\documentclass[12pt]{article}

%%%%%%%%%%%%%%%%%%%%%%%% Balíky %%%%%%%%%%%%%%%%%%%%%%%%%%%%%%%%%%%%%
\usepackage[utf8]{inputenc}
\usepackage[T1]{fontenc}
\usepackage[czech]{babel} 
\usepackage{amsmath, amsfonts, amssymb, amsthm}
\usepackage{graphicx}
\usepackage{booktabs} % Pre profesionálne vyzerajúce tabuľky
\usepackage{geometry}
\geometry{a4paper, margin=2.5cm}
\usepackage[colorlinks=true, linkcolor=black, citecolor=red]{hyperref}
\usepackage{titlesec}
\titleformat{\section}{\large\bfseries}{\thesection}{1em}{}
\usepackage{lmodern} 
%\usepackage{newtxtext,newtxmath}
%\usepackage{microtype}
\usepackage{pgfplots}
\usepackage{float}


\pgfplotsset{compat=1.18}
\usetikzlibrary{arrows.meta}

%%%%%%%%%%%%%%%%%%%%%%% Obsah %%%%%%%%%%%%%%%%%%%%%%%%%%%%%%%%%%%%%%%
\usepackage{tocloft}
\renewcommand{\cftsecdotsep}{\cftdotsep}
\renewcommand{\cfttoctitlefont}{\Large\bfseries}
\renewcommand{\cftaftertoctitle}{\vspace{0.5cm}}
\setlength{\cftsecindent}{0pt}      % Sekcia bez odsadenia
\setlength{\cftsubsecindent}{1.5em} % Podsekcia odsadená
\setlength{\cftsubsubsecindent}{3em} % Podpodsekcia odsadená viac
\renewcommand{\cftsecpagefont}{\normalfont}
\renewcommand{\cftsubsecpagefont}{\normalfont}

%%%%%%%%%%%%%%%%%%%%% Okraje a Odsadeni %%%%%%%%%%%%%%%%%%%%%%%%%%%%%

\geometry{
    a4paper,
    total={170mm,257mm},
    left=2.5cm,
    right=2.5cm,
    top=5cm,      % Toto spraví ten veľký odskok zhora ako v PDF
    bottom=5cm    % Spodný okraj pre číslovanie
}

%%%%%%%%%%%%%%%%%%%%%%%%%%%%%%%%%%%%%%%%%%%%%%%%%%%%%%%%%%%%%%%%%%%%%

\begin{document}

\begin{titlepage}
    \centering
    \vspace*{2cm} % Odsadenie zhora, aby bol hlavný názov približne v strede hornej polovice

    % Hlavný názov
    {\huge Projekt – numerické metódy v $\mathbb{R}$ a $\mathbb{R}^n$ \par}
    
    \vspace{0.8cm} % Medzera medzi názvom a predmetom
    
    % Názov predmetu - stredne veľké písmo
    {\Large Matematické programovanie (M5170) \par}
    
    \vspace{0.8cm} % Väčšia medzera pred menami autorov
    
    % Mená autorov 
    {\large Ján Húska, Radim Čech \par}
    
    \vfill % Vyplnenie miesta, aby sa číslo strany dostalo na spodok

    % Číslo strany na titulnom liste 
    \vfill % Toto vytlačí zvyšok textu úplne na spodok strany
    
    % Dátum a číslo strany
    {\large december 2026 \par}
    \vspace{0.5cm}
    {\large 1 \par}
\end{titlepage}


\pagenumbering{arabic}
\setcounter{page}{2}
%%%%%%%%%%%%%%%%%%%%%%%%%%%%%%%%%%%%%%%%%%%%%%%%%%%%%%%%%%%%%%%%%%%%%

%Obsah 
\tableofcontents
\newpage


%%%%%%%%%%%%%%%%%%%%%%%%%%%%%%%%%%%%%%%%%%%%%%%%%%%%%%%%%%%%%%%%%%%%%
\section{Analýza numerických metód pre funkciu jednej premennej}

Na analýzu správania numerických metód pre funkciu jednej premennej sme si zvolili funkciu
$$ f(x) = (x-2)^2 + \frac{1}{2}\sin(3x) - \frac{1}{3}\cos(2x) + 1, $$
ktorá je spojitá na celom $\mathbb{R}$. Svoje „presné“ minimum táto funkcia nadobúda v bode
$$ x^* \doteq 1{,}7876 $$
s funkčnou hodnotou
$$ f(x^*) \doteq 0{,}9959. $$
Na nasledujúcej strane je vykreslený graf tejto funkcie pre obmedzený rozsah hodnôt $x$ a $y$.

Pre tento projekt sme vybrali \textbf{metódu polenia intervalu} a \textbf{metódu zlatého rezu}, ktorých správanie budeme na tejto funkcii skúmať.

\newpage
%%%%%%%%%%%%%%%%%%%%%%%%%%%%%%%%%%%%%%%%%%%%%%%%%%%%%%%%%%%%%%%%%%%%%
% --- 4. STRANA: GRAF FUNKCIE 
\thispagestyle{plain}
\vspace*{\fill} 

\begin{figure}[h!]
    \centering
    \begin{tikzpicture}
        \begin{axis}[
            width=16cm, height=12cm,    % ZVÄČŠENIE GRAFU na maximum strany
            axis lines=middle,
            xlabel={$x$}, ylabel={$f(x)$},
            xlabel style={at={(ticklabel* cs:1)}, anchor=north west},
            ylabel style={at={(ticklabel* cs:1)}, anchor=south west},
            xmin=-2.5, xmax=6.5,
            ymin=-2, ymax=18,
            grid=major,
            grid style={dotted, gray!50},
            domain=-2.2:6.2,
            samples=200,
            trig format plots=rad,
            legend style={at={(0.95,0.95)}, anchor=north east, draw=none, fill=none},
            tick label style={font=\footnotesize}
        ]
            
            % 1. Funkcia
            \addplot[color=blue!80!black, thick] 
            { (x-2)^2 + 0.5*sin(3*x) - (1/3)*cos(2*x) + 1 };
            \addlegendentry{$f(x)$}

            % 2. Minimum (bodka)
            \addplot[mark=*, color=red, mark size=2.5pt, only marks] 
            coordinates {(1.7876, 0.9959)};

            % 3. Vodiace čiary
            \draw[dashed, gray] (axis cs:1.7876, 0) -- (axis cs:1.7876, 0.9959);
            \draw[dashed, gray] (axis cs:0, 0.9959) -- (axis cs:1.7876, 0.9959);

            % 4. Popis minima 
            \node[
                anchor=west,           % Zarovnanie textu
                fill=white,            % Biele pozadie (aby bol text čitateľný)
                font=\footnotesize,
                inner sep=2pt,         % Odsadenie textu od okraja
                draw=gray!30,          % Jemný rámček (voliteľné)
                rounded corners        % Zaoblené rohy rámčeka
            ] (popis) at (axis cs:2.5, 4) {Minimum $[1{,}79; 0{,}99]$};

            % Šípka od popisu k bodu (aby bolo jasné, kam patrí)
            \draw[->, gray, thick, shorten >=2pt] (popis.south west) -- (axis cs:1.85, 1.1);

        \end{axis}
    \end{tikzpicture}
    \caption{Graf funkcie $ f(x) = (x-2)^2 + \frac{1}{2}\sin(3x) - \frac{1}{3}\cos(2x) + 1. $}
\end{figure}

\vspace*{\fill}
\newpage
%%%%%%%%%%%%%%%%%%%%%%%%%%%%%%%%%%%%%%%%%%%%%%%%%%%%%%%%%%%%%%%%%%%%%

\subsection{Metóda polenia intervalu}

\subsubsection{Analýza správania metódy pri rôznych hodnotách počtu vyčíslení}

Ako prvé sa pozrieme na to, ako rôzny zvolený počet vyčíslení ovplyvní presnosť tejto metódy. Konkrétne budeme voliť $N \in \{4, 10, 20, 50, 100\}$, metóda totiž vyžaduje párny počet vyčíslení.

Pre dané vyčíslenia tiež zvolíme pevne rovnaký počiatočný interval. Keďže vopred vieme, že naša funkcia nadobúda v bode $x^* \doteq 1{,}7876$ svoje minimum, zvolíme tak interval, ktorý je približne symetrický okolo daného bodu. My teda volíme počiatočný interval $I_0$ ako $[-2; 6]$.

Pre túto metódu je nutné tiež zvoliť hodnotu $\delta \in (0, \frac{1}{2}(b-a))$. Z teórie vieme, že je vhodné voliť hodnotu delta „dostatočne malú“. V našom prípade teda volíme $\delta = 0{,}01$. Teraz už prejdeme k samotnej analýze správania metódy polenia intervalu a porovnáme výsledky získané pre rôzny počet vyčíslení.

\vspace{0.5cm}

\noindent \textbf{N=4} \\
Pre $N=4$ vyčíslení dostávame výsledný ILM (interval lokalizácie minima) približne ako $I_2 = [0{,}01; 2{,}01]$ a apriórny odhad maximálnej chyby je $\epsilon_{max} \doteq 1{,}0$. Aproximáciu minima dostávame ako stred tohto intervalu $\tilde{x} \doteq 1{,}01$ s funkčnou hodnotou $f(\tilde{x}) \doteq 1{,}623$. Skutočná chyba je potom $\epsilon \doteq 0{,}777$. Vidíme, že po iba dvoch krokoch metódy je chyba stále značná, čo je vzhľadom na šírku pôvodného intervalu očakávateľné.

\vspace{0.3cm}

\noindent \textbf{N=10} \\
Pre $N=10$ vyčíslení (5 krokov) sa interval výrazne zúži. Dostávame výsledný ILM približne ako $I_5 = [1{,}51; 2{,}01]$ a apriórny odhad maximálnej chyby je $\epsilon_{max} \doteq 0{,}25$. Aproximáciu minima dostávame ako bod $\tilde{x} \doteq 1{,}76$ s funkčnou hodnotou $f(\tilde{x}) \doteq 0{,}9965$. Skutočná chyba je potom $\epsilon \doteq 0{,}027$. Chyba sa rádovo zmenšila a aproximácia sa už blíži k skutočnému minimu.

\vspace{0.3cm}

\noindent \textbf{N=20} \\
Pre $N=20$ vyčíslení dostávame výsledný ILM ako $I_{10} = [1{,}78; 1{,}79]$ a apriórny odhad maximálnej chyby klesá pod $0{,}01$. Aproximáciu minima dostávame ako bod $\tilde{x} \doteq 1{,}785$ s funkčnou hodnotou $f(\tilde{x}) \doteq 0{,}9959$. Skutočná chyba je už veľmi malá, rádovo $10^{-3}$. Zdvojnásobením počtu vyčíslení sme dosiahli výrazné spresnenie.

\vspace{0.3cm}

\noindent \textbf{N=50} \\
Pre $N=50$ vyčíslení je interval lokalizácie minima extrémne úzky. Apriórny odhad chyby je už v rádoch $10^{-6}$. Aproximácia $\tilde{x}$ je prakticky totožná s $x^*$ na 6 desatinných miest. Tu sa už dostávame na veľmi dobrú presnosť, a to najmä pre funkčnú hodnotu. Je však otázkou, či je na praktické účely takto vysoký počet iterácií nutný, keď už pri $N=20$ bola chyba zanedbateľná.

\vspace{0.3cm}

\noindent \textbf{N=100} \\
Pre $N=100$ vyčíslení sa chyba prakticky nevylepšila oproti predchádzajúcim vyčísleniam, narážame tu už skôr na limity presnosti zvoleného $\delta$ a numerického zobrazenia. Chyby sa líšia až na vysokom mieste za desatinnou čiarkou. Pri našej ďalšej analýze sa tak obmedzíme len na nižšie $N$, ktoré dostatočne ilustrujú správanie metódy. \\

%%%%%%%%%%%%%%%%%%%%%%%%%%%%%%%%%%%%%%%%%%%%%%%%%%%%%%%%%%%%%%%%%%%%%

\subsubsection{Analýza správania metódy pri rôznych východiskových intervaloch lokalizácie minima}

Najskôr sa zameriame na rozdielne dĺžky počiatočných intervalov. Konkrétne náš pôvodný interval $I_0 = [-2; 6]$ najprve skrátime na polovicu a potom dvakrát predĺžime. Apriórny odhad maximálnej chyby totiž závisí od konkrétnej dĺžky intervalu.

Konkrétne, zo vzťahu pre dĺžku výsledného intervalu po $k = N/2$ krokoch
$$ l_k = \frac{b-a}{2^k} + \delta \frac{2^k - 1}{2^{k-1}} $$
dostávame, že skrátenie intervalu by malo viesť k lepšej presnosti, zatiaľ čo jeho predĺženie k horšej. Overme to na konkrétnych dátach.

\vspace{0.5cm}
\noindent \textbf{Kratší počiatočný interval} \\
Pre východiskový interval $I_0 = [0; 4]$ (dĺžka 4) a $N=10$ vyčíslení dostávame výsledný ILM približne ako $I_5 = [1{,}75; 1{,}87]$. Apriórny odhad maximálnej chyby je $\epsilon_{max} \doteq 0{,}06$. Aproximáciu minima dostávame ako bod $\tilde{x} \doteq 1{,}81$ s funkčnou hodnotou $f(\tilde{x}) \doteq 0{,}997$. Skutočná chyba je v tomto prípade výrazne menšia než pri pôvodnom dlhšom intervale $[-2; 6]$ pri rovnakom počte vyčíslení.

Záverom teda môžeme konštatovať, že kratší interval je nezanedbateľne výhodný predovšetkým pre nižší počet vyčíslení.

\vspace{0.5cm}
\noindent \textbf{Dlhší počiatočný interval} \\
Pre východiskový interval $I_0 = [-6; 10]$ (dĺžka 16) a $N=10$ vyčíslení je apriórny odhad chyby prirodzene väčší. Výsledný ILM vychádza širší a aproximácia minima $\tilde{x}$ je menej presná. Konkrétne pre $N=10$ je chyba stále v ráde desatín. To potvrdzuje teoretický predpoklad, že ak o polohe minima nič nevieme a musíme zvoliť široký interval „pre istotu“, zaplatíme za to nižšou presnosťou alebo nutnosťou vyššieho počtu iterácií.

\vspace{0.5cm}
\noindent \textbf{Nesymetrický východiskový interval} \\
Na záver ešte preskúmajme interval, ktorého jeden z krajných bodov nie je od skutočného bodu minima príliš ďaleko, napr. $I_0 = [-10; 2]$. V úvode sme spomínali, že takýto interval sa bude pri metóde prevažne skracovať z jednej strany.

Ak v našom základnom uvažovanom intervale trafíme jeho krajom príliš blízko k bodu $x^*$, potom v ďalšom kroku môže byť stred intervalu výrazne vzdialený od bodu $x^*$. Pre malý počet vyčíslení to teda môže robiť problém. V našom prípade sa však ukazuje, že metóda je robustná a aj napriek počiatočnému „tápaniu“ sa po niekoľkých iteráciách interval začne sťahovať okolo skutočného minima.
%%%%%%%%%%%%%%%%%%%%%%%%%%%%%%%%%%%%%%%%%%%%%%%%%%%%%%%%%%%%%%%%%%%%%
% --- METÓDA ZLATÉHO REZU ---
\newpage
\subsection{Metóda zlatého rezu}

Podobne ako pri metóde polenia intervalu sa najskôr pozrieme na to, ako rôzny zvolený počet vyčíslení ovplyvní presnosť tejto metódy. Konkrétne budeme opäť voliť $N \in \{4, 10, 20, 50, 100\}$. Rovnako tiež zvolíme počiatočný interval $I_0$ ako $[-2; 6]$. Pre túto metódu už nevolíme hodnotu $\delta$.

Princíp tejto metódy vychádza práve z metódy polenia intervalu, avšak vhodnou voľbou bodov budeme okrem prvého kroku vyčísľovať vždy iba jeden nový bod. Teda v prípade rovnakého počtu vyčíslení $N$ by sme mali dostať $N-1$ nových intervalov, zatiaľ čo pri MPI dostaneme iba $N/2$.

\subsubsection{Analýza správania metódy pri rôznych hodnotách počtu vyčíslení}

\vspace{0.3cm}
\noindent \textbf{N=4} \\
Pre $N=4$ vyčíslení dostávame pri metóde zlatého rezu (MZR) o jeden interval viac oproti MPI (konkrétne 3 oproti 2). Pri takto malých hodnotách vyčíslení každý nový krok prináša zlepšenie.

% TABUĽKA PRE N=4
\begin{table}[h!]
    \centering
    \begin{tabular}{lcc}
        \toprule
        \textbf{N=4} & \textbf{MZR} & \textbf{MPI} \\
        \midrule
        Aproximácia $\tilde{x}$ & $1{,}054$ & $1{,}010$ \\
        Chyba $|\tilde{x}-x^*|$ & $0{,}733$ & $0{,}777$ \\
        Hodnota $f(\tilde{x})$ & $1{,}538$ & $1{,}623$ \\
        Posledný ILM & $[0{,}14; 2{,}94]$ & $[0{,}01; 2{,}01]$ \\
        \bottomrule
    \end{tabular}
    \caption{Porovnanie pre $N=4$.}
\end{table}

\noindent \textbf{N=10} \\
Tu už dostávame 9 intervalov oproti 5 z metódy polenia intervalu. To sa prejaví na presnosti. Očakávame, že chyba MZR bude menšia.

% TABUĽKA PRE N=10
\begin{table}[h!]
    \centering
    \begin{tabular}{lcc}
        \toprule
        \textbf{N=10} & \textbf{MZR} & \textbf{MPI} \\
        \midrule
        Aproximácia $\tilde{x}$ & $1{,}792$ & $1{,}760$ \\
        Chyba $|\tilde{x}-x^*|$ & $0{,}004$ & $0{,}027$ \\
        Hodnota $f(\tilde{x})$ & $0{,}9960$ & $0{,}9965$ \\
        \bottomrule
    \end{tabular}
    \caption{Porovnanie pre $N=10$.}
\end{table}

\noindent \textbf{N=20} \\
Pre $N=20$ je metóda zlatého rezu výrazne efektívnejšia. Zatiaľ čo pri MPI sa interval zmenšuje faktorom 0,5 každé dva kroky, pri MZR sa zmenšuje faktorom približne 0,618 každý jeden krok.

% TABUĽKA PRE N=20 (Doplnená pre kompletnosť)
\begin{table}[h!]
    \centering
    \begin{tabular}{lcc}
        \toprule
        \textbf{N=20} & \textbf{MZR} & \textbf{MPI} \\
        \midrule
        Aproximácia $\tilde{x}$ & $1{,}7876$ & $1{,}7850$ \\
        Chyba $|\tilde{x}-x^*|$ & $10^{-5}$ & $10^{-3}$ \\
        \bottomrule
    \end{tabular}
    \caption{Porovnanie pre $N=20$.}
\end{table}

%%%%%%%%%%%%%%%%%%%%%%%%%%%%%%%%%%%%%%%%%%%%%%%%%%%%%%%%%%%%%%%%%%%%%

\subsubsection{Analýza správania metódy pri rôznych východiskových intervaloch lokalizácie minima}

Aj tu, rovnako ako pri metóde polenia intervalu, by sme očakávali lepšie výsledky pre kratší počiatočný interval, respektíve horšie výsledky pre dlhší počiatočný interval. Overíme, či je to skutočne tak aj pre túto metódu.

\vspace{0.5cm}
\noindent \textbf{Kratší počiatočný interval} \\
Pre východiskový interval $I_0 = [0; 4]$ a $N=10$ vyčíslení dostávame výrazne presnejšiu aproximáciu než pri pôvodnom intervale $[-2; 6]$.

% TABUĽKA PRE KRATŠÍ INTERVAL
\begin{table}[h!]
    \centering
    \begin{tabular}{lcc}
        \toprule
        \textbf{Parameter} & \textbf{Pôvodný } $[-2; 6]$ & \textbf{Kratší } $[0; 4]$ \\
        \midrule
        Dĺžka intervalu & 8 & 4 \\
        Chyba pre $N=10$ & $0{,}0042$ & $0{,}0015$ \\
        Hodnota $f(\tilde{x})$ & $0{,}9960$ & $0{,}9959$ \\
        \bottomrule
    \end{tabular}
    \caption{Vplyv skrátenia intervalu na presnosť (N=10).}
\end{table}

\noindent Ako vidíme, skrátenie intervalu na polovicu viedlo k rádovému zlepšeniu chyby. To potvrdzuje, že ak máme o polohe minima dobrú predbežnú informáciu, oplatí sa zvoliť interval čo najužší.

\vspace{0.5cm}
\noindent \textbf{Dlhší počiatočný interval} \\
Teraz skúsme interval zdvojnásobiť na $I_0 = [-6; 10]$. Tu očakávame, že metóda bude potrebovať viac krokov na dosiahnutie rovnakej presnosti.

% TABUĽKA PRE DLHŠÍ INTERVAL
\begin{table}[h!]
    \centering
    \begin{tabular}{lcc}
        \toprule
        \textbf{Parameter} & \textbf{Pôvodný } $[-2; 6]$ & \textbf{Dlhší } $[-6; 10]$ \\
        \midrule
        Dĺžka intervalu & 8 & 16 \\
        Chyba pre $N=10$ & $0{,}0042$ & $0{,}0210$ \\
        Chyba pre $N=20$ & $10^{-5}$ & $10^{-4}$ \\
        \bottomrule
    \end{tabular}
    \caption{Vplyv predĺženia intervalu na presnosť.}
\end{table}

\noindent Pre dvakrát dlhší počiatočný interval dostávame zhoršenie chyby. Aj pre vyšší počet vyčíslení ($N=20$) je rozdiel stále viditeľný.

\vspace{0.5cm}
\noindent \textbf{Nesymetrický východiskový interval} \\
Na záver ešte preskúmajme nesymetrický interval $I_0 = [-10; 2]$. Pri metóde zlatého rezu, ktorá zmenšuje interval s konštantným pomerom $\tau \approx 0{,}618$, môže takáto asymetria spôsobiť, že v prvých krokoch sa interval skracuje „z nesprávnej strany“, kým sa nový bod nedostane do blízkosti minima.

Pre $N=10$ vyčíslení dostávame chybu približne $0{,}015$, čo je horší výsledok než pri symetrickom intervale $[-2; 6]$. To ukazuje, že aj poloha minima v rámci intervalu zohráva úlohu v rýchlosti konvergencie v prvých iteráciách.
\newpage
%%%%%%%%%%%%%%%%%%%%%%%%%%%%%%%%%%%%%%%%%%%%%%%%%%%%%%%%%%%%%%%%%%%%%

\section{Analýza numerických metód pre funkciu viac premenných}

Na analýzu numerických metód pre funkciu viac (konkrétne dvoch) premenných sme si zvolili funkciu
$$ f(x,y) = e^{-x} + e^{y} + (x-y^2)^2 + x, $$
ktorá je definovaná a spojitá na celom priestore $\mathbb{R}^2$. Svoje „presné“ minimum táto funkcia nadobúda v bode
$$ [x^*; y^*] \doteq [0.3881288312483782; -0.7409240564423611] $$
s funkčnou hodnotou
$$ f(x^*; y^*) \doteq 1.5689964037485506 $$
Na nasledujúcich dvoch stranách sú potom vykreslené vrstevnicový graf a 3D graf tejto funkcie z rôznych pohľadov, ktoré ilustrujú jej tvar a globálne vlastnosti.

Pre tento projekt sme zvolili \textit{Newtonovu metódu} a \textit{metódu združených gradientov}, ktorých správanie a rýchlosť konvergencie budeme pre túto funkciu analyzovať.

\newpage

% --- 10. STRANA: VRSTEVNICOVÝ GRAF ---
\newpage
\thispagestyle{plain}
\vspace*{\fill}

\begin{figure}[H]
    \centering
    \resizebox{0.8\textwidth}{!}{
    %% Creator: Matplotlib, PGF backend
%%
%% To include the figure in your LaTeX document, write
%%   \input{<filename>.pgf}
%%
%% Make sure the required packages are loaded in your preamble
%%   \usepackage{pgf}
%%
%% Also ensure that all the required font packages are loaded; for instance,
%% the lmodern package is sometimes necessary when using math font.
%%   \usepackage{lmodern}
%%
%% Figures using additional raster images can only be included by \input if
%% they are in the same directory as the main LaTeX file. For loading figures
%% from other directories you can use the `import` package
%%   \usepackage{import}
%%
%% and then include the figures with
%%   \import{<path to file>}{<filename>.pgf}
%%
%% Matplotlib used the following preamble
%%   
%%   \usepackage{fontspec}
%%   \setmainfont{DejaVuSerif.ttf}[Path=\detokenize{/home/radimek/Documents/projekt_mat_prog/mat_prog_kernel/lib/python3.12/site-packages/matplotlib/mpl-data/fonts/ttf/}]
%%   \setsansfont{DejaVuSans.ttf}[Path=\detokenize{/home/radimek/Documents/projekt_mat_prog/mat_prog_kernel/lib/python3.12/site-packages/matplotlib/mpl-data/fonts/ttf/}]
%%   \setmonofont{DejaVuSansMono.ttf}[Path=\detokenize{/home/radimek/Documents/projekt_mat_prog/mat_prog_kernel/lib/python3.12/site-packages/matplotlib/mpl-data/fonts/ttf/}]
%%   \makeatletter\@ifpackageloaded{underscore}{}{\usepackage[strings]{underscore}}\makeatother
%%
\begingroup%
\makeatletter%
\begin{pgfpicture}%
\pgfpathrectangle{\pgfpointorigin}{\pgfqpoint{7.000000in}{6.000000in}}%
\pgfusepath{use as bounding box, clip}%
\begin{pgfscope}%
\pgfsetbuttcap%
\pgfsetmiterjoin%
\definecolor{currentfill}{rgb}{1.000000,1.000000,1.000000}%
\pgfsetfillcolor{currentfill}%
\pgfsetlinewidth{0.000000pt}%
\definecolor{currentstroke}{rgb}{1.000000,1.000000,1.000000}%
\pgfsetstrokecolor{currentstroke}%
\pgfsetdash{}{0pt}%
\pgfpathmoveto{\pgfqpoint{0.000000in}{0.000000in}}%
\pgfpathlineto{\pgfqpoint{7.000000in}{0.000000in}}%
\pgfpathlineto{\pgfqpoint{7.000000in}{6.000000in}}%
\pgfpathlineto{\pgfqpoint{0.000000in}{6.000000in}}%
\pgfpathlineto{\pgfqpoint{0.000000in}{0.000000in}}%
\pgfpathclose%
\pgfusepath{fill}%
\end{pgfscope}%
\begin{pgfscope}%
\pgfsetbuttcap%
\pgfsetmiterjoin%
\definecolor{currentfill}{rgb}{1.000000,1.000000,1.000000}%
\pgfsetfillcolor{currentfill}%
\pgfsetlinewidth{0.000000pt}%
\definecolor{currentstroke}{rgb}{0.000000,0.000000,0.000000}%
\pgfsetstrokecolor{currentstroke}%
\pgfsetstrokeopacity{0.000000}%
\pgfsetdash{}{0pt}%
\pgfpathmoveto{\pgfqpoint{0.854460in}{0.571603in}}%
\pgfpathlineto{\pgfqpoint{6.739560in}{0.571603in}}%
\pgfpathlineto{\pgfqpoint{6.739560in}{5.797238in}}%
\pgfpathlineto{\pgfqpoint{0.854460in}{5.797238in}}%
\pgfpathlineto{\pgfqpoint{0.854460in}{0.571603in}}%
\pgfpathclose%
\pgfusepath{fill}%
\end{pgfscope}%
\begin{pgfscope}%
\pgfsetbuttcap%
\pgfsetroundjoin%
\definecolor{currentfill}{rgb}{0.000000,0.000000,0.000000}%
\pgfsetfillcolor{currentfill}%
\pgfsetlinewidth{0.803000pt}%
\definecolor{currentstroke}{rgb}{0.000000,0.000000,0.000000}%
\pgfsetstrokecolor{currentstroke}%
\pgfsetdash{}{0pt}%
\pgfsys@defobject{currentmarker}{\pgfqpoint{0.000000in}{-0.048611in}}{\pgfqpoint{0.000000in}{0.000000in}}{%
\pgfpathmoveto{\pgfqpoint{0.000000in}{0.000000in}}%
\pgfpathlineto{\pgfqpoint{0.000000in}{-0.048611in}}%
\pgfusepath{stroke,fill}%
}%
\begin{pgfscope}%
\pgfsys@transformshift{0.854460in}{0.571603in}%
\pgfsys@useobject{currentmarker}{}%
\end{pgfscope}%
\end{pgfscope}%
\begin{pgfscope}%
\definecolor{textcolor}{rgb}{0.000000,0.000000,0.000000}%
\pgfsetstrokecolor{textcolor}%
\pgfsetfillcolor{textcolor}%
\pgftext[x=0.854460in,y=0.474381in,,top]{\color{textcolor}\sffamily\fontsize{10.000000}{12.000000}\selectfont \ensuremath{-}1.0}%
\end{pgfscope}%
\begin{pgfscope}%
\pgfsetbuttcap%
\pgfsetroundjoin%
\definecolor{currentfill}{rgb}{0.000000,0.000000,0.000000}%
\pgfsetfillcolor{currentfill}%
\pgfsetlinewidth{0.803000pt}%
\definecolor{currentstroke}{rgb}{0.000000,0.000000,0.000000}%
\pgfsetstrokecolor{currentstroke}%
\pgfsetdash{}{0pt}%
\pgfsys@defobject{currentmarker}{\pgfqpoint{0.000000in}{-0.048611in}}{\pgfqpoint{0.000000in}{0.000000in}}{%
\pgfpathmoveto{\pgfqpoint{0.000000in}{0.000000in}}%
\pgfpathlineto{\pgfqpoint{0.000000in}{-0.048611in}}%
\pgfusepath{stroke,fill}%
}%
\begin{pgfscope}%
\pgfsys@transformshift{1.835310in}{0.571603in}%
\pgfsys@useobject{currentmarker}{}%
\end{pgfscope}%
\end{pgfscope}%
\begin{pgfscope}%
\definecolor{textcolor}{rgb}{0.000000,0.000000,0.000000}%
\pgfsetstrokecolor{textcolor}%
\pgfsetfillcolor{textcolor}%
\pgftext[x=1.835310in,y=0.474381in,,top]{\color{textcolor}\sffamily\fontsize{10.000000}{12.000000}\selectfont \ensuremath{-}0.5}%
\end{pgfscope}%
\begin{pgfscope}%
\pgfsetbuttcap%
\pgfsetroundjoin%
\definecolor{currentfill}{rgb}{0.000000,0.000000,0.000000}%
\pgfsetfillcolor{currentfill}%
\pgfsetlinewidth{0.803000pt}%
\definecolor{currentstroke}{rgb}{0.000000,0.000000,0.000000}%
\pgfsetstrokecolor{currentstroke}%
\pgfsetdash{}{0pt}%
\pgfsys@defobject{currentmarker}{\pgfqpoint{0.000000in}{-0.048611in}}{\pgfqpoint{0.000000in}{0.000000in}}{%
\pgfpathmoveto{\pgfqpoint{0.000000in}{0.000000in}}%
\pgfpathlineto{\pgfqpoint{0.000000in}{-0.048611in}}%
\pgfusepath{stroke,fill}%
}%
\begin{pgfscope}%
\pgfsys@transformshift{2.816160in}{0.571603in}%
\pgfsys@useobject{currentmarker}{}%
\end{pgfscope}%
\end{pgfscope}%
\begin{pgfscope}%
\definecolor{textcolor}{rgb}{0.000000,0.000000,0.000000}%
\pgfsetstrokecolor{textcolor}%
\pgfsetfillcolor{textcolor}%
\pgftext[x=2.816160in,y=0.474381in,,top]{\color{textcolor}\sffamily\fontsize{10.000000}{12.000000}\selectfont 0.0}%
\end{pgfscope}%
\begin{pgfscope}%
\pgfsetbuttcap%
\pgfsetroundjoin%
\definecolor{currentfill}{rgb}{0.000000,0.000000,0.000000}%
\pgfsetfillcolor{currentfill}%
\pgfsetlinewidth{0.803000pt}%
\definecolor{currentstroke}{rgb}{0.000000,0.000000,0.000000}%
\pgfsetstrokecolor{currentstroke}%
\pgfsetdash{}{0pt}%
\pgfsys@defobject{currentmarker}{\pgfqpoint{0.000000in}{-0.048611in}}{\pgfqpoint{0.000000in}{0.000000in}}{%
\pgfpathmoveto{\pgfqpoint{0.000000in}{0.000000in}}%
\pgfpathlineto{\pgfqpoint{0.000000in}{-0.048611in}}%
\pgfusepath{stroke,fill}%
}%
\begin{pgfscope}%
\pgfsys@transformshift{3.797010in}{0.571603in}%
\pgfsys@useobject{currentmarker}{}%
\end{pgfscope}%
\end{pgfscope}%
\begin{pgfscope}%
\definecolor{textcolor}{rgb}{0.000000,0.000000,0.000000}%
\pgfsetstrokecolor{textcolor}%
\pgfsetfillcolor{textcolor}%
\pgftext[x=3.797010in,y=0.474381in,,top]{\color{textcolor}\sffamily\fontsize{10.000000}{12.000000}\selectfont 0.5}%
\end{pgfscope}%
\begin{pgfscope}%
\pgfsetbuttcap%
\pgfsetroundjoin%
\definecolor{currentfill}{rgb}{0.000000,0.000000,0.000000}%
\pgfsetfillcolor{currentfill}%
\pgfsetlinewidth{0.803000pt}%
\definecolor{currentstroke}{rgb}{0.000000,0.000000,0.000000}%
\pgfsetstrokecolor{currentstroke}%
\pgfsetdash{}{0pt}%
\pgfsys@defobject{currentmarker}{\pgfqpoint{0.000000in}{-0.048611in}}{\pgfqpoint{0.000000in}{0.000000in}}{%
\pgfpathmoveto{\pgfqpoint{0.000000in}{0.000000in}}%
\pgfpathlineto{\pgfqpoint{0.000000in}{-0.048611in}}%
\pgfusepath{stroke,fill}%
}%
\begin{pgfscope}%
\pgfsys@transformshift{4.777860in}{0.571603in}%
\pgfsys@useobject{currentmarker}{}%
\end{pgfscope}%
\end{pgfscope}%
\begin{pgfscope}%
\definecolor{textcolor}{rgb}{0.000000,0.000000,0.000000}%
\pgfsetstrokecolor{textcolor}%
\pgfsetfillcolor{textcolor}%
\pgftext[x=4.777860in,y=0.474381in,,top]{\color{textcolor}\sffamily\fontsize{10.000000}{12.000000}\selectfont 1.0}%
\end{pgfscope}%
\begin{pgfscope}%
\pgfsetbuttcap%
\pgfsetroundjoin%
\definecolor{currentfill}{rgb}{0.000000,0.000000,0.000000}%
\pgfsetfillcolor{currentfill}%
\pgfsetlinewidth{0.803000pt}%
\definecolor{currentstroke}{rgb}{0.000000,0.000000,0.000000}%
\pgfsetstrokecolor{currentstroke}%
\pgfsetdash{}{0pt}%
\pgfsys@defobject{currentmarker}{\pgfqpoint{0.000000in}{-0.048611in}}{\pgfqpoint{0.000000in}{0.000000in}}{%
\pgfpathmoveto{\pgfqpoint{0.000000in}{0.000000in}}%
\pgfpathlineto{\pgfqpoint{0.000000in}{-0.048611in}}%
\pgfusepath{stroke,fill}%
}%
\begin{pgfscope}%
\pgfsys@transformshift{5.758710in}{0.571603in}%
\pgfsys@useobject{currentmarker}{}%
\end{pgfscope}%
\end{pgfscope}%
\begin{pgfscope}%
\definecolor{textcolor}{rgb}{0.000000,0.000000,0.000000}%
\pgfsetstrokecolor{textcolor}%
\pgfsetfillcolor{textcolor}%
\pgftext[x=5.758710in,y=0.474381in,,top]{\color{textcolor}\sffamily\fontsize{10.000000}{12.000000}\selectfont 1.5}%
\end{pgfscope}%
\begin{pgfscope}%
\pgfsetbuttcap%
\pgfsetroundjoin%
\definecolor{currentfill}{rgb}{0.000000,0.000000,0.000000}%
\pgfsetfillcolor{currentfill}%
\pgfsetlinewidth{0.803000pt}%
\definecolor{currentstroke}{rgb}{0.000000,0.000000,0.000000}%
\pgfsetstrokecolor{currentstroke}%
\pgfsetdash{}{0pt}%
\pgfsys@defobject{currentmarker}{\pgfqpoint{0.000000in}{-0.048611in}}{\pgfqpoint{0.000000in}{0.000000in}}{%
\pgfpathmoveto{\pgfqpoint{0.000000in}{0.000000in}}%
\pgfpathlineto{\pgfqpoint{0.000000in}{-0.048611in}}%
\pgfusepath{stroke,fill}%
}%
\begin{pgfscope}%
\pgfsys@transformshift{6.739560in}{0.571603in}%
\pgfsys@useobject{currentmarker}{}%
\end{pgfscope}%
\end{pgfscope}%
\begin{pgfscope}%
\definecolor{textcolor}{rgb}{0.000000,0.000000,0.000000}%
\pgfsetstrokecolor{textcolor}%
\pgfsetfillcolor{textcolor}%
\pgftext[x=6.739560in,y=0.474381in,,top]{\color{textcolor}\sffamily\fontsize{10.000000}{12.000000}\selectfont 2.0}%
\end{pgfscope}%
\begin{pgfscope}%
\definecolor{textcolor}{rgb}{0.000000,0.000000,0.000000}%
\pgfsetstrokecolor{textcolor}%
\pgfsetfillcolor{textcolor}%
\pgftext[x=3.797010in,y=0.284413in,,top]{\color{textcolor}\sffamily\fontsize{10.000000}{12.000000}\selectfont x}%
\end{pgfscope}%
\begin{pgfscope}%
\pgfsetbuttcap%
\pgfsetroundjoin%
\definecolor{currentfill}{rgb}{0.000000,0.000000,0.000000}%
\pgfsetfillcolor{currentfill}%
\pgfsetlinewidth{0.803000pt}%
\definecolor{currentstroke}{rgb}{0.000000,0.000000,0.000000}%
\pgfsetstrokecolor{currentstroke}%
\pgfsetdash{}{0pt}%
\pgfsys@defobject{currentmarker}{\pgfqpoint{-0.048611in}{0.000000in}}{\pgfqpoint{-0.000000in}{0.000000in}}{%
\pgfpathmoveto{\pgfqpoint{-0.000000in}{0.000000in}}%
\pgfpathlineto{\pgfqpoint{-0.048611in}{0.000000in}}%
\pgfusepath{stroke,fill}%
}%
\begin{pgfscope}%
\pgfsys@transformshift{0.854460in}{0.571603in}%
\pgfsys@useobject{currentmarker}{}%
\end{pgfscope}%
\end{pgfscope}%
\begin{pgfscope}%
\definecolor{textcolor}{rgb}{0.000000,0.000000,0.000000}%
\pgfsetstrokecolor{textcolor}%
\pgfsetfillcolor{textcolor}%
\pgftext[x=0.339968in, y=0.518842in, left, base]{\color{textcolor}\sffamily\fontsize{10.000000}{12.000000}\selectfont \ensuremath{-}1.00}%
\end{pgfscope}%
\begin{pgfscope}%
\pgfsetbuttcap%
\pgfsetroundjoin%
\definecolor{currentfill}{rgb}{0.000000,0.000000,0.000000}%
\pgfsetfillcolor{currentfill}%
\pgfsetlinewidth{0.803000pt}%
\definecolor{currentstroke}{rgb}{0.000000,0.000000,0.000000}%
\pgfsetstrokecolor{currentstroke}%
\pgfsetdash{}{0pt}%
\pgfsys@defobject{currentmarker}{\pgfqpoint{-0.048611in}{0.000000in}}{\pgfqpoint{-0.000000in}{0.000000in}}{%
\pgfpathmoveto{\pgfqpoint{-0.000000in}{0.000000in}}%
\pgfpathlineto{\pgfqpoint{-0.048611in}{0.000000in}}%
\pgfusepath{stroke,fill}%
}%
\begin{pgfscope}%
\pgfsys@transformshift{0.854460in}{1.224808in}%
\pgfsys@useobject{currentmarker}{}%
\end{pgfscope}%
\end{pgfscope}%
\begin{pgfscope}%
\definecolor{textcolor}{rgb}{0.000000,0.000000,0.000000}%
\pgfsetstrokecolor{textcolor}%
\pgfsetfillcolor{textcolor}%
\pgftext[x=0.339968in, y=1.172046in, left, base]{\color{textcolor}\sffamily\fontsize{10.000000}{12.000000}\selectfont \ensuremath{-}0.75}%
\end{pgfscope}%
\begin{pgfscope}%
\pgfsetbuttcap%
\pgfsetroundjoin%
\definecolor{currentfill}{rgb}{0.000000,0.000000,0.000000}%
\pgfsetfillcolor{currentfill}%
\pgfsetlinewidth{0.803000pt}%
\definecolor{currentstroke}{rgb}{0.000000,0.000000,0.000000}%
\pgfsetstrokecolor{currentstroke}%
\pgfsetdash{}{0pt}%
\pgfsys@defobject{currentmarker}{\pgfqpoint{-0.048611in}{0.000000in}}{\pgfqpoint{-0.000000in}{0.000000in}}{%
\pgfpathmoveto{\pgfqpoint{-0.000000in}{0.000000in}}%
\pgfpathlineto{\pgfqpoint{-0.048611in}{0.000000in}}%
\pgfusepath{stroke,fill}%
}%
\begin{pgfscope}%
\pgfsys@transformshift{0.854460in}{1.878012in}%
\pgfsys@useobject{currentmarker}{}%
\end{pgfscope}%
\end{pgfscope}%
\begin{pgfscope}%
\definecolor{textcolor}{rgb}{0.000000,0.000000,0.000000}%
\pgfsetstrokecolor{textcolor}%
\pgfsetfillcolor{textcolor}%
\pgftext[x=0.339968in, y=1.825251in, left, base]{\color{textcolor}\sffamily\fontsize{10.000000}{12.000000}\selectfont \ensuremath{-}0.50}%
\end{pgfscope}%
\begin{pgfscope}%
\pgfsetbuttcap%
\pgfsetroundjoin%
\definecolor{currentfill}{rgb}{0.000000,0.000000,0.000000}%
\pgfsetfillcolor{currentfill}%
\pgfsetlinewidth{0.803000pt}%
\definecolor{currentstroke}{rgb}{0.000000,0.000000,0.000000}%
\pgfsetstrokecolor{currentstroke}%
\pgfsetdash{}{0pt}%
\pgfsys@defobject{currentmarker}{\pgfqpoint{-0.048611in}{0.000000in}}{\pgfqpoint{-0.000000in}{0.000000in}}{%
\pgfpathmoveto{\pgfqpoint{-0.000000in}{0.000000in}}%
\pgfpathlineto{\pgfqpoint{-0.048611in}{0.000000in}}%
\pgfusepath{stroke,fill}%
}%
\begin{pgfscope}%
\pgfsys@transformshift{0.854460in}{2.531217in}%
\pgfsys@useobject{currentmarker}{}%
\end{pgfscope}%
\end{pgfscope}%
\begin{pgfscope}%
\definecolor{textcolor}{rgb}{0.000000,0.000000,0.000000}%
\pgfsetstrokecolor{textcolor}%
\pgfsetfillcolor{textcolor}%
\pgftext[x=0.339968in, y=2.478455in, left, base]{\color{textcolor}\sffamily\fontsize{10.000000}{12.000000}\selectfont \ensuremath{-}0.25}%
\end{pgfscope}%
\begin{pgfscope}%
\pgfsetbuttcap%
\pgfsetroundjoin%
\definecolor{currentfill}{rgb}{0.000000,0.000000,0.000000}%
\pgfsetfillcolor{currentfill}%
\pgfsetlinewidth{0.803000pt}%
\definecolor{currentstroke}{rgb}{0.000000,0.000000,0.000000}%
\pgfsetstrokecolor{currentstroke}%
\pgfsetdash{}{0pt}%
\pgfsys@defobject{currentmarker}{\pgfqpoint{-0.048611in}{0.000000in}}{\pgfqpoint{-0.000000in}{0.000000in}}{%
\pgfpathmoveto{\pgfqpoint{-0.000000in}{0.000000in}}%
\pgfpathlineto{\pgfqpoint{-0.048611in}{0.000000in}}%
\pgfusepath{stroke,fill}%
}%
\begin{pgfscope}%
\pgfsys@transformshift{0.854460in}{3.184421in}%
\pgfsys@useobject{currentmarker}{}%
\end{pgfscope}%
\end{pgfscope}%
\begin{pgfscope}%
\definecolor{textcolor}{rgb}{0.000000,0.000000,0.000000}%
\pgfsetstrokecolor{textcolor}%
\pgfsetfillcolor{textcolor}%
\pgftext[x=0.447993in, y=3.131659in, left, base]{\color{textcolor}\sffamily\fontsize{10.000000}{12.000000}\selectfont 0.00}%
\end{pgfscope}%
\begin{pgfscope}%
\pgfsetbuttcap%
\pgfsetroundjoin%
\definecolor{currentfill}{rgb}{0.000000,0.000000,0.000000}%
\pgfsetfillcolor{currentfill}%
\pgfsetlinewidth{0.803000pt}%
\definecolor{currentstroke}{rgb}{0.000000,0.000000,0.000000}%
\pgfsetstrokecolor{currentstroke}%
\pgfsetdash{}{0pt}%
\pgfsys@defobject{currentmarker}{\pgfqpoint{-0.048611in}{0.000000in}}{\pgfqpoint{-0.000000in}{0.000000in}}{%
\pgfpathmoveto{\pgfqpoint{-0.000000in}{0.000000in}}%
\pgfpathlineto{\pgfqpoint{-0.048611in}{0.000000in}}%
\pgfusepath{stroke,fill}%
}%
\begin{pgfscope}%
\pgfsys@transformshift{0.854460in}{3.837625in}%
\pgfsys@useobject{currentmarker}{}%
\end{pgfscope}%
\end{pgfscope}%
\begin{pgfscope}%
\definecolor{textcolor}{rgb}{0.000000,0.000000,0.000000}%
\pgfsetstrokecolor{textcolor}%
\pgfsetfillcolor{textcolor}%
\pgftext[x=0.447993in, y=3.784864in, left, base]{\color{textcolor}\sffamily\fontsize{10.000000}{12.000000}\selectfont 0.25}%
\end{pgfscope}%
\begin{pgfscope}%
\pgfsetbuttcap%
\pgfsetroundjoin%
\definecolor{currentfill}{rgb}{0.000000,0.000000,0.000000}%
\pgfsetfillcolor{currentfill}%
\pgfsetlinewidth{0.803000pt}%
\definecolor{currentstroke}{rgb}{0.000000,0.000000,0.000000}%
\pgfsetstrokecolor{currentstroke}%
\pgfsetdash{}{0pt}%
\pgfsys@defobject{currentmarker}{\pgfqpoint{-0.048611in}{0.000000in}}{\pgfqpoint{-0.000000in}{0.000000in}}{%
\pgfpathmoveto{\pgfqpoint{-0.000000in}{0.000000in}}%
\pgfpathlineto{\pgfqpoint{-0.048611in}{0.000000in}}%
\pgfusepath{stroke,fill}%
}%
\begin{pgfscope}%
\pgfsys@transformshift{0.854460in}{4.490830in}%
\pgfsys@useobject{currentmarker}{}%
\end{pgfscope}%
\end{pgfscope}%
\begin{pgfscope}%
\definecolor{textcolor}{rgb}{0.000000,0.000000,0.000000}%
\pgfsetstrokecolor{textcolor}%
\pgfsetfillcolor{textcolor}%
\pgftext[x=0.447993in, y=4.438068in, left, base]{\color{textcolor}\sffamily\fontsize{10.000000}{12.000000}\selectfont 0.50}%
\end{pgfscope}%
\begin{pgfscope}%
\pgfsetbuttcap%
\pgfsetroundjoin%
\definecolor{currentfill}{rgb}{0.000000,0.000000,0.000000}%
\pgfsetfillcolor{currentfill}%
\pgfsetlinewidth{0.803000pt}%
\definecolor{currentstroke}{rgb}{0.000000,0.000000,0.000000}%
\pgfsetstrokecolor{currentstroke}%
\pgfsetdash{}{0pt}%
\pgfsys@defobject{currentmarker}{\pgfqpoint{-0.048611in}{0.000000in}}{\pgfqpoint{-0.000000in}{0.000000in}}{%
\pgfpathmoveto{\pgfqpoint{-0.000000in}{0.000000in}}%
\pgfpathlineto{\pgfqpoint{-0.048611in}{0.000000in}}%
\pgfusepath{stroke,fill}%
}%
\begin{pgfscope}%
\pgfsys@transformshift{0.854460in}{5.144034in}%
\pgfsys@useobject{currentmarker}{}%
\end{pgfscope}%
\end{pgfscope}%
\begin{pgfscope}%
\definecolor{textcolor}{rgb}{0.000000,0.000000,0.000000}%
\pgfsetstrokecolor{textcolor}%
\pgfsetfillcolor{textcolor}%
\pgftext[x=0.447993in, y=5.091273in, left, base]{\color{textcolor}\sffamily\fontsize{10.000000}{12.000000}\selectfont 0.75}%
\end{pgfscope}%
\begin{pgfscope}%
\pgfsetbuttcap%
\pgfsetroundjoin%
\definecolor{currentfill}{rgb}{0.000000,0.000000,0.000000}%
\pgfsetfillcolor{currentfill}%
\pgfsetlinewidth{0.803000pt}%
\definecolor{currentstroke}{rgb}{0.000000,0.000000,0.000000}%
\pgfsetstrokecolor{currentstroke}%
\pgfsetdash{}{0pt}%
\pgfsys@defobject{currentmarker}{\pgfqpoint{-0.048611in}{0.000000in}}{\pgfqpoint{-0.000000in}{0.000000in}}{%
\pgfpathmoveto{\pgfqpoint{-0.000000in}{0.000000in}}%
\pgfpathlineto{\pgfqpoint{-0.048611in}{0.000000in}}%
\pgfusepath{stroke,fill}%
}%
\begin{pgfscope}%
\pgfsys@transformshift{0.854460in}{5.797238in}%
\pgfsys@useobject{currentmarker}{}%
\end{pgfscope}%
\end{pgfscope}%
\begin{pgfscope}%
\definecolor{textcolor}{rgb}{0.000000,0.000000,0.000000}%
\pgfsetstrokecolor{textcolor}%
\pgfsetfillcolor{textcolor}%
\pgftext[x=0.447993in, y=5.744477in, left, base]{\color{textcolor}\sffamily\fontsize{10.000000}{12.000000}\selectfont 1.00}%
\end{pgfscope}%
\begin{pgfscope}%
\definecolor{textcolor}{rgb}{0.000000,0.000000,0.000000}%
\pgfsetstrokecolor{textcolor}%
\pgfsetfillcolor{textcolor}%
\pgftext[x=0.284413in,y=3.184421in,,bottom,rotate=90.000000]{\color{textcolor}\sffamily\fontsize{10.000000}{12.000000}\selectfont y}%
\end{pgfscope}%
\begin{pgfscope}%
\pgfpathrectangle{\pgfqpoint{0.854460in}{0.571603in}}{\pgfqpoint{5.885100in}{5.225635in}}%
\pgfusepath{clip}%
\pgfsetbuttcap%
\pgfsetroundjoin%
\pgfsetlinewidth{1.505625pt}%
\definecolor{currentstroke}{rgb}{0.273809,0.031497,0.358853}%
\pgfsetstrokecolor{currentstroke}%
\pgfsetdash{}{0pt}%
\pgfpathmoveto{\pgfqpoint{4.070867in}{1.241281in}}%
\pgfpathlineto{\pgfqpoint{4.125782in}{1.175571in}}%
\pgfpathlineto{\pgfqpoint{4.168246in}{1.123052in}}%
\pgfpathlineto{\pgfqpoint{4.228536in}{1.044274in}}%
\pgfpathlineto{\pgfqpoint{4.284278in}{0.965495in}}%
\pgfpathlineto{\pgfqpoint{4.318017in}{0.912976in}}%
\pgfpathlineto{\pgfqpoint{4.348371in}{0.860458in}}%
\pgfpathlineto{\pgfqpoint{4.374350in}{0.807939in}}%
\pgfpathlineto{\pgfqpoint{4.384843in}{0.781679in}}%
\pgfpathlineto{\pgfqpoint{4.393739in}{0.755420in}}%
\pgfpathlineto{\pgfqpoint{4.400495in}{0.729160in}}%
\pgfpathlineto{\pgfqpoint{4.404221in}{0.702901in}}%
\pgfpathlineto{\pgfqpoint{4.403264in}{0.673127in}}%
\pgfpathlineto{\pgfqpoint{4.397683in}{0.650382in}}%
\pgfpathlineto{\pgfqpoint{4.382086in}{0.624122in}}%
\pgfpathlineto{\pgfqpoint{4.373691in}{0.615959in}}%
\pgfpathlineto{\pgfqpoint{4.344118in}{0.597011in}}%
\pgfpathlineto{\pgfqpoint{4.314544in}{0.588172in}}%
\pgfpathlineto{\pgfqpoint{4.284971in}{0.583610in}}%
\pgfpathlineto{\pgfqpoint{4.255398in}{0.582147in}}%
\pgfpathlineto{\pgfqpoint{4.225824in}{0.583004in}}%
\pgfpathlineto{\pgfqpoint{4.196251in}{0.585645in}}%
\pgfpathlineto{\pgfqpoint{4.166677in}{0.589688in}}%
\pgfpathlineto{\pgfqpoint{4.122964in}{0.597863in}}%
\pgfpathlineto{\pgfqpoint{4.077957in}{0.608820in}}%
\pgfpathlineto{\pgfqpoint{4.018811in}{0.625812in}}%
\pgfpathlineto{\pgfqpoint{3.948066in}{0.650382in}}%
\pgfpathlineto{\pgfqpoint{3.881868in}{0.676641in}}%
\pgfpathlineto{\pgfqpoint{3.822205in}{0.702901in}}%
\pgfpathlineto{\pgfqpoint{3.767335in}{0.729160in}}%
\pgfpathlineto{\pgfqpoint{3.716076in}{0.755420in}}%
\pgfpathlineto{\pgfqpoint{3.634357in}{0.801055in}}%
\pgfpathlineto{\pgfqpoint{3.575210in}{0.836798in}}%
\pgfpathlineto{\pgfqpoint{3.499258in}{0.886717in}}%
\pgfpathlineto{\pgfqpoint{3.456917in}{0.916419in}}%
\pgfpathlineto{\pgfqpoint{3.391535in}{0.965495in}}%
\pgfpathlineto{\pgfqpoint{3.327111in}{1.018014in}}%
\pgfpathlineto{\pgfqpoint{3.267775in}{1.070533in}}%
\pgfpathlineto{\pgfqpoint{3.213147in}{1.123052in}}%
\pgfpathlineto{\pgfqpoint{3.161183in}{1.177492in}}%
\pgfpathlineto{\pgfqpoint{3.117242in}{1.228090in}}%
\pgfpathlineto{\pgfqpoint{3.075310in}{1.280609in}}%
\pgfpathlineto{\pgfqpoint{3.042890in}{1.325448in}}%
\pgfpathlineto{\pgfqpoint{3.020151in}{1.359388in}}%
\pgfpathlineto{\pgfqpoint{2.988135in}{1.411906in}}%
\pgfpathlineto{\pgfqpoint{2.973705in}{1.438166in}}%
\pgfpathlineto{\pgfqpoint{2.947533in}{1.490685in}}%
\pgfpathlineto{\pgfqpoint{2.924596in}{1.544954in}}%
\pgfpathlineto{\pgfqpoint{2.907476in}{1.595723in}}%
\pgfpathlineto{\pgfqpoint{2.893639in}{1.648242in}}%
\pgfpathlineto{\pgfqpoint{2.888732in}{1.674501in}}%
\pgfpathlineto{\pgfqpoint{2.885009in}{1.700761in}}%
\pgfpathlineto{\pgfqpoint{2.882585in}{1.727020in}}%
\pgfpathlineto{\pgfqpoint{2.881588in}{1.753280in}}%
\pgfpathlineto{\pgfqpoint{2.882167in}{1.779539in}}%
\pgfpathlineto{\pgfqpoint{2.884491in}{1.805799in}}%
\pgfpathlineto{\pgfqpoint{2.888754in}{1.832058in}}%
\pgfpathlineto{\pgfqpoint{2.895198in}{1.858318in}}%
\pgfpathlineto{\pgfqpoint{2.905163in}{1.884577in}}%
\pgfpathlineto{\pgfqpoint{2.924596in}{1.920928in}}%
\pgfpathlineto{\pgfqpoint{2.936985in}{1.937096in}}%
\pgfpathlineto{\pgfqpoint{2.954169in}{1.955505in}}%
\pgfpathlineto{\pgfqpoint{2.963745in}{1.963355in}}%
\pgfpathlineto{\pgfqpoint{2.983743in}{1.977341in}}%
\pgfpathlineto{\pgfqpoint{3.013316in}{1.991481in}}%
\pgfpathlineto{\pgfqpoint{3.042890in}{1.999641in}}%
\pgfpathlineto{\pgfqpoint{3.072463in}{2.003499in}}%
\pgfpathlineto{\pgfqpoint{3.102036in}{2.003735in}}%
\pgfpathlineto{\pgfqpoint{3.131610in}{2.000894in}}%
\pgfpathlineto{\pgfqpoint{3.161183in}{1.995420in}}%
\pgfpathlineto{\pgfqpoint{3.190756in}{1.987599in}}%
\pgfpathlineto{\pgfqpoint{3.220330in}{1.977529in}}%
\pgfpathlineto{\pgfqpoint{3.255016in}{1.963355in}}%
\pgfpathlineto{\pgfqpoint{3.279476in}{1.952053in}}%
\pgfpathlineto{\pgfqpoint{3.309050in}{1.936987in}}%
\pgfpathlineto{\pgfqpoint{3.368197in}{1.902303in}}%
\pgfpathlineto{\pgfqpoint{3.427343in}{1.862827in}}%
\pgfpathlineto{\pgfqpoint{3.486490in}{1.819041in}}%
\pgfpathlineto{\pgfqpoint{3.545637in}{1.771672in}}%
\pgfpathlineto{\pgfqpoint{3.604783in}{1.721129in}}%
\pgfpathlineto{\pgfqpoint{3.684793in}{1.648242in}}%
\pgfpathlineto{\pgfqpoint{3.739690in}{1.595723in}}%
\pgfpathlineto{\pgfqpoint{3.792774in}{1.543204in}}%
\pgfpathlineto{\pgfqpoint{3.810573in}{1.525316in}}%
\pgfpathlineto{\pgfqpoint{3.810573in}{1.525316in}}%
\pgfusepath{stroke}%
\end{pgfscope}%
\begin{pgfscope}%
\pgfpathrectangle{\pgfqpoint{0.854460in}{0.571603in}}{\pgfqpoint{5.885100in}{5.225635in}}%
\pgfusepath{clip}%
\pgfsetbuttcap%
\pgfsetroundjoin%
\pgfsetlinewidth{1.505625pt}%
\definecolor{currentstroke}{rgb}{0.278791,0.062145,0.386592}%
\pgfsetstrokecolor{currentstroke}%
\pgfsetdash{}{0pt}%
\pgfpathmoveto{\pgfqpoint{3.596928in}{0.571603in}}%
\pgfpathlineto{\pgfqpoint{3.507962in}{0.624122in}}%
\pgfpathlineto{\pgfqpoint{3.424344in}{0.676641in}}%
\pgfpathlineto{\pgfqpoint{3.338623in}{0.734146in}}%
\pgfpathlineto{\pgfqpoint{3.271786in}{0.781679in}}%
\pgfpathlineto{\pgfqpoint{3.190756in}{0.843047in}}%
\pgfpathlineto{\pgfqpoint{3.131610in}{0.890565in}}%
\pgfpathlineto{\pgfqpoint{3.072463in}{0.940814in}}%
\pgfpathlineto{\pgfqpoint{3.013316in}{0.994202in}}%
\pgfpathlineto{\pgfqpoint{2.954169in}{1.051179in}}%
\pgfpathlineto{\pgfqpoint{2.895023in}{1.112244in}}%
\pgfpathlineto{\pgfqpoint{2.861102in}{1.149312in}}%
\pgfpathlineto{\pgfqpoint{2.815844in}{1.201831in}}%
\pgfpathlineto{\pgfqpoint{2.773448in}{1.254350in}}%
\pgfpathlineto{\pgfqpoint{2.734109in}{1.306869in}}%
\pgfpathlineto{\pgfqpoint{2.697497in}{1.359388in}}%
\pgfpathlineto{\pgfqpoint{2.663625in}{1.411906in}}%
\pgfpathlineto{\pgfqpoint{2.628862in}{1.470916in}}%
\pgfpathlineto{\pgfqpoint{2.599289in}{1.526338in}}%
\pgfpathlineto{\pgfqpoint{2.578304in}{1.569463in}}%
\pgfpathlineto{\pgfqpoint{2.555142in}{1.621982in}}%
\pgfpathlineto{\pgfqpoint{2.534494in}{1.674501in}}%
\pgfpathlineto{\pgfqpoint{2.516507in}{1.727020in}}%
\pgfpathlineto{\pgfqpoint{2.501163in}{1.779539in}}%
\pgfpathlineto{\pgfqpoint{2.488396in}{1.832058in}}%
\pgfpathlineto{\pgfqpoint{2.478251in}{1.884577in}}%
\pgfpathlineto{\pgfqpoint{2.471010in}{1.937096in}}%
\pgfpathlineto{\pgfqpoint{2.466512in}{1.989615in}}%
\pgfpathlineto{\pgfqpoint{2.464931in}{2.042134in}}%
\pgfpathlineto{\pgfqpoint{2.466447in}{2.094653in}}%
\pgfpathlineto{\pgfqpoint{2.471242in}{2.147172in}}%
\pgfpathlineto{\pgfqpoint{2.479501in}{2.199691in}}%
\pgfpathlineto{\pgfqpoint{2.485234in}{2.225950in}}%
\pgfpathlineto{\pgfqpoint{2.499924in}{2.278469in}}%
\pgfpathlineto{\pgfqpoint{2.510569in}{2.309328in}}%
\pgfpathlineto{\pgfqpoint{2.531363in}{2.357248in}}%
\pgfpathlineto{\pgfqpoint{2.544884in}{2.383507in}}%
\pgfpathlineto{\pgfqpoint{2.569716in}{2.424058in}}%
\pgfpathlineto{\pgfqpoint{2.578291in}{2.436026in}}%
\pgfpathlineto{\pgfqpoint{2.599289in}{2.463158in}}%
\pgfpathlineto{\pgfqpoint{2.628862in}{2.494724in}}%
\pgfpathlineto{\pgfqpoint{2.658436in}{2.520641in}}%
\pgfpathlineto{\pgfqpoint{2.688009in}{2.541897in}}%
\pgfpathlineto{\pgfqpoint{2.717582in}{2.559018in}}%
\pgfpathlineto{\pgfqpoint{2.747156in}{2.572694in}}%
\pgfpathlineto{\pgfqpoint{2.776729in}{2.583170in}}%
\pgfpathlineto{\pgfqpoint{2.806303in}{2.590825in}}%
\pgfpathlineto{\pgfqpoint{2.835876in}{2.595789in}}%
\pgfpathlineto{\pgfqpoint{2.865449in}{2.598257in}}%
\pgfpathlineto{\pgfqpoint{2.895023in}{2.598418in}}%
\pgfpathlineto{\pgfqpoint{2.924596in}{2.596395in}}%
\pgfpathlineto{\pgfqpoint{2.954169in}{2.592289in}}%
\pgfpathlineto{\pgfqpoint{2.983743in}{2.586165in}}%
\pgfpathlineto{\pgfqpoint{3.013316in}{2.578175in}}%
\pgfpathlineto{\pgfqpoint{3.045689in}{2.567323in}}%
\pgfpathlineto{\pgfqpoint{3.072463in}{2.556883in}}%
\pgfpathlineto{\pgfqpoint{3.107404in}{2.541064in}}%
\pgfpathlineto{\pgfqpoint{3.131610in}{2.529015in}}%
\pgfpathlineto{\pgfqpoint{3.161183in}{2.512831in}}%
\pgfpathlineto{\pgfqpoint{3.201057in}{2.488545in}}%
\pgfpathlineto{\pgfqpoint{3.249903in}{2.455987in}}%
\pgfpathlineto{\pgfqpoint{3.311829in}{2.409766in}}%
\pgfpathlineto{\pgfqpoint{3.368197in}{2.363794in}}%
\pgfpathlineto{\pgfqpoint{3.435131in}{2.304729in}}%
\pgfpathlineto{\pgfqpoint{3.491374in}{2.252210in}}%
\pgfpathlineto{\pgfqpoint{3.545637in}{2.199381in}}%
\pgfpathlineto{\pgfqpoint{3.634357in}{2.109002in}}%
\pgfpathlineto{\pgfqpoint{3.752650in}{1.983160in}}%
\pgfpathlineto{\pgfqpoint{3.913574in}{1.805799in}}%
\pgfpathlineto{\pgfqpoint{4.048384in}{1.654649in}}%
\pgfpathlineto{\pgfqpoint{4.216859in}{1.463870in}}%
\pgfpathlineto{\pgfqpoint{4.216859in}{1.463870in}}%
\pgfusepath{stroke}%
\end{pgfscope}%
\begin{pgfscope}%
\pgfpathrectangle{\pgfqpoint{0.854460in}{0.571603in}}{\pgfqpoint{5.885100in}{5.225635in}}%
\pgfusepath{clip}%
\pgfsetbuttcap%
\pgfsetroundjoin%
\pgfsetlinewidth{1.505625pt}%
\definecolor{currentstroke}{rgb}{0.278791,0.062145,0.386592}%
\pgfsetstrokecolor{currentstroke}%
\pgfsetdash{}{0pt}%
\pgfpathmoveto{\pgfqpoint{4.470801in}{1.173541in}}%
\pgfpathlineto{\pgfqpoint{4.491834in}{1.149312in}}%
\pgfpathlineto{\pgfqpoint{4.491984in}{1.149137in}}%
\pgfpathlineto{\pgfqpoint{4.514386in}{1.123052in}}%
\pgfpathlineto{\pgfqpoint{4.521558in}{1.114719in}}%
\pgfpathlineto{\pgfqpoint{4.536919in}{1.096793in}}%
\pgfpathlineto{\pgfqpoint{4.551131in}{1.080234in}}%
\pgfpathlineto{\pgfqpoint{4.559426in}{1.070533in}}%
\pgfpathlineto{\pgfqpoint{4.580705in}{1.045672in}}%
\pgfpathlineto{\pgfqpoint{4.581897in}{1.044274in}}%
\pgfpathlineto{\pgfqpoint{4.604115in}{1.018014in}}%
\pgfpathlineto{\pgfqpoint{4.610278in}{1.010713in}}%
\pgfpathlineto{\pgfqpoint{4.626239in}{0.991755in}}%
\pgfpathlineto{\pgfqpoint{4.639851in}{0.975564in}}%
\pgfpathlineto{\pgfqpoint{4.648299in}{0.965495in}}%
\pgfpathlineto{\pgfqpoint{4.669425in}{0.940263in}}%
\pgfpathlineto{\pgfqpoint{4.670283in}{0.939236in}}%
\pgfpathlineto{\pgfqpoint{4.691938in}{0.912976in}}%
\pgfpathlineto{\pgfqpoint{4.698998in}{0.904367in}}%
\pgfpathlineto{\pgfqpoint{4.713461in}{0.886717in}}%
\pgfpathlineto{\pgfqpoint{4.728571in}{0.868183in}}%
\pgfpathlineto{\pgfqpoint{4.734868in}{0.860458in}}%
\pgfpathlineto{\pgfqpoint{4.756072in}{0.834198in}}%
\pgfpathlineto{\pgfqpoint{4.758145in}{0.831582in}}%
\pgfpathlineto{\pgfqpoint{4.776894in}{0.807939in}}%
\pgfpathlineto{\pgfqpoint{4.787718in}{0.794169in}}%
\pgfpathlineto{\pgfqpoint{4.797548in}{0.781679in}}%
\pgfpathlineto{\pgfqpoint{4.817291in}{0.756347in}}%
\pgfpathlineto{\pgfqpoint{4.818015in}{0.755420in}}%
\pgfpathlineto{\pgfqpoint{4.837949in}{0.729160in}}%
\pgfpathlineto{\pgfqpoint{4.846865in}{0.717241in}}%
\pgfpathlineto{\pgfqpoint{4.857619in}{0.702901in}}%
\pgfpathlineto{\pgfqpoint{4.876438in}{0.677437in}}%
\pgfpathlineto{\pgfqpoint{4.877028in}{0.676641in}}%
\pgfpathlineto{\pgfqpoint{4.895758in}{0.650382in}}%
\pgfpathlineto{\pgfqpoint{4.906012in}{0.635703in}}%
\pgfpathlineto{\pgfqpoint{4.914136in}{0.624122in}}%
\pgfpathlineto{\pgfqpoint{4.932012in}{0.597863in}}%
\pgfpathlineto{\pgfqpoint{4.935585in}{0.592358in}}%
\pgfpathlineto{\pgfqpoint{4.949135in}{0.571603in}}%
\pgfusepath{stroke}%
\end{pgfscope}%
\begin{pgfscope}%
\pgfpathrectangle{\pgfqpoint{0.854460in}{0.571603in}}{\pgfqpoint{5.885100in}{5.225635in}}%
\pgfusepath{clip}%
\pgfsetbuttcap%
\pgfsetroundjoin%
\pgfsetlinewidth{1.505625pt}%
\definecolor{currentstroke}{rgb}{0.282327,0.094955,0.417331}%
\pgfsetstrokecolor{currentstroke}%
\pgfsetdash{}{0pt}%
\pgfpathmoveto{\pgfqpoint{3.320897in}{0.571603in}}%
\pgfpathlineto{\pgfqpoint{3.241386in}{0.624122in}}%
\pgfpathlineto{\pgfqpoint{3.161183in}{0.679973in}}%
\pgfpathlineto{\pgfqpoint{3.094084in}{0.729160in}}%
\pgfpathlineto{\pgfqpoint{3.013316in}{0.791741in}}%
\pgfpathlineto{\pgfqpoint{2.954169in}{0.840053in}}%
\pgfpathlineto{\pgfqpoint{2.895023in}{0.890818in}}%
\pgfpathlineto{\pgfqpoint{2.835876in}{0.944361in}}%
\pgfpathlineto{\pgfqpoint{2.776729in}{1.001038in}}%
\pgfpathlineto{\pgfqpoint{2.717582in}{1.061240in}}%
\pgfpathlineto{\pgfqpoint{2.684320in}{1.096793in}}%
\pgfpathlineto{\pgfqpoint{2.628862in}{1.159588in}}%
\pgfpathlineto{\pgfqpoint{2.593597in}{1.201831in}}%
\pgfpathlineto{\pgfqpoint{2.552204in}{1.254350in}}%
\pgfpathlineto{\pgfqpoint{2.510569in}{1.310614in}}%
\pgfpathlineto{\pgfqpoint{2.476815in}{1.359388in}}%
\pgfpathlineto{\pgfqpoint{2.442837in}{1.411906in}}%
\pgfpathlineto{\pgfqpoint{2.411190in}{1.464425in}}%
\pgfpathlineto{\pgfqpoint{2.381843in}{1.516944in}}%
\pgfpathlineto{\pgfqpoint{2.354752in}{1.569463in}}%
\pgfpathlineto{\pgfqpoint{2.329861in}{1.621982in}}%
\pgfpathlineto{\pgfqpoint{2.303555in}{1.683593in}}%
\pgfpathlineto{\pgfqpoint{2.286822in}{1.727020in}}%
\pgfpathlineto{\pgfqpoint{2.268455in}{1.779539in}}%
\pgfpathlineto{\pgfqpoint{2.252290in}{1.832058in}}%
\pgfpathlineto{\pgfqpoint{2.238208in}{1.884577in}}%
\pgfpathlineto{\pgfqpoint{2.226295in}{1.937096in}}%
\pgfpathlineto{\pgfqpoint{2.214835in}{1.999222in}}%
\pgfpathlineto{\pgfqpoint{2.208712in}{2.042134in}}%
\pgfpathlineto{\pgfqpoint{2.203184in}{2.094653in}}%
\pgfpathlineto{\pgfqpoint{2.199788in}{2.147172in}}%
\pgfpathlineto{\pgfqpoint{2.198587in}{2.199691in}}%
\pgfpathlineto{\pgfqpoint{2.199638in}{2.252210in}}%
\pgfpathlineto{\pgfqpoint{2.202997in}{2.304729in}}%
\pgfpathlineto{\pgfqpoint{2.208708in}{2.357248in}}%
\pgfpathlineto{\pgfqpoint{2.216899in}{2.409766in}}%
\pgfpathlineto{\pgfqpoint{2.227905in}{2.462285in}}%
\pgfpathlineto{\pgfqpoint{2.244409in}{2.524783in}}%
\pgfpathlineto{\pgfqpoint{2.258291in}{2.567323in}}%
\pgfpathlineto{\pgfqpoint{2.278191in}{2.619842in}}%
\pgfpathlineto{\pgfqpoint{2.303555in}{2.675954in}}%
\pgfpathlineto{\pgfqpoint{2.333129in}{2.730852in}}%
\pgfpathlineto{\pgfqpoint{2.362702in}{2.777810in}}%
\pgfpathlineto{\pgfqpoint{2.392275in}{2.818450in}}%
\pgfpathlineto{\pgfqpoint{2.423604in}{2.856177in}}%
\pgfpathlineto{\pgfqpoint{2.451422in}{2.885754in}}%
\pgfpathlineto{\pgfqpoint{2.480996in}{2.913618in}}%
\pgfpathlineto{\pgfqpoint{2.510569in}{2.938265in}}%
\pgfpathlineto{\pgfqpoint{2.542015in}{2.961215in}}%
\pgfpathlineto{\pgfqpoint{2.584806in}{2.987475in}}%
\pgfpathlineto{\pgfqpoint{2.599289in}{2.995471in}}%
\pgfpathlineto{\pgfqpoint{2.638922in}{3.013734in}}%
\pgfpathlineto{\pgfqpoint{2.658436in}{3.021531in}}%
\pgfpathlineto{\pgfqpoint{2.688009in}{3.031290in}}%
\pgfpathlineto{\pgfqpoint{2.722650in}{3.039994in}}%
\pgfpathlineto{\pgfqpoint{2.747156in}{3.044671in}}%
\pgfpathlineto{\pgfqpoint{2.776729in}{3.048378in}}%
\pgfpathlineto{\pgfqpoint{2.806303in}{3.050171in}}%
\pgfpathlineto{\pgfqpoint{2.835876in}{3.050081in}}%
\pgfpathlineto{\pgfqpoint{2.865449in}{3.048132in}}%
\pgfpathlineto{\pgfqpoint{2.895023in}{3.044353in}}%
\pgfpathlineto{\pgfqpoint{2.924596in}{3.038760in}}%
\pgfpathlineto{\pgfqpoint{2.954169in}{3.031344in}}%
\pgfpathlineto{\pgfqpoint{2.983743in}{3.022173in}}%
\pgfpathlineto{\pgfqpoint{3.013316in}{3.011260in}}%
\pgfpathlineto{\pgfqpoint{3.042890in}{2.998596in}}%
\pgfpathlineto{\pgfqpoint{3.072463in}{2.984253in}}%
\pgfpathlineto{\pgfqpoint{3.113769in}{2.961215in}}%
\pgfpathlineto{\pgfqpoint{3.131610in}{2.950569in}}%
\pgfpathlineto{\pgfqpoint{3.161183in}{2.931330in}}%
\pgfpathlineto{\pgfqpoint{3.193200in}{2.908696in}}%
\pgfpathlineto{\pgfqpoint{3.227570in}{2.882437in}}%
\pgfpathlineto{\pgfqpoint{3.279476in}{2.839487in}}%
\pgfpathlineto{\pgfqpoint{3.338623in}{2.785546in}}%
\pgfpathlineto{\pgfqpoint{3.399841in}{2.724880in}}%
\pgfpathlineto{\pgfqpoint{3.456917in}{2.664673in}}%
\pgfpathlineto{\pgfqpoint{3.545637in}{2.565274in}}%
\pgfpathlineto{\pgfqpoint{3.634357in}{2.460792in}}%
\pgfpathlineto{\pgfqpoint{3.783692in}{2.278469in}}%
\pgfpathlineto{\pgfqpoint{4.060099in}{1.937096in}}%
\pgfpathlineto{\pgfqpoint{4.255874in}{1.700761in}}%
\pgfpathlineto{\pgfqpoint{4.389314in}{1.543204in}}%
\pgfpathlineto{\pgfqpoint{4.580705in}{1.322359in}}%
\pgfpathlineto{\pgfqpoint{4.722109in}{1.162052in}}%
\pgfpathlineto{\pgfqpoint{4.722109in}{1.162052in}}%
\pgfusepath{stroke}%
\end{pgfscope}%
\begin{pgfscope}%
\pgfpathrectangle{\pgfqpoint{0.854460in}{0.571603in}}{\pgfqpoint{5.885100in}{5.225635in}}%
\pgfusepath{clip}%
\pgfsetbuttcap%
\pgfsetroundjoin%
\pgfsetlinewidth{1.505625pt}%
\definecolor{currentstroke}{rgb}{0.282327,0.094955,0.417331}%
\pgfsetstrokecolor{currentstroke}%
\pgfsetdash{}{0pt}%
\pgfpathmoveto{\pgfqpoint{4.978414in}{0.873986in}}%
\pgfpathlineto{\pgfqpoint{4.990401in}{0.860458in}}%
\pgfpathlineto{\pgfqpoint{4.994732in}{0.855554in}}%
\pgfpathlineto{\pgfqpoint{5.013603in}{0.834198in}}%
\pgfpathlineto{\pgfqpoint{5.024305in}{0.822067in}}%
\pgfpathlineto{\pgfqpoint{5.036783in}{0.807939in}}%
\pgfpathlineto{\pgfqpoint{5.053878in}{0.788537in}}%
\pgfpathlineto{\pgfqpoint{5.059929in}{0.781679in}}%
\pgfpathlineto{\pgfqpoint{5.083020in}{0.755420in}}%
\pgfpathlineto{\pgfqpoint{5.083452in}{0.754923in}}%
\pgfpathlineto{\pgfqpoint{5.105885in}{0.729160in}}%
\pgfpathlineto{\pgfqpoint{5.113025in}{0.720921in}}%
\pgfpathlineto{\pgfqpoint{5.128680in}{0.702901in}}%
\pgfpathlineto{\pgfqpoint{5.142599in}{0.686790in}}%
\pgfpathlineto{\pgfqpoint{5.151392in}{0.676641in}}%
\pgfpathlineto{\pgfqpoint{5.172172in}{0.652504in}}%
\pgfpathlineto{\pgfqpoint{5.174005in}{0.650382in}}%
\pgfpathlineto{\pgfqpoint{5.196355in}{0.624122in}}%
\pgfpathlineto{\pgfqpoint{5.201745in}{0.617715in}}%
\pgfpathlineto{\pgfqpoint{5.218518in}{0.597863in}}%
\pgfpathlineto{\pgfqpoint{5.231319in}{0.582563in}}%
\pgfpathlineto{\pgfqpoint{5.240530in}{0.571603in}}%
\pgfusepath{stroke}%
\end{pgfscope}%
\begin{pgfscope}%
\pgfpathrectangle{\pgfqpoint{0.854460in}{0.571603in}}{\pgfqpoint{5.885100in}{5.225635in}}%
\pgfusepath{clip}%
\pgfsetbuttcap%
\pgfsetroundjoin%
\pgfsetlinewidth{1.505625pt}%
\definecolor{currentstroke}{rgb}{0.283229,0.120777,0.440584}%
\pgfsetstrokecolor{currentstroke}%
\pgfsetdash{}{0pt}%
\pgfpathmoveto{\pgfqpoint{3.111874in}{0.571603in}}%
\pgfpathlineto{\pgfqpoint{3.036974in}{0.624122in}}%
\pgfpathlineto{\pgfqpoint{2.954169in}{0.685284in}}%
\pgfpathlineto{\pgfqpoint{2.895023in}{0.730992in}}%
\pgfpathlineto{\pgfqpoint{2.832351in}{0.781679in}}%
\pgfpathlineto{\pgfqpoint{2.770439in}{0.834198in}}%
\pgfpathlineto{\pgfqpoint{2.711451in}{0.886717in}}%
\pgfpathlineto{\pgfqpoint{2.655289in}{0.939236in}}%
\pgfpathlineto{\pgfqpoint{2.599289in}{0.994422in}}%
\pgfpathlineto{\pgfqpoint{2.540142in}{1.056223in}}%
\pgfpathlineto{\pgfqpoint{2.503165in}{1.096793in}}%
\pgfpathlineto{\pgfqpoint{2.451422in}{1.156674in}}%
\pgfpathlineto{\pgfqpoint{2.414477in}{1.201831in}}%
\pgfpathlineto{\pgfqpoint{2.362702in}{1.269182in}}%
\pgfpathlineto{\pgfqpoint{2.333129in}{1.309959in}}%
\pgfpathlineto{\pgfqpoint{2.299196in}{1.359388in}}%
\pgfpathlineto{\pgfqpoint{2.265319in}{1.411906in}}%
\pgfpathlineto{\pgfqpoint{2.233586in}{1.464425in}}%
\pgfpathlineto{\pgfqpoint{2.203964in}{1.516944in}}%
\pgfpathlineto{\pgfqpoint{2.176410in}{1.569463in}}%
\pgfpathlineto{\pgfqpoint{2.150874in}{1.621982in}}%
\pgfpathlineto{\pgfqpoint{2.126115in}{1.677374in}}%
\pgfpathlineto{\pgfqpoint{2.095749in}{1.753280in}}%
\pgfpathlineto{\pgfqpoint{2.068520in}{1.832058in}}%
\pgfpathlineto{\pgfqpoint{2.052817in}{1.884577in}}%
\pgfpathlineto{\pgfqpoint{2.037395in}{1.942959in}}%
\pgfpathlineto{\pgfqpoint{2.026824in}{1.989615in}}%
\pgfpathlineto{\pgfqpoint{2.016594in}{2.042134in}}%
\pgfpathlineto{\pgfqpoint{2.007822in}{2.096853in}}%
\pgfpathlineto{\pgfqpoint{2.001639in}{2.147172in}}%
\pgfpathlineto{\pgfqpoint{1.996955in}{2.199691in}}%
\pgfpathlineto{\pgfqpoint{1.994088in}{2.252210in}}%
\pgfpathlineto{\pgfqpoint{1.993068in}{2.304729in}}%
\pgfpathlineto{\pgfqpoint{1.993917in}{2.357248in}}%
\pgfpathlineto{\pgfqpoint{1.996657in}{2.409766in}}%
\pgfpathlineto{\pgfqpoint{2.001300in}{2.462285in}}%
\pgfpathlineto{\pgfqpoint{2.007856in}{2.514804in}}%
\pgfpathlineto{\pgfqpoint{2.016639in}{2.567323in}}%
\pgfpathlineto{\pgfqpoint{2.027408in}{2.619842in}}%
\pgfpathlineto{\pgfqpoint{2.040259in}{2.672361in}}%
\pgfpathlineto{\pgfqpoint{2.055541in}{2.724880in}}%
\pgfpathlineto{\pgfqpoint{2.073079in}{2.777399in}}%
\pgfpathlineto{\pgfqpoint{2.096542in}{2.838179in}}%
\pgfpathlineto{\pgfqpoint{2.116068in}{2.882437in}}%
\pgfpathlineto{\pgfqpoint{2.141871in}{2.934956in}}%
\pgfpathlineto{\pgfqpoint{2.170909in}{2.987475in}}%
\pgfpathlineto{\pgfqpoint{2.186612in}{3.013734in}}%
\pgfpathlineto{\pgfqpoint{2.221387in}{3.066253in}}%
\pgfpathlineto{\pgfqpoint{2.260633in}{3.118772in}}%
\pgfpathlineto{\pgfqpoint{2.282138in}{3.145032in}}%
\pgfpathlineto{\pgfqpoint{2.305035in}{3.171291in}}%
\pgfpathlineto{\pgfqpoint{2.333129in}{3.201024in}}%
\pgfpathlineto{\pgfqpoint{2.362702in}{3.229777in}}%
\pgfpathlineto{\pgfqpoint{2.392275in}{3.256177in}}%
\pgfpathlineto{\pgfqpoint{2.421849in}{3.280394in}}%
\pgfpathlineto{\pgfqpoint{2.451447in}{3.302589in}}%
\pgfpathlineto{\pgfqpoint{2.490766in}{3.328848in}}%
\pgfpathlineto{\pgfqpoint{2.510569in}{3.341074in}}%
\pgfpathlineto{\pgfqpoint{2.540142in}{3.357678in}}%
\pgfpathlineto{\pgfqpoint{2.589522in}{3.381367in}}%
\pgfpathlineto{\pgfqpoint{2.599289in}{3.385693in}}%
\pgfpathlineto{\pgfqpoint{2.628862in}{3.397209in}}%
\pgfpathlineto{\pgfqpoint{2.660050in}{3.407626in}}%
\pgfpathlineto{\pgfqpoint{2.688009in}{3.415462in}}%
\pgfpathlineto{\pgfqpoint{2.717582in}{3.422196in}}%
\pgfpathlineto{\pgfqpoint{2.747156in}{3.427364in}}%
\pgfpathlineto{\pgfqpoint{2.776729in}{3.430951in}}%
\pgfpathlineto{\pgfqpoint{2.806303in}{3.432943in}}%
\pgfpathlineto{\pgfqpoint{2.835876in}{3.433324in}}%
\pgfpathlineto{\pgfqpoint{2.865449in}{3.432078in}}%
\pgfpathlineto{\pgfqpoint{2.895023in}{3.429188in}}%
\pgfpathlineto{\pgfqpoint{2.924596in}{3.424637in}}%
\pgfpathlineto{\pgfqpoint{2.954169in}{3.418408in}}%
\pgfpathlineto{\pgfqpoint{2.992540in}{3.407626in}}%
\pgfpathlineto{\pgfqpoint{3.013316in}{3.400794in}}%
\pgfpathlineto{\pgfqpoint{3.060808in}{3.381367in}}%
\pgfpathlineto{\pgfqpoint{3.072463in}{3.376120in}}%
\pgfpathlineto{\pgfqpoint{3.112557in}{3.355107in}}%
\pgfpathlineto{\pgfqpoint{3.131610in}{3.344204in}}%
\pgfpathlineto{\pgfqpoint{3.161183in}{3.325509in}}%
\pgfpathlineto{\pgfqpoint{3.193920in}{3.302589in}}%
\pgfpathlineto{\pgfqpoint{3.228022in}{3.276329in}}%
\pgfpathlineto{\pgfqpoint{3.259437in}{3.250070in}}%
\pgfpathlineto{\pgfqpoint{3.288731in}{3.223810in}}%
\pgfpathlineto{\pgfqpoint{3.316328in}{3.197551in}}%
\pgfpathlineto{\pgfqpoint{3.342552in}{3.171291in}}%
\pgfpathlineto{\pgfqpoint{3.397770in}{3.111951in}}%
\pgfpathlineto{\pgfqpoint{3.459181in}{3.039994in}}%
\pgfpathlineto{\pgfqpoint{3.521898in}{2.961215in}}%
\pgfpathlineto{\pgfqpoint{3.581555in}{2.882437in}}%
\pgfpathlineto{\pgfqpoint{3.663930in}{2.769436in}}%
\pgfpathlineto{\pgfqpoint{3.824952in}{2.541543in}}%
\pgfpathlineto{\pgfqpoint{3.824952in}{2.541543in}}%
\pgfusepath{stroke}%
\end{pgfscope}%
\begin{pgfscope}%
\pgfpathrectangle{\pgfqpoint{0.854460in}{0.571603in}}{\pgfqpoint{5.885100in}{5.225635in}}%
\pgfusepath{clip}%
\pgfsetbuttcap%
\pgfsetroundjoin%
\pgfsetlinewidth{1.505625pt}%
\definecolor{currentstroke}{rgb}{0.283229,0.120777,0.440584}%
\pgfsetstrokecolor{currentstroke}%
\pgfsetdash{}{0pt}%
\pgfpathmoveto{\pgfqpoint{4.050252in}{2.226569in}}%
\pgfpathlineto{\pgfqpoint{4.050698in}{2.225950in}}%
\pgfpathlineto{\pgfqpoint{4.069901in}{2.199691in}}%
\pgfpathlineto{\pgfqpoint{4.077957in}{2.188846in}}%
\pgfpathlineto{\pgfqpoint{4.089233in}{2.173431in}}%
\pgfpathlineto{\pgfqpoint{4.107531in}{2.148836in}}%
\pgfpathlineto{\pgfqpoint{4.108750in}{2.147172in}}%
\pgfpathlineto{\pgfqpoint{4.128230in}{2.120912in}}%
\pgfpathlineto{\pgfqpoint{4.137104in}{2.109138in}}%
\pgfpathlineto{\pgfqpoint{4.147868in}{2.094653in}}%
\pgfpathlineto{\pgfqpoint{4.166677in}{2.069756in}}%
\pgfpathlineto{\pgfqpoint{4.167693in}{2.068393in}}%
\pgfpathlineto{\pgfqpoint{4.187488in}{2.042134in}}%
\pgfpathlineto{\pgfqpoint{4.196251in}{2.030690in}}%
\pgfpathlineto{\pgfqpoint{4.207448in}{2.015874in}}%
\pgfpathlineto{\pgfqpoint{4.225824in}{1.991948in}}%
\pgfpathlineto{\pgfqpoint{4.227594in}{1.989615in}}%
\pgfpathlineto{\pgfqpoint{4.247739in}{1.963355in}}%
\pgfpathlineto{\pgfqpoint{4.255398in}{1.953518in}}%
\pgfpathlineto{\pgfqpoint{4.268030in}{1.937096in}}%
\pgfpathlineto{\pgfqpoint{4.284971in}{1.915412in}}%
\pgfpathlineto{\pgfqpoint{4.288505in}{1.910836in}}%
\pgfpathlineto{\pgfqpoint{4.309031in}{1.884577in}}%
\pgfpathlineto{\pgfqpoint{4.314544in}{1.877614in}}%
\pgfpathlineto{\pgfqpoint{4.329657in}{1.858318in}}%
\pgfpathlineto{\pgfqpoint{4.344118in}{1.840127in}}%
\pgfpathlineto{\pgfqpoint{4.350466in}{1.832058in}}%
\pgfpathlineto{\pgfqpoint{4.371399in}{1.805799in}}%
\pgfpathlineto{\pgfqpoint{4.373691in}{1.802953in}}%
\pgfpathlineto{\pgfqpoint{4.392363in}{1.779539in}}%
\pgfpathlineto{\pgfqpoint{4.403264in}{1.766061in}}%
\pgfpathlineto{\pgfqpoint{4.413505in}{1.753280in}}%
\pgfpathlineto{\pgfqpoint{4.432838in}{1.729485in}}%
\pgfpathlineto{\pgfqpoint{4.434822in}{1.727020in}}%
\pgfpathlineto{\pgfqpoint{4.456168in}{1.700761in}}%
\pgfpathlineto{\pgfqpoint{4.462411in}{1.693173in}}%
\pgfpathlineto{\pgfqpoint{4.477643in}{1.674501in}}%
\pgfpathlineto{\pgfqpoint{4.491984in}{1.657149in}}%
\pgfpathlineto{\pgfqpoint{4.499287in}{1.648242in}}%
\pgfpathlineto{\pgfqpoint{4.521087in}{1.621982in}}%
\pgfpathlineto{\pgfqpoint{4.521558in}{1.621419in}}%
\pgfpathlineto{\pgfqpoint{4.542887in}{1.595723in}}%
\pgfpathlineto{\pgfqpoint{4.551131in}{1.585911in}}%
\pgfpathlineto{\pgfqpoint{4.564852in}{1.569463in}}%
\pgfpathlineto{\pgfqpoint{4.580705in}{1.550686in}}%
\pgfpathlineto{\pgfqpoint{4.586978in}{1.543204in}}%
\pgfpathlineto{\pgfqpoint{4.609238in}{1.516944in}}%
\pgfpathlineto{\pgfqpoint{4.610278in}{1.515726in}}%
\pgfpathlineto{\pgfqpoint{4.631514in}{1.490685in}}%
\pgfpathlineto{\pgfqpoint{4.639851in}{1.480962in}}%
\pgfpathlineto{\pgfqpoint{4.653945in}{1.464425in}}%
\pgfpathlineto{\pgfqpoint{4.669425in}{1.446458in}}%
\pgfpathlineto{\pgfqpoint{4.676527in}{1.438166in}}%
\pgfpathlineto{\pgfqpoint{4.698998in}{1.412207in}}%
\pgfpathlineto{\pgfqpoint{4.699256in}{1.411906in}}%
\pgfpathlineto{\pgfqpoint{4.721985in}{1.385647in}}%
\pgfpathlineto{\pgfqpoint{4.728571in}{1.378109in}}%
\pgfpathlineto{\pgfqpoint{4.744851in}{1.359388in}}%
\pgfpathlineto{\pgfqpoint{4.758145in}{1.344243in}}%
\pgfpathlineto{\pgfqpoint{4.767857in}{1.333128in}}%
\pgfpathlineto{\pgfqpoint{4.787718in}{1.310604in}}%
\pgfpathlineto{\pgfqpoint{4.790998in}{1.306869in}}%
\pgfpathlineto{\pgfqpoint{4.814203in}{1.280609in}}%
\pgfpathlineto{\pgfqpoint{4.817291in}{1.277134in}}%
\pgfpathlineto{\pgfqpoint{4.837466in}{1.254350in}}%
\pgfpathlineto{\pgfqpoint{4.846865in}{1.243819in}}%
\pgfpathlineto{\pgfqpoint{4.860855in}{1.228090in}}%
\pgfpathlineto{\pgfqpoint{4.876438in}{1.210704in}}%
\pgfpathlineto{\pgfqpoint{4.884366in}{1.201831in}}%
\pgfpathlineto{\pgfqpoint{4.906012in}{1.177781in}}%
\pgfpathlineto{\pgfqpoint{4.907995in}{1.175571in}}%
\pgfpathlineto{\pgfqpoint{4.931650in}{1.149312in}}%
\pgfpathlineto{\pgfqpoint{4.935585in}{1.144965in}}%
\pgfpathlineto{\pgfqpoint{4.955373in}{1.123052in}}%
\pgfpathlineto{\pgfqpoint{4.965158in}{1.112282in}}%
\pgfpathlineto{\pgfqpoint{4.979202in}{1.096793in}}%
\pgfpathlineto{\pgfqpoint{4.994732in}{1.079762in}}%
\pgfpathlineto{\pgfqpoint{5.003133in}{1.070533in}}%
\pgfpathlineto{\pgfqpoint{5.024305in}{1.047397in}}%
\pgfpathlineto{\pgfqpoint{5.027159in}{1.044274in}}%
\pgfpathlineto{\pgfqpoint{5.051217in}{1.018014in}}%
\pgfpathlineto{\pgfqpoint{5.053878in}{1.015115in}}%
\pgfpathlineto{\pgfqpoint{5.075300in}{0.991755in}}%
\pgfpathlineto{\pgfqpoint{5.083452in}{0.982900in}}%
\pgfpathlineto{\pgfqpoint{5.099464in}{0.965495in}}%
\pgfpathlineto{\pgfqpoint{5.113025in}{0.950807in}}%
\pgfpathlineto{\pgfqpoint{5.123704in}{0.939236in}}%
\pgfpathlineto{\pgfqpoint{5.142599in}{0.918826in}}%
\pgfpathlineto{\pgfqpoint{5.148014in}{0.912976in}}%
\pgfpathlineto{\pgfqpoint{5.172172in}{0.886948in}}%
\pgfpathlineto{\pgfqpoint{5.172387in}{0.886717in}}%
\pgfpathlineto{\pgfqpoint{5.196705in}{0.860458in}}%
\pgfpathlineto{\pgfqpoint{5.201745in}{0.855021in}}%
\pgfpathlineto{\pgfqpoint{5.221069in}{0.834198in}}%
\pgfpathlineto{\pgfqpoint{5.231319in}{0.823164in}}%
\pgfpathlineto{\pgfqpoint{5.245477in}{0.807939in}}%
\pgfpathlineto{\pgfqpoint{5.260892in}{0.791370in}}%
\pgfpathlineto{\pgfqpoint{5.269921in}{0.781679in}}%
\pgfpathlineto{\pgfqpoint{5.290465in}{0.759629in}}%
\pgfpathlineto{\pgfqpoint{5.294394in}{0.755420in}}%
\pgfpathlineto{\pgfqpoint{5.318860in}{0.729160in}}%
\pgfpathlineto{\pgfqpoint{5.320039in}{0.727888in}}%
\pgfpathlineto{\pgfqpoint{5.343246in}{0.702901in}}%
\pgfpathlineto{\pgfqpoint{5.349612in}{0.696033in}}%
\pgfpathlineto{\pgfqpoint{5.367635in}{0.676641in}}%
\pgfpathlineto{\pgfqpoint{5.379185in}{0.664181in}}%
\pgfpathlineto{\pgfqpoint{5.392017in}{0.650382in}}%
\pgfpathlineto{\pgfqpoint{5.408759in}{0.632318in}}%
\pgfpathlineto{\pgfqpoint{5.416380in}{0.624122in}}%
\pgfpathlineto{\pgfqpoint{5.438332in}{0.600426in}}%
\pgfpathlineto{\pgfqpoint{5.440715in}{0.597863in}}%
\pgfpathlineto{\pgfqpoint{5.464941in}{0.571603in}}%
\pgfusepath{stroke}%
\end{pgfscope}%
\begin{pgfscope}%
\pgfpathrectangle{\pgfqpoint{0.854460in}{0.571603in}}{\pgfqpoint{5.885100in}{5.225635in}}%
\pgfusepath{clip}%
\pgfsetbuttcap%
\pgfsetroundjoin%
\pgfsetlinewidth{1.505625pt}%
\definecolor{currentstroke}{rgb}{0.281887,0.150881,0.465405}%
\pgfsetstrokecolor{currentstroke}%
\pgfsetdash{}{0pt}%
\pgfpathmoveto{\pgfqpoint{2.937449in}{0.571603in}}%
\pgfpathlineto{\pgfqpoint{2.865449in}{0.624148in}}%
\pgfpathlineto{\pgfqpoint{2.776729in}{0.692488in}}%
\pgfpathlineto{\pgfqpoint{2.717582in}{0.740298in}}%
\pgfpathlineto{\pgfqpoint{2.658436in}{0.790164in}}%
\pgfpathlineto{\pgfqpoint{2.599289in}{0.842331in}}%
\pgfpathlineto{\pgfqpoint{2.540142in}{0.897064in}}%
\pgfpathlineto{\pgfqpoint{2.480996in}{0.954654in}}%
\pgfpathlineto{\pgfqpoint{2.421849in}{1.015418in}}%
\pgfpathlineto{\pgfqpoint{2.392275in}{1.047209in}}%
\pgfpathlineto{\pgfqpoint{2.333129in}{1.113940in}}%
\pgfpathlineto{\pgfqpoint{2.303200in}{1.149312in}}%
\pgfpathlineto{\pgfqpoint{2.244409in}{1.223053in}}%
\pgfpathlineto{\pgfqpoint{2.214835in}{1.262397in}}%
\pgfpathlineto{\pgfqpoint{2.182802in}{1.306869in}}%
\pgfpathlineto{\pgfqpoint{2.147058in}{1.359388in}}%
\pgfpathlineto{\pgfqpoint{2.113369in}{1.411906in}}%
\pgfpathlineto{\pgfqpoint{2.081708in}{1.464425in}}%
\pgfpathlineto{\pgfqpoint{2.052039in}{1.516944in}}%
\pgfpathlineto{\pgfqpoint{2.024323in}{1.569463in}}%
\pgfpathlineto{\pgfqpoint{1.998509in}{1.621982in}}%
\pgfpathlineto{\pgfqpoint{1.974542in}{1.674501in}}%
\pgfpathlineto{\pgfqpoint{1.948675in}{1.736538in}}%
\pgfpathlineto{\pgfqpoint{1.922716in}{1.805799in}}%
\pgfpathlineto{\pgfqpoint{1.905159in}{1.858318in}}%
\pgfpathlineto{\pgfqpoint{1.889196in}{1.910836in}}%
\pgfpathlineto{\pgfqpoint{1.868691in}{1.989615in}}%
\pgfpathlineto{\pgfqpoint{1.857039in}{2.042134in}}%
\pgfpathlineto{\pgfqpoint{1.847176in}{2.094653in}}%
\pgfpathlineto{\pgfqpoint{1.838897in}{2.147172in}}%
\pgfpathlineto{\pgfqpoint{1.830381in}{2.217546in}}%
\pgfpathlineto{\pgfqpoint{1.827285in}{2.252210in}}%
\pgfpathlineto{\pgfqpoint{1.824006in}{2.304729in}}%
\pgfpathlineto{\pgfqpoint{1.822355in}{2.357248in}}%
\pgfpathlineto{\pgfqpoint{1.822344in}{2.409766in}}%
\pgfpathlineto{\pgfqpoint{1.823983in}{2.462285in}}%
\pgfpathlineto{\pgfqpoint{1.827276in}{2.514804in}}%
\pgfpathlineto{\pgfqpoint{1.832281in}{2.567323in}}%
\pgfpathlineto{\pgfqpoint{1.839089in}{2.619842in}}%
\pgfpathlineto{\pgfqpoint{1.847595in}{2.672361in}}%
\pgfpathlineto{\pgfqpoint{1.859955in}{2.734863in}}%
\pgfpathlineto{\pgfqpoint{1.869968in}{2.777399in}}%
\pgfpathlineto{\pgfqpoint{1.889528in}{2.849272in}}%
\pgfpathlineto{\pgfqpoint{1.899907in}{2.882437in}}%
\pgfpathlineto{\pgfqpoint{1.919102in}{2.938585in}}%
\pgfpathlineto{\pgfqpoint{1.937976in}{2.987475in}}%
\pgfpathlineto{\pgfqpoint{1.960306in}{3.039994in}}%
\pgfpathlineto{\pgfqpoint{1.984961in}{3.092513in}}%
\pgfpathlineto{\pgfqpoint{2.012147in}{3.145032in}}%
\pgfpathlineto{\pgfqpoint{2.042089in}{3.197551in}}%
\pgfpathlineto{\pgfqpoint{2.075035in}{3.250070in}}%
\pgfpathlineto{\pgfqpoint{2.111252in}{3.302589in}}%
\pgfpathlineto{\pgfqpoint{2.130609in}{3.328848in}}%
\pgfpathlineto{\pgfqpoint{2.172569in}{3.381367in}}%
\pgfpathlineto{\pgfqpoint{2.195145in}{3.407626in}}%
\pgfpathlineto{\pgfqpoint{2.218884in}{3.433886in}}%
\pgfpathlineto{\pgfqpoint{2.244409in}{3.460607in}}%
\pgfpathlineto{\pgfqpoint{2.273982in}{3.489617in}}%
\pgfpathlineto{\pgfqpoint{2.303555in}{3.516816in}}%
\pgfpathlineto{\pgfqpoint{2.333129in}{3.542315in}}%
\pgfpathlineto{\pgfqpoint{2.362702in}{3.566208in}}%
\pgfpathlineto{\pgfqpoint{2.396372in}{3.591443in}}%
\pgfpathlineto{\pgfqpoint{2.434455in}{3.617702in}}%
\pgfpathlineto{\pgfqpoint{2.480996in}{3.646920in}}%
\pgfpathlineto{\pgfqpoint{2.523110in}{3.670221in}}%
\pgfpathlineto{\pgfqpoint{2.569716in}{3.693264in}}%
\pgfpathlineto{\pgfqpoint{2.599289in}{3.706134in}}%
\pgfpathlineto{\pgfqpoint{2.642880in}{3.722740in}}%
\pgfpathlineto{\pgfqpoint{2.658436in}{3.728192in}}%
\pgfpathlineto{\pgfqpoint{2.688009in}{3.737328in}}%
\pgfpathlineto{\pgfqpoint{2.734196in}{3.749000in}}%
\pgfpathlineto{\pgfqpoint{2.747156in}{3.751905in}}%
\pgfpathlineto{\pgfqpoint{2.776729in}{3.757253in}}%
\pgfpathlineto{\pgfqpoint{2.806303in}{3.761327in}}%
\pgfpathlineto{\pgfqpoint{2.835876in}{3.764093in}}%
\pgfpathlineto{\pgfqpoint{2.865449in}{3.765518in}}%
\pgfpathlineto{\pgfqpoint{2.895023in}{3.765566in}}%
\pgfpathlineto{\pgfqpoint{2.924596in}{3.764199in}}%
\pgfpathlineto{\pgfqpoint{2.954169in}{3.761378in}}%
\pgfpathlineto{\pgfqpoint{2.983743in}{3.757063in}}%
\pgfpathlineto{\pgfqpoint{3.022177in}{3.749000in}}%
\pgfpathlineto{\pgfqpoint{3.042890in}{3.743715in}}%
\pgfpathlineto{\pgfqpoint{3.072463in}{3.734552in}}%
\pgfpathlineto{\pgfqpoint{3.104313in}{3.722740in}}%
\pgfpathlineto{\pgfqpoint{3.131610in}{3.710994in}}%
\pgfpathlineto{\pgfqpoint{3.161186in}{3.696481in}}%
\pgfpathlineto{\pgfqpoint{3.206521in}{3.670221in}}%
\pgfpathlineto{\pgfqpoint{3.220330in}{3.661515in}}%
\pgfpathlineto{\pgfqpoint{3.249903in}{3.640997in}}%
\pgfpathlineto{\pgfqpoint{3.280268in}{3.617702in}}%
\pgfpathlineto{\pgfqpoint{3.311382in}{3.591443in}}%
\pgfpathlineto{\pgfqpoint{3.340018in}{3.565183in}}%
\pgfpathlineto{\pgfqpoint{3.368197in}{3.537333in}}%
\pgfpathlineto{\pgfqpoint{3.414998in}{3.486405in}}%
\pgfpathlineto{\pgfqpoint{3.437464in}{3.460145in}}%
\pgfpathlineto{\pgfqpoint{3.479636in}{3.407626in}}%
\pgfpathlineto{\pgfqpoint{3.499539in}{3.381367in}}%
\pgfpathlineto{\pgfqpoint{3.545637in}{3.317728in}}%
\pgfpathlineto{\pgfqpoint{3.608856in}{3.223810in}}%
\pgfpathlineto{\pgfqpoint{3.675582in}{3.118772in}}%
\pgfpathlineto{\pgfqpoint{3.752650in}{2.993144in}}%
\pgfpathlineto{\pgfqpoint{3.851039in}{2.829918in}}%
\pgfpathlineto{\pgfqpoint{3.979633in}{2.619842in}}%
\pgfpathlineto{\pgfqpoint{4.096572in}{2.436026in}}%
\pgfpathlineto{\pgfqpoint{4.201375in}{2.278469in}}%
\pgfpathlineto{\pgfqpoint{4.284971in}{2.157724in}}%
\pgfpathlineto{\pgfqpoint{4.348668in}{2.068393in}}%
\pgfpathlineto{\pgfqpoint{4.432838in}{1.953993in}}%
\pgfpathlineto{\pgfqpoint{4.505314in}{1.858318in}}%
\pgfpathlineto{\pgfqpoint{4.610278in}{1.724457in}}%
\pgfpathlineto{\pgfqpoint{4.698998in}{1.614956in}}%
\pgfpathlineto{\pgfqpoint{4.787718in}{1.508522in}}%
\pgfpathlineto{\pgfqpoint{4.876438in}{1.404842in}}%
\pgfpathlineto{\pgfqpoint{4.965158in}{1.303629in}}%
\pgfpathlineto{\pgfqpoint{5.056368in}{1.201831in}}%
\pgfpathlineto{\pgfqpoint{5.176738in}{1.070533in}}%
\pgfpathlineto{\pgfqpoint{5.317428in}{0.920721in}}%
\pgfpathlineto{\pgfqpoint{5.317428in}{0.920721in}}%
\pgfusepath{stroke}%
\end{pgfscope}%
\begin{pgfscope}%
\pgfpathrectangle{\pgfqpoint{0.854460in}{0.571603in}}{\pgfqpoint{5.885100in}{5.225635in}}%
\pgfusepath{clip}%
\pgfsetbuttcap%
\pgfsetroundjoin%
\pgfsetlinewidth{1.505625pt}%
\definecolor{currentstroke}{rgb}{0.281887,0.150881,0.465405}%
\pgfsetstrokecolor{currentstroke}%
\pgfsetdash{}{0pt}%
\pgfpathmoveto{\pgfqpoint{5.584550in}{0.643548in}}%
\pgfpathlineto{\pgfqpoint{5.586199in}{0.641851in}}%
\pgfpathlineto{\pgfqpoint{5.603483in}{0.624122in}}%
\pgfpathlineto{\pgfqpoint{5.615772in}{0.611496in}}%
\pgfpathlineto{\pgfqpoint{5.629088in}{0.597863in}}%
\pgfpathlineto{\pgfqpoint{5.645346in}{0.581183in}}%
\pgfpathlineto{\pgfqpoint{5.654718in}{0.571603in}}%
\pgfusepath{stroke}%
\end{pgfscope}%
\begin{pgfscope}%
\pgfpathrectangle{\pgfqpoint{0.854460in}{0.571603in}}{\pgfqpoint{5.885100in}{5.225635in}}%
\pgfusepath{clip}%
\pgfsetbuttcap%
\pgfsetroundjoin%
\pgfsetlinewidth{1.505625pt}%
\definecolor{currentstroke}{rgb}{0.278826,0.175490,0.483397}%
\pgfsetstrokecolor{currentstroke}%
\pgfsetdash{}{0pt}%
\pgfpathmoveto{\pgfqpoint{2.785093in}{0.571603in}}%
\pgfpathlineto{\pgfqpoint{2.715383in}{0.624122in}}%
\pgfpathlineto{\pgfqpoint{2.628862in}{0.692749in}}%
\pgfpathlineto{\pgfqpoint{2.569716in}{0.741958in}}%
\pgfpathlineto{\pgfqpoint{2.510569in}{0.793289in}}%
\pgfpathlineto{\pgfqpoint{2.451422in}{0.846979in}}%
\pgfpathlineto{\pgfqpoint{2.392275in}{0.903287in}}%
\pgfpathlineto{\pgfqpoint{2.333129in}{0.962497in}}%
\pgfpathlineto{\pgfqpoint{2.303555in}{0.993342in}}%
\pgfpathlineto{\pgfqpoint{2.244409in}{1.057921in}}%
\pgfpathlineto{\pgfqpoint{2.188127in}{1.123052in}}%
\pgfpathlineto{\pgfqpoint{2.145356in}{1.175571in}}%
\pgfpathlineto{\pgfqpoint{2.096542in}{1.239009in}}%
\pgfpathlineto{\pgfqpoint{2.047697in}{1.306869in}}%
\pgfpathlineto{\pgfqpoint{2.007822in}{1.366145in}}%
\pgfpathlineto{\pgfqpoint{1.962806in}{1.438166in}}%
\pgfpathlineto{\pgfqpoint{1.932252in}{1.490685in}}%
\pgfpathlineto{\pgfqpoint{1.903583in}{1.543204in}}%
\pgfpathlineto{\pgfqpoint{1.876752in}{1.595723in}}%
\pgfpathlineto{\pgfqpoint{1.851712in}{1.648242in}}%
\pgfpathlineto{\pgfqpoint{1.828407in}{1.700761in}}%
\pgfpathlineto{\pgfqpoint{1.796785in}{1.779539in}}%
\pgfpathlineto{\pgfqpoint{1.768930in}{1.858318in}}%
\pgfpathlineto{\pgfqpoint{1.744809in}{1.937096in}}%
\pgfpathlineto{\pgfqpoint{1.730795in}{1.989615in}}%
\pgfpathlineto{\pgfqpoint{1.712599in}{2.068393in}}%
\pgfpathlineto{\pgfqpoint{1.702558in}{2.120912in}}%
\pgfpathlineto{\pgfqpoint{1.690277in}{2.199691in}}%
\pgfpathlineto{\pgfqpoint{1.682515in}{2.266866in}}%
\pgfpathlineto{\pgfqpoint{1.679240in}{2.304729in}}%
\pgfpathlineto{\pgfqpoint{1.676049in}{2.357248in}}%
\pgfpathlineto{\pgfqpoint{1.674356in}{2.409766in}}%
\pgfpathlineto{\pgfqpoint{1.674170in}{2.462285in}}%
\pgfpathlineto{\pgfqpoint{1.675492in}{2.514804in}}%
\pgfpathlineto{\pgfqpoint{1.678324in}{2.567323in}}%
\pgfpathlineto{\pgfqpoint{1.682666in}{2.619842in}}%
\pgfpathlineto{\pgfqpoint{1.688672in}{2.672361in}}%
\pgfpathlineto{\pgfqpoint{1.696214in}{2.724880in}}%
\pgfpathlineto{\pgfqpoint{1.705280in}{2.777399in}}%
\pgfpathlineto{\pgfqpoint{1.715965in}{2.829918in}}%
\pgfpathlineto{\pgfqpoint{1.735165in}{2.908696in}}%
\pgfpathlineto{\pgfqpoint{1.750092in}{2.961215in}}%
\pgfpathlineto{\pgfqpoint{1.771235in}{3.027067in}}%
\pgfpathlineto{\pgfqpoint{1.795185in}{3.092513in}}%
\pgfpathlineto{\pgfqpoint{1.816583in}{3.145032in}}%
\pgfpathlineto{\pgfqpoint{1.839933in}{3.197551in}}%
\pgfpathlineto{\pgfqpoint{1.865373in}{3.250070in}}%
\pgfpathlineto{\pgfqpoint{1.893055in}{3.302589in}}%
\pgfpathlineto{\pgfqpoint{1.923135in}{3.355107in}}%
\pgfpathlineto{\pgfqpoint{1.955780in}{3.407626in}}%
\pgfpathlineto{\pgfqpoint{1.991163in}{3.460145in}}%
\pgfpathlineto{\pgfqpoint{2.029464in}{3.512664in}}%
\pgfpathlineto{\pgfqpoint{2.066968in}{3.560290in}}%
\pgfpathlineto{\pgfqpoint{2.096542in}{3.595375in}}%
\pgfpathlineto{\pgfqpoint{2.140567in}{3.643962in}}%
\pgfpathlineto{\pgfqpoint{2.185262in}{3.689702in}}%
\pgfpathlineto{\pgfqpoint{2.220068in}{3.722740in}}%
\pgfpathlineto{\pgfqpoint{2.273982in}{3.770194in}}%
\pgfpathlineto{\pgfqpoint{2.312775in}{3.801519in}}%
\pgfpathlineto{\pgfqpoint{2.362702in}{3.838957in}}%
\pgfpathlineto{\pgfqpoint{2.423815in}{3.880297in}}%
\pgfpathlineto{\pgfqpoint{2.480996in}{3.914781in}}%
\pgfpathlineto{\pgfqpoint{2.540142in}{3.946416in}}%
\pgfpathlineto{\pgfqpoint{2.599289in}{3.974072in}}%
\pgfpathlineto{\pgfqpoint{2.658436in}{3.997801in}}%
\pgfpathlineto{\pgfqpoint{2.717582in}{4.017630in}}%
\pgfpathlineto{\pgfqpoint{2.776729in}{4.033485in}}%
\pgfpathlineto{\pgfqpoint{2.835876in}{4.045247in}}%
\pgfpathlineto{\pgfqpoint{2.895023in}{4.052781in}}%
\pgfpathlineto{\pgfqpoint{2.924596in}{4.054900in}}%
\pgfpathlineto{\pgfqpoint{2.954169in}{4.055868in}}%
\pgfpathlineto{\pgfqpoint{2.983743in}{4.055642in}}%
\pgfpathlineto{\pgfqpoint{3.013316in}{4.054178in}}%
\pgfpathlineto{\pgfqpoint{3.042890in}{4.051426in}}%
\pgfpathlineto{\pgfqpoint{3.072463in}{4.047337in}}%
\pgfpathlineto{\pgfqpoint{3.119233in}{4.037854in}}%
\pgfpathlineto{\pgfqpoint{3.131610in}{4.034883in}}%
\pgfpathlineto{\pgfqpoint{3.161183in}{4.026318in}}%
\pgfpathlineto{\pgfqpoint{3.202287in}{4.011594in}}%
\pgfpathlineto{\pgfqpoint{3.220330in}{4.004236in}}%
\pgfpathlineto{\pgfqpoint{3.259804in}{3.985335in}}%
\pgfpathlineto{\pgfqpoint{3.279476in}{3.974786in}}%
\pgfpathlineto{\pgfqpoint{3.309050in}{3.957060in}}%
\pgfpathlineto{\pgfqpoint{3.344457in}{3.932816in}}%
\pgfpathlineto{\pgfqpoint{3.378316in}{3.906556in}}%
\pgfpathlineto{\pgfqpoint{3.408674in}{3.880297in}}%
\pgfpathlineto{\pgfqpoint{3.436309in}{3.854037in}}%
\pgfpathlineto{\pgfqpoint{3.461792in}{3.827778in}}%
\pgfpathlineto{\pgfqpoint{3.486490in}{3.800410in}}%
\pgfpathlineto{\pgfqpoint{3.528436in}{3.749000in}}%
\pgfpathlineto{\pgfqpoint{3.567260in}{3.696481in}}%
\pgfpathlineto{\pgfqpoint{3.585399in}{3.670221in}}%
\pgfpathlineto{\pgfqpoint{3.619737in}{3.617702in}}%
\pgfpathlineto{\pgfqpoint{3.652007in}{3.565183in}}%
\pgfpathlineto{\pgfqpoint{3.693504in}{3.493790in}}%
\pgfpathlineto{\pgfqpoint{3.751765in}{3.386975in}}%
\pgfpathlineto{\pgfqpoint{3.751765in}{3.386975in}}%
\pgfusepath{stroke}%
\end{pgfscope}%
\begin{pgfscope}%
\pgfpathrectangle{\pgfqpoint{0.854460in}{0.571603in}}{\pgfqpoint{5.885100in}{5.225635in}}%
\pgfusepath{clip}%
\pgfsetbuttcap%
\pgfsetroundjoin%
\pgfsetlinewidth{1.505625pt}%
\definecolor{currentstroke}{rgb}{0.278826,0.175490,0.483397}%
\pgfsetstrokecolor{currentstroke}%
\pgfsetdash{}{0pt}%
\pgfpathmoveto{\pgfqpoint{3.929964in}{3.040749in}}%
\pgfpathlineto{\pgfqpoint{4.026504in}{2.856177in}}%
\pgfpathlineto{\pgfqpoint{4.098026in}{2.724880in}}%
\pgfpathlineto{\pgfqpoint{4.187928in}{2.567323in}}%
\pgfpathlineto{\pgfqpoint{4.266724in}{2.436026in}}%
\pgfpathlineto{\pgfqpoint{4.344118in}{2.313056in}}%
\pgfpathlineto{\pgfqpoint{4.383608in}{2.252210in}}%
\pgfpathlineto{\pgfqpoint{4.462411in}{2.135025in}}%
\pgfpathlineto{\pgfqpoint{4.527177in}{2.042134in}}%
\pgfpathlineto{\pgfqpoint{4.610278in}{1.927384in}}%
\pgfpathlineto{\pgfqpoint{4.681731in}{1.832058in}}%
\pgfpathlineto{\pgfqpoint{4.763219in}{1.727020in}}%
\pgfpathlineto{\pgfqpoint{4.847472in}{1.621982in}}%
\pgfpathlineto{\pgfqpoint{4.935585in}{1.515556in}}%
\pgfpathlineto{\pgfqpoint{5.024305in}{1.411586in}}%
\pgfpathlineto{\pgfqpoint{5.116201in}{1.306869in}}%
\pgfpathlineto{\pgfqpoint{5.231319in}{1.179522in}}%
\pgfpathlineto{\pgfqpoint{5.332531in}{1.070533in}}%
\pgfpathlineto{\pgfqpoint{5.467906in}{0.928601in}}%
\pgfpathlineto{\pgfqpoint{5.586199in}{0.807544in}}%
\pgfpathlineto{\pgfqpoint{5.742861in}{0.650382in}}%
\pgfpathlineto{\pgfqpoint{5.822537in}{0.571603in}}%
\pgfpathlineto{\pgfqpoint{5.822537in}{0.571603in}}%
\pgfusepath{stroke}%
\end{pgfscope}%
\begin{pgfscope}%
\pgfpathrectangle{\pgfqpoint{0.854460in}{0.571603in}}{\pgfqpoint{5.885100in}{5.225635in}}%
\pgfusepath{clip}%
\pgfsetbuttcap%
\pgfsetroundjoin%
\pgfsetlinewidth{1.505625pt}%
\definecolor{currentstroke}{rgb}{0.273006,0.204520,0.501721}%
\pgfsetstrokecolor{currentstroke}%
\pgfsetdash{}{0pt}%
\pgfpathmoveto{\pgfqpoint{2.648461in}{0.571603in}}%
\pgfpathlineto{\pgfqpoint{2.569716in}{0.632750in}}%
\pgfpathlineto{\pgfqpoint{2.510569in}{0.680656in}}%
\pgfpathlineto{\pgfqpoint{2.451422in}{0.730504in}}%
\pgfpathlineto{\pgfqpoint{2.392275in}{0.782508in}}%
\pgfpathlineto{\pgfqpoint{2.333129in}{0.836903in}}%
\pgfpathlineto{\pgfqpoint{2.273982in}{0.893944in}}%
\pgfpathlineto{\pgfqpoint{2.214835in}{0.953909in}}%
\pgfpathlineto{\pgfqpoint{2.154856in}{1.018014in}}%
\pgfpathlineto{\pgfqpoint{2.096542in}{1.084327in}}%
\pgfpathlineto{\pgfqpoint{2.042639in}{1.149312in}}%
\pgfpathlineto{\pgfqpoint{2.001497in}{1.201831in}}%
\pgfpathlineto{\pgfqpoint{1.948675in}{1.273505in}}%
\pgfpathlineto{\pgfqpoint{1.907566in}{1.333128in}}%
\pgfpathlineto{\pgfqpoint{1.873426in}{1.385647in}}%
\pgfpathlineto{\pgfqpoint{1.830381in}{1.456433in}}%
\pgfpathlineto{\pgfqpoint{1.796225in}{1.516944in}}%
\pgfpathlineto{\pgfqpoint{1.768513in}{1.569463in}}%
\pgfpathlineto{\pgfqpoint{1.730293in}{1.648242in}}%
\pgfpathlineto{\pgfqpoint{1.706880in}{1.700761in}}%
\pgfpathlineto{\pgfqpoint{1.674934in}{1.779539in}}%
\pgfpathlineto{\pgfqpoint{1.646612in}{1.858318in}}%
\pgfpathlineto{\pgfqpoint{1.621809in}{1.937096in}}%
\pgfpathlineto{\pgfqpoint{1.600559in}{2.015874in}}%
\pgfpathlineto{\pgfqpoint{1.588247in}{2.068393in}}%
\pgfpathlineto{\pgfqpoint{1.572598in}{2.147172in}}%
\pgfpathlineto{\pgfqpoint{1.563922in}{2.199691in}}%
\pgfpathlineto{\pgfqpoint{1.553806in}{2.278469in}}%
\pgfpathlineto{\pgfqpoint{1.546851in}{2.357248in}}%
\pgfpathlineto{\pgfqpoint{1.543083in}{2.436026in}}%
\pgfpathlineto{\pgfqpoint{1.542519in}{2.514804in}}%
\pgfpathlineto{\pgfqpoint{1.545165in}{2.593583in}}%
\pgfpathlineto{\pgfqpoint{1.551013in}{2.672361in}}%
\pgfpathlineto{\pgfqpoint{1.560041in}{2.751140in}}%
\pgfpathlineto{\pgfqpoint{1.567906in}{2.803659in}}%
\pgfpathlineto{\pgfqpoint{1.582535in}{2.882437in}}%
\pgfpathlineto{\pgfqpoint{1.594046in}{2.934956in}}%
\pgfpathlineto{\pgfqpoint{1.614425in}{3.013734in}}%
\pgfpathlineto{\pgfqpoint{1.629920in}{3.066253in}}%
\pgfpathlineto{\pgfqpoint{1.656227in}{3.145032in}}%
\pgfpathlineto{\pgfqpoint{1.686409in}{3.223810in}}%
\pgfpathlineto{\pgfqpoint{1.712088in}{3.283837in}}%
\pgfpathlineto{\pgfqpoint{1.745810in}{3.355107in}}%
\pgfpathlineto{\pgfqpoint{1.772956in}{3.407626in}}%
\pgfpathlineto{\pgfqpoint{1.802225in}{3.460145in}}%
\pgfpathlineto{\pgfqpoint{1.833741in}{3.512664in}}%
\pgfpathlineto{\pgfqpoint{1.867632in}{3.565183in}}%
\pgfpathlineto{\pgfqpoint{1.904024in}{3.617702in}}%
\pgfpathlineto{\pgfqpoint{1.943046in}{3.670221in}}%
\pgfpathlineto{\pgfqpoint{1.978248in}{3.714528in}}%
\pgfpathlineto{\pgfqpoint{2.007822in}{3.749758in}}%
\pgfpathlineto{\pgfqpoint{2.054260in}{3.801519in}}%
\pgfpathlineto{\pgfqpoint{2.096542in}{3.845658in}}%
\pgfpathlineto{\pgfqpoint{2.131872in}{3.880297in}}%
\pgfpathlineto{\pgfqpoint{2.189212in}{3.932816in}}%
\pgfpathlineto{\pgfqpoint{2.244409in}{3.979321in}}%
\pgfpathlineto{\pgfqpoint{2.303555in}{4.025087in}}%
\pgfpathlineto{\pgfqpoint{2.362702in}{4.067050in}}%
\pgfpathlineto{\pgfqpoint{2.421849in}{4.105319in}}%
\pgfpathlineto{\pgfqpoint{2.485828in}{4.142892in}}%
\pgfpathlineto{\pgfqpoint{2.540142in}{4.171814in}}%
\pgfpathlineto{\pgfqpoint{2.599289in}{4.200178in}}%
\pgfpathlineto{\pgfqpoint{2.658436in}{4.225418in}}%
\pgfpathlineto{\pgfqpoint{2.718717in}{4.247930in}}%
\pgfpathlineto{\pgfqpoint{2.776729in}{4.266456in}}%
\pgfpathlineto{\pgfqpoint{2.835876in}{4.282206in}}%
\pgfpathlineto{\pgfqpoint{2.895023in}{4.294666in}}%
\pgfpathlineto{\pgfqpoint{2.954169in}{4.303684in}}%
\pgfpathlineto{\pgfqpoint{3.013316in}{4.309014in}}%
\pgfpathlineto{\pgfqpoint{3.072463in}{4.310471in}}%
\pgfpathlineto{\pgfqpoint{3.131610in}{4.307713in}}%
\pgfpathlineto{\pgfqpoint{3.161183in}{4.304633in}}%
\pgfpathlineto{\pgfqpoint{3.190756in}{4.300348in}}%
\pgfpathlineto{\pgfqpoint{3.220330in}{4.294694in}}%
\pgfpathlineto{\pgfqpoint{3.249903in}{4.287692in}}%
\pgfpathlineto{\pgfqpoint{3.294791in}{4.274189in}}%
\pgfpathlineto{\pgfqpoint{3.309050in}{4.269277in}}%
\pgfpathlineto{\pgfqpoint{3.360115in}{4.247930in}}%
\pgfpathlineto{\pgfqpoint{3.368197in}{4.244161in}}%
\pgfpathlineto{\pgfqpoint{3.410022in}{4.221670in}}%
\pgfpathlineto{\pgfqpoint{3.427343in}{4.211280in}}%
\pgfpathlineto{\pgfqpoint{3.456917in}{4.191586in}}%
\pgfpathlineto{\pgfqpoint{3.486860in}{4.169151in}}%
\pgfpathlineto{\pgfqpoint{3.517996in}{4.142892in}}%
\pgfpathlineto{\pgfqpoint{3.545997in}{4.116632in}}%
\pgfpathlineto{\pgfqpoint{3.575210in}{4.086234in}}%
\pgfpathlineto{\pgfqpoint{3.616258in}{4.037854in}}%
\pgfpathlineto{\pgfqpoint{3.636564in}{4.011594in}}%
\pgfpathlineto{\pgfqpoint{3.673376in}{3.959075in}}%
\pgfpathlineto{\pgfqpoint{3.706598in}{3.906556in}}%
\pgfpathlineto{\pgfqpoint{3.737052in}{3.854037in}}%
\pgfpathlineto{\pgfqpoint{3.765369in}{3.801519in}}%
\pgfpathlineto{\pgfqpoint{3.804883in}{3.722740in}}%
\pgfpathlineto{\pgfqpoint{3.829709in}{3.670221in}}%
\pgfpathlineto{\pgfqpoint{3.870944in}{3.579318in}}%
\pgfpathlineto{\pgfqpoint{3.967515in}{3.355107in}}%
\pgfpathlineto{\pgfqpoint{4.048384in}{3.168100in}}%
\pgfpathlineto{\pgfqpoint{4.093758in}{3.066253in}}%
\pgfpathlineto{\pgfqpoint{4.154695in}{2.934956in}}%
\pgfpathlineto{\pgfqpoint{4.218835in}{2.803659in}}%
\pgfpathlineto{\pgfqpoint{4.286657in}{2.672361in}}%
\pgfpathlineto{\pgfqpoint{4.344118in}{2.566727in}}%
\pgfpathlineto{\pgfqpoint{4.388328in}{2.488545in}}%
\pgfpathlineto{\pgfqpoint{4.450274in}{2.383507in}}%
\pgfpathlineto{\pgfqpoint{4.521558in}{2.268405in}}%
\pgfpathlineto{\pgfqpoint{4.565681in}{2.199691in}}%
\pgfpathlineto{\pgfqpoint{4.639851in}{2.088720in}}%
\pgfpathlineto{\pgfqpoint{4.698998in}{2.003576in}}%
\pgfpathlineto{\pgfqpoint{4.758145in}{1.921168in}}%
\pgfpathlineto{\pgfqpoint{4.817291in}{1.841256in}}%
\pgfpathlineto{\pgfqpoint{4.884393in}{1.753280in}}%
\pgfpathlineto{\pgfqpoint{4.967374in}{1.648242in}}%
\pgfpathlineto{\pgfqpoint{5.053878in}{1.542541in}}%
\pgfpathlineto{\pgfqpoint{5.133820in}{1.447933in}}%
\pgfpathlineto{\pgfqpoint{5.133820in}{1.447933in}}%
\pgfusepath{stroke}%
\end{pgfscope}%
\begin{pgfscope}%
\pgfpathrectangle{\pgfqpoint{0.854460in}{0.571603in}}{\pgfqpoint{5.885100in}{5.225635in}}%
\pgfusepath{clip}%
\pgfsetbuttcap%
\pgfsetroundjoin%
\pgfsetlinewidth{1.505625pt}%
\definecolor{currentstroke}{rgb}{0.273006,0.204520,0.501721}%
\pgfsetstrokecolor{currentstroke}%
\pgfsetdash{}{0pt}%
\pgfpathmoveto{\pgfqpoint{5.390741in}{1.160502in}}%
\pgfpathlineto{\pgfqpoint{5.401122in}{1.149312in}}%
\pgfpathlineto{\pgfqpoint{5.408759in}{1.141148in}}%
\pgfpathlineto{\pgfqpoint{5.425664in}{1.123052in}}%
\pgfpathlineto{\pgfqpoint{5.438332in}{1.109600in}}%
\pgfpathlineto{\pgfqpoint{5.450381in}{1.096793in}}%
\pgfpathlineto{\pgfqpoint{5.467906in}{1.078306in}}%
\pgfpathlineto{\pgfqpoint{5.475268in}{1.070533in}}%
\pgfpathlineto{\pgfqpoint{5.497479in}{1.047256in}}%
\pgfpathlineto{\pgfqpoint{5.500323in}{1.044274in}}%
\pgfpathlineto{\pgfqpoint{5.525518in}{1.018014in}}%
\pgfpathlineto{\pgfqpoint{5.527052in}{1.016423in}}%
\pgfpathlineto{\pgfqpoint{5.550831in}{0.991755in}}%
\pgfpathlineto{\pgfqpoint{5.556626in}{0.985781in}}%
\pgfpathlineto{\pgfqpoint{5.576303in}{0.965495in}}%
\pgfpathlineto{\pgfqpoint{5.586199in}{0.955355in}}%
\pgfpathlineto{\pgfqpoint{5.601932in}{0.939236in}}%
\pgfpathlineto{\pgfqpoint{5.615772in}{0.925137in}}%
\pgfpathlineto{\pgfqpoint{5.627714in}{0.912976in}}%
\pgfpathlineto{\pgfqpoint{5.645346in}{0.895118in}}%
\pgfpathlineto{\pgfqpoint{5.653645in}{0.886717in}}%
\pgfpathlineto{\pgfqpoint{5.674919in}{0.865290in}}%
\pgfpathlineto{\pgfqpoint{5.679722in}{0.860458in}}%
\pgfpathlineto{\pgfqpoint{5.704492in}{0.835646in}}%
\pgfpathlineto{\pgfqpoint{5.705939in}{0.834198in}}%
\pgfpathlineto{\pgfqpoint{5.732266in}{0.807939in}}%
\pgfpathlineto{\pgfqpoint{5.734066in}{0.806149in}}%
\pgfpathlineto{\pgfqpoint{5.758706in}{0.781679in}}%
\pgfpathlineto{\pgfqpoint{5.763639in}{0.776797in}}%
\pgfpathlineto{\pgfqpoint{5.785277in}{0.755420in}}%
\pgfpathlineto{\pgfqpoint{5.793213in}{0.747604in}}%
\pgfpathlineto{\pgfqpoint{5.811976in}{0.729160in}}%
\pgfpathlineto{\pgfqpoint{5.822786in}{0.718563in}}%
\pgfpathlineto{\pgfqpoint{5.838798in}{0.702901in}}%
\pgfpathlineto{\pgfqpoint{5.852359in}{0.689667in}}%
\pgfpathlineto{\pgfqpoint{5.865739in}{0.676641in}}%
\pgfpathlineto{\pgfqpoint{5.881933in}{0.660907in}}%
\pgfpathlineto{\pgfqpoint{5.892795in}{0.650382in}}%
\pgfpathlineto{\pgfqpoint{5.911506in}{0.632278in}}%
\pgfpathlineto{\pgfqpoint{5.919959in}{0.624122in}}%
\pgfpathlineto{\pgfqpoint{5.941079in}{0.603770in}}%
\pgfpathlineto{\pgfqpoint{5.947229in}{0.597863in}}%
\pgfpathlineto{\pgfqpoint{5.970653in}{0.575378in}}%
\pgfpathlineto{\pgfqpoint{5.974598in}{0.571603in}}%
\pgfusepath{stroke}%
\end{pgfscope}%
\begin{pgfscope}%
\pgfpathrectangle{\pgfqpoint{0.854460in}{0.571603in}}{\pgfqpoint{5.885100in}{5.225635in}}%
\pgfusepath{clip}%
\pgfsetbuttcap%
\pgfsetroundjoin%
\pgfsetlinewidth{1.505625pt}%
\definecolor{currentstroke}{rgb}{0.266580,0.228262,0.514349}%
\pgfsetstrokecolor{currentstroke}%
\pgfsetdash{}{0pt}%
\pgfpathmoveto{\pgfqpoint{2.523741in}{0.571603in}}%
\pgfpathlineto{\pgfqpoint{2.451422in}{0.628877in}}%
\pgfpathlineto{\pgfqpoint{2.392275in}{0.677661in}}%
\pgfpathlineto{\pgfqpoint{2.332323in}{0.729160in}}%
\pgfpathlineto{\pgfqpoint{2.273700in}{0.781679in}}%
\pgfpathlineto{\pgfqpoint{2.214835in}{0.836828in}}%
\pgfpathlineto{\pgfqpoint{2.155689in}{0.894930in}}%
\pgfpathlineto{\pgfqpoint{2.096542in}{0.955995in}}%
\pgfpathlineto{\pgfqpoint{2.039514in}{1.018014in}}%
\pgfpathlineto{\pgfqpoint{1.993724in}{1.070533in}}%
\pgfpathlineto{\pgfqpoint{1.948675in}{1.124630in}}%
\pgfpathlineto{\pgfqpoint{1.888250in}{1.201831in}}%
\pgfpathlineto{\pgfqpoint{1.830381in}{1.281560in}}%
\pgfpathlineto{\pgfqpoint{1.778215in}{1.359388in}}%
\pgfpathlineto{\pgfqpoint{1.741661in}{1.417885in}}%
\pgfpathlineto{\pgfqpoint{1.699324in}{1.490685in}}%
\pgfpathlineto{\pgfqpoint{1.670859in}{1.543204in}}%
\pgfpathlineto{\pgfqpoint{1.644100in}{1.595723in}}%
\pgfpathlineto{\pgfqpoint{1.619002in}{1.648242in}}%
\pgfpathlineto{\pgfqpoint{1.584491in}{1.727020in}}%
\pgfpathlineto{\pgfqpoint{1.553544in}{1.805799in}}%
\pgfpathlineto{\pgfqpoint{1.526073in}{1.884577in}}%
\pgfpathlineto{\pgfqpoint{1.501972in}{1.963355in}}%
\pgfpathlineto{\pgfqpoint{1.481244in}{2.042134in}}%
\pgfpathlineto{\pgfqpoint{1.469223in}{2.094653in}}%
\pgfpathlineto{\pgfqpoint{1.453860in}{2.173431in}}%
\pgfpathlineto{\pgfqpoint{1.441611in}{2.252210in}}%
\pgfpathlineto{\pgfqpoint{1.432530in}{2.330988in}}%
\pgfpathlineto{\pgfqpoint{1.426468in}{2.409766in}}%
\pgfpathlineto{\pgfqpoint{1.423441in}{2.488545in}}%
\pgfpathlineto{\pgfqpoint{1.423456in}{2.567323in}}%
\pgfpathlineto{\pgfqpoint{1.426511in}{2.646102in}}%
\pgfpathlineto{\pgfqpoint{1.432594in}{2.724880in}}%
\pgfpathlineto{\pgfqpoint{1.441680in}{2.803659in}}%
\pgfpathlineto{\pgfqpoint{1.449476in}{2.856177in}}%
\pgfpathlineto{\pgfqpoint{1.463840in}{2.934956in}}%
\pgfpathlineto{\pgfqpoint{1.481311in}{3.013734in}}%
\pgfpathlineto{\pgfqpoint{1.502055in}{3.092513in}}%
\pgfpathlineto{\pgfqpoint{1.517825in}{3.145032in}}%
\pgfpathlineto{\pgfqpoint{1.534977in}{3.197551in}}%
\pgfpathlineto{\pgfqpoint{1.564221in}{3.277397in}}%
\pgfpathlineto{\pgfqpoint{1.596357in}{3.355107in}}%
\pgfpathlineto{\pgfqpoint{1.632833in}{3.433886in}}%
\pgfpathlineto{\pgfqpoint{1.659367in}{3.486405in}}%
\pgfpathlineto{\pgfqpoint{1.687824in}{3.538924in}}%
\pgfpathlineto{\pgfqpoint{1.718303in}{3.591443in}}%
\pgfpathlineto{\pgfqpoint{1.750902in}{3.643962in}}%
\pgfpathlineto{\pgfqpoint{1.785723in}{3.696481in}}%
\pgfpathlineto{\pgfqpoint{1.830381in}{3.759265in}}%
\pgfpathlineto{\pgfqpoint{1.862502in}{3.801519in}}%
\pgfpathlineto{\pgfqpoint{1.904983in}{3.854037in}}%
\pgfpathlineto{\pgfqpoint{1.950213in}{3.906556in}}%
\pgfpathlineto{\pgfqpoint{2.007822in}{3.968486in}}%
\pgfpathlineto{\pgfqpoint{2.066968in}{4.027172in}}%
\pgfpathlineto{\pgfqpoint{2.126115in}{4.081422in}}%
\pgfpathlineto{\pgfqpoint{2.185262in}{4.131657in}}%
\pgfpathlineto{\pgfqpoint{2.244409in}{4.178244in}}%
\pgfpathlineto{\pgfqpoint{2.303790in}{4.221670in}}%
\pgfpathlineto{\pgfqpoint{2.362702in}{4.261490in}}%
\pgfpathlineto{\pgfqpoint{2.424939in}{4.300449in}}%
\pgfpathlineto{\pgfqpoint{2.480996in}{4.332823in}}%
\pgfpathlineto{\pgfqpoint{2.540142in}{4.364320in}}%
\pgfpathlineto{\pgfqpoint{2.599289in}{4.393192in}}%
\pgfpathlineto{\pgfqpoint{2.658436in}{4.419474in}}%
\pgfpathlineto{\pgfqpoint{2.717582in}{4.443195in}}%
\pgfpathlineto{\pgfqpoint{2.776729in}{4.464376in}}%
\pgfpathlineto{\pgfqpoint{2.840371in}{4.484265in}}%
\pgfpathlineto{\pgfqpoint{2.895023in}{4.498899in}}%
\pgfpathlineto{\pgfqpoint{2.954169in}{4.512204in}}%
\pgfpathlineto{\pgfqpoint{3.013316in}{4.522583in}}%
\pgfpathlineto{\pgfqpoint{3.072463in}{4.530069in}}%
\pgfpathlineto{\pgfqpoint{3.131610in}{4.534409in}}%
\pgfpathlineto{\pgfqpoint{3.190756in}{4.535316in}}%
\pgfpathlineto{\pgfqpoint{3.249903in}{4.532462in}}%
\pgfpathlineto{\pgfqpoint{3.309050in}{4.525470in}}%
\pgfpathlineto{\pgfqpoint{3.338623in}{4.520288in}}%
\pgfpathlineto{\pgfqpoint{3.381388in}{4.510524in}}%
\pgfpathlineto{\pgfqpoint{3.397770in}{4.506150in}}%
\pgfpathlineto{\pgfqpoint{3.427343in}{4.496950in}}%
\pgfpathlineto{\pgfqpoint{3.461941in}{4.484265in}}%
\pgfpathlineto{\pgfqpoint{3.486490in}{4.473886in}}%
\pgfpathlineto{\pgfqpoint{3.519392in}{4.458005in}}%
\pgfpathlineto{\pgfqpoint{3.565002in}{4.431746in}}%
\pgfpathlineto{\pgfqpoint{3.575210in}{4.425246in}}%
\pgfpathlineto{\pgfqpoint{3.604783in}{4.404572in}}%
\pgfpathlineto{\pgfqpoint{3.636667in}{4.379227in}}%
\pgfpathlineto{\pgfqpoint{3.665953in}{4.352967in}}%
\pgfpathlineto{\pgfqpoint{3.693504in}{4.325360in}}%
\pgfpathlineto{\pgfqpoint{3.723077in}{4.292142in}}%
\pgfpathlineto{\pgfqpoint{3.757812in}{4.247930in}}%
\pgfpathlineto{\pgfqpoint{3.793851in}{4.195411in}}%
\pgfpathlineto{\pgfqpoint{3.825657in}{4.142892in}}%
\pgfpathlineto{\pgfqpoint{3.854213in}{4.090373in}}%
\pgfpathlineto{\pgfqpoint{3.880260in}{4.037854in}}%
\pgfpathlineto{\pgfqpoint{3.904370in}{3.985335in}}%
\pgfpathlineto{\pgfqpoint{3.937655in}{3.906556in}}%
\pgfpathlineto{\pgfqpoint{3.968576in}{3.827778in}}%
\pgfpathlineto{\pgfqpoint{4.016975in}{3.696481in}}%
\pgfpathlineto{\pgfqpoint{4.054229in}{3.591443in}}%
\pgfpathlineto{\pgfqpoint{4.072686in}{3.539146in}}%
\pgfpathlineto{\pgfqpoint{4.072686in}{3.539146in}}%
\pgfusepath{stroke}%
\end{pgfscope}%
\begin{pgfscope}%
\pgfpathrectangle{\pgfqpoint{0.854460in}{0.571603in}}{\pgfqpoint{5.885100in}{5.225635in}}%
\pgfusepath{clip}%
\pgfsetbuttcap%
\pgfsetroundjoin%
\pgfsetlinewidth{1.505625pt}%
\definecolor{currentstroke}{rgb}{0.266580,0.228262,0.514349}%
\pgfsetstrokecolor{currentstroke}%
\pgfsetdash{}{0pt}%
\pgfpathmoveto{\pgfqpoint{4.207778in}{3.172350in}}%
\pgfpathlineto{\pgfqpoint{4.239728in}{3.092513in}}%
\pgfpathlineto{\pgfqpoint{4.284971in}{2.984687in}}%
\pgfpathlineto{\pgfqpoint{4.330071in}{2.882437in}}%
\pgfpathlineto{\pgfqpoint{4.379087in}{2.777399in}}%
\pgfpathlineto{\pgfqpoint{4.430876in}{2.672361in}}%
\pgfpathlineto{\pgfqpoint{4.491984in}{2.555403in}}%
\pgfpathlineto{\pgfqpoint{4.528506in}{2.488545in}}%
\pgfpathlineto{\pgfqpoint{4.588514in}{2.383507in}}%
\pgfpathlineto{\pgfqpoint{4.651661in}{2.278469in}}%
\pgfpathlineto{\pgfqpoint{4.718018in}{2.173431in}}%
\pgfpathlineto{\pgfqpoint{4.787718in}{2.068300in}}%
\pgfpathlineto{\pgfqpoint{4.846865in}{1.982657in}}%
\pgfpathlineto{\pgfqpoint{4.906012in}{1.899973in}}%
\pgfpathlineto{\pgfqpoint{4.965158in}{1.819963in}}%
\pgfpathlineto{\pgfqpoint{5.036166in}{1.727020in}}%
\pgfpathlineto{\pgfqpoint{5.119525in}{1.621982in}}%
\pgfpathlineto{\pgfqpoint{5.206055in}{1.516944in}}%
\pgfpathlineto{\pgfqpoint{5.295698in}{1.411906in}}%
\pgfpathlineto{\pgfqpoint{5.388396in}{1.306869in}}%
\pgfpathlineto{\pgfqpoint{5.497479in}{1.187427in}}%
\pgfpathlineto{\pgfqpoint{5.586199in}{1.093162in}}%
\pgfpathlineto{\pgfqpoint{5.684106in}{0.991755in}}%
\pgfpathlineto{\pgfqpoint{5.793213in}{0.881718in}}%
\pgfpathlineto{\pgfqpoint{5.921782in}{0.755420in}}%
\pgfpathlineto{\pgfqpoint{6.059373in}{0.623802in}}%
\pgfpathlineto{\pgfqpoint{6.114810in}{0.571603in}}%
\pgfpathlineto{\pgfqpoint{6.114810in}{0.571603in}}%
\pgfusepath{stroke}%
\end{pgfscope}%
\begin{pgfscope}%
\pgfpathrectangle{\pgfqpoint{0.854460in}{0.571603in}}{\pgfqpoint{5.885100in}{5.225635in}}%
\pgfusepath{clip}%
\pgfsetbuttcap%
\pgfsetroundjoin%
\pgfsetlinewidth{1.505625pt}%
\definecolor{currentstroke}{rgb}{0.257322,0.256130,0.526563}%
\pgfsetstrokecolor{currentstroke}%
\pgfsetdash{}{0pt}%
\pgfpathmoveto{\pgfqpoint{2.408459in}{0.571603in}}%
\pgfpathlineto{\pgfqpoint{2.333129in}{0.632527in}}%
\pgfpathlineto{\pgfqpoint{2.273982in}{0.682415in}}%
\pgfpathlineto{\pgfqpoint{2.214835in}{0.734351in}}%
\pgfpathlineto{\pgfqpoint{2.155689in}{0.788548in}}%
\pgfpathlineto{\pgfqpoint{2.096542in}{0.845235in}}%
\pgfpathlineto{\pgfqpoint{2.029326in}{0.912976in}}%
\pgfpathlineto{\pgfqpoint{1.978248in}{0.967162in}}%
\pgfpathlineto{\pgfqpoint{1.919102in}{1.033349in}}%
\pgfpathlineto{\pgfqpoint{1.865440in}{1.096793in}}%
\pgfpathlineto{\pgfqpoint{1.823323in}{1.149312in}}%
\pgfpathlineto{\pgfqpoint{1.771235in}{1.218021in}}%
\pgfpathlineto{\pgfqpoint{1.726633in}{1.280609in}}%
\pgfpathlineto{\pgfqpoint{1.682515in}{1.346560in}}%
\pgfpathlineto{\pgfqpoint{1.641616in}{1.411906in}}%
\pgfpathlineto{\pgfqpoint{1.610735in}{1.464425in}}%
\pgfpathlineto{\pgfqpoint{1.581573in}{1.516944in}}%
\pgfpathlineto{\pgfqpoint{1.554090in}{1.569463in}}%
\pgfpathlineto{\pgfqpoint{1.528242in}{1.621982in}}%
\pgfpathlineto{\pgfqpoint{1.492531in}{1.700761in}}%
\pgfpathlineto{\pgfqpoint{1.470637in}{1.753280in}}%
\pgfpathlineto{\pgfqpoint{1.440711in}{1.832058in}}%
\pgfpathlineto{\pgfqpoint{1.414113in}{1.910836in}}%
\pgfpathlineto{\pgfqpoint{1.390820in}{1.989615in}}%
\pgfpathlineto{\pgfqpoint{1.377088in}{2.042134in}}%
\pgfpathlineto{\pgfqpoint{1.359004in}{2.120912in}}%
\pgfpathlineto{\pgfqpoint{1.348741in}{2.173431in}}%
\pgfpathlineto{\pgfqpoint{1.335805in}{2.252210in}}%
\pgfpathlineto{\pgfqpoint{1.325816in}{2.330988in}}%
\pgfpathlineto{\pgfqpoint{1.318875in}{2.409766in}}%
\pgfpathlineto{\pgfqpoint{1.314819in}{2.488545in}}%
\pgfpathlineto{\pgfqpoint{1.313656in}{2.567323in}}%
\pgfpathlineto{\pgfqpoint{1.315382in}{2.646102in}}%
\pgfpathlineto{\pgfqpoint{1.319987in}{2.724880in}}%
\pgfpathlineto{\pgfqpoint{1.327634in}{2.805204in}}%
\pgfpathlineto{\pgfqpoint{1.337984in}{2.882437in}}%
\pgfpathlineto{\pgfqpoint{1.351383in}{2.961215in}}%
\pgfpathlineto{\pgfqpoint{1.361996in}{3.013734in}}%
\pgfpathlineto{\pgfqpoint{1.380493in}{3.092513in}}%
\pgfpathlineto{\pgfqpoint{1.394562in}{3.145032in}}%
\pgfpathlineto{\pgfqpoint{1.418274in}{3.223810in}}%
\pgfpathlineto{\pgfqpoint{1.445928in}{3.304244in}}%
\pgfpathlineto{\pgfqpoint{1.475808in}{3.381367in}}%
\pgfpathlineto{\pgfqpoint{1.509930in}{3.460145in}}%
\pgfpathlineto{\pgfqpoint{1.547812in}{3.538924in}}%
\pgfpathlineto{\pgfqpoint{1.575228in}{3.591443in}}%
\pgfpathlineto{\pgfqpoint{1.604490in}{3.643962in}}%
\pgfpathlineto{\pgfqpoint{1.635680in}{3.696481in}}%
\pgfpathlineto{\pgfqpoint{1.668884in}{3.749000in}}%
\pgfpathlineto{\pgfqpoint{1.712088in}{3.812886in}}%
\pgfpathlineto{\pgfqpoint{1.761415in}{3.880297in}}%
\pgfpathlineto{\pgfqpoint{1.802519in}{3.932816in}}%
\pgfpathlineto{\pgfqpoint{1.859955in}{4.001016in}}%
\pgfpathlineto{\pgfqpoint{1.892930in}{4.037854in}}%
\pgfpathlineto{\pgfqpoint{1.948675in}{4.096609in}}%
\pgfpathlineto{\pgfqpoint{2.007822in}{4.154608in}}%
\pgfpathlineto{\pgfqpoint{2.066968in}{4.208636in}}%
\pgfpathlineto{\pgfqpoint{2.126115in}{4.259062in}}%
\pgfpathlineto{\pgfqpoint{2.185262in}{4.306209in}}%
\pgfpathlineto{\pgfqpoint{2.248121in}{4.352967in}}%
\pgfpathlineto{\pgfqpoint{2.303555in}{4.391524in}}%
\pgfpathlineto{\pgfqpoint{2.365231in}{4.431746in}}%
\pgfpathlineto{\pgfqpoint{2.421849in}{4.466193in}}%
\pgfpathlineto{\pgfqpoint{2.480996in}{4.499852in}}%
\pgfpathlineto{\pgfqpoint{2.551170in}{4.536784in}}%
\pgfpathlineto{\pgfqpoint{2.605036in}{4.563043in}}%
\pgfpathlineto{\pgfqpoint{2.663107in}{4.589303in}}%
\pgfpathlineto{\pgfqpoint{2.726489in}{4.615562in}}%
\pgfpathlineto{\pgfqpoint{2.796708in}{4.641822in}}%
\pgfpathlineto{\pgfqpoint{2.835876in}{4.655169in}}%
\pgfpathlineto{\pgfqpoint{2.895023in}{4.673550in}}%
\pgfpathlineto{\pgfqpoint{2.954169in}{4.689723in}}%
\pgfpathlineto{\pgfqpoint{3.013316in}{4.703618in}}%
\pgfpathlineto{\pgfqpoint{3.072463in}{4.715230in}}%
\pgfpathlineto{\pgfqpoint{3.131610in}{4.724395in}}%
\pgfpathlineto{\pgfqpoint{3.190756in}{4.730932in}}%
\pgfpathlineto{\pgfqpoint{3.249903in}{4.734743in}}%
\pgfpathlineto{\pgfqpoint{3.309050in}{4.735570in}}%
\pgfpathlineto{\pgfqpoint{3.368197in}{4.733118in}}%
\pgfpathlineto{\pgfqpoint{3.427343in}{4.727045in}}%
\pgfpathlineto{\pgfqpoint{3.467291in}{4.720600in}}%
\pgfpathlineto{\pgfqpoint{3.516063in}{4.710000in}}%
\pgfpathlineto{\pgfqpoint{3.569663in}{4.694341in}}%
\pgfpathlineto{\pgfqpoint{3.575210in}{4.692507in}}%
\pgfpathlineto{\pgfqpoint{3.604783in}{4.681493in}}%
\pgfpathlineto{\pgfqpoint{3.636289in}{4.668081in}}%
\pgfpathlineto{\pgfqpoint{3.663930in}{4.654537in}}%
\pgfpathlineto{\pgfqpoint{3.693504in}{4.638227in}}%
\pgfpathlineto{\pgfqpoint{3.729076in}{4.615562in}}%
\pgfpathlineto{\pgfqpoint{3.764577in}{4.589303in}}%
\pgfpathlineto{\pgfqpoint{3.795522in}{4.563043in}}%
\pgfpathlineto{\pgfqpoint{3.822915in}{4.536784in}}%
\pgfpathlineto{\pgfqpoint{3.847496in}{4.510524in}}%
\pgfpathlineto{\pgfqpoint{3.870944in}{4.482823in}}%
\pgfpathlineto{\pgfqpoint{3.900517in}{4.443433in}}%
\pgfpathlineto{\pgfqpoint{3.930090in}{4.398676in}}%
\pgfpathlineto{\pgfqpoint{3.959664in}{4.347642in}}%
\pgfpathlineto{\pgfqpoint{3.989237in}{4.289426in}}%
\pgfpathlineto{\pgfqpoint{4.018811in}{4.223240in}}%
\pgfpathlineto{\pgfqpoint{4.040435in}{4.169151in}}%
\pgfpathlineto{\pgfqpoint{4.059863in}{4.116632in}}%
\pgfpathlineto{\pgfqpoint{4.086762in}{4.037854in}}%
\pgfpathlineto{\pgfqpoint{4.119682in}{3.932816in}}%
\pgfpathlineto{\pgfqpoint{4.166677in}{3.772576in}}%
\pgfpathlineto{\pgfqpoint{4.226620in}{3.565183in}}%
\pgfpathlineto{\pgfqpoint{4.266281in}{3.433886in}}%
\pgfpathlineto{\pgfqpoint{4.308516in}{3.302589in}}%
\pgfpathlineto{\pgfqpoint{4.353909in}{3.171291in}}%
\pgfpathlineto{\pgfqpoint{4.392902in}{3.066253in}}%
\pgfpathlineto{\pgfqpoint{4.434557in}{2.961215in}}%
\pgfpathlineto{\pgfqpoint{4.467572in}{2.882437in}}%
\pgfpathlineto{\pgfqpoint{4.514173in}{2.777399in}}%
\pgfpathlineto{\pgfqpoint{4.563818in}{2.672361in}}%
\pgfpathlineto{\pgfqpoint{4.616645in}{2.567323in}}%
\pgfpathlineto{\pgfqpoint{4.669425in}{2.468340in}}%
\pgfpathlineto{\pgfqpoint{4.701965in}{2.409766in}}%
\pgfpathlineto{\pgfqpoint{4.758145in}{2.312910in}}%
\pgfpathlineto{\pgfqpoint{4.794731in}{2.252210in}}%
\pgfpathlineto{\pgfqpoint{4.860797in}{2.147172in}}%
\pgfpathlineto{\pgfqpoint{4.930307in}{2.042134in}}%
\pgfpathlineto{\pgfqpoint{5.003198in}{1.937096in}}%
\pgfpathlineto{\pgfqpoint{5.083452in}{1.826827in}}%
\pgfpathlineto{\pgfqpoint{5.142599in}{1.748630in}}%
\pgfpathlineto{\pgfqpoint{5.201745in}{1.672799in}}%
\pgfpathlineto{\pgfqpoint{5.285134in}{1.569463in}}%
\pgfpathlineto{\pgfqpoint{5.379185in}{1.457483in}}%
\pgfpathlineto{\pgfqpoint{5.464590in}{1.359388in}}%
\pgfpathlineto{\pgfqpoint{5.559161in}{1.254350in}}%
\pgfpathlineto{\pgfqpoint{5.656831in}{1.149312in}}%
\pgfpathlineto{\pgfqpoint{5.700715in}{1.103224in}}%
\pgfpathlineto{\pgfqpoint{5.700715in}{1.103224in}}%
\pgfusepath{stroke}%
\end{pgfscope}%
\begin{pgfscope}%
\pgfpathrectangle{\pgfqpoint{0.854460in}{0.571603in}}{\pgfqpoint{5.885100in}{5.225635in}}%
\pgfusepath{clip}%
\pgfsetbuttcap%
\pgfsetroundjoin%
\pgfsetlinewidth{1.505625pt}%
\definecolor{currentstroke}{rgb}{0.257322,0.256130,0.526563}%
\pgfsetstrokecolor{currentstroke}%
\pgfsetdash{}{0pt}%
\pgfpathmoveto{\pgfqpoint{5.971831in}{0.830327in}}%
\pgfpathlineto{\pgfqpoint{5.994893in}{0.807939in}}%
\pgfpathlineto{\pgfqpoint{6.000226in}{0.802787in}}%
\pgfpathlineto{\pgfqpoint{6.022110in}{0.781679in}}%
\pgfpathlineto{\pgfqpoint{6.029799in}{0.774296in}}%
\pgfpathlineto{\pgfqpoint{6.049494in}{0.755420in}}%
\pgfpathlineto{\pgfqpoint{6.059373in}{0.745991in}}%
\pgfpathlineto{\pgfqpoint{6.077041in}{0.729160in}}%
\pgfpathlineto{\pgfqpoint{6.088946in}{0.717864in}}%
\pgfpathlineto{\pgfqpoint{6.104749in}{0.702901in}}%
\pgfpathlineto{\pgfqpoint{6.118520in}{0.689908in}}%
\pgfpathlineto{\pgfqpoint{6.132614in}{0.676641in}}%
\pgfpathlineto{\pgfqpoint{6.148093in}{0.662119in}}%
\pgfpathlineto{\pgfqpoint{6.160634in}{0.650382in}}%
\pgfpathlineto{\pgfqpoint{6.177666in}{0.634488in}}%
\pgfpathlineto{\pgfqpoint{6.188805in}{0.624122in}}%
\pgfpathlineto{\pgfqpoint{6.207240in}{0.607011in}}%
\pgfpathlineto{\pgfqpoint{6.217123in}{0.597863in}}%
\pgfpathlineto{\pgfqpoint{6.236813in}{0.579680in}}%
\pgfpathlineto{\pgfqpoint{6.245586in}{0.571603in}}%
\pgfusepath{stroke}%
\end{pgfscope}%
\begin{pgfscope}%
\pgfpathrectangle{\pgfqpoint{0.854460in}{0.571603in}}{\pgfqpoint{5.885100in}{5.225635in}}%
\pgfusepath{clip}%
\pgfsetbuttcap%
\pgfsetroundjoin%
\pgfsetlinewidth{1.505625pt}%
\definecolor{currentstroke}{rgb}{0.248629,0.278775,0.534556}%
\pgfsetstrokecolor{currentstroke}%
\pgfsetdash{}{0pt}%
\pgfpathmoveto{\pgfqpoint{2.300862in}{0.571603in}}%
\pgfpathlineto{\pgfqpoint{2.214835in}{0.642685in}}%
\pgfpathlineto{\pgfqpoint{2.145448in}{0.702901in}}%
\pgfpathlineto{\pgfqpoint{2.066968in}{0.774661in}}%
\pgfpathlineto{\pgfqpoint{2.005096in}{0.834198in}}%
\pgfpathlineto{\pgfqpoint{1.948675in}{0.891202in}}%
\pgfpathlineto{\pgfqpoint{1.878991in}{0.965495in}}%
\pgfpathlineto{\pgfqpoint{1.830381in}{1.020181in}}%
\pgfpathlineto{\pgfqpoint{1.766099in}{1.096793in}}%
\pgfpathlineto{\pgfqpoint{1.712088in}{1.165492in}}%
\pgfpathlineto{\pgfqpoint{1.665705in}{1.228090in}}%
\pgfpathlineto{\pgfqpoint{1.623368in}{1.288646in}}%
\pgfpathlineto{\pgfqpoint{1.576976in}{1.359388in}}%
\pgfpathlineto{\pgfqpoint{1.534648in}{1.428631in}}%
\pgfpathlineto{\pgfqpoint{1.499224in}{1.490685in}}%
\pgfpathlineto{\pgfqpoint{1.457630in}{1.569463in}}%
\pgfpathlineto{\pgfqpoint{1.431903in}{1.621982in}}%
\pgfpathlineto{\pgfqpoint{1.407732in}{1.674501in}}%
\pgfpathlineto{\pgfqpoint{1.374381in}{1.753280in}}%
\pgfpathlineto{\pgfqpoint{1.344379in}{1.832058in}}%
\pgfpathlineto{\pgfqpoint{1.317635in}{1.910836in}}%
\pgfpathlineto{\pgfqpoint{1.294045in}{1.989615in}}%
\pgfpathlineto{\pgfqpoint{1.273593in}{2.068393in}}%
\pgfpathlineto{\pgfqpoint{1.261653in}{2.120912in}}%
\pgfpathlineto{\pgfqpoint{1.246213in}{2.199691in}}%
\pgfpathlineto{\pgfqpoint{1.233692in}{2.278469in}}%
\pgfpathlineto{\pgfqpoint{1.224100in}{2.357248in}}%
\pgfpathlineto{\pgfqpoint{1.217308in}{2.436026in}}%
\pgfpathlineto{\pgfqpoint{1.213324in}{2.514804in}}%
\pgfpathlineto{\pgfqpoint{1.212153in}{2.593583in}}%
\pgfpathlineto{\pgfqpoint{1.213789in}{2.672361in}}%
\pgfpathlineto{\pgfqpoint{1.218224in}{2.751140in}}%
\pgfpathlineto{\pgfqpoint{1.225437in}{2.829918in}}%
\pgfpathlineto{\pgfqpoint{1.235403in}{2.908696in}}%
\pgfpathlineto{\pgfqpoint{1.248296in}{2.987475in}}%
\pgfpathlineto{\pgfqpoint{1.264009in}{3.066253in}}%
\pgfpathlineto{\pgfqpoint{1.276157in}{3.118772in}}%
\pgfpathlineto{\pgfqpoint{1.298061in}{3.202016in}}%
\pgfpathlineto{\pgfqpoint{1.320592in}{3.276329in}}%
\pgfpathlineto{\pgfqpoint{1.338197in}{3.328848in}}%
\pgfpathlineto{\pgfqpoint{1.367289in}{3.407626in}}%
\pgfpathlineto{\pgfqpoint{1.399761in}{3.486405in}}%
\pgfpathlineto{\pgfqpoint{1.423327in}{3.538924in}}%
\pgfpathlineto{\pgfqpoint{1.461784in}{3.617702in}}%
\pgfpathlineto{\pgfqpoint{1.489523in}{3.670221in}}%
\pgfpathlineto{\pgfqpoint{1.519033in}{3.722740in}}%
\pgfpathlineto{\pgfqpoint{1.550387in}{3.775259in}}%
\pgfpathlineto{\pgfqpoint{1.593795in}{3.843232in}}%
\pgfpathlineto{\pgfqpoint{1.637408in}{3.906556in}}%
\pgfpathlineto{\pgfqpoint{1.682515in}{3.967815in}}%
\pgfpathlineto{\pgfqpoint{1.738024in}{4.037854in}}%
\pgfpathlineto{\pgfqpoint{1.782622in}{4.090373in}}%
\pgfpathlineto{\pgfqpoint{1.830381in}{4.143559in}}%
\pgfpathlineto{\pgfqpoint{1.889528in}{4.205113in}}%
\pgfpathlineto{\pgfqpoint{1.948675in}{4.262641in}}%
\pgfpathlineto{\pgfqpoint{2.007822in}{4.316514in}}%
\pgfpathlineto{\pgfqpoint{2.066968in}{4.367064in}}%
\pgfpathlineto{\pgfqpoint{2.126115in}{4.414587in}}%
\pgfpathlineto{\pgfqpoint{2.185262in}{4.459353in}}%
\pgfpathlineto{\pgfqpoint{2.257714in}{4.510524in}}%
\pgfpathlineto{\pgfqpoint{2.303555in}{4.541116in}}%
\pgfpathlineto{\pgfqpoint{2.380565in}{4.589303in}}%
\pgfpathlineto{\pgfqpoint{2.425097in}{4.615562in}}%
\pgfpathlineto{\pgfqpoint{2.480996in}{4.646787in}}%
\pgfpathlineto{\pgfqpoint{2.540142in}{4.677868in}}%
\pgfpathlineto{\pgfqpoint{2.599289in}{4.707027in}}%
\pgfpathlineto{\pgfqpoint{2.658436in}{4.734304in}}%
\pgfpathlineto{\pgfqpoint{2.717582in}{4.759729in}}%
\pgfpathlineto{\pgfqpoint{2.776729in}{4.783334in}}%
\pgfpathlineto{\pgfqpoint{2.835876in}{4.805144in}}%
\pgfpathlineto{\pgfqpoint{2.896511in}{4.825638in}}%
\pgfpathlineto{\pgfqpoint{2.954169in}{4.843258in}}%
\pgfpathlineto{\pgfqpoint{3.013316in}{4.859561in}}%
\pgfpathlineto{\pgfqpoint{3.091390in}{4.878157in}}%
\pgfpathlineto{\pgfqpoint{3.131610in}{4.886412in}}%
\pgfpathlineto{\pgfqpoint{3.190756in}{4.896831in}}%
\pgfpathlineto{\pgfqpoint{3.249903in}{4.905169in}}%
\pgfpathlineto{\pgfqpoint{3.309050in}{4.911129in}}%
\pgfpathlineto{\pgfqpoint{3.368197in}{4.914700in}}%
\pgfpathlineto{\pgfqpoint{3.427343in}{4.915654in}}%
\pgfpathlineto{\pgfqpoint{3.486490in}{4.913729in}}%
\pgfpathlineto{\pgfqpoint{3.545637in}{4.908626in}}%
\pgfpathlineto{\pgfqpoint{3.604783in}{4.899871in}}%
\pgfpathlineto{\pgfqpoint{3.663930in}{4.886945in}}%
\pgfpathlineto{\pgfqpoint{3.695608in}{4.878157in}}%
\pgfpathlineto{\pgfqpoint{3.723077in}{4.869214in}}%
\pgfpathlineto{\pgfqpoint{3.768080in}{4.851897in}}%
\pgfpathlineto{\pgfqpoint{3.782224in}{4.845744in}}%
\pgfpathlineto{\pgfqpoint{3.822770in}{4.825638in}}%
\pgfpathlineto{\pgfqpoint{3.841370in}{4.815217in}}%
\pgfpathlineto{\pgfqpoint{3.870944in}{4.796864in}}%
\pgfpathlineto{\pgfqpoint{3.904254in}{4.773119in}}%
\pgfpathlineto{\pgfqpoint{3.936265in}{4.746860in}}%
\pgfpathlineto{\pgfqpoint{3.964398in}{4.720600in}}%
\pgfpathlineto{\pgfqpoint{3.989449in}{4.694341in}}%
\pgfpathlineto{\pgfqpoint{4.018811in}{4.659283in}}%
\pgfpathlineto{\pgfqpoint{4.050572in}{4.615562in}}%
\pgfpathlineto{\pgfqpoint{4.083043in}{4.563043in}}%
\pgfpathlineto{\pgfqpoint{4.110873in}{4.510524in}}%
\pgfpathlineto{\pgfqpoint{4.137104in}{4.453284in}}%
\pgfpathlineto{\pgfqpoint{4.156342in}{4.405486in}}%
\pgfpathlineto{\pgfqpoint{4.175426in}{4.352967in}}%
\pgfpathlineto{\pgfqpoint{4.200808in}{4.274189in}}%
\pgfpathlineto{\pgfqpoint{4.225824in}{4.186073in}}%
\pgfpathlineto{\pgfqpoint{4.250214in}{4.090373in}}%
\pgfpathlineto{\pgfqpoint{4.270759in}{4.003979in}}%
\pgfpathlineto{\pgfqpoint{4.270759in}{4.003979in}}%
\pgfusepath{stroke}%
\end{pgfscope}%
\begin{pgfscope}%
\pgfpathrectangle{\pgfqpoint{0.854460in}{0.571603in}}{\pgfqpoint{5.885100in}{5.225635in}}%
\pgfusepath{clip}%
\pgfsetbuttcap%
\pgfsetroundjoin%
\pgfsetlinewidth{1.505625pt}%
\definecolor{currentstroke}{rgb}{0.248629,0.278775,0.534556}%
\pgfsetstrokecolor{currentstroke}%
\pgfsetdash{}{0pt}%
\pgfpathmoveto{\pgfqpoint{4.359434in}{3.622081in}}%
\pgfpathlineto{\pgfqpoint{4.387134in}{3.512664in}}%
\pgfpathlineto{\pgfqpoint{4.415603in}{3.407626in}}%
\pgfpathlineto{\pgfqpoint{4.446234in}{3.302589in}}%
\pgfpathlineto{\pgfqpoint{4.479304in}{3.197551in}}%
\pgfpathlineto{\pgfqpoint{4.515054in}{3.092513in}}%
\pgfpathlineto{\pgfqpoint{4.553661in}{2.987475in}}%
\pgfpathlineto{\pgfqpoint{4.584560in}{2.908696in}}%
\pgfpathlineto{\pgfqpoint{4.617202in}{2.829918in}}%
\pgfpathlineto{\pgfqpoint{4.651645in}{2.751140in}}%
\pgfpathlineto{\pgfqpoint{4.700500in}{2.646102in}}%
\pgfpathlineto{\pgfqpoint{4.739268in}{2.567323in}}%
\pgfpathlineto{\pgfqpoint{4.794010in}{2.462285in}}%
\pgfpathlineto{\pgfqpoint{4.846865in}{2.366718in}}%
\pgfpathlineto{\pgfqpoint{4.882606in}{2.304729in}}%
\pgfpathlineto{\pgfqpoint{4.935585in}{2.216716in}}%
\pgfpathlineto{\pgfqpoint{4.979094in}{2.147172in}}%
\pgfpathlineto{\pgfqpoint{5.047879in}{2.042134in}}%
\pgfpathlineto{\pgfqpoint{5.120193in}{1.937096in}}%
\pgfpathlineto{\pgfqpoint{5.196066in}{1.832058in}}%
\pgfpathlineto{\pgfqpoint{5.275448in}{1.727020in}}%
\pgfpathlineto{\pgfqpoint{5.358370in}{1.621982in}}%
\pgfpathlineto{\pgfqpoint{5.444771in}{1.516944in}}%
\pgfpathlineto{\pgfqpoint{5.534601in}{1.411906in}}%
\pgfpathlineto{\pgfqpoint{5.627812in}{1.306869in}}%
\pgfpathlineto{\pgfqpoint{5.734066in}{1.191503in}}%
\pgfpathlineto{\pgfqpoint{5.824208in}{1.096793in}}%
\pgfpathlineto{\pgfqpoint{5.927151in}{0.991755in}}%
\pgfpathlineto{\pgfqpoint{6.033240in}{0.886717in}}%
\pgfpathlineto{\pgfqpoint{6.148093in}{0.776141in}}%
\pgfpathlineto{\pgfqpoint{6.266386in}{0.665275in}}%
\pgfpathlineto{\pgfqpoint{6.368678in}{0.571603in}}%
\pgfpathlineto{\pgfqpoint{6.368678in}{0.571603in}}%
\pgfusepath{stroke}%
\end{pgfscope}%
\begin{pgfscope}%
\pgfpathrectangle{\pgfqpoint{0.854460in}{0.571603in}}{\pgfqpoint{5.885100in}{5.225635in}}%
\pgfusepath{clip}%
\pgfsetbuttcap%
\pgfsetroundjoin%
\pgfsetlinewidth{1.505625pt}%
\definecolor{currentstroke}{rgb}{0.239346,0.300855,0.540844}%
\pgfsetstrokecolor{currentstroke}%
\pgfsetdash{}{0pt}%
\pgfpathmoveto{\pgfqpoint{2.199834in}{0.571603in}}%
\pgfpathlineto{\pgfqpoint{2.126115in}{0.633238in}}%
\pgfpathlineto{\pgfqpoint{2.066968in}{0.684827in}}%
\pgfpathlineto{\pgfqpoint{2.007822in}{0.738559in}}%
\pgfpathlineto{\pgfqpoint{1.935002in}{0.807939in}}%
\pgfpathlineto{\pgfqpoint{1.882494in}{0.860458in}}%
\pgfpathlineto{\pgfqpoint{1.830381in}{0.914880in}}%
\pgfpathlineto{\pgfqpoint{1.760715in}{0.991755in}}%
\pgfpathlineto{\pgfqpoint{1.712088in}{1.048483in}}%
\pgfpathlineto{\pgfqpoint{1.651680in}{1.123052in}}%
\pgfpathlineto{\pgfqpoint{1.592106in}{1.201831in}}%
\pgfpathlineto{\pgfqpoint{1.536696in}{1.280609in}}%
\pgfpathlineto{\pgfqpoint{1.502001in}{1.333128in}}%
\pgfpathlineto{\pgfqpoint{1.453222in}{1.411906in}}%
\pgfpathlineto{\pgfqpoint{1.416354in}{1.475945in}}%
\pgfpathlineto{\pgfqpoint{1.380218in}{1.543204in}}%
\pgfpathlineto{\pgfqpoint{1.341235in}{1.621982in}}%
\pgfpathlineto{\pgfqpoint{1.317154in}{1.674501in}}%
\pgfpathlineto{\pgfqpoint{1.283855in}{1.753280in}}%
\pgfpathlineto{\pgfqpoint{1.253832in}{1.832058in}}%
\pgfpathlineto{\pgfqpoint{1.226994in}{1.910836in}}%
\pgfpathlineto{\pgfqpoint{1.203240in}{1.989615in}}%
\pgfpathlineto{\pgfqpoint{1.182508in}{2.068393in}}%
\pgfpathlineto{\pgfqpoint{1.170386in}{2.120912in}}%
\pgfpathlineto{\pgfqpoint{1.154541in}{2.199691in}}%
\pgfpathlineto{\pgfqpoint{1.141602in}{2.278469in}}%
\pgfpathlineto{\pgfqpoint{1.131455in}{2.357248in}}%
\pgfpathlineto{\pgfqpoint{1.124030in}{2.436026in}}%
\pgfpathlineto{\pgfqpoint{1.119360in}{2.514804in}}%
\pgfpathlineto{\pgfqpoint{1.117439in}{2.593583in}}%
\pgfpathlineto{\pgfqpoint{1.118192in}{2.672361in}}%
\pgfpathlineto{\pgfqpoint{1.121631in}{2.751140in}}%
\pgfpathlineto{\pgfqpoint{1.127827in}{2.829918in}}%
\pgfpathlineto{\pgfqpoint{1.136707in}{2.908696in}}%
\pgfpathlineto{\pgfqpoint{1.148241in}{2.987475in}}%
\pgfpathlineto{\pgfqpoint{1.157546in}{3.039994in}}%
\pgfpathlineto{\pgfqpoint{1.173771in}{3.118772in}}%
\pgfpathlineto{\pgfqpoint{1.186178in}{3.171291in}}%
\pgfpathlineto{\pgfqpoint{1.209341in}{3.257593in}}%
\pgfpathlineto{\pgfqpoint{1.231212in}{3.328848in}}%
\pgfpathlineto{\pgfqpoint{1.248919in}{3.381367in}}%
\pgfpathlineto{\pgfqpoint{1.278088in}{3.460145in}}%
\pgfpathlineto{\pgfqpoint{1.310523in}{3.538924in}}%
\pgfpathlineto{\pgfqpoint{1.333993in}{3.591443in}}%
\pgfpathlineto{\pgfqpoint{1.372200in}{3.670221in}}%
\pgfpathlineto{\pgfqpoint{1.399685in}{3.722740in}}%
\pgfpathlineto{\pgfqpoint{1.428867in}{3.775259in}}%
\pgfpathlineto{\pgfqpoint{1.459814in}{3.827778in}}%
\pgfpathlineto{\pgfqpoint{1.492593in}{3.880297in}}%
\pgfpathlineto{\pgfqpoint{1.534648in}{3.943640in}}%
\pgfpathlineto{\pgfqpoint{1.583121in}{4.011594in}}%
\pgfpathlineto{\pgfqpoint{1.623368in}{4.064663in}}%
\pgfpathlineto{\pgfqpoint{1.682515in}{4.137440in}}%
\pgfpathlineto{\pgfqpoint{1.733009in}{4.195411in}}%
\pgfpathlineto{\pgfqpoint{1.781493in}{4.247930in}}%
\pgfpathlineto{\pgfqpoint{1.832725in}{4.300449in}}%
\pgfpathlineto{\pgfqpoint{1.889528in}{4.355306in}}%
\pgfpathlineto{\pgfqpoint{1.948675in}{4.409063in}}%
\pgfpathlineto{\pgfqpoint{2.007822in}{4.459742in}}%
\pgfpathlineto{\pgfqpoint{2.070772in}{4.510524in}}%
\pgfpathlineto{\pgfqpoint{2.140132in}{4.563043in}}%
\pgfpathlineto{\pgfqpoint{2.185262in}{4.595492in}}%
\pgfpathlineto{\pgfqpoint{2.253239in}{4.641822in}}%
\pgfpathlineto{\pgfqpoint{2.333129in}{4.692772in}}%
\pgfpathlineto{\pgfqpoint{2.392275in}{4.728081in}}%
\pgfpathlineto{\pgfqpoint{2.472648in}{4.773119in}}%
\pgfpathlineto{\pgfqpoint{2.540142in}{4.808393in}}%
\pgfpathlineto{\pgfqpoint{2.599289in}{4.837469in}}%
\pgfpathlineto{\pgfqpoint{2.658436in}{4.864892in}}%
\pgfpathlineto{\pgfqpoint{2.717582in}{4.890693in}}%
\pgfpathlineto{\pgfqpoint{2.776729in}{4.914904in}}%
\pgfpathlineto{\pgfqpoint{2.835876in}{4.937553in}}%
\pgfpathlineto{\pgfqpoint{2.924596in}{4.968558in}}%
\pgfpathlineto{\pgfqpoint{2.983743in}{4.987318in}}%
\pgfpathlineto{\pgfqpoint{3.072463in}{5.012445in}}%
\pgfpathlineto{\pgfqpoint{3.161183in}{5.033920in}}%
\pgfpathlineto{\pgfqpoint{3.220330in}{5.046064in}}%
\pgfpathlineto{\pgfqpoint{3.279476in}{5.056533in}}%
\pgfpathlineto{\pgfqpoint{3.338623in}{5.065154in}}%
\pgfpathlineto{\pgfqpoint{3.397770in}{5.071765in}}%
\pgfpathlineto{\pgfqpoint{3.456917in}{5.076339in}}%
\pgfpathlineto{\pgfqpoint{3.516063in}{5.078686in}}%
\pgfpathlineto{\pgfqpoint{3.575210in}{5.078591in}}%
\pgfpathlineto{\pgfqpoint{3.634357in}{5.075804in}}%
\pgfpathlineto{\pgfqpoint{3.693504in}{5.070037in}}%
\pgfpathlineto{\pgfqpoint{3.752650in}{5.060927in}}%
\pgfpathlineto{\pgfqpoint{3.811797in}{5.047752in}}%
\pgfpathlineto{\pgfqpoint{3.853806in}{5.035714in}}%
\pgfpathlineto{\pgfqpoint{3.870944in}{5.030097in}}%
\pgfpathlineto{\pgfqpoint{3.924283in}{5.009454in}}%
\pgfpathlineto{\pgfqpoint{3.930090in}{5.006932in}}%
\pgfpathlineto{\pgfqpoint{3.977959in}{4.983195in}}%
\pgfpathlineto{\pgfqpoint{3.989237in}{4.976910in}}%
\pgfpathlineto{\pgfqpoint{4.021760in}{4.956935in}}%
\pgfpathlineto{\pgfqpoint{4.058440in}{4.930676in}}%
\pgfpathlineto{\pgfqpoint{4.090096in}{4.904416in}}%
\pgfpathlineto{\pgfqpoint{4.117805in}{4.878157in}}%
\pgfpathlineto{\pgfqpoint{4.142367in}{4.851897in}}%
\pgfpathlineto{\pgfqpoint{4.166677in}{4.822621in}}%
\pgfpathlineto{\pgfqpoint{4.196251in}{4.781607in}}%
\pgfpathlineto{\pgfqpoint{4.218060in}{4.746860in}}%
\pgfpathlineto{\pgfqpoint{4.232928in}{4.720600in}}%
\pgfpathlineto{\pgfqpoint{4.259187in}{4.668081in}}%
\pgfpathlineto{\pgfqpoint{4.284971in}{4.607037in}}%
\pgfpathlineto{\pgfqpoint{4.301007in}{4.563043in}}%
\pgfpathlineto{\pgfqpoint{4.318101in}{4.510524in}}%
\pgfpathlineto{\pgfqpoint{4.340179in}{4.431746in}}%
\pgfpathlineto{\pgfqpoint{4.353074in}{4.379227in}}%
\pgfpathlineto{\pgfqpoint{4.373691in}{4.285165in}}%
\pgfpathlineto{\pgfqpoint{4.391102in}{4.195411in}}%
\pgfpathlineto{\pgfqpoint{4.419208in}{4.037854in}}%
\pgfpathlineto{\pgfqpoint{4.466010in}{3.775259in}}%
\pgfpathlineto{\pgfqpoint{4.491984in}{3.643777in}}%
\pgfpathlineto{\pgfqpoint{4.514644in}{3.538924in}}%
\pgfpathlineto{\pgfqpoint{4.539528in}{3.433886in}}%
\pgfpathlineto{\pgfqpoint{4.566859in}{3.328848in}}%
\pgfpathlineto{\pgfqpoint{4.596864in}{3.223810in}}%
\pgfpathlineto{\pgfqpoint{4.629743in}{3.118772in}}%
\pgfpathlineto{\pgfqpoint{4.665675in}{3.013734in}}%
\pgfpathlineto{\pgfqpoint{4.704746in}{2.908696in}}%
\pgfpathlineto{\pgfqpoint{4.736204in}{2.829918in}}%
\pgfpathlineto{\pgfqpoint{4.769553in}{2.751140in}}%
\pgfpathlineto{\pgfqpoint{4.804840in}{2.672361in}}%
\pgfpathlineto{\pgfqpoint{4.854954in}{2.567323in}}%
\pgfpathlineto{\pgfqpoint{4.894848in}{2.488545in}}%
\pgfpathlineto{\pgfqpoint{4.951184in}{2.383507in}}%
\pgfpathlineto{\pgfqpoint{5.011142in}{2.278469in}}%
\pgfpathlineto{\pgfqpoint{5.074752in}{2.173431in}}%
\pgfpathlineto{\pgfqpoint{5.142599in}{2.067570in}}%
\pgfpathlineto{\pgfqpoint{5.201745in}{1.979665in}}%
\pgfpathlineto{\pgfqpoint{5.260892in}{1.895327in}}%
\pgfpathlineto{\pgfqpoint{5.320039in}{1.814163in}}%
\pgfpathlineto{\pgfqpoint{5.379185in}{1.735835in}}%
\pgfpathlineto{\pgfqpoint{5.438332in}{1.660051in}}%
\pgfpathlineto{\pgfqpoint{5.497479in}{1.586566in}}%
\pgfpathlineto{\pgfqpoint{5.556626in}{1.515170in}}%
\pgfpathlineto{\pgfqpoint{5.622235in}{1.438166in}}%
\pgfpathlineto{\pgfqpoint{5.714815in}{1.333128in}}%
\pgfpathlineto{\pgfqpoint{5.810879in}{1.228090in}}%
\pgfpathlineto{\pgfqpoint{5.911506in}{1.121905in}}%
\pgfpathlineto{\pgfqpoint{6.013183in}{1.018014in}}%
\pgfpathlineto{\pgfqpoint{6.036192in}{0.994973in}}%
\pgfpathlineto{\pgfqpoint{6.036192in}{0.994973in}}%
\pgfusepath{stroke}%
\end{pgfscope}%
\begin{pgfscope}%
\pgfpathrectangle{\pgfqpoint{0.854460in}{0.571603in}}{\pgfqpoint{5.885100in}{5.225635in}}%
\pgfusepath{clip}%
\pgfsetbuttcap%
\pgfsetroundjoin%
\pgfsetlinewidth{1.505625pt}%
\definecolor{currentstroke}{rgb}{0.239346,0.300855,0.540844}%
\pgfsetstrokecolor{currentstroke}%
\pgfsetdash{}{0pt}%
\pgfpathmoveto{\pgfqpoint{6.313150in}{0.728529in}}%
\pgfpathlineto{\pgfqpoint{6.325533in}{0.717050in}}%
\pgfpathlineto{\pgfqpoint{6.340827in}{0.702901in}}%
\pgfpathlineto{\pgfqpoint{6.355107in}{0.689748in}}%
\pgfpathlineto{\pgfqpoint{6.369368in}{0.676641in}}%
\pgfpathlineto{\pgfqpoint{6.384680in}{0.662627in}}%
\pgfpathlineto{\pgfqpoint{6.398091in}{0.650382in}}%
\pgfpathlineto{\pgfqpoint{6.414253in}{0.635681in}}%
\pgfpathlineto{\pgfqpoint{6.426993in}{0.624122in}}%
\pgfpathlineto{\pgfqpoint{6.443827in}{0.608904in}}%
\pgfpathlineto{\pgfqpoint{6.456073in}{0.597863in}}%
\pgfpathlineto{\pgfqpoint{6.473400in}{0.582291in}}%
\pgfpathlineto{\pgfqpoint{6.485327in}{0.571603in}}%
\pgfusepath{stroke}%
\end{pgfscope}%
\begin{pgfscope}%
\pgfpathrectangle{\pgfqpoint{0.854460in}{0.571603in}}{\pgfqpoint{5.885100in}{5.225635in}}%
\pgfusepath{clip}%
\pgfsetbuttcap%
\pgfsetroundjoin%
\pgfsetlinewidth{1.505625pt}%
\definecolor{currentstroke}{rgb}{0.227802,0.326594,0.546532}%
\pgfsetstrokecolor{currentstroke}%
\pgfsetdash{}{0pt}%
\pgfpathmoveto{\pgfqpoint{2.104286in}{0.571603in}}%
\pgfpathlineto{\pgfqpoint{2.037395in}{0.628224in}}%
\pgfpathlineto{\pgfqpoint{1.978248in}{0.680367in}}%
\pgfpathlineto{\pgfqpoint{1.919102in}{0.734689in}}%
\pgfpathlineto{\pgfqpoint{1.859955in}{0.791401in}}%
\pgfpathlineto{\pgfqpoint{1.791350in}{0.860458in}}%
\pgfpathlineto{\pgfqpoint{1.741628in}{0.912976in}}%
\pgfpathlineto{\pgfqpoint{1.671078in}{0.991755in}}%
\pgfpathlineto{\pgfqpoint{1.623368in}{1.048057in}}%
\pgfpathlineto{\pgfqpoint{1.563287in}{1.123052in}}%
\pgfpathlineto{\pgfqpoint{1.504338in}{1.201831in}}%
\pgfpathlineto{\pgfqpoint{1.449491in}{1.280609in}}%
\pgfpathlineto{\pgfqpoint{1.415088in}{1.333128in}}%
\pgfpathlineto{\pgfqpoint{1.366741in}{1.411906in}}%
\pgfpathlineto{\pgfqpoint{1.327634in}{1.480483in}}%
\pgfpathlineto{\pgfqpoint{1.294223in}{1.543204in}}%
\pgfpathlineto{\pgfqpoint{1.255463in}{1.621982in}}%
\pgfpathlineto{\pgfqpoint{1.220046in}{1.700761in}}%
\pgfpathlineto{\pgfqpoint{1.198266in}{1.753280in}}%
\pgfpathlineto{\pgfqpoint{1.168264in}{1.832058in}}%
\pgfpathlineto{\pgfqpoint{1.141372in}{1.910836in}}%
\pgfpathlineto{\pgfqpoint{1.117491in}{1.989615in}}%
\pgfpathlineto{\pgfqpoint{1.096614in}{2.068393in}}%
\pgfpathlineto{\pgfqpoint{1.078636in}{2.147172in}}%
\pgfpathlineto{\pgfqpoint{1.063407in}{2.225950in}}%
\pgfpathlineto{\pgfqpoint{1.051049in}{2.304729in}}%
\pgfpathlineto{\pgfqpoint{1.041366in}{2.383507in}}%
\pgfpathlineto{\pgfqpoint{1.034333in}{2.462285in}}%
\pgfpathlineto{\pgfqpoint{1.029993in}{2.541064in}}%
\pgfpathlineto{\pgfqpoint{1.028305in}{2.619842in}}%
\pgfpathlineto{\pgfqpoint{1.029217in}{2.698621in}}%
\pgfpathlineto{\pgfqpoint{1.032734in}{2.777399in}}%
\pgfpathlineto{\pgfqpoint{1.038933in}{2.856177in}}%
\pgfpathlineto{\pgfqpoint{1.047737in}{2.934956in}}%
\pgfpathlineto{\pgfqpoint{1.059120in}{3.013734in}}%
\pgfpathlineto{\pgfqpoint{1.073293in}{3.092513in}}%
\pgfpathlineto{\pgfqpoint{1.091047in}{3.175419in}}%
\pgfpathlineto{\pgfqpoint{1.109776in}{3.250070in}}%
\pgfpathlineto{\pgfqpoint{1.132262in}{3.328848in}}%
\pgfpathlineto{\pgfqpoint{1.157660in}{3.407626in}}%
\pgfpathlineto{\pgfqpoint{1.186092in}{3.486405in}}%
\pgfpathlineto{\pgfqpoint{1.217675in}{3.565183in}}%
\pgfpathlineto{\pgfqpoint{1.252523in}{3.643962in}}%
\pgfpathlineto{\pgfqpoint{1.277617in}{3.696481in}}%
\pgfpathlineto{\pgfqpoint{1.318222in}{3.775259in}}%
\pgfpathlineto{\pgfqpoint{1.357208in}{3.844980in}}%
\pgfpathlineto{\pgfqpoint{1.394215in}{3.906556in}}%
\pgfpathlineto{\pgfqpoint{1.427737in}{3.959075in}}%
\pgfpathlineto{\pgfqpoint{1.475501in}{4.029356in}}%
\pgfpathlineto{\pgfqpoint{1.519941in}{4.090373in}}%
\pgfpathlineto{\pgfqpoint{1.564221in}{4.147729in}}%
\pgfpathlineto{\pgfqpoint{1.623368in}{4.219402in}}%
\pgfpathlineto{\pgfqpoint{1.671661in}{4.274189in}}%
\pgfpathlineto{\pgfqpoint{1.720487in}{4.326708in}}%
\pgfpathlineto{\pgfqpoint{1.771971in}{4.379227in}}%
\pgfpathlineto{\pgfqpoint{1.830381in}{4.435432in}}%
\pgfpathlineto{\pgfqpoint{1.889528in}{4.489145in}}%
\pgfpathlineto{\pgfqpoint{1.948675in}{4.539935in}}%
\pgfpathlineto{\pgfqpoint{2.009432in}{4.589303in}}%
\pgfpathlineto{\pgfqpoint{2.078015in}{4.641822in}}%
\pgfpathlineto{\pgfqpoint{2.150852in}{4.694341in}}%
\pgfpathlineto{\pgfqpoint{2.214835in}{4.737794in}}%
\pgfpathlineto{\pgfqpoint{2.273982in}{4.775973in}}%
\pgfpathlineto{\pgfqpoint{2.362702in}{4.829729in}}%
\pgfpathlineto{\pgfqpoint{2.451422in}{4.879643in}}%
\pgfpathlineto{\pgfqpoint{2.540142in}{4.925874in}}%
\pgfpathlineto{\pgfqpoint{2.603777in}{4.956935in}}%
\pgfpathlineto{\pgfqpoint{2.688009in}{4.995371in}}%
\pgfpathlineto{\pgfqpoint{2.776729in}{5.032811in}}%
\pgfpathlineto{\pgfqpoint{2.851808in}{5.061973in}}%
\pgfpathlineto{\pgfqpoint{2.924607in}{5.088233in}}%
\pgfpathlineto{\pgfqpoint{3.013316in}{5.117327in}}%
\pgfpathlineto{\pgfqpoint{3.102036in}{5.143369in}}%
\pgfpathlineto{\pgfqpoint{3.193847in}{5.167011in}}%
\pgfpathlineto{\pgfqpoint{3.279476in}{5.185880in}}%
\pgfpathlineto{\pgfqpoint{3.338623in}{5.197130in}}%
\pgfpathlineto{\pgfqpoint{3.397770in}{5.206742in}}%
\pgfpathlineto{\pgfqpoint{3.456917in}{5.214785in}}%
\pgfpathlineto{\pgfqpoint{3.516063in}{5.221085in}}%
\pgfpathlineto{\pgfqpoint{3.575210in}{5.225452in}}%
\pgfpathlineto{\pgfqpoint{3.634357in}{5.227848in}}%
\pgfpathlineto{\pgfqpoint{3.693504in}{5.228084in}}%
\pgfpathlineto{\pgfqpoint{3.752650in}{5.225945in}}%
\pgfpathlineto{\pgfqpoint{3.811797in}{5.221181in}}%
\pgfpathlineto{\pgfqpoint{3.870944in}{5.213317in}}%
\pgfpathlineto{\pgfqpoint{3.930090in}{5.202039in}}%
\pgfpathlineto{\pgfqpoint{3.966184in}{5.193271in}}%
\pgfpathlineto{\pgfqpoint{4.018811in}{5.177377in}}%
\pgfpathlineto{\pgfqpoint{4.048384in}{5.166789in}}%
\pgfpathlineto{\pgfqpoint{4.077957in}{5.154539in}}%
\pgfpathlineto{\pgfqpoint{4.107787in}{5.140752in}}%
\pgfpathlineto{\pgfqpoint{4.155426in}{5.114492in}}%
\pgfpathlineto{\pgfqpoint{4.166677in}{5.107517in}}%
\pgfpathlineto{\pgfqpoint{4.196251in}{5.087528in}}%
\pgfpathlineto{\pgfqpoint{4.229038in}{5.061973in}}%
\pgfpathlineto{\pgfqpoint{4.258340in}{5.035714in}}%
\pgfpathlineto{\pgfqpoint{4.284971in}{5.008484in}}%
\pgfpathlineto{\pgfqpoint{4.314544in}{4.973504in}}%
\pgfpathlineto{\pgfqpoint{4.344118in}{4.932669in}}%
\pgfpathlineto{\pgfqpoint{4.361842in}{4.904416in}}%
\pgfpathlineto{\pgfqpoint{4.376852in}{4.878157in}}%
\pgfpathlineto{\pgfqpoint{4.403264in}{4.824843in}}%
\pgfpathlineto{\pgfqpoint{4.424655in}{4.773119in}}%
\pgfpathlineto{\pgfqpoint{4.443142in}{4.720600in}}%
\pgfpathlineto{\pgfqpoint{4.462411in}{4.655731in}}%
\pgfpathlineto{\pgfqpoint{4.472658in}{4.615562in}}%
\pgfpathlineto{\pgfqpoint{4.490084in}{4.536784in}}%
\pgfpathlineto{\pgfqpoint{4.499991in}{4.484265in}}%
\pgfpathlineto{\pgfqpoint{4.513101in}{4.405486in}}%
\pgfpathlineto{\pgfqpoint{4.528253in}{4.300449in}}%
\pgfpathlineto{\pgfqpoint{4.551739in}{4.116632in}}%
\pgfpathlineto{\pgfqpoint{4.575227in}{3.932816in}}%
\pgfpathlineto{\pgfqpoint{4.593905in}{3.801519in}}%
\pgfpathlineto{\pgfqpoint{4.610752in}{3.696481in}}%
\pgfpathlineto{\pgfqpoint{4.629524in}{3.591443in}}%
\pgfpathlineto{\pgfqpoint{4.650674in}{3.486405in}}%
\pgfpathlineto{\pgfqpoint{4.674431in}{3.381367in}}%
\pgfpathlineto{\pgfqpoint{4.698998in}{3.283955in}}%
\pgfpathlineto{\pgfqpoint{4.715330in}{3.223810in}}%
\pgfpathlineto{\pgfqpoint{4.746413in}{3.118772in}}%
\pgfpathlineto{\pgfqpoint{4.780720in}{3.013734in}}%
\pgfpathlineto{\pgfqpoint{4.818378in}{2.908696in}}%
\pgfpathlineto{\pgfqpoint{4.848845in}{2.829918in}}%
\pgfpathlineto{\pgfqpoint{4.881272in}{2.751140in}}%
\pgfpathlineto{\pgfqpoint{4.915700in}{2.672361in}}%
\pgfpathlineto{\pgfqpoint{4.952168in}{2.593583in}}%
\pgfpathlineto{\pgfqpoint{5.004000in}{2.488545in}}%
\pgfpathlineto{\pgfqpoint{5.053878in}{2.393981in}}%
\pgfpathlineto{\pgfqpoint{5.088685in}{2.330988in}}%
\pgfpathlineto{\pgfqpoint{5.142599in}{2.238080in}}%
\pgfpathlineto{\pgfqpoint{5.181750in}{2.173431in}}%
\pgfpathlineto{\pgfqpoint{5.231500in}{2.094653in}}%
\pgfpathlineto{\pgfqpoint{5.290465in}{2.005300in}}%
\pgfpathlineto{\pgfqpoint{5.349612in}{1.919527in}}%
\pgfpathlineto{\pgfqpoint{5.393299in}{1.858318in}}%
\pgfpathlineto{\pgfqpoint{5.471361in}{1.753280in}}%
\pgfpathlineto{\pgfqpoint{5.532333in}{1.674501in}}%
\pgfpathlineto{\pgfqpoint{5.580965in}{1.613555in}}%
\pgfpathlineto{\pgfqpoint{5.580965in}{1.613555in}}%
\pgfusepath{stroke}%
\end{pgfscope}%
\begin{pgfscope}%
\pgfpathrectangle{\pgfqpoint{0.854460in}{0.571603in}}{\pgfqpoint{5.885100in}{5.225635in}}%
\pgfusepath{clip}%
\pgfsetbuttcap%
\pgfsetroundjoin%
\pgfsetlinewidth{1.505625pt}%
\definecolor{currentstroke}{rgb}{0.227802,0.326594,0.546532}%
\pgfsetstrokecolor{currentstroke}%
\pgfsetdash{}{0pt}%
\pgfpathmoveto{\pgfqpoint{5.831969in}{1.320530in}}%
\pgfpathlineto{\pgfqpoint{5.844318in}{1.306869in}}%
\pgfpathlineto{\pgfqpoint{5.852359in}{1.298071in}}%
\pgfpathlineto{\pgfqpoint{5.868295in}{1.280609in}}%
\pgfpathlineto{\pgfqpoint{5.881933in}{1.265827in}}%
\pgfpathlineto{\pgfqpoint{5.892507in}{1.254350in}}%
\pgfpathlineto{\pgfqpoint{5.911506in}{1.233946in}}%
\pgfpathlineto{\pgfqpoint{5.916952in}{1.228090in}}%
\pgfpathlineto{\pgfqpoint{5.941079in}{1.202413in}}%
\pgfpathlineto{\pgfqpoint{5.941626in}{1.201831in}}%
\pgfpathlineto{\pgfqpoint{5.966480in}{1.175571in}}%
\pgfpathlineto{\pgfqpoint{5.970653in}{1.171204in}}%
\pgfpathlineto{\pgfqpoint{5.991556in}{1.149312in}}%
\pgfpathlineto{\pgfqpoint{6.000226in}{1.140316in}}%
\pgfpathlineto{\pgfqpoint{6.016857in}{1.123052in}}%
\pgfpathlineto{\pgfqpoint{6.029799in}{1.109741in}}%
\pgfpathlineto{\pgfqpoint{6.042383in}{1.096793in}}%
\pgfpathlineto{\pgfqpoint{6.059373in}{1.079467in}}%
\pgfpathlineto{\pgfqpoint{6.068131in}{1.070533in}}%
\pgfpathlineto{\pgfqpoint{6.088946in}{1.049484in}}%
\pgfpathlineto{\pgfqpoint{6.094098in}{1.044274in}}%
\pgfpathlineto{\pgfqpoint{6.118520in}{1.019783in}}%
\pgfpathlineto{\pgfqpoint{6.120283in}{1.018014in}}%
\pgfpathlineto{\pgfqpoint{6.146667in}{0.991755in}}%
\pgfpathlineto{\pgfqpoint{6.148093in}{0.990345in}}%
\pgfpathlineto{\pgfqpoint{6.173246in}{0.965495in}}%
\pgfpathlineto{\pgfqpoint{6.177666in}{0.961161in}}%
\pgfpathlineto{\pgfqpoint{6.200039in}{0.939236in}}%
\pgfpathlineto{\pgfqpoint{6.207240in}{0.932230in}}%
\pgfpathlineto{\pgfqpoint{6.227044in}{0.912976in}}%
\pgfpathlineto{\pgfqpoint{6.236813in}{0.903545in}}%
\pgfpathlineto{\pgfqpoint{6.254259in}{0.886717in}}%
\pgfpathlineto{\pgfqpoint{6.266386in}{0.875097in}}%
\pgfpathlineto{\pgfqpoint{6.281681in}{0.860458in}}%
\pgfpathlineto{\pgfqpoint{6.295960in}{0.846879in}}%
\pgfpathlineto{\pgfqpoint{6.309310in}{0.834198in}}%
\pgfpathlineto{\pgfqpoint{6.325533in}{0.818883in}}%
\pgfpathlineto{\pgfqpoint{6.337143in}{0.807939in}}%
\pgfpathlineto{\pgfqpoint{6.355107in}{0.791103in}}%
\pgfpathlineto{\pgfqpoint{6.365177in}{0.781679in}}%
\pgfpathlineto{\pgfqpoint{6.384680in}{0.763531in}}%
\pgfpathlineto{\pgfqpoint{6.393411in}{0.755420in}}%
\pgfpathlineto{\pgfqpoint{6.414253in}{0.736161in}}%
\pgfpathlineto{\pgfqpoint{6.421844in}{0.729160in}}%
\pgfpathlineto{\pgfqpoint{6.443827in}{0.708987in}}%
\pgfpathlineto{\pgfqpoint{6.450472in}{0.702901in}}%
\pgfpathlineto{\pgfqpoint{6.473400in}{0.682002in}}%
\pgfpathlineto{\pgfqpoint{6.479294in}{0.676641in}}%
\pgfpathlineto{\pgfqpoint{6.502973in}{0.655201in}}%
\pgfpathlineto{\pgfqpoint{6.508309in}{0.650382in}}%
\pgfpathlineto{\pgfqpoint{6.532547in}{0.628578in}}%
\pgfpathlineto{\pgfqpoint{6.537513in}{0.624122in}}%
\pgfpathlineto{\pgfqpoint{6.562120in}{0.602128in}}%
\pgfpathlineto{\pgfqpoint{6.566905in}{0.597863in}}%
\pgfpathlineto{\pgfqpoint{6.591693in}{0.575845in}}%
\pgfpathlineto{\pgfqpoint{6.596482in}{0.571603in}}%
\pgfusepath{stroke}%
\end{pgfscope}%
\begin{pgfscope}%
\pgfpathrectangle{\pgfqpoint{0.854460in}{0.571603in}}{\pgfqpoint{5.885100in}{5.225635in}}%
\pgfusepath{clip}%
\pgfsetbuttcap%
\pgfsetroundjoin%
\pgfsetlinewidth{1.505625pt}%
\definecolor{currentstroke}{rgb}{0.218130,0.347432,0.550038}%
\pgfsetstrokecolor{currentstroke}%
\pgfsetdash{}{0pt}%
\pgfpathmoveto{\pgfqpoint{2.013550in}{0.571603in}}%
\pgfpathlineto{\pgfqpoint{1.948675in}{0.627249in}}%
\pgfpathlineto{\pgfqpoint{1.889528in}{0.680074in}}%
\pgfpathlineto{\pgfqpoint{1.830381in}{0.735120in}}%
\pgfpathlineto{\pgfqpoint{1.755856in}{0.807939in}}%
\pgfpathlineto{\pgfqpoint{1.704674in}{0.860458in}}%
\pgfpathlineto{\pgfqpoint{1.652941in}{0.915876in}}%
\pgfpathlineto{\pgfqpoint{1.585818in}{0.991755in}}%
\pgfpathlineto{\pgfqpoint{1.534648in}{1.053002in}}%
\pgfpathlineto{\pgfqpoint{1.479239in}{1.123052in}}%
\pgfpathlineto{\pgfqpoint{1.439902in}{1.175571in}}%
\pgfpathlineto{\pgfqpoint{1.384219in}{1.254350in}}%
\pgfpathlineto{\pgfqpoint{1.332476in}{1.333128in}}%
\pgfpathlineto{\pgfqpoint{1.298061in}{1.388984in}}%
\pgfpathlineto{\pgfqpoint{1.254544in}{1.464425in}}%
\pgfpathlineto{\pgfqpoint{1.226152in}{1.516944in}}%
\pgfpathlineto{\pgfqpoint{1.186426in}{1.595723in}}%
\pgfpathlineto{\pgfqpoint{1.161842in}{1.648242in}}%
\pgfpathlineto{\pgfqpoint{1.138700in}{1.700761in}}%
\pgfpathlineto{\pgfqpoint{1.106659in}{1.779539in}}%
\pgfpathlineto{\pgfqpoint{1.077734in}{1.858318in}}%
\pgfpathlineto{\pgfqpoint{1.051830in}{1.937096in}}%
\pgfpathlineto{\pgfqpoint{1.028848in}{2.015874in}}%
\pgfpathlineto{\pgfqpoint{1.008794in}{2.094653in}}%
\pgfpathlineto{\pgfqpoint{0.991544in}{2.173431in}}%
\pgfpathlineto{\pgfqpoint{0.976994in}{2.252210in}}%
\pgfpathlineto{\pgfqpoint{0.965190in}{2.330988in}}%
\pgfpathlineto{\pgfqpoint{0.956035in}{2.409766in}}%
\pgfpathlineto{\pgfqpoint{0.949458in}{2.488545in}}%
\pgfpathlineto{\pgfqpoint{0.945462in}{2.567323in}}%
\pgfpathlineto{\pgfqpoint{0.944046in}{2.646102in}}%
\pgfpathlineto{\pgfqpoint{0.945205in}{2.724880in}}%
\pgfpathlineto{\pgfqpoint{0.948927in}{2.803659in}}%
\pgfpathlineto{\pgfqpoint{0.955198in}{2.882437in}}%
\pgfpathlineto{\pgfqpoint{0.963998in}{2.961215in}}%
\pgfpathlineto{\pgfqpoint{0.975354in}{3.039994in}}%
\pgfpathlineto{\pgfqpoint{0.989412in}{3.118772in}}%
\pgfpathlineto{\pgfqpoint{1.006036in}{3.197551in}}%
\pgfpathlineto{\pgfqpoint{1.025427in}{3.276329in}}%
\pgfpathlineto{\pgfqpoint{1.047595in}{3.355107in}}%
\pgfpathlineto{\pgfqpoint{1.072580in}{3.433886in}}%
\pgfpathlineto{\pgfqpoint{1.100498in}{3.512664in}}%
\pgfpathlineto{\pgfqpoint{1.131463in}{3.591443in}}%
\pgfpathlineto{\pgfqpoint{1.165583in}{3.670221in}}%
\pgfpathlineto{\pgfqpoint{1.190138in}{3.722740in}}%
\pgfpathlineto{\pgfqpoint{1.229814in}{3.801519in}}%
\pgfpathlineto{\pgfqpoint{1.268488in}{3.872339in}}%
\pgfpathlineto{\pgfqpoint{1.303931in}{3.932816in}}%
\pgfpathlineto{\pgfqpoint{1.353521in}{4.011594in}}%
\pgfpathlineto{\pgfqpoint{1.388855in}{4.064113in}}%
\pgfpathlineto{\pgfqpoint{1.445928in}{4.143499in}}%
\pgfpathlineto{\pgfqpoint{1.506726in}{4.221670in}}%
\pgfpathlineto{\pgfqpoint{1.564221in}{4.290401in}}%
\pgfpathlineto{\pgfqpoint{1.619995in}{4.352967in}}%
\pgfpathlineto{\pgfqpoint{1.669622in}{4.405486in}}%
\pgfpathlineto{\pgfqpoint{1.721849in}{4.458005in}}%
\pgfpathlineto{\pgfqpoint{1.776889in}{4.510524in}}%
\pgfpathlineto{\pgfqpoint{1.834963in}{4.563043in}}%
\pgfpathlineto{\pgfqpoint{1.896301in}{4.615562in}}%
\pgfpathlineto{\pgfqpoint{1.961135in}{4.668081in}}%
\pgfpathlineto{\pgfqpoint{2.029706in}{4.720600in}}%
\pgfpathlineto{\pgfqpoint{2.096542in}{4.769004in}}%
\pgfpathlineto{\pgfqpoint{2.155689in}{4.809660in}}%
\pgfpathlineto{\pgfqpoint{2.220144in}{4.851897in}}%
\pgfpathlineto{\pgfqpoint{2.305151in}{4.904416in}}%
\pgfpathlineto{\pgfqpoint{2.396281in}{4.956935in}}%
\pgfpathlineto{\pgfqpoint{2.480996in}{5.002505in}}%
\pgfpathlineto{\pgfqpoint{2.546417in}{5.035714in}}%
\pgfpathlineto{\pgfqpoint{2.628862in}{5.075115in}}%
\pgfpathlineto{\pgfqpoint{2.717582in}{5.114758in}}%
\pgfpathlineto{\pgfqpoint{2.806303in}{5.151393in}}%
\pgfpathlineto{\pgfqpoint{2.895023in}{5.185320in}}%
\pgfpathlineto{\pgfqpoint{2.983743in}{5.216533in}}%
\pgfpathlineto{\pgfqpoint{3.075024in}{5.245790in}}%
\pgfpathlineto{\pgfqpoint{3.165999in}{5.272049in}}%
\pgfpathlineto{\pgfqpoint{3.249903in}{5.293683in}}%
\pgfpathlineto{\pgfqpoint{3.309050in}{5.307395in}}%
\pgfpathlineto{\pgfqpoint{3.397770in}{5.325631in}}%
\pgfpathlineto{\pgfqpoint{3.486490in}{5.340673in}}%
\pgfpathlineto{\pgfqpoint{3.560583in}{5.350827in}}%
\pgfpathlineto{\pgfqpoint{3.604783in}{5.355730in}}%
\pgfpathlineto{\pgfqpoint{3.663930in}{5.360830in}}%
\pgfpathlineto{\pgfqpoint{3.723077in}{5.364221in}}%
\pgfpathlineto{\pgfqpoint{3.782224in}{5.365746in}}%
\pgfpathlineto{\pgfqpoint{3.841370in}{5.365225in}}%
\pgfpathlineto{\pgfqpoint{3.900517in}{5.362451in}}%
\pgfpathlineto{\pgfqpoint{3.959664in}{5.357186in}}%
\pgfpathlineto{\pgfqpoint{4.018811in}{5.349100in}}%
\pgfpathlineto{\pgfqpoint{4.077957in}{5.337608in}}%
\pgfpathlineto{\pgfqpoint{4.137104in}{5.322442in}}%
\pgfpathlineto{\pgfqpoint{4.196251in}{5.302590in}}%
\pgfpathlineto{\pgfqpoint{4.225824in}{5.290618in}}%
\pgfpathlineto{\pgfqpoint{4.265634in}{5.272049in}}%
\pgfpathlineto{\pgfqpoint{4.284971in}{5.261824in}}%
\pgfpathlineto{\pgfqpoint{4.314544in}{5.244629in}}%
\pgfpathlineto{\pgfqpoint{4.351651in}{5.219530in}}%
\pgfpathlineto{\pgfqpoint{4.384920in}{5.193271in}}%
\pgfpathlineto{\pgfqpoint{4.413745in}{5.167011in}}%
\pgfpathlineto{\pgfqpoint{4.439011in}{5.140752in}}%
\pgfpathlineto{\pgfqpoint{4.462411in}{5.113161in}}%
\pgfpathlineto{\pgfqpoint{4.491984in}{5.072419in}}%
\pgfpathlineto{\pgfqpoint{4.514721in}{5.035714in}}%
\pgfpathlineto{\pgfqpoint{4.529049in}{5.009454in}}%
\pgfpathlineto{\pgfqpoint{4.553787in}{4.956935in}}%
\pgfpathlineto{\pgfqpoint{4.574124in}{4.904416in}}%
\pgfpathlineto{\pgfqpoint{4.591037in}{4.851897in}}%
\pgfpathlineto{\pgfqpoint{4.605228in}{4.799378in}}%
\pgfpathlineto{\pgfqpoint{4.617135in}{4.746860in}}%
\pgfpathlineto{\pgfqpoint{4.627209in}{4.694341in}}%
\pgfpathlineto{\pgfqpoint{4.639851in}{4.614696in}}%
\pgfpathlineto{\pgfqpoint{4.649695in}{4.536784in}}%
\pgfpathlineto{\pgfqpoint{4.660514in}{4.431746in}}%
\pgfpathlineto{\pgfqpoint{4.673836in}{4.274189in}}%
\pgfpathlineto{\pgfqpoint{4.679091in}{4.207309in}}%
\pgfpathlineto{\pgfqpoint{4.679091in}{4.207309in}}%
\pgfusepath{stroke}%
\end{pgfscope}%
\begin{pgfscope}%
\pgfpathrectangle{\pgfqpoint{0.854460in}{0.571603in}}{\pgfqpoint{5.885100in}{5.225635in}}%
\pgfusepath{clip}%
\pgfsetbuttcap%
\pgfsetroundjoin%
\pgfsetlinewidth{1.505625pt}%
\definecolor{currentstroke}{rgb}{0.218130,0.347432,0.550038}%
\pgfsetstrokecolor{currentstroke}%
\pgfsetdash{}{0pt}%
\pgfpathmoveto{\pgfqpoint{4.716136in}{3.816309in}}%
\pgfpathlineto{\pgfqpoint{4.728571in}{3.720952in}}%
\pgfpathlineto{\pgfqpoint{4.744042in}{3.617702in}}%
\pgfpathlineto{\pgfqpoint{4.762373in}{3.512664in}}%
\pgfpathlineto{\pgfqpoint{4.777880in}{3.433886in}}%
\pgfpathlineto{\pgfqpoint{4.801135in}{3.328848in}}%
\pgfpathlineto{\pgfqpoint{4.820623in}{3.250070in}}%
\pgfpathlineto{\pgfqpoint{4.846865in}{3.153906in}}%
\pgfpathlineto{\pgfqpoint{4.864967in}{3.092513in}}%
\pgfpathlineto{\pgfqpoint{4.889983in}{3.013734in}}%
\pgfpathlineto{\pgfqpoint{4.916955in}{2.934956in}}%
\pgfpathlineto{\pgfqpoint{4.945914in}{2.856177in}}%
\pgfpathlineto{\pgfqpoint{4.976893in}{2.777399in}}%
\pgfpathlineto{\pgfqpoint{5.009927in}{2.698621in}}%
\pgfpathlineto{\pgfqpoint{5.053878in}{2.600870in}}%
\pgfpathlineto{\pgfqpoint{5.083452in}{2.538718in}}%
\pgfpathlineto{\pgfqpoint{5.121601in}{2.462285in}}%
\pgfpathlineto{\pgfqpoint{5.172172in}{2.366855in}}%
\pgfpathlineto{\pgfqpoint{5.206701in}{2.304729in}}%
\pgfpathlineto{\pgfqpoint{5.260892in}{2.211934in}}%
\pgfpathlineto{\pgfqpoint{5.300397in}{2.147172in}}%
\pgfpathlineto{\pgfqpoint{5.350533in}{2.068393in}}%
\pgfpathlineto{\pgfqpoint{5.408759in}{1.980872in}}%
\pgfpathlineto{\pgfqpoint{5.467906in}{1.895789in}}%
\pgfpathlineto{\pgfqpoint{5.527052in}{1.814116in}}%
\pgfpathlineto{\pgfqpoint{5.586199in}{1.735472in}}%
\pgfpathlineto{\pgfqpoint{5.645346in}{1.659531in}}%
\pgfpathlineto{\pgfqpoint{5.704492in}{1.586020in}}%
\pgfpathlineto{\pgfqpoint{5.763639in}{1.514708in}}%
\pgfpathlineto{\pgfqpoint{5.829029in}{1.438166in}}%
\pgfpathlineto{\pgfqpoint{5.922018in}{1.333128in}}%
\pgfpathlineto{\pgfqpoint{6.018707in}{1.228090in}}%
\pgfpathlineto{\pgfqpoint{6.119074in}{1.123052in}}%
\pgfpathlineto{\pgfqpoint{6.222950in}{1.018014in}}%
\pgfpathlineto{\pgfqpoint{6.330398in}{0.912976in}}%
\pgfpathlineto{\pgfqpoint{6.443827in}{0.805565in}}%
\pgfpathlineto{\pgfqpoint{6.562120in}{0.696898in}}%
\pgfpathlineto{\pgfqpoint{6.680414in}{0.591301in}}%
\pgfpathlineto{\pgfqpoint{6.702832in}{0.571603in}}%
\pgfpathlineto{\pgfqpoint{6.702832in}{0.571603in}}%
\pgfusepath{stroke}%
\end{pgfscope}%
\begin{pgfscope}%
\pgfpathrectangle{\pgfqpoint{0.854460in}{0.571603in}}{\pgfqpoint{5.885100in}{5.225635in}}%
\pgfusepath{clip}%
\pgfsetbuttcap%
\pgfsetroundjoin%
\pgfsetlinewidth{1.505625pt}%
\definecolor{currentstroke}{rgb}{0.206756,0.371758,0.553117}%
\pgfsetstrokecolor{currentstroke}%
\pgfsetdash{}{0pt}%
\pgfpathmoveto{\pgfqpoint{1.927055in}{0.571603in}}%
\pgfpathlineto{\pgfqpoint{1.859955in}{0.629974in}}%
\pgfpathlineto{\pgfqpoint{1.780068in}{0.702901in}}%
\pgfpathlineto{\pgfqpoint{1.712088in}{0.768349in}}%
\pgfpathlineto{\pgfqpoint{1.646978in}{0.834198in}}%
\pgfpathlineto{\pgfqpoint{1.593795in}{0.890682in}}%
\pgfpathlineto{\pgfqpoint{1.526955in}{0.965495in}}%
\pgfpathlineto{\pgfqpoint{1.475501in}{1.026398in}}%
\pgfpathlineto{\pgfqpoint{1.416354in}{1.100402in}}%
\pgfpathlineto{\pgfqpoint{1.357208in}{1.179482in}}%
\pgfpathlineto{\pgfqpoint{1.304912in}{1.254350in}}%
\pgfpathlineto{\pgfqpoint{1.268488in}{1.309653in}}%
\pgfpathlineto{\pgfqpoint{1.221495in}{1.385647in}}%
\pgfpathlineto{\pgfqpoint{1.179767in}{1.458108in}}%
\pgfpathlineto{\pgfqpoint{1.148054in}{1.516944in}}%
\pgfpathlineto{\pgfqpoint{1.108637in}{1.595723in}}%
\pgfpathlineto{\pgfqpoint{1.072470in}{1.674501in}}%
\pgfpathlineto{\pgfqpoint{1.050140in}{1.727020in}}%
\pgfpathlineto{\pgfqpoint{1.019239in}{1.805799in}}%
\pgfpathlineto{\pgfqpoint{0.991365in}{1.884577in}}%
\pgfpathlineto{\pgfqpoint{0.966422in}{1.963355in}}%
\pgfpathlineto{\pgfqpoint{0.943181in}{2.046592in}}%
\pgfpathlineto{\pgfqpoint{0.925123in}{2.120912in}}%
\pgfpathlineto{\pgfqpoint{0.908570in}{2.199691in}}%
\pgfpathlineto{\pgfqpoint{0.894743in}{2.278469in}}%
\pgfpathlineto{\pgfqpoint{0.883463in}{2.357248in}}%
\pgfpathlineto{\pgfqpoint{0.874877in}{2.436026in}}%
\pgfpathlineto{\pgfqpoint{0.868800in}{2.514804in}}%
\pgfpathlineto{\pgfqpoint{0.865230in}{2.593583in}}%
\pgfpathlineto{\pgfqpoint{0.864167in}{2.672361in}}%
\pgfpathlineto{\pgfqpoint{0.865601in}{2.751140in}}%
\pgfpathlineto{\pgfqpoint{0.869523in}{2.829918in}}%
\pgfpathlineto{\pgfqpoint{0.875918in}{2.908696in}}%
\pgfpathlineto{\pgfqpoint{0.884781in}{2.987475in}}%
\pgfpathlineto{\pgfqpoint{0.896280in}{3.066253in}}%
\pgfpathlineto{\pgfqpoint{0.910232in}{3.145032in}}%
\pgfpathlineto{\pgfqpoint{0.926860in}{3.223810in}}%
\pgfpathlineto{\pgfqpoint{0.946045in}{3.302589in}}%
\pgfpathlineto{\pgfqpoint{0.967994in}{3.381367in}}%
\pgfpathlineto{\pgfqpoint{0.984185in}{3.433886in}}%
\pgfpathlineto{\pgfqpoint{1.010784in}{3.512664in}}%
\pgfpathlineto{\pgfqpoint{1.040316in}{3.591443in}}%
\pgfpathlineto{\pgfqpoint{1.072887in}{3.670221in}}%
\pgfpathlineto{\pgfqpoint{1.108602in}{3.749000in}}%
\pgfpathlineto{\pgfqpoint{1.134230in}{3.801519in}}%
\pgfpathlineto{\pgfqpoint{1.161360in}{3.854037in}}%
\pgfpathlineto{\pgfqpoint{1.190046in}{3.906556in}}%
\pgfpathlineto{\pgfqpoint{1.220347in}{3.959075in}}%
\pgfpathlineto{\pgfqpoint{1.252318in}{4.011594in}}%
\pgfpathlineto{\pgfqpoint{1.298061in}{4.082296in}}%
\pgfpathlineto{\pgfqpoint{1.339954in}{4.142892in}}%
\pgfpathlineto{\pgfqpoint{1.386781in}{4.206723in}}%
\pgfpathlineto{\pgfqpoint{1.439543in}{4.274189in}}%
\pgfpathlineto{\pgfqpoint{1.483055in}{4.326708in}}%
\pgfpathlineto{\pgfqpoint{1.534648in}{4.385763in}}%
\pgfpathlineto{\pgfqpoint{1.593795in}{4.449560in}}%
\pgfpathlineto{\pgfqpoint{1.653753in}{4.510524in}}%
\pgfpathlineto{\pgfqpoint{1.712088in}{4.566499in}}%
\pgfpathlineto{\pgfqpoint{1.771235in}{4.620294in}}%
\pgfpathlineto{\pgfqpoint{1.830381in}{4.671381in}}%
\pgfpathlineto{\pgfqpoint{1.890302in}{4.720600in}}%
\pgfpathlineto{\pgfqpoint{1.957769in}{4.773119in}}%
\pgfpathlineto{\pgfqpoint{2.037395in}{4.831650in}}%
\pgfpathlineto{\pgfqpoint{2.104372in}{4.878157in}}%
\pgfpathlineto{\pgfqpoint{2.185262in}{4.931319in}}%
\pgfpathlineto{\pgfqpoint{2.273982in}{4.986008in}}%
\pgfpathlineto{\pgfqpoint{2.362702in}{5.037290in}}%
\pgfpathlineto{\pgfqpoint{2.456952in}{5.088233in}}%
\pgfpathlineto{\pgfqpoint{2.540142in}{5.130362in}}%
\pgfpathlineto{\pgfqpoint{2.628862in}{5.172597in}}%
\pgfpathlineto{\pgfqpoint{2.717582in}{5.212087in}}%
\pgfpathlineto{\pgfqpoint{2.806303in}{5.249008in}}%
\pgfpathlineto{\pgfqpoint{2.895023in}{5.283328in}}%
\pgfpathlineto{\pgfqpoint{2.983743in}{5.315225in}}%
\pgfpathlineto{\pgfqpoint{3.072463in}{5.344695in}}%
\pgfpathlineto{\pgfqpoint{3.161183in}{5.371730in}}%
\pgfpathlineto{\pgfqpoint{3.249903in}{5.396310in}}%
\pgfpathlineto{\pgfqpoint{3.338623in}{5.418410in}}%
\pgfpathlineto{\pgfqpoint{3.427343in}{5.437993in}}%
\pgfpathlineto{\pgfqpoint{3.521176in}{5.455865in}}%
\pgfpathlineto{\pgfqpoint{3.604783in}{5.469069in}}%
\pgfpathlineto{\pgfqpoint{3.693504in}{5.480323in}}%
\pgfpathlineto{\pgfqpoint{3.752650in}{5.485989in}}%
\pgfpathlineto{\pgfqpoint{3.811797in}{5.490125in}}%
\pgfpathlineto{\pgfqpoint{3.870944in}{5.492648in}}%
\pgfpathlineto{\pgfqpoint{3.930090in}{5.493410in}}%
\pgfpathlineto{\pgfqpoint{3.989237in}{5.492242in}}%
\pgfpathlineto{\pgfqpoint{4.048384in}{5.488948in}}%
\pgfpathlineto{\pgfqpoint{4.116872in}{5.482125in}}%
\pgfpathlineto{\pgfqpoint{4.166677in}{5.474813in}}%
\pgfpathlineto{\pgfqpoint{4.225824in}{5.463256in}}%
\pgfpathlineto{\pgfqpoint{4.256774in}{5.455865in}}%
\pgfpathlineto{\pgfqpoint{4.314544in}{5.438804in}}%
\pgfpathlineto{\pgfqpoint{4.344118in}{5.428472in}}%
\pgfpathlineto{\pgfqpoint{4.373691in}{5.416600in}}%
\pgfpathlineto{\pgfqpoint{4.403459in}{5.403346in}}%
\pgfpathlineto{\pgfqpoint{4.452909in}{5.377087in}}%
\pgfpathlineto{\pgfqpoint{4.462411in}{5.371403in}}%
\pgfpathlineto{\pgfqpoint{4.494010in}{5.350827in}}%
\pgfpathlineto{\pgfqpoint{4.528663in}{5.324568in}}%
\pgfpathlineto{\pgfqpoint{4.558542in}{5.298308in}}%
\pgfpathlineto{\pgfqpoint{4.584594in}{5.272049in}}%
\pgfpathlineto{\pgfqpoint{4.610278in}{5.242279in}}%
\pgfpathlineto{\pgfqpoint{4.639851in}{5.201918in}}%
\pgfpathlineto{\pgfqpoint{4.645565in}{5.193271in}}%
\pgfpathlineto{\pgfqpoint{4.669425in}{5.152920in}}%
\pgfpathlineto{\pgfqpoint{4.688728in}{5.114492in}}%
\pgfpathlineto{\pgfqpoint{4.700349in}{5.088233in}}%
\pgfpathlineto{\pgfqpoint{4.720066in}{5.035714in}}%
\pgfpathlineto{\pgfqpoint{4.736157in}{4.983195in}}%
\pgfpathlineto{\pgfqpoint{4.749260in}{4.930676in}}%
\pgfpathlineto{\pgfqpoint{4.759995in}{4.878157in}}%
\pgfpathlineto{\pgfqpoint{4.768657in}{4.825638in}}%
\pgfpathlineto{\pgfqpoint{4.778797in}{4.746860in}}%
\pgfpathlineto{\pgfqpoint{4.786301in}{4.668081in}}%
\pgfpathlineto{\pgfqpoint{4.791822in}{4.589303in}}%
\pgfpathlineto{\pgfqpoint{4.798351in}{4.458005in}}%
\pgfpathlineto{\pgfqpoint{4.816272in}{4.037854in}}%
\pgfpathlineto{\pgfqpoint{4.825305in}{3.906556in}}%
\pgfpathlineto{\pgfqpoint{4.834760in}{3.801519in}}%
\pgfpathlineto{\pgfqpoint{4.846865in}{3.694214in}}%
\pgfpathlineto{\pgfqpoint{4.860864in}{3.591443in}}%
\pgfpathlineto{\pgfqpoint{4.878034in}{3.486405in}}%
\pgfpathlineto{\pgfqpoint{4.892798in}{3.407626in}}%
\pgfpathlineto{\pgfqpoint{4.909381in}{3.328848in}}%
\pgfpathlineto{\pgfqpoint{4.927738in}{3.250070in}}%
\pgfpathlineto{\pgfqpoint{4.947980in}{3.171291in}}%
\pgfpathlineto{\pgfqpoint{4.970176in}{3.092513in}}%
\pgfpathlineto{\pgfqpoint{4.994732in}{3.012562in}}%
\pgfpathlineto{\pgfqpoint{5.024305in}{2.924111in}}%
\pgfpathlineto{\pgfqpoint{5.048701in}{2.856177in}}%
\pgfpathlineto{\pgfqpoint{5.083452in}{2.766376in}}%
\pgfpathlineto{\pgfqpoint{5.113025in}{2.694869in}}%
\pgfpathlineto{\pgfqpoint{5.145926in}{2.619842in}}%
\pgfpathlineto{\pgfqpoint{5.182583in}{2.541064in}}%
\pgfpathlineto{\pgfqpoint{5.221429in}{2.462285in}}%
\pgfpathlineto{\pgfqpoint{5.262485in}{2.383507in}}%
\pgfpathlineto{\pgfqpoint{5.305670in}{2.304729in}}%
\pgfpathlineto{\pgfqpoint{5.351123in}{2.225950in}}%
\pgfpathlineto{\pgfqpoint{5.398719in}{2.147172in}}%
\pgfpathlineto{\pgfqpoint{5.448554in}{2.068393in}}%
\pgfpathlineto{\pgfqpoint{5.500622in}{1.989615in}}%
\pgfpathlineto{\pgfqpoint{5.556626in}{1.908382in}}%
\pgfpathlineto{\pgfqpoint{5.615772in}{1.826041in}}%
\pgfpathlineto{\pgfqpoint{5.674919in}{1.746851in}}%
\pgfpathlineto{\pgfqpoint{5.734066in}{1.670472in}}%
\pgfpathlineto{\pgfqpoint{5.793934in}{1.595723in}}%
\pgfpathlineto{\pgfqpoint{5.859136in}{1.516944in}}%
\pgfpathlineto{\pgfqpoint{5.949465in}{1.411906in}}%
\pgfpathlineto{\pgfqpoint{5.988268in}{1.368147in}}%
\pgfpathlineto{\pgfqpoint{5.988268in}{1.368147in}}%
\pgfusepath{stroke}%
\end{pgfscope}%
\begin{pgfscope}%
\pgfpathrectangle{\pgfqpoint{0.854460in}{0.571603in}}{\pgfqpoint{5.885100in}{5.225635in}}%
\pgfusepath{clip}%
\pgfsetbuttcap%
\pgfsetroundjoin%
\pgfsetlinewidth{1.505625pt}%
\definecolor{currentstroke}{rgb}{0.206756,0.371758,0.553117}%
\pgfsetstrokecolor{currentstroke}%
\pgfsetdash{}{0pt}%
\pgfpathmoveto{\pgfqpoint{6.252171in}{1.087699in}}%
\pgfpathlineto{\pgfqpoint{6.266386in}{1.073341in}}%
\pgfpathlineto{\pgfqpoint{6.269166in}{1.070533in}}%
\pgfpathlineto{\pgfqpoint{6.295390in}{1.044274in}}%
\pgfpathlineto{\pgfqpoint{6.295960in}{1.043707in}}%
\pgfpathlineto{\pgfqpoint{6.321816in}{1.018014in}}%
\pgfpathlineto{\pgfqpoint{6.325533in}{1.014351in}}%
\pgfpathlineto{\pgfqpoint{6.348473in}{0.991755in}}%
\pgfpathlineto{\pgfqpoint{6.355107in}{0.985273in}}%
\pgfpathlineto{\pgfqpoint{6.375358in}{0.965495in}}%
\pgfpathlineto{\pgfqpoint{6.384680in}{0.956464in}}%
\pgfpathlineto{\pgfqpoint{6.402471in}{0.939236in}}%
\pgfpathlineto{\pgfqpoint{6.414253in}{0.927914in}}%
\pgfpathlineto{\pgfqpoint{6.429810in}{0.912976in}}%
\pgfpathlineto{\pgfqpoint{6.443827in}{0.899616in}}%
\pgfpathlineto{\pgfqpoint{6.457372in}{0.886717in}}%
\pgfpathlineto{\pgfqpoint{6.473400in}{0.871562in}}%
\pgfpathlineto{\pgfqpoint{6.485156in}{0.860458in}}%
\pgfpathlineto{\pgfqpoint{6.502973in}{0.843744in}}%
\pgfpathlineto{\pgfqpoint{6.513161in}{0.834198in}}%
\pgfpathlineto{\pgfqpoint{6.532547in}{0.816155in}}%
\pgfpathlineto{\pgfqpoint{6.541385in}{0.807939in}}%
\pgfpathlineto{\pgfqpoint{6.562120in}{0.788787in}}%
\pgfpathlineto{\pgfqpoint{6.569827in}{0.781679in}}%
\pgfpathlineto{\pgfqpoint{6.591693in}{0.761634in}}%
\pgfpathlineto{\pgfqpoint{6.598484in}{0.755420in}}%
\pgfpathlineto{\pgfqpoint{6.621267in}{0.734690in}}%
\pgfpathlineto{\pgfqpoint{6.627356in}{0.729160in}}%
\pgfpathlineto{\pgfqpoint{6.650840in}{0.707949in}}%
\pgfpathlineto{\pgfqpoint{6.656440in}{0.702901in}}%
\pgfpathlineto{\pgfqpoint{6.680414in}{0.681403in}}%
\pgfpathlineto{\pgfqpoint{6.685735in}{0.676641in}}%
\pgfpathlineto{\pgfqpoint{6.709987in}{0.655049in}}%
\pgfpathlineto{\pgfqpoint{6.715240in}{0.650382in}}%
\pgfpathlineto{\pgfqpoint{6.739560in}{0.628879in}}%
\pgfusepath{stroke}%
\end{pgfscope}%
\begin{pgfscope}%
\pgfpathrectangle{\pgfqpoint{0.854460in}{0.571603in}}{\pgfqpoint{5.885100in}{5.225635in}}%
\pgfusepath{clip}%
\pgfsetbuttcap%
\pgfsetroundjoin%
\pgfsetlinewidth{1.505625pt}%
\definecolor{currentstroke}{rgb}{0.197636,0.391528,0.554969}%
\pgfsetstrokecolor{currentstroke}%
\pgfsetdash{}{0pt}%
\pgfpathmoveto{\pgfqpoint{1.844312in}{0.571603in}}%
\pgfpathlineto{\pgfqpoint{1.830381in}{0.583696in}}%
\pgfpathlineto{\pgfqpoint{1.814129in}{0.597863in}}%
\pgfpathlineto{\pgfqpoint{1.812832in}{0.599010in}}%
\pgfusepath{stroke}%
\end{pgfscope}%
\begin{pgfscope}%
\pgfpathrectangle{\pgfqpoint{0.854460in}{0.571603in}}{\pgfqpoint{5.885100in}{5.225635in}}%
\pgfusepath{clip}%
\pgfsetbuttcap%
\pgfsetroundjoin%
\pgfsetlinewidth{1.505625pt}%
\definecolor{currentstroke}{rgb}{0.197636,0.391528,0.554969}%
\pgfsetstrokecolor{currentstroke}%
\pgfsetdash{}{0pt}%
\pgfpathmoveto{\pgfqpoint{1.537085in}{0.866600in}}%
\pgfpathlineto{\pgfqpoint{1.534648in}{0.869231in}}%
\pgfpathlineto{\pgfqpoint{1.518549in}{0.886717in}}%
\pgfpathlineto{\pgfqpoint{1.505074in}{0.901579in}}%
\pgfpathlineto{\pgfqpoint{1.494807in}{0.912976in}}%
\pgfpathlineto{\pgfqpoint{1.475501in}{0.934739in}}%
\pgfpathlineto{\pgfqpoint{1.471538in}{0.939236in}}%
\pgfpathlineto{\pgfqpoint{1.448775in}{0.965495in}}%
\pgfpathlineto{\pgfqpoint{1.445928in}{0.968837in}}%
\pgfpathlineto{\pgfqpoint{1.426533in}{0.991755in}}%
\pgfpathlineto{\pgfqpoint{1.416354in}{1.003971in}}%
\pgfpathlineto{\pgfqpoint{1.404737in}{1.018014in}}%
\pgfpathlineto{\pgfqpoint{1.386781in}{1.040061in}}%
\pgfpathlineto{\pgfqpoint{1.383375in}{1.044274in}}%
\pgfpathlineto{\pgfqpoint{1.362518in}{1.070533in}}%
\pgfpathlineto{\pgfqpoint{1.357208in}{1.077336in}}%
\pgfpathlineto{\pgfqpoint{1.342133in}{1.096793in}}%
\pgfpathlineto{\pgfqpoint{1.327634in}{1.115807in}}%
\pgfpathlineto{\pgfqpoint{1.322152in}{1.123052in}}%
\pgfpathlineto{\pgfqpoint{1.302636in}{1.149312in}}%
\pgfpathlineto{\pgfqpoint{1.298061in}{1.155582in}}%
\pgfpathlineto{\pgfqpoint{1.283593in}{1.175571in}}%
\pgfpathlineto{\pgfqpoint{1.268488in}{1.196783in}}%
\pgfpathlineto{\pgfqpoint{1.264923in}{1.201831in}}%
\pgfpathlineto{\pgfqpoint{1.246738in}{1.228090in}}%
\pgfpathlineto{\pgfqpoint{1.238914in}{1.239592in}}%
\pgfpathlineto{\pgfqpoint{1.228963in}{1.254350in}}%
\pgfpathlineto{\pgfqpoint{1.211563in}{1.280609in}}%
\pgfpathlineto{\pgfqpoint{1.209341in}{1.284034in}}%
\pgfpathlineto{\pgfqpoint{1.194658in}{1.306869in}}%
\pgfpathlineto{\pgfqpoint{1.179767in}{1.330421in}}%
\pgfpathlineto{\pgfqpoint{1.178071in}{1.333128in}}%
\pgfpathlineto{\pgfqpoint{1.161980in}{1.359388in}}%
\pgfpathlineto{\pgfqpoint{1.150194in}{1.378972in}}%
\pgfpathlineto{\pgfqpoint{1.146215in}{1.385647in}}%
\pgfpathlineto{\pgfqpoint{1.130900in}{1.411906in}}%
\pgfpathlineto{\pgfqpoint{1.120621in}{1.429878in}}%
\pgfpathlineto{\pgfqpoint{1.115926in}{1.438166in}}%
\pgfpathlineto{\pgfqpoint{1.101382in}{1.464425in}}%
\pgfpathlineto{\pgfqpoint{1.091047in}{1.483468in}}%
\pgfpathlineto{\pgfqpoint{1.087169in}{1.490685in}}%
\pgfpathlineto{\pgfqpoint{1.073392in}{1.516944in}}%
\pgfpathlineto{\pgfqpoint{1.061474in}{1.540127in}}%
\pgfpathlineto{\pgfqpoint{1.059908in}{1.543204in}}%
\pgfpathlineto{\pgfqpoint{1.046889in}{1.569463in}}%
\pgfpathlineto{\pgfqpoint{1.034140in}{1.595723in}}%
\pgfpathlineto{\pgfqpoint{1.031901in}{1.600460in}}%
\pgfpathlineto{\pgfqpoint{1.021834in}{1.621982in}}%
\pgfpathlineto{\pgfqpoint{1.009837in}{1.648242in}}%
\pgfpathlineto{\pgfqpoint{1.002327in}{1.665097in}}%
\pgfpathlineto{\pgfqpoint{0.998184in}{1.674501in}}%
\pgfpathlineto{\pgfqpoint{0.986928in}{1.700761in}}%
\pgfpathlineto{\pgfqpoint{0.975946in}{1.727020in}}%
\pgfpathlineto{\pgfqpoint{0.972754in}{1.734881in}}%
\pgfpathlineto{\pgfqpoint{0.965367in}{1.753280in}}%
\pgfpathlineto{\pgfqpoint{0.955117in}{1.779539in}}%
\pgfpathlineto{\pgfqpoint{0.945142in}{1.805799in}}%
\pgfpathlineto{\pgfqpoint{0.943181in}{1.811131in}}%
\pgfpathlineto{\pgfqpoint{0.935576in}{1.832058in}}%
\pgfpathlineto{\pgfqpoint{0.926318in}{1.858318in}}%
\pgfpathlineto{\pgfqpoint{0.917338in}{1.884577in}}%
\pgfpathlineto{\pgfqpoint{0.913607in}{1.895866in}}%
\pgfpathlineto{\pgfqpoint{0.908720in}{1.910836in}}%
\pgfpathlineto{\pgfqpoint{0.900442in}{1.937096in}}%
\pgfpathlineto{\pgfqpoint{0.892441in}{1.963355in}}%
\pgfpathlineto{\pgfqpoint{0.884718in}{1.989615in}}%
\pgfpathlineto{\pgfqpoint{0.884034in}{1.992037in}}%
\pgfpathlineto{\pgfqpoint{0.877390in}{2.015874in}}%
\pgfpathlineto{\pgfqpoint{0.870350in}{2.042134in}}%
\pgfpathlineto{\pgfqpoint{0.863588in}{2.068393in}}%
\pgfpathlineto{\pgfqpoint{0.857103in}{2.094653in}}%
\pgfpathlineto{\pgfqpoint{0.854460in}{2.105859in}}%
\pgfusepath{stroke}%
\end{pgfscope}%
\begin{pgfscope}%
\pgfpathrectangle{\pgfqpoint{0.854460in}{0.571603in}}{\pgfqpoint{5.885100in}{5.225635in}}%
\pgfusepath{clip}%
\pgfsetbuttcap%
\pgfsetroundjoin%
\pgfsetlinewidth{1.505625pt}%
\definecolor{currentstroke}{rgb}{0.197636,0.391528,0.554969}%
\pgfsetstrokecolor{currentstroke}%
\pgfsetdash{}{0pt}%
\pgfpathmoveto{\pgfqpoint{0.854460in}{3.261442in}}%
\pgfpathlineto{\pgfqpoint{0.871070in}{3.328848in}}%
\pgfpathlineto{\pgfqpoint{0.892890in}{3.407626in}}%
\pgfpathlineto{\pgfqpoint{0.917409in}{3.486405in}}%
\pgfpathlineto{\pgfqpoint{0.944736in}{3.565183in}}%
\pgfpathlineto{\pgfqpoint{0.974979in}{3.643962in}}%
\pgfpathlineto{\pgfqpoint{1.008241in}{3.722740in}}%
\pgfpathlineto{\pgfqpoint{1.044619in}{3.801519in}}%
\pgfpathlineto{\pgfqpoint{1.084208in}{3.880297in}}%
\pgfpathlineto{\pgfqpoint{1.120621in}{3.947404in}}%
\pgfpathlineto{\pgfqpoint{1.157885in}{4.011594in}}%
\pgfpathlineto{\pgfqpoint{1.206943in}{4.090373in}}%
\pgfpathlineto{\pgfqpoint{1.241838in}{4.142892in}}%
\pgfpathlineto{\pgfqpoint{1.298061in}{4.222338in}}%
\pgfpathlineto{\pgfqpoint{1.357653in}{4.300449in}}%
\pgfpathlineto{\pgfqpoint{1.416354in}{4.372106in}}%
\pgfpathlineto{\pgfqpoint{1.468280in}{4.431746in}}%
\pgfpathlineto{\pgfqpoint{1.516473in}{4.484265in}}%
\pgfpathlineto{\pgfqpoint{1.567031in}{4.536784in}}%
\pgfpathlineto{\pgfqpoint{1.623368in}{4.592349in}}%
\pgfpathlineto{\pgfqpoint{1.682515in}{4.647684in}}%
\pgfpathlineto{\pgfqpoint{1.741661in}{4.700257in}}%
\pgfpathlineto{\pgfqpoint{1.800808in}{4.750305in}}%
\pgfpathlineto{\pgfqpoint{1.861687in}{4.799378in}}%
\pgfpathlineto{\pgfqpoint{1.930308in}{4.851897in}}%
\pgfpathlineto{\pgfqpoint{2.007822in}{4.908104in}}%
\pgfpathlineto{\pgfqpoint{2.079026in}{4.956935in}}%
\pgfpathlineto{\pgfqpoint{2.159813in}{5.009454in}}%
\pgfpathlineto{\pgfqpoint{2.245485in}{5.061973in}}%
\pgfpathlineto{\pgfqpoint{2.336615in}{5.114492in}}%
\pgfpathlineto{\pgfqpoint{2.433832in}{5.167011in}}%
\pgfpathlineto{\pgfqpoint{2.510569in}{5.206078in}}%
\pgfpathlineto{\pgfqpoint{2.599289in}{5.248832in}}%
\pgfpathlineto{\pgfqpoint{2.688009in}{5.288968in}}%
\pgfpathlineto{\pgfqpoint{2.776729in}{5.326758in}}%
\pgfpathlineto{\pgfqpoint{2.865449in}{5.362111in}}%
\pgfpathlineto{\pgfqpoint{2.954169in}{5.395250in}}%
\pgfpathlineto{\pgfqpoint{3.053137in}{5.429606in}}%
\pgfpathlineto{\pgfqpoint{3.134672in}{5.455865in}}%
\pgfpathlineto{\pgfqpoint{3.222709in}{5.482125in}}%
\pgfpathlineto{\pgfqpoint{3.319161in}{5.508384in}}%
\pgfpathlineto{\pgfqpoint{3.397770in}{5.527847in}}%
\pgfpathlineto{\pgfqpoint{3.486490in}{5.547673in}}%
\pgfpathlineto{\pgfqpoint{3.575210in}{5.565213in}}%
\pgfpathlineto{\pgfqpoint{3.663930in}{5.580241in}}%
\pgfpathlineto{\pgfqpoint{3.752650in}{5.592701in}}%
\pgfpathlineto{\pgfqpoint{3.841370in}{5.602335in}}%
\pgfpathlineto{\pgfqpoint{3.930090in}{5.609001in}}%
\pgfpathlineto{\pgfqpoint{3.989237in}{5.611604in}}%
\pgfpathlineto{\pgfqpoint{4.048384in}{5.612577in}}%
\pgfpathlineto{\pgfqpoint{4.107531in}{5.611764in}}%
\pgfpathlineto{\pgfqpoint{4.166677in}{5.608987in}}%
\pgfpathlineto{\pgfqpoint{4.225824in}{5.604040in}}%
\pgfpathlineto{\pgfqpoint{4.284971in}{5.596687in}}%
\pgfpathlineto{\pgfqpoint{4.344118in}{5.586635in}}%
\pgfpathlineto{\pgfqpoint{4.403264in}{5.573122in}}%
\pgfpathlineto{\pgfqpoint{4.462411in}{5.555890in}}%
\pgfpathlineto{\pgfqpoint{4.521558in}{5.533967in}}%
\pgfpathlineto{\pgfqpoint{4.576240in}{5.508384in}}%
\pgfpathlineto{\pgfqpoint{4.610278in}{5.489276in}}%
\pgfpathlineto{\pgfqpoint{4.639851in}{5.470324in}}%
\pgfpathlineto{\pgfqpoint{4.669425in}{5.448803in}}%
\pgfpathlineto{\pgfqpoint{4.698998in}{5.424110in}}%
\pgfpathlineto{\pgfqpoint{4.728571in}{5.395479in}}%
\pgfpathlineto{\pgfqpoint{4.758145in}{5.361914in}}%
\pgfpathlineto{\pgfqpoint{4.766937in}{5.350827in}}%
\pgfpathlineto{\pgfqpoint{4.787718in}{5.321971in}}%
\pgfpathlineto{\pgfqpoint{4.802723in}{5.298308in}}%
\pgfpathlineto{\pgfqpoint{4.817754in}{5.272049in}}%
\pgfpathlineto{\pgfqpoint{4.830972in}{5.245790in}}%
\pgfpathlineto{\pgfqpoint{4.846865in}{5.209901in}}%
\pgfpathlineto{\pgfqpoint{4.862875in}{5.167011in}}%
\pgfpathlineto{\pgfqpoint{4.878890in}{5.114492in}}%
\pgfpathlineto{\pgfqpoint{4.891506in}{5.061973in}}%
\pgfpathlineto{\pgfqpoint{4.901533in}{5.009454in}}%
\pgfpathlineto{\pgfqpoint{4.909337in}{4.956935in}}%
\pgfpathlineto{\pgfqpoint{4.915300in}{4.904416in}}%
\pgfpathlineto{\pgfqpoint{4.921600in}{4.825638in}}%
\pgfpathlineto{\pgfqpoint{4.925464in}{4.746860in}}%
\pgfpathlineto{\pgfqpoint{4.927987in}{4.641822in}}%
\pgfpathlineto{\pgfqpoint{4.928650in}{4.496916in}}%
\pgfpathlineto{\pgfqpoint{4.928650in}{4.496916in}}%
\pgfusepath{stroke}%
\end{pgfscope}%
\begin{pgfscope}%
\pgfpathrectangle{\pgfqpoint{0.854460in}{0.571603in}}{\pgfqpoint{5.885100in}{5.225635in}}%
\pgfusepath{clip}%
\pgfsetbuttcap%
\pgfsetroundjoin%
\pgfsetlinewidth{1.505625pt}%
\definecolor{currentstroke}{rgb}{0.197636,0.391528,0.554969}%
\pgfsetstrokecolor{currentstroke}%
\pgfsetdash{}{0pt}%
\pgfpathmoveto{\pgfqpoint{4.930491in}{4.103998in}}%
\pgfpathlineto{\pgfqpoint{4.933295in}{4.011594in}}%
\pgfpathlineto{\pgfqpoint{4.938317in}{3.906556in}}%
\pgfpathlineto{\pgfqpoint{4.945574in}{3.801519in}}%
\pgfpathlineto{\pgfqpoint{4.955389in}{3.696481in}}%
\pgfpathlineto{\pgfqpoint{4.967993in}{3.591443in}}%
\pgfpathlineto{\pgfqpoint{4.979334in}{3.512664in}}%
\pgfpathlineto{\pgfqpoint{4.994732in}{3.421582in}}%
\pgfpathlineto{\pgfqpoint{5.007375in}{3.355107in}}%
\pgfpathlineto{\pgfqpoint{5.024305in}{3.275940in}}%
\pgfpathlineto{\pgfqpoint{5.042897in}{3.197551in}}%
\pgfpathlineto{\pgfqpoint{5.063595in}{3.118772in}}%
\pgfpathlineto{\pgfqpoint{5.086330in}{3.039994in}}%
\pgfpathlineto{\pgfqpoint{5.113025in}{2.955458in}}%
\pgfpathlineto{\pgfqpoint{5.137947in}{2.882437in}}%
\pgfpathlineto{\pgfqpoint{5.172172in}{2.790114in}}%
\pgfpathlineto{\pgfqpoint{5.201745in}{2.715976in}}%
\pgfpathlineto{\pgfqpoint{5.231354in}{2.646102in}}%
\pgfpathlineto{\pgfqpoint{5.266804in}{2.567323in}}%
\pgfpathlineto{\pgfqpoint{5.304469in}{2.488545in}}%
\pgfpathlineto{\pgfqpoint{5.349612in}{2.399851in}}%
\pgfpathlineto{\pgfqpoint{5.386504in}{2.330988in}}%
\pgfpathlineto{\pgfqpoint{5.438332in}{2.239428in}}%
\pgfpathlineto{\pgfqpoint{5.477471in}{2.173431in}}%
\pgfpathlineto{\pgfqpoint{5.527052in}{2.093570in}}%
\pgfpathlineto{\pgfqpoint{5.586199in}{2.002780in}}%
\pgfpathlineto{\pgfqpoint{5.630782in}{1.937096in}}%
\pgfpathlineto{\pgfqpoint{5.704492in}{1.833385in}}%
\pgfpathlineto{\pgfqpoint{5.744235in}{1.779539in}}%
\pgfpathlineto{\pgfqpoint{5.824870in}{1.674501in}}%
\pgfpathlineto{\pgfqpoint{5.887912in}{1.595723in}}%
\pgfpathlineto{\pgfqpoint{5.975441in}{1.490685in}}%
\pgfpathlineto{\pgfqpoint{6.066872in}{1.385647in}}%
\pgfpathlineto{\pgfqpoint{6.162185in}{1.280609in}}%
\pgfpathlineto{\pgfqpoint{6.261365in}{1.175571in}}%
\pgfpathlineto{\pgfqpoint{6.364315in}{1.070533in}}%
\pgfpathlineto{\pgfqpoint{6.473400in}{0.963222in}}%
\pgfpathlineto{\pgfqpoint{6.581411in}{0.860458in}}%
\pgfpathlineto{\pgfqpoint{6.695420in}{0.755420in}}%
\pgfpathlineto{\pgfqpoint{6.739560in}{0.715624in}}%
\pgfpathlineto{\pgfqpoint{6.739560in}{0.715624in}}%
\pgfusepath{stroke}%
\end{pgfscope}%
\begin{pgfscope}%
\pgfpathrectangle{\pgfqpoint{0.854460in}{0.571603in}}{\pgfqpoint{5.885100in}{5.225635in}}%
\pgfusepath{clip}%
\pgfsetbuttcap%
\pgfsetroundjoin%
\pgfsetlinewidth{1.505625pt}%
\definecolor{currentstroke}{rgb}{0.187231,0.414746,0.556547}%
\pgfsetstrokecolor{currentstroke}%
\pgfsetdash{}{0pt}%
\pgfpathmoveto{\pgfqpoint{1.764898in}{0.571603in}}%
\pgfpathlineto{\pgfqpoint{1.741661in}{0.592038in}}%
\pgfpathlineto{\pgfqpoint{1.735067in}{0.597863in}}%
\pgfpathlineto{\pgfqpoint{1.712088in}{0.618460in}}%
\pgfpathlineto{\pgfqpoint{1.705799in}{0.624122in}}%
\pgfpathlineto{\pgfqpoint{1.682515in}{0.645400in}}%
\pgfpathlineto{\pgfqpoint{1.677089in}{0.650382in}}%
\pgfpathlineto{\pgfqpoint{1.652941in}{0.672882in}}%
\pgfpathlineto{\pgfqpoint{1.648927in}{0.676641in}}%
\pgfpathlineto{\pgfqpoint{1.623368in}{0.700931in}}%
\pgfpathlineto{\pgfqpoint{1.621306in}{0.702901in}}%
\pgfpathlineto{\pgfqpoint{1.594224in}{0.729160in}}%
\pgfpathlineto{\pgfqpoint{1.593795in}{0.729584in}}%
\pgfpathlineto{\pgfqpoint{1.567705in}{0.755420in}}%
\pgfpathlineto{\pgfqpoint{1.564221in}{0.758922in}}%
\pgfpathlineto{\pgfqpoint{1.541709in}{0.781679in}}%
\pgfpathlineto{\pgfqpoint{1.534648in}{0.788925in}}%
\pgfpathlineto{\pgfqpoint{1.516225in}{0.807939in}}%
\pgfpathlineto{\pgfqpoint{1.505074in}{0.819622in}}%
\pgfpathlineto{\pgfqpoint{1.491244in}{0.834198in}}%
\pgfpathlineto{\pgfqpoint{1.475501in}{0.851044in}}%
\pgfpathlineto{\pgfqpoint{1.466757in}{0.860458in}}%
\pgfpathlineto{\pgfqpoint{1.445928in}{0.883223in}}%
\pgfpathlineto{\pgfqpoint{1.442751in}{0.886717in}}%
\pgfpathlineto{\pgfqpoint{1.419262in}{0.912976in}}%
\pgfpathlineto{\pgfqpoint{1.416354in}{0.916281in}}%
\pgfpathlineto{\pgfqpoint{1.396290in}{0.939236in}}%
\pgfpathlineto{\pgfqpoint{1.390182in}{0.946332in}}%
\pgfusepath{stroke}%
\end{pgfscope}%
\begin{pgfscope}%
\pgfpathrectangle{\pgfqpoint{0.854460in}{0.571603in}}{\pgfqpoint{5.885100in}{5.225635in}}%
\pgfusepath{clip}%
\pgfsetbuttcap%
\pgfsetroundjoin%
\pgfsetlinewidth{1.505625pt}%
\definecolor{currentstroke}{rgb}{0.187231,0.414746,0.556547}%
\pgfsetstrokecolor{currentstroke}%
\pgfsetdash{}{0pt}%
\pgfpathmoveto{\pgfqpoint{1.156225in}{1.254043in}}%
\pgfpathlineto{\pgfqpoint{1.156020in}{1.254350in}}%
\pgfpathlineto{\pgfqpoint{1.150194in}{1.263196in}}%
\pgfpathlineto{\pgfqpoint{1.138827in}{1.280609in}}%
\pgfpathlineto{\pgfqpoint{1.121977in}{1.306869in}}%
\pgfpathlineto{\pgfqpoint{1.120621in}{1.309029in}}%
\pgfpathlineto{\pgfqpoint{1.105625in}{1.333128in}}%
\pgfpathlineto{\pgfqpoint{1.091047in}{1.356957in}}%
\pgfpathlineto{\pgfqpoint{1.089574in}{1.359388in}}%
\pgfpathlineto{\pgfqpoint{1.074010in}{1.385647in}}%
\pgfpathlineto{\pgfqpoint{1.061474in}{1.407183in}}%
\pgfpathlineto{\pgfqpoint{1.058751in}{1.411906in}}%
\pgfpathlineto{\pgfqpoint{1.043950in}{1.438166in}}%
\pgfpathlineto{\pgfqpoint{1.031901in}{1.459955in}}%
\pgfpathlineto{\pgfqpoint{1.029452in}{1.464425in}}%
\pgfpathlineto{\pgfqpoint{1.015410in}{1.490685in}}%
\pgfpathlineto{\pgfqpoint{1.002327in}{1.515627in}}%
\pgfpathlineto{\pgfqpoint{1.001643in}{1.516944in}}%
\pgfpathlineto{\pgfqpoint{0.988351in}{1.543204in}}%
\pgfpathlineto{\pgfqpoint{0.975326in}{1.569463in}}%
\pgfpathlineto{\pgfqpoint{0.972754in}{1.574783in}}%
\pgfpathlineto{\pgfqpoint{0.962735in}{1.595723in}}%
\pgfpathlineto{\pgfqpoint{0.950454in}{1.621982in}}%
\pgfpathlineto{\pgfqpoint{0.943181in}{1.637919in}}%
\pgfpathlineto{\pgfqpoint{0.938520in}{1.648242in}}%
\pgfpathlineto{\pgfqpoint{0.926974in}{1.674501in}}%
\pgfpathlineto{\pgfqpoint{0.915697in}{1.700761in}}%
\pgfpathlineto{\pgfqpoint{0.913607in}{1.705770in}}%
\pgfpathlineto{\pgfqpoint{0.904840in}{1.727020in}}%
\pgfpathlineto{\pgfqpoint{0.894288in}{1.753280in}}%
\pgfpathlineto{\pgfqpoint{0.884034in}{1.779472in}}%
\pgfpathlineto{\pgfqpoint{0.884008in}{1.779539in}}%
\pgfpathlineto{\pgfqpoint{0.874167in}{1.805799in}}%
\pgfpathlineto{\pgfqpoint{0.864598in}{1.832058in}}%
\pgfpathlineto{\pgfqpoint{0.855302in}{1.858318in}}%
\pgfpathlineto{\pgfqpoint{0.854460in}{1.860779in}}%
\pgfusepath{stroke}%
\end{pgfscope}%
\begin{pgfscope}%
\pgfpathrectangle{\pgfqpoint{0.854460in}{0.571603in}}{\pgfqpoint{5.885100in}{5.225635in}}%
\pgfusepath{clip}%
\pgfsetbuttcap%
\pgfsetroundjoin%
\pgfsetlinewidth{1.505625pt}%
\definecolor{currentstroke}{rgb}{0.187231,0.414746,0.556547}%
\pgfsetstrokecolor{currentstroke}%
\pgfsetdash{}{0pt}%
\pgfpathmoveto{\pgfqpoint{0.854460in}{3.537749in}}%
\pgfpathlineto{\pgfqpoint{0.873372in}{3.591443in}}%
\pgfpathlineto{\pgfqpoint{0.903426in}{3.670221in}}%
\pgfpathlineto{\pgfqpoint{0.936405in}{3.749000in}}%
\pgfpathlineto{\pgfqpoint{0.960110in}{3.801519in}}%
\pgfpathlineto{\pgfqpoint{0.985190in}{3.854037in}}%
\pgfpathlineto{\pgfqpoint{1.025496in}{3.932816in}}%
\pgfpathlineto{\pgfqpoint{1.061474in}{3.998110in}}%
\pgfpathlineto{\pgfqpoint{1.100327in}{4.064113in}}%
\pgfpathlineto{\pgfqpoint{1.150194in}{4.143173in}}%
\pgfpathlineto{\pgfqpoint{1.203682in}{4.221670in}}%
\pgfpathlineto{\pgfqpoint{1.241686in}{4.274189in}}%
\pgfpathlineto{\pgfqpoint{1.298061in}{4.347604in}}%
\pgfpathlineto{\pgfqpoint{1.345307in}{4.405486in}}%
\pgfpathlineto{\pgfqpoint{1.390329in}{4.458005in}}%
\pgfpathlineto{\pgfqpoint{1.445928in}{4.519478in}}%
\pgfpathlineto{\pgfqpoint{1.505074in}{4.581312in}}%
\pgfpathlineto{\pgfqpoint{1.566263in}{4.641822in}}%
\pgfpathlineto{\pgfqpoint{1.623368in}{4.695374in}}%
\pgfpathlineto{\pgfqpoint{1.682515in}{4.748154in}}%
\pgfpathlineto{\pgfqpoint{1.742781in}{4.799378in}}%
\pgfpathlineto{\pgfqpoint{1.807741in}{4.851897in}}%
\pgfpathlineto{\pgfqpoint{1.889528in}{4.914471in}}%
\pgfpathlineto{\pgfqpoint{1.948675in}{4.957513in}}%
\pgfpathlineto{\pgfqpoint{2.037395in}{5.018638in}}%
\pgfpathlineto{\pgfqpoint{2.103665in}{5.061973in}}%
\pgfpathlineto{\pgfqpoint{2.188176in}{5.114492in}}%
\pgfpathlineto{\pgfqpoint{2.277741in}{5.167011in}}%
\pgfpathlineto{\pgfqpoint{2.372889in}{5.219530in}}%
\pgfpathlineto{\pgfqpoint{2.451422in}{5.260509in}}%
\pgfpathlineto{\pgfqpoint{2.540142in}{5.304433in}}%
\pgfpathlineto{\pgfqpoint{2.639721in}{5.350827in}}%
\pgfpathlineto{\pgfqpoint{2.717582in}{5.385048in}}%
\pgfpathlineto{\pgfqpoint{2.825233in}{5.429606in}}%
\pgfpathlineto{\pgfqpoint{2.895023in}{5.456830in}}%
\pgfpathlineto{\pgfqpoint{3.013316in}{5.499942in}}%
\pgfpathlineto{\pgfqpoint{3.102036in}{5.529981in}}%
\pgfpathlineto{\pgfqpoint{3.200278in}{5.560903in}}%
\pgfpathlineto{\pgfqpoint{3.309050in}{5.592287in}}%
\pgfpathlineto{\pgfqpoint{3.397770in}{5.615708in}}%
\pgfpathlineto{\pgfqpoint{3.497928in}{5.639682in}}%
\pgfpathlineto{\pgfqpoint{3.575210in}{5.656355in}}%
\pgfpathlineto{\pgfqpoint{3.663930in}{5.673553in}}%
\pgfpathlineto{\pgfqpoint{3.752650in}{5.688552in}}%
\pgfpathlineto{\pgfqpoint{3.841370in}{5.701136in}}%
\pgfpathlineto{\pgfqpoint{3.930090in}{5.711261in}}%
\pgfpathlineto{\pgfqpoint{4.018811in}{5.718734in}}%
\pgfpathlineto{\pgfqpoint{4.107531in}{5.723096in}}%
\pgfpathlineto{\pgfqpoint{4.166677in}{5.724199in}}%
\pgfpathlineto{\pgfqpoint{4.225824in}{5.723702in}}%
\pgfpathlineto{\pgfqpoint{4.284971in}{5.721453in}}%
\pgfpathlineto{\pgfqpoint{4.344118in}{5.717235in}}%
\pgfpathlineto{\pgfqpoint{4.403264in}{5.710706in}}%
\pgfpathlineto{\pgfqpoint{4.462411in}{5.701717in}}%
\pgfpathlineto{\pgfqpoint{4.521558in}{5.689896in}}%
\pgfpathlineto{\pgfqpoint{4.580705in}{5.674500in}}%
\pgfpathlineto{\pgfqpoint{4.610278in}{5.665439in}}%
\pgfpathlineto{\pgfqpoint{4.669425in}{5.643504in}}%
\pgfpathlineto{\pgfqpoint{4.698998in}{5.630435in}}%
\pgfpathlineto{\pgfqpoint{4.733190in}{5.613422in}}%
\pgfpathlineto{\pgfqpoint{4.777773in}{5.587163in}}%
\pgfpathlineto{\pgfqpoint{4.787718in}{5.580573in}}%
\pgfpathlineto{\pgfqpoint{4.817291in}{5.559343in}}%
\pgfpathlineto{\pgfqpoint{4.847183in}{5.534644in}}%
\pgfpathlineto{\pgfqpoint{4.876438in}{5.506547in}}%
\pgfpathlineto{\pgfqpoint{4.906012in}{5.473104in}}%
\pgfpathlineto{\pgfqpoint{4.919488in}{5.455865in}}%
\pgfpathlineto{\pgfqpoint{4.937987in}{5.429606in}}%
\pgfpathlineto{\pgfqpoint{4.965158in}{5.383604in}}%
\pgfpathlineto{\pgfqpoint{4.981282in}{5.350827in}}%
\pgfpathlineto{\pgfqpoint{4.994732in}{5.319152in}}%
\pgfpathlineto{\pgfqpoint{5.011351in}{5.272049in}}%
\pgfpathlineto{\pgfqpoint{5.026062in}{5.219530in}}%
\pgfpathlineto{\pgfqpoint{5.037303in}{5.167011in}}%
\pgfpathlineto{\pgfqpoint{5.045871in}{5.114492in}}%
\pgfpathlineto{\pgfqpoint{5.052208in}{5.061973in}}%
\pgfpathlineto{\pgfqpoint{5.056640in}{5.009454in}}%
\pgfpathlineto{\pgfqpoint{5.060495in}{4.930676in}}%
\pgfpathlineto{\pgfqpoint{5.061808in}{4.851897in}}%
\pgfpathlineto{\pgfqpoint{5.060792in}{4.746860in}}%
\pgfpathlineto{\pgfqpoint{5.056856in}{4.615562in}}%
\pgfpathlineto{\pgfqpoint{5.043871in}{4.247930in}}%
\pgfpathlineto{\pgfqpoint{5.042193in}{4.116632in}}%
\pgfpathlineto{\pgfqpoint{5.042923in}{4.011594in}}%
\pgfpathlineto{\pgfqpoint{5.045854in}{3.906556in}}%
\pgfpathlineto{\pgfqpoint{5.051277in}{3.801519in}}%
\pgfpathlineto{\pgfqpoint{5.059383in}{3.696481in}}%
\pgfpathlineto{\pgfqpoint{5.067367in}{3.617702in}}%
\pgfpathlineto{\pgfqpoint{5.077108in}{3.538924in}}%
\pgfpathlineto{\pgfqpoint{5.088635in}{3.460145in}}%
\pgfpathlineto{\pgfqpoint{5.102002in}{3.381367in}}%
\pgfpathlineto{\pgfqpoint{5.117319in}{3.302589in}}%
\pgfpathlineto{\pgfqpoint{5.134573in}{3.223810in}}%
\pgfpathlineto{\pgfqpoint{5.153847in}{3.145032in}}%
\pgfpathlineto{\pgfqpoint{5.175207in}{3.066253in}}%
\pgfpathlineto{\pgfqpoint{5.201745in}{2.977657in}}%
\pgfpathlineto{\pgfqpoint{5.224170in}{2.908696in}}%
\pgfpathlineto{\pgfqpoint{5.251871in}{2.829918in}}%
\pgfpathlineto{\pgfqpoint{5.281756in}{2.751140in}}%
\pgfpathlineto{\pgfqpoint{5.320039in}{2.657889in}}%
\pgfpathlineto{\pgfqpoint{5.349612in}{2.590428in}}%
\pgfpathlineto{\pgfqpoint{5.384698in}{2.514804in}}%
\pgfpathlineto{\pgfqpoint{5.423475in}{2.436026in}}%
\pgfpathlineto{\pgfqpoint{5.467906in}{2.351065in}}%
\pgfpathlineto{\pgfqpoint{5.507858in}{2.278469in}}%
\pgfpathlineto{\pgfqpoint{5.556626in}{2.194449in}}%
\pgfpathlineto{\pgfqpoint{5.601340in}{2.120912in}}%
\pgfpathlineto{\pgfqpoint{5.626516in}{2.081004in}}%
\pgfpathlineto{\pgfqpoint{5.626516in}{2.081004in}}%
\pgfusepath{stroke}%
\end{pgfscope}%
\begin{pgfscope}%
\pgfpathrectangle{\pgfqpoint{0.854460in}{0.571603in}}{\pgfqpoint{5.885100in}{5.225635in}}%
\pgfusepath{clip}%
\pgfsetbuttcap%
\pgfsetroundjoin%
\pgfsetlinewidth{1.505625pt}%
\definecolor{currentstroke}{rgb}{0.187231,0.414746,0.556547}%
\pgfsetstrokecolor{currentstroke}%
\pgfsetdash{}{0pt}%
\pgfpathmoveto{\pgfqpoint{5.847701in}{1.762996in}}%
\pgfpathlineto{\pgfqpoint{5.852359in}{1.756816in}}%
\pgfpathlineto{\pgfqpoint{5.855016in}{1.753280in}}%
\pgfpathlineto{\pgfqpoint{5.875019in}{1.727020in}}%
\pgfpathlineto{\pgfqpoint{5.881933in}{1.718078in}}%
\pgfpathlineto{\pgfqpoint{5.895278in}{1.700761in}}%
\pgfpathlineto{\pgfqpoint{5.911506in}{1.680027in}}%
\pgfpathlineto{\pgfqpoint{5.915818in}{1.674501in}}%
\pgfpathlineto{\pgfqpoint{5.936590in}{1.648242in}}%
\pgfpathlineto{\pgfqpoint{5.941079in}{1.642642in}}%
\pgfpathlineto{\pgfqpoint{5.957597in}{1.621982in}}%
\pgfpathlineto{\pgfqpoint{5.970653in}{1.605890in}}%
\pgfpathlineto{\pgfqpoint{5.978879in}{1.595723in}}%
\pgfpathlineto{\pgfqpoint{6.000226in}{1.569717in}}%
\pgfpathlineto{\pgfqpoint{6.000434in}{1.569463in}}%
\pgfpathlineto{\pgfqpoint{6.022183in}{1.543204in}}%
\pgfpathlineto{\pgfqpoint{6.029799in}{1.534133in}}%
\pgfpathlineto{\pgfqpoint{6.044200in}{1.516944in}}%
\pgfpathlineto{\pgfqpoint{6.059373in}{1.499079in}}%
\pgfpathlineto{\pgfqpoint{6.066486in}{1.490685in}}%
\pgfpathlineto{\pgfqpoint{6.088946in}{1.464532in}}%
\pgfpathlineto{\pgfqpoint{6.089037in}{1.464425in}}%
\pgfpathlineto{\pgfqpoint{6.111785in}{1.438166in}}%
\pgfpathlineto{\pgfqpoint{6.118520in}{1.430490in}}%
\pgfpathlineto{\pgfqpoint{6.134796in}{1.411906in}}%
\pgfpathlineto{\pgfqpoint{6.148093in}{1.396914in}}%
\pgfpathlineto{\pgfqpoint{6.158069in}{1.385647in}}%
\pgfpathlineto{\pgfqpoint{6.177666in}{1.363784in}}%
\pgfpathlineto{\pgfqpoint{6.181601in}{1.359388in}}%
\pgfpathlineto{\pgfqpoint{6.205372in}{1.333128in}}%
\pgfpathlineto{\pgfqpoint{6.207240in}{1.331086in}}%
\pgfpathlineto{\pgfqpoint{6.229362in}{1.306869in}}%
\pgfpathlineto{\pgfqpoint{6.236813in}{1.298805in}}%
\pgfpathlineto{\pgfqpoint{6.253608in}{1.280609in}}%
\pgfpathlineto{\pgfqpoint{6.266386in}{1.266920in}}%
\pgfpathlineto{\pgfqpoint{6.278108in}{1.254350in}}%
\pgfpathlineto{\pgfqpoint{6.295960in}{1.235416in}}%
\pgfpathlineto{\pgfqpoint{6.302861in}{1.228090in}}%
\pgfpathlineto{\pgfqpoint{6.325533in}{1.204280in}}%
\pgfpathlineto{\pgfqpoint{6.327863in}{1.201831in}}%
\pgfpathlineto{\pgfqpoint{6.353095in}{1.175571in}}%
\pgfpathlineto{\pgfqpoint{6.355107in}{1.173497in}}%
\pgfpathlineto{\pgfqpoint{6.378552in}{1.149312in}}%
\pgfpathlineto{\pgfqpoint{6.384680in}{1.143054in}}%
\pgfpathlineto{\pgfqpoint{6.404257in}{1.123052in}}%
\pgfpathlineto{\pgfqpoint{6.414253in}{1.112939in}}%
\pgfpathlineto{\pgfqpoint{6.430208in}{1.096793in}}%
\pgfpathlineto{\pgfqpoint{6.443827in}{1.083142in}}%
\pgfpathlineto{\pgfqpoint{6.456404in}{1.070533in}}%
\pgfpathlineto{\pgfqpoint{6.473400in}{1.053652in}}%
\pgfpathlineto{\pgfqpoint{6.482842in}{1.044274in}}%
\pgfpathlineto{\pgfqpoint{6.502973in}{1.024458in}}%
\pgfpathlineto{\pgfqpoint{6.509521in}{1.018014in}}%
\pgfpathlineto{\pgfqpoint{6.532547in}{0.995552in}}%
\pgfpathlineto{\pgfqpoint{6.536440in}{0.991755in}}%
\pgfpathlineto{\pgfqpoint{6.562120in}{0.966923in}}%
\pgfpathlineto{\pgfqpoint{6.563597in}{0.965495in}}%
\pgfpathlineto{\pgfqpoint{6.590984in}{0.939236in}}%
\pgfpathlineto{\pgfqpoint{6.591693in}{0.938560in}}%
\pgfpathlineto{\pgfqpoint{6.618593in}{0.912976in}}%
\pgfpathlineto{\pgfqpoint{6.621267in}{0.910453in}}%
\pgfpathlineto{\pgfqpoint{6.646439in}{0.886717in}}%
\pgfpathlineto{\pgfqpoint{6.650840in}{0.882597in}}%
\pgfpathlineto{\pgfqpoint{6.674519in}{0.860458in}}%
\pgfpathlineto{\pgfqpoint{6.680414in}{0.854986in}}%
\pgfpathlineto{\pgfqpoint{6.702833in}{0.834198in}}%
\pgfpathlineto{\pgfqpoint{6.709987in}{0.827611in}}%
\pgfpathlineto{\pgfqpoint{6.731379in}{0.807939in}}%
\pgfpathlineto{\pgfqpoint{6.739560in}{0.800465in}}%
\pgfusepath{stroke}%
\end{pgfscope}%
\begin{pgfscope}%
\pgfpathrectangle{\pgfqpoint{0.854460in}{0.571603in}}{\pgfqpoint{5.885100in}{5.225635in}}%
\pgfusepath{clip}%
\pgfsetbuttcap%
\pgfsetroundjoin%
\pgfsetlinewidth{1.505625pt}%
\definecolor{currentstroke}{rgb}{0.179019,0.433756,0.557430}%
\pgfsetstrokecolor{currentstroke}%
\pgfsetdash{}{0pt}%
\pgfpathmoveto{\pgfqpoint{1.688535in}{0.571603in}}%
\pgfpathlineto{\pgfqpoint{1.682515in}{0.576944in}}%
\pgfpathlineto{\pgfqpoint{1.659034in}{0.597863in}}%
\pgfpathlineto{\pgfqpoint{1.652941in}{0.603371in}}%
\pgfpathlineto{\pgfqpoint{1.630094in}{0.624122in}}%
\pgfpathlineto{\pgfqpoint{1.623368in}{0.630321in}}%
\pgfpathlineto{\pgfqpoint{1.601706in}{0.650382in}}%
\pgfpathlineto{\pgfqpoint{1.593795in}{0.657817in}}%
\pgfpathlineto{\pgfqpoint{1.573864in}{0.676641in}}%
\pgfpathlineto{\pgfqpoint{1.564221in}{0.685884in}}%
\pgfpathlineto{\pgfqpoint{1.546558in}{0.702901in}}%
\pgfpathlineto{\pgfqpoint{1.534648in}{0.714547in}}%
\pgfpathlineto{\pgfqpoint{1.519781in}{0.729160in}}%
\pgfpathlineto{\pgfqpoint{1.505074in}{0.743832in}}%
\pgfpathlineto{\pgfqpoint{1.493523in}{0.755420in}}%
\pgfpathlineto{\pgfqpoint{1.475501in}{0.773768in}}%
\pgfpathlineto{\pgfqpoint{1.467775in}{0.781679in}}%
\pgfpathlineto{\pgfqpoint{1.445928in}{0.804384in}}%
\pgfpathlineto{\pgfqpoint{1.442527in}{0.807939in}}%
\pgfpathlineto{\pgfqpoint{1.417792in}{0.834198in}}%
\pgfpathlineto{\pgfqpoint{1.416354in}{0.835749in}}%
\pgfpathlineto{\pgfqpoint{1.393595in}{0.860458in}}%
\pgfpathlineto{\pgfqpoint{1.386781in}{0.867967in}}%
\pgfpathlineto{\pgfqpoint{1.369876in}{0.886717in}}%
\pgfpathlineto{\pgfqpoint{1.357208in}{0.900982in}}%
\pgfpathlineto{\pgfqpoint{1.346624in}{0.912976in}}%
\pgfpathlineto{\pgfqpoint{1.327634in}{0.934827in}}%
\pgfpathlineto{\pgfqpoint{1.323828in}{0.939236in}}%
\pgfpathlineto{\pgfqpoint{1.301529in}{0.965495in}}%
\pgfpathlineto{\pgfqpoint{1.298061in}{0.969649in}}%
\pgfpathlineto{\pgfqpoint{1.279729in}{0.991755in}}%
\pgfpathlineto{\pgfqpoint{1.268488in}{1.005521in}}%
\pgfpathlineto{\pgfqpoint{1.258358in}{1.018014in}}%
\pgfpathlineto{\pgfqpoint{1.238914in}{1.042370in}}%
\pgfpathlineto{\pgfqpoint{1.237405in}{1.044274in}}%
\pgfpathlineto{\pgfqpoint{1.216973in}{1.070533in}}%
\pgfpathlineto{\pgfqpoint{1.209341in}{1.080503in}}%
\pgfpathlineto{\pgfqpoint{1.196965in}{1.096793in}}%
\pgfpathlineto{\pgfqpoint{1.179767in}{1.119788in}}%
\pgfpathlineto{\pgfqpoint{1.177345in}{1.123052in}}%
\pgfpathlineto{\pgfqpoint{1.158224in}{1.149312in}}%
\pgfpathlineto{\pgfqpoint{1.150194in}{1.160527in}}%
\pgfpathlineto{\pgfqpoint{1.139510in}{1.175571in}}%
\pgfpathlineto{\pgfqpoint{1.121162in}{1.201831in}}%
\pgfpathlineto{\pgfqpoint{1.120621in}{1.202621in}}%
\pgfpathlineto{\pgfqpoint{1.103330in}{1.228090in}}%
\pgfpathlineto{\pgfqpoint{1.091047in}{1.246481in}}%
\pgfpathlineto{\pgfqpoint{1.085837in}{1.254350in}}%
\pgfpathlineto{\pgfqpoint{1.068782in}{1.280609in}}%
\pgfpathlineto{\pgfqpoint{1.061474in}{1.292073in}}%
\pgfpathlineto{\pgfqpoint{1.052126in}{1.306869in}}%
\pgfpathlineto{\pgfqpoint{1.042764in}{1.321960in}}%
\pgfusepath{stroke}%
\end{pgfscope}%
\begin{pgfscope}%
\pgfpathrectangle{\pgfqpoint{0.854460in}{0.571603in}}{\pgfqpoint{5.885100in}{5.225635in}}%
\pgfusepath{clip}%
\pgfsetbuttcap%
\pgfsetroundjoin%
\pgfsetlinewidth{1.505625pt}%
\definecolor{currentstroke}{rgb}{0.179019,0.433756,0.557430}%
\pgfsetstrokecolor{currentstroke}%
\pgfsetdash{}{0pt}%
\pgfpathmoveto{\pgfqpoint{0.862013in}{1.666348in}}%
\pgfpathlineto{\pgfqpoint{0.858429in}{1.674501in}}%
\pgfpathlineto{\pgfqpoint{0.854460in}{1.683776in}}%
\pgfusepath{stroke}%
\end{pgfscope}%
\begin{pgfscope}%
\pgfpathrectangle{\pgfqpoint{0.854460in}{0.571603in}}{\pgfqpoint{5.885100in}{5.225635in}}%
\pgfusepath{clip}%
\pgfsetbuttcap%
\pgfsetroundjoin%
\pgfsetlinewidth{1.505625pt}%
\definecolor{currentstroke}{rgb}{0.179019,0.433756,0.557430}%
\pgfsetstrokecolor{currentstroke}%
\pgfsetdash{}{0pt}%
\pgfpathmoveto{\pgfqpoint{0.854460in}{3.742772in}}%
\pgfpathlineto{\pgfqpoint{0.857082in}{3.749000in}}%
\pgfpathlineto{\pgfqpoint{0.868405in}{3.775259in}}%
\pgfpathlineto{\pgfqpoint{0.879979in}{3.801519in}}%
\pgfpathlineto{\pgfqpoint{0.884034in}{3.810508in}}%
\pgfpathlineto{\pgfqpoint{0.891949in}{3.827778in}}%
\pgfpathlineto{\pgfqpoint{0.904246in}{3.854037in}}%
\pgfpathlineto{\pgfqpoint{0.913607in}{3.873619in}}%
\pgfpathlineto{\pgfqpoint{0.916851in}{3.880297in}}%
\pgfpathlineto{\pgfqpoint{0.929884in}{3.906556in}}%
\pgfpathlineto{\pgfqpoint{0.943163in}{3.932816in}}%
\pgfpathlineto{\pgfqpoint{0.943181in}{3.932850in}}%
\pgfpathlineto{\pgfqpoint{0.956944in}{3.959075in}}%
\pgfpathlineto{\pgfqpoint{0.970970in}{3.985335in}}%
\pgfpathlineto{\pgfqpoint{0.972754in}{3.988607in}}%
\pgfpathlineto{\pgfqpoint{0.985478in}{4.011594in}}%
\pgfpathlineto{\pgfqpoint{1.000263in}{4.037854in}}%
\pgfpathlineto{\pgfqpoint{1.002327in}{4.041450in}}%
\pgfpathlineto{\pgfqpoint{1.015536in}{4.064113in}}%
\pgfpathlineto{\pgfqpoint{1.031090in}{4.090373in}}%
\pgfpathlineto{\pgfqpoint{1.031901in}{4.091716in}}%
\pgfpathlineto{\pgfqpoint{1.047169in}{4.116632in}}%
\pgfpathlineto{\pgfqpoint{1.061474in}{4.139622in}}%
\pgfpathlineto{\pgfqpoint{1.063539in}{4.142892in}}%
\pgfpathlineto{\pgfqpoint{1.080423in}{4.169151in}}%
\pgfpathlineto{\pgfqpoint{1.091047in}{4.185422in}}%
\pgfpathlineto{\pgfqpoint{1.097666in}{4.195411in}}%
\pgfpathlineto{\pgfqpoint{1.115347in}{4.221670in}}%
\pgfpathlineto{\pgfqpoint{1.120621in}{4.229377in}}%
\pgfpathlineto{\pgfqpoint{1.133502in}{4.247930in}}%
\pgfpathlineto{\pgfqpoint{1.150194in}{4.271632in}}%
\pgfpathlineto{\pgfqpoint{1.152021in}{4.274189in}}%
\pgfpathlineto{\pgfqpoint{1.171091in}{4.300449in}}%
\pgfpathlineto{\pgfqpoint{1.179767in}{4.312225in}}%
\pgfpathlineto{\pgfqpoint{1.190590in}{4.326708in}}%
\pgfpathlineto{\pgfqpoint{1.209341in}{4.351461in}}%
\pgfpathlineto{\pgfqpoint{1.210498in}{4.352967in}}%
\pgfpathlineto{\pgfqpoint{1.230991in}{4.379227in}}%
\pgfpathlineto{\pgfqpoint{1.238914in}{4.389241in}}%
\pgfpathlineto{\pgfqpoint{1.251946in}{4.405486in}}%
\pgfpathlineto{\pgfqpoint{1.268488in}{4.425835in}}%
\pgfpathlineto{\pgfqpoint{1.273359in}{4.431746in}}%
\pgfpathlineto{\pgfqpoint{1.295302in}{4.458005in}}%
\pgfpathlineto{\pgfqpoint{1.298061in}{4.461259in}}%
\pgfpathlineto{\pgfqpoint{1.317827in}{4.484265in}}%
\pgfpathlineto{\pgfqpoint{1.327634in}{4.495534in}}%
\pgfpathlineto{\pgfqpoint{1.340856in}{4.510524in}}%
\pgfpathlineto{\pgfqpoint{1.357208in}{4.528828in}}%
\pgfpathlineto{\pgfqpoint{1.364410in}{4.536784in}}%
\pgfpathlineto{\pgfqpoint{1.386781in}{4.561185in}}%
\pgfpathlineto{\pgfqpoint{1.388507in}{4.563043in}}%
\pgfpathlineto{\pgfqpoint{1.413230in}{4.589303in}}%
\pgfpathlineto{\pgfqpoint{1.416354in}{4.592579in}}%
\pgfpathlineto{\pgfqpoint{1.438558in}{4.615562in}}%
\pgfpathlineto{\pgfqpoint{1.445928in}{4.623096in}}%
\pgfpathlineto{\pgfqpoint{1.464478in}{4.641822in}}%
\pgfpathlineto{\pgfqpoint{1.475501in}{4.652812in}}%
\pgfpathlineto{\pgfqpoint{1.491009in}{4.668081in}}%
\pgfpathlineto{\pgfqpoint{1.505074in}{4.681761in}}%
\pgfpathlineto{\pgfqpoint{1.518170in}{4.694341in}}%
\pgfpathlineto{\pgfqpoint{1.534648in}{4.709977in}}%
\pgfpathlineto{\pgfqpoint{1.545981in}{4.720600in}}%
\pgfpathlineto{\pgfqpoint{1.564221in}{4.737491in}}%
\pgfpathlineto{\pgfqpoint{1.574462in}{4.746860in}}%
\pgfpathlineto{\pgfqpoint{1.593795in}{4.764333in}}%
\pgfpathlineto{\pgfqpoint{1.603633in}{4.773119in}}%
\pgfpathlineto{\pgfqpoint{1.623368in}{4.790532in}}%
\pgfpathlineto{\pgfqpoint{1.633514in}{4.799378in}}%
\pgfpathlineto{\pgfqpoint{1.652941in}{4.816116in}}%
\pgfpathlineto{\pgfqpoint{1.664124in}{4.825638in}}%
\pgfpathlineto{\pgfqpoint{1.682515in}{4.841111in}}%
\pgfpathlineto{\pgfqpoint{1.695484in}{4.851897in}}%
\pgfpathlineto{\pgfqpoint{1.712088in}{4.865542in}}%
\pgfpathlineto{\pgfqpoint{1.727613in}{4.878157in}}%
\pgfpathlineto{\pgfqpoint{1.741661in}{4.889436in}}%
\pgfpathlineto{\pgfqpoint{1.760531in}{4.904416in}}%
\pgfpathlineto{\pgfqpoint{1.771235in}{4.912814in}}%
\pgfpathlineto{\pgfqpoint{1.794255in}{4.930676in}}%
\pgfpathlineto{\pgfqpoint{1.800808in}{4.935700in}}%
\pgfpathlineto{\pgfqpoint{1.828807in}{4.956935in}}%
\pgfpathlineto{\pgfqpoint{1.830381in}{4.958115in}}%
\pgfpathlineto{\pgfqpoint{1.859955in}{4.980016in}}%
\pgfpathlineto{\pgfqpoint{1.864295in}{4.983195in}}%
\pgfpathlineto{\pgfqpoint{1.889528in}{5.001454in}}%
\pgfpathlineto{\pgfqpoint{1.900700in}{5.009454in}}%
\pgfpathlineto{\pgfqpoint{1.919102in}{5.022475in}}%
\pgfpathlineto{\pgfqpoint{1.938006in}{5.035714in}}%
\pgfpathlineto{\pgfqpoint{1.948675in}{5.043097in}}%
\pgfpathlineto{\pgfqpoint{1.976231in}{5.061973in}}%
\pgfpathlineto{\pgfqpoint{1.978248in}{5.063339in}}%
\pgfpathlineto{\pgfqpoint{2.007822in}{5.083114in}}%
\pgfpathlineto{\pgfqpoint{2.015559in}{5.088233in}}%
\pgfpathlineto{\pgfqpoint{2.037395in}{5.102508in}}%
\pgfpathlineto{\pgfqpoint{2.055907in}{5.114492in}}%
\pgfpathlineto{\pgfqpoint{2.066968in}{5.121568in}}%
\pgfpathlineto{\pgfqpoint{2.096542in}{5.140302in}}%
\pgfpathlineto{\pgfqpoint{2.097261in}{5.140752in}}%
\pgfpathlineto{\pgfqpoint{2.126115in}{5.158577in}}%
\pgfpathlineto{\pgfqpoint{2.139896in}{5.167011in}}%
\pgfpathlineto{\pgfqpoint{2.155689in}{5.176561in}}%
\pgfpathlineto{\pgfqpoint{2.183575in}{5.193271in}}%
\pgfpathlineto{\pgfqpoint{2.185262in}{5.194269in}}%
\pgfpathlineto{\pgfqpoint{2.214835in}{5.211548in}}%
\pgfpathlineto{\pgfqpoint{2.228628in}{5.219530in}}%
\pgfpathlineto{\pgfqpoint{2.244409in}{5.228553in}}%
\pgfpathlineto{\pgfqpoint{2.273982in}{5.245312in}}%
\pgfpathlineto{\pgfqpoint{2.274837in}{5.245790in}}%
\pgfpathlineto{\pgfqpoint{2.303555in}{5.261646in}}%
\pgfpathlineto{\pgfqpoint{2.322559in}{5.272049in}}%
\pgfpathlineto{\pgfqpoint{2.333129in}{5.277766in}}%
\pgfpathlineto{\pgfqpoint{2.362702in}{5.293584in}}%
\pgfpathlineto{\pgfqpoint{2.371641in}{5.298308in}}%
\pgfpathlineto{\pgfqpoint{2.392275in}{5.309084in}}%
\pgfpathlineto{\pgfqpoint{2.421849in}{5.324402in}}%
\pgfpathlineto{\pgfqpoint{2.422174in}{5.324568in}}%
\pgfpathlineto{\pgfqpoint{2.451422in}{5.339312in}}%
\pgfpathlineto{\pgfqpoint{2.474446in}{5.350827in}}%
\pgfpathlineto{\pgfqpoint{2.480996in}{5.354063in}}%
\pgfpathlineto{\pgfqpoint{2.510569in}{5.368482in}}%
\pgfpathlineto{\pgfqpoint{2.528393in}{5.377087in}}%
\pgfpathlineto{\pgfqpoint{2.540142in}{5.382690in}}%
\pgfpathlineto{\pgfqpoint{2.569716in}{5.396625in}}%
\pgfpathlineto{\pgfqpoint{2.584142in}{5.403346in}}%
\pgfpathlineto{\pgfqpoint{2.599289in}{5.410317in}}%
\pgfpathlineto{\pgfqpoint{2.602810in}{5.411919in}}%
\pgfusepath{stroke}%
\end{pgfscope}%
\begin{pgfscope}%
\pgfpathrectangle{\pgfqpoint{0.854460in}{0.571603in}}{\pgfqpoint{5.885100in}{5.225635in}}%
\pgfusepath{clip}%
\pgfsetbuttcap%
\pgfsetroundjoin%
\pgfsetlinewidth{1.505625pt}%
\definecolor{currentstroke}{rgb}{0.179019,0.433756,0.557430}%
\pgfsetstrokecolor{currentstroke}%
\pgfsetdash{}{0pt}%
\pgfpathmoveto{\pgfqpoint{2.953235in}{5.556238in}}%
\pgfpathlineto{\pgfqpoint{2.954169in}{5.556586in}}%
\pgfpathlineto{\pgfqpoint{2.965953in}{5.560903in}}%
\pgfpathlineto{\pgfqpoint{2.983743in}{5.567338in}}%
\pgfpathlineto{\pgfqpoint{3.013316in}{5.577882in}}%
\pgfpathlineto{\pgfqpoint{3.039648in}{5.587163in}}%
\pgfpathlineto{\pgfqpoint{3.042890in}{5.588291in}}%
\pgfpathlineto{\pgfqpoint{3.072463in}{5.598372in}}%
\pgfpathlineto{\pgfqpoint{3.102036in}{5.608340in}}%
\pgfpathlineto{\pgfqpoint{3.117372in}{5.613422in}}%
\pgfpathlineto{\pgfqpoint{3.131610in}{5.618079in}}%
\pgfpathlineto{\pgfqpoint{3.161183in}{5.627581in}}%
\pgfpathlineto{\pgfqpoint{3.190756in}{5.636962in}}%
\pgfpathlineto{\pgfqpoint{3.199512in}{5.639682in}}%
\pgfpathlineto{\pgfqpoint{3.220330in}{5.646063in}}%
\pgfpathlineto{\pgfqpoint{3.249903in}{5.654974in}}%
\pgfpathlineto{\pgfqpoint{3.279476in}{5.663757in}}%
\pgfpathlineto{\pgfqpoint{3.287007in}{5.665941in}}%
\pgfpathlineto{\pgfqpoint{3.309050in}{5.672250in}}%
\pgfpathlineto{\pgfqpoint{3.338623in}{5.680558in}}%
\pgfpathlineto{\pgfqpoint{3.368197in}{5.688728in}}%
\pgfpathlineto{\pgfqpoint{3.381072in}{5.692201in}}%
\pgfpathlineto{\pgfqpoint{3.397770in}{5.696644in}}%
\pgfpathlineto{\pgfqpoint{3.427343in}{5.704333in}}%
\pgfpathlineto{\pgfqpoint{3.456917in}{5.711874in}}%
\pgfpathlineto{\pgfqpoint{3.483302in}{5.718460in}}%
\pgfpathlineto{\pgfqpoint{3.486490in}{5.719245in}}%
\pgfpathlineto{\pgfqpoint{3.516063in}{5.726298in}}%
\pgfpathlineto{\pgfqpoint{3.545637in}{5.733193in}}%
\pgfpathlineto{\pgfqpoint{3.575210in}{5.739926in}}%
\pgfpathlineto{\pgfqpoint{3.596873in}{5.744720in}}%
\pgfpathlineto{\pgfqpoint{3.604783in}{5.746446in}}%
\pgfpathlineto{\pgfqpoint{3.634357in}{5.752674in}}%
\pgfpathlineto{\pgfqpoint{3.663930in}{5.758727in}}%
\pgfpathlineto{\pgfqpoint{3.693504in}{5.764600in}}%
\pgfpathlineto{\pgfqpoint{3.723077in}{5.770287in}}%
\pgfpathlineto{\pgfqpoint{3.726842in}{5.770979in}}%
\pgfpathlineto{\pgfqpoint{3.752650in}{5.775655in}}%
\pgfpathlineto{\pgfqpoint{3.782224in}{5.780809in}}%
\pgfpathlineto{\pgfqpoint{3.811797in}{5.785762in}}%
\pgfpathlineto{\pgfqpoint{3.841370in}{5.790506in}}%
\pgfpathlineto{\pgfqpoint{3.870944in}{5.795035in}}%
\pgfpathlineto{\pgfqpoint{3.886176in}{5.797238in}}%
\pgfusepath{stroke}%
\end{pgfscope}%
\begin{pgfscope}%
\pgfpathrectangle{\pgfqpoint{0.854460in}{0.571603in}}{\pgfqpoint{5.885100in}{5.225635in}}%
\pgfusepath{clip}%
\pgfsetbuttcap%
\pgfsetroundjoin%
\pgfsetlinewidth{1.505625pt}%
\definecolor{currentstroke}{rgb}{0.179019,0.433756,0.557430}%
\pgfsetstrokecolor{currentstroke}%
\pgfsetdash{}{0pt}%
\pgfpathmoveto{\pgfqpoint{4.649720in}{5.797238in}}%
\pgfpathlineto{\pgfqpoint{4.669425in}{5.792848in}}%
\pgfpathlineto{\pgfqpoint{4.698998in}{5.785523in}}%
\pgfpathlineto{\pgfqpoint{4.728571in}{5.777368in}}%
\pgfpathlineto{\pgfqpoint{4.749608in}{5.770979in}}%
\pgfpathlineto{\pgfqpoint{4.758145in}{5.768222in}}%
\pgfpathlineto{\pgfqpoint{4.787718in}{5.757827in}}%
\pgfpathlineto{\pgfqpoint{4.817291in}{5.746368in}}%
\pgfpathlineto{\pgfqpoint{4.821242in}{5.744720in}}%
\pgfpathlineto{\pgfqpoint{4.846865in}{5.733291in}}%
\pgfpathlineto{\pgfqpoint{4.876438in}{5.718842in}}%
\pgfpathlineto{\pgfqpoint{4.877169in}{5.718460in}}%
\pgfpathlineto{\pgfqpoint{4.906012in}{5.702274in}}%
\pgfpathlineto{\pgfqpoint{4.922579in}{5.692201in}}%
\pgfpathlineto{\pgfqpoint{4.935585in}{5.683676in}}%
\pgfpathlineto{\pgfqpoint{4.960719in}{5.665941in}}%
\pgfpathlineto{\pgfqpoint{4.965158in}{5.662551in}}%
\pgfpathlineto{\pgfqpoint{4.993131in}{5.639682in}}%
\pgfpathlineto{\pgfqpoint{4.994732in}{5.638259in}}%
\pgfpathlineto{\pgfqpoint{5.020957in}{5.613422in}}%
\pgfpathlineto{\pgfqpoint{5.024305in}{5.609958in}}%
\pgfpathlineto{\pgfqpoint{5.045064in}{5.587163in}}%
\pgfpathlineto{\pgfqpoint{5.053878in}{5.576525in}}%
\pgfpathlineto{\pgfqpoint{5.066120in}{5.560903in}}%
\pgfpathlineto{\pgfqpoint{5.083452in}{5.536426in}}%
\pgfpathlineto{\pgfqpoint{5.084649in}{5.534644in}}%
\pgfpathlineto{\pgfqpoint{5.100776in}{5.508384in}}%
\pgfpathlineto{\pgfqpoint{5.113025in}{5.486066in}}%
\pgfpathlineto{\pgfqpoint{5.115087in}{5.482125in}}%
\pgfpathlineto{\pgfqpoint{5.127560in}{5.455865in}}%
\pgfpathlineto{\pgfqpoint{5.138628in}{5.429606in}}%
\pgfpathlineto{\pgfqpoint{5.142599in}{5.419096in}}%
\pgfpathlineto{\pgfqpoint{5.148298in}{5.403346in}}%
\pgfpathlineto{\pgfqpoint{5.156753in}{5.377087in}}%
\pgfpathlineto{\pgfqpoint{5.164179in}{5.350827in}}%
\pgfpathlineto{\pgfqpoint{5.170667in}{5.324568in}}%
\pgfpathlineto{\pgfqpoint{5.172172in}{5.317643in}}%
\pgfpathlineto{\pgfqpoint{5.176218in}{5.298308in}}%
\pgfpathlineto{\pgfqpoint{5.180976in}{5.272049in}}%
\pgfpathlineto{\pgfqpoint{5.185036in}{5.245790in}}%
\pgfpathlineto{\pgfqpoint{5.188457in}{5.219530in}}%
\pgfpathlineto{\pgfqpoint{5.191294in}{5.193271in}}%
\pgfpathlineto{\pgfqpoint{5.193598in}{5.167011in}}%
\pgfpathlineto{\pgfqpoint{5.195416in}{5.140752in}}%
\pgfpathlineto{\pgfqpoint{5.196789in}{5.114492in}}%
\pgfpathlineto{\pgfqpoint{5.197758in}{5.088233in}}%
\pgfpathlineto{\pgfqpoint{5.198358in}{5.061973in}}%
\pgfpathlineto{\pgfqpoint{5.198625in}{5.035714in}}%
\pgfpathlineto{\pgfqpoint{5.198588in}{5.009454in}}%
\pgfpathlineto{\pgfqpoint{5.198279in}{4.983195in}}%
\pgfpathlineto{\pgfqpoint{5.197722in}{4.956935in}}%
\pgfpathlineto{\pgfqpoint{5.196944in}{4.930676in}}%
\pgfpathlineto{\pgfqpoint{5.195969in}{4.904416in}}%
\pgfpathlineto{\pgfqpoint{5.195355in}{4.890407in}}%
\pgfusepath{stroke}%
\end{pgfscope}%
\begin{pgfscope}%
\pgfpathrectangle{\pgfqpoint{0.854460in}{0.571603in}}{\pgfqpoint{5.885100in}{5.225635in}}%
\pgfusepath{clip}%
\pgfsetbuttcap%
\pgfsetroundjoin%
\pgfsetlinewidth{1.505625pt}%
\definecolor{currentstroke}{rgb}{0.179019,0.433756,0.557430}%
\pgfsetstrokecolor{currentstroke}%
\pgfsetdash{}{0pt}%
\pgfpathmoveto{\pgfqpoint{5.168913in}{4.498456in}}%
\pgfpathlineto{\pgfqpoint{5.159124in}{4.352967in}}%
\pgfpathlineto{\pgfqpoint{5.152381in}{4.221670in}}%
\pgfpathlineto{\pgfqpoint{5.148959in}{4.116632in}}%
\pgfpathlineto{\pgfqpoint{5.147649in}{4.011594in}}%
\pgfpathlineto{\pgfqpoint{5.148737in}{3.906556in}}%
\pgfpathlineto{\pgfqpoint{5.152479in}{3.801519in}}%
\pgfpathlineto{\pgfqpoint{5.157166in}{3.722740in}}%
\pgfpathlineto{\pgfqpoint{5.163565in}{3.643962in}}%
\pgfpathlineto{\pgfqpoint{5.172172in}{3.561788in}}%
\pgfpathlineto{\pgfqpoint{5.181738in}{3.486405in}}%
\pgfpathlineto{\pgfqpoint{5.193651in}{3.407626in}}%
\pgfpathlineto{\pgfqpoint{5.207522in}{3.328848in}}%
\pgfpathlineto{\pgfqpoint{5.223386in}{3.250070in}}%
\pgfpathlineto{\pgfqpoint{5.241300in}{3.171291in}}%
\pgfpathlineto{\pgfqpoint{5.261341in}{3.092513in}}%
\pgfpathlineto{\pgfqpoint{5.283446in}{3.013734in}}%
\pgfpathlineto{\pgfqpoint{5.307724in}{2.934956in}}%
\pgfpathlineto{\pgfqpoint{5.334194in}{2.856177in}}%
\pgfpathlineto{\pgfqpoint{5.362875in}{2.777399in}}%
\pgfpathlineto{\pgfqpoint{5.393786in}{2.698621in}}%
\pgfpathlineto{\pgfqpoint{5.426949in}{2.619842in}}%
\pgfpathlineto{\pgfqpoint{5.467906in}{2.529343in}}%
\pgfpathlineto{\pgfqpoint{5.500105in}{2.462285in}}%
\pgfpathlineto{\pgfqpoint{5.540078in}{2.383507in}}%
\pgfpathlineto{\pgfqpoint{5.586199in}{2.297915in}}%
\pgfpathlineto{\pgfqpoint{5.626982in}{2.225950in}}%
\pgfpathlineto{\pgfqpoint{5.674919in}{2.145567in}}%
\pgfpathlineto{\pgfqpoint{5.723132in}{2.068393in}}%
\pgfpathlineto{\pgfqpoint{5.774681in}{1.989615in}}%
\pgfpathlineto{\pgfqpoint{5.828558in}{1.910836in}}%
\pgfpathlineto{\pgfqpoint{5.884748in}{1.832058in}}%
\pgfpathlineto{\pgfqpoint{5.943241in}{1.753280in}}%
\pgfpathlineto{\pgfqpoint{6.004030in}{1.674501in}}%
\pgfpathlineto{\pgfqpoint{6.067113in}{1.595723in}}%
\pgfpathlineto{\pgfqpoint{6.154802in}{1.490685in}}%
\pgfpathlineto{\pgfqpoint{6.246531in}{1.385647in}}%
\pgfpathlineto{\pgfqpoint{6.342282in}{1.280609in}}%
\pgfpathlineto{\pgfqpoint{6.442049in}{1.175571in}}%
\pgfpathlineto{\pgfqpoint{6.545710in}{1.070533in}}%
\pgfpathlineto{\pgfqpoint{6.653326in}{0.965495in}}%
\pgfpathlineto{\pgfqpoint{6.739560in}{0.883914in}}%
\pgfpathlineto{\pgfqpoint{6.739560in}{0.883914in}}%
\pgfusepath{stroke}%
\end{pgfscope}%
\begin{pgfscope}%
\pgfpathrectangle{\pgfqpoint{0.854460in}{0.571603in}}{\pgfqpoint{5.885100in}{5.225635in}}%
\pgfusepath{clip}%
\pgfsetbuttcap%
\pgfsetroundjoin%
\pgfsetlinewidth{1.505625pt}%
\definecolor{currentstroke}{rgb}{0.169646,0.456262,0.558030}%
\pgfsetstrokecolor{currentstroke}%
\pgfsetdash{}{0pt}%
\pgfpathmoveto{\pgfqpoint{1.614939in}{0.571603in}}%
\pgfpathlineto{\pgfqpoint{1.593795in}{0.590603in}}%
\pgfpathlineto{\pgfqpoint{1.585752in}{0.597863in}}%
\pgfpathlineto{\pgfqpoint{1.584965in}{0.598583in}}%
\pgfusepath{stroke}%
\end{pgfscope}%
\begin{pgfscope}%
\pgfpathrectangle{\pgfqpoint{0.854460in}{0.571603in}}{\pgfqpoint{5.885100in}{5.225635in}}%
\pgfusepath{clip}%
\pgfsetbuttcap%
\pgfsetroundjoin%
\pgfsetlinewidth{1.505625pt}%
\definecolor{currentstroke}{rgb}{0.169646,0.456262,0.558030}%
\pgfsetstrokecolor{currentstroke}%
\pgfsetdash{}{0pt}%
\pgfpathmoveto{\pgfqpoint{1.313602in}{0.870997in}}%
\pgfpathlineto{\pgfqpoint{1.299525in}{0.886717in}}%
\pgfpathlineto{\pgfqpoint{1.298061in}{0.888379in}}%
\pgfpathlineto{\pgfqpoint{1.276538in}{0.912976in}}%
\pgfpathlineto{\pgfqpoint{1.268488in}{0.922317in}}%
\pgfpathlineto{\pgfqpoint{1.254002in}{0.939236in}}%
\pgfpathlineto{\pgfqpoint{1.238914in}{0.957128in}}%
\pgfpathlineto{\pgfqpoint{1.231906in}{0.965495in}}%
\pgfpathlineto{\pgfqpoint{1.210252in}{0.991755in}}%
\pgfpathlineto{\pgfqpoint{1.209341in}{0.992880in}}%
\pgfpathlineto{\pgfqpoint{1.189128in}{1.018014in}}%
\pgfpathlineto{\pgfqpoint{1.179767in}{1.029834in}}%
\pgfpathlineto{\pgfqpoint{1.168416in}{1.044274in}}%
\pgfpathlineto{\pgfqpoint{1.150194in}{1.067814in}}%
\pgfpathlineto{\pgfqpoint{1.148105in}{1.070533in}}%
\pgfpathlineto{\pgfqpoint{1.128297in}{1.096793in}}%
\pgfpathlineto{\pgfqpoint{1.120621in}{1.107138in}}%
\pgfpathlineto{\pgfqpoint{1.108904in}{1.123052in}}%
\pgfpathlineto{\pgfqpoint{1.091047in}{1.147691in}}%
\pgfpathlineto{\pgfqpoint{1.089882in}{1.149312in}}%
\pgfpathlineto{\pgfqpoint{1.071368in}{1.175571in}}%
\pgfpathlineto{\pgfqpoint{1.061474in}{1.189837in}}%
\pgfpathlineto{\pgfqpoint{1.053224in}{1.201831in}}%
\pgfpathlineto{\pgfqpoint{1.035474in}{1.228090in}}%
\pgfpathlineto{\pgfqpoint{1.031901in}{1.233481in}}%
\pgfpathlineto{\pgfqpoint{1.018184in}{1.254350in}}%
\pgfpathlineto{\pgfqpoint{1.002327in}{1.278873in}}%
\pgfpathlineto{\pgfqpoint{1.001215in}{1.280609in}}%
\pgfpathlineto{\pgfqpoint{0.984736in}{1.306869in}}%
\pgfpathlineto{\pgfqpoint{0.972754in}{1.326292in}}%
\pgfpathlineto{\pgfqpoint{0.968574in}{1.333128in}}%
\pgfpathlineto{\pgfqpoint{0.952849in}{1.359388in}}%
\pgfpathlineto{\pgfqpoint{0.943181in}{1.375839in}}%
\pgfpathlineto{\pgfqpoint{0.937469in}{1.385647in}}%
\pgfpathlineto{\pgfqpoint{0.922493in}{1.411906in}}%
\pgfpathlineto{\pgfqpoint{0.913607in}{1.427799in}}%
\pgfpathlineto{\pgfqpoint{0.907865in}{1.438166in}}%
\pgfpathlineto{\pgfqpoint{0.893633in}{1.464425in}}%
\pgfpathlineto{\pgfqpoint{0.884034in}{1.482498in}}%
\pgfpathlineto{\pgfqpoint{0.879728in}{1.490685in}}%
\pgfpathlineto{\pgfqpoint{0.866232in}{1.516944in}}%
\pgfpathlineto{\pgfqpoint{0.854460in}{1.540315in}}%
\pgfusepath{stroke}%
\end{pgfscope}%
\begin{pgfscope}%
\pgfpathrectangle{\pgfqpoint{0.854460in}{0.571603in}}{\pgfqpoint{5.885100in}{5.225635in}}%
\pgfusepath{clip}%
\pgfsetbuttcap%
\pgfsetroundjoin%
\pgfsetlinewidth{1.505625pt}%
\definecolor{currentstroke}{rgb}{0.169646,0.456262,0.558030}%
\pgfsetstrokecolor{currentstroke}%
\pgfsetdash{}{0pt}%
\pgfpathmoveto{\pgfqpoint{0.854460in}{3.911451in}}%
\pgfpathlineto{\pgfqpoint{0.865051in}{3.932816in}}%
\pgfpathlineto{\pgfqpoint{0.878320in}{3.959075in}}%
\pgfpathlineto{\pgfqpoint{0.884034in}{3.970159in}}%
\pgfpathlineto{\pgfqpoint{0.891977in}{3.985335in}}%
\pgfpathlineto{\pgfqpoint{0.905985in}{4.011594in}}%
\pgfpathlineto{\pgfqpoint{0.913607in}{4.025618in}}%
\pgfpathlineto{\pgfqpoint{0.920358in}{4.037854in}}%
\pgfpathlineto{\pgfqpoint{0.935116in}{4.064113in}}%
\pgfpathlineto{\pgfqpoint{0.943181in}{4.078209in}}%
\pgfpathlineto{\pgfqpoint{0.950243in}{4.090373in}}%
\pgfpathlineto{\pgfqpoint{0.965761in}{4.116632in}}%
\pgfpathlineto{\pgfqpoint{0.972754in}{4.128261in}}%
\pgfpathlineto{\pgfqpoint{0.981681in}{4.142892in}}%
\pgfpathlineto{\pgfqpoint{0.997969in}{4.169151in}}%
\pgfpathlineto{\pgfqpoint{1.002327in}{4.176057in}}%
\pgfpathlineto{\pgfqpoint{1.014719in}{4.195411in}}%
\pgfpathlineto{\pgfqpoint{1.031784in}{4.221670in}}%
\pgfpathlineto{\pgfqpoint{1.031901in}{4.221846in}}%
\pgfpathlineto{\pgfqpoint{1.049403in}{4.247930in}}%
\pgfpathlineto{\pgfqpoint{1.061474in}{4.265662in}}%
\pgfpathlineto{\pgfqpoint{1.067362in}{4.274189in}}%
\pgfpathlineto{\pgfqpoint{1.085777in}{4.300449in}}%
\pgfpathlineto{\pgfqpoint{1.091047in}{4.307848in}}%
\pgfpathlineto{\pgfqpoint{1.104669in}{4.326708in}}%
\pgfpathlineto{\pgfqpoint{1.120621in}{4.348489in}}%
\pgfpathlineto{\pgfqpoint{1.123946in}{4.352967in}}%
\pgfpathlineto{\pgfqpoint{1.143747in}{4.379227in}}%
\pgfpathlineto{\pgfqpoint{1.150194in}{4.387654in}}%
\pgfpathlineto{\pgfqpoint{1.164022in}{4.405486in}}%
\pgfpathlineto{\pgfqpoint{1.179767in}{4.425521in}}%
\pgfpathlineto{\pgfqpoint{1.184726in}{4.431746in}}%
\pgfpathlineto{\pgfqpoint{1.205942in}{4.458005in}}%
\pgfpathlineto{\pgfqpoint{1.209341in}{4.462152in}}%
\pgfpathlineto{\pgfqpoint{1.227708in}{4.484265in}}%
\pgfpathlineto{\pgfqpoint{1.238914in}{4.497582in}}%
\pgfpathlineto{\pgfqpoint{1.249948in}{4.510524in}}%
\pgfpathlineto{\pgfqpoint{1.268488in}{4.531991in}}%
\pgfpathlineto{\pgfqpoint{1.272681in}{4.536784in}}%
\pgfpathlineto{\pgfqpoint{1.295965in}{4.563043in}}%
\pgfpathlineto{\pgfqpoint{1.298061in}{4.565374in}}%
\pgfpathlineto{\pgfqpoint{1.319848in}{4.589303in}}%
\pgfpathlineto{\pgfqpoint{1.327634in}{4.597746in}}%
\pgfpathlineto{\pgfqpoint{1.344270in}{4.615562in}}%
\pgfpathlineto{\pgfqpoint{1.357208in}{4.629245in}}%
\pgfpathlineto{\pgfqpoint{1.369248in}{4.641822in}}%
\pgfpathlineto{\pgfqpoint{1.386781in}{4.659909in}}%
\pgfpathlineto{\pgfqpoint{1.394801in}{4.668081in}}%
\pgfpathlineto{\pgfqpoint{1.416354in}{4.689773in}}%
\pgfpathlineto{\pgfqpoint{1.420948in}{4.694341in}}%
\pgfpathlineto{\pgfqpoint{1.445928in}{4.718873in}}%
\pgfpathlineto{\pgfqpoint{1.447707in}{4.720600in}}%
\pgfpathlineto{\pgfqpoint{1.475106in}{4.746860in}}%
\pgfpathlineto{\pgfqpoint{1.475501in}{4.747234in}}%
\pgfpathlineto{\pgfqpoint{1.503177in}{4.773119in}}%
\pgfpathlineto{\pgfqpoint{1.505074in}{4.774872in}}%
\pgfpathlineto{\pgfqpoint{1.531904in}{4.799378in}}%
\pgfpathlineto{\pgfqpoint{1.534648in}{4.801854in}}%
\pgfpathlineto{\pgfqpoint{1.561307in}{4.825638in}}%
\pgfpathlineto{\pgfqpoint{1.564221in}{4.828207in}}%
\pgfpathlineto{\pgfqpoint{1.591403in}{4.851897in}}%
\pgfpathlineto{\pgfqpoint{1.593795in}{4.853957in}}%
\pgfpathlineto{\pgfqpoint{1.622210in}{4.878157in}}%
\pgfpathlineto{\pgfqpoint{1.623368in}{4.879131in}}%
\pgfpathlineto{\pgfqpoint{1.652941in}{4.903740in}}%
\pgfpathlineto{\pgfqpoint{1.653763in}{4.904416in}}%
\pgfpathlineto{\pgfqpoint{1.682515in}{4.927790in}}%
\pgfpathlineto{\pgfqpoint{1.686103in}{4.930676in}}%
\pgfpathlineto{\pgfqpoint{1.712088in}{4.951325in}}%
\pgfpathlineto{\pgfqpoint{1.719223in}{4.956935in}}%
\pgfpathlineto{\pgfqpoint{1.741661in}{4.974367in}}%
\pgfpathlineto{\pgfqpoint{1.753143in}{4.983195in}}%
\pgfpathlineto{\pgfqpoint{1.771235in}{4.996938in}}%
\pgfpathlineto{\pgfqpoint{1.787880in}{5.009454in}}%
\pgfpathlineto{\pgfqpoint{1.800808in}{5.019060in}}%
\pgfpathlineto{\pgfqpoint{1.823450in}{5.035714in}}%
\pgfpathlineto{\pgfqpoint{1.830381in}{5.040751in}}%
\pgfpathlineto{\pgfqpoint{1.859871in}{5.061973in}}%
\pgfpathlineto{\pgfqpoint{1.859955in}{5.062033in}}%
\pgfpathlineto{\pgfqpoint{1.889528in}{5.082812in}}%
\pgfpathlineto{\pgfqpoint{1.897319in}{5.088233in}}%
\pgfpathlineto{\pgfqpoint{1.919102in}{5.103209in}}%
\pgfpathlineto{\pgfqpoint{1.935670in}{5.114492in}}%
\pgfpathlineto{\pgfqpoint{1.948675in}{5.123243in}}%
\pgfpathlineto{\pgfqpoint{1.974938in}{5.140752in}}%
\pgfpathlineto{\pgfqpoint{1.978248in}{5.142932in}}%
\pgfpathlineto{\pgfqpoint{2.007822in}{5.162193in}}%
\pgfpathlineto{\pgfqpoint{2.015296in}{5.167011in}}%
\pgfpathlineto{\pgfqpoint{2.037395in}{5.181088in}}%
\pgfpathlineto{\pgfqpoint{2.056691in}{5.193271in}}%
\pgfpathlineto{\pgfqpoint{2.066968in}{5.199681in}}%
\pgfpathlineto{\pgfqpoint{2.096542in}{5.217955in}}%
\pgfpathlineto{\pgfqpoint{2.099122in}{5.219530in}}%
\pgfpathlineto{\pgfqpoint{2.126115in}{5.235814in}}%
\pgfpathlineto{\pgfqpoint{2.142793in}{5.245790in}}%
\pgfpathlineto{\pgfqpoint{2.155689in}{5.253410in}}%
\pgfpathlineto{\pgfqpoint{2.185262in}{5.270730in}}%
\pgfpathlineto{\pgfqpoint{2.187542in}{5.272049in}}%
\pgfpathlineto{\pgfqpoint{2.214835in}{5.287647in}}%
\pgfpathlineto{\pgfqpoint{2.224866in}{5.293334in}}%
\pgfusepath{stroke}%
\end{pgfscope}%
\begin{pgfscope}%
\pgfpathrectangle{\pgfqpoint{0.854460in}{0.571603in}}{\pgfqpoint{5.885100in}{5.225635in}}%
\pgfusepath{clip}%
\pgfsetbuttcap%
\pgfsetroundjoin%
\pgfsetlinewidth{1.505625pt}%
\definecolor{currentstroke}{rgb}{0.169646,0.456262,0.558030}%
\pgfsetstrokecolor{currentstroke}%
\pgfsetdash{}{0pt}%
\pgfpathmoveto{\pgfqpoint{2.562937in}{5.466735in}}%
\pgfpathlineto{\pgfqpoint{2.569716in}{5.469882in}}%
\pgfpathlineto{\pgfqpoint{2.596260in}{5.482125in}}%
\pgfpathlineto{\pgfqpoint{2.599289in}{5.483504in}}%
\pgfpathlineto{\pgfqpoint{2.628862in}{5.496783in}}%
\pgfpathlineto{\pgfqpoint{2.654882in}{5.508384in}}%
\pgfpathlineto{\pgfqpoint{2.658436in}{5.509949in}}%
\pgfpathlineto{\pgfqpoint{2.688009in}{5.522782in}}%
\pgfpathlineto{\pgfqpoint{2.715535in}{5.534644in}}%
\pgfpathlineto{\pgfqpoint{2.717582in}{5.535515in}}%
\pgfpathlineto{\pgfqpoint{2.747156in}{5.547905in}}%
\pgfpathlineto{\pgfqpoint{2.776729in}{5.560214in}}%
\pgfpathlineto{\pgfqpoint{2.778412in}{5.560903in}}%
\pgfpathlineto{\pgfqpoint{2.806303in}{5.572182in}}%
\pgfpathlineto{\pgfqpoint{2.835876in}{5.584049in}}%
\pgfpathlineto{\pgfqpoint{2.843749in}{5.587163in}}%
\pgfpathlineto{\pgfqpoint{2.865449in}{5.595636in}}%
\pgfpathlineto{\pgfqpoint{2.895023in}{5.607065in}}%
\pgfpathlineto{\pgfqpoint{2.911673in}{5.613422in}}%
\pgfpathlineto{\pgfqpoint{2.924596in}{5.618293in}}%
\pgfpathlineto{\pgfqpoint{2.954169in}{5.629287in}}%
\pgfpathlineto{\pgfqpoint{2.982379in}{5.639682in}}%
\pgfpathlineto{\pgfqpoint{2.983743in}{5.640178in}}%
\pgfpathlineto{\pgfqpoint{3.013316in}{5.650739in}}%
\pgfpathlineto{\pgfqpoint{3.042890in}{5.661205in}}%
\pgfpathlineto{\pgfqpoint{3.056473in}{5.665941in}}%
\pgfpathlineto{\pgfqpoint{3.072463in}{5.671444in}}%
\pgfpathlineto{\pgfqpoint{3.102036in}{5.681476in}}%
\pgfpathlineto{\pgfqpoint{3.131610in}{5.691408in}}%
\pgfpathlineto{\pgfqpoint{3.134017in}{5.692201in}}%
\pgfpathlineto{\pgfqpoint{3.161183in}{5.701026in}}%
\pgfpathlineto{\pgfqpoint{3.190756in}{5.710521in}}%
\pgfpathlineto{\pgfqpoint{3.215801in}{5.718460in}}%
\pgfpathlineto{\pgfqpoint{3.220330in}{5.719876in}}%
\pgfpathlineto{\pgfqpoint{3.249903in}{5.728937in}}%
\pgfpathlineto{\pgfqpoint{3.279476in}{5.737887in}}%
\pgfpathlineto{\pgfqpoint{3.302396in}{5.744720in}}%
\pgfpathlineto{\pgfqpoint{3.309050in}{5.746677in}}%
\pgfpathlineto{\pgfqpoint{3.338623in}{5.755188in}}%
\pgfpathlineto{\pgfqpoint{3.368197in}{5.763583in}}%
\pgfpathlineto{\pgfqpoint{3.394661in}{5.770979in}}%
\pgfpathlineto{\pgfqpoint{3.397770in}{5.771836in}}%
\pgfpathlineto{\pgfqpoint{3.427343in}{5.779788in}}%
\pgfpathlineto{\pgfqpoint{3.456917in}{5.787615in}}%
\pgfpathlineto{\pgfqpoint{3.486490in}{5.795314in}}%
\pgfpathlineto{\pgfqpoint{3.494076in}{5.797238in}}%
\pgfusepath{stroke}%
\end{pgfscope}%
\begin{pgfscope}%
\pgfpathrectangle{\pgfqpoint{0.854460in}{0.571603in}}{\pgfqpoint{5.885100in}{5.225635in}}%
\pgfusepath{clip}%
\pgfsetbuttcap%
\pgfsetroundjoin%
\pgfsetlinewidth{1.505625pt}%
\definecolor{currentstroke}{rgb}{0.169646,0.456262,0.558030}%
\pgfsetstrokecolor{currentstroke}%
\pgfsetdash{}{0pt}%
\pgfpathmoveto{\pgfqpoint{5.057130in}{5.797238in}}%
\pgfpathlineto{\pgfqpoint{5.097244in}{5.770979in}}%
\pgfpathlineto{\pgfqpoint{5.131244in}{5.744720in}}%
\pgfpathlineto{\pgfqpoint{5.160345in}{5.718460in}}%
\pgfpathlineto{\pgfqpoint{5.185468in}{5.692201in}}%
\pgfpathlineto{\pgfqpoint{5.207326in}{5.665941in}}%
\pgfpathlineto{\pgfqpoint{5.231319in}{5.632101in}}%
\pgfpathlineto{\pgfqpoint{5.242946in}{5.613422in}}%
\pgfpathlineto{\pgfqpoint{5.260892in}{5.580461in}}%
\pgfpathlineto{\pgfqpoint{5.270228in}{5.560903in}}%
\pgfpathlineto{\pgfqpoint{5.281351in}{5.534644in}}%
\pgfpathlineto{\pgfqpoint{5.291106in}{5.508384in}}%
\pgfpathlineto{\pgfqpoint{5.306748in}{5.455865in}}%
\pgfpathlineto{\pgfqpoint{5.318399in}{5.403346in}}%
\pgfpathlineto{\pgfqpoint{5.326602in}{5.350827in}}%
\pgfpathlineto{\pgfqpoint{5.332056in}{5.298308in}}%
\pgfpathlineto{\pgfqpoint{5.335238in}{5.245790in}}%
\pgfpathlineto{\pgfqpoint{5.336523in}{5.193271in}}%
\pgfpathlineto{\pgfqpoint{5.335585in}{5.114492in}}%
\pgfpathlineto{\pgfqpoint{5.331986in}{5.035714in}}%
\pgfpathlineto{\pgfqpoint{5.324268in}{4.930676in}}%
\pgfpathlineto{\pgfqpoint{5.311646in}{4.799378in}}%
\pgfpathlineto{\pgfqpoint{5.268436in}{4.379227in}}%
\pgfpathlineto{\pgfqpoint{5.258302in}{4.247930in}}%
\pgfpathlineto{\pgfqpoint{5.252307in}{4.142892in}}%
\pgfpathlineto{\pgfqpoint{5.248580in}{4.037854in}}%
\pgfpathlineto{\pgfqpoint{5.247385in}{3.932816in}}%
\pgfpathlineto{\pgfqpoint{5.248952in}{3.827778in}}%
\pgfpathlineto{\pgfqpoint{5.252069in}{3.749000in}}%
\pgfpathlineto{\pgfqpoint{5.256942in}{3.670221in}}%
\pgfpathlineto{\pgfqpoint{5.263623in}{3.591443in}}%
\pgfpathlineto{\pgfqpoint{5.272155in}{3.512664in}}%
\pgfpathlineto{\pgfqpoint{5.282655in}{3.433886in}}%
\pgfpathlineto{\pgfqpoint{5.295149in}{3.355107in}}%
\pgfpathlineto{\pgfqpoint{5.309656in}{3.276329in}}%
\pgfpathlineto{\pgfqpoint{5.326267in}{3.197551in}}%
\pgfpathlineto{\pgfqpoint{5.344993in}{3.118772in}}%
\pgfpathlineto{\pgfqpoint{5.365848in}{3.039994in}}%
\pgfpathlineto{\pgfqpoint{5.388909in}{2.961215in}}%
\pgfpathlineto{\pgfqpoint{5.414187in}{2.882437in}}%
\pgfpathlineto{\pgfqpoint{5.441698in}{2.803659in}}%
\pgfpathlineto{\pgfqpoint{5.471459in}{2.724880in}}%
\pgfpathlineto{\pgfqpoint{5.503487in}{2.646102in}}%
\pgfpathlineto{\pgfqpoint{5.537806in}{2.567323in}}%
\pgfpathlineto{\pgfqpoint{5.574441in}{2.488545in}}%
\pgfpathlineto{\pgfqpoint{5.615772in}{2.405202in}}%
\pgfpathlineto{\pgfqpoint{5.654677in}{2.330988in}}%
\pgfpathlineto{\pgfqpoint{5.704492in}{2.241439in}}%
\pgfpathlineto{\pgfqpoint{5.744244in}{2.173431in}}%
\pgfpathlineto{\pgfqpoint{5.793213in}{2.093642in}}%
\pgfpathlineto{\pgfqpoint{5.852359in}{2.002065in}}%
\pgfpathlineto{\pgfqpoint{5.896172in}{1.937096in}}%
\pgfpathlineto{\pgfqpoint{5.951509in}{1.858318in}}%
\pgfpathlineto{\pgfqpoint{6.009188in}{1.779539in}}%
\pgfpathlineto{\pgfqpoint{6.069200in}{1.700761in}}%
\pgfpathlineto{\pgfqpoint{6.148093in}{1.601602in}}%
\pgfpathlineto{\pgfqpoint{6.207240in}{1.530100in}}%
\pgfpathlineto{\pgfqpoint{6.266386in}{1.460787in}}%
\pgfpathlineto{\pgfqpoint{6.347093in}{1.369430in}}%
\pgfpathlineto{\pgfqpoint{6.347093in}{1.369430in}}%
\pgfusepath{stroke}%
\end{pgfscope}%
\begin{pgfscope}%
\pgfpathrectangle{\pgfqpoint{0.854460in}{0.571603in}}{\pgfqpoint{5.885100in}{5.225635in}}%
\pgfusepath{clip}%
\pgfsetbuttcap%
\pgfsetroundjoin%
\pgfsetlinewidth{1.505625pt}%
\definecolor{currentstroke}{rgb}{0.169646,0.456262,0.558030}%
\pgfsetstrokecolor{currentstroke}%
\pgfsetdash{}{0pt}%
\pgfpathmoveto{\pgfqpoint{6.612442in}{1.090477in}}%
\pgfpathlineto{\pgfqpoint{6.621267in}{1.081693in}}%
\pgfpathlineto{\pgfqpoint{6.632478in}{1.070533in}}%
\pgfpathlineto{\pgfqpoint{6.650840in}{1.052428in}}%
\pgfpathlineto{\pgfqpoint{6.659110in}{1.044274in}}%
\pgfpathlineto{\pgfqpoint{6.680414in}{1.023463in}}%
\pgfpathlineto{\pgfqpoint{6.685992in}{1.018014in}}%
\pgfpathlineto{\pgfqpoint{6.709987in}{0.994789in}}%
\pgfpathlineto{\pgfqpoint{6.713123in}{0.991755in}}%
\pgfpathlineto{\pgfqpoint{6.739560in}{0.966396in}}%
\pgfusepath{stroke}%
\end{pgfscope}%
\begin{pgfscope}%
\pgfpathrectangle{\pgfqpoint{0.854460in}{0.571603in}}{\pgfqpoint{5.885100in}{5.225635in}}%
\pgfusepath{clip}%
\pgfsetbuttcap%
\pgfsetroundjoin%
\pgfsetlinewidth{1.505625pt}%
\definecolor{currentstroke}{rgb}{0.162142,0.474838,0.558140}%
\pgfsetstrokecolor{currentstroke}%
\pgfsetdash{}{0pt}%
\pgfpathmoveto{\pgfqpoint{1.543857in}{0.571603in}}%
\pgfpathlineto{\pgfqpoint{1.534648in}{0.579950in}}%
\pgfpathlineto{\pgfqpoint{1.517055in}{0.595968in}}%
\pgfusepath{stroke}%
\end{pgfscope}%
\begin{pgfscope}%
\pgfpathrectangle{\pgfqpoint{0.854460in}{0.571603in}}{\pgfqpoint{5.885100in}{5.225635in}}%
\pgfusepath{clip}%
\pgfsetbuttcap%
\pgfsetroundjoin%
\pgfsetlinewidth{1.505625pt}%
\definecolor{currentstroke}{rgb}{0.162142,0.474838,0.558140}%
\pgfsetstrokecolor{currentstroke}%
\pgfsetdash{}{0pt}%
\pgfpathmoveto{\pgfqpoint{1.246806in}{0.869582in}}%
\pgfpathlineto{\pgfqpoint{1.238914in}{0.878465in}}%
\pgfpathlineto{\pgfqpoint{1.231629in}{0.886717in}}%
\pgfpathlineto{\pgfqpoint{1.209341in}{0.912343in}}%
\pgfpathlineto{\pgfqpoint{1.208794in}{0.912976in}}%
\pgfpathlineto{\pgfqpoint{1.186494in}{0.939236in}}%
\pgfpathlineto{\pgfqpoint{1.179767in}{0.947279in}}%
\pgfpathlineto{\pgfqpoint{1.164638in}{0.965495in}}%
\pgfpathlineto{\pgfqpoint{1.150194in}{0.983151in}}%
\pgfpathlineto{\pgfqpoint{1.143204in}{0.991755in}}%
\pgfpathlineto{\pgfqpoint{1.122206in}{1.018014in}}%
\pgfpathlineto{\pgfqpoint{1.120621in}{1.020033in}}%
\pgfpathlineto{\pgfqpoint{1.101719in}{1.044274in}}%
\pgfpathlineto{\pgfqpoint{1.091047in}{1.058172in}}%
\pgfpathlineto{\pgfqpoint{1.081627in}{1.070533in}}%
\pgfpathlineto{\pgfqpoint{1.061925in}{1.096793in}}%
\pgfpathlineto{\pgfqpoint{1.061474in}{1.097405in}}%
\pgfpathlineto{\pgfqpoint{1.042742in}{1.123052in}}%
\pgfpathlineto{\pgfqpoint{1.031901in}{1.138129in}}%
\pgfpathlineto{\pgfqpoint{1.023924in}{1.149312in}}%
\pgfpathlineto{\pgfqpoint{1.005507in}{1.175571in}}%
\pgfpathlineto{\pgfqpoint{1.002327in}{1.180192in}}%
\pgfpathlineto{\pgfqpoint{0.987558in}{1.201831in}}%
\pgfpathlineto{\pgfqpoint{0.972754in}{1.223871in}}%
\pgfpathlineto{\pgfqpoint{0.969944in}{1.228090in}}%
\pgfpathlineto{\pgfqpoint{0.952796in}{1.254350in}}%
\pgfpathlineto{\pgfqpoint{0.943181in}{1.269332in}}%
\pgfpathlineto{\pgfqpoint{0.936006in}{1.280609in}}%
\pgfpathlineto{\pgfqpoint{0.919610in}{1.306869in}}%
\pgfpathlineto{\pgfqpoint{0.913607in}{1.316671in}}%
\pgfpathlineto{\pgfqpoint{0.903619in}{1.333128in}}%
\pgfpathlineto{\pgfqpoint{0.887971in}{1.359388in}}%
\pgfpathlineto{\pgfqpoint{0.884034in}{1.366134in}}%
\pgfpathlineto{\pgfqpoint{0.872751in}{1.385647in}}%
\pgfpathlineto{\pgfqpoint{0.857846in}{1.411906in}}%
\pgfpathlineto{\pgfqpoint{0.854460in}{1.418003in}}%
\pgfusepath{stroke}%
\end{pgfscope}%
\begin{pgfscope}%
\pgfpathrectangle{\pgfqpoint{0.854460in}{0.571603in}}{\pgfqpoint{5.885100in}{5.225635in}}%
\pgfusepath{clip}%
\pgfsetbuttcap%
\pgfsetroundjoin%
\pgfsetlinewidth{1.505625pt}%
\definecolor{currentstroke}{rgb}{0.162142,0.474838,0.558140}%
\pgfsetstrokecolor{currentstroke}%
\pgfsetdash{}{0pt}%
\pgfpathmoveto{\pgfqpoint{0.854460in}{4.056693in}}%
\pgfpathlineto{\pgfqpoint{0.858544in}{4.064113in}}%
\pgfpathlineto{\pgfqpoint{0.873273in}{4.090373in}}%
\pgfpathlineto{\pgfqpoint{0.884034in}{4.109232in}}%
\pgfpathlineto{\pgfqpoint{0.888318in}{4.116632in}}%
\pgfpathlineto{\pgfqpoint{0.903798in}{4.142892in}}%
\pgfpathlineto{\pgfqpoint{0.913607in}{4.159256in}}%
\pgfpathlineto{\pgfqpoint{0.919623in}{4.169151in}}%
\pgfpathlineto{\pgfqpoint{0.935865in}{4.195411in}}%
\pgfpathlineto{\pgfqpoint{0.943181in}{4.207044in}}%
\pgfpathlineto{\pgfqpoint{0.952508in}{4.221670in}}%
\pgfpathlineto{\pgfqpoint{0.969518in}{4.247930in}}%
\pgfpathlineto{\pgfqpoint{0.972754in}{4.252841in}}%
\pgfpathlineto{\pgfqpoint{0.987016in}{4.274189in}}%
\pgfpathlineto{\pgfqpoint{1.002327in}{4.296781in}}%
\pgfpathlineto{\pgfqpoint{1.004848in}{4.300449in}}%
\pgfpathlineto{\pgfqpoint{1.023191in}{4.326708in}}%
\pgfpathlineto{\pgfqpoint{1.031901in}{4.338994in}}%
\pgfpathlineto{\pgfqpoint{1.041941in}{4.352967in}}%
\pgfpathlineto{\pgfqpoint{1.061075in}{4.379227in}}%
\pgfpathlineto{\pgfqpoint{1.061474in}{4.379766in}}%
\pgfpathlineto{\pgfqpoint{1.080778in}{4.405486in}}%
\pgfpathlineto{\pgfqpoint{1.091047in}{4.418984in}}%
\pgfpathlineto{\pgfqpoint{1.100886in}{4.431746in}}%
\pgfpathlineto{\pgfqpoint{1.120621in}{4.457004in}}%
\pgfpathlineto{\pgfqpoint{1.121413in}{4.458005in}}%
\pgfpathlineto{\pgfqpoint{1.142522in}{4.484265in}}%
\pgfpathlineto{\pgfqpoint{1.150194in}{4.493682in}}%
\pgfpathlineto{\pgfqpoint{1.164092in}{4.510524in}}%
\pgfpathlineto{\pgfqpoint{1.179767in}{4.529274in}}%
\pgfpathlineto{\pgfqpoint{1.186126in}{4.536784in}}%
\pgfpathlineto{\pgfqpoint{1.208654in}{4.563043in}}%
\pgfpathlineto{\pgfqpoint{1.209341in}{4.563832in}}%
\pgfpathlineto{\pgfqpoint{1.231790in}{4.589303in}}%
\pgfpathlineto{\pgfqpoint{1.238914in}{4.597282in}}%
\pgfpathlineto{\pgfqpoint{1.255434in}{4.615562in}}%
\pgfpathlineto{\pgfqpoint{1.268488in}{4.629823in}}%
\pgfpathlineto{\pgfqpoint{1.279604in}{4.641822in}}%
\pgfpathlineto{\pgfqpoint{1.298061in}{4.661493in}}%
\pgfpathlineto{\pgfqpoint{1.304317in}{4.668081in}}%
\pgfpathlineto{\pgfqpoint{1.327634in}{4.692330in}}%
\pgfpathlineto{\pgfqpoint{1.329590in}{4.694341in}}%
\pgfpathlineto{\pgfqpoint{1.355476in}{4.720600in}}%
\pgfpathlineto{\pgfqpoint{1.357208in}{4.722334in}}%
\pgfpathlineto{\pgfqpoint{1.381984in}{4.746860in}}%
\pgfpathlineto{\pgfqpoint{1.386781in}{4.751549in}}%
\pgfpathlineto{\pgfqpoint{1.409096in}{4.773119in}}%
\pgfpathlineto{\pgfqpoint{1.416354in}{4.780049in}}%
\pgfpathlineto{\pgfqpoint{1.436830in}{4.799378in}}%
\pgfpathlineto{\pgfqpoint{1.445928in}{4.807862in}}%
\pgfpathlineto{\pgfqpoint{1.465203in}{4.825638in}}%
\pgfpathlineto{\pgfqpoint{1.475501in}{4.835020in}}%
\pgfpathlineto{\pgfqpoint{1.494233in}{4.851897in}}%
\pgfpathlineto{\pgfqpoint{1.505074in}{4.861548in}}%
\pgfpathlineto{\pgfqpoint{1.523937in}{4.878157in}}%
\pgfpathlineto{\pgfqpoint{1.534648in}{4.887475in}}%
\pgfpathlineto{\pgfqpoint{1.554332in}{4.904416in}}%
\pgfpathlineto{\pgfqpoint{1.564221in}{4.912825in}}%
\pgfpathlineto{\pgfqpoint{1.585437in}{4.930676in}}%
\pgfpathlineto{\pgfqpoint{1.593795in}{4.937623in}}%
\pgfpathlineto{\pgfqpoint{1.617268in}{4.956935in}}%
\pgfpathlineto{\pgfqpoint{1.623368in}{4.961893in}}%
\pgfpathlineto{\pgfqpoint{1.649842in}{4.983195in}}%
\pgfpathlineto{\pgfqpoint{1.652941in}{4.985658in}}%
\pgfpathlineto{\pgfqpoint{1.682515in}{5.008929in}}%
\pgfpathlineto{\pgfqpoint{1.683189in}{5.009454in}}%
\pgfpathlineto{\pgfqpoint{1.712088in}{5.031678in}}%
\pgfpathlineto{\pgfqpoint{1.717389in}{5.035714in}}%
\pgfpathlineto{\pgfqpoint{1.741661in}{5.053975in}}%
\pgfpathlineto{\pgfqpoint{1.752396in}{5.061973in}}%
\pgfpathlineto{\pgfqpoint{1.771235in}{5.075842in}}%
\pgfpathlineto{\pgfqpoint{1.788227in}{5.088233in}}%
\pgfpathlineto{\pgfqpoint{1.800808in}{5.097297in}}%
\pgfpathlineto{\pgfqpoint{1.824898in}{5.114492in}}%
\pgfpathlineto{\pgfqpoint{1.830381in}{5.118359in}}%
\pgfpathlineto{\pgfqpoint{1.848788in}{5.131214in}}%
\pgfusepath{stroke}%
\end{pgfscope}%
\begin{pgfscope}%
\pgfpathrectangle{\pgfqpoint{0.854460in}{0.571603in}}{\pgfqpoint{5.885100in}{5.225635in}}%
\pgfusepath{clip}%
\pgfsetbuttcap%
\pgfsetroundjoin%
\pgfsetlinewidth{1.505625pt}%
\definecolor{currentstroke}{rgb}{0.162142,0.474838,0.558140}%
\pgfsetstrokecolor{currentstroke}%
\pgfsetdash{}{0pt}%
\pgfpathmoveto{\pgfqpoint{2.171091in}{5.334606in}}%
\pgfpathlineto{\pgfqpoint{2.185262in}{5.342697in}}%
\pgfpathlineto{\pgfqpoint{2.199619in}{5.350827in}}%
\pgfpathlineto{\pgfqpoint{2.214835in}{5.359340in}}%
\pgfpathlineto{\pgfqpoint{2.244409in}{5.375753in}}%
\pgfpathlineto{\pgfqpoint{2.246841in}{5.377087in}}%
\pgfpathlineto{\pgfqpoint{2.273982in}{5.391790in}}%
\pgfpathlineto{\pgfqpoint{2.295464in}{5.403346in}}%
\pgfpathlineto{\pgfqpoint{2.303555in}{5.407646in}}%
\pgfpathlineto{\pgfqpoint{2.333129in}{5.423198in}}%
\pgfpathlineto{\pgfqpoint{2.345428in}{5.429606in}}%
\pgfpathlineto{\pgfqpoint{2.362702in}{5.438495in}}%
\pgfpathlineto{\pgfqpoint{2.392275in}{5.453595in}}%
\pgfpathlineto{\pgfqpoint{2.396774in}{5.455865in}}%
\pgfpathlineto{\pgfqpoint{2.421849in}{5.468361in}}%
\pgfpathlineto{\pgfqpoint{2.449641in}{5.482125in}}%
\pgfpathlineto{\pgfqpoint{2.451422in}{5.482996in}}%
\pgfpathlineto{\pgfqpoint{2.480996in}{5.497274in}}%
\pgfpathlineto{\pgfqpoint{2.504160in}{5.508384in}}%
\pgfpathlineto{\pgfqpoint{2.510569in}{5.511420in}}%
\pgfpathlineto{\pgfqpoint{2.540142in}{5.525264in}}%
\pgfpathlineto{\pgfqpoint{2.560336in}{5.534644in}}%
\pgfpathlineto{\pgfqpoint{2.569716in}{5.538946in}}%
\pgfpathlineto{\pgfqpoint{2.599289in}{5.552360in}}%
\pgfpathlineto{\pgfqpoint{2.618284in}{5.560903in}}%
\pgfpathlineto{\pgfqpoint{2.628862in}{5.565602in}}%
\pgfpathlineto{\pgfqpoint{2.658436in}{5.578590in}}%
\pgfpathlineto{\pgfqpoint{2.678122in}{5.587163in}}%
\pgfpathlineto{\pgfqpoint{2.688009in}{5.591414in}}%
\pgfpathlineto{\pgfqpoint{2.717582in}{5.603981in}}%
\pgfpathlineto{\pgfqpoint{2.739978in}{5.613422in}}%
\pgfpathlineto{\pgfqpoint{2.747156in}{5.616410in}}%
\pgfpathlineto{\pgfqpoint{2.776729in}{5.628559in}}%
\pgfpathlineto{\pgfqpoint{2.803986in}{5.639682in}}%
\pgfpathlineto{\pgfqpoint{2.806303in}{5.640615in}}%
\pgfpathlineto{\pgfqpoint{2.835876in}{5.652350in}}%
\pgfpathlineto{\pgfqpoint{2.865449in}{5.664011in}}%
\pgfpathlineto{\pgfqpoint{2.870417in}{5.665941in}}%
\pgfpathlineto{\pgfqpoint{2.895023in}{5.675379in}}%
\pgfpathlineto{\pgfqpoint{2.924596in}{5.686627in}}%
\pgfpathlineto{\pgfqpoint{2.939428in}{5.692201in}}%
\pgfpathlineto{\pgfqpoint{2.954169in}{5.697669in}}%
\pgfpathlineto{\pgfqpoint{2.983743in}{5.708507in}}%
\pgfpathlineto{\pgfqpoint{3.011129in}{5.718460in}}%
\pgfpathlineto{\pgfqpoint{3.013316in}{5.719245in}}%
\pgfpathlineto{\pgfqpoint{3.042890in}{5.729674in}}%
\pgfpathlineto{\pgfqpoint{3.072463in}{5.740020in}}%
\pgfpathlineto{\pgfqpoint{3.086081in}{5.744720in}}%
\pgfpathlineto{\pgfqpoint{3.102036in}{5.750153in}}%
\pgfpathlineto{\pgfqpoint{3.131610in}{5.760091in}}%
\pgfpathlineto{\pgfqpoint{3.161183in}{5.769940in}}%
\pgfpathlineto{\pgfqpoint{3.164360in}{5.770979in}}%
\pgfpathlineto{\pgfqpoint{3.190756in}{5.779496in}}%
\pgfpathlineto{\pgfqpoint{3.220330in}{5.788936in}}%
\pgfpathlineto{\pgfqpoint{3.246629in}{5.797238in}}%
\pgfusepath{stroke}%
\end{pgfscope}%
\begin{pgfscope}%
\pgfpathrectangle{\pgfqpoint{0.854460in}{0.571603in}}{\pgfqpoint{5.885100in}{5.225635in}}%
\pgfusepath{clip}%
\pgfsetbuttcap%
\pgfsetroundjoin%
\pgfsetlinewidth{1.505625pt}%
\definecolor{currentstroke}{rgb}{0.162142,0.474838,0.558140}%
\pgfsetstrokecolor{currentstroke}%
\pgfsetdash{}{0pt}%
\pgfpathmoveto{\pgfqpoint{5.319889in}{5.797238in}}%
\pgfpathlineto{\pgfqpoint{5.343080in}{5.770979in}}%
\pgfpathlineto{\pgfqpoint{5.363234in}{5.744720in}}%
\pgfpathlineto{\pgfqpoint{5.380860in}{5.718460in}}%
\pgfpathlineto{\pgfqpoint{5.396085in}{5.692201in}}%
\pgfpathlineto{\pgfqpoint{5.409467in}{5.665941in}}%
\pgfpathlineto{\pgfqpoint{5.420967in}{5.639682in}}%
\pgfpathlineto{\pgfqpoint{5.438332in}{5.591459in}}%
\pgfpathlineto{\pgfqpoint{5.447080in}{5.560903in}}%
\pgfpathlineto{\pgfqpoint{5.458713in}{5.508384in}}%
\pgfpathlineto{\pgfqpoint{5.466774in}{5.455865in}}%
\pgfpathlineto{\pgfqpoint{5.471784in}{5.403346in}}%
\pgfpathlineto{\pgfqpoint{5.474329in}{5.350827in}}%
\pgfpathlineto{\pgfqpoint{5.474831in}{5.298308in}}%
\pgfpathlineto{\pgfqpoint{5.473630in}{5.245790in}}%
\pgfpathlineto{\pgfqpoint{5.469261in}{5.167011in}}%
\pgfpathlineto{\pgfqpoint{5.462445in}{5.088233in}}%
\pgfpathlineto{\pgfqpoint{5.450740in}{4.983195in}}%
\pgfpathlineto{\pgfqpoint{5.433594in}{4.851897in}}%
\pgfpathlineto{\pgfqpoint{5.383680in}{4.484265in}}%
\pgfpathlineto{\pgfqpoint{5.368694in}{4.352967in}}%
\pgfpathlineto{\pgfqpoint{5.358744in}{4.247930in}}%
\pgfpathlineto{\pgfqpoint{5.350973in}{4.142892in}}%
\pgfpathlineto{\pgfqpoint{5.345599in}{4.037854in}}%
\pgfpathlineto{\pgfqpoint{5.342913in}{3.932816in}}%
\pgfpathlineto{\pgfqpoint{5.342799in}{3.854037in}}%
\pgfpathlineto{\pgfqpoint{5.344404in}{3.775259in}}%
\pgfpathlineto{\pgfqpoint{5.347804in}{3.696481in}}%
\pgfpathlineto{\pgfqpoint{5.353038in}{3.617702in}}%
\pgfpathlineto{\pgfqpoint{5.360175in}{3.538924in}}%
\pgfpathlineto{\pgfqpoint{5.369303in}{3.460145in}}%
\pgfpathlineto{\pgfqpoint{5.380478in}{3.381367in}}%
\pgfpathlineto{\pgfqpoint{5.393661in}{3.302589in}}%
\pgfpathlineto{\pgfqpoint{5.409018in}{3.223810in}}%
\pgfpathlineto{\pgfqpoint{5.426450in}{3.145032in}}%
\pgfpathlineto{\pgfqpoint{5.446103in}{3.066253in}}%
\pgfpathlineto{\pgfqpoint{5.467985in}{2.987475in}}%
\pgfpathlineto{\pgfqpoint{5.492053in}{2.908696in}}%
\pgfpathlineto{\pgfqpoint{5.518394in}{2.829918in}}%
\pgfpathlineto{\pgfqpoint{5.547023in}{2.751140in}}%
\pgfpathlineto{\pgfqpoint{5.577957in}{2.672361in}}%
\pgfpathlineto{\pgfqpoint{5.615772in}{2.583299in}}%
\pgfpathlineto{\pgfqpoint{5.646802in}{2.514804in}}%
\pgfpathlineto{\pgfqpoint{5.684696in}{2.436026in}}%
\pgfpathlineto{\pgfqpoint{5.724967in}{2.357248in}}%
\pgfpathlineto{\pgfqpoint{5.767609in}{2.278469in}}%
\pgfpathlineto{\pgfqpoint{5.812588in}{2.199691in}}%
\pgfpathlineto{\pgfqpoint{5.859957in}{2.120912in}}%
\pgfpathlineto{\pgfqpoint{5.911506in}{2.039385in}}%
\pgfpathlineto{\pgfqpoint{5.970653in}{1.950390in}}%
\pgfpathlineto{\pgfqpoint{6.016274in}{1.884577in}}%
\pgfpathlineto{\pgfqpoint{6.088946in}{1.784502in}}%
\pgfpathlineto{\pgfqpoint{6.132344in}{1.727020in}}%
\pgfpathlineto{\pgfqpoint{6.145535in}{1.709951in}}%
\pgfpathlineto{\pgfqpoint{6.145535in}{1.709951in}}%
\pgfusepath{stroke}%
\end{pgfscope}%
\begin{pgfscope}%
\pgfpathrectangle{\pgfqpoint{0.854460in}{0.571603in}}{\pgfqpoint{5.885100in}{5.225635in}}%
\pgfusepath{clip}%
\pgfsetbuttcap%
\pgfsetroundjoin%
\pgfsetlinewidth{1.505625pt}%
\definecolor{currentstroke}{rgb}{0.162142,0.474838,0.558140}%
\pgfsetstrokecolor{currentstroke}%
\pgfsetdash{}{0pt}%
\pgfpathmoveto{\pgfqpoint{6.391998in}{1.412793in}}%
\pgfpathlineto{\pgfqpoint{6.392778in}{1.411906in}}%
\pgfpathlineto{\pgfqpoint{6.414253in}{1.387822in}}%
\pgfpathlineto{\pgfqpoint{6.416190in}{1.385647in}}%
\pgfpathlineto{\pgfqpoint{6.439834in}{1.359388in}}%
\pgfpathlineto{\pgfqpoint{6.443827in}{1.355005in}}%
\pgfpathlineto{\pgfqpoint{6.463730in}{1.333128in}}%
\pgfpathlineto{\pgfqpoint{6.473400in}{1.322626in}}%
\pgfpathlineto{\pgfqpoint{6.487894in}{1.306869in}}%
\pgfpathlineto{\pgfqpoint{6.502973in}{1.290665in}}%
\pgfpathlineto{\pgfqpoint{6.512323in}{1.280609in}}%
\pgfpathlineto{\pgfqpoint{6.532547in}{1.259106in}}%
\pgfpathlineto{\pgfqpoint{6.537016in}{1.254350in}}%
\pgfpathlineto{\pgfqpoint{6.561971in}{1.228090in}}%
\pgfpathlineto{\pgfqpoint{6.562120in}{1.227934in}}%
\pgfpathlineto{\pgfqpoint{6.587147in}{1.201831in}}%
\pgfpathlineto{\pgfqpoint{6.591693in}{1.197139in}}%
\pgfpathlineto{\pgfqpoint{6.612585in}{1.175571in}}%
\pgfpathlineto{\pgfqpoint{6.621267in}{1.166702in}}%
\pgfpathlineto{\pgfqpoint{6.638283in}{1.149312in}}%
\pgfpathlineto{\pgfqpoint{6.650840in}{1.136610in}}%
\pgfpathlineto{\pgfqpoint{6.664240in}{1.123052in}}%
\pgfpathlineto{\pgfqpoint{6.680414in}{1.106851in}}%
\pgfpathlineto{\pgfqpoint{6.690454in}{1.096793in}}%
\pgfpathlineto{\pgfqpoint{6.709987in}{1.077415in}}%
\pgfpathlineto{\pgfqpoint{6.716924in}{1.070533in}}%
\pgfpathlineto{\pgfqpoint{6.739560in}{1.048290in}}%
\pgfusepath{stroke}%
\end{pgfscope}%
\begin{pgfscope}%
\pgfpathrectangle{\pgfqpoint{0.854460in}{0.571603in}}{\pgfqpoint{5.885100in}{5.225635in}}%
\pgfusepath{clip}%
\pgfsetbuttcap%
\pgfsetroundjoin%
\pgfsetlinewidth{1.505625pt}%
\definecolor{currentstroke}{rgb}{0.154815,0.493313,0.557840}%
\pgfsetstrokecolor{currentstroke}%
\pgfsetdash{}{0pt}%
\pgfpathmoveto{\pgfqpoint{1.475048in}{0.571603in}}%
\pgfpathlineto{\pgfqpoint{1.446453in}{0.597863in}}%
\pgfpathlineto{\pgfqpoint{1.445928in}{0.598352in}}%
\pgfpathlineto{\pgfqpoint{1.435295in}{0.608307in}}%
\pgfusepath{stroke}%
\end{pgfscope}%
\begin{pgfscope}%
\pgfpathrectangle{\pgfqpoint{0.854460in}{0.571603in}}{\pgfqpoint{5.885100in}{5.225635in}}%
\pgfusepath{clip}%
\pgfsetbuttcap%
\pgfsetroundjoin%
\pgfsetlinewidth{1.505625pt}%
\definecolor{currentstroke}{rgb}{0.154815,0.493313,0.557840}%
\pgfsetstrokecolor{currentstroke}%
\pgfsetdash{}{0pt}%
\pgfpathmoveto{\pgfqpoint{1.167670in}{0.884705in}}%
\pgfpathlineto{\pgfqpoint{1.165909in}{0.886717in}}%
\pgfpathlineto{\pgfqpoint{1.150194in}{0.904939in}}%
\pgfpathlineto{\pgfqpoint{1.143308in}{0.912976in}}%
\pgfpathlineto{\pgfqpoint{1.121147in}{0.939236in}}%
\pgfpathlineto{\pgfqpoint{1.120621in}{0.939871in}}%
\pgfpathlineto{\pgfqpoint{1.099515in}{0.965495in}}%
\pgfpathlineto{\pgfqpoint{1.091047in}{0.975932in}}%
\pgfpathlineto{\pgfqpoint{1.078301in}{0.991755in}}%
\pgfpathlineto{\pgfqpoint{1.061474in}{1.012961in}}%
\pgfpathlineto{\pgfqpoint{1.057493in}{1.018014in}}%
\pgfpathlineto{\pgfqpoint{1.037156in}{1.044274in}}%
\pgfpathlineto{\pgfqpoint{1.031901in}{1.051175in}}%
\pgfpathlineto{\pgfqpoint{1.017268in}{1.070533in}}%
\pgfpathlineto{\pgfqpoint{1.002327in}{1.090606in}}%
\pgfpathlineto{\pgfqpoint{0.997757in}{1.096793in}}%
\pgfpathlineto{\pgfqpoint{0.978700in}{1.123052in}}%
\pgfpathlineto{\pgfqpoint{0.972754in}{1.131388in}}%
\pgfpathlineto{\pgfqpoint{0.960071in}{1.149312in}}%
\pgfpathlineto{\pgfqpoint{0.943181in}{1.173558in}}%
\pgfpathlineto{\pgfqpoint{0.941790in}{1.175571in}}%
\pgfpathlineto{\pgfqpoint{0.923998in}{1.201831in}}%
\pgfpathlineto{\pgfqpoint{0.913607in}{1.217421in}}%
\pgfpathlineto{\pgfqpoint{0.906556in}{1.228090in}}%
\pgfpathlineto{\pgfqpoint{0.889513in}{1.254350in}}%
\pgfpathlineto{\pgfqpoint{0.884034in}{1.262952in}}%
\pgfpathlineto{\pgfqpoint{0.872886in}{1.280609in}}%
\pgfpathlineto{\pgfqpoint{0.856588in}{1.306869in}}%
\pgfpathlineto{\pgfqpoint{0.854460in}{1.310369in}}%
\pgfusepath{stroke}%
\end{pgfscope}%
\begin{pgfscope}%
\pgfpathrectangle{\pgfqpoint{0.854460in}{0.571603in}}{\pgfqpoint{5.885100in}{5.225635in}}%
\pgfusepath{clip}%
\pgfsetbuttcap%
\pgfsetroundjoin%
\pgfsetlinewidth{1.505625pt}%
\definecolor{currentstroke}{rgb}{0.154815,0.493313,0.557840}%
\pgfsetstrokecolor{currentstroke}%
\pgfsetdash{}{0pt}%
\pgfpathmoveto{\pgfqpoint{0.854460in}{4.185177in}}%
\pgfpathlineto{\pgfqpoint{0.860665in}{4.195411in}}%
\pgfpathlineto{\pgfqpoint{0.876858in}{4.221670in}}%
\pgfpathlineto{\pgfqpoint{0.884034in}{4.233116in}}%
\pgfpathlineto{\pgfqpoint{0.893449in}{4.247930in}}%
\pgfpathlineto{\pgfqpoint{0.910402in}{4.274189in}}%
\pgfpathlineto{\pgfqpoint{0.913607in}{4.279070in}}%
\pgfpathlineto{\pgfqpoint{0.927836in}{4.300449in}}%
\pgfpathlineto{\pgfqpoint{0.943181in}{4.323174in}}%
\pgfpathlineto{\pgfqpoint{0.945599in}{4.326708in}}%
\pgfpathlineto{\pgfqpoint{0.963867in}{4.352967in}}%
\pgfpathlineto{\pgfqpoint{0.972754in}{4.365556in}}%
\pgfpathlineto{\pgfqpoint{0.982533in}{4.379227in}}%
\pgfpathlineto{\pgfqpoint{1.001583in}{4.405486in}}%
\pgfpathlineto{\pgfqpoint{1.002327in}{4.406495in}}%
\pgfpathlineto{\pgfqpoint{1.021188in}{4.431746in}}%
\pgfpathlineto{\pgfqpoint{1.031901in}{4.445894in}}%
\pgfpathlineto{\pgfqpoint{1.041189in}{4.458005in}}%
\pgfpathlineto{\pgfqpoint{1.061474in}{4.484101in}}%
\pgfpathlineto{\pgfqpoint{1.061603in}{4.484265in}}%
\pgfpathlineto{\pgfqpoint{1.082600in}{4.510524in}}%
\pgfpathlineto{\pgfqpoint{1.091047in}{4.520950in}}%
\pgfpathlineto{\pgfqpoint{1.104038in}{4.536784in}}%
\pgfpathlineto{\pgfqpoint{1.120621in}{4.556733in}}%
\pgfpathlineto{\pgfqpoint{1.125931in}{4.563043in}}%
\pgfpathlineto{\pgfqpoint{1.148330in}{4.589303in}}%
\pgfpathlineto{\pgfqpoint{1.150194in}{4.591456in}}%
\pgfpathlineto{\pgfqpoint{1.171305in}{4.615562in}}%
\pgfpathlineto{\pgfqpoint{1.179767in}{4.625101in}}%
\pgfpathlineto{\pgfqpoint{1.194778in}{4.641822in}}%
\pgfpathlineto{\pgfqpoint{1.209341in}{4.657838in}}%
\pgfpathlineto{\pgfqpoint{1.218765in}{4.668081in}}%
\pgfpathlineto{\pgfqpoint{1.238914in}{4.689705in}}%
\pgfpathlineto{\pgfqpoint{1.243284in}{4.694341in}}%
\pgfpathlineto{\pgfqpoint{1.268354in}{4.720600in}}%
\pgfpathlineto{\pgfqpoint{1.268488in}{4.720739in}}%
\pgfpathlineto{\pgfqpoint{1.294058in}{4.746860in}}%
\pgfpathlineto{\pgfqpoint{1.298061in}{4.750898in}}%
\pgfpathlineto{\pgfqpoint{1.320336in}{4.773119in}}%
\pgfpathlineto{\pgfqpoint{1.327634in}{4.780310in}}%
\pgfpathlineto{\pgfqpoint{1.347203in}{4.799378in}}%
\pgfpathlineto{\pgfqpoint{1.357208in}{4.809007in}}%
\pgfpathlineto{\pgfqpoint{1.374677in}{4.825638in}}%
\pgfpathlineto{\pgfqpoint{1.386781in}{4.837019in}}%
\pgfpathlineto{\pgfqpoint{1.402775in}{4.851897in}}%
\pgfpathlineto{\pgfqpoint{1.416354in}{4.864375in}}%
\pgfpathlineto{\pgfqpoint{1.431514in}{4.878157in}}%
\pgfpathlineto{\pgfqpoint{1.445928in}{4.891101in}}%
\pgfpathlineto{\pgfqpoint{1.460910in}{4.904416in}}%
\pgfpathlineto{\pgfqpoint{1.475501in}{4.917226in}}%
\pgfpathlineto{\pgfqpoint{1.490981in}{4.930676in}}%
\pgfpathlineto{\pgfqpoint{1.503351in}{4.941294in}}%
\pgfusepath{stroke}%
\end{pgfscope}%
\begin{pgfscope}%
\pgfpathrectangle{\pgfqpoint{0.854460in}{0.571603in}}{\pgfqpoint{5.885100in}{5.225635in}}%
\pgfusepath{clip}%
\pgfsetbuttcap%
\pgfsetroundjoin%
\pgfsetlinewidth{1.505625pt}%
\definecolor{currentstroke}{rgb}{0.154815,0.493313,0.557840}%
\pgfsetstrokecolor{currentstroke}%
\pgfsetdash{}{0pt}%
\pgfpathmoveto{\pgfqpoint{1.806253in}{5.174806in}}%
\pgfpathlineto{\pgfqpoint{1.830381in}{5.191519in}}%
\pgfpathlineto{\pgfqpoint{1.832936in}{5.193271in}}%
\pgfpathlineto{\pgfqpoint{1.859955in}{5.211571in}}%
\pgfpathlineto{\pgfqpoint{1.871805in}{5.219530in}}%
\pgfpathlineto{\pgfqpoint{1.889528in}{5.231291in}}%
\pgfpathlineto{\pgfqpoint{1.911558in}{5.245790in}}%
\pgfpathlineto{\pgfqpoint{1.919102in}{5.250694in}}%
\pgfpathlineto{\pgfqpoint{1.948675in}{5.269750in}}%
\pgfpathlineto{\pgfqpoint{1.952279in}{5.272049in}}%
\pgfpathlineto{\pgfqpoint{1.978248in}{5.288412in}}%
\pgfpathlineto{\pgfqpoint{1.994077in}{5.298308in}}%
\pgfpathlineto{\pgfqpoint{2.007822in}{5.306798in}}%
\pgfpathlineto{\pgfqpoint{2.036811in}{5.324568in}}%
\pgfpathlineto{\pgfqpoint{2.037395in}{5.324922in}}%
\pgfpathlineto{\pgfqpoint{2.066968in}{5.342629in}}%
\pgfpathlineto{\pgfqpoint{2.080765in}{5.350827in}}%
\pgfpathlineto{\pgfqpoint{2.096542in}{5.360089in}}%
\pgfpathlineto{\pgfqpoint{2.125706in}{5.377087in}}%
\pgfpathlineto{\pgfqpoint{2.126115in}{5.377322in}}%
\pgfpathlineto{\pgfqpoint{2.155689in}{5.394153in}}%
\pgfpathlineto{\pgfqpoint{2.171957in}{5.403346in}}%
\pgfpathlineto{\pgfqpoint{2.185262in}{5.410773in}}%
\pgfpathlineto{\pgfqpoint{2.214835in}{5.427150in}}%
\pgfpathlineto{\pgfqpoint{2.219317in}{5.429606in}}%
\pgfpathlineto{\pgfqpoint{2.244409in}{5.443185in}}%
\pgfpathlineto{\pgfqpoint{2.267990in}{5.455865in}}%
\pgfpathlineto{\pgfqpoint{2.273982in}{5.459048in}}%
\pgfpathlineto{\pgfqpoint{2.303555in}{5.474591in}}%
\pgfpathlineto{\pgfqpoint{2.318003in}{5.482125in}}%
\pgfpathlineto{\pgfqpoint{2.333129in}{5.489914in}}%
\pgfpathlineto{\pgfqpoint{2.362702in}{5.505024in}}%
\pgfpathlineto{\pgfqpoint{2.369348in}{5.508384in}}%
\pgfpathlineto{\pgfqpoint{2.392275in}{5.519833in}}%
\pgfpathlineto{\pgfqpoint{2.421849in}{5.534515in}}%
\pgfpathlineto{\pgfqpoint{2.422111in}{5.534644in}}%
\pgfpathlineto{\pgfqpoint{2.451422in}{5.548834in}}%
\pgfpathlineto{\pgfqpoint{2.476487in}{5.560903in}}%
\pgfpathlineto{\pgfqpoint{2.480996in}{5.563047in}}%
\pgfpathlineto{\pgfqpoint{2.510569in}{5.576946in}}%
\pgfpathlineto{\pgfqpoint{2.532447in}{5.587163in}}%
\pgfpathlineto{\pgfqpoint{2.540142in}{5.590711in}}%
\pgfpathlineto{\pgfqpoint{2.569716in}{5.604199in}}%
\pgfpathlineto{\pgfqpoint{2.590084in}{5.613422in}}%
\pgfpathlineto{\pgfqpoint{2.599289in}{5.617538in}}%
\pgfpathlineto{\pgfqpoint{2.628862in}{5.630619in}}%
\pgfpathlineto{\pgfqpoint{2.649502in}{5.639682in}}%
\pgfpathlineto{\pgfqpoint{2.658436in}{5.643554in}}%
\pgfpathlineto{\pgfqpoint{2.688009in}{5.656233in}}%
\pgfpathlineto{\pgfqpoint{2.710812in}{5.665941in}}%
\pgfpathlineto{\pgfqpoint{2.717582in}{5.668787in}}%
\pgfpathlineto{\pgfqpoint{2.747156in}{5.681067in}}%
\pgfpathlineto{\pgfqpoint{2.774130in}{5.692201in}}%
\pgfpathlineto{\pgfqpoint{2.776729in}{5.693260in}}%
\pgfpathlineto{\pgfqpoint{2.806303in}{5.705145in}}%
\pgfpathlineto{\pgfqpoint{2.835876in}{5.716965in}}%
\pgfpathlineto{\pgfqpoint{2.839667in}{5.718460in}}%
\pgfpathlineto{\pgfqpoint{2.865449in}{5.728493in}}%
\pgfpathlineto{\pgfqpoint{2.895023in}{5.739920in}}%
\pgfpathlineto{\pgfqpoint{2.907587in}{5.744720in}}%
\pgfpathlineto{\pgfqpoint{2.924596in}{5.751133in}}%
\pgfpathlineto{\pgfqpoint{2.954169in}{5.762170in}}%
\pgfpathlineto{\pgfqpoint{2.977966in}{5.770979in}}%
\pgfpathlineto{\pgfqpoint{2.983743in}{5.773089in}}%
\pgfpathlineto{\pgfqpoint{3.013316in}{5.783740in}}%
\pgfpathlineto{\pgfqpoint{3.042890in}{5.794318in}}%
\pgfpathlineto{\pgfqpoint{3.051169in}{5.797238in}}%
\pgfusepath{stroke}%
\end{pgfscope}%
\begin{pgfscope}%
\pgfpathrectangle{\pgfqpoint{0.854460in}{0.571603in}}{\pgfqpoint{5.885100in}{5.225635in}}%
\pgfusepath{clip}%
\pgfsetbuttcap%
\pgfsetroundjoin%
\pgfsetlinewidth{1.505625pt}%
\definecolor{currentstroke}{rgb}{0.154815,0.493313,0.557840}%
\pgfsetstrokecolor{currentstroke}%
\pgfsetdash{}{0pt}%
\pgfpathmoveto{\pgfqpoint{5.530714in}{5.797238in}}%
\pgfpathlineto{\pgfqpoint{5.545001in}{5.770979in}}%
\pgfpathlineto{\pgfqpoint{5.556626in}{5.746573in}}%
\pgfpathlineto{\pgfqpoint{5.557469in}{5.744720in}}%
\pgfpathlineto{\pgfqpoint{5.568107in}{5.718460in}}%
\pgfpathlineto{\pgfqpoint{5.577306in}{5.692201in}}%
\pgfpathlineto{\pgfqpoint{5.585196in}{5.665941in}}%
\pgfpathlineto{\pgfqpoint{5.586199in}{5.662081in}}%
\pgfpathlineto{\pgfqpoint{5.591790in}{5.639682in}}%
\pgfpathlineto{\pgfqpoint{5.597305in}{5.613422in}}%
\pgfpathlineto{\pgfqpoint{5.601851in}{5.587163in}}%
\pgfpathlineto{\pgfqpoint{5.605513in}{5.560903in}}%
\pgfpathlineto{\pgfqpoint{5.608367in}{5.534644in}}%
\pgfpathlineto{\pgfqpoint{5.610481in}{5.508384in}}%
\pgfpathlineto{\pgfqpoint{5.611920in}{5.482125in}}%
\pgfpathlineto{\pgfqpoint{5.612741in}{5.455865in}}%
\pgfpathlineto{\pgfqpoint{5.612997in}{5.429606in}}%
\pgfpathlineto{\pgfqpoint{5.612738in}{5.403346in}}%
\pgfpathlineto{\pgfqpoint{5.612008in}{5.377087in}}%
\pgfpathlineto{\pgfqpoint{5.610849in}{5.350827in}}%
\pgfpathlineto{\pgfqpoint{5.609298in}{5.324568in}}%
\pgfpathlineto{\pgfqpoint{5.607392in}{5.298308in}}%
\pgfpathlineto{\pgfqpoint{5.605162in}{5.272049in}}%
\pgfpathlineto{\pgfqpoint{5.602640in}{5.245790in}}%
\pgfpathlineto{\pgfqpoint{5.599853in}{5.219530in}}%
\pgfpathlineto{\pgfqpoint{5.596829in}{5.193271in}}%
\pgfpathlineto{\pgfqpoint{5.593590in}{5.167011in}}%
\pgfpathlineto{\pgfqpoint{5.590161in}{5.140752in}}%
\pgfpathlineto{\pgfqpoint{5.587949in}{5.124606in}}%
\pgfusepath{stroke}%
\end{pgfscope}%
\begin{pgfscope}%
\pgfpathrectangle{\pgfqpoint{0.854460in}{0.571603in}}{\pgfqpoint{5.885100in}{5.225635in}}%
\pgfusepath{clip}%
\pgfsetbuttcap%
\pgfsetroundjoin%
\pgfsetlinewidth{1.505625pt}%
\definecolor{currentstroke}{rgb}{0.154815,0.493313,0.557840}%
\pgfsetstrokecolor{currentstroke}%
\pgfsetdash{}{0pt}%
\pgfpathmoveto{\pgfqpoint{5.524927in}{4.737221in}}%
\pgfpathlineto{\pgfqpoint{5.497479in}{4.568906in}}%
\pgfpathlineto{\pgfqpoint{5.481014in}{4.458005in}}%
\pgfpathlineto{\pgfqpoint{5.467239in}{4.352967in}}%
\pgfpathlineto{\pgfqpoint{5.455432in}{4.247930in}}%
\pgfpathlineto{\pgfqpoint{5.446008in}{4.142892in}}%
\pgfpathlineto{\pgfqpoint{5.439200in}{4.037854in}}%
\pgfpathlineto{\pgfqpoint{5.435181in}{3.932816in}}%
\pgfpathlineto{\pgfqpoint{5.434144in}{3.854037in}}%
\pgfpathlineto{\pgfqpoint{5.434891in}{3.775259in}}%
\pgfpathlineto{\pgfqpoint{5.438332in}{3.679404in}}%
\pgfpathlineto{\pgfqpoint{5.441977in}{3.617702in}}%
\pgfpathlineto{\pgfqpoint{5.448421in}{3.538924in}}%
\pgfpathlineto{\pgfqpoint{5.456897in}{3.460145in}}%
\pgfpathlineto{\pgfqpoint{5.467906in}{3.378514in}}%
\pgfpathlineto{\pgfqpoint{5.480084in}{3.302589in}}%
\pgfpathlineto{\pgfqpoint{5.497479in}{3.211361in}}%
\pgfpathlineto{\pgfqpoint{5.511847in}{3.145032in}}%
\pgfpathlineto{\pgfqpoint{5.531050in}{3.066253in}}%
\pgfpathlineto{\pgfqpoint{5.552474in}{2.987475in}}%
\pgfpathlineto{\pgfqpoint{5.576151in}{2.908696in}}%
\pgfpathlineto{\pgfqpoint{5.602127in}{2.829918in}}%
\pgfpathlineto{\pgfqpoint{5.630416in}{2.751140in}}%
\pgfpathlineto{\pgfqpoint{5.661033in}{2.672361in}}%
\pgfpathlineto{\pgfqpoint{5.693995in}{2.593583in}}%
\pgfpathlineto{\pgfqpoint{5.734066in}{2.504710in}}%
\pgfpathlineto{\pgfqpoint{5.767010in}{2.436026in}}%
\pgfpathlineto{\pgfqpoint{5.807049in}{2.357248in}}%
\pgfpathlineto{\pgfqpoint{5.852359in}{2.273394in}}%
\pgfpathlineto{\pgfqpoint{5.894326in}{2.199691in}}%
\pgfpathlineto{\pgfqpoint{5.941584in}{2.120912in}}%
\pgfpathlineto{\pgfqpoint{5.991182in}{2.042134in}}%
\pgfpathlineto{\pgfqpoint{6.043199in}{1.963355in}}%
\pgfpathlineto{\pgfqpoint{6.097624in}{1.884577in}}%
\pgfpathlineto{\pgfqpoint{6.154445in}{1.805799in}}%
\pgfpathlineto{\pgfqpoint{6.213653in}{1.727020in}}%
\pgfpathlineto{\pgfqpoint{6.275244in}{1.648242in}}%
\pgfpathlineto{\pgfqpoint{6.355107in}{1.550394in}}%
\pgfpathlineto{\pgfqpoint{6.414253in}{1.480634in}}%
\pgfpathlineto{\pgfqpoint{6.474325in}{1.411906in}}%
\pgfpathlineto{\pgfqpoint{6.562120in}{1.314986in}}%
\pgfpathlineto{\pgfqpoint{6.621267in}{1.251791in}}%
\pgfpathlineto{\pgfqpoint{6.709987in}{1.159871in}}%
\pgfpathlineto{\pgfqpoint{6.739560in}{1.129938in}}%
\pgfpathlineto{\pgfqpoint{6.739560in}{1.129938in}}%
\pgfusepath{stroke}%
\end{pgfscope}%
\begin{pgfscope}%
\pgfpathrectangle{\pgfqpoint{0.854460in}{0.571603in}}{\pgfqpoint{5.885100in}{5.225635in}}%
\pgfusepath{clip}%
\pgfsetbuttcap%
\pgfsetroundjoin%
\pgfsetlinewidth{1.505625pt}%
\definecolor{currentstroke}{rgb}{0.146180,0.515413,0.556823}%
\pgfsetstrokecolor{currentstroke}%
\pgfsetdash{}{0pt}%
\pgfpathmoveto{\pgfqpoint{1.408457in}{0.571603in}}%
\pgfpathlineto{\pgfqpoint{1.386781in}{0.591677in}}%
\pgfpathlineto{\pgfqpoint{1.380134in}{0.597863in}}%
\pgfpathlineto{\pgfqpoint{1.357208in}{0.619503in}}%
\pgfpathlineto{\pgfqpoint{1.352338in}{0.624122in}}%
\pgfpathlineto{\pgfqpoint{1.327634in}{0.647893in}}%
\pgfpathlineto{\pgfqpoint{1.325061in}{0.650382in}}%
\pgfpathlineto{\pgfqpoint{1.298298in}{0.676641in}}%
\pgfpathlineto{\pgfqpoint{1.298061in}{0.676878in}}%
\pgfpathlineto{\pgfqpoint{1.272081in}{0.702901in}}%
\pgfpathlineto{\pgfqpoint{1.268488in}{0.706552in}}%
\pgfpathlineto{\pgfqpoint{1.246364in}{0.729160in}}%
\pgfpathlineto{\pgfqpoint{1.238914in}{0.736884in}}%
\pgfpathlineto{\pgfqpoint{1.221138in}{0.755420in}}%
\pgfpathlineto{\pgfqpoint{1.209341in}{0.767901in}}%
\pgfpathlineto{\pgfqpoint{1.196395in}{0.781679in}}%
\pgfpathlineto{\pgfqpoint{1.179767in}{0.799634in}}%
\pgfpathlineto{\pgfqpoint{1.172124in}{0.807939in}}%
\pgfpathlineto{\pgfqpoint{1.150194in}{0.832114in}}%
\pgfpathlineto{\pgfqpoint{1.148315in}{0.834198in}}%
\pgfpathlineto{\pgfqpoint{1.125020in}{0.860458in}}%
\pgfpathlineto{\pgfqpoint{1.120621in}{0.865494in}}%
\pgfpathlineto{\pgfqpoint{1.102201in}{0.886717in}}%
\pgfpathlineto{\pgfqpoint{1.091047in}{0.899760in}}%
\pgfpathlineto{\pgfqpoint{1.079820in}{0.912976in}}%
\pgfpathlineto{\pgfqpoint{1.061474in}{0.934896in}}%
\pgfpathlineto{\pgfqpoint{1.057866in}{0.939236in}}%
\pgfpathlineto{\pgfqpoint{1.053968in}{0.944003in}}%
\pgfusepath{stroke}%
\end{pgfscope}%
\begin{pgfscope}%
\pgfpathrectangle{\pgfqpoint{0.854460in}{0.571603in}}{\pgfqpoint{5.885100in}{5.225635in}}%
\pgfusepath{clip}%
\pgfsetbuttcap%
\pgfsetroundjoin%
\pgfsetlinewidth{1.505625pt}%
\definecolor{currentstroke}{rgb}{0.146180,0.515413,0.556823}%
\pgfsetstrokecolor{currentstroke}%
\pgfsetdash{}{0pt}%
\pgfpathmoveto{\pgfqpoint{0.854460in}{4.300944in}}%
\pgfpathlineto{\pgfqpoint{0.871553in}{4.326708in}}%
\pgfpathlineto{\pgfqpoint{0.884034in}{4.345251in}}%
\pgfpathlineto{\pgfqpoint{0.889296in}{4.352967in}}%
\pgfpathlineto{\pgfqpoint{0.907487in}{4.379227in}}%
\pgfpathlineto{\pgfqpoint{0.913607in}{4.387926in}}%
\pgfpathlineto{\pgfqpoint{0.926120in}{4.405486in}}%
\pgfpathlineto{\pgfqpoint{0.943181in}{4.429099in}}%
\pgfpathlineto{\pgfqpoint{0.945118in}{4.431746in}}%
\pgfpathlineto{\pgfqpoint{0.964640in}{4.458005in}}%
\pgfpathlineto{\pgfqpoint{0.972754in}{4.468766in}}%
\pgfpathlineto{\pgfqpoint{0.984587in}{4.484265in}}%
\pgfpathlineto{\pgfqpoint{1.002327in}{4.507190in}}%
\pgfpathlineto{\pgfqpoint{1.004940in}{4.510524in}}%
\pgfpathlineto{\pgfqpoint{1.025823in}{4.536784in}}%
\pgfpathlineto{\pgfqpoint{1.031901in}{4.544321in}}%
\pgfpathlineto{\pgfqpoint{1.047183in}{4.563043in}}%
\pgfpathlineto{\pgfqpoint{1.061474in}{4.580323in}}%
\pgfpathlineto{\pgfqpoint{1.068990in}{4.589303in}}%
\pgfpathlineto{\pgfqpoint{1.091047in}{4.615314in}}%
\pgfpathlineto{\pgfqpoint{1.091261in}{4.615562in}}%
\pgfpathlineto{\pgfqpoint{1.114127in}{4.641822in}}%
\pgfpathlineto{\pgfqpoint{1.120621in}{4.649183in}}%
\pgfpathlineto{\pgfqpoint{1.137486in}{4.668081in}}%
\pgfpathlineto{\pgfqpoint{1.150194in}{4.682140in}}%
\pgfpathlineto{\pgfqpoint{1.161349in}{4.694341in}}%
\pgfpathlineto{\pgfqpoint{1.179767in}{4.714230in}}%
\pgfpathlineto{\pgfqpoint{1.180814in}{4.715348in}}%
\pgfusepath{stroke}%
\end{pgfscope}%
\begin{pgfscope}%
\pgfpathrectangle{\pgfqpoint{0.854460in}{0.571603in}}{\pgfqpoint{5.885100in}{5.225635in}}%
\pgfusepath{clip}%
\pgfsetbuttcap%
\pgfsetroundjoin%
\pgfsetlinewidth{1.505625pt}%
\definecolor{currentstroke}{rgb}{0.146180,0.515413,0.556823}%
\pgfsetstrokecolor{currentstroke}%
\pgfsetdash{}{0pt}%
\pgfpathmoveto{\pgfqpoint{1.458807in}{4.980346in}}%
\pgfpathlineto{\pgfqpoint{1.462130in}{4.983195in}}%
\pgfpathlineto{\pgfqpoint{1.475501in}{4.994518in}}%
\pgfpathlineto{\pgfqpoint{1.493312in}{5.009454in}}%
\pgfpathlineto{\pgfqpoint{1.505074in}{5.019200in}}%
\pgfpathlineto{\pgfqpoint{1.525198in}{5.035714in}}%
\pgfpathlineto{\pgfqpoint{1.534648in}{5.043374in}}%
\pgfpathlineto{\pgfqpoint{1.557806in}{5.061973in}}%
\pgfpathlineto{\pgfqpoint{1.564221in}{5.067064in}}%
\pgfpathlineto{\pgfqpoint{1.591148in}{5.088233in}}%
\pgfpathlineto{\pgfqpoint{1.593795in}{5.090288in}}%
\pgfpathlineto{\pgfqpoint{1.623368in}{5.113039in}}%
\pgfpathlineto{\pgfqpoint{1.625276in}{5.114492in}}%
\pgfpathlineto{\pgfqpoint{1.652941in}{5.135314in}}%
\pgfpathlineto{\pgfqpoint{1.660230in}{5.140752in}}%
\pgfpathlineto{\pgfqpoint{1.682515in}{5.157175in}}%
\pgfpathlineto{\pgfqpoint{1.695978in}{5.167011in}}%
\pgfpathlineto{\pgfqpoint{1.712088in}{5.178640in}}%
\pgfpathlineto{\pgfqpoint{1.732531in}{5.193271in}}%
\pgfpathlineto{\pgfqpoint{1.741661in}{5.199726in}}%
\pgfpathlineto{\pgfqpoint{1.769903in}{5.219530in}}%
\pgfpathlineto{\pgfqpoint{1.771235in}{5.220452in}}%
\pgfpathlineto{\pgfqpoint{1.800808in}{5.240733in}}%
\pgfpathlineto{\pgfqpoint{1.808245in}{5.245790in}}%
\pgfpathlineto{\pgfqpoint{1.830381in}{5.260660in}}%
\pgfpathlineto{\pgfqpoint{1.847472in}{5.272049in}}%
\pgfpathlineto{\pgfqpoint{1.859955in}{5.280268in}}%
\pgfpathlineto{\pgfqpoint{1.887569in}{5.298308in}}%
\pgfpathlineto{\pgfqpoint{1.889528in}{5.299573in}}%
\pgfpathlineto{\pgfqpoint{1.919102in}{5.318467in}}%
\pgfpathlineto{\pgfqpoint{1.928728in}{5.324568in}}%
\pgfpathlineto{\pgfqpoint{1.948675in}{5.337056in}}%
\pgfpathlineto{\pgfqpoint{1.970834in}{5.350827in}}%
\pgfpathlineto{\pgfqpoint{1.978248in}{5.355380in}}%
\pgfpathlineto{\pgfqpoint{2.007822in}{5.373377in}}%
\pgfpathlineto{\pgfqpoint{2.013974in}{5.377087in}}%
\pgfpathlineto{\pgfqpoint{2.037395in}{5.391037in}}%
\pgfpathlineto{\pgfqpoint{2.058203in}{5.403346in}}%
\pgfpathlineto{\pgfqpoint{2.066968in}{5.408468in}}%
\pgfpathlineto{\pgfqpoint{2.096542in}{5.425601in}}%
\pgfpathlineto{\pgfqpoint{2.103518in}{5.429606in}}%
\pgfpathlineto{\pgfqpoint{2.126115in}{5.442419in}}%
\pgfpathlineto{\pgfqpoint{2.149982in}{5.455865in}}%
\pgfpathlineto{\pgfqpoint{2.155689in}{5.459041in}}%
\pgfpathlineto{\pgfqpoint{2.185262in}{5.475339in}}%
\pgfpathlineto{\pgfqpoint{2.197671in}{5.482125in}}%
\pgfpathlineto{\pgfqpoint{2.214835in}{5.491395in}}%
\pgfpathlineto{\pgfqpoint{2.244409in}{5.507263in}}%
\pgfpathlineto{\pgfqpoint{2.246521in}{5.508384in}}%
\pgfpathlineto{\pgfqpoint{2.273982in}{5.522779in}}%
\pgfpathlineto{\pgfqpoint{2.296747in}{5.534644in}}%
\pgfpathlineto{\pgfqpoint{2.303555in}{5.538148in}}%
\pgfpathlineto{\pgfqpoint{2.333129in}{5.553223in}}%
\pgfpathlineto{\pgfqpoint{2.348308in}{5.560903in}}%
\pgfpathlineto{\pgfqpoint{2.362702in}{5.568095in}}%
\pgfpathlineto{\pgfqpoint{2.392275in}{5.582759in}}%
\pgfpathlineto{\pgfqpoint{2.401242in}{5.587163in}}%
\pgfpathlineto{\pgfqpoint{2.421849in}{5.597158in}}%
\pgfpathlineto{\pgfqpoint{2.451422in}{5.611415in}}%
\pgfpathlineto{\pgfqpoint{2.455632in}{5.613422in}}%
\pgfpathlineto{\pgfqpoint{2.480996in}{5.625364in}}%
\pgfpathlineto{\pgfqpoint{2.510569in}{5.639220in}}%
\pgfpathlineto{\pgfqpoint{2.511566in}{5.639682in}}%
\pgfpathlineto{\pgfqpoint{2.540142in}{5.652742in}}%
\pgfpathlineto{\pgfqpoint{2.569147in}{5.665941in}}%
\pgfpathlineto{\pgfqpoint{2.569716in}{5.666197in}}%
\pgfpathlineto{\pgfqpoint{2.599289in}{5.679318in}}%
\pgfpathlineto{\pgfqpoint{2.628444in}{5.692201in}}%
\pgfpathlineto{\pgfqpoint{2.628862in}{5.692383in}}%
\pgfpathlineto{\pgfqpoint{2.658436in}{5.705119in}}%
\pgfpathlineto{\pgfqpoint{2.688009in}{5.717801in}}%
\pgfpathlineto{\pgfqpoint{2.689565in}{5.718460in}}%
\pgfpathlineto{\pgfqpoint{2.717582in}{5.730170in}}%
\pgfpathlineto{\pgfqpoint{2.747156in}{5.742469in}}%
\pgfpathlineto{\pgfqpoint{2.752632in}{5.744720in}}%
\pgfpathlineto{\pgfqpoint{2.776729in}{5.754495in}}%
\pgfpathlineto{\pgfqpoint{2.806303in}{5.766415in}}%
\pgfpathlineto{\pgfqpoint{2.817744in}{5.770979in}}%
\pgfpathlineto{\pgfqpoint{2.835876in}{5.778118in}}%
\pgfpathlineto{\pgfqpoint{2.865449in}{5.789663in}}%
\pgfpathlineto{\pgfqpoint{2.885014in}{5.797238in}}%
\pgfusepath{stroke}%
\end{pgfscope}%
\begin{pgfscope}%
\pgfpathrectangle{\pgfqpoint{0.854460in}{0.571603in}}{\pgfqpoint{5.885100in}{5.225635in}}%
\pgfusepath{clip}%
\pgfsetbuttcap%
\pgfsetroundjoin%
\pgfsetlinewidth{1.505625pt}%
\definecolor{currentstroke}{rgb}{0.146180,0.515413,0.556823}%
\pgfsetstrokecolor{currentstroke}%
\pgfsetdash{}{0pt}%
\pgfpathmoveto{\pgfqpoint{5.712248in}{5.797238in}}%
\pgfpathlineto{\pgfqpoint{5.720805in}{5.770979in}}%
\pgfpathlineto{\pgfqpoint{5.728054in}{5.744720in}}%
\pgfpathlineto{\pgfqpoint{5.734110in}{5.718460in}}%
\pgfpathlineto{\pgfqpoint{5.742908in}{5.665941in}}%
\pgfpathlineto{\pgfqpoint{5.748150in}{5.613422in}}%
\pgfpathlineto{\pgfqpoint{5.750421in}{5.560903in}}%
\pgfpathlineto{\pgfqpoint{5.750206in}{5.508384in}}%
\pgfpathlineto{\pgfqpoint{5.747915in}{5.455865in}}%
\pgfpathlineto{\pgfqpoint{5.743893in}{5.403346in}}%
\pgfpathlineto{\pgfqpoint{5.735248in}{5.324568in}}%
\pgfpathlineto{\pgfqpoint{5.724056in}{5.245790in}}%
\pgfpathlineto{\pgfqpoint{5.706472in}{5.140752in}}%
\pgfpathlineto{\pgfqpoint{5.681692in}{5.009454in}}%
\pgfpathlineto{\pgfqpoint{5.595356in}{4.563043in}}%
\pgfpathlineto{\pgfqpoint{5.573702in}{4.431746in}}%
\pgfpathlineto{\pgfqpoint{5.558628in}{4.326708in}}%
\pgfpathlineto{\pgfqpoint{5.545798in}{4.221670in}}%
\pgfpathlineto{\pgfqpoint{5.535562in}{4.116632in}}%
\pgfpathlineto{\pgfqpoint{5.528140in}{4.011594in}}%
\pgfpathlineto{\pgfqpoint{5.524504in}{3.932816in}}%
\pgfpathlineto{\pgfqpoint{5.522630in}{3.854037in}}%
\pgfpathlineto{\pgfqpoint{5.522596in}{3.775259in}}%
\pgfpathlineto{\pgfqpoint{5.524466in}{3.696481in}}%
\pgfpathlineto{\pgfqpoint{5.528291in}{3.617702in}}%
\pgfpathlineto{\pgfqpoint{5.534097in}{3.538924in}}%
\pgfpathlineto{\pgfqpoint{5.541972in}{3.460145in}}%
\pgfpathlineto{\pgfqpoint{5.551975in}{3.381367in}}%
\pgfpathlineto{\pgfqpoint{5.564095in}{3.302589in}}%
\pgfpathlineto{\pgfqpoint{5.578401in}{3.223810in}}%
\pgfpathlineto{\pgfqpoint{5.594919in}{3.145032in}}%
\pgfpathlineto{\pgfqpoint{5.615772in}{3.058240in}}%
\pgfpathlineto{\pgfqpoint{5.634710in}{2.987475in}}%
\pgfpathlineto{\pgfqpoint{5.658033in}{2.908696in}}%
\pgfpathlineto{\pgfqpoint{5.683680in}{2.829918in}}%
\pgfpathlineto{\pgfqpoint{5.711661in}{2.751140in}}%
\pgfpathlineto{\pgfqpoint{5.741988in}{2.672361in}}%
\pgfpathlineto{\pgfqpoint{5.774680in}{2.593583in}}%
\pgfpathlineto{\pgfqpoint{5.809755in}{2.514804in}}%
\pgfpathlineto{\pgfqpoint{5.852359in}{2.425726in}}%
\pgfpathlineto{\pgfqpoint{5.887104in}{2.357248in}}%
\pgfpathlineto{\pgfqpoint{5.929366in}{2.278469in}}%
\pgfpathlineto{\pgfqpoint{5.974072in}{2.199691in}}%
\pgfpathlineto{\pgfqpoint{6.021164in}{2.120912in}}%
\pgfpathlineto{\pgfqpoint{6.068671in}{2.045293in}}%
\pgfpathlineto{\pgfqpoint{6.068671in}{2.045293in}}%
\pgfusepath{stroke}%
\end{pgfscope}%
\begin{pgfscope}%
\pgfpathrectangle{\pgfqpoint{0.854460in}{0.571603in}}{\pgfqpoint{5.885100in}{5.225635in}}%
\pgfusepath{clip}%
\pgfsetbuttcap%
\pgfsetroundjoin%
\pgfsetlinewidth{1.505625pt}%
\definecolor{currentstroke}{rgb}{0.146180,0.515413,0.556823}%
\pgfsetstrokecolor{currentstroke}%
\pgfsetdash{}{0pt}%
\pgfpathmoveto{\pgfqpoint{6.291718in}{1.728712in}}%
\pgfpathlineto{\pgfqpoint{6.293007in}{1.727020in}}%
\pgfpathlineto{\pgfqpoint{6.295960in}{1.723200in}}%
\pgfpathlineto{\pgfqpoint{6.313258in}{1.700761in}}%
\pgfpathlineto{\pgfqpoint{6.325533in}{1.685087in}}%
\pgfpathlineto{\pgfqpoint{6.333802in}{1.674501in}}%
\pgfpathlineto{\pgfqpoint{6.354633in}{1.648242in}}%
\pgfpathlineto{\pgfqpoint{6.355107in}{1.647652in}}%
\pgfpathlineto{\pgfqpoint{6.375682in}{1.621982in}}%
\pgfpathlineto{\pgfqpoint{6.384680in}{1.610922in}}%
\pgfpathlineto{\pgfqpoint{6.397019in}{1.595723in}}%
\pgfpathlineto{\pgfqpoint{6.414253in}{1.574802in}}%
\pgfpathlineto{\pgfqpoint{6.418642in}{1.569463in}}%
\pgfpathlineto{\pgfqpoint{6.440521in}{1.543204in}}%
\pgfpathlineto{\pgfqpoint{6.443827in}{1.539287in}}%
\pgfpathlineto{\pgfqpoint{6.462646in}{1.516944in}}%
\pgfpathlineto{\pgfqpoint{6.473400in}{1.504353in}}%
\pgfpathlineto{\pgfqpoint{6.485053in}{1.490685in}}%
\pgfpathlineto{\pgfqpoint{6.502973in}{1.469951in}}%
\pgfpathlineto{\pgfqpoint{6.507741in}{1.464425in}}%
\pgfpathlineto{\pgfqpoint{6.530690in}{1.438166in}}%
\pgfpathlineto{\pgfqpoint{6.532547in}{1.436066in}}%
\pgfpathlineto{\pgfqpoint{6.553877in}{1.411906in}}%
\pgfpathlineto{\pgfqpoint{6.562120in}{1.402690in}}%
\pgfpathlineto{\pgfqpoint{6.577341in}{1.385647in}}%
\pgfpathlineto{\pgfqpoint{6.591693in}{1.369779in}}%
\pgfpathlineto{\pgfqpoint{6.601081in}{1.359388in}}%
\pgfpathlineto{\pgfqpoint{6.621267in}{1.337316in}}%
\pgfpathlineto{\pgfqpoint{6.625093in}{1.333128in}}%
\pgfpathlineto{\pgfqpoint{6.649365in}{1.306869in}}%
\pgfpathlineto{\pgfqpoint{6.650840in}{1.305289in}}%
\pgfpathlineto{\pgfqpoint{6.673876in}{1.280609in}}%
\pgfpathlineto{\pgfqpoint{6.680414in}{1.273685in}}%
\pgfpathlineto{\pgfqpoint{6.698658in}{1.254350in}}%
\pgfpathlineto{\pgfqpoint{6.709987in}{1.242478in}}%
\pgfpathlineto{\pgfqpoint{6.723709in}{1.228090in}}%
\pgfpathlineto{\pgfqpoint{6.739560in}{1.211653in}}%
\pgfusepath{stroke}%
\end{pgfscope}%
\begin{pgfscope}%
\pgfpathrectangle{\pgfqpoint{0.854460in}{0.571603in}}{\pgfqpoint{5.885100in}{5.225635in}}%
\pgfusepath{clip}%
\pgfsetbuttcap%
\pgfsetroundjoin%
\pgfsetlinewidth{1.505625pt}%
\definecolor{currentstroke}{rgb}{0.139147,0.533812,0.555298}%
\pgfsetstrokecolor{currentstroke}%
\pgfsetdash{}{0pt}%
\pgfpathmoveto{\pgfqpoint{1.343822in}{0.571603in}}%
\pgfpathlineto{\pgfqpoint{1.327634in}{0.586725in}}%
\pgfpathlineto{\pgfqpoint{1.316287in}{0.597376in}}%
\pgfusepath{stroke}%
\end{pgfscope}%
\begin{pgfscope}%
\pgfpathrectangle{\pgfqpoint{0.854460in}{0.571603in}}{\pgfqpoint{5.885100in}{5.225635in}}%
\pgfusepath{clip}%
\pgfsetbuttcap%
\pgfsetroundjoin%
\pgfsetlinewidth{1.505625pt}%
\definecolor{currentstroke}{rgb}{0.139147,0.533812,0.555298}%
\pgfsetstrokecolor{currentstroke}%
\pgfsetdash{}{0pt}%
\pgfpathmoveto{\pgfqpoint{1.050159in}{0.875342in}}%
\pgfpathlineto{\pgfqpoint{1.040371in}{0.886717in}}%
\pgfpathlineto{\pgfqpoint{1.031901in}{0.896707in}}%
\pgfpathlineto{\pgfqpoint{1.018198in}{0.912976in}}%
\pgfpathlineto{\pgfqpoint{1.002327in}{0.932101in}}%
\pgfpathlineto{\pgfqpoint{0.996447in}{0.939236in}}%
\pgfpathlineto{\pgfqpoint{0.975140in}{0.965495in}}%
\pgfpathlineto{\pgfqpoint{0.972754in}{0.968487in}}%
\pgfpathlineto{\pgfqpoint{0.954323in}{0.991755in}}%
\pgfpathlineto{\pgfqpoint{0.943181in}{1.006034in}}%
\pgfpathlineto{\pgfqpoint{0.933900in}{1.018014in}}%
\pgfpathlineto{\pgfqpoint{0.913865in}{1.044274in}}%
\pgfpathlineto{\pgfqpoint{0.913607in}{1.044619in}}%
\pgfpathlineto{\pgfqpoint{0.894341in}{1.070533in}}%
\pgfpathlineto{\pgfqpoint{0.884034in}{1.084612in}}%
\pgfpathlineto{\pgfqpoint{0.875184in}{1.096793in}}%
\pgfpathlineto{\pgfqpoint{0.856410in}{1.123052in}}%
\pgfpathlineto{\pgfqpoint{0.854460in}{1.125830in}}%
\pgfusepath{stroke}%
\end{pgfscope}%
\begin{pgfscope}%
\pgfpathrectangle{\pgfqpoint{0.854460in}{0.571603in}}{\pgfqpoint{5.885100in}{5.225635in}}%
\pgfusepath{clip}%
\pgfsetbuttcap%
\pgfsetroundjoin%
\pgfsetlinewidth{1.505625pt}%
\definecolor{currentstroke}{rgb}{0.139147,0.533812,0.555298}%
\pgfsetstrokecolor{currentstroke}%
\pgfsetdash{}{0pt}%
\pgfpathmoveto{\pgfqpoint{0.854460in}{4.406469in}}%
\pgfpathlineto{\pgfqpoint{0.872412in}{4.431746in}}%
\pgfpathlineto{\pgfqpoint{0.883859in}{4.447642in}}%
\pgfusepath{stroke}%
\end{pgfscope}%
\begin{pgfscope}%
\pgfpathrectangle{\pgfqpoint{0.854460in}{0.571603in}}{\pgfqpoint{5.885100in}{5.225635in}}%
\pgfusepath{clip}%
\pgfsetbuttcap%
\pgfsetroundjoin%
\pgfsetlinewidth{1.505625pt}%
\definecolor{currentstroke}{rgb}{0.139147,0.533812,0.555298}%
\pgfsetstrokecolor{currentstroke}%
\pgfsetdash{}{0pt}%
\pgfpathmoveto{\pgfqpoint{1.129692in}{4.745129in}}%
\pgfpathlineto{\pgfqpoint{1.131303in}{4.746860in}}%
\pgfpathlineto{\pgfqpoint{1.150194in}{4.766893in}}%
\pgfpathlineto{\pgfqpoint{1.156130in}{4.773119in}}%
\pgfpathlineto{\pgfqpoint{1.179767in}{4.797602in}}%
\pgfpathlineto{\pgfqpoint{1.181502in}{4.799378in}}%
\pgfpathlineto{\pgfqpoint{1.207465in}{4.825638in}}%
\pgfpathlineto{\pgfqpoint{1.209341in}{4.827510in}}%
\pgfpathlineto{\pgfqpoint{1.234025in}{4.851897in}}%
\pgfpathlineto{\pgfqpoint{1.238914in}{4.856668in}}%
\pgfpathlineto{\pgfqpoint{1.261166in}{4.878157in}}%
\pgfpathlineto{\pgfqpoint{1.268488in}{4.885140in}}%
\pgfpathlineto{\pgfqpoint{1.288905in}{4.904416in}}%
\pgfpathlineto{\pgfqpoint{1.298061in}{4.912954in}}%
\pgfpathlineto{\pgfqpoint{1.317257in}{4.930676in}}%
\pgfpathlineto{\pgfqpoint{1.327634in}{4.940139in}}%
\pgfpathlineto{\pgfqpoint{1.346238in}{4.956935in}}%
\pgfpathlineto{\pgfqpoint{1.357208in}{4.966719in}}%
\pgfpathlineto{\pgfqpoint{1.375862in}{4.983195in}}%
\pgfpathlineto{\pgfqpoint{1.386781in}{4.992721in}}%
\pgfpathlineto{\pgfqpoint{1.406146in}{5.009454in}}%
\pgfpathlineto{\pgfqpoint{1.416354in}{5.018168in}}%
\pgfpathlineto{\pgfqpoint{1.437104in}{5.035714in}}%
\pgfpathlineto{\pgfqpoint{1.445928in}{5.043085in}}%
\pgfpathlineto{\pgfqpoint{1.468751in}{5.061973in}}%
\pgfpathlineto{\pgfqpoint{1.475501in}{5.067492in}}%
\pgfpathlineto{\pgfqpoint{1.501101in}{5.088233in}}%
\pgfpathlineto{\pgfqpoint{1.505074in}{5.091413in}}%
\pgfpathlineto{\pgfqpoint{1.534170in}{5.114492in}}%
\pgfpathlineto{\pgfqpoint{1.534648in}{5.114867in}}%
\pgfpathlineto{\pgfqpoint{1.564221in}{5.137816in}}%
\pgfpathlineto{\pgfqpoint{1.568038in}{5.140752in}}%
\pgfpathlineto{\pgfqpoint{1.593795in}{5.160323in}}%
\pgfpathlineto{\pgfqpoint{1.602673in}{5.167011in}}%
\pgfpathlineto{\pgfqpoint{1.623368in}{5.182413in}}%
\pgfpathlineto{\pgfqpoint{1.638081in}{5.193271in}}%
\pgfpathlineto{\pgfqpoint{1.652941in}{5.204105in}}%
\pgfpathlineto{\pgfqpoint{1.674274in}{5.219530in}}%
\pgfpathlineto{\pgfqpoint{1.682515in}{5.225417in}}%
\pgfpathlineto{\pgfqpoint{1.711265in}{5.245790in}}%
\pgfpathlineto{\pgfqpoint{1.712088in}{5.246366in}}%
\pgfpathlineto{\pgfqpoint{1.741661in}{5.266865in}}%
\pgfpathlineto{\pgfqpoint{1.749201in}{5.272049in}}%
\pgfpathlineto{\pgfqpoint{1.771235in}{5.287016in}}%
\pgfpathlineto{\pgfqpoint{1.787988in}{5.298308in}}%
\pgfpathlineto{\pgfqpoint{1.800808in}{5.306845in}}%
\pgfpathlineto{\pgfqpoint{1.827622in}{5.324568in}}%
\pgfpathlineto{\pgfqpoint{1.830381in}{5.326370in}}%
\pgfpathlineto{\pgfqpoint{1.859955in}{5.345497in}}%
\pgfpathlineto{\pgfqpoint{1.868264in}{5.350827in}}%
\pgfpathlineto{\pgfqpoint{1.889528in}{5.364304in}}%
\pgfpathlineto{\pgfqpoint{1.909840in}{5.377087in}}%
\pgfpathlineto{\pgfqpoint{1.919102in}{5.382845in}}%
\pgfpathlineto{\pgfqpoint{1.948675in}{5.401086in}}%
\pgfpathlineto{\pgfqpoint{1.952375in}{5.403346in}}%
\pgfpathlineto{\pgfqpoint{1.978248in}{5.418963in}}%
\pgfpathlineto{\pgfqpoint{1.996000in}{5.429606in}}%
\pgfpathlineto{\pgfqpoint{2.007822in}{5.436608in}}%
\pgfpathlineto{\pgfqpoint{2.037395in}{5.453995in}}%
\pgfpathlineto{\pgfqpoint{2.040607in}{5.455865in}}%
\pgfpathlineto{\pgfqpoint{2.066968in}{5.471026in}}%
\pgfpathlineto{\pgfqpoint{2.086386in}{5.482125in}}%
\pgfpathlineto{\pgfqpoint{2.096542in}{5.487858in}}%
\pgfpathlineto{\pgfqpoint{2.126115in}{5.504423in}}%
\pgfpathlineto{\pgfqpoint{2.133251in}{5.508384in}}%
\pgfpathlineto{\pgfqpoint{2.155689in}{5.520688in}}%
\pgfpathlineto{\pgfqpoint{2.181285in}{5.534644in}}%
\pgfpathlineto{\pgfqpoint{2.185262in}{5.536785in}}%
\pgfpathlineto{\pgfqpoint{2.214835in}{5.552555in}}%
\pgfpathlineto{\pgfqpoint{2.230596in}{5.560903in}}%
\pgfpathlineto{\pgfqpoint{2.244409in}{5.568130in}}%
\pgfpathlineto{\pgfqpoint{2.273982in}{5.583490in}}%
\pgfpathlineto{\pgfqpoint{2.281118in}{5.587163in}}%
\pgfpathlineto{\pgfqpoint{2.303555in}{5.598567in}}%
\pgfpathlineto{\pgfqpoint{2.332928in}{5.613422in}}%
\pgfpathlineto{\pgfqpoint{2.333129in}{5.613523in}}%
\pgfpathlineto{\pgfqpoint{2.362702in}{5.628127in}}%
\pgfpathlineto{\pgfqpoint{2.386210in}{5.639682in}}%
\pgfpathlineto{\pgfqpoint{2.392275in}{5.642625in}}%
\pgfpathlineto{\pgfqpoint{2.421849in}{5.656839in}}%
\pgfpathlineto{\pgfqpoint{2.440904in}{5.665941in}}%
\pgfpathlineto{\pgfqpoint{2.451422in}{5.670902in}}%
\pgfpathlineto{\pgfqpoint{2.480996in}{5.684730in}}%
\pgfpathlineto{\pgfqpoint{2.497088in}{5.692201in}}%
\pgfpathlineto{\pgfqpoint{2.510569in}{5.698380in}}%
\pgfpathlineto{\pgfqpoint{2.540142in}{5.711827in}}%
\pgfpathlineto{\pgfqpoint{2.554842in}{5.718460in}}%
\pgfpathlineto{\pgfqpoint{2.569716in}{5.725086in}}%
\pgfpathlineto{\pgfqpoint{2.599289in}{5.738157in}}%
\pgfpathlineto{\pgfqpoint{2.614253in}{5.744720in}}%
\pgfpathlineto{\pgfqpoint{2.628862in}{5.751044in}}%
\pgfpathlineto{\pgfqpoint{2.658436in}{5.763744in}}%
\pgfpathlineto{\pgfqpoint{2.675409in}{5.770979in}}%
\pgfpathlineto{\pgfqpoint{2.688009in}{5.776281in}}%
\pgfpathlineto{\pgfqpoint{2.717582in}{5.788614in}}%
\pgfpathlineto{\pgfqpoint{2.738400in}{5.797238in}}%
\pgfusepath{stroke}%
\end{pgfscope}%
\begin{pgfscope}%
\pgfpathrectangle{\pgfqpoint{0.854460in}{0.571603in}}{\pgfqpoint{5.885100in}{5.225635in}}%
\pgfusepath{clip}%
\pgfsetbuttcap%
\pgfsetroundjoin%
\pgfsetlinewidth{1.505625pt}%
\definecolor{currentstroke}{rgb}{0.139147,0.533812,0.555298}%
\pgfsetstrokecolor{currentstroke}%
\pgfsetdash{}{0pt}%
\pgfpathmoveto{\pgfqpoint{5.874289in}{5.797238in}}%
\pgfpathlineto{\pgfqpoint{5.878766in}{5.770979in}}%
\pgfpathlineto{\pgfqpoint{5.881933in}{5.747098in}}%
\pgfpathlineto{\pgfqpoint{5.882237in}{5.744720in}}%
\pgfpathlineto{\pgfqpoint{5.884756in}{5.718460in}}%
\pgfpathlineto{\pgfqpoint{5.886455in}{5.692201in}}%
\pgfpathlineto{\pgfqpoint{5.887402in}{5.665941in}}%
\pgfpathlineto{\pgfqpoint{5.887663in}{5.639682in}}%
\pgfpathlineto{\pgfqpoint{5.887295in}{5.613422in}}%
\pgfpathlineto{\pgfqpoint{5.886352in}{5.587163in}}%
\pgfpathlineto{\pgfqpoint{5.884883in}{5.560903in}}%
\pgfpathlineto{\pgfqpoint{5.882934in}{5.534644in}}%
\pgfpathlineto{\pgfqpoint{5.881933in}{5.523728in}}%
\pgfpathlineto{\pgfqpoint{5.880527in}{5.508384in}}%
\pgfpathlineto{\pgfqpoint{5.877700in}{5.482125in}}%
\pgfpathlineto{\pgfqpoint{5.874505in}{5.455865in}}%
\pgfpathlineto{\pgfqpoint{5.870975in}{5.429606in}}%
\pgfpathlineto{\pgfqpoint{5.867141in}{5.403346in}}%
\pgfpathlineto{\pgfqpoint{5.863034in}{5.377087in}}%
\pgfpathlineto{\pgfqpoint{5.858678in}{5.350827in}}%
\pgfpathlineto{\pgfqpoint{5.854101in}{5.324568in}}%
\pgfpathlineto{\pgfqpoint{5.852359in}{5.315073in}}%
\pgfpathlineto{\pgfqpoint{5.849284in}{5.298308in}}%
\pgfpathlineto{\pgfqpoint{5.844266in}{5.272049in}}%
\pgfpathlineto{\pgfqpoint{5.839091in}{5.245790in}}%
\pgfpathlineto{\pgfqpoint{5.833778in}{5.219530in}}%
\pgfpathlineto{\pgfqpoint{5.828344in}{5.193271in}}%
\pgfpathlineto{\pgfqpoint{5.822808in}{5.167011in}}%
\pgfpathlineto{\pgfqpoint{5.822786in}{5.166911in}}%
\pgfpathlineto{\pgfqpoint{5.817115in}{5.140752in}}%
\pgfpathlineto{\pgfqpoint{5.811349in}{5.114492in}}%
\pgfpathlineto{\pgfqpoint{5.805527in}{5.088233in}}%
\pgfpathlineto{\pgfqpoint{5.799660in}{5.061973in}}%
\pgfpathlineto{\pgfqpoint{5.793761in}{5.035714in}}%
\pgfpathlineto{\pgfqpoint{5.793213in}{5.033304in}}%
\pgfpathlineto{\pgfqpoint{5.787778in}{5.009454in}}%
\pgfpathlineto{\pgfqpoint{5.781782in}{4.983195in}}%
\pgfpathlineto{\pgfqpoint{5.775788in}{4.956935in}}%
\pgfpathlineto{\pgfqpoint{5.769807in}{4.930676in}}%
\pgfpathlineto{\pgfqpoint{5.763849in}{4.904416in}}%
\pgfpathlineto{\pgfqpoint{5.763639in}{4.903493in}}%
\pgfpathlineto{\pgfqpoint{5.757857in}{4.878157in}}%
\pgfpathlineto{\pgfqpoint{5.751904in}{4.851897in}}%
\pgfpathlineto{\pgfqpoint{5.746001in}{4.825638in}}%
\pgfpathlineto{\pgfqpoint{5.740156in}{4.799378in}}%
\pgfpathlineto{\pgfqpoint{5.734374in}{4.773119in}}%
\pgfpathlineto{\pgfqpoint{5.734066in}{4.771710in}}%
\pgfpathlineto{\pgfqpoint{5.728605in}{4.746860in}}%
\pgfpathlineto{\pgfqpoint{5.722912in}{4.720600in}}%
\pgfpathlineto{\pgfqpoint{5.717306in}{4.694341in}}%
\pgfpathlineto{\pgfqpoint{5.711792in}{4.668081in}}%
\pgfpathlineto{\pgfqpoint{5.708198in}{4.650654in}}%
\pgfusepath{stroke}%
\end{pgfscope}%
\begin{pgfscope}%
\pgfpathrectangle{\pgfqpoint{0.854460in}{0.571603in}}{\pgfqpoint{5.885100in}{5.225635in}}%
\pgfusepath{clip}%
\pgfsetbuttcap%
\pgfsetroundjoin%
\pgfsetlinewidth{1.505625pt}%
\definecolor{currentstroke}{rgb}{0.139147,0.533812,0.555298}%
\pgfsetstrokecolor{currentstroke}%
\pgfsetdash{}{0pt}%
\pgfpathmoveto{\pgfqpoint{5.641245in}{4.264062in}}%
\pgfpathlineto{\pgfqpoint{5.629644in}{4.169151in}}%
\pgfpathlineto{\pgfqpoint{5.619550in}{4.064113in}}%
\pgfpathlineto{\pgfqpoint{5.613938in}{3.985335in}}%
\pgfpathlineto{\pgfqpoint{5.610075in}{3.906556in}}%
\pgfpathlineto{\pgfqpoint{5.608061in}{3.827778in}}%
\pgfpathlineto{\pgfqpoint{5.607956in}{3.749000in}}%
\pgfpathlineto{\pgfqpoint{5.609819in}{3.670221in}}%
\pgfpathlineto{\pgfqpoint{5.613705in}{3.591443in}}%
\pgfpathlineto{\pgfqpoint{5.619637in}{3.512664in}}%
\pgfpathlineto{\pgfqpoint{5.627666in}{3.433886in}}%
\pgfpathlineto{\pgfqpoint{5.637870in}{3.355107in}}%
\pgfpathlineto{\pgfqpoint{5.650261in}{3.276329in}}%
\pgfpathlineto{\pgfqpoint{5.664855in}{3.197551in}}%
\pgfpathlineto{\pgfqpoint{5.681718in}{3.118772in}}%
\pgfpathlineto{\pgfqpoint{5.700864in}{3.039994in}}%
\pgfpathlineto{\pgfqpoint{5.722287in}{2.961215in}}%
\pgfpathlineto{\pgfqpoint{5.746055in}{2.882437in}}%
\pgfpathlineto{\pgfqpoint{5.772175in}{2.803659in}}%
\pgfpathlineto{\pgfqpoint{5.800658in}{2.724880in}}%
\pgfpathlineto{\pgfqpoint{5.831516in}{2.646102in}}%
\pgfpathlineto{\pgfqpoint{5.864767in}{2.567323in}}%
\pgfpathlineto{\pgfqpoint{5.900430in}{2.488545in}}%
\pgfpathlineto{\pgfqpoint{5.941079in}{2.404717in}}%
\pgfpathlineto{\pgfqpoint{5.979014in}{2.330988in}}%
\pgfpathlineto{\pgfqpoint{6.029799in}{2.238361in}}%
\pgfpathlineto{\pgfqpoint{6.067317in}{2.173431in}}%
\pgfpathlineto{\pgfqpoint{6.118520in}{2.089275in}}%
\pgfpathlineto{\pgfqpoint{6.165371in}{2.015874in}}%
\pgfpathlineto{\pgfqpoint{6.218066in}{1.937096in}}%
\pgfpathlineto{\pgfqpoint{6.273209in}{1.858318in}}%
\pgfpathlineto{\pgfqpoint{6.330790in}{1.779539in}}%
\pgfpathlineto{\pgfqpoint{6.390803in}{1.700761in}}%
\pgfpathlineto{\pgfqpoint{6.453246in}{1.621982in}}%
\pgfpathlineto{\pgfqpoint{6.532547in}{1.526130in}}%
\pgfpathlineto{\pgfqpoint{6.591693in}{1.457274in}}%
\pgfpathlineto{\pgfqpoint{6.655156in}{1.385647in}}%
\pgfpathlineto{\pgfqpoint{6.739560in}{1.293737in}}%
\pgfpathlineto{\pgfqpoint{6.739560in}{1.293737in}}%
\pgfusepath{stroke}%
\end{pgfscope}%
\begin{pgfscope}%
\pgfpathrectangle{\pgfqpoint{0.854460in}{0.571603in}}{\pgfqpoint{5.885100in}{5.225635in}}%
\pgfusepath{clip}%
\pgfsetbuttcap%
\pgfsetroundjoin%
\pgfsetlinewidth{1.505625pt}%
\definecolor{currentstroke}{rgb}{0.131172,0.555899,0.552459}%
\pgfsetstrokecolor{currentstroke}%
\pgfsetdash{}{0pt}%
\pgfpathmoveto{\pgfqpoint{1.281021in}{0.571603in}}%
\pgfpathlineto{\pgfqpoint{1.268488in}{0.583413in}}%
\pgfpathlineto{\pgfqpoint{1.256686in}{0.594588in}}%
\pgfusepath{stroke}%
\end{pgfscope}%
\begin{pgfscope}%
\pgfpathrectangle{\pgfqpoint{0.854460in}{0.571603in}}{\pgfqpoint{5.885100in}{5.225635in}}%
\pgfusepath{clip}%
\pgfsetbuttcap%
\pgfsetroundjoin%
\pgfsetlinewidth{1.505625pt}%
\definecolor{currentstroke}{rgb}{0.131172,0.555899,0.552459}%
\pgfsetstrokecolor{currentstroke}%
\pgfsetdash{}{0pt}%
\pgfpathmoveto{\pgfqpoint{0.991526in}{0.873556in}}%
\pgfpathlineto{\pgfqpoint{0.980299in}{0.886717in}}%
\pgfpathlineto{\pgfqpoint{0.972754in}{0.895692in}}%
\pgfpathlineto{\pgfqpoint{0.958322in}{0.912976in}}%
\pgfpathlineto{\pgfqpoint{0.943181in}{0.931379in}}%
\pgfpathlineto{\pgfqpoint{0.936761in}{0.939236in}}%
\pgfpathlineto{\pgfqpoint{0.915633in}{0.965495in}}%
\pgfpathlineto{\pgfqpoint{0.913607in}{0.968056in}}%
\pgfpathlineto{\pgfqpoint{0.894995in}{0.991755in}}%
\pgfpathlineto{\pgfqpoint{0.884034in}{1.005922in}}%
\pgfpathlineto{\pgfqpoint{0.874746in}{1.018014in}}%
\pgfpathlineto{\pgfqpoint{0.854881in}{1.044274in}}%
\pgfpathlineto{\pgfqpoint{0.854460in}{1.044840in}}%
\pgfusepath{stroke}%
\end{pgfscope}%
\begin{pgfscope}%
\pgfpathrectangle{\pgfqpoint{0.854460in}{0.571603in}}{\pgfqpoint{5.885100in}{5.225635in}}%
\pgfusepath{clip}%
\pgfsetbuttcap%
\pgfsetroundjoin%
\pgfsetlinewidth{1.505625pt}%
\definecolor{currentstroke}{rgb}{0.131172,0.555899,0.552459}%
\pgfsetstrokecolor{currentstroke}%
\pgfsetdash{}{0pt}%
\pgfpathmoveto{\pgfqpoint{0.854460in}{4.503619in}}%
\pgfpathlineto{\pgfqpoint{0.859611in}{4.510524in}}%
\pgfpathlineto{\pgfqpoint{0.879488in}{4.536784in}}%
\pgfpathlineto{\pgfqpoint{0.884034in}{4.542702in}}%
\pgfpathlineto{\pgfqpoint{0.899846in}{4.563043in}}%
\pgfpathlineto{\pgfqpoint{0.913607in}{4.580512in}}%
\pgfpathlineto{\pgfqpoint{0.920614in}{4.589303in}}%
\pgfpathlineto{\pgfqpoint{0.941830in}{4.615562in}}%
\pgfpathlineto{\pgfqpoint{0.943181in}{4.617208in}}%
\pgfpathlineto{\pgfqpoint{0.963598in}{4.641822in}}%
\pgfpathlineto{\pgfqpoint{0.972754in}{4.652716in}}%
\pgfpathlineto{\pgfqpoint{0.985814in}{4.668081in}}%
\pgfpathlineto{\pgfqpoint{1.002327in}{4.687258in}}%
\pgfpathlineto{\pgfqpoint{1.008495in}{4.694341in}}%
\pgfpathlineto{\pgfqpoint{1.031660in}{4.720600in}}%
\pgfpathlineto{\pgfqpoint{1.031901in}{4.720869in}}%
\pgfpathlineto{\pgfqpoint{1.055416in}{4.746860in}}%
\pgfpathlineto{\pgfqpoint{1.061474in}{4.753470in}}%
\pgfpathlineto{\pgfqpoint{1.079675in}{4.773119in}}%
\pgfpathlineto{\pgfqpoint{1.091047in}{4.785242in}}%
\pgfpathlineto{\pgfqpoint{1.104452in}{4.799378in}}%
\pgfpathlineto{\pgfqpoint{1.120621in}{4.816217in}}%
\pgfpathlineto{\pgfqpoint{1.129763in}{4.825638in}}%
\pgfpathlineto{\pgfqpoint{1.150194in}{4.846429in}}%
\pgfpathlineto{\pgfqpoint{1.155624in}{4.851897in}}%
\pgfpathlineto{\pgfqpoint{1.179767in}{4.875911in}}%
\pgfpathlineto{\pgfqpoint{1.182049in}{4.878157in}}%
\pgfpathlineto{\pgfqpoint{1.209060in}{4.904416in}}%
\pgfpathlineto{\pgfqpoint{1.209341in}{4.904686in}}%
\pgfpathlineto{\pgfqpoint{1.236695in}{4.930676in}}%
\pgfpathlineto{\pgfqpoint{1.238914in}{4.932758in}}%
\pgfpathlineto{\pgfqpoint{1.264931in}{4.956935in}}%
\pgfpathlineto{\pgfqpoint{1.268488in}{4.960200in}}%
\pgfpathlineto{\pgfqpoint{1.293783in}{4.983195in}}%
\pgfpathlineto{\pgfqpoint{1.298061in}{4.987036in}}%
\pgfpathlineto{\pgfqpoint{1.323266in}{5.009454in}}%
\pgfpathlineto{\pgfqpoint{1.327634in}{5.013292in}}%
\pgfpathlineto{\pgfqpoint{1.353394in}{5.035714in}}%
\pgfpathlineto{\pgfqpoint{1.357208in}{5.038992in}}%
\pgfpathlineto{\pgfqpoint{1.384183in}{5.061973in}}%
\pgfpathlineto{\pgfqpoint{1.386781in}{5.064160in}}%
\pgfpathlineto{\pgfqpoint{1.415645in}{5.088233in}}%
\pgfpathlineto{\pgfqpoint{1.416354in}{5.088817in}}%
\pgfpathlineto{\pgfqpoint{1.445928in}{5.112955in}}%
\pgfpathlineto{\pgfqpoint{1.447828in}{5.114492in}}%
\pgfpathlineto{\pgfqpoint{1.475501in}{5.136603in}}%
\pgfpathlineto{\pgfqpoint{1.480738in}{5.140752in}}%
\pgfpathlineto{\pgfqpoint{1.505074in}{5.159795in}}%
\pgfpathlineto{\pgfqpoint{1.514376in}{5.167011in}}%
\pgfpathlineto{\pgfqpoint{1.534648in}{5.182549in}}%
\pgfpathlineto{\pgfqpoint{1.548755in}{5.193271in}}%
\pgfpathlineto{\pgfqpoint{1.564221in}{5.204884in}}%
\pgfpathlineto{\pgfqpoint{1.583887in}{5.219530in}}%
\pgfpathlineto{\pgfqpoint{1.593795in}{5.226819in}}%
\pgfpathlineto{\pgfqpoint{1.619787in}{5.245790in}}%
\pgfpathlineto{\pgfqpoint{1.623368in}{5.248372in}}%
\pgfpathlineto{\pgfqpoint{1.652941in}{5.269509in}}%
\pgfpathlineto{\pgfqpoint{1.656526in}{5.272049in}}%
\pgfpathlineto{\pgfqpoint{1.682515in}{5.290238in}}%
\pgfpathlineto{\pgfqpoint{1.694133in}{5.298308in}}%
\pgfpathlineto{\pgfqpoint{1.712088in}{5.310629in}}%
\pgfpathlineto{\pgfqpoint{1.732555in}{5.324568in}}%
\pgfpathlineto{\pgfqpoint{1.741661in}{5.330695in}}%
\pgfpathlineto{\pgfqpoint{1.771235in}{5.350446in}}%
\pgfpathlineto{\pgfqpoint{1.771812in}{5.350827in}}%
\pgfpathlineto{\pgfqpoint{1.800808in}{5.369773in}}%
\pgfpathlineto{\pgfqpoint{1.812082in}{5.377087in}}%
\pgfpathlineto{\pgfqpoint{1.830381in}{5.388815in}}%
\pgfpathlineto{\pgfqpoint{1.853210in}{5.403346in}}%
\pgfpathlineto{\pgfqpoint{1.859955in}{5.407587in}}%
\pgfpathlineto{\pgfqpoint{1.889528in}{5.426031in}}%
\pgfpathlineto{\pgfqpoint{1.895309in}{5.429606in}}%
\pgfpathlineto{\pgfqpoint{1.919102in}{5.444140in}}%
\pgfpathlineto{\pgfqpoint{1.938421in}{5.455865in}}%
\pgfpathlineto{\pgfqpoint{1.948675in}{5.462013in}}%
\pgfpathlineto{\pgfqpoint{1.978248in}{5.479613in}}%
\pgfpathlineto{\pgfqpoint{1.982507in}{5.482125in}}%
\pgfpathlineto{\pgfqpoint{2.007822in}{5.496873in}}%
\pgfpathlineto{\pgfqpoint{2.027700in}{5.508384in}}%
\pgfpathlineto{\pgfqpoint{2.037395in}{5.513930in}}%
\pgfpathlineto{\pgfqpoint{2.066968in}{5.530717in}}%
\pgfpathlineto{\pgfqpoint{2.073944in}{5.534644in}}%
\pgfpathlineto{\pgfqpoint{2.096542in}{5.547207in}}%
\pgfpathlineto{\pgfqpoint{2.121313in}{5.560903in}}%
\pgfpathlineto{\pgfqpoint{2.126115in}{5.563526in}}%
\pgfpathlineto{\pgfqpoint{2.155689in}{5.579529in}}%
\pgfpathlineto{\pgfqpoint{2.169890in}{5.587163in}}%
\pgfpathlineto{\pgfqpoint{2.185262in}{5.595324in}}%
\pgfpathlineto{\pgfqpoint{2.214835in}{5.610926in}}%
\pgfpathlineto{\pgfqpoint{2.219612in}{5.613422in}}%
\pgfpathlineto{\pgfqpoint{2.244409in}{5.626222in}}%
\pgfpathlineto{\pgfqpoint{2.270609in}{5.639682in}}%
\pgfpathlineto{\pgfqpoint{2.273982in}{5.641393in}}%
\pgfpathlineto{\pgfqpoint{2.303555in}{5.656249in}}%
\pgfpathlineto{\pgfqpoint{2.322953in}{5.665941in}}%
\pgfpathlineto{\pgfqpoint{2.333129in}{5.670962in}}%
\pgfpathlineto{\pgfqpoint{2.362702in}{5.685435in}}%
\pgfpathlineto{\pgfqpoint{2.376624in}{5.692201in}}%
\pgfpathlineto{\pgfqpoint{2.392275in}{5.699711in}}%
\pgfpathlineto{\pgfqpoint{2.421849in}{5.713807in}}%
\pgfpathlineto{\pgfqpoint{2.426481in}{5.715997in}}%
\pgfusepath{stroke}%
\end{pgfscope}%
\begin{pgfscope}%
\pgfpathrectangle{\pgfqpoint{0.854460in}{0.571603in}}{\pgfqpoint{5.885100in}{5.225635in}}%
\pgfusepath{clip}%
\pgfsetbuttcap%
\pgfsetroundjoin%
\pgfsetlinewidth{1.505625pt}%
\definecolor{currentstroke}{rgb}{0.131172,0.555899,0.552459}%
\pgfsetstrokecolor{currentstroke}%
\pgfsetdash{}{0pt}%
\pgfpathmoveto{\pgfqpoint{6.022114in}{5.797238in}}%
\pgfpathlineto{\pgfqpoint{6.023430in}{5.770979in}}%
\pgfpathlineto{\pgfqpoint{6.023966in}{5.744720in}}%
\pgfpathlineto{\pgfqpoint{6.023790in}{5.718460in}}%
\pgfpathlineto{\pgfqpoint{6.022961in}{5.692201in}}%
\pgfpathlineto{\pgfqpoint{6.021536in}{5.665941in}}%
\pgfpathlineto{\pgfqpoint{6.019566in}{5.639682in}}%
\pgfpathlineto{\pgfqpoint{6.017098in}{5.613422in}}%
\pgfpathlineto{\pgfqpoint{6.014176in}{5.587163in}}%
\pgfpathlineto{\pgfqpoint{6.010839in}{5.560903in}}%
\pgfpathlineto{\pgfqpoint{6.007124in}{5.534644in}}%
\pgfpathlineto{\pgfqpoint{6.003066in}{5.508384in}}%
\pgfpathlineto{\pgfqpoint{6.000226in}{5.491406in}}%
\pgfpathlineto{\pgfqpoint{5.998675in}{5.482125in}}%
\pgfpathlineto{\pgfqpoint{5.993962in}{5.455865in}}%
\pgfpathlineto{\pgfqpoint{5.988991in}{5.429606in}}%
\pgfpathlineto{\pgfqpoint{5.983790in}{5.403346in}}%
\pgfpathlineto{\pgfqpoint{5.978382in}{5.377087in}}%
\pgfpathlineto{\pgfqpoint{5.972789in}{5.350827in}}%
\pgfpathlineto{\pgfqpoint{5.970653in}{5.341160in}}%
\pgfpathlineto{\pgfqpoint{5.966987in}{5.324568in}}%
\pgfpathlineto{\pgfqpoint{5.961014in}{5.298308in}}%
\pgfpathlineto{\pgfqpoint{5.954915in}{5.272049in}}%
\pgfpathlineto{\pgfqpoint{5.948707in}{5.245790in}}%
\pgfpathlineto{\pgfqpoint{5.942408in}{5.219530in}}%
\pgfpathlineto{\pgfqpoint{5.941079in}{5.214106in}}%
\pgfpathlineto{\pgfqpoint{5.935971in}{5.193271in}}%
\pgfpathlineto{\pgfqpoint{5.930078in}{5.169513in}}%
\pgfusepath{stroke}%
\end{pgfscope}%
\begin{pgfscope}%
\pgfpathrectangle{\pgfqpoint{0.854460in}{0.571603in}}{\pgfqpoint{5.885100in}{5.225635in}}%
\pgfusepath{clip}%
\pgfsetbuttcap%
\pgfsetroundjoin%
\pgfsetlinewidth{1.505625pt}%
\definecolor{currentstroke}{rgb}{0.131172,0.555899,0.552459}%
\pgfsetstrokecolor{currentstroke}%
\pgfsetdash{}{0pt}%
\pgfpathmoveto{\pgfqpoint{5.834333in}{4.789472in}}%
\pgfpathlineto{\pgfqpoint{5.805733in}{4.668081in}}%
\pgfpathlineto{\pgfqpoint{5.777265in}{4.536784in}}%
\pgfpathlineto{\pgfqpoint{5.756775in}{4.431746in}}%
\pgfpathlineto{\pgfqpoint{5.738623in}{4.326708in}}%
\pgfpathlineto{\pgfqpoint{5.723045in}{4.221670in}}%
\pgfpathlineto{\pgfqpoint{5.710316in}{4.116632in}}%
\pgfpathlineto{\pgfqpoint{5.702748in}{4.037854in}}%
\pgfpathlineto{\pgfqpoint{5.696923in}{3.959075in}}%
\pgfpathlineto{\pgfqpoint{5.692956in}{3.880297in}}%
\pgfpathlineto{\pgfqpoint{5.690903in}{3.801519in}}%
\pgfpathlineto{\pgfqpoint{5.690819in}{3.722740in}}%
\pgfpathlineto{\pgfqpoint{5.692758in}{3.643962in}}%
\pgfpathlineto{\pgfqpoint{5.696771in}{3.565183in}}%
\pgfpathlineto{\pgfqpoint{5.702912in}{3.486405in}}%
\pgfpathlineto{\pgfqpoint{5.711173in}{3.407626in}}%
\pgfpathlineto{\pgfqpoint{5.721636in}{3.328848in}}%
\pgfpathlineto{\pgfqpoint{5.734366in}{3.250070in}}%
\pgfpathlineto{\pgfqpoint{5.749293in}{3.171291in}}%
\pgfpathlineto{\pgfqpoint{5.766561in}{3.092513in}}%
\pgfpathlineto{\pgfqpoint{5.786108in}{3.013734in}}%
\pgfpathlineto{\pgfqpoint{5.807996in}{2.934956in}}%
\pgfpathlineto{\pgfqpoint{5.832254in}{2.856177in}}%
\pgfpathlineto{\pgfqpoint{5.858889in}{2.777399in}}%
\pgfpathlineto{\pgfqpoint{5.887913in}{2.698621in}}%
\pgfpathlineto{\pgfqpoint{5.919338in}{2.619842in}}%
\pgfpathlineto{\pgfqpoint{5.953180in}{2.541064in}}%
\pgfpathlineto{\pgfqpoint{5.989460in}{2.462285in}}%
\pgfpathlineto{\pgfqpoint{6.029799in}{2.380396in}}%
\pgfpathlineto{\pgfqpoint{6.069345in}{2.304729in}}%
\pgfpathlineto{\pgfqpoint{6.118520in}{2.216317in}}%
\pgfpathlineto{\pgfqpoint{6.159038in}{2.147172in}}%
\pgfpathlineto{\pgfqpoint{6.207603in}{2.068393in}}%
\pgfpathlineto{\pgfqpoint{6.266386in}{1.977953in}}%
\pgfpathlineto{\pgfqpoint{6.312036in}{1.910836in}}%
\pgfpathlineto{\pgfqpoint{6.367962in}{1.832058in}}%
\pgfpathlineto{\pgfqpoint{6.426349in}{1.753280in}}%
\pgfpathlineto{\pgfqpoint{6.502973in}{1.654630in}}%
\pgfpathlineto{\pgfqpoint{6.562120in}{1.581632in}}%
\pgfpathlineto{\pgfqpoint{6.621267in}{1.511089in}}%
\pgfpathlineto{\pgfqpoint{6.684450in}{1.438166in}}%
\pgfpathlineto{\pgfqpoint{6.739560in}{1.376493in}}%
\pgfpathlineto{\pgfqpoint{6.739560in}{1.376493in}}%
\pgfusepath{stroke}%
\end{pgfscope}%
\begin{pgfscope}%
\pgfpathrectangle{\pgfqpoint{0.854460in}{0.571603in}}{\pgfqpoint{5.885100in}{5.225635in}}%
\pgfusepath{clip}%
\pgfsetbuttcap%
\pgfsetroundjoin%
\pgfsetlinewidth{1.505625pt}%
\definecolor{currentstroke}{rgb}{0.125394,0.574318,0.549086}%
\pgfsetstrokecolor{currentstroke}%
\pgfsetdash{}{0pt}%
\pgfpathmoveto{\pgfqpoint{1.219941in}{0.571603in}}%
\pgfpathlineto{\pgfqpoint{1.209341in}{0.581679in}}%
\pgfpathlineto{\pgfqpoint{1.197001in}{0.593467in}}%
\pgfusepath{stroke}%
\end{pgfscope}%
\begin{pgfscope}%
\pgfpathrectangle{\pgfqpoint{0.854460in}{0.571603in}}{\pgfqpoint{5.885100in}{5.225635in}}%
\pgfusepath{clip}%
\pgfsetbuttcap%
\pgfsetroundjoin%
\pgfsetlinewidth{1.505625pt}%
\definecolor{currentstroke}{rgb}{0.125394,0.574318,0.549086}%
\pgfsetstrokecolor{currentstroke}%
\pgfsetdash{}{0pt}%
\pgfpathmoveto{\pgfqpoint{0.932968in}{0.873593in}}%
\pgfpathlineto{\pgfqpoint{0.921870in}{0.886717in}}%
\pgfpathlineto{\pgfqpoint{0.913607in}{0.896632in}}%
\pgfpathlineto{\pgfqpoint{0.900078in}{0.912976in}}%
\pgfpathlineto{\pgfqpoint{0.884034in}{0.932646in}}%
\pgfpathlineto{\pgfqpoint{0.878696in}{0.939236in}}%
\pgfpathlineto{\pgfqpoint{0.857758in}{0.965495in}}%
\pgfpathlineto{\pgfqpoint{0.854460in}{0.969700in}}%
\pgfusepath{stroke}%
\end{pgfscope}%
\begin{pgfscope}%
\pgfpathrectangle{\pgfqpoint{0.854460in}{0.571603in}}{\pgfqpoint{5.885100in}{5.225635in}}%
\pgfusepath{clip}%
\pgfsetbuttcap%
\pgfsetroundjoin%
\pgfsetlinewidth{1.505625pt}%
\definecolor{currentstroke}{rgb}{0.125394,0.574318,0.549086}%
\pgfsetstrokecolor{currentstroke}%
\pgfsetdash{}{0pt}%
\pgfpathmoveto{\pgfqpoint{0.854460in}{4.593842in}}%
\pgfpathlineto{\pgfqpoint{0.871713in}{4.615562in}}%
\pgfpathlineto{\pgfqpoint{0.884034in}{4.630870in}}%
\pgfpathlineto{\pgfqpoint{0.892949in}{4.641822in}}%
\pgfpathlineto{\pgfqpoint{0.913607in}{4.666866in}}%
\pgfpathlineto{\pgfqpoint{0.914620in}{4.668081in}}%
\pgfpathlineto{\pgfqpoint{0.936847in}{4.694341in}}%
\pgfpathlineto{\pgfqpoint{0.943181in}{4.701725in}}%
\pgfpathlineto{\pgfqpoint{0.959548in}{4.720600in}}%
\pgfpathlineto{\pgfqpoint{0.972754in}{4.735633in}}%
\pgfpathlineto{\pgfqpoint{0.982722in}{4.746860in}}%
\pgfpathlineto{\pgfqpoint{1.002327in}{4.768657in}}%
\pgfpathlineto{\pgfqpoint{1.006384in}{4.773119in}}%
\pgfpathlineto{\pgfqpoint{1.030571in}{4.799378in}}%
\pgfpathlineto{\pgfqpoint{1.031901in}{4.800802in}}%
\pgfpathlineto{\pgfqpoint{1.055335in}{4.825638in}}%
\pgfpathlineto{\pgfqpoint{1.061474in}{4.832062in}}%
\pgfpathlineto{\pgfqpoint{1.080624in}{4.851897in}}%
\pgfpathlineto{\pgfqpoint{1.091047in}{4.862559in}}%
\pgfpathlineto{\pgfqpoint{1.106453in}{4.878157in}}%
\pgfpathlineto{\pgfqpoint{1.120621in}{4.892324in}}%
\pgfpathlineto{\pgfqpoint{1.132836in}{4.904416in}}%
\pgfpathlineto{\pgfqpoint{1.150194in}{4.921386in}}%
\pgfpathlineto{\pgfqpoint{1.159790in}{4.930676in}}%
\pgfpathlineto{\pgfqpoint{1.179767in}{4.949776in}}%
\pgfpathlineto{\pgfqpoint{1.187329in}{4.956935in}}%
\pgfpathlineto{\pgfqpoint{1.209341in}{4.977520in}}%
\pgfpathlineto{\pgfqpoint{1.215467in}{4.983195in}}%
\pgfpathlineto{\pgfqpoint{1.238914in}{5.004646in}}%
\pgfpathlineto{\pgfqpoint{1.244220in}{5.009454in}}%
\pgfpathlineto{\pgfqpoint{1.268488in}{5.031178in}}%
\pgfpathlineto{\pgfqpoint{1.273602in}{5.035714in}}%
\pgfpathlineto{\pgfqpoint{1.298061in}{5.057141in}}%
\pgfpathlineto{\pgfqpoint{1.303627in}{5.061973in}}%
\pgfpathlineto{\pgfqpoint{1.327634in}{5.082560in}}%
\pgfpathlineto{\pgfqpoint{1.334310in}{5.088233in}}%
\pgfpathlineto{\pgfqpoint{1.357208in}{5.107455in}}%
\pgfpathlineto{\pgfqpoint{1.365664in}{5.114492in}}%
\pgfpathlineto{\pgfqpoint{1.386781in}{5.131850in}}%
\pgfpathlineto{\pgfqpoint{1.397704in}{5.140752in}}%
\pgfpathlineto{\pgfqpoint{1.416354in}{5.155765in}}%
\pgfpathlineto{\pgfqpoint{1.430443in}{5.167011in}}%
\pgfpathlineto{\pgfqpoint{1.445928in}{5.179221in}}%
\pgfpathlineto{\pgfqpoint{1.463894in}{5.193271in}}%
\pgfpathlineto{\pgfqpoint{1.475501in}{5.202237in}}%
\pgfpathlineto{\pgfqpoint{1.498069in}{5.219530in}}%
\pgfpathlineto{\pgfqpoint{1.505074in}{5.224833in}}%
\pgfpathlineto{\pgfqpoint{1.532982in}{5.245790in}}%
\pgfpathlineto{\pgfqpoint{1.534648in}{5.247026in}}%
\pgfpathlineto{\pgfqpoint{1.564221in}{5.268769in}}%
\pgfpathlineto{\pgfqpoint{1.568719in}{5.272049in}}%
\pgfpathlineto{\pgfqpoint{1.593795in}{5.290113in}}%
\pgfpathlineto{\pgfqpoint{1.605258in}{5.298308in}}%
\pgfpathlineto{\pgfqpoint{1.623368in}{5.311099in}}%
\pgfpathlineto{\pgfqpoint{1.642582in}{5.324568in}}%
\pgfpathlineto{\pgfqpoint{1.652941in}{5.331742in}}%
\pgfpathlineto{\pgfqpoint{1.680701in}{5.350827in}}%
\pgfpathlineto{\pgfqpoint{1.682515in}{5.352059in}}%
\pgfpathlineto{\pgfqpoint{1.712088in}{5.371965in}}%
\pgfpathlineto{\pgfqpoint{1.719756in}{5.377087in}}%
\pgfpathlineto{\pgfqpoint{1.741661in}{5.391543in}}%
\pgfpathlineto{\pgfqpoint{1.759671in}{5.403346in}}%
\pgfpathlineto{\pgfqpoint{1.771235in}{5.410834in}}%
\pgfpathlineto{\pgfqpoint{1.800423in}{5.429606in}}%
\pgfpathlineto{\pgfqpoint{1.800808in}{5.429851in}}%
\pgfpathlineto{\pgfqpoint{1.830381in}{5.448461in}}%
\pgfpathlineto{\pgfqpoint{1.842224in}{5.455865in}}%
\pgfpathlineto{\pgfqpoint{1.859955in}{5.466815in}}%
\pgfpathlineto{\pgfqpoint{1.884900in}{5.482125in}}%
\pgfpathlineto{\pgfqpoint{1.889528in}{5.484931in}}%
\pgfpathlineto{\pgfqpoint{1.919102in}{5.502708in}}%
\pgfpathlineto{\pgfqpoint{1.928614in}{5.508384in}}%
\pgfpathlineto{\pgfqpoint{1.948675in}{5.520209in}}%
\pgfpathlineto{\pgfqpoint{1.973306in}{5.534644in}}%
\pgfpathlineto{\pgfqpoint{1.978248in}{5.537505in}}%
\pgfpathlineto{\pgfqpoint{1.996419in}{5.547933in}}%
\pgfusepath{stroke}%
\end{pgfscope}%
\begin{pgfscope}%
\pgfpathrectangle{\pgfqpoint{0.854460in}{0.571603in}}{\pgfqpoint{5.885100in}{5.225635in}}%
\pgfusepath{clip}%
\pgfsetbuttcap%
\pgfsetroundjoin%
\pgfsetlinewidth{1.505625pt}%
\definecolor{currentstroke}{rgb}{0.125394,0.574318,0.549086}%
\pgfsetstrokecolor{currentstroke}%
\pgfsetdash{}{0pt}%
\pgfpathmoveto{\pgfqpoint{2.332474in}{5.725605in}}%
\pgfpathlineto{\pgfqpoint{2.333129in}{5.725924in}}%
\pgfpathlineto{\pgfqpoint{2.362702in}{5.740247in}}%
\pgfpathlineto{\pgfqpoint{2.372011in}{5.744720in}}%
\pgfpathlineto{\pgfqpoint{2.392275in}{5.754333in}}%
\pgfpathlineto{\pgfqpoint{2.421849in}{5.768292in}}%
\pgfpathlineto{\pgfqpoint{2.427594in}{5.770979in}}%
\pgfpathlineto{\pgfqpoint{2.451422in}{5.781982in}}%
\pgfpathlineto{\pgfqpoint{2.480996in}{5.795582in}}%
\pgfpathlineto{\pgfqpoint{2.484635in}{5.797238in}}%
\pgfusepath{stroke}%
\end{pgfscope}%
\begin{pgfscope}%
\pgfpathrectangle{\pgfqpoint{0.854460in}{0.571603in}}{\pgfqpoint{5.885100in}{5.225635in}}%
\pgfusepath{clip}%
\pgfsetbuttcap%
\pgfsetroundjoin%
\pgfsetlinewidth{1.505625pt}%
\definecolor{currentstroke}{rgb}{0.125394,0.574318,0.549086}%
\pgfsetstrokecolor{currentstroke}%
\pgfsetdash{}{0pt}%
\pgfpathmoveto{\pgfqpoint{6.158977in}{5.797238in}}%
\pgfpathlineto{\pgfqpoint{6.155981in}{5.744720in}}%
\pgfpathlineto{\pgfqpoint{6.150734in}{5.692201in}}%
\pgfpathlineto{\pgfqpoint{6.143547in}{5.639682in}}%
\pgfpathlineto{\pgfqpoint{6.129836in}{5.560903in}}%
\pgfpathlineto{\pgfqpoint{6.113391in}{5.482125in}}%
\pgfpathlineto{\pgfqpoint{6.088313in}{5.377087in}}%
\pgfpathlineto{\pgfqpoint{6.053614in}{5.245790in}}%
\pgfpathlineto{\pgfqpoint{5.987486in}{5.009454in}}%
\pgfpathlineto{\pgfqpoint{5.937015in}{4.825638in}}%
\pgfpathlineto{\pgfqpoint{5.903129in}{4.694341in}}%
\pgfpathlineto{\pgfqpoint{5.877958in}{4.589303in}}%
\pgfpathlineto{\pgfqpoint{5.854844in}{4.484265in}}%
\pgfpathlineto{\pgfqpoint{5.834054in}{4.379227in}}%
\pgfpathlineto{\pgfqpoint{5.815902in}{4.274189in}}%
\pgfpathlineto{\pgfqpoint{5.800570in}{4.169151in}}%
\pgfpathlineto{\pgfqpoint{5.791066in}{4.090373in}}%
\pgfpathlineto{\pgfqpoint{5.783304in}{4.011594in}}%
\pgfpathlineto{\pgfqpoint{5.777409in}{3.932816in}}%
\pgfpathlineto{\pgfqpoint{5.773434in}{3.854037in}}%
\pgfpathlineto{\pgfqpoint{5.771429in}{3.775259in}}%
\pgfpathlineto{\pgfqpoint{5.771446in}{3.696481in}}%
\pgfpathlineto{\pgfqpoint{5.773532in}{3.617702in}}%
\pgfpathlineto{\pgfqpoint{5.777737in}{3.538924in}}%
\pgfpathlineto{\pgfqpoint{5.784112in}{3.460145in}}%
\pgfpathlineto{\pgfqpoint{5.793213in}{3.377415in}}%
\pgfpathlineto{\pgfqpoint{5.803480in}{3.302589in}}%
\pgfpathlineto{\pgfqpoint{5.816555in}{3.223810in}}%
\pgfpathlineto{\pgfqpoint{5.831910in}{3.145032in}}%
\pgfpathlineto{\pgfqpoint{5.849608in}{3.066253in}}%
\pgfpathlineto{\pgfqpoint{5.869614in}{2.987475in}}%
\pgfpathlineto{\pgfqpoint{5.892009in}{2.908696in}}%
\pgfpathlineto{\pgfqpoint{5.916796in}{2.829918in}}%
\pgfpathlineto{\pgfqpoint{5.943984in}{2.751140in}}%
\pgfpathlineto{\pgfqpoint{5.973582in}{2.672361in}}%
\pgfpathlineto{\pgfqpoint{6.005604in}{2.593583in}}%
\pgfpathlineto{\pgfqpoint{6.033868in}{2.528678in}}%
\pgfpathlineto{\pgfqpoint{6.033868in}{2.528678in}}%
\pgfusepath{stroke}%
\end{pgfscope}%
\begin{pgfscope}%
\pgfpathrectangle{\pgfqpoint{0.854460in}{0.571603in}}{\pgfqpoint{5.885100in}{5.225635in}}%
\pgfusepath{clip}%
\pgfsetbuttcap%
\pgfsetroundjoin%
\pgfsetlinewidth{1.505625pt}%
\definecolor{currentstroke}{rgb}{0.125394,0.574318,0.549086}%
\pgfsetstrokecolor{currentstroke}%
\pgfsetdash{}{0pt}%
\pgfpathmoveto{\pgfqpoint{6.212394in}{2.182862in}}%
\pgfpathlineto{\pgfqpoint{6.217887in}{2.173431in}}%
\pgfpathlineto{\pgfqpoint{6.233505in}{2.147172in}}%
\pgfpathlineto{\pgfqpoint{6.236813in}{2.141713in}}%
\pgfpathlineto{\pgfqpoint{6.249365in}{2.120912in}}%
\pgfpathlineto{\pgfqpoint{6.265544in}{2.094653in}}%
\pgfpathlineto{\pgfqpoint{6.266386in}{2.093310in}}%
\pgfpathlineto{\pgfqpoint{6.281946in}{2.068393in}}%
\pgfpathlineto{\pgfqpoint{6.295960in}{2.046404in}}%
\pgfpathlineto{\pgfqpoint{6.298671in}{2.042134in}}%
\pgfpathlineto{\pgfqpoint{6.315635in}{2.015874in}}%
\pgfpathlineto{\pgfqpoint{6.325533in}{2.000838in}}%
\pgfpathlineto{\pgfqpoint{6.332895in}{1.989615in}}%
\pgfpathlineto{\pgfqpoint{6.350434in}{1.963355in}}%
\pgfpathlineto{\pgfqpoint{6.355107in}{1.956477in}}%
\pgfpathlineto{\pgfqpoint{6.368228in}{1.937096in}}%
\pgfpathlineto{\pgfqpoint{6.384680in}{1.913244in}}%
\pgfpathlineto{\pgfqpoint{6.386335in}{1.910836in}}%
\pgfpathlineto{\pgfqpoint{6.404675in}{1.884577in}}%
\pgfpathlineto{\pgfqpoint{6.414253in}{1.871101in}}%
\pgfpathlineto{\pgfqpoint{6.423311in}{1.858318in}}%
\pgfpathlineto{\pgfqpoint{6.442242in}{1.832058in}}%
\pgfpathlineto{\pgfqpoint{6.443827in}{1.829892in}}%
\pgfpathlineto{\pgfqpoint{6.461406in}{1.805799in}}%
\pgfpathlineto{\pgfqpoint{6.473400in}{1.789638in}}%
\pgfpathlineto{\pgfqpoint{6.480874in}{1.779539in}}%
\pgfpathlineto{\pgfqpoint{6.500626in}{1.753280in}}%
\pgfpathlineto{\pgfqpoint{6.502973in}{1.750204in}}%
\pgfpathlineto{\pgfqpoint{6.520619in}{1.727020in}}%
\pgfpathlineto{\pgfqpoint{6.532547in}{1.711601in}}%
\pgfpathlineto{\pgfqpoint{6.540912in}{1.700761in}}%
\pgfpathlineto{\pgfqpoint{6.561498in}{1.674501in}}%
\pgfpathlineto{\pgfqpoint{6.562120in}{1.673718in}}%
\pgfpathlineto{\pgfqpoint{6.582313in}{1.648242in}}%
\pgfpathlineto{\pgfqpoint{6.591693in}{1.636587in}}%
\pgfpathlineto{\pgfqpoint{6.603423in}{1.621982in}}%
\pgfpathlineto{\pgfqpoint{6.621267in}{1.600098in}}%
\pgfpathlineto{\pgfqpoint{6.624827in}{1.595723in}}%
\pgfpathlineto{\pgfqpoint{6.646488in}{1.569463in}}%
\pgfpathlineto{\pgfqpoint{6.650840in}{1.564257in}}%
\pgfpathlineto{\pgfqpoint{6.668411in}{1.543204in}}%
\pgfpathlineto{\pgfqpoint{6.680414in}{1.529025in}}%
\pgfpathlineto{\pgfqpoint{6.690623in}{1.516944in}}%
\pgfpathlineto{\pgfqpoint{6.709987in}{1.494351in}}%
\pgfpathlineto{\pgfqpoint{6.713123in}{1.490685in}}%
\pgfpathlineto{\pgfqpoint{6.735880in}{1.464425in}}%
\pgfpathlineto{\pgfqpoint{6.739560in}{1.460232in}}%
\pgfusepath{stroke}%
\end{pgfscope}%
\begin{pgfscope}%
\pgfpathrectangle{\pgfqpoint{0.854460in}{0.571603in}}{\pgfqpoint{5.885100in}{5.225635in}}%
\pgfusepath{clip}%
\pgfsetbuttcap%
\pgfsetroundjoin%
\pgfsetlinewidth{1.505625pt}%
\definecolor{currentstroke}{rgb}{0.120565,0.596422,0.543611}%
\pgfsetstrokecolor{currentstroke}%
\pgfsetdash{}{0pt}%
\pgfpathmoveto{\pgfqpoint{1.160477in}{0.571603in}}%
\pgfpathlineto{\pgfqpoint{1.150194in}{0.581464in}}%
\pgfpathlineto{\pgfqpoint{1.133178in}{0.597863in}}%
\pgfpathlineto{\pgfqpoint{1.120621in}{0.610136in}}%
\pgfpathlineto{\pgfqpoint{1.106385in}{0.624122in}}%
\pgfpathlineto{\pgfqpoint{1.091047in}{0.639404in}}%
\pgfpathlineto{\pgfqpoint{1.080089in}{0.650382in}}%
\pgfpathlineto{\pgfqpoint{1.061474in}{0.669293in}}%
\pgfpathlineto{\pgfqpoint{1.054281in}{0.676641in}}%
\pgfpathlineto{\pgfqpoint{1.031901in}{0.699830in}}%
\pgfpathlineto{\pgfqpoint{1.028953in}{0.702901in}}%
\pgfpathlineto{\pgfqpoint{1.004120in}{0.729160in}}%
\pgfpathlineto{\pgfqpoint{1.002327in}{0.731084in}}%
\pgfpathlineto{\pgfqpoint{0.979793in}{0.755420in}}%
\pgfpathlineto{\pgfqpoint{0.972754in}{0.763131in}}%
\pgfpathlineto{\pgfqpoint{0.955925in}{0.781679in}}%
\pgfpathlineto{\pgfqpoint{0.943181in}{0.795929in}}%
\pgfpathlineto{\pgfqpoint{0.932506in}{0.807939in}}%
\pgfpathlineto{\pgfqpoint{0.913607in}{0.829509in}}%
\pgfpathlineto{\pgfqpoint{0.909525in}{0.834198in}}%
\pgfpathlineto{\pgfqpoint{0.887013in}{0.860458in}}%
\pgfpathlineto{\pgfqpoint{0.884034in}{0.863988in}}%
\pgfpathlineto{\pgfqpoint{0.864981in}{0.886717in}}%
\pgfpathlineto{\pgfqpoint{0.854460in}{0.899452in}}%
\pgfusepath{stroke}%
\end{pgfscope}%
\begin{pgfscope}%
\pgfpathrectangle{\pgfqpoint{0.854460in}{0.571603in}}{\pgfqpoint{5.885100in}{5.225635in}}%
\pgfusepath{clip}%
\pgfsetbuttcap%
\pgfsetroundjoin%
\pgfsetlinewidth{1.505625pt}%
\definecolor{currentstroke}{rgb}{0.120565,0.596422,0.543611}%
\pgfsetstrokecolor{currentstroke}%
\pgfsetdash{}{0pt}%
\pgfpathmoveto{\pgfqpoint{0.854460in}{4.678071in}}%
\pgfpathlineto{\pgfqpoint{0.867984in}{4.694341in}}%
\pgfpathlineto{\pgfqpoint{0.884034in}{4.713398in}}%
\pgfpathlineto{\pgfqpoint{0.890165in}{4.720600in}}%
\pgfpathlineto{\pgfqpoint{0.912817in}{4.746860in}}%
\pgfpathlineto{\pgfqpoint{0.913607in}{4.747762in}}%
\pgfpathlineto{\pgfqpoint{0.936035in}{4.773119in}}%
\pgfpathlineto{\pgfqpoint{0.943181in}{4.781096in}}%
\pgfpathlineto{\pgfqpoint{0.959732in}{4.799378in}}%
\pgfpathlineto{\pgfqpoint{0.972754in}{4.813580in}}%
\pgfpathlineto{\pgfqpoint{0.983926in}{4.825638in}}%
\pgfpathlineto{\pgfqpoint{1.002327in}{4.845249in}}%
\pgfpathlineto{\pgfqpoint{1.008629in}{4.851897in}}%
\pgfpathlineto{\pgfqpoint{1.031901in}{4.876139in}}%
\pgfpathlineto{\pgfqpoint{1.033858in}{4.878157in}}%
\pgfpathlineto{\pgfqpoint{1.059656in}{4.904416in}}%
\pgfpathlineto{\pgfqpoint{1.061474in}{4.906243in}}%
\pgfpathlineto{\pgfqpoint{1.086032in}{4.930676in}}%
\pgfpathlineto{\pgfqpoint{1.091047in}{4.935604in}}%
\pgfpathlineto{\pgfqpoint{1.112968in}{4.956935in}}%
\pgfpathlineto{\pgfqpoint{1.120621in}{4.964290in}}%
\pgfpathlineto{\pgfqpoint{1.140479in}{4.983195in}}%
\pgfpathlineto{\pgfqpoint{1.150194in}{4.992330in}}%
\pgfpathlineto{\pgfqpoint{1.168578in}{5.009454in}}%
\pgfpathlineto{\pgfqpoint{1.179767in}{5.019749in}}%
\pgfpathlineto{\pgfqpoint{1.197281in}{5.035714in}}%
\pgfpathlineto{\pgfqpoint{1.209341in}{5.046572in}}%
\pgfpathlineto{\pgfqpoint{1.226602in}{5.061973in}}%
\pgfpathlineto{\pgfqpoint{1.238914in}{5.072825in}}%
\pgfpathlineto{\pgfqpoint{1.256553in}{5.088233in}}%
\pgfpathlineto{\pgfqpoint{1.268488in}{5.098531in}}%
\pgfpathlineto{\pgfqpoint{1.287149in}{5.114492in}}%
\pgfpathlineto{\pgfqpoint{1.298061in}{5.123711in}}%
\pgfpathlineto{\pgfqpoint{1.318404in}{5.140752in}}%
\pgfpathlineto{\pgfqpoint{1.327634in}{5.148389in}}%
\pgfpathlineto{\pgfqpoint{1.350330in}{5.167011in}}%
\pgfpathlineto{\pgfqpoint{1.357208in}{5.172585in}}%
\pgfpathlineto{\pgfqpoint{1.382941in}{5.193271in}}%
\pgfpathlineto{\pgfqpoint{1.386781in}{5.196320in}}%
\pgfpathlineto{\pgfqpoint{1.416249in}{5.219530in}}%
\pgfpathlineto{\pgfqpoint{1.416354in}{5.219612in}}%
\pgfpathlineto{\pgfqpoint{1.445928in}{5.242415in}}%
\pgfpathlineto{\pgfqpoint{1.450339in}{5.245790in}}%
\pgfpathlineto{\pgfqpoint{1.475501in}{5.264804in}}%
\pgfpathlineto{\pgfqpoint{1.485164in}{5.272049in}}%
\pgfpathlineto{\pgfqpoint{1.505074in}{5.286797in}}%
\pgfpathlineto{\pgfqpoint{1.520734in}{5.298308in}}%
\pgfpathlineto{\pgfqpoint{1.534648in}{5.308413in}}%
\pgfpathlineto{\pgfqpoint{1.557060in}{5.324568in}}%
\pgfpathlineto{\pgfqpoint{1.564221in}{5.329668in}}%
\pgfpathlineto{\pgfqpoint{1.592626in}{5.349747in}}%
\pgfusepath{stroke}%
\end{pgfscope}%
\begin{pgfscope}%
\pgfpathrectangle{\pgfqpoint{0.854460in}{0.571603in}}{\pgfqpoint{5.885100in}{5.225635in}}%
\pgfusepath{clip}%
\pgfsetbuttcap%
\pgfsetroundjoin%
\pgfsetlinewidth{1.505625pt}%
\definecolor{currentstroke}{rgb}{0.120565,0.596422,0.543611}%
\pgfsetstrokecolor{currentstroke}%
\pgfsetdash{}{0pt}%
\pgfpathmoveto{\pgfqpoint{1.913581in}{5.555498in}}%
\pgfpathlineto{\pgfqpoint{1.919102in}{5.558765in}}%
\pgfpathlineto{\pgfqpoint{1.922744in}{5.560903in}}%
\pgfpathlineto{\pgfqpoint{1.948675in}{5.575934in}}%
\pgfpathlineto{\pgfqpoint{1.968151in}{5.587163in}}%
\pgfpathlineto{\pgfqpoint{1.978248in}{5.592913in}}%
\pgfpathlineto{\pgfqpoint{2.007822in}{5.609637in}}%
\pgfpathlineto{\pgfqpoint{2.014567in}{5.613422in}}%
\pgfpathlineto{\pgfqpoint{2.037395in}{5.626073in}}%
\pgfpathlineto{\pgfqpoint{2.062073in}{5.639682in}}%
\pgfpathlineto{\pgfqpoint{2.066968in}{5.642348in}}%
\pgfpathlineto{\pgfqpoint{2.096542in}{5.658320in}}%
\pgfpathlineto{\pgfqpoint{2.110737in}{5.665941in}}%
\pgfpathlineto{\pgfqpoint{2.126115in}{5.674095in}}%
\pgfpathlineto{\pgfqpoint{2.155689in}{5.689685in}}%
\pgfpathlineto{\pgfqpoint{2.160500in}{5.692201in}}%
\pgfpathlineto{\pgfqpoint{2.185262in}{5.704983in}}%
\pgfpathlineto{\pgfqpoint{2.211481in}{5.718460in}}%
\pgfpathlineto{\pgfqpoint{2.214835in}{5.720163in}}%
\pgfpathlineto{\pgfqpoint{2.244409in}{5.735040in}}%
\pgfpathlineto{\pgfqpoint{2.263744in}{5.744720in}}%
\pgfpathlineto{\pgfqpoint{2.273982in}{5.749781in}}%
\pgfpathlineto{\pgfqpoint{2.303555in}{5.764295in}}%
\pgfpathlineto{\pgfqpoint{2.317263in}{5.770979in}}%
\pgfpathlineto{\pgfqpoint{2.333129in}{5.778618in}}%
\pgfpathlineto{\pgfqpoint{2.362702in}{5.792774in}}%
\pgfpathlineto{\pgfqpoint{2.372101in}{5.797238in}}%
\pgfusepath{stroke}%
\end{pgfscope}%
\begin{pgfscope}%
\pgfpathrectangle{\pgfqpoint{0.854460in}{0.571603in}}{\pgfqpoint{5.885100in}{5.225635in}}%
\pgfusepath{clip}%
\pgfsetbuttcap%
\pgfsetroundjoin%
\pgfsetlinewidth{1.505625pt}%
\definecolor{currentstroke}{rgb}{0.120565,0.596422,0.543611}%
\pgfsetstrokecolor{currentstroke}%
\pgfsetdash{}{0pt}%
\pgfpathmoveto{\pgfqpoint{6.287072in}{5.797238in}}%
\pgfpathlineto{\pgfqpoint{6.283813in}{5.770979in}}%
\pgfpathlineto{\pgfqpoint{6.280061in}{5.744720in}}%
\pgfpathlineto{\pgfqpoint{6.275859in}{5.718460in}}%
\pgfpathlineto{\pgfqpoint{6.271247in}{5.692201in}}%
\pgfpathlineto{\pgfqpoint{6.266386in}{5.666612in}}%
\pgfpathlineto{\pgfqpoint{6.266259in}{5.665941in}}%
\pgfpathlineto{\pgfqpoint{6.260865in}{5.639682in}}%
\pgfpathlineto{\pgfqpoint{6.255161in}{5.613422in}}%
\pgfpathlineto{\pgfqpoint{6.249176in}{5.587163in}}%
\pgfpathlineto{\pgfqpoint{6.242937in}{5.560903in}}%
\pgfpathlineto{\pgfqpoint{6.236813in}{5.536057in}}%
\pgfpathlineto{\pgfqpoint{6.236465in}{5.534644in}}%
\pgfpathlineto{\pgfqpoint{6.229713in}{5.508384in}}%
\pgfpathlineto{\pgfqpoint{6.222779in}{5.482125in}}%
\pgfpathlineto{\pgfqpoint{6.215683in}{5.455865in}}%
\pgfpathlineto{\pgfqpoint{6.208445in}{5.429606in}}%
\pgfpathlineto{\pgfqpoint{6.207240in}{5.425350in}}%
\pgfpathlineto{\pgfqpoint{6.201012in}{5.403346in}}%
\pgfpathlineto{\pgfqpoint{6.193459in}{5.377087in}}%
\pgfpathlineto{\pgfqpoint{6.185816in}{5.350827in}}%
\pgfpathlineto{\pgfqpoint{6.178098in}{5.324568in}}%
\pgfpathlineto{\pgfqpoint{6.177666in}{5.323124in}}%
\pgfpathlineto{\pgfqpoint{6.170238in}{5.298308in}}%
\pgfpathlineto{\pgfqpoint{6.162327in}{5.272049in}}%
\pgfpathlineto{\pgfqpoint{6.154383in}{5.245790in}}%
\pgfpathlineto{\pgfqpoint{6.148093in}{5.225097in}}%
\pgfpathlineto{\pgfqpoint{6.146400in}{5.219530in}}%
\pgfpathlineto{\pgfqpoint{6.138338in}{5.193271in}}%
\pgfpathlineto{\pgfqpoint{6.130279in}{5.167011in}}%
\pgfpathlineto{\pgfqpoint{6.122233in}{5.140752in}}%
\pgfpathlineto{\pgfqpoint{6.118520in}{5.128664in}}%
\pgfpathlineto{\pgfqpoint{6.114162in}{5.114492in}}%
\pgfpathlineto{\pgfqpoint{6.106084in}{5.088233in}}%
\pgfpathlineto{\pgfqpoint{6.098047in}{5.061973in}}%
\pgfpathlineto{\pgfqpoint{6.090059in}{5.035714in}}%
\pgfpathlineto{\pgfqpoint{6.088946in}{5.032057in}}%
\pgfpathlineto{\pgfqpoint{6.082057in}{5.009454in}}%
\pgfpathlineto{\pgfqpoint{6.074110in}{4.983195in}}%
\pgfpathlineto{\pgfqpoint{6.066236in}{4.956935in}}%
\pgfpathlineto{\pgfqpoint{6.059373in}{4.933839in}}%
\pgfpathlineto{\pgfqpoint{6.058431in}{4.930676in}}%
\pgfpathlineto{\pgfqpoint{6.050644in}{4.904416in}}%
\pgfpathlineto{\pgfqpoint{6.042950in}{4.878157in}}%
\pgfpathlineto{\pgfqpoint{6.035356in}{4.851897in}}%
\pgfpathlineto{\pgfqpoint{6.029799in}{4.832455in}}%
\pgfpathlineto{\pgfqpoint{6.027847in}{4.825638in}}%
\pgfpathlineto{\pgfqpoint{6.020394in}{4.799378in}}%
\pgfpathlineto{\pgfqpoint{6.013057in}{4.773119in}}%
\pgfpathlineto{\pgfqpoint{6.005842in}{4.746860in}}%
\pgfpathlineto{\pgfqpoint{6.000226in}{4.726087in}}%
\pgfpathlineto{\pgfqpoint{5.998739in}{4.720600in}}%
\pgfpathlineto{\pgfqpoint{5.991711in}{4.694341in}}%
\pgfpathlineto{\pgfqpoint{5.984821in}{4.668081in}}%
\pgfpathlineto{\pgfqpoint{5.978070in}{4.641822in}}%
\pgfpathlineto{\pgfqpoint{5.971464in}{4.615562in}}%
\pgfpathlineto{\pgfqpoint{5.970653in}{4.612283in}}%
\pgfpathlineto{\pgfqpoint{5.964950in}{4.589303in}}%
\pgfpathlineto{\pgfqpoint{5.958582in}{4.563043in}}%
\pgfpathlineto{\pgfqpoint{5.952370in}{4.536784in}}%
\pgfpathlineto{\pgfqpoint{5.946318in}{4.510524in}}%
\pgfpathlineto{\pgfqpoint{5.941079in}{4.487183in}}%
\pgfpathlineto{\pgfqpoint{5.940422in}{4.484265in}}%
\pgfpathlineto{\pgfqpoint{5.934644in}{4.458005in}}%
\pgfpathlineto{\pgfqpoint{5.929036in}{4.431746in}}%
\pgfpathlineto{\pgfqpoint{5.923601in}{4.405486in}}%
\pgfpathlineto{\pgfqpoint{5.918343in}{4.379227in}}%
\pgfpathlineto{\pgfqpoint{5.913263in}{4.352967in}}%
\pgfpathlineto{\pgfqpoint{5.911506in}{4.343583in}}%
\pgfpathlineto{\pgfqpoint{5.908334in}{4.326708in}}%
\pgfpathlineto{\pgfqpoint{5.903574in}{4.300449in}}%
\pgfpathlineto{\pgfqpoint{5.899002in}{4.274189in}}%
\pgfpathlineto{\pgfqpoint{5.894621in}{4.247930in}}%
\pgfpathlineto{\pgfqpoint{5.890431in}{4.221670in}}%
\pgfpathlineto{\pgfqpoint{5.886436in}{4.195411in}}%
\pgfpathlineto{\pgfqpoint{5.882637in}{4.169151in}}%
\pgfpathlineto{\pgfqpoint{5.881933in}{4.164027in}}%
\pgfpathlineto{\pgfqpoint{5.879011in}{4.142892in}}%
\pgfpathlineto{\pgfqpoint{5.875580in}{4.116632in}}%
\pgfpathlineto{\pgfqpoint{5.872355in}{4.090373in}}%
\pgfpathlineto{\pgfqpoint{5.869335in}{4.064113in}}%
\pgfpathlineto{\pgfqpoint{5.866524in}{4.037854in}}%
\pgfpathlineto{\pgfqpoint{5.863924in}{4.011594in}}%
\pgfpathlineto{\pgfqpoint{5.861535in}{3.985335in}}%
\pgfpathlineto{\pgfqpoint{5.859361in}{3.959075in}}%
\pgfpathlineto{\pgfqpoint{5.857402in}{3.932816in}}%
\pgfpathlineto{\pgfqpoint{5.855661in}{3.906556in}}%
\pgfpathlineto{\pgfqpoint{5.854139in}{3.880297in}}%
\pgfpathlineto{\pgfqpoint{5.852838in}{3.854037in}}%
\pgfpathlineto{\pgfqpoint{5.852359in}{3.842403in}}%
\pgfpathlineto{\pgfqpoint{5.851754in}{3.827778in}}%
\pgfpathlineto{\pgfqpoint{5.850893in}{3.801519in}}%
\pgfpathlineto{\pgfqpoint{5.850259in}{3.775259in}}%
\pgfpathlineto{\pgfqpoint{5.849856in}{3.749000in}}%
\pgfpathlineto{\pgfqpoint{5.849684in}{3.722740in}}%
\pgfpathlineto{\pgfqpoint{5.849746in}{3.696481in}}%
\pgfpathlineto{\pgfqpoint{5.850042in}{3.670221in}}%
\pgfpathlineto{\pgfqpoint{5.850575in}{3.643962in}}%
\pgfpathlineto{\pgfqpoint{5.851347in}{3.617702in}}%
\pgfpathlineto{\pgfqpoint{5.852359in}{3.591443in}}%
\pgfpathlineto{\pgfqpoint{5.852359in}{3.591434in}}%
\pgfpathlineto{\pgfqpoint{5.853602in}{3.565183in}}%
\pgfpathlineto{\pgfqpoint{5.855086in}{3.538924in}}%
\pgfpathlineto{\pgfqpoint{5.856814in}{3.512664in}}%
\pgfpathlineto{\pgfqpoint{5.858786in}{3.486405in}}%
\pgfpathlineto{\pgfqpoint{5.861005in}{3.460145in}}%
\pgfpathlineto{\pgfqpoint{5.863473in}{3.433886in}}%
\pgfpathlineto{\pgfqpoint{5.866192in}{3.407626in}}%
\pgfpathlineto{\pgfqpoint{5.869162in}{3.381367in}}%
\pgfpathlineto{\pgfqpoint{5.872386in}{3.355107in}}%
\pgfpathlineto{\pgfqpoint{5.875867in}{3.328848in}}%
\pgfpathlineto{\pgfqpoint{5.879605in}{3.302589in}}%
\pgfpathlineto{\pgfqpoint{5.881933in}{3.287299in}}%
\pgfpathlineto{\pgfqpoint{5.883588in}{3.276329in}}%
\pgfpathlineto{\pgfqpoint{5.887811in}{3.250070in}}%
\pgfpathlineto{\pgfqpoint{5.892296in}{3.223810in}}%
\pgfpathlineto{\pgfqpoint{5.897044in}{3.197551in}}%
\pgfpathlineto{\pgfqpoint{5.902057in}{3.171291in}}%
\pgfpathlineto{\pgfqpoint{5.907338in}{3.145032in}}%
\pgfpathlineto{\pgfqpoint{5.911506in}{3.125310in}}%
\pgfpathlineto{\pgfqpoint{5.912876in}{3.118772in}}%
\pgfpathlineto{\pgfqpoint{5.918649in}{3.092513in}}%
\pgfpathlineto{\pgfqpoint{5.924694in}{3.066253in}}%
\pgfpathlineto{\pgfqpoint{5.929870in}{3.044742in}}%
\pgfusepath{stroke}%
\end{pgfscope}%
\begin{pgfscope}%
\pgfpathrectangle{\pgfqpoint{0.854460in}{0.571603in}}{\pgfqpoint{5.885100in}{5.225635in}}%
\pgfusepath{clip}%
\pgfsetbuttcap%
\pgfsetroundjoin%
\pgfsetlinewidth{1.505625pt}%
\definecolor{currentstroke}{rgb}{0.120565,0.596422,0.543611}%
\pgfsetstrokecolor{currentstroke}%
\pgfsetdash{}{0pt}%
\pgfpathmoveto{\pgfqpoint{6.047648in}{2.671808in}}%
\pgfpathlineto{\pgfqpoint{6.057793in}{2.646102in}}%
\pgfpathlineto{\pgfqpoint{6.059373in}{2.642205in}}%
\pgfpathlineto{\pgfqpoint{6.068386in}{2.619842in}}%
\pgfpathlineto{\pgfqpoint{6.079271in}{2.593583in}}%
\pgfpathlineto{\pgfqpoint{6.088946in}{2.570877in}}%
\pgfpathlineto{\pgfqpoint{6.090452in}{2.567323in}}%
\pgfpathlineto{\pgfqpoint{6.101861in}{2.541064in}}%
\pgfpathlineto{\pgfqpoint{6.113582in}{2.514804in}}%
\pgfpathlineto{\pgfqpoint{6.118520in}{2.504014in}}%
\pgfpathlineto{\pgfqpoint{6.125559in}{2.488545in}}%
\pgfpathlineto{\pgfqpoint{6.137811in}{2.462285in}}%
\pgfpathlineto{\pgfqpoint{6.148093in}{2.440793in}}%
\pgfpathlineto{\pgfqpoint{6.150362in}{2.436026in}}%
\pgfpathlineto{\pgfqpoint{6.163148in}{2.409766in}}%
\pgfpathlineto{\pgfqpoint{6.176256in}{2.383507in}}%
\pgfpathlineto{\pgfqpoint{6.177666in}{2.380743in}}%
\pgfpathlineto{\pgfqpoint{6.189592in}{2.357248in}}%
\pgfpathlineto{\pgfqpoint{6.203243in}{2.330988in}}%
\pgfpathlineto{\pgfqpoint{6.207240in}{2.323463in}}%
\pgfpathlineto{\pgfqpoint{6.217143in}{2.304729in}}%
\pgfpathlineto{\pgfqpoint{6.231340in}{2.278469in}}%
\pgfpathlineto{\pgfqpoint{6.236813in}{2.268558in}}%
\pgfpathlineto{\pgfqpoint{6.245799in}{2.252210in}}%
\pgfpathlineto{\pgfqpoint{6.260548in}{2.225950in}}%
\pgfpathlineto{\pgfqpoint{6.266386in}{2.215765in}}%
\pgfpathlineto{\pgfqpoint{6.275561in}{2.199691in}}%
\pgfpathlineto{\pgfqpoint{6.290867in}{2.173431in}}%
\pgfpathlineto{\pgfqpoint{6.295960in}{2.164863in}}%
\pgfpathlineto{\pgfqpoint{6.306431in}{2.147172in}}%
\pgfpathlineto{\pgfqpoint{6.322299in}{2.120912in}}%
\pgfpathlineto{\pgfqpoint{6.325533in}{2.115658in}}%
\pgfpathlineto{\pgfqpoint{6.338412in}{2.094653in}}%
\pgfpathlineto{\pgfqpoint{6.354847in}{2.068393in}}%
\pgfpathlineto{\pgfqpoint{6.355107in}{2.067986in}}%
\pgfpathlineto{\pgfqpoint{6.371504in}{2.042134in}}%
\pgfpathlineto{\pgfqpoint{6.384680in}{2.021773in}}%
\pgfpathlineto{\pgfqpoint{6.388483in}{2.015874in}}%
\pgfpathlineto{\pgfqpoint{6.405713in}{1.989615in}}%
\pgfpathlineto{\pgfqpoint{6.414253in}{1.976836in}}%
\pgfpathlineto{\pgfqpoint{6.423232in}{1.963355in}}%
\pgfpathlineto{\pgfqpoint{6.441043in}{1.937096in}}%
\pgfpathlineto{\pgfqpoint{6.443827in}{1.933059in}}%
\pgfpathlineto{\pgfqpoint{6.459098in}{1.910836in}}%
\pgfpathlineto{\pgfqpoint{6.473400in}{1.890404in}}%
\pgfpathlineto{\pgfqpoint{6.477466in}{1.884577in}}%
\pgfpathlineto{\pgfqpoint{6.496090in}{1.858318in}}%
\pgfpathlineto{\pgfqpoint{6.502973in}{1.848771in}}%
\pgfpathlineto{\pgfqpoint{6.514990in}{1.832058in}}%
\pgfpathlineto{\pgfqpoint{6.532547in}{1.808062in}}%
\pgfpathlineto{\pgfqpoint{6.534198in}{1.805799in}}%
\pgfpathlineto{\pgfqpoint{6.553644in}{1.779539in}}%
\pgfpathlineto{\pgfqpoint{6.562120in}{1.768279in}}%
\pgfpathlineto{\pgfqpoint{6.573381in}{1.753280in}}%
\pgfpathlineto{\pgfqpoint{6.591693in}{1.729289in}}%
\pgfpathlineto{\pgfqpoint{6.593421in}{1.727020in}}%
\pgfpathlineto{\pgfqpoint{6.613701in}{1.700761in}}%
\pgfpathlineto{\pgfqpoint{6.621267in}{1.691114in}}%
\pgfpathlineto{\pgfqpoint{6.634267in}{1.674501in}}%
\pgfpathlineto{\pgfqpoint{6.650840in}{1.653652in}}%
\pgfpathlineto{\pgfqpoint{6.655132in}{1.648242in}}%
\pgfpathlineto{\pgfqpoint{6.676259in}{1.621982in}}%
\pgfpathlineto{\pgfqpoint{6.680414in}{1.616890in}}%
\pgfpathlineto{\pgfqpoint{6.697649in}{1.595723in}}%
\pgfpathlineto{\pgfqpoint{6.709987in}{1.580793in}}%
\pgfpathlineto{\pgfqpoint{6.719333in}{1.569463in}}%
\pgfpathlineto{\pgfqpoint{6.739560in}{1.545296in}}%
\pgfusepath{stroke}%
\end{pgfscope}%
\begin{pgfscope}%
\pgfpathrectangle{\pgfqpoint{0.854460in}{0.571603in}}{\pgfqpoint{5.885100in}{5.225635in}}%
\pgfusepath{clip}%
\pgfsetbuttcap%
\pgfsetroundjoin%
\pgfsetlinewidth{1.505625pt}%
\definecolor{currentstroke}{rgb}{0.119483,0.614817,0.537692}%
\pgfsetstrokecolor{currentstroke}%
\pgfsetdash{}{0pt}%
\pgfpathmoveto{\pgfqpoint{1.102532in}{0.571603in}}%
\pgfpathlineto{\pgfqpoint{1.091047in}{0.582713in}}%
\pgfpathlineto{\pgfqpoint{1.075466in}{0.597863in}}%
\pgfpathlineto{\pgfqpoint{1.061474in}{0.611659in}}%
\pgfpathlineto{\pgfqpoint{1.048901in}{0.624122in}}%
\pgfpathlineto{\pgfqpoint{1.031901in}{0.641211in}}%
\pgfpathlineto{\pgfqpoint{1.022827in}{0.650382in}}%
\pgfpathlineto{\pgfqpoint{1.002327in}{0.671392in}}%
\pgfpathlineto{\pgfqpoint{0.997235in}{0.676641in}}%
\pgfpathlineto{\pgfqpoint{0.972754in}{0.702231in}}%
\pgfpathlineto{\pgfqpoint{0.972117in}{0.702901in}}%
\pgfpathlineto{\pgfqpoint{0.947520in}{0.729160in}}%
\pgfpathlineto{\pgfqpoint{0.943181in}{0.733860in}}%
\pgfpathlineto{\pgfqpoint{0.923394in}{0.755420in}}%
\pgfpathlineto{\pgfqpoint{0.913607in}{0.766235in}}%
\pgfpathlineto{\pgfqpoint{0.899719in}{0.781679in}}%
\pgfpathlineto{\pgfqpoint{0.884034in}{0.799372in}}%
\pgfpathlineto{\pgfqpoint{0.876487in}{0.807939in}}%
\pgfpathlineto{\pgfqpoint{0.854460in}{0.833302in}}%
\pgfusepath{stroke}%
\end{pgfscope}%
\begin{pgfscope}%
\pgfpathrectangle{\pgfqpoint{0.854460in}{0.571603in}}{\pgfqpoint{5.885100in}{5.225635in}}%
\pgfusepath{clip}%
\pgfsetbuttcap%
\pgfsetroundjoin%
\pgfsetlinewidth{1.505625pt}%
\definecolor{currentstroke}{rgb}{0.119483,0.614817,0.537692}%
\pgfsetstrokecolor{currentstroke}%
\pgfsetdash{}{0pt}%
\pgfpathmoveto{\pgfqpoint{0.854460in}{4.757175in}}%
\pgfpathlineto{\pgfqpoint{0.868311in}{4.773119in}}%
\pgfpathlineto{\pgfqpoint{0.884034in}{4.790986in}}%
\pgfpathlineto{\pgfqpoint{0.891496in}{4.799378in}}%
\pgfpathlineto{\pgfqpoint{0.913607in}{4.823928in}}%
\pgfpathlineto{\pgfqpoint{0.915163in}{4.825638in}}%
\pgfpathlineto{\pgfqpoint{0.939387in}{4.851897in}}%
\pgfpathlineto{\pgfqpoint{0.943181in}{4.855957in}}%
\pgfpathlineto{\pgfqpoint{0.964138in}{4.878157in}}%
\pgfpathlineto{\pgfqpoint{0.972754in}{4.887169in}}%
\pgfpathlineto{\pgfqpoint{0.989407in}{4.904416in}}%
\pgfpathlineto{\pgfqpoint{1.002327in}{4.917631in}}%
\pgfpathlineto{\pgfqpoint{1.015206in}{4.930676in}}%
\pgfpathlineto{\pgfqpoint{1.031901in}{4.947374in}}%
\pgfpathlineto{\pgfqpoint{1.041552in}{4.956935in}}%
\pgfpathlineto{\pgfqpoint{1.061474in}{4.976427in}}%
\pgfpathlineto{\pgfqpoint{1.068457in}{4.983195in}}%
\pgfpathlineto{\pgfqpoint{1.091047in}{5.004817in}}%
\pgfpathlineto{\pgfqpoint{1.095937in}{5.009454in}}%
\pgfpathlineto{\pgfqpoint{1.120621in}{5.032573in}}%
\pgfpathlineto{\pgfqpoint{1.124005in}{5.035714in}}%
\pgfpathlineto{\pgfqpoint{1.150194in}{5.059719in}}%
\pgfpathlineto{\pgfqpoint{1.152676in}{5.061973in}}%
\pgfpathlineto{\pgfqpoint{1.179767in}{5.086281in}}%
\pgfpathlineto{\pgfqpoint{1.181962in}{5.088233in}}%
\pgfpathlineto{\pgfqpoint{1.209341in}{5.112283in}}%
\pgfpathlineto{\pgfqpoint{1.211878in}{5.114492in}}%
\pgfpathlineto{\pgfqpoint{1.223297in}{5.124314in}}%
\pgfusepath{stroke}%
\end{pgfscope}%
\begin{pgfscope}%
\pgfpathrectangle{\pgfqpoint{0.854460in}{0.571603in}}{\pgfqpoint{5.885100in}{5.225635in}}%
\pgfusepath{clip}%
\pgfsetbuttcap%
\pgfsetroundjoin%
\pgfsetlinewidth{1.505625pt}%
\definecolor{currentstroke}{rgb}{0.119483,0.614817,0.537692}%
\pgfsetstrokecolor{currentstroke}%
\pgfsetdash{}{0pt}%
\pgfpathmoveto{\pgfqpoint{1.524648in}{5.360032in}}%
\pgfpathlineto{\pgfqpoint{1.534648in}{5.367143in}}%
\pgfpathlineto{\pgfqpoint{1.548731in}{5.377087in}}%
\pgfpathlineto{\pgfqpoint{1.564221in}{5.387892in}}%
\pgfpathlineto{\pgfqpoint{1.586530in}{5.403346in}}%
\pgfpathlineto{\pgfqpoint{1.593795in}{5.408318in}}%
\pgfpathlineto{\pgfqpoint{1.623368in}{5.428412in}}%
\pgfpathlineto{\pgfqpoint{1.625140in}{5.429606in}}%
\pgfpathlineto{\pgfqpoint{1.652941in}{5.448108in}}%
\pgfpathlineto{\pgfqpoint{1.664673in}{5.455865in}}%
\pgfpathlineto{\pgfqpoint{1.682515in}{5.467519in}}%
\pgfpathlineto{\pgfqpoint{1.705018in}{5.482125in}}%
\pgfpathlineto{\pgfqpoint{1.712088in}{5.486658in}}%
\pgfpathlineto{\pgfqpoint{1.741661in}{5.505482in}}%
\pgfpathlineto{\pgfqpoint{1.746257in}{5.508384in}}%
\pgfpathlineto{\pgfqpoint{1.771235in}{5.523966in}}%
\pgfpathlineto{\pgfqpoint{1.788453in}{5.534644in}}%
\pgfpathlineto{\pgfqpoint{1.800808in}{5.542212in}}%
\pgfpathlineto{\pgfqpoint{1.830381in}{5.560220in}}%
\pgfpathlineto{\pgfqpoint{1.831514in}{5.560903in}}%
\pgfpathlineto{\pgfqpoint{1.859955in}{5.577859in}}%
\pgfpathlineto{\pgfqpoint{1.875646in}{5.587163in}}%
\pgfpathlineto{\pgfqpoint{1.889528in}{5.595293in}}%
\pgfpathlineto{\pgfqpoint{1.919102in}{5.612515in}}%
\pgfpathlineto{\pgfqpoint{1.920673in}{5.613422in}}%
\pgfpathlineto{\pgfqpoint{1.948675in}{5.629387in}}%
\pgfpathlineto{\pgfqpoint{1.966821in}{5.639682in}}%
\pgfpathlineto{\pgfqpoint{1.978248in}{5.646084in}}%
\pgfpathlineto{\pgfqpoint{2.007822in}{5.662549in}}%
\pgfpathlineto{\pgfqpoint{2.013962in}{5.665941in}}%
\pgfpathlineto{\pgfqpoint{2.037395in}{5.678727in}}%
\pgfpathlineto{\pgfqpoint{2.062201in}{5.692201in}}%
\pgfpathlineto{\pgfqpoint{2.066968in}{5.694758in}}%
\pgfpathlineto{\pgfqpoint{2.096542in}{5.710493in}}%
\pgfpathlineto{\pgfqpoint{2.111600in}{5.718460in}}%
\pgfpathlineto{\pgfqpoint{2.126115in}{5.726045in}}%
\pgfpathlineto{\pgfqpoint{2.155689in}{5.741410in}}%
\pgfpathlineto{\pgfqpoint{2.162109in}{5.744720in}}%
\pgfpathlineto{\pgfqpoint{2.185262in}{5.756506in}}%
\pgfpathlineto{\pgfqpoint{2.213800in}{5.770979in}}%
\pgfpathlineto{\pgfqpoint{2.214835in}{5.771497in}}%
\pgfpathlineto{\pgfqpoint{2.244409in}{5.786170in}}%
\pgfpathlineto{\pgfqpoint{2.266804in}{5.797238in}}%
\pgfusepath{stroke}%
\end{pgfscope}%
\begin{pgfscope}%
\pgfpathrectangle{\pgfqpoint{0.854460in}{0.571603in}}{\pgfqpoint{5.885100in}{5.225635in}}%
\pgfusepath{clip}%
\pgfsetbuttcap%
\pgfsetroundjoin%
\pgfsetlinewidth{1.505625pt}%
\definecolor{currentstroke}{rgb}{0.119483,0.614817,0.537692}%
\pgfsetstrokecolor{currentstroke}%
\pgfsetdash{}{0pt}%
\pgfpathmoveto{\pgfqpoint{6.407925in}{5.797238in}}%
\pgfpathlineto{\pgfqpoint{6.402925in}{5.770979in}}%
\pgfpathlineto{\pgfqpoint{6.397530in}{5.744720in}}%
\pgfpathlineto{\pgfqpoint{6.391775in}{5.718460in}}%
\pgfpathlineto{\pgfqpoint{6.385694in}{5.692201in}}%
\pgfpathlineto{\pgfqpoint{6.384680in}{5.688076in}}%
\pgfpathlineto{\pgfqpoint{6.379250in}{5.665941in}}%
\pgfpathlineto{\pgfqpoint{6.372525in}{5.639682in}}%
\pgfpathlineto{\pgfqpoint{6.365558in}{5.613422in}}%
\pgfpathlineto{\pgfqpoint{6.358373in}{5.587163in}}%
\pgfpathlineto{\pgfqpoint{6.355107in}{5.575626in}}%
\pgfpathlineto{\pgfqpoint{6.350945in}{5.560903in}}%
\pgfpathlineto{\pgfqpoint{6.343305in}{5.534644in}}%
\pgfpathlineto{\pgfqpoint{6.335514in}{5.508384in}}%
\pgfpathlineto{\pgfqpoint{6.327588in}{5.482125in}}%
\pgfpathlineto{\pgfqpoint{6.325533in}{5.475474in}}%
\pgfpathlineto{\pgfqpoint{6.319480in}{5.455865in}}%
\pgfpathlineto{\pgfqpoint{6.311249in}{5.429606in}}%
\pgfpathlineto{\pgfqpoint{6.302937in}{5.403346in}}%
\pgfpathlineto{\pgfqpoint{6.295960in}{5.381526in}}%
\pgfpathlineto{\pgfqpoint{6.294541in}{5.377087in}}%
\pgfpathlineto{\pgfqpoint{6.286015in}{5.350827in}}%
\pgfpathlineto{\pgfqpoint{6.277450in}{5.324568in}}%
\pgfpathlineto{\pgfqpoint{6.268858in}{5.298308in}}%
\pgfpathlineto{\pgfqpoint{6.266386in}{5.290826in}}%
\pgfpathlineto{\pgfqpoint{6.260184in}{5.272049in}}%
\pgfpathlineto{\pgfqpoint{6.251481in}{5.245790in}}%
\pgfpathlineto{\pgfqpoint{6.242786in}{5.219530in}}%
\pgfpathlineto{\pgfqpoint{6.236813in}{5.201512in}}%
\pgfpathlineto{\pgfqpoint{6.234079in}{5.193271in}}%
\pgfpathlineto{\pgfqpoint{6.225338in}{5.167011in}}%
\pgfpathlineto{\pgfqpoint{6.216634in}{5.140752in}}%
\pgfpathlineto{\pgfqpoint{6.207976in}{5.114492in}}%
\pgfpathlineto{\pgfqpoint{6.207240in}{5.112265in}}%
\pgfpathlineto{\pgfqpoint{6.199290in}{5.088233in}}%
\pgfpathlineto{\pgfqpoint{6.190659in}{5.061973in}}%
\pgfpathlineto{\pgfqpoint{6.182099in}{5.035714in}}%
\pgfpathlineto{\pgfqpoint{6.177666in}{5.022052in}}%
\pgfpathlineto{\pgfqpoint{6.173573in}{5.009454in}}%
\pgfpathlineto{\pgfqpoint{6.165088in}{4.983195in}}%
\pgfpathlineto{\pgfqpoint{6.156694in}{4.956935in}}%
\pgfpathlineto{\pgfqpoint{6.148396in}{4.930676in}}%
\pgfpathlineto{\pgfqpoint{6.148093in}{4.929711in}}%
\pgfpathlineto{\pgfqpoint{6.140123in}{4.904416in}}%
\pgfpathlineto{\pgfqpoint{6.131956in}{4.878157in}}%
\pgfpathlineto{\pgfqpoint{6.123904in}{4.851897in}}%
\pgfpathlineto{\pgfqpoint{6.118520in}{4.834118in}}%
\pgfpathlineto{\pgfqpoint{6.115946in}{4.825638in}}%
\pgfpathlineto{\pgfqpoint{6.108061in}{4.799378in}}%
\pgfpathlineto{\pgfqpoint{6.100306in}{4.773119in}}%
\pgfpathlineto{\pgfqpoint{6.092685in}{4.746860in}}%
\pgfpathlineto{\pgfqpoint{6.088946in}{4.733784in}}%
\pgfpathlineto{\pgfqpoint{6.086734in}{4.726070in}}%
\pgfusepath{stroke}%
\end{pgfscope}%
\begin{pgfscope}%
\pgfpathrectangle{\pgfqpoint{0.854460in}{0.571603in}}{\pgfqpoint{5.885100in}{5.225635in}}%
\pgfusepath{clip}%
\pgfsetbuttcap%
\pgfsetroundjoin%
\pgfsetlinewidth{1.505625pt}%
\definecolor{currentstroke}{rgb}{0.119483,0.614817,0.537692}%
\pgfsetstrokecolor{currentstroke}%
\pgfsetdash{}{0pt}%
\pgfpathmoveto{\pgfqpoint{5.993560in}{4.345453in}}%
\pgfpathlineto{\pgfqpoint{5.980004in}{4.274189in}}%
\pgfpathlineto{\pgfqpoint{5.962623in}{4.169151in}}%
\pgfpathlineto{\pgfqpoint{5.951678in}{4.090373in}}%
\pgfpathlineto{\pgfqpoint{5.941079in}{3.995821in}}%
\pgfpathlineto{\pgfqpoint{5.935487in}{3.932816in}}%
\pgfpathlineto{\pgfqpoint{5.930336in}{3.854037in}}%
\pgfpathlineto{\pgfqpoint{5.927236in}{3.775259in}}%
\pgfpathlineto{\pgfqpoint{5.926232in}{3.696481in}}%
\pgfpathlineto{\pgfqpoint{5.927366in}{3.617702in}}%
\pgfpathlineto{\pgfqpoint{5.930681in}{3.538924in}}%
\pgfpathlineto{\pgfqpoint{5.936222in}{3.460145in}}%
\pgfpathlineto{\pgfqpoint{5.944009in}{3.381367in}}%
\pgfpathlineto{\pgfqpoint{5.954054in}{3.302589in}}%
\pgfpathlineto{\pgfqpoint{5.966448in}{3.223810in}}%
\pgfpathlineto{\pgfqpoint{5.981152in}{3.145032in}}%
\pgfpathlineto{\pgfqpoint{6.000226in}{3.057968in}}%
\pgfpathlineto{\pgfqpoint{6.017701in}{2.987475in}}%
\pgfpathlineto{\pgfqpoint{6.039579in}{2.908696in}}%
\pgfpathlineto{\pgfqpoint{6.063889in}{2.829918in}}%
\pgfpathlineto{\pgfqpoint{6.090638in}{2.751140in}}%
\pgfpathlineto{\pgfqpoint{6.119834in}{2.672361in}}%
\pgfpathlineto{\pgfqpoint{6.151490in}{2.593583in}}%
\pgfpathlineto{\pgfqpoint{6.185620in}{2.514804in}}%
\pgfpathlineto{\pgfqpoint{6.222243in}{2.436026in}}%
\pgfpathlineto{\pgfqpoint{6.266386in}{2.347606in}}%
\pgfpathlineto{\pgfqpoint{6.303006in}{2.278469in}}%
\pgfpathlineto{\pgfqpoint{6.347141in}{2.199691in}}%
\pgfpathlineto{\pgfqpoint{6.393789in}{2.120912in}}%
\pgfpathlineto{\pgfqpoint{6.443827in}{2.040817in}}%
\pgfpathlineto{\pgfqpoint{6.502973in}{1.951069in}}%
\pgfpathlineto{\pgfqpoint{6.548833in}{1.884577in}}%
\pgfpathlineto{\pgfqpoint{6.605551in}{1.805799in}}%
\pgfpathlineto{\pgfqpoint{6.664786in}{1.727020in}}%
\pgfpathlineto{\pgfqpoint{6.739560in}{1.632079in}}%
\pgfpathlineto{\pgfqpoint{6.739560in}{1.632079in}}%
\pgfusepath{stroke}%
\end{pgfscope}%
\begin{pgfscope}%
\pgfpathrectangle{\pgfqpoint{0.854460in}{0.571603in}}{\pgfqpoint{5.885100in}{5.225635in}}%
\pgfusepath{clip}%
\pgfsetbuttcap%
\pgfsetroundjoin%
\pgfsetlinewidth{1.505625pt}%
\definecolor{currentstroke}{rgb}{0.123444,0.636809,0.528763}%
\pgfsetstrokecolor{currentstroke}%
\pgfsetdash{}{0pt}%
\pgfpathmoveto{\pgfqpoint{1.046014in}{0.571603in}}%
\pgfpathlineto{\pgfqpoint{1.031901in}{0.585377in}}%
\pgfpathlineto{\pgfqpoint{1.019173in}{0.597863in}}%
\pgfpathlineto{\pgfqpoint{1.002327in}{0.614621in}}%
\pgfpathlineto{\pgfqpoint{0.992827in}{0.624122in}}%
\pgfpathlineto{\pgfqpoint{0.972754in}{0.644479in}}%
\pgfpathlineto{\pgfqpoint{0.966966in}{0.650382in}}%
\pgfpathlineto{\pgfqpoint{0.943181in}{0.674978in}}%
\pgfpathlineto{\pgfqpoint{0.941581in}{0.676641in}}%
\pgfpathlineto{\pgfqpoint{0.916704in}{0.702901in}}%
\pgfpathlineto{\pgfqpoint{0.913607in}{0.706217in}}%
\pgfpathlineto{\pgfqpoint{0.892313in}{0.729160in}}%
\pgfpathlineto{\pgfqpoint{0.884034in}{0.738206in}}%
\pgfpathlineto{\pgfqpoint{0.868376in}{0.755420in}}%
\pgfpathlineto{\pgfqpoint{0.854460in}{0.770936in}}%
\pgfusepath{stroke}%
\end{pgfscope}%
\begin{pgfscope}%
\pgfpathrectangle{\pgfqpoint{0.854460in}{0.571603in}}{\pgfqpoint{5.885100in}{5.225635in}}%
\pgfusepath{clip}%
\pgfsetbuttcap%
\pgfsetroundjoin%
\pgfsetlinewidth{1.505625pt}%
\definecolor{currentstroke}{rgb}{0.123444,0.636809,0.528763}%
\pgfsetstrokecolor{currentstroke}%
\pgfsetdash{}{0pt}%
\pgfpathmoveto{\pgfqpoint{0.854460in}{4.831793in}}%
\pgfpathlineto{\pgfqpoint{0.872690in}{4.851897in}}%
\pgfpathlineto{\pgfqpoint{0.884034in}{4.864249in}}%
\pgfpathlineto{\pgfqpoint{0.891351in}{4.872138in}}%
\pgfusepath{stroke}%
\end{pgfscope}%
\begin{pgfscope}%
\pgfpathrectangle{\pgfqpoint{0.854460in}{0.571603in}}{\pgfqpoint{5.885100in}{5.225635in}}%
\pgfusepath{clip}%
\pgfsetbuttcap%
\pgfsetroundjoin%
\pgfsetlinewidth{1.505625pt}%
\definecolor{currentstroke}{rgb}{0.123444,0.636809,0.528763}%
\pgfsetstrokecolor{currentstroke}%
\pgfsetdash{}{0pt}%
\pgfpathmoveto{\pgfqpoint{1.167645in}{5.139087in}}%
\pgfpathlineto{\pgfqpoint{1.169547in}{5.140752in}}%
\pgfpathlineto{\pgfqpoint{1.179767in}{5.149590in}}%
\pgfpathlineto{\pgfqpoint{1.200083in}{5.167011in}}%
\pgfpathlineto{\pgfqpoint{1.209341in}{5.174854in}}%
\pgfpathlineto{\pgfqpoint{1.231263in}{5.193271in}}%
\pgfpathlineto{\pgfqpoint{1.238914in}{5.199620in}}%
\pgfpathlineto{\pgfqpoint{1.263100in}{5.219530in}}%
\pgfpathlineto{\pgfqpoint{1.268488in}{5.223911in}}%
\pgfpathlineto{\pgfqpoint{1.295606in}{5.245790in}}%
\pgfpathlineto{\pgfqpoint{1.298061in}{5.247746in}}%
\pgfpathlineto{\pgfqpoint{1.327634in}{5.271125in}}%
\pgfpathlineto{\pgfqpoint{1.328812in}{5.272049in}}%
\pgfpathlineto{\pgfqpoint{1.357208in}{5.294040in}}%
\pgfpathlineto{\pgfqpoint{1.362761in}{5.298308in}}%
\pgfpathlineto{\pgfqpoint{1.386781in}{5.316546in}}%
\pgfpathlineto{\pgfqpoint{1.397426in}{5.324568in}}%
\pgfpathlineto{\pgfqpoint{1.416354in}{5.338661in}}%
\pgfpathlineto{\pgfqpoint{1.432816in}{5.350827in}}%
\pgfpathlineto{\pgfqpoint{1.445928in}{5.360401in}}%
\pgfpathlineto{\pgfqpoint{1.468943in}{5.377087in}}%
\pgfpathlineto{\pgfqpoint{1.475501in}{5.381784in}}%
\pgfpathlineto{\pgfqpoint{1.505074in}{5.402815in}}%
\pgfpathlineto{\pgfqpoint{1.505828in}{5.403346in}}%
\pgfpathlineto{\pgfqpoint{1.534648in}{5.423421in}}%
\pgfpathlineto{\pgfqpoint{1.543587in}{5.429606in}}%
\pgfpathlineto{\pgfqpoint{1.564221in}{5.443709in}}%
\pgfpathlineto{\pgfqpoint{1.582123in}{5.455865in}}%
\pgfpathlineto{\pgfqpoint{1.593795in}{5.463695in}}%
\pgfpathlineto{\pgfqpoint{1.621445in}{5.482125in}}%
\pgfpathlineto{\pgfqpoint{1.623368in}{5.483391in}}%
\pgfpathlineto{\pgfqpoint{1.652941in}{5.502703in}}%
\pgfpathlineto{\pgfqpoint{1.661700in}{5.508384in}}%
\pgfpathlineto{\pgfqpoint{1.682515in}{5.521722in}}%
\pgfpathlineto{\pgfqpoint{1.702799in}{5.534644in}}%
\pgfpathlineto{\pgfqpoint{1.712088in}{5.540489in}}%
\pgfpathlineto{\pgfqpoint{1.741661in}{5.558976in}}%
\pgfpathlineto{\pgfqpoint{1.744769in}{5.560903in}}%
\pgfpathlineto{\pgfqpoint{1.771235in}{5.577115in}}%
\pgfpathlineto{\pgfqpoint{1.787729in}{5.587163in}}%
\pgfpathlineto{\pgfqpoint{1.800808in}{5.595033in}}%
\pgfpathlineto{\pgfqpoint{1.830381in}{5.612728in}}%
\pgfpathlineto{\pgfqpoint{1.831551in}{5.613422in}}%
\pgfpathlineto{\pgfqpoint{1.859955in}{5.630065in}}%
\pgfpathlineto{\pgfqpoint{1.876450in}{5.639682in}}%
\pgfpathlineto{\pgfqpoint{1.889528in}{5.647212in}}%
\pgfpathlineto{\pgfqpoint{1.919102in}{5.664144in}}%
\pgfpathlineto{\pgfqpoint{1.922265in}{5.665941in}}%
\pgfpathlineto{\pgfqpoint{1.948675in}{5.680754in}}%
\pgfpathlineto{\pgfqpoint{1.969178in}{5.692201in}}%
\pgfpathlineto{\pgfqpoint{1.978248in}{5.697202in}}%
\pgfpathlineto{\pgfqpoint{2.007822in}{5.713399in}}%
\pgfpathlineto{\pgfqpoint{2.017125in}{5.718460in}}%
\pgfpathlineto{\pgfqpoint{2.037395in}{5.729351in}}%
\pgfpathlineto{\pgfqpoint{2.066119in}{5.744720in}}%
\pgfpathlineto{\pgfqpoint{2.066968in}{5.745168in}}%
\pgfpathlineto{\pgfqpoint{2.096542in}{5.760657in}}%
\pgfpathlineto{\pgfqpoint{2.116334in}{5.770979in}}%
\pgfpathlineto{\pgfqpoint{2.126115in}{5.776017in}}%
\pgfpathlineto{\pgfqpoint{2.155689in}{5.791147in}}%
\pgfpathlineto{\pgfqpoint{2.167670in}{5.797238in}}%
\pgfusepath{stroke}%
\end{pgfscope}%
\begin{pgfscope}%
\pgfpathrectangle{\pgfqpoint{0.854460in}{0.571603in}}{\pgfqpoint{5.885100in}{5.225635in}}%
\pgfusepath{clip}%
\pgfsetbuttcap%
\pgfsetroundjoin%
\pgfsetlinewidth{1.505625pt}%
\definecolor{currentstroke}{rgb}{0.123444,0.636809,0.528763}%
\pgfsetstrokecolor{currentstroke}%
\pgfsetdash{}{0pt}%
\pgfpathmoveto{\pgfqpoint{6.522648in}{5.797238in}}%
\pgfpathlineto{\pgfqpoint{6.516168in}{5.770979in}}%
\pgfpathlineto{\pgfqpoint{6.509373in}{5.744720in}}%
\pgfpathlineto{\pgfqpoint{6.502973in}{5.721018in}}%
\pgfpathlineto{\pgfqpoint{6.502284in}{5.718460in}}%
\pgfpathlineto{\pgfqpoint{6.494859in}{5.692201in}}%
\pgfpathlineto{\pgfqpoint{6.487200in}{5.665941in}}%
\pgfpathlineto{\pgfqpoint{6.479333in}{5.639682in}}%
\pgfpathlineto{\pgfqpoint{6.473400in}{5.620402in}}%
\pgfpathlineto{\pgfqpoint{6.471256in}{5.613422in}}%
\pgfpathlineto{\pgfqpoint{6.462945in}{5.587163in}}%
\pgfpathlineto{\pgfqpoint{6.454489in}{5.560903in}}%
\pgfpathlineto{\pgfqpoint{6.445907in}{5.534644in}}%
\pgfpathlineto{\pgfqpoint{6.443827in}{5.528415in}}%
\pgfpathlineto{\pgfqpoint{6.437144in}{5.508384in}}%
\pgfpathlineto{\pgfqpoint{6.428265in}{5.482125in}}%
\pgfpathlineto{\pgfqpoint{6.419311in}{5.455865in}}%
\pgfpathlineto{\pgfqpoint{6.414253in}{5.441208in}}%
\pgfpathlineto{\pgfqpoint{6.410252in}{5.429606in}}%
\pgfpathlineto{\pgfqpoint{6.401092in}{5.403346in}}%
\pgfpathlineto{\pgfqpoint{6.391897in}{5.377087in}}%
\pgfpathlineto{\pgfqpoint{6.384680in}{5.356570in}}%
\pgfpathlineto{\pgfqpoint{6.382660in}{5.350827in}}%
\pgfpathlineto{\pgfqpoint{6.373337in}{5.324568in}}%
\pgfpathlineto{\pgfqpoint{6.364016in}{5.298308in}}%
\pgfpathlineto{\pgfqpoint{6.355107in}{5.273186in}}%
\pgfpathlineto{\pgfqpoint{6.354703in}{5.272049in}}%
\pgfpathlineto{\pgfqpoint{6.345518in}{5.246342in}}%
\pgfusepath{stroke}%
\end{pgfscope}%
\begin{pgfscope}%
\pgfpathrectangle{\pgfqpoint{0.854460in}{0.571603in}}{\pgfqpoint{5.885100in}{5.225635in}}%
\pgfusepath{clip}%
\pgfsetbuttcap%
\pgfsetroundjoin%
\pgfsetlinewidth{1.505625pt}%
\definecolor{currentstroke}{rgb}{0.123444,0.636809,0.528763}%
\pgfsetstrokecolor{currentstroke}%
\pgfsetdash{}{0pt}%
\pgfpathmoveto{\pgfqpoint{6.217917in}{4.876630in}}%
\pgfpathlineto{\pgfqpoint{6.177153in}{4.746860in}}%
\pgfpathlineto{\pgfqpoint{6.146427in}{4.641822in}}%
\pgfpathlineto{\pgfqpoint{6.118073in}{4.536784in}}%
\pgfpathlineto{\pgfqpoint{6.092313in}{4.431746in}}%
\pgfpathlineto{\pgfqpoint{6.069381in}{4.326708in}}%
\pgfpathlineto{\pgfqpoint{6.049495in}{4.221670in}}%
\pgfpathlineto{\pgfqpoint{6.036682in}{4.142892in}}%
\pgfpathlineto{\pgfqpoint{6.025742in}{4.064113in}}%
\pgfpathlineto{\pgfqpoint{6.016711in}{3.985335in}}%
\pgfpathlineto{\pgfqpoint{6.009692in}{3.906556in}}%
\pgfpathlineto{\pgfqpoint{6.004724in}{3.827778in}}%
\pgfpathlineto{\pgfqpoint{6.001848in}{3.749000in}}%
\pgfpathlineto{\pgfqpoint{6.001103in}{3.670221in}}%
\pgfpathlineto{\pgfqpoint{6.002530in}{3.591443in}}%
\pgfpathlineto{\pgfqpoint{6.006169in}{3.512664in}}%
\pgfpathlineto{\pgfqpoint{6.012064in}{3.433886in}}%
\pgfpathlineto{\pgfqpoint{6.020257in}{3.355107in}}%
\pgfpathlineto{\pgfqpoint{6.030785in}{3.276329in}}%
\pgfpathlineto{\pgfqpoint{6.043610in}{3.197551in}}%
\pgfpathlineto{\pgfqpoint{6.059373in}{3.116360in}}%
\pgfpathlineto{\pgfqpoint{6.076439in}{3.039994in}}%
\pgfpathlineto{\pgfqpoint{6.096468in}{2.961215in}}%
\pgfpathlineto{\pgfqpoint{6.118941in}{2.882437in}}%
\pgfpathlineto{\pgfqpoint{6.143830in}{2.803659in}}%
\pgfpathlineto{\pgfqpoint{6.171188in}{2.724880in}}%
\pgfpathlineto{\pgfqpoint{6.201029in}{2.646102in}}%
\pgfpathlineto{\pgfqpoint{6.236813in}{2.559353in}}%
\pgfpathlineto{\pgfqpoint{6.268196in}{2.488545in}}%
\pgfpathlineto{\pgfqpoint{6.305511in}{2.409766in}}%
\pgfpathlineto{\pgfqpoint{6.345360in}{2.330988in}}%
\pgfpathlineto{\pgfqpoint{6.387744in}{2.252210in}}%
\pgfpathlineto{\pgfqpoint{6.432623in}{2.173431in}}%
\pgfpathlineto{\pgfqpoint{6.480055in}{2.094653in}}%
\pgfpathlineto{\pgfqpoint{6.532547in}{2.012032in}}%
\pgfpathlineto{\pgfqpoint{6.582499in}{1.937096in}}%
\pgfpathlineto{\pgfqpoint{6.637521in}{1.858318in}}%
\pgfpathlineto{\pgfqpoint{6.695081in}{1.779539in}}%
\pgfpathlineto{\pgfqpoint{6.739560in}{1.720969in}}%
\pgfpathlineto{\pgfqpoint{6.739560in}{1.720969in}}%
\pgfusepath{stroke}%
\end{pgfscope}%
\begin{pgfscope}%
\pgfpathrectangle{\pgfqpoint{0.854460in}{0.571603in}}{\pgfqpoint{5.885100in}{5.225635in}}%
\pgfusepath{clip}%
\pgfsetbuttcap%
\pgfsetroundjoin%
\pgfsetlinewidth{1.505625pt}%
\definecolor{currentstroke}{rgb}{0.132268,0.655014,0.519661}%
\pgfsetstrokecolor{currentstroke}%
\pgfsetdash{}{0pt}%
\pgfpathmoveto{\pgfqpoint{0.990838in}{0.571603in}}%
\pgfpathlineto{\pgfqpoint{0.972754in}{0.589410in}}%
\pgfpathlineto{\pgfqpoint{0.964215in}{0.597863in}}%
\pgfpathlineto{\pgfqpoint{0.943181in}{0.618974in}}%
\pgfpathlineto{\pgfqpoint{0.938079in}{0.624122in}}%
\pgfpathlineto{\pgfqpoint{0.913607in}{0.649163in}}%
\pgfpathlineto{\pgfqpoint{0.912423in}{0.650382in}}%
\pgfpathlineto{\pgfqpoint{0.887278in}{0.676641in}}%
\pgfpathlineto{\pgfqpoint{0.884034in}{0.680078in}}%
\pgfpathlineto{\pgfqpoint{0.862616in}{0.702901in}}%
\pgfpathlineto{\pgfqpoint{0.854460in}{0.711713in}}%
\pgfusepath{stroke}%
\end{pgfscope}%
\begin{pgfscope}%
\pgfpathrectangle{\pgfqpoint{0.854460in}{0.571603in}}{\pgfqpoint{5.885100in}{5.225635in}}%
\pgfusepath{clip}%
\pgfsetbuttcap%
\pgfsetroundjoin%
\pgfsetlinewidth{1.505625pt}%
\definecolor{currentstroke}{rgb}{0.132268,0.655014,0.519661}%
\pgfsetstrokecolor{currentstroke}%
\pgfsetdash{}{0pt}%
\pgfpathmoveto{\pgfqpoint{0.854460in}{4.902448in}}%
\pgfpathlineto{\pgfqpoint{0.856296in}{4.904416in}}%
\pgfpathlineto{\pgfqpoint{0.881110in}{4.930676in}}%
\pgfpathlineto{\pgfqpoint{0.884034in}{4.933730in}}%
\pgfpathlineto{\pgfqpoint{0.906462in}{4.956935in}}%
\pgfpathlineto{\pgfqpoint{0.913607in}{4.964236in}}%
\pgfpathlineto{\pgfqpoint{0.932338in}{4.983195in}}%
\pgfpathlineto{\pgfqpoint{0.943181in}{4.994033in}}%
\pgfpathlineto{\pgfqpoint{0.958751in}{5.009454in}}%
\pgfpathlineto{\pgfqpoint{0.972754in}{5.023150in}}%
\pgfpathlineto{\pgfqpoint{0.985716in}{5.035714in}}%
\pgfpathlineto{\pgfqpoint{1.002327in}{5.051615in}}%
\pgfpathlineto{\pgfqpoint{1.013246in}{5.061973in}}%
\pgfpathlineto{\pgfqpoint{1.031901in}{5.079453in}}%
\pgfpathlineto{\pgfqpoint{1.041354in}{5.088233in}}%
\pgfpathlineto{\pgfqpoint{1.061474in}{5.106690in}}%
\pgfpathlineto{\pgfqpoint{1.070054in}{5.114492in}}%
\pgfpathlineto{\pgfqpoint{1.091047in}{5.133350in}}%
\pgfpathlineto{\pgfqpoint{1.099358in}{5.140752in}}%
\pgfpathlineto{\pgfqpoint{1.120621in}{5.159457in}}%
\pgfpathlineto{\pgfqpoint{1.129280in}{5.167011in}}%
\pgfpathlineto{\pgfqpoint{1.150194in}{5.185033in}}%
\pgfpathlineto{\pgfqpoint{1.159833in}{5.193271in}}%
\pgfpathlineto{\pgfqpoint{1.179767in}{5.210100in}}%
\pgfpathlineto{\pgfqpoint{1.191029in}{5.219530in}}%
\pgfpathlineto{\pgfqpoint{1.209341in}{5.234678in}}%
\pgfpathlineto{\pgfqpoint{1.222880in}{5.245790in}}%
\pgfpathlineto{\pgfqpoint{1.238914in}{5.258789in}}%
\pgfpathlineto{\pgfqpoint{1.255398in}{5.272049in}}%
\pgfpathlineto{\pgfqpoint{1.268488in}{5.282451in}}%
\pgfpathlineto{\pgfqpoint{1.288595in}{5.298308in}}%
\pgfpathlineto{\pgfqpoint{1.298061in}{5.305683in}}%
\pgfpathlineto{\pgfqpoint{1.322483in}{5.324568in}}%
\pgfpathlineto{\pgfqpoint{1.327634in}{5.328503in}}%
\pgfpathlineto{\pgfqpoint{1.357072in}{5.350827in}}%
\pgfpathlineto{\pgfqpoint{1.357208in}{5.350929in}}%
\pgfpathlineto{\pgfqpoint{1.386781in}{5.372898in}}%
\pgfpathlineto{\pgfqpoint{1.392460in}{5.377087in}}%
\pgfpathlineto{\pgfqpoint{1.416354in}{5.394497in}}%
\pgfpathlineto{\pgfqpoint{1.428584in}{5.403346in}}%
\pgfpathlineto{\pgfqpoint{1.445928in}{5.415744in}}%
\pgfpathlineto{\pgfqpoint{1.465450in}{5.429606in}}%
\pgfpathlineto{\pgfqpoint{1.475501in}{5.436655in}}%
\pgfpathlineto{\pgfqpoint{1.503069in}{5.455865in}}%
\pgfpathlineto{\pgfqpoint{1.505074in}{5.457246in}}%
\pgfpathlineto{\pgfqpoint{1.534648in}{5.477440in}}%
\pgfpathlineto{\pgfqpoint{1.541556in}{5.482125in}}%
\pgfpathlineto{\pgfqpoint{1.564221in}{5.497309in}}%
\pgfpathlineto{\pgfqpoint{1.580856in}{5.508384in}}%
\pgfpathlineto{\pgfqpoint{1.593795in}{5.516894in}}%
\pgfpathlineto{\pgfqpoint{1.620945in}{5.534644in}}%
\pgfpathlineto{\pgfqpoint{1.623368in}{5.536209in}}%
\pgfpathlineto{\pgfqpoint{1.628346in}{5.539398in}}%
\pgfusepath{stroke}%
\end{pgfscope}%
\begin{pgfscope}%
\pgfpathrectangle{\pgfqpoint{0.854460in}{0.571603in}}{\pgfqpoint{5.885100in}{5.225635in}}%
\pgfusepath{clip}%
\pgfsetbuttcap%
\pgfsetroundjoin%
\pgfsetlinewidth{1.505625pt}%
\definecolor{currentstroke}{rgb}{0.132268,0.655014,0.519661}%
\pgfsetstrokecolor{currentstroke}%
\pgfsetdash{}{0pt}%
\pgfpathmoveto{\pgfqpoint{1.955637in}{5.734033in}}%
\pgfpathlineto{\pgfqpoint{1.975084in}{5.744720in}}%
\pgfpathlineto{\pgfqpoint{1.978248in}{5.746437in}}%
\pgfpathlineto{\pgfqpoint{2.007822in}{5.762359in}}%
\pgfpathlineto{\pgfqpoint{2.023909in}{5.770979in}}%
\pgfpathlineto{\pgfqpoint{2.037395in}{5.778115in}}%
\pgfpathlineto{\pgfqpoint{2.066968in}{5.793678in}}%
\pgfpathlineto{\pgfqpoint{2.073783in}{5.797238in}}%
\pgfusepath{stroke}%
\end{pgfscope}%
\begin{pgfscope}%
\pgfpathrectangle{\pgfqpoint{0.854460in}{0.571603in}}{\pgfqpoint{5.885100in}{5.225635in}}%
\pgfusepath{clip}%
\pgfsetbuttcap%
\pgfsetroundjoin%
\pgfsetlinewidth{1.505625pt}%
\definecolor{currentstroke}{rgb}{0.132268,0.655014,0.519661}%
\pgfsetstrokecolor{currentstroke}%
\pgfsetdash{}{0pt}%
\pgfpathmoveto{\pgfqpoint{6.632130in}{5.797238in}}%
\pgfpathlineto{\pgfqpoint{6.624383in}{5.770979in}}%
\pgfpathlineto{\pgfqpoint{6.621267in}{5.760835in}}%
\pgfpathlineto{\pgfqpoint{6.616777in}{5.746181in}}%
\pgfusepath{stroke}%
\end{pgfscope}%
\begin{pgfscope}%
\pgfpathrectangle{\pgfqpoint{0.854460in}{0.571603in}}{\pgfqpoint{5.885100in}{5.225635in}}%
\pgfusepath{clip}%
\pgfsetbuttcap%
\pgfsetroundjoin%
\pgfsetlinewidth{1.505625pt}%
\definecolor{currentstroke}{rgb}{0.132268,0.655014,0.519661}%
\pgfsetstrokecolor{currentstroke}%
\pgfsetdash{}{0pt}%
\pgfpathmoveto{\pgfqpoint{6.486197in}{5.377655in}}%
\pgfpathlineto{\pgfqpoint{6.358027in}{5.035714in}}%
\pgfpathlineto{\pgfqpoint{6.311512in}{4.904416in}}%
\pgfpathlineto{\pgfqpoint{6.276261in}{4.799378in}}%
\pgfpathlineto{\pgfqpoint{6.243087in}{4.694341in}}%
\pgfpathlineto{\pgfqpoint{6.212267in}{4.589303in}}%
\pgfpathlineto{\pgfqpoint{6.184045in}{4.484265in}}%
\pgfpathlineto{\pgfqpoint{6.158647in}{4.379227in}}%
\pgfpathlineto{\pgfqpoint{6.136280in}{4.274189in}}%
\pgfpathlineto{\pgfqpoint{6.118520in}{4.177219in}}%
\pgfpathlineto{\pgfqpoint{6.108787in}{4.116632in}}%
\pgfpathlineto{\pgfqpoint{6.097906in}{4.037854in}}%
\pgfpathlineto{\pgfqpoint{6.088946in}{3.958099in}}%
\pgfpathlineto{\pgfqpoint{6.082168in}{3.880297in}}%
\pgfpathlineto{\pgfqpoint{6.077403in}{3.801519in}}%
\pgfpathlineto{\pgfqpoint{6.074781in}{3.722740in}}%
\pgfpathlineto{\pgfqpoint{6.074341in}{3.643962in}}%
\pgfpathlineto{\pgfqpoint{6.076119in}{3.565183in}}%
\pgfpathlineto{\pgfqpoint{6.080156in}{3.486405in}}%
\pgfpathlineto{\pgfqpoint{6.086492in}{3.407626in}}%
\pgfpathlineto{\pgfqpoint{6.095120in}{3.328848in}}%
\pgfpathlineto{\pgfqpoint{6.106096in}{3.250070in}}%
\pgfpathlineto{\pgfqpoint{6.119477in}{3.171291in}}%
\pgfpathlineto{\pgfqpoint{6.135204in}{3.092513in}}%
\pgfpathlineto{\pgfqpoint{6.153385in}{3.013734in}}%
\pgfpathlineto{\pgfqpoint{6.173994in}{2.934956in}}%
\pgfpathlineto{\pgfqpoint{6.197040in}{2.856177in}}%
\pgfpathlineto{\pgfqpoint{6.222566in}{2.777399in}}%
\pgfpathlineto{\pgfqpoint{6.250578in}{2.698621in}}%
\pgfpathlineto{\pgfqpoint{6.281087in}{2.619842in}}%
\pgfpathlineto{\pgfqpoint{6.314104in}{2.541064in}}%
\pgfpathlineto{\pgfqpoint{6.355107in}{2.450758in}}%
\pgfpathlineto{\pgfqpoint{6.387712in}{2.383507in}}%
\pgfpathlineto{\pgfqpoint{6.428285in}{2.304729in}}%
\pgfpathlineto{\pgfqpoint{6.473400in}{2.222502in}}%
\pgfpathlineto{\pgfqpoint{6.517084in}{2.147172in}}%
\pgfpathlineto{\pgfqpoint{6.565318in}{2.068393in}}%
\pgfpathlineto{\pgfqpoint{6.621267in}{1.981835in}}%
\pgfpathlineto{\pgfqpoint{6.669382in}{1.910836in}}%
\pgfpathlineto{\pgfqpoint{6.725245in}{1.832058in}}%
\pgfpathlineto{\pgfqpoint{6.739560in}{1.812493in}}%
\pgfpathlineto{\pgfqpoint{6.739560in}{1.812493in}}%
\pgfusepath{stroke}%
\end{pgfscope}%
\begin{pgfscope}%
\pgfpathrectangle{\pgfqpoint{0.854460in}{0.571603in}}{\pgfqpoint{5.885100in}{5.225635in}}%
\pgfusepath{clip}%
\pgfsetbuttcap%
\pgfsetroundjoin%
\pgfsetlinewidth{1.505625pt}%
\definecolor{currentstroke}{rgb}{0.146616,0.673050,0.508936}%
\pgfsetstrokecolor{currentstroke}%
\pgfsetdash{}{0pt}%
\pgfpathmoveto{\pgfqpoint{0.936926in}{0.571603in}}%
\pgfpathlineto{\pgfqpoint{0.913607in}{0.594771in}}%
\pgfpathlineto{\pgfqpoint{0.910512in}{0.597863in}}%
\pgfpathlineto{\pgfqpoint{0.884585in}{0.624122in}}%
\pgfpathlineto{\pgfqpoint{0.884034in}{0.624689in}}%
\pgfpathlineto{\pgfqpoint{0.859178in}{0.650382in}}%
\pgfpathlineto{\pgfqpoint{0.854460in}{0.655326in}}%
\pgfusepath{stroke}%
\end{pgfscope}%
\begin{pgfscope}%
\pgfpathrectangle{\pgfqpoint{0.854460in}{0.571603in}}{\pgfqpoint{5.885100in}{5.225635in}}%
\pgfusepath{clip}%
\pgfsetbuttcap%
\pgfsetroundjoin%
\pgfsetlinewidth{1.505625pt}%
\definecolor{currentstroke}{rgb}{0.146616,0.673050,0.508936}%
\pgfsetstrokecolor{currentstroke}%
\pgfsetdash{}{0pt}%
\pgfpathmoveto{\pgfqpoint{0.854460in}{4.969488in}}%
\pgfpathlineto{\pgfqpoint{0.867775in}{4.983195in}}%
\pgfpathlineto{\pgfqpoint{0.884034in}{4.999724in}}%
\pgfpathlineto{\pgfqpoint{0.893694in}{5.009454in}}%
\pgfpathlineto{\pgfqpoint{0.913607in}{5.029264in}}%
\pgfpathlineto{\pgfqpoint{0.920150in}{5.035714in}}%
\pgfpathlineto{\pgfqpoint{0.943181in}{5.058136in}}%
\pgfpathlineto{\pgfqpoint{0.947158in}{5.061973in}}%
\pgfpathlineto{\pgfqpoint{0.972754in}{5.086366in}}%
\pgfpathlineto{\pgfqpoint{0.974730in}{5.088233in}}%
\pgfpathlineto{\pgfqpoint{1.002327in}{5.113980in}}%
\pgfpathlineto{\pgfqpoint{1.002881in}{5.114492in}}%
\pgfpathlineto{\pgfqpoint{1.031627in}{5.140752in}}%
\pgfpathlineto{\pgfqpoint{1.031901in}{5.140998in}}%
\pgfpathlineto{\pgfqpoint{1.060976in}{5.167011in}}%
\pgfpathlineto{\pgfqpoint{1.061474in}{5.167451in}}%
\pgfpathlineto{\pgfqpoint{1.090933in}{5.193271in}}%
\pgfpathlineto{\pgfqpoint{1.091047in}{5.193369in}}%
\pgfpathlineto{\pgfqpoint{1.120621in}{5.218761in}}%
\pgfpathlineto{\pgfqpoint{1.121524in}{5.219530in}}%
\pgfpathlineto{\pgfqpoint{1.150194in}{5.243650in}}%
\pgfpathlineto{\pgfqpoint{1.152758in}{5.245790in}}%
\pgfpathlineto{\pgfqpoint{1.179767in}{5.268059in}}%
\pgfpathlineto{\pgfqpoint{1.184645in}{5.272049in}}%
\pgfpathlineto{\pgfqpoint{1.209341in}{5.292007in}}%
\pgfpathlineto{\pgfqpoint{1.217198in}{5.298308in}}%
\pgfpathlineto{\pgfqpoint{1.237679in}{5.314534in}}%
\pgfusepath{stroke}%
\end{pgfscope}%
\begin{pgfscope}%
\pgfpathrectangle{\pgfqpoint{0.854460in}{0.571603in}}{\pgfqpoint{5.885100in}{5.225635in}}%
\pgfusepath{clip}%
\pgfsetbuttcap%
\pgfsetroundjoin%
\pgfsetlinewidth{1.505625pt}%
\definecolor{currentstroke}{rgb}{0.146616,0.673050,0.508936}%
\pgfsetstrokecolor{currentstroke}%
\pgfsetdash{}{0pt}%
\pgfpathmoveto{\pgfqpoint{1.547574in}{5.537931in}}%
\pgfpathlineto{\pgfqpoint{1.564221in}{5.548866in}}%
\pgfpathlineto{\pgfqpoint{1.582655in}{5.560903in}}%
\pgfpathlineto{\pgfqpoint{1.593795in}{5.568089in}}%
\pgfpathlineto{\pgfqpoint{1.623368in}{5.587057in}}%
\pgfpathlineto{\pgfqpoint{1.623534in}{5.587163in}}%
\pgfpathlineto{\pgfqpoint{1.652941in}{5.605639in}}%
\pgfpathlineto{\pgfqpoint{1.665396in}{5.613422in}}%
\pgfpathlineto{\pgfqpoint{1.682515in}{5.623989in}}%
\pgfpathlineto{\pgfqpoint{1.708073in}{5.639682in}}%
\pgfpathlineto{\pgfqpoint{1.712088in}{5.642117in}}%
\pgfpathlineto{\pgfqpoint{1.741661in}{5.659921in}}%
\pgfpathlineto{\pgfqpoint{1.751722in}{5.665941in}}%
\pgfpathlineto{\pgfqpoint{1.771235in}{5.677474in}}%
\pgfpathlineto{\pgfqpoint{1.796270in}{5.692201in}}%
\pgfpathlineto{\pgfqpoint{1.800808in}{5.694837in}}%
\pgfpathlineto{\pgfqpoint{1.830381in}{5.711893in}}%
\pgfpathlineto{\pgfqpoint{1.841833in}{5.718460in}}%
\pgfpathlineto{\pgfqpoint{1.859955in}{5.728724in}}%
\pgfpathlineto{\pgfqpoint{1.888320in}{5.744720in}}%
\pgfpathlineto{\pgfqpoint{1.889528in}{5.745393in}}%
\pgfpathlineto{\pgfqpoint{1.919102in}{5.761729in}}%
\pgfpathlineto{\pgfqpoint{1.935922in}{5.770979in}}%
\pgfpathlineto{\pgfqpoint{1.948675in}{5.777906in}}%
\pgfpathlineto{\pgfqpoint{1.978248in}{5.793881in}}%
\pgfpathlineto{\pgfqpoint{1.984508in}{5.797238in}}%
\pgfusepath{stroke}%
\end{pgfscope}%
\begin{pgfscope}%
\pgfpathrectangle{\pgfqpoint{0.854460in}{0.571603in}}{\pgfqpoint{5.885100in}{5.225635in}}%
\pgfusepath{clip}%
\pgfsetbuttcap%
\pgfsetroundjoin%
\pgfsetlinewidth{1.505625pt}%
\definecolor{currentstroke}{rgb}{0.146616,0.673050,0.508936}%
\pgfsetstrokecolor{currentstroke}%
\pgfsetdash{}{0pt}%
\pgfpathmoveto{\pgfqpoint{6.737069in}{5.797238in}}%
\pgfpathlineto{\pgfqpoint{6.728128in}{5.770979in}}%
\pgfpathlineto{\pgfqpoint{6.718992in}{5.744720in}}%
\pgfpathlineto{\pgfqpoint{6.709987in}{5.719330in}}%
\pgfpathlineto{\pgfqpoint{6.709679in}{5.718460in}}%
\pgfpathlineto{\pgfqpoint{6.700114in}{5.692201in}}%
\pgfpathlineto{\pgfqpoint{6.690416in}{5.665941in}}%
\pgfpathlineto{\pgfqpoint{6.680603in}{5.639682in}}%
\pgfpathlineto{\pgfqpoint{6.680414in}{5.639186in}}%
\pgfpathlineto{\pgfqpoint{6.670588in}{5.613422in}}%
\pgfpathlineto{\pgfqpoint{6.660491in}{5.587163in}}%
\pgfpathlineto{\pgfqpoint{6.650840in}{5.562244in}}%
\pgfpathlineto{\pgfqpoint{6.650322in}{5.560903in}}%
\pgfpathlineto{\pgfqpoint{6.640000in}{5.534644in}}%
\pgfpathlineto{\pgfqpoint{6.629640in}{5.508384in}}%
\pgfpathlineto{\pgfqpoint{6.621267in}{5.487257in}}%
\pgfpathlineto{\pgfqpoint{6.619235in}{5.482125in}}%
\pgfpathlineto{\pgfqpoint{6.608731in}{5.455865in}}%
\pgfpathlineto{\pgfqpoint{6.598227in}{5.429606in}}%
\pgfpathlineto{\pgfqpoint{6.591693in}{5.413325in}}%
\pgfpathlineto{\pgfqpoint{6.587692in}{5.403346in}}%
\pgfpathlineto{\pgfqpoint{6.577112in}{5.377087in}}%
\pgfpathlineto{\pgfqpoint{6.566563in}{5.350827in}}%
\pgfpathlineto{\pgfqpoint{6.562120in}{5.339791in}}%
\pgfpathlineto{\pgfqpoint{6.555993in}{5.324568in}}%
\pgfpathlineto{\pgfqpoint{6.545429in}{5.298308in}}%
\pgfpathlineto{\pgfqpoint{6.534921in}{5.272049in}}%
\pgfpathlineto{\pgfqpoint{6.532547in}{5.266121in}}%
\pgfpathlineto{\pgfqpoint{6.524401in}{5.245790in}}%
\pgfpathlineto{\pgfqpoint{6.513931in}{5.219530in}}%
\pgfpathlineto{\pgfqpoint{6.503541in}{5.193271in}}%
\pgfpathlineto{\pgfqpoint{6.502973in}{5.191836in}}%
\pgfpathlineto{\pgfqpoint{6.493145in}{5.167011in}}%
\pgfpathlineto{\pgfqpoint{6.482839in}{5.140752in}}%
\pgfpathlineto{\pgfqpoint{6.473400in}{5.116483in}}%
\pgfpathlineto{\pgfqpoint{6.472625in}{5.114492in}}%
\pgfpathlineto{\pgfqpoint{6.462430in}{5.088233in}}%
\pgfpathlineto{\pgfqpoint{6.452348in}{5.061973in}}%
\pgfpathlineto{\pgfqpoint{6.443827in}{5.039545in}}%
\pgfpathlineto{\pgfqpoint{6.442369in}{5.035714in}}%
\pgfpathlineto{\pgfqpoint{6.432436in}{5.009454in}}%
\pgfpathlineto{\pgfqpoint{6.422632in}{4.983195in}}%
\pgfpathlineto{\pgfqpoint{6.414253in}{4.960469in}}%
\pgfpathlineto{\pgfqpoint{6.412948in}{4.956935in}}%
\pgfpathlineto{\pgfqpoint{6.403327in}{4.930676in}}%
\pgfpathlineto{\pgfqpoint{6.393849in}{4.904416in}}%
\pgfpathlineto{\pgfqpoint{6.384680in}{4.878621in}}%
\pgfpathlineto{\pgfqpoint{6.384515in}{4.878157in}}%
\pgfpathlineto{\pgfqpoint{6.375250in}{4.851897in}}%
\pgfpathlineto{\pgfqpoint{6.366140in}{4.825638in}}%
\pgfpathlineto{\pgfqpoint{6.357189in}{4.799378in}}%
\pgfpathlineto{\pgfqpoint{6.355107in}{4.793189in}}%
\pgfpathlineto{\pgfqpoint{6.348339in}{4.773119in}}%
\pgfpathlineto{\pgfqpoint{6.339638in}{4.746860in}}%
\pgfpathlineto{\pgfqpoint{6.331105in}{4.720600in}}%
\pgfpathlineto{\pgfqpoint{6.325533in}{4.703138in}}%
\pgfpathlineto{\pgfqpoint{6.322719in}{4.694341in}}%
\pgfpathlineto{\pgfqpoint{6.314461in}{4.668081in}}%
\pgfpathlineto{\pgfqpoint{6.306381in}{4.641822in}}%
\pgfpathlineto{\pgfqpoint{6.298482in}{4.615562in}}%
\pgfpathlineto{\pgfqpoint{6.295960in}{4.607011in}}%
\pgfpathlineto{\pgfqpoint{6.290721in}{4.589303in}}%
\pgfpathlineto{\pgfqpoint{6.283126in}{4.563043in}}%
\pgfpathlineto{\pgfqpoint{6.275721in}{4.536784in}}%
\pgfpathlineto{\pgfqpoint{6.268507in}{4.510524in}}%
\pgfpathlineto{\pgfqpoint{6.266386in}{4.502622in}}%
\pgfpathlineto{\pgfqpoint{6.261444in}{4.484265in}}%
\pgfpathlineto{\pgfqpoint{6.254561in}{4.458005in}}%
\pgfpathlineto{\pgfqpoint{6.247877in}{4.431746in}}%
\pgfpathlineto{\pgfqpoint{6.241394in}{4.405486in}}%
\pgfpathlineto{\pgfqpoint{6.236813in}{4.386361in}}%
\pgfpathlineto{\pgfqpoint{6.235098in}{4.379227in}}%
\pgfpathlineto{\pgfqpoint{6.228971in}{4.352967in}}%
\pgfpathlineto{\pgfqpoint{6.223051in}{4.326708in}}%
\pgfpathlineto{\pgfqpoint{6.217340in}{4.300449in}}%
\pgfpathlineto{\pgfqpoint{6.211841in}{4.274189in}}%
\pgfpathlineto{\pgfqpoint{6.211678in}{4.273380in}}%
\pgfusepath{stroke}%
\end{pgfscope}%
\begin{pgfscope}%
\pgfpathrectangle{\pgfqpoint{0.854460in}{0.571603in}}{\pgfqpoint{5.885100in}{5.225635in}}%
\pgfusepath{clip}%
\pgfsetbuttcap%
\pgfsetroundjoin%
\pgfsetlinewidth{1.505625pt}%
\definecolor{currentstroke}{rgb}{0.146616,0.673050,0.508936}%
\pgfsetstrokecolor{currentstroke}%
\pgfsetdash{}{0pt}%
\pgfpathmoveto{\pgfqpoint{6.155420in}{3.885115in}}%
\pgfpathlineto{\pgfqpoint{6.155016in}{3.880297in}}%
\pgfpathlineto{\pgfqpoint{6.153049in}{3.854037in}}%
\pgfpathlineto{\pgfqpoint{6.151320in}{3.827778in}}%
\pgfpathlineto{\pgfqpoint{6.149829in}{3.801519in}}%
\pgfpathlineto{\pgfqpoint{6.148579in}{3.775259in}}%
\pgfpathlineto{\pgfqpoint{6.148093in}{3.762624in}}%
\pgfpathlineto{\pgfqpoint{6.147566in}{3.749000in}}%
\pgfpathlineto{\pgfqpoint{6.146794in}{3.722740in}}%
\pgfpathlineto{\pgfqpoint{6.146268in}{3.696481in}}%
\pgfpathlineto{\pgfqpoint{6.145989in}{3.670221in}}%
\pgfpathlineto{\pgfqpoint{6.145959in}{3.643962in}}%
\pgfpathlineto{\pgfqpoint{6.146179in}{3.617702in}}%
\pgfpathlineto{\pgfqpoint{6.146651in}{3.591443in}}%
\pgfpathlineto{\pgfqpoint{6.147375in}{3.565183in}}%
\pgfpathlineto{\pgfqpoint{6.148093in}{3.545930in}}%
\pgfpathlineto{\pgfqpoint{6.148352in}{3.538924in}}%
\pgfpathlineto{\pgfqpoint{6.149576in}{3.512664in}}%
\pgfpathlineto{\pgfqpoint{6.151056in}{3.486405in}}%
\pgfpathlineto{\pgfqpoint{6.152792in}{3.460145in}}%
\pgfpathlineto{\pgfqpoint{6.154786in}{3.433886in}}%
\pgfpathlineto{\pgfqpoint{6.157039in}{3.407626in}}%
\pgfpathlineto{\pgfqpoint{6.159554in}{3.381367in}}%
\pgfpathlineto{\pgfqpoint{6.162331in}{3.355107in}}%
\pgfpathlineto{\pgfqpoint{6.165372in}{3.328848in}}%
\pgfpathlineto{\pgfqpoint{6.168679in}{3.302589in}}%
\pgfpathlineto{\pgfqpoint{6.172253in}{3.276329in}}%
\pgfpathlineto{\pgfqpoint{6.176096in}{3.250070in}}%
\pgfpathlineto{\pgfqpoint{6.177666in}{3.240048in}}%
\pgfpathlineto{\pgfqpoint{6.180190in}{3.223810in}}%
\pgfpathlineto{\pgfqpoint{6.184543in}{3.197551in}}%
\pgfpathlineto{\pgfqpoint{6.189167in}{3.171291in}}%
\pgfpathlineto{\pgfqpoint{6.194066in}{3.145032in}}%
\pgfpathlineto{\pgfqpoint{6.199240in}{3.118772in}}%
\pgfpathlineto{\pgfqpoint{6.204691in}{3.092513in}}%
\pgfpathlineto{\pgfqpoint{6.207240in}{3.080834in}}%
\pgfpathlineto{\pgfqpoint{6.210398in}{3.066253in}}%
\pgfpathlineto{\pgfqpoint{6.216364in}{3.039994in}}%
\pgfpathlineto{\pgfqpoint{6.222612in}{3.013734in}}%
\pgfpathlineto{\pgfqpoint{6.229143in}{2.987475in}}%
\pgfpathlineto{\pgfqpoint{6.235959in}{2.961215in}}%
\pgfpathlineto{\pgfqpoint{6.236813in}{2.958055in}}%
\pgfpathlineto{\pgfqpoint{6.243014in}{2.934956in}}%
\pgfpathlineto{\pgfqpoint{6.250351in}{2.908696in}}%
\pgfpathlineto{\pgfqpoint{6.257977in}{2.882437in}}%
\pgfpathlineto{\pgfqpoint{6.265894in}{2.856177in}}%
\pgfpathlineto{\pgfqpoint{6.266386in}{2.854601in}}%
\pgfpathlineto{\pgfqpoint{6.274047in}{2.829918in}}%
\pgfpathlineto{\pgfqpoint{6.282490in}{2.803659in}}%
\pgfpathlineto{\pgfqpoint{6.291229in}{2.777399in}}%
\pgfpathlineto{\pgfqpoint{6.295960in}{2.763645in}}%
\pgfpathlineto{\pgfqpoint{6.300235in}{2.751140in}}%
\pgfpathlineto{\pgfqpoint{6.309503in}{2.724880in}}%
\pgfpathlineto{\pgfqpoint{6.319073in}{2.698621in}}%
\pgfpathlineto{\pgfqpoint{6.325533in}{2.681428in}}%
\pgfpathlineto{\pgfqpoint{6.328920in}{2.672361in}}%
\pgfpathlineto{\pgfqpoint{6.339024in}{2.646102in}}%
\pgfpathlineto{\pgfqpoint{6.349433in}{2.619842in}}%
\pgfpathlineto{\pgfqpoint{6.355107in}{2.605931in}}%
\pgfpathlineto{\pgfqpoint{6.360115in}{2.593583in}}%
\pgfpathlineto{\pgfqpoint{6.371063in}{2.567323in}}%
\pgfpathlineto{\pgfqpoint{6.382324in}{2.541064in}}%
\pgfpathlineto{\pgfqpoint{6.384680in}{2.535709in}}%
\pgfpathlineto{\pgfqpoint{6.393831in}{2.514804in}}%
\pgfpathlineto{\pgfqpoint{6.405636in}{2.488545in}}%
\pgfpathlineto{\pgfqpoint{6.414253in}{2.469866in}}%
\pgfpathlineto{\pgfqpoint{6.417733in}{2.462285in}}%
\pgfpathlineto{\pgfqpoint{6.430086in}{2.436026in}}%
\pgfpathlineto{\pgfqpoint{6.442760in}{2.409766in}}%
\pgfpathlineto{\pgfqpoint{6.443827in}{2.407606in}}%
\pgfpathlineto{\pgfqpoint{6.455671in}{2.383507in}}%
\pgfpathlineto{\pgfqpoint{6.468900in}{2.357248in}}%
\pgfpathlineto{\pgfqpoint{6.473400in}{2.348514in}}%
\pgfpathlineto{\pgfqpoint{6.482390in}{2.330988in}}%
\pgfpathlineto{\pgfqpoint{6.496177in}{2.304729in}}%
\pgfpathlineto{\pgfqpoint{6.502973in}{2.292069in}}%
\pgfpathlineto{\pgfqpoint{6.510243in}{2.278469in}}%
\pgfpathlineto{\pgfqpoint{6.524592in}{2.252210in}}%
\pgfpathlineto{\pgfqpoint{6.532547in}{2.237966in}}%
\pgfpathlineto{\pgfqpoint{6.539229in}{2.225950in}}%
\pgfpathlineto{\pgfqpoint{6.554145in}{2.199691in}}%
\pgfpathlineto{\pgfqpoint{6.562120in}{2.185943in}}%
\pgfpathlineto{\pgfqpoint{6.569350in}{2.173431in}}%
\pgfpathlineto{\pgfqpoint{6.584838in}{2.147172in}}%
\pgfpathlineto{\pgfqpoint{6.591693in}{2.135778in}}%
\pgfpathlineto{\pgfqpoint{6.600606in}{2.120912in}}%
\pgfpathlineto{\pgfqpoint{6.616671in}{2.094653in}}%
\pgfpathlineto{\pgfqpoint{6.621267in}{2.087280in}}%
\pgfpathlineto{\pgfqpoint{6.633000in}{2.068393in}}%
\pgfpathlineto{\pgfqpoint{6.649648in}{2.042134in}}%
\pgfpathlineto{\pgfqpoint{6.650840in}{2.040285in}}%
\pgfpathlineto{\pgfqpoint{6.666536in}{2.015874in}}%
\pgfpathlineto{\pgfqpoint{6.680414in}{1.994714in}}%
\pgfpathlineto{\pgfqpoint{6.683747in}{1.989615in}}%
\pgfpathlineto{\pgfqpoint{6.701216in}{1.963355in}}%
\pgfpathlineto{\pgfqpoint{6.709987in}{1.950410in}}%
\pgfpathlineto{\pgfqpoint{6.718981in}{1.937096in}}%
\pgfpathlineto{\pgfqpoint{6.737046in}{1.910836in}}%
\pgfpathlineto{\pgfqpoint{6.739560in}{1.907241in}}%
\pgfusepath{stroke}%
\end{pgfscope}%
\begin{pgfscope}%
\pgfpathrectangle{\pgfqpoint{0.854460in}{0.571603in}}{\pgfqpoint{5.885100in}{5.225635in}}%
\pgfusepath{clip}%
\pgfsetbuttcap%
\pgfsetroundjoin%
\pgfsetlinewidth{1.505625pt}%
\definecolor{currentstroke}{rgb}{0.170948,0.694384,0.493803}%
\pgfsetstrokecolor{currentstroke}%
\pgfsetdash{}{0pt}%
\pgfpathmoveto{\pgfqpoint{0.884205in}{0.571603in}}%
\pgfpathlineto{\pgfqpoint{0.884034in}{0.571774in}}%
\pgfpathlineto{\pgfqpoint{0.858034in}{0.597863in}}%
\pgfpathlineto{\pgfqpoint{0.854460in}{0.601498in}}%
\pgfusepath{stroke}%
\end{pgfscope}%
\begin{pgfscope}%
\pgfpathrectangle{\pgfqpoint{0.854460in}{0.571603in}}{\pgfqpoint{5.885100in}{5.225635in}}%
\pgfusepath{clip}%
\pgfsetbuttcap%
\pgfsetroundjoin%
\pgfsetlinewidth{1.505625pt}%
\definecolor{currentstroke}{rgb}{0.170948,0.694384,0.493803}%
\pgfsetstrokecolor{currentstroke}%
\pgfsetdash{}{0pt}%
\pgfpathmoveto{\pgfqpoint{0.854460in}{5.033443in}}%
\pgfpathlineto{\pgfqpoint{0.856726in}{5.035714in}}%
\pgfpathlineto{\pgfqpoint{0.883257in}{5.061973in}}%
\pgfpathlineto{\pgfqpoint{0.884034in}{5.062732in}}%
\pgfpathlineto{\pgfqpoint{0.910365in}{5.088233in}}%
\pgfpathlineto{\pgfqpoint{0.913607in}{5.091334in}}%
\pgfpathlineto{\pgfqpoint{0.938030in}{5.114492in}}%
\pgfpathlineto{\pgfqpoint{0.943181in}{5.119316in}}%
\pgfpathlineto{\pgfqpoint{0.966264in}{5.140752in}}%
\pgfpathlineto{\pgfqpoint{0.972754in}{5.146704in}}%
\pgfpathlineto{\pgfqpoint{0.995080in}{5.167011in}}%
\pgfpathlineto{\pgfqpoint{1.002327in}{5.173523in}}%
\pgfpathlineto{\pgfqpoint{1.024490in}{5.193271in}}%
\pgfpathlineto{\pgfqpoint{1.031901in}{5.199793in}}%
\pgfpathlineto{\pgfqpoint{1.054507in}{5.219530in}}%
\pgfpathlineto{\pgfqpoint{1.061474in}{5.225538in}}%
\pgfpathlineto{\pgfqpoint{1.085143in}{5.245790in}}%
\pgfpathlineto{\pgfqpoint{1.091047in}{5.250780in}}%
\pgfpathlineto{\pgfqpoint{1.116410in}{5.272049in}}%
\pgfpathlineto{\pgfqpoint{1.120621in}{5.275537in}}%
\pgfpathlineto{\pgfqpoint{1.148319in}{5.298308in}}%
\pgfpathlineto{\pgfqpoint{1.150194in}{5.299831in}}%
\pgfpathlineto{\pgfqpoint{1.179767in}{5.323663in}}%
\pgfpathlineto{\pgfqpoint{1.180900in}{5.324568in}}%
\pgfpathlineto{\pgfqpoint{1.209341in}{5.347031in}}%
\pgfpathlineto{\pgfqpoint{1.214183in}{5.350827in}}%
\pgfpathlineto{\pgfqpoint{1.237296in}{5.368729in}}%
\pgfusepath{stroke}%
\end{pgfscope}%
\begin{pgfscope}%
\pgfpathrectangle{\pgfqpoint{0.854460in}{0.571603in}}{\pgfqpoint{5.885100in}{5.225635in}}%
\pgfusepath{clip}%
\pgfsetbuttcap%
\pgfsetroundjoin%
\pgfsetlinewidth{1.505625pt}%
\definecolor{currentstroke}{rgb}{0.170948,0.694384,0.493803}%
\pgfsetstrokecolor{currentstroke}%
\pgfsetdash{}{0pt}%
\pgfpathmoveto{\pgfqpoint{1.549295in}{5.588930in}}%
\pgfpathlineto{\pgfqpoint{1.564221in}{5.598546in}}%
\pgfpathlineto{\pgfqpoint{1.587443in}{5.613422in}}%
\pgfpathlineto{\pgfqpoint{1.593795in}{5.617441in}}%
\pgfpathlineto{\pgfqpoint{1.623368in}{5.636031in}}%
\pgfpathlineto{\pgfqpoint{1.629215in}{5.639682in}}%
\pgfpathlineto{\pgfqpoint{1.652941in}{5.654314in}}%
\pgfpathlineto{\pgfqpoint{1.671893in}{5.665941in}}%
\pgfpathlineto{\pgfqpoint{1.682515in}{5.672379in}}%
\pgfpathlineto{\pgfqpoint{1.712088in}{5.690199in}}%
\pgfpathlineto{\pgfqpoint{1.715434in}{5.692201in}}%
\pgfpathlineto{\pgfqpoint{1.741661in}{5.707696in}}%
\pgfpathlineto{\pgfqpoint{1.759967in}{5.718460in}}%
\pgfpathlineto{\pgfqpoint{1.771235in}{5.725005in}}%
\pgfpathlineto{\pgfqpoint{1.800808in}{5.742086in}}%
\pgfpathlineto{\pgfqpoint{1.805401in}{5.744720in}}%
\pgfpathlineto{\pgfqpoint{1.830381in}{5.758869in}}%
\pgfpathlineto{\pgfqpoint{1.851854in}{5.770979in}}%
\pgfpathlineto{\pgfqpoint{1.859955in}{5.775491in}}%
\pgfpathlineto{\pgfqpoint{1.889528in}{5.791860in}}%
\pgfpathlineto{\pgfqpoint{1.899303in}{5.797238in}}%
\pgfusepath{stroke}%
\end{pgfscope}%
\begin{pgfscope}%
\pgfpathrectangle{\pgfqpoint{0.854460in}{0.571603in}}{\pgfqpoint{5.885100in}{5.225635in}}%
\pgfusepath{clip}%
\pgfsetbuttcap%
\pgfsetroundjoin%
\pgfsetlinewidth{1.505625pt}%
\definecolor{currentstroke}{rgb}{0.170948,0.694384,0.493803}%
\pgfsetstrokecolor{currentstroke}%
\pgfsetdash{}{0pt}%
\pgfpathmoveto{\pgfqpoint{6.739560in}{5.551949in}}%
\pgfpathlineto{\pgfqpoint{6.532547in}{5.055914in}}%
\pgfpathlineto{\pgfqpoint{6.483999in}{4.930676in}}%
\pgfpathlineto{\pgfqpoint{6.443827in}{4.820983in}}%
\pgfpathlineto{\pgfqpoint{6.409186in}{4.720600in}}%
\pgfpathlineto{\pgfqpoint{6.375428in}{4.615562in}}%
\pgfpathlineto{\pgfqpoint{6.344433in}{4.510524in}}%
\pgfpathlineto{\pgfqpoint{6.316403in}{4.405486in}}%
\pgfpathlineto{\pgfqpoint{6.295960in}{4.320078in}}%
\pgfpathlineto{\pgfqpoint{6.280296in}{4.247930in}}%
\pgfpathlineto{\pgfqpoint{6.265092in}{4.169151in}}%
\pgfpathlineto{\pgfqpoint{6.251802in}{4.090373in}}%
\pgfpathlineto{\pgfqpoint{6.240572in}{4.011594in}}%
\pgfpathlineto{\pgfqpoint{6.231391in}{3.932816in}}%
\pgfpathlineto{\pgfqpoint{6.224323in}{3.854037in}}%
\pgfpathlineto{\pgfqpoint{6.219429in}{3.775259in}}%
\pgfpathlineto{\pgfqpoint{6.216740in}{3.696481in}}%
\pgfpathlineto{\pgfqpoint{6.216291in}{3.617702in}}%
\pgfpathlineto{\pgfqpoint{6.218116in}{3.538924in}}%
\pgfpathlineto{\pgfqpoint{6.222251in}{3.460145in}}%
\pgfpathlineto{\pgfqpoint{6.228732in}{3.381367in}}%
\pgfpathlineto{\pgfqpoint{6.237594in}{3.302589in}}%
\pgfpathlineto{\pgfqpoint{6.248805in}{3.223810in}}%
\pgfpathlineto{\pgfqpoint{6.262470in}{3.145032in}}%
\pgfpathlineto{\pgfqpoint{6.278544in}{3.066253in}}%
\pgfpathlineto{\pgfqpoint{6.297112in}{2.987475in}}%
\pgfpathlineto{\pgfqpoint{6.318114in}{2.908696in}}%
\pgfpathlineto{\pgfqpoint{6.341618in}{2.829918in}}%
\pgfpathlineto{\pgfqpoint{6.367633in}{2.751140in}}%
\pgfpathlineto{\pgfqpoint{6.396167in}{2.672361in}}%
\pgfpathlineto{\pgfqpoint{6.427229in}{2.593583in}}%
\pgfpathlineto{\pgfqpoint{6.460831in}{2.514804in}}%
\pgfpathlineto{\pgfqpoint{6.502513in}{2.424560in}}%
\pgfpathlineto{\pgfqpoint{6.502513in}{2.424560in}}%
\pgfusepath{stroke}%
\end{pgfscope}%
\begin{pgfscope}%
\pgfpathrectangle{\pgfqpoint{0.854460in}{0.571603in}}{\pgfqpoint{5.885100in}{5.225635in}}%
\pgfusepath{clip}%
\pgfsetbuttcap%
\pgfsetroundjoin%
\pgfsetlinewidth{1.505625pt}%
\definecolor{currentstroke}{rgb}{0.170948,0.694384,0.493803}%
\pgfsetstrokecolor{currentstroke}%
\pgfsetdash{}{0pt}%
\pgfpathmoveto{\pgfqpoint{6.689942in}{2.083902in}}%
\pgfpathlineto{\pgfqpoint{6.699576in}{2.068393in}}%
\pgfpathlineto{\pgfqpoint{6.709987in}{2.051966in}}%
\pgfpathlineto{\pgfqpoint{6.716197in}{2.042134in}}%
\pgfpathlineto{\pgfqpoint{6.733098in}{2.015874in}}%
\pgfpathlineto{\pgfqpoint{6.739560in}{2.006016in}}%
\pgfusepath{stroke}%
\end{pgfscope}%
\begin{pgfscope}%
\pgfpathrectangle{\pgfqpoint{0.854460in}{0.571603in}}{\pgfqpoint{5.885100in}{5.225635in}}%
\pgfusepath{clip}%
\pgfsetbuttcap%
\pgfsetroundjoin%
\pgfsetlinewidth{1.505625pt}%
\definecolor{currentstroke}{rgb}{0.196571,0.711827,0.479221}%
\pgfsetstrokecolor{currentstroke}%
\pgfsetdash{}{0pt}%
\pgfpathmoveto{\pgfqpoint{0.854460in}{5.094464in}}%
\pgfpathlineto{\pgfqpoint{0.875240in}{5.114492in}}%
\pgfpathlineto{\pgfqpoint{0.884034in}{5.122864in}}%
\pgfpathlineto{\pgfqpoint{0.902984in}{5.140752in}}%
\pgfpathlineto{\pgfqpoint{0.913607in}{5.150656in}}%
\pgfpathlineto{\pgfqpoint{0.931298in}{5.167011in}}%
\pgfpathlineto{\pgfqpoint{0.943181in}{5.177862in}}%
\pgfpathlineto{\pgfqpoint{0.960193in}{5.193271in}}%
\pgfpathlineto{\pgfqpoint{0.972754in}{5.204508in}}%
\pgfpathlineto{\pgfqpoint{0.989683in}{5.219530in}}%
\pgfpathlineto{\pgfqpoint{1.002327in}{5.230614in}}%
\pgfpathlineto{\pgfqpoint{1.019778in}{5.245790in}}%
\pgfpathlineto{\pgfqpoint{1.031901in}{5.256203in}}%
\pgfpathlineto{\pgfqpoint{1.050492in}{5.272049in}}%
\pgfpathlineto{\pgfqpoint{1.061474in}{5.281296in}}%
\pgfpathlineto{\pgfqpoint{1.081835in}{5.298308in}}%
\pgfpathlineto{\pgfqpoint{1.091047in}{5.305912in}}%
\pgfpathlineto{\pgfqpoint{1.113820in}{5.324568in}}%
\pgfpathlineto{\pgfqpoint{1.120621in}{5.330072in}}%
\pgfpathlineto{\pgfqpoint{1.146457in}{5.350827in}}%
\pgfpathlineto{\pgfqpoint{1.150194in}{5.353793in}}%
\pgfpathlineto{\pgfqpoint{1.179757in}{5.377087in}}%
\pgfpathlineto{\pgfqpoint{1.179767in}{5.377095in}}%
\pgfpathlineto{\pgfqpoint{1.209341in}{5.399930in}}%
\pgfpathlineto{\pgfqpoint{1.213796in}{5.403346in}}%
\pgfpathlineto{\pgfqpoint{1.238914in}{5.422372in}}%
\pgfpathlineto{\pgfqpoint{1.248529in}{5.429606in}}%
\pgfpathlineto{\pgfqpoint{1.260747in}{5.438686in}}%
\pgfusepath{stroke}%
\end{pgfscope}%
\begin{pgfscope}%
\pgfpathrectangle{\pgfqpoint{0.854460in}{0.571603in}}{\pgfqpoint{5.885100in}{5.225635in}}%
\pgfusepath{clip}%
\pgfsetbuttcap%
\pgfsetroundjoin%
\pgfsetlinewidth{1.505625pt}%
\definecolor{currentstroke}{rgb}{0.196571,0.711827,0.479221}%
\pgfsetstrokecolor{currentstroke}%
\pgfsetdash{}{0pt}%
\pgfpathmoveto{\pgfqpoint{1.575956in}{5.653874in}}%
\pgfpathlineto{\pgfqpoint{1.593795in}{5.665085in}}%
\pgfpathlineto{\pgfqpoint{1.595168in}{5.665941in}}%
\pgfpathlineto{\pgfqpoint{1.623368in}{5.683313in}}%
\pgfpathlineto{\pgfqpoint{1.637867in}{5.692201in}}%
\pgfpathlineto{\pgfqpoint{1.652941in}{5.701327in}}%
\pgfpathlineto{\pgfqpoint{1.681374in}{5.718460in}}%
\pgfpathlineto{\pgfqpoint{1.682515in}{5.719139in}}%
\pgfpathlineto{\pgfqpoint{1.712088in}{5.736604in}}%
\pgfpathlineto{\pgfqpoint{1.725897in}{5.744720in}}%
\pgfpathlineto{\pgfqpoint{1.741661in}{5.753871in}}%
\pgfpathlineto{\pgfqpoint{1.771235in}{5.770964in}}%
\pgfpathlineto{\pgfqpoint{1.771262in}{5.770979in}}%
\pgfpathlineto{\pgfqpoint{1.800808in}{5.787711in}}%
\pgfpathlineto{\pgfqpoint{1.817704in}{5.797238in}}%
\pgfusepath{stroke}%
\end{pgfscope}%
\begin{pgfscope}%
\pgfpathrectangle{\pgfqpoint{0.854460in}{0.571603in}}{\pgfqpoint{5.885100in}{5.225635in}}%
\pgfusepath{clip}%
\pgfsetbuttcap%
\pgfsetroundjoin%
\pgfsetlinewidth{1.505625pt}%
\definecolor{currentstroke}{rgb}{0.196571,0.711827,0.479221}%
\pgfsetstrokecolor{currentstroke}%
\pgfsetdash{}{0pt}%
\pgfpathmoveto{\pgfqpoint{6.739560in}{5.350069in}}%
\pgfpathlineto{\pgfqpoint{6.728236in}{5.324568in}}%
\pgfpathlineto{\pgfqpoint{6.716658in}{5.298308in}}%
\pgfpathlineto{\pgfqpoint{6.714626in}{5.293682in}}%
\pgfusepath{stroke}%
\end{pgfscope}%
\begin{pgfscope}%
\pgfpathrectangle{\pgfqpoint{0.854460in}{0.571603in}}{\pgfqpoint{5.885100in}{5.225635in}}%
\pgfusepath{clip}%
\pgfsetbuttcap%
\pgfsetroundjoin%
\pgfsetlinewidth{1.505625pt}%
\definecolor{currentstroke}{rgb}{0.196571,0.711827,0.479221}%
\pgfsetstrokecolor{currentstroke}%
\pgfsetdash{}{0pt}%
\pgfpathmoveto{\pgfqpoint{6.563934in}{4.933562in}}%
\pgfpathlineto{\pgfqpoint{6.532547in}{4.851404in}}%
\pgfpathlineto{\pgfqpoint{6.494655in}{4.746860in}}%
\pgfpathlineto{\pgfqpoint{6.459174in}{4.641822in}}%
\pgfpathlineto{\pgfqpoint{6.426496in}{4.536784in}}%
\pgfpathlineto{\pgfqpoint{6.396813in}{4.431746in}}%
\pgfpathlineto{\pgfqpoint{6.376631in}{4.352967in}}%
\pgfpathlineto{\pgfqpoint{6.355107in}{4.259362in}}%
\pgfpathlineto{\pgfqpoint{6.341880in}{4.195411in}}%
\pgfpathlineto{\pgfqpoint{6.325533in}{4.104974in}}%
\pgfpathlineto{\pgfqpoint{6.315043in}{4.037854in}}%
\pgfpathlineto{\pgfqpoint{6.304725in}{3.959075in}}%
\pgfpathlineto{\pgfqpoint{6.295960in}{3.873313in}}%
\pgfpathlineto{\pgfqpoint{6.290525in}{3.801519in}}%
\pgfpathlineto{\pgfqpoint{6.286716in}{3.722740in}}%
\pgfpathlineto{\pgfqpoint{6.285164in}{3.643962in}}%
\pgfpathlineto{\pgfqpoint{6.285901in}{3.565183in}}%
\pgfpathlineto{\pgfqpoint{6.288961in}{3.486405in}}%
\pgfpathlineto{\pgfqpoint{6.294380in}{3.407626in}}%
\pgfpathlineto{\pgfqpoint{6.302147in}{3.328848in}}%
\pgfpathlineto{\pgfqpoint{6.312320in}{3.250070in}}%
\pgfpathlineto{\pgfqpoint{6.325533in}{3.168089in}}%
\pgfpathlineto{\pgfqpoint{6.339980in}{3.092513in}}%
\pgfpathlineto{\pgfqpoint{6.357519in}{3.013734in}}%
\pgfpathlineto{\pgfqpoint{6.377510in}{2.934956in}}%
\pgfpathlineto{\pgfqpoint{6.400006in}{2.856177in}}%
\pgfpathlineto{\pgfqpoint{6.425027in}{2.777399in}}%
\pgfpathlineto{\pgfqpoint{6.452575in}{2.698621in}}%
\pgfpathlineto{\pgfqpoint{6.482660in}{2.619842in}}%
\pgfpathlineto{\pgfqpoint{6.515292in}{2.541064in}}%
\pgfpathlineto{\pgfqpoint{6.550487in}{2.462285in}}%
\pgfpathlineto{\pgfqpoint{6.591693in}{2.376677in}}%
\pgfpathlineto{\pgfqpoint{6.628590in}{2.304729in}}%
\pgfpathlineto{\pgfqpoint{6.671502in}{2.225950in}}%
\pgfpathlineto{\pgfqpoint{6.717003in}{2.147172in}}%
\pgfpathlineto{\pgfqpoint{6.739560in}{2.109766in}}%
\pgfpathlineto{\pgfqpoint{6.739560in}{2.109766in}}%
\pgfusepath{stroke}%
\end{pgfscope}%
\begin{pgfscope}%
\pgfpathrectangle{\pgfqpoint{0.854460in}{0.571603in}}{\pgfqpoint{5.885100in}{5.225635in}}%
\pgfusepath{clip}%
\pgfsetbuttcap%
\pgfsetroundjoin%
\pgfsetlinewidth{1.505625pt}%
\definecolor{currentstroke}{rgb}{0.232815,0.732247,0.459277}%
\pgfsetstrokecolor{currentstroke}%
\pgfsetdash{}{0pt}%
\pgfpathmoveto{\pgfqpoint{0.854460in}{5.152895in}}%
\pgfpathlineto{\pgfqpoint{0.869485in}{5.167011in}}%
\pgfpathlineto{\pgfqpoint{0.884034in}{5.180513in}}%
\pgfpathlineto{\pgfqpoint{0.897895in}{5.193271in}}%
\pgfpathlineto{\pgfqpoint{0.913607in}{5.207555in}}%
\pgfpathlineto{\pgfqpoint{0.926887in}{5.219530in}}%
\pgfpathlineto{\pgfqpoint{0.943181in}{5.234044in}}%
\pgfpathlineto{\pgfqpoint{0.956472in}{5.245790in}}%
\pgfpathlineto{\pgfqpoint{0.972754in}{5.260003in}}%
\pgfpathlineto{\pgfqpoint{0.986662in}{5.272049in}}%
\pgfpathlineto{\pgfqpoint{1.002327in}{5.285452in}}%
\pgfpathlineto{\pgfqpoint{1.017470in}{5.298308in}}%
\pgfpathlineto{\pgfqpoint{1.031901in}{5.310411in}}%
\pgfpathlineto{\pgfqpoint{1.048907in}{5.324568in}}%
\pgfpathlineto{\pgfqpoint{1.061474in}{5.334902in}}%
\pgfpathlineto{\pgfqpoint{1.080983in}{5.350827in}}%
\pgfpathlineto{\pgfqpoint{1.091047in}{5.358943in}}%
\pgfpathlineto{\pgfqpoint{1.113711in}{5.377087in}}%
\pgfpathlineto{\pgfqpoint{1.120621in}{5.382552in}}%
\pgfpathlineto{\pgfqpoint{1.147100in}{5.403346in}}%
\pgfpathlineto{\pgfqpoint{1.150194in}{5.405747in}}%
\pgfpathlineto{\pgfqpoint{1.179767in}{5.428526in}}%
\pgfpathlineto{\pgfqpoint{1.181181in}{5.429606in}}%
\pgfpathlineto{\pgfqpoint{1.209341in}{5.450872in}}%
\pgfpathlineto{\pgfqpoint{1.215998in}{5.455865in}}%
\pgfpathlineto{\pgfqpoint{1.238914in}{5.472847in}}%
\pgfpathlineto{\pgfqpoint{1.251517in}{5.482125in}}%
\pgfpathlineto{\pgfqpoint{1.259804in}{5.488152in}}%
\pgfusepath{stroke}%
\end{pgfscope}%
\begin{pgfscope}%
\pgfpathrectangle{\pgfqpoint{0.854460in}{0.571603in}}{\pgfqpoint{5.885100in}{5.225635in}}%
\pgfusepath{clip}%
\pgfsetbuttcap%
\pgfsetroundjoin%
\pgfsetlinewidth{1.505625pt}%
\definecolor{currentstroke}{rgb}{0.232815,0.732247,0.459277}%
\pgfsetstrokecolor{currentstroke}%
\pgfsetdash{}{0pt}%
\pgfpathmoveto{\pgfqpoint{1.576740in}{5.700573in}}%
\pgfpathlineto{\pgfqpoint{1.593795in}{5.711069in}}%
\pgfpathlineto{\pgfqpoint{1.605867in}{5.718460in}}%
\pgfpathlineto{\pgfqpoint{1.623368in}{5.729044in}}%
\pgfpathlineto{\pgfqpoint{1.649410in}{5.744720in}}%
\pgfpathlineto{\pgfqpoint{1.652941in}{5.746819in}}%
\pgfpathlineto{\pgfqpoint{1.682515in}{5.764279in}}%
\pgfpathlineto{\pgfqpoint{1.693923in}{5.770979in}}%
\pgfpathlineto{\pgfqpoint{1.712088in}{5.781516in}}%
\pgfpathlineto{\pgfqpoint{1.739307in}{5.797238in}}%
\pgfusepath{stroke}%
\end{pgfscope}%
\begin{pgfscope}%
\pgfpathrectangle{\pgfqpoint{0.854460in}{0.571603in}}{\pgfqpoint{5.885100in}{5.225635in}}%
\pgfusepath{clip}%
\pgfsetbuttcap%
\pgfsetroundjoin%
\pgfsetlinewidth{1.505625pt}%
\definecolor{currentstroke}{rgb}{0.232815,0.732247,0.459277}%
\pgfsetstrokecolor{currentstroke}%
\pgfsetdash{}{0pt}%
\pgfpathmoveto{\pgfqpoint{6.739560in}{5.166380in}}%
\pgfpathlineto{\pgfqpoint{6.728245in}{5.140752in}}%
\pgfpathlineto{\pgfqpoint{6.716788in}{5.114492in}}%
\pgfpathlineto{\pgfqpoint{6.709987in}{5.098755in}}%
\pgfpathlineto{\pgfqpoint{6.705436in}{5.088233in}}%
\pgfpathlineto{\pgfqpoint{6.694176in}{5.061973in}}%
\pgfpathlineto{\pgfqpoint{6.683068in}{5.035714in}}%
\pgfpathlineto{\pgfqpoint{6.680414in}{5.029385in}}%
\pgfpathlineto{\pgfqpoint{6.672046in}{5.009454in}}%
\pgfpathlineto{\pgfqpoint{6.661163in}{4.983195in}}%
\pgfpathlineto{\pgfqpoint{6.650840in}{4.957913in}}%
\pgfpathlineto{\pgfqpoint{6.650440in}{4.956935in}}%
\pgfpathlineto{\pgfqpoint{6.639800in}{4.930676in}}%
\pgfpathlineto{\pgfqpoint{6.629331in}{4.904416in}}%
\pgfpathlineto{\pgfqpoint{6.621267in}{4.883874in}}%
\pgfpathlineto{\pgfqpoint{6.619019in}{4.878157in}}%
\pgfpathlineto{\pgfqpoint{6.608817in}{4.851897in}}%
\pgfpathlineto{\pgfqpoint{6.598797in}{4.825638in}}%
\pgfpathlineto{\pgfqpoint{6.591693in}{4.806708in}}%
\pgfpathlineto{\pgfqpoint{6.588938in}{4.799378in}}%
\pgfpathlineto{\pgfqpoint{6.579207in}{4.773119in}}%
\pgfpathlineto{\pgfqpoint{6.569666in}{4.746860in}}%
\pgfpathlineto{\pgfqpoint{6.562120in}{4.725691in}}%
\pgfpathlineto{\pgfqpoint{6.560301in}{4.720600in}}%
\pgfpathlineto{\pgfqpoint{6.551070in}{4.694341in}}%
\pgfpathlineto{\pgfqpoint{6.542037in}{4.668081in}}%
\pgfpathlineto{\pgfqpoint{6.533202in}{4.641822in}}%
\pgfpathlineto{\pgfqpoint{6.532547in}{4.639840in}}%
\pgfpathlineto{\pgfqpoint{6.530733in}{4.634367in}}%
\pgfusepath{stroke}%
\end{pgfscope}%
\begin{pgfscope}%
\pgfpathrectangle{\pgfqpoint{0.854460in}{0.571603in}}{\pgfqpoint{5.885100in}{5.225635in}}%
\pgfusepath{clip}%
\pgfsetbuttcap%
\pgfsetroundjoin%
\pgfsetlinewidth{1.505625pt}%
\definecolor{currentstroke}{rgb}{0.232815,0.732247,0.459277}%
\pgfsetstrokecolor{currentstroke}%
\pgfsetdash{}{0pt}%
\pgfpathmoveto{\pgfqpoint{6.425615in}{4.257219in}}%
\pgfpathlineto{\pgfqpoint{6.423542in}{4.247930in}}%
\pgfpathlineto{\pgfqpoint{6.417911in}{4.221670in}}%
\pgfpathlineto{\pgfqpoint{6.414253in}{4.203910in}}%
\pgfpathlineto{\pgfqpoint{6.412495in}{4.195411in}}%
\pgfpathlineto{\pgfqpoint{6.407283in}{4.169151in}}%
\pgfpathlineto{\pgfqpoint{6.402303in}{4.142892in}}%
\pgfpathlineto{\pgfqpoint{6.397556in}{4.116632in}}%
\pgfpathlineto{\pgfqpoint{6.393044in}{4.090373in}}%
\pgfpathlineto{\pgfqpoint{6.388768in}{4.064113in}}%
\pgfpathlineto{\pgfqpoint{6.384728in}{4.037854in}}%
\pgfpathlineto{\pgfqpoint{6.384680in}{4.037525in}}%
\pgfpathlineto{\pgfqpoint{6.380897in}{4.011594in}}%
\pgfpathlineto{\pgfqpoint{6.377306in}{3.985335in}}%
\pgfpathlineto{\pgfqpoint{6.373956in}{3.959075in}}%
\pgfpathlineto{\pgfqpoint{6.370849in}{3.932816in}}%
\pgfpathlineto{\pgfqpoint{6.367986in}{3.906556in}}%
\pgfpathlineto{\pgfqpoint{6.365367in}{3.880297in}}%
\pgfpathlineto{\pgfqpoint{6.362994in}{3.854037in}}%
\pgfpathlineto{\pgfqpoint{6.360867in}{3.827778in}}%
\pgfpathlineto{\pgfqpoint{6.358988in}{3.801519in}}%
\pgfpathlineto{\pgfqpoint{6.357357in}{3.775259in}}%
\pgfpathlineto{\pgfqpoint{6.355976in}{3.749000in}}%
\pgfpathlineto{\pgfqpoint{6.355107in}{3.728801in}}%
\pgfpathlineto{\pgfqpoint{6.354844in}{3.722740in}}%
\pgfpathlineto{\pgfqpoint{6.353959in}{3.696481in}}%
\pgfpathlineto{\pgfqpoint{6.353330in}{3.670221in}}%
\pgfpathlineto{\pgfqpoint{6.352956in}{3.643962in}}%
\pgfpathlineto{\pgfqpoint{6.352839in}{3.617702in}}%
\pgfpathlineto{\pgfqpoint{6.352981in}{3.591443in}}%
\pgfpathlineto{\pgfqpoint{6.353382in}{3.565183in}}%
\pgfpathlineto{\pgfqpoint{6.354044in}{3.538924in}}%
\pgfpathlineto{\pgfqpoint{6.354968in}{3.512664in}}%
\pgfpathlineto{\pgfqpoint{6.355107in}{3.509605in}}%
\pgfpathlineto{\pgfqpoint{6.356147in}{3.486405in}}%
\pgfpathlineto{\pgfqpoint{6.357588in}{3.460145in}}%
\pgfpathlineto{\pgfqpoint{6.359293in}{3.433886in}}%
\pgfpathlineto{\pgfqpoint{6.361263in}{3.407626in}}%
\pgfpathlineto{\pgfqpoint{6.363500in}{3.381367in}}%
\pgfpathlineto{\pgfqpoint{6.366004in}{3.355107in}}%
\pgfpathlineto{\pgfqpoint{6.368778in}{3.328848in}}%
\pgfpathlineto{\pgfqpoint{6.371822in}{3.302589in}}%
\pgfpathlineto{\pgfqpoint{6.375139in}{3.276329in}}%
\pgfpathlineto{\pgfqpoint{6.378729in}{3.250070in}}%
\pgfpathlineto{\pgfqpoint{6.382595in}{3.223810in}}%
\pgfpathlineto{\pgfqpoint{6.384680in}{3.210595in}}%
\pgfpathlineto{\pgfqpoint{6.386723in}{3.197551in}}%
\pgfpathlineto{\pgfqpoint{6.391111in}{3.171291in}}%
\pgfpathlineto{\pgfqpoint{6.395778in}{3.145032in}}%
\pgfpathlineto{\pgfqpoint{6.400725in}{3.118772in}}%
\pgfpathlineto{\pgfqpoint{6.405953in}{3.092513in}}%
\pgfpathlineto{\pgfqpoint{6.411464in}{3.066253in}}%
\pgfpathlineto{\pgfqpoint{6.414253in}{3.053613in}}%
\pgfpathlineto{\pgfqpoint{6.417238in}{3.039994in}}%
\pgfpathlineto{\pgfqpoint{6.423277in}{3.013734in}}%
\pgfpathlineto{\pgfqpoint{6.429603in}{2.987475in}}%
\pgfpathlineto{\pgfqpoint{6.436217in}{2.961215in}}%
\pgfpathlineto{\pgfqpoint{6.443121in}{2.934956in}}%
\pgfpathlineto{\pgfqpoint{6.443827in}{2.932380in}}%
\pgfpathlineto{\pgfqpoint{6.450272in}{2.908696in}}%
\pgfpathlineto{\pgfqpoint{6.457709in}{2.882437in}}%
\pgfpathlineto{\pgfqpoint{6.465441in}{2.856177in}}%
\pgfpathlineto{\pgfqpoint{6.473400in}{2.830143in}}%
\pgfpathlineto{\pgfqpoint{6.473468in}{2.829918in}}%
\pgfpathlineto{\pgfqpoint{6.481736in}{2.803659in}}%
\pgfpathlineto{\pgfqpoint{6.490302in}{2.777399in}}%
\pgfpathlineto{\pgfqpoint{6.499169in}{2.751140in}}%
\pgfpathlineto{\pgfqpoint{6.502973in}{2.740235in}}%
\pgfpathlineto{\pgfqpoint{6.508301in}{2.724880in}}%
\pgfpathlineto{\pgfqpoint{6.517709in}{2.698621in}}%
\pgfpathlineto{\pgfqpoint{6.527423in}{2.672361in}}%
\pgfpathlineto{\pgfqpoint{6.532547in}{2.658923in}}%
\pgfpathlineto{\pgfqpoint{6.537410in}{2.646102in}}%
\pgfpathlineto{\pgfqpoint{6.547669in}{2.619842in}}%
\pgfpathlineto{\pgfqpoint{6.558239in}{2.593583in}}%
\pgfpathlineto{\pgfqpoint{6.562120in}{2.584206in}}%
\pgfpathlineto{\pgfqpoint{6.569073in}{2.567323in}}%
\pgfpathlineto{\pgfqpoint{6.580193in}{2.541064in}}%
\pgfpathlineto{\pgfqpoint{6.591630in}{2.514804in}}%
\pgfpathlineto{\pgfqpoint{6.591693in}{2.514663in}}%
\pgfpathlineto{\pgfqpoint{6.603304in}{2.488545in}}%
\pgfpathlineto{\pgfqpoint{6.615296in}{2.462285in}}%
\pgfpathlineto{\pgfqpoint{6.621267in}{2.449538in}}%
\pgfpathlineto{\pgfqpoint{6.627567in}{2.436026in}}%
\pgfpathlineto{\pgfqpoint{6.640118in}{2.409766in}}%
\pgfpathlineto{\pgfqpoint{6.650840in}{2.387893in}}%
\pgfpathlineto{\pgfqpoint{6.652981in}{2.383507in}}%
\pgfpathlineto{\pgfqpoint{6.666094in}{2.357248in}}%
\pgfpathlineto{\pgfqpoint{6.679538in}{2.330988in}}%
\pgfpathlineto{\pgfqpoint{6.680414in}{2.329314in}}%
\pgfpathlineto{\pgfqpoint{6.693223in}{2.304729in}}%
\pgfpathlineto{\pgfqpoint{6.707236in}{2.278469in}}%
\pgfpathlineto{\pgfqpoint{6.709987in}{2.273422in}}%
\pgfpathlineto{\pgfqpoint{6.721504in}{2.252210in}}%
\pgfpathlineto{\pgfqpoint{6.736091in}{2.225950in}}%
\pgfpathlineto{\pgfqpoint{6.739560in}{2.219832in}}%
\pgfusepath{stroke}%
\end{pgfscope}%
\begin{pgfscope}%
\pgfpathrectangle{\pgfqpoint{0.854460in}{0.571603in}}{\pgfqpoint{5.885100in}{5.225635in}}%
\pgfusepath{clip}%
\pgfsetbuttcap%
\pgfsetroundjoin%
\pgfsetlinewidth{1.505625pt}%
\definecolor{currentstroke}{rgb}{0.266941,0.748751,0.440573}%
\pgfsetstrokecolor{currentstroke}%
\pgfsetdash{}{0pt}%
\pgfpathmoveto{\pgfqpoint{6.739560in}{4.988824in}}%
\pgfpathlineto{\pgfqpoint{6.737162in}{4.983195in}}%
\pgfpathlineto{\pgfqpoint{6.726088in}{4.956935in}}%
\pgfpathlineto{\pgfqpoint{6.722791in}{4.948990in}}%
\pgfusepath{stroke}%
\end{pgfscope}%
\begin{pgfscope}%
\pgfpathrectangle{\pgfqpoint{0.854460in}{0.571603in}}{\pgfqpoint{5.885100in}{5.225635in}}%
\pgfusepath{clip}%
\pgfsetbuttcap%
\pgfsetroundjoin%
\pgfsetlinewidth{1.505625pt}%
\definecolor{currentstroke}{rgb}{0.266941,0.748751,0.440573}%
\pgfsetstrokecolor{currentstroke}%
\pgfsetdash{}{0pt}%
\pgfpathmoveto{\pgfqpoint{6.586045in}{4.582953in}}%
\pgfpathlineto{\pgfqpoint{6.579572in}{4.563043in}}%
\pgfpathlineto{\pgfqpoint{6.571246in}{4.536784in}}%
\pgfpathlineto{\pgfqpoint{6.563136in}{4.510524in}}%
\pgfpathlineto{\pgfqpoint{6.562120in}{4.507161in}}%
\pgfpathlineto{\pgfqpoint{6.555187in}{4.484265in}}%
\pgfpathlineto{\pgfqpoint{6.547448in}{4.458005in}}%
\pgfpathlineto{\pgfqpoint{6.539930in}{4.431746in}}%
\pgfpathlineto{\pgfqpoint{6.532633in}{4.405486in}}%
\pgfpathlineto{\pgfqpoint{6.532547in}{4.405169in}}%
\pgfpathlineto{\pgfqpoint{6.525503in}{4.379227in}}%
\pgfpathlineto{\pgfqpoint{6.518598in}{4.352967in}}%
\pgfpathlineto{\pgfqpoint{6.511920in}{4.326708in}}%
\pgfpathlineto{\pgfqpoint{6.505468in}{4.300449in}}%
\pgfpathlineto{\pgfqpoint{6.502973in}{4.289949in}}%
\pgfpathlineto{\pgfqpoint{6.499216in}{4.274189in}}%
\pgfpathlineto{\pgfqpoint{6.493175in}{4.247930in}}%
\pgfpathlineto{\pgfqpoint{6.487367in}{4.221670in}}%
\pgfpathlineto{\pgfqpoint{6.481791in}{4.195411in}}%
\pgfpathlineto{\pgfqpoint{6.476450in}{4.169151in}}%
\pgfpathlineto{\pgfqpoint{6.473400in}{4.153492in}}%
\pgfpathlineto{\pgfqpoint{6.471327in}{4.142892in}}%
\pgfpathlineto{\pgfqpoint{6.466419in}{4.116632in}}%
\pgfpathlineto{\pgfqpoint{6.461749in}{4.090373in}}%
\pgfpathlineto{\pgfqpoint{6.457319in}{4.064113in}}%
\pgfpathlineto{\pgfqpoint{6.453129in}{4.037854in}}%
\pgfpathlineto{\pgfqpoint{6.449181in}{4.011594in}}%
\pgfpathlineto{\pgfqpoint{6.445474in}{3.985335in}}%
\pgfpathlineto{\pgfqpoint{6.443827in}{3.972859in}}%
\pgfpathlineto{\pgfqpoint{6.441998in}{3.959075in}}%
\pgfpathlineto{\pgfqpoint{6.438754in}{3.932816in}}%
\pgfpathlineto{\pgfqpoint{6.435758in}{3.906556in}}%
\pgfpathlineto{\pgfqpoint{6.433009in}{3.880297in}}%
\pgfpathlineto{\pgfqpoint{6.430509in}{3.854037in}}%
\pgfpathlineto{\pgfqpoint{6.428259in}{3.827778in}}%
\pgfpathlineto{\pgfqpoint{6.426259in}{3.801519in}}%
\pgfpathlineto{\pgfqpoint{6.424512in}{3.775259in}}%
\pgfpathlineto{\pgfqpoint{6.423016in}{3.749000in}}%
\pgfpathlineto{\pgfqpoint{6.421775in}{3.722740in}}%
\pgfpathlineto{\pgfqpoint{6.420788in}{3.696481in}}%
\pgfpathlineto{\pgfqpoint{6.420057in}{3.670221in}}%
\pgfpathlineto{\pgfqpoint{6.419582in}{3.643962in}}%
\pgfpathlineto{\pgfqpoint{6.419366in}{3.617702in}}%
\pgfpathlineto{\pgfqpoint{6.419409in}{3.591443in}}%
\pgfpathlineto{\pgfqpoint{6.419711in}{3.565183in}}%
\pgfpathlineto{\pgfqpoint{6.420275in}{3.538924in}}%
\pgfpathlineto{\pgfqpoint{6.421102in}{3.512664in}}%
\pgfpathlineto{\pgfqpoint{6.422192in}{3.486405in}}%
\pgfpathlineto{\pgfqpoint{6.423547in}{3.460145in}}%
\pgfpathlineto{\pgfqpoint{6.425168in}{3.433886in}}%
\pgfpathlineto{\pgfqpoint{6.427056in}{3.407626in}}%
\pgfpathlineto{\pgfqpoint{6.429213in}{3.381367in}}%
\pgfpathlineto{\pgfqpoint{6.431640in}{3.355107in}}%
\pgfpathlineto{\pgfqpoint{6.434339in}{3.328848in}}%
\pgfpathlineto{\pgfqpoint{6.437310in}{3.302589in}}%
\pgfpathlineto{\pgfqpoint{6.440556in}{3.276329in}}%
\pgfpathlineto{\pgfqpoint{6.443827in}{3.251936in}}%
\pgfpathlineto{\pgfqpoint{6.444075in}{3.250070in}}%
\pgfpathlineto{\pgfqpoint{6.447846in}{3.223810in}}%
\pgfpathlineto{\pgfqpoint{6.451894in}{3.197551in}}%
\pgfpathlineto{\pgfqpoint{6.456220in}{3.171291in}}%
\pgfpathlineto{\pgfqpoint{6.460826in}{3.145032in}}%
\pgfpathlineto{\pgfqpoint{6.465714in}{3.118772in}}%
\pgfpathlineto{\pgfqpoint{6.470884in}{3.092513in}}%
\pgfpathlineto{\pgfqpoint{6.473400in}{3.080401in}}%
\pgfpathlineto{\pgfqpoint{6.476319in}{3.066253in}}%
\pgfpathlineto{\pgfqpoint{6.482020in}{3.039994in}}%
\pgfpathlineto{\pgfqpoint{6.488008in}{3.013734in}}%
\pgfpathlineto{\pgfqpoint{6.494283in}{2.987475in}}%
\pgfpathlineto{\pgfqpoint{6.500849in}{2.961215in}}%
\pgfpathlineto{\pgfqpoint{6.502973in}{2.953076in}}%
\pgfpathlineto{\pgfqpoint{6.507673in}{2.934956in}}%
\pgfpathlineto{\pgfqpoint{6.514774in}{2.908696in}}%
\pgfpathlineto{\pgfqpoint{6.522169in}{2.882437in}}%
\pgfpathlineto{\pgfqpoint{6.529859in}{2.856177in}}%
\pgfpathlineto{\pgfqpoint{6.532547in}{2.847337in}}%
\pgfpathlineto{\pgfqpoint{6.537811in}{2.829918in}}%
\pgfpathlineto{\pgfqpoint{6.546041in}{2.803659in}}%
\pgfpathlineto{\pgfqpoint{6.554571in}{2.777399in}}%
\pgfpathlineto{\pgfqpoint{6.562120in}{2.754949in}}%
\pgfpathlineto{\pgfqpoint{6.563394in}{2.751140in}}%
\pgfpathlineto{\pgfqpoint{6.572466in}{2.724880in}}%
\pgfpathlineto{\pgfqpoint{6.581844in}{2.698621in}}%
\pgfpathlineto{\pgfqpoint{6.591528in}{2.672361in}}%
\pgfpathlineto{\pgfqpoint{6.591693in}{2.671925in}}%
\pgfpathlineto{\pgfqpoint{6.601452in}{2.646102in}}%
\pgfpathlineto{\pgfqpoint{6.611686in}{2.619842in}}%
\pgfpathlineto{\pgfqpoint{6.621267in}{2.595979in}}%
\pgfpathlineto{\pgfqpoint{6.622224in}{2.593583in}}%
\pgfpathlineto{\pgfqpoint{6.633008in}{2.567323in}}%
\pgfpathlineto{\pgfqpoint{6.644107in}{2.541064in}}%
\pgfpathlineto{\pgfqpoint{6.650840in}{2.525562in}}%
\pgfpathlineto{\pgfqpoint{6.655491in}{2.514804in}}%
\pgfpathlineto{\pgfqpoint{6.667146in}{2.488545in}}%
\pgfpathlineto{\pgfqpoint{6.679122in}{2.462285in}}%
\pgfpathlineto{\pgfqpoint{6.680414in}{2.459521in}}%
\pgfpathlineto{\pgfqpoint{6.691345in}{2.436026in}}%
\pgfpathlineto{\pgfqpoint{6.703883in}{2.409766in}}%
\pgfpathlineto{\pgfqpoint{6.709987in}{2.397290in}}%
\pgfpathlineto{\pgfqpoint{6.716701in}{2.383507in}}%
\pgfpathlineto{\pgfqpoint{6.729804in}{2.357248in}}%
\pgfpathlineto{\pgfqpoint{6.739560in}{2.338163in}}%
\pgfusepath{stroke}%
\end{pgfscope}%
\begin{pgfscope}%
\pgfpathrectangle{\pgfqpoint{0.854460in}{0.571603in}}{\pgfqpoint{5.885100in}{5.225635in}}%
\pgfusepath{clip}%
\pgfsetbuttcap%
\pgfsetroundjoin%
\pgfsetlinewidth{1.505625pt}%
\definecolor{currentstroke}{rgb}{0.266941,0.748751,0.440573}%
\pgfsetstrokecolor{currentstroke}%
\pgfsetdash{}{0pt}%
\pgfpathmoveto{\pgfqpoint{0.854460in}{5.208975in}}%
\pgfpathlineto{\pgfqpoint{0.865981in}{5.219530in}}%
\pgfpathlineto{\pgfqpoint{0.884034in}{5.235869in}}%
\pgfpathlineto{\pgfqpoint{0.886302in}{5.237905in}}%
\pgfusepath{stroke}%
\end{pgfscope}%
\begin{pgfscope}%
\pgfpathrectangle{\pgfqpoint{0.854460in}{0.571603in}}{\pgfqpoint{5.885100in}{5.225635in}}%
\pgfusepath{clip}%
\pgfsetbuttcap%
\pgfsetroundjoin%
\pgfsetlinewidth{1.505625pt}%
\definecolor{currentstroke}{rgb}{0.266941,0.748751,0.440573}%
\pgfsetstrokecolor{currentstroke}%
\pgfsetdash{}{0pt}%
\pgfpathmoveto{\pgfqpoint{1.182729in}{5.480327in}}%
\pgfpathlineto{\pgfqpoint{1.185132in}{5.482125in}}%
\pgfpathlineto{\pgfqpoint{1.209341in}{5.500010in}}%
\pgfpathlineto{\pgfqpoint{1.220749in}{5.508384in}}%
\pgfpathlineto{\pgfqpoint{1.238914in}{5.521558in}}%
\pgfpathlineto{\pgfqpoint{1.257073in}{5.534644in}}%
\pgfpathlineto{\pgfqpoint{1.268488in}{5.542771in}}%
\pgfpathlineto{\pgfqpoint{1.294112in}{5.560903in}}%
\pgfpathlineto{\pgfqpoint{1.298061in}{5.563664in}}%
\pgfpathlineto{\pgfqpoint{1.327634in}{5.584196in}}%
\pgfpathlineto{\pgfqpoint{1.331937in}{5.587163in}}%
\pgfpathlineto{\pgfqpoint{1.357208in}{5.604378in}}%
\pgfpathlineto{\pgfqpoint{1.370560in}{5.613422in}}%
\pgfpathlineto{\pgfqpoint{1.386781in}{5.624276in}}%
\pgfpathlineto{\pgfqpoint{1.409933in}{5.639682in}}%
\pgfpathlineto{\pgfqpoint{1.416354in}{5.643903in}}%
\pgfpathlineto{\pgfqpoint{1.445928in}{5.663221in}}%
\pgfpathlineto{\pgfqpoint{1.450120in}{5.665941in}}%
\pgfpathlineto{\pgfqpoint{1.475501in}{5.682209in}}%
\pgfpathlineto{\pgfqpoint{1.491171in}{5.692201in}}%
\pgfpathlineto{\pgfqpoint{1.505074in}{5.700958in}}%
\pgfpathlineto{\pgfqpoint{1.533000in}{5.718460in}}%
\pgfpathlineto{\pgfqpoint{1.534648in}{5.719481in}}%
\pgfpathlineto{\pgfqpoint{1.564221in}{5.737656in}}%
\pgfpathlineto{\pgfqpoint{1.575774in}{5.744720in}}%
\pgfpathlineto{\pgfqpoint{1.593795in}{5.755604in}}%
\pgfpathlineto{\pgfqpoint{1.619369in}{5.770979in}}%
\pgfpathlineto{\pgfqpoint{1.623368in}{5.773354in}}%
\pgfpathlineto{\pgfqpoint{1.652941in}{5.790797in}}%
\pgfpathlineto{\pgfqpoint{1.663920in}{5.797238in}}%
\pgfusepath{stroke}%
\end{pgfscope}%
\begin{pgfscope}%
\pgfpathrectangle{\pgfqpoint{0.854460in}{0.571603in}}{\pgfqpoint{5.885100in}{5.225635in}}%
\pgfusepath{clip}%
\pgfsetbuttcap%
\pgfsetroundjoin%
\pgfsetlinewidth{1.505625pt}%
\definecolor{currentstroke}{rgb}{0.311925,0.767822,0.415586}%
\pgfsetstrokecolor{currentstroke}%
\pgfsetdash{}{0pt}%
\pgfpathmoveto{\pgfqpoint{6.739560in}{4.809727in}}%
\pgfpathlineto{\pgfqpoint{6.735466in}{4.799378in}}%
\pgfpathlineto{\pgfqpoint{6.725239in}{4.773119in}}%
\pgfpathlineto{\pgfqpoint{6.715214in}{4.746860in}}%
\pgfpathlineto{\pgfqpoint{6.709987in}{4.732922in}}%
\pgfpathlineto{\pgfqpoint{6.705357in}{4.720600in}}%
\pgfpathlineto{\pgfqpoint{6.695666in}{4.694341in}}%
\pgfpathlineto{\pgfqpoint{6.686184in}{4.668081in}}%
\pgfpathlineto{\pgfqpoint{6.680414in}{4.651773in}}%
\pgfpathlineto{\pgfqpoint{6.676884in}{4.641822in}}%
\pgfpathlineto{\pgfqpoint{6.667753in}{4.615562in}}%
\pgfpathlineto{\pgfqpoint{6.658837in}{4.589303in}}%
\pgfpathlineto{\pgfqpoint{6.650840in}{4.565178in}}%
\pgfpathlineto{\pgfqpoint{6.650131in}{4.563043in}}%
\pgfpathlineto{\pgfqpoint{6.641581in}{4.536784in}}%
\pgfpathlineto{\pgfqpoint{6.633252in}{4.510524in}}%
\pgfpathlineto{\pgfqpoint{6.625144in}{4.484265in}}%
\pgfpathlineto{\pgfqpoint{6.621267in}{4.471385in}}%
\pgfpathlineto{\pgfqpoint{6.617227in}{4.458005in}}%
\pgfpathlineto{\pgfqpoint{6.609506in}{4.431746in}}%
\pgfpathlineto{\pgfqpoint{6.602010in}{4.405486in}}%
\pgfpathlineto{\pgfqpoint{6.594741in}{4.379227in}}%
\pgfpathlineto{\pgfqpoint{6.591693in}{4.367887in}}%
\pgfpathlineto{\pgfqpoint{6.587670in}{4.352967in}}%
\pgfpathlineto{\pgfqpoint{6.580807in}{4.326708in}}%
\pgfpathlineto{\pgfqpoint{6.574175in}{4.300449in}}%
\pgfpathlineto{\pgfqpoint{6.567775in}{4.274189in}}%
\pgfpathlineto{\pgfqpoint{6.563578in}{4.256322in}}%
\pgfusepath{stroke}%
\end{pgfscope}%
\begin{pgfscope}%
\pgfpathrectangle{\pgfqpoint{0.854460in}{0.571603in}}{\pgfqpoint{5.885100in}{5.225635in}}%
\pgfusepath{clip}%
\pgfsetbuttcap%
\pgfsetroundjoin%
\pgfsetlinewidth{1.505625pt}%
\definecolor{currentstroke}{rgb}{0.311925,0.767822,0.415586}%
\pgfsetstrokecolor{currentstroke}%
\pgfsetdash{}{0pt}%
\pgfpathmoveto{\pgfqpoint{6.498470in}{3.869602in}}%
\pgfpathlineto{\pgfqpoint{6.496917in}{3.854037in}}%
\pgfpathlineto{\pgfqpoint{6.494550in}{3.827778in}}%
\pgfpathlineto{\pgfqpoint{6.492436in}{3.801519in}}%
\pgfpathlineto{\pgfqpoint{6.490576in}{3.775259in}}%
\pgfpathlineto{\pgfqpoint{6.488972in}{3.749000in}}%
\pgfpathlineto{\pgfqpoint{6.487624in}{3.722740in}}%
\pgfpathlineto{\pgfqpoint{6.486533in}{3.696481in}}%
\pgfpathlineto{\pgfqpoint{6.485701in}{3.670221in}}%
\pgfpathlineto{\pgfqpoint{6.485128in}{3.643962in}}%
\pgfpathlineto{\pgfqpoint{6.484816in}{3.617702in}}%
\pgfpathlineto{\pgfqpoint{6.484764in}{3.591443in}}%
\pgfpathlineto{\pgfqpoint{6.484976in}{3.565183in}}%
\pgfpathlineto{\pgfqpoint{6.485450in}{3.538924in}}%
\pgfpathlineto{\pgfqpoint{6.486190in}{3.512664in}}%
\pgfpathlineto{\pgfqpoint{6.487195in}{3.486405in}}%
\pgfpathlineto{\pgfqpoint{6.488468in}{3.460145in}}%
\pgfpathlineto{\pgfqpoint{6.490008in}{3.433886in}}%
\pgfpathlineto{\pgfqpoint{6.491818in}{3.407626in}}%
\pgfpathlineto{\pgfqpoint{6.493899in}{3.381367in}}%
\pgfpathlineto{\pgfqpoint{6.496252in}{3.355107in}}%
\pgfpathlineto{\pgfqpoint{6.498878in}{3.328848in}}%
\pgfpathlineto{\pgfqpoint{6.501779in}{3.302589in}}%
\pgfpathlineto{\pgfqpoint{6.502973in}{3.292716in}}%
\pgfpathlineto{\pgfqpoint{6.504942in}{3.276329in}}%
\pgfpathlineto{\pgfqpoint{6.508372in}{3.250070in}}%
\pgfpathlineto{\pgfqpoint{6.512079in}{3.223810in}}%
\pgfpathlineto{\pgfqpoint{6.516064in}{3.197551in}}%
\pgfpathlineto{\pgfqpoint{6.520329in}{3.171291in}}%
\pgfpathlineto{\pgfqpoint{6.524876in}{3.145032in}}%
\pgfpathlineto{\pgfqpoint{6.529706in}{3.118772in}}%
\pgfpathlineto{\pgfqpoint{6.532547in}{3.104187in}}%
\pgfpathlineto{\pgfqpoint{6.534805in}{3.092513in}}%
\pgfpathlineto{\pgfqpoint{6.540168in}{3.066253in}}%
\pgfpathlineto{\pgfqpoint{6.545818in}{3.039994in}}%
\pgfpathlineto{\pgfqpoint{6.551755in}{3.013734in}}%
\pgfpathlineto{\pgfqpoint{6.557982in}{2.987475in}}%
\pgfpathlineto{\pgfqpoint{6.562120in}{2.970800in}}%
\pgfpathlineto{\pgfqpoint{6.564484in}{2.961215in}}%
\pgfpathlineto{\pgfqpoint{6.571248in}{2.934956in}}%
\pgfpathlineto{\pgfqpoint{6.578306in}{2.908696in}}%
\pgfpathlineto{\pgfqpoint{6.585658in}{2.882437in}}%
\pgfpathlineto{\pgfqpoint{6.591693in}{2.861714in}}%
\pgfpathlineto{\pgfqpoint{6.593296in}{2.856177in}}%
\pgfpathlineto{\pgfqpoint{6.601190in}{2.829918in}}%
\pgfpathlineto{\pgfqpoint{6.609383in}{2.803659in}}%
\pgfpathlineto{\pgfqpoint{6.617877in}{2.777399in}}%
\pgfpathlineto{\pgfqpoint{6.621267in}{2.767272in}}%
\pgfpathlineto{\pgfqpoint{6.626637in}{2.751140in}}%
\pgfpathlineto{\pgfqpoint{6.635677in}{2.724880in}}%
\pgfpathlineto{\pgfqpoint{6.645023in}{2.698621in}}%
\pgfpathlineto{\pgfqpoint{6.650840in}{2.682787in}}%
\pgfpathlineto{\pgfqpoint{6.654650in}{2.672361in}}%
\pgfpathlineto{\pgfqpoint{6.664546in}{2.646102in}}%
\pgfpathlineto{\pgfqpoint{6.674753in}{2.619842in}}%
\pgfpathlineto{\pgfqpoint{6.680414in}{2.605698in}}%
\pgfpathlineto{\pgfqpoint{6.685239in}{2.593583in}}%
\pgfpathlineto{\pgfqpoint{6.695999in}{2.567323in}}%
\pgfpathlineto{\pgfqpoint{6.707076in}{2.541064in}}%
\pgfpathlineto{\pgfqpoint{6.709987in}{2.534344in}}%
\pgfpathlineto{\pgfqpoint{6.718413in}{2.514804in}}%
\pgfpathlineto{\pgfqpoint{6.730049in}{2.488545in}}%
\pgfpathlineto{\pgfqpoint{6.739560in}{2.467646in}}%
\pgfusepath{stroke}%
\end{pgfscope}%
\begin{pgfscope}%
\pgfpathrectangle{\pgfqpoint{0.854460in}{0.571603in}}{\pgfqpoint{5.885100in}{5.225635in}}%
\pgfusepath{clip}%
\pgfsetbuttcap%
\pgfsetroundjoin%
\pgfsetlinewidth{1.505625pt}%
\definecolor{currentstroke}{rgb}{0.311925,0.767822,0.415586}%
\pgfsetstrokecolor{currentstroke}%
\pgfsetdash{}{0pt}%
\pgfpathmoveto{\pgfqpoint{0.854460in}{5.262885in}}%
\pgfpathlineto{\pgfqpoint{0.864712in}{5.272049in}}%
\pgfpathlineto{\pgfqpoint{0.884034in}{5.289110in}}%
\pgfpathlineto{\pgfqpoint{0.884966in}{5.289926in}}%
\pgfusepath{stroke}%
\end{pgfscope}%
\begin{pgfscope}%
\pgfpathrectangle{\pgfqpoint{0.854460in}{0.571603in}}{\pgfqpoint{5.885100in}{5.225635in}}%
\pgfusepath{clip}%
\pgfsetbuttcap%
\pgfsetroundjoin%
\pgfsetlinewidth{1.505625pt}%
\definecolor{currentstroke}{rgb}{0.311925,0.767822,0.415586}%
\pgfsetstrokecolor{currentstroke}%
\pgfsetdash{}{0pt}%
\pgfpathmoveto{\pgfqpoint{1.183880in}{5.529013in}}%
\pgfpathlineto{\pgfqpoint{1.191573in}{5.534644in}}%
\pgfpathlineto{\pgfqpoint{1.209341in}{5.547490in}}%
\pgfpathlineto{\pgfqpoint{1.228006in}{5.560903in}}%
\pgfpathlineto{\pgfqpoint{1.238914in}{5.568647in}}%
\pgfpathlineto{\pgfqpoint{1.265151in}{5.587163in}}%
\pgfpathlineto{\pgfqpoint{1.268488in}{5.589489in}}%
\pgfpathlineto{\pgfqpoint{1.298061in}{5.609964in}}%
\pgfpathlineto{\pgfqpoint{1.303088in}{5.613422in}}%
\pgfpathlineto{\pgfqpoint{1.327634in}{5.630102in}}%
\pgfpathlineto{\pgfqpoint{1.341810in}{5.639682in}}%
\pgfpathlineto{\pgfqpoint{1.357208in}{5.649960in}}%
\pgfpathlineto{\pgfqpoint{1.381277in}{5.665941in}}%
\pgfpathlineto{\pgfqpoint{1.386781in}{5.669551in}}%
\pgfpathlineto{\pgfqpoint{1.416354in}{5.688825in}}%
\pgfpathlineto{\pgfqpoint{1.421568in}{5.692201in}}%
\pgfpathlineto{\pgfqpoint{1.445928in}{5.707783in}}%
\pgfpathlineto{\pgfqpoint{1.462705in}{5.718460in}}%
\pgfpathlineto{\pgfqpoint{1.475501in}{5.726505in}}%
\pgfpathlineto{\pgfqpoint{1.504613in}{5.744720in}}%
\pgfpathlineto{\pgfqpoint{1.505074in}{5.745004in}}%
\pgfpathlineto{\pgfqpoint{1.534648in}{5.763146in}}%
\pgfpathlineto{\pgfqpoint{1.547478in}{5.770979in}}%
\pgfpathlineto{\pgfqpoint{1.564221in}{5.781076in}}%
\pgfpathlineto{\pgfqpoint{1.591141in}{5.797238in}}%
\pgfusepath{stroke}%
\end{pgfscope}%
\begin{pgfscope}%
\pgfpathrectangle{\pgfqpoint{0.854460in}{0.571603in}}{\pgfqpoint{5.885100in}{5.225635in}}%
\pgfusepath{clip}%
\pgfsetbuttcap%
\pgfsetroundjoin%
\pgfsetlinewidth{1.505625pt}%
\definecolor{currentstroke}{rgb}{0.352360,0.783011,0.392636}%
\pgfsetstrokecolor{currentstroke}%
\pgfsetdash{}{0pt}%
\pgfpathmoveto{\pgfqpoint{6.739560in}{4.621564in}}%
\pgfpathlineto{\pgfqpoint{6.737430in}{4.615562in}}%
\pgfpathlineto{\pgfqpoint{6.728292in}{4.589303in}}%
\pgfpathlineto{\pgfqpoint{6.719376in}{4.563043in}}%
\pgfpathlineto{\pgfqpoint{6.710681in}{4.536784in}}%
\pgfpathlineto{\pgfqpoint{6.709987in}{4.534642in}}%
\pgfpathlineto{\pgfqpoint{6.702151in}{4.510524in}}%
\pgfpathlineto{\pgfqpoint{6.693842in}{4.484265in}}%
\pgfpathlineto{\pgfqpoint{6.685759in}{4.458005in}}%
\pgfpathlineto{\pgfqpoint{6.680414in}{4.440159in}}%
\pgfpathlineto{\pgfqpoint{6.677886in}{4.431746in}}%
\pgfpathlineto{\pgfqpoint{6.670203in}{4.405486in}}%
\pgfpathlineto{\pgfqpoint{6.662751in}{4.379227in}}%
\pgfpathlineto{\pgfqpoint{6.655532in}{4.352967in}}%
\pgfpathlineto{\pgfqpoint{6.650840in}{4.335356in}}%
\pgfpathlineto{\pgfqpoint{6.648529in}{4.326708in}}%
\pgfpathlineto{\pgfqpoint{6.641726in}{4.300449in}}%
\pgfpathlineto{\pgfqpoint{6.635161in}{4.274189in}}%
\pgfpathlineto{\pgfqpoint{6.628832in}{4.247930in}}%
\pgfpathlineto{\pgfqpoint{6.622741in}{4.221670in}}%
\pgfpathlineto{\pgfqpoint{6.621267in}{4.215070in}}%
\pgfpathlineto{\pgfqpoint{6.616858in}{4.195411in}}%
\pgfpathlineto{\pgfqpoint{6.611204in}{4.169151in}}%
\pgfpathlineto{\pgfqpoint{6.605794in}{4.142892in}}%
\pgfpathlineto{\pgfqpoint{6.600626in}{4.116632in}}%
\pgfpathlineto{\pgfqpoint{6.595701in}{4.090373in}}%
\pgfpathlineto{\pgfqpoint{6.591693in}{4.067894in}}%
\pgfpathlineto{\pgfqpoint{6.591017in}{4.064113in}}%
\pgfpathlineto{\pgfqpoint{6.586550in}{4.037854in}}%
\pgfpathlineto{\pgfqpoint{6.582331in}{4.011594in}}%
\pgfpathlineto{\pgfqpoint{6.578361in}{3.985335in}}%
\pgfpathlineto{\pgfqpoint{6.574640in}{3.959075in}}%
\pgfpathlineto{\pgfqpoint{6.571169in}{3.932816in}}%
\pgfpathlineto{\pgfqpoint{6.567950in}{3.906556in}}%
\pgfpathlineto{\pgfqpoint{6.564982in}{3.880297in}}%
\pgfpathlineto{\pgfqpoint{6.562267in}{3.854037in}}%
\pgfpathlineto{\pgfqpoint{6.562120in}{3.852475in}}%
\pgfpathlineto{\pgfqpoint{6.559789in}{3.827778in}}%
\pgfpathlineto{\pgfqpoint{6.557566in}{3.801519in}}%
\pgfpathlineto{\pgfqpoint{6.555599in}{3.775259in}}%
\pgfpathlineto{\pgfqpoint{6.553891in}{3.749000in}}%
\pgfpathlineto{\pgfqpoint{6.552442in}{3.722740in}}%
\pgfpathlineto{\pgfqpoint{6.551253in}{3.696481in}}%
\pgfpathlineto{\pgfqpoint{6.550324in}{3.670221in}}%
\pgfpathlineto{\pgfqpoint{6.549657in}{3.643962in}}%
\pgfpathlineto{\pgfqpoint{6.549252in}{3.617702in}}%
\pgfpathlineto{\pgfqpoint{6.549111in}{3.591443in}}%
\pgfpathlineto{\pgfqpoint{6.549235in}{3.565183in}}%
\pgfpathlineto{\pgfqpoint{6.549624in}{3.538924in}}%
\pgfpathlineto{\pgfqpoint{6.550281in}{3.512664in}}%
\pgfpathlineto{\pgfqpoint{6.551205in}{3.486405in}}%
\pgfpathlineto{\pgfqpoint{6.552398in}{3.460145in}}%
\pgfpathlineto{\pgfqpoint{6.553212in}{3.445537in}}%
\pgfusepath{stroke}%
\end{pgfscope}%
\begin{pgfscope}%
\pgfpathrectangle{\pgfqpoint{0.854460in}{0.571603in}}{\pgfqpoint{5.885100in}{5.225635in}}%
\pgfusepath{clip}%
\pgfsetbuttcap%
\pgfsetroundjoin%
\pgfsetlinewidth{1.505625pt}%
\definecolor{currentstroke}{rgb}{0.352360,0.783011,0.392636}%
\pgfsetstrokecolor{currentstroke}%
\pgfsetdash{}{0pt}%
\pgfpathmoveto{\pgfqpoint{6.605099in}{3.056763in}}%
\pgfpathlineto{\pgfqpoint{6.608675in}{3.039994in}}%
\pgfpathlineto{\pgfqpoint{6.614563in}{3.013734in}}%
\pgfpathlineto{\pgfqpoint{6.620742in}{2.987475in}}%
\pgfpathlineto{\pgfqpoint{6.621267in}{2.985342in}}%
\pgfpathlineto{\pgfqpoint{6.627173in}{2.961215in}}%
\pgfpathlineto{\pgfqpoint{6.633893in}{2.934956in}}%
\pgfpathlineto{\pgfqpoint{6.640907in}{2.908696in}}%
\pgfpathlineto{\pgfqpoint{6.648218in}{2.882437in}}%
\pgfpathlineto{\pgfqpoint{6.650840in}{2.873381in}}%
\pgfpathlineto{\pgfqpoint{6.655793in}{2.856177in}}%
\pgfpathlineto{\pgfqpoint{6.663649in}{2.829918in}}%
\pgfpathlineto{\pgfqpoint{6.671804in}{2.803659in}}%
\pgfpathlineto{\pgfqpoint{6.680262in}{2.777399in}}%
\pgfpathlineto{\pgfqpoint{6.680414in}{2.776945in}}%
\pgfpathlineto{\pgfqpoint{6.688966in}{2.751140in}}%
\pgfpathlineto{\pgfqpoint{6.697974in}{2.724880in}}%
\pgfpathlineto{\pgfqpoint{6.707288in}{2.698621in}}%
\pgfpathlineto{\pgfqpoint{6.709987in}{2.691247in}}%
\pgfpathlineto{\pgfqpoint{6.716864in}{2.672361in}}%
\pgfpathlineto{\pgfqpoint{6.726732in}{2.646102in}}%
\pgfpathlineto{\pgfqpoint{6.736911in}{2.619842in}}%
\pgfpathlineto{\pgfqpoint{6.739560in}{2.613202in}}%
\pgfusepath{stroke}%
\end{pgfscope}%
\begin{pgfscope}%
\pgfpathrectangle{\pgfqpoint{0.854460in}{0.571603in}}{\pgfqpoint{5.885100in}{5.225635in}}%
\pgfusepath{clip}%
\pgfsetbuttcap%
\pgfsetroundjoin%
\pgfsetlinewidth{1.505625pt}%
\definecolor{currentstroke}{rgb}{0.352360,0.783011,0.392636}%
\pgfsetstrokecolor{currentstroke}%
\pgfsetdash{}{0pt}%
\pgfpathmoveto{\pgfqpoint{0.854460in}{5.314795in}}%
\pgfpathlineto{\pgfqpoint{0.865663in}{5.324568in}}%
\pgfpathlineto{\pgfqpoint{0.883727in}{5.340135in}}%
\pgfusepath{stroke}%
\end{pgfscope}%
\begin{pgfscope}%
\pgfpathrectangle{\pgfqpoint{0.854460in}{0.571603in}}{\pgfqpoint{5.885100in}{5.225635in}}%
\pgfusepath{clip}%
\pgfsetbuttcap%
\pgfsetroundjoin%
\pgfsetlinewidth{1.505625pt}%
\definecolor{currentstroke}{rgb}{0.352360,0.783011,0.392636}%
\pgfsetstrokecolor{currentstroke}%
\pgfsetdash{}{0pt}%
\pgfpathmoveto{\pgfqpoint{1.184956in}{5.576052in}}%
\pgfpathlineto{\pgfqpoint{1.200463in}{5.587163in}}%
\pgfpathlineto{\pgfqpoint{1.209341in}{5.593447in}}%
\pgfpathlineto{\pgfqpoint{1.237727in}{5.613422in}}%
\pgfpathlineto{\pgfqpoint{1.238914in}{5.614247in}}%
\pgfpathlineto{\pgfqpoint{1.268488in}{5.634658in}}%
\pgfpathlineto{\pgfqpoint{1.275810in}{5.639682in}}%
\pgfpathlineto{\pgfqpoint{1.298061in}{5.654763in}}%
\pgfpathlineto{\pgfqpoint{1.314644in}{5.665941in}}%
\pgfpathlineto{\pgfqpoint{1.327634in}{5.674592in}}%
\pgfpathlineto{\pgfqpoint{1.354219in}{5.692201in}}%
\pgfpathlineto{\pgfqpoint{1.357208in}{5.694156in}}%
\pgfpathlineto{\pgfqpoint{1.386781in}{5.713376in}}%
\pgfpathlineto{\pgfqpoint{1.394649in}{5.718460in}}%
\pgfpathlineto{\pgfqpoint{1.416354in}{5.732315in}}%
\pgfpathlineto{\pgfqpoint{1.435884in}{5.744720in}}%
\pgfpathlineto{\pgfqpoint{1.445928in}{5.751022in}}%
\pgfpathlineto{\pgfqpoint{1.475501in}{5.769480in}}%
\pgfpathlineto{\pgfqpoint{1.477920in}{5.770979in}}%
\pgfpathlineto{\pgfqpoint{1.505074in}{5.787606in}}%
\pgfpathlineto{\pgfqpoint{1.520878in}{5.797238in}}%
\pgfusepath{stroke}%
\end{pgfscope}%
\begin{pgfscope}%
\pgfpathrectangle{\pgfqpoint{0.854460in}{0.571603in}}{\pgfqpoint{5.885100in}{5.225635in}}%
\pgfusepath{clip}%
\pgfsetbuttcap%
\pgfsetroundjoin%
\pgfsetlinewidth{1.505625pt}%
\definecolor{currentstroke}{rgb}{0.404001,0.800275,0.362552}%
\pgfsetstrokecolor{currentstroke}%
\pgfsetdash{}{0pt}%
\pgfpathmoveto{\pgfqpoint{6.739560in}{4.413175in}}%
\pgfpathlineto{\pgfqpoint{6.737266in}{4.405486in}}%
\pgfpathlineto{\pgfqpoint{6.729642in}{4.379227in}}%
\pgfpathlineto{\pgfqpoint{6.722254in}{4.352967in}}%
\pgfpathlineto{\pgfqpoint{6.715102in}{4.326708in}}%
\pgfpathlineto{\pgfqpoint{6.709987in}{4.307297in}}%
\pgfpathlineto{\pgfqpoint{6.708176in}{4.300449in}}%
\pgfpathlineto{\pgfqpoint{6.701453in}{4.274189in}}%
\pgfpathlineto{\pgfqpoint{6.694970in}{4.247930in}}%
\pgfpathlineto{\pgfqpoint{6.688730in}{4.221670in}}%
\pgfpathlineto{\pgfqpoint{6.682731in}{4.195411in}}%
\pgfpathlineto{\pgfqpoint{6.680414in}{4.184858in}}%
\pgfpathlineto{\pgfqpoint{6.676951in}{4.169151in}}%
\pgfpathlineto{\pgfqpoint{6.671400in}{4.142892in}}%
\pgfpathlineto{\pgfqpoint{6.666095in}{4.116632in}}%
\pgfpathlineto{\pgfqpoint{6.661037in}{4.090373in}}%
\pgfpathlineto{\pgfqpoint{6.656226in}{4.064113in}}%
\pgfpathlineto{\pgfqpoint{6.651664in}{4.037854in}}%
\pgfpathlineto{\pgfqpoint{6.650840in}{4.032849in}}%
\pgfpathlineto{\pgfqpoint{6.647326in}{4.011594in}}%
\pgfpathlineto{\pgfqpoint{6.643234in}{3.985335in}}%
\pgfpathlineto{\pgfqpoint{6.639394in}{3.959075in}}%
\pgfpathlineto{\pgfqpoint{6.635807in}{3.932816in}}%
\pgfpathlineto{\pgfqpoint{6.632474in}{3.906556in}}%
\pgfpathlineto{\pgfqpoint{6.629395in}{3.880297in}}%
\pgfpathlineto{\pgfqpoint{6.626571in}{3.854037in}}%
\pgfpathlineto{\pgfqpoint{6.624003in}{3.827778in}}%
\pgfpathlineto{\pgfqpoint{6.621693in}{3.801519in}}%
\pgfpathlineto{\pgfqpoint{6.621267in}{3.796081in}}%
\pgfpathlineto{\pgfqpoint{6.620606in}{3.787678in}}%
\pgfusepath{stroke}%
\end{pgfscope}%
\begin{pgfscope}%
\pgfpathrectangle{\pgfqpoint{0.854460in}{0.571603in}}{\pgfqpoint{5.885100in}{5.225635in}}%
\pgfusepath{clip}%
\pgfsetbuttcap%
\pgfsetroundjoin%
\pgfsetlinewidth{1.505625pt}%
\definecolor{currentstroke}{rgb}{0.404001,0.800275,0.362552}%
\pgfsetstrokecolor{currentstroke}%
\pgfsetdash{}{0pt}%
\pgfpathmoveto{\pgfqpoint{6.619351in}{3.395158in}}%
\pgfpathlineto{\pgfqpoint{6.620368in}{3.381367in}}%
\pgfpathlineto{\pgfqpoint{6.621267in}{3.370701in}}%
\pgfpathlineto{\pgfqpoint{6.622571in}{3.355107in}}%
\pgfpathlineto{\pgfqpoint{6.625043in}{3.328848in}}%
\pgfpathlineto{\pgfqpoint{6.627791in}{3.302589in}}%
\pgfpathlineto{\pgfqpoint{6.630817in}{3.276329in}}%
\pgfpathlineto{\pgfqpoint{6.634121in}{3.250070in}}%
\pgfpathlineto{\pgfqpoint{6.637706in}{3.223810in}}%
\pgfpathlineto{\pgfqpoint{6.641572in}{3.197551in}}%
\pgfpathlineto{\pgfqpoint{6.645721in}{3.171291in}}%
\pgfpathlineto{\pgfqpoint{6.650155in}{3.145032in}}%
\pgfpathlineto{\pgfqpoint{6.650840in}{3.141217in}}%
\pgfpathlineto{\pgfqpoint{6.654847in}{3.118772in}}%
\pgfpathlineto{\pgfqpoint{6.659820in}{3.092513in}}%
\pgfpathlineto{\pgfqpoint{6.665081in}{3.066253in}}%
\pgfpathlineto{\pgfqpoint{6.670630in}{3.039994in}}%
\pgfpathlineto{\pgfqpoint{6.676470in}{3.013734in}}%
\pgfpathlineto{\pgfqpoint{6.680414in}{2.996844in}}%
\pgfpathlineto{\pgfqpoint{6.682587in}{2.987475in}}%
\pgfpathlineto{\pgfqpoint{6.688970in}{2.961215in}}%
\pgfpathlineto{\pgfqpoint{6.695647in}{2.934956in}}%
\pgfpathlineto{\pgfqpoint{6.702619in}{2.908696in}}%
\pgfpathlineto{\pgfqpoint{6.709888in}{2.882437in}}%
\pgfpathlineto{\pgfqpoint{6.709987in}{2.882095in}}%
\pgfpathlineto{\pgfqpoint{6.717407in}{2.856177in}}%
\pgfpathlineto{\pgfqpoint{6.725225in}{2.829918in}}%
\pgfpathlineto{\pgfqpoint{6.733343in}{2.803659in}}%
\pgfpathlineto{\pgfqpoint{6.739560in}{2.784268in}}%
\pgfusepath{stroke}%
\end{pgfscope}%
\begin{pgfscope}%
\pgfpathrectangle{\pgfqpoint{0.854460in}{0.571603in}}{\pgfqpoint{5.885100in}{5.225635in}}%
\pgfusepath{clip}%
\pgfsetbuttcap%
\pgfsetroundjoin%
\pgfsetlinewidth{1.505625pt}%
\definecolor{currentstroke}{rgb}{0.404001,0.800275,0.362552}%
\pgfsetstrokecolor{currentstroke}%
\pgfsetdash{}{0pt}%
\pgfpathmoveto{\pgfqpoint{0.854460in}{5.364861in}}%
\pgfpathlineto{\pgfqpoint{0.868815in}{5.377087in}}%
\pgfpathlineto{\pgfqpoint{0.882580in}{5.388669in}}%
\pgfusepath{stroke}%
\end{pgfscope}%
\begin{pgfscope}%
\pgfpathrectangle{\pgfqpoint{0.854460in}{0.571603in}}{\pgfqpoint{5.885100in}{5.225635in}}%
\pgfusepath{clip}%
\pgfsetbuttcap%
\pgfsetroundjoin%
\pgfsetlinewidth{1.505625pt}%
\definecolor{currentstroke}{rgb}{0.404001,0.800275,0.362552}%
\pgfsetstrokecolor{currentstroke}%
\pgfsetdash{}{0pt}%
\pgfpathmoveto{\pgfqpoint{1.185961in}{5.621578in}}%
\pgfpathlineto{\pgfqpoint{1.209341in}{5.637973in}}%
\pgfpathlineto{\pgfqpoint{1.211794in}{5.639682in}}%
\pgfpathlineto{\pgfqpoint{1.238914in}{5.658342in}}%
\pgfpathlineto{\pgfqpoint{1.250021in}{5.665941in}}%
\pgfpathlineto{\pgfqpoint{1.268488in}{5.678424in}}%
\pgfpathlineto{\pgfqpoint{1.288978in}{5.692201in}}%
\pgfpathlineto{\pgfqpoint{1.298061in}{5.698233in}}%
\pgfpathlineto{\pgfqpoint{1.327634in}{5.717769in}}%
\pgfpathlineto{\pgfqpoint{1.328687in}{5.718460in}}%
\pgfpathlineto{\pgfqpoint{1.357208in}{5.736940in}}%
\pgfpathlineto{\pgfqpoint{1.369275in}{5.744720in}}%
\pgfpathlineto{\pgfqpoint{1.386781in}{5.755869in}}%
\pgfpathlineto{\pgfqpoint{1.410620in}{5.770979in}}%
\pgfpathlineto{\pgfqpoint{1.416354in}{5.774569in}}%
\pgfpathlineto{\pgfqpoint{1.445928in}{5.792973in}}%
\pgfpathlineto{\pgfqpoint{1.452822in}{5.797238in}}%
\pgfusepath{stroke}%
\end{pgfscope}%
\begin{pgfscope}%
\pgfpathrectangle{\pgfqpoint{0.854460in}{0.571603in}}{\pgfqpoint{5.885100in}{5.225635in}}%
\pgfusepath{clip}%
\pgfsetbuttcap%
\pgfsetroundjoin%
\pgfsetlinewidth{1.505625pt}%
\definecolor{currentstroke}{rgb}{0.449368,0.813768,0.335384}%
\pgfsetstrokecolor{currentstroke}%
\pgfsetdash{}{0pt}%
\pgfpathmoveto{\pgfqpoint{6.739560in}{4.159402in}}%
\pgfpathlineto{\pgfqpoint{6.735991in}{4.142892in}}%
\pgfpathlineto{\pgfqpoint{6.730556in}{4.116632in}}%
\pgfpathlineto{\pgfqpoint{6.725371in}{4.090373in}}%
\pgfpathlineto{\pgfqpoint{6.722350in}{4.074299in}}%
\pgfusepath{stroke}%
\end{pgfscope}%
\begin{pgfscope}%
\pgfpathrectangle{\pgfqpoint{0.854460in}{0.571603in}}{\pgfqpoint{5.885100in}{5.225635in}}%
\pgfusepath{clip}%
\pgfsetbuttcap%
\pgfsetroundjoin%
\pgfsetlinewidth{1.505625pt}%
\definecolor{currentstroke}{rgb}{0.449368,0.813768,0.335384}%
\pgfsetstrokecolor{currentstroke}%
\pgfsetdash{}{0pt}%
\pgfpathmoveto{\pgfqpoint{6.677286in}{3.684548in}}%
\pgfpathlineto{\pgfqpoint{6.676681in}{3.670221in}}%
\pgfpathlineto{\pgfqpoint{6.675838in}{3.643962in}}%
\pgfpathlineto{\pgfqpoint{6.675261in}{3.617702in}}%
\pgfpathlineto{\pgfqpoint{6.674952in}{3.591443in}}%
\pgfpathlineto{\pgfqpoint{6.674911in}{3.565183in}}%
\pgfpathlineto{\pgfqpoint{6.675140in}{3.538924in}}%
\pgfpathlineto{\pgfqpoint{6.675640in}{3.512664in}}%
\pgfpathlineto{\pgfqpoint{6.676411in}{3.486405in}}%
\pgfpathlineto{\pgfqpoint{6.677455in}{3.460145in}}%
\pgfpathlineto{\pgfqpoint{6.678773in}{3.433886in}}%
\pgfpathlineto{\pgfqpoint{6.680365in}{3.407626in}}%
\pgfpathlineto{\pgfqpoint{6.680414in}{3.406948in}}%
\pgfpathlineto{\pgfqpoint{6.682221in}{3.381367in}}%
\pgfpathlineto{\pgfqpoint{6.684353in}{3.355107in}}%
\pgfpathlineto{\pgfqpoint{6.686760in}{3.328848in}}%
\pgfpathlineto{\pgfqpoint{6.689446in}{3.302589in}}%
\pgfpathlineto{\pgfqpoint{6.692410in}{3.276329in}}%
\pgfpathlineto{\pgfqpoint{6.695655in}{3.250070in}}%
\pgfpathlineto{\pgfqpoint{6.699181in}{3.223810in}}%
\pgfpathlineto{\pgfqpoint{6.702990in}{3.197551in}}%
\pgfpathlineto{\pgfqpoint{6.707083in}{3.171291in}}%
\pgfpathlineto{\pgfqpoint{6.709987in}{3.153881in}}%
\pgfpathlineto{\pgfqpoint{6.711453in}{3.145032in}}%
\pgfpathlineto{\pgfqpoint{6.716089in}{3.118772in}}%
\pgfpathlineto{\pgfqpoint{6.721011in}{3.092513in}}%
\pgfpathlineto{\pgfqpoint{6.726222in}{3.066253in}}%
\pgfpathlineto{\pgfqpoint{6.731723in}{3.039994in}}%
\pgfpathlineto{\pgfqpoint{6.737516in}{3.013734in}}%
\pgfpathlineto{\pgfqpoint{6.739560in}{3.004909in}}%
\pgfusepath{stroke}%
\end{pgfscope}%
\begin{pgfscope}%
\pgfpathrectangle{\pgfqpoint{0.854460in}{0.571603in}}{\pgfqpoint{5.885100in}{5.225635in}}%
\pgfusepath{clip}%
\pgfsetbuttcap%
\pgfsetroundjoin%
\pgfsetlinewidth{1.505625pt}%
\definecolor{currentstroke}{rgb}{0.449368,0.813768,0.335384}%
\pgfsetstrokecolor{currentstroke}%
\pgfsetdash{}{0pt}%
\pgfpathmoveto{\pgfqpoint{0.854460in}{5.413227in}}%
\pgfpathlineto{\pgfqpoint{0.874148in}{5.429606in}}%
\pgfpathlineto{\pgfqpoint{0.884034in}{5.437731in}}%
\pgfpathlineto{\pgfqpoint{0.890509in}{5.443016in}}%
\pgfusepath{stroke}%
\end{pgfscope}%
\begin{pgfscope}%
\pgfpathrectangle{\pgfqpoint{0.854460in}{0.571603in}}{\pgfqpoint{5.885100in}{5.225635in}}%
\pgfusepath{clip}%
\pgfsetbuttcap%
\pgfsetroundjoin%
\pgfsetlinewidth{1.505625pt}%
\definecolor{currentstroke}{rgb}{0.449368,0.813768,0.335384}%
\pgfsetstrokecolor{currentstroke}%
\pgfsetdash{}{0pt}%
\pgfpathmoveto{\pgfqpoint{1.196465in}{5.672243in}}%
\pgfpathlineto{\pgfqpoint{1.209341in}{5.681077in}}%
\pgfpathlineto{\pgfqpoint{1.225644in}{5.692201in}}%
\pgfpathlineto{\pgfqpoint{1.238914in}{5.701146in}}%
\pgfpathlineto{\pgfqpoint{1.264735in}{5.718460in}}%
\pgfpathlineto{\pgfqpoint{1.268488in}{5.720946in}}%
\pgfpathlineto{\pgfqpoint{1.298061in}{5.740409in}}%
\pgfpathlineto{\pgfqpoint{1.304649in}{5.744720in}}%
\pgfpathlineto{\pgfqpoint{1.327634in}{5.759576in}}%
\pgfpathlineto{\pgfqpoint{1.345363in}{5.770979in}}%
\pgfpathlineto{\pgfqpoint{1.357208in}{5.778506in}}%
\pgfpathlineto{\pgfqpoint{1.386781in}{5.797208in}}%
\pgfpathlineto{\pgfqpoint{1.386830in}{5.797238in}}%
\pgfusepath{stroke}%
\end{pgfscope}%
\begin{pgfscope}%
\pgfpathrectangle{\pgfqpoint{0.854460in}{0.571603in}}{\pgfqpoint{5.885100in}{5.225635in}}%
\pgfusepath{clip}%
\pgfsetbuttcap%
\pgfsetroundjoin%
\pgfsetlinewidth{1.505625pt}%
\definecolor{currentstroke}{rgb}{0.496615,0.826376,0.306377}%
\pgfsetstrokecolor{currentstroke}%
\pgfsetdash{}{0pt}%
\pgfpathmoveto{\pgfqpoint{6.739560in}{3.693595in}}%
\pgfpathlineto{\pgfqpoint{6.738498in}{3.670221in}}%
\pgfpathlineto{\pgfqpoint{6.737572in}{3.643962in}}%
\pgfpathlineto{\pgfqpoint{6.736914in}{3.617702in}}%
\pgfpathlineto{\pgfqpoint{6.736526in}{3.591443in}}%
\pgfpathlineto{\pgfqpoint{6.736409in}{3.565183in}}%
\pgfpathlineto{\pgfqpoint{6.736562in}{3.538924in}}%
\pgfpathlineto{\pgfqpoint{6.736988in}{3.512664in}}%
\pgfpathlineto{\pgfqpoint{6.737688in}{3.486405in}}%
\pgfpathlineto{\pgfqpoint{6.738661in}{3.460145in}}%
\pgfpathlineto{\pgfqpoint{6.739560in}{3.441246in}}%
\pgfusepath{stroke}%
\end{pgfscope}%
\begin{pgfscope}%
\pgfpathrectangle{\pgfqpoint{0.854460in}{0.571603in}}{\pgfqpoint{5.885100in}{5.225635in}}%
\pgfusepath{clip}%
\pgfsetbuttcap%
\pgfsetroundjoin%
\pgfsetlinewidth{1.505625pt}%
\definecolor{currentstroke}{rgb}{0.496615,0.826376,0.306377}%
\pgfsetstrokecolor{currentstroke}%
\pgfsetdash{}{0pt}%
\pgfpathmoveto{\pgfqpoint{0.854460in}{5.460030in}}%
\pgfpathlineto{\pgfqpoint{0.881642in}{5.482125in}}%
\pgfpathlineto{\pgfqpoint{0.884034in}{5.484046in}}%
\pgfpathlineto{\pgfqpoint{0.900856in}{5.497463in}}%
\pgfusepath{stroke}%
\end{pgfscope}%
\begin{pgfscope}%
\pgfpathrectangle{\pgfqpoint{0.854460in}{0.571603in}}{\pgfqpoint{5.885100in}{5.225635in}}%
\pgfusepath{clip}%
\pgfsetbuttcap%
\pgfsetroundjoin%
\pgfsetlinewidth{1.505625pt}%
\definecolor{currentstroke}{rgb}{0.496615,0.826376,0.306377}%
\pgfsetstrokecolor{currentstroke}%
\pgfsetdash{}{0pt}%
\pgfpathmoveto{\pgfqpoint{1.209342in}{5.722987in}}%
\pgfpathlineto{\pgfqpoint{1.238914in}{5.742749in}}%
\pgfpathlineto{\pgfqpoint{1.241882in}{5.744720in}}%
\pgfpathlineto{\pgfqpoint{1.268488in}{5.762169in}}%
\pgfpathlineto{\pgfqpoint{1.281987in}{5.770979in}}%
\pgfpathlineto{\pgfqpoint{1.298061in}{5.781342in}}%
\pgfpathlineto{\pgfqpoint{1.322837in}{5.797238in}}%
\pgfusepath{stroke}%
\end{pgfscope}%
\begin{pgfscope}%
\pgfpathrectangle{\pgfqpoint{0.854460in}{0.571603in}}{\pgfqpoint{5.885100in}{5.225635in}}%
\pgfusepath{clip}%
\pgfsetbuttcap%
\pgfsetroundjoin%
\pgfsetlinewidth{1.505625pt}%
\definecolor{currentstroke}{rgb}{0.555484,0.840254,0.269281}%
\pgfsetstrokecolor{currentstroke}%
\pgfsetdash{}{0pt}%
\pgfpathmoveto{\pgfqpoint{0.854460in}{5.505341in}}%
\pgfpathlineto{\pgfqpoint{0.858247in}{5.508384in}}%
\pgfpathlineto{\pgfqpoint{0.879483in}{5.525244in}}%
\pgfusepath{stroke}%
\end{pgfscope}%
\begin{pgfscope}%
\pgfpathrectangle{\pgfqpoint{0.854460in}{0.571603in}}{\pgfqpoint{5.885100in}{5.225635in}}%
\pgfusepath{clip}%
\pgfsetbuttcap%
\pgfsetroundjoin%
\pgfsetlinewidth{1.505625pt}%
\definecolor{currentstroke}{rgb}{0.555484,0.840254,0.269281}%
\pgfsetstrokecolor{currentstroke}%
\pgfsetdash{}{0pt}%
\pgfpathmoveto{\pgfqpoint{1.188581in}{5.749859in}}%
\pgfpathlineto{\pgfqpoint{1.209341in}{5.763673in}}%
\pgfpathlineto{\pgfqpoint{1.220377in}{5.770979in}}%
\pgfpathlineto{\pgfqpoint{1.238914in}{5.783104in}}%
\pgfpathlineto{\pgfqpoint{1.260631in}{5.797238in}}%
\pgfusepath{stroke}%
\end{pgfscope}%
\begin{pgfscope}%
\pgfpathrectangle{\pgfqpoint{0.854460in}{0.571603in}}{\pgfqpoint{5.885100in}{5.225635in}}%
\pgfusepath{clip}%
\pgfsetbuttcap%
\pgfsetroundjoin%
\pgfsetlinewidth{1.505625pt}%
\definecolor{currentstroke}{rgb}{0.606045,0.850733,0.236712}%
\pgfsetstrokecolor{currentstroke}%
\pgfsetdash{}{0pt}%
\pgfpathmoveto{\pgfqpoint{0.854460in}{5.549232in}}%
\pgfpathlineto{\pgfqpoint{0.869317in}{5.560903in}}%
\pgfpathlineto{\pgfqpoint{0.878551in}{5.568070in}}%
\pgfusepath{stroke}%
\end{pgfscope}%
\begin{pgfscope}%
\pgfpathrectangle{\pgfqpoint{0.854460in}{0.571603in}}{\pgfqpoint{5.885100in}{5.225635in}}%
\pgfusepath{clip}%
\pgfsetbuttcap%
\pgfsetroundjoin%
\pgfsetlinewidth{1.505625pt}%
\definecolor{currentstroke}{rgb}{0.606045,0.850733,0.236712}%
\pgfsetstrokecolor{currentstroke}%
\pgfsetdash{}{0pt}%
\pgfpathmoveto{\pgfqpoint{1.189343in}{5.790135in}}%
\pgfpathlineto{\pgfqpoint{1.200103in}{5.797238in}}%
\pgfusepath{stroke}%
\end{pgfscope}%
\begin{pgfscope}%
\pgfpathrectangle{\pgfqpoint{0.854460in}{0.571603in}}{\pgfqpoint{5.885100in}{5.225635in}}%
\pgfusepath{clip}%
\pgfsetbuttcap%
\pgfsetroundjoin%
\pgfsetlinewidth{1.505625pt}%
\definecolor{currentstroke}{rgb}{0.668054,0.861999,0.196293}%
\pgfsetstrokecolor{currentstroke}%
\pgfsetdash{}{0pt}%
\pgfpathmoveto{\pgfqpoint{0.854460in}{5.591890in}}%
\pgfpathlineto{\pgfqpoint{0.882488in}{5.613422in}}%
\pgfpathlineto{\pgfqpoint{0.884034in}{5.614596in}}%
\pgfpathlineto{\pgfqpoint{0.913607in}{5.636891in}}%
\pgfpathlineto{\pgfqpoint{0.917332in}{5.639682in}}%
\pgfpathlineto{\pgfqpoint{0.943181in}{5.658813in}}%
\pgfpathlineto{\pgfqpoint{0.952869in}{5.665941in}}%
\pgfpathlineto{\pgfqpoint{0.972754in}{5.680397in}}%
\pgfpathlineto{\pgfqpoint{0.989085in}{5.692201in}}%
\pgfpathlineto{\pgfqpoint{1.002327in}{5.701658in}}%
\pgfpathlineto{\pgfqpoint{1.025987in}{5.718460in}}%
\pgfpathlineto{\pgfqpoint{1.031901in}{5.722609in}}%
\pgfpathlineto{\pgfqpoint{1.061474in}{5.743238in}}%
\pgfpathlineto{\pgfqpoint{1.063612in}{5.744720in}}%
\pgfpathlineto{\pgfqpoint{1.091047in}{5.763504in}}%
\pgfpathlineto{\pgfqpoint{1.102023in}{5.770979in}}%
\pgfpathlineto{\pgfqpoint{1.120621in}{5.783494in}}%
\pgfpathlineto{\pgfqpoint{1.141151in}{5.797238in}}%
\pgfusepath{stroke}%
\end{pgfscope}%
\begin{pgfscope}%
\pgfpathrectangle{\pgfqpoint{0.854460in}{0.571603in}}{\pgfqpoint{5.885100in}{5.225635in}}%
\pgfusepath{clip}%
\pgfsetbuttcap%
\pgfsetroundjoin%
\pgfsetlinewidth{1.505625pt}%
\definecolor{currentstroke}{rgb}{0.720391,0.870350,0.162603}%
\pgfsetstrokecolor{currentstroke}%
\pgfsetdash{}{0pt}%
\pgfpathmoveto{\pgfqpoint{0.854460in}{5.633306in}}%
\pgfpathlineto{\pgfqpoint{0.862851in}{5.639682in}}%
\pgfpathlineto{\pgfqpoint{0.884034in}{5.655585in}}%
\pgfpathlineto{\pgfqpoint{0.897911in}{5.665941in}}%
\pgfpathlineto{\pgfqpoint{0.913607in}{5.677515in}}%
\pgfpathlineto{\pgfqpoint{0.933640in}{5.692201in}}%
\pgfpathlineto{\pgfqpoint{0.943181in}{5.699111in}}%
\pgfpathlineto{\pgfqpoint{0.970046in}{5.718460in}}%
\pgfpathlineto{\pgfqpoint{0.972754in}{5.720387in}}%
\pgfpathlineto{\pgfqpoint{1.002327in}{5.741296in}}%
\pgfpathlineto{\pgfqpoint{1.007198in}{5.744720in}}%
\pgfpathlineto{\pgfqpoint{1.031901in}{5.761872in}}%
\pgfpathlineto{\pgfqpoint{1.045086in}{5.770979in}}%
\pgfpathlineto{\pgfqpoint{1.061474in}{5.782163in}}%
\pgfpathlineto{\pgfqpoint{1.083680in}{5.797238in}}%
\pgfusepath{stroke}%
\end{pgfscope}%
\begin{pgfscope}%
\pgfpathrectangle{\pgfqpoint{0.854460in}{0.571603in}}{\pgfqpoint{5.885100in}{5.225635in}}%
\pgfusepath{clip}%
\pgfsetbuttcap%
\pgfsetroundjoin%
\pgfsetlinewidth{1.505625pt}%
\definecolor{currentstroke}{rgb}{0.783315,0.879285,0.125405}%
\pgfsetstrokecolor{currentstroke}%
\pgfsetdash{}{0pt}%
\pgfpathmoveto{\pgfqpoint{0.854460in}{5.673587in}}%
\pgfpathlineto{\pgfqpoint{0.879497in}{5.692201in}}%
\pgfpathlineto{\pgfqpoint{0.884034in}{5.695533in}}%
\pgfpathlineto{\pgfqpoint{0.913607in}{5.717125in}}%
\pgfpathlineto{\pgfqpoint{0.915448in}{5.718460in}}%
\pgfpathlineto{\pgfqpoint{0.943181in}{5.738334in}}%
\pgfpathlineto{\pgfqpoint{0.952140in}{5.744720in}}%
\pgfpathlineto{\pgfqpoint{0.972754in}{5.759235in}}%
\pgfpathlineto{\pgfqpoint{0.989523in}{5.770979in}}%
\pgfpathlineto{\pgfqpoint{1.002327in}{5.779840in}}%
\pgfpathlineto{\pgfqpoint{1.027602in}{5.797238in}}%
\pgfusepath{stroke}%
\end{pgfscope}%
\begin{pgfscope}%
\pgfpathrectangle{\pgfqpoint{0.854460in}{0.571603in}}{\pgfqpoint{5.885100in}{5.225635in}}%
\pgfusepath{clip}%
\pgfsetbuttcap%
\pgfsetroundjoin%
\pgfsetlinewidth{1.505625pt}%
\definecolor{currentstroke}{rgb}{0.835270,0.886029,0.102646}%
\pgfsetstrokecolor{currentstroke}%
\pgfsetdash{}{0pt}%
\pgfpathmoveto{\pgfqpoint{0.854460in}{5.712811in}}%
\pgfpathlineto{\pgfqpoint{0.862142in}{5.718460in}}%
\pgfpathlineto{\pgfqpoint{0.884034in}{5.734370in}}%
\pgfpathlineto{\pgfqpoint{0.898357in}{5.744720in}}%
\pgfpathlineto{\pgfqpoint{0.913607in}{5.755609in}}%
\pgfpathlineto{\pgfqpoint{0.935252in}{5.770979in}}%
\pgfpathlineto{\pgfqpoint{0.943181in}{5.776542in}}%
\pgfpathlineto{\pgfqpoint{0.972754in}{5.797182in}}%
\pgfpathlineto{\pgfqpoint{0.972835in}{5.797238in}}%
\pgfusepath{stroke}%
\end{pgfscope}%
\begin{pgfscope}%
\pgfpathrectangle{\pgfqpoint{0.854460in}{0.571603in}}{\pgfqpoint{5.885100in}{5.225635in}}%
\pgfusepath{clip}%
\pgfsetbuttcap%
\pgfsetroundjoin%
\pgfsetlinewidth{1.505625pt}%
\definecolor{currentstroke}{rgb}{0.896320,0.893616,0.096335}%
\pgfsetstrokecolor{currentstroke}%
\pgfsetdash{}{0pt}%
\pgfpathmoveto{\pgfqpoint{0.854460in}{5.751010in}}%
\pgfpathlineto{\pgfqpoint{0.882197in}{5.770979in}}%
\pgfpathlineto{\pgfqpoint{0.884034in}{5.772286in}}%
\pgfpathlineto{\pgfqpoint{0.913607in}{5.793188in}}%
\pgfpathlineto{\pgfqpoint{0.919371in}{5.797238in}}%
\pgfusepath{stroke}%
\end{pgfscope}%
\begin{pgfscope}%
\pgfpathrectangle{\pgfqpoint{0.854460in}{0.571603in}}{\pgfqpoint{5.885100in}{5.225635in}}%
\pgfusepath{clip}%
\pgfsetbuttcap%
\pgfsetroundjoin%
\pgfsetlinewidth{1.505625pt}%
\definecolor{currentstroke}{rgb}{0.945636,0.899815,0.112838}%
\pgfsetstrokecolor{currentstroke}%
\pgfsetdash{}{0pt}%
\pgfpathmoveto{\pgfqpoint{0.854460in}{5.788247in}}%
\pgfpathlineto{\pgfqpoint{0.867082in}{5.797238in}}%
\pgfusepath{stroke}%
\end{pgfscope}%
\begin{pgfscope}%
\pgfsetrectcap%
\pgfsetmiterjoin%
\pgfsetlinewidth{0.803000pt}%
\definecolor{currentstroke}{rgb}{0.000000,0.000000,0.000000}%
\pgfsetstrokecolor{currentstroke}%
\pgfsetdash{}{0pt}%
\pgfpathmoveto{\pgfqpoint{0.854460in}{0.571603in}}%
\pgfpathlineto{\pgfqpoint{0.854460in}{5.797238in}}%
\pgfusepath{stroke}%
\end{pgfscope}%
\begin{pgfscope}%
\pgfsetrectcap%
\pgfsetmiterjoin%
\pgfsetlinewidth{0.803000pt}%
\definecolor{currentstroke}{rgb}{0.000000,0.000000,0.000000}%
\pgfsetstrokecolor{currentstroke}%
\pgfsetdash{}{0pt}%
\pgfpathmoveto{\pgfqpoint{6.739560in}{0.571603in}}%
\pgfpathlineto{\pgfqpoint{6.739560in}{5.797238in}}%
\pgfusepath{stroke}%
\end{pgfscope}%
\begin{pgfscope}%
\pgfsetrectcap%
\pgfsetmiterjoin%
\pgfsetlinewidth{0.803000pt}%
\definecolor{currentstroke}{rgb}{0.000000,0.000000,0.000000}%
\pgfsetstrokecolor{currentstroke}%
\pgfsetdash{}{0pt}%
\pgfpathmoveto{\pgfqpoint{0.854460in}{0.571603in}}%
\pgfpathlineto{\pgfqpoint{6.739560in}{0.571603in}}%
\pgfusepath{stroke}%
\end{pgfscope}%
\begin{pgfscope}%
\pgfsetrectcap%
\pgfsetmiterjoin%
\pgfsetlinewidth{0.803000pt}%
\definecolor{currentstroke}{rgb}{0.000000,0.000000,0.000000}%
\pgfsetstrokecolor{currentstroke}%
\pgfsetdash{}{0pt}%
\pgfpathmoveto{\pgfqpoint{0.854460in}{5.797238in}}%
\pgfpathlineto{\pgfqpoint{6.739560in}{5.797238in}}%
\pgfusepath{stroke}%
\end{pgfscope}%
\begin{pgfscope}%
\definecolor{textcolor}{rgb}{0.273809,0.031497,0.358853}%
\pgfsetstrokecolor{textcolor}%
\pgfsetfillcolor{textcolor}%
\pgftext[x=3.835467in, y=1.453917in, left, base,rotate=313.690413]{\color{textcolor}\sffamily\fontsize{8.000000}{9.600000}\selectfont 1.65}%
\end{pgfscope}%
\begin{pgfscope}%
\definecolor{textcolor}{rgb}{0.278791,0.062145,0.386592}%
\pgfsetstrokecolor{textcolor}%
\pgfsetfillcolor{textcolor}%
\pgftext[x=4.238110in, y=1.389713in, left, base,rotate=312.368660]{\color{textcolor}\sffamily\fontsize{8.000000}{9.600000}\selectfont 1.80}%
\end{pgfscope}%
\begin{pgfscope}%
\definecolor{textcolor}{rgb}{0.282327,0.094955,0.417331}%
\pgfsetstrokecolor{textcolor}%
\pgfsetfillcolor{textcolor}%
\pgftext[x=4.743645in, y=1.087834in, left, base,rotate=312.861190]{\color{textcolor}\sffamily\fontsize{8.000000}{9.600000}\selectfont 1.95}%
\end{pgfscope}%
\begin{pgfscope}%
\definecolor{textcolor}{rgb}{0.283229,0.120777,0.440584}%
\pgfsetstrokecolor{textcolor}%
\pgfsetfillcolor{textcolor}%
\pgftext[x=3.838407in, y=2.464382in, left, base,rotate=306.690648]{\color{textcolor}\sffamily\fontsize{8.000000}{9.600000}\selectfont 2.10}%
\end{pgfscope}%
\begin{pgfscope}%
\definecolor{textcolor}{rgb}{0.281887,0.150881,0.465405}%
\pgfsetstrokecolor{textcolor}%
\pgfsetfillcolor{textcolor}%
\pgftext[x=5.341234in, y=0.847176in, left, base,rotate=315.154933]{\color{textcolor}\sffamily\fontsize{8.000000}{9.600000}\selectfont 2.25}%
\end{pgfscope}%
\begin{pgfscope}%
\definecolor{textcolor}{rgb}{0.278826,0.175490,0.483397}%
\pgfsetstrokecolor{textcolor}%
\pgfsetfillcolor{textcolor}%
\pgftext[x=3.756018in, y=3.308780in, left, base,rotate=298.129815]{\color{textcolor}\sffamily\fontsize{8.000000}{9.600000}\selectfont 2.40}%
\end{pgfscope}%
\begin{pgfscope}%
\definecolor{textcolor}{rgb}{0.273006,0.204520,0.501721}%
\pgfsetstrokecolor{textcolor}%
\pgfsetfillcolor{textcolor}%
\pgftext[x=5.154117in, y=1.372439in, left, base,rotate=312.995477]{\color{textcolor}\sffamily\fontsize{8.000000}{9.600000}\selectfont 2.55}%
\end{pgfscope}%
\begin{pgfscope}%
\definecolor{textcolor}{rgb}{0.266580,0.228262,0.514349}%
\pgfsetstrokecolor{textcolor}%
\pgfsetfillcolor{textcolor}%
\pgftext[x=4.065769in, y=3.459660in, left, base,rotate=290.956508]{\color{textcolor}\sffamily\fontsize{8.000000}{9.600000}\selectfont 2.70}%
\end{pgfscope}%
\begin{pgfscope}%
\definecolor{textcolor}{rgb}{0.257322,0.256130,0.526563}%
\pgfsetstrokecolor{textcolor}%
\pgfsetfillcolor{textcolor}%
\pgftext[x=5.724790in, y=1.029333in, left, base,rotate=316.020956]{\color{textcolor}\sffamily\fontsize{8.000000}{9.600000}\selectfont 2.85}%
\end{pgfscope}%
\begin{pgfscope}%
\definecolor{textcolor}{rgb}{0.248629,0.278775,0.534556}%
\pgfsetstrokecolor{textcolor}%
\pgfsetfillcolor{textcolor}%
\pgftext[x=4.255841in, y=3.926004in, left, base,rotate=283.510037]{\color{textcolor}\sffamily\fontsize{8.000000}{9.600000}\selectfont 3.00}%
\end{pgfscope}%
\begin{pgfscope}%
\definecolor{textcolor}{rgb}{0.239346,0.300855,0.540844}%
\pgfsetstrokecolor{textcolor}%
\pgfsetfillcolor{textcolor}%
\pgftext[x=6.061820in, y=0.921798in, left, base,rotate=317.309850]{\color{textcolor}\sffamily\fontsize{8.000000}{9.600000}\selectfont 3.15}%
\end{pgfscope}%
\begin{pgfscope}%
\definecolor{textcolor}{rgb}{0.227802,0.326594,0.546532}%
\pgfsetstrokecolor{textcolor}%
\pgfsetfillcolor{textcolor}%
\pgftext[x=5.599219in, y=1.537026in, left, base,rotate=311.778721]{\color{textcolor}\sffamily\fontsize{8.000000}{9.600000}\selectfont 3.30}%
\end{pgfscope}%
\begin{pgfscope}%
\definecolor{textcolor}{rgb}{0.218130,0.347432,0.550038}%
\pgfsetstrokecolor{textcolor}%
\pgfsetfillcolor{textcolor}%
\pgftext[x=4.653047in, y=4.131735in, left, base,rotate=275.567428]{\color{textcolor}\sffamily\fontsize{8.000000}{9.600000}\selectfont 3.45}%
\end{pgfscope}%
\begin{pgfscope}%
\definecolor{textcolor}{rgb}{0.206756,0.371758,0.553117}%
\pgfsetstrokecolor{textcolor}%
\pgfsetfillcolor{textcolor}%
\pgftext[x=6.009996in, y=1.292993in, left, base,rotate=314.465189]{\color{textcolor}\sffamily\fontsize{8.000000}{9.600000}\selectfont 3.60}%
\end{pgfscope}%
\begin{pgfscope}%
\definecolor{textcolor}{rgb}{0.197636,0.391528,0.554969}%
\pgfsetstrokecolor{textcolor}%
\pgfsetfillcolor{textcolor}%
\pgftext[x=1.560240in, y=0.790944in, left, base,rotate=317.082108]{\color{textcolor}\sffamily\fontsize{8.000000}{9.600000}\selectfont 3.75}%
\end{pgfscope}%
\begin{pgfscope}%
\definecolor{textcolor}{rgb}{0.197636,0.391528,0.554969}%
\pgfsetstrokecolor{textcolor}%
\pgfsetfillcolor{textcolor}%
\pgftext[x=4.897248in, y=4.424045in, left, base,rotate=270.187789]{\color{textcolor}\sffamily\fontsize{8.000000}{9.600000}\selectfont 3.75}%
\end{pgfscope}%
\begin{pgfscope}%
\definecolor{textcolor}{rgb}{0.187231,0.414746,0.556547}%
\pgfsetstrokecolor{textcolor}%
\pgfsetfillcolor{textcolor}%
\pgftext[x=1.167604in, y=1.174159in, left, base,rotate=308.419345]{\color{textcolor}\sffamily\fontsize{8.000000}{9.600000}\selectfont 3.90}%
\end{pgfscope}%
\begin{pgfscope}%
\definecolor{textcolor}{rgb}{0.187231,0.414746,0.556547}%
\pgfsetstrokecolor{textcolor}%
\pgfsetfillcolor{textcolor}%
\pgftext[x=5.636613in, y=2.001820in, left, base,rotate=305.962539]{\color{textcolor}\sffamily\fontsize{8.000000}{9.600000}\selectfont 3.90}%
\end{pgfscope}%
\begin{pgfscope}%
\definecolor{textcolor}{rgb}{0.179019,0.433756,0.557430}%
\pgfsetstrokecolor{textcolor}%
\pgfsetfillcolor{textcolor}%
\pgftext[x=0.859998in, y=1.584449in, left, base,rotate=298.711312]{\color{textcolor}\sffamily\fontsize{8.000000}{9.600000}\selectfont 4.05}%
\end{pgfscope}%
\begin{pgfscope}%
\definecolor{textcolor}{rgb}{0.179019,0.433756,0.557430}%
\pgfsetstrokecolor{textcolor}%
\pgfsetfillcolor{textcolor}%
\pgftext[x=2.672938in, y=5.413545in, left, base,rotate=21.550188]{\color{textcolor}\sffamily\fontsize{8.000000}{9.600000}\selectfont 4.05}%
\end{pgfscope}%
\begin{pgfscope}%
\definecolor{textcolor}{rgb}{0.179019,0.433756,0.557430}%
\pgfsetstrokecolor{textcolor}%
\pgfsetfillcolor{textcolor}%
\pgftext[x=5.205035in, y=4.568747in, left, base,rotate=85.845100]{\color{textcolor}\sffamily\fontsize{8.000000}{9.600000}\selectfont 4.05}%
\end{pgfscope}%
\begin{pgfscope}%
\definecolor{textcolor}{rgb}{0.169646,0.456262,0.558030}%
\pgfsetstrokecolor{textcolor}%
\pgfsetfillcolor{textcolor}%
\pgftext[x=1.335526in, y=0.794816in, left, base,rotate=316.114333]{\color{textcolor}\sffamily\fontsize{8.000000}{9.600000}\selectfont 4.20}%
\end{pgfscope}%
\begin{pgfscope}%
\definecolor{textcolor}{rgb}{0.169646,0.456262,0.558030}%
\pgfsetstrokecolor{textcolor}%
\pgfsetfillcolor{textcolor}%
\pgftext[x=2.294789in, y=5.301400in, left, base,rotate=26.177995]{\color{textcolor}\sffamily\fontsize{8.000000}{9.600000}\selectfont 4.20}%
\end{pgfscope}%
\begin{pgfscope}%
\definecolor{textcolor}{rgb}{0.169646,0.456262,0.558030}%
\pgfsetstrokecolor{textcolor}%
\pgfsetfillcolor{textcolor}%
\pgftext[x=6.369078in, y=1.294305in, left, base,rotate=314.777759]{\color{textcolor}\sffamily\fontsize{8.000000}{9.600000}\selectfont 4.20}%
\end{pgfscope}%
\begin{pgfscope}%
\definecolor{textcolor}{rgb}{0.162142,0.474838,0.558140}%
\pgfsetstrokecolor{textcolor}%
\pgfsetfillcolor{textcolor}%
\pgftext[x=1.268435in, y=0.793294in, left, base,rotate=315.866959]{\color{textcolor}\sffamily\fontsize{8.000000}{9.600000}\selectfont 4.35}%
\end{pgfscope}%
\begin{pgfscope}%
\definecolor{textcolor}{rgb}{0.162142,0.474838,0.558140}%
\pgfsetstrokecolor{textcolor}%
\pgfsetfillcolor{textcolor}%
\pgftext[x=1.917844in, y=5.146308in, left, base,rotate=31.167936]{\color{textcolor}\sffamily\fontsize{8.000000}{9.600000}\selectfont 4.35}%
\end{pgfscope}%
\begin{pgfscope}%
\definecolor{textcolor}{rgb}{0.162142,0.474838,0.558140}%
\pgfsetstrokecolor{textcolor}%
\pgfsetfillcolor{textcolor}%
\pgftext[x=6.162264in, y=1.632708in, left, base,rotate=310.870830]{\color{textcolor}\sffamily\fontsize{8.000000}{9.600000}\selectfont 4.35}%
\end{pgfscope}%
\begin{pgfscope}%
\definecolor{textcolor}{rgb}{0.154815,0.493313,0.557840}%
\pgfsetstrokecolor{textcolor}%
\pgfsetfillcolor{textcolor}%
\pgftext[x=1.188584in, y=0.808143in, left, base,rotate=315.291812]{\color{textcolor}\sffamily\fontsize{8.000000}{9.600000}\selectfont 4.50}%
\end{pgfscope}%
\begin{pgfscope}%
\definecolor{textcolor}{rgb}{0.154815,0.493313,0.557840}%
\pgfsetstrokecolor{textcolor}%
\pgfsetfillcolor{textcolor}%
\pgftext[x=1.570753in, y=4.963806in, left, base,rotate=36.463096]{\color{textcolor}\sffamily\fontsize{8.000000}{9.600000}\selectfont 4.50}%
\end{pgfscope}%
\begin{pgfscope}%
\definecolor{textcolor}{rgb}{0.154815,0.493313,0.557840}%
\pgfsetstrokecolor{textcolor}%
\pgfsetfillcolor{textcolor}%
\pgftext[x=5.567048in, y=4.803573in, left, base,rotate=80.244277]{\color{textcolor}\sffamily\fontsize{8.000000}{9.600000}\selectfont 4.50}%
\end{pgfscope}%
\begin{pgfscope}%
\definecolor{textcolor}{rgb}{0.146180,0.515413,0.556823}%
\pgfsetstrokecolor{textcolor}%
\pgfsetfillcolor{textcolor}%
\pgftext[x=0.836395in, y=1.176842in, left, base,rotate=307.169580]{\color{textcolor}\sffamily\fontsize{8.000000}{9.600000}\selectfont 4.65}%
\end{pgfscope}%
\begin{pgfscope}%
\definecolor{textcolor}{rgb}{0.146180,0.515413,0.556823}%
\pgfsetstrokecolor{textcolor}%
\pgfsetfillcolor{textcolor}%
\pgftext[x=1.245619in, y=4.745825in, left, base,rotate=42.421140]{\color{textcolor}\sffamily\fontsize{8.000000}{9.600000}\selectfont 4.65}%
\end{pgfscope}%
\begin{pgfscope}%
\definecolor{textcolor}{rgb}{0.146180,0.515413,0.556823}%
\pgfsetstrokecolor{textcolor}%
\pgfsetfillcolor{textcolor}%
\pgftext[x=6.079067in, y=1.966092in, left, base,rotate=306.317042]{\color{textcolor}\sffamily\fontsize{8.000000}{9.600000}\selectfont 4.65}%
\end{pgfscope}%
\begin{pgfscope}%
\definecolor{textcolor}{rgb}{0.139147,0.533812,0.555298}%
\pgfsetstrokecolor{textcolor}%
\pgfsetfillcolor{textcolor}%
\pgftext[x=1.070645in, y=0.798591in, left, base,rotate=314.982643]{\color{textcolor}\sffamily\fontsize{8.000000}{9.600000}\selectfont 4.80}%
\end{pgfscope}%
\begin{pgfscope}%
\definecolor{textcolor}{rgb}{0.139147,0.533812,0.555298}%
\pgfsetstrokecolor{textcolor}%
\pgfsetfillcolor{textcolor}%
\pgftext[x=0.944780in, y=4.486673in, left, base,rotate=49.237941]{\color{textcolor}\sffamily\fontsize{8.000000}{9.600000}\selectfont 4.80}%
\end{pgfscope}%
\begin{pgfscope}%
\definecolor{textcolor}{rgb}{0.139147,0.533812,0.555298}%
\pgfsetstrokecolor{textcolor}%
\pgfsetfillcolor{textcolor}%
\pgftext[x=5.679344in, y=4.330822in, left, base,rotate=79.716205]{\color{textcolor}\sffamily\fontsize{8.000000}{9.600000}\selectfont 4.80}%
\end{pgfscope}%
\begin{pgfscope}%
\definecolor{textcolor}{rgb}{0.131172,0.555899,0.552459}%
\pgfsetstrokecolor{textcolor}%
\pgfsetfillcolor{textcolor}%
\pgftext[x=1.011743in, y=0.796703in, left, base,rotate=314.770236]{\color{textcolor}\sffamily\fontsize{8.000000}{9.600000}\selectfont 4.95}%
\end{pgfscope}%
\begin{pgfscope}%
\definecolor{textcolor}{rgb}{0.131172,0.555899,0.552459}%
\pgfsetstrokecolor{textcolor}%
\pgfsetfillcolor{textcolor}%
\pgftext[x=2.497697in, y=5.717366in, left, base,rotate=23.220107]{\color{textcolor}\sffamily\fontsize{8.000000}{9.600000}\selectfont 4.95}%
\end{pgfscope}%
\begin{pgfscope}%
\definecolor{textcolor}{rgb}{0.131172,0.555899,0.552459}%
\pgfsetstrokecolor{textcolor}%
\pgfsetfillcolor{textcolor}%
\pgftext[x=5.879960in, y=4.852142in, left, base,rotate=75.195251]{\color{textcolor}\sffamily\fontsize{8.000000}{9.600000}\selectfont 4.95}%
\end{pgfscope}%
\begin{pgfscope}%
\definecolor{textcolor}{rgb}{0.125394,0.574318,0.549086}%
\pgfsetstrokecolor{textcolor}%
\pgfsetfillcolor{textcolor}%
\pgftext[x=0.952876in, y=0.796624in, left, base,rotate=314.528502]{\color{textcolor}\sffamily\fontsize{8.000000}{9.600000}\selectfont 5.10}%
\end{pgfscope}%
\begin{pgfscope}%
\definecolor{textcolor}{rgb}{0.125394,0.574318,0.549086}%
\pgfsetstrokecolor{textcolor}%
\pgfsetfillcolor{textcolor}%
\pgftext[x=2.066548in, y=5.556422in, left, base,rotate=26.875090]{\color{textcolor}\sffamily\fontsize{8.000000}{9.600000}\selectfont 5.10}%
\end{pgfscope}%
\begin{pgfscope}%
\definecolor{textcolor}{rgb}{0.125394,0.574318,0.549086}%
\pgfsetstrokecolor{textcolor}%
\pgfsetfillcolor{textcolor}%
\pgftext[x=6.032868in, y=2.447549in, left, base,rotate=298.311033]{\color{textcolor}\sffamily\fontsize{8.000000}{9.600000}\selectfont 5.10}%
\end{pgfscope}%
\begin{pgfscope}%
\definecolor{textcolor}{rgb}{0.120565,0.596422,0.543611}%
\pgfsetstrokecolor{textcolor}%
\pgfsetfillcolor{textcolor}%
\pgftext[x=1.661852in, y=5.364984in, left, base,rotate=31.572470]{\color{textcolor}\sffamily\fontsize{8.000000}{9.600000}\selectfont 5.25}%
\end{pgfscope}%
\begin{pgfscope}%
\definecolor{textcolor}{rgb}{0.120565,0.596422,0.543611}%
\pgfsetstrokecolor{textcolor}%
\pgfsetfillcolor{textcolor}%
\pgftext[x=5.914806in, y=2.964062in, left, base,rotate=288.243380]{\color{textcolor}\sffamily\fontsize{8.000000}{9.600000}\selectfont 5.25}%
\end{pgfscope}%
\begin{pgfscope}%
\definecolor{textcolor}{rgb}{0.119483,0.614817,0.537692}%
\pgfsetstrokecolor{textcolor}%
\pgfsetfillcolor{textcolor}%
\pgftext[x=1.290793in, y=5.147056in, left, base,rotate=36.864246]{\color{textcolor}\sffamily\fontsize{8.000000}{9.600000}\selectfont 5.40}%
\end{pgfscope}%
\begin{pgfscope}%
\definecolor{textcolor}{rgb}{0.119483,0.614817,0.537692}%
\pgfsetstrokecolor{textcolor}%
\pgfsetfillcolor{textcolor}%
\pgftext[x=6.035320in, y=4.409348in, left, base,rotate=75.643307]{\color{textcolor}\sffamily\fontsize{8.000000}{9.600000}\selectfont 5.40}%
\end{pgfscope}%
\begin{pgfscope}%
\definecolor{textcolor}{rgb}{0.123444,0.636809,0.528763}%
\pgfsetstrokecolor{textcolor}%
\pgfsetfillcolor{textcolor}%
\pgftext[x=0.956149in, y=4.902916in, left, base,rotate=42.801183]{\color{textcolor}\sffamily\fontsize{8.000000}{9.600000}\selectfont 5.55}%
\end{pgfscope}%
\begin{pgfscope}%
\definecolor{textcolor}{rgb}{0.123444,0.636809,0.528763}%
\pgfsetstrokecolor{textcolor}%
\pgfsetfillcolor{textcolor}%
\pgftext[x=6.267239in, y=4.935217in, left, base,rotate=70.147409]{\color{textcolor}\sffamily\fontsize{8.000000}{9.600000}\selectfont 5.55}%
\end{pgfscope}%
\begin{pgfscope}%
\definecolor{textcolor}{rgb}{0.132268,0.655014,0.519661}%
\pgfsetstrokecolor{textcolor}%
\pgfsetfillcolor{textcolor}%
\pgftext[x=1.698091in, y=5.551786in, left, base,rotate=29.688537]{\color{textcolor}\sffamily\fontsize{8.000000}{9.600000}\selectfont 5.70}%
\end{pgfscope}%
\begin{pgfscope}%
\definecolor{textcolor}{rgb}{0.132268,0.655014,0.519661}%
\pgfsetstrokecolor{textcolor}%
\pgfsetfillcolor{textcolor}%
\pgftext[x=6.539746in, y=5.434279in, left, base,rotate=69.596688]{\color{textcolor}\sffamily\fontsize{8.000000}{9.600000}\selectfont 5.70}%
\end{pgfscope}%
\begin{pgfscope}%
\definecolor{textcolor}{rgb}{0.146616,0.673050,0.508936}%
\pgfsetstrokecolor{textcolor}%
\pgfsetfillcolor{textcolor}%
\pgftext[x=1.306055in, y=5.334061in, left, base,rotate=34.648669]{\color{textcolor}\sffamily\fontsize{8.000000}{9.600000}\selectfont 5.85}%
\end{pgfscope}%
\begin{pgfscope}%
\definecolor{textcolor}{rgb}{0.146616,0.673050,0.508936}%
\pgfsetstrokecolor{textcolor}%
\pgfsetfillcolor{textcolor}%
\pgftext[x=6.189433in, y=3.953276in, left, base,rotate=81.379693]{\color{textcolor}\sffamily\fontsize{8.000000}{9.600000}\selectfont 5.85}%
\end{pgfscope}%
\begin{pgfscope}%
\definecolor{textcolor}{rgb}{0.170948,0.694384,0.493803}%
\pgfsetstrokecolor{textcolor}%
\pgfsetfillcolor{textcolor}%
\pgftext[x=1.305882in, y=5.387419in, left, base,rotate=34.082045]{\color{textcolor}\sffamily\fontsize{8.000000}{9.600000}\selectfont 6.00}%
\end{pgfscope}%
\begin{pgfscope}%
\definecolor{textcolor}{rgb}{0.170948,0.694384,0.493803}%
\pgfsetstrokecolor{textcolor}%
\pgfsetfillcolor{textcolor}%
\pgftext[x=6.503512in, y=2.343526in, left, base,rotate=299.866712]{\color{textcolor}\sffamily\fontsize{8.000000}{9.600000}\selectfont 6.00}%
\end{pgfscope}%
\begin{pgfscope}%
\definecolor{textcolor}{rgb}{0.196571,0.711827,0.479221}%
\pgfsetstrokecolor{textcolor}%
\pgfsetfillcolor{textcolor}%
\pgftext[x=1.329632in, y=5.456078in, left, base,rotate=33.199178]{\color{textcolor}\sffamily\fontsize{8.000000}{9.600000}\selectfont 6.15}%
\end{pgfscope}%
\begin{pgfscope}%
\definecolor{textcolor}{rgb}{0.196571,0.711827,0.479221}%
\pgfsetstrokecolor{textcolor}%
\pgfsetfillcolor{textcolor}%
\pgftext[x=6.615890in, y=4.988875in, left, base,rotate=66.381696]{\color{textcolor}\sffamily\fontsize{8.000000}{9.600000}\selectfont 6.15}%
\end{pgfscope}%
\begin{pgfscope}%
\definecolor{textcolor}{rgb}{0.232815,0.732247,0.459277}%
\pgfsetstrokecolor{textcolor}%
\pgfsetfillcolor{textcolor}%
\pgftext[x=1.328853in, y=5.504819in, left, base,rotate=32.719600]{\color{textcolor}\sffamily\fontsize{8.000000}{9.600000}\selectfont 6.30}%
\end{pgfscope}%
\begin{pgfscope}%
\definecolor{textcolor}{rgb}{0.232815,0.732247,0.459277}%
\pgfsetstrokecolor{textcolor}%
\pgfsetfillcolor{textcolor}%
\pgftext[x=6.468250in, y=4.319901in, left, base,rotate=73.766077]{\color{textcolor}\sffamily\fontsize{8.000000}{9.600000}\selectfont 6.30}%
\end{pgfscope}%
\begin{pgfscope}%
\definecolor{textcolor}{rgb}{0.266941,0.748751,0.440573}%
\pgfsetstrokecolor{textcolor}%
\pgfsetfillcolor{textcolor}%
\pgftext[x=6.634421in, y=4.641045in, left, base,rotate=68.679374]{\color{textcolor}\sffamily\fontsize{8.000000}{9.600000}\selectfont 6.45}%
\end{pgfscope}%
\begin{pgfscope}%
\definecolor{textcolor}{rgb}{0.266941,0.748751,0.440573}%
\pgfsetstrokecolor{textcolor}%
\pgfsetfillcolor{textcolor}%
\pgftext[x=0.953453in, y=5.262157in, left, base,rotate=38.088586]{\color{textcolor}\sffamily\fontsize{8.000000}{9.600000}\selectfont 6.45}%
\end{pgfscope}%
\begin{pgfscope}%
\definecolor{textcolor}{rgb}{0.311925,0.767822,0.415586}%
\pgfsetstrokecolor{textcolor}%
\pgfsetfillcolor{textcolor}%
\pgftext[x=6.533473in, y=3.936975in, left, base,rotate=80.021325]{\color{textcolor}\sffamily\fontsize{8.000000}{9.600000}\selectfont 6.60}%
\end{pgfscope}%
\begin{pgfscope}%
\definecolor{textcolor}{rgb}{0.311925,0.767822,0.415586}%
\pgfsetstrokecolor{textcolor}%
\pgfsetfillcolor{textcolor}%
\pgftext[x=0.952380in, y=5.313311in, left, base,rotate=37.473849]{\color{textcolor}\sffamily\fontsize{8.000000}{9.600000}\selectfont 6.60}%
\end{pgfscope}%
\begin{pgfscope}%
\definecolor{textcolor}{rgb}{0.352360,0.783011,0.392636}%
\pgfsetstrokecolor{textcolor}%
\pgfsetfillcolor{textcolor}%
\pgftext[x=6.524283in, y=3.368356in, left, base,rotate=277.928949]{\color{textcolor}\sffamily\fontsize{8.000000}{9.600000}\selectfont 6.75}%
\end{pgfscope}%
\begin{pgfscope}%
\definecolor{textcolor}{rgb}{0.352360,0.783011,0.392636}%
\pgfsetstrokecolor{textcolor}%
\pgfsetfillcolor{textcolor}%
\pgftext[x=0.951374in, y=5.362706in, left, base,rotate=36.893021]{\color{textcolor}\sffamily\fontsize{8.000000}{9.600000}\selectfont 6.75}%
\end{pgfscope}%
\begin{pgfscope}%
\definecolor{textcolor}{rgb}{0.404001,0.800275,0.362552}%
\pgfsetstrokecolor{textcolor}%
\pgfsetfillcolor{textcolor}%
\pgftext[x=6.642711in, y=3.467638in, left, base,rotate=89.798304]{\color{textcolor}\sffamily\fontsize{8.000000}{9.600000}\selectfont 6.90}%
\end{pgfscope}%
\begin{pgfscope}%
\definecolor{textcolor}{rgb}{0.404001,0.800275,0.362552}%
\pgfsetstrokecolor{textcolor}%
\pgfsetfillcolor{textcolor}%
\pgftext[x=0.950442in, y=5.410471in, left, base,rotate=36.349940]{\color{textcolor}\sffamily\fontsize{8.000000}{9.600000}\selectfont 6.90}%
\end{pgfscope}%
\begin{pgfscope}%
\definecolor{textcolor}{rgb}{0.449368,0.813768,0.335384}%
\pgfsetstrokecolor{textcolor}%
\pgfsetfillcolor{textcolor}%
\pgftext[x=6.708421in, y=3.753813in, left, base,rotate=83.103022]{\color{textcolor}\sffamily\fontsize{8.000000}{9.600000}\selectfont 7.05}%
\end{pgfscope}%
\begin{pgfscope}%
\definecolor{textcolor}{rgb}{0.449368,0.813768,0.335384}%
\pgfsetstrokecolor{textcolor}%
\pgfsetfillcolor{textcolor}%
\pgftext[x=0.958646in, y=5.463843in, left, base,rotate=35.689435]{\color{textcolor}\sffamily\fontsize{8.000000}{9.600000}\selectfont 7.05}%
\end{pgfscope}%
\begin{pgfscope}%
\definecolor{textcolor}{rgb}{0.496615,0.826376,0.306377}%
\pgfsetstrokecolor{textcolor}%
\pgfsetfillcolor{textcolor}%
\pgftext[x=0.969230in, y=5.517338in, left, base,rotate=35.016807]{\color{textcolor}\sffamily\fontsize{8.000000}{9.600000}\selectfont 7.20}%
\end{pgfscope}%
\begin{pgfscope}%
\definecolor{textcolor}{rgb}{0.555484,0.840254,0.269281}%
\pgfsetstrokecolor{textcolor}%
\pgfsetfillcolor{textcolor}%
\pgftext[x=0.947920in, y=5.544883in, left, base,rotate=34.859578]{\color{textcolor}\sffamily\fontsize{8.000000}{9.600000}\selectfont 7.35}%
\end{pgfscope}%
\begin{pgfscope}%
\definecolor{textcolor}{rgb}{0.606045,0.850733,0.236712}%
\pgfsetstrokecolor{textcolor}%
\pgfsetfillcolor{textcolor}%
\pgftext[x=0.947167in, y=5.587025in, left, base,rotate=34.407845]{\color{textcolor}\sffamily\fontsize{8.000000}{9.600000}\selectfont 7.50}%
\end{pgfscope}%
\end{pgfpicture}%
\makeatother%
\endgroup%
}
    \caption{Vrstevnicový graf funkcie $f(x,y)$.}
    \label{fig:graph_contour}
\end{figure}

\vspace*{\fill}

% --- 11. STRANA: 3D GRAF (Optimalizovaný pre rýchlosť) ---
\newpage
\thispagestyle{plain}
\vspace*{\fill}
\begin{figure}[H]
    \centering
    \resizebox{1\textwidth}{!}{
    %% Creator: Matplotlib, PGF backend
%%
%% To include the figure in your LaTeX document, write
%%   \input{<filename>.pgf}
%%
%% Make sure the required packages are loaded in your preamble
%%   \usepackage{pgf}
%%
%% Also ensure that all the required font packages are loaded; for instance,
%% the lmodern package is sometimes necessary when using math font.
%%   \usepackage{lmodern}
%%
%% Figures using additional raster images can only be included by \input if
%% they are in the same directory as the main LaTeX file. For loading figures
%% from other directories you can use the `import` package
%%   \usepackage{import}
%%
%% and then include the figures with
%%   \import{<path to file>}{<filename>.pgf}
%%
%% Matplotlib used the following preamble
%%   
%%   \usepackage{fontspec}
%%   \setmainfont{DejaVuSerif.ttf}[Path=\detokenize{/home/radimek/Documents/projekt_mat_prog/mat_prog_kernel/lib/python3.12/site-packages/matplotlib/mpl-data/fonts/ttf/}]
%%   \setsansfont{DejaVuSans.ttf}[Path=\detokenize{/home/radimek/Documents/projekt_mat_prog/mat_prog_kernel/lib/python3.12/site-packages/matplotlib/mpl-data/fonts/ttf/}]
%%   \setmonofont{DejaVuSansMono.ttf}[Path=\detokenize{/home/radimek/Documents/projekt_mat_prog/mat_prog_kernel/lib/python3.12/site-packages/matplotlib/mpl-data/fonts/ttf/}]
%%   \makeatletter\@ifpackageloaded{underscore}{}{\usepackage[strings]{underscore}}\makeatother
%%
\begingroup%
\makeatletter%
\begin{pgfpicture}%
\pgfpathrectangle{\pgfpointorigin}{\pgfqpoint{8.000000in}{6.000000in}}%
\pgfusepath{use as bounding box, clip}%
\begin{pgfscope}%
\pgfsetbuttcap%
\pgfsetmiterjoin%
\definecolor{currentfill}{rgb}{1.000000,1.000000,1.000000}%
\pgfsetfillcolor{currentfill}%
\pgfsetlinewidth{0.000000pt}%
\definecolor{currentstroke}{rgb}{1.000000,1.000000,1.000000}%
\pgfsetstrokecolor{currentstroke}%
\pgfsetdash{}{0pt}%
\pgfpathmoveto{\pgfqpoint{0.000000in}{0.000000in}}%
\pgfpathlineto{\pgfqpoint{8.000000in}{0.000000in}}%
\pgfpathlineto{\pgfqpoint{8.000000in}{6.000000in}}%
\pgfpathlineto{\pgfqpoint{0.000000in}{6.000000in}}%
\pgfpathlineto{\pgfqpoint{0.000000in}{0.000000in}}%
\pgfpathclose%
\pgfusepath{fill}%
\end{pgfscope}%
\begin{pgfscope}%
\pgfsetbuttcap%
\pgfsetmiterjoin%
\definecolor{currentfill}{rgb}{1.000000,1.000000,1.000000}%
\pgfsetfillcolor{currentfill}%
\pgfsetlinewidth{0.000000pt}%
\definecolor{currentstroke}{rgb}{0.000000,0.000000,0.000000}%
\pgfsetstrokecolor{currentstroke}%
\pgfsetstrokeopacity{0.000000}%
\pgfsetdash{}{0pt}%
\pgfpathmoveto{\pgfqpoint{1.150000in}{0.150000in}}%
\pgfpathlineto{\pgfqpoint{6.850000in}{0.150000in}}%
\pgfpathlineto{\pgfqpoint{6.850000in}{5.850000in}}%
\pgfpathlineto{\pgfqpoint{1.150000in}{5.850000in}}%
\pgfpathlineto{\pgfqpoint{1.150000in}{0.150000in}}%
\pgfpathclose%
\pgfusepath{fill}%
\end{pgfscope}%
\begin{pgfscope}%
\pgfsetbuttcap%
\pgfsetmiterjoin%
\definecolor{currentfill}{rgb}{0.950000,0.950000,0.950000}%
\pgfsetfillcolor{currentfill}%
\pgfsetfillopacity{0.500000}%
\pgfsetlinewidth{1.003750pt}%
\definecolor{currentstroke}{rgb}{0.950000,0.950000,0.950000}%
\pgfsetstrokecolor{currentstroke}%
\pgfsetstrokeopacity{0.500000}%
\pgfsetdash{}{0pt}%
\pgfpathmoveto{\pgfqpoint{1.580389in}{1.555437in}}%
\pgfpathlineto{\pgfqpoint{3.462715in}{3.133240in}}%
\pgfpathlineto{\pgfqpoint{3.436549in}{5.408715in}}%
\pgfpathlineto{\pgfqpoint{1.464144in}{3.969343in}}%
\pgfusepath{stroke,fill}%
\end{pgfscope}%
\begin{pgfscope}%
\pgfsetbuttcap%
\pgfsetmiterjoin%
\definecolor{currentfill}{rgb}{0.900000,0.900000,0.900000}%
\pgfsetfillcolor{currentfill}%
\pgfsetfillopacity{0.500000}%
\pgfsetlinewidth{1.003750pt}%
\definecolor{currentstroke}{rgb}{0.900000,0.900000,0.900000}%
\pgfsetstrokecolor{currentstroke}%
\pgfsetstrokeopacity{0.500000}%
\pgfsetdash{}{0pt}%
\pgfpathmoveto{\pgfqpoint{3.462715in}{3.133240in}}%
\pgfpathlineto{\pgfqpoint{6.483177in}{2.255311in}}%
\pgfpathlineto{\pgfqpoint{6.590967in}{4.609162in}}%
\pgfpathlineto{\pgfqpoint{3.436549in}{5.408715in}}%
\pgfusepath{stroke,fill}%
\end{pgfscope}%
\begin{pgfscope}%
\pgfsetbuttcap%
\pgfsetmiterjoin%
\definecolor{currentfill}{rgb}{0.925000,0.925000,0.925000}%
\pgfsetfillcolor{currentfill}%
\pgfsetfillopacity{0.500000}%
\pgfsetlinewidth{1.003750pt}%
\definecolor{currentstroke}{rgb}{0.925000,0.925000,0.925000}%
\pgfsetstrokecolor{currentstroke}%
\pgfsetstrokeopacity{0.500000}%
\pgfsetdash{}{0pt}%
\pgfpathmoveto{\pgfqpoint{1.580389in}{1.555437in}}%
\pgfpathlineto{\pgfqpoint{4.782226in}{0.509717in}}%
\pgfpathlineto{\pgfqpoint{6.483177in}{2.255311in}}%
\pgfpathlineto{\pgfqpoint{3.462715in}{3.133240in}}%
\pgfusepath{stroke,fill}%
\end{pgfscope}%
\begin{pgfscope}%
\pgfsetrectcap%
\pgfsetroundjoin%
\pgfsetlinewidth{0.803000pt}%
\definecolor{currentstroke}{rgb}{0.000000,0.000000,0.000000}%
\pgfsetstrokecolor{currentstroke}%
\pgfsetdash{}{0pt}%
\pgfpathmoveto{\pgfqpoint{1.580389in}{1.555437in}}%
\pgfpathlineto{\pgfqpoint{4.782226in}{0.509717in}}%
\pgfusepath{stroke}%
\end{pgfscope}%
\begin{pgfscope}%
\definecolor{textcolor}{rgb}{0.000000,0.000000,0.000000}%
\pgfsetstrokecolor{textcolor}%
\pgfsetfillcolor{textcolor}%
\pgftext[x=2.913491in,y=0.557898in,,]{\color{textcolor}\sffamily\fontsize{10.000000}{12.000000}\selectfont x}%
\end{pgfscope}%
\begin{pgfscope}%
\pgfsetbuttcap%
\pgfsetroundjoin%
\pgfsetlinewidth{0.803000pt}%
\definecolor{currentstroke}{rgb}{0.690196,0.690196,0.690196}%
\pgfsetstrokecolor{currentstroke}%
\pgfsetdash{}{0pt}%
\pgfpathmoveto{\pgfqpoint{1.774309in}{1.492103in}}%
\pgfpathlineto{\pgfqpoint{3.646411in}{3.079847in}}%
\pgfpathlineto{\pgfqpoint{3.628011in}{5.360185in}}%
\pgfusepath{stroke}%
\end{pgfscope}%
\begin{pgfscope}%
\pgfsetbuttcap%
\pgfsetroundjoin%
\pgfsetlinewidth{0.803000pt}%
\definecolor{currentstroke}{rgb}{0.690196,0.690196,0.690196}%
\pgfsetstrokecolor{currentstroke}%
\pgfsetdash{}{0pt}%
\pgfpathmoveto{\pgfqpoint{2.222368in}{1.345767in}}%
\pgfpathlineto{\pgfqpoint{4.070468in}{2.956591in}}%
\pgfpathlineto{\pgfqpoint{4.070186in}{5.248106in}}%
\pgfusepath{stroke}%
\end{pgfscope}%
\begin{pgfscope}%
\pgfsetbuttcap%
\pgfsetroundjoin%
\pgfsetlinewidth{0.803000pt}%
\definecolor{currentstroke}{rgb}{0.690196,0.690196,0.690196}%
\pgfsetstrokecolor{currentstroke}%
\pgfsetdash{}{0pt}%
\pgfpathmoveto{\pgfqpoint{2.677247in}{1.197204in}}%
\pgfpathlineto{\pgfqpoint{4.500444in}{2.831614in}}%
\pgfpathlineto{\pgfqpoint{4.518800in}{5.134396in}}%
\pgfusepath{stroke}%
\end{pgfscope}%
\begin{pgfscope}%
\pgfsetbuttcap%
\pgfsetroundjoin%
\pgfsetlinewidth{0.803000pt}%
\definecolor{currentstroke}{rgb}{0.690196,0.690196,0.690196}%
\pgfsetstrokecolor{currentstroke}%
\pgfsetdash{}{0pt}%
\pgfpathmoveto{\pgfqpoint{3.139103in}{1.046362in}}%
\pgfpathlineto{\pgfqpoint{4.936464in}{2.704880in}}%
\pgfpathlineto{\pgfqpoint{4.973994in}{5.019017in}}%
\pgfusepath{stroke}%
\end{pgfscope}%
\begin{pgfscope}%
\pgfsetbuttcap%
\pgfsetroundjoin%
\pgfsetlinewidth{0.803000pt}%
\definecolor{currentstroke}{rgb}{0.690196,0.690196,0.690196}%
\pgfsetstrokecolor{currentstroke}%
\pgfsetdash{}{0pt}%
\pgfpathmoveto{\pgfqpoint{3.608098in}{0.893188in}}%
\pgfpathlineto{\pgfqpoint{5.378655in}{2.576352in}}%
\pgfpathlineto{\pgfqpoint{5.435914in}{4.901934in}}%
\pgfusepath{stroke}%
\end{pgfscope}%
\begin{pgfscope}%
\pgfsetbuttcap%
\pgfsetroundjoin%
\pgfsetlinewidth{0.803000pt}%
\definecolor{currentstroke}{rgb}{0.690196,0.690196,0.690196}%
\pgfsetstrokecolor{currentstroke}%
\pgfsetdash{}{0pt}%
\pgfpathmoveto{\pgfqpoint{4.084398in}{0.737628in}}%
\pgfpathlineto{\pgfqpoint{5.827149in}{2.445993in}}%
\pgfpathlineto{\pgfqpoint{5.904712in}{4.783107in}}%
\pgfusepath{stroke}%
\end{pgfscope}%
\begin{pgfscope}%
\pgfsetbuttcap%
\pgfsetroundjoin%
\pgfsetlinewidth{0.803000pt}%
\definecolor{currentstroke}{rgb}{0.690196,0.690196,0.690196}%
\pgfsetstrokecolor{currentstroke}%
\pgfsetdash{}{0pt}%
\pgfpathmoveto{\pgfqpoint{4.568177in}{0.579626in}}%
\pgfpathlineto{\pgfqpoint{6.282083in}{2.313762in}}%
\pgfpathlineto{\pgfqpoint{6.380540in}{4.662499in}}%
\pgfusepath{stroke}%
\end{pgfscope}%
\begin{pgfscope}%
\pgfsetrectcap%
\pgfsetroundjoin%
\pgfsetlinewidth{0.803000pt}%
\definecolor{currentstroke}{rgb}{0.000000,0.000000,0.000000}%
\pgfsetstrokecolor{currentstroke}%
\pgfsetdash{}{0pt}%
\pgfpathmoveto{\pgfqpoint{1.790612in}{1.505929in}}%
\pgfpathlineto{\pgfqpoint{1.741635in}{1.464392in}}%
\pgfusepath{stroke}%
\end{pgfscope}%
\begin{pgfscope}%
\definecolor{textcolor}{rgb}{0.000000,0.000000,0.000000}%
\pgfsetstrokecolor{textcolor}%
\pgfsetfillcolor{textcolor}%
\pgftext[x=1.669876in,y=1.274184in,,top]{\color{textcolor}\sffamily\fontsize{10.000000}{12.000000}\selectfont \ensuremath{-}1.0}%
\end{pgfscope}%
\begin{pgfscope}%
\pgfsetrectcap%
\pgfsetroundjoin%
\pgfsetlinewidth{0.803000pt}%
\definecolor{currentstroke}{rgb}{0.000000,0.000000,0.000000}%
\pgfsetstrokecolor{currentstroke}%
\pgfsetdash{}{0pt}%
\pgfpathmoveto{\pgfqpoint{2.238471in}{1.359803in}}%
\pgfpathlineto{\pgfqpoint{2.190092in}{1.317635in}}%
\pgfusepath{stroke}%
\end{pgfscope}%
\begin{pgfscope}%
\definecolor{textcolor}{rgb}{0.000000,0.000000,0.000000}%
\pgfsetstrokecolor{textcolor}%
\pgfsetfillcolor{textcolor}%
\pgftext[x=2.118230in,y=1.125843in,,top]{\color{textcolor}\sffamily\fontsize{10.000000}{12.000000}\selectfont \ensuremath{-}0.5}%
\end{pgfscope}%
\begin{pgfscope}%
\pgfsetrectcap%
\pgfsetroundjoin%
\pgfsetlinewidth{0.803000pt}%
\definecolor{currentstroke}{rgb}{0.000000,0.000000,0.000000}%
\pgfsetstrokecolor{currentstroke}%
\pgfsetdash{}{0pt}%
\pgfpathmoveto{\pgfqpoint{2.693143in}{1.211454in}}%
\pgfpathlineto{\pgfqpoint{2.645386in}{1.168642in}}%
\pgfusepath{stroke}%
\end{pgfscope}%
\begin{pgfscope}%
\definecolor{textcolor}{rgb}{0.000000,0.000000,0.000000}%
\pgfsetstrokecolor{textcolor}%
\pgfsetfillcolor{textcolor}%
\pgftext[x=2.573426in,y=0.975238in,,top]{\color{textcolor}\sffamily\fontsize{10.000000}{12.000000}\selectfont 0.0}%
\end{pgfscope}%
\begin{pgfscope}%
\pgfsetrectcap%
\pgfsetroundjoin%
\pgfsetlinewidth{0.803000pt}%
\definecolor{currentstroke}{rgb}{0.000000,0.000000,0.000000}%
\pgfsetstrokecolor{currentstroke}%
\pgfsetdash{}{0pt}%
\pgfpathmoveto{\pgfqpoint{3.154783in}{1.060831in}}%
\pgfpathlineto{\pgfqpoint{3.107673in}{1.017359in}}%
\pgfusepath{stroke}%
\end{pgfscope}%
\begin{pgfscope}%
\definecolor{textcolor}{rgb}{0.000000,0.000000,0.000000}%
\pgfsetstrokecolor{textcolor}%
\pgfsetfillcolor{textcolor}%
\pgftext[x=3.035622in,y=0.822318in,,top]{\color{textcolor}\sffamily\fontsize{10.000000}{12.000000}\selectfont 0.5}%
\end{pgfscope}%
\begin{pgfscope}%
\pgfsetrectcap%
\pgfsetroundjoin%
\pgfsetlinewidth{0.803000pt}%
\definecolor{currentstroke}{rgb}{0.000000,0.000000,0.000000}%
\pgfsetstrokecolor{currentstroke}%
\pgfsetdash{}{0pt}%
\pgfpathmoveto{\pgfqpoint{3.623554in}{0.907881in}}%
\pgfpathlineto{\pgfqpoint{3.577116in}{0.863735in}}%
\pgfusepath{stroke}%
\end{pgfscope}%
\begin{pgfscope}%
\definecolor{textcolor}{rgb}{0.000000,0.000000,0.000000}%
\pgfsetstrokecolor{textcolor}%
\pgfsetfillcolor{textcolor}%
\pgftext[x=3.504979in,y=0.667028in,,top]{\color{textcolor}\sffamily\fontsize{10.000000}{12.000000}\selectfont 1.0}%
\end{pgfscope}%
\begin{pgfscope}%
\pgfsetrectcap%
\pgfsetroundjoin%
\pgfsetlinewidth{0.803000pt}%
\definecolor{currentstroke}{rgb}{0.000000,0.000000,0.000000}%
\pgfsetstrokecolor{currentstroke}%
\pgfsetdash{}{0pt}%
\pgfpathmoveto{\pgfqpoint{4.099622in}{0.752551in}}%
\pgfpathlineto{\pgfqpoint{4.053883in}{0.707715in}}%
\pgfusepath{stroke}%
\end{pgfscope}%
\begin{pgfscope}%
\definecolor{textcolor}{rgb}{0.000000,0.000000,0.000000}%
\pgfsetstrokecolor{textcolor}%
\pgfsetfillcolor{textcolor}%
\pgftext[x=3.981665in,y=0.509314in,,top]{\color{textcolor}\sffamily\fontsize{10.000000}{12.000000}\selectfont 1.5}%
\end{pgfscope}%
\begin{pgfscope}%
\pgfsetrectcap%
\pgfsetroundjoin%
\pgfsetlinewidth{0.803000pt}%
\definecolor{currentstroke}{rgb}{0.000000,0.000000,0.000000}%
\pgfsetstrokecolor{currentstroke}%
\pgfsetdash{}{0pt}%
\pgfpathmoveto{\pgfqpoint{4.583158in}{0.594784in}}%
\pgfpathlineto{\pgfqpoint{4.538146in}{0.549241in}}%
\pgfusepath{stroke}%
\end{pgfscope}%
\begin{pgfscope}%
\definecolor{textcolor}{rgb}{0.000000,0.000000,0.000000}%
\pgfsetstrokecolor{textcolor}%
\pgfsetfillcolor{textcolor}%
\pgftext[x=4.465855in,y=0.349116in,,top]{\color{textcolor}\sffamily\fontsize{10.000000}{12.000000}\selectfont 2.0}%
\end{pgfscope}%
\begin{pgfscope}%
\pgfsetrectcap%
\pgfsetroundjoin%
\pgfsetlinewidth{0.803000pt}%
\definecolor{currentstroke}{rgb}{0.000000,0.000000,0.000000}%
\pgfsetstrokecolor{currentstroke}%
\pgfsetdash{}{0pt}%
\pgfpathmoveto{\pgfqpoint{6.483177in}{2.255311in}}%
\pgfpathlineto{\pgfqpoint{4.782226in}{0.509717in}}%
\pgfusepath{stroke}%
\end{pgfscope}%
\begin{pgfscope}%
\definecolor{textcolor}{rgb}{0.000000,0.000000,0.000000}%
\pgfsetstrokecolor{textcolor}%
\pgfsetfillcolor{textcolor}%
\pgftext[x=6.045209in,y=1.032725in,,]{\color{textcolor}\sffamily\fontsize{10.000000}{12.000000}\selectfont y}%
\end{pgfscope}%
\begin{pgfscope}%
\pgfsetbuttcap%
\pgfsetroundjoin%
\pgfsetlinewidth{0.803000pt}%
\definecolor{currentstroke}{rgb}{0.690196,0.690196,0.690196}%
\pgfsetstrokecolor{currentstroke}%
\pgfsetdash{}{0pt}%
\pgfpathmoveto{\pgfqpoint{1.600541in}{4.068879in}}%
\pgfpathlineto{\pgfqpoint{1.710097in}{1.664161in}}%
\pgfpathlineto{\pgfqpoint{4.899919in}{0.630499in}}%
\pgfusepath{stroke}%
\end{pgfscope}%
\begin{pgfscope}%
\pgfsetbuttcap%
\pgfsetroundjoin%
\pgfsetlinewidth{0.803000pt}%
\definecolor{currentstroke}{rgb}{0.690196,0.690196,0.690196}%
\pgfsetstrokecolor{currentstroke}%
\pgfsetdash{}{0pt}%
\pgfpathmoveto{\pgfqpoint{1.830662in}{4.236811in}}%
\pgfpathlineto{\pgfqpoint{1.929089in}{1.847724in}}%
\pgfpathlineto{\pgfqpoint{5.098461in}{0.834252in}}%
\pgfusepath{stroke}%
\end{pgfscope}%
\begin{pgfscope}%
\pgfsetbuttcap%
\pgfsetroundjoin%
\pgfsetlinewidth{0.803000pt}%
\definecolor{currentstroke}{rgb}{0.690196,0.690196,0.690196}%
\pgfsetstrokecolor{currentstroke}%
\pgfsetdash{}{0pt}%
\pgfpathmoveto{\pgfqpoint{2.056228in}{4.401419in}}%
\pgfpathlineto{\pgfqpoint{2.143933in}{2.027810in}}%
\pgfpathlineto{\pgfqpoint{5.293045in}{1.033943in}}%
\pgfusepath{stroke}%
\end{pgfscope}%
\begin{pgfscope}%
\pgfsetbuttcap%
\pgfsetroundjoin%
\pgfsetlinewidth{0.803000pt}%
\definecolor{currentstroke}{rgb}{0.690196,0.690196,0.690196}%
\pgfsetstrokecolor{currentstroke}%
\pgfsetdash{}{0pt}%
\pgfpathmoveto{\pgfqpoint{2.277372in}{4.562799in}}%
\pgfpathlineto{\pgfqpoint{2.354746in}{2.204518in}}%
\pgfpathlineto{\pgfqpoint{5.483787in}{1.229691in}}%
\pgfusepath{stroke}%
\end{pgfscope}%
\begin{pgfscope}%
\pgfsetbuttcap%
\pgfsetroundjoin%
\pgfsetlinewidth{0.803000pt}%
\definecolor{currentstroke}{rgb}{0.690196,0.690196,0.690196}%
\pgfsetstrokecolor{currentstroke}%
\pgfsetdash{}{0pt}%
\pgfpathmoveto{\pgfqpoint{2.494222in}{4.721047in}}%
\pgfpathlineto{\pgfqpoint{2.561641in}{2.377942in}}%
\pgfpathlineto{\pgfqpoint{5.670801in}{1.421614in}}%
\pgfusepath{stroke}%
\end{pgfscope}%
\begin{pgfscope}%
\pgfsetbuttcap%
\pgfsetroundjoin%
\pgfsetlinewidth{0.803000pt}%
\definecolor{currentstroke}{rgb}{0.690196,0.690196,0.690196}%
\pgfsetstrokecolor{currentstroke}%
\pgfsetdash{}{0pt}%
\pgfpathmoveto{\pgfqpoint{2.706903in}{4.876252in}}%
\pgfpathlineto{\pgfqpoint{2.764725in}{2.548171in}}%
\pgfpathlineto{\pgfqpoint{5.854194in}{1.609820in}}%
\pgfusepath{stroke}%
\end{pgfscope}%
\begin{pgfscope}%
\pgfsetbuttcap%
\pgfsetroundjoin%
\pgfsetlinewidth{0.803000pt}%
\definecolor{currentstroke}{rgb}{0.690196,0.690196,0.690196}%
\pgfsetstrokecolor{currentstroke}%
\pgfsetdash{}{0pt}%
\pgfpathmoveto{\pgfqpoint{2.915534in}{5.028501in}}%
\pgfpathlineto{\pgfqpoint{2.964104in}{2.715295in}}%
\pgfpathlineto{\pgfqpoint{6.034072in}{1.794419in}}%
\pgfusepath{stroke}%
\end{pgfscope}%
\begin{pgfscope}%
\pgfsetbuttcap%
\pgfsetroundjoin%
\pgfsetlinewidth{0.803000pt}%
\definecolor{currentstroke}{rgb}{0.690196,0.690196,0.690196}%
\pgfsetstrokecolor{currentstroke}%
\pgfsetdash{}{0pt}%
\pgfpathmoveto{\pgfqpoint{3.120229in}{5.177879in}}%
\pgfpathlineto{\pgfqpoint{3.159878in}{2.879396in}}%
\pgfpathlineto{\pgfqpoint{6.210533in}{1.975511in}}%
\pgfusepath{stroke}%
\end{pgfscope}%
\begin{pgfscope}%
\pgfsetbuttcap%
\pgfsetroundjoin%
\pgfsetlinewidth{0.803000pt}%
\definecolor{currentstroke}{rgb}{0.690196,0.690196,0.690196}%
\pgfsetstrokecolor{currentstroke}%
\pgfsetdash{}{0pt}%
\pgfpathmoveto{\pgfqpoint{3.321099in}{5.324464in}}%
\pgfpathlineto{\pgfqpoint{3.352144in}{3.040557in}}%
\pgfpathlineto{\pgfqpoint{6.383674in}{2.153197in}}%
\pgfusepath{stroke}%
\end{pgfscope}%
\begin{pgfscope}%
\pgfsetrectcap%
\pgfsetroundjoin%
\pgfsetlinewidth{0.803000pt}%
\definecolor{currentstroke}{rgb}{0.000000,0.000000,0.000000}%
\pgfsetstrokecolor{currentstroke}%
\pgfsetdash{}{0pt}%
\pgfpathmoveto{\pgfqpoint{4.873038in}{0.639210in}}%
\pgfpathlineto{\pgfqpoint{4.953750in}{0.613055in}}%
\pgfusepath{stroke}%
\end{pgfscope}%
\begin{pgfscope}%
\definecolor{textcolor}{rgb}{0.000000,0.000000,0.000000}%
\pgfsetstrokecolor{textcolor}%
\pgfsetfillcolor{textcolor}%
\pgftext[x=5.078779in,y=0.444104in,,top]{\color{textcolor}\sffamily\fontsize{10.000000}{12.000000}\selectfont \ensuremath{-}1.00}%
\end{pgfscope}%
\begin{pgfscope}%
\pgfsetrectcap%
\pgfsetroundjoin%
\pgfsetlinewidth{0.803000pt}%
\definecolor{currentstroke}{rgb}{0.000000,0.000000,0.000000}%
\pgfsetstrokecolor{currentstroke}%
\pgfsetdash{}{0pt}%
\pgfpathmoveto{\pgfqpoint{5.071766in}{0.842788in}}%
\pgfpathlineto{\pgfqpoint{5.151919in}{0.817158in}}%
\pgfusepath{stroke}%
\end{pgfscope}%
\begin{pgfscope}%
\definecolor{textcolor}{rgb}{0.000000,0.000000,0.000000}%
\pgfsetstrokecolor{textcolor}%
\pgfsetfillcolor{textcolor}%
\pgftext[x=5.275596in,y=0.649813in,,top]{\color{textcolor}\sffamily\fontsize{10.000000}{12.000000}\selectfont \ensuremath{-}0.75}%
\end{pgfscope}%
\begin{pgfscope}%
\pgfsetrectcap%
\pgfsetroundjoin%
\pgfsetlinewidth{0.803000pt}%
\definecolor{currentstroke}{rgb}{0.000000,0.000000,0.000000}%
\pgfsetstrokecolor{currentstroke}%
\pgfsetdash{}{0pt}%
\pgfpathmoveto{\pgfqpoint{5.266534in}{1.042310in}}%
\pgfpathlineto{\pgfqpoint{5.346134in}{1.017188in}}%
\pgfusepath{stroke}%
\end{pgfscope}%
\begin{pgfscope}%
\definecolor{textcolor}{rgb}{0.000000,0.000000,0.000000}%
\pgfsetstrokecolor{textcolor}%
\pgfsetfillcolor{textcolor}%
\pgftext[x=5.468487in,y=0.851419in,,top]{\color{textcolor}\sffamily\fontsize{10.000000}{12.000000}\selectfont \ensuremath{-}0.50}%
\end{pgfscope}%
\begin{pgfscope}%
\pgfsetrectcap%
\pgfsetroundjoin%
\pgfsetlinewidth{0.803000pt}%
\definecolor{currentstroke}{rgb}{0.000000,0.000000,0.000000}%
\pgfsetstrokecolor{currentstroke}%
\pgfsetdash{}{0pt}%
\pgfpathmoveto{\pgfqpoint{5.457458in}{1.237894in}}%
\pgfpathlineto{\pgfqpoint{5.536511in}{1.213266in}}%
\pgfusepath{stroke}%
\end{pgfscope}%
\begin{pgfscope}%
\definecolor{textcolor}{rgb}{0.000000,0.000000,0.000000}%
\pgfsetstrokecolor{textcolor}%
\pgfsetfillcolor{textcolor}%
\pgftext[x=5.657569in,y=1.049043in,,top]{\color{textcolor}\sffamily\fontsize{10.000000}{12.000000}\selectfont \ensuremath{-}0.25}%
\end{pgfscope}%
\begin{pgfscope}%
\pgfsetrectcap%
\pgfsetroundjoin%
\pgfsetlinewidth{0.803000pt}%
\definecolor{currentstroke}{rgb}{0.000000,0.000000,0.000000}%
\pgfsetstrokecolor{currentstroke}%
\pgfsetdash{}{0pt}%
\pgfpathmoveto{\pgfqpoint{5.644652in}{1.429657in}}%
\pgfpathlineto{\pgfqpoint{5.723164in}{1.405507in}}%
\pgfusepath{stroke}%
\end{pgfscope}%
\begin{pgfscope}%
\definecolor{textcolor}{rgb}{0.000000,0.000000,0.000000}%
\pgfsetstrokecolor{textcolor}%
\pgfsetfillcolor{textcolor}%
\pgftext[x=5.842953in,y=1.242803in,,top]{\color{textcolor}\sffamily\fontsize{10.000000}{12.000000}\selectfont 0.00}%
\end{pgfscope}%
\begin{pgfscope}%
\pgfsetrectcap%
\pgfsetroundjoin%
\pgfsetlinewidth{0.803000pt}%
\definecolor{currentstroke}{rgb}{0.000000,0.000000,0.000000}%
\pgfsetstrokecolor{currentstroke}%
\pgfsetdash{}{0pt}%
\pgfpathmoveto{\pgfqpoint{5.828223in}{1.617708in}}%
\pgfpathlineto{\pgfqpoint{5.906201in}{1.594025in}}%
\pgfusepath{stroke}%
\end{pgfscope}%
\begin{pgfscope}%
\definecolor{textcolor}{rgb}{0.000000,0.000000,0.000000}%
\pgfsetstrokecolor{textcolor}%
\pgfsetfillcolor{textcolor}%
\pgftext[x=6.024748in,y=1.432811in,,top]{\color{textcolor}\sffamily\fontsize{10.000000}{12.000000}\selectfont 0.25}%
\end{pgfscope}%
\begin{pgfscope}%
\pgfsetrectcap%
\pgfsetroundjoin%
\pgfsetlinewidth{0.803000pt}%
\definecolor{currentstroke}{rgb}{0.000000,0.000000,0.000000}%
\pgfsetstrokecolor{currentstroke}%
\pgfsetdash{}{0pt}%
\pgfpathmoveto{\pgfqpoint{6.008276in}{1.802156in}}%
\pgfpathlineto{\pgfqpoint{6.085725in}{1.778924in}}%
\pgfusepath{stroke}%
\end{pgfscope}%
\begin{pgfscope}%
\definecolor{textcolor}{rgb}{0.000000,0.000000,0.000000}%
\pgfsetstrokecolor{textcolor}%
\pgfsetfillcolor{textcolor}%
\pgftext[x=6.203055in,y=1.619174in,,top]{\color{textcolor}\sffamily\fontsize{10.000000}{12.000000}\selectfont 0.50}%
\end{pgfscope}%
\begin{pgfscope}%
\pgfsetrectcap%
\pgfsetroundjoin%
\pgfsetlinewidth{0.803000pt}%
\definecolor{currentstroke}{rgb}{0.000000,0.000000,0.000000}%
\pgfsetstrokecolor{currentstroke}%
\pgfsetdash{}{0pt}%
\pgfpathmoveto{\pgfqpoint{6.184911in}{1.983102in}}%
\pgfpathlineto{\pgfqpoint{6.261837in}{1.960310in}}%
\pgfusepath{stroke}%
\end{pgfscope}%
\begin{pgfscope}%
\definecolor{textcolor}{rgb}{0.000000,0.000000,0.000000}%
\pgfsetstrokecolor{textcolor}%
\pgfsetfillcolor{textcolor}%
\pgftext[x=6.377975in,y=1.801997in,,top]{\color{textcolor}\sffamily\fontsize{10.000000}{12.000000}\selectfont 0.75}%
\end{pgfscope}%
\begin{pgfscope}%
\pgfsetrectcap%
\pgfsetroundjoin%
\pgfsetlinewidth{0.803000pt}%
\definecolor{currentstroke}{rgb}{0.000000,0.000000,0.000000}%
\pgfsetstrokecolor{currentstroke}%
\pgfsetdash{}{0pt}%
\pgfpathmoveto{\pgfqpoint{6.358225in}{2.160646in}}%
\pgfpathlineto{\pgfqpoint{6.434634in}{2.138280in}}%
\pgfusepath{stroke}%
\end{pgfscope}%
\begin{pgfscope}%
\definecolor{textcolor}{rgb}{0.000000,0.000000,0.000000}%
\pgfsetstrokecolor{textcolor}%
\pgfsetfillcolor{textcolor}%
\pgftext[x=6.549603in,y=1.981379in,,top]{\color{textcolor}\sffamily\fontsize{10.000000}{12.000000}\selectfont 1.00}%
\end{pgfscope}%
\begin{pgfscope}%
\pgfsetrectcap%
\pgfsetroundjoin%
\pgfsetlinewidth{0.803000pt}%
\definecolor{currentstroke}{rgb}{0.000000,0.000000,0.000000}%
\pgfsetstrokecolor{currentstroke}%
\pgfsetdash{}{0pt}%
\pgfpathmoveto{\pgfqpoint{6.483177in}{2.255311in}}%
\pgfpathlineto{\pgfqpoint{6.590967in}{4.609162in}}%
\pgfusepath{stroke}%
\end{pgfscope}%
\begin{pgfscope}%
\definecolor{textcolor}{rgb}{0.000000,0.000000,0.000000}%
\pgfsetstrokecolor{textcolor}%
\pgfsetfillcolor{textcolor}%
\pgftext[x=7.097978in,y=3.481758in,,,rotate=87.378092]{\color{textcolor}\sffamily\fontsize{10.000000}{12.000000}\selectfont f(x, y)}%
\end{pgfscope}%
\begin{pgfscope}%
\pgfsetbuttcap%
\pgfsetroundjoin%
\pgfsetlinewidth{0.803000pt}%
\definecolor{currentstroke}{rgb}{0.690196,0.690196,0.690196}%
\pgfsetstrokecolor{currentstroke}%
\pgfsetdash{}{0pt}%
\pgfpathmoveto{\pgfqpoint{6.491485in}{2.436728in}}%
\pgfpathlineto{\pgfqpoint{3.460695in}{3.308959in}}%
\pgfpathlineto{\pgfqpoint{1.571444in}{1.741193in}}%
\pgfusepath{stroke}%
\end{pgfscope}%
\begin{pgfscope}%
\pgfsetbuttcap%
\pgfsetroundjoin%
\pgfsetlinewidth{0.803000pt}%
\definecolor{currentstroke}{rgb}{0.690196,0.690196,0.690196}%
\pgfsetstrokecolor{currentstroke}%
\pgfsetdash{}{0pt}%
\pgfpathmoveto{\pgfqpoint{6.506094in}{2.755756in}}%
\pgfpathlineto{\pgfqpoint{3.457143in}{3.617827in}}%
\pgfpathlineto{\pgfqpoint{1.555708in}{2.067969in}}%
\pgfusepath{stroke}%
\end{pgfscope}%
\begin{pgfscope}%
\pgfsetbuttcap%
\pgfsetroundjoin%
\pgfsetlinewidth{0.803000pt}%
\definecolor{currentstroke}{rgb}{0.690196,0.690196,0.690196}%
\pgfsetstrokecolor{currentstroke}%
\pgfsetdash{}{0pt}%
\pgfpathmoveto{\pgfqpoint{6.520881in}{3.078669in}}%
\pgfpathlineto{\pgfqpoint{3.453550in}{3.930275in}}%
\pgfpathlineto{\pgfqpoint{1.539772in}{2.398876in}}%
\pgfusepath{stroke}%
\end{pgfscope}%
\begin{pgfscope}%
\pgfsetbuttcap%
\pgfsetroundjoin%
\pgfsetlinewidth{0.803000pt}%
\definecolor{currentstroke}{rgb}{0.690196,0.690196,0.690196}%
\pgfsetstrokecolor{currentstroke}%
\pgfsetdash{}{0pt}%
\pgfpathmoveto{\pgfqpoint{6.535849in}{3.405537in}}%
\pgfpathlineto{\pgfqpoint{3.449915in}{4.246367in}}%
\pgfpathlineto{\pgfqpoint{1.523634in}{2.733992in}}%
\pgfusepath{stroke}%
\end{pgfscope}%
\begin{pgfscope}%
\pgfsetbuttcap%
\pgfsetroundjoin%
\pgfsetlinewidth{0.803000pt}%
\definecolor{currentstroke}{rgb}{0.690196,0.690196,0.690196}%
\pgfsetstrokecolor{currentstroke}%
\pgfsetdash{}{0pt}%
\pgfpathmoveto{\pgfqpoint{6.551002in}{3.736434in}}%
\pgfpathlineto{\pgfqpoint{3.446238in}{4.566166in}}%
\pgfpathlineto{\pgfqpoint{1.507290in}{3.073399in}}%
\pgfusepath{stroke}%
\end{pgfscope}%
\begin{pgfscope}%
\pgfsetbuttcap%
\pgfsetroundjoin%
\pgfsetlinewidth{0.803000pt}%
\definecolor{currentstroke}{rgb}{0.690196,0.690196,0.690196}%
\pgfsetstrokecolor{currentstroke}%
\pgfsetdash{}{0pt}%
\pgfpathmoveto{\pgfqpoint{6.566343in}{4.071434in}}%
\pgfpathlineto{\pgfqpoint{3.442517in}{4.889738in}}%
\pgfpathlineto{\pgfqpoint{1.490734in}{3.417178in}}%
\pgfusepath{stroke}%
\end{pgfscope}%
\begin{pgfscope}%
\pgfsetbuttcap%
\pgfsetroundjoin%
\pgfsetlinewidth{0.803000pt}%
\definecolor{currentstroke}{rgb}{0.690196,0.690196,0.690196}%
\pgfsetstrokecolor{currentstroke}%
\pgfsetdash{}{0pt}%
\pgfpathmoveto{\pgfqpoint{6.581875in}{4.410615in}}%
\pgfpathlineto{\pgfqpoint{3.438752in}{5.217149in}}%
\pgfpathlineto{\pgfqpoint{1.473965in}{3.765416in}}%
\pgfusepath{stroke}%
\end{pgfscope}%
\begin{pgfscope}%
\pgfsetrectcap%
\pgfsetroundjoin%
\pgfsetlinewidth{0.803000pt}%
\definecolor{currentstroke}{rgb}{0.000000,0.000000,0.000000}%
\pgfsetstrokecolor{currentstroke}%
\pgfsetdash{}{0pt}%
\pgfpathmoveto{\pgfqpoint{6.466044in}{2.444050in}}%
\pgfpathlineto{\pgfqpoint{6.542427in}{2.422067in}}%
\pgfusepath{stroke}%
\end{pgfscope}%
\begin{pgfscope}%
\definecolor{textcolor}{rgb}{0.000000,0.000000,0.000000}%
\pgfsetstrokecolor{textcolor}%
\pgfsetfillcolor{textcolor}%
\pgftext[x=6.746064in,y=2.472717in,,top]{\color{textcolor}\sffamily\fontsize{10.000000}{12.000000}\selectfont 2}%
\end{pgfscope}%
\begin{pgfscope}%
\pgfsetrectcap%
\pgfsetroundjoin%
\pgfsetlinewidth{0.803000pt}%
\definecolor{currentstroke}{rgb}{0.000000,0.000000,0.000000}%
\pgfsetstrokecolor{currentstroke}%
\pgfsetdash{}{0pt}%
\pgfpathmoveto{\pgfqpoint{6.480494in}{2.762995in}}%
\pgfpathlineto{\pgfqpoint{6.557357in}{2.741262in}}%
\pgfusepath{stroke}%
\end{pgfscope}%
\begin{pgfscope}%
\definecolor{textcolor}{rgb}{0.000000,0.000000,0.000000}%
\pgfsetstrokecolor{textcolor}%
\pgfsetfillcolor{textcolor}%
\pgftext[x=6.762182in,y=2.791336in,,top]{\color{textcolor}\sffamily\fontsize{10.000000}{12.000000}\selectfont 3}%
\end{pgfscope}%
\begin{pgfscope}%
\pgfsetrectcap%
\pgfsetroundjoin%
\pgfsetlinewidth{0.803000pt}%
\definecolor{currentstroke}{rgb}{0.000000,0.000000,0.000000}%
\pgfsetstrokecolor{currentstroke}%
\pgfsetdash{}{0pt}%
\pgfpathmoveto{\pgfqpoint{6.495119in}{3.085821in}}%
\pgfpathlineto{\pgfqpoint{6.572468in}{3.064346in}}%
\pgfusepath{stroke}%
\end{pgfscope}%
\begin{pgfscope}%
\definecolor{textcolor}{rgb}{0.000000,0.000000,0.000000}%
\pgfsetstrokecolor{textcolor}%
\pgfsetfillcolor{textcolor}%
\pgftext[x=6.778495in,y=3.113826in,,top]{\color{textcolor}\sffamily\fontsize{10.000000}{12.000000}\selectfont 4}%
\end{pgfscope}%
\begin{pgfscope}%
\pgfsetrectcap%
\pgfsetroundjoin%
\pgfsetlinewidth{0.803000pt}%
\definecolor{currentstroke}{rgb}{0.000000,0.000000,0.000000}%
\pgfsetstrokecolor{currentstroke}%
\pgfsetdash{}{0pt}%
\pgfpathmoveto{\pgfqpoint{6.509923in}{3.412601in}}%
\pgfpathlineto{\pgfqpoint{6.587765in}{3.391391in}}%
\pgfusepath{stroke}%
\end{pgfscope}%
\begin{pgfscope}%
\definecolor{textcolor}{rgb}{0.000000,0.000000,0.000000}%
\pgfsetstrokecolor{textcolor}%
\pgfsetfillcolor{textcolor}%
\pgftext[x=6.795009in,y=3.440259in,,top]{\color{textcolor}\sffamily\fontsize{10.000000}{12.000000}\selectfont 5}%
\end{pgfscope}%
\begin{pgfscope}%
\pgfsetrectcap%
\pgfsetroundjoin%
\pgfsetlinewidth{0.803000pt}%
\definecolor{currentstroke}{rgb}{0.000000,0.000000,0.000000}%
\pgfsetstrokecolor{currentstroke}%
\pgfsetdash{}{0pt}%
\pgfpathmoveto{\pgfqpoint{6.524910in}{3.743407in}}%
\pgfpathlineto{\pgfqpoint{6.603250in}{3.722471in}}%
\pgfusepath{stroke}%
\end{pgfscope}%
\begin{pgfscope}%
\definecolor{textcolor}{rgb}{0.000000,0.000000,0.000000}%
\pgfsetstrokecolor{textcolor}%
\pgfsetfillcolor{textcolor}%
\pgftext[x=6.811725in,y=3.770708in,,top]{\color{textcolor}\sffamily\fontsize{10.000000}{12.000000}\selectfont 6}%
\end{pgfscope}%
\begin{pgfscope}%
\pgfsetrectcap%
\pgfsetroundjoin%
\pgfsetlinewidth{0.803000pt}%
\definecolor{currentstroke}{rgb}{0.000000,0.000000,0.000000}%
\pgfsetstrokecolor{currentstroke}%
\pgfsetdash{}{0pt}%
\pgfpathmoveto{\pgfqpoint{6.540083in}{4.078313in}}%
\pgfpathlineto{\pgfqpoint{6.618927in}{4.057659in}}%
\pgfusepath{stroke}%
\end{pgfscope}%
\begin{pgfscope}%
\definecolor{textcolor}{rgb}{0.000000,0.000000,0.000000}%
\pgfsetstrokecolor{textcolor}%
\pgfsetfillcolor{textcolor}%
\pgftext[x=6.828648in,y=4.105246in,,top]{\color{textcolor}\sffamily\fontsize{10.000000}{12.000000}\selectfont 7}%
\end{pgfscope}%
\begin{pgfscope}%
\pgfsetrectcap%
\pgfsetroundjoin%
\pgfsetlinewidth{0.803000pt}%
\definecolor{currentstroke}{rgb}{0.000000,0.000000,0.000000}%
\pgfsetstrokecolor{currentstroke}%
\pgfsetdash{}{0pt}%
\pgfpathmoveto{\pgfqpoint{6.555445in}{4.417397in}}%
\pgfpathlineto{\pgfqpoint{6.634801in}{4.397034in}}%
\pgfusepath{stroke}%
\end{pgfscope}%
\begin{pgfscope}%
\definecolor{textcolor}{rgb}{0.000000,0.000000,0.000000}%
\pgfsetstrokecolor{textcolor}%
\pgfsetfillcolor{textcolor}%
\pgftext[x=6.845782in,y=4.443950in,,top]{\color{textcolor}\sffamily\fontsize{10.000000}{12.000000}\selectfont 8}%
\end{pgfscope}%
\begin{pgfscope}%
\pgfpathrectangle{\pgfqpoint{1.150000in}{0.150000in}}{\pgfqpoint{5.700000in}{5.700000in}}%
\pgfusepath{clip}%
\pgfsetbuttcap%
\pgfsetroundjoin%
\definecolor{currentfill}{rgb}{0.136408,0.541173,0.554483}%
\pgfsetfillcolor{currentfill}%
\pgfsetfillopacity{0.800000}%
\pgfsetlinewidth{0.000000pt}%
\definecolor{currentstroke}{rgb}{0.000000,0.000000,0.000000}%
\pgfsetstrokecolor{currentstroke}%
\pgfsetdash{}{0pt}%
\pgfpathmoveto{\pgfqpoint{4.071654in}{3.739275in}}%
\pgfpathlineto{\pgfqpoint{4.084938in}{3.723813in}}%
\pgfpathlineto{\pgfqpoint{4.098221in}{3.708593in}}%
\pgfpathlineto{\pgfqpoint{4.111505in}{3.693611in}}%
\pgfpathlineto{\pgfqpoint{4.124790in}{3.678867in}}%
\pgfpathlineto{\pgfqpoint{4.132459in}{3.704914in}}%
\pgfpathlineto{\pgfqpoint{4.140128in}{3.731384in}}%
\pgfpathlineto{\pgfqpoint{4.147798in}{3.758285in}}%
\pgfpathlineto{\pgfqpoint{4.155469in}{3.785625in}}%
\pgfpathlineto{\pgfqpoint{4.142185in}{3.801078in}}%
\pgfpathlineto{\pgfqpoint{4.128901in}{3.816769in}}%
\pgfpathlineto{\pgfqpoint{4.115617in}{3.832700in}}%
\pgfpathlineto{\pgfqpoint{4.102333in}{3.848873in}}%
\pgfpathlineto{\pgfqpoint{4.094663in}{3.820809in}}%
\pgfpathlineto{\pgfqpoint{4.086993in}{3.793193in}}%
\pgfpathlineto{\pgfqpoint{4.079323in}{3.766018in}}%
\pgfpathlineto{\pgfqpoint{4.071654in}{3.739275in}}%
\pgfpathclose%
\pgfusepath{fill}%
\end{pgfscope}%
\begin{pgfscope}%
\pgfpathrectangle{\pgfqpoint{1.150000in}{0.150000in}}{\pgfqpoint{5.700000in}{5.700000in}}%
\pgfusepath{clip}%
\pgfsetbuttcap%
\pgfsetroundjoin%
\definecolor{currentfill}{rgb}{0.141935,0.526453,0.555991}%
\pgfsetfillcolor{currentfill}%
\pgfsetfillopacity{0.800000}%
\pgfsetlinewidth{0.000000pt}%
\definecolor{currentstroke}{rgb}{0.000000,0.000000,0.000000}%
\pgfsetstrokecolor{currentstroke}%
\pgfsetdash{}{0pt}%
\pgfpathmoveto{\pgfqpoint{3.987835in}{3.698031in}}%
\pgfpathlineto{\pgfqpoint{4.001120in}{3.682267in}}%
\pgfpathlineto{\pgfqpoint{4.014405in}{3.666750in}}%
\pgfpathlineto{\pgfqpoint{4.027689in}{3.651479in}}%
\pgfpathlineto{\pgfqpoint{4.040974in}{3.636451in}}%
\pgfpathlineto{\pgfqpoint{4.048644in}{3.661551in}}%
\pgfpathlineto{\pgfqpoint{4.056315in}{3.687050in}}%
\pgfpathlineto{\pgfqpoint{4.063984in}{3.712955in}}%
\pgfpathlineto{\pgfqpoint{4.071654in}{3.739275in}}%
\pgfpathlineto{\pgfqpoint{4.058370in}{3.754979in}}%
\pgfpathlineto{\pgfqpoint{4.045086in}{3.770928in}}%
\pgfpathlineto{\pgfqpoint{4.031801in}{3.787123in}}%
\pgfpathlineto{\pgfqpoint{4.018515in}{3.803565in}}%
\pgfpathlineto{\pgfqpoint{4.010846in}{3.776554in}}%
\pgfpathlineto{\pgfqpoint{4.003176in}{3.749966in}}%
\pgfpathlineto{\pgfqpoint{3.995506in}{3.723795in}}%
\pgfpathlineto{\pgfqpoint{3.987835in}{3.698031in}}%
\pgfpathclose%
\pgfusepath{fill}%
\end{pgfscope}%
\begin{pgfscope}%
\pgfpathrectangle{\pgfqpoint{1.150000in}{0.150000in}}{\pgfqpoint{5.700000in}{5.700000in}}%
\pgfusepath{clip}%
\pgfsetbuttcap%
\pgfsetroundjoin%
\definecolor{currentfill}{rgb}{0.144759,0.519093,0.556572}%
\pgfsetfillcolor{currentfill}%
\pgfsetfillopacity{0.800000}%
\pgfsetlinewidth{0.000000pt}%
\definecolor{currentstroke}{rgb}{0.000000,0.000000,0.000000}%
\pgfsetstrokecolor{currentstroke}%
\pgfsetdash{}{0pt}%
\pgfpathmoveto{\pgfqpoint{4.124790in}{3.678867in}}%
\pgfpathlineto{\pgfqpoint{4.138075in}{3.664360in}}%
\pgfpathlineto{\pgfqpoint{4.151360in}{3.650087in}}%
\pgfpathlineto{\pgfqpoint{4.164647in}{3.636047in}}%
\pgfpathlineto{\pgfqpoint{4.177935in}{3.622239in}}%
\pgfpathlineto{\pgfqpoint{4.185603in}{3.647593in}}%
\pgfpathlineto{\pgfqpoint{4.193271in}{3.673361in}}%
\pgfpathlineto{\pgfqpoint{4.200940in}{3.699550in}}%
\pgfpathlineto{\pgfqpoint{4.208610in}{3.726171in}}%
\pgfpathlineto{\pgfqpoint{4.195324in}{3.740684in}}%
\pgfpathlineto{\pgfqpoint{4.182038in}{3.755430in}}%
\pgfpathlineto{\pgfqpoint{4.168753in}{3.770410in}}%
\pgfpathlineto{\pgfqpoint{4.155469in}{3.785625in}}%
\pgfpathlineto{\pgfqpoint{4.147798in}{3.758285in}}%
\pgfpathlineto{\pgfqpoint{4.140128in}{3.731384in}}%
\pgfpathlineto{\pgfqpoint{4.132459in}{3.704914in}}%
\pgfpathlineto{\pgfqpoint{4.124790in}{3.678867in}}%
\pgfpathclose%
\pgfusepath{fill}%
\end{pgfscope}%
\begin{pgfscope}%
\pgfpathrectangle{\pgfqpoint{1.150000in}{0.150000in}}{\pgfqpoint{5.700000in}{5.700000in}}%
\pgfusepath{clip}%
\pgfsetbuttcap%
\pgfsetroundjoin%
\definecolor{currentfill}{rgb}{0.150476,0.504369,0.557430}%
\pgfsetfillcolor{currentfill}%
\pgfsetfillopacity{0.800000}%
\pgfsetlinewidth{0.000000pt}%
\definecolor{currentstroke}{rgb}{0.000000,0.000000,0.000000}%
\pgfsetstrokecolor{currentstroke}%
\pgfsetdash{}{0pt}%
\pgfpathmoveto{\pgfqpoint{4.040974in}{3.636451in}}%
\pgfpathlineto{\pgfqpoint{4.054258in}{3.621664in}}%
\pgfpathlineto{\pgfqpoint{4.067543in}{3.607118in}}%
\pgfpathlineto{\pgfqpoint{4.080828in}{3.592811in}}%
\pgfpathlineto{\pgfqpoint{4.094114in}{3.578740in}}%
\pgfpathlineto{\pgfqpoint{4.101783in}{3.603179in}}%
\pgfpathlineto{\pgfqpoint{4.109452in}{3.628008in}}%
\pgfpathlineto{\pgfqpoint{4.117121in}{3.653235in}}%
\pgfpathlineto{\pgfqpoint{4.124790in}{3.678867in}}%
\pgfpathlineto{\pgfqpoint{4.111505in}{3.693611in}}%
\pgfpathlineto{\pgfqpoint{4.098221in}{3.708593in}}%
\pgfpathlineto{\pgfqpoint{4.084938in}{3.723813in}}%
\pgfpathlineto{\pgfqpoint{4.071654in}{3.739275in}}%
\pgfpathlineto{\pgfqpoint{4.063984in}{3.712955in}}%
\pgfpathlineto{\pgfqpoint{4.056315in}{3.687050in}}%
\pgfpathlineto{\pgfqpoint{4.048644in}{3.661551in}}%
\pgfpathlineto{\pgfqpoint{4.040974in}{3.636451in}}%
\pgfpathclose%
\pgfusepath{fill}%
\end{pgfscope}%
\begin{pgfscope}%
\pgfpathrectangle{\pgfqpoint{1.150000in}{0.150000in}}{\pgfqpoint{5.700000in}{5.700000in}}%
\pgfusepath{clip}%
\pgfsetbuttcap%
\pgfsetroundjoin%
\definecolor{currentfill}{rgb}{0.128729,0.563265,0.551229}%
\pgfsetfillcolor{currentfill}%
\pgfsetfillopacity{0.800000}%
\pgfsetlinewidth{0.000000pt}%
\definecolor{currentstroke}{rgb}{0.000000,0.000000,0.000000}%
\pgfsetstrokecolor{currentstroke}%
\pgfsetdash{}{0pt}%
\pgfpathmoveto{\pgfqpoint{4.018515in}{3.803565in}}%
\pgfpathlineto{\pgfqpoint{4.031801in}{3.787123in}}%
\pgfpathlineto{\pgfqpoint{4.045086in}{3.770928in}}%
\pgfpathlineto{\pgfqpoint{4.058370in}{3.754979in}}%
\pgfpathlineto{\pgfqpoint{4.071654in}{3.739275in}}%
\pgfpathlineto{\pgfqpoint{4.079323in}{3.766018in}}%
\pgfpathlineto{\pgfqpoint{4.086993in}{3.793193in}}%
\pgfpathlineto{\pgfqpoint{4.094663in}{3.820809in}}%
\pgfpathlineto{\pgfqpoint{4.102333in}{3.848873in}}%
\pgfpathlineto{\pgfqpoint{4.089049in}{3.865290in}}%
\pgfpathlineto{\pgfqpoint{4.075764in}{3.881951in}}%
\pgfpathlineto{\pgfqpoint{4.062478in}{3.898860in}}%
\pgfpathlineto{\pgfqpoint{4.049191in}{3.916018in}}%
\pgfpathlineto{\pgfqpoint{4.041522in}{3.887226in}}%
\pgfpathlineto{\pgfqpoint{4.033853in}{3.858892in}}%
\pgfpathlineto{\pgfqpoint{4.026184in}{3.831008in}}%
\pgfpathlineto{\pgfqpoint{4.018515in}{3.803565in}}%
\pgfpathclose%
\pgfusepath{fill}%
\end{pgfscope}%
\begin{pgfscope}%
\pgfpathrectangle{\pgfqpoint{1.150000in}{0.150000in}}{\pgfqpoint{5.700000in}{5.700000in}}%
\pgfusepath{clip}%
\pgfsetbuttcap%
\pgfsetroundjoin%
\definecolor{currentfill}{rgb}{0.131172,0.555899,0.552459}%
\pgfsetfillcolor{currentfill}%
\pgfsetfillopacity{0.800000}%
\pgfsetlinewidth{0.000000pt}%
\definecolor{currentstroke}{rgb}{0.000000,0.000000,0.000000}%
\pgfsetstrokecolor{currentstroke}%
\pgfsetdash{}{0pt}%
\pgfpathmoveto{\pgfqpoint{4.155469in}{3.785625in}}%
\pgfpathlineto{\pgfqpoint{4.168753in}{3.770410in}}%
\pgfpathlineto{\pgfqpoint{4.182038in}{3.755430in}}%
\pgfpathlineto{\pgfqpoint{4.195324in}{3.740684in}}%
\pgfpathlineto{\pgfqpoint{4.208610in}{3.726171in}}%
\pgfpathlineto{\pgfqpoint{4.216281in}{3.753231in}}%
\pgfpathlineto{\pgfqpoint{4.223954in}{3.780739in}}%
\pgfpathlineto{\pgfqpoint{4.231628in}{3.808705in}}%
\pgfpathlineto{\pgfqpoint{4.239304in}{3.837137in}}%
\pgfpathlineto{\pgfqpoint{4.226018in}{3.852392in}}%
\pgfpathlineto{\pgfqpoint{4.212732in}{3.867880in}}%
\pgfpathlineto{\pgfqpoint{4.199447in}{3.883603in}}%
\pgfpathlineto{\pgfqpoint{4.186163in}{3.899563in}}%
\pgfpathlineto{\pgfqpoint{4.178487in}{3.870374in}}%
\pgfpathlineto{\pgfqpoint{4.170813in}{3.841661in}}%
\pgfpathlineto{\pgfqpoint{4.163140in}{3.813414in}}%
\pgfpathlineto{\pgfqpoint{4.155469in}{3.785625in}}%
\pgfpathclose%
\pgfusepath{fill}%
\end{pgfscope}%
\begin{pgfscope}%
\pgfpathrectangle{\pgfqpoint{1.150000in}{0.150000in}}{\pgfqpoint{5.700000in}{5.700000in}}%
\pgfusepath{clip}%
\pgfsetbuttcap%
\pgfsetroundjoin%
\definecolor{currentfill}{rgb}{0.133743,0.548535,0.553541}%
\pgfsetfillcolor{currentfill}%
\pgfsetfillopacity{0.800000}%
\pgfsetlinewidth{0.000000pt}%
\definecolor{currentstroke}{rgb}{0.000000,0.000000,0.000000}%
\pgfsetstrokecolor{currentstroke}%
\pgfsetdash{}{0pt}%
\pgfpathmoveto{\pgfqpoint{3.934685in}{3.763589in}}%
\pgfpathlineto{\pgfqpoint{3.947974in}{3.746820in}}%
\pgfpathlineto{\pgfqpoint{3.961262in}{3.730305in}}%
\pgfpathlineto{\pgfqpoint{3.974549in}{3.714043in}}%
\pgfpathlineto{\pgfqpoint{3.987835in}{3.698031in}}%
\pgfpathlineto{\pgfqpoint{3.995506in}{3.723795in}}%
\pgfpathlineto{\pgfqpoint{4.003176in}{3.749966in}}%
\pgfpathlineto{\pgfqpoint{4.010846in}{3.776554in}}%
\pgfpathlineto{\pgfqpoint{4.018515in}{3.803565in}}%
\pgfpathlineto{\pgfqpoint{4.005229in}{3.820257in}}%
\pgfpathlineto{\pgfqpoint{3.991942in}{3.837200in}}%
\pgfpathlineto{\pgfqpoint{3.978653in}{3.854396in}}%
\pgfpathlineto{\pgfqpoint{3.965363in}{3.871848in}}%
\pgfpathlineto{\pgfqpoint{3.957695in}{3.844142in}}%
\pgfpathlineto{\pgfqpoint{3.950026in}{3.816869in}}%
\pgfpathlineto{\pgfqpoint{3.942356in}{3.790021in}}%
\pgfpathlineto{\pgfqpoint{3.934685in}{3.763589in}}%
\pgfpathclose%
\pgfusepath{fill}%
\end{pgfscope}%
\begin{pgfscope}%
\pgfpathrectangle{\pgfqpoint{1.150000in}{0.150000in}}{\pgfqpoint{5.700000in}{5.700000in}}%
\pgfusepath{clip}%
\pgfsetbuttcap%
\pgfsetroundjoin%
\definecolor{currentfill}{rgb}{0.124395,0.578002,0.548287}%
\pgfsetfillcolor{currentfill}%
\pgfsetfillopacity{0.800000}%
\pgfsetlinewidth{0.000000pt}%
\definecolor{currentstroke}{rgb}{0.000000,0.000000,0.000000}%
\pgfsetstrokecolor{currentstroke}%
\pgfsetdash{}{0pt}%
\pgfpathmoveto{\pgfqpoint{4.102333in}{3.848873in}}%
\pgfpathlineto{\pgfqpoint{4.115617in}{3.832700in}}%
\pgfpathlineto{\pgfqpoint{4.128901in}{3.816769in}}%
\pgfpathlineto{\pgfqpoint{4.142185in}{3.801078in}}%
\pgfpathlineto{\pgfqpoint{4.155469in}{3.785625in}}%
\pgfpathlineto{\pgfqpoint{4.163140in}{3.813414in}}%
\pgfpathlineto{\pgfqpoint{4.170813in}{3.841661in}}%
\pgfpathlineto{\pgfqpoint{4.178487in}{3.870374in}}%
\pgfpathlineto{\pgfqpoint{4.186163in}{3.899563in}}%
\pgfpathlineto{\pgfqpoint{4.172878in}{3.915760in}}%
\pgfpathlineto{\pgfqpoint{4.159593in}{3.932197in}}%
\pgfpathlineto{\pgfqpoint{4.146309in}{3.948876in}}%
\pgfpathlineto{\pgfqpoint{4.133023in}{3.965797in}}%
\pgfpathlineto{\pgfqpoint{4.125349in}{3.935848in}}%
\pgfpathlineto{\pgfqpoint{4.117676in}{3.906383in}}%
\pgfpathlineto{\pgfqpoint{4.110004in}{3.877395in}}%
\pgfpathlineto{\pgfqpoint{4.102333in}{3.848873in}}%
\pgfpathclose%
\pgfusepath{fill}%
\end{pgfscope}%
\begin{pgfscope}%
\pgfpathrectangle{\pgfqpoint{1.150000in}{0.150000in}}{\pgfqpoint{5.700000in}{5.700000in}}%
\pgfusepath{clip}%
\pgfsetbuttcap%
\pgfsetroundjoin%
\definecolor{currentfill}{rgb}{0.157729,0.485932,0.558013}%
\pgfsetfillcolor{currentfill}%
\pgfsetfillopacity{0.800000}%
\pgfsetlinewidth{0.000000pt}%
\definecolor{currentstroke}{rgb}{0.000000,0.000000,0.000000}%
\pgfsetstrokecolor{currentstroke}%
\pgfsetdash{}{0pt}%
\pgfpathmoveto{\pgfqpoint{4.094114in}{3.578740in}}%
\pgfpathlineto{\pgfqpoint{4.107400in}{3.564905in}}%
\pgfpathlineto{\pgfqpoint{4.120687in}{3.551303in}}%
\pgfpathlineto{\pgfqpoint{4.133976in}{3.537934in}}%
\pgfpathlineto{\pgfqpoint{4.147265in}{3.524797in}}%
\pgfpathlineto{\pgfqpoint{4.154933in}{3.548578in}}%
\pgfpathlineto{\pgfqpoint{4.162600in}{3.572740in}}%
\pgfpathlineto{\pgfqpoint{4.170267in}{3.597291in}}%
\pgfpathlineto{\pgfqpoint{4.177935in}{3.622239in}}%
\pgfpathlineto{\pgfqpoint{4.164647in}{3.636047in}}%
\pgfpathlineto{\pgfqpoint{4.151360in}{3.650087in}}%
\pgfpathlineto{\pgfqpoint{4.138075in}{3.664360in}}%
\pgfpathlineto{\pgfqpoint{4.124790in}{3.678867in}}%
\pgfpathlineto{\pgfqpoint{4.117121in}{3.653235in}}%
\pgfpathlineto{\pgfqpoint{4.109452in}{3.628008in}}%
\pgfpathlineto{\pgfqpoint{4.101783in}{3.603179in}}%
\pgfpathlineto{\pgfqpoint{4.094114in}{3.578740in}}%
\pgfpathclose%
\pgfusepath{fill}%
\end{pgfscope}%
\begin{pgfscope}%
\pgfpathrectangle{\pgfqpoint{1.150000in}{0.150000in}}{\pgfqpoint{5.700000in}{5.700000in}}%
\pgfusepath{clip}%
\pgfsetbuttcap%
\pgfsetroundjoin%
\definecolor{currentfill}{rgb}{0.137770,0.537492,0.554906}%
\pgfsetfillcolor{currentfill}%
\pgfsetfillopacity{0.800000}%
\pgfsetlinewidth{0.000000pt}%
\definecolor{currentstroke}{rgb}{0.000000,0.000000,0.000000}%
\pgfsetstrokecolor{currentstroke}%
\pgfsetdash{}{0pt}%
\pgfpathmoveto{\pgfqpoint{4.208610in}{3.726171in}}%
\pgfpathlineto{\pgfqpoint{4.221897in}{3.711888in}}%
\pgfpathlineto{\pgfqpoint{4.235186in}{3.697836in}}%
\pgfpathlineto{\pgfqpoint{4.248476in}{3.684011in}}%
\pgfpathlineto{\pgfqpoint{4.261768in}{3.670413in}}%
\pgfpathlineto{\pgfqpoint{4.269437in}{3.696747in}}%
\pgfpathlineto{\pgfqpoint{4.277109in}{3.723521in}}%
\pgfpathlineto{\pgfqpoint{4.284782in}{3.750742in}}%
\pgfpathlineto{\pgfqpoint{4.292457in}{3.778422in}}%
\pgfpathlineto{\pgfqpoint{4.279167in}{3.792758in}}%
\pgfpathlineto{\pgfqpoint{4.265878in}{3.807321in}}%
\pgfpathlineto{\pgfqpoint{4.252590in}{3.822114in}}%
\pgfpathlineto{\pgfqpoint{4.239304in}{3.837137in}}%
\pgfpathlineto{\pgfqpoint{4.231628in}{3.808705in}}%
\pgfpathlineto{\pgfqpoint{4.223954in}{3.780739in}}%
\pgfpathlineto{\pgfqpoint{4.216281in}{3.753231in}}%
\pgfpathlineto{\pgfqpoint{4.208610in}{3.726171in}}%
\pgfpathclose%
\pgfusepath{fill}%
\end{pgfscope}%
\begin{pgfscope}%
\pgfpathrectangle{\pgfqpoint{1.150000in}{0.150000in}}{\pgfqpoint{5.700000in}{5.700000in}}%
\pgfusepath{clip}%
\pgfsetbuttcap%
\pgfsetroundjoin%
\definecolor{currentfill}{rgb}{0.154815,0.493313,0.557840}%
\pgfsetfillcolor{currentfill}%
\pgfsetfillopacity{0.800000}%
\pgfsetlinewidth{0.000000pt}%
\definecolor{currentstroke}{rgb}{0.000000,0.000000,0.000000}%
\pgfsetstrokecolor{currentstroke}%
\pgfsetdash{}{0pt}%
\pgfpathmoveto{\pgfqpoint{3.957139in}{3.598891in}}%
\pgfpathlineto{\pgfqpoint{3.970425in}{3.583770in}}%
\pgfpathlineto{\pgfqpoint{3.983711in}{3.568896in}}%
\pgfpathlineto{\pgfqpoint{3.996996in}{3.554266in}}%
\pgfpathlineto{\pgfqpoint{4.010282in}{3.539879in}}%
\pgfpathlineto{\pgfqpoint{4.017957in}{3.563463in}}%
\pgfpathlineto{\pgfqpoint{4.025630in}{3.587415in}}%
\pgfpathlineto{\pgfqpoint{4.033302in}{3.611741in}}%
\pgfpathlineto{\pgfqpoint{4.040974in}{3.636451in}}%
\pgfpathlineto{\pgfqpoint{4.027689in}{3.651479in}}%
\pgfpathlineto{\pgfqpoint{4.014405in}{3.666750in}}%
\pgfpathlineto{\pgfqpoint{4.001120in}{3.682267in}}%
\pgfpathlineto{\pgfqpoint{3.987835in}{3.698031in}}%
\pgfpathlineto{\pgfqpoint{3.980163in}{3.672666in}}%
\pgfpathlineto{\pgfqpoint{3.972489in}{3.647693in}}%
\pgfpathlineto{\pgfqpoint{3.964815in}{3.623104in}}%
\pgfpathlineto{\pgfqpoint{3.957139in}{3.598891in}}%
\pgfpathclose%
\pgfusepath{fill}%
\end{pgfscope}%
\begin{pgfscope}%
\pgfpathrectangle{\pgfqpoint{1.150000in}{0.150000in}}{\pgfqpoint{5.700000in}{5.700000in}}%
\pgfusepath{clip}%
\pgfsetbuttcap%
\pgfsetroundjoin%
\definecolor{currentfill}{rgb}{0.151918,0.500685,0.557587}%
\pgfsetfillcolor{currentfill}%
\pgfsetfillopacity{0.800000}%
\pgfsetlinewidth{0.000000pt}%
\definecolor{currentstroke}{rgb}{0.000000,0.000000,0.000000}%
\pgfsetstrokecolor{currentstroke}%
\pgfsetdash{}{0pt}%
\pgfpathmoveto{\pgfqpoint{4.177935in}{3.622239in}}%
\pgfpathlineto{\pgfqpoint{4.191224in}{3.608661in}}%
\pgfpathlineto{\pgfqpoint{4.204514in}{3.595312in}}%
\pgfpathlineto{\pgfqpoint{4.217806in}{3.582191in}}%
\pgfpathlineto{\pgfqpoint{4.231100in}{3.569295in}}%
\pgfpathlineto{\pgfqpoint{4.238765in}{3.593958in}}%
\pgfpathlineto{\pgfqpoint{4.246432in}{3.619027in}}%
\pgfpathlineto{\pgfqpoint{4.254099in}{3.644509in}}%
\pgfpathlineto{\pgfqpoint{4.261768in}{3.670413in}}%
\pgfpathlineto{\pgfqpoint{4.248476in}{3.684011in}}%
\pgfpathlineto{\pgfqpoint{4.235186in}{3.697836in}}%
\pgfpathlineto{\pgfqpoint{4.221897in}{3.711888in}}%
\pgfpathlineto{\pgfqpoint{4.208610in}{3.726171in}}%
\pgfpathlineto{\pgfqpoint{4.200940in}{3.699550in}}%
\pgfpathlineto{\pgfqpoint{4.193271in}{3.673361in}}%
\pgfpathlineto{\pgfqpoint{4.185603in}{3.647593in}}%
\pgfpathlineto{\pgfqpoint{4.177935in}{3.622239in}}%
\pgfpathclose%
\pgfusepath{fill}%
\end{pgfscope}%
\begin{pgfscope}%
\pgfpathrectangle{\pgfqpoint{1.150000in}{0.150000in}}{\pgfqpoint{5.700000in}{5.700000in}}%
\pgfusepath{clip}%
\pgfsetbuttcap%
\pgfsetroundjoin%
\definecolor{currentfill}{rgb}{0.147607,0.511733,0.557049}%
\pgfsetfillcolor{currentfill}%
\pgfsetfillopacity{0.800000}%
\pgfsetlinewidth{0.000000pt}%
\definecolor{currentstroke}{rgb}{0.000000,0.000000,0.000000}%
\pgfsetstrokecolor{currentstroke}%
\pgfsetdash{}{0pt}%
\pgfpathmoveto{\pgfqpoint{3.903988in}{3.661868in}}%
\pgfpathlineto{\pgfqpoint{3.917277in}{3.645746in}}%
\pgfpathlineto{\pgfqpoint{3.930565in}{3.629877in}}%
\pgfpathlineto{\pgfqpoint{3.943853in}{3.614259in}}%
\pgfpathlineto{\pgfqpoint{3.957139in}{3.598891in}}%
\pgfpathlineto{\pgfqpoint{3.964815in}{3.623104in}}%
\pgfpathlineto{\pgfqpoint{3.972489in}{3.647693in}}%
\pgfpathlineto{\pgfqpoint{3.980163in}{3.672666in}}%
\pgfpathlineto{\pgfqpoint{3.987835in}{3.698031in}}%
\pgfpathlineto{\pgfqpoint{3.974549in}{3.714043in}}%
\pgfpathlineto{\pgfqpoint{3.961262in}{3.730305in}}%
\pgfpathlineto{\pgfqpoint{3.947974in}{3.746820in}}%
\pgfpathlineto{\pgfqpoint{3.934685in}{3.763589in}}%
\pgfpathlineto{\pgfqpoint{3.927013in}{3.737566in}}%
\pgfpathlineto{\pgfqpoint{3.919339in}{3.711943in}}%
\pgfpathlineto{\pgfqpoint{3.911665in}{3.686713in}}%
\pgfpathlineto{\pgfqpoint{3.903988in}{3.661868in}}%
\pgfpathclose%
\pgfusepath{fill}%
\end{pgfscope}%
\begin{pgfscope}%
\pgfpathrectangle{\pgfqpoint{1.150000in}{0.150000in}}{\pgfqpoint{5.700000in}{5.700000in}}%
\pgfusepath{clip}%
\pgfsetbuttcap%
\pgfsetroundjoin%
\definecolor{currentfill}{rgb}{0.122606,0.585371,0.546557}%
\pgfsetfillcolor{currentfill}%
\pgfsetfillopacity{0.800000}%
\pgfsetlinewidth{0.000000pt}%
\definecolor{currentstroke}{rgb}{0.000000,0.000000,0.000000}%
\pgfsetstrokecolor{currentstroke}%
\pgfsetdash{}{0pt}%
\pgfpathmoveto{\pgfqpoint{3.965363in}{3.871848in}}%
\pgfpathlineto{\pgfqpoint{3.978653in}{3.854396in}}%
\pgfpathlineto{\pgfqpoint{3.991942in}{3.837200in}}%
\pgfpathlineto{\pgfqpoint{4.005229in}{3.820257in}}%
\pgfpathlineto{\pgfqpoint{4.018515in}{3.803565in}}%
\pgfpathlineto{\pgfqpoint{4.026184in}{3.831008in}}%
\pgfpathlineto{\pgfqpoint{4.033853in}{3.858892in}}%
\pgfpathlineto{\pgfqpoint{4.041522in}{3.887226in}}%
\pgfpathlineto{\pgfqpoint{4.049191in}{3.916018in}}%
\pgfpathlineto{\pgfqpoint{4.035904in}{3.933426in}}%
\pgfpathlineto{\pgfqpoint{4.022615in}{3.951086in}}%
\pgfpathlineto{\pgfqpoint{4.009325in}{3.969001in}}%
\pgfpathlineto{\pgfqpoint{3.996033in}{3.987172in}}%
\pgfpathlineto{\pgfqpoint{3.988365in}{3.957648in}}%
\pgfpathlineto{\pgfqpoint{3.980698in}{3.928592in}}%
\pgfpathlineto{\pgfqpoint{3.973031in}{3.899995in}}%
\pgfpathlineto{\pgfqpoint{3.965363in}{3.871848in}}%
\pgfpathclose%
\pgfusepath{fill}%
\end{pgfscope}%
\begin{pgfscope}%
\pgfpathrectangle{\pgfqpoint{1.150000in}{0.150000in}}{\pgfqpoint{5.700000in}{5.700000in}}%
\pgfusepath{clip}%
\pgfsetbuttcap%
\pgfsetroundjoin%
\definecolor{currentfill}{rgb}{0.163625,0.471133,0.558148}%
\pgfsetfillcolor{currentfill}%
\pgfsetfillopacity{0.800000}%
\pgfsetlinewidth{0.000000pt}%
\definecolor{currentstroke}{rgb}{0.000000,0.000000,0.000000}%
\pgfsetstrokecolor{currentstroke}%
\pgfsetdash{}{0pt}%
\pgfpathmoveto{\pgfqpoint{4.010282in}{3.539879in}}%
\pgfpathlineto{\pgfqpoint{4.023568in}{3.525733in}}%
\pgfpathlineto{\pgfqpoint{4.036854in}{3.511826in}}%
\pgfpathlineto{\pgfqpoint{4.050141in}{3.498157in}}%
\pgfpathlineto{\pgfqpoint{4.063429in}{3.484725in}}%
\pgfpathlineto{\pgfqpoint{4.071101in}{3.507683in}}%
\pgfpathlineto{\pgfqpoint{4.078773in}{3.530999in}}%
\pgfpathlineto{\pgfqpoint{4.086444in}{3.554683in}}%
\pgfpathlineto{\pgfqpoint{4.094114in}{3.578740in}}%
\pgfpathlineto{\pgfqpoint{4.080828in}{3.592811in}}%
\pgfpathlineto{\pgfqpoint{4.067543in}{3.607118in}}%
\pgfpathlineto{\pgfqpoint{4.054258in}{3.621664in}}%
\pgfpathlineto{\pgfqpoint{4.040974in}{3.636451in}}%
\pgfpathlineto{\pgfqpoint{4.033302in}{3.611741in}}%
\pgfpathlineto{\pgfqpoint{4.025630in}{3.587415in}}%
\pgfpathlineto{\pgfqpoint{4.017957in}{3.563463in}}%
\pgfpathlineto{\pgfqpoint{4.010282in}{3.539879in}}%
\pgfpathclose%
\pgfusepath{fill}%
\end{pgfscope}%
\begin{pgfscope}%
\pgfpathrectangle{\pgfqpoint{1.150000in}{0.150000in}}{\pgfqpoint{5.700000in}{5.700000in}}%
\pgfusepath{clip}%
\pgfsetbuttcap%
\pgfsetroundjoin%
\definecolor{currentfill}{rgb}{0.119738,0.603785,0.541400}%
\pgfsetfillcolor{currentfill}%
\pgfsetfillopacity{0.800000}%
\pgfsetlinewidth{0.000000pt}%
\definecolor{currentstroke}{rgb}{0.000000,0.000000,0.000000}%
\pgfsetstrokecolor{currentstroke}%
\pgfsetdash{}{0pt}%
\pgfpathmoveto{\pgfqpoint{4.049191in}{3.916018in}}%
\pgfpathlineto{\pgfqpoint{4.062478in}{3.898860in}}%
\pgfpathlineto{\pgfqpoint{4.075764in}{3.881951in}}%
\pgfpathlineto{\pgfqpoint{4.089049in}{3.865290in}}%
\pgfpathlineto{\pgfqpoint{4.102333in}{3.848873in}}%
\pgfpathlineto{\pgfqpoint{4.110004in}{3.877395in}}%
\pgfpathlineto{\pgfqpoint{4.117676in}{3.906383in}}%
\pgfpathlineto{\pgfqpoint{4.125349in}{3.935848in}}%
\pgfpathlineto{\pgfqpoint{4.133023in}{3.965797in}}%
\pgfpathlineto{\pgfqpoint{4.119738in}{3.982963in}}%
\pgfpathlineto{\pgfqpoint{4.106451in}{4.000375in}}%
\pgfpathlineto{\pgfqpoint{4.093164in}{4.018035in}}%
\pgfpathlineto{\pgfqpoint{4.079875in}{4.035946in}}%
\pgfpathlineto{\pgfqpoint{4.072203in}{4.005231in}}%
\pgfpathlineto{\pgfqpoint{4.064531in}{3.975011in}}%
\pgfpathlineto{\pgfqpoint{4.056861in}{3.945277in}}%
\pgfpathlineto{\pgfqpoint{4.049191in}{3.916018in}}%
\pgfpathclose%
\pgfusepath{fill}%
\end{pgfscope}%
\begin{pgfscope}%
\pgfpathrectangle{\pgfqpoint{1.150000in}{0.150000in}}{\pgfqpoint{5.700000in}{5.700000in}}%
\pgfusepath{clip}%
\pgfsetbuttcap%
\pgfsetroundjoin%
\definecolor{currentfill}{rgb}{0.126453,0.570633,0.549841}%
\pgfsetfillcolor{currentfill}%
\pgfsetfillopacity{0.800000}%
\pgfsetlinewidth{0.000000pt}%
\definecolor{currentstroke}{rgb}{0.000000,0.000000,0.000000}%
\pgfsetstrokecolor{currentstroke}%
\pgfsetdash{}{0pt}%
\pgfpathmoveto{\pgfqpoint{4.239304in}{3.837137in}}%
\pgfpathlineto{\pgfqpoint{4.252590in}{3.822114in}}%
\pgfpathlineto{\pgfqpoint{4.265878in}{3.807321in}}%
\pgfpathlineto{\pgfqpoint{4.279167in}{3.792758in}}%
\pgfpathlineto{\pgfqpoint{4.292457in}{3.778422in}}%
\pgfpathlineto{\pgfqpoint{4.300135in}{3.806567in}}%
\pgfpathlineto{\pgfqpoint{4.307815in}{3.835188in}}%
\pgfpathlineto{\pgfqpoint{4.315497in}{3.864295in}}%
\pgfpathlineto{\pgfqpoint{4.302207in}{3.879208in}}%
\pgfpathlineto{\pgfqpoint{4.288919in}{3.894350in}}%
\pgfpathlineto{\pgfqpoint{4.275631in}{3.909722in}}%
\pgfpathlineto{\pgfqpoint{4.262344in}{3.925324in}}%
\pgfpathlineto{\pgfqpoint{4.254662in}{3.895437in}}%
\pgfpathlineto{\pgfqpoint{4.246981in}{3.866045in}}%
\pgfpathlineto{\pgfqpoint{4.239304in}{3.837137in}}%
\pgfpathclose%
\pgfusepath{fill}%
\end{pgfscope}%
\begin{pgfscope}%
\pgfpathrectangle{\pgfqpoint{1.150000in}{0.150000in}}{\pgfqpoint{5.700000in}{5.700000in}}%
\pgfusepath{clip}%
\pgfsetbuttcap%
\pgfsetroundjoin%
\definecolor{currentfill}{rgb}{0.121148,0.592739,0.544641}%
\pgfsetfillcolor{currentfill}%
\pgfsetfillopacity{0.800000}%
\pgfsetlinewidth{0.000000pt}%
\definecolor{currentstroke}{rgb}{0.000000,0.000000,0.000000}%
\pgfsetstrokecolor{currentstroke}%
\pgfsetdash{}{0pt}%
\pgfpathmoveto{\pgfqpoint{4.186163in}{3.899563in}}%
\pgfpathlineto{\pgfqpoint{4.199447in}{3.883603in}}%
\pgfpathlineto{\pgfqpoint{4.212732in}{3.867880in}}%
\pgfpathlineto{\pgfqpoint{4.226018in}{3.852392in}}%
\pgfpathlineto{\pgfqpoint{4.239304in}{3.837137in}}%
\pgfpathlineto{\pgfqpoint{4.246981in}{3.866045in}}%
\pgfpathlineto{\pgfqpoint{4.254662in}{3.895437in}}%
\pgfpathlineto{\pgfqpoint{4.262344in}{3.925324in}}%
\pgfpathlineto{\pgfqpoint{4.249058in}{3.941159in}}%
\pgfpathlineto{\pgfqpoint{4.235772in}{3.957228in}}%
\pgfpathlineto{\pgfqpoint{4.222486in}{3.973533in}}%
\pgfpathlineto{\pgfqpoint{4.209201in}{3.990076in}}%
\pgfpathlineto{\pgfqpoint{4.201519in}{3.959404in}}%
\pgfpathlineto{\pgfqpoint{4.193840in}{3.929236in}}%
\pgfpathlineto{\pgfqpoint{4.186163in}{3.899563in}}%
\pgfpathclose%
\pgfusepath{fill}%
\end{pgfscope}%
\begin{pgfscope}%
\pgfpathrectangle{\pgfqpoint{1.150000in}{0.150000in}}{\pgfqpoint{5.700000in}{5.700000in}}%
\pgfusepath{clip}%
\pgfsetbuttcap%
\pgfsetroundjoin%
\definecolor{currentfill}{rgb}{0.137770,0.537492,0.554906}%
\pgfsetfillcolor{currentfill}%
\pgfsetfillopacity{0.800000}%
\pgfsetlinewidth{0.000000pt}%
\definecolor{currentstroke}{rgb}{0.000000,0.000000,0.000000}%
\pgfsetstrokecolor{currentstroke}%
\pgfsetdash{}{0pt}%
\pgfpathmoveto{\pgfqpoint{3.850817in}{3.728927in}}%
\pgfpathlineto{\pgfqpoint{3.864112in}{3.711773in}}%
\pgfpathlineto{\pgfqpoint{3.877406in}{3.694880in}}%
\pgfpathlineto{\pgfqpoint{3.890698in}{3.678245in}}%
\pgfpathlineto{\pgfqpoint{3.903988in}{3.661868in}}%
\pgfpathlineto{\pgfqpoint{3.911665in}{3.686713in}}%
\pgfpathlineto{\pgfqpoint{3.919339in}{3.711943in}}%
\pgfpathlineto{\pgfqpoint{3.927013in}{3.737566in}}%
\pgfpathlineto{\pgfqpoint{3.934685in}{3.763589in}}%
\pgfpathlineto{\pgfqpoint{3.921394in}{3.780614in}}%
\pgfpathlineto{\pgfqpoint{3.908102in}{3.797896in}}%
\pgfpathlineto{\pgfqpoint{3.894808in}{3.815439in}}%
\pgfpathlineto{\pgfqpoint{3.881512in}{3.833243in}}%
\pgfpathlineto{\pgfqpoint{3.873841in}{3.806557in}}%
\pgfpathlineto{\pgfqpoint{3.866168in}{3.780281in}}%
\pgfpathlineto{\pgfqpoint{3.858493in}{3.754407in}}%
\pgfpathlineto{\pgfqpoint{3.850817in}{3.728927in}}%
\pgfpathclose%
\pgfusepath{fill}%
\end{pgfscope}%
\begin{pgfscope}%
\pgfpathrectangle{\pgfqpoint{1.150000in}{0.150000in}}{\pgfqpoint{5.700000in}{5.700000in}}%
\pgfusepath{clip}%
\pgfsetbuttcap%
\pgfsetroundjoin%
\definecolor{currentfill}{rgb}{0.125394,0.574318,0.549086}%
\pgfsetfillcolor{currentfill}%
\pgfsetfillopacity{0.800000}%
\pgfsetlinewidth{0.000000pt}%
\definecolor{currentstroke}{rgb}{0.000000,0.000000,0.000000}%
\pgfsetstrokecolor{currentstroke}%
\pgfsetdash{}{0pt}%
\pgfpathmoveto{\pgfqpoint{3.881512in}{3.833243in}}%
\pgfpathlineto{\pgfqpoint{3.894808in}{3.815439in}}%
\pgfpathlineto{\pgfqpoint{3.908102in}{3.797896in}}%
\pgfpathlineto{\pgfqpoint{3.921394in}{3.780614in}}%
\pgfpathlineto{\pgfqpoint{3.934685in}{3.763589in}}%
\pgfpathlineto{\pgfqpoint{3.942356in}{3.790021in}}%
\pgfpathlineto{\pgfqpoint{3.950026in}{3.816869in}}%
\pgfpathlineto{\pgfqpoint{3.957695in}{3.844142in}}%
\pgfpathlineto{\pgfqpoint{3.965363in}{3.871848in}}%
\pgfpathlineto{\pgfqpoint{3.952071in}{3.889556in}}%
\pgfpathlineto{\pgfqpoint{3.938778in}{3.907523in}}%
\pgfpathlineto{\pgfqpoint{3.925482in}{3.925751in}}%
\pgfpathlineto{\pgfqpoint{3.912184in}{3.944242in}}%
\pgfpathlineto{\pgfqpoint{3.904518in}{3.915837in}}%
\pgfpathlineto{\pgfqpoint{3.896850in}{3.887874in}}%
\pgfpathlineto{\pgfqpoint{3.889182in}{3.860346in}}%
\pgfpathlineto{\pgfqpoint{3.881512in}{3.833243in}}%
\pgfpathclose%
\pgfusepath{fill}%
\end{pgfscope}%
\begin{pgfscope}%
\pgfpathrectangle{\pgfqpoint{1.150000in}{0.150000in}}{\pgfqpoint{5.700000in}{5.700000in}}%
\pgfusepath{clip}%
\pgfsetbuttcap%
\pgfsetroundjoin%
\definecolor{currentfill}{rgb}{0.144759,0.519093,0.556572}%
\pgfsetfillcolor{currentfill}%
\pgfsetfillopacity{0.800000}%
\pgfsetlinewidth{0.000000pt}%
\definecolor{currentstroke}{rgb}{0.000000,0.000000,0.000000}%
\pgfsetstrokecolor{currentstroke}%
\pgfsetdash{}{0pt}%
\pgfpathmoveto{\pgfqpoint{4.261768in}{3.670413in}}%
\pgfpathlineto{\pgfqpoint{4.275061in}{3.657040in}}%
\pgfpathlineto{\pgfqpoint{4.288356in}{3.643891in}}%
\pgfpathlineto{\pgfqpoint{4.301653in}{3.630965in}}%
\pgfpathlineto{\pgfqpoint{4.314952in}{3.618260in}}%
\pgfpathlineto{\pgfqpoint{4.322620in}{3.643872in}}%
\pgfpathlineto{\pgfqpoint{4.330289in}{3.669914in}}%
\pgfpathlineto{\pgfqpoint{4.337961in}{3.696395in}}%
\pgfpathlineto{\pgfqpoint{4.345634in}{3.723325in}}%
\pgfpathlineto{\pgfqpoint{4.332337in}{3.736765in}}%
\pgfpathlineto{\pgfqpoint{4.319042in}{3.750426in}}%
\pgfpathlineto{\pgfqpoint{4.305749in}{3.764312in}}%
\pgfpathlineto{\pgfqpoint{4.292457in}{3.778422in}}%
\pgfpathlineto{\pgfqpoint{4.284782in}{3.750742in}}%
\pgfpathlineto{\pgfqpoint{4.277109in}{3.723521in}}%
\pgfpathlineto{\pgfqpoint{4.269437in}{3.696747in}}%
\pgfpathlineto{\pgfqpoint{4.261768in}{3.670413in}}%
\pgfpathclose%
\pgfusepath{fill}%
\end{pgfscope}%
\begin{pgfscope}%
\pgfpathrectangle{\pgfqpoint{1.150000in}{0.150000in}}{\pgfqpoint{5.700000in}{5.700000in}}%
\pgfusepath{clip}%
\pgfsetbuttcap%
\pgfsetroundjoin%
\definecolor{currentfill}{rgb}{0.165117,0.467423,0.558141}%
\pgfsetfillcolor{currentfill}%
\pgfsetfillopacity{0.800000}%
\pgfsetlinewidth{0.000000pt}%
\definecolor{currentstroke}{rgb}{0.000000,0.000000,0.000000}%
\pgfsetstrokecolor{currentstroke}%
\pgfsetdash{}{0pt}%
\pgfpathmoveto{\pgfqpoint{4.147265in}{3.524797in}}%
\pgfpathlineto{\pgfqpoint{4.160557in}{3.511888in}}%
\pgfpathlineto{\pgfqpoint{4.173849in}{3.499208in}}%
\pgfpathlineto{\pgfqpoint{4.187144in}{3.486754in}}%
\pgfpathlineto{\pgfqpoint{4.200441in}{3.474526in}}%
\pgfpathlineto{\pgfqpoint{4.208105in}{3.497652in}}%
\pgfpathlineto{\pgfqpoint{4.215770in}{3.521150in}}%
\pgfpathlineto{\pgfqpoint{4.223435in}{3.545028in}}%
\pgfpathlineto{\pgfqpoint{4.231100in}{3.569295in}}%
\pgfpathlineto{\pgfqpoint{4.217806in}{3.582191in}}%
\pgfpathlineto{\pgfqpoint{4.204514in}{3.595312in}}%
\pgfpathlineto{\pgfqpoint{4.191224in}{3.608661in}}%
\pgfpathlineto{\pgfqpoint{4.177935in}{3.622239in}}%
\pgfpathlineto{\pgfqpoint{4.170267in}{3.597291in}}%
\pgfpathlineto{\pgfqpoint{4.162600in}{3.572740in}}%
\pgfpathlineto{\pgfqpoint{4.154933in}{3.548578in}}%
\pgfpathlineto{\pgfqpoint{4.147265in}{3.524797in}}%
\pgfpathclose%
\pgfusepath{fill}%
\end{pgfscope}%
\begin{pgfscope}%
\pgfpathrectangle{\pgfqpoint{1.150000in}{0.150000in}}{\pgfqpoint{5.700000in}{5.700000in}}%
\pgfusepath{clip}%
\pgfsetbuttcap%
\pgfsetroundjoin%
\definecolor{currentfill}{rgb}{0.132444,0.552216,0.553018}%
\pgfsetfillcolor{currentfill}%
\pgfsetfillopacity{0.800000}%
\pgfsetlinewidth{0.000000pt}%
\definecolor{currentstroke}{rgb}{0.000000,0.000000,0.000000}%
\pgfsetstrokecolor{currentstroke}%
\pgfsetdash{}{0pt}%
\pgfpathmoveto{\pgfqpoint{4.292457in}{3.778422in}}%
\pgfpathlineto{\pgfqpoint{4.305749in}{3.764312in}}%
\pgfpathlineto{\pgfqpoint{4.319042in}{3.750426in}}%
\pgfpathlineto{\pgfqpoint{4.332337in}{3.736765in}}%
\pgfpathlineto{\pgfqpoint{4.345634in}{3.723325in}}%
\pgfpathlineto{\pgfqpoint{4.353310in}{3.750712in}}%
\pgfpathlineto{\pgfqpoint{4.360989in}{3.778566in}}%
\pgfpathlineto{\pgfqpoint{4.368670in}{3.806895in}}%
\pgfpathlineto{\pgfqpoint{4.355375in}{3.820910in}}%
\pgfpathlineto{\pgfqpoint{4.342081in}{3.835147in}}%
\pgfpathlineto{\pgfqpoint{4.328788in}{3.849608in}}%
\pgfpathlineto{\pgfqpoint{4.315497in}{3.864295in}}%
\pgfpathlineto{\pgfqpoint{4.307815in}{3.835188in}}%
\pgfpathlineto{\pgfqpoint{4.300135in}{3.806567in}}%
\pgfpathlineto{\pgfqpoint{4.292457in}{3.778422in}}%
\pgfpathclose%
\pgfusepath{fill}%
\end{pgfscope}%
\begin{pgfscope}%
\pgfpathrectangle{\pgfqpoint{1.150000in}{0.150000in}}{\pgfqpoint{5.700000in}{5.700000in}}%
\pgfusepath{clip}%
\pgfsetbuttcap%
\pgfsetroundjoin%
\definecolor{currentfill}{rgb}{0.157729,0.485932,0.558013}%
\pgfsetfillcolor{currentfill}%
\pgfsetfillopacity{0.800000}%
\pgfsetlinewidth{0.000000pt}%
\definecolor{currentstroke}{rgb}{0.000000,0.000000,0.000000}%
\pgfsetstrokecolor{currentstroke}%
\pgfsetdash{}{0pt}%
\pgfpathmoveto{\pgfqpoint{4.231100in}{3.569295in}}%
\pgfpathlineto{\pgfqpoint{4.244395in}{3.556624in}}%
\pgfpathlineto{\pgfqpoint{4.257693in}{3.544176in}}%
\pgfpathlineto{\pgfqpoint{4.270993in}{3.531951in}}%
\pgfpathlineto{\pgfqpoint{4.284295in}{3.519946in}}%
\pgfpathlineto{\pgfqpoint{4.291958in}{3.543921in}}%
\pgfpathlineto{\pgfqpoint{4.299622in}{3.568294in}}%
\pgfpathlineto{\pgfqpoint{4.307286in}{3.593070in}}%
\pgfpathlineto{\pgfqpoint{4.314952in}{3.618260in}}%
\pgfpathlineto{\pgfqpoint{4.301653in}{3.630965in}}%
\pgfpathlineto{\pgfqpoint{4.288356in}{3.643891in}}%
\pgfpathlineto{\pgfqpoint{4.275061in}{3.657040in}}%
\pgfpathlineto{\pgfqpoint{4.261768in}{3.670413in}}%
\pgfpathlineto{\pgfqpoint{4.254099in}{3.644509in}}%
\pgfpathlineto{\pgfqpoint{4.246432in}{3.619027in}}%
\pgfpathlineto{\pgfqpoint{4.238765in}{3.593958in}}%
\pgfpathlineto{\pgfqpoint{4.231100in}{3.569295in}}%
\pgfpathclose%
\pgfusepath{fill}%
\end{pgfscope}%
\begin{pgfscope}%
\pgfpathrectangle{\pgfqpoint{1.150000in}{0.150000in}}{\pgfqpoint{5.700000in}{5.700000in}}%
\pgfusepath{clip}%
\pgfsetbuttcap%
\pgfsetroundjoin%
\definecolor{currentfill}{rgb}{0.119483,0.614817,0.537692}%
\pgfsetfillcolor{currentfill}%
\pgfsetfillopacity{0.800000}%
\pgfsetlinewidth{0.000000pt}%
\definecolor{currentstroke}{rgb}{0.000000,0.000000,0.000000}%
\pgfsetstrokecolor{currentstroke}%
\pgfsetdash{}{0pt}%
\pgfpathmoveto{\pgfqpoint{4.133023in}{3.965797in}}%
\pgfpathlineto{\pgfqpoint{4.146309in}{3.948876in}}%
\pgfpathlineto{\pgfqpoint{4.159593in}{3.932197in}}%
\pgfpathlineto{\pgfqpoint{4.172878in}{3.915760in}}%
\pgfpathlineto{\pgfqpoint{4.186163in}{3.899563in}}%
\pgfpathlineto{\pgfqpoint{4.193840in}{3.929236in}}%
\pgfpathlineto{\pgfqpoint{4.201519in}{3.959404in}}%
\pgfpathlineto{\pgfqpoint{4.209201in}{3.990076in}}%
\pgfpathlineto{\pgfqpoint{4.195915in}{4.006857in}}%
\pgfpathlineto{\pgfqpoint{4.182630in}{4.023878in}}%
\pgfpathlineto{\pgfqpoint{4.169343in}{4.041141in}}%
\pgfpathlineto{\pgfqpoint{4.156057in}{4.058649in}}%
\pgfpathlineto{\pgfqpoint{4.148377in}{4.027188in}}%
\pgfpathlineto{\pgfqpoint{4.140699in}{3.996241in}}%
\pgfpathlineto{\pgfqpoint{4.133023in}{3.965797in}}%
\pgfpathclose%
\pgfusepath{fill}%
\end{pgfscope}%
\begin{pgfscope}%
\pgfpathrectangle{\pgfqpoint{1.150000in}{0.150000in}}{\pgfqpoint{5.700000in}{5.700000in}}%
\pgfusepath{clip}%
\pgfsetbuttcap%
\pgfsetroundjoin%
\definecolor{currentfill}{rgb}{0.159194,0.482237,0.558073}%
\pgfsetfillcolor{currentfill}%
\pgfsetfillopacity{0.800000}%
\pgfsetlinewidth{0.000000pt}%
\definecolor{currentstroke}{rgb}{0.000000,0.000000,0.000000}%
\pgfsetstrokecolor{currentstroke}%
\pgfsetdash{}{0pt}%
\pgfpathmoveto{\pgfqpoint{3.873263in}{3.566187in}}%
\pgfpathlineto{\pgfqpoint{3.886553in}{3.550676in}}%
\pgfpathlineto{\pgfqpoint{3.899842in}{3.535417in}}%
\pgfpathlineto{\pgfqpoint{3.913131in}{3.520409in}}%
\pgfpathlineto{\pgfqpoint{3.926419in}{3.505650in}}%
\pgfpathlineto{\pgfqpoint{3.934102in}{3.528433in}}%
\pgfpathlineto{\pgfqpoint{3.941783in}{3.551563in}}%
\pgfpathlineto{\pgfqpoint{3.949462in}{3.575046in}}%
\pgfpathlineto{\pgfqpoint{3.957139in}{3.598891in}}%
\pgfpathlineto{\pgfqpoint{3.943853in}{3.614259in}}%
\pgfpathlineto{\pgfqpoint{3.930565in}{3.629877in}}%
\pgfpathlineto{\pgfqpoint{3.917277in}{3.645746in}}%
\pgfpathlineto{\pgfqpoint{3.903988in}{3.661868in}}%
\pgfpathlineto{\pgfqpoint{3.896310in}{3.637400in}}%
\pgfpathlineto{\pgfqpoint{3.888630in}{3.613302in}}%
\pgfpathlineto{\pgfqpoint{3.880948in}{3.589567in}}%
\pgfpathlineto{\pgfqpoint{3.873263in}{3.566187in}}%
\pgfpathclose%
\pgfusepath{fill}%
\end{pgfscope}%
\begin{pgfscope}%
\pgfpathrectangle{\pgfqpoint{1.150000in}{0.150000in}}{\pgfqpoint{5.700000in}{5.700000in}}%
\pgfusepath{clip}%
\pgfsetbuttcap%
\pgfsetroundjoin%
\definecolor{currentfill}{rgb}{0.171176,0.452530,0.557965}%
\pgfsetfillcolor{currentfill}%
\pgfsetfillopacity{0.800000}%
\pgfsetlinewidth{0.000000pt}%
\definecolor{currentstroke}{rgb}{0.000000,0.000000,0.000000}%
\pgfsetstrokecolor{currentstroke}%
\pgfsetdash{}{0pt}%
\pgfpathmoveto{\pgfqpoint{4.063429in}{3.484725in}}%
\pgfpathlineto{\pgfqpoint{4.076717in}{3.471527in}}%
\pgfpathlineto{\pgfqpoint{4.090007in}{3.458562in}}%
\pgfpathlineto{\pgfqpoint{4.103298in}{3.445830in}}%
\pgfpathlineto{\pgfqpoint{4.116590in}{3.433327in}}%
\pgfpathlineto{\pgfqpoint{4.124260in}{3.455662in}}%
\pgfpathlineto{\pgfqpoint{4.131929in}{3.478346in}}%
\pgfpathlineto{\pgfqpoint{4.139598in}{3.501389in}}%
\pgfpathlineto{\pgfqpoint{4.147265in}{3.524797in}}%
\pgfpathlineto{\pgfqpoint{4.133976in}{3.537934in}}%
\pgfpathlineto{\pgfqpoint{4.120687in}{3.551303in}}%
\pgfpathlineto{\pgfqpoint{4.107400in}{3.564905in}}%
\pgfpathlineto{\pgfqpoint{4.094114in}{3.578740in}}%
\pgfpathlineto{\pgfqpoint{4.086444in}{3.554683in}}%
\pgfpathlineto{\pgfqpoint{4.078773in}{3.530999in}}%
\pgfpathlineto{\pgfqpoint{4.071101in}{3.507683in}}%
\pgfpathlineto{\pgfqpoint{4.063429in}{3.484725in}}%
\pgfpathclose%
\pgfusepath{fill}%
\end{pgfscope}%
\begin{pgfscope}%
\pgfpathrectangle{\pgfqpoint{1.150000in}{0.150000in}}{\pgfqpoint{5.700000in}{5.700000in}}%
\pgfusepath{clip}%
\pgfsetbuttcap%
\pgfsetroundjoin%
\definecolor{currentfill}{rgb}{0.168126,0.459988,0.558082}%
\pgfsetfillcolor{currentfill}%
\pgfsetfillopacity{0.800000}%
\pgfsetlinewidth{0.000000pt}%
\definecolor{currentstroke}{rgb}{0.000000,0.000000,0.000000}%
\pgfsetstrokecolor{currentstroke}%
\pgfsetdash{}{0pt}%
\pgfpathmoveto{\pgfqpoint{3.926419in}{3.505650in}}%
\pgfpathlineto{\pgfqpoint{3.939706in}{3.491138in}}%
\pgfpathlineto{\pgfqpoint{3.952994in}{3.476871in}}%
\pgfpathlineto{\pgfqpoint{3.966281in}{3.462848in}}%
\pgfpathlineto{\pgfqpoint{3.979569in}{3.449067in}}%
\pgfpathlineto{\pgfqpoint{3.987250in}{3.471256in}}%
\pgfpathlineto{\pgfqpoint{3.994929in}{3.493783in}}%
\pgfpathlineto{\pgfqpoint{4.002606in}{3.516655in}}%
\pgfpathlineto{\pgfqpoint{4.010282in}{3.539879in}}%
\pgfpathlineto{\pgfqpoint{3.996996in}{3.554266in}}%
\pgfpathlineto{\pgfqpoint{3.983711in}{3.568896in}}%
\pgfpathlineto{\pgfqpoint{3.970425in}{3.583770in}}%
\pgfpathlineto{\pgfqpoint{3.957139in}{3.598891in}}%
\pgfpathlineto{\pgfqpoint{3.949462in}{3.575046in}}%
\pgfpathlineto{\pgfqpoint{3.941783in}{3.551563in}}%
\pgfpathlineto{\pgfqpoint{3.934102in}{3.528433in}}%
\pgfpathlineto{\pgfqpoint{3.926419in}{3.505650in}}%
\pgfpathclose%
\pgfusepath{fill}%
\end{pgfscope}%
\begin{pgfscope}%
\pgfpathrectangle{\pgfqpoint{1.150000in}{0.150000in}}{\pgfqpoint{5.700000in}{5.700000in}}%
\pgfusepath{clip}%
\pgfsetbuttcap%
\pgfsetroundjoin%
\definecolor{currentfill}{rgb}{0.150476,0.504369,0.557430}%
\pgfsetfillcolor{currentfill}%
\pgfsetfillopacity{0.800000}%
\pgfsetlinewidth{0.000000pt}%
\definecolor{currentstroke}{rgb}{0.000000,0.000000,0.000000}%
\pgfsetstrokecolor{currentstroke}%
\pgfsetdash{}{0pt}%
\pgfpathmoveto{\pgfqpoint{3.820090in}{3.630793in}}%
\pgfpathlineto{\pgfqpoint{3.833386in}{3.614254in}}%
\pgfpathlineto{\pgfqpoint{3.846680in}{3.597974in}}%
\pgfpathlineto{\pgfqpoint{3.859972in}{3.581953in}}%
\pgfpathlineto{\pgfqpoint{3.873263in}{3.566187in}}%
\pgfpathlineto{\pgfqpoint{3.880948in}{3.589567in}}%
\pgfpathlineto{\pgfqpoint{3.888630in}{3.613302in}}%
\pgfpathlineto{\pgfqpoint{3.896310in}{3.637400in}}%
\pgfpathlineto{\pgfqpoint{3.903988in}{3.661868in}}%
\pgfpathlineto{\pgfqpoint{3.890698in}{3.678245in}}%
\pgfpathlineto{\pgfqpoint{3.877406in}{3.694880in}}%
\pgfpathlineto{\pgfqpoint{3.864112in}{3.711773in}}%
\pgfpathlineto{\pgfqpoint{3.850817in}{3.728927in}}%
\pgfpathlineto{\pgfqpoint{3.843138in}{3.703833in}}%
\pgfpathlineto{\pgfqpoint{3.835458in}{3.679117in}}%
\pgfpathlineto{\pgfqpoint{3.827775in}{3.654773in}}%
\pgfpathlineto{\pgfqpoint{3.820090in}{3.630793in}}%
\pgfpathclose%
\pgfusepath{fill}%
\end{pgfscope}%
\begin{pgfscope}%
\pgfpathrectangle{\pgfqpoint{1.150000in}{0.150000in}}{\pgfqpoint{5.700000in}{5.700000in}}%
\pgfusepath{clip}%
\pgfsetbuttcap%
\pgfsetroundjoin%
\definecolor{currentfill}{rgb}{0.119423,0.611141,0.538982}%
\pgfsetfillcolor{currentfill}%
\pgfsetfillopacity{0.800000}%
\pgfsetlinewidth{0.000000pt}%
\definecolor{currentstroke}{rgb}{0.000000,0.000000,0.000000}%
\pgfsetstrokecolor{currentstroke}%
\pgfsetdash{}{0pt}%
\pgfpathmoveto{\pgfqpoint{3.912184in}{3.944242in}}%
\pgfpathlineto{\pgfqpoint{3.925482in}{3.925751in}}%
\pgfpathlineto{\pgfqpoint{3.938778in}{3.907523in}}%
\pgfpathlineto{\pgfqpoint{3.952071in}{3.889556in}}%
\pgfpathlineto{\pgfqpoint{3.965363in}{3.871848in}}%
\pgfpathlineto{\pgfqpoint{3.973031in}{3.899995in}}%
\pgfpathlineto{\pgfqpoint{3.980698in}{3.928592in}}%
\pgfpathlineto{\pgfqpoint{3.988365in}{3.957648in}}%
\pgfpathlineto{\pgfqpoint{3.996033in}{3.987172in}}%
\pgfpathlineto{\pgfqpoint{3.982739in}{4.005601in}}%
\pgfpathlineto{\pgfqpoint{3.969443in}{4.024290in}}%
\pgfpathlineto{\pgfqpoint{3.956145in}{4.043241in}}%
\pgfpathlineto{\pgfqpoint{3.942845in}{4.062456in}}%
\pgfpathlineto{\pgfqpoint{3.935180in}{4.032196in}}%
\pgfpathlineto{\pgfqpoint{3.927515in}{4.002413in}}%
\pgfpathlineto{\pgfqpoint{3.919850in}{3.973098in}}%
\pgfpathlineto{\pgfqpoint{3.912184in}{3.944242in}}%
\pgfpathclose%
\pgfusepath{fill}%
\end{pgfscope}%
\begin{pgfscope}%
\pgfpathrectangle{\pgfqpoint{1.150000in}{0.150000in}}{\pgfqpoint{5.700000in}{5.700000in}}%
\pgfusepath{clip}%
\pgfsetbuttcap%
\pgfsetroundjoin%
\definecolor{currentfill}{rgb}{0.120638,0.625828,0.533488}%
\pgfsetfillcolor{currentfill}%
\pgfsetfillopacity{0.800000}%
\pgfsetlinewidth{0.000000pt}%
\definecolor{currentstroke}{rgb}{0.000000,0.000000,0.000000}%
\pgfsetstrokecolor{currentstroke}%
\pgfsetdash{}{0pt}%
\pgfpathmoveto{\pgfqpoint{3.996033in}{3.987172in}}%
\pgfpathlineto{\pgfqpoint{4.009325in}{3.969001in}}%
\pgfpathlineto{\pgfqpoint{4.022615in}{3.951086in}}%
\pgfpathlineto{\pgfqpoint{4.035904in}{3.933426in}}%
\pgfpathlineto{\pgfqpoint{4.049191in}{3.916018in}}%
\pgfpathlineto{\pgfqpoint{4.056861in}{3.945277in}}%
\pgfpathlineto{\pgfqpoint{4.064531in}{3.975011in}}%
\pgfpathlineto{\pgfqpoint{4.072203in}{4.005231in}}%
\pgfpathlineto{\pgfqpoint{4.079875in}{4.035946in}}%
\pgfpathlineto{\pgfqpoint{4.066585in}{4.054108in}}%
\pgfpathlineto{\pgfqpoint{4.053294in}{4.072523in}}%
\pgfpathlineto{\pgfqpoint{4.040001in}{4.091194in}}%
\pgfpathlineto{\pgfqpoint{4.026706in}{4.110122in}}%
\pgfpathlineto{\pgfqpoint{4.019036in}{4.078638in}}%
\pgfpathlineto{\pgfqpoint{4.011368in}{4.047658in}}%
\pgfpathlineto{\pgfqpoint{4.003700in}{4.017172in}}%
\pgfpathlineto{\pgfqpoint{3.996033in}{3.987172in}}%
\pgfpathclose%
\pgfusepath{fill}%
\end{pgfscope}%
\begin{pgfscope}%
\pgfpathrectangle{\pgfqpoint{1.150000in}{0.150000in}}{\pgfqpoint{5.700000in}{5.700000in}}%
\pgfusepath{clip}%
\pgfsetbuttcap%
\pgfsetroundjoin%
\definecolor{currentfill}{rgb}{0.129933,0.559582,0.551864}%
\pgfsetfillcolor{currentfill}%
\pgfsetfillopacity{0.800000}%
\pgfsetlinewidth{0.000000pt}%
\definecolor{currentstroke}{rgb}{0.000000,0.000000,0.000000}%
\pgfsetstrokecolor{currentstroke}%
\pgfsetdash{}{0pt}%
\pgfpathmoveto{\pgfqpoint{3.797613in}{3.800192in}}%
\pgfpathlineto{\pgfqpoint{3.810917in}{3.781974in}}%
\pgfpathlineto{\pgfqpoint{3.824220in}{3.764025in}}%
\pgfpathlineto{\pgfqpoint{3.837519in}{3.746344in}}%
\pgfpathlineto{\pgfqpoint{3.850817in}{3.728927in}}%
\pgfpathlineto{\pgfqpoint{3.858493in}{3.754407in}}%
\pgfpathlineto{\pgfqpoint{3.866168in}{3.780281in}}%
\pgfpathlineto{\pgfqpoint{3.873841in}{3.806557in}}%
\pgfpathlineto{\pgfqpoint{3.881512in}{3.833243in}}%
\pgfpathlineto{\pgfqpoint{3.868214in}{3.851311in}}%
\pgfpathlineto{\pgfqpoint{3.854913in}{3.869645in}}%
\pgfpathlineto{\pgfqpoint{3.841610in}{3.888246in}}%
\pgfpathlineto{\pgfqpoint{3.828303in}{3.907118in}}%
\pgfpathlineto{\pgfqpoint{3.820634in}{3.879767in}}%
\pgfpathlineto{\pgfqpoint{3.812962in}{3.852833in}}%
\pgfpathlineto{\pgfqpoint{3.805288in}{3.826311in}}%
\pgfpathlineto{\pgfqpoint{3.797613in}{3.800192in}}%
\pgfpathclose%
\pgfusepath{fill}%
\end{pgfscope}%
\begin{pgfscope}%
\pgfpathrectangle{\pgfqpoint{1.150000in}{0.150000in}}{\pgfqpoint{5.700000in}{5.700000in}}%
\pgfusepath{clip}%
\pgfsetbuttcap%
\pgfsetroundjoin%
\definecolor{currentfill}{rgb}{0.150476,0.504369,0.557430}%
\pgfsetfillcolor{currentfill}%
\pgfsetfillopacity{0.800000}%
\pgfsetlinewidth{0.000000pt}%
\definecolor{currentstroke}{rgb}{0.000000,0.000000,0.000000}%
\pgfsetstrokecolor{currentstroke}%
\pgfsetdash{}{0pt}%
\pgfpathmoveto{\pgfqpoint{4.314952in}{3.618260in}}%
\pgfpathlineto{\pgfqpoint{4.328254in}{3.605776in}}%
\pgfpathlineto{\pgfqpoint{4.341558in}{3.593510in}}%
\pgfpathlineto{\pgfqpoint{4.354865in}{3.581462in}}%
\pgfpathlineto{\pgfqpoint{4.368174in}{3.569630in}}%
\pgfpathlineto{\pgfqpoint{4.375839in}{3.594522in}}%
\pgfpathlineto{\pgfqpoint{4.383505in}{3.619835in}}%
\pgfpathlineto{\pgfqpoint{4.391174in}{3.645579in}}%
\pgfpathlineto{\pgfqpoint{4.398845in}{3.671762in}}%
\pgfpathlineto{\pgfqpoint{4.385538in}{3.684326in}}%
\pgfpathlineto{\pgfqpoint{4.372235in}{3.697107in}}%
\pgfpathlineto{\pgfqpoint{4.358933in}{3.710106in}}%
\pgfpathlineto{\pgfqpoint{4.345634in}{3.723325in}}%
\pgfpathlineto{\pgfqpoint{4.337961in}{3.696395in}}%
\pgfpathlineto{\pgfqpoint{4.330289in}{3.669914in}}%
\pgfpathlineto{\pgfqpoint{4.322620in}{3.643872in}}%
\pgfpathlineto{\pgfqpoint{4.314952in}{3.618260in}}%
\pgfpathclose%
\pgfusepath{fill}%
\end{pgfscope}%
\begin{pgfscope}%
\pgfpathrectangle{\pgfqpoint{1.150000in}{0.150000in}}{\pgfqpoint{5.700000in}{5.700000in}}%
\pgfusepath{clip}%
\pgfsetbuttcap%
\pgfsetroundjoin%
\definecolor{currentfill}{rgb}{0.139147,0.533812,0.555298}%
\pgfsetfillcolor{currentfill}%
\pgfsetfillopacity{0.800000}%
\pgfsetlinewidth{0.000000pt}%
\definecolor{currentstroke}{rgb}{0.000000,0.000000,0.000000}%
\pgfsetstrokecolor{currentstroke}%
\pgfsetdash{}{0pt}%
\pgfpathmoveto{\pgfqpoint{4.345634in}{3.723325in}}%
\pgfpathlineto{\pgfqpoint{4.358933in}{3.710106in}}%
\pgfpathlineto{\pgfqpoint{4.372235in}{3.697107in}}%
\pgfpathlineto{\pgfqpoint{4.385538in}{3.684326in}}%
\pgfpathlineto{\pgfqpoint{4.398845in}{3.671762in}}%
\pgfpathlineto{\pgfqpoint{4.406518in}{3.698394in}}%
\pgfpathlineto{\pgfqpoint{4.414194in}{3.725483in}}%
\pgfpathlineto{\pgfqpoint{4.421874in}{3.753039in}}%
\pgfpathlineto{\pgfqpoint{4.408570in}{3.766175in}}%
\pgfpathlineto{\pgfqpoint{4.395268in}{3.779529in}}%
\pgfpathlineto{\pgfqpoint{4.381968in}{3.793102in}}%
\pgfpathlineto{\pgfqpoint{4.368670in}{3.806895in}}%
\pgfpathlineto{\pgfqpoint{4.360989in}{3.778566in}}%
\pgfpathlineto{\pgfqpoint{4.353310in}{3.750712in}}%
\pgfpathlineto{\pgfqpoint{4.345634in}{3.723325in}}%
\pgfpathclose%
\pgfusepath{fill}%
\end{pgfscope}%
\begin{pgfscope}%
\pgfpathrectangle{\pgfqpoint{1.150000in}{0.150000in}}{\pgfqpoint{5.700000in}{5.700000in}}%
\pgfusepath{clip}%
\pgfsetbuttcap%
\pgfsetroundjoin%
\definecolor{currentfill}{rgb}{0.175841,0.441290,0.557685}%
\pgfsetfillcolor{currentfill}%
\pgfsetfillopacity{0.800000}%
\pgfsetlinewidth{0.000000pt}%
\definecolor{currentstroke}{rgb}{0.000000,0.000000,0.000000}%
\pgfsetstrokecolor{currentstroke}%
\pgfsetdash{}{0pt}%
\pgfpathmoveto{\pgfqpoint{3.979569in}{3.449067in}}%
\pgfpathlineto{\pgfqpoint{3.992857in}{3.435527in}}%
\pgfpathlineto{\pgfqpoint{4.006146in}{3.422225in}}%
\pgfpathlineto{\pgfqpoint{4.019435in}{3.409161in}}%
\pgfpathlineto{\pgfqpoint{4.032725in}{3.396332in}}%
\pgfpathlineto{\pgfqpoint{4.040403in}{3.417929in}}%
\pgfpathlineto{\pgfqpoint{4.048080in}{3.439855in}}%
\pgfpathlineto{\pgfqpoint{4.055755in}{3.462118in}}%
\pgfpathlineto{\pgfqpoint{4.063429in}{3.484725in}}%
\pgfpathlineto{\pgfqpoint{4.050141in}{3.498157in}}%
\pgfpathlineto{\pgfqpoint{4.036854in}{3.511826in}}%
\pgfpathlineto{\pgfqpoint{4.023568in}{3.525733in}}%
\pgfpathlineto{\pgfqpoint{4.010282in}{3.539879in}}%
\pgfpathlineto{\pgfqpoint{4.002606in}{3.516655in}}%
\pgfpathlineto{\pgfqpoint{3.994929in}{3.493783in}}%
\pgfpathlineto{\pgfqpoint{3.987250in}{3.471256in}}%
\pgfpathlineto{\pgfqpoint{3.979569in}{3.449067in}}%
\pgfpathclose%
\pgfusepath{fill}%
\end{pgfscope}%
\begin{pgfscope}%
\pgfpathrectangle{\pgfqpoint{1.150000in}{0.150000in}}{\pgfqpoint{5.700000in}{5.700000in}}%
\pgfusepath{clip}%
\pgfsetbuttcap%
\pgfsetroundjoin%
\definecolor{currentfill}{rgb}{0.120092,0.600104,0.542530}%
\pgfsetfillcolor{currentfill}%
\pgfsetfillopacity{0.800000}%
\pgfsetlinewidth{0.000000pt}%
\definecolor{currentstroke}{rgb}{0.000000,0.000000,0.000000}%
\pgfsetstrokecolor{currentstroke}%
\pgfsetdash{}{0pt}%
\pgfpathmoveto{\pgfqpoint{3.828303in}{3.907118in}}%
\pgfpathlineto{\pgfqpoint{3.841610in}{3.888246in}}%
\pgfpathlineto{\pgfqpoint{3.854913in}{3.869645in}}%
\pgfpathlineto{\pgfqpoint{3.868214in}{3.851311in}}%
\pgfpathlineto{\pgfqpoint{3.881512in}{3.833243in}}%
\pgfpathlineto{\pgfqpoint{3.889182in}{3.860346in}}%
\pgfpathlineto{\pgfqpoint{3.896850in}{3.887874in}}%
\pgfpathlineto{\pgfqpoint{3.904518in}{3.915837in}}%
\pgfpathlineto{\pgfqpoint{3.912184in}{3.944242in}}%
\pgfpathlineto{\pgfqpoint{3.898884in}{3.962998in}}%
\pgfpathlineto{\pgfqpoint{3.885582in}{3.982020in}}%
\pgfpathlineto{\pgfqpoint{3.872276in}{4.001312in}}%
\pgfpathlineto{\pgfqpoint{3.858967in}{4.020876in}}%
\pgfpathlineto{\pgfqpoint{3.851303in}{3.991767in}}%
\pgfpathlineto{\pgfqpoint{3.843638in}{3.963110in}}%
\pgfpathlineto{\pgfqpoint{3.835972in}{3.934897in}}%
\pgfpathlineto{\pgfqpoint{3.828303in}{3.907118in}}%
\pgfpathclose%
\pgfusepath{fill}%
\end{pgfscope}%
\begin{pgfscope}%
\pgfpathrectangle{\pgfqpoint{1.150000in}{0.150000in}}{\pgfqpoint{5.700000in}{5.700000in}}%
\pgfusepath{clip}%
\pgfsetbuttcap%
\pgfsetroundjoin%
\definecolor{currentfill}{rgb}{0.124780,0.640461,0.527068}%
\pgfsetfillcolor{currentfill}%
\pgfsetfillopacity{0.800000}%
\pgfsetlinewidth{0.000000pt}%
\definecolor{currentstroke}{rgb}{0.000000,0.000000,0.000000}%
\pgfsetstrokecolor{currentstroke}%
\pgfsetdash{}{0pt}%
\pgfpathmoveto{\pgfqpoint{4.079875in}{4.035946in}}%
\pgfpathlineto{\pgfqpoint{4.093164in}{4.018035in}}%
\pgfpathlineto{\pgfqpoint{4.106451in}{4.000375in}}%
\pgfpathlineto{\pgfqpoint{4.119738in}{3.982963in}}%
\pgfpathlineto{\pgfqpoint{4.133023in}{3.965797in}}%
\pgfpathlineto{\pgfqpoint{4.140699in}{3.996241in}}%
\pgfpathlineto{\pgfqpoint{4.148377in}{4.027188in}}%
\pgfpathlineto{\pgfqpoint{4.156057in}{4.058649in}}%
\pgfpathlineto{\pgfqpoint{4.142769in}{4.076402in}}%
\pgfpathlineto{\pgfqpoint{4.129481in}{4.094402in}}%
\pgfpathlineto{\pgfqpoint{4.116191in}{4.112651in}}%
\pgfpathlineto{\pgfqpoint{4.102900in}{4.131151in}}%
\pgfpathlineto{\pgfqpoint{4.095223in}{4.098896in}}%
\pgfpathlineto{\pgfqpoint{4.087548in}{4.067164in}}%
\pgfpathlineto{\pgfqpoint{4.079875in}{4.035946in}}%
\pgfpathclose%
\pgfusepath{fill}%
\end{pgfscope}%
\begin{pgfscope}%
\pgfpathrectangle{\pgfqpoint{1.150000in}{0.150000in}}{\pgfqpoint{5.700000in}{5.700000in}}%
\pgfusepath{clip}%
\pgfsetbuttcap%
\pgfsetroundjoin%
\definecolor{currentfill}{rgb}{0.141935,0.526453,0.555991}%
\pgfsetfillcolor{currentfill}%
\pgfsetfillopacity{0.800000}%
\pgfsetlinewidth{0.000000pt}%
\definecolor{currentstroke}{rgb}{0.000000,0.000000,0.000000}%
\pgfsetstrokecolor{currentstroke}%
\pgfsetdash{}{0pt}%
\pgfpathmoveto{\pgfqpoint{3.766886in}{3.699592in}}%
\pgfpathlineto{\pgfqpoint{3.780191in}{3.681992in}}%
\pgfpathlineto{\pgfqpoint{3.793493in}{3.664660in}}%
\pgfpathlineto{\pgfqpoint{3.806792in}{3.647595in}}%
\pgfpathlineto{\pgfqpoint{3.820090in}{3.630793in}}%
\pgfpathlineto{\pgfqpoint{3.827775in}{3.654773in}}%
\pgfpathlineto{\pgfqpoint{3.835458in}{3.679117in}}%
\pgfpathlineto{\pgfqpoint{3.843138in}{3.703833in}}%
\pgfpathlineto{\pgfqpoint{3.850817in}{3.728927in}}%
\pgfpathlineto{\pgfqpoint{3.837519in}{3.746344in}}%
\pgfpathlineto{\pgfqpoint{3.824220in}{3.764025in}}%
\pgfpathlineto{\pgfqpoint{3.810917in}{3.781974in}}%
\pgfpathlineto{\pgfqpoint{3.797613in}{3.800192in}}%
\pgfpathlineto{\pgfqpoint{3.789935in}{3.774468in}}%
\pgfpathlineto{\pgfqpoint{3.782254in}{3.749131in}}%
\pgfpathlineto{\pgfqpoint{3.774572in}{3.724175in}}%
\pgfpathlineto{\pgfqpoint{3.766886in}{3.699592in}}%
\pgfpathclose%
\pgfusepath{fill}%
\end{pgfscope}%
\begin{pgfscope}%
\pgfpathrectangle{\pgfqpoint{1.150000in}{0.150000in}}{\pgfqpoint{5.700000in}{5.700000in}}%
\pgfusepath{clip}%
\pgfsetbuttcap%
\pgfsetroundjoin%
\definecolor{currentfill}{rgb}{0.171176,0.452530,0.557965}%
\pgfsetfillcolor{currentfill}%
\pgfsetfillopacity{0.800000}%
\pgfsetlinewidth{0.000000pt}%
\definecolor{currentstroke}{rgb}{0.000000,0.000000,0.000000}%
\pgfsetstrokecolor{currentstroke}%
\pgfsetdash{}{0pt}%
\pgfpathmoveto{\pgfqpoint{4.200441in}{3.474526in}}%
\pgfpathlineto{\pgfqpoint{4.213739in}{3.462522in}}%
\pgfpathlineto{\pgfqpoint{4.227040in}{3.450741in}}%
\pgfpathlineto{\pgfqpoint{4.240343in}{3.439181in}}%
\pgfpathlineto{\pgfqpoint{4.253649in}{3.427842in}}%
\pgfpathlineto{\pgfqpoint{4.261310in}{3.450314in}}%
\pgfpathlineto{\pgfqpoint{4.268972in}{3.473150in}}%
\pgfpathlineto{\pgfqpoint{4.276633in}{3.496358in}}%
\pgfpathlineto{\pgfqpoint{4.284295in}{3.519946in}}%
\pgfpathlineto{\pgfqpoint{4.270993in}{3.531951in}}%
\pgfpathlineto{\pgfqpoint{4.257693in}{3.544176in}}%
\pgfpathlineto{\pgfqpoint{4.244395in}{3.556624in}}%
\pgfpathlineto{\pgfqpoint{4.231100in}{3.569295in}}%
\pgfpathlineto{\pgfqpoint{4.223435in}{3.545028in}}%
\pgfpathlineto{\pgfqpoint{4.215770in}{3.521150in}}%
\pgfpathlineto{\pgfqpoint{4.208105in}{3.497652in}}%
\pgfpathlineto{\pgfqpoint{4.200441in}{3.474526in}}%
\pgfpathclose%
\pgfusepath{fill}%
\end{pgfscope}%
\begin{pgfscope}%
\pgfpathrectangle{\pgfqpoint{1.150000in}{0.150000in}}{\pgfqpoint{5.700000in}{5.700000in}}%
\pgfusepath{clip}%
\pgfsetbuttcap%
\pgfsetroundjoin%
\definecolor{currentfill}{rgb}{0.177423,0.437527,0.557565}%
\pgfsetfillcolor{currentfill}%
\pgfsetfillopacity{0.800000}%
\pgfsetlinewidth{0.000000pt}%
\definecolor{currentstroke}{rgb}{0.000000,0.000000,0.000000}%
\pgfsetstrokecolor{currentstroke}%
\pgfsetdash{}{0pt}%
\pgfpathmoveto{\pgfqpoint{4.116590in}{3.433327in}}%
\pgfpathlineto{\pgfqpoint{4.129885in}{3.421054in}}%
\pgfpathlineto{\pgfqpoint{4.143180in}{3.409008in}}%
\pgfpathlineto{\pgfqpoint{4.156478in}{3.397189in}}%
\pgfpathlineto{\pgfqpoint{4.169778in}{3.385594in}}%
\pgfpathlineto{\pgfqpoint{4.177445in}{3.407307in}}%
\pgfpathlineto{\pgfqpoint{4.185110in}{3.429362in}}%
\pgfpathlineto{\pgfqpoint{4.192776in}{3.451766in}}%
\pgfpathlineto{\pgfqpoint{4.200441in}{3.474526in}}%
\pgfpathlineto{\pgfqpoint{4.187144in}{3.486754in}}%
\pgfpathlineto{\pgfqpoint{4.173849in}{3.499208in}}%
\pgfpathlineto{\pgfqpoint{4.160557in}{3.511888in}}%
\pgfpathlineto{\pgfqpoint{4.147265in}{3.524797in}}%
\pgfpathlineto{\pgfqpoint{4.139598in}{3.501389in}}%
\pgfpathlineto{\pgfqpoint{4.131929in}{3.478346in}}%
\pgfpathlineto{\pgfqpoint{4.124260in}{3.455662in}}%
\pgfpathlineto{\pgfqpoint{4.116590in}{3.433327in}}%
\pgfpathclose%
\pgfusepath{fill}%
\end{pgfscope}%
\begin{pgfscope}%
\pgfpathrectangle{\pgfqpoint{1.150000in}{0.150000in}}{\pgfqpoint{5.700000in}{5.700000in}}%
\pgfusepath{clip}%
\pgfsetbuttcap%
\pgfsetroundjoin%
\definecolor{currentfill}{rgb}{0.163625,0.471133,0.558148}%
\pgfsetfillcolor{currentfill}%
\pgfsetfillopacity{0.800000}%
\pgfsetlinewidth{0.000000pt}%
\definecolor{currentstroke}{rgb}{0.000000,0.000000,0.000000}%
\pgfsetstrokecolor{currentstroke}%
\pgfsetdash{}{0pt}%
\pgfpathmoveto{\pgfqpoint{4.284295in}{3.519946in}}%
\pgfpathlineto{\pgfqpoint{4.297600in}{3.508160in}}%
\pgfpathlineto{\pgfqpoint{4.310908in}{3.496593in}}%
\pgfpathlineto{\pgfqpoint{4.324218in}{3.485243in}}%
\pgfpathlineto{\pgfqpoint{4.337532in}{3.474109in}}%
\pgfpathlineto{\pgfqpoint{4.345190in}{3.497399in}}%
\pgfpathlineto{\pgfqpoint{4.352850in}{3.521077in}}%
\pgfpathlineto{\pgfqpoint{4.360512in}{3.545152in}}%
\pgfpathlineto{\pgfqpoint{4.368174in}{3.569630in}}%
\pgfpathlineto{\pgfqpoint{4.354865in}{3.581462in}}%
\pgfpathlineto{\pgfqpoint{4.341558in}{3.593510in}}%
\pgfpathlineto{\pgfqpoint{4.328254in}{3.605776in}}%
\pgfpathlineto{\pgfqpoint{4.314952in}{3.618260in}}%
\pgfpathlineto{\pgfqpoint{4.307286in}{3.593070in}}%
\pgfpathlineto{\pgfqpoint{4.299622in}{3.568294in}}%
\pgfpathlineto{\pgfqpoint{4.291958in}{3.543921in}}%
\pgfpathlineto{\pgfqpoint{4.284295in}{3.519946in}}%
\pgfpathclose%
\pgfusepath{fill}%
\end{pgfscope}%
\begin{pgfscope}%
\pgfpathrectangle{\pgfqpoint{1.150000in}{0.150000in}}{\pgfqpoint{5.700000in}{5.700000in}}%
\pgfusepath{clip}%
\pgfsetbuttcap%
\pgfsetroundjoin%
\definecolor{currentfill}{rgb}{0.171176,0.452530,0.557965}%
\pgfsetfillcolor{currentfill}%
\pgfsetfillopacity{0.800000}%
\pgfsetlinewidth{0.000000pt}%
\definecolor{currentstroke}{rgb}{0.000000,0.000000,0.000000}%
\pgfsetstrokecolor{currentstroke}%
\pgfsetdash{}{0pt}%
\pgfpathmoveto{\pgfqpoint{3.842502in}{3.476078in}}%
\pgfpathlineto{\pgfqpoint{3.855794in}{3.461144in}}%
\pgfpathlineto{\pgfqpoint{3.869085in}{3.446461in}}%
\pgfpathlineto{\pgfqpoint{3.882375in}{3.432028in}}%
\pgfpathlineto{\pgfqpoint{3.895665in}{3.417844in}}%
\pgfpathlineto{\pgfqpoint{3.903357in}{3.439310in}}%
\pgfpathlineto{\pgfqpoint{3.911047in}{3.461096in}}%
\pgfpathlineto{\pgfqpoint{3.918734in}{3.483207in}}%
\pgfpathlineto{\pgfqpoint{3.926419in}{3.505650in}}%
\pgfpathlineto{\pgfqpoint{3.913131in}{3.520409in}}%
\pgfpathlineto{\pgfqpoint{3.899842in}{3.535417in}}%
\pgfpathlineto{\pgfqpoint{3.886553in}{3.550676in}}%
\pgfpathlineto{\pgfqpoint{3.873263in}{3.566187in}}%
\pgfpathlineto{\pgfqpoint{3.865577in}{3.543155in}}%
\pgfpathlineto{\pgfqpoint{3.857888in}{3.520464in}}%
\pgfpathlineto{\pgfqpoint{3.850196in}{3.498108in}}%
\pgfpathlineto{\pgfqpoint{3.842502in}{3.476078in}}%
\pgfpathclose%
\pgfusepath{fill}%
\end{pgfscope}%
\begin{pgfscope}%
\pgfpathrectangle{\pgfqpoint{1.150000in}{0.150000in}}{\pgfqpoint{5.700000in}{5.700000in}}%
\pgfusepath{clip}%
\pgfsetbuttcap%
\pgfsetroundjoin%
\definecolor{currentfill}{rgb}{0.163625,0.471133,0.558148}%
\pgfsetfillcolor{currentfill}%
\pgfsetfillopacity{0.800000}%
\pgfsetlinewidth{0.000000pt}%
\definecolor{currentstroke}{rgb}{0.000000,0.000000,0.000000}%
\pgfsetstrokecolor{currentstroke}%
\pgfsetdash{}{0pt}%
\pgfpathmoveto{\pgfqpoint{3.789323in}{3.538371in}}%
\pgfpathlineto{\pgfqpoint{3.802620in}{3.522411in}}%
\pgfpathlineto{\pgfqpoint{3.815915in}{3.506710in}}%
\pgfpathlineto{\pgfqpoint{3.829209in}{3.491266in}}%
\pgfpathlineto{\pgfqpoint{3.842502in}{3.476078in}}%
\pgfpathlineto{\pgfqpoint{3.850196in}{3.498108in}}%
\pgfpathlineto{\pgfqpoint{3.857888in}{3.520464in}}%
\pgfpathlineto{\pgfqpoint{3.865577in}{3.543155in}}%
\pgfpathlineto{\pgfqpoint{3.873263in}{3.566187in}}%
\pgfpathlineto{\pgfqpoint{3.859972in}{3.581953in}}%
\pgfpathlineto{\pgfqpoint{3.846680in}{3.597974in}}%
\pgfpathlineto{\pgfqpoint{3.833386in}{3.614254in}}%
\pgfpathlineto{\pgfqpoint{3.820090in}{3.630793in}}%
\pgfpathlineto{\pgfqpoint{3.812402in}{3.607170in}}%
\pgfpathlineto{\pgfqpoint{3.804712in}{3.583896in}}%
\pgfpathlineto{\pgfqpoint{3.797019in}{3.560966in}}%
\pgfpathlineto{\pgfqpoint{3.789323in}{3.538371in}}%
\pgfpathclose%
\pgfusepath{fill}%
\end{pgfscope}%
\begin{pgfscope}%
\pgfpathrectangle{\pgfqpoint{1.150000in}{0.150000in}}{\pgfqpoint{5.700000in}{5.700000in}}%
\pgfusepath{clip}%
\pgfsetbuttcap%
\pgfsetroundjoin%
\definecolor{currentfill}{rgb}{0.144759,0.519093,0.556572}%
\pgfsetfillcolor{currentfill}%
\pgfsetfillopacity{0.800000}%
\pgfsetlinewidth{0.000000pt}%
\definecolor{currentstroke}{rgb}{0.000000,0.000000,0.000000}%
\pgfsetstrokecolor{currentstroke}%
\pgfsetdash{}{0pt}%
\pgfpathmoveto{\pgfqpoint{4.398845in}{3.671762in}}%
\pgfpathlineto{\pgfqpoint{4.412154in}{3.659414in}}%
\pgfpathlineto{\pgfqpoint{4.425465in}{3.647281in}}%
\pgfpathlineto{\pgfqpoint{4.438780in}{3.635361in}}%
\pgfpathlineto{\pgfqpoint{4.452099in}{3.623654in}}%
\pgfpathlineto{\pgfqpoint{4.459769in}{3.649533in}}%
\pgfpathlineto{\pgfqpoint{4.467442in}{3.675861in}}%
\pgfpathlineto{\pgfqpoint{4.475118in}{3.702646in}}%
\pgfpathlineto{\pgfqpoint{4.461803in}{3.714924in}}%
\pgfpathlineto{\pgfqpoint{4.448490in}{3.727415in}}%
\pgfpathlineto{\pgfqpoint{4.435181in}{3.740119in}}%
\pgfpathlineto{\pgfqpoint{4.421874in}{3.753039in}}%
\pgfpathlineto{\pgfqpoint{4.414194in}{3.725483in}}%
\pgfpathlineto{\pgfqpoint{4.406518in}{3.698394in}}%
\pgfpathlineto{\pgfqpoint{4.398845in}{3.671762in}}%
\pgfpathclose%
\pgfusepath{fill}%
\end{pgfscope}%
\begin{pgfscope}%
\pgfpathrectangle{\pgfqpoint{1.150000in}{0.150000in}}{\pgfqpoint{5.700000in}{5.700000in}}%
\pgfusepath{clip}%
\pgfsetbuttcap%
\pgfsetroundjoin%
\definecolor{currentfill}{rgb}{0.182256,0.426184,0.557120}%
\pgfsetfillcolor{currentfill}%
\pgfsetfillopacity{0.800000}%
\pgfsetlinewidth{0.000000pt}%
\definecolor{currentstroke}{rgb}{0.000000,0.000000,0.000000}%
\pgfsetstrokecolor{currentstroke}%
\pgfsetdash{}{0pt}%
\pgfpathmoveto{\pgfqpoint{4.032725in}{3.396332in}}%
\pgfpathlineto{\pgfqpoint{4.046017in}{3.383738in}}%
\pgfpathlineto{\pgfqpoint{4.059309in}{3.371376in}}%
\pgfpathlineto{\pgfqpoint{4.072604in}{3.359246in}}%
\pgfpathlineto{\pgfqpoint{4.085899in}{3.347345in}}%
\pgfpathlineto{\pgfqpoint{4.093574in}{3.368352in}}%
\pgfpathlineto{\pgfqpoint{4.101248in}{3.389680in}}%
\pgfpathlineto{\pgfqpoint{4.108920in}{3.411336in}}%
\pgfpathlineto{\pgfqpoint{4.116590in}{3.433327in}}%
\pgfpathlineto{\pgfqpoint{4.103298in}{3.445830in}}%
\pgfpathlineto{\pgfqpoint{4.090007in}{3.458562in}}%
\pgfpathlineto{\pgfqpoint{4.076717in}{3.471527in}}%
\pgfpathlineto{\pgfqpoint{4.063429in}{3.484725in}}%
\pgfpathlineto{\pgfqpoint{4.055755in}{3.462118in}}%
\pgfpathlineto{\pgfqpoint{4.048080in}{3.439855in}}%
\pgfpathlineto{\pgfqpoint{4.040403in}{3.417929in}}%
\pgfpathlineto{\pgfqpoint{4.032725in}{3.396332in}}%
\pgfpathclose%
\pgfusepath{fill}%
\end{pgfscope}%
\begin{pgfscope}%
\pgfpathrectangle{\pgfqpoint{1.150000in}{0.150000in}}{\pgfqpoint{5.700000in}{5.700000in}}%
\pgfusepath{clip}%
\pgfsetbuttcap%
\pgfsetroundjoin%
\definecolor{currentfill}{rgb}{0.132268,0.655014,0.519661}%
\pgfsetfillcolor{currentfill}%
\pgfsetfillopacity{0.800000}%
\pgfsetlinewidth{0.000000pt}%
\definecolor{currentstroke}{rgb}{0.000000,0.000000,0.000000}%
\pgfsetstrokecolor{currentstroke}%
\pgfsetdash{}{0pt}%
\pgfpathmoveto{\pgfqpoint{3.942845in}{4.062456in}}%
\pgfpathlineto{\pgfqpoint{3.956145in}{4.043241in}}%
\pgfpathlineto{\pgfqpoint{3.969443in}{4.024290in}}%
\pgfpathlineto{\pgfqpoint{3.982739in}{4.005601in}}%
\pgfpathlineto{\pgfqpoint{3.996033in}{3.987172in}}%
\pgfpathlineto{\pgfqpoint{4.003700in}{4.017172in}}%
\pgfpathlineto{\pgfqpoint{4.011368in}{4.047658in}}%
\pgfpathlineto{\pgfqpoint{4.019036in}{4.078638in}}%
\pgfpathlineto{\pgfqpoint{4.026706in}{4.110122in}}%
\pgfpathlineto{\pgfqpoint{4.013409in}{4.129310in}}%
\pgfpathlineto{\pgfqpoint{4.000109in}{4.148759in}}%
\pgfpathlineto{\pgfqpoint{3.986808in}{4.168471in}}%
\pgfpathlineto{\pgfqpoint{3.973503in}{4.188449in}}%
\pgfpathlineto{\pgfqpoint{3.965838in}{4.156189in}}%
\pgfpathlineto{\pgfqpoint{3.958173in}{4.124444in}}%
\pgfpathlineto{\pgfqpoint{3.950509in}{4.093202in}}%
\pgfpathlineto{\pgfqpoint{3.942845in}{4.062456in}}%
\pgfpathclose%
\pgfusepath{fill}%
\end{pgfscope}%
\begin{pgfscope}%
\pgfpathrectangle{\pgfqpoint{1.150000in}{0.150000in}}{\pgfqpoint{5.700000in}{5.700000in}}%
\pgfusepath{clip}%
\pgfsetbuttcap%
\pgfsetroundjoin%
\definecolor{currentfill}{rgb}{0.124780,0.640461,0.527068}%
\pgfsetfillcolor{currentfill}%
\pgfsetfillopacity{0.800000}%
\pgfsetlinewidth{0.000000pt}%
\definecolor{currentstroke}{rgb}{0.000000,0.000000,0.000000}%
\pgfsetstrokecolor{currentstroke}%
\pgfsetdash{}{0pt}%
\pgfpathmoveto{\pgfqpoint{3.858967in}{4.020876in}}%
\pgfpathlineto{\pgfqpoint{3.872276in}{4.001312in}}%
\pgfpathlineto{\pgfqpoint{3.885582in}{3.982020in}}%
\pgfpathlineto{\pgfqpoint{3.898884in}{3.962998in}}%
\pgfpathlineto{\pgfqpoint{3.912184in}{3.944242in}}%
\pgfpathlineto{\pgfqpoint{3.919850in}{3.973098in}}%
\pgfpathlineto{\pgfqpoint{3.927515in}{4.002413in}}%
\pgfpathlineto{\pgfqpoint{3.935180in}{4.032196in}}%
\pgfpathlineto{\pgfqpoint{3.942845in}{4.062456in}}%
\pgfpathlineto{\pgfqpoint{3.929541in}{4.081937in}}%
\pgfpathlineto{\pgfqpoint{3.916236in}{4.101687in}}%
\pgfpathlineto{\pgfqpoint{3.902927in}{4.121706in}}%
\pgfpathlineto{\pgfqpoint{3.889615in}{4.141999in}}%
\pgfpathlineto{\pgfqpoint{3.881954in}{4.110997in}}%
\pgfpathlineto{\pgfqpoint{3.874293in}{4.080482in}}%
\pgfpathlineto{\pgfqpoint{3.866630in}{4.050444in}}%
\pgfpathlineto{\pgfqpoint{3.858967in}{4.020876in}}%
\pgfpathclose%
\pgfusepath{fill}%
\end{pgfscope}%
\begin{pgfscope}%
\pgfpathrectangle{\pgfqpoint{1.150000in}{0.150000in}}{\pgfqpoint{5.700000in}{5.700000in}}%
\pgfusepath{clip}%
\pgfsetbuttcap%
\pgfsetroundjoin%
\definecolor{currentfill}{rgb}{0.121831,0.589055,0.545623}%
\pgfsetfillcolor{currentfill}%
\pgfsetfillopacity{0.800000}%
\pgfsetlinewidth{0.000000pt}%
\definecolor{currentstroke}{rgb}{0.000000,0.000000,0.000000}%
\pgfsetstrokecolor{currentstroke}%
\pgfsetdash{}{0pt}%
\pgfpathmoveto{\pgfqpoint{3.744363in}{3.875797in}}%
\pgfpathlineto{\pgfqpoint{3.757680in}{3.856481in}}%
\pgfpathlineto{\pgfqpoint{3.770994in}{3.837443in}}%
\pgfpathlineto{\pgfqpoint{3.784305in}{3.818681in}}%
\pgfpathlineto{\pgfqpoint{3.797613in}{3.800192in}}%
\pgfpathlineto{\pgfqpoint{3.805288in}{3.826311in}}%
\pgfpathlineto{\pgfqpoint{3.812962in}{3.852833in}}%
\pgfpathlineto{\pgfqpoint{3.820634in}{3.879767in}}%
\pgfpathlineto{\pgfqpoint{3.828303in}{3.907118in}}%
\pgfpathlineto{\pgfqpoint{3.814994in}{3.926263in}}%
\pgfpathlineto{\pgfqpoint{3.801682in}{3.945681in}}%
\pgfpathlineto{\pgfqpoint{3.788366in}{3.965377in}}%
\pgfpathlineto{\pgfqpoint{3.775046in}{3.985351in}}%
\pgfpathlineto{\pgfqpoint{3.767379in}{3.957329in}}%
\pgfpathlineto{\pgfqpoint{3.759709in}{3.929734in}}%
\pgfpathlineto{\pgfqpoint{3.752037in}{3.902560in}}%
\pgfpathlineto{\pgfqpoint{3.744363in}{3.875797in}}%
\pgfpathclose%
\pgfusepath{fill}%
\end{pgfscope}%
\begin{pgfscope}%
\pgfpathrectangle{\pgfqpoint{1.150000in}{0.150000in}}{\pgfqpoint{5.700000in}{5.700000in}}%
\pgfusepath{clip}%
\pgfsetbuttcap%
\pgfsetroundjoin%
\definecolor{currentfill}{rgb}{0.156270,0.489624,0.557936}%
\pgfsetfillcolor{currentfill}%
\pgfsetfillopacity{0.800000}%
\pgfsetlinewidth{0.000000pt}%
\definecolor{currentstroke}{rgb}{0.000000,0.000000,0.000000}%
\pgfsetstrokecolor{currentstroke}%
\pgfsetdash{}{0pt}%
\pgfpathmoveto{\pgfqpoint{4.368174in}{3.569630in}}%
\pgfpathlineto{\pgfqpoint{4.381487in}{3.558014in}}%
\pgfpathlineto{\pgfqpoint{4.394803in}{3.546611in}}%
\pgfpathlineto{\pgfqpoint{4.408121in}{3.535422in}}%
\pgfpathlineto{\pgfqpoint{4.421444in}{3.524445in}}%
\pgfpathlineto{\pgfqpoint{4.429104in}{3.548619in}}%
\pgfpathlineto{\pgfqpoint{4.436767in}{3.573206in}}%
\pgfpathlineto{\pgfqpoint{4.444431in}{3.598214in}}%
\pgfpathlineto{\pgfqpoint{4.452099in}{3.623654in}}%
\pgfpathlineto{\pgfqpoint{4.438780in}{3.635361in}}%
\pgfpathlineto{\pgfqpoint{4.425465in}{3.647281in}}%
\pgfpathlineto{\pgfqpoint{4.412154in}{3.659414in}}%
\pgfpathlineto{\pgfqpoint{4.398845in}{3.671762in}}%
\pgfpathlineto{\pgfqpoint{4.391174in}{3.645579in}}%
\pgfpathlineto{\pgfqpoint{4.383505in}{3.619835in}}%
\pgfpathlineto{\pgfqpoint{4.375839in}{3.594522in}}%
\pgfpathlineto{\pgfqpoint{4.368174in}{3.569630in}}%
\pgfpathclose%
\pgfusepath{fill}%
\end{pgfscope}%
\begin{pgfscope}%
\pgfpathrectangle{\pgfqpoint{1.150000in}{0.150000in}}{\pgfqpoint{5.700000in}{5.700000in}}%
\pgfusepath{clip}%
\pgfsetbuttcap%
\pgfsetroundjoin%
\definecolor{currentfill}{rgb}{0.180629,0.429975,0.557282}%
\pgfsetfillcolor{currentfill}%
\pgfsetfillopacity{0.800000}%
\pgfsetlinewidth{0.000000pt}%
\definecolor{currentstroke}{rgb}{0.000000,0.000000,0.000000}%
\pgfsetstrokecolor{currentstroke}%
\pgfsetdash{}{0pt}%
\pgfpathmoveto{\pgfqpoint{3.895665in}{3.417844in}}%
\pgfpathlineto{\pgfqpoint{3.908955in}{3.403906in}}%
\pgfpathlineto{\pgfqpoint{3.922245in}{3.390212in}}%
\pgfpathlineto{\pgfqpoint{3.935535in}{3.376762in}}%
\pgfpathlineto{\pgfqpoint{3.948825in}{3.363554in}}%
\pgfpathlineto{\pgfqpoint{3.956514in}{3.384460in}}%
\pgfpathlineto{\pgfqpoint{3.964201in}{3.405676in}}%
\pgfpathlineto{\pgfqpoint{3.971886in}{3.427210in}}%
\pgfpathlineto{\pgfqpoint{3.979569in}{3.449067in}}%
\pgfpathlineto{\pgfqpoint{3.966281in}{3.462848in}}%
\pgfpathlineto{\pgfqpoint{3.952994in}{3.476871in}}%
\pgfpathlineto{\pgfqpoint{3.939706in}{3.491138in}}%
\pgfpathlineto{\pgfqpoint{3.926419in}{3.505650in}}%
\pgfpathlineto{\pgfqpoint{3.918734in}{3.483207in}}%
\pgfpathlineto{\pgfqpoint{3.911047in}{3.461096in}}%
\pgfpathlineto{\pgfqpoint{3.903357in}{3.439310in}}%
\pgfpathlineto{\pgfqpoint{3.895665in}{3.417844in}}%
\pgfpathclose%
\pgfusepath{fill}%
\end{pgfscope}%
\begin{pgfscope}%
\pgfpathrectangle{\pgfqpoint{1.150000in}{0.150000in}}{\pgfqpoint{5.700000in}{5.700000in}}%
\pgfusepath{clip}%
\pgfsetbuttcap%
\pgfsetroundjoin%
\definecolor{currentfill}{rgb}{0.154815,0.493313,0.557840}%
\pgfsetfillcolor{currentfill}%
\pgfsetfillopacity{0.800000}%
\pgfsetlinewidth{0.000000pt}%
\definecolor{currentstroke}{rgb}{0.000000,0.000000,0.000000}%
\pgfsetstrokecolor{currentstroke}%
\pgfsetdash{}{0pt}%
\pgfpathmoveto{\pgfqpoint{3.736116in}{3.604844in}}%
\pgfpathlineto{\pgfqpoint{3.749421in}{3.587827in}}%
\pgfpathlineto{\pgfqpoint{3.762724in}{3.571077in}}%
\pgfpathlineto{\pgfqpoint{3.776024in}{3.554592in}}%
\pgfpathlineto{\pgfqpoint{3.789323in}{3.538371in}}%
\pgfpathlineto{\pgfqpoint{3.797019in}{3.560966in}}%
\pgfpathlineto{\pgfqpoint{3.804712in}{3.583896in}}%
\pgfpathlineto{\pgfqpoint{3.812402in}{3.607170in}}%
\pgfpathlineto{\pgfqpoint{3.820090in}{3.630793in}}%
\pgfpathlineto{\pgfqpoint{3.806792in}{3.647595in}}%
\pgfpathlineto{\pgfqpoint{3.793493in}{3.664660in}}%
\pgfpathlineto{\pgfqpoint{3.780191in}{3.681992in}}%
\pgfpathlineto{\pgfqpoint{3.766886in}{3.699592in}}%
\pgfpathlineto{\pgfqpoint{3.759198in}{3.675374in}}%
\pgfpathlineto{\pgfqpoint{3.751507in}{3.651515in}}%
\pgfpathlineto{\pgfqpoint{3.743813in}{3.628007in}}%
\pgfpathlineto{\pgfqpoint{3.736116in}{3.604844in}}%
\pgfpathclose%
\pgfusepath{fill}%
\end{pgfscope}%
\begin{pgfscope}%
\pgfpathrectangle{\pgfqpoint{1.150000in}{0.150000in}}{\pgfqpoint{5.700000in}{5.700000in}}%
\pgfusepath{clip}%
\pgfsetbuttcap%
\pgfsetroundjoin%
\definecolor{currentfill}{rgb}{0.140210,0.665859,0.513427}%
\pgfsetfillcolor{currentfill}%
\pgfsetfillopacity{0.800000}%
\pgfsetlinewidth{0.000000pt}%
\definecolor{currentstroke}{rgb}{0.000000,0.000000,0.000000}%
\pgfsetstrokecolor{currentstroke}%
\pgfsetdash{}{0pt}%
\pgfpathmoveto{\pgfqpoint{4.026706in}{4.110122in}}%
\pgfpathlineto{\pgfqpoint{4.040001in}{4.091194in}}%
\pgfpathlineto{\pgfqpoint{4.053294in}{4.072523in}}%
\pgfpathlineto{\pgfqpoint{4.066585in}{4.054108in}}%
\pgfpathlineto{\pgfqpoint{4.079875in}{4.035946in}}%
\pgfpathlineto{\pgfqpoint{4.087548in}{4.067164in}}%
\pgfpathlineto{\pgfqpoint{4.095223in}{4.098896in}}%
\pgfpathlineto{\pgfqpoint{4.102900in}{4.131151in}}%
\pgfpathlineto{\pgfqpoint{4.089608in}{4.149903in}}%
\pgfpathlineto{\pgfqpoint{4.076314in}{4.168910in}}%
\pgfpathlineto{\pgfqpoint{4.063018in}{4.188174in}}%
\pgfpathlineto{\pgfqpoint{4.049720in}{4.207696in}}%
\pgfpathlineto{\pgfqpoint{4.042047in}{4.174642in}}%
\pgfpathlineto{\pgfqpoint{4.034376in}{4.142120in}}%
\pgfpathlineto{\pgfqpoint{4.026706in}{4.110122in}}%
\pgfpathclose%
\pgfusepath{fill}%
\end{pgfscope}%
\begin{pgfscope}%
\pgfpathrectangle{\pgfqpoint{1.150000in}{0.150000in}}{\pgfqpoint{5.700000in}{5.700000in}}%
\pgfusepath{clip}%
\pgfsetbuttcap%
\pgfsetroundjoin%
\definecolor{currentfill}{rgb}{0.132444,0.552216,0.553018}%
\pgfsetfillcolor{currentfill}%
\pgfsetfillopacity{0.800000}%
\pgfsetlinewidth{0.000000pt}%
\definecolor{currentstroke}{rgb}{0.000000,0.000000,0.000000}%
\pgfsetstrokecolor{currentstroke}%
\pgfsetdash{}{0pt}%
\pgfpathmoveto{\pgfqpoint{3.713639in}{3.772715in}}%
\pgfpathlineto{\pgfqpoint{3.726956in}{3.754021in}}%
\pgfpathlineto{\pgfqpoint{3.740269in}{3.735604in}}%
\pgfpathlineto{\pgfqpoint{3.753579in}{3.717462in}}%
\pgfpathlineto{\pgfqpoint{3.766886in}{3.699592in}}%
\pgfpathlineto{\pgfqpoint{3.774572in}{3.724175in}}%
\pgfpathlineto{\pgfqpoint{3.782254in}{3.749131in}}%
\pgfpathlineto{\pgfqpoint{3.789935in}{3.774468in}}%
\pgfpathlineto{\pgfqpoint{3.797613in}{3.800192in}}%
\pgfpathlineto{\pgfqpoint{3.784305in}{3.818681in}}%
\pgfpathlineto{\pgfqpoint{3.770994in}{3.837443in}}%
\pgfpathlineto{\pgfqpoint{3.757680in}{3.856481in}}%
\pgfpathlineto{\pgfqpoint{3.744363in}{3.875797in}}%
\pgfpathlineto{\pgfqpoint{3.736686in}{3.849439in}}%
\pgfpathlineto{\pgfqpoint{3.729006in}{3.823478in}}%
\pgfpathlineto{\pgfqpoint{3.721324in}{3.797906in}}%
\pgfpathlineto{\pgfqpoint{3.713639in}{3.772715in}}%
\pgfpathclose%
\pgfusepath{fill}%
\end{pgfscope}%
\begin{pgfscope}%
\pgfpathrectangle{\pgfqpoint{1.150000in}{0.150000in}}{\pgfqpoint{5.700000in}{5.700000in}}%
\pgfusepath{clip}%
\pgfsetbuttcap%
\pgfsetroundjoin%
\definecolor{currentfill}{rgb}{0.120638,0.625828,0.533488}%
\pgfsetfillcolor{currentfill}%
\pgfsetfillopacity{0.800000}%
\pgfsetlinewidth{0.000000pt}%
\definecolor{currentstroke}{rgb}{0.000000,0.000000,0.000000}%
\pgfsetstrokecolor{currentstroke}%
\pgfsetdash{}{0pt}%
\pgfpathmoveto{\pgfqpoint{3.775046in}{3.985351in}}%
\pgfpathlineto{\pgfqpoint{3.788366in}{3.965377in}}%
\pgfpathlineto{\pgfqpoint{3.801682in}{3.945681in}}%
\pgfpathlineto{\pgfqpoint{3.814994in}{3.926263in}}%
\pgfpathlineto{\pgfqpoint{3.828303in}{3.907118in}}%
\pgfpathlineto{\pgfqpoint{3.835972in}{3.934897in}}%
\pgfpathlineto{\pgfqpoint{3.843638in}{3.963110in}}%
\pgfpathlineto{\pgfqpoint{3.851303in}{3.991767in}}%
\pgfpathlineto{\pgfqpoint{3.858967in}{4.020876in}}%
\pgfpathlineto{\pgfqpoint{3.845656in}{4.040712in}}%
\pgfpathlineto{\pgfqpoint{3.832341in}{4.060825in}}%
\pgfpathlineto{\pgfqpoint{3.819022in}{4.081215in}}%
\pgfpathlineto{\pgfqpoint{3.805699in}{4.101886in}}%
\pgfpathlineto{\pgfqpoint{3.798038in}{4.072069in}}%
\pgfpathlineto{\pgfqpoint{3.790376in}{4.042713in}}%
\pgfpathlineto{\pgfqpoint{3.782712in}{4.013810in}}%
\pgfpathlineto{\pgfqpoint{3.775046in}{3.985351in}}%
\pgfpathclose%
\pgfusepath{fill}%
\end{pgfscope}%
\begin{pgfscope}%
\pgfpathrectangle{\pgfqpoint{1.150000in}{0.150000in}}{\pgfqpoint{5.700000in}{5.700000in}}%
\pgfusepath{clip}%
\pgfsetbuttcap%
\pgfsetroundjoin%
\definecolor{currentfill}{rgb}{0.177423,0.437527,0.557565}%
\pgfsetfillcolor{currentfill}%
\pgfsetfillopacity{0.800000}%
\pgfsetlinewidth{0.000000pt}%
\definecolor{currentstroke}{rgb}{0.000000,0.000000,0.000000}%
\pgfsetstrokecolor{currentstroke}%
\pgfsetdash{}{0pt}%
\pgfpathmoveto{\pgfqpoint{4.253649in}{3.427842in}}%
\pgfpathlineto{\pgfqpoint{4.266958in}{3.416721in}}%
\pgfpathlineto{\pgfqpoint{4.280270in}{3.405818in}}%
\pgfpathlineto{\pgfqpoint{4.293585in}{3.395132in}}%
\pgfpathlineto{\pgfqpoint{4.306903in}{3.384661in}}%
\pgfpathlineto{\pgfqpoint{4.314559in}{3.406482in}}%
\pgfpathlineto{\pgfqpoint{4.322216in}{3.428658in}}%
\pgfpathlineto{\pgfqpoint{4.329873in}{3.451198in}}%
\pgfpathlineto{\pgfqpoint{4.337532in}{3.474109in}}%
\pgfpathlineto{\pgfqpoint{4.324218in}{3.485243in}}%
\pgfpathlineto{\pgfqpoint{4.310908in}{3.496593in}}%
\pgfpathlineto{\pgfqpoint{4.297600in}{3.508160in}}%
\pgfpathlineto{\pgfqpoint{4.284295in}{3.519946in}}%
\pgfpathlineto{\pgfqpoint{4.276633in}{3.496358in}}%
\pgfpathlineto{\pgfqpoint{4.268972in}{3.473150in}}%
\pgfpathlineto{\pgfqpoint{4.261310in}{3.450314in}}%
\pgfpathlineto{\pgfqpoint{4.253649in}{3.427842in}}%
\pgfpathclose%
\pgfusepath{fill}%
\end{pgfscope}%
\begin{pgfscope}%
\pgfpathrectangle{\pgfqpoint{1.150000in}{0.150000in}}{\pgfqpoint{5.700000in}{5.700000in}}%
\pgfusepath{clip}%
\pgfsetbuttcap%
\pgfsetroundjoin%
\definecolor{currentfill}{rgb}{0.183898,0.422383,0.556944}%
\pgfsetfillcolor{currentfill}%
\pgfsetfillopacity{0.800000}%
\pgfsetlinewidth{0.000000pt}%
\definecolor{currentstroke}{rgb}{0.000000,0.000000,0.000000}%
\pgfsetstrokecolor{currentstroke}%
\pgfsetdash{}{0pt}%
\pgfpathmoveto{\pgfqpoint{4.169778in}{3.385594in}}%
\pgfpathlineto{\pgfqpoint{4.183081in}{3.374223in}}%
\pgfpathlineto{\pgfqpoint{4.196385in}{3.363074in}}%
\pgfpathlineto{\pgfqpoint{4.209693in}{3.352147in}}%
\pgfpathlineto{\pgfqpoint{4.223003in}{3.341439in}}%
\pgfpathlineto{\pgfqpoint{4.230665in}{3.362532in}}%
\pgfpathlineto{\pgfqpoint{4.238327in}{3.383958in}}%
\pgfpathlineto{\pgfqpoint{4.245988in}{3.405726in}}%
\pgfpathlineto{\pgfqpoint{4.253649in}{3.427842in}}%
\pgfpathlineto{\pgfqpoint{4.240343in}{3.439181in}}%
\pgfpathlineto{\pgfqpoint{4.227040in}{3.450741in}}%
\pgfpathlineto{\pgfqpoint{4.213739in}{3.462522in}}%
\pgfpathlineto{\pgfqpoint{4.200441in}{3.474526in}}%
\pgfpathlineto{\pgfqpoint{4.192776in}{3.451766in}}%
\pgfpathlineto{\pgfqpoint{4.185110in}{3.429362in}}%
\pgfpathlineto{\pgfqpoint{4.177445in}{3.407307in}}%
\pgfpathlineto{\pgfqpoint{4.169778in}{3.385594in}}%
\pgfpathclose%
\pgfusepath{fill}%
\end{pgfscope}%
\begin{pgfscope}%
\pgfpathrectangle{\pgfqpoint{1.150000in}{0.150000in}}{\pgfqpoint{5.700000in}{5.700000in}}%
\pgfusepath{clip}%
\pgfsetbuttcap%
\pgfsetroundjoin%
\definecolor{currentfill}{rgb}{0.187231,0.414746,0.556547}%
\pgfsetfillcolor{currentfill}%
\pgfsetfillopacity{0.800000}%
\pgfsetlinewidth{0.000000pt}%
\definecolor{currentstroke}{rgb}{0.000000,0.000000,0.000000}%
\pgfsetstrokecolor{currentstroke}%
\pgfsetdash{}{0pt}%
\pgfpathmoveto{\pgfqpoint{3.948825in}{3.363554in}}%
\pgfpathlineto{\pgfqpoint{3.962116in}{3.350585in}}%
\pgfpathlineto{\pgfqpoint{3.975408in}{3.337854in}}%
\pgfpathlineto{\pgfqpoint{3.988700in}{3.325361in}}%
\pgfpathlineto{\pgfqpoint{4.001994in}{3.313102in}}%
\pgfpathlineto{\pgfqpoint{4.009680in}{3.333450in}}%
\pgfpathlineto{\pgfqpoint{4.017364in}{3.354100in}}%
\pgfpathlineto{\pgfqpoint{4.025045in}{3.375058in}}%
\pgfpathlineto{\pgfqpoint{4.032725in}{3.396332in}}%
\pgfpathlineto{\pgfqpoint{4.019435in}{3.409161in}}%
\pgfpathlineto{\pgfqpoint{4.006146in}{3.422225in}}%
\pgfpathlineto{\pgfqpoint{3.992857in}{3.435527in}}%
\pgfpathlineto{\pgfqpoint{3.979569in}{3.449067in}}%
\pgfpathlineto{\pgfqpoint{3.971886in}{3.427210in}}%
\pgfpathlineto{\pgfqpoint{3.964201in}{3.405676in}}%
\pgfpathlineto{\pgfqpoint{3.956514in}{3.384460in}}%
\pgfpathlineto{\pgfqpoint{3.948825in}{3.363554in}}%
\pgfpathclose%
\pgfusepath{fill}%
\end{pgfscope}%
\begin{pgfscope}%
\pgfpathrectangle{\pgfqpoint{1.150000in}{0.150000in}}{\pgfqpoint{5.700000in}{5.700000in}}%
\pgfusepath{clip}%
\pgfsetbuttcap%
\pgfsetroundjoin%
\definecolor{currentfill}{rgb}{0.144759,0.519093,0.556572}%
\pgfsetfillcolor{currentfill}%
\pgfsetfillopacity{0.800000}%
\pgfsetlinewidth{0.000000pt}%
\definecolor{currentstroke}{rgb}{0.000000,0.000000,0.000000}%
\pgfsetstrokecolor{currentstroke}%
\pgfsetdash{}{0pt}%
\pgfpathmoveto{\pgfqpoint{3.682868in}{3.675629in}}%
\pgfpathlineto{\pgfqpoint{3.696184in}{3.657521in}}%
\pgfpathlineto{\pgfqpoint{3.709498in}{3.639689in}}%
\pgfpathlineto{\pgfqpoint{3.722808in}{3.622131in}}%
\pgfpathlineto{\pgfqpoint{3.736116in}{3.604844in}}%
\pgfpathlineto{\pgfqpoint{3.743813in}{3.628007in}}%
\pgfpathlineto{\pgfqpoint{3.751507in}{3.651515in}}%
\pgfpathlineto{\pgfqpoint{3.759198in}{3.675374in}}%
\pgfpathlineto{\pgfqpoint{3.766886in}{3.699592in}}%
\pgfpathlineto{\pgfqpoint{3.753579in}{3.717462in}}%
\pgfpathlineto{\pgfqpoint{3.740269in}{3.735604in}}%
\pgfpathlineto{\pgfqpoint{3.726956in}{3.754021in}}%
\pgfpathlineto{\pgfqpoint{3.713639in}{3.772715in}}%
\pgfpathlineto{\pgfqpoint{3.705951in}{3.747900in}}%
\pgfpathlineto{\pgfqpoint{3.698260in}{3.723451in}}%
\pgfpathlineto{\pgfqpoint{3.690565in}{3.699363in}}%
\pgfpathlineto{\pgfqpoint{3.682868in}{3.675629in}}%
\pgfpathclose%
\pgfusepath{fill}%
\end{pgfscope}%
\begin{pgfscope}%
\pgfpathrectangle{\pgfqpoint{1.150000in}{0.150000in}}{\pgfqpoint{5.700000in}{5.700000in}}%
\pgfusepath{clip}%
\pgfsetbuttcap%
\pgfsetroundjoin%
\definecolor{currentfill}{rgb}{0.169646,0.456262,0.558030}%
\pgfsetfillcolor{currentfill}%
\pgfsetfillopacity{0.800000}%
\pgfsetlinewidth{0.000000pt}%
\definecolor{currentstroke}{rgb}{0.000000,0.000000,0.000000}%
\pgfsetstrokecolor{currentstroke}%
\pgfsetdash{}{0pt}%
\pgfpathmoveto{\pgfqpoint{4.337532in}{3.474109in}}%
\pgfpathlineto{\pgfqpoint{4.350848in}{3.463189in}}%
\pgfpathlineto{\pgfqpoint{4.364168in}{3.452483in}}%
\pgfpathlineto{\pgfqpoint{4.377492in}{3.441990in}}%
\pgfpathlineto{\pgfqpoint{4.390819in}{3.431708in}}%
\pgfpathlineto{\pgfqpoint{4.398473in}{3.454315in}}%
\pgfpathlineto{\pgfqpoint{4.406129in}{3.477301in}}%
\pgfpathlineto{\pgfqpoint{4.413785in}{3.500675in}}%
\pgfpathlineto{\pgfqpoint{4.421444in}{3.524445in}}%
\pgfpathlineto{\pgfqpoint{4.408121in}{3.535422in}}%
\pgfpathlineto{\pgfqpoint{4.394803in}{3.546611in}}%
\pgfpathlineto{\pgfqpoint{4.381487in}{3.558014in}}%
\pgfpathlineto{\pgfqpoint{4.368174in}{3.569630in}}%
\pgfpathlineto{\pgfqpoint{4.360512in}{3.545152in}}%
\pgfpathlineto{\pgfqpoint{4.352850in}{3.521077in}}%
\pgfpathlineto{\pgfqpoint{4.345190in}{3.497399in}}%
\pgfpathlineto{\pgfqpoint{4.337532in}{3.474109in}}%
\pgfpathclose%
\pgfusepath{fill}%
\end{pgfscope}%
\begin{pgfscope}%
\pgfpathrectangle{\pgfqpoint{1.150000in}{0.150000in}}{\pgfqpoint{5.700000in}{5.700000in}}%
\pgfusepath{clip}%
\pgfsetbuttcap%
\pgfsetroundjoin%
\definecolor{currentfill}{rgb}{0.188923,0.410910,0.556326}%
\pgfsetfillcolor{currentfill}%
\pgfsetfillopacity{0.800000}%
\pgfsetlinewidth{0.000000pt}%
\definecolor{currentstroke}{rgb}{0.000000,0.000000,0.000000}%
\pgfsetstrokecolor{currentstroke}%
\pgfsetdash{}{0pt}%
\pgfpathmoveto{\pgfqpoint{4.085899in}{3.347345in}}%
\pgfpathlineto{\pgfqpoint{4.099197in}{3.335673in}}%
\pgfpathlineto{\pgfqpoint{4.112497in}{3.324229in}}%
\pgfpathlineto{\pgfqpoint{4.125798in}{3.313009in}}%
\pgfpathlineto{\pgfqpoint{4.139102in}{3.302015in}}%
\pgfpathlineto{\pgfqpoint{4.146773in}{3.322433in}}%
\pgfpathlineto{\pgfqpoint{4.154443in}{3.343165in}}%
\pgfpathlineto{\pgfqpoint{4.162111in}{3.364216in}}%
\pgfpathlineto{\pgfqpoint{4.169778in}{3.385594in}}%
\pgfpathlineto{\pgfqpoint{4.156478in}{3.397189in}}%
\pgfpathlineto{\pgfqpoint{4.143180in}{3.409008in}}%
\pgfpathlineto{\pgfqpoint{4.129885in}{3.421054in}}%
\pgfpathlineto{\pgfqpoint{4.116590in}{3.433327in}}%
\pgfpathlineto{\pgfqpoint{4.108920in}{3.411336in}}%
\pgfpathlineto{\pgfqpoint{4.101248in}{3.389680in}}%
\pgfpathlineto{\pgfqpoint{4.093574in}{3.368352in}}%
\pgfpathlineto{\pgfqpoint{4.085899in}{3.347345in}}%
\pgfpathclose%
\pgfusepath{fill}%
\end{pgfscope}%
\begin{pgfscope}%
\pgfpathrectangle{\pgfqpoint{1.150000in}{0.150000in}}{\pgfqpoint{5.700000in}{5.700000in}}%
\pgfusepath{clip}%
\pgfsetbuttcap%
\pgfsetroundjoin%
\definecolor{currentfill}{rgb}{0.150476,0.504369,0.557430}%
\pgfsetfillcolor{currentfill}%
\pgfsetfillopacity{0.800000}%
\pgfsetlinewidth{0.000000pt}%
\definecolor{currentstroke}{rgb}{0.000000,0.000000,0.000000}%
\pgfsetstrokecolor{currentstroke}%
\pgfsetdash{}{0pt}%
\pgfpathmoveto{\pgfqpoint{4.452099in}{3.623654in}}%
\pgfpathlineto{\pgfqpoint{4.465420in}{3.612158in}}%
\pgfpathlineto{\pgfqpoint{4.478745in}{3.600872in}}%
\pgfpathlineto{\pgfqpoint{4.492073in}{3.589796in}}%
\pgfpathlineto{\pgfqpoint{4.505406in}{3.578927in}}%
\pgfpathlineto{\pgfqpoint{4.513071in}{3.604056in}}%
\pgfpathlineto{\pgfqpoint{4.520740in}{3.629625in}}%
\pgfpathlineto{\pgfqpoint{4.528413in}{3.655643in}}%
\pgfpathlineto{\pgfqpoint{4.515084in}{3.667080in}}%
\pgfpathlineto{\pgfqpoint{4.501758in}{3.678725in}}%
\pgfpathlineto{\pgfqpoint{4.488437in}{3.690581in}}%
\pgfpathlineto{\pgfqpoint{4.475118in}{3.702646in}}%
\pgfpathlineto{\pgfqpoint{4.467442in}{3.675861in}}%
\pgfpathlineto{\pgfqpoint{4.459769in}{3.649533in}}%
\pgfpathlineto{\pgfqpoint{4.452099in}{3.623654in}}%
\pgfpathclose%
\pgfusepath{fill}%
\end{pgfscope}%
\begin{pgfscope}%
\pgfpathrectangle{\pgfqpoint{1.150000in}{0.150000in}}{\pgfqpoint{5.700000in}{5.700000in}}%
\pgfusepath{clip}%
\pgfsetbuttcap%
\pgfsetroundjoin%
\definecolor{currentfill}{rgb}{0.174274,0.445044,0.557792}%
\pgfsetfillcolor{currentfill}%
\pgfsetfillopacity{0.800000}%
\pgfsetlinewidth{0.000000pt}%
\definecolor{currentstroke}{rgb}{0.000000,0.000000,0.000000}%
\pgfsetstrokecolor{currentstroke}%
\pgfsetdash{}{0pt}%
\pgfpathmoveto{\pgfqpoint{3.758509in}{3.451217in}}%
\pgfpathlineto{\pgfqpoint{3.771808in}{3.435802in}}%
\pgfpathlineto{\pgfqpoint{3.785105in}{3.420645in}}%
\pgfpathlineto{\pgfqpoint{3.798401in}{3.405745in}}%
\pgfpathlineto{\pgfqpoint{3.811696in}{3.391100in}}%
\pgfpathlineto{\pgfqpoint{3.819402in}{3.411887in}}%
\pgfpathlineto{\pgfqpoint{3.827105in}{3.432974in}}%
\pgfpathlineto{\pgfqpoint{3.834805in}{3.454369in}}%
\pgfpathlineto{\pgfqpoint{3.842502in}{3.476078in}}%
\pgfpathlineto{\pgfqpoint{3.829209in}{3.491266in}}%
\pgfpathlineto{\pgfqpoint{3.815915in}{3.506710in}}%
\pgfpathlineto{\pgfqpoint{3.802620in}{3.522411in}}%
\pgfpathlineto{\pgfqpoint{3.789323in}{3.538371in}}%
\pgfpathlineto{\pgfqpoint{3.781624in}{3.516105in}}%
\pgfpathlineto{\pgfqpoint{3.773922in}{3.494162in}}%
\pgfpathlineto{\pgfqpoint{3.766217in}{3.472535in}}%
\pgfpathlineto{\pgfqpoint{3.758509in}{3.451217in}}%
\pgfpathclose%
\pgfusepath{fill}%
\end{pgfscope}%
\begin{pgfscope}%
\pgfpathrectangle{\pgfqpoint{1.150000in}{0.150000in}}{\pgfqpoint{5.700000in}{5.700000in}}%
\pgfusepath{clip}%
\pgfsetbuttcap%
\pgfsetroundjoin%
\definecolor{currentfill}{rgb}{0.165117,0.467423,0.558141}%
\pgfsetfillcolor{currentfill}%
\pgfsetfillopacity{0.800000}%
\pgfsetlinewidth{0.000000pt}%
\definecolor{currentstroke}{rgb}{0.000000,0.000000,0.000000}%
\pgfsetstrokecolor{currentstroke}%
\pgfsetdash{}{0pt}%
\pgfpathmoveto{\pgfqpoint{3.705295in}{3.515503in}}%
\pgfpathlineto{\pgfqpoint{3.718602in}{3.499034in}}%
\pgfpathlineto{\pgfqpoint{3.731906in}{3.482831in}}%
\pgfpathlineto{\pgfqpoint{3.745208in}{3.466893in}}%
\pgfpathlineto{\pgfqpoint{3.758509in}{3.451217in}}%
\pgfpathlineto{\pgfqpoint{3.766217in}{3.472535in}}%
\pgfpathlineto{\pgfqpoint{3.773922in}{3.494162in}}%
\pgfpathlineto{\pgfqpoint{3.781624in}{3.516105in}}%
\pgfpathlineto{\pgfqpoint{3.789323in}{3.538371in}}%
\pgfpathlineto{\pgfqpoint{3.776024in}{3.554592in}}%
\pgfpathlineto{\pgfqpoint{3.762724in}{3.571077in}}%
\pgfpathlineto{\pgfqpoint{3.749421in}{3.587827in}}%
\pgfpathlineto{\pgfqpoint{3.736116in}{3.604844in}}%
\pgfpathlineto{\pgfqpoint{3.728416in}{3.582019in}}%
\pgfpathlineto{\pgfqpoint{3.720712in}{3.559524in}}%
\pgfpathlineto{\pgfqpoint{3.713005in}{3.537355in}}%
\pgfpathlineto{\pgfqpoint{3.705295in}{3.515503in}}%
\pgfpathclose%
\pgfusepath{fill}%
\end{pgfscope}%
\begin{pgfscope}%
\pgfpathrectangle{\pgfqpoint{1.150000in}{0.150000in}}{\pgfqpoint{5.700000in}{5.700000in}}%
\pgfusepath{clip}%
\pgfsetbuttcap%
\pgfsetroundjoin%
\definecolor{currentfill}{rgb}{0.157851,0.683765,0.501686}%
\pgfsetfillcolor{currentfill}%
\pgfsetfillopacity{0.800000}%
\pgfsetlinewidth{0.000000pt}%
\definecolor{currentstroke}{rgb}{0.000000,0.000000,0.000000}%
\pgfsetstrokecolor{currentstroke}%
\pgfsetdash{}{0pt}%
\pgfpathmoveto{\pgfqpoint{3.889615in}{4.141999in}}%
\pgfpathlineto{\pgfqpoint{3.902927in}{4.121706in}}%
\pgfpathlineto{\pgfqpoint{3.916236in}{4.101687in}}%
\pgfpathlineto{\pgfqpoint{3.929541in}{4.081937in}}%
\pgfpathlineto{\pgfqpoint{3.942845in}{4.062456in}}%
\pgfpathlineto{\pgfqpoint{3.950509in}{4.093202in}}%
\pgfpathlineto{\pgfqpoint{3.958173in}{4.124444in}}%
\pgfpathlineto{\pgfqpoint{3.965838in}{4.156189in}}%
\pgfpathlineto{\pgfqpoint{3.973503in}{4.188449in}}%
\pgfpathlineto{\pgfqpoint{3.960196in}{4.208694in}}%
\pgfpathlineto{\pgfqpoint{3.946886in}{4.229209in}}%
\pgfpathlineto{\pgfqpoint{3.933573in}{4.249996in}}%
\pgfpathlineto{\pgfqpoint{3.920256in}{4.271057in}}%
\pgfpathlineto{\pgfqpoint{3.912596in}{4.238016in}}%
\pgfpathlineto{\pgfqpoint{3.904935in}{4.205499in}}%
\pgfpathlineto{\pgfqpoint{3.897275in}{4.173496in}}%
\pgfpathlineto{\pgfqpoint{3.889615in}{4.141999in}}%
\pgfpathclose%
\pgfusepath{fill}%
\end{pgfscope}%
\begin{pgfscope}%
\pgfpathrectangle{\pgfqpoint{1.150000in}{0.150000in}}{\pgfqpoint{5.700000in}{5.700000in}}%
\pgfusepath{clip}%
\pgfsetbuttcap%
\pgfsetroundjoin%
\definecolor{currentfill}{rgb}{0.183898,0.422383,0.556944}%
\pgfsetfillcolor{currentfill}%
\pgfsetfillopacity{0.800000}%
\pgfsetlinewidth{0.000000pt}%
\definecolor{currentstroke}{rgb}{0.000000,0.000000,0.000000}%
\pgfsetstrokecolor{currentstroke}%
\pgfsetdash{}{0pt}%
\pgfpathmoveto{\pgfqpoint{3.811696in}{3.391100in}}%
\pgfpathlineto{\pgfqpoint{3.824990in}{3.376708in}}%
\pgfpathlineto{\pgfqpoint{3.838284in}{3.362568in}}%
\pgfpathlineto{\pgfqpoint{3.851577in}{3.348676in}}%
\pgfpathlineto{\pgfqpoint{3.864870in}{3.335033in}}%
\pgfpathlineto{\pgfqpoint{3.872573in}{3.355290in}}%
\pgfpathlineto{\pgfqpoint{3.880273in}{3.375840in}}%
\pgfpathlineto{\pgfqpoint{3.887970in}{3.396689in}}%
\pgfpathlineto{\pgfqpoint{3.895665in}{3.417844in}}%
\pgfpathlineto{\pgfqpoint{3.882375in}{3.432028in}}%
\pgfpathlineto{\pgfqpoint{3.869085in}{3.446461in}}%
\pgfpathlineto{\pgfqpoint{3.855794in}{3.461144in}}%
\pgfpathlineto{\pgfqpoint{3.842502in}{3.476078in}}%
\pgfpathlineto{\pgfqpoint{3.834805in}{3.454369in}}%
\pgfpathlineto{\pgfqpoint{3.827105in}{3.432974in}}%
\pgfpathlineto{\pgfqpoint{3.819402in}{3.411887in}}%
\pgfpathlineto{\pgfqpoint{3.811696in}{3.391100in}}%
\pgfpathclose%
\pgfusepath{fill}%
\end{pgfscope}%
\begin{pgfscope}%
\pgfpathrectangle{\pgfqpoint{1.150000in}{0.150000in}}{\pgfqpoint{5.700000in}{5.700000in}}%
\pgfusepath{clip}%
\pgfsetbuttcap%
\pgfsetroundjoin%
\definecolor{currentfill}{rgb}{0.123463,0.581687,0.547445}%
\pgfsetfillcolor{currentfill}%
\pgfsetfillopacity{0.800000}%
\pgfsetlinewidth{0.000000pt}%
\definecolor{currentstroke}{rgb}{0.000000,0.000000,0.000000}%
\pgfsetstrokecolor{currentstroke}%
\pgfsetdash{}{0pt}%
\pgfpathmoveto{\pgfqpoint{3.660334in}{3.850307in}}%
\pgfpathlineto{\pgfqpoint{3.673667in}{3.830482in}}%
\pgfpathlineto{\pgfqpoint{3.686995in}{3.810943in}}%
\pgfpathlineto{\pgfqpoint{3.700319in}{3.791689in}}%
\pgfpathlineto{\pgfqpoint{3.713639in}{3.772715in}}%
\pgfpathlineto{\pgfqpoint{3.721324in}{3.797906in}}%
\pgfpathlineto{\pgfqpoint{3.729006in}{3.823478in}}%
\pgfpathlineto{\pgfqpoint{3.736686in}{3.849439in}}%
\pgfpathlineto{\pgfqpoint{3.744363in}{3.875797in}}%
\pgfpathlineto{\pgfqpoint{3.731041in}{3.895393in}}%
\pgfpathlineto{\pgfqpoint{3.717716in}{3.915272in}}%
\pgfpathlineto{\pgfqpoint{3.704387in}{3.935436in}}%
\pgfpathlineto{\pgfqpoint{3.691053in}{3.955887in}}%
\pgfpathlineto{\pgfqpoint{3.683378in}{3.928891in}}%
\pgfpathlineto{\pgfqpoint{3.675700in}{3.902300in}}%
\pgfpathlineto{\pgfqpoint{3.668019in}{3.876108in}}%
\pgfpathlineto{\pgfqpoint{3.660334in}{3.850307in}}%
\pgfpathclose%
\pgfusepath{fill}%
\end{pgfscope}%
\begin{pgfscope}%
\pgfpathrectangle{\pgfqpoint{1.150000in}{0.150000in}}{\pgfqpoint{5.700000in}{5.700000in}}%
\pgfusepath{clip}%
\pgfsetbuttcap%
\pgfsetroundjoin%
\definecolor{currentfill}{rgb}{0.162142,0.474838,0.558140}%
\pgfsetfillcolor{currentfill}%
\pgfsetfillopacity{0.800000}%
\pgfsetlinewidth{0.000000pt}%
\definecolor{currentstroke}{rgb}{0.000000,0.000000,0.000000}%
\pgfsetstrokecolor{currentstroke}%
\pgfsetdash{}{0pt}%
\pgfpathmoveto{\pgfqpoint{4.421444in}{3.524445in}}%
\pgfpathlineto{\pgfqpoint{4.434770in}{3.513678in}}%
\pgfpathlineto{\pgfqpoint{4.448099in}{3.503121in}}%
\pgfpathlineto{\pgfqpoint{4.461433in}{3.492772in}}%
\pgfpathlineto{\pgfqpoint{4.474770in}{3.482631in}}%
\pgfpathlineto{\pgfqpoint{4.482426in}{3.506090in}}%
\pgfpathlineto{\pgfqpoint{4.490083in}{3.529953in}}%
\pgfpathlineto{\pgfqpoint{4.497743in}{3.554229in}}%
\pgfpathlineto{\pgfqpoint{4.505406in}{3.578927in}}%
\pgfpathlineto{\pgfqpoint{4.492073in}{3.589796in}}%
\pgfpathlineto{\pgfqpoint{4.478745in}{3.600872in}}%
\pgfpathlineto{\pgfqpoint{4.465420in}{3.612158in}}%
\pgfpathlineto{\pgfqpoint{4.452099in}{3.623654in}}%
\pgfpathlineto{\pgfqpoint{4.444431in}{3.598214in}}%
\pgfpathlineto{\pgfqpoint{4.436767in}{3.573206in}}%
\pgfpathlineto{\pgfqpoint{4.429104in}{3.548619in}}%
\pgfpathlineto{\pgfqpoint{4.421444in}{3.524445in}}%
\pgfpathclose%
\pgfusepath{fill}%
\end{pgfscope}%
\begin{pgfscope}%
\pgfpathrectangle{\pgfqpoint{1.150000in}{0.150000in}}{\pgfqpoint{5.700000in}{5.700000in}}%
\pgfusepath{clip}%
\pgfsetbuttcap%
\pgfsetroundjoin%
\definecolor{currentfill}{rgb}{0.170948,0.694384,0.493803}%
\pgfsetfillcolor{currentfill}%
\pgfsetfillopacity{0.800000}%
\pgfsetlinewidth{0.000000pt}%
\definecolor{currentstroke}{rgb}{0.000000,0.000000,0.000000}%
\pgfsetstrokecolor{currentstroke}%
\pgfsetdash{}{0pt}%
\pgfpathmoveto{\pgfqpoint{3.973503in}{4.188449in}}%
\pgfpathlineto{\pgfqpoint{3.986808in}{4.168471in}}%
\pgfpathlineto{\pgfqpoint{4.000109in}{4.148759in}}%
\pgfpathlineto{\pgfqpoint{4.013409in}{4.129310in}}%
\pgfpathlineto{\pgfqpoint{4.026706in}{4.110122in}}%
\pgfpathlineto{\pgfqpoint{4.034376in}{4.142120in}}%
\pgfpathlineto{\pgfqpoint{4.042047in}{4.174642in}}%
\pgfpathlineto{\pgfqpoint{4.049720in}{4.207696in}}%
\pgfpathlineto{\pgfqpoint{4.036420in}{4.227478in}}%
\pgfpathlineto{\pgfqpoint{4.023117in}{4.247523in}}%
\pgfpathlineto{\pgfqpoint{4.009812in}{4.267832in}}%
\pgfpathlineto{\pgfqpoint{3.996504in}{4.288408in}}%
\pgfpathlineto{\pgfqpoint{3.988836in}{4.254548in}}%
\pgfpathlineto{\pgfqpoint{3.981169in}{4.221232in}}%
\pgfpathlineto{\pgfqpoint{3.973503in}{4.188449in}}%
\pgfpathclose%
\pgfusepath{fill}%
\end{pgfscope}%
\begin{pgfscope}%
\pgfpathrectangle{\pgfqpoint{1.150000in}{0.150000in}}{\pgfqpoint{5.700000in}{5.700000in}}%
\pgfusepath{clip}%
\pgfsetbuttcap%
\pgfsetroundjoin%
\definecolor{currentfill}{rgb}{0.119699,0.618490,0.536347}%
\pgfsetfillcolor{currentfill}%
\pgfsetfillopacity{0.800000}%
\pgfsetlinewidth{0.000000pt}%
\definecolor{currentstroke}{rgb}{0.000000,0.000000,0.000000}%
\pgfsetstrokecolor{currentstroke}%
\pgfsetdash{}{0pt}%
\pgfpathmoveto{\pgfqpoint{3.691053in}{3.955887in}}%
\pgfpathlineto{\pgfqpoint{3.704387in}{3.935436in}}%
\pgfpathlineto{\pgfqpoint{3.717716in}{3.915272in}}%
\pgfpathlineto{\pgfqpoint{3.731041in}{3.895393in}}%
\pgfpathlineto{\pgfqpoint{3.744363in}{3.875797in}}%
\pgfpathlineto{\pgfqpoint{3.752037in}{3.902560in}}%
\pgfpathlineto{\pgfqpoint{3.759709in}{3.929734in}}%
\pgfpathlineto{\pgfqpoint{3.767379in}{3.957329in}}%
\pgfpathlineto{\pgfqpoint{3.775046in}{3.985351in}}%
\pgfpathlineto{\pgfqpoint{3.761723in}{4.005607in}}%
\pgfpathlineto{\pgfqpoint{3.748395in}{4.026147in}}%
\pgfpathlineto{\pgfqpoint{3.735063in}{4.046973in}}%
\pgfpathlineto{\pgfqpoint{3.721727in}{4.068087in}}%
\pgfpathlineto{\pgfqpoint{3.714062in}{4.039389in}}%
\pgfpathlineto{\pgfqpoint{3.706395in}{4.011128in}}%
\pgfpathlineto{\pgfqpoint{3.698725in}{3.983297in}}%
\pgfpathlineto{\pgfqpoint{3.691053in}{3.955887in}}%
\pgfpathclose%
\pgfusepath{fill}%
\end{pgfscope}%
\begin{pgfscope}%
\pgfpathrectangle{\pgfqpoint{1.150000in}{0.150000in}}{\pgfqpoint{5.700000in}{5.700000in}}%
\pgfusepath{clip}%
\pgfsetbuttcap%
\pgfsetroundjoin%
\definecolor{currentfill}{rgb}{0.143303,0.669459,0.511215}%
\pgfsetfillcolor{currentfill}%
\pgfsetfillopacity{0.800000}%
\pgfsetlinewidth{0.000000pt}%
\definecolor{currentstroke}{rgb}{0.000000,0.000000,0.000000}%
\pgfsetstrokecolor{currentstroke}%
\pgfsetdash{}{0pt}%
\pgfpathmoveto{\pgfqpoint{3.805699in}{4.101886in}}%
\pgfpathlineto{\pgfqpoint{3.819022in}{4.081215in}}%
\pgfpathlineto{\pgfqpoint{3.832341in}{4.060825in}}%
\pgfpathlineto{\pgfqpoint{3.845656in}{4.040712in}}%
\pgfpathlineto{\pgfqpoint{3.858967in}{4.020876in}}%
\pgfpathlineto{\pgfqpoint{3.866630in}{4.050444in}}%
\pgfpathlineto{\pgfqpoint{3.874293in}{4.080482in}}%
\pgfpathlineto{\pgfqpoint{3.881954in}{4.110997in}}%
\pgfpathlineto{\pgfqpoint{3.889615in}{4.141999in}}%
\pgfpathlineto{\pgfqpoint{3.876299in}{4.162566in}}%
\pgfpathlineto{\pgfqpoint{3.862980in}{4.183410in}}%
\pgfpathlineto{\pgfqpoint{3.849657in}{4.204534in}}%
\pgfpathlineto{\pgfqpoint{3.836330in}{4.225939in}}%
\pgfpathlineto{\pgfqpoint{3.828674in}{4.194190in}}%
\pgfpathlineto{\pgfqpoint{3.821017in}{4.162938in}}%
\pgfpathlineto{\pgfqpoint{3.813358in}{4.132172in}}%
\pgfpathlineto{\pgfqpoint{3.805699in}{4.101886in}}%
\pgfpathclose%
\pgfusepath{fill}%
\end{pgfscope}%
\begin{pgfscope}%
\pgfpathrectangle{\pgfqpoint{1.150000in}{0.150000in}}{\pgfqpoint{5.700000in}{5.700000in}}%
\pgfusepath{clip}%
\pgfsetbuttcap%
\pgfsetroundjoin%
\definecolor{currentfill}{rgb}{0.194100,0.399323,0.555565}%
\pgfsetfillcolor{currentfill}%
\pgfsetfillopacity{0.800000}%
\pgfsetlinewidth{0.000000pt}%
\definecolor{currentstroke}{rgb}{0.000000,0.000000,0.000000}%
\pgfsetstrokecolor{currentstroke}%
\pgfsetdash{}{0pt}%
\pgfpathmoveto{\pgfqpoint{4.001994in}{3.313102in}}%
\pgfpathlineto{\pgfqpoint{4.015289in}{3.301078in}}%
\pgfpathlineto{\pgfqpoint{4.028585in}{3.289285in}}%
\pgfpathlineto{\pgfqpoint{4.041883in}{3.277724in}}%
\pgfpathlineto{\pgfqpoint{4.055183in}{3.266392in}}%
\pgfpathlineto{\pgfqpoint{4.062865in}{3.286183in}}%
\pgfpathlineto{\pgfqpoint{4.070545in}{3.306268in}}%
\pgfpathlineto{\pgfqpoint{4.078223in}{3.326653in}}%
\pgfpathlineto{\pgfqpoint{4.085899in}{3.347345in}}%
\pgfpathlineto{\pgfqpoint{4.072604in}{3.359246in}}%
\pgfpathlineto{\pgfqpoint{4.059309in}{3.371376in}}%
\pgfpathlineto{\pgfqpoint{4.046017in}{3.383738in}}%
\pgfpathlineto{\pgfqpoint{4.032725in}{3.396332in}}%
\pgfpathlineto{\pgfqpoint{4.025045in}{3.375058in}}%
\pgfpathlineto{\pgfqpoint{4.017364in}{3.354100in}}%
\pgfpathlineto{\pgfqpoint{4.009680in}{3.333450in}}%
\pgfpathlineto{\pgfqpoint{4.001994in}{3.313102in}}%
\pgfpathclose%
\pgfusepath{fill}%
\end{pgfscope}%
\begin{pgfscope}%
\pgfpathrectangle{\pgfqpoint{1.150000in}{0.150000in}}{\pgfqpoint{5.700000in}{5.700000in}}%
\pgfusepath{clip}%
\pgfsetbuttcap%
\pgfsetroundjoin%
\definecolor{currentfill}{rgb}{0.135066,0.544853,0.554029}%
\pgfsetfillcolor{currentfill}%
\pgfsetfillopacity{0.800000}%
\pgfsetlinewidth{0.000000pt}%
\definecolor{currentstroke}{rgb}{0.000000,0.000000,0.000000}%
\pgfsetstrokecolor{currentstroke}%
\pgfsetdash{}{0pt}%
\pgfpathmoveto{\pgfqpoint{3.629564in}{3.750867in}}%
\pgfpathlineto{\pgfqpoint{3.642896in}{3.731632in}}%
\pgfpathlineto{\pgfqpoint{3.656224in}{3.712682in}}%
\pgfpathlineto{\pgfqpoint{3.669548in}{3.694015in}}%
\pgfpathlineto{\pgfqpoint{3.682868in}{3.675629in}}%
\pgfpathlineto{\pgfqpoint{3.690565in}{3.699363in}}%
\pgfpathlineto{\pgfqpoint{3.698260in}{3.723451in}}%
\pgfpathlineto{\pgfqpoint{3.705951in}{3.747900in}}%
\pgfpathlineto{\pgfqpoint{3.713639in}{3.772715in}}%
\pgfpathlineto{\pgfqpoint{3.700319in}{3.791689in}}%
\pgfpathlineto{\pgfqpoint{3.686995in}{3.810943in}}%
\pgfpathlineto{\pgfqpoint{3.673667in}{3.830482in}}%
\pgfpathlineto{\pgfqpoint{3.660334in}{3.850307in}}%
\pgfpathlineto{\pgfqpoint{3.652647in}{3.824890in}}%
\pgfpathlineto{\pgfqpoint{3.644956in}{3.799848in}}%
\pgfpathlineto{\pgfqpoint{3.637262in}{3.775177in}}%
\pgfpathlineto{\pgfqpoint{3.629564in}{3.750867in}}%
\pgfpathclose%
\pgfusepath{fill}%
\end{pgfscope}%
\begin{pgfscope}%
\pgfpathrectangle{\pgfqpoint{1.150000in}{0.150000in}}{\pgfqpoint{5.700000in}{5.700000in}}%
\pgfusepath{clip}%
\pgfsetbuttcap%
\pgfsetroundjoin%
\definecolor{currentfill}{rgb}{0.156270,0.489624,0.557936}%
\pgfsetfillcolor{currentfill}%
\pgfsetfillopacity{0.800000}%
\pgfsetlinewidth{0.000000pt}%
\definecolor{currentstroke}{rgb}{0.000000,0.000000,0.000000}%
\pgfsetstrokecolor{currentstroke}%
\pgfsetdash{}{0pt}%
\pgfpathmoveto{\pgfqpoint{3.652042in}{3.584090in}}%
\pgfpathlineto{\pgfqpoint{3.665360in}{3.566532in}}%
\pgfpathlineto{\pgfqpoint{3.678674in}{3.549250in}}%
\pgfpathlineto{\pgfqpoint{3.691986in}{3.532241in}}%
\pgfpathlineto{\pgfqpoint{3.705295in}{3.515503in}}%
\pgfpathlineto{\pgfqpoint{3.713005in}{3.537355in}}%
\pgfpathlineto{\pgfqpoint{3.720712in}{3.559524in}}%
\pgfpathlineto{\pgfqpoint{3.728416in}{3.582019in}}%
\pgfpathlineto{\pgfqpoint{3.736116in}{3.604844in}}%
\pgfpathlineto{\pgfqpoint{3.722808in}{3.622131in}}%
\pgfpathlineto{\pgfqpoint{3.709498in}{3.639689in}}%
\pgfpathlineto{\pgfqpoint{3.696184in}{3.657521in}}%
\pgfpathlineto{\pgfqpoint{3.682868in}{3.675629in}}%
\pgfpathlineto{\pgfqpoint{3.675167in}{3.652241in}}%
\pgfpathlineto{\pgfqpoint{3.667462in}{3.629193in}}%
\pgfpathlineto{\pgfqpoint{3.659754in}{3.606478in}}%
\pgfpathlineto{\pgfqpoint{3.652042in}{3.584090in}}%
\pgfpathclose%
\pgfusepath{fill}%
\end{pgfscope}%
\begin{pgfscope}%
\pgfpathrectangle{\pgfqpoint{1.150000in}{0.150000in}}{\pgfqpoint{5.700000in}{5.700000in}}%
\pgfusepath{clip}%
\pgfsetbuttcap%
\pgfsetroundjoin%
\definecolor{currentfill}{rgb}{0.192357,0.403199,0.555836}%
\pgfsetfillcolor{currentfill}%
\pgfsetfillopacity{0.800000}%
\pgfsetlinewidth{0.000000pt}%
\definecolor{currentstroke}{rgb}{0.000000,0.000000,0.000000}%
\pgfsetstrokecolor{currentstroke}%
\pgfsetdash{}{0pt}%
\pgfpathmoveto{\pgfqpoint{3.864870in}{3.335033in}}%
\pgfpathlineto{\pgfqpoint{3.878163in}{3.321635in}}%
\pgfpathlineto{\pgfqpoint{3.891456in}{3.308481in}}%
\pgfpathlineto{\pgfqpoint{3.904749in}{3.295571in}}%
\pgfpathlineto{\pgfqpoint{3.918043in}{3.282901in}}%
\pgfpathlineto{\pgfqpoint{3.925743in}{3.302631in}}%
\pgfpathlineto{\pgfqpoint{3.933439in}{3.322646in}}%
\pgfpathlineto{\pgfqpoint{3.941134in}{3.342951in}}%
\pgfpathlineto{\pgfqpoint{3.948825in}{3.363554in}}%
\pgfpathlineto{\pgfqpoint{3.935535in}{3.376762in}}%
\pgfpathlineto{\pgfqpoint{3.922245in}{3.390212in}}%
\pgfpathlineto{\pgfqpoint{3.908955in}{3.403906in}}%
\pgfpathlineto{\pgfqpoint{3.895665in}{3.417844in}}%
\pgfpathlineto{\pgfqpoint{3.887970in}{3.396689in}}%
\pgfpathlineto{\pgfqpoint{3.880273in}{3.375840in}}%
\pgfpathlineto{\pgfqpoint{3.872573in}{3.355290in}}%
\pgfpathlineto{\pgfqpoint{3.864870in}{3.335033in}}%
\pgfpathclose%
\pgfusepath{fill}%
\end{pgfscope}%
\begin{pgfscope}%
\pgfpathrectangle{\pgfqpoint{1.150000in}{0.150000in}}{\pgfqpoint{5.700000in}{5.700000in}}%
\pgfusepath{clip}%
\pgfsetbuttcap%
\pgfsetroundjoin%
\definecolor{currentfill}{rgb}{0.188923,0.410910,0.556326}%
\pgfsetfillcolor{currentfill}%
\pgfsetfillopacity{0.800000}%
\pgfsetlinewidth{0.000000pt}%
\definecolor{currentstroke}{rgb}{0.000000,0.000000,0.000000}%
\pgfsetstrokecolor{currentstroke}%
\pgfsetdash{}{0pt}%
\pgfpathmoveto{\pgfqpoint{4.223003in}{3.341439in}}%
\pgfpathlineto{\pgfqpoint{4.236316in}{3.330949in}}%
\pgfpathlineto{\pgfqpoint{4.249633in}{3.320677in}}%
\pgfpathlineto{\pgfqpoint{4.262952in}{3.310621in}}%
\pgfpathlineto{\pgfqpoint{4.276275in}{3.300780in}}%
\pgfpathlineto{\pgfqpoint{4.283932in}{3.321255in}}%
\pgfpathlineto{\pgfqpoint{4.291589in}{3.342056in}}%
\pgfpathlineto{\pgfqpoint{4.299246in}{3.363188in}}%
\pgfpathlineto{\pgfqpoint{4.306903in}{3.384661in}}%
\pgfpathlineto{\pgfqpoint{4.293585in}{3.395132in}}%
\pgfpathlineto{\pgfqpoint{4.280270in}{3.405818in}}%
\pgfpathlineto{\pgfqpoint{4.266958in}{3.416721in}}%
\pgfpathlineto{\pgfqpoint{4.253649in}{3.427842in}}%
\pgfpathlineto{\pgfqpoint{4.245988in}{3.405726in}}%
\pgfpathlineto{\pgfqpoint{4.238327in}{3.383958in}}%
\pgfpathlineto{\pgfqpoint{4.230665in}{3.362532in}}%
\pgfpathlineto{\pgfqpoint{4.223003in}{3.341439in}}%
\pgfpathclose%
\pgfusepath{fill}%
\end{pgfscope}%
\begin{pgfscope}%
\pgfpathrectangle{\pgfqpoint{1.150000in}{0.150000in}}{\pgfqpoint{5.700000in}{5.700000in}}%
\pgfusepath{clip}%
\pgfsetbuttcap%
\pgfsetroundjoin%
\definecolor{currentfill}{rgb}{0.182256,0.426184,0.557120}%
\pgfsetfillcolor{currentfill}%
\pgfsetfillopacity{0.800000}%
\pgfsetlinewidth{0.000000pt}%
\definecolor{currentstroke}{rgb}{0.000000,0.000000,0.000000}%
\pgfsetstrokecolor{currentstroke}%
\pgfsetdash{}{0pt}%
\pgfpathmoveto{\pgfqpoint{4.306903in}{3.384661in}}%
\pgfpathlineto{\pgfqpoint{4.320224in}{3.374405in}}%
\pgfpathlineto{\pgfqpoint{4.333549in}{3.364361in}}%
\pgfpathlineto{\pgfqpoint{4.346878in}{3.354530in}}%
\pgfpathlineto{\pgfqpoint{4.360210in}{3.344909in}}%
\pgfpathlineto{\pgfqpoint{4.367861in}{3.366080in}}%
\pgfpathlineto{\pgfqpoint{4.375513in}{3.387598in}}%
\pgfpathlineto{\pgfqpoint{4.383166in}{3.409472in}}%
\pgfpathlineto{\pgfqpoint{4.390819in}{3.431708in}}%
\pgfpathlineto{\pgfqpoint{4.377492in}{3.441990in}}%
\pgfpathlineto{\pgfqpoint{4.364168in}{3.452483in}}%
\pgfpathlineto{\pgfqpoint{4.350848in}{3.463189in}}%
\pgfpathlineto{\pgfqpoint{4.337532in}{3.474109in}}%
\pgfpathlineto{\pgfqpoint{4.329873in}{3.451198in}}%
\pgfpathlineto{\pgfqpoint{4.322216in}{3.428658in}}%
\pgfpathlineto{\pgfqpoint{4.314559in}{3.406482in}}%
\pgfpathlineto{\pgfqpoint{4.306903in}{3.384661in}}%
\pgfpathclose%
\pgfusepath{fill}%
\end{pgfscope}%
\begin{pgfscope}%
\pgfpathrectangle{\pgfqpoint{1.150000in}{0.150000in}}{\pgfqpoint{5.700000in}{5.700000in}}%
\pgfusepath{clip}%
\pgfsetbuttcap%
\pgfsetroundjoin%
\definecolor{currentfill}{rgb}{0.132268,0.655014,0.519661}%
\pgfsetfillcolor{currentfill}%
\pgfsetfillopacity{0.800000}%
\pgfsetlinewidth{0.000000pt}%
\definecolor{currentstroke}{rgb}{0.000000,0.000000,0.000000}%
\pgfsetstrokecolor{currentstroke}%
\pgfsetdash{}{0pt}%
\pgfpathmoveto{\pgfqpoint{3.721727in}{4.068087in}}%
\pgfpathlineto{\pgfqpoint{3.735063in}{4.046973in}}%
\pgfpathlineto{\pgfqpoint{3.748395in}{4.026147in}}%
\pgfpathlineto{\pgfqpoint{3.761723in}{4.005607in}}%
\pgfpathlineto{\pgfqpoint{3.775046in}{3.985351in}}%
\pgfpathlineto{\pgfqpoint{3.782712in}{4.013810in}}%
\pgfpathlineto{\pgfqpoint{3.790376in}{4.042713in}}%
\pgfpathlineto{\pgfqpoint{3.798038in}{4.072069in}}%
\pgfpathlineto{\pgfqpoint{3.805699in}{4.101886in}}%
\pgfpathlineto{\pgfqpoint{3.792372in}{4.122839in}}%
\pgfpathlineto{\pgfqpoint{3.779041in}{4.144078in}}%
\pgfpathlineto{\pgfqpoint{3.765706in}{4.165603in}}%
\pgfpathlineto{\pgfqpoint{3.752365in}{4.187419in}}%
\pgfpathlineto{\pgfqpoint{3.744709in}{4.156889in}}%
\pgfpathlineto{\pgfqpoint{3.737050in}{4.126829in}}%
\pgfpathlineto{\pgfqpoint{3.729390in}{4.097231in}}%
\pgfpathlineto{\pgfqpoint{3.721727in}{4.068087in}}%
\pgfpathclose%
\pgfusepath{fill}%
\end{pgfscope}%
\begin{pgfscope}%
\pgfpathrectangle{\pgfqpoint{1.150000in}{0.150000in}}{\pgfqpoint{5.700000in}{5.700000in}}%
\pgfusepath{clip}%
\pgfsetbuttcap%
\pgfsetroundjoin%
\definecolor{currentfill}{rgb}{0.195860,0.395433,0.555276}%
\pgfsetfillcolor{currentfill}%
\pgfsetfillopacity{0.800000}%
\pgfsetlinewidth{0.000000pt}%
\definecolor{currentstroke}{rgb}{0.000000,0.000000,0.000000}%
\pgfsetstrokecolor{currentstroke}%
\pgfsetdash{}{0pt}%
\pgfpathmoveto{\pgfqpoint{4.139102in}{3.302015in}}%
\pgfpathlineto{\pgfqpoint{4.152409in}{3.291243in}}%
\pgfpathlineto{\pgfqpoint{4.165718in}{3.280694in}}%
\pgfpathlineto{\pgfqpoint{4.179030in}{3.270365in}}%
\pgfpathlineto{\pgfqpoint{4.192345in}{3.260255in}}%
\pgfpathlineto{\pgfqpoint{4.200011in}{3.280087in}}%
\pgfpathlineto{\pgfqpoint{4.207676in}{3.300224in}}%
\pgfpathlineto{\pgfqpoint{4.215340in}{3.320672in}}%
\pgfpathlineto{\pgfqpoint{4.223003in}{3.341439in}}%
\pgfpathlineto{\pgfqpoint{4.209693in}{3.352147in}}%
\pgfpathlineto{\pgfqpoint{4.196385in}{3.363074in}}%
\pgfpathlineto{\pgfqpoint{4.183081in}{3.374223in}}%
\pgfpathlineto{\pgfqpoint{4.169778in}{3.385594in}}%
\pgfpathlineto{\pgfqpoint{4.162111in}{3.364216in}}%
\pgfpathlineto{\pgfqpoint{4.154443in}{3.343165in}}%
\pgfpathlineto{\pgfqpoint{4.146773in}{3.322433in}}%
\pgfpathlineto{\pgfqpoint{4.139102in}{3.302015in}}%
\pgfpathclose%
\pgfusepath{fill}%
\end{pgfscope}%
\begin{pgfscope}%
\pgfpathrectangle{\pgfqpoint{1.150000in}{0.150000in}}{\pgfqpoint{5.700000in}{5.700000in}}%
\pgfusepath{clip}%
\pgfsetbuttcap%
\pgfsetroundjoin%
\definecolor{currentfill}{rgb}{0.174274,0.445044,0.557792}%
\pgfsetfillcolor{currentfill}%
\pgfsetfillopacity{0.800000}%
\pgfsetlinewidth{0.000000pt}%
\definecolor{currentstroke}{rgb}{0.000000,0.000000,0.000000}%
\pgfsetstrokecolor{currentstroke}%
\pgfsetdash{}{0pt}%
\pgfpathmoveto{\pgfqpoint{4.390819in}{3.431708in}}%
\pgfpathlineto{\pgfqpoint{4.404150in}{3.421636in}}%
\pgfpathlineto{\pgfqpoint{4.417485in}{3.411773in}}%
\pgfpathlineto{\pgfqpoint{4.430824in}{3.402119in}}%
\pgfpathlineto{\pgfqpoint{4.444167in}{3.392671in}}%
\pgfpathlineto{\pgfqpoint{4.451816in}{3.414597in}}%
\pgfpathlineto{\pgfqpoint{4.459466in}{3.436893in}}%
\pgfpathlineto{\pgfqpoint{4.467117in}{3.459569in}}%
\pgfpathlineto{\pgfqpoint{4.474770in}{3.482631in}}%
\pgfpathlineto{\pgfqpoint{4.461433in}{3.492772in}}%
\pgfpathlineto{\pgfqpoint{4.448099in}{3.503121in}}%
\pgfpathlineto{\pgfqpoint{4.434770in}{3.513678in}}%
\pgfpathlineto{\pgfqpoint{4.421444in}{3.524445in}}%
\pgfpathlineto{\pgfqpoint{4.413785in}{3.500675in}}%
\pgfpathlineto{\pgfqpoint{4.406129in}{3.477301in}}%
\pgfpathlineto{\pgfqpoint{4.398473in}{3.454315in}}%
\pgfpathlineto{\pgfqpoint{4.390819in}{3.431708in}}%
\pgfpathclose%
\pgfusepath{fill}%
\end{pgfscope}%
\begin{pgfscope}%
\pgfpathrectangle{\pgfqpoint{1.150000in}{0.150000in}}{\pgfqpoint{5.700000in}{5.700000in}}%
\pgfusepath{clip}%
\pgfsetbuttcap%
\pgfsetroundjoin%
\definecolor{currentfill}{rgb}{0.156270,0.489624,0.557936}%
\pgfsetfillcolor{currentfill}%
\pgfsetfillopacity{0.800000}%
\pgfsetlinewidth{0.000000pt}%
\definecolor{currentstroke}{rgb}{0.000000,0.000000,0.000000}%
\pgfsetstrokecolor{currentstroke}%
\pgfsetdash{}{0pt}%
\pgfpathmoveto{\pgfqpoint{4.505406in}{3.578927in}}%
\pgfpathlineto{\pgfqpoint{4.518742in}{3.568266in}}%
\pgfpathlineto{\pgfqpoint{4.532082in}{3.557810in}}%
\pgfpathlineto{\pgfqpoint{4.545427in}{3.547560in}}%
\pgfpathlineto{\pgfqpoint{4.558776in}{3.537513in}}%
\pgfpathlineto{\pgfqpoint{4.566436in}{3.561895in}}%
\pgfpathlineto{\pgfqpoint{4.574100in}{3.586707in}}%
\pgfpathlineto{\pgfqpoint{4.581767in}{3.611960in}}%
\pgfpathlineto{\pgfqpoint{4.568422in}{3.622573in}}%
\pgfpathlineto{\pgfqpoint{4.555082in}{3.633390in}}%
\pgfpathlineto{\pgfqpoint{4.541745in}{3.644413in}}%
\pgfpathlineto{\pgfqpoint{4.528413in}{3.655643in}}%
\pgfpathlineto{\pgfqpoint{4.520740in}{3.629625in}}%
\pgfpathlineto{\pgfqpoint{4.513071in}{3.604056in}}%
\pgfpathlineto{\pgfqpoint{4.505406in}{3.578927in}}%
\pgfpathclose%
\pgfusepath{fill}%
\end{pgfscope}%
\begin{pgfscope}%
\pgfpathrectangle{\pgfqpoint{1.150000in}{0.150000in}}{\pgfqpoint{5.700000in}{5.700000in}}%
\pgfusepath{clip}%
\pgfsetbuttcap%
\pgfsetroundjoin%
\definecolor{currentfill}{rgb}{0.146180,0.515413,0.556823}%
\pgfsetfillcolor{currentfill}%
\pgfsetfillopacity{0.800000}%
\pgfsetlinewidth{0.000000pt}%
\definecolor{currentstroke}{rgb}{0.000000,0.000000,0.000000}%
\pgfsetstrokecolor{currentstroke}%
\pgfsetdash{}{0pt}%
\pgfpathmoveto{\pgfqpoint{3.598735in}{3.657116in}}%
\pgfpathlineto{\pgfqpoint{3.612068in}{3.638435in}}%
\pgfpathlineto{\pgfqpoint{3.625396in}{3.620039in}}%
\pgfpathlineto{\pgfqpoint{3.638721in}{3.601924in}}%
\pgfpathlineto{\pgfqpoint{3.652042in}{3.584090in}}%
\pgfpathlineto{\pgfqpoint{3.659754in}{3.606478in}}%
\pgfpathlineto{\pgfqpoint{3.667462in}{3.629193in}}%
\pgfpathlineto{\pgfqpoint{3.675167in}{3.652241in}}%
\pgfpathlineto{\pgfqpoint{3.682868in}{3.675629in}}%
\pgfpathlineto{\pgfqpoint{3.669548in}{3.694015in}}%
\pgfpathlineto{\pgfqpoint{3.656224in}{3.712682in}}%
\pgfpathlineto{\pgfqpoint{3.642896in}{3.731632in}}%
\pgfpathlineto{\pgfqpoint{3.629564in}{3.750867in}}%
\pgfpathlineto{\pgfqpoint{3.621863in}{3.726913in}}%
\pgfpathlineto{\pgfqpoint{3.614157in}{3.703308in}}%
\pgfpathlineto{\pgfqpoint{3.606448in}{3.680044in}}%
\pgfpathlineto{\pgfqpoint{3.598735in}{3.657116in}}%
\pgfpathclose%
\pgfusepath{fill}%
\end{pgfscope}%
\begin{pgfscope}%
\pgfpathrectangle{\pgfqpoint{1.150000in}{0.150000in}}{\pgfqpoint{5.700000in}{5.700000in}}%
\pgfusepath{clip}%
\pgfsetbuttcap%
\pgfsetroundjoin%
\definecolor{currentfill}{rgb}{0.199430,0.387607,0.554642}%
\pgfsetfillcolor{currentfill}%
\pgfsetfillopacity{0.800000}%
\pgfsetlinewidth{0.000000pt}%
\definecolor{currentstroke}{rgb}{0.000000,0.000000,0.000000}%
\pgfsetstrokecolor{currentstroke}%
\pgfsetdash{}{0pt}%
\pgfpathmoveto{\pgfqpoint{3.918043in}{3.282901in}}%
\pgfpathlineto{\pgfqpoint{3.931337in}{3.270471in}}%
\pgfpathlineto{\pgfqpoint{3.944633in}{3.258278in}}%
\pgfpathlineto{\pgfqpoint{3.957929in}{3.246322in}}%
\pgfpathlineto{\pgfqpoint{3.971227in}{3.234601in}}%
\pgfpathlineto{\pgfqpoint{3.978922in}{3.253806in}}%
\pgfpathlineto{\pgfqpoint{3.986615in}{3.273287in}}%
\pgfpathlineto{\pgfqpoint{3.994306in}{3.293050in}}%
\pgfpathlineto{\pgfqpoint{4.001994in}{3.313102in}}%
\pgfpathlineto{\pgfqpoint{3.988700in}{3.325361in}}%
\pgfpathlineto{\pgfqpoint{3.975408in}{3.337854in}}%
\pgfpathlineto{\pgfqpoint{3.962116in}{3.350585in}}%
\pgfpathlineto{\pgfqpoint{3.948825in}{3.363554in}}%
\pgfpathlineto{\pgfqpoint{3.941134in}{3.342951in}}%
\pgfpathlineto{\pgfqpoint{3.933439in}{3.322646in}}%
\pgfpathlineto{\pgfqpoint{3.925743in}{3.302631in}}%
\pgfpathlineto{\pgfqpoint{3.918043in}{3.282901in}}%
\pgfpathclose%
\pgfusepath{fill}%
\end{pgfscope}%
\begin{pgfscope}%
\pgfpathrectangle{\pgfqpoint{1.150000in}{0.150000in}}{\pgfqpoint{5.700000in}{5.700000in}}%
\pgfusepath{clip}%
\pgfsetbuttcap%
\pgfsetroundjoin%
\definecolor{currentfill}{rgb}{0.201239,0.383670,0.554294}%
\pgfsetfillcolor{currentfill}%
\pgfsetfillopacity{0.800000}%
\pgfsetlinewidth{0.000000pt}%
\definecolor{currentstroke}{rgb}{0.000000,0.000000,0.000000}%
\pgfsetstrokecolor{currentstroke}%
\pgfsetdash{}{0pt}%
\pgfpathmoveto{\pgfqpoint{4.055183in}{3.266392in}}%
\pgfpathlineto{\pgfqpoint{4.068485in}{3.255288in}}%
\pgfpathlineto{\pgfqpoint{4.081789in}{3.244411in}}%
\pgfpathlineto{\pgfqpoint{4.095095in}{3.233759in}}%
\pgfpathlineto{\pgfqpoint{4.108403in}{3.223331in}}%
\pgfpathlineto{\pgfqpoint{4.116081in}{3.242567in}}%
\pgfpathlineto{\pgfqpoint{4.123756in}{3.262089in}}%
\pgfpathlineto{\pgfqpoint{4.131430in}{3.281902in}}%
\pgfpathlineto{\pgfqpoint{4.139102in}{3.302015in}}%
\pgfpathlineto{\pgfqpoint{4.125798in}{3.313009in}}%
\pgfpathlineto{\pgfqpoint{4.112497in}{3.324229in}}%
\pgfpathlineto{\pgfqpoint{4.099197in}{3.335673in}}%
\pgfpathlineto{\pgfqpoint{4.085899in}{3.347345in}}%
\pgfpathlineto{\pgfqpoint{4.078223in}{3.326653in}}%
\pgfpathlineto{\pgfqpoint{4.070545in}{3.306268in}}%
\pgfpathlineto{\pgfqpoint{4.062865in}{3.286183in}}%
\pgfpathlineto{\pgfqpoint{4.055183in}{3.266392in}}%
\pgfpathclose%
\pgfusepath{fill}%
\end{pgfscope}%
\begin{pgfscope}%
\pgfpathrectangle{\pgfqpoint{1.150000in}{0.150000in}}{\pgfqpoint{5.700000in}{5.700000in}}%
\pgfusepath{clip}%
\pgfsetbuttcap%
\pgfsetroundjoin%
\definecolor{currentfill}{rgb}{0.119423,0.611141,0.538982}%
\pgfsetfillcolor{currentfill}%
\pgfsetfillopacity{0.800000}%
\pgfsetlinewidth{0.000000pt}%
\definecolor{currentstroke}{rgb}{0.000000,0.000000,0.000000}%
\pgfsetstrokecolor{currentstroke}%
\pgfsetdash{}{0pt}%
\pgfpathmoveto{\pgfqpoint{3.606958in}{3.932520in}}%
\pgfpathlineto{\pgfqpoint{3.620310in}{3.911525in}}%
\pgfpathlineto{\pgfqpoint{3.633656in}{3.890826in}}%
\pgfpathlineto{\pgfqpoint{3.646998in}{3.870421in}}%
\pgfpathlineto{\pgfqpoint{3.660334in}{3.850307in}}%
\pgfpathlineto{\pgfqpoint{3.668019in}{3.876108in}}%
\pgfpathlineto{\pgfqpoint{3.675700in}{3.902300in}}%
\pgfpathlineto{\pgfqpoint{3.683378in}{3.928891in}}%
\pgfpathlineto{\pgfqpoint{3.691053in}{3.955887in}}%
\pgfpathlineto{\pgfqpoint{3.677714in}{3.976628in}}%
\pgfpathlineto{\pgfqpoint{3.664371in}{3.997661in}}%
\pgfpathlineto{\pgfqpoint{3.651023in}{4.018990in}}%
\pgfpathlineto{\pgfqpoint{3.637669in}{4.040616in}}%
\pgfpathlineto{\pgfqpoint{3.629996in}{4.012977in}}%
\pgfpathlineto{\pgfqpoint{3.622320in}{3.985753in}}%
\pgfpathlineto{\pgfqpoint{3.614641in}{3.958936in}}%
\pgfpathlineto{\pgfqpoint{3.606958in}{3.932520in}}%
\pgfpathclose%
\pgfusepath{fill}%
\end{pgfscope}%
\begin{pgfscope}%
\pgfpathrectangle{\pgfqpoint{1.150000in}{0.150000in}}{\pgfqpoint{5.700000in}{5.700000in}}%
\pgfusepath{clip}%
\pgfsetbuttcap%
\pgfsetroundjoin%
\definecolor{currentfill}{rgb}{0.166617,0.463708,0.558119}%
\pgfsetfillcolor{currentfill}%
\pgfsetfillopacity{0.800000}%
\pgfsetlinewidth{0.000000pt}%
\definecolor{currentstroke}{rgb}{0.000000,0.000000,0.000000}%
\pgfsetstrokecolor{currentstroke}%
\pgfsetdash{}{0pt}%
\pgfpathmoveto{\pgfqpoint{4.474770in}{3.482631in}}%
\pgfpathlineto{\pgfqpoint{4.488112in}{3.472697in}}%
\pgfpathlineto{\pgfqpoint{4.501458in}{3.462968in}}%
\pgfpathlineto{\pgfqpoint{4.514809in}{3.453444in}}%
\pgfpathlineto{\pgfqpoint{4.528164in}{3.444124in}}%
\pgfpathlineto{\pgfqpoint{4.535813in}{3.466868in}}%
\pgfpathlineto{\pgfqpoint{4.543465in}{3.490009in}}%
\pgfpathlineto{\pgfqpoint{4.551119in}{3.513555in}}%
\pgfpathlineto{\pgfqpoint{4.558776in}{3.537513in}}%
\pgfpathlineto{\pgfqpoint{4.545427in}{3.547560in}}%
\pgfpathlineto{\pgfqpoint{4.532082in}{3.557810in}}%
\pgfpathlineto{\pgfqpoint{4.518742in}{3.568266in}}%
\pgfpathlineto{\pgfqpoint{4.505406in}{3.578927in}}%
\pgfpathlineto{\pgfqpoint{4.497743in}{3.554229in}}%
\pgfpathlineto{\pgfqpoint{4.490083in}{3.529953in}}%
\pgfpathlineto{\pgfqpoint{4.482426in}{3.506090in}}%
\pgfpathlineto{\pgfqpoint{4.474770in}{3.482631in}}%
\pgfpathclose%
\pgfusepath{fill}%
\end{pgfscope}%
\begin{pgfscope}%
\pgfpathrectangle{\pgfqpoint{1.150000in}{0.150000in}}{\pgfqpoint{5.700000in}{5.700000in}}%
\pgfusepath{clip}%
\pgfsetbuttcap%
\pgfsetroundjoin%
\definecolor{currentfill}{rgb}{0.177423,0.437527,0.557565}%
\pgfsetfillcolor{currentfill}%
\pgfsetfillopacity{0.800000}%
\pgfsetlinewidth{0.000000pt}%
\definecolor{currentstroke}{rgb}{0.000000,0.000000,0.000000}%
\pgfsetstrokecolor{currentstroke}%
\pgfsetdash{}{0pt}%
\pgfpathmoveto{\pgfqpoint{3.674417in}{3.431151in}}%
\pgfpathlineto{\pgfqpoint{3.687726in}{3.415195in}}%
\pgfpathlineto{\pgfqpoint{3.701032in}{3.399506in}}%
\pgfpathlineto{\pgfqpoint{3.714337in}{3.384080in}}%
\pgfpathlineto{\pgfqpoint{3.727640in}{3.368916in}}%
\pgfpathlineto{\pgfqpoint{3.735363in}{3.389058in}}%
\pgfpathlineto{\pgfqpoint{3.743082in}{3.409485in}}%
\pgfpathlineto{\pgfqpoint{3.750797in}{3.430202in}}%
\pgfpathlineto{\pgfqpoint{3.758509in}{3.451217in}}%
\pgfpathlineto{\pgfqpoint{3.745208in}{3.466893in}}%
\pgfpathlineto{\pgfqpoint{3.731906in}{3.482831in}}%
\pgfpathlineto{\pgfqpoint{3.718602in}{3.499034in}}%
\pgfpathlineto{\pgfqpoint{3.705295in}{3.515503in}}%
\pgfpathlineto{\pgfqpoint{3.697581in}{3.493963in}}%
\pgfpathlineto{\pgfqpoint{3.689863in}{3.472729in}}%
\pgfpathlineto{\pgfqpoint{3.682142in}{3.451793in}}%
\pgfpathlineto{\pgfqpoint{3.674417in}{3.431151in}}%
\pgfpathclose%
\pgfusepath{fill}%
\end{pgfscope}%
\begin{pgfscope}%
\pgfpathrectangle{\pgfqpoint{1.150000in}{0.150000in}}{\pgfqpoint{5.700000in}{5.700000in}}%
\pgfusepath{clip}%
\pgfsetbuttcap%
\pgfsetroundjoin%
\definecolor{currentfill}{rgb}{0.185556,0.418570,0.556753}%
\pgfsetfillcolor{currentfill}%
\pgfsetfillopacity{0.800000}%
\pgfsetlinewidth{0.000000pt}%
\definecolor{currentstroke}{rgb}{0.000000,0.000000,0.000000}%
\pgfsetstrokecolor{currentstroke}%
\pgfsetdash{}{0pt}%
\pgfpathmoveto{\pgfqpoint{3.727640in}{3.368916in}}%
\pgfpathlineto{\pgfqpoint{3.740942in}{3.354012in}}%
\pgfpathlineto{\pgfqpoint{3.754242in}{3.339366in}}%
\pgfpathlineto{\pgfqpoint{3.767541in}{3.324976in}}%
\pgfpathlineto{\pgfqpoint{3.780839in}{3.310840in}}%
\pgfpathlineto{\pgfqpoint{3.788559in}{3.330485in}}%
\pgfpathlineto{\pgfqpoint{3.796275in}{3.350406in}}%
\pgfpathlineto{\pgfqpoint{3.803987in}{3.370609in}}%
\pgfpathlineto{\pgfqpoint{3.811696in}{3.391100in}}%
\pgfpathlineto{\pgfqpoint{3.798401in}{3.405745in}}%
\pgfpathlineto{\pgfqpoint{3.785105in}{3.420645in}}%
\pgfpathlineto{\pgfqpoint{3.771808in}{3.435802in}}%
\pgfpathlineto{\pgfqpoint{3.758509in}{3.451217in}}%
\pgfpathlineto{\pgfqpoint{3.750797in}{3.430202in}}%
\pgfpathlineto{\pgfqpoint{3.743082in}{3.409485in}}%
\pgfpathlineto{\pgfqpoint{3.735363in}{3.389058in}}%
\pgfpathlineto{\pgfqpoint{3.727640in}{3.368916in}}%
\pgfpathclose%
\pgfusepath{fill}%
\end{pgfscope}%
\begin{pgfscope}%
\pgfpathrectangle{\pgfqpoint{1.150000in}{0.150000in}}{\pgfqpoint{5.700000in}{5.700000in}}%
\pgfusepath{clip}%
\pgfsetbuttcap%
\pgfsetroundjoin%
\definecolor{currentfill}{rgb}{0.214000,0.722114,0.469588}%
\pgfsetfillcolor{currentfill}%
\pgfsetfillopacity{0.800000}%
\pgfsetlinewidth{0.000000pt}%
\definecolor{currentstroke}{rgb}{0.000000,0.000000,0.000000}%
\pgfsetstrokecolor{currentstroke}%
\pgfsetdash{}{0pt}%
\pgfpathmoveto{\pgfqpoint{3.920256in}{4.271057in}}%
\pgfpathlineto{\pgfqpoint{3.933573in}{4.249996in}}%
\pgfpathlineto{\pgfqpoint{3.946886in}{4.229209in}}%
\pgfpathlineto{\pgfqpoint{3.960196in}{4.208694in}}%
\pgfpathlineto{\pgfqpoint{3.973503in}{4.188449in}}%
\pgfpathlineto{\pgfqpoint{3.981169in}{4.221232in}}%
\pgfpathlineto{\pgfqpoint{3.988836in}{4.254548in}}%
\pgfpathlineto{\pgfqpoint{3.996504in}{4.288408in}}%
\pgfpathlineto{\pgfqpoint{3.983193in}{4.309252in}}%
\pgfpathlineto{\pgfqpoint{3.969879in}{4.330367in}}%
\pgfpathlineto{\pgfqpoint{3.956561in}{4.351755in}}%
\pgfpathlineto{\pgfqpoint{3.943240in}{4.373419in}}%
\pgfpathlineto{\pgfqpoint{3.935578in}{4.338748in}}%
\pgfpathlineto{\pgfqpoint{3.927917in}{4.304631in}}%
\pgfpathlineto{\pgfqpoint{3.920256in}{4.271057in}}%
\pgfpathclose%
\pgfusepath{fill}%
\end{pgfscope}%
\begin{pgfscope}%
\pgfpathrectangle{\pgfqpoint{1.150000in}{0.150000in}}{\pgfqpoint{5.700000in}{5.700000in}}%
\pgfusepath{clip}%
\pgfsetbuttcap%
\pgfsetroundjoin%
\definecolor{currentfill}{rgb}{0.196571,0.711827,0.479221}%
\pgfsetfillcolor{currentfill}%
\pgfsetfillopacity{0.800000}%
\pgfsetlinewidth{0.000000pt}%
\definecolor{currentstroke}{rgb}{0.000000,0.000000,0.000000}%
\pgfsetstrokecolor{currentstroke}%
\pgfsetdash{}{0pt}%
\pgfpathmoveto{\pgfqpoint{3.836330in}{4.225939in}}%
\pgfpathlineto{\pgfqpoint{3.849657in}{4.204534in}}%
\pgfpathlineto{\pgfqpoint{3.862980in}{4.183410in}}%
\pgfpathlineto{\pgfqpoint{3.876299in}{4.162566in}}%
\pgfpathlineto{\pgfqpoint{3.889615in}{4.141999in}}%
\pgfpathlineto{\pgfqpoint{3.897275in}{4.173496in}}%
\pgfpathlineto{\pgfqpoint{3.904935in}{4.205499in}}%
\pgfpathlineto{\pgfqpoint{3.912596in}{4.238016in}}%
\pgfpathlineto{\pgfqpoint{3.920256in}{4.271057in}}%
\pgfpathlineto{\pgfqpoint{3.906935in}{4.292394in}}%
\pgfpathlineto{\pgfqpoint{3.893611in}{4.314009in}}%
\pgfpathlineto{\pgfqpoint{3.880283in}{4.335906in}}%
\pgfpathlineto{\pgfqpoint{3.866950in}{4.358086in}}%
\pgfpathlineto{\pgfqpoint{3.859296in}{4.324258in}}%
\pgfpathlineto{\pgfqpoint{3.851641in}{4.290964in}}%
\pgfpathlineto{\pgfqpoint{3.843986in}{4.258194in}}%
\pgfpathlineto{\pgfqpoint{3.836330in}{4.225939in}}%
\pgfpathclose%
\pgfusepath{fill}%
\end{pgfscope}%
\begin{pgfscope}%
\pgfpathrectangle{\pgfqpoint{1.150000in}{0.150000in}}{\pgfqpoint{5.700000in}{5.700000in}}%
\pgfusepath{clip}%
\pgfsetbuttcap%
\pgfsetroundjoin%
\definecolor{currentfill}{rgb}{0.125394,0.574318,0.549086}%
\pgfsetfillcolor{currentfill}%
\pgfsetfillopacity{0.800000}%
\pgfsetlinewidth{0.000000pt}%
\definecolor{currentstroke}{rgb}{0.000000,0.000000,0.000000}%
\pgfsetstrokecolor{currentstroke}%
\pgfsetdash{}{0pt}%
\pgfpathmoveto{\pgfqpoint{3.576191in}{3.830710in}}%
\pgfpathlineto{\pgfqpoint{3.589542in}{3.810309in}}%
\pgfpathlineto{\pgfqpoint{3.602887in}{3.790203in}}%
\pgfpathlineto{\pgfqpoint{3.616228in}{3.770390in}}%
\pgfpathlineto{\pgfqpoint{3.629564in}{3.750867in}}%
\pgfpathlineto{\pgfqpoint{3.637262in}{3.775177in}}%
\pgfpathlineto{\pgfqpoint{3.644956in}{3.799848in}}%
\pgfpathlineto{\pgfqpoint{3.652647in}{3.824890in}}%
\pgfpathlineto{\pgfqpoint{3.660334in}{3.850307in}}%
\pgfpathlineto{\pgfqpoint{3.646998in}{3.870421in}}%
\pgfpathlineto{\pgfqpoint{3.633656in}{3.890826in}}%
\pgfpathlineto{\pgfqpoint{3.620310in}{3.911525in}}%
\pgfpathlineto{\pgfqpoint{3.606958in}{3.932520in}}%
\pgfpathlineto{\pgfqpoint{3.599272in}{3.906496in}}%
\pgfpathlineto{\pgfqpoint{3.591582in}{3.880858in}}%
\pgfpathlineto{\pgfqpoint{3.583889in}{3.855599in}}%
\pgfpathlineto{\pgfqpoint{3.576191in}{3.830710in}}%
\pgfpathclose%
\pgfusepath{fill}%
\end{pgfscope}%
\begin{pgfscope}%
\pgfpathrectangle{\pgfqpoint{1.150000in}{0.150000in}}{\pgfqpoint{5.700000in}{5.700000in}}%
\pgfusepath{clip}%
\pgfsetbuttcap%
\pgfsetroundjoin%
\definecolor{currentfill}{rgb}{0.128087,0.647749,0.523491}%
\pgfsetfillcolor{currentfill}%
\pgfsetfillopacity{0.800000}%
\pgfsetlinewidth{0.000000pt}%
\definecolor{currentstroke}{rgb}{0.000000,0.000000,0.000000}%
\pgfsetstrokecolor{currentstroke}%
\pgfsetdash{}{0pt}%
\pgfpathmoveto{\pgfqpoint{3.637669in}{4.040616in}}%
\pgfpathlineto{\pgfqpoint{3.651023in}{4.018990in}}%
\pgfpathlineto{\pgfqpoint{3.664371in}{3.997661in}}%
\pgfpathlineto{\pgfqpoint{3.677714in}{3.976628in}}%
\pgfpathlineto{\pgfqpoint{3.691053in}{3.955887in}}%
\pgfpathlineto{\pgfqpoint{3.698725in}{3.983297in}}%
\pgfpathlineto{\pgfqpoint{3.706395in}{4.011128in}}%
\pgfpathlineto{\pgfqpoint{3.714062in}{4.039389in}}%
\pgfpathlineto{\pgfqpoint{3.721727in}{4.068087in}}%
\pgfpathlineto{\pgfqpoint{3.708386in}{4.089493in}}%
\pgfpathlineto{\pgfqpoint{3.695040in}{4.111192in}}%
\pgfpathlineto{\pgfqpoint{3.681688in}{4.133188in}}%
\pgfpathlineto{\pgfqpoint{3.668331in}{4.155483in}}%
\pgfpathlineto{\pgfqpoint{3.660670in}{4.126104in}}%
\pgfpathlineto{\pgfqpoint{3.653006in}{4.097172in}}%
\pgfpathlineto{\pgfqpoint{3.645339in}{4.068678in}}%
\pgfpathlineto{\pgfqpoint{3.637669in}{4.040616in}}%
\pgfpathclose%
\pgfusepath{fill}%
\end{pgfscope}%
\begin{pgfscope}%
\pgfpathrectangle{\pgfqpoint{1.150000in}{0.150000in}}{\pgfqpoint{5.700000in}{5.700000in}}%
\pgfusepath{clip}%
\pgfsetbuttcap%
\pgfsetroundjoin%
\definecolor{currentfill}{rgb}{0.168126,0.459988,0.558082}%
\pgfsetfillcolor{currentfill}%
\pgfsetfillopacity{0.800000}%
\pgfsetlinewidth{0.000000pt}%
\definecolor{currentstroke}{rgb}{0.000000,0.000000,0.000000}%
\pgfsetstrokecolor{currentstroke}%
\pgfsetdash{}{0pt}%
\pgfpathmoveto{\pgfqpoint{3.621155in}{3.497677in}}%
\pgfpathlineto{\pgfqpoint{3.634475in}{3.480636in}}%
\pgfpathlineto{\pgfqpoint{3.647792in}{3.463869in}}%
\pgfpathlineto{\pgfqpoint{3.661105in}{3.447375in}}%
\pgfpathlineto{\pgfqpoint{3.674417in}{3.431151in}}%
\pgfpathlineto{\pgfqpoint{3.682142in}{3.451793in}}%
\pgfpathlineto{\pgfqpoint{3.689863in}{3.472729in}}%
\pgfpathlineto{\pgfqpoint{3.697581in}{3.493963in}}%
\pgfpathlineto{\pgfqpoint{3.705295in}{3.515503in}}%
\pgfpathlineto{\pgfqpoint{3.691986in}{3.532241in}}%
\pgfpathlineto{\pgfqpoint{3.678674in}{3.549250in}}%
\pgfpathlineto{\pgfqpoint{3.665360in}{3.566532in}}%
\pgfpathlineto{\pgfqpoint{3.652042in}{3.584090in}}%
\pgfpathlineto{\pgfqpoint{3.644326in}{3.562022in}}%
\pgfpathlineto{\pgfqpoint{3.636607in}{3.540268in}}%
\pgfpathlineto{\pgfqpoint{3.628883in}{3.518821in}}%
\pgfpathlineto{\pgfqpoint{3.621155in}{3.497677in}}%
\pgfpathclose%
\pgfusepath{fill}%
\end{pgfscope}%
\begin{pgfscope}%
\pgfpathrectangle{\pgfqpoint{1.150000in}{0.150000in}}{\pgfqpoint{5.700000in}{5.700000in}}%
\pgfusepath{clip}%
\pgfsetbuttcap%
\pgfsetroundjoin%
\definecolor{currentfill}{rgb}{0.175707,0.697900,0.491033}%
\pgfsetfillcolor{currentfill}%
\pgfsetfillopacity{0.800000}%
\pgfsetlinewidth{0.000000pt}%
\definecolor{currentstroke}{rgb}{0.000000,0.000000,0.000000}%
\pgfsetstrokecolor{currentstroke}%
\pgfsetdash{}{0pt}%
\pgfpathmoveto{\pgfqpoint{3.752365in}{4.187419in}}%
\pgfpathlineto{\pgfqpoint{3.765706in}{4.165603in}}%
\pgfpathlineto{\pgfqpoint{3.779041in}{4.144078in}}%
\pgfpathlineto{\pgfqpoint{3.792372in}{4.122839in}}%
\pgfpathlineto{\pgfqpoint{3.805699in}{4.101886in}}%
\pgfpathlineto{\pgfqpoint{3.813358in}{4.132172in}}%
\pgfpathlineto{\pgfqpoint{3.821017in}{4.162938in}}%
\pgfpathlineto{\pgfqpoint{3.828674in}{4.194190in}}%
\pgfpathlineto{\pgfqpoint{3.836330in}{4.225939in}}%
\pgfpathlineto{\pgfqpoint{3.822999in}{4.247629in}}%
\pgfpathlineto{\pgfqpoint{3.809663in}{4.269605in}}%
\pgfpathlineto{\pgfqpoint{3.796323in}{4.291870in}}%
\pgfpathlineto{\pgfqpoint{3.782977in}{4.314427in}}%
\pgfpathlineto{\pgfqpoint{3.775326in}{4.281924in}}%
\pgfpathlineto{\pgfqpoint{3.767674in}{4.249928in}}%
\pgfpathlineto{\pgfqpoint{3.760021in}{4.218430in}}%
\pgfpathlineto{\pgfqpoint{3.752365in}{4.187419in}}%
\pgfpathclose%
\pgfusepath{fill}%
\end{pgfscope}%
\begin{pgfscope}%
\pgfpathrectangle{\pgfqpoint{1.150000in}{0.150000in}}{\pgfqpoint{5.700000in}{5.700000in}}%
\pgfusepath{clip}%
\pgfsetbuttcap%
\pgfsetroundjoin%
\definecolor{currentfill}{rgb}{0.195860,0.395433,0.555276}%
\pgfsetfillcolor{currentfill}%
\pgfsetfillopacity{0.800000}%
\pgfsetlinewidth{0.000000pt}%
\definecolor{currentstroke}{rgb}{0.000000,0.000000,0.000000}%
\pgfsetstrokecolor{currentstroke}%
\pgfsetdash{}{0pt}%
\pgfpathmoveto{\pgfqpoint{3.780839in}{3.310840in}}%
\pgfpathlineto{\pgfqpoint{3.794137in}{3.296957in}}%
\pgfpathlineto{\pgfqpoint{3.807434in}{3.283325in}}%
\pgfpathlineto{\pgfqpoint{3.820730in}{3.269942in}}%
\pgfpathlineto{\pgfqpoint{3.834027in}{3.256806in}}%
\pgfpathlineto{\pgfqpoint{3.841743in}{3.275954in}}%
\pgfpathlineto{\pgfqpoint{3.849455in}{3.295371in}}%
\pgfpathlineto{\pgfqpoint{3.857164in}{3.315062in}}%
\pgfpathlineto{\pgfqpoint{3.864870in}{3.335033in}}%
\pgfpathlineto{\pgfqpoint{3.851577in}{3.348676in}}%
\pgfpathlineto{\pgfqpoint{3.838284in}{3.362568in}}%
\pgfpathlineto{\pgfqpoint{3.824990in}{3.376708in}}%
\pgfpathlineto{\pgfqpoint{3.811696in}{3.391100in}}%
\pgfpathlineto{\pgfqpoint{3.803987in}{3.370609in}}%
\pgfpathlineto{\pgfqpoint{3.796275in}{3.350406in}}%
\pgfpathlineto{\pgfqpoint{3.788559in}{3.330485in}}%
\pgfpathlineto{\pgfqpoint{3.780839in}{3.310840in}}%
\pgfpathclose%
\pgfusepath{fill}%
\end{pgfscope}%
\begin{pgfscope}%
\pgfpathrectangle{\pgfqpoint{1.150000in}{0.150000in}}{\pgfqpoint{5.700000in}{5.700000in}}%
\pgfusepath{clip}%
\pgfsetbuttcap%
\pgfsetroundjoin%
\definecolor{currentfill}{rgb}{0.194100,0.399323,0.555565}%
\pgfsetfillcolor{currentfill}%
\pgfsetfillopacity{0.800000}%
\pgfsetlinewidth{0.000000pt}%
\definecolor{currentstroke}{rgb}{0.000000,0.000000,0.000000}%
\pgfsetstrokecolor{currentstroke}%
\pgfsetdash{}{0pt}%
\pgfpathmoveto{\pgfqpoint{4.276275in}{3.300780in}}%
\pgfpathlineto{\pgfqpoint{4.289602in}{3.291153in}}%
\pgfpathlineto{\pgfqpoint{4.302932in}{3.281739in}}%
\pgfpathlineto{\pgfqpoint{4.316266in}{3.272537in}}%
\pgfpathlineto{\pgfqpoint{4.329605in}{3.263545in}}%
\pgfpathlineto{\pgfqpoint{4.337256in}{3.283403in}}%
\pgfpathlineto{\pgfqpoint{4.344908in}{3.303578in}}%
\pgfpathlineto{\pgfqpoint{4.352559in}{3.324078in}}%
\pgfpathlineto{\pgfqpoint{4.360210in}{3.344909in}}%
\pgfpathlineto{\pgfqpoint{4.346878in}{3.354530in}}%
\pgfpathlineto{\pgfqpoint{4.333549in}{3.364361in}}%
\pgfpathlineto{\pgfqpoint{4.320224in}{3.374405in}}%
\pgfpathlineto{\pgfqpoint{4.306903in}{3.384661in}}%
\pgfpathlineto{\pgfqpoint{4.299246in}{3.363188in}}%
\pgfpathlineto{\pgfqpoint{4.291589in}{3.342056in}}%
\pgfpathlineto{\pgfqpoint{4.283932in}{3.321255in}}%
\pgfpathlineto{\pgfqpoint{4.276275in}{3.300780in}}%
\pgfpathclose%
\pgfusepath{fill}%
\end{pgfscope}%
\begin{pgfscope}%
\pgfpathrectangle{\pgfqpoint{1.150000in}{0.150000in}}{\pgfqpoint{5.700000in}{5.700000in}}%
\pgfusepath{clip}%
\pgfsetbuttcap%
\pgfsetroundjoin%
\definecolor{currentfill}{rgb}{0.187231,0.414746,0.556547}%
\pgfsetfillcolor{currentfill}%
\pgfsetfillopacity{0.800000}%
\pgfsetlinewidth{0.000000pt}%
\definecolor{currentstroke}{rgb}{0.000000,0.000000,0.000000}%
\pgfsetstrokecolor{currentstroke}%
\pgfsetdash{}{0pt}%
\pgfpathmoveto{\pgfqpoint{4.360210in}{3.344909in}}%
\pgfpathlineto{\pgfqpoint{4.373547in}{3.335499in}}%
\pgfpathlineto{\pgfqpoint{4.386887in}{3.326297in}}%
\pgfpathlineto{\pgfqpoint{4.400232in}{3.317303in}}%
\pgfpathlineto{\pgfqpoint{4.413582in}{3.308517in}}%
\pgfpathlineto{\pgfqpoint{4.421227in}{3.329039in}}%
\pgfpathlineto{\pgfqpoint{4.428873in}{3.349900in}}%
\pgfpathlineto{\pgfqpoint{4.436520in}{3.371109in}}%
\pgfpathlineto{\pgfqpoint{4.444167in}{3.392671in}}%
\pgfpathlineto{\pgfqpoint{4.430824in}{3.402119in}}%
\pgfpathlineto{\pgfqpoint{4.417485in}{3.411773in}}%
\pgfpathlineto{\pgfqpoint{4.404150in}{3.421636in}}%
\pgfpathlineto{\pgfqpoint{4.390819in}{3.431708in}}%
\pgfpathlineto{\pgfqpoint{4.383166in}{3.409472in}}%
\pgfpathlineto{\pgfqpoint{4.375513in}{3.387598in}}%
\pgfpathlineto{\pgfqpoint{4.367861in}{3.366080in}}%
\pgfpathlineto{\pgfqpoint{4.360210in}{3.344909in}}%
\pgfpathclose%
\pgfusepath{fill}%
\end{pgfscope}%
\begin{pgfscope}%
\pgfpathrectangle{\pgfqpoint{1.150000in}{0.150000in}}{\pgfqpoint{5.700000in}{5.700000in}}%
\pgfusepath{clip}%
\pgfsetbuttcap%
\pgfsetroundjoin%
\definecolor{currentfill}{rgb}{0.206756,0.371758,0.553117}%
\pgfsetfillcolor{currentfill}%
\pgfsetfillopacity{0.800000}%
\pgfsetlinewidth{0.000000pt}%
\definecolor{currentstroke}{rgb}{0.000000,0.000000,0.000000}%
\pgfsetstrokecolor{currentstroke}%
\pgfsetdash{}{0pt}%
\pgfpathmoveto{\pgfqpoint{3.971227in}{3.234601in}}%
\pgfpathlineto{\pgfqpoint{3.984526in}{3.223113in}}%
\pgfpathlineto{\pgfqpoint{3.997826in}{3.211857in}}%
\pgfpathlineto{\pgfqpoint{4.011129in}{3.200832in}}%
\pgfpathlineto{\pgfqpoint{4.024433in}{3.190035in}}%
\pgfpathlineto{\pgfqpoint{4.032124in}{3.208716in}}%
\pgfpathlineto{\pgfqpoint{4.039813in}{3.227665in}}%
\pgfpathlineto{\pgfqpoint{4.047499in}{3.246888in}}%
\pgfpathlineto{\pgfqpoint{4.055183in}{3.266392in}}%
\pgfpathlineto{\pgfqpoint{4.041883in}{3.277724in}}%
\pgfpathlineto{\pgfqpoint{4.028585in}{3.289285in}}%
\pgfpathlineto{\pgfqpoint{4.015289in}{3.301078in}}%
\pgfpathlineto{\pgfqpoint{4.001994in}{3.313102in}}%
\pgfpathlineto{\pgfqpoint{3.994306in}{3.293050in}}%
\pgfpathlineto{\pgfqpoint{3.986615in}{3.273287in}}%
\pgfpathlineto{\pgfqpoint{3.978922in}{3.253806in}}%
\pgfpathlineto{\pgfqpoint{3.971227in}{3.234601in}}%
\pgfpathclose%
\pgfusepath{fill}%
\end{pgfscope}%
\begin{pgfscope}%
\pgfpathrectangle{\pgfqpoint{1.150000in}{0.150000in}}{\pgfqpoint{5.700000in}{5.700000in}}%
\pgfusepath{clip}%
\pgfsetbuttcap%
\pgfsetroundjoin%
\definecolor{currentfill}{rgb}{0.136408,0.541173,0.554483}%
\pgfsetfillcolor{currentfill}%
\pgfsetfillopacity{0.800000}%
\pgfsetlinewidth{0.000000pt}%
\definecolor{currentstroke}{rgb}{0.000000,0.000000,0.000000}%
\pgfsetstrokecolor{currentstroke}%
\pgfsetdash{}{0pt}%
\pgfpathmoveto{\pgfqpoint{3.545361in}{3.734733in}}%
\pgfpathlineto{\pgfqpoint{3.558712in}{3.714890in}}%
\pgfpathlineto{\pgfqpoint{3.572058in}{3.695341in}}%
\pgfpathlineto{\pgfqpoint{3.585399in}{3.676084in}}%
\pgfpathlineto{\pgfqpoint{3.598735in}{3.657116in}}%
\pgfpathlineto{\pgfqpoint{3.606448in}{3.680044in}}%
\pgfpathlineto{\pgfqpoint{3.614157in}{3.703308in}}%
\pgfpathlineto{\pgfqpoint{3.621863in}{3.726913in}}%
\pgfpathlineto{\pgfqpoint{3.629564in}{3.750867in}}%
\pgfpathlineto{\pgfqpoint{3.616228in}{3.770390in}}%
\pgfpathlineto{\pgfqpoint{3.602887in}{3.790203in}}%
\pgfpathlineto{\pgfqpoint{3.589542in}{3.810309in}}%
\pgfpathlineto{\pgfqpoint{3.576191in}{3.830710in}}%
\pgfpathlineto{\pgfqpoint{3.568490in}{3.806186in}}%
\pgfpathlineto{\pgfqpoint{3.560784in}{3.782020in}}%
\pgfpathlineto{\pgfqpoint{3.553075in}{3.758205in}}%
\pgfpathlineto{\pgfqpoint{3.545361in}{3.734733in}}%
\pgfpathclose%
\pgfusepath{fill}%
\end{pgfscope}%
\begin{pgfscope}%
\pgfpathrectangle{\pgfqpoint{1.150000in}{0.150000in}}{\pgfqpoint{5.700000in}{5.700000in}}%
\pgfusepath{clip}%
\pgfsetbuttcap%
\pgfsetroundjoin%
\definecolor{currentfill}{rgb}{0.201239,0.383670,0.554294}%
\pgfsetfillcolor{currentfill}%
\pgfsetfillopacity{0.800000}%
\pgfsetlinewidth{0.000000pt}%
\definecolor{currentstroke}{rgb}{0.000000,0.000000,0.000000}%
\pgfsetstrokecolor{currentstroke}%
\pgfsetdash{}{0pt}%
\pgfpathmoveto{\pgfqpoint{4.192345in}{3.260255in}}%
\pgfpathlineto{\pgfqpoint{4.205663in}{3.250364in}}%
\pgfpathlineto{\pgfqpoint{4.218985in}{3.240689in}}%
\pgfpathlineto{\pgfqpoint{4.232310in}{3.231231in}}%
\pgfpathlineto{\pgfqpoint{4.245638in}{3.221987in}}%
\pgfpathlineto{\pgfqpoint{4.253299in}{3.241234in}}%
\pgfpathlineto{\pgfqpoint{4.260959in}{3.260777in}}%
\pgfpathlineto{\pgfqpoint{4.268617in}{3.280623in}}%
\pgfpathlineto{\pgfqpoint{4.276275in}{3.300780in}}%
\pgfpathlineto{\pgfqpoint{4.262952in}{3.310621in}}%
\pgfpathlineto{\pgfqpoint{4.249633in}{3.320677in}}%
\pgfpathlineto{\pgfqpoint{4.236316in}{3.330949in}}%
\pgfpathlineto{\pgfqpoint{4.223003in}{3.341439in}}%
\pgfpathlineto{\pgfqpoint{4.215340in}{3.320672in}}%
\pgfpathlineto{\pgfqpoint{4.207676in}{3.300224in}}%
\pgfpathlineto{\pgfqpoint{4.200011in}{3.280087in}}%
\pgfpathlineto{\pgfqpoint{4.192345in}{3.260255in}}%
\pgfpathclose%
\pgfusepath{fill}%
\end{pgfscope}%
\begin{pgfscope}%
\pgfpathrectangle{\pgfqpoint{1.150000in}{0.150000in}}{\pgfqpoint{5.700000in}{5.700000in}}%
\pgfusepath{clip}%
\pgfsetbuttcap%
\pgfsetroundjoin%
\definecolor{currentfill}{rgb}{0.157729,0.485932,0.558013}%
\pgfsetfillcolor{currentfill}%
\pgfsetfillopacity{0.800000}%
\pgfsetlinewidth{0.000000pt}%
\definecolor{currentstroke}{rgb}{0.000000,0.000000,0.000000}%
\pgfsetstrokecolor{currentstroke}%
\pgfsetdash{}{0pt}%
\pgfpathmoveto{\pgfqpoint{3.567842in}{3.568631in}}%
\pgfpathlineto{\pgfqpoint{3.581176in}{3.550469in}}%
\pgfpathlineto{\pgfqpoint{3.594506in}{3.532591in}}%
\pgfpathlineto{\pgfqpoint{3.607832in}{3.514995in}}%
\pgfpathlineto{\pgfqpoint{3.621155in}{3.497677in}}%
\pgfpathlineto{\pgfqpoint{3.628883in}{3.518821in}}%
\pgfpathlineto{\pgfqpoint{3.636607in}{3.540268in}}%
\pgfpathlineto{\pgfqpoint{3.644326in}{3.562022in}}%
\pgfpathlineto{\pgfqpoint{3.652042in}{3.584090in}}%
\pgfpathlineto{\pgfqpoint{3.638721in}{3.601924in}}%
\pgfpathlineto{\pgfqpoint{3.625396in}{3.620039in}}%
\pgfpathlineto{\pgfqpoint{3.612068in}{3.638435in}}%
\pgfpathlineto{\pgfqpoint{3.598735in}{3.657116in}}%
\pgfpathlineto{\pgfqpoint{3.591018in}{3.634517in}}%
\pgfpathlineto{\pgfqpoint{3.583297in}{3.612241in}}%
\pgfpathlineto{\pgfqpoint{3.575572in}{3.590281in}}%
\pgfpathlineto{\pgfqpoint{3.567842in}{3.568631in}}%
\pgfpathclose%
\pgfusepath{fill}%
\end{pgfscope}%
\begin{pgfscope}%
\pgfpathrectangle{\pgfqpoint{1.150000in}{0.150000in}}{\pgfqpoint{5.700000in}{5.700000in}}%
\pgfusepath{clip}%
\pgfsetbuttcap%
\pgfsetroundjoin%
\definecolor{currentfill}{rgb}{0.160665,0.478540,0.558115}%
\pgfsetfillcolor{currentfill}%
\pgfsetfillopacity{0.800000}%
\pgfsetlinewidth{0.000000pt}%
\definecolor{currentstroke}{rgb}{0.000000,0.000000,0.000000}%
\pgfsetstrokecolor{currentstroke}%
\pgfsetdash{}{0pt}%
\pgfpathmoveto{\pgfqpoint{4.558776in}{3.537513in}}%
\pgfpathlineto{\pgfqpoint{4.572129in}{3.527670in}}%
\pgfpathlineto{\pgfqpoint{4.585488in}{3.518029in}}%
\pgfpathlineto{\pgfqpoint{4.598851in}{3.508590in}}%
\pgfpathlineto{\pgfqpoint{4.612218in}{3.499350in}}%
\pgfpathlineto{\pgfqpoint{4.619872in}{3.522985in}}%
\pgfpathlineto{\pgfqpoint{4.627529in}{3.547043in}}%
\pgfpathlineto{\pgfqpoint{4.635190in}{3.571533in}}%
\pgfpathlineto{\pgfqpoint{4.621827in}{3.581337in}}%
\pgfpathlineto{\pgfqpoint{4.608469in}{3.591343in}}%
\pgfpathlineto{\pgfqpoint{4.595116in}{3.601550in}}%
\pgfpathlineto{\pgfqpoint{4.581767in}{3.611960in}}%
\pgfpathlineto{\pgfqpoint{4.574100in}{3.586707in}}%
\pgfpathlineto{\pgfqpoint{4.566436in}{3.561895in}}%
\pgfpathlineto{\pgfqpoint{4.558776in}{3.537513in}}%
\pgfpathclose%
\pgfusepath{fill}%
\end{pgfscope}%
\begin{pgfscope}%
\pgfpathrectangle{\pgfqpoint{1.150000in}{0.150000in}}{\pgfqpoint{5.700000in}{5.700000in}}%
\pgfusepath{clip}%
\pgfsetbuttcap%
\pgfsetroundjoin%
\definecolor{currentfill}{rgb}{0.203063,0.379716,0.553925}%
\pgfsetfillcolor{currentfill}%
\pgfsetfillopacity{0.800000}%
\pgfsetlinewidth{0.000000pt}%
\definecolor{currentstroke}{rgb}{0.000000,0.000000,0.000000}%
\pgfsetstrokecolor{currentstroke}%
\pgfsetdash{}{0pt}%
\pgfpathmoveto{\pgfqpoint{3.834027in}{3.256806in}}%
\pgfpathlineto{\pgfqpoint{3.847323in}{3.243915in}}%
\pgfpathlineto{\pgfqpoint{3.860620in}{3.231268in}}%
\pgfpathlineto{\pgfqpoint{3.873917in}{3.218864in}}%
\pgfpathlineto{\pgfqpoint{3.887215in}{3.206700in}}%
\pgfpathlineto{\pgfqpoint{3.894927in}{3.225354in}}%
\pgfpathlineto{\pgfqpoint{3.902635in}{3.244269in}}%
\pgfpathlineto{\pgfqpoint{3.910341in}{3.263449in}}%
\pgfpathlineto{\pgfqpoint{3.918043in}{3.282901in}}%
\pgfpathlineto{\pgfqpoint{3.904749in}{3.295571in}}%
\pgfpathlineto{\pgfqpoint{3.891456in}{3.308481in}}%
\pgfpathlineto{\pgfqpoint{3.878163in}{3.321635in}}%
\pgfpathlineto{\pgfqpoint{3.864870in}{3.335033in}}%
\pgfpathlineto{\pgfqpoint{3.857164in}{3.315062in}}%
\pgfpathlineto{\pgfqpoint{3.849455in}{3.295371in}}%
\pgfpathlineto{\pgfqpoint{3.841743in}{3.275954in}}%
\pgfpathlineto{\pgfqpoint{3.834027in}{3.256806in}}%
\pgfpathclose%
\pgfusepath{fill}%
\end{pgfscope}%
\begin{pgfscope}%
\pgfpathrectangle{\pgfqpoint{1.150000in}{0.150000in}}{\pgfqpoint{5.700000in}{5.700000in}}%
\pgfusepath{clip}%
\pgfsetbuttcap%
\pgfsetroundjoin%
\definecolor{currentfill}{rgb}{0.162016,0.687316,0.499129}%
\pgfsetfillcolor{currentfill}%
\pgfsetfillopacity{0.800000}%
\pgfsetlinewidth{0.000000pt}%
\definecolor{currentstroke}{rgb}{0.000000,0.000000,0.000000}%
\pgfsetstrokecolor{currentstroke}%
\pgfsetdash{}{0pt}%
\pgfpathmoveto{\pgfqpoint{3.668331in}{4.155483in}}%
\pgfpathlineto{\pgfqpoint{3.681688in}{4.133188in}}%
\pgfpathlineto{\pgfqpoint{3.695040in}{4.111192in}}%
\pgfpathlineto{\pgfqpoint{3.708386in}{4.089493in}}%
\pgfpathlineto{\pgfqpoint{3.721727in}{4.068087in}}%
\pgfpathlineto{\pgfqpoint{3.729390in}{4.097231in}}%
\pgfpathlineto{\pgfqpoint{3.737050in}{4.126829in}}%
\pgfpathlineto{\pgfqpoint{3.744709in}{4.156889in}}%
\pgfpathlineto{\pgfqpoint{3.752365in}{4.187419in}}%
\pgfpathlineto{\pgfqpoint{3.739020in}{4.209528in}}%
\pgfpathlineto{\pgfqpoint{3.725670in}{4.231932in}}%
\pgfpathlineto{\pgfqpoint{3.712314in}{4.254633in}}%
\pgfpathlineto{\pgfqpoint{3.698953in}{4.277635in}}%
\pgfpathlineto{\pgfqpoint{3.691301in}{4.246385in}}%
\pgfpathlineto{\pgfqpoint{3.683647in}{4.215615in}}%
\pgfpathlineto{\pgfqpoint{3.675990in}{4.185317in}}%
\pgfpathlineto{\pgfqpoint{3.668331in}{4.155483in}}%
\pgfpathclose%
\pgfusepath{fill}%
\end{pgfscope}%
\begin{pgfscope}%
\pgfpathrectangle{\pgfqpoint{1.150000in}{0.150000in}}{\pgfqpoint{5.700000in}{5.700000in}}%
\pgfusepath{clip}%
\pgfsetbuttcap%
\pgfsetroundjoin%
\definecolor{currentfill}{rgb}{0.179019,0.433756,0.557430}%
\pgfsetfillcolor{currentfill}%
\pgfsetfillopacity{0.800000}%
\pgfsetlinewidth{0.000000pt}%
\definecolor{currentstroke}{rgb}{0.000000,0.000000,0.000000}%
\pgfsetstrokecolor{currentstroke}%
\pgfsetdash{}{0pt}%
\pgfpathmoveto{\pgfqpoint{4.444167in}{3.392671in}}%
\pgfpathlineto{\pgfqpoint{4.457515in}{3.383430in}}%
\pgfpathlineto{\pgfqpoint{4.470867in}{3.374395in}}%
\pgfpathlineto{\pgfqpoint{4.484224in}{3.365563in}}%
\pgfpathlineto{\pgfqpoint{4.497586in}{3.356935in}}%
\pgfpathlineto{\pgfqpoint{4.505228in}{3.378180in}}%
\pgfpathlineto{\pgfqpoint{4.512872in}{3.399788in}}%
\pgfpathlineto{\pgfqpoint{4.520517in}{3.421766in}}%
\pgfpathlineto{\pgfqpoint{4.528164in}{3.444124in}}%
\pgfpathlineto{\pgfqpoint{4.514809in}{3.453444in}}%
\pgfpathlineto{\pgfqpoint{4.501458in}{3.462968in}}%
\pgfpathlineto{\pgfqpoint{4.488112in}{3.472697in}}%
\pgfpathlineto{\pgfqpoint{4.474770in}{3.482631in}}%
\pgfpathlineto{\pgfqpoint{4.467117in}{3.459569in}}%
\pgfpathlineto{\pgfqpoint{4.459466in}{3.436893in}}%
\pgfpathlineto{\pgfqpoint{4.451816in}{3.414597in}}%
\pgfpathlineto{\pgfqpoint{4.444167in}{3.392671in}}%
\pgfpathclose%
\pgfusepath{fill}%
\end{pgfscope}%
\begin{pgfscope}%
\pgfpathrectangle{\pgfqpoint{1.150000in}{0.150000in}}{\pgfqpoint{5.700000in}{5.700000in}}%
\pgfusepath{clip}%
\pgfsetbuttcap%
\pgfsetroundjoin%
\definecolor{currentfill}{rgb}{0.208623,0.367752,0.552675}%
\pgfsetfillcolor{currentfill}%
\pgfsetfillopacity{0.800000}%
\pgfsetlinewidth{0.000000pt}%
\definecolor{currentstroke}{rgb}{0.000000,0.000000,0.000000}%
\pgfsetstrokecolor{currentstroke}%
\pgfsetdash{}{0pt}%
\pgfpathmoveto{\pgfqpoint{4.108403in}{3.223331in}}%
\pgfpathlineto{\pgfqpoint{4.121715in}{3.213127in}}%
\pgfpathlineto{\pgfqpoint{4.135029in}{3.203143in}}%
\pgfpathlineto{\pgfqpoint{4.148346in}{3.193380in}}%
\pgfpathlineto{\pgfqpoint{4.161666in}{3.183836in}}%
\pgfpathlineto{\pgfqpoint{4.169338in}{3.202518in}}%
\pgfpathlineto{\pgfqpoint{4.177009in}{3.221477in}}%
\pgfpathlineto{\pgfqpoint{4.184678in}{3.240721in}}%
\pgfpathlineto{\pgfqpoint{4.192345in}{3.260255in}}%
\pgfpathlineto{\pgfqpoint{4.179030in}{3.270365in}}%
\pgfpathlineto{\pgfqpoint{4.165718in}{3.280694in}}%
\pgfpathlineto{\pgfqpoint{4.152409in}{3.291243in}}%
\pgfpathlineto{\pgfqpoint{4.139102in}{3.302015in}}%
\pgfpathlineto{\pgfqpoint{4.131430in}{3.281902in}}%
\pgfpathlineto{\pgfqpoint{4.123756in}{3.262089in}}%
\pgfpathlineto{\pgfqpoint{4.116081in}{3.242567in}}%
\pgfpathlineto{\pgfqpoint{4.108403in}{3.223331in}}%
\pgfpathclose%
\pgfusepath{fill}%
\end{pgfscope}%
\begin{pgfscope}%
\pgfpathrectangle{\pgfqpoint{1.150000in}{0.150000in}}{\pgfqpoint{5.700000in}{5.700000in}}%
\pgfusepath{clip}%
\pgfsetbuttcap%
\pgfsetroundjoin%
\definecolor{currentfill}{rgb}{0.171176,0.452530,0.557965}%
\pgfsetfillcolor{currentfill}%
\pgfsetfillopacity{0.800000}%
\pgfsetlinewidth{0.000000pt}%
\definecolor{currentstroke}{rgb}{0.000000,0.000000,0.000000}%
\pgfsetstrokecolor{currentstroke}%
\pgfsetdash{}{0pt}%
\pgfpathmoveto{\pgfqpoint{4.528164in}{3.444124in}}%
\pgfpathlineto{\pgfqpoint{4.541524in}{3.435006in}}%
\pgfpathlineto{\pgfqpoint{4.554889in}{3.426090in}}%
\pgfpathlineto{\pgfqpoint{4.568259in}{3.417375in}}%
\pgfpathlineto{\pgfqpoint{4.581634in}{3.408860in}}%
\pgfpathlineto{\pgfqpoint{4.589276in}{3.430892in}}%
\pgfpathlineto{\pgfqpoint{4.596921in}{3.453313in}}%
\pgfpathlineto{\pgfqpoint{4.604568in}{3.476129in}}%
\pgfpathlineto{\pgfqpoint{4.612218in}{3.499350in}}%
\pgfpathlineto{\pgfqpoint{4.598851in}{3.508590in}}%
\pgfpathlineto{\pgfqpoint{4.585488in}{3.518029in}}%
\pgfpathlineto{\pgfqpoint{4.572129in}{3.527670in}}%
\pgfpathlineto{\pgfqpoint{4.558776in}{3.537513in}}%
\pgfpathlineto{\pgfqpoint{4.551119in}{3.513555in}}%
\pgfpathlineto{\pgfqpoint{4.543465in}{3.490009in}}%
\pgfpathlineto{\pgfqpoint{4.535813in}{3.466868in}}%
\pgfpathlineto{\pgfqpoint{4.528164in}{3.444124in}}%
\pgfpathclose%
\pgfusepath{fill}%
\end{pgfscope}%
\begin{pgfscope}%
\pgfpathrectangle{\pgfqpoint{1.150000in}{0.150000in}}{\pgfqpoint{5.700000in}{5.700000in}}%
\pgfusepath{clip}%
\pgfsetbuttcap%
\pgfsetroundjoin%
\definecolor{currentfill}{rgb}{0.124780,0.640461,0.527068}%
\pgfsetfillcolor{currentfill}%
\pgfsetfillopacity{0.800000}%
\pgfsetlinewidth{0.000000pt}%
\definecolor{currentstroke}{rgb}{0.000000,0.000000,0.000000}%
\pgfsetstrokecolor{currentstroke}%
\pgfsetdash{}{0pt}%
\pgfpathmoveto{\pgfqpoint{3.553496in}{4.019519in}}%
\pgfpathlineto{\pgfqpoint{3.566870in}{3.997311in}}%
\pgfpathlineto{\pgfqpoint{3.580239in}{3.975410in}}%
\pgfpathlineto{\pgfqpoint{3.593601in}{3.953814in}}%
\pgfpathlineto{\pgfqpoint{3.606958in}{3.932520in}}%
\pgfpathlineto{\pgfqpoint{3.614641in}{3.958936in}}%
\pgfpathlineto{\pgfqpoint{3.622320in}{3.985753in}}%
\pgfpathlineto{\pgfqpoint{3.629996in}{4.012977in}}%
\pgfpathlineto{\pgfqpoint{3.637669in}{4.040616in}}%
\pgfpathlineto{\pgfqpoint{3.624310in}{4.062542in}}%
\pgfpathlineto{\pgfqpoint{3.610945in}{4.084772in}}%
\pgfpathlineto{\pgfqpoint{3.597574in}{4.107307in}}%
\pgfpathlineto{\pgfqpoint{3.584197in}{4.130151in}}%
\pgfpathlineto{\pgfqpoint{3.576527in}{4.101863in}}%
\pgfpathlineto{\pgfqpoint{3.568854in}{4.074001in}}%
\pgfpathlineto{\pgfqpoint{3.561176in}{4.046555in}}%
\pgfpathlineto{\pgfqpoint{3.553496in}{4.019519in}}%
\pgfpathclose%
\pgfusepath{fill}%
\end{pgfscope}%
\begin{pgfscope}%
\pgfpathrectangle{\pgfqpoint{1.150000in}{0.150000in}}{\pgfqpoint{5.700000in}{5.700000in}}%
\pgfusepath{clip}%
\pgfsetbuttcap%
\pgfsetroundjoin%
\definecolor{currentfill}{rgb}{0.119512,0.607464,0.540218}%
\pgfsetfillcolor{currentfill}%
\pgfsetfillopacity{0.800000}%
\pgfsetlinewidth{0.000000pt}%
\definecolor{currentstroke}{rgb}{0.000000,0.000000,0.000000}%
\pgfsetstrokecolor{currentstroke}%
\pgfsetdash{}{0pt}%
\pgfpathmoveto{\pgfqpoint{3.522733in}{3.915322in}}%
\pgfpathlineto{\pgfqpoint{3.536107in}{3.893713in}}%
\pgfpathlineto{\pgfqpoint{3.549474in}{3.872409in}}%
\pgfpathlineto{\pgfqpoint{3.562835in}{3.851410in}}%
\pgfpathlineto{\pgfqpoint{3.576191in}{3.830710in}}%
\pgfpathlineto{\pgfqpoint{3.583889in}{3.855599in}}%
\pgfpathlineto{\pgfqpoint{3.591582in}{3.880858in}}%
\pgfpathlineto{\pgfqpoint{3.599272in}{3.906496in}}%
\pgfpathlineto{\pgfqpoint{3.606958in}{3.932520in}}%
\pgfpathlineto{\pgfqpoint{3.593601in}{3.953814in}}%
\pgfpathlineto{\pgfqpoint{3.580239in}{3.975410in}}%
\pgfpathlineto{\pgfqpoint{3.566870in}{3.997311in}}%
\pgfpathlineto{\pgfqpoint{3.553496in}{4.019519in}}%
\pgfpathlineto{\pgfqpoint{3.545811in}{3.992884in}}%
\pgfpathlineto{\pgfqpoint{3.538123in}{3.966645in}}%
\pgfpathlineto{\pgfqpoint{3.530430in}{3.940793in}}%
\pgfpathlineto{\pgfqpoint{3.522733in}{3.915322in}}%
\pgfpathclose%
\pgfusepath{fill}%
\end{pgfscope}%
\begin{pgfscope}%
\pgfpathrectangle{\pgfqpoint{1.150000in}{0.150000in}}{\pgfqpoint{5.700000in}{5.700000in}}%
\pgfusepath{clip}%
\pgfsetbuttcap%
\pgfsetroundjoin%
\definecolor{currentfill}{rgb}{0.266941,0.748751,0.440573}%
\pgfsetfillcolor{currentfill}%
\pgfsetfillopacity{0.800000}%
\pgfsetlinewidth{0.000000pt}%
\definecolor{currentstroke}{rgb}{0.000000,0.000000,0.000000}%
\pgfsetstrokecolor{currentstroke}%
\pgfsetdash{}{0pt}%
\pgfpathmoveto{\pgfqpoint{3.866950in}{4.358086in}}%
\pgfpathlineto{\pgfqpoint{3.880283in}{4.335906in}}%
\pgfpathlineto{\pgfqpoint{3.893611in}{4.314009in}}%
\pgfpathlineto{\pgfqpoint{3.906935in}{4.292394in}}%
\pgfpathlineto{\pgfqpoint{3.920256in}{4.271057in}}%
\pgfpathlineto{\pgfqpoint{3.927917in}{4.304631in}}%
\pgfpathlineto{\pgfqpoint{3.935578in}{4.338748in}}%
\pgfpathlineto{\pgfqpoint{3.943240in}{4.373419in}}%
\pgfpathlineto{\pgfqpoint{3.929915in}{4.395359in}}%
\pgfpathlineto{\pgfqpoint{3.916586in}{4.417580in}}%
\pgfpathlineto{\pgfqpoint{3.903253in}{4.440083in}}%
\pgfpathlineto{\pgfqpoint{3.889915in}{4.462870in}}%
\pgfpathlineto{\pgfqpoint{3.882260in}{4.427382in}}%
\pgfpathlineto{\pgfqpoint{3.874605in}{4.392457in}}%
\pgfpathlineto{\pgfqpoint{3.866950in}{4.358086in}}%
\pgfpathclose%
\pgfusepath{fill}%
\end{pgfscope}%
\begin{pgfscope}%
\pgfpathrectangle{\pgfqpoint{1.150000in}{0.150000in}}{\pgfqpoint{5.700000in}{5.700000in}}%
\pgfusepath{clip}%
\pgfsetbuttcap%
\pgfsetroundjoin%
\definecolor{currentfill}{rgb}{0.147607,0.511733,0.557049}%
\pgfsetfillcolor{currentfill}%
\pgfsetfillopacity{0.800000}%
\pgfsetlinewidth{0.000000pt}%
\definecolor{currentstroke}{rgb}{0.000000,0.000000,0.000000}%
\pgfsetstrokecolor{currentstroke}%
\pgfsetdash{}{0pt}%
\pgfpathmoveto{\pgfqpoint{3.514462in}{3.644163in}}%
\pgfpathlineto{\pgfqpoint{3.527814in}{3.624843in}}%
\pgfpathlineto{\pgfqpoint{3.541161in}{3.605815in}}%
\pgfpathlineto{\pgfqpoint{3.554504in}{3.587079in}}%
\pgfpathlineto{\pgfqpoint{3.567842in}{3.568631in}}%
\pgfpathlineto{\pgfqpoint{3.575572in}{3.590281in}}%
\pgfpathlineto{\pgfqpoint{3.583297in}{3.612241in}}%
\pgfpathlineto{\pgfqpoint{3.591018in}{3.634517in}}%
\pgfpathlineto{\pgfqpoint{3.598735in}{3.657116in}}%
\pgfpathlineto{\pgfqpoint{3.585399in}{3.676084in}}%
\pgfpathlineto{\pgfqpoint{3.572058in}{3.695341in}}%
\pgfpathlineto{\pgfqpoint{3.558712in}{3.714890in}}%
\pgfpathlineto{\pgfqpoint{3.545361in}{3.734733in}}%
\pgfpathlineto{\pgfqpoint{3.537643in}{3.711600in}}%
\pgfpathlineto{\pgfqpoint{3.529921in}{3.688798in}}%
\pgfpathlineto{\pgfqpoint{3.522194in}{3.666321in}}%
\pgfpathlineto{\pgfqpoint{3.514462in}{3.644163in}}%
\pgfpathclose%
\pgfusepath{fill}%
\end{pgfscope}%
\begin{pgfscope}%
\pgfpathrectangle{\pgfqpoint{1.150000in}{0.150000in}}{\pgfqpoint{5.700000in}{5.700000in}}%
\pgfusepath{clip}%
\pgfsetbuttcap%
\pgfsetroundjoin%
\definecolor{currentfill}{rgb}{0.210503,0.363727,0.552206}%
\pgfsetfillcolor{currentfill}%
\pgfsetfillopacity{0.800000}%
\pgfsetlinewidth{0.000000pt}%
\definecolor{currentstroke}{rgb}{0.000000,0.000000,0.000000}%
\pgfsetstrokecolor{currentstroke}%
\pgfsetdash{}{0pt}%
\pgfpathmoveto{\pgfqpoint{3.887215in}{3.206700in}}%
\pgfpathlineto{\pgfqpoint{3.900514in}{3.194775in}}%
\pgfpathlineto{\pgfqpoint{3.913813in}{3.183088in}}%
\pgfpathlineto{\pgfqpoint{3.927114in}{3.171637in}}%
\pgfpathlineto{\pgfqpoint{3.940416in}{3.160420in}}%
\pgfpathlineto{\pgfqpoint{3.948123in}{3.178581in}}%
\pgfpathlineto{\pgfqpoint{3.955827in}{3.196995in}}%
\pgfpathlineto{\pgfqpoint{3.963528in}{3.215666in}}%
\pgfpathlineto{\pgfqpoint{3.971227in}{3.234601in}}%
\pgfpathlineto{\pgfqpoint{3.957929in}{3.246322in}}%
\pgfpathlineto{\pgfqpoint{3.944633in}{3.258278in}}%
\pgfpathlineto{\pgfqpoint{3.931337in}{3.270471in}}%
\pgfpathlineto{\pgfqpoint{3.918043in}{3.282901in}}%
\pgfpathlineto{\pgfqpoint{3.910341in}{3.263449in}}%
\pgfpathlineto{\pgfqpoint{3.902635in}{3.244269in}}%
\pgfpathlineto{\pgfqpoint{3.894927in}{3.225354in}}%
\pgfpathlineto{\pgfqpoint{3.887215in}{3.206700in}}%
\pgfpathclose%
\pgfusepath{fill}%
\end{pgfscope}%
\begin{pgfscope}%
\pgfpathrectangle{\pgfqpoint{1.150000in}{0.150000in}}{\pgfqpoint{5.700000in}{5.700000in}}%
\pgfusepath{clip}%
\pgfsetbuttcap%
\pgfsetroundjoin%
\definecolor{currentfill}{rgb}{0.187231,0.414746,0.556547}%
\pgfsetfillcolor{currentfill}%
\pgfsetfillopacity{0.800000}%
\pgfsetlinewidth{0.000000pt}%
\definecolor{currentstroke}{rgb}{0.000000,0.000000,0.000000}%
\pgfsetstrokecolor{currentstroke}%
\pgfsetdash{}{0pt}%
\pgfpathmoveto{\pgfqpoint{3.643475in}{3.351397in}}%
\pgfpathlineto{\pgfqpoint{3.656787in}{3.335921in}}%
\pgfpathlineto{\pgfqpoint{3.670097in}{3.320710in}}%
\pgfpathlineto{\pgfqpoint{3.683406in}{3.305763in}}%
\pgfpathlineto{\pgfqpoint{3.696712in}{3.291077in}}%
\pgfpathlineto{\pgfqpoint{3.704450in}{3.310139in}}%
\pgfpathlineto{\pgfqpoint{3.712184in}{3.329462in}}%
\pgfpathlineto{\pgfqpoint{3.719914in}{3.349052in}}%
\pgfpathlineto{\pgfqpoint{3.727640in}{3.368916in}}%
\pgfpathlineto{\pgfqpoint{3.714337in}{3.384080in}}%
\pgfpathlineto{\pgfqpoint{3.701032in}{3.399506in}}%
\pgfpathlineto{\pgfqpoint{3.687726in}{3.415195in}}%
\pgfpathlineto{\pgfqpoint{3.674417in}{3.431151in}}%
\pgfpathlineto{\pgfqpoint{3.666688in}{3.410796in}}%
\pgfpathlineto{\pgfqpoint{3.658954in}{3.390723in}}%
\pgfpathlineto{\pgfqpoint{3.651217in}{3.370925in}}%
\pgfpathlineto{\pgfqpoint{3.643475in}{3.351397in}}%
\pgfpathclose%
\pgfusepath{fill}%
\end{pgfscope}%
\begin{pgfscope}%
\pgfpathrectangle{\pgfqpoint{1.150000in}{0.150000in}}{\pgfqpoint{5.700000in}{5.700000in}}%
\pgfusepath{clip}%
\pgfsetbuttcap%
\pgfsetroundjoin%
\definecolor{currentfill}{rgb}{0.252899,0.742211,0.448284}%
\pgfsetfillcolor{currentfill}%
\pgfsetfillopacity{0.800000}%
\pgfsetlinewidth{0.000000pt}%
\definecolor{currentstroke}{rgb}{0.000000,0.000000,0.000000}%
\pgfsetstrokecolor{currentstroke}%
\pgfsetdash{}{0pt}%
\pgfpathmoveto{\pgfqpoint{3.782977in}{4.314427in}}%
\pgfpathlineto{\pgfqpoint{3.796323in}{4.291870in}}%
\pgfpathlineto{\pgfqpoint{3.809663in}{4.269605in}}%
\pgfpathlineto{\pgfqpoint{3.822999in}{4.247629in}}%
\pgfpathlineto{\pgfqpoint{3.836330in}{4.225939in}}%
\pgfpathlineto{\pgfqpoint{3.843986in}{4.258194in}}%
\pgfpathlineto{\pgfqpoint{3.851641in}{4.290964in}}%
\pgfpathlineto{\pgfqpoint{3.859296in}{4.324258in}}%
\pgfpathlineto{\pgfqpoint{3.866950in}{4.358086in}}%
\pgfpathlineto{\pgfqpoint{3.853613in}{4.380551in}}%
\pgfpathlineto{\pgfqpoint{3.840272in}{4.403305in}}%
\pgfpathlineto{\pgfqpoint{3.826925in}{4.426349in}}%
\pgfpathlineto{\pgfqpoint{3.813573in}{4.449687in}}%
\pgfpathlineto{\pgfqpoint{3.805925in}{4.415065in}}%
\pgfpathlineto{\pgfqpoint{3.798277in}{4.380988in}}%
\pgfpathlineto{\pgfqpoint{3.790627in}{4.347445in}}%
\pgfpathlineto{\pgfqpoint{3.782977in}{4.314427in}}%
\pgfpathclose%
\pgfusepath{fill}%
\end{pgfscope}%
\begin{pgfscope}%
\pgfpathrectangle{\pgfqpoint{1.150000in}{0.150000in}}{\pgfqpoint{5.700000in}{5.700000in}}%
\pgfusepath{clip}%
\pgfsetbuttcap%
\pgfsetroundjoin%
\definecolor{currentfill}{rgb}{0.179019,0.433756,0.557430}%
\pgfsetfillcolor{currentfill}%
\pgfsetfillopacity{0.800000}%
\pgfsetlinewidth{0.000000pt}%
\definecolor{currentstroke}{rgb}{0.000000,0.000000,0.000000}%
\pgfsetstrokecolor{currentstroke}%
\pgfsetdash{}{0pt}%
\pgfpathmoveto{\pgfqpoint{3.590202in}{3.415997in}}%
\pgfpathlineto{\pgfqpoint{3.603525in}{3.399438in}}%
\pgfpathlineto{\pgfqpoint{3.616844in}{3.383154in}}%
\pgfpathlineto{\pgfqpoint{3.630161in}{3.367140in}}%
\pgfpathlineto{\pgfqpoint{3.643475in}{3.351397in}}%
\pgfpathlineto{\pgfqpoint{3.651217in}{3.370925in}}%
\pgfpathlineto{\pgfqpoint{3.658954in}{3.390723in}}%
\pgfpathlineto{\pgfqpoint{3.666688in}{3.410796in}}%
\pgfpathlineto{\pgfqpoint{3.674417in}{3.431151in}}%
\pgfpathlineto{\pgfqpoint{3.661105in}{3.447375in}}%
\pgfpathlineto{\pgfqpoint{3.647792in}{3.463869in}}%
\pgfpathlineto{\pgfqpoint{3.634475in}{3.480636in}}%
\pgfpathlineto{\pgfqpoint{3.621155in}{3.497677in}}%
\pgfpathlineto{\pgfqpoint{3.613423in}{3.476828in}}%
\pgfpathlineto{\pgfqpoint{3.605687in}{3.456269in}}%
\pgfpathlineto{\pgfqpoint{3.597947in}{3.435994in}}%
\pgfpathlineto{\pgfqpoint{3.590202in}{3.415997in}}%
\pgfpathclose%
\pgfusepath{fill}%
\end{pgfscope}%
\begin{pgfscope}%
\pgfpathrectangle{\pgfqpoint{1.150000in}{0.150000in}}{\pgfqpoint{5.700000in}{5.700000in}}%
\pgfusepath{clip}%
\pgfsetbuttcap%
\pgfsetroundjoin%
\definecolor{currentfill}{rgb}{0.197636,0.391528,0.554969}%
\pgfsetfillcolor{currentfill}%
\pgfsetfillopacity{0.800000}%
\pgfsetlinewidth{0.000000pt}%
\definecolor{currentstroke}{rgb}{0.000000,0.000000,0.000000}%
\pgfsetstrokecolor{currentstroke}%
\pgfsetdash{}{0pt}%
\pgfpathmoveto{\pgfqpoint{3.696712in}{3.291077in}}%
\pgfpathlineto{\pgfqpoint{3.710017in}{3.276651in}}%
\pgfpathlineto{\pgfqpoint{3.723321in}{3.262482in}}%
\pgfpathlineto{\pgfqpoint{3.736624in}{3.248569in}}%
\pgfpathlineto{\pgfqpoint{3.749926in}{3.234910in}}%
\pgfpathlineto{\pgfqpoint{3.757660in}{3.253507in}}%
\pgfpathlineto{\pgfqpoint{3.765390in}{3.272357in}}%
\pgfpathlineto{\pgfqpoint{3.773117in}{3.291467in}}%
\pgfpathlineto{\pgfqpoint{3.780839in}{3.310840in}}%
\pgfpathlineto{\pgfqpoint{3.767541in}{3.324976in}}%
\pgfpathlineto{\pgfqpoint{3.754242in}{3.339366in}}%
\pgfpathlineto{\pgfqpoint{3.740942in}{3.354012in}}%
\pgfpathlineto{\pgfqpoint{3.727640in}{3.368916in}}%
\pgfpathlineto{\pgfqpoint{3.719914in}{3.349052in}}%
\pgfpathlineto{\pgfqpoint{3.712184in}{3.329462in}}%
\pgfpathlineto{\pgfqpoint{3.704450in}{3.310139in}}%
\pgfpathlineto{\pgfqpoint{3.696712in}{3.291077in}}%
\pgfpathclose%
\pgfusepath{fill}%
\end{pgfscope}%
\begin{pgfscope}%
\pgfpathrectangle{\pgfqpoint{1.150000in}{0.150000in}}{\pgfqpoint{5.700000in}{5.700000in}}%
\pgfusepath{clip}%
\pgfsetbuttcap%
\pgfsetroundjoin%
\definecolor{currentfill}{rgb}{0.214298,0.355619,0.551184}%
\pgfsetfillcolor{currentfill}%
\pgfsetfillopacity{0.800000}%
\pgfsetlinewidth{0.000000pt}%
\definecolor{currentstroke}{rgb}{0.000000,0.000000,0.000000}%
\pgfsetstrokecolor{currentstroke}%
\pgfsetdash{}{0pt}%
\pgfpathmoveto{\pgfqpoint{4.024433in}{3.190035in}}%
\pgfpathlineto{\pgfqpoint{4.037739in}{3.179467in}}%
\pgfpathlineto{\pgfqpoint{4.051048in}{3.169124in}}%
\pgfpathlineto{\pgfqpoint{4.064359in}{3.159007in}}%
\pgfpathlineto{\pgfqpoint{4.077673in}{3.149114in}}%
\pgfpathlineto{\pgfqpoint{4.085359in}{3.167272in}}%
\pgfpathlineto{\pgfqpoint{4.093043in}{3.185690in}}%
\pgfpathlineto{\pgfqpoint{4.100724in}{3.204375in}}%
\pgfpathlineto{\pgfqpoint{4.108403in}{3.223331in}}%
\pgfpathlineto{\pgfqpoint{4.095095in}{3.233759in}}%
\pgfpathlineto{\pgfqpoint{4.081789in}{3.244411in}}%
\pgfpathlineto{\pgfqpoint{4.068485in}{3.255288in}}%
\pgfpathlineto{\pgfqpoint{4.055183in}{3.266392in}}%
\pgfpathlineto{\pgfqpoint{4.047499in}{3.246888in}}%
\pgfpathlineto{\pgfqpoint{4.039813in}{3.227665in}}%
\pgfpathlineto{\pgfqpoint{4.032124in}{3.208716in}}%
\pgfpathlineto{\pgfqpoint{4.024433in}{3.190035in}}%
\pgfpathclose%
\pgfusepath{fill}%
\end{pgfscope}%
\begin{pgfscope}%
\pgfpathrectangle{\pgfqpoint{1.150000in}{0.150000in}}{\pgfqpoint{5.700000in}{5.700000in}}%
\pgfusepath{clip}%
\pgfsetbuttcap%
\pgfsetroundjoin%
\definecolor{currentfill}{rgb}{0.153894,0.680203,0.504172}%
\pgfsetfillcolor{currentfill}%
\pgfsetfillopacity{0.800000}%
\pgfsetlinewidth{0.000000pt}%
\definecolor{currentstroke}{rgb}{0.000000,0.000000,0.000000}%
\pgfsetstrokecolor{currentstroke}%
\pgfsetdash{}{0pt}%
\pgfpathmoveto{\pgfqpoint{3.584197in}{4.130151in}}%
\pgfpathlineto{\pgfqpoint{3.597574in}{4.107307in}}%
\pgfpathlineto{\pgfqpoint{3.610945in}{4.084772in}}%
\pgfpathlineto{\pgfqpoint{3.624310in}{4.062542in}}%
\pgfpathlineto{\pgfqpoint{3.637669in}{4.040616in}}%
\pgfpathlineto{\pgfqpoint{3.645339in}{4.068678in}}%
\pgfpathlineto{\pgfqpoint{3.653006in}{4.097172in}}%
\pgfpathlineto{\pgfqpoint{3.660670in}{4.126104in}}%
\pgfpathlineto{\pgfqpoint{3.668331in}{4.155483in}}%
\pgfpathlineto{\pgfqpoint{3.654969in}{4.178079in}}%
\pgfpathlineto{\pgfqpoint{3.641600in}{4.200980in}}%
\pgfpathlineto{\pgfqpoint{3.628226in}{4.224188in}}%
\pgfpathlineto{\pgfqpoint{3.614844in}{4.247706in}}%
\pgfpathlineto{\pgfqpoint{3.607187in}{4.217640in}}%
\pgfpathlineto{\pgfqpoint{3.599527in}{4.188031in}}%
\pgfpathlineto{\pgfqpoint{3.591864in}{4.158871in}}%
\pgfpathlineto{\pgfqpoint{3.584197in}{4.130151in}}%
\pgfpathclose%
\pgfusepath{fill}%
\end{pgfscope}%
\begin{pgfscope}%
\pgfpathrectangle{\pgfqpoint{1.150000in}{0.150000in}}{\pgfqpoint{5.700000in}{5.700000in}}%
\pgfusepath{clip}%
\pgfsetbuttcap%
\pgfsetroundjoin%
\definecolor{currentfill}{rgb}{0.126453,0.570633,0.549841}%
\pgfsetfillcolor{currentfill}%
\pgfsetfillopacity{0.800000}%
\pgfsetlinewidth{0.000000pt}%
\definecolor{currentstroke}{rgb}{0.000000,0.000000,0.000000}%
\pgfsetstrokecolor{currentstroke}%
\pgfsetdash{}{0pt}%
\pgfpathmoveto{\pgfqpoint{3.491904in}{3.817103in}}%
\pgfpathlineto{\pgfqpoint{3.505277in}{3.796056in}}%
\pgfpathlineto{\pgfqpoint{3.518644in}{3.775314in}}%
\pgfpathlineto{\pgfqpoint{3.532005in}{3.754874in}}%
\pgfpathlineto{\pgfqpoint{3.545361in}{3.734733in}}%
\pgfpathlineto{\pgfqpoint{3.553075in}{3.758205in}}%
\pgfpathlineto{\pgfqpoint{3.560784in}{3.782020in}}%
\pgfpathlineto{\pgfqpoint{3.568490in}{3.806186in}}%
\pgfpathlineto{\pgfqpoint{3.576191in}{3.830710in}}%
\pgfpathlineto{\pgfqpoint{3.562835in}{3.851410in}}%
\pgfpathlineto{\pgfqpoint{3.549474in}{3.872409in}}%
\pgfpathlineto{\pgfqpoint{3.536107in}{3.893713in}}%
\pgfpathlineto{\pgfqpoint{3.522733in}{3.915322in}}%
\pgfpathlineto{\pgfqpoint{3.515033in}{3.890224in}}%
\pgfpathlineto{\pgfqpoint{3.507327in}{3.865493in}}%
\pgfpathlineto{\pgfqpoint{3.499618in}{3.841121in}}%
\pgfpathlineto{\pgfqpoint{3.491904in}{3.817103in}}%
\pgfpathclose%
\pgfusepath{fill}%
\end{pgfscope}%
\begin{pgfscope}%
\pgfpathrectangle{\pgfqpoint{1.150000in}{0.150000in}}{\pgfqpoint{5.700000in}{5.700000in}}%
\pgfusepath{clip}%
\pgfsetbuttcap%
\pgfsetroundjoin%
\definecolor{currentfill}{rgb}{0.226397,0.728888,0.462789}%
\pgfsetfillcolor{currentfill}%
\pgfsetfillopacity{0.800000}%
\pgfsetlinewidth{0.000000pt}%
\definecolor{currentstroke}{rgb}{0.000000,0.000000,0.000000}%
\pgfsetstrokecolor{currentstroke}%
\pgfsetdash{}{0pt}%
\pgfpathmoveto{\pgfqpoint{3.698953in}{4.277635in}}%
\pgfpathlineto{\pgfqpoint{3.712314in}{4.254633in}}%
\pgfpathlineto{\pgfqpoint{3.725670in}{4.231932in}}%
\pgfpathlineto{\pgfqpoint{3.739020in}{4.209528in}}%
\pgfpathlineto{\pgfqpoint{3.752365in}{4.187419in}}%
\pgfpathlineto{\pgfqpoint{3.760021in}{4.218430in}}%
\pgfpathlineto{\pgfqpoint{3.767674in}{4.249928in}}%
\pgfpathlineto{\pgfqpoint{3.775326in}{4.281924in}}%
\pgfpathlineto{\pgfqpoint{3.782977in}{4.314427in}}%
\pgfpathlineto{\pgfqpoint{3.769627in}{4.337278in}}%
\pgfpathlineto{\pgfqpoint{3.756271in}{4.360425in}}%
\pgfpathlineto{\pgfqpoint{3.742910in}{4.383872in}}%
\pgfpathlineto{\pgfqpoint{3.729542in}{4.407621in}}%
\pgfpathlineto{\pgfqpoint{3.721897in}{4.374359in}}%
\pgfpathlineto{\pgfqpoint{3.714251in}{4.341613in}}%
\pgfpathlineto{\pgfqpoint{3.706603in}{4.309375in}}%
\pgfpathlineto{\pgfqpoint{3.698953in}{4.277635in}}%
\pgfpathclose%
\pgfusepath{fill}%
\end{pgfscope}%
\begin{pgfscope}%
\pgfpathrectangle{\pgfqpoint{1.150000in}{0.150000in}}{\pgfqpoint{5.700000in}{5.700000in}}%
\pgfusepath{clip}%
\pgfsetbuttcap%
\pgfsetroundjoin%
\definecolor{currentfill}{rgb}{0.168126,0.459988,0.558082}%
\pgfsetfillcolor{currentfill}%
\pgfsetfillopacity{0.800000}%
\pgfsetlinewidth{0.000000pt}%
\definecolor{currentstroke}{rgb}{0.000000,0.000000,0.000000}%
\pgfsetstrokecolor{currentstroke}%
\pgfsetdash{}{0pt}%
\pgfpathmoveto{\pgfqpoint{3.536878in}{3.485016in}}%
\pgfpathlineto{\pgfqpoint{3.550215in}{3.467339in}}%
\pgfpathlineto{\pgfqpoint{3.563547in}{3.449945in}}%
\pgfpathlineto{\pgfqpoint{3.576876in}{3.432832in}}%
\pgfpathlineto{\pgfqpoint{3.590202in}{3.415997in}}%
\pgfpathlineto{\pgfqpoint{3.597947in}{3.435994in}}%
\pgfpathlineto{\pgfqpoint{3.605687in}{3.456269in}}%
\pgfpathlineto{\pgfqpoint{3.613423in}{3.476828in}}%
\pgfpathlineto{\pgfqpoint{3.621155in}{3.497677in}}%
\pgfpathlineto{\pgfqpoint{3.607832in}{3.514995in}}%
\pgfpathlineto{\pgfqpoint{3.594506in}{3.532591in}}%
\pgfpathlineto{\pgfqpoint{3.581176in}{3.550469in}}%
\pgfpathlineto{\pgfqpoint{3.567842in}{3.568631in}}%
\pgfpathlineto{\pgfqpoint{3.560108in}{3.547286in}}%
\pgfpathlineto{\pgfqpoint{3.552369in}{3.526239in}}%
\pgfpathlineto{\pgfqpoint{3.544626in}{3.505484in}}%
\pgfpathlineto{\pgfqpoint{3.536878in}{3.485016in}}%
\pgfpathclose%
\pgfusepath{fill}%
\end{pgfscope}%
\begin{pgfscope}%
\pgfpathrectangle{\pgfqpoint{1.150000in}{0.150000in}}{\pgfqpoint{5.700000in}{5.700000in}}%
\pgfusepath{clip}%
\pgfsetbuttcap%
\pgfsetroundjoin%
\definecolor{currentfill}{rgb}{0.199430,0.387607,0.554642}%
\pgfsetfillcolor{currentfill}%
\pgfsetfillopacity{0.800000}%
\pgfsetlinewidth{0.000000pt}%
\definecolor{currentstroke}{rgb}{0.000000,0.000000,0.000000}%
\pgfsetstrokecolor{currentstroke}%
\pgfsetdash{}{0pt}%
\pgfpathmoveto{\pgfqpoint{4.329605in}{3.263545in}}%
\pgfpathlineto{\pgfqpoint{4.342947in}{3.254763in}}%
\pgfpathlineto{\pgfqpoint{4.356294in}{3.246189in}}%
\pgfpathlineto{\pgfqpoint{4.369646in}{3.237823in}}%
\pgfpathlineto{\pgfqpoint{4.383002in}{3.229664in}}%
\pgfpathlineto{\pgfqpoint{4.390647in}{3.248906in}}%
\pgfpathlineto{\pgfqpoint{4.398292in}{3.268458in}}%
\pgfpathlineto{\pgfqpoint{4.405937in}{3.288325in}}%
\pgfpathlineto{\pgfqpoint{4.413582in}{3.308517in}}%
\pgfpathlineto{\pgfqpoint{4.400232in}{3.317303in}}%
\pgfpathlineto{\pgfqpoint{4.386887in}{3.326297in}}%
\pgfpathlineto{\pgfqpoint{4.373547in}{3.335499in}}%
\pgfpathlineto{\pgfqpoint{4.360210in}{3.344909in}}%
\pgfpathlineto{\pgfqpoint{4.352559in}{3.324078in}}%
\pgfpathlineto{\pgfqpoint{4.344908in}{3.303578in}}%
\pgfpathlineto{\pgfqpoint{4.337256in}{3.283403in}}%
\pgfpathlineto{\pgfqpoint{4.329605in}{3.263545in}}%
\pgfpathclose%
\pgfusepath{fill}%
\end{pgfscope}%
\begin{pgfscope}%
\pgfpathrectangle{\pgfqpoint{1.150000in}{0.150000in}}{\pgfqpoint{5.700000in}{5.700000in}}%
\pgfusepath{clip}%
\pgfsetbuttcap%
\pgfsetroundjoin%
\definecolor{currentfill}{rgb}{0.206756,0.371758,0.553117}%
\pgfsetfillcolor{currentfill}%
\pgfsetfillopacity{0.800000}%
\pgfsetlinewidth{0.000000pt}%
\definecolor{currentstroke}{rgb}{0.000000,0.000000,0.000000}%
\pgfsetstrokecolor{currentstroke}%
\pgfsetdash{}{0pt}%
\pgfpathmoveto{\pgfqpoint{3.749926in}{3.234910in}}%
\pgfpathlineto{\pgfqpoint{3.763227in}{3.221503in}}%
\pgfpathlineto{\pgfqpoint{3.776528in}{3.208347in}}%
\pgfpathlineto{\pgfqpoint{3.789829in}{3.195438in}}%
\pgfpathlineto{\pgfqpoint{3.803129in}{3.182777in}}%
\pgfpathlineto{\pgfqpoint{3.810859in}{3.200911in}}%
\pgfpathlineto{\pgfqpoint{3.818585in}{3.219290in}}%
\pgfpathlineto{\pgfqpoint{3.826308in}{3.237919in}}%
\pgfpathlineto{\pgfqpoint{3.834027in}{3.256806in}}%
\pgfpathlineto{\pgfqpoint{3.820730in}{3.269942in}}%
\pgfpathlineto{\pgfqpoint{3.807434in}{3.283325in}}%
\pgfpathlineto{\pgfqpoint{3.794137in}{3.296957in}}%
\pgfpathlineto{\pgfqpoint{3.780839in}{3.310840in}}%
\pgfpathlineto{\pgfqpoint{3.773117in}{3.291467in}}%
\pgfpathlineto{\pgfqpoint{3.765390in}{3.272357in}}%
\pgfpathlineto{\pgfqpoint{3.757660in}{3.253507in}}%
\pgfpathlineto{\pgfqpoint{3.749926in}{3.234910in}}%
\pgfpathclose%
\pgfusepath{fill}%
\end{pgfscope}%
\begin{pgfscope}%
\pgfpathrectangle{\pgfqpoint{1.150000in}{0.150000in}}{\pgfqpoint{5.700000in}{5.700000in}}%
\pgfusepath{clip}%
\pgfsetbuttcap%
\pgfsetroundjoin%
\definecolor{currentfill}{rgb}{0.192357,0.403199,0.555836}%
\pgfsetfillcolor{currentfill}%
\pgfsetfillopacity{0.800000}%
\pgfsetlinewidth{0.000000pt}%
\definecolor{currentstroke}{rgb}{0.000000,0.000000,0.000000}%
\pgfsetstrokecolor{currentstroke}%
\pgfsetdash{}{0pt}%
\pgfpathmoveto{\pgfqpoint{4.413582in}{3.308517in}}%
\pgfpathlineto{\pgfqpoint{4.426936in}{3.299935in}}%
\pgfpathlineto{\pgfqpoint{4.440295in}{3.291559in}}%
\pgfpathlineto{\pgfqpoint{4.453659in}{3.283387in}}%
\pgfpathlineto{\pgfqpoint{4.467028in}{3.275418in}}%
\pgfpathlineto{\pgfqpoint{4.474667in}{3.295293in}}%
\pgfpathlineto{\pgfqpoint{4.482306in}{3.315499in}}%
\pgfpathlineto{\pgfqpoint{4.489945in}{3.336044in}}%
\pgfpathlineto{\pgfqpoint{4.497586in}{3.356935in}}%
\pgfpathlineto{\pgfqpoint{4.484224in}{3.365563in}}%
\pgfpathlineto{\pgfqpoint{4.470867in}{3.374395in}}%
\pgfpathlineto{\pgfqpoint{4.457515in}{3.383430in}}%
\pgfpathlineto{\pgfqpoint{4.444167in}{3.392671in}}%
\pgfpathlineto{\pgfqpoint{4.436520in}{3.371109in}}%
\pgfpathlineto{\pgfqpoint{4.428873in}{3.349900in}}%
\pgfpathlineto{\pgfqpoint{4.421227in}{3.329039in}}%
\pgfpathlineto{\pgfqpoint{4.413582in}{3.308517in}}%
\pgfpathclose%
\pgfusepath{fill}%
\end{pgfscope}%
\begin{pgfscope}%
\pgfpathrectangle{\pgfqpoint{1.150000in}{0.150000in}}{\pgfqpoint{5.700000in}{5.700000in}}%
\pgfusepath{clip}%
\pgfsetbuttcap%
\pgfsetroundjoin%
\definecolor{currentfill}{rgb}{0.206756,0.371758,0.553117}%
\pgfsetfillcolor{currentfill}%
\pgfsetfillopacity{0.800000}%
\pgfsetlinewidth{0.000000pt}%
\definecolor{currentstroke}{rgb}{0.000000,0.000000,0.000000}%
\pgfsetstrokecolor{currentstroke}%
\pgfsetdash{}{0pt}%
\pgfpathmoveto{\pgfqpoint{4.245638in}{3.221987in}}%
\pgfpathlineto{\pgfqpoint{4.258971in}{3.212957in}}%
\pgfpathlineto{\pgfqpoint{4.272307in}{3.204139in}}%
\pgfpathlineto{\pgfqpoint{4.285647in}{3.195533in}}%
\pgfpathlineto{\pgfqpoint{4.298992in}{3.187137in}}%
\pgfpathlineto{\pgfqpoint{4.306646in}{3.205800in}}%
\pgfpathlineto{\pgfqpoint{4.314300in}{3.224750in}}%
\pgfpathlineto{\pgfqpoint{4.321953in}{3.243996in}}%
\pgfpathlineto{\pgfqpoint{4.329605in}{3.263545in}}%
\pgfpathlineto{\pgfqpoint{4.316266in}{3.272537in}}%
\pgfpathlineto{\pgfqpoint{4.302932in}{3.281739in}}%
\pgfpathlineto{\pgfqpoint{4.289602in}{3.291153in}}%
\pgfpathlineto{\pgfqpoint{4.276275in}{3.300780in}}%
\pgfpathlineto{\pgfqpoint{4.268617in}{3.280623in}}%
\pgfpathlineto{\pgfqpoint{4.260959in}{3.260777in}}%
\pgfpathlineto{\pgfqpoint{4.253299in}{3.241234in}}%
\pgfpathlineto{\pgfqpoint{4.245638in}{3.221987in}}%
\pgfpathclose%
\pgfusepath{fill}%
\end{pgfscope}%
\begin{pgfscope}%
\pgfpathrectangle{\pgfqpoint{1.150000in}{0.150000in}}{\pgfqpoint{5.700000in}{5.700000in}}%
\pgfusepath{clip}%
\pgfsetbuttcap%
\pgfsetroundjoin%
\definecolor{currentfill}{rgb}{0.165117,0.467423,0.558141}%
\pgfsetfillcolor{currentfill}%
\pgfsetfillopacity{0.800000}%
\pgfsetlinewidth{0.000000pt}%
\definecolor{currentstroke}{rgb}{0.000000,0.000000,0.000000}%
\pgfsetstrokecolor{currentstroke}%
\pgfsetdash{}{0pt}%
\pgfpathmoveto{\pgfqpoint{4.612218in}{3.499350in}}%
\pgfpathlineto{\pgfqpoint{4.625592in}{3.490310in}}%
\pgfpathlineto{\pgfqpoint{4.638970in}{3.481469in}}%
\pgfpathlineto{\pgfqpoint{4.652353in}{3.472826in}}%
\pgfpathlineto{\pgfqpoint{4.665743in}{3.464379in}}%
\pgfpathlineto{\pgfqpoint{4.673389in}{3.487270in}}%
\pgfpathlineto{\pgfqpoint{4.681038in}{3.510575in}}%
\pgfpathlineto{\pgfqpoint{4.688692in}{3.534303in}}%
\pgfpathlineto{\pgfqpoint{4.675308in}{3.543314in}}%
\pgfpathlineto{\pgfqpoint{4.661930in}{3.552522in}}%
\pgfpathlineto{\pgfqpoint{4.648558in}{3.561928in}}%
\pgfpathlineto{\pgfqpoint{4.635190in}{3.571533in}}%
\pgfpathlineto{\pgfqpoint{4.627529in}{3.547043in}}%
\pgfpathlineto{\pgfqpoint{4.619872in}{3.522985in}}%
\pgfpathlineto{\pgfqpoint{4.612218in}{3.499350in}}%
\pgfpathclose%
\pgfusepath{fill}%
\end{pgfscope}%
\begin{pgfscope}%
\pgfpathrectangle{\pgfqpoint{1.150000in}{0.150000in}}{\pgfqpoint{5.700000in}{5.700000in}}%
\pgfusepath{clip}%
\pgfsetbuttcap%
\pgfsetroundjoin%
\definecolor{currentfill}{rgb}{0.214298,0.355619,0.551184}%
\pgfsetfillcolor{currentfill}%
\pgfsetfillopacity{0.800000}%
\pgfsetlinewidth{0.000000pt}%
\definecolor{currentstroke}{rgb}{0.000000,0.000000,0.000000}%
\pgfsetstrokecolor{currentstroke}%
\pgfsetdash{}{0pt}%
\pgfpathmoveto{\pgfqpoint{4.161666in}{3.183836in}}%
\pgfpathlineto{\pgfqpoint{4.174990in}{3.174510in}}%
\pgfpathlineto{\pgfqpoint{4.188317in}{3.165401in}}%
\pgfpathlineto{\pgfqpoint{4.201647in}{3.156507in}}%
\pgfpathlineto{\pgfqpoint{4.214982in}{3.147828in}}%
\pgfpathlineto{\pgfqpoint{4.222648in}{3.165957in}}%
\pgfpathlineto{\pgfqpoint{4.230313in}{3.184356in}}%
\pgfpathlineto{\pgfqpoint{4.237976in}{3.203030in}}%
\pgfpathlineto{\pgfqpoint{4.245638in}{3.221987in}}%
\pgfpathlineto{\pgfqpoint{4.232310in}{3.231231in}}%
\pgfpathlineto{\pgfqpoint{4.218985in}{3.240689in}}%
\pgfpathlineto{\pgfqpoint{4.205663in}{3.250364in}}%
\pgfpathlineto{\pgfqpoint{4.192345in}{3.260255in}}%
\pgfpathlineto{\pgfqpoint{4.184678in}{3.240721in}}%
\pgfpathlineto{\pgfqpoint{4.177009in}{3.221477in}}%
\pgfpathlineto{\pgfqpoint{4.169338in}{3.202518in}}%
\pgfpathlineto{\pgfqpoint{4.161666in}{3.183836in}}%
\pgfpathclose%
\pgfusepath{fill}%
\end{pgfscope}%
\begin{pgfscope}%
\pgfpathrectangle{\pgfqpoint{1.150000in}{0.150000in}}{\pgfqpoint{5.700000in}{5.700000in}}%
\pgfusepath{clip}%
\pgfsetbuttcap%
\pgfsetroundjoin%
\definecolor{currentfill}{rgb}{0.183898,0.422383,0.556944}%
\pgfsetfillcolor{currentfill}%
\pgfsetfillopacity{0.800000}%
\pgfsetlinewidth{0.000000pt}%
\definecolor{currentstroke}{rgb}{0.000000,0.000000,0.000000}%
\pgfsetstrokecolor{currentstroke}%
\pgfsetdash{}{0pt}%
\pgfpathmoveto{\pgfqpoint{4.497586in}{3.356935in}}%
\pgfpathlineto{\pgfqpoint{4.510953in}{3.348509in}}%
\pgfpathlineto{\pgfqpoint{4.524325in}{3.340284in}}%
\pgfpathlineto{\pgfqpoint{4.537703in}{3.332260in}}%
\pgfpathlineto{\pgfqpoint{4.551085in}{3.324436in}}%
\pgfpathlineto{\pgfqpoint{4.558720in}{3.345002in}}%
\pgfpathlineto{\pgfqpoint{4.566356in}{3.365923in}}%
\pgfpathlineto{\pgfqpoint{4.573994in}{3.387206in}}%
\pgfpathlineto{\pgfqpoint{4.581634in}{3.408860in}}%
\pgfpathlineto{\pgfqpoint{4.568259in}{3.417375in}}%
\pgfpathlineto{\pgfqpoint{4.554889in}{3.426090in}}%
\pgfpathlineto{\pgfqpoint{4.541524in}{3.435006in}}%
\pgfpathlineto{\pgfqpoint{4.528164in}{3.444124in}}%
\pgfpathlineto{\pgfqpoint{4.520517in}{3.421766in}}%
\pgfpathlineto{\pgfqpoint{4.512872in}{3.399788in}}%
\pgfpathlineto{\pgfqpoint{4.505228in}{3.378180in}}%
\pgfpathlineto{\pgfqpoint{4.497586in}{3.356935in}}%
\pgfpathclose%
\pgfusepath{fill}%
\end{pgfscope}%
\begin{pgfscope}%
\pgfpathrectangle{\pgfqpoint{1.150000in}{0.150000in}}{\pgfqpoint{5.700000in}{5.700000in}}%
\pgfusepath{clip}%
\pgfsetbuttcap%
\pgfsetroundjoin%
\definecolor{currentfill}{rgb}{0.218130,0.347432,0.550038}%
\pgfsetfillcolor{currentfill}%
\pgfsetfillopacity{0.800000}%
\pgfsetlinewidth{0.000000pt}%
\definecolor{currentstroke}{rgb}{0.000000,0.000000,0.000000}%
\pgfsetstrokecolor{currentstroke}%
\pgfsetdash{}{0pt}%
\pgfpathmoveto{\pgfqpoint{3.940416in}{3.160420in}}%
\pgfpathlineto{\pgfqpoint{3.953720in}{3.149436in}}%
\pgfpathlineto{\pgfqpoint{3.967025in}{3.138684in}}%
\pgfpathlineto{\pgfqpoint{3.980332in}{3.128162in}}%
\pgfpathlineto{\pgfqpoint{3.993641in}{3.117869in}}%
\pgfpathlineto{\pgfqpoint{4.001344in}{3.135539in}}%
\pgfpathlineto{\pgfqpoint{4.009043in}{3.153453in}}%
\pgfpathlineto{\pgfqpoint{4.016739in}{3.171616in}}%
\pgfpathlineto{\pgfqpoint{4.024433in}{3.190035in}}%
\pgfpathlineto{\pgfqpoint{4.011129in}{3.200832in}}%
\pgfpathlineto{\pgfqpoint{3.997826in}{3.211857in}}%
\pgfpathlineto{\pgfqpoint{3.984526in}{3.223113in}}%
\pgfpathlineto{\pgfqpoint{3.971227in}{3.234601in}}%
\pgfpathlineto{\pgfqpoint{3.963528in}{3.215666in}}%
\pgfpathlineto{\pgfqpoint{3.955827in}{3.196995in}}%
\pgfpathlineto{\pgfqpoint{3.948123in}{3.178581in}}%
\pgfpathlineto{\pgfqpoint{3.940416in}{3.160420in}}%
\pgfpathclose%
\pgfusepath{fill}%
\end{pgfscope}%
\begin{pgfscope}%
\pgfpathrectangle{\pgfqpoint{1.150000in}{0.150000in}}{\pgfqpoint{5.700000in}{5.700000in}}%
\pgfusepath{clip}%
\pgfsetbuttcap%
\pgfsetroundjoin%
\definecolor{currentfill}{rgb}{0.136408,0.541173,0.554483}%
\pgfsetfillcolor{currentfill}%
\pgfsetfillopacity{0.800000}%
\pgfsetlinewidth{0.000000pt}%
\definecolor{currentstroke}{rgb}{0.000000,0.000000,0.000000}%
\pgfsetstrokecolor{currentstroke}%
\pgfsetdash{}{0pt}%
\pgfpathmoveto{\pgfqpoint{3.461001in}{3.724434in}}%
\pgfpathlineto{\pgfqpoint{3.474375in}{3.703913in}}%
\pgfpathlineto{\pgfqpoint{3.487743in}{3.683696in}}%
\pgfpathlineto{\pgfqpoint{3.501105in}{3.663780in}}%
\pgfpathlineto{\pgfqpoint{3.514462in}{3.644163in}}%
\pgfpathlineto{\pgfqpoint{3.522194in}{3.666321in}}%
\pgfpathlineto{\pgfqpoint{3.529921in}{3.688798in}}%
\pgfpathlineto{\pgfqpoint{3.537643in}{3.711600in}}%
\pgfpathlineto{\pgfqpoint{3.545361in}{3.734733in}}%
\pgfpathlineto{\pgfqpoint{3.532005in}{3.754874in}}%
\pgfpathlineto{\pgfqpoint{3.518644in}{3.775314in}}%
\pgfpathlineto{\pgfqpoint{3.505277in}{3.796056in}}%
\pgfpathlineto{\pgfqpoint{3.491904in}{3.817103in}}%
\pgfpathlineto{\pgfqpoint{3.484185in}{3.793432in}}%
\pgfpathlineto{\pgfqpoint{3.476462in}{3.770101in}}%
\pgfpathlineto{\pgfqpoint{3.468734in}{3.747104in}}%
\pgfpathlineto{\pgfqpoint{3.461001in}{3.724434in}}%
\pgfpathclose%
\pgfusepath{fill}%
\end{pgfscope}%
\begin{pgfscope}%
\pgfpathrectangle{\pgfqpoint{1.150000in}{0.150000in}}{\pgfqpoint{5.700000in}{5.700000in}}%
\pgfusepath{clip}%
\pgfsetbuttcap%
\pgfsetroundjoin%
\definecolor{currentfill}{rgb}{0.208030,0.718701,0.472873}%
\pgfsetfillcolor{currentfill}%
\pgfsetfillopacity{0.800000}%
\pgfsetlinewidth{0.000000pt}%
\definecolor{currentstroke}{rgb}{0.000000,0.000000,0.000000}%
\pgfsetstrokecolor{currentstroke}%
\pgfsetdash{}{0pt}%
\pgfpathmoveto{\pgfqpoint{3.614844in}{4.247706in}}%
\pgfpathlineto{\pgfqpoint{3.628226in}{4.224188in}}%
\pgfpathlineto{\pgfqpoint{3.641600in}{4.200980in}}%
\pgfpathlineto{\pgfqpoint{3.654969in}{4.178079in}}%
\pgfpathlineto{\pgfqpoint{3.668331in}{4.155483in}}%
\pgfpathlineto{\pgfqpoint{3.675990in}{4.185317in}}%
\pgfpathlineto{\pgfqpoint{3.683647in}{4.215615in}}%
\pgfpathlineto{\pgfqpoint{3.691301in}{4.246385in}}%
\pgfpathlineto{\pgfqpoint{3.698953in}{4.277635in}}%
\pgfpathlineto{\pgfqpoint{3.685585in}{4.300940in}}%
\pgfpathlineto{\pgfqpoint{3.672212in}{4.324551in}}%
\pgfpathlineto{\pgfqpoint{3.658832in}{4.348471in}}%
\pgfpathlineto{\pgfqpoint{3.645446in}{4.372703in}}%
\pgfpathlineto{\pgfqpoint{3.637799in}{4.340726in}}%
\pgfpathlineto{\pgfqpoint{3.630150in}{4.309240in}}%
\pgfpathlineto{\pgfqpoint{3.622499in}{4.278236in}}%
\pgfpathlineto{\pgfqpoint{3.614844in}{4.247706in}}%
\pgfpathclose%
\pgfusepath{fill}%
\end{pgfscope}%
\begin{pgfscope}%
\pgfpathrectangle{\pgfqpoint{1.150000in}{0.150000in}}{\pgfqpoint{5.700000in}{5.700000in}}%
\pgfusepath{clip}%
\pgfsetbuttcap%
\pgfsetroundjoin%
\definecolor{currentfill}{rgb}{0.157729,0.485932,0.558013}%
\pgfsetfillcolor{currentfill}%
\pgfsetfillopacity{0.800000}%
\pgfsetlinewidth{0.000000pt}%
\definecolor{currentstroke}{rgb}{0.000000,0.000000,0.000000}%
\pgfsetstrokecolor{currentstroke}%
\pgfsetdash{}{0pt}%
\pgfpathmoveto{\pgfqpoint{3.483489in}{3.558602in}}%
\pgfpathlineto{\pgfqpoint{3.496844in}{3.539769in}}%
\pgfpathlineto{\pgfqpoint{3.510193in}{3.521229in}}%
\pgfpathlineto{\pgfqpoint{3.523538in}{3.502979in}}%
\pgfpathlineto{\pgfqpoint{3.536878in}{3.485016in}}%
\pgfpathlineto{\pgfqpoint{3.544626in}{3.505484in}}%
\pgfpathlineto{\pgfqpoint{3.552369in}{3.526239in}}%
\pgfpathlineto{\pgfqpoint{3.560108in}{3.547286in}}%
\pgfpathlineto{\pgfqpoint{3.567842in}{3.568631in}}%
\pgfpathlineto{\pgfqpoint{3.554504in}{3.587079in}}%
\pgfpathlineto{\pgfqpoint{3.541161in}{3.605815in}}%
\pgfpathlineto{\pgfqpoint{3.527814in}{3.624843in}}%
\pgfpathlineto{\pgfqpoint{3.514462in}{3.644163in}}%
\pgfpathlineto{\pgfqpoint{3.506726in}{3.622318in}}%
\pgfpathlineto{\pgfqpoint{3.498985in}{3.600781in}}%
\pgfpathlineto{\pgfqpoint{3.491240in}{3.579544in}}%
\pgfpathlineto{\pgfqpoint{3.483489in}{3.558602in}}%
\pgfpathclose%
\pgfusepath{fill}%
\end{pgfscope}%
\begin{pgfscope}%
\pgfpathrectangle{\pgfqpoint{1.150000in}{0.150000in}}{\pgfqpoint{5.700000in}{5.700000in}}%
\pgfusepath{clip}%
\pgfsetbuttcap%
\pgfsetroundjoin%
\definecolor{currentfill}{rgb}{0.214298,0.355619,0.551184}%
\pgfsetfillcolor{currentfill}%
\pgfsetfillopacity{0.800000}%
\pgfsetlinewidth{0.000000pt}%
\definecolor{currentstroke}{rgb}{0.000000,0.000000,0.000000}%
\pgfsetstrokecolor{currentstroke}%
\pgfsetdash{}{0pt}%
\pgfpathmoveto{\pgfqpoint{3.803129in}{3.182777in}}%
\pgfpathlineto{\pgfqpoint{3.816430in}{3.170361in}}%
\pgfpathlineto{\pgfqpoint{3.829731in}{3.158188in}}%
\pgfpathlineto{\pgfqpoint{3.843033in}{3.146257in}}%
\pgfpathlineto{\pgfqpoint{3.856335in}{3.134566in}}%
\pgfpathlineto{\pgfqpoint{3.864060in}{3.152239in}}%
\pgfpathlineto{\pgfqpoint{3.871782in}{3.170148in}}%
\pgfpathlineto{\pgfqpoint{3.879500in}{3.188300in}}%
\pgfpathlineto{\pgfqpoint{3.887215in}{3.206700in}}%
\pgfpathlineto{\pgfqpoint{3.873917in}{3.218864in}}%
\pgfpathlineto{\pgfqpoint{3.860620in}{3.231268in}}%
\pgfpathlineto{\pgfqpoint{3.847323in}{3.243915in}}%
\pgfpathlineto{\pgfqpoint{3.834027in}{3.256806in}}%
\pgfpathlineto{\pgfqpoint{3.826308in}{3.237919in}}%
\pgfpathlineto{\pgfqpoint{3.818585in}{3.219290in}}%
\pgfpathlineto{\pgfqpoint{3.810859in}{3.200911in}}%
\pgfpathlineto{\pgfqpoint{3.803129in}{3.182777in}}%
\pgfpathclose%
\pgfusepath{fill}%
\end{pgfscope}%
\begin{pgfscope}%
\pgfpathrectangle{\pgfqpoint{1.150000in}{0.150000in}}{\pgfqpoint{5.700000in}{5.700000in}}%
\pgfusepath{clip}%
\pgfsetbuttcap%
\pgfsetroundjoin%
\definecolor{currentfill}{rgb}{0.220057,0.343307,0.549413}%
\pgfsetfillcolor{currentfill}%
\pgfsetfillopacity{0.800000}%
\pgfsetlinewidth{0.000000pt}%
\definecolor{currentstroke}{rgb}{0.000000,0.000000,0.000000}%
\pgfsetstrokecolor{currentstroke}%
\pgfsetdash{}{0pt}%
\pgfpathmoveto{\pgfqpoint{4.077673in}{3.149114in}}%
\pgfpathlineto{\pgfqpoint{4.090989in}{3.139443in}}%
\pgfpathlineto{\pgfqpoint{4.104309in}{3.129994in}}%
\pgfpathlineto{\pgfqpoint{4.117631in}{3.120764in}}%
\pgfpathlineto{\pgfqpoint{4.130957in}{3.111754in}}%
\pgfpathlineto{\pgfqpoint{4.138638in}{3.129390in}}%
\pgfpathlineto{\pgfqpoint{4.146316in}{3.147279in}}%
\pgfpathlineto{\pgfqpoint{4.153992in}{3.165425in}}%
\pgfpathlineto{\pgfqpoint{4.161666in}{3.183836in}}%
\pgfpathlineto{\pgfqpoint{4.148346in}{3.193380in}}%
\pgfpathlineto{\pgfqpoint{4.135029in}{3.203143in}}%
\pgfpathlineto{\pgfqpoint{4.121715in}{3.213127in}}%
\pgfpathlineto{\pgfqpoint{4.108403in}{3.223331in}}%
\pgfpathlineto{\pgfqpoint{4.100724in}{3.204375in}}%
\pgfpathlineto{\pgfqpoint{4.093043in}{3.185690in}}%
\pgfpathlineto{\pgfqpoint{4.085359in}{3.167272in}}%
\pgfpathlineto{\pgfqpoint{4.077673in}{3.149114in}}%
\pgfpathclose%
\pgfusepath{fill}%
\end{pgfscope}%
\begin{pgfscope}%
\pgfpathrectangle{\pgfqpoint{1.150000in}{0.150000in}}{\pgfqpoint{5.700000in}{5.700000in}}%
\pgfusepath{clip}%
\pgfsetbuttcap%
\pgfsetroundjoin%
\definecolor{currentfill}{rgb}{0.175841,0.441290,0.557685}%
\pgfsetfillcolor{currentfill}%
\pgfsetfillopacity{0.800000}%
\pgfsetlinewidth{0.000000pt}%
\definecolor{currentstroke}{rgb}{0.000000,0.000000,0.000000}%
\pgfsetstrokecolor{currentstroke}%
\pgfsetdash{}{0pt}%
\pgfpathmoveto{\pgfqpoint{4.581634in}{3.408860in}}%
\pgfpathlineto{\pgfqpoint{4.595015in}{3.400544in}}%
\pgfpathlineto{\pgfqpoint{4.608401in}{3.392426in}}%
\pgfpathlineto{\pgfqpoint{4.621792in}{3.384506in}}%
\pgfpathlineto{\pgfqpoint{4.635190in}{3.376782in}}%
\pgfpathlineto{\pgfqpoint{4.642824in}{3.398104in}}%
\pgfpathlineto{\pgfqpoint{4.650460in}{3.419805in}}%
\pgfpathlineto{\pgfqpoint{4.658100in}{3.441894in}}%
\pgfpathlineto{\pgfqpoint{4.665743in}{3.464379in}}%
\pgfpathlineto{\pgfqpoint{4.652353in}{3.472826in}}%
\pgfpathlineto{\pgfqpoint{4.638970in}{3.481469in}}%
\pgfpathlineto{\pgfqpoint{4.625592in}{3.490310in}}%
\pgfpathlineto{\pgfqpoint{4.612218in}{3.499350in}}%
\pgfpathlineto{\pgfqpoint{4.604568in}{3.476129in}}%
\pgfpathlineto{\pgfqpoint{4.596921in}{3.453313in}}%
\pgfpathlineto{\pgfqpoint{4.589276in}{3.430892in}}%
\pgfpathlineto{\pgfqpoint{4.581634in}{3.408860in}}%
\pgfpathclose%
\pgfusepath{fill}%
\end{pgfscope}%
\begin{pgfscope}%
\pgfpathrectangle{\pgfqpoint{1.150000in}{0.150000in}}{\pgfqpoint{5.700000in}{5.700000in}}%
\pgfusepath{clip}%
\pgfsetbuttcap%
\pgfsetroundjoin%
\definecolor{currentfill}{rgb}{0.123444,0.636809,0.528763}%
\pgfsetfillcolor{currentfill}%
\pgfsetfillopacity{0.800000}%
\pgfsetlinewidth{0.000000pt}%
\definecolor{currentstroke}{rgb}{0.000000,0.000000,0.000000}%
\pgfsetstrokecolor{currentstroke}%
\pgfsetdash{}{0pt}%
\pgfpathmoveto{\pgfqpoint{3.469176in}{4.004877in}}%
\pgfpathlineto{\pgfqpoint{3.482575in}{3.982015in}}%
\pgfpathlineto{\pgfqpoint{3.495968in}{3.959470in}}%
\pgfpathlineto{\pgfqpoint{3.509354in}{3.937240in}}%
\pgfpathlineto{\pgfqpoint{3.522733in}{3.915322in}}%
\pgfpathlineto{\pgfqpoint{3.530430in}{3.940793in}}%
\pgfpathlineto{\pgfqpoint{3.538123in}{3.966645in}}%
\pgfpathlineto{\pgfqpoint{3.545811in}{3.992884in}}%
\pgfpathlineto{\pgfqpoint{3.553496in}{4.019519in}}%
\pgfpathlineto{\pgfqpoint{3.540115in}{4.042037in}}%
\pgfpathlineto{\pgfqpoint{3.526727in}{4.064868in}}%
\pgfpathlineto{\pgfqpoint{3.513333in}{4.088014in}}%
\pgfpathlineto{\pgfqpoint{3.499931in}{4.111480in}}%
\pgfpathlineto{\pgfqpoint{3.492249in}{4.084231in}}%
\pgfpathlineto{\pgfqpoint{3.484562in}{4.057385in}}%
\pgfpathlineto{\pgfqpoint{3.476871in}{4.030936in}}%
\pgfpathlineto{\pgfqpoint{3.469176in}{4.004877in}}%
\pgfpathclose%
\pgfusepath{fill}%
\end{pgfscope}%
\begin{pgfscope}%
\pgfpathrectangle{\pgfqpoint{1.150000in}{0.150000in}}{\pgfqpoint{5.700000in}{5.700000in}}%
\pgfusepath{clip}%
\pgfsetbuttcap%
\pgfsetroundjoin%
\definecolor{currentfill}{rgb}{0.150148,0.676631,0.506589}%
\pgfsetfillcolor{currentfill}%
\pgfsetfillopacity{0.800000}%
\pgfsetlinewidth{0.000000pt}%
\definecolor{currentstroke}{rgb}{0.000000,0.000000,0.000000}%
\pgfsetstrokecolor{currentstroke}%
\pgfsetdash{}{0pt}%
\pgfpathmoveto{\pgfqpoint{3.499931in}{4.111480in}}%
\pgfpathlineto{\pgfqpoint{3.513333in}{4.088014in}}%
\pgfpathlineto{\pgfqpoint{3.526727in}{4.064868in}}%
\pgfpathlineto{\pgfqpoint{3.540115in}{4.042037in}}%
\pgfpathlineto{\pgfqpoint{3.553496in}{4.019519in}}%
\pgfpathlineto{\pgfqpoint{3.561176in}{4.046555in}}%
\pgfpathlineto{\pgfqpoint{3.568854in}{4.074001in}}%
\pgfpathlineto{\pgfqpoint{3.576527in}{4.101863in}}%
\pgfpathlineto{\pgfqpoint{3.584197in}{4.130151in}}%
\pgfpathlineto{\pgfqpoint{3.570813in}{4.153306in}}%
\pgfpathlineto{\pgfqpoint{3.557423in}{4.176775in}}%
\pgfpathlineto{\pgfqpoint{3.544025in}{4.200562in}}%
\pgfpathlineto{\pgfqpoint{3.530620in}{4.224670in}}%
\pgfpathlineto{\pgfqpoint{3.522954in}{4.195729in}}%
\pgfpathlineto{\pgfqpoint{3.515284in}{4.167222in}}%
\pgfpathlineto{\pgfqpoint{3.507609in}{4.139142in}}%
\pgfpathlineto{\pgfqpoint{3.499931in}{4.111480in}}%
\pgfpathclose%
\pgfusepath{fill}%
\end{pgfscope}%
\begin{pgfscope}%
\pgfpathrectangle{\pgfqpoint{1.150000in}{0.150000in}}{\pgfqpoint{5.700000in}{5.700000in}}%
\pgfusepath{clip}%
\pgfsetbuttcap%
\pgfsetroundjoin%
\definecolor{currentfill}{rgb}{0.335885,0.777018,0.402049}%
\pgfsetfillcolor{currentfill}%
\pgfsetfillopacity{0.800000}%
\pgfsetlinewidth{0.000000pt}%
\definecolor{currentstroke}{rgb}{0.000000,0.000000,0.000000}%
\pgfsetstrokecolor{currentstroke}%
\pgfsetdash{}{0pt}%
\pgfpathmoveto{\pgfqpoint{3.813573in}{4.449687in}}%
\pgfpathlineto{\pgfqpoint{3.826925in}{4.426349in}}%
\pgfpathlineto{\pgfqpoint{3.840272in}{4.403305in}}%
\pgfpathlineto{\pgfqpoint{3.853613in}{4.380551in}}%
\pgfpathlineto{\pgfqpoint{3.866950in}{4.358086in}}%
\pgfpathlineto{\pgfqpoint{3.874605in}{4.392457in}}%
\pgfpathlineto{\pgfqpoint{3.882260in}{4.427382in}}%
\pgfpathlineto{\pgfqpoint{3.889915in}{4.462870in}}%
\pgfpathlineto{\pgfqpoint{3.876573in}{4.485945in}}%
\pgfpathlineto{\pgfqpoint{3.863226in}{4.509309in}}%
\pgfpathlineto{\pgfqpoint{3.849874in}{4.532965in}}%
\pgfpathlineto{\pgfqpoint{3.836516in}{4.556916in}}%
\pgfpathlineto{\pgfqpoint{3.828868in}{4.520602in}}%
\pgfpathlineto{\pgfqpoint{3.821221in}{4.484863in}}%
\pgfpathlineto{\pgfqpoint{3.813573in}{4.449687in}}%
\pgfpathclose%
\pgfusepath{fill}%
\end{pgfscope}%
\begin{pgfscope}%
\pgfpathrectangle{\pgfqpoint{1.150000in}{0.150000in}}{\pgfqpoint{5.700000in}{5.700000in}}%
\pgfusepath{clip}%
\pgfsetbuttcap%
\pgfsetroundjoin%
\definecolor{currentfill}{rgb}{0.119738,0.603785,0.541400}%
\pgfsetfillcolor{currentfill}%
\pgfsetfillopacity{0.800000}%
\pgfsetlinewidth{0.000000pt}%
\definecolor{currentstroke}{rgb}{0.000000,0.000000,0.000000}%
\pgfsetstrokecolor{currentstroke}%
\pgfsetdash{}{0pt}%
\pgfpathmoveto{\pgfqpoint{3.438348in}{3.904400in}}%
\pgfpathlineto{\pgfqpoint{3.451747in}{3.882104in}}%
\pgfpathlineto{\pgfqpoint{3.465139in}{3.860124in}}%
\pgfpathlineto{\pgfqpoint{3.478525in}{3.838458in}}%
\pgfpathlineto{\pgfqpoint{3.491904in}{3.817103in}}%
\pgfpathlineto{\pgfqpoint{3.499618in}{3.841121in}}%
\pgfpathlineto{\pgfqpoint{3.507327in}{3.865493in}}%
\pgfpathlineto{\pgfqpoint{3.515033in}{3.890224in}}%
\pgfpathlineto{\pgfqpoint{3.522733in}{3.915322in}}%
\pgfpathlineto{\pgfqpoint{3.509354in}{3.937240in}}%
\pgfpathlineto{\pgfqpoint{3.495968in}{3.959470in}}%
\pgfpathlineto{\pgfqpoint{3.482575in}{3.982015in}}%
\pgfpathlineto{\pgfqpoint{3.469176in}{4.004877in}}%
\pgfpathlineto{\pgfqpoint{3.461476in}{3.979201in}}%
\pgfpathlineto{\pgfqpoint{3.453771in}{3.953900in}}%
\pgfpathlineto{\pgfqpoint{3.446062in}{3.928969in}}%
\pgfpathlineto{\pgfqpoint{3.438348in}{3.904400in}}%
\pgfpathclose%
\pgfusepath{fill}%
\end{pgfscope}%
\begin{pgfscope}%
\pgfpathrectangle{\pgfqpoint{1.150000in}{0.150000in}}{\pgfqpoint{5.700000in}{5.700000in}}%
\pgfusepath{clip}%
\pgfsetbuttcap%
\pgfsetroundjoin%
\definecolor{currentfill}{rgb}{0.199430,0.387607,0.554642}%
\pgfsetfillcolor{currentfill}%
\pgfsetfillopacity{0.800000}%
\pgfsetlinewidth{0.000000pt}%
\definecolor{currentstroke}{rgb}{0.000000,0.000000,0.000000}%
\pgfsetstrokecolor{currentstroke}%
\pgfsetdash{}{0pt}%
\pgfpathmoveto{\pgfqpoint{3.612466in}{3.275874in}}%
\pgfpathlineto{\pgfqpoint{3.625782in}{3.260844in}}%
\pgfpathlineto{\pgfqpoint{3.639096in}{3.246080in}}%
\pgfpathlineto{\pgfqpoint{3.652408in}{3.231578in}}%
\pgfpathlineto{\pgfqpoint{3.665719in}{3.217338in}}%
\pgfpathlineto{\pgfqpoint{3.673473in}{3.235407in}}%
\pgfpathlineto{\pgfqpoint{3.681224in}{3.253717in}}%
\pgfpathlineto{\pgfqpoint{3.688970in}{3.272272in}}%
\pgfpathlineto{\pgfqpoint{3.696712in}{3.291077in}}%
\pgfpathlineto{\pgfqpoint{3.683406in}{3.305763in}}%
\pgfpathlineto{\pgfqpoint{3.670097in}{3.320710in}}%
\pgfpathlineto{\pgfqpoint{3.656787in}{3.335921in}}%
\pgfpathlineto{\pgfqpoint{3.643475in}{3.351397in}}%
\pgfpathlineto{\pgfqpoint{3.635730in}{3.332133in}}%
\pgfpathlineto{\pgfqpoint{3.627979in}{3.313128in}}%
\pgfpathlineto{\pgfqpoint{3.620225in}{3.294377in}}%
\pgfpathlineto{\pgfqpoint{3.612466in}{3.275874in}}%
\pgfpathclose%
\pgfusepath{fill}%
\end{pgfscope}%
\begin{pgfscope}%
\pgfpathrectangle{\pgfqpoint{1.150000in}{0.150000in}}{\pgfqpoint{5.700000in}{5.700000in}}%
\pgfusepath{clip}%
\pgfsetbuttcap%
\pgfsetroundjoin%
\definecolor{currentfill}{rgb}{0.188923,0.410910,0.556326}%
\pgfsetfillcolor{currentfill}%
\pgfsetfillopacity{0.800000}%
\pgfsetlinewidth{0.000000pt}%
\definecolor{currentstroke}{rgb}{0.000000,0.000000,0.000000}%
\pgfsetstrokecolor{currentstroke}%
\pgfsetdash{}{0pt}%
\pgfpathmoveto{\pgfqpoint{3.559177in}{3.338683in}}%
\pgfpathlineto{\pgfqpoint{3.572504in}{3.322573in}}%
\pgfpathlineto{\pgfqpoint{3.585827in}{3.306736in}}%
\pgfpathlineto{\pgfqpoint{3.599148in}{3.291170in}}%
\pgfpathlineto{\pgfqpoint{3.612466in}{3.275874in}}%
\pgfpathlineto{\pgfqpoint{3.620225in}{3.294377in}}%
\pgfpathlineto{\pgfqpoint{3.627979in}{3.313128in}}%
\pgfpathlineto{\pgfqpoint{3.635730in}{3.332133in}}%
\pgfpathlineto{\pgfqpoint{3.643475in}{3.351397in}}%
\pgfpathlineto{\pgfqpoint{3.630161in}{3.367140in}}%
\pgfpathlineto{\pgfqpoint{3.616844in}{3.383154in}}%
\pgfpathlineto{\pgfqpoint{3.603525in}{3.399438in}}%
\pgfpathlineto{\pgfqpoint{3.590202in}{3.415997in}}%
\pgfpathlineto{\pgfqpoint{3.582453in}{3.396273in}}%
\pgfpathlineto{\pgfqpoint{3.574699in}{3.376816in}}%
\pgfpathlineto{\pgfqpoint{3.566941in}{3.357621in}}%
\pgfpathlineto{\pgfqpoint{3.559177in}{3.338683in}}%
\pgfpathclose%
\pgfusepath{fill}%
\end{pgfscope}%
\begin{pgfscope}%
\pgfpathrectangle{\pgfqpoint{1.150000in}{0.150000in}}{\pgfqpoint{5.700000in}{5.700000in}}%
\pgfusepath{clip}%
\pgfsetbuttcap%
\pgfsetroundjoin%
\definecolor{currentfill}{rgb}{0.319809,0.770914,0.411152}%
\pgfsetfillcolor{currentfill}%
\pgfsetfillopacity{0.800000}%
\pgfsetlinewidth{0.000000pt}%
\definecolor{currentstroke}{rgb}{0.000000,0.000000,0.000000}%
\pgfsetstrokecolor{currentstroke}%
\pgfsetdash{}{0pt}%
\pgfpathmoveto{\pgfqpoint{3.729542in}{4.407621in}}%
\pgfpathlineto{\pgfqpoint{3.742910in}{4.383872in}}%
\pgfpathlineto{\pgfqpoint{3.756271in}{4.360425in}}%
\pgfpathlineto{\pgfqpoint{3.769627in}{4.337278in}}%
\pgfpathlineto{\pgfqpoint{3.782977in}{4.314427in}}%
\pgfpathlineto{\pgfqpoint{3.790627in}{4.347445in}}%
\pgfpathlineto{\pgfqpoint{3.798277in}{4.380988in}}%
\pgfpathlineto{\pgfqpoint{3.805925in}{4.415065in}}%
\pgfpathlineto{\pgfqpoint{3.813573in}{4.449687in}}%
\pgfpathlineto{\pgfqpoint{3.800216in}{4.473320in}}%
\pgfpathlineto{\pgfqpoint{3.786853in}{4.497252in}}%
\pgfpathlineto{\pgfqpoint{3.773485in}{4.521486in}}%
\pgfpathlineto{\pgfqpoint{3.760110in}{4.546023in}}%
\pgfpathlineto{\pgfqpoint{3.752470in}{4.510600in}}%
\pgfpathlineto{\pgfqpoint{3.744828in}{4.475732in}}%
\pgfpathlineto{\pgfqpoint{3.737186in}{4.441409in}}%
\pgfpathlineto{\pgfqpoint{3.729542in}{4.407621in}}%
\pgfpathclose%
\pgfusepath{fill}%
\end{pgfscope}%
\begin{pgfscope}%
\pgfpathrectangle{\pgfqpoint{1.150000in}{0.150000in}}{\pgfqpoint{5.700000in}{5.700000in}}%
\pgfusepath{clip}%
\pgfsetbuttcap%
\pgfsetroundjoin%
\definecolor{currentfill}{rgb}{0.221989,0.339161,0.548752}%
\pgfsetfillcolor{currentfill}%
\pgfsetfillopacity{0.800000}%
\pgfsetlinewidth{0.000000pt}%
\definecolor{currentstroke}{rgb}{0.000000,0.000000,0.000000}%
\pgfsetstrokecolor{currentstroke}%
\pgfsetdash{}{0pt}%
\pgfpathmoveto{\pgfqpoint{3.856335in}{3.134566in}}%
\pgfpathlineto{\pgfqpoint{3.869638in}{3.123114in}}%
\pgfpathlineto{\pgfqpoint{3.882943in}{3.111900in}}%
\pgfpathlineto{\pgfqpoint{3.896248in}{3.100921in}}%
\pgfpathlineto{\pgfqpoint{3.909556in}{3.090176in}}%
\pgfpathlineto{\pgfqpoint{3.917276in}{3.107388in}}%
\pgfpathlineto{\pgfqpoint{3.924992in}{3.124829in}}%
\pgfpathlineto{\pgfqpoint{3.932706in}{3.142504in}}%
\pgfpathlineto{\pgfqpoint{3.940416in}{3.160420in}}%
\pgfpathlineto{\pgfqpoint{3.927114in}{3.171637in}}%
\pgfpathlineto{\pgfqpoint{3.913813in}{3.183088in}}%
\pgfpathlineto{\pgfqpoint{3.900514in}{3.194775in}}%
\pgfpathlineto{\pgfqpoint{3.887215in}{3.206700in}}%
\pgfpathlineto{\pgfqpoint{3.879500in}{3.188300in}}%
\pgfpathlineto{\pgfqpoint{3.871782in}{3.170148in}}%
\pgfpathlineto{\pgfqpoint{3.864060in}{3.152239in}}%
\pgfpathlineto{\pgfqpoint{3.856335in}{3.134566in}}%
\pgfpathclose%
\pgfusepath{fill}%
\end{pgfscope}%
\begin{pgfscope}%
\pgfpathrectangle{\pgfqpoint{1.150000in}{0.150000in}}{\pgfqpoint{5.700000in}{5.700000in}}%
\pgfusepath{clip}%
\pgfsetbuttcap%
\pgfsetroundjoin%
\definecolor{currentfill}{rgb}{0.208623,0.367752,0.552675}%
\pgfsetfillcolor{currentfill}%
\pgfsetfillopacity{0.800000}%
\pgfsetlinewidth{0.000000pt}%
\definecolor{currentstroke}{rgb}{0.000000,0.000000,0.000000}%
\pgfsetstrokecolor{currentstroke}%
\pgfsetdash{}{0pt}%
\pgfpathmoveto{\pgfqpoint{3.665719in}{3.217338in}}%
\pgfpathlineto{\pgfqpoint{3.679028in}{3.203356in}}%
\pgfpathlineto{\pgfqpoint{3.692336in}{3.189632in}}%
\pgfpathlineto{\pgfqpoint{3.705643in}{3.176163in}}%
\pgfpathlineto{\pgfqpoint{3.718950in}{3.162948in}}%
\pgfpathlineto{\pgfqpoint{3.726700in}{3.180585in}}%
\pgfpathlineto{\pgfqpoint{3.734446in}{3.198455in}}%
\pgfpathlineto{\pgfqpoint{3.742188in}{3.216561in}}%
\pgfpathlineto{\pgfqpoint{3.749926in}{3.234910in}}%
\pgfpathlineto{\pgfqpoint{3.736624in}{3.248569in}}%
\pgfpathlineto{\pgfqpoint{3.723321in}{3.262482in}}%
\pgfpathlineto{\pgfqpoint{3.710017in}{3.276651in}}%
\pgfpathlineto{\pgfqpoint{3.696712in}{3.291077in}}%
\pgfpathlineto{\pgfqpoint{3.688970in}{3.272272in}}%
\pgfpathlineto{\pgfqpoint{3.681224in}{3.253717in}}%
\pgfpathlineto{\pgfqpoint{3.673473in}{3.235407in}}%
\pgfpathlineto{\pgfqpoint{3.665719in}{3.217338in}}%
\pgfpathclose%
\pgfusepath{fill}%
\end{pgfscope}%
\begin{pgfscope}%
\pgfpathrectangle{\pgfqpoint{1.150000in}{0.150000in}}{\pgfqpoint{5.700000in}{5.700000in}}%
\pgfusepath{clip}%
\pgfsetbuttcap%
\pgfsetroundjoin%
\definecolor{currentfill}{rgb}{0.179019,0.433756,0.557430}%
\pgfsetfillcolor{currentfill}%
\pgfsetfillopacity{0.800000}%
\pgfsetlinewidth{0.000000pt}%
\definecolor{currentstroke}{rgb}{0.000000,0.000000,0.000000}%
\pgfsetstrokecolor{currentstroke}%
\pgfsetdash{}{0pt}%
\pgfpathmoveto{\pgfqpoint{3.505840in}{3.405902in}}%
\pgfpathlineto{\pgfqpoint{3.519180in}{3.388676in}}%
\pgfpathlineto{\pgfqpoint{3.532516in}{3.371732in}}%
\pgfpathlineto{\pgfqpoint{3.545848in}{3.355069in}}%
\pgfpathlineto{\pgfqpoint{3.559177in}{3.338683in}}%
\pgfpathlineto{\pgfqpoint{3.566941in}{3.357621in}}%
\pgfpathlineto{\pgfqpoint{3.574699in}{3.376816in}}%
\pgfpathlineto{\pgfqpoint{3.582453in}{3.396273in}}%
\pgfpathlineto{\pgfqpoint{3.590202in}{3.415997in}}%
\pgfpathlineto{\pgfqpoint{3.576876in}{3.432832in}}%
\pgfpathlineto{\pgfqpoint{3.563547in}{3.449945in}}%
\pgfpathlineto{\pgfqpoint{3.550215in}{3.467339in}}%
\pgfpathlineto{\pgfqpoint{3.536878in}{3.485016in}}%
\pgfpathlineto{\pgfqpoint{3.529126in}{3.464830in}}%
\pgfpathlineto{\pgfqpoint{3.521369in}{3.444919in}}%
\pgfpathlineto{\pgfqpoint{3.513607in}{3.425278in}}%
\pgfpathlineto{\pgfqpoint{3.505840in}{3.405902in}}%
\pgfpathclose%
\pgfusepath{fill}%
\end{pgfscope}%
\begin{pgfscope}%
\pgfpathrectangle{\pgfqpoint{1.150000in}{0.150000in}}{\pgfqpoint{5.700000in}{5.700000in}}%
\pgfusepath{clip}%
\pgfsetbuttcap%
\pgfsetroundjoin%
\definecolor{currentfill}{rgb}{0.147607,0.511733,0.557049}%
\pgfsetfillcolor{currentfill}%
\pgfsetfillopacity{0.800000}%
\pgfsetlinewidth{0.000000pt}%
\definecolor{currentstroke}{rgb}{0.000000,0.000000,0.000000}%
\pgfsetstrokecolor{currentstroke}%
\pgfsetdash{}{0pt}%
\pgfpathmoveto{\pgfqpoint{3.430020in}{3.636915in}}%
\pgfpathlineto{\pgfqpoint{3.443396in}{3.616884in}}%
\pgfpathlineto{\pgfqpoint{3.456766in}{3.597157in}}%
\pgfpathlineto{\pgfqpoint{3.470130in}{3.577731in}}%
\pgfpathlineto{\pgfqpoint{3.483489in}{3.558602in}}%
\pgfpathlineto{\pgfqpoint{3.491240in}{3.579544in}}%
\pgfpathlineto{\pgfqpoint{3.498985in}{3.600781in}}%
\pgfpathlineto{\pgfqpoint{3.506726in}{3.622318in}}%
\pgfpathlineto{\pgfqpoint{3.514462in}{3.644163in}}%
\pgfpathlineto{\pgfqpoint{3.501105in}{3.663780in}}%
\pgfpathlineto{\pgfqpoint{3.487743in}{3.683696in}}%
\pgfpathlineto{\pgfqpoint{3.474375in}{3.703913in}}%
\pgfpathlineto{\pgfqpoint{3.461001in}{3.724434in}}%
\pgfpathlineto{\pgfqpoint{3.453263in}{3.702087in}}%
\pgfpathlineto{\pgfqpoint{3.445521in}{3.680055in}}%
\pgfpathlineto{\pgfqpoint{3.437773in}{3.658333in}}%
\pgfpathlineto{\pgfqpoint{3.430020in}{3.636915in}}%
\pgfpathclose%
\pgfusepath{fill}%
\end{pgfscope}%
\begin{pgfscope}%
\pgfpathrectangle{\pgfqpoint{1.150000in}{0.150000in}}{\pgfqpoint{5.700000in}{5.700000in}}%
\pgfusepath{clip}%
\pgfsetbuttcap%
\pgfsetroundjoin%
\definecolor{currentfill}{rgb}{0.196571,0.711827,0.479221}%
\pgfsetfillcolor{currentfill}%
\pgfsetfillopacity{0.800000}%
\pgfsetlinewidth{0.000000pt}%
\definecolor{currentstroke}{rgb}{0.000000,0.000000,0.000000}%
\pgfsetstrokecolor{currentstroke}%
\pgfsetdash{}{0pt}%
\pgfpathmoveto{\pgfqpoint{3.530620in}{4.224670in}}%
\pgfpathlineto{\pgfqpoint{3.544025in}{4.200562in}}%
\pgfpathlineto{\pgfqpoint{3.557423in}{4.176775in}}%
\pgfpathlineto{\pgfqpoint{3.570813in}{4.153306in}}%
\pgfpathlineto{\pgfqpoint{3.584197in}{4.130151in}}%
\pgfpathlineto{\pgfqpoint{3.591864in}{4.158871in}}%
\pgfpathlineto{\pgfqpoint{3.599527in}{4.188031in}}%
\pgfpathlineto{\pgfqpoint{3.607187in}{4.217640in}}%
\pgfpathlineto{\pgfqpoint{3.614844in}{4.247706in}}%
\pgfpathlineto{\pgfqpoint{3.601457in}{4.271537in}}%
\pgfpathlineto{\pgfqpoint{3.588062in}{4.295683in}}%
\pgfpathlineto{\pgfqpoint{3.574660in}{4.320149in}}%
\pgfpathlineto{\pgfqpoint{3.561251in}{4.344937in}}%
\pgfpathlineto{\pgfqpoint{3.553598in}{4.314178in}}%
\pgfpathlineto{\pgfqpoint{3.545943in}{4.283886in}}%
\pgfpathlineto{\pgfqpoint{3.538283in}{4.254053in}}%
\pgfpathlineto{\pgfqpoint{3.530620in}{4.224670in}}%
\pgfpathclose%
\pgfusepath{fill}%
\end{pgfscope}%
\begin{pgfscope}%
\pgfpathrectangle{\pgfqpoint{1.150000in}{0.150000in}}{\pgfqpoint{5.700000in}{5.700000in}}%
\pgfusepath{clip}%
\pgfsetbuttcap%
\pgfsetroundjoin%
\definecolor{currentfill}{rgb}{0.223925,0.334994,0.548053}%
\pgfsetfillcolor{currentfill}%
\pgfsetfillopacity{0.800000}%
\pgfsetlinewidth{0.000000pt}%
\definecolor{currentstroke}{rgb}{0.000000,0.000000,0.000000}%
\pgfsetstrokecolor{currentstroke}%
\pgfsetdash{}{0pt}%
\pgfpathmoveto{\pgfqpoint{3.993641in}{3.117869in}}%
\pgfpathlineto{\pgfqpoint{4.006953in}{3.107803in}}%
\pgfpathlineto{\pgfqpoint{4.020267in}{3.097963in}}%
\pgfpathlineto{\pgfqpoint{4.033583in}{3.088349in}}%
\pgfpathlineto{\pgfqpoint{4.046902in}{3.078957in}}%
\pgfpathlineto{\pgfqpoint{4.054599in}{3.096137in}}%
\pgfpathlineto{\pgfqpoint{4.062293in}{3.113552in}}%
\pgfpathlineto{\pgfqpoint{4.069984in}{3.131209in}}%
\pgfpathlineto{\pgfqpoint{4.077673in}{3.149114in}}%
\pgfpathlineto{\pgfqpoint{4.064359in}{3.159007in}}%
\pgfpathlineto{\pgfqpoint{4.051048in}{3.169124in}}%
\pgfpathlineto{\pgfqpoint{4.037739in}{3.179467in}}%
\pgfpathlineto{\pgfqpoint{4.024433in}{3.190035in}}%
\pgfpathlineto{\pgfqpoint{4.016739in}{3.171616in}}%
\pgfpathlineto{\pgfqpoint{4.009043in}{3.153453in}}%
\pgfpathlineto{\pgfqpoint{4.001344in}{3.135539in}}%
\pgfpathlineto{\pgfqpoint{3.993641in}{3.117869in}}%
\pgfpathclose%
\pgfusepath{fill}%
\end{pgfscope}%
\begin{pgfscope}%
\pgfpathrectangle{\pgfqpoint{1.150000in}{0.150000in}}{\pgfqpoint{5.700000in}{5.700000in}}%
\pgfusepath{clip}%
\pgfsetbuttcap%
\pgfsetroundjoin%
\definecolor{currentfill}{rgb}{0.203063,0.379716,0.553925}%
\pgfsetfillcolor{currentfill}%
\pgfsetfillopacity{0.800000}%
\pgfsetlinewidth{0.000000pt}%
\definecolor{currentstroke}{rgb}{0.000000,0.000000,0.000000}%
\pgfsetstrokecolor{currentstroke}%
\pgfsetdash{}{0pt}%
\pgfpathmoveto{\pgfqpoint{4.383002in}{3.229664in}}%
\pgfpathlineto{\pgfqpoint{4.396363in}{3.221710in}}%
\pgfpathlineto{\pgfqpoint{4.409729in}{3.213961in}}%
\pgfpathlineto{\pgfqpoint{4.423100in}{3.206415in}}%
\pgfpathlineto{\pgfqpoint{4.436477in}{3.199073in}}%
\pgfpathlineto{\pgfqpoint{4.444115in}{3.217701in}}%
\pgfpathlineto{\pgfqpoint{4.451753in}{3.236629in}}%
\pgfpathlineto{\pgfqpoint{4.459390in}{3.255866in}}%
\pgfpathlineto{\pgfqpoint{4.467028in}{3.275418in}}%
\pgfpathlineto{\pgfqpoint{4.453659in}{3.283387in}}%
\pgfpathlineto{\pgfqpoint{4.440295in}{3.291559in}}%
\pgfpathlineto{\pgfqpoint{4.426936in}{3.299935in}}%
\pgfpathlineto{\pgfqpoint{4.413582in}{3.308517in}}%
\pgfpathlineto{\pgfqpoint{4.405937in}{3.288325in}}%
\pgfpathlineto{\pgfqpoint{4.398292in}{3.268458in}}%
\pgfpathlineto{\pgfqpoint{4.390647in}{3.248906in}}%
\pgfpathlineto{\pgfqpoint{4.383002in}{3.229664in}}%
\pgfpathclose%
\pgfusepath{fill}%
\end{pgfscope}%
\begin{pgfscope}%
\pgfpathrectangle{\pgfqpoint{1.150000in}{0.150000in}}{\pgfqpoint{5.700000in}{5.700000in}}%
\pgfusepath{clip}%
\pgfsetbuttcap%
\pgfsetroundjoin%
\definecolor{currentfill}{rgb}{0.210503,0.363727,0.552206}%
\pgfsetfillcolor{currentfill}%
\pgfsetfillopacity{0.800000}%
\pgfsetlinewidth{0.000000pt}%
\definecolor{currentstroke}{rgb}{0.000000,0.000000,0.000000}%
\pgfsetstrokecolor{currentstroke}%
\pgfsetdash{}{0pt}%
\pgfpathmoveto{\pgfqpoint{4.298992in}{3.187137in}}%
\pgfpathlineto{\pgfqpoint{4.312341in}{3.178951in}}%
\pgfpathlineto{\pgfqpoint{4.325694in}{3.170973in}}%
\pgfpathlineto{\pgfqpoint{4.339053in}{3.163202in}}%
\pgfpathlineto{\pgfqpoint{4.352416in}{3.155638in}}%
\pgfpathlineto{\pgfqpoint{4.360064in}{3.173717in}}%
\pgfpathlineto{\pgfqpoint{4.367710in}{3.192076in}}%
\pgfpathlineto{\pgfqpoint{4.375357in}{3.210723in}}%
\pgfpathlineto{\pgfqpoint{4.383002in}{3.229664in}}%
\pgfpathlineto{\pgfqpoint{4.369646in}{3.237823in}}%
\pgfpathlineto{\pgfqpoint{4.356294in}{3.246189in}}%
\pgfpathlineto{\pgfqpoint{4.342947in}{3.254763in}}%
\pgfpathlineto{\pgfqpoint{4.329605in}{3.263545in}}%
\pgfpathlineto{\pgfqpoint{4.321953in}{3.243996in}}%
\pgfpathlineto{\pgfqpoint{4.314300in}{3.224750in}}%
\pgfpathlineto{\pgfqpoint{4.306646in}{3.205800in}}%
\pgfpathlineto{\pgfqpoint{4.298992in}{3.187137in}}%
\pgfpathclose%
\pgfusepath{fill}%
\end{pgfscope}%
\begin{pgfscope}%
\pgfpathrectangle{\pgfqpoint{1.150000in}{0.150000in}}{\pgfqpoint{5.700000in}{5.700000in}}%
\pgfusepath{clip}%
\pgfsetbuttcap%
\pgfsetroundjoin%
\definecolor{currentfill}{rgb}{0.168126,0.459988,0.558082}%
\pgfsetfillcolor{currentfill}%
\pgfsetfillopacity{0.800000}%
\pgfsetlinewidth{0.000000pt}%
\definecolor{currentstroke}{rgb}{0.000000,0.000000,0.000000}%
\pgfsetstrokecolor{currentstroke}%
\pgfsetdash{}{0pt}%
\pgfpathmoveto{\pgfqpoint{4.665743in}{3.464379in}}%
\pgfpathlineto{\pgfqpoint{4.679137in}{3.456128in}}%
\pgfpathlineto{\pgfqpoint{4.692538in}{3.448073in}}%
\pgfpathlineto{\pgfqpoint{4.705945in}{3.440213in}}%
\pgfpathlineto{\pgfqpoint{4.719358in}{3.432546in}}%
\pgfpathlineto{\pgfqpoint{4.726995in}{3.454693in}}%
\pgfpathlineto{\pgfqpoint{4.734636in}{3.477247in}}%
\pgfpathlineto{\pgfqpoint{4.742281in}{3.500215in}}%
\pgfpathlineto{\pgfqpoint{4.728875in}{3.508445in}}%
\pgfpathlineto{\pgfqpoint{4.715475in}{3.516869in}}%
\pgfpathlineto{\pgfqpoint{4.702080in}{3.525488in}}%
\pgfpathlineto{\pgfqpoint{4.688692in}{3.534303in}}%
\pgfpathlineto{\pgfqpoint{4.681038in}{3.510575in}}%
\pgfpathlineto{\pgfqpoint{4.673389in}{3.487270in}}%
\pgfpathlineto{\pgfqpoint{4.665743in}{3.464379in}}%
\pgfpathclose%
\pgfusepath{fill}%
\end{pgfscope}%
\begin{pgfscope}%
\pgfpathrectangle{\pgfqpoint{1.150000in}{0.150000in}}{\pgfqpoint{5.700000in}{5.700000in}}%
\pgfusepath{clip}%
\pgfsetbuttcap%
\pgfsetroundjoin%
\definecolor{currentfill}{rgb}{0.296479,0.761561,0.424223}%
\pgfsetfillcolor{currentfill}%
\pgfsetfillopacity{0.800000}%
\pgfsetlinewidth{0.000000pt}%
\definecolor{currentstroke}{rgb}{0.000000,0.000000,0.000000}%
\pgfsetstrokecolor{currentstroke}%
\pgfsetdash{}{0pt}%
\pgfpathmoveto{\pgfqpoint{3.645446in}{4.372703in}}%
\pgfpathlineto{\pgfqpoint{3.658832in}{4.348471in}}%
\pgfpathlineto{\pgfqpoint{3.672212in}{4.324551in}}%
\pgfpathlineto{\pgfqpoint{3.685585in}{4.300940in}}%
\pgfpathlineto{\pgfqpoint{3.698953in}{4.277635in}}%
\pgfpathlineto{\pgfqpoint{3.706603in}{4.309375in}}%
\pgfpathlineto{\pgfqpoint{3.714251in}{4.341613in}}%
\pgfpathlineto{\pgfqpoint{3.721897in}{4.374359in}}%
\pgfpathlineto{\pgfqpoint{3.729542in}{4.407621in}}%
\pgfpathlineto{\pgfqpoint{3.716169in}{4.431675in}}%
\pgfpathlineto{\pgfqpoint{3.702789in}{4.456036in}}%
\pgfpathlineto{\pgfqpoint{3.689403in}{4.480708in}}%
\pgfpathlineto{\pgfqpoint{3.676010in}{4.505694in}}%
\pgfpathlineto{\pgfqpoint{3.668372in}{4.471665in}}%
\pgfpathlineto{\pgfqpoint{3.660732in}{4.438163in}}%
\pgfpathlineto{\pgfqpoint{3.653090in}{4.405179in}}%
\pgfpathlineto{\pgfqpoint{3.645446in}{4.372703in}}%
\pgfpathclose%
\pgfusepath{fill}%
\end{pgfscope}%
\begin{pgfscope}%
\pgfpathrectangle{\pgfqpoint{1.150000in}{0.150000in}}{\pgfqpoint{5.700000in}{5.700000in}}%
\pgfusepath{clip}%
\pgfsetbuttcap%
\pgfsetroundjoin%
\definecolor{currentfill}{rgb}{0.195860,0.395433,0.555276}%
\pgfsetfillcolor{currentfill}%
\pgfsetfillopacity{0.800000}%
\pgfsetlinewidth{0.000000pt}%
\definecolor{currentstroke}{rgb}{0.000000,0.000000,0.000000}%
\pgfsetstrokecolor{currentstroke}%
\pgfsetdash{}{0pt}%
\pgfpathmoveto{\pgfqpoint{4.467028in}{3.275418in}}%
\pgfpathlineto{\pgfqpoint{4.480402in}{3.267651in}}%
\pgfpathlineto{\pgfqpoint{4.493782in}{3.260085in}}%
\pgfpathlineto{\pgfqpoint{4.507167in}{3.252720in}}%
\pgfpathlineto{\pgfqpoint{4.520558in}{3.245554in}}%
\pgfpathlineto{\pgfqpoint{4.528189in}{3.264783in}}%
\pgfpathlineto{\pgfqpoint{4.535820in}{3.284334in}}%
\pgfpathlineto{\pgfqpoint{4.543452in}{3.304216in}}%
\pgfpathlineto{\pgfqpoint{4.551085in}{3.324436in}}%
\pgfpathlineto{\pgfqpoint{4.537703in}{3.332260in}}%
\pgfpathlineto{\pgfqpoint{4.524325in}{3.340284in}}%
\pgfpathlineto{\pgfqpoint{4.510953in}{3.348509in}}%
\pgfpathlineto{\pgfqpoint{4.497586in}{3.356935in}}%
\pgfpathlineto{\pgfqpoint{4.489945in}{3.336044in}}%
\pgfpathlineto{\pgfqpoint{4.482306in}{3.315499in}}%
\pgfpathlineto{\pgfqpoint{4.474667in}{3.295293in}}%
\pgfpathlineto{\pgfqpoint{4.467028in}{3.275418in}}%
\pgfpathclose%
\pgfusepath{fill}%
\end{pgfscope}%
\begin{pgfscope}%
\pgfpathrectangle{\pgfqpoint{1.150000in}{0.150000in}}{\pgfqpoint{5.700000in}{5.700000in}}%
\pgfusepath{clip}%
\pgfsetbuttcap%
\pgfsetroundjoin%
\definecolor{currentfill}{rgb}{0.126453,0.570633,0.549841}%
\pgfsetfillcolor{currentfill}%
\pgfsetfillopacity{0.800000}%
\pgfsetlinewidth{0.000000pt}%
\definecolor{currentstroke}{rgb}{0.000000,0.000000,0.000000}%
\pgfsetstrokecolor{currentstroke}%
\pgfsetdash{}{0pt}%
\pgfpathmoveto{\pgfqpoint{3.407443in}{3.809617in}}%
\pgfpathlineto{\pgfqpoint{3.420842in}{3.787851in}}%
\pgfpathlineto{\pgfqpoint{3.434235in}{3.766400in}}%
\pgfpathlineto{\pgfqpoint{3.447621in}{3.745263in}}%
\pgfpathlineto{\pgfqpoint{3.461001in}{3.724434in}}%
\pgfpathlineto{\pgfqpoint{3.468734in}{3.747104in}}%
\pgfpathlineto{\pgfqpoint{3.476462in}{3.770101in}}%
\pgfpathlineto{\pgfqpoint{3.484185in}{3.793432in}}%
\pgfpathlineto{\pgfqpoint{3.491904in}{3.817103in}}%
\pgfpathlineto{\pgfqpoint{3.478525in}{3.838458in}}%
\pgfpathlineto{\pgfqpoint{3.465139in}{3.860124in}}%
\pgfpathlineto{\pgfqpoint{3.451747in}{3.882104in}}%
\pgfpathlineto{\pgfqpoint{3.438348in}{3.904400in}}%
\pgfpathlineto{\pgfqpoint{3.430629in}{3.880186in}}%
\pgfpathlineto{\pgfqpoint{3.422905in}{3.856322in}}%
\pgfpathlineto{\pgfqpoint{3.415177in}{3.832801in}}%
\pgfpathlineto{\pgfqpoint{3.407443in}{3.809617in}}%
\pgfpathclose%
\pgfusepath{fill}%
\end{pgfscope}%
\begin{pgfscope}%
\pgfpathrectangle{\pgfqpoint{1.150000in}{0.150000in}}{\pgfqpoint{5.700000in}{5.700000in}}%
\pgfusepath{clip}%
\pgfsetbuttcap%
\pgfsetroundjoin%
\definecolor{currentfill}{rgb}{0.218130,0.347432,0.550038}%
\pgfsetfillcolor{currentfill}%
\pgfsetfillopacity{0.800000}%
\pgfsetlinewidth{0.000000pt}%
\definecolor{currentstroke}{rgb}{0.000000,0.000000,0.000000}%
\pgfsetstrokecolor{currentstroke}%
\pgfsetdash{}{0pt}%
\pgfpathmoveto{\pgfqpoint{4.214982in}{3.147828in}}%
\pgfpathlineto{\pgfqpoint{4.228320in}{3.139362in}}%
\pgfpathlineto{\pgfqpoint{4.241663in}{3.131109in}}%
\pgfpathlineto{\pgfqpoint{4.255009in}{3.123067in}}%
\pgfpathlineto{\pgfqpoint{4.268361in}{3.115235in}}%
\pgfpathlineto{\pgfqpoint{4.276021in}{3.132812in}}%
\pgfpathlineto{\pgfqpoint{4.283679in}{3.150650in}}%
\pgfpathlineto{\pgfqpoint{4.291336in}{3.168756in}}%
\pgfpathlineto{\pgfqpoint{4.298992in}{3.187137in}}%
\pgfpathlineto{\pgfqpoint{4.285647in}{3.195533in}}%
\pgfpathlineto{\pgfqpoint{4.272307in}{3.204139in}}%
\pgfpathlineto{\pgfqpoint{4.258971in}{3.212957in}}%
\pgfpathlineto{\pgfqpoint{4.245638in}{3.221987in}}%
\pgfpathlineto{\pgfqpoint{4.237976in}{3.203030in}}%
\pgfpathlineto{\pgfqpoint{4.230313in}{3.184356in}}%
\pgfpathlineto{\pgfqpoint{4.222648in}{3.165957in}}%
\pgfpathlineto{\pgfqpoint{4.214982in}{3.147828in}}%
\pgfpathclose%
\pgfusepath{fill}%
\end{pgfscope}%
\begin{pgfscope}%
\pgfpathrectangle{\pgfqpoint{1.150000in}{0.150000in}}{\pgfqpoint{5.700000in}{5.700000in}}%
\pgfusepath{clip}%
\pgfsetbuttcap%
\pgfsetroundjoin%
\definecolor{currentfill}{rgb}{0.216210,0.351535,0.550627}%
\pgfsetfillcolor{currentfill}%
\pgfsetfillopacity{0.800000}%
\pgfsetlinewidth{0.000000pt}%
\definecolor{currentstroke}{rgb}{0.000000,0.000000,0.000000}%
\pgfsetstrokecolor{currentstroke}%
\pgfsetdash{}{0pt}%
\pgfpathmoveto{\pgfqpoint{3.718950in}{3.162948in}}%
\pgfpathlineto{\pgfqpoint{3.732256in}{3.149984in}}%
\pgfpathlineto{\pgfqpoint{3.745561in}{3.137270in}}%
\pgfpathlineto{\pgfqpoint{3.758866in}{3.124804in}}%
\pgfpathlineto{\pgfqpoint{3.772172in}{3.112585in}}%
\pgfpathlineto{\pgfqpoint{3.779917in}{3.129792in}}%
\pgfpathlineto{\pgfqpoint{3.787658in}{3.147223in}}%
\pgfpathlineto{\pgfqpoint{3.795396in}{3.164883in}}%
\pgfpathlineto{\pgfqpoint{3.803129in}{3.182777in}}%
\pgfpathlineto{\pgfqpoint{3.789829in}{3.195438in}}%
\pgfpathlineto{\pgfqpoint{3.776528in}{3.208347in}}%
\pgfpathlineto{\pgfqpoint{3.763227in}{3.221503in}}%
\pgfpathlineto{\pgfqpoint{3.749926in}{3.234910in}}%
\pgfpathlineto{\pgfqpoint{3.742188in}{3.216561in}}%
\pgfpathlineto{\pgfqpoint{3.734446in}{3.198455in}}%
\pgfpathlineto{\pgfqpoint{3.726700in}{3.180585in}}%
\pgfpathlineto{\pgfqpoint{3.718950in}{3.162948in}}%
\pgfpathclose%
\pgfusepath{fill}%
\end{pgfscope}%
\begin{pgfscope}%
\pgfpathrectangle{\pgfqpoint{1.150000in}{0.150000in}}{\pgfqpoint{5.700000in}{5.700000in}}%
\pgfusepath{clip}%
\pgfsetbuttcap%
\pgfsetroundjoin%
\definecolor{currentfill}{rgb}{0.168126,0.459988,0.558082}%
\pgfsetfillcolor{currentfill}%
\pgfsetfillopacity{0.800000}%
\pgfsetlinewidth{0.000000pt}%
\definecolor{currentstroke}{rgb}{0.000000,0.000000,0.000000}%
\pgfsetstrokecolor{currentstroke}%
\pgfsetdash{}{0pt}%
\pgfpathmoveto{\pgfqpoint{3.452438in}{3.477678in}}%
\pgfpathlineto{\pgfqpoint{3.465795in}{3.459299in}}%
\pgfpathlineto{\pgfqpoint{3.479148in}{3.441211in}}%
\pgfpathlineto{\pgfqpoint{3.492496in}{3.423413in}}%
\pgfpathlineto{\pgfqpoint{3.505840in}{3.405902in}}%
\pgfpathlineto{\pgfqpoint{3.513607in}{3.425278in}}%
\pgfpathlineto{\pgfqpoint{3.521369in}{3.444919in}}%
\pgfpathlineto{\pgfqpoint{3.529126in}{3.464830in}}%
\pgfpathlineto{\pgfqpoint{3.536878in}{3.485016in}}%
\pgfpathlineto{\pgfqpoint{3.523538in}{3.502979in}}%
\pgfpathlineto{\pgfqpoint{3.510193in}{3.521229in}}%
\pgfpathlineto{\pgfqpoint{3.496844in}{3.539769in}}%
\pgfpathlineto{\pgfqpoint{3.483489in}{3.558602in}}%
\pgfpathlineto{\pgfqpoint{3.475734in}{3.537951in}}%
\pgfpathlineto{\pgfqpoint{3.467974in}{3.517583in}}%
\pgfpathlineto{\pgfqpoint{3.460208in}{3.497494in}}%
\pgfpathlineto{\pgfqpoint{3.452438in}{3.477678in}}%
\pgfpathclose%
\pgfusepath{fill}%
\end{pgfscope}%
\begin{pgfscope}%
\pgfpathrectangle{\pgfqpoint{1.150000in}{0.150000in}}{\pgfqpoint{5.700000in}{5.700000in}}%
\pgfusepath{clip}%
\pgfsetbuttcap%
\pgfsetroundjoin%
\definecolor{currentfill}{rgb}{0.187231,0.414746,0.556547}%
\pgfsetfillcolor{currentfill}%
\pgfsetfillopacity{0.800000}%
\pgfsetlinewidth{0.000000pt}%
\definecolor{currentstroke}{rgb}{0.000000,0.000000,0.000000}%
\pgfsetstrokecolor{currentstroke}%
\pgfsetdash{}{0pt}%
\pgfpathmoveto{\pgfqpoint{4.551085in}{3.324436in}}%
\pgfpathlineto{\pgfqpoint{4.564474in}{3.316811in}}%
\pgfpathlineto{\pgfqpoint{4.577868in}{3.309384in}}%
\pgfpathlineto{\pgfqpoint{4.591268in}{3.302154in}}%
\pgfpathlineto{\pgfqpoint{4.604674in}{3.295120in}}%
\pgfpathlineto{\pgfqpoint{4.612300in}{3.315008in}}%
\pgfpathlineto{\pgfqpoint{4.619928in}{3.335243in}}%
\pgfpathlineto{\pgfqpoint{4.627558in}{3.355831in}}%
\pgfpathlineto{\pgfqpoint{4.635190in}{3.376782in}}%
\pgfpathlineto{\pgfqpoint{4.621792in}{3.384506in}}%
\pgfpathlineto{\pgfqpoint{4.608401in}{3.392426in}}%
\pgfpathlineto{\pgfqpoint{4.595015in}{3.400544in}}%
\pgfpathlineto{\pgfqpoint{4.581634in}{3.408860in}}%
\pgfpathlineto{\pgfqpoint{4.573994in}{3.387206in}}%
\pgfpathlineto{\pgfqpoint{4.566356in}{3.365923in}}%
\pgfpathlineto{\pgfqpoint{4.558720in}{3.345002in}}%
\pgfpathlineto{\pgfqpoint{4.551085in}{3.324436in}}%
\pgfpathclose%
\pgfusepath{fill}%
\end{pgfscope}%
\begin{pgfscope}%
\pgfpathrectangle{\pgfqpoint{1.150000in}{0.150000in}}{\pgfqpoint{5.700000in}{5.700000in}}%
\pgfusepath{clip}%
\pgfsetbuttcap%
\pgfsetroundjoin%
\definecolor{currentfill}{rgb}{0.223925,0.334994,0.548053}%
\pgfsetfillcolor{currentfill}%
\pgfsetfillopacity{0.800000}%
\pgfsetlinewidth{0.000000pt}%
\definecolor{currentstroke}{rgb}{0.000000,0.000000,0.000000}%
\pgfsetstrokecolor{currentstroke}%
\pgfsetdash{}{0pt}%
\pgfpathmoveto{\pgfqpoint{4.130957in}{3.111754in}}%
\pgfpathlineto{\pgfqpoint{4.144287in}{3.102961in}}%
\pgfpathlineto{\pgfqpoint{4.157619in}{3.094384in}}%
\pgfpathlineto{\pgfqpoint{4.170956in}{3.086024in}}%
\pgfpathlineto{\pgfqpoint{4.184297in}{3.077877in}}%
\pgfpathlineto{\pgfqpoint{4.191971in}{3.094993in}}%
\pgfpathlineto{\pgfqpoint{4.199643in}{3.112352in}}%
\pgfpathlineto{\pgfqpoint{4.207314in}{3.129962in}}%
\pgfpathlineto{\pgfqpoint{4.214982in}{3.147828in}}%
\pgfpathlineto{\pgfqpoint{4.201647in}{3.156507in}}%
\pgfpathlineto{\pgfqpoint{4.188317in}{3.165401in}}%
\pgfpathlineto{\pgfqpoint{4.174990in}{3.174510in}}%
\pgfpathlineto{\pgfqpoint{4.161666in}{3.183836in}}%
\pgfpathlineto{\pgfqpoint{4.153992in}{3.165425in}}%
\pgfpathlineto{\pgfqpoint{4.146316in}{3.147279in}}%
\pgfpathlineto{\pgfqpoint{4.138638in}{3.129390in}}%
\pgfpathlineto{\pgfqpoint{4.130957in}{3.111754in}}%
\pgfpathclose%
\pgfusepath{fill}%
\end{pgfscope}%
\begin{pgfscope}%
\pgfpathrectangle{\pgfqpoint{1.150000in}{0.150000in}}{\pgfqpoint{5.700000in}{5.700000in}}%
\pgfusepath{clip}%
\pgfsetbuttcap%
\pgfsetroundjoin%
\definecolor{currentfill}{rgb}{0.229739,0.322361,0.545706}%
\pgfsetfillcolor{currentfill}%
\pgfsetfillopacity{0.800000}%
\pgfsetlinewidth{0.000000pt}%
\definecolor{currentstroke}{rgb}{0.000000,0.000000,0.000000}%
\pgfsetstrokecolor{currentstroke}%
\pgfsetdash{}{0pt}%
\pgfpathmoveto{\pgfqpoint{3.909556in}{3.090176in}}%
\pgfpathlineto{\pgfqpoint{3.922864in}{3.079664in}}%
\pgfpathlineto{\pgfqpoint{3.936175in}{3.069383in}}%
\pgfpathlineto{\pgfqpoint{3.949487in}{3.059333in}}%
\pgfpathlineto{\pgfqpoint{3.962802in}{3.049511in}}%
\pgfpathlineto{\pgfqpoint{3.970517in}{3.066263in}}%
\pgfpathlineto{\pgfqpoint{3.978228in}{3.083236in}}%
\pgfpathlineto{\pgfqpoint{3.985936in}{3.100436in}}%
\pgfpathlineto{\pgfqpoint{3.993641in}{3.117869in}}%
\pgfpathlineto{\pgfqpoint{3.980332in}{3.128162in}}%
\pgfpathlineto{\pgfqpoint{3.967025in}{3.138684in}}%
\pgfpathlineto{\pgfqpoint{3.953720in}{3.149436in}}%
\pgfpathlineto{\pgfqpoint{3.940416in}{3.160420in}}%
\pgfpathlineto{\pgfqpoint{3.932706in}{3.142504in}}%
\pgfpathlineto{\pgfqpoint{3.924992in}{3.124829in}}%
\pgfpathlineto{\pgfqpoint{3.917276in}{3.107388in}}%
\pgfpathlineto{\pgfqpoint{3.909556in}{3.090176in}}%
\pgfpathclose%
\pgfusepath{fill}%
\end{pgfscope}%
\begin{pgfscope}%
\pgfpathrectangle{\pgfqpoint{1.150000in}{0.150000in}}{\pgfqpoint{5.700000in}{5.700000in}}%
\pgfusepath{clip}%
\pgfsetbuttcap%
\pgfsetroundjoin%
\definecolor{currentfill}{rgb}{0.274149,0.751988,0.436601}%
\pgfsetfillcolor{currentfill}%
\pgfsetfillopacity{0.800000}%
\pgfsetlinewidth{0.000000pt}%
\definecolor{currentstroke}{rgb}{0.000000,0.000000,0.000000}%
\pgfsetstrokecolor{currentstroke}%
\pgfsetdash{}{0pt}%
\pgfpathmoveto{\pgfqpoint{3.561251in}{4.344937in}}%
\pgfpathlineto{\pgfqpoint{3.574660in}{4.320149in}}%
\pgfpathlineto{\pgfqpoint{3.588062in}{4.295683in}}%
\pgfpathlineto{\pgfqpoint{3.601457in}{4.271537in}}%
\pgfpathlineto{\pgfqpoint{3.614844in}{4.247706in}}%
\pgfpathlineto{\pgfqpoint{3.622499in}{4.278236in}}%
\pgfpathlineto{\pgfqpoint{3.630150in}{4.309240in}}%
\pgfpathlineto{\pgfqpoint{3.637799in}{4.340726in}}%
\pgfpathlineto{\pgfqpoint{3.645446in}{4.372703in}}%
\pgfpathlineto{\pgfqpoint{3.632053in}{4.397249in}}%
\pgfpathlineto{\pgfqpoint{3.618653in}{4.422112in}}%
\pgfpathlineto{\pgfqpoint{3.605245in}{4.447296in}}%
\pgfpathlineto{\pgfqpoint{3.591830in}{4.472804in}}%
\pgfpathlineto{\pgfqpoint{3.584190in}{4.440095in}}%
\pgfpathlineto{\pgfqpoint{3.576546in}{4.407886in}}%
\pgfpathlineto{\pgfqpoint{3.568900in}{4.376170in}}%
\pgfpathlineto{\pgfqpoint{3.561251in}{4.344937in}}%
\pgfpathclose%
\pgfusepath{fill}%
\end{pgfscope}%
\begin{pgfscope}%
\pgfpathrectangle{\pgfqpoint{1.150000in}{0.150000in}}{\pgfqpoint{5.700000in}{5.700000in}}%
\pgfusepath{clip}%
\pgfsetbuttcap%
\pgfsetroundjoin%
\definecolor{currentfill}{rgb}{0.179019,0.433756,0.557430}%
\pgfsetfillcolor{currentfill}%
\pgfsetfillopacity{0.800000}%
\pgfsetlinewidth{0.000000pt}%
\definecolor{currentstroke}{rgb}{0.000000,0.000000,0.000000}%
\pgfsetstrokecolor{currentstroke}%
\pgfsetdash{}{0pt}%
\pgfpathmoveto{\pgfqpoint{4.635190in}{3.376782in}}%
\pgfpathlineto{\pgfqpoint{4.648593in}{3.369254in}}%
\pgfpathlineto{\pgfqpoint{4.662003in}{3.361922in}}%
\pgfpathlineto{\pgfqpoint{4.675418in}{3.354783in}}%
\pgfpathlineto{\pgfqpoint{4.688840in}{3.347838in}}%
\pgfpathlineto{\pgfqpoint{4.696465in}{3.368450in}}%
\pgfpathlineto{\pgfqpoint{4.704093in}{3.389433in}}%
\pgfpathlineto{\pgfqpoint{4.711724in}{3.410795in}}%
\pgfpathlineto{\pgfqpoint{4.719358in}{3.432546in}}%
\pgfpathlineto{\pgfqpoint{4.705945in}{3.440213in}}%
\pgfpathlineto{\pgfqpoint{4.692538in}{3.448073in}}%
\pgfpathlineto{\pgfqpoint{4.679137in}{3.456128in}}%
\pgfpathlineto{\pgfqpoint{4.665743in}{3.464379in}}%
\pgfpathlineto{\pgfqpoint{4.658100in}{3.441894in}}%
\pgfpathlineto{\pgfqpoint{4.650460in}{3.419805in}}%
\pgfpathlineto{\pgfqpoint{4.642824in}{3.398104in}}%
\pgfpathlineto{\pgfqpoint{4.635190in}{3.376782in}}%
\pgfpathclose%
\pgfusepath{fill}%
\end{pgfscope}%
\begin{pgfscope}%
\pgfpathrectangle{\pgfqpoint{1.150000in}{0.150000in}}{\pgfqpoint{5.700000in}{5.700000in}}%
\pgfusepath{clip}%
\pgfsetbuttcap%
\pgfsetroundjoin%
\definecolor{currentfill}{rgb}{0.135066,0.544853,0.554029}%
\pgfsetfillcolor{currentfill}%
\pgfsetfillopacity{0.800000}%
\pgfsetlinewidth{0.000000pt}%
\definecolor{currentstroke}{rgb}{0.000000,0.000000,0.000000}%
\pgfsetstrokecolor{currentstroke}%
\pgfsetdash{}{0pt}%
\pgfpathmoveto{\pgfqpoint{3.376455in}{3.720125in}}%
\pgfpathlineto{\pgfqpoint{3.389856in}{3.698853in}}%
\pgfpathlineto{\pgfqpoint{3.403251in}{3.677896in}}%
\pgfpathlineto{\pgfqpoint{3.416639in}{3.657251in}}%
\pgfpathlineto{\pgfqpoint{3.430020in}{3.636915in}}%
\pgfpathlineto{\pgfqpoint{3.437773in}{3.658333in}}%
\pgfpathlineto{\pgfqpoint{3.445521in}{3.680055in}}%
\pgfpathlineto{\pgfqpoint{3.453263in}{3.702087in}}%
\pgfpathlineto{\pgfqpoint{3.461001in}{3.724434in}}%
\pgfpathlineto{\pgfqpoint{3.447621in}{3.745263in}}%
\pgfpathlineto{\pgfqpoint{3.434235in}{3.766400in}}%
\pgfpathlineto{\pgfqpoint{3.420842in}{3.787851in}}%
\pgfpathlineto{\pgfqpoint{3.407443in}{3.809617in}}%
\pgfpathlineto{\pgfqpoint{3.399704in}{3.786763in}}%
\pgfpathlineto{\pgfqpoint{3.391959in}{3.764234in}}%
\pgfpathlineto{\pgfqpoint{3.384210in}{3.742023in}}%
\pgfpathlineto{\pgfqpoint{3.376455in}{3.720125in}}%
\pgfpathclose%
\pgfusepath{fill}%
\end{pgfscope}%
\begin{pgfscope}%
\pgfpathrectangle{\pgfqpoint{1.150000in}{0.150000in}}{\pgfqpoint{5.700000in}{5.700000in}}%
\pgfusepath{clip}%
\pgfsetbuttcap%
\pgfsetroundjoin%
\definecolor{currentfill}{rgb}{0.225863,0.330805,0.547314}%
\pgfsetfillcolor{currentfill}%
\pgfsetfillopacity{0.800000}%
\pgfsetlinewidth{0.000000pt}%
\definecolor{currentstroke}{rgb}{0.000000,0.000000,0.000000}%
\pgfsetstrokecolor{currentstroke}%
\pgfsetdash{}{0pt}%
\pgfpathmoveto{\pgfqpoint{3.772172in}{3.112585in}}%
\pgfpathlineto{\pgfqpoint{3.785477in}{3.100611in}}%
\pgfpathlineto{\pgfqpoint{3.798784in}{3.088879in}}%
\pgfpathlineto{\pgfqpoint{3.812090in}{3.077390in}}%
\pgfpathlineto{\pgfqpoint{3.825398in}{3.066140in}}%
\pgfpathlineto{\pgfqpoint{3.833138in}{3.082918in}}%
\pgfpathlineto{\pgfqpoint{3.840874in}{3.099911in}}%
\pgfpathlineto{\pgfqpoint{3.848606in}{3.117126in}}%
\pgfpathlineto{\pgfqpoint{3.856335in}{3.134566in}}%
\pgfpathlineto{\pgfqpoint{3.843033in}{3.146257in}}%
\pgfpathlineto{\pgfqpoint{3.829731in}{3.158188in}}%
\pgfpathlineto{\pgfqpoint{3.816430in}{3.170361in}}%
\pgfpathlineto{\pgfqpoint{3.803129in}{3.182777in}}%
\pgfpathlineto{\pgfqpoint{3.795396in}{3.164883in}}%
\pgfpathlineto{\pgfqpoint{3.787658in}{3.147223in}}%
\pgfpathlineto{\pgfqpoint{3.779917in}{3.129792in}}%
\pgfpathlineto{\pgfqpoint{3.772172in}{3.112585in}}%
\pgfpathclose%
\pgfusepath{fill}%
\end{pgfscope}%
\begin{pgfscope}%
\pgfpathrectangle{\pgfqpoint{1.150000in}{0.150000in}}{\pgfqpoint{5.700000in}{5.700000in}}%
\pgfusepath{clip}%
\pgfsetbuttcap%
\pgfsetroundjoin%
\definecolor{currentfill}{rgb}{0.146616,0.673050,0.508936}%
\pgfsetfillcolor{currentfill}%
\pgfsetfillopacity{0.800000}%
\pgfsetlinewidth{0.000000pt}%
\definecolor{currentstroke}{rgb}{0.000000,0.000000,0.000000}%
\pgfsetstrokecolor{currentstroke}%
\pgfsetdash{}{0pt}%
\pgfpathmoveto{\pgfqpoint{3.415501in}{4.099565in}}%
\pgfpathlineto{\pgfqpoint{3.428931in}{4.075401in}}%
\pgfpathlineto{\pgfqpoint{3.442354in}{4.051567in}}%
\pgfpathlineto{\pgfqpoint{3.455768in}{4.028060in}}%
\pgfpathlineto{\pgfqpoint{3.469176in}{4.004877in}}%
\pgfpathlineto{\pgfqpoint{3.476871in}{4.030936in}}%
\pgfpathlineto{\pgfqpoint{3.484562in}{4.057385in}}%
\pgfpathlineto{\pgfqpoint{3.492249in}{4.084231in}}%
\pgfpathlineto{\pgfqpoint{3.499931in}{4.111480in}}%
\pgfpathlineto{\pgfqpoint{3.486522in}{4.135268in}}%
\pgfpathlineto{\pgfqpoint{3.473105in}{4.159381in}}%
\pgfpathlineto{\pgfqpoint{3.459681in}{4.183822in}}%
\pgfpathlineto{\pgfqpoint{3.446248in}{4.208595in}}%
\pgfpathlineto{\pgfqpoint{3.438568in}{4.180724in}}%
\pgfpathlineto{\pgfqpoint{3.430884in}{4.153267in}}%
\pgfpathlineto{\pgfqpoint{3.423195in}{4.126217in}}%
\pgfpathlineto{\pgfqpoint{3.415501in}{4.099565in}}%
\pgfpathclose%
\pgfusepath{fill}%
\end{pgfscope}%
\begin{pgfscope}%
\pgfpathrectangle{\pgfqpoint{1.150000in}{0.150000in}}{\pgfqpoint{5.700000in}{5.700000in}}%
\pgfusepath{clip}%
\pgfsetbuttcap%
\pgfsetroundjoin%
\definecolor{currentfill}{rgb}{0.157729,0.485932,0.558013}%
\pgfsetfillcolor{currentfill}%
\pgfsetfillopacity{0.800000}%
\pgfsetlinewidth{0.000000pt}%
\definecolor{currentstroke}{rgb}{0.000000,0.000000,0.000000}%
\pgfsetstrokecolor{currentstroke}%
\pgfsetdash{}{0pt}%
\pgfpathmoveto{\pgfqpoint{3.398957in}{3.554170in}}%
\pgfpathlineto{\pgfqpoint{3.412335in}{3.534596in}}%
\pgfpathlineto{\pgfqpoint{3.425708in}{3.515325in}}%
\pgfpathlineto{\pgfqpoint{3.439076in}{3.496353in}}%
\pgfpathlineto{\pgfqpoint{3.452438in}{3.477678in}}%
\pgfpathlineto{\pgfqpoint{3.460208in}{3.497494in}}%
\pgfpathlineto{\pgfqpoint{3.467974in}{3.517583in}}%
\pgfpathlineto{\pgfqpoint{3.475734in}{3.537951in}}%
\pgfpathlineto{\pgfqpoint{3.483489in}{3.558602in}}%
\pgfpathlineto{\pgfqpoint{3.470130in}{3.577731in}}%
\pgfpathlineto{\pgfqpoint{3.456766in}{3.597157in}}%
\pgfpathlineto{\pgfqpoint{3.443396in}{3.616884in}}%
\pgfpathlineto{\pgfqpoint{3.430020in}{3.636915in}}%
\pgfpathlineto{\pgfqpoint{3.422262in}{3.615795in}}%
\pgfpathlineto{\pgfqpoint{3.414499in}{3.594968in}}%
\pgfpathlineto{\pgfqpoint{3.406731in}{3.574428in}}%
\pgfpathlineto{\pgfqpoint{3.398957in}{3.554170in}}%
\pgfpathclose%
\pgfusepath{fill}%
\end{pgfscope}%
\begin{pgfscope}%
\pgfpathrectangle{\pgfqpoint{1.150000in}{0.150000in}}{\pgfqpoint{5.700000in}{5.700000in}}%
\pgfusepath{clip}%
\pgfsetbuttcap%
\pgfsetroundjoin%
\definecolor{currentfill}{rgb}{0.123444,0.636809,0.528763}%
\pgfsetfillcolor{currentfill}%
\pgfsetfillopacity{0.800000}%
\pgfsetlinewidth{0.000000pt}%
\definecolor{currentstroke}{rgb}{0.000000,0.000000,0.000000}%
\pgfsetstrokecolor{currentstroke}%
\pgfsetdash{}{0pt}%
\pgfpathmoveto{\pgfqpoint{3.384677in}{3.996810in}}%
\pgfpathlineto{\pgfqpoint{3.398106in}{3.973217in}}%
\pgfpathlineto{\pgfqpoint{3.411528in}{3.949954in}}%
\pgfpathlineto{\pgfqpoint{3.424942in}{3.927015in}}%
\pgfpathlineto{\pgfqpoint{3.438348in}{3.904400in}}%
\pgfpathlineto{\pgfqpoint{3.446062in}{3.928969in}}%
\pgfpathlineto{\pgfqpoint{3.453771in}{3.953900in}}%
\pgfpathlineto{\pgfqpoint{3.461476in}{3.979201in}}%
\pgfpathlineto{\pgfqpoint{3.469176in}{4.004877in}}%
\pgfpathlineto{\pgfqpoint{3.455768in}{4.028060in}}%
\pgfpathlineto{\pgfqpoint{3.442354in}{4.051567in}}%
\pgfpathlineto{\pgfqpoint{3.428931in}{4.075401in}}%
\pgfpathlineto{\pgfqpoint{3.415501in}{4.099565in}}%
\pgfpathlineto{\pgfqpoint{3.407802in}{4.073305in}}%
\pgfpathlineto{\pgfqpoint{3.400099in}{4.047431in}}%
\pgfpathlineto{\pgfqpoint{3.392390in}{4.021935in}}%
\pgfpathlineto{\pgfqpoint{3.384677in}{3.996810in}}%
\pgfpathclose%
\pgfusepath{fill}%
\end{pgfscope}%
\begin{pgfscope}%
\pgfpathrectangle{\pgfqpoint{1.150000in}{0.150000in}}{\pgfqpoint{5.700000in}{5.700000in}}%
\pgfusepath{clip}%
\pgfsetbuttcap%
\pgfsetroundjoin%
\definecolor{currentfill}{rgb}{0.229739,0.322361,0.545706}%
\pgfsetfillcolor{currentfill}%
\pgfsetfillopacity{0.800000}%
\pgfsetlinewidth{0.000000pt}%
\definecolor{currentstroke}{rgb}{0.000000,0.000000,0.000000}%
\pgfsetstrokecolor{currentstroke}%
\pgfsetdash{}{0pt}%
\pgfpathmoveto{\pgfqpoint{4.046902in}{3.078957in}}%
\pgfpathlineto{\pgfqpoint{4.060225in}{3.069789in}}%
\pgfpathlineto{\pgfqpoint{4.073550in}{3.060841in}}%
\pgfpathlineto{\pgfqpoint{4.086878in}{3.052113in}}%
\pgfpathlineto{\pgfqpoint{4.100210in}{3.043604in}}%
\pgfpathlineto{\pgfqpoint{4.107901in}{3.060294in}}%
\pgfpathlineto{\pgfqpoint{4.115589in}{3.077211in}}%
\pgfpathlineto{\pgfqpoint{4.123274in}{3.094363in}}%
\pgfpathlineto{\pgfqpoint{4.130957in}{3.111754in}}%
\pgfpathlineto{\pgfqpoint{4.117631in}{3.120764in}}%
\pgfpathlineto{\pgfqpoint{4.104309in}{3.129994in}}%
\pgfpathlineto{\pgfqpoint{4.090989in}{3.139443in}}%
\pgfpathlineto{\pgfqpoint{4.077673in}{3.149114in}}%
\pgfpathlineto{\pgfqpoint{4.069984in}{3.131209in}}%
\pgfpathlineto{\pgfqpoint{4.062293in}{3.113552in}}%
\pgfpathlineto{\pgfqpoint{4.054599in}{3.096137in}}%
\pgfpathlineto{\pgfqpoint{4.046902in}{3.078957in}}%
\pgfpathclose%
\pgfusepath{fill}%
\end{pgfscope}%
\begin{pgfscope}%
\pgfpathrectangle{\pgfqpoint{1.150000in}{0.150000in}}{\pgfqpoint{5.700000in}{5.700000in}}%
\pgfusepath{clip}%
\pgfsetbuttcap%
\pgfsetroundjoin%
\definecolor{currentfill}{rgb}{0.421908,0.805774,0.351910}%
\pgfsetfillcolor{currentfill}%
\pgfsetfillopacity{0.800000}%
\pgfsetlinewidth{0.000000pt}%
\definecolor{currentstroke}{rgb}{0.000000,0.000000,0.000000}%
\pgfsetstrokecolor{currentstroke}%
\pgfsetdash{}{0pt}%
\pgfpathmoveto{\pgfqpoint{3.760110in}{4.546023in}}%
\pgfpathlineto{\pgfqpoint{3.773485in}{4.521486in}}%
\pgfpathlineto{\pgfqpoint{3.786853in}{4.497252in}}%
\pgfpathlineto{\pgfqpoint{3.800216in}{4.473320in}}%
\pgfpathlineto{\pgfqpoint{3.813573in}{4.449687in}}%
\pgfpathlineto{\pgfqpoint{3.821221in}{4.484863in}}%
\pgfpathlineto{\pgfqpoint{3.828868in}{4.520602in}}%
\pgfpathlineto{\pgfqpoint{3.836516in}{4.556916in}}%
\pgfpathlineto{\pgfqpoint{3.823153in}{4.581164in}}%
\pgfpathlineto{\pgfqpoint{3.809784in}{4.605712in}}%
\pgfpathlineto{\pgfqpoint{3.796410in}{4.630562in}}%
\pgfpathlineto{\pgfqpoint{3.783029in}{4.655718in}}%
\pgfpathlineto{\pgfqpoint{3.775390in}{4.618571in}}%
\pgfpathlineto{\pgfqpoint{3.767750in}{4.582010in}}%
\pgfpathlineto{\pgfqpoint{3.760110in}{4.546023in}}%
\pgfpathclose%
\pgfusepath{fill}%
\end{pgfscope}%
\begin{pgfscope}%
\pgfpathrectangle{\pgfqpoint{1.150000in}{0.150000in}}{\pgfqpoint{5.700000in}{5.700000in}}%
\pgfusepath{clip}%
\pgfsetbuttcap%
\pgfsetroundjoin%
\definecolor{currentfill}{rgb}{0.191090,0.708366,0.482284}%
\pgfsetfillcolor{currentfill}%
\pgfsetfillopacity{0.800000}%
\pgfsetlinewidth{0.000000pt}%
\definecolor{currentstroke}{rgb}{0.000000,0.000000,0.000000}%
\pgfsetstrokecolor{currentstroke}%
\pgfsetdash{}{0pt}%
\pgfpathmoveto{\pgfqpoint{3.446248in}{4.208595in}}%
\pgfpathlineto{\pgfqpoint{3.459681in}{4.183822in}}%
\pgfpathlineto{\pgfqpoint{3.473105in}{4.159381in}}%
\pgfpathlineto{\pgfqpoint{3.486522in}{4.135268in}}%
\pgfpathlineto{\pgfqpoint{3.499931in}{4.111480in}}%
\pgfpathlineto{\pgfqpoint{3.507609in}{4.139142in}}%
\pgfpathlineto{\pgfqpoint{3.515284in}{4.167222in}}%
\pgfpathlineto{\pgfqpoint{3.522954in}{4.195729in}}%
\pgfpathlineto{\pgfqpoint{3.530620in}{4.224670in}}%
\pgfpathlineto{\pgfqpoint{3.517208in}{4.249100in}}%
\pgfpathlineto{\pgfqpoint{3.503788in}{4.273858in}}%
\pgfpathlineto{\pgfqpoint{3.490360in}{4.298945in}}%
\pgfpathlineto{\pgfqpoint{3.476923in}{4.324365in}}%
\pgfpathlineto{\pgfqpoint{3.469261in}{4.294764in}}%
\pgfpathlineto{\pgfqpoint{3.461594in}{4.265607in}}%
\pgfpathlineto{\pgfqpoint{3.453923in}{4.236887in}}%
\pgfpathlineto{\pgfqpoint{3.446248in}{4.208595in}}%
\pgfpathclose%
\pgfusepath{fill}%
\end{pgfscope}%
\begin{pgfscope}%
\pgfpathrectangle{\pgfqpoint{1.150000in}{0.150000in}}{\pgfqpoint{5.700000in}{5.700000in}}%
\pgfusepath{clip}%
\pgfsetbuttcap%
\pgfsetroundjoin%
\definecolor{currentfill}{rgb}{0.199430,0.387607,0.554642}%
\pgfsetfillcolor{currentfill}%
\pgfsetfillopacity{0.800000}%
\pgfsetlinewidth{0.000000pt}%
\definecolor{currentstroke}{rgb}{0.000000,0.000000,0.000000}%
\pgfsetstrokecolor{currentstroke}%
\pgfsetdash{}{0pt}%
\pgfpathmoveto{\pgfqpoint{3.528077in}{3.265392in}}%
\pgfpathlineto{\pgfqpoint{3.541408in}{3.249697in}}%
\pgfpathlineto{\pgfqpoint{3.554735in}{3.234275in}}%
\pgfpathlineto{\pgfqpoint{3.568061in}{3.219124in}}%
\pgfpathlineto{\pgfqpoint{3.581383in}{3.204241in}}%
\pgfpathlineto{\pgfqpoint{3.589161in}{3.221803in}}%
\pgfpathlineto{\pgfqpoint{3.596934in}{3.239592in}}%
\pgfpathlineto{\pgfqpoint{3.604702in}{3.257614in}}%
\pgfpathlineto{\pgfqpoint{3.612466in}{3.275874in}}%
\pgfpathlineto{\pgfqpoint{3.599148in}{3.291170in}}%
\pgfpathlineto{\pgfqpoint{3.585827in}{3.306736in}}%
\pgfpathlineto{\pgfqpoint{3.572504in}{3.322573in}}%
\pgfpathlineto{\pgfqpoint{3.559177in}{3.338683in}}%
\pgfpathlineto{\pgfqpoint{3.551410in}{3.319996in}}%
\pgfpathlineto{\pgfqpoint{3.543637in}{3.301555in}}%
\pgfpathlineto{\pgfqpoint{3.535859in}{3.283355in}}%
\pgfpathlineto{\pgfqpoint{3.528077in}{3.265392in}}%
\pgfpathclose%
\pgfusepath{fill}%
\end{pgfscope}%
\begin{pgfscope}%
\pgfpathrectangle{\pgfqpoint{1.150000in}{0.150000in}}{\pgfqpoint{5.700000in}{5.700000in}}%
\pgfusepath{clip}%
\pgfsetbuttcap%
\pgfsetroundjoin%
\definecolor{currentfill}{rgb}{0.208623,0.367752,0.552675}%
\pgfsetfillcolor{currentfill}%
\pgfsetfillopacity{0.800000}%
\pgfsetlinewidth{0.000000pt}%
\definecolor{currentstroke}{rgb}{0.000000,0.000000,0.000000}%
\pgfsetstrokecolor{currentstroke}%
\pgfsetdash{}{0pt}%
\pgfpathmoveto{\pgfqpoint{3.581383in}{3.204241in}}%
\pgfpathlineto{\pgfqpoint{3.594704in}{3.189626in}}%
\pgfpathlineto{\pgfqpoint{3.608023in}{3.175274in}}%
\pgfpathlineto{\pgfqpoint{3.621340in}{3.161186in}}%
\pgfpathlineto{\pgfqpoint{3.634655in}{3.147358in}}%
\pgfpathlineto{\pgfqpoint{3.642428in}{3.164518in}}%
\pgfpathlineto{\pgfqpoint{3.650196in}{3.181898in}}%
\pgfpathlineto{\pgfqpoint{3.657960in}{3.199503in}}%
\pgfpathlineto{\pgfqpoint{3.665719in}{3.217338in}}%
\pgfpathlineto{\pgfqpoint{3.652408in}{3.231578in}}%
\pgfpathlineto{\pgfqpoint{3.639096in}{3.246080in}}%
\pgfpathlineto{\pgfqpoint{3.625782in}{3.260844in}}%
\pgfpathlineto{\pgfqpoint{3.612466in}{3.275874in}}%
\pgfpathlineto{\pgfqpoint{3.604702in}{3.257614in}}%
\pgfpathlineto{\pgfqpoint{3.596934in}{3.239592in}}%
\pgfpathlineto{\pgfqpoint{3.589161in}{3.221803in}}%
\pgfpathlineto{\pgfqpoint{3.581383in}{3.204241in}}%
\pgfpathclose%
\pgfusepath{fill}%
\end{pgfscope}%
\begin{pgfscope}%
\pgfpathrectangle{\pgfqpoint{1.150000in}{0.150000in}}{\pgfqpoint{5.700000in}{5.700000in}}%
\pgfusepath{clip}%
\pgfsetbuttcap%
\pgfsetroundjoin%
\definecolor{currentfill}{rgb}{0.188923,0.410910,0.556326}%
\pgfsetfillcolor{currentfill}%
\pgfsetfillopacity{0.800000}%
\pgfsetlinewidth{0.000000pt}%
\definecolor{currentstroke}{rgb}{0.000000,0.000000,0.000000}%
\pgfsetstrokecolor{currentstroke}%
\pgfsetdash{}{0pt}%
\pgfpathmoveto{\pgfqpoint{3.474722in}{3.330944in}}%
\pgfpathlineto{\pgfqpoint{3.488066in}{3.314136in}}%
\pgfpathlineto{\pgfqpoint{3.501406in}{3.297609in}}%
\pgfpathlineto{\pgfqpoint{3.514743in}{3.281362in}}%
\pgfpathlineto{\pgfqpoint{3.528077in}{3.265392in}}%
\pgfpathlineto{\pgfqpoint{3.535859in}{3.283355in}}%
\pgfpathlineto{\pgfqpoint{3.543637in}{3.301555in}}%
\pgfpathlineto{\pgfqpoint{3.551410in}{3.319996in}}%
\pgfpathlineto{\pgfqpoint{3.559177in}{3.338683in}}%
\pgfpathlineto{\pgfqpoint{3.545848in}{3.355069in}}%
\pgfpathlineto{\pgfqpoint{3.532516in}{3.371732in}}%
\pgfpathlineto{\pgfqpoint{3.519180in}{3.388676in}}%
\pgfpathlineto{\pgfqpoint{3.505840in}{3.405902in}}%
\pgfpathlineto{\pgfqpoint{3.498068in}{3.386786in}}%
\pgfpathlineto{\pgfqpoint{3.490291in}{3.367924in}}%
\pgfpathlineto{\pgfqpoint{3.482509in}{3.349312in}}%
\pgfpathlineto{\pgfqpoint{3.474722in}{3.330944in}}%
\pgfpathclose%
\pgfusepath{fill}%
\end{pgfscope}%
\begin{pgfscope}%
\pgfpathrectangle{\pgfqpoint{1.150000in}{0.150000in}}{\pgfqpoint{5.700000in}{5.700000in}}%
\pgfusepath{clip}%
\pgfsetbuttcap%
\pgfsetroundjoin%
\definecolor{currentfill}{rgb}{0.119738,0.603785,0.541400}%
\pgfsetfillcolor{currentfill}%
\pgfsetfillopacity{0.800000}%
\pgfsetlinewidth{0.000000pt}%
\definecolor{currentstroke}{rgb}{0.000000,0.000000,0.000000}%
\pgfsetstrokecolor{currentstroke}%
\pgfsetdash{}{0pt}%
\pgfpathmoveto{\pgfqpoint{3.353770in}{3.899896in}}%
\pgfpathlineto{\pgfqpoint{3.367200in}{3.876838in}}%
\pgfpathlineto{\pgfqpoint{3.380622in}{3.854107in}}%
\pgfpathlineto{\pgfqpoint{3.394036in}{3.831701in}}%
\pgfpathlineto{\pgfqpoint{3.407443in}{3.809617in}}%
\pgfpathlineto{\pgfqpoint{3.415177in}{3.832801in}}%
\pgfpathlineto{\pgfqpoint{3.422905in}{3.856322in}}%
\pgfpathlineto{\pgfqpoint{3.430629in}{3.880186in}}%
\pgfpathlineto{\pgfqpoint{3.438348in}{3.904400in}}%
\pgfpathlineto{\pgfqpoint{3.424942in}{3.927015in}}%
\pgfpathlineto{\pgfqpoint{3.411528in}{3.949954in}}%
\pgfpathlineto{\pgfqpoint{3.398106in}{3.973217in}}%
\pgfpathlineto{\pgfqpoint{3.384677in}{3.996810in}}%
\pgfpathlineto{\pgfqpoint{3.376958in}{3.972050in}}%
\pgfpathlineto{\pgfqpoint{3.369234in}{3.947649in}}%
\pgfpathlineto{\pgfqpoint{3.361505in}{3.923600in}}%
\pgfpathlineto{\pgfqpoint{3.353770in}{3.899896in}}%
\pgfpathclose%
\pgfusepath{fill}%
\end{pgfscope}%
\begin{pgfscope}%
\pgfpathrectangle{\pgfqpoint{1.150000in}{0.150000in}}{\pgfqpoint{5.700000in}{5.700000in}}%
\pgfusepath{clip}%
\pgfsetbuttcap%
\pgfsetroundjoin%
\definecolor{currentfill}{rgb}{0.404001,0.800275,0.362552}%
\pgfsetfillcolor{currentfill}%
\pgfsetfillopacity{0.800000}%
\pgfsetlinewidth{0.000000pt}%
\definecolor{currentstroke}{rgb}{0.000000,0.000000,0.000000}%
\pgfsetstrokecolor{currentstroke}%
\pgfsetdash{}{0pt}%
\pgfpathmoveto{\pgfqpoint{3.676010in}{4.505694in}}%
\pgfpathlineto{\pgfqpoint{3.689403in}{4.480708in}}%
\pgfpathlineto{\pgfqpoint{3.702789in}{4.456036in}}%
\pgfpathlineto{\pgfqpoint{3.716169in}{4.431675in}}%
\pgfpathlineto{\pgfqpoint{3.729542in}{4.407621in}}%
\pgfpathlineto{\pgfqpoint{3.737186in}{4.441409in}}%
\pgfpathlineto{\pgfqpoint{3.744828in}{4.475732in}}%
\pgfpathlineto{\pgfqpoint{3.752470in}{4.510600in}}%
\pgfpathlineto{\pgfqpoint{3.760110in}{4.546023in}}%
\pgfpathlineto{\pgfqpoint{3.746730in}{4.570866in}}%
\pgfpathlineto{\pgfqpoint{3.733342in}{4.596020in}}%
\pgfpathlineto{\pgfqpoint{3.719948in}{4.621486in}}%
\pgfpathlineto{\pgfqpoint{3.706547in}{4.647267in}}%
\pgfpathlineto{\pgfqpoint{3.698915in}{4.611035in}}%
\pgfpathlineto{\pgfqpoint{3.691282in}{4.575369in}}%
\pgfpathlineto{\pgfqpoint{3.683647in}{4.540259in}}%
\pgfpathlineto{\pgfqpoint{3.676010in}{4.505694in}}%
\pgfpathclose%
\pgfusepath{fill}%
\end{pgfscope}%
\begin{pgfscope}%
\pgfpathrectangle{\pgfqpoint{1.150000in}{0.150000in}}{\pgfqpoint{5.700000in}{5.700000in}}%
\pgfusepath{clip}%
\pgfsetbuttcap%
\pgfsetroundjoin%
\definecolor{currentfill}{rgb}{0.206756,0.371758,0.553117}%
\pgfsetfillcolor{currentfill}%
\pgfsetfillopacity{0.800000}%
\pgfsetlinewidth{0.000000pt}%
\definecolor{currentstroke}{rgb}{0.000000,0.000000,0.000000}%
\pgfsetstrokecolor{currentstroke}%
\pgfsetdash{}{0pt}%
\pgfpathmoveto{\pgfqpoint{4.436477in}{3.199073in}}%
\pgfpathlineto{\pgfqpoint{4.449859in}{3.191932in}}%
\pgfpathlineto{\pgfqpoint{4.463246in}{3.184993in}}%
\pgfpathlineto{\pgfqpoint{4.476640in}{3.178254in}}%
\pgfpathlineto{\pgfqpoint{4.490039in}{3.171714in}}%
\pgfpathlineto{\pgfqpoint{4.497669in}{3.189727in}}%
\pgfpathlineto{\pgfqpoint{4.505298in}{3.208034in}}%
\pgfpathlineto{\pgfqpoint{4.512928in}{3.226640in}}%
\pgfpathlineto{\pgfqpoint{4.520558in}{3.245554in}}%
\pgfpathlineto{\pgfqpoint{4.507167in}{3.252720in}}%
\pgfpathlineto{\pgfqpoint{4.493782in}{3.260085in}}%
\pgfpathlineto{\pgfqpoint{4.480402in}{3.267651in}}%
\pgfpathlineto{\pgfqpoint{4.467028in}{3.275418in}}%
\pgfpathlineto{\pgfqpoint{4.459390in}{3.255866in}}%
\pgfpathlineto{\pgfqpoint{4.451753in}{3.236629in}}%
\pgfpathlineto{\pgfqpoint{4.444115in}{3.217701in}}%
\pgfpathlineto{\pgfqpoint{4.436477in}{3.199073in}}%
\pgfpathclose%
\pgfusepath{fill}%
\end{pgfscope}%
\begin{pgfscope}%
\pgfpathrectangle{\pgfqpoint{1.150000in}{0.150000in}}{\pgfqpoint{5.700000in}{5.700000in}}%
\pgfusepath{clip}%
\pgfsetbuttcap%
\pgfsetroundjoin%
\definecolor{currentfill}{rgb}{0.214298,0.355619,0.551184}%
\pgfsetfillcolor{currentfill}%
\pgfsetfillopacity{0.800000}%
\pgfsetlinewidth{0.000000pt}%
\definecolor{currentstroke}{rgb}{0.000000,0.000000,0.000000}%
\pgfsetstrokecolor{currentstroke}%
\pgfsetdash{}{0pt}%
\pgfpathmoveto{\pgfqpoint{4.352416in}{3.155638in}}%
\pgfpathlineto{\pgfqpoint{4.365784in}{3.148279in}}%
\pgfpathlineto{\pgfqpoint{4.379158in}{3.141125in}}%
\pgfpathlineto{\pgfqpoint{4.392536in}{3.134174in}}%
\pgfpathlineto{\pgfqpoint{4.405921in}{3.127426in}}%
\pgfpathlineto{\pgfqpoint{4.413561in}{3.144922in}}%
\pgfpathlineto{\pgfqpoint{4.421200in}{3.162691in}}%
\pgfpathlineto{\pgfqpoint{4.428839in}{3.180739in}}%
\pgfpathlineto{\pgfqpoint{4.436477in}{3.199073in}}%
\pgfpathlineto{\pgfqpoint{4.423100in}{3.206415in}}%
\pgfpathlineto{\pgfqpoint{4.409729in}{3.213961in}}%
\pgfpathlineto{\pgfqpoint{4.396363in}{3.221710in}}%
\pgfpathlineto{\pgfqpoint{4.383002in}{3.229664in}}%
\pgfpathlineto{\pgfqpoint{4.375357in}{3.210723in}}%
\pgfpathlineto{\pgfqpoint{4.367710in}{3.192076in}}%
\pgfpathlineto{\pgfqpoint{4.360064in}{3.173717in}}%
\pgfpathlineto{\pgfqpoint{4.352416in}{3.155638in}}%
\pgfpathclose%
\pgfusepath{fill}%
\end{pgfscope}%
\begin{pgfscope}%
\pgfpathrectangle{\pgfqpoint{1.150000in}{0.150000in}}{\pgfqpoint{5.700000in}{5.700000in}}%
\pgfusepath{clip}%
\pgfsetbuttcap%
\pgfsetroundjoin%
\definecolor{currentfill}{rgb}{0.171176,0.452530,0.557965}%
\pgfsetfillcolor{currentfill}%
\pgfsetfillopacity{0.800000}%
\pgfsetlinewidth{0.000000pt}%
\definecolor{currentstroke}{rgb}{0.000000,0.000000,0.000000}%
\pgfsetstrokecolor{currentstroke}%
\pgfsetdash{}{0pt}%
\pgfpathmoveto{\pgfqpoint{4.719358in}{3.432546in}}%
\pgfpathlineto{\pgfqpoint{4.732777in}{3.425072in}}%
\pgfpathlineto{\pgfqpoint{4.746202in}{3.417790in}}%
\pgfpathlineto{\pgfqpoint{4.759634in}{3.410700in}}%
\pgfpathlineto{\pgfqpoint{4.773072in}{3.403801in}}%
\pgfpathlineto{\pgfqpoint{4.780700in}{3.425207in}}%
\pgfpathlineto{\pgfqpoint{4.788331in}{3.447010in}}%
\pgfpathlineto{\pgfqpoint{4.795967in}{3.469219in}}%
\pgfpathlineto{\pgfqpoint{4.782536in}{3.476681in}}%
\pgfpathlineto{\pgfqpoint{4.769111in}{3.484334in}}%
\pgfpathlineto{\pgfqpoint{4.755693in}{3.492178in}}%
\pgfpathlineto{\pgfqpoint{4.742281in}{3.500215in}}%
\pgfpathlineto{\pgfqpoint{4.734636in}{3.477247in}}%
\pgfpathlineto{\pgfqpoint{4.726995in}{3.454693in}}%
\pgfpathlineto{\pgfqpoint{4.719358in}{3.432546in}}%
\pgfpathclose%
\pgfusepath{fill}%
\end{pgfscope}%
\begin{pgfscope}%
\pgfpathrectangle{\pgfqpoint{1.150000in}{0.150000in}}{\pgfqpoint{5.700000in}{5.700000in}}%
\pgfusepath{clip}%
\pgfsetbuttcap%
\pgfsetroundjoin%
\definecolor{currentfill}{rgb}{0.218130,0.347432,0.550038}%
\pgfsetfillcolor{currentfill}%
\pgfsetfillopacity{0.800000}%
\pgfsetlinewidth{0.000000pt}%
\definecolor{currentstroke}{rgb}{0.000000,0.000000,0.000000}%
\pgfsetstrokecolor{currentstroke}%
\pgfsetdash{}{0pt}%
\pgfpathmoveto{\pgfqpoint{3.634655in}{3.147358in}}%
\pgfpathlineto{\pgfqpoint{3.647970in}{3.133788in}}%
\pgfpathlineto{\pgfqpoint{3.661283in}{3.120476in}}%
\pgfpathlineto{\pgfqpoint{3.674595in}{3.107418in}}%
\pgfpathlineto{\pgfqpoint{3.687906in}{3.094614in}}%
\pgfpathlineto{\pgfqpoint{3.695674in}{3.111375in}}%
\pgfpathlineto{\pgfqpoint{3.703437in}{3.128347in}}%
\pgfpathlineto{\pgfqpoint{3.711195in}{3.145537in}}%
\pgfpathlineto{\pgfqpoint{3.718950in}{3.162948in}}%
\pgfpathlineto{\pgfqpoint{3.705643in}{3.176163in}}%
\pgfpathlineto{\pgfqpoint{3.692336in}{3.189632in}}%
\pgfpathlineto{\pgfqpoint{3.679028in}{3.203356in}}%
\pgfpathlineto{\pgfqpoint{3.665719in}{3.217338in}}%
\pgfpathlineto{\pgfqpoint{3.657960in}{3.199503in}}%
\pgfpathlineto{\pgfqpoint{3.650196in}{3.181898in}}%
\pgfpathlineto{\pgfqpoint{3.642428in}{3.164518in}}%
\pgfpathlineto{\pgfqpoint{3.634655in}{3.147358in}}%
\pgfpathclose%
\pgfusepath{fill}%
\end{pgfscope}%
\begin{pgfscope}%
\pgfpathrectangle{\pgfqpoint{1.150000in}{0.150000in}}{\pgfqpoint{5.700000in}{5.700000in}}%
\pgfusepath{clip}%
\pgfsetbuttcap%
\pgfsetroundjoin%
\definecolor{currentfill}{rgb}{0.231674,0.318106,0.544834}%
\pgfsetfillcolor{currentfill}%
\pgfsetfillopacity{0.800000}%
\pgfsetlinewidth{0.000000pt}%
\definecolor{currentstroke}{rgb}{0.000000,0.000000,0.000000}%
\pgfsetstrokecolor{currentstroke}%
\pgfsetdash{}{0pt}%
\pgfpathmoveto{\pgfqpoint{3.825398in}{3.066140in}}%
\pgfpathlineto{\pgfqpoint{3.838706in}{3.055129in}}%
\pgfpathlineto{\pgfqpoint{3.852016in}{3.044355in}}%
\pgfpathlineto{\pgfqpoint{3.865327in}{3.033816in}}%
\pgfpathlineto{\pgfqpoint{3.878639in}{3.023511in}}%
\pgfpathlineto{\pgfqpoint{3.886374in}{3.039861in}}%
\pgfpathlineto{\pgfqpoint{3.894105in}{3.056418in}}%
\pgfpathlineto{\pgfqpoint{3.901832in}{3.073188in}}%
\pgfpathlineto{\pgfqpoint{3.909556in}{3.090176in}}%
\pgfpathlineto{\pgfqpoint{3.896248in}{3.100921in}}%
\pgfpathlineto{\pgfqpoint{3.882943in}{3.111900in}}%
\pgfpathlineto{\pgfqpoint{3.869638in}{3.123114in}}%
\pgfpathlineto{\pgfqpoint{3.856335in}{3.134566in}}%
\pgfpathlineto{\pgfqpoint{3.848606in}{3.117126in}}%
\pgfpathlineto{\pgfqpoint{3.840874in}{3.099911in}}%
\pgfpathlineto{\pgfqpoint{3.833138in}{3.082918in}}%
\pgfpathlineto{\pgfqpoint{3.825398in}{3.066140in}}%
\pgfpathclose%
\pgfusepath{fill}%
\end{pgfscope}%
\begin{pgfscope}%
\pgfpathrectangle{\pgfqpoint{1.150000in}{0.150000in}}{\pgfqpoint{5.700000in}{5.700000in}}%
\pgfusepath{clip}%
\pgfsetbuttcap%
\pgfsetroundjoin%
\definecolor{currentfill}{rgb}{0.199430,0.387607,0.554642}%
\pgfsetfillcolor{currentfill}%
\pgfsetfillopacity{0.800000}%
\pgfsetlinewidth{0.000000pt}%
\definecolor{currentstroke}{rgb}{0.000000,0.000000,0.000000}%
\pgfsetstrokecolor{currentstroke}%
\pgfsetdash{}{0pt}%
\pgfpathmoveto{\pgfqpoint{4.520558in}{3.245554in}}%
\pgfpathlineto{\pgfqpoint{4.533955in}{3.238587in}}%
\pgfpathlineto{\pgfqpoint{4.547357in}{3.231817in}}%
\pgfpathlineto{\pgfqpoint{4.560766in}{3.225245in}}%
\pgfpathlineto{\pgfqpoint{4.574181in}{3.218869in}}%
\pgfpathlineto{\pgfqpoint{4.581803in}{3.237452in}}%
\pgfpathlineto{\pgfqpoint{4.589426in}{3.256350in}}%
\pgfpathlineto{\pgfqpoint{4.597049in}{3.275570in}}%
\pgfpathlineto{\pgfqpoint{4.604674in}{3.295120in}}%
\pgfpathlineto{\pgfqpoint{4.591268in}{3.302154in}}%
\pgfpathlineto{\pgfqpoint{4.577868in}{3.309384in}}%
\pgfpathlineto{\pgfqpoint{4.564474in}{3.316811in}}%
\pgfpathlineto{\pgfqpoint{4.551085in}{3.324436in}}%
\pgfpathlineto{\pgfqpoint{4.543452in}{3.304216in}}%
\pgfpathlineto{\pgfqpoint{4.535820in}{3.284334in}}%
\pgfpathlineto{\pgfqpoint{4.528189in}{3.264783in}}%
\pgfpathlineto{\pgfqpoint{4.520558in}{3.245554in}}%
\pgfpathclose%
\pgfusepath{fill}%
\end{pgfscope}%
\begin{pgfscope}%
\pgfpathrectangle{\pgfqpoint{1.150000in}{0.150000in}}{\pgfqpoint{5.700000in}{5.700000in}}%
\pgfusepath{clip}%
\pgfsetbuttcap%
\pgfsetroundjoin%
\definecolor{currentfill}{rgb}{0.221989,0.339161,0.548752}%
\pgfsetfillcolor{currentfill}%
\pgfsetfillopacity{0.800000}%
\pgfsetlinewidth{0.000000pt}%
\definecolor{currentstroke}{rgb}{0.000000,0.000000,0.000000}%
\pgfsetstrokecolor{currentstroke}%
\pgfsetdash{}{0pt}%
\pgfpathmoveto{\pgfqpoint{4.268361in}{3.115235in}}%
\pgfpathlineto{\pgfqpoint{4.281716in}{3.107612in}}%
\pgfpathlineto{\pgfqpoint{4.295077in}{3.100197in}}%
\pgfpathlineto{\pgfqpoint{4.308442in}{3.092990in}}%
\pgfpathlineto{\pgfqpoint{4.321813in}{3.085989in}}%
\pgfpathlineto{\pgfqpoint{4.329466in}{3.103014in}}%
\pgfpathlineto{\pgfqpoint{4.337117in}{3.120293in}}%
\pgfpathlineto{\pgfqpoint{4.344767in}{3.137832in}}%
\pgfpathlineto{\pgfqpoint{4.352416in}{3.155638in}}%
\pgfpathlineto{\pgfqpoint{4.339053in}{3.163202in}}%
\pgfpathlineto{\pgfqpoint{4.325694in}{3.170973in}}%
\pgfpathlineto{\pgfqpoint{4.312341in}{3.178951in}}%
\pgfpathlineto{\pgfqpoint{4.298992in}{3.187137in}}%
\pgfpathlineto{\pgfqpoint{4.291336in}{3.168756in}}%
\pgfpathlineto{\pgfqpoint{4.283679in}{3.150650in}}%
\pgfpathlineto{\pgfqpoint{4.276021in}{3.132812in}}%
\pgfpathlineto{\pgfqpoint{4.268361in}{3.115235in}}%
\pgfpathclose%
\pgfusepath{fill}%
\end{pgfscope}%
\begin{pgfscope}%
\pgfpathrectangle{\pgfqpoint{1.150000in}{0.150000in}}{\pgfqpoint{5.700000in}{5.700000in}}%
\pgfusepath{clip}%
\pgfsetbuttcap%
\pgfsetroundjoin%
\definecolor{currentfill}{rgb}{0.177423,0.437527,0.557565}%
\pgfsetfillcolor{currentfill}%
\pgfsetfillopacity{0.800000}%
\pgfsetlinewidth{0.000000pt}%
\definecolor{currentstroke}{rgb}{0.000000,0.000000,0.000000}%
\pgfsetstrokecolor{currentstroke}%
\pgfsetdash{}{0pt}%
\pgfpathmoveto{\pgfqpoint{3.421303in}{3.401046in}}%
\pgfpathlineto{\pgfqpoint{3.434665in}{3.383085in}}%
\pgfpathlineto{\pgfqpoint{3.448022in}{3.365417in}}%
\pgfpathlineto{\pgfqpoint{3.461374in}{3.348037in}}%
\pgfpathlineto{\pgfqpoint{3.474722in}{3.330944in}}%
\pgfpathlineto{\pgfqpoint{3.482509in}{3.349312in}}%
\pgfpathlineto{\pgfqpoint{3.490291in}{3.367924in}}%
\pgfpathlineto{\pgfqpoint{3.498068in}{3.386786in}}%
\pgfpathlineto{\pgfqpoint{3.505840in}{3.405902in}}%
\pgfpathlineto{\pgfqpoint{3.492496in}{3.423413in}}%
\pgfpathlineto{\pgfqpoint{3.479148in}{3.441211in}}%
\pgfpathlineto{\pgfqpoint{3.465795in}{3.459299in}}%
\pgfpathlineto{\pgfqpoint{3.452438in}{3.477678in}}%
\pgfpathlineto{\pgfqpoint{3.444662in}{3.458131in}}%
\pgfpathlineto{\pgfqpoint{3.436881in}{3.438846in}}%
\pgfpathlineto{\pgfqpoint{3.429095in}{3.419819in}}%
\pgfpathlineto{\pgfqpoint{3.421303in}{3.401046in}}%
\pgfpathclose%
\pgfusepath{fill}%
\end{pgfscope}%
\begin{pgfscope}%
\pgfpathrectangle{\pgfqpoint{1.150000in}{0.150000in}}{\pgfqpoint{5.700000in}{5.700000in}}%
\pgfusepath{clip}%
\pgfsetbuttcap%
\pgfsetroundjoin%
\definecolor{currentfill}{rgb}{0.266941,0.748751,0.440573}%
\pgfsetfillcolor{currentfill}%
\pgfsetfillopacity{0.800000}%
\pgfsetlinewidth{0.000000pt}%
\definecolor{currentstroke}{rgb}{0.000000,0.000000,0.000000}%
\pgfsetstrokecolor{currentstroke}%
\pgfsetdash{}{0pt}%
\pgfpathmoveto{\pgfqpoint{3.476923in}{4.324365in}}%
\pgfpathlineto{\pgfqpoint{3.490360in}{4.298945in}}%
\pgfpathlineto{\pgfqpoint{3.503788in}{4.273858in}}%
\pgfpathlineto{\pgfqpoint{3.517208in}{4.249100in}}%
\pgfpathlineto{\pgfqpoint{3.530620in}{4.224670in}}%
\pgfpathlineto{\pgfqpoint{3.538283in}{4.254053in}}%
\pgfpathlineto{\pgfqpoint{3.545943in}{4.283886in}}%
\pgfpathlineto{\pgfqpoint{3.553598in}{4.314178in}}%
\pgfpathlineto{\pgfqpoint{3.561251in}{4.344937in}}%
\pgfpathlineto{\pgfqpoint{3.547834in}{4.370049in}}%
\pgfpathlineto{\pgfqpoint{3.534409in}{4.395490in}}%
\pgfpathlineto{\pgfqpoint{3.520976in}{4.421262in}}%
\pgfpathlineto{\pgfqpoint{3.507534in}{4.447369in}}%
\pgfpathlineto{\pgfqpoint{3.499887in}{4.415912in}}%
\pgfpathlineto{\pgfqpoint{3.492236in}{4.384930in}}%
\pgfpathlineto{\pgfqpoint{3.484582in}{4.354418in}}%
\pgfpathlineto{\pgfqpoint{3.476923in}{4.324365in}}%
\pgfpathclose%
\pgfusepath{fill}%
\end{pgfscope}%
\begin{pgfscope}%
\pgfpathrectangle{\pgfqpoint{1.150000in}{0.150000in}}{\pgfqpoint{5.700000in}{5.700000in}}%
\pgfusepath{clip}%
\pgfsetbuttcap%
\pgfsetroundjoin%
\definecolor{currentfill}{rgb}{0.235526,0.309527,0.542944}%
\pgfsetfillcolor{currentfill}%
\pgfsetfillopacity{0.800000}%
\pgfsetlinewidth{0.000000pt}%
\definecolor{currentstroke}{rgb}{0.000000,0.000000,0.000000}%
\pgfsetstrokecolor{currentstroke}%
\pgfsetdash{}{0pt}%
\pgfpathmoveto{\pgfqpoint{3.962802in}{3.049511in}}%
\pgfpathlineto{\pgfqpoint{3.976119in}{3.039916in}}%
\pgfpathlineto{\pgfqpoint{3.989439in}{3.030547in}}%
\pgfpathlineto{\pgfqpoint{4.002761in}{3.021403in}}%
\pgfpathlineto{\pgfqpoint{4.016086in}{3.012482in}}%
\pgfpathlineto{\pgfqpoint{4.023795in}{3.028776in}}%
\pgfpathlineto{\pgfqpoint{4.031500in}{3.045282in}}%
\pgfpathlineto{\pgfqpoint{4.039203in}{3.062008in}}%
\pgfpathlineto{\pgfqpoint{4.046902in}{3.078957in}}%
\pgfpathlineto{\pgfqpoint{4.033583in}{3.088349in}}%
\pgfpathlineto{\pgfqpoint{4.020267in}{3.097963in}}%
\pgfpathlineto{\pgfqpoint{4.006953in}{3.107803in}}%
\pgfpathlineto{\pgfqpoint{3.993641in}{3.117869in}}%
\pgfpathlineto{\pgfqpoint{3.985936in}{3.100436in}}%
\pgfpathlineto{\pgfqpoint{3.978228in}{3.083236in}}%
\pgfpathlineto{\pgfqpoint{3.970517in}{3.066263in}}%
\pgfpathlineto{\pgfqpoint{3.962802in}{3.049511in}}%
\pgfpathclose%
\pgfusepath{fill}%
\end{pgfscope}%
\begin{pgfscope}%
\pgfpathrectangle{\pgfqpoint{1.150000in}{0.150000in}}{\pgfqpoint{5.700000in}{5.700000in}}%
\pgfusepath{clip}%
\pgfsetbuttcap%
\pgfsetroundjoin%
\definecolor{currentfill}{rgb}{0.146180,0.515413,0.556823}%
\pgfsetfillcolor{currentfill}%
\pgfsetfillopacity{0.800000}%
\pgfsetlinewidth{0.000000pt}%
\definecolor{currentstroke}{rgb}{0.000000,0.000000,0.000000}%
\pgfsetstrokecolor{currentstroke}%
\pgfsetdash{}{0pt}%
\pgfpathmoveto{\pgfqpoint{3.345380in}{3.635548in}}%
\pgfpathlineto{\pgfqpoint{3.358784in}{3.614736in}}%
\pgfpathlineto{\pgfqpoint{3.372181in}{3.594237in}}%
\pgfpathlineto{\pgfqpoint{3.385572in}{3.574050in}}%
\pgfpathlineto{\pgfqpoint{3.398957in}{3.554170in}}%
\pgfpathlineto{\pgfqpoint{3.406731in}{3.574428in}}%
\pgfpathlineto{\pgfqpoint{3.414499in}{3.594968in}}%
\pgfpathlineto{\pgfqpoint{3.422262in}{3.615795in}}%
\pgfpathlineto{\pgfqpoint{3.430020in}{3.636915in}}%
\pgfpathlineto{\pgfqpoint{3.416639in}{3.657251in}}%
\pgfpathlineto{\pgfqpoint{3.403251in}{3.677896in}}%
\pgfpathlineto{\pgfqpoint{3.389856in}{3.698853in}}%
\pgfpathlineto{\pgfqpoint{3.376455in}{3.720125in}}%
\pgfpathlineto{\pgfqpoint{3.368694in}{3.698534in}}%
\pgfpathlineto{\pgfqpoint{3.360929in}{3.677245in}}%
\pgfpathlineto{\pgfqpoint{3.353157in}{3.656251in}}%
\pgfpathlineto{\pgfqpoint{3.345380in}{3.635548in}}%
\pgfpathclose%
\pgfusepath{fill}%
\end{pgfscope}%
\begin{pgfscope}%
\pgfpathrectangle{\pgfqpoint{1.150000in}{0.150000in}}{\pgfqpoint{5.700000in}{5.700000in}}%
\pgfusepath{clip}%
\pgfsetbuttcap%
\pgfsetroundjoin%
\definecolor{currentfill}{rgb}{0.377779,0.791781,0.377939}%
\pgfsetfillcolor{currentfill}%
\pgfsetfillopacity{0.800000}%
\pgfsetlinewidth{0.000000pt}%
\definecolor{currentstroke}{rgb}{0.000000,0.000000,0.000000}%
\pgfsetstrokecolor{currentstroke}%
\pgfsetdash{}{0pt}%
\pgfpathmoveto{\pgfqpoint{3.591830in}{4.472804in}}%
\pgfpathlineto{\pgfqpoint{3.605245in}{4.447296in}}%
\pgfpathlineto{\pgfqpoint{3.618653in}{4.422112in}}%
\pgfpathlineto{\pgfqpoint{3.632053in}{4.397249in}}%
\pgfpathlineto{\pgfqpoint{3.645446in}{4.372703in}}%
\pgfpathlineto{\pgfqpoint{3.653090in}{4.405179in}}%
\pgfpathlineto{\pgfqpoint{3.660732in}{4.438163in}}%
\pgfpathlineto{\pgfqpoint{3.668372in}{4.471665in}}%
\pgfpathlineto{\pgfqpoint{3.676010in}{4.505694in}}%
\pgfpathlineto{\pgfqpoint{3.662610in}{4.530996in}}%
\pgfpathlineto{\pgfqpoint{3.649203in}{4.556617in}}%
\pgfpathlineto{\pgfqpoint{3.635788in}{4.582560in}}%
\pgfpathlineto{\pgfqpoint{3.622366in}{4.608830in}}%
\pgfpathlineto{\pgfqpoint{3.614736in}{4.574027in}}%
\pgfpathlineto{\pgfqpoint{3.607103in}{4.539761in}}%
\pgfpathlineto{\pgfqpoint{3.599468in}{4.506023in}}%
\pgfpathlineto{\pgfqpoint{3.591830in}{4.472804in}}%
\pgfpathclose%
\pgfusepath{fill}%
\end{pgfscope}%
\begin{pgfscope}%
\pgfpathrectangle{\pgfqpoint{1.150000in}{0.150000in}}{\pgfqpoint{5.700000in}{5.700000in}}%
\pgfusepath{clip}%
\pgfsetbuttcap%
\pgfsetroundjoin%
\definecolor{currentfill}{rgb}{0.190631,0.407061,0.556089}%
\pgfsetfillcolor{currentfill}%
\pgfsetfillopacity{0.800000}%
\pgfsetlinewidth{0.000000pt}%
\definecolor{currentstroke}{rgb}{0.000000,0.000000,0.000000}%
\pgfsetstrokecolor{currentstroke}%
\pgfsetdash{}{0pt}%
\pgfpathmoveto{\pgfqpoint{4.604674in}{3.295120in}}%
\pgfpathlineto{\pgfqpoint{4.618086in}{3.288282in}}%
\pgfpathlineto{\pgfqpoint{4.631505in}{3.281639in}}%
\pgfpathlineto{\pgfqpoint{4.644930in}{3.275190in}}%
\pgfpathlineto{\pgfqpoint{4.658362in}{3.268935in}}%
\pgfpathlineto{\pgfqpoint{4.665979in}{3.288145in}}%
\pgfpathlineto{\pgfqpoint{4.673597in}{3.307694in}}%
\pgfpathlineto{\pgfqpoint{4.681218in}{3.327589in}}%
\pgfpathlineto{\pgfqpoint{4.688840in}{3.347838in}}%
\pgfpathlineto{\pgfqpoint{4.675418in}{3.354783in}}%
\pgfpathlineto{\pgfqpoint{4.662003in}{3.361922in}}%
\pgfpathlineto{\pgfqpoint{4.648593in}{3.369254in}}%
\pgfpathlineto{\pgfqpoint{4.635190in}{3.376782in}}%
\pgfpathlineto{\pgfqpoint{4.627558in}{3.355831in}}%
\pgfpathlineto{\pgfqpoint{4.619928in}{3.335243in}}%
\pgfpathlineto{\pgfqpoint{4.612300in}{3.315008in}}%
\pgfpathlineto{\pgfqpoint{4.604674in}{3.295120in}}%
\pgfpathclose%
\pgfusepath{fill}%
\end{pgfscope}%
\begin{pgfscope}%
\pgfpathrectangle{\pgfqpoint{1.150000in}{0.150000in}}{\pgfqpoint{5.700000in}{5.700000in}}%
\pgfusepath{clip}%
\pgfsetbuttcap%
\pgfsetroundjoin%
\definecolor{currentfill}{rgb}{0.227802,0.326594,0.546532}%
\pgfsetfillcolor{currentfill}%
\pgfsetfillopacity{0.800000}%
\pgfsetlinewidth{0.000000pt}%
\definecolor{currentstroke}{rgb}{0.000000,0.000000,0.000000}%
\pgfsetstrokecolor{currentstroke}%
\pgfsetdash{}{0pt}%
\pgfpathmoveto{\pgfqpoint{4.184297in}{3.077877in}}%
\pgfpathlineto{\pgfqpoint{4.197642in}{3.069944in}}%
\pgfpathlineto{\pgfqpoint{4.210991in}{3.062222in}}%
\pgfpathlineto{\pgfqpoint{4.224344in}{3.054712in}}%
\pgfpathlineto{\pgfqpoint{4.237702in}{3.047412in}}%
\pgfpathlineto{\pgfqpoint{4.245370in}{3.064008in}}%
\pgfpathlineto{\pgfqpoint{4.253035in}{3.080839in}}%
\pgfpathlineto{\pgfqpoint{4.260699in}{3.097913in}}%
\pgfpathlineto{\pgfqpoint{4.268361in}{3.115235in}}%
\pgfpathlineto{\pgfqpoint{4.255009in}{3.123067in}}%
\pgfpathlineto{\pgfqpoint{4.241663in}{3.131109in}}%
\pgfpathlineto{\pgfqpoint{4.228320in}{3.139362in}}%
\pgfpathlineto{\pgfqpoint{4.214982in}{3.147828in}}%
\pgfpathlineto{\pgfqpoint{4.207314in}{3.129962in}}%
\pgfpathlineto{\pgfqpoint{4.199643in}{3.112352in}}%
\pgfpathlineto{\pgfqpoint{4.191971in}{3.094993in}}%
\pgfpathlineto{\pgfqpoint{4.184297in}{3.077877in}}%
\pgfpathclose%
\pgfusepath{fill}%
\end{pgfscope}%
\begin{pgfscope}%
\pgfpathrectangle{\pgfqpoint{1.150000in}{0.150000in}}{\pgfqpoint{5.700000in}{5.700000in}}%
\pgfusepath{clip}%
\pgfsetbuttcap%
\pgfsetroundjoin%
\definecolor{currentfill}{rgb}{0.125394,0.574318,0.549086}%
\pgfsetfillcolor{currentfill}%
\pgfsetfillopacity{0.800000}%
\pgfsetlinewidth{0.000000pt}%
\definecolor{currentstroke}{rgb}{0.000000,0.000000,0.000000}%
\pgfsetstrokecolor{currentstroke}%
\pgfsetdash{}{0pt}%
\pgfpathmoveto{\pgfqpoint{3.322777in}{3.808417in}}%
\pgfpathlineto{\pgfqpoint{3.336208in}{3.785857in}}%
\pgfpathlineto{\pgfqpoint{3.349631in}{3.763624in}}%
\pgfpathlineto{\pgfqpoint{3.363047in}{3.741714in}}%
\pgfpathlineto{\pgfqpoint{3.376455in}{3.720125in}}%
\pgfpathlineto{\pgfqpoint{3.384210in}{3.742023in}}%
\pgfpathlineto{\pgfqpoint{3.391959in}{3.764234in}}%
\pgfpathlineto{\pgfqpoint{3.399704in}{3.786763in}}%
\pgfpathlineto{\pgfqpoint{3.407443in}{3.809617in}}%
\pgfpathlineto{\pgfqpoint{3.394036in}{3.831701in}}%
\pgfpathlineto{\pgfqpoint{3.380622in}{3.854107in}}%
\pgfpathlineto{\pgfqpoint{3.367200in}{3.876838in}}%
\pgfpathlineto{\pgfqpoint{3.353770in}{3.899896in}}%
\pgfpathlineto{\pgfqpoint{3.346030in}{3.876532in}}%
\pgfpathlineto{\pgfqpoint{3.338285in}{3.853502in}}%
\pgfpathlineto{\pgfqpoint{3.330533in}{3.830799in}}%
\pgfpathlineto{\pgfqpoint{3.322777in}{3.808417in}}%
\pgfpathclose%
\pgfusepath{fill}%
\end{pgfscope}%
\begin{pgfscope}%
\pgfpathrectangle{\pgfqpoint{1.150000in}{0.150000in}}{\pgfqpoint{5.700000in}{5.700000in}}%
\pgfusepath{clip}%
\pgfsetbuttcap%
\pgfsetroundjoin%
\definecolor{currentfill}{rgb}{0.227802,0.326594,0.546532}%
\pgfsetfillcolor{currentfill}%
\pgfsetfillopacity{0.800000}%
\pgfsetlinewidth{0.000000pt}%
\definecolor{currentstroke}{rgb}{0.000000,0.000000,0.000000}%
\pgfsetstrokecolor{currentstroke}%
\pgfsetdash{}{0pt}%
\pgfpathmoveto{\pgfqpoint{3.687906in}{3.094614in}}%
\pgfpathlineto{\pgfqpoint{3.701217in}{3.082061in}}%
\pgfpathlineto{\pgfqpoint{3.714528in}{3.069758in}}%
\pgfpathlineto{\pgfqpoint{3.727839in}{3.057702in}}%
\pgfpathlineto{\pgfqpoint{3.741149in}{3.045893in}}%
\pgfpathlineto{\pgfqpoint{3.748911in}{3.062255in}}%
\pgfpathlineto{\pgfqpoint{3.756669in}{3.078822in}}%
\pgfpathlineto{\pgfqpoint{3.764422in}{3.095596in}}%
\pgfpathlineto{\pgfqpoint{3.772172in}{3.112585in}}%
\pgfpathlineto{\pgfqpoint{3.758866in}{3.124804in}}%
\pgfpathlineto{\pgfqpoint{3.745561in}{3.137270in}}%
\pgfpathlineto{\pgfqpoint{3.732256in}{3.149984in}}%
\pgfpathlineto{\pgfqpoint{3.718950in}{3.162948in}}%
\pgfpathlineto{\pgfqpoint{3.711195in}{3.145537in}}%
\pgfpathlineto{\pgfqpoint{3.703437in}{3.128347in}}%
\pgfpathlineto{\pgfqpoint{3.695674in}{3.111375in}}%
\pgfpathlineto{\pgfqpoint{3.687906in}{3.094614in}}%
\pgfpathclose%
\pgfusepath{fill}%
\end{pgfscope}%
\begin{pgfscope}%
\pgfpathrectangle{\pgfqpoint{1.150000in}{0.150000in}}{\pgfqpoint{5.700000in}{5.700000in}}%
\pgfusepath{clip}%
\pgfsetbuttcap%
\pgfsetroundjoin%
\definecolor{currentfill}{rgb}{0.166617,0.463708,0.558119}%
\pgfsetfillcolor{currentfill}%
\pgfsetfillopacity{0.800000}%
\pgfsetlinewidth{0.000000pt}%
\definecolor{currentstroke}{rgb}{0.000000,0.000000,0.000000}%
\pgfsetstrokecolor{currentstroke}%
\pgfsetdash{}{0pt}%
\pgfpathmoveto{\pgfqpoint{3.367806in}{3.475853in}}%
\pgfpathlineto{\pgfqpoint{3.381189in}{3.456701in}}%
\pgfpathlineto{\pgfqpoint{3.394566in}{3.437851in}}%
\pgfpathlineto{\pgfqpoint{3.407937in}{3.419300in}}%
\pgfpathlineto{\pgfqpoint{3.421303in}{3.401046in}}%
\pgfpathlineto{\pgfqpoint{3.429095in}{3.419819in}}%
\pgfpathlineto{\pgfqpoint{3.436881in}{3.438846in}}%
\pgfpathlineto{\pgfqpoint{3.444662in}{3.458131in}}%
\pgfpathlineto{\pgfqpoint{3.452438in}{3.477678in}}%
\pgfpathlineto{\pgfqpoint{3.439076in}{3.496353in}}%
\pgfpathlineto{\pgfqpoint{3.425708in}{3.515325in}}%
\pgfpathlineto{\pgfqpoint{3.412335in}{3.534596in}}%
\pgfpathlineto{\pgfqpoint{3.398957in}{3.554170in}}%
\pgfpathlineto{\pgfqpoint{3.391177in}{3.534189in}}%
\pgfpathlineto{\pgfqpoint{3.383392in}{3.514479in}}%
\pgfpathlineto{\pgfqpoint{3.375602in}{3.495035in}}%
\pgfpathlineto{\pgfqpoint{3.367806in}{3.475853in}}%
\pgfpathclose%
\pgfusepath{fill}%
\end{pgfscope}%
\begin{pgfscope}%
\pgfpathrectangle{\pgfqpoint{1.150000in}{0.150000in}}{\pgfqpoint{5.700000in}{5.700000in}}%
\pgfusepath{clip}%
\pgfsetbuttcap%
\pgfsetroundjoin%
\definecolor{currentfill}{rgb}{0.180629,0.429975,0.557282}%
\pgfsetfillcolor{currentfill}%
\pgfsetfillopacity{0.800000}%
\pgfsetlinewidth{0.000000pt}%
\definecolor{currentstroke}{rgb}{0.000000,0.000000,0.000000}%
\pgfsetstrokecolor{currentstroke}%
\pgfsetdash{}{0pt}%
\pgfpathmoveto{\pgfqpoint{4.688840in}{3.347838in}}%
\pgfpathlineto{\pgfqpoint{4.702269in}{3.341086in}}%
\pgfpathlineto{\pgfqpoint{4.715704in}{3.334526in}}%
\pgfpathlineto{\pgfqpoint{4.729146in}{3.328157in}}%
\pgfpathlineto{\pgfqpoint{4.742595in}{3.321980in}}%
\pgfpathlineto{\pgfqpoint{4.750210in}{3.341882in}}%
\pgfpathlineto{\pgfqpoint{4.757828in}{3.362147in}}%
\pgfpathlineto{\pgfqpoint{4.765448in}{3.382784in}}%
\pgfpathlineto{\pgfqpoint{4.773072in}{3.403801in}}%
\pgfpathlineto{\pgfqpoint{4.759634in}{3.410700in}}%
\pgfpathlineto{\pgfqpoint{4.746202in}{3.417790in}}%
\pgfpathlineto{\pgfqpoint{4.732777in}{3.425072in}}%
\pgfpathlineto{\pgfqpoint{4.719358in}{3.432546in}}%
\pgfpathlineto{\pgfqpoint{4.711724in}{3.410795in}}%
\pgfpathlineto{\pgfqpoint{4.704093in}{3.389433in}}%
\pgfpathlineto{\pgfqpoint{4.696465in}{3.368450in}}%
\pgfpathlineto{\pgfqpoint{4.688840in}{3.347838in}}%
\pgfpathclose%
\pgfusepath{fill}%
\end{pgfscope}%
\begin{pgfscope}%
\pgfpathrectangle{\pgfqpoint{1.150000in}{0.150000in}}{\pgfqpoint{5.700000in}{5.700000in}}%
\pgfusepath{clip}%
\pgfsetbuttcap%
\pgfsetroundjoin%
\definecolor{currentfill}{rgb}{0.233603,0.313828,0.543914}%
\pgfsetfillcolor{currentfill}%
\pgfsetfillopacity{0.800000}%
\pgfsetlinewidth{0.000000pt}%
\definecolor{currentstroke}{rgb}{0.000000,0.000000,0.000000}%
\pgfsetstrokecolor{currentstroke}%
\pgfsetdash{}{0pt}%
\pgfpathmoveto{\pgfqpoint{4.100210in}{3.043604in}}%
\pgfpathlineto{\pgfqpoint{4.113546in}{3.035312in}}%
\pgfpathlineto{\pgfqpoint{4.126885in}{3.027237in}}%
\pgfpathlineto{\pgfqpoint{4.140228in}{3.019377in}}%
\pgfpathlineto{\pgfqpoint{4.153575in}{3.011731in}}%
\pgfpathlineto{\pgfqpoint{4.161260in}{3.027932in}}%
\pgfpathlineto{\pgfqpoint{4.168941in}{3.044353in}}%
\pgfpathlineto{\pgfqpoint{4.176620in}{3.060999in}}%
\pgfpathlineto{\pgfqpoint{4.184297in}{3.077877in}}%
\pgfpathlineto{\pgfqpoint{4.170956in}{3.086024in}}%
\pgfpathlineto{\pgfqpoint{4.157619in}{3.094384in}}%
\pgfpathlineto{\pgfqpoint{4.144287in}{3.102961in}}%
\pgfpathlineto{\pgfqpoint{4.130957in}{3.111754in}}%
\pgfpathlineto{\pgfqpoint{4.123274in}{3.094363in}}%
\pgfpathlineto{\pgfqpoint{4.115589in}{3.077211in}}%
\pgfpathlineto{\pgfqpoint{4.107901in}{3.060294in}}%
\pgfpathlineto{\pgfqpoint{4.100210in}{3.043604in}}%
\pgfpathclose%
\pgfusepath{fill}%
\end{pgfscope}%
\begin{pgfscope}%
\pgfpathrectangle{\pgfqpoint{1.150000in}{0.150000in}}{\pgfqpoint{5.700000in}{5.700000in}}%
\pgfusepath{clip}%
\pgfsetbuttcap%
\pgfsetroundjoin%
\definecolor{currentfill}{rgb}{0.360741,0.785964,0.387814}%
\pgfsetfillcolor{currentfill}%
\pgfsetfillopacity{0.800000}%
\pgfsetlinewidth{0.000000pt}%
\definecolor{currentstroke}{rgb}{0.000000,0.000000,0.000000}%
\pgfsetstrokecolor{currentstroke}%
\pgfsetdash{}{0pt}%
\pgfpathmoveto{\pgfqpoint{3.507534in}{4.447369in}}%
\pgfpathlineto{\pgfqpoint{3.520976in}{4.421262in}}%
\pgfpathlineto{\pgfqpoint{3.534409in}{4.395490in}}%
\pgfpathlineto{\pgfqpoint{3.547834in}{4.370049in}}%
\pgfpathlineto{\pgfqpoint{3.561251in}{4.344937in}}%
\pgfpathlineto{\pgfqpoint{3.568900in}{4.376170in}}%
\pgfpathlineto{\pgfqpoint{3.576546in}{4.407886in}}%
\pgfpathlineto{\pgfqpoint{3.584190in}{4.440095in}}%
\pgfpathlineto{\pgfqpoint{3.591830in}{4.472804in}}%
\pgfpathlineto{\pgfqpoint{3.578407in}{4.498639in}}%
\pgfpathlineto{\pgfqpoint{3.564976in}{4.524803in}}%
\pgfpathlineto{\pgfqpoint{3.551536in}{4.551301in}}%
\pgfpathlineto{\pgfqpoint{3.538088in}{4.578136in}}%
\pgfpathlineto{\pgfqpoint{3.530455in}{4.544686in}}%
\pgfpathlineto{\pgfqpoint{3.522818in}{4.511748in}}%
\pgfpathlineto{\pgfqpoint{3.515178in}{4.479312in}}%
\pgfpathlineto{\pgfqpoint{3.507534in}{4.447369in}}%
\pgfpathclose%
\pgfusepath{fill}%
\end{pgfscope}%
\begin{pgfscope}%
\pgfpathrectangle{\pgfqpoint{1.150000in}{0.150000in}}{\pgfqpoint{5.700000in}{5.700000in}}%
\pgfusepath{clip}%
\pgfsetbuttcap%
\pgfsetroundjoin%
\definecolor{currentfill}{rgb}{0.239346,0.300855,0.540844}%
\pgfsetfillcolor{currentfill}%
\pgfsetfillopacity{0.800000}%
\pgfsetlinewidth{0.000000pt}%
\definecolor{currentstroke}{rgb}{0.000000,0.000000,0.000000}%
\pgfsetstrokecolor{currentstroke}%
\pgfsetdash{}{0pt}%
\pgfpathmoveto{\pgfqpoint{3.878639in}{3.023511in}}%
\pgfpathlineto{\pgfqpoint{3.891954in}{3.013439in}}%
\pgfpathlineto{\pgfqpoint{3.905270in}{3.003598in}}%
\pgfpathlineto{\pgfqpoint{3.918588in}{2.993987in}}%
\pgfpathlineto{\pgfqpoint{3.931909in}{2.984604in}}%
\pgfpathlineto{\pgfqpoint{3.939637in}{3.000526in}}%
\pgfpathlineto{\pgfqpoint{3.947362in}{3.016647in}}%
\pgfpathlineto{\pgfqpoint{3.955084in}{3.032974in}}%
\pgfpathlineto{\pgfqpoint{3.962802in}{3.049511in}}%
\pgfpathlineto{\pgfqpoint{3.949487in}{3.059333in}}%
\pgfpathlineto{\pgfqpoint{3.936175in}{3.069383in}}%
\pgfpathlineto{\pgfqpoint{3.922864in}{3.079664in}}%
\pgfpathlineto{\pgfqpoint{3.909556in}{3.090176in}}%
\pgfpathlineto{\pgfqpoint{3.901832in}{3.073188in}}%
\pgfpathlineto{\pgfqpoint{3.894105in}{3.056418in}}%
\pgfpathlineto{\pgfqpoint{3.886374in}{3.039861in}}%
\pgfpathlineto{\pgfqpoint{3.878639in}{3.023511in}}%
\pgfpathclose%
\pgfusepath{fill}%
\end{pgfscope}%
\begin{pgfscope}%
\pgfpathrectangle{\pgfqpoint{1.150000in}{0.150000in}}{\pgfqpoint{5.700000in}{5.700000in}}%
\pgfusepath{clip}%
\pgfsetbuttcap%
\pgfsetroundjoin%
\definecolor{currentfill}{rgb}{0.146616,0.673050,0.508936}%
\pgfsetfillcolor{currentfill}%
\pgfsetfillopacity{0.800000}%
\pgfsetlinewidth{0.000000pt}%
\definecolor{currentstroke}{rgb}{0.000000,0.000000,0.000000}%
\pgfsetstrokecolor{currentstroke}%
\pgfsetdash{}{0pt}%
\pgfpathmoveto{\pgfqpoint{3.330873in}{4.094535in}}%
\pgfpathlineto{\pgfqpoint{3.344338in}{4.069594in}}%
\pgfpathlineto{\pgfqpoint{3.357793in}{4.044995in}}%
\pgfpathlineto{\pgfqpoint{3.371239in}{4.020735in}}%
\pgfpathlineto{\pgfqpoint{3.384677in}{3.996810in}}%
\pgfpathlineto{\pgfqpoint{3.392390in}{4.021935in}}%
\pgfpathlineto{\pgfqpoint{3.400099in}{4.047431in}}%
\pgfpathlineto{\pgfqpoint{3.407802in}{4.073305in}}%
\pgfpathlineto{\pgfqpoint{3.415501in}{4.099565in}}%
\pgfpathlineto{\pgfqpoint{3.402062in}{4.124062in}}%
\pgfpathlineto{\pgfqpoint{3.388614in}{4.148896in}}%
\pgfpathlineto{\pgfqpoint{3.375158in}{4.174070in}}%
\pgfpathlineto{\pgfqpoint{3.361692in}{4.199588in}}%
\pgfpathlineto{\pgfqpoint{3.353995in}{4.172740in}}%
\pgfpathlineto{\pgfqpoint{3.346293in}{4.146286in}}%
\pgfpathlineto{\pgfqpoint{3.338586in}{4.120220in}}%
\pgfpathlineto{\pgfqpoint{3.330873in}{4.094535in}}%
\pgfpathclose%
\pgfusepath{fill}%
\end{pgfscope}%
\begin{pgfscope}%
\pgfpathrectangle{\pgfqpoint{1.150000in}{0.150000in}}{\pgfqpoint{5.700000in}{5.700000in}}%
\pgfusepath{clip}%
\pgfsetbuttcap%
\pgfsetroundjoin%
\definecolor{currentfill}{rgb}{0.191090,0.708366,0.482284}%
\pgfsetfillcolor{currentfill}%
\pgfsetfillopacity{0.800000}%
\pgfsetlinewidth{0.000000pt}%
\definecolor{currentstroke}{rgb}{0.000000,0.000000,0.000000}%
\pgfsetstrokecolor{currentstroke}%
\pgfsetdash{}{0pt}%
\pgfpathmoveto{\pgfqpoint{3.361692in}{4.199588in}}%
\pgfpathlineto{\pgfqpoint{3.375158in}{4.174070in}}%
\pgfpathlineto{\pgfqpoint{3.388614in}{4.148896in}}%
\pgfpathlineto{\pgfqpoint{3.402062in}{4.124062in}}%
\pgfpathlineto{\pgfqpoint{3.415501in}{4.099565in}}%
\pgfpathlineto{\pgfqpoint{3.423195in}{4.126217in}}%
\pgfpathlineto{\pgfqpoint{3.430884in}{4.153267in}}%
\pgfpathlineto{\pgfqpoint{3.438568in}{4.180724in}}%
\pgfpathlineto{\pgfqpoint{3.446248in}{4.208595in}}%
\pgfpathlineto{\pgfqpoint{3.432806in}{4.233703in}}%
\pgfpathlineto{\pgfqpoint{3.419356in}{4.259148in}}%
\pgfpathlineto{\pgfqpoint{3.405897in}{4.284935in}}%
\pgfpathlineto{\pgfqpoint{3.392429in}{4.311068in}}%
\pgfpathlineto{\pgfqpoint{3.384752in}{4.282570in}}%
\pgfpathlineto{\pgfqpoint{3.377070in}{4.254496in}}%
\pgfpathlineto{\pgfqpoint{3.369384in}{4.226837in}}%
\pgfpathlineto{\pgfqpoint{3.361692in}{4.199588in}}%
\pgfpathclose%
\pgfusepath{fill}%
\end{pgfscope}%
\begin{pgfscope}%
\pgfpathrectangle{\pgfqpoint{1.150000in}{0.150000in}}{\pgfqpoint{5.700000in}{5.700000in}}%
\pgfusepath{clip}%
\pgfsetbuttcap%
\pgfsetroundjoin%
\definecolor{currentfill}{rgb}{0.235526,0.309527,0.542944}%
\pgfsetfillcolor{currentfill}%
\pgfsetfillopacity{0.800000}%
\pgfsetlinewidth{0.000000pt}%
\definecolor{currentstroke}{rgb}{0.000000,0.000000,0.000000}%
\pgfsetstrokecolor{currentstroke}%
\pgfsetdash{}{0pt}%
\pgfpathmoveto{\pgfqpoint{3.741149in}{3.045893in}}%
\pgfpathlineto{\pgfqpoint{3.754461in}{3.034328in}}%
\pgfpathlineto{\pgfqpoint{3.767772in}{3.023006in}}%
\pgfpathlineto{\pgfqpoint{3.781084in}{3.011926in}}%
\pgfpathlineto{\pgfqpoint{3.794397in}{3.001086in}}%
\pgfpathlineto{\pgfqpoint{3.802154in}{3.017051in}}%
\pgfpathlineto{\pgfqpoint{3.809906in}{3.033211in}}%
\pgfpathlineto{\pgfqpoint{3.817654in}{3.049573in}}%
\pgfpathlineto{\pgfqpoint{3.825398in}{3.066140in}}%
\pgfpathlineto{\pgfqpoint{3.812090in}{3.077390in}}%
\pgfpathlineto{\pgfqpoint{3.798784in}{3.088879in}}%
\pgfpathlineto{\pgfqpoint{3.785477in}{3.100611in}}%
\pgfpathlineto{\pgfqpoint{3.772172in}{3.112585in}}%
\pgfpathlineto{\pgfqpoint{3.764422in}{3.095596in}}%
\pgfpathlineto{\pgfqpoint{3.756669in}{3.078822in}}%
\pgfpathlineto{\pgfqpoint{3.748911in}{3.062255in}}%
\pgfpathlineto{\pgfqpoint{3.741149in}{3.045893in}}%
\pgfpathclose%
\pgfusepath{fill}%
\end{pgfscope}%
\begin{pgfscope}%
\pgfpathrectangle{\pgfqpoint{1.150000in}{0.150000in}}{\pgfqpoint{5.700000in}{5.700000in}}%
\pgfusepath{clip}%
\pgfsetbuttcap%
\pgfsetroundjoin%
\definecolor{currentfill}{rgb}{0.133743,0.548535,0.553541}%
\pgfsetfillcolor{currentfill}%
\pgfsetfillopacity{0.800000}%
\pgfsetlinewidth{0.000000pt}%
\definecolor{currentstroke}{rgb}{0.000000,0.000000,0.000000}%
\pgfsetstrokecolor{currentstroke}%
\pgfsetdash{}{0pt}%
\pgfpathmoveto{\pgfqpoint{3.291691in}{3.721995in}}%
\pgfpathlineto{\pgfqpoint{3.305125in}{3.699897in}}%
\pgfpathlineto{\pgfqpoint{3.318551in}{3.678126in}}%
\pgfpathlineto{\pgfqpoint{3.331969in}{3.656677in}}%
\pgfpathlineto{\pgfqpoint{3.345380in}{3.635548in}}%
\pgfpathlineto{\pgfqpoint{3.353157in}{3.656251in}}%
\pgfpathlineto{\pgfqpoint{3.360929in}{3.677245in}}%
\pgfpathlineto{\pgfqpoint{3.368694in}{3.698534in}}%
\pgfpathlineto{\pgfqpoint{3.376455in}{3.720125in}}%
\pgfpathlineto{\pgfqpoint{3.363047in}{3.741714in}}%
\pgfpathlineto{\pgfqpoint{3.349631in}{3.763624in}}%
\pgfpathlineto{\pgfqpoint{3.336208in}{3.785857in}}%
\pgfpathlineto{\pgfqpoint{3.322777in}{3.808417in}}%
\pgfpathlineto{\pgfqpoint{3.315014in}{3.786352in}}%
\pgfpathlineto{\pgfqpoint{3.307245in}{3.764597in}}%
\pgfpathlineto{\pgfqpoint{3.299471in}{3.743146in}}%
\pgfpathlineto{\pgfqpoint{3.291691in}{3.721995in}}%
\pgfpathclose%
\pgfusepath{fill}%
\end{pgfscope}%
\begin{pgfscope}%
\pgfpathrectangle{\pgfqpoint{1.150000in}{0.150000in}}{\pgfqpoint{5.700000in}{5.700000in}}%
\pgfusepath{clip}%
\pgfsetbuttcap%
\pgfsetroundjoin%
\definecolor{currentfill}{rgb}{0.208623,0.367752,0.552675}%
\pgfsetfillcolor{currentfill}%
\pgfsetfillopacity{0.800000}%
\pgfsetlinewidth{0.000000pt}%
\definecolor{currentstroke}{rgb}{0.000000,0.000000,0.000000}%
\pgfsetstrokecolor{currentstroke}%
\pgfsetdash{}{0pt}%
\pgfpathmoveto{\pgfqpoint{3.496897in}{3.195805in}}%
\pgfpathlineto{\pgfqpoint{3.510232in}{3.180493in}}%
\pgfpathlineto{\pgfqpoint{3.523566in}{3.165453in}}%
\pgfpathlineto{\pgfqpoint{3.536896in}{3.150684in}}%
\pgfpathlineto{\pgfqpoint{3.550224in}{3.136183in}}%
\pgfpathlineto{\pgfqpoint{3.558021in}{3.152879in}}%
\pgfpathlineto{\pgfqpoint{3.565814in}{3.169784in}}%
\pgfpathlineto{\pgfqpoint{3.573601in}{3.186904in}}%
\pgfpathlineto{\pgfqpoint{3.581383in}{3.204241in}}%
\pgfpathlineto{\pgfqpoint{3.568061in}{3.219124in}}%
\pgfpathlineto{\pgfqpoint{3.554735in}{3.234275in}}%
\pgfpathlineto{\pgfqpoint{3.541408in}{3.249697in}}%
\pgfpathlineto{\pgfqpoint{3.528077in}{3.265392in}}%
\pgfpathlineto{\pgfqpoint{3.520290in}{3.247659in}}%
\pgfpathlineto{\pgfqpoint{3.512497in}{3.230154in}}%
\pgfpathlineto{\pgfqpoint{3.504699in}{3.212871in}}%
\pgfpathlineto{\pgfqpoint{3.496897in}{3.195805in}}%
\pgfpathclose%
\pgfusepath{fill}%
\end{pgfscope}%
\begin{pgfscope}%
\pgfpathrectangle{\pgfqpoint{1.150000in}{0.150000in}}{\pgfqpoint{5.700000in}{5.700000in}}%
\pgfusepath{clip}%
\pgfsetbuttcap%
\pgfsetroundjoin%
\definecolor{currentfill}{rgb}{0.124780,0.640461,0.527068}%
\pgfsetfillcolor{currentfill}%
\pgfsetfillopacity{0.800000}%
\pgfsetlinewidth{0.000000pt}%
\definecolor{currentstroke}{rgb}{0.000000,0.000000,0.000000}%
\pgfsetstrokecolor{currentstroke}%
\pgfsetdash{}{0pt}%
\pgfpathmoveto{\pgfqpoint{3.299967in}{3.995472in}}%
\pgfpathlineto{\pgfqpoint{3.313431in}{3.971070in}}%
\pgfpathlineto{\pgfqpoint{3.326886in}{3.947009in}}%
\pgfpathlineto{\pgfqpoint{3.340332in}{3.923286in}}%
\pgfpathlineto{\pgfqpoint{3.353770in}{3.899896in}}%
\pgfpathlineto{\pgfqpoint{3.361505in}{3.923600in}}%
\pgfpathlineto{\pgfqpoint{3.369234in}{3.947649in}}%
\pgfpathlineto{\pgfqpoint{3.376958in}{3.972050in}}%
\pgfpathlineto{\pgfqpoint{3.384677in}{3.996810in}}%
\pgfpathlineto{\pgfqpoint{3.371239in}{4.020735in}}%
\pgfpathlineto{\pgfqpoint{3.357793in}{4.044995in}}%
\pgfpathlineto{\pgfqpoint{3.344338in}{4.069594in}}%
\pgfpathlineto{\pgfqpoint{3.330873in}{4.094535in}}%
\pgfpathlineto{\pgfqpoint{3.323155in}{4.069224in}}%
\pgfpathlineto{\pgfqpoint{3.315431in}{4.044281in}}%
\pgfpathlineto{\pgfqpoint{3.307702in}{4.019699in}}%
\pgfpathlineto{\pgfqpoint{3.299967in}{3.995472in}}%
\pgfpathclose%
\pgfusepath{fill}%
\end{pgfscope}%
\begin{pgfscope}%
\pgfpathrectangle{\pgfqpoint{1.150000in}{0.150000in}}{\pgfqpoint{5.700000in}{5.700000in}}%
\pgfusepath{clip}%
\pgfsetbuttcap%
\pgfsetroundjoin%
\definecolor{currentfill}{rgb}{0.515992,0.831158,0.294279}%
\pgfsetfillcolor{currentfill}%
\pgfsetfillopacity{0.800000}%
\pgfsetlinewidth{0.000000pt}%
\definecolor{currentstroke}{rgb}{0.000000,0.000000,0.000000}%
\pgfsetstrokecolor{currentstroke}%
\pgfsetdash{}{0pt}%
\pgfpathmoveto{\pgfqpoint{3.706547in}{4.647267in}}%
\pgfpathlineto{\pgfqpoint{3.719948in}{4.621486in}}%
\pgfpathlineto{\pgfqpoint{3.733342in}{4.596020in}}%
\pgfpathlineto{\pgfqpoint{3.746730in}{4.570866in}}%
\pgfpathlineto{\pgfqpoint{3.760110in}{4.546023in}}%
\pgfpathlineto{\pgfqpoint{3.767750in}{4.582010in}}%
\pgfpathlineto{\pgfqpoint{3.775390in}{4.618571in}}%
\pgfpathlineto{\pgfqpoint{3.783029in}{4.655718in}}%
\pgfpathlineto{\pgfqpoint{3.769641in}{4.681182in}}%
\pgfpathlineto{\pgfqpoint{3.756247in}{4.706957in}}%
\pgfpathlineto{\pgfqpoint{3.742846in}{4.733046in}}%
\pgfpathlineto{\pgfqpoint{3.729438in}{4.759453in}}%
\pgfpathlineto{\pgfqpoint{3.721809in}{4.721465in}}%
\pgfpathlineto{\pgfqpoint{3.714178in}{4.684073in}}%
\pgfpathlineto{\pgfqpoint{3.706547in}{4.647267in}}%
\pgfpathclose%
\pgfusepath{fill}%
\end{pgfscope}%
\begin{pgfscope}%
\pgfpathrectangle{\pgfqpoint{1.150000in}{0.150000in}}{\pgfqpoint{5.700000in}{5.700000in}}%
\pgfusepath{clip}%
\pgfsetbuttcap%
\pgfsetroundjoin%
\definecolor{currentfill}{rgb}{0.239346,0.300855,0.540844}%
\pgfsetfillcolor{currentfill}%
\pgfsetfillopacity{0.800000}%
\pgfsetlinewidth{0.000000pt}%
\definecolor{currentstroke}{rgb}{0.000000,0.000000,0.000000}%
\pgfsetstrokecolor{currentstroke}%
\pgfsetdash{}{0pt}%
\pgfpathmoveto{\pgfqpoint{4.016086in}{3.012482in}}%
\pgfpathlineto{\pgfqpoint{4.029414in}{3.003783in}}%
\pgfpathlineto{\pgfqpoint{4.042745in}{2.995305in}}%
\pgfpathlineto{\pgfqpoint{4.056080in}{2.987047in}}%
\pgfpathlineto{\pgfqpoint{4.069418in}{2.979008in}}%
\pgfpathlineto{\pgfqpoint{4.077121in}{2.994843in}}%
\pgfpathlineto{\pgfqpoint{4.084820in}{3.010884in}}%
\pgfpathlineto{\pgfqpoint{4.092517in}{3.027136in}}%
\pgfpathlineto{\pgfqpoint{4.100210in}{3.043604in}}%
\pgfpathlineto{\pgfqpoint{4.086878in}{3.052113in}}%
\pgfpathlineto{\pgfqpoint{4.073550in}{3.060841in}}%
\pgfpathlineto{\pgfqpoint{4.060225in}{3.069789in}}%
\pgfpathlineto{\pgfqpoint{4.046902in}{3.078957in}}%
\pgfpathlineto{\pgfqpoint{4.039203in}{3.062008in}}%
\pgfpathlineto{\pgfqpoint{4.031500in}{3.045282in}}%
\pgfpathlineto{\pgfqpoint{4.023795in}{3.028776in}}%
\pgfpathlineto{\pgfqpoint{4.016086in}{3.012482in}}%
\pgfpathclose%
\pgfusepath{fill}%
\end{pgfscope}%
\begin{pgfscope}%
\pgfpathrectangle{\pgfqpoint{1.150000in}{0.150000in}}{\pgfqpoint{5.700000in}{5.700000in}}%
\pgfusepath{clip}%
\pgfsetbuttcap%
\pgfsetroundjoin%
\definecolor{currentfill}{rgb}{0.197636,0.391528,0.554969}%
\pgfsetfillcolor{currentfill}%
\pgfsetfillopacity{0.800000}%
\pgfsetlinewidth{0.000000pt}%
\definecolor{currentstroke}{rgb}{0.000000,0.000000,0.000000}%
\pgfsetstrokecolor{currentstroke}%
\pgfsetdash{}{0pt}%
\pgfpathmoveto{\pgfqpoint{3.443521in}{3.259823in}}%
\pgfpathlineto{\pgfqpoint{3.456870in}{3.243399in}}%
\pgfpathlineto{\pgfqpoint{3.470216in}{3.227256in}}%
\pgfpathlineto{\pgfqpoint{3.483558in}{3.211392in}}%
\pgfpathlineto{\pgfqpoint{3.496897in}{3.195805in}}%
\pgfpathlineto{\pgfqpoint{3.504699in}{3.212871in}}%
\pgfpathlineto{\pgfqpoint{3.512497in}{3.230154in}}%
\pgfpathlineto{\pgfqpoint{3.520290in}{3.247659in}}%
\pgfpathlineto{\pgfqpoint{3.528077in}{3.265392in}}%
\pgfpathlineto{\pgfqpoint{3.514743in}{3.281362in}}%
\pgfpathlineto{\pgfqpoint{3.501406in}{3.297609in}}%
\pgfpathlineto{\pgfqpoint{3.488066in}{3.314136in}}%
\pgfpathlineto{\pgfqpoint{3.474722in}{3.330944in}}%
\pgfpathlineto{\pgfqpoint{3.466930in}{3.312816in}}%
\pgfpathlineto{\pgfqpoint{3.459132in}{3.294923in}}%
\pgfpathlineto{\pgfqpoint{3.451329in}{3.277260in}}%
\pgfpathlineto{\pgfqpoint{3.443521in}{3.259823in}}%
\pgfpathclose%
\pgfusepath{fill}%
\end{pgfscope}%
\begin{pgfscope}%
\pgfpathrectangle{\pgfqpoint{1.150000in}{0.150000in}}{\pgfqpoint{5.700000in}{5.700000in}}%
\pgfusepath{clip}%
\pgfsetbuttcap%
\pgfsetroundjoin%
\definecolor{currentfill}{rgb}{0.259857,0.745492,0.444467}%
\pgfsetfillcolor{currentfill}%
\pgfsetfillopacity{0.800000}%
\pgfsetlinewidth{0.000000pt}%
\definecolor{currentstroke}{rgb}{0.000000,0.000000,0.000000}%
\pgfsetstrokecolor{currentstroke}%
\pgfsetdash{}{0pt}%
\pgfpathmoveto{\pgfqpoint{3.392429in}{4.311068in}}%
\pgfpathlineto{\pgfqpoint{3.405897in}{4.284935in}}%
\pgfpathlineto{\pgfqpoint{3.419356in}{4.259148in}}%
\pgfpathlineto{\pgfqpoint{3.432806in}{4.233703in}}%
\pgfpathlineto{\pgfqpoint{3.446248in}{4.208595in}}%
\pgfpathlineto{\pgfqpoint{3.453923in}{4.236887in}}%
\pgfpathlineto{\pgfqpoint{3.461594in}{4.265607in}}%
\pgfpathlineto{\pgfqpoint{3.469261in}{4.294764in}}%
\pgfpathlineto{\pgfqpoint{3.476923in}{4.324365in}}%
\pgfpathlineto{\pgfqpoint{3.463478in}{4.350121in}}%
\pgfpathlineto{\pgfqpoint{3.450024in}{4.376217in}}%
\pgfpathlineto{\pgfqpoint{3.436561in}{4.402657in}}%
\pgfpathlineto{\pgfqpoint{3.423088in}{4.429443in}}%
\pgfpathlineto{\pgfqpoint{3.415430in}{4.399176in}}%
\pgfpathlineto{\pgfqpoint{3.407768in}{4.369363in}}%
\pgfpathlineto{\pgfqpoint{3.400101in}{4.339996in}}%
\pgfpathlineto{\pgfqpoint{3.392429in}{4.311068in}}%
\pgfpathclose%
\pgfusepath{fill}%
\end{pgfscope}%
\begin{pgfscope}%
\pgfpathrectangle{\pgfqpoint{1.150000in}{0.150000in}}{\pgfqpoint{5.700000in}{5.700000in}}%
\pgfusepath{clip}%
\pgfsetbuttcap%
\pgfsetroundjoin%
\definecolor{currentfill}{rgb}{0.156270,0.489624,0.557936}%
\pgfsetfillcolor{currentfill}%
\pgfsetfillopacity{0.800000}%
\pgfsetlinewidth{0.000000pt}%
\definecolor{currentstroke}{rgb}{0.000000,0.000000,0.000000}%
\pgfsetstrokecolor{currentstroke}%
\pgfsetdash{}{0pt}%
\pgfpathmoveto{\pgfqpoint{3.314214in}{3.555536in}}%
\pgfpathlineto{\pgfqpoint{3.327622in}{3.535149in}}%
\pgfpathlineto{\pgfqpoint{3.341023in}{3.515074in}}%
\pgfpathlineto{\pgfqpoint{3.354418in}{3.495310in}}%
\pgfpathlineto{\pgfqpoint{3.367806in}{3.475853in}}%
\pgfpathlineto{\pgfqpoint{3.375602in}{3.495035in}}%
\pgfpathlineto{\pgfqpoint{3.383392in}{3.514479in}}%
\pgfpathlineto{\pgfqpoint{3.391177in}{3.534189in}}%
\pgfpathlineto{\pgfqpoint{3.398957in}{3.554170in}}%
\pgfpathlineto{\pgfqpoint{3.385572in}{3.574050in}}%
\pgfpathlineto{\pgfqpoint{3.372181in}{3.594237in}}%
\pgfpathlineto{\pgfqpoint{3.358784in}{3.614736in}}%
\pgfpathlineto{\pgfqpoint{3.345380in}{3.635548in}}%
\pgfpathlineto{\pgfqpoint{3.337597in}{3.615130in}}%
\pgfpathlineto{\pgfqpoint{3.329809in}{3.594992in}}%
\pgfpathlineto{\pgfqpoint{3.322014in}{3.575129in}}%
\pgfpathlineto{\pgfqpoint{3.314214in}{3.555536in}}%
\pgfpathclose%
\pgfusepath{fill}%
\end{pgfscope}%
\begin{pgfscope}%
\pgfpathrectangle{\pgfqpoint{1.150000in}{0.150000in}}{\pgfqpoint{5.700000in}{5.700000in}}%
\pgfusepath{clip}%
\pgfsetbuttcap%
\pgfsetroundjoin%
\definecolor{currentfill}{rgb}{0.174274,0.445044,0.557792}%
\pgfsetfillcolor{currentfill}%
\pgfsetfillopacity{0.800000}%
\pgfsetlinewidth{0.000000pt}%
\definecolor{currentstroke}{rgb}{0.000000,0.000000,0.000000}%
\pgfsetstrokecolor{currentstroke}%
\pgfsetdash{}{0pt}%
\pgfpathmoveto{\pgfqpoint{4.773072in}{3.403801in}}%
\pgfpathlineto{\pgfqpoint{4.786518in}{3.397092in}}%
\pgfpathlineto{\pgfqpoint{4.799970in}{3.390573in}}%
\pgfpathlineto{\pgfqpoint{4.813429in}{3.384242in}}%
\pgfpathlineto{\pgfqpoint{4.826896in}{3.378099in}}%
\pgfpathlineto{\pgfqpoint{4.834512in}{3.398763in}}%
\pgfpathlineto{\pgfqpoint{4.842133in}{3.419817in}}%
\pgfpathlineto{\pgfqpoint{4.849758in}{3.441269in}}%
\pgfpathlineto{\pgfqpoint{4.836300in}{3.447974in}}%
\pgfpathlineto{\pgfqpoint{4.822848in}{3.454866in}}%
\pgfpathlineto{\pgfqpoint{4.809404in}{3.461948in}}%
\pgfpathlineto{\pgfqpoint{4.795967in}{3.469219in}}%
\pgfpathlineto{\pgfqpoint{4.788331in}{3.447010in}}%
\pgfpathlineto{\pgfqpoint{4.780700in}{3.425207in}}%
\pgfpathlineto{\pgfqpoint{4.773072in}{3.403801in}}%
\pgfpathclose%
\pgfusepath{fill}%
\end{pgfscope}%
\begin{pgfscope}%
\pgfpathrectangle{\pgfqpoint{1.150000in}{0.150000in}}{\pgfqpoint{5.700000in}{5.700000in}}%
\pgfusepath{clip}%
\pgfsetbuttcap%
\pgfsetroundjoin%
\definecolor{currentfill}{rgb}{0.220057,0.343307,0.549413}%
\pgfsetfillcolor{currentfill}%
\pgfsetfillopacity{0.800000}%
\pgfsetlinewidth{0.000000pt}%
\definecolor{currentstroke}{rgb}{0.000000,0.000000,0.000000}%
\pgfsetstrokecolor{currentstroke}%
\pgfsetdash{}{0pt}%
\pgfpathmoveto{\pgfqpoint{3.550224in}{3.136183in}}%
\pgfpathlineto{\pgfqpoint{3.563550in}{3.121948in}}%
\pgfpathlineto{\pgfqpoint{3.576875in}{3.107977in}}%
\pgfpathlineto{\pgfqpoint{3.590197in}{3.094269in}}%
\pgfpathlineto{\pgfqpoint{3.603518in}{3.080821in}}%
\pgfpathlineto{\pgfqpoint{3.611309in}{3.097149in}}%
\pgfpathlineto{\pgfqpoint{3.619096in}{3.113678in}}%
\pgfpathlineto{\pgfqpoint{3.626878in}{3.130412in}}%
\pgfpathlineto{\pgfqpoint{3.634655in}{3.147358in}}%
\pgfpathlineto{\pgfqpoint{3.621340in}{3.161186in}}%
\pgfpathlineto{\pgfqpoint{3.608023in}{3.175274in}}%
\pgfpathlineto{\pgfqpoint{3.594704in}{3.189626in}}%
\pgfpathlineto{\pgfqpoint{3.581383in}{3.204241in}}%
\pgfpathlineto{\pgfqpoint{3.573601in}{3.186904in}}%
\pgfpathlineto{\pgfqpoint{3.565814in}{3.169784in}}%
\pgfpathlineto{\pgfqpoint{3.558021in}{3.152879in}}%
\pgfpathlineto{\pgfqpoint{3.550224in}{3.136183in}}%
\pgfpathclose%
\pgfusepath{fill}%
\end{pgfscope}%
\begin{pgfscope}%
\pgfpathrectangle{\pgfqpoint{1.150000in}{0.150000in}}{\pgfqpoint{5.700000in}{5.700000in}}%
\pgfusepath{clip}%
\pgfsetbuttcap%
\pgfsetroundjoin%
\definecolor{currentfill}{rgb}{0.218130,0.347432,0.550038}%
\pgfsetfillcolor{currentfill}%
\pgfsetfillopacity{0.800000}%
\pgfsetlinewidth{0.000000pt}%
\definecolor{currentstroke}{rgb}{0.000000,0.000000,0.000000}%
\pgfsetstrokecolor{currentstroke}%
\pgfsetdash{}{0pt}%
\pgfpathmoveto{\pgfqpoint{4.405921in}{3.127426in}}%
\pgfpathlineto{\pgfqpoint{4.419311in}{3.120880in}}%
\pgfpathlineto{\pgfqpoint{4.432706in}{3.114535in}}%
\pgfpathlineto{\pgfqpoint{4.446108in}{3.108390in}}%
\pgfpathlineto{\pgfqpoint{4.459516in}{3.102444in}}%
\pgfpathlineto{\pgfqpoint{4.467147in}{3.119358in}}%
\pgfpathlineto{\pgfqpoint{4.474778in}{3.136536in}}%
\pgfpathlineto{\pgfqpoint{4.482409in}{3.153986in}}%
\pgfpathlineto{\pgfqpoint{4.490039in}{3.171714in}}%
\pgfpathlineto{\pgfqpoint{4.476640in}{3.178254in}}%
\pgfpathlineto{\pgfqpoint{4.463246in}{3.184993in}}%
\pgfpathlineto{\pgfqpoint{4.449859in}{3.191932in}}%
\pgfpathlineto{\pgfqpoint{4.436477in}{3.199073in}}%
\pgfpathlineto{\pgfqpoint{4.428839in}{3.180739in}}%
\pgfpathlineto{\pgfqpoint{4.421200in}{3.162691in}}%
\pgfpathlineto{\pgfqpoint{4.413561in}{3.144922in}}%
\pgfpathlineto{\pgfqpoint{4.405921in}{3.127426in}}%
\pgfpathclose%
\pgfusepath{fill}%
\end{pgfscope}%
\begin{pgfscope}%
\pgfpathrectangle{\pgfqpoint{1.150000in}{0.150000in}}{\pgfqpoint{5.700000in}{5.700000in}}%
\pgfusepath{clip}%
\pgfsetbuttcap%
\pgfsetroundjoin%
\definecolor{currentfill}{rgb}{0.210503,0.363727,0.552206}%
\pgfsetfillcolor{currentfill}%
\pgfsetfillopacity{0.800000}%
\pgfsetlinewidth{0.000000pt}%
\definecolor{currentstroke}{rgb}{0.000000,0.000000,0.000000}%
\pgfsetstrokecolor{currentstroke}%
\pgfsetdash{}{0pt}%
\pgfpathmoveto{\pgfqpoint{4.490039in}{3.171714in}}%
\pgfpathlineto{\pgfqpoint{4.503444in}{3.165373in}}%
\pgfpathlineto{\pgfqpoint{4.516855in}{3.159229in}}%
\pgfpathlineto{\pgfqpoint{4.530273in}{3.153283in}}%
\pgfpathlineto{\pgfqpoint{4.543697in}{3.147532in}}%
\pgfpathlineto{\pgfqpoint{4.551318in}{3.164932in}}%
\pgfpathlineto{\pgfqpoint{4.558939in}{3.182616in}}%
\pgfpathlineto{\pgfqpoint{4.566560in}{3.200593in}}%
\pgfpathlineto{\pgfqpoint{4.574181in}{3.218869in}}%
\pgfpathlineto{\pgfqpoint{4.560766in}{3.225245in}}%
\pgfpathlineto{\pgfqpoint{4.547357in}{3.231817in}}%
\pgfpathlineto{\pgfqpoint{4.533955in}{3.238587in}}%
\pgfpathlineto{\pgfqpoint{4.520558in}{3.245554in}}%
\pgfpathlineto{\pgfqpoint{4.512928in}{3.226640in}}%
\pgfpathlineto{\pgfqpoint{4.505298in}{3.208034in}}%
\pgfpathlineto{\pgfqpoint{4.497669in}{3.189727in}}%
\pgfpathlineto{\pgfqpoint{4.490039in}{3.171714in}}%
\pgfpathclose%
\pgfusepath{fill}%
\end{pgfscope}%
\begin{pgfscope}%
\pgfpathrectangle{\pgfqpoint{1.150000in}{0.150000in}}{\pgfqpoint{5.700000in}{5.700000in}}%
\pgfusepath{clip}%
\pgfsetbuttcap%
\pgfsetroundjoin%
\definecolor{currentfill}{rgb}{0.225863,0.330805,0.547314}%
\pgfsetfillcolor{currentfill}%
\pgfsetfillopacity{0.800000}%
\pgfsetlinewidth{0.000000pt}%
\definecolor{currentstroke}{rgb}{0.000000,0.000000,0.000000}%
\pgfsetstrokecolor{currentstroke}%
\pgfsetdash{}{0pt}%
\pgfpathmoveto{\pgfqpoint{4.321813in}{3.085989in}}%
\pgfpathlineto{\pgfqpoint{4.335189in}{3.079193in}}%
\pgfpathlineto{\pgfqpoint{4.348570in}{3.072602in}}%
\pgfpathlineto{\pgfqpoint{4.361956in}{3.066214in}}%
\pgfpathlineto{\pgfqpoint{4.375349in}{3.060029in}}%
\pgfpathlineto{\pgfqpoint{4.382994in}{3.076504in}}%
\pgfpathlineto{\pgfqpoint{4.390637in}{3.093223in}}%
\pgfpathlineto{\pgfqpoint{4.398279in}{3.110195in}}%
\pgfpathlineto{\pgfqpoint{4.405921in}{3.127426in}}%
\pgfpathlineto{\pgfqpoint{4.392536in}{3.134174in}}%
\pgfpathlineto{\pgfqpoint{4.379158in}{3.141125in}}%
\pgfpathlineto{\pgfqpoint{4.365784in}{3.148279in}}%
\pgfpathlineto{\pgfqpoint{4.352416in}{3.155638in}}%
\pgfpathlineto{\pgfqpoint{4.344767in}{3.137832in}}%
\pgfpathlineto{\pgfqpoint{4.337117in}{3.120293in}}%
\pgfpathlineto{\pgfqpoint{4.329466in}{3.103014in}}%
\pgfpathlineto{\pgfqpoint{4.321813in}{3.085989in}}%
\pgfpathclose%
\pgfusepath{fill}%
\end{pgfscope}%
\begin{pgfscope}%
\pgfpathrectangle{\pgfqpoint{1.150000in}{0.150000in}}{\pgfqpoint{5.700000in}{5.700000in}}%
\pgfusepath{clip}%
\pgfsetbuttcap%
\pgfsetroundjoin%
\definecolor{currentfill}{rgb}{0.187231,0.414746,0.556547}%
\pgfsetfillcolor{currentfill}%
\pgfsetfillopacity{0.800000}%
\pgfsetlinewidth{0.000000pt}%
\definecolor{currentstroke}{rgb}{0.000000,0.000000,0.000000}%
\pgfsetstrokecolor{currentstroke}%
\pgfsetdash{}{0pt}%
\pgfpathmoveto{\pgfqpoint{3.390082in}{3.328383in}}%
\pgfpathlineto{\pgfqpoint{3.403449in}{3.310809in}}%
\pgfpathlineto{\pgfqpoint{3.416810in}{3.293526in}}%
\pgfpathlineto{\pgfqpoint{3.430168in}{3.276531in}}%
\pgfpathlineto{\pgfqpoint{3.443521in}{3.259823in}}%
\pgfpathlineto{\pgfqpoint{3.451329in}{3.277260in}}%
\pgfpathlineto{\pgfqpoint{3.459132in}{3.294923in}}%
\pgfpathlineto{\pgfqpoint{3.466930in}{3.312816in}}%
\pgfpathlineto{\pgfqpoint{3.474722in}{3.330944in}}%
\pgfpathlineto{\pgfqpoint{3.461374in}{3.348037in}}%
\pgfpathlineto{\pgfqpoint{3.448022in}{3.365417in}}%
\pgfpathlineto{\pgfqpoint{3.434665in}{3.383085in}}%
\pgfpathlineto{\pgfqpoint{3.421303in}{3.401046in}}%
\pgfpathlineto{\pgfqpoint{3.413506in}{3.382520in}}%
\pgfpathlineto{\pgfqpoint{3.405704in}{3.364237in}}%
\pgfpathlineto{\pgfqpoint{3.397896in}{3.346193in}}%
\pgfpathlineto{\pgfqpoint{3.390082in}{3.328383in}}%
\pgfpathclose%
\pgfusepath{fill}%
\end{pgfscope}%
\begin{pgfscope}%
\pgfpathrectangle{\pgfqpoint{1.150000in}{0.150000in}}{\pgfqpoint{5.700000in}{5.700000in}}%
\pgfusepath{clip}%
\pgfsetbuttcap%
\pgfsetroundjoin%
\definecolor{currentfill}{rgb}{0.201239,0.383670,0.554294}%
\pgfsetfillcolor{currentfill}%
\pgfsetfillopacity{0.800000}%
\pgfsetlinewidth{0.000000pt}%
\definecolor{currentstroke}{rgb}{0.000000,0.000000,0.000000}%
\pgfsetstrokecolor{currentstroke}%
\pgfsetdash{}{0pt}%
\pgfpathmoveto{\pgfqpoint{4.574181in}{3.218869in}}%
\pgfpathlineto{\pgfqpoint{4.587603in}{3.212689in}}%
\pgfpathlineto{\pgfqpoint{4.601031in}{3.206703in}}%
\pgfpathlineto{\pgfqpoint{4.614465in}{3.200911in}}%
\pgfpathlineto{\pgfqpoint{4.627907in}{3.195313in}}%
\pgfpathlineto{\pgfqpoint{4.635519in}{3.213251in}}%
\pgfpathlineto{\pgfqpoint{4.643132in}{3.231495in}}%
\pgfpathlineto{\pgfqpoint{4.650746in}{3.250054in}}%
\pgfpathlineto{\pgfqpoint{4.658362in}{3.268935in}}%
\pgfpathlineto{\pgfqpoint{4.644930in}{3.275190in}}%
\pgfpathlineto{\pgfqpoint{4.631505in}{3.281639in}}%
\pgfpathlineto{\pgfqpoint{4.618086in}{3.288282in}}%
\pgfpathlineto{\pgfqpoint{4.604674in}{3.295120in}}%
\pgfpathlineto{\pgfqpoint{4.597049in}{3.275570in}}%
\pgfpathlineto{\pgfqpoint{4.589426in}{3.256350in}}%
\pgfpathlineto{\pgfqpoint{4.581803in}{3.237452in}}%
\pgfpathlineto{\pgfqpoint{4.574181in}{3.218869in}}%
\pgfpathclose%
\pgfusepath{fill}%
\end{pgfscope}%
\begin{pgfscope}%
\pgfpathrectangle{\pgfqpoint{1.150000in}{0.150000in}}{\pgfqpoint{5.700000in}{5.700000in}}%
\pgfusepath{clip}%
\pgfsetbuttcap%
\pgfsetroundjoin%
\definecolor{currentfill}{rgb}{0.506271,0.828786,0.300362}%
\pgfsetfillcolor{currentfill}%
\pgfsetfillopacity{0.800000}%
\pgfsetlinewidth{0.000000pt}%
\definecolor{currentstroke}{rgb}{0.000000,0.000000,0.000000}%
\pgfsetstrokecolor{currentstroke}%
\pgfsetdash{}{0pt}%
\pgfpathmoveto{\pgfqpoint{3.622366in}{4.608830in}}%
\pgfpathlineto{\pgfqpoint{3.635788in}{4.582560in}}%
\pgfpathlineto{\pgfqpoint{3.649203in}{4.556617in}}%
\pgfpathlineto{\pgfqpoint{3.662610in}{4.530996in}}%
\pgfpathlineto{\pgfqpoint{3.676010in}{4.505694in}}%
\pgfpathlineto{\pgfqpoint{3.683647in}{4.540259in}}%
\pgfpathlineto{\pgfqpoint{3.691282in}{4.575369in}}%
\pgfpathlineto{\pgfqpoint{3.698915in}{4.611035in}}%
\pgfpathlineto{\pgfqpoint{3.706547in}{4.647267in}}%
\pgfpathlineto{\pgfqpoint{3.693139in}{4.673366in}}%
\pgfpathlineto{\pgfqpoint{3.679723in}{4.699787in}}%
\pgfpathlineto{\pgfqpoint{3.666300in}{4.726533in}}%
\pgfpathlineto{\pgfqpoint{3.652868in}{4.753606in}}%
\pgfpathlineto{\pgfqpoint{3.645245in}{4.716558in}}%
\pgfpathlineto{\pgfqpoint{3.637621in}{4.680085in}}%
\pgfpathlineto{\pgfqpoint{3.629994in}{4.644179in}}%
\pgfpathlineto{\pgfqpoint{3.622366in}{4.608830in}}%
\pgfpathclose%
\pgfusepath{fill}%
\end{pgfscope}%
\begin{pgfscope}%
\pgfpathrectangle{\pgfqpoint{1.150000in}{0.150000in}}{\pgfqpoint{5.700000in}{5.700000in}}%
\pgfusepath{clip}%
\pgfsetbuttcap%
\pgfsetroundjoin%
\definecolor{currentfill}{rgb}{0.227802,0.326594,0.546532}%
\pgfsetfillcolor{currentfill}%
\pgfsetfillopacity{0.800000}%
\pgfsetlinewidth{0.000000pt}%
\definecolor{currentstroke}{rgb}{0.000000,0.000000,0.000000}%
\pgfsetstrokecolor{currentstroke}%
\pgfsetdash{}{0pt}%
\pgfpathmoveto{\pgfqpoint{3.603518in}{3.080821in}}%
\pgfpathlineto{\pgfqpoint{3.616838in}{3.067632in}}%
\pgfpathlineto{\pgfqpoint{3.630156in}{3.054699in}}%
\pgfpathlineto{\pgfqpoint{3.643474in}{3.042020in}}%
\pgfpathlineto{\pgfqpoint{3.656792in}{3.029595in}}%
\pgfpathlineto{\pgfqpoint{3.664577in}{3.045555in}}%
\pgfpathlineto{\pgfqpoint{3.672358in}{3.061709in}}%
\pgfpathlineto{\pgfqpoint{3.680134in}{3.078060in}}%
\pgfpathlineto{\pgfqpoint{3.687906in}{3.094614in}}%
\pgfpathlineto{\pgfqpoint{3.674595in}{3.107418in}}%
\pgfpathlineto{\pgfqpoint{3.661283in}{3.120476in}}%
\pgfpathlineto{\pgfqpoint{3.647970in}{3.133788in}}%
\pgfpathlineto{\pgfqpoint{3.634655in}{3.147358in}}%
\pgfpathlineto{\pgfqpoint{3.626878in}{3.130412in}}%
\pgfpathlineto{\pgfqpoint{3.619096in}{3.113678in}}%
\pgfpathlineto{\pgfqpoint{3.611309in}{3.097149in}}%
\pgfpathlineto{\pgfqpoint{3.603518in}{3.080821in}}%
\pgfpathclose%
\pgfusepath{fill}%
\end{pgfscope}%
\begin{pgfscope}%
\pgfpathrectangle{\pgfqpoint{1.150000in}{0.150000in}}{\pgfqpoint{5.700000in}{5.700000in}}%
\pgfusepath{clip}%
\pgfsetbuttcap%
\pgfsetroundjoin%
\definecolor{currentfill}{rgb}{0.231674,0.318106,0.544834}%
\pgfsetfillcolor{currentfill}%
\pgfsetfillopacity{0.800000}%
\pgfsetlinewidth{0.000000pt}%
\definecolor{currentstroke}{rgb}{0.000000,0.000000,0.000000}%
\pgfsetstrokecolor{currentstroke}%
\pgfsetdash{}{0pt}%
\pgfpathmoveto{\pgfqpoint{4.237702in}{3.047412in}}%
\pgfpathlineto{\pgfqpoint{4.251065in}{3.040321in}}%
\pgfpathlineto{\pgfqpoint{4.264433in}{3.033439in}}%
\pgfpathlineto{\pgfqpoint{4.277806in}{3.026763in}}%
\pgfpathlineto{\pgfqpoint{4.291184in}{3.020293in}}%
\pgfpathlineto{\pgfqpoint{4.298844in}{3.036369in}}%
\pgfpathlineto{\pgfqpoint{4.306502in}{3.052673in}}%
\pgfpathlineto{\pgfqpoint{4.314158in}{3.069211in}}%
\pgfpathlineto{\pgfqpoint{4.321813in}{3.085989in}}%
\pgfpathlineto{\pgfqpoint{4.308442in}{3.092990in}}%
\pgfpathlineto{\pgfqpoint{4.295077in}{3.100197in}}%
\pgfpathlineto{\pgfqpoint{4.281716in}{3.107612in}}%
\pgfpathlineto{\pgfqpoint{4.268361in}{3.115235in}}%
\pgfpathlineto{\pgfqpoint{4.260699in}{3.097913in}}%
\pgfpathlineto{\pgfqpoint{4.253035in}{3.080839in}}%
\pgfpathlineto{\pgfqpoint{4.245370in}{3.064008in}}%
\pgfpathlineto{\pgfqpoint{4.237702in}{3.047412in}}%
\pgfpathclose%
\pgfusepath{fill}%
\end{pgfscope}%
\begin{pgfscope}%
\pgfpathrectangle{\pgfqpoint{1.150000in}{0.150000in}}{\pgfqpoint{5.700000in}{5.700000in}}%
\pgfusepath{clip}%
\pgfsetbuttcap%
\pgfsetroundjoin%
\definecolor{currentfill}{rgb}{0.119423,0.611141,0.538982}%
\pgfsetfillcolor{currentfill}%
\pgfsetfillopacity{0.800000}%
\pgfsetlinewidth{0.000000pt}%
\definecolor{currentstroke}{rgb}{0.000000,0.000000,0.000000}%
\pgfsetstrokecolor{currentstroke}%
\pgfsetdash{}{0pt}%
\pgfpathmoveto{\pgfqpoint{3.268968in}{3.901990in}}%
\pgfpathlineto{\pgfqpoint{3.282433in}{3.878091in}}%
\pgfpathlineto{\pgfqpoint{3.295890in}{3.854531in}}%
\pgfpathlineto{\pgfqpoint{3.309337in}{3.831307in}}%
\pgfpathlineto{\pgfqpoint{3.322777in}{3.808417in}}%
\pgfpathlineto{\pgfqpoint{3.330533in}{3.830799in}}%
\pgfpathlineto{\pgfqpoint{3.338285in}{3.853502in}}%
\pgfpathlineto{\pgfqpoint{3.346030in}{3.876532in}}%
\pgfpathlineto{\pgfqpoint{3.353770in}{3.899896in}}%
\pgfpathlineto{\pgfqpoint{3.340332in}{3.923286in}}%
\pgfpathlineto{\pgfqpoint{3.326886in}{3.947009in}}%
\pgfpathlineto{\pgfqpoint{3.313431in}{3.971070in}}%
\pgfpathlineto{\pgfqpoint{3.299967in}{3.995472in}}%
\pgfpathlineto{\pgfqpoint{3.292226in}{3.971594in}}%
\pgfpathlineto{\pgfqpoint{3.284479in}{3.948058in}}%
\pgfpathlineto{\pgfqpoint{3.276726in}{3.924859in}}%
\pgfpathlineto{\pgfqpoint{3.268968in}{3.901990in}}%
\pgfpathclose%
\pgfusepath{fill}%
\end{pgfscope}%
\begin{pgfscope}%
\pgfpathrectangle{\pgfqpoint{1.150000in}{0.150000in}}{\pgfqpoint{5.700000in}{5.700000in}}%
\pgfusepath{clip}%
\pgfsetbuttcap%
\pgfsetroundjoin%
\definecolor{currentfill}{rgb}{0.241237,0.296485,0.539709}%
\pgfsetfillcolor{currentfill}%
\pgfsetfillopacity{0.800000}%
\pgfsetlinewidth{0.000000pt}%
\definecolor{currentstroke}{rgb}{0.000000,0.000000,0.000000}%
\pgfsetstrokecolor{currentstroke}%
\pgfsetdash{}{0pt}%
\pgfpathmoveto{\pgfqpoint{3.794397in}{3.001086in}}%
\pgfpathlineto{\pgfqpoint{3.807712in}{2.990483in}}%
\pgfpathlineto{\pgfqpoint{3.821027in}{2.980118in}}%
\pgfpathlineto{\pgfqpoint{3.834344in}{2.969987in}}%
\pgfpathlineto{\pgfqpoint{3.847662in}{2.960091in}}%
\pgfpathlineto{\pgfqpoint{3.855413in}{2.975659in}}%
\pgfpathlineto{\pgfqpoint{3.863159in}{2.991416in}}%
\pgfpathlineto{\pgfqpoint{3.870901in}{3.007365in}}%
\pgfpathlineto{\pgfqpoint{3.878639in}{3.023511in}}%
\pgfpathlineto{\pgfqpoint{3.865327in}{3.033816in}}%
\pgfpathlineto{\pgfqpoint{3.852016in}{3.044355in}}%
\pgfpathlineto{\pgfqpoint{3.838706in}{3.055129in}}%
\pgfpathlineto{\pgfqpoint{3.825398in}{3.066140in}}%
\pgfpathlineto{\pgfqpoint{3.817654in}{3.049573in}}%
\pgfpathlineto{\pgfqpoint{3.809906in}{3.033211in}}%
\pgfpathlineto{\pgfqpoint{3.802154in}{3.017051in}}%
\pgfpathlineto{\pgfqpoint{3.794397in}{3.001086in}}%
\pgfpathclose%
\pgfusepath{fill}%
\end{pgfscope}%
\begin{pgfscope}%
\pgfpathrectangle{\pgfqpoint{1.150000in}{0.150000in}}{\pgfqpoint{5.700000in}{5.700000in}}%
\pgfusepath{clip}%
\pgfsetbuttcap%
\pgfsetroundjoin%
\definecolor{currentfill}{rgb}{0.192357,0.403199,0.555836}%
\pgfsetfillcolor{currentfill}%
\pgfsetfillopacity{0.800000}%
\pgfsetlinewidth{0.000000pt}%
\definecolor{currentstroke}{rgb}{0.000000,0.000000,0.000000}%
\pgfsetstrokecolor{currentstroke}%
\pgfsetdash{}{0pt}%
\pgfpathmoveto{\pgfqpoint{4.658362in}{3.268935in}}%
\pgfpathlineto{\pgfqpoint{4.671800in}{3.262872in}}%
\pgfpathlineto{\pgfqpoint{4.685246in}{3.257001in}}%
\pgfpathlineto{\pgfqpoint{4.698698in}{3.251321in}}%
\pgfpathlineto{\pgfqpoint{4.712157in}{3.245832in}}%
\pgfpathlineto{\pgfqpoint{4.719764in}{3.264366in}}%
\pgfpathlineto{\pgfqpoint{4.727372in}{3.283230in}}%
\pgfpathlineto{\pgfqpoint{4.734982in}{3.302432in}}%
\pgfpathlineto{\pgfqpoint{4.742595in}{3.321980in}}%
\pgfpathlineto{\pgfqpoint{4.729146in}{3.328157in}}%
\pgfpathlineto{\pgfqpoint{4.715704in}{3.334526in}}%
\pgfpathlineto{\pgfqpoint{4.702269in}{3.341086in}}%
\pgfpathlineto{\pgfqpoint{4.688840in}{3.347838in}}%
\pgfpathlineto{\pgfqpoint{4.681218in}{3.327589in}}%
\pgfpathlineto{\pgfqpoint{4.673597in}{3.307694in}}%
\pgfpathlineto{\pgfqpoint{4.665979in}{3.288145in}}%
\pgfpathlineto{\pgfqpoint{4.658362in}{3.268935in}}%
\pgfpathclose%
\pgfusepath{fill}%
\end{pgfscope}%
\begin{pgfscope}%
\pgfpathrectangle{\pgfqpoint{1.150000in}{0.150000in}}{\pgfqpoint{5.700000in}{5.700000in}}%
\pgfusepath{clip}%
\pgfsetbuttcap%
\pgfsetroundjoin%
\definecolor{currentfill}{rgb}{0.344074,0.780029,0.397381}%
\pgfsetfillcolor{currentfill}%
\pgfsetfillopacity{0.800000}%
\pgfsetlinewidth{0.000000pt}%
\definecolor{currentstroke}{rgb}{0.000000,0.000000,0.000000}%
\pgfsetstrokecolor{currentstroke}%
\pgfsetdash{}{0pt}%
\pgfpathmoveto{\pgfqpoint{3.423088in}{4.429443in}}%
\pgfpathlineto{\pgfqpoint{3.436561in}{4.402657in}}%
\pgfpathlineto{\pgfqpoint{3.450024in}{4.376217in}}%
\pgfpathlineto{\pgfqpoint{3.463478in}{4.350121in}}%
\pgfpathlineto{\pgfqpoint{3.476923in}{4.324365in}}%
\pgfpathlineto{\pgfqpoint{3.484582in}{4.354418in}}%
\pgfpathlineto{\pgfqpoint{3.492236in}{4.384930in}}%
\pgfpathlineto{\pgfqpoint{3.499887in}{4.415912in}}%
\pgfpathlineto{\pgfqpoint{3.507534in}{4.447369in}}%
\pgfpathlineto{\pgfqpoint{3.494084in}{4.473814in}}%
\pgfpathlineto{\pgfqpoint{3.480625in}{4.500601in}}%
\pgfpathlineto{\pgfqpoint{3.467156in}{4.527732in}}%
\pgfpathlineto{\pgfqpoint{3.453678in}{4.555212in}}%
\pgfpathlineto{\pgfqpoint{3.446037in}{4.523048in}}%
\pgfpathlineto{\pgfqpoint{3.438391in}{4.491370in}}%
\pgfpathlineto{\pgfqpoint{3.430742in}{4.460171in}}%
\pgfpathlineto{\pgfqpoint{3.423088in}{4.429443in}}%
\pgfpathclose%
\pgfusepath{fill}%
\end{pgfscope}%
\begin{pgfscope}%
\pgfpathrectangle{\pgfqpoint{1.150000in}{0.150000in}}{\pgfqpoint{5.700000in}{5.700000in}}%
\pgfusepath{clip}%
\pgfsetbuttcap%
\pgfsetroundjoin%
\definecolor{currentfill}{rgb}{0.243113,0.292092,0.538516}%
\pgfsetfillcolor{currentfill}%
\pgfsetfillopacity{0.800000}%
\pgfsetlinewidth{0.000000pt}%
\definecolor{currentstroke}{rgb}{0.000000,0.000000,0.000000}%
\pgfsetstrokecolor{currentstroke}%
\pgfsetdash{}{0pt}%
\pgfpathmoveto{\pgfqpoint{3.931909in}{2.984604in}}%
\pgfpathlineto{\pgfqpoint{3.945232in}{2.975448in}}%
\pgfpathlineto{\pgfqpoint{3.958557in}{2.966518in}}%
\pgfpathlineto{\pgfqpoint{3.971886in}{2.957813in}}%
\pgfpathlineto{\pgfqpoint{3.985217in}{2.949331in}}%
\pgfpathlineto{\pgfqpoint{3.992939in}{2.964825in}}%
\pgfpathlineto{\pgfqpoint{4.000658in}{2.980512in}}%
\pgfpathlineto{\pgfqpoint{4.008374in}{2.996396in}}%
\pgfpathlineto{\pgfqpoint{4.016086in}{3.012482in}}%
\pgfpathlineto{\pgfqpoint{4.002761in}{3.021403in}}%
\pgfpathlineto{\pgfqpoint{3.989439in}{3.030547in}}%
\pgfpathlineto{\pgfqpoint{3.976119in}{3.039916in}}%
\pgfpathlineto{\pgfqpoint{3.962802in}{3.049511in}}%
\pgfpathlineto{\pgfqpoint{3.955084in}{3.032974in}}%
\pgfpathlineto{\pgfqpoint{3.947362in}{3.016647in}}%
\pgfpathlineto{\pgfqpoint{3.939637in}{3.000526in}}%
\pgfpathlineto{\pgfqpoint{3.931909in}{2.984604in}}%
\pgfpathclose%
\pgfusepath{fill}%
\end{pgfscope}%
\begin{pgfscope}%
\pgfpathrectangle{\pgfqpoint{1.150000in}{0.150000in}}{\pgfqpoint{5.700000in}{5.700000in}}%
\pgfusepath{clip}%
\pgfsetbuttcap%
\pgfsetroundjoin%
\definecolor{currentfill}{rgb}{0.175841,0.441290,0.557685}%
\pgfsetfillcolor{currentfill}%
\pgfsetfillopacity{0.800000}%
\pgfsetlinewidth{0.000000pt}%
\definecolor{currentstroke}{rgb}{0.000000,0.000000,0.000000}%
\pgfsetstrokecolor{currentstroke}%
\pgfsetdash{}{0pt}%
\pgfpathmoveto{\pgfqpoint{3.336565in}{3.401641in}}%
\pgfpathlineto{\pgfqpoint{3.349953in}{3.382877in}}%
\pgfpathlineto{\pgfqpoint{3.363334in}{3.364414in}}%
\pgfpathlineto{\pgfqpoint{3.376711in}{3.346250in}}%
\pgfpathlineto{\pgfqpoint{3.390082in}{3.328383in}}%
\pgfpathlineto{\pgfqpoint{3.397896in}{3.346193in}}%
\pgfpathlineto{\pgfqpoint{3.405704in}{3.364237in}}%
\pgfpathlineto{\pgfqpoint{3.413506in}{3.382520in}}%
\pgfpathlineto{\pgfqpoint{3.421303in}{3.401046in}}%
\pgfpathlineto{\pgfqpoint{3.407937in}{3.419300in}}%
\pgfpathlineto{\pgfqpoint{3.394566in}{3.437851in}}%
\pgfpathlineto{\pgfqpoint{3.381189in}{3.456701in}}%
\pgfpathlineto{\pgfqpoint{3.367806in}{3.475853in}}%
\pgfpathlineto{\pgfqpoint{3.360004in}{3.456927in}}%
\pgfpathlineto{\pgfqpoint{3.352197in}{3.438253in}}%
\pgfpathlineto{\pgfqpoint{3.344384in}{3.419826in}}%
\pgfpathlineto{\pgfqpoint{3.336565in}{3.401641in}}%
\pgfpathclose%
\pgfusepath{fill}%
\end{pgfscope}%
\begin{pgfscope}%
\pgfpathrectangle{\pgfqpoint{1.150000in}{0.150000in}}{\pgfqpoint{5.700000in}{5.700000in}}%
\pgfusepath{clip}%
\pgfsetbuttcap%
\pgfsetroundjoin%
\definecolor{currentfill}{rgb}{0.477504,0.821444,0.318195}%
\pgfsetfillcolor{currentfill}%
\pgfsetfillopacity{0.800000}%
\pgfsetlinewidth{0.000000pt}%
\definecolor{currentstroke}{rgb}{0.000000,0.000000,0.000000}%
\pgfsetstrokecolor{currentstroke}%
\pgfsetdash{}{0pt}%
\pgfpathmoveto{\pgfqpoint{3.538088in}{4.578136in}}%
\pgfpathlineto{\pgfqpoint{3.551536in}{4.551301in}}%
\pgfpathlineto{\pgfqpoint{3.564976in}{4.524803in}}%
\pgfpathlineto{\pgfqpoint{3.578407in}{4.498639in}}%
\pgfpathlineto{\pgfqpoint{3.591830in}{4.472804in}}%
\pgfpathlineto{\pgfqpoint{3.599468in}{4.506023in}}%
\pgfpathlineto{\pgfqpoint{3.607103in}{4.539761in}}%
\pgfpathlineto{\pgfqpoint{3.614736in}{4.574027in}}%
\pgfpathlineto{\pgfqpoint{3.622366in}{4.608830in}}%
\pgfpathlineto{\pgfqpoint{3.608935in}{4.635428in}}%
\pgfpathlineto{\pgfqpoint{3.595497in}{4.662358in}}%
\pgfpathlineto{\pgfqpoint{3.582049in}{4.689623in}}%
\pgfpathlineto{\pgfqpoint{3.568593in}{4.717227in}}%
\pgfpathlineto{\pgfqpoint{3.560971in}{4.681641in}}%
\pgfpathlineto{\pgfqpoint{3.553346in}{4.646604in}}%
\pgfpathlineto{\pgfqpoint{3.545719in}{4.612105in}}%
\pgfpathlineto{\pgfqpoint{3.538088in}{4.578136in}}%
\pgfpathclose%
\pgfusepath{fill}%
\end{pgfscope}%
\begin{pgfscope}%
\pgfpathrectangle{\pgfqpoint{1.150000in}{0.150000in}}{\pgfqpoint{5.700000in}{5.700000in}}%
\pgfusepath{clip}%
\pgfsetbuttcap%
\pgfsetroundjoin%
\definecolor{currentfill}{rgb}{0.237441,0.305202,0.541921}%
\pgfsetfillcolor{currentfill}%
\pgfsetfillopacity{0.800000}%
\pgfsetlinewidth{0.000000pt}%
\definecolor{currentstroke}{rgb}{0.000000,0.000000,0.000000}%
\pgfsetstrokecolor{currentstroke}%
\pgfsetdash{}{0pt}%
\pgfpathmoveto{\pgfqpoint{4.153575in}{3.011731in}}%
\pgfpathlineto{\pgfqpoint{4.166927in}{3.004298in}}%
\pgfpathlineto{\pgfqpoint{4.180283in}{2.997078in}}%
\pgfpathlineto{\pgfqpoint{4.193643in}{2.990068in}}%
\pgfpathlineto{\pgfqpoint{4.207009in}{2.983269in}}%
\pgfpathlineto{\pgfqpoint{4.214686in}{2.998981in}}%
\pgfpathlineto{\pgfqpoint{4.222360in}{3.014905in}}%
\pgfpathlineto{\pgfqpoint{4.230033in}{3.031047in}}%
\pgfpathlineto{\pgfqpoint{4.237702in}{3.047412in}}%
\pgfpathlineto{\pgfqpoint{4.224344in}{3.054712in}}%
\pgfpathlineto{\pgfqpoint{4.210991in}{3.062222in}}%
\pgfpathlineto{\pgfqpoint{4.197642in}{3.069944in}}%
\pgfpathlineto{\pgfqpoint{4.184297in}{3.077877in}}%
\pgfpathlineto{\pgfqpoint{4.176620in}{3.060999in}}%
\pgfpathlineto{\pgfqpoint{4.168941in}{3.044353in}}%
\pgfpathlineto{\pgfqpoint{4.161260in}{3.027932in}}%
\pgfpathlineto{\pgfqpoint{4.153575in}{3.011731in}}%
\pgfpathclose%
\pgfusepath{fill}%
\end{pgfscope}%
\begin{pgfscope}%
\pgfpathrectangle{\pgfqpoint{1.150000in}{0.150000in}}{\pgfqpoint{5.700000in}{5.700000in}}%
\pgfusepath{clip}%
\pgfsetbuttcap%
\pgfsetroundjoin%
\definecolor{currentfill}{rgb}{0.237441,0.305202,0.541921}%
\pgfsetfillcolor{currentfill}%
\pgfsetfillopacity{0.800000}%
\pgfsetlinewidth{0.000000pt}%
\definecolor{currentstroke}{rgb}{0.000000,0.000000,0.000000}%
\pgfsetstrokecolor{currentstroke}%
\pgfsetdash{}{0pt}%
\pgfpathmoveto{\pgfqpoint{3.656792in}{3.029595in}}%
\pgfpathlineto{\pgfqpoint{3.670108in}{3.017421in}}%
\pgfpathlineto{\pgfqpoint{3.683425in}{3.005496in}}%
\pgfpathlineto{\pgfqpoint{3.696741in}{2.993819in}}%
\pgfpathlineto{\pgfqpoint{3.710058in}{2.982387in}}%
\pgfpathlineto{\pgfqpoint{3.717838in}{2.997981in}}%
\pgfpathlineto{\pgfqpoint{3.725613in}{3.013760in}}%
\pgfpathlineto{\pgfqpoint{3.733383in}{3.029729in}}%
\pgfpathlineto{\pgfqpoint{3.741149in}{3.045893in}}%
\pgfpathlineto{\pgfqpoint{3.727839in}{3.057702in}}%
\pgfpathlineto{\pgfqpoint{3.714528in}{3.069758in}}%
\pgfpathlineto{\pgfqpoint{3.701217in}{3.082061in}}%
\pgfpathlineto{\pgfqpoint{3.687906in}{3.094614in}}%
\pgfpathlineto{\pgfqpoint{3.680134in}{3.078060in}}%
\pgfpathlineto{\pgfqpoint{3.672358in}{3.061709in}}%
\pgfpathlineto{\pgfqpoint{3.664577in}{3.045555in}}%
\pgfpathlineto{\pgfqpoint{3.656792in}{3.029595in}}%
\pgfpathclose%
\pgfusepath{fill}%
\end{pgfscope}%
\begin{pgfscope}%
\pgfpathrectangle{\pgfqpoint{1.150000in}{0.150000in}}{\pgfqpoint{5.700000in}{5.700000in}}%
\pgfusepath{clip}%
\pgfsetbuttcap%
\pgfsetroundjoin%
\definecolor{currentfill}{rgb}{0.143343,0.522773,0.556295}%
\pgfsetfillcolor{currentfill}%
\pgfsetfillopacity{0.800000}%
\pgfsetlinewidth{0.000000pt}%
\definecolor{currentstroke}{rgb}{0.000000,0.000000,0.000000}%
\pgfsetstrokecolor{currentstroke}%
\pgfsetdash{}{0pt}%
\pgfpathmoveto{\pgfqpoint{3.260510in}{3.640277in}}%
\pgfpathlineto{\pgfqpoint{3.273947in}{3.618607in}}%
\pgfpathlineto{\pgfqpoint{3.287377in}{3.597262in}}%
\pgfpathlineto{\pgfqpoint{3.300799in}{3.576240in}}%
\pgfpathlineto{\pgfqpoint{3.314214in}{3.555536in}}%
\pgfpathlineto{\pgfqpoint{3.322014in}{3.575129in}}%
\pgfpathlineto{\pgfqpoint{3.329809in}{3.594992in}}%
\pgfpathlineto{\pgfqpoint{3.337597in}{3.615130in}}%
\pgfpathlineto{\pgfqpoint{3.345380in}{3.635548in}}%
\pgfpathlineto{\pgfqpoint{3.331969in}{3.656677in}}%
\pgfpathlineto{\pgfqpoint{3.318551in}{3.678126in}}%
\pgfpathlineto{\pgfqpoint{3.305125in}{3.699897in}}%
\pgfpathlineto{\pgfqpoint{3.291691in}{3.721995in}}%
\pgfpathlineto{\pgfqpoint{3.283905in}{3.701137in}}%
\pgfpathlineto{\pgfqpoint{3.276113in}{3.680569in}}%
\pgfpathlineto{\pgfqpoint{3.268314in}{3.660283in}}%
\pgfpathlineto{\pgfqpoint{3.260510in}{3.640277in}}%
\pgfpathclose%
\pgfusepath{fill}%
\end{pgfscope}%
\begin{pgfscope}%
\pgfpathrectangle{\pgfqpoint{1.150000in}{0.150000in}}{\pgfqpoint{5.700000in}{5.700000in}}%
\pgfusepath{clip}%
\pgfsetbuttcap%
\pgfsetroundjoin%
\definecolor{currentfill}{rgb}{0.183898,0.422383,0.556944}%
\pgfsetfillcolor{currentfill}%
\pgfsetfillopacity{0.800000}%
\pgfsetlinewidth{0.000000pt}%
\definecolor{currentstroke}{rgb}{0.000000,0.000000,0.000000}%
\pgfsetstrokecolor{currentstroke}%
\pgfsetdash{}{0pt}%
\pgfpathmoveto{\pgfqpoint{4.742595in}{3.321980in}}%
\pgfpathlineto{\pgfqpoint{4.756051in}{3.315992in}}%
\pgfpathlineto{\pgfqpoint{4.769514in}{3.310193in}}%
\pgfpathlineto{\pgfqpoint{4.782984in}{3.304583in}}%
\pgfpathlineto{\pgfqpoint{4.796462in}{3.299161in}}%
\pgfpathlineto{\pgfqpoint{4.804066in}{3.318355in}}%
\pgfpathlineto{\pgfqpoint{4.811673in}{3.337904in}}%
\pgfpathlineto{\pgfqpoint{4.819283in}{3.357815in}}%
\pgfpathlineto{\pgfqpoint{4.826896in}{3.378099in}}%
\pgfpathlineto{\pgfqpoint{4.813429in}{3.384242in}}%
\pgfpathlineto{\pgfqpoint{4.799970in}{3.390573in}}%
\pgfpathlineto{\pgfqpoint{4.786518in}{3.397092in}}%
\pgfpathlineto{\pgfqpoint{4.773072in}{3.403801in}}%
\pgfpathlineto{\pgfqpoint{4.765448in}{3.382784in}}%
\pgfpathlineto{\pgfqpoint{4.757828in}{3.362147in}}%
\pgfpathlineto{\pgfqpoint{4.750210in}{3.341882in}}%
\pgfpathlineto{\pgfqpoint{4.742595in}{3.321980in}}%
\pgfpathclose%
\pgfusepath{fill}%
\end{pgfscope}%
\begin{pgfscope}%
\pgfpathrectangle{\pgfqpoint{1.150000in}{0.150000in}}{\pgfqpoint{5.700000in}{5.700000in}}%
\pgfusepath{clip}%
\pgfsetbuttcap%
\pgfsetroundjoin%
\definecolor{currentfill}{rgb}{0.123463,0.581687,0.547445}%
\pgfsetfillcolor{currentfill}%
\pgfsetfillopacity{0.800000}%
\pgfsetlinewidth{0.000000pt}%
\definecolor{currentstroke}{rgb}{0.000000,0.000000,0.000000}%
\pgfsetstrokecolor{currentstroke}%
\pgfsetdash{}{0pt}%
\pgfpathmoveto{\pgfqpoint{3.237872in}{3.813708in}}%
\pgfpathlineto{\pgfqpoint{3.251340in}{3.790275in}}%
\pgfpathlineto{\pgfqpoint{3.264799in}{3.767180in}}%
\pgfpathlineto{\pgfqpoint{3.278249in}{3.744421in}}%
\pgfpathlineto{\pgfqpoint{3.291691in}{3.721995in}}%
\pgfpathlineto{\pgfqpoint{3.299471in}{3.743146in}}%
\pgfpathlineto{\pgfqpoint{3.307245in}{3.764597in}}%
\pgfpathlineto{\pgfqpoint{3.315014in}{3.786352in}}%
\pgfpathlineto{\pgfqpoint{3.322777in}{3.808417in}}%
\pgfpathlineto{\pgfqpoint{3.309337in}{3.831307in}}%
\pgfpathlineto{\pgfqpoint{3.295890in}{3.854531in}}%
\pgfpathlineto{\pgfqpoint{3.282433in}{3.878091in}}%
\pgfpathlineto{\pgfqpoint{3.268968in}{3.901990in}}%
\pgfpathlineto{\pgfqpoint{3.261203in}{3.879446in}}%
\pgfpathlineto{\pgfqpoint{3.253432in}{3.857222in}}%
\pgfpathlineto{\pgfqpoint{3.245655in}{3.835311in}}%
\pgfpathlineto{\pgfqpoint{3.237872in}{3.813708in}}%
\pgfpathclose%
\pgfusepath{fill}%
\end{pgfscope}%
\begin{pgfscope}%
\pgfpathrectangle{\pgfqpoint{1.150000in}{0.150000in}}{\pgfqpoint{5.700000in}{5.700000in}}%
\pgfusepath{clip}%
\pgfsetbuttcap%
\pgfsetroundjoin%
\definecolor{currentfill}{rgb}{0.243113,0.292092,0.538516}%
\pgfsetfillcolor{currentfill}%
\pgfsetfillopacity{0.800000}%
\pgfsetlinewidth{0.000000pt}%
\definecolor{currentstroke}{rgb}{0.000000,0.000000,0.000000}%
\pgfsetstrokecolor{currentstroke}%
\pgfsetdash{}{0pt}%
\pgfpathmoveto{\pgfqpoint{4.069418in}{2.979008in}}%
\pgfpathlineto{\pgfqpoint{4.082760in}{2.971186in}}%
\pgfpathlineto{\pgfqpoint{4.096106in}{2.963580in}}%
\pgfpathlineto{\pgfqpoint{4.109456in}{2.956189in}}%
\pgfpathlineto{\pgfqpoint{4.122810in}{2.949013in}}%
\pgfpathlineto{\pgfqpoint{4.130506in}{2.964391in}}%
\pgfpathlineto{\pgfqpoint{4.138199in}{2.979966in}}%
\pgfpathlineto{\pgfqpoint{4.145889in}{2.995744in}}%
\pgfpathlineto{\pgfqpoint{4.153575in}{3.011731in}}%
\pgfpathlineto{\pgfqpoint{4.140228in}{3.019377in}}%
\pgfpathlineto{\pgfqpoint{4.126885in}{3.027237in}}%
\pgfpathlineto{\pgfqpoint{4.113546in}{3.035312in}}%
\pgfpathlineto{\pgfqpoint{4.100210in}{3.043604in}}%
\pgfpathlineto{\pgfqpoint{4.092517in}{3.027136in}}%
\pgfpathlineto{\pgfqpoint{4.084820in}{3.010884in}}%
\pgfpathlineto{\pgfqpoint{4.077121in}{2.994843in}}%
\pgfpathlineto{\pgfqpoint{4.069418in}{2.979008in}}%
\pgfpathclose%
\pgfusepath{fill}%
\end{pgfscope}%
\begin{pgfscope}%
\pgfpathrectangle{\pgfqpoint{1.150000in}{0.150000in}}{\pgfqpoint{5.700000in}{5.700000in}}%
\pgfusepath{clip}%
\pgfsetbuttcap%
\pgfsetroundjoin%
\definecolor{currentfill}{rgb}{0.165117,0.467423,0.558141}%
\pgfsetfillcolor{currentfill}%
\pgfsetfillopacity{0.800000}%
\pgfsetlinewidth{0.000000pt}%
\definecolor{currentstroke}{rgb}{0.000000,0.000000,0.000000}%
\pgfsetstrokecolor{currentstroke}%
\pgfsetdash{}{0pt}%
\pgfpathmoveto{\pgfqpoint{3.282953in}{3.479765in}}%
\pgfpathlineto{\pgfqpoint{3.296366in}{3.459768in}}%
\pgfpathlineto{\pgfqpoint{3.309772in}{3.440083in}}%
\pgfpathlineto{\pgfqpoint{3.323172in}{3.420709in}}%
\pgfpathlineto{\pgfqpoint{3.336565in}{3.401641in}}%
\pgfpathlineto{\pgfqpoint{3.344384in}{3.419826in}}%
\pgfpathlineto{\pgfqpoint{3.352197in}{3.438253in}}%
\pgfpathlineto{\pgfqpoint{3.360004in}{3.456927in}}%
\pgfpathlineto{\pgfqpoint{3.367806in}{3.475853in}}%
\pgfpathlineto{\pgfqpoint{3.354418in}{3.495310in}}%
\pgfpathlineto{\pgfqpoint{3.341023in}{3.515074in}}%
\pgfpathlineto{\pgfqpoint{3.327622in}{3.535149in}}%
\pgfpathlineto{\pgfqpoint{3.314214in}{3.555536in}}%
\pgfpathlineto{\pgfqpoint{3.306408in}{3.536208in}}%
\pgfpathlineto{\pgfqpoint{3.298596in}{3.517140in}}%
\pgfpathlineto{\pgfqpoint{3.290778in}{3.498327in}}%
\pgfpathlineto{\pgfqpoint{3.282953in}{3.479765in}}%
\pgfpathclose%
\pgfusepath{fill}%
\end{pgfscope}%
\begin{pgfscope}%
\pgfpathrectangle{\pgfqpoint{1.150000in}{0.150000in}}{\pgfqpoint{5.700000in}{5.700000in}}%
\pgfusepath{clip}%
\pgfsetbuttcap%
\pgfsetroundjoin%
\definecolor{currentfill}{rgb}{0.196571,0.711827,0.479221}%
\pgfsetfillcolor{currentfill}%
\pgfsetfillopacity{0.800000}%
\pgfsetlinewidth{0.000000pt}%
\definecolor{currentstroke}{rgb}{0.000000,0.000000,0.000000}%
\pgfsetstrokecolor{currentstroke}%
\pgfsetdash{}{0pt}%
\pgfpathmoveto{\pgfqpoint{3.276919in}{4.197791in}}%
\pgfpathlineto{\pgfqpoint{3.290422in}{4.171446in}}%
\pgfpathlineto{\pgfqpoint{3.303916in}{4.145458in}}%
\pgfpathlineto{\pgfqpoint{3.317399in}{4.119822in}}%
\pgfpathlineto{\pgfqpoint{3.330873in}{4.094535in}}%
\pgfpathlineto{\pgfqpoint{3.338586in}{4.120220in}}%
\pgfpathlineto{\pgfqpoint{3.346293in}{4.146286in}}%
\pgfpathlineto{\pgfqpoint{3.353995in}{4.172740in}}%
\pgfpathlineto{\pgfqpoint{3.361692in}{4.199588in}}%
\pgfpathlineto{\pgfqpoint{3.348217in}{4.225452in}}%
\pgfpathlineto{\pgfqpoint{3.334732in}{4.251667in}}%
\pgfpathlineto{\pgfqpoint{3.321237in}{4.278236in}}%
\pgfpathlineto{\pgfqpoint{3.307731in}{4.305163in}}%
\pgfpathlineto{\pgfqpoint{3.300036in}{4.277721in}}%
\pgfpathlineto{\pgfqpoint{3.292336in}{4.250683in}}%
\pgfpathlineto{\pgfqpoint{3.284630in}{4.224042in}}%
\pgfpathlineto{\pgfqpoint{3.276919in}{4.197791in}}%
\pgfpathclose%
\pgfusepath{fill}%
\end{pgfscope}%
\begin{pgfscope}%
\pgfpathrectangle{\pgfqpoint{1.150000in}{0.150000in}}{\pgfqpoint{5.700000in}{5.700000in}}%
\pgfusepath{clip}%
\pgfsetbuttcap%
\pgfsetroundjoin%
\definecolor{currentfill}{rgb}{0.246811,0.283237,0.535941}%
\pgfsetfillcolor{currentfill}%
\pgfsetfillopacity{0.800000}%
\pgfsetlinewidth{0.000000pt}%
\definecolor{currentstroke}{rgb}{0.000000,0.000000,0.000000}%
\pgfsetstrokecolor{currentstroke}%
\pgfsetdash{}{0pt}%
\pgfpathmoveto{\pgfqpoint{3.847662in}{2.960091in}}%
\pgfpathlineto{\pgfqpoint{3.860983in}{2.950427in}}%
\pgfpathlineto{\pgfqpoint{3.874305in}{2.940994in}}%
\pgfpathlineto{\pgfqpoint{3.887629in}{2.931790in}}%
\pgfpathlineto{\pgfqpoint{3.900956in}{2.922815in}}%
\pgfpathlineto{\pgfqpoint{3.908700in}{2.937987in}}%
\pgfpathlineto{\pgfqpoint{3.916440in}{2.953340in}}%
\pgfpathlineto{\pgfqpoint{3.924176in}{2.968877in}}%
\pgfpathlineto{\pgfqpoint{3.931909in}{2.984604in}}%
\pgfpathlineto{\pgfqpoint{3.918588in}{2.993987in}}%
\pgfpathlineto{\pgfqpoint{3.905270in}{3.003598in}}%
\pgfpathlineto{\pgfqpoint{3.891954in}{3.013439in}}%
\pgfpathlineto{\pgfqpoint{3.878639in}{3.023511in}}%
\pgfpathlineto{\pgfqpoint{3.870901in}{3.007365in}}%
\pgfpathlineto{\pgfqpoint{3.863159in}{2.991416in}}%
\pgfpathlineto{\pgfqpoint{3.855413in}{2.975659in}}%
\pgfpathlineto{\pgfqpoint{3.847662in}{2.960091in}}%
\pgfpathclose%
\pgfusepath{fill}%
\end{pgfscope}%
\begin{pgfscope}%
\pgfpathrectangle{\pgfqpoint{1.150000in}{0.150000in}}{\pgfqpoint{5.700000in}{5.700000in}}%
\pgfusepath{clip}%
\pgfsetbuttcap%
\pgfsetroundjoin%
\definecolor{currentfill}{rgb}{0.259857,0.745492,0.444467}%
\pgfsetfillcolor{currentfill}%
\pgfsetfillopacity{0.800000}%
\pgfsetlinewidth{0.000000pt}%
\definecolor{currentstroke}{rgb}{0.000000,0.000000,0.000000}%
\pgfsetstrokecolor{currentstroke}%
\pgfsetdash{}{0pt}%
\pgfpathmoveto{\pgfqpoint{3.307731in}{4.305163in}}%
\pgfpathlineto{\pgfqpoint{3.321237in}{4.278236in}}%
\pgfpathlineto{\pgfqpoint{3.334732in}{4.251667in}}%
\pgfpathlineto{\pgfqpoint{3.348217in}{4.225452in}}%
\pgfpathlineto{\pgfqpoint{3.361692in}{4.199588in}}%
\pgfpathlineto{\pgfqpoint{3.369384in}{4.226837in}}%
\pgfpathlineto{\pgfqpoint{3.377070in}{4.254496in}}%
\pgfpathlineto{\pgfqpoint{3.384752in}{4.282570in}}%
\pgfpathlineto{\pgfqpoint{3.392429in}{4.311068in}}%
\pgfpathlineto{\pgfqpoint{3.378951in}{4.337548in}}%
\pgfpathlineto{\pgfqpoint{3.365463in}{4.364381in}}%
\pgfpathlineto{\pgfqpoint{3.351964in}{4.391569in}}%
\pgfpathlineto{\pgfqpoint{3.338456in}{4.419117in}}%
\pgfpathlineto{\pgfqpoint{3.330782in}{4.389987in}}%
\pgfpathlineto{\pgfqpoint{3.323104in}{4.361289in}}%
\pgfpathlineto{\pgfqpoint{3.315420in}{4.333017in}}%
\pgfpathlineto{\pgfqpoint{3.307731in}{4.305163in}}%
\pgfpathclose%
\pgfusepath{fill}%
\end{pgfscope}%
\begin{pgfscope}%
\pgfpathrectangle{\pgfqpoint{1.150000in}{0.150000in}}{\pgfqpoint{5.700000in}{5.700000in}}%
\pgfusepath{clip}%
\pgfsetbuttcap%
\pgfsetroundjoin%
\definecolor{currentfill}{rgb}{0.458674,0.816363,0.329727}%
\pgfsetfillcolor{currentfill}%
\pgfsetfillopacity{0.800000}%
\pgfsetlinewidth{0.000000pt}%
\definecolor{currentstroke}{rgb}{0.000000,0.000000,0.000000}%
\pgfsetstrokecolor{currentstroke}%
\pgfsetdash{}{0pt}%
\pgfpathmoveto{\pgfqpoint{3.453678in}{4.555212in}}%
\pgfpathlineto{\pgfqpoint{3.467156in}{4.527732in}}%
\pgfpathlineto{\pgfqpoint{3.480625in}{4.500601in}}%
\pgfpathlineto{\pgfqpoint{3.494084in}{4.473814in}}%
\pgfpathlineto{\pgfqpoint{3.507534in}{4.447369in}}%
\pgfpathlineto{\pgfqpoint{3.515178in}{4.479312in}}%
\pgfpathlineto{\pgfqpoint{3.522818in}{4.511748in}}%
\pgfpathlineto{\pgfqpoint{3.530455in}{4.544686in}}%
\pgfpathlineto{\pgfqpoint{3.538088in}{4.578136in}}%
\pgfpathlineto{\pgfqpoint{3.524631in}{4.605310in}}%
\pgfpathlineto{\pgfqpoint{3.511165in}{4.632828in}}%
\pgfpathlineto{\pgfqpoint{3.497689in}{4.660693in}}%
\pgfpathlineto{\pgfqpoint{3.484204in}{4.688908in}}%
\pgfpathlineto{\pgfqpoint{3.476578in}{4.654710in}}%
\pgfpathlineto{\pgfqpoint{3.468948in}{4.621035in}}%
\pgfpathlineto{\pgfqpoint{3.461315in}{4.587871in}}%
\pgfpathlineto{\pgfqpoint{3.453678in}{4.555212in}}%
\pgfpathclose%
\pgfusepath{fill}%
\end{pgfscope}%
\begin{pgfscope}%
\pgfpathrectangle{\pgfqpoint{1.150000in}{0.150000in}}{\pgfqpoint{5.700000in}{5.700000in}}%
\pgfusepath{clip}%
\pgfsetbuttcap%
\pgfsetroundjoin%
\definecolor{currentfill}{rgb}{0.150148,0.676631,0.506589}%
\pgfsetfillcolor{currentfill}%
\pgfsetfillopacity{0.800000}%
\pgfsetlinewidth{0.000000pt}%
\definecolor{currentstroke}{rgb}{0.000000,0.000000,0.000000}%
\pgfsetstrokecolor{currentstroke}%
\pgfsetdash{}{0pt}%
\pgfpathmoveto{\pgfqpoint{3.246013in}{4.096560in}}%
\pgfpathlineto{\pgfqpoint{3.259516in}{4.070759in}}%
\pgfpathlineto{\pgfqpoint{3.273010in}{4.045313in}}%
\pgfpathlineto{\pgfqpoint{3.286493in}{4.020219in}}%
\pgfpathlineto{\pgfqpoint{3.299967in}{3.995472in}}%
\pgfpathlineto{\pgfqpoint{3.307702in}{4.019699in}}%
\pgfpathlineto{\pgfqpoint{3.315431in}{4.044281in}}%
\pgfpathlineto{\pgfqpoint{3.323155in}{4.069224in}}%
\pgfpathlineto{\pgfqpoint{3.330873in}{4.094535in}}%
\pgfpathlineto{\pgfqpoint{3.317399in}{4.119822in}}%
\pgfpathlineto{\pgfqpoint{3.303916in}{4.145458in}}%
\pgfpathlineto{\pgfqpoint{3.290422in}{4.171446in}}%
\pgfpathlineto{\pgfqpoint{3.276919in}{4.197791in}}%
\pgfpathlineto{\pgfqpoint{3.269201in}{4.171924in}}%
\pgfpathlineto{\pgfqpoint{3.261478in}{4.146434in}}%
\pgfpathlineto{\pgfqpoint{3.253748in}{4.121315in}}%
\pgfpathlineto{\pgfqpoint{3.246013in}{4.096560in}}%
\pgfpathclose%
\pgfusepath{fill}%
\end{pgfscope}%
\begin{pgfscope}%
\pgfpathrectangle{\pgfqpoint{1.150000in}{0.150000in}}{\pgfqpoint{5.700000in}{5.700000in}}%
\pgfusepath{clip}%
\pgfsetbuttcap%
\pgfsetroundjoin%
\definecolor{currentfill}{rgb}{0.243113,0.292092,0.538516}%
\pgfsetfillcolor{currentfill}%
\pgfsetfillopacity{0.800000}%
\pgfsetlinewidth{0.000000pt}%
\definecolor{currentstroke}{rgb}{0.000000,0.000000,0.000000}%
\pgfsetstrokecolor{currentstroke}%
\pgfsetdash{}{0pt}%
\pgfpathmoveto{\pgfqpoint{3.710058in}{2.982387in}}%
\pgfpathlineto{\pgfqpoint{3.723375in}{2.971201in}}%
\pgfpathlineto{\pgfqpoint{3.736693in}{2.960256in}}%
\pgfpathlineto{\pgfqpoint{3.750011in}{2.949553in}}%
\pgfpathlineto{\pgfqpoint{3.763330in}{2.939090in}}%
\pgfpathlineto{\pgfqpoint{3.771103in}{2.954318in}}%
\pgfpathlineto{\pgfqpoint{3.778872in}{2.969724in}}%
\pgfpathlineto{\pgfqpoint{3.786637in}{2.985312in}}%
\pgfpathlineto{\pgfqpoint{3.794397in}{3.001086in}}%
\pgfpathlineto{\pgfqpoint{3.781084in}{3.011926in}}%
\pgfpathlineto{\pgfqpoint{3.767772in}{3.023006in}}%
\pgfpathlineto{\pgfqpoint{3.754461in}{3.034328in}}%
\pgfpathlineto{\pgfqpoint{3.741149in}{3.045893in}}%
\pgfpathlineto{\pgfqpoint{3.733383in}{3.029729in}}%
\pgfpathlineto{\pgfqpoint{3.725613in}{3.013760in}}%
\pgfpathlineto{\pgfqpoint{3.717838in}{2.997981in}}%
\pgfpathlineto{\pgfqpoint{3.710058in}{2.982387in}}%
\pgfpathclose%
\pgfusepath{fill}%
\end{pgfscope}%
\begin{pgfscope}%
\pgfpathrectangle{\pgfqpoint{1.150000in}{0.150000in}}{\pgfqpoint{5.700000in}{5.700000in}}%
\pgfusepath{clip}%
\pgfsetbuttcap%
\pgfsetroundjoin%
\definecolor{currentfill}{rgb}{0.208623,0.367752,0.552675}%
\pgfsetfillcolor{currentfill}%
\pgfsetfillopacity{0.800000}%
\pgfsetlinewidth{0.000000pt}%
\definecolor{currentstroke}{rgb}{0.000000,0.000000,0.000000}%
\pgfsetstrokecolor{currentstroke}%
\pgfsetdash{}{0pt}%
\pgfpathmoveto{\pgfqpoint{3.412233in}{3.192243in}}%
\pgfpathlineto{\pgfqpoint{3.425588in}{3.176170in}}%
\pgfpathlineto{\pgfqpoint{3.438940in}{3.160378in}}%
\pgfpathlineto{\pgfqpoint{3.452288in}{3.144865in}}%
\pgfpathlineto{\pgfqpoint{3.465633in}{3.129628in}}%
\pgfpathlineto{\pgfqpoint{3.473457in}{3.145868in}}%
\pgfpathlineto{\pgfqpoint{3.481275in}{3.162308in}}%
\pgfpathlineto{\pgfqpoint{3.489089in}{3.178952in}}%
\pgfpathlineto{\pgfqpoint{3.496897in}{3.195805in}}%
\pgfpathlineto{\pgfqpoint{3.483558in}{3.211392in}}%
\pgfpathlineto{\pgfqpoint{3.470216in}{3.227256in}}%
\pgfpathlineto{\pgfqpoint{3.456870in}{3.243399in}}%
\pgfpathlineto{\pgfqpoint{3.443521in}{3.259823in}}%
\pgfpathlineto{\pgfqpoint{3.435707in}{3.242607in}}%
\pgfpathlineto{\pgfqpoint{3.427888in}{3.225608in}}%
\pgfpathlineto{\pgfqpoint{3.420063in}{3.208821in}}%
\pgfpathlineto{\pgfqpoint{3.412233in}{3.192243in}}%
\pgfpathclose%
\pgfusepath{fill}%
\end{pgfscope}%
\begin{pgfscope}%
\pgfpathrectangle{\pgfqpoint{1.150000in}{0.150000in}}{\pgfqpoint{5.700000in}{5.700000in}}%
\pgfusepath{clip}%
\pgfsetbuttcap%
\pgfsetroundjoin%
\definecolor{currentfill}{rgb}{0.175841,0.441290,0.557685}%
\pgfsetfillcolor{currentfill}%
\pgfsetfillopacity{0.800000}%
\pgfsetlinewidth{0.000000pt}%
\definecolor{currentstroke}{rgb}{0.000000,0.000000,0.000000}%
\pgfsetstrokecolor{currentstroke}%
\pgfsetdash{}{0pt}%
\pgfpathmoveto{\pgfqpoint{4.826896in}{3.378099in}}%
\pgfpathlineto{\pgfqpoint{4.840369in}{3.372144in}}%
\pgfpathlineto{\pgfqpoint{4.853851in}{3.366376in}}%
\pgfpathlineto{\pgfqpoint{4.867340in}{3.360794in}}%
\pgfpathlineto{\pgfqpoint{4.880837in}{3.355398in}}%
\pgfpathlineto{\pgfqpoint{4.888441in}{3.375322in}}%
\pgfpathlineto{\pgfqpoint{4.896050in}{3.395626in}}%
\pgfpathlineto{\pgfqpoint{4.903663in}{3.416321in}}%
\pgfpathlineto{\pgfqpoint{4.890175in}{3.422279in}}%
\pgfpathlineto{\pgfqpoint{4.876695in}{3.428422in}}%
\pgfpathlineto{\pgfqpoint{4.863223in}{3.434752in}}%
\pgfpathlineto{\pgfqpoint{4.849758in}{3.441269in}}%
\pgfpathlineto{\pgfqpoint{4.842133in}{3.419817in}}%
\pgfpathlineto{\pgfqpoint{4.834512in}{3.398763in}}%
\pgfpathlineto{\pgfqpoint{4.826896in}{3.378099in}}%
\pgfpathclose%
\pgfusepath{fill}%
\end{pgfscope}%
\begin{pgfscope}%
\pgfpathrectangle{\pgfqpoint{1.150000in}{0.150000in}}{\pgfqpoint{5.700000in}{5.700000in}}%
\pgfusepath{clip}%
\pgfsetbuttcap%
\pgfsetroundjoin%
\definecolor{currentfill}{rgb}{0.218130,0.347432,0.550038}%
\pgfsetfillcolor{currentfill}%
\pgfsetfillopacity{0.800000}%
\pgfsetlinewidth{0.000000pt}%
\definecolor{currentstroke}{rgb}{0.000000,0.000000,0.000000}%
\pgfsetstrokecolor{currentstroke}%
\pgfsetdash{}{0pt}%
\pgfpathmoveto{\pgfqpoint{3.465633in}{3.129628in}}%
\pgfpathlineto{\pgfqpoint{3.478975in}{3.114666in}}%
\pgfpathlineto{\pgfqpoint{3.492314in}{3.099976in}}%
\pgfpathlineto{\pgfqpoint{3.505650in}{3.085556in}}%
\pgfpathlineto{\pgfqpoint{3.518985in}{3.071404in}}%
\pgfpathlineto{\pgfqpoint{3.526802in}{3.087306in}}%
\pgfpathlineto{\pgfqpoint{3.534615in}{3.103401in}}%
\pgfpathlineto{\pgfqpoint{3.542422in}{3.119691in}}%
\pgfpathlineto{\pgfqpoint{3.550224in}{3.136183in}}%
\pgfpathlineto{\pgfqpoint{3.536896in}{3.150684in}}%
\pgfpathlineto{\pgfqpoint{3.523566in}{3.165453in}}%
\pgfpathlineto{\pgfqpoint{3.510232in}{3.180493in}}%
\pgfpathlineto{\pgfqpoint{3.496897in}{3.195805in}}%
\pgfpathlineto{\pgfqpoint{3.489089in}{3.178952in}}%
\pgfpathlineto{\pgfqpoint{3.481275in}{3.162308in}}%
\pgfpathlineto{\pgfqpoint{3.473457in}{3.145868in}}%
\pgfpathlineto{\pgfqpoint{3.465633in}{3.129628in}}%
\pgfpathclose%
\pgfusepath{fill}%
\end{pgfscope}%
\begin{pgfscope}%
\pgfpathrectangle{\pgfqpoint{1.150000in}{0.150000in}}{\pgfqpoint{5.700000in}{5.700000in}}%
\pgfusepath{clip}%
\pgfsetbuttcap%
\pgfsetroundjoin%
\definecolor{currentfill}{rgb}{0.220057,0.343307,0.549413}%
\pgfsetfillcolor{currentfill}%
\pgfsetfillopacity{0.800000}%
\pgfsetlinewidth{0.000000pt}%
\definecolor{currentstroke}{rgb}{0.000000,0.000000,0.000000}%
\pgfsetstrokecolor{currentstroke}%
\pgfsetdash{}{0pt}%
\pgfpathmoveto{\pgfqpoint{4.459516in}{3.102444in}}%
\pgfpathlineto{\pgfqpoint{4.472930in}{3.096697in}}%
\pgfpathlineto{\pgfqpoint{4.486350in}{3.091148in}}%
\pgfpathlineto{\pgfqpoint{4.499777in}{3.085795in}}%
\pgfpathlineto{\pgfqpoint{4.513210in}{3.080639in}}%
\pgfpathlineto{\pgfqpoint{4.520833in}{3.096971in}}%
\pgfpathlineto{\pgfqpoint{4.528455in}{3.113559in}}%
\pgfpathlineto{\pgfqpoint{4.536076in}{3.130410in}}%
\pgfpathlineto{\pgfqpoint{4.543697in}{3.147532in}}%
\pgfpathlineto{\pgfqpoint{4.530273in}{3.153283in}}%
\pgfpathlineto{\pgfqpoint{4.516855in}{3.159229in}}%
\pgfpathlineto{\pgfqpoint{4.503444in}{3.165373in}}%
\pgfpathlineto{\pgfqpoint{4.490039in}{3.171714in}}%
\pgfpathlineto{\pgfqpoint{4.482409in}{3.153986in}}%
\pgfpathlineto{\pgfqpoint{4.474778in}{3.136536in}}%
\pgfpathlineto{\pgfqpoint{4.467147in}{3.119358in}}%
\pgfpathlineto{\pgfqpoint{4.459516in}{3.102444in}}%
\pgfpathclose%
\pgfusepath{fill}%
\end{pgfscope}%
\begin{pgfscope}%
\pgfpathrectangle{\pgfqpoint{1.150000in}{0.150000in}}{\pgfqpoint{5.700000in}{5.700000in}}%
\pgfusepath{clip}%
\pgfsetbuttcap%
\pgfsetroundjoin%
\definecolor{currentfill}{rgb}{0.212395,0.359683,0.551710}%
\pgfsetfillcolor{currentfill}%
\pgfsetfillopacity{0.800000}%
\pgfsetlinewidth{0.000000pt}%
\definecolor{currentstroke}{rgb}{0.000000,0.000000,0.000000}%
\pgfsetstrokecolor{currentstroke}%
\pgfsetdash{}{0pt}%
\pgfpathmoveto{\pgfqpoint{4.543697in}{3.147532in}}%
\pgfpathlineto{\pgfqpoint{4.557128in}{3.141977in}}%
\pgfpathlineto{\pgfqpoint{4.570566in}{3.136617in}}%
\pgfpathlineto{\pgfqpoint{4.584010in}{3.131451in}}%
\pgfpathlineto{\pgfqpoint{4.597462in}{3.126479in}}%
\pgfpathlineto{\pgfqpoint{4.605073in}{3.143264in}}%
\pgfpathlineto{\pgfqpoint{4.612684in}{3.160327in}}%
\pgfpathlineto{\pgfqpoint{4.620295in}{3.177674in}}%
\pgfpathlineto{\pgfqpoint{4.627907in}{3.195313in}}%
\pgfpathlineto{\pgfqpoint{4.614465in}{3.200911in}}%
\pgfpathlineto{\pgfqpoint{4.601031in}{3.206703in}}%
\pgfpathlineto{\pgfqpoint{4.587603in}{3.212689in}}%
\pgfpathlineto{\pgfqpoint{4.574181in}{3.218869in}}%
\pgfpathlineto{\pgfqpoint{4.566560in}{3.200593in}}%
\pgfpathlineto{\pgfqpoint{4.558939in}{3.182616in}}%
\pgfpathlineto{\pgfqpoint{4.551318in}{3.164932in}}%
\pgfpathlineto{\pgfqpoint{4.543697in}{3.147532in}}%
\pgfpathclose%
\pgfusepath{fill}%
\end{pgfscope}%
\begin{pgfscope}%
\pgfpathrectangle{\pgfqpoint{1.150000in}{0.150000in}}{\pgfqpoint{5.700000in}{5.700000in}}%
\pgfusepath{clip}%
\pgfsetbuttcap%
\pgfsetroundjoin%
\definecolor{currentfill}{rgb}{0.227802,0.326594,0.546532}%
\pgfsetfillcolor{currentfill}%
\pgfsetfillopacity{0.800000}%
\pgfsetlinewidth{0.000000pt}%
\definecolor{currentstroke}{rgb}{0.000000,0.000000,0.000000}%
\pgfsetstrokecolor{currentstroke}%
\pgfsetdash{}{0pt}%
\pgfpathmoveto{\pgfqpoint{4.375349in}{3.060029in}}%
\pgfpathlineto{\pgfqpoint{4.388747in}{3.054046in}}%
\pgfpathlineto{\pgfqpoint{4.402151in}{3.048264in}}%
\pgfpathlineto{\pgfqpoint{4.415561in}{3.042682in}}%
\pgfpathlineto{\pgfqpoint{4.428977in}{3.037299in}}%
\pgfpathlineto{\pgfqpoint{4.436614in}{3.053222in}}%
\pgfpathlineto{\pgfqpoint{4.444249in}{3.069383in}}%
\pgfpathlineto{\pgfqpoint{4.451883in}{3.085788in}}%
\pgfpathlineto{\pgfqpoint{4.459516in}{3.102444in}}%
\pgfpathlineto{\pgfqpoint{4.446108in}{3.108390in}}%
\pgfpathlineto{\pgfqpoint{4.432706in}{3.114535in}}%
\pgfpathlineto{\pgfqpoint{4.419311in}{3.120880in}}%
\pgfpathlineto{\pgfqpoint{4.405921in}{3.127426in}}%
\pgfpathlineto{\pgfqpoint{4.398279in}{3.110195in}}%
\pgfpathlineto{\pgfqpoint{4.390637in}{3.093223in}}%
\pgfpathlineto{\pgfqpoint{4.382994in}{3.076504in}}%
\pgfpathlineto{\pgfqpoint{4.375349in}{3.060029in}}%
\pgfpathclose%
\pgfusepath{fill}%
\end{pgfscope}%
\begin{pgfscope}%
\pgfpathrectangle{\pgfqpoint{1.150000in}{0.150000in}}{\pgfqpoint{5.700000in}{5.700000in}}%
\pgfusepath{clip}%
\pgfsetbuttcap%
\pgfsetroundjoin%
\definecolor{currentfill}{rgb}{0.626579,0.854645,0.223353}%
\pgfsetfillcolor{currentfill}%
\pgfsetfillopacity{0.800000}%
\pgfsetlinewidth{0.000000pt}%
\definecolor{currentstroke}{rgb}{0.000000,0.000000,0.000000}%
\pgfsetstrokecolor{currentstroke}%
\pgfsetdash{}{0pt}%
\pgfpathmoveto{\pgfqpoint{3.652868in}{4.753606in}}%
\pgfpathlineto{\pgfqpoint{3.666300in}{4.726533in}}%
\pgfpathlineto{\pgfqpoint{3.679723in}{4.699787in}}%
\pgfpathlineto{\pgfqpoint{3.693139in}{4.673366in}}%
\pgfpathlineto{\pgfqpoint{3.706547in}{4.647267in}}%
\pgfpathlineto{\pgfqpoint{3.714178in}{4.684073in}}%
\pgfpathlineto{\pgfqpoint{3.721809in}{4.721465in}}%
\pgfpathlineto{\pgfqpoint{3.729438in}{4.759453in}}%
\pgfpathlineto{\pgfqpoint{3.716023in}{4.786179in}}%
\pgfpathlineto{\pgfqpoint{3.702600in}{4.813228in}}%
\pgfpathlineto{\pgfqpoint{3.689169in}{4.840604in}}%
\pgfpathlineto{\pgfqpoint{3.675730in}{4.868308in}}%
\pgfpathlineto{\pgfqpoint{3.668110in}{4.829471in}}%
\pgfpathlineto{\pgfqpoint{3.660490in}{4.791240in}}%
\pgfpathlineto{\pgfqpoint{3.652868in}{4.753606in}}%
\pgfpathclose%
\pgfusepath{fill}%
\end{pgfscope}%
\begin{pgfscope}%
\pgfpathrectangle{\pgfqpoint{1.150000in}{0.150000in}}{\pgfqpoint{5.700000in}{5.700000in}}%
\pgfusepath{clip}%
\pgfsetbuttcap%
\pgfsetroundjoin%
\definecolor{currentfill}{rgb}{0.195860,0.395433,0.555276}%
\pgfsetfillcolor{currentfill}%
\pgfsetfillopacity{0.800000}%
\pgfsetlinewidth{0.000000pt}%
\definecolor{currentstroke}{rgb}{0.000000,0.000000,0.000000}%
\pgfsetstrokecolor{currentstroke}%
\pgfsetdash{}{0pt}%
\pgfpathmoveto{\pgfqpoint{3.358771in}{3.259393in}}%
\pgfpathlineto{\pgfqpoint{3.372143in}{3.242172in}}%
\pgfpathlineto{\pgfqpoint{3.385511in}{3.225242in}}%
\pgfpathlineto{\pgfqpoint{3.398874in}{3.208599in}}%
\pgfpathlineto{\pgfqpoint{3.412233in}{3.192243in}}%
\pgfpathlineto{\pgfqpoint{3.420063in}{3.208821in}}%
\pgfpathlineto{\pgfqpoint{3.427888in}{3.225608in}}%
\pgfpathlineto{\pgfqpoint{3.435707in}{3.242607in}}%
\pgfpathlineto{\pgfqpoint{3.443521in}{3.259823in}}%
\pgfpathlineto{\pgfqpoint{3.430168in}{3.276531in}}%
\pgfpathlineto{\pgfqpoint{3.416810in}{3.293526in}}%
\pgfpathlineto{\pgfqpoint{3.403449in}{3.310809in}}%
\pgfpathlineto{\pgfqpoint{3.390082in}{3.328383in}}%
\pgfpathlineto{\pgfqpoint{3.382263in}{3.310802in}}%
\pgfpathlineto{\pgfqpoint{3.374438in}{3.293446in}}%
\pgfpathlineto{\pgfqpoint{3.366607in}{3.276311in}}%
\pgfpathlineto{\pgfqpoint{3.358771in}{3.259393in}}%
\pgfpathclose%
\pgfusepath{fill}%
\end{pgfscope}%
\begin{pgfscope}%
\pgfpathrectangle{\pgfqpoint{1.150000in}{0.150000in}}{\pgfqpoint{5.700000in}{5.700000in}}%
\pgfusepath{clip}%
\pgfsetbuttcap%
\pgfsetroundjoin%
\definecolor{currentfill}{rgb}{0.248629,0.278775,0.534556}%
\pgfsetfillcolor{currentfill}%
\pgfsetfillopacity{0.800000}%
\pgfsetlinewidth{0.000000pt}%
\definecolor{currentstroke}{rgb}{0.000000,0.000000,0.000000}%
\pgfsetstrokecolor{currentstroke}%
\pgfsetdash{}{0pt}%
\pgfpathmoveto{\pgfqpoint{3.985217in}{2.949331in}}%
\pgfpathlineto{\pgfqpoint{3.998552in}{2.941070in}}%
\pgfpathlineto{\pgfqpoint{4.011889in}{2.933031in}}%
\pgfpathlineto{\pgfqpoint{4.025231in}{2.925211in}}%
\pgfpathlineto{\pgfqpoint{4.038576in}{2.917610in}}%
\pgfpathlineto{\pgfqpoint{4.046291in}{2.932678in}}%
\pgfpathlineto{\pgfqpoint{4.054004in}{2.947930in}}%
\pgfpathlineto{\pgfqpoint{4.061713in}{2.963372in}}%
\pgfpathlineto{\pgfqpoint{4.069418in}{2.979008in}}%
\pgfpathlineto{\pgfqpoint{4.056080in}{2.987047in}}%
\pgfpathlineto{\pgfqpoint{4.042745in}{2.995305in}}%
\pgfpathlineto{\pgfqpoint{4.029414in}{3.003783in}}%
\pgfpathlineto{\pgfqpoint{4.016086in}{3.012482in}}%
\pgfpathlineto{\pgfqpoint{4.008374in}{2.996396in}}%
\pgfpathlineto{\pgfqpoint{4.000658in}{2.980512in}}%
\pgfpathlineto{\pgfqpoint{3.992939in}{2.964825in}}%
\pgfpathlineto{\pgfqpoint{3.985217in}{2.949331in}}%
\pgfpathclose%
\pgfusepath{fill}%
\end{pgfscope}%
\begin{pgfscope}%
\pgfpathrectangle{\pgfqpoint{1.150000in}{0.150000in}}{\pgfqpoint{5.700000in}{5.700000in}}%
\pgfusepath{clip}%
\pgfsetbuttcap%
\pgfsetroundjoin%
\definecolor{currentfill}{rgb}{0.227802,0.326594,0.546532}%
\pgfsetfillcolor{currentfill}%
\pgfsetfillopacity{0.800000}%
\pgfsetlinewidth{0.000000pt}%
\definecolor{currentstroke}{rgb}{0.000000,0.000000,0.000000}%
\pgfsetstrokecolor{currentstroke}%
\pgfsetdash{}{0pt}%
\pgfpathmoveto{\pgfqpoint{3.518985in}{3.071404in}}%
\pgfpathlineto{\pgfqpoint{3.532317in}{3.057518in}}%
\pgfpathlineto{\pgfqpoint{3.545647in}{3.043896in}}%
\pgfpathlineto{\pgfqpoint{3.558976in}{3.030536in}}%
\pgfpathlineto{\pgfqpoint{3.572303in}{3.017436in}}%
\pgfpathlineto{\pgfqpoint{3.580114in}{3.033002in}}%
\pgfpathlineto{\pgfqpoint{3.587920in}{3.048752in}}%
\pgfpathlineto{\pgfqpoint{3.595722in}{3.064690in}}%
\pgfpathlineto{\pgfqpoint{3.603518in}{3.080821in}}%
\pgfpathlineto{\pgfqpoint{3.590197in}{3.094269in}}%
\pgfpathlineto{\pgfqpoint{3.576875in}{3.107977in}}%
\pgfpathlineto{\pgfqpoint{3.563550in}{3.121948in}}%
\pgfpathlineto{\pgfqpoint{3.550224in}{3.136183in}}%
\pgfpathlineto{\pgfqpoint{3.542422in}{3.119691in}}%
\pgfpathlineto{\pgfqpoint{3.534615in}{3.103401in}}%
\pgfpathlineto{\pgfqpoint{3.526802in}{3.087306in}}%
\pgfpathlineto{\pgfqpoint{3.518985in}{3.071404in}}%
\pgfpathclose%
\pgfusepath{fill}%
\end{pgfscope}%
\begin{pgfscope}%
\pgfpathrectangle{\pgfqpoint{1.150000in}{0.150000in}}{\pgfqpoint{5.700000in}{5.700000in}}%
\pgfusepath{clip}%
\pgfsetbuttcap%
\pgfsetroundjoin%
\definecolor{currentfill}{rgb}{0.344074,0.780029,0.397381}%
\pgfsetfillcolor{currentfill}%
\pgfsetfillopacity{0.800000}%
\pgfsetlinewidth{0.000000pt}%
\definecolor{currentstroke}{rgb}{0.000000,0.000000,0.000000}%
\pgfsetstrokecolor{currentstroke}%
\pgfsetdash{}{0pt}%
\pgfpathmoveto{\pgfqpoint{3.338456in}{4.419117in}}%
\pgfpathlineto{\pgfqpoint{3.351964in}{4.391569in}}%
\pgfpathlineto{\pgfqpoint{3.365463in}{4.364381in}}%
\pgfpathlineto{\pgfqpoint{3.378951in}{4.337548in}}%
\pgfpathlineto{\pgfqpoint{3.392429in}{4.311068in}}%
\pgfpathlineto{\pgfqpoint{3.400101in}{4.339996in}}%
\pgfpathlineto{\pgfqpoint{3.407768in}{4.369363in}}%
\pgfpathlineto{\pgfqpoint{3.415430in}{4.399176in}}%
\pgfpathlineto{\pgfqpoint{3.423088in}{4.429443in}}%
\pgfpathlineto{\pgfqpoint{3.409606in}{4.456579in}}%
\pgfpathlineto{\pgfqpoint{3.396114in}{4.484069in}}%
\pgfpathlineto{\pgfqpoint{3.382611in}{4.511916in}}%
\pgfpathlineto{\pgfqpoint{3.369097in}{4.540125in}}%
\pgfpathlineto{\pgfqpoint{3.361444in}{4.509185in}}%
\pgfpathlineto{\pgfqpoint{3.353786in}{4.478709in}}%
\pgfpathlineto{\pgfqpoint{3.346124in}{4.448689in}}%
\pgfpathlineto{\pgfqpoint{3.338456in}{4.419117in}}%
\pgfpathclose%
\pgfusepath{fill}%
\end{pgfscope}%
\begin{pgfscope}%
\pgfpathrectangle{\pgfqpoint{1.150000in}{0.150000in}}{\pgfqpoint{5.700000in}{5.700000in}}%
\pgfusepath{clip}%
\pgfsetbuttcap%
\pgfsetroundjoin%
\definecolor{currentfill}{rgb}{0.235526,0.309527,0.542944}%
\pgfsetfillcolor{currentfill}%
\pgfsetfillopacity{0.800000}%
\pgfsetlinewidth{0.000000pt}%
\definecolor{currentstroke}{rgb}{0.000000,0.000000,0.000000}%
\pgfsetstrokecolor{currentstroke}%
\pgfsetdash{}{0pt}%
\pgfpathmoveto{\pgfqpoint{4.291184in}{3.020293in}}%
\pgfpathlineto{\pgfqpoint{4.304567in}{3.014029in}}%
\pgfpathlineto{\pgfqpoint{4.317956in}{3.007970in}}%
\pgfpathlineto{\pgfqpoint{4.331351in}{3.002113in}}%
\pgfpathlineto{\pgfqpoint{4.344751in}{2.996460in}}%
\pgfpathlineto{\pgfqpoint{4.352403in}{3.012016in}}%
\pgfpathlineto{\pgfqpoint{4.360054in}{3.027792in}}%
\pgfpathlineto{\pgfqpoint{4.367702in}{3.043794in}}%
\pgfpathlineto{\pgfqpoint{4.375349in}{3.060029in}}%
\pgfpathlineto{\pgfqpoint{4.361956in}{3.066214in}}%
\pgfpathlineto{\pgfqpoint{4.348570in}{3.072602in}}%
\pgfpathlineto{\pgfqpoint{4.335189in}{3.079193in}}%
\pgfpathlineto{\pgfqpoint{4.321813in}{3.085989in}}%
\pgfpathlineto{\pgfqpoint{4.314158in}{3.069211in}}%
\pgfpathlineto{\pgfqpoint{4.306502in}{3.052673in}}%
\pgfpathlineto{\pgfqpoint{4.298844in}{3.036369in}}%
\pgfpathlineto{\pgfqpoint{4.291184in}{3.020293in}}%
\pgfpathclose%
\pgfusepath{fill}%
\end{pgfscope}%
\begin{pgfscope}%
\pgfpathrectangle{\pgfqpoint{1.150000in}{0.150000in}}{\pgfqpoint{5.700000in}{5.700000in}}%
\pgfusepath{clip}%
\pgfsetbuttcap%
\pgfsetroundjoin%
\definecolor{currentfill}{rgb}{0.203063,0.379716,0.553925}%
\pgfsetfillcolor{currentfill}%
\pgfsetfillopacity{0.800000}%
\pgfsetlinewidth{0.000000pt}%
\definecolor{currentstroke}{rgb}{0.000000,0.000000,0.000000}%
\pgfsetstrokecolor{currentstroke}%
\pgfsetdash{}{0pt}%
\pgfpathmoveto{\pgfqpoint{4.627907in}{3.195313in}}%
\pgfpathlineto{\pgfqpoint{4.641356in}{3.189907in}}%
\pgfpathlineto{\pgfqpoint{4.654811in}{3.184694in}}%
\pgfpathlineto{\pgfqpoint{4.668274in}{3.179671in}}%
\pgfpathlineto{\pgfqpoint{4.681744in}{3.174839in}}%
\pgfpathlineto{\pgfqpoint{4.689346in}{3.192132in}}%
\pgfpathlineto{\pgfqpoint{4.696949in}{3.209723in}}%
\pgfpathlineto{\pgfqpoint{4.704552in}{3.227620in}}%
\pgfpathlineto{\pgfqpoint{4.712157in}{3.245832in}}%
\pgfpathlineto{\pgfqpoint{4.698698in}{3.251321in}}%
\pgfpathlineto{\pgfqpoint{4.685246in}{3.257001in}}%
\pgfpathlineto{\pgfqpoint{4.671800in}{3.262872in}}%
\pgfpathlineto{\pgfqpoint{4.658362in}{3.268935in}}%
\pgfpathlineto{\pgfqpoint{4.650746in}{3.250054in}}%
\pgfpathlineto{\pgfqpoint{4.643132in}{3.231495in}}%
\pgfpathlineto{\pgfqpoint{4.635519in}{3.213251in}}%
\pgfpathlineto{\pgfqpoint{4.627907in}{3.195313in}}%
\pgfpathclose%
\pgfusepath{fill}%
\end{pgfscope}%
\begin{pgfscope}%
\pgfpathrectangle{\pgfqpoint{1.150000in}{0.150000in}}{\pgfqpoint{5.700000in}{5.700000in}}%
\pgfusepath{clip}%
\pgfsetbuttcap%
\pgfsetroundjoin%
\definecolor{currentfill}{rgb}{0.128087,0.647749,0.523491}%
\pgfsetfillcolor{currentfill}%
\pgfsetfillopacity{0.800000}%
\pgfsetlinewidth{0.000000pt}%
\definecolor{currentstroke}{rgb}{0.000000,0.000000,0.000000}%
\pgfsetstrokecolor{currentstroke}%
\pgfsetdash{}{0pt}%
\pgfpathmoveto{\pgfqpoint{3.215010in}{4.001057in}}%
\pgfpathlineto{\pgfqpoint{3.228514in}{3.975763in}}%
\pgfpathlineto{\pgfqpoint{3.242008in}{3.950823in}}%
\pgfpathlineto{\pgfqpoint{3.255493in}{3.926233in}}%
\pgfpathlineto{\pgfqpoint{3.268968in}{3.901990in}}%
\pgfpathlineto{\pgfqpoint{3.276726in}{3.924859in}}%
\pgfpathlineto{\pgfqpoint{3.284479in}{3.948058in}}%
\pgfpathlineto{\pgfqpoint{3.292226in}{3.971594in}}%
\pgfpathlineto{\pgfqpoint{3.299967in}{3.995472in}}%
\pgfpathlineto{\pgfqpoint{3.286493in}{4.020219in}}%
\pgfpathlineto{\pgfqpoint{3.273010in}{4.045313in}}%
\pgfpathlineto{\pgfqpoint{3.259516in}{4.070759in}}%
\pgfpathlineto{\pgfqpoint{3.246013in}{4.096560in}}%
\pgfpathlineto{\pgfqpoint{3.238271in}{4.072162in}}%
\pgfpathlineto{\pgfqpoint{3.230524in}{4.048117in}}%
\pgfpathlineto{\pgfqpoint{3.222770in}{4.024417in}}%
\pgfpathlineto{\pgfqpoint{3.215010in}{4.001057in}}%
\pgfpathclose%
\pgfusepath{fill}%
\end{pgfscope}%
\begin{pgfscope}%
\pgfpathrectangle{\pgfqpoint{1.150000in}{0.150000in}}{\pgfqpoint{5.700000in}{5.700000in}}%
\pgfusepath{clip}%
\pgfsetbuttcap%
\pgfsetroundjoin%
\definecolor{currentfill}{rgb}{0.131172,0.555899,0.552459}%
\pgfsetfillcolor{currentfill}%
\pgfsetfillopacity{0.800000}%
\pgfsetlinewidth{0.000000pt}%
\definecolor{currentstroke}{rgb}{0.000000,0.000000,0.000000}%
\pgfsetstrokecolor{currentstroke}%
\pgfsetdash{}{0pt}%
\pgfpathmoveto{\pgfqpoint{3.206676in}{3.730271in}}%
\pgfpathlineto{\pgfqpoint{3.220148in}{3.707268in}}%
\pgfpathlineto{\pgfqpoint{3.233611in}{3.684604in}}%
\pgfpathlineto{\pgfqpoint{3.247064in}{3.662275in}}%
\pgfpathlineto{\pgfqpoint{3.260510in}{3.640277in}}%
\pgfpathlineto{\pgfqpoint{3.268314in}{3.660283in}}%
\pgfpathlineto{\pgfqpoint{3.276113in}{3.680569in}}%
\pgfpathlineto{\pgfqpoint{3.283905in}{3.701137in}}%
\pgfpathlineto{\pgfqpoint{3.291691in}{3.721995in}}%
\pgfpathlineto{\pgfqpoint{3.278249in}{3.744421in}}%
\pgfpathlineto{\pgfqpoint{3.264799in}{3.767180in}}%
\pgfpathlineto{\pgfqpoint{3.251340in}{3.790275in}}%
\pgfpathlineto{\pgfqpoint{3.237872in}{3.813708in}}%
\pgfpathlineto{\pgfqpoint{3.230083in}{3.792407in}}%
\pgfpathlineto{\pgfqpoint{3.222287in}{3.771405in}}%
\pgfpathlineto{\pgfqpoint{3.214485in}{3.750694in}}%
\pgfpathlineto{\pgfqpoint{3.206676in}{3.730271in}}%
\pgfpathclose%
\pgfusepath{fill}%
\end{pgfscope}%
\begin{pgfscope}%
\pgfpathrectangle{\pgfqpoint{1.150000in}{0.150000in}}{\pgfqpoint{5.700000in}{5.700000in}}%
\pgfusepath{clip}%
\pgfsetbuttcap%
\pgfsetroundjoin%
\definecolor{currentfill}{rgb}{0.153364,0.497000,0.557724}%
\pgfsetfillcolor{currentfill}%
\pgfsetfillopacity{0.800000}%
\pgfsetlinewidth{0.000000pt}%
\definecolor{currentstroke}{rgb}{0.000000,0.000000,0.000000}%
\pgfsetstrokecolor{currentstroke}%
\pgfsetdash{}{0pt}%
\pgfpathmoveto{\pgfqpoint{3.229230in}{3.562936in}}%
\pgfpathlineto{\pgfqpoint{3.242672in}{3.541660in}}%
\pgfpathlineto{\pgfqpoint{3.256107in}{3.520708in}}%
\pgfpathlineto{\pgfqpoint{3.269534in}{3.500077in}}%
\pgfpathlineto{\pgfqpoint{3.282953in}{3.479765in}}%
\pgfpathlineto{\pgfqpoint{3.290778in}{3.498327in}}%
\pgfpathlineto{\pgfqpoint{3.298596in}{3.517140in}}%
\pgfpathlineto{\pgfqpoint{3.306408in}{3.536208in}}%
\pgfpathlineto{\pgfqpoint{3.314214in}{3.555536in}}%
\pgfpathlineto{\pgfqpoint{3.300799in}{3.576240in}}%
\pgfpathlineto{\pgfqpoint{3.287377in}{3.597262in}}%
\pgfpathlineto{\pgfqpoint{3.273947in}{3.618607in}}%
\pgfpathlineto{\pgfqpoint{3.260510in}{3.640277in}}%
\pgfpathlineto{\pgfqpoint{3.252699in}{3.620543in}}%
\pgfpathlineto{\pgfqpoint{3.244883in}{3.601079in}}%
\pgfpathlineto{\pgfqpoint{3.237059in}{3.581878in}}%
\pgfpathlineto{\pgfqpoint{3.229230in}{3.562936in}}%
\pgfpathclose%
\pgfusepath{fill}%
\end{pgfscope}%
\begin{pgfscope}%
\pgfpathrectangle{\pgfqpoint{1.150000in}{0.150000in}}{\pgfqpoint{5.700000in}{5.700000in}}%
\pgfusepath{clip}%
\pgfsetbuttcap%
\pgfsetroundjoin%
\definecolor{currentfill}{rgb}{0.185556,0.418570,0.556753}%
\pgfsetfillcolor{currentfill}%
\pgfsetfillopacity{0.800000}%
\pgfsetlinewidth{0.000000pt}%
\definecolor{currentstroke}{rgb}{0.000000,0.000000,0.000000}%
\pgfsetstrokecolor{currentstroke}%
\pgfsetdash{}{0pt}%
\pgfpathmoveto{\pgfqpoint{3.305230in}{3.331234in}}%
\pgfpathlineto{\pgfqpoint{3.318623in}{3.312825in}}%
\pgfpathlineto{\pgfqpoint{3.332011in}{3.294717in}}%
\pgfpathlineto{\pgfqpoint{3.345394in}{3.276907in}}%
\pgfpathlineto{\pgfqpoint{3.358771in}{3.259393in}}%
\pgfpathlineto{\pgfqpoint{3.366607in}{3.276311in}}%
\pgfpathlineto{\pgfqpoint{3.374438in}{3.293446in}}%
\pgfpathlineto{\pgfqpoint{3.382263in}{3.310802in}}%
\pgfpathlineto{\pgfqpoint{3.390082in}{3.328383in}}%
\pgfpathlineto{\pgfqpoint{3.376711in}{3.346250in}}%
\pgfpathlineto{\pgfqpoint{3.363334in}{3.364414in}}%
\pgfpathlineto{\pgfqpoint{3.349953in}{3.382877in}}%
\pgfpathlineto{\pgfqpoint{3.336565in}{3.401641in}}%
\pgfpathlineto{\pgfqpoint{3.328740in}{3.383693in}}%
\pgfpathlineto{\pgfqpoint{3.320909in}{3.365979in}}%
\pgfpathlineto{\pgfqpoint{3.313073in}{3.348494in}}%
\pgfpathlineto{\pgfqpoint{3.305230in}{3.331234in}}%
\pgfpathclose%
\pgfusepath{fill}%
\end{pgfscope}%
\begin{pgfscope}%
\pgfpathrectangle{\pgfqpoint{1.150000in}{0.150000in}}{\pgfqpoint{5.700000in}{5.700000in}}%
\pgfusepath{clip}%
\pgfsetbuttcap%
\pgfsetroundjoin%
\definecolor{currentfill}{rgb}{0.241237,0.296485,0.539709}%
\pgfsetfillcolor{currentfill}%
\pgfsetfillopacity{0.800000}%
\pgfsetlinewidth{0.000000pt}%
\definecolor{currentstroke}{rgb}{0.000000,0.000000,0.000000}%
\pgfsetstrokecolor{currentstroke}%
\pgfsetdash{}{0pt}%
\pgfpathmoveto{\pgfqpoint{4.207009in}{2.983269in}}%
\pgfpathlineto{\pgfqpoint{4.220379in}{2.976678in}}%
\pgfpathlineto{\pgfqpoint{4.233754in}{2.970296in}}%
\pgfpathlineto{\pgfqpoint{4.247134in}{2.964120in}}%
\pgfpathlineto{\pgfqpoint{4.260520in}{2.958151in}}%
\pgfpathlineto{\pgfqpoint{4.268190in}{2.973375in}}%
\pgfpathlineto{\pgfqpoint{4.275857in}{2.988802in}}%
\pgfpathlineto{\pgfqpoint{4.283521in}{3.004440in}}%
\pgfpathlineto{\pgfqpoint{4.291184in}{3.020293in}}%
\pgfpathlineto{\pgfqpoint{4.277806in}{3.026763in}}%
\pgfpathlineto{\pgfqpoint{4.264433in}{3.033439in}}%
\pgfpathlineto{\pgfqpoint{4.251065in}{3.040321in}}%
\pgfpathlineto{\pgfqpoint{4.237702in}{3.047412in}}%
\pgfpathlineto{\pgfqpoint{4.230033in}{3.031047in}}%
\pgfpathlineto{\pgfqpoint{4.222360in}{3.014905in}}%
\pgfpathlineto{\pgfqpoint{4.214686in}{2.998981in}}%
\pgfpathlineto{\pgfqpoint{4.207009in}{2.983269in}}%
\pgfpathclose%
\pgfusepath{fill}%
\end{pgfscope}%
\begin{pgfscope}%
\pgfpathrectangle{\pgfqpoint{1.150000in}{0.150000in}}{\pgfqpoint{5.700000in}{5.700000in}}%
\pgfusepath{clip}%
\pgfsetbuttcap%
\pgfsetroundjoin%
\definecolor{currentfill}{rgb}{0.616293,0.852709,0.230052}%
\pgfsetfillcolor{currentfill}%
\pgfsetfillopacity{0.800000}%
\pgfsetlinewidth{0.000000pt}%
\definecolor{currentstroke}{rgb}{0.000000,0.000000,0.000000}%
\pgfsetstrokecolor{currentstroke}%
\pgfsetdash{}{0pt}%
\pgfpathmoveto{\pgfqpoint{3.568593in}{4.717227in}}%
\pgfpathlineto{\pgfqpoint{3.582049in}{4.689623in}}%
\pgfpathlineto{\pgfqpoint{3.595497in}{4.662358in}}%
\pgfpathlineto{\pgfqpoint{3.608935in}{4.635428in}}%
\pgfpathlineto{\pgfqpoint{3.622366in}{4.608830in}}%
\pgfpathlineto{\pgfqpoint{3.629994in}{4.644179in}}%
\pgfpathlineto{\pgfqpoint{3.637621in}{4.680085in}}%
\pgfpathlineto{\pgfqpoint{3.645245in}{4.716558in}}%
\pgfpathlineto{\pgfqpoint{3.652868in}{4.753606in}}%
\pgfpathlineto{\pgfqpoint{3.639429in}{4.781010in}}%
\pgfpathlineto{\pgfqpoint{3.625981in}{4.808748in}}%
\pgfpathlineto{\pgfqpoint{3.612524in}{4.836824in}}%
\pgfpathlineto{\pgfqpoint{3.599058in}{4.865240in}}%
\pgfpathlineto{\pgfqpoint{3.591445in}{4.827366in}}%
\pgfpathlineto{\pgfqpoint{3.583830in}{4.790079in}}%
\pgfpathlineto{\pgfqpoint{3.576213in}{4.753369in}}%
\pgfpathlineto{\pgfqpoint{3.568593in}{4.717227in}}%
\pgfpathclose%
\pgfusepath{fill}%
\end{pgfscope}%
\begin{pgfscope}%
\pgfpathrectangle{\pgfqpoint{1.150000in}{0.150000in}}{\pgfqpoint{5.700000in}{5.700000in}}%
\pgfusepath{clip}%
\pgfsetbuttcap%
\pgfsetroundjoin%
\definecolor{currentfill}{rgb}{0.194100,0.399323,0.555565}%
\pgfsetfillcolor{currentfill}%
\pgfsetfillopacity{0.800000}%
\pgfsetlinewidth{0.000000pt}%
\definecolor{currentstroke}{rgb}{0.000000,0.000000,0.000000}%
\pgfsetstrokecolor{currentstroke}%
\pgfsetdash{}{0pt}%
\pgfpathmoveto{\pgfqpoint{4.712157in}{3.245832in}}%
\pgfpathlineto{\pgfqpoint{4.725624in}{3.240533in}}%
\pgfpathlineto{\pgfqpoint{4.739098in}{3.235423in}}%
\pgfpathlineto{\pgfqpoint{4.752580in}{3.230502in}}%
\pgfpathlineto{\pgfqpoint{4.766070in}{3.225769in}}%
\pgfpathlineto{\pgfqpoint{4.773665in}{3.243626in}}%
\pgfpathlineto{\pgfqpoint{4.781262in}{3.261805in}}%
\pgfpathlineto{\pgfqpoint{4.788861in}{3.280314in}}%
\pgfpathlineto{\pgfqpoint{4.796462in}{3.299161in}}%
\pgfpathlineto{\pgfqpoint{4.782984in}{3.304583in}}%
\pgfpathlineto{\pgfqpoint{4.769514in}{3.310193in}}%
\pgfpathlineto{\pgfqpoint{4.756051in}{3.315992in}}%
\pgfpathlineto{\pgfqpoint{4.742595in}{3.321980in}}%
\pgfpathlineto{\pgfqpoint{4.734982in}{3.302432in}}%
\pgfpathlineto{\pgfqpoint{4.727372in}{3.283230in}}%
\pgfpathlineto{\pgfqpoint{4.719764in}{3.264366in}}%
\pgfpathlineto{\pgfqpoint{4.712157in}{3.245832in}}%
\pgfpathclose%
\pgfusepath{fill}%
\end{pgfscope}%
\begin{pgfscope}%
\pgfpathrectangle{\pgfqpoint{1.150000in}{0.150000in}}{\pgfqpoint{5.700000in}{5.700000in}}%
\pgfusepath{clip}%
\pgfsetbuttcap%
\pgfsetroundjoin%
\definecolor{currentfill}{rgb}{0.237441,0.305202,0.541921}%
\pgfsetfillcolor{currentfill}%
\pgfsetfillopacity{0.800000}%
\pgfsetlinewidth{0.000000pt}%
\definecolor{currentstroke}{rgb}{0.000000,0.000000,0.000000}%
\pgfsetstrokecolor{currentstroke}%
\pgfsetdash{}{0pt}%
\pgfpathmoveto{\pgfqpoint{3.572303in}{3.017436in}}%
\pgfpathlineto{\pgfqpoint{3.585629in}{3.004594in}}%
\pgfpathlineto{\pgfqpoint{3.598954in}{2.992009in}}%
\pgfpathlineto{\pgfqpoint{3.612278in}{2.979678in}}%
\pgfpathlineto{\pgfqpoint{3.625602in}{2.967599in}}%
\pgfpathlineto{\pgfqpoint{3.633406in}{2.982830in}}%
\pgfpathlineto{\pgfqpoint{3.641206in}{2.998237in}}%
\pgfpathlineto{\pgfqpoint{3.649001in}{3.013823in}}%
\pgfpathlineto{\pgfqpoint{3.656792in}{3.029595in}}%
\pgfpathlineto{\pgfqpoint{3.643474in}{3.042020in}}%
\pgfpathlineto{\pgfqpoint{3.630156in}{3.054699in}}%
\pgfpathlineto{\pgfqpoint{3.616838in}{3.067632in}}%
\pgfpathlineto{\pgfqpoint{3.603518in}{3.080821in}}%
\pgfpathlineto{\pgfqpoint{3.595722in}{3.064690in}}%
\pgfpathlineto{\pgfqpoint{3.587920in}{3.048752in}}%
\pgfpathlineto{\pgfqpoint{3.580114in}{3.033002in}}%
\pgfpathlineto{\pgfqpoint{3.572303in}{3.017436in}}%
\pgfpathclose%
\pgfusepath{fill}%
\end{pgfscope}%
\begin{pgfscope}%
\pgfpathrectangle{\pgfqpoint{1.150000in}{0.150000in}}{\pgfqpoint{5.700000in}{5.700000in}}%
\pgfusepath{clip}%
\pgfsetbuttcap%
\pgfsetroundjoin%
\definecolor{currentfill}{rgb}{0.250425,0.274290,0.533103}%
\pgfsetfillcolor{currentfill}%
\pgfsetfillopacity{0.800000}%
\pgfsetlinewidth{0.000000pt}%
\definecolor{currentstroke}{rgb}{0.000000,0.000000,0.000000}%
\pgfsetstrokecolor{currentstroke}%
\pgfsetdash{}{0pt}%
\pgfpathmoveto{\pgfqpoint{3.763330in}{2.939090in}}%
\pgfpathlineto{\pgfqpoint{3.776650in}{2.928865in}}%
\pgfpathlineto{\pgfqpoint{3.789972in}{2.918876in}}%
\pgfpathlineto{\pgfqpoint{3.803295in}{2.909123in}}%
\pgfpathlineto{\pgfqpoint{3.816620in}{2.899603in}}%
\pgfpathlineto{\pgfqpoint{3.824387in}{2.914466in}}%
\pgfpathlineto{\pgfqpoint{3.832150in}{2.929499in}}%
\pgfpathlineto{\pgfqpoint{3.839908in}{2.944706in}}%
\pgfpathlineto{\pgfqpoint{3.847662in}{2.960091in}}%
\pgfpathlineto{\pgfqpoint{3.834344in}{2.969987in}}%
\pgfpathlineto{\pgfqpoint{3.821027in}{2.980118in}}%
\pgfpathlineto{\pgfqpoint{3.807712in}{2.990483in}}%
\pgfpathlineto{\pgfqpoint{3.794397in}{3.001086in}}%
\pgfpathlineto{\pgfqpoint{3.786637in}{2.985312in}}%
\pgfpathlineto{\pgfqpoint{3.778872in}{2.969724in}}%
\pgfpathlineto{\pgfqpoint{3.771103in}{2.954318in}}%
\pgfpathlineto{\pgfqpoint{3.763330in}{2.939090in}}%
\pgfpathclose%
\pgfusepath{fill}%
\end{pgfscope}%
\begin{pgfscope}%
\pgfpathrectangle{\pgfqpoint{1.150000in}{0.150000in}}{\pgfqpoint{5.700000in}{5.700000in}}%
\pgfusepath{clip}%
\pgfsetbuttcap%
\pgfsetroundjoin%
\definecolor{currentfill}{rgb}{0.252194,0.269783,0.531579}%
\pgfsetfillcolor{currentfill}%
\pgfsetfillopacity{0.800000}%
\pgfsetlinewidth{0.000000pt}%
\definecolor{currentstroke}{rgb}{0.000000,0.000000,0.000000}%
\pgfsetstrokecolor{currentstroke}%
\pgfsetdash{}{0pt}%
\pgfpathmoveto{\pgfqpoint{3.900956in}{2.922815in}}%
\pgfpathlineto{\pgfqpoint{3.914286in}{2.914067in}}%
\pgfpathlineto{\pgfqpoint{3.927618in}{2.905544in}}%
\pgfpathlineto{\pgfqpoint{3.940952in}{2.897246in}}%
\pgfpathlineto{\pgfqpoint{3.954290in}{2.889171in}}%
\pgfpathlineto{\pgfqpoint{3.962028in}{2.903948in}}%
\pgfpathlineto{\pgfqpoint{3.969761in}{2.918897in}}%
\pgfpathlineto{\pgfqpoint{3.977491in}{2.934023in}}%
\pgfpathlineto{\pgfqpoint{3.985217in}{2.949331in}}%
\pgfpathlineto{\pgfqpoint{3.971886in}{2.957813in}}%
\pgfpathlineto{\pgfqpoint{3.958557in}{2.966518in}}%
\pgfpathlineto{\pgfqpoint{3.945232in}{2.975448in}}%
\pgfpathlineto{\pgfqpoint{3.931909in}{2.984604in}}%
\pgfpathlineto{\pgfqpoint{3.924176in}{2.968877in}}%
\pgfpathlineto{\pgfqpoint{3.916440in}{2.953340in}}%
\pgfpathlineto{\pgfqpoint{3.908700in}{2.937987in}}%
\pgfpathlineto{\pgfqpoint{3.900956in}{2.922815in}}%
\pgfpathclose%
\pgfusepath{fill}%
\end{pgfscope}%
\begin{pgfscope}%
\pgfpathrectangle{\pgfqpoint{1.150000in}{0.150000in}}{\pgfqpoint{5.700000in}{5.700000in}}%
\pgfusepath{clip}%
\pgfsetbuttcap%
\pgfsetroundjoin%
\definecolor{currentfill}{rgb}{0.119699,0.618490,0.536347}%
\pgfsetfillcolor{currentfill}%
\pgfsetfillopacity{0.800000}%
\pgfsetlinewidth{0.000000pt}%
\definecolor{currentstroke}{rgb}{0.000000,0.000000,0.000000}%
\pgfsetstrokecolor{currentstroke}%
\pgfsetdash{}{0pt}%
\pgfpathmoveto{\pgfqpoint{3.183905in}{3.910898in}}%
\pgfpathlineto{\pgfqpoint{3.197412in}{3.886075in}}%
\pgfpathlineto{\pgfqpoint{3.210908in}{3.861605in}}%
\pgfpathlineto{\pgfqpoint{3.224395in}{3.837483in}}%
\pgfpathlineto{\pgfqpoint{3.237872in}{3.813708in}}%
\pgfpathlineto{\pgfqpoint{3.245655in}{3.835311in}}%
\pgfpathlineto{\pgfqpoint{3.253432in}{3.857222in}}%
\pgfpathlineto{\pgfqpoint{3.261203in}{3.879446in}}%
\pgfpathlineto{\pgfqpoint{3.268968in}{3.901990in}}%
\pgfpathlineto{\pgfqpoint{3.255493in}{3.926233in}}%
\pgfpathlineto{\pgfqpoint{3.242008in}{3.950823in}}%
\pgfpathlineto{\pgfqpoint{3.228514in}{3.975763in}}%
\pgfpathlineto{\pgfqpoint{3.215010in}{4.001057in}}%
\pgfpathlineto{\pgfqpoint{3.207243in}{3.978030in}}%
\pgfpathlineto{\pgfqpoint{3.199470in}{3.955332in}}%
\pgfpathlineto{\pgfqpoint{3.191691in}{3.932957in}}%
\pgfpathlineto{\pgfqpoint{3.183905in}{3.910898in}}%
\pgfpathclose%
\pgfusepath{fill}%
\end{pgfscope}%
\begin{pgfscope}%
\pgfpathrectangle{\pgfqpoint{1.150000in}{0.150000in}}{\pgfqpoint{5.700000in}{5.700000in}}%
\pgfusepath{clip}%
\pgfsetbuttcap%
\pgfsetroundjoin%
\definecolor{currentfill}{rgb}{0.449368,0.813768,0.335384}%
\pgfsetfillcolor{currentfill}%
\pgfsetfillopacity{0.800000}%
\pgfsetlinewidth{0.000000pt}%
\definecolor{currentstroke}{rgb}{0.000000,0.000000,0.000000}%
\pgfsetstrokecolor{currentstroke}%
\pgfsetdash{}{0pt}%
\pgfpathmoveto{\pgfqpoint{3.369097in}{4.540125in}}%
\pgfpathlineto{\pgfqpoint{3.382611in}{4.511916in}}%
\pgfpathlineto{\pgfqpoint{3.396114in}{4.484069in}}%
\pgfpathlineto{\pgfqpoint{3.409606in}{4.456579in}}%
\pgfpathlineto{\pgfqpoint{3.423088in}{4.429443in}}%
\pgfpathlineto{\pgfqpoint{3.430742in}{4.460171in}}%
\pgfpathlineto{\pgfqpoint{3.438391in}{4.491370in}}%
\pgfpathlineto{\pgfqpoint{3.446037in}{4.523048in}}%
\pgfpathlineto{\pgfqpoint{3.453678in}{4.555212in}}%
\pgfpathlineto{\pgfqpoint{3.440190in}{4.583044in}}%
\pgfpathlineto{\pgfqpoint{3.426691in}{4.611231in}}%
\pgfpathlineto{\pgfqpoint{3.413182in}{4.639779in}}%
\pgfpathlineto{\pgfqpoint{3.399663in}{4.668689in}}%
\pgfpathlineto{\pgfqpoint{3.392028in}{4.635811in}}%
\pgfpathlineto{\pgfqpoint{3.384389in}{4.603429in}}%
\pgfpathlineto{\pgfqpoint{3.376746in}{4.571537in}}%
\pgfpathlineto{\pgfqpoint{3.369097in}{4.540125in}}%
\pgfpathclose%
\pgfusepath{fill}%
\end{pgfscope}%
\begin{pgfscope}%
\pgfpathrectangle{\pgfqpoint{1.150000in}{0.150000in}}{\pgfqpoint{5.700000in}{5.700000in}}%
\pgfusepath{clip}%
\pgfsetbuttcap%
\pgfsetroundjoin%
\definecolor{currentfill}{rgb}{0.246811,0.283237,0.535941}%
\pgfsetfillcolor{currentfill}%
\pgfsetfillopacity{0.800000}%
\pgfsetlinewidth{0.000000pt}%
\definecolor{currentstroke}{rgb}{0.000000,0.000000,0.000000}%
\pgfsetstrokecolor{currentstroke}%
\pgfsetdash{}{0pt}%
\pgfpathmoveto{\pgfqpoint{4.122810in}{2.949013in}}%
\pgfpathlineto{\pgfqpoint{4.136169in}{2.942049in}}%
\pgfpathlineto{\pgfqpoint{4.149532in}{2.935298in}}%
\pgfpathlineto{\pgfqpoint{4.162900in}{2.928758in}}%
\pgfpathlineto{\pgfqpoint{4.176272in}{2.922427in}}%
\pgfpathlineto{\pgfqpoint{4.183961in}{2.937348in}}%
\pgfpathlineto{\pgfqpoint{4.191646in}{2.952458in}}%
\pgfpathlineto{\pgfqpoint{4.199329in}{2.967763in}}%
\pgfpathlineto{\pgfqpoint{4.207009in}{2.983269in}}%
\pgfpathlineto{\pgfqpoint{4.193643in}{2.990068in}}%
\pgfpathlineto{\pgfqpoint{4.180283in}{2.997078in}}%
\pgfpathlineto{\pgfqpoint{4.166927in}{3.004298in}}%
\pgfpathlineto{\pgfqpoint{4.153575in}{3.011731in}}%
\pgfpathlineto{\pgfqpoint{4.145889in}{2.995744in}}%
\pgfpathlineto{\pgfqpoint{4.138199in}{2.979966in}}%
\pgfpathlineto{\pgfqpoint{4.130506in}{2.964391in}}%
\pgfpathlineto{\pgfqpoint{4.122810in}{2.949013in}}%
\pgfpathclose%
\pgfusepath{fill}%
\end{pgfscope}%
\begin{pgfscope}%
\pgfpathrectangle{\pgfqpoint{1.150000in}{0.150000in}}{\pgfqpoint{5.700000in}{5.700000in}}%
\pgfusepath{clip}%
\pgfsetbuttcap%
\pgfsetroundjoin%
\definecolor{currentfill}{rgb}{0.185556,0.418570,0.556753}%
\pgfsetfillcolor{currentfill}%
\pgfsetfillopacity{0.800000}%
\pgfsetlinewidth{0.000000pt}%
\definecolor{currentstroke}{rgb}{0.000000,0.000000,0.000000}%
\pgfsetstrokecolor{currentstroke}%
\pgfsetdash{}{0pt}%
\pgfpathmoveto{\pgfqpoint{4.796462in}{3.299161in}}%
\pgfpathlineto{\pgfqpoint{4.809948in}{3.293927in}}%
\pgfpathlineto{\pgfqpoint{4.823441in}{3.288879in}}%
\pgfpathlineto{\pgfqpoint{4.836943in}{3.284018in}}%
\pgfpathlineto{\pgfqpoint{4.850452in}{3.279343in}}%
\pgfpathlineto{\pgfqpoint{4.858043in}{3.297828in}}%
\pgfpathlineto{\pgfqpoint{4.865638in}{3.316660in}}%
\pgfpathlineto{\pgfqpoint{4.873235in}{3.335847in}}%
\pgfpathlineto{\pgfqpoint{4.880837in}{3.355398in}}%
\pgfpathlineto{\pgfqpoint{4.867340in}{3.360794in}}%
\pgfpathlineto{\pgfqpoint{4.853851in}{3.366376in}}%
\pgfpathlineto{\pgfqpoint{4.840369in}{3.372144in}}%
\pgfpathlineto{\pgfqpoint{4.826896in}{3.378099in}}%
\pgfpathlineto{\pgfqpoint{4.819283in}{3.357815in}}%
\pgfpathlineto{\pgfqpoint{4.811673in}{3.337904in}}%
\pgfpathlineto{\pgfqpoint{4.804066in}{3.318355in}}%
\pgfpathlineto{\pgfqpoint{4.796462in}{3.299161in}}%
\pgfpathclose%
\pgfusepath{fill}%
\end{pgfscope}%
\begin{pgfscope}%
\pgfpathrectangle{\pgfqpoint{1.150000in}{0.150000in}}{\pgfqpoint{5.700000in}{5.700000in}}%
\pgfusepath{clip}%
\pgfsetbuttcap%
\pgfsetroundjoin%
\definecolor{currentfill}{rgb}{0.172719,0.448791,0.557885}%
\pgfsetfillcolor{currentfill}%
\pgfsetfillopacity{0.800000}%
\pgfsetlinewidth{0.000000pt}%
\definecolor{currentstroke}{rgb}{0.000000,0.000000,0.000000}%
\pgfsetstrokecolor{currentstroke}%
\pgfsetdash{}{0pt}%
\pgfpathmoveto{\pgfqpoint{3.251595in}{3.407934in}}%
\pgfpathlineto{\pgfqpoint{3.265013in}{3.388294in}}%
\pgfpathlineto{\pgfqpoint{3.278425in}{3.368966in}}%
\pgfpathlineto{\pgfqpoint{3.291831in}{3.349947in}}%
\pgfpathlineto{\pgfqpoint{3.305230in}{3.331234in}}%
\pgfpathlineto{\pgfqpoint{3.313073in}{3.348494in}}%
\pgfpathlineto{\pgfqpoint{3.320909in}{3.365979in}}%
\pgfpathlineto{\pgfqpoint{3.328740in}{3.383693in}}%
\pgfpathlineto{\pgfqpoint{3.336565in}{3.401641in}}%
\pgfpathlineto{\pgfqpoint{3.323172in}{3.420709in}}%
\pgfpathlineto{\pgfqpoint{3.309772in}{3.440083in}}%
\pgfpathlineto{\pgfqpoint{3.296366in}{3.459768in}}%
\pgfpathlineto{\pgfqpoint{3.282953in}{3.479765in}}%
\pgfpathlineto{\pgfqpoint{3.275123in}{3.461449in}}%
\pgfpathlineto{\pgfqpoint{3.267286in}{3.443375in}}%
\pgfpathlineto{\pgfqpoint{3.259444in}{3.425538in}}%
\pgfpathlineto{\pgfqpoint{3.251595in}{3.407934in}}%
\pgfpathclose%
\pgfusepath{fill}%
\end{pgfscope}%
\begin{pgfscope}%
\pgfpathrectangle{\pgfqpoint{1.150000in}{0.150000in}}{\pgfqpoint{5.700000in}{5.700000in}}%
\pgfusepath{clip}%
\pgfsetbuttcap%
\pgfsetroundjoin%
\definecolor{currentfill}{rgb}{0.585678,0.846661,0.249897}%
\pgfsetfillcolor{currentfill}%
\pgfsetfillopacity{0.800000}%
\pgfsetlinewidth{0.000000pt}%
\definecolor{currentstroke}{rgb}{0.000000,0.000000,0.000000}%
\pgfsetstrokecolor{currentstroke}%
\pgfsetdash{}{0pt}%
\pgfpathmoveto{\pgfqpoint{3.484204in}{4.688908in}}%
\pgfpathlineto{\pgfqpoint{3.497689in}{4.660693in}}%
\pgfpathlineto{\pgfqpoint{3.511165in}{4.632828in}}%
\pgfpathlineto{\pgfqpoint{3.524631in}{4.605310in}}%
\pgfpathlineto{\pgfqpoint{3.538088in}{4.578136in}}%
\pgfpathlineto{\pgfqpoint{3.545719in}{4.612105in}}%
\pgfpathlineto{\pgfqpoint{3.553346in}{4.646604in}}%
\pgfpathlineto{\pgfqpoint{3.560971in}{4.681641in}}%
\pgfpathlineto{\pgfqpoint{3.568593in}{4.717227in}}%
\pgfpathlineto{\pgfqpoint{3.555128in}{4.745173in}}%
\pgfpathlineto{\pgfqpoint{3.541653in}{4.773464in}}%
\pgfpathlineto{\pgfqpoint{3.528169in}{4.802105in}}%
\pgfpathlineto{\pgfqpoint{3.514674in}{4.831098in}}%
\pgfpathlineto{\pgfqpoint{3.507061in}{4.794722in}}%
\pgfpathlineto{\pgfqpoint{3.499445in}{4.758904in}}%
\pgfpathlineto{\pgfqpoint{3.491826in}{4.723636in}}%
\pgfpathlineto{\pgfqpoint{3.484204in}{4.688908in}}%
\pgfpathclose%
\pgfusepath{fill}%
\end{pgfscope}%
\begin{pgfscope}%
\pgfpathrectangle{\pgfqpoint{1.150000in}{0.150000in}}{\pgfqpoint{5.700000in}{5.700000in}}%
\pgfusepath{clip}%
\pgfsetbuttcap%
\pgfsetroundjoin%
\definecolor{currentfill}{rgb}{0.244972,0.287675,0.537260}%
\pgfsetfillcolor{currentfill}%
\pgfsetfillopacity{0.800000}%
\pgfsetlinewidth{0.000000pt}%
\definecolor{currentstroke}{rgb}{0.000000,0.000000,0.000000}%
\pgfsetstrokecolor{currentstroke}%
\pgfsetdash{}{0pt}%
\pgfpathmoveto{\pgfqpoint{3.625602in}{2.967599in}}%
\pgfpathlineto{\pgfqpoint{3.638925in}{2.955772in}}%
\pgfpathlineto{\pgfqpoint{3.652248in}{2.944194in}}%
\pgfpathlineto{\pgfqpoint{3.665571in}{2.932863in}}%
\pgfpathlineto{\pgfqpoint{3.678894in}{2.921778in}}%
\pgfpathlineto{\pgfqpoint{3.686692in}{2.936674in}}%
\pgfpathlineto{\pgfqpoint{3.694485in}{2.951738in}}%
\pgfpathlineto{\pgfqpoint{3.702274in}{2.966974in}}%
\pgfpathlineto{\pgfqpoint{3.710058in}{2.982387in}}%
\pgfpathlineto{\pgfqpoint{3.696741in}{2.993819in}}%
\pgfpathlineto{\pgfqpoint{3.683425in}{3.005496in}}%
\pgfpathlineto{\pgfqpoint{3.670108in}{3.017421in}}%
\pgfpathlineto{\pgfqpoint{3.656792in}{3.029595in}}%
\pgfpathlineto{\pgfqpoint{3.649001in}{3.013823in}}%
\pgfpathlineto{\pgfqpoint{3.641206in}{2.998237in}}%
\pgfpathlineto{\pgfqpoint{3.633406in}{2.982830in}}%
\pgfpathlineto{\pgfqpoint{3.625602in}{2.967599in}}%
\pgfpathclose%
\pgfusepath{fill}%
\end{pgfscope}%
\begin{pgfscope}%
\pgfpathrectangle{\pgfqpoint{1.150000in}{0.150000in}}{\pgfqpoint{5.700000in}{5.700000in}}%
\pgfusepath{clip}%
\pgfsetbuttcap%
\pgfsetroundjoin%
\definecolor{currentfill}{rgb}{0.252194,0.269783,0.531579}%
\pgfsetfillcolor{currentfill}%
\pgfsetfillopacity{0.800000}%
\pgfsetlinewidth{0.000000pt}%
\definecolor{currentstroke}{rgb}{0.000000,0.000000,0.000000}%
\pgfsetstrokecolor{currentstroke}%
\pgfsetdash{}{0pt}%
\pgfpathmoveto{\pgfqpoint{4.038576in}{2.917610in}}%
\pgfpathlineto{\pgfqpoint{4.051924in}{2.910226in}}%
\pgfpathlineto{\pgfqpoint{4.065277in}{2.903059in}}%
\pgfpathlineto{\pgfqpoint{4.078634in}{2.896106in}}%
\pgfpathlineto{\pgfqpoint{4.091995in}{2.889368in}}%
\pgfpathlineto{\pgfqpoint{4.099704in}{2.904009in}}%
\pgfpathlineto{\pgfqpoint{4.107409in}{2.918827in}}%
\pgfpathlineto{\pgfqpoint{4.115111in}{2.933827in}}%
\pgfpathlineto{\pgfqpoint{4.122810in}{2.949013in}}%
\pgfpathlineto{\pgfqpoint{4.109456in}{2.956189in}}%
\pgfpathlineto{\pgfqpoint{4.096106in}{2.963580in}}%
\pgfpathlineto{\pgfqpoint{4.082760in}{2.971186in}}%
\pgfpathlineto{\pgfqpoint{4.069418in}{2.979008in}}%
\pgfpathlineto{\pgfqpoint{4.061713in}{2.963372in}}%
\pgfpathlineto{\pgfqpoint{4.054004in}{2.947930in}}%
\pgfpathlineto{\pgfqpoint{4.046291in}{2.932678in}}%
\pgfpathlineto{\pgfqpoint{4.038576in}{2.917610in}}%
\pgfpathclose%
\pgfusepath{fill}%
\end{pgfscope}%
\begin{pgfscope}%
\pgfpathrectangle{\pgfqpoint{1.150000in}{0.150000in}}{\pgfqpoint{5.700000in}{5.700000in}}%
\pgfusepath{clip}%
\pgfsetbuttcap%
\pgfsetroundjoin%
\definecolor{currentfill}{rgb}{0.140536,0.530132,0.555659}%
\pgfsetfillcolor{currentfill}%
\pgfsetfillopacity{0.800000}%
\pgfsetlinewidth{0.000000pt}%
\definecolor{currentstroke}{rgb}{0.000000,0.000000,0.000000}%
\pgfsetstrokecolor{currentstroke}%
\pgfsetdash{}{0pt}%
\pgfpathmoveto{\pgfqpoint{3.175377in}{3.651350in}}%
\pgfpathlineto{\pgfqpoint{3.188853in}{3.628744in}}%
\pgfpathlineto{\pgfqpoint{3.202321in}{3.606475in}}%
\pgfpathlineto{\pgfqpoint{3.215780in}{3.584540in}}%
\pgfpathlineto{\pgfqpoint{3.229230in}{3.562936in}}%
\pgfpathlineto{\pgfqpoint{3.237059in}{3.581878in}}%
\pgfpathlineto{\pgfqpoint{3.244883in}{3.601079in}}%
\pgfpathlineto{\pgfqpoint{3.252699in}{3.620543in}}%
\pgfpathlineto{\pgfqpoint{3.260510in}{3.640277in}}%
\pgfpathlineto{\pgfqpoint{3.247064in}{3.662275in}}%
\pgfpathlineto{\pgfqpoint{3.233611in}{3.684604in}}%
\pgfpathlineto{\pgfqpoint{3.220148in}{3.707268in}}%
\pgfpathlineto{\pgfqpoint{3.206676in}{3.730271in}}%
\pgfpathlineto{\pgfqpoint{3.198861in}{3.710129in}}%
\pgfpathlineto{\pgfqpoint{3.191040in}{3.690265in}}%
\pgfpathlineto{\pgfqpoint{3.183212in}{3.670674in}}%
\pgfpathlineto{\pgfqpoint{3.175377in}{3.651350in}}%
\pgfpathclose%
\pgfusepath{fill}%
\end{pgfscope}%
\begin{pgfscope}%
\pgfpathrectangle{\pgfqpoint{1.150000in}{0.150000in}}{\pgfqpoint{5.700000in}{5.700000in}}%
\pgfusepath{clip}%
\pgfsetbuttcap%
\pgfsetroundjoin%
\definecolor{currentfill}{rgb}{0.177423,0.437527,0.557565}%
\pgfsetfillcolor{currentfill}%
\pgfsetfillopacity{0.800000}%
\pgfsetlinewidth{0.000000pt}%
\definecolor{currentstroke}{rgb}{0.000000,0.000000,0.000000}%
\pgfsetstrokecolor{currentstroke}%
\pgfsetdash{}{0pt}%
\pgfpathmoveto{\pgfqpoint{4.880837in}{3.355398in}}%
\pgfpathlineto{\pgfqpoint{4.894341in}{3.350187in}}%
\pgfpathlineto{\pgfqpoint{4.907854in}{3.345161in}}%
\pgfpathlineto{\pgfqpoint{4.921375in}{3.340318in}}%
\pgfpathlineto{\pgfqpoint{4.934904in}{3.335659in}}%
\pgfpathlineto{\pgfqpoint{4.942495in}{3.354842in}}%
\pgfpathlineto{\pgfqpoint{4.950091in}{3.374398in}}%
\pgfpathlineto{\pgfqpoint{4.957691in}{3.394337in}}%
\pgfpathlineto{\pgfqpoint{4.944172in}{3.399557in}}%
\pgfpathlineto{\pgfqpoint{4.930661in}{3.404961in}}%
\pgfpathlineto{\pgfqpoint{4.917158in}{3.410549in}}%
\pgfpathlineto{\pgfqpoint{4.903663in}{3.416321in}}%
\pgfpathlineto{\pgfqpoint{4.896050in}{3.395626in}}%
\pgfpathlineto{\pgfqpoint{4.888441in}{3.375322in}}%
\pgfpathlineto{\pgfqpoint{4.880837in}{3.355398in}}%
\pgfpathclose%
\pgfusepath{fill}%
\end{pgfscope}%
\begin{pgfscope}%
\pgfpathrectangle{\pgfqpoint{1.150000in}{0.150000in}}{\pgfqpoint{5.700000in}{5.700000in}}%
\pgfusepath{clip}%
\pgfsetbuttcap%
\pgfsetroundjoin%
\definecolor{currentfill}{rgb}{0.216210,0.351535,0.550627}%
\pgfsetfillcolor{currentfill}%
\pgfsetfillopacity{0.800000}%
\pgfsetlinewidth{0.000000pt}%
\definecolor{currentstroke}{rgb}{0.000000,0.000000,0.000000}%
\pgfsetstrokecolor{currentstroke}%
\pgfsetdash{}{0pt}%
\pgfpathmoveto{\pgfqpoint{3.380856in}{3.127932in}}%
\pgfpathlineto{\pgfqpoint{3.394218in}{3.112178in}}%
\pgfpathlineto{\pgfqpoint{3.407576in}{3.096704in}}%
\pgfpathlineto{\pgfqpoint{3.420931in}{3.081509in}}%
\pgfpathlineto{\pgfqpoint{3.434283in}{3.066591in}}%
\pgfpathlineto{\pgfqpoint{3.442128in}{3.082070in}}%
\pgfpathlineto{\pgfqpoint{3.449969in}{3.097733in}}%
\pgfpathlineto{\pgfqpoint{3.457804in}{3.113585in}}%
\pgfpathlineto{\pgfqpoint{3.465633in}{3.129628in}}%
\pgfpathlineto{\pgfqpoint{3.452288in}{3.144865in}}%
\pgfpathlineto{\pgfqpoint{3.438940in}{3.160378in}}%
\pgfpathlineto{\pgfqpoint{3.425588in}{3.176170in}}%
\pgfpathlineto{\pgfqpoint{3.412233in}{3.192243in}}%
\pgfpathlineto{\pgfqpoint{3.404397in}{3.175869in}}%
\pgfpathlineto{\pgfqpoint{3.396556in}{3.159695in}}%
\pgfpathlineto{\pgfqpoint{3.388709in}{3.143717in}}%
\pgfpathlineto{\pgfqpoint{3.380856in}{3.127932in}}%
\pgfpathclose%
\pgfusepath{fill}%
\end{pgfscope}%
\begin{pgfscope}%
\pgfpathrectangle{\pgfqpoint{1.150000in}{0.150000in}}{\pgfqpoint{5.700000in}{5.700000in}}%
\pgfusepath{clip}%
\pgfsetbuttcap%
\pgfsetroundjoin%
\definecolor{currentfill}{rgb}{0.121831,0.589055,0.545623}%
\pgfsetfillcolor{currentfill}%
\pgfsetfillopacity{0.800000}%
\pgfsetlinewidth{0.000000pt}%
\definecolor{currentstroke}{rgb}{0.000000,0.000000,0.000000}%
\pgfsetstrokecolor{currentstroke}%
\pgfsetdash{}{0pt}%
\pgfpathmoveto{\pgfqpoint{3.152695in}{3.825728in}}%
\pgfpathlineto{\pgfqpoint{3.166205in}{3.801340in}}%
\pgfpathlineto{\pgfqpoint{3.179705in}{3.777303in}}%
\pgfpathlineto{\pgfqpoint{3.193196in}{3.753614in}}%
\pgfpathlineto{\pgfqpoint{3.206676in}{3.730271in}}%
\pgfpathlineto{\pgfqpoint{3.214485in}{3.750694in}}%
\pgfpathlineto{\pgfqpoint{3.222287in}{3.771405in}}%
\pgfpathlineto{\pgfqpoint{3.230083in}{3.792407in}}%
\pgfpathlineto{\pgfqpoint{3.237872in}{3.813708in}}%
\pgfpathlineto{\pgfqpoint{3.224395in}{3.837483in}}%
\pgfpathlineto{\pgfqpoint{3.210908in}{3.861605in}}%
\pgfpathlineto{\pgfqpoint{3.197412in}{3.886075in}}%
\pgfpathlineto{\pgfqpoint{3.183905in}{3.910898in}}%
\pgfpathlineto{\pgfqpoint{3.176112in}{3.889152in}}%
\pgfpathlineto{\pgfqpoint{3.168313in}{3.867711in}}%
\pgfpathlineto{\pgfqpoint{3.160507in}{3.846572in}}%
\pgfpathlineto{\pgfqpoint{3.152695in}{3.825728in}}%
\pgfpathclose%
\pgfusepath{fill}%
\end{pgfscope}%
\begin{pgfscope}%
\pgfpathrectangle{\pgfqpoint{1.150000in}{0.150000in}}{\pgfqpoint{5.700000in}{5.700000in}}%
\pgfusepath{clip}%
\pgfsetbuttcap%
\pgfsetroundjoin%
\definecolor{currentfill}{rgb}{0.221989,0.339161,0.548752}%
\pgfsetfillcolor{currentfill}%
\pgfsetfillopacity{0.800000}%
\pgfsetlinewidth{0.000000pt}%
\definecolor{currentstroke}{rgb}{0.000000,0.000000,0.000000}%
\pgfsetstrokecolor{currentstroke}%
\pgfsetdash{}{0pt}%
\pgfpathmoveto{\pgfqpoint{4.513210in}{3.080639in}}%
\pgfpathlineto{\pgfqpoint{4.526651in}{3.075679in}}%
\pgfpathlineto{\pgfqpoint{4.540098in}{3.070913in}}%
\pgfpathlineto{\pgfqpoint{4.553553in}{3.066341in}}%
\pgfpathlineto{\pgfqpoint{4.567014in}{3.061962in}}%
\pgfpathlineto{\pgfqpoint{4.574627in}{3.077711in}}%
\pgfpathlineto{\pgfqpoint{4.582239in}{3.093709in}}%
\pgfpathlineto{\pgfqpoint{4.589851in}{3.109962in}}%
\pgfpathlineto{\pgfqpoint{4.597462in}{3.126479in}}%
\pgfpathlineto{\pgfqpoint{4.584010in}{3.131451in}}%
\pgfpathlineto{\pgfqpoint{4.570566in}{3.136617in}}%
\pgfpathlineto{\pgfqpoint{4.557128in}{3.141977in}}%
\pgfpathlineto{\pgfqpoint{4.543697in}{3.147532in}}%
\pgfpathlineto{\pgfqpoint{4.536076in}{3.130410in}}%
\pgfpathlineto{\pgfqpoint{4.528455in}{3.113559in}}%
\pgfpathlineto{\pgfqpoint{4.520833in}{3.096971in}}%
\pgfpathlineto{\pgfqpoint{4.513210in}{3.080639in}}%
\pgfpathclose%
\pgfusepath{fill}%
\end{pgfscope}%
\begin{pgfscope}%
\pgfpathrectangle{\pgfqpoint{1.150000in}{0.150000in}}{\pgfqpoint{5.700000in}{5.700000in}}%
\pgfusepath{clip}%
\pgfsetbuttcap%
\pgfsetroundjoin%
\definecolor{currentfill}{rgb}{0.253935,0.265254,0.529983}%
\pgfsetfillcolor{currentfill}%
\pgfsetfillopacity{0.800000}%
\pgfsetlinewidth{0.000000pt}%
\definecolor{currentstroke}{rgb}{0.000000,0.000000,0.000000}%
\pgfsetstrokecolor{currentstroke}%
\pgfsetdash{}{0pt}%
\pgfpathmoveto{\pgfqpoint{3.816620in}{2.899603in}}%
\pgfpathlineto{\pgfqpoint{3.829947in}{2.890315in}}%
\pgfpathlineto{\pgfqpoint{3.843276in}{2.881259in}}%
\pgfpathlineto{\pgfqpoint{3.856607in}{2.872431in}}%
\pgfpathlineto{\pgfqpoint{3.869940in}{2.863832in}}%
\pgfpathlineto{\pgfqpoint{3.877700in}{2.878331in}}%
\pgfpathlineto{\pgfqpoint{3.885456in}{2.892991in}}%
\pgfpathlineto{\pgfqpoint{3.893208in}{2.907818in}}%
\pgfpathlineto{\pgfqpoint{3.900956in}{2.922815in}}%
\pgfpathlineto{\pgfqpoint{3.887629in}{2.931790in}}%
\pgfpathlineto{\pgfqpoint{3.874305in}{2.940994in}}%
\pgfpathlineto{\pgfqpoint{3.860983in}{2.950427in}}%
\pgfpathlineto{\pgfqpoint{3.847662in}{2.960091in}}%
\pgfpathlineto{\pgfqpoint{3.839908in}{2.944706in}}%
\pgfpathlineto{\pgfqpoint{3.832150in}{2.929499in}}%
\pgfpathlineto{\pgfqpoint{3.824387in}{2.914466in}}%
\pgfpathlineto{\pgfqpoint{3.816620in}{2.899603in}}%
\pgfpathclose%
\pgfusepath{fill}%
\end{pgfscope}%
\begin{pgfscope}%
\pgfpathrectangle{\pgfqpoint{1.150000in}{0.150000in}}{\pgfqpoint{5.700000in}{5.700000in}}%
\pgfusepath{clip}%
\pgfsetbuttcap%
\pgfsetroundjoin%
\definecolor{currentfill}{rgb}{0.229739,0.322361,0.545706}%
\pgfsetfillcolor{currentfill}%
\pgfsetfillopacity{0.800000}%
\pgfsetlinewidth{0.000000pt}%
\definecolor{currentstroke}{rgb}{0.000000,0.000000,0.000000}%
\pgfsetstrokecolor{currentstroke}%
\pgfsetdash{}{0pt}%
\pgfpathmoveto{\pgfqpoint{4.428977in}{3.037299in}}%
\pgfpathlineto{\pgfqpoint{4.442400in}{3.032114in}}%
\pgfpathlineto{\pgfqpoint{4.455830in}{3.027128in}}%
\pgfpathlineto{\pgfqpoint{4.469266in}{3.022338in}}%
\pgfpathlineto{\pgfqpoint{4.482709in}{3.017745in}}%
\pgfpathlineto{\pgfqpoint{4.490336in}{3.033117in}}%
\pgfpathlineto{\pgfqpoint{4.497962in}{3.048719in}}%
\pgfpathlineto{\pgfqpoint{4.505587in}{3.064558in}}%
\pgfpathlineto{\pgfqpoint{4.513210in}{3.080639in}}%
\pgfpathlineto{\pgfqpoint{4.499777in}{3.085795in}}%
\pgfpathlineto{\pgfqpoint{4.486350in}{3.091148in}}%
\pgfpathlineto{\pgfqpoint{4.472930in}{3.096697in}}%
\pgfpathlineto{\pgfqpoint{4.459516in}{3.102444in}}%
\pgfpathlineto{\pgfqpoint{4.451883in}{3.085788in}}%
\pgfpathlineto{\pgfqpoint{4.444249in}{3.069383in}}%
\pgfpathlineto{\pgfqpoint{4.436614in}{3.053222in}}%
\pgfpathlineto{\pgfqpoint{4.428977in}{3.037299in}}%
\pgfpathclose%
\pgfusepath{fill}%
\end{pgfscope}%
\begin{pgfscope}%
\pgfpathrectangle{\pgfqpoint{1.150000in}{0.150000in}}{\pgfqpoint{5.700000in}{5.700000in}}%
\pgfusepath{clip}%
\pgfsetbuttcap%
\pgfsetroundjoin%
\definecolor{currentfill}{rgb}{0.227802,0.326594,0.546532}%
\pgfsetfillcolor{currentfill}%
\pgfsetfillopacity{0.800000}%
\pgfsetlinewidth{0.000000pt}%
\definecolor{currentstroke}{rgb}{0.000000,0.000000,0.000000}%
\pgfsetstrokecolor{currentstroke}%
\pgfsetdash{}{0pt}%
\pgfpathmoveto{\pgfqpoint{3.434283in}{3.066591in}}%
\pgfpathlineto{\pgfqpoint{3.447631in}{3.051946in}}%
\pgfpathlineto{\pgfqpoint{3.460977in}{3.037574in}}%
\pgfpathlineto{\pgfqpoint{3.474321in}{3.023471in}}%
\pgfpathlineto{\pgfqpoint{3.487662in}{3.009636in}}%
\pgfpathlineto{\pgfqpoint{3.495500in}{3.024810in}}%
\pgfpathlineto{\pgfqpoint{3.503334in}{3.040160in}}%
\pgfpathlineto{\pgfqpoint{3.511162in}{3.055690in}}%
\pgfpathlineto{\pgfqpoint{3.518985in}{3.071404in}}%
\pgfpathlineto{\pgfqpoint{3.505650in}{3.085556in}}%
\pgfpathlineto{\pgfqpoint{3.492314in}{3.099976in}}%
\pgfpathlineto{\pgfqpoint{3.478975in}{3.114666in}}%
\pgfpathlineto{\pgfqpoint{3.465633in}{3.129628in}}%
\pgfpathlineto{\pgfqpoint{3.457804in}{3.113585in}}%
\pgfpathlineto{\pgfqpoint{3.449969in}{3.097733in}}%
\pgfpathlineto{\pgfqpoint{3.442128in}{3.082070in}}%
\pgfpathlineto{\pgfqpoint{3.434283in}{3.066591in}}%
\pgfpathclose%
\pgfusepath{fill}%
\end{pgfscope}%
\begin{pgfscope}%
\pgfpathrectangle{\pgfqpoint{1.150000in}{0.150000in}}{\pgfqpoint{5.700000in}{5.700000in}}%
\pgfusepath{clip}%
\pgfsetbuttcap%
\pgfsetroundjoin%
\definecolor{currentfill}{rgb}{0.162142,0.474838,0.558140}%
\pgfsetfillcolor{currentfill}%
\pgfsetfillopacity{0.800000}%
\pgfsetlinewidth{0.000000pt}%
\definecolor{currentstroke}{rgb}{0.000000,0.000000,0.000000}%
\pgfsetstrokecolor{currentstroke}%
\pgfsetdash{}{0pt}%
\pgfpathmoveto{\pgfqpoint{3.197848in}{3.489673in}}%
\pgfpathlineto{\pgfqpoint{3.211296in}{3.468755in}}%
\pgfpathlineto{\pgfqpoint{3.224736in}{3.448162in}}%
\pgfpathlineto{\pgfqpoint{3.238169in}{3.427889in}}%
\pgfpathlineto{\pgfqpoint{3.251595in}{3.407934in}}%
\pgfpathlineto{\pgfqpoint{3.259444in}{3.425538in}}%
\pgfpathlineto{\pgfqpoint{3.267286in}{3.443375in}}%
\pgfpathlineto{\pgfqpoint{3.275123in}{3.461449in}}%
\pgfpathlineto{\pgfqpoint{3.282953in}{3.479765in}}%
\pgfpathlineto{\pgfqpoint{3.269534in}{3.500077in}}%
\pgfpathlineto{\pgfqpoint{3.256107in}{3.520708in}}%
\pgfpathlineto{\pgfqpoint{3.242672in}{3.541660in}}%
\pgfpathlineto{\pgfqpoint{3.229230in}{3.562936in}}%
\pgfpathlineto{\pgfqpoint{3.221394in}{3.544250in}}%
\pgfpathlineto{\pgfqpoint{3.213552in}{3.525813in}}%
\pgfpathlineto{\pgfqpoint{3.205703in}{3.507622in}}%
\pgfpathlineto{\pgfqpoint{3.197848in}{3.489673in}}%
\pgfpathclose%
\pgfusepath{fill}%
\end{pgfscope}%
\begin{pgfscope}%
\pgfpathrectangle{\pgfqpoint{1.150000in}{0.150000in}}{\pgfqpoint{5.700000in}{5.700000in}}%
\pgfusepath{clip}%
\pgfsetbuttcap%
\pgfsetroundjoin%
\definecolor{currentfill}{rgb}{0.212395,0.359683,0.551710}%
\pgfsetfillcolor{currentfill}%
\pgfsetfillopacity{0.800000}%
\pgfsetlinewidth{0.000000pt}%
\definecolor{currentstroke}{rgb}{0.000000,0.000000,0.000000}%
\pgfsetstrokecolor{currentstroke}%
\pgfsetdash{}{0pt}%
\pgfpathmoveto{\pgfqpoint{4.597462in}{3.126479in}}%
\pgfpathlineto{\pgfqpoint{4.610921in}{3.121698in}}%
\pgfpathlineto{\pgfqpoint{4.624387in}{3.117110in}}%
\pgfpathlineto{\pgfqpoint{4.637861in}{3.112713in}}%
\pgfpathlineto{\pgfqpoint{4.651342in}{3.108507in}}%
\pgfpathlineto{\pgfqpoint{4.658942in}{3.124679in}}%
\pgfpathlineto{\pgfqpoint{4.666543in}{3.141120in}}%
\pgfpathlineto{\pgfqpoint{4.674143in}{3.157838in}}%
\pgfpathlineto{\pgfqpoint{4.681744in}{3.174839in}}%
\pgfpathlineto{\pgfqpoint{4.668274in}{3.179671in}}%
\pgfpathlineto{\pgfqpoint{4.654811in}{3.184694in}}%
\pgfpathlineto{\pgfqpoint{4.641356in}{3.189907in}}%
\pgfpathlineto{\pgfqpoint{4.627907in}{3.195313in}}%
\pgfpathlineto{\pgfqpoint{4.620295in}{3.177674in}}%
\pgfpathlineto{\pgfqpoint{4.612684in}{3.160327in}}%
\pgfpathlineto{\pgfqpoint{4.605073in}{3.143264in}}%
\pgfpathlineto{\pgfqpoint{4.597462in}{3.126479in}}%
\pgfpathclose%
\pgfusepath{fill}%
\end{pgfscope}%
\begin{pgfscope}%
\pgfpathrectangle{\pgfqpoint{1.150000in}{0.150000in}}{\pgfqpoint{5.700000in}{5.700000in}}%
\pgfusepath{clip}%
\pgfsetbuttcap%
\pgfsetroundjoin%
\definecolor{currentfill}{rgb}{0.204903,0.375746,0.553533}%
\pgfsetfillcolor{currentfill}%
\pgfsetfillopacity{0.800000}%
\pgfsetlinewidth{0.000000pt}%
\definecolor{currentstroke}{rgb}{0.000000,0.000000,0.000000}%
\pgfsetstrokecolor{currentstroke}%
\pgfsetdash{}{0pt}%
\pgfpathmoveto{\pgfqpoint{3.327366in}{3.193803in}}%
\pgfpathlineto{\pgfqpoint{3.340745in}{3.176902in}}%
\pgfpathlineto{\pgfqpoint{3.354120in}{3.160292in}}%
\pgfpathlineto{\pgfqpoint{3.367490in}{3.143969in}}%
\pgfpathlineto{\pgfqpoint{3.380856in}{3.127932in}}%
\pgfpathlineto{\pgfqpoint{3.388709in}{3.143717in}}%
\pgfpathlineto{\pgfqpoint{3.396556in}{3.159695in}}%
\pgfpathlineto{\pgfqpoint{3.404397in}{3.175869in}}%
\pgfpathlineto{\pgfqpoint{3.412233in}{3.192243in}}%
\pgfpathlineto{\pgfqpoint{3.398874in}{3.208599in}}%
\pgfpathlineto{\pgfqpoint{3.385511in}{3.225242in}}%
\pgfpathlineto{\pgfqpoint{3.372143in}{3.242172in}}%
\pgfpathlineto{\pgfqpoint{3.358771in}{3.259393in}}%
\pgfpathlineto{\pgfqpoint{3.350928in}{3.242687in}}%
\pgfpathlineto{\pgfqpoint{3.343080in}{3.226189in}}%
\pgfpathlineto{\pgfqpoint{3.335226in}{3.209896in}}%
\pgfpathlineto{\pgfqpoint{3.327366in}{3.193803in}}%
\pgfpathclose%
\pgfusepath{fill}%
\end{pgfscope}%
\begin{pgfscope}%
\pgfpathrectangle{\pgfqpoint{1.150000in}{0.150000in}}{\pgfqpoint{5.700000in}{5.700000in}}%
\pgfusepath{clip}%
\pgfsetbuttcap%
\pgfsetroundjoin%
\definecolor{currentfill}{rgb}{0.252194,0.269783,0.531579}%
\pgfsetfillcolor{currentfill}%
\pgfsetfillopacity{0.800000}%
\pgfsetlinewidth{0.000000pt}%
\definecolor{currentstroke}{rgb}{0.000000,0.000000,0.000000}%
\pgfsetstrokecolor{currentstroke}%
\pgfsetdash{}{0pt}%
\pgfpathmoveto{\pgfqpoint{3.678894in}{2.921778in}}%
\pgfpathlineto{\pgfqpoint{3.692217in}{2.910937in}}%
\pgfpathlineto{\pgfqpoint{3.705541in}{2.900339in}}%
\pgfpathlineto{\pgfqpoint{3.718866in}{2.889982in}}%
\pgfpathlineto{\pgfqpoint{3.732192in}{2.879864in}}%
\pgfpathlineto{\pgfqpoint{3.739983in}{2.894426in}}%
\pgfpathlineto{\pgfqpoint{3.747770in}{2.909148in}}%
\pgfpathlineto{\pgfqpoint{3.755552in}{2.924035in}}%
\pgfpathlineto{\pgfqpoint{3.763330in}{2.939090in}}%
\pgfpathlineto{\pgfqpoint{3.750011in}{2.949553in}}%
\pgfpathlineto{\pgfqpoint{3.736693in}{2.960256in}}%
\pgfpathlineto{\pgfqpoint{3.723375in}{2.971201in}}%
\pgfpathlineto{\pgfqpoint{3.710058in}{2.982387in}}%
\pgfpathlineto{\pgfqpoint{3.702274in}{2.966974in}}%
\pgfpathlineto{\pgfqpoint{3.694485in}{2.951738in}}%
\pgfpathlineto{\pgfqpoint{3.686692in}{2.936674in}}%
\pgfpathlineto{\pgfqpoint{3.678894in}{2.921778in}}%
\pgfpathclose%
\pgfusepath{fill}%
\end{pgfscope}%
\begin{pgfscope}%
\pgfpathrectangle{\pgfqpoint{1.150000in}{0.150000in}}{\pgfqpoint{5.700000in}{5.700000in}}%
\pgfusepath{clip}%
\pgfsetbuttcap%
\pgfsetroundjoin%
\definecolor{currentfill}{rgb}{0.237441,0.305202,0.541921}%
\pgfsetfillcolor{currentfill}%
\pgfsetfillopacity{0.800000}%
\pgfsetlinewidth{0.000000pt}%
\definecolor{currentstroke}{rgb}{0.000000,0.000000,0.000000}%
\pgfsetstrokecolor{currentstroke}%
\pgfsetdash{}{0pt}%
\pgfpathmoveto{\pgfqpoint{4.344751in}{2.996460in}}%
\pgfpathlineto{\pgfqpoint{4.358158in}{2.991008in}}%
\pgfpathlineto{\pgfqpoint{4.371570in}{2.985757in}}%
\pgfpathlineto{\pgfqpoint{4.384989in}{2.980706in}}%
\pgfpathlineto{\pgfqpoint{4.398414in}{2.975855in}}%
\pgfpathlineto{\pgfqpoint{4.406058in}{2.990891in}}%
\pgfpathlineto{\pgfqpoint{4.413699in}{3.006140in}}%
\pgfpathlineto{\pgfqpoint{4.421339in}{3.021607in}}%
\pgfpathlineto{\pgfqpoint{4.428977in}{3.037299in}}%
\pgfpathlineto{\pgfqpoint{4.415561in}{3.042682in}}%
\pgfpathlineto{\pgfqpoint{4.402151in}{3.048264in}}%
\pgfpathlineto{\pgfqpoint{4.388747in}{3.054046in}}%
\pgfpathlineto{\pgfqpoint{4.375349in}{3.060029in}}%
\pgfpathlineto{\pgfqpoint{4.367702in}{3.043794in}}%
\pgfpathlineto{\pgfqpoint{4.360054in}{3.027792in}}%
\pgfpathlineto{\pgfqpoint{4.352403in}{3.012016in}}%
\pgfpathlineto{\pgfqpoint{4.344751in}{2.996460in}}%
\pgfpathclose%
\pgfusepath{fill}%
\end{pgfscope}%
\begin{pgfscope}%
\pgfpathrectangle{\pgfqpoint{1.150000in}{0.150000in}}{\pgfqpoint{5.700000in}{5.700000in}}%
\pgfusepath{clip}%
\pgfsetbuttcap%
\pgfsetroundjoin%
\definecolor{currentfill}{rgb}{0.565498,0.842430,0.262877}%
\pgfsetfillcolor{currentfill}%
\pgfsetfillopacity{0.800000}%
\pgfsetlinewidth{0.000000pt}%
\definecolor{currentstroke}{rgb}{0.000000,0.000000,0.000000}%
\pgfsetstrokecolor{currentstroke}%
\pgfsetdash{}{0pt}%
\pgfpathmoveto{\pgfqpoint{3.399663in}{4.668689in}}%
\pgfpathlineto{\pgfqpoint{3.413182in}{4.639779in}}%
\pgfpathlineto{\pgfqpoint{3.426691in}{4.611231in}}%
\pgfpathlineto{\pgfqpoint{3.440190in}{4.583044in}}%
\pgfpathlineto{\pgfqpoint{3.453678in}{4.555212in}}%
\pgfpathlineto{\pgfqpoint{3.461315in}{4.587871in}}%
\pgfpathlineto{\pgfqpoint{3.468948in}{4.621035in}}%
\pgfpathlineto{\pgfqpoint{3.476578in}{4.654710in}}%
\pgfpathlineto{\pgfqpoint{3.484204in}{4.688908in}}%
\pgfpathlineto{\pgfqpoint{3.470708in}{4.717477in}}%
\pgfpathlineto{\pgfqpoint{3.457202in}{4.746404in}}%
\pgfpathlineto{\pgfqpoint{3.443686in}{4.775693in}}%
\pgfpathlineto{\pgfqpoint{3.430159in}{4.805347in}}%
\pgfpathlineto{\pgfqpoint{3.422541in}{4.770393in}}%
\pgfpathlineto{\pgfqpoint{3.414919in}{4.735971in}}%
\pgfpathlineto{\pgfqpoint{3.407293in}{4.702073in}}%
\pgfpathlineto{\pgfqpoint{3.399663in}{4.668689in}}%
\pgfpathclose%
\pgfusepath{fill}%
\end{pgfscope}%
\begin{pgfscope}%
\pgfpathrectangle{\pgfqpoint{1.150000in}{0.150000in}}{\pgfqpoint{5.700000in}{5.700000in}}%
\pgfusepath{clip}%
\pgfsetbuttcap%
\pgfsetroundjoin%
\definecolor{currentfill}{rgb}{0.204903,0.375746,0.553533}%
\pgfsetfillcolor{currentfill}%
\pgfsetfillopacity{0.800000}%
\pgfsetlinewidth{0.000000pt}%
\definecolor{currentstroke}{rgb}{0.000000,0.000000,0.000000}%
\pgfsetstrokecolor{currentstroke}%
\pgfsetdash{}{0pt}%
\pgfpathmoveto{\pgfqpoint{4.681744in}{3.174839in}}%
\pgfpathlineto{\pgfqpoint{4.695222in}{3.170197in}}%
\pgfpathlineto{\pgfqpoint{4.708708in}{3.165744in}}%
\pgfpathlineto{\pgfqpoint{4.722201in}{3.161480in}}%
\pgfpathlineto{\pgfqpoint{4.735703in}{3.157405in}}%
\pgfpathlineto{\pgfqpoint{4.743293in}{3.174052in}}%
\pgfpathlineto{\pgfqpoint{4.750884in}{3.190990in}}%
\pgfpathlineto{\pgfqpoint{4.758476in}{3.208226in}}%
\pgfpathlineto{\pgfqpoint{4.766070in}{3.225769in}}%
\pgfpathlineto{\pgfqpoint{4.752580in}{3.230502in}}%
\pgfpathlineto{\pgfqpoint{4.739098in}{3.235423in}}%
\pgfpathlineto{\pgfqpoint{4.725624in}{3.240533in}}%
\pgfpathlineto{\pgfqpoint{4.712157in}{3.245832in}}%
\pgfpathlineto{\pgfqpoint{4.704552in}{3.227620in}}%
\pgfpathlineto{\pgfqpoint{4.696949in}{3.209723in}}%
\pgfpathlineto{\pgfqpoint{4.689346in}{3.192132in}}%
\pgfpathlineto{\pgfqpoint{4.681744in}{3.174839in}}%
\pgfpathclose%
\pgfusepath{fill}%
\end{pgfscope}%
\begin{pgfscope}%
\pgfpathrectangle{\pgfqpoint{1.150000in}{0.150000in}}{\pgfqpoint{5.700000in}{5.700000in}}%
\pgfusepath{clip}%
\pgfsetbuttcap%
\pgfsetroundjoin%
\definecolor{currentfill}{rgb}{0.237441,0.305202,0.541921}%
\pgfsetfillcolor{currentfill}%
\pgfsetfillopacity{0.800000}%
\pgfsetlinewidth{0.000000pt}%
\definecolor{currentstroke}{rgb}{0.000000,0.000000,0.000000}%
\pgfsetstrokecolor{currentstroke}%
\pgfsetdash{}{0pt}%
\pgfpathmoveto{\pgfqpoint{3.487662in}{3.009636in}}%
\pgfpathlineto{\pgfqpoint{3.501001in}{2.996067in}}%
\pgfpathlineto{\pgfqpoint{3.514338in}{2.982761in}}%
\pgfpathlineto{\pgfqpoint{3.527673in}{2.969717in}}%
\pgfpathlineto{\pgfqpoint{3.541007in}{2.956933in}}%
\pgfpathlineto{\pgfqpoint{3.548839in}{2.971803in}}%
\pgfpathlineto{\pgfqpoint{3.556665in}{2.986840in}}%
\pgfpathlineto{\pgfqpoint{3.564487in}{3.002050in}}%
\pgfpathlineto{\pgfqpoint{3.572303in}{3.017436in}}%
\pgfpathlineto{\pgfqpoint{3.558976in}{3.030536in}}%
\pgfpathlineto{\pgfqpoint{3.545647in}{3.043896in}}%
\pgfpathlineto{\pgfqpoint{3.532317in}{3.057518in}}%
\pgfpathlineto{\pgfqpoint{3.518985in}{3.071404in}}%
\pgfpathlineto{\pgfqpoint{3.511162in}{3.055690in}}%
\pgfpathlineto{\pgfqpoint{3.503334in}{3.040160in}}%
\pgfpathlineto{\pgfqpoint{3.495500in}{3.024810in}}%
\pgfpathlineto{\pgfqpoint{3.487662in}{3.009636in}}%
\pgfpathclose%
\pgfusepath{fill}%
\end{pgfscope}%
\begin{pgfscope}%
\pgfpathrectangle{\pgfqpoint{1.150000in}{0.150000in}}{\pgfqpoint{5.700000in}{5.700000in}}%
\pgfusepath{clip}%
\pgfsetbuttcap%
\pgfsetroundjoin%
\definecolor{currentfill}{rgb}{0.751884,0.874951,0.143228}%
\pgfsetfillcolor{currentfill}%
\pgfsetfillopacity{0.800000}%
\pgfsetlinewidth{0.000000pt}%
\definecolor{currentstroke}{rgb}{0.000000,0.000000,0.000000}%
\pgfsetstrokecolor{currentstroke}%
\pgfsetdash{}{0pt}%
\pgfpathmoveto{\pgfqpoint{3.599058in}{4.865240in}}%
\pgfpathlineto{\pgfqpoint{3.612524in}{4.836824in}}%
\pgfpathlineto{\pgfqpoint{3.625981in}{4.808748in}}%
\pgfpathlineto{\pgfqpoint{3.639429in}{4.781010in}}%
\pgfpathlineto{\pgfqpoint{3.652868in}{4.753606in}}%
\pgfpathlineto{\pgfqpoint{3.660490in}{4.791240in}}%
\pgfpathlineto{\pgfqpoint{3.668110in}{4.829471in}}%
\pgfpathlineto{\pgfqpoint{3.675730in}{4.868308in}}%
\pgfpathlineto{\pgfqpoint{3.662282in}{4.896346in}}%
\pgfpathlineto{\pgfqpoint{3.648826in}{4.924719in}}%
\pgfpathlineto{\pgfqpoint{3.635361in}{4.953432in}}%
\pgfpathlineto{\pgfqpoint{3.621886in}{4.982488in}}%
\pgfpathlineto{\pgfqpoint{3.614278in}{4.942790in}}%
\pgfpathlineto{\pgfqpoint{3.606669in}{4.903712in}}%
\pgfpathlineto{\pgfqpoint{3.599058in}{4.865240in}}%
\pgfpathclose%
\pgfusepath{fill}%
\end{pgfscope}%
\begin{pgfscope}%
\pgfpathrectangle{\pgfqpoint{1.150000in}{0.150000in}}{\pgfqpoint{5.700000in}{5.700000in}}%
\pgfusepath{clip}%
\pgfsetbuttcap%
\pgfsetroundjoin%
\definecolor{currentfill}{rgb}{0.255645,0.260703,0.528312}%
\pgfsetfillcolor{currentfill}%
\pgfsetfillopacity{0.800000}%
\pgfsetlinewidth{0.000000pt}%
\definecolor{currentstroke}{rgb}{0.000000,0.000000,0.000000}%
\pgfsetstrokecolor{currentstroke}%
\pgfsetdash{}{0pt}%
\pgfpathmoveto{\pgfqpoint{3.954290in}{2.889171in}}%
\pgfpathlineto{\pgfqpoint{3.967632in}{2.881318in}}%
\pgfpathlineto{\pgfqpoint{3.980976in}{2.873686in}}%
\pgfpathlineto{\pgfqpoint{3.994324in}{2.866273in}}%
\pgfpathlineto{\pgfqpoint{4.007676in}{2.859079in}}%
\pgfpathlineto{\pgfqpoint{4.015407in}{2.873460in}}%
\pgfpathlineto{\pgfqpoint{4.023133in}{2.888006in}}%
\pgfpathlineto{\pgfqpoint{4.030856in}{2.902721in}}%
\pgfpathlineto{\pgfqpoint{4.038576in}{2.917610in}}%
\pgfpathlineto{\pgfqpoint{4.025231in}{2.925211in}}%
\pgfpathlineto{\pgfqpoint{4.011889in}{2.933031in}}%
\pgfpathlineto{\pgfqpoint{3.998552in}{2.941070in}}%
\pgfpathlineto{\pgfqpoint{3.985217in}{2.949331in}}%
\pgfpathlineto{\pgfqpoint{3.977491in}{2.934023in}}%
\pgfpathlineto{\pgfqpoint{3.969761in}{2.918897in}}%
\pgfpathlineto{\pgfqpoint{3.962028in}{2.903948in}}%
\pgfpathlineto{\pgfqpoint{3.954290in}{2.889171in}}%
\pgfpathclose%
\pgfusepath{fill}%
\end{pgfscope}%
\begin{pgfscope}%
\pgfpathrectangle{\pgfqpoint{1.150000in}{0.150000in}}{\pgfqpoint{5.700000in}{5.700000in}}%
\pgfusepath{clip}%
\pgfsetbuttcap%
\pgfsetroundjoin%
\definecolor{currentfill}{rgb}{0.243113,0.292092,0.538516}%
\pgfsetfillcolor{currentfill}%
\pgfsetfillopacity{0.800000}%
\pgfsetlinewidth{0.000000pt}%
\definecolor{currentstroke}{rgb}{0.000000,0.000000,0.000000}%
\pgfsetstrokecolor{currentstroke}%
\pgfsetdash{}{0pt}%
\pgfpathmoveto{\pgfqpoint{4.260520in}{2.958151in}}%
\pgfpathlineto{\pgfqpoint{4.273911in}{2.952387in}}%
\pgfpathlineto{\pgfqpoint{4.287308in}{2.946828in}}%
\pgfpathlineto{\pgfqpoint{4.300711in}{2.941472in}}%
\pgfpathlineto{\pgfqpoint{4.314120in}{2.936318in}}%
\pgfpathlineto{\pgfqpoint{4.321781in}{2.951053in}}%
\pgfpathlineto{\pgfqpoint{4.329440in}{2.965985in}}%
\pgfpathlineto{\pgfqpoint{4.337097in}{2.981118in}}%
\pgfpathlineto{\pgfqpoint{4.344751in}{2.996460in}}%
\pgfpathlineto{\pgfqpoint{4.331351in}{3.002113in}}%
\pgfpathlineto{\pgfqpoint{4.317956in}{3.007970in}}%
\pgfpathlineto{\pgfqpoint{4.304567in}{3.014029in}}%
\pgfpathlineto{\pgfqpoint{4.291184in}{3.020293in}}%
\pgfpathlineto{\pgfqpoint{4.283521in}{3.004440in}}%
\pgfpathlineto{\pgfqpoint{4.275857in}{2.988802in}}%
\pgfpathlineto{\pgfqpoint{4.268190in}{2.973375in}}%
\pgfpathlineto{\pgfqpoint{4.260520in}{2.958151in}}%
\pgfpathclose%
\pgfusepath{fill}%
\end{pgfscope}%
\begin{pgfscope}%
\pgfpathrectangle{\pgfqpoint{1.150000in}{0.150000in}}{\pgfqpoint{5.700000in}{5.700000in}}%
\pgfusepath{clip}%
\pgfsetbuttcap%
\pgfsetroundjoin%
\definecolor{currentfill}{rgb}{0.194100,0.399323,0.555565}%
\pgfsetfillcolor{currentfill}%
\pgfsetfillopacity{0.800000}%
\pgfsetlinewidth{0.000000pt}%
\definecolor{currentstroke}{rgb}{0.000000,0.000000,0.000000}%
\pgfsetstrokecolor{currentstroke}%
\pgfsetdash{}{0pt}%
\pgfpathmoveto{\pgfqpoint{3.273798in}{3.264360in}}%
\pgfpathlineto{\pgfqpoint{3.287198in}{3.246272in}}%
\pgfpathlineto{\pgfqpoint{3.300593in}{3.228486in}}%
\pgfpathlineto{\pgfqpoint{3.313982in}{3.210997in}}%
\pgfpathlineto{\pgfqpoint{3.327366in}{3.193803in}}%
\pgfpathlineto{\pgfqpoint{3.335226in}{3.209896in}}%
\pgfpathlineto{\pgfqpoint{3.343080in}{3.226189in}}%
\pgfpathlineto{\pgfqpoint{3.350928in}{3.242687in}}%
\pgfpathlineto{\pgfqpoint{3.358771in}{3.259393in}}%
\pgfpathlineto{\pgfqpoint{3.345394in}{3.276907in}}%
\pgfpathlineto{\pgfqpoint{3.332011in}{3.294717in}}%
\pgfpathlineto{\pgfqpoint{3.318623in}{3.312825in}}%
\pgfpathlineto{\pgfqpoint{3.305230in}{3.331234in}}%
\pgfpathlineto{\pgfqpoint{3.297381in}{3.314195in}}%
\pgfpathlineto{\pgfqpoint{3.289526in}{3.297372in}}%
\pgfpathlineto{\pgfqpoint{3.281665in}{3.280761in}}%
\pgfpathlineto{\pgfqpoint{3.273798in}{3.264360in}}%
\pgfpathclose%
\pgfusepath{fill}%
\end{pgfscope}%
\begin{pgfscope}%
\pgfpathrectangle{\pgfqpoint{1.150000in}{0.150000in}}{\pgfqpoint{5.700000in}{5.700000in}}%
\pgfusepath{clip}%
\pgfsetbuttcap%
\pgfsetroundjoin%
\definecolor{currentfill}{rgb}{0.195860,0.395433,0.555276}%
\pgfsetfillcolor{currentfill}%
\pgfsetfillopacity{0.800000}%
\pgfsetlinewidth{0.000000pt}%
\definecolor{currentstroke}{rgb}{0.000000,0.000000,0.000000}%
\pgfsetstrokecolor{currentstroke}%
\pgfsetdash{}{0pt}%
\pgfpathmoveto{\pgfqpoint{4.766070in}{3.225769in}}%
\pgfpathlineto{\pgfqpoint{4.779567in}{3.221224in}}%
\pgfpathlineto{\pgfqpoint{4.793073in}{3.216865in}}%
\pgfpathlineto{\pgfqpoint{4.806587in}{3.212692in}}%
\pgfpathlineto{\pgfqpoint{4.820109in}{3.208706in}}%
\pgfpathlineto{\pgfqpoint{4.827691in}{3.225886in}}%
\pgfpathlineto{\pgfqpoint{4.835276in}{3.243380in}}%
\pgfpathlineto{\pgfqpoint{4.842863in}{3.261196in}}%
\pgfpathlineto{\pgfqpoint{4.850452in}{3.279343in}}%
\pgfpathlineto{\pgfqpoint{4.836943in}{3.284018in}}%
\pgfpathlineto{\pgfqpoint{4.823441in}{3.288879in}}%
\pgfpathlineto{\pgfqpoint{4.809948in}{3.293927in}}%
\pgfpathlineto{\pgfqpoint{4.796462in}{3.299161in}}%
\pgfpathlineto{\pgfqpoint{4.788861in}{3.280314in}}%
\pgfpathlineto{\pgfqpoint{4.781262in}{3.261805in}}%
\pgfpathlineto{\pgfqpoint{4.773665in}{3.243626in}}%
\pgfpathlineto{\pgfqpoint{4.766070in}{3.225769in}}%
\pgfpathclose%
\pgfusepath{fill}%
\end{pgfscope}%
\begin{pgfscope}%
\pgfpathrectangle{\pgfqpoint{1.150000in}{0.150000in}}{\pgfqpoint{5.700000in}{5.700000in}}%
\pgfusepath{clip}%
\pgfsetbuttcap%
\pgfsetroundjoin%
\definecolor{currentfill}{rgb}{0.248629,0.278775,0.534556}%
\pgfsetfillcolor{currentfill}%
\pgfsetfillopacity{0.800000}%
\pgfsetlinewidth{0.000000pt}%
\definecolor{currentstroke}{rgb}{0.000000,0.000000,0.000000}%
\pgfsetstrokecolor{currentstroke}%
\pgfsetdash{}{0pt}%
\pgfpathmoveto{\pgfqpoint{4.176272in}{2.922427in}}%
\pgfpathlineto{\pgfqpoint{4.189650in}{2.916306in}}%
\pgfpathlineto{\pgfqpoint{4.203033in}{2.910392in}}%
\pgfpathlineto{\pgfqpoint{4.216421in}{2.904686in}}%
\pgfpathlineto{\pgfqpoint{4.229814in}{2.899186in}}%
\pgfpathlineto{\pgfqpoint{4.237495in}{2.913649in}}%
\pgfpathlineto{\pgfqpoint{4.245173in}{2.928294in}}%
\pgfpathlineto{\pgfqpoint{4.252848in}{2.943126in}}%
\pgfpathlineto{\pgfqpoint{4.260520in}{2.958151in}}%
\pgfpathlineto{\pgfqpoint{4.247134in}{2.964120in}}%
\pgfpathlineto{\pgfqpoint{4.233754in}{2.970296in}}%
\pgfpathlineto{\pgfqpoint{4.220379in}{2.976678in}}%
\pgfpathlineto{\pgfqpoint{4.207009in}{2.983269in}}%
\pgfpathlineto{\pgfqpoint{4.199329in}{2.967763in}}%
\pgfpathlineto{\pgfqpoint{4.191646in}{2.952458in}}%
\pgfpathlineto{\pgfqpoint{4.183961in}{2.937348in}}%
\pgfpathlineto{\pgfqpoint{4.176272in}{2.922427in}}%
\pgfpathclose%
\pgfusepath{fill}%
\end{pgfscope}%
\begin{pgfscope}%
\pgfpathrectangle{\pgfqpoint{1.150000in}{0.150000in}}{\pgfqpoint{5.700000in}{5.700000in}}%
\pgfusepath{clip}%
\pgfsetbuttcap%
\pgfsetroundjoin%
\definecolor{currentfill}{rgb}{0.244972,0.287675,0.537260}%
\pgfsetfillcolor{currentfill}%
\pgfsetfillopacity{0.800000}%
\pgfsetlinewidth{0.000000pt}%
\definecolor{currentstroke}{rgb}{0.000000,0.000000,0.000000}%
\pgfsetstrokecolor{currentstroke}%
\pgfsetdash{}{0pt}%
\pgfpathmoveto{\pgfqpoint{3.541007in}{2.956933in}}%
\pgfpathlineto{\pgfqpoint{3.554340in}{2.944407in}}%
\pgfpathlineto{\pgfqpoint{3.567672in}{2.932137in}}%
\pgfpathlineto{\pgfqpoint{3.581003in}{2.920122in}}%
\pgfpathlineto{\pgfqpoint{3.594333in}{2.908359in}}%
\pgfpathlineto{\pgfqpoint{3.602158in}{2.922924in}}%
\pgfpathlineto{\pgfqpoint{3.609977in}{2.937651in}}%
\pgfpathlineto{\pgfqpoint{3.617792in}{2.952541in}}%
\pgfpathlineto{\pgfqpoint{3.625602in}{2.967599in}}%
\pgfpathlineto{\pgfqpoint{3.612278in}{2.979678in}}%
\pgfpathlineto{\pgfqpoint{3.598954in}{2.992009in}}%
\pgfpathlineto{\pgfqpoint{3.585629in}{3.004594in}}%
\pgfpathlineto{\pgfqpoint{3.572303in}{3.017436in}}%
\pgfpathlineto{\pgfqpoint{3.564487in}{3.002050in}}%
\pgfpathlineto{\pgfqpoint{3.556665in}{2.986840in}}%
\pgfpathlineto{\pgfqpoint{3.548839in}{2.971803in}}%
\pgfpathlineto{\pgfqpoint{3.541007in}{2.956933in}}%
\pgfpathclose%
\pgfusepath{fill}%
\end{pgfscope}%
\begin{pgfscope}%
\pgfpathrectangle{\pgfqpoint{1.150000in}{0.150000in}}{\pgfqpoint{5.700000in}{5.700000in}}%
\pgfusepath{clip}%
\pgfsetbuttcap%
\pgfsetroundjoin%
\definecolor{currentfill}{rgb}{0.182256,0.426184,0.557120}%
\pgfsetfillcolor{currentfill}%
\pgfsetfillopacity{0.800000}%
\pgfsetlinewidth{0.000000pt}%
\definecolor{currentstroke}{rgb}{0.000000,0.000000,0.000000}%
\pgfsetstrokecolor{currentstroke}%
\pgfsetdash{}{0pt}%
\pgfpathmoveto{\pgfqpoint{3.220135in}{3.339768in}}%
\pgfpathlineto{\pgfqpoint{3.233561in}{3.320452in}}%
\pgfpathlineto{\pgfqpoint{3.246980in}{3.301446in}}%
\pgfpathlineto{\pgfqpoint{3.260392in}{3.282750in}}%
\pgfpathlineto{\pgfqpoint{3.273798in}{3.264360in}}%
\pgfpathlineto{\pgfqpoint{3.281665in}{3.280761in}}%
\pgfpathlineto{\pgfqpoint{3.289526in}{3.297372in}}%
\pgfpathlineto{\pgfqpoint{3.297381in}{3.314195in}}%
\pgfpathlineto{\pgfqpoint{3.305230in}{3.331234in}}%
\pgfpathlineto{\pgfqpoint{3.291831in}{3.349947in}}%
\pgfpathlineto{\pgfqpoint{3.278425in}{3.368966in}}%
\pgfpathlineto{\pgfqpoint{3.265013in}{3.388294in}}%
\pgfpathlineto{\pgfqpoint{3.251595in}{3.407934in}}%
\pgfpathlineto{\pgfqpoint{3.243739in}{3.390559in}}%
\pgfpathlineto{\pgfqpoint{3.235878in}{3.373409in}}%
\pgfpathlineto{\pgfqpoint{3.228010in}{3.356480in}}%
\pgfpathlineto{\pgfqpoint{3.220135in}{3.339768in}}%
\pgfpathclose%
\pgfusepath{fill}%
\end{pgfscope}%
\begin{pgfscope}%
\pgfpathrectangle{\pgfqpoint{1.150000in}{0.150000in}}{\pgfqpoint{5.700000in}{5.700000in}}%
\pgfusepath{clip}%
\pgfsetbuttcap%
\pgfsetroundjoin%
\definecolor{currentfill}{rgb}{0.730889,0.871916,0.156029}%
\pgfsetfillcolor{currentfill}%
\pgfsetfillopacity{0.800000}%
\pgfsetlinewidth{0.000000pt}%
\definecolor{currentstroke}{rgb}{0.000000,0.000000,0.000000}%
\pgfsetstrokecolor{currentstroke}%
\pgfsetdash{}{0pt}%
\pgfpathmoveto{\pgfqpoint{3.514674in}{4.831098in}}%
\pgfpathlineto{\pgfqpoint{3.528169in}{4.802105in}}%
\pgfpathlineto{\pgfqpoint{3.541653in}{4.773464in}}%
\pgfpathlineto{\pgfqpoint{3.555128in}{4.745173in}}%
\pgfpathlineto{\pgfqpoint{3.568593in}{4.717227in}}%
\pgfpathlineto{\pgfqpoint{3.576213in}{4.753369in}}%
\pgfpathlineto{\pgfqpoint{3.583830in}{4.790079in}}%
\pgfpathlineto{\pgfqpoint{3.591445in}{4.827366in}}%
\pgfpathlineto{\pgfqpoint{3.599058in}{4.865240in}}%
\pgfpathlineto{\pgfqpoint{3.585583in}{4.894001in}}%
\pgfpathlineto{\pgfqpoint{3.572098in}{4.923110in}}%
\pgfpathlineto{\pgfqpoint{3.558603in}{4.952570in}}%
\pgfpathlineto{\pgfqpoint{3.545098in}{4.982386in}}%
\pgfpathlineto{\pgfqpoint{3.537496in}{4.943676in}}%
\pgfpathlineto{\pgfqpoint{3.529892in}{4.905565in}}%
\pgfpathlineto{\pgfqpoint{3.522284in}{4.868042in}}%
\pgfpathlineto{\pgfqpoint{3.514674in}{4.831098in}}%
\pgfpathclose%
\pgfusepath{fill}%
\end{pgfscope}%
\begin{pgfscope}%
\pgfpathrectangle{\pgfqpoint{1.150000in}{0.150000in}}{\pgfqpoint{5.700000in}{5.700000in}}%
\pgfusepath{clip}%
\pgfsetbuttcap%
\pgfsetroundjoin%
\definecolor{currentfill}{rgb}{0.185556,0.418570,0.556753}%
\pgfsetfillcolor{currentfill}%
\pgfsetfillopacity{0.800000}%
\pgfsetlinewidth{0.000000pt}%
\definecolor{currentstroke}{rgb}{0.000000,0.000000,0.000000}%
\pgfsetstrokecolor{currentstroke}%
\pgfsetdash{}{0pt}%
\pgfpathmoveto{\pgfqpoint{4.850452in}{3.279343in}}%
\pgfpathlineto{\pgfqpoint{4.863969in}{3.274852in}}%
\pgfpathlineto{\pgfqpoint{4.877495in}{3.270546in}}%
\pgfpathlineto{\pgfqpoint{4.891029in}{3.266424in}}%
\pgfpathlineto{\pgfqpoint{4.904572in}{3.262486in}}%
\pgfpathlineto{\pgfqpoint{4.912150in}{3.280263in}}%
\pgfpathlineto{\pgfqpoint{4.919731in}{3.298378in}}%
\pgfpathlineto{\pgfqpoint{4.927316in}{3.316841in}}%
\pgfpathlineto{\pgfqpoint{4.934904in}{3.335659in}}%
\pgfpathlineto{\pgfqpoint{4.921375in}{3.340318in}}%
\pgfpathlineto{\pgfqpoint{4.907854in}{3.345161in}}%
\pgfpathlineto{\pgfqpoint{4.894341in}{3.350187in}}%
\pgfpathlineto{\pgfqpoint{4.880837in}{3.355398in}}%
\pgfpathlineto{\pgfqpoint{4.873235in}{3.335847in}}%
\pgfpathlineto{\pgfqpoint{4.865638in}{3.316660in}}%
\pgfpathlineto{\pgfqpoint{4.858043in}{3.297828in}}%
\pgfpathlineto{\pgfqpoint{4.850452in}{3.279343in}}%
\pgfpathclose%
\pgfusepath{fill}%
\end{pgfscope}%
\begin{pgfscope}%
\pgfpathrectangle{\pgfqpoint{1.150000in}{0.150000in}}{\pgfqpoint{5.700000in}{5.700000in}}%
\pgfusepath{clip}%
\pgfsetbuttcap%
\pgfsetroundjoin%
\definecolor{currentfill}{rgb}{0.257322,0.256130,0.526563}%
\pgfsetfillcolor{currentfill}%
\pgfsetfillopacity{0.800000}%
\pgfsetlinewidth{0.000000pt}%
\definecolor{currentstroke}{rgb}{0.000000,0.000000,0.000000}%
\pgfsetstrokecolor{currentstroke}%
\pgfsetdash{}{0pt}%
\pgfpathmoveto{\pgfqpoint{3.732192in}{2.879864in}}%
\pgfpathlineto{\pgfqpoint{3.745519in}{2.869985in}}%
\pgfpathlineto{\pgfqpoint{3.758847in}{2.860341in}}%
\pgfpathlineto{\pgfqpoint{3.772177in}{2.850933in}}%
\pgfpathlineto{\pgfqpoint{3.785509in}{2.841759in}}%
\pgfpathlineto{\pgfqpoint{3.793293in}{2.855987in}}%
\pgfpathlineto{\pgfqpoint{3.801073in}{2.870367in}}%
\pgfpathlineto{\pgfqpoint{3.808849in}{2.884905in}}%
\pgfpathlineto{\pgfqpoint{3.816620in}{2.899603in}}%
\pgfpathlineto{\pgfqpoint{3.803295in}{2.909123in}}%
\pgfpathlineto{\pgfqpoint{3.789972in}{2.918876in}}%
\pgfpathlineto{\pgfqpoint{3.776650in}{2.928865in}}%
\pgfpathlineto{\pgfqpoint{3.763330in}{2.939090in}}%
\pgfpathlineto{\pgfqpoint{3.755552in}{2.924035in}}%
\pgfpathlineto{\pgfqpoint{3.747770in}{2.909148in}}%
\pgfpathlineto{\pgfqpoint{3.739983in}{2.894426in}}%
\pgfpathlineto{\pgfqpoint{3.732192in}{2.879864in}}%
\pgfpathclose%
\pgfusepath{fill}%
\end{pgfscope}%
\begin{pgfscope}%
\pgfpathrectangle{\pgfqpoint{1.150000in}{0.150000in}}{\pgfqpoint{5.700000in}{5.700000in}}%
\pgfusepath{clip}%
\pgfsetbuttcap%
\pgfsetroundjoin%
\definecolor{currentfill}{rgb}{0.128729,0.563265,0.551229}%
\pgfsetfillcolor{currentfill}%
\pgfsetfillopacity{0.800000}%
\pgfsetlinewidth{0.000000pt}%
\definecolor{currentstroke}{rgb}{0.000000,0.000000,0.000000}%
\pgfsetstrokecolor{currentstroke}%
\pgfsetdash{}{0pt}%
\pgfpathmoveto{\pgfqpoint{3.121376in}{3.745216in}}%
\pgfpathlineto{\pgfqpoint{3.134891in}{3.721226in}}%
\pgfpathlineto{\pgfqpoint{3.148396in}{3.697588in}}%
\pgfpathlineto{\pgfqpoint{3.161891in}{3.674297in}}%
\pgfpathlineto{\pgfqpoint{3.175377in}{3.651350in}}%
\pgfpathlineto{\pgfqpoint{3.183212in}{3.670674in}}%
\pgfpathlineto{\pgfqpoint{3.191040in}{3.690265in}}%
\pgfpathlineto{\pgfqpoint{3.198861in}{3.710129in}}%
\pgfpathlineto{\pgfqpoint{3.206676in}{3.730271in}}%
\pgfpathlineto{\pgfqpoint{3.193196in}{3.753614in}}%
\pgfpathlineto{\pgfqpoint{3.179705in}{3.777303in}}%
\pgfpathlineto{\pgfqpoint{3.166205in}{3.801340in}}%
\pgfpathlineto{\pgfqpoint{3.152695in}{3.825728in}}%
\pgfpathlineto{\pgfqpoint{3.144875in}{3.805176in}}%
\pgfpathlineto{\pgfqpoint{3.137049in}{3.784910in}}%
\pgfpathlineto{\pgfqpoint{3.129216in}{3.764924in}}%
\pgfpathlineto{\pgfqpoint{3.121376in}{3.745216in}}%
\pgfpathclose%
\pgfusepath{fill}%
\end{pgfscope}%
\begin{pgfscope}%
\pgfpathrectangle{\pgfqpoint{1.150000in}{0.150000in}}{\pgfqpoint{5.700000in}{5.700000in}}%
\pgfusepath{clip}%
\pgfsetbuttcap%
\pgfsetroundjoin%
\definecolor{currentfill}{rgb}{0.149039,0.508051,0.557250}%
\pgfsetfillcolor{currentfill}%
\pgfsetfillopacity{0.800000}%
\pgfsetlinewidth{0.000000pt}%
\definecolor{currentstroke}{rgb}{0.000000,0.000000,0.000000}%
\pgfsetstrokecolor{currentstroke}%
\pgfsetdash{}{0pt}%
\pgfpathmoveto{\pgfqpoint{3.143971in}{3.576644in}}%
\pgfpathlineto{\pgfqpoint{3.157453in}{3.554399in}}%
\pgfpathlineto{\pgfqpoint{3.170927in}{3.532491in}}%
\pgfpathlineto{\pgfqpoint{3.184391in}{3.510917in}}%
\pgfpathlineto{\pgfqpoint{3.197848in}{3.489673in}}%
\pgfpathlineto{\pgfqpoint{3.205703in}{3.507622in}}%
\pgfpathlineto{\pgfqpoint{3.213552in}{3.525813in}}%
\pgfpathlineto{\pgfqpoint{3.221394in}{3.544250in}}%
\pgfpathlineto{\pgfqpoint{3.229230in}{3.562936in}}%
\pgfpathlineto{\pgfqpoint{3.215780in}{3.584540in}}%
\pgfpathlineto{\pgfqpoint{3.202321in}{3.606475in}}%
\pgfpathlineto{\pgfqpoint{3.188853in}{3.628744in}}%
\pgfpathlineto{\pgfqpoint{3.175377in}{3.651350in}}%
\pgfpathlineto{\pgfqpoint{3.167536in}{3.632290in}}%
\pgfpathlineto{\pgfqpoint{3.159688in}{3.613488in}}%
\pgfpathlineto{\pgfqpoint{3.151833in}{3.594941in}}%
\pgfpathlineto{\pgfqpoint{3.143971in}{3.576644in}}%
\pgfpathclose%
\pgfusepath{fill}%
\end{pgfscope}%
\begin{pgfscope}%
\pgfpathrectangle{\pgfqpoint{1.150000in}{0.150000in}}{\pgfqpoint{5.700000in}{5.700000in}}%
\pgfusepath{clip}%
\pgfsetbuttcap%
\pgfsetroundjoin%
\definecolor{currentfill}{rgb}{0.258965,0.251537,0.524736}%
\pgfsetfillcolor{currentfill}%
\pgfsetfillopacity{0.800000}%
\pgfsetlinewidth{0.000000pt}%
\definecolor{currentstroke}{rgb}{0.000000,0.000000,0.000000}%
\pgfsetstrokecolor{currentstroke}%
\pgfsetdash{}{0pt}%
\pgfpathmoveto{\pgfqpoint{3.869940in}{2.863832in}}%
\pgfpathlineto{\pgfqpoint{3.883276in}{2.855460in}}%
\pgfpathlineto{\pgfqpoint{3.896615in}{2.847314in}}%
\pgfpathlineto{\pgfqpoint{3.909957in}{2.839392in}}%
\pgfpathlineto{\pgfqpoint{3.923301in}{2.831693in}}%
\pgfpathlineto{\pgfqpoint{3.931055in}{2.845827in}}%
\pgfpathlineto{\pgfqpoint{3.938804in}{2.860115in}}%
\pgfpathlineto{\pgfqpoint{3.946549in}{2.874562in}}%
\pgfpathlineto{\pgfqpoint{3.954290in}{2.889171in}}%
\pgfpathlineto{\pgfqpoint{3.940952in}{2.897246in}}%
\pgfpathlineto{\pgfqpoint{3.927618in}{2.905544in}}%
\pgfpathlineto{\pgfqpoint{3.914286in}{2.914067in}}%
\pgfpathlineto{\pgfqpoint{3.900956in}{2.922815in}}%
\pgfpathlineto{\pgfqpoint{3.893208in}{2.907818in}}%
\pgfpathlineto{\pgfqpoint{3.885456in}{2.892991in}}%
\pgfpathlineto{\pgfqpoint{3.877700in}{2.878331in}}%
\pgfpathlineto{\pgfqpoint{3.869940in}{2.863832in}}%
\pgfpathclose%
\pgfusepath{fill}%
\end{pgfscope}%
\begin{pgfscope}%
\pgfpathrectangle{\pgfqpoint{1.150000in}{0.150000in}}{\pgfqpoint{5.700000in}{5.700000in}}%
\pgfusepath{clip}%
\pgfsetbuttcap%
\pgfsetroundjoin%
\definecolor{currentfill}{rgb}{0.253935,0.265254,0.529983}%
\pgfsetfillcolor{currentfill}%
\pgfsetfillopacity{0.800000}%
\pgfsetlinewidth{0.000000pt}%
\definecolor{currentstroke}{rgb}{0.000000,0.000000,0.000000}%
\pgfsetstrokecolor{currentstroke}%
\pgfsetdash{}{0pt}%
\pgfpathmoveto{\pgfqpoint{4.091995in}{2.889368in}}%
\pgfpathlineto{\pgfqpoint{4.105361in}{2.882842in}}%
\pgfpathlineto{\pgfqpoint{4.118731in}{2.876529in}}%
\pgfpathlineto{\pgfqpoint{4.132106in}{2.870426in}}%
\pgfpathlineto{\pgfqpoint{4.145486in}{2.864534in}}%
\pgfpathlineto{\pgfqpoint{4.153188in}{2.878749in}}%
\pgfpathlineto{\pgfqpoint{4.160886in}{2.893133in}}%
\pgfpathlineto{\pgfqpoint{4.168581in}{2.907691in}}%
\pgfpathlineto{\pgfqpoint{4.176272in}{2.922427in}}%
\pgfpathlineto{\pgfqpoint{4.162900in}{2.928758in}}%
\pgfpathlineto{\pgfqpoint{4.149532in}{2.935298in}}%
\pgfpathlineto{\pgfqpoint{4.136169in}{2.942049in}}%
\pgfpathlineto{\pgfqpoint{4.122810in}{2.949013in}}%
\pgfpathlineto{\pgfqpoint{4.115111in}{2.933827in}}%
\pgfpathlineto{\pgfqpoint{4.107409in}{2.918827in}}%
\pgfpathlineto{\pgfqpoint{4.099704in}{2.904009in}}%
\pgfpathlineto{\pgfqpoint{4.091995in}{2.889368in}}%
\pgfpathclose%
\pgfusepath{fill}%
\end{pgfscope}%
\begin{pgfscope}%
\pgfpathrectangle{\pgfqpoint{1.150000in}{0.150000in}}{\pgfqpoint{5.700000in}{5.700000in}}%
\pgfusepath{clip}%
\pgfsetbuttcap%
\pgfsetroundjoin%
\definecolor{currentfill}{rgb}{0.252194,0.269783,0.531579}%
\pgfsetfillcolor{currentfill}%
\pgfsetfillopacity{0.800000}%
\pgfsetlinewidth{0.000000pt}%
\definecolor{currentstroke}{rgb}{0.000000,0.000000,0.000000}%
\pgfsetstrokecolor{currentstroke}%
\pgfsetdash{}{0pt}%
\pgfpathmoveto{\pgfqpoint{3.594333in}{2.908359in}}%
\pgfpathlineto{\pgfqpoint{3.607663in}{2.896846in}}%
\pgfpathlineto{\pgfqpoint{3.620993in}{2.885583in}}%
\pgfpathlineto{\pgfqpoint{3.634323in}{2.874567in}}%
\pgfpathlineto{\pgfqpoint{3.647653in}{2.863797in}}%
\pgfpathlineto{\pgfqpoint{3.655470in}{2.878060in}}%
\pgfpathlineto{\pgfqpoint{3.663283in}{2.892475in}}%
\pgfpathlineto{\pgfqpoint{3.671091in}{2.907046in}}%
\pgfpathlineto{\pgfqpoint{3.678894in}{2.921778in}}%
\pgfpathlineto{\pgfqpoint{3.665571in}{2.932863in}}%
\pgfpathlineto{\pgfqpoint{3.652248in}{2.944194in}}%
\pgfpathlineto{\pgfqpoint{3.638925in}{2.955772in}}%
\pgfpathlineto{\pgfqpoint{3.625602in}{2.967599in}}%
\pgfpathlineto{\pgfqpoint{3.617792in}{2.952541in}}%
\pgfpathlineto{\pgfqpoint{3.609977in}{2.937651in}}%
\pgfpathlineto{\pgfqpoint{3.602158in}{2.922924in}}%
\pgfpathlineto{\pgfqpoint{3.594333in}{2.908359in}}%
\pgfpathclose%
\pgfusepath{fill}%
\end{pgfscope}%
\begin{pgfscope}%
\pgfpathrectangle{\pgfqpoint{1.150000in}{0.150000in}}{\pgfqpoint{5.700000in}{5.700000in}}%
\pgfusepath{clip}%
\pgfsetbuttcap%
\pgfsetroundjoin%
\definecolor{currentfill}{rgb}{0.709898,0.868751,0.169257}%
\pgfsetfillcolor{currentfill}%
\pgfsetfillopacity{0.800000}%
\pgfsetlinewidth{0.000000pt}%
\definecolor{currentstroke}{rgb}{0.000000,0.000000,0.000000}%
\pgfsetstrokecolor{currentstroke}%
\pgfsetdash{}{0pt}%
\pgfpathmoveto{\pgfqpoint{3.430159in}{4.805347in}}%
\pgfpathlineto{\pgfqpoint{3.443686in}{4.775693in}}%
\pgfpathlineto{\pgfqpoint{3.457202in}{4.746404in}}%
\pgfpathlineto{\pgfqpoint{3.470708in}{4.717477in}}%
\pgfpathlineto{\pgfqpoint{3.484204in}{4.688908in}}%
\pgfpathlineto{\pgfqpoint{3.491826in}{4.723636in}}%
\pgfpathlineto{\pgfqpoint{3.499445in}{4.758904in}}%
\pgfpathlineto{\pgfqpoint{3.507061in}{4.794722in}}%
\pgfpathlineto{\pgfqpoint{3.514674in}{4.831098in}}%
\pgfpathlineto{\pgfqpoint{3.501170in}{4.860447in}}%
\pgfpathlineto{\pgfqpoint{3.487655in}{4.890157in}}%
\pgfpathlineto{\pgfqpoint{3.474129in}{4.920231in}}%
\pgfpathlineto{\pgfqpoint{3.460592in}{4.950672in}}%
\pgfpathlineto{\pgfqpoint{3.452989in}{4.913495in}}%
\pgfpathlineto{\pgfqpoint{3.445383in}{4.876889in}}%
\pgfpathlineto{\pgfqpoint{3.437772in}{4.840843in}}%
\pgfpathlineto{\pgfqpoint{3.430159in}{4.805347in}}%
\pgfpathclose%
\pgfusepath{fill}%
\end{pgfscope}%
\begin{pgfscope}%
\pgfpathrectangle{\pgfqpoint{1.150000in}{0.150000in}}{\pgfqpoint{5.700000in}{5.700000in}}%
\pgfusepath{clip}%
\pgfsetbuttcap%
\pgfsetroundjoin%
\definecolor{currentfill}{rgb}{0.169646,0.456262,0.558030}%
\pgfsetfillcolor{currentfill}%
\pgfsetfillopacity{0.800000}%
\pgfsetlinewidth{0.000000pt}%
\definecolor{currentstroke}{rgb}{0.000000,0.000000,0.000000}%
\pgfsetstrokecolor{currentstroke}%
\pgfsetdash{}{0pt}%
\pgfpathmoveto{\pgfqpoint{3.166360in}{3.420208in}}%
\pgfpathlineto{\pgfqpoint{3.179815in}{3.399616in}}%
\pgfpathlineto{\pgfqpoint{3.193263in}{3.379347in}}%
\pgfpathlineto{\pgfqpoint{3.206703in}{3.359399in}}%
\pgfpathlineto{\pgfqpoint{3.220135in}{3.339768in}}%
\pgfpathlineto{\pgfqpoint{3.228010in}{3.356480in}}%
\pgfpathlineto{\pgfqpoint{3.235878in}{3.373409in}}%
\pgfpathlineto{\pgfqpoint{3.243739in}{3.390559in}}%
\pgfpathlineto{\pgfqpoint{3.251595in}{3.407934in}}%
\pgfpathlineto{\pgfqpoint{3.238169in}{3.427889in}}%
\pgfpathlineto{\pgfqpoint{3.224736in}{3.448162in}}%
\pgfpathlineto{\pgfqpoint{3.211296in}{3.468755in}}%
\pgfpathlineto{\pgfqpoint{3.197848in}{3.489673in}}%
\pgfpathlineto{\pgfqpoint{3.189986in}{3.471960in}}%
\pgfpathlineto{\pgfqpoint{3.182117in}{3.454482in}}%
\pgfpathlineto{\pgfqpoint{3.174242in}{3.437232in}}%
\pgfpathlineto{\pgfqpoint{3.166360in}{3.420208in}}%
\pgfpathclose%
\pgfusepath{fill}%
\end{pgfscope}%
\begin{pgfscope}%
\pgfpathrectangle{\pgfqpoint{1.150000in}{0.150000in}}{\pgfqpoint{5.700000in}{5.700000in}}%
\pgfusepath{clip}%
\pgfsetbuttcap%
\pgfsetroundjoin%
\definecolor{currentfill}{rgb}{0.177423,0.437527,0.557565}%
\pgfsetfillcolor{currentfill}%
\pgfsetfillopacity{0.800000}%
\pgfsetlinewidth{0.000000pt}%
\definecolor{currentstroke}{rgb}{0.000000,0.000000,0.000000}%
\pgfsetstrokecolor{currentstroke}%
\pgfsetdash{}{0pt}%
\pgfpathmoveto{\pgfqpoint{4.934904in}{3.335659in}}%
\pgfpathlineto{\pgfqpoint{4.948441in}{3.331183in}}%
\pgfpathlineto{\pgfqpoint{4.961987in}{3.326889in}}%
\pgfpathlineto{\pgfqpoint{4.975542in}{3.322778in}}%
\pgfpathlineto{\pgfqpoint{4.989106in}{3.318848in}}%
\pgfpathlineto{\pgfqpoint{4.996683in}{3.337290in}}%
\pgfpathlineto{\pgfqpoint{5.004265in}{3.356098in}}%
\pgfpathlineto{\pgfqpoint{5.011850in}{3.375279in}}%
\pgfpathlineto{\pgfqpoint{4.998298in}{3.379771in}}%
\pgfpathlineto{\pgfqpoint{4.984754in}{3.384444in}}%
\pgfpathlineto{\pgfqpoint{4.971218in}{3.389299in}}%
\pgfpathlineto{\pgfqpoint{4.957691in}{3.394337in}}%
\pgfpathlineto{\pgfqpoint{4.950091in}{3.374398in}}%
\pgfpathlineto{\pgfqpoint{4.942495in}{3.354842in}}%
\pgfpathlineto{\pgfqpoint{4.934904in}{3.335659in}}%
\pgfpathclose%
\pgfusepath{fill}%
\end{pgfscope}%
\begin{pgfscope}%
\pgfpathrectangle{\pgfqpoint{1.150000in}{0.150000in}}{\pgfqpoint{5.700000in}{5.700000in}}%
\pgfusepath{clip}%
\pgfsetbuttcap%
\pgfsetroundjoin%
\definecolor{currentfill}{rgb}{0.257322,0.256130,0.526563}%
\pgfsetfillcolor{currentfill}%
\pgfsetfillopacity{0.800000}%
\pgfsetlinewidth{0.000000pt}%
\definecolor{currentstroke}{rgb}{0.000000,0.000000,0.000000}%
\pgfsetstrokecolor{currentstroke}%
\pgfsetdash{}{0pt}%
\pgfpathmoveto{\pgfqpoint{4.007676in}{2.859079in}}%
\pgfpathlineto{\pgfqpoint{4.021032in}{2.852102in}}%
\pgfpathlineto{\pgfqpoint{4.034392in}{2.845341in}}%
\pgfpathlineto{\pgfqpoint{4.047756in}{2.838795in}}%
\pgfpathlineto{\pgfqpoint{4.061124in}{2.832464in}}%
\pgfpathlineto{\pgfqpoint{4.068848in}{2.846450in}}%
\pgfpathlineto{\pgfqpoint{4.076567in}{2.860593in}}%
\pgfpathlineto{\pgfqpoint{4.084283in}{2.874897in}}%
\pgfpathlineto{\pgfqpoint{4.091995in}{2.889368in}}%
\pgfpathlineto{\pgfqpoint{4.078634in}{2.896106in}}%
\pgfpathlineto{\pgfqpoint{4.065277in}{2.903059in}}%
\pgfpathlineto{\pgfqpoint{4.051924in}{2.910226in}}%
\pgfpathlineto{\pgfqpoint{4.038576in}{2.917610in}}%
\pgfpathlineto{\pgfqpoint{4.030856in}{2.902721in}}%
\pgfpathlineto{\pgfqpoint{4.023133in}{2.888006in}}%
\pgfpathlineto{\pgfqpoint{4.015407in}{2.873460in}}%
\pgfpathlineto{\pgfqpoint{4.007676in}{2.859079in}}%
\pgfpathclose%
\pgfusepath{fill}%
\end{pgfscope}%
\begin{pgfscope}%
\pgfpathrectangle{\pgfqpoint{1.150000in}{0.150000in}}{\pgfqpoint{5.700000in}{5.700000in}}%
\pgfusepath{clip}%
\pgfsetbuttcap%
\pgfsetroundjoin%
\definecolor{currentfill}{rgb}{0.229739,0.322361,0.545706}%
\pgfsetfillcolor{currentfill}%
\pgfsetfillopacity{0.800000}%
\pgfsetlinewidth{0.000000pt}%
\definecolor{currentstroke}{rgb}{0.000000,0.000000,0.000000}%
\pgfsetstrokecolor{currentstroke}%
\pgfsetdash{}{0pt}%
\pgfpathmoveto{\pgfqpoint{4.482709in}{3.017745in}}%
\pgfpathlineto{\pgfqpoint{4.496159in}{3.013347in}}%
\pgfpathlineto{\pgfqpoint{4.509616in}{3.009144in}}%
\pgfpathlineto{\pgfqpoint{4.523081in}{3.005135in}}%
\pgfpathlineto{\pgfqpoint{4.536553in}{3.001319in}}%
\pgfpathlineto{\pgfqpoint{4.544170in}{3.016140in}}%
\pgfpathlineto{\pgfqpoint{4.551786in}{3.031183in}}%
\pgfpathlineto{\pgfqpoint{4.559401in}{3.046455in}}%
\pgfpathlineto{\pgfqpoint{4.567014in}{3.061962in}}%
\pgfpathlineto{\pgfqpoint{4.553553in}{3.066341in}}%
\pgfpathlineto{\pgfqpoint{4.540098in}{3.070913in}}%
\pgfpathlineto{\pgfqpoint{4.526651in}{3.075679in}}%
\pgfpathlineto{\pgfqpoint{4.513210in}{3.080639in}}%
\pgfpathlineto{\pgfqpoint{4.505587in}{3.064558in}}%
\pgfpathlineto{\pgfqpoint{4.497962in}{3.048719in}}%
\pgfpathlineto{\pgfqpoint{4.490336in}{3.033117in}}%
\pgfpathlineto{\pgfqpoint{4.482709in}{3.017745in}}%
\pgfpathclose%
\pgfusepath{fill}%
\end{pgfscope}%
\begin{pgfscope}%
\pgfpathrectangle{\pgfqpoint{1.150000in}{0.150000in}}{\pgfqpoint{5.700000in}{5.700000in}}%
\pgfusepath{clip}%
\pgfsetbuttcap%
\pgfsetroundjoin%
\definecolor{currentfill}{rgb}{0.221989,0.339161,0.548752}%
\pgfsetfillcolor{currentfill}%
\pgfsetfillopacity{0.800000}%
\pgfsetlinewidth{0.000000pt}%
\definecolor{currentstroke}{rgb}{0.000000,0.000000,0.000000}%
\pgfsetstrokecolor{currentstroke}%
\pgfsetdash{}{0pt}%
\pgfpathmoveto{\pgfqpoint{4.567014in}{3.061962in}}%
\pgfpathlineto{\pgfqpoint{4.580484in}{3.057776in}}%
\pgfpathlineto{\pgfqpoint{4.593960in}{3.053782in}}%
\pgfpathlineto{\pgfqpoint{4.607445in}{3.049979in}}%
\pgfpathlineto{\pgfqpoint{4.620937in}{3.046368in}}%
\pgfpathlineto{\pgfqpoint{4.628539in}{3.061534in}}%
\pgfpathlineto{\pgfqpoint{4.636141in}{3.076941in}}%
\pgfpathlineto{\pgfqpoint{4.643741in}{3.092597in}}%
\pgfpathlineto{\pgfqpoint{4.651342in}{3.108507in}}%
\pgfpathlineto{\pgfqpoint{4.637861in}{3.112713in}}%
\pgfpathlineto{\pgfqpoint{4.624387in}{3.117110in}}%
\pgfpathlineto{\pgfqpoint{4.610921in}{3.121698in}}%
\pgfpathlineto{\pgfqpoint{4.597462in}{3.126479in}}%
\pgfpathlineto{\pgfqpoint{4.589851in}{3.109962in}}%
\pgfpathlineto{\pgfqpoint{4.582239in}{3.093709in}}%
\pgfpathlineto{\pgfqpoint{4.574627in}{3.077711in}}%
\pgfpathlineto{\pgfqpoint{4.567014in}{3.061962in}}%
\pgfpathclose%
\pgfusepath{fill}%
\end{pgfscope}%
\begin{pgfscope}%
\pgfpathrectangle{\pgfqpoint{1.150000in}{0.150000in}}{\pgfqpoint{5.700000in}{5.700000in}}%
\pgfusepath{clip}%
\pgfsetbuttcap%
\pgfsetroundjoin%
\definecolor{currentfill}{rgb}{0.225863,0.330805,0.547314}%
\pgfsetfillcolor{currentfill}%
\pgfsetfillopacity{0.800000}%
\pgfsetlinewidth{0.000000pt}%
\definecolor{currentstroke}{rgb}{0.000000,0.000000,0.000000}%
\pgfsetstrokecolor{currentstroke}%
\pgfsetdash{}{0pt}%
\pgfpathmoveto{\pgfqpoint{3.349386in}{3.066641in}}%
\pgfpathlineto{\pgfqpoint{3.362755in}{3.051173in}}%
\pgfpathlineto{\pgfqpoint{3.376121in}{3.035986in}}%
\pgfpathlineto{\pgfqpoint{3.389484in}{3.021077in}}%
\pgfpathlineto{\pgfqpoint{3.402843in}{3.006444in}}%
\pgfpathlineto{\pgfqpoint{3.410711in}{3.021223in}}%
\pgfpathlineto{\pgfqpoint{3.418574in}{3.036171in}}%
\pgfpathlineto{\pgfqpoint{3.426431in}{3.051292in}}%
\pgfpathlineto{\pgfqpoint{3.434283in}{3.066591in}}%
\pgfpathlineto{\pgfqpoint{3.420931in}{3.081509in}}%
\pgfpathlineto{\pgfqpoint{3.407576in}{3.096704in}}%
\pgfpathlineto{\pgfqpoint{3.394218in}{3.112178in}}%
\pgfpathlineto{\pgfqpoint{3.380856in}{3.127932in}}%
\pgfpathlineto{\pgfqpoint{3.372997in}{3.112335in}}%
\pgfpathlineto{\pgfqpoint{3.365133in}{3.096923in}}%
\pgfpathlineto{\pgfqpoint{3.357262in}{3.081693in}}%
\pgfpathlineto{\pgfqpoint{3.349386in}{3.066641in}}%
\pgfpathclose%
\pgfusepath{fill}%
\end{pgfscope}%
\begin{pgfscope}%
\pgfpathrectangle{\pgfqpoint{1.150000in}{0.150000in}}{\pgfqpoint{5.700000in}{5.700000in}}%
\pgfusepath{clip}%
\pgfsetbuttcap%
\pgfsetroundjoin%
\definecolor{currentfill}{rgb}{0.237441,0.305202,0.541921}%
\pgfsetfillcolor{currentfill}%
\pgfsetfillopacity{0.800000}%
\pgfsetlinewidth{0.000000pt}%
\definecolor{currentstroke}{rgb}{0.000000,0.000000,0.000000}%
\pgfsetstrokecolor{currentstroke}%
\pgfsetdash{}{0pt}%
\pgfpathmoveto{\pgfqpoint{4.398414in}{2.975855in}}%
\pgfpathlineto{\pgfqpoint{4.411846in}{2.971202in}}%
\pgfpathlineto{\pgfqpoint{4.425285in}{2.966747in}}%
\pgfpathlineto{\pgfqpoint{4.438730in}{2.962489in}}%
\pgfpathlineto{\pgfqpoint{4.452183in}{2.958427in}}%
\pgfpathlineto{\pgfqpoint{4.459817in}{2.972943in}}%
\pgfpathlineto{\pgfqpoint{4.467449in}{2.987664in}}%
\pgfpathlineto{\pgfqpoint{4.475080in}{3.002596in}}%
\pgfpathlineto{\pgfqpoint{4.482709in}{3.017745in}}%
\pgfpathlineto{\pgfqpoint{4.469266in}{3.022338in}}%
\pgfpathlineto{\pgfqpoint{4.455830in}{3.027128in}}%
\pgfpathlineto{\pgfqpoint{4.442400in}{3.032114in}}%
\pgfpathlineto{\pgfqpoint{4.428977in}{3.037299in}}%
\pgfpathlineto{\pgfqpoint{4.421339in}{3.021607in}}%
\pgfpathlineto{\pgfqpoint{4.413699in}{3.006140in}}%
\pgfpathlineto{\pgfqpoint{4.406058in}{2.990891in}}%
\pgfpathlineto{\pgfqpoint{4.398414in}{2.975855in}}%
\pgfpathclose%
\pgfusepath{fill}%
\end{pgfscope}%
\begin{pgfscope}%
\pgfpathrectangle{\pgfqpoint{1.150000in}{0.150000in}}{\pgfqpoint{5.700000in}{5.700000in}}%
\pgfusepath{clip}%
\pgfsetbuttcap%
\pgfsetroundjoin%
\definecolor{currentfill}{rgb}{0.214298,0.355619,0.551184}%
\pgfsetfillcolor{currentfill}%
\pgfsetfillopacity{0.800000}%
\pgfsetlinewidth{0.000000pt}%
\definecolor{currentstroke}{rgb}{0.000000,0.000000,0.000000}%
\pgfsetstrokecolor{currentstroke}%
\pgfsetdash{}{0pt}%
\pgfpathmoveto{\pgfqpoint{3.295866in}{3.131363in}}%
\pgfpathlineto{\pgfqpoint{3.309252in}{3.114750in}}%
\pgfpathlineto{\pgfqpoint{3.322635in}{3.098427in}}%
\pgfpathlineto{\pgfqpoint{3.336012in}{3.082391in}}%
\pgfpathlineto{\pgfqpoint{3.349386in}{3.066641in}}%
\pgfpathlineto{\pgfqpoint{3.357262in}{3.081693in}}%
\pgfpathlineto{\pgfqpoint{3.365133in}{3.096923in}}%
\pgfpathlineto{\pgfqpoint{3.372997in}{3.112335in}}%
\pgfpathlineto{\pgfqpoint{3.380856in}{3.127932in}}%
\pgfpathlineto{\pgfqpoint{3.367490in}{3.143969in}}%
\pgfpathlineto{\pgfqpoint{3.354120in}{3.160292in}}%
\pgfpathlineto{\pgfqpoint{3.340745in}{3.176902in}}%
\pgfpathlineto{\pgfqpoint{3.327366in}{3.193803in}}%
\pgfpathlineto{\pgfqpoint{3.319500in}{3.177907in}}%
\pgfpathlineto{\pgfqpoint{3.311628in}{3.162204in}}%
\pgfpathlineto{\pgfqpoint{3.303750in}{3.146691in}}%
\pgfpathlineto{\pgfqpoint{3.295866in}{3.131363in}}%
\pgfpathclose%
\pgfusepath{fill}%
\end{pgfscope}%
\begin{pgfscope}%
\pgfpathrectangle{\pgfqpoint{1.150000in}{0.150000in}}{\pgfqpoint{5.700000in}{5.700000in}}%
\pgfusepath{clip}%
\pgfsetbuttcap%
\pgfsetroundjoin%
\definecolor{currentfill}{rgb}{0.214298,0.355619,0.551184}%
\pgfsetfillcolor{currentfill}%
\pgfsetfillopacity{0.800000}%
\pgfsetlinewidth{0.000000pt}%
\definecolor{currentstroke}{rgb}{0.000000,0.000000,0.000000}%
\pgfsetstrokecolor{currentstroke}%
\pgfsetdash{}{0pt}%
\pgfpathmoveto{\pgfqpoint{4.651342in}{3.108507in}}%
\pgfpathlineto{\pgfqpoint{4.664831in}{3.104491in}}%
\pgfpathlineto{\pgfqpoint{4.678328in}{3.100664in}}%
\pgfpathlineto{\pgfqpoint{4.691833in}{3.097025in}}%
\pgfpathlineto{\pgfqpoint{4.705346in}{3.093576in}}%
\pgfpathlineto{\pgfqpoint{4.712935in}{3.109134in}}%
\pgfpathlineto{\pgfqpoint{4.720524in}{3.124953in}}%
\pgfpathlineto{\pgfqpoint{4.728113in}{3.141041in}}%
\pgfpathlineto{\pgfqpoint{4.735703in}{3.157405in}}%
\pgfpathlineto{\pgfqpoint{4.722201in}{3.161480in}}%
\pgfpathlineto{\pgfqpoint{4.708708in}{3.165744in}}%
\pgfpathlineto{\pgfqpoint{4.695222in}{3.170197in}}%
\pgfpathlineto{\pgfqpoint{4.681744in}{3.174839in}}%
\pgfpathlineto{\pgfqpoint{4.674143in}{3.157838in}}%
\pgfpathlineto{\pgfqpoint{4.666543in}{3.141120in}}%
\pgfpathlineto{\pgfqpoint{4.658942in}{3.124679in}}%
\pgfpathlineto{\pgfqpoint{4.651342in}{3.108507in}}%
\pgfpathclose%
\pgfusepath{fill}%
\end{pgfscope}%
\begin{pgfscope}%
\pgfpathrectangle{\pgfqpoint{1.150000in}{0.150000in}}{\pgfqpoint{5.700000in}{5.700000in}}%
\pgfusepath{clip}%
\pgfsetbuttcap%
\pgfsetroundjoin%
\definecolor{currentfill}{rgb}{0.235526,0.309527,0.542944}%
\pgfsetfillcolor{currentfill}%
\pgfsetfillopacity{0.800000}%
\pgfsetlinewidth{0.000000pt}%
\definecolor{currentstroke}{rgb}{0.000000,0.000000,0.000000}%
\pgfsetstrokecolor{currentstroke}%
\pgfsetdash{}{0pt}%
\pgfpathmoveto{\pgfqpoint{3.402843in}{3.006444in}}%
\pgfpathlineto{\pgfqpoint{3.416199in}{2.992085in}}%
\pgfpathlineto{\pgfqpoint{3.429553in}{2.977998in}}%
\pgfpathlineto{\pgfqpoint{3.442904in}{2.964181in}}%
\pgfpathlineto{\pgfqpoint{3.456252in}{2.950631in}}%
\pgfpathlineto{\pgfqpoint{3.464113in}{2.965136in}}%
\pgfpathlineto{\pgfqpoint{3.471968in}{2.979802in}}%
\pgfpathlineto{\pgfqpoint{3.479817in}{2.994635in}}%
\pgfpathlineto{\pgfqpoint{3.487662in}{3.009636in}}%
\pgfpathlineto{\pgfqpoint{3.474321in}{3.023471in}}%
\pgfpathlineto{\pgfqpoint{3.460977in}{3.037574in}}%
\pgfpathlineto{\pgfqpoint{3.447631in}{3.051946in}}%
\pgfpathlineto{\pgfqpoint{3.434283in}{3.066591in}}%
\pgfpathlineto{\pgfqpoint{3.426431in}{3.051292in}}%
\pgfpathlineto{\pgfqpoint{3.418574in}{3.036171in}}%
\pgfpathlineto{\pgfqpoint{3.410711in}{3.021223in}}%
\pgfpathlineto{\pgfqpoint{3.402843in}{3.006444in}}%
\pgfpathclose%
\pgfusepath{fill}%
\end{pgfscope}%
\begin{pgfscope}%
\pgfpathrectangle{\pgfqpoint{1.150000in}{0.150000in}}{\pgfqpoint{5.700000in}{5.700000in}}%
\pgfusepath{clip}%
\pgfsetbuttcap%
\pgfsetroundjoin%
\definecolor{currentfill}{rgb}{0.260571,0.246922,0.522828}%
\pgfsetfillcolor{currentfill}%
\pgfsetfillopacity{0.800000}%
\pgfsetlinewidth{0.000000pt}%
\definecolor{currentstroke}{rgb}{0.000000,0.000000,0.000000}%
\pgfsetstrokecolor{currentstroke}%
\pgfsetdash{}{0pt}%
\pgfpathmoveto{\pgfqpoint{3.785509in}{2.841759in}}%
\pgfpathlineto{\pgfqpoint{3.798842in}{2.832816in}}%
\pgfpathlineto{\pgfqpoint{3.812178in}{2.824105in}}%
\pgfpathlineto{\pgfqpoint{3.825516in}{2.815623in}}%
\pgfpathlineto{\pgfqpoint{3.838856in}{2.807369in}}%
\pgfpathlineto{\pgfqpoint{3.846634in}{2.821263in}}%
\pgfpathlineto{\pgfqpoint{3.854407in}{2.835303in}}%
\pgfpathlineto{\pgfqpoint{3.862176in}{2.849491in}}%
\pgfpathlineto{\pgfqpoint{3.869940in}{2.863832in}}%
\pgfpathlineto{\pgfqpoint{3.856607in}{2.872431in}}%
\pgfpathlineto{\pgfqpoint{3.843276in}{2.881259in}}%
\pgfpathlineto{\pgfqpoint{3.829947in}{2.890315in}}%
\pgfpathlineto{\pgfqpoint{3.816620in}{2.899603in}}%
\pgfpathlineto{\pgfqpoint{3.808849in}{2.884905in}}%
\pgfpathlineto{\pgfqpoint{3.801073in}{2.870367in}}%
\pgfpathlineto{\pgfqpoint{3.793293in}{2.855987in}}%
\pgfpathlineto{\pgfqpoint{3.785509in}{2.841759in}}%
\pgfpathclose%
\pgfusepath{fill}%
\end{pgfscope}%
\begin{pgfscope}%
\pgfpathrectangle{\pgfqpoint{1.150000in}{0.150000in}}{\pgfqpoint{5.700000in}{5.700000in}}%
\pgfusepath{clip}%
\pgfsetbuttcap%
\pgfsetroundjoin%
\definecolor{currentfill}{rgb}{0.244972,0.287675,0.537260}%
\pgfsetfillcolor{currentfill}%
\pgfsetfillopacity{0.800000}%
\pgfsetlinewidth{0.000000pt}%
\definecolor{currentstroke}{rgb}{0.000000,0.000000,0.000000}%
\pgfsetstrokecolor{currentstroke}%
\pgfsetdash{}{0pt}%
\pgfpathmoveto{\pgfqpoint{4.314120in}{2.936318in}}%
\pgfpathlineto{\pgfqpoint{4.327535in}{2.931367in}}%
\pgfpathlineto{\pgfqpoint{4.340956in}{2.926616in}}%
\pgfpathlineto{\pgfqpoint{4.354383in}{2.922066in}}%
\pgfpathlineto{\pgfqpoint{4.367818in}{2.917715in}}%
\pgfpathlineto{\pgfqpoint{4.375470in}{2.931961in}}%
\pgfpathlineto{\pgfqpoint{4.383121in}{2.946396in}}%
\pgfpathlineto{\pgfqpoint{4.390769in}{2.961025in}}%
\pgfpathlineto{\pgfqpoint{4.398414in}{2.975855in}}%
\pgfpathlineto{\pgfqpoint{4.384989in}{2.980706in}}%
\pgfpathlineto{\pgfqpoint{4.371570in}{2.985757in}}%
\pgfpathlineto{\pgfqpoint{4.358158in}{2.991008in}}%
\pgfpathlineto{\pgfqpoint{4.344751in}{2.996460in}}%
\pgfpathlineto{\pgfqpoint{4.337097in}{2.981118in}}%
\pgfpathlineto{\pgfqpoint{4.329440in}{2.965985in}}%
\pgfpathlineto{\pgfqpoint{4.321781in}{2.951053in}}%
\pgfpathlineto{\pgfqpoint{4.314120in}{2.936318in}}%
\pgfpathclose%
\pgfusepath{fill}%
\end{pgfscope}%
\begin{pgfscope}%
\pgfpathrectangle{\pgfqpoint{1.150000in}{0.150000in}}{\pgfqpoint{5.700000in}{5.700000in}}%
\pgfusepath{clip}%
\pgfsetbuttcap%
\pgfsetroundjoin%
\definecolor{currentfill}{rgb}{0.204903,0.375746,0.553533}%
\pgfsetfillcolor{currentfill}%
\pgfsetfillopacity{0.800000}%
\pgfsetlinewidth{0.000000pt}%
\definecolor{currentstroke}{rgb}{0.000000,0.000000,0.000000}%
\pgfsetstrokecolor{currentstroke}%
\pgfsetdash{}{0pt}%
\pgfpathmoveto{\pgfqpoint{4.735703in}{3.157405in}}%
\pgfpathlineto{\pgfqpoint{4.749212in}{3.153516in}}%
\pgfpathlineto{\pgfqpoint{4.762730in}{3.149815in}}%
\pgfpathlineto{\pgfqpoint{4.776256in}{3.146300in}}%
\pgfpathlineto{\pgfqpoint{4.789791in}{3.142970in}}%
\pgfpathlineto{\pgfqpoint{4.797369in}{3.158972in}}%
\pgfpathlineto{\pgfqpoint{4.804948in}{3.175257in}}%
\pgfpathlineto{\pgfqpoint{4.812527in}{3.191832in}}%
\pgfpathlineto{\pgfqpoint{4.820109in}{3.208706in}}%
\pgfpathlineto{\pgfqpoint{4.806587in}{3.212692in}}%
\pgfpathlineto{\pgfqpoint{4.793073in}{3.216865in}}%
\pgfpathlineto{\pgfqpoint{4.779567in}{3.221224in}}%
\pgfpathlineto{\pgfqpoint{4.766070in}{3.225769in}}%
\pgfpathlineto{\pgfqpoint{4.758476in}{3.208226in}}%
\pgfpathlineto{\pgfqpoint{4.750884in}{3.190990in}}%
\pgfpathlineto{\pgfqpoint{4.743293in}{3.174052in}}%
\pgfpathlineto{\pgfqpoint{4.735703in}{3.157405in}}%
\pgfpathclose%
\pgfusepath{fill}%
\end{pgfscope}%
\begin{pgfscope}%
\pgfpathrectangle{\pgfqpoint{1.150000in}{0.150000in}}{\pgfqpoint{5.700000in}{5.700000in}}%
\pgfusepath{clip}%
\pgfsetbuttcap%
\pgfsetroundjoin%
\definecolor{currentfill}{rgb}{0.257322,0.256130,0.526563}%
\pgfsetfillcolor{currentfill}%
\pgfsetfillopacity{0.800000}%
\pgfsetlinewidth{0.000000pt}%
\definecolor{currentstroke}{rgb}{0.000000,0.000000,0.000000}%
\pgfsetstrokecolor{currentstroke}%
\pgfsetdash{}{0pt}%
\pgfpathmoveto{\pgfqpoint{3.647653in}{2.863797in}}%
\pgfpathlineto{\pgfqpoint{3.660983in}{2.853271in}}%
\pgfpathlineto{\pgfqpoint{3.674315in}{2.842987in}}%
\pgfpathlineto{\pgfqpoint{3.687646in}{2.832945in}}%
\pgfpathlineto{\pgfqpoint{3.700979in}{2.823141in}}%
\pgfpathlineto{\pgfqpoint{3.708790in}{2.837101in}}%
\pgfpathlineto{\pgfqpoint{3.716595in}{2.851206in}}%
\pgfpathlineto{\pgfqpoint{3.724396in}{2.865459in}}%
\pgfpathlineto{\pgfqpoint{3.732192in}{2.879864in}}%
\pgfpathlineto{\pgfqpoint{3.718866in}{2.889982in}}%
\pgfpathlineto{\pgfqpoint{3.705541in}{2.900339in}}%
\pgfpathlineto{\pgfqpoint{3.692217in}{2.910937in}}%
\pgfpathlineto{\pgfqpoint{3.678894in}{2.921778in}}%
\pgfpathlineto{\pgfqpoint{3.671091in}{2.907046in}}%
\pgfpathlineto{\pgfqpoint{3.663283in}{2.892475in}}%
\pgfpathlineto{\pgfqpoint{3.655470in}{2.878060in}}%
\pgfpathlineto{\pgfqpoint{3.647653in}{2.863797in}}%
\pgfpathclose%
\pgfusepath{fill}%
\end{pgfscope}%
\begin{pgfscope}%
\pgfpathrectangle{\pgfqpoint{1.150000in}{0.150000in}}{\pgfqpoint{5.700000in}{5.700000in}}%
\pgfusepath{clip}%
\pgfsetbuttcap%
\pgfsetroundjoin%
\definecolor{currentfill}{rgb}{0.203063,0.379716,0.553925}%
\pgfsetfillcolor{currentfill}%
\pgfsetfillopacity{0.800000}%
\pgfsetlinewidth{0.000000pt}%
\definecolor{currentstroke}{rgb}{0.000000,0.000000,0.000000}%
\pgfsetstrokecolor{currentstroke}%
\pgfsetdash{}{0pt}%
\pgfpathmoveto{\pgfqpoint{3.242267in}{3.200766in}}%
\pgfpathlineto{\pgfqpoint{3.255675in}{3.182968in}}%
\pgfpathlineto{\pgfqpoint{3.269077in}{3.165469in}}%
\pgfpathlineto{\pgfqpoint{3.282474in}{3.148269in}}%
\pgfpathlineto{\pgfqpoint{3.295866in}{3.131363in}}%
\pgfpathlineto{\pgfqpoint{3.303750in}{3.146691in}}%
\pgfpathlineto{\pgfqpoint{3.311628in}{3.162204in}}%
\pgfpathlineto{\pgfqpoint{3.319500in}{3.177907in}}%
\pgfpathlineto{\pgfqpoint{3.327366in}{3.193803in}}%
\pgfpathlineto{\pgfqpoint{3.313982in}{3.210997in}}%
\pgfpathlineto{\pgfqpoint{3.300593in}{3.228486in}}%
\pgfpathlineto{\pgfqpoint{3.287198in}{3.246272in}}%
\pgfpathlineto{\pgfqpoint{3.273798in}{3.264360in}}%
\pgfpathlineto{\pgfqpoint{3.265925in}{3.248163in}}%
\pgfpathlineto{\pgfqpoint{3.258045in}{3.232167in}}%
\pgfpathlineto{\pgfqpoint{3.250159in}{3.216370in}}%
\pgfpathlineto{\pgfqpoint{3.242267in}{3.200766in}}%
\pgfpathclose%
\pgfusepath{fill}%
\end{pgfscope}%
\begin{pgfscope}%
\pgfpathrectangle{\pgfqpoint{1.150000in}{0.150000in}}{\pgfqpoint{5.700000in}{5.700000in}}%
\pgfusepath{clip}%
\pgfsetbuttcap%
\pgfsetroundjoin%
\definecolor{currentfill}{rgb}{0.136408,0.541173,0.554483}%
\pgfsetfillcolor{currentfill}%
\pgfsetfillopacity{0.800000}%
\pgfsetlinewidth{0.000000pt}%
\definecolor{currentstroke}{rgb}{0.000000,0.000000,0.000000}%
\pgfsetstrokecolor{currentstroke}%
\pgfsetdash{}{0pt}%
\pgfpathmoveto{\pgfqpoint{3.089945in}{3.669056in}}%
\pgfpathlineto{\pgfqpoint{3.103467in}{3.645431in}}%
\pgfpathlineto{\pgfqpoint{3.116978in}{3.622156in}}%
\pgfpathlineto{\pgfqpoint{3.130479in}{3.599228in}}%
\pgfpathlineto{\pgfqpoint{3.143971in}{3.576644in}}%
\pgfpathlineto{\pgfqpoint{3.151833in}{3.594941in}}%
\pgfpathlineto{\pgfqpoint{3.159688in}{3.613488in}}%
\pgfpathlineto{\pgfqpoint{3.167536in}{3.632290in}}%
\pgfpathlineto{\pgfqpoint{3.175377in}{3.651350in}}%
\pgfpathlineto{\pgfqpoint{3.161891in}{3.674297in}}%
\pgfpathlineto{\pgfqpoint{3.148396in}{3.697588in}}%
\pgfpathlineto{\pgfqpoint{3.134891in}{3.721226in}}%
\pgfpathlineto{\pgfqpoint{3.121376in}{3.745216in}}%
\pgfpathlineto{\pgfqpoint{3.113529in}{3.725779in}}%
\pgfpathlineto{\pgfqpoint{3.105675in}{3.706610in}}%
\pgfpathlineto{\pgfqpoint{3.097814in}{3.687704in}}%
\pgfpathlineto{\pgfqpoint{3.089945in}{3.669056in}}%
\pgfpathclose%
\pgfusepath{fill}%
\end{pgfscope}%
\begin{pgfscope}%
\pgfpathrectangle{\pgfqpoint{1.150000in}{0.150000in}}{\pgfqpoint{5.700000in}{5.700000in}}%
\pgfusepath{clip}%
\pgfsetbuttcap%
\pgfsetroundjoin%
\definecolor{currentfill}{rgb}{0.244972,0.287675,0.537260}%
\pgfsetfillcolor{currentfill}%
\pgfsetfillopacity{0.800000}%
\pgfsetlinewidth{0.000000pt}%
\definecolor{currentstroke}{rgb}{0.000000,0.000000,0.000000}%
\pgfsetstrokecolor{currentstroke}%
\pgfsetdash{}{0pt}%
\pgfpathmoveto{\pgfqpoint{3.456252in}{2.950631in}}%
\pgfpathlineto{\pgfqpoint{3.469599in}{2.937346in}}%
\pgfpathlineto{\pgfqpoint{3.482943in}{2.924325in}}%
\pgfpathlineto{\pgfqpoint{3.496286in}{2.911566in}}%
\pgfpathlineto{\pgfqpoint{3.509628in}{2.899066in}}%
\pgfpathlineto{\pgfqpoint{3.517481in}{2.913298in}}%
\pgfpathlineto{\pgfqpoint{3.525328in}{2.927685in}}%
\pgfpathlineto{\pgfqpoint{3.533170in}{2.942228in}}%
\pgfpathlineto{\pgfqpoint{3.541007in}{2.956933in}}%
\pgfpathlineto{\pgfqpoint{3.527673in}{2.969717in}}%
\pgfpathlineto{\pgfqpoint{3.514338in}{2.982761in}}%
\pgfpathlineto{\pgfqpoint{3.501001in}{2.996067in}}%
\pgfpathlineto{\pgfqpoint{3.487662in}{3.009636in}}%
\pgfpathlineto{\pgfqpoint{3.479817in}{2.994635in}}%
\pgfpathlineto{\pgfqpoint{3.471968in}{2.979802in}}%
\pgfpathlineto{\pgfqpoint{3.464113in}{2.965136in}}%
\pgfpathlineto{\pgfqpoint{3.456252in}{2.950631in}}%
\pgfpathclose%
\pgfusepath{fill}%
\end{pgfscope}%
\begin{pgfscope}%
\pgfpathrectangle{\pgfqpoint{1.150000in}{0.150000in}}{\pgfqpoint{5.700000in}{5.700000in}}%
\pgfusepath{clip}%
\pgfsetbuttcap%
\pgfsetroundjoin%
\definecolor{currentfill}{rgb}{0.250425,0.274290,0.533103}%
\pgfsetfillcolor{currentfill}%
\pgfsetfillopacity{0.800000}%
\pgfsetlinewidth{0.000000pt}%
\definecolor{currentstroke}{rgb}{0.000000,0.000000,0.000000}%
\pgfsetstrokecolor{currentstroke}%
\pgfsetdash{}{0pt}%
\pgfpathmoveto{\pgfqpoint{4.229814in}{2.899186in}}%
\pgfpathlineto{\pgfqpoint{4.243214in}{2.893891in}}%
\pgfpathlineto{\pgfqpoint{4.256619in}{2.888801in}}%
\pgfpathlineto{\pgfqpoint{4.270030in}{2.883914in}}%
\pgfpathlineto{\pgfqpoint{4.283447in}{2.879230in}}%
\pgfpathlineto{\pgfqpoint{4.291119in}{2.893236in}}%
\pgfpathlineto{\pgfqpoint{4.298789in}{2.907415in}}%
\pgfpathlineto{\pgfqpoint{4.306456in}{2.921774in}}%
\pgfpathlineto{\pgfqpoint{4.314120in}{2.936318in}}%
\pgfpathlineto{\pgfqpoint{4.300711in}{2.941472in}}%
\pgfpathlineto{\pgfqpoint{4.287308in}{2.946828in}}%
\pgfpathlineto{\pgfqpoint{4.273911in}{2.952387in}}%
\pgfpathlineto{\pgfqpoint{4.260520in}{2.958151in}}%
\pgfpathlineto{\pgfqpoint{4.252848in}{2.943126in}}%
\pgfpathlineto{\pgfqpoint{4.245173in}{2.928294in}}%
\pgfpathlineto{\pgfqpoint{4.237495in}{2.913649in}}%
\pgfpathlineto{\pgfqpoint{4.229814in}{2.899186in}}%
\pgfpathclose%
\pgfusepath{fill}%
\end{pgfscope}%
\begin{pgfscope}%
\pgfpathrectangle{\pgfqpoint{1.150000in}{0.150000in}}{\pgfqpoint{5.700000in}{5.700000in}}%
\pgfusepath{clip}%
\pgfsetbuttcap%
\pgfsetroundjoin%
\definecolor{currentfill}{rgb}{0.262138,0.242286,0.520837}%
\pgfsetfillcolor{currentfill}%
\pgfsetfillopacity{0.800000}%
\pgfsetlinewidth{0.000000pt}%
\definecolor{currentstroke}{rgb}{0.000000,0.000000,0.000000}%
\pgfsetstrokecolor{currentstroke}%
\pgfsetdash{}{0pt}%
\pgfpathmoveto{\pgfqpoint{3.923301in}{2.831693in}}%
\pgfpathlineto{\pgfqpoint{3.936650in}{2.824216in}}%
\pgfpathlineto{\pgfqpoint{3.950001in}{2.816959in}}%
\pgfpathlineto{\pgfqpoint{3.963356in}{2.809922in}}%
\pgfpathlineto{\pgfqpoint{3.976715in}{2.803104in}}%
\pgfpathlineto{\pgfqpoint{3.984462in}{2.816874in}}%
\pgfpathlineto{\pgfqpoint{3.992204in}{2.830790in}}%
\pgfpathlineto{\pgfqpoint{3.999942in}{2.844857in}}%
\pgfpathlineto{\pgfqpoint{4.007676in}{2.859079in}}%
\pgfpathlineto{\pgfqpoint{3.994324in}{2.866273in}}%
\pgfpathlineto{\pgfqpoint{3.980976in}{2.873686in}}%
\pgfpathlineto{\pgfqpoint{3.967632in}{2.881318in}}%
\pgfpathlineto{\pgfqpoint{3.954290in}{2.889171in}}%
\pgfpathlineto{\pgfqpoint{3.946549in}{2.874562in}}%
\pgfpathlineto{\pgfqpoint{3.938804in}{2.860115in}}%
\pgfpathlineto{\pgfqpoint{3.931055in}{2.845827in}}%
\pgfpathlineto{\pgfqpoint{3.923301in}{2.831693in}}%
\pgfpathclose%
\pgfusepath{fill}%
\end{pgfscope}%
\begin{pgfscope}%
\pgfpathrectangle{\pgfqpoint{1.150000in}{0.150000in}}{\pgfqpoint{5.700000in}{5.700000in}}%
\pgfusepath{clip}%
\pgfsetbuttcap%
\pgfsetroundjoin%
\definecolor{currentfill}{rgb}{0.195860,0.395433,0.555276}%
\pgfsetfillcolor{currentfill}%
\pgfsetfillopacity{0.800000}%
\pgfsetlinewidth{0.000000pt}%
\definecolor{currentstroke}{rgb}{0.000000,0.000000,0.000000}%
\pgfsetstrokecolor{currentstroke}%
\pgfsetdash{}{0pt}%
\pgfpathmoveto{\pgfqpoint{4.820109in}{3.208706in}}%
\pgfpathlineto{\pgfqpoint{4.833639in}{3.204904in}}%
\pgfpathlineto{\pgfqpoint{4.847178in}{3.201287in}}%
\pgfpathlineto{\pgfqpoint{4.860726in}{3.197854in}}%
\pgfpathlineto{\pgfqpoint{4.874282in}{3.194605in}}%
\pgfpathlineto{\pgfqpoint{4.881852in}{3.211108in}}%
\pgfpathlineto{\pgfqpoint{4.889423in}{3.227917in}}%
\pgfpathlineto{\pgfqpoint{4.896996in}{3.245040in}}%
\pgfpathlineto{\pgfqpoint{4.904572in}{3.262486in}}%
\pgfpathlineto{\pgfqpoint{4.891029in}{3.266424in}}%
\pgfpathlineto{\pgfqpoint{4.877495in}{3.270546in}}%
\pgfpathlineto{\pgfqpoint{4.863969in}{3.274852in}}%
\pgfpathlineto{\pgfqpoint{4.850452in}{3.279343in}}%
\pgfpathlineto{\pgfqpoint{4.842863in}{3.261196in}}%
\pgfpathlineto{\pgfqpoint{4.835276in}{3.243380in}}%
\pgfpathlineto{\pgfqpoint{4.827691in}{3.225886in}}%
\pgfpathlineto{\pgfqpoint{4.820109in}{3.208706in}}%
\pgfpathclose%
\pgfusepath{fill}%
\end{pgfscope}%
\begin{pgfscope}%
\pgfpathrectangle{\pgfqpoint{1.150000in}{0.150000in}}{\pgfqpoint{5.700000in}{5.700000in}}%
\pgfusepath{clip}%
\pgfsetbuttcap%
\pgfsetroundjoin%
\definecolor{currentfill}{rgb}{0.157729,0.485932,0.558013}%
\pgfsetfillcolor{currentfill}%
\pgfsetfillopacity{0.800000}%
\pgfsetlinewidth{0.000000pt}%
\definecolor{currentstroke}{rgb}{0.000000,0.000000,0.000000}%
\pgfsetstrokecolor{currentstroke}%
\pgfsetdash{}{0pt}%
\pgfpathmoveto{\pgfqpoint{3.112455in}{3.505872in}}%
\pgfpathlineto{\pgfqpoint{3.125945in}{3.483955in}}%
\pgfpathlineto{\pgfqpoint{3.139425in}{3.462374in}}%
\pgfpathlineto{\pgfqpoint{3.152897in}{3.441126in}}%
\pgfpathlineto{\pgfqpoint{3.166360in}{3.420208in}}%
\pgfpathlineto{\pgfqpoint{3.174242in}{3.437232in}}%
\pgfpathlineto{\pgfqpoint{3.182117in}{3.454482in}}%
\pgfpathlineto{\pgfqpoint{3.189986in}{3.471960in}}%
\pgfpathlineto{\pgfqpoint{3.197848in}{3.489673in}}%
\pgfpathlineto{\pgfqpoint{3.184391in}{3.510917in}}%
\pgfpathlineto{\pgfqpoint{3.170927in}{3.532491in}}%
\pgfpathlineto{\pgfqpoint{3.157453in}{3.554399in}}%
\pgfpathlineto{\pgfqpoint{3.143971in}{3.576644in}}%
\pgfpathlineto{\pgfqpoint{3.136102in}{3.558592in}}%
\pgfpathlineto{\pgfqpoint{3.128227in}{3.540782in}}%
\pgfpathlineto{\pgfqpoint{3.120344in}{3.523210in}}%
\pgfpathlineto{\pgfqpoint{3.112455in}{3.505872in}}%
\pgfpathclose%
\pgfusepath{fill}%
\end{pgfscope}%
\begin{pgfscope}%
\pgfpathrectangle{\pgfqpoint{1.150000in}{0.150000in}}{\pgfqpoint{5.700000in}{5.700000in}}%
\pgfusepath{clip}%
\pgfsetbuttcap%
\pgfsetroundjoin%
\definecolor{currentfill}{rgb}{0.876168,0.891125,0.095250}%
\pgfsetfillcolor{currentfill}%
\pgfsetfillopacity{0.800000}%
\pgfsetlinewidth{0.000000pt}%
\definecolor{currentstroke}{rgb}{0.000000,0.000000,0.000000}%
\pgfsetstrokecolor{currentstroke}%
\pgfsetdash{}{0pt}%
\pgfpathmoveto{\pgfqpoint{3.545098in}{4.982386in}}%
\pgfpathlineto{\pgfqpoint{3.558603in}{4.952570in}}%
\pgfpathlineto{\pgfqpoint{3.572098in}{4.923110in}}%
\pgfpathlineto{\pgfqpoint{3.585583in}{4.894001in}}%
\pgfpathlineto{\pgfqpoint{3.599058in}{4.865240in}}%
\pgfpathlineto{\pgfqpoint{3.606669in}{4.903712in}}%
\pgfpathlineto{\pgfqpoint{3.614278in}{4.942790in}}%
\pgfpathlineto{\pgfqpoint{3.621886in}{4.982488in}}%
\pgfpathlineto{\pgfqpoint{3.608403in}{5.011889in}}%
\pgfpathlineto{\pgfqpoint{3.594909in}{5.041641in}}%
\pgfpathlineto{\pgfqpoint{3.581406in}{5.071745in}}%
\pgfpathlineto{\pgfqpoint{3.567892in}{5.102207in}}%
\pgfpathlineto{\pgfqpoint{3.560296in}{5.061641in}}%
\pgfpathlineto{\pgfqpoint{3.552698in}{5.021704in}}%
\pgfpathlineto{\pgfqpoint{3.545098in}{4.982386in}}%
\pgfpathclose%
\pgfusepath{fill}%
\end{pgfscope}%
\begin{pgfscope}%
\pgfpathrectangle{\pgfqpoint{1.150000in}{0.150000in}}{\pgfqpoint{5.700000in}{5.700000in}}%
\pgfusepath{clip}%
\pgfsetbuttcap%
\pgfsetroundjoin%
\definecolor{currentfill}{rgb}{0.190631,0.407061,0.556089}%
\pgfsetfillcolor{currentfill}%
\pgfsetfillopacity{0.800000}%
\pgfsetlinewidth{0.000000pt}%
\definecolor{currentstroke}{rgb}{0.000000,0.000000,0.000000}%
\pgfsetstrokecolor{currentstroke}%
\pgfsetdash{}{0pt}%
\pgfpathmoveto{\pgfqpoint{3.188572in}{3.275015in}}%
\pgfpathlineto{\pgfqpoint{3.202006in}{3.255989in}}%
\pgfpathlineto{\pgfqpoint{3.215432in}{3.237274in}}%
\pgfpathlineto{\pgfqpoint{3.228853in}{3.218867in}}%
\pgfpathlineto{\pgfqpoint{3.242267in}{3.200766in}}%
\pgfpathlineto{\pgfqpoint{3.250159in}{3.216370in}}%
\pgfpathlineto{\pgfqpoint{3.258045in}{3.232167in}}%
\pgfpathlineto{\pgfqpoint{3.265925in}{3.248163in}}%
\pgfpathlineto{\pgfqpoint{3.273798in}{3.264360in}}%
\pgfpathlineto{\pgfqpoint{3.260392in}{3.282750in}}%
\pgfpathlineto{\pgfqpoint{3.246980in}{3.301446in}}%
\pgfpathlineto{\pgfqpoint{3.233561in}{3.320452in}}%
\pgfpathlineto{\pgfqpoint{3.220135in}{3.339768in}}%
\pgfpathlineto{\pgfqpoint{3.212254in}{3.323269in}}%
\pgfpathlineto{\pgfqpoint{3.204367in}{3.306980in}}%
\pgfpathlineto{\pgfqpoint{3.196473in}{3.290896in}}%
\pgfpathlineto{\pgfqpoint{3.188572in}{3.275015in}}%
\pgfpathclose%
\pgfusepath{fill}%
\end{pgfscope}%
\begin{pgfscope}%
\pgfpathrectangle{\pgfqpoint{1.150000in}{0.150000in}}{\pgfqpoint{5.700000in}{5.700000in}}%
\pgfusepath{clip}%
\pgfsetbuttcap%
\pgfsetroundjoin%
\definecolor{currentfill}{rgb}{0.255645,0.260703,0.528312}%
\pgfsetfillcolor{currentfill}%
\pgfsetfillopacity{0.800000}%
\pgfsetlinewidth{0.000000pt}%
\definecolor{currentstroke}{rgb}{0.000000,0.000000,0.000000}%
\pgfsetstrokecolor{currentstroke}%
\pgfsetdash{}{0pt}%
\pgfpathmoveto{\pgfqpoint{4.145486in}{2.864534in}}%
\pgfpathlineto{\pgfqpoint{4.158872in}{2.858850in}}%
\pgfpathlineto{\pgfqpoint{4.172262in}{2.853375in}}%
\pgfpathlineto{\pgfqpoint{4.185658in}{2.848107in}}%
\pgfpathlineto{\pgfqpoint{4.199060in}{2.843045in}}%
\pgfpathlineto{\pgfqpoint{4.206753in}{2.856834in}}%
\pgfpathlineto{\pgfqpoint{4.214444in}{2.870784in}}%
\pgfpathlineto{\pgfqpoint{4.222130in}{2.884899in}}%
\pgfpathlineto{\pgfqpoint{4.229814in}{2.899186in}}%
\pgfpathlineto{\pgfqpoint{4.216421in}{2.904686in}}%
\pgfpathlineto{\pgfqpoint{4.203033in}{2.910392in}}%
\pgfpathlineto{\pgfqpoint{4.189650in}{2.916306in}}%
\pgfpathlineto{\pgfqpoint{4.176272in}{2.922427in}}%
\pgfpathlineto{\pgfqpoint{4.168581in}{2.907691in}}%
\pgfpathlineto{\pgfqpoint{4.160886in}{2.893133in}}%
\pgfpathlineto{\pgfqpoint{4.153188in}{2.878749in}}%
\pgfpathlineto{\pgfqpoint{4.145486in}{2.864534in}}%
\pgfpathclose%
\pgfusepath{fill}%
\end{pgfscope}%
\begin{pgfscope}%
\pgfpathrectangle{\pgfqpoint{1.150000in}{0.150000in}}{\pgfqpoint{5.700000in}{5.700000in}}%
\pgfusepath{clip}%
\pgfsetbuttcap%
\pgfsetroundjoin%
\definecolor{currentfill}{rgb}{0.252194,0.269783,0.531579}%
\pgfsetfillcolor{currentfill}%
\pgfsetfillopacity{0.800000}%
\pgfsetlinewidth{0.000000pt}%
\definecolor{currentstroke}{rgb}{0.000000,0.000000,0.000000}%
\pgfsetstrokecolor{currentstroke}%
\pgfsetdash{}{0pt}%
\pgfpathmoveto{\pgfqpoint{3.509628in}{2.899066in}}%
\pgfpathlineto{\pgfqpoint{3.522968in}{2.886824in}}%
\pgfpathlineto{\pgfqpoint{3.536307in}{2.874838in}}%
\pgfpathlineto{\pgfqpoint{3.549646in}{2.863107in}}%
\pgfpathlineto{\pgfqpoint{3.562984in}{2.851627in}}%
\pgfpathlineto{\pgfqpoint{3.570829in}{2.865588in}}%
\pgfpathlineto{\pgfqpoint{3.578669in}{2.879694in}}%
\pgfpathlineto{\pgfqpoint{3.586504in}{2.893950in}}%
\pgfpathlineto{\pgfqpoint{3.594333in}{2.908359in}}%
\pgfpathlineto{\pgfqpoint{3.581003in}{2.920122in}}%
\pgfpathlineto{\pgfqpoint{3.567672in}{2.932137in}}%
\pgfpathlineto{\pgfqpoint{3.554340in}{2.944407in}}%
\pgfpathlineto{\pgfqpoint{3.541007in}{2.956933in}}%
\pgfpathlineto{\pgfqpoint{3.533170in}{2.942228in}}%
\pgfpathlineto{\pgfqpoint{3.525328in}{2.927685in}}%
\pgfpathlineto{\pgfqpoint{3.517481in}{2.913298in}}%
\pgfpathlineto{\pgfqpoint{3.509628in}{2.899066in}}%
\pgfpathclose%
\pgfusepath{fill}%
\end{pgfscope}%
\begin{pgfscope}%
\pgfpathrectangle{\pgfqpoint{1.150000in}{0.150000in}}{\pgfqpoint{5.700000in}{5.700000in}}%
\pgfusepath{clip}%
\pgfsetbuttcap%
\pgfsetroundjoin%
\definecolor{currentfill}{rgb}{0.185556,0.418570,0.556753}%
\pgfsetfillcolor{currentfill}%
\pgfsetfillopacity{0.800000}%
\pgfsetlinewidth{0.000000pt}%
\definecolor{currentstroke}{rgb}{0.000000,0.000000,0.000000}%
\pgfsetstrokecolor{currentstroke}%
\pgfsetdash{}{0pt}%
\pgfpathmoveto{\pgfqpoint{4.904572in}{3.262486in}}%
\pgfpathlineto{\pgfqpoint{4.918123in}{3.258731in}}%
\pgfpathlineto{\pgfqpoint{4.931684in}{3.255158in}}%
\pgfpathlineto{\pgfqpoint{4.945253in}{3.251767in}}%
\pgfpathlineto{\pgfqpoint{4.958831in}{3.248558in}}%
\pgfpathlineto{\pgfqpoint{4.966395in}{3.265625in}}%
\pgfpathlineto{\pgfqpoint{4.973962in}{3.283024in}}%
\pgfpathlineto{\pgfqpoint{4.981532in}{3.300762in}}%
\pgfpathlineto{\pgfqpoint{4.989106in}{3.318848in}}%
\pgfpathlineto{\pgfqpoint{4.975542in}{3.322778in}}%
\pgfpathlineto{\pgfqpoint{4.961987in}{3.326889in}}%
\pgfpathlineto{\pgfqpoint{4.948441in}{3.331183in}}%
\pgfpathlineto{\pgfqpoint{4.934904in}{3.335659in}}%
\pgfpathlineto{\pgfqpoint{4.927316in}{3.316841in}}%
\pgfpathlineto{\pgfqpoint{4.919731in}{3.298378in}}%
\pgfpathlineto{\pgfqpoint{4.912150in}{3.280263in}}%
\pgfpathlineto{\pgfqpoint{4.904572in}{3.262486in}}%
\pgfpathclose%
\pgfusepath{fill}%
\end{pgfscope}%
\begin{pgfscope}%
\pgfpathrectangle{\pgfqpoint{1.150000in}{0.150000in}}{\pgfqpoint{5.700000in}{5.700000in}}%
\pgfusepath{clip}%
\pgfsetbuttcap%
\pgfsetroundjoin%
\definecolor{currentfill}{rgb}{0.262138,0.242286,0.520837}%
\pgfsetfillcolor{currentfill}%
\pgfsetfillopacity{0.800000}%
\pgfsetlinewidth{0.000000pt}%
\definecolor{currentstroke}{rgb}{0.000000,0.000000,0.000000}%
\pgfsetstrokecolor{currentstroke}%
\pgfsetdash{}{0pt}%
\pgfpathmoveto{\pgfqpoint{3.700979in}{2.823141in}}%
\pgfpathlineto{\pgfqpoint{3.714313in}{2.813576in}}%
\pgfpathlineto{\pgfqpoint{3.727649in}{2.804247in}}%
\pgfpathlineto{\pgfqpoint{3.740986in}{2.795152in}}%
\pgfpathlineto{\pgfqpoint{3.754325in}{2.786292in}}%
\pgfpathlineto{\pgfqpoint{3.762128in}{2.799949in}}%
\pgfpathlineto{\pgfqpoint{3.769926in}{2.813744in}}%
\pgfpathlineto{\pgfqpoint{3.777720in}{2.827679in}}%
\pgfpathlineto{\pgfqpoint{3.785509in}{2.841759in}}%
\pgfpathlineto{\pgfqpoint{3.772177in}{2.850933in}}%
\pgfpathlineto{\pgfqpoint{3.758847in}{2.860341in}}%
\pgfpathlineto{\pgfqpoint{3.745519in}{2.869985in}}%
\pgfpathlineto{\pgfqpoint{3.732192in}{2.879864in}}%
\pgfpathlineto{\pgfqpoint{3.724396in}{2.865459in}}%
\pgfpathlineto{\pgfqpoint{3.716595in}{2.851206in}}%
\pgfpathlineto{\pgfqpoint{3.708790in}{2.837101in}}%
\pgfpathlineto{\pgfqpoint{3.700979in}{2.823141in}}%
\pgfpathclose%
\pgfusepath{fill}%
\end{pgfscope}%
\begin{pgfscope}%
\pgfpathrectangle{\pgfqpoint{1.150000in}{0.150000in}}{\pgfqpoint{5.700000in}{5.700000in}}%
\pgfusepath{clip}%
\pgfsetbuttcap%
\pgfsetroundjoin%
\definecolor{currentfill}{rgb}{0.866013,0.889868,0.095953}%
\pgfsetfillcolor{currentfill}%
\pgfsetfillopacity{0.800000}%
\pgfsetlinewidth{0.000000pt}%
\definecolor{currentstroke}{rgb}{0.000000,0.000000,0.000000}%
\pgfsetstrokecolor{currentstroke}%
\pgfsetdash{}{0pt}%
\pgfpathmoveto{\pgfqpoint{3.460592in}{4.950672in}}%
\pgfpathlineto{\pgfqpoint{3.474129in}{4.920231in}}%
\pgfpathlineto{\pgfqpoint{3.487655in}{4.890157in}}%
\pgfpathlineto{\pgfqpoint{3.501170in}{4.860447in}}%
\pgfpathlineto{\pgfqpoint{3.514674in}{4.831098in}}%
\pgfpathlineto{\pgfqpoint{3.522284in}{4.868042in}}%
\pgfpathlineto{\pgfqpoint{3.529892in}{4.905565in}}%
\pgfpathlineto{\pgfqpoint{3.537496in}{4.943676in}}%
\pgfpathlineto{\pgfqpoint{3.545098in}{4.982386in}}%
\pgfpathlineto{\pgfqpoint{3.531583in}{5.012560in}}%
\pgfpathlineto{\pgfqpoint{3.518057in}{5.043097in}}%
\pgfpathlineto{\pgfqpoint{3.504521in}{5.074000in}}%
\pgfpathlineto{\pgfqpoint{3.490972in}{5.105273in}}%
\pgfpathlineto{\pgfqpoint{3.483382in}{5.065718in}}%
\pgfpathlineto{\pgfqpoint{3.475788in}{5.026773in}}%
\pgfpathlineto{\pgfqpoint{3.468192in}{4.988428in}}%
\pgfpathlineto{\pgfqpoint{3.460592in}{4.950672in}}%
\pgfpathclose%
\pgfusepath{fill}%
\end{pgfscope}%
\begin{pgfscope}%
\pgfpathrectangle{\pgfqpoint{1.150000in}{0.150000in}}{\pgfqpoint{5.700000in}{5.700000in}}%
\pgfusepath{clip}%
\pgfsetbuttcap%
\pgfsetroundjoin%
\definecolor{currentfill}{rgb}{0.263663,0.237631,0.518762}%
\pgfsetfillcolor{currentfill}%
\pgfsetfillopacity{0.800000}%
\pgfsetlinewidth{0.000000pt}%
\definecolor{currentstroke}{rgb}{0.000000,0.000000,0.000000}%
\pgfsetstrokecolor{currentstroke}%
\pgfsetdash{}{0pt}%
\pgfpathmoveto{\pgfqpoint{3.838856in}{2.807369in}}%
\pgfpathlineto{\pgfqpoint{3.852199in}{2.799341in}}%
\pgfpathlineto{\pgfqpoint{3.865545in}{2.791540in}}%
\pgfpathlineto{\pgfqpoint{3.878894in}{2.783963in}}%
\pgfpathlineto{\pgfqpoint{3.892246in}{2.776608in}}%
\pgfpathlineto{\pgfqpoint{3.900016in}{2.790170in}}%
\pgfpathlineto{\pgfqpoint{3.907782in}{2.803868in}}%
\pgfpathlineto{\pgfqpoint{3.915544in}{2.817708in}}%
\pgfpathlineto{\pgfqpoint{3.923301in}{2.831693in}}%
\pgfpathlineto{\pgfqpoint{3.909957in}{2.839392in}}%
\pgfpathlineto{\pgfqpoint{3.896615in}{2.847314in}}%
\pgfpathlineto{\pgfqpoint{3.883276in}{2.855460in}}%
\pgfpathlineto{\pgfqpoint{3.869940in}{2.863832in}}%
\pgfpathlineto{\pgfqpoint{3.862176in}{2.849491in}}%
\pgfpathlineto{\pgfqpoint{3.854407in}{2.835303in}}%
\pgfpathlineto{\pgfqpoint{3.846634in}{2.821263in}}%
\pgfpathlineto{\pgfqpoint{3.838856in}{2.807369in}}%
\pgfpathclose%
\pgfusepath{fill}%
\end{pgfscope}%
\begin{pgfscope}%
\pgfpathrectangle{\pgfqpoint{1.150000in}{0.150000in}}{\pgfqpoint{5.700000in}{5.700000in}}%
\pgfusepath{clip}%
\pgfsetbuttcap%
\pgfsetroundjoin%
\definecolor{currentfill}{rgb}{0.177423,0.437527,0.557565}%
\pgfsetfillcolor{currentfill}%
\pgfsetfillopacity{0.800000}%
\pgfsetlinewidth{0.000000pt}%
\definecolor{currentstroke}{rgb}{0.000000,0.000000,0.000000}%
\pgfsetstrokecolor{currentstroke}%
\pgfsetdash{}{0pt}%
\pgfpathmoveto{\pgfqpoint{3.134765in}{3.354289in}}%
\pgfpathlineto{\pgfqpoint{3.148228in}{3.333989in}}%
\pgfpathlineto{\pgfqpoint{3.161684in}{3.314012in}}%
\pgfpathlineto{\pgfqpoint{3.175131in}{3.294355in}}%
\pgfpathlineto{\pgfqpoint{3.188572in}{3.275015in}}%
\pgfpathlineto{\pgfqpoint{3.196473in}{3.290896in}}%
\pgfpathlineto{\pgfqpoint{3.204367in}{3.306980in}}%
\pgfpathlineto{\pgfqpoint{3.212254in}{3.323269in}}%
\pgfpathlineto{\pgfqpoint{3.220135in}{3.339768in}}%
\pgfpathlineto{\pgfqpoint{3.206703in}{3.359399in}}%
\pgfpathlineto{\pgfqpoint{3.193263in}{3.379347in}}%
\pgfpathlineto{\pgfqpoint{3.179815in}{3.399616in}}%
\pgfpathlineto{\pgfqpoint{3.166360in}{3.420208in}}%
\pgfpathlineto{\pgfqpoint{3.158472in}{3.403405in}}%
\pgfpathlineto{\pgfqpoint{3.150576in}{3.386820in}}%
\pgfpathlineto{\pgfqpoint{3.142674in}{3.370450in}}%
\pgfpathlineto{\pgfqpoint{3.134765in}{3.354289in}}%
\pgfpathclose%
\pgfusepath{fill}%
\end{pgfscope}%
\begin{pgfscope}%
\pgfpathrectangle{\pgfqpoint{1.150000in}{0.150000in}}{\pgfqpoint{5.700000in}{5.700000in}}%
\pgfusepath{clip}%
\pgfsetbuttcap%
\pgfsetroundjoin%
\definecolor{currentfill}{rgb}{0.260571,0.246922,0.522828}%
\pgfsetfillcolor{currentfill}%
\pgfsetfillopacity{0.800000}%
\pgfsetlinewidth{0.000000pt}%
\definecolor{currentstroke}{rgb}{0.000000,0.000000,0.000000}%
\pgfsetstrokecolor{currentstroke}%
\pgfsetdash{}{0pt}%
\pgfpathmoveto{\pgfqpoint{4.061124in}{2.832464in}}%
\pgfpathlineto{\pgfqpoint{4.074498in}{2.826345in}}%
\pgfpathlineto{\pgfqpoint{4.087875in}{2.820439in}}%
\pgfpathlineto{\pgfqpoint{4.101258in}{2.814743in}}%
\pgfpathlineto{\pgfqpoint{4.114646in}{2.809258in}}%
\pgfpathlineto{\pgfqpoint{4.122361in}{2.822849in}}%
\pgfpathlineto{\pgfqpoint{4.130073in}{2.836588in}}%
\pgfpathlineto{\pgfqpoint{4.137782in}{2.850482in}}%
\pgfpathlineto{\pgfqpoint{4.145486in}{2.864534in}}%
\pgfpathlineto{\pgfqpoint{4.132106in}{2.870426in}}%
\pgfpathlineto{\pgfqpoint{4.118731in}{2.876529in}}%
\pgfpathlineto{\pgfqpoint{4.105361in}{2.882842in}}%
\pgfpathlineto{\pgfqpoint{4.091995in}{2.889368in}}%
\pgfpathlineto{\pgfqpoint{4.084283in}{2.874897in}}%
\pgfpathlineto{\pgfqpoint{4.076567in}{2.860593in}}%
\pgfpathlineto{\pgfqpoint{4.068848in}{2.846450in}}%
\pgfpathlineto{\pgfqpoint{4.061124in}{2.832464in}}%
\pgfpathclose%
\pgfusepath{fill}%
\end{pgfscope}%
\begin{pgfscope}%
\pgfpathrectangle{\pgfqpoint{1.150000in}{0.150000in}}{\pgfqpoint{5.700000in}{5.700000in}}%
\pgfusepath{clip}%
\pgfsetbuttcap%
\pgfsetroundjoin%
\definecolor{currentfill}{rgb}{0.258965,0.251537,0.524736}%
\pgfsetfillcolor{currentfill}%
\pgfsetfillopacity{0.800000}%
\pgfsetlinewidth{0.000000pt}%
\definecolor{currentstroke}{rgb}{0.000000,0.000000,0.000000}%
\pgfsetstrokecolor{currentstroke}%
\pgfsetdash{}{0pt}%
\pgfpathmoveto{\pgfqpoint{3.562984in}{2.851627in}}%
\pgfpathlineto{\pgfqpoint{3.576321in}{2.840399in}}%
\pgfpathlineto{\pgfqpoint{3.589658in}{2.829419in}}%
\pgfpathlineto{\pgfqpoint{3.602996in}{2.818686in}}%
\pgfpathlineto{\pgfqpoint{3.616333in}{2.808199in}}%
\pgfpathlineto{\pgfqpoint{3.624171in}{2.821888in}}%
\pgfpathlineto{\pgfqpoint{3.632003in}{2.835715in}}%
\pgfpathlineto{\pgfqpoint{3.639831in}{2.849683in}}%
\pgfpathlineto{\pgfqpoint{3.647653in}{2.863797in}}%
\pgfpathlineto{\pgfqpoint{3.634323in}{2.874567in}}%
\pgfpathlineto{\pgfqpoint{3.620993in}{2.885583in}}%
\pgfpathlineto{\pgfqpoint{3.607663in}{2.896846in}}%
\pgfpathlineto{\pgfqpoint{3.594333in}{2.908359in}}%
\pgfpathlineto{\pgfqpoint{3.586504in}{2.893950in}}%
\pgfpathlineto{\pgfqpoint{3.578669in}{2.879694in}}%
\pgfpathlineto{\pgfqpoint{3.570829in}{2.865588in}}%
\pgfpathlineto{\pgfqpoint{3.562984in}{2.851627in}}%
\pgfpathclose%
\pgfusepath{fill}%
\end{pgfscope}%
\begin{pgfscope}%
\pgfpathrectangle{\pgfqpoint{1.150000in}{0.150000in}}{\pgfqpoint{5.700000in}{5.700000in}}%
\pgfusepath{clip}%
\pgfsetbuttcap%
\pgfsetroundjoin%
\definecolor{currentfill}{rgb}{0.177423,0.437527,0.557565}%
\pgfsetfillcolor{currentfill}%
\pgfsetfillopacity{0.800000}%
\pgfsetlinewidth{0.000000pt}%
\definecolor{currentstroke}{rgb}{0.000000,0.000000,0.000000}%
\pgfsetstrokecolor{currentstroke}%
\pgfsetdash{}{0pt}%
\pgfpathmoveto{\pgfqpoint{4.989106in}{3.318848in}}%
\pgfpathlineto{\pgfqpoint{5.002679in}{3.315098in}}%
\pgfpathlineto{\pgfqpoint{5.016260in}{3.311529in}}%
\pgfpathlineto{\pgfqpoint{5.029851in}{3.308141in}}%
\pgfpathlineto{\pgfqpoint{5.043452in}{3.304931in}}%
\pgfpathlineto{\pgfqpoint{5.051014in}{3.322633in}}%
\pgfpathlineto{\pgfqpoint{5.058579in}{3.340692in}}%
\pgfpathlineto{\pgfqpoint{5.066150in}{3.359117in}}%
\pgfpathlineto{\pgfqpoint{5.052561in}{3.362888in}}%
\pgfpathlineto{\pgfqpoint{5.038982in}{3.366838in}}%
\pgfpathlineto{\pgfqpoint{5.025412in}{3.370969in}}%
\pgfpathlineto{\pgfqpoint{5.011850in}{3.375279in}}%
\pgfpathlineto{\pgfqpoint{5.004265in}{3.356098in}}%
\pgfpathlineto{\pgfqpoint{4.996683in}{3.337290in}}%
\pgfpathlineto{\pgfqpoint{4.989106in}{3.318848in}}%
\pgfpathclose%
\pgfusepath{fill}%
\end{pgfscope}%
\begin{pgfscope}%
\pgfpathrectangle{\pgfqpoint{1.150000in}{0.150000in}}{\pgfqpoint{5.700000in}{5.700000in}}%
\pgfusepath{clip}%
\pgfsetbuttcap%
\pgfsetroundjoin%
\definecolor{currentfill}{rgb}{0.144759,0.519093,0.556572}%
\pgfsetfillcolor{currentfill}%
\pgfsetfillopacity{0.800000}%
\pgfsetlinewidth{0.000000pt}%
\definecolor{currentstroke}{rgb}{0.000000,0.000000,0.000000}%
\pgfsetstrokecolor{currentstroke}%
\pgfsetdash{}{0pt}%
\pgfpathmoveto{\pgfqpoint{3.058401in}{3.596968in}}%
\pgfpathlineto{\pgfqpoint{3.071929in}{3.573673in}}%
\pgfpathlineto{\pgfqpoint{3.085448in}{3.550728in}}%
\pgfpathlineto{\pgfqpoint{3.098956in}{3.528128in}}%
\pgfpathlineto{\pgfqpoint{3.112455in}{3.505872in}}%
\pgfpathlineto{\pgfqpoint{3.120344in}{3.523210in}}%
\pgfpathlineto{\pgfqpoint{3.128227in}{3.540782in}}%
\pgfpathlineto{\pgfqpoint{3.136102in}{3.558592in}}%
\pgfpathlineto{\pgfqpoint{3.143971in}{3.576644in}}%
\pgfpathlineto{\pgfqpoint{3.130479in}{3.599228in}}%
\pgfpathlineto{\pgfqpoint{3.116978in}{3.622156in}}%
\pgfpathlineto{\pgfqpoint{3.103467in}{3.645431in}}%
\pgfpathlineto{\pgfqpoint{3.089945in}{3.669056in}}%
\pgfpathlineto{\pgfqpoint{3.082070in}{3.650663in}}%
\pgfpathlineto{\pgfqpoint{3.074188in}{3.632520in}}%
\pgfpathlineto{\pgfqpoint{3.066298in}{3.614623in}}%
\pgfpathlineto{\pgfqpoint{3.058401in}{3.596968in}}%
\pgfpathclose%
\pgfusepath{fill}%
\end{pgfscope}%
\begin{pgfscope}%
\pgfpathrectangle{\pgfqpoint{1.150000in}{0.150000in}}{\pgfqpoint{5.700000in}{5.700000in}}%
\pgfusepath{clip}%
\pgfsetbuttcap%
\pgfsetroundjoin%
\definecolor{currentfill}{rgb}{0.229739,0.322361,0.545706}%
\pgfsetfillcolor{currentfill}%
\pgfsetfillopacity{0.800000}%
\pgfsetlinewidth{0.000000pt}%
\definecolor{currentstroke}{rgb}{0.000000,0.000000,0.000000}%
\pgfsetstrokecolor{currentstroke}%
\pgfsetdash{}{0pt}%
\pgfpathmoveto{\pgfqpoint{4.536553in}{3.001319in}}%
\pgfpathlineto{\pgfqpoint{4.550032in}{2.997696in}}%
\pgfpathlineto{\pgfqpoint{4.563520in}{2.994266in}}%
\pgfpathlineto{\pgfqpoint{4.577015in}{2.991027in}}%
\pgfpathlineto{\pgfqpoint{4.590518in}{2.987978in}}%
\pgfpathlineto{\pgfqpoint{4.598125in}{3.002247in}}%
\pgfpathlineto{\pgfqpoint{4.605730in}{3.016731in}}%
\pgfpathlineto{\pgfqpoint{4.613334in}{3.031436in}}%
\pgfpathlineto{\pgfqpoint{4.620937in}{3.046368in}}%
\pgfpathlineto{\pgfqpoint{4.607445in}{3.049979in}}%
\pgfpathlineto{\pgfqpoint{4.593960in}{3.053782in}}%
\pgfpathlineto{\pgfqpoint{4.580484in}{3.057776in}}%
\pgfpathlineto{\pgfqpoint{4.567014in}{3.061962in}}%
\pgfpathlineto{\pgfqpoint{4.559401in}{3.046455in}}%
\pgfpathlineto{\pgfqpoint{4.551786in}{3.031183in}}%
\pgfpathlineto{\pgfqpoint{4.544170in}{3.016140in}}%
\pgfpathlineto{\pgfqpoint{4.536553in}{3.001319in}}%
\pgfpathclose%
\pgfusepath{fill}%
\end{pgfscope}%
\begin{pgfscope}%
\pgfpathrectangle{\pgfqpoint{1.150000in}{0.150000in}}{\pgfqpoint{5.700000in}{5.700000in}}%
\pgfusepath{clip}%
\pgfsetbuttcap%
\pgfsetroundjoin%
\definecolor{currentfill}{rgb}{0.221989,0.339161,0.548752}%
\pgfsetfillcolor{currentfill}%
\pgfsetfillopacity{0.800000}%
\pgfsetlinewidth{0.000000pt}%
\definecolor{currentstroke}{rgb}{0.000000,0.000000,0.000000}%
\pgfsetstrokecolor{currentstroke}%
\pgfsetdash{}{0pt}%
\pgfpathmoveto{\pgfqpoint{4.620937in}{3.046368in}}%
\pgfpathlineto{\pgfqpoint{4.634437in}{3.042946in}}%
\pgfpathlineto{\pgfqpoint{4.647946in}{3.039714in}}%
\pgfpathlineto{\pgfqpoint{4.661462in}{3.036670in}}%
\pgfpathlineto{\pgfqpoint{4.674988in}{3.033815in}}%
\pgfpathlineto{\pgfqpoint{4.682578in}{3.048399in}}%
\pgfpathlineto{\pgfqpoint{4.690168in}{3.063215in}}%
\pgfpathlineto{\pgfqpoint{4.697757in}{3.078272in}}%
\pgfpathlineto{\pgfqpoint{4.705346in}{3.093576in}}%
\pgfpathlineto{\pgfqpoint{4.691833in}{3.097025in}}%
\pgfpathlineto{\pgfqpoint{4.678328in}{3.100664in}}%
\pgfpathlineto{\pgfqpoint{4.664831in}{3.104491in}}%
\pgfpathlineto{\pgfqpoint{4.651342in}{3.108507in}}%
\pgfpathlineto{\pgfqpoint{4.643741in}{3.092597in}}%
\pgfpathlineto{\pgfqpoint{4.636141in}{3.076941in}}%
\pgfpathlineto{\pgfqpoint{4.628539in}{3.061534in}}%
\pgfpathlineto{\pgfqpoint{4.620937in}{3.046368in}}%
\pgfpathclose%
\pgfusepath{fill}%
\end{pgfscope}%
\begin{pgfscope}%
\pgfpathrectangle{\pgfqpoint{1.150000in}{0.150000in}}{\pgfqpoint{5.700000in}{5.700000in}}%
\pgfusepath{clip}%
\pgfsetbuttcap%
\pgfsetroundjoin%
\definecolor{currentfill}{rgb}{0.237441,0.305202,0.541921}%
\pgfsetfillcolor{currentfill}%
\pgfsetfillopacity{0.800000}%
\pgfsetlinewidth{0.000000pt}%
\definecolor{currentstroke}{rgb}{0.000000,0.000000,0.000000}%
\pgfsetstrokecolor{currentstroke}%
\pgfsetdash{}{0pt}%
\pgfpathmoveto{\pgfqpoint{4.452183in}{2.958427in}}%
\pgfpathlineto{\pgfqpoint{4.465643in}{2.954561in}}%
\pgfpathlineto{\pgfqpoint{4.479110in}{2.950890in}}%
\pgfpathlineto{\pgfqpoint{4.492584in}{2.947413in}}%
\pgfpathlineto{\pgfqpoint{4.506067in}{2.944130in}}%
\pgfpathlineto{\pgfqpoint{4.513691in}{2.958126in}}%
\pgfpathlineto{\pgfqpoint{4.521313in}{2.972318in}}%
\pgfpathlineto{\pgfqpoint{4.528934in}{2.986714in}}%
\pgfpathlineto{\pgfqpoint{4.536553in}{3.001319in}}%
\pgfpathlineto{\pgfqpoint{4.523081in}{3.005135in}}%
\pgfpathlineto{\pgfqpoint{4.509616in}{3.009144in}}%
\pgfpathlineto{\pgfqpoint{4.496159in}{3.013347in}}%
\pgfpathlineto{\pgfqpoint{4.482709in}{3.017745in}}%
\pgfpathlineto{\pgfqpoint{4.475080in}{3.002596in}}%
\pgfpathlineto{\pgfqpoint{4.467449in}{2.987664in}}%
\pgfpathlineto{\pgfqpoint{4.459817in}{2.972943in}}%
\pgfpathlineto{\pgfqpoint{4.452183in}{2.958427in}}%
\pgfpathclose%
\pgfusepath{fill}%
\end{pgfscope}%
\begin{pgfscope}%
\pgfpathrectangle{\pgfqpoint{1.150000in}{0.150000in}}{\pgfqpoint{5.700000in}{5.700000in}}%
\pgfusepath{clip}%
\pgfsetbuttcap%
\pgfsetroundjoin%
\definecolor{currentfill}{rgb}{0.233603,0.313828,0.543914}%
\pgfsetfillcolor{currentfill}%
\pgfsetfillopacity{0.800000}%
\pgfsetlinewidth{0.000000pt}%
\definecolor{currentstroke}{rgb}{0.000000,0.000000,0.000000}%
\pgfsetstrokecolor{currentstroke}%
\pgfsetdash{}{0pt}%
\pgfpathmoveto{\pgfqpoint{3.317821in}{3.008143in}}%
\pgfpathlineto{\pgfqpoint{3.331199in}{2.992930in}}%
\pgfpathlineto{\pgfqpoint{3.344573in}{2.977997in}}%
\pgfpathlineto{\pgfqpoint{3.357944in}{2.963342in}}%
\pgfpathlineto{\pgfqpoint{3.371311in}{2.948963in}}%
\pgfpathlineto{\pgfqpoint{3.379203in}{2.963095in}}%
\pgfpathlineto{\pgfqpoint{3.387089in}{2.977384in}}%
\pgfpathlineto{\pgfqpoint{3.394969in}{2.991833in}}%
\pgfpathlineto{\pgfqpoint{3.402843in}{3.006444in}}%
\pgfpathlineto{\pgfqpoint{3.389484in}{3.021077in}}%
\pgfpathlineto{\pgfqpoint{3.376121in}{3.035986in}}%
\pgfpathlineto{\pgfqpoint{3.362755in}{3.051173in}}%
\pgfpathlineto{\pgfqpoint{3.349386in}{3.066641in}}%
\pgfpathlineto{\pgfqpoint{3.341504in}{3.051763in}}%
\pgfpathlineto{\pgfqpoint{3.333615in}{3.037056in}}%
\pgfpathlineto{\pgfqpoint{3.325721in}{3.022517in}}%
\pgfpathlineto{\pgfqpoint{3.317821in}{3.008143in}}%
\pgfpathclose%
\pgfusepath{fill}%
\end{pgfscope}%
\begin{pgfscope}%
\pgfpathrectangle{\pgfqpoint{1.150000in}{0.150000in}}{\pgfqpoint{5.700000in}{5.700000in}}%
\pgfusepath{clip}%
\pgfsetbuttcap%
\pgfsetroundjoin%
\definecolor{currentfill}{rgb}{0.221989,0.339161,0.548752}%
\pgfsetfillcolor{currentfill}%
\pgfsetfillopacity{0.800000}%
\pgfsetlinewidth{0.000000pt}%
\definecolor{currentstroke}{rgb}{0.000000,0.000000,0.000000}%
\pgfsetstrokecolor{currentstroke}%
\pgfsetdash{}{0pt}%
\pgfpathmoveto{\pgfqpoint{3.264266in}{3.071846in}}%
\pgfpathlineto{\pgfqpoint{3.277662in}{3.055488in}}%
\pgfpathlineto{\pgfqpoint{3.291053in}{3.039420in}}%
\pgfpathlineto{\pgfqpoint{3.304439in}{3.023639in}}%
\pgfpathlineto{\pgfqpoint{3.317821in}{3.008143in}}%
\pgfpathlineto{\pgfqpoint{3.325721in}{3.022517in}}%
\pgfpathlineto{\pgfqpoint{3.333615in}{3.037056in}}%
\pgfpathlineto{\pgfqpoint{3.341504in}{3.051763in}}%
\pgfpathlineto{\pgfqpoint{3.349386in}{3.066641in}}%
\pgfpathlineto{\pgfqpoint{3.336012in}{3.082391in}}%
\pgfpathlineto{\pgfqpoint{3.322635in}{3.098427in}}%
\pgfpathlineto{\pgfqpoint{3.309252in}{3.114750in}}%
\pgfpathlineto{\pgfqpoint{3.295866in}{3.131363in}}%
\pgfpathlineto{\pgfqpoint{3.287975in}{3.116218in}}%
\pgfpathlineto{\pgfqpoint{3.280079in}{3.101252in}}%
\pgfpathlineto{\pgfqpoint{3.272176in}{3.086463in}}%
\pgfpathlineto{\pgfqpoint{3.264266in}{3.071846in}}%
\pgfpathclose%
\pgfusepath{fill}%
\end{pgfscope}%
\begin{pgfscope}%
\pgfpathrectangle{\pgfqpoint{1.150000in}{0.150000in}}{\pgfqpoint{5.700000in}{5.700000in}}%
\pgfusepath{clip}%
\pgfsetbuttcap%
\pgfsetroundjoin%
\definecolor{currentfill}{rgb}{0.263663,0.237631,0.518762}%
\pgfsetfillcolor{currentfill}%
\pgfsetfillopacity{0.800000}%
\pgfsetlinewidth{0.000000pt}%
\definecolor{currentstroke}{rgb}{0.000000,0.000000,0.000000}%
\pgfsetstrokecolor{currentstroke}%
\pgfsetdash{}{0pt}%
\pgfpathmoveto{\pgfqpoint{3.976715in}{2.803104in}}%
\pgfpathlineto{\pgfqpoint{3.990078in}{2.796503in}}%
\pgfpathlineto{\pgfqpoint{4.003446in}{2.790118in}}%
\pgfpathlineto{\pgfqpoint{4.016817in}{2.783948in}}%
\pgfpathlineto{\pgfqpoint{4.030193in}{2.777993in}}%
\pgfpathlineto{\pgfqpoint{4.037932in}{2.791398in}}%
\pgfpathlineto{\pgfqpoint{4.045667in}{2.804943in}}%
\pgfpathlineto{\pgfqpoint{4.053397in}{2.818630in}}%
\pgfpathlineto{\pgfqpoint{4.061124in}{2.832464in}}%
\pgfpathlineto{\pgfqpoint{4.047756in}{2.838795in}}%
\pgfpathlineto{\pgfqpoint{4.034392in}{2.845341in}}%
\pgfpathlineto{\pgfqpoint{4.021032in}{2.852102in}}%
\pgfpathlineto{\pgfqpoint{4.007676in}{2.859079in}}%
\pgfpathlineto{\pgfqpoint{3.999942in}{2.844857in}}%
\pgfpathlineto{\pgfqpoint{3.992204in}{2.830790in}}%
\pgfpathlineto{\pgfqpoint{3.984462in}{2.816874in}}%
\pgfpathlineto{\pgfqpoint{3.976715in}{2.803104in}}%
\pgfpathclose%
\pgfusepath{fill}%
\end{pgfscope}%
\begin{pgfscope}%
\pgfpathrectangle{\pgfqpoint{1.150000in}{0.150000in}}{\pgfqpoint{5.700000in}{5.700000in}}%
\pgfusepath{clip}%
\pgfsetbuttcap%
\pgfsetroundjoin%
\definecolor{currentfill}{rgb}{0.212395,0.359683,0.551710}%
\pgfsetfillcolor{currentfill}%
\pgfsetfillopacity{0.800000}%
\pgfsetlinewidth{0.000000pt}%
\definecolor{currentstroke}{rgb}{0.000000,0.000000,0.000000}%
\pgfsetstrokecolor{currentstroke}%
\pgfsetdash{}{0pt}%
\pgfpathmoveto{\pgfqpoint{4.705346in}{3.093576in}}%
\pgfpathlineto{\pgfqpoint{4.718868in}{3.090313in}}%
\pgfpathlineto{\pgfqpoint{4.732398in}{3.087238in}}%
\pgfpathlineto{\pgfqpoint{4.745937in}{3.084349in}}%
\pgfpathlineto{\pgfqpoint{4.759484in}{3.081646in}}%
\pgfpathlineto{\pgfqpoint{4.767061in}{3.096590in}}%
\pgfpathlineto{\pgfqpoint{4.774637in}{3.111787in}}%
\pgfpathlineto{\pgfqpoint{4.782214in}{3.127244in}}%
\pgfpathlineto{\pgfqpoint{4.789791in}{3.142970in}}%
\pgfpathlineto{\pgfqpoint{4.776256in}{3.146300in}}%
\pgfpathlineto{\pgfqpoint{4.762730in}{3.149815in}}%
\pgfpathlineto{\pgfqpoint{4.749212in}{3.153516in}}%
\pgfpathlineto{\pgfqpoint{4.735703in}{3.157405in}}%
\pgfpathlineto{\pgfqpoint{4.728113in}{3.141041in}}%
\pgfpathlineto{\pgfqpoint{4.720524in}{3.124953in}}%
\pgfpathlineto{\pgfqpoint{4.712935in}{3.109134in}}%
\pgfpathlineto{\pgfqpoint{4.705346in}{3.093576in}}%
\pgfpathclose%
\pgfusepath{fill}%
\end{pgfscope}%
\begin{pgfscope}%
\pgfpathrectangle{\pgfqpoint{1.150000in}{0.150000in}}{\pgfqpoint{5.700000in}{5.700000in}}%
\pgfusepath{clip}%
\pgfsetbuttcap%
\pgfsetroundjoin%
\definecolor{currentfill}{rgb}{0.244972,0.287675,0.537260}%
\pgfsetfillcolor{currentfill}%
\pgfsetfillopacity{0.800000}%
\pgfsetlinewidth{0.000000pt}%
\definecolor{currentstroke}{rgb}{0.000000,0.000000,0.000000}%
\pgfsetstrokecolor{currentstroke}%
\pgfsetdash{}{0pt}%
\pgfpathmoveto{\pgfqpoint{4.367818in}{2.917715in}}%
\pgfpathlineto{\pgfqpoint{4.381259in}{2.913563in}}%
\pgfpathlineto{\pgfqpoint{4.394706in}{2.909609in}}%
\pgfpathlineto{\pgfqpoint{4.408161in}{2.905851in}}%
\pgfpathlineto{\pgfqpoint{4.421624in}{2.902291in}}%
\pgfpathlineto{\pgfqpoint{4.429267in}{2.916047in}}%
\pgfpathlineto{\pgfqpoint{4.436908in}{2.929985in}}%
\pgfpathlineto{\pgfqpoint{4.444546in}{2.944110in}}%
\pgfpathlineto{\pgfqpoint{4.452183in}{2.958427in}}%
\pgfpathlineto{\pgfqpoint{4.438730in}{2.962489in}}%
\pgfpathlineto{\pgfqpoint{4.425285in}{2.966747in}}%
\pgfpathlineto{\pgfqpoint{4.411846in}{2.971202in}}%
\pgfpathlineto{\pgfqpoint{4.398414in}{2.975855in}}%
\pgfpathlineto{\pgfqpoint{4.390769in}{2.961025in}}%
\pgfpathlineto{\pgfqpoint{4.383121in}{2.946396in}}%
\pgfpathlineto{\pgfqpoint{4.375470in}{2.931961in}}%
\pgfpathlineto{\pgfqpoint{4.367818in}{2.917715in}}%
\pgfpathclose%
\pgfusepath{fill}%
\end{pgfscope}%
\begin{pgfscope}%
\pgfpathrectangle{\pgfqpoint{1.150000in}{0.150000in}}{\pgfqpoint{5.700000in}{5.700000in}}%
\pgfusepath{clip}%
\pgfsetbuttcap%
\pgfsetroundjoin%
\definecolor{currentfill}{rgb}{0.243113,0.292092,0.538516}%
\pgfsetfillcolor{currentfill}%
\pgfsetfillopacity{0.800000}%
\pgfsetlinewidth{0.000000pt}%
\definecolor{currentstroke}{rgb}{0.000000,0.000000,0.000000}%
\pgfsetstrokecolor{currentstroke}%
\pgfsetdash{}{0pt}%
\pgfpathmoveto{\pgfqpoint{3.371311in}{2.948963in}}%
\pgfpathlineto{\pgfqpoint{3.384676in}{2.934858in}}%
\pgfpathlineto{\pgfqpoint{3.398038in}{2.921024in}}%
\pgfpathlineto{\pgfqpoint{3.411397in}{2.907460in}}%
\pgfpathlineto{\pgfqpoint{3.424754in}{2.894163in}}%
\pgfpathlineto{\pgfqpoint{3.432637in}{2.908054in}}%
\pgfpathlineto{\pgfqpoint{3.440514in}{2.922093in}}%
\pgfpathlineto{\pgfqpoint{3.448386in}{2.936284in}}%
\pgfpathlineto{\pgfqpoint{3.456252in}{2.950631in}}%
\pgfpathlineto{\pgfqpoint{3.442904in}{2.964181in}}%
\pgfpathlineto{\pgfqpoint{3.429553in}{2.977998in}}%
\pgfpathlineto{\pgfqpoint{3.416199in}{2.992085in}}%
\pgfpathlineto{\pgfqpoint{3.402843in}{3.006444in}}%
\pgfpathlineto{\pgfqpoint{3.394969in}{2.991833in}}%
\pgfpathlineto{\pgfqpoint{3.387089in}{2.977384in}}%
\pgfpathlineto{\pgfqpoint{3.379203in}{2.963095in}}%
\pgfpathlineto{\pgfqpoint{3.371311in}{2.948963in}}%
\pgfpathclose%
\pgfusepath{fill}%
\end{pgfscope}%
\begin{pgfscope}%
\pgfpathrectangle{\pgfqpoint{1.150000in}{0.150000in}}{\pgfqpoint{5.700000in}{5.700000in}}%
\pgfusepath{clip}%
\pgfsetbuttcap%
\pgfsetroundjoin%
\definecolor{currentfill}{rgb}{0.210503,0.363727,0.552206}%
\pgfsetfillcolor{currentfill}%
\pgfsetfillopacity{0.800000}%
\pgfsetlinewidth{0.000000pt}%
\definecolor{currentstroke}{rgb}{0.000000,0.000000,0.000000}%
\pgfsetstrokecolor{currentstroke}%
\pgfsetdash{}{0pt}%
\pgfpathmoveto{\pgfqpoint{3.210633in}{3.140225in}}%
\pgfpathlineto{\pgfqpoint{3.224049in}{3.122683in}}%
\pgfpathlineto{\pgfqpoint{3.237461in}{3.105441in}}%
\pgfpathlineto{\pgfqpoint{3.250866in}{3.088496in}}%
\pgfpathlineto{\pgfqpoint{3.264266in}{3.071846in}}%
\pgfpathlineto{\pgfqpoint{3.272176in}{3.086463in}}%
\pgfpathlineto{\pgfqpoint{3.280079in}{3.101252in}}%
\pgfpathlineto{\pgfqpoint{3.287975in}{3.116218in}}%
\pgfpathlineto{\pgfqpoint{3.295866in}{3.131363in}}%
\pgfpathlineto{\pgfqpoint{3.282474in}{3.148269in}}%
\pgfpathlineto{\pgfqpoint{3.269077in}{3.165469in}}%
\pgfpathlineto{\pgfqpoint{3.255675in}{3.182968in}}%
\pgfpathlineto{\pgfqpoint{3.242267in}{3.200766in}}%
\pgfpathlineto{\pgfqpoint{3.234368in}{3.185353in}}%
\pgfpathlineto{\pgfqpoint{3.226463in}{3.170127in}}%
\pgfpathlineto{\pgfqpoint{3.218551in}{3.155086in}}%
\pgfpathlineto{\pgfqpoint{3.210633in}{3.140225in}}%
\pgfpathclose%
\pgfusepath{fill}%
\end{pgfscope}%
\begin{pgfscope}%
\pgfpathrectangle{\pgfqpoint{1.150000in}{0.150000in}}{\pgfqpoint{5.700000in}{5.700000in}}%
\pgfusepath{clip}%
\pgfsetbuttcap%
\pgfsetroundjoin%
\definecolor{currentfill}{rgb}{0.165117,0.467423,0.558141}%
\pgfsetfillcolor{currentfill}%
\pgfsetfillopacity{0.800000}%
\pgfsetlinewidth{0.000000pt}%
\definecolor{currentstroke}{rgb}{0.000000,0.000000,0.000000}%
\pgfsetstrokecolor{currentstroke}%
\pgfsetdash{}{0pt}%
\pgfpathmoveto{\pgfqpoint{3.080827in}{3.438781in}}%
\pgfpathlineto{\pgfqpoint{3.094325in}{3.417158in}}%
\pgfpathlineto{\pgfqpoint{3.107813in}{3.395870in}}%
\pgfpathlineto{\pgfqpoint{3.121293in}{3.374915in}}%
\pgfpathlineto{\pgfqpoint{3.134765in}{3.354289in}}%
\pgfpathlineto{\pgfqpoint{3.142674in}{3.370450in}}%
\pgfpathlineto{\pgfqpoint{3.150576in}{3.386820in}}%
\pgfpathlineto{\pgfqpoint{3.158472in}{3.403405in}}%
\pgfpathlineto{\pgfqpoint{3.166360in}{3.420208in}}%
\pgfpathlineto{\pgfqpoint{3.152897in}{3.441126in}}%
\pgfpathlineto{\pgfqpoint{3.139425in}{3.462374in}}%
\pgfpathlineto{\pgfqpoint{3.125945in}{3.483955in}}%
\pgfpathlineto{\pgfqpoint{3.112455in}{3.505872in}}%
\pgfpathlineto{\pgfqpoint{3.104559in}{3.488764in}}%
\pgfpathlineto{\pgfqpoint{3.096655in}{3.471882in}}%
\pgfpathlineto{\pgfqpoint{3.088744in}{3.455222in}}%
\pgfpathlineto{\pgfqpoint{3.080827in}{3.438781in}}%
\pgfpathclose%
\pgfusepath{fill}%
\end{pgfscope}%
\begin{pgfscope}%
\pgfpathrectangle{\pgfqpoint{1.150000in}{0.150000in}}{\pgfqpoint{5.700000in}{5.700000in}}%
\pgfusepath{clip}%
\pgfsetbuttcap%
\pgfsetroundjoin%
\definecolor{currentfill}{rgb}{0.204903,0.375746,0.553533}%
\pgfsetfillcolor{currentfill}%
\pgfsetfillopacity{0.800000}%
\pgfsetlinewidth{0.000000pt}%
\definecolor{currentstroke}{rgb}{0.000000,0.000000,0.000000}%
\pgfsetstrokecolor{currentstroke}%
\pgfsetdash{}{0pt}%
\pgfpathmoveto{\pgfqpoint{4.789791in}{3.142970in}}%
\pgfpathlineto{\pgfqpoint{4.803334in}{3.139826in}}%
\pgfpathlineto{\pgfqpoint{4.816887in}{3.136867in}}%
\pgfpathlineto{\pgfqpoint{4.830448in}{3.134092in}}%
\pgfpathlineto{\pgfqpoint{4.844018in}{3.131501in}}%
\pgfpathlineto{\pgfqpoint{4.851582in}{3.146856in}}%
\pgfpathlineto{\pgfqpoint{4.859148in}{3.162487in}}%
\pgfpathlineto{\pgfqpoint{4.866714in}{3.178401in}}%
\pgfpathlineto{\pgfqpoint{4.874282in}{3.194605in}}%
\pgfpathlineto{\pgfqpoint{4.860726in}{3.197854in}}%
\pgfpathlineto{\pgfqpoint{4.847178in}{3.201287in}}%
\pgfpathlineto{\pgfqpoint{4.833639in}{3.204904in}}%
\pgfpathlineto{\pgfqpoint{4.820109in}{3.208706in}}%
\pgfpathlineto{\pgfqpoint{4.812527in}{3.191832in}}%
\pgfpathlineto{\pgfqpoint{4.804948in}{3.175257in}}%
\pgfpathlineto{\pgfqpoint{4.797369in}{3.158972in}}%
\pgfpathlineto{\pgfqpoint{4.789791in}{3.142970in}}%
\pgfpathclose%
\pgfusepath{fill}%
\end{pgfscope}%
\begin{pgfscope}%
\pgfpathrectangle{\pgfqpoint{1.150000in}{0.150000in}}{\pgfqpoint{5.700000in}{5.700000in}}%
\pgfusepath{clip}%
\pgfsetbuttcap%
\pgfsetroundjoin%
\definecolor{currentfill}{rgb}{0.250425,0.274290,0.533103}%
\pgfsetfillcolor{currentfill}%
\pgfsetfillopacity{0.800000}%
\pgfsetlinewidth{0.000000pt}%
\definecolor{currentstroke}{rgb}{0.000000,0.000000,0.000000}%
\pgfsetstrokecolor{currentstroke}%
\pgfsetdash{}{0pt}%
\pgfpathmoveto{\pgfqpoint{4.283447in}{2.879230in}}%
\pgfpathlineto{\pgfqpoint{4.296870in}{2.874748in}}%
\pgfpathlineto{\pgfqpoint{4.310300in}{2.870467in}}%
\pgfpathlineto{\pgfqpoint{4.323737in}{2.866386in}}%
\pgfpathlineto{\pgfqpoint{4.337180in}{2.862505in}}%
\pgfpathlineto{\pgfqpoint{4.344843in}{2.876052in}}%
\pgfpathlineto{\pgfqpoint{4.352504in}{2.889766in}}%
\pgfpathlineto{\pgfqpoint{4.360162in}{2.903652in}}%
\pgfpathlineto{\pgfqpoint{4.367818in}{2.917715in}}%
\pgfpathlineto{\pgfqpoint{4.354383in}{2.922066in}}%
\pgfpathlineto{\pgfqpoint{4.340956in}{2.926616in}}%
\pgfpathlineto{\pgfqpoint{4.327535in}{2.931367in}}%
\pgfpathlineto{\pgfqpoint{4.314120in}{2.936318in}}%
\pgfpathlineto{\pgfqpoint{4.306456in}{2.921774in}}%
\pgfpathlineto{\pgfqpoint{4.298789in}{2.907415in}}%
\pgfpathlineto{\pgfqpoint{4.291119in}{2.893236in}}%
\pgfpathlineto{\pgfqpoint{4.283447in}{2.879230in}}%
\pgfpathclose%
\pgfusepath{fill}%
\end{pgfscope}%
\begin{pgfscope}%
\pgfpathrectangle{\pgfqpoint{1.150000in}{0.150000in}}{\pgfqpoint{5.700000in}{5.700000in}}%
\pgfusepath{clip}%
\pgfsetbuttcap%
\pgfsetroundjoin%
\definecolor{currentfill}{rgb}{0.266580,0.228262,0.514349}%
\pgfsetfillcolor{currentfill}%
\pgfsetfillopacity{0.800000}%
\pgfsetlinewidth{0.000000pt}%
\definecolor{currentstroke}{rgb}{0.000000,0.000000,0.000000}%
\pgfsetstrokecolor{currentstroke}%
\pgfsetdash{}{0pt}%
\pgfpathmoveto{\pgfqpoint{3.754325in}{2.786292in}}%
\pgfpathlineto{\pgfqpoint{3.767666in}{2.777663in}}%
\pgfpathlineto{\pgfqpoint{3.781008in}{2.769266in}}%
\pgfpathlineto{\pgfqpoint{3.794354in}{2.761097in}}%
\pgfpathlineto{\pgfqpoint{3.807701in}{2.753157in}}%
\pgfpathlineto{\pgfqpoint{3.815497in}{2.766513in}}%
\pgfpathlineto{\pgfqpoint{3.823288in}{2.779997in}}%
\pgfpathlineto{\pgfqpoint{3.831074in}{2.793614in}}%
\pgfpathlineto{\pgfqpoint{3.838856in}{2.807369in}}%
\pgfpathlineto{\pgfqpoint{3.825516in}{2.815623in}}%
\pgfpathlineto{\pgfqpoint{3.812178in}{2.824105in}}%
\pgfpathlineto{\pgfqpoint{3.798842in}{2.832816in}}%
\pgfpathlineto{\pgfqpoint{3.785509in}{2.841759in}}%
\pgfpathlineto{\pgfqpoint{3.777720in}{2.827679in}}%
\pgfpathlineto{\pgfqpoint{3.769926in}{2.813744in}}%
\pgfpathlineto{\pgfqpoint{3.762128in}{2.799949in}}%
\pgfpathlineto{\pgfqpoint{3.754325in}{2.786292in}}%
\pgfpathclose%
\pgfusepath{fill}%
\end{pgfscope}%
\begin{pgfscope}%
\pgfpathrectangle{\pgfqpoint{1.150000in}{0.150000in}}{\pgfqpoint{5.700000in}{5.700000in}}%
\pgfusepath{clip}%
\pgfsetbuttcap%
\pgfsetroundjoin%
\definecolor{currentfill}{rgb}{0.263663,0.237631,0.518762}%
\pgfsetfillcolor{currentfill}%
\pgfsetfillopacity{0.800000}%
\pgfsetlinewidth{0.000000pt}%
\definecolor{currentstroke}{rgb}{0.000000,0.000000,0.000000}%
\pgfsetstrokecolor{currentstroke}%
\pgfsetdash{}{0pt}%
\pgfpathmoveto{\pgfqpoint{3.616333in}{2.808199in}}%
\pgfpathlineto{\pgfqpoint{3.629671in}{2.797956in}}%
\pgfpathlineto{\pgfqpoint{3.643010in}{2.787956in}}%
\pgfpathlineto{\pgfqpoint{3.656349in}{2.778196in}}%
\pgfpathlineto{\pgfqpoint{3.669690in}{2.768676in}}%
\pgfpathlineto{\pgfqpoint{3.677519in}{2.782093in}}%
\pgfpathlineto{\pgfqpoint{3.685344in}{2.795641in}}%
\pgfpathlineto{\pgfqpoint{3.693164in}{2.809322in}}%
\pgfpathlineto{\pgfqpoint{3.700979in}{2.823141in}}%
\pgfpathlineto{\pgfqpoint{3.687646in}{2.832945in}}%
\pgfpathlineto{\pgfqpoint{3.674315in}{2.842987in}}%
\pgfpathlineto{\pgfqpoint{3.660983in}{2.853271in}}%
\pgfpathlineto{\pgfqpoint{3.647653in}{2.863797in}}%
\pgfpathlineto{\pgfqpoint{3.639831in}{2.849683in}}%
\pgfpathlineto{\pgfqpoint{3.632003in}{2.835715in}}%
\pgfpathlineto{\pgfqpoint{3.624171in}{2.821888in}}%
\pgfpathlineto{\pgfqpoint{3.616333in}{2.808199in}}%
\pgfpathclose%
\pgfusepath{fill}%
\end{pgfscope}%
\begin{pgfscope}%
\pgfpathrectangle{\pgfqpoint{1.150000in}{0.150000in}}{\pgfqpoint{5.700000in}{5.700000in}}%
\pgfusepath{clip}%
\pgfsetbuttcap%
\pgfsetroundjoin%
\definecolor{currentfill}{rgb}{0.250425,0.274290,0.533103}%
\pgfsetfillcolor{currentfill}%
\pgfsetfillopacity{0.800000}%
\pgfsetlinewidth{0.000000pt}%
\definecolor{currentstroke}{rgb}{0.000000,0.000000,0.000000}%
\pgfsetstrokecolor{currentstroke}%
\pgfsetdash{}{0pt}%
\pgfpathmoveto{\pgfqpoint{3.424754in}{2.894163in}}%
\pgfpathlineto{\pgfqpoint{3.438108in}{2.881132in}}%
\pgfpathlineto{\pgfqpoint{3.451461in}{2.868364in}}%
\pgfpathlineto{\pgfqpoint{3.464812in}{2.855857in}}%
\pgfpathlineto{\pgfqpoint{3.478162in}{2.843610in}}%
\pgfpathlineto{\pgfqpoint{3.486036in}{2.857260in}}%
\pgfpathlineto{\pgfqpoint{3.493906in}{2.871050in}}%
\pgfpathlineto{\pgfqpoint{3.501769in}{2.884984in}}%
\pgfpathlineto{\pgfqpoint{3.509628in}{2.899066in}}%
\pgfpathlineto{\pgfqpoint{3.496286in}{2.911566in}}%
\pgfpathlineto{\pgfqpoint{3.482943in}{2.924325in}}%
\pgfpathlineto{\pgfqpoint{3.469599in}{2.937346in}}%
\pgfpathlineto{\pgfqpoint{3.456252in}{2.950631in}}%
\pgfpathlineto{\pgfqpoint{3.448386in}{2.936284in}}%
\pgfpathlineto{\pgfqpoint{3.440514in}{2.922093in}}%
\pgfpathlineto{\pgfqpoint{3.432637in}{2.908054in}}%
\pgfpathlineto{\pgfqpoint{3.424754in}{2.894163in}}%
\pgfpathclose%
\pgfusepath{fill}%
\end{pgfscope}%
\begin{pgfscope}%
\pgfpathrectangle{\pgfqpoint{1.150000in}{0.150000in}}{\pgfqpoint{5.700000in}{5.700000in}}%
\pgfusepath{clip}%
\pgfsetbuttcap%
\pgfsetroundjoin%
\definecolor{currentfill}{rgb}{0.195860,0.395433,0.555276}%
\pgfsetfillcolor{currentfill}%
\pgfsetfillopacity{0.800000}%
\pgfsetlinewidth{0.000000pt}%
\definecolor{currentstroke}{rgb}{0.000000,0.000000,0.000000}%
\pgfsetstrokecolor{currentstroke}%
\pgfsetdash{}{0pt}%
\pgfpathmoveto{\pgfqpoint{4.874282in}{3.194605in}}%
\pgfpathlineto{\pgfqpoint{4.887848in}{3.191539in}}%
\pgfpathlineto{\pgfqpoint{4.901422in}{3.188655in}}%
\pgfpathlineto{\pgfqpoint{4.915006in}{3.185954in}}%
\pgfpathlineto{\pgfqpoint{4.928599in}{3.183434in}}%
\pgfpathlineto{\pgfqpoint{4.936154in}{3.199259in}}%
\pgfpathlineto{\pgfqpoint{4.943711in}{3.215383in}}%
\pgfpathlineto{\pgfqpoint{4.951270in}{3.231813in}}%
\pgfpathlineto{\pgfqpoint{4.958831in}{3.248558in}}%
\pgfpathlineto{\pgfqpoint{4.945253in}{3.251767in}}%
\pgfpathlineto{\pgfqpoint{4.931684in}{3.255158in}}%
\pgfpathlineto{\pgfqpoint{4.918123in}{3.258731in}}%
\pgfpathlineto{\pgfqpoint{4.904572in}{3.262486in}}%
\pgfpathlineto{\pgfqpoint{4.896996in}{3.245040in}}%
\pgfpathlineto{\pgfqpoint{4.889423in}{3.227917in}}%
\pgfpathlineto{\pgfqpoint{4.881852in}{3.211108in}}%
\pgfpathlineto{\pgfqpoint{4.874282in}{3.194605in}}%
\pgfpathclose%
\pgfusepath{fill}%
\end{pgfscope}%
\begin{pgfscope}%
\pgfpathrectangle{\pgfqpoint{1.150000in}{0.150000in}}{\pgfqpoint{5.700000in}{5.700000in}}%
\pgfusepath{clip}%
\pgfsetbuttcap%
\pgfsetroundjoin%
\definecolor{currentfill}{rgb}{0.197636,0.391528,0.554969}%
\pgfsetfillcolor{currentfill}%
\pgfsetfillopacity{0.800000}%
\pgfsetlinewidth{0.000000pt}%
\definecolor{currentstroke}{rgb}{0.000000,0.000000,0.000000}%
\pgfsetstrokecolor{currentstroke}%
\pgfsetdash{}{0pt}%
\pgfpathmoveto{\pgfqpoint{3.156903in}{3.213446in}}%
\pgfpathlineto{\pgfqpoint{3.170345in}{3.194677in}}%
\pgfpathlineto{\pgfqpoint{3.183781in}{3.176219in}}%
\pgfpathlineto{\pgfqpoint{3.197210in}{3.158069in}}%
\pgfpathlineto{\pgfqpoint{3.210633in}{3.140225in}}%
\pgfpathlineto{\pgfqpoint{3.218551in}{3.155086in}}%
\pgfpathlineto{\pgfqpoint{3.226463in}{3.170127in}}%
\pgfpathlineto{\pgfqpoint{3.234368in}{3.185353in}}%
\pgfpathlineto{\pgfqpoint{3.242267in}{3.200766in}}%
\pgfpathlineto{\pgfqpoint{3.228853in}{3.218867in}}%
\pgfpathlineto{\pgfqpoint{3.215432in}{3.237274in}}%
\pgfpathlineto{\pgfqpoint{3.202006in}{3.255989in}}%
\pgfpathlineto{\pgfqpoint{3.188572in}{3.275015in}}%
\pgfpathlineto{\pgfqpoint{3.180665in}{3.259333in}}%
\pgfpathlineto{\pgfqpoint{3.172751in}{3.243846in}}%
\pgfpathlineto{\pgfqpoint{3.164830in}{3.228551in}}%
\pgfpathlineto{\pgfqpoint{3.156903in}{3.213446in}}%
\pgfpathclose%
\pgfusepath{fill}%
\end{pgfscope}%
\begin{pgfscope}%
\pgfpathrectangle{\pgfqpoint{1.150000in}{0.150000in}}{\pgfqpoint{5.700000in}{5.700000in}}%
\pgfusepath{clip}%
\pgfsetbuttcap%
\pgfsetroundjoin%
\definecolor{currentfill}{rgb}{0.257322,0.256130,0.526563}%
\pgfsetfillcolor{currentfill}%
\pgfsetfillopacity{0.800000}%
\pgfsetlinewidth{0.000000pt}%
\definecolor{currentstroke}{rgb}{0.000000,0.000000,0.000000}%
\pgfsetstrokecolor{currentstroke}%
\pgfsetdash{}{0pt}%
\pgfpathmoveto{\pgfqpoint{4.199060in}{2.843045in}}%
\pgfpathlineto{\pgfqpoint{4.212467in}{2.838188in}}%
\pgfpathlineto{\pgfqpoint{4.225881in}{2.833536in}}%
\pgfpathlineto{\pgfqpoint{4.239300in}{2.829088in}}%
\pgfpathlineto{\pgfqpoint{4.252726in}{2.824842in}}%
\pgfpathlineto{\pgfqpoint{4.260411in}{2.838204in}}%
\pgfpathlineto{\pgfqpoint{4.268093in}{2.851720in}}%
\pgfpathlineto{\pgfqpoint{4.275771in}{2.865393in}}%
\pgfpathlineto{\pgfqpoint{4.283447in}{2.879230in}}%
\pgfpathlineto{\pgfqpoint{4.270030in}{2.883914in}}%
\pgfpathlineto{\pgfqpoint{4.256619in}{2.888801in}}%
\pgfpathlineto{\pgfqpoint{4.243214in}{2.893891in}}%
\pgfpathlineto{\pgfqpoint{4.229814in}{2.899186in}}%
\pgfpathlineto{\pgfqpoint{4.222130in}{2.884899in}}%
\pgfpathlineto{\pgfqpoint{4.214444in}{2.870784in}}%
\pgfpathlineto{\pgfqpoint{4.206753in}{2.856834in}}%
\pgfpathlineto{\pgfqpoint{4.199060in}{2.843045in}}%
\pgfpathclose%
\pgfusepath{fill}%
\end{pgfscope}%
\begin{pgfscope}%
\pgfpathrectangle{\pgfqpoint{1.150000in}{0.150000in}}{\pgfqpoint{5.700000in}{5.700000in}}%
\pgfusepath{clip}%
\pgfsetbuttcap%
\pgfsetroundjoin%
\definecolor{currentfill}{rgb}{0.266580,0.228262,0.514349}%
\pgfsetfillcolor{currentfill}%
\pgfsetfillopacity{0.800000}%
\pgfsetlinewidth{0.000000pt}%
\definecolor{currentstroke}{rgb}{0.000000,0.000000,0.000000}%
\pgfsetstrokecolor{currentstroke}%
\pgfsetdash{}{0pt}%
\pgfpathmoveto{\pgfqpoint{3.892246in}{2.776608in}}%
\pgfpathlineto{\pgfqpoint{3.905601in}{2.769476in}}%
\pgfpathlineto{\pgfqpoint{3.918960in}{2.762564in}}%
\pgfpathlineto{\pgfqpoint{3.932323in}{2.755872in}}%
\pgfpathlineto{\pgfqpoint{3.945689in}{2.749399in}}%
\pgfpathlineto{\pgfqpoint{3.953452in}{2.762627in}}%
\pgfpathlineto{\pgfqpoint{3.961211in}{2.775985in}}%
\pgfpathlineto{\pgfqpoint{3.968965in}{2.789476in}}%
\pgfpathlineto{\pgfqpoint{3.976715in}{2.803104in}}%
\pgfpathlineto{\pgfqpoint{3.963356in}{2.809922in}}%
\pgfpathlineto{\pgfqpoint{3.950001in}{2.816959in}}%
\pgfpathlineto{\pgfqpoint{3.936650in}{2.824216in}}%
\pgfpathlineto{\pgfqpoint{3.923301in}{2.831693in}}%
\pgfpathlineto{\pgfqpoint{3.915544in}{2.817708in}}%
\pgfpathlineto{\pgfqpoint{3.907782in}{2.803868in}}%
\pgfpathlineto{\pgfqpoint{3.900016in}{2.790170in}}%
\pgfpathlineto{\pgfqpoint{3.892246in}{2.776608in}}%
\pgfpathclose%
\pgfusepath{fill}%
\end{pgfscope}%
\begin{pgfscope}%
\pgfpathrectangle{\pgfqpoint{1.150000in}{0.150000in}}{\pgfqpoint{5.700000in}{5.700000in}}%
\pgfusepath{clip}%
\pgfsetbuttcap%
\pgfsetroundjoin%
\definecolor{currentfill}{rgb}{0.185556,0.418570,0.556753}%
\pgfsetfillcolor{currentfill}%
\pgfsetfillopacity{0.800000}%
\pgfsetlinewidth{0.000000pt}%
\definecolor{currentstroke}{rgb}{0.000000,0.000000,0.000000}%
\pgfsetstrokecolor{currentstroke}%
\pgfsetdash{}{0pt}%
\pgfpathmoveto{\pgfqpoint{4.958831in}{3.248558in}}%
\pgfpathlineto{\pgfqpoint{4.972419in}{3.245529in}}%
\pgfpathlineto{\pgfqpoint{4.986016in}{3.242682in}}%
\pgfpathlineto{\pgfqpoint{4.999623in}{3.240014in}}%
\pgfpathlineto{\pgfqpoint{5.013239in}{3.237526in}}%
\pgfpathlineto{\pgfqpoint{5.020788in}{3.253884in}}%
\pgfpathlineto{\pgfqpoint{5.028339in}{3.270565in}}%
\pgfpathlineto{\pgfqpoint{5.035894in}{3.287578in}}%
\pgfpathlineto{\pgfqpoint{5.043452in}{3.304931in}}%
\pgfpathlineto{\pgfqpoint{5.029851in}{3.308141in}}%
\pgfpathlineto{\pgfqpoint{5.016260in}{3.311529in}}%
\pgfpathlineto{\pgfqpoint{5.002679in}{3.315098in}}%
\pgfpathlineto{\pgfqpoint{4.989106in}{3.318848in}}%
\pgfpathlineto{\pgfqpoint{4.981532in}{3.300762in}}%
\pgfpathlineto{\pgfqpoint{4.973962in}{3.283024in}}%
\pgfpathlineto{\pgfqpoint{4.966395in}{3.265625in}}%
\pgfpathlineto{\pgfqpoint{4.958831in}{3.248558in}}%
\pgfpathclose%
\pgfusepath{fill}%
\end{pgfscope}%
\begin{pgfscope}%
\pgfpathrectangle{\pgfqpoint{1.150000in}{0.150000in}}{\pgfqpoint{5.700000in}{5.700000in}}%
\pgfusepath{clip}%
\pgfsetbuttcap%
\pgfsetroundjoin%
\definecolor{currentfill}{rgb}{0.262138,0.242286,0.520837}%
\pgfsetfillcolor{currentfill}%
\pgfsetfillopacity{0.800000}%
\pgfsetlinewidth{0.000000pt}%
\definecolor{currentstroke}{rgb}{0.000000,0.000000,0.000000}%
\pgfsetstrokecolor{currentstroke}%
\pgfsetdash{}{0pt}%
\pgfpathmoveto{\pgfqpoint{4.114646in}{2.809258in}}%
\pgfpathlineto{\pgfqpoint{4.128039in}{2.803981in}}%
\pgfpathlineto{\pgfqpoint{4.141437in}{2.798913in}}%
\pgfpathlineto{\pgfqpoint{4.154841in}{2.794052in}}%
\pgfpathlineto{\pgfqpoint{4.168251in}{2.789397in}}%
\pgfpathlineto{\pgfqpoint{4.175959in}{2.802593in}}%
\pgfpathlineto{\pgfqpoint{4.183663in}{2.815929in}}%
\pgfpathlineto{\pgfqpoint{4.191363in}{2.829411in}}%
\pgfpathlineto{\pgfqpoint{4.199060in}{2.843045in}}%
\pgfpathlineto{\pgfqpoint{4.185658in}{2.848107in}}%
\pgfpathlineto{\pgfqpoint{4.172262in}{2.853375in}}%
\pgfpathlineto{\pgfqpoint{4.158872in}{2.858850in}}%
\pgfpathlineto{\pgfqpoint{4.145486in}{2.864534in}}%
\pgfpathlineto{\pgfqpoint{4.137782in}{2.850482in}}%
\pgfpathlineto{\pgfqpoint{4.130073in}{2.836588in}}%
\pgfpathlineto{\pgfqpoint{4.122361in}{2.822849in}}%
\pgfpathlineto{\pgfqpoint{4.114646in}{2.809258in}}%
\pgfpathclose%
\pgfusepath{fill}%
\end{pgfscope}%
\begin{pgfscope}%
\pgfpathrectangle{\pgfqpoint{1.150000in}{0.150000in}}{\pgfqpoint{5.700000in}{5.700000in}}%
\pgfusepath{clip}%
\pgfsetbuttcap%
\pgfsetroundjoin%
\definecolor{currentfill}{rgb}{0.258965,0.251537,0.524736}%
\pgfsetfillcolor{currentfill}%
\pgfsetfillopacity{0.800000}%
\pgfsetlinewidth{0.000000pt}%
\definecolor{currentstroke}{rgb}{0.000000,0.000000,0.000000}%
\pgfsetstrokecolor{currentstroke}%
\pgfsetdash{}{0pt}%
\pgfpathmoveto{\pgfqpoint{3.478162in}{2.843610in}}%
\pgfpathlineto{\pgfqpoint{3.491510in}{2.831621in}}%
\pgfpathlineto{\pgfqpoint{3.504857in}{2.819887in}}%
\pgfpathlineto{\pgfqpoint{3.518204in}{2.808408in}}%
\pgfpathlineto{\pgfqpoint{3.531550in}{2.797181in}}%
\pgfpathlineto{\pgfqpoint{3.539416in}{2.810590in}}%
\pgfpathlineto{\pgfqpoint{3.547277in}{2.824132in}}%
\pgfpathlineto{\pgfqpoint{3.555133in}{2.837810in}}%
\pgfpathlineto{\pgfqpoint{3.562984in}{2.851627in}}%
\pgfpathlineto{\pgfqpoint{3.549646in}{2.863107in}}%
\pgfpathlineto{\pgfqpoint{3.536307in}{2.874838in}}%
\pgfpathlineto{\pgfqpoint{3.522968in}{2.886824in}}%
\pgfpathlineto{\pgfqpoint{3.509628in}{2.899066in}}%
\pgfpathlineto{\pgfqpoint{3.501769in}{2.884984in}}%
\pgfpathlineto{\pgfqpoint{3.493906in}{2.871050in}}%
\pgfpathlineto{\pgfqpoint{3.486036in}{2.857260in}}%
\pgfpathlineto{\pgfqpoint{3.478162in}{2.843610in}}%
\pgfpathclose%
\pgfusepath{fill}%
\end{pgfscope}%
\begin{pgfscope}%
\pgfpathrectangle{\pgfqpoint{1.150000in}{0.150000in}}{\pgfqpoint{5.700000in}{5.700000in}}%
\pgfusepath{clip}%
\pgfsetbuttcap%
\pgfsetroundjoin%
\definecolor{currentfill}{rgb}{0.993248,0.906157,0.143936}%
\pgfsetfillcolor{currentfill}%
\pgfsetfillopacity{0.800000}%
\pgfsetlinewidth{0.000000pt}%
\definecolor{currentstroke}{rgb}{0.000000,0.000000,0.000000}%
\pgfsetstrokecolor{currentstroke}%
\pgfsetdash{}{0pt}%
\pgfpathmoveto{\pgfqpoint{3.490972in}{5.105273in}}%
\pgfpathlineto{\pgfqpoint{3.504521in}{5.074000in}}%
\pgfpathlineto{\pgfqpoint{3.518057in}{5.043097in}}%
\pgfpathlineto{\pgfqpoint{3.531583in}{5.012560in}}%
\pgfpathlineto{\pgfqpoint{3.545098in}{4.982386in}}%
\pgfpathlineto{\pgfqpoint{3.552698in}{5.021704in}}%
\pgfpathlineto{\pgfqpoint{3.560296in}{5.061641in}}%
\pgfpathlineto{\pgfqpoint{3.567892in}{5.102207in}}%
\pgfpathlineto{\pgfqpoint{3.554367in}{5.133030in}}%
\pgfpathlineto{\pgfqpoint{3.540832in}{5.164217in}}%
\pgfpathlineto{\pgfqpoint{3.527286in}{5.195773in}}%
\pgfpathlineto{\pgfqpoint{3.513728in}{5.227701in}}%
\pgfpathlineto{\pgfqpoint{3.506145in}{5.186254in}}%
\pgfpathlineto{\pgfqpoint{3.498560in}{5.145448in}}%
\pgfpathlineto{\pgfqpoint{3.490972in}{5.105273in}}%
\pgfpathclose%
\pgfusepath{fill}%
\end{pgfscope}%
\begin{pgfscope}%
\pgfpathrectangle{\pgfqpoint{1.150000in}{0.150000in}}{\pgfqpoint{5.700000in}{5.700000in}}%
\pgfusepath{clip}%
\pgfsetbuttcap%
\pgfsetroundjoin%
\definecolor{currentfill}{rgb}{0.185556,0.418570,0.556753}%
\pgfsetfillcolor{currentfill}%
\pgfsetfillopacity{0.800000}%
\pgfsetlinewidth{0.000000pt}%
\definecolor{currentstroke}{rgb}{0.000000,0.000000,0.000000}%
\pgfsetstrokecolor{currentstroke}%
\pgfsetdash{}{0pt}%
\pgfpathmoveto{\pgfqpoint{3.103059in}{3.291687in}}%
\pgfpathlineto{\pgfqpoint{3.116531in}{3.271646in}}%
\pgfpathlineto{\pgfqpoint{3.129996in}{3.251927in}}%
\pgfpathlineto{\pgfqpoint{3.143453in}{3.232528in}}%
\pgfpathlineto{\pgfqpoint{3.156903in}{3.213446in}}%
\pgfpathlineto{\pgfqpoint{3.164830in}{3.228551in}}%
\pgfpathlineto{\pgfqpoint{3.172751in}{3.243846in}}%
\pgfpathlineto{\pgfqpoint{3.180665in}{3.259333in}}%
\pgfpathlineto{\pgfqpoint{3.188572in}{3.275015in}}%
\pgfpathlineto{\pgfqpoint{3.175131in}{3.294355in}}%
\pgfpathlineto{\pgfqpoint{3.161684in}{3.314012in}}%
\pgfpathlineto{\pgfqpoint{3.148228in}{3.333989in}}%
\pgfpathlineto{\pgfqpoint{3.134765in}{3.354289in}}%
\pgfpathlineto{\pgfqpoint{3.126849in}{3.338336in}}%
\pgfpathlineto{\pgfqpoint{3.118926in}{3.322587in}}%
\pgfpathlineto{\pgfqpoint{3.110996in}{3.307038in}}%
\pgfpathlineto{\pgfqpoint{3.103059in}{3.291687in}}%
\pgfpathclose%
\pgfusepath{fill}%
\end{pgfscope}%
\begin{pgfscope}%
\pgfpathrectangle{\pgfqpoint{1.150000in}{0.150000in}}{\pgfqpoint{5.700000in}{5.700000in}}%
\pgfusepath{clip}%
\pgfsetbuttcap%
\pgfsetroundjoin%
\definecolor{currentfill}{rgb}{0.151918,0.500685,0.557587}%
\pgfsetfillcolor{currentfill}%
\pgfsetfillopacity{0.800000}%
\pgfsetlinewidth{0.000000pt}%
\definecolor{currentstroke}{rgb}{0.000000,0.000000,0.000000}%
\pgfsetstrokecolor{currentstroke}%
\pgfsetdash{}{0pt}%
\pgfpathmoveto{\pgfqpoint{3.026738in}{3.528697in}}%
\pgfpathlineto{\pgfqpoint{3.040276in}{3.505698in}}%
\pgfpathlineto{\pgfqpoint{3.053803in}{3.483047in}}%
\pgfpathlineto{\pgfqpoint{3.067319in}{3.460743in}}%
\pgfpathlineto{\pgfqpoint{3.080827in}{3.438781in}}%
\pgfpathlineto{\pgfqpoint{3.088744in}{3.455222in}}%
\pgfpathlineto{\pgfqpoint{3.096655in}{3.471882in}}%
\pgfpathlineto{\pgfqpoint{3.104559in}{3.488764in}}%
\pgfpathlineto{\pgfqpoint{3.112455in}{3.505872in}}%
\pgfpathlineto{\pgfqpoint{3.098956in}{3.528128in}}%
\pgfpathlineto{\pgfqpoint{3.085448in}{3.550728in}}%
\pgfpathlineto{\pgfqpoint{3.071929in}{3.573673in}}%
\pgfpathlineto{\pgfqpoint{3.058401in}{3.596968in}}%
\pgfpathlineto{\pgfqpoint{3.050496in}{3.579552in}}%
\pgfpathlineto{\pgfqpoint{3.042584in}{3.562371in}}%
\pgfpathlineto{\pgfqpoint{3.034665in}{3.545420in}}%
\pgfpathlineto{\pgfqpoint{3.026738in}{3.528697in}}%
\pgfpathclose%
\pgfusepath{fill}%
\end{pgfscope}%
\begin{pgfscope}%
\pgfpathrectangle{\pgfqpoint{1.150000in}{0.150000in}}{\pgfqpoint{5.700000in}{5.700000in}}%
\pgfusepath{clip}%
\pgfsetbuttcap%
\pgfsetroundjoin%
\definecolor{currentfill}{rgb}{0.267968,0.223549,0.512008}%
\pgfsetfillcolor{currentfill}%
\pgfsetfillopacity{0.800000}%
\pgfsetlinewidth{0.000000pt}%
\definecolor{currentstroke}{rgb}{0.000000,0.000000,0.000000}%
\pgfsetstrokecolor{currentstroke}%
\pgfsetdash{}{0pt}%
\pgfpathmoveto{\pgfqpoint{3.669690in}{2.768676in}}%
\pgfpathlineto{\pgfqpoint{3.683031in}{2.759393in}}%
\pgfpathlineto{\pgfqpoint{3.696374in}{2.750347in}}%
\pgfpathlineto{\pgfqpoint{3.709719in}{2.741536in}}%
\pgfpathlineto{\pgfqpoint{3.723065in}{2.732958in}}%
\pgfpathlineto{\pgfqpoint{3.730887in}{2.746104in}}%
\pgfpathlineto{\pgfqpoint{3.738705in}{2.759373in}}%
\pgfpathlineto{\pgfqpoint{3.746517in}{2.772767in}}%
\pgfpathlineto{\pgfqpoint{3.754325in}{2.786292in}}%
\pgfpathlineto{\pgfqpoint{3.740986in}{2.795152in}}%
\pgfpathlineto{\pgfqpoint{3.727649in}{2.804247in}}%
\pgfpathlineto{\pgfqpoint{3.714313in}{2.813576in}}%
\pgfpathlineto{\pgfqpoint{3.700979in}{2.823141in}}%
\pgfpathlineto{\pgfqpoint{3.693164in}{2.809322in}}%
\pgfpathlineto{\pgfqpoint{3.685344in}{2.795641in}}%
\pgfpathlineto{\pgfqpoint{3.677519in}{2.782093in}}%
\pgfpathlineto{\pgfqpoint{3.669690in}{2.768676in}}%
\pgfpathclose%
\pgfusepath{fill}%
\end{pgfscope}%
\begin{pgfscope}%
\pgfpathrectangle{\pgfqpoint{1.150000in}{0.150000in}}{\pgfqpoint{5.700000in}{5.700000in}}%
\pgfusepath{clip}%
\pgfsetbuttcap%
\pgfsetroundjoin%
\definecolor{currentfill}{rgb}{0.269308,0.218818,0.509577}%
\pgfsetfillcolor{currentfill}%
\pgfsetfillopacity{0.800000}%
\pgfsetlinewidth{0.000000pt}%
\definecolor{currentstroke}{rgb}{0.000000,0.000000,0.000000}%
\pgfsetstrokecolor{currentstroke}%
\pgfsetdash{}{0pt}%
\pgfpathmoveto{\pgfqpoint{3.807701in}{2.753157in}}%
\pgfpathlineto{\pgfqpoint{3.821052in}{2.745444in}}%
\pgfpathlineto{\pgfqpoint{3.834405in}{2.737956in}}%
\pgfpathlineto{\pgfqpoint{3.847761in}{2.730693in}}%
\pgfpathlineto{\pgfqpoint{3.861121in}{2.723652in}}%
\pgfpathlineto{\pgfqpoint{3.868909in}{2.736705in}}%
\pgfpathlineto{\pgfqpoint{3.876692in}{2.749880in}}%
\pgfpathlineto{\pgfqpoint{3.884471in}{2.763180in}}%
\pgfpathlineto{\pgfqpoint{3.892246in}{2.776608in}}%
\pgfpathlineto{\pgfqpoint{3.878894in}{2.783963in}}%
\pgfpathlineto{\pgfqpoint{3.865545in}{2.791540in}}%
\pgfpathlineto{\pgfqpoint{3.852199in}{2.799341in}}%
\pgfpathlineto{\pgfqpoint{3.838856in}{2.807369in}}%
\pgfpathlineto{\pgfqpoint{3.831074in}{2.793614in}}%
\pgfpathlineto{\pgfqpoint{3.823288in}{2.779997in}}%
\pgfpathlineto{\pgfqpoint{3.815497in}{2.766513in}}%
\pgfpathlineto{\pgfqpoint{3.807701in}{2.753157in}}%
\pgfpathclose%
\pgfusepath{fill}%
\end{pgfscope}%
\begin{pgfscope}%
\pgfpathrectangle{\pgfqpoint{1.150000in}{0.150000in}}{\pgfqpoint{5.700000in}{5.700000in}}%
\pgfusepath{clip}%
\pgfsetbuttcap%
\pgfsetroundjoin%
\definecolor{currentfill}{rgb}{0.265145,0.232956,0.516599}%
\pgfsetfillcolor{currentfill}%
\pgfsetfillopacity{0.800000}%
\pgfsetlinewidth{0.000000pt}%
\definecolor{currentstroke}{rgb}{0.000000,0.000000,0.000000}%
\pgfsetstrokecolor{currentstroke}%
\pgfsetdash{}{0pt}%
\pgfpathmoveto{\pgfqpoint{4.030193in}{2.777993in}}%
\pgfpathlineto{\pgfqpoint{4.043574in}{2.772250in}}%
\pgfpathlineto{\pgfqpoint{4.056959in}{2.766719in}}%
\pgfpathlineto{\pgfqpoint{4.070350in}{2.761400in}}%
\pgfpathlineto{\pgfqpoint{4.083745in}{2.756291in}}%
\pgfpathlineto{\pgfqpoint{4.091476in}{2.769332in}}%
\pgfpathlineto{\pgfqpoint{4.099203in}{2.782504in}}%
\pgfpathlineto{\pgfqpoint{4.106927in}{2.795811in}}%
\pgfpathlineto{\pgfqpoint{4.114646in}{2.809258in}}%
\pgfpathlineto{\pgfqpoint{4.101258in}{2.814743in}}%
\pgfpathlineto{\pgfqpoint{4.087875in}{2.820439in}}%
\pgfpathlineto{\pgfqpoint{4.074498in}{2.826345in}}%
\pgfpathlineto{\pgfqpoint{4.061124in}{2.832464in}}%
\pgfpathlineto{\pgfqpoint{4.053397in}{2.818630in}}%
\pgfpathlineto{\pgfqpoint{4.045667in}{2.804943in}}%
\pgfpathlineto{\pgfqpoint{4.037932in}{2.791398in}}%
\pgfpathlineto{\pgfqpoint{4.030193in}{2.777993in}}%
\pgfpathclose%
\pgfusepath{fill}%
\end{pgfscope}%
\begin{pgfscope}%
\pgfpathrectangle{\pgfqpoint{1.150000in}{0.150000in}}{\pgfqpoint{5.700000in}{5.700000in}}%
\pgfusepath{clip}%
\pgfsetbuttcap%
\pgfsetroundjoin%
\definecolor{currentfill}{rgb}{0.177423,0.437527,0.557565}%
\pgfsetfillcolor{currentfill}%
\pgfsetfillopacity{0.800000}%
\pgfsetlinewidth{0.000000pt}%
\definecolor{currentstroke}{rgb}{0.000000,0.000000,0.000000}%
\pgfsetstrokecolor{currentstroke}%
\pgfsetdash{}{0pt}%
\pgfpathmoveto{\pgfqpoint{5.043452in}{3.304931in}}%
\pgfpathlineto{\pgfqpoint{5.057062in}{3.301901in}}%
\pgfpathlineto{\pgfqpoint{5.070681in}{3.299050in}}%
\pgfpathlineto{\pgfqpoint{5.084311in}{3.296377in}}%
\pgfpathlineto{\pgfqpoint{5.097950in}{3.293881in}}%
\pgfpathlineto{\pgfqpoint{5.105495in}{3.310842in}}%
\pgfpathlineto{\pgfqpoint{5.113044in}{3.328152in}}%
\pgfpathlineto{\pgfqpoint{5.120598in}{3.345820in}}%
\pgfpathlineto{\pgfqpoint{5.106972in}{3.348877in}}%
\pgfpathlineto{\pgfqpoint{5.093355in}{3.352112in}}%
\pgfpathlineto{\pgfqpoint{5.079748in}{3.355525in}}%
\pgfpathlineto{\pgfqpoint{5.066150in}{3.359117in}}%
\pgfpathlineto{\pgfqpoint{5.058579in}{3.340692in}}%
\pgfpathlineto{\pgfqpoint{5.051014in}{3.322633in}}%
\pgfpathlineto{\pgfqpoint{5.043452in}{3.304931in}}%
\pgfpathclose%
\pgfusepath{fill}%
\end{pgfscope}%
\begin{pgfscope}%
\pgfpathrectangle{\pgfqpoint{1.150000in}{0.150000in}}{\pgfqpoint{5.700000in}{5.700000in}}%
\pgfusepath{clip}%
\pgfsetbuttcap%
\pgfsetroundjoin%
\definecolor{currentfill}{rgb}{0.263663,0.237631,0.518762}%
\pgfsetfillcolor{currentfill}%
\pgfsetfillopacity{0.800000}%
\pgfsetlinewidth{0.000000pt}%
\definecolor{currentstroke}{rgb}{0.000000,0.000000,0.000000}%
\pgfsetstrokecolor{currentstroke}%
\pgfsetdash{}{0pt}%
\pgfpathmoveto{\pgfqpoint{3.531550in}{2.797181in}}%
\pgfpathlineto{\pgfqpoint{3.544895in}{2.786204in}}%
\pgfpathlineto{\pgfqpoint{3.558241in}{2.775477in}}%
\pgfpathlineto{\pgfqpoint{3.571586in}{2.764996in}}%
\pgfpathlineto{\pgfqpoint{3.584932in}{2.754761in}}%
\pgfpathlineto{\pgfqpoint{3.592790in}{2.767930in}}%
\pgfpathlineto{\pgfqpoint{3.600643in}{2.781223in}}%
\pgfpathlineto{\pgfqpoint{3.608491in}{2.794646in}}%
\pgfpathlineto{\pgfqpoint{3.616333in}{2.808199in}}%
\pgfpathlineto{\pgfqpoint{3.602996in}{2.818686in}}%
\pgfpathlineto{\pgfqpoint{3.589658in}{2.829419in}}%
\pgfpathlineto{\pgfqpoint{3.576321in}{2.840399in}}%
\pgfpathlineto{\pgfqpoint{3.562984in}{2.851627in}}%
\pgfpathlineto{\pgfqpoint{3.555133in}{2.837810in}}%
\pgfpathlineto{\pgfqpoint{3.547277in}{2.824132in}}%
\pgfpathlineto{\pgfqpoint{3.539416in}{2.810590in}}%
\pgfpathlineto{\pgfqpoint{3.531550in}{2.797181in}}%
\pgfpathclose%
\pgfusepath{fill}%
\end{pgfscope}%
\begin{pgfscope}%
\pgfpathrectangle{\pgfqpoint{1.150000in}{0.150000in}}{\pgfqpoint{5.700000in}{5.700000in}}%
\pgfusepath{clip}%
\pgfsetbuttcap%
\pgfsetroundjoin%
\definecolor{currentfill}{rgb}{0.229739,0.322361,0.545706}%
\pgfsetfillcolor{currentfill}%
\pgfsetfillopacity{0.800000}%
\pgfsetlinewidth{0.000000pt}%
\definecolor{currentstroke}{rgb}{0.000000,0.000000,0.000000}%
\pgfsetstrokecolor{currentstroke}%
\pgfsetdash{}{0pt}%
\pgfpathmoveto{\pgfqpoint{4.590518in}{2.987978in}}%
\pgfpathlineto{\pgfqpoint{4.604030in}{2.985120in}}%
\pgfpathlineto{\pgfqpoint{4.617549in}{2.982452in}}%
\pgfpathlineto{\pgfqpoint{4.631078in}{2.979973in}}%
\pgfpathlineto{\pgfqpoint{4.644615in}{2.977682in}}%
\pgfpathlineto{\pgfqpoint{4.652210in}{2.991398in}}%
\pgfpathlineto{\pgfqpoint{4.659804in}{3.005322in}}%
\pgfpathlineto{\pgfqpoint{4.667396in}{3.019459in}}%
\pgfpathlineto{\pgfqpoint{4.674988in}{3.033815in}}%
\pgfpathlineto{\pgfqpoint{4.661462in}{3.036670in}}%
\pgfpathlineto{\pgfqpoint{4.647946in}{3.039714in}}%
\pgfpathlineto{\pgfqpoint{4.634437in}{3.042946in}}%
\pgfpathlineto{\pgfqpoint{4.620937in}{3.046368in}}%
\pgfpathlineto{\pgfqpoint{4.613334in}{3.031436in}}%
\pgfpathlineto{\pgfqpoint{4.605730in}{3.016731in}}%
\pgfpathlineto{\pgfqpoint{4.598125in}{3.002247in}}%
\pgfpathlineto{\pgfqpoint{4.590518in}{2.987978in}}%
\pgfpathclose%
\pgfusepath{fill}%
\end{pgfscope}%
\begin{pgfscope}%
\pgfpathrectangle{\pgfqpoint{1.150000in}{0.150000in}}{\pgfqpoint{5.700000in}{5.700000in}}%
\pgfusepath{clip}%
\pgfsetbuttcap%
\pgfsetroundjoin%
\definecolor{currentfill}{rgb}{0.237441,0.305202,0.541921}%
\pgfsetfillcolor{currentfill}%
\pgfsetfillopacity{0.800000}%
\pgfsetlinewidth{0.000000pt}%
\definecolor{currentstroke}{rgb}{0.000000,0.000000,0.000000}%
\pgfsetstrokecolor{currentstroke}%
\pgfsetdash{}{0pt}%
\pgfpathmoveto{\pgfqpoint{4.506067in}{2.944130in}}%
\pgfpathlineto{\pgfqpoint{4.519557in}{2.941040in}}%
\pgfpathlineto{\pgfqpoint{4.533055in}{2.938141in}}%
\pgfpathlineto{\pgfqpoint{4.546561in}{2.935435in}}%
\pgfpathlineto{\pgfqpoint{4.560075in}{2.932919in}}%
\pgfpathlineto{\pgfqpoint{4.567688in}{2.946394in}}%
\pgfpathlineto{\pgfqpoint{4.575300in}{2.960058in}}%
\pgfpathlineto{\pgfqpoint{4.582910in}{2.973917in}}%
\pgfpathlineto{\pgfqpoint{4.590518in}{2.987978in}}%
\pgfpathlineto{\pgfqpoint{4.577015in}{2.991027in}}%
\pgfpathlineto{\pgfqpoint{4.563520in}{2.994266in}}%
\pgfpathlineto{\pgfqpoint{4.550032in}{2.997696in}}%
\pgfpathlineto{\pgfqpoint{4.536553in}{3.001319in}}%
\pgfpathlineto{\pgfqpoint{4.528934in}{2.986714in}}%
\pgfpathlineto{\pgfqpoint{4.521313in}{2.972318in}}%
\pgfpathlineto{\pgfqpoint{4.513691in}{2.958126in}}%
\pgfpathlineto{\pgfqpoint{4.506067in}{2.944130in}}%
\pgfpathclose%
\pgfusepath{fill}%
\end{pgfscope}%
\begin{pgfscope}%
\pgfpathrectangle{\pgfqpoint{1.150000in}{0.150000in}}{\pgfqpoint{5.700000in}{5.700000in}}%
\pgfusepath{clip}%
\pgfsetbuttcap%
\pgfsetroundjoin%
\definecolor{currentfill}{rgb}{0.221989,0.339161,0.548752}%
\pgfsetfillcolor{currentfill}%
\pgfsetfillopacity{0.800000}%
\pgfsetlinewidth{0.000000pt}%
\definecolor{currentstroke}{rgb}{0.000000,0.000000,0.000000}%
\pgfsetstrokecolor{currentstroke}%
\pgfsetdash{}{0pt}%
\pgfpathmoveto{\pgfqpoint{4.674988in}{3.033815in}}%
\pgfpathlineto{\pgfqpoint{4.688521in}{3.031148in}}%
\pgfpathlineto{\pgfqpoint{4.702064in}{3.028668in}}%
\pgfpathlineto{\pgfqpoint{4.715615in}{3.026375in}}%
\pgfpathlineto{\pgfqpoint{4.729175in}{3.024268in}}%
\pgfpathlineto{\pgfqpoint{4.736753in}{3.038267in}}%
\pgfpathlineto{\pgfqpoint{4.744331in}{3.052492in}}%
\pgfpathlineto{\pgfqpoint{4.751908in}{3.066949in}}%
\pgfpathlineto{\pgfqpoint{4.759484in}{3.081646in}}%
\pgfpathlineto{\pgfqpoint{4.745937in}{3.084349in}}%
\pgfpathlineto{\pgfqpoint{4.732398in}{3.087238in}}%
\pgfpathlineto{\pgfqpoint{4.718868in}{3.090313in}}%
\pgfpathlineto{\pgfqpoint{4.705346in}{3.093576in}}%
\pgfpathlineto{\pgfqpoint{4.697757in}{3.078272in}}%
\pgfpathlineto{\pgfqpoint{4.690168in}{3.063215in}}%
\pgfpathlineto{\pgfqpoint{4.682578in}{3.048399in}}%
\pgfpathlineto{\pgfqpoint{4.674988in}{3.033815in}}%
\pgfpathclose%
\pgfusepath{fill}%
\end{pgfscope}%
\begin{pgfscope}%
\pgfpathrectangle{\pgfqpoint{1.150000in}{0.150000in}}{\pgfqpoint{5.700000in}{5.700000in}}%
\pgfusepath{clip}%
\pgfsetbuttcap%
\pgfsetroundjoin%
\definecolor{currentfill}{rgb}{0.171176,0.452530,0.557965}%
\pgfsetfillcolor{currentfill}%
\pgfsetfillopacity{0.800000}%
\pgfsetlinewidth{0.000000pt}%
\definecolor{currentstroke}{rgb}{0.000000,0.000000,0.000000}%
\pgfsetstrokecolor{currentstroke}%
\pgfsetdash{}{0pt}%
\pgfpathmoveto{\pgfqpoint{3.049083in}{3.375139in}}%
\pgfpathlineto{\pgfqpoint{3.062591in}{3.353777in}}%
\pgfpathlineto{\pgfqpoint{3.076089in}{3.332749in}}%
\pgfpathlineto{\pgfqpoint{3.089578in}{3.312053in}}%
\pgfpathlineto{\pgfqpoint{3.103059in}{3.291687in}}%
\pgfpathlineto{\pgfqpoint{3.110996in}{3.307038in}}%
\pgfpathlineto{\pgfqpoint{3.118926in}{3.322587in}}%
\pgfpathlineto{\pgfqpoint{3.126849in}{3.338336in}}%
\pgfpathlineto{\pgfqpoint{3.134765in}{3.354289in}}%
\pgfpathlineto{\pgfqpoint{3.121293in}{3.374915in}}%
\pgfpathlineto{\pgfqpoint{3.107813in}{3.395870in}}%
\pgfpathlineto{\pgfqpoint{3.094325in}{3.417158in}}%
\pgfpathlineto{\pgfqpoint{3.080827in}{3.438781in}}%
\pgfpathlineto{\pgfqpoint{3.072902in}{3.422556in}}%
\pgfpathlineto{\pgfqpoint{3.064970in}{3.406543in}}%
\pgfpathlineto{\pgfqpoint{3.057030in}{3.390738in}}%
\pgfpathlineto{\pgfqpoint{3.049083in}{3.375139in}}%
\pgfpathclose%
\pgfusepath{fill}%
\end{pgfscope}%
\begin{pgfscope}%
\pgfpathrectangle{\pgfqpoint{1.150000in}{0.150000in}}{\pgfqpoint{5.700000in}{5.700000in}}%
\pgfusepath{clip}%
\pgfsetbuttcap%
\pgfsetroundjoin%
\definecolor{currentfill}{rgb}{0.244972,0.287675,0.537260}%
\pgfsetfillcolor{currentfill}%
\pgfsetfillopacity{0.800000}%
\pgfsetlinewidth{0.000000pt}%
\definecolor{currentstroke}{rgb}{0.000000,0.000000,0.000000}%
\pgfsetstrokecolor{currentstroke}%
\pgfsetdash{}{0pt}%
\pgfpathmoveto{\pgfqpoint{4.421624in}{2.902291in}}%
\pgfpathlineto{\pgfqpoint{4.435093in}{2.898925in}}%
\pgfpathlineto{\pgfqpoint{4.448570in}{2.895755in}}%
\pgfpathlineto{\pgfqpoint{4.462055in}{2.892780in}}%
\pgfpathlineto{\pgfqpoint{4.475547in}{2.889998in}}%
\pgfpathlineto{\pgfqpoint{4.483181in}{2.903265in}}%
\pgfpathlineto{\pgfqpoint{4.490812in}{2.916706in}}%
\pgfpathlineto{\pgfqpoint{4.498440in}{2.930325in}}%
\pgfpathlineto{\pgfqpoint{4.506067in}{2.944130in}}%
\pgfpathlineto{\pgfqpoint{4.492584in}{2.947413in}}%
\pgfpathlineto{\pgfqpoint{4.479110in}{2.950890in}}%
\pgfpathlineto{\pgfqpoint{4.465643in}{2.954561in}}%
\pgfpathlineto{\pgfqpoint{4.452183in}{2.958427in}}%
\pgfpathlineto{\pgfqpoint{4.444546in}{2.944110in}}%
\pgfpathlineto{\pgfqpoint{4.436908in}{2.929985in}}%
\pgfpathlineto{\pgfqpoint{4.429267in}{2.916047in}}%
\pgfpathlineto{\pgfqpoint{4.421624in}{2.902291in}}%
\pgfpathclose%
\pgfusepath{fill}%
\end{pgfscope}%
\begin{pgfscope}%
\pgfpathrectangle{\pgfqpoint{1.150000in}{0.150000in}}{\pgfqpoint{5.700000in}{5.700000in}}%
\pgfusepath{clip}%
\pgfsetbuttcap%
\pgfsetroundjoin%
\definecolor{currentfill}{rgb}{0.229739,0.322361,0.545706}%
\pgfsetfillcolor{currentfill}%
\pgfsetfillopacity{0.800000}%
\pgfsetlinewidth{0.000000pt}%
\definecolor{currentstroke}{rgb}{0.000000,0.000000,0.000000}%
\pgfsetstrokecolor{currentstroke}%
\pgfsetdash{}{0pt}%
\pgfpathmoveto{\pgfqpoint{3.232566in}{3.015047in}}%
\pgfpathlineto{\pgfqpoint{3.245971in}{2.998912in}}%
\pgfpathlineto{\pgfqpoint{3.259371in}{2.983067in}}%
\pgfpathlineto{\pgfqpoint{3.272767in}{2.967509in}}%
\pgfpathlineto{\pgfqpoint{3.286158in}{2.952236in}}%
\pgfpathlineto{\pgfqpoint{3.294083in}{2.965980in}}%
\pgfpathlineto{\pgfqpoint{3.302002in}{2.979878in}}%
\pgfpathlineto{\pgfqpoint{3.309914in}{2.993931in}}%
\pgfpathlineto{\pgfqpoint{3.317821in}{3.008143in}}%
\pgfpathlineto{\pgfqpoint{3.304439in}{3.023639in}}%
\pgfpathlineto{\pgfqpoint{3.291053in}{3.039420in}}%
\pgfpathlineto{\pgfqpoint{3.277662in}{3.055488in}}%
\pgfpathlineto{\pgfqpoint{3.264266in}{3.071846in}}%
\pgfpathlineto{\pgfqpoint{3.256351in}{3.057399in}}%
\pgfpathlineto{\pgfqpoint{3.248429in}{3.043118in}}%
\pgfpathlineto{\pgfqpoint{3.240501in}{3.029002in}}%
\pgfpathlineto{\pgfqpoint{3.232566in}{3.015047in}}%
\pgfpathclose%
\pgfusepath{fill}%
\end{pgfscope}%
\begin{pgfscope}%
\pgfpathrectangle{\pgfqpoint{1.150000in}{0.150000in}}{\pgfqpoint{5.700000in}{5.700000in}}%
\pgfusepath{clip}%
\pgfsetbuttcap%
\pgfsetroundjoin%
\definecolor{currentfill}{rgb}{0.239346,0.300855,0.540844}%
\pgfsetfillcolor{currentfill}%
\pgfsetfillopacity{0.800000}%
\pgfsetlinewidth{0.000000pt}%
\definecolor{currentstroke}{rgb}{0.000000,0.000000,0.000000}%
\pgfsetstrokecolor{currentstroke}%
\pgfsetdash{}{0pt}%
\pgfpathmoveto{\pgfqpoint{3.286158in}{2.952236in}}%
\pgfpathlineto{\pgfqpoint{3.299545in}{2.937245in}}%
\pgfpathlineto{\pgfqpoint{3.312929in}{2.922534in}}%
\pgfpathlineto{\pgfqpoint{3.326309in}{2.908102in}}%
\pgfpathlineto{\pgfqpoint{3.339686in}{2.893945in}}%
\pgfpathlineto{\pgfqpoint{3.347601in}{2.907479in}}%
\pgfpathlineto{\pgfqpoint{3.355510in}{2.921158in}}%
\pgfpathlineto{\pgfqpoint{3.363414in}{2.934985in}}%
\pgfpathlineto{\pgfqpoint{3.371311in}{2.948963in}}%
\pgfpathlineto{\pgfqpoint{3.357944in}{2.963342in}}%
\pgfpathlineto{\pgfqpoint{3.344573in}{2.977997in}}%
\pgfpathlineto{\pgfqpoint{3.331199in}{2.992930in}}%
\pgfpathlineto{\pgfqpoint{3.317821in}{3.008143in}}%
\pgfpathlineto{\pgfqpoint{3.309914in}{2.993931in}}%
\pgfpathlineto{\pgfqpoint{3.302002in}{2.979878in}}%
\pgfpathlineto{\pgfqpoint{3.294083in}{2.965980in}}%
\pgfpathlineto{\pgfqpoint{3.286158in}{2.952236in}}%
\pgfpathclose%
\pgfusepath{fill}%
\end{pgfscope}%
\begin{pgfscope}%
\pgfpathrectangle{\pgfqpoint{1.150000in}{0.150000in}}{\pgfqpoint{5.700000in}{5.700000in}}%
\pgfusepath{clip}%
\pgfsetbuttcap%
\pgfsetroundjoin%
\definecolor{currentfill}{rgb}{0.212395,0.359683,0.551710}%
\pgfsetfillcolor{currentfill}%
\pgfsetfillopacity{0.800000}%
\pgfsetlinewidth{0.000000pt}%
\definecolor{currentstroke}{rgb}{0.000000,0.000000,0.000000}%
\pgfsetstrokecolor{currentstroke}%
\pgfsetdash{}{0pt}%
\pgfpathmoveto{\pgfqpoint{4.759484in}{3.081646in}}%
\pgfpathlineto{\pgfqpoint{4.773041in}{3.079129in}}%
\pgfpathlineto{\pgfqpoint{4.786606in}{3.076797in}}%
\pgfpathlineto{\pgfqpoint{4.800181in}{3.074649in}}%
\pgfpathlineto{\pgfqpoint{4.813765in}{3.072685in}}%
\pgfpathlineto{\pgfqpoint{4.821328in}{3.087013in}}%
\pgfpathlineto{\pgfqpoint{4.828891in}{3.101587in}}%
\pgfpathlineto{\pgfqpoint{4.836454in}{3.116414in}}%
\pgfpathlineto{\pgfqpoint{4.844018in}{3.131501in}}%
\pgfpathlineto{\pgfqpoint{4.830448in}{3.134092in}}%
\pgfpathlineto{\pgfqpoint{4.816887in}{3.136867in}}%
\pgfpathlineto{\pgfqpoint{4.803334in}{3.139826in}}%
\pgfpathlineto{\pgfqpoint{4.789791in}{3.142970in}}%
\pgfpathlineto{\pgfqpoint{4.782214in}{3.127244in}}%
\pgfpathlineto{\pgfqpoint{4.774637in}{3.111787in}}%
\pgfpathlineto{\pgfqpoint{4.767061in}{3.096590in}}%
\pgfpathlineto{\pgfqpoint{4.759484in}{3.081646in}}%
\pgfpathclose%
\pgfusepath{fill}%
\end{pgfscope}%
\begin{pgfscope}%
\pgfpathrectangle{\pgfqpoint{1.150000in}{0.150000in}}{\pgfqpoint{5.700000in}{5.700000in}}%
\pgfusepath{clip}%
\pgfsetbuttcap%
\pgfsetroundjoin%
\definecolor{currentfill}{rgb}{0.252194,0.269783,0.531579}%
\pgfsetfillcolor{currentfill}%
\pgfsetfillopacity{0.800000}%
\pgfsetlinewidth{0.000000pt}%
\definecolor{currentstroke}{rgb}{0.000000,0.000000,0.000000}%
\pgfsetstrokecolor{currentstroke}%
\pgfsetdash{}{0pt}%
\pgfpathmoveto{\pgfqpoint{4.337180in}{2.862505in}}%
\pgfpathlineto{\pgfqpoint{4.350630in}{2.858822in}}%
\pgfpathlineto{\pgfqpoint{4.364087in}{2.855338in}}%
\pgfpathlineto{\pgfqpoint{4.377551in}{2.852050in}}%
\pgfpathlineto{\pgfqpoint{4.391023in}{2.848960in}}%
\pgfpathlineto{\pgfqpoint{4.398677in}{2.862049in}}%
\pgfpathlineto{\pgfqpoint{4.406329in}{2.875297in}}%
\pgfpathlineto{\pgfqpoint{4.413978in}{2.888709in}}%
\pgfpathlineto{\pgfqpoint{4.421624in}{2.902291in}}%
\pgfpathlineto{\pgfqpoint{4.408161in}{2.905851in}}%
\pgfpathlineto{\pgfqpoint{4.394706in}{2.909609in}}%
\pgfpathlineto{\pgfqpoint{4.381259in}{2.913563in}}%
\pgfpathlineto{\pgfqpoint{4.367818in}{2.917715in}}%
\pgfpathlineto{\pgfqpoint{4.360162in}{2.903652in}}%
\pgfpathlineto{\pgfqpoint{4.352504in}{2.889766in}}%
\pgfpathlineto{\pgfqpoint{4.344843in}{2.876052in}}%
\pgfpathlineto{\pgfqpoint{4.337180in}{2.862505in}}%
\pgfpathclose%
\pgfusepath{fill}%
\end{pgfscope}%
\begin{pgfscope}%
\pgfpathrectangle{\pgfqpoint{1.150000in}{0.150000in}}{\pgfqpoint{5.700000in}{5.700000in}}%
\pgfusepath{clip}%
\pgfsetbuttcap%
\pgfsetroundjoin%
\definecolor{currentfill}{rgb}{0.218130,0.347432,0.550038}%
\pgfsetfillcolor{currentfill}%
\pgfsetfillopacity{0.800000}%
\pgfsetlinewidth{0.000000pt}%
\definecolor{currentstroke}{rgb}{0.000000,0.000000,0.000000}%
\pgfsetstrokecolor{currentstroke}%
\pgfsetdash{}{0pt}%
\pgfpathmoveto{\pgfqpoint{3.178894in}{3.082531in}}%
\pgfpathlineto{\pgfqpoint{3.192320in}{3.065213in}}%
\pgfpathlineto{\pgfqpoint{3.205741in}{3.048195in}}%
\pgfpathlineto{\pgfqpoint{3.219156in}{3.031473in}}%
\pgfpathlineto{\pgfqpoint{3.232566in}{3.015047in}}%
\pgfpathlineto{\pgfqpoint{3.240501in}{3.029002in}}%
\pgfpathlineto{\pgfqpoint{3.248429in}{3.043118in}}%
\pgfpathlineto{\pgfqpoint{3.256351in}{3.057399in}}%
\pgfpathlineto{\pgfqpoint{3.264266in}{3.071846in}}%
\pgfpathlineto{\pgfqpoint{3.250866in}{3.088496in}}%
\pgfpathlineto{\pgfqpoint{3.237461in}{3.105441in}}%
\pgfpathlineto{\pgfqpoint{3.224049in}{3.122683in}}%
\pgfpathlineto{\pgfqpoint{3.210633in}{3.140225in}}%
\pgfpathlineto{\pgfqpoint{3.202708in}{3.125542in}}%
\pgfpathlineto{\pgfqpoint{3.194777in}{3.111034in}}%
\pgfpathlineto{\pgfqpoint{3.186839in}{3.096698in}}%
\pgfpathlineto{\pgfqpoint{3.178894in}{3.082531in}}%
\pgfpathclose%
\pgfusepath{fill}%
\end{pgfscope}%
\begin{pgfscope}%
\pgfpathrectangle{\pgfqpoint{1.150000in}{0.150000in}}{\pgfqpoint{5.700000in}{5.700000in}}%
\pgfusepath{clip}%
\pgfsetbuttcap%
\pgfsetroundjoin%
\definecolor{currentfill}{rgb}{0.267968,0.223549,0.512008}%
\pgfsetfillcolor{currentfill}%
\pgfsetfillopacity{0.800000}%
\pgfsetlinewidth{0.000000pt}%
\definecolor{currentstroke}{rgb}{0.000000,0.000000,0.000000}%
\pgfsetstrokecolor{currentstroke}%
\pgfsetdash{}{0pt}%
\pgfpathmoveto{\pgfqpoint{3.945689in}{2.749399in}}%
\pgfpathlineto{\pgfqpoint{3.959060in}{2.743143in}}%
\pgfpathlineto{\pgfqpoint{3.972435in}{2.737103in}}%
\pgfpathlineto{\pgfqpoint{3.985814in}{2.731278in}}%
\pgfpathlineto{\pgfqpoint{3.999197in}{2.725667in}}%
\pgfpathlineto{\pgfqpoint{4.006953in}{2.738562in}}%
\pgfpathlineto{\pgfqpoint{4.014704in}{2.751579in}}%
\pgfpathlineto{\pgfqpoint{4.022450in}{2.764721in}}%
\pgfpathlineto{\pgfqpoint{4.030193in}{2.777993in}}%
\pgfpathlineto{\pgfqpoint{4.016817in}{2.783948in}}%
\pgfpathlineto{\pgfqpoint{4.003446in}{2.790118in}}%
\pgfpathlineto{\pgfqpoint{3.990078in}{2.796503in}}%
\pgfpathlineto{\pgfqpoint{3.976715in}{2.803104in}}%
\pgfpathlineto{\pgfqpoint{3.968965in}{2.789476in}}%
\pgfpathlineto{\pgfqpoint{3.961211in}{2.775985in}}%
\pgfpathlineto{\pgfqpoint{3.953452in}{2.762627in}}%
\pgfpathlineto{\pgfqpoint{3.945689in}{2.749399in}}%
\pgfpathclose%
\pgfusepath{fill}%
\end{pgfscope}%
\begin{pgfscope}%
\pgfpathrectangle{\pgfqpoint{1.150000in}{0.150000in}}{\pgfqpoint{5.700000in}{5.700000in}}%
\pgfusepath{clip}%
\pgfsetbuttcap%
\pgfsetroundjoin%
\definecolor{currentfill}{rgb}{0.248629,0.278775,0.534556}%
\pgfsetfillcolor{currentfill}%
\pgfsetfillopacity{0.800000}%
\pgfsetlinewidth{0.000000pt}%
\definecolor{currentstroke}{rgb}{0.000000,0.000000,0.000000}%
\pgfsetstrokecolor{currentstroke}%
\pgfsetdash{}{0pt}%
\pgfpathmoveto{\pgfqpoint{3.339686in}{2.893945in}}%
\pgfpathlineto{\pgfqpoint{3.353059in}{2.880061in}}%
\pgfpathlineto{\pgfqpoint{3.366430in}{2.866449in}}%
\pgfpathlineto{\pgfqpoint{3.379798in}{2.853107in}}%
\pgfpathlineto{\pgfqpoint{3.393164in}{2.840031in}}%
\pgfpathlineto{\pgfqpoint{3.401070in}{2.853356in}}%
\pgfpathlineto{\pgfqpoint{3.408970in}{2.866817in}}%
\pgfpathlineto{\pgfqpoint{3.416865in}{2.880419in}}%
\pgfpathlineto{\pgfqpoint{3.424754in}{2.894163in}}%
\pgfpathlineto{\pgfqpoint{3.411397in}{2.907460in}}%
\pgfpathlineto{\pgfqpoint{3.398038in}{2.921024in}}%
\pgfpathlineto{\pgfqpoint{3.384676in}{2.934858in}}%
\pgfpathlineto{\pgfqpoint{3.371311in}{2.948963in}}%
\pgfpathlineto{\pgfqpoint{3.363414in}{2.934985in}}%
\pgfpathlineto{\pgfqpoint{3.355510in}{2.921158in}}%
\pgfpathlineto{\pgfqpoint{3.347601in}{2.907479in}}%
\pgfpathlineto{\pgfqpoint{3.339686in}{2.893945in}}%
\pgfpathclose%
\pgfusepath{fill}%
\end{pgfscope}%
\begin{pgfscope}%
\pgfpathrectangle{\pgfqpoint{1.150000in}{0.150000in}}{\pgfqpoint{5.700000in}{5.700000in}}%
\pgfusepath{clip}%
\pgfsetbuttcap%
\pgfsetroundjoin%
\definecolor{currentfill}{rgb}{0.203063,0.379716,0.553925}%
\pgfsetfillcolor{currentfill}%
\pgfsetfillopacity{0.800000}%
\pgfsetlinewidth{0.000000pt}%
\definecolor{currentstroke}{rgb}{0.000000,0.000000,0.000000}%
\pgfsetstrokecolor{currentstroke}%
\pgfsetdash{}{0pt}%
\pgfpathmoveto{\pgfqpoint{4.844018in}{3.131501in}}%
\pgfpathlineto{\pgfqpoint{4.857598in}{3.129093in}}%
\pgfpathlineto{\pgfqpoint{4.871186in}{3.126868in}}%
\pgfpathlineto{\pgfqpoint{4.884785in}{3.124825in}}%
\pgfpathlineto{\pgfqpoint{4.898393in}{3.122964in}}%
\pgfpathlineto{\pgfqpoint{4.905943in}{3.137673in}}%
\pgfpathlineto{\pgfqpoint{4.913494in}{3.152649in}}%
\pgfpathlineto{\pgfqpoint{4.921046in}{3.167900in}}%
\pgfpathlineto{\pgfqpoint{4.928599in}{3.183434in}}%
\pgfpathlineto{\pgfqpoint{4.915006in}{3.185954in}}%
\pgfpathlineto{\pgfqpoint{4.901422in}{3.188655in}}%
\pgfpathlineto{\pgfqpoint{4.887848in}{3.191539in}}%
\pgfpathlineto{\pgfqpoint{4.874282in}{3.194605in}}%
\pgfpathlineto{\pgfqpoint{4.866714in}{3.178401in}}%
\pgfpathlineto{\pgfqpoint{4.859148in}{3.162487in}}%
\pgfpathlineto{\pgfqpoint{4.851582in}{3.146856in}}%
\pgfpathlineto{\pgfqpoint{4.844018in}{3.131501in}}%
\pgfpathclose%
\pgfusepath{fill}%
\end{pgfscope}%
\begin{pgfscope}%
\pgfpathrectangle{\pgfqpoint{1.150000in}{0.150000in}}{\pgfqpoint{5.700000in}{5.700000in}}%
\pgfusepath{clip}%
\pgfsetbuttcap%
\pgfsetroundjoin%
\definecolor{currentfill}{rgb}{0.257322,0.256130,0.526563}%
\pgfsetfillcolor{currentfill}%
\pgfsetfillopacity{0.800000}%
\pgfsetlinewidth{0.000000pt}%
\definecolor{currentstroke}{rgb}{0.000000,0.000000,0.000000}%
\pgfsetstrokecolor{currentstroke}%
\pgfsetdash{}{0pt}%
\pgfpathmoveto{\pgfqpoint{4.252726in}{2.824842in}}%
\pgfpathlineto{\pgfqpoint{4.266158in}{2.820798in}}%
\pgfpathlineto{\pgfqpoint{4.279596in}{2.816956in}}%
\pgfpathlineto{\pgfqpoint{4.293041in}{2.813314in}}%
\pgfpathlineto{\pgfqpoint{4.306494in}{2.809871in}}%
\pgfpathlineto{\pgfqpoint{4.314170in}{2.822806in}}%
\pgfpathlineto{\pgfqpoint{4.321843in}{2.835887in}}%
\pgfpathlineto{\pgfqpoint{4.329513in}{2.849118in}}%
\pgfpathlineto{\pgfqpoint{4.337180in}{2.862505in}}%
\pgfpathlineto{\pgfqpoint{4.323737in}{2.866386in}}%
\pgfpathlineto{\pgfqpoint{4.310300in}{2.870467in}}%
\pgfpathlineto{\pgfqpoint{4.296870in}{2.874748in}}%
\pgfpathlineto{\pgfqpoint{4.283447in}{2.879230in}}%
\pgfpathlineto{\pgfqpoint{4.275771in}{2.865393in}}%
\pgfpathlineto{\pgfqpoint{4.268093in}{2.851720in}}%
\pgfpathlineto{\pgfqpoint{4.260411in}{2.838204in}}%
\pgfpathlineto{\pgfqpoint{4.252726in}{2.824842in}}%
\pgfpathclose%
\pgfusepath{fill}%
\end{pgfscope}%
\begin{pgfscope}%
\pgfpathrectangle{\pgfqpoint{1.150000in}{0.150000in}}{\pgfqpoint{5.700000in}{5.700000in}}%
\pgfusepath{clip}%
\pgfsetbuttcap%
\pgfsetroundjoin%
\definecolor{currentfill}{rgb}{0.270595,0.214069,0.507052}%
\pgfsetfillcolor{currentfill}%
\pgfsetfillopacity{0.800000}%
\pgfsetlinewidth{0.000000pt}%
\definecolor{currentstroke}{rgb}{0.000000,0.000000,0.000000}%
\pgfsetstrokecolor{currentstroke}%
\pgfsetdash{}{0pt}%
\pgfpathmoveto{\pgfqpoint{3.723065in}{2.732958in}}%
\pgfpathlineto{\pgfqpoint{3.736414in}{2.724612in}}%
\pgfpathlineto{\pgfqpoint{3.749764in}{2.716497in}}%
\pgfpathlineto{\pgfqpoint{3.763117in}{2.708612in}}%
\pgfpathlineto{\pgfqpoint{3.776472in}{2.700954in}}%
\pgfpathlineto{\pgfqpoint{3.784287in}{2.713829in}}%
\pgfpathlineto{\pgfqpoint{3.792096in}{2.726819in}}%
\pgfpathlineto{\pgfqpoint{3.799901in}{2.739927in}}%
\pgfpathlineto{\pgfqpoint{3.807701in}{2.753157in}}%
\pgfpathlineto{\pgfqpoint{3.794354in}{2.761097in}}%
\pgfpathlineto{\pgfqpoint{3.781008in}{2.769266in}}%
\pgfpathlineto{\pgfqpoint{3.767666in}{2.777663in}}%
\pgfpathlineto{\pgfqpoint{3.754325in}{2.786292in}}%
\pgfpathlineto{\pgfqpoint{3.746517in}{2.772767in}}%
\pgfpathlineto{\pgfqpoint{3.738705in}{2.759373in}}%
\pgfpathlineto{\pgfqpoint{3.730887in}{2.746104in}}%
\pgfpathlineto{\pgfqpoint{3.723065in}{2.732958in}}%
\pgfpathclose%
\pgfusepath{fill}%
\end{pgfscope}%
\begin{pgfscope}%
\pgfpathrectangle{\pgfqpoint{1.150000in}{0.150000in}}{\pgfqpoint{5.700000in}{5.700000in}}%
\pgfusepath{clip}%
\pgfsetbuttcap%
\pgfsetroundjoin%
\definecolor{currentfill}{rgb}{0.204903,0.375746,0.553533}%
\pgfsetfillcolor{currentfill}%
\pgfsetfillopacity{0.800000}%
\pgfsetlinewidth{0.000000pt}%
\definecolor{currentstroke}{rgb}{0.000000,0.000000,0.000000}%
\pgfsetstrokecolor{currentstroke}%
\pgfsetdash{}{0pt}%
\pgfpathmoveto{\pgfqpoint{3.125124in}{3.154854in}}%
\pgfpathlineto{\pgfqpoint{3.138577in}{3.136310in}}%
\pgfpathlineto{\pgfqpoint{3.152022in}{3.118077in}}%
\pgfpathlineto{\pgfqpoint{3.165461in}{3.100152in}}%
\pgfpathlineto{\pgfqpoint{3.178894in}{3.082531in}}%
\pgfpathlineto{\pgfqpoint{3.186839in}{3.096698in}}%
\pgfpathlineto{\pgfqpoint{3.194777in}{3.111034in}}%
\pgfpathlineto{\pgfqpoint{3.202708in}{3.125542in}}%
\pgfpathlineto{\pgfqpoint{3.210633in}{3.140225in}}%
\pgfpathlineto{\pgfqpoint{3.197210in}{3.158069in}}%
\pgfpathlineto{\pgfqpoint{3.183781in}{3.176219in}}%
\pgfpathlineto{\pgfqpoint{3.170345in}{3.194677in}}%
\pgfpathlineto{\pgfqpoint{3.156903in}{3.213446in}}%
\pgfpathlineto{\pgfqpoint{3.148968in}{3.198526in}}%
\pgfpathlineto{\pgfqpoint{3.141027in}{3.183790in}}%
\pgfpathlineto{\pgfqpoint{3.133079in}{3.169233in}}%
\pgfpathlineto{\pgfqpoint{3.125124in}{3.154854in}}%
\pgfpathclose%
\pgfusepath{fill}%
\end{pgfscope}%
\begin{pgfscope}%
\pgfpathrectangle{\pgfqpoint{1.150000in}{0.150000in}}{\pgfqpoint{5.700000in}{5.700000in}}%
\pgfusepath{clip}%
\pgfsetbuttcap%
\pgfsetroundjoin%
\definecolor{currentfill}{rgb}{0.267968,0.223549,0.512008}%
\pgfsetfillcolor{currentfill}%
\pgfsetfillopacity{0.800000}%
\pgfsetlinewidth{0.000000pt}%
\definecolor{currentstroke}{rgb}{0.000000,0.000000,0.000000}%
\pgfsetstrokecolor{currentstroke}%
\pgfsetdash{}{0pt}%
\pgfpathmoveto{\pgfqpoint{3.584932in}{2.754761in}}%
\pgfpathlineto{\pgfqpoint{3.598278in}{2.744770in}}%
\pgfpathlineto{\pgfqpoint{3.611624in}{2.735022in}}%
\pgfpathlineto{\pgfqpoint{3.624972in}{2.725514in}}%
\pgfpathlineto{\pgfqpoint{3.638320in}{2.716245in}}%
\pgfpathlineto{\pgfqpoint{3.646170in}{2.729173in}}%
\pgfpathlineto{\pgfqpoint{3.654015in}{2.742219in}}%
\pgfpathlineto{\pgfqpoint{3.661855in}{2.755386in}}%
\pgfpathlineto{\pgfqpoint{3.669690in}{2.768676in}}%
\pgfpathlineto{\pgfqpoint{3.656349in}{2.778196in}}%
\pgfpathlineto{\pgfqpoint{3.643010in}{2.787956in}}%
\pgfpathlineto{\pgfqpoint{3.629671in}{2.797956in}}%
\pgfpathlineto{\pgfqpoint{3.616333in}{2.808199in}}%
\pgfpathlineto{\pgfqpoint{3.608491in}{2.794646in}}%
\pgfpathlineto{\pgfqpoint{3.600643in}{2.781223in}}%
\pgfpathlineto{\pgfqpoint{3.592790in}{2.767930in}}%
\pgfpathlineto{\pgfqpoint{3.584932in}{2.754761in}}%
\pgfpathclose%
\pgfusepath{fill}%
\end{pgfscope}%
\begin{pgfscope}%
\pgfpathrectangle{\pgfqpoint{1.150000in}{0.150000in}}{\pgfqpoint{5.700000in}{5.700000in}}%
\pgfusepath{clip}%
\pgfsetbuttcap%
\pgfsetroundjoin%
\definecolor{currentfill}{rgb}{0.194100,0.399323,0.555565}%
\pgfsetfillcolor{currentfill}%
\pgfsetfillopacity{0.800000}%
\pgfsetlinewidth{0.000000pt}%
\definecolor{currentstroke}{rgb}{0.000000,0.000000,0.000000}%
\pgfsetstrokecolor{currentstroke}%
\pgfsetdash{}{0pt}%
\pgfpathmoveto{\pgfqpoint{4.928599in}{3.183434in}}%
\pgfpathlineto{\pgfqpoint{4.942202in}{3.181096in}}%
\pgfpathlineto{\pgfqpoint{4.955815in}{3.178938in}}%
\pgfpathlineto{\pgfqpoint{4.969437in}{3.176961in}}%
\pgfpathlineto{\pgfqpoint{4.983069in}{3.175164in}}%
\pgfpathlineto{\pgfqpoint{4.990609in}{3.190310in}}%
\pgfpathlineto{\pgfqpoint{4.998150in}{3.205747in}}%
\pgfpathlineto{\pgfqpoint{5.005693in}{3.221483in}}%
\pgfpathlineto{\pgfqpoint{5.013239in}{3.237526in}}%
\pgfpathlineto{\pgfqpoint{4.999623in}{3.240014in}}%
\pgfpathlineto{\pgfqpoint{4.986016in}{3.242682in}}%
\pgfpathlineto{\pgfqpoint{4.972419in}{3.245529in}}%
\pgfpathlineto{\pgfqpoint{4.958831in}{3.248558in}}%
\pgfpathlineto{\pgfqpoint{4.951270in}{3.231813in}}%
\pgfpathlineto{\pgfqpoint{4.943711in}{3.215383in}}%
\pgfpathlineto{\pgfqpoint{4.936154in}{3.199259in}}%
\pgfpathlineto{\pgfqpoint{4.928599in}{3.183434in}}%
\pgfpathclose%
\pgfusepath{fill}%
\end{pgfscope}%
\begin{pgfscope}%
\pgfpathrectangle{\pgfqpoint{1.150000in}{0.150000in}}{\pgfqpoint{5.700000in}{5.700000in}}%
\pgfusepath{clip}%
\pgfsetbuttcap%
\pgfsetroundjoin%
\definecolor{currentfill}{rgb}{0.257322,0.256130,0.526563}%
\pgfsetfillcolor{currentfill}%
\pgfsetfillopacity{0.800000}%
\pgfsetlinewidth{0.000000pt}%
\definecolor{currentstroke}{rgb}{0.000000,0.000000,0.000000}%
\pgfsetstrokecolor{currentstroke}%
\pgfsetdash{}{0pt}%
\pgfpathmoveto{\pgfqpoint{3.393164in}{2.840031in}}%
\pgfpathlineto{\pgfqpoint{3.406527in}{2.827221in}}%
\pgfpathlineto{\pgfqpoint{3.419889in}{2.814674in}}%
\pgfpathlineto{\pgfqpoint{3.433249in}{2.802389in}}%
\pgfpathlineto{\pgfqpoint{3.446607in}{2.790363in}}%
\pgfpathlineto{\pgfqpoint{3.454504in}{2.803478in}}%
\pgfpathlineto{\pgfqpoint{3.462396in}{2.816722in}}%
\pgfpathlineto{\pgfqpoint{3.470281in}{2.830099in}}%
\pgfpathlineto{\pgfqpoint{3.478162in}{2.843610in}}%
\pgfpathlineto{\pgfqpoint{3.464812in}{2.855857in}}%
\pgfpathlineto{\pgfqpoint{3.451461in}{2.868364in}}%
\pgfpathlineto{\pgfqpoint{3.438108in}{2.881132in}}%
\pgfpathlineto{\pgfqpoint{3.424754in}{2.894163in}}%
\pgfpathlineto{\pgfqpoint{3.416865in}{2.880419in}}%
\pgfpathlineto{\pgfqpoint{3.408970in}{2.866817in}}%
\pgfpathlineto{\pgfqpoint{3.401070in}{2.853356in}}%
\pgfpathlineto{\pgfqpoint{3.393164in}{2.840031in}}%
\pgfpathclose%
\pgfusepath{fill}%
\end{pgfscope}%
\begin{pgfscope}%
\pgfpathrectangle{\pgfqpoint{1.150000in}{0.150000in}}{\pgfqpoint{5.700000in}{5.700000in}}%
\pgfusepath{clip}%
\pgfsetbuttcap%
\pgfsetroundjoin%
\definecolor{currentfill}{rgb}{0.262138,0.242286,0.520837}%
\pgfsetfillcolor{currentfill}%
\pgfsetfillopacity{0.800000}%
\pgfsetlinewidth{0.000000pt}%
\definecolor{currentstroke}{rgb}{0.000000,0.000000,0.000000}%
\pgfsetstrokecolor{currentstroke}%
\pgfsetdash{}{0pt}%
\pgfpathmoveto{\pgfqpoint{4.168251in}{2.789397in}}%
\pgfpathlineto{\pgfqpoint{4.181667in}{2.784948in}}%
\pgfpathlineto{\pgfqpoint{4.195088in}{2.780703in}}%
\pgfpathlineto{\pgfqpoint{4.208516in}{2.776662in}}%
\pgfpathlineto{\pgfqpoint{4.221950in}{2.772824in}}%
\pgfpathlineto{\pgfqpoint{4.229649in}{2.785624in}}%
\pgfpathlineto{\pgfqpoint{4.237345in}{2.798557in}}%
\pgfpathlineto{\pgfqpoint{4.245037in}{2.811628in}}%
\pgfpathlineto{\pgfqpoint{4.252726in}{2.824842in}}%
\pgfpathlineto{\pgfqpoint{4.239300in}{2.829088in}}%
\pgfpathlineto{\pgfqpoint{4.225881in}{2.833536in}}%
\pgfpathlineto{\pgfqpoint{4.212467in}{2.838188in}}%
\pgfpathlineto{\pgfqpoint{4.199060in}{2.843045in}}%
\pgfpathlineto{\pgfqpoint{4.191363in}{2.829411in}}%
\pgfpathlineto{\pgfqpoint{4.183663in}{2.815929in}}%
\pgfpathlineto{\pgfqpoint{4.175959in}{2.802593in}}%
\pgfpathlineto{\pgfqpoint{4.168251in}{2.789397in}}%
\pgfpathclose%
\pgfusepath{fill}%
\end{pgfscope}%
\begin{pgfscope}%
\pgfpathrectangle{\pgfqpoint{1.150000in}{0.150000in}}{\pgfqpoint{5.700000in}{5.700000in}}%
\pgfusepath{clip}%
\pgfsetbuttcap%
\pgfsetroundjoin%
\definecolor{currentfill}{rgb}{0.270595,0.214069,0.507052}%
\pgfsetfillcolor{currentfill}%
\pgfsetfillopacity{0.800000}%
\pgfsetlinewidth{0.000000pt}%
\definecolor{currentstroke}{rgb}{0.000000,0.000000,0.000000}%
\pgfsetstrokecolor{currentstroke}%
\pgfsetdash{}{0pt}%
\pgfpathmoveto{\pgfqpoint{3.861121in}{2.723652in}}%
\pgfpathlineto{\pgfqpoint{3.874484in}{2.716834in}}%
\pgfpathlineto{\pgfqpoint{3.887850in}{2.710236in}}%
\pgfpathlineto{\pgfqpoint{3.901220in}{2.703858in}}%
\pgfpathlineto{\pgfqpoint{3.914594in}{2.697698in}}%
\pgfpathlineto{\pgfqpoint{3.922375in}{2.710449in}}%
\pgfpathlineto{\pgfqpoint{3.930151in}{2.723314in}}%
\pgfpathlineto{\pgfqpoint{3.937922in}{2.736296in}}%
\pgfpathlineto{\pgfqpoint{3.945689in}{2.749399in}}%
\pgfpathlineto{\pgfqpoint{3.932323in}{2.755872in}}%
\pgfpathlineto{\pgfqpoint{3.918960in}{2.762564in}}%
\pgfpathlineto{\pgfqpoint{3.905601in}{2.769476in}}%
\pgfpathlineto{\pgfqpoint{3.892246in}{2.776608in}}%
\pgfpathlineto{\pgfqpoint{3.884471in}{2.763180in}}%
\pgfpathlineto{\pgfqpoint{3.876692in}{2.749880in}}%
\pgfpathlineto{\pgfqpoint{3.868909in}{2.736705in}}%
\pgfpathlineto{\pgfqpoint{3.861121in}{2.723652in}}%
\pgfpathclose%
\pgfusepath{fill}%
\end{pgfscope}%
\begin{pgfscope}%
\pgfpathrectangle{\pgfqpoint{1.150000in}{0.150000in}}{\pgfqpoint{5.700000in}{5.700000in}}%
\pgfusepath{clip}%
\pgfsetbuttcap%
\pgfsetroundjoin%
\definecolor{currentfill}{rgb}{0.159194,0.482237,0.558073}%
\pgfsetfillcolor{currentfill}%
\pgfsetfillopacity{0.800000}%
\pgfsetlinewidth{0.000000pt}%
\definecolor{currentstroke}{rgb}{0.000000,0.000000,0.000000}%
\pgfsetstrokecolor{currentstroke}%
\pgfsetdash{}{0pt}%
\pgfpathmoveto{\pgfqpoint{2.994957in}{3.464010in}}%
\pgfpathlineto{\pgfqpoint{3.008503in}{3.441273in}}%
\pgfpathlineto{\pgfqpoint{3.022040in}{3.418884in}}%
\pgfpathlineto{\pgfqpoint{3.035567in}{3.396841in}}%
\pgfpathlineto{\pgfqpoint{3.049083in}{3.375139in}}%
\pgfpathlineto{\pgfqpoint{3.057030in}{3.390738in}}%
\pgfpathlineto{\pgfqpoint{3.064970in}{3.406543in}}%
\pgfpathlineto{\pgfqpoint{3.072902in}{3.422556in}}%
\pgfpathlineto{\pgfqpoint{3.080827in}{3.438781in}}%
\pgfpathlineto{\pgfqpoint{3.067319in}{3.460743in}}%
\pgfpathlineto{\pgfqpoint{3.053803in}{3.483047in}}%
\pgfpathlineto{\pgfqpoint{3.040276in}{3.505698in}}%
\pgfpathlineto{\pgfqpoint{3.026738in}{3.528697in}}%
\pgfpathlineto{\pgfqpoint{3.018804in}{3.512198in}}%
\pgfpathlineto{\pgfqpoint{3.010863in}{3.495919in}}%
\pgfpathlineto{\pgfqpoint{3.002913in}{3.479858in}}%
\pgfpathlineto{\pgfqpoint{2.994957in}{3.464010in}}%
\pgfpathclose%
\pgfusepath{fill}%
\end{pgfscope}%
\begin{pgfscope}%
\pgfpathrectangle{\pgfqpoint{1.150000in}{0.150000in}}{\pgfqpoint{5.700000in}{5.700000in}}%
\pgfusepath{clip}%
\pgfsetbuttcap%
\pgfsetroundjoin%
\definecolor{currentfill}{rgb}{0.185556,0.418570,0.556753}%
\pgfsetfillcolor{currentfill}%
\pgfsetfillopacity{0.800000}%
\pgfsetlinewidth{0.000000pt}%
\definecolor{currentstroke}{rgb}{0.000000,0.000000,0.000000}%
\pgfsetstrokecolor{currentstroke}%
\pgfsetdash{}{0pt}%
\pgfpathmoveto{\pgfqpoint{5.013239in}{3.237526in}}%
\pgfpathlineto{\pgfqpoint{5.026865in}{3.235218in}}%
\pgfpathlineto{\pgfqpoint{5.040501in}{3.233088in}}%
\pgfpathlineto{\pgfqpoint{5.054148in}{3.231137in}}%
\pgfpathlineto{\pgfqpoint{5.067804in}{3.229364in}}%
\pgfpathlineto{\pgfqpoint{5.075336in}{3.245012in}}%
\pgfpathlineto{\pgfqpoint{5.082870in}{3.260975in}}%
\pgfpathlineto{\pgfqpoint{5.090408in}{3.277262in}}%
\pgfpathlineto{\pgfqpoint{5.097950in}{3.293881in}}%
\pgfpathlineto{\pgfqpoint{5.084311in}{3.296377in}}%
\pgfpathlineto{\pgfqpoint{5.070681in}{3.299050in}}%
\pgfpathlineto{\pgfqpoint{5.057062in}{3.301901in}}%
\pgfpathlineto{\pgfqpoint{5.043452in}{3.304931in}}%
\pgfpathlineto{\pgfqpoint{5.035894in}{3.287578in}}%
\pgfpathlineto{\pgfqpoint{5.028339in}{3.270565in}}%
\pgfpathlineto{\pgfqpoint{5.020788in}{3.253884in}}%
\pgfpathlineto{\pgfqpoint{5.013239in}{3.237526in}}%
\pgfpathclose%
\pgfusepath{fill}%
\end{pgfscope}%
\begin{pgfscope}%
\pgfpathrectangle{\pgfqpoint{1.150000in}{0.150000in}}{\pgfqpoint{5.700000in}{5.700000in}}%
\pgfusepath{clip}%
\pgfsetbuttcap%
\pgfsetroundjoin%
\definecolor{currentfill}{rgb}{0.192357,0.403199,0.555836}%
\pgfsetfillcolor{currentfill}%
\pgfsetfillopacity{0.800000}%
\pgfsetlinewidth{0.000000pt}%
\definecolor{currentstroke}{rgb}{0.000000,0.000000,0.000000}%
\pgfsetstrokecolor{currentstroke}%
\pgfsetdash{}{0pt}%
\pgfpathmoveto{\pgfqpoint{3.071240in}{3.232193in}}%
\pgfpathlineto{\pgfqpoint{3.084723in}{3.212378in}}%
\pgfpathlineto{\pgfqpoint{3.098198in}{3.192885in}}%
\pgfpathlineto{\pgfqpoint{3.111665in}{3.173711in}}%
\pgfpathlineto{\pgfqpoint{3.125124in}{3.154854in}}%
\pgfpathlineto{\pgfqpoint{3.133079in}{3.169233in}}%
\pgfpathlineto{\pgfqpoint{3.141027in}{3.183790in}}%
\pgfpathlineto{\pgfqpoint{3.148968in}{3.198526in}}%
\pgfpathlineto{\pgfqpoint{3.156903in}{3.213446in}}%
\pgfpathlineto{\pgfqpoint{3.143453in}{3.232528in}}%
\pgfpathlineto{\pgfqpoint{3.129996in}{3.251927in}}%
\pgfpathlineto{\pgfqpoint{3.116531in}{3.271646in}}%
\pgfpathlineto{\pgfqpoint{3.103059in}{3.291687in}}%
\pgfpathlineto{\pgfqpoint{3.095115in}{3.276529in}}%
\pgfpathlineto{\pgfqpoint{3.087164in}{3.261563in}}%
\pgfpathlineto{\pgfqpoint{3.079205in}{3.246786in}}%
\pgfpathlineto{\pgfqpoint{3.071240in}{3.232193in}}%
\pgfpathclose%
\pgfusepath{fill}%
\end{pgfscope}%
\begin{pgfscope}%
\pgfpathrectangle{\pgfqpoint{1.150000in}{0.150000in}}{\pgfqpoint{5.700000in}{5.700000in}}%
\pgfusepath{clip}%
\pgfsetbuttcap%
\pgfsetroundjoin%
\definecolor{currentfill}{rgb}{0.263663,0.237631,0.518762}%
\pgfsetfillcolor{currentfill}%
\pgfsetfillopacity{0.800000}%
\pgfsetlinewidth{0.000000pt}%
\definecolor{currentstroke}{rgb}{0.000000,0.000000,0.000000}%
\pgfsetstrokecolor{currentstroke}%
\pgfsetdash{}{0pt}%
\pgfpathmoveto{\pgfqpoint{3.446607in}{2.790363in}}%
\pgfpathlineto{\pgfqpoint{3.459964in}{2.778595in}}%
\pgfpathlineto{\pgfqpoint{3.473320in}{2.767083in}}%
\pgfpathlineto{\pgfqpoint{3.486676in}{2.755824in}}%
\pgfpathlineto{\pgfqpoint{3.500030in}{2.744818in}}%
\pgfpathlineto{\pgfqpoint{3.507918in}{2.757724in}}%
\pgfpathlineto{\pgfqpoint{3.515801in}{2.770751in}}%
\pgfpathlineto{\pgfqpoint{3.523678in}{2.783902in}}%
\pgfpathlineto{\pgfqpoint{3.531550in}{2.797181in}}%
\pgfpathlineto{\pgfqpoint{3.518204in}{2.808408in}}%
\pgfpathlineto{\pgfqpoint{3.504857in}{2.819887in}}%
\pgfpathlineto{\pgfqpoint{3.491510in}{2.831621in}}%
\pgfpathlineto{\pgfqpoint{3.478162in}{2.843610in}}%
\pgfpathlineto{\pgfqpoint{3.470281in}{2.830099in}}%
\pgfpathlineto{\pgfqpoint{3.462396in}{2.816722in}}%
\pgfpathlineto{\pgfqpoint{3.454504in}{2.803478in}}%
\pgfpathlineto{\pgfqpoint{3.446607in}{2.790363in}}%
\pgfpathclose%
\pgfusepath{fill}%
\end{pgfscope}%
\begin{pgfscope}%
\pgfpathrectangle{\pgfqpoint{1.150000in}{0.150000in}}{\pgfqpoint{5.700000in}{5.700000in}}%
\pgfusepath{clip}%
\pgfsetbuttcap%
\pgfsetroundjoin%
\definecolor{currentfill}{rgb}{0.266580,0.228262,0.514349}%
\pgfsetfillcolor{currentfill}%
\pgfsetfillopacity{0.800000}%
\pgfsetlinewidth{0.000000pt}%
\definecolor{currentstroke}{rgb}{0.000000,0.000000,0.000000}%
\pgfsetstrokecolor{currentstroke}%
\pgfsetdash{}{0pt}%
\pgfpathmoveto{\pgfqpoint{4.083745in}{2.756291in}}%
\pgfpathlineto{\pgfqpoint{4.097146in}{2.751390in}}%
\pgfpathlineto{\pgfqpoint{4.110553in}{2.746698in}}%
\pgfpathlineto{\pgfqpoint{4.123965in}{2.742213in}}%
\pgfpathlineto{\pgfqpoint{4.137383in}{2.737935in}}%
\pgfpathlineto{\pgfqpoint{4.145106in}{2.750612in}}%
\pgfpathlineto{\pgfqpoint{4.152825in}{2.763411in}}%
\pgfpathlineto{\pgfqpoint{4.160540in}{2.776338in}}%
\pgfpathlineto{\pgfqpoint{4.168251in}{2.789397in}}%
\pgfpathlineto{\pgfqpoint{4.154841in}{2.794052in}}%
\pgfpathlineto{\pgfqpoint{4.141437in}{2.798913in}}%
\pgfpathlineto{\pgfqpoint{4.128039in}{2.803981in}}%
\pgfpathlineto{\pgfqpoint{4.114646in}{2.809258in}}%
\pgfpathlineto{\pgfqpoint{4.106927in}{2.795811in}}%
\pgfpathlineto{\pgfqpoint{4.099203in}{2.782504in}}%
\pgfpathlineto{\pgfqpoint{4.091476in}{2.769332in}}%
\pgfpathlineto{\pgfqpoint{4.083745in}{2.756291in}}%
\pgfpathclose%
\pgfusepath{fill}%
\end{pgfscope}%
\begin{pgfscope}%
\pgfpathrectangle{\pgfqpoint{1.150000in}{0.150000in}}{\pgfqpoint{5.700000in}{5.700000in}}%
\pgfusepath{clip}%
\pgfsetbuttcap%
\pgfsetroundjoin%
\definecolor{currentfill}{rgb}{0.271828,0.209303,0.504434}%
\pgfsetfillcolor{currentfill}%
\pgfsetfillopacity{0.800000}%
\pgfsetlinewidth{0.000000pt}%
\definecolor{currentstroke}{rgb}{0.000000,0.000000,0.000000}%
\pgfsetstrokecolor{currentstroke}%
\pgfsetdash{}{0pt}%
\pgfpathmoveto{\pgfqpoint{3.638320in}{2.716245in}}%
\pgfpathlineto{\pgfqpoint{3.651670in}{2.707214in}}%
\pgfpathlineto{\pgfqpoint{3.665021in}{2.698420in}}%
\pgfpathlineto{\pgfqpoint{3.678373in}{2.689860in}}%
\pgfpathlineto{\pgfqpoint{3.691728in}{2.681534in}}%
\pgfpathlineto{\pgfqpoint{3.699569in}{2.694222in}}%
\pgfpathlineto{\pgfqpoint{3.707406in}{2.707020in}}%
\pgfpathlineto{\pgfqpoint{3.715238in}{2.719931in}}%
\pgfpathlineto{\pgfqpoint{3.723065in}{2.732958in}}%
\pgfpathlineto{\pgfqpoint{3.709719in}{2.741536in}}%
\pgfpathlineto{\pgfqpoint{3.696374in}{2.750347in}}%
\pgfpathlineto{\pgfqpoint{3.683031in}{2.759393in}}%
\pgfpathlineto{\pgfqpoint{3.669690in}{2.768676in}}%
\pgfpathlineto{\pgfqpoint{3.661855in}{2.755386in}}%
\pgfpathlineto{\pgfqpoint{3.654015in}{2.742219in}}%
\pgfpathlineto{\pgfqpoint{3.646170in}{2.729173in}}%
\pgfpathlineto{\pgfqpoint{3.638320in}{2.716245in}}%
\pgfpathclose%
\pgfusepath{fill}%
\end{pgfscope}%
\begin{pgfscope}%
\pgfpathrectangle{\pgfqpoint{1.150000in}{0.150000in}}{\pgfqpoint{5.700000in}{5.700000in}}%
\pgfusepath{clip}%
\pgfsetbuttcap%
\pgfsetroundjoin%
\definecolor{currentfill}{rgb}{0.177423,0.437527,0.557565}%
\pgfsetfillcolor{currentfill}%
\pgfsetfillopacity{0.800000}%
\pgfsetlinewidth{0.000000pt}%
\definecolor{currentstroke}{rgb}{0.000000,0.000000,0.000000}%
\pgfsetstrokecolor{currentstroke}%
\pgfsetdash{}{0pt}%
\pgfpathmoveto{\pgfqpoint{5.097950in}{3.293881in}}%
\pgfpathlineto{\pgfqpoint{5.111599in}{3.291564in}}%
\pgfpathlineto{\pgfqpoint{5.125259in}{3.289423in}}%
\pgfpathlineto{\pgfqpoint{5.138929in}{3.287459in}}%
\pgfpathlineto{\pgfqpoint{5.152609in}{3.285672in}}%
\pgfpathlineto{\pgfqpoint{5.160136in}{3.301890in}}%
\pgfpathlineto{\pgfqpoint{5.167668in}{3.318450in}}%
\pgfpathlineto{\pgfqpoint{5.175204in}{3.335361in}}%
\pgfpathlineto{\pgfqpoint{5.161537in}{3.337711in}}%
\pgfpathlineto{\pgfqpoint{5.147881in}{3.340237in}}%
\pgfpathlineto{\pgfqpoint{5.134234in}{3.342940in}}%
\pgfpathlineto{\pgfqpoint{5.120598in}{3.345820in}}%
\pgfpathlineto{\pgfqpoint{5.113044in}{3.328152in}}%
\pgfpathlineto{\pgfqpoint{5.105495in}{3.310842in}}%
\pgfpathlineto{\pgfqpoint{5.097950in}{3.293881in}}%
\pgfpathclose%
\pgfusepath{fill}%
\end{pgfscope}%
\begin{pgfscope}%
\pgfpathrectangle{\pgfqpoint{1.150000in}{0.150000in}}{\pgfqpoint{5.700000in}{5.700000in}}%
\pgfusepath{clip}%
\pgfsetbuttcap%
\pgfsetroundjoin%
\definecolor{currentfill}{rgb}{0.273006,0.204520,0.501721}%
\pgfsetfillcolor{currentfill}%
\pgfsetfillopacity{0.800000}%
\pgfsetlinewidth{0.000000pt}%
\definecolor{currentstroke}{rgb}{0.000000,0.000000,0.000000}%
\pgfsetstrokecolor{currentstroke}%
\pgfsetdash{}{0pt}%
\pgfpathmoveto{\pgfqpoint{3.776472in}{2.700954in}}%
\pgfpathlineto{\pgfqpoint{3.789830in}{2.693524in}}%
\pgfpathlineto{\pgfqpoint{3.803191in}{2.686319in}}%
\pgfpathlineto{\pgfqpoint{3.816555in}{2.679338in}}%
\pgfpathlineto{\pgfqpoint{3.829922in}{2.672580in}}%
\pgfpathlineto{\pgfqpoint{3.837729in}{2.685184in}}%
\pgfpathlineto{\pgfqpoint{3.845531in}{2.697895in}}%
\pgfpathlineto{\pgfqpoint{3.853328in}{2.710717in}}%
\pgfpathlineto{\pgfqpoint{3.861121in}{2.723652in}}%
\pgfpathlineto{\pgfqpoint{3.847761in}{2.730693in}}%
\pgfpathlineto{\pgfqpoint{3.834405in}{2.737956in}}%
\pgfpathlineto{\pgfqpoint{3.821052in}{2.745444in}}%
\pgfpathlineto{\pgfqpoint{3.807701in}{2.753157in}}%
\pgfpathlineto{\pgfqpoint{3.799901in}{2.739927in}}%
\pgfpathlineto{\pgfqpoint{3.792096in}{2.726819in}}%
\pgfpathlineto{\pgfqpoint{3.784287in}{2.713829in}}%
\pgfpathlineto{\pgfqpoint{3.776472in}{2.700954in}}%
\pgfpathclose%
\pgfusepath{fill}%
\end{pgfscope}%
\begin{pgfscope}%
\pgfpathrectangle{\pgfqpoint{1.150000in}{0.150000in}}{\pgfqpoint{5.700000in}{5.700000in}}%
\pgfusepath{clip}%
\pgfsetbuttcap%
\pgfsetroundjoin%
\definecolor{currentfill}{rgb}{0.269308,0.218818,0.509577}%
\pgfsetfillcolor{currentfill}%
\pgfsetfillopacity{0.800000}%
\pgfsetlinewidth{0.000000pt}%
\definecolor{currentstroke}{rgb}{0.000000,0.000000,0.000000}%
\pgfsetstrokecolor{currentstroke}%
\pgfsetdash{}{0pt}%
\pgfpathmoveto{\pgfqpoint{3.999197in}{2.725667in}}%
\pgfpathlineto{\pgfqpoint{4.012586in}{2.720270in}}%
\pgfpathlineto{\pgfqpoint{4.025979in}{2.715084in}}%
\pgfpathlineto{\pgfqpoint{4.039377in}{2.710110in}}%
\pgfpathlineto{\pgfqpoint{4.052781in}{2.705346in}}%
\pgfpathlineto{\pgfqpoint{4.060528in}{2.717907in}}%
\pgfpathlineto{\pgfqpoint{4.068271in}{2.730582in}}%
\pgfpathlineto{\pgfqpoint{4.076010in}{2.743375in}}%
\pgfpathlineto{\pgfqpoint{4.083745in}{2.756291in}}%
\pgfpathlineto{\pgfqpoint{4.070350in}{2.761400in}}%
\pgfpathlineto{\pgfqpoint{4.056959in}{2.766719in}}%
\pgfpathlineto{\pgfqpoint{4.043574in}{2.772250in}}%
\pgfpathlineto{\pgfqpoint{4.030193in}{2.777993in}}%
\pgfpathlineto{\pgfqpoint{4.022450in}{2.764721in}}%
\pgfpathlineto{\pgfqpoint{4.014704in}{2.751579in}}%
\pgfpathlineto{\pgfqpoint{4.006953in}{2.738562in}}%
\pgfpathlineto{\pgfqpoint{3.999197in}{2.725667in}}%
\pgfpathclose%
\pgfusepath{fill}%
\end{pgfscope}%
\begin{pgfscope}%
\pgfpathrectangle{\pgfqpoint{1.150000in}{0.150000in}}{\pgfqpoint{5.700000in}{5.700000in}}%
\pgfusepath{clip}%
\pgfsetbuttcap%
\pgfsetroundjoin%
\definecolor{currentfill}{rgb}{0.179019,0.433756,0.557430}%
\pgfsetfillcolor{currentfill}%
\pgfsetfillopacity{0.800000}%
\pgfsetlinewidth{0.000000pt}%
\definecolor{currentstroke}{rgb}{0.000000,0.000000,0.000000}%
\pgfsetstrokecolor{currentstroke}%
\pgfsetdash{}{0pt}%
\pgfpathmoveto{\pgfqpoint{3.017222in}{3.314739in}}%
\pgfpathlineto{\pgfqpoint{3.030740in}{3.293604in}}%
\pgfpathlineto{\pgfqpoint{3.044249in}{3.272803in}}%
\pgfpathlineto{\pgfqpoint{3.057749in}{3.252334in}}%
\pgfpathlineto{\pgfqpoint{3.071240in}{3.232193in}}%
\pgfpathlineto{\pgfqpoint{3.079205in}{3.246786in}}%
\pgfpathlineto{\pgfqpoint{3.087164in}{3.261563in}}%
\pgfpathlineto{\pgfqpoint{3.095115in}{3.276529in}}%
\pgfpathlineto{\pgfqpoint{3.103059in}{3.291687in}}%
\pgfpathlineto{\pgfqpoint{3.089578in}{3.312053in}}%
\pgfpathlineto{\pgfqpoint{3.076089in}{3.332749in}}%
\pgfpathlineto{\pgfqpoint{3.062591in}{3.353777in}}%
\pgfpathlineto{\pgfqpoint{3.049083in}{3.375139in}}%
\pgfpathlineto{\pgfqpoint{3.041129in}{3.359743in}}%
\pgfpathlineto{\pgfqpoint{3.033168in}{3.344546in}}%
\pgfpathlineto{\pgfqpoint{3.025199in}{3.329546in}}%
\pgfpathlineto{\pgfqpoint{3.017222in}{3.314739in}}%
\pgfpathclose%
\pgfusepath{fill}%
\end{pgfscope}%
\begin{pgfscope}%
\pgfpathrectangle{\pgfqpoint{1.150000in}{0.150000in}}{\pgfqpoint{5.700000in}{5.700000in}}%
\pgfusepath{clip}%
\pgfsetbuttcap%
\pgfsetroundjoin%
\definecolor{currentfill}{rgb}{0.267968,0.223549,0.512008}%
\pgfsetfillcolor{currentfill}%
\pgfsetfillopacity{0.800000}%
\pgfsetlinewidth{0.000000pt}%
\definecolor{currentstroke}{rgb}{0.000000,0.000000,0.000000}%
\pgfsetstrokecolor{currentstroke}%
\pgfsetdash{}{0pt}%
\pgfpathmoveto{\pgfqpoint{3.500030in}{2.744818in}}%
\pgfpathlineto{\pgfqpoint{3.513384in}{2.734062in}}%
\pgfpathlineto{\pgfqpoint{3.526738in}{2.723555in}}%
\pgfpathlineto{\pgfqpoint{3.540092in}{2.713296in}}%
\pgfpathlineto{\pgfqpoint{3.553446in}{2.703281in}}%
\pgfpathlineto{\pgfqpoint{3.561326in}{2.715978in}}%
\pgfpathlineto{\pgfqpoint{3.569200in}{2.728788in}}%
\pgfpathlineto{\pgfqpoint{3.577068in}{2.741715in}}%
\pgfpathlineto{\pgfqpoint{3.584932in}{2.754761in}}%
\pgfpathlineto{\pgfqpoint{3.571586in}{2.764996in}}%
\pgfpathlineto{\pgfqpoint{3.558241in}{2.775477in}}%
\pgfpathlineto{\pgfqpoint{3.544895in}{2.786204in}}%
\pgfpathlineto{\pgfqpoint{3.531550in}{2.797181in}}%
\pgfpathlineto{\pgfqpoint{3.523678in}{2.783902in}}%
\pgfpathlineto{\pgfqpoint{3.515801in}{2.770751in}}%
\pgfpathlineto{\pgfqpoint{3.507918in}{2.757724in}}%
\pgfpathlineto{\pgfqpoint{3.500030in}{2.744818in}}%
\pgfpathclose%
\pgfusepath{fill}%
\end{pgfscope}%
\begin{pgfscope}%
\pgfpathrectangle{\pgfqpoint{1.150000in}{0.150000in}}{\pgfqpoint{5.700000in}{5.700000in}}%
\pgfusepath{clip}%
\pgfsetbuttcap%
\pgfsetroundjoin%
\definecolor{currentfill}{rgb}{0.235526,0.309527,0.542944}%
\pgfsetfillcolor{currentfill}%
\pgfsetfillopacity{0.800000}%
\pgfsetlinewidth{0.000000pt}%
\definecolor{currentstroke}{rgb}{0.000000,0.000000,0.000000}%
\pgfsetstrokecolor{currentstroke}%
\pgfsetdash{}{0pt}%
\pgfpathmoveto{\pgfqpoint{4.560075in}{2.932919in}}%
\pgfpathlineto{\pgfqpoint{4.573597in}{2.930594in}}%
\pgfpathlineto{\pgfqpoint{4.587129in}{2.928459in}}%
\pgfpathlineto{\pgfqpoint{4.600668in}{2.926513in}}%
\pgfpathlineto{\pgfqpoint{4.614217in}{2.924756in}}%
\pgfpathlineto{\pgfqpoint{4.621819in}{2.937709in}}%
\pgfpathlineto{\pgfqpoint{4.629420in}{2.950843in}}%
\pgfpathlineto{\pgfqpoint{4.637018in}{2.964165in}}%
\pgfpathlineto{\pgfqpoint{4.644615in}{2.977682in}}%
\pgfpathlineto{\pgfqpoint{4.631078in}{2.979973in}}%
\pgfpathlineto{\pgfqpoint{4.617549in}{2.982452in}}%
\pgfpathlineto{\pgfqpoint{4.604030in}{2.985120in}}%
\pgfpathlineto{\pgfqpoint{4.590518in}{2.987978in}}%
\pgfpathlineto{\pgfqpoint{4.582910in}{2.973917in}}%
\pgfpathlineto{\pgfqpoint{4.575300in}{2.960058in}}%
\pgfpathlineto{\pgfqpoint{4.567688in}{2.946394in}}%
\pgfpathlineto{\pgfqpoint{4.560075in}{2.932919in}}%
\pgfpathclose%
\pgfusepath{fill}%
\end{pgfscope}%
\begin{pgfscope}%
\pgfpathrectangle{\pgfqpoint{1.150000in}{0.150000in}}{\pgfqpoint{5.700000in}{5.700000in}}%
\pgfusepath{clip}%
\pgfsetbuttcap%
\pgfsetroundjoin%
\definecolor{currentfill}{rgb}{0.227802,0.326594,0.546532}%
\pgfsetfillcolor{currentfill}%
\pgfsetfillopacity{0.800000}%
\pgfsetlinewidth{0.000000pt}%
\definecolor{currentstroke}{rgb}{0.000000,0.000000,0.000000}%
\pgfsetstrokecolor{currentstroke}%
\pgfsetdash{}{0pt}%
\pgfpathmoveto{\pgfqpoint{4.644615in}{2.977682in}}%
\pgfpathlineto{\pgfqpoint{4.658160in}{2.975579in}}%
\pgfpathlineto{\pgfqpoint{4.671715in}{2.973664in}}%
\pgfpathlineto{\pgfqpoint{4.685279in}{2.971935in}}%
\pgfpathlineto{\pgfqpoint{4.698851in}{2.970393in}}%
\pgfpathlineto{\pgfqpoint{4.706434in}{2.983557in}}%
\pgfpathlineto{\pgfqpoint{4.714016in}{2.996919in}}%
\pgfpathlineto{\pgfqpoint{4.721596in}{3.010487in}}%
\pgfpathlineto{\pgfqpoint{4.729175in}{3.024268in}}%
\pgfpathlineto{\pgfqpoint{4.715615in}{3.026375in}}%
\pgfpathlineto{\pgfqpoint{4.702064in}{3.028668in}}%
\pgfpathlineto{\pgfqpoint{4.688521in}{3.031148in}}%
\pgfpathlineto{\pgfqpoint{4.674988in}{3.033815in}}%
\pgfpathlineto{\pgfqpoint{4.667396in}{3.019459in}}%
\pgfpathlineto{\pgfqpoint{4.659804in}{3.005322in}}%
\pgfpathlineto{\pgfqpoint{4.652210in}{2.991398in}}%
\pgfpathlineto{\pgfqpoint{4.644615in}{2.977682in}}%
\pgfpathclose%
\pgfusepath{fill}%
\end{pgfscope}%
\begin{pgfscope}%
\pgfpathrectangle{\pgfqpoint{1.150000in}{0.150000in}}{\pgfqpoint{5.700000in}{5.700000in}}%
\pgfusepath{clip}%
\pgfsetbuttcap%
\pgfsetroundjoin%
\definecolor{currentfill}{rgb}{0.243113,0.292092,0.538516}%
\pgfsetfillcolor{currentfill}%
\pgfsetfillopacity{0.800000}%
\pgfsetlinewidth{0.000000pt}%
\definecolor{currentstroke}{rgb}{0.000000,0.000000,0.000000}%
\pgfsetstrokecolor{currentstroke}%
\pgfsetdash{}{0pt}%
\pgfpathmoveto{\pgfqpoint{4.475547in}{2.889998in}}%
\pgfpathlineto{\pgfqpoint{4.489047in}{2.887409in}}%
\pgfpathlineto{\pgfqpoint{4.502556in}{2.885013in}}%
\pgfpathlineto{\pgfqpoint{4.516073in}{2.882808in}}%
\pgfpathlineto{\pgfqpoint{4.529598in}{2.880795in}}%
\pgfpathlineto{\pgfqpoint{4.537221in}{2.893572in}}%
\pgfpathlineto{\pgfqpoint{4.544841in}{2.906514in}}%
\pgfpathlineto{\pgfqpoint{4.552459in}{2.919628in}}%
\pgfpathlineto{\pgfqpoint{4.560075in}{2.932919in}}%
\pgfpathlineto{\pgfqpoint{4.546561in}{2.935435in}}%
\pgfpathlineto{\pgfqpoint{4.533055in}{2.938141in}}%
\pgfpathlineto{\pgfqpoint{4.519557in}{2.941040in}}%
\pgfpathlineto{\pgfqpoint{4.506067in}{2.944130in}}%
\pgfpathlineto{\pgfqpoint{4.498440in}{2.930325in}}%
\pgfpathlineto{\pgfqpoint{4.490812in}{2.916706in}}%
\pgfpathlineto{\pgfqpoint{4.483181in}{2.903265in}}%
\pgfpathlineto{\pgfqpoint{4.475547in}{2.889998in}}%
\pgfpathclose%
\pgfusepath{fill}%
\end{pgfscope}%
\begin{pgfscope}%
\pgfpathrectangle{\pgfqpoint{1.150000in}{0.150000in}}{\pgfqpoint{5.700000in}{5.700000in}}%
\pgfusepath{clip}%
\pgfsetbuttcap%
\pgfsetroundjoin%
\definecolor{currentfill}{rgb}{0.220057,0.343307,0.549413}%
\pgfsetfillcolor{currentfill}%
\pgfsetfillopacity{0.800000}%
\pgfsetlinewidth{0.000000pt}%
\definecolor{currentstroke}{rgb}{0.000000,0.000000,0.000000}%
\pgfsetstrokecolor{currentstroke}%
\pgfsetdash{}{0pt}%
\pgfpathmoveto{\pgfqpoint{4.729175in}{3.024268in}}%
\pgfpathlineto{\pgfqpoint{4.742744in}{3.022346in}}%
\pgfpathlineto{\pgfqpoint{4.756323in}{3.020610in}}%
\pgfpathlineto{\pgfqpoint{4.769911in}{3.019058in}}%
\pgfpathlineto{\pgfqpoint{4.783509in}{3.017691in}}%
\pgfpathlineto{\pgfqpoint{4.791074in}{3.031106in}}%
\pgfpathlineto{\pgfqpoint{4.798638in}{3.044738in}}%
\pgfpathlineto{\pgfqpoint{4.806201in}{3.058595in}}%
\pgfpathlineto{\pgfqpoint{4.813765in}{3.072685in}}%
\pgfpathlineto{\pgfqpoint{4.800181in}{3.074649in}}%
\pgfpathlineto{\pgfqpoint{4.786606in}{3.076797in}}%
\pgfpathlineto{\pgfqpoint{4.773041in}{3.079129in}}%
\pgfpathlineto{\pgfqpoint{4.759484in}{3.081646in}}%
\pgfpathlineto{\pgfqpoint{4.751908in}{3.066949in}}%
\pgfpathlineto{\pgfqpoint{4.744331in}{3.052492in}}%
\pgfpathlineto{\pgfqpoint{4.736753in}{3.038267in}}%
\pgfpathlineto{\pgfqpoint{4.729175in}{3.024268in}}%
\pgfpathclose%
\pgfusepath{fill}%
\end{pgfscope}%
\begin{pgfscope}%
\pgfpathrectangle{\pgfqpoint{1.150000in}{0.150000in}}{\pgfqpoint{5.700000in}{5.700000in}}%
\pgfusepath{clip}%
\pgfsetbuttcap%
\pgfsetroundjoin%
\definecolor{currentfill}{rgb}{0.235526,0.309527,0.542944}%
\pgfsetfillcolor{currentfill}%
\pgfsetfillopacity{0.800000}%
\pgfsetlinewidth{0.000000pt}%
\definecolor{currentstroke}{rgb}{0.000000,0.000000,0.000000}%
\pgfsetstrokecolor{currentstroke}%
\pgfsetdash{}{0pt}%
\pgfpathmoveto{\pgfqpoint{3.200763in}{2.960783in}}%
\pgfpathlineto{\pgfqpoint{3.214178in}{2.944840in}}%
\pgfpathlineto{\pgfqpoint{3.227588in}{2.929185in}}%
\pgfpathlineto{\pgfqpoint{3.240994in}{2.913818in}}%
\pgfpathlineto{\pgfqpoint{3.254396in}{2.898736in}}%
\pgfpathlineto{\pgfqpoint{3.262346in}{2.911894in}}%
\pgfpathlineto{\pgfqpoint{3.270290in}{2.925195in}}%
\pgfpathlineto{\pgfqpoint{3.278227in}{2.938641in}}%
\pgfpathlineto{\pgfqpoint{3.286158in}{2.952236in}}%
\pgfpathlineto{\pgfqpoint{3.272767in}{2.967509in}}%
\pgfpathlineto{\pgfqpoint{3.259371in}{2.983067in}}%
\pgfpathlineto{\pgfqpoint{3.245971in}{2.998912in}}%
\pgfpathlineto{\pgfqpoint{3.232566in}{3.015047in}}%
\pgfpathlineto{\pgfqpoint{3.224625in}{3.001250in}}%
\pgfpathlineto{\pgfqpoint{3.216678in}{2.987609in}}%
\pgfpathlineto{\pgfqpoint{3.208723in}{2.974120in}}%
\pgfpathlineto{\pgfqpoint{3.200763in}{2.960783in}}%
\pgfpathclose%
\pgfusepath{fill}%
\end{pgfscope}%
\begin{pgfscope}%
\pgfpathrectangle{\pgfqpoint{1.150000in}{0.150000in}}{\pgfqpoint{5.700000in}{5.700000in}}%
\pgfusepath{clip}%
\pgfsetbuttcap%
\pgfsetroundjoin%
\definecolor{currentfill}{rgb}{0.250425,0.274290,0.533103}%
\pgfsetfillcolor{currentfill}%
\pgfsetfillopacity{0.800000}%
\pgfsetlinewidth{0.000000pt}%
\definecolor{currentstroke}{rgb}{0.000000,0.000000,0.000000}%
\pgfsetstrokecolor{currentstroke}%
\pgfsetdash{}{0pt}%
\pgfpathmoveto{\pgfqpoint{4.391023in}{2.848960in}}%
\pgfpathlineto{\pgfqpoint{4.404502in}{2.846065in}}%
\pgfpathlineto{\pgfqpoint{4.417989in}{2.843365in}}%
\pgfpathlineto{\pgfqpoint{4.431483in}{2.840860in}}%
\pgfpathlineto{\pgfqpoint{4.444986in}{2.838549in}}%
\pgfpathlineto{\pgfqpoint{4.452631in}{2.851179in}}%
\pgfpathlineto{\pgfqpoint{4.460272in}{2.863961in}}%
\pgfpathlineto{\pgfqpoint{4.467911in}{2.876898in}}%
\pgfpathlineto{\pgfqpoint{4.475547in}{2.889998in}}%
\pgfpathlineto{\pgfqpoint{4.462055in}{2.892780in}}%
\pgfpathlineto{\pgfqpoint{4.448570in}{2.895755in}}%
\pgfpathlineto{\pgfqpoint{4.435093in}{2.898925in}}%
\pgfpathlineto{\pgfqpoint{4.421624in}{2.902291in}}%
\pgfpathlineto{\pgfqpoint{4.413978in}{2.888709in}}%
\pgfpathlineto{\pgfqpoint{4.406329in}{2.875297in}}%
\pgfpathlineto{\pgfqpoint{4.398677in}{2.862049in}}%
\pgfpathlineto{\pgfqpoint{4.391023in}{2.848960in}}%
\pgfpathclose%
\pgfusepath{fill}%
\end{pgfscope}%
\begin{pgfscope}%
\pgfpathrectangle{\pgfqpoint{1.150000in}{0.150000in}}{\pgfqpoint{5.700000in}{5.700000in}}%
\pgfusepath{clip}%
\pgfsetbuttcap%
\pgfsetroundjoin%
\definecolor{currentfill}{rgb}{0.246811,0.283237,0.535941}%
\pgfsetfillcolor{currentfill}%
\pgfsetfillopacity{0.800000}%
\pgfsetlinewidth{0.000000pt}%
\definecolor{currentstroke}{rgb}{0.000000,0.000000,0.000000}%
\pgfsetstrokecolor{currentstroke}%
\pgfsetdash{}{0pt}%
\pgfpathmoveto{\pgfqpoint{3.254396in}{2.898736in}}%
\pgfpathlineto{\pgfqpoint{3.267793in}{2.883935in}}%
\pgfpathlineto{\pgfqpoint{3.281187in}{2.869415in}}%
\pgfpathlineto{\pgfqpoint{3.294577in}{2.855173in}}%
\pgfpathlineto{\pgfqpoint{3.307963in}{2.841206in}}%
\pgfpathlineto{\pgfqpoint{3.315903in}{2.854186in}}%
\pgfpathlineto{\pgfqpoint{3.323836in}{2.867301in}}%
\pgfpathlineto{\pgfqpoint{3.331764in}{2.880553in}}%
\pgfpathlineto{\pgfqpoint{3.339686in}{2.893945in}}%
\pgfpathlineto{\pgfqpoint{3.326309in}{2.908102in}}%
\pgfpathlineto{\pgfqpoint{3.312929in}{2.922534in}}%
\pgfpathlineto{\pgfqpoint{3.299545in}{2.937245in}}%
\pgfpathlineto{\pgfqpoint{3.286158in}{2.952236in}}%
\pgfpathlineto{\pgfqpoint{3.278227in}{2.938641in}}%
\pgfpathlineto{\pgfqpoint{3.270290in}{2.925195in}}%
\pgfpathlineto{\pgfqpoint{3.262346in}{2.911894in}}%
\pgfpathlineto{\pgfqpoint{3.254396in}{2.898736in}}%
\pgfpathclose%
\pgfusepath{fill}%
\end{pgfscope}%
\begin{pgfscope}%
\pgfpathrectangle{\pgfqpoint{1.150000in}{0.150000in}}{\pgfqpoint{5.700000in}{5.700000in}}%
\pgfusepath{clip}%
\pgfsetbuttcap%
\pgfsetroundjoin%
\definecolor{currentfill}{rgb}{0.210503,0.363727,0.552206}%
\pgfsetfillcolor{currentfill}%
\pgfsetfillopacity{0.800000}%
\pgfsetlinewidth{0.000000pt}%
\definecolor{currentstroke}{rgb}{0.000000,0.000000,0.000000}%
\pgfsetstrokecolor{currentstroke}%
\pgfsetdash{}{0pt}%
\pgfpathmoveto{\pgfqpoint{4.813765in}{3.072685in}}%
\pgfpathlineto{\pgfqpoint{4.827358in}{3.070904in}}%
\pgfpathlineto{\pgfqpoint{4.840961in}{3.069307in}}%
\pgfpathlineto{\pgfqpoint{4.854574in}{3.067892in}}%
\pgfpathlineto{\pgfqpoint{4.868197in}{3.066659in}}%
\pgfpathlineto{\pgfqpoint{4.875746in}{3.080371in}}%
\pgfpathlineto{\pgfqpoint{4.883294in}{3.094321in}}%
\pgfpathlineto{\pgfqpoint{4.890843in}{3.108516in}}%
\pgfpathlineto{\pgfqpoint{4.898393in}{3.122964in}}%
\pgfpathlineto{\pgfqpoint{4.884785in}{3.124825in}}%
\pgfpathlineto{\pgfqpoint{4.871186in}{3.126868in}}%
\pgfpathlineto{\pgfqpoint{4.857598in}{3.129093in}}%
\pgfpathlineto{\pgfqpoint{4.844018in}{3.131501in}}%
\pgfpathlineto{\pgfqpoint{4.836454in}{3.116414in}}%
\pgfpathlineto{\pgfqpoint{4.828891in}{3.101587in}}%
\pgfpathlineto{\pgfqpoint{4.821328in}{3.087013in}}%
\pgfpathlineto{\pgfqpoint{4.813765in}{3.072685in}}%
\pgfpathclose%
\pgfusepath{fill}%
\end{pgfscope}%
\begin{pgfscope}%
\pgfpathrectangle{\pgfqpoint{1.150000in}{0.150000in}}{\pgfqpoint{5.700000in}{5.700000in}}%
\pgfusepath{clip}%
\pgfsetbuttcap%
\pgfsetroundjoin%
\definecolor{currentfill}{rgb}{0.223925,0.334994,0.548053}%
\pgfsetfillcolor{currentfill}%
\pgfsetfillopacity{0.800000}%
\pgfsetlinewidth{0.000000pt}%
\definecolor{currentstroke}{rgb}{0.000000,0.000000,0.000000}%
\pgfsetstrokecolor{currentstroke}%
\pgfsetdash{}{0pt}%
\pgfpathmoveto{\pgfqpoint{3.147048in}{3.027502in}}%
\pgfpathlineto{\pgfqpoint{3.160485in}{3.010376in}}%
\pgfpathlineto{\pgfqpoint{3.173917in}{2.993549in}}%
\pgfpathlineto{\pgfqpoint{3.187342in}{2.977019in}}%
\pgfpathlineto{\pgfqpoint{3.200763in}{2.960783in}}%
\pgfpathlineto{\pgfqpoint{3.208723in}{2.974120in}}%
\pgfpathlineto{\pgfqpoint{3.216678in}{2.987609in}}%
\pgfpathlineto{\pgfqpoint{3.224625in}{3.001250in}}%
\pgfpathlineto{\pgfqpoint{3.232566in}{3.015047in}}%
\pgfpathlineto{\pgfqpoint{3.219156in}{3.031473in}}%
\pgfpathlineto{\pgfqpoint{3.205741in}{3.048195in}}%
\pgfpathlineto{\pgfqpoint{3.192320in}{3.065213in}}%
\pgfpathlineto{\pgfqpoint{3.178894in}{3.082531in}}%
\pgfpathlineto{\pgfqpoint{3.170943in}{3.068531in}}%
\pgfpathlineto{\pgfqpoint{3.162985in}{3.054694in}}%
\pgfpathlineto{\pgfqpoint{3.155020in}{3.041019in}}%
\pgfpathlineto{\pgfqpoint{3.147048in}{3.027502in}}%
\pgfpathclose%
\pgfusepath{fill}%
\end{pgfscope}%
\begin{pgfscope}%
\pgfpathrectangle{\pgfqpoint{1.150000in}{0.150000in}}{\pgfqpoint{5.700000in}{5.700000in}}%
\pgfusepath{clip}%
\pgfsetbuttcap%
\pgfsetroundjoin%
\definecolor{currentfill}{rgb}{0.257322,0.256130,0.526563}%
\pgfsetfillcolor{currentfill}%
\pgfsetfillopacity{0.800000}%
\pgfsetlinewidth{0.000000pt}%
\definecolor{currentstroke}{rgb}{0.000000,0.000000,0.000000}%
\pgfsetstrokecolor{currentstroke}%
\pgfsetdash{}{0pt}%
\pgfpathmoveto{\pgfqpoint{4.306494in}{2.809871in}}%
\pgfpathlineto{\pgfqpoint{4.319953in}{2.806628in}}%
\pgfpathlineto{\pgfqpoint{4.333419in}{2.803582in}}%
\pgfpathlineto{\pgfqpoint{4.346893in}{2.800734in}}%
\pgfpathlineto{\pgfqpoint{4.360374in}{2.798083in}}%
\pgfpathlineto{\pgfqpoint{4.368041in}{2.810591in}}%
\pgfpathlineto{\pgfqpoint{4.375705in}{2.823236in}}%
\pgfpathlineto{\pgfqpoint{4.383365in}{2.836024in}}%
\pgfpathlineto{\pgfqpoint{4.391023in}{2.848960in}}%
\pgfpathlineto{\pgfqpoint{4.377551in}{2.852050in}}%
\pgfpathlineto{\pgfqpoint{4.364087in}{2.855338in}}%
\pgfpathlineto{\pgfqpoint{4.350630in}{2.858822in}}%
\pgfpathlineto{\pgfqpoint{4.337180in}{2.862505in}}%
\pgfpathlineto{\pgfqpoint{4.329513in}{2.849118in}}%
\pgfpathlineto{\pgfqpoint{4.321843in}{2.835887in}}%
\pgfpathlineto{\pgfqpoint{4.314170in}{2.822806in}}%
\pgfpathlineto{\pgfqpoint{4.306494in}{2.809871in}}%
\pgfpathclose%
\pgfusepath{fill}%
\end{pgfscope}%
\begin{pgfscope}%
\pgfpathrectangle{\pgfqpoint{1.150000in}{0.150000in}}{\pgfqpoint{5.700000in}{5.700000in}}%
\pgfusepath{clip}%
\pgfsetbuttcap%
\pgfsetroundjoin%
\definecolor{currentfill}{rgb}{0.271828,0.209303,0.504434}%
\pgfsetfillcolor{currentfill}%
\pgfsetfillopacity{0.800000}%
\pgfsetlinewidth{0.000000pt}%
\definecolor{currentstroke}{rgb}{0.000000,0.000000,0.000000}%
\pgfsetstrokecolor{currentstroke}%
\pgfsetdash{}{0pt}%
\pgfpathmoveto{\pgfqpoint{3.914594in}{2.697698in}}%
\pgfpathlineto{\pgfqpoint{3.927972in}{2.691756in}}%
\pgfpathlineto{\pgfqpoint{3.941355in}{2.686030in}}%
\pgfpathlineto{\pgfqpoint{3.954742in}{2.680519in}}%
\pgfpathlineto{\pgfqpoint{3.968133in}{2.675222in}}%
\pgfpathlineto{\pgfqpoint{3.975906in}{2.687671in}}%
\pgfpathlineto{\pgfqpoint{3.983674in}{2.700225in}}%
\pgfpathlineto{\pgfqpoint{3.991438in}{2.712890in}}%
\pgfpathlineto{\pgfqpoint{3.999197in}{2.725667in}}%
\pgfpathlineto{\pgfqpoint{3.985814in}{2.731278in}}%
\pgfpathlineto{\pgfqpoint{3.972435in}{2.737103in}}%
\pgfpathlineto{\pgfqpoint{3.959060in}{2.743143in}}%
\pgfpathlineto{\pgfqpoint{3.945689in}{2.749399in}}%
\pgfpathlineto{\pgfqpoint{3.937922in}{2.736296in}}%
\pgfpathlineto{\pgfqpoint{3.930151in}{2.723314in}}%
\pgfpathlineto{\pgfqpoint{3.922375in}{2.710449in}}%
\pgfpathlineto{\pgfqpoint{3.914594in}{2.697698in}}%
\pgfpathclose%
\pgfusepath{fill}%
\end{pgfscope}%
\begin{pgfscope}%
\pgfpathrectangle{\pgfqpoint{1.150000in}{0.150000in}}{\pgfqpoint{5.700000in}{5.700000in}}%
\pgfusepath{clip}%
\pgfsetbuttcap%
\pgfsetroundjoin%
\definecolor{currentfill}{rgb}{0.255645,0.260703,0.528312}%
\pgfsetfillcolor{currentfill}%
\pgfsetfillopacity{0.800000}%
\pgfsetlinewidth{0.000000pt}%
\definecolor{currentstroke}{rgb}{0.000000,0.000000,0.000000}%
\pgfsetstrokecolor{currentstroke}%
\pgfsetdash{}{0pt}%
\pgfpathmoveto{\pgfqpoint{3.307963in}{2.841206in}}%
\pgfpathlineto{\pgfqpoint{3.321346in}{2.827513in}}%
\pgfpathlineto{\pgfqpoint{3.334727in}{2.814091in}}%
\pgfpathlineto{\pgfqpoint{3.348105in}{2.800939in}}%
\pgfpathlineto{\pgfqpoint{3.361480in}{2.788053in}}%
\pgfpathlineto{\pgfqpoint{3.369410in}{2.800855in}}%
\pgfpathlineto{\pgfqpoint{3.377334in}{2.813783in}}%
\pgfpathlineto{\pgfqpoint{3.385252in}{2.826841in}}%
\pgfpathlineto{\pgfqpoint{3.393164in}{2.840031in}}%
\pgfpathlineto{\pgfqpoint{3.379798in}{2.853107in}}%
\pgfpathlineto{\pgfqpoint{3.366430in}{2.866449in}}%
\pgfpathlineto{\pgfqpoint{3.353059in}{2.880061in}}%
\pgfpathlineto{\pgfqpoint{3.339686in}{2.893945in}}%
\pgfpathlineto{\pgfqpoint{3.331764in}{2.880553in}}%
\pgfpathlineto{\pgfqpoint{3.323836in}{2.867301in}}%
\pgfpathlineto{\pgfqpoint{3.315903in}{2.854186in}}%
\pgfpathlineto{\pgfqpoint{3.307963in}{2.841206in}}%
\pgfpathclose%
\pgfusepath{fill}%
\end{pgfscope}%
\begin{pgfscope}%
\pgfpathrectangle{\pgfqpoint{1.150000in}{0.150000in}}{\pgfqpoint{5.700000in}{5.700000in}}%
\pgfusepath{clip}%
\pgfsetbuttcap%
\pgfsetroundjoin%
\definecolor{currentfill}{rgb}{0.201239,0.383670,0.554294}%
\pgfsetfillcolor{currentfill}%
\pgfsetfillopacity{0.800000}%
\pgfsetlinewidth{0.000000pt}%
\definecolor{currentstroke}{rgb}{0.000000,0.000000,0.000000}%
\pgfsetstrokecolor{currentstroke}%
\pgfsetdash{}{0pt}%
\pgfpathmoveto{\pgfqpoint{4.898393in}{3.122964in}}%
\pgfpathlineto{\pgfqpoint{4.912011in}{3.121285in}}%
\pgfpathlineto{\pgfqpoint{4.925638in}{3.119786in}}%
\pgfpathlineto{\pgfqpoint{4.939276in}{3.118469in}}%
\pgfpathlineto{\pgfqpoint{4.952924in}{3.117331in}}%
\pgfpathlineto{\pgfqpoint{4.960459in}{3.131392in}}%
\pgfpathlineto{\pgfqpoint{4.967995in}{3.145712in}}%
\pgfpathlineto{\pgfqpoint{4.975531in}{3.160300in}}%
\pgfpathlineto{\pgfqpoint{4.983069in}{3.175164in}}%
\pgfpathlineto{\pgfqpoint{4.969437in}{3.176961in}}%
\pgfpathlineto{\pgfqpoint{4.955815in}{3.178938in}}%
\pgfpathlineto{\pgfqpoint{4.942202in}{3.181096in}}%
\pgfpathlineto{\pgfqpoint{4.928599in}{3.183434in}}%
\pgfpathlineto{\pgfqpoint{4.921046in}{3.167900in}}%
\pgfpathlineto{\pgfqpoint{4.913494in}{3.152649in}}%
\pgfpathlineto{\pgfqpoint{4.905943in}{3.137673in}}%
\pgfpathlineto{\pgfqpoint{4.898393in}{3.122964in}}%
\pgfpathclose%
\pgfusepath{fill}%
\end{pgfscope}%
\begin{pgfscope}%
\pgfpathrectangle{\pgfqpoint{1.150000in}{0.150000in}}{\pgfqpoint{5.700000in}{5.700000in}}%
\pgfusepath{clip}%
\pgfsetbuttcap%
\pgfsetroundjoin%
\definecolor{currentfill}{rgb}{0.212395,0.359683,0.551710}%
\pgfsetfillcolor{currentfill}%
\pgfsetfillopacity{0.800000}%
\pgfsetlinewidth{0.000000pt}%
\definecolor{currentstroke}{rgb}{0.000000,0.000000,0.000000}%
\pgfsetstrokecolor{currentstroke}%
\pgfsetdash{}{0pt}%
\pgfpathmoveto{\pgfqpoint{3.093235in}{3.099057in}}%
\pgfpathlineto{\pgfqpoint{3.106698in}{3.080705in}}%
\pgfpathlineto{\pgfqpoint{3.120155in}{3.062664in}}%
\pgfpathlineto{\pgfqpoint{3.133605in}{3.044931in}}%
\pgfpathlineto{\pgfqpoint{3.147048in}{3.027502in}}%
\pgfpathlineto{\pgfqpoint{3.155020in}{3.041019in}}%
\pgfpathlineto{\pgfqpoint{3.162985in}{3.054694in}}%
\pgfpathlineto{\pgfqpoint{3.170943in}{3.068531in}}%
\pgfpathlineto{\pgfqpoint{3.178894in}{3.082531in}}%
\pgfpathlineto{\pgfqpoint{3.165461in}{3.100152in}}%
\pgfpathlineto{\pgfqpoint{3.152022in}{3.118077in}}%
\pgfpathlineto{\pgfqpoint{3.138577in}{3.136310in}}%
\pgfpathlineto{\pgfqpoint{3.125124in}{3.154854in}}%
\pgfpathlineto{\pgfqpoint{3.117162in}{3.140650in}}%
\pgfpathlineto{\pgfqpoint{3.109194in}{3.126617in}}%
\pgfpathlineto{\pgfqpoint{3.101218in}{3.112753in}}%
\pgfpathlineto{\pgfqpoint{3.093235in}{3.099057in}}%
\pgfpathclose%
\pgfusepath{fill}%
\end{pgfscope}%
\begin{pgfscope}%
\pgfpathrectangle{\pgfqpoint{1.150000in}{0.150000in}}{\pgfqpoint{5.700000in}{5.700000in}}%
\pgfusepath{clip}%
\pgfsetbuttcap%
\pgfsetroundjoin%
\definecolor{currentfill}{rgb}{0.262138,0.242286,0.520837}%
\pgfsetfillcolor{currentfill}%
\pgfsetfillopacity{0.800000}%
\pgfsetlinewidth{0.000000pt}%
\definecolor{currentstroke}{rgb}{0.000000,0.000000,0.000000}%
\pgfsetstrokecolor{currentstroke}%
\pgfsetdash{}{0pt}%
\pgfpathmoveto{\pgfqpoint{4.221950in}{2.772824in}}%
\pgfpathlineto{\pgfqpoint{4.235391in}{2.769188in}}%
\pgfpathlineto{\pgfqpoint{4.248838in}{2.765754in}}%
\pgfpathlineto{\pgfqpoint{4.262292in}{2.762520in}}%
\pgfpathlineto{\pgfqpoint{4.275753in}{2.759485in}}%
\pgfpathlineto{\pgfqpoint{4.283444in}{2.771888in}}%
\pgfpathlineto{\pgfqpoint{4.291131in}{2.784417in}}%
\pgfpathlineto{\pgfqpoint{4.298814in}{2.797076in}}%
\pgfpathlineto{\pgfqpoint{4.306494in}{2.809871in}}%
\pgfpathlineto{\pgfqpoint{4.293041in}{2.813314in}}%
\pgfpathlineto{\pgfqpoint{4.279596in}{2.816956in}}%
\pgfpathlineto{\pgfqpoint{4.266158in}{2.820798in}}%
\pgfpathlineto{\pgfqpoint{4.252726in}{2.824842in}}%
\pgfpathlineto{\pgfqpoint{4.245037in}{2.811628in}}%
\pgfpathlineto{\pgfqpoint{4.237345in}{2.798557in}}%
\pgfpathlineto{\pgfqpoint{4.229649in}{2.785624in}}%
\pgfpathlineto{\pgfqpoint{4.221950in}{2.772824in}}%
\pgfpathclose%
\pgfusepath{fill}%
\end{pgfscope}%
\begin{pgfscope}%
\pgfpathrectangle{\pgfqpoint{1.150000in}{0.150000in}}{\pgfqpoint{5.700000in}{5.700000in}}%
\pgfusepath{clip}%
\pgfsetbuttcap%
\pgfsetroundjoin%
\definecolor{currentfill}{rgb}{0.165117,0.467423,0.558141}%
\pgfsetfillcolor{currentfill}%
\pgfsetfillopacity{0.800000}%
\pgfsetlinewidth{0.000000pt}%
\definecolor{currentstroke}{rgb}{0.000000,0.000000,0.000000}%
\pgfsetstrokecolor{currentstroke}%
\pgfsetdash{}{0pt}%
\pgfpathmoveto{\pgfqpoint{2.963052in}{3.402698in}}%
\pgfpathlineto{\pgfqpoint{2.976610in}{3.380189in}}%
\pgfpathlineto{\pgfqpoint{2.990158in}{3.358029in}}%
\pgfpathlineto{\pgfqpoint{3.003695in}{3.336213in}}%
\pgfpathlineto{\pgfqpoint{3.017222in}{3.314739in}}%
\pgfpathlineto{\pgfqpoint{3.025199in}{3.329546in}}%
\pgfpathlineto{\pgfqpoint{3.033168in}{3.344546in}}%
\pgfpathlineto{\pgfqpoint{3.041129in}{3.359743in}}%
\pgfpathlineto{\pgfqpoint{3.049083in}{3.375139in}}%
\pgfpathlineto{\pgfqpoint{3.035567in}{3.396841in}}%
\pgfpathlineto{\pgfqpoint{3.022040in}{3.418884in}}%
\pgfpathlineto{\pgfqpoint{3.008503in}{3.441273in}}%
\pgfpathlineto{\pgfqpoint{2.994957in}{3.464010in}}%
\pgfpathlineto{\pgfqpoint{2.986992in}{3.448373in}}%
\pgfpathlineto{\pgfqpoint{2.979020in}{3.432944in}}%
\pgfpathlineto{\pgfqpoint{2.971040in}{3.417720in}}%
\pgfpathlineto{\pgfqpoint{2.963052in}{3.402698in}}%
\pgfpathclose%
\pgfusepath{fill}%
\end{pgfscope}%
\begin{pgfscope}%
\pgfpathrectangle{\pgfqpoint{1.150000in}{0.150000in}}{\pgfqpoint{5.700000in}{5.700000in}}%
\pgfusepath{clip}%
\pgfsetbuttcap%
\pgfsetroundjoin%
\definecolor{currentfill}{rgb}{0.274128,0.199721,0.498911}%
\pgfsetfillcolor{currentfill}%
\pgfsetfillopacity{0.800000}%
\pgfsetlinewidth{0.000000pt}%
\definecolor{currentstroke}{rgb}{0.000000,0.000000,0.000000}%
\pgfsetstrokecolor{currentstroke}%
\pgfsetdash{}{0pt}%
\pgfpathmoveto{\pgfqpoint{3.691728in}{2.681534in}}%
\pgfpathlineto{\pgfqpoint{3.705084in}{2.673440in}}%
\pgfpathlineto{\pgfqpoint{3.718442in}{2.665577in}}%
\pgfpathlineto{\pgfqpoint{3.731803in}{2.657943in}}%
\pgfpathlineto{\pgfqpoint{3.745166in}{2.650537in}}%
\pgfpathlineto{\pgfqpoint{3.753000in}{2.662986in}}%
\pgfpathlineto{\pgfqpoint{3.760829in}{2.675536in}}%
\pgfpathlineto{\pgfqpoint{3.768653in}{2.688191in}}%
\pgfpathlineto{\pgfqpoint{3.776472in}{2.700954in}}%
\pgfpathlineto{\pgfqpoint{3.763117in}{2.708612in}}%
\pgfpathlineto{\pgfqpoint{3.749764in}{2.716497in}}%
\pgfpathlineto{\pgfqpoint{3.736414in}{2.724612in}}%
\pgfpathlineto{\pgfqpoint{3.723065in}{2.732958in}}%
\pgfpathlineto{\pgfqpoint{3.715238in}{2.719931in}}%
\pgfpathlineto{\pgfqpoint{3.707406in}{2.707020in}}%
\pgfpathlineto{\pgfqpoint{3.699569in}{2.694222in}}%
\pgfpathlineto{\pgfqpoint{3.691728in}{2.681534in}}%
\pgfpathclose%
\pgfusepath{fill}%
\end{pgfscope}%
\begin{pgfscope}%
\pgfpathrectangle{\pgfqpoint{1.150000in}{0.150000in}}{\pgfqpoint{5.700000in}{5.700000in}}%
\pgfusepath{clip}%
\pgfsetbuttcap%
\pgfsetroundjoin%
\definecolor{currentfill}{rgb}{0.192357,0.403199,0.555836}%
\pgfsetfillcolor{currentfill}%
\pgfsetfillopacity{0.800000}%
\pgfsetlinewidth{0.000000pt}%
\definecolor{currentstroke}{rgb}{0.000000,0.000000,0.000000}%
\pgfsetstrokecolor{currentstroke}%
\pgfsetdash{}{0pt}%
\pgfpathmoveto{\pgfqpoint{4.983069in}{3.175164in}}%
\pgfpathlineto{\pgfqpoint{4.996711in}{3.173546in}}%
\pgfpathlineto{\pgfqpoint{5.010364in}{3.172107in}}%
\pgfpathlineto{\pgfqpoint{5.024027in}{3.170847in}}%
\pgfpathlineto{\pgfqpoint{5.037700in}{3.169766in}}%
\pgfpathlineto{\pgfqpoint{5.045223in}{3.184232in}}%
\pgfpathlineto{\pgfqpoint{5.052748in}{3.198982in}}%
\pgfpathlineto{\pgfqpoint{5.060275in}{3.214024in}}%
\pgfpathlineto{\pgfqpoint{5.067804in}{3.229364in}}%
\pgfpathlineto{\pgfqpoint{5.054148in}{3.231137in}}%
\pgfpathlineto{\pgfqpoint{5.040501in}{3.233088in}}%
\pgfpathlineto{\pgfqpoint{5.026865in}{3.235218in}}%
\pgfpathlineto{\pgfqpoint{5.013239in}{3.237526in}}%
\pgfpathlineto{\pgfqpoint{5.005693in}{3.221483in}}%
\pgfpathlineto{\pgfqpoint{4.998150in}{3.205747in}}%
\pgfpathlineto{\pgfqpoint{4.990609in}{3.190310in}}%
\pgfpathlineto{\pgfqpoint{4.983069in}{3.175164in}}%
\pgfpathclose%
\pgfusepath{fill}%
\end{pgfscope}%
\begin{pgfscope}%
\pgfpathrectangle{\pgfqpoint{1.150000in}{0.150000in}}{\pgfqpoint{5.700000in}{5.700000in}}%
\pgfusepath{clip}%
\pgfsetbuttcap%
\pgfsetroundjoin%
\definecolor{currentfill}{rgb}{0.271828,0.209303,0.504434}%
\pgfsetfillcolor{currentfill}%
\pgfsetfillopacity{0.800000}%
\pgfsetlinewidth{0.000000pt}%
\definecolor{currentstroke}{rgb}{0.000000,0.000000,0.000000}%
\pgfsetstrokecolor{currentstroke}%
\pgfsetdash{}{0pt}%
\pgfpathmoveto{\pgfqpoint{3.553446in}{2.703281in}}%
\pgfpathlineto{\pgfqpoint{3.566801in}{2.693511in}}%
\pgfpathlineto{\pgfqpoint{3.580156in}{2.683983in}}%
\pgfpathlineto{\pgfqpoint{3.593512in}{2.674696in}}%
\pgfpathlineto{\pgfqpoint{3.606868in}{2.665648in}}%
\pgfpathlineto{\pgfqpoint{3.614739in}{2.678135in}}%
\pgfpathlineto{\pgfqpoint{3.622605in}{2.690729in}}%
\pgfpathlineto{\pgfqpoint{3.630465in}{2.703431in}}%
\pgfpathlineto{\pgfqpoint{3.638320in}{2.716245in}}%
\pgfpathlineto{\pgfqpoint{3.624972in}{2.725514in}}%
\pgfpathlineto{\pgfqpoint{3.611624in}{2.735022in}}%
\pgfpathlineto{\pgfqpoint{3.598278in}{2.744770in}}%
\pgfpathlineto{\pgfqpoint{3.584932in}{2.754761in}}%
\pgfpathlineto{\pgfqpoint{3.577068in}{2.741715in}}%
\pgfpathlineto{\pgfqpoint{3.569200in}{2.728788in}}%
\pgfpathlineto{\pgfqpoint{3.561326in}{2.715978in}}%
\pgfpathlineto{\pgfqpoint{3.553446in}{2.703281in}}%
\pgfpathclose%
\pgfusepath{fill}%
\end{pgfscope}%
\begin{pgfscope}%
\pgfpathrectangle{\pgfqpoint{1.150000in}{0.150000in}}{\pgfqpoint{5.700000in}{5.700000in}}%
\pgfusepath{clip}%
\pgfsetbuttcap%
\pgfsetroundjoin%
\definecolor{currentfill}{rgb}{0.262138,0.242286,0.520837}%
\pgfsetfillcolor{currentfill}%
\pgfsetfillopacity{0.800000}%
\pgfsetlinewidth{0.000000pt}%
\definecolor{currentstroke}{rgb}{0.000000,0.000000,0.000000}%
\pgfsetstrokecolor{currentstroke}%
\pgfsetdash{}{0pt}%
\pgfpathmoveto{\pgfqpoint{3.361480in}{2.788053in}}%
\pgfpathlineto{\pgfqpoint{3.374853in}{2.775433in}}%
\pgfpathlineto{\pgfqpoint{3.388224in}{2.763076in}}%
\pgfpathlineto{\pgfqpoint{3.401594in}{2.750981in}}%
\pgfpathlineto{\pgfqpoint{3.414962in}{2.739144in}}%
\pgfpathlineto{\pgfqpoint{3.422882in}{2.751768in}}%
\pgfpathlineto{\pgfqpoint{3.430796in}{2.764511in}}%
\pgfpathlineto{\pgfqpoint{3.438704in}{2.777375in}}%
\pgfpathlineto{\pgfqpoint{3.446607in}{2.790363in}}%
\pgfpathlineto{\pgfqpoint{3.433249in}{2.802389in}}%
\pgfpathlineto{\pgfqpoint{3.419889in}{2.814674in}}%
\pgfpathlineto{\pgfqpoint{3.406527in}{2.827221in}}%
\pgfpathlineto{\pgfqpoint{3.393164in}{2.840031in}}%
\pgfpathlineto{\pgfqpoint{3.385252in}{2.826841in}}%
\pgfpathlineto{\pgfqpoint{3.377334in}{2.813783in}}%
\pgfpathlineto{\pgfqpoint{3.369410in}{2.800855in}}%
\pgfpathlineto{\pgfqpoint{3.361480in}{2.788053in}}%
\pgfpathclose%
\pgfusepath{fill}%
\end{pgfscope}%
\begin{pgfscope}%
\pgfpathrectangle{\pgfqpoint{1.150000in}{0.150000in}}{\pgfqpoint{5.700000in}{5.700000in}}%
\pgfusepath{clip}%
\pgfsetbuttcap%
\pgfsetroundjoin%
\definecolor{currentfill}{rgb}{0.266580,0.228262,0.514349}%
\pgfsetfillcolor{currentfill}%
\pgfsetfillopacity{0.800000}%
\pgfsetlinewidth{0.000000pt}%
\definecolor{currentstroke}{rgb}{0.000000,0.000000,0.000000}%
\pgfsetstrokecolor{currentstroke}%
\pgfsetdash{}{0pt}%
\pgfpathmoveto{\pgfqpoint{4.137383in}{2.737935in}}%
\pgfpathlineto{\pgfqpoint{4.150806in}{2.733862in}}%
\pgfpathlineto{\pgfqpoint{4.164236in}{2.729994in}}%
\pgfpathlineto{\pgfqpoint{4.177673in}{2.726330in}}%
\pgfpathlineto{\pgfqpoint{4.191115in}{2.722869in}}%
\pgfpathlineto{\pgfqpoint{4.198830in}{2.735181in}}%
\pgfpathlineto{\pgfqpoint{4.206540in}{2.747607in}}%
\pgfpathlineto{\pgfqpoint{4.214247in}{2.760154in}}%
\pgfpathlineto{\pgfqpoint{4.221950in}{2.772824in}}%
\pgfpathlineto{\pgfqpoint{4.208516in}{2.776662in}}%
\pgfpathlineto{\pgfqpoint{4.195088in}{2.780703in}}%
\pgfpathlineto{\pgfqpoint{4.181667in}{2.784948in}}%
\pgfpathlineto{\pgfqpoint{4.168251in}{2.789397in}}%
\pgfpathlineto{\pgfqpoint{4.160540in}{2.776338in}}%
\pgfpathlineto{\pgfqpoint{4.152825in}{2.763411in}}%
\pgfpathlineto{\pgfqpoint{4.145106in}{2.750612in}}%
\pgfpathlineto{\pgfqpoint{4.137383in}{2.737935in}}%
\pgfpathclose%
\pgfusepath{fill}%
\end{pgfscope}%
\begin{pgfscope}%
\pgfpathrectangle{\pgfqpoint{1.150000in}{0.150000in}}{\pgfqpoint{5.700000in}{5.700000in}}%
\pgfusepath{clip}%
\pgfsetbuttcap%
\pgfsetroundjoin%
\definecolor{currentfill}{rgb}{0.197636,0.391528,0.554969}%
\pgfsetfillcolor{currentfill}%
\pgfsetfillopacity{0.800000}%
\pgfsetlinewidth{0.000000pt}%
\definecolor{currentstroke}{rgb}{0.000000,0.000000,0.000000}%
\pgfsetstrokecolor{currentstroke}%
\pgfsetdash{}{0pt}%
\pgfpathmoveto{\pgfqpoint{3.039305in}{3.175624in}}%
\pgfpathlineto{\pgfqpoint{3.052800in}{3.156002in}}%
\pgfpathlineto{\pgfqpoint{3.066286in}{3.136702in}}%
\pgfpathlineto{\pgfqpoint{3.079764in}{3.117721in}}%
\pgfpathlineto{\pgfqpoint{3.093235in}{3.099057in}}%
\pgfpathlineto{\pgfqpoint{3.101218in}{3.112753in}}%
\pgfpathlineto{\pgfqpoint{3.109194in}{3.126617in}}%
\pgfpathlineto{\pgfqpoint{3.117162in}{3.140650in}}%
\pgfpathlineto{\pgfqpoint{3.125124in}{3.154854in}}%
\pgfpathlineto{\pgfqpoint{3.111665in}{3.173711in}}%
\pgfpathlineto{\pgfqpoint{3.098198in}{3.192885in}}%
\pgfpathlineto{\pgfqpoint{3.084723in}{3.212378in}}%
\pgfpathlineto{\pgfqpoint{3.071240in}{3.232193in}}%
\pgfpathlineto{\pgfqpoint{3.063267in}{3.217783in}}%
\pgfpathlineto{\pgfqpoint{3.055287in}{3.203554in}}%
\pgfpathlineto{\pgfqpoint{3.047300in}{3.189502in}}%
\pgfpathlineto{\pgfqpoint{3.039305in}{3.175624in}}%
\pgfpathclose%
\pgfusepath{fill}%
\end{pgfscope}%
\begin{pgfscope}%
\pgfpathrectangle{\pgfqpoint{1.150000in}{0.150000in}}{\pgfqpoint{5.700000in}{5.700000in}}%
\pgfusepath{clip}%
\pgfsetbuttcap%
\pgfsetroundjoin%
\definecolor{currentfill}{rgb}{0.183898,0.422383,0.556944}%
\pgfsetfillcolor{currentfill}%
\pgfsetfillopacity{0.800000}%
\pgfsetlinewidth{0.000000pt}%
\definecolor{currentstroke}{rgb}{0.000000,0.000000,0.000000}%
\pgfsetstrokecolor{currentstroke}%
\pgfsetdash{}{0pt}%
\pgfpathmoveto{\pgfqpoint{5.067804in}{3.229364in}}%
\pgfpathlineto{\pgfqpoint{5.081471in}{3.227769in}}%
\pgfpathlineto{\pgfqpoint{5.095148in}{3.226351in}}%
\pgfpathlineto{\pgfqpoint{5.108836in}{3.225110in}}%
\pgfpathlineto{\pgfqpoint{5.122534in}{3.224046in}}%
\pgfpathlineto{\pgfqpoint{5.130048in}{3.238982in}}%
\pgfpathlineto{\pgfqpoint{5.137565in}{3.254226in}}%
\pgfpathlineto{\pgfqpoint{5.145085in}{3.269787in}}%
\pgfpathlineto{\pgfqpoint{5.152609in}{3.285672in}}%
\pgfpathlineto{\pgfqpoint{5.138929in}{3.287459in}}%
\pgfpathlineto{\pgfqpoint{5.125259in}{3.289423in}}%
\pgfpathlineto{\pgfqpoint{5.111599in}{3.291564in}}%
\pgfpathlineto{\pgfqpoint{5.097950in}{3.293881in}}%
\pgfpathlineto{\pgfqpoint{5.090408in}{3.277262in}}%
\pgfpathlineto{\pgfqpoint{5.082870in}{3.260975in}}%
\pgfpathlineto{\pgfqpoint{5.075336in}{3.245012in}}%
\pgfpathlineto{\pgfqpoint{5.067804in}{3.229364in}}%
\pgfpathclose%
\pgfusepath{fill}%
\end{pgfscope}%
\begin{pgfscope}%
\pgfpathrectangle{\pgfqpoint{1.150000in}{0.150000in}}{\pgfqpoint{5.700000in}{5.700000in}}%
\pgfusepath{clip}%
\pgfsetbuttcap%
\pgfsetroundjoin%
\definecolor{currentfill}{rgb}{0.274128,0.199721,0.498911}%
\pgfsetfillcolor{currentfill}%
\pgfsetfillopacity{0.800000}%
\pgfsetlinewidth{0.000000pt}%
\definecolor{currentstroke}{rgb}{0.000000,0.000000,0.000000}%
\pgfsetstrokecolor{currentstroke}%
\pgfsetdash{}{0pt}%
\pgfpathmoveto{\pgfqpoint{3.829922in}{2.672580in}}%
\pgfpathlineto{\pgfqpoint{3.843293in}{2.666045in}}%
\pgfpathlineto{\pgfqpoint{3.856667in}{2.659730in}}%
\pgfpathlineto{\pgfqpoint{3.870045in}{2.653635in}}%
\pgfpathlineto{\pgfqpoint{3.883427in}{2.647758in}}%
\pgfpathlineto{\pgfqpoint{3.891226in}{2.660091in}}%
\pgfpathlineto{\pgfqpoint{3.899020in}{2.672522in}}%
\pgfpathlineto{\pgfqpoint{3.906809in}{2.685057in}}%
\pgfpathlineto{\pgfqpoint{3.914594in}{2.697698in}}%
\pgfpathlineto{\pgfqpoint{3.901220in}{2.703858in}}%
\pgfpathlineto{\pgfqpoint{3.887850in}{2.710236in}}%
\pgfpathlineto{\pgfqpoint{3.874484in}{2.716834in}}%
\pgfpathlineto{\pgfqpoint{3.861121in}{2.723652in}}%
\pgfpathlineto{\pgfqpoint{3.853328in}{2.710717in}}%
\pgfpathlineto{\pgfqpoint{3.845531in}{2.697895in}}%
\pgfpathlineto{\pgfqpoint{3.837729in}{2.685184in}}%
\pgfpathlineto{\pgfqpoint{3.829922in}{2.672580in}}%
\pgfpathclose%
\pgfusepath{fill}%
\end{pgfscope}%
\begin{pgfscope}%
\pgfpathrectangle{\pgfqpoint{1.150000in}{0.150000in}}{\pgfqpoint{5.700000in}{5.700000in}}%
\pgfusepath{clip}%
\pgfsetbuttcap%
\pgfsetroundjoin%
\definecolor{currentfill}{rgb}{0.267968,0.223549,0.512008}%
\pgfsetfillcolor{currentfill}%
\pgfsetfillopacity{0.800000}%
\pgfsetlinewidth{0.000000pt}%
\definecolor{currentstroke}{rgb}{0.000000,0.000000,0.000000}%
\pgfsetstrokecolor{currentstroke}%
\pgfsetdash{}{0pt}%
\pgfpathmoveto{\pgfqpoint{3.414962in}{2.739144in}}%
\pgfpathlineto{\pgfqpoint{3.428328in}{2.727566in}}%
\pgfpathlineto{\pgfqpoint{3.441694in}{2.716243in}}%
\pgfpathlineto{\pgfqpoint{3.455058in}{2.705174in}}%
\pgfpathlineto{\pgfqpoint{3.468422in}{2.694358in}}%
\pgfpathlineto{\pgfqpoint{3.476332in}{2.706803in}}%
\pgfpathlineto{\pgfqpoint{3.484237in}{2.719360in}}%
\pgfpathlineto{\pgfqpoint{3.492136in}{2.732031in}}%
\pgfpathlineto{\pgfqpoint{3.500030in}{2.744818in}}%
\pgfpathlineto{\pgfqpoint{3.486676in}{2.755824in}}%
\pgfpathlineto{\pgfqpoint{3.473320in}{2.767083in}}%
\pgfpathlineto{\pgfqpoint{3.459964in}{2.778595in}}%
\pgfpathlineto{\pgfqpoint{3.446607in}{2.790363in}}%
\pgfpathlineto{\pgfqpoint{3.438704in}{2.777375in}}%
\pgfpathlineto{\pgfqpoint{3.430796in}{2.764511in}}%
\pgfpathlineto{\pgfqpoint{3.422882in}{2.751768in}}%
\pgfpathlineto{\pgfqpoint{3.414962in}{2.739144in}}%
\pgfpathclose%
\pgfusepath{fill}%
\end{pgfscope}%
\begin{pgfscope}%
\pgfpathrectangle{\pgfqpoint{1.150000in}{0.150000in}}{\pgfqpoint{5.700000in}{5.700000in}}%
\pgfusepath{clip}%
\pgfsetbuttcap%
\pgfsetroundjoin%
\definecolor{currentfill}{rgb}{0.270595,0.214069,0.507052}%
\pgfsetfillcolor{currentfill}%
\pgfsetfillopacity{0.800000}%
\pgfsetlinewidth{0.000000pt}%
\definecolor{currentstroke}{rgb}{0.000000,0.000000,0.000000}%
\pgfsetstrokecolor{currentstroke}%
\pgfsetdash{}{0pt}%
\pgfpathmoveto{\pgfqpoint{4.052781in}{2.705346in}}%
\pgfpathlineto{\pgfqpoint{4.066190in}{2.700791in}}%
\pgfpathlineto{\pgfqpoint{4.079604in}{2.696444in}}%
\pgfpathlineto{\pgfqpoint{4.093024in}{2.692305in}}%
\pgfpathlineto{\pgfqpoint{4.106450in}{2.688372in}}%
\pgfpathlineto{\pgfqpoint{4.114190in}{2.700600in}}%
\pgfpathlineto{\pgfqpoint{4.121925in}{2.712933in}}%
\pgfpathlineto{\pgfqpoint{4.129656in}{2.725377in}}%
\pgfpathlineto{\pgfqpoint{4.137383in}{2.737935in}}%
\pgfpathlineto{\pgfqpoint{4.123965in}{2.742213in}}%
\pgfpathlineto{\pgfqpoint{4.110553in}{2.746698in}}%
\pgfpathlineto{\pgfqpoint{4.097146in}{2.751390in}}%
\pgfpathlineto{\pgfqpoint{4.083745in}{2.756291in}}%
\pgfpathlineto{\pgfqpoint{4.076010in}{2.743375in}}%
\pgfpathlineto{\pgfqpoint{4.068271in}{2.730582in}}%
\pgfpathlineto{\pgfqpoint{4.060528in}{2.717907in}}%
\pgfpathlineto{\pgfqpoint{4.052781in}{2.705346in}}%
\pgfpathclose%
\pgfusepath{fill}%
\end{pgfscope}%
\begin{pgfscope}%
\pgfpathrectangle{\pgfqpoint{1.150000in}{0.150000in}}{\pgfqpoint{5.700000in}{5.700000in}}%
\pgfusepath{clip}%
\pgfsetbuttcap%
\pgfsetroundjoin%
\definecolor{currentfill}{rgb}{0.185556,0.418570,0.556753}%
\pgfsetfillcolor{currentfill}%
\pgfsetfillopacity{0.800000}%
\pgfsetlinewidth{0.000000pt}%
\definecolor{currentstroke}{rgb}{0.000000,0.000000,0.000000}%
\pgfsetstrokecolor{currentstroke}%
\pgfsetdash{}{0pt}%
\pgfpathmoveto{\pgfqpoint{2.985241in}{3.257395in}}%
\pgfpathlineto{\pgfqpoint{2.998771in}{3.236454in}}%
\pgfpathlineto{\pgfqpoint{3.012291in}{3.215847in}}%
\pgfpathlineto{\pgfqpoint{3.025803in}{3.195571in}}%
\pgfpathlineto{\pgfqpoint{3.039305in}{3.175624in}}%
\pgfpathlineto{\pgfqpoint{3.047300in}{3.189502in}}%
\pgfpathlineto{\pgfqpoint{3.055287in}{3.203554in}}%
\pgfpathlineto{\pgfqpoint{3.063267in}{3.217783in}}%
\pgfpathlineto{\pgfqpoint{3.071240in}{3.232193in}}%
\pgfpathlineto{\pgfqpoint{3.057749in}{3.252334in}}%
\pgfpathlineto{\pgfqpoint{3.044249in}{3.272803in}}%
\pgfpathlineto{\pgfqpoint{3.030740in}{3.293604in}}%
\pgfpathlineto{\pgfqpoint{3.017222in}{3.314739in}}%
\pgfpathlineto{\pgfqpoint{3.009238in}{3.300123in}}%
\pgfpathlineto{\pgfqpoint{3.001247in}{3.285696in}}%
\pgfpathlineto{\pgfqpoint{2.993248in}{3.271454in}}%
\pgfpathlineto{\pgfqpoint{2.985241in}{3.257395in}}%
\pgfpathclose%
\pgfusepath{fill}%
\end{pgfscope}%
\begin{pgfscope}%
\pgfpathrectangle{\pgfqpoint{1.150000in}{0.150000in}}{\pgfqpoint{5.700000in}{5.700000in}}%
\pgfusepath{clip}%
\pgfsetbuttcap%
\pgfsetroundjoin%
\definecolor{currentfill}{rgb}{0.175841,0.441290,0.557685}%
\pgfsetfillcolor{currentfill}%
\pgfsetfillopacity{0.800000}%
\pgfsetlinewidth{0.000000pt}%
\definecolor{currentstroke}{rgb}{0.000000,0.000000,0.000000}%
\pgfsetstrokecolor{currentstroke}%
\pgfsetdash{}{0pt}%
\pgfpathmoveto{\pgfqpoint{5.152609in}{3.285672in}}%
\pgfpathlineto{\pgfqpoint{5.166300in}{3.284061in}}%
\pgfpathlineto{\pgfqpoint{5.180001in}{3.282625in}}%
\pgfpathlineto{\pgfqpoint{5.193714in}{3.281365in}}%
\pgfpathlineto{\pgfqpoint{5.207437in}{3.280280in}}%
\pgfpathlineto{\pgfqpoint{5.214946in}{3.295755in}}%
\pgfpathlineto{\pgfqpoint{5.222458in}{3.311564in}}%
\pgfpathlineto{\pgfqpoint{5.229975in}{3.327716in}}%
\pgfpathlineto{\pgfqpoint{5.216266in}{3.329365in}}%
\pgfpathlineto{\pgfqpoint{5.202568in}{3.331188in}}%
\pgfpathlineto{\pgfqpoint{5.188881in}{3.333187in}}%
\pgfpathlineto{\pgfqpoint{5.175204in}{3.335361in}}%
\pgfpathlineto{\pgfqpoint{5.167668in}{3.318450in}}%
\pgfpathlineto{\pgfqpoint{5.160136in}{3.301890in}}%
\pgfpathlineto{\pgfqpoint{5.152609in}{3.285672in}}%
\pgfpathclose%
\pgfusepath{fill}%
\end{pgfscope}%
\begin{pgfscope}%
\pgfpathrectangle{\pgfqpoint{1.150000in}{0.150000in}}{\pgfqpoint{5.700000in}{5.700000in}}%
\pgfusepath{clip}%
\pgfsetbuttcap%
\pgfsetroundjoin%
\definecolor{currentfill}{rgb}{0.275191,0.194905,0.496005}%
\pgfsetfillcolor{currentfill}%
\pgfsetfillopacity{0.800000}%
\pgfsetlinewidth{0.000000pt}%
\definecolor{currentstroke}{rgb}{0.000000,0.000000,0.000000}%
\pgfsetstrokecolor{currentstroke}%
\pgfsetdash{}{0pt}%
\pgfpathmoveto{\pgfqpoint{3.606868in}{2.665648in}}%
\pgfpathlineto{\pgfqpoint{3.620226in}{2.656838in}}%
\pgfpathlineto{\pgfqpoint{3.633586in}{2.648264in}}%
\pgfpathlineto{\pgfqpoint{3.646947in}{2.639925in}}%
\pgfpathlineto{\pgfqpoint{3.660309in}{2.631819in}}%
\pgfpathlineto{\pgfqpoint{3.668172in}{2.644098in}}%
\pgfpathlineto{\pgfqpoint{3.676029in}{2.656475in}}%
\pgfpathlineto{\pgfqpoint{3.683881in}{2.668953in}}%
\pgfpathlineto{\pgfqpoint{3.691728in}{2.681534in}}%
\pgfpathlineto{\pgfqpoint{3.678373in}{2.689860in}}%
\pgfpathlineto{\pgfqpoint{3.665021in}{2.698420in}}%
\pgfpathlineto{\pgfqpoint{3.651670in}{2.707214in}}%
\pgfpathlineto{\pgfqpoint{3.638320in}{2.716245in}}%
\pgfpathlineto{\pgfqpoint{3.630465in}{2.703431in}}%
\pgfpathlineto{\pgfqpoint{3.622605in}{2.690729in}}%
\pgfpathlineto{\pgfqpoint{3.614739in}{2.678135in}}%
\pgfpathlineto{\pgfqpoint{3.606868in}{2.665648in}}%
\pgfpathclose%
\pgfusepath{fill}%
\end{pgfscope}%
\begin{pgfscope}%
\pgfpathrectangle{\pgfqpoint{1.150000in}{0.150000in}}{\pgfqpoint{5.700000in}{5.700000in}}%
\pgfusepath{clip}%
\pgfsetbuttcap%
\pgfsetroundjoin%
\definecolor{currentfill}{rgb}{0.276194,0.190074,0.493001}%
\pgfsetfillcolor{currentfill}%
\pgfsetfillopacity{0.800000}%
\pgfsetlinewidth{0.000000pt}%
\definecolor{currentstroke}{rgb}{0.000000,0.000000,0.000000}%
\pgfsetstrokecolor{currentstroke}%
\pgfsetdash{}{0pt}%
\pgfpathmoveto{\pgfqpoint{3.745166in}{2.650537in}}%
\pgfpathlineto{\pgfqpoint{3.758532in}{2.643359in}}%
\pgfpathlineto{\pgfqpoint{3.771901in}{2.636405in}}%
\pgfpathlineto{\pgfqpoint{3.785273in}{2.629677in}}%
\pgfpathlineto{\pgfqpoint{3.798649in}{2.623171in}}%
\pgfpathlineto{\pgfqpoint{3.806474in}{2.635379in}}%
\pgfpathlineto{\pgfqpoint{3.814295in}{2.647681in}}%
\pgfpathlineto{\pgfqpoint{3.822111in}{2.660080in}}%
\pgfpathlineto{\pgfqpoint{3.829922in}{2.672580in}}%
\pgfpathlineto{\pgfqpoint{3.816555in}{2.679338in}}%
\pgfpathlineto{\pgfqpoint{3.803191in}{2.686319in}}%
\pgfpathlineto{\pgfqpoint{3.789830in}{2.693524in}}%
\pgfpathlineto{\pgfqpoint{3.776472in}{2.700954in}}%
\pgfpathlineto{\pgfqpoint{3.768653in}{2.688191in}}%
\pgfpathlineto{\pgfqpoint{3.760829in}{2.675536in}}%
\pgfpathlineto{\pgfqpoint{3.753000in}{2.662986in}}%
\pgfpathlineto{\pgfqpoint{3.745166in}{2.650537in}}%
\pgfpathclose%
\pgfusepath{fill}%
\end{pgfscope}%
\begin{pgfscope}%
\pgfpathrectangle{\pgfqpoint{1.150000in}{0.150000in}}{\pgfqpoint{5.700000in}{5.700000in}}%
\pgfusepath{clip}%
\pgfsetbuttcap%
\pgfsetroundjoin%
\definecolor{currentfill}{rgb}{0.273006,0.204520,0.501721}%
\pgfsetfillcolor{currentfill}%
\pgfsetfillopacity{0.800000}%
\pgfsetlinewidth{0.000000pt}%
\definecolor{currentstroke}{rgb}{0.000000,0.000000,0.000000}%
\pgfsetstrokecolor{currentstroke}%
\pgfsetdash{}{0pt}%
\pgfpathmoveto{\pgfqpoint{3.968133in}{2.675222in}}%
\pgfpathlineto{\pgfqpoint{3.981529in}{2.670139in}}%
\pgfpathlineto{\pgfqpoint{3.994931in}{2.665268in}}%
\pgfpathlineto{\pgfqpoint{4.008337in}{2.660608in}}%
\pgfpathlineto{\pgfqpoint{4.021748in}{2.656158in}}%
\pgfpathlineto{\pgfqpoint{4.029513in}{2.668304in}}%
\pgfpathlineto{\pgfqpoint{4.037273in}{2.680548in}}%
\pgfpathlineto{\pgfqpoint{4.045029in}{2.692894in}}%
\pgfpathlineto{\pgfqpoint{4.052781in}{2.705346in}}%
\pgfpathlineto{\pgfqpoint{4.039377in}{2.710110in}}%
\pgfpathlineto{\pgfqpoint{4.025979in}{2.715084in}}%
\pgfpathlineto{\pgfqpoint{4.012586in}{2.720270in}}%
\pgfpathlineto{\pgfqpoint{3.999197in}{2.725667in}}%
\pgfpathlineto{\pgfqpoint{3.991438in}{2.712890in}}%
\pgfpathlineto{\pgfqpoint{3.983674in}{2.700225in}}%
\pgfpathlineto{\pgfqpoint{3.975906in}{2.687671in}}%
\pgfpathlineto{\pgfqpoint{3.968133in}{2.675222in}}%
\pgfpathclose%
\pgfusepath{fill}%
\end{pgfscope}%
\begin{pgfscope}%
\pgfpathrectangle{\pgfqpoint{1.150000in}{0.150000in}}{\pgfqpoint{5.700000in}{5.700000in}}%
\pgfusepath{clip}%
\pgfsetbuttcap%
\pgfsetroundjoin%
\definecolor{currentfill}{rgb}{0.233603,0.313828,0.543914}%
\pgfsetfillcolor{currentfill}%
\pgfsetfillopacity{0.800000}%
\pgfsetlinewidth{0.000000pt}%
\definecolor{currentstroke}{rgb}{0.000000,0.000000,0.000000}%
\pgfsetstrokecolor{currentstroke}%
\pgfsetdash{}{0pt}%
\pgfpathmoveto{\pgfqpoint{4.614217in}{2.924756in}}%
\pgfpathlineto{\pgfqpoint{4.627775in}{2.923187in}}%
\pgfpathlineto{\pgfqpoint{4.641341in}{2.921806in}}%
\pgfpathlineto{\pgfqpoint{4.654917in}{2.920612in}}%
\pgfpathlineto{\pgfqpoint{4.668502in}{2.919605in}}%
\pgfpathlineto{\pgfqpoint{4.676093in}{2.932035in}}%
\pgfpathlineto{\pgfqpoint{4.683681in}{2.944638in}}%
\pgfpathlineto{\pgfqpoint{4.691267in}{2.957423in}}%
\pgfpathlineto{\pgfqpoint{4.698851in}{2.970393in}}%
\pgfpathlineto{\pgfqpoint{4.685279in}{2.971935in}}%
\pgfpathlineto{\pgfqpoint{4.671715in}{2.973664in}}%
\pgfpathlineto{\pgfqpoint{4.658160in}{2.975579in}}%
\pgfpathlineto{\pgfqpoint{4.644615in}{2.977682in}}%
\pgfpathlineto{\pgfqpoint{4.637018in}{2.964165in}}%
\pgfpathlineto{\pgfqpoint{4.629420in}{2.950843in}}%
\pgfpathlineto{\pgfqpoint{4.621819in}{2.937709in}}%
\pgfpathlineto{\pgfqpoint{4.614217in}{2.924756in}}%
\pgfpathclose%
\pgfusepath{fill}%
\end{pgfscope}%
\begin{pgfscope}%
\pgfpathrectangle{\pgfqpoint{1.150000in}{0.150000in}}{\pgfqpoint{5.700000in}{5.700000in}}%
\pgfusepath{clip}%
\pgfsetbuttcap%
\pgfsetroundjoin%
\definecolor{currentfill}{rgb}{0.243113,0.292092,0.538516}%
\pgfsetfillcolor{currentfill}%
\pgfsetfillopacity{0.800000}%
\pgfsetlinewidth{0.000000pt}%
\definecolor{currentstroke}{rgb}{0.000000,0.000000,0.000000}%
\pgfsetstrokecolor{currentstroke}%
\pgfsetdash{}{0pt}%
\pgfpathmoveto{\pgfqpoint{4.529598in}{2.880795in}}%
\pgfpathlineto{\pgfqpoint{4.543131in}{2.878972in}}%
\pgfpathlineto{\pgfqpoint{4.556674in}{2.877340in}}%
\pgfpathlineto{\pgfqpoint{4.570225in}{2.875897in}}%
\pgfpathlineto{\pgfqpoint{4.583785in}{2.874643in}}%
\pgfpathlineto{\pgfqpoint{4.591397in}{2.886928in}}%
\pgfpathlineto{\pgfqpoint{4.599006in}{2.899372in}}%
\pgfpathlineto{\pgfqpoint{4.606613in}{2.911979in}}%
\pgfpathlineto{\pgfqpoint{4.614217in}{2.924756in}}%
\pgfpathlineto{\pgfqpoint{4.600668in}{2.926513in}}%
\pgfpathlineto{\pgfqpoint{4.587129in}{2.928459in}}%
\pgfpathlineto{\pgfqpoint{4.573597in}{2.930594in}}%
\pgfpathlineto{\pgfqpoint{4.560075in}{2.932919in}}%
\pgfpathlineto{\pgfqpoint{4.552459in}{2.919628in}}%
\pgfpathlineto{\pgfqpoint{4.544841in}{2.906514in}}%
\pgfpathlineto{\pgfqpoint{4.537221in}{2.893572in}}%
\pgfpathlineto{\pgfqpoint{4.529598in}{2.880795in}}%
\pgfpathclose%
\pgfusepath{fill}%
\end{pgfscope}%
\begin{pgfscope}%
\pgfpathrectangle{\pgfqpoint{1.150000in}{0.150000in}}{\pgfqpoint{5.700000in}{5.700000in}}%
\pgfusepath{clip}%
\pgfsetbuttcap%
\pgfsetroundjoin%
\definecolor{currentfill}{rgb}{0.225863,0.330805,0.547314}%
\pgfsetfillcolor{currentfill}%
\pgfsetfillopacity{0.800000}%
\pgfsetlinewidth{0.000000pt}%
\definecolor{currentstroke}{rgb}{0.000000,0.000000,0.000000}%
\pgfsetstrokecolor{currentstroke}%
\pgfsetdash{}{0pt}%
\pgfpathmoveto{\pgfqpoint{4.698851in}{2.970393in}}%
\pgfpathlineto{\pgfqpoint{4.712434in}{2.969037in}}%
\pgfpathlineto{\pgfqpoint{4.726025in}{2.967867in}}%
\pgfpathlineto{\pgfqpoint{4.739626in}{2.966881in}}%
\pgfpathlineto{\pgfqpoint{4.753237in}{2.966080in}}%
\pgfpathlineto{\pgfqpoint{4.760807in}{2.978689in}}%
\pgfpathlineto{\pgfqpoint{4.768376in}{2.991489in}}%
\pgfpathlineto{\pgfqpoint{4.775943in}{3.004488in}}%
\pgfpathlineto{\pgfqpoint{4.783509in}{3.017691in}}%
\pgfpathlineto{\pgfqpoint{4.769911in}{3.019058in}}%
\pgfpathlineto{\pgfqpoint{4.756323in}{3.020610in}}%
\pgfpathlineto{\pgfqpoint{4.742744in}{3.022346in}}%
\pgfpathlineto{\pgfqpoint{4.729175in}{3.024268in}}%
\pgfpathlineto{\pgfqpoint{4.721596in}{3.010487in}}%
\pgfpathlineto{\pgfqpoint{4.714016in}{2.996919in}}%
\pgfpathlineto{\pgfqpoint{4.706434in}{2.983557in}}%
\pgfpathlineto{\pgfqpoint{4.698851in}{2.970393in}}%
\pgfpathclose%
\pgfusepath{fill}%
\end{pgfscope}%
\begin{pgfscope}%
\pgfpathrectangle{\pgfqpoint{1.150000in}{0.150000in}}{\pgfqpoint{5.700000in}{5.700000in}}%
\pgfusepath{clip}%
\pgfsetbuttcap%
\pgfsetroundjoin%
\definecolor{currentfill}{rgb}{0.271828,0.209303,0.504434}%
\pgfsetfillcolor{currentfill}%
\pgfsetfillopacity{0.800000}%
\pgfsetlinewidth{0.000000pt}%
\definecolor{currentstroke}{rgb}{0.000000,0.000000,0.000000}%
\pgfsetstrokecolor{currentstroke}%
\pgfsetdash{}{0pt}%
\pgfpathmoveto{\pgfqpoint{3.468422in}{2.694358in}}%
\pgfpathlineto{\pgfqpoint{3.481785in}{2.683791in}}%
\pgfpathlineto{\pgfqpoint{3.495148in}{2.673474in}}%
\pgfpathlineto{\pgfqpoint{3.508511in}{2.663404in}}%
\pgfpathlineto{\pgfqpoint{3.521875in}{2.653579in}}%
\pgfpathlineto{\pgfqpoint{3.529776in}{2.665847in}}%
\pgfpathlineto{\pgfqpoint{3.537671in}{2.678218in}}%
\pgfpathlineto{\pgfqpoint{3.545561in}{2.690696in}}%
\pgfpathlineto{\pgfqpoint{3.553446in}{2.703281in}}%
\pgfpathlineto{\pgfqpoint{3.540092in}{2.713296in}}%
\pgfpathlineto{\pgfqpoint{3.526738in}{2.723555in}}%
\pgfpathlineto{\pgfqpoint{3.513384in}{2.734062in}}%
\pgfpathlineto{\pgfqpoint{3.500030in}{2.744818in}}%
\pgfpathlineto{\pgfqpoint{3.492136in}{2.732031in}}%
\pgfpathlineto{\pgfqpoint{3.484237in}{2.719360in}}%
\pgfpathlineto{\pgfqpoint{3.476332in}{2.706803in}}%
\pgfpathlineto{\pgfqpoint{3.468422in}{2.694358in}}%
\pgfpathclose%
\pgfusepath{fill}%
\end{pgfscope}%
\begin{pgfscope}%
\pgfpathrectangle{\pgfqpoint{1.150000in}{0.150000in}}{\pgfqpoint{5.700000in}{5.700000in}}%
\pgfusepath{clip}%
\pgfsetbuttcap%
\pgfsetroundjoin%
\definecolor{currentfill}{rgb}{0.248629,0.278775,0.534556}%
\pgfsetfillcolor{currentfill}%
\pgfsetfillopacity{0.800000}%
\pgfsetlinewidth{0.000000pt}%
\definecolor{currentstroke}{rgb}{0.000000,0.000000,0.000000}%
\pgfsetstrokecolor{currentstroke}%
\pgfsetdash{}{0pt}%
\pgfpathmoveto{\pgfqpoint{4.444986in}{2.838549in}}%
\pgfpathlineto{\pgfqpoint{4.458497in}{2.836432in}}%
\pgfpathlineto{\pgfqpoint{4.472016in}{2.834506in}}%
\pgfpathlineto{\pgfqpoint{4.485543in}{2.832773in}}%
\pgfpathlineto{\pgfqpoint{4.499079in}{2.831232in}}%
\pgfpathlineto{\pgfqpoint{4.506713in}{2.843402in}}%
\pgfpathlineto{\pgfqpoint{4.514344in}{2.855716in}}%
\pgfpathlineto{\pgfqpoint{4.521972in}{2.868178in}}%
\pgfpathlineto{\pgfqpoint{4.529598in}{2.880795in}}%
\pgfpathlineto{\pgfqpoint{4.516073in}{2.882808in}}%
\pgfpathlineto{\pgfqpoint{4.502556in}{2.885013in}}%
\pgfpathlineto{\pgfqpoint{4.489047in}{2.887409in}}%
\pgfpathlineto{\pgfqpoint{4.475547in}{2.889998in}}%
\pgfpathlineto{\pgfqpoint{4.467911in}{2.876898in}}%
\pgfpathlineto{\pgfqpoint{4.460272in}{2.863961in}}%
\pgfpathlineto{\pgfqpoint{4.452631in}{2.851179in}}%
\pgfpathlineto{\pgfqpoint{4.444986in}{2.838549in}}%
\pgfpathclose%
\pgfusepath{fill}%
\end{pgfscope}%
\begin{pgfscope}%
\pgfpathrectangle{\pgfqpoint{1.150000in}{0.150000in}}{\pgfqpoint{5.700000in}{5.700000in}}%
\pgfusepath{clip}%
\pgfsetbuttcap%
\pgfsetroundjoin%
\definecolor{currentfill}{rgb}{0.218130,0.347432,0.550038}%
\pgfsetfillcolor{currentfill}%
\pgfsetfillopacity{0.800000}%
\pgfsetlinewidth{0.000000pt}%
\definecolor{currentstroke}{rgb}{0.000000,0.000000,0.000000}%
\pgfsetstrokecolor{currentstroke}%
\pgfsetdash{}{0pt}%
\pgfpathmoveto{\pgfqpoint{4.783509in}{3.017691in}}%
\pgfpathlineto{\pgfqpoint{4.797116in}{3.016508in}}%
\pgfpathlineto{\pgfqpoint{4.810733in}{3.015507in}}%
\pgfpathlineto{\pgfqpoint{4.824360in}{3.014690in}}%
\pgfpathlineto{\pgfqpoint{4.837997in}{3.014055in}}%
\pgfpathlineto{\pgfqpoint{4.845548in}{3.026883in}}%
\pgfpathlineto{\pgfqpoint{4.853098in}{3.039922in}}%
\pgfpathlineto{\pgfqpoint{4.860648in}{3.053179in}}%
\pgfpathlineto{\pgfqpoint{4.868197in}{3.066659in}}%
\pgfpathlineto{\pgfqpoint{4.854574in}{3.067892in}}%
\pgfpathlineto{\pgfqpoint{4.840961in}{3.069307in}}%
\pgfpathlineto{\pgfqpoint{4.827358in}{3.070904in}}%
\pgfpathlineto{\pgfqpoint{4.813765in}{3.072685in}}%
\pgfpathlineto{\pgfqpoint{4.806201in}{3.058595in}}%
\pgfpathlineto{\pgfqpoint{4.798638in}{3.044738in}}%
\pgfpathlineto{\pgfqpoint{4.791074in}{3.031106in}}%
\pgfpathlineto{\pgfqpoint{4.783509in}{3.017691in}}%
\pgfpathclose%
\pgfusepath{fill}%
\end{pgfscope}%
\begin{pgfscope}%
\pgfpathrectangle{\pgfqpoint{1.150000in}{0.150000in}}{\pgfqpoint{5.700000in}{5.700000in}}%
\pgfusepath{clip}%
\pgfsetbuttcap%
\pgfsetroundjoin%
\definecolor{currentfill}{rgb}{0.241237,0.296485,0.539709}%
\pgfsetfillcolor{currentfill}%
\pgfsetfillopacity{0.800000}%
\pgfsetlinewidth{0.000000pt}%
\definecolor{currentstroke}{rgb}{0.000000,0.000000,0.000000}%
\pgfsetstrokecolor{currentstroke}%
\pgfsetdash{}{0pt}%
\pgfpathmoveto{\pgfqpoint{3.168853in}{2.908895in}}%
\pgfpathlineto{\pgfqpoint{3.182280in}{2.893110in}}%
\pgfpathlineto{\pgfqpoint{3.195702in}{2.877615in}}%
\pgfpathlineto{\pgfqpoint{3.209119in}{2.862407in}}%
\pgfpathlineto{\pgfqpoint{3.222531in}{2.847483in}}%
\pgfpathlineto{\pgfqpoint{3.230507in}{2.860094in}}%
\pgfpathlineto{\pgfqpoint{3.238476in}{2.872838in}}%
\pgfpathlineto{\pgfqpoint{3.246439in}{2.885718in}}%
\pgfpathlineto{\pgfqpoint{3.254396in}{2.898736in}}%
\pgfpathlineto{\pgfqpoint{3.240994in}{2.913818in}}%
\pgfpathlineto{\pgfqpoint{3.227588in}{2.929185in}}%
\pgfpathlineto{\pgfqpoint{3.214178in}{2.944840in}}%
\pgfpathlineto{\pgfqpoint{3.200763in}{2.960783in}}%
\pgfpathlineto{\pgfqpoint{3.192795in}{2.947594in}}%
\pgfpathlineto{\pgfqpoint{3.184821in}{2.934551in}}%
\pgfpathlineto{\pgfqpoint{3.176841in}{2.921652in}}%
\pgfpathlineto{\pgfqpoint{3.168853in}{2.908895in}}%
\pgfpathclose%
\pgfusepath{fill}%
\end{pgfscope}%
\begin{pgfscope}%
\pgfpathrectangle{\pgfqpoint{1.150000in}{0.150000in}}{\pgfqpoint{5.700000in}{5.700000in}}%
\pgfusepath{clip}%
\pgfsetbuttcap%
\pgfsetroundjoin%
\definecolor{currentfill}{rgb}{0.229739,0.322361,0.545706}%
\pgfsetfillcolor{currentfill}%
\pgfsetfillopacity{0.800000}%
\pgfsetlinewidth{0.000000pt}%
\definecolor{currentstroke}{rgb}{0.000000,0.000000,0.000000}%
\pgfsetstrokecolor{currentstroke}%
\pgfsetdash{}{0pt}%
\pgfpathmoveto{\pgfqpoint{3.115093in}{2.974976in}}%
\pgfpathlineto{\pgfqpoint{3.128542in}{2.958009in}}%
\pgfpathlineto{\pgfqpoint{3.141984in}{2.941342in}}%
\pgfpathlineto{\pgfqpoint{3.155422in}{2.924971in}}%
\pgfpathlineto{\pgfqpoint{3.168853in}{2.908895in}}%
\pgfpathlineto{\pgfqpoint{3.176841in}{2.921652in}}%
\pgfpathlineto{\pgfqpoint{3.184821in}{2.934551in}}%
\pgfpathlineto{\pgfqpoint{3.192795in}{2.947594in}}%
\pgfpathlineto{\pgfqpoint{3.200763in}{2.960783in}}%
\pgfpathlineto{\pgfqpoint{3.187342in}{2.977019in}}%
\pgfpathlineto{\pgfqpoint{3.173917in}{2.993549in}}%
\pgfpathlineto{\pgfqpoint{3.160485in}{3.010376in}}%
\pgfpathlineto{\pgfqpoint{3.147048in}{3.027502in}}%
\pgfpathlineto{\pgfqpoint{3.139070in}{3.014142in}}%
\pgfpathlineto{\pgfqpoint{3.131084in}{3.000935in}}%
\pgfpathlineto{\pgfqpoint{3.123092in}{2.987881in}}%
\pgfpathlineto{\pgfqpoint{3.115093in}{2.974976in}}%
\pgfpathclose%
\pgfusepath{fill}%
\end{pgfscope}%
\begin{pgfscope}%
\pgfpathrectangle{\pgfqpoint{1.150000in}{0.150000in}}{\pgfqpoint{5.700000in}{5.700000in}}%
\pgfusepath{clip}%
\pgfsetbuttcap%
\pgfsetroundjoin%
\definecolor{currentfill}{rgb}{0.255645,0.260703,0.528312}%
\pgfsetfillcolor{currentfill}%
\pgfsetfillopacity{0.800000}%
\pgfsetlinewidth{0.000000pt}%
\definecolor{currentstroke}{rgb}{0.000000,0.000000,0.000000}%
\pgfsetstrokecolor{currentstroke}%
\pgfsetdash{}{0pt}%
\pgfpathmoveto{\pgfqpoint{4.360374in}{2.798083in}}%
\pgfpathlineto{\pgfqpoint{4.373863in}{2.795628in}}%
\pgfpathlineto{\pgfqpoint{4.387359in}{2.793368in}}%
\pgfpathlineto{\pgfqpoint{4.400864in}{2.791304in}}%
\pgfpathlineto{\pgfqpoint{4.414376in}{2.789433in}}%
\pgfpathlineto{\pgfqpoint{4.422034in}{2.801512in}}%
\pgfpathlineto{\pgfqpoint{4.429688in}{2.813721in}}%
\pgfpathlineto{\pgfqpoint{4.437338in}{2.826065in}}%
\pgfpathlineto{\pgfqpoint{4.444986in}{2.838549in}}%
\pgfpathlineto{\pgfqpoint{4.431483in}{2.840860in}}%
\pgfpathlineto{\pgfqpoint{4.417989in}{2.843365in}}%
\pgfpathlineto{\pgfqpoint{4.404502in}{2.846065in}}%
\pgfpathlineto{\pgfqpoint{4.391023in}{2.848960in}}%
\pgfpathlineto{\pgfqpoint{4.383365in}{2.836024in}}%
\pgfpathlineto{\pgfqpoint{4.375705in}{2.823236in}}%
\pgfpathlineto{\pgfqpoint{4.368041in}{2.810591in}}%
\pgfpathlineto{\pgfqpoint{4.360374in}{2.798083in}}%
\pgfpathclose%
\pgfusepath{fill}%
\end{pgfscope}%
\begin{pgfscope}%
\pgfpathrectangle{\pgfqpoint{1.150000in}{0.150000in}}{\pgfqpoint{5.700000in}{5.700000in}}%
\pgfusepath{clip}%
\pgfsetbuttcap%
\pgfsetroundjoin%
\definecolor{currentfill}{rgb}{0.252194,0.269783,0.531579}%
\pgfsetfillcolor{currentfill}%
\pgfsetfillopacity{0.800000}%
\pgfsetlinewidth{0.000000pt}%
\definecolor{currentstroke}{rgb}{0.000000,0.000000,0.000000}%
\pgfsetstrokecolor{currentstroke}%
\pgfsetdash{}{0pt}%
\pgfpathmoveto{\pgfqpoint{3.222531in}{2.847483in}}%
\pgfpathlineto{\pgfqpoint{3.235939in}{2.832842in}}%
\pgfpathlineto{\pgfqpoint{3.249344in}{2.818480in}}%
\pgfpathlineto{\pgfqpoint{3.262745in}{2.804397in}}%
\pgfpathlineto{\pgfqpoint{3.276142in}{2.790589in}}%
\pgfpathlineto{\pgfqpoint{3.284106in}{2.803052in}}%
\pgfpathlineto{\pgfqpoint{3.292065in}{2.815642in}}%
\pgfpathlineto{\pgfqpoint{3.300017in}{2.828359in}}%
\pgfpathlineto{\pgfqpoint{3.307963in}{2.841206in}}%
\pgfpathlineto{\pgfqpoint{3.294577in}{2.855173in}}%
\pgfpathlineto{\pgfqpoint{3.281187in}{2.869415in}}%
\pgfpathlineto{\pgfqpoint{3.267793in}{2.883935in}}%
\pgfpathlineto{\pgfqpoint{3.254396in}{2.898736in}}%
\pgfpathlineto{\pgfqpoint{3.246439in}{2.885718in}}%
\pgfpathlineto{\pgfqpoint{3.238476in}{2.872838in}}%
\pgfpathlineto{\pgfqpoint{3.230507in}{2.860094in}}%
\pgfpathlineto{\pgfqpoint{3.222531in}{2.847483in}}%
\pgfpathclose%
\pgfusepath{fill}%
\end{pgfscope}%
\begin{pgfscope}%
\pgfpathrectangle{\pgfqpoint{1.150000in}{0.150000in}}{\pgfqpoint{5.700000in}{5.700000in}}%
\pgfusepath{clip}%
\pgfsetbuttcap%
\pgfsetroundjoin%
\definecolor{currentfill}{rgb}{0.208623,0.367752,0.552675}%
\pgfsetfillcolor{currentfill}%
\pgfsetfillopacity{0.800000}%
\pgfsetlinewidth{0.000000pt}%
\definecolor{currentstroke}{rgb}{0.000000,0.000000,0.000000}%
\pgfsetstrokecolor{currentstroke}%
\pgfsetdash{}{0pt}%
\pgfpathmoveto{\pgfqpoint{4.868197in}{3.066659in}}%
\pgfpathlineto{\pgfqpoint{4.881830in}{3.065608in}}%
\pgfpathlineto{\pgfqpoint{4.895473in}{3.064738in}}%
\pgfpathlineto{\pgfqpoint{4.909126in}{3.064050in}}%
\pgfpathlineto{\pgfqpoint{4.922789in}{3.063542in}}%
\pgfpathlineto{\pgfqpoint{4.930323in}{3.076636in}}%
\pgfpathlineto{\pgfqpoint{4.937857in}{3.089960in}}%
\pgfpathlineto{\pgfqpoint{4.945390in}{3.103523in}}%
\pgfpathlineto{\pgfqpoint{4.952924in}{3.117331in}}%
\pgfpathlineto{\pgfqpoint{4.939276in}{3.118469in}}%
\pgfpathlineto{\pgfqpoint{4.925638in}{3.119786in}}%
\pgfpathlineto{\pgfqpoint{4.912011in}{3.121285in}}%
\pgfpathlineto{\pgfqpoint{4.898393in}{3.122964in}}%
\pgfpathlineto{\pgfqpoint{4.890843in}{3.108516in}}%
\pgfpathlineto{\pgfqpoint{4.883294in}{3.094321in}}%
\pgfpathlineto{\pgfqpoint{4.875746in}{3.080371in}}%
\pgfpathlineto{\pgfqpoint{4.868197in}{3.066659in}}%
\pgfpathclose%
\pgfusepath{fill}%
\end{pgfscope}%
\begin{pgfscope}%
\pgfpathrectangle{\pgfqpoint{1.150000in}{0.150000in}}{\pgfqpoint{5.700000in}{5.700000in}}%
\pgfusepath{clip}%
\pgfsetbuttcap%
\pgfsetroundjoin%
\definecolor{currentfill}{rgb}{0.262138,0.242286,0.520837}%
\pgfsetfillcolor{currentfill}%
\pgfsetfillopacity{0.800000}%
\pgfsetlinewidth{0.000000pt}%
\definecolor{currentstroke}{rgb}{0.000000,0.000000,0.000000}%
\pgfsetstrokecolor{currentstroke}%
\pgfsetdash{}{0pt}%
\pgfpathmoveto{\pgfqpoint{4.275753in}{2.759485in}}%
\pgfpathlineto{\pgfqpoint{4.289221in}{2.756650in}}%
\pgfpathlineto{\pgfqpoint{4.302697in}{2.754013in}}%
\pgfpathlineto{\pgfqpoint{4.316180in}{2.751574in}}%
\pgfpathlineto{\pgfqpoint{4.329670in}{2.749332in}}%
\pgfpathlineto{\pgfqpoint{4.337352in}{2.761338in}}%
\pgfpathlineto{\pgfqpoint{4.345029in}{2.773462in}}%
\pgfpathlineto{\pgfqpoint{4.352703in}{2.785709in}}%
\pgfpathlineto{\pgfqpoint{4.360374in}{2.798083in}}%
\pgfpathlineto{\pgfqpoint{4.346893in}{2.800734in}}%
\pgfpathlineto{\pgfqpoint{4.333419in}{2.803582in}}%
\pgfpathlineto{\pgfqpoint{4.319953in}{2.806628in}}%
\pgfpathlineto{\pgfqpoint{4.306494in}{2.809871in}}%
\pgfpathlineto{\pgfqpoint{4.298814in}{2.797076in}}%
\pgfpathlineto{\pgfqpoint{4.291131in}{2.784417in}}%
\pgfpathlineto{\pgfqpoint{4.283444in}{2.771888in}}%
\pgfpathlineto{\pgfqpoint{4.275753in}{2.759485in}}%
\pgfpathclose%
\pgfusepath{fill}%
\end{pgfscope}%
\begin{pgfscope}%
\pgfpathrectangle{\pgfqpoint{1.150000in}{0.150000in}}{\pgfqpoint{5.700000in}{5.700000in}}%
\pgfusepath{clip}%
\pgfsetbuttcap%
\pgfsetroundjoin%
\definecolor{currentfill}{rgb}{0.171176,0.452530,0.557965}%
\pgfsetfillcolor{currentfill}%
\pgfsetfillopacity{0.800000}%
\pgfsetlinewidth{0.000000pt}%
\definecolor{currentstroke}{rgb}{0.000000,0.000000,0.000000}%
\pgfsetstrokecolor{currentstroke}%
\pgfsetdash{}{0pt}%
\pgfpathmoveto{\pgfqpoint{2.931023in}{3.344576in}}%
\pgfpathlineto{\pgfqpoint{2.944593in}{3.322262in}}%
\pgfpathlineto{\pgfqpoint{2.958153in}{3.300296in}}%
\pgfpathlineto{\pgfqpoint{2.971702in}{3.278675in}}%
\pgfpathlineto{\pgfqpoint{2.985241in}{3.257395in}}%
\pgfpathlineto{\pgfqpoint{2.993248in}{3.271454in}}%
\pgfpathlineto{\pgfqpoint{3.001247in}{3.285696in}}%
\pgfpathlineto{\pgfqpoint{3.009238in}{3.300123in}}%
\pgfpathlineto{\pgfqpoint{3.017222in}{3.314739in}}%
\pgfpathlineto{\pgfqpoint{3.003695in}{3.336213in}}%
\pgfpathlineto{\pgfqpoint{2.990158in}{3.358029in}}%
\pgfpathlineto{\pgfqpoint{2.976610in}{3.380189in}}%
\pgfpathlineto{\pgfqpoint{2.963052in}{3.402698in}}%
\pgfpathlineto{\pgfqpoint{2.955057in}{3.387875in}}%
\pgfpathlineto{\pgfqpoint{2.947053in}{3.373249in}}%
\pgfpathlineto{\pgfqpoint{2.939042in}{3.358816in}}%
\pgfpathlineto{\pgfqpoint{2.931023in}{3.344576in}}%
\pgfpathclose%
\pgfusepath{fill}%
\end{pgfscope}%
\begin{pgfscope}%
\pgfpathrectangle{\pgfqpoint{1.150000in}{0.150000in}}{\pgfqpoint{5.700000in}{5.700000in}}%
\pgfusepath{clip}%
\pgfsetbuttcap%
\pgfsetroundjoin%
\definecolor{currentfill}{rgb}{0.218130,0.347432,0.550038}%
\pgfsetfillcolor{currentfill}%
\pgfsetfillopacity{0.800000}%
\pgfsetlinewidth{0.000000pt}%
\definecolor{currentstroke}{rgb}{0.000000,0.000000,0.000000}%
\pgfsetstrokecolor{currentstroke}%
\pgfsetdash{}{0pt}%
\pgfpathmoveto{\pgfqpoint{3.061232in}{3.045892in}}%
\pgfpathlineto{\pgfqpoint{3.074707in}{3.027700in}}%
\pgfpathlineto{\pgfqpoint{3.088176in}{3.009819in}}%
\pgfpathlineto{\pgfqpoint{3.101637in}{2.992245in}}%
\pgfpathlineto{\pgfqpoint{3.115093in}{2.974976in}}%
\pgfpathlineto{\pgfqpoint{3.123092in}{2.987881in}}%
\pgfpathlineto{\pgfqpoint{3.131084in}{3.000935in}}%
\pgfpathlineto{\pgfqpoint{3.139070in}{3.014142in}}%
\pgfpathlineto{\pgfqpoint{3.147048in}{3.027502in}}%
\pgfpathlineto{\pgfqpoint{3.133605in}{3.044931in}}%
\pgfpathlineto{\pgfqpoint{3.120155in}{3.062664in}}%
\pgfpathlineto{\pgfqpoint{3.106698in}{3.080705in}}%
\pgfpathlineto{\pgfqpoint{3.093235in}{3.099057in}}%
\pgfpathlineto{\pgfqpoint{3.085245in}{3.085524in}}%
\pgfpathlineto{\pgfqpoint{3.077248in}{3.072154in}}%
\pgfpathlineto{\pgfqpoint{3.069243in}{3.058944in}}%
\pgfpathlineto{\pgfqpoint{3.061232in}{3.045892in}}%
\pgfpathclose%
\pgfusepath{fill}%
\end{pgfscope}%
\begin{pgfscope}%
\pgfpathrectangle{\pgfqpoint{1.150000in}{0.150000in}}{\pgfqpoint{5.700000in}{5.700000in}}%
\pgfusepath{clip}%
\pgfsetbuttcap%
\pgfsetroundjoin%
\definecolor{currentfill}{rgb}{0.260571,0.246922,0.522828}%
\pgfsetfillcolor{currentfill}%
\pgfsetfillopacity{0.800000}%
\pgfsetlinewidth{0.000000pt}%
\definecolor{currentstroke}{rgb}{0.000000,0.000000,0.000000}%
\pgfsetstrokecolor{currentstroke}%
\pgfsetdash{}{0pt}%
\pgfpathmoveto{\pgfqpoint{3.276142in}{2.790589in}}%
\pgfpathlineto{\pgfqpoint{3.289536in}{2.777054in}}%
\pgfpathlineto{\pgfqpoint{3.302927in}{2.763791in}}%
\pgfpathlineto{\pgfqpoint{3.316315in}{2.750797in}}%
\pgfpathlineto{\pgfqpoint{3.329701in}{2.738070in}}%
\pgfpathlineto{\pgfqpoint{3.337655in}{2.750387in}}%
\pgfpathlineto{\pgfqpoint{3.345603in}{2.762821in}}%
\pgfpathlineto{\pgfqpoint{3.353544in}{2.775376in}}%
\pgfpathlineto{\pgfqpoint{3.361480in}{2.788053in}}%
\pgfpathlineto{\pgfqpoint{3.348105in}{2.800939in}}%
\pgfpathlineto{\pgfqpoint{3.334727in}{2.814091in}}%
\pgfpathlineto{\pgfqpoint{3.321346in}{2.827513in}}%
\pgfpathlineto{\pgfqpoint{3.307963in}{2.841206in}}%
\pgfpathlineto{\pgfqpoint{3.300017in}{2.828359in}}%
\pgfpathlineto{\pgfqpoint{3.292065in}{2.815642in}}%
\pgfpathlineto{\pgfqpoint{3.284106in}{2.803052in}}%
\pgfpathlineto{\pgfqpoint{3.276142in}{2.790589in}}%
\pgfpathclose%
\pgfusepath{fill}%
\end{pgfscope}%
\begin{pgfscope}%
\pgfpathrectangle{\pgfqpoint{1.150000in}{0.150000in}}{\pgfqpoint{5.700000in}{5.700000in}}%
\pgfusepath{clip}%
\pgfsetbuttcap%
\pgfsetroundjoin%
\definecolor{currentfill}{rgb}{0.199430,0.387607,0.554642}%
\pgfsetfillcolor{currentfill}%
\pgfsetfillopacity{0.800000}%
\pgfsetlinewidth{0.000000pt}%
\definecolor{currentstroke}{rgb}{0.000000,0.000000,0.000000}%
\pgfsetstrokecolor{currentstroke}%
\pgfsetdash{}{0pt}%
\pgfpathmoveto{\pgfqpoint{4.952924in}{3.117331in}}%
\pgfpathlineto{\pgfqpoint{4.966583in}{3.116373in}}%
\pgfpathlineto{\pgfqpoint{4.980252in}{3.115595in}}%
\pgfpathlineto{\pgfqpoint{4.993931in}{3.114996in}}%
\pgfpathlineto{\pgfqpoint{5.007621in}{3.114575in}}%
\pgfpathlineto{\pgfqpoint{5.015140in}{3.127987in}}%
\pgfpathlineto{\pgfqpoint{5.022659in}{3.141650in}}%
\pgfpathlineto{\pgfqpoint{5.030179in}{3.155574in}}%
\pgfpathlineto{\pgfqpoint{5.037700in}{3.169766in}}%
\pgfpathlineto{\pgfqpoint{5.024027in}{3.170847in}}%
\pgfpathlineto{\pgfqpoint{5.010364in}{3.172107in}}%
\pgfpathlineto{\pgfqpoint{4.996711in}{3.173546in}}%
\pgfpathlineto{\pgfqpoint{4.983069in}{3.175164in}}%
\pgfpathlineto{\pgfqpoint{4.975531in}{3.160300in}}%
\pgfpathlineto{\pgfqpoint{4.967995in}{3.145712in}}%
\pgfpathlineto{\pgfqpoint{4.960459in}{3.131392in}}%
\pgfpathlineto{\pgfqpoint{4.952924in}{3.117331in}}%
\pgfpathclose%
\pgfusepath{fill}%
\end{pgfscope}%
\begin{pgfscope}%
\pgfpathrectangle{\pgfqpoint{1.150000in}{0.150000in}}{\pgfqpoint{5.700000in}{5.700000in}}%
\pgfusepath{clip}%
\pgfsetbuttcap%
\pgfsetroundjoin%
\definecolor{currentfill}{rgb}{0.275191,0.194905,0.496005}%
\pgfsetfillcolor{currentfill}%
\pgfsetfillopacity{0.800000}%
\pgfsetlinewidth{0.000000pt}%
\definecolor{currentstroke}{rgb}{0.000000,0.000000,0.000000}%
\pgfsetstrokecolor{currentstroke}%
\pgfsetdash{}{0pt}%
\pgfpathmoveto{\pgfqpoint{3.883427in}{2.647758in}}%
\pgfpathlineto{\pgfqpoint{3.896813in}{2.642099in}}%
\pgfpathlineto{\pgfqpoint{3.910204in}{2.636656in}}%
\pgfpathlineto{\pgfqpoint{3.923598in}{2.631428in}}%
\pgfpathlineto{\pgfqpoint{3.936998in}{2.626415in}}%
\pgfpathlineto{\pgfqpoint{3.944788in}{2.638476in}}%
\pgfpathlineto{\pgfqpoint{3.952575in}{2.650629in}}%
\pgfpathlineto{\pgfqpoint{3.960356in}{2.662876in}}%
\pgfpathlineto{\pgfqpoint{3.968133in}{2.675222in}}%
\pgfpathlineto{\pgfqpoint{3.954742in}{2.680519in}}%
\pgfpathlineto{\pgfqpoint{3.941355in}{2.686030in}}%
\pgfpathlineto{\pgfqpoint{3.927972in}{2.691756in}}%
\pgfpathlineto{\pgfqpoint{3.914594in}{2.697698in}}%
\pgfpathlineto{\pgfqpoint{3.906809in}{2.685057in}}%
\pgfpathlineto{\pgfqpoint{3.899020in}{2.672522in}}%
\pgfpathlineto{\pgfqpoint{3.891226in}{2.660091in}}%
\pgfpathlineto{\pgfqpoint{3.883427in}{2.647758in}}%
\pgfpathclose%
\pgfusepath{fill}%
\end{pgfscope}%
\begin{pgfscope}%
\pgfpathrectangle{\pgfqpoint{1.150000in}{0.150000in}}{\pgfqpoint{5.700000in}{5.700000in}}%
\pgfusepath{clip}%
\pgfsetbuttcap%
\pgfsetroundjoin%
\definecolor{currentfill}{rgb}{0.266580,0.228262,0.514349}%
\pgfsetfillcolor{currentfill}%
\pgfsetfillopacity{0.800000}%
\pgfsetlinewidth{0.000000pt}%
\definecolor{currentstroke}{rgb}{0.000000,0.000000,0.000000}%
\pgfsetstrokecolor{currentstroke}%
\pgfsetdash{}{0pt}%
\pgfpathmoveto{\pgfqpoint{4.191115in}{2.722869in}}%
\pgfpathlineto{\pgfqpoint{4.204564in}{2.719610in}}%
\pgfpathlineto{\pgfqpoint{4.218020in}{2.716553in}}%
\pgfpathlineto{\pgfqpoint{4.231483in}{2.713696in}}%
\pgfpathlineto{\pgfqpoint{4.244953in}{2.711039in}}%
\pgfpathlineto{\pgfqpoint{4.252659in}{2.722985in}}%
\pgfpathlineto{\pgfqpoint{4.260361in}{2.735038in}}%
\pgfpathlineto{\pgfqpoint{4.268059in}{2.747204in}}%
\pgfpathlineto{\pgfqpoint{4.275753in}{2.759485in}}%
\pgfpathlineto{\pgfqpoint{4.262292in}{2.762520in}}%
\pgfpathlineto{\pgfqpoint{4.248838in}{2.765754in}}%
\pgfpathlineto{\pgfqpoint{4.235391in}{2.769188in}}%
\pgfpathlineto{\pgfqpoint{4.221950in}{2.772824in}}%
\pgfpathlineto{\pgfqpoint{4.214247in}{2.760154in}}%
\pgfpathlineto{\pgfqpoint{4.206540in}{2.747607in}}%
\pgfpathlineto{\pgfqpoint{4.198830in}{2.735181in}}%
\pgfpathlineto{\pgfqpoint{4.191115in}{2.722869in}}%
\pgfpathclose%
\pgfusepath{fill}%
\end{pgfscope}%
\begin{pgfscope}%
\pgfpathrectangle{\pgfqpoint{1.150000in}{0.150000in}}{\pgfqpoint{5.700000in}{5.700000in}}%
\pgfusepath{clip}%
\pgfsetbuttcap%
\pgfsetroundjoin%
\definecolor{currentfill}{rgb}{0.190631,0.407061,0.556089}%
\pgfsetfillcolor{currentfill}%
\pgfsetfillopacity{0.800000}%
\pgfsetlinewidth{0.000000pt}%
\definecolor{currentstroke}{rgb}{0.000000,0.000000,0.000000}%
\pgfsetstrokecolor{currentstroke}%
\pgfsetdash{}{0pt}%
\pgfpathmoveto{\pgfqpoint{5.037700in}{3.169766in}}%
\pgfpathlineto{\pgfqpoint{5.051384in}{3.168862in}}%
\pgfpathlineto{\pgfqpoint{5.065079in}{3.168136in}}%
\pgfpathlineto{\pgfqpoint{5.078785in}{3.167588in}}%
\pgfpathlineto{\pgfqpoint{5.092501in}{3.167216in}}%
\pgfpathlineto{\pgfqpoint{5.100007in}{3.181002in}}%
\pgfpathlineto{\pgfqpoint{5.107514in}{3.195064in}}%
\pgfpathlineto{\pgfqpoint{5.115023in}{3.209409in}}%
\pgfpathlineto{\pgfqpoint{5.122534in}{3.224046in}}%
\pgfpathlineto{\pgfqpoint{5.108836in}{3.225110in}}%
\pgfpathlineto{\pgfqpoint{5.095148in}{3.226351in}}%
\pgfpathlineto{\pgfqpoint{5.081471in}{3.227769in}}%
\pgfpathlineto{\pgfqpoint{5.067804in}{3.229364in}}%
\pgfpathlineto{\pgfqpoint{5.060275in}{3.214024in}}%
\pgfpathlineto{\pgfqpoint{5.052748in}{3.198982in}}%
\pgfpathlineto{\pgfqpoint{5.045223in}{3.184232in}}%
\pgfpathlineto{\pgfqpoint{5.037700in}{3.169766in}}%
\pgfpathclose%
\pgfusepath{fill}%
\end{pgfscope}%
\begin{pgfscope}%
\pgfpathrectangle{\pgfqpoint{1.150000in}{0.150000in}}{\pgfqpoint{5.700000in}{5.700000in}}%
\pgfusepath{clip}%
\pgfsetbuttcap%
\pgfsetroundjoin%
\definecolor{currentfill}{rgb}{0.277134,0.185228,0.489898}%
\pgfsetfillcolor{currentfill}%
\pgfsetfillopacity{0.800000}%
\pgfsetlinewidth{0.000000pt}%
\definecolor{currentstroke}{rgb}{0.000000,0.000000,0.000000}%
\pgfsetstrokecolor{currentstroke}%
\pgfsetdash{}{0pt}%
\pgfpathmoveto{\pgfqpoint{3.660309in}{2.631819in}}%
\pgfpathlineto{\pgfqpoint{3.673674in}{2.623946in}}%
\pgfpathlineto{\pgfqpoint{3.687041in}{2.616304in}}%
\pgfpathlineto{\pgfqpoint{3.700410in}{2.608891in}}%
\pgfpathlineto{\pgfqpoint{3.713782in}{2.601706in}}%
\pgfpathlineto{\pgfqpoint{3.721635in}{2.613775in}}%
\pgfpathlineto{\pgfqpoint{3.729484in}{2.625935in}}%
\pgfpathlineto{\pgfqpoint{3.737328in}{2.638188in}}%
\pgfpathlineto{\pgfqpoint{3.745166in}{2.650537in}}%
\pgfpathlineto{\pgfqpoint{3.731803in}{2.657943in}}%
\pgfpathlineto{\pgfqpoint{3.718442in}{2.665577in}}%
\pgfpathlineto{\pgfqpoint{3.705084in}{2.673440in}}%
\pgfpathlineto{\pgfqpoint{3.691728in}{2.681534in}}%
\pgfpathlineto{\pgfqpoint{3.683881in}{2.668953in}}%
\pgfpathlineto{\pgfqpoint{3.676029in}{2.656475in}}%
\pgfpathlineto{\pgfqpoint{3.668172in}{2.644098in}}%
\pgfpathlineto{\pgfqpoint{3.660309in}{2.631819in}}%
\pgfpathclose%
\pgfusepath{fill}%
\end{pgfscope}%
\begin{pgfscope}%
\pgfpathrectangle{\pgfqpoint{1.150000in}{0.150000in}}{\pgfqpoint{5.700000in}{5.700000in}}%
\pgfusepath{clip}%
\pgfsetbuttcap%
\pgfsetroundjoin%
\definecolor{currentfill}{rgb}{0.275191,0.194905,0.496005}%
\pgfsetfillcolor{currentfill}%
\pgfsetfillopacity{0.800000}%
\pgfsetlinewidth{0.000000pt}%
\definecolor{currentstroke}{rgb}{0.000000,0.000000,0.000000}%
\pgfsetstrokecolor{currentstroke}%
\pgfsetdash{}{0pt}%
\pgfpathmoveto{\pgfqpoint{3.521875in}{2.653579in}}%
\pgfpathlineto{\pgfqpoint{3.535238in}{2.643998in}}%
\pgfpathlineto{\pgfqpoint{3.548602in}{2.634660in}}%
\pgfpathlineto{\pgfqpoint{3.561967in}{2.625562in}}%
\pgfpathlineto{\pgfqpoint{3.575333in}{2.616704in}}%
\pgfpathlineto{\pgfqpoint{3.583225in}{2.628794in}}%
\pgfpathlineto{\pgfqpoint{3.591111in}{2.640979in}}%
\pgfpathlineto{\pgfqpoint{3.598992in}{2.653263in}}%
\pgfpathlineto{\pgfqpoint{3.606868in}{2.665648in}}%
\pgfpathlineto{\pgfqpoint{3.593512in}{2.674696in}}%
\pgfpathlineto{\pgfqpoint{3.580156in}{2.683983in}}%
\pgfpathlineto{\pgfqpoint{3.566801in}{2.693511in}}%
\pgfpathlineto{\pgfqpoint{3.553446in}{2.703281in}}%
\pgfpathlineto{\pgfqpoint{3.545561in}{2.690696in}}%
\pgfpathlineto{\pgfqpoint{3.537671in}{2.678218in}}%
\pgfpathlineto{\pgfqpoint{3.529776in}{2.665847in}}%
\pgfpathlineto{\pgfqpoint{3.521875in}{2.653579in}}%
\pgfpathclose%
\pgfusepath{fill}%
\end{pgfscope}%
\begin{pgfscope}%
\pgfpathrectangle{\pgfqpoint{1.150000in}{0.150000in}}{\pgfqpoint{5.700000in}{5.700000in}}%
\pgfusepath{clip}%
\pgfsetbuttcap%
\pgfsetroundjoin%
\definecolor{currentfill}{rgb}{0.204903,0.375746,0.553533}%
\pgfsetfillcolor{currentfill}%
\pgfsetfillopacity{0.800000}%
\pgfsetlinewidth{0.000000pt}%
\definecolor{currentstroke}{rgb}{0.000000,0.000000,0.000000}%
\pgfsetstrokecolor{currentstroke}%
\pgfsetdash{}{0pt}%
\pgfpathmoveto{\pgfqpoint{3.007253in}{3.121818in}}%
\pgfpathlineto{\pgfqpoint{3.020760in}{3.102357in}}%
\pgfpathlineto{\pgfqpoint{3.034258in}{3.083217in}}%
\pgfpathlineto{\pgfqpoint{3.047749in}{3.064396in}}%
\pgfpathlineto{\pgfqpoint{3.061232in}{3.045892in}}%
\pgfpathlineto{\pgfqpoint{3.069243in}{3.058944in}}%
\pgfpathlineto{\pgfqpoint{3.077248in}{3.072154in}}%
\pgfpathlineto{\pgfqpoint{3.085245in}{3.085524in}}%
\pgfpathlineto{\pgfqpoint{3.093235in}{3.099057in}}%
\pgfpathlineto{\pgfqpoint{3.079764in}{3.117721in}}%
\pgfpathlineto{\pgfqpoint{3.066286in}{3.136702in}}%
\pgfpathlineto{\pgfqpoint{3.052800in}{3.156002in}}%
\pgfpathlineto{\pgfqpoint{3.039305in}{3.175624in}}%
\pgfpathlineto{\pgfqpoint{3.031303in}{3.161920in}}%
\pgfpathlineto{\pgfqpoint{3.023294in}{3.148385in}}%
\pgfpathlineto{\pgfqpoint{3.015277in}{3.135019in}}%
\pgfpathlineto{\pgfqpoint{3.007253in}{3.121818in}}%
\pgfpathclose%
\pgfusepath{fill}%
\end{pgfscope}%
\begin{pgfscope}%
\pgfpathrectangle{\pgfqpoint{1.150000in}{0.150000in}}{\pgfqpoint{5.700000in}{5.700000in}}%
\pgfusepath{clip}%
\pgfsetbuttcap%
\pgfsetroundjoin%
\definecolor{currentfill}{rgb}{0.266580,0.228262,0.514349}%
\pgfsetfillcolor{currentfill}%
\pgfsetfillopacity{0.800000}%
\pgfsetlinewidth{0.000000pt}%
\definecolor{currentstroke}{rgb}{0.000000,0.000000,0.000000}%
\pgfsetstrokecolor{currentstroke}%
\pgfsetdash{}{0pt}%
\pgfpathmoveto{\pgfqpoint{3.329701in}{2.738070in}}%
\pgfpathlineto{\pgfqpoint{3.343084in}{2.725608in}}%
\pgfpathlineto{\pgfqpoint{3.356466in}{2.713410in}}%
\pgfpathlineto{\pgfqpoint{3.369845in}{2.701472in}}%
\pgfpathlineto{\pgfqpoint{3.383223in}{2.689795in}}%
\pgfpathlineto{\pgfqpoint{3.391167in}{2.701965in}}%
\pgfpathlineto{\pgfqpoint{3.399104in}{2.714245in}}%
\pgfpathlineto{\pgfqpoint{3.407036in}{2.726638in}}%
\pgfpathlineto{\pgfqpoint{3.414962in}{2.739144in}}%
\pgfpathlineto{\pgfqpoint{3.401594in}{2.750981in}}%
\pgfpathlineto{\pgfqpoint{3.388224in}{2.763076in}}%
\pgfpathlineto{\pgfqpoint{3.374853in}{2.775433in}}%
\pgfpathlineto{\pgfqpoint{3.361480in}{2.788053in}}%
\pgfpathlineto{\pgfqpoint{3.353544in}{2.775376in}}%
\pgfpathlineto{\pgfqpoint{3.345603in}{2.762821in}}%
\pgfpathlineto{\pgfqpoint{3.337655in}{2.750387in}}%
\pgfpathlineto{\pgfqpoint{3.329701in}{2.738070in}}%
\pgfpathclose%
\pgfusepath{fill}%
\end{pgfscope}%
\begin{pgfscope}%
\pgfpathrectangle{\pgfqpoint{1.150000in}{0.150000in}}{\pgfqpoint{5.700000in}{5.700000in}}%
\pgfusepath{clip}%
\pgfsetbuttcap%
\pgfsetroundjoin%
\definecolor{currentfill}{rgb}{0.270595,0.214069,0.507052}%
\pgfsetfillcolor{currentfill}%
\pgfsetfillopacity{0.800000}%
\pgfsetlinewidth{0.000000pt}%
\definecolor{currentstroke}{rgb}{0.000000,0.000000,0.000000}%
\pgfsetstrokecolor{currentstroke}%
\pgfsetdash{}{0pt}%
\pgfpathmoveto{\pgfqpoint{4.106450in}{2.688372in}}%
\pgfpathlineto{\pgfqpoint{4.119882in}{2.684645in}}%
\pgfpathlineto{\pgfqpoint{4.133321in}{2.681123in}}%
\pgfpathlineto{\pgfqpoint{4.146765in}{2.677805in}}%
\pgfpathlineto{\pgfqpoint{4.160216in}{2.674690in}}%
\pgfpathlineto{\pgfqpoint{4.167947in}{2.686583in}}%
\pgfpathlineto{\pgfqpoint{4.175674in}{2.698575in}}%
\pgfpathlineto{\pgfqpoint{4.183397in}{2.710669in}}%
\pgfpathlineto{\pgfqpoint{4.191115in}{2.722869in}}%
\pgfpathlineto{\pgfqpoint{4.177673in}{2.726330in}}%
\pgfpathlineto{\pgfqpoint{4.164236in}{2.729994in}}%
\pgfpathlineto{\pgfqpoint{4.150806in}{2.733862in}}%
\pgfpathlineto{\pgfqpoint{4.137383in}{2.737935in}}%
\pgfpathlineto{\pgfqpoint{4.129656in}{2.725377in}}%
\pgfpathlineto{\pgfqpoint{4.121925in}{2.712933in}}%
\pgfpathlineto{\pgfqpoint{4.114190in}{2.700600in}}%
\pgfpathlineto{\pgfqpoint{4.106450in}{2.688372in}}%
\pgfpathclose%
\pgfusepath{fill}%
\end{pgfscope}%
\begin{pgfscope}%
\pgfpathrectangle{\pgfqpoint{1.150000in}{0.150000in}}{\pgfqpoint{5.700000in}{5.700000in}}%
\pgfusepath{clip}%
\pgfsetbuttcap%
\pgfsetroundjoin%
\definecolor{currentfill}{rgb}{0.182256,0.426184,0.557120}%
\pgfsetfillcolor{currentfill}%
\pgfsetfillopacity{0.800000}%
\pgfsetlinewidth{0.000000pt}%
\definecolor{currentstroke}{rgb}{0.000000,0.000000,0.000000}%
\pgfsetstrokecolor{currentstroke}%
\pgfsetdash{}{0pt}%
\pgfpathmoveto{\pgfqpoint{5.122534in}{3.224046in}}%
\pgfpathlineto{\pgfqpoint{5.136244in}{3.223158in}}%
\pgfpathlineto{\pgfqpoint{5.149964in}{3.222446in}}%
\pgfpathlineto{\pgfqpoint{5.163696in}{3.221910in}}%
\pgfpathlineto{\pgfqpoint{5.177439in}{3.221549in}}%
\pgfpathlineto{\pgfqpoint{5.184934in}{3.235773in}}%
\pgfpathlineto{\pgfqpoint{5.192432in}{3.250297in}}%
\pgfpathlineto{\pgfqpoint{5.199933in}{3.265130in}}%
\pgfpathlineto{\pgfqpoint{5.207437in}{3.280280in}}%
\pgfpathlineto{\pgfqpoint{5.193714in}{3.281365in}}%
\pgfpathlineto{\pgfqpoint{5.180001in}{3.282625in}}%
\pgfpathlineto{\pgfqpoint{5.166300in}{3.284061in}}%
\pgfpathlineto{\pgfqpoint{5.152609in}{3.285672in}}%
\pgfpathlineto{\pgfqpoint{5.145085in}{3.269787in}}%
\pgfpathlineto{\pgfqpoint{5.137565in}{3.254226in}}%
\pgfpathlineto{\pgfqpoint{5.130048in}{3.238982in}}%
\pgfpathlineto{\pgfqpoint{5.122534in}{3.224046in}}%
\pgfpathclose%
\pgfusepath{fill}%
\end{pgfscope}%
\begin{pgfscope}%
\pgfpathrectangle{\pgfqpoint{1.150000in}{0.150000in}}{\pgfqpoint{5.700000in}{5.700000in}}%
\pgfusepath{clip}%
\pgfsetbuttcap%
\pgfsetroundjoin%
\definecolor{currentfill}{rgb}{0.277134,0.185228,0.489898}%
\pgfsetfillcolor{currentfill}%
\pgfsetfillopacity{0.800000}%
\pgfsetlinewidth{0.000000pt}%
\definecolor{currentstroke}{rgb}{0.000000,0.000000,0.000000}%
\pgfsetstrokecolor{currentstroke}%
\pgfsetdash{}{0pt}%
\pgfpathmoveto{\pgfqpoint{3.798649in}{2.623171in}}%
\pgfpathlineto{\pgfqpoint{3.812027in}{2.616887in}}%
\pgfpathlineto{\pgfqpoint{3.825409in}{2.610824in}}%
\pgfpathlineto{\pgfqpoint{3.838795in}{2.604981in}}%
\pgfpathlineto{\pgfqpoint{3.852185in}{2.599357in}}%
\pgfpathlineto{\pgfqpoint{3.860003in}{2.611324in}}%
\pgfpathlineto{\pgfqpoint{3.867816in}{2.623378in}}%
\pgfpathlineto{\pgfqpoint{3.875624in}{2.635522in}}%
\pgfpathlineto{\pgfqpoint{3.883427in}{2.647758in}}%
\pgfpathlineto{\pgfqpoint{3.870045in}{2.653635in}}%
\pgfpathlineto{\pgfqpoint{3.856667in}{2.659730in}}%
\pgfpathlineto{\pgfqpoint{3.843293in}{2.666045in}}%
\pgfpathlineto{\pgfqpoint{3.829922in}{2.672580in}}%
\pgfpathlineto{\pgfqpoint{3.822111in}{2.660080in}}%
\pgfpathlineto{\pgfqpoint{3.814295in}{2.647681in}}%
\pgfpathlineto{\pgfqpoint{3.806474in}{2.635379in}}%
\pgfpathlineto{\pgfqpoint{3.798649in}{2.623171in}}%
\pgfpathclose%
\pgfusepath{fill}%
\end{pgfscope}%
\begin{pgfscope}%
\pgfpathrectangle{\pgfqpoint{1.150000in}{0.150000in}}{\pgfqpoint{5.700000in}{5.700000in}}%
\pgfusepath{clip}%
\pgfsetbuttcap%
\pgfsetroundjoin%
\definecolor{currentfill}{rgb}{0.271828,0.209303,0.504434}%
\pgfsetfillcolor{currentfill}%
\pgfsetfillopacity{0.800000}%
\pgfsetlinewidth{0.000000pt}%
\definecolor{currentstroke}{rgb}{0.000000,0.000000,0.000000}%
\pgfsetstrokecolor{currentstroke}%
\pgfsetdash{}{0pt}%
\pgfpathmoveto{\pgfqpoint{3.383223in}{2.689795in}}%
\pgfpathlineto{\pgfqpoint{3.396600in}{2.678374in}}%
\pgfpathlineto{\pgfqpoint{3.409975in}{2.667210in}}%
\pgfpathlineto{\pgfqpoint{3.423350in}{2.656300in}}%
\pgfpathlineto{\pgfqpoint{3.436723in}{2.645641in}}%
\pgfpathlineto{\pgfqpoint{3.444657in}{2.657665in}}%
\pgfpathlineto{\pgfqpoint{3.452584in}{2.669791in}}%
\pgfpathlineto{\pgfqpoint{3.460506in}{2.682021in}}%
\pgfpathlineto{\pgfqpoint{3.468422in}{2.694358in}}%
\pgfpathlineto{\pgfqpoint{3.455058in}{2.705174in}}%
\pgfpathlineto{\pgfqpoint{3.441694in}{2.716243in}}%
\pgfpathlineto{\pgfqpoint{3.428328in}{2.727566in}}%
\pgfpathlineto{\pgfqpoint{3.414962in}{2.739144in}}%
\pgfpathlineto{\pgfqpoint{3.407036in}{2.726638in}}%
\pgfpathlineto{\pgfqpoint{3.399104in}{2.714245in}}%
\pgfpathlineto{\pgfqpoint{3.391167in}{2.701965in}}%
\pgfpathlineto{\pgfqpoint{3.383223in}{2.689795in}}%
\pgfpathclose%
\pgfusepath{fill}%
\end{pgfscope}%
\begin{pgfscope}%
\pgfpathrectangle{\pgfqpoint{1.150000in}{0.150000in}}{\pgfqpoint{5.700000in}{5.700000in}}%
\pgfusepath{clip}%
\pgfsetbuttcap%
\pgfsetroundjoin%
\definecolor{currentfill}{rgb}{0.190631,0.407061,0.556089}%
\pgfsetfillcolor{currentfill}%
\pgfsetfillopacity{0.800000}%
\pgfsetlinewidth{0.000000pt}%
\definecolor{currentstroke}{rgb}{0.000000,0.000000,0.000000}%
\pgfsetstrokecolor{currentstroke}%
\pgfsetdash{}{0pt}%
\pgfpathmoveto{\pgfqpoint{2.953137in}{3.202946in}}%
\pgfpathlineto{\pgfqpoint{2.966680in}{3.182165in}}%
\pgfpathlineto{\pgfqpoint{2.980213in}{3.161719in}}%
\pgfpathlineto{\pgfqpoint{2.993737in}{3.141605in}}%
\pgfpathlineto{\pgfqpoint{3.007253in}{3.121818in}}%
\pgfpathlineto{\pgfqpoint{3.015277in}{3.135019in}}%
\pgfpathlineto{\pgfqpoint{3.023294in}{3.148385in}}%
\pgfpathlineto{\pgfqpoint{3.031303in}{3.161920in}}%
\pgfpathlineto{\pgfqpoint{3.039305in}{3.175624in}}%
\pgfpathlineto{\pgfqpoint{3.025803in}{3.195571in}}%
\pgfpathlineto{\pgfqpoint{3.012291in}{3.215847in}}%
\pgfpathlineto{\pgfqpoint{2.998771in}{3.236454in}}%
\pgfpathlineto{\pgfqpoint{2.985241in}{3.257395in}}%
\pgfpathlineto{\pgfqpoint{2.977227in}{3.243518in}}%
\pgfpathlineto{\pgfqpoint{2.969205in}{3.229818in}}%
\pgfpathlineto{\pgfqpoint{2.961175in}{3.216295in}}%
\pgfpathlineto{\pgfqpoint{2.953137in}{3.202946in}}%
\pgfpathclose%
\pgfusepath{fill}%
\end{pgfscope}%
\begin{pgfscope}%
\pgfpathrectangle{\pgfqpoint{1.150000in}{0.150000in}}{\pgfqpoint{5.700000in}{5.700000in}}%
\pgfusepath{clip}%
\pgfsetbuttcap%
\pgfsetroundjoin%
\definecolor{currentfill}{rgb}{0.273006,0.204520,0.501721}%
\pgfsetfillcolor{currentfill}%
\pgfsetfillopacity{0.800000}%
\pgfsetlinewidth{0.000000pt}%
\definecolor{currentstroke}{rgb}{0.000000,0.000000,0.000000}%
\pgfsetstrokecolor{currentstroke}%
\pgfsetdash{}{0pt}%
\pgfpathmoveto{\pgfqpoint{4.021748in}{2.656158in}}%
\pgfpathlineto{\pgfqpoint{4.035165in}{2.651918in}}%
\pgfpathlineto{\pgfqpoint{4.048588in}{2.647886in}}%
\pgfpathlineto{\pgfqpoint{4.062016in}{2.644062in}}%
\pgfpathlineto{\pgfqpoint{4.075451in}{2.640444in}}%
\pgfpathlineto{\pgfqpoint{4.083207in}{2.652287in}}%
\pgfpathlineto{\pgfqpoint{4.090959in}{2.664220in}}%
\pgfpathlineto{\pgfqpoint{4.098707in}{2.676247in}}%
\pgfpathlineto{\pgfqpoint{4.106450in}{2.688372in}}%
\pgfpathlineto{\pgfqpoint{4.093024in}{2.692305in}}%
\pgfpathlineto{\pgfqpoint{4.079604in}{2.696444in}}%
\pgfpathlineto{\pgfqpoint{4.066190in}{2.700791in}}%
\pgfpathlineto{\pgfqpoint{4.052781in}{2.705346in}}%
\pgfpathlineto{\pgfqpoint{4.045029in}{2.692894in}}%
\pgfpathlineto{\pgfqpoint{4.037273in}{2.680548in}}%
\pgfpathlineto{\pgfqpoint{4.029513in}{2.668304in}}%
\pgfpathlineto{\pgfqpoint{4.021748in}{2.656158in}}%
\pgfpathclose%
\pgfusepath{fill}%
\end{pgfscope}%
\begin{pgfscope}%
\pgfpathrectangle{\pgfqpoint{1.150000in}{0.150000in}}{\pgfqpoint{5.700000in}{5.700000in}}%
\pgfusepath{clip}%
\pgfsetbuttcap%
\pgfsetroundjoin%
\definecolor{currentfill}{rgb}{0.174274,0.445044,0.557792}%
\pgfsetfillcolor{currentfill}%
\pgfsetfillopacity{0.800000}%
\pgfsetlinewidth{0.000000pt}%
\definecolor{currentstroke}{rgb}{0.000000,0.000000,0.000000}%
\pgfsetstrokecolor{currentstroke}%
\pgfsetdash{}{0pt}%
\pgfpathmoveto{\pgfqpoint{5.207437in}{3.280280in}}%
\pgfpathlineto{\pgfqpoint{5.221172in}{3.279369in}}%
\pgfpathlineto{\pgfqpoint{5.234918in}{3.278633in}}%
\pgfpathlineto{\pgfqpoint{5.248675in}{3.278072in}}%
\pgfpathlineto{\pgfqpoint{5.262444in}{3.277684in}}%
\pgfpathlineto{\pgfqpoint{5.269932in}{3.292415in}}%
\pgfpathlineto{\pgfqpoint{5.277424in}{3.307472in}}%
\pgfpathlineto{\pgfqpoint{5.284921in}{3.322864in}}%
\pgfpathlineto{\pgfqpoint{5.271167in}{3.323816in}}%
\pgfpathlineto{\pgfqpoint{5.257425in}{3.324942in}}%
\pgfpathlineto{\pgfqpoint{5.243695in}{3.326242in}}%
\pgfpathlineto{\pgfqpoint{5.229975in}{3.327716in}}%
\pgfpathlineto{\pgfqpoint{5.222458in}{3.311564in}}%
\pgfpathlineto{\pgfqpoint{5.214946in}{3.295755in}}%
\pgfpathlineto{\pgfqpoint{5.207437in}{3.280280in}}%
\pgfpathclose%
\pgfusepath{fill}%
\end{pgfscope}%
\begin{pgfscope}%
\pgfpathrectangle{\pgfqpoint{1.150000in}{0.150000in}}{\pgfqpoint{5.700000in}{5.700000in}}%
\pgfusepath{clip}%
\pgfsetbuttcap%
\pgfsetroundjoin%
\definecolor{currentfill}{rgb}{0.278012,0.180367,0.486697}%
\pgfsetfillcolor{currentfill}%
\pgfsetfillopacity{0.800000}%
\pgfsetlinewidth{0.000000pt}%
\definecolor{currentstroke}{rgb}{0.000000,0.000000,0.000000}%
\pgfsetstrokecolor{currentstroke}%
\pgfsetdash{}{0pt}%
\pgfpathmoveto{\pgfqpoint{3.575333in}{2.616704in}}%
\pgfpathlineto{\pgfqpoint{3.588700in}{2.608083in}}%
\pgfpathlineto{\pgfqpoint{3.602068in}{2.599699in}}%
\pgfpathlineto{\pgfqpoint{3.615438in}{2.591550in}}%
\pgfpathlineto{\pgfqpoint{3.628809in}{2.583634in}}%
\pgfpathlineto{\pgfqpoint{3.636692in}{2.595546in}}%
\pgfpathlineto{\pgfqpoint{3.644570in}{2.607546in}}%
\pgfpathlineto{\pgfqpoint{3.652442in}{2.619636in}}%
\pgfpathlineto{\pgfqpoint{3.660309in}{2.631819in}}%
\pgfpathlineto{\pgfqpoint{3.646947in}{2.639925in}}%
\pgfpathlineto{\pgfqpoint{3.633586in}{2.648264in}}%
\pgfpathlineto{\pgfqpoint{3.620226in}{2.656838in}}%
\pgfpathlineto{\pgfqpoint{3.606868in}{2.665648in}}%
\pgfpathlineto{\pgfqpoint{3.598992in}{2.653263in}}%
\pgfpathlineto{\pgfqpoint{3.591111in}{2.640979in}}%
\pgfpathlineto{\pgfqpoint{3.583225in}{2.628794in}}%
\pgfpathlineto{\pgfqpoint{3.575333in}{2.616704in}}%
\pgfpathclose%
\pgfusepath{fill}%
\end{pgfscope}%
\begin{pgfscope}%
\pgfpathrectangle{\pgfqpoint{1.150000in}{0.150000in}}{\pgfqpoint{5.700000in}{5.700000in}}%
\pgfusepath{clip}%
\pgfsetbuttcap%
\pgfsetroundjoin%
\definecolor{currentfill}{rgb}{0.278012,0.180367,0.486697}%
\pgfsetfillcolor{currentfill}%
\pgfsetfillopacity{0.800000}%
\pgfsetlinewidth{0.000000pt}%
\definecolor{currentstroke}{rgb}{0.000000,0.000000,0.000000}%
\pgfsetstrokecolor{currentstroke}%
\pgfsetdash{}{0pt}%
\pgfpathmoveto{\pgfqpoint{3.713782in}{2.601706in}}%
\pgfpathlineto{\pgfqpoint{3.727156in}{2.594748in}}%
\pgfpathlineto{\pgfqpoint{3.740533in}{2.588015in}}%
\pgfpathlineto{\pgfqpoint{3.753913in}{2.581508in}}%
\pgfpathlineto{\pgfqpoint{3.767297in}{2.575223in}}%
\pgfpathlineto{\pgfqpoint{3.775142in}{2.587083in}}%
\pgfpathlineto{\pgfqpoint{3.782983in}{2.599026in}}%
\pgfpathlineto{\pgfqpoint{3.790818in}{2.611054in}}%
\pgfpathlineto{\pgfqpoint{3.798649in}{2.623171in}}%
\pgfpathlineto{\pgfqpoint{3.785273in}{2.629677in}}%
\pgfpathlineto{\pgfqpoint{3.771901in}{2.636405in}}%
\pgfpathlineto{\pgfqpoint{3.758532in}{2.643359in}}%
\pgfpathlineto{\pgfqpoint{3.745166in}{2.650537in}}%
\pgfpathlineto{\pgfqpoint{3.737328in}{2.638188in}}%
\pgfpathlineto{\pgfqpoint{3.729484in}{2.625935in}}%
\pgfpathlineto{\pgfqpoint{3.721635in}{2.613775in}}%
\pgfpathlineto{\pgfqpoint{3.713782in}{2.601706in}}%
\pgfpathclose%
\pgfusepath{fill}%
\end{pgfscope}%
\begin{pgfscope}%
\pgfpathrectangle{\pgfqpoint{1.150000in}{0.150000in}}{\pgfqpoint{5.700000in}{5.700000in}}%
\pgfusepath{clip}%
\pgfsetbuttcap%
\pgfsetroundjoin%
\definecolor{currentfill}{rgb}{0.276194,0.190074,0.493001}%
\pgfsetfillcolor{currentfill}%
\pgfsetfillopacity{0.800000}%
\pgfsetlinewidth{0.000000pt}%
\definecolor{currentstroke}{rgb}{0.000000,0.000000,0.000000}%
\pgfsetstrokecolor{currentstroke}%
\pgfsetdash{}{0pt}%
\pgfpathmoveto{\pgfqpoint{3.936998in}{2.626415in}}%
\pgfpathlineto{\pgfqpoint{3.950402in}{2.621615in}}%
\pgfpathlineto{\pgfqpoint{3.963811in}{2.617028in}}%
\pgfpathlineto{\pgfqpoint{3.977226in}{2.612651in}}%
\pgfpathlineto{\pgfqpoint{3.990645in}{2.608485in}}%
\pgfpathlineto{\pgfqpoint{3.998428in}{2.620274in}}%
\pgfpathlineto{\pgfqpoint{4.006206in}{2.632147in}}%
\pgfpathlineto{\pgfqpoint{4.013979in}{2.644107in}}%
\pgfpathlineto{\pgfqpoint{4.021748in}{2.656158in}}%
\pgfpathlineto{\pgfqpoint{4.008337in}{2.660608in}}%
\pgfpathlineto{\pgfqpoint{3.994931in}{2.665268in}}%
\pgfpathlineto{\pgfqpoint{3.981529in}{2.670139in}}%
\pgfpathlineto{\pgfqpoint{3.968133in}{2.675222in}}%
\pgfpathlineto{\pgfqpoint{3.960356in}{2.662876in}}%
\pgfpathlineto{\pgfqpoint{3.952575in}{2.650629in}}%
\pgfpathlineto{\pgfqpoint{3.944788in}{2.638476in}}%
\pgfpathlineto{\pgfqpoint{3.936998in}{2.626415in}}%
\pgfpathclose%
\pgfusepath{fill}%
\end{pgfscope}%
\begin{pgfscope}%
\pgfpathrectangle{\pgfqpoint{1.150000in}{0.150000in}}{\pgfqpoint{5.700000in}{5.700000in}}%
\pgfusepath{clip}%
\pgfsetbuttcap%
\pgfsetroundjoin%
\definecolor{currentfill}{rgb}{0.239346,0.300855,0.540844}%
\pgfsetfillcolor{currentfill}%
\pgfsetfillopacity{0.800000}%
\pgfsetlinewidth{0.000000pt}%
\definecolor{currentstroke}{rgb}{0.000000,0.000000,0.000000}%
\pgfsetstrokecolor{currentstroke}%
\pgfsetdash{}{0pt}%
\pgfpathmoveto{\pgfqpoint{4.583785in}{2.874643in}}%
\pgfpathlineto{\pgfqpoint{4.597354in}{2.873577in}}%
\pgfpathlineto{\pgfqpoint{4.610933in}{2.872700in}}%
\pgfpathlineto{\pgfqpoint{4.624521in}{2.872010in}}%
\pgfpathlineto{\pgfqpoint{4.638118in}{2.871507in}}%
\pgfpathlineto{\pgfqpoint{4.645718in}{2.883300in}}%
\pgfpathlineto{\pgfqpoint{4.653315in}{2.895243in}}%
\pgfpathlineto{\pgfqpoint{4.660910in}{2.907343in}}%
\pgfpathlineto{\pgfqpoint{4.668502in}{2.919605in}}%
\pgfpathlineto{\pgfqpoint{4.654917in}{2.920612in}}%
\pgfpathlineto{\pgfqpoint{4.641341in}{2.921806in}}%
\pgfpathlineto{\pgfqpoint{4.627775in}{2.923187in}}%
\pgfpathlineto{\pgfqpoint{4.614217in}{2.924756in}}%
\pgfpathlineto{\pgfqpoint{4.606613in}{2.911979in}}%
\pgfpathlineto{\pgfqpoint{4.599006in}{2.899372in}}%
\pgfpathlineto{\pgfqpoint{4.591397in}{2.886928in}}%
\pgfpathlineto{\pgfqpoint{4.583785in}{2.874643in}}%
\pgfpathclose%
\pgfusepath{fill}%
\end{pgfscope}%
\begin{pgfscope}%
\pgfpathrectangle{\pgfqpoint{1.150000in}{0.150000in}}{\pgfqpoint{5.700000in}{5.700000in}}%
\pgfusepath{clip}%
\pgfsetbuttcap%
\pgfsetroundjoin%
\definecolor{currentfill}{rgb}{0.231674,0.318106,0.544834}%
\pgfsetfillcolor{currentfill}%
\pgfsetfillopacity{0.800000}%
\pgfsetlinewidth{0.000000pt}%
\definecolor{currentstroke}{rgb}{0.000000,0.000000,0.000000}%
\pgfsetstrokecolor{currentstroke}%
\pgfsetdash{}{0pt}%
\pgfpathmoveto{\pgfqpoint{4.668502in}{2.919605in}}%
\pgfpathlineto{\pgfqpoint{4.682097in}{2.918784in}}%
\pgfpathlineto{\pgfqpoint{4.695702in}{2.918148in}}%
\pgfpathlineto{\pgfqpoint{4.709316in}{2.917698in}}%
\pgfpathlineto{\pgfqpoint{4.722940in}{2.917433in}}%
\pgfpathlineto{\pgfqpoint{4.730518in}{2.929339in}}%
\pgfpathlineto{\pgfqpoint{4.738093in}{2.941411in}}%
\pgfpathlineto{\pgfqpoint{4.745666in}{2.953656in}}%
\pgfpathlineto{\pgfqpoint{4.753237in}{2.966080in}}%
\pgfpathlineto{\pgfqpoint{4.739626in}{2.966881in}}%
\pgfpathlineto{\pgfqpoint{4.726025in}{2.967867in}}%
\pgfpathlineto{\pgfqpoint{4.712434in}{2.969037in}}%
\pgfpathlineto{\pgfqpoint{4.698851in}{2.970393in}}%
\pgfpathlineto{\pgfqpoint{4.691267in}{2.957423in}}%
\pgfpathlineto{\pgfqpoint{4.683681in}{2.944638in}}%
\pgfpathlineto{\pgfqpoint{4.676093in}{2.932035in}}%
\pgfpathlineto{\pgfqpoint{4.668502in}{2.919605in}}%
\pgfpathclose%
\pgfusepath{fill}%
\end{pgfscope}%
\begin{pgfscope}%
\pgfpathrectangle{\pgfqpoint{1.150000in}{0.150000in}}{\pgfqpoint{5.700000in}{5.700000in}}%
\pgfusepath{clip}%
\pgfsetbuttcap%
\pgfsetroundjoin%
\definecolor{currentfill}{rgb}{0.246811,0.283237,0.535941}%
\pgfsetfillcolor{currentfill}%
\pgfsetfillopacity{0.800000}%
\pgfsetlinewidth{0.000000pt}%
\definecolor{currentstroke}{rgb}{0.000000,0.000000,0.000000}%
\pgfsetstrokecolor{currentstroke}%
\pgfsetdash{}{0pt}%
\pgfpathmoveto{\pgfqpoint{4.499079in}{2.831232in}}%
\pgfpathlineto{\pgfqpoint{4.512623in}{2.829881in}}%
\pgfpathlineto{\pgfqpoint{4.526177in}{2.828721in}}%
\pgfpathlineto{\pgfqpoint{4.539739in}{2.827750in}}%
\pgfpathlineto{\pgfqpoint{4.553310in}{2.826969in}}%
\pgfpathlineto{\pgfqpoint{4.560933in}{2.838678in}}%
\pgfpathlineto{\pgfqpoint{4.568554in}{2.850524in}}%
\pgfpathlineto{\pgfqpoint{4.576171in}{2.862510in}}%
\pgfpathlineto{\pgfqpoint{4.583785in}{2.874643in}}%
\pgfpathlineto{\pgfqpoint{4.570225in}{2.875897in}}%
\pgfpathlineto{\pgfqpoint{4.556674in}{2.877340in}}%
\pgfpathlineto{\pgfqpoint{4.543131in}{2.878972in}}%
\pgfpathlineto{\pgfqpoint{4.529598in}{2.880795in}}%
\pgfpathlineto{\pgfqpoint{4.521972in}{2.868178in}}%
\pgfpathlineto{\pgfqpoint{4.514344in}{2.855716in}}%
\pgfpathlineto{\pgfqpoint{4.506713in}{2.843402in}}%
\pgfpathlineto{\pgfqpoint{4.499079in}{2.831232in}}%
\pgfpathclose%
\pgfusepath{fill}%
\end{pgfscope}%
\begin{pgfscope}%
\pgfpathrectangle{\pgfqpoint{1.150000in}{0.150000in}}{\pgfqpoint{5.700000in}{5.700000in}}%
\pgfusepath{clip}%
\pgfsetbuttcap%
\pgfsetroundjoin%
\definecolor{currentfill}{rgb}{0.223925,0.334994,0.548053}%
\pgfsetfillcolor{currentfill}%
\pgfsetfillopacity{0.800000}%
\pgfsetlinewidth{0.000000pt}%
\definecolor{currentstroke}{rgb}{0.000000,0.000000,0.000000}%
\pgfsetstrokecolor{currentstroke}%
\pgfsetdash{}{0pt}%
\pgfpathmoveto{\pgfqpoint{4.753237in}{2.966080in}}%
\pgfpathlineto{\pgfqpoint{4.766858in}{2.965463in}}%
\pgfpathlineto{\pgfqpoint{4.780489in}{2.965029in}}%
\pgfpathlineto{\pgfqpoint{4.794131in}{2.964779in}}%
\pgfpathlineto{\pgfqpoint{4.807782in}{2.964711in}}%
\pgfpathlineto{\pgfqpoint{4.815338in}{2.976765in}}%
\pgfpathlineto{\pgfqpoint{4.822893in}{2.989002in}}%
\pgfpathlineto{\pgfqpoint{4.830446in}{3.001430in}}%
\pgfpathlineto{\pgfqpoint{4.837997in}{3.014055in}}%
\pgfpathlineto{\pgfqpoint{4.824360in}{3.014690in}}%
\pgfpathlineto{\pgfqpoint{4.810733in}{3.015507in}}%
\pgfpathlineto{\pgfqpoint{4.797116in}{3.016508in}}%
\pgfpathlineto{\pgfqpoint{4.783509in}{3.017691in}}%
\pgfpathlineto{\pgfqpoint{4.775943in}{3.004488in}}%
\pgfpathlineto{\pgfqpoint{4.768376in}{2.991489in}}%
\pgfpathlineto{\pgfqpoint{4.760807in}{2.978689in}}%
\pgfpathlineto{\pgfqpoint{4.753237in}{2.966080in}}%
\pgfpathclose%
\pgfusepath{fill}%
\end{pgfscope}%
\begin{pgfscope}%
\pgfpathrectangle{\pgfqpoint{1.150000in}{0.150000in}}{\pgfqpoint{5.700000in}{5.700000in}}%
\pgfusepath{clip}%
\pgfsetbuttcap%
\pgfsetroundjoin%
\definecolor{currentfill}{rgb}{0.275191,0.194905,0.496005}%
\pgfsetfillcolor{currentfill}%
\pgfsetfillopacity{0.800000}%
\pgfsetlinewidth{0.000000pt}%
\definecolor{currentstroke}{rgb}{0.000000,0.000000,0.000000}%
\pgfsetstrokecolor{currentstroke}%
\pgfsetdash{}{0pt}%
\pgfpathmoveto{\pgfqpoint{3.436723in}{2.645641in}}%
\pgfpathlineto{\pgfqpoint{3.450097in}{2.635233in}}%
\pgfpathlineto{\pgfqpoint{3.463469in}{2.625074in}}%
\pgfpathlineto{\pgfqpoint{3.476842in}{2.615163in}}%
\pgfpathlineto{\pgfqpoint{3.490215in}{2.605496in}}%
\pgfpathlineto{\pgfqpoint{3.498138in}{2.617373in}}%
\pgfpathlineto{\pgfqpoint{3.506056in}{2.629344in}}%
\pgfpathlineto{\pgfqpoint{3.513968in}{2.641412in}}%
\pgfpathlineto{\pgfqpoint{3.521875in}{2.653579in}}%
\pgfpathlineto{\pgfqpoint{3.508511in}{2.663404in}}%
\pgfpathlineto{\pgfqpoint{3.495148in}{2.673474in}}%
\pgfpathlineto{\pgfqpoint{3.481785in}{2.683791in}}%
\pgfpathlineto{\pgfqpoint{3.468422in}{2.694358in}}%
\pgfpathlineto{\pgfqpoint{3.460506in}{2.682021in}}%
\pgfpathlineto{\pgfqpoint{3.452584in}{2.669791in}}%
\pgfpathlineto{\pgfqpoint{3.444657in}{2.657665in}}%
\pgfpathlineto{\pgfqpoint{3.436723in}{2.645641in}}%
\pgfpathclose%
\pgfusepath{fill}%
\end{pgfscope}%
\begin{pgfscope}%
\pgfpathrectangle{\pgfqpoint{1.150000in}{0.150000in}}{\pgfqpoint{5.700000in}{5.700000in}}%
\pgfusepath{clip}%
\pgfsetbuttcap%
\pgfsetroundjoin%
\definecolor{currentfill}{rgb}{0.253935,0.265254,0.529983}%
\pgfsetfillcolor{currentfill}%
\pgfsetfillopacity{0.800000}%
\pgfsetlinewidth{0.000000pt}%
\definecolor{currentstroke}{rgb}{0.000000,0.000000,0.000000}%
\pgfsetstrokecolor{currentstroke}%
\pgfsetdash{}{0pt}%
\pgfpathmoveto{\pgfqpoint{4.414376in}{2.789433in}}%
\pgfpathlineto{\pgfqpoint{4.427897in}{2.787755in}}%
\pgfpathlineto{\pgfqpoint{4.441426in}{2.786271in}}%
\pgfpathlineto{\pgfqpoint{4.454964in}{2.784979in}}%
\pgfpathlineto{\pgfqpoint{4.468510in}{2.783879in}}%
\pgfpathlineto{\pgfqpoint{4.476157in}{2.795528in}}%
\pgfpathlineto{\pgfqpoint{4.483801in}{2.807300in}}%
\pgfpathlineto{\pgfqpoint{4.491441in}{2.819200in}}%
\pgfpathlineto{\pgfqpoint{4.499079in}{2.831232in}}%
\pgfpathlineto{\pgfqpoint{4.485543in}{2.832773in}}%
\pgfpathlineto{\pgfqpoint{4.472016in}{2.834506in}}%
\pgfpathlineto{\pgfqpoint{4.458497in}{2.836432in}}%
\pgfpathlineto{\pgfqpoint{4.444986in}{2.838549in}}%
\pgfpathlineto{\pgfqpoint{4.437338in}{2.826065in}}%
\pgfpathlineto{\pgfqpoint{4.429688in}{2.813721in}}%
\pgfpathlineto{\pgfqpoint{4.422034in}{2.801512in}}%
\pgfpathlineto{\pgfqpoint{4.414376in}{2.789433in}}%
\pgfpathclose%
\pgfusepath{fill}%
\end{pgfscope}%
\begin{pgfscope}%
\pgfpathrectangle{\pgfqpoint{1.150000in}{0.150000in}}{\pgfqpoint{5.700000in}{5.700000in}}%
\pgfusepath{clip}%
\pgfsetbuttcap%
\pgfsetroundjoin%
\definecolor{currentfill}{rgb}{0.246811,0.283237,0.535941}%
\pgfsetfillcolor{currentfill}%
\pgfsetfillopacity{0.800000}%
\pgfsetlinewidth{0.000000pt}%
\definecolor{currentstroke}{rgb}{0.000000,0.000000,0.000000}%
\pgfsetstrokecolor{currentstroke}%
\pgfsetdash{}{0pt}%
\pgfpathmoveto{\pgfqpoint{3.136836in}{2.859243in}}%
\pgfpathlineto{\pgfqpoint{3.150275in}{2.843586in}}%
\pgfpathlineto{\pgfqpoint{3.163709in}{2.828218in}}%
\pgfpathlineto{\pgfqpoint{3.177138in}{2.813137in}}%
\pgfpathlineto{\pgfqpoint{3.190562in}{2.798340in}}%
\pgfpathlineto{\pgfqpoint{3.198564in}{2.810435in}}%
\pgfpathlineto{\pgfqpoint{3.206560in}{2.822656in}}%
\pgfpathlineto{\pgfqpoint{3.214549in}{2.835005in}}%
\pgfpathlineto{\pgfqpoint{3.222531in}{2.847483in}}%
\pgfpathlineto{\pgfqpoint{3.209119in}{2.862407in}}%
\pgfpathlineto{\pgfqpoint{3.195702in}{2.877615in}}%
\pgfpathlineto{\pgfqpoint{3.182280in}{2.893110in}}%
\pgfpathlineto{\pgfqpoint{3.168853in}{2.908895in}}%
\pgfpathlineto{\pgfqpoint{3.160859in}{2.896277in}}%
\pgfpathlineto{\pgfqpoint{3.152858in}{2.883797in}}%
\pgfpathlineto{\pgfqpoint{3.144851in}{2.871453in}}%
\pgfpathlineto{\pgfqpoint{3.136836in}{2.859243in}}%
\pgfpathclose%
\pgfusepath{fill}%
\end{pgfscope}%
\begin{pgfscope}%
\pgfpathrectangle{\pgfqpoint{1.150000in}{0.150000in}}{\pgfqpoint{5.700000in}{5.700000in}}%
\pgfusepath{clip}%
\pgfsetbuttcap%
\pgfsetroundjoin%
\definecolor{currentfill}{rgb}{0.214298,0.355619,0.551184}%
\pgfsetfillcolor{currentfill}%
\pgfsetfillopacity{0.800000}%
\pgfsetlinewidth{0.000000pt}%
\definecolor{currentstroke}{rgb}{0.000000,0.000000,0.000000}%
\pgfsetstrokecolor{currentstroke}%
\pgfsetdash{}{0pt}%
\pgfpathmoveto{\pgfqpoint{4.837997in}{3.014055in}}%
\pgfpathlineto{\pgfqpoint{4.851645in}{3.013602in}}%
\pgfpathlineto{\pgfqpoint{4.865303in}{3.013331in}}%
\pgfpathlineto{\pgfqpoint{4.878971in}{3.013241in}}%
\pgfpathlineto{\pgfqpoint{4.892650in}{3.013332in}}%
\pgfpathlineto{\pgfqpoint{4.900186in}{3.025573in}}%
\pgfpathlineto{\pgfqpoint{4.907721in}{3.038017in}}%
\pgfpathlineto{\pgfqpoint{4.915256in}{3.050671in}}%
\pgfpathlineto{\pgfqpoint{4.922789in}{3.063542in}}%
\pgfpathlineto{\pgfqpoint{4.909126in}{3.064050in}}%
\pgfpathlineto{\pgfqpoint{4.895473in}{3.064738in}}%
\pgfpathlineto{\pgfqpoint{4.881830in}{3.065608in}}%
\pgfpathlineto{\pgfqpoint{4.868197in}{3.066659in}}%
\pgfpathlineto{\pgfqpoint{4.860648in}{3.053179in}}%
\pgfpathlineto{\pgfqpoint{4.853098in}{3.039922in}}%
\pgfpathlineto{\pgfqpoint{4.845548in}{3.026883in}}%
\pgfpathlineto{\pgfqpoint{4.837997in}{3.014055in}}%
\pgfpathclose%
\pgfusepath{fill}%
\end{pgfscope}%
\begin{pgfscope}%
\pgfpathrectangle{\pgfqpoint{1.150000in}{0.150000in}}{\pgfqpoint{5.700000in}{5.700000in}}%
\pgfusepath{clip}%
\pgfsetbuttcap%
\pgfsetroundjoin%
\definecolor{currentfill}{rgb}{0.235526,0.309527,0.542944}%
\pgfsetfillcolor{currentfill}%
\pgfsetfillopacity{0.800000}%
\pgfsetlinewidth{0.000000pt}%
\definecolor{currentstroke}{rgb}{0.000000,0.000000,0.000000}%
\pgfsetstrokecolor{currentstroke}%
\pgfsetdash{}{0pt}%
\pgfpathmoveto{\pgfqpoint{3.083025in}{2.924815in}}%
\pgfpathlineto{\pgfqpoint{3.096487in}{2.907975in}}%
\pgfpathlineto{\pgfqpoint{3.109942in}{2.891435in}}%
\pgfpathlineto{\pgfqpoint{3.123392in}{2.875192in}}%
\pgfpathlineto{\pgfqpoint{3.136836in}{2.859243in}}%
\pgfpathlineto{\pgfqpoint{3.144851in}{2.871453in}}%
\pgfpathlineto{\pgfqpoint{3.152858in}{2.883797in}}%
\pgfpathlineto{\pgfqpoint{3.160859in}{2.896277in}}%
\pgfpathlineto{\pgfqpoint{3.168853in}{2.908895in}}%
\pgfpathlineto{\pgfqpoint{3.155422in}{2.924971in}}%
\pgfpathlineto{\pgfqpoint{3.141984in}{2.941342in}}%
\pgfpathlineto{\pgfqpoint{3.128542in}{2.958009in}}%
\pgfpathlineto{\pgfqpoint{3.115093in}{2.974976in}}%
\pgfpathlineto{\pgfqpoint{3.107086in}{2.962219in}}%
\pgfpathlineto{\pgfqpoint{3.099073in}{2.949608in}}%
\pgfpathlineto{\pgfqpoint{3.091053in}{2.937140in}}%
\pgfpathlineto{\pgfqpoint{3.083025in}{2.924815in}}%
\pgfpathclose%
\pgfusepath{fill}%
\end{pgfscope}%
\begin{pgfscope}%
\pgfpathrectangle{\pgfqpoint{1.150000in}{0.150000in}}{\pgfqpoint{5.700000in}{5.700000in}}%
\pgfusepath{clip}%
\pgfsetbuttcap%
\pgfsetroundjoin%
\definecolor{currentfill}{rgb}{0.177423,0.437527,0.557565}%
\pgfsetfillcolor{currentfill}%
\pgfsetfillopacity{0.800000}%
\pgfsetlinewidth{0.000000pt}%
\definecolor{currentstroke}{rgb}{0.000000,0.000000,0.000000}%
\pgfsetstrokecolor{currentstroke}%
\pgfsetdash{}{0pt}%
\pgfpathmoveto{\pgfqpoint{2.898866in}{3.289479in}}%
\pgfpathlineto{\pgfqpoint{2.912449in}{3.267327in}}%
\pgfpathlineto{\pgfqpoint{2.926022in}{3.245523in}}%
\pgfpathlineto{\pgfqpoint{2.939585in}{3.224064in}}%
\pgfpathlineto{\pgfqpoint{2.953137in}{3.202946in}}%
\pgfpathlineto{\pgfqpoint{2.961175in}{3.216295in}}%
\pgfpathlineto{\pgfqpoint{2.969205in}{3.229818in}}%
\pgfpathlineto{\pgfqpoint{2.977227in}{3.243518in}}%
\pgfpathlineto{\pgfqpoint{2.985241in}{3.257395in}}%
\pgfpathlineto{\pgfqpoint{2.971702in}{3.278675in}}%
\pgfpathlineto{\pgfqpoint{2.958153in}{3.300296in}}%
\pgfpathlineto{\pgfqpoint{2.944593in}{3.322262in}}%
\pgfpathlineto{\pgfqpoint{2.931023in}{3.344576in}}%
\pgfpathlineto{\pgfqpoint{2.922996in}{3.330524in}}%
\pgfpathlineto{\pgfqpoint{2.914960in}{3.316658in}}%
\pgfpathlineto{\pgfqpoint{2.906917in}{3.302978in}}%
\pgfpathlineto{\pgfqpoint{2.898866in}{3.289479in}}%
\pgfpathclose%
\pgfusepath{fill}%
\end{pgfscope}%
\begin{pgfscope}%
\pgfpathrectangle{\pgfqpoint{1.150000in}{0.150000in}}{\pgfqpoint{5.700000in}{5.700000in}}%
\pgfusepath{clip}%
\pgfsetbuttcap%
\pgfsetroundjoin%
\definecolor{currentfill}{rgb}{0.257322,0.256130,0.526563}%
\pgfsetfillcolor{currentfill}%
\pgfsetfillopacity{0.800000}%
\pgfsetlinewidth{0.000000pt}%
\definecolor{currentstroke}{rgb}{0.000000,0.000000,0.000000}%
\pgfsetstrokecolor{currentstroke}%
\pgfsetdash{}{0pt}%
\pgfpathmoveto{\pgfqpoint{3.190562in}{2.798340in}}%
\pgfpathlineto{\pgfqpoint{3.203982in}{2.783826in}}%
\pgfpathlineto{\pgfqpoint{3.217399in}{2.769592in}}%
\pgfpathlineto{\pgfqpoint{3.230811in}{2.755635in}}%
\pgfpathlineto{\pgfqpoint{3.244220in}{2.741954in}}%
\pgfpathlineto{\pgfqpoint{3.252210in}{2.753934in}}%
\pgfpathlineto{\pgfqpoint{3.260193in}{2.766031in}}%
\pgfpathlineto{\pgfqpoint{3.268171in}{2.778249in}}%
\pgfpathlineto{\pgfqpoint{3.276142in}{2.790589in}}%
\pgfpathlineto{\pgfqpoint{3.262745in}{2.804397in}}%
\pgfpathlineto{\pgfqpoint{3.249344in}{2.818480in}}%
\pgfpathlineto{\pgfqpoint{3.235939in}{2.832842in}}%
\pgfpathlineto{\pgfqpoint{3.222531in}{2.847483in}}%
\pgfpathlineto{\pgfqpoint{3.214549in}{2.835005in}}%
\pgfpathlineto{\pgfqpoint{3.206560in}{2.822656in}}%
\pgfpathlineto{\pgfqpoint{3.198564in}{2.810435in}}%
\pgfpathlineto{\pgfqpoint{3.190562in}{2.798340in}}%
\pgfpathclose%
\pgfusepath{fill}%
\end{pgfscope}%
\begin{pgfscope}%
\pgfpathrectangle{\pgfqpoint{1.150000in}{0.150000in}}{\pgfqpoint{5.700000in}{5.700000in}}%
\pgfusepath{clip}%
\pgfsetbuttcap%
\pgfsetroundjoin%
\definecolor{currentfill}{rgb}{0.260571,0.246922,0.522828}%
\pgfsetfillcolor{currentfill}%
\pgfsetfillopacity{0.800000}%
\pgfsetlinewidth{0.000000pt}%
\definecolor{currentstroke}{rgb}{0.000000,0.000000,0.000000}%
\pgfsetstrokecolor{currentstroke}%
\pgfsetdash{}{0pt}%
\pgfpathmoveto{\pgfqpoint{4.329670in}{2.749332in}}%
\pgfpathlineto{\pgfqpoint{4.343169in}{2.747286in}}%
\pgfpathlineto{\pgfqpoint{4.356675in}{2.745436in}}%
\pgfpathlineto{\pgfqpoint{4.370189in}{2.743780in}}%
\pgfpathlineto{\pgfqpoint{4.383711in}{2.742319in}}%
\pgfpathlineto{\pgfqpoint{4.391383in}{2.753927in}}%
\pgfpathlineto{\pgfqpoint{4.399051in}{2.765645in}}%
\pgfpathlineto{\pgfqpoint{4.406715in}{2.777479in}}%
\pgfpathlineto{\pgfqpoint{4.414376in}{2.789433in}}%
\pgfpathlineto{\pgfqpoint{4.400864in}{2.791304in}}%
\pgfpathlineto{\pgfqpoint{4.387359in}{2.793368in}}%
\pgfpathlineto{\pgfqpoint{4.373863in}{2.795628in}}%
\pgfpathlineto{\pgfqpoint{4.360374in}{2.798083in}}%
\pgfpathlineto{\pgfqpoint{4.352703in}{2.785709in}}%
\pgfpathlineto{\pgfqpoint{4.345029in}{2.773462in}}%
\pgfpathlineto{\pgfqpoint{4.337352in}{2.761338in}}%
\pgfpathlineto{\pgfqpoint{4.329670in}{2.749332in}}%
\pgfpathclose%
\pgfusepath{fill}%
\end{pgfscope}%
\begin{pgfscope}%
\pgfpathrectangle{\pgfqpoint{1.150000in}{0.150000in}}{\pgfqpoint{5.700000in}{5.700000in}}%
\pgfusepath{clip}%
\pgfsetbuttcap%
\pgfsetroundjoin%
\definecolor{currentfill}{rgb}{0.204903,0.375746,0.553533}%
\pgfsetfillcolor{currentfill}%
\pgfsetfillopacity{0.800000}%
\pgfsetlinewidth{0.000000pt}%
\definecolor{currentstroke}{rgb}{0.000000,0.000000,0.000000}%
\pgfsetstrokecolor{currentstroke}%
\pgfsetdash{}{0pt}%
\pgfpathmoveto{\pgfqpoint{4.922789in}{3.063542in}}%
\pgfpathlineto{\pgfqpoint{4.936464in}{3.063214in}}%
\pgfpathlineto{\pgfqpoint{4.950149in}{3.063065in}}%
\pgfpathlineto{\pgfqpoint{4.963845in}{3.063096in}}%
\pgfpathlineto{\pgfqpoint{4.977552in}{3.063307in}}%
\pgfpathlineto{\pgfqpoint{4.985069in}{3.075782in}}%
\pgfpathlineto{\pgfqpoint{4.992586in}{3.088480in}}%
\pgfpathlineto{\pgfqpoint{5.000104in}{3.101409in}}%
\pgfpathlineto{\pgfqpoint{5.007621in}{3.114575in}}%
\pgfpathlineto{\pgfqpoint{4.993931in}{3.114996in}}%
\pgfpathlineto{\pgfqpoint{4.980252in}{3.115595in}}%
\pgfpathlineto{\pgfqpoint{4.966583in}{3.116373in}}%
\pgfpathlineto{\pgfqpoint{4.952924in}{3.117331in}}%
\pgfpathlineto{\pgfqpoint{4.945390in}{3.103523in}}%
\pgfpathlineto{\pgfqpoint{4.937857in}{3.089960in}}%
\pgfpathlineto{\pgfqpoint{4.930323in}{3.076636in}}%
\pgfpathlineto{\pgfqpoint{4.922789in}{3.063542in}}%
\pgfpathclose%
\pgfusepath{fill}%
\end{pgfscope}%
\begin{pgfscope}%
\pgfpathrectangle{\pgfqpoint{1.150000in}{0.150000in}}{\pgfqpoint{5.700000in}{5.700000in}}%
\pgfusepath{clip}%
\pgfsetbuttcap%
\pgfsetroundjoin%
\definecolor{currentfill}{rgb}{0.223925,0.334994,0.548053}%
\pgfsetfillcolor{currentfill}%
\pgfsetfillopacity{0.800000}%
\pgfsetlinewidth{0.000000pt}%
\definecolor{currentstroke}{rgb}{0.000000,0.000000,0.000000}%
\pgfsetstrokecolor{currentstroke}%
\pgfsetdash{}{0pt}%
\pgfpathmoveto{\pgfqpoint{3.029112in}{2.995220in}}%
\pgfpathlineto{\pgfqpoint{3.042601in}{2.977156in}}%
\pgfpathlineto{\pgfqpoint{3.056083in}{2.959403in}}%
\pgfpathlineto{\pgfqpoint{3.069557in}{2.941956in}}%
\pgfpathlineto{\pgfqpoint{3.083025in}{2.924815in}}%
\pgfpathlineto{\pgfqpoint{3.091053in}{2.937140in}}%
\pgfpathlineto{\pgfqpoint{3.099073in}{2.949608in}}%
\pgfpathlineto{\pgfqpoint{3.107086in}{2.962219in}}%
\pgfpathlineto{\pgfqpoint{3.115093in}{2.974976in}}%
\pgfpathlineto{\pgfqpoint{3.101637in}{2.992245in}}%
\pgfpathlineto{\pgfqpoint{3.088176in}{3.009819in}}%
\pgfpathlineto{\pgfqpoint{3.074707in}{3.027700in}}%
\pgfpathlineto{\pgfqpoint{3.061232in}{3.045892in}}%
\pgfpathlineto{\pgfqpoint{3.053213in}{3.032995in}}%
\pgfpathlineto{\pgfqpoint{3.045187in}{3.020252in}}%
\pgfpathlineto{\pgfqpoint{3.037153in}{3.007661in}}%
\pgfpathlineto{\pgfqpoint{3.029112in}{2.995220in}}%
\pgfpathclose%
\pgfusepath{fill}%
\end{pgfscope}%
\begin{pgfscope}%
\pgfpathrectangle{\pgfqpoint{1.150000in}{0.150000in}}{\pgfqpoint{5.700000in}{5.700000in}}%
\pgfusepath{clip}%
\pgfsetbuttcap%
\pgfsetroundjoin%
\definecolor{currentfill}{rgb}{0.265145,0.232956,0.516599}%
\pgfsetfillcolor{currentfill}%
\pgfsetfillopacity{0.800000}%
\pgfsetlinewidth{0.000000pt}%
\definecolor{currentstroke}{rgb}{0.000000,0.000000,0.000000}%
\pgfsetstrokecolor{currentstroke}%
\pgfsetdash{}{0pt}%
\pgfpathmoveto{\pgfqpoint{4.244953in}{2.711039in}}%
\pgfpathlineto{\pgfqpoint{4.258430in}{2.708582in}}%
\pgfpathlineto{\pgfqpoint{4.271915in}{2.706323in}}%
\pgfpathlineto{\pgfqpoint{4.285407in}{2.704262in}}%
\pgfpathlineto{\pgfqpoint{4.298907in}{2.702398in}}%
\pgfpathlineto{\pgfqpoint{4.306604in}{2.713977in}}%
\pgfpathlineto{\pgfqpoint{4.314296in}{2.725656in}}%
\pgfpathlineto{\pgfqpoint{4.321985in}{2.737440in}}%
\pgfpathlineto{\pgfqpoint{4.329670in}{2.749332in}}%
\pgfpathlineto{\pgfqpoint{4.316180in}{2.751574in}}%
\pgfpathlineto{\pgfqpoint{4.302697in}{2.754013in}}%
\pgfpathlineto{\pgfqpoint{4.289221in}{2.756650in}}%
\pgfpathlineto{\pgfqpoint{4.275753in}{2.759485in}}%
\pgfpathlineto{\pgfqpoint{4.268059in}{2.747204in}}%
\pgfpathlineto{\pgfqpoint{4.260361in}{2.735038in}}%
\pgfpathlineto{\pgfqpoint{4.252659in}{2.722985in}}%
\pgfpathlineto{\pgfqpoint{4.244953in}{2.711039in}}%
\pgfpathclose%
\pgfusepath{fill}%
\end{pgfscope}%
\begin{pgfscope}%
\pgfpathrectangle{\pgfqpoint{1.150000in}{0.150000in}}{\pgfqpoint{5.700000in}{5.700000in}}%
\pgfusepath{clip}%
\pgfsetbuttcap%
\pgfsetroundjoin%
\definecolor{currentfill}{rgb}{0.263663,0.237631,0.518762}%
\pgfsetfillcolor{currentfill}%
\pgfsetfillopacity{0.800000}%
\pgfsetlinewidth{0.000000pt}%
\definecolor{currentstroke}{rgb}{0.000000,0.000000,0.000000}%
\pgfsetstrokecolor{currentstroke}%
\pgfsetdash{}{0pt}%
\pgfpathmoveto{\pgfqpoint{3.244220in}{2.741954in}}%
\pgfpathlineto{\pgfqpoint{3.257625in}{2.728546in}}%
\pgfpathlineto{\pgfqpoint{3.271028in}{2.715410in}}%
\pgfpathlineto{\pgfqpoint{3.284427in}{2.702543in}}%
\pgfpathlineto{\pgfqpoint{3.297824in}{2.689943in}}%
\pgfpathlineto{\pgfqpoint{3.305803in}{2.701808in}}%
\pgfpathlineto{\pgfqpoint{3.313775in}{2.713782in}}%
\pgfpathlineto{\pgfqpoint{3.321741in}{2.725869in}}%
\pgfpathlineto{\pgfqpoint{3.329701in}{2.738070in}}%
\pgfpathlineto{\pgfqpoint{3.316315in}{2.750797in}}%
\pgfpathlineto{\pgfqpoint{3.302927in}{2.763791in}}%
\pgfpathlineto{\pgfqpoint{3.289536in}{2.777054in}}%
\pgfpathlineto{\pgfqpoint{3.276142in}{2.790589in}}%
\pgfpathlineto{\pgfqpoint{3.268171in}{2.778249in}}%
\pgfpathlineto{\pgfqpoint{3.260193in}{2.766031in}}%
\pgfpathlineto{\pgfqpoint{3.252210in}{2.753934in}}%
\pgfpathlineto{\pgfqpoint{3.244220in}{2.741954in}}%
\pgfpathclose%
\pgfusepath{fill}%
\end{pgfscope}%
\begin{pgfscope}%
\pgfpathrectangle{\pgfqpoint{1.150000in}{0.150000in}}{\pgfqpoint{5.700000in}{5.700000in}}%
\pgfusepath{clip}%
\pgfsetbuttcap%
\pgfsetroundjoin%
\definecolor{currentfill}{rgb}{0.197636,0.391528,0.554969}%
\pgfsetfillcolor{currentfill}%
\pgfsetfillopacity{0.800000}%
\pgfsetlinewidth{0.000000pt}%
\definecolor{currentstroke}{rgb}{0.000000,0.000000,0.000000}%
\pgfsetstrokecolor{currentstroke}%
\pgfsetdash{}{0pt}%
\pgfpathmoveto{\pgfqpoint{5.007621in}{3.114575in}}%
\pgfpathlineto{\pgfqpoint{5.021322in}{3.114333in}}%
\pgfpathlineto{\pgfqpoint{5.035035in}{3.114269in}}%
\pgfpathlineto{\pgfqpoint{5.048758in}{3.114383in}}%
\pgfpathlineto{\pgfqpoint{5.062493in}{3.114674in}}%
\pgfpathlineto{\pgfqpoint{5.069993in}{3.127435in}}%
\pgfpathlineto{\pgfqpoint{5.077495in}{3.140440in}}%
\pgfpathlineto{\pgfqpoint{5.084997in}{3.153698in}}%
\pgfpathlineto{\pgfqpoint{5.092501in}{3.167216in}}%
\pgfpathlineto{\pgfqpoint{5.078785in}{3.167588in}}%
\pgfpathlineto{\pgfqpoint{5.065079in}{3.168136in}}%
\pgfpathlineto{\pgfqpoint{5.051384in}{3.168862in}}%
\pgfpathlineto{\pgfqpoint{5.037700in}{3.169766in}}%
\pgfpathlineto{\pgfqpoint{5.030179in}{3.155574in}}%
\pgfpathlineto{\pgfqpoint{5.022659in}{3.141650in}}%
\pgfpathlineto{\pgfqpoint{5.015140in}{3.127987in}}%
\pgfpathlineto{\pgfqpoint{5.007621in}{3.114575in}}%
\pgfpathclose%
\pgfusepath{fill}%
\end{pgfscope}%
\begin{pgfscope}%
\pgfpathrectangle{\pgfqpoint{1.150000in}{0.150000in}}{\pgfqpoint{5.700000in}{5.700000in}}%
\pgfusepath{clip}%
\pgfsetbuttcap%
\pgfsetroundjoin%
\definecolor{currentfill}{rgb}{0.278012,0.180367,0.486697}%
\pgfsetfillcolor{currentfill}%
\pgfsetfillopacity{0.800000}%
\pgfsetlinewidth{0.000000pt}%
\definecolor{currentstroke}{rgb}{0.000000,0.000000,0.000000}%
\pgfsetstrokecolor{currentstroke}%
\pgfsetdash{}{0pt}%
\pgfpathmoveto{\pgfqpoint{3.852185in}{2.599357in}}%
\pgfpathlineto{\pgfqpoint{3.865580in}{2.593950in}}%
\pgfpathlineto{\pgfqpoint{3.878978in}{2.588759in}}%
\pgfpathlineto{\pgfqpoint{3.892381in}{2.583784in}}%
\pgfpathlineto{\pgfqpoint{3.905788in}{2.579023in}}%
\pgfpathlineto{\pgfqpoint{3.913598in}{2.590750in}}%
\pgfpathlineto{\pgfqpoint{3.921402in}{2.602555in}}%
\pgfpathlineto{\pgfqpoint{3.929202in}{2.614443in}}%
\pgfpathlineto{\pgfqpoint{3.936998in}{2.626415in}}%
\pgfpathlineto{\pgfqpoint{3.923598in}{2.631428in}}%
\pgfpathlineto{\pgfqpoint{3.910204in}{2.636656in}}%
\pgfpathlineto{\pgfqpoint{3.896813in}{2.642099in}}%
\pgfpathlineto{\pgfqpoint{3.883427in}{2.647758in}}%
\pgfpathlineto{\pgfqpoint{3.875624in}{2.635522in}}%
\pgfpathlineto{\pgfqpoint{3.867816in}{2.623378in}}%
\pgfpathlineto{\pgfqpoint{3.860003in}{2.611324in}}%
\pgfpathlineto{\pgfqpoint{3.852185in}{2.599357in}}%
\pgfpathclose%
\pgfusepath{fill}%
\end{pgfscope}%
\begin{pgfscope}%
\pgfpathrectangle{\pgfqpoint{1.150000in}{0.150000in}}{\pgfqpoint{5.700000in}{5.700000in}}%
\pgfusepath{clip}%
\pgfsetbuttcap%
\pgfsetroundjoin%
\definecolor{currentfill}{rgb}{0.269308,0.218818,0.509577}%
\pgfsetfillcolor{currentfill}%
\pgfsetfillopacity{0.800000}%
\pgfsetlinewidth{0.000000pt}%
\definecolor{currentstroke}{rgb}{0.000000,0.000000,0.000000}%
\pgfsetstrokecolor{currentstroke}%
\pgfsetdash{}{0pt}%
\pgfpathmoveto{\pgfqpoint{4.160216in}{2.674690in}}%
\pgfpathlineto{\pgfqpoint{4.173674in}{2.671778in}}%
\pgfpathlineto{\pgfqpoint{4.187139in}{2.669067in}}%
\pgfpathlineto{\pgfqpoint{4.200610in}{2.666557in}}%
\pgfpathlineto{\pgfqpoint{4.214089in}{2.664247in}}%
\pgfpathlineto{\pgfqpoint{4.221811in}{2.675805in}}%
\pgfpathlineto{\pgfqpoint{4.229529in}{2.687454in}}%
\pgfpathlineto{\pgfqpoint{4.237243in}{2.699197in}}%
\pgfpathlineto{\pgfqpoint{4.244953in}{2.711039in}}%
\pgfpathlineto{\pgfqpoint{4.231483in}{2.713696in}}%
\pgfpathlineto{\pgfqpoint{4.218020in}{2.716553in}}%
\pgfpathlineto{\pgfqpoint{4.204564in}{2.719610in}}%
\pgfpathlineto{\pgfqpoint{4.191115in}{2.722869in}}%
\pgfpathlineto{\pgfqpoint{4.183397in}{2.710669in}}%
\pgfpathlineto{\pgfqpoint{4.175674in}{2.698575in}}%
\pgfpathlineto{\pgfqpoint{4.167947in}{2.686583in}}%
\pgfpathlineto{\pgfqpoint{4.160216in}{2.674690in}}%
\pgfpathclose%
\pgfusepath{fill}%
\end{pgfscope}%
\begin{pgfscope}%
\pgfpathrectangle{\pgfqpoint{1.150000in}{0.150000in}}{\pgfqpoint{5.700000in}{5.700000in}}%
\pgfusepath{clip}%
\pgfsetbuttcap%
\pgfsetroundjoin%
\definecolor{currentfill}{rgb}{0.210503,0.363727,0.552206}%
\pgfsetfillcolor{currentfill}%
\pgfsetfillopacity{0.800000}%
\pgfsetlinewidth{0.000000pt}%
\definecolor{currentstroke}{rgb}{0.000000,0.000000,0.000000}%
\pgfsetstrokecolor{currentstroke}%
\pgfsetdash{}{0pt}%
\pgfpathmoveto{\pgfqpoint{2.975079in}{3.070635in}}%
\pgfpathlineto{\pgfqpoint{2.988600in}{3.051301in}}%
\pgfpathlineto{\pgfqpoint{3.002112in}{3.032290in}}%
\pgfpathlineto{\pgfqpoint{3.015616in}{3.013597in}}%
\pgfpathlineto{\pgfqpoint{3.029112in}{2.995220in}}%
\pgfpathlineto{\pgfqpoint{3.037153in}{3.007661in}}%
\pgfpathlineto{\pgfqpoint{3.045187in}{3.020252in}}%
\pgfpathlineto{\pgfqpoint{3.053213in}{3.032995in}}%
\pgfpathlineto{\pgfqpoint{3.061232in}{3.045892in}}%
\pgfpathlineto{\pgfqpoint{3.047749in}{3.064396in}}%
\pgfpathlineto{\pgfqpoint{3.034258in}{3.083217in}}%
\pgfpathlineto{\pgfqpoint{3.020760in}{3.102357in}}%
\pgfpathlineto{\pgfqpoint{3.007253in}{3.121818in}}%
\pgfpathlineto{\pgfqpoint{2.999221in}{3.108781in}}%
\pgfpathlineto{\pgfqpoint{2.991181in}{3.095906in}}%
\pgfpathlineto{\pgfqpoint{2.983134in}{3.083192in}}%
\pgfpathlineto{\pgfqpoint{2.975079in}{3.070635in}}%
\pgfpathclose%
\pgfusepath{fill}%
\end{pgfscope}%
\begin{pgfscope}%
\pgfpathrectangle{\pgfqpoint{1.150000in}{0.150000in}}{\pgfqpoint{5.700000in}{5.700000in}}%
\pgfusepath{clip}%
\pgfsetbuttcap%
\pgfsetroundjoin%
\definecolor{currentfill}{rgb}{0.187231,0.414746,0.556547}%
\pgfsetfillcolor{currentfill}%
\pgfsetfillopacity{0.800000}%
\pgfsetlinewidth{0.000000pt}%
\definecolor{currentstroke}{rgb}{0.000000,0.000000,0.000000}%
\pgfsetstrokecolor{currentstroke}%
\pgfsetdash{}{0pt}%
\pgfpathmoveto{\pgfqpoint{5.092501in}{3.167216in}}%
\pgfpathlineto{\pgfqpoint{5.106229in}{3.167022in}}%
\pgfpathlineto{\pgfqpoint{5.119968in}{3.167003in}}%
\pgfpathlineto{\pgfqpoint{5.133719in}{3.167161in}}%
\pgfpathlineto{\pgfqpoint{5.147481in}{3.167495in}}%
\pgfpathlineto{\pgfqpoint{5.154968in}{3.180598in}}%
\pgfpathlineto{\pgfqpoint{5.162456in}{3.193970in}}%
\pgfpathlineto{\pgfqpoint{5.169946in}{3.207618in}}%
\pgfpathlineto{\pgfqpoint{5.177439in}{3.221549in}}%
\pgfpathlineto{\pgfqpoint{5.163696in}{3.221910in}}%
\pgfpathlineto{\pgfqpoint{5.149964in}{3.222446in}}%
\pgfpathlineto{\pgfqpoint{5.136244in}{3.223158in}}%
\pgfpathlineto{\pgfqpoint{5.122534in}{3.224046in}}%
\pgfpathlineto{\pgfqpoint{5.115023in}{3.209409in}}%
\pgfpathlineto{\pgfqpoint{5.107514in}{3.195064in}}%
\pgfpathlineto{\pgfqpoint{5.100007in}{3.181002in}}%
\pgfpathlineto{\pgfqpoint{5.092501in}{3.167216in}}%
\pgfpathclose%
\pgfusepath{fill}%
\end{pgfscope}%
\begin{pgfscope}%
\pgfpathrectangle{\pgfqpoint{1.150000in}{0.150000in}}{\pgfqpoint{5.700000in}{5.700000in}}%
\pgfusepath{clip}%
\pgfsetbuttcap%
\pgfsetroundjoin%
\definecolor{currentfill}{rgb}{0.278826,0.175490,0.483397}%
\pgfsetfillcolor{currentfill}%
\pgfsetfillopacity{0.800000}%
\pgfsetlinewidth{0.000000pt}%
\definecolor{currentstroke}{rgb}{0.000000,0.000000,0.000000}%
\pgfsetstrokecolor{currentstroke}%
\pgfsetdash{}{0pt}%
\pgfpathmoveto{\pgfqpoint{3.628809in}{2.583634in}}%
\pgfpathlineto{\pgfqpoint{3.642182in}{2.575950in}}%
\pgfpathlineto{\pgfqpoint{3.655558in}{2.568498in}}%
\pgfpathlineto{\pgfqpoint{3.668936in}{2.561274in}}%
\pgfpathlineto{\pgfqpoint{3.682316in}{2.554279in}}%
\pgfpathlineto{\pgfqpoint{3.690190in}{2.566013in}}%
\pgfpathlineto{\pgfqpoint{3.698059in}{2.577827in}}%
\pgfpathlineto{\pgfqpoint{3.705923in}{2.589724in}}%
\pgfpathlineto{\pgfqpoint{3.713782in}{2.601706in}}%
\pgfpathlineto{\pgfqpoint{3.700410in}{2.608891in}}%
\pgfpathlineto{\pgfqpoint{3.687041in}{2.616304in}}%
\pgfpathlineto{\pgfqpoint{3.673674in}{2.623946in}}%
\pgfpathlineto{\pgfqpoint{3.660309in}{2.631819in}}%
\pgfpathlineto{\pgfqpoint{3.652442in}{2.619636in}}%
\pgfpathlineto{\pgfqpoint{3.644570in}{2.607546in}}%
\pgfpathlineto{\pgfqpoint{3.636692in}{2.595546in}}%
\pgfpathlineto{\pgfqpoint{3.628809in}{2.583634in}}%
\pgfpathclose%
\pgfusepath{fill}%
\end{pgfscope}%
\begin{pgfscope}%
\pgfpathrectangle{\pgfqpoint{1.150000in}{0.150000in}}{\pgfqpoint{5.700000in}{5.700000in}}%
\pgfusepath{clip}%
\pgfsetbuttcap%
\pgfsetroundjoin%
\definecolor{currentfill}{rgb}{0.278012,0.180367,0.486697}%
\pgfsetfillcolor{currentfill}%
\pgfsetfillopacity{0.800000}%
\pgfsetlinewidth{0.000000pt}%
\definecolor{currentstroke}{rgb}{0.000000,0.000000,0.000000}%
\pgfsetstrokecolor{currentstroke}%
\pgfsetdash{}{0pt}%
\pgfpathmoveto{\pgfqpoint{3.490215in}{2.605496in}}%
\pgfpathlineto{\pgfqpoint{3.503588in}{2.596074in}}%
\pgfpathlineto{\pgfqpoint{3.516962in}{2.586894in}}%
\pgfpathlineto{\pgfqpoint{3.530336in}{2.577955in}}%
\pgfpathlineto{\pgfqpoint{3.543711in}{2.569255in}}%
\pgfpathlineto{\pgfqpoint{3.551625in}{2.580985in}}%
\pgfpathlineto{\pgfqpoint{3.559533in}{2.592801in}}%
\pgfpathlineto{\pgfqpoint{3.567435in}{2.604707in}}%
\pgfpathlineto{\pgfqpoint{3.575333in}{2.616704in}}%
\pgfpathlineto{\pgfqpoint{3.561967in}{2.625562in}}%
\pgfpathlineto{\pgfqpoint{3.548602in}{2.634660in}}%
\pgfpathlineto{\pgfqpoint{3.535238in}{2.643998in}}%
\pgfpathlineto{\pgfqpoint{3.521875in}{2.653579in}}%
\pgfpathlineto{\pgfqpoint{3.513968in}{2.641412in}}%
\pgfpathlineto{\pgfqpoint{3.506056in}{2.629344in}}%
\pgfpathlineto{\pgfqpoint{3.498138in}{2.617373in}}%
\pgfpathlineto{\pgfqpoint{3.490215in}{2.605496in}}%
\pgfpathclose%
\pgfusepath{fill}%
\end{pgfscope}%
\begin{pgfscope}%
\pgfpathrectangle{\pgfqpoint{1.150000in}{0.150000in}}{\pgfqpoint{5.700000in}{5.700000in}}%
\pgfusepath{clip}%
\pgfsetbuttcap%
\pgfsetroundjoin%
\definecolor{currentfill}{rgb}{0.269308,0.218818,0.509577}%
\pgfsetfillcolor{currentfill}%
\pgfsetfillopacity{0.800000}%
\pgfsetlinewidth{0.000000pt}%
\definecolor{currentstroke}{rgb}{0.000000,0.000000,0.000000}%
\pgfsetstrokecolor{currentstroke}%
\pgfsetdash{}{0pt}%
\pgfpathmoveto{\pgfqpoint{3.297824in}{2.689943in}}%
\pgfpathlineto{\pgfqpoint{3.311219in}{2.677608in}}%
\pgfpathlineto{\pgfqpoint{3.324611in}{2.665537in}}%
\pgfpathlineto{\pgfqpoint{3.338002in}{2.653727in}}%
\pgfpathlineto{\pgfqpoint{3.351390in}{2.642176in}}%
\pgfpathlineto{\pgfqpoint{3.359358in}{2.653925in}}%
\pgfpathlineto{\pgfqpoint{3.367319in}{2.665777in}}%
\pgfpathlineto{\pgfqpoint{3.375274in}{2.677733in}}%
\pgfpathlineto{\pgfqpoint{3.383223in}{2.689795in}}%
\pgfpathlineto{\pgfqpoint{3.369845in}{2.701472in}}%
\pgfpathlineto{\pgfqpoint{3.356466in}{2.713410in}}%
\pgfpathlineto{\pgfqpoint{3.343084in}{2.725608in}}%
\pgfpathlineto{\pgfqpoint{3.329701in}{2.738070in}}%
\pgfpathlineto{\pgfqpoint{3.321741in}{2.725869in}}%
\pgfpathlineto{\pgfqpoint{3.313775in}{2.713782in}}%
\pgfpathlineto{\pgfqpoint{3.305803in}{2.701808in}}%
\pgfpathlineto{\pgfqpoint{3.297824in}{2.689943in}}%
\pgfpathclose%
\pgfusepath{fill}%
\end{pgfscope}%
\begin{pgfscope}%
\pgfpathrectangle{\pgfqpoint{1.150000in}{0.150000in}}{\pgfqpoint{5.700000in}{5.700000in}}%
\pgfusepath{clip}%
\pgfsetbuttcap%
\pgfsetroundjoin%
\definecolor{currentfill}{rgb}{0.273006,0.204520,0.501721}%
\pgfsetfillcolor{currentfill}%
\pgfsetfillopacity{0.800000}%
\pgfsetlinewidth{0.000000pt}%
\definecolor{currentstroke}{rgb}{0.000000,0.000000,0.000000}%
\pgfsetstrokecolor{currentstroke}%
\pgfsetdash{}{0pt}%
\pgfpathmoveto{\pgfqpoint{4.075451in}{2.640444in}}%
\pgfpathlineto{\pgfqpoint{4.088891in}{2.637032in}}%
\pgfpathlineto{\pgfqpoint{4.102338in}{2.633825in}}%
\pgfpathlineto{\pgfqpoint{4.115791in}{2.630823in}}%
\pgfpathlineto{\pgfqpoint{4.129250in}{2.628023in}}%
\pgfpathlineto{\pgfqpoint{4.136998in}{2.639562in}}%
\pgfpathlineto{\pgfqpoint{4.144742in}{2.651184in}}%
\pgfpathlineto{\pgfqpoint{4.152481in}{2.662892in}}%
\pgfpathlineto{\pgfqpoint{4.160216in}{2.674690in}}%
\pgfpathlineto{\pgfqpoint{4.146765in}{2.677805in}}%
\pgfpathlineto{\pgfqpoint{4.133321in}{2.681123in}}%
\pgfpathlineto{\pgfqpoint{4.119882in}{2.684645in}}%
\pgfpathlineto{\pgfqpoint{4.106450in}{2.688372in}}%
\pgfpathlineto{\pgfqpoint{4.098707in}{2.676247in}}%
\pgfpathlineto{\pgfqpoint{4.090959in}{2.664220in}}%
\pgfpathlineto{\pgfqpoint{4.083207in}{2.652287in}}%
\pgfpathlineto{\pgfqpoint{4.075451in}{2.640444in}}%
\pgfpathclose%
\pgfusepath{fill}%
\end{pgfscope}%
\begin{pgfscope}%
\pgfpathrectangle{\pgfqpoint{1.150000in}{0.150000in}}{\pgfqpoint{5.700000in}{5.700000in}}%
\pgfusepath{clip}%
\pgfsetbuttcap%
\pgfsetroundjoin%
\definecolor{currentfill}{rgb}{0.179019,0.433756,0.557430}%
\pgfsetfillcolor{currentfill}%
\pgfsetfillopacity{0.800000}%
\pgfsetlinewidth{0.000000pt}%
\definecolor{currentstroke}{rgb}{0.000000,0.000000,0.000000}%
\pgfsetstrokecolor{currentstroke}%
\pgfsetdash{}{0pt}%
\pgfpathmoveto{\pgfqpoint{5.177439in}{3.221549in}}%
\pgfpathlineto{\pgfqpoint{5.191193in}{3.221364in}}%
\pgfpathlineto{\pgfqpoint{5.204959in}{3.221353in}}%
\pgfpathlineto{\pgfqpoint{5.218737in}{3.221517in}}%
\pgfpathlineto{\pgfqpoint{5.232526in}{3.221855in}}%
\pgfpathlineto{\pgfqpoint{5.240001in}{3.235365in}}%
\pgfpathlineto{\pgfqpoint{5.247479in}{3.249167in}}%
\pgfpathlineto{\pgfqpoint{5.254960in}{3.263271in}}%
\pgfpathlineto{\pgfqpoint{5.262444in}{3.277684in}}%
\pgfpathlineto{\pgfqpoint{5.248675in}{3.278072in}}%
\pgfpathlineto{\pgfqpoint{5.234918in}{3.278633in}}%
\pgfpathlineto{\pgfqpoint{5.221172in}{3.279369in}}%
\pgfpathlineto{\pgfqpoint{5.207437in}{3.280280in}}%
\pgfpathlineto{\pgfqpoint{5.199933in}{3.265130in}}%
\pgfpathlineto{\pgfqpoint{5.192432in}{3.250297in}}%
\pgfpathlineto{\pgfqpoint{5.184934in}{3.235773in}}%
\pgfpathlineto{\pgfqpoint{5.177439in}{3.221549in}}%
\pgfpathclose%
\pgfusepath{fill}%
\end{pgfscope}%
\begin{pgfscope}%
\pgfpathrectangle{\pgfqpoint{1.150000in}{0.150000in}}{\pgfqpoint{5.700000in}{5.700000in}}%
\pgfusepath{clip}%
\pgfsetbuttcap%
\pgfsetroundjoin%
\definecolor{currentfill}{rgb}{0.279574,0.170599,0.479997}%
\pgfsetfillcolor{currentfill}%
\pgfsetfillopacity{0.800000}%
\pgfsetlinewidth{0.000000pt}%
\definecolor{currentstroke}{rgb}{0.000000,0.000000,0.000000}%
\pgfsetstrokecolor{currentstroke}%
\pgfsetdash{}{0pt}%
\pgfpathmoveto{\pgfqpoint{3.767297in}{2.575223in}}%
\pgfpathlineto{\pgfqpoint{3.780684in}{2.569160in}}%
\pgfpathlineto{\pgfqpoint{3.794074in}{2.563319in}}%
\pgfpathlineto{\pgfqpoint{3.807469in}{2.557697in}}%
\pgfpathlineto{\pgfqpoint{3.820867in}{2.552294in}}%
\pgfpathlineto{\pgfqpoint{3.828704in}{2.563944in}}%
\pgfpathlineto{\pgfqpoint{3.836536in}{2.575670in}}%
\pgfpathlineto{\pgfqpoint{3.844363in}{2.587473in}}%
\pgfpathlineto{\pgfqpoint{3.852185in}{2.599357in}}%
\pgfpathlineto{\pgfqpoint{3.838795in}{2.604981in}}%
\pgfpathlineto{\pgfqpoint{3.825409in}{2.610824in}}%
\pgfpathlineto{\pgfqpoint{3.812027in}{2.616887in}}%
\pgfpathlineto{\pgfqpoint{3.798649in}{2.623171in}}%
\pgfpathlineto{\pgfqpoint{3.790818in}{2.611054in}}%
\pgfpathlineto{\pgfqpoint{3.782983in}{2.599026in}}%
\pgfpathlineto{\pgfqpoint{3.775142in}{2.587083in}}%
\pgfpathlineto{\pgfqpoint{3.767297in}{2.575223in}}%
\pgfpathclose%
\pgfusepath{fill}%
\end{pgfscope}%
\begin{pgfscope}%
\pgfpathrectangle{\pgfqpoint{1.150000in}{0.150000in}}{\pgfqpoint{5.700000in}{5.700000in}}%
\pgfusepath{clip}%
\pgfsetbuttcap%
\pgfsetroundjoin%
\definecolor{currentfill}{rgb}{0.195860,0.395433,0.555276}%
\pgfsetfillcolor{currentfill}%
\pgfsetfillopacity{0.800000}%
\pgfsetlinewidth{0.000000pt}%
\definecolor{currentstroke}{rgb}{0.000000,0.000000,0.000000}%
\pgfsetstrokecolor{currentstroke}%
\pgfsetdash{}{0pt}%
\pgfpathmoveto{\pgfqpoint{2.920908in}{3.151249in}}%
\pgfpathlineto{\pgfqpoint{2.934465in}{3.130597in}}%
\pgfpathlineto{\pgfqpoint{2.948012in}{3.110280in}}%
\pgfpathlineto{\pgfqpoint{2.961550in}{3.090293in}}%
\pgfpathlineto{\pgfqpoint{2.975079in}{3.070635in}}%
\pgfpathlineto{\pgfqpoint{2.983134in}{3.083192in}}%
\pgfpathlineto{\pgfqpoint{2.991181in}{3.095906in}}%
\pgfpathlineto{\pgfqpoint{2.999221in}{3.108781in}}%
\pgfpathlineto{\pgfqpoint{3.007253in}{3.121818in}}%
\pgfpathlineto{\pgfqpoint{2.993737in}{3.141605in}}%
\pgfpathlineto{\pgfqpoint{2.980213in}{3.161719in}}%
\pgfpathlineto{\pgfqpoint{2.966680in}{3.182165in}}%
\pgfpathlineto{\pgfqpoint{2.953137in}{3.202946in}}%
\pgfpathlineto{\pgfqpoint{2.945092in}{3.189768in}}%
\pgfpathlineto{\pgfqpoint{2.937038in}{3.176761in}}%
\pgfpathlineto{\pgfqpoint{2.928977in}{3.163922in}}%
\pgfpathlineto{\pgfqpoint{2.920908in}{3.151249in}}%
\pgfpathclose%
\pgfusepath{fill}%
\end{pgfscope}%
\begin{pgfscope}%
\pgfpathrectangle{\pgfqpoint{1.150000in}{0.150000in}}{\pgfqpoint{5.700000in}{5.700000in}}%
\pgfusepath{clip}%
\pgfsetbuttcap%
\pgfsetroundjoin%
\definecolor{currentfill}{rgb}{0.274128,0.199721,0.498911}%
\pgfsetfillcolor{currentfill}%
\pgfsetfillopacity{0.800000}%
\pgfsetlinewidth{0.000000pt}%
\definecolor{currentstroke}{rgb}{0.000000,0.000000,0.000000}%
\pgfsetstrokecolor{currentstroke}%
\pgfsetdash{}{0pt}%
\pgfpathmoveto{\pgfqpoint{3.351390in}{2.642176in}}%
\pgfpathlineto{\pgfqpoint{3.364778in}{2.630883in}}%
\pgfpathlineto{\pgfqpoint{3.378164in}{2.619846in}}%
\pgfpathlineto{\pgfqpoint{3.391549in}{2.609062in}}%
\pgfpathlineto{\pgfqpoint{3.404933in}{2.598531in}}%
\pgfpathlineto{\pgfqpoint{3.412889in}{2.610165in}}%
\pgfpathlineto{\pgfqpoint{3.420840in}{2.621893in}}%
\pgfpathlineto{\pgfqpoint{3.428785in}{2.633718in}}%
\pgfpathlineto{\pgfqpoint{3.436723in}{2.645641in}}%
\pgfpathlineto{\pgfqpoint{3.423350in}{2.656300in}}%
\pgfpathlineto{\pgfqpoint{3.409975in}{2.667210in}}%
\pgfpathlineto{\pgfqpoint{3.396600in}{2.678374in}}%
\pgfpathlineto{\pgfqpoint{3.383223in}{2.689795in}}%
\pgfpathlineto{\pgfqpoint{3.375274in}{2.677733in}}%
\pgfpathlineto{\pgfqpoint{3.367319in}{2.665777in}}%
\pgfpathlineto{\pgfqpoint{3.359358in}{2.653925in}}%
\pgfpathlineto{\pgfqpoint{3.351390in}{2.642176in}}%
\pgfpathclose%
\pgfusepath{fill}%
\end{pgfscope}%
\begin{pgfscope}%
\pgfpathrectangle{\pgfqpoint{1.150000in}{0.150000in}}{\pgfqpoint{5.700000in}{5.700000in}}%
\pgfusepath{clip}%
\pgfsetbuttcap%
\pgfsetroundjoin%
\definecolor{currentfill}{rgb}{0.276194,0.190074,0.493001}%
\pgfsetfillcolor{currentfill}%
\pgfsetfillopacity{0.800000}%
\pgfsetlinewidth{0.000000pt}%
\definecolor{currentstroke}{rgb}{0.000000,0.000000,0.000000}%
\pgfsetstrokecolor{currentstroke}%
\pgfsetdash{}{0pt}%
\pgfpathmoveto{\pgfqpoint{3.990645in}{2.608485in}}%
\pgfpathlineto{\pgfqpoint{4.004070in}{2.604529in}}%
\pgfpathlineto{\pgfqpoint{4.017501in}{2.600781in}}%
\pgfpathlineto{\pgfqpoint{4.030938in}{2.597241in}}%
\pgfpathlineto{\pgfqpoint{4.044380in}{2.593907in}}%
\pgfpathlineto{\pgfqpoint{4.052154in}{2.605424in}}%
\pgfpathlineto{\pgfqpoint{4.059924in}{2.617016in}}%
\pgfpathlineto{\pgfqpoint{4.067690in}{2.628689in}}%
\pgfpathlineto{\pgfqpoint{4.075451in}{2.640444in}}%
\pgfpathlineto{\pgfqpoint{4.062016in}{2.644062in}}%
\pgfpathlineto{\pgfqpoint{4.048588in}{2.647886in}}%
\pgfpathlineto{\pgfqpoint{4.035165in}{2.651918in}}%
\pgfpathlineto{\pgfqpoint{4.021748in}{2.656158in}}%
\pgfpathlineto{\pgfqpoint{4.013979in}{2.644107in}}%
\pgfpathlineto{\pgfqpoint{4.006206in}{2.632147in}}%
\pgfpathlineto{\pgfqpoint{3.998428in}{2.620274in}}%
\pgfpathlineto{\pgfqpoint{3.990645in}{2.608485in}}%
\pgfpathclose%
\pgfusepath{fill}%
\end{pgfscope}%
\begin{pgfscope}%
\pgfpathrectangle{\pgfqpoint{1.150000in}{0.150000in}}{\pgfqpoint{5.700000in}{5.700000in}}%
\pgfusepath{clip}%
\pgfsetbuttcap%
\pgfsetroundjoin%
\definecolor{currentfill}{rgb}{0.171176,0.452530,0.557965}%
\pgfsetfillcolor{currentfill}%
\pgfsetfillopacity{0.800000}%
\pgfsetlinewidth{0.000000pt}%
\definecolor{currentstroke}{rgb}{0.000000,0.000000,0.000000}%
\pgfsetstrokecolor{currentstroke}%
\pgfsetdash{}{0pt}%
\pgfpathmoveto{\pgfqpoint{5.262444in}{3.277684in}}%
\pgfpathlineto{\pgfqpoint{5.276224in}{3.277470in}}%
\pgfpathlineto{\pgfqpoint{5.290017in}{3.277429in}}%
\pgfpathlineto{\pgfqpoint{5.303821in}{3.277561in}}%
\pgfpathlineto{\pgfqpoint{5.317637in}{3.277866in}}%
\pgfpathlineto{\pgfqpoint{5.325104in}{3.291851in}}%
\pgfpathlineto{\pgfqpoint{5.332574in}{3.306155in}}%
\pgfpathlineto{\pgfqpoint{5.340049in}{3.320786in}}%
\pgfpathlineto{\pgfqpoint{5.326249in}{3.321047in}}%
\pgfpathlineto{\pgfqpoint{5.312461in}{3.321480in}}%
\pgfpathlineto{\pgfqpoint{5.298685in}{3.322085in}}%
\pgfpathlineto{\pgfqpoint{5.284921in}{3.322864in}}%
\pgfpathlineto{\pgfqpoint{5.277424in}{3.307472in}}%
\pgfpathlineto{\pgfqpoint{5.269932in}{3.292415in}}%
\pgfpathlineto{\pgfqpoint{5.262444in}{3.277684in}}%
\pgfpathclose%
\pgfusepath{fill}%
\end{pgfscope}%
\begin{pgfscope}%
\pgfpathrectangle{\pgfqpoint{1.150000in}{0.150000in}}{\pgfqpoint{5.700000in}{5.700000in}}%
\pgfusepath{clip}%
\pgfsetbuttcap%
\pgfsetroundjoin%
\definecolor{currentfill}{rgb}{0.279574,0.170599,0.479997}%
\pgfsetfillcolor{currentfill}%
\pgfsetfillopacity{0.800000}%
\pgfsetlinewidth{0.000000pt}%
\definecolor{currentstroke}{rgb}{0.000000,0.000000,0.000000}%
\pgfsetstrokecolor{currentstroke}%
\pgfsetdash{}{0pt}%
\pgfpathmoveto{\pgfqpoint{3.543711in}{2.569255in}}%
\pgfpathlineto{\pgfqpoint{3.557087in}{2.560793in}}%
\pgfpathlineto{\pgfqpoint{3.570465in}{2.552567in}}%
\pgfpathlineto{\pgfqpoint{3.583844in}{2.544576in}}%
\pgfpathlineto{\pgfqpoint{3.597224in}{2.536819in}}%
\pgfpathlineto{\pgfqpoint{3.605129in}{2.548402in}}%
\pgfpathlineto{\pgfqpoint{3.613027in}{2.560064in}}%
\pgfpathlineto{\pgfqpoint{3.620921in}{2.571808in}}%
\pgfpathlineto{\pgfqpoint{3.628809in}{2.583634in}}%
\pgfpathlineto{\pgfqpoint{3.615438in}{2.591550in}}%
\pgfpathlineto{\pgfqpoint{3.602068in}{2.599699in}}%
\pgfpathlineto{\pgfqpoint{3.588700in}{2.608083in}}%
\pgfpathlineto{\pgfqpoint{3.575333in}{2.616704in}}%
\pgfpathlineto{\pgfqpoint{3.567435in}{2.604707in}}%
\pgfpathlineto{\pgfqpoint{3.559533in}{2.592801in}}%
\pgfpathlineto{\pgfqpoint{3.551625in}{2.580985in}}%
\pgfpathlineto{\pgfqpoint{3.543711in}{2.569255in}}%
\pgfpathclose%
\pgfusepath{fill}%
\end{pgfscope}%
\begin{pgfscope}%
\pgfpathrectangle{\pgfqpoint{1.150000in}{0.150000in}}{\pgfqpoint{5.700000in}{5.700000in}}%
\pgfusepath{clip}%
\pgfsetbuttcap%
\pgfsetroundjoin%
\definecolor{currentfill}{rgb}{0.237441,0.305202,0.541921}%
\pgfsetfillcolor{currentfill}%
\pgfsetfillopacity{0.800000}%
\pgfsetlinewidth{0.000000pt}%
\definecolor{currentstroke}{rgb}{0.000000,0.000000,0.000000}%
\pgfsetstrokecolor{currentstroke}%
\pgfsetdash{}{0pt}%
\pgfpathmoveto{\pgfqpoint{4.638118in}{2.871507in}}%
\pgfpathlineto{\pgfqpoint{4.651725in}{2.871190in}}%
\pgfpathlineto{\pgfqpoint{4.665343in}{2.871060in}}%
\pgfpathlineto{\pgfqpoint{4.678970in}{2.871115in}}%
\pgfpathlineto{\pgfqpoint{4.692607in}{2.871355in}}%
\pgfpathlineto{\pgfqpoint{4.700194in}{2.882654in}}%
\pgfpathlineto{\pgfqpoint{4.707779in}{2.894097in}}%
\pgfpathlineto{\pgfqpoint{4.715361in}{2.905687in}}%
\pgfpathlineto{\pgfqpoint{4.722940in}{2.917433in}}%
\pgfpathlineto{\pgfqpoint{4.709316in}{2.917698in}}%
\pgfpathlineto{\pgfqpoint{4.695702in}{2.918148in}}%
\pgfpathlineto{\pgfqpoint{4.682097in}{2.918784in}}%
\pgfpathlineto{\pgfqpoint{4.668502in}{2.919605in}}%
\pgfpathlineto{\pgfqpoint{4.660910in}{2.907343in}}%
\pgfpathlineto{\pgfqpoint{4.653315in}{2.895243in}}%
\pgfpathlineto{\pgfqpoint{4.645718in}{2.883300in}}%
\pgfpathlineto{\pgfqpoint{4.638118in}{2.871507in}}%
\pgfpathclose%
\pgfusepath{fill}%
\end{pgfscope}%
\begin{pgfscope}%
\pgfpathrectangle{\pgfqpoint{1.150000in}{0.150000in}}{\pgfqpoint{5.700000in}{5.700000in}}%
\pgfusepath{clip}%
\pgfsetbuttcap%
\pgfsetroundjoin%
\definecolor{currentfill}{rgb}{0.244972,0.287675,0.537260}%
\pgfsetfillcolor{currentfill}%
\pgfsetfillopacity{0.800000}%
\pgfsetlinewidth{0.000000pt}%
\definecolor{currentstroke}{rgb}{0.000000,0.000000,0.000000}%
\pgfsetstrokecolor{currentstroke}%
\pgfsetdash{}{0pt}%
\pgfpathmoveto{\pgfqpoint{4.553310in}{2.826969in}}%
\pgfpathlineto{\pgfqpoint{4.566891in}{2.826377in}}%
\pgfpathlineto{\pgfqpoint{4.580481in}{2.825972in}}%
\pgfpathlineto{\pgfqpoint{4.594081in}{2.825756in}}%
\pgfpathlineto{\pgfqpoint{4.607690in}{2.825727in}}%
\pgfpathlineto{\pgfqpoint{4.615302in}{2.836974in}}%
\pgfpathlineto{\pgfqpoint{4.622910in}{2.848350in}}%
\pgfpathlineto{\pgfqpoint{4.630516in}{2.859859in}}%
\pgfpathlineto{\pgfqpoint{4.638118in}{2.871507in}}%
\pgfpathlineto{\pgfqpoint{4.624521in}{2.872010in}}%
\pgfpathlineto{\pgfqpoint{4.610933in}{2.872700in}}%
\pgfpathlineto{\pgfqpoint{4.597354in}{2.873577in}}%
\pgfpathlineto{\pgfqpoint{4.583785in}{2.874643in}}%
\pgfpathlineto{\pgfqpoint{4.576171in}{2.862510in}}%
\pgfpathlineto{\pgfqpoint{4.568554in}{2.850524in}}%
\pgfpathlineto{\pgfqpoint{4.560933in}{2.838678in}}%
\pgfpathlineto{\pgfqpoint{4.553310in}{2.826969in}}%
\pgfpathclose%
\pgfusepath{fill}%
\end{pgfscope}%
\begin{pgfscope}%
\pgfpathrectangle{\pgfqpoint{1.150000in}{0.150000in}}{\pgfqpoint{5.700000in}{5.700000in}}%
\pgfusepath{clip}%
\pgfsetbuttcap%
\pgfsetroundjoin%
\definecolor{currentfill}{rgb}{0.280255,0.165693,0.476498}%
\pgfsetfillcolor{currentfill}%
\pgfsetfillopacity{0.800000}%
\pgfsetlinewidth{0.000000pt}%
\definecolor{currentstroke}{rgb}{0.000000,0.000000,0.000000}%
\pgfsetstrokecolor{currentstroke}%
\pgfsetdash{}{0pt}%
\pgfpathmoveto{\pgfqpoint{3.682316in}{2.554279in}}%
\pgfpathlineto{\pgfqpoint{3.695699in}{2.547511in}}%
\pgfpathlineto{\pgfqpoint{3.709085in}{2.540969in}}%
\pgfpathlineto{\pgfqpoint{3.722474in}{2.534651in}}%
\pgfpathlineto{\pgfqpoint{3.735866in}{2.528557in}}%
\pgfpathlineto{\pgfqpoint{3.743731in}{2.540112in}}%
\pgfpathlineto{\pgfqpoint{3.751591in}{2.551740in}}%
\pgfpathlineto{\pgfqpoint{3.759447in}{2.563443in}}%
\pgfpathlineto{\pgfqpoint{3.767297in}{2.575223in}}%
\pgfpathlineto{\pgfqpoint{3.753913in}{2.581508in}}%
\pgfpathlineto{\pgfqpoint{3.740533in}{2.588015in}}%
\pgfpathlineto{\pgfqpoint{3.727156in}{2.594748in}}%
\pgfpathlineto{\pgfqpoint{3.713782in}{2.601706in}}%
\pgfpathlineto{\pgfqpoint{3.705923in}{2.589724in}}%
\pgfpathlineto{\pgfqpoint{3.698059in}{2.577827in}}%
\pgfpathlineto{\pgfqpoint{3.690190in}{2.566013in}}%
\pgfpathlineto{\pgfqpoint{3.682316in}{2.554279in}}%
\pgfpathclose%
\pgfusepath{fill}%
\end{pgfscope}%
\begin{pgfscope}%
\pgfpathrectangle{\pgfqpoint{1.150000in}{0.150000in}}{\pgfqpoint{5.700000in}{5.700000in}}%
\pgfusepath{clip}%
\pgfsetbuttcap%
\pgfsetroundjoin%
\definecolor{currentfill}{rgb}{0.278012,0.180367,0.486697}%
\pgfsetfillcolor{currentfill}%
\pgfsetfillopacity{0.800000}%
\pgfsetlinewidth{0.000000pt}%
\definecolor{currentstroke}{rgb}{0.000000,0.000000,0.000000}%
\pgfsetstrokecolor{currentstroke}%
\pgfsetdash{}{0pt}%
\pgfpathmoveto{\pgfqpoint{3.905788in}{2.579023in}}%
\pgfpathlineto{\pgfqpoint{3.919201in}{2.574476in}}%
\pgfpathlineto{\pgfqpoint{3.932618in}{2.570141in}}%
\pgfpathlineto{\pgfqpoint{3.946041in}{2.566018in}}%
\pgfpathlineto{\pgfqpoint{3.959468in}{2.562105in}}%
\pgfpathlineto{\pgfqpoint{3.967270in}{2.573590in}}%
\pgfpathlineto{\pgfqpoint{3.975066in}{2.585147in}}%
\pgfpathlineto{\pgfqpoint{3.982858in}{2.596777in}}%
\pgfpathlineto{\pgfqpoint{3.990645in}{2.608485in}}%
\pgfpathlineto{\pgfqpoint{3.977226in}{2.612651in}}%
\pgfpathlineto{\pgfqpoint{3.963811in}{2.617028in}}%
\pgfpathlineto{\pgfqpoint{3.950402in}{2.621615in}}%
\pgfpathlineto{\pgfqpoint{3.936998in}{2.626415in}}%
\pgfpathlineto{\pgfqpoint{3.929202in}{2.614443in}}%
\pgfpathlineto{\pgfqpoint{3.921402in}{2.602555in}}%
\pgfpathlineto{\pgfqpoint{3.913598in}{2.590750in}}%
\pgfpathlineto{\pgfqpoint{3.905788in}{2.579023in}}%
\pgfpathclose%
\pgfusepath{fill}%
\end{pgfscope}%
\begin{pgfscope}%
\pgfpathrectangle{\pgfqpoint{1.150000in}{0.150000in}}{\pgfqpoint{5.700000in}{5.700000in}}%
\pgfusepath{clip}%
\pgfsetbuttcap%
\pgfsetroundjoin%
\definecolor{currentfill}{rgb}{0.227802,0.326594,0.546532}%
\pgfsetfillcolor{currentfill}%
\pgfsetfillopacity{0.800000}%
\pgfsetlinewidth{0.000000pt}%
\definecolor{currentstroke}{rgb}{0.000000,0.000000,0.000000}%
\pgfsetstrokecolor{currentstroke}%
\pgfsetdash{}{0pt}%
\pgfpathmoveto{\pgfqpoint{4.722940in}{2.917433in}}%
\pgfpathlineto{\pgfqpoint{4.736575in}{2.917352in}}%
\pgfpathlineto{\pgfqpoint{4.750219in}{2.917455in}}%
\pgfpathlineto{\pgfqpoint{4.763874in}{2.917741in}}%
\pgfpathlineto{\pgfqpoint{4.777540in}{2.918211in}}%
\pgfpathlineto{\pgfqpoint{4.785103in}{2.929592in}}%
\pgfpathlineto{\pgfqpoint{4.792665in}{2.941131in}}%
\pgfpathlineto{\pgfqpoint{4.800225in}{2.952836in}}%
\pgfpathlineto{\pgfqpoint{4.807782in}{2.964711in}}%
\pgfpathlineto{\pgfqpoint{4.794131in}{2.964779in}}%
\pgfpathlineto{\pgfqpoint{4.780489in}{2.965029in}}%
\pgfpathlineto{\pgfqpoint{4.766858in}{2.965463in}}%
\pgfpathlineto{\pgfqpoint{4.753237in}{2.966080in}}%
\pgfpathlineto{\pgfqpoint{4.745666in}{2.953656in}}%
\pgfpathlineto{\pgfqpoint{4.738093in}{2.941411in}}%
\pgfpathlineto{\pgfqpoint{4.730518in}{2.929339in}}%
\pgfpathlineto{\pgfqpoint{4.722940in}{2.917433in}}%
\pgfpathclose%
\pgfusepath{fill}%
\end{pgfscope}%
\begin{pgfscope}%
\pgfpathrectangle{\pgfqpoint{1.150000in}{0.150000in}}{\pgfqpoint{5.700000in}{5.700000in}}%
\pgfusepath{clip}%
\pgfsetbuttcap%
\pgfsetroundjoin%
\definecolor{currentfill}{rgb}{0.182256,0.426184,0.557120}%
\pgfsetfillcolor{currentfill}%
\pgfsetfillopacity{0.800000}%
\pgfsetlinewidth{0.000000pt}%
\definecolor{currentstroke}{rgb}{0.000000,0.000000,0.000000}%
\pgfsetstrokecolor{currentstroke}%
\pgfsetdash{}{0pt}%
\pgfpathmoveto{\pgfqpoint{2.866578in}{3.237268in}}%
\pgfpathlineto{\pgfqpoint{2.880176in}{3.215245in}}%
\pgfpathlineto{\pgfqpoint{2.893764in}{3.193570in}}%
\pgfpathlineto{\pgfqpoint{2.907341in}{3.172239in}}%
\pgfpathlineto{\pgfqpoint{2.920908in}{3.151249in}}%
\pgfpathlineto{\pgfqpoint{2.928977in}{3.163922in}}%
\pgfpathlineto{\pgfqpoint{2.937038in}{3.176761in}}%
\pgfpathlineto{\pgfqpoint{2.945092in}{3.189768in}}%
\pgfpathlineto{\pgfqpoint{2.953137in}{3.202946in}}%
\pgfpathlineto{\pgfqpoint{2.939585in}{3.224064in}}%
\pgfpathlineto{\pgfqpoint{2.926022in}{3.245523in}}%
\pgfpathlineto{\pgfqpoint{2.912449in}{3.267327in}}%
\pgfpathlineto{\pgfqpoint{2.898866in}{3.289479in}}%
\pgfpathlineto{\pgfqpoint{2.890806in}{3.276161in}}%
\pgfpathlineto{\pgfqpoint{2.882738in}{3.263021in}}%
\pgfpathlineto{\pgfqpoint{2.874662in}{3.250057in}}%
\pgfpathlineto{\pgfqpoint{2.866578in}{3.237268in}}%
\pgfpathclose%
\pgfusepath{fill}%
\end{pgfscope}%
\begin{pgfscope}%
\pgfpathrectangle{\pgfqpoint{1.150000in}{0.150000in}}{\pgfqpoint{5.700000in}{5.700000in}}%
\pgfusepath{clip}%
\pgfsetbuttcap%
\pgfsetroundjoin%
\definecolor{currentfill}{rgb}{0.252194,0.269783,0.531579}%
\pgfsetfillcolor{currentfill}%
\pgfsetfillopacity{0.800000}%
\pgfsetlinewidth{0.000000pt}%
\definecolor{currentstroke}{rgb}{0.000000,0.000000,0.000000}%
\pgfsetstrokecolor{currentstroke}%
\pgfsetdash{}{0pt}%
\pgfpathmoveto{\pgfqpoint{4.468510in}{2.783879in}}%
\pgfpathlineto{\pgfqpoint{4.482065in}{2.782969in}}%
\pgfpathlineto{\pgfqpoint{4.495629in}{2.782251in}}%
\pgfpathlineto{\pgfqpoint{4.509203in}{2.781722in}}%
\pgfpathlineto{\pgfqpoint{4.522785in}{2.781384in}}%
\pgfpathlineto{\pgfqpoint{4.530422in}{2.792603in}}%
\pgfpathlineto{\pgfqpoint{4.538055in}{2.803936in}}%
\pgfpathlineto{\pgfqpoint{4.545684in}{2.815390in}}%
\pgfpathlineto{\pgfqpoint{4.553310in}{2.826969in}}%
\pgfpathlineto{\pgfqpoint{4.539739in}{2.827750in}}%
\pgfpathlineto{\pgfqpoint{4.526177in}{2.828721in}}%
\pgfpathlineto{\pgfqpoint{4.512623in}{2.829881in}}%
\pgfpathlineto{\pgfqpoint{4.499079in}{2.831232in}}%
\pgfpathlineto{\pgfqpoint{4.491441in}{2.819200in}}%
\pgfpathlineto{\pgfqpoint{4.483801in}{2.807300in}}%
\pgfpathlineto{\pgfqpoint{4.476157in}{2.795528in}}%
\pgfpathlineto{\pgfqpoint{4.468510in}{2.783879in}}%
\pgfpathclose%
\pgfusepath{fill}%
\end{pgfscope}%
\begin{pgfscope}%
\pgfpathrectangle{\pgfqpoint{1.150000in}{0.150000in}}{\pgfqpoint{5.700000in}{5.700000in}}%
\pgfusepath{clip}%
\pgfsetbuttcap%
\pgfsetroundjoin%
\definecolor{currentfill}{rgb}{0.220057,0.343307,0.549413}%
\pgfsetfillcolor{currentfill}%
\pgfsetfillopacity{0.800000}%
\pgfsetlinewidth{0.000000pt}%
\definecolor{currentstroke}{rgb}{0.000000,0.000000,0.000000}%
\pgfsetstrokecolor{currentstroke}%
\pgfsetdash{}{0pt}%
\pgfpathmoveto{\pgfqpoint{4.807782in}{2.964711in}}%
\pgfpathlineto{\pgfqpoint{4.821444in}{2.964826in}}%
\pgfpathlineto{\pgfqpoint{4.835117in}{2.965123in}}%
\pgfpathlineto{\pgfqpoint{4.848800in}{2.965602in}}%
\pgfpathlineto{\pgfqpoint{4.862494in}{2.966262in}}%
\pgfpathlineto{\pgfqpoint{4.870036in}{2.977758in}}%
\pgfpathlineto{\pgfqpoint{4.877575in}{2.989431in}}%
\pgfpathlineto{\pgfqpoint{4.885113in}{3.001286in}}%
\pgfpathlineto{\pgfqpoint{4.892650in}{3.013332in}}%
\pgfpathlineto{\pgfqpoint{4.878971in}{3.013241in}}%
\pgfpathlineto{\pgfqpoint{4.865303in}{3.013331in}}%
\pgfpathlineto{\pgfqpoint{4.851645in}{3.013602in}}%
\pgfpathlineto{\pgfqpoint{4.837997in}{3.014055in}}%
\pgfpathlineto{\pgfqpoint{4.830446in}{3.001430in}}%
\pgfpathlineto{\pgfqpoint{4.822893in}{2.989002in}}%
\pgfpathlineto{\pgfqpoint{4.815338in}{2.976765in}}%
\pgfpathlineto{\pgfqpoint{4.807782in}{2.964711in}}%
\pgfpathclose%
\pgfusepath{fill}%
\end{pgfscope}%
\begin{pgfscope}%
\pgfpathrectangle{\pgfqpoint{1.150000in}{0.150000in}}{\pgfqpoint{5.700000in}{5.700000in}}%
\pgfusepath{clip}%
\pgfsetbuttcap%
\pgfsetroundjoin%
\definecolor{currentfill}{rgb}{0.277134,0.185228,0.489898}%
\pgfsetfillcolor{currentfill}%
\pgfsetfillopacity{0.800000}%
\pgfsetlinewidth{0.000000pt}%
\definecolor{currentstroke}{rgb}{0.000000,0.000000,0.000000}%
\pgfsetstrokecolor{currentstroke}%
\pgfsetdash{}{0pt}%
\pgfpathmoveto{\pgfqpoint{3.404933in}{2.598531in}}%
\pgfpathlineto{\pgfqpoint{3.418316in}{2.588250in}}%
\pgfpathlineto{\pgfqpoint{3.431700in}{2.578219in}}%
\pgfpathlineto{\pgfqpoint{3.445083in}{2.568434in}}%
\pgfpathlineto{\pgfqpoint{3.458466in}{2.558895in}}%
\pgfpathlineto{\pgfqpoint{3.466411in}{2.570413in}}%
\pgfpathlineto{\pgfqpoint{3.474352in}{2.582018in}}%
\pgfpathlineto{\pgfqpoint{3.482286in}{2.593712in}}%
\pgfpathlineto{\pgfqpoint{3.490215in}{2.605496in}}%
\pgfpathlineto{\pgfqpoint{3.476842in}{2.615163in}}%
\pgfpathlineto{\pgfqpoint{3.463469in}{2.625074in}}%
\pgfpathlineto{\pgfqpoint{3.450097in}{2.635233in}}%
\pgfpathlineto{\pgfqpoint{3.436723in}{2.645641in}}%
\pgfpathlineto{\pgfqpoint{3.428785in}{2.633718in}}%
\pgfpathlineto{\pgfqpoint{3.420840in}{2.621893in}}%
\pgfpathlineto{\pgfqpoint{3.412889in}{2.610165in}}%
\pgfpathlineto{\pgfqpoint{3.404933in}{2.598531in}}%
\pgfpathclose%
\pgfusepath{fill}%
\end{pgfscope}%
\begin{pgfscope}%
\pgfpathrectangle{\pgfqpoint{1.150000in}{0.150000in}}{\pgfqpoint{5.700000in}{5.700000in}}%
\pgfusepath{clip}%
\pgfsetbuttcap%
\pgfsetroundjoin%
\definecolor{currentfill}{rgb}{0.252194,0.269783,0.531579}%
\pgfsetfillcolor{currentfill}%
\pgfsetfillopacity{0.800000}%
\pgfsetlinewidth{0.000000pt}%
\definecolor{currentstroke}{rgb}{0.000000,0.000000,0.000000}%
\pgfsetstrokecolor{currentstroke}%
\pgfsetdash{}{0pt}%
\pgfpathmoveto{\pgfqpoint{3.104708in}{2.811711in}}%
\pgfpathlineto{\pgfqpoint{3.118160in}{2.796149in}}%
\pgfpathlineto{\pgfqpoint{3.131607in}{2.780876in}}%
\pgfpathlineto{\pgfqpoint{3.145049in}{2.765891in}}%
\pgfpathlineto{\pgfqpoint{3.158487in}{2.751189in}}%
\pgfpathlineto{\pgfqpoint{3.166516in}{2.762796in}}%
\pgfpathlineto{\pgfqpoint{3.174538in}{2.774522in}}%
\pgfpathlineto{\pgfqpoint{3.182553in}{2.786370in}}%
\pgfpathlineto{\pgfqpoint{3.190562in}{2.798340in}}%
\pgfpathlineto{\pgfqpoint{3.177138in}{2.813137in}}%
\pgfpathlineto{\pgfqpoint{3.163709in}{2.828218in}}%
\pgfpathlineto{\pgfqpoint{3.150275in}{2.843586in}}%
\pgfpathlineto{\pgfqpoint{3.136836in}{2.859243in}}%
\pgfpathlineto{\pgfqpoint{3.128815in}{2.847165in}}%
\pgfpathlineto{\pgfqpoint{3.120786in}{2.835218in}}%
\pgfpathlineto{\pgfqpoint{3.112751in}{2.823401in}}%
\pgfpathlineto{\pgfqpoint{3.104708in}{2.811711in}}%
\pgfpathclose%
\pgfusepath{fill}%
\end{pgfscope}%
\begin{pgfscope}%
\pgfpathrectangle{\pgfqpoint{1.150000in}{0.150000in}}{\pgfqpoint{5.700000in}{5.700000in}}%
\pgfusepath{clip}%
\pgfsetbuttcap%
\pgfsetroundjoin%
\definecolor{currentfill}{rgb}{0.258965,0.251537,0.524736}%
\pgfsetfillcolor{currentfill}%
\pgfsetfillopacity{0.800000}%
\pgfsetlinewidth{0.000000pt}%
\definecolor{currentstroke}{rgb}{0.000000,0.000000,0.000000}%
\pgfsetstrokecolor{currentstroke}%
\pgfsetdash{}{0pt}%
\pgfpathmoveto{\pgfqpoint{4.383711in}{2.742319in}}%
\pgfpathlineto{\pgfqpoint{4.397242in}{2.741052in}}%
\pgfpathlineto{\pgfqpoint{4.410781in}{2.739978in}}%
\pgfpathlineto{\pgfqpoint{4.424329in}{2.739096in}}%
\pgfpathlineto{\pgfqpoint{4.437886in}{2.738406in}}%
\pgfpathlineto{\pgfqpoint{4.445547in}{2.749615in}}%
\pgfpathlineto{\pgfqpoint{4.453205in}{2.760927in}}%
\pgfpathlineto{\pgfqpoint{4.460860in}{2.772347in}}%
\pgfpathlineto{\pgfqpoint{4.468510in}{2.783879in}}%
\pgfpathlineto{\pgfqpoint{4.454964in}{2.784979in}}%
\pgfpathlineto{\pgfqpoint{4.441426in}{2.786271in}}%
\pgfpathlineto{\pgfqpoint{4.427897in}{2.787755in}}%
\pgfpathlineto{\pgfqpoint{4.414376in}{2.789433in}}%
\pgfpathlineto{\pgfqpoint{4.406715in}{2.777479in}}%
\pgfpathlineto{\pgfqpoint{4.399051in}{2.765645in}}%
\pgfpathlineto{\pgfqpoint{4.391383in}{2.753927in}}%
\pgfpathlineto{\pgfqpoint{4.383711in}{2.742319in}}%
\pgfpathclose%
\pgfusepath{fill}%
\end{pgfscope}%
\begin{pgfscope}%
\pgfpathrectangle{\pgfqpoint{1.150000in}{0.150000in}}{\pgfqpoint{5.700000in}{5.700000in}}%
\pgfusepath{clip}%
\pgfsetbuttcap%
\pgfsetroundjoin%
\definecolor{currentfill}{rgb}{0.241237,0.296485,0.539709}%
\pgfsetfillcolor{currentfill}%
\pgfsetfillopacity{0.800000}%
\pgfsetlinewidth{0.000000pt}%
\definecolor{currentstroke}{rgb}{0.000000,0.000000,0.000000}%
\pgfsetstrokecolor{currentstroke}%
\pgfsetdash{}{0pt}%
\pgfpathmoveto{\pgfqpoint{3.050843in}{2.876901in}}%
\pgfpathlineto{\pgfqpoint{3.064319in}{2.860157in}}%
\pgfpathlineto{\pgfqpoint{3.077788in}{2.843712in}}%
\pgfpathlineto{\pgfqpoint{3.091251in}{2.827564in}}%
\pgfpathlineto{\pgfqpoint{3.104708in}{2.811711in}}%
\pgfpathlineto{\pgfqpoint{3.112751in}{2.823401in}}%
\pgfpathlineto{\pgfqpoint{3.120786in}{2.835218in}}%
\pgfpathlineto{\pgfqpoint{3.128815in}{2.847165in}}%
\pgfpathlineto{\pgfqpoint{3.136836in}{2.859243in}}%
\pgfpathlineto{\pgfqpoint{3.123392in}{2.875192in}}%
\pgfpathlineto{\pgfqpoint{3.109942in}{2.891435in}}%
\pgfpathlineto{\pgfqpoint{3.096487in}{2.907975in}}%
\pgfpathlineto{\pgfqpoint{3.083025in}{2.924815in}}%
\pgfpathlineto{\pgfqpoint{3.074991in}{2.912630in}}%
\pgfpathlineto{\pgfqpoint{3.066949in}{2.900583in}}%
\pgfpathlineto{\pgfqpoint{3.058900in}{2.888674in}}%
\pgfpathlineto{\pgfqpoint{3.050843in}{2.876901in}}%
\pgfpathclose%
\pgfusepath{fill}%
\end{pgfscope}%
\begin{pgfscope}%
\pgfpathrectangle{\pgfqpoint{1.150000in}{0.150000in}}{\pgfqpoint{5.700000in}{5.700000in}}%
\pgfusepath{clip}%
\pgfsetbuttcap%
\pgfsetroundjoin%
\definecolor{currentfill}{rgb}{0.210503,0.363727,0.552206}%
\pgfsetfillcolor{currentfill}%
\pgfsetfillopacity{0.800000}%
\pgfsetlinewidth{0.000000pt}%
\definecolor{currentstroke}{rgb}{0.000000,0.000000,0.000000}%
\pgfsetstrokecolor{currentstroke}%
\pgfsetdash{}{0pt}%
\pgfpathmoveto{\pgfqpoint{4.892650in}{3.013332in}}%
\pgfpathlineto{\pgfqpoint{4.906340in}{3.013603in}}%
\pgfpathlineto{\pgfqpoint{4.920041in}{3.014055in}}%
\pgfpathlineto{\pgfqpoint{4.933753in}{3.014686in}}%
\pgfpathlineto{\pgfqpoint{4.947477in}{3.015497in}}%
\pgfpathlineto{\pgfqpoint{4.954997in}{3.027150in}}%
\pgfpathlineto{\pgfqpoint{4.962516in}{3.038998in}}%
\pgfpathlineto{\pgfqpoint{4.970034in}{3.051048in}}%
\pgfpathlineto{\pgfqpoint{4.977552in}{3.063307in}}%
\pgfpathlineto{\pgfqpoint{4.963845in}{3.063096in}}%
\pgfpathlineto{\pgfqpoint{4.950149in}{3.063065in}}%
\pgfpathlineto{\pgfqpoint{4.936464in}{3.063214in}}%
\pgfpathlineto{\pgfqpoint{4.922789in}{3.063542in}}%
\pgfpathlineto{\pgfqpoint{4.915256in}{3.050671in}}%
\pgfpathlineto{\pgfqpoint{4.907721in}{3.038017in}}%
\pgfpathlineto{\pgfqpoint{4.900186in}{3.025573in}}%
\pgfpathlineto{\pgfqpoint{4.892650in}{3.013332in}}%
\pgfpathclose%
\pgfusepath{fill}%
\end{pgfscope}%
\begin{pgfscope}%
\pgfpathrectangle{\pgfqpoint{1.150000in}{0.150000in}}{\pgfqpoint{5.700000in}{5.700000in}}%
\pgfusepath{clip}%
\pgfsetbuttcap%
\pgfsetroundjoin%
\definecolor{currentfill}{rgb}{0.260571,0.246922,0.522828}%
\pgfsetfillcolor{currentfill}%
\pgfsetfillopacity{0.800000}%
\pgfsetlinewidth{0.000000pt}%
\definecolor{currentstroke}{rgb}{0.000000,0.000000,0.000000}%
\pgfsetstrokecolor{currentstroke}%
\pgfsetdash{}{0pt}%
\pgfpathmoveto{\pgfqpoint{3.158487in}{2.751189in}}%
\pgfpathlineto{\pgfqpoint{3.171920in}{2.736771in}}%
\pgfpathlineto{\pgfqpoint{3.185348in}{2.722632in}}%
\pgfpathlineto{\pgfqpoint{3.198773in}{2.708771in}}%
\pgfpathlineto{\pgfqpoint{3.212194in}{2.695185in}}%
\pgfpathlineto{\pgfqpoint{3.220210in}{2.706708in}}%
\pgfpathlineto{\pgfqpoint{3.228220in}{2.718343in}}%
\pgfpathlineto{\pgfqpoint{3.236223in}{2.730091in}}%
\pgfpathlineto{\pgfqpoint{3.244220in}{2.741954in}}%
\pgfpathlineto{\pgfqpoint{3.230811in}{2.755635in}}%
\pgfpathlineto{\pgfqpoint{3.217399in}{2.769592in}}%
\pgfpathlineto{\pgfqpoint{3.203982in}{2.783826in}}%
\pgfpathlineto{\pgfqpoint{3.190562in}{2.798340in}}%
\pgfpathlineto{\pgfqpoint{3.182553in}{2.786370in}}%
\pgfpathlineto{\pgfqpoint{3.174538in}{2.774522in}}%
\pgfpathlineto{\pgfqpoint{3.166516in}{2.762796in}}%
\pgfpathlineto{\pgfqpoint{3.158487in}{2.751189in}}%
\pgfpathclose%
\pgfusepath{fill}%
\end{pgfscope}%
\begin{pgfscope}%
\pgfpathrectangle{\pgfqpoint{1.150000in}{0.150000in}}{\pgfqpoint{5.700000in}{5.700000in}}%
\pgfusepath{clip}%
\pgfsetbuttcap%
\pgfsetroundjoin%
\definecolor{currentfill}{rgb}{0.263663,0.237631,0.518762}%
\pgfsetfillcolor{currentfill}%
\pgfsetfillopacity{0.800000}%
\pgfsetlinewidth{0.000000pt}%
\definecolor{currentstroke}{rgb}{0.000000,0.000000,0.000000}%
\pgfsetstrokecolor{currentstroke}%
\pgfsetdash{}{0pt}%
\pgfpathmoveto{\pgfqpoint{4.298907in}{2.702398in}}%
\pgfpathlineto{\pgfqpoint{4.312414in}{2.700731in}}%
\pgfpathlineto{\pgfqpoint{4.325930in}{2.699259in}}%
\pgfpathlineto{\pgfqpoint{4.339454in}{2.697982in}}%
\pgfpathlineto{\pgfqpoint{4.352986in}{2.696901in}}%
\pgfpathlineto{\pgfqpoint{4.360673in}{2.708112in}}%
\pgfpathlineto{\pgfqpoint{4.368356in}{2.719416in}}%
\pgfpathlineto{\pgfqpoint{4.376036in}{2.730817in}}%
\pgfpathlineto{\pgfqpoint{4.383711in}{2.742319in}}%
\pgfpathlineto{\pgfqpoint{4.370189in}{2.743780in}}%
\pgfpathlineto{\pgfqpoint{4.356675in}{2.745436in}}%
\pgfpathlineto{\pgfqpoint{4.343169in}{2.747286in}}%
\pgfpathlineto{\pgfqpoint{4.329670in}{2.749332in}}%
\pgfpathlineto{\pgfqpoint{4.321985in}{2.737440in}}%
\pgfpathlineto{\pgfqpoint{4.314296in}{2.725656in}}%
\pgfpathlineto{\pgfqpoint{4.306604in}{2.713977in}}%
\pgfpathlineto{\pgfqpoint{4.298907in}{2.702398in}}%
\pgfpathclose%
\pgfusepath{fill}%
\end{pgfscope}%
\begin{pgfscope}%
\pgfpathrectangle{\pgfqpoint{1.150000in}{0.150000in}}{\pgfqpoint{5.700000in}{5.700000in}}%
\pgfusepath{clip}%
\pgfsetbuttcap%
\pgfsetroundjoin%
\definecolor{currentfill}{rgb}{0.229739,0.322361,0.545706}%
\pgfsetfillcolor{currentfill}%
\pgfsetfillopacity{0.800000}%
\pgfsetlinewidth{0.000000pt}%
\definecolor{currentstroke}{rgb}{0.000000,0.000000,0.000000}%
\pgfsetstrokecolor{currentstroke}%
\pgfsetdash{}{0pt}%
\pgfpathmoveto{\pgfqpoint{2.996874in}{2.946924in}}%
\pgfpathlineto{\pgfqpoint{3.010377in}{2.928956in}}%
\pgfpathlineto{\pgfqpoint{3.023873in}{2.911297in}}%
\pgfpathlineto{\pgfqpoint{3.037361in}{2.893947in}}%
\pgfpathlineto{\pgfqpoint{3.050843in}{2.876901in}}%
\pgfpathlineto{\pgfqpoint{3.058900in}{2.888674in}}%
\pgfpathlineto{\pgfqpoint{3.066949in}{2.900583in}}%
\pgfpathlineto{\pgfqpoint{3.074991in}{2.912630in}}%
\pgfpathlineto{\pgfqpoint{3.083025in}{2.924815in}}%
\pgfpathlineto{\pgfqpoint{3.069557in}{2.941956in}}%
\pgfpathlineto{\pgfqpoint{3.056083in}{2.959403in}}%
\pgfpathlineto{\pgfqpoint{3.042601in}{2.977156in}}%
\pgfpathlineto{\pgfqpoint{3.029112in}{2.995220in}}%
\pgfpathlineto{\pgfqpoint{3.021064in}{2.982927in}}%
\pgfpathlineto{\pgfqpoint{3.013008in}{2.970782in}}%
\pgfpathlineto{\pgfqpoint{3.004945in}{2.958781in}}%
\pgfpathlineto{\pgfqpoint{2.996874in}{2.946924in}}%
\pgfpathclose%
\pgfusepath{fill}%
\end{pgfscope}%
\begin{pgfscope}%
\pgfpathrectangle{\pgfqpoint{1.150000in}{0.150000in}}{\pgfqpoint{5.700000in}{5.700000in}}%
\pgfusepath{clip}%
\pgfsetbuttcap%
\pgfsetroundjoin%
\definecolor{currentfill}{rgb}{0.203063,0.379716,0.553925}%
\pgfsetfillcolor{currentfill}%
\pgfsetfillopacity{0.800000}%
\pgfsetlinewidth{0.000000pt}%
\definecolor{currentstroke}{rgb}{0.000000,0.000000,0.000000}%
\pgfsetstrokecolor{currentstroke}%
\pgfsetdash{}{0pt}%
\pgfpathmoveto{\pgfqpoint{4.977552in}{3.063307in}}%
\pgfpathlineto{\pgfqpoint{4.991270in}{3.063696in}}%
\pgfpathlineto{\pgfqpoint{5.004999in}{3.064263in}}%
\pgfpathlineto{\pgfqpoint{5.018740in}{3.065009in}}%
\pgfpathlineto{\pgfqpoint{5.032492in}{3.065932in}}%
\pgfpathlineto{\pgfqpoint{5.039992in}{3.077787in}}%
\pgfpathlineto{\pgfqpoint{5.047492in}{3.089857in}}%
\pgfpathlineto{\pgfqpoint{5.054992in}{3.102151in}}%
\pgfpathlineto{\pgfqpoint{5.062493in}{3.114674in}}%
\pgfpathlineto{\pgfqpoint{5.048758in}{3.114383in}}%
\pgfpathlineto{\pgfqpoint{5.035035in}{3.114269in}}%
\pgfpathlineto{\pgfqpoint{5.021322in}{3.114333in}}%
\pgfpathlineto{\pgfqpoint{5.007621in}{3.114575in}}%
\pgfpathlineto{\pgfqpoint{5.000104in}{3.101409in}}%
\pgfpathlineto{\pgfqpoint{4.992586in}{3.088480in}}%
\pgfpathlineto{\pgfqpoint{4.985069in}{3.075782in}}%
\pgfpathlineto{\pgfqpoint{4.977552in}{3.063307in}}%
\pgfpathclose%
\pgfusepath{fill}%
\end{pgfscope}%
\begin{pgfscope}%
\pgfpathrectangle{\pgfqpoint{1.150000in}{0.150000in}}{\pgfqpoint{5.700000in}{5.700000in}}%
\pgfusepath{clip}%
\pgfsetbuttcap%
\pgfsetroundjoin%
\definecolor{currentfill}{rgb}{0.269308,0.218818,0.509577}%
\pgfsetfillcolor{currentfill}%
\pgfsetfillopacity{0.800000}%
\pgfsetlinewidth{0.000000pt}%
\definecolor{currentstroke}{rgb}{0.000000,0.000000,0.000000}%
\pgfsetstrokecolor{currentstroke}%
\pgfsetdash{}{0pt}%
\pgfpathmoveto{\pgfqpoint{4.214089in}{2.664247in}}%
\pgfpathlineto{\pgfqpoint{4.227575in}{2.662137in}}%
\pgfpathlineto{\pgfqpoint{4.241069in}{2.660225in}}%
\pgfpathlineto{\pgfqpoint{4.254570in}{2.658512in}}%
\pgfpathlineto{\pgfqpoint{4.268079in}{2.656996in}}%
\pgfpathlineto{\pgfqpoint{4.275792in}{2.668217in}}%
\pgfpathlineto{\pgfqpoint{4.283501in}{2.679522in}}%
\pgfpathlineto{\pgfqpoint{4.291206in}{2.690914in}}%
\pgfpathlineto{\pgfqpoint{4.298907in}{2.702398in}}%
\pgfpathlineto{\pgfqpoint{4.285407in}{2.704262in}}%
\pgfpathlineto{\pgfqpoint{4.271915in}{2.706323in}}%
\pgfpathlineto{\pgfqpoint{4.258430in}{2.708582in}}%
\pgfpathlineto{\pgfqpoint{4.244953in}{2.711039in}}%
\pgfpathlineto{\pgfqpoint{4.237243in}{2.699197in}}%
\pgfpathlineto{\pgfqpoint{4.229529in}{2.687454in}}%
\pgfpathlineto{\pgfqpoint{4.221811in}{2.675805in}}%
\pgfpathlineto{\pgfqpoint{4.214089in}{2.664247in}}%
\pgfpathclose%
\pgfusepath{fill}%
\end{pgfscope}%
\begin{pgfscope}%
\pgfpathrectangle{\pgfqpoint{1.150000in}{0.150000in}}{\pgfqpoint{5.700000in}{5.700000in}}%
\pgfusepath{clip}%
\pgfsetbuttcap%
\pgfsetroundjoin%
\definecolor{currentfill}{rgb}{0.267968,0.223549,0.512008}%
\pgfsetfillcolor{currentfill}%
\pgfsetfillopacity{0.800000}%
\pgfsetlinewidth{0.000000pt}%
\definecolor{currentstroke}{rgb}{0.000000,0.000000,0.000000}%
\pgfsetstrokecolor{currentstroke}%
\pgfsetdash{}{0pt}%
\pgfpathmoveto{\pgfqpoint{3.212194in}{2.695185in}}%
\pgfpathlineto{\pgfqpoint{3.225612in}{2.681873in}}%
\pgfpathlineto{\pgfqpoint{3.239027in}{2.668832in}}%
\pgfpathlineto{\pgfqpoint{3.252438in}{2.656061in}}%
\pgfpathlineto{\pgfqpoint{3.265847in}{2.643557in}}%
\pgfpathlineto{\pgfqpoint{3.273851in}{2.654996in}}%
\pgfpathlineto{\pgfqpoint{3.281848in}{2.666539in}}%
\pgfpathlineto{\pgfqpoint{3.289839in}{2.678188in}}%
\pgfpathlineto{\pgfqpoint{3.297824in}{2.689943in}}%
\pgfpathlineto{\pgfqpoint{3.284427in}{2.702543in}}%
\pgfpathlineto{\pgfqpoint{3.271028in}{2.715410in}}%
\pgfpathlineto{\pgfqpoint{3.257625in}{2.728546in}}%
\pgfpathlineto{\pgfqpoint{3.244220in}{2.741954in}}%
\pgfpathlineto{\pgfqpoint{3.236223in}{2.730091in}}%
\pgfpathlineto{\pgfqpoint{3.228220in}{2.718343in}}%
\pgfpathlineto{\pgfqpoint{3.220210in}{2.706708in}}%
\pgfpathlineto{\pgfqpoint{3.212194in}{2.695185in}}%
\pgfpathclose%
\pgfusepath{fill}%
\end{pgfscope}%
\begin{pgfscope}%
\pgfpathrectangle{\pgfqpoint{1.150000in}{0.150000in}}{\pgfqpoint{5.700000in}{5.700000in}}%
\pgfusepath{clip}%
\pgfsetbuttcap%
\pgfsetroundjoin%
\definecolor{currentfill}{rgb}{0.279574,0.170599,0.479997}%
\pgfsetfillcolor{currentfill}%
\pgfsetfillopacity{0.800000}%
\pgfsetlinewidth{0.000000pt}%
\definecolor{currentstroke}{rgb}{0.000000,0.000000,0.000000}%
\pgfsetstrokecolor{currentstroke}%
\pgfsetdash{}{0pt}%
\pgfpathmoveto{\pgfqpoint{3.820867in}{2.552294in}}%
\pgfpathlineto{\pgfqpoint{3.834269in}{2.547108in}}%
\pgfpathlineto{\pgfqpoint{3.847676in}{2.542139in}}%
\pgfpathlineto{\pgfqpoint{3.861087in}{2.537385in}}%
\pgfpathlineto{\pgfqpoint{3.874503in}{2.532846in}}%
\pgfpathlineto{\pgfqpoint{3.882331in}{2.544287in}}%
\pgfpathlineto{\pgfqpoint{3.890155in}{2.555795in}}%
\pgfpathlineto{\pgfqpoint{3.897974in}{2.567373in}}%
\pgfpathlineto{\pgfqpoint{3.905788in}{2.579023in}}%
\pgfpathlineto{\pgfqpoint{3.892381in}{2.583784in}}%
\pgfpathlineto{\pgfqpoint{3.878978in}{2.588759in}}%
\pgfpathlineto{\pgfqpoint{3.865580in}{2.593950in}}%
\pgfpathlineto{\pgfqpoint{3.852185in}{2.599357in}}%
\pgfpathlineto{\pgfqpoint{3.844363in}{2.587473in}}%
\pgfpathlineto{\pgfqpoint{3.836536in}{2.575670in}}%
\pgfpathlineto{\pgfqpoint{3.828704in}{2.563944in}}%
\pgfpathlineto{\pgfqpoint{3.820867in}{2.552294in}}%
\pgfpathclose%
\pgfusepath{fill}%
\end{pgfscope}%
\begin{pgfscope}%
\pgfpathrectangle{\pgfqpoint{1.150000in}{0.150000in}}{\pgfqpoint{5.700000in}{5.700000in}}%
\pgfusepath{clip}%
\pgfsetbuttcap%
\pgfsetroundjoin%
\definecolor{currentfill}{rgb}{0.194100,0.399323,0.555565}%
\pgfsetfillcolor{currentfill}%
\pgfsetfillopacity{0.800000}%
\pgfsetlinewidth{0.000000pt}%
\definecolor{currentstroke}{rgb}{0.000000,0.000000,0.000000}%
\pgfsetstrokecolor{currentstroke}%
\pgfsetdash{}{0pt}%
\pgfpathmoveto{\pgfqpoint{5.062493in}{3.114674in}}%
\pgfpathlineto{\pgfqpoint{5.076239in}{3.115142in}}%
\pgfpathlineto{\pgfqpoint{5.089996in}{3.115787in}}%
\pgfpathlineto{\pgfqpoint{5.103766in}{3.116608in}}%
\pgfpathlineto{\pgfqpoint{5.117547in}{3.117606in}}%
\pgfpathlineto{\pgfqpoint{5.125029in}{3.129714in}}%
\pgfpathlineto{\pgfqpoint{5.132512in}{3.142060in}}%
\pgfpathlineto{\pgfqpoint{5.139996in}{3.154651in}}%
\pgfpathlineto{\pgfqpoint{5.147481in}{3.167495in}}%
\pgfpathlineto{\pgfqpoint{5.133719in}{3.167161in}}%
\pgfpathlineto{\pgfqpoint{5.119968in}{3.167003in}}%
\pgfpathlineto{\pgfqpoint{5.106229in}{3.167022in}}%
\pgfpathlineto{\pgfqpoint{5.092501in}{3.167216in}}%
\pgfpathlineto{\pgfqpoint{5.084997in}{3.153698in}}%
\pgfpathlineto{\pgfqpoint{5.077495in}{3.140440in}}%
\pgfpathlineto{\pgfqpoint{5.069993in}{3.127435in}}%
\pgfpathlineto{\pgfqpoint{5.062493in}{3.114674in}}%
\pgfpathclose%
\pgfusepath{fill}%
\end{pgfscope}%
\begin{pgfscope}%
\pgfpathrectangle{\pgfqpoint{1.150000in}{0.150000in}}{\pgfqpoint{5.700000in}{5.700000in}}%
\pgfusepath{clip}%
\pgfsetbuttcap%
\pgfsetroundjoin%
\definecolor{currentfill}{rgb}{0.216210,0.351535,0.550627}%
\pgfsetfillcolor{currentfill}%
\pgfsetfillopacity{0.800000}%
\pgfsetlinewidth{0.000000pt}%
\definecolor{currentstroke}{rgb}{0.000000,0.000000,0.000000}%
\pgfsetstrokecolor{currentstroke}%
\pgfsetdash{}{0pt}%
\pgfpathmoveto{\pgfqpoint{2.942782in}{3.021956in}}%
\pgfpathlineto{\pgfqpoint{2.956318in}{3.002719in}}%
\pgfpathlineto{\pgfqpoint{2.969845in}{2.983802in}}%
\pgfpathlineto{\pgfqpoint{2.983363in}{2.965205in}}%
\pgfpathlineto{\pgfqpoint{2.996874in}{2.946924in}}%
\pgfpathlineto{\pgfqpoint{3.004945in}{2.958781in}}%
\pgfpathlineto{\pgfqpoint{3.013008in}{2.970782in}}%
\pgfpathlineto{\pgfqpoint{3.021064in}{2.982927in}}%
\pgfpathlineto{\pgfqpoint{3.029112in}{2.995220in}}%
\pgfpathlineto{\pgfqpoint{3.015616in}{3.013597in}}%
\pgfpathlineto{\pgfqpoint{3.002112in}{3.032290in}}%
\pgfpathlineto{\pgfqpoint{2.988600in}{3.051301in}}%
\pgfpathlineto{\pgfqpoint{2.975079in}{3.070635in}}%
\pgfpathlineto{\pgfqpoint{2.967017in}{3.058234in}}%
\pgfpathlineto{\pgfqpoint{2.958947in}{3.045989in}}%
\pgfpathlineto{\pgfqpoint{2.950868in}{3.033897in}}%
\pgfpathlineto{\pgfqpoint{2.942782in}{3.021956in}}%
\pgfpathclose%
\pgfusepath{fill}%
\end{pgfscope}%
\begin{pgfscope}%
\pgfpathrectangle{\pgfqpoint{1.150000in}{0.150000in}}{\pgfqpoint{5.700000in}{5.700000in}}%
\pgfusepath{clip}%
\pgfsetbuttcap%
\pgfsetroundjoin%
\definecolor{currentfill}{rgb}{0.273006,0.204520,0.501721}%
\pgfsetfillcolor{currentfill}%
\pgfsetfillopacity{0.800000}%
\pgfsetlinewidth{0.000000pt}%
\definecolor{currentstroke}{rgb}{0.000000,0.000000,0.000000}%
\pgfsetstrokecolor{currentstroke}%
\pgfsetdash{}{0pt}%
\pgfpathmoveto{\pgfqpoint{4.129250in}{2.628023in}}%
\pgfpathlineto{\pgfqpoint{4.142716in}{2.625427in}}%
\pgfpathlineto{\pgfqpoint{4.156190in}{2.623032in}}%
\pgfpathlineto{\pgfqpoint{4.169670in}{2.620838in}}%
\pgfpathlineto{\pgfqpoint{4.183157in}{2.618844in}}%
\pgfpathlineto{\pgfqpoint{4.190897in}{2.630079in}}%
\pgfpathlineto{\pgfqpoint{4.198632in}{2.641388in}}%
\pgfpathlineto{\pgfqpoint{4.206363in}{2.652776in}}%
\pgfpathlineto{\pgfqpoint{4.214089in}{2.664247in}}%
\pgfpathlineto{\pgfqpoint{4.200610in}{2.666557in}}%
\pgfpathlineto{\pgfqpoint{4.187139in}{2.669067in}}%
\pgfpathlineto{\pgfqpoint{4.173674in}{2.671778in}}%
\pgfpathlineto{\pgfqpoint{4.160216in}{2.674690in}}%
\pgfpathlineto{\pgfqpoint{4.152481in}{2.662892in}}%
\pgfpathlineto{\pgfqpoint{4.144742in}{2.651184in}}%
\pgfpathlineto{\pgfqpoint{4.136998in}{2.639562in}}%
\pgfpathlineto{\pgfqpoint{4.129250in}{2.628023in}}%
\pgfpathclose%
\pgfusepath{fill}%
\end{pgfscope}%
\begin{pgfscope}%
\pgfpathrectangle{\pgfqpoint{1.150000in}{0.150000in}}{\pgfqpoint{5.700000in}{5.700000in}}%
\pgfusepath{clip}%
\pgfsetbuttcap%
\pgfsetroundjoin%
\definecolor{currentfill}{rgb}{0.280868,0.160771,0.472899}%
\pgfsetfillcolor{currentfill}%
\pgfsetfillopacity{0.800000}%
\pgfsetlinewidth{0.000000pt}%
\definecolor{currentstroke}{rgb}{0.000000,0.000000,0.000000}%
\pgfsetstrokecolor{currentstroke}%
\pgfsetdash{}{0pt}%
\pgfpathmoveto{\pgfqpoint{3.597224in}{2.536819in}}%
\pgfpathlineto{\pgfqpoint{3.610607in}{2.529294in}}%
\pgfpathlineto{\pgfqpoint{3.623992in}{2.522000in}}%
\pgfpathlineto{\pgfqpoint{3.637379in}{2.514936in}}%
\pgfpathlineto{\pgfqpoint{3.650768in}{2.508100in}}%
\pgfpathlineto{\pgfqpoint{3.658663in}{2.519536in}}%
\pgfpathlineto{\pgfqpoint{3.666552in}{2.531043in}}%
\pgfpathlineto{\pgfqpoint{3.674437in}{2.542623in}}%
\pgfpathlineto{\pgfqpoint{3.682316in}{2.554279in}}%
\pgfpathlineto{\pgfqpoint{3.668936in}{2.561274in}}%
\pgfpathlineto{\pgfqpoint{3.655558in}{2.568498in}}%
\pgfpathlineto{\pgfqpoint{3.642182in}{2.575950in}}%
\pgfpathlineto{\pgfqpoint{3.628809in}{2.583634in}}%
\pgfpathlineto{\pgfqpoint{3.620921in}{2.571808in}}%
\pgfpathlineto{\pgfqpoint{3.613027in}{2.560064in}}%
\pgfpathlineto{\pgfqpoint{3.605129in}{2.548402in}}%
\pgfpathlineto{\pgfqpoint{3.597224in}{2.536819in}}%
\pgfpathclose%
\pgfusepath{fill}%
\end{pgfscope}%
\begin{pgfscope}%
\pgfpathrectangle{\pgfqpoint{1.150000in}{0.150000in}}{\pgfqpoint{5.700000in}{5.700000in}}%
\pgfusepath{clip}%
\pgfsetbuttcap%
\pgfsetroundjoin%
\definecolor{currentfill}{rgb}{0.183898,0.422383,0.556944}%
\pgfsetfillcolor{currentfill}%
\pgfsetfillopacity{0.800000}%
\pgfsetlinewidth{0.000000pt}%
\definecolor{currentstroke}{rgb}{0.000000,0.000000,0.000000}%
\pgfsetstrokecolor{currentstroke}%
\pgfsetdash{}{0pt}%
\pgfpathmoveto{\pgfqpoint{5.147481in}{3.167495in}}%
\pgfpathlineto{\pgfqpoint{5.161255in}{3.168004in}}%
\pgfpathlineto{\pgfqpoint{5.175041in}{3.168688in}}%
\pgfpathlineto{\pgfqpoint{5.188839in}{3.169547in}}%
\pgfpathlineto{\pgfqpoint{5.202649in}{3.170581in}}%
\pgfpathlineto{\pgfqpoint{5.210115in}{3.183001in}}%
\pgfpathlineto{\pgfqpoint{5.217584in}{3.195681in}}%
\pgfpathlineto{\pgfqpoint{5.225054in}{3.208630in}}%
\pgfpathlineto{\pgfqpoint{5.232526in}{3.221855in}}%
\pgfpathlineto{\pgfqpoint{5.218737in}{3.221517in}}%
\pgfpathlineto{\pgfqpoint{5.204959in}{3.221353in}}%
\pgfpathlineto{\pgfqpoint{5.191193in}{3.221364in}}%
\pgfpathlineto{\pgfqpoint{5.177439in}{3.221549in}}%
\pgfpathlineto{\pgfqpoint{5.169946in}{3.207618in}}%
\pgfpathlineto{\pgfqpoint{5.162456in}{3.193970in}}%
\pgfpathlineto{\pgfqpoint{5.154968in}{3.180598in}}%
\pgfpathlineto{\pgfqpoint{5.147481in}{3.167495in}}%
\pgfpathclose%
\pgfusepath{fill}%
\end{pgfscope}%
\begin{pgfscope}%
\pgfpathrectangle{\pgfqpoint{1.150000in}{0.150000in}}{\pgfqpoint{5.700000in}{5.700000in}}%
\pgfusepath{clip}%
\pgfsetbuttcap%
\pgfsetroundjoin%
\definecolor{currentfill}{rgb}{0.273006,0.204520,0.501721}%
\pgfsetfillcolor{currentfill}%
\pgfsetfillopacity{0.800000}%
\pgfsetlinewidth{0.000000pt}%
\definecolor{currentstroke}{rgb}{0.000000,0.000000,0.000000}%
\pgfsetstrokecolor{currentstroke}%
\pgfsetdash{}{0pt}%
\pgfpathmoveto{\pgfqpoint{3.265847in}{2.643557in}}%
\pgfpathlineto{\pgfqpoint{3.279254in}{2.631318in}}%
\pgfpathlineto{\pgfqpoint{3.292658in}{2.619342in}}%
\pgfpathlineto{\pgfqpoint{3.306060in}{2.607627in}}%
\pgfpathlineto{\pgfqpoint{3.319460in}{2.596172in}}%
\pgfpathlineto{\pgfqpoint{3.327452in}{2.607528in}}%
\pgfpathlineto{\pgfqpoint{3.335438in}{2.618979in}}%
\pgfpathlineto{\pgfqpoint{3.343417in}{2.630528in}}%
\pgfpathlineto{\pgfqpoint{3.351390in}{2.642176in}}%
\pgfpathlineto{\pgfqpoint{3.338002in}{2.653727in}}%
\pgfpathlineto{\pgfqpoint{3.324611in}{2.665537in}}%
\pgfpathlineto{\pgfqpoint{3.311219in}{2.677608in}}%
\pgfpathlineto{\pgfqpoint{3.297824in}{2.689943in}}%
\pgfpathlineto{\pgfqpoint{3.289839in}{2.678188in}}%
\pgfpathlineto{\pgfqpoint{3.281848in}{2.666539in}}%
\pgfpathlineto{\pgfqpoint{3.273851in}{2.654996in}}%
\pgfpathlineto{\pgfqpoint{3.265847in}{2.643557in}}%
\pgfpathclose%
\pgfusepath{fill}%
\end{pgfscope}%
\begin{pgfscope}%
\pgfpathrectangle{\pgfqpoint{1.150000in}{0.150000in}}{\pgfqpoint{5.700000in}{5.700000in}}%
\pgfusepath{clip}%
\pgfsetbuttcap%
\pgfsetroundjoin%
\definecolor{currentfill}{rgb}{0.279574,0.170599,0.479997}%
\pgfsetfillcolor{currentfill}%
\pgfsetfillopacity{0.800000}%
\pgfsetlinewidth{0.000000pt}%
\definecolor{currentstroke}{rgb}{0.000000,0.000000,0.000000}%
\pgfsetstrokecolor{currentstroke}%
\pgfsetdash{}{0pt}%
\pgfpathmoveto{\pgfqpoint{3.458466in}{2.558895in}}%
\pgfpathlineto{\pgfqpoint{3.471849in}{2.549600in}}%
\pgfpathlineto{\pgfqpoint{3.485232in}{2.540547in}}%
\pgfpathlineto{\pgfqpoint{3.498617in}{2.531735in}}%
\pgfpathlineto{\pgfqpoint{3.512002in}{2.523163in}}%
\pgfpathlineto{\pgfqpoint{3.519937in}{2.534565in}}%
\pgfpathlineto{\pgfqpoint{3.527867in}{2.546047in}}%
\pgfpathlineto{\pgfqpoint{3.535792in}{2.557609in}}%
\pgfpathlineto{\pgfqpoint{3.543711in}{2.569255in}}%
\pgfpathlineto{\pgfqpoint{3.530336in}{2.577955in}}%
\pgfpathlineto{\pgfqpoint{3.516962in}{2.586894in}}%
\pgfpathlineto{\pgfqpoint{3.503588in}{2.596074in}}%
\pgfpathlineto{\pgfqpoint{3.490215in}{2.605496in}}%
\pgfpathlineto{\pgfqpoint{3.482286in}{2.593712in}}%
\pgfpathlineto{\pgfqpoint{3.474352in}{2.582018in}}%
\pgfpathlineto{\pgfqpoint{3.466411in}{2.570413in}}%
\pgfpathlineto{\pgfqpoint{3.458466in}{2.558895in}}%
\pgfpathclose%
\pgfusepath{fill}%
\end{pgfscope}%
\begin{pgfscope}%
\pgfpathrectangle{\pgfqpoint{1.150000in}{0.150000in}}{\pgfqpoint{5.700000in}{5.700000in}}%
\pgfusepath{clip}%
\pgfsetbuttcap%
\pgfsetroundjoin%
\definecolor{currentfill}{rgb}{0.276194,0.190074,0.493001}%
\pgfsetfillcolor{currentfill}%
\pgfsetfillopacity{0.800000}%
\pgfsetlinewidth{0.000000pt}%
\definecolor{currentstroke}{rgb}{0.000000,0.000000,0.000000}%
\pgfsetstrokecolor{currentstroke}%
\pgfsetdash{}{0pt}%
\pgfpathmoveto{\pgfqpoint{4.044380in}{2.593907in}}%
\pgfpathlineto{\pgfqpoint{4.057829in}{2.590780in}}%
\pgfpathlineto{\pgfqpoint{4.071284in}{2.587858in}}%
\pgfpathlineto{\pgfqpoint{4.084745in}{2.585140in}}%
\pgfpathlineto{\pgfqpoint{4.098213in}{2.582625in}}%
\pgfpathlineto{\pgfqpoint{4.105979in}{2.593868in}}%
\pgfpathlineto{\pgfqpoint{4.113741in}{2.605180in}}%
\pgfpathlineto{\pgfqpoint{4.121498in}{2.616564in}}%
\pgfpathlineto{\pgfqpoint{4.129250in}{2.628023in}}%
\pgfpathlineto{\pgfqpoint{4.115791in}{2.630823in}}%
\pgfpathlineto{\pgfqpoint{4.102338in}{2.633825in}}%
\pgfpathlineto{\pgfqpoint{4.088891in}{2.637032in}}%
\pgfpathlineto{\pgfqpoint{4.075451in}{2.640444in}}%
\pgfpathlineto{\pgfqpoint{4.067690in}{2.628689in}}%
\pgfpathlineto{\pgfqpoint{4.059924in}{2.617016in}}%
\pgfpathlineto{\pgfqpoint{4.052154in}{2.605424in}}%
\pgfpathlineto{\pgfqpoint{4.044380in}{2.593907in}}%
\pgfpathclose%
\pgfusepath{fill}%
\end{pgfscope}%
\begin{pgfscope}%
\pgfpathrectangle{\pgfqpoint{1.150000in}{0.150000in}}{\pgfqpoint{5.700000in}{5.700000in}}%
\pgfusepath{clip}%
\pgfsetbuttcap%
\pgfsetroundjoin%
\definecolor{currentfill}{rgb}{0.201239,0.383670,0.554294}%
\pgfsetfillcolor{currentfill}%
\pgfsetfillopacity{0.800000}%
\pgfsetlinewidth{0.000000pt}%
\definecolor{currentstroke}{rgb}{0.000000,0.000000,0.000000}%
\pgfsetstrokecolor{currentstroke}%
\pgfsetdash{}{0pt}%
\pgfpathmoveto{\pgfqpoint{2.888550in}{3.102188in}}%
\pgfpathlineto{\pgfqpoint{2.902122in}{3.081632in}}%
\pgfpathlineto{\pgfqpoint{2.915685in}{3.061410in}}%
\pgfpathlineto{\pgfqpoint{2.929238in}{3.041519in}}%
\pgfpathlineto{\pgfqpoint{2.942782in}{3.021956in}}%
\pgfpathlineto{\pgfqpoint{2.950868in}{3.033897in}}%
\pgfpathlineto{\pgfqpoint{2.958947in}{3.045989in}}%
\pgfpathlineto{\pgfqpoint{2.967017in}{3.058234in}}%
\pgfpathlineto{\pgfqpoint{2.975079in}{3.070635in}}%
\pgfpathlineto{\pgfqpoint{2.961550in}{3.090293in}}%
\pgfpathlineto{\pgfqpoint{2.948012in}{3.110280in}}%
\pgfpathlineto{\pgfqpoint{2.934465in}{3.130597in}}%
\pgfpathlineto{\pgfqpoint{2.920908in}{3.151249in}}%
\pgfpathlineto{\pgfqpoint{2.912831in}{3.138741in}}%
\pgfpathlineto{\pgfqpoint{2.904745in}{3.126396in}}%
\pgfpathlineto{\pgfqpoint{2.896652in}{3.114212in}}%
\pgfpathlineto{\pgfqpoint{2.888550in}{3.102188in}}%
\pgfpathclose%
\pgfusepath{fill}%
\end{pgfscope}%
\begin{pgfscope}%
\pgfpathrectangle{\pgfqpoint{1.150000in}{0.150000in}}{\pgfqpoint{5.700000in}{5.700000in}}%
\pgfusepath{clip}%
\pgfsetbuttcap%
\pgfsetroundjoin%
\definecolor{currentfill}{rgb}{0.175841,0.441290,0.557685}%
\pgfsetfillcolor{currentfill}%
\pgfsetfillopacity{0.800000}%
\pgfsetlinewidth{0.000000pt}%
\definecolor{currentstroke}{rgb}{0.000000,0.000000,0.000000}%
\pgfsetstrokecolor{currentstroke}%
\pgfsetdash{}{0pt}%
\pgfpathmoveto{\pgfqpoint{5.232526in}{3.221855in}}%
\pgfpathlineto{\pgfqpoint{5.246328in}{3.222367in}}%
\pgfpathlineto{\pgfqpoint{5.260141in}{3.223053in}}%
\pgfpathlineto{\pgfqpoint{5.273967in}{3.223913in}}%
\pgfpathlineto{\pgfqpoint{5.287806in}{3.224946in}}%
\pgfpathlineto{\pgfqpoint{5.295259in}{3.237740in}}%
\pgfpathlineto{\pgfqpoint{5.302715in}{3.250819in}}%
\pgfpathlineto{\pgfqpoint{5.310175in}{3.264192in}}%
\pgfpathlineto{\pgfqpoint{5.317637in}{3.277866in}}%
\pgfpathlineto{\pgfqpoint{5.303821in}{3.277561in}}%
\pgfpathlineto{\pgfqpoint{5.290017in}{3.277429in}}%
\pgfpathlineto{\pgfqpoint{5.276224in}{3.277470in}}%
\pgfpathlineto{\pgfqpoint{5.262444in}{3.277684in}}%
\pgfpathlineto{\pgfqpoint{5.254960in}{3.263271in}}%
\pgfpathlineto{\pgfqpoint{5.247479in}{3.249167in}}%
\pgfpathlineto{\pgfqpoint{5.240001in}{3.235365in}}%
\pgfpathlineto{\pgfqpoint{5.232526in}{3.221855in}}%
\pgfpathclose%
\pgfusepath{fill}%
\end{pgfscope}%
\begin{pgfscope}%
\pgfpathrectangle{\pgfqpoint{1.150000in}{0.150000in}}{\pgfqpoint{5.700000in}{5.700000in}}%
\pgfusepath{clip}%
\pgfsetbuttcap%
\pgfsetroundjoin%
\definecolor{currentfill}{rgb}{0.280868,0.160771,0.472899}%
\pgfsetfillcolor{currentfill}%
\pgfsetfillopacity{0.800000}%
\pgfsetlinewidth{0.000000pt}%
\definecolor{currentstroke}{rgb}{0.000000,0.000000,0.000000}%
\pgfsetstrokecolor{currentstroke}%
\pgfsetdash{}{0pt}%
\pgfpathmoveto{\pgfqpoint{3.735866in}{2.528557in}}%
\pgfpathlineto{\pgfqpoint{3.749261in}{2.522684in}}%
\pgfpathlineto{\pgfqpoint{3.762660in}{2.517033in}}%
\pgfpathlineto{\pgfqpoint{3.776063in}{2.511601in}}%
\pgfpathlineto{\pgfqpoint{3.789470in}{2.506389in}}%
\pgfpathlineto{\pgfqpoint{3.797326in}{2.517765in}}%
\pgfpathlineto{\pgfqpoint{3.805178in}{2.529207in}}%
\pgfpathlineto{\pgfqpoint{3.813025in}{2.540715in}}%
\pgfpathlineto{\pgfqpoint{3.820867in}{2.552294in}}%
\pgfpathlineto{\pgfqpoint{3.807469in}{2.557697in}}%
\pgfpathlineto{\pgfqpoint{3.794074in}{2.563319in}}%
\pgfpathlineto{\pgfqpoint{3.780684in}{2.569160in}}%
\pgfpathlineto{\pgfqpoint{3.767297in}{2.575223in}}%
\pgfpathlineto{\pgfqpoint{3.759447in}{2.563443in}}%
\pgfpathlineto{\pgfqpoint{3.751591in}{2.551740in}}%
\pgfpathlineto{\pgfqpoint{3.743731in}{2.540112in}}%
\pgfpathlineto{\pgfqpoint{3.735866in}{2.528557in}}%
\pgfpathclose%
\pgfusepath{fill}%
\end{pgfscope}%
\begin{pgfscope}%
\pgfpathrectangle{\pgfqpoint{1.150000in}{0.150000in}}{\pgfqpoint{5.700000in}{5.700000in}}%
\pgfusepath{clip}%
\pgfsetbuttcap%
\pgfsetroundjoin%
\definecolor{currentfill}{rgb}{0.277134,0.185228,0.489898}%
\pgfsetfillcolor{currentfill}%
\pgfsetfillopacity{0.800000}%
\pgfsetlinewidth{0.000000pt}%
\definecolor{currentstroke}{rgb}{0.000000,0.000000,0.000000}%
\pgfsetstrokecolor{currentstroke}%
\pgfsetdash{}{0pt}%
\pgfpathmoveto{\pgfqpoint{3.319460in}{2.596172in}}%
\pgfpathlineto{\pgfqpoint{3.332859in}{2.584975in}}%
\pgfpathlineto{\pgfqpoint{3.346257in}{2.574034in}}%
\pgfpathlineto{\pgfqpoint{3.359653in}{2.563346in}}%
\pgfpathlineto{\pgfqpoint{3.373048in}{2.552911in}}%
\pgfpathlineto{\pgfqpoint{3.381028in}{2.564182in}}%
\pgfpathlineto{\pgfqpoint{3.389002in}{2.575541in}}%
\pgfpathlineto{\pgfqpoint{3.396971in}{2.586990in}}%
\pgfpathlineto{\pgfqpoint{3.404933in}{2.598531in}}%
\pgfpathlineto{\pgfqpoint{3.391549in}{2.609062in}}%
\pgfpathlineto{\pgfqpoint{3.378164in}{2.619846in}}%
\pgfpathlineto{\pgfqpoint{3.364778in}{2.630883in}}%
\pgfpathlineto{\pgfqpoint{3.351390in}{2.642176in}}%
\pgfpathlineto{\pgfqpoint{3.343417in}{2.630528in}}%
\pgfpathlineto{\pgfqpoint{3.335438in}{2.618979in}}%
\pgfpathlineto{\pgfqpoint{3.327452in}{2.607528in}}%
\pgfpathlineto{\pgfqpoint{3.319460in}{2.596172in}}%
\pgfpathclose%
\pgfusepath{fill}%
\end{pgfscope}%
\begin{pgfscope}%
\pgfpathrectangle{\pgfqpoint{1.150000in}{0.150000in}}{\pgfqpoint{5.700000in}{5.700000in}}%
\pgfusepath{clip}%
\pgfsetbuttcap%
\pgfsetroundjoin%
\definecolor{currentfill}{rgb}{0.278012,0.180367,0.486697}%
\pgfsetfillcolor{currentfill}%
\pgfsetfillopacity{0.800000}%
\pgfsetlinewidth{0.000000pt}%
\definecolor{currentstroke}{rgb}{0.000000,0.000000,0.000000}%
\pgfsetstrokecolor{currentstroke}%
\pgfsetdash{}{0pt}%
\pgfpathmoveto{\pgfqpoint{3.959468in}{2.562105in}}%
\pgfpathlineto{\pgfqpoint{3.972902in}{2.558401in}}%
\pgfpathlineto{\pgfqpoint{3.986341in}{2.554907in}}%
\pgfpathlineto{\pgfqpoint{3.999785in}{2.551620in}}%
\pgfpathlineto{\pgfqpoint{4.013236in}{2.548540in}}%
\pgfpathlineto{\pgfqpoint{4.021029in}{2.559783in}}%
\pgfpathlineto{\pgfqpoint{4.028817in}{2.571091in}}%
\pgfpathlineto{\pgfqpoint{4.036601in}{2.582464in}}%
\pgfpathlineto{\pgfqpoint{4.044380in}{2.593907in}}%
\pgfpathlineto{\pgfqpoint{4.030938in}{2.597241in}}%
\pgfpathlineto{\pgfqpoint{4.017501in}{2.600781in}}%
\pgfpathlineto{\pgfqpoint{4.004070in}{2.604529in}}%
\pgfpathlineto{\pgfqpoint{3.990645in}{2.608485in}}%
\pgfpathlineto{\pgfqpoint{3.982858in}{2.596777in}}%
\pgfpathlineto{\pgfqpoint{3.975066in}{2.585147in}}%
\pgfpathlineto{\pgfqpoint{3.967270in}{2.573590in}}%
\pgfpathlineto{\pgfqpoint{3.959468in}{2.562105in}}%
\pgfpathclose%
\pgfusepath{fill}%
\end{pgfscope}%
\begin{pgfscope}%
\pgfpathrectangle{\pgfqpoint{1.150000in}{0.150000in}}{\pgfqpoint{5.700000in}{5.700000in}}%
\pgfusepath{clip}%
\pgfsetbuttcap%
\pgfsetroundjoin%
\definecolor{currentfill}{rgb}{0.168126,0.459988,0.558082}%
\pgfsetfillcolor{currentfill}%
\pgfsetfillopacity{0.800000}%
\pgfsetlinewidth{0.000000pt}%
\definecolor{currentstroke}{rgb}{0.000000,0.000000,0.000000}%
\pgfsetstrokecolor{currentstroke}%
\pgfsetdash{}{0pt}%
\pgfpathmoveto{\pgfqpoint{5.317637in}{3.277866in}}%
\pgfpathlineto{\pgfqpoint{5.331466in}{3.278344in}}%
\pgfpathlineto{\pgfqpoint{5.345307in}{3.278994in}}%
\pgfpathlineto{\pgfqpoint{5.359160in}{3.279817in}}%
\pgfpathlineto{\pgfqpoint{5.373026in}{3.280811in}}%
\pgfpathlineto{\pgfqpoint{5.380470in}{3.294048in}}%
\pgfpathlineto{\pgfqpoint{5.387917in}{3.307597in}}%
\pgfpathlineto{\pgfqpoint{5.395369in}{3.321466in}}%
\pgfpathlineto{\pgfqpoint{5.381521in}{3.321038in}}%
\pgfpathlineto{\pgfqpoint{5.367685in}{3.320782in}}%
\pgfpathlineto{\pgfqpoint{5.353861in}{3.320698in}}%
\pgfpathlineto{\pgfqpoint{5.340049in}{3.320786in}}%
\pgfpathlineto{\pgfqpoint{5.332574in}{3.306155in}}%
\pgfpathlineto{\pgfqpoint{5.325104in}{3.291851in}}%
\pgfpathlineto{\pgfqpoint{5.317637in}{3.277866in}}%
\pgfpathclose%
\pgfusepath{fill}%
\end{pgfscope}%
\begin{pgfscope}%
\pgfpathrectangle{\pgfqpoint{1.150000in}{0.150000in}}{\pgfqpoint{5.700000in}{5.700000in}}%
\pgfusepath{clip}%
\pgfsetbuttcap%
\pgfsetroundjoin%
\definecolor{currentfill}{rgb}{0.280868,0.160771,0.472899}%
\pgfsetfillcolor{currentfill}%
\pgfsetfillopacity{0.800000}%
\pgfsetlinewidth{0.000000pt}%
\definecolor{currentstroke}{rgb}{0.000000,0.000000,0.000000}%
\pgfsetstrokecolor{currentstroke}%
\pgfsetdash{}{0pt}%
\pgfpathmoveto{\pgfqpoint{3.512002in}{2.523163in}}%
\pgfpathlineto{\pgfqpoint{3.525388in}{2.514828in}}%
\pgfpathlineto{\pgfqpoint{3.538775in}{2.506730in}}%
\pgfpathlineto{\pgfqpoint{3.552164in}{2.498867in}}%
\pgfpathlineto{\pgfqpoint{3.565554in}{2.491238in}}%
\pgfpathlineto{\pgfqpoint{3.573480in}{2.502524in}}%
\pgfpathlineto{\pgfqpoint{3.581400in}{2.513882in}}%
\pgfpathlineto{\pgfqpoint{3.589315in}{2.525313in}}%
\pgfpathlineto{\pgfqpoint{3.597224in}{2.536819in}}%
\pgfpathlineto{\pgfqpoint{3.583844in}{2.544576in}}%
\pgfpathlineto{\pgfqpoint{3.570465in}{2.552567in}}%
\pgfpathlineto{\pgfqpoint{3.557087in}{2.560793in}}%
\pgfpathlineto{\pgfqpoint{3.543711in}{2.569255in}}%
\pgfpathlineto{\pgfqpoint{3.535792in}{2.557609in}}%
\pgfpathlineto{\pgfqpoint{3.527867in}{2.546047in}}%
\pgfpathlineto{\pgfqpoint{3.519937in}{2.534565in}}%
\pgfpathlineto{\pgfqpoint{3.512002in}{2.523163in}}%
\pgfpathclose%
\pgfusepath{fill}%
\end{pgfscope}%
\begin{pgfscope}%
\pgfpathrectangle{\pgfqpoint{1.150000in}{0.150000in}}{\pgfqpoint{5.700000in}{5.700000in}}%
\pgfusepath{clip}%
\pgfsetbuttcap%
\pgfsetroundjoin%
\definecolor{currentfill}{rgb}{0.187231,0.414746,0.556547}%
\pgfsetfillcolor{currentfill}%
\pgfsetfillopacity{0.800000}%
\pgfsetlinewidth{0.000000pt}%
\definecolor{currentstroke}{rgb}{0.000000,0.000000,0.000000}%
\pgfsetstrokecolor{currentstroke}%
\pgfsetdash{}{0pt}%
\pgfpathmoveto{\pgfqpoint{2.834156in}{3.187823in}}%
\pgfpathlineto{\pgfqpoint{2.847771in}{3.165896in}}%
\pgfpathlineto{\pgfqpoint{2.861375in}{3.144317in}}%
\pgfpathlineto{\pgfqpoint{2.874967in}{3.123082in}}%
\pgfpathlineto{\pgfqpoint{2.888550in}{3.102188in}}%
\pgfpathlineto{\pgfqpoint{2.896652in}{3.114212in}}%
\pgfpathlineto{\pgfqpoint{2.904745in}{3.126396in}}%
\pgfpathlineto{\pgfqpoint{2.912831in}{3.138741in}}%
\pgfpathlineto{\pgfqpoint{2.920908in}{3.151249in}}%
\pgfpathlineto{\pgfqpoint{2.907341in}{3.172239in}}%
\pgfpathlineto{\pgfqpoint{2.893764in}{3.193570in}}%
\pgfpathlineto{\pgfqpoint{2.880176in}{3.215245in}}%
\pgfpathlineto{\pgfqpoint{2.866578in}{3.237268in}}%
\pgfpathlineto{\pgfqpoint{2.858485in}{3.224652in}}%
\pgfpathlineto{\pgfqpoint{2.850384in}{3.212206in}}%
\pgfpathlineto{\pgfqpoint{2.842274in}{3.199931in}}%
\pgfpathlineto{\pgfqpoint{2.834156in}{3.187823in}}%
\pgfpathclose%
\pgfusepath{fill}%
\end{pgfscope}%
\begin{pgfscope}%
\pgfpathrectangle{\pgfqpoint{1.150000in}{0.150000in}}{\pgfqpoint{5.700000in}{5.700000in}}%
\pgfusepath{clip}%
\pgfsetbuttcap%
\pgfsetroundjoin%
\definecolor{currentfill}{rgb}{0.241237,0.296485,0.539709}%
\pgfsetfillcolor{currentfill}%
\pgfsetfillopacity{0.800000}%
\pgfsetlinewidth{0.000000pt}%
\definecolor{currentstroke}{rgb}{0.000000,0.000000,0.000000}%
\pgfsetstrokecolor{currentstroke}%
\pgfsetdash{}{0pt}%
\pgfpathmoveto{\pgfqpoint{4.607690in}{2.825727in}}%
\pgfpathlineto{\pgfqpoint{4.621310in}{2.825885in}}%
\pgfpathlineto{\pgfqpoint{4.634939in}{2.826229in}}%
\pgfpathlineto{\pgfqpoint{4.648579in}{2.826759in}}%
\pgfpathlineto{\pgfqpoint{4.662228in}{2.827474in}}%
\pgfpathlineto{\pgfqpoint{4.669828in}{2.838258in}}%
\pgfpathlineto{\pgfqpoint{4.677424in}{2.849163in}}%
\pgfpathlineto{\pgfqpoint{4.685017in}{2.860193in}}%
\pgfpathlineto{\pgfqpoint{4.692607in}{2.871355in}}%
\pgfpathlineto{\pgfqpoint{4.678970in}{2.871115in}}%
\pgfpathlineto{\pgfqpoint{4.665343in}{2.871060in}}%
\pgfpathlineto{\pgfqpoint{4.651725in}{2.871190in}}%
\pgfpathlineto{\pgfqpoint{4.638118in}{2.871507in}}%
\pgfpathlineto{\pgfqpoint{4.630516in}{2.859859in}}%
\pgfpathlineto{\pgfqpoint{4.622910in}{2.848350in}}%
\pgfpathlineto{\pgfqpoint{4.615302in}{2.836974in}}%
\pgfpathlineto{\pgfqpoint{4.607690in}{2.825727in}}%
\pgfpathclose%
\pgfusepath{fill}%
\end{pgfscope}%
\begin{pgfscope}%
\pgfpathrectangle{\pgfqpoint{1.150000in}{0.150000in}}{\pgfqpoint{5.700000in}{5.700000in}}%
\pgfusepath{clip}%
\pgfsetbuttcap%
\pgfsetroundjoin%
\definecolor{currentfill}{rgb}{0.281412,0.155834,0.469201}%
\pgfsetfillcolor{currentfill}%
\pgfsetfillopacity{0.800000}%
\pgfsetlinewidth{0.000000pt}%
\definecolor{currentstroke}{rgb}{0.000000,0.000000,0.000000}%
\pgfsetstrokecolor{currentstroke}%
\pgfsetdash{}{0pt}%
\pgfpathmoveto{\pgfqpoint{3.650768in}{2.508100in}}%
\pgfpathlineto{\pgfqpoint{3.664160in}{2.501491in}}%
\pgfpathlineto{\pgfqpoint{3.677554in}{2.495108in}}%
\pgfpathlineto{\pgfqpoint{3.690952in}{2.488949in}}%
\pgfpathlineto{\pgfqpoint{3.704353in}{2.483014in}}%
\pgfpathlineto{\pgfqpoint{3.712239in}{2.494302in}}%
\pgfpathlineto{\pgfqpoint{3.720120in}{2.505654in}}%
\pgfpathlineto{\pgfqpoint{3.727995in}{2.517071in}}%
\pgfpathlineto{\pgfqpoint{3.735866in}{2.528557in}}%
\pgfpathlineto{\pgfqpoint{3.722474in}{2.534651in}}%
\pgfpathlineto{\pgfqpoint{3.709085in}{2.540969in}}%
\pgfpathlineto{\pgfqpoint{3.695699in}{2.547511in}}%
\pgfpathlineto{\pgfqpoint{3.682316in}{2.554279in}}%
\pgfpathlineto{\pgfqpoint{3.674437in}{2.542623in}}%
\pgfpathlineto{\pgfqpoint{3.666552in}{2.531043in}}%
\pgfpathlineto{\pgfqpoint{3.658663in}{2.519536in}}%
\pgfpathlineto{\pgfqpoint{3.650768in}{2.508100in}}%
\pgfpathclose%
\pgfusepath{fill}%
\end{pgfscope}%
\begin{pgfscope}%
\pgfpathrectangle{\pgfqpoint{1.150000in}{0.150000in}}{\pgfqpoint{5.700000in}{5.700000in}}%
\pgfusepath{clip}%
\pgfsetbuttcap%
\pgfsetroundjoin%
\definecolor{currentfill}{rgb}{0.248629,0.278775,0.534556}%
\pgfsetfillcolor{currentfill}%
\pgfsetfillopacity{0.800000}%
\pgfsetlinewidth{0.000000pt}%
\definecolor{currentstroke}{rgb}{0.000000,0.000000,0.000000}%
\pgfsetstrokecolor{currentstroke}%
\pgfsetdash{}{0pt}%
\pgfpathmoveto{\pgfqpoint{4.522785in}{2.781384in}}%
\pgfpathlineto{\pgfqpoint{4.536377in}{2.781234in}}%
\pgfpathlineto{\pgfqpoint{4.549979in}{2.781273in}}%
\pgfpathlineto{\pgfqpoint{4.563590in}{2.781499in}}%
\pgfpathlineto{\pgfqpoint{4.577211in}{2.781914in}}%
\pgfpathlineto{\pgfqpoint{4.584836in}{2.792701in}}%
\pgfpathlineto{\pgfqpoint{4.592458in}{2.803596in}}%
\pgfpathlineto{\pgfqpoint{4.600076in}{2.814603in}}%
\pgfpathlineto{\pgfqpoint{4.607690in}{2.825727in}}%
\pgfpathlineto{\pgfqpoint{4.594081in}{2.825756in}}%
\pgfpathlineto{\pgfqpoint{4.580481in}{2.825972in}}%
\pgfpathlineto{\pgfqpoint{4.566891in}{2.826377in}}%
\pgfpathlineto{\pgfqpoint{4.553310in}{2.826969in}}%
\pgfpathlineto{\pgfqpoint{4.545684in}{2.815390in}}%
\pgfpathlineto{\pgfqpoint{4.538055in}{2.803936in}}%
\pgfpathlineto{\pgfqpoint{4.530422in}{2.792603in}}%
\pgfpathlineto{\pgfqpoint{4.522785in}{2.781384in}}%
\pgfpathclose%
\pgfusepath{fill}%
\end{pgfscope}%
\begin{pgfscope}%
\pgfpathrectangle{\pgfqpoint{1.150000in}{0.150000in}}{\pgfqpoint{5.700000in}{5.700000in}}%
\pgfusepath{clip}%
\pgfsetbuttcap%
\pgfsetroundjoin%
\definecolor{currentfill}{rgb}{0.280255,0.165693,0.476498}%
\pgfsetfillcolor{currentfill}%
\pgfsetfillopacity{0.800000}%
\pgfsetlinewidth{0.000000pt}%
\definecolor{currentstroke}{rgb}{0.000000,0.000000,0.000000}%
\pgfsetstrokecolor{currentstroke}%
\pgfsetdash{}{0pt}%
\pgfpathmoveto{\pgfqpoint{3.874503in}{2.532846in}}%
\pgfpathlineto{\pgfqpoint{3.887923in}{2.528521in}}%
\pgfpathlineto{\pgfqpoint{3.901349in}{2.524408in}}%
\pgfpathlineto{\pgfqpoint{3.914780in}{2.520506in}}%
\pgfpathlineto{\pgfqpoint{3.928216in}{2.516816in}}%
\pgfpathlineto{\pgfqpoint{3.936036in}{2.528046in}}%
\pgfpathlineto{\pgfqpoint{3.943852in}{2.539335in}}%
\pgfpathlineto{\pgfqpoint{3.951662in}{2.550687in}}%
\pgfpathlineto{\pgfqpoint{3.959468in}{2.562105in}}%
\pgfpathlineto{\pgfqpoint{3.946041in}{2.566018in}}%
\pgfpathlineto{\pgfqpoint{3.932618in}{2.570141in}}%
\pgfpathlineto{\pgfqpoint{3.919201in}{2.574476in}}%
\pgfpathlineto{\pgfqpoint{3.905788in}{2.579023in}}%
\pgfpathlineto{\pgfqpoint{3.897974in}{2.567373in}}%
\pgfpathlineto{\pgfqpoint{3.890155in}{2.555795in}}%
\pgfpathlineto{\pgfqpoint{3.882331in}{2.544287in}}%
\pgfpathlineto{\pgfqpoint{3.874503in}{2.532846in}}%
\pgfpathclose%
\pgfusepath{fill}%
\end{pgfscope}%
\begin{pgfscope}%
\pgfpathrectangle{\pgfqpoint{1.150000in}{0.150000in}}{\pgfqpoint{5.700000in}{5.700000in}}%
\pgfusepath{clip}%
\pgfsetbuttcap%
\pgfsetroundjoin%
\definecolor{currentfill}{rgb}{0.233603,0.313828,0.543914}%
\pgfsetfillcolor{currentfill}%
\pgfsetfillopacity{0.800000}%
\pgfsetlinewidth{0.000000pt}%
\definecolor{currentstroke}{rgb}{0.000000,0.000000,0.000000}%
\pgfsetstrokecolor{currentstroke}%
\pgfsetdash{}{0pt}%
\pgfpathmoveto{\pgfqpoint{4.692607in}{2.871355in}}%
\pgfpathlineto{\pgfqpoint{4.706254in}{2.871780in}}%
\pgfpathlineto{\pgfqpoint{4.719912in}{2.872390in}}%
\pgfpathlineto{\pgfqpoint{4.733581in}{2.873183in}}%
\pgfpathlineto{\pgfqpoint{4.747260in}{2.874160in}}%
\pgfpathlineto{\pgfqpoint{4.754834in}{2.884964in}}%
\pgfpathlineto{\pgfqpoint{4.762405in}{2.895903in}}%
\pgfpathlineto{\pgfqpoint{4.769974in}{2.906984in}}%
\pgfpathlineto{\pgfqpoint{4.777540in}{2.918211in}}%
\pgfpathlineto{\pgfqpoint{4.763874in}{2.917741in}}%
\pgfpathlineto{\pgfqpoint{4.750219in}{2.917455in}}%
\pgfpathlineto{\pgfqpoint{4.736575in}{2.917352in}}%
\pgfpathlineto{\pgfqpoint{4.722940in}{2.917433in}}%
\pgfpathlineto{\pgfqpoint{4.715361in}{2.905687in}}%
\pgfpathlineto{\pgfqpoint{4.707779in}{2.894097in}}%
\pgfpathlineto{\pgfqpoint{4.700194in}{2.882654in}}%
\pgfpathlineto{\pgfqpoint{4.692607in}{2.871355in}}%
\pgfpathclose%
\pgfusepath{fill}%
\end{pgfscope}%
\begin{pgfscope}%
\pgfpathrectangle{\pgfqpoint{1.150000in}{0.150000in}}{\pgfqpoint{5.700000in}{5.700000in}}%
\pgfusepath{clip}%
\pgfsetbuttcap%
\pgfsetroundjoin%
\definecolor{currentfill}{rgb}{0.255645,0.260703,0.528312}%
\pgfsetfillcolor{currentfill}%
\pgfsetfillopacity{0.800000}%
\pgfsetlinewidth{0.000000pt}%
\definecolor{currentstroke}{rgb}{0.000000,0.000000,0.000000}%
\pgfsetstrokecolor{currentstroke}%
\pgfsetdash{}{0pt}%
\pgfpathmoveto{\pgfqpoint{4.437886in}{2.738406in}}%
\pgfpathlineto{\pgfqpoint{4.451451in}{2.737908in}}%
\pgfpathlineto{\pgfqpoint{4.465026in}{2.737601in}}%
\pgfpathlineto{\pgfqpoint{4.478610in}{2.737484in}}%
\pgfpathlineto{\pgfqpoint{4.492204in}{2.737557in}}%
\pgfpathlineto{\pgfqpoint{4.499855in}{2.748366in}}%
\pgfpathlineto{\pgfqpoint{4.507502in}{2.759270in}}%
\pgfpathlineto{\pgfqpoint{4.515145in}{2.770274in}}%
\pgfpathlineto{\pgfqpoint{4.522785in}{2.781384in}}%
\pgfpathlineto{\pgfqpoint{4.509203in}{2.781722in}}%
\pgfpathlineto{\pgfqpoint{4.495629in}{2.782251in}}%
\pgfpathlineto{\pgfqpoint{4.482065in}{2.782969in}}%
\pgfpathlineto{\pgfqpoint{4.468510in}{2.783879in}}%
\pgfpathlineto{\pgfqpoint{4.460860in}{2.772347in}}%
\pgfpathlineto{\pgfqpoint{4.453205in}{2.760927in}}%
\pgfpathlineto{\pgfqpoint{4.445547in}{2.749615in}}%
\pgfpathlineto{\pgfqpoint{4.437886in}{2.738406in}}%
\pgfpathclose%
\pgfusepath{fill}%
\end{pgfscope}%
\begin{pgfscope}%
\pgfpathrectangle{\pgfqpoint{1.150000in}{0.150000in}}{\pgfqpoint{5.700000in}{5.700000in}}%
\pgfusepath{clip}%
\pgfsetbuttcap%
\pgfsetroundjoin%
\definecolor{currentfill}{rgb}{0.223925,0.334994,0.548053}%
\pgfsetfillcolor{currentfill}%
\pgfsetfillopacity{0.800000}%
\pgfsetlinewidth{0.000000pt}%
\definecolor{currentstroke}{rgb}{0.000000,0.000000,0.000000}%
\pgfsetstrokecolor{currentstroke}%
\pgfsetdash{}{0pt}%
\pgfpathmoveto{\pgfqpoint{4.777540in}{2.918211in}}%
\pgfpathlineto{\pgfqpoint{4.791216in}{2.918864in}}%
\pgfpathlineto{\pgfqpoint{4.804903in}{2.919699in}}%
\pgfpathlineto{\pgfqpoint{4.818601in}{2.920716in}}%
\pgfpathlineto{\pgfqpoint{4.832310in}{2.921914in}}%
\pgfpathlineto{\pgfqpoint{4.839859in}{2.932768in}}%
\pgfpathlineto{\pgfqpoint{4.847406in}{2.943773in}}%
\pgfpathlineto{\pgfqpoint{4.854951in}{2.954935in}}%
\pgfpathlineto{\pgfqpoint{4.862494in}{2.966262in}}%
\pgfpathlineto{\pgfqpoint{4.848800in}{2.965602in}}%
\pgfpathlineto{\pgfqpoint{4.835117in}{2.965123in}}%
\pgfpathlineto{\pgfqpoint{4.821444in}{2.964826in}}%
\pgfpathlineto{\pgfqpoint{4.807782in}{2.964711in}}%
\pgfpathlineto{\pgfqpoint{4.800225in}{2.952836in}}%
\pgfpathlineto{\pgfqpoint{4.792665in}{2.941131in}}%
\pgfpathlineto{\pgfqpoint{4.785103in}{2.929592in}}%
\pgfpathlineto{\pgfqpoint{4.777540in}{2.918211in}}%
\pgfpathclose%
\pgfusepath{fill}%
\end{pgfscope}%
\begin{pgfscope}%
\pgfpathrectangle{\pgfqpoint{1.150000in}{0.150000in}}{\pgfqpoint{5.700000in}{5.700000in}}%
\pgfusepath{clip}%
\pgfsetbuttcap%
\pgfsetroundjoin%
\definecolor{currentfill}{rgb}{0.255645,0.260703,0.528312}%
\pgfsetfillcolor{currentfill}%
\pgfsetfillopacity{0.800000}%
\pgfsetlinewidth{0.000000pt}%
\definecolor{currentstroke}{rgb}{0.000000,0.000000,0.000000}%
\pgfsetstrokecolor{currentstroke}%
\pgfsetdash{}{0pt}%
\pgfpathmoveto{\pgfqpoint{3.072468in}{2.766202in}}%
\pgfpathlineto{\pgfqpoint{3.085934in}{2.750704in}}%
\pgfpathlineto{\pgfqpoint{3.099395in}{2.735495in}}%
\pgfpathlineto{\pgfqpoint{3.112851in}{2.720573in}}%
\pgfpathlineto{\pgfqpoint{3.126302in}{2.705936in}}%
\pgfpathlineto{\pgfqpoint{3.134358in}{2.717076in}}%
\pgfpathlineto{\pgfqpoint{3.142408in}{2.728331in}}%
\pgfpathlineto{\pgfqpoint{3.150451in}{2.739701in}}%
\pgfpathlineto{\pgfqpoint{3.158487in}{2.751189in}}%
\pgfpathlineto{\pgfqpoint{3.145049in}{2.765891in}}%
\pgfpathlineto{\pgfqpoint{3.131607in}{2.780876in}}%
\pgfpathlineto{\pgfqpoint{3.118160in}{2.796149in}}%
\pgfpathlineto{\pgfqpoint{3.104708in}{2.811711in}}%
\pgfpathlineto{\pgfqpoint{3.096659in}{2.800147in}}%
\pgfpathlineto{\pgfqpoint{3.088602in}{2.788708in}}%
\pgfpathlineto{\pgfqpoint{3.080538in}{2.777394in}}%
\pgfpathlineto{\pgfqpoint{3.072468in}{2.766202in}}%
\pgfpathclose%
\pgfusepath{fill}%
\end{pgfscope}%
\begin{pgfscope}%
\pgfpathrectangle{\pgfqpoint{1.150000in}{0.150000in}}{\pgfqpoint{5.700000in}{5.700000in}}%
\pgfusepath{clip}%
\pgfsetbuttcap%
\pgfsetroundjoin%
\definecolor{currentfill}{rgb}{0.279574,0.170599,0.479997}%
\pgfsetfillcolor{currentfill}%
\pgfsetfillopacity{0.800000}%
\pgfsetlinewidth{0.000000pt}%
\definecolor{currentstroke}{rgb}{0.000000,0.000000,0.000000}%
\pgfsetstrokecolor{currentstroke}%
\pgfsetdash{}{0pt}%
\pgfpathmoveto{\pgfqpoint{3.373048in}{2.552911in}}%
\pgfpathlineto{\pgfqpoint{3.386443in}{2.542726in}}%
\pgfpathlineto{\pgfqpoint{3.399837in}{2.532790in}}%
\pgfpathlineto{\pgfqpoint{3.413231in}{2.523102in}}%
\pgfpathlineto{\pgfqpoint{3.426625in}{2.513659in}}%
\pgfpathlineto{\pgfqpoint{3.434594in}{2.524846in}}%
\pgfpathlineto{\pgfqpoint{3.442557in}{2.536113in}}%
\pgfpathlineto{\pgfqpoint{3.450514in}{2.547462in}}%
\pgfpathlineto{\pgfqpoint{3.458466in}{2.558895in}}%
\pgfpathlineto{\pgfqpoint{3.445083in}{2.568434in}}%
\pgfpathlineto{\pgfqpoint{3.431700in}{2.578219in}}%
\pgfpathlineto{\pgfqpoint{3.418316in}{2.588250in}}%
\pgfpathlineto{\pgfqpoint{3.404933in}{2.598531in}}%
\pgfpathlineto{\pgfqpoint{3.396971in}{2.586990in}}%
\pgfpathlineto{\pgfqpoint{3.389002in}{2.575541in}}%
\pgfpathlineto{\pgfqpoint{3.381028in}{2.564182in}}%
\pgfpathlineto{\pgfqpoint{3.373048in}{2.552911in}}%
\pgfpathclose%
\pgfusepath{fill}%
\end{pgfscope}%
\begin{pgfscope}%
\pgfpathrectangle{\pgfqpoint{1.150000in}{0.150000in}}{\pgfqpoint{5.700000in}{5.700000in}}%
\pgfusepath{clip}%
\pgfsetbuttcap%
\pgfsetroundjoin%
\definecolor{currentfill}{rgb}{0.244972,0.287675,0.537260}%
\pgfsetfillcolor{currentfill}%
\pgfsetfillopacity{0.800000}%
\pgfsetlinewidth{0.000000pt}%
\definecolor{currentstroke}{rgb}{0.000000,0.000000,0.000000}%
\pgfsetstrokecolor{currentstroke}%
\pgfsetdash{}{0pt}%
\pgfpathmoveto{\pgfqpoint{3.018544in}{2.831138in}}%
\pgfpathlineto{\pgfqpoint{3.032034in}{2.814457in}}%
\pgfpathlineto{\pgfqpoint{3.045518in}{2.798076in}}%
\pgfpathlineto{\pgfqpoint{3.058996in}{2.781992in}}%
\pgfpathlineto{\pgfqpoint{3.072468in}{2.766202in}}%
\pgfpathlineto{\pgfqpoint{3.080538in}{2.777394in}}%
\pgfpathlineto{\pgfqpoint{3.088602in}{2.788708in}}%
\pgfpathlineto{\pgfqpoint{3.096659in}{2.800147in}}%
\pgfpathlineto{\pgfqpoint{3.104708in}{2.811711in}}%
\pgfpathlineto{\pgfqpoint{3.091251in}{2.827564in}}%
\pgfpathlineto{\pgfqpoint{3.077788in}{2.843712in}}%
\pgfpathlineto{\pgfqpoint{3.064319in}{2.860157in}}%
\pgfpathlineto{\pgfqpoint{3.050843in}{2.876901in}}%
\pgfpathlineto{\pgfqpoint{3.042779in}{2.865262in}}%
\pgfpathlineto{\pgfqpoint{3.034708in}{2.853756in}}%
\pgfpathlineto{\pgfqpoint{3.026630in}{2.842381in}}%
\pgfpathlineto{\pgfqpoint{3.018544in}{2.831138in}}%
\pgfpathclose%
\pgfusepath{fill}%
\end{pgfscope}%
\begin{pgfscope}%
\pgfpathrectangle{\pgfqpoint{1.150000in}{0.150000in}}{\pgfqpoint{5.700000in}{5.700000in}}%
\pgfusepath{clip}%
\pgfsetbuttcap%
\pgfsetroundjoin%
\definecolor{currentfill}{rgb}{0.262138,0.242286,0.520837}%
\pgfsetfillcolor{currentfill}%
\pgfsetfillopacity{0.800000}%
\pgfsetlinewidth{0.000000pt}%
\definecolor{currentstroke}{rgb}{0.000000,0.000000,0.000000}%
\pgfsetstrokecolor{currentstroke}%
\pgfsetdash{}{0pt}%
\pgfpathmoveto{\pgfqpoint{4.352986in}{2.696901in}}%
\pgfpathlineto{\pgfqpoint{4.366526in}{2.696013in}}%
\pgfpathlineto{\pgfqpoint{4.380075in}{2.695318in}}%
\pgfpathlineto{\pgfqpoint{4.393633in}{2.694817in}}%
\pgfpathlineto{\pgfqpoint{4.407200in}{2.694508in}}%
\pgfpathlineto{\pgfqpoint{4.414878in}{2.705351in}}%
\pgfpathlineto{\pgfqpoint{4.422551in}{2.716279in}}%
\pgfpathlineto{\pgfqpoint{4.430220in}{2.727296in}}%
\pgfpathlineto{\pgfqpoint{4.437886in}{2.738406in}}%
\pgfpathlineto{\pgfqpoint{4.424329in}{2.739096in}}%
\pgfpathlineto{\pgfqpoint{4.410781in}{2.739978in}}%
\pgfpathlineto{\pgfqpoint{4.397242in}{2.741052in}}%
\pgfpathlineto{\pgfqpoint{4.383711in}{2.742319in}}%
\pgfpathlineto{\pgfqpoint{4.376036in}{2.730817in}}%
\pgfpathlineto{\pgfqpoint{4.368356in}{2.719416in}}%
\pgfpathlineto{\pgfqpoint{4.360673in}{2.708112in}}%
\pgfpathlineto{\pgfqpoint{4.352986in}{2.696901in}}%
\pgfpathclose%
\pgfusepath{fill}%
\end{pgfscope}%
\begin{pgfscope}%
\pgfpathrectangle{\pgfqpoint{1.150000in}{0.150000in}}{\pgfqpoint{5.700000in}{5.700000in}}%
\pgfusepath{clip}%
\pgfsetbuttcap%
\pgfsetroundjoin%
\definecolor{currentfill}{rgb}{0.216210,0.351535,0.550627}%
\pgfsetfillcolor{currentfill}%
\pgfsetfillopacity{0.800000}%
\pgfsetlinewidth{0.000000pt}%
\definecolor{currentstroke}{rgb}{0.000000,0.000000,0.000000}%
\pgfsetstrokecolor{currentstroke}%
\pgfsetdash{}{0pt}%
\pgfpathmoveto{\pgfqpoint{4.862494in}{2.966262in}}%
\pgfpathlineto{\pgfqpoint{4.876200in}{2.967102in}}%
\pgfpathlineto{\pgfqpoint{4.889916in}{2.968124in}}%
\pgfpathlineto{\pgfqpoint{4.903644in}{2.969325in}}%
\pgfpathlineto{\pgfqpoint{4.917383in}{2.970707in}}%
\pgfpathlineto{\pgfqpoint{4.924909in}{2.981644in}}%
\pgfpathlineto{\pgfqpoint{4.932433in}{2.992751in}}%
\pgfpathlineto{\pgfqpoint{4.939956in}{3.004033in}}%
\pgfpathlineto{\pgfqpoint{4.947477in}{3.015497in}}%
\pgfpathlineto{\pgfqpoint{4.933753in}{3.014686in}}%
\pgfpathlineto{\pgfqpoint{4.920041in}{3.014055in}}%
\pgfpathlineto{\pgfqpoint{4.906340in}{3.013603in}}%
\pgfpathlineto{\pgfqpoint{4.892650in}{3.013332in}}%
\pgfpathlineto{\pgfqpoint{4.885113in}{3.001286in}}%
\pgfpathlineto{\pgfqpoint{4.877575in}{2.989431in}}%
\pgfpathlineto{\pgfqpoint{4.870036in}{2.977758in}}%
\pgfpathlineto{\pgfqpoint{4.862494in}{2.966262in}}%
\pgfpathclose%
\pgfusepath{fill}%
\end{pgfscope}%
\begin{pgfscope}%
\pgfpathrectangle{\pgfqpoint{1.150000in}{0.150000in}}{\pgfqpoint{5.700000in}{5.700000in}}%
\pgfusepath{clip}%
\pgfsetbuttcap%
\pgfsetroundjoin%
\definecolor{currentfill}{rgb}{0.263663,0.237631,0.518762}%
\pgfsetfillcolor{currentfill}%
\pgfsetfillopacity{0.800000}%
\pgfsetlinewidth{0.000000pt}%
\definecolor{currentstroke}{rgb}{0.000000,0.000000,0.000000}%
\pgfsetstrokecolor{currentstroke}%
\pgfsetdash{}{0pt}%
\pgfpathmoveto{\pgfqpoint{3.126302in}{2.705936in}}%
\pgfpathlineto{\pgfqpoint{3.139749in}{2.691581in}}%
\pgfpathlineto{\pgfqpoint{3.153191in}{2.677506in}}%
\pgfpathlineto{\pgfqpoint{3.166629in}{2.663709in}}%
\pgfpathlineto{\pgfqpoint{3.180064in}{2.650187in}}%
\pgfpathlineto{\pgfqpoint{3.188106in}{2.661275in}}%
\pgfpathlineto{\pgfqpoint{3.196142in}{2.672470in}}%
\pgfpathlineto{\pgfqpoint{3.204172in}{2.683773in}}%
\pgfpathlineto{\pgfqpoint{3.212194in}{2.695185in}}%
\pgfpathlineto{\pgfqpoint{3.198773in}{2.708771in}}%
\pgfpathlineto{\pgfqpoint{3.185348in}{2.722632in}}%
\pgfpathlineto{\pgfqpoint{3.171920in}{2.736771in}}%
\pgfpathlineto{\pgfqpoint{3.158487in}{2.751189in}}%
\pgfpathlineto{\pgfqpoint{3.150451in}{2.739701in}}%
\pgfpathlineto{\pgfqpoint{3.142408in}{2.728331in}}%
\pgfpathlineto{\pgfqpoint{3.134358in}{2.717076in}}%
\pgfpathlineto{\pgfqpoint{3.126302in}{2.705936in}}%
\pgfpathclose%
\pgfusepath{fill}%
\end{pgfscope}%
\begin{pgfscope}%
\pgfpathrectangle{\pgfqpoint{1.150000in}{0.150000in}}{\pgfqpoint{5.700000in}{5.700000in}}%
\pgfusepath{clip}%
\pgfsetbuttcap%
\pgfsetroundjoin%
\definecolor{currentfill}{rgb}{0.233603,0.313828,0.543914}%
\pgfsetfillcolor{currentfill}%
\pgfsetfillopacity{0.800000}%
\pgfsetlinewidth{0.000000pt}%
\definecolor{currentstroke}{rgb}{0.000000,0.000000,0.000000}%
\pgfsetstrokecolor{currentstroke}%
\pgfsetdash{}{0pt}%
\pgfpathmoveto{\pgfqpoint{2.964514in}{2.900908in}}%
\pgfpathlineto{\pgfqpoint{2.978032in}{2.883003in}}%
\pgfpathlineto{\pgfqpoint{2.991543in}{2.865408in}}%
\pgfpathlineto{\pgfqpoint{3.005047in}{2.848121in}}%
\pgfpathlineto{\pgfqpoint{3.018544in}{2.831138in}}%
\pgfpathlineto{\pgfqpoint{3.026630in}{2.842381in}}%
\pgfpathlineto{\pgfqpoint{3.034708in}{2.853756in}}%
\pgfpathlineto{\pgfqpoint{3.042779in}{2.865262in}}%
\pgfpathlineto{\pgfqpoint{3.050843in}{2.876901in}}%
\pgfpathlineto{\pgfqpoint{3.037361in}{2.893947in}}%
\pgfpathlineto{\pgfqpoint{3.023873in}{2.911297in}}%
\pgfpathlineto{\pgfqpoint{3.010377in}{2.928956in}}%
\pgfpathlineto{\pgfqpoint{2.996874in}{2.946924in}}%
\pgfpathlineto{\pgfqpoint{2.988795in}{2.935209in}}%
\pgfpathlineto{\pgfqpoint{2.980709in}{2.923636in}}%
\pgfpathlineto{\pgfqpoint{2.972615in}{2.912202in}}%
\pgfpathlineto{\pgfqpoint{2.964514in}{2.900908in}}%
\pgfpathclose%
\pgfusepath{fill}%
\end{pgfscope}%
\begin{pgfscope}%
\pgfpathrectangle{\pgfqpoint{1.150000in}{0.150000in}}{\pgfqpoint{5.700000in}{5.700000in}}%
\pgfusepath{clip}%
\pgfsetbuttcap%
\pgfsetroundjoin%
\definecolor{currentfill}{rgb}{0.206756,0.371758,0.553117}%
\pgfsetfillcolor{currentfill}%
\pgfsetfillopacity{0.800000}%
\pgfsetlinewidth{0.000000pt}%
\definecolor{currentstroke}{rgb}{0.000000,0.000000,0.000000}%
\pgfsetstrokecolor{currentstroke}%
\pgfsetdash{}{0pt}%
\pgfpathmoveto{\pgfqpoint{4.947477in}{3.015497in}}%
\pgfpathlineto{\pgfqpoint{4.961211in}{3.016487in}}%
\pgfpathlineto{\pgfqpoint{4.974957in}{3.017656in}}%
\pgfpathlineto{\pgfqpoint{4.988715in}{3.019004in}}%
\pgfpathlineto{\pgfqpoint{5.002485in}{3.020529in}}%
\pgfpathlineto{\pgfqpoint{5.009988in}{3.031592in}}%
\pgfpathlineto{\pgfqpoint{5.017490in}{3.042841in}}%
\pgfpathlineto{\pgfqpoint{5.024991in}{3.054286in}}%
\pgfpathlineto{\pgfqpoint{5.032492in}{3.065932in}}%
\pgfpathlineto{\pgfqpoint{5.018740in}{3.065009in}}%
\pgfpathlineto{\pgfqpoint{5.004999in}{3.064263in}}%
\pgfpathlineto{\pgfqpoint{4.991270in}{3.063696in}}%
\pgfpathlineto{\pgfqpoint{4.977552in}{3.063307in}}%
\pgfpathlineto{\pgfqpoint{4.970034in}{3.051048in}}%
\pgfpathlineto{\pgfqpoint{4.962516in}{3.038998in}}%
\pgfpathlineto{\pgfqpoint{4.954997in}{3.027150in}}%
\pgfpathlineto{\pgfqpoint{4.947477in}{3.015497in}}%
\pgfpathclose%
\pgfusepath{fill}%
\end{pgfscope}%
\begin{pgfscope}%
\pgfpathrectangle{\pgfqpoint{1.150000in}{0.150000in}}{\pgfqpoint{5.700000in}{5.700000in}}%
\pgfusepath{clip}%
\pgfsetbuttcap%
\pgfsetroundjoin%
\definecolor{currentfill}{rgb}{0.266580,0.228262,0.514349}%
\pgfsetfillcolor{currentfill}%
\pgfsetfillopacity{0.800000}%
\pgfsetlinewidth{0.000000pt}%
\definecolor{currentstroke}{rgb}{0.000000,0.000000,0.000000}%
\pgfsetstrokecolor{currentstroke}%
\pgfsetdash{}{0pt}%
\pgfpathmoveto{\pgfqpoint{4.268079in}{2.656996in}}%
\pgfpathlineto{\pgfqpoint{4.281596in}{2.655676in}}%
\pgfpathlineto{\pgfqpoint{4.295121in}{2.654553in}}%
\pgfpathlineto{\pgfqpoint{4.308654in}{2.653625in}}%
\pgfpathlineto{\pgfqpoint{4.322195in}{2.652891in}}%
\pgfpathlineto{\pgfqpoint{4.329899in}{2.663776in}}%
\pgfpathlineto{\pgfqpoint{4.337599in}{2.674737in}}%
\pgfpathlineto{\pgfqpoint{4.345294in}{2.685777in}}%
\pgfpathlineto{\pgfqpoint{4.352986in}{2.696901in}}%
\pgfpathlineto{\pgfqpoint{4.339454in}{2.697982in}}%
\pgfpathlineto{\pgfqpoint{4.325930in}{2.699259in}}%
\pgfpathlineto{\pgfqpoint{4.312414in}{2.700731in}}%
\pgfpathlineto{\pgfqpoint{4.298907in}{2.702398in}}%
\pgfpathlineto{\pgfqpoint{4.291206in}{2.690914in}}%
\pgfpathlineto{\pgfqpoint{4.283501in}{2.679522in}}%
\pgfpathlineto{\pgfqpoint{4.275792in}{2.668217in}}%
\pgfpathlineto{\pgfqpoint{4.268079in}{2.656996in}}%
\pgfpathclose%
\pgfusepath{fill}%
\end{pgfscope}%
\begin{pgfscope}%
\pgfpathrectangle{\pgfqpoint{1.150000in}{0.150000in}}{\pgfqpoint{5.700000in}{5.700000in}}%
\pgfusepath{clip}%
\pgfsetbuttcap%
\pgfsetroundjoin%
\definecolor{currentfill}{rgb}{0.197636,0.391528,0.554969}%
\pgfsetfillcolor{currentfill}%
\pgfsetfillopacity{0.800000}%
\pgfsetlinewidth{0.000000pt}%
\definecolor{currentstroke}{rgb}{0.000000,0.000000,0.000000}%
\pgfsetstrokecolor{currentstroke}%
\pgfsetdash{}{0pt}%
\pgfpathmoveto{\pgfqpoint{5.032492in}{3.065932in}}%
\pgfpathlineto{\pgfqpoint{5.046256in}{3.067033in}}%
\pgfpathlineto{\pgfqpoint{5.060032in}{3.068311in}}%
\pgfpathlineto{\pgfqpoint{5.073820in}{3.069766in}}%
\pgfpathlineto{\pgfqpoint{5.087620in}{3.071398in}}%
\pgfpathlineto{\pgfqpoint{5.095102in}{3.082631in}}%
\pgfpathlineto{\pgfqpoint{5.102584in}{3.094071in}}%
\pgfpathlineto{\pgfqpoint{5.110065in}{3.105727in}}%
\pgfpathlineto{\pgfqpoint{5.117547in}{3.117606in}}%
\pgfpathlineto{\pgfqpoint{5.103766in}{3.116608in}}%
\pgfpathlineto{\pgfqpoint{5.089996in}{3.115787in}}%
\pgfpathlineto{\pgfqpoint{5.076239in}{3.115142in}}%
\pgfpathlineto{\pgfqpoint{5.062493in}{3.114674in}}%
\pgfpathlineto{\pgfqpoint{5.054992in}{3.102151in}}%
\pgfpathlineto{\pgfqpoint{5.047492in}{3.089857in}}%
\pgfpathlineto{\pgfqpoint{5.039992in}{3.077787in}}%
\pgfpathlineto{\pgfqpoint{5.032492in}{3.065932in}}%
\pgfpathclose%
\pgfusepath{fill}%
\end{pgfscope}%
\begin{pgfscope}%
\pgfpathrectangle{\pgfqpoint{1.150000in}{0.150000in}}{\pgfqpoint{5.700000in}{5.700000in}}%
\pgfusepath{clip}%
\pgfsetbuttcap%
\pgfsetroundjoin%
\definecolor{currentfill}{rgb}{0.270595,0.214069,0.507052}%
\pgfsetfillcolor{currentfill}%
\pgfsetfillopacity{0.800000}%
\pgfsetlinewidth{0.000000pt}%
\definecolor{currentstroke}{rgb}{0.000000,0.000000,0.000000}%
\pgfsetstrokecolor{currentstroke}%
\pgfsetdash{}{0pt}%
\pgfpathmoveto{\pgfqpoint{3.180064in}{2.650187in}}%
\pgfpathlineto{\pgfqpoint{3.193495in}{2.636939in}}%
\pgfpathlineto{\pgfqpoint{3.206922in}{2.623962in}}%
\pgfpathlineto{\pgfqpoint{3.220347in}{2.611255in}}%
\pgfpathlineto{\pgfqpoint{3.233769in}{2.598815in}}%
\pgfpathlineto{\pgfqpoint{3.241798in}{2.609851in}}%
\pgfpathlineto{\pgfqpoint{3.249821in}{2.620985in}}%
\pgfpathlineto{\pgfqpoint{3.257837in}{2.632220in}}%
\pgfpathlineto{\pgfqpoint{3.265847in}{2.643557in}}%
\pgfpathlineto{\pgfqpoint{3.252438in}{2.656061in}}%
\pgfpathlineto{\pgfqpoint{3.239027in}{2.668832in}}%
\pgfpathlineto{\pgfqpoint{3.225612in}{2.681873in}}%
\pgfpathlineto{\pgfqpoint{3.212194in}{2.695185in}}%
\pgfpathlineto{\pgfqpoint{3.204172in}{2.683773in}}%
\pgfpathlineto{\pgfqpoint{3.196142in}{2.672470in}}%
\pgfpathlineto{\pgfqpoint{3.188106in}{2.661275in}}%
\pgfpathlineto{\pgfqpoint{3.180064in}{2.650187in}}%
\pgfpathclose%
\pgfusepath{fill}%
\end{pgfscope}%
\begin{pgfscope}%
\pgfpathrectangle{\pgfqpoint{1.150000in}{0.150000in}}{\pgfqpoint{5.700000in}{5.700000in}}%
\pgfusepath{clip}%
\pgfsetbuttcap%
\pgfsetroundjoin%
\definecolor{currentfill}{rgb}{0.271828,0.209303,0.504434}%
\pgfsetfillcolor{currentfill}%
\pgfsetfillopacity{0.800000}%
\pgfsetlinewidth{0.000000pt}%
\definecolor{currentstroke}{rgb}{0.000000,0.000000,0.000000}%
\pgfsetstrokecolor{currentstroke}%
\pgfsetdash{}{0pt}%
\pgfpathmoveto{\pgfqpoint{4.183157in}{2.618844in}}%
\pgfpathlineto{\pgfqpoint{4.196652in}{2.617051in}}%
\pgfpathlineto{\pgfqpoint{4.210155in}{2.615456in}}%
\pgfpathlineto{\pgfqpoint{4.223665in}{2.614059in}}%
\pgfpathlineto{\pgfqpoint{4.237182in}{2.612860in}}%
\pgfpathlineto{\pgfqpoint{4.244913in}{2.623789in}}%
\pgfpathlineto{\pgfqpoint{4.252639in}{2.634785in}}%
\pgfpathlineto{\pgfqpoint{4.260361in}{2.645853in}}%
\pgfpathlineto{\pgfqpoint{4.268079in}{2.656996in}}%
\pgfpathlineto{\pgfqpoint{4.254570in}{2.658512in}}%
\pgfpathlineto{\pgfqpoint{4.241069in}{2.660225in}}%
\pgfpathlineto{\pgfqpoint{4.227575in}{2.662137in}}%
\pgfpathlineto{\pgfqpoint{4.214089in}{2.664247in}}%
\pgfpathlineto{\pgfqpoint{4.206363in}{2.652776in}}%
\pgfpathlineto{\pgfqpoint{4.198632in}{2.641388in}}%
\pgfpathlineto{\pgfqpoint{4.190897in}{2.630079in}}%
\pgfpathlineto{\pgfqpoint{4.183157in}{2.618844in}}%
\pgfpathclose%
\pgfusepath{fill}%
\end{pgfscope}%
\begin{pgfscope}%
\pgfpathrectangle{\pgfqpoint{1.150000in}{0.150000in}}{\pgfqpoint{5.700000in}{5.700000in}}%
\pgfusepath{clip}%
\pgfsetbuttcap%
\pgfsetroundjoin%
\definecolor{currentfill}{rgb}{0.281412,0.155834,0.469201}%
\pgfsetfillcolor{currentfill}%
\pgfsetfillopacity{0.800000}%
\pgfsetlinewidth{0.000000pt}%
\definecolor{currentstroke}{rgb}{0.000000,0.000000,0.000000}%
\pgfsetstrokecolor{currentstroke}%
\pgfsetdash{}{0pt}%
\pgfpathmoveto{\pgfqpoint{3.789470in}{2.506389in}}%
\pgfpathlineto{\pgfqpoint{3.802880in}{2.501394in}}%
\pgfpathlineto{\pgfqpoint{3.816296in}{2.496615in}}%
\pgfpathlineto{\pgfqpoint{3.829715in}{2.492053in}}%
\pgfpathlineto{\pgfqpoint{3.843139in}{2.487704in}}%
\pgfpathlineto{\pgfqpoint{3.850988in}{2.498902in}}%
\pgfpathlineto{\pgfqpoint{3.858831in}{2.510156in}}%
\pgfpathlineto{\pgfqpoint{3.866669in}{2.521470in}}%
\pgfpathlineto{\pgfqpoint{3.874503in}{2.532846in}}%
\pgfpathlineto{\pgfqpoint{3.861087in}{2.537385in}}%
\pgfpathlineto{\pgfqpoint{3.847676in}{2.542139in}}%
\pgfpathlineto{\pgfqpoint{3.834269in}{2.547108in}}%
\pgfpathlineto{\pgfqpoint{3.820867in}{2.552294in}}%
\pgfpathlineto{\pgfqpoint{3.813025in}{2.540715in}}%
\pgfpathlineto{\pgfqpoint{3.805178in}{2.529207in}}%
\pgfpathlineto{\pgfqpoint{3.797326in}{2.517765in}}%
\pgfpathlineto{\pgfqpoint{3.789470in}{2.506389in}}%
\pgfpathclose%
\pgfusepath{fill}%
\end{pgfscope}%
\begin{pgfscope}%
\pgfpathrectangle{\pgfqpoint{1.150000in}{0.150000in}}{\pgfqpoint{5.700000in}{5.700000in}}%
\pgfusepath{clip}%
\pgfsetbuttcap%
\pgfsetroundjoin%
\definecolor{currentfill}{rgb}{0.220057,0.343307,0.549413}%
\pgfsetfillcolor{currentfill}%
\pgfsetfillopacity{0.800000}%
\pgfsetlinewidth{0.000000pt}%
\definecolor{currentstroke}{rgb}{0.000000,0.000000,0.000000}%
\pgfsetstrokecolor{currentstroke}%
\pgfsetdash{}{0pt}%
\pgfpathmoveto{\pgfqpoint{2.910359in}{2.975687in}}%
\pgfpathlineto{\pgfqpoint{2.923910in}{2.956512in}}%
\pgfpathlineto{\pgfqpoint{2.937453in}{2.937660in}}%
\pgfpathlineto{\pgfqpoint{2.950987in}{2.919126in}}%
\pgfpathlineto{\pgfqpoint{2.964514in}{2.900908in}}%
\pgfpathlineto{\pgfqpoint{2.972615in}{2.912202in}}%
\pgfpathlineto{\pgfqpoint{2.980709in}{2.923636in}}%
\pgfpathlineto{\pgfqpoint{2.988795in}{2.935209in}}%
\pgfpathlineto{\pgfqpoint{2.996874in}{2.946924in}}%
\pgfpathlineto{\pgfqpoint{2.983363in}{2.965205in}}%
\pgfpathlineto{\pgfqpoint{2.969845in}{2.983802in}}%
\pgfpathlineto{\pgfqpoint{2.956318in}{3.002719in}}%
\pgfpathlineto{\pgfqpoint{2.942782in}{3.021956in}}%
\pgfpathlineto{\pgfqpoint{2.934689in}{3.010167in}}%
\pgfpathlineto{\pgfqpoint{2.926587in}{2.998526in}}%
\pgfpathlineto{\pgfqpoint{2.918477in}{2.987033in}}%
\pgfpathlineto{\pgfqpoint{2.910359in}{2.975687in}}%
\pgfpathclose%
\pgfusepath{fill}%
\end{pgfscope}%
\begin{pgfscope}%
\pgfpathrectangle{\pgfqpoint{1.150000in}{0.150000in}}{\pgfqpoint{5.700000in}{5.700000in}}%
\pgfusepath{clip}%
\pgfsetbuttcap%
\pgfsetroundjoin%
\definecolor{currentfill}{rgb}{0.188923,0.410910,0.556326}%
\pgfsetfillcolor{currentfill}%
\pgfsetfillopacity{0.800000}%
\pgfsetlinewidth{0.000000pt}%
\definecolor{currentstroke}{rgb}{0.000000,0.000000,0.000000}%
\pgfsetstrokecolor{currentstroke}%
\pgfsetdash{}{0pt}%
\pgfpathmoveto{\pgfqpoint{5.117547in}{3.117606in}}%
\pgfpathlineto{\pgfqpoint{5.131340in}{3.118780in}}%
\pgfpathlineto{\pgfqpoint{5.145146in}{3.120129in}}%
\pgfpathlineto{\pgfqpoint{5.158963in}{3.121654in}}%
\pgfpathlineto{\pgfqpoint{5.172794in}{3.123354in}}%
\pgfpathlineto{\pgfqpoint{5.180256in}{3.134808in}}%
\pgfpathlineto{\pgfqpoint{5.187720in}{3.146493in}}%
\pgfpathlineto{\pgfqpoint{5.195184in}{3.158414in}}%
\pgfpathlineto{\pgfqpoint{5.202649in}{3.170581in}}%
\pgfpathlineto{\pgfqpoint{5.188839in}{3.169547in}}%
\pgfpathlineto{\pgfqpoint{5.175041in}{3.168688in}}%
\pgfpathlineto{\pgfqpoint{5.161255in}{3.168004in}}%
\pgfpathlineto{\pgfqpoint{5.147481in}{3.167495in}}%
\pgfpathlineto{\pgfqpoint{5.139996in}{3.154651in}}%
\pgfpathlineto{\pgfqpoint{5.132512in}{3.142060in}}%
\pgfpathlineto{\pgfqpoint{5.125029in}{3.129714in}}%
\pgfpathlineto{\pgfqpoint{5.117547in}{3.117606in}}%
\pgfpathclose%
\pgfusepath{fill}%
\end{pgfscope}%
\begin{pgfscope}%
\pgfpathrectangle{\pgfqpoint{1.150000in}{0.150000in}}{\pgfqpoint{5.700000in}{5.700000in}}%
\pgfusepath{clip}%
\pgfsetbuttcap%
\pgfsetroundjoin%
\definecolor{currentfill}{rgb}{0.275191,0.194905,0.496005}%
\pgfsetfillcolor{currentfill}%
\pgfsetfillopacity{0.800000}%
\pgfsetlinewidth{0.000000pt}%
\definecolor{currentstroke}{rgb}{0.000000,0.000000,0.000000}%
\pgfsetstrokecolor{currentstroke}%
\pgfsetdash{}{0pt}%
\pgfpathmoveto{\pgfqpoint{4.098213in}{2.582625in}}%
\pgfpathlineto{\pgfqpoint{4.111688in}{2.580313in}}%
\pgfpathlineto{\pgfqpoint{4.125170in}{2.578204in}}%
\pgfpathlineto{\pgfqpoint{4.138658in}{2.576295in}}%
\pgfpathlineto{\pgfqpoint{4.152154in}{2.574587in}}%
\pgfpathlineto{\pgfqpoint{4.159912in}{2.585557in}}%
\pgfpathlineto{\pgfqpoint{4.167665in}{2.596587in}}%
\pgfpathlineto{\pgfqpoint{4.175413in}{2.607682in}}%
\pgfpathlineto{\pgfqpoint{4.183157in}{2.618844in}}%
\pgfpathlineto{\pgfqpoint{4.169670in}{2.620838in}}%
\pgfpathlineto{\pgfqpoint{4.156190in}{2.623032in}}%
\pgfpathlineto{\pgfqpoint{4.142716in}{2.625427in}}%
\pgfpathlineto{\pgfqpoint{4.129250in}{2.628023in}}%
\pgfpathlineto{\pgfqpoint{4.121498in}{2.616564in}}%
\pgfpathlineto{\pgfqpoint{4.113741in}{2.605180in}}%
\pgfpathlineto{\pgfqpoint{4.105979in}{2.593868in}}%
\pgfpathlineto{\pgfqpoint{4.098213in}{2.582625in}}%
\pgfpathclose%
\pgfusepath{fill}%
\end{pgfscope}%
\begin{pgfscope}%
\pgfpathrectangle{\pgfqpoint{1.150000in}{0.150000in}}{\pgfqpoint{5.700000in}{5.700000in}}%
\pgfusepath{clip}%
\pgfsetbuttcap%
\pgfsetroundjoin%
\definecolor{currentfill}{rgb}{0.281887,0.150881,0.465405}%
\pgfsetfillcolor{currentfill}%
\pgfsetfillopacity{0.800000}%
\pgfsetlinewidth{0.000000pt}%
\definecolor{currentstroke}{rgb}{0.000000,0.000000,0.000000}%
\pgfsetstrokecolor{currentstroke}%
\pgfsetdash{}{0pt}%
\pgfpathmoveto{\pgfqpoint{3.565554in}{2.491238in}}%
\pgfpathlineto{\pgfqpoint{3.578946in}{2.483841in}}%
\pgfpathlineto{\pgfqpoint{3.592340in}{2.476675in}}%
\pgfpathlineto{\pgfqpoint{3.605737in}{2.469738in}}%
\pgfpathlineto{\pgfqpoint{3.619135in}{2.463030in}}%
\pgfpathlineto{\pgfqpoint{3.627052in}{2.474200in}}%
\pgfpathlineto{\pgfqpoint{3.634962in}{2.485434in}}%
\pgfpathlineto{\pgfqpoint{3.642868in}{2.496733in}}%
\pgfpathlineto{\pgfqpoint{3.650768in}{2.508100in}}%
\pgfpathlineto{\pgfqpoint{3.637379in}{2.514936in}}%
\pgfpathlineto{\pgfqpoint{3.623992in}{2.522000in}}%
\pgfpathlineto{\pgfqpoint{3.610607in}{2.529294in}}%
\pgfpathlineto{\pgfqpoint{3.597224in}{2.536819in}}%
\pgfpathlineto{\pgfqpoint{3.589315in}{2.525313in}}%
\pgfpathlineto{\pgfqpoint{3.581400in}{2.513882in}}%
\pgfpathlineto{\pgfqpoint{3.573480in}{2.502524in}}%
\pgfpathlineto{\pgfqpoint{3.565554in}{2.491238in}}%
\pgfpathclose%
\pgfusepath{fill}%
\end{pgfscope}%
\begin{pgfscope}%
\pgfpathrectangle{\pgfqpoint{1.150000in}{0.150000in}}{\pgfqpoint{5.700000in}{5.700000in}}%
\pgfusepath{clip}%
\pgfsetbuttcap%
\pgfsetroundjoin%
\definecolor{currentfill}{rgb}{0.280868,0.160771,0.472899}%
\pgfsetfillcolor{currentfill}%
\pgfsetfillopacity{0.800000}%
\pgfsetlinewidth{0.000000pt}%
\definecolor{currentstroke}{rgb}{0.000000,0.000000,0.000000}%
\pgfsetstrokecolor{currentstroke}%
\pgfsetdash{}{0pt}%
\pgfpathmoveto{\pgfqpoint{3.426625in}{2.513659in}}%
\pgfpathlineto{\pgfqpoint{3.440019in}{2.504460in}}%
\pgfpathlineto{\pgfqpoint{3.453413in}{2.495504in}}%
\pgfpathlineto{\pgfqpoint{3.466808in}{2.486788in}}%
\pgfpathlineto{\pgfqpoint{3.480203in}{2.478312in}}%
\pgfpathlineto{\pgfqpoint{3.488161in}{2.489414in}}%
\pgfpathlineto{\pgfqpoint{3.496114in}{2.500589in}}%
\pgfpathlineto{\pgfqpoint{3.504060in}{2.511838in}}%
\pgfpathlineto{\pgfqpoint{3.512002in}{2.523163in}}%
\pgfpathlineto{\pgfqpoint{3.498617in}{2.531735in}}%
\pgfpathlineto{\pgfqpoint{3.485232in}{2.540547in}}%
\pgfpathlineto{\pgfqpoint{3.471849in}{2.549600in}}%
\pgfpathlineto{\pgfqpoint{3.458466in}{2.558895in}}%
\pgfpathlineto{\pgfqpoint{3.450514in}{2.547462in}}%
\pgfpathlineto{\pgfqpoint{3.442557in}{2.536113in}}%
\pgfpathlineto{\pgfqpoint{3.434594in}{2.524846in}}%
\pgfpathlineto{\pgfqpoint{3.426625in}{2.513659in}}%
\pgfpathclose%
\pgfusepath{fill}%
\end{pgfscope}%
\begin{pgfscope}%
\pgfpathrectangle{\pgfqpoint{1.150000in}{0.150000in}}{\pgfqpoint{5.700000in}{5.700000in}}%
\pgfusepath{clip}%
\pgfsetbuttcap%
\pgfsetroundjoin%
\definecolor{currentfill}{rgb}{0.275191,0.194905,0.496005}%
\pgfsetfillcolor{currentfill}%
\pgfsetfillopacity{0.800000}%
\pgfsetlinewidth{0.000000pt}%
\definecolor{currentstroke}{rgb}{0.000000,0.000000,0.000000}%
\pgfsetstrokecolor{currentstroke}%
\pgfsetdash{}{0pt}%
\pgfpathmoveto{\pgfqpoint{3.233769in}{2.598815in}}%
\pgfpathlineto{\pgfqpoint{3.247188in}{2.586640in}}%
\pgfpathlineto{\pgfqpoint{3.260604in}{2.574729in}}%
\pgfpathlineto{\pgfqpoint{3.274019in}{2.563079in}}%
\pgfpathlineto{\pgfqpoint{3.287432in}{2.551688in}}%
\pgfpathlineto{\pgfqpoint{3.295448in}{2.562671in}}%
\pgfpathlineto{\pgfqpoint{3.303459in}{2.573746in}}%
\pgfpathlineto{\pgfqpoint{3.311463in}{2.584912in}}%
\pgfpathlineto{\pgfqpoint{3.319460in}{2.596172in}}%
\pgfpathlineto{\pgfqpoint{3.306060in}{2.607627in}}%
\pgfpathlineto{\pgfqpoint{3.292658in}{2.619342in}}%
\pgfpathlineto{\pgfqpoint{3.279254in}{2.631318in}}%
\pgfpathlineto{\pgfqpoint{3.265847in}{2.643557in}}%
\pgfpathlineto{\pgfqpoint{3.257837in}{2.632220in}}%
\pgfpathlineto{\pgfqpoint{3.249821in}{2.620985in}}%
\pgfpathlineto{\pgfqpoint{3.241798in}{2.609851in}}%
\pgfpathlineto{\pgfqpoint{3.233769in}{2.598815in}}%
\pgfpathclose%
\pgfusepath{fill}%
\end{pgfscope}%
\begin{pgfscope}%
\pgfpathrectangle{\pgfqpoint{1.150000in}{0.150000in}}{\pgfqpoint{5.700000in}{5.700000in}}%
\pgfusepath{clip}%
\pgfsetbuttcap%
\pgfsetroundjoin%
\definecolor{currentfill}{rgb}{0.180629,0.429975,0.557282}%
\pgfsetfillcolor{currentfill}%
\pgfsetfillopacity{0.800000}%
\pgfsetlinewidth{0.000000pt}%
\definecolor{currentstroke}{rgb}{0.000000,0.000000,0.000000}%
\pgfsetstrokecolor{currentstroke}%
\pgfsetdash{}{0pt}%
\pgfpathmoveto{\pgfqpoint{5.202649in}{3.170581in}}%
\pgfpathlineto{\pgfqpoint{5.216471in}{3.171790in}}%
\pgfpathlineto{\pgfqpoint{5.230306in}{3.173173in}}%
\pgfpathlineto{\pgfqpoint{5.244153in}{3.174730in}}%
\pgfpathlineto{\pgfqpoint{5.258013in}{3.176460in}}%
\pgfpathlineto{\pgfqpoint{5.265458in}{3.188194in}}%
\pgfpathlineto{\pgfqpoint{5.272906in}{3.200181in}}%
\pgfpathlineto{\pgfqpoint{5.280354in}{3.212429in}}%
\pgfpathlineto{\pgfqpoint{5.287806in}{3.224946in}}%
\pgfpathlineto{\pgfqpoint{5.273967in}{3.223913in}}%
\pgfpathlineto{\pgfqpoint{5.260141in}{3.223053in}}%
\pgfpathlineto{\pgfqpoint{5.246328in}{3.222367in}}%
\pgfpathlineto{\pgfqpoint{5.232526in}{3.221855in}}%
\pgfpathlineto{\pgfqpoint{5.225054in}{3.208630in}}%
\pgfpathlineto{\pgfqpoint{5.217584in}{3.195681in}}%
\pgfpathlineto{\pgfqpoint{5.210115in}{3.183001in}}%
\pgfpathlineto{\pgfqpoint{5.202649in}{3.170581in}}%
\pgfpathclose%
\pgfusepath{fill}%
\end{pgfscope}%
\begin{pgfscope}%
\pgfpathrectangle{\pgfqpoint{1.150000in}{0.150000in}}{\pgfqpoint{5.700000in}{5.700000in}}%
\pgfusepath{clip}%
\pgfsetbuttcap%
\pgfsetroundjoin%
\definecolor{currentfill}{rgb}{0.206756,0.371758,0.553117}%
\pgfsetfillcolor{currentfill}%
\pgfsetfillopacity{0.800000}%
\pgfsetlinewidth{0.000000pt}%
\definecolor{currentstroke}{rgb}{0.000000,0.000000,0.000000}%
\pgfsetstrokecolor{currentstroke}%
\pgfsetdash{}{0pt}%
\pgfpathmoveto{\pgfqpoint{2.856060in}{3.055666in}}%
\pgfpathlineto{\pgfqpoint{2.869650in}{3.035173in}}%
\pgfpathlineto{\pgfqpoint{2.883229in}{3.015014in}}%
\pgfpathlineto{\pgfqpoint{2.896798in}{2.995187in}}%
\pgfpathlineto{\pgfqpoint{2.910359in}{2.975687in}}%
\pgfpathlineto{\pgfqpoint{2.918477in}{2.987033in}}%
\pgfpathlineto{\pgfqpoint{2.926587in}{2.998526in}}%
\pgfpathlineto{\pgfqpoint{2.934689in}{3.010167in}}%
\pgfpathlineto{\pgfqpoint{2.942782in}{3.021956in}}%
\pgfpathlineto{\pgfqpoint{2.929238in}{3.041519in}}%
\pgfpathlineto{\pgfqpoint{2.915685in}{3.061410in}}%
\pgfpathlineto{\pgfqpoint{2.902122in}{3.081632in}}%
\pgfpathlineto{\pgfqpoint{2.888550in}{3.102188in}}%
\pgfpathlineto{\pgfqpoint{2.880440in}{3.090323in}}%
\pgfpathlineto{\pgfqpoint{2.872322in}{3.078615in}}%
\pgfpathlineto{\pgfqpoint{2.864195in}{3.067063in}}%
\pgfpathlineto{\pgfqpoint{2.856060in}{3.055666in}}%
\pgfpathclose%
\pgfusepath{fill}%
\end{pgfscope}%
\begin{pgfscope}%
\pgfpathrectangle{\pgfqpoint{1.150000in}{0.150000in}}{\pgfqpoint{5.700000in}{5.700000in}}%
\pgfusepath{clip}%
\pgfsetbuttcap%
\pgfsetroundjoin%
\definecolor{currentfill}{rgb}{0.278012,0.180367,0.486697}%
\pgfsetfillcolor{currentfill}%
\pgfsetfillopacity{0.800000}%
\pgfsetlinewidth{0.000000pt}%
\definecolor{currentstroke}{rgb}{0.000000,0.000000,0.000000}%
\pgfsetstrokecolor{currentstroke}%
\pgfsetdash{}{0pt}%
\pgfpathmoveto{\pgfqpoint{4.013236in}{2.548540in}}%
\pgfpathlineto{\pgfqpoint{4.026693in}{2.545666in}}%
\pgfpathlineto{\pgfqpoint{4.040156in}{2.542998in}}%
\pgfpathlineto{\pgfqpoint{4.053626in}{2.540534in}}%
\pgfpathlineto{\pgfqpoint{4.067103in}{2.538273in}}%
\pgfpathlineto{\pgfqpoint{4.074887in}{2.549275in}}%
\pgfpathlineto{\pgfqpoint{4.082667in}{2.560331in}}%
\pgfpathlineto{\pgfqpoint{4.090442in}{2.571447in}}%
\pgfpathlineto{\pgfqpoint{4.098213in}{2.582625in}}%
\pgfpathlineto{\pgfqpoint{4.084745in}{2.585140in}}%
\pgfpathlineto{\pgfqpoint{4.071284in}{2.587858in}}%
\pgfpathlineto{\pgfqpoint{4.057829in}{2.590780in}}%
\pgfpathlineto{\pgfqpoint{4.044380in}{2.593907in}}%
\pgfpathlineto{\pgfqpoint{4.036601in}{2.582464in}}%
\pgfpathlineto{\pgfqpoint{4.028817in}{2.571091in}}%
\pgfpathlineto{\pgfqpoint{4.021029in}{2.559783in}}%
\pgfpathlineto{\pgfqpoint{4.013236in}{2.548540in}}%
\pgfpathclose%
\pgfusepath{fill}%
\end{pgfscope}%
\begin{pgfscope}%
\pgfpathrectangle{\pgfqpoint{1.150000in}{0.150000in}}{\pgfqpoint{5.700000in}{5.700000in}}%
\pgfusepath{clip}%
\pgfsetbuttcap%
\pgfsetroundjoin%
\definecolor{currentfill}{rgb}{0.281887,0.150881,0.465405}%
\pgfsetfillcolor{currentfill}%
\pgfsetfillopacity{0.800000}%
\pgfsetlinewidth{0.000000pt}%
\definecolor{currentstroke}{rgb}{0.000000,0.000000,0.000000}%
\pgfsetstrokecolor{currentstroke}%
\pgfsetdash{}{0pt}%
\pgfpathmoveto{\pgfqpoint{3.704353in}{2.483014in}}%
\pgfpathlineto{\pgfqpoint{3.717757in}{2.477301in}}%
\pgfpathlineto{\pgfqpoint{3.731165in}{2.471809in}}%
\pgfpathlineto{\pgfqpoint{3.744577in}{2.466537in}}%
\pgfpathlineto{\pgfqpoint{3.757992in}{2.461484in}}%
\pgfpathlineto{\pgfqpoint{3.765869in}{2.472624in}}%
\pgfpathlineto{\pgfqpoint{3.773741in}{2.483820in}}%
\pgfpathlineto{\pgfqpoint{3.781608in}{2.495074in}}%
\pgfpathlineto{\pgfqpoint{3.789470in}{2.506389in}}%
\pgfpathlineto{\pgfqpoint{3.776063in}{2.511601in}}%
\pgfpathlineto{\pgfqpoint{3.762660in}{2.517033in}}%
\pgfpathlineto{\pgfqpoint{3.749261in}{2.522684in}}%
\pgfpathlineto{\pgfqpoint{3.735866in}{2.528557in}}%
\pgfpathlineto{\pgfqpoint{3.727995in}{2.517071in}}%
\pgfpathlineto{\pgfqpoint{3.720120in}{2.505654in}}%
\pgfpathlineto{\pgfqpoint{3.712239in}{2.494302in}}%
\pgfpathlineto{\pgfqpoint{3.704353in}{2.483014in}}%
\pgfpathclose%
\pgfusepath{fill}%
\end{pgfscope}%
\begin{pgfscope}%
\pgfpathrectangle{\pgfqpoint{1.150000in}{0.150000in}}{\pgfqpoint{5.700000in}{5.700000in}}%
\pgfusepath{clip}%
\pgfsetbuttcap%
\pgfsetroundjoin%
\definecolor{currentfill}{rgb}{0.172719,0.448791,0.557885}%
\pgfsetfillcolor{currentfill}%
\pgfsetfillopacity{0.800000}%
\pgfsetlinewidth{0.000000pt}%
\definecolor{currentstroke}{rgb}{0.000000,0.000000,0.000000}%
\pgfsetstrokecolor{currentstroke}%
\pgfsetdash{}{0pt}%
\pgfpathmoveto{\pgfqpoint{5.287806in}{3.224946in}}%
\pgfpathlineto{\pgfqpoint{5.301656in}{3.226152in}}%
\pgfpathlineto{\pgfqpoint{5.315520in}{3.227531in}}%
\pgfpathlineto{\pgfqpoint{5.329396in}{3.229082in}}%
\pgfpathlineto{\pgfqpoint{5.343285in}{3.230806in}}%
\pgfpathlineto{\pgfqpoint{5.350716in}{3.242882in}}%
\pgfpathlineto{\pgfqpoint{5.358150in}{3.255236in}}%
\pgfpathlineto{\pgfqpoint{5.365586in}{3.267876in}}%
\pgfpathlineto{\pgfqpoint{5.373026in}{3.280811in}}%
\pgfpathlineto{\pgfqpoint{5.359160in}{3.279817in}}%
\pgfpathlineto{\pgfqpoint{5.345307in}{3.278994in}}%
\pgfpathlineto{\pgfqpoint{5.331466in}{3.278344in}}%
\pgfpathlineto{\pgfqpoint{5.317637in}{3.277866in}}%
\pgfpathlineto{\pgfqpoint{5.310175in}{3.264192in}}%
\pgfpathlineto{\pgfqpoint{5.302715in}{3.250819in}}%
\pgfpathlineto{\pgfqpoint{5.295259in}{3.237740in}}%
\pgfpathlineto{\pgfqpoint{5.287806in}{3.224946in}}%
\pgfpathclose%
\pgfusepath{fill}%
\end{pgfscope}%
\begin{pgfscope}%
\pgfpathrectangle{\pgfqpoint{1.150000in}{0.150000in}}{\pgfqpoint{5.700000in}{5.700000in}}%
\pgfusepath{clip}%
\pgfsetbuttcap%
\pgfsetroundjoin%
\definecolor{currentfill}{rgb}{0.278826,0.175490,0.483397}%
\pgfsetfillcolor{currentfill}%
\pgfsetfillopacity{0.800000}%
\pgfsetlinewidth{0.000000pt}%
\definecolor{currentstroke}{rgb}{0.000000,0.000000,0.000000}%
\pgfsetstrokecolor{currentstroke}%
\pgfsetdash{}{0pt}%
\pgfpathmoveto{\pgfqpoint{3.287432in}{2.551688in}}%
\pgfpathlineto{\pgfqpoint{3.300843in}{2.540555in}}%
\pgfpathlineto{\pgfqpoint{3.314252in}{2.529679in}}%
\pgfpathlineto{\pgfqpoint{3.327660in}{2.519056in}}%
\pgfpathlineto{\pgfqpoint{3.341068in}{2.508685in}}%
\pgfpathlineto{\pgfqpoint{3.349072in}{2.519615in}}%
\pgfpathlineto{\pgfqpoint{3.357070in}{2.530629in}}%
\pgfpathlineto{\pgfqpoint{3.365062in}{2.541727in}}%
\pgfpathlineto{\pgfqpoint{3.373048in}{2.552911in}}%
\pgfpathlineto{\pgfqpoint{3.359653in}{2.563346in}}%
\pgfpathlineto{\pgfqpoint{3.346257in}{2.574034in}}%
\pgfpathlineto{\pgfqpoint{3.332859in}{2.584975in}}%
\pgfpathlineto{\pgfqpoint{3.319460in}{2.596172in}}%
\pgfpathlineto{\pgfqpoint{3.311463in}{2.584912in}}%
\pgfpathlineto{\pgfqpoint{3.303459in}{2.573746in}}%
\pgfpathlineto{\pgfqpoint{3.295448in}{2.562671in}}%
\pgfpathlineto{\pgfqpoint{3.287432in}{2.551688in}}%
\pgfpathclose%
\pgfusepath{fill}%
\end{pgfscope}%
\begin{pgfscope}%
\pgfpathrectangle{\pgfqpoint{1.150000in}{0.150000in}}{\pgfqpoint{5.700000in}{5.700000in}}%
\pgfusepath{clip}%
\pgfsetbuttcap%
\pgfsetroundjoin%
\definecolor{currentfill}{rgb}{0.279574,0.170599,0.479997}%
\pgfsetfillcolor{currentfill}%
\pgfsetfillopacity{0.800000}%
\pgfsetlinewidth{0.000000pt}%
\definecolor{currentstroke}{rgb}{0.000000,0.000000,0.000000}%
\pgfsetstrokecolor{currentstroke}%
\pgfsetdash{}{0pt}%
\pgfpathmoveto{\pgfqpoint{3.928216in}{2.516816in}}%
\pgfpathlineto{\pgfqpoint{3.941657in}{2.513335in}}%
\pgfpathlineto{\pgfqpoint{3.955105in}{2.510062in}}%
\pgfpathlineto{\pgfqpoint{3.968558in}{2.506998in}}%
\pgfpathlineto{\pgfqpoint{3.982017in}{2.504141in}}%
\pgfpathlineto{\pgfqpoint{3.989829in}{2.515160in}}%
\pgfpathlineto{\pgfqpoint{3.997636in}{2.526231in}}%
\pgfpathlineto{\pgfqpoint{4.005439in}{2.537357in}}%
\pgfpathlineto{\pgfqpoint{4.013236in}{2.548540in}}%
\pgfpathlineto{\pgfqpoint{3.999785in}{2.551620in}}%
\pgfpathlineto{\pgfqpoint{3.986341in}{2.554907in}}%
\pgfpathlineto{\pgfqpoint{3.972902in}{2.558401in}}%
\pgfpathlineto{\pgfqpoint{3.959468in}{2.562105in}}%
\pgfpathlineto{\pgfqpoint{3.951662in}{2.550687in}}%
\pgfpathlineto{\pgfqpoint{3.943852in}{2.539335in}}%
\pgfpathlineto{\pgfqpoint{3.936036in}{2.528046in}}%
\pgfpathlineto{\pgfqpoint{3.928216in}{2.516816in}}%
\pgfpathclose%
\pgfusepath{fill}%
\end{pgfscope}%
\begin{pgfscope}%
\pgfpathrectangle{\pgfqpoint{1.150000in}{0.150000in}}{\pgfqpoint{5.700000in}{5.700000in}}%
\pgfusepath{clip}%
\pgfsetbuttcap%
\pgfsetroundjoin%
\definecolor{currentfill}{rgb}{0.190631,0.407061,0.556089}%
\pgfsetfillcolor{currentfill}%
\pgfsetfillopacity{0.800000}%
\pgfsetlinewidth{0.000000pt}%
\definecolor{currentstroke}{rgb}{0.000000,0.000000,0.000000}%
\pgfsetstrokecolor{currentstroke}%
\pgfsetdash{}{0pt}%
\pgfpathmoveto{\pgfqpoint{2.801598in}{3.141049in}}%
\pgfpathlineto{\pgfqpoint{2.815230in}{3.119185in}}%
\pgfpathlineto{\pgfqpoint{2.828851in}{3.097669in}}%
\pgfpathlineto{\pgfqpoint{2.842461in}{3.076497in}}%
\pgfpathlineto{\pgfqpoint{2.856060in}{3.055666in}}%
\pgfpathlineto{\pgfqpoint{2.864195in}{3.067063in}}%
\pgfpathlineto{\pgfqpoint{2.872322in}{3.078615in}}%
\pgfpathlineto{\pgfqpoint{2.880440in}{3.090323in}}%
\pgfpathlineto{\pgfqpoint{2.888550in}{3.102188in}}%
\pgfpathlineto{\pgfqpoint{2.874967in}{3.123082in}}%
\pgfpathlineto{\pgfqpoint{2.861375in}{3.144317in}}%
\pgfpathlineto{\pgfqpoint{2.847771in}{3.165896in}}%
\pgfpathlineto{\pgfqpoint{2.834156in}{3.187823in}}%
\pgfpathlineto{\pgfqpoint{2.826030in}{3.175883in}}%
\pgfpathlineto{\pgfqpoint{2.817894in}{3.164108in}}%
\pgfpathlineto{\pgfqpoint{2.809751in}{3.152497in}}%
\pgfpathlineto{\pgfqpoint{2.801598in}{3.141049in}}%
\pgfpathclose%
\pgfusepath{fill}%
\end{pgfscope}%
\begin{pgfscope}%
\pgfpathrectangle{\pgfqpoint{1.150000in}{0.150000in}}{\pgfqpoint{5.700000in}{5.700000in}}%
\pgfusepath{clip}%
\pgfsetbuttcap%
\pgfsetroundjoin%
\definecolor{currentfill}{rgb}{0.165117,0.467423,0.558141}%
\pgfsetfillcolor{currentfill}%
\pgfsetfillopacity{0.800000}%
\pgfsetlinewidth{0.000000pt}%
\definecolor{currentstroke}{rgb}{0.000000,0.000000,0.000000}%
\pgfsetstrokecolor{currentstroke}%
\pgfsetdash{}{0pt}%
\pgfpathmoveto{\pgfqpoint{5.373026in}{3.280811in}}%
\pgfpathlineto{\pgfqpoint{5.386905in}{3.281977in}}%
\pgfpathlineto{\pgfqpoint{5.400797in}{3.283315in}}%
\pgfpathlineto{\pgfqpoint{5.414701in}{3.284824in}}%
\pgfpathlineto{\pgfqpoint{5.428619in}{3.286504in}}%
\pgfpathlineto{\pgfqpoint{5.436038in}{3.298992in}}%
\pgfpathlineto{\pgfqpoint{5.443462in}{3.311784in}}%
\pgfpathlineto{\pgfqpoint{5.450889in}{3.324888in}}%
\pgfpathlineto{\pgfqpoint{5.436990in}{3.323776in}}%
\pgfpathlineto{\pgfqpoint{5.423104in}{3.322835in}}%
\pgfpathlineto{\pgfqpoint{5.409230in}{3.322065in}}%
\pgfpathlineto{\pgfqpoint{5.395369in}{3.321466in}}%
\pgfpathlineto{\pgfqpoint{5.387917in}{3.307597in}}%
\pgfpathlineto{\pgfqpoint{5.380470in}{3.294048in}}%
\pgfpathlineto{\pgfqpoint{5.373026in}{3.280811in}}%
\pgfpathclose%
\pgfusepath{fill}%
\end{pgfscope}%
\begin{pgfscope}%
\pgfpathrectangle{\pgfqpoint{1.150000in}{0.150000in}}{\pgfqpoint{5.700000in}{5.700000in}}%
\pgfusepath{clip}%
\pgfsetbuttcap%
\pgfsetroundjoin%
\definecolor{currentfill}{rgb}{0.281887,0.150881,0.465405}%
\pgfsetfillcolor{currentfill}%
\pgfsetfillopacity{0.800000}%
\pgfsetlinewidth{0.000000pt}%
\definecolor{currentstroke}{rgb}{0.000000,0.000000,0.000000}%
\pgfsetstrokecolor{currentstroke}%
\pgfsetdash{}{0pt}%
\pgfpathmoveto{\pgfqpoint{3.480203in}{2.478312in}}%
\pgfpathlineto{\pgfqpoint{3.493600in}{2.470074in}}%
\pgfpathlineto{\pgfqpoint{3.506997in}{2.462073in}}%
\pgfpathlineto{\pgfqpoint{3.520396in}{2.454306in}}%
\pgfpathlineto{\pgfqpoint{3.533797in}{2.446773in}}%
\pgfpathlineto{\pgfqpoint{3.541744in}{2.457790in}}%
\pgfpathlineto{\pgfqpoint{3.549686in}{2.468873in}}%
\pgfpathlineto{\pgfqpoint{3.557623in}{2.480021in}}%
\pgfpathlineto{\pgfqpoint{3.565554in}{2.491238in}}%
\pgfpathlineto{\pgfqpoint{3.552164in}{2.498867in}}%
\pgfpathlineto{\pgfqpoint{3.538775in}{2.506730in}}%
\pgfpathlineto{\pgfqpoint{3.525388in}{2.514828in}}%
\pgfpathlineto{\pgfqpoint{3.512002in}{2.523163in}}%
\pgfpathlineto{\pgfqpoint{3.504060in}{2.511838in}}%
\pgfpathlineto{\pgfqpoint{3.496114in}{2.500589in}}%
\pgfpathlineto{\pgfqpoint{3.488161in}{2.489414in}}%
\pgfpathlineto{\pgfqpoint{3.480203in}{2.478312in}}%
\pgfpathclose%
\pgfusepath{fill}%
\end{pgfscope}%
\begin{pgfscope}%
\pgfpathrectangle{\pgfqpoint{1.150000in}{0.150000in}}{\pgfqpoint{5.700000in}{5.700000in}}%
\pgfusepath{clip}%
\pgfsetbuttcap%
\pgfsetroundjoin%
\definecolor{currentfill}{rgb}{0.282290,0.145912,0.461510}%
\pgfsetfillcolor{currentfill}%
\pgfsetfillopacity{0.800000}%
\pgfsetlinewidth{0.000000pt}%
\definecolor{currentstroke}{rgb}{0.000000,0.000000,0.000000}%
\pgfsetstrokecolor{currentstroke}%
\pgfsetdash{}{0pt}%
\pgfpathmoveto{\pgfqpoint{3.619135in}{2.463030in}}%
\pgfpathlineto{\pgfqpoint{3.632537in}{2.456549in}}%
\pgfpathlineto{\pgfqpoint{3.645941in}{2.450294in}}%
\pgfpathlineto{\pgfqpoint{3.659348in}{2.444264in}}%
\pgfpathlineto{\pgfqpoint{3.672758in}{2.438457in}}%
\pgfpathlineto{\pgfqpoint{3.680665in}{2.449510in}}%
\pgfpathlineto{\pgfqpoint{3.688566in}{2.460620in}}%
\pgfpathlineto{\pgfqpoint{3.696462in}{2.471787in}}%
\pgfpathlineto{\pgfqpoint{3.704353in}{2.483014in}}%
\pgfpathlineto{\pgfqpoint{3.690952in}{2.488949in}}%
\pgfpathlineto{\pgfqpoint{3.677554in}{2.495108in}}%
\pgfpathlineto{\pgfqpoint{3.664160in}{2.501491in}}%
\pgfpathlineto{\pgfqpoint{3.650768in}{2.508100in}}%
\pgfpathlineto{\pgfqpoint{3.642868in}{2.496733in}}%
\pgfpathlineto{\pgfqpoint{3.634962in}{2.485434in}}%
\pgfpathlineto{\pgfqpoint{3.627052in}{2.474200in}}%
\pgfpathlineto{\pgfqpoint{3.619135in}{2.463030in}}%
\pgfpathclose%
\pgfusepath{fill}%
\end{pgfscope}%
\begin{pgfscope}%
\pgfpathrectangle{\pgfqpoint{1.150000in}{0.150000in}}{\pgfqpoint{5.700000in}{5.700000in}}%
\pgfusepath{clip}%
\pgfsetbuttcap%
\pgfsetroundjoin%
\definecolor{currentfill}{rgb}{0.281412,0.155834,0.469201}%
\pgfsetfillcolor{currentfill}%
\pgfsetfillopacity{0.800000}%
\pgfsetlinewidth{0.000000pt}%
\definecolor{currentstroke}{rgb}{0.000000,0.000000,0.000000}%
\pgfsetstrokecolor{currentstroke}%
\pgfsetdash{}{0pt}%
\pgfpathmoveto{\pgfqpoint{3.843139in}{2.487704in}}%
\pgfpathlineto{\pgfqpoint{3.856568in}{2.483570in}}%
\pgfpathlineto{\pgfqpoint{3.870002in}{2.479648in}}%
\pgfpathlineto{\pgfqpoint{3.883441in}{2.475938in}}%
\pgfpathlineto{\pgfqpoint{3.896886in}{2.472438in}}%
\pgfpathlineto{\pgfqpoint{3.904726in}{2.483456in}}%
\pgfpathlineto{\pgfqpoint{3.912561in}{2.494523in}}%
\pgfpathlineto{\pgfqpoint{3.920391in}{2.505642in}}%
\pgfpathlineto{\pgfqpoint{3.928216in}{2.516816in}}%
\pgfpathlineto{\pgfqpoint{3.914780in}{2.520506in}}%
\pgfpathlineto{\pgfqpoint{3.901349in}{2.524408in}}%
\pgfpathlineto{\pgfqpoint{3.887923in}{2.528521in}}%
\pgfpathlineto{\pgfqpoint{3.874503in}{2.532846in}}%
\pgfpathlineto{\pgfqpoint{3.866669in}{2.521470in}}%
\pgfpathlineto{\pgfqpoint{3.858831in}{2.510156in}}%
\pgfpathlineto{\pgfqpoint{3.850988in}{2.498902in}}%
\pgfpathlineto{\pgfqpoint{3.843139in}{2.487704in}}%
\pgfpathclose%
\pgfusepath{fill}%
\end{pgfscope}%
\begin{pgfscope}%
\pgfpathrectangle{\pgfqpoint{1.150000in}{0.150000in}}{\pgfqpoint{5.700000in}{5.700000in}}%
\pgfusepath{clip}%
\pgfsetbuttcap%
\pgfsetroundjoin%
\definecolor{currentfill}{rgb}{0.244972,0.287675,0.537260}%
\pgfsetfillcolor{currentfill}%
\pgfsetfillopacity{0.800000}%
\pgfsetlinewidth{0.000000pt}%
\definecolor{currentstroke}{rgb}{0.000000,0.000000,0.000000}%
\pgfsetstrokecolor{currentstroke}%
\pgfsetdash{}{0pt}%
\pgfpathmoveto{\pgfqpoint{4.577211in}{2.781914in}}%
\pgfpathlineto{\pgfqpoint{4.590842in}{2.782516in}}%
\pgfpathlineto{\pgfqpoint{4.604484in}{2.783304in}}%
\pgfpathlineto{\pgfqpoint{4.618135in}{2.784279in}}%
\pgfpathlineto{\pgfqpoint{4.631797in}{2.785439in}}%
\pgfpathlineto{\pgfqpoint{4.639411in}{2.795793in}}%
\pgfpathlineto{\pgfqpoint{4.647020in}{2.806247in}}%
\pgfpathlineto{\pgfqpoint{4.654626in}{2.816806in}}%
\pgfpathlineto{\pgfqpoint{4.662228in}{2.827474in}}%
\pgfpathlineto{\pgfqpoint{4.648579in}{2.826759in}}%
\pgfpathlineto{\pgfqpoint{4.634939in}{2.826229in}}%
\pgfpathlineto{\pgfqpoint{4.621310in}{2.825885in}}%
\pgfpathlineto{\pgfqpoint{4.607690in}{2.825727in}}%
\pgfpathlineto{\pgfqpoint{4.600076in}{2.814603in}}%
\pgfpathlineto{\pgfqpoint{4.592458in}{2.803596in}}%
\pgfpathlineto{\pgfqpoint{4.584836in}{2.792701in}}%
\pgfpathlineto{\pgfqpoint{4.577211in}{2.781914in}}%
\pgfpathclose%
\pgfusepath{fill}%
\end{pgfscope}%
\begin{pgfscope}%
\pgfpathrectangle{\pgfqpoint{1.150000in}{0.150000in}}{\pgfqpoint{5.700000in}{5.700000in}}%
\pgfusepath{clip}%
\pgfsetbuttcap%
\pgfsetroundjoin%
\definecolor{currentfill}{rgb}{0.252194,0.269783,0.531579}%
\pgfsetfillcolor{currentfill}%
\pgfsetfillopacity{0.800000}%
\pgfsetlinewidth{0.000000pt}%
\definecolor{currentstroke}{rgb}{0.000000,0.000000,0.000000}%
\pgfsetstrokecolor{currentstroke}%
\pgfsetdash{}{0pt}%
\pgfpathmoveto{\pgfqpoint{4.492204in}{2.737557in}}%
\pgfpathlineto{\pgfqpoint{4.505806in}{2.737820in}}%
\pgfpathlineto{\pgfqpoint{4.519419in}{2.738271in}}%
\pgfpathlineto{\pgfqpoint{4.533042in}{2.738911in}}%
\pgfpathlineto{\pgfqpoint{4.546674in}{2.739738in}}%
\pgfpathlineto{\pgfqpoint{4.554314in}{2.750146in}}%
\pgfpathlineto{\pgfqpoint{4.561950in}{2.760641in}}%
\pgfpathlineto{\pgfqpoint{4.569583in}{2.771229in}}%
\pgfpathlineto{\pgfqpoint{4.577211in}{2.781914in}}%
\pgfpathlineto{\pgfqpoint{4.563590in}{2.781499in}}%
\pgfpathlineto{\pgfqpoint{4.549979in}{2.781273in}}%
\pgfpathlineto{\pgfqpoint{4.536377in}{2.781234in}}%
\pgfpathlineto{\pgfqpoint{4.522785in}{2.781384in}}%
\pgfpathlineto{\pgfqpoint{4.515145in}{2.770274in}}%
\pgfpathlineto{\pgfqpoint{4.507502in}{2.759270in}}%
\pgfpathlineto{\pgfqpoint{4.499855in}{2.748366in}}%
\pgfpathlineto{\pgfqpoint{4.492204in}{2.737557in}}%
\pgfpathclose%
\pgfusepath{fill}%
\end{pgfscope}%
\begin{pgfscope}%
\pgfpathrectangle{\pgfqpoint{1.150000in}{0.150000in}}{\pgfqpoint{5.700000in}{5.700000in}}%
\pgfusepath{clip}%
\pgfsetbuttcap%
\pgfsetroundjoin%
\definecolor{currentfill}{rgb}{0.237441,0.305202,0.541921}%
\pgfsetfillcolor{currentfill}%
\pgfsetfillopacity{0.800000}%
\pgfsetlinewidth{0.000000pt}%
\definecolor{currentstroke}{rgb}{0.000000,0.000000,0.000000}%
\pgfsetstrokecolor{currentstroke}%
\pgfsetdash{}{0pt}%
\pgfpathmoveto{\pgfqpoint{4.662228in}{2.827474in}}%
\pgfpathlineto{\pgfqpoint{4.675889in}{2.828375in}}%
\pgfpathlineto{\pgfqpoint{4.689560in}{2.829461in}}%
\pgfpathlineto{\pgfqpoint{4.703241in}{2.830730in}}%
\pgfpathlineto{\pgfqpoint{4.716934in}{2.832184in}}%
\pgfpathlineto{\pgfqpoint{4.724520in}{2.842503in}}%
\pgfpathlineto{\pgfqpoint{4.732103in}{2.852935in}}%
\pgfpathlineto{\pgfqpoint{4.739683in}{2.863485in}}%
\pgfpathlineto{\pgfqpoint{4.747260in}{2.874160in}}%
\pgfpathlineto{\pgfqpoint{4.733581in}{2.873183in}}%
\pgfpathlineto{\pgfqpoint{4.719912in}{2.872390in}}%
\pgfpathlineto{\pgfqpoint{4.706254in}{2.871780in}}%
\pgfpathlineto{\pgfqpoint{4.692607in}{2.871355in}}%
\pgfpathlineto{\pgfqpoint{4.685017in}{2.860193in}}%
\pgfpathlineto{\pgfqpoint{4.677424in}{2.849163in}}%
\pgfpathlineto{\pgfqpoint{4.669828in}{2.838258in}}%
\pgfpathlineto{\pgfqpoint{4.662228in}{2.827474in}}%
\pgfpathclose%
\pgfusepath{fill}%
\end{pgfscope}%
\begin{pgfscope}%
\pgfpathrectangle{\pgfqpoint{1.150000in}{0.150000in}}{\pgfqpoint{5.700000in}{5.700000in}}%
\pgfusepath{clip}%
\pgfsetbuttcap%
\pgfsetroundjoin%
\definecolor{currentfill}{rgb}{0.227802,0.326594,0.546532}%
\pgfsetfillcolor{currentfill}%
\pgfsetfillopacity{0.800000}%
\pgfsetlinewidth{0.000000pt}%
\definecolor{currentstroke}{rgb}{0.000000,0.000000,0.000000}%
\pgfsetstrokecolor{currentstroke}%
\pgfsetdash{}{0pt}%
\pgfpathmoveto{\pgfqpoint{4.747260in}{2.874160in}}%
\pgfpathlineto{\pgfqpoint{4.760950in}{2.875320in}}%
\pgfpathlineto{\pgfqpoint{4.774651in}{2.876662in}}%
\pgfpathlineto{\pgfqpoint{4.788363in}{2.878188in}}%
\pgfpathlineto{\pgfqpoint{4.802087in}{2.879895in}}%
\pgfpathlineto{\pgfqpoint{4.809647in}{2.890202in}}%
\pgfpathlineto{\pgfqpoint{4.817204in}{2.900638in}}%
\pgfpathlineto{\pgfqpoint{4.824758in}{2.911206in}}%
\pgfpathlineto{\pgfqpoint{4.832310in}{2.921914in}}%
\pgfpathlineto{\pgfqpoint{4.818601in}{2.920716in}}%
\pgfpathlineto{\pgfqpoint{4.804903in}{2.919699in}}%
\pgfpathlineto{\pgfqpoint{4.791216in}{2.918864in}}%
\pgfpathlineto{\pgfqpoint{4.777540in}{2.918211in}}%
\pgfpathlineto{\pgfqpoint{4.769974in}{2.906984in}}%
\pgfpathlineto{\pgfqpoint{4.762405in}{2.895903in}}%
\pgfpathlineto{\pgfqpoint{4.754834in}{2.884964in}}%
\pgfpathlineto{\pgfqpoint{4.747260in}{2.874160in}}%
\pgfpathclose%
\pgfusepath{fill}%
\end{pgfscope}%
\begin{pgfscope}%
\pgfpathrectangle{\pgfqpoint{1.150000in}{0.150000in}}{\pgfqpoint{5.700000in}{5.700000in}}%
\pgfusepath{clip}%
\pgfsetbuttcap%
\pgfsetroundjoin%
\definecolor{currentfill}{rgb}{0.258965,0.251537,0.524736}%
\pgfsetfillcolor{currentfill}%
\pgfsetfillopacity{0.800000}%
\pgfsetlinewidth{0.000000pt}%
\definecolor{currentstroke}{rgb}{0.000000,0.000000,0.000000}%
\pgfsetstrokecolor{currentstroke}%
\pgfsetdash{}{0pt}%
\pgfpathmoveto{\pgfqpoint{4.407200in}{2.694508in}}%
\pgfpathlineto{\pgfqpoint{4.420776in}{2.694390in}}%
\pgfpathlineto{\pgfqpoint{4.434361in}{2.694464in}}%
\pgfpathlineto{\pgfqpoint{4.447955in}{2.694728in}}%
\pgfpathlineto{\pgfqpoint{4.461559in}{2.695182in}}%
\pgfpathlineto{\pgfqpoint{4.469226in}{2.705656in}}%
\pgfpathlineto{\pgfqpoint{4.476889in}{2.716207in}}%
\pgfpathlineto{\pgfqpoint{4.484548in}{2.726839in}}%
\pgfpathlineto{\pgfqpoint{4.492204in}{2.737557in}}%
\pgfpathlineto{\pgfqpoint{4.478610in}{2.737484in}}%
\pgfpathlineto{\pgfqpoint{4.465026in}{2.737601in}}%
\pgfpathlineto{\pgfqpoint{4.451451in}{2.737908in}}%
\pgfpathlineto{\pgfqpoint{4.437886in}{2.738406in}}%
\pgfpathlineto{\pgfqpoint{4.430220in}{2.727296in}}%
\pgfpathlineto{\pgfqpoint{4.422551in}{2.716279in}}%
\pgfpathlineto{\pgfqpoint{4.414878in}{2.705351in}}%
\pgfpathlineto{\pgfqpoint{4.407200in}{2.694508in}}%
\pgfpathclose%
\pgfusepath{fill}%
\end{pgfscope}%
\begin{pgfscope}%
\pgfpathrectangle{\pgfqpoint{1.150000in}{0.150000in}}{\pgfqpoint{5.700000in}{5.700000in}}%
\pgfusepath{clip}%
\pgfsetbuttcap%
\pgfsetroundjoin%
\definecolor{currentfill}{rgb}{0.258965,0.251537,0.524736}%
\pgfsetfillcolor{currentfill}%
\pgfsetfillopacity{0.800000}%
\pgfsetlinewidth{0.000000pt}%
\definecolor{currentstroke}{rgb}{0.000000,0.000000,0.000000}%
\pgfsetstrokecolor{currentstroke}%
\pgfsetdash{}{0pt}%
\pgfpathmoveto{\pgfqpoint{3.040111in}{2.722643in}}%
\pgfpathlineto{\pgfqpoint{3.053592in}{2.707177in}}%
\pgfpathlineto{\pgfqpoint{3.067069in}{2.692000in}}%
\pgfpathlineto{\pgfqpoint{3.080540in}{2.677110in}}%
\pgfpathlineto{\pgfqpoint{3.094006in}{2.662505in}}%
\pgfpathlineto{\pgfqpoint{3.102090in}{2.673195in}}%
\pgfpathlineto{\pgfqpoint{3.110168in}{2.683996in}}%
\pgfpathlineto{\pgfqpoint{3.118238in}{2.694909in}}%
\pgfpathlineto{\pgfqpoint{3.126302in}{2.705936in}}%
\pgfpathlineto{\pgfqpoint{3.112851in}{2.720573in}}%
\pgfpathlineto{\pgfqpoint{3.099395in}{2.735495in}}%
\pgfpathlineto{\pgfqpoint{3.085934in}{2.750704in}}%
\pgfpathlineto{\pgfqpoint{3.072468in}{2.766202in}}%
\pgfpathlineto{\pgfqpoint{3.064389in}{2.755132in}}%
\pgfpathlineto{\pgfqpoint{3.056304in}{2.744183in}}%
\pgfpathlineto{\pgfqpoint{3.048211in}{2.733353in}}%
\pgfpathlineto{\pgfqpoint{3.040111in}{2.722643in}}%
\pgfpathclose%
\pgfusepath{fill}%
\end{pgfscope}%
\begin{pgfscope}%
\pgfpathrectangle{\pgfqpoint{1.150000in}{0.150000in}}{\pgfqpoint{5.700000in}{5.700000in}}%
\pgfusepath{clip}%
\pgfsetbuttcap%
\pgfsetroundjoin%
\definecolor{currentfill}{rgb}{0.250425,0.274290,0.533103}%
\pgfsetfillcolor{currentfill}%
\pgfsetfillopacity{0.800000}%
\pgfsetlinewidth{0.000000pt}%
\definecolor{currentstroke}{rgb}{0.000000,0.000000,0.000000}%
\pgfsetstrokecolor{currentstroke}%
\pgfsetdash{}{0pt}%
\pgfpathmoveto{\pgfqpoint{2.986124in}{2.787452in}}%
\pgfpathlineto{\pgfqpoint{2.999630in}{2.770803in}}%
\pgfpathlineto{\pgfqpoint{3.013130in}{2.754454in}}%
\pgfpathlineto{\pgfqpoint{3.026623in}{2.738401in}}%
\pgfpathlineto{\pgfqpoint{3.040111in}{2.722643in}}%
\pgfpathlineto{\pgfqpoint{3.048211in}{2.733353in}}%
\pgfpathlineto{\pgfqpoint{3.056304in}{2.744183in}}%
\pgfpathlineto{\pgfqpoint{3.064389in}{2.755132in}}%
\pgfpathlineto{\pgfqpoint{3.072468in}{2.766202in}}%
\pgfpathlineto{\pgfqpoint{3.058996in}{2.781992in}}%
\pgfpathlineto{\pgfqpoint{3.045518in}{2.798076in}}%
\pgfpathlineto{\pgfqpoint{3.032034in}{2.814457in}}%
\pgfpathlineto{\pgfqpoint{3.018544in}{2.831138in}}%
\pgfpathlineto{\pgfqpoint{3.010450in}{2.820024in}}%
\pgfpathlineto{\pgfqpoint{3.002349in}{2.809039in}}%
\pgfpathlineto{\pgfqpoint{2.994241in}{2.798182in}}%
\pgfpathlineto{\pgfqpoint{2.986124in}{2.787452in}}%
\pgfpathclose%
\pgfusepath{fill}%
\end{pgfscope}%
\begin{pgfscope}%
\pgfpathrectangle{\pgfqpoint{1.150000in}{0.150000in}}{\pgfqpoint{5.700000in}{5.700000in}}%
\pgfusepath{clip}%
\pgfsetbuttcap%
\pgfsetroundjoin%
\definecolor{currentfill}{rgb}{0.280868,0.160771,0.472899}%
\pgfsetfillcolor{currentfill}%
\pgfsetfillopacity{0.800000}%
\pgfsetlinewidth{0.000000pt}%
\definecolor{currentstroke}{rgb}{0.000000,0.000000,0.000000}%
\pgfsetstrokecolor{currentstroke}%
\pgfsetdash{}{0pt}%
\pgfpathmoveto{\pgfqpoint{3.341068in}{2.508685in}}%
\pgfpathlineto{\pgfqpoint{3.354474in}{2.498565in}}%
\pgfpathlineto{\pgfqpoint{3.367880in}{2.488694in}}%
\pgfpathlineto{\pgfqpoint{3.381285in}{2.479071in}}%
\pgfpathlineto{\pgfqpoint{3.394691in}{2.469693in}}%
\pgfpathlineto{\pgfqpoint{3.402683in}{2.480570in}}%
\pgfpathlineto{\pgfqpoint{3.410669in}{2.491522in}}%
\pgfpathlineto{\pgfqpoint{3.418650in}{2.502552in}}%
\pgfpathlineto{\pgfqpoint{3.426625in}{2.513659in}}%
\pgfpathlineto{\pgfqpoint{3.413231in}{2.523102in}}%
\pgfpathlineto{\pgfqpoint{3.399837in}{2.532790in}}%
\pgfpathlineto{\pgfqpoint{3.386443in}{2.542726in}}%
\pgfpathlineto{\pgfqpoint{3.373048in}{2.552911in}}%
\pgfpathlineto{\pgfqpoint{3.365062in}{2.541727in}}%
\pgfpathlineto{\pgfqpoint{3.357070in}{2.530629in}}%
\pgfpathlineto{\pgfqpoint{3.349072in}{2.519615in}}%
\pgfpathlineto{\pgfqpoint{3.341068in}{2.508685in}}%
\pgfpathclose%
\pgfusepath{fill}%
\end{pgfscope}%
\begin{pgfscope}%
\pgfpathrectangle{\pgfqpoint{1.150000in}{0.150000in}}{\pgfqpoint{5.700000in}{5.700000in}}%
\pgfusepath{clip}%
\pgfsetbuttcap%
\pgfsetroundjoin%
\definecolor{currentfill}{rgb}{0.220057,0.343307,0.549413}%
\pgfsetfillcolor{currentfill}%
\pgfsetfillopacity{0.800000}%
\pgfsetlinewidth{0.000000pt}%
\definecolor{currentstroke}{rgb}{0.000000,0.000000,0.000000}%
\pgfsetstrokecolor{currentstroke}%
\pgfsetdash{}{0pt}%
\pgfpathmoveto{\pgfqpoint{4.832310in}{2.921914in}}%
\pgfpathlineto{\pgfqpoint{4.846030in}{2.923294in}}%
\pgfpathlineto{\pgfqpoint{4.859762in}{2.924855in}}%
\pgfpathlineto{\pgfqpoint{4.873505in}{2.926597in}}%
\pgfpathlineto{\pgfqpoint{4.887260in}{2.928519in}}%
\pgfpathlineto{\pgfqpoint{4.894794in}{2.938844in}}%
\pgfpathlineto{\pgfqpoint{4.902326in}{2.949313in}}%
\pgfpathlineto{\pgfqpoint{4.909856in}{2.959932in}}%
\pgfpathlineto{\pgfqpoint{4.917383in}{2.970707in}}%
\pgfpathlineto{\pgfqpoint{4.903644in}{2.969325in}}%
\pgfpathlineto{\pgfqpoint{4.889916in}{2.968124in}}%
\pgfpathlineto{\pgfqpoint{4.876200in}{2.967102in}}%
\pgfpathlineto{\pgfqpoint{4.862494in}{2.966262in}}%
\pgfpathlineto{\pgfqpoint{4.854951in}{2.954935in}}%
\pgfpathlineto{\pgfqpoint{4.847406in}{2.943773in}}%
\pgfpathlineto{\pgfqpoint{4.839859in}{2.932768in}}%
\pgfpathlineto{\pgfqpoint{4.832310in}{2.921914in}}%
\pgfpathclose%
\pgfusepath{fill}%
\end{pgfscope}%
\begin{pgfscope}%
\pgfpathrectangle{\pgfqpoint{1.150000in}{0.150000in}}{\pgfqpoint{5.700000in}{5.700000in}}%
\pgfusepath{clip}%
\pgfsetbuttcap%
\pgfsetroundjoin%
\definecolor{currentfill}{rgb}{0.265145,0.232956,0.516599}%
\pgfsetfillcolor{currentfill}%
\pgfsetfillopacity{0.800000}%
\pgfsetlinewidth{0.000000pt}%
\definecolor{currentstroke}{rgb}{0.000000,0.000000,0.000000}%
\pgfsetstrokecolor{currentstroke}%
\pgfsetdash{}{0pt}%
\pgfpathmoveto{\pgfqpoint{4.322195in}{2.652891in}}%
\pgfpathlineto{\pgfqpoint{4.335745in}{2.652353in}}%
\pgfpathlineto{\pgfqpoint{4.349304in}{2.652007in}}%
\pgfpathlineto{\pgfqpoint{4.362872in}{2.651855in}}%
\pgfpathlineto{\pgfqpoint{4.376448in}{2.651896in}}%
\pgfpathlineto{\pgfqpoint{4.384143in}{2.662443in}}%
\pgfpathlineto{\pgfqpoint{4.391833in}{2.673058in}}%
\pgfpathlineto{\pgfqpoint{4.399518in}{2.683745in}}%
\pgfpathlineto{\pgfqpoint{4.407200in}{2.694508in}}%
\pgfpathlineto{\pgfqpoint{4.393633in}{2.694817in}}%
\pgfpathlineto{\pgfqpoint{4.380075in}{2.695318in}}%
\pgfpathlineto{\pgfqpoint{4.366526in}{2.696013in}}%
\pgfpathlineto{\pgfqpoint{4.352986in}{2.696901in}}%
\pgfpathlineto{\pgfqpoint{4.345294in}{2.685777in}}%
\pgfpathlineto{\pgfqpoint{4.337599in}{2.674737in}}%
\pgfpathlineto{\pgfqpoint{4.329899in}{2.663776in}}%
\pgfpathlineto{\pgfqpoint{4.322195in}{2.652891in}}%
\pgfpathclose%
\pgfusepath{fill}%
\end{pgfscope}%
\begin{pgfscope}%
\pgfpathrectangle{\pgfqpoint{1.150000in}{0.150000in}}{\pgfqpoint{5.700000in}{5.700000in}}%
\pgfusepath{clip}%
\pgfsetbuttcap%
\pgfsetroundjoin%
\definecolor{currentfill}{rgb}{0.266580,0.228262,0.514349}%
\pgfsetfillcolor{currentfill}%
\pgfsetfillopacity{0.800000}%
\pgfsetlinewidth{0.000000pt}%
\definecolor{currentstroke}{rgb}{0.000000,0.000000,0.000000}%
\pgfsetstrokecolor{currentstroke}%
\pgfsetdash{}{0pt}%
\pgfpathmoveto{\pgfqpoint{3.094006in}{2.662505in}}%
\pgfpathlineto{\pgfqpoint{3.107467in}{2.648182in}}%
\pgfpathlineto{\pgfqpoint{3.120924in}{2.634139in}}%
\pgfpathlineto{\pgfqpoint{3.134376in}{2.620374in}}%
\pgfpathlineto{\pgfqpoint{3.147825in}{2.606885in}}%
\pgfpathlineto{\pgfqpoint{3.155895in}{2.617555in}}%
\pgfpathlineto{\pgfqpoint{3.163958in}{2.628328in}}%
\pgfpathlineto{\pgfqpoint{3.172014in}{2.639205in}}%
\pgfpathlineto{\pgfqpoint{3.180064in}{2.650187in}}%
\pgfpathlineto{\pgfqpoint{3.166629in}{2.663709in}}%
\pgfpathlineto{\pgfqpoint{3.153191in}{2.677506in}}%
\pgfpathlineto{\pgfqpoint{3.139749in}{2.691581in}}%
\pgfpathlineto{\pgfqpoint{3.126302in}{2.705936in}}%
\pgfpathlineto{\pgfqpoint{3.118238in}{2.694909in}}%
\pgfpathlineto{\pgfqpoint{3.110168in}{2.683996in}}%
\pgfpathlineto{\pgfqpoint{3.102090in}{2.673195in}}%
\pgfpathlineto{\pgfqpoint{3.094006in}{2.662505in}}%
\pgfpathclose%
\pgfusepath{fill}%
\end{pgfscope}%
\begin{pgfscope}%
\pgfpathrectangle{\pgfqpoint{1.150000in}{0.150000in}}{\pgfqpoint{5.700000in}{5.700000in}}%
\pgfusepath{clip}%
\pgfsetbuttcap%
\pgfsetroundjoin%
\definecolor{currentfill}{rgb}{0.210503,0.363727,0.552206}%
\pgfsetfillcolor{currentfill}%
\pgfsetfillopacity{0.800000}%
\pgfsetlinewidth{0.000000pt}%
\definecolor{currentstroke}{rgb}{0.000000,0.000000,0.000000}%
\pgfsetstrokecolor{currentstroke}%
\pgfsetdash{}{0pt}%
\pgfpathmoveto{\pgfqpoint{4.917383in}{2.970707in}}%
\pgfpathlineto{\pgfqpoint{4.931134in}{2.972268in}}%
\pgfpathlineto{\pgfqpoint{4.944897in}{2.974008in}}%
\pgfpathlineto{\pgfqpoint{4.958672in}{2.975928in}}%
\pgfpathlineto{\pgfqpoint{4.972458in}{2.978026in}}%
\pgfpathlineto{\pgfqpoint{4.979967in}{2.988403in}}%
\pgfpathlineto{\pgfqpoint{4.987475in}{2.998942in}}%
\pgfpathlineto{\pgfqpoint{4.994981in}{3.009648in}}%
\pgfpathlineto{\pgfqpoint{5.002485in}{3.020529in}}%
\pgfpathlineto{\pgfqpoint{4.988715in}{3.019004in}}%
\pgfpathlineto{\pgfqpoint{4.974957in}{3.017656in}}%
\pgfpathlineto{\pgfqpoint{4.961211in}{3.016487in}}%
\pgfpathlineto{\pgfqpoint{4.947477in}{3.015497in}}%
\pgfpathlineto{\pgfqpoint{4.939956in}{3.004033in}}%
\pgfpathlineto{\pgfqpoint{4.932433in}{2.992751in}}%
\pgfpathlineto{\pgfqpoint{4.924909in}{2.981644in}}%
\pgfpathlineto{\pgfqpoint{4.917383in}{2.970707in}}%
\pgfpathclose%
\pgfusepath{fill}%
\end{pgfscope}%
\begin{pgfscope}%
\pgfpathrectangle{\pgfqpoint{1.150000in}{0.150000in}}{\pgfqpoint{5.700000in}{5.700000in}}%
\pgfusepath{clip}%
\pgfsetbuttcap%
\pgfsetroundjoin%
\definecolor{currentfill}{rgb}{0.237441,0.305202,0.541921}%
\pgfsetfillcolor{currentfill}%
\pgfsetfillopacity{0.800000}%
\pgfsetlinewidth{0.000000pt}%
\definecolor{currentstroke}{rgb}{0.000000,0.000000,0.000000}%
\pgfsetstrokecolor{currentstroke}%
\pgfsetdash{}{0pt}%
\pgfpathmoveto{\pgfqpoint{2.932029in}{2.857097in}}%
\pgfpathlineto{\pgfqpoint{2.945564in}{2.839223in}}%
\pgfpathlineto{\pgfqpoint{2.959091in}{2.821659in}}%
\pgfpathlineto{\pgfqpoint{2.972611in}{2.804403in}}%
\pgfpathlineto{\pgfqpoint{2.986124in}{2.787452in}}%
\pgfpathlineto{\pgfqpoint{2.994241in}{2.798182in}}%
\pgfpathlineto{\pgfqpoint{3.002349in}{2.809039in}}%
\pgfpathlineto{\pgfqpoint{3.010450in}{2.820024in}}%
\pgfpathlineto{\pgfqpoint{3.018544in}{2.831138in}}%
\pgfpathlineto{\pgfqpoint{3.005047in}{2.848121in}}%
\pgfpathlineto{\pgfqpoint{2.991543in}{2.865408in}}%
\pgfpathlineto{\pgfqpoint{2.978032in}{2.883003in}}%
\pgfpathlineto{\pgfqpoint{2.964514in}{2.900908in}}%
\pgfpathlineto{\pgfqpoint{2.956404in}{2.889750in}}%
\pgfpathlineto{\pgfqpoint{2.948287in}{2.878730in}}%
\pgfpathlineto{\pgfqpoint{2.940162in}{2.867846in}}%
\pgfpathlineto{\pgfqpoint{2.932029in}{2.857097in}}%
\pgfpathclose%
\pgfusepath{fill}%
\end{pgfscope}%
\begin{pgfscope}%
\pgfpathrectangle{\pgfqpoint{1.150000in}{0.150000in}}{\pgfqpoint{5.700000in}{5.700000in}}%
\pgfusepath{clip}%
\pgfsetbuttcap%
\pgfsetroundjoin%
\definecolor{currentfill}{rgb}{0.269308,0.218818,0.509577}%
\pgfsetfillcolor{currentfill}%
\pgfsetfillopacity{0.800000}%
\pgfsetlinewidth{0.000000pt}%
\definecolor{currentstroke}{rgb}{0.000000,0.000000,0.000000}%
\pgfsetstrokecolor{currentstroke}%
\pgfsetdash{}{0pt}%
\pgfpathmoveto{\pgfqpoint{4.237182in}{2.612860in}}%
\pgfpathlineto{\pgfqpoint{4.250708in}{2.611858in}}%
\pgfpathlineto{\pgfqpoint{4.264242in}{2.611052in}}%
\pgfpathlineto{\pgfqpoint{4.277785in}{2.610442in}}%
\pgfpathlineto{\pgfqpoint{4.291335in}{2.610027in}}%
\pgfpathlineto{\pgfqpoint{4.299057in}{2.620649in}}%
\pgfpathlineto{\pgfqpoint{4.306774in}{2.631332in}}%
\pgfpathlineto{\pgfqpoint{4.314487in}{2.642078in}}%
\pgfpathlineto{\pgfqpoint{4.322195in}{2.652891in}}%
\pgfpathlineto{\pgfqpoint{4.308654in}{2.653625in}}%
\pgfpathlineto{\pgfqpoint{4.295121in}{2.654553in}}%
\pgfpathlineto{\pgfqpoint{4.281596in}{2.655676in}}%
\pgfpathlineto{\pgfqpoint{4.268079in}{2.656996in}}%
\pgfpathlineto{\pgfqpoint{4.260361in}{2.645853in}}%
\pgfpathlineto{\pgfqpoint{4.252639in}{2.634785in}}%
\pgfpathlineto{\pgfqpoint{4.244913in}{2.623789in}}%
\pgfpathlineto{\pgfqpoint{4.237182in}{2.612860in}}%
\pgfpathclose%
\pgfusepath{fill}%
\end{pgfscope}%
\begin{pgfscope}%
\pgfpathrectangle{\pgfqpoint{1.150000in}{0.150000in}}{\pgfqpoint{5.700000in}{5.700000in}}%
\pgfusepath{clip}%
\pgfsetbuttcap%
\pgfsetroundjoin%
\definecolor{currentfill}{rgb}{0.203063,0.379716,0.553925}%
\pgfsetfillcolor{currentfill}%
\pgfsetfillopacity{0.800000}%
\pgfsetlinewidth{0.000000pt}%
\definecolor{currentstroke}{rgb}{0.000000,0.000000,0.000000}%
\pgfsetstrokecolor{currentstroke}%
\pgfsetdash{}{0pt}%
\pgfpathmoveto{\pgfqpoint{5.002485in}{3.020529in}}%
\pgfpathlineto{\pgfqpoint{5.016267in}{3.022233in}}%
\pgfpathlineto{\pgfqpoint{5.030060in}{3.024115in}}%
\pgfpathlineto{\pgfqpoint{5.043866in}{3.026174in}}%
\pgfpathlineto{\pgfqpoint{5.057685in}{3.028410in}}%
\pgfpathlineto{\pgfqpoint{5.065170in}{3.038880in}}%
\pgfpathlineto{\pgfqpoint{5.072654in}{3.049530in}}%
\pgfpathlineto{\pgfqpoint{5.080137in}{3.060367in}}%
\pgfpathlineto{\pgfqpoint{5.087620in}{3.071398in}}%
\pgfpathlineto{\pgfqpoint{5.073820in}{3.069766in}}%
\pgfpathlineto{\pgfqpoint{5.060032in}{3.068311in}}%
\pgfpathlineto{\pgfqpoint{5.046256in}{3.067033in}}%
\pgfpathlineto{\pgfqpoint{5.032492in}{3.065932in}}%
\pgfpathlineto{\pgfqpoint{5.024991in}{3.054286in}}%
\pgfpathlineto{\pgfqpoint{5.017490in}{3.042841in}}%
\pgfpathlineto{\pgfqpoint{5.009988in}{3.031592in}}%
\pgfpathlineto{\pgfqpoint{5.002485in}{3.020529in}}%
\pgfpathclose%
\pgfusepath{fill}%
\end{pgfscope}%
\begin{pgfscope}%
\pgfpathrectangle{\pgfqpoint{1.150000in}{0.150000in}}{\pgfqpoint{5.700000in}{5.700000in}}%
\pgfusepath{clip}%
\pgfsetbuttcap%
\pgfsetroundjoin%
\definecolor{currentfill}{rgb}{0.273006,0.204520,0.501721}%
\pgfsetfillcolor{currentfill}%
\pgfsetfillopacity{0.800000}%
\pgfsetlinewidth{0.000000pt}%
\definecolor{currentstroke}{rgb}{0.000000,0.000000,0.000000}%
\pgfsetstrokecolor{currentstroke}%
\pgfsetdash{}{0pt}%
\pgfpathmoveto{\pgfqpoint{3.147825in}{2.606885in}}%
\pgfpathlineto{\pgfqpoint{3.161270in}{2.593670in}}%
\pgfpathlineto{\pgfqpoint{3.174712in}{2.580726in}}%
\pgfpathlineto{\pgfqpoint{3.188150in}{2.568051in}}%
\pgfpathlineto{\pgfqpoint{3.201586in}{2.555644in}}%
\pgfpathlineto{\pgfqpoint{3.209641in}{2.566293in}}%
\pgfpathlineto{\pgfqpoint{3.217690in}{2.577037in}}%
\pgfpathlineto{\pgfqpoint{3.225733in}{2.587877in}}%
\pgfpathlineto{\pgfqpoint{3.233769in}{2.598815in}}%
\pgfpathlineto{\pgfqpoint{3.220347in}{2.611255in}}%
\pgfpathlineto{\pgfqpoint{3.206922in}{2.623962in}}%
\pgfpathlineto{\pgfqpoint{3.193495in}{2.636939in}}%
\pgfpathlineto{\pgfqpoint{3.180064in}{2.650187in}}%
\pgfpathlineto{\pgfqpoint{3.172014in}{2.639205in}}%
\pgfpathlineto{\pgfqpoint{3.163958in}{2.628328in}}%
\pgfpathlineto{\pgfqpoint{3.155895in}{2.617555in}}%
\pgfpathlineto{\pgfqpoint{3.147825in}{2.606885in}}%
\pgfpathclose%
\pgfusepath{fill}%
\end{pgfscope}%
\begin{pgfscope}%
\pgfpathrectangle{\pgfqpoint{1.150000in}{0.150000in}}{\pgfqpoint{5.700000in}{5.700000in}}%
\pgfusepath{clip}%
\pgfsetbuttcap%
\pgfsetroundjoin%
\definecolor{currentfill}{rgb}{0.274128,0.199721,0.498911}%
\pgfsetfillcolor{currentfill}%
\pgfsetfillopacity{0.800000}%
\pgfsetlinewidth{0.000000pt}%
\definecolor{currentstroke}{rgb}{0.000000,0.000000,0.000000}%
\pgfsetstrokecolor{currentstroke}%
\pgfsetdash{}{0pt}%
\pgfpathmoveto{\pgfqpoint{4.152154in}{2.574587in}}%
\pgfpathlineto{\pgfqpoint{4.165658in}{2.573079in}}%
\pgfpathlineto{\pgfqpoint{4.179169in}{2.571770in}}%
\pgfpathlineto{\pgfqpoint{4.192688in}{2.570660in}}%
\pgfpathlineto{\pgfqpoint{4.206214in}{2.569747in}}%
\pgfpathlineto{\pgfqpoint{4.213963in}{2.580442in}}%
\pgfpathlineto{\pgfqpoint{4.221708in}{2.591190in}}%
\pgfpathlineto{\pgfqpoint{4.229447in}{2.601995in}}%
\pgfpathlineto{\pgfqpoint{4.237182in}{2.612860in}}%
\pgfpathlineto{\pgfqpoint{4.223665in}{2.614059in}}%
\pgfpathlineto{\pgfqpoint{4.210155in}{2.615456in}}%
\pgfpathlineto{\pgfqpoint{4.196652in}{2.617051in}}%
\pgfpathlineto{\pgfqpoint{4.183157in}{2.618844in}}%
\pgfpathlineto{\pgfqpoint{4.175413in}{2.607682in}}%
\pgfpathlineto{\pgfqpoint{4.167665in}{2.596587in}}%
\pgfpathlineto{\pgfqpoint{4.159912in}{2.585557in}}%
\pgfpathlineto{\pgfqpoint{4.152154in}{2.574587in}}%
\pgfpathclose%
\pgfusepath{fill}%
\end{pgfscope}%
\begin{pgfscope}%
\pgfpathrectangle{\pgfqpoint{1.150000in}{0.150000in}}{\pgfqpoint{5.700000in}{5.700000in}}%
\pgfusepath{clip}%
\pgfsetbuttcap%
\pgfsetroundjoin%
\definecolor{currentfill}{rgb}{0.282290,0.145912,0.461510}%
\pgfsetfillcolor{currentfill}%
\pgfsetfillopacity{0.800000}%
\pgfsetlinewidth{0.000000pt}%
\definecolor{currentstroke}{rgb}{0.000000,0.000000,0.000000}%
\pgfsetstrokecolor{currentstroke}%
\pgfsetdash{}{0pt}%
\pgfpathmoveto{\pgfqpoint{3.757992in}{2.461484in}}%
\pgfpathlineto{\pgfqpoint{3.771412in}{2.456648in}}%
\pgfpathlineto{\pgfqpoint{3.784835in}{2.452030in}}%
\pgfpathlineto{\pgfqpoint{3.798264in}{2.447627in}}%
\pgfpathlineto{\pgfqpoint{3.811696in}{2.443439in}}%
\pgfpathlineto{\pgfqpoint{3.819565in}{2.454431in}}%
\pgfpathlineto{\pgfqpoint{3.827428in}{2.465471in}}%
\pgfpathlineto{\pgfqpoint{3.835286in}{2.476562in}}%
\pgfpathlineto{\pgfqpoint{3.843139in}{2.487704in}}%
\pgfpathlineto{\pgfqpoint{3.829715in}{2.492053in}}%
\pgfpathlineto{\pgfqpoint{3.816296in}{2.496615in}}%
\pgfpathlineto{\pgfqpoint{3.802880in}{2.501394in}}%
\pgfpathlineto{\pgfqpoint{3.789470in}{2.506389in}}%
\pgfpathlineto{\pgfqpoint{3.781608in}{2.495074in}}%
\pgfpathlineto{\pgfqpoint{3.773741in}{2.483820in}}%
\pgfpathlineto{\pgfqpoint{3.765869in}{2.472624in}}%
\pgfpathlineto{\pgfqpoint{3.757992in}{2.461484in}}%
\pgfpathclose%
\pgfusepath{fill}%
\end{pgfscope}%
\begin{pgfscope}%
\pgfpathrectangle{\pgfqpoint{1.150000in}{0.150000in}}{\pgfqpoint{5.700000in}{5.700000in}}%
\pgfusepath{clip}%
\pgfsetbuttcap%
\pgfsetroundjoin%
\definecolor{currentfill}{rgb}{0.223925,0.334994,0.548053}%
\pgfsetfillcolor{currentfill}%
\pgfsetfillopacity{0.800000}%
\pgfsetlinewidth{0.000000pt}%
\definecolor{currentstroke}{rgb}{0.000000,0.000000,0.000000}%
\pgfsetstrokecolor{currentstroke}%
\pgfsetdash{}{0pt}%
\pgfpathmoveto{\pgfqpoint{2.877805in}{2.931752in}}%
\pgfpathlineto{\pgfqpoint{2.891374in}{2.912608in}}%
\pgfpathlineto{\pgfqpoint{2.904934in}{2.893786in}}%
\pgfpathlineto{\pgfqpoint{2.918485in}{2.875284in}}%
\pgfpathlineto{\pgfqpoint{2.932029in}{2.857097in}}%
\pgfpathlineto{\pgfqpoint{2.940162in}{2.867846in}}%
\pgfpathlineto{\pgfqpoint{2.948287in}{2.878730in}}%
\pgfpathlineto{\pgfqpoint{2.956404in}{2.889750in}}%
\pgfpathlineto{\pgfqpoint{2.964514in}{2.900908in}}%
\pgfpathlineto{\pgfqpoint{2.950987in}{2.919126in}}%
\pgfpathlineto{\pgfqpoint{2.937453in}{2.937660in}}%
\pgfpathlineto{\pgfqpoint{2.923910in}{2.956512in}}%
\pgfpathlineto{\pgfqpoint{2.910359in}{2.975687in}}%
\pgfpathlineto{\pgfqpoint{2.902233in}{2.964487in}}%
\pgfpathlineto{\pgfqpoint{2.894099in}{2.953432in}}%
\pgfpathlineto{\pgfqpoint{2.885956in}{2.942520in}}%
\pgfpathlineto{\pgfqpoint{2.877805in}{2.931752in}}%
\pgfpathclose%
\pgfusepath{fill}%
\end{pgfscope}%
\begin{pgfscope}%
\pgfpathrectangle{\pgfqpoint{1.150000in}{0.150000in}}{\pgfqpoint{5.700000in}{5.700000in}}%
\pgfusepath{clip}%
\pgfsetbuttcap%
\pgfsetroundjoin%
\definecolor{currentfill}{rgb}{0.194100,0.399323,0.555565}%
\pgfsetfillcolor{currentfill}%
\pgfsetfillopacity{0.800000}%
\pgfsetlinewidth{0.000000pt}%
\definecolor{currentstroke}{rgb}{0.000000,0.000000,0.000000}%
\pgfsetstrokecolor{currentstroke}%
\pgfsetdash{}{0pt}%
\pgfpathmoveto{\pgfqpoint{5.087620in}{3.071398in}}%
\pgfpathlineto{\pgfqpoint{5.101432in}{3.073207in}}%
\pgfpathlineto{\pgfqpoint{5.115257in}{3.075191in}}%
\pgfpathlineto{\pgfqpoint{5.129094in}{3.077352in}}%
\pgfpathlineto{\pgfqpoint{5.142944in}{3.079688in}}%
\pgfpathlineto{\pgfqpoint{5.150407in}{3.090296in}}%
\pgfpathlineto{\pgfqpoint{5.157869in}{3.101105in}}%
\pgfpathlineto{\pgfqpoint{5.165331in}{3.112122in}}%
\pgfpathlineto{\pgfqpoint{5.172794in}{3.123354in}}%
\pgfpathlineto{\pgfqpoint{5.158963in}{3.121654in}}%
\pgfpathlineto{\pgfqpoint{5.145146in}{3.120129in}}%
\pgfpathlineto{\pgfqpoint{5.131340in}{3.118780in}}%
\pgfpathlineto{\pgfqpoint{5.117547in}{3.117606in}}%
\pgfpathlineto{\pgfqpoint{5.110065in}{3.105727in}}%
\pgfpathlineto{\pgfqpoint{5.102584in}{3.094071in}}%
\pgfpathlineto{\pgfqpoint{5.095102in}{3.082631in}}%
\pgfpathlineto{\pgfqpoint{5.087620in}{3.071398in}}%
\pgfpathclose%
\pgfusepath{fill}%
\end{pgfscope}%
\begin{pgfscope}%
\pgfpathrectangle{\pgfqpoint{1.150000in}{0.150000in}}{\pgfqpoint{5.700000in}{5.700000in}}%
\pgfusepath{clip}%
\pgfsetbuttcap%
\pgfsetroundjoin%
\definecolor{currentfill}{rgb}{0.277134,0.185228,0.489898}%
\pgfsetfillcolor{currentfill}%
\pgfsetfillopacity{0.800000}%
\pgfsetlinewidth{0.000000pt}%
\definecolor{currentstroke}{rgb}{0.000000,0.000000,0.000000}%
\pgfsetstrokecolor{currentstroke}%
\pgfsetdash{}{0pt}%
\pgfpathmoveto{\pgfqpoint{4.067103in}{2.538273in}}%
\pgfpathlineto{\pgfqpoint{4.080586in}{2.536216in}}%
\pgfpathlineto{\pgfqpoint{4.094076in}{2.534361in}}%
\pgfpathlineto{\pgfqpoint{4.107573in}{2.532707in}}%
\pgfpathlineto{\pgfqpoint{4.121078in}{2.531254in}}%
\pgfpathlineto{\pgfqpoint{4.128854in}{2.542012in}}%
\pgfpathlineto{\pgfqpoint{4.136625in}{2.552818in}}%
\pgfpathlineto{\pgfqpoint{4.144392in}{2.563676in}}%
\pgfpathlineto{\pgfqpoint{4.152154in}{2.574587in}}%
\pgfpathlineto{\pgfqpoint{4.138658in}{2.576295in}}%
\pgfpathlineto{\pgfqpoint{4.125170in}{2.578204in}}%
\pgfpathlineto{\pgfqpoint{4.111688in}{2.580313in}}%
\pgfpathlineto{\pgfqpoint{4.098213in}{2.582625in}}%
\pgfpathlineto{\pgfqpoint{4.090442in}{2.571447in}}%
\pgfpathlineto{\pgfqpoint{4.082667in}{2.560331in}}%
\pgfpathlineto{\pgfqpoint{4.074887in}{2.549275in}}%
\pgfpathlineto{\pgfqpoint{4.067103in}{2.538273in}}%
\pgfpathclose%
\pgfusepath{fill}%
\end{pgfscope}%
\begin{pgfscope}%
\pgfpathrectangle{\pgfqpoint{1.150000in}{0.150000in}}{\pgfqpoint{5.700000in}{5.700000in}}%
\pgfusepath{clip}%
\pgfsetbuttcap%
\pgfsetroundjoin%
\definecolor{currentfill}{rgb}{0.185556,0.418570,0.556753}%
\pgfsetfillcolor{currentfill}%
\pgfsetfillopacity{0.800000}%
\pgfsetlinewidth{0.000000pt}%
\definecolor{currentstroke}{rgb}{0.000000,0.000000,0.000000}%
\pgfsetstrokecolor{currentstroke}%
\pgfsetdash{}{0pt}%
\pgfpathmoveto{\pgfqpoint{5.172794in}{3.123354in}}%
\pgfpathlineto{\pgfqpoint{5.186636in}{3.125229in}}%
\pgfpathlineto{\pgfqpoint{5.200492in}{3.127279in}}%
\pgfpathlineto{\pgfqpoint{5.214360in}{3.129503in}}%
\pgfpathlineto{\pgfqpoint{5.228241in}{3.131902in}}%
\pgfpathlineto{\pgfqpoint{5.235683in}{3.142700in}}%
\pgfpathlineto{\pgfqpoint{5.243125in}{3.153721in}}%
\pgfpathlineto{\pgfqpoint{5.250569in}{3.164972in}}%
\pgfpathlineto{\pgfqpoint{5.258013in}{3.176460in}}%
\pgfpathlineto{\pgfqpoint{5.244153in}{3.174730in}}%
\pgfpathlineto{\pgfqpoint{5.230306in}{3.173173in}}%
\pgfpathlineto{\pgfqpoint{5.216471in}{3.171790in}}%
\pgfpathlineto{\pgfqpoint{5.202649in}{3.170581in}}%
\pgfpathlineto{\pgfqpoint{5.195184in}{3.158414in}}%
\pgfpathlineto{\pgfqpoint{5.187720in}{3.146493in}}%
\pgfpathlineto{\pgfqpoint{5.180256in}{3.134808in}}%
\pgfpathlineto{\pgfqpoint{5.172794in}{3.123354in}}%
\pgfpathclose%
\pgfusepath{fill}%
\end{pgfscope}%
\begin{pgfscope}%
\pgfpathrectangle{\pgfqpoint{1.150000in}{0.150000in}}{\pgfqpoint{5.700000in}{5.700000in}}%
\pgfusepath{clip}%
\pgfsetbuttcap%
\pgfsetroundjoin%
\definecolor{currentfill}{rgb}{0.282623,0.140926,0.457517}%
\pgfsetfillcolor{currentfill}%
\pgfsetfillopacity{0.800000}%
\pgfsetlinewidth{0.000000pt}%
\definecolor{currentstroke}{rgb}{0.000000,0.000000,0.000000}%
\pgfsetstrokecolor{currentstroke}%
\pgfsetdash{}{0pt}%
\pgfpathmoveto{\pgfqpoint{3.533797in}{2.446773in}}%
\pgfpathlineto{\pgfqpoint{3.547199in}{2.439473in}}%
\pgfpathlineto{\pgfqpoint{3.560603in}{2.432404in}}%
\pgfpathlineto{\pgfqpoint{3.574009in}{2.425564in}}%
\pgfpathlineto{\pgfqpoint{3.587418in}{2.418953in}}%
\pgfpathlineto{\pgfqpoint{3.595355in}{2.429885in}}%
\pgfpathlineto{\pgfqpoint{3.603287in}{2.440874in}}%
\pgfpathlineto{\pgfqpoint{3.611214in}{2.451922in}}%
\pgfpathlineto{\pgfqpoint{3.619135in}{2.463030in}}%
\pgfpathlineto{\pgfqpoint{3.605737in}{2.469738in}}%
\pgfpathlineto{\pgfqpoint{3.592340in}{2.476675in}}%
\pgfpathlineto{\pgfqpoint{3.578946in}{2.483841in}}%
\pgfpathlineto{\pgfqpoint{3.565554in}{2.491238in}}%
\pgfpathlineto{\pgfqpoint{3.557623in}{2.480021in}}%
\pgfpathlineto{\pgfqpoint{3.549686in}{2.468873in}}%
\pgfpathlineto{\pgfqpoint{3.541744in}{2.457790in}}%
\pgfpathlineto{\pgfqpoint{3.533797in}{2.446773in}}%
\pgfpathclose%
\pgfusepath{fill}%
\end{pgfscope}%
\begin{pgfscope}%
\pgfpathrectangle{\pgfqpoint{1.150000in}{0.150000in}}{\pgfqpoint{5.700000in}{5.700000in}}%
\pgfusepath{clip}%
\pgfsetbuttcap%
\pgfsetroundjoin%
\definecolor{currentfill}{rgb}{0.281887,0.150881,0.465405}%
\pgfsetfillcolor{currentfill}%
\pgfsetfillopacity{0.800000}%
\pgfsetlinewidth{0.000000pt}%
\definecolor{currentstroke}{rgb}{0.000000,0.000000,0.000000}%
\pgfsetstrokecolor{currentstroke}%
\pgfsetdash{}{0pt}%
\pgfpathmoveto{\pgfqpoint{3.394691in}{2.469693in}}%
\pgfpathlineto{\pgfqpoint{3.408096in}{2.460559in}}%
\pgfpathlineto{\pgfqpoint{3.421501in}{2.451668in}}%
\pgfpathlineto{\pgfqpoint{3.434907in}{2.443018in}}%
\pgfpathlineto{\pgfqpoint{3.448314in}{2.434607in}}%
\pgfpathlineto{\pgfqpoint{3.456295in}{2.445430in}}%
\pgfpathlineto{\pgfqpoint{3.464270in}{2.456321in}}%
\pgfpathlineto{\pgfqpoint{3.472239in}{2.467282in}}%
\pgfpathlineto{\pgfqpoint{3.480203in}{2.478312in}}%
\pgfpathlineto{\pgfqpoint{3.466808in}{2.486788in}}%
\pgfpathlineto{\pgfqpoint{3.453413in}{2.495504in}}%
\pgfpathlineto{\pgfqpoint{3.440019in}{2.504460in}}%
\pgfpathlineto{\pgfqpoint{3.426625in}{2.513659in}}%
\pgfpathlineto{\pgfqpoint{3.418650in}{2.502552in}}%
\pgfpathlineto{\pgfqpoint{3.410669in}{2.491522in}}%
\pgfpathlineto{\pgfqpoint{3.402683in}{2.480570in}}%
\pgfpathlineto{\pgfqpoint{3.394691in}{2.469693in}}%
\pgfpathclose%
\pgfusepath{fill}%
\end{pgfscope}%
\begin{pgfscope}%
\pgfpathrectangle{\pgfqpoint{1.150000in}{0.150000in}}{\pgfqpoint{5.700000in}{5.700000in}}%
\pgfusepath{clip}%
\pgfsetbuttcap%
\pgfsetroundjoin%
\definecolor{currentfill}{rgb}{0.277134,0.185228,0.489898}%
\pgfsetfillcolor{currentfill}%
\pgfsetfillopacity{0.800000}%
\pgfsetlinewidth{0.000000pt}%
\definecolor{currentstroke}{rgb}{0.000000,0.000000,0.000000}%
\pgfsetstrokecolor{currentstroke}%
\pgfsetdash{}{0pt}%
\pgfpathmoveto{\pgfqpoint{3.201586in}{2.555644in}}%
\pgfpathlineto{\pgfqpoint{3.215018in}{2.543502in}}%
\pgfpathlineto{\pgfqpoint{3.228448in}{2.531624in}}%
\pgfpathlineto{\pgfqpoint{3.241876in}{2.520007in}}%
\pgfpathlineto{\pgfqpoint{3.255302in}{2.508649in}}%
\pgfpathlineto{\pgfqpoint{3.263344in}{2.519277in}}%
\pgfpathlineto{\pgfqpoint{3.271380in}{2.529992in}}%
\pgfpathlineto{\pgfqpoint{3.279409in}{2.540795in}}%
\pgfpathlineto{\pgfqpoint{3.287432in}{2.551688in}}%
\pgfpathlineto{\pgfqpoint{3.274019in}{2.563079in}}%
\pgfpathlineto{\pgfqpoint{3.260604in}{2.574729in}}%
\pgfpathlineto{\pgfqpoint{3.247188in}{2.586640in}}%
\pgfpathlineto{\pgfqpoint{3.233769in}{2.598815in}}%
\pgfpathlineto{\pgfqpoint{3.225733in}{2.587877in}}%
\pgfpathlineto{\pgfqpoint{3.217690in}{2.577037in}}%
\pgfpathlineto{\pgfqpoint{3.209641in}{2.566293in}}%
\pgfpathlineto{\pgfqpoint{3.201586in}{2.555644in}}%
\pgfpathclose%
\pgfusepath{fill}%
\end{pgfscope}%
\begin{pgfscope}%
\pgfpathrectangle{\pgfqpoint{1.150000in}{0.150000in}}{\pgfqpoint{5.700000in}{5.700000in}}%
\pgfusepath{clip}%
\pgfsetbuttcap%
\pgfsetroundjoin%
\definecolor{currentfill}{rgb}{0.210503,0.363727,0.552206}%
\pgfsetfillcolor{currentfill}%
\pgfsetfillopacity{0.800000}%
\pgfsetlinewidth{0.000000pt}%
\definecolor{currentstroke}{rgb}{0.000000,0.000000,0.000000}%
\pgfsetstrokecolor{currentstroke}%
\pgfsetdash{}{0pt}%
\pgfpathmoveto{\pgfqpoint{2.823436in}{3.011609in}}%
\pgfpathlineto{\pgfqpoint{2.837043in}{2.991146in}}%
\pgfpathlineto{\pgfqpoint{2.850640in}{2.971018in}}%
\pgfpathlineto{\pgfqpoint{2.864227in}{2.951221in}}%
\pgfpathlineto{\pgfqpoint{2.877805in}{2.931752in}}%
\pgfpathlineto{\pgfqpoint{2.885956in}{2.942520in}}%
\pgfpathlineto{\pgfqpoint{2.894099in}{2.953432in}}%
\pgfpathlineto{\pgfqpoint{2.902233in}{2.964487in}}%
\pgfpathlineto{\pgfqpoint{2.910359in}{2.975687in}}%
\pgfpathlineto{\pgfqpoint{2.896798in}{2.995187in}}%
\pgfpathlineto{\pgfqpoint{2.883229in}{3.015014in}}%
\pgfpathlineto{\pgfqpoint{2.869650in}{3.035173in}}%
\pgfpathlineto{\pgfqpoint{2.856060in}{3.055666in}}%
\pgfpathlineto{\pgfqpoint{2.847917in}{3.044423in}}%
\pgfpathlineto{\pgfqpoint{2.839765in}{3.033333in}}%
\pgfpathlineto{\pgfqpoint{2.831605in}{3.022395in}}%
\pgfpathlineto{\pgfqpoint{2.823436in}{3.011609in}}%
\pgfpathclose%
\pgfusepath{fill}%
\end{pgfscope}%
\begin{pgfscope}%
\pgfpathrectangle{\pgfqpoint{1.150000in}{0.150000in}}{\pgfqpoint{5.700000in}{5.700000in}}%
\pgfusepath{clip}%
\pgfsetbuttcap%
\pgfsetroundjoin%
\definecolor{currentfill}{rgb}{0.177423,0.437527,0.557565}%
\pgfsetfillcolor{currentfill}%
\pgfsetfillopacity{0.800000}%
\pgfsetlinewidth{0.000000pt}%
\definecolor{currentstroke}{rgb}{0.000000,0.000000,0.000000}%
\pgfsetstrokecolor{currentstroke}%
\pgfsetdash{}{0pt}%
\pgfpathmoveto{\pgfqpoint{5.258013in}{3.176460in}}%
\pgfpathlineto{\pgfqpoint{5.271886in}{3.178365in}}%
\pgfpathlineto{\pgfqpoint{5.285771in}{3.180442in}}%
\pgfpathlineto{\pgfqpoint{5.299670in}{3.182693in}}%
\pgfpathlineto{\pgfqpoint{5.313582in}{3.185117in}}%
\pgfpathlineto{\pgfqpoint{5.321005in}{3.196163in}}%
\pgfpathlineto{\pgfqpoint{5.328430in}{3.207455in}}%
\pgfpathlineto{\pgfqpoint{5.335857in}{3.219000in}}%
\pgfpathlineto{\pgfqpoint{5.343285in}{3.230806in}}%
\pgfpathlineto{\pgfqpoint{5.329396in}{3.229082in}}%
\pgfpathlineto{\pgfqpoint{5.315520in}{3.227531in}}%
\pgfpathlineto{\pgfqpoint{5.301656in}{3.226152in}}%
\pgfpathlineto{\pgfqpoint{5.287806in}{3.224946in}}%
\pgfpathlineto{\pgfqpoint{5.280354in}{3.212429in}}%
\pgfpathlineto{\pgfqpoint{5.272906in}{3.200181in}}%
\pgfpathlineto{\pgfqpoint{5.265458in}{3.188194in}}%
\pgfpathlineto{\pgfqpoint{5.258013in}{3.176460in}}%
\pgfpathclose%
\pgfusepath{fill}%
\end{pgfscope}%
\begin{pgfscope}%
\pgfpathrectangle{\pgfqpoint{1.150000in}{0.150000in}}{\pgfqpoint{5.700000in}{5.700000in}}%
\pgfusepath{clip}%
\pgfsetbuttcap%
\pgfsetroundjoin%
\definecolor{currentfill}{rgb}{0.279574,0.170599,0.479997}%
\pgfsetfillcolor{currentfill}%
\pgfsetfillopacity{0.800000}%
\pgfsetlinewidth{0.000000pt}%
\definecolor{currentstroke}{rgb}{0.000000,0.000000,0.000000}%
\pgfsetstrokecolor{currentstroke}%
\pgfsetdash{}{0pt}%
\pgfpathmoveto{\pgfqpoint{3.982017in}{2.504141in}}%
\pgfpathlineto{\pgfqpoint{3.995482in}{2.501490in}}%
\pgfpathlineto{\pgfqpoint{4.008954in}{2.499045in}}%
\pgfpathlineto{\pgfqpoint{4.022432in}{2.496804in}}%
\pgfpathlineto{\pgfqpoint{4.035916in}{2.494767in}}%
\pgfpathlineto{\pgfqpoint{4.043720in}{2.505575in}}%
\pgfpathlineto{\pgfqpoint{4.051519in}{2.516426in}}%
\pgfpathlineto{\pgfqpoint{4.059313in}{2.527325in}}%
\pgfpathlineto{\pgfqpoint{4.067103in}{2.538273in}}%
\pgfpathlineto{\pgfqpoint{4.053626in}{2.540534in}}%
\pgfpathlineto{\pgfqpoint{4.040156in}{2.542998in}}%
\pgfpathlineto{\pgfqpoint{4.026693in}{2.545666in}}%
\pgfpathlineto{\pgfqpoint{4.013236in}{2.548540in}}%
\pgfpathlineto{\pgfqpoint{4.005439in}{2.537357in}}%
\pgfpathlineto{\pgfqpoint{3.997636in}{2.526231in}}%
\pgfpathlineto{\pgfqpoint{3.989829in}{2.515160in}}%
\pgfpathlineto{\pgfqpoint{3.982017in}{2.504141in}}%
\pgfpathclose%
\pgfusepath{fill}%
\end{pgfscope}%
\begin{pgfscope}%
\pgfpathrectangle{\pgfqpoint{1.150000in}{0.150000in}}{\pgfqpoint{5.700000in}{5.700000in}}%
\pgfusepath{clip}%
\pgfsetbuttcap%
\pgfsetroundjoin%
\definecolor{currentfill}{rgb}{0.282623,0.140926,0.457517}%
\pgfsetfillcolor{currentfill}%
\pgfsetfillopacity{0.800000}%
\pgfsetlinewidth{0.000000pt}%
\definecolor{currentstroke}{rgb}{0.000000,0.000000,0.000000}%
\pgfsetstrokecolor{currentstroke}%
\pgfsetdash{}{0pt}%
\pgfpathmoveto{\pgfqpoint{3.672758in}{2.438457in}}%
\pgfpathlineto{\pgfqpoint{3.686171in}{2.432872in}}%
\pgfpathlineto{\pgfqpoint{3.699588in}{2.427509in}}%
\pgfpathlineto{\pgfqpoint{3.713009in}{2.422365in}}%
\pgfpathlineto{\pgfqpoint{3.726433in}{2.417441in}}%
\pgfpathlineto{\pgfqpoint{3.734331in}{2.428377in}}%
\pgfpathlineto{\pgfqpoint{3.742223in}{2.439362in}}%
\pgfpathlineto{\pgfqpoint{3.750110in}{2.450397in}}%
\pgfpathlineto{\pgfqpoint{3.757992in}{2.461484in}}%
\pgfpathlineto{\pgfqpoint{3.744577in}{2.466537in}}%
\pgfpathlineto{\pgfqpoint{3.731165in}{2.471809in}}%
\pgfpathlineto{\pgfqpoint{3.717757in}{2.477301in}}%
\pgfpathlineto{\pgfqpoint{3.704353in}{2.483014in}}%
\pgfpathlineto{\pgfqpoint{3.696462in}{2.471787in}}%
\pgfpathlineto{\pgfqpoint{3.688566in}{2.460620in}}%
\pgfpathlineto{\pgfqpoint{3.680665in}{2.449510in}}%
\pgfpathlineto{\pgfqpoint{3.672758in}{2.438457in}}%
\pgfpathclose%
\pgfusepath{fill}%
\end{pgfscope}%
\begin{pgfscope}%
\pgfpathrectangle{\pgfqpoint{1.150000in}{0.150000in}}{\pgfqpoint{5.700000in}{5.700000in}}%
\pgfusepath{clip}%
\pgfsetbuttcap%
\pgfsetroundjoin%
\definecolor{currentfill}{rgb}{0.168126,0.459988,0.558082}%
\pgfsetfillcolor{currentfill}%
\pgfsetfillopacity{0.800000}%
\pgfsetlinewidth{0.000000pt}%
\definecolor{currentstroke}{rgb}{0.000000,0.000000,0.000000}%
\pgfsetstrokecolor{currentstroke}%
\pgfsetdash{}{0pt}%
\pgfpathmoveto{\pgfqpoint{5.343285in}{3.230806in}}%
\pgfpathlineto{\pgfqpoint{5.357187in}{3.232703in}}%
\pgfpathlineto{\pgfqpoint{5.371103in}{3.234771in}}%
\pgfpathlineto{\pgfqpoint{5.385032in}{3.237012in}}%
\pgfpathlineto{\pgfqpoint{5.398974in}{3.239424in}}%
\pgfpathlineto{\pgfqpoint{5.406381in}{3.250780in}}%
\pgfpathlineto{\pgfqpoint{5.413791in}{3.262407in}}%
\pgfpathlineto{\pgfqpoint{5.421203in}{3.274312in}}%
\pgfpathlineto{\pgfqpoint{5.428619in}{3.286504in}}%
\pgfpathlineto{\pgfqpoint{5.414701in}{3.284824in}}%
\pgfpathlineto{\pgfqpoint{5.400797in}{3.283315in}}%
\pgfpathlineto{\pgfqpoint{5.386905in}{3.281977in}}%
\pgfpathlineto{\pgfqpoint{5.373026in}{3.280811in}}%
\pgfpathlineto{\pgfqpoint{5.365586in}{3.267876in}}%
\pgfpathlineto{\pgfqpoint{5.358150in}{3.255236in}}%
\pgfpathlineto{\pgfqpoint{5.350716in}{3.242882in}}%
\pgfpathlineto{\pgfqpoint{5.343285in}{3.230806in}}%
\pgfpathclose%
\pgfusepath{fill}%
\end{pgfscope}%
\begin{pgfscope}%
\pgfpathrectangle{\pgfqpoint{1.150000in}{0.150000in}}{\pgfqpoint{5.700000in}{5.700000in}}%
\pgfusepath{clip}%
\pgfsetbuttcap%
\pgfsetroundjoin%
\definecolor{currentfill}{rgb}{0.280255,0.165693,0.476498}%
\pgfsetfillcolor{currentfill}%
\pgfsetfillopacity{0.800000}%
\pgfsetlinewidth{0.000000pt}%
\definecolor{currentstroke}{rgb}{0.000000,0.000000,0.000000}%
\pgfsetstrokecolor{currentstroke}%
\pgfsetdash{}{0pt}%
\pgfpathmoveto{\pgfqpoint{3.255302in}{2.508649in}}%
\pgfpathlineto{\pgfqpoint{3.268726in}{2.497550in}}%
\pgfpathlineto{\pgfqpoint{3.282148in}{2.486706in}}%
\pgfpathlineto{\pgfqpoint{3.295569in}{2.476117in}}%
\pgfpathlineto{\pgfqpoint{3.308989in}{2.465780in}}%
\pgfpathlineto{\pgfqpoint{3.317018in}{2.476386in}}%
\pgfpathlineto{\pgfqpoint{3.325041in}{2.487071in}}%
\pgfpathlineto{\pgfqpoint{3.333057in}{2.497837in}}%
\pgfpathlineto{\pgfqpoint{3.341068in}{2.508685in}}%
\pgfpathlineto{\pgfqpoint{3.327660in}{2.519056in}}%
\pgfpathlineto{\pgfqpoint{3.314252in}{2.529679in}}%
\pgfpathlineto{\pgfqpoint{3.300843in}{2.540555in}}%
\pgfpathlineto{\pgfqpoint{3.287432in}{2.551688in}}%
\pgfpathlineto{\pgfqpoint{3.279409in}{2.540795in}}%
\pgfpathlineto{\pgfqpoint{3.271380in}{2.529992in}}%
\pgfpathlineto{\pgfqpoint{3.263344in}{2.519277in}}%
\pgfpathlineto{\pgfqpoint{3.255302in}{2.508649in}}%
\pgfpathclose%
\pgfusepath{fill}%
\end{pgfscope}%
\begin{pgfscope}%
\pgfpathrectangle{\pgfqpoint{1.150000in}{0.150000in}}{\pgfqpoint{5.700000in}{5.700000in}}%
\pgfusepath{clip}%
\pgfsetbuttcap%
\pgfsetroundjoin%
\definecolor{currentfill}{rgb}{0.280868,0.160771,0.472899}%
\pgfsetfillcolor{currentfill}%
\pgfsetfillopacity{0.800000}%
\pgfsetlinewidth{0.000000pt}%
\definecolor{currentstroke}{rgb}{0.000000,0.000000,0.000000}%
\pgfsetstrokecolor{currentstroke}%
\pgfsetdash{}{0pt}%
\pgfpathmoveto{\pgfqpoint{3.896886in}{2.472438in}}%
\pgfpathlineto{\pgfqpoint{3.910336in}{2.469149in}}%
\pgfpathlineto{\pgfqpoint{3.923791in}{2.466068in}}%
\pgfpathlineto{\pgfqpoint{3.937253in}{2.463196in}}%
\pgfpathlineto{\pgfqpoint{3.950720in}{2.460531in}}%
\pgfpathlineto{\pgfqpoint{3.958552in}{2.471368in}}%
\pgfpathlineto{\pgfqpoint{3.966378in}{2.482248in}}%
\pgfpathlineto{\pgfqpoint{3.974200in}{2.493171in}}%
\pgfpathlineto{\pgfqpoint{3.982017in}{2.504141in}}%
\pgfpathlineto{\pgfqpoint{3.968558in}{2.506998in}}%
\pgfpathlineto{\pgfqpoint{3.955105in}{2.510062in}}%
\pgfpathlineto{\pgfqpoint{3.941657in}{2.513335in}}%
\pgfpathlineto{\pgfqpoint{3.928216in}{2.516816in}}%
\pgfpathlineto{\pgfqpoint{3.920391in}{2.505642in}}%
\pgfpathlineto{\pgfqpoint{3.912561in}{2.494523in}}%
\pgfpathlineto{\pgfqpoint{3.904726in}{2.483456in}}%
\pgfpathlineto{\pgfqpoint{3.896886in}{2.472438in}}%
\pgfpathclose%
\pgfusepath{fill}%
\end{pgfscope}%
\begin{pgfscope}%
\pgfpathrectangle{\pgfqpoint{1.150000in}{0.150000in}}{\pgfqpoint{5.700000in}{5.700000in}}%
\pgfusepath{clip}%
\pgfsetbuttcap%
\pgfsetroundjoin%
\definecolor{currentfill}{rgb}{0.195860,0.395433,0.555276}%
\pgfsetfillcolor{currentfill}%
\pgfsetfillopacity{0.800000}%
\pgfsetlinewidth{0.000000pt}%
\definecolor{currentstroke}{rgb}{0.000000,0.000000,0.000000}%
\pgfsetstrokecolor{currentstroke}%
\pgfsetdash{}{0pt}%
\pgfpathmoveto{\pgfqpoint{2.768899in}{3.096870in}}%
\pgfpathlineto{\pgfqpoint{2.782550in}{3.075036in}}%
\pgfpathlineto{\pgfqpoint{2.796189in}{3.053550in}}%
\pgfpathlineto{\pgfqpoint{2.809818in}{3.032409in}}%
\pgfpathlineto{\pgfqpoint{2.823436in}{3.011609in}}%
\pgfpathlineto{\pgfqpoint{2.831605in}{3.022395in}}%
\pgfpathlineto{\pgfqpoint{2.839765in}{3.033333in}}%
\pgfpathlineto{\pgfqpoint{2.847917in}{3.044423in}}%
\pgfpathlineto{\pgfqpoint{2.856060in}{3.055666in}}%
\pgfpathlineto{\pgfqpoint{2.842461in}{3.076497in}}%
\pgfpathlineto{\pgfqpoint{2.828851in}{3.097669in}}%
\pgfpathlineto{\pgfqpoint{2.815230in}{3.119185in}}%
\pgfpathlineto{\pgfqpoint{2.801598in}{3.141049in}}%
\pgfpathlineto{\pgfqpoint{2.793436in}{3.129763in}}%
\pgfpathlineto{\pgfqpoint{2.785266in}{3.118639in}}%
\pgfpathlineto{\pgfqpoint{2.777087in}{3.107675in}}%
\pgfpathlineto{\pgfqpoint{2.768899in}{3.096870in}}%
\pgfpathclose%
\pgfusepath{fill}%
\end{pgfscope}%
\begin{pgfscope}%
\pgfpathrectangle{\pgfqpoint{1.150000in}{0.150000in}}{\pgfqpoint{5.700000in}{5.700000in}}%
\pgfusepath{clip}%
\pgfsetbuttcap%
\pgfsetroundjoin%
\definecolor{currentfill}{rgb}{0.282623,0.140926,0.457517}%
\pgfsetfillcolor{currentfill}%
\pgfsetfillopacity{0.800000}%
\pgfsetlinewidth{0.000000pt}%
\definecolor{currentstroke}{rgb}{0.000000,0.000000,0.000000}%
\pgfsetstrokecolor{currentstroke}%
\pgfsetdash{}{0pt}%
\pgfpathmoveto{\pgfqpoint{3.448314in}{2.434607in}}%
\pgfpathlineto{\pgfqpoint{3.461721in}{2.426434in}}%
\pgfpathlineto{\pgfqpoint{3.475130in}{2.418498in}}%
\pgfpathlineto{\pgfqpoint{3.488539in}{2.410797in}}%
\pgfpathlineto{\pgfqpoint{3.501950in}{2.403330in}}%
\pgfpathlineto{\pgfqpoint{3.509920in}{2.414100in}}%
\pgfpathlineto{\pgfqpoint{3.517885in}{2.424929in}}%
\pgfpathlineto{\pgfqpoint{3.525843in}{2.435820in}}%
\pgfpathlineto{\pgfqpoint{3.533797in}{2.446773in}}%
\pgfpathlineto{\pgfqpoint{3.520396in}{2.454306in}}%
\pgfpathlineto{\pgfqpoint{3.506997in}{2.462073in}}%
\pgfpathlineto{\pgfqpoint{3.493600in}{2.470074in}}%
\pgfpathlineto{\pgfqpoint{3.480203in}{2.478312in}}%
\pgfpathlineto{\pgfqpoint{3.472239in}{2.467282in}}%
\pgfpathlineto{\pgfqpoint{3.464270in}{2.456321in}}%
\pgfpathlineto{\pgfqpoint{3.456295in}{2.445430in}}%
\pgfpathlineto{\pgfqpoint{3.448314in}{2.434607in}}%
\pgfpathclose%
\pgfusepath{fill}%
\end{pgfscope}%
\begin{pgfscope}%
\pgfpathrectangle{\pgfqpoint{1.150000in}{0.150000in}}{\pgfqpoint{5.700000in}{5.700000in}}%
\pgfusepath{clip}%
\pgfsetbuttcap%
\pgfsetroundjoin%
\definecolor{currentfill}{rgb}{0.160665,0.478540,0.558115}%
\pgfsetfillcolor{currentfill}%
\pgfsetfillopacity{0.800000}%
\pgfsetlinewidth{0.000000pt}%
\definecolor{currentstroke}{rgb}{0.000000,0.000000,0.000000}%
\pgfsetstrokecolor{currentstroke}%
\pgfsetdash{}{0pt}%
\pgfpathmoveto{\pgfqpoint{5.428619in}{3.286504in}}%
\pgfpathlineto{\pgfqpoint{5.442550in}{3.288356in}}%
\pgfpathlineto{\pgfqpoint{5.456495in}{3.290378in}}%
\pgfpathlineto{\pgfqpoint{5.470453in}{3.292572in}}%
\pgfpathlineto{\pgfqpoint{5.484425in}{3.294936in}}%
\pgfpathlineto{\pgfqpoint{5.491819in}{3.306671in}}%
\pgfpathlineto{\pgfqpoint{5.499216in}{3.318704in}}%
\pgfpathlineto{\pgfqpoint{5.506618in}{3.331041in}}%
\pgfpathlineto{\pgfqpoint{5.492666in}{3.329248in}}%
\pgfpathlineto{\pgfqpoint{5.478727in}{3.327624in}}%
\pgfpathlineto{\pgfqpoint{5.464802in}{3.326171in}}%
\pgfpathlineto{\pgfqpoint{5.450889in}{3.324888in}}%
\pgfpathlineto{\pgfqpoint{5.443462in}{3.311784in}}%
\pgfpathlineto{\pgfqpoint{5.436038in}{3.298992in}}%
\pgfpathlineto{\pgfqpoint{5.428619in}{3.286504in}}%
\pgfpathclose%
\pgfusepath{fill}%
\end{pgfscope}%
\begin{pgfscope}%
\pgfpathrectangle{\pgfqpoint{1.150000in}{0.150000in}}{\pgfqpoint{5.700000in}{5.700000in}}%
\pgfusepath{clip}%
\pgfsetbuttcap%
\pgfsetroundjoin%
\definecolor{currentfill}{rgb}{0.282884,0.135920,0.453427}%
\pgfsetfillcolor{currentfill}%
\pgfsetfillopacity{0.800000}%
\pgfsetlinewidth{0.000000pt}%
\definecolor{currentstroke}{rgb}{0.000000,0.000000,0.000000}%
\pgfsetstrokecolor{currentstroke}%
\pgfsetdash{}{0pt}%
\pgfpathmoveto{\pgfqpoint{3.587418in}{2.418953in}}%
\pgfpathlineto{\pgfqpoint{3.600829in}{2.412569in}}%
\pgfpathlineto{\pgfqpoint{3.614242in}{2.406411in}}%
\pgfpathlineto{\pgfqpoint{3.627659in}{2.400478in}}%
\pgfpathlineto{\pgfqpoint{3.641079in}{2.394769in}}%
\pgfpathlineto{\pgfqpoint{3.649006in}{2.405615in}}%
\pgfpathlineto{\pgfqpoint{3.656929in}{2.416510in}}%
\pgfpathlineto{\pgfqpoint{3.664846in}{2.427457in}}%
\pgfpathlineto{\pgfqpoint{3.672758in}{2.438457in}}%
\pgfpathlineto{\pgfqpoint{3.659348in}{2.444264in}}%
\pgfpathlineto{\pgfqpoint{3.645941in}{2.450294in}}%
\pgfpathlineto{\pgfqpoint{3.632537in}{2.456549in}}%
\pgfpathlineto{\pgfqpoint{3.619135in}{2.463030in}}%
\pgfpathlineto{\pgfqpoint{3.611214in}{2.451922in}}%
\pgfpathlineto{\pgfqpoint{3.603287in}{2.440874in}}%
\pgfpathlineto{\pgfqpoint{3.595355in}{2.429885in}}%
\pgfpathlineto{\pgfqpoint{3.587418in}{2.418953in}}%
\pgfpathclose%
\pgfusepath{fill}%
\end{pgfscope}%
\begin{pgfscope}%
\pgfpathrectangle{\pgfqpoint{1.150000in}{0.150000in}}{\pgfqpoint{5.700000in}{5.700000in}}%
\pgfusepath{clip}%
\pgfsetbuttcap%
\pgfsetroundjoin%
\definecolor{currentfill}{rgb}{0.282290,0.145912,0.461510}%
\pgfsetfillcolor{currentfill}%
\pgfsetfillopacity{0.800000}%
\pgfsetlinewidth{0.000000pt}%
\definecolor{currentstroke}{rgb}{0.000000,0.000000,0.000000}%
\pgfsetstrokecolor{currentstroke}%
\pgfsetdash{}{0pt}%
\pgfpathmoveto{\pgfqpoint{3.811696in}{2.443439in}}%
\pgfpathlineto{\pgfqpoint{3.825134in}{2.439464in}}%
\pgfpathlineto{\pgfqpoint{3.838576in}{2.435703in}}%
\pgfpathlineto{\pgfqpoint{3.852024in}{2.432153in}}%
\pgfpathlineto{\pgfqpoint{3.865477in}{2.428814in}}%
\pgfpathlineto{\pgfqpoint{3.873337in}{2.439658in}}%
\pgfpathlineto{\pgfqpoint{3.881191in}{2.450541in}}%
\pgfpathlineto{\pgfqpoint{3.889041in}{2.461467in}}%
\pgfpathlineto{\pgfqpoint{3.896886in}{2.472438in}}%
\pgfpathlineto{\pgfqpoint{3.883441in}{2.475938in}}%
\pgfpathlineto{\pgfqpoint{3.870002in}{2.479648in}}%
\pgfpathlineto{\pgfqpoint{3.856568in}{2.483570in}}%
\pgfpathlineto{\pgfqpoint{3.843139in}{2.487704in}}%
\pgfpathlineto{\pgfqpoint{3.835286in}{2.476562in}}%
\pgfpathlineto{\pgfqpoint{3.827428in}{2.465471in}}%
\pgfpathlineto{\pgfqpoint{3.819565in}{2.454431in}}%
\pgfpathlineto{\pgfqpoint{3.811696in}{2.443439in}}%
\pgfpathclose%
\pgfusepath{fill}%
\end{pgfscope}%
\begin{pgfscope}%
\pgfpathrectangle{\pgfqpoint{1.150000in}{0.150000in}}{\pgfqpoint{5.700000in}{5.700000in}}%
\pgfusepath{clip}%
\pgfsetbuttcap%
\pgfsetroundjoin%
\definecolor{currentfill}{rgb}{0.248629,0.278775,0.534556}%
\pgfsetfillcolor{currentfill}%
\pgfsetfillopacity{0.800000}%
\pgfsetlinewidth{0.000000pt}%
\definecolor{currentstroke}{rgb}{0.000000,0.000000,0.000000}%
\pgfsetstrokecolor{currentstroke}%
\pgfsetdash{}{0pt}%
\pgfpathmoveto{\pgfqpoint{4.546674in}{2.739738in}}%
\pgfpathlineto{\pgfqpoint{4.560317in}{2.740754in}}%
\pgfpathlineto{\pgfqpoint{4.573970in}{2.741956in}}%
\pgfpathlineto{\pgfqpoint{4.587633in}{2.743345in}}%
\pgfpathlineto{\pgfqpoint{4.601307in}{2.744920in}}%
\pgfpathlineto{\pgfqpoint{4.608935in}{2.754925in}}%
\pgfpathlineto{\pgfqpoint{4.616560in}{2.765010in}}%
\pgfpathlineto{\pgfqpoint{4.624181in}{2.775180in}}%
\pgfpathlineto{\pgfqpoint{4.631797in}{2.785439in}}%
\pgfpathlineto{\pgfqpoint{4.618135in}{2.784279in}}%
\pgfpathlineto{\pgfqpoint{4.604484in}{2.783304in}}%
\pgfpathlineto{\pgfqpoint{4.590842in}{2.782516in}}%
\pgfpathlineto{\pgfqpoint{4.577211in}{2.781914in}}%
\pgfpathlineto{\pgfqpoint{4.569583in}{2.771229in}}%
\pgfpathlineto{\pgfqpoint{4.561950in}{2.760641in}}%
\pgfpathlineto{\pgfqpoint{4.554314in}{2.750146in}}%
\pgfpathlineto{\pgfqpoint{4.546674in}{2.739738in}}%
\pgfpathclose%
\pgfusepath{fill}%
\end{pgfscope}%
\begin{pgfscope}%
\pgfpathrectangle{\pgfqpoint{1.150000in}{0.150000in}}{\pgfqpoint{5.700000in}{5.700000in}}%
\pgfusepath{clip}%
\pgfsetbuttcap%
\pgfsetroundjoin%
\definecolor{currentfill}{rgb}{0.255645,0.260703,0.528312}%
\pgfsetfillcolor{currentfill}%
\pgfsetfillopacity{0.800000}%
\pgfsetlinewidth{0.000000pt}%
\definecolor{currentstroke}{rgb}{0.000000,0.000000,0.000000}%
\pgfsetstrokecolor{currentstroke}%
\pgfsetdash{}{0pt}%
\pgfpathmoveto{\pgfqpoint{4.461559in}{2.695182in}}%
\pgfpathlineto{\pgfqpoint{4.475173in}{2.695827in}}%
\pgfpathlineto{\pgfqpoint{4.488796in}{2.696660in}}%
\pgfpathlineto{\pgfqpoint{4.502429in}{2.697682in}}%
\pgfpathlineto{\pgfqpoint{4.516073in}{2.698893in}}%
\pgfpathlineto{\pgfqpoint{4.523729in}{2.708995in}}%
\pgfpathlineto{\pgfqpoint{4.531382in}{2.719168in}}%
\pgfpathlineto{\pgfqpoint{4.539030in}{2.729414in}}%
\pgfpathlineto{\pgfqpoint{4.546674in}{2.739738in}}%
\pgfpathlineto{\pgfqpoint{4.533042in}{2.738911in}}%
\pgfpathlineto{\pgfqpoint{4.519419in}{2.738271in}}%
\pgfpathlineto{\pgfqpoint{4.505806in}{2.737820in}}%
\pgfpathlineto{\pgfqpoint{4.492204in}{2.737557in}}%
\pgfpathlineto{\pgfqpoint{4.484548in}{2.726839in}}%
\pgfpathlineto{\pgfqpoint{4.476889in}{2.716207in}}%
\pgfpathlineto{\pgfqpoint{4.469226in}{2.705656in}}%
\pgfpathlineto{\pgfqpoint{4.461559in}{2.695182in}}%
\pgfpathclose%
\pgfusepath{fill}%
\end{pgfscope}%
\begin{pgfscope}%
\pgfpathrectangle{\pgfqpoint{1.150000in}{0.150000in}}{\pgfqpoint{5.700000in}{5.700000in}}%
\pgfusepath{clip}%
\pgfsetbuttcap%
\pgfsetroundjoin%
\definecolor{currentfill}{rgb}{0.241237,0.296485,0.539709}%
\pgfsetfillcolor{currentfill}%
\pgfsetfillopacity{0.800000}%
\pgfsetlinewidth{0.000000pt}%
\definecolor{currentstroke}{rgb}{0.000000,0.000000,0.000000}%
\pgfsetstrokecolor{currentstroke}%
\pgfsetdash{}{0pt}%
\pgfpathmoveto{\pgfqpoint{4.631797in}{2.785439in}}%
\pgfpathlineto{\pgfqpoint{4.645470in}{2.786785in}}%
\pgfpathlineto{\pgfqpoint{4.659154in}{2.788317in}}%
\pgfpathlineto{\pgfqpoint{4.672848in}{2.790032in}}%
\pgfpathlineto{\pgfqpoint{4.686553in}{2.791933in}}%
\pgfpathlineto{\pgfqpoint{4.694154in}{2.801852in}}%
\pgfpathlineto{\pgfqpoint{4.701751in}{2.811864in}}%
\pgfpathlineto{\pgfqpoint{4.709344in}{2.821973in}}%
\pgfpathlineto{\pgfqpoint{4.716934in}{2.832184in}}%
\pgfpathlineto{\pgfqpoint{4.703241in}{2.830730in}}%
\pgfpathlineto{\pgfqpoint{4.689560in}{2.829461in}}%
\pgfpathlineto{\pgfqpoint{4.675889in}{2.828375in}}%
\pgfpathlineto{\pgfqpoint{4.662228in}{2.827474in}}%
\pgfpathlineto{\pgfqpoint{4.654626in}{2.816806in}}%
\pgfpathlineto{\pgfqpoint{4.647020in}{2.806247in}}%
\pgfpathlineto{\pgfqpoint{4.639411in}{2.795793in}}%
\pgfpathlineto{\pgfqpoint{4.631797in}{2.785439in}}%
\pgfpathclose%
\pgfusepath{fill}%
\end{pgfscope}%
\begin{pgfscope}%
\pgfpathrectangle{\pgfqpoint{1.150000in}{0.150000in}}{\pgfqpoint{5.700000in}{5.700000in}}%
\pgfusepath{clip}%
\pgfsetbuttcap%
\pgfsetroundjoin%
\definecolor{currentfill}{rgb}{0.262138,0.242286,0.520837}%
\pgfsetfillcolor{currentfill}%
\pgfsetfillopacity{0.800000}%
\pgfsetlinewidth{0.000000pt}%
\definecolor{currentstroke}{rgb}{0.000000,0.000000,0.000000}%
\pgfsetstrokecolor{currentstroke}%
\pgfsetdash{}{0pt}%
\pgfpathmoveto{\pgfqpoint{4.376448in}{2.651896in}}%
\pgfpathlineto{\pgfqpoint{4.390034in}{2.652128in}}%
\pgfpathlineto{\pgfqpoint{4.403629in}{2.652552in}}%
\pgfpathlineto{\pgfqpoint{4.417233in}{2.653167in}}%
\pgfpathlineto{\pgfqpoint{4.430847in}{2.653972in}}%
\pgfpathlineto{\pgfqpoint{4.438532in}{2.664180in}}%
\pgfpathlineto{\pgfqpoint{4.446212in}{2.674449in}}%
\pgfpathlineto{\pgfqpoint{4.453888in}{2.684781in}}%
\pgfpathlineto{\pgfqpoint{4.461559in}{2.695182in}}%
\pgfpathlineto{\pgfqpoint{4.447955in}{2.694728in}}%
\pgfpathlineto{\pgfqpoint{4.434361in}{2.694464in}}%
\pgfpathlineto{\pgfqpoint{4.420776in}{2.694390in}}%
\pgfpathlineto{\pgfqpoint{4.407200in}{2.694508in}}%
\pgfpathlineto{\pgfqpoint{4.399518in}{2.683745in}}%
\pgfpathlineto{\pgfqpoint{4.391833in}{2.673058in}}%
\pgfpathlineto{\pgfqpoint{4.384143in}{2.662443in}}%
\pgfpathlineto{\pgfqpoint{4.376448in}{2.651896in}}%
\pgfpathclose%
\pgfusepath{fill}%
\end{pgfscope}%
\begin{pgfscope}%
\pgfpathrectangle{\pgfqpoint{1.150000in}{0.150000in}}{\pgfqpoint{5.700000in}{5.700000in}}%
\pgfusepath{clip}%
\pgfsetbuttcap%
\pgfsetroundjoin%
\definecolor{currentfill}{rgb}{0.231674,0.318106,0.544834}%
\pgfsetfillcolor{currentfill}%
\pgfsetfillopacity{0.800000}%
\pgfsetlinewidth{0.000000pt}%
\definecolor{currentstroke}{rgb}{0.000000,0.000000,0.000000}%
\pgfsetstrokecolor{currentstroke}%
\pgfsetdash{}{0pt}%
\pgfpathmoveto{\pgfqpoint{4.716934in}{2.832184in}}%
\pgfpathlineto{\pgfqpoint{4.730637in}{2.833821in}}%
\pgfpathlineto{\pgfqpoint{4.744352in}{2.835642in}}%
\pgfpathlineto{\pgfqpoint{4.758078in}{2.837645in}}%
\pgfpathlineto{\pgfqpoint{4.771815in}{2.839831in}}%
\pgfpathlineto{\pgfqpoint{4.779388in}{2.849683in}}%
\pgfpathlineto{\pgfqpoint{4.786958in}{2.859641in}}%
\pgfpathlineto{\pgfqpoint{4.794524in}{2.869710in}}%
\pgfpathlineto{\pgfqpoint{4.802087in}{2.879895in}}%
\pgfpathlineto{\pgfqpoint{4.788363in}{2.878188in}}%
\pgfpathlineto{\pgfqpoint{4.774651in}{2.876662in}}%
\pgfpathlineto{\pgfqpoint{4.760950in}{2.875320in}}%
\pgfpathlineto{\pgfqpoint{4.747260in}{2.874160in}}%
\pgfpathlineto{\pgfqpoint{4.739683in}{2.863485in}}%
\pgfpathlineto{\pgfqpoint{4.732103in}{2.852935in}}%
\pgfpathlineto{\pgfqpoint{4.724520in}{2.842503in}}%
\pgfpathlineto{\pgfqpoint{4.716934in}{2.832184in}}%
\pgfpathclose%
\pgfusepath{fill}%
\end{pgfscope}%
\begin{pgfscope}%
\pgfpathrectangle{\pgfqpoint{1.150000in}{0.150000in}}{\pgfqpoint{5.700000in}{5.700000in}}%
\pgfusepath{clip}%
\pgfsetbuttcap%
\pgfsetroundjoin%
\definecolor{currentfill}{rgb}{0.281887,0.150881,0.465405}%
\pgfsetfillcolor{currentfill}%
\pgfsetfillopacity{0.800000}%
\pgfsetlinewidth{0.000000pt}%
\definecolor{currentstroke}{rgb}{0.000000,0.000000,0.000000}%
\pgfsetstrokecolor{currentstroke}%
\pgfsetdash{}{0pt}%
\pgfpathmoveto{\pgfqpoint{3.308989in}{2.465780in}}%
\pgfpathlineto{\pgfqpoint{3.322408in}{2.455693in}}%
\pgfpathlineto{\pgfqpoint{3.335826in}{2.445856in}}%
\pgfpathlineto{\pgfqpoint{3.349244in}{2.436266in}}%
\pgfpathlineto{\pgfqpoint{3.362661in}{2.426922in}}%
\pgfpathlineto{\pgfqpoint{3.370678in}{2.437506in}}%
\pgfpathlineto{\pgfqpoint{3.378688in}{2.448162in}}%
\pgfpathlineto{\pgfqpoint{3.386692in}{2.458891in}}%
\pgfpathlineto{\pgfqpoint{3.394691in}{2.469693in}}%
\pgfpathlineto{\pgfqpoint{3.381285in}{2.479071in}}%
\pgfpathlineto{\pgfqpoint{3.367880in}{2.488694in}}%
\pgfpathlineto{\pgfqpoint{3.354474in}{2.498565in}}%
\pgfpathlineto{\pgfqpoint{3.341068in}{2.508685in}}%
\pgfpathlineto{\pgfqpoint{3.333057in}{2.497837in}}%
\pgfpathlineto{\pgfqpoint{3.325041in}{2.487071in}}%
\pgfpathlineto{\pgfqpoint{3.317018in}{2.476386in}}%
\pgfpathlineto{\pgfqpoint{3.308989in}{2.465780in}}%
\pgfpathclose%
\pgfusepath{fill}%
\end{pgfscope}%
\begin{pgfscope}%
\pgfpathrectangle{\pgfqpoint{1.150000in}{0.150000in}}{\pgfqpoint{5.700000in}{5.700000in}}%
\pgfusepath{clip}%
\pgfsetbuttcap%
\pgfsetroundjoin%
\definecolor{currentfill}{rgb}{0.262138,0.242286,0.520837}%
\pgfsetfillcolor{currentfill}%
\pgfsetfillopacity{0.800000}%
\pgfsetlinewidth{0.000000pt}%
\definecolor{currentstroke}{rgb}{0.000000,0.000000,0.000000}%
\pgfsetstrokecolor{currentstroke}%
\pgfsetdash{}{0pt}%
\pgfpathmoveto{\pgfqpoint{3.007635in}{2.680981in}}%
\pgfpathlineto{\pgfqpoint{3.021133in}{2.665514in}}%
\pgfpathlineto{\pgfqpoint{3.034626in}{2.650338in}}%
\pgfpathlineto{\pgfqpoint{3.048113in}{2.635448in}}%
\pgfpathlineto{\pgfqpoint{3.061594in}{2.620843in}}%
\pgfpathlineto{\pgfqpoint{3.069708in}{2.631095in}}%
\pgfpathlineto{\pgfqpoint{3.077815in}{2.641455in}}%
\pgfpathlineto{\pgfqpoint{3.085914in}{2.651925in}}%
\pgfpathlineto{\pgfqpoint{3.094006in}{2.662505in}}%
\pgfpathlineto{\pgfqpoint{3.080540in}{2.677110in}}%
\pgfpathlineto{\pgfqpoint{3.067069in}{2.692000in}}%
\pgfpathlineto{\pgfqpoint{3.053592in}{2.707177in}}%
\pgfpathlineto{\pgfqpoint{3.040111in}{2.722643in}}%
\pgfpathlineto{\pgfqpoint{3.032003in}{2.712051in}}%
\pgfpathlineto{\pgfqpoint{3.023888in}{2.701577in}}%
\pgfpathlineto{\pgfqpoint{3.015765in}{2.691221in}}%
\pgfpathlineto{\pgfqpoint{3.007635in}{2.680981in}}%
\pgfpathclose%
\pgfusepath{fill}%
\end{pgfscope}%
\begin{pgfscope}%
\pgfpathrectangle{\pgfqpoint{1.150000in}{0.150000in}}{\pgfqpoint{5.700000in}{5.700000in}}%
\pgfusepath{clip}%
\pgfsetbuttcap%
\pgfsetroundjoin%
\definecolor{currentfill}{rgb}{0.252194,0.269783,0.531579}%
\pgfsetfillcolor{currentfill}%
\pgfsetfillopacity{0.800000}%
\pgfsetlinewidth{0.000000pt}%
\definecolor{currentstroke}{rgb}{0.000000,0.000000,0.000000}%
\pgfsetstrokecolor{currentstroke}%
\pgfsetdash{}{0pt}%
\pgfpathmoveto{\pgfqpoint{2.953581in}{2.745791in}}%
\pgfpathlineto{\pgfqpoint{2.967104in}{2.729142in}}%
\pgfpathlineto{\pgfqpoint{2.980621in}{2.712792in}}%
\pgfpathlineto{\pgfqpoint{2.994131in}{2.696739in}}%
\pgfpathlineto{\pgfqpoint{3.007635in}{2.680981in}}%
\pgfpathlineto{\pgfqpoint{3.015765in}{2.691221in}}%
\pgfpathlineto{\pgfqpoint{3.023888in}{2.701577in}}%
\pgfpathlineto{\pgfqpoint{3.032003in}{2.712051in}}%
\pgfpathlineto{\pgfqpoint{3.040111in}{2.722643in}}%
\pgfpathlineto{\pgfqpoint{3.026623in}{2.738401in}}%
\pgfpathlineto{\pgfqpoint{3.013130in}{2.754454in}}%
\pgfpathlineto{\pgfqpoint{2.999630in}{2.770803in}}%
\pgfpathlineto{\pgfqpoint{2.986124in}{2.787452in}}%
\pgfpathlineto{\pgfqpoint{2.978000in}{2.776849in}}%
\pgfpathlineto{\pgfqpoint{2.969868in}{2.766371in}}%
\pgfpathlineto{\pgfqpoint{2.961729in}{2.756019in}}%
\pgfpathlineto{\pgfqpoint{2.953581in}{2.745791in}}%
\pgfpathclose%
\pgfusepath{fill}%
\end{pgfscope}%
\begin{pgfscope}%
\pgfpathrectangle{\pgfqpoint{1.150000in}{0.150000in}}{\pgfqpoint{5.700000in}{5.700000in}}%
\pgfusepath{clip}%
\pgfsetbuttcap%
\pgfsetroundjoin%
\definecolor{currentfill}{rgb}{0.223925,0.334994,0.548053}%
\pgfsetfillcolor{currentfill}%
\pgfsetfillopacity{0.800000}%
\pgfsetlinewidth{0.000000pt}%
\definecolor{currentstroke}{rgb}{0.000000,0.000000,0.000000}%
\pgfsetstrokecolor{currentstroke}%
\pgfsetdash{}{0pt}%
\pgfpathmoveto{\pgfqpoint{4.802087in}{2.879895in}}%
\pgfpathlineto{\pgfqpoint{4.815822in}{2.881784in}}%
\pgfpathlineto{\pgfqpoint{4.829568in}{2.883855in}}%
\pgfpathlineto{\pgfqpoint{4.843326in}{2.886107in}}%
\pgfpathlineto{\pgfqpoint{4.857097in}{2.888539in}}%
\pgfpathlineto{\pgfqpoint{4.864642in}{2.898348in}}%
\pgfpathlineto{\pgfqpoint{4.872184in}{2.908277in}}%
\pgfpathlineto{\pgfqpoint{4.879724in}{2.918332in}}%
\pgfpathlineto{\pgfqpoint{4.887260in}{2.928519in}}%
\pgfpathlineto{\pgfqpoint{4.873505in}{2.926597in}}%
\pgfpathlineto{\pgfqpoint{4.859762in}{2.924855in}}%
\pgfpathlineto{\pgfqpoint{4.846030in}{2.923294in}}%
\pgfpathlineto{\pgfqpoint{4.832310in}{2.921914in}}%
\pgfpathlineto{\pgfqpoint{4.824758in}{2.911206in}}%
\pgfpathlineto{\pgfqpoint{4.817204in}{2.900638in}}%
\pgfpathlineto{\pgfqpoint{4.809647in}{2.890202in}}%
\pgfpathlineto{\pgfqpoint{4.802087in}{2.879895in}}%
\pgfpathclose%
\pgfusepath{fill}%
\end{pgfscope}%
\begin{pgfscope}%
\pgfpathrectangle{\pgfqpoint{1.150000in}{0.150000in}}{\pgfqpoint{5.700000in}{5.700000in}}%
\pgfusepath{clip}%
\pgfsetbuttcap%
\pgfsetroundjoin%
\definecolor{currentfill}{rgb}{0.267968,0.223549,0.512008}%
\pgfsetfillcolor{currentfill}%
\pgfsetfillopacity{0.800000}%
\pgfsetlinewidth{0.000000pt}%
\definecolor{currentstroke}{rgb}{0.000000,0.000000,0.000000}%
\pgfsetstrokecolor{currentstroke}%
\pgfsetdash{}{0pt}%
\pgfpathmoveto{\pgfqpoint{4.291335in}{2.610027in}}%
\pgfpathlineto{\pgfqpoint{4.304895in}{2.609806in}}%
\pgfpathlineto{\pgfqpoint{4.318463in}{2.609779in}}%
\pgfpathlineto{\pgfqpoint{4.332040in}{2.609946in}}%
\pgfpathlineto{\pgfqpoint{4.345626in}{2.610306in}}%
\pgfpathlineto{\pgfqpoint{4.353338in}{2.620621in}}%
\pgfpathlineto{\pgfqpoint{4.361046in}{2.630989in}}%
\pgfpathlineto{\pgfqpoint{4.368749in}{2.641412in}}%
\pgfpathlineto{\pgfqpoint{4.376448in}{2.651896in}}%
\pgfpathlineto{\pgfqpoint{4.362872in}{2.651855in}}%
\pgfpathlineto{\pgfqpoint{4.349304in}{2.652007in}}%
\pgfpathlineto{\pgfqpoint{4.335745in}{2.652353in}}%
\pgfpathlineto{\pgfqpoint{4.322195in}{2.652891in}}%
\pgfpathlineto{\pgfqpoint{4.314487in}{2.642078in}}%
\pgfpathlineto{\pgfqpoint{4.306774in}{2.631332in}}%
\pgfpathlineto{\pgfqpoint{4.299057in}{2.620649in}}%
\pgfpathlineto{\pgfqpoint{4.291335in}{2.610027in}}%
\pgfpathclose%
\pgfusepath{fill}%
\end{pgfscope}%
\begin{pgfscope}%
\pgfpathrectangle{\pgfqpoint{1.150000in}{0.150000in}}{\pgfqpoint{5.700000in}{5.700000in}}%
\pgfusepath{clip}%
\pgfsetbuttcap%
\pgfsetroundjoin%
\definecolor{currentfill}{rgb}{0.269308,0.218818,0.509577}%
\pgfsetfillcolor{currentfill}%
\pgfsetfillopacity{0.800000}%
\pgfsetlinewidth{0.000000pt}%
\definecolor{currentstroke}{rgb}{0.000000,0.000000,0.000000}%
\pgfsetstrokecolor{currentstroke}%
\pgfsetdash{}{0pt}%
\pgfpathmoveto{\pgfqpoint{3.061594in}{2.620843in}}%
\pgfpathlineto{\pgfqpoint{3.075071in}{2.606521in}}%
\pgfpathlineto{\pgfqpoint{3.088544in}{2.592479in}}%
\pgfpathlineto{\pgfqpoint{3.102012in}{2.578715in}}%
\pgfpathlineto{\pgfqpoint{3.115476in}{2.565227in}}%
\pgfpathlineto{\pgfqpoint{3.123574in}{2.575490in}}%
\pgfpathlineto{\pgfqpoint{3.131665in}{2.585853in}}%
\pgfpathlineto{\pgfqpoint{3.139748in}{2.596318in}}%
\pgfpathlineto{\pgfqpoint{3.147825in}{2.606885in}}%
\pgfpathlineto{\pgfqpoint{3.134376in}{2.620374in}}%
\pgfpathlineto{\pgfqpoint{3.120924in}{2.634139in}}%
\pgfpathlineto{\pgfqpoint{3.107467in}{2.648182in}}%
\pgfpathlineto{\pgfqpoint{3.094006in}{2.662505in}}%
\pgfpathlineto{\pgfqpoint{3.085914in}{2.651925in}}%
\pgfpathlineto{\pgfqpoint{3.077815in}{2.641455in}}%
\pgfpathlineto{\pgfqpoint{3.069708in}{2.631095in}}%
\pgfpathlineto{\pgfqpoint{3.061594in}{2.620843in}}%
\pgfpathclose%
\pgfusepath{fill}%
\end{pgfscope}%
\begin{pgfscope}%
\pgfpathrectangle{\pgfqpoint{1.150000in}{0.150000in}}{\pgfqpoint{5.700000in}{5.700000in}}%
\pgfusepath{clip}%
\pgfsetbuttcap%
\pgfsetroundjoin%
\definecolor{currentfill}{rgb}{0.241237,0.296485,0.539709}%
\pgfsetfillcolor{currentfill}%
\pgfsetfillopacity{0.800000}%
\pgfsetlinewidth{0.000000pt}%
\definecolor{currentstroke}{rgb}{0.000000,0.000000,0.000000}%
\pgfsetstrokecolor{currentstroke}%
\pgfsetdash{}{0pt}%
\pgfpathmoveto{\pgfqpoint{2.899415in}{2.815438in}}%
\pgfpathlineto{\pgfqpoint{2.912968in}{2.797564in}}%
\pgfpathlineto{\pgfqpoint{2.926513in}{2.779999in}}%
\pgfpathlineto{\pgfqpoint{2.940051in}{2.762743in}}%
\pgfpathlineto{\pgfqpoint{2.953581in}{2.745791in}}%
\pgfpathlineto{\pgfqpoint{2.961729in}{2.756019in}}%
\pgfpathlineto{\pgfqpoint{2.969868in}{2.766371in}}%
\pgfpathlineto{\pgfqpoint{2.978000in}{2.776849in}}%
\pgfpathlineto{\pgfqpoint{2.986124in}{2.787452in}}%
\pgfpathlineto{\pgfqpoint{2.972611in}{2.804403in}}%
\pgfpathlineto{\pgfqpoint{2.959091in}{2.821659in}}%
\pgfpathlineto{\pgfqpoint{2.945564in}{2.839223in}}%
\pgfpathlineto{\pgfqpoint{2.932029in}{2.857097in}}%
\pgfpathlineto{\pgfqpoint{2.923887in}{2.846482in}}%
\pgfpathlineto{\pgfqpoint{2.915738in}{2.836001in}}%
\pgfpathlineto{\pgfqpoint{2.907581in}{2.825653in}}%
\pgfpathlineto{\pgfqpoint{2.899415in}{2.815438in}}%
\pgfpathclose%
\pgfusepath{fill}%
\end{pgfscope}%
\begin{pgfscope}%
\pgfpathrectangle{\pgfqpoint{1.150000in}{0.150000in}}{\pgfqpoint{5.700000in}{5.700000in}}%
\pgfusepath{clip}%
\pgfsetbuttcap%
\pgfsetroundjoin%
\definecolor{currentfill}{rgb}{0.214298,0.355619,0.551184}%
\pgfsetfillcolor{currentfill}%
\pgfsetfillopacity{0.800000}%
\pgfsetlinewidth{0.000000pt}%
\definecolor{currentstroke}{rgb}{0.000000,0.000000,0.000000}%
\pgfsetstrokecolor{currentstroke}%
\pgfsetdash{}{0pt}%
\pgfpathmoveto{\pgfqpoint{4.887260in}{2.928519in}}%
\pgfpathlineto{\pgfqpoint{4.901027in}{2.930621in}}%
\pgfpathlineto{\pgfqpoint{4.914806in}{2.932903in}}%
\pgfpathlineto{\pgfqpoint{4.928597in}{2.935365in}}%
\pgfpathlineto{\pgfqpoint{4.942400in}{2.938006in}}%
\pgfpathlineto{\pgfqpoint{4.949918in}{2.947800in}}%
\pgfpathlineto{\pgfqpoint{4.957434in}{2.957731in}}%
\pgfpathlineto{\pgfqpoint{4.964947in}{2.967804in}}%
\pgfpathlineto{\pgfqpoint{4.972458in}{2.978026in}}%
\pgfpathlineto{\pgfqpoint{4.958672in}{2.975928in}}%
\pgfpathlineto{\pgfqpoint{4.944897in}{2.974008in}}%
\pgfpathlineto{\pgfqpoint{4.931134in}{2.972268in}}%
\pgfpathlineto{\pgfqpoint{4.917383in}{2.970707in}}%
\pgfpathlineto{\pgfqpoint{4.909856in}{2.959932in}}%
\pgfpathlineto{\pgfqpoint{4.902326in}{2.949313in}}%
\pgfpathlineto{\pgfqpoint{4.894794in}{2.938844in}}%
\pgfpathlineto{\pgfqpoint{4.887260in}{2.928519in}}%
\pgfpathclose%
\pgfusepath{fill}%
\end{pgfscope}%
\begin{pgfscope}%
\pgfpathrectangle{\pgfqpoint{1.150000in}{0.150000in}}{\pgfqpoint{5.700000in}{5.700000in}}%
\pgfusepath{clip}%
\pgfsetbuttcap%
\pgfsetroundjoin%
\definecolor{currentfill}{rgb}{0.271828,0.209303,0.504434}%
\pgfsetfillcolor{currentfill}%
\pgfsetfillopacity{0.800000}%
\pgfsetlinewidth{0.000000pt}%
\definecolor{currentstroke}{rgb}{0.000000,0.000000,0.000000}%
\pgfsetstrokecolor{currentstroke}%
\pgfsetdash{}{0pt}%
\pgfpathmoveto{\pgfqpoint{4.206214in}{2.569747in}}%
\pgfpathlineto{\pgfqpoint{4.219749in}{2.569032in}}%
\pgfpathlineto{\pgfqpoint{4.233292in}{2.568513in}}%
\pgfpathlineto{\pgfqpoint{4.246843in}{2.568190in}}%
\pgfpathlineto{\pgfqpoint{4.260403in}{2.568062in}}%
\pgfpathlineto{\pgfqpoint{4.268143in}{2.578481in}}%
\pgfpathlineto{\pgfqpoint{4.275878in}{2.588946in}}%
\pgfpathlineto{\pgfqpoint{4.283609in}{2.599460in}}%
\pgfpathlineto{\pgfqpoint{4.291335in}{2.610027in}}%
\pgfpathlineto{\pgfqpoint{4.277785in}{2.610442in}}%
\pgfpathlineto{\pgfqpoint{4.264242in}{2.611052in}}%
\pgfpathlineto{\pgfqpoint{4.250708in}{2.611858in}}%
\pgfpathlineto{\pgfqpoint{4.237182in}{2.612860in}}%
\pgfpathlineto{\pgfqpoint{4.229447in}{2.601995in}}%
\pgfpathlineto{\pgfqpoint{4.221708in}{2.591190in}}%
\pgfpathlineto{\pgfqpoint{4.213963in}{2.580442in}}%
\pgfpathlineto{\pgfqpoint{4.206214in}{2.569747in}}%
\pgfpathclose%
\pgfusepath{fill}%
\end{pgfscope}%
\begin{pgfscope}%
\pgfpathrectangle{\pgfqpoint{1.150000in}{0.150000in}}{\pgfqpoint{5.700000in}{5.700000in}}%
\pgfusepath{clip}%
\pgfsetbuttcap%
\pgfsetroundjoin%
\definecolor{currentfill}{rgb}{0.206756,0.371758,0.553117}%
\pgfsetfillcolor{currentfill}%
\pgfsetfillopacity{0.800000}%
\pgfsetlinewidth{0.000000pt}%
\definecolor{currentstroke}{rgb}{0.000000,0.000000,0.000000}%
\pgfsetstrokecolor{currentstroke}%
\pgfsetdash{}{0pt}%
\pgfpathmoveto{\pgfqpoint{4.972458in}{2.978026in}}%
\pgfpathlineto{\pgfqpoint{4.986257in}{2.980303in}}%
\pgfpathlineto{\pgfqpoint{5.000068in}{2.982758in}}%
\pgfpathlineto{\pgfqpoint{5.013892in}{2.985391in}}%
\pgfpathlineto{\pgfqpoint{5.027728in}{2.988202in}}%
\pgfpathlineto{\pgfqpoint{5.035220in}{2.998016in}}%
\pgfpathlineto{\pgfqpoint{5.042710in}{3.007985in}}%
\pgfpathlineto{\pgfqpoint{5.050198in}{3.018114in}}%
\pgfpathlineto{\pgfqpoint{5.057685in}{3.028410in}}%
\pgfpathlineto{\pgfqpoint{5.043866in}{3.026174in}}%
\pgfpathlineto{\pgfqpoint{5.030060in}{3.024115in}}%
\pgfpathlineto{\pgfqpoint{5.016267in}{3.022233in}}%
\pgfpathlineto{\pgfqpoint{5.002485in}{3.020529in}}%
\pgfpathlineto{\pgfqpoint{4.994981in}{3.009648in}}%
\pgfpathlineto{\pgfqpoint{4.987475in}{2.998942in}}%
\pgfpathlineto{\pgfqpoint{4.979967in}{2.988403in}}%
\pgfpathlineto{\pgfqpoint{4.972458in}{2.978026in}}%
\pgfpathclose%
\pgfusepath{fill}%
\end{pgfscope}%
\begin{pgfscope}%
\pgfpathrectangle{\pgfqpoint{1.150000in}{0.150000in}}{\pgfqpoint{5.700000in}{5.700000in}}%
\pgfusepath{clip}%
\pgfsetbuttcap%
\pgfsetroundjoin%
\definecolor{currentfill}{rgb}{0.282884,0.135920,0.453427}%
\pgfsetfillcolor{currentfill}%
\pgfsetfillopacity{0.800000}%
\pgfsetlinewidth{0.000000pt}%
\definecolor{currentstroke}{rgb}{0.000000,0.000000,0.000000}%
\pgfsetstrokecolor{currentstroke}%
\pgfsetdash{}{0pt}%
\pgfpathmoveto{\pgfqpoint{3.726433in}{2.417441in}}%
\pgfpathlineto{\pgfqpoint{3.739862in}{2.412734in}}%
\pgfpathlineto{\pgfqpoint{3.753294in}{2.408245in}}%
\pgfpathlineto{\pgfqpoint{3.766731in}{2.403971in}}%
\pgfpathlineto{\pgfqpoint{3.780173in}{2.399912in}}%
\pgfpathlineto{\pgfqpoint{3.788061in}{2.410731in}}%
\pgfpathlineto{\pgfqpoint{3.795945in}{2.421591in}}%
\pgfpathlineto{\pgfqpoint{3.803823in}{2.432493in}}%
\pgfpathlineto{\pgfqpoint{3.811696in}{2.443439in}}%
\pgfpathlineto{\pgfqpoint{3.798264in}{2.447627in}}%
\pgfpathlineto{\pgfqpoint{3.784835in}{2.452030in}}%
\pgfpathlineto{\pgfqpoint{3.771412in}{2.456648in}}%
\pgfpathlineto{\pgfqpoint{3.757992in}{2.461484in}}%
\pgfpathlineto{\pgfqpoint{3.750110in}{2.450397in}}%
\pgfpathlineto{\pgfqpoint{3.742223in}{2.439362in}}%
\pgfpathlineto{\pgfqpoint{3.734331in}{2.428377in}}%
\pgfpathlineto{\pgfqpoint{3.726433in}{2.417441in}}%
\pgfpathclose%
\pgfusepath{fill}%
\end{pgfscope}%
\begin{pgfscope}%
\pgfpathrectangle{\pgfqpoint{1.150000in}{0.150000in}}{\pgfqpoint{5.700000in}{5.700000in}}%
\pgfusepath{clip}%
\pgfsetbuttcap%
\pgfsetroundjoin%
\definecolor{currentfill}{rgb}{0.276194,0.190074,0.493001}%
\pgfsetfillcolor{currentfill}%
\pgfsetfillopacity{0.800000}%
\pgfsetlinewidth{0.000000pt}%
\definecolor{currentstroke}{rgb}{0.000000,0.000000,0.000000}%
\pgfsetstrokecolor{currentstroke}%
\pgfsetdash{}{0pt}%
\pgfpathmoveto{\pgfqpoint{4.121078in}{2.531254in}}%
\pgfpathlineto{\pgfqpoint{4.134590in}{2.530001in}}%
\pgfpathlineto{\pgfqpoint{4.148109in}{2.528948in}}%
\pgfpathlineto{\pgfqpoint{4.161637in}{2.528093in}}%
\pgfpathlineto{\pgfqpoint{4.175172in}{2.527436in}}%
\pgfpathlineto{\pgfqpoint{4.182940in}{2.537950in}}%
\pgfpathlineto{\pgfqpoint{4.190703in}{2.548505in}}%
\pgfpathlineto{\pgfqpoint{4.198461in}{2.559103in}}%
\pgfpathlineto{\pgfqpoint{4.206214in}{2.569747in}}%
\pgfpathlineto{\pgfqpoint{4.192688in}{2.570660in}}%
\pgfpathlineto{\pgfqpoint{4.179169in}{2.571770in}}%
\pgfpathlineto{\pgfqpoint{4.165658in}{2.573079in}}%
\pgfpathlineto{\pgfqpoint{4.152154in}{2.574587in}}%
\pgfpathlineto{\pgfqpoint{4.144392in}{2.563676in}}%
\pgfpathlineto{\pgfqpoint{4.136625in}{2.552818in}}%
\pgfpathlineto{\pgfqpoint{4.128854in}{2.542012in}}%
\pgfpathlineto{\pgfqpoint{4.121078in}{2.531254in}}%
\pgfpathclose%
\pgfusepath{fill}%
\end{pgfscope}%
\begin{pgfscope}%
\pgfpathrectangle{\pgfqpoint{1.150000in}{0.150000in}}{\pgfqpoint{5.700000in}{5.700000in}}%
\pgfusepath{clip}%
\pgfsetbuttcap%
\pgfsetroundjoin%
\definecolor{currentfill}{rgb}{0.275191,0.194905,0.496005}%
\pgfsetfillcolor{currentfill}%
\pgfsetfillopacity{0.800000}%
\pgfsetlinewidth{0.000000pt}%
\definecolor{currentstroke}{rgb}{0.000000,0.000000,0.000000}%
\pgfsetstrokecolor{currentstroke}%
\pgfsetdash{}{0pt}%
\pgfpathmoveto{\pgfqpoint{3.115476in}{2.565227in}}%
\pgfpathlineto{\pgfqpoint{3.128936in}{2.552013in}}%
\pgfpathlineto{\pgfqpoint{3.142393in}{2.539070in}}%
\pgfpathlineto{\pgfqpoint{3.155846in}{2.526397in}}%
\pgfpathlineto{\pgfqpoint{3.169296in}{2.513991in}}%
\pgfpathlineto{\pgfqpoint{3.177378in}{2.524264in}}%
\pgfpathlineto{\pgfqpoint{3.185454in}{2.534630in}}%
\pgfpathlineto{\pgfqpoint{3.193523in}{2.545090in}}%
\pgfpathlineto{\pgfqpoint{3.201586in}{2.555644in}}%
\pgfpathlineto{\pgfqpoint{3.188150in}{2.568051in}}%
\pgfpathlineto{\pgfqpoint{3.174712in}{2.580726in}}%
\pgfpathlineto{\pgfqpoint{3.161270in}{2.593670in}}%
\pgfpathlineto{\pgfqpoint{3.147825in}{2.606885in}}%
\pgfpathlineto{\pgfqpoint{3.139748in}{2.596318in}}%
\pgfpathlineto{\pgfqpoint{3.131665in}{2.585853in}}%
\pgfpathlineto{\pgfqpoint{3.123574in}{2.575490in}}%
\pgfpathlineto{\pgfqpoint{3.115476in}{2.565227in}}%
\pgfpathclose%
\pgfusepath{fill}%
\end{pgfscope}%
\begin{pgfscope}%
\pgfpathrectangle{\pgfqpoint{1.150000in}{0.150000in}}{\pgfqpoint{5.700000in}{5.700000in}}%
\pgfusepath{clip}%
\pgfsetbuttcap%
\pgfsetroundjoin%
\definecolor{currentfill}{rgb}{0.197636,0.391528,0.554969}%
\pgfsetfillcolor{currentfill}%
\pgfsetfillopacity{0.800000}%
\pgfsetlinewidth{0.000000pt}%
\definecolor{currentstroke}{rgb}{0.000000,0.000000,0.000000}%
\pgfsetstrokecolor{currentstroke}%
\pgfsetdash{}{0pt}%
\pgfpathmoveto{\pgfqpoint{5.057685in}{3.028410in}}%
\pgfpathlineto{\pgfqpoint{5.071515in}{3.030823in}}%
\pgfpathlineto{\pgfqpoint{5.085359in}{3.033413in}}%
\pgfpathlineto{\pgfqpoint{5.099215in}{3.036180in}}%
\pgfpathlineto{\pgfqpoint{5.113084in}{3.039122in}}%
\pgfpathlineto{\pgfqpoint{5.120551in}{3.048998in}}%
\pgfpathlineto{\pgfqpoint{5.128016in}{3.059046in}}%
\pgfpathlineto{\pgfqpoint{5.135480in}{3.069274in}}%
\pgfpathlineto{\pgfqpoint{5.142944in}{3.079688in}}%
\pgfpathlineto{\pgfqpoint{5.129094in}{3.077352in}}%
\pgfpathlineto{\pgfqpoint{5.115257in}{3.075191in}}%
\pgfpathlineto{\pgfqpoint{5.101432in}{3.073207in}}%
\pgfpathlineto{\pgfqpoint{5.087620in}{3.071398in}}%
\pgfpathlineto{\pgfqpoint{5.080137in}{3.060367in}}%
\pgfpathlineto{\pgfqpoint{5.072654in}{3.049530in}}%
\pgfpathlineto{\pgfqpoint{5.065170in}{3.038880in}}%
\pgfpathlineto{\pgfqpoint{5.057685in}{3.028410in}}%
\pgfpathclose%
\pgfusepath{fill}%
\end{pgfscope}%
\begin{pgfscope}%
\pgfpathrectangle{\pgfqpoint{1.150000in}{0.150000in}}{\pgfqpoint{5.700000in}{5.700000in}}%
\pgfusepath{clip}%
\pgfsetbuttcap%
\pgfsetroundjoin%
\definecolor{currentfill}{rgb}{0.227802,0.326594,0.546532}%
\pgfsetfillcolor{currentfill}%
\pgfsetfillopacity{0.800000}%
\pgfsetlinewidth{0.000000pt}%
\definecolor{currentstroke}{rgb}{0.000000,0.000000,0.000000}%
\pgfsetstrokecolor{currentstroke}%
\pgfsetdash{}{0pt}%
\pgfpathmoveto{\pgfqpoint{2.845118in}{2.890099in}}%
\pgfpathlineto{\pgfqpoint{2.858706in}{2.870954in}}%
\pgfpathlineto{\pgfqpoint{2.872284in}{2.852131in}}%
\pgfpathlineto{\pgfqpoint{2.885854in}{2.833626in}}%
\pgfpathlineto{\pgfqpoint{2.899415in}{2.815438in}}%
\pgfpathlineto{\pgfqpoint{2.907581in}{2.825653in}}%
\pgfpathlineto{\pgfqpoint{2.915738in}{2.836001in}}%
\pgfpathlineto{\pgfqpoint{2.923887in}{2.846482in}}%
\pgfpathlineto{\pgfqpoint{2.932029in}{2.857097in}}%
\pgfpathlineto{\pgfqpoint{2.918485in}{2.875284in}}%
\pgfpathlineto{\pgfqpoint{2.904934in}{2.893786in}}%
\pgfpathlineto{\pgfqpoint{2.891374in}{2.912608in}}%
\pgfpathlineto{\pgfqpoint{2.877805in}{2.931752in}}%
\pgfpathlineto{\pgfqpoint{2.869646in}{2.921127in}}%
\pgfpathlineto{\pgfqpoint{2.861479in}{2.910643in}}%
\pgfpathlineto{\pgfqpoint{2.853303in}{2.900301in}}%
\pgfpathlineto{\pgfqpoint{2.845118in}{2.890099in}}%
\pgfpathclose%
\pgfusepath{fill}%
\end{pgfscope}%
\begin{pgfscope}%
\pgfpathrectangle{\pgfqpoint{1.150000in}{0.150000in}}{\pgfqpoint{5.700000in}{5.700000in}}%
\pgfusepath{clip}%
\pgfsetbuttcap%
\pgfsetroundjoin%
\definecolor{currentfill}{rgb}{0.188923,0.410910,0.556326}%
\pgfsetfillcolor{currentfill}%
\pgfsetfillopacity{0.800000}%
\pgfsetlinewidth{0.000000pt}%
\definecolor{currentstroke}{rgb}{0.000000,0.000000,0.000000}%
\pgfsetstrokecolor{currentstroke}%
\pgfsetdash{}{0pt}%
\pgfpathmoveto{\pgfqpoint{5.142944in}{3.079688in}}%
\pgfpathlineto{\pgfqpoint{5.156807in}{3.082200in}}%
\pgfpathlineto{\pgfqpoint{5.170682in}{3.084887in}}%
\pgfpathlineto{\pgfqpoint{5.184571in}{3.087749in}}%
\pgfpathlineto{\pgfqpoint{5.198473in}{3.090787in}}%
\pgfpathlineto{\pgfqpoint{5.205916in}{3.100768in}}%
\pgfpathlineto{\pgfqpoint{5.213358in}{3.110943in}}%
\pgfpathlineto{\pgfqpoint{5.220799in}{3.121319in}}%
\pgfpathlineto{\pgfqpoint{5.228241in}{3.131902in}}%
\pgfpathlineto{\pgfqpoint{5.214360in}{3.129503in}}%
\pgfpathlineto{\pgfqpoint{5.200492in}{3.127279in}}%
\pgfpathlineto{\pgfqpoint{5.186636in}{3.125229in}}%
\pgfpathlineto{\pgfqpoint{5.172794in}{3.123354in}}%
\pgfpathlineto{\pgfqpoint{5.165331in}{3.112122in}}%
\pgfpathlineto{\pgfqpoint{5.157869in}{3.101105in}}%
\pgfpathlineto{\pgfqpoint{5.150407in}{3.090296in}}%
\pgfpathlineto{\pgfqpoint{5.142944in}{3.079688in}}%
\pgfpathclose%
\pgfusepath{fill}%
\end{pgfscope}%
\begin{pgfscope}%
\pgfpathrectangle{\pgfqpoint{1.150000in}{0.150000in}}{\pgfqpoint{5.700000in}{5.700000in}}%
\pgfusepath{clip}%
\pgfsetbuttcap%
\pgfsetroundjoin%
\definecolor{currentfill}{rgb}{0.278826,0.175490,0.483397}%
\pgfsetfillcolor{currentfill}%
\pgfsetfillopacity{0.800000}%
\pgfsetlinewidth{0.000000pt}%
\definecolor{currentstroke}{rgb}{0.000000,0.000000,0.000000}%
\pgfsetstrokecolor{currentstroke}%
\pgfsetdash{}{0pt}%
\pgfpathmoveto{\pgfqpoint{4.035916in}{2.494767in}}%
\pgfpathlineto{\pgfqpoint{4.049408in}{2.492933in}}%
\pgfpathlineto{\pgfqpoint{4.062906in}{2.491302in}}%
\pgfpathlineto{\pgfqpoint{4.076412in}{2.489872in}}%
\pgfpathlineto{\pgfqpoint{4.089925in}{2.488643in}}%
\pgfpathlineto{\pgfqpoint{4.097720in}{2.499238in}}%
\pgfpathlineto{\pgfqpoint{4.105511in}{2.509870in}}%
\pgfpathlineto{\pgfqpoint{4.113297in}{2.520541in}}%
\pgfpathlineto{\pgfqpoint{4.121078in}{2.531254in}}%
\pgfpathlineto{\pgfqpoint{4.107573in}{2.532707in}}%
\pgfpathlineto{\pgfqpoint{4.094076in}{2.534361in}}%
\pgfpathlineto{\pgfqpoint{4.080586in}{2.536216in}}%
\pgfpathlineto{\pgfqpoint{4.067103in}{2.538273in}}%
\pgfpathlineto{\pgfqpoint{4.059313in}{2.527325in}}%
\pgfpathlineto{\pgfqpoint{4.051519in}{2.516426in}}%
\pgfpathlineto{\pgfqpoint{4.043720in}{2.505575in}}%
\pgfpathlineto{\pgfqpoint{4.035916in}{2.494767in}}%
\pgfpathclose%
\pgfusepath{fill}%
\end{pgfscope}%
\begin{pgfscope}%
\pgfpathrectangle{\pgfqpoint{1.150000in}{0.150000in}}{\pgfqpoint{5.700000in}{5.700000in}}%
\pgfusepath{clip}%
\pgfsetbuttcap%
\pgfsetroundjoin%
\definecolor{currentfill}{rgb}{0.283072,0.130895,0.449241}%
\pgfsetfillcolor{currentfill}%
\pgfsetfillopacity{0.800000}%
\pgfsetlinewidth{0.000000pt}%
\definecolor{currentstroke}{rgb}{0.000000,0.000000,0.000000}%
\pgfsetstrokecolor{currentstroke}%
\pgfsetdash{}{0pt}%
\pgfpathmoveto{\pgfqpoint{3.501950in}{2.403330in}}%
\pgfpathlineto{\pgfqpoint{3.515363in}{2.396096in}}%
\pgfpathlineto{\pgfqpoint{3.528778in}{2.389092in}}%
\pgfpathlineto{\pgfqpoint{3.542194in}{2.382319in}}%
\pgfpathlineto{\pgfqpoint{3.555613in}{2.375774in}}%
\pgfpathlineto{\pgfqpoint{3.563572in}{2.386489in}}%
\pgfpathlineto{\pgfqpoint{3.571526in}{2.397257in}}%
\pgfpathlineto{\pgfqpoint{3.579475in}{2.408078in}}%
\pgfpathlineto{\pgfqpoint{3.587418in}{2.418953in}}%
\pgfpathlineto{\pgfqpoint{3.574009in}{2.425564in}}%
\pgfpathlineto{\pgfqpoint{3.560603in}{2.432404in}}%
\pgfpathlineto{\pgfqpoint{3.547199in}{2.439473in}}%
\pgfpathlineto{\pgfqpoint{3.533797in}{2.446773in}}%
\pgfpathlineto{\pgfqpoint{3.525843in}{2.435820in}}%
\pgfpathlineto{\pgfqpoint{3.517885in}{2.424929in}}%
\pgfpathlineto{\pgfqpoint{3.509920in}{2.414100in}}%
\pgfpathlineto{\pgfqpoint{3.501950in}{2.403330in}}%
\pgfpathclose%
\pgfusepath{fill}%
\end{pgfscope}%
\begin{pgfscope}%
\pgfpathrectangle{\pgfqpoint{1.150000in}{0.150000in}}{\pgfqpoint{5.700000in}{5.700000in}}%
\pgfusepath{clip}%
\pgfsetbuttcap%
\pgfsetroundjoin%
\definecolor{currentfill}{rgb}{0.282623,0.140926,0.457517}%
\pgfsetfillcolor{currentfill}%
\pgfsetfillopacity{0.800000}%
\pgfsetlinewidth{0.000000pt}%
\definecolor{currentstroke}{rgb}{0.000000,0.000000,0.000000}%
\pgfsetstrokecolor{currentstroke}%
\pgfsetdash{}{0pt}%
\pgfpathmoveto{\pgfqpoint{3.362661in}{2.426922in}}%
\pgfpathlineto{\pgfqpoint{3.376079in}{2.417823in}}%
\pgfpathlineto{\pgfqpoint{3.389496in}{2.408965in}}%
\pgfpathlineto{\pgfqpoint{3.402914in}{2.400349in}}%
\pgfpathlineto{\pgfqpoint{3.416332in}{2.391973in}}%
\pgfpathlineto{\pgfqpoint{3.424336in}{2.402534in}}%
\pgfpathlineto{\pgfqpoint{3.432334in}{2.413160in}}%
\pgfpathlineto{\pgfqpoint{3.440327in}{2.423850in}}%
\pgfpathlineto{\pgfqpoint{3.448314in}{2.434607in}}%
\pgfpathlineto{\pgfqpoint{3.434907in}{2.443018in}}%
\pgfpathlineto{\pgfqpoint{3.421501in}{2.451668in}}%
\pgfpathlineto{\pgfqpoint{3.408096in}{2.460559in}}%
\pgfpathlineto{\pgfqpoint{3.394691in}{2.469693in}}%
\pgfpathlineto{\pgfqpoint{3.386692in}{2.458891in}}%
\pgfpathlineto{\pgfqpoint{3.378688in}{2.448162in}}%
\pgfpathlineto{\pgfqpoint{3.370678in}{2.437506in}}%
\pgfpathlineto{\pgfqpoint{3.362661in}{2.426922in}}%
\pgfpathclose%
\pgfusepath{fill}%
\end{pgfscope}%
\begin{pgfscope}%
\pgfpathrectangle{\pgfqpoint{1.150000in}{0.150000in}}{\pgfqpoint{5.700000in}{5.700000in}}%
\pgfusepath{clip}%
\pgfsetbuttcap%
\pgfsetroundjoin%
\definecolor{currentfill}{rgb}{0.278826,0.175490,0.483397}%
\pgfsetfillcolor{currentfill}%
\pgfsetfillopacity{0.800000}%
\pgfsetlinewidth{0.000000pt}%
\definecolor{currentstroke}{rgb}{0.000000,0.000000,0.000000}%
\pgfsetstrokecolor{currentstroke}%
\pgfsetdash{}{0pt}%
\pgfpathmoveto{\pgfqpoint{3.169296in}{2.513991in}}%
\pgfpathlineto{\pgfqpoint{3.182743in}{2.501850in}}%
\pgfpathlineto{\pgfqpoint{3.196187in}{2.489973in}}%
\pgfpathlineto{\pgfqpoint{3.209629in}{2.478358in}}%
\pgfpathlineto{\pgfqpoint{3.223069in}{2.467003in}}%
\pgfpathlineto{\pgfqpoint{3.231137in}{2.477286in}}%
\pgfpathlineto{\pgfqpoint{3.239199in}{2.487654in}}%
\pgfpathlineto{\pgfqpoint{3.247254in}{2.498109in}}%
\pgfpathlineto{\pgfqpoint{3.255302in}{2.508649in}}%
\pgfpathlineto{\pgfqpoint{3.241876in}{2.520007in}}%
\pgfpathlineto{\pgfqpoint{3.228448in}{2.531624in}}%
\pgfpathlineto{\pgfqpoint{3.215018in}{2.543502in}}%
\pgfpathlineto{\pgfqpoint{3.201586in}{2.555644in}}%
\pgfpathlineto{\pgfqpoint{3.193523in}{2.545090in}}%
\pgfpathlineto{\pgfqpoint{3.185454in}{2.534630in}}%
\pgfpathlineto{\pgfqpoint{3.177378in}{2.524264in}}%
\pgfpathlineto{\pgfqpoint{3.169296in}{2.513991in}}%
\pgfpathclose%
\pgfusepath{fill}%
\end{pgfscope}%
\begin{pgfscope}%
\pgfpathrectangle{\pgfqpoint{1.150000in}{0.150000in}}{\pgfqpoint{5.700000in}{5.700000in}}%
\pgfusepath{clip}%
\pgfsetbuttcap%
\pgfsetroundjoin%
\definecolor{currentfill}{rgb}{0.180629,0.429975,0.557282}%
\pgfsetfillcolor{currentfill}%
\pgfsetfillopacity{0.800000}%
\pgfsetlinewidth{0.000000pt}%
\definecolor{currentstroke}{rgb}{0.000000,0.000000,0.000000}%
\pgfsetstrokecolor{currentstroke}%
\pgfsetdash{}{0pt}%
\pgfpathmoveto{\pgfqpoint{5.228241in}{3.131902in}}%
\pgfpathlineto{\pgfqpoint{5.242135in}{3.134475in}}%
\pgfpathlineto{\pgfqpoint{5.256043in}{3.137222in}}%
\pgfpathlineto{\pgfqpoint{5.269964in}{3.140143in}}%
\pgfpathlineto{\pgfqpoint{5.283898in}{3.143237in}}%
\pgfpathlineto{\pgfqpoint{5.291318in}{3.153377in}}%
\pgfpathlineto{\pgfqpoint{5.298739in}{3.163732in}}%
\pgfpathlineto{\pgfqpoint{5.306160in}{3.174310in}}%
\pgfpathlineto{\pgfqpoint{5.313582in}{3.185117in}}%
\pgfpathlineto{\pgfqpoint{5.299670in}{3.182693in}}%
\pgfpathlineto{\pgfqpoint{5.285771in}{3.180442in}}%
\pgfpathlineto{\pgfqpoint{5.271886in}{3.178365in}}%
\pgfpathlineto{\pgfqpoint{5.258013in}{3.176460in}}%
\pgfpathlineto{\pgfqpoint{5.250569in}{3.164972in}}%
\pgfpathlineto{\pgfqpoint{5.243125in}{3.153721in}}%
\pgfpathlineto{\pgfqpoint{5.235683in}{3.142700in}}%
\pgfpathlineto{\pgfqpoint{5.228241in}{3.131902in}}%
\pgfpathclose%
\pgfusepath{fill}%
\end{pgfscope}%
\begin{pgfscope}%
\pgfpathrectangle{\pgfqpoint{1.150000in}{0.150000in}}{\pgfqpoint{5.700000in}{5.700000in}}%
\pgfusepath{clip}%
\pgfsetbuttcap%
\pgfsetroundjoin%
\definecolor{currentfill}{rgb}{0.214298,0.355619,0.551184}%
\pgfsetfillcolor{currentfill}%
\pgfsetfillopacity{0.800000}%
\pgfsetlinewidth{0.000000pt}%
\definecolor{currentstroke}{rgb}{0.000000,0.000000,0.000000}%
\pgfsetstrokecolor{currentstroke}%
\pgfsetdash{}{0pt}%
\pgfpathmoveto{\pgfqpoint{2.790672in}{2.969963in}}%
\pgfpathlineto{\pgfqpoint{2.804299in}{2.949498in}}%
\pgfpathlineto{\pgfqpoint{2.817915in}{2.929368in}}%
\pgfpathlineto{\pgfqpoint{2.831522in}{2.909570in}}%
\pgfpathlineto{\pgfqpoint{2.845118in}{2.890099in}}%
\pgfpathlineto{\pgfqpoint{2.853303in}{2.900301in}}%
\pgfpathlineto{\pgfqpoint{2.861479in}{2.910643in}}%
\pgfpathlineto{\pgfqpoint{2.869646in}{2.921127in}}%
\pgfpathlineto{\pgfqpoint{2.877805in}{2.931752in}}%
\pgfpathlineto{\pgfqpoint{2.864227in}{2.951221in}}%
\pgfpathlineto{\pgfqpoint{2.850640in}{2.971018in}}%
\pgfpathlineto{\pgfqpoint{2.837043in}{2.991146in}}%
\pgfpathlineto{\pgfqpoint{2.823436in}{3.011609in}}%
\pgfpathlineto{\pgfqpoint{2.815258in}{3.000973in}}%
\pgfpathlineto{\pgfqpoint{2.807071in}{2.990487in}}%
\pgfpathlineto{\pgfqpoint{2.798876in}{2.980150in}}%
\pgfpathlineto{\pgfqpoint{2.790672in}{2.969963in}}%
\pgfpathclose%
\pgfusepath{fill}%
\end{pgfscope}%
\begin{pgfscope}%
\pgfpathrectangle{\pgfqpoint{1.150000in}{0.150000in}}{\pgfqpoint{5.700000in}{5.700000in}}%
\pgfusepath{clip}%
\pgfsetbuttcap%
\pgfsetroundjoin%
\definecolor{currentfill}{rgb}{0.280868,0.160771,0.472899}%
\pgfsetfillcolor{currentfill}%
\pgfsetfillopacity{0.800000}%
\pgfsetlinewidth{0.000000pt}%
\definecolor{currentstroke}{rgb}{0.000000,0.000000,0.000000}%
\pgfsetstrokecolor{currentstroke}%
\pgfsetdash{}{0pt}%
\pgfpathmoveto{\pgfqpoint{3.950720in}{2.460531in}}%
\pgfpathlineto{\pgfqpoint{3.964194in}{2.458072in}}%
\pgfpathlineto{\pgfqpoint{3.977674in}{2.455819in}}%
\pgfpathlineto{\pgfqpoint{3.991160in}{2.453770in}}%
\pgfpathlineto{\pgfqpoint{4.004653in}{2.451926in}}%
\pgfpathlineto{\pgfqpoint{4.012476in}{2.462583in}}%
\pgfpathlineto{\pgfqpoint{4.020294in}{2.473274in}}%
\pgfpathlineto{\pgfqpoint{4.028108in}{2.484001in}}%
\pgfpathlineto{\pgfqpoint{4.035916in}{2.494767in}}%
\pgfpathlineto{\pgfqpoint{4.022432in}{2.496804in}}%
\pgfpathlineto{\pgfqpoint{4.008954in}{2.499045in}}%
\pgfpathlineto{\pgfqpoint{3.995482in}{2.501490in}}%
\pgfpathlineto{\pgfqpoint{3.982017in}{2.504141in}}%
\pgfpathlineto{\pgfqpoint{3.974200in}{2.493171in}}%
\pgfpathlineto{\pgfqpoint{3.966378in}{2.482248in}}%
\pgfpathlineto{\pgfqpoint{3.958552in}{2.471368in}}%
\pgfpathlineto{\pgfqpoint{3.950720in}{2.460531in}}%
\pgfpathclose%
\pgfusepath{fill}%
\end{pgfscope}%
\begin{pgfscope}%
\pgfpathrectangle{\pgfqpoint{1.150000in}{0.150000in}}{\pgfqpoint{5.700000in}{5.700000in}}%
\pgfusepath{clip}%
\pgfsetbuttcap%
\pgfsetroundjoin%
\definecolor{currentfill}{rgb}{0.283072,0.130895,0.449241}%
\pgfsetfillcolor{currentfill}%
\pgfsetfillopacity{0.800000}%
\pgfsetlinewidth{0.000000pt}%
\definecolor{currentstroke}{rgb}{0.000000,0.000000,0.000000}%
\pgfsetstrokecolor{currentstroke}%
\pgfsetdash{}{0pt}%
\pgfpathmoveto{\pgfqpoint{3.641079in}{2.394769in}}%
\pgfpathlineto{\pgfqpoint{3.654501in}{2.389282in}}%
\pgfpathlineto{\pgfqpoint{3.667928in}{2.384016in}}%
\pgfpathlineto{\pgfqpoint{3.681358in}{2.378970in}}%
\pgfpathlineto{\pgfqpoint{3.694791in}{2.374143in}}%
\pgfpathlineto{\pgfqpoint{3.702710in}{2.384903in}}%
\pgfpathlineto{\pgfqpoint{3.710623in}{2.395705in}}%
\pgfpathlineto{\pgfqpoint{3.718530in}{2.406551in}}%
\pgfpathlineto{\pgfqpoint{3.726433in}{2.417441in}}%
\pgfpathlineto{\pgfqpoint{3.713009in}{2.422365in}}%
\pgfpathlineto{\pgfqpoint{3.699588in}{2.427509in}}%
\pgfpathlineto{\pgfqpoint{3.686171in}{2.432872in}}%
\pgfpathlineto{\pgfqpoint{3.672758in}{2.438457in}}%
\pgfpathlineto{\pgfqpoint{3.664846in}{2.427457in}}%
\pgfpathlineto{\pgfqpoint{3.656929in}{2.416510in}}%
\pgfpathlineto{\pgfqpoint{3.649006in}{2.405615in}}%
\pgfpathlineto{\pgfqpoint{3.641079in}{2.394769in}}%
\pgfpathclose%
\pgfusepath{fill}%
\end{pgfscope}%
\begin{pgfscope}%
\pgfpathrectangle{\pgfqpoint{1.150000in}{0.150000in}}{\pgfqpoint{5.700000in}{5.700000in}}%
\pgfusepath{clip}%
\pgfsetbuttcap%
\pgfsetroundjoin%
\definecolor{currentfill}{rgb}{0.172719,0.448791,0.557885}%
\pgfsetfillcolor{currentfill}%
\pgfsetfillopacity{0.800000}%
\pgfsetlinewidth{0.000000pt}%
\definecolor{currentstroke}{rgb}{0.000000,0.000000,0.000000}%
\pgfsetstrokecolor{currentstroke}%
\pgfsetdash{}{0pt}%
\pgfpathmoveto{\pgfqpoint{5.313582in}{3.185117in}}%
\pgfpathlineto{\pgfqpoint{5.327508in}{3.187714in}}%
\pgfpathlineto{\pgfqpoint{5.341447in}{3.190484in}}%
\pgfpathlineto{\pgfqpoint{5.355399in}{3.193426in}}%
\pgfpathlineto{\pgfqpoint{5.369366in}{3.196541in}}%
\pgfpathlineto{\pgfqpoint{5.376765in}{3.206896in}}%
\pgfpathlineto{\pgfqpoint{5.384166in}{3.217490in}}%
\pgfpathlineto{\pgfqpoint{5.391569in}{3.228330in}}%
\pgfpathlineto{\pgfqpoint{5.398974in}{3.239424in}}%
\pgfpathlineto{\pgfqpoint{5.385032in}{3.237012in}}%
\pgfpathlineto{\pgfqpoint{5.371103in}{3.234771in}}%
\pgfpathlineto{\pgfqpoint{5.357187in}{3.232703in}}%
\pgfpathlineto{\pgfqpoint{5.343285in}{3.230806in}}%
\pgfpathlineto{\pgfqpoint{5.335857in}{3.219000in}}%
\pgfpathlineto{\pgfqpoint{5.328430in}{3.207455in}}%
\pgfpathlineto{\pgfqpoint{5.321005in}{3.196163in}}%
\pgfpathlineto{\pgfqpoint{5.313582in}{3.185117in}}%
\pgfpathclose%
\pgfusepath{fill}%
\end{pgfscope}%
\begin{pgfscope}%
\pgfpathrectangle{\pgfqpoint{1.150000in}{0.150000in}}{\pgfqpoint{5.700000in}{5.700000in}}%
\pgfusepath{clip}%
\pgfsetbuttcap%
\pgfsetroundjoin%
\definecolor{currentfill}{rgb}{0.281887,0.150881,0.465405}%
\pgfsetfillcolor{currentfill}%
\pgfsetfillopacity{0.800000}%
\pgfsetlinewidth{0.000000pt}%
\definecolor{currentstroke}{rgb}{0.000000,0.000000,0.000000}%
\pgfsetstrokecolor{currentstroke}%
\pgfsetdash{}{0pt}%
\pgfpathmoveto{\pgfqpoint{3.865477in}{2.428814in}}%
\pgfpathlineto{\pgfqpoint{3.878935in}{2.425685in}}%
\pgfpathlineto{\pgfqpoint{3.892399in}{2.422766in}}%
\pgfpathlineto{\pgfqpoint{3.905869in}{2.420054in}}%
\pgfpathlineto{\pgfqpoint{3.919345in}{2.417550in}}%
\pgfpathlineto{\pgfqpoint{3.927196in}{2.428244in}}%
\pgfpathlineto{\pgfqpoint{3.935042in}{2.438971in}}%
\pgfpathlineto{\pgfqpoint{3.942884in}{2.449732in}}%
\pgfpathlineto{\pgfqpoint{3.950720in}{2.460531in}}%
\pgfpathlineto{\pgfqpoint{3.937253in}{2.463196in}}%
\pgfpathlineto{\pgfqpoint{3.923791in}{2.466068in}}%
\pgfpathlineto{\pgfqpoint{3.910336in}{2.469149in}}%
\pgfpathlineto{\pgfqpoint{3.896886in}{2.472438in}}%
\pgfpathlineto{\pgfqpoint{3.889041in}{2.461467in}}%
\pgfpathlineto{\pgfqpoint{3.881191in}{2.450541in}}%
\pgfpathlineto{\pgfqpoint{3.873337in}{2.439658in}}%
\pgfpathlineto{\pgfqpoint{3.865477in}{2.428814in}}%
\pgfpathclose%
\pgfusepath{fill}%
\end{pgfscope}%
\begin{pgfscope}%
\pgfpathrectangle{\pgfqpoint{1.150000in}{0.150000in}}{\pgfqpoint{5.700000in}{5.700000in}}%
\pgfusepath{clip}%
\pgfsetbuttcap%
\pgfsetroundjoin%
\definecolor{currentfill}{rgb}{0.280868,0.160771,0.472899}%
\pgfsetfillcolor{currentfill}%
\pgfsetfillopacity{0.800000}%
\pgfsetlinewidth{0.000000pt}%
\definecolor{currentstroke}{rgb}{0.000000,0.000000,0.000000}%
\pgfsetstrokecolor{currentstroke}%
\pgfsetdash{}{0pt}%
\pgfpathmoveto{\pgfqpoint{3.223069in}{2.467003in}}%
\pgfpathlineto{\pgfqpoint{3.236507in}{2.455905in}}%
\pgfpathlineto{\pgfqpoint{3.249943in}{2.445063in}}%
\pgfpathlineto{\pgfqpoint{3.263378in}{2.434476in}}%
\pgfpathlineto{\pgfqpoint{3.276811in}{2.424141in}}%
\pgfpathlineto{\pgfqpoint{3.284865in}{2.434434in}}%
\pgfpathlineto{\pgfqpoint{3.292913in}{2.444805in}}%
\pgfpathlineto{\pgfqpoint{3.300954in}{2.455253in}}%
\pgfpathlineto{\pgfqpoint{3.308989in}{2.465780in}}%
\pgfpathlineto{\pgfqpoint{3.295569in}{2.476117in}}%
\pgfpathlineto{\pgfqpoint{3.282148in}{2.486706in}}%
\pgfpathlineto{\pgfqpoint{3.268726in}{2.497550in}}%
\pgfpathlineto{\pgfqpoint{3.255302in}{2.508649in}}%
\pgfpathlineto{\pgfqpoint{3.247254in}{2.498109in}}%
\pgfpathlineto{\pgfqpoint{3.239199in}{2.487654in}}%
\pgfpathlineto{\pgfqpoint{3.231137in}{2.477286in}}%
\pgfpathlineto{\pgfqpoint{3.223069in}{2.467003in}}%
\pgfpathclose%
\pgfusepath{fill}%
\end{pgfscope}%
\begin{pgfscope}%
\pgfpathrectangle{\pgfqpoint{1.150000in}{0.150000in}}{\pgfqpoint{5.700000in}{5.700000in}}%
\pgfusepath{clip}%
\pgfsetbuttcap%
\pgfsetroundjoin%
\definecolor{currentfill}{rgb}{0.163625,0.471133,0.558148}%
\pgfsetfillcolor{currentfill}%
\pgfsetfillopacity{0.800000}%
\pgfsetlinewidth{0.000000pt}%
\definecolor{currentstroke}{rgb}{0.000000,0.000000,0.000000}%
\pgfsetstrokecolor{currentstroke}%
\pgfsetdash{}{0pt}%
\pgfpathmoveto{\pgfqpoint{5.398974in}{3.239424in}}%
\pgfpathlineto{\pgfqpoint{5.412930in}{3.242008in}}%
\pgfpathlineto{\pgfqpoint{5.426900in}{3.244763in}}%
\pgfpathlineto{\pgfqpoint{5.440883in}{3.247690in}}%
\pgfpathlineto{\pgfqpoint{5.454881in}{3.250789in}}%
\pgfpathlineto{\pgfqpoint{5.462263in}{3.261422in}}%
\pgfpathlineto{\pgfqpoint{5.469647in}{3.272319in}}%
\pgfpathlineto{\pgfqpoint{5.477034in}{3.283488in}}%
\pgfpathlineto{\pgfqpoint{5.484425in}{3.294936in}}%
\pgfpathlineto{\pgfqpoint{5.470453in}{3.292572in}}%
\pgfpathlineto{\pgfqpoint{5.456495in}{3.290378in}}%
\pgfpathlineto{\pgfqpoint{5.442550in}{3.288356in}}%
\pgfpathlineto{\pgfqpoint{5.428619in}{3.286504in}}%
\pgfpathlineto{\pgfqpoint{5.421203in}{3.274312in}}%
\pgfpathlineto{\pgfqpoint{5.413791in}{3.262407in}}%
\pgfpathlineto{\pgfqpoint{5.406381in}{3.250780in}}%
\pgfpathlineto{\pgfqpoint{5.398974in}{3.239424in}}%
\pgfpathclose%
\pgfusepath{fill}%
\end{pgfscope}%
\begin{pgfscope}%
\pgfpathrectangle{\pgfqpoint{1.150000in}{0.150000in}}{\pgfqpoint{5.700000in}{5.700000in}}%
\pgfusepath{clip}%
\pgfsetbuttcap%
\pgfsetroundjoin%
\definecolor{currentfill}{rgb}{0.199430,0.387607,0.554642}%
\pgfsetfillcolor{currentfill}%
\pgfsetfillopacity{0.800000}%
\pgfsetlinewidth{0.000000pt}%
\definecolor{currentstroke}{rgb}{0.000000,0.000000,0.000000}%
\pgfsetstrokecolor{currentstroke}%
\pgfsetdash{}{0pt}%
\pgfpathmoveto{\pgfqpoint{2.736055in}{3.055234in}}%
\pgfpathlineto{\pgfqpoint{2.749726in}{3.033398in}}%
\pgfpathlineto{\pgfqpoint{2.763386in}{3.011909in}}%
\pgfpathlineto{\pgfqpoint{2.777034in}{2.990766in}}%
\pgfpathlineto{\pgfqpoint{2.790672in}{2.969963in}}%
\pgfpathlineto{\pgfqpoint{2.798876in}{2.980150in}}%
\pgfpathlineto{\pgfqpoint{2.807071in}{2.990487in}}%
\pgfpathlineto{\pgfqpoint{2.815258in}{3.000973in}}%
\pgfpathlineto{\pgfqpoint{2.823436in}{3.011609in}}%
\pgfpathlineto{\pgfqpoint{2.809818in}{3.032409in}}%
\pgfpathlineto{\pgfqpoint{2.796189in}{3.053550in}}%
\pgfpathlineto{\pgfqpoint{2.782550in}{3.075036in}}%
\pgfpathlineto{\pgfqpoint{2.768899in}{3.096870in}}%
\pgfpathlineto{\pgfqpoint{2.760701in}{3.086224in}}%
\pgfpathlineto{\pgfqpoint{2.752495in}{3.075737in}}%
\pgfpathlineto{\pgfqpoint{2.744280in}{3.065407in}}%
\pgfpathlineto{\pgfqpoint{2.736055in}{3.055234in}}%
\pgfpathclose%
\pgfusepath{fill}%
\end{pgfscope}%
\begin{pgfscope}%
\pgfpathrectangle{\pgfqpoint{1.150000in}{0.150000in}}{\pgfqpoint{5.700000in}{5.700000in}}%
\pgfusepath{clip}%
\pgfsetbuttcap%
\pgfsetroundjoin%
\definecolor{currentfill}{rgb}{0.283072,0.130895,0.449241}%
\pgfsetfillcolor{currentfill}%
\pgfsetfillopacity{0.800000}%
\pgfsetlinewidth{0.000000pt}%
\definecolor{currentstroke}{rgb}{0.000000,0.000000,0.000000}%
\pgfsetstrokecolor{currentstroke}%
\pgfsetdash{}{0pt}%
\pgfpathmoveto{\pgfqpoint{3.416332in}{2.391973in}}%
\pgfpathlineto{\pgfqpoint{3.429751in}{2.383835in}}%
\pgfpathlineto{\pgfqpoint{3.443171in}{2.375933in}}%
\pgfpathlineto{\pgfqpoint{3.456592in}{2.368267in}}%
\pgfpathlineto{\pgfqpoint{3.470014in}{2.360834in}}%
\pgfpathlineto{\pgfqpoint{3.478006in}{2.371373in}}%
\pgfpathlineto{\pgfqpoint{3.485993in}{2.381968in}}%
\pgfpathlineto{\pgfqpoint{3.493975in}{2.392620in}}%
\pgfpathlineto{\pgfqpoint{3.501950in}{2.403330in}}%
\pgfpathlineto{\pgfqpoint{3.488539in}{2.410797in}}%
\pgfpathlineto{\pgfqpoint{3.475130in}{2.418498in}}%
\pgfpathlineto{\pgfqpoint{3.461721in}{2.426434in}}%
\pgfpathlineto{\pgfqpoint{3.448314in}{2.434607in}}%
\pgfpathlineto{\pgfqpoint{3.440327in}{2.423850in}}%
\pgfpathlineto{\pgfqpoint{3.432334in}{2.413160in}}%
\pgfpathlineto{\pgfqpoint{3.424336in}{2.402534in}}%
\pgfpathlineto{\pgfqpoint{3.416332in}{2.391973in}}%
\pgfpathclose%
\pgfusepath{fill}%
\end{pgfscope}%
\begin{pgfscope}%
\pgfpathrectangle{\pgfqpoint{1.150000in}{0.150000in}}{\pgfqpoint{5.700000in}{5.700000in}}%
\pgfusepath{clip}%
\pgfsetbuttcap%
\pgfsetroundjoin%
\definecolor{currentfill}{rgb}{0.283187,0.125848,0.444960}%
\pgfsetfillcolor{currentfill}%
\pgfsetfillopacity{0.800000}%
\pgfsetlinewidth{0.000000pt}%
\definecolor{currentstroke}{rgb}{0.000000,0.000000,0.000000}%
\pgfsetstrokecolor{currentstroke}%
\pgfsetdash{}{0pt}%
\pgfpathmoveto{\pgfqpoint{3.555613in}{2.375774in}}%
\pgfpathlineto{\pgfqpoint{3.569034in}{2.369456in}}%
\pgfpathlineto{\pgfqpoint{3.582458in}{2.363364in}}%
\pgfpathlineto{\pgfqpoint{3.595885in}{2.357498in}}%
\pgfpathlineto{\pgfqpoint{3.609314in}{2.351854in}}%
\pgfpathlineto{\pgfqpoint{3.617263in}{2.362515in}}%
\pgfpathlineto{\pgfqpoint{3.625207in}{2.373220in}}%
\pgfpathlineto{\pgfqpoint{3.633145in}{2.383971in}}%
\pgfpathlineto{\pgfqpoint{3.641079in}{2.394769in}}%
\pgfpathlineto{\pgfqpoint{3.627659in}{2.400478in}}%
\pgfpathlineto{\pgfqpoint{3.614242in}{2.406411in}}%
\pgfpathlineto{\pgfqpoint{3.600829in}{2.412569in}}%
\pgfpathlineto{\pgfqpoint{3.587418in}{2.418953in}}%
\pgfpathlineto{\pgfqpoint{3.579475in}{2.408078in}}%
\pgfpathlineto{\pgfqpoint{3.571526in}{2.397257in}}%
\pgfpathlineto{\pgfqpoint{3.563572in}{2.386489in}}%
\pgfpathlineto{\pgfqpoint{3.555613in}{2.375774in}}%
\pgfpathclose%
\pgfusepath{fill}%
\end{pgfscope}%
\begin{pgfscope}%
\pgfpathrectangle{\pgfqpoint{1.150000in}{0.150000in}}{\pgfqpoint{5.700000in}{5.700000in}}%
\pgfusepath{clip}%
\pgfsetbuttcap%
\pgfsetroundjoin%
\definecolor{currentfill}{rgb}{0.282884,0.135920,0.453427}%
\pgfsetfillcolor{currentfill}%
\pgfsetfillopacity{0.800000}%
\pgfsetlinewidth{0.000000pt}%
\definecolor{currentstroke}{rgb}{0.000000,0.000000,0.000000}%
\pgfsetstrokecolor{currentstroke}%
\pgfsetdash{}{0pt}%
\pgfpathmoveto{\pgfqpoint{3.780173in}{2.399912in}}%
\pgfpathlineto{\pgfqpoint{3.793619in}{2.396067in}}%
\pgfpathlineto{\pgfqpoint{3.807070in}{2.392435in}}%
\pgfpathlineto{\pgfqpoint{3.820526in}{2.389014in}}%
\pgfpathlineto{\pgfqpoint{3.833988in}{2.385805in}}%
\pgfpathlineto{\pgfqpoint{3.841868in}{2.396506in}}%
\pgfpathlineto{\pgfqpoint{3.849743in}{2.407241in}}%
\pgfpathlineto{\pgfqpoint{3.857612in}{2.418009in}}%
\pgfpathlineto{\pgfqpoint{3.865477in}{2.428814in}}%
\pgfpathlineto{\pgfqpoint{3.852024in}{2.432153in}}%
\pgfpathlineto{\pgfqpoint{3.838576in}{2.435703in}}%
\pgfpathlineto{\pgfqpoint{3.825134in}{2.439464in}}%
\pgfpathlineto{\pgfqpoint{3.811696in}{2.443439in}}%
\pgfpathlineto{\pgfqpoint{3.803823in}{2.432493in}}%
\pgfpathlineto{\pgfqpoint{3.795945in}{2.421591in}}%
\pgfpathlineto{\pgfqpoint{3.788061in}{2.410731in}}%
\pgfpathlineto{\pgfqpoint{3.780173in}{2.399912in}}%
\pgfpathclose%
\pgfusepath{fill}%
\end{pgfscope}%
\begin{pgfscope}%
\pgfpathrectangle{\pgfqpoint{1.150000in}{0.150000in}}{\pgfqpoint{5.700000in}{5.700000in}}%
\pgfusepath{clip}%
\pgfsetbuttcap%
\pgfsetroundjoin%
\definecolor{currentfill}{rgb}{0.156270,0.489624,0.557936}%
\pgfsetfillcolor{currentfill}%
\pgfsetfillopacity{0.800000}%
\pgfsetlinewidth{0.000000pt}%
\definecolor{currentstroke}{rgb}{0.000000,0.000000,0.000000}%
\pgfsetstrokecolor{currentstroke}%
\pgfsetdash{}{0pt}%
\pgfpathmoveto{\pgfqpoint{5.484425in}{3.294936in}}%
\pgfpathlineto{\pgfqpoint{5.498410in}{3.297470in}}%
\pgfpathlineto{\pgfqpoint{5.512410in}{3.300175in}}%
\pgfpathlineto{\pgfqpoint{5.526424in}{3.303050in}}%
\pgfpathlineto{\pgfqpoint{5.540452in}{3.306095in}}%
\pgfpathlineto{\pgfqpoint{5.547819in}{3.317076in}}%
\pgfpathlineto{\pgfqpoint{5.555190in}{3.328347in}}%
\pgfpathlineto{\pgfqpoint{5.562564in}{3.339916in}}%
\pgfpathlineto{\pgfqpoint{5.548557in}{3.337443in}}%
\pgfpathlineto{\pgfqpoint{5.534564in}{3.335139in}}%
\pgfpathlineto{\pgfqpoint{5.520584in}{3.333005in}}%
\pgfpathlineto{\pgfqpoint{5.506618in}{3.331041in}}%
\pgfpathlineto{\pgfqpoint{5.499216in}{3.318704in}}%
\pgfpathlineto{\pgfqpoint{5.491819in}{3.306671in}}%
\pgfpathlineto{\pgfqpoint{5.484425in}{3.294936in}}%
\pgfpathclose%
\pgfusepath{fill}%
\end{pgfscope}%
\begin{pgfscope}%
\pgfpathrectangle{\pgfqpoint{1.150000in}{0.150000in}}{\pgfqpoint{5.700000in}{5.700000in}}%
\pgfusepath{clip}%
\pgfsetbuttcap%
\pgfsetroundjoin%
\definecolor{currentfill}{rgb}{0.250425,0.274290,0.533103}%
\pgfsetfillcolor{currentfill}%
\pgfsetfillopacity{0.800000}%
\pgfsetlinewidth{0.000000pt}%
\definecolor{currentstroke}{rgb}{0.000000,0.000000,0.000000}%
\pgfsetstrokecolor{currentstroke}%
\pgfsetdash{}{0pt}%
\pgfpathmoveto{\pgfqpoint{4.516073in}{2.698893in}}%
\pgfpathlineto{\pgfqpoint{4.529727in}{2.700291in}}%
\pgfpathlineto{\pgfqpoint{4.543391in}{2.701877in}}%
\pgfpathlineto{\pgfqpoint{4.557065in}{2.703650in}}%
\pgfpathlineto{\pgfqpoint{4.570751in}{2.705609in}}%
\pgfpathlineto{\pgfqpoint{4.578396in}{2.715340in}}%
\pgfpathlineto{\pgfqpoint{4.586037in}{2.725132in}}%
\pgfpathlineto{\pgfqpoint{4.593674in}{2.734991in}}%
\pgfpathlineto{\pgfqpoint{4.601307in}{2.744920in}}%
\pgfpathlineto{\pgfqpoint{4.587633in}{2.743345in}}%
\pgfpathlineto{\pgfqpoint{4.573970in}{2.741956in}}%
\pgfpathlineto{\pgfqpoint{4.560317in}{2.740754in}}%
\pgfpathlineto{\pgfqpoint{4.546674in}{2.739738in}}%
\pgfpathlineto{\pgfqpoint{4.539030in}{2.729414in}}%
\pgfpathlineto{\pgfqpoint{4.531382in}{2.719168in}}%
\pgfpathlineto{\pgfqpoint{4.523729in}{2.708995in}}%
\pgfpathlineto{\pgfqpoint{4.516073in}{2.698893in}}%
\pgfpathclose%
\pgfusepath{fill}%
\end{pgfscope}%
\begin{pgfscope}%
\pgfpathrectangle{\pgfqpoint{1.150000in}{0.150000in}}{\pgfqpoint{5.700000in}{5.700000in}}%
\pgfusepath{clip}%
\pgfsetbuttcap%
\pgfsetroundjoin%
\definecolor{currentfill}{rgb}{0.258965,0.251537,0.524736}%
\pgfsetfillcolor{currentfill}%
\pgfsetfillopacity{0.800000}%
\pgfsetlinewidth{0.000000pt}%
\definecolor{currentstroke}{rgb}{0.000000,0.000000,0.000000}%
\pgfsetstrokecolor{currentstroke}%
\pgfsetdash{}{0pt}%
\pgfpathmoveto{\pgfqpoint{4.430847in}{2.653972in}}%
\pgfpathlineto{\pgfqpoint{4.444471in}{2.654968in}}%
\pgfpathlineto{\pgfqpoint{4.458105in}{2.656153in}}%
\pgfpathlineto{\pgfqpoint{4.471749in}{2.657527in}}%
\pgfpathlineto{\pgfqpoint{4.485403in}{2.659090in}}%
\pgfpathlineto{\pgfqpoint{4.493077in}{2.668957in}}%
\pgfpathlineto{\pgfqpoint{4.500747in}{2.678878in}}%
\pgfpathlineto{\pgfqpoint{4.508412in}{2.688855in}}%
\pgfpathlineto{\pgfqpoint{4.516073in}{2.698893in}}%
\pgfpathlineto{\pgfqpoint{4.502429in}{2.697682in}}%
\pgfpathlineto{\pgfqpoint{4.488796in}{2.696660in}}%
\pgfpathlineto{\pgfqpoint{4.475173in}{2.695827in}}%
\pgfpathlineto{\pgfqpoint{4.461559in}{2.695182in}}%
\pgfpathlineto{\pgfqpoint{4.453888in}{2.684781in}}%
\pgfpathlineto{\pgfqpoint{4.446212in}{2.674449in}}%
\pgfpathlineto{\pgfqpoint{4.438532in}{2.664180in}}%
\pgfpathlineto{\pgfqpoint{4.430847in}{2.653972in}}%
\pgfpathclose%
\pgfusepath{fill}%
\end{pgfscope}%
\begin{pgfscope}%
\pgfpathrectangle{\pgfqpoint{1.150000in}{0.150000in}}{\pgfqpoint{5.700000in}{5.700000in}}%
\pgfusepath{clip}%
\pgfsetbuttcap%
\pgfsetroundjoin%
\definecolor{currentfill}{rgb}{0.243113,0.292092,0.538516}%
\pgfsetfillcolor{currentfill}%
\pgfsetfillopacity{0.800000}%
\pgfsetlinewidth{0.000000pt}%
\definecolor{currentstroke}{rgb}{0.000000,0.000000,0.000000}%
\pgfsetstrokecolor{currentstroke}%
\pgfsetdash{}{0pt}%
\pgfpathmoveto{\pgfqpoint{4.601307in}{2.744920in}}%
\pgfpathlineto{\pgfqpoint{4.614991in}{2.746681in}}%
\pgfpathlineto{\pgfqpoint{4.628687in}{2.748628in}}%
\pgfpathlineto{\pgfqpoint{4.642394in}{2.750760in}}%
\pgfpathlineto{\pgfqpoint{4.656111in}{2.753076in}}%
\pgfpathlineto{\pgfqpoint{4.663728in}{2.762677in}}%
\pgfpathlineto{\pgfqpoint{4.671340in}{2.772350in}}%
\pgfpathlineto{\pgfqpoint{4.678949in}{2.782100in}}%
\pgfpathlineto{\pgfqpoint{4.686553in}{2.791933in}}%
\pgfpathlineto{\pgfqpoint{4.672848in}{2.790032in}}%
\pgfpathlineto{\pgfqpoint{4.659154in}{2.788317in}}%
\pgfpathlineto{\pgfqpoint{4.645470in}{2.786785in}}%
\pgfpathlineto{\pgfqpoint{4.631797in}{2.785439in}}%
\pgfpathlineto{\pgfqpoint{4.624181in}{2.775180in}}%
\pgfpathlineto{\pgfqpoint{4.616560in}{2.765010in}}%
\pgfpathlineto{\pgfqpoint{4.608935in}{2.754925in}}%
\pgfpathlineto{\pgfqpoint{4.601307in}{2.744920in}}%
\pgfpathclose%
\pgfusepath{fill}%
\end{pgfscope}%
\begin{pgfscope}%
\pgfpathrectangle{\pgfqpoint{1.150000in}{0.150000in}}{\pgfqpoint{5.700000in}{5.700000in}}%
\pgfusepath{clip}%
\pgfsetbuttcap%
\pgfsetroundjoin%
\definecolor{currentfill}{rgb}{0.265145,0.232956,0.516599}%
\pgfsetfillcolor{currentfill}%
\pgfsetfillopacity{0.800000}%
\pgfsetlinewidth{0.000000pt}%
\definecolor{currentstroke}{rgb}{0.000000,0.000000,0.000000}%
\pgfsetstrokecolor{currentstroke}%
\pgfsetdash{}{0pt}%
\pgfpathmoveto{\pgfqpoint{4.345626in}{2.610306in}}%
\pgfpathlineto{\pgfqpoint{4.359221in}{2.610858in}}%
\pgfpathlineto{\pgfqpoint{4.372826in}{2.611601in}}%
\pgfpathlineto{\pgfqpoint{4.386440in}{2.612536in}}%
\pgfpathlineto{\pgfqpoint{4.400064in}{2.613662in}}%
\pgfpathlineto{\pgfqpoint{4.407767in}{2.623669in}}%
\pgfpathlineto{\pgfqpoint{4.415465in}{2.633720in}}%
\pgfpathlineto{\pgfqpoint{4.423159in}{2.643820in}}%
\pgfpathlineto{\pgfqpoint{4.430847in}{2.653972in}}%
\pgfpathlineto{\pgfqpoint{4.417233in}{2.653167in}}%
\pgfpathlineto{\pgfqpoint{4.403629in}{2.652552in}}%
\pgfpathlineto{\pgfqpoint{4.390034in}{2.652128in}}%
\pgfpathlineto{\pgfqpoint{4.376448in}{2.651896in}}%
\pgfpathlineto{\pgfqpoint{4.368749in}{2.641412in}}%
\pgfpathlineto{\pgfqpoint{4.361046in}{2.630989in}}%
\pgfpathlineto{\pgfqpoint{4.353338in}{2.620621in}}%
\pgfpathlineto{\pgfqpoint{4.345626in}{2.610306in}}%
\pgfpathclose%
\pgfusepath{fill}%
\end{pgfscope}%
\begin{pgfscope}%
\pgfpathrectangle{\pgfqpoint{1.150000in}{0.150000in}}{\pgfqpoint{5.700000in}{5.700000in}}%
\pgfusepath{clip}%
\pgfsetbuttcap%
\pgfsetroundjoin%
\definecolor{currentfill}{rgb}{0.282290,0.145912,0.461510}%
\pgfsetfillcolor{currentfill}%
\pgfsetfillopacity{0.800000}%
\pgfsetlinewidth{0.000000pt}%
\definecolor{currentstroke}{rgb}{0.000000,0.000000,0.000000}%
\pgfsetstrokecolor{currentstroke}%
\pgfsetdash{}{0pt}%
\pgfpathmoveto{\pgfqpoint{3.276811in}{2.424141in}}%
\pgfpathlineto{\pgfqpoint{3.290243in}{2.414057in}}%
\pgfpathlineto{\pgfqpoint{3.303674in}{2.404223in}}%
\pgfpathlineto{\pgfqpoint{3.317105in}{2.394636in}}%
\pgfpathlineto{\pgfqpoint{3.330535in}{2.385294in}}%
\pgfpathlineto{\pgfqpoint{3.338576in}{2.395596in}}%
\pgfpathlineto{\pgfqpoint{3.346610in}{2.405968in}}%
\pgfpathlineto{\pgfqpoint{3.354639in}{2.416410in}}%
\pgfpathlineto{\pgfqpoint{3.362661in}{2.426922in}}%
\pgfpathlineto{\pgfqpoint{3.349244in}{2.436266in}}%
\pgfpathlineto{\pgfqpoint{3.335826in}{2.445856in}}%
\pgfpathlineto{\pgfqpoint{3.322408in}{2.455693in}}%
\pgfpathlineto{\pgfqpoint{3.308989in}{2.465780in}}%
\pgfpathlineto{\pgfqpoint{3.300954in}{2.455253in}}%
\pgfpathlineto{\pgfqpoint{3.292913in}{2.444805in}}%
\pgfpathlineto{\pgfqpoint{3.284865in}{2.434434in}}%
\pgfpathlineto{\pgfqpoint{3.276811in}{2.424141in}}%
\pgfpathclose%
\pgfusepath{fill}%
\end{pgfscope}%
\begin{pgfscope}%
\pgfpathrectangle{\pgfqpoint{1.150000in}{0.150000in}}{\pgfqpoint{5.700000in}{5.700000in}}%
\pgfusepath{clip}%
\pgfsetbuttcap%
\pgfsetroundjoin%
\definecolor{currentfill}{rgb}{0.235526,0.309527,0.542944}%
\pgfsetfillcolor{currentfill}%
\pgfsetfillopacity{0.800000}%
\pgfsetlinewidth{0.000000pt}%
\definecolor{currentstroke}{rgb}{0.000000,0.000000,0.000000}%
\pgfsetstrokecolor{currentstroke}%
\pgfsetdash{}{0pt}%
\pgfpathmoveto{\pgfqpoint{4.686553in}{2.791933in}}%
\pgfpathlineto{\pgfqpoint{4.700270in}{2.794017in}}%
\pgfpathlineto{\pgfqpoint{4.713998in}{2.796285in}}%
\pgfpathlineto{\pgfqpoint{4.727737in}{2.798736in}}%
\pgfpathlineto{\pgfqpoint{4.741488in}{2.801370in}}%
\pgfpathlineto{\pgfqpoint{4.749075in}{2.810854in}}%
\pgfpathlineto{\pgfqpoint{4.756659in}{2.820421in}}%
\pgfpathlineto{\pgfqpoint{4.764239in}{2.830079in}}%
\pgfpathlineto{\pgfqpoint{4.771815in}{2.839831in}}%
\pgfpathlineto{\pgfqpoint{4.758078in}{2.837645in}}%
\pgfpathlineto{\pgfqpoint{4.744352in}{2.835642in}}%
\pgfpathlineto{\pgfqpoint{4.730637in}{2.833821in}}%
\pgfpathlineto{\pgfqpoint{4.716934in}{2.832184in}}%
\pgfpathlineto{\pgfqpoint{4.709344in}{2.821973in}}%
\pgfpathlineto{\pgfqpoint{4.701751in}{2.811864in}}%
\pgfpathlineto{\pgfqpoint{4.694154in}{2.801852in}}%
\pgfpathlineto{\pgfqpoint{4.686553in}{2.791933in}}%
\pgfpathclose%
\pgfusepath{fill}%
\end{pgfscope}%
\begin{pgfscope}%
\pgfpathrectangle{\pgfqpoint{1.150000in}{0.150000in}}{\pgfqpoint{5.700000in}{5.700000in}}%
\pgfusepath{clip}%
\pgfsetbuttcap%
\pgfsetroundjoin%
\definecolor{currentfill}{rgb}{0.265145,0.232956,0.516599}%
\pgfsetfillcolor{currentfill}%
\pgfsetfillopacity{0.800000}%
\pgfsetlinewidth{0.000000pt}%
\definecolor{currentstroke}{rgb}{0.000000,0.000000,0.000000}%
\pgfsetstrokecolor{currentstroke}%
\pgfsetdash{}{0pt}%
\pgfpathmoveto{\pgfqpoint{2.975037in}{2.641184in}}%
\pgfpathlineto{\pgfqpoint{2.988553in}{2.625686in}}%
\pgfpathlineto{\pgfqpoint{3.002062in}{2.610478in}}%
\pgfpathlineto{\pgfqpoint{3.015567in}{2.595557in}}%
\pgfpathlineto{\pgfqpoint{3.029066in}{2.580921in}}%
\pgfpathlineto{\pgfqpoint{3.037209in}{2.590739in}}%
\pgfpathlineto{\pgfqpoint{3.045345in}{2.600666in}}%
\pgfpathlineto{\pgfqpoint{3.053473in}{2.610701in}}%
\pgfpathlineto{\pgfqpoint{3.061594in}{2.620843in}}%
\pgfpathlineto{\pgfqpoint{3.048113in}{2.635448in}}%
\pgfpathlineto{\pgfqpoint{3.034626in}{2.650338in}}%
\pgfpathlineto{\pgfqpoint{3.021133in}{2.665514in}}%
\pgfpathlineto{\pgfqpoint{3.007635in}{2.680981in}}%
\pgfpathlineto{\pgfqpoint{2.999497in}{2.670857in}}%
\pgfpathlineto{\pgfqpoint{2.991352in}{2.660850in}}%
\pgfpathlineto{\pgfqpoint{2.983198in}{2.650959in}}%
\pgfpathlineto{\pgfqpoint{2.975037in}{2.641184in}}%
\pgfpathclose%
\pgfusepath{fill}%
\end{pgfscope}%
\begin{pgfscope}%
\pgfpathrectangle{\pgfqpoint{1.150000in}{0.150000in}}{\pgfqpoint{5.700000in}{5.700000in}}%
\pgfusepath{clip}%
\pgfsetbuttcap%
\pgfsetroundjoin%
\definecolor{currentfill}{rgb}{0.255645,0.260703,0.528312}%
\pgfsetfillcolor{currentfill}%
\pgfsetfillopacity{0.800000}%
\pgfsetlinewidth{0.000000pt}%
\definecolor{currentstroke}{rgb}{0.000000,0.000000,0.000000}%
\pgfsetstrokecolor{currentstroke}%
\pgfsetdash{}{0pt}%
\pgfpathmoveto{\pgfqpoint{2.920911in}{2.706123in}}%
\pgfpathlineto{\pgfqpoint{2.934452in}{2.689441in}}%
\pgfpathlineto{\pgfqpoint{2.947987in}{2.673059in}}%
\pgfpathlineto{\pgfqpoint{2.961515in}{2.656974in}}%
\pgfpathlineto{\pgfqpoint{2.975037in}{2.641184in}}%
\pgfpathlineto{\pgfqpoint{2.983198in}{2.650959in}}%
\pgfpathlineto{\pgfqpoint{2.991352in}{2.660850in}}%
\pgfpathlineto{\pgfqpoint{2.999497in}{2.670857in}}%
\pgfpathlineto{\pgfqpoint{3.007635in}{2.680981in}}%
\pgfpathlineto{\pgfqpoint{2.994131in}{2.696739in}}%
\pgfpathlineto{\pgfqpoint{2.980621in}{2.712792in}}%
\pgfpathlineto{\pgfqpoint{2.967104in}{2.729142in}}%
\pgfpathlineto{\pgfqpoint{2.953581in}{2.745791in}}%
\pgfpathlineto{\pgfqpoint{2.945426in}{2.735688in}}%
\pgfpathlineto{\pgfqpoint{2.937262in}{2.725709in}}%
\pgfpathlineto{\pgfqpoint{2.929090in}{2.715854in}}%
\pgfpathlineto{\pgfqpoint{2.920911in}{2.706123in}}%
\pgfpathclose%
\pgfusepath{fill}%
\end{pgfscope}%
\begin{pgfscope}%
\pgfpathrectangle{\pgfqpoint{1.150000in}{0.150000in}}{\pgfqpoint{5.700000in}{5.700000in}}%
\pgfusepath{clip}%
\pgfsetbuttcap%
\pgfsetroundjoin%
\definecolor{currentfill}{rgb}{0.269308,0.218818,0.509577}%
\pgfsetfillcolor{currentfill}%
\pgfsetfillopacity{0.800000}%
\pgfsetlinewidth{0.000000pt}%
\definecolor{currentstroke}{rgb}{0.000000,0.000000,0.000000}%
\pgfsetstrokecolor{currentstroke}%
\pgfsetdash{}{0pt}%
\pgfpathmoveto{\pgfqpoint{4.260403in}{2.568062in}}%
\pgfpathlineto{\pgfqpoint{4.273971in}{2.568129in}}%
\pgfpathlineto{\pgfqpoint{4.287549in}{2.568391in}}%
\pgfpathlineto{\pgfqpoint{4.301135in}{2.568846in}}%
\pgfpathlineto{\pgfqpoint{4.314730in}{2.569494in}}%
\pgfpathlineto{\pgfqpoint{4.322461in}{2.579636in}}%
\pgfpathlineto{\pgfqpoint{4.330188in}{2.589817in}}%
\pgfpathlineto{\pgfqpoint{4.337909in}{2.600039in}}%
\pgfpathlineto{\pgfqpoint{4.345626in}{2.610306in}}%
\pgfpathlineto{\pgfqpoint{4.332040in}{2.609946in}}%
\pgfpathlineto{\pgfqpoint{4.318463in}{2.609779in}}%
\pgfpathlineto{\pgfqpoint{4.304895in}{2.609806in}}%
\pgfpathlineto{\pgfqpoint{4.291335in}{2.610027in}}%
\pgfpathlineto{\pgfqpoint{4.283609in}{2.599460in}}%
\pgfpathlineto{\pgfqpoint{4.275878in}{2.588946in}}%
\pgfpathlineto{\pgfqpoint{4.268143in}{2.578481in}}%
\pgfpathlineto{\pgfqpoint{4.260403in}{2.568062in}}%
\pgfpathclose%
\pgfusepath{fill}%
\end{pgfscope}%
\begin{pgfscope}%
\pgfpathrectangle{\pgfqpoint{1.150000in}{0.150000in}}{\pgfqpoint{5.700000in}{5.700000in}}%
\pgfusepath{clip}%
\pgfsetbuttcap%
\pgfsetroundjoin%
\definecolor{currentfill}{rgb}{0.225863,0.330805,0.547314}%
\pgfsetfillcolor{currentfill}%
\pgfsetfillopacity{0.800000}%
\pgfsetlinewidth{0.000000pt}%
\definecolor{currentstroke}{rgb}{0.000000,0.000000,0.000000}%
\pgfsetstrokecolor{currentstroke}%
\pgfsetdash{}{0pt}%
\pgfpathmoveto{\pgfqpoint{4.771815in}{2.839831in}}%
\pgfpathlineto{\pgfqpoint{4.785564in}{2.842199in}}%
\pgfpathlineto{\pgfqpoint{4.799325in}{2.844749in}}%
\pgfpathlineto{\pgfqpoint{4.813098in}{2.847481in}}%
\pgfpathlineto{\pgfqpoint{4.826883in}{2.850394in}}%
\pgfpathlineto{\pgfqpoint{4.834441in}{2.859778in}}%
\pgfpathlineto{\pgfqpoint{4.841996in}{2.869260in}}%
\pgfpathlineto{\pgfqpoint{4.849548in}{2.878845in}}%
\pgfpathlineto{\pgfqpoint{4.857097in}{2.888539in}}%
\pgfpathlineto{\pgfqpoint{4.843326in}{2.886107in}}%
\pgfpathlineto{\pgfqpoint{4.829568in}{2.883855in}}%
\pgfpathlineto{\pgfqpoint{4.815822in}{2.881784in}}%
\pgfpathlineto{\pgfqpoint{4.802087in}{2.879895in}}%
\pgfpathlineto{\pgfqpoint{4.794524in}{2.869710in}}%
\pgfpathlineto{\pgfqpoint{4.786958in}{2.859641in}}%
\pgfpathlineto{\pgfqpoint{4.779388in}{2.849683in}}%
\pgfpathlineto{\pgfqpoint{4.771815in}{2.839831in}}%
\pgfpathclose%
\pgfusepath{fill}%
\end{pgfscope}%
\begin{pgfscope}%
\pgfpathrectangle{\pgfqpoint{1.150000in}{0.150000in}}{\pgfqpoint{5.700000in}{5.700000in}}%
\pgfusepath{clip}%
\pgfsetbuttcap%
\pgfsetroundjoin%
\definecolor{currentfill}{rgb}{0.271828,0.209303,0.504434}%
\pgfsetfillcolor{currentfill}%
\pgfsetfillopacity{0.800000}%
\pgfsetlinewidth{0.000000pt}%
\definecolor{currentstroke}{rgb}{0.000000,0.000000,0.000000}%
\pgfsetstrokecolor{currentstroke}%
\pgfsetdash{}{0pt}%
\pgfpathmoveto{\pgfqpoint{3.029066in}{2.580921in}}%
\pgfpathlineto{\pgfqpoint{3.042559in}{2.566567in}}%
\pgfpathlineto{\pgfqpoint{3.056048in}{2.552494in}}%
\pgfpathlineto{\pgfqpoint{3.069533in}{2.538699in}}%
\pgfpathlineto{\pgfqpoint{3.083013in}{2.525181in}}%
\pgfpathlineto{\pgfqpoint{3.091140in}{2.535042in}}%
\pgfpathlineto{\pgfqpoint{3.099259in}{2.545004in}}%
\pgfpathlineto{\pgfqpoint{3.107371in}{2.555065in}}%
\pgfpathlineto{\pgfqpoint{3.115476in}{2.565227in}}%
\pgfpathlineto{\pgfqpoint{3.102012in}{2.578715in}}%
\pgfpathlineto{\pgfqpoint{3.088544in}{2.592479in}}%
\pgfpathlineto{\pgfqpoint{3.075071in}{2.606521in}}%
\pgfpathlineto{\pgfqpoint{3.061594in}{2.620843in}}%
\pgfpathlineto{\pgfqpoint{3.053473in}{2.610701in}}%
\pgfpathlineto{\pgfqpoint{3.045345in}{2.600666in}}%
\pgfpathlineto{\pgfqpoint{3.037209in}{2.590739in}}%
\pgfpathlineto{\pgfqpoint{3.029066in}{2.580921in}}%
\pgfpathclose%
\pgfusepath{fill}%
\end{pgfscope}%
\begin{pgfscope}%
\pgfpathrectangle{\pgfqpoint{1.150000in}{0.150000in}}{\pgfqpoint{5.700000in}{5.700000in}}%
\pgfusepath{clip}%
\pgfsetbuttcap%
\pgfsetroundjoin%
\definecolor{currentfill}{rgb}{0.244972,0.287675,0.537260}%
\pgfsetfillcolor{currentfill}%
\pgfsetfillopacity{0.800000}%
\pgfsetlinewidth{0.000000pt}%
\definecolor{currentstroke}{rgb}{0.000000,0.000000,0.000000}%
\pgfsetstrokecolor{currentstroke}%
\pgfsetdash{}{0pt}%
\pgfpathmoveto{\pgfqpoint{2.866669in}{2.775902in}}%
\pgfpathlineto{\pgfqpoint{2.880241in}{2.757994in}}%
\pgfpathlineto{\pgfqpoint{2.893805in}{2.740397in}}%
\pgfpathlineto{\pgfqpoint{2.907362in}{2.723107in}}%
\pgfpathlineto{\pgfqpoint{2.920911in}{2.706123in}}%
\pgfpathlineto{\pgfqpoint{2.929090in}{2.715854in}}%
\pgfpathlineto{\pgfqpoint{2.937262in}{2.725709in}}%
\pgfpathlineto{\pgfqpoint{2.945426in}{2.735688in}}%
\pgfpathlineto{\pgfqpoint{2.953581in}{2.745791in}}%
\pgfpathlineto{\pgfqpoint{2.940051in}{2.762743in}}%
\pgfpathlineto{\pgfqpoint{2.926513in}{2.779999in}}%
\pgfpathlineto{\pgfqpoint{2.912968in}{2.797564in}}%
\pgfpathlineto{\pgfqpoint{2.899415in}{2.815438in}}%
\pgfpathlineto{\pgfqpoint{2.891241in}{2.805356in}}%
\pgfpathlineto{\pgfqpoint{2.883059in}{2.795406in}}%
\pgfpathlineto{\pgfqpoint{2.874868in}{2.785588in}}%
\pgfpathlineto{\pgfqpoint{2.866669in}{2.775902in}}%
\pgfpathclose%
\pgfusepath{fill}%
\end{pgfscope}%
\begin{pgfscope}%
\pgfpathrectangle{\pgfqpoint{1.150000in}{0.150000in}}{\pgfqpoint{5.700000in}{5.700000in}}%
\pgfusepath{clip}%
\pgfsetbuttcap%
\pgfsetroundjoin%
\definecolor{currentfill}{rgb}{0.274128,0.199721,0.498911}%
\pgfsetfillcolor{currentfill}%
\pgfsetfillopacity{0.800000}%
\pgfsetlinewidth{0.000000pt}%
\definecolor{currentstroke}{rgb}{0.000000,0.000000,0.000000}%
\pgfsetstrokecolor{currentstroke}%
\pgfsetdash{}{0pt}%
\pgfpathmoveto{\pgfqpoint{4.175172in}{2.527436in}}%
\pgfpathlineto{\pgfqpoint{4.188715in}{2.526976in}}%
\pgfpathlineto{\pgfqpoint{4.202267in}{2.526713in}}%
\pgfpathlineto{\pgfqpoint{4.215827in}{2.526647in}}%
\pgfpathlineto{\pgfqpoint{4.229396in}{2.526776in}}%
\pgfpathlineto{\pgfqpoint{4.237155in}{2.537045in}}%
\pgfpathlineto{\pgfqpoint{4.244909in}{2.547347in}}%
\pgfpathlineto{\pgfqpoint{4.252658in}{2.557685in}}%
\pgfpathlineto{\pgfqpoint{4.260403in}{2.568062in}}%
\pgfpathlineto{\pgfqpoint{4.246843in}{2.568190in}}%
\pgfpathlineto{\pgfqpoint{4.233292in}{2.568513in}}%
\pgfpathlineto{\pgfqpoint{4.219749in}{2.569032in}}%
\pgfpathlineto{\pgfqpoint{4.206214in}{2.569747in}}%
\pgfpathlineto{\pgfqpoint{4.198461in}{2.559103in}}%
\pgfpathlineto{\pgfqpoint{4.190703in}{2.548505in}}%
\pgfpathlineto{\pgfqpoint{4.182940in}{2.537950in}}%
\pgfpathlineto{\pgfqpoint{4.175172in}{2.527436in}}%
\pgfpathclose%
\pgfusepath{fill}%
\end{pgfscope}%
\begin{pgfscope}%
\pgfpathrectangle{\pgfqpoint{1.150000in}{0.150000in}}{\pgfqpoint{5.700000in}{5.700000in}}%
\pgfusepath{clip}%
\pgfsetbuttcap%
\pgfsetroundjoin%
\definecolor{currentfill}{rgb}{0.218130,0.347432,0.550038}%
\pgfsetfillcolor{currentfill}%
\pgfsetfillopacity{0.800000}%
\pgfsetlinewidth{0.000000pt}%
\definecolor{currentstroke}{rgb}{0.000000,0.000000,0.000000}%
\pgfsetstrokecolor{currentstroke}%
\pgfsetdash{}{0pt}%
\pgfpathmoveto{\pgfqpoint{4.857097in}{2.888539in}}%
\pgfpathlineto{\pgfqpoint{4.870879in}{2.891152in}}%
\pgfpathlineto{\pgfqpoint{4.884673in}{2.893946in}}%
\pgfpathlineto{\pgfqpoint{4.898480in}{2.896920in}}%
\pgfpathlineto{\pgfqpoint{4.912299in}{2.900073in}}%
\pgfpathlineto{\pgfqpoint{4.919828in}{2.909381in}}%
\pgfpathlineto{\pgfqpoint{4.927355in}{2.918802in}}%
\pgfpathlineto{\pgfqpoint{4.934879in}{2.928342in}}%
\pgfpathlineto{\pgfqpoint{4.942400in}{2.938006in}}%
\pgfpathlineto{\pgfqpoint{4.928597in}{2.935365in}}%
\pgfpathlineto{\pgfqpoint{4.914806in}{2.932903in}}%
\pgfpathlineto{\pgfqpoint{4.901027in}{2.930621in}}%
\pgfpathlineto{\pgfqpoint{4.887260in}{2.928519in}}%
\pgfpathlineto{\pgfqpoint{4.879724in}{2.918332in}}%
\pgfpathlineto{\pgfqpoint{4.872184in}{2.908277in}}%
\pgfpathlineto{\pgfqpoint{4.864642in}{2.898348in}}%
\pgfpathlineto{\pgfqpoint{4.857097in}{2.888539in}}%
\pgfpathclose%
\pgfusepath{fill}%
\end{pgfscope}%
\begin{pgfscope}%
\pgfpathrectangle{\pgfqpoint{1.150000in}{0.150000in}}{\pgfqpoint{5.700000in}{5.700000in}}%
\pgfusepath{clip}%
\pgfsetbuttcap%
\pgfsetroundjoin%
\definecolor{currentfill}{rgb}{0.208623,0.367752,0.552675}%
\pgfsetfillcolor{currentfill}%
\pgfsetfillopacity{0.800000}%
\pgfsetlinewidth{0.000000pt}%
\definecolor{currentstroke}{rgb}{0.000000,0.000000,0.000000}%
\pgfsetstrokecolor{currentstroke}%
\pgfsetdash{}{0pt}%
\pgfpathmoveto{\pgfqpoint{4.942400in}{2.938006in}}%
\pgfpathlineto{\pgfqpoint{4.956215in}{2.940825in}}%
\pgfpathlineto{\pgfqpoint{4.970043in}{2.943824in}}%
\pgfpathlineto{\pgfqpoint{4.983884in}{2.947001in}}%
\pgfpathlineto{\pgfqpoint{4.997737in}{2.950356in}}%
\pgfpathlineto{\pgfqpoint{5.005239in}{2.959618in}}%
\pgfpathlineto{\pgfqpoint{5.012738in}{2.969009in}}%
\pgfpathlineto{\pgfqpoint{5.020234in}{2.978535in}}%
\pgfpathlineto{\pgfqpoint{5.027728in}{2.988202in}}%
\pgfpathlineto{\pgfqpoint{5.013892in}{2.985391in}}%
\pgfpathlineto{\pgfqpoint{5.000068in}{2.982758in}}%
\pgfpathlineto{\pgfqpoint{4.986257in}{2.980303in}}%
\pgfpathlineto{\pgfqpoint{4.972458in}{2.978026in}}%
\pgfpathlineto{\pgfqpoint{4.964947in}{2.967804in}}%
\pgfpathlineto{\pgfqpoint{4.957434in}{2.957731in}}%
\pgfpathlineto{\pgfqpoint{4.949918in}{2.947800in}}%
\pgfpathlineto{\pgfqpoint{4.942400in}{2.938006in}}%
\pgfpathclose%
\pgfusepath{fill}%
\end{pgfscope}%
\begin{pgfscope}%
\pgfpathrectangle{\pgfqpoint{1.150000in}{0.150000in}}{\pgfqpoint{5.700000in}{5.700000in}}%
\pgfusepath{clip}%
\pgfsetbuttcap%
\pgfsetroundjoin%
\definecolor{currentfill}{rgb}{0.283187,0.125848,0.444960}%
\pgfsetfillcolor{currentfill}%
\pgfsetfillopacity{0.800000}%
\pgfsetlinewidth{0.000000pt}%
\definecolor{currentstroke}{rgb}{0.000000,0.000000,0.000000}%
\pgfsetstrokecolor{currentstroke}%
\pgfsetdash{}{0pt}%
\pgfpathmoveto{\pgfqpoint{3.694791in}{2.374143in}}%
\pgfpathlineto{\pgfqpoint{3.708229in}{2.369535in}}%
\pgfpathlineto{\pgfqpoint{3.721671in}{2.365143in}}%
\pgfpathlineto{\pgfqpoint{3.735117in}{2.360968in}}%
\pgfpathlineto{\pgfqpoint{3.748567in}{2.357007in}}%
\pgfpathlineto{\pgfqpoint{3.756476in}{2.367680in}}%
\pgfpathlineto{\pgfqpoint{3.764380in}{2.378388in}}%
\pgfpathlineto{\pgfqpoint{3.772279in}{2.389131in}}%
\pgfpathlineto{\pgfqpoint{3.780173in}{2.399912in}}%
\pgfpathlineto{\pgfqpoint{3.766731in}{2.403971in}}%
\pgfpathlineto{\pgfqpoint{3.753294in}{2.408245in}}%
\pgfpathlineto{\pgfqpoint{3.739862in}{2.412734in}}%
\pgfpathlineto{\pgfqpoint{3.726433in}{2.417441in}}%
\pgfpathlineto{\pgfqpoint{3.718530in}{2.406551in}}%
\pgfpathlineto{\pgfqpoint{3.710623in}{2.395705in}}%
\pgfpathlineto{\pgfqpoint{3.702710in}{2.384903in}}%
\pgfpathlineto{\pgfqpoint{3.694791in}{2.374143in}}%
\pgfpathclose%
\pgfusepath{fill}%
\end{pgfscope}%
\begin{pgfscope}%
\pgfpathrectangle{\pgfqpoint{1.150000in}{0.150000in}}{\pgfqpoint{5.700000in}{5.700000in}}%
\pgfusepath{clip}%
\pgfsetbuttcap%
\pgfsetroundjoin%
\definecolor{currentfill}{rgb}{0.277134,0.185228,0.489898}%
\pgfsetfillcolor{currentfill}%
\pgfsetfillopacity{0.800000}%
\pgfsetlinewidth{0.000000pt}%
\definecolor{currentstroke}{rgb}{0.000000,0.000000,0.000000}%
\pgfsetstrokecolor{currentstroke}%
\pgfsetdash{}{0pt}%
\pgfpathmoveto{\pgfqpoint{4.089925in}{2.488643in}}%
\pgfpathlineto{\pgfqpoint{4.103445in}{2.487615in}}%
\pgfpathlineto{\pgfqpoint{4.116974in}{2.486785in}}%
\pgfpathlineto{\pgfqpoint{4.130509in}{2.486155in}}%
\pgfpathlineto{\pgfqpoint{4.144053in}{2.485723in}}%
\pgfpathlineto{\pgfqpoint{4.151840in}{2.496105in}}%
\pgfpathlineto{\pgfqpoint{4.159622in}{2.506516in}}%
\pgfpathlineto{\pgfqpoint{4.167400in}{2.516959in}}%
\pgfpathlineto{\pgfqpoint{4.175172in}{2.527436in}}%
\pgfpathlineto{\pgfqpoint{4.161637in}{2.528093in}}%
\pgfpathlineto{\pgfqpoint{4.148109in}{2.528948in}}%
\pgfpathlineto{\pgfqpoint{4.134590in}{2.530001in}}%
\pgfpathlineto{\pgfqpoint{4.121078in}{2.531254in}}%
\pgfpathlineto{\pgfqpoint{4.113297in}{2.520541in}}%
\pgfpathlineto{\pgfqpoint{4.105511in}{2.509870in}}%
\pgfpathlineto{\pgfqpoint{4.097720in}{2.499238in}}%
\pgfpathlineto{\pgfqpoint{4.089925in}{2.488643in}}%
\pgfpathclose%
\pgfusepath{fill}%
\end{pgfscope}%
\begin{pgfscope}%
\pgfpathrectangle{\pgfqpoint{1.150000in}{0.150000in}}{\pgfqpoint{5.700000in}{5.700000in}}%
\pgfusepath{clip}%
\pgfsetbuttcap%
\pgfsetroundjoin%
\definecolor{currentfill}{rgb}{0.276194,0.190074,0.493001}%
\pgfsetfillcolor{currentfill}%
\pgfsetfillopacity{0.800000}%
\pgfsetlinewidth{0.000000pt}%
\definecolor{currentstroke}{rgb}{0.000000,0.000000,0.000000}%
\pgfsetstrokecolor{currentstroke}%
\pgfsetdash{}{0pt}%
\pgfpathmoveto{\pgfqpoint{3.083013in}{2.525181in}}%
\pgfpathlineto{\pgfqpoint{3.096490in}{2.511936in}}%
\pgfpathlineto{\pgfqpoint{3.109962in}{2.498962in}}%
\pgfpathlineto{\pgfqpoint{3.123431in}{2.486259in}}%
\pgfpathlineto{\pgfqpoint{3.136896in}{2.473823in}}%
\pgfpathlineto{\pgfqpoint{3.145007in}{2.483726in}}%
\pgfpathlineto{\pgfqpoint{3.153110in}{2.493722in}}%
\pgfpathlineto{\pgfqpoint{3.161206in}{2.503810in}}%
\pgfpathlineto{\pgfqpoint{3.169296in}{2.513991in}}%
\pgfpathlineto{\pgfqpoint{3.155846in}{2.526397in}}%
\pgfpathlineto{\pgfqpoint{3.142393in}{2.539070in}}%
\pgfpathlineto{\pgfqpoint{3.128936in}{2.552013in}}%
\pgfpathlineto{\pgfqpoint{3.115476in}{2.565227in}}%
\pgfpathlineto{\pgfqpoint{3.107371in}{2.555065in}}%
\pgfpathlineto{\pgfqpoint{3.099259in}{2.545004in}}%
\pgfpathlineto{\pgfqpoint{3.091140in}{2.535042in}}%
\pgfpathlineto{\pgfqpoint{3.083013in}{2.525181in}}%
\pgfpathclose%
\pgfusepath{fill}%
\end{pgfscope}%
\begin{pgfscope}%
\pgfpathrectangle{\pgfqpoint{1.150000in}{0.150000in}}{\pgfqpoint{5.700000in}{5.700000in}}%
\pgfusepath{clip}%
\pgfsetbuttcap%
\pgfsetroundjoin%
\definecolor{currentfill}{rgb}{0.199430,0.387607,0.554642}%
\pgfsetfillcolor{currentfill}%
\pgfsetfillopacity{0.800000}%
\pgfsetlinewidth{0.000000pt}%
\definecolor{currentstroke}{rgb}{0.000000,0.000000,0.000000}%
\pgfsetstrokecolor{currentstroke}%
\pgfsetdash{}{0pt}%
\pgfpathmoveto{\pgfqpoint{5.027728in}{2.988202in}}%
\pgfpathlineto{\pgfqpoint{5.041577in}{2.991190in}}%
\pgfpathlineto{\pgfqpoint{5.055439in}{2.994355in}}%
\pgfpathlineto{\pgfqpoint{5.069313in}{2.997698in}}%
\pgfpathlineto{\pgfqpoint{5.083201in}{3.001218in}}%
\pgfpathlineto{\pgfqpoint{5.090675in}{3.010468in}}%
\pgfpathlineto{\pgfqpoint{5.098147in}{3.019864in}}%
\pgfpathlineto{\pgfqpoint{5.105616in}{3.029413in}}%
\pgfpathlineto{\pgfqpoint{5.113084in}{3.039122in}}%
\pgfpathlineto{\pgfqpoint{5.099215in}{3.036180in}}%
\pgfpathlineto{\pgfqpoint{5.085359in}{3.033413in}}%
\pgfpathlineto{\pgfqpoint{5.071515in}{3.030823in}}%
\pgfpathlineto{\pgfqpoint{5.057685in}{3.028410in}}%
\pgfpathlineto{\pgfqpoint{5.050198in}{3.018114in}}%
\pgfpathlineto{\pgfqpoint{5.042710in}{3.007985in}}%
\pgfpathlineto{\pgfqpoint{5.035220in}{2.998016in}}%
\pgfpathlineto{\pgfqpoint{5.027728in}{2.988202in}}%
\pgfpathclose%
\pgfusepath{fill}%
\end{pgfscope}%
\begin{pgfscope}%
\pgfpathrectangle{\pgfqpoint{1.150000in}{0.150000in}}{\pgfqpoint{5.700000in}{5.700000in}}%
\pgfusepath{clip}%
\pgfsetbuttcap%
\pgfsetroundjoin%
\definecolor{currentfill}{rgb}{0.231674,0.318106,0.544834}%
\pgfsetfillcolor{currentfill}%
\pgfsetfillopacity{0.800000}%
\pgfsetlinewidth{0.000000pt}%
\definecolor{currentstroke}{rgb}{0.000000,0.000000,0.000000}%
\pgfsetstrokecolor{currentstroke}%
\pgfsetdash{}{0pt}%
\pgfpathmoveto{\pgfqpoint{2.812294in}{2.850697in}}%
\pgfpathlineto{\pgfqpoint{2.825901in}{2.831518in}}%
\pgfpathlineto{\pgfqpoint{2.839499in}{2.812661in}}%
\pgfpathlineto{\pgfqpoint{2.853089in}{2.794123in}}%
\pgfpathlineto{\pgfqpoint{2.866669in}{2.775902in}}%
\pgfpathlineto{\pgfqpoint{2.874868in}{2.785588in}}%
\pgfpathlineto{\pgfqpoint{2.883059in}{2.795406in}}%
\pgfpathlineto{\pgfqpoint{2.891241in}{2.805356in}}%
\pgfpathlineto{\pgfqpoint{2.899415in}{2.815438in}}%
\pgfpathlineto{\pgfqpoint{2.885854in}{2.833626in}}%
\pgfpathlineto{\pgfqpoint{2.872284in}{2.852131in}}%
\pgfpathlineto{\pgfqpoint{2.858706in}{2.870954in}}%
\pgfpathlineto{\pgfqpoint{2.845118in}{2.890099in}}%
\pgfpathlineto{\pgfqpoint{2.836925in}{2.880038in}}%
\pgfpathlineto{\pgfqpoint{2.828723in}{2.870118in}}%
\pgfpathlineto{\pgfqpoint{2.820513in}{2.860337in}}%
\pgfpathlineto{\pgfqpoint{2.812294in}{2.850697in}}%
\pgfpathclose%
\pgfusepath{fill}%
\end{pgfscope}%
\begin{pgfscope}%
\pgfpathrectangle{\pgfqpoint{1.150000in}{0.150000in}}{\pgfqpoint{5.700000in}{5.700000in}}%
\pgfusepath{clip}%
\pgfsetbuttcap%
\pgfsetroundjoin%
\definecolor{currentfill}{rgb}{0.280255,0.165693,0.476498}%
\pgfsetfillcolor{currentfill}%
\pgfsetfillopacity{0.800000}%
\pgfsetlinewidth{0.000000pt}%
\definecolor{currentstroke}{rgb}{0.000000,0.000000,0.000000}%
\pgfsetstrokecolor{currentstroke}%
\pgfsetdash{}{0pt}%
\pgfpathmoveto{\pgfqpoint{4.004653in}{2.451926in}}%
\pgfpathlineto{\pgfqpoint{4.018153in}{2.450285in}}%
\pgfpathlineto{\pgfqpoint{4.031660in}{2.448847in}}%
\pgfpathlineto{\pgfqpoint{4.045174in}{2.447610in}}%
\pgfpathlineto{\pgfqpoint{4.058695in}{2.446574in}}%
\pgfpathlineto{\pgfqpoint{4.066510in}{2.457050in}}%
\pgfpathlineto{\pgfqpoint{4.074320in}{2.467551in}}%
\pgfpathlineto{\pgfqpoint{4.082125in}{2.478082in}}%
\pgfpathlineto{\pgfqpoint{4.089925in}{2.488643in}}%
\pgfpathlineto{\pgfqpoint{4.076412in}{2.489872in}}%
\pgfpathlineto{\pgfqpoint{4.062906in}{2.491302in}}%
\pgfpathlineto{\pgfqpoint{4.049408in}{2.492933in}}%
\pgfpathlineto{\pgfqpoint{4.035916in}{2.494767in}}%
\pgfpathlineto{\pgfqpoint{4.028108in}{2.484001in}}%
\pgfpathlineto{\pgfqpoint{4.020294in}{2.473274in}}%
\pgfpathlineto{\pgfqpoint{4.012476in}{2.462583in}}%
\pgfpathlineto{\pgfqpoint{4.004653in}{2.451926in}}%
\pgfpathclose%
\pgfusepath{fill}%
\end{pgfscope}%
\begin{pgfscope}%
\pgfpathrectangle{\pgfqpoint{1.150000in}{0.150000in}}{\pgfqpoint{5.700000in}{5.700000in}}%
\pgfusepath{clip}%
\pgfsetbuttcap%
\pgfsetroundjoin%
\definecolor{currentfill}{rgb}{0.192357,0.403199,0.555836}%
\pgfsetfillcolor{currentfill}%
\pgfsetfillopacity{0.800000}%
\pgfsetlinewidth{0.000000pt}%
\definecolor{currentstroke}{rgb}{0.000000,0.000000,0.000000}%
\pgfsetstrokecolor{currentstroke}%
\pgfsetdash{}{0pt}%
\pgfpathmoveto{\pgfqpoint{5.113084in}{3.039122in}}%
\pgfpathlineto{\pgfqpoint{5.126967in}{3.042241in}}%
\pgfpathlineto{\pgfqpoint{5.140862in}{3.045536in}}%
\pgfpathlineto{\pgfqpoint{5.154771in}{3.049007in}}%
\pgfpathlineto{\pgfqpoint{5.168693in}{3.052653in}}%
\pgfpathlineto{\pgfqpoint{5.176140in}{3.061931in}}%
\pgfpathlineto{\pgfqpoint{5.183586in}{3.071375in}}%
\pgfpathlineto{\pgfqpoint{5.191030in}{3.080991in}}%
\pgfpathlineto{\pgfqpoint{5.198473in}{3.090787in}}%
\pgfpathlineto{\pgfqpoint{5.184571in}{3.087749in}}%
\pgfpathlineto{\pgfqpoint{5.170682in}{3.084887in}}%
\pgfpathlineto{\pgfqpoint{5.156807in}{3.082200in}}%
\pgfpathlineto{\pgfqpoint{5.142944in}{3.079688in}}%
\pgfpathlineto{\pgfqpoint{5.135480in}{3.069274in}}%
\pgfpathlineto{\pgfqpoint{5.128016in}{3.059046in}}%
\pgfpathlineto{\pgfqpoint{5.120551in}{3.048998in}}%
\pgfpathlineto{\pgfqpoint{5.113084in}{3.039122in}}%
\pgfpathclose%
\pgfusepath{fill}%
\end{pgfscope}%
\begin{pgfscope}%
\pgfpathrectangle{\pgfqpoint{1.150000in}{0.150000in}}{\pgfqpoint{5.700000in}{5.700000in}}%
\pgfusepath{clip}%
\pgfsetbuttcap%
\pgfsetroundjoin%
\definecolor{currentfill}{rgb}{0.283229,0.120777,0.440584}%
\pgfsetfillcolor{currentfill}%
\pgfsetfillopacity{0.800000}%
\pgfsetlinewidth{0.000000pt}%
\definecolor{currentstroke}{rgb}{0.000000,0.000000,0.000000}%
\pgfsetstrokecolor{currentstroke}%
\pgfsetdash{}{0pt}%
\pgfpathmoveto{\pgfqpoint{3.470014in}{2.360834in}}%
\pgfpathlineto{\pgfqpoint{3.483438in}{2.353635in}}%
\pgfpathlineto{\pgfqpoint{3.496863in}{2.346666in}}%
\pgfpathlineto{\pgfqpoint{3.510291in}{2.339927in}}%
\pgfpathlineto{\pgfqpoint{3.523720in}{2.333417in}}%
\pgfpathlineto{\pgfqpoint{3.531702in}{2.343933in}}%
\pgfpathlineto{\pgfqpoint{3.539678in}{2.354497in}}%
\pgfpathlineto{\pgfqpoint{3.547648in}{2.365110in}}%
\pgfpathlineto{\pgfqpoint{3.555613in}{2.375774in}}%
\pgfpathlineto{\pgfqpoint{3.542194in}{2.382319in}}%
\pgfpathlineto{\pgfqpoint{3.528778in}{2.389092in}}%
\pgfpathlineto{\pgfqpoint{3.515363in}{2.396096in}}%
\pgfpathlineto{\pgfqpoint{3.501950in}{2.403330in}}%
\pgfpathlineto{\pgfqpoint{3.493975in}{2.392620in}}%
\pgfpathlineto{\pgfqpoint{3.485993in}{2.381968in}}%
\pgfpathlineto{\pgfqpoint{3.478006in}{2.371373in}}%
\pgfpathlineto{\pgfqpoint{3.470014in}{2.360834in}}%
\pgfpathclose%
\pgfusepath{fill}%
\end{pgfscope}%
\begin{pgfscope}%
\pgfpathrectangle{\pgfqpoint{1.150000in}{0.150000in}}{\pgfqpoint{5.700000in}{5.700000in}}%
\pgfusepath{clip}%
\pgfsetbuttcap%
\pgfsetroundjoin%
\definecolor{currentfill}{rgb}{0.283072,0.130895,0.449241}%
\pgfsetfillcolor{currentfill}%
\pgfsetfillopacity{0.800000}%
\pgfsetlinewidth{0.000000pt}%
\definecolor{currentstroke}{rgb}{0.000000,0.000000,0.000000}%
\pgfsetstrokecolor{currentstroke}%
\pgfsetdash{}{0pt}%
\pgfpathmoveto{\pgfqpoint{3.330535in}{2.385294in}}%
\pgfpathlineto{\pgfqpoint{3.343965in}{2.376197in}}%
\pgfpathlineto{\pgfqpoint{3.357395in}{2.367343in}}%
\pgfpathlineto{\pgfqpoint{3.370825in}{2.358730in}}%
\pgfpathlineto{\pgfqpoint{3.384256in}{2.350357in}}%
\pgfpathlineto{\pgfqpoint{3.392284in}{2.360667in}}%
\pgfpathlineto{\pgfqpoint{3.400306in}{2.371040in}}%
\pgfpathlineto{\pgfqpoint{3.408322in}{2.381475in}}%
\pgfpathlineto{\pgfqpoint{3.416332in}{2.391973in}}%
\pgfpathlineto{\pgfqpoint{3.402914in}{2.400349in}}%
\pgfpathlineto{\pgfqpoint{3.389496in}{2.408965in}}%
\pgfpathlineto{\pgfqpoint{3.376079in}{2.417823in}}%
\pgfpathlineto{\pgfqpoint{3.362661in}{2.426922in}}%
\pgfpathlineto{\pgfqpoint{3.354639in}{2.416410in}}%
\pgfpathlineto{\pgfqpoint{3.346610in}{2.405968in}}%
\pgfpathlineto{\pgfqpoint{3.338576in}{2.395596in}}%
\pgfpathlineto{\pgfqpoint{3.330535in}{2.385294in}}%
\pgfpathclose%
\pgfusepath{fill}%
\end{pgfscope}%
\begin{pgfscope}%
\pgfpathrectangle{\pgfqpoint{1.150000in}{0.150000in}}{\pgfqpoint{5.700000in}{5.700000in}}%
\pgfusepath{clip}%
\pgfsetbuttcap%
\pgfsetroundjoin%
\definecolor{currentfill}{rgb}{0.183898,0.422383,0.556944}%
\pgfsetfillcolor{currentfill}%
\pgfsetfillopacity{0.800000}%
\pgfsetlinewidth{0.000000pt}%
\definecolor{currentstroke}{rgb}{0.000000,0.000000,0.000000}%
\pgfsetstrokecolor{currentstroke}%
\pgfsetdash{}{0pt}%
\pgfpathmoveto{\pgfqpoint{5.198473in}{3.090787in}}%
\pgfpathlineto{\pgfqpoint{5.212388in}{3.093999in}}%
\pgfpathlineto{\pgfqpoint{5.226317in}{3.097386in}}%
\pgfpathlineto{\pgfqpoint{5.240260in}{3.100947in}}%
\pgfpathlineto{\pgfqpoint{5.254216in}{3.104682in}}%
\pgfpathlineto{\pgfqpoint{5.261638in}{3.114035in}}%
\pgfpathlineto{\pgfqpoint{5.269058in}{3.123574in}}%
\pgfpathlineto{\pgfqpoint{5.276478in}{3.133305in}}%
\pgfpathlineto{\pgfqpoint{5.283898in}{3.143237in}}%
\pgfpathlineto{\pgfqpoint{5.269964in}{3.140143in}}%
\pgfpathlineto{\pgfqpoint{5.256043in}{3.137222in}}%
\pgfpathlineto{\pgfqpoint{5.242135in}{3.134475in}}%
\pgfpathlineto{\pgfqpoint{5.228241in}{3.131902in}}%
\pgfpathlineto{\pgfqpoint{5.220799in}{3.121319in}}%
\pgfpathlineto{\pgfqpoint{5.213358in}{3.110943in}}%
\pgfpathlineto{\pgfqpoint{5.205916in}{3.100768in}}%
\pgfpathlineto{\pgfqpoint{5.198473in}{3.090787in}}%
\pgfpathclose%
\pgfusepath{fill}%
\end{pgfscope}%
\begin{pgfscope}%
\pgfpathrectangle{\pgfqpoint{1.150000in}{0.150000in}}{\pgfqpoint{5.700000in}{5.700000in}}%
\pgfusepath{clip}%
\pgfsetbuttcap%
\pgfsetroundjoin%
\definecolor{currentfill}{rgb}{0.279574,0.170599,0.479997}%
\pgfsetfillcolor{currentfill}%
\pgfsetfillopacity{0.800000}%
\pgfsetlinewidth{0.000000pt}%
\definecolor{currentstroke}{rgb}{0.000000,0.000000,0.000000}%
\pgfsetstrokecolor{currentstroke}%
\pgfsetdash{}{0pt}%
\pgfpathmoveto{\pgfqpoint{3.136896in}{2.473823in}}%
\pgfpathlineto{\pgfqpoint{3.150359in}{2.461652in}}%
\pgfpathlineto{\pgfqpoint{3.163819in}{2.449746in}}%
\pgfpathlineto{\pgfqpoint{3.177276in}{2.438101in}}%
\pgfpathlineto{\pgfqpoint{3.190730in}{2.426716in}}%
\pgfpathlineto{\pgfqpoint{3.198825in}{2.436661in}}%
\pgfpathlineto{\pgfqpoint{3.206913in}{2.446690in}}%
\pgfpathlineto{\pgfqpoint{3.214995in}{2.456804in}}%
\pgfpathlineto{\pgfqpoint{3.223069in}{2.467003in}}%
\pgfpathlineto{\pgfqpoint{3.209629in}{2.478358in}}%
\pgfpathlineto{\pgfqpoint{3.196187in}{2.489973in}}%
\pgfpathlineto{\pgfqpoint{3.182743in}{2.501850in}}%
\pgfpathlineto{\pgfqpoint{3.169296in}{2.513991in}}%
\pgfpathlineto{\pgfqpoint{3.161206in}{2.503810in}}%
\pgfpathlineto{\pgfqpoint{3.153110in}{2.493722in}}%
\pgfpathlineto{\pgfqpoint{3.145007in}{2.483726in}}%
\pgfpathlineto{\pgfqpoint{3.136896in}{2.473823in}}%
\pgfpathclose%
\pgfusepath{fill}%
\end{pgfscope}%
\begin{pgfscope}%
\pgfpathrectangle{\pgfqpoint{1.150000in}{0.150000in}}{\pgfqpoint{5.700000in}{5.700000in}}%
\pgfusepath{clip}%
\pgfsetbuttcap%
\pgfsetroundjoin%
\definecolor{currentfill}{rgb}{0.281887,0.150881,0.465405}%
\pgfsetfillcolor{currentfill}%
\pgfsetfillopacity{0.800000}%
\pgfsetlinewidth{0.000000pt}%
\definecolor{currentstroke}{rgb}{0.000000,0.000000,0.000000}%
\pgfsetstrokecolor{currentstroke}%
\pgfsetdash{}{0pt}%
\pgfpathmoveto{\pgfqpoint{3.919345in}{2.417550in}}%
\pgfpathlineto{\pgfqpoint{3.932827in}{2.415252in}}%
\pgfpathlineto{\pgfqpoint{3.946315in}{2.413161in}}%
\pgfpathlineto{\pgfqpoint{3.959810in}{2.411274in}}%
\pgfpathlineto{\pgfqpoint{3.973311in}{2.409591in}}%
\pgfpathlineto{\pgfqpoint{3.981154in}{2.420135in}}%
\pgfpathlineto{\pgfqpoint{3.988992in}{2.430705in}}%
\pgfpathlineto{\pgfqpoint{3.996825in}{2.441301in}}%
\pgfpathlineto{\pgfqpoint{4.004653in}{2.451926in}}%
\pgfpathlineto{\pgfqpoint{3.991160in}{2.453770in}}%
\pgfpathlineto{\pgfqpoint{3.977674in}{2.455819in}}%
\pgfpathlineto{\pgfqpoint{3.964194in}{2.458072in}}%
\pgfpathlineto{\pgfqpoint{3.950720in}{2.460531in}}%
\pgfpathlineto{\pgfqpoint{3.942884in}{2.449732in}}%
\pgfpathlineto{\pgfqpoint{3.935042in}{2.438971in}}%
\pgfpathlineto{\pgfqpoint{3.927196in}{2.428244in}}%
\pgfpathlineto{\pgfqpoint{3.919345in}{2.417550in}}%
\pgfpathclose%
\pgfusepath{fill}%
\end{pgfscope}%
\begin{pgfscope}%
\pgfpathrectangle{\pgfqpoint{1.150000in}{0.150000in}}{\pgfqpoint{5.700000in}{5.700000in}}%
\pgfusepath{clip}%
\pgfsetbuttcap%
\pgfsetroundjoin%
\definecolor{currentfill}{rgb}{0.216210,0.351535,0.550627}%
\pgfsetfillcolor{currentfill}%
\pgfsetfillopacity{0.800000}%
\pgfsetlinewidth{0.000000pt}%
\definecolor{currentstroke}{rgb}{0.000000,0.000000,0.000000}%
\pgfsetstrokecolor{currentstroke}%
\pgfsetdash{}{0pt}%
\pgfpathmoveto{\pgfqpoint{2.757765in}{2.930698in}}%
\pgfpathlineto{\pgfqpoint{2.771412in}{2.910199in}}%
\pgfpathlineto{\pgfqpoint{2.785049in}{2.890034in}}%
\pgfpathlineto{\pgfqpoint{2.798676in}{2.870201in}}%
\pgfpathlineto{\pgfqpoint{2.812294in}{2.850697in}}%
\pgfpathlineto{\pgfqpoint{2.820513in}{2.860337in}}%
\pgfpathlineto{\pgfqpoint{2.828723in}{2.870118in}}%
\pgfpathlineto{\pgfqpoint{2.836925in}{2.880038in}}%
\pgfpathlineto{\pgfqpoint{2.845118in}{2.890099in}}%
\pgfpathlineto{\pgfqpoint{2.831522in}{2.909570in}}%
\pgfpathlineto{\pgfqpoint{2.817915in}{2.929368in}}%
\pgfpathlineto{\pgfqpoint{2.804299in}{2.949498in}}%
\pgfpathlineto{\pgfqpoint{2.790672in}{2.969963in}}%
\pgfpathlineto{\pgfqpoint{2.782459in}{2.959924in}}%
\pgfpathlineto{\pgfqpoint{2.774236in}{2.950034in}}%
\pgfpathlineto{\pgfqpoint{2.766005in}{2.940292in}}%
\pgfpathlineto{\pgfqpoint{2.757765in}{2.930698in}}%
\pgfpathclose%
\pgfusepath{fill}%
\end{pgfscope}%
\begin{pgfscope}%
\pgfpathrectangle{\pgfqpoint{1.150000in}{0.150000in}}{\pgfqpoint{5.700000in}{5.700000in}}%
\pgfusepath{clip}%
\pgfsetbuttcap%
\pgfsetroundjoin%
\definecolor{currentfill}{rgb}{0.283229,0.120777,0.440584}%
\pgfsetfillcolor{currentfill}%
\pgfsetfillopacity{0.800000}%
\pgfsetlinewidth{0.000000pt}%
\definecolor{currentstroke}{rgb}{0.000000,0.000000,0.000000}%
\pgfsetstrokecolor{currentstroke}%
\pgfsetdash{}{0pt}%
\pgfpathmoveto{\pgfqpoint{3.609314in}{2.351854in}}%
\pgfpathlineto{\pgfqpoint{3.622747in}{2.346434in}}%
\pgfpathlineto{\pgfqpoint{3.636183in}{2.341235in}}%
\pgfpathlineto{\pgfqpoint{3.649622in}{2.336256in}}%
\pgfpathlineto{\pgfqpoint{3.663066in}{2.331496in}}%
\pgfpathlineto{\pgfqpoint{3.671005in}{2.342101in}}%
\pgfpathlineto{\pgfqpoint{3.678939in}{2.352744in}}%
\pgfpathlineto{\pgfqpoint{3.686868in}{2.363424in}}%
\pgfpathlineto{\pgfqpoint{3.694791in}{2.374143in}}%
\pgfpathlineto{\pgfqpoint{3.681358in}{2.378970in}}%
\pgfpathlineto{\pgfqpoint{3.667928in}{2.384016in}}%
\pgfpathlineto{\pgfqpoint{3.654501in}{2.389282in}}%
\pgfpathlineto{\pgfqpoint{3.641079in}{2.394769in}}%
\pgfpathlineto{\pgfqpoint{3.633145in}{2.383971in}}%
\pgfpathlineto{\pgfqpoint{3.625207in}{2.373220in}}%
\pgfpathlineto{\pgfqpoint{3.617263in}{2.362515in}}%
\pgfpathlineto{\pgfqpoint{3.609314in}{2.351854in}}%
\pgfpathclose%
\pgfusepath{fill}%
\end{pgfscope}%
\begin{pgfscope}%
\pgfpathrectangle{\pgfqpoint{1.150000in}{0.150000in}}{\pgfqpoint{5.700000in}{5.700000in}}%
\pgfusepath{clip}%
\pgfsetbuttcap%
\pgfsetroundjoin%
\definecolor{currentfill}{rgb}{0.175841,0.441290,0.557685}%
\pgfsetfillcolor{currentfill}%
\pgfsetfillopacity{0.800000}%
\pgfsetlinewidth{0.000000pt}%
\definecolor{currentstroke}{rgb}{0.000000,0.000000,0.000000}%
\pgfsetstrokecolor{currentstroke}%
\pgfsetdash{}{0pt}%
\pgfpathmoveto{\pgfqpoint{5.283898in}{3.143237in}}%
\pgfpathlineto{\pgfqpoint{5.297846in}{3.146505in}}%
\pgfpathlineto{\pgfqpoint{5.311808in}{3.149947in}}%
\pgfpathlineto{\pgfqpoint{5.325784in}{3.153561in}}%
\pgfpathlineto{\pgfqpoint{5.339774in}{3.157349in}}%
\pgfpathlineto{\pgfqpoint{5.347172in}{3.166828in}}%
\pgfpathlineto{\pgfqpoint{5.354569in}{3.176515in}}%
\pgfpathlineto{\pgfqpoint{5.361967in}{3.186416in}}%
\pgfpathlineto{\pgfqpoint{5.369366in}{3.196541in}}%
\pgfpathlineto{\pgfqpoint{5.355399in}{3.193426in}}%
\pgfpathlineto{\pgfqpoint{5.341447in}{3.190484in}}%
\pgfpathlineto{\pgfqpoint{5.327508in}{3.187714in}}%
\pgfpathlineto{\pgfqpoint{5.313582in}{3.185117in}}%
\pgfpathlineto{\pgfqpoint{5.306160in}{3.174310in}}%
\pgfpathlineto{\pgfqpoint{5.298739in}{3.163732in}}%
\pgfpathlineto{\pgfqpoint{5.291318in}{3.153377in}}%
\pgfpathlineto{\pgfqpoint{5.283898in}{3.143237in}}%
\pgfpathclose%
\pgfusepath{fill}%
\end{pgfscope}%
\begin{pgfscope}%
\pgfpathrectangle{\pgfqpoint{1.150000in}{0.150000in}}{\pgfqpoint{5.700000in}{5.700000in}}%
\pgfusepath{clip}%
\pgfsetbuttcap%
\pgfsetroundjoin%
\definecolor{currentfill}{rgb}{0.166617,0.463708,0.558119}%
\pgfsetfillcolor{currentfill}%
\pgfsetfillopacity{0.800000}%
\pgfsetlinewidth{0.000000pt}%
\definecolor{currentstroke}{rgb}{0.000000,0.000000,0.000000}%
\pgfsetstrokecolor{currentstroke}%
\pgfsetdash{}{0pt}%
\pgfpathmoveto{\pgfqpoint{5.369366in}{3.196541in}}%
\pgfpathlineto{\pgfqpoint{5.383346in}{3.199828in}}%
\pgfpathlineto{\pgfqpoint{5.397340in}{3.203287in}}%
\pgfpathlineto{\pgfqpoint{5.411349in}{3.206919in}}%
\pgfpathlineto{\pgfqpoint{5.425372in}{3.210722in}}%
\pgfpathlineto{\pgfqpoint{5.432747in}{3.220384in}}%
\pgfpathlineto{\pgfqpoint{5.440123in}{3.230277in}}%
\pgfpathlineto{\pgfqpoint{5.447501in}{3.240410in}}%
\pgfpathlineto{\pgfqpoint{5.454881in}{3.250789in}}%
\pgfpathlineto{\pgfqpoint{5.440883in}{3.247690in}}%
\pgfpathlineto{\pgfqpoint{5.426900in}{3.244763in}}%
\pgfpathlineto{\pgfqpoint{5.412930in}{3.242008in}}%
\pgfpathlineto{\pgfqpoint{5.398974in}{3.239424in}}%
\pgfpathlineto{\pgfqpoint{5.391569in}{3.228330in}}%
\pgfpathlineto{\pgfqpoint{5.384166in}{3.217490in}}%
\pgfpathlineto{\pgfqpoint{5.376765in}{3.206896in}}%
\pgfpathlineto{\pgfqpoint{5.369366in}{3.196541in}}%
\pgfpathclose%
\pgfusepath{fill}%
\end{pgfscope}%
\begin{pgfscope}%
\pgfpathrectangle{\pgfqpoint{1.150000in}{0.150000in}}{\pgfqpoint{5.700000in}{5.700000in}}%
\pgfusepath{clip}%
\pgfsetbuttcap%
\pgfsetroundjoin%
\definecolor{currentfill}{rgb}{0.282623,0.140926,0.457517}%
\pgfsetfillcolor{currentfill}%
\pgfsetfillopacity{0.800000}%
\pgfsetlinewidth{0.000000pt}%
\definecolor{currentstroke}{rgb}{0.000000,0.000000,0.000000}%
\pgfsetstrokecolor{currentstroke}%
\pgfsetdash{}{0pt}%
\pgfpathmoveto{\pgfqpoint{3.833988in}{2.385805in}}%
\pgfpathlineto{\pgfqpoint{3.847455in}{2.382806in}}%
\pgfpathlineto{\pgfqpoint{3.860927in}{2.380016in}}%
\pgfpathlineto{\pgfqpoint{3.874406in}{2.377435in}}%
\pgfpathlineto{\pgfqpoint{3.887890in}{2.375061in}}%
\pgfpathlineto{\pgfqpoint{3.895761in}{2.385644in}}%
\pgfpathlineto{\pgfqpoint{3.903627in}{2.396252in}}%
\pgfpathlineto{\pgfqpoint{3.911489in}{2.406887in}}%
\pgfpathlineto{\pgfqpoint{3.919345in}{2.417550in}}%
\pgfpathlineto{\pgfqpoint{3.905869in}{2.420054in}}%
\pgfpathlineto{\pgfqpoint{3.892399in}{2.422766in}}%
\pgfpathlineto{\pgfqpoint{3.878935in}{2.425685in}}%
\pgfpathlineto{\pgfqpoint{3.865477in}{2.428814in}}%
\pgfpathlineto{\pgfqpoint{3.857612in}{2.418009in}}%
\pgfpathlineto{\pgfqpoint{3.849743in}{2.407241in}}%
\pgfpathlineto{\pgfqpoint{3.841868in}{2.396506in}}%
\pgfpathlineto{\pgfqpoint{3.833988in}{2.385805in}}%
\pgfpathclose%
\pgfusepath{fill}%
\end{pgfscope}%
\begin{pgfscope}%
\pgfpathrectangle{\pgfqpoint{1.150000in}{0.150000in}}{\pgfqpoint{5.700000in}{5.700000in}}%
\pgfusepath{clip}%
\pgfsetbuttcap%
\pgfsetroundjoin%
\definecolor{currentfill}{rgb}{0.281887,0.150881,0.465405}%
\pgfsetfillcolor{currentfill}%
\pgfsetfillopacity{0.800000}%
\pgfsetlinewidth{0.000000pt}%
\definecolor{currentstroke}{rgb}{0.000000,0.000000,0.000000}%
\pgfsetstrokecolor{currentstroke}%
\pgfsetdash{}{0pt}%
\pgfpathmoveto{\pgfqpoint{3.190730in}{2.426716in}}%
\pgfpathlineto{\pgfqpoint{3.204183in}{2.415589in}}%
\pgfpathlineto{\pgfqpoint{3.217634in}{2.404718in}}%
\pgfpathlineto{\pgfqpoint{3.231083in}{2.394102in}}%
\pgfpathlineto{\pgfqpoint{3.244530in}{2.383738in}}%
\pgfpathlineto{\pgfqpoint{3.252610in}{2.393724in}}%
\pgfpathlineto{\pgfqpoint{3.260683in}{2.403786in}}%
\pgfpathlineto{\pgfqpoint{3.268750in}{2.413926in}}%
\pgfpathlineto{\pgfqpoint{3.276811in}{2.424141in}}%
\pgfpathlineto{\pgfqpoint{3.263378in}{2.434476in}}%
\pgfpathlineto{\pgfqpoint{3.249943in}{2.445063in}}%
\pgfpathlineto{\pgfqpoint{3.236507in}{2.455905in}}%
\pgfpathlineto{\pgfqpoint{3.223069in}{2.467003in}}%
\pgfpathlineto{\pgfqpoint{3.214995in}{2.456804in}}%
\pgfpathlineto{\pgfqpoint{3.206913in}{2.446690in}}%
\pgfpathlineto{\pgfqpoint{3.198825in}{2.436661in}}%
\pgfpathlineto{\pgfqpoint{3.190730in}{2.426716in}}%
\pgfpathclose%
\pgfusepath{fill}%
\end{pgfscope}%
\begin{pgfscope}%
\pgfpathrectangle{\pgfqpoint{1.150000in}{0.150000in}}{\pgfqpoint{5.700000in}{5.700000in}}%
\pgfusepath{clip}%
\pgfsetbuttcap%
\pgfsetroundjoin%
\definecolor{currentfill}{rgb}{0.201239,0.383670,0.554294}%
\pgfsetfillcolor{currentfill}%
\pgfsetfillopacity{0.800000}%
\pgfsetlinewidth{0.000000pt}%
\definecolor{currentstroke}{rgb}{0.000000,0.000000,0.000000}%
\pgfsetstrokecolor{currentstroke}%
\pgfsetdash{}{0pt}%
\pgfpathmoveto{\pgfqpoint{2.703061in}{3.016110in}}%
\pgfpathlineto{\pgfqpoint{2.716755in}{2.994238in}}%
\pgfpathlineto{\pgfqpoint{2.730436in}{2.972715in}}%
\pgfpathlineto{\pgfqpoint{2.744106in}{2.951536in}}%
\pgfpathlineto{\pgfqpoint{2.757765in}{2.930698in}}%
\pgfpathlineto{\pgfqpoint{2.766005in}{2.940292in}}%
\pgfpathlineto{\pgfqpoint{2.774236in}{2.950034in}}%
\pgfpathlineto{\pgfqpoint{2.782459in}{2.959924in}}%
\pgfpathlineto{\pgfqpoint{2.790672in}{2.969963in}}%
\pgfpathlineto{\pgfqpoint{2.777034in}{2.990766in}}%
\pgfpathlineto{\pgfqpoint{2.763386in}{3.011909in}}%
\pgfpathlineto{\pgfqpoint{2.749726in}{3.033398in}}%
\pgfpathlineto{\pgfqpoint{2.736055in}{3.055234in}}%
\pgfpathlineto{\pgfqpoint{2.727821in}{3.045219in}}%
\pgfpathlineto{\pgfqpoint{2.719577in}{3.035359in}}%
\pgfpathlineto{\pgfqpoint{2.711324in}{3.025657in}}%
\pgfpathlineto{\pgfqpoint{2.703061in}{3.016110in}}%
\pgfpathclose%
\pgfusepath{fill}%
\end{pgfscope}%
\begin{pgfscope}%
\pgfpathrectangle{\pgfqpoint{1.150000in}{0.150000in}}{\pgfqpoint{5.700000in}{5.700000in}}%
\pgfusepath{clip}%
\pgfsetbuttcap%
\pgfsetroundjoin%
\definecolor{currentfill}{rgb}{0.159194,0.482237,0.558073}%
\pgfsetfillcolor{currentfill}%
\pgfsetfillopacity{0.800000}%
\pgfsetlinewidth{0.000000pt}%
\definecolor{currentstroke}{rgb}{0.000000,0.000000,0.000000}%
\pgfsetstrokecolor{currentstroke}%
\pgfsetdash{}{0pt}%
\pgfpathmoveto{\pgfqpoint{5.454881in}{3.250789in}}%
\pgfpathlineto{\pgfqpoint{5.468893in}{3.254058in}}%
\pgfpathlineto{\pgfqpoint{5.482919in}{3.257498in}}%
\pgfpathlineto{\pgfqpoint{5.496959in}{3.261110in}}%
\pgfpathlineto{\pgfqpoint{5.511015in}{3.264892in}}%
\pgfpathlineto{\pgfqpoint{5.518370in}{3.274801in}}%
\pgfpathlineto{\pgfqpoint{5.525728in}{3.284965in}}%
\pgfpathlineto{\pgfqpoint{5.533088in}{3.295394in}}%
\pgfpathlineto{\pgfqpoint{5.540452in}{3.306095in}}%
\pgfpathlineto{\pgfqpoint{5.526424in}{3.303050in}}%
\pgfpathlineto{\pgfqpoint{5.512410in}{3.300175in}}%
\pgfpathlineto{\pgfqpoint{5.498410in}{3.297470in}}%
\pgfpathlineto{\pgfqpoint{5.484425in}{3.294936in}}%
\pgfpathlineto{\pgfqpoint{5.477034in}{3.283488in}}%
\pgfpathlineto{\pgfqpoint{5.469647in}{3.272319in}}%
\pgfpathlineto{\pgfqpoint{5.462263in}{3.261422in}}%
\pgfpathlineto{\pgfqpoint{5.454881in}{3.250789in}}%
\pgfpathclose%
\pgfusepath{fill}%
\end{pgfscope}%
\begin{pgfscope}%
\pgfpathrectangle{\pgfqpoint{1.150000in}{0.150000in}}{\pgfqpoint{5.700000in}{5.700000in}}%
\pgfusepath{clip}%
\pgfsetbuttcap%
\pgfsetroundjoin%
\definecolor{currentfill}{rgb}{0.283229,0.120777,0.440584}%
\pgfsetfillcolor{currentfill}%
\pgfsetfillopacity{0.800000}%
\pgfsetlinewidth{0.000000pt}%
\definecolor{currentstroke}{rgb}{0.000000,0.000000,0.000000}%
\pgfsetstrokecolor{currentstroke}%
\pgfsetdash{}{0pt}%
\pgfpathmoveto{\pgfqpoint{3.384256in}{2.350357in}}%
\pgfpathlineto{\pgfqpoint{3.397687in}{2.342221in}}%
\pgfpathlineto{\pgfqpoint{3.411119in}{2.334323in}}%
\pgfpathlineto{\pgfqpoint{3.424551in}{2.326660in}}%
\pgfpathlineto{\pgfqpoint{3.437985in}{2.319232in}}%
\pgfpathlineto{\pgfqpoint{3.446001in}{2.329551in}}%
\pgfpathlineto{\pgfqpoint{3.454011in}{2.339924in}}%
\pgfpathlineto{\pgfqpoint{3.462015in}{2.350352in}}%
\pgfpathlineto{\pgfqpoint{3.470014in}{2.360834in}}%
\pgfpathlineto{\pgfqpoint{3.456592in}{2.368267in}}%
\pgfpathlineto{\pgfqpoint{3.443171in}{2.375933in}}%
\pgfpathlineto{\pgfqpoint{3.429751in}{2.383835in}}%
\pgfpathlineto{\pgfqpoint{3.416332in}{2.391973in}}%
\pgfpathlineto{\pgfqpoint{3.408322in}{2.381475in}}%
\pgfpathlineto{\pgfqpoint{3.400306in}{2.371040in}}%
\pgfpathlineto{\pgfqpoint{3.392284in}{2.360667in}}%
\pgfpathlineto{\pgfqpoint{3.384256in}{2.350357in}}%
\pgfpathclose%
\pgfusepath{fill}%
\end{pgfscope}%
\begin{pgfscope}%
\pgfpathrectangle{\pgfqpoint{1.150000in}{0.150000in}}{\pgfqpoint{5.700000in}{5.700000in}}%
\pgfusepath{clip}%
\pgfsetbuttcap%
\pgfsetroundjoin%
\definecolor{currentfill}{rgb}{0.283072,0.130895,0.449241}%
\pgfsetfillcolor{currentfill}%
\pgfsetfillopacity{0.800000}%
\pgfsetlinewidth{0.000000pt}%
\definecolor{currentstroke}{rgb}{0.000000,0.000000,0.000000}%
\pgfsetstrokecolor{currentstroke}%
\pgfsetdash{}{0pt}%
\pgfpathmoveto{\pgfqpoint{3.748567in}{2.357007in}}%
\pgfpathlineto{\pgfqpoint{3.762022in}{2.353260in}}%
\pgfpathlineto{\pgfqpoint{3.775482in}{2.349726in}}%
\pgfpathlineto{\pgfqpoint{3.788948in}{2.346405in}}%
\pgfpathlineto{\pgfqpoint{3.802418in}{2.343294in}}%
\pgfpathlineto{\pgfqpoint{3.810318in}{2.353881in}}%
\pgfpathlineto{\pgfqpoint{3.818213in}{2.364494in}}%
\pgfpathlineto{\pgfqpoint{3.826103in}{2.375135in}}%
\pgfpathlineto{\pgfqpoint{3.833988in}{2.385805in}}%
\pgfpathlineto{\pgfqpoint{3.820526in}{2.389014in}}%
\pgfpathlineto{\pgfqpoint{3.807070in}{2.392435in}}%
\pgfpathlineto{\pgfqpoint{3.793619in}{2.396067in}}%
\pgfpathlineto{\pgfqpoint{3.780173in}{2.399912in}}%
\pgfpathlineto{\pgfqpoint{3.772279in}{2.389131in}}%
\pgfpathlineto{\pgfqpoint{3.764380in}{2.378388in}}%
\pgfpathlineto{\pgfqpoint{3.756476in}{2.367680in}}%
\pgfpathlineto{\pgfqpoint{3.748567in}{2.357007in}}%
\pgfpathclose%
\pgfusepath{fill}%
\end{pgfscope}%
\begin{pgfscope}%
\pgfpathrectangle{\pgfqpoint{1.150000in}{0.150000in}}{\pgfqpoint{5.700000in}{5.700000in}}%
\pgfusepath{clip}%
\pgfsetbuttcap%
\pgfsetroundjoin%
\definecolor{currentfill}{rgb}{0.283197,0.115680,0.436115}%
\pgfsetfillcolor{currentfill}%
\pgfsetfillopacity{0.800000}%
\pgfsetlinewidth{0.000000pt}%
\definecolor{currentstroke}{rgb}{0.000000,0.000000,0.000000}%
\pgfsetstrokecolor{currentstroke}%
\pgfsetdash{}{0pt}%
\pgfpathmoveto{\pgfqpoint{3.523720in}{2.333417in}}%
\pgfpathlineto{\pgfqpoint{3.537152in}{2.327135in}}%
\pgfpathlineto{\pgfqpoint{3.550586in}{2.321078in}}%
\pgfpathlineto{\pgfqpoint{3.564023in}{2.315247in}}%
\pgfpathlineto{\pgfqpoint{3.577463in}{2.309639in}}%
\pgfpathlineto{\pgfqpoint{3.585434in}{2.320131in}}%
\pgfpathlineto{\pgfqpoint{3.593399in}{2.330663in}}%
\pgfpathlineto{\pgfqpoint{3.601359in}{2.341238in}}%
\pgfpathlineto{\pgfqpoint{3.609314in}{2.351854in}}%
\pgfpathlineto{\pgfqpoint{3.595885in}{2.357498in}}%
\pgfpathlineto{\pgfqpoint{3.582458in}{2.363364in}}%
\pgfpathlineto{\pgfqpoint{3.569034in}{2.369456in}}%
\pgfpathlineto{\pgfqpoint{3.555613in}{2.375774in}}%
\pgfpathlineto{\pgfqpoint{3.547648in}{2.365110in}}%
\pgfpathlineto{\pgfqpoint{3.539678in}{2.354497in}}%
\pgfpathlineto{\pgfqpoint{3.531702in}{2.343933in}}%
\pgfpathlineto{\pgfqpoint{3.523720in}{2.333417in}}%
\pgfpathclose%
\pgfusepath{fill}%
\end{pgfscope}%
\begin{pgfscope}%
\pgfpathrectangle{\pgfqpoint{1.150000in}{0.150000in}}{\pgfqpoint{5.700000in}{5.700000in}}%
\pgfusepath{clip}%
\pgfsetbuttcap%
\pgfsetroundjoin%
\definecolor{currentfill}{rgb}{0.260571,0.246922,0.522828}%
\pgfsetfillcolor{currentfill}%
\pgfsetfillopacity{0.800000}%
\pgfsetlinewidth{0.000000pt}%
\definecolor{currentstroke}{rgb}{0.000000,0.000000,0.000000}%
\pgfsetstrokecolor{currentstroke}%
\pgfsetdash{}{0pt}%
\pgfpathmoveto{\pgfqpoint{4.400064in}{2.613662in}}%
\pgfpathlineto{\pgfqpoint{4.413698in}{2.614978in}}%
\pgfpathlineto{\pgfqpoint{4.427342in}{2.616484in}}%
\pgfpathlineto{\pgfqpoint{4.440996in}{2.618179in}}%
\pgfpathlineto{\pgfqpoint{4.454660in}{2.620064in}}%
\pgfpathlineto{\pgfqpoint{4.462353in}{2.629761in}}%
\pgfpathlineto{\pgfqpoint{4.470041in}{2.639495in}}%
\pgfpathlineto{\pgfqpoint{4.477724in}{2.649270in}}%
\pgfpathlineto{\pgfqpoint{4.485403in}{2.659090in}}%
\pgfpathlineto{\pgfqpoint{4.471749in}{2.657527in}}%
\pgfpathlineto{\pgfqpoint{4.458105in}{2.656153in}}%
\pgfpathlineto{\pgfqpoint{4.444471in}{2.654968in}}%
\pgfpathlineto{\pgfqpoint{4.430847in}{2.653972in}}%
\pgfpathlineto{\pgfqpoint{4.423159in}{2.643820in}}%
\pgfpathlineto{\pgfqpoint{4.415465in}{2.633720in}}%
\pgfpathlineto{\pgfqpoint{4.407767in}{2.623669in}}%
\pgfpathlineto{\pgfqpoint{4.400064in}{2.613662in}}%
\pgfpathclose%
\pgfusepath{fill}%
\end{pgfscope}%
\begin{pgfscope}%
\pgfpathrectangle{\pgfqpoint{1.150000in}{0.150000in}}{\pgfqpoint{5.700000in}{5.700000in}}%
\pgfusepath{clip}%
\pgfsetbuttcap%
\pgfsetroundjoin%
\definecolor{currentfill}{rgb}{0.253935,0.265254,0.529983}%
\pgfsetfillcolor{currentfill}%
\pgfsetfillopacity{0.800000}%
\pgfsetlinewidth{0.000000pt}%
\definecolor{currentstroke}{rgb}{0.000000,0.000000,0.000000}%
\pgfsetstrokecolor{currentstroke}%
\pgfsetdash{}{0pt}%
\pgfpathmoveto{\pgfqpoint{4.485403in}{2.659090in}}%
\pgfpathlineto{\pgfqpoint{4.499067in}{2.660841in}}%
\pgfpathlineto{\pgfqpoint{4.512742in}{2.662779in}}%
\pgfpathlineto{\pgfqpoint{4.526428in}{2.664905in}}%
\pgfpathlineto{\pgfqpoint{4.540124in}{2.667219in}}%
\pgfpathlineto{\pgfqpoint{4.547787in}{2.676745in}}%
\pgfpathlineto{\pgfqpoint{4.555446in}{2.686316in}}%
\pgfpathlineto{\pgfqpoint{4.563101in}{2.695936in}}%
\pgfpathlineto{\pgfqpoint{4.570751in}{2.705609in}}%
\pgfpathlineto{\pgfqpoint{4.557065in}{2.703650in}}%
\pgfpathlineto{\pgfqpoint{4.543391in}{2.701877in}}%
\pgfpathlineto{\pgfqpoint{4.529727in}{2.700291in}}%
\pgfpathlineto{\pgfqpoint{4.516073in}{2.698893in}}%
\pgfpathlineto{\pgfqpoint{4.508412in}{2.688855in}}%
\pgfpathlineto{\pgfqpoint{4.500747in}{2.678878in}}%
\pgfpathlineto{\pgfqpoint{4.493077in}{2.668957in}}%
\pgfpathlineto{\pgfqpoint{4.485403in}{2.659090in}}%
\pgfpathclose%
\pgfusepath{fill}%
\end{pgfscope}%
\begin{pgfscope}%
\pgfpathrectangle{\pgfqpoint{1.150000in}{0.150000in}}{\pgfqpoint{5.700000in}{5.700000in}}%
\pgfusepath{clip}%
\pgfsetbuttcap%
\pgfsetroundjoin%
\definecolor{currentfill}{rgb}{0.266580,0.228262,0.514349}%
\pgfsetfillcolor{currentfill}%
\pgfsetfillopacity{0.800000}%
\pgfsetlinewidth{0.000000pt}%
\definecolor{currentstroke}{rgb}{0.000000,0.000000,0.000000}%
\pgfsetstrokecolor{currentstroke}%
\pgfsetdash{}{0pt}%
\pgfpathmoveto{\pgfqpoint{4.314730in}{2.569494in}}%
\pgfpathlineto{\pgfqpoint{4.328335in}{2.570334in}}%
\pgfpathlineto{\pgfqpoint{4.341949in}{2.571367in}}%
\pgfpathlineto{\pgfqpoint{4.355573in}{2.572591in}}%
\pgfpathlineto{\pgfqpoint{4.369206in}{2.574007in}}%
\pgfpathlineto{\pgfqpoint{4.376928in}{2.583872in}}%
\pgfpathlineto{\pgfqpoint{4.384645in}{2.593767in}}%
\pgfpathlineto{\pgfqpoint{4.392357in}{2.603696in}}%
\pgfpathlineto{\pgfqpoint{4.400064in}{2.613662in}}%
\pgfpathlineto{\pgfqpoint{4.386440in}{2.612536in}}%
\pgfpathlineto{\pgfqpoint{4.372826in}{2.611601in}}%
\pgfpathlineto{\pgfqpoint{4.359221in}{2.610858in}}%
\pgfpathlineto{\pgfqpoint{4.345626in}{2.610306in}}%
\pgfpathlineto{\pgfqpoint{4.337909in}{2.600039in}}%
\pgfpathlineto{\pgfqpoint{4.330188in}{2.589817in}}%
\pgfpathlineto{\pgfqpoint{4.322461in}{2.579636in}}%
\pgfpathlineto{\pgfqpoint{4.314730in}{2.569494in}}%
\pgfpathclose%
\pgfusepath{fill}%
\end{pgfscope}%
\begin{pgfscope}%
\pgfpathrectangle{\pgfqpoint{1.150000in}{0.150000in}}{\pgfqpoint{5.700000in}{5.700000in}}%
\pgfusepath{clip}%
\pgfsetbuttcap%
\pgfsetroundjoin%
\definecolor{currentfill}{rgb}{0.282884,0.135920,0.453427}%
\pgfsetfillcolor{currentfill}%
\pgfsetfillopacity{0.800000}%
\pgfsetlinewidth{0.000000pt}%
\definecolor{currentstroke}{rgb}{0.000000,0.000000,0.000000}%
\pgfsetstrokecolor{currentstroke}%
\pgfsetdash{}{0pt}%
\pgfpathmoveto{\pgfqpoint{3.244530in}{2.383738in}}%
\pgfpathlineto{\pgfqpoint{3.257976in}{2.373625in}}%
\pgfpathlineto{\pgfqpoint{3.271422in}{2.363762in}}%
\pgfpathlineto{\pgfqpoint{3.284866in}{2.354146in}}%
\pgfpathlineto{\pgfqpoint{3.298310in}{2.344776in}}%
\pgfpathlineto{\pgfqpoint{3.306375in}{2.354802in}}%
\pgfpathlineto{\pgfqpoint{3.314435in}{2.364897in}}%
\pgfpathlineto{\pgfqpoint{3.322488in}{2.375061in}}%
\pgfpathlineto{\pgfqpoint{3.330535in}{2.385294in}}%
\pgfpathlineto{\pgfqpoint{3.317105in}{2.394636in}}%
\pgfpathlineto{\pgfqpoint{3.303674in}{2.404223in}}%
\pgfpathlineto{\pgfqpoint{3.290243in}{2.414057in}}%
\pgfpathlineto{\pgfqpoint{3.276811in}{2.424141in}}%
\pgfpathlineto{\pgfqpoint{3.268750in}{2.413926in}}%
\pgfpathlineto{\pgfqpoint{3.260683in}{2.403786in}}%
\pgfpathlineto{\pgfqpoint{3.252610in}{2.393724in}}%
\pgfpathlineto{\pgfqpoint{3.244530in}{2.383738in}}%
\pgfpathclose%
\pgfusepath{fill}%
\end{pgfscope}%
\begin{pgfscope}%
\pgfpathrectangle{\pgfqpoint{1.150000in}{0.150000in}}{\pgfqpoint{5.700000in}{5.700000in}}%
\pgfusepath{clip}%
\pgfsetbuttcap%
\pgfsetroundjoin%
\definecolor{currentfill}{rgb}{0.246811,0.283237,0.535941}%
\pgfsetfillcolor{currentfill}%
\pgfsetfillopacity{0.800000}%
\pgfsetlinewidth{0.000000pt}%
\definecolor{currentstroke}{rgb}{0.000000,0.000000,0.000000}%
\pgfsetstrokecolor{currentstroke}%
\pgfsetdash{}{0pt}%
\pgfpathmoveto{\pgfqpoint{4.570751in}{2.705609in}}%
\pgfpathlineto{\pgfqpoint{4.584447in}{2.707755in}}%
\pgfpathlineto{\pgfqpoint{4.598154in}{2.710087in}}%
\pgfpathlineto{\pgfqpoint{4.611872in}{2.712604in}}%
\pgfpathlineto{\pgfqpoint{4.625602in}{2.715306in}}%
\pgfpathlineto{\pgfqpoint{4.633236in}{2.724663in}}%
\pgfpathlineto{\pgfqpoint{4.640865in}{2.734074in}}%
\pgfpathlineto{\pgfqpoint{4.648491in}{2.743544in}}%
\pgfpathlineto{\pgfqpoint{4.656111in}{2.753076in}}%
\pgfpathlineto{\pgfqpoint{4.642394in}{2.750760in}}%
\pgfpathlineto{\pgfqpoint{4.628687in}{2.748628in}}%
\pgfpathlineto{\pgfqpoint{4.614991in}{2.746681in}}%
\pgfpathlineto{\pgfqpoint{4.601307in}{2.744920in}}%
\pgfpathlineto{\pgfqpoint{4.593674in}{2.734991in}}%
\pgfpathlineto{\pgfqpoint{4.586037in}{2.725132in}}%
\pgfpathlineto{\pgfqpoint{4.578396in}{2.715340in}}%
\pgfpathlineto{\pgfqpoint{4.570751in}{2.705609in}}%
\pgfpathclose%
\pgfusepath{fill}%
\end{pgfscope}%
\begin{pgfscope}%
\pgfpathrectangle{\pgfqpoint{1.150000in}{0.150000in}}{\pgfqpoint{5.700000in}{5.700000in}}%
\pgfusepath{clip}%
\pgfsetbuttcap%
\pgfsetroundjoin%
\definecolor{currentfill}{rgb}{0.271828,0.209303,0.504434}%
\pgfsetfillcolor{currentfill}%
\pgfsetfillopacity{0.800000}%
\pgfsetlinewidth{0.000000pt}%
\definecolor{currentstroke}{rgb}{0.000000,0.000000,0.000000}%
\pgfsetstrokecolor{currentstroke}%
\pgfsetdash{}{0pt}%
\pgfpathmoveto{\pgfqpoint{4.229396in}{2.526776in}}%
\pgfpathlineto{\pgfqpoint{4.242973in}{2.527100in}}%
\pgfpathlineto{\pgfqpoint{4.256559in}{2.527618in}}%
\pgfpathlineto{\pgfqpoint{4.270154in}{2.528331in}}%
\pgfpathlineto{\pgfqpoint{4.283758in}{2.529236in}}%
\pgfpathlineto{\pgfqpoint{4.291509in}{2.539260in}}%
\pgfpathlineto{\pgfqpoint{4.299254in}{2.549309in}}%
\pgfpathlineto{\pgfqpoint{4.306995in}{2.559386in}}%
\pgfpathlineto{\pgfqpoint{4.314730in}{2.569494in}}%
\pgfpathlineto{\pgfqpoint{4.301135in}{2.568846in}}%
\pgfpathlineto{\pgfqpoint{4.287549in}{2.568391in}}%
\pgfpathlineto{\pgfqpoint{4.273971in}{2.568129in}}%
\pgfpathlineto{\pgfqpoint{4.260403in}{2.568062in}}%
\pgfpathlineto{\pgfqpoint{4.252658in}{2.557685in}}%
\pgfpathlineto{\pgfqpoint{4.244909in}{2.547347in}}%
\pgfpathlineto{\pgfqpoint{4.237155in}{2.537045in}}%
\pgfpathlineto{\pgfqpoint{4.229396in}{2.526776in}}%
\pgfpathclose%
\pgfusepath{fill}%
\end{pgfscope}%
\begin{pgfscope}%
\pgfpathrectangle{\pgfqpoint{1.150000in}{0.150000in}}{\pgfqpoint{5.700000in}{5.700000in}}%
\pgfusepath{clip}%
\pgfsetbuttcap%
\pgfsetroundjoin%
\definecolor{currentfill}{rgb}{0.151918,0.500685,0.557587}%
\pgfsetfillcolor{currentfill}%
\pgfsetfillopacity{0.800000}%
\pgfsetlinewidth{0.000000pt}%
\definecolor{currentstroke}{rgb}{0.000000,0.000000,0.000000}%
\pgfsetstrokecolor{currentstroke}%
\pgfsetdash{}{0pt}%
\pgfpathmoveto{\pgfqpoint{5.540452in}{3.306095in}}%
\pgfpathlineto{\pgfqpoint{5.554494in}{3.309310in}}%
\pgfpathlineto{\pgfqpoint{5.568552in}{3.312695in}}%
\pgfpathlineto{\pgfqpoint{5.582623in}{3.316250in}}%
\pgfpathlineto{\pgfqpoint{5.596710in}{3.319975in}}%
\pgfpathlineto{\pgfqpoint{5.604048in}{3.330200in}}%
\pgfpathlineto{\pgfqpoint{5.611391in}{3.340706in}}%
\pgfpathlineto{\pgfqpoint{5.618737in}{3.351503in}}%
\pgfpathlineto{\pgfqpoint{5.604672in}{3.348352in}}%
\pgfpathlineto{\pgfqpoint{5.590622in}{3.345371in}}%
\pgfpathlineto{\pgfqpoint{5.576586in}{3.342558in}}%
\pgfpathlineto{\pgfqpoint{5.562564in}{3.339916in}}%
\pgfpathlineto{\pgfqpoint{5.555190in}{3.328347in}}%
\pgfpathlineto{\pgfqpoint{5.547819in}{3.317076in}}%
\pgfpathlineto{\pgfqpoint{5.540452in}{3.306095in}}%
\pgfpathclose%
\pgfusepath{fill}%
\end{pgfscope}%
\begin{pgfscope}%
\pgfpathrectangle{\pgfqpoint{1.150000in}{0.150000in}}{\pgfqpoint{5.700000in}{5.700000in}}%
\pgfusepath{clip}%
\pgfsetbuttcap%
\pgfsetroundjoin%
\definecolor{currentfill}{rgb}{0.237441,0.305202,0.541921}%
\pgfsetfillcolor{currentfill}%
\pgfsetfillopacity{0.800000}%
\pgfsetlinewidth{0.000000pt}%
\definecolor{currentstroke}{rgb}{0.000000,0.000000,0.000000}%
\pgfsetstrokecolor{currentstroke}%
\pgfsetdash{}{0pt}%
\pgfpathmoveto{\pgfqpoint{4.656111in}{2.753076in}}%
\pgfpathlineto{\pgfqpoint{4.669840in}{2.755577in}}%
\pgfpathlineto{\pgfqpoint{4.683581in}{2.758262in}}%
\pgfpathlineto{\pgfqpoint{4.697333in}{2.761131in}}%
\pgfpathlineto{\pgfqpoint{4.711097in}{2.764183in}}%
\pgfpathlineto{\pgfqpoint{4.718701in}{2.773378in}}%
\pgfpathlineto{\pgfqpoint{4.726301in}{2.782638in}}%
\pgfpathlineto{\pgfqpoint{4.733896in}{2.791967in}}%
\pgfpathlineto{\pgfqpoint{4.741488in}{2.801370in}}%
\pgfpathlineto{\pgfqpoint{4.727737in}{2.798736in}}%
\pgfpathlineto{\pgfqpoint{4.713998in}{2.796285in}}%
\pgfpathlineto{\pgfqpoint{4.700270in}{2.794017in}}%
\pgfpathlineto{\pgfqpoint{4.686553in}{2.791933in}}%
\pgfpathlineto{\pgfqpoint{4.678949in}{2.782100in}}%
\pgfpathlineto{\pgfqpoint{4.671340in}{2.772350in}}%
\pgfpathlineto{\pgfqpoint{4.663728in}{2.762677in}}%
\pgfpathlineto{\pgfqpoint{4.656111in}{2.753076in}}%
\pgfpathclose%
\pgfusepath{fill}%
\end{pgfscope}%
\begin{pgfscope}%
\pgfpathrectangle{\pgfqpoint{1.150000in}{0.150000in}}{\pgfqpoint{5.700000in}{5.700000in}}%
\pgfusepath{clip}%
\pgfsetbuttcap%
\pgfsetroundjoin%
\definecolor{currentfill}{rgb}{0.266580,0.228262,0.514349}%
\pgfsetfillcolor{currentfill}%
\pgfsetfillopacity{0.800000}%
\pgfsetlinewidth{0.000000pt}%
\definecolor{currentstroke}{rgb}{0.000000,0.000000,0.000000}%
\pgfsetstrokecolor{currentstroke}%
\pgfsetdash{}{0pt}%
\pgfpathmoveto{\pgfqpoint{2.942313in}{2.603242in}}%
\pgfpathlineto{\pgfqpoint{2.955847in}{2.587680in}}%
\pgfpathlineto{\pgfqpoint{2.969375in}{2.572409in}}%
\pgfpathlineto{\pgfqpoint{2.982898in}{2.557424in}}%
\pgfpathlineto{\pgfqpoint{2.996415in}{2.542725in}}%
\pgfpathlineto{\pgfqpoint{3.004589in}{2.552112in}}%
\pgfpathlineto{\pgfqpoint{3.012756in}{2.561607in}}%
\pgfpathlineto{\pgfqpoint{3.020914in}{2.571210in}}%
\pgfpathlineto{\pgfqpoint{3.029066in}{2.580921in}}%
\pgfpathlineto{\pgfqpoint{3.015567in}{2.595557in}}%
\pgfpathlineto{\pgfqpoint{3.002062in}{2.610478in}}%
\pgfpathlineto{\pgfqpoint{2.988553in}{2.625686in}}%
\pgfpathlineto{\pgfqpoint{2.975037in}{2.641184in}}%
\pgfpathlineto{\pgfqpoint{2.966868in}{2.631524in}}%
\pgfpathlineto{\pgfqpoint{2.958691in}{2.621981in}}%
\pgfpathlineto{\pgfqpoint{2.950506in}{2.612553in}}%
\pgfpathlineto{\pgfqpoint{2.942313in}{2.603242in}}%
\pgfpathclose%
\pgfusepath{fill}%
\end{pgfscope}%
\begin{pgfscope}%
\pgfpathrectangle{\pgfqpoint{1.150000in}{0.150000in}}{\pgfqpoint{5.700000in}{5.700000in}}%
\pgfusepath{clip}%
\pgfsetbuttcap%
\pgfsetroundjoin%
\definecolor{currentfill}{rgb}{0.257322,0.256130,0.526563}%
\pgfsetfillcolor{currentfill}%
\pgfsetfillopacity{0.800000}%
\pgfsetlinewidth{0.000000pt}%
\definecolor{currentstroke}{rgb}{0.000000,0.000000,0.000000}%
\pgfsetstrokecolor{currentstroke}%
\pgfsetdash{}{0pt}%
\pgfpathmoveto{\pgfqpoint{2.888109in}{2.668438in}}%
\pgfpathlineto{\pgfqpoint{2.901670in}{2.651691in}}%
\pgfpathlineto{\pgfqpoint{2.915225in}{2.635245in}}%
\pgfpathlineto{\pgfqpoint{2.928772in}{2.619096in}}%
\pgfpathlineto{\pgfqpoint{2.942313in}{2.603242in}}%
\pgfpathlineto{\pgfqpoint{2.950506in}{2.612553in}}%
\pgfpathlineto{\pgfqpoint{2.958691in}{2.621981in}}%
\pgfpathlineto{\pgfqpoint{2.966868in}{2.631524in}}%
\pgfpathlineto{\pgfqpoint{2.975037in}{2.641184in}}%
\pgfpathlineto{\pgfqpoint{2.961515in}{2.656974in}}%
\pgfpathlineto{\pgfqpoint{2.947987in}{2.673059in}}%
\pgfpathlineto{\pgfqpoint{2.934452in}{2.689441in}}%
\pgfpathlineto{\pgfqpoint{2.920911in}{2.706123in}}%
\pgfpathlineto{\pgfqpoint{2.912723in}{2.696516in}}%
\pgfpathlineto{\pgfqpoint{2.904527in}{2.687032in}}%
\pgfpathlineto{\pgfqpoint{2.896322in}{2.677673in}}%
\pgfpathlineto{\pgfqpoint{2.888109in}{2.668438in}}%
\pgfpathclose%
\pgfusepath{fill}%
\end{pgfscope}%
\begin{pgfscope}%
\pgfpathrectangle{\pgfqpoint{1.150000in}{0.150000in}}{\pgfqpoint{5.700000in}{5.700000in}}%
\pgfusepath{clip}%
\pgfsetbuttcap%
\pgfsetroundjoin%
\definecolor{currentfill}{rgb}{0.229739,0.322361,0.545706}%
\pgfsetfillcolor{currentfill}%
\pgfsetfillopacity{0.800000}%
\pgfsetlinewidth{0.000000pt}%
\definecolor{currentstroke}{rgb}{0.000000,0.000000,0.000000}%
\pgfsetstrokecolor{currentstroke}%
\pgfsetdash{}{0pt}%
\pgfpathmoveto{\pgfqpoint{4.741488in}{2.801370in}}%
\pgfpathlineto{\pgfqpoint{4.755251in}{2.804187in}}%
\pgfpathlineto{\pgfqpoint{4.769025in}{2.807187in}}%
\pgfpathlineto{\pgfqpoint{4.782812in}{2.810368in}}%
\pgfpathlineto{\pgfqpoint{4.796611in}{2.813731in}}%
\pgfpathlineto{\pgfqpoint{4.804185in}{2.822776in}}%
\pgfpathlineto{\pgfqpoint{4.811755in}{2.831898in}}%
\pgfpathlineto{\pgfqpoint{4.819321in}{2.841103in}}%
\pgfpathlineto{\pgfqpoint{4.826883in}{2.850394in}}%
\pgfpathlineto{\pgfqpoint{4.813098in}{2.847481in}}%
\pgfpathlineto{\pgfqpoint{4.799325in}{2.844749in}}%
\pgfpathlineto{\pgfqpoint{4.785564in}{2.842199in}}%
\pgfpathlineto{\pgfqpoint{4.771815in}{2.839831in}}%
\pgfpathlineto{\pgfqpoint{4.764239in}{2.830079in}}%
\pgfpathlineto{\pgfqpoint{4.756659in}{2.820421in}}%
\pgfpathlineto{\pgfqpoint{4.749075in}{2.810854in}}%
\pgfpathlineto{\pgfqpoint{4.741488in}{2.801370in}}%
\pgfpathclose%
\pgfusepath{fill}%
\end{pgfscope}%
\begin{pgfscope}%
\pgfpathrectangle{\pgfqpoint{1.150000in}{0.150000in}}{\pgfqpoint{5.700000in}{5.700000in}}%
\pgfusepath{clip}%
\pgfsetbuttcap%
\pgfsetroundjoin%
\definecolor{currentfill}{rgb}{0.276194,0.190074,0.493001}%
\pgfsetfillcolor{currentfill}%
\pgfsetfillopacity{0.800000}%
\pgfsetlinewidth{0.000000pt}%
\definecolor{currentstroke}{rgb}{0.000000,0.000000,0.000000}%
\pgfsetstrokecolor{currentstroke}%
\pgfsetdash{}{0pt}%
\pgfpathmoveto{\pgfqpoint{4.144053in}{2.485723in}}%
\pgfpathlineto{\pgfqpoint{4.157605in}{2.485489in}}%
\pgfpathlineto{\pgfqpoint{4.171165in}{2.485452in}}%
\pgfpathlineto{\pgfqpoint{4.184734in}{2.485611in}}%
\pgfpathlineto{\pgfqpoint{4.198311in}{2.485966in}}%
\pgfpathlineto{\pgfqpoint{4.206089in}{2.496134in}}%
\pgfpathlineto{\pgfqpoint{4.213863in}{2.506323in}}%
\pgfpathlineto{\pgfqpoint{4.221632in}{2.516536in}}%
\pgfpathlineto{\pgfqpoint{4.229396in}{2.526776in}}%
\pgfpathlineto{\pgfqpoint{4.215827in}{2.526647in}}%
\pgfpathlineto{\pgfqpoint{4.202267in}{2.526713in}}%
\pgfpathlineto{\pgfqpoint{4.188715in}{2.526976in}}%
\pgfpathlineto{\pgfqpoint{4.175172in}{2.527436in}}%
\pgfpathlineto{\pgfqpoint{4.167400in}{2.516959in}}%
\pgfpathlineto{\pgfqpoint{4.159622in}{2.506516in}}%
\pgfpathlineto{\pgfqpoint{4.151840in}{2.496105in}}%
\pgfpathlineto{\pgfqpoint{4.144053in}{2.485723in}}%
\pgfpathclose%
\pgfusepath{fill}%
\end{pgfscope}%
\begin{pgfscope}%
\pgfpathrectangle{\pgfqpoint{1.150000in}{0.150000in}}{\pgfqpoint{5.700000in}{5.700000in}}%
\pgfusepath{clip}%
\pgfsetbuttcap%
\pgfsetroundjoin%
\definecolor{currentfill}{rgb}{0.273006,0.204520,0.501721}%
\pgfsetfillcolor{currentfill}%
\pgfsetfillopacity{0.800000}%
\pgfsetlinewidth{0.000000pt}%
\definecolor{currentstroke}{rgb}{0.000000,0.000000,0.000000}%
\pgfsetstrokecolor{currentstroke}%
\pgfsetdash{}{0pt}%
\pgfpathmoveto{\pgfqpoint{2.996415in}{2.542725in}}%
\pgfpathlineto{\pgfqpoint{3.009927in}{2.528309in}}%
\pgfpathlineto{\pgfqpoint{3.023434in}{2.514174in}}%
\pgfpathlineto{\pgfqpoint{3.036936in}{2.500316in}}%
\pgfpathlineto{\pgfqpoint{3.050434in}{2.486735in}}%
\pgfpathlineto{\pgfqpoint{3.058590in}{2.496196in}}%
\pgfpathlineto{\pgfqpoint{3.066738in}{2.505758in}}%
\pgfpathlineto{\pgfqpoint{3.074879in}{2.515419in}}%
\pgfpathlineto{\pgfqpoint{3.083013in}{2.525181in}}%
\pgfpathlineto{\pgfqpoint{3.069533in}{2.538699in}}%
\pgfpathlineto{\pgfqpoint{3.056048in}{2.552494in}}%
\pgfpathlineto{\pgfqpoint{3.042559in}{2.566567in}}%
\pgfpathlineto{\pgfqpoint{3.029066in}{2.580921in}}%
\pgfpathlineto{\pgfqpoint{3.020914in}{2.571210in}}%
\pgfpathlineto{\pgfqpoint{3.012756in}{2.561607in}}%
\pgfpathlineto{\pgfqpoint{3.004589in}{2.552112in}}%
\pgfpathlineto{\pgfqpoint{2.996415in}{2.542725in}}%
\pgfpathclose%
\pgfusepath{fill}%
\end{pgfscope}%
\begin{pgfscope}%
\pgfpathrectangle{\pgfqpoint{1.150000in}{0.150000in}}{\pgfqpoint{5.700000in}{5.700000in}}%
\pgfusepath{clip}%
\pgfsetbuttcap%
\pgfsetroundjoin%
\definecolor{currentfill}{rgb}{0.246811,0.283237,0.535941}%
\pgfsetfillcolor{currentfill}%
\pgfsetfillopacity{0.800000}%
\pgfsetlinewidth{0.000000pt}%
\definecolor{currentstroke}{rgb}{0.000000,0.000000,0.000000}%
\pgfsetstrokecolor{currentstroke}%
\pgfsetdash{}{0pt}%
\pgfpathmoveto{\pgfqpoint{2.833787in}{2.738477in}}%
\pgfpathlineto{\pgfqpoint{2.847379in}{2.720503in}}%
\pgfpathlineto{\pgfqpoint{2.860964in}{2.702841in}}%
\pgfpathlineto{\pgfqpoint{2.874540in}{2.685487in}}%
\pgfpathlineto{\pgfqpoint{2.888109in}{2.668438in}}%
\pgfpathlineto{\pgfqpoint{2.896322in}{2.677673in}}%
\pgfpathlineto{\pgfqpoint{2.904527in}{2.687032in}}%
\pgfpathlineto{\pgfqpoint{2.912723in}{2.696516in}}%
\pgfpathlineto{\pgfqpoint{2.920911in}{2.706123in}}%
\pgfpathlineto{\pgfqpoint{2.907362in}{2.723107in}}%
\pgfpathlineto{\pgfqpoint{2.893805in}{2.740397in}}%
\pgfpathlineto{\pgfqpoint{2.880241in}{2.757994in}}%
\pgfpathlineto{\pgfqpoint{2.866669in}{2.775902in}}%
\pgfpathlineto{\pgfqpoint{2.858461in}{2.766347in}}%
\pgfpathlineto{\pgfqpoint{2.850245in}{2.756925in}}%
\pgfpathlineto{\pgfqpoint{2.842020in}{2.747635in}}%
\pgfpathlineto{\pgfqpoint{2.833787in}{2.738477in}}%
\pgfpathclose%
\pgfusepath{fill}%
\end{pgfscope}%
\begin{pgfscope}%
\pgfpathrectangle{\pgfqpoint{1.150000in}{0.150000in}}{\pgfqpoint{5.700000in}{5.700000in}}%
\pgfusepath{clip}%
\pgfsetbuttcap%
\pgfsetroundjoin%
\definecolor{currentfill}{rgb}{0.220057,0.343307,0.549413}%
\pgfsetfillcolor{currentfill}%
\pgfsetfillopacity{0.800000}%
\pgfsetlinewidth{0.000000pt}%
\definecolor{currentstroke}{rgb}{0.000000,0.000000,0.000000}%
\pgfsetstrokecolor{currentstroke}%
\pgfsetdash{}{0pt}%
\pgfpathmoveto{\pgfqpoint{4.826883in}{2.850394in}}%
\pgfpathlineto{\pgfqpoint{4.840680in}{2.853488in}}%
\pgfpathlineto{\pgfqpoint{4.854489in}{2.856763in}}%
\pgfpathlineto{\pgfqpoint{4.868311in}{2.860218in}}%
\pgfpathlineto{\pgfqpoint{4.882145in}{2.863854in}}%
\pgfpathlineto{\pgfqpoint{4.889689in}{2.872768in}}%
\pgfpathlineto{\pgfqpoint{4.897229in}{2.881772in}}%
\pgfpathlineto{\pgfqpoint{4.904766in}{2.890872in}}%
\pgfpathlineto{\pgfqpoint{4.912299in}{2.900073in}}%
\pgfpathlineto{\pgfqpoint{4.898480in}{2.896920in}}%
\pgfpathlineto{\pgfqpoint{4.884673in}{2.893946in}}%
\pgfpathlineto{\pgfqpoint{4.870879in}{2.891152in}}%
\pgfpathlineto{\pgfqpoint{4.857097in}{2.888539in}}%
\pgfpathlineto{\pgfqpoint{4.849548in}{2.878845in}}%
\pgfpathlineto{\pgfqpoint{4.841996in}{2.869260in}}%
\pgfpathlineto{\pgfqpoint{4.834441in}{2.859778in}}%
\pgfpathlineto{\pgfqpoint{4.826883in}{2.850394in}}%
\pgfpathclose%
\pgfusepath{fill}%
\end{pgfscope}%
\begin{pgfscope}%
\pgfpathrectangle{\pgfqpoint{1.150000in}{0.150000in}}{\pgfqpoint{5.700000in}{5.700000in}}%
\pgfusepath{clip}%
\pgfsetbuttcap%
\pgfsetroundjoin%
\definecolor{currentfill}{rgb}{0.283229,0.120777,0.440584}%
\pgfsetfillcolor{currentfill}%
\pgfsetfillopacity{0.800000}%
\pgfsetlinewidth{0.000000pt}%
\definecolor{currentstroke}{rgb}{0.000000,0.000000,0.000000}%
\pgfsetstrokecolor{currentstroke}%
\pgfsetdash{}{0pt}%
\pgfpathmoveto{\pgfqpoint{3.663066in}{2.331496in}}%
\pgfpathlineto{\pgfqpoint{3.676513in}{2.326954in}}%
\pgfpathlineto{\pgfqpoint{3.689964in}{2.322630in}}%
\pgfpathlineto{\pgfqpoint{3.703419in}{2.318521in}}%
\pgfpathlineto{\pgfqpoint{3.716879in}{2.314628in}}%
\pgfpathlineto{\pgfqpoint{3.724809in}{2.325178in}}%
\pgfpathlineto{\pgfqpoint{3.732733in}{2.335757in}}%
\pgfpathlineto{\pgfqpoint{3.740653in}{2.346366in}}%
\pgfpathlineto{\pgfqpoint{3.748567in}{2.357007in}}%
\pgfpathlineto{\pgfqpoint{3.735117in}{2.360968in}}%
\pgfpathlineto{\pgfqpoint{3.721671in}{2.365143in}}%
\pgfpathlineto{\pgfqpoint{3.708229in}{2.369535in}}%
\pgfpathlineto{\pgfqpoint{3.694791in}{2.374143in}}%
\pgfpathlineto{\pgfqpoint{3.686868in}{2.363424in}}%
\pgfpathlineto{\pgfqpoint{3.678939in}{2.352744in}}%
\pgfpathlineto{\pgfqpoint{3.671005in}{2.342101in}}%
\pgfpathlineto{\pgfqpoint{3.663066in}{2.331496in}}%
\pgfpathclose%
\pgfusepath{fill}%
\end{pgfscope}%
\begin{pgfscope}%
\pgfpathrectangle{\pgfqpoint{1.150000in}{0.150000in}}{\pgfqpoint{5.700000in}{5.700000in}}%
\pgfusepath{clip}%
\pgfsetbuttcap%
\pgfsetroundjoin%
\definecolor{currentfill}{rgb}{0.278826,0.175490,0.483397}%
\pgfsetfillcolor{currentfill}%
\pgfsetfillopacity{0.800000}%
\pgfsetlinewidth{0.000000pt}%
\definecolor{currentstroke}{rgb}{0.000000,0.000000,0.000000}%
\pgfsetstrokecolor{currentstroke}%
\pgfsetdash{}{0pt}%
\pgfpathmoveto{\pgfqpoint{4.058695in}{2.446574in}}%
\pgfpathlineto{\pgfqpoint{4.072224in}{2.445739in}}%
\pgfpathlineto{\pgfqpoint{4.085760in}{2.445104in}}%
\pgfpathlineto{\pgfqpoint{4.099304in}{2.444667in}}%
\pgfpathlineto{\pgfqpoint{4.112856in}{2.444430in}}%
\pgfpathlineto{\pgfqpoint{4.120663in}{2.454723in}}%
\pgfpathlineto{\pgfqpoint{4.128464in}{2.465035in}}%
\pgfpathlineto{\pgfqpoint{4.136261in}{2.475367in}}%
\pgfpathlineto{\pgfqpoint{4.144053in}{2.485723in}}%
\pgfpathlineto{\pgfqpoint{4.130509in}{2.486155in}}%
\pgfpathlineto{\pgfqpoint{4.116974in}{2.486785in}}%
\pgfpathlineto{\pgfqpoint{4.103445in}{2.487615in}}%
\pgfpathlineto{\pgfqpoint{4.089925in}{2.488643in}}%
\pgfpathlineto{\pgfqpoint{4.082125in}{2.478082in}}%
\pgfpathlineto{\pgfqpoint{4.074320in}{2.467551in}}%
\pgfpathlineto{\pgfqpoint{4.066510in}{2.457050in}}%
\pgfpathlineto{\pgfqpoint{4.058695in}{2.446574in}}%
\pgfpathclose%
\pgfusepath{fill}%
\end{pgfscope}%
\begin{pgfscope}%
\pgfpathrectangle{\pgfqpoint{1.150000in}{0.150000in}}{\pgfqpoint{5.700000in}{5.700000in}}%
\pgfusepath{clip}%
\pgfsetbuttcap%
\pgfsetroundjoin%
\definecolor{currentfill}{rgb}{0.210503,0.363727,0.552206}%
\pgfsetfillcolor{currentfill}%
\pgfsetfillopacity{0.800000}%
\pgfsetlinewidth{0.000000pt}%
\definecolor{currentstroke}{rgb}{0.000000,0.000000,0.000000}%
\pgfsetstrokecolor{currentstroke}%
\pgfsetdash{}{0pt}%
\pgfpathmoveto{\pgfqpoint{4.912299in}{2.900073in}}%
\pgfpathlineto{\pgfqpoint{4.926130in}{2.903406in}}%
\pgfpathlineto{\pgfqpoint{4.939975in}{2.906918in}}%
\pgfpathlineto{\pgfqpoint{4.953832in}{2.910609in}}%
\pgfpathlineto{\pgfqpoint{4.967702in}{2.914479in}}%
\pgfpathlineto{\pgfqpoint{4.975216in}{2.923285in}}%
\pgfpathlineto{\pgfqpoint{4.982726in}{2.932196in}}%
\pgfpathlineto{\pgfqpoint{4.990233in}{2.941218in}}%
\pgfpathlineto{\pgfqpoint{4.997737in}{2.950356in}}%
\pgfpathlineto{\pgfqpoint{4.983884in}{2.947001in}}%
\pgfpathlineto{\pgfqpoint{4.970043in}{2.943824in}}%
\pgfpathlineto{\pgfqpoint{4.956215in}{2.940825in}}%
\pgfpathlineto{\pgfqpoint{4.942400in}{2.938006in}}%
\pgfpathlineto{\pgfqpoint{4.934879in}{2.928342in}}%
\pgfpathlineto{\pgfqpoint{4.927355in}{2.918802in}}%
\pgfpathlineto{\pgfqpoint{4.919828in}{2.909381in}}%
\pgfpathlineto{\pgfqpoint{4.912299in}{2.900073in}}%
\pgfpathclose%
\pgfusepath{fill}%
\end{pgfscope}%
\begin{pgfscope}%
\pgfpathrectangle{\pgfqpoint{1.150000in}{0.150000in}}{\pgfqpoint{5.700000in}{5.700000in}}%
\pgfusepath{clip}%
\pgfsetbuttcap%
\pgfsetroundjoin%
\definecolor{currentfill}{rgb}{0.278012,0.180367,0.486697}%
\pgfsetfillcolor{currentfill}%
\pgfsetfillopacity{0.800000}%
\pgfsetlinewidth{0.000000pt}%
\definecolor{currentstroke}{rgb}{0.000000,0.000000,0.000000}%
\pgfsetstrokecolor{currentstroke}%
\pgfsetdash{}{0pt}%
\pgfpathmoveto{\pgfqpoint{3.050434in}{2.486735in}}%
\pgfpathlineto{\pgfqpoint{3.063927in}{2.473428in}}%
\pgfpathlineto{\pgfqpoint{3.077417in}{2.460393in}}%
\pgfpathlineto{\pgfqpoint{3.090902in}{2.447627in}}%
\pgfpathlineto{\pgfqpoint{3.104384in}{2.435130in}}%
\pgfpathlineto{\pgfqpoint{3.112523in}{2.444665in}}%
\pgfpathlineto{\pgfqpoint{3.120655in}{2.454292in}}%
\pgfpathlineto{\pgfqpoint{3.128779in}{2.464011in}}%
\pgfpathlineto{\pgfqpoint{3.136896in}{2.473823in}}%
\pgfpathlineto{\pgfqpoint{3.123431in}{2.486259in}}%
\pgfpathlineto{\pgfqpoint{3.109962in}{2.498962in}}%
\pgfpathlineto{\pgfqpoint{3.096490in}{2.511936in}}%
\pgfpathlineto{\pgfqpoint{3.083013in}{2.525181in}}%
\pgfpathlineto{\pgfqpoint{3.074879in}{2.515419in}}%
\pgfpathlineto{\pgfqpoint{3.066738in}{2.505758in}}%
\pgfpathlineto{\pgfqpoint{3.058590in}{2.496196in}}%
\pgfpathlineto{\pgfqpoint{3.050434in}{2.486735in}}%
\pgfpathclose%
\pgfusepath{fill}%
\end{pgfscope}%
\begin{pgfscope}%
\pgfpathrectangle{\pgfqpoint{1.150000in}{0.150000in}}{\pgfqpoint{5.700000in}{5.700000in}}%
\pgfusepath{clip}%
\pgfsetbuttcap%
\pgfsetroundjoin%
\definecolor{currentfill}{rgb}{0.203063,0.379716,0.553925}%
\pgfsetfillcolor{currentfill}%
\pgfsetfillopacity{0.800000}%
\pgfsetlinewidth{0.000000pt}%
\definecolor{currentstroke}{rgb}{0.000000,0.000000,0.000000}%
\pgfsetstrokecolor{currentstroke}%
\pgfsetdash{}{0pt}%
\pgfpathmoveto{\pgfqpoint{4.997737in}{2.950356in}}%
\pgfpathlineto{\pgfqpoint{5.011604in}{2.953890in}}%
\pgfpathlineto{\pgfqpoint{5.025483in}{2.957601in}}%
\pgfpathlineto{\pgfqpoint{5.039376in}{2.961490in}}%
\pgfpathlineto{\pgfqpoint{5.053282in}{2.965557in}}%
\pgfpathlineto{\pgfqpoint{5.060766in}{2.974284in}}%
\pgfpathlineto{\pgfqpoint{5.068247in}{2.983132in}}%
\pgfpathlineto{\pgfqpoint{5.075725in}{2.992108in}}%
\pgfpathlineto{\pgfqpoint{5.083201in}{3.001218in}}%
\pgfpathlineto{\pgfqpoint{5.069313in}{2.997698in}}%
\pgfpathlineto{\pgfqpoint{5.055439in}{2.994355in}}%
\pgfpathlineto{\pgfqpoint{5.041577in}{2.991190in}}%
\pgfpathlineto{\pgfqpoint{5.027728in}{2.988202in}}%
\pgfpathlineto{\pgfqpoint{5.020234in}{2.978535in}}%
\pgfpathlineto{\pgfqpoint{5.012738in}{2.969009in}}%
\pgfpathlineto{\pgfqpoint{5.005239in}{2.959618in}}%
\pgfpathlineto{\pgfqpoint{4.997737in}{2.950356in}}%
\pgfpathclose%
\pgfusepath{fill}%
\end{pgfscope}%
\begin{pgfscope}%
\pgfpathrectangle{\pgfqpoint{1.150000in}{0.150000in}}{\pgfqpoint{5.700000in}{5.700000in}}%
\pgfusepath{clip}%
\pgfsetbuttcap%
\pgfsetroundjoin%
\definecolor{currentfill}{rgb}{0.233603,0.313828,0.543914}%
\pgfsetfillcolor{currentfill}%
\pgfsetfillopacity{0.800000}%
\pgfsetlinewidth{0.000000pt}%
\definecolor{currentstroke}{rgb}{0.000000,0.000000,0.000000}%
\pgfsetstrokecolor{currentstroke}%
\pgfsetdash{}{0pt}%
\pgfpathmoveto{\pgfqpoint{2.779326in}{2.813536in}}%
\pgfpathlineto{\pgfqpoint{2.792955in}{2.794290in}}%
\pgfpathlineto{\pgfqpoint{2.806575in}{2.775367in}}%
\pgfpathlineto{\pgfqpoint{2.820185in}{2.756764in}}%
\pgfpathlineto{\pgfqpoint{2.833787in}{2.738477in}}%
\pgfpathlineto{\pgfqpoint{2.842020in}{2.747635in}}%
\pgfpathlineto{\pgfqpoint{2.850245in}{2.756925in}}%
\pgfpathlineto{\pgfqpoint{2.858461in}{2.766347in}}%
\pgfpathlineto{\pgfqpoint{2.866669in}{2.775902in}}%
\pgfpathlineto{\pgfqpoint{2.853089in}{2.794123in}}%
\pgfpathlineto{\pgfqpoint{2.839499in}{2.812661in}}%
\pgfpathlineto{\pgfqpoint{2.825901in}{2.831518in}}%
\pgfpathlineto{\pgfqpoint{2.812294in}{2.850697in}}%
\pgfpathlineto{\pgfqpoint{2.804065in}{2.841196in}}%
\pgfpathlineto{\pgfqpoint{2.795828in}{2.831836in}}%
\pgfpathlineto{\pgfqpoint{2.787582in}{2.822616in}}%
\pgfpathlineto{\pgfqpoint{2.779326in}{2.813536in}}%
\pgfpathclose%
\pgfusepath{fill}%
\end{pgfscope}%
\begin{pgfscope}%
\pgfpathrectangle{\pgfqpoint{1.150000in}{0.150000in}}{\pgfqpoint{5.700000in}{5.700000in}}%
\pgfusepath{clip}%
\pgfsetbuttcap%
\pgfsetroundjoin%
\definecolor{currentfill}{rgb}{0.280868,0.160771,0.472899}%
\pgfsetfillcolor{currentfill}%
\pgfsetfillopacity{0.800000}%
\pgfsetlinewidth{0.000000pt}%
\definecolor{currentstroke}{rgb}{0.000000,0.000000,0.000000}%
\pgfsetstrokecolor{currentstroke}%
\pgfsetdash{}{0pt}%
\pgfpathmoveto{\pgfqpoint{3.973311in}{2.409591in}}%
\pgfpathlineto{\pgfqpoint{3.986819in}{2.408112in}}%
\pgfpathlineto{\pgfqpoint{4.000334in}{2.406836in}}%
\pgfpathlineto{\pgfqpoint{4.013856in}{2.405761in}}%
\pgfpathlineto{\pgfqpoint{4.027386in}{2.404888in}}%
\pgfpathlineto{\pgfqpoint{4.035221in}{2.415282in}}%
\pgfpathlineto{\pgfqpoint{4.043050in}{2.425693in}}%
\pgfpathlineto{\pgfqpoint{4.050875in}{2.436123in}}%
\pgfpathlineto{\pgfqpoint{4.058695in}{2.446574in}}%
\pgfpathlineto{\pgfqpoint{4.045174in}{2.447610in}}%
\pgfpathlineto{\pgfqpoint{4.031660in}{2.448847in}}%
\pgfpathlineto{\pgfqpoint{4.018153in}{2.450285in}}%
\pgfpathlineto{\pgfqpoint{4.004653in}{2.451926in}}%
\pgfpathlineto{\pgfqpoint{3.996825in}{2.441301in}}%
\pgfpathlineto{\pgfqpoint{3.988992in}{2.430705in}}%
\pgfpathlineto{\pgfqpoint{3.981154in}{2.420135in}}%
\pgfpathlineto{\pgfqpoint{3.973311in}{2.409591in}}%
\pgfpathclose%
\pgfusepath{fill}%
\end{pgfscope}%
\begin{pgfscope}%
\pgfpathrectangle{\pgfqpoint{1.150000in}{0.150000in}}{\pgfqpoint{5.700000in}{5.700000in}}%
\pgfusepath{clip}%
\pgfsetbuttcap%
\pgfsetroundjoin%
\definecolor{currentfill}{rgb}{0.283091,0.110553,0.431554}%
\pgfsetfillcolor{currentfill}%
\pgfsetfillopacity{0.800000}%
\pgfsetlinewidth{0.000000pt}%
\definecolor{currentstroke}{rgb}{0.000000,0.000000,0.000000}%
\pgfsetstrokecolor{currentstroke}%
\pgfsetdash{}{0pt}%
\pgfpathmoveto{\pgfqpoint{3.437985in}{2.319232in}}%
\pgfpathlineto{\pgfqpoint{3.451421in}{2.312035in}}%
\pgfpathlineto{\pgfqpoint{3.464858in}{2.305071in}}%
\pgfpathlineto{\pgfqpoint{3.478297in}{2.298336in}}%
\pgfpathlineto{\pgfqpoint{3.491738in}{2.291830in}}%
\pgfpathlineto{\pgfqpoint{3.499742in}{2.302157in}}%
\pgfpathlineto{\pgfqpoint{3.507740in}{2.312530in}}%
\pgfpathlineto{\pgfqpoint{3.515733in}{2.322950in}}%
\pgfpathlineto{\pgfqpoint{3.523720in}{2.333417in}}%
\pgfpathlineto{\pgfqpoint{3.510291in}{2.339927in}}%
\pgfpathlineto{\pgfqpoint{3.496863in}{2.346666in}}%
\pgfpathlineto{\pgfqpoint{3.483438in}{2.353635in}}%
\pgfpathlineto{\pgfqpoint{3.470014in}{2.360834in}}%
\pgfpathlineto{\pgfqpoint{3.462015in}{2.350352in}}%
\pgfpathlineto{\pgfqpoint{3.454011in}{2.339924in}}%
\pgfpathlineto{\pgfqpoint{3.446001in}{2.329551in}}%
\pgfpathlineto{\pgfqpoint{3.437985in}{2.319232in}}%
\pgfpathclose%
\pgfusepath{fill}%
\end{pgfscope}%
\begin{pgfscope}%
\pgfpathrectangle{\pgfqpoint{1.150000in}{0.150000in}}{\pgfqpoint{5.700000in}{5.700000in}}%
\pgfusepath{clip}%
\pgfsetbuttcap%
\pgfsetroundjoin%
\definecolor{currentfill}{rgb}{0.194100,0.399323,0.555565}%
\pgfsetfillcolor{currentfill}%
\pgfsetfillopacity{0.800000}%
\pgfsetlinewidth{0.000000pt}%
\definecolor{currentstroke}{rgb}{0.000000,0.000000,0.000000}%
\pgfsetstrokecolor{currentstroke}%
\pgfsetdash{}{0pt}%
\pgfpathmoveto{\pgfqpoint{5.083201in}{3.001218in}}%
\pgfpathlineto{\pgfqpoint{5.097103in}{3.004914in}}%
\pgfpathlineto{\pgfqpoint{5.111017in}{3.008787in}}%
\pgfpathlineto{\pgfqpoint{5.124946in}{3.012836in}}%
\pgfpathlineto{\pgfqpoint{5.138888in}{3.017061in}}%
\pgfpathlineto{\pgfqpoint{5.146342in}{3.025744in}}%
\pgfpathlineto{\pgfqpoint{5.153795in}{3.034566in}}%
\pgfpathlineto{\pgfqpoint{5.161245in}{3.043533in}}%
\pgfpathlineto{\pgfqpoint{5.168693in}{3.052653in}}%
\pgfpathlineto{\pgfqpoint{5.154771in}{3.049007in}}%
\pgfpathlineto{\pgfqpoint{5.140862in}{3.045536in}}%
\pgfpathlineto{\pgfqpoint{5.126967in}{3.042241in}}%
\pgfpathlineto{\pgfqpoint{5.113084in}{3.039122in}}%
\pgfpathlineto{\pgfqpoint{5.105616in}{3.029413in}}%
\pgfpathlineto{\pgfqpoint{5.098147in}{3.019864in}}%
\pgfpathlineto{\pgfqpoint{5.090675in}{3.010468in}}%
\pgfpathlineto{\pgfqpoint{5.083201in}{3.001218in}}%
\pgfpathclose%
\pgfusepath{fill}%
\end{pgfscope}%
\begin{pgfscope}%
\pgfpathrectangle{\pgfqpoint{1.150000in}{0.150000in}}{\pgfqpoint{5.700000in}{5.700000in}}%
\pgfusepath{clip}%
\pgfsetbuttcap%
\pgfsetroundjoin%
\definecolor{currentfill}{rgb}{0.283229,0.120777,0.440584}%
\pgfsetfillcolor{currentfill}%
\pgfsetfillopacity{0.800000}%
\pgfsetlinewidth{0.000000pt}%
\definecolor{currentstroke}{rgb}{0.000000,0.000000,0.000000}%
\pgfsetstrokecolor{currentstroke}%
\pgfsetdash{}{0pt}%
\pgfpathmoveto{\pgfqpoint{3.298310in}{2.344776in}}%
\pgfpathlineto{\pgfqpoint{3.311753in}{2.335650in}}%
\pgfpathlineto{\pgfqpoint{3.325196in}{2.326768in}}%
\pgfpathlineto{\pgfqpoint{3.338640in}{2.318127in}}%
\pgfpathlineto{\pgfqpoint{3.352083in}{2.309725in}}%
\pgfpathlineto{\pgfqpoint{3.360135in}{2.319792in}}%
\pgfpathlineto{\pgfqpoint{3.368181in}{2.329919in}}%
\pgfpathlineto{\pgfqpoint{3.376222in}{2.340107in}}%
\pgfpathlineto{\pgfqpoint{3.384256in}{2.350357in}}%
\pgfpathlineto{\pgfqpoint{3.370825in}{2.358730in}}%
\pgfpathlineto{\pgfqpoint{3.357395in}{2.367343in}}%
\pgfpathlineto{\pgfqpoint{3.343965in}{2.376197in}}%
\pgfpathlineto{\pgfqpoint{3.330535in}{2.385294in}}%
\pgfpathlineto{\pgfqpoint{3.322488in}{2.375061in}}%
\pgfpathlineto{\pgfqpoint{3.314435in}{2.364897in}}%
\pgfpathlineto{\pgfqpoint{3.306375in}{2.354802in}}%
\pgfpathlineto{\pgfqpoint{3.298310in}{2.344776in}}%
\pgfpathclose%
\pgfusepath{fill}%
\end{pgfscope}%
\begin{pgfscope}%
\pgfpathrectangle{\pgfqpoint{1.150000in}{0.150000in}}{\pgfqpoint{5.700000in}{5.700000in}}%
\pgfusepath{clip}%
\pgfsetbuttcap%
\pgfsetroundjoin%
\definecolor{currentfill}{rgb}{0.282290,0.145912,0.461510}%
\pgfsetfillcolor{currentfill}%
\pgfsetfillopacity{0.800000}%
\pgfsetlinewidth{0.000000pt}%
\definecolor{currentstroke}{rgb}{0.000000,0.000000,0.000000}%
\pgfsetstrokecolor{currentstroke}%
\pgfsetdash{}{0pt}%
\pgfpathmoveto{\pgfqpoint{3.887890in}{2.375061in}}%
\pgfpathlineto{\pgfqpoint{3.901380in}{2.372894in}}%
\pgfpathlineto{\pgfqpoint{3.914877in}{2.370932in}}%
\pgfpathlineto{\pgfqpoint{3.928380in}{2.369176in}}%
\pgfpathlineto{\pgfqpoint{3.941889in}{2.367624in}}%
\pgfpathlineto{\pgfqpoint{3.949752in}{2.378088in}}%
\pgfpathlineto{\pgfqpoint{3.957610in}{2.388570in}}%
\pgfpathlineto{\pgfqpoint{3.965463in}{2.399070in}}%
\pgfpathlineto{\pgfqpoint{3.973311in}{2.409591in}}%
\pgfpathlineto{\pgfqpoint{3.959810in}{2.411274in}}%
\pgfpathlineto{\pgfqpoint{3.946315in}{2.413161in}}%
\pgfpathlineto{\pgfqpoint{3.932827in}{2.415252in}}%
\pgfpathlineto{\pgfqpoint{3.919345in}{2.417550in}}%
\pgfpathlineto{\pgfqpoint{3.911489in}{2.406887in}}%
\pgfpathlineto{\pgfqpoint{3.903627in}{2.396252in}}%
\pgfpathlineto{\pgfqpoint{3.895761in}{2.385644in}}%
\pgfpathlineto{\pgfqpoint{3.887890in}{2.375061in}}%
\pgfpathclose%
\pgfusepath{fill}%
\end{pgfscope}%
\begin{pgfscope}%
\pgfpathrectangle{\pgfqpoint{1.150000in}{0.150000in}}{\pgfqpoint{5.700000in}{5.700000in}}%
\pgfusepath{clip}%
\pgfsetbuttcap%
\pgfsetroundjoin%
\definecolor{currentfill}{rgb}{0.280868,0.160771,0.472899}%
\pgfsetfillcolor{currentfill}%
\pgfsetfillopacity{0.800000}%
\pgfsetlinewidth{0.000000pt}%
\definecolor{currentstroke}{rgb}{0.000000,0.000000,0.000000}%
\pgfsetstrokecolor{currentstroke}%
\pgfsetdash{}{0pt}%
\pgfpathmoveto{\pgfqpoint{3.104384in}{2.435130in}}%
\pgfpathlineto{\pgfqpoint{3.117863in}{2.422898in}}%
\pgfpathlineto{\pgfqpoint{3.131339in}{2.410930in}}%
\pgfpathlineto{\pgfqpoint{3.144812in}{2.399224in}}%
\pgfpathlineto{\pgfqpoint{3.158283in}{2.387778in}}%
\pgfpathlineto{\pgfqpoint{3.166405in}{2.397386in}}%
\pgfpathlineto{\pgfqpoint{3.174520in}{2.407078in}}%
\pgfpathlineto{\pgfqpoint{3.182629in}{2.416855in}}%
\pgfpathlineto{\pgfqpoint{3.190730in}{2.426716in}}%
\pgfpathlineto{\pgfqpoint{3.177276in}{2.438101in}}%
\pgfpathlineto{\pgfqpoint{3.163819in}{2.449746in}}%
\pgfpathlineto{\pgfqpoint{3.150359in}{2.461652in}}%
\pgfpathlineto{\pgfqpoint{3.136896in}{2.473823in}}%
\pgfpathlineto{\pgfqpoint{3.128779in}{2.464011in}}%
\pgfpathlineto{\pgfqpoint{3.120655in}{2.454292in}}%
\pgfpathlineto{\pgfqpoint{3.112523in}{2.444665in}}%
\pgfpathlineto{\pgfqpoint{3.104384in}{2.435130in}}%
\pgfpathclose%
\pgfusepath{fill}%
\end{pgfscope}%
\begin{pgfscope}%
\pgfpathrectangle{\pgfqpoint{1.150000in}{0.150000in}}{\pgfqpoint{5.700000in}{5.700000in}}%
\pgfusepath{clip}%
\pgfsetbuttcap%
\pgfsetroundjoin%
\definecolor{currentfill}{rgb}{0.185556,0.418570,0.556753}%
\pgfsetfillcolor{currentfill}%
\pgfsetfillopacity{0.800000}%
\pgfsetlinewidth{0.000000pt}%
\definecolor{currentstroke}{rgb}{0.000000,0.000000,0.000000}%
\pgfsetstrokecolor{currentstroke}%
\pgfsetdash{}{0pt}%
\pgfpathmoveto{\pgfqpoint{5.168693in}{3.052653in}}%
\pgfpathlineto{\pgfqpoint{5.182629in}{3.056474in}}%
\pgfpathlineto{\pgfqpoint{5.196579in}{3.060471in}}%
\pgfpathlineto{\pgfqpoint{5.210543in}{3.064643in}}%
\pgfpathlineto{\pgfqpoint{5.224521in}{3.068990in}}%
\pgfpathlineto{\pgfqpoint{5.231947in}{3.077669in}}%
\pgfpathlineto{\pgfqpoint{5.239371in}{3.086506in}}%
\pgfpathlineto{\pgfqpoint{5.246794in}{3.095508in}}%
\pgfpathlineto{\pgfqpoint{5.254216in}{3.104682in}}%
\pgfpathlineto{\pgfqpoint{5.240260in}{3.100947in}}%
\pgfpathlineto{\pgfqpoint{5.226317in}{3.097386in}}%
\pgfpathlineto{\pgfqpoint{5.212388in}{3.093999in}}%
\pgfpathlineto{\pgfqpoint{5.198473in}{3.090787in}}%
\pgfpathlineto{\pgfqpoint{5.191030in}{3.080991in}}%
\pgfpathlineto{\pgfqpoint{5.183586in}{3.071375in}}%
\pgfpathlineto{\pgfqpoint{5.176140in}{3.061931in}}%
\pgfpathlineto{\pgfqpoint{5.168693in}{3.052653in}}%
\pgfpathclose%
\pgfusepath{fill}%
\end{pgfscope}%
\begin{pgfscope}%
\pgfpathrectangle{\pgfqpoint{1.150000in}{0.150000in}}{\pgfqpoint{5.700000in}{5.700000in}}%
\pgfusepath{clip}%
\pgfsetbuttcap%
\pgfsetroundjoin%
\definecolor{currentfill}{rgb}{0.283091,0.110553,0.431554}%
\pgfsetfillcolor{currentfill}%
\pgfsetfillopacity{0.800000}%
\pgfsetlinewidth{0.000000pt}%
\definecolor{currentstroke}{rgb}{0.000000,0.000000,0.000000}%
\pgfsetstrokecolor{currentstroke}%
\pgfsetdash{}{0pt}%
\pgfpathmoveto{\pgfqpoint{3.577463in}{2.309639in}}%
\pgfpathlineto{\pgfqpoint{3.590906in}{2.304254in}}%
\pgfpathlineto{\pgfqpoint{3.604352in}{2.299090in}}%
\pgfpathlineto{\pgfqpoint{3.617802in}{2.294147in}}%
\pgfpathlineto{\pgfqpoint{3.631255in}{2.289423in}}%
\pgfpathlineto{\pgfqpoint{3.639216in}{2.299891in}}%
\pgfpathlineto{\pgfqpoint{3.647171in}{2.310392in}}%
\pgfpathlineto{\pgfqpoint{3.655121in}{2.320926in}}%
\pgfpathlineto{\pgfqpoint{3.663066in}{2.331496in}}%
\pgfpathlineto{\pgfqpoint{3.649622in}{2.336256in}}%
\pgfpathlineto{\pgfqpoint{3.636183in}{2.341235in}}%
\pgfpathlineto{\pgfqpoint{3.622747in}{2.346434in}}%
\pgfpathlineto{\pgfqpoint{3.609314in}{2.351854in}}%
\pgfpathlineto{\pgfqpoint{3.601359in}{2.341238in}}%
\pgfpathlineto{\pgfqpoint{3.593399in}{2.330663in}}%
\pgfpathlineto{\pgfqpoint{3.585434in}{2.320131in}}%
\pgfpathlineto{\pgfqpoint{3.577463in}{2.309639in}}%
\pgfpathclose%
\pgfusepath{fill}%
\end{pgfscope}%
\begin{pgfscope}%
\pgfpathrectangle{\pgfqpoint{1.150000in}{0.150000in}}{\pgfqpoint{5.700000in}{5.700000in}}%
\pgfusepath{clip}%
\pgfsetbuttcap%
\pgfsetroundjoin%
\definecolor{currentfill}{rgb}{0.220057,0.343307,0.549413}%
\pgfsetfillcolor{currentfill}%
\pgfsetfillopacity{0.800000}%
\pgfsetlinewidth{0.000000pt}%
\definecolor{currentstroke}{rgb}{0.000000,0.000000,0.000000}%
\pgfsetstrokecolor{currentstroke}%
\pgfsetdash{}{0pt}%
\pgfpathmoveto{\pgfqpoint{2.724709in}{2.893805in}}%
\pgfpathlineto{\pgfqpoint{2.738379in}{2.873238in}}%
\pgfpathlineto{\pgfqpoint{2.752039in}{2.853007in}}%
\pgfpathlineto{\pgfqpoint{2.765688in}{2.833107in}}%
\pgfpathlineto{\pgfqpoint{2.779326in}{2.813536in}}%
\pgfpathlineto{\pgfqpoint{2.787582in}{2.822616in}}%
\pgfpathlineto{\pgfqpoint{2.795828in}{2.831836in}}%
\pgfpathlineto{\pgfqpoint{2.804065in}{2.841196in}}%
\pgfpathlineto{\pgfqpoint{2.812294in}{2.850697in}}%
\pgfpathlineto{\pgfqpoint{2.798676in}{2.870201in}}%
\pgfpathlineto{\pgfqpoint{2.785049in}{2.890034in}}%
\pgfpathlineto{\pgfqpoint{2.771412in}{2.910199in}}%
\pgfpathlineto{\pgfqpoint{2.757765in}{2.930698in}}%
\pgfpathlineto{\pgfqpoint{2.749515in}{2.921252in}}%
\pgfpathlineto{\pgfqpoint{2.741256in}{2.911955in}}%
\pgfpathlineto{\pgfqpoint{2.732987in}{2.902806in}}%
\pgfpathlineto{\pgfqpoint{2.724709in}{2.893805in}}%
\pgfpathclose%
\pgfusepath{fill}%
\end{pgfscope}%
\begin{pgfscope}%
\pgfpathrectangle{\pgfqpoint{1.150000in}{0.150000in}}{\pgfqpoint{5.700000in}{5.700000in}}%
\pgfusepath{clip}%
\pgfsetbuttcap%
\pgfsetroundjoin%
\definecolor{currentfill}{rgb}{0.177423,0.437527,0.557565}%
\pgfsetfillcolor{currentfill}%
\pgfsetfillopacity{0.800000}%
\pgfsetlinewidth{0.000000pt}%
\definecolor{currentstroke}{rgb}{0.000000,0.000000,0.000000}%
\pgfsetstrokecolor{currentstroke}%
\pgfsetdash{}{0pt}%
\pgfpathmoveto{\pgfqpoint{5.254216in}{3.104682in}}%
\pgfpathlineto{\pgfqpoint{5.268187in}{3.108592in}}%
\pgfpathlineto{\pgfqpoint{5.282171in}{3.112675in}}%
\pgfpathlineto{\pgfqpoint{5.296170in}{3.116933in}}%
\pgfpathlineto{\pgfqpoint{5.310183in}{3.121364in}}%
\pgfpathlineto{\pgfqpoint{5.317582in}{3.130085in}}%
\pgfpathlineto{\pgfqpoint{5.324980in}{3.138985in}}%
\pgfpathlineto{\pgfqpoint{5.332377in}{3.148071in}}%
\pgfpathlineto{\pgfqpoint{5.339774in}{3.157349in}}%
\pgfpathlineto{\pgfqpoint{5.325784in}{3.153561in}}%
\pgfpathlineto{\pgfqpoint{5.311808in}{3.149947in}}%
\pgfpathlineto{\pgfqpoint{5.297846in}{3.146505in}}%
\pgfpathlineto{\pgfqpoint{5.283898in}{3.143237in}}%
\pgfpathlineto{\pgfqpoint{5.276478in}{3.133305in}}%
\pgfpathlineto{\pgfqpoint{5.269058in}{3.123574in}}%
\pgfpathlineto{\pgfqpoint{5.261638in}{3.114035in}}%
\pgfpathlineto{\pgfqpoint{5.254216in}{3.104682in}}%
\pgfpathclose%
\pgfusepath{fill}%
\end{pgfscope}%
\begin{pgfscope}%
\pgfpathrectangle{\pgfqpoint{1.150000in}{0.150000in}}{\pgfqpoint{5.700000in}{5.700000in}}%
\pgfusepath{clip}%
\pgfsetbuttcap%
\pgfsetroundjoin%
\definecolor{currentfill}{rgb}{0.283072,0.130895,0.449241}%
\pgfsetfillcolor{currentfill}%
\pgfsetfillopacity{0.800000}%
\pgfsetlinewidth{0.000000pt}%
\definecolor{currentstroke}{rgb}{0.000000,0.000000,0.000000}%
\pgfsetstrokecolor{currentstroke}%
\pgfsetdash{}{0pt}%
\pgfpathmoveto{\pgfqpoint{3.802418in}{2.343294in}}%
\pgfpathlineto{\pgfqpoint{3.815893in}{2.340394in}}%
\pgfpathlineto{\pgfqpoint{3.829375in}{2.337703in}}%
\pgfpathlineto{\pgfqpoint{3.842861in}{2.335221in}}%
\pgfpathlineto{\pgfqpoint{3.856354in}{2.332946in}}%
\pgfpathlineto{\pgfqpoint{3.864246in}{2.343445in}}%
\pgfpathlineto{\pgfqpoint{3.872132in}{2.353963in}}%
\pgfpathlineto{\pgfqpoint{3.880014in}{2.364501in}}%
\pgfpathlineto{\pgfqpoint{3.887890in}{2.375061in}}%
\pgfpathlineto{\pgfqpoint{3.874406in}{2.377435in}}%
\pgfpathlineto{\pgfqpoint{3.860927in}{2.380016in}}%
\pgfpathlineto{\pgfqpoint{3.847455in}{2.382806in}}%
\pgfpathlineto{\pgfqpoint{3.833988in}{2.385805in}}%
\pgfpathlineto{\pgfqpoint{3.826103in}{2.375135in}}%
\pgfpathlineto{\pgfqpoint{3.818213in}{2.364494in}}%
\pgfpathlineto{\pgfqpoint{3.810318in}{2.353881in}}%
\pgfpathlineto{\pgfqpoint{3.802418in}{2.343294in}}%
\pgfpathclose%
\pgfusepath{fill}%
\end{pgfscope}%
\begin{pgfscope}%
\pgfpathrectangle{\pgfqpoint{1.150000in}{0.150000in}}{\pgfqpoint{5.700000in}{5.700000in}}%
\pgfusepath{clip}%
\pgfsetbuttcap%
\pgfsetroundjoin%
\definecolor{currentfill}{rgb}{0.169646,0.456262,0.558030}%
\pgfsetfillcolor{currentfill}%
\pgfsetfillopacity{0.800000}%
\pgfsetlinewidth{0.000000pt}%
\definecolor{currentstroke}{rgb}{0.000000,0.000000,0.000000}%
\pgfsetstrokecolor{currentstroke}%
\pgfsetdash{}{0pt}%
\pgfpathmoveto{\pgfqpoint{5.339774in}{3.157349in}}%
\pgfpathlineto{\pgfqpoint{5.353779in}{3.161310in}}%
\pgfpathlineto{\pgfqpoint{5.367797in}{3.165443in}}%
\pgfpathlineto{\pgfqpoint{5.381830in}{3.169750in}}%
\pgfpathlineto{\pgfqpoint{5.395878in}{3.174229in}}%
\pgfpathlineto{\pgfqpoint{5.403251in}{3.183044in}}%
\pgfpathlineto{\pgfqpoint{5.410624in}{3.192059in}}%
\pgfpathlineto{\pgfqpoint{5.417998in}{3.201283in}}%
\pgfpathlineto{\pgfqpoint{5.425372in}{3.210722in}}%
\pgfpathlineto{\pgfqpoint{5.411349in}{3.206919in}}%
\pgfpathlineto{\pgfqpoint{5.397340in}{3.203287in}}%
\pgfpathlineto{\pgfqpoint{5.383346in}{3.199828in}}%
\pgfpathlineto{\pgfqpoint{5.369366in}{3.196541in}}%
\pgfpathlineto{\pgfqpoint{5.361967in}{3.186416in}}%
\pgfpathlineto{\pgfqpoint{5.354569in}{3.176515in}}%
\pgfpathlineto{\pgfqpoint{5.347172in}{3.166828in}}%
\pgfpathlineto{\pgfqpoint{5.339774in}{3.157349in}}%
\pgfpathclose%
\pgfusepath{fill}%
\end{pgfscope}%
\begin{pgfscope}%
\pgfpathrectangle{\pgfqpoint{1.150000in}{0.150000in}}{\pgfqpoint{5.700000in}{5.700000in}}%
\pgfusepath{clip}%
\pgfsetbuttcap%
\pgfsetroundjoin%
\definecolor{currentfill}{rgb}{0.282290,0.145912,0.461510}%
\pgfsetfillcolor{currentfill}%
\pgfsetfillopacity{0.800000}%
\pgfsetlinewidth{0.000000pt}%
\definecolor{currentstroke}{rgb}{0.000000,0.000000,0.000000}%
\pgfsetstrokecolor{currentstroke}%
\pgfsetdash{}{0pt}%
\pgfpathmoveto{\pgfqpoint{3.158283in}{2.387778in}}%
\pgfpathlineto{\pgfqpoint{3.171751in}{2.376590in}}%
\pgfpathlineto{\pgfqpoint{3.185217in}{2.365659in}}%
\pgfpathlineto{\pgfqpoint{3.198682in}{2.354982in}}%
\pgfpathlineto{\pgfqpoint{3.212144in}{2.344558in}}%
\pgfpathlineto{\pgfqpoint{3.220251in}{2.354238in}}%
\pgfpathlineto{\pgfqpoint{3.228350in}{2.363995in}}%
\pgfpathlineto{\pgfqpoint{3.236444in}{2.373828in}}%
\pgfpathlineto{\pgfqpoint{3.244530in}{2.383738in}}%
\pgfpathlineto{\pgfqpoint{3.231083in}{2.394102in}}%
\pgfpathlineto{\pgfqpoint{3.217634in}{2.404718in}}%
\pgfpathlineto{\pgfqpoint{3.204183in}{2.415589in}}%
\pgfpathlineto{\pgfqpoint{3.190730in}{2.426716in}}%
\pgfpathlineto{\pgfqpoint{3.182629in}{2.416855in}}%
\pgfpathlineto{\pgfqpoint{3.174520in}{2.407078in}}%
\pgfpathlineto{\pgfqpoint{3.166405in}{2.397386in}}%
\pgfpathlineto{\pgfqpoint{3.158283in}{2.387778in}}%
\pgfpathclose%
\pgfusepath{fill}%
\end{pgfscope}%
\begin{pgfscope}%
\pgfpathrectangle{\pgfqpoint{1.150000in}{0.150000in}}{\pgfqpoint{5.700000in}{5.700000in}}%
\pgfusepath{clip}%
\pgfsetbuttcap%
\pgfsetroundjoin%
\definecolor{currentfill}{rgb}{0.162142,0.474838,0.558140}%
\pgfsetfillcolor{currentfill}%
\pgfsetfillopacity{0.800000}%
\pgfsetlinewidth{0.000000pt}%
\definecolor{currentstroke}{rgb}{0.000000,0.000000,0.000000}%
\pgfsetstrokecolor{currentstroke}%
\pgfsetdash{}{0pt}%
\pgfpathmoveto{\pgfqpoint{5.425372in}{3.210722in}}%
\pgfpathlineto{\pgfqpoint{5.439409in}{3.214697in}}%
\pgfpathlineto{\pgfqpoint{5.453462in}{3.218844in}}%
\pgfpathlineto{\pgfqpoint{5.467528in}{3.223162in}}%
\pgfpathlineto{\pgfqpoint{5.481610in}{3.227652in}}%
\pgfpathlineto{\pgfqpoint{5.488959in}{3.236619in}}%
\pgfpathlineto{\pgfqpoint{5.496310in}{3.245809in}}%
\pgfpathlineto{\pgfqpoint{5.503661in}{3.255231in}}%
\pgfpathlineto{\pgfqpoint{5.511015in}{3.264892in}}%
\pgfpathlineto{\pgfqpoint{5.496959in}{3.261110in}}%
\pgfpathlineto{\pgfqpoint{5.482919in}{3.257498in}}%
\pgfpathlineto{\pgfqpoint{5.468893in}{3.254058in}}%
\pgfpathlineto{\pgfqpoint{5.454881in}{3.250789in}}%
\pgfpathlineto{\pgfqpoint{5.447501in}{3.240410in}}%
\pgfpathlineto{\pgfqpoint{5.440123in}{3.230277in}}%
\pgfpathlineto{\pgfqpoint{5.432747in}{3.220384in}}%
\pgfpathlineto{\pgfqpoint{5.425372in}{3.210722in}}%
\pgfpathclose%
\pgfusepath{fill}%
\end{pgfscope}%
\begin{pgfscope}%
\pgfpathrectangle{\pgfqpoint{1.150000in}{0.150000in}}{\pgfqpoint{5.700000in}{5.700000in}}%
\pgfusepath{clip}%
\pgfsetbuttcap%
\pgfsetroundjoin%
\definecolor{currentfill}{rgb}{0.204903,0.375746,0.553533}%
\pgfsetfillcolor{currentfill}%
\pgfsetfillopacity{0.800000}%
\pgfsetlinewidth{0.000000pt}%
\definecolor{currentstroke}{rgb}{0.000000,0.000000,0.000000}%
\pgfsetstrokecolor{currentstroke}%
\pgfsetdash{}{0pt}%
\pgfpathmoveto{\pgfqpoint{2.669914in}{2.979489in}}%
\pgfpathlineto{\pgfqpoint{2.683630in}{2.957549in}}%
\pgfpathlineto{\pgfqpoint{2.697335in}{2.935957in}}%
\pgfpathlineto{\pgfqpoint{2.711028in}{2.914710in}}%
\pgfpathlineto{\pgfqpoint{2.724709in}{2.893805in}}%
\pgfpathlineto{\pgfqpoint{2.732987in}{2.902806in}}%
\pgfpathlineto{\pgfqpoint{2.741256in}{2.911955in}}%
\pgfpathlineto{\pgfqpoint{2.749515in}{2.921252in}}%
\pgfpathlineto{\pgfqpoint{2.757765in}{2.930698in}}%
\pgfpathlineto{\pgfqpoint{2.744106in}{2.951536in}}%
\pgfpathlineto{\pgfqpoint{2.730436in}{2.972715in}}%
\pgfpathlineto{\pgfqpoint{2.716755in}{2.994238in}}%
\pgfpathlineto{\pgfqpoint{2.703061in}{3.016110in}}%
\pgfpathlineto{\pgfqpoint{2.694789in}{3.006720in}}%
\pgfpathlineto{\pgfqpoint{2.686507in}{2.997487in}}%
\pgfpathlineto{\pgfqpoint{2.678215in}{2.988410in}}%
\pgfpathlineto{\pgfqpoint{2.669914in}{2.979489in}}%
\pgfpathclose%
\pgfusepath{fill}%
\end{pgfscope}%
\begin{pgfscope}%
\pgfpathrectangle{\pgfqpoint{1.150000in}{0.150000in}}{\pgfqpoint{5.700000in}{5.700000in}}%
\pgfusepath{clip}%
\pgfsetbuttcap%
\pgfsetroundjoin%
\definecolor{currentfill}{rgb}{0.283091,0.110553,0.431554}%
\pgfsetfillcolor{currentfill}%
\pgfsetfillopacity{0.800000}%
\pgfsetlinewidth{0.000000pt}%
\definecolor{currentstroke}{rgb}{0.000000,0.000000,0.000000}%
\pgfsetstrokecolor{currentstroke}%
\pgfsetdash{}{0pt}%
\pgfpathmoveto{\pgfqpoint{3.352083in}{2.309725in}}%
\pgfpathlineto{\pgfqpoint{3.365527in}{2.301562in}}%
\pgfpathlineto{\pgfqpoint{3.378972in}{2.293636in}}%
\pgfpathlineto{\pgfqpoint{3.392417in}{2.285946in}}%
\pgfpathlineto{\pgfqpoint{3.405864in}{2.278489in}}%
\pgfpathlineto{\pgfqpoint{3.413903in}{2.288595in}}%
\pgfpathlineto{\pgfqpoint{3.421936in}{2.298754in}}%
\pgfpathlineto{\pgfqpoint{3.429964in}{2.308966in}}%
\pgfpathlineto{\pgfqpoint{3.437985in}{2.319232in}}%
\pgfpathlineto{\pgfqpoint{3.424551in}{2.326660in}}%
\pgfpathlineto{\pgfqpoint{3.411119in}{2.334323in}}%
\pgfpathlineto{\pgfqpoint{3.397687in}{2.342221in}}%
\pgfpathlineto{\pgfqpoint{3.384256in}{2.350357in}}%
\pgfpathlineto{\pgfqpoint{3.376222in}{2.340107in}}%
\pgfpathlineto{\pgfqpoint{3.368181in}{2.329919in}}%
\pgfpathlineto{\pgfqpoint{3.360135in}{2.319792in}}%
\pgfpathlineto{\pgfqpoint{3.352083in}{2.309725in}}%
\pgfpathclose%
\pgfusepath{fill}%
\end{pgfscope}%
\begin{pgfscope}%
\pgfpathrectangle{\pgfqpoint{1.150000in}{0.150000in}}{\pgfqpoint{5.700000in}{5.700000in}}%
\pgfusepath{clip}%
\pgfsetbuttcap%
\pgfsetroundjoin%
\definecolor{currentfill}{rgb}{0.283229,0.120777,0.440584}%
\pgfsetfillcolor{currentfill}%
\pgfsetfillopacity{0.800000}%
\pgfsetlinewidth{0.000000pt}%
\definecolor{currentstroke}{rgb}{0.000000,0.000000,0.000000}%
\pgfsetstrokecolor{currentstroke}%
\pgfsetdash{}{0pt}%
\pgfpathmoveto{\pgfqpoint{3.716879in}{2.314628in}}%
\pgfpathlineto{\pgfqpoint{3.730343in}{2.310948in}}%
\pgfpathlineto{\pgfqpoint{3.743812in}{2.307482in}}%
\pgfpathlineto{\pgfqpoint{3.757287in}{2.304228in}}%
\pgfpathlineto{\pgfqpoint{3.770766in}{2.301185in}}%
\pgfpathlineto{\pgfqpoint{3.778687in}{2.311679in}}%
\pgfpathlineto{\pgfqpoint{3.786602in}{2.322195in}}%
\pgfpathlineto{\pgfqpoint{3.794513in}{2.332733in}}%
\pgfpathlineto{\pgfqpoint{3.802418in}{2.343294in}}%
\pgfpathlineto{\pgfqpoint{3.788948in}{2.346405in}}%
\pgfpathlineto{\pgfqpoint{3.775482in}{2.349726in}}%
\pgfpathlineto{\pgfqpoint{3.762022in}{2.353260in}}%
\pgfpathlineto{\pgfqpoint{3.748567in}{2.357007in}}%
\pgfpathlineto{\pgfqpoint{3.740653in}{2.346366in}}%
\pgfpathlineto{\pgfqpoint{3.732733in}{2.335757in}}%
\pgfpathlineto{\pgfqpoint{3.724809in}{2.325178in}}%
\pgfpathlineto{\pgfqpoint{3.716879in}{2.314628in}}%
\pgfpathclose%
\pgfusepath{fill}%
\end{pgfscope}%
\begin{pgfscope}%
\pgfpathrectangle{\pgfqpoint{1.150000in}{0.150000in}}{\pgfqpoint{5.700000in}{5.700000in}}%
\pgfusepath{clip}%
\pgfsetbuttcap%
\pgfsetroundjoin%
\definecolor{currentfill}{rgb}{0.282910,0.105393,0.426902}%
\pgfsetfillcolor{currentfill}%
\pgfsetfillopacity{0.800000}%
\pgfsetlinewidth{0.000000pt}%
\definecolor{currentstroke}{rgb}{0.000000,0.000000,0.000000}%
\pgfsetstrokecolor{currentstroke}%
\pgfsetdash{}{0pt}%
\pgfpathmoveto{\pgfqpoint{3.491738in}{2.291830in}}%
\pgfpathlineto{\pgfqpoint{3.505181in}{2.285551in}}%
\pgfpathlineto{\pgfqpoint{3.518626in}{2.279499in}}%
\pgfpathlineto{\pgfqpoint{3.532074in}{2.273672in}}%
\pgfpathlineto{\pgfqpoint{3.545525in}{2.268068in}}%
\pgfpathlineto{\pgfqpoint{3.553517in}{2.278403in}}%
\pgfpathlineto{\pgfqpoint{3.561505in}{2.288776in}}%
\pgfpathlineto{\pgfqpoint{3.569487in}{2.299188in}}%
\pgfpathlineto{\pgfqpoint{3.577463in}{2.309639in}}%
\pgfpathlineto{\pgfqpoint{3.564023in}{2.315247in}}%
\pgfpathlineto{\pgfqpoint{3.550586in}{2.321078in}}%
\pgfpathlineto{\pgfqpoint{3.537152in}{2.327135in}}%
\pgfpathlineto{\pgfqpoint{3.523720in}{2.333417in}}%
\pgfpathlineto{\pgfqpoint{3.515733in}{2.322950in}}%
\pgfpathlineto{\pgfqpoint{3.507740in}{2.312530in}}%
\pgfpathlineto{\pgfqpoint{3.499742in}{2.302157in}}%
\pgfpathlineto{\pgfqpoint{3.491738in}{2.291830in}}%
\pgfpathclose%
\pgfusepath{fill}%
\end{pgfscope}%
\begin{pgfscope}%
\pgfpathrectangle{\pgfqpoint{1.150000in}{0.150000in}}{\pgfqpoint{5.700000in}{5.700000in}}%
\pgfusepath{clip}%
\pgfsetbuttcap%
\pgfsetroundjoin%
\definecolor{currentfill}{rgb}{0.154815,0.493313,0.557840}%
\pgfsetfillcolor{currentfill}%
\pgfsetfillopacity{0.800000}%
\pgfsetlinewidth{0.000000pt}%
\definecolor{currentstroke}{rgb}{0.000000,0.000000,0.000000}%
\pgfsetstrokecolor{currentstroke}%
\pgfsetdash{}{0pt}%
\pgfpathmoveto{\pgfqpoint{5.511015in}{3.264892in}}%
\pgfpathlineto{\pgfqpoint{5.525085in}{3.268845in}}%
\pgfpathlineto{\pgfqpoint{5.539169in}{3.272968in}}%
\pgfpathlineto{\pgfqpoint{5.553269in}{3.277263in}}%
\pgfpathlineto{\pgfqpoint{5.567384in}{3.281727in}}%
\pgfpathlineto{\pgfqpoint{5.574712in}{3.290908in}}%
\pgfpathlineto{\pgfqpoint{5.582042in}{3.300338in}}%
\pgfpathlineto{\pgfqpoint{5.589374in}{3.310024in}}%
\pgfpathlineto{\pgfqpoint{5.596710in}{3.319975in}}%
\pgfpathlineto{\pgfqpoint{5.582623in}{3.316250in}}%
\pgfpathlineto{\pgfqpoint{5.568552in}{3.312695in}}%
\pgfpathlineto{\pgfqpoint{5.554494in}{3.309310in}}%
\pgfpathlineto{\pgfqpoint{5.540452in}{3.306095in}}%
\pgfpathlineto{\pgfqpoint{5.533088in}{3.295394in}}%
\pgfpathlineto{\pgfqpoint{5.525728in}{3.284965in}}%
\pgfpathlineto{\pgfqpoint{5.518370in}{3.274801in}}%
\pgfpathlineto{\pgfqpoint{5.511015in}{3.264892in}}%
\pgfpathclose%
\pgfusepath{fill}%
\end{pgfscope}%
\begin{pgfscope}%
\pgfpathrectangle{\pgfqpoint{1.150000in}{0.150000in}}{\pgfqpoint{5.700000in}{5.700000in}}%
\pgfusepath{clip}%
\pgfsetbuttcap%
\pgfsetroundjoin%
\definecolor{currentfill}{rgb}{0.263663,0.237631,0.518762}%
\pgfsetfillcolor{currentfill}%
\pgfsetfillopacity{0.800000}%
\pgfsetlinewidth{0.000000pt}%
\definecolor{currentstroke}{rgb}{0.000000,0.000000,0.000000}%
\pgfsetstrokecolor{currentstroke}%
\pgfsetdash{}{0pt}%
\pgfpathmoveto{\pgfqpoint{4.369206in}{2.574007in}}%
\pgfpathlineto{\pgfqpoint{4.382850in}{2.575613in}}%
\pgfpathlineto{\pgfqpoint{4.396503in}{2.577409in}}%
\pgfpathlineto{\pgfqpoint{4.410166in}{2.579395in}}%
\pgfpathlineto{\pgfqpoint{4.423840in}{2.581570in}}%
\pgfpathlineto{\pgfqpoint{4.431553in}{2.591156in}}%
\pgfpathlineto{\pgfqpoint{4.439260in}{2.600765in}}%
\pgfpathlineto{\pgfqpoint{4.446962in}{2.610399in}}%
\pgfpathlineto{\pgfqpoint{4.454660in}{2.620064in}}%
\pgfpathlineto{\pgfqpoint{4.440996in}{2.618179in}}%
\pgfpathlineto{\pgfqpoint{4.427342in}{2.616484in}}%
\pgfpathlineto{\pgfqpoint{4.413698in}{2.614978in}}%
\pgfpathlineto{\pgfqpoint{4.400064in}{2.613662in}}%
\pgfpathlineto{\pgfqpoint{4.392357in}{2.603696in}}%
\pgfpathlineto{\pgfqpoint{4.384645in}{2.593767in}}%
\pgfpathlineto{\pgfqpoint{4.376928in}{2.583872in}}%
\pgfpathlineto{\pgfqpoint{4.369206in}{2.574007in}}%
\pgfpathclose%
\pgfusepath{fill}%
\end{pgfscope}%
\begin{pgfscope}%
\pgfpathrectangle{\pgfqpoint{1.150000in}{0.150000in}}{\pgfqpoint{5.700000in}{5.700000in}}%
\pgfusepath{clip}%
\pgfsetbuttcap%
\pgfsetroundjoin%
\definecolor{currentfill}{rgb}{0.269308,0.218818,0.509577}%
\pgfsetfillcolor{currentfill}%
\pgfsetfillopacity{0.800000}%
\pgfsetlinewidth{0.000000pt}%
\definecolor{currentstroke}{rgb}{0.000000,0.000000,0.000000}%
\pgfsetstrokecolor{currentstroke}%
\pgfsetdash{}{0pt}%
\pgfpathmoveto{\pgfqpoint{4.283758in}{2.529236in}}%
\pgfpathlineto{\pgfqpoint{4.297372in}{2.530335in}}%
\pgfpathlineto{\pgfqpoint{4.310995in}{2.531626in}}%
\pgfpathlineto{\pgfqpoint{4.324628in}{2.533109in}}%
\pgfpathlineto{\pgfqpoint{4.338271in}{2.534783in}}%
\pgfpathlineto{\pgfqpoint{4.346012in}{2.544560in}}%
\pgfpathlineto{\pgfqpoint{4.353748in}{2.554354in}}%
\pgfpathlineto{\pgfqpoint{4.361480in}{2.564168in}}%
\pgfpathlineto{\pgfqpoint{4.369206in}{2.574007in}}%
\pgfpathlineto{\pgfqpoint{4.355573in}{2.572591in}}%
\pgfpathlineto{\pgfqpoint{4.341949in}{2.571367in}}%
\pgfpathlineto{\pgfqpoint{4.328335in}{2.570334in}}%
\pgfpathlineto{\pgfqpoint{4.314730in}{2.569494in}}%
\pgfpathlineto{\pgfqpoint{4.306995in}{2.559386in}}%
\pgfpathlineto{\pgfqpoint{4.299254in}{2.549309in}}%
\pgfpathlineto{\pgfqpoint{4.291509in}{2.539260in}}%
\pgfpathlineto{\pgfqpoint{4.283758in}{2.529236in}}%
\pgfpathclose%
\pgfusepath{fill}%
\end{pgfscope}%
\begin{pgfscope}%
\pgfpathrectangle{\pgfqpoint{1.150000in}{0.150000in}}{\pgfqpoint{5.700000in}{5.700000in}}%
\pgfusepath{clip}%
\pgfsetbuttcap%
\pgfsetroundjoin%
\definecolor{currentfill}{rgb}{0.283072,0.130895,0.449241}%
\pgfsetfillcolor{currentfill}%
\pgfsetfillopacity{0.800000}%
\pgfsetlinewidth{0.000000pt}%
\definecolor{currentstroke}{rgb}{0.000000,0.000000,0.000000}%
\pgfsetstrokecolor{currentstroke}%
\pgfsetdash{}{0pt}%
\pgfpathmoveto{\pgfqpoint{3.212144in}{2.344558in}}%
\pgfpathlineto{\pgfqpoint{3.225605in}{2.334385in}}%
\pgfpathlineto{\pgfqpoint{3.239065in}{2.324461in}}%
\pgfpathlineto{\pgfqpoint{3.252524in}{2.314785in}}%
\pgfpathlineto{\pgfqpoint{3.265983in}{2.305356in}}%
\pgfpathlineto{\pgfqpoint{3.274074in}{2.315108in}}%
\pgfpathlineto{\pgfqpoint{3.282159in}{2.324928in}}%
\pgfpathlineto{\pgfqpoint{3.290238in}{2.334818in}}%
\pgfpathlineto{\pgfqpoint{3.298310in}{2.344776in}}%
\pgfpathlineto{\pgfqpoint{3.284866in}{2.354146in}}%
\pgfpathlineto{\pgfqpoint{3.271422in}{2.363762in}}%
\pgfpathlineto{\pgfqpoint{3.257976in}{2.373625in}}%
\pgfpathlineto{\pgfqpoint{3.244530in}{2.383738in}}%
\pgfpathlineto{\pgfqpoint{3.236444in}{2.373828in}}%
\pgfpathlineto{\pgfqpoint{3.228350in}{2.363995in}}%
\pgfpathlineto{\pgfqpoint{3.220251in}{2.354238in}}%
\pgfpathlineto{\pgfqpoint{3.212144in}{2.344558in}}%
\pgfpathclose%
\pgfusepath{fill}%
\end{pgfscope}%
\begin{pgfscope}%
\pgfpathrectangle{\pgfqpoint{1.150000in}{0.150000in}}{\pgfqpoint{5.700000in}{5.700000in}}%
\pgfusepath{clip}%
\pgfsetbuttcap%
\pgfsetroundjoin%
\definecolor{currentfill}{rgb}{0.255645,0.260703,0.528312}%
\pgfsetfillcolor{currentfill}%
\pgfsetfillopacity{0.800000}%
\pgfsetlinewidth{0.000000pt}%
\definecolor{currentstroke}{rgb}{0.000000,0.000000,0.000000}%
\pgfsetstrokecolor{currentstroke}%
\pgfsetdash{}{0pt}%
\pgfpathmoveto{\pgfqpoint{4.454660in}{2.620064in}}%
\pgfpathlineto{\pgfqpoint{4.468334in}{2.622137in}}%
\pgfpathlineto{\pgfqpoint{4.482020in}{2.624398in}}%
\pgfpathlineto{\pgfqpoint{4.495716in}{2.626847in}}%
\pgfpathlineto{\pgfqpoint{4.509422in}{2.629483in}}%
\pgfpathlineto{\pgfqpoint{4.517105in}{2.638869in}}%
\pgfpathlineto{\pgfqpoint{4.524783in}{2.648285in}}%
\pgfpathlineto{\pgfqpoint{4.532456in}{2.657733in}}%
\pgfpathlineto{\pgfqpoint{4.540124in}{2.667219in}}%
\pgfpathlineto{\pgfqpoint{4.526428in}{2.664905in}}%
\pgfpathlineto{\pgfqpoint{4.512742in}{2.662779in}}%
\pgfpathlineto{\pgfqpoint{4.499067in}{2.660841in}}%
\pgfpathlineto{\pgfqpoint{4.485403in}{2.659090in}}%
\pgfpathlineto{\pgfqpoint{4.477724in}{2.649270in}}%
\pgfpathlineto{\pgfqpoint{4.470041in}{2.639495in}}%
\pgfpathlineto{\pgfqpoint{4.462353in}{2.629761in}}%
\pgfpathlineto{\pgfqpoint{4.454660in}{2.620064in}}%
\pgfpathclose%
\pgfusepath{fill}%
\end{pgfscope}%
\begin{pgfscope}%
\pgfpathrectangle{\pgfqpoint{1.150000in}{0.150000in}}{\pgfqpoint{5.700000in}{5.700000in}}%
\pgfusepath{clip}%
\pgfsetbuttcap%
\pgfsetroundjoin%
\definecolor{currentfill}{rgb}{0.274128,0.199721,0.498911}%
\pgfsetfillcolor{currentfill}%
\pgfsetfillopacity{0.800000}%
\pgfsetlinewidth{0.000000pt}%
\definecolor{currentstroke}{rgb}{0.000000,0.000000,0.000000}%
\pgfsetstrokecolor{currentstroke}%
\pgfsetdash{}{0pt}%
\pgfpathmoveto{\pgfqpoint{4.198311in}{2.485966in}}%
\pgfpathlineto{\pgfqpoint{4.211897in}{2.486516in}}%
\pgfpathlineto{\pgfqpoint{4.225491in}{2.487260in}}%
\pgfpathlineto{\pgfqpoint{4.239095in}{2.488199in}}%
\pgfpathlineto{\pgfqpoint{4.252708in}{2.489332in}}%
\pgfpathlineto{\pgfqpoint{4.260478in}{2.499285in}}%
\pgfpathlineto{\pgfqpoint{4.268243in}{2.509252in}}%
\pgfpathlineto{\pgfqpoint{4.276003in}{2.519235in}}%
\pgfpathlineto{\pgfqpoint{4.283758in}{2.529236in}}%
\pgfpathlineto{\pgfqpoint{4.270154in}{2.528331in}}%
\pgfpathlineto{\pgfqpoint{4.256559in}{2.527618in}}%
\pgfpathlineto{\pgfqpoint{4.242973in}{2.527100in}}%
\pgfpathlineto{\pgfqpoint{4.229396in}{2.526776in}}%
\pgfpathlineto{\pgfqpoint{4.221632in}{2.516536in}}%
\pgfpathlineto{\pgfqpoint{4.213863in}{2.506323in}}%
\pgfpathlineto{\pgfqpoint{4.206089in}{2.496134in}}%
\pgfpathlineto{\pgfqpoint{4.198311in}{2.485966in}}%
\pgfpathclose%
\pgfusepath{fill}%
\end{pgfscope}%
\begin{pgfscope}%
\pgfpathrectangle{\pgfqpoint{1.150000in}{0.150000in}}{\pgfqpoint{5.700000in}{5.700000in}}%
\pgfusepath{clip}%
\pgfsetbuttcap%
\pgfsetroundjoin%
\definecolor{currentfill}{rgb}{0.248629,0.278775,0.534556}%
\pgfsetfillcolor{currentfill}%
\pgfsetfillopacity{0.800000}%
\pgfsetlinewidth{0.000000pt}%
\definecolor{currentstroke}{rgb}{0.000000,0.000000,0.000000}%
\pgfsetstrokecolor{currentstroke}%
\pgfsetdash{}{0pt}%
\pgfpathmoveto{\pgfqpoint{4.540124in}{2.667219in}}%
\pgfpathlineto{\pgfqpoint{4.553831in}{2.669719in}}%
\pgfpathlineto{\pgfqpoint{4.567550in}{2.672405in}}%
\pgfpathlineto{\pgfqpoint{4.581279in}{2.675277in}}%
\pgfpathlineto{\pgfqpoint{4.595020in}{2.678334in}}%
\pgfpathlineto{\pgfqpoint{4.602673in}{2.687517in}}%
\pgfpathlineto{\pgfqpoint{4.610321in}{2.696738in}}%
\pgfpathlineto{\pgfqpoint{4.617964in}{2.705999in}}%
\pgfpathlineto{\pgfqpoint{4.625602in}{2.715306in}}%
\pgfpathlineto{\pgfqpoint{4.611872in}{2.712604in}}%
\pgfpathlineto{\pgfqpoint{4.598154in}{2.710087in}}%
\pgfpathlineto{\pgfqpoint{4.584447in}{2.707755in}}%
\pgfpathlineto{\pgfqpoint{4.570751in}{2.705609in}}%
\pgfpathlineto{\pgfqpoint{4.563101in}{2.695936in}}%
\pgfpathlineto{\pgfqpoint{4.555446in}{2.686316in}}%
\pgfpathlineto{\pgfqpoint{4.547787in}{2.676745in}}%
\pgfpathlineto{\pgfqpoint{4.540124in}{2.667219in}}%
\pgfpathclose%
\pgfusepath{fill}%
\end{pgfscope}%
\begin{pgfscope}%
\pgfpathrectangle{\pgfqpoint{1.150000in}{0.150000in}}{\pgfqpoint{5.700000in}{5.700000in}}%
\pgfusepath{clip}%
\pgfsetbuttcap%
\pgfsetroundjoin%
\definecolor{currentfill}{rgb}{0.239346,0.300855,0.540844}%
\pgfsetfillcolor{currentfill}%
\pgfsetfillopacity{0.800000}%
\pgfsetlinewidth{0.000000pt}%
\definecolor{currentstroke}{rgb}{0.000000,0.000000,0.000000}%
\pgfsetstrokecolor{currentstroke}%
\pgfsetdash{}{0pt}%
\pgfpathmoveto{\pgfqpoint{4.625602in}{2.715306in}}%
\pgfpathlineto{\pgfqpoint{4.639343in}{2.718193in}}%
\pgfpathlineto{\pgfqpoint{4.653096in}{2.721265in}}%
\pgfpathlineto{\pgfqpoint{4.666860in}{2.724521in}}%
\pgfpathlineto{\pgfqpoint{4.680637in}{2.727961in}}%
\pgfpathlineto{\pgfqpoint{4.688259in}{2.736942in}}%
\pgfpathlineto{\pgfqpoint{4.695876in}{2.745970in}}%
\pgfpathlineto{\pgfqpoint{4.703489in}{2.755049in}}%
\pgfpathlineto{\pgfqpoint{4.711097in}{2.764183in}}%
\pgfpathlineto{\pgfqpoint{4.697333in}{2.761131in}}%
\pgfpathlineto{\pgfqpoint{4.683581in}{2.758262in}}%
\pgfpathlineto{\pgfqpoint{4.669840in}{2.755577in}}%
\pgfpathlineto{\pgfqpoint{4.656111in}{2.753076in}}%
\pgfpathlineto{\pgfqpoint{4.648491in}{2.743544in}}%
\pgfpathlineto{\pgfqpoint{4.640865in}{2.734074in}}%
\pgfpathlineto{\pgfqpoint{4.633236in}{2.724663in}}%
\pgfpathlineto{\pgfqpoint{4.625602in}{2.715306in}}%
\pgfpathclose%
\pgfusepath{fill}%
\end{pgfscope}%
\begin{pgfscope}%
\pgfpathrectangle{\pgfqpoint{1.150000in}{0.150000in}}{\pgfqpoint{5.700000in}{5.700000in}}%
\pgfusepath{clip}%
\pgfsetbuttcap%
\pgfsetroundjoin%
\definecolor{currentfill}{rgb}{0.277134,0.185228,0.489898}%
\pgfsetfillcolor{currentfill}%
\pgfsetfillopacity{0.800000}%
\pgfsetlinewidth{0.000000pt}%
\definecolor{currentstroke}{rgb}{0.000000,0.000000,0.000000}%
\pgfsetstrokecolor{currentstroke}%
\pgfsetdash{}{0pt}%
\pgfpathmoveto{\pgfqpoint{4.112856in}{2.444430in}}%
\pgfpathlineto{\pgfqpoint{4.126416in}{2.444390in}}%
\pgfpathlineto{\pgfqpoint{4.139985in}{2.444547in}}%
\pgfpathlineto{\pgfqpoint{4.153562in}{2.444900in}}%
\pgfpathlineto{\pgfqpoint{4.167147in}{2.445450in}}%
\pgfpathlineto{\pgfqpoint{4.174945in}{2.455561in}}%
\pgfpathlineto{\pgfqpoint{4.182739in}{2.465681in}}%
\pgfpathlineto{\pgfqpoint{4.190527in}{2.475816in}}%
\pgfpathlineto{\pgfqpoint{4.198311in}{2.485966in}}%
\pgfpathlineto{\pgfqpoint{4.184734in}{2.485611in}}%
\pgfpathlineto{\pgfqpoint{4.171165in}{2.485452in}}%
\pgfpathlineto{\pgfqpoint{4.157605in}{2.485489in}}%
\pgfpathlineto{\pgfqpoint{4.144053in}{2.485723in}}%
\pgfpathlineto{\pgfqpoint{4.136261in}{2.475367in}}%
\pgfpathlineto{\pgfqpoint{4.128464in}{2.465035in}}%
\pgfpathlineto{\pgfqpoint{4.120663in}{2.454723in}}%
\pgfpathlineto{\pgfqpoint{4.112856in}{2.444430in}}%
\pgfpathclose%
\pgfusepath{fill}%
\end{pgfscope}%
\begin{pgfscope}%
\pgfpathrectangle{\pgfqpoint{1.150000in}{0.150000in}}{\pgfqpoint{5.700000in}{5.700000in}}%
\pgfusepath{clip}%
\pgfsetbuttcap%
\pgfsetroundjoin%
\definecolor{currentfill}{rgb}{0.267968,0.223549,0.512008}%
\pgfsetfillcolor{currentfill}%
\pgfsetfillopacity{0.800000}%
\pgfsetlinewidth{0.000000pt}%
\definecolor{currentstroke}{rgb}{0.000000,0.000000,0.000000}%
\pgfsetstrokecolor{currentstroke}%
\pgfsetdash{}{0pt}%
\pgfpathmoveto{\pgfqpoint{2.909458in}{2.567166in}}%
\pgfpathlineto{\pgfqpoint{2.923012in}{2.551509in}}%
\pgfpathlineto{\pgfqpoint{2.936560in}{2.536142in}}%
\pgfpathlineto{\pgfqpoint{2.950103in}{2.521063in}}%
\pgfpathlineto{\pgfqpoint{2.963639in}{2.506269in}}%
\pgfpathlineto{\pgfqpoint{2.971845in}{2.515219in}}%
\pgfpathlineto{\pgfqpoint{2.980043in}{2.524278in}}%
\pgfpathlineto{\pgfqpoint{2.988233in}{2.533447in}}%
\pgfpathlineto{\pgfqpoint{2.996415in}{2.542725in}}%
\pgfpathlineto{\pgfqpoint{2.982898in}{2.557424in}}%
\pgfpathlineto{\pgfqpoint{2.969375in}{2.572409in}}%
\pgfpathlineto{\pgfqpoint{2.955847in}{2.587680in}}%
\pgfpathlineto{\pgfqpoint{2.942313in}{2.603242in}}%
\pgfpathlineto{\pgfqpoint{2.934111in}{2.594047in}}%
\pgfpathlineto{\pgfqpoint{2.925902in}{2.584969in}}%
\pgfpathlineto{\pgfqpoint{2.917684in}{2.576009in}}%
\pgfpathlineto{\pgfqpoint{2.909458in}{2.567166in}}%
\pgfpathclose%
\pgfusepath{fill}%
\end{pgfscope}%
\begin{pgfscope}%
\pgfpathrectangle{\pgfqpoint{1.150000in}{0.150000in}}{\pgfqpoint{5.700000in}{5.700000in}}%
\pgfusepath{clip}%
\pgfsetbuttcap%
\pgfsetroundjoin%
\definecolor{currentfill}{rgb}{0.258965,0.251537,0.524736}%
\pgfsetfillcolor{currentfill}%
\pgfsetfillopacity{0.800000}%
\pgfsetlinewidth{0.000000pt}%
\definecolor{currentstroke}{rgb}{0.000000,0.000000,0.000000}%
\pgfsetstrokecolor{currentstroke}%
\pgfsetdash{}{0pt}%
\pgfpathmoveto{\pgfqpoint{2.855171in}{2.632747in}}%
\pgfpathlineto{\pgfqpoint{2.868754in}{2.615904in}}%
\pgfpathlineto{\pgfqpoint{2.882329in}{2.599361in}}%
\pgfpathlineto{\pgfqpoint{2.895897in}{2.583116in}}%
\pgfpathlineto{\pgfqpoint{2.909458in}{2.567166in}}%
\pgfpathlineto{\pgfqpoint{2.917684in}{2.576009in}}%
\pgfpathlineto{\pgfqpoint{2.925902in}{2.584969in}}%
\pgfpathlineto{\pgfqpoint{2.934111in}{2.594047in}}%
\pgfpathlineto{\pgfqpoint{2.942313in}{2.603242in}}%
\pgfpathlineto{\pgfqpoint{2.928772in}{2.619096in}}%
\pgfpathlineto{\pgfqpoint{2.915225in}{2.635245in}}%
\pgfpathlineto{\pgfqpoint{2.901670in}{2.651691in}}%
\pgfpathlineto{\pgfqpoint{2.888109in}{2.668438in}}%
\pgfpathlineto{\pgfqpoint{2.879888in}{2.659327in}}%
\pgfpathlineto{\pgfqpoint{2.871658in}{2.650342in}}%
\pgfpathlineto{\pgfqpoint{2.863419in}{2.641481in}}%
\pgfpathlineto{\pgfqpoint{2.855171in}{2.632747in}}%
\pgfpathclose%
\pgfusepath{fill}%
\end{pgfscope}%
\begin{pgfscope}%
\pgfpathrectangle{\pgfqpoint{1.150000in}{0.150000in}}{\pgfqpoint{5.700000in}{5.700000in}}%
\pgfusepath{clip}%
\pgfsetbuttcap%
\pgfsetroundjoin%
\definecolor{currentfill}{rgb}{0.283091,0.110553,0.431554}%
\pgfsetfillcolor{currentfill}%
\pgfsetfillopacity{0.800000}%
\pgfsetlinewidth{0.000000pt}%
\definecolor{currentstroke}{rgb}{0.000000,0.000000,0.000000}%
\pgfsetstrokecolor{currentstroke}%
\pgfsetdash{}{0pt}%
\pgfpathmoveto{\pgfqpoint{3.631255in}{2.289423in}}%
\pgfpathlineto{\pgfqpoint{3.644712in}{2.284917in}}%
\pgfpathlineto{\pgfqpoint{3.658173in}{2.280629in}}%
\pgfpathlineto{\pgfqpoint{3.671638in}{2.276557in}}%
\pgfpathlineto{\pgfqpoint{3.685107in}{2.272699in}}%
\pgfpathlineto{\pgfqpoint{3.693058in}{2.283143in}}%
\pgfpathlineto{\pgfqpoint{3.701003in}{2.293611in}}%
\pgfpathlineto{\pgfqpoint{3.708944in}{2.304106in}}%
\pgfpathlineto{\pgfqpoint{3.716879in}{2.314628in}}%
\pgfpathlineto{\pgfqpoint{3.703419in}{2.318521in}}%
\pgfpathlineto{\pgfqpoint{3.689964in}{2.322630in}}%
\pgfpathlineto{\pgfqpoint{3.676513in}{2.326954in}}%
\pgfpathlineto{\pgfqpoint{3.663066in}{2.331496in}}%
\pgfpathlineto{\pgfqpoint{3.655121in}{2.320926in}}%
\pgfpathlineto{\pgfqpoint{3.647171in}{2.310392in}}%
\pgfpathlineto{\pgfqpoint{3.639216in}{2.299891in}}%
\pgfpathlineto{\pgfqpoint{3.631255in}{2.289423in}}%
\pgfpathclose%
\pgfusepath{fill}%
\end{pgfscope}%
\begin{pgfscope}%
\pgfpathrectangle{\pgfqpoint{1.150000in}{0.150000in}}{\pgfqpoint{5.700000in}{5.700000in}}%
\pgfusepath{clip}%
\pgfsetbuttcap%
\pgfsetroundjoin%
\definecolor{currentfill}{rgb}{0.231674,0.318106,0.544834}%
\pgfsetfillcolor{currentfill}%
\pgfsetfillopacity{0.800000}%
\pgfsetlinewidth{0.000000pt}%
\definecolor{currentstroke}{rgb}{0.000000,0.000000,0.000000}%
\pgfsetstrokecolor{currentstroke}%
\pgfsetdash{}{0pt}%
\pgfpathmoveto{\pgfqpoint{4.711097in}{2.764183in}}%
\pgfpathlineto{\pgfqpoint{4.724873in}{2.767419in}}%
\pgfpathlineto{\pgfqpoint{4.738661in}{2.770837in}}%
\pgfpathlineto{\pgfqpoint{4.752461in}{2.774438in}}%
\pgfpathlineto{\pgfqpoint{4.766273in}{2.778221in}}%
\pgfpathlineto{\pgfqpoint{4.773864in}{2.787008in}}%
\pgfpathlineto{\pgfqpoint{4.781451in}{2.795852in}}%
\pgfpathlineto{\pgfqpoint{4.789033in}{2.804758in}}%
\pgfpathlineto{\pgfqpoint{4.796611in}{2.813731in}}%
\pgfpathlineto{\pgfqpoint{4.782812in}{2.810368in}}%
\pgfpathlineto{\pgfqpoint{4.769025in}{2.807187in}}%
\pgfpathlineto{\pgfqpoint{4.755251in}{2.804187in}}%
\pgfpathlineto{\pgfqpoint{4.741488in}{2.801370in}}%
\pgfpathlineto{\pgfqpoint{4.733896in}{2.791967in}}%
\pgfpathlineto{\pgfqpoint{4.726301in}{2.782638in}}%
\pgfpathlineto{\pgfqpoint{4.718701in}{2.773378in}}%
\pgfpathlineto{\pgfqpoint{4.711097in}{2.764183in}}%
\pgfpathclose%
\pgfusepath{fill}%
\end{pgfscope}%
\begin{pgfscope}%
\pgfpathrectangle{\pgfqpoint{1.150000in}{0.150000in}}{\pgfqpoint{5.700000in}{5.700000in}}%
\pgfusepath{clip}%
\pgfsetbuttcap%
\pgfsetroundjoin%
\definecolor{currentfill}{rgb}{0.147607,0.511733,0.557049}%
\pgfsetfillcolor{currentfill}%
\pgfsetfillopacity{0.800000}%
\pgfsetlinewidth{0.000000pt}%
\definecolor{currentstroke}{rgb}{0.000000,0.000000,0.000000}%
\pgfsetstrokecolor{currentstroke}%
\pgfsetdash{}{0pt}%
\pgfpathmoveto{\pgfqpoint{5.596710in}{3.319975in}}%
\pgfpathlineto{\pgfqpoint{5.610811in}{3.323869in}}%
\pgfpathlineto{\pgfqpoint{5.624928in}{3.327934in}}%
\pgfpathlineto{\pgfqpoint{5.639060in}{3.332167in}}%
\pgfpathlineto{\pgfqpoint{5.653207in}{3.336571in}}%
\pgfpathlineto{\pgfqpoint{5.660516in}{3.346035in}}%
\pgfpathlineto{\pgfqpoint{5.667828in}{3.355775in}}%
\pgfpathlineto{\pgfqpoint{5.675145in}{3.365797in}}%
\pgfpathlineto{\pgfqpoint{5.661020in}{3.361970in}}%
\pgfpathlineto{\pgfqpoint{5.646911in}{3.358312in}}%
\pgfpathlineto{\pgfqpoint{5.632817in}{3.354823in}}%
\pgfpathlineto{\pgfqpoint{5.618737in}{3.351503in}}%
\pgfpathlineto{\pgfqpoint{5.611391in}{3.340706in}}%
\pgfpathlineto{\pgfqpoint{5.604048in}{3.330200in}}%
\pgfpathlineto{\pgfqpoint{5.596710in}{3.319975in}}%
\pgfpathclose%
\pgfusepath{fill}%
\end{pgfscope}%
\begin{pgfscope}%
\pgfpathrectangle{\pgfqpoint{1.150000in}{0.150000in}}{\pgfqpoint{5.700000in}{5.700000in}}%
\pgfusepath{clip}%
\pgfsetbuttcap%
\pgfsetroundjoin%
\definecolor{currentfill}{rgb}{0.274128,0.199721,0.498911}%
\pgfsetfillcolor{currentfill}%
\pgfsetfillopacity{0.800000}%
\pgfsetlinewidth{0.000000pt}%
\definecolor{currentstroke}{rgb}{0.000000,0.000000,0.000000}%
\pgfsetstrokecolor{currentstroke}%
\pgfsetdash{}{0pt}%
\pgfpathmoveto{\pgfqpoint{2.963639in}{2.506269in}}%
\pgfpathlineto{\pgfqpoint{2.977170in}{2.491758in}}%
\pgfpathlineto{\pgfqpoint{2.990696in}{2.477528in}}%
\pgfpathlineto{\pgfqpoint{3.004217in}{2.463576in}}%
\pgfpathlineto{\pgfqpoint{3.017733in}{2.449901in}}%
\pgfpathlineto{\pgfqpoint{3.025920in}{2.458957in}}%
\pgfpathlineto{\pgfqpoint{3.034099in}{2.468115in}}%
\pgfpathlineto{\pgfqpoint{3.042270in}{2.477375in}}%
\pgfpathlineto{\pgfqpoint{3.050434in}{2.486735in}}%
\pgfpathlineto{\pgfqpoint{3.036936in}{2.500316in}}%
\pgfpathlineto{\pgfqpoint{3.023434in}{2.514174in}}%
\pgfpathlineto{\pgfqpoint{3.009927in}{2.528309in}}%
\pgfpathlineto{\pgfqpoint{2.996415in}{2.542725in}}%
\pgfpathlineto{\pgfqpoint{2.988233in}{2.533447in}}%
\pgfpathlineto{\pgfqpoint{2.980043in}{2.524278in}}%
\pgfpathlineto{\pgfqpoint{2.971845in}{2.515219in}}%
\pgfpathlineto{\pgfqpoint{2.963639in}{2.506269in}}%
\pgfpathclose%
\pgfusepath{fill}%
\end{pgfscope}%
\begin{pgfscope}%
\pgfpathrectangle{\pgfqpoint{1.150000in}{0.150000in}}{\pgfqpoint{5.700000in}{5.700000in}}%
\pgfusepath{clip}%
\pgfsetbuttcap%
\pgfsetroundjoin%
\definecolor{currentfill}{rgb}{0.280255,0.165693,0.476498}%
\pgfsetfillcolor{currentfill}%
\pgfsetfillopacity{0.800000}%
\pgfsetlinewidth{0.000000pt}%
\definecolor{currentstroke}{rgb}{0.000000,0.000000,0.000000}%
\pgfsetstrokecolor{currentstroke}%
\pgfsetdash{}{0pt}%
\pgfpathmoveto{\pgfqpoint{4.027386in}{2.404888in}}%
\pgfpathlineto{\pgfqpoint{4.040923in}{2.404216in}}%
\pgfpathlineto{\pgfqpoint{4.054467in}{2.403743in}}%
\pgfpathlineto{\pgfqpoint{4.068020in}{2.403470in}}%
\pgfpathlineto{\pgfqpoint{4.081580in}{2.403395in}}%
\pgfpathlineto{\pgfqpoint{4.089406in}{2.413637in}}%
\pgfpathlineto{\pgfqpoint{4.097228in}{2.423889in}}%
\pgfpathlineto{\pgfqpoint{4.105045in}{2.434152in}}%
\pgfpathlineto{\pgfqpoint{4.112856in}{2.444430in}}%
\pgfpathlineto{\pgfqpoint{4.099304in}{2.444667in}}%
\pgfpathlineto{\pgfqpoint{4.085760in}{2.445104in}}%
\pgfpathlineto{\pgfqpoint{4.072224in}{2.445739in}}%
\pgfpathlineto{\pgfqpoint{4.058695in}{2.446574in}}%
\pgfpathlineto{\pgfqpoint{4.050875in}{2.436123in}}%
\pgfpathlineto{\pgfqpoint{4.043050in}{2.425693in}}%
\pgfpathlineto{\pgfqpoint{4.035221in}{2.415282in}}%
\pgfpathlineto{\pgfqpoint{4.027386in}{2.404888in}}%
\pgfpathclose%
\pgfusepath{fill}%
\end{pgfscope}%
\begin{pgfscope}%
\pgfpathrectangle{\pgfqpoint{1.150000in}{0.150000in}}{\pgfqpoint{5.700000in}{5.700000in}}%
\pgfusepath{clip}%
\pgfsetbuttcap%
\pgfsetroundjoin%
\definecolor{currentfill}{rgb}{0.248629,0.278775,0.534556}%
\pgfsetfillcolor{currentfill}%
\pgfsetfillopacity{0.800000}%
\pgfsetlinewidth{0.000000pt}%
\definecolor{currentstroke}{rgb}{0.000000,0.000000,0.000000}%
\pgfsetstrokecolor{currentstroke}%
\pgfsetdash{}{0pt}%
\pgfpathmoveto{\pgfqpoint{2.800762in}{2.703175in}}%
\pgfpathlineto{\pgfqpoint{2.814377in}{2.685104in}}%
\pgfpathlineto{\pgfqpoint{2.827983in}{2.667344in}}%
\pgfpathlineto{\pgfqpoint{2.841581in}{2.649893in}}%
\pgfpathlineto{\pgfqpoint{2.855171in}{2.632747in}}%
\pgfpathlineto{\pgfqpoint{2.863419in}{2.641481in}}%
\pgfpathlineto{\pgfqpoint{2.871658in}{2.650342in}}%
\pgfpathlineto{\pgfqpoint{2.879888in}{2.659327in}}%
\pgfpathlineto{\pgfqpoint{2.888109in}{2.668438in}}%
\pgfpathlineto{\pgfqpoint{2.874540in}{2.685487in}}%
\pgfpathlineto{\pgfqpoint{2.860964in}{2.702841in}}%
\pgfpathlineto{\pgfqpoint{2.847379in}{2.720503in}}%
\pgfpathlineto{\pgfqpoint{2.833787in}{2.738477in}}%
\pgfpathlineto{\pgfqpoint{2.825544in}{2.729451in}}%
\pgfpathlineto{\pgfqpoint{2.817293in}{2.720559in}}%
\pgfpathlineto{\pgfqpoint{2.809032in}{2.711800in}}%
\pgfpathlineto{\pgfqpoint{2.800762in}{2.703175in}}%
\pgfpathclose%
\pgfusepath{fill}%
\end{pgfscope}%
\begin{pgfscope}%
\pgfpathrectangle{\pgfqpoint{1.150000in}{0.150000in}}{\pgfqpoint{5.700000in}{5.700000in}}%
\pgfusepath{clip}%
\pgfsetbuttcap%
\pgfsetroundjoin%
\definecolor{currentfill}{rgb}{0.221989,0.339161,0.548752}%
\pgfsetfillcolor{currentfill}%
\pgfsetfillopacity{0.800000}%
\pgfsetlinewidth{0.000000pt}%
\definecolor{currentstroke}{rgb}{0.000000,0.000000,0.000000}%
\pgfsetstrokecolor{currentstroke}%
\pgfsetdash{}{0pt}%
\pgfpathmoveto{\pgfqpoint{4.796611in}{2.813731in}}%
\pgfpathlineto{\pgfqpoint{4.810422in}{2.817276in}}%
\pgfpathlineto{\pgfqpoint{4.824246in}{2.821002in}}%
\pgfpathlineto{\pgfqpoint{4.838082in}{2.824910in}}%
\pgfpathlineto{\pgfqpoint{4.851932in}{2.828998in}}%
\pgfpathlineto{\pgfqpoint{4.859491in}{2.837603in}}%
\pgfpathlineto{\pgfqpoint{4.867047in}{2.846277in}}%
\pgfpathlineto{\pgfqpoint{4.874598in}{2.855026in}}%
\pgfpathlineto{\pgfqpoint{4.882145in}{2.863854in}}%
\pgfpathlineto{\pgfqpoint{4.868311in}{2.860218in}}%
\pgfpathlineto{\pgfqpoint{4.854489in}{2.856763in}}%
\pgfpathlineto{\pgfqpoint{4.840680in}{2.853488in}}%
\pgfpathlineto{\pgfqpoint{4.826883in}{2.850394in}}%
\pgfpathlineto{\pgfqpoint{4.819321in}{2.841103in}}%
\pgfpathlineto{\pgfqpoint{4.811755in}{2.831898in}}%
\pgfpathlineto{\pgfqpoint{4.804185in}{2.822776in}}%
\pgfpathlineto{\pgfqpoint{4.796611in}{2.813731in}}%
\pgfpathclose%
\pgfusepath{fill}%
\end{pgfscope}%
\begin{pgfscope}%
\pgfpathrectangle{\pgfqpoint{1.150000in}{0.150000in}}{\pgfqpoint{5.700000in}{5.700000in}}%
\pgfusepath{clip}%
\pgfsetbuttcap%
\pgfsetroundjoin%
\definecolor{currentfill}{rgb}{0.214298,0.355619,0.551184}%
\pgfsetfillcolor{currentfill}%
\pgfsetfillopacity{0.800000}%
\pgfsetlinewidth{0.000000pt}%
\definecolor{currentstroke}{rgb}{0.000000,0.000000,0.000000}%
\pgfsetstrokecolor{currentstroke}%
\pgfsetdash{}{0pt}%
\pgfpathmoveto{\pgfqpoint{4.882145in}{2.863854in}}%
\pgfpathlineto{\pgfqpoint{4.895993in}{2.867670in}}%
\pgfpathlineto{\pgfqpoint{4.909853in}{2.871666in}}%
\pgfpathlineto{\pgfqpoint{4.923726in}{2.875841in}}%
\pgfpathlineto{\pgfqpoint{4.937612in}{2.880196in}}%
\pgfpathlineto{\pgfqpoint{4.945140in}{2.888637in}}%
\pgfpathlineto{\pgfqpoint{4.952665in}{2.897161in}}%
\pgfpathlineto{\pgfqpoint{4.960185in}{2.905773in}}%
\pgfpathlineto{\pgfqpoint{4.967702in}{2.914479in}}%
\pgfpathlineto{\pgfqpoint{4.953832in}{2.910609in}}%
\pgfpathlineto{\pgfqpoint{4.939975in}{2.906918in}}%
\pgfpathlineto{\pgfqpoint{4.926130in}{2.903406in}}%
\pgfpathlineto{\pgfqpoint{4.912299in}{2.900073in}}%
\pgfpathlineto{\pgfqpoint{4.904766in}{2.890872in}}%
\pgfpathlineto{\pgfqpoint{4.897229in}{2.881772in}}%
\pgfpathlineto{\pgfqpoint{4.889689in}{2.872768in}}%
\pgfpathlineto{\pgfqpoint{4.882145in}{2.863854in}}%
\pgfpathclose%
\pgfusepath{fill}%
\end{pgfscope}%
\begin{pgfscope}%
\pgfpathrectangle{\pgfqpoint{1.150000in}{0.150000in}}{\pgfqpoint{5.700000in}{5.700000in}}%
\pgfusepath{clip}%
\pgfsetbuttcap%
\pgfsetroundjoin%
\definecolor{currentfill}{rgb}{0.281887,0.150881,0.465405}%
\pgfsetfillcolor{currentfill}%
\pgfsetfillopacity{0.800000}%
\pgfsetlinewidth{0.000000pt}%
\definecolor{currentstroke}{rgb}{0.000000,0.000000,0.000000}%
\pgfsetstrokecolor{currentstroke}%
\pgfsetdash{}{0pt}%
\pgfpathmoveto{\pgfqpoint{3.941889in}{2.367624in}}%
\pgfpathlineto{\pgfqpoint{3.955406in}{2.366276in}}%
\pgfpathlineto{\pgfqpoint{3.968929in}{2.365131in}}%
\pgfpathlineto{\pgfqpoint{3.982459in}{2.364188in}}%
\pgfpathlineto{\pgfqpoint{3.995997in}{2.363446in}}%
\pgfpathlineto{\pgfqpoint{4.003852in}{2.373790in}}%
\pgfpathlineto{\pgfqpoint{4.011701in}{2.384144in}}%
\pgfpathlineto{\pgfqpoint{4.019546in}{2.394510in}}%
\pgfpathlineto{\pgfqpoint{4.027386in}{2.404888in}}%
\pgfpathlineto{\pgfqpoint{4.013856in}{2.405761in}}%
\pgfpathlineto{\pgfqpoint{4.000334in}{2.406836in}}%
\pgfpathlineto{\pgfqpoint{3.986819in}{2.408112in}}%
\pgfpathlineto{\pgfqpoint{3.973311in}{2.409591in}}%
\pgfpathlineto{\pgfqpoint{3.965463in}{2.399070in}}%
\pgfpathlineto{\pgfqpoint{3.957610in}{2.388570in}}%
\pgfpathlineto{\pgfqpoint{3.949752in}{2.378088in}}%
\pgfpathlineto{\pgfqpoint{3.941889in}{2.367624in}}%
\pgfpathclose%
\pgfusepath{fill}%
\end{pgfscope}%
\begin{pgfscope}%
\pgfpathrectangle{\pgfqpoint{1.150000in}{0.150000in}}{\pgfqpoint{5.700000in}{5.700000in}}%
\pgfusepath{clip}%
\pgfsetbuttcap%
\pgfsetroundjoin%
\definecolor{currentfill}{rgb}{0.278826,0.175490,0.483397}%
\pgfsetfillcolor{currentfill}%
\pgfsetfillopacity{0.800000}%
\pgfsetlinewidth{0.000000pt}%
\definecolor{currentstroke}{rgb}{0.000000,0.000000,0.000000}%
\pgfsetstrokecolor{currentstroke}%
\pgfsetdash{}{0pt}%
\pgfpathmoveto{\pgfqpoint{3.017733in}{2.449901in}}%
\pgfpathlineto{\pgfqpoint{3.031245in}{2.436500in}}%
\pgfpathlineto{\pgfqpoint{3.044753in}{2.423372in}}%
\pgfpathlineto{\pgfqpoint{3.058256in}{2.410513in}}%
\pgfpathlineto{\pgfqpoint{3.071756in}{2.397922in}}%
\pgfpathlineto{\pgfqpoint{3.079924in}{2.407083in}}%
\pgfpathlineto{\pgfqpoint{3.088085in}{2.416339in}}%
\pgfpathlineto{\pgfqpoint{3.096238in}{2.425688in}}%
\pgfpathlineto{\pgfqpoint{3.104384in}{2.435130in}}%
\pgfpathlineto{\pgfqpoint{3.090902in}{2.447627in}}%
\pgfpathlineto{\pgfqpoint{3.077417in}{2.460393in}}%
\pgfpathlineto{\pgfqpoint{3.063927in}{2.473428in}}%
\pgfpathlineto{\pgfqpoint{3.050434in}{2.486735in}}%
\pgfpathlineto{\pgfqpoint{3.042270in}{2.477375in}}%
\pgfpathlineto{\pgfqpoint{3.034099in}{2.468115in}}%
\pgfpathlineto{\pgfqpoint{3.025920in}{2.458957in}}%
\pgfpathlineto{\pgfqpoint{3.017733in}{2.449901in}}%
\pgfpathclose%
\pgfusepath{fill}%
\end{pgfscope}%
\begin{pgfscope}%
\pgfpathrectangle{\pgfqpoint{1.150000in}{0.150000in}}{\pgfqpoint{5.700000in}{5.700000in}}%
\pgfusepath{clip}%
\pgfsetbuttcap%
\pgfsetroundjoin%
\definecolor{currentfill}{rgb}{0.282910,0.105393,0.426902}%
\pgfsetfillcolor{currentfill}%
\pgfsetfillopacity{0.800000}%
\pgfsetlinewidth{0.000000pt}%
\definecolor{currentstroke}{rgb}{0.000000,0.000000,0.000000}%
\pgfsetstrokecolor{currentstroke}%
\pgfsetdash{}{0pt}%
\pgfpathmoveto{\pgfqpoint{3.405864in}{2.278489in}}%
\pgfpathlineto{\pgfqpoint{3.419311in}{2.271266in}}%
\pgfpathlineto{\pgfqpoint{3.432760in}{2.264274in}}%
\pgfpathlineto{\pgfqpoint{3.446211in}{2.257511in}}%
\pgfpathlineto{\pgfqpoint{3.459664in}{2.250978in}}%
\pgfpathlineto{\pgfqpoint{3.467691in}{2.261123in}}%
\pgfpathlineto{\pgfqpoint{3.475712in}{2.271313in}}%
\pgfpathlineto{\pgfqpoint{3.483728in}{2.281549in}}%
\pgfpathlineto{\pgfqpoint{3.491738in}{2.291830in}}%
\pgfpathlineto{\pgfqpoint{3.478297in}{2.298336in}}%
\pgfpathlineto{\pgfqpoint{3.464858in}{2.305071in}}%
\pgfpathlineto{\pgfqpoint{3.451421in}{2.312035in}}%
\pgfpathlineto{\pgfqpoint{3.437985in}{2.319232in}}%
\pgfpathlineto{\pgfqpoint{3.429964in}{2.308966in}}%
\pgfpathlineto{\pgfqpoint{3.421936in}{2.298754in}}%
\pgfpathlineto{\pgfqpoint{3.413903in}{2.288595in}}%
\pgfpathlineto{\pgfqpoint{3.405864in}{2.278489in}}%
\pgfpathclose%
\pgfusepath{fill}%
\end{pgfscope}%
\begin{pgfscope}%
\pgfpathrectangle{\pgfqpoint{1.150000in}{0.150000in}}{\pgfqpoint{5.700000in}{5.700000in}}%
\pgfusepath{clip}%
\pgfsetbuttcap%
\pgfsetroundjoin%
\definecolor{currentfill}{rgb}{0.235526,0.309527,0.542944}%
\pgfsetfillcolor{currentfill}%
\pgfsetfillopacity{0.800000}%
\pgfsetlinewidth{0.000000pt}%
\definecolor{currentstroke}{rgb}{0.000000,0.000000,0.000000}%
\pgfsetstrokecolor{currentstroke}%
\pgfsetdash{}{0pt}%
\pgfpathmoveto{\pgfqpoint{2.746212in}{2.778628in}}%
\pgfpathlineto{\pgfqpoint{2.759864in}{2.759283in}}%
\pgfpathlineto{\pgfqpoint{2.773506in}{2.740262in}}%
\pgfpathlineto{\pgfqpoint{2.787139in}{2.721560in}}%
\pgfpathlineto{\pgfqpoint{2.800762in}{2.703175in}}%
\pgfpathlineto{\pgfqpoint{2.809032in}{2.711800in}}%
\pgfpathlineto{\pgfqpoint{2.817293in}{2.720559in}}%
\pgfpathlineto{\pgfqpoint{2.825544in}{2.729451in}}%
\pgfpathlineto{\pgfqpoint{2.833787in}{2.738477in}}%
\pgfpathlineto{\pgfqpoint{2.820185in}{2.756764in}}%
\pgfpathlineto{\pgfqpoint{2.806575in}{2.775367in}}%
\pgfpathlineto{\pgfqpoint{2.792955in}{2.794290in}}%
\pgfpathlineto{\pgfqpoint{2.779326in}{2.813536in}}%
\pgfpathlineto{\pgfqpoint{2.771062in}{2.804597in}}%
\pgfpathlineto{\pgfqpoint{2.762788in}{2.795798in}}%
\pgfpathlineto{\pgfqpoint{2.754505in}{2.787142in}}%
\pgfpathlineto{\pgfqpoint{2.746212in}{2.778628in}}%
\pgfpathclose%
\pgfusepath{fill}%
\end{pgfscope}%
\begin{pgfscope}%
\pgfpathrectangle{\pgfqpoint{1.150000in}{0.150000in}}{\pgfqpoint{5.700000in}{5.700000in}}%
\pgfusepath{clip}%
\pgfsetbuttcap%
\pgfsetroundjoin%
\definecolor{currentfill}{rgb}{0.204903,0.375746,0.553533}%
\pgfsetfillcolor{currentfill}%
\pgfsetfillopacity{0.800000}%
\pgfsetlinewidth{0.000000pt}%
\definecolor{currentstroke}{rgb}{0.000000,0.000000,0.000000}%
\pgfsetstrokecolor{currentstroke}%
\pgfsetdash{}{0pt}%
\pgfpathmoveto{\pgfqpoint{4.967702in}{2.914479in}}%
\pgfpathlineto{\pgfqpoint{4.981586in}{2.918528in}}%
\pgfpathlineto{\pgfqpoint{4.995482in}{2.922755in}}%
\pgfpathlineto{\pgfqpoint{5.009392in}{2.927161in}}%
\pgfpathlineto{\pgfqpoint{5.023316in}{2.931744in}}%
\pgfpathlineto{\pgfqpoint{5.030813in}{2.940045in}}%
\pgfpathlineto{\pgfqpoint{5.038306in}{2.948443in}}%
\pgfpathlineto{\pgfqpoint{5.045796in}{2.956945in}}%
\pgfpathlineto{\pgfqpoint{5.053282in}{2.965557in}}%
\pgfpathlineto{\pgfqpoint{5.039376in}{2.961490in}}%
\pgfpathlineto{\pgfqpoint{5.025483in}{2.957601in}}%
\pgfpathlineto{\pgfqpoint{5.011604in}{2.953890in}}%
\pgfpathlineto{\pgfqpoint{4.997737in}{2.950356in}}%
\pgfpathlineto{\pgfqpoint{4.990233in}{2.941218in}}%
\pgfpathlineto{\pgfqpoint{4.982726in}{2.932196in}}%
\pgfpathlineto{\pgfqpoint{4.975216in}{2.923285in}}%
\pgfpathlineto{\pgfqpoint{4.967702in}{2.914479in}}%
\pgfpathclose%
\pgfusepath{fill}%
\end{pgfscope}%
\begin{pgfscope}%
\pgfpathrectangle{\pgfqpoint{1.150000in}{0.150000in}}{\pgfqpoint{5.700000in}{5.700000in}}%
\pgfusepath{clip}%
\pgfsetbuttcap%
\pgfsetroundjoin%
\definecolor{currentfill}{rgb}{0.283197,0.115680,0.436115}%
\pgfsetfillcolor{currentfill}%
\pgfsetfillopacity{0.800000}%
\pgfsetlinewidth{0.000000pt}%
\definecolor{currentstroke}{rgb}{0.000000,0.000000,0.000000}%
\pgfsetstrokecolor{currentstroke}%
\pgfsetdash{}{0pt}%
\pgfpathmoveto{\pgfqpoint{3.265983in}{2.305356in}}%
\pgfpathlineto{\pgfqpoint{3.279440in}{2.296171in}}%
\pgfpathlineto{\pgfqpoint{3.292898in}{2.287229in}}%
\pgfpathlineto{\pgfqpoint{3.306355in}{2.278528in}}%
\pgfpathlineto{\pgfqpoint{3.319812in}{2.270068in}}%
\pgfpathlineto{\pgfqpoint{3.327889in}{2.279891in}}%
\pgfpathlineto{\pgfqpoint{3.335960in}{2.289775in}}%
\pgfpathlineto{\pgfqpoint{3.344025in}{2.299720in}}%
\pgfpathlineto{\pgfqpoint{3.352083in}{2.309725in}}%
\pgfpathlineto{\pgfqpoint{3.338640in}{2.318127in}}%
\pgfpathlineto{\pgfqpoint{3.325196in}{2.326768in}}%
\pgfpathlineto{\pgfqpoint{3.311753in}{2.335650in}}%
\pgfpathlineto{\pgfqpoint{3.298310in}{2.344776in}}%
\pgfpathlineto{\pgfqpoint{3.290238in}{2.334818in}}%
\pgfpathlineto{\pgfqpoint{3.282159in}{2.324928in}}%
\pgfpathlineto{\pgfqpoint{3.274074in}{2.315108in}}%
\pgfpathlineto{\pgfqpoint{3.265983in}{2.305356in}}%
\pgfpathclose%
\pgfusepath{fill}%
\end{pgfscope}%
\begin{pgfscope}%
\pgfpathrectangle{\pgfqpoint{1.150000in}{0.150000in}}{\pgfqpoint{5.700000in}{5.700000in}}%
\pgfusepath{clip}%
\pgfsetbuttcap%
\pgfsetroundjoin%
\definecolor{currentfill}{rgb}{0.195860,0.395433,0.555276}%
\pgfsetfillcolor{currentfill}%
\pgfsetfillopacity{0.800000}%
\pgfsetlinewidth{0.000000pt}%
\definecolor{currentstroke}{rgb}{0.000000,0.000000,0.000000}%
\pgfsetstrokecolor{currentstroke}%
\pgfsetdash{}{0pt}%
\pgfpathmoveto{\pgfqpoint{5.053282in}{2.965557in}}%
\pgfpathlineto{\pgfqpoint{5.067202in}{2.969801in}}%
\pgfpathlineto{\pgfqpoint{5.081136in}{2.974222in}}%
\pgfpathlineto{\pgfqpoint{5.095083in}{2.978820in}}%
\pgfpathlineto{\pgfqpoint{5.109044in}{2.983594in}}%
\pgfpathlineto{\pgfqpoint{5.116509in}{2.991784in}}%
\pgfpathlineto{\pgfqpoint{5.123972in}{3.000087in}}%
\pgfpathlineto{\pgfqpoint{5.131431in}{3.008511in}}%
\pgfpathlineto{\pgfqpoint{5.138888in}{3.017061in}}%
\pgfpathlineto{\pgfqpoint{5.124946in}{3.012836in}}%
\pgfpathlineto{\pgfqpoint{5.111017in}{3.008787in}}%
\pgfpathlineto{\pgfqpoint{5.097103in}{3.004914in}}%
\pgfpathlineto{\pgfqpoint{5.083201in}{3.001218in}}%
\pgfpathlineto{\pgfqpoint{5.075725in}{2.992108in}}%
\pgfpathlineto{\pgfqpoint{5.068247in}{2.983132in}}%
\pgfpathlineto{\pgfqpoint{5.060766in}{2.974284in}}%
\pgfpathlineto{\pgfqpoint{5.053282in}{2.965557in}}%
\pgfpathclose%
\pgfusepath{fill}%
\end{pgfscope}%
\begin{pgfscope}%
\pgfpathrectangle{\pgfqpoint{1.150000in}{0.150000in}}{\pgfqpoint{5.700000in}{5.700000in}}%
\pgfusepath{clip}%
\pgfsetbuttcap%
\pgfsetroundjoin%
\definecolor{currentfill}{rgb}{0.282884,0.135920,0.453427}%
\pgfsetfillcolor{currentfill}%
\pgfsetfillopacity{0.800000}%
\pgfsetlinewidth{0.000000pt}%
\definecolor{currentstroke}{rgb}{0.000000,0.000000,0.000000}%
\pgfsetstrokecolor{currentstroke}%
\pgfsetdash{}{0pt}%
\pgfpathmoveto{\pgfqpoint{3.856354in}{2.332946in}}%
\pgfpathlineto{\pgfqpoint{3.869853in}{2.330878in}}%
\pgfpathlineto{\pgfqpoint{3.883358in}{2.329017in}}%
\pgfpathlineto{\pgfqpoint{3.896869in}{2.327360in}}%
\pgfpathlineto{\pgfqpoint{3.910387in}{2.325908in}}%
\pgfpathlineto{\pgfqpoint{3.918270in}{2.336319in}}%
\pgfpathlineto{\pgfqpoint{3.926148in}{2.346741in}}%
\pgfpathlineto{\pgfqpoint{3.934021in}{2.357176in}}%
\pgfpathlineto{\pgfqpoint{3.941889in}{2.367624in}}%
\pgfpathlineto{\pgfqpoint{3.928380in}{2.369176in}}%
\pgfpathlineto{\pgfqpoint{3.914877in}{2.370932in}}%
\pgfpathlineto{\pgfqpoint{3.901380in}{2.372894in}}%
\pgfpathlineto{\pgfqpoint{3.887890in}{2.375061in}}%
\pgfpathlineto{\pgfqpoint{3.880014in}{2.364501in}}%
\pgfpathlineto{\pgfqpoint{3.872132in}{2.353963in}}%
\pgfpathlineto{\pgfqpoint{3.864246in}{2.343445in}}%
\pgfpathlineto{\pgfqpoint{3.856354in}{2.332946in}}%
\pgfpathclose%
\pgfusepath{fill}%
\end{pgfscope}%
\begin{pgfscope}%
\pgfpathrectangle{\pgfqpoint{1.150000in}{0.150000in}}{\pgfqpoint{5.700000in}{5.700000in}}%
\pgfusepath{clip}%
\pgfsetbuttcap%
\pgfsetroundjoin%
\definecolor{currentfill}{rgb}{0.282910,0.105393,0.426902}%
\pgfsetfillcolor{currentfill}%
\pgfsetfillopacity{0.800000}%
\pgfsetlinewidth{0.000000pt}%
\definecolor{currentstroke}{rgb}{0.000000,0.000000,0.000000}%
\pgfsetstrokecolor{currentstroke}%
\pgfsetdash{}{0pt}%
\pgfpathmoveto{\pgfqpoint{3.545525in}{2.268068in}}%
\pgfpathlineto{\pgfqpoint{3.558978in}{2.262688in}}%
\pgfpathlineto{\pgfqpoint{3.572435in}{2.257529in}}%
\pgfpathlineto{\pgfqpoint{3.585895in}{2.252590in}}%
\pgfpathlineto{\pgfqpoint{3.599358in}{2.247871in}}%
\pgfpathlineto{\pgfqpoint{3.607340in}{2.258212in}}%
\pgfpathlineto{\pgfqpoint{3.615317in}{2.268585in}}%
\pgfpathlineto{\pgfqpoint{3.623289in}{2.278988in}}%
\pgfpathlineto{\pgfqpoint{3.631255in}{2.289423in}}%
\pgfpathlineto{\pgfqpoint{3.617802in}{2.294147in}}%
\pgfpathlineto{\pgfqpoint{3.604352in}{2.299090in}}%
\pgfpathlineto{\pgfqpoint{3.590906in}{2.304254in}}%
\pgfpathlineto{\pgfqpoint{3.577463in}{2.309639in}}%
\pgfpathlineto{\pgfqpoint{3.569487in}{2.299188in}}%
\pgfpathlineto{\pgfqpoint{3.561505in}{2.288776in}}%
\pgfpathlineto{\pgfqpoint{3.553517in}{2.278403in}}%
\pgfpathlineto{\pgfqpoint{3.545525in}{2.268068in}}%
\pgfpathclose%
\pgfusepath{fill}%
\end{pgfscope}%
\begin{pgfscope}%
\pgfpathrectangle{\pgfqpoint{1.150000in}{0.150000in}}{\pgfqpoint{5.700000in}{5.700000in}}%
\pgfusepath{clip}%
\pgfsetbuttcap%
\pgfsetroundjoin%
\definecolor{currentfill}{rgb}{0.281412,0.155834,0.469201}%
\pgfsetfillcolor{currentfill}%
\pgfsetfillopacity{0.800000}%
\pgfsetlinewidth{0.000000pt}%
\definecolor{currentstroke}{rgb}{0.000000,0.000000,0.000000}%
\pgfsetstrokecolor{currentstroke}%
\pgfsetdash{}{0pt}%
\pgfpathmoveto{\pgfqpoint{3.071756in}{2.397922in}}%
\pgfpathlineto{\pgfqpoint{3.085253in}{2.385597in}}%
\pgfpathlineto{\pgfqpoint{3.098746in}{2.373536in}}%
\pgfpathlineto{\pgfqpoint{3.112236in}{2.361738in}}%
\pgfpathlineto{\pgfqpoint{3.125723in}{2.350199in}}%
\pgfpathlineto{\pgfqpoint{3.133874in}{2.359465in}}%
\pgfpathlineto{\pgfqpoint{3.142017in}{2.368817in}}%
\pgfpathlineto{\pgfqpoint{3.150154in}{2.378255in}}%
\pgfpathlineto{\pgfqpoint{3.158283in}{2.387778in}}%
\pgfpathlineto{\pgfqpoint{3.144812in}{2.399224in}}%
\pgfpathlineto{\pgfqpoint{3.131339in}{2.410930in}}%
\pgfpathlineto{\pgfqpoint{3.117863in}{2.422898in}}%
\pgfpathlineto{\pgfqpoint{3.104384in}{2.435130in}}%
\pgfpathlineto{\pgfqpoint{3.096238in}{2.425688in}}%
\pgfpathlineto{\pgfqpoint{3.088085in}{2.416339in}}%
\pgfpathlineto{\pgfqpoint{3.079924in}{2.407083in}}%
\pgfpathlineto{\pgfqpoint{3.071756in}{2.397922in}}%
\pgfpathclose%
\pgfusepath{fill}%
\end{pgfscope}%
\begin{pgfscope}%
\pgfpathrectangle{\pgfqpoint{1.150000in}{0.150000in}}{\pgfqpoint{5.700000in}{5.700000in}}%
\pgfusepath{clip}%
\pgfsetbuttcap%
\pgfsetroundjoin%
\definecolor{currentfill}{rgb}{0.187231,0.414746,0.556547}%
\pgfsetfillcolor{currentfill}%
\pgfsetfillopacity{0.800000}%
\pgfsetlinewidth{0.000000pt}%
\definecolor{currentstroke}{rgb}{0.000000,0.000000,0.000000}%
\pgfsetstrokecolor{currentstroke}%
\pgfsetdash{}{0pt}%
\pgfpathmoveto{\pgfqpoint{5.138888in}{3.017061in}}%
\pgfpathlineto{\pgfqpoint{5.152844in}{3.021463in}}%
\pgfpathlineto{\pgfqpoint{5.166814in}{3.026040in}}%
\pgfpathlineto{\pgfqpoint{5.180798in}{3.030793in}}%
\pgfpathlineto{\pgfqpoint{5.194797in}{3.035722in}}%
\pgfpathlineto{\pgfqpoint{5.202231in}{3.043835in}}%
\pgfpathlineto{\pgfqpoint{5.209663in}{3.052079in}}%
\pgfpathlineto{\pgfqpoint{5.217093in}{3.060462in}}%
\pgfpathlineto{\pgfqpoint{5.224521in}{3.068990in}}%
\pgfpathlineto{\pgfqpoint{5.210543in}{3.064643in}}%
\pgfpathlineto{\pgfqpoint{5.196579in}{3.060471in}}%
\pgfpathlineto{\pgfqpoint{5.182629in}{3.056474in}}%
\pgfpathlineto{\pgfqpoint{5.168693in}{3.052653in}}%
\pgfpathlineto{\pgfqpoint{5.161245in}{3.043533in}}%
\pgfpathlineto{\pgfqpoint{5.153795in}{3.034566in}}%
\pgfpathlineto{\pgfqpoint{5.146342in}{3.025744in}}%
\pgfpathlineto{\pgfqpoint{5.138888in}{3.017061in}}%
\pgfpathclose%
\pgfusepath{fill}%
\end{pgfscope}%
\begin{pgfscope}%
\pgfpathrectangle{\pgfqpoint{1.150000in}{0.150000in}}{\pgfqpoint{5.700000in}{5.700000in}}%
\pgfusepath{clip}%
\pgfsetbuttcap%
\pgfsetroundjoin%
\definecolor{currentfill}{rgb}{0.221989,0.339161,0.548752}%
\pgfsetfillcolor{currentfill}%
\pgfsetfillopacity{0.800000}%
\pgfsetlinewidth{0.000000pt}%
\definecolor{currentstroke}{rgb}{0.000000,0.000000,0.000000}%
\pgfsetstrokecolor{currentstroke}%
\pgfsetdash{}{0pt}%
\pgfpathmoveto{\pgfqpoint{2.691500in}{2.859296in}}%
\pgfpathlineto{\pgfqpoint{2.705194in}{2.838629in}}%
\pgfpathlineto{\pgfqpoint{2.718877in}{2.818297in}}%
\pgfpathlineto{\pgfqpoint{2.732550in}{2.798298in}}%
\pgfpathlineto{\pgfqpoint{2.746212in}{2.778628in}}%
\pgfpathlineto{\pgfqpoint{2.754505in}{2.787142in}}%
\pgfpathlineto{\pgfqpoint{2.762788in}{2.795798in}}%
\pgfpathlineto{\pgfqpoint{2.771062in}{2.804597in}}%
\pgfpathlineto{\pgfqpoint{2.779326in}{2.813536in}}%
\pgfpathlineto{\pgfqpoint{2.765688in}{2.833107in}}%
\pgfpathlineto{\pgfqpoint{2.752039in}{2.853007in}}%
\pgfpathlineto{\pgfqpoint{2.738379in}{2.873238in}}%
\pgfpathlineto{\pgfqpoint{2.724709in}{2.893805in}}%
\pgfpathlineto{\pgfqpoint{2.716421in}{2.884953in}}%
\pgfpathlineto{\pgfqpoint{2.708124in}{2.876251in}}%
\pgfpathlineto{\pgfqpoint{2.699817in}{2.867698in}}%
\pgfpathlineto{\pgfqpoint{2.691500in}{2.859296in}}%
\pgfpathclose%
\pgfusepath{fill}%
\end{pgfscope}%
\begin{pgfscope}%
\pgfpathrectangle{\pgfqpoint{1.150000in}{0.150000in}}{\pgfqpoint{5.700000in}{5.700000in}}%
\pgfusepath{clip}%
\pgfsetbuttcap%
\pgfsetroundjoin%
\definecolor{currentfill}{rgb}{0.179019,0.433756,0.557430}%
\pgfsetfillcolor{currentfill}%
\pgfsetfillopacity{0.800000}%
\pgfsetlinewidth{0.000000pt}%
\definecolor{currentstroke}{rgb}{0.000000,0.000000,0.000000}%
\pgfsetstrokecolor{currentstroke}%
\pgfsetdash{}{0pt}%
\pgfpathmoveto{\pgfqpoint{5.224521in}{3.068990in}}%
\pgfpathlineto{\pgfqpoint{5.238513in}{3.073512in}}%
\pgfpathlineto{\pgfqpoint{5.252519in}{3.078208in}}%
\pgfpathlineto{\pgfqpoint{5.266540in}{3.083079in}}%
\pgfpathlineto{\pgfqpoint{5.280576in}{3.088125in}}%
\pgfpathlineto{\pgfqpoint{5.287980in}{3.096202in}}%
\pgfpathlineto{\pgfqpoint{5.295382in}{3.104429in}}%
\pgfpathlineto{\pgfqpoint{5.302783in}{3.112815in}}%
\pgfpathlineto{\pgfqpoint{5.310183in}{3.121364in}}%
\pgfpathlineto{\pgfqpoint{5.296170in}{3.116933in}}%
\pgfpathlineto{\pgfqpoint{5.282171in}{3.112675in}}%
\pgfpathlineto{\pgfqpoint{5.268187in}{3.108592in}}%
\pgfpathlineto{\pgfqpoint{5.254216in}{3.104682in}}%
\pgfpathlineto{\pgfqpoint{5.246794in}{3.095508in}}%
\pgfpathlineto{\pgfqpoint{5.239371in}{3.086506in}}%
\pgfpathlineto{\pgfqpoint{5.231947in}{3.077669in}}%
\pgfpathlineto{\pgfqpoint{5.224521in}{3.068990in}}%
\pgfpathclose%
\pgfusepath{fill}%
\end{pgfscope}%
\begin{pgfscope}%
\pgfpathrectangle{\pgfqpoint{1.150000in}{0.150000in}}{\pgfqpoint{5.700000in}{5.700000in}}%
\pgfusepath{clip}%
\pgfsetbuttcap%
\pgfsetroundjoin%
\definecolor{currentfill}{rgb}{0.283229,0.120777,0.440584}%
\pgfsetfillcolor{currentfill}%
\pgfsetfillopacity{0.800000}%
\pgfsetlinewidth{0.000000pt}%
\definecolor{currentstroke}{rgb}{0.000000,0.000000,0.000000}%
\pgfsetstrokecolor{currentstroke}%
\pgfsetdash{}{0pt}%
\pgfpathmoveto{\pgfqpoint{3.770766in}{2.301185in}}%
\pgfpathlineto{\pgfqpoint{3.784250in}{2.298353in}}%
\pgfpathlineto{\pgfqpoint{3.797740in}{2.295730in}}%
\pgfpathlineto{\pgfqpoint{3.811236in}{2.293316in}}%
\pgfpathlineto{\pgfqpoint{3.824737in}{2.291109in}}%
\pgfpathlineto{\pgfqpoint{3.832649in}{2.301547in}}%
\pgfpathlineto{\pgfqpoint{3.840556in}{2.311998in}}%
\pgfpathlineto{\pgfqpoint{3.848458in}{2.322464in}}%
\pgfpathlineto{\pgfqpoint{3.856354in}{2.332946in}}%
\pgfpathlineto{\pgfqpoint{3.842861in}{2.335221in}}%
\pgfpathlineto{\pgfqpoint{3.829375in}{2.337703in}}%
\pgfpathlineto{\pgfqpoint{3.815893in}{2.340394in}}%
\pgfpathlineto{\pgfqpoint{3.802418in}{2.343294in}}%
\pgfpathlineto{\pgfqpoint{3.794513in}{2.332733in}}%
\pgfpathlineto{\pgfqpoint{3.786602in}{2.322195in}}%
\pgfpathlineto{\pgfqpoint{3.778687in}{2.311679in}}%
\pgfpathlineto{\pgfqpoint{3.770766in}{2.301185in}}%
\pgfpathclose%
\pgfusepath{fill}%
\end{pgfscope}%
\begin{pgfscope}%
\pgfpathrectangle{\pgfqpoint{1.150000in}{0.150000in}}{\pgfqpoint{5.700000in}{5.700000in}}%
\pgfusepath{clip}%
\pgfsetbuttcap%
\pgfsetroundjoin%
\definecolor{currentfill}{rgb}{0.171176,0.452530,0.557965}%
\pgfsetfillcolor{currentfill}%
\pgfsetfillopacity{0.800000}%
\pgfsetlinewidth{0.000000pt}%
\definecolor{currentstroke}{rgb}{0.000000,0.000000,0.000000}%
\pgfsetstrokecolor{currentstroke}%
\pgfsetdash{}{0pt}%
\pgfpathmoveto{\pgfqpoint{5.310183in}{3.121364in}}%
\pgfpathlineto{\pgfqpoint{5.324211in}{3.125969in}}%
\pgfpathlineto{\pgfqpoint{5.338253in}{3.130748in}}%
\pgfpathlineto{\pgfqpoint{5.352310in}{3.135700in}}%
\pgfpathlineto{\pgfqpoint{5.366382in}{3.140825in}}%
\pgfpathlineto{\pgfqpoint{5.373757in}{3.148912in}}%
\pgfpathlineto{\pgfqpoint{5.381132in}{3.157170in}}%
\pgfpathlineto{\pgfqpoint{5.388505in}{3.165607in}}%
\pgfpathlineto{\pgfqpoint{5.395878in}{3.174229in}}%
\pgfpathlineto{\pgfqpoint{5.381830in}{3.169750in}}%
\pgfpathlineto{\pgfqpoint{5.367797in}{3.165443in}}%
\pgfpathlineto{\pgfqpoint{5.353779in}{3.161310in}}%
\pgfpathlineto{\pgfqpoint{5.339774in}{3.157349in}}%
\pgfpathlineto{\pgfqpoint{5.332377in}{3.148071in}}%
\pgfpathlineto{\pgfqpoint{5.324980in}{3.138985in}}%
\pgfpathlineto{\pgfqpoint{5.317582in}{3.130085in}}%
\pgfpathlineto{\pgfqpoint{5.310183in}{3.121364in}}%
\pgfpathclose%
\pgfusepath{fill}%
\end{pgfscope}%
\begin{pgfscope}%
\pgfpathrectangle{\pgfqpoint{1.150000in}{0.150000in}}{\pgfqpoint{5.700000in}{5.700000in}}%
\pgfusepath{clip}%
\pgfsetbuttcap%
\pgfsetroundjoin%
\definecolor{currentfill}{rgb}{0.282884,0.135920,0.453427}%
\pgfsetfillcolor{currentfill}%
\pgfsetfillopacity{0.800000}%
\pgfsetlinewidth{0.000000pt}%
\definecolor{currentstroke}{rgb}{0.000000,0.000000,0.000000}%
\pgfsetstrokecolor{currentstroke}%
\pgfsetdash{}{0pt}%
\pgfpathmoveto{\pgfqpoint{3.125723in}{2.350199in}}%
\pgfpathlineto{\pgfqpoint{3.139208in}{2.338919in}}%
\pgfpathlineto{\pgfqpoint{3.152691in}{2.327896in}}%
\pgfpathlineto{\pgfqpoint{3.166172in}{2.317127in}}%
\pgfpathlineto{\pgfqpoint{3.179650in}{2.306611in}}%
\pgfpathlineto{\pgfqpoint{3.187784in}{2.315981in}}%
\pgfpathlineto{\pgfqpoint{3.195911in}{2.325429in}}%
\pgfpathlineto{\pgfqpoint{3.204031in}{2.334955in}}%
\pgfpathlineto{\pgfqpoint{3.212144in}{2.344558in}}%
\pgfpathlineto{\pgfqpoint{3.198682in}{2.354982in}}%
\pgfpathlineto{\pgfqpoint{3.185217in}{2.365659in}}%
\pgfpathlineto{\pgfqpoint{3.171751in}{2.376590in}}%
\pgfpathlineto{\pgfqpoint{3.158283in}{2.387778in}}%
\pgfpathlineto{\pgfqpoint{3.150154in}{2.378255in}}%
\pgfpathlineto{\pgfqpoint{3.142017in}{2.368817in}}%
\pgfpathlineto{\pgfqpoint{3.133874in}{2.359465in}}%
\pgfpathlineto{\pgfqpoint{3.125723in}{2.350199in}}%
\pgfpathclose%
\pgfusepath{fill}%
\end{pgfscope}%
\begin{pgfscope}%
\pgfpathrectangle{\pgfqpoint{1.150000in}{0.150000in}}{\pgfqpoint{5.700000in}{5.700000in}}%
\pgfusepath{clip}%
\pgfsetbuttcap%
\pgfsetroundjoin%
\definecolor{currentfill}{rgb}{0.163625,0.471133,0.558148}%
\pgfsetfillcolor{currentfill}%
\pgfsetfillopacity{0.800000}%
\pgfsetlinewidth{0.000000pt}%
\definecolor{currentstroke}{rgb}{0.000000,0.000000,0.000000}%
\pgfsetstrokecolor{currentstroke}%
\pgfsetdash{}{0pt}%
\pgfpathmoveto{\pgfqpoint{5.395878in}{3.174229in}}%
\pgfpathlineto{\pgfqpoint{5.409941in}{3.178880in}}%
\pgfpathlineto{\pgfqpoint{5.424018in}{3.183704in}}%
\pgfpathlineto{\pgfqpoint{5.438111in}{3.188701in}}%
\pgfpathlineto{\pgfqpoint{5.452219in}{3.193870in}}%
\pgfpathlineto{\pgfqpoint{5.459567in}{3.202018in}}%
\pgfpathlineto{\pgfqpoint{5.466914in}{3.210360in}}%
\pgfpathlineto{\pgfqpoint{5.474262in}{3.218902in}}%
\pgfpathlineto{\pgfqpoint{5.481610in}{3.227652in}}%
\pgfpathlineto{\pgfqpoint{5.467528in}{3.223162in}}%
\pgfpathlineto{\pgfqpoint{5.453462in}{3.218844in}}%
\pgfpathlineto{\pgfqpoint{5.439409in}{3.214697in}}%
\pgfpathlineto{\pgfqpoint{5.425372in}{3.210722in}}%
\pgfpathlineto{\pgfqpoint{5.417998in}{3.201283in}}%
\pgfpathlineto{\pgfqpoint{5.410624in}{3.192059in}}%
\pgfpathlineto{\pgfqpoint{5.403251in}{3.183044in}}%
\pgfpathlineto{\pgfqpoint{5.395878in}{3.174229in}}%
\pgfpathclose%
\pgfusepath{fill}%
\end{pgfscope}%
\begin{pgfscope}%
\pgfpathrectangle{\pgfqpoint{1.150000in}{0.150000in}}{\pgfqpoint{5.700000in}{5.700000in}}%
\pgfusepath{clip}%
\pgfsetbuttcap%
\pgfsetroundjoin%
\definecolor{currentfill}{rgb}{0.282910,0.105393,0.426902}%
\pgfsetfillcolor{currentfill}%
\pgfsetfillopacity{0.800000}%
\pgfsetlinewidth{0.000000pt}%
\definecolor{currentstroke}{rgb}{0.000000,0.000000,0.000000}%
\pgfsetstrokecolor{currentstroke}%
\pgfsetdash{}{0pt}%
\pgfpathmoveto{\pgfqpoint{3.319812in}{2.270068in}}%
\pgfpathlineto{\pgfqpoint{3.333270in}{2.261845in}}%
\pgfpathlineto{\pgfqpoint{3.346728in}{2.253860in}}%
\pgfpathlineto{\pgfqpoint{3.360187in}{2.246111in}}%
\pgfpathlineto{\pgfqpoint{3.373646in}{2.238596in}}%
\pgfpathlineto{\pgfqpoint{3.381710in}{2.248489in}}%
\pgfpathlineto{\pgfqpoint{3.389767in}{2.258436in}}%
\pgfpathlineto{\pgfqpoint{3.397818in}{2.268436in}}%
\pgfpathlineto{\pgfqpoint{3.405864in}{2.278489in}}%
\pgfpathlineto{\pgfqpoint{3.392417in}{2.285946in}}%
\pgfpathlineto{\pgfqpoint{3.378972in}{2.293636in}}%
\pgfpathlineto{\pgfqpoint{3.365527in}{2.301562in}}%
\pgfpathlineto{\pgfqpoint{3.352083in}{2.309725in}}%
\pgfpathlineto{\pgfqpoint{3.344025in}{2.299720in}}%
\pgfpathlineto{\pgfqpoint{3.335960in}{2.289775in}}%
\pgfpathlineto{\pgfqpoint{3.327889in}{2.279891in}}%
\pgfpathlineto{\pgfqpoint{3.319812in}{2.270068in}}%
\pgfpathclose%
\pgfusepath{fill}%
\end{pgfscope}%
\begin{pgfscope}%
\pgfpathrectangle{\pgfqpoint{1.150000in}{0.150000in}}{\pgfqpoint{5.700000in}{5.700000in}}%
\pgfusepath{clip}%
\pgfsetbuttcap%
\pgfsetroundjoin%
\definecolor{currentfill}{rgb}{0.283091,0.110553,0.431554}%
\pgfsetfillcolor{currentfill}%
\pgfsetfillopacity{0.800000}%
\pgfsetlinewidth{0.000000pt}%
\definecolor{currentstroke}{rgb}{0.000000,0.000000,0.000000}%
\pgfsetstrokecolor{currentstroke}%
\pgfsetdash{}{0pt}%
\pgfpathmoveto{\pgfqpoint{3.685107in}{2.272699in}}%
\pgfpathlineto{\pgfqpoint{3.698581in}{2.269056in}}%
\pgfpathlineto{\pgfqpoint{3.712059in}{2.265627in}}%
\pgfpathlineto{\pgfqpoint{3.725543in}{2.262409in}}%
\pgfpathlineto{\pgfqpoint{3.739031in}{2.259403in}}%
\pgfpathlineto{\pgfqpoint{3.746973in}{2.269821in}}%
\pgfpathlineto{\pgfqpoint{3.754909in}{2.280257in}}%
\pgfpathlineto{\pgfqpoint{3.762840in}{2.290712in}}%
\pgfpathlineto{\pgfqpoint{3.770766in}{2.301185in}}%
\pgfpathlineto{\pgfqpoint{3.757287in}{2.304228in}}%
\pgfpathlineto{\pgfqpoint{3.743812in}{2.307482in}}%
\pgfpathlineto{\pgfqpoint{3.730343in}{2.310948in}}%
\pgfpathlineto{\pgfqpoint{3.716879in}{2.314628in}}%
\pgfpathlineto{\pgfqpoint{3.708944in}{2.304106in}}%
\pgfpathlineto{\pgfqpoint{3.701003in}{2.293611in}}%
\pgfpathlineto{\pgfqpoint{3.693058in}{2.283143in}}%
\pgfpathlineto{\pgfqpoint{3.685107in}{2.272699in}}%
\pgfpathclose%
\pgfusepath{fill}%
\end{pgfscope}%
\begin{pgfscope}%
\pgfpathrectangle{\pgfqpoint{1.150000in}{0.150000in}}{\pgfqpoint{5.700000in}{5.700000in}}%
\pgfusepath{clip}%
\pgfsetbuttcap%
\pgfsetroundjoin%
\definecolor{currentfill}{rgb}{0.282656,0.100196,0.422160}%
\pgfsetfillcolor{currentfill}%
\pgfsetfillopacity{0.800000}%
\pgfsetlinewidth{0.000000pt}%
\definecolor{currentstroke}{rgb}{0.000000,0.000000,0.000000}%
\pgfsetstrokecolor{currentstroke}%
\pgfsetdash{}{0pt}%
\pgfpathmoveto{\pgfqpoint{3.459664in}{2.250978in}}%
\pgfpathlineto{\pgfqpoint{3.473119in}{2.244673in}}%
\pgfpathlineto{\pgfqpoint{3.486576in}{2.238594in}}%
\pgfpathlineto{\pgfqpoint{3.500035in}{2.232739in}}%
\pgfpathlineto{\pgfqpoint{3.513497in}{2.227109in}}%
\pgfpathlineto{\pgfqpoint{3.521512in}{2.237293in}}%
\pgfpathlineto{\pgfqpoint{3.529522in}{2.247514in}}%
\pgfpathlineto{\pgfqpoint{3.537526in}{2.257772in}}%
\pgfpathlineto{\pgfqpoint{3.545525in}{2.268068in}}%
\pgfpathlineto{\pgfqpoint{3.532074in}{2.273672in}}%
\pgfpathlineto{\pgfqpoint{3.518626in}{2.279499in}}%
\pgfpathlineto{\pgfqpoint{3.505181in}{2.285551in}}%
\pgfpathlineto{\pgfqpoint{3.491738in}{2.291830in}}%
\pgfpathlineto{\pgfqpoint{3.483728in}{2.281549in}}%
\pgfpathlineto{\pgfqpoint{3.475712in}{2.271313in}}%
\pgfpathlineto{\pgfqpoint{3.467691in}{2.261123in}}%
\pgfpathlineto{\pgfqpoint{3.459664in}{2.250978in}}%
\pgfpathclose%
\pgfusepath{fill}%
\end{pgfscope}%
\begin{pgfscope}%
\pgfpathrectangle{\pgfqpoint{1.150000in}{0.150000in}}{\pgfqpoint{5.700000in}{5.700000in}}%
\pgfusepath{clip}%
\pgfsetbuttcap%
\pgfsetroundjoin%
\definecolor{currentfill}{rgb}{0.206756,0.371758,0.553117}%
\pgfsetfillcolor{currentfill}%
\pgfsetfillopacity{0.800000}%
\pgfsetlinewidth{0.000000pt}%
\definecolor{currentstroke}{rgb}{0.000000,0.000000,0.000000}%
\pgfsetstrokecolor{currentstroke}%
\pgfsetdash{}{0pt}%
\pgfpathmoveto{\pgfqpoint{2.636606in}{2.945384in}}%
\pgfpathlineto{\pgfqpoint{2.650348in}{2.923342in}}%
\pgfpathlineto{\pgfqpoint{2.664077in}{2.901649in}}%
\pgfpathlineto{\pgfqpoint{2.677794in}{2.880301in}}%
\pgfpathlineto{\pgfqpoint{2.691500in}{2.859296in}}%
\pgfpathlineto{\pgfqpoint{2.699817in}{2.867698in}}%
\pgfpathlineto{\pgfqpoint{2.708124in}{2.876251in}}%
\pgfpathlineto{\pgfqpoint{2.716421in}{2.884953in}}%
\pgfpathlineto{\pgfqpoint{2.724709in}{2.893805in}}%
\pgfpathlineto{\pgfqpoint{2.711028in}{2.914710in}}%
\pgfpathlineto{\pgfqpoint{2.697335in}{2.935957in}}%
\pgfpathlineto{\pgfqpoint{2.683630in}{2.957549in}}%
\pgfpathlineto{\pgfqpoint{2.669914in}{2.979489in}}%
\pgfpathlineto{\pgfqpoint{2.661602in}{2.970726in}}%
\pgfpathlineto{\pgfqpoint{2.653280in}{2.962120in}}%
\pgfpathlineto{\pgfqpoint{2.644948in}{2.953673in}}%
\pgfpathlineto{\pgfqpoint{2.636606in}{2.945384in}}%
\pgfpathclose%
\pgfusepath{fill}%
\end{pgfscope}%
\begin{pgfscope}%
\pgfpathrectangle{\pgfqpoint{1.150000in}{0.150000in}}{\pgfqpoint{5.700000in}{5.700000in}}%
\pgfusepath{clip}%
\pgfsetbuttcap%
\pgfsetroundjoin%
\definecolor{currentfill}{rgb}{0.156270,0.489624,0.557936}%
\pgfsetfillcolor{currentfill}%
\pgfsetfillopacity{0.800000}%
\pgfsetlinewidth{0.000000pt}%
\definecolor{currentstroke}{rgb}{0.000000,0.000000,0.000000}%
\pgfsetstrokecolor{currentstroke}%
\pgfsetdash{}{0pt}%
\pgfpathmoveto{\pgfqpoint{5.481610in}{3.227652in}}%
\pgfpathlineto{\pgfqpoint{5.495707in}{3.232314in}}%
\pgfpathlineto{\pgfqpoint{5.509819in}{3.237147in}}%
\pgfpathlineto{\pgfqpoint{5.523947in}{3.242151in}}%
\pgfpathlineto{\pgfqpoint{5.538090in}{3.247327in}}%
\pgfpathlineto{\pgfqpoint{5.545412in}{3.255594in}}%
\pgfpathlineto{\pgfqpoint{5.552734in}{3.264078in}}%
\pgfpathlineto{\pgfqpoint{5.560059in}{3.272787in}}%
\pgfpathlineto{\pgfqpoint{5.567384in}{3.281727in}}%
\pgfpathlineto{\pgfqpoint{5.553269in}{3.277263in}}%
\pgfpathlineto{\pgfqpoint{5.539169in}{3.272968in}}%
\pgfpathlineto{\pgfqpoint{5.525085in}{3.268845in}}%
\pgfpathlineto{\pgfqpoint{5.511015in}{3.264892in}}%
\pgfpathlineto{\pgfqpoint{5.503661in}{3.255231in}}%
\pgfpathlineto{\pgfqpoint{5.496310in}{3.245809in}}%
\pgfpathlineto{\pgfqpoint{5.488959in}{3.236619in}}%
\pgfpathlineto{\pgfqpoint{5.481610in}{3.227652in}}%
\pgfpathclose%
\pgfusepath{fill}%
\end{pgfscope}%
\begin{pgfscope}%
\pgfpathrectangle{\pgfqpoint{1.150000in}{0.150000in}}{\pgfqpoint{5.700000in}{5.700000in}}%
\pgfusepath{clip}%
\pgfsetbuttcap%
\pgfsetroundjoin%
\definecolor{currentfill}{rgb}{0.270595,0.214069,0.507052}%
\pgfsetfillcolor{currentfill}%
\pgfsetfillopacity{0.800000}%
\pgfsetlinewidth{0.000000pt}%
\definecolor{currentstroke}{rgb}{0.000000,0.000000,0.000000}%
\pgfsetstrokecolor{currentstroke}%
\pgfsetdash{}{0pt}%
\pgfpathmoveto{\pgfqpoint{4.252708in}{2.489332in}}%
\pgfpathlineto{\pgfqpoint{4.266330in}{2.490658in}}%
\pgfpathlineto{\pgfqpoint{4.279962in}{2.492176in}}%
\pgfpathlineto{\pgfqpoint{4.293604in}{2.493886in}}%
\pgfpathlineto{\pgfqpoint{4.307255in}{2.495788in}}%
\pgfpathlineto{\pgfqpoint{4.315017in}{2.505526in}}%
\pgfpathlineto{\pgfqpoint{4.322773in}{2.515269in}}%
\pgfpathlineto{\pgfqpoint{4.330524in}{2.525020in}}%
\pgfpathlineto{\pgfqpoint{4.338271in}{2.534783in}}%
\pgfpathlineto{\pgfqpoint{4.324628in}{2.533109in}}%
\pgfpathlineto{\pgfqpoint{4.310995in}{2.531626in}}%
\pgfpathlineto{\pgfqpoint{4.297372in}{2.530335in}}%
\pgfpathlineto{\pgfqpoint{4.283758in}{2.529236in}}%
\pgfpathlineto{\pgfqpoint{4.276003in}{2.519235in}}%
\pgfpathlineto{\pgfqpoint{4.268243in}{2.509252in}}%
\pgfpathlineto{\pgfqpoint{4.260478in}{2.499285in}}%
\pgfpathlineto{\pgfqpoint{4.252708in}{2.489332in}}%
\pgfpathclose%
\pgfusepath{fill}%
\end{pgfscope}%
\begin{pgfscope}%
\pgfpathrectangle{\pgfqpoint{1.150000in}{0.150000in}}{\pgfqpoint{5.700000in}{5.700000in}}%
\pgfusepath{clip}%
\pgfsetbuttcap%
\pgfsetroundjoin%
\definecolor{currentfill}{rgb}{0.265145,0.232956,0.516599}%
\pgfsetfillcolor{currentfill}%
\pgfsetfillopacity{0.800000}%
\pgfsetlinewidth{0.000000pt}%
\definecolor{currentstroke}{rgb}{0.000000,0.000000,0.000000}%
\pgfsetstrokecolor{currentstroke}%
\pgfsetdash{}{0pt}%
\pgfpathmoveto{\pgfqpoint{4.338271in}{2.534783in}}%
\pgfpathlineto{\pgfqpoint{4.351923in}{2.536648in}}%
\pgfpathlineto{\pgfqpoint{4.365586in}{2.538704in}}%
\pgfpathlineto{\pgfqpoint{4.379258in}{2.540949in}}%
\pgfpathlineto{\pgfqpoint{4.392942in}{2.543385in}}%
\pgfpathlineto{\pgfqpoint{4.400674in}{2.552914in}}%
\pgfpathlineto{\pgfqpoint{4.408401in}{2.562452in}}%
\pgfpathlineto{\pgfqpoint{4.416123in}{2.572003in}}%
\pgfpathlineto{\pgfqpoint{4.423840in}{2.581570in}}%
\pgfpathlineto{\pgfqpoint{4.410166in}{2.579395in}}%
\pgfpathlineto{\pgfqpoint{4.396503in}{2.577409in}}%
\pgfpathlineto{\pgfqpoint{4.382850in}{2.575613in}}%
\pgfpathlineto{\pgfqpoint{4.369206in}{2.574007in}}%
\pgfpathlineto{\pgfqpoint{4.361480in}{2.564168in}}%
\pgfpathlineto{\pgfqpoint{4.353748in}{2.554354in}}%
\pgfpathlineto{\pgfqpoint{4.346012in}{2.544560in}}%
\pgfpathlineto{\pgfqpoint{4.338271in}{2.534783in}}%
\pgfpathclose%
\pgfusepath{fill}%
\end{pgfscope}%
\begin{pgfscope}%
\pgfpathrectangle{\pgfqpoint{1.150000in}{0.150000in}}{\pgfqpoint{5.700000in}{5.700000in}}%
\pgfusepath{clip}%
\pgfsetbuttcap%
\pgfsetroundjoin%
\definecolor{currentfill}{rgb}{0.275191,0.194905,0.496005}%
\pgfsetfillcolor{currentfill}%
\pgfsetfillopacity{0.800000}%
\pgfsetlinewidth{0.000000pt}%
\definecolor{currentstroke}{rgb}{0.000000,0.000000,0.000000}%
\pgfsetstrokecolor{currentstroke}%
\pgfsetdash{}{0pt}%
\pgfpathmoveto{\pgfqpoint{4.167147in}{2.445450in}}%
\pgfpathlineto{\pgfqpoint{4.180741in}{2.446196in}}%
\pgfpathlineto{\pgfqpoint{4.194344in}{2.447136in}}%
\pgfpathlineto{\pgfqpoint{4.207957in}{2.448270in}}%
\pgfpathlineto{\pgfqpoint{4.221578in}{2.449599in}}%
\pgfpathlineto{\pgfqpoint{4.229368in}{2.459525in}}%
\pgfpathlineto{\pgfqpoint{4.237153in}{2.469454in}}%
\pgfpathlineto{\pgfqpoint{4.244933in}{2.479389in}}%
\pgfpathlineto{\pgfqpoint{4.252708in}{2.489332in}}%
\pgfpathlineto{\pgfqpoint{4.239095in}{2.488199in}}%
\pgfpathlineto{\pgfqpoint{4.225491in}{2.487260in}}%
\pgfpathlineto{\pgfqpoint{4.211897in}{2.486516in}}%
\pgfpathlineto{\pgfqpoint{4.198311in}{2.485966in}}%
\pgfpathlineto{\pgfqpoint{4.190527in}{2.475816in}}%
\pgfpathlineto{\pgfqpoint{4.182739in}{2.465681in}}%
\pgfpathlineto{\pgfqpoint{4.174945in}{2.455561in}}%
\pgfpathlineto{\pgfqpoint{4.167147in}{2.445450in}}%
\pgfpathclose%
\pgfusepath{fill}%
\end{pgfscope}%
\begin{pgfscope}%
\pgfpathrectangle{\pgfqpoint{1.150000in}{0.150000in}}{\pgfqpoint{5.700000in}{5.700000in}}%
\pgfusepath{clip}%
\pgfsetbuttcap%
\pgfsetroundjoin%
\definecolor{currentfill}{rgb}{0.149039,0.508051,0.557250}%
\pgfsetfillcolor{currentfill}%
\pgfsetfillopacity{0.800000}%
\pgfsetlinewidth{0.000000pt}%
\definecolor{currentstroke}{rgb}{0.000000,0.000000,0.000000}%
\pgfsetstrokecolor{currentstroke}%
\pgfsetdash{}{0pt}%
\pgfpathmoveto{\pgfqpoint{5.567384in}{3.281727in}}%
\pgfpathlineto{\pgfqpoint{5.581515in}{3.286363in}}%
\pgfpathlineto{\pgfqpoint{5.595660in}{3.291168in}}%
\pgfpathlineto{\pgfqpoint{5.609822in}{3.296144in}}%
\pgfpathlineto{\pgfqpoint{5.623999in}{3.301291in}}%
\pgfpathlineto{\pgfqpoint{5.631297in}{3.309741in}}%
\pgfpathlineto{\pgfqpoint{5.638598in}{3.318432in}}%
\pgfpathlineto{\pgfqpoint{5.645901in}{3.327372in}}%
\pgfpathlineto{\pgfqpoint{5.653207in}{3.336571in}}%
\pgfpathlineto{\pgfqpoint{5.639060in}{3.332167in}}%
\pgfpathlineto{\pgfqpoint{5.624928in}{3.327934in}}%
\pgfpathlineto{\pgfqpoint{5.610811in}{3.323869in}}%
\pgfpathlineto{\pgfqpoint{5.596710in}{3.319975in}}%
\pgfpathlineto{\pgfqpoint{5.589374in}{3.310024in}}%
\pgfpathlineto{\pgfqpoint{5.582042in}{3.300338in}}%
\pgfpathlineto{\pgfqpoint{5.574712in}{3.290908in}}%
\pgfpathlineto{\pgfqpoint{5.567384in}{3.281727in}}%
\pgfpathclose%
\pgfusepath{fill}%
\end{pgfscope}%
\begin{pgfscope}%
\pgfpathrectangle{\pgfqpoint{1.150000in}{0.150000in}}{\pgfqpoint{5.700000in}{5.700000in}}%
\pgfusepath{clip}%
\pgfsetbuttcap%
\pgfsetroundjoin%
\definecolor{currentfill}{rgb}{0.283229,0.120777,0.440584}%
\pgfsetfillcolor{currentfill}%
\pgfsetfillopacity{0.800000}%
\pgfsetlinewidth{0.000000pt}%
\definecolor{currentstroke}{rgb}{0.000000,0.000000,0.000000}%
\pgfsetstrokecolor{currentstroke}%
\pgfsetdash{}{0pt}%
\pgfpathmoveto{\pgfqpoint{3.179650in}{2.306611in}}%
\pgfpathlineto{\pgfqpoint{3.193128in}{2.296346in}}%
\pgfpathlineto{\pgfqpoint{3.206603in}{2.286331in}}%
\pgfpathlineto{\pgfqpoint{3.220078in}{2.276564in}}%
\pgfpathlineto{\pgfqpoint{3.233551in}{2.267044in}}%
\pgfpathlineto{\pgfqpoint{3.241669in}{2.276517in}}%
\pgfpathlineto{\pgfqpoint{3.249780in}{2.286060in}}%
\pgfpathlineto{\pgfqpoint{3.257885in}{2.295673in}}%
\pgfpathlineto{\pgfqpoint{3.265983in}{2.305356in}}%
\pgfpathlineto{\pgfqpoint{3.252524in}{2.314785in}}%
\pgfpathlineto{\pgfqpoint{3.239065in}{2.324461in}}%
\pgfpathlineto{\pgfqpoint{3.225605in}{2.334385in}}%
\pgfpathlineto{\pgfqpoint{3.212144in}{2.344558in}}%
\pgfpathlineto{\pgfqpoint{3.204031in}{2.334955in}}%
\pgfpathlineto{\pgfqpoint{3.195911in}{2.325429in}}%
\pgfpathlineto{\pgfqpoint{3.187784in}{2.315981in}}%
\pgfpathlineto{\pgfqpoint{3.179650in}{2.306611in}}%
\pgfpathclose%
\pgfusepath{fill}%
\end{pgfscope}%
\begin{pgfscope}%
\pgfpathrectangle{\pgfqpoint{1.150000in}{0.150000in}}{\pgfqpoint{5.700000in}{5.700000in}}%
\pgfusepath{clip}%
\pgfsetbuttcap%
\pgfsetroundjoin%
\definecolor{currentfill}{rgb}{0.258965,0.251537,0.524736}%
\pgfsetfillcolor{currentfill}%
\pgfsetfillopacity{0.800000}%
\pgfsetlinewidth{0.000000pt}%
\definecolor{currentstroke}{rgb}{0.000000,0.000000,0.000000}%
\pgfsetstrokecolor{currentstroke}%
\pgfsetdash{}{0pt}%
\pgfpathmoveto{\pgfqpoint{4.423840in}{2.581570in}}%
\pgfpathlineto{\pgfqpoint{4.437525in}{2.583934in}}%
\pgfpathlineto{\pgfqpoint{4.451220in}{2.586487in}}%
\pgfpathlineto{\pgfqpoint{4.464926in}{2.589228in}}%
\pgfpathlineto{\pgfqpoint{4.478643in}{2.592156in}}%
\pgfpathlineto{\pgfqpoint{4.486345in}{2.601462in}}%
\pgfpathlineto{\pgfqpoint{4.494043in}{2.610783in}}%
\pgfpathlineto{\pgfqpoint{4.501735in}{2.620122in}}%
\pgfpathlineto{\pgfqpoint{4.509422in}{2.629483in}}%
\pgfpathlineto{\pgfqpoint{4.495716in}{2.626847in}}%
\pgfpathlineto{\pgfqpoint{4.482020in}{2.624398in}}%
\pgfpathlineto{\pgfqpoint{4.468334in}{2.622137in}}%
\pgfpathlineto{\pgfqpoint{4.454660in}{2.620064in}}%
\pgfpathlineto{\pgfqpoint{4.446962in}{2.610399in}}%
\pgfpathlineto{\pgfqpoint{4.439260in}{2.600765in}}%
\pgfpathlineto{\pgfqpoint{4.431553in}{2.591156in}}%
\pgfpathlineto{\pgfqpoint{4.423840in}{2.581570in}}%
\pgfpathclose%
\pgfusepath{fill}%
\end{pgfscope}%
\begin{pgfscope}%
\pgfpathrectangle{\pgfqpoint{1.150000in}{0.150000in}}{\pgfqpoint{5.700000in}{5.700000in}}%
\pgfusepath{clip}%
\pgfsetbuttcap%
\pgfsetroundjoin%
\definecolor{currentfill}{rgb}{0.278826,0.175490,0.483397}%
\pgfsetfillcolor{currentfill}%
\pgfsetfillopacity{0.800000}%
\pgfsetlinewidth{0.000000pt}%
\definecolor{currentstroke}{rgb}{0.000000,0.000000,0.000000}%
\pgfsetstrokecolor{currentstroke}%
\pgfsetdash{}{0pt}%
\pgfpathmoveto{\pgfqpoint{4.081580in}{2.403395in}}%
\pgfpathlineto{\pgfqpoint{4.095148in}{2.403518in}}%
\pgfpathlineto{\pgfqpoint{4.108725in}{2.403839in}}%
\pgfpathlineto{\pgfqpoint{4.122310in}{2.404356in}}%
\pgfpathlineto{\pgfqpoint{4.135903in}{2.405070in}}%
\pgfpathlineto{\pgfqpoint{4.143722in}{2.415160in}}%
\pgfpathlineto{\pgfqpoint{4.151535in}{2.425252in}}%
\pgfpathlineto{\pgfqpoint{4.159344in}{2.435348in}}%
\pgfpathlineto{\pgfqpoint{4.167147in}{2.445450in}}%
\pgfpathlineto{\pgfqpoint{4.153562in}{2.444900in}}%
\pgfpathlineto{\pgfqpoint{4.139985in}{2.444547in}}%
\pgfpathlineto{\pgfqpoint{4.126416in}{2.444390in}}%
\pgfpathlineto{\pgfqpoint{4.112856in}{2.444430in}}%
\pgfpathlineto{\pgfqpoint{4.105045in}{2.434152in}}%
\pgfpathlineto{\pgfqpoint{4.097228in}{2.423889in}}%
\pgfpathlineto{\pgfqpoint{4.089406in}{2.413637in}}%
\pgfpathlineto{\pgfqpoint{4.081580in}{2.403395in}}%
\pgfpathclose%
\pgfusepath{fill}%
\end{pgfscope}%
\begin{pgfscope}%
\pgfpathrectangle{\pgfqpoint{1.150000in}{0.150000in}}{\pgfqpoint{5.700000in}{5.700000in}}%
\pgfusepath{clip}%
\pgfsetbuttcap%
\pgfsetroundjoin%
\definecolor{currentfill}{rgb}{0.250425,0.274290,0.533103}%
\pgfsetfillcolor{currentfill}%
\pgfsetfillopacity{0.800000}%
\pgfsetlinewidth{0.000000pt}%
\definecolor{currentstroke}{rgb}{0.000000,0.000000,0.000000}%
\pgfsetstrokecolor{currentstroke}%
\pgfsetdash{}{0pt}%
\pgfpathmoveto{\pgfqpoint{4.509422in}{2.629483in}}%
\pgfpathlineto{\pgfqpoint{4.523140in}{2.632306in}}%
\pgfpathlineto{\pgfqpoint{4.536869in}{2.635316in}}%
\pgfpathlineto{\pgfqpoint{4.550610in}{2.638512in}}%
\pgfpathlineto{\pgfqpoint{4.564362in}{2.641894in}}%
\pgfpathlineto{\pgfqpoint{4.572034in}{2.650968in}}%
\pgfpathlineto{\pgfqpoint{4.579701in}{2.660064in}}%
\pgfpathlineto{\pgfqpoint{4.587363in}{2.669184in}}%
\pgfpathlineto{\pgfqpoint{4.595020in}{2.678334in}}%
\pgfpathlineto{\pgfqpoint{4.581279in}{2.675277in}}%
\pgfpathlineto{\pgfqpoint{4.567550in}{2.672405in}}%
\pgfpathlineto{\pgfqpoint{4.553831in}{2.669719in}}%
\pgfpathlineto{\pgfqpoint{4.540124in}{2.667219in}}%
\pgfpathlineto{\pgfqpoint{4.532456in}{2.657733in}}%
\pgfpathlineto{\pgfqpoint{4.524783in}{2.648285in}}%
\pgfpathlineto{\pgfqpoint{4.517105in}{2.638869in}}%
\pgfpathlineto{\pgfqpoint{4.509422in}{2.629483in}}%
\pgfpathclose%
\pgfusepath{fill}%
\end{pgfscope}%
\begin{pgfscope}%
\pgfpathrectangle{\pgfqpoint{1.150000in}{0.150000in}}{\pgfqpoint{5.700000in}{5.700000in}}%
\pgfusepath{clip}%
\pgfsetbuttcap%
\pgfsetroundjoin%
\definecolor{currentfill}{rgb}{0.282656,0.100196,0.422160}%
\pgfsetfillcolor{currentfill}%
\pgfsetfillopacity{0.800000}%
\pgfsetlinewidth{0.000000pt}%
\definecolor{currentstroke}{rgb}{0.000000,0.000000,0.000000}%
\pgfsetstrokecolor{currentstroke}%
\pgfsetdash{}{0pt}%
\pgfpathmoveto{\pgfqpoint{3.599358in}{2.247871in}}%
\pgfpathlineto{\pgfqpoint{3.612825in}{2.243370in}}%
\pgfpathlineto{\pgfqpoint{3.626296in}{2.239087in}}%
\pgfpathlineto{\pgfqpoint{3.639771in}{2.235019in}}%
\pgfpathlineto{\pgfqpoint{3.653251in}{2.231167in}}%
\pgfpathlineto{\pgfqpoint{3.661223in}{2.241515in}}%
\pgfpathlineto{\pgfqpoint{3.669189in}{2.251886in}}%
\pgfpathlineto{\pgfqpoint{3.677151in}{2.262281in}}%
\pgfpathlineto{\pgfqpoint{3.685107in}{2.272699in}}%
\pgfpathlineto{\pgfqpoint{3.671638in}{2.276557in}}%
\pgfpathlineto{\pgfqpoint{3.658173in}{2.280629in}}%
\pgfpathlineto{\pgfqpoint{3.644712in}{2.284917in}}%
\pgfpathlineto{\pgfqpoint{3.631255in}{2.289423in}}%
\pgfpathlineto{\pgfqpoint{3.623289in}{2.278988in}}%
\pgfpathlineto{\pgfqpoint{3.615317in}{2.268585in}}%
\pgfpathlineto{\pgfqpoint{3.607340in}{2.258212in}}%
\pgfpathlineto{\pgfqpoint{3.599358in}{2.247871in}}%
\pgfpathclose%
\pgfusepath{fill}%
\end{pgfscope}%
\begin{pgfscope}%
\pgfpathrectangle{\pgfqpoint{1.150000in}{0.150000in}}{\pgfqpoint{5.700000in}{5.700000in}}%
\pgfusepath{clip}%
\pgfsetbuttcap%
\pgfsetroundjoin%
\definecolor{currentfill}{rgb}{0.243113,0.292092,0.538516}%
\pgfsetfillcolor{currentfill}%
\pgfsetfillopacity{0.800000}%
\pgfsetlinewidth{0.000000pt}%
\definecolor{currentstroke}{rgb}{0.000000,0.000000,0.000000}%
\pgfsetstrokecolor{currentstroke}%
\pgfsetdash{}{0pt}%
\pgfpathmoveto{\pgfqpoint{4.595020in}{2.678334in}}%
\pgfpathlineto{\pgfqpoint{4.608773in}{2.681577in}}%
\pgfpathlineto{\pgfqpoint{4.622537in}{2.685005in}}%
\pgfpathlineto{\pgfqpoint{4.636313in}{2.688617in}}%
\pgfpathlineto{\pgfqpoint{4.650102in}{2.692414in}}%
\pgfpathlineto{\pgfqpoint{4.657743in}{2.701252in}}%
\pgfpathlineto{\pgfqpoint{4.665379in}{2.710120in}}%
\pgfpathlineto{\pgfqpoint{4.673010in}{2.719021in}}%
\pgfpathlineto{\pgfqpoint{4.680637in}{2.727961in}}%
\pgfpathlineto{\pgfqpoint{4.666860in}{2.724521in}}%
\pgfpathlineto{\pgfqpoint{4.653096in}{2.721265in}}%
\pgfpathlineto{\pgfqpoint{4.639343in}{2.718193in}}%
\pgfpathlineto{\pgfqpoint{4.625602in}{2.715306in}}%
\pgfpathlineto{\pgfqpoint{4.617964in}{2.705999in}}%
\pgfpathlineto{\pgfqpoint{4.610321in}{2.696738in}}%
\pgfpathlineto{\pgfqpoint{4.602673in}{2.687517in}}%
\pgfpathlineto{\pgfqpoint{4.595020in}{2.678334in}}%
\pgfpathclose%
\pgfusepath{fill}%
\end{pgfscope}%
\begin{pgfscope}%
\pgfpathrectangle{\pgfqpoint{1.150000in}{0.150000in}}{\pgfqpoint{5.700000in}{5.700000in}}%
\pgfusepath{clip}%
\pgfsetbuttcap%
\pgfsetroundjoin%
\definecolor{currentfill}{rgb}{0.280868,0.160771,0.472899}%
\pgfsetfillcolor{currentfill}%
\pgfsetfillopacity{0.800000}%
\pgfsetlinewidth{0.000000pt}%
\definecolor{currentstroke}{rgb}{0.000000,0.000000,0.000000}%
\pgfsetstrokecolor{currentstroke}%
\pgfsetdash{}{0pt}%
\pgfpathmoveto{\pgfqpoint{3.995997in}{2.363446in}}%
\pgfpathlineto{\pgfqpoint{4.009542in}{2.362905in}}%
\pgfpathlineto{\pgfqpoint{4.023095in}{2.362564in}}%
\pgfpathlineto{\pgfqpoint{4.036655in}{2.362423in}}%
\pgfpathlineto{\pgfqpoint{4.050223in}{2.362480in}}%
\pgfpathlineto{\pgfqpoint{4.058070in}{2.372704in}}%
\pgfpathlineto{\pgfqpoint{4.065912in}{2.382930in}}%
\pgfpathlineto{\pgfqpoint{4.073748in}{2.393160in}}%
\pgfpathlineto{\pgfqpoint{4.081580in}{2.403395in}}%
\pgfpathlineto{\pgfqpoint{4.068020in}{2.403470in}}%
\pgfpathlineto{\pgfqpoint{4.054467in}{2.403743in}}%
\pgfpathlineto{\pgfqpoint{4.040923in}{2.404216in}}%
\pgfpathlineto{\pgfqpoint{4.027386in}{2.404888in}}%
\pgfpathlineto{\pgfqpoint{4.019546in}{2.394510in}}%
\pgfpathlineto{\pgfqpoint{4.011701in}{2.384144in}}%
\pgfpathlineto{\pgfqpoint{4.003852in}{2.373790in}}%
\pgfpathlineto{\pgfqpoint{3.995997in}{2.363446in}}%
\pgfpathclose%
\pgfusepath{fill}%
\end{pgfscope}%
\begin{pgfscope}%
\pgfpathrectangle{\pgfqpoint{1.150000in}{0.150000in}}{\pgfqpoint{5.700000in}{5.700000in}}%
\pgfusepath{clip}%
\pgfsetbuttcap%
\pgfsetroundjoin%
\definecolor{currentfill}{rgb}{0.269308,0.218818,0.509577}%
\pgfsetfillcolor{currentfill}%
\pgfsetfillopacity{0.800000}%
\pgfsetlinewidth{0.000000pt}%
\definecolor{currentstroke}{rgb}{0.000000,0.000000,0.000000}%
\pgfsetstrokecolor{currentstroke}%
\pgfsetdash{}{0pt}%
\pgfpathmoveto{\pgfqpoint{2.876467in}{2.532990in}}%
\pgfpathlineto{\pgfqpoint{2.890043in}{2.517205in}}%
\pgfpathlineto{\pgfqpoint{2.903613in}{2.501710in}}%
\pgfpathlineto{\pgfqpoint{2.917176in}{2.486504in}}%
\pgfpathlineto{\pgfqpoint{2.930733in}{2.471583in}}%
\pgfpathlineto{\pgfqpoint{2.938972in}{2.480086in}}%
\pgfpathlineto{\pgfqpoint{2.947203in}{2.488702in}}%
\pgfpathlineto{\pgfqpoint{2.955425in}{2.497430in}}%
\pgfpathlineto{\pgfqpoint{2.963639in}{2.506269in}}%
\pgfpathlineto{\pgfqpoint{2.950103in}{2.521063in}}%
\pgfpathlineto{\pgfqpoint{2.936560in}{2.536142in}}%
\pgfpathlineto{\pgfqpoint{2.923012in}{2.551509in}}%
\pgfpathlineto{\pgfqpoint{2.909458in}{2.567166in}}%
\pgfpathlineto{\pgfqpoint{2.901223in}{2.558442in}}%
\pgfpathlineto{\pgfqpoint{2.892980in}{2.549838in}}%
\pgfpathlineto{\pgfqpoint{2.884728in}{2.541353in}}%
\pgfpathlineto{\pgfqpoint{2.876467in}{2.532990in}}%
\pgfpathclose%
\pgfusepath{fill}%
\end{pgfscope}%
\begin{pgfscope}%
\pgfpathrectangle{\pgfqpoint{1.150000in}{0.150000in}}{\pgfqpoint{5.700000in}{5.700000in}}%
\pgfusepath{clip}%
\pgfsetbuttcap%
\pgfsetroundjoin%
\definecolor{currentfill}{rgb}{0.260571,0.246922,0.522828}%
\pgfsetfillcolor{currentfill}%
\pgfsetfillopacity{0.800000}%
\pgfsetlinewidth{0.000000pt}%
\definecolor{currentstroke}{rgb}{0.000000,0.000000,0.000000}%
\pgfsetstrokecolor{currentstroke}%
\pgfsetdash{}{0pt}%
\pgfpathmoveto{\pgfqpoint{2.822093in}{2.599084in}}%
\pgfpathlineto{\pgfqpoint{2.835698in}{2.582112in}}%
\pgfpathlineto{\pgfqpoint{2.849295in}{2.565441in}}%
\pgfpathlineto{\pgfqpoint{2.862885in}{2.549067in}}%
\pgfpathlineto{\pgfqpoint{2.876467in}{2.532990in}}%
\pgfpathlineto{\pgfqpoint{2.884728in}{2.541353in}}%
\pgfpathlineto{\pgfqpoint{2.892980in}{2.549838in}}%
\pgfpathlineto{\pgfqpoint{2.901223in}{2.558442in}}%
\pgfpathlineto{\pgfqpoint{2.909458in}{2.567166in}}%
\pgfpathlineto{\pgfqpoint{2.895897in}{2.583116in}}%
\pgfpathlineto{\pgfqpoint{2.882329in}{2.599361in}}%
\pgfpathlineto{\pgfqpoint{2.868754in}{2.615904in}}%
\pgfpathlineto{\pgfqpoint{2.855171in}{2.632747in}}%
\pgfpathlineto{\pgfqpoint{2.846915in}{2.624139in}}%
\pgfpathlineto{\pgfqpoint{2.838650in}{2.615659in}}%
\pgfpathlineto{\pgfqpoint{2.830376in}{2.607307in}}%
\pgfpathlineto{\pgfqpoint{2.822093in}{2.599084in}}%
\pgfpathclose%
\pgfusepath{fill}%
\end{pgfscope}%
\begin{pgfscope}%
\pgfpathrectangle{\pgfqpoint{1.150000in}{0.150000in}}{\pgfqpoint{5.700000in}{5.700000in}}%
\pgfusepath{clip}%
\pgfsetbuttcap%
\pgfsetroundjoin%
\definecolor{currentfill}{rgb}{0.233603,0.313828,0.543914}%
\pgfsetfillcolor{currentfill}%
\pgfsetfillopacity{0.800000}%
\pgfsetlinewidth{0.000000pt}%
\definecolor{currentstroke}{rgb}{0.000000,0.000000,0.000000}%
\pgfsetstrokecolor{currentstroke}%
\pgfsetdash{}{0pt}%
\pgfpathmoveto{\pgfqpoint{4.680637in}{2.727961in}}%
\pgfpathlineto{\pgfqpoint{4.694425in}{2.731584in}}%
\pgfpathlineto{\pgfqpoint{4.708226in}{2.735391in}}%
\pgfpathlineto{\pgfqpoint{4.722039in}{2.739381in}}%
\pgfpathlineto{\pgfqpoint{4.735864in}{2.743553in}}%
\pgfpathlineto{\pgfqpoint{4.743473in}{2.752157in}}%
\pgfpathlineto{\pgfqpoint{4.751078in}{2.760800in}}%
\pgfpathlineto{\pgfqpoint{4.758678in}{2.769487in}}%
\pgfpathlineto{\pgfqpoint{4.766273in}{2.778221in}}%
\pgfpathlineto{\pgfqpoint{4.752461in}{2.774438in}}%
\pgfpathlineto{\pgfqpoint{4.738661in}{2.770837in}}%
\pgfpathlineto{\pgfqpoint{4.724873in}{2.767419in}}%
\pgfpathlineto{\pgfqpoint{4.711097in}{2.764183in}}%
\pgfpathlineto{\pgfqpoint{4.703489in}{2.755049in}}%
\pgfpathlineto{\pgfqpoint{4.695876in}{2.745970in}}%
\pgfpathlineto{\pgfqpoint{4.688259in}{2.736942in}}%
\pgfpathlineto{\pgfqpoint{4.680637in}{2.727961in}}%
\pgfpathclose%
\pgfusepath{fill}%
\end{pgfscope}%
\begin{pgfscope}%
\pgfpathrectangle{\pgfqpoint{1.150000in}{0.150000in}}{\pgfqpoint{5.700000in}{5.700000in}}%
\pgfusepath{clip}%
\pgfsetbuttcap%
\pgfsetroundjoin%
\definecolor{currentfill}{rgb}{0.275191,0.194905,0.496005}%
\pgfsetfillcolor{currentfill}%
\pgfsetfillopacity{0.800000}%
\pgfsetlinewidth{0.000000pt}%
\definecolor{currentstroke}{rgb}{0.000000,0.000000,0.000000}%
\pgfsetstrokecolor{currentstroke}%
\pgfsetdash{}{0pt}%
\pgfpathmoveto{\pgfqpoint{2.930733in}{2.471583in}}%
\pgfpathlineto{\pgfqpoint{2.944285in}{2.456946in}}%
\pgfpathlineto{\pgfqpoint{2.957831in}{2.442589in}}%
\pgfpathlineto{\pgfqpoint{2.971372in}{2.428512in}}%
\pgfpathlineto{\pgfqpoint{2.984908in}{2.414711in}}%
\pgfpathlineto{\pgfqpoint{2.993126in}{2.423352in}}%
\pgfpathlineto{\pgfqpoint{3.001337in}{2.432098in}}%
\pgfpathlineto{\pgfqpoint{3.009539in}{2.440948in}}%
\pgfpathlineto{\pgfqpoint{3.017733in}{2.449901in}}%
\pgfpathlineto{\pgfqpoint{3.004217in}{2.463576in}}%
\pgfpathlineto{\pgfqpoint{2.990696in}{2.477528in}}%
\pgfpathlineto{\pgfqpoint{2.977170in}{2.491758in}}%
\pgfpathlineto{\pgfqpoint{2.963639in}{2.506269in}}%
\pgfpathlineto{\pgfqpoint{2.955425in}{2.497430in}}%
\pgfpathlineto{\pgfqpoint{2.947203in}{2.488702in}}%
\pgfpathlineto{\pgfqpoint{2.938972in}{2.480086in}}%
\pgfpathlineto{\pgfqpoint{2.930733in}{2.471583in}}%
\pgfpathclose%
\pgfusepath{fill}%
\end{pgfscope}%
\begin{pgfscope}%
\pgfpathrectangle{\pgfqpoint{1.150000in}{0.150000in}}{\pgfqpoint{5.700000in}{5.700000in}}%
\pgfusepath{clip}%
\pgfsetbuttcap%
\pgfsetroundjoin%
\definecolor{currentfill}{rgb}{0.282623,0.140926,0.457517}%
\pgfsetfillcolor{currentfill}%
\pgfsetfillopacity{0.800000}%
\pgfsetlinewidth{0.000000pt}%
\definecolor{currentstroke}{rgb}{0.000000,0.000000,0.000000}%
\pgfsetstrokecolor{currentstroke}%
\pgfsetdash{}{0pt}%
\pgfpathmoveto{\pgfqpoint{3.910387in}{2.325908in}}%
\pgfpathlineto{\pgfqpoint{3.923912in}{2.324660in}}%
\pgfpathlineto{\pgfqpoint{3.937443in}{2.323615in}}%
\pgfpathlineto{\pgfqpoint{3.950982in}{2.322772in}}%
\pgfpathlineto{\pgfqpoint{3.964528in}{2.322130in}}%
\pgfpathlineto{\pgfqpoint{3.972402in}{2.332453in}}%
\pgfpathlineto{\pgfqpoint{3.980272in}{2.342779in}}%
\pgfpathlineto{\pgfqpoint{3.988137in}{2.353109in}}%
\pgfpathlineto{\pgfqpoint{3.995997in}{2.363446in}}%
\pgfpathlineto{\pgfqpoint{3.982459in}{2.364188in}}%
\pgfpathlineto{\pgfqpoint{3.968929in}{2.365131in}}%
\pgfpathlineto{\pgfqpoint{3.955406in}{2.366276in}}%
\pgfpathlineto{\pgfqpoint{3.941889in}{2.367624in}}%
\pgfpathlineto{\pgfqpoint{3.934021in}{2.357176in}}%
\pgfpathlineto{\pgfqpoint{3.926148in}{2.346741in}}%
\pgfpathlineto{\pgfqpoint{3.918270in}{2.336319in}}%
\pgfpathlineto{\pgfqpoint{3.910387in}{2.325908in}}%
\pgfpathclose%
\pgfusepath{fill}%
\end{pgfscope}%
\begin{pgfscope}%
\pgfpathrectangle{\pgfqpoint{1.150000in}{0.150000in}}{\pgfqpoint{5.700000in}{5.700000in}}%
\pgfusepath{clip}%
\pgfsetbuttcap%
\pgfsetroundjoin%
\definecolor{currentfill}{rgb}{0.223925,0.334994,0.548053}%
\pgfsetfillcolor{currentfill}%
\pgfsetfillopacity{0.800000}%
\pgfsetlinewidth{0.000000pt}%
\definecolor{currentstroke}{rgb}{0.000000,0.000000,0.000000}%
\pgfsetstrokecolor{currentstroke}%
\pgfsetdash{}{0pt}%
\pgfpathmoveto{\pgfqpoint{4.766273in}{2.778221in}}%
\pgfpathlineto{\pgfqpoint{4.780098in}{2.782186in}}%
\pgfpathlineto{\pgfqpoint{4.793936in}{2.786333in}}%
\pgfpathlineto{\pgfqpoint{4.807786in}{2.790662in}}%
\pgfpathlineto{\pgfqpoint{4.821650in}{2.795172in}}%
\pgfpathlineto{\pgfqpoint{4.829227in}{2.803549in}}%
\pgfpathlineto{\pgfqpoint{4.836800in}{2.811976in}}%
\pgfpathlineto{\pgfqpoint{4.844368in}{2.820457in}}%
\pgfpathlineto{\pgfqpoint{4.851932in}{2.828998in}}%
\pgfpathlineto{\pgfqpoint{4.838082in}{2.824910in}}%
\pgfpathlineto{\pgfqpoint{4.824246in}{2.821002in}}%
\pgfpathlineto{\pgfqpoint{4.810422in}{2.817276in}}%
\pgfpathlineto{\pgfqpoint{4.796611in}{2.813731in}}%
\pgfpathlineto{\pgfqpoint{4.789033in}{2.804758in}}%
\pgfpathlineto{\pgfqpoint{4.781451in}{2.795852in}}%
\pgfpathlineto{\pgfqpoint{4.773864in}{2.787008in}}%
\pgfpathlineto{\pgfqpoint{4.766273in}{2.778221in}}%
\pgfpathclose%
\pgfusepath{fill}%
\end{pgfscope}%
\begin{pgfscope}%
\pgfpathrectangle{\pgfqpoint{1.150000in}{0.150000in}}{\pgfqpoint{5.700000in}{5.700000in}}%
\pgfusepath{clip}%
\pgfsetbuttcap%
\pgfsetroundjoin%
\definecolor{currentfill}{rgb}{0.250425,0.274290,0.533103}%
\pgfsetfillcolor{currentfill}%
\pgfsetfillopacity{0.800000}%
\pgfsetlinewidth{0.000000pt}%
\definecolor{currentstroke}{rgb}{0.000000,0.000000,0.000000}%
\pgfsetstrokecolor{currentstroke}%
\pgfsetdash{}{0pt}%
\pgfpathmoveto{\pgfqpoint{2.767591in}{2.670030in}}%
\pgfpathlineto{\pgfqpoint{2.781230in}{2.651829in}}%
\pgfpathlineto{\pgfqpoint{2.794859in}{2.633940in}}%
\pgfpathlineto{\pgfqpoint{2.808480in}{2.616359in}}%
\pgfpathlineto{\pgfqpoint{2.822093in}{2.599084in}}%
\pgfpathlineto{\pgfqpoint{2.830376in}{2.607307in}}%
\pgfpathlineto{\pgfqpoint{2.838650in}{2.615659in}}%
\pgfpathlineto{\pgfqpoint{2.846915in}{2.624139in}}%
\pgfpathlineto{\pgfqpoint{2.855171in}{2.632747in}}%
\pgfpathlineto{\pgfqpoint{2.841581in}{2.649893in}}%
\pgfpathlineto{\pgfqpoint{2.827983in}{2.667344in}}%
\pgfpathlineto{\pgfqpoint{2.814377in}{2.685104in}}%
\pgfpathlineto{\pgfqpoint{2.800762in}{2.703175in}}%
\pgfpathlineto{\pgfqpoint{2.792484in}{2.694685in}}%
\pgfpathlineto{\pgfqpoint{2.784196in}{2.686330in}}%
\pgfpathlineto{\pgfqpoint{2.775898in}{2.678112in}}%
\pgfpathlineto{\pgfqpoint{2.767591in}{2.670030in}}%
\pgfpathclose%
\pgfusepath{fill}%
\end{pgfscope}%
\begin{pgfscope}%
\pgfpathrectangle{\pgfqpoint{1.150000in}{0.150000in}}{\pgfqpoint{5.700000in}{5.700000in}}%
\pgfusepath{clip}%
\pgfsetbuttcap%
\pgfsetroundjoin%
\definecolor{currentfill}{rgb}{0.141935,0.526453,0.555991}%
\pgfsetfillcolor{currentfill}%
\pgfsetfillopacity{0.800000}%
\pgfsetlinewidth{0.000000pt}%
\definecolor{currentstroke}{rgb}{0.000000,0.000000,0.000000}%
\pgfsetstrokecolor{currentstroke}%
\pgfsetdash{}{0pt}%
\pgfpathmoveto{\pgfqpoint{5.653207in}{3.336571in}}%
\pgfpathlineto{\pgfqpoint{5.667369in}{3.341144in}}%
\pgfpathlineto{\pgfqpoint{5.681548in}{3.345886in}}%
\pgfpathlineto{\pgfqpoint{5.695742in}{3.350798in}}%
\pgfpathlineto{\pgfqpoint{5.709952in}{3.355879in}}%
\pgfpathlineto{\pgfqpoint{5.717230in}{3.364580in}}%
\pgfpathlineto{\pgfqpoint{5.724511in}{3.373549in}}%
\pgfpathlineto{\pgfqpoint{5.731796in}{3.382794in}}%
\pgfpathlineto{\pgfqpoint{5.717610in}{3.378292in}}%
\pgfpathlineto{\pgfqpoint{5.703440in}{3.373958in}}%
\pgfpathlineto{\pgfqpoint{5.689284in}{3.369793in}}%
\pgfpathlineto{\pgfqpoint{5.675145in}{3.365797in}}%
\pgfpathlineto{\pgfqpoint{5.667828in}{3.355775in}}%
\pgfpathlineto{\pgfqpoint{5.660516in}{3.346035in}}%
\pgfpathlineto{\pgfqpoint{5.653207in}{3.336571in}}%
\pgfpathclose%
\pgfusepath{fill}%
\end{pgfscope}%
\begin{pgfscope}%
\pgfpathrectangle{\pgfqpoint{1.150000in}{0.150000in}}{\pgfqpoint{5.700000in}{5.700000in}}%
\pgfusepath{clip}%
\pgfsetbuttcap%
\pgfsetroundjoin%
\definecolor{currentfill}{rgb}{0.282327,0.094955,0.417331}%
\pgfsetfillcolor{currentfill}%
\pgfsetfillopacity{0.800000}%
\pgfsetlinewidth{0.000000pt}%
\definecolor{currentstroke}{rgb}{0.000000,0.000000,0.000000}%
\pgfsetstrokecolor{currentstroke}%
\pgfsetdash{}{0pt}%
\pgfpathmoveto{\pgfqpoint{3.373646in}{2.238596in}}%
\pgfpathlineto{\pgfqpoint{3.387107in}{2.231314in}}%
\pgfpathlineto{\pgfqpoint{3.400569in}{2.224263in}}%
\pgfpathlineto{\pgfqpoint{3.414032in}{2.217442in}}%
\pgfpathlineto{\pgfqpoint{3.427497in}{2.210851in}}%
\pgfpathlineto{\pgfqpoint{3.435548in}{2.220815in}}%
\pgfpathlineto{\pgfqpoint{3.443592in}{2.230824in}}%
\pgfpathlineto{\pgfqpoint{3.451631in}{2.240879in}}%
\pgfpathlineto{\pgfqpoint{3.459664in}{2.250978in}}%
\pgfpathlineto{\pgfqpoint{3.446211in}{2.257511in}}%
\pgfpathlineto{\pgfqpoint{3.432760in}{2.264274in}}%
\pgfpathlineto{\pgfqpoint{3.419311in}{2.271266in}}%
\pgfpathlineto{\pgfqpoint{3.405864in}{2.278489in}}%
\pgfpathlineto{\pgfqpoint{3.397818in}{2.268436in}}%
\pgfpathlineto{\pgfqpoint{3.389767in}{2.258436in}}%
\pgfpathlineto{\pgfqpoint{3.381710in}{2.248489in}}%
\pgfpathlineto{\pgfqpoint{3.373646in}{2.238596in}}%
\pgfpathclose%
\pgfusepath{fill}%
\end{pgfscope}%
\begin{pgfscope}%
\pgfpathrectangle{\pgfqpoint{1.150000in}{0.150000in}}{\pgfqpoint{5.700000in}{5.700000in}}%
\pgfusepath{clip}%
\pgfsetbuttcap%
\pgfsetroundjoin%
\definecolor{currentfill}{rgb}{0.214298,0.355619,0.551184}%
\pgfsetfillcolor{currentfill}%
\pgfsetfillopacity{0.800000}%
\pgfsetlinewidth{0.000000pt}%
\definecolor{currentstroke}{rgb}{0.000000,0.000000,0.000000}%
\pgfsetstrokecolor{currentstroke}%
\pgfsetdash{}{0pt}%
\pgfpathmoveto{\pgfqpoint{4.851932in}{2.828998in}}%
\pgfpathlineto{\pgfqpoint{4.865794in}{2.833266in}}%
\pgfpathlineto{\pgfqpoint{4.879669in}{2.837715in}}%
\pgfpathlineto{\pgfqpoint{4.893557in}{2.842344in}}%
\pgfpathlineto{\pgfqpoint{4.907459in}{2.847154in}}%
\pgfpathlineto{\pgfqpoint{4.915004in}{2.855316in}}%
\pgfpathlineto{\pgfqpoint{4.922544in}{2.863541in}}%
\pgfpathlineto{\pgfqpoint{4.930080in}{2.871832in}}%
\pgfpathlineto{\pgfqpoint{4.937612in}{2.880196in}}%
\pgfpathlineto{\pgfqpoint{4.923726in}{2.875841in}}%
\pgfpathlineto{\pgfqpoint{4.909853in}{2.871666in}}%
\pgfpathlineto{\pgfqpoint{4.895993in}{2.867670in}}%
\pgfpathlineto{\pgfqpoint{4.882145in}{2.863854in}}%
\pgfpathlineto{\pgfqpoint{4.874598in}{2.855026in}}%
\pgfpathlineto{\pgfqpoint{4.867047in}{2.846277in}}%
\pgfpathlineto{\pgfqpoint{4.859491in}{2.837603in}}%
\pgfpathlineto{\pgfqpoint{4.851932in}{2.828998in}}%
\pgfpathclose%
\pgfusepath{fill}%
\end{pgfscope}%
\begin{pgfscope}%
\pgfpathrectangle{\pgfqpoint{1.150000in}{0.150000in}}{\pgfqpoint{5.700000in}{5.700000in}}%
\pgfusepath{clip}%
\pgfsetbuttcap%
\pgfsetroundjoin%
\definecolor{currentfill}{rgb}{0.279574,0.170599,0.479997}%
\pgfsetfillcolor{currentfill}%
\pgfsetfillopacity{0.800000}%
\pgfsetlinewidth{0.000000pt}%
\definecolor{currentstroke}{rgb}{0.000000,0.000000,0.000000}%
\pgfsetstrokecolor{currentstroke}%
\pgfsetdash{}{0pt}%
\pgfpathmoveto{\pgfqpoint{2.984908in}{2.414711in}}%
\pgfpathlineto{\pgfqpoint{2.998439in}{2.401184in}}%
\pgfpathlineto{\pgfqpoint{3.011966in}{2.387930in}}%
\pgfpathlineto{\pgfqpoint{3.025489in}{2.374946in}}%
\pgfpathlineto{\pgfqpoint{3.039008in}{2.362231in}}%
\pgfpathlineto{\pgfqpoint{3.047206in}{2.371009in}}%
\pgfpathlineto{\pgfqpoint{3.055397in}{2.379884in}}%
\pgfpathlineto{\pgfqpoint{3.063580in}{2.388855in}}%
\pgfpathlineto{\pgfqpoint{3.071756in}{2.397922in}}%
\pgfpathlineto{\pgfqpoint{3.058256in}{2.410513in}}%
\pgfpathlineto{\pgfqpoint{3.044753in}{2.423372in}}%
\pgfpathlineto{\pgfqpoint{3.031245in}{2.436500in}}%
\pgfpathlineto{\pgfqpoint{3.017733in}{2.449901in}}%
\pgfpathlineto{\pgfqpoint{3.009539in}{2.440948in}}%
\pgfpathlineto{\pgfqpoint{3.001337in}{2.432098in}}%
\pgfpathlineto{\pgfqpoint{2.993126in}{2.423352in}}%
\pgfpathlineto{\pgfqpoint{2.984908in}{2.414711in}}%
\pgfpathclose%
\pgfusepath{fill}%
\end{pgfscope}%
\begin{pgfscope}%
\pgfpathrectangle{\pgfqpoint{1.150000in}{0.150000in}}{\pgfqpoint{5.700000in}{5.700000in}}%
\pgfusepath{clip}%
\pgfsetbuttcap%
\pgfsetroundjoin%
\definecolor{currentfill}{rgb}{0.283187,0.125848,0.444960}%
\pgfsetfillcolor{currentfill}%
\pgfsetfillopacity{0.800000}%
\pgfsetlinewidth{0.000000pt}%
\definecolor{currentstroke}{rgb}{0.000000,0.000000,0.000000}%
\pgfsetstrokecolor{currentstroke}%
\pgfsetdash{}{0pt}%
\pgfpathmoveto{\pgfqpoint{3.824737in}{2.291109in}}%
\pgfpathlineto{\pgfqpoint{3.838245in}{2.289110in}}%
\pgfpathlineto{\pgfqpoint{3.851758in}{2.287317in}}%
\pgfpathlineto{\pgfqpoint{3.865278in}{2.285729in}}%
\pgfpathlineto{\pgfqpoint{3.878804in}{2.284345in}}%
\pgfpathlineto{\pgfqpoint{3.886707in}{2.294726in}}%
\pgfpathlineto{\pgfqpoint{3.894606in}{2.305113in}}%
\pgfpathlineto{\pgfqpoint{3.902499in}{2.315506in}}%
\pgfpathlineto{\pgfqpoint{3.910387in}{2.325908in}}%
\pgfpathlineto{\pgfqpoint{3.896869in}{2.327360in}}%
\pgfpathlineto{\pgfqpoint{3.883358in}{2.329017in}}%
\pgfpathlineto{\pgfqpoint{3.869853in}{2.330878in}}%
\pgfpathlineto{\pgfqpoint{3.856354in}{2.332946in}}%
\pgfpathlineto{\pgfqpoint{3.848458in}{2.322464in}}%
\pgfpathlineto{\pgfqpoint{3.840556in}{2.311998in}}%
\pgfpathlineto{\pgfqpoint{3.832649in}{2.301547in}}%
\pgfpathlineto{\pgfqpoint{3.824737in}{2.291109in}}%
\pgfpathclose%
\pgfusepath{fill}%
\end{pgfscope}%
\begin{pgfscope}%
\pgfpathrectangle{\pgfqpoint{1.150000in}{0.150000in}}{\pgfqpoint{5.700000in}{5.700000in}}%
\pgfusepath{clip}%
\pgfsetbuttcap%
\pgfsetroundjoin%
\definecolor{currentfill}{rgb}{0.283091,0.110553,0.431554}%
\pgfsetfillcolor{currentfill}%
\pgfsetfillopacity{0.800000}%
\pgfsetlinewidth{0.000000pt}%
\definecolor{currentstroke}{rgb}{0.000000,0.000000,0.000000}%
\pgfsetstrokecolor{currentstroke}%
\pgfsetdash{}{0pt}%
\pgfpathmoveto{\pgfqpoint{3.233551in}{2.267044in}}%
\pgfpathlineto{\pgfqpoint{3.247024in}{2.257768in}}%
\pgfpathlineto{\pgfqpoint{3.260497in}{2.248735in}}%
\pgfpathlineto{\pgfqpoint{3.273969in}{2.239944in}}%
\pgfpathlineto{\pgfqpoint{3.287440in}{2.231392in}}%
\pgfpathlineto{\pgfqpoint{3.295543in}{2.240968in}}%
\pgfpathlineto{\pgfqpoint{3.303639in}{2.250606in}}%
\pgfpathlineto{\pgfqpoint{3.311729in}{2.260306in}}%
\pgfpathlineto{\pgfqpoint{3.319812in}{2.270068in}}%
\pgfpathlineto{\pgfqpoint{3.306355in}{2.278528in}}%
\pgfpathlineto{\pgfqpoint{3.292898in}{2.287229in}}%
\pgfpathlineto{\pgfqpoint{3.279440in}{2.296171in}}%
\pgfpathlineto{\pgfqpoint{3.265983in}{2.305356in}}%
\pgfpathlineto{\pgfqpoint{3.257885in}{2.295673in}}%
\pgfpathlineto{\pgfqpoint{3.249780in}{2.286060in}}%
\pgfpathlineto{\pgfqpoint{3.241669in}{2.276517in}}%
\pgfpathlineto{\pgfqpoint{3.233551in}{2.267044in}}%
\pgfpathclose%
\pgfusepath{fill}%
\end{pgfscope}%
\begin{pgfscope}%
\pgfpathrectangle{\pgfqpoint{1.150000in}{0.150000in}}{\pgfqpoint{5.700000in}{5.700000in}}%
\pgfusepath{clip}%
\pgfsetbuttcap%
\pgfsetroundjoin%
\definecolor{currentfill}{rgb}{0.206756,0.371758,0.553117}%
\pgfsetfillcolor{currentfill}%
\pgfsetfillopacity{0.800000}%
\pgfsetlinewidth{0.000000pt}%
\definecolor{currentstroke}{rgb}{0.000000,0.000000,0.000000}%
\pgfsetstrokecolor{currentstroke}%
\pgfsetdash{}{0pt}%
\pgfpathmoveto{\pgfqpoint{4.937612in}{2.880196in}}%
\pgfpathlineto{\pgfqpoint{4.951512in}{2.884730in}}%
\pgfpathlineto{\pgfqpoint{4.965425in}{2.889443in}}%
\pgfpathlineto{\pgfqpoint{4.979352in}{2.894335in}}%
\pgfpathlineto{\pgfqpoint{4.993293in}{2.899405in}}%
\pgfpathlineto{\pgfqpoint{5.000805in}{2.907371in}}%
\pgfpathlineto{\pgfqpoint{5.008312in}{2.915413in}}%
\pgfpathlineto{\pgfqpoint{5.015816in}{2.923535in}}%
\pgfpathlineto{\pgfqpoint{5.023316in}{2.931744in}}%
\pgfpathlineto{\pgfqpoint{5.009392in}{2.927161in}}%
\pgfpathlineto{\pgfqpoint{4.995482in}{2.922755in}}%
\pgfpathlineto{\pgfqpoint{4.981586in}{2.918528in}}%
\pgfpathlineto{\pgfqpoint{4.967702in}{2.914479in}}%
\pgfpathlineto{\pgfqpoint{4.960185in}{2.905773in}}%
\pgfpathlineto{\pgfqpoint{4.952665in}{2.897161in}}%
\pgfpathlineto{\pgfqpoint{4.945140in}{2.888637in}}%
\pgfpathlineto{\pgfqpoint{4.937612in}{2.880196in}}%
\pgfpathclose%
\pgfusepath{fill}%
\end{pgfscope}%
\begin{pgfscope}%
\pgfpathrectangle{\pgfqpoint{1.150000in}{0.150000in}}{\pgfqpoint{5.700000in}{5.700000in}}%
\pgfusepath{clip}%
\pgfsetbuttcap%
\pgfsetroundjoin%
\definecolor{currentfill}{rgb}{0.237441,0.305202,0.541921}%
\pgfsetfillcolor{currentfill}%
\pgfsetfillopacity{0.800000}%
\pgfsetlinewidth{0.000000pt}%
\definecolor{currentstroke}{rgb}{0.000000,0.000000,0.000000}%
\pgfsetstrokecolor{currentstroke}%
\pgfsetdash{}{0pt}%
\pgfpathmoveto{\pgfqpoint{2.712944in}{2.746007in}}%
\pgfpathlineto{\pgfqpoint{2.726621in}{2.726531in}}%
\pgfpathlineto{\pgfqpoint{2.740287in}{2.707378in}}%
\pgfpathlineto{\pgfqpoint{2.753944in}{2.688546in}}%
\pgfpathlineto{\pgfqpoint{2.767591in}{2.670030in}}%
\pgfpathlineto{\pgfqpoint{2.775898in}{2.678112in}}%
\pgfpathlineto{\pgfqpoint{2.784196in}{2.686330in}}%
\pgfpathlineto{\pgfqpoint{2.792484in}{2.694685in}}%
\pgfpathlineto{\pgfqpoint{2.800762in}{2.703175in}}%
\pgfpathlineto{\pgfqpoint{2.787139in}{2.721560in}}%
\pgfpathlineto{\pgfqpoint{2.773506in}{2.740262in}}%
\pgfpathlineto{\pgfqpoint{2.759864in}{2.759283in}}%
\pgfpathlineto{\pgfqpoint{2.746212in}{2.778628in}}%
\pgfpathlineto{\pgfqpoint{2.737910in}{2.770256in}}%
\pgfpathlineto{\pgfqpoint{2.729598in}{2.762028in}}%
\pgfpathlineto{\pgfqpoint{2.721276in}{2.753945in}}%
\pgfpathlineto{\pgfqpoint{2.712944in}{2.746007in}}%
\pgfpathclose%
\pgfusepath{fill}%
\end{pgfscope}%
\begin{pgfscope}%
\pgfpathrectangle{\pgfqpoint{1.150000in}{0.150000in}}{\pgfqpoint{5.700000in}{5.700000in}}%
\pgfusepath{clip}%
\pgfsetbuttcap%
\pgfsetroundjoin%
\definecolor{currentfill}{rgb}{0.282327,0.094955,0.417331}%
\pgfsetfillcolor{currentfill}%
\pgfsetfillopacity{0.800000}%
\pgfsetlinewidth{0.000000pt}%
\definecolor{currentstroke}{rgb}{0.000000,0.000000,0.000000}%
\pgfsetstrokecolor{currentstroke}%
\pgfsetdash{}{0pt}%
\pgfpathmoveto{\pgfqpoint{3.513497in}{2.227109in}}%
\pgfpathlineto{\pgfqpoint{3.526962in}{2.221702in}}%
\pgfpathlineto{\pgfqpoint{3.540429in}{2.216517in}}%
\pgfpathlineto{\pgfqpoint{3.553900in}{2.211552in}}%
\pgfpathlineto{\pgfqpoint{3.567374in}{2.206806in}}%
\pgfpathlineto{\pgfqpoint{3.575379in}{2.217028in}}%
\pgfpathlineto{\pgfqpoint{3.583377in}{2.227279in}}%
\pgfpathlineto{\pgfqpoint{3.591370in}{2.237560in}}%
\pgfpathlineto{\pgfqpoint{3.599358in}{2.247871in}}%
\pgfpathlineto{\pgfqpoint{3.585895in}{2.252590in}}%
\pgfpathlineto{\pgfqpoint{3.572435in}{2.257529in}}%
\pgfpathlineto{\pgfqpoint{3.558978in}{2.262688in}}%
\pgfpathlineto{\pgfqpoint{3.545525in}{2.268068in}}%
\pgfpathlineto{\pgfqpoint{3.537526in}{2.257772in}}%
\pgfpathlineto{\pgfqpoint{3.529522in}{2.247514in}}%
\pgfpathlineto{\pgfqpoint{3.521512in}{2.237293in}}%
\pgfpathlineto{\pgfqpoint{3.513497in}{2.227109in}}%
\pgfpathclose%
\pgfusepath{fill}%
\end{pgfscope}%
\begin{pgfscope}%
\pgfpathrectangle{\pgfqpoint{1.150000in}{0.150000in}}{\pgfqpoint{5.700000in}{5.700000in}}%
\pgfusepath{clip}%
\pgfsetbuttcap%
\pgfsetroundjoin%
\definecolor{currentfill}{rgb}{0.197636,0.391528,0.554969}%
\pgfsetfillcolor{currentfill}%
\pgfsetfillopacity{0.800000}%
\pgfsetlinewidth{0.000000pt}%
\definecolor{currentstroke}{rgb}{0.000000,0.000000,0.000000}%
\pgfsetstrokecolor{currentstroke}%
\pgfsetdash{}{0pt}%
\pgfpathmoveto{\pgfqpoint{5.023316in}{2.931744in}}%
\pgfpathlineto{\pgfqpoint{5.037254in}{2.936506in}}%
\pgfpathlineto{\pgfqpoint{5.051205in}{2.941445in}}%
\pgfpathlineto{\pgfqpoint{5.065171in}{2.946562in}}%
\pgfpathlineto{\pgfqpoint{5.079150in}{2.951856in}}%
\pgfpathlineto{\pgfqpoint{5.086629in}{2.959650in}}%
\pgfpathlineto{\pgfqpoint{5.094104in}{2.967533in}}%
\pgfpathlineto{\pgfqpoint{5.101576in}{2.975513in}}%
\pgfpathlineto{\pgfqpoint{5.109044in}{2.983594in}}%
\pgfpathlineto{\pgfqpoint{5.095083in}{2.978820in}}%
\pgfpathlineto{\pgfqpoint{5.081136in}{2.974222in}}%
\pgfpathlineto{\pgfqpoint{5.067202in}{2.969801in}}%
\pgfpathlineto{\pgfqpoint{5.053282in}{2.965557in}}%
\pgfpathlineto{\pgfqpoint{5.045796in}{2.956945in}}%
\pgfpathlineto{\pgfqpoint{5.038306in}{2.948443in}}%
\pgfpathlineto{\pgfqpoint{5.030813in}{2.940045in}}%
\pgfpathlineto{\pgfqpoint{5.023316in}{2.931744in}}%
\pgfpathclose%
\pgfusepath{fill}%
\end{pgfscope}%
\begin{pgfscope}%
\pgfpathrectangle{\pgfqpoint{1.150000in}{0.150000in}}{\pgfqpoint{5.700000in}{5.700000in}}%
\pgfusepath{clip}%
\pgfsetbuttcap%
\pgfsetroundjoin%
\definecolor{currentfill}{rgb}{0.281887,0.150881,0.465405}%
\pgfsetfillcolor{currentfill}%
\pgfsetfillopacity{0.800000}%
\pgfsetlinewidth{0.000000pt}%
\definecolor{currentstroke}{rgb}{0.000000,0.000000,0.000000}%
\pgfsetstrokecolor{currentstroke}%
\pgfsetdash{}{0pt}%
\pgfpathmoveto{\pgfqpoint{3.039008in}{2.362231in}}%
\pgfpathlineto{\pgfqpoint{3.052523in}{2.349781in}}%
\pgfpathlineto{\pgfqpoint{3.066034in}{2.337596in}}%
\pgfpathlineto{\pgfqpoint{3.079543in}{2.325673in}}%
\pgfpathlineto{\pgfqpoint{3.093048in}{2.314011in}}%
\pgfpathlineto{\pgfqpoint{3.101228in}{2.322925in}}%
\pgfpathlineto{\pgfqpoint{3.109401in}{2.331928in}}%
\pgfpathlineto{\pgfqpoint{3.117566in}{2.341020in}}%
\pgfpathlineto{\pgfqpoint{3.125723in}{2.350199in}}%
\pgfpathlineto{\pgfqpoint{3.112236in}{2.361738in}}%
\pgfpathlineto{\pgfqpoint{3.098746in}{2.373536in}}%
\pgfpathlineto{\pgfqpoint{3.085253in}{2.385597in}}%
\pgfpathlineto{\pgfqpoint{3.071756in}{2.397922in}}%
\pgfpathlineto{\pgfqpoint{3.063580in}{2.388855in}}%
\pgfpathlineto{\pgfqpoint{3.055397in}{2.379884in}}%
\pgfpathlineto{\pgfqpoint{3.047206in}{2.371009in}}%
\pgfpathlineto{\pgfqpoint{3.039008in}{2.362231in}}%
\pgfpathclose%
\pgfusepath{fill}%
\end{pgfscope}%
\begin{pgfscope}%
\pgfpathrectangle{\pgfqpoint{1.150000in}{0.150000in}}{\pgfqpoint{5.700000in}{5.700000in}}%
\pgfusepath{clip}%
\pgfsetbuttcap%
\pgfsetroundjoin%
\definecolor{currentfill}{rgb}{0.283197,0.115680,0.436115}%
\pgfsetfillcolor{currentfill}%
\pgfsetfillopacity{0.800000}%
\pgfsetlinewidth{0.000000pt}%
\definecolor{currentstroke}{rgb}{0.000000,0.000000,0.000000}%
\pgfsetstrokecolor{currentstroke}%
\pgfsetdash{}{0pt}%
\pgfpathmoveto{\pgfqpoint{3.739031in}{2.259403in}}%
\pgfpathlineto{\pgfqpoint{3.752525in}{2.256608in}}%
\pgfpathlineto{\pgfqpoint{3.766024in}{2.254022in}}%
\pgfpathlineto{\pgfqpoint{3.779528in}{2.251644in}}%
\pgfpathlineto{\pgfqpoint{3.793038in}{2.249475in}}%
\pgfpathlineto{\pgfqpoint{3.800971in}{2.259868in}}%
\pgfpathlineto{\pgfqpoint{3.808898in}{2.270271in}}%
\pgfpathlineto{\pgfqpoint{3.816820in}{2.280684in}}%
\pgfpathlineto{\pgfqpoint{3.824737in}{2.291109in}}%
\pgfpathlineto{\pgfqpoint{3.811236in}{2.293316in}}%
\pgfpathlineto{\pgfqpoint{3.797740in}{2.295730in}}%
\pgfpathlineto{\pgfqpoint{3.784250in}{2.298353in}}%
\pgfpathlineto{\pgfqpoint{3.770766in}{2.301185in}}%
\pgfpathlineto{\pgfqpoint{3.762840in}{2.290712in}}%
\pgfpathlineto{\pgfqpoint{3.754909in}{2.280257in}}%
\pgfpathlineto{\pgfqpoint{3.746973in}{2.269821in}}%
\pgfpathlineto{\pgfqpoint{3.739031in}{2.259403in}}%
\pgfpathclose%
\pgfusepath{fill}%
\end{pgfscope}%
\begin{pgfscope}%
\pgfpathrectangle{\pgfqpoint{1.150000in}{0.150000in}}{\pgfqpoint{5.700000in}{5.700000in}}%
\pgfusepath{clip}%
\pgfsetbuttcap%
\pgfsetroundjoin%
\definecolor{currentfill}{rgb}{0.188923,0.410910,0.556326}%
\pgfsetfillcolor{currentfill}%
\pgfsetfillopacity{0.800000}%
\pgfsetlinewidth{0.000000pt}%
\definecolor{currentstroke}{rgb}{0.000000,0.000000,0.000000}%
\pgfsetstrokecolor{currentstroke}%
\pgfsetdash{}{0pt}%
\pgfpathmoveto{\pgfqpoint{5.109044in}{2.983594in}}%
\pgfpathlineto{\pgfqpoint{5.123020in}{2.988546in}}%
\pgfpathlineto{\pgfqpoint{5.137009in}{2.993674in}}%
\pgfpathlineto{\pgfqpoint{5.151013in}{2.998978in}}%
\pgfpathlineto{\pgfqpoint{5.165032in}{3.004459in}}%
\pgfpathlineto{\pgfqpoint{5.172478in}{3.012109in}}%
\pgfpathlineto{\pgfqpoint{5.179920in}{3.019865in}}%
\pgfpathlineto{\pgfqpoint{5.187360in}{3.027734in}}%
\pgfpathlineto{\pgfqpoint{5.194797in}{3.035722in}}%
\pgfpathlineto{\pgfqpoint{5.180798in}{3.030793in}}%
\pgfpathlineto{\pgfqpoint{5.166814in}{3.026040in}}%
\pgfpathlineto{\pgfqpoint{5.152844in}{3.021463in}}%
\pgfpathlineto{\pgfqpoint{5.138888in}{3.017061in}}%
\pgfpathlineto{\pgfqpoint{5.131431in}{3.008511in}}%
\pgfpathlineto{\pgfqpoint{5.123972in}{3.000087in}}%
\pgfpathlineto{\pgfqpoint{5.116509in}{2.991784in}}%
\pgfpathlineto{\pgfqpoint{5.109044in}{2.983594in}}%
\pgfpathclose%
\pgfusepath{fill}%
\end{pgfscope}%
\begin{pgfscope}%
\pgfpathrectangle{\pgfqpoint{1.150000in}{0.150000in}}{\pgfqpoint{5.700000in}{5.700000in}}%
\pgfusepath{clip}%
\pgfsetbuttcap%
\pgfsetroundjoin%
\definecolor{currentfill}{rgb}{0.221989,0.339161,0.548752}%
\pgfsetfillcolor{currentfill}%
\pgfsetfillopacity{0.800000}%
\pgfsetlinewidth{0.000000pt}%
\definecolor{currentstroke}{rgb}{0.000000,0.000000,0.000000}%
\pgfsetstrokecolor{currentstroke}%
\pgfsetdash{}{0pt}%
\pgfpathmoveto{\pgfqpoint{2.658131in}{2.827204in}}%
\pgfpathlineto{\pgfqpoint{2.671851in}{2.806404in}}%
\pgfpathlineto{\pgfqpoint{2.685560in}{2.785940in}}%
\pgfpathlineto{\pgfqpoint{2.699257in}{2.765809in}}%
\pgfpathlineto{\pgfqpoint{2.712944in}{2.746007in}}%
\pgfpathlineto{\pgfqpoint{2.721276in}{2.753945in}}%
\pgfpathlineto{\pgfqpoint{2.729598in}{2.762028in}}%
\pgfpathlineto{\pgfqpoint{2.737910in}{2.770256in}}%
\pgfpathlineto{\pgfqpoint{2.746212in}{2.778628in}}%
\pgfpathlineto{\pgfqpoint{2.732550in}{2.798298in}}%
\pgfpathlineto{\pgfqpoint{2.718877in}{2.818297in}}%
\pgfpathlineto{\pgfqpoint{2.705194in}{2.838629in}}%
\pgfpathlineto{\pgfqpoint{2.691500in}{2.859296in}}%
\pgfpathlineto{\pgfqpoint{2.683173in}{2.851044in}}%
\pgfpathlineto{\pgfqpoint{2.674836in}{2.842945in}}%
\pgfpathlineto{\pgfqpoint{2.666489in}{2.834998in}}%
\pgfpathlineto{\pgfqpoint{2.658131in}{2.827204in}}%
\pgfpathclose%
\pgfusepath{fill}%
\end{pgfscope}%
\begin{pgfscope}%
\pgfpathrectangle{\pgfqpoint{1.150000in}{0.150000in}}{\pgfqpoint{5.700000in}{5.700000in}}%
\pgfusepath{clip}%
\pgfsetbuttcap%
\pgfsetroundjoin%
\definecolor{currentfill}{rgb}{0.180629,0.429975,0.557282}%
\pgfsetfillcolor{currentfill}%
\pgfsetfillopacity{0.800000}%
\pgfsetlinewidth{0.000000pt}%
\definecolor{currentstroke}{rgb}{0.000000,0.000000,0.000000}%
\pgfsetstrokecolor{currentstroke}%
\pgfsetdash{}{0pt}%
\pgfpathmoveto{\pgfqpoint{5.194797in}{3.035722in}}%
\pgfpathlineto{\pgfqpoint{5.208810in}{3.040826in}}%
\pgfpathlineto{\pgfqpoint{5.222838in}{3.046106in}}%
\pgfpathlineto{\pgfqpoint{5.236880in}{3.051560in}}%
\pgfpathlineto{\pgfqpoint{5.250937in}{3.057191in}}%
\pgfpathlineto{\pgfqpoint{5.258350in}{3.064731in}}%
\pgfpathlineto{\pgfqpoint{5.265761in}{3.072396in}}%
\pgfpathlineto{\pgfqpoint{5.273169in}{3.080192in}}%
\pgfpathlineto{\pgfqpoint{5.280576in}{3.088125in}}%
\pgfpathlineto{\pgfqpoint{5.266540in}{3.083079in}}%
\pgfpathlineto{\pgfqpoint{5.252519in}{3.078208in}}%
\pgfpathlineto{\pgfqpoint{5.238513in}{3.073512in}}%
\pgfpathlineto{\pgfqpoint{5.224521in}{3.068990in}}%
\pgfpathlineto{\pgfqpoint{5.217093in}{3.060462in}}%
\pgfpathlineto{\pgfqpoint{5.209663in}{3.052079in}}%
\pgfpathlineto{\pgfqpoint{5.202231in}{3.043835in}}%
\pgfpathlineto{\pgfqpoint{5.194797in}{3.035722in}}%
\pgfpathclose%
\pgfusepath{fill}%
\end{pgfscope}%
\begin{pgfscope}%
\pgfpathrectangle{\pgfqpoint{1.150000in}{0.150000in}}{\pgfqpoint{5.700000in}{5.700000in}}%
\pgfusepath{clip}%
\pgfsetbuttcap%
\pgfsetroundjoin%
\definecolor{currentfill}{rgb}{0.283072,0.130895,0.449241}%
\pgfsetfillcolor{currentfill}%
\pgfsetfillopacity{0.800000}%
\pgfsetlinewidth{0.000000pt}%
\definecolor{currentstroke}{rgb}{0.000000,0.000000,0.000000}%
\pgfsetstrokecolor{currentstroke}%
\pgfsetdash{}{0pt}%
\pgfpathmoveto{\pgfqpoint{3.093048in}{2.314011in}}%
\pgfpathlineto{\pgfqpoint{3.106551in}{2.302607in}}%
\pgfpathlineto{\pgfqpoint{3.120051in}{2.291460in}}%
\pgfpathlineto{\pgfqpoint{3.133549in}{2.280567in}}%
\pgfpathlineto{\pgfqpoint{3.147045in}{2.269928in}}%
\pgfpathlineto{\pgfqpoint{3.155207in}{2.278978in}}%
\pgfpathlineto{\pgfqpoint{3.163362in}{2.288109in}}%
\pgfpathlineto{\pgfqpoint{3.171510in}{2.297320in}}%
\pgfpathlineto{\pgfqpoint{3.179650in}{2.306611in}}%
\pgfpathlineto{\pgfqpoint{3.166172in}{2.317127in}}%
\pgfpathlineto{\pgfqpoint{3.152691in}{2.327896in}}%
\pgfpathlineto{\pgfqpoint{3.139208in}{2.338919in}}%
\pgfpathlineto{\pgfqpoint{3.125723in}{2.350199in}}%
\pgfpathlineto{\pgfqpoint{3.117566in}{2.341020in}}%
\pgfpathlineto{\pgfqpoint{3.109401in}{2.331928in}}%
\pgfpathlineto{\pgfqpoint{3.101228in}{2.322925in}}%
\pgfpathlineto{\pgfqpoint{3.093048in}{2.314011in}}%
\pgfpathclose%
\pgfusepath{fill}%
\end{pgfscope}%
\begin{pgfscope}%
\pgfpathrectangle{\pgfqpoint{1.150000in}{0.150000in}}{\pgfqpoint{5.700000in}{5.700000in}}%
\pgfusepath{clip}%
\pgfsetbuttcap%
\pgfsetroundjoin%
\definecolor{currentfill}{rgb}{0.172719,0.448791,0.557885}%
\pgfsetfillcolor{currentfill}%
\pgfsetfillopacity{0.800000}%
\pgfsetlinewidth{0.000000pt}%
\definecolor{currentstroke}{rgb}{0.000000,0.000000,0.000000}%
\pgfsetstrokecolor{currentstroke}%
\pgfsetdash{}{0pt}%
\pgfpathmoveto{\pgfqpoint{5.280576in}{3.088125in}}%
\pgfpathlineto{\pgfqpoint{5.294626in}{3.093345in}}%
\pgfpathlineto{\pgfqpoint{5.308691in}{3.098739in}}%
\pgfpathlineto{\pgfqpoint{5.322771in}{3.104307in}}%
\pgfpathlineto{\pgfqpoint{5.336867in}{3.110050in}}%
\pgfpathlineto{\pgfqpoint{5.344248in}{3.117522in}}%
\pgfpathlineto{\pgfqpoint{5.351628in}{3.125137in}}%
\pgfpathlineto{\pgfqpoint{5.359006in}{3.132903in}}%
\pgfpathlineto{\pgfqpoint{5.366382in}{3.140825in}}%
\pgfpathlineto{\pgfqpoint{5.352310in}{3.135700in}}%
\pgfpathlineto{\pgfqpoint{5.338253in}{3.130748in}}%
\pgfpathlineto{\pgfqpoint{5.324211in}{3.125969in}}%
\pgfpathlineto{\pgfqpoint{5.310183in}{3.121364in}}%
\pgfpathlineto{\pgfqpoint{5.302783in}{3.112815in}}%
\pgfpathlineto{\pgfqpoint{5.295382in}{3.104429in}}%
\pgfpathlineto{\pgfqpoint{5.287980in}{3.096202in}}%
\pgfpathlineto{\pgfqpoint{5.280576in}{3.088125in}}%
\pgfpathclose%
\pgfusepath{fill}%
\end{pgfscope}%
\begin{pgfscope}%
\pgfpathrectangle{\pgfqpoint{1.150000in}{0.150000in}}{\pgfqpoint{5.700000in}{5.700000in}}%
\pgfusepath{clip}%
\pgfsetbuttcap%
\pgfsetroundjoin%
\definecolor{currentfill}{rgb}{0.282656,0.100196,0.422160}%
\pgfsetfillcolor{currentfill}%
\pgfsetfillopacity{0.800000}%
\pgfsetlinewidth{0.000000pt}%
\definecolor{currentstroke}{rgb}{0.000000,0.000000,0.000000}%
\pgfsetstrokecolor{currentstroke}%
\pgfsetdash{}{0pt}%
\pgfpathmoveto{\pgfqpoint{3.653251in}{2.231167in}}%
\pgfpathlineto{\pgfqpoint{3.666734in}{2.227530in}}%
\pgfpathlineto{\pgfqpoint{3.680222in}{2.224105in}}%
\pgfpathlineto{\pgfqpoint{3.693715in}{2.220894in}}%
\pgfpathlineto{\pgfqpoint{3.707213in}{2.217893in}}%
\pgfpathlineto{\pgfqpoint{3.715175in}{2.228247in}}%
\pgfpathlineto{\pgfqpoint{3.723132in}{2.238617in}}%
\pgfpathlineto{\pgfqpoint{3.731084in}{2.249002in}}%
\pgfpathlineto{\pgfqpoint{3.739031in}{2.259403in}}%
\pgfpathlineto{\pgfqpoint{3.725543in}{2.262409in}}%
\pgfpathlineto{\pgfqpoint{3.712059in}{2.265627in}}%
\pgfpathlineto{\pgfqpoint{3.698581in}{2.269056in}}%
\pgfpathlineto{\pgfqpoint{3.685107in}{2.272699in}}%
\pgfpathlineto{\pgfqpoint{3.677151in}{2.262281in}}%
\pgfpathlineto{\pgfqpoint{3.669189in}{2.251886in}}%
\pgfpathlineto{\pgfqpoint{3.661223in}{2.241515in}}%
\pgfpathlineto{\pgfqpoint{3.653251in}{2.231167in}}%
\pgfpathclose%
\pgfusepath{fill}%
\end{pgfscope}%
\begin{pgfscope}%
\pgfpathrectangle{\pgfqpoint{1.150000in}{0.150000in}}{\pgfqpoint{5.700000in}{5.700000in}}%
\pgfusepath{clip}%
\pgfsetbuttcap%
\pgfsetroundjoin%
\definecolor{currentfill}{rgb}{0.282656,0.100196,0.422160}%
\pgfsetfillcolor{currentfill}%
\pgfsetfillopacity{0.800000}%
\pgfsetlinewidth{0.000000pt}%
\definecolor{currentstroke}{rgb}{0.000000,0.000000,0.000000}%
\pgfsetstrokecolor{currentstroke}%
\pgfsetdash{}{0pt}%
\pgfpathmoveto{\pgfqpoint{3.287440in}{2.231392in}}%
\pgfpathlineto{\pgfqpoint{3.300912in}{2.223080in}}%
\pgfpathlineto{\pgfqpoint{3.314385in}{2.215005in}}%
\pgfpathlineto{\pgfqpoint{3.327857in}{2.207165in}}%
\pgfpathlineto{\pgfqpoint{3.341331in}{2.199560in}}%
\pgfpathlineto{\pgfqpoint{3.349419in}{2.209237in}}%
\pgfpathlineto{\pgfqpoint{3.357501in}{2.218969in}}%
\pgfpathlineto{\pgfqpoint{3.365577in}{2.228756in}}%
\pgfpathlineto{\pgfqpoint{3.373646in}{2.238596in}}%
\pgfpathlineto{\pgfqpoint{3.360187in}{2.246111in}}%
\pgfpathlineto{\pgfqpoint{3.346728in}{2.253860in}}%
\pgfpathlineto{\pgfqpoint{3.333270in}{2.261845in}}%
\pgfpathlineto{\pgfqpoint{3.319812in}{2.270068in}}%
\pgfpathlineto{\pgfqpoint{3.311729in}{2.260306in}}%
\pgfpathlineto{\pgfqpoint{3.303639in}{2.250606in}}%
\pgfpathlineto{\pgfqpoint{3.295543in}{2.240968in}}%
\pgfpathlineto{\pgfqpoint{3.287440in}{2.231392in}}%
\pgfpathclose%
\pgfusepath{fill}%
\end{pgfscope}%
\begin{pgfscope}%
\pgfpathrectangle{\pgfqpoint{1.150000in}{0.150000in}}{\pgfqpoint{5.700000in}{5.700000in}}%
\pgfusepath{clip}%
\pgfsetbuttcap%
\pgfsetroundjoin%
\definecolor{currentfill}{rgb}{0.281924,0.089666,0.412415}%
\pgfsetfillcolor{currentfill}%
\pgfsetfillopacity{0.800000}%
\pgfsetlinewidth{0.000000pt}%
\definecolor{currentstroke}{rgb}{0.000000,0.000000,0.000000}%
\pgfsetstrokecolor{currentstroke}%
\pgfsetdash{}{0pt}%
\pgfpathmoveto{\pgfqpoint{3.427497in}{2.210851in}}%
\pgfpathlineto{\pgfqpoint{3.440964in}{2.204487in}}%
\pgfpathlineto{\pgfqpoint{3.454434in}{2.198350in}}%
\pgfpathlineto{\pgfqpoint{3.467905in}{2.192438in}}%
\pgfpathlineto{\pgfqpoint{3.481379in}{2.186750in}}%
\pgfpathlineto{\pgfqpoint{3.489417in}{2.196784in}}%
\pgfpathlineto{\pgfqpoint{3.497449in}{2.206855in}}%
\pgfpathlineto{\pgfqpoint{3.505476in}{2.216963in}}%
\pgfpathlineto{\pgfqpoint{3.513497in}{2.227109in}}%
\pgfpathlineto{\pgfqpoint{3.500035in}{2.232739in}}%
\pgfpathlineto{\pgfqpoint{3.486576in}{2.238594in}}%
\pgfpathlineto{\pgfqpoint{3.473119in}{2.244673in}}%
\pgfpathlineto{\pgfqpoint{3.459664in}{2.250978in}}%
\pgfpathlineto{\pgfqpoint{3.451631in}{2.240879in}}%
\pgfpathlineto{\pgfqpoint{3.443592in}{2.230824in}}%
\pgfpathlineto{\pgfqpoint{3.435548in}{2.220815in}}%
\pgfpathlineto{\pgfqpoint{3.427497in}{2.210851in}}%
\pgfpathclose%
\pgfusepath{fill}%
\end{pgfscope}%
\begin{pgfscope}%
\pgfpathrectangle{\pgfqpoint{1.150000in}{0.150000in}}{\pgfqpoint{5.700000in}{5.700000in}}%
\pgfusepath{clip}%
\pgfsetbuttcap%
\pgfsetroundjoin%
\definecolor{currentfill}{rgb}{0.165117,0.467423,0.558141}%
\pgfsetfillcolor{currentfill}%
\pgfsetfillopacity{0.800000}%
\pgfsetlinewidth{0.000000pt}%
\definecolor{currentstroke}{rgb}{0.000000,0.000000,0.000000}%
\pgfsetstrokecolor{currentstroke}%
\pgfsetdash{}{0pt}%
\pgfpathmoveto{\pgfqpoint{5.366382in}{3.140825in}}%
\pgfpathlineto{\pgfqpoint{5.380469in}{3.146124in}}%
\pgfpathlineto{\pgfqpoint{5.394572in}{3.151596in}}%
\pgfpathlineto{\pgfqpoint{5.408689in}{3.157241in}}%
\pgfpathlineto{\pgfqpoint{5.422823in}{3.163060in}}%
\pgfpathlineto{\pgfqpoint{5.430173in}{3.170509in}}%
\pgfpathlineto{\pgfqpoint{5.437522in}{3.178122in}}%
\pgfpathlineto{\pgfqpoint{5.444871in}{3.185907in}}%
\pgfpathlineto{\pgfqpoint{5.452219in}{3.193870in}}%
\pgfpathlineto{\pgfqpoint{5.438111in}{3.188701in}}%
\pgfpathlineto{\pgfqpoint{5.424018in}{3.183704in}}%
\pgfpathlineto{\pgfqpoint{5.409941in}{3.178880in}}%
\pgfpathlineto{\pgfqpoint{5.395878in}{3.174229in}}%
\pgfpathlineto{\pgfqpoint{5.388505in}{3.165607in}}%
\pgfpathlineto{\pgfqpoint{5.381132in}{3.157170in}}%
\pgfpathlineto{\pgfqpoint{5.373757in}{3.148912in}}%
\pgfpathlineto{\pgfqpoint{5.366382in}{3.140825in}}%
\pgfpathclose%
\pgfusepath{fill}%
\end{pgfscope}%
\begin{pgfscope}%
\pgfpathrectangle{\pgfqpoint{1.150000in}{0.150000in}}{\pgfqpoint{5.700000in}{5.700000in}}%
\pgfusepath{clip}%
\pgfsetbuttcap%
\pgfsetroundjoin%
\definecolor{currentfill}{rgb}{0.206756,0.371758,0.553117}%
\pgfsetfillcolor{currentfill}%
\pgfsetfillopacity{0.800000}%
\pgfsetlinewidth{0.000000pt}%
\definecolor{currentstroke}{rgb}{0.000000,0.000000,0.000000}%
\pgfsetstrokecolor{currentstroke}%
\pgfsetdash{}{0pt}%
\pgfpathmoveto{\pgfqpoint{2.603131in}{2.913829in}}%
\pgfpathlineto{\pgfqpoint{2.616900in}{2.891652in}}%
\pgfpathlineto{\pgfqpoint{2.630656in}{2.869825in}}%
\pgfpathlineto{\pgfqpoint{2.644400in}{2.848343in}}%
\pgfpathlineto{\pgfqpoint{2.658131in}{2.827204in}}%
\pgfpathlineto{\pgfqpoint{2.666489in}{2.834998in}}%
\pgfpathlineto{\pgfqpoint{2.674836in}{2.842945in}}%
\pgfpathlineto{\pgfqpoint{2.683173in}{2.851044in}}%
\pgfpathlineto{\pgfqpoint{2.691500in}{2.859296in}}%
\pgfpathlineto{\pgfqpoint{2.677794in}{2.880301in}}%
\pgfpathlineto{\pgfqpoint{2.664077in}{2.901649in}}%
\pgfpathlineto{\pgfqpoint{2.650348in}{2.923342in}}%
\pgfpathlineto{\pgfqpoint{2.636606in}{2.945384in}}%
\pgfpathlineto{\pgfqpoint{2.628253in}{2.937254in}}%
\pgfpathlineto{\pgfqpoint{2.619890in}{2.929284in}}%
\pgfpathlineto{\pgfqpoint{2.611516in}{2.921476in}}%
\pgfpathlineto{\pgfqpoint{2.603131in}{2.913829in}}%
\pgfpathclose%
\pgfusepath{fill}%
\end{pgfscope}%
\begin{pgfscope}%
\pgfpathrectangle{\pgfqpoint{1.150000in}{0.150000in}}{\pgfqpoint{5.700000in}{5.700000in}}%
\pgfusepath{clip}%
\pgfsetbuttcap%
\pgfsetroundjoin%
\definecolor{currentfill}{rgb}{0.157729,0.485932,0.558013}%
\pgfsetfillcolor{currentfill}%
\pgfsetfillopacity{0.800000}%
\pgfsetlinewidth{0.000000pt}%
\definecolor{currentstroke}{rgb}{0.000000,0.000000,0.000000}%
\pgfsetstrokecolor{currentstroke}%
\pgfsetdash{}{0pt}%
\pgfpathmoveto{\pgfqpoint{5.452219in}{3.193870in}}%
\pgfpathlineto{\pgfqpoint{5.466342in}{3.199211in}}%
\pgfpathlineto{\pgfqpoint{5.480481in}{3.204724in}}%
\pgfpathlineto{\pgfqpoint{5.494635in}{3.210409in}}%
\pgfpathlineto{\pgfqpoint{5.508806in}{3.216267in}}%
\pgfpathlineto{\pgfqpoint{5.516127in}{3.223746in}}%
\pgfpathlineto{\pgfqpoint{5.523447in}{3.231410in}}%
\pgfpathlineto{\pgfqpoint{5.530768in}{3.239268in}}%
\pgfpathlineto{\pgfqpoint{5.538090in}{3.247327in}}%
\pgfpathlineto{\pgfqpoint{5.523947in}{3.242151in}}%
\pgfpathlineto{\pgfqpoint{5.509819in}{3.237147in}}%
\pgfpathlineto{\pgfqpoint{5.495707in}{3.232314in}}%
\pgfpathlineto{\pgfqpoint{5.481610in}{3.227652in}}%
\pgfpathlineto{\pgfqpoint{5.474262in}{3.218902in}}%
\pgfpathlineto{\pgfqpoint{5.466914in}{3.210360in}}%
\pgfpathlineto{\pgfqpoint{5.459567in}{3.202018in}}%
\pgfpathlineto{\pgfqpoint{5.452219in}{3.193870in}}%
\pgfpathclose%
\pgfusepath{fill}%
\end{pgfscope}%
\begin{pgfscope}%
\pgfpathrectangle{\pgfqpoint{1.150000in}{0.150000in}}{\pgfqpoint{5.700000in}{5.700000in}}%
\pgfusepath{clip}%
\pgfsetbuttcap%
\pgfsetroundjoin%
\definecolor{currentfill}{rgb}{0.271828,0.209303,0.504434}%
\pgfsetfillcolor{currentfill}%
\pgfsetfillopacity{0.800000}%
\pgfsetlinewidth{0.000000pt}%
\definecolor{currentstroke}{rgb}{0.000000,0.000000,0.000000}%
\pgfsetstrokecolor{currentstroke}%
\pgfsetdash{}{0pt}%
\pgfpathmoveto{\pgfqpoint{4.221578in}{2.449599in}}%
\pgfpathlineto{\pgfqpoint{4.235209in}{2.451121in}}%
\pgfpathlineto{\pgfqpoint{4.248849in}{2.452835in}}%
\pgfpathlineto{\pgfqpoint{4.262499in}{2.454742in}}%
\pgfpathlineto{\pgfqpoint{4.276158in}{2.456841in}}%
\pgfpathlineto{\pgfqpoint{4.283940in}{2.466583in}}%
\pgfpathlineto{\pgfqpoint{4.291717in}{2.476320in}}%
\pgfpathlineto{\pgfqpoint{4.299489in}{2.486054in}}%
\pgfpathlineto{\pgfqpoint{4.307255in}{2.495788in}}%
\pgfpathlineto{\pgfqpoint{4.293604in}{2.493886in}}%
\pgfpathlineto{\pgfqpoint{4.279962in}{2.492176in}}%
\pgfpathlineto{\pgfqpoint{4.266330in}{2.490658in}}%
\pgfpathlineto{\pgfqpoint{4.252708in}{2.489332in}}%
\pgfpathlineto{\pgfqpoint{4.244933in}{2.479389in}}%
\pgfpathlineto{\pgfqpoint{4.237153in}{2.469454in}}%
\pgfpathlineto{\pgfqpoint{4.229368in}{2.459525in}}%
\pgfpathlineto{\pgfqpoint{4.221578in}{2.449599in}}%
\pgfpathclose%
\pgfusepath{fill}%
\end{pgfscope}%
\begin{pgfscope}%
\pgfpathrectangle{\pgfqpoint{1.150000in}{0.150000in}}{\pgfqpoint{5.700000in}{5.700000in}}%
\pgfusepath{clip}%
\pgfsetbuttcap%
\pgfsetroundjoin%
\definecolor{currentfill}{rgb}{0.277134,0.185228,0.489898}%
\pgfsetfillcolor{currentfill}%
\pgfsetfillopacity{0.800000}%
\pgfsetlinewidth{0.000000pt}%
\definecolor{currentstroke}{rgb}{0.000000,0.000000,0.000000}%
\pgfsetstrokecolor{currentstroke}%
\pgfsetdash{}{0pt}%
\pgfpathmoveto{\pgfqpoint{4.135903in}{2.405070in}}%
\pgfpathlineto{\pgfqpoint{4.149506in}{2.405980in}}%
\pgfpathlineto{\pgfqpoint{4.163117in}{2.407084in}}%
\pgfpathlineto{\pgfqpoint{4.176737in}{2.408383in}}%
\pgfpathlineto{\pgfqpoint{4.190366in}{2.409877in}}%
\pgfpathlineto{\pgfqpoint{4.198177in}{2.419814in}}%
\pgfpathlineto{\pgfqpoint{4.205982in}{2.429745in}}%
\pgfpathlineto{\pgfqpoint{4.213783in}{2.439673in}}%
\pgfpathlineto{\pgfqpoint{4.221578in}{2.449599in}}%
\pgfpathlineto{\pgfqpoint{4.207957in}{2.448270in}}%
\pgfpathlineto{\pgfqpoint{4.194344in}{2.447136in}}%
\pgfpathlineto{\pgfqpoint{4.180741in}{2.446196in}}%
\pgfpathlineto{\pgfqpoint{4.167147in}{2.445450in}}%
\pgfpathlineto{\pgfqpoint{4.159344in}{2.435348in}}%
\pgfpathlineto{\pgfqpoint{4.151535in}{2.425252in}}%
\pgfpathlineto{\pgfqpoint{4.143722in}{2.415160in}}%
\pgfpathlineto{\pgfqpoint{4.135903in}{2.405070in}}%
\pgfpathclose%
\pgfusepath{fill}%
\end{pgfscope}%
\begin{pgfscope}%
\pgfpathrectangle{\pgfqpoint{1.150000in}{0.150000in}}{\pgfqpoint{5.700000in}{5.700000in}}%
\pgfusepath{clip}%
\pgfsetbuttcap%
\pgfsetroundjoin%
\definecolor{currentfill}{rgb}{0.266580,0.228262,0.514349}%
\pgfsetfillcolor{currentfill}%
\pgfsetfillopacity{0.800000}%
\pgfsetlinewidth{0.000000pt}%
\definecolor{currentstroke}{rgb}{0.000000,0.000000,0.000000}%
\pgfsetstrokecolor{currentstroke}%
\pgfsetdash{}{0pt}%
\pgfpathmoveto{\pgfqpoint{4.307255in}{2.495788in}}%
\pgfpathlineto{\pgfqpoint{4.320916in}{2.497882in}}%
\pgfpathlineto{\pgfqpoint{4.334588in}{2.500166in}}%
\pgfpathlineto{\pgfqpoint{4.348270in}{2.502640in}}%
\pgfpathlineto{\pgfqpoint{4.361962in}{2.505305in}}%
\pgfpathlineto{\pgfqpoint{4.369714in}{2.514825in}}%
\pgfpathlineto{\pgfqpoint{4.377462in}{2.524343in}}%
\pgfpathlineto{\pgfqpoint{4.385204in}{2.533862in}}%
\pgfpathlineto{\pgfqpoint{4.392942in}{2.543385in}}%
\pgfpathlineto{\pgfqpoint{4.379258in}{2.540949in}}%
\pgfpathlineto{\pgfqpoint{4.365586in}{2.538704in}}%
\pgfpathlineto{\pgfqpoint{4.351923in}{2.536648in}}%
\pgfpathlineto{\pgfqpoint{4.338271in}{2.534783in}}%
\pgfpathlineto{\pgfqpoint{4.330524in}{2.525020in}}%
\pgfpathlineto{\pgfqpoint{4.322773in}{2.515269in}}%
\pgfpathlineto{\pgfqpoint{4.315017in}{2.505526in}}%
\pgfpathlineto{\pgfqpoint{4.307255in}{2.495788in}}%
\pgfpathclose%
\pgfusepath{fill}%
\end{pgfscope}%
\begin{pgfscope}%
\pgfpathrectangle{\pgfqpoint{1.150000in}{0.150000in}}{\pgfqpoint{5.700000in}{5.700000in}}%
\pgfusepath{clip}%
\pgfsetbuttcap%
\pgfsetroundjoin%
\definecolor{currentfill}{rgb}{0.279574,0.170599,0.479997}%
\pgfsetfillcolor{currentfill}%
\pgfsetfillopacity{0.800000}%
\pgfsetlinewidth{0.000000pt}%
\definecolor{currentstroke}{rgb}{0.000000,0.000000,0.000000}%
\pgfsetstrokecolor{currentstroke}%
\pgfsetdash{}{0pt}%
\pgfpathmoveto{\pgfqpoint{4.050223in}{2.362480in}}%
\pgfpathlineto{\pgfqpoint{4.063800in}{2.362736in}}%
\pgfpathlineto{\pgfqpoint{4.077384in}{2.363189in}}%
\pgfpathlineto{\pgfqpoint{4.090977in}{2.363839in}}%
\pgfpathlineto{\pgfqpoint{4.104579in}{2.364686in}}%
\pgfpathlineto{\pgfqpoint{4.112417in}{2.374789in}}%
\pgfpathlineto{\pgfqpoint{4.120251in}{2.384886in}}%
\pgfpathlineto{\pgfqpoint{4.128080in}{2.394979in}}%
\pgfpathlineto{\pgfqpoint{4.135903in}{2.405070in}}%
\pgfpathlineto{\pgfqpoint{4.122310in}{2.404356in}}%
\pgfpathlineto{\pgfqpoint{4.108725in}{2.403839in}}%
\pgfpathlineto{\pgfqpoint{4.095148in}{2.403518in}}%
\pgfpathlineto{\pgfqpoint{4.081580in}{2.403395in}}%
\pgfpathlineto{\pgfqpoint{4.073748in}{2.393160in}}%
\pgfpathlineto{\pgfqpoint{4.065912in}{2.382930in}}%
\pgfpathlineto{\pgfqpoint{4.058070in}{2.372704in}}%
\pgfpathlineto{\pgfqpoint{4.050223in}{2.362480in}}%
\pgfpathclose%
\pgfusepath{fill}%
\end{pgfscope}%
\begin{pgfscope}%
\pgfpathrectangle{\pgfqpoint{1.150000in}{0.150000in}}{\pgfqpoint{5.700000in}{5.700000in}}%
\pgfusepath{clip}%
\pgfsetbuttcap%
\pgfsetroundjoin%
\definecolor{currentfill}{rgb}{0.260571,0.246922,0.522828}%
\pgfsetfillcolor{currentfill}%
\pgfsetfillopacity{0.800000}%
\pgfsetlinewidth{0.000000pt}%
\definecolor{currentstroke}{rgb}{0.000000,0.000000,0.000000}%
\pgfsetstrokecolor{currentstroke}%
\pgfsetdash{}{0pt}%
\pgfpathmoveto{\pgfqpoint{4.392942in}{2.543385in}}%
\pgfpathlineto{\pgfqpoint{4.406636in}{2.546009in}}%
\pgfpathlineto{\pgfqpoint{4.420340in}{2.548823in}}%
\pgfpathlineto{\pgfqpoint{4.434055in}{2.551824in}}%
\pgfpathlineto{\pgfqpoint{4.447782in}{2.555014in}}%
\pgfpathlineto{\pgfqpoint{4.455505in}{2.564294in}}%
\pgfpathlineto{\pgfqpoint{4.463223in}{2.573575in}}%
\pgfpathlineto{\pgfqpoint{4.470935in}{2.582862in}}%
\pgfpathlineto{\pgfqpoint{4.478643in}{2.592156in}}%
\pgfpathlineto{\pgfqpoint{4.464926in}{2.589228in}}%
\pgfpathlineto{\pgfqpoint{4.451220in}{2.586487in}}%
\pgfpathlineto{\pgfqpoint{4.437525in}{2.583934in}}%
\pgfpathlineto{\pgfqpoint{4.423840in}{2.581570in}}%
\pgfpathlineto{\pgfqpoint{4.416123in}{2.572003in}}%
\pgfpathlineto{\pgfqpoint{4.408401in}{2.562452in}}%
\pgfpathlineto{\pgfqpoint{4.400674in}{2.552914in}}%
\pgfpathlineto{\pgfqpoint{4.392942in}{2.543385in}}%
\pgfpathclose%
\pgfusepath{fill}%
\end{pgfscope}%
\begin{pgfscope}%
\pgfpathrectangle{\pgfqpoint{1.150000in}{0.150000in}}{\pgfqpoint{5.700000in}{5.700000in}}%
\pgfusepath{clip}%
\pgfsetbuttcap%
\pgfsetroundjoin%
\definecolor{currentfill}{rgb}{0.283197,0.115680,0.436115}%
\pgfsetfillcolor{currentfill}%
\pgfsetfillopacity{0.800000}%
\pgfsetlinewidth{0.000000pt}%
\definecolor{currentstroke}{rgb}{0.000000,0.000000,0.000000}%
\pgfsetstrokecolor{currentstroke}%
\pgfsetdash{}{0pt}%
\pgfpathmoveto{\pgfqpoint{3.147045in}{2.269928in}}%
\pgfpathlineto{\pgfqpoint{3.160539in}{2.259541in}}%
\pgfpathlineto{\pgfqpoint{3.174032in}{2.249403in}}%
\pgfpathlineto{\pgfqpoint{3.187523in}{2.239513in}}%
\pgfpathlineto{\pgfqpoint{3.201013in}{2.229870in}}%
\pgfpathlineto{\pgfqpoint{3.209158in}{2.239054in}}%
\pgfpathlineto{\pgfqpoint{3.217296in}{2.248312in}}%
\pgfpathlineto{\pgfqpoint{3.225427in}{2.257642in}}%
\pgfpathlineto{\pgfqpoint{3.233551in}{2.267044in}}%
\pgfpathlineto{\pgfqpoint{3.220078in}{2.276564in}}%
\pgfpathlineto{\pgfqpoint{3.206603in}{2.286331in}}%
\pgfpathlineto{\pgfqpoint{3.193128in}{2.296346in}}%
\pgfpathlineto{\pgfqpoint{3.179650in}{2.306611in}}%
\pgfpathlineto{\pgfqpoint{3.171510in}{2.297320in}}%
\pgfpathlineto{\pgfqpoint{3.163362in}{2.288109in}}%
\pgfpathlineto{\pgfqpoint{3.155207in}{2.278978in}}%
\pgfpathlineto{\pgfqpoint{3.147045in}{2.269928in}}%
\pgfpathclose%
\pgfusepath{fill}%
\end{pgfscope}%
\begin{pgfscope}%
\pgfpathrectangle{\pgfqpoint{1.150000in}{0.150000in}}{\pgfqpoint{5.700000in}{5.700000in}}%
\pgfusepath{clip}%
\pgfsetbuttcap%
\pgfsetroundjoin%
\definecolor{currentfill}{rgb}{0.150476,0.504369,0.557430}%
\pgfsetfillcolor{currentfill}%
\pgfsetfillopacity{0.800000}%
\pgfsetlinewidth{0.000000pt}%
\definecolor{currentstroke}{rgb}{0.000000,0.000000,0.000000}%
\pgfsetstrokecolor{currentstroke}%
\pgfsetdash{}{0pt}%
\pgfpathmoveto{\pgfqpoint{5.538090in}{3.247327in}}%
\pgfpathlineto{\pgfqpoint{5.552248in}{3.252674in}}%
\pgfpathlineto{\pgfqpoint{5.566423in}{3.258192in}}%
\pgfpathlineto{\pgfqpoint{5.580613in}{3.263882in}}%
\pgfpathlineto{\pgfqpoint{5.594819in}{3.269742in}}%
\pgfpathlineto{\pgfqpoint{5.602113in}{3.277308in}}%
\pgfpathlineto{\pgfqpoint{5.609407in}{3.285082in}}%
\pgfpathlineto{\pgfqpoint{5.616702in}{3.293074in}}%
\pgfpathlineto{\pgfqpoint{5.623999in}{3.301291in}}%
\pgfpathlineto{\pgfqpoint{5.609822in}{3.296144in}}%
\pgfpathlineto{\pgfqpoint{5.595660in}{3.291168in}}%
\pgfpathlineto{\pgfqpoint{5.581515in}{3.286363in}}%
\pgfpathlineto{\pgfqpoint{5.567384in}{3.281727in}}%
\pgfpathlineto{\pgfqpoint{5.560059in}{3.272787in}}%
\pgfpathlineto{\pgfqpoint{5.552734in}{3.264078in}}%
\pgfpathlineto{\pgfqpoint{5.545412in}{3.255594in}}%
\pgfpathlineto{\pgfqpoint{5.538090in}{3.247327in}}%
\pgfpathclose%
\pgfusepath{fill}%
\end{pgfscope}%
\begin{pgfscope}%
\pgfpathrectangle{\pgfqpoint{1.150000in}{0.150000in}}{\pgfqpoint{5.700000in}{5.700000in}}%
\pgfusepath{clip}%
\pgfsetbuttcap%
\pgfsetroundjoin%
\definecolor{currentfill}{rgb}{0.282327,0.094955,0.417331}%
\pgfsetfillcolor{currentfill}%
\pgfsetfillopacity{0.800000}%
\pgfsetlinewidth{0.000000pt}%
\definecolor{currentstroke}{rgb}{0.000000,0.000000,0.000000}%
\pgfsetstrokecolor{currentstroke}%
\pgfsetdash{}{0pt}%
\pgfpathmoveto{\pgfqpoint{3.567374in}{2.206806in}}%
\pgfpathlineto{\pgfqpoint{3.580852in}{2.202279in}}%
\pgfpathlineto{\pgfqpoint{3.594334in}{2.197970in}}%
\pgfpathlineto{\pgfqpoint{3.607819in}{2.193876in}}%
\pgfpathlineto{\pgfqpoint{3.621309in}{2.189998in}}%
\pgfpathlineto{\pgfqpoint{3.629302in}{2.200258in}}%
\pgfpathlineto{\pgfqpoint{3.637290in}{2.210539in}}%
\pgfpathlineto{\pgfqpoint{3.645273in}{2.220842in}}%
\pgfpathlineto{\pgfqpoint{3.653251in}{2.231167in}}%
\pgfpathlineto{\pgfqpoint{3.639771in}{2.235019in}}%
\pgfpathlineto{\pgfqpoint{3.626296in}{2.239087in}}%
\pgfpathlineto{\pgfqpoint{3.612825in}{2.243370in}}%
\pgfpathlineto{\pgfqpoint{3.599358in}{2.247871in}}%
\pgfpathlineto{\pgfqpoint{3.591370in}{2.237560in}}%
\pgfpathlineto{\pgfqpoint{3.583377in}{2.227279in}}%
\pgfpathlineto{\pgfqpoint{3.575379in}{2.217028in}}%
\pgfpathlineto{\pgfqpoint{3.567374in}{2.206806in}}%
\pgfpathclose%
\pgfusepath{fill}%
\end{pgfscope}%
\begin{pgfscope}%
\pgfpathrectangle{\pgfqpoint{1.150000in}{0.150000in}}{\pgfqpoint{5.700000in}{5.700000in}}%
\pgfusepath{clip}%
\pgfsetbuttcap%
\pgfsetroundjoin%
\definecolor{currentfill}{rgb}{0.252194,0.269783,0.531579}%
\pgfsetfillcolor{currentfill}%
\pgfsetfillopacity{0.800000}%
\pgfsetlinewidth{0.000000pt}%
\definecolor{currentstroke}{rgb}{0.000000,0.000000,0.000000}%
\pgfsetstrokecolor{currentstroke}%
\pgfsetdash{}{0pt}%
\pgfpathmoveto{\pgfqpoint{4.478643in}{2.592156in}}%
\pgfpathlineto{\pgfqpoint{4.492371in}{2.595272in}}%
\pgfpathlineto{\pgfqpoint{4.506110in}{2.598575in}}%
\pgfpathlineto{\pgfqpoint{4.519861in}{2.602065in}}%
\pgfpathlineto{\pgfqpoint{4.533623in}{2.605740in}}%
\pgfpathlineto{\pgfqpoint{4.541315in}{2.614765in}}%
\pgfpathlineto{\pgfqpoint{4.549003in}{2.623796in}}%
\pgfpathlineto{\pgfqpoint{4.556685in}{2.632838in}}%
\pgfpathlineto{\pgfqpoint{4.564362in}{2.641894in}}%
\pgfpathlineto{\pgfqpoint{4.550610in}{2.638512in}}%
\pgfpathlineto{\pgfqpoint{4.536869in}{2.635316in}}%
\pgfpathlineto{\pgfqpoint{4.523140in}{2.632306in}}%
\pgfpathlineto{\pgfqpoint{4.509422in}{2.629483in}}%
\pgfpathlineto{\pgfqpoint{4.501735in}{2.620122in}}%
\pgfpathlineto{\pgfqpoint{4.494043in}{2.610783in}}%
\pgfpathlineto{\pgfqpoint{4.486345in}{2.601462in}}%
\pgfpathlineto{\pgfqpoint{4.478643in}{2.592156in}}%
\pgfpathclose%
\pgfusepath{fill}%
\end{pgfscope}%
\begin{pgfscope}%
\pgfpathrectangle{\pgfqpoint{1.150000in}{0.150000in}}{\pgfqpoint{5.700000in}{5.700000in}}%
\pgfusepath{clip}%
\pgfsetbuttcap%
\pgfsetroundjoin%
\definecolor{currentfill}{rgb}{0.281887,0.150881,0.465405}%
\pgfsetfillcolor{currentfill}%
\pgfsetfillopacity{0.800000}%
\pgfsetlinewidth{0.000000pt}%
\definecolor{currentstroke}{rgb}{0.000000,0.000000,0.000000}%
\pgfsetstrokecolor{currentstroke}%
\pgfsetdash{}{0pt}%
\pgfpathmoveto{\pgfqpoint{3.964528in}{2.322130in}}%
\pgfpathlineto{\pgfqpoint{3.978081in}{2.321690in}}%
\pgfpathlineto{\pgfqpoint{3.991641in}{2.321450in}}%
\pgfpathlineto{\pgfqpoint{4.005210in}{2.321409in}}%
\pgfpathlineto{\pgfqpoint{4.018786in}{2.321567in}}%
\pgfpathlineto{\pgfqpoint{4.026653in}{2.331801in}}%
\pgfpathlineto{\pgfqpoint{4.034515in}{2.342030in}}%
\pgfpathlineto{\pgfqpoint{4.042371in}{2.352256in}}%
\pgfpathlineto{\pgfqpoint{4.050223in}{2.362480in}}%
\pgfpathlineto{\pgfqpoint{4.036655in}{2.362423in}}%
\pgfpathlineto{\pgfqpoint{4.023095in}{2.362564in}}%
\pgfpathlineto{\pgfqpoint{4.009542in}{2.362905in}}%
\pgfpathlineto{\pgfqpoint{3.995997in}{2.363446in}}%
\pgfpathlineto{\pgfqpoint{3.988137in}{2.353109in}}%
\pgfpathlineto{\pgfqpoint{3.980272in}{2.342779in}}%
\pgfpathlineto{\pgfqpoint{3.972402in}{2.332453in}}%
\pgfpathlineto{\pgfqpoint{3.964528in}{2.322130in}}%
\pgfpathclose%
\pgfusepath{fill}%
\end{pgfscope}%
\begin{pgfscope}%
\pgfpathrectangle{\pgfqpoint{1.150000in}{0.150000in}}{\pgfqpoint{5.700000in}{5.700000in}}%
\pgfusepath{clip}%
\pgfsetbuttcap%
\pgfsetroundjoin%
\definecolor{currentfill}{rgb}{0.244972,0.287675,0.537260}%
\pgfsetfillcolor{currentfill}%
\pgfsetfillopacity{0.800000}%
\pgfsetlinewidth{0.000000pt}%
\definecolor{currentstroke}{rgb}{0.000000,0.000000,0.000000}%
\pgfsetstrokecolor{currentstroke}%
\pgfsetdash{}{0pt}%
\pgfpathmoveto{\pgfqpoint{4.564362in}{2.641894in}}%
\pgfpathlineto{\pgfqpoint{4.578125in}{2.645462in}}%
\pgfpathlineto{\pgfqpoint{4.591901in}{2.649215in}}%
\pgfpathlineto{\pgfqpoint{4.605688in}{2.653154in}}%
\pgfpathlineto{\pgfqpoint{4.619487in}{2.657277in}}%
\pgfpathlineto{\pgfqpoint{4.627149in}{2.666037in}}%
\pgfpathlineto{\pgfqpoint{4.634805in}{2.674810in}}%
\pgfpathlineto{\pgfqpoint{4.642456in}{2.683602in}}%
\pgfpathlineto{\pgfqpoint{4.650102in}{2.692414in}}%
\pgfpathlineto{\pgfqpoint{4.636313in}{2.688617in}}%
\pgfpathlineto{\pgfqpoint{4.622537in}{2.685005in}}%
\pgfpathlineto{\pgfqpoint{4.608773in}{2.681577in}}%
\pgfpathlineto{\pgfqpoint{4.595020in}{2.678334in}}%
\pgfpathlineto{\pgfqpoint{4.587363in}{2.669184in}}%
\pgfpathlineto{\pgfqpoint{4.579701in}{2.660064in}}%
\pgfpathlineto{\pgfqpoint{4.572034in}{2.650968in}}%
\pgfpathlineto{\pgfqpoint{4.564362in}{2.641894in}}%
\pgfpathclose%
\pgfusepath{fill}%
\end{pgfscope}%
\begin{pgfscope}%
\pgfpathrectangle{\pgfqpoint{1.150000in}{0.150000in}}{\pgfqpoint{5.700000in}{5.700000in}}%
\pgfusepath{clip}%
\pgfsetbuttcap%
\pgfsetroundjoin%
\definecolor{currentfill}{rgb}{0.143343,0.522773,0.556295}%
\pgfsetfillcolor{currentfill}%
\pgfsetfillopacity{0.800000}%
\pgfsetlinewidth{0.000000pt}%
\definecolor{currentstroke}{rgb}{0.000000,0.000000,0.000000}%
\pgfsetstrokecolor{currentstroke}%
\pgfsetdash{}{0pt}%
\pgfpathmoveto{\pgfqpoint{5.623999in}{3.301291in}}%
\pgfpathlineto{\pgfqpoint{5.638192in}{3.306607in}}%
\pgfpathlineto{\pgfqpoint{5.652401in}{3.312095in}}%
\pgfpathlineto{\pgfqpoint{5.666626in}{3.317752in}}%
\pgfpathlineto{\pgfqpoint{5.680867in}{3.323580in}}%
\pgfpathlineto{\pgfqpoint{5.688135in}{3.331295in}}%
\pgfpathlineto{\pgfqpoint{5.695405in}{3.339245in}}%
\pgfpathlineto{\pgfqpoint{5.702677in}{3.347436in}}%
\pgfpathlineto{\pgfqpoint{5.709952in}{3.355879in}}%
\pgfpathlineto{\pgfqpoint{5.695742in}{3.350798in}}%
\pgfpathlineto{\pgfqpoint{5.681548in}{3.345886in}}%
\pgfpathlineto{\pgfqpoint{5.667369in}{3.341144in}}%
\pgfpathlineto{\pgfqpoint{5.653207in}{3.336571in}}%
\pgfpathlineto{\pgfqpoint{5.645901in}{3.327372in}}%
\pgfpathlineto{\pgfqpoint{5.638598in}{3.318432in}}%
\pgfpathlineto{\pgfqpoint{5.631297in}{3.309741in}}%
\pgfpathlineto{\pgfqpoint{5.623999in}{3.301291in}}%
\pgfpathclose%
\pgfusepath{fill}%
\end{pgfscope}%
\begin{pgfscope}%
\pgfpathrectangle{\pgfqpoint{1.150000in}{0.150000in}}{\pgfqpoint{5.700000in}{5.700000in}}%
\pgfusepath{clip}%
\pgfsetbuttcap%
\pgfsetroundjoin%
\definecolor{currentfill}{rgb}{0.282884,0.135920,0.453427}%
\pgfsetfillcolor{currentfill}%
\pgfsetfillopacity{0.800000}%
\pgfsetlinewidth{0.000000pt}%
\definecolor{currentstroke}{rgb}{0.000000,0.000000,0.000000}%
\pgfsetstrokecolor{currentstroke}%
\pgfsetdash{}{0pt}%
\pgfpathmoveto{\pgfqpoint{3.878804in}{2.284345in}}%
\pgfpathlineto{\pgfqpoint{3.892337in}{2.283166in}}%
\pgfpathlineto{\pgfqpoint{3.905877in}{2.282190in}}%
\pgfpathlineto{\pgfqpoint{3.919423in}{2.281416in}}%
\pgfpathlineto{\pgfqpoint{3.932977in}{2.280844in}}%
\pgfpathlineto{\pgfqpoint{3.940872in}{2.291167in}}%
\pgfpathlineto{\pgfqpoint{3.948763in}{2.301489in}}%
\pgfpathlineto{\pgfqpoint{3.956648in}{2.311809in}}%
\pgfpathlineto{\pgfqpoint{3.964528in}{2.322130in}}%
\pgfpathlineto{\pgfqpoint{3.950982in}{2.322772in}}%
\pgfpathlineto{\pgfqpoint{3.937443in}{2.323615in}}%
\pgfpathlineto{\pgfqpoint{3.923912in}{2.324660in}}%
\pgfpathlineto{\pgfqpoint{3.910387in}{2.325908in}}%
\pgfpathlineto{\pgfqpoint{3.902499in}{2.315506in}}%
\pgfpathlineto{\pgfqpoint{3.894606in}{2.305113in}}%
\pgfpathlineto{\pgfqpoint{3.886707in}{2.294726in}}%
\pgfpathlineto{\pgfqpoint{3.878804in}{2.284345in}}%
\pgfpathclose%
\pgfusepath{fill}%
\end{pgfscope}%
\begin{pgfscope}%
\pgfpathrectangle{\pgfqpoint{1.150000in}{0.150000in}}{\pgfqpoint{5.700000in}{5.700000in}}%
\pgfusepath{clip}%
\pgfsetbuttcap%
\pgfsetroundjoin%
\definecolor{currentfill}{rgb}{0.235526,0.309527,0.542944}%
\pgfsetfillcolor{currentfill}%
\pgfsetfillopacity{0.800000}%
\pgfsetlinewidth{0.000000pt}%
\definecolor{currentstroke}{rgb}{0.000000,0.000000,0.000000}%
\pgfsetstrokecolor{currentstroke}%
\pgfsetdash{}{0pt}%
\pgfpathmoveto{\pgfqpoint{4.650102in}{2.692414in}}%
\pgfpathlineto{\pgfqpoint{4.663902in}{2.696395in}}%
\pgfpathlineto{\pgfqpoint{4.677715in}{2.700560in}}%
\pgfpathlineto{\pgfqpoint{4.691540in}{2.704908in}}%
\pgfpathlineto{\pgfqpoint{4.705377in}{2.709439in}}%
\pgfpathlineto{\pgfqpoint{4.713006in}{2.717931in}}%
\pgfpathlineto{\pgfqpoint{4.720631in}{2.726444in}}%
\pgfpathlineto{\pgfqpoint{4.728250in}{2.734984in}}%
\pgfpathlineto{\pgfqpoint{4.735864in}{2.743553in}}%
\pgfpathlineto{\pgfqpoint{4.722039in}{2.739381in}}%
\pgfpathlineto{\pgfqpoint{4.708226in}{2.735391in}}%
\pgfpathlineto{\pgfqpoint{4.694425in}{2.731584in}}%
\pgfpathlineto{\pgfqpoint{4.680637in}{2.727961in}}%
\pgfpathlineto{\pgfqpoint{4.673010in}{2.719021in}}%
\pgfpathlineto{\pgfqpoint{4.665379in}{2.710120in}}%
\pgfpathlineto{\pgfqpoint{4.657743in}{2.701252in}}%
\pgfpathlineto{\pgfqpoint{4.650102in}{2.692414in}}%
\pgfpathclose%
\pgfusepath{fill}%
\end{pgfscope}%
\begin{pgfscope}%
\pgfpathrectangle{\pgfqpoint{1.150000in}{0.150000in}}{\pgfqpoint{5.700000in}{5.700000in}}%
\pgfusepath{clip}%
\pgfsetbuttcap%
\pgfsetroundjoin%
\definecolor{currentfill}{rgb}{0.270595,0.214069,0.507052}%
\pgfsetfillcolor{currentfill}%
\pgfsetfillopacity{0.800000}%
\pgfsetlinewidth{0.000000pt}%
\definecolor{currentstroke}{rgb}{0.000000,0.000000,0.000000}%
\pgfsetstrokecolor{currentstroke}%
\pgfsetdash{}{0pt}%
\pgfpathmoveto{\pgfqpoint{2.843336in}{2.500765in}}%
\pgfpathlineto{\pgfqpoint{2.856935in}{2.484820in}}%
\pgfpathlineto{\pgfqpoint{2.870527in}{2.469166in}}%
\pgfpathlineto{\pgfqpoint{2.884113in}{2.453801in}}%
\pgfpathlineto{\pgfqpoint{2.897692in}{2.438721in}}%
\pgfpathlineto{\pgfqpoint{2.905965in}{2.446762in}}%
\pgfpathlineto{\pgfqpoint{2.914230in}{2.454920in}}%
\pgfpathlineto{\pgfqpoint{2.922486in}{2.463194in}}%
\pgfpathlineto{\pgfqpoint{2.930733in}{2.471583in}}%
\pgfpathlineto{\pgfqpoint{2.917176in}{2.486504in}}%
\pgfpathlineto{\pgfqpoint{2.903613in}{2.501710in}}%
\pgfpathlineto{\pgfqpoint{2.890043in}{2.517205in}}%
\pgfpathlineto{\pgfqpoint{2.876467in}{2.532990in}}%
\pgfpathlineto{\pgfqpoint{2.868198in}{2.524748in}}%
\pgfpathlineto{\pgfqpoint{2.859920in}{2.516629in}}%
\pgfpathlineto{\pgfqpoint{2.851633in}{2.508634in}}%
\pgfpathlineto{\pgfqpoint{2.843336in}{2.500765in}}%
\pgfpathclose%
\pgfusepath{fill}%
\end{pgfscope}%
\begin{pgfscope}%
\pgfpathrectangle{\pgfqpoint{1.150000in}{0.150000in}}{\pgfqpoint{5.700000in}{5.700000in}}%
\pgfusepath{clip}%
\pgfsetbuttcap%
\pgfsetroundjoin%
\definecolor{currentfill}{rgb}{0.262138,0.242286,0.520837}%
\pgfsetfillcolor{currentfill}%
\pgfsetfillopacity{0.800000}%
\pgfsetlinewidth{0.000000pt}%
\definecolor{currentstroke}{rgb}{0.000000,0.000000,0.000000}%
\pgfsetstrokecolor{currentstroke}%
\pgfsetdash{}{0pt}%
\pgfpathmoveto{\pgfqpoint{2.788867in}{2.567502in}}%
\pgfpathlineto{\pgfqpoint{2.802496in}{2.550369in}}%
\pgfpathlineto{\pgfqpoint{2.816117in}{2.533537in}}%
\pgfpathlineto{\pgfqpoint{2.829730in}{2.517003in}}%
\pgfpathlineto{\pgfqpoint{2.843336in}{2.500765in}}%
\pgfpathlineto{\pgfqpoint{2.851633in}{2.508634in}}%
\pgfpathlineto{\pgfqpoint{2.859920in}{2.516629in}}%
\pgfpathlineto{\pgfqpoint{2.868198in}{2.524748in}}%
\pgfpathlineto{\pgfqpoint{2.876467in}{2.532990in}}%
\pgfpathlineto{\pgfqpoint{2.862885in}{2.549067in}}%
\pgfpathlineto{\pgfqpoint{2.849295in}{2.565441in}}%
\pgfpathlineto{\pgfqpoint{2.835698in}{2.582112in}}%
\pgfpathlineto{\pgfqpoint{2.822093in}{2.599084in}}%
\pgfpathlineto{\pgfqpoint{2.813800in}{2.590990in}}%
\pgfpathlineto{\pgfqpoint{2.805499in}{2.583028in}}%
\pgfpathlineto{\pgfqpoint{2.797188in}{2.575198in}}%
\pgfpathlineto{\pgfqpoint{2.788867in}{2.567502in}}%
\pgfpathclose%
\pgfusepath{fill}%
\end{pgfscope}%
\begin{pgfscope}%
\pgfpathrectangle{\pgfqpoint{1.150000in}{0.150000in}}{\pgfqpoint{5.700000in}{5.700000in}}%
\pgfusepath{clip}%
\pgfsetbuttcap%
\pgfsetroundjoin%
\definecolor{currentfill}{rgb}{0.276194,0.190074,0.493001}%
\pgfsetfillcolor{currentfill}%
\pgfsetfillopacity{0.800000}%
\pgfsetlinewidth{0.000000pt}%
\definecolor{currentstroke}{rgb}{0.000000,0.000000,0.000000}%
\pgfsetstrokecolor{currentstroke}%
\pgfsetdash{}{0pt}%
\pgfpathmoveto{\pgfqpoint{2.897692in}{2.438721in}}%
\pgfpathlineto{\pgfqpoint{2.911266in}{2.423925in}}%
\pgfpathlineto{\pgfqpoint{2.924833in}{2.409411in}}%
\pgfpathlineto{\pgfqpoint{2.938396in}{2.395175in}}%
\pgfpathlineto{\pgfqpoint{2.951953in}{2.381217in}}%
\pgfpathlineto{\pgfqpoint{2.960204in}{2.389427in}}%
\pgfpathlineto{\pgfqpoint{2.968447in}{2.397747in}}%
\pgfpathlineto{\pgfqpoint{2.976682in}{2.406176in}}%
\pgfpathlineto{\pgfqpoint{2.984908in}{2.414711in}}%
\pgfpathlineto{\pgfqpoint{2.971372in}{2.428512in}}%
\pgfpathlineto{\pgfqpoint{2.957831in}{2.442589in}}%
\pgfpathlineto{\pgfqpoint{2.944285in}{2.456946in}}%
\pgfpathlineto{\pgfqpoint{2.930733in}{2.471583in}}%
\pgfpathlineto{\pgfqpoint{2.922486in}{2.463194in}}%
\pgfpathlineto{\pgfqpoint{2.914230in}{2.454920in}}%
\pgfpathlineto{\pgfqpoint{2.905965in}{2.446762in}}%
\pgfpathlineto{\pgfqpoint{2.897692in}{2.438721in}}%
\pgfpathclose%
\pgfusepath{fill}%
\end{pgfscope}%
\begin{pgfscope}%
\pgfpathrectangle{\pgfqpoint{1.150000in}{0.150000in}}{\pgfqpoint{5.700000in}{5.700000in}}%
\pgfusepath{clip}%
\pgfsetbuttcap%
\pgfsetroundjoin%
\definecolor{currentfill}{rgb}{0.281924,0.089666,0.412415}%
\pgfsetfillcolor{currentfill}%
\pgfsetfillopacity{0.800000}%
\pgfsetlinewidth{0.000000pt}%
\definecolor{currentstroke}{rgb}{0.000000,0.000000,0.000000}%
\pgfsetstrokecolor{currentstroke}%
\pgfsetdash{}{0pt}%
\pgfpathmoveto{\pgfqpoint{3.341331in}{2.199560in}}%
\pgfpathlineto{\pgfqpoint{3.354805in}{2.192188in}}%
\pgfpathlineto{\pgfqpoint{3.368281in}{2.185047in}}%
\pgfpathlineto{\pgfqpoint{3.381758in}{2.178137in}}%
\pgfpathlineto{\pgfqpoint{3.395236in}{2.171456in}}%
\pgfpathlineto{\pgfqpoint{3.403310in}{2.181235in}}%
\pgfpathlineto{\pgfqpoint{3.411379in}{2.191061in}}%
\pgfpathlineto{\pgfqpoint{3.419441in}{2.200933in}}%
\pgfpathlineto{\pgfqpoint{3.427497in}{2.210851in}}%
\pgfpathlineto{\pgfqpoint{3.414032in}{2.217442in}}%
\pgfpathlineto{\pgfqpoint{3.400569in}{2.224263in}}%
\pgfpathlineto{\pgfqpoint{3.387107in}{2.231314in}}%
\pgfpathlineto{\pgfqpoint{3.373646in}{2.238596in}}%
\pgfpathlineto{\pgfqpoint{3.365577in}{2.228756in}}%
\pgfpathlineto{\pgfqpoint{3.357501in}{2.218969in}}%
\pgfpathlineto{\pgfqpoint{3.349419in}{2.209237in}}%
\pgfpathlineto{\pgfqpoint{3.341331in}{2.199560in}}%
\pgfpathclose%
\pgfusepath{fill}%
\end{pgfscope}%
\begin{pgfscope}%
\pgfpathrectangle{\pgfqpoint{1.150000in}{0.150000in}}{\pgfqpoint{5.700000in}{5.700000in}}%
\pgfusepath{clip}%
\pgfsetbuttcap%
\pgfsetroundjoin%
\definecolor{currentfill}{rgb}{0.225863,0.330805,0.547314}%
\pgfsetfillcolor{currentfill}%
\pgfsetfillopacity{0.800000}%
\pgfsetlinewidth{0.000000pt}%
\definecolor{currentstroke}{rgb}{0.000000,0.000000,0.000000}%
\pgfsetstrokecolor{currentstroke}%
\pgfsetdash{}{0pt}%
\pgfpathmoveto{\pgfqpoint{4.735864in}{2.743553in}}%
\pgfpathlineto{\pgfqpoint{4.749702in}{2.747908in}}%
\pgfpathlineto{\pgfqpoint{4.763553in}{2.752446in}}%
\pgfpathlineto{\pgfqpoint{4.777417in}{2.757165in}}%
\pgfpathlineto{\pgfqpoint{4.791293in}{2.762067in}}%
\pgfpathlineto{\pgfqpoint{4.798890in}{2.770292in}}%
\pgfpathlineto{\pgfqpoint{4.806481in}{2.778548in}}%
\pgfpathlineto{\pgfqpoint{4.814068in}{2.786840in}}%
\pgfpathlineto{\pgfqpoint{4.821650in}{2.795172in}}%
\pgfpathlineto{\pgfqpoint{4.807786in}{2.790662in}}%
\pgfpathlineto{\pgfqpoint{4.793936in}{2.786333in}}%
\pgfpathlineto{\pgfqpoint{4.780098in}{2.782186in}}%
\pgfpathlineto{\pgfqpoint{4.766273in}{2.778221in}}%
\pgfpathlineto{\pgfqpoint{4.758678in}{2.769487in}}%
\pgfpathlineto{\pgfqpoint{4.751078in}{2.760800in}}%
\pgfpathlineto{\pgfqpoint{4.743473in}{2.752157in}}%
\pgfpathlineto{\pgfqpoint{4.735864in}{2.743553in}}%
\pgfpathclose%
\pgfusepath{fill}%
\end{pgfscope}%
\begin{pgfscope}%
\pgfpathrectangle{\pgfqpoint{1.150000in}{0.150000in}}{\pgfqpoint{5.700000in}{5.700000in}}%
\pgfusepath{clip}%
\pgfsetbuttcap%
\pgfsetroundjoin%
\definecolor{currentfill}{rgb}{0.252194,0.269783,0.531579}%
\pgfsetfillcolor{currentfill}%
\pgfsetfillopacity{0.800000}%
\pgfsetlinewidth{0.000000pt}%
\definecolor{currentstroke}{rgb}{0.000000,0.000000,0.000000}%
\pgfsetstrokecolor{currentstroke}%
\pgfsetdash{}{0pt}%
\pgfpathmoveto{\pgfqpoint{2.734267in}{2.639097in}}%
\pgfpathlineto{\pgfqpoint{2.747931in}{2.620733in}}%
\pgfpathlineto{\pgfqpoint{2.761585in}{2.602681in}}%
\pgfpathlineto{\pgfqpoint{2.775230in}{2.584938in}}%
\pgfpathlineto{\pgfqpoint{2.788867in}{2.567502in}}%
\pgfpathlineto{\pgfqpoint{2.797188in}{2.575198in}}%
\pgfpathlineto{\pgfqpoint{2.805499in}{2.583028in}}%
\pgfpathlineto{\pgfqpoint{2.813800in}{2.590990in}}%
\pgfpathlineto{\pgfqpoint{2.822093in}{2.599084in}}%
\pgfpathlineto{\pgfqpoint{2.808480in}{2.616359in}}%
\pgfpathlineto{\pgfqpoint{2.794859in}{2.633940in}}%
\pgfpathlineto{\pgfqpoint{2.781230in}{2.651829in}}%
\pgfpathlineto{\pgfqpoint{2.767591in}{2.670030in}}%
\pgfpathlineto{\pgfqpoint{2.759275in}{2.662087in}}%
\pgfpathlineto{\pgfqpoint{2.750949in}{2.654283in}}%
\pgfpathlineto{\pgfqpoint{2.742613in}{2.646619in}}%
\pgfpathlineto{\pgfqpoint{2.734267in}{2.639097in}}%
\pgfpathclose%
\pgfusepath{fill}%
\end{pgfscope}%
\begin{pgfscope}%
\pgfpathrectangle{\pgfqpoint{1.150000in}{0.150000in}}{\pgfqpoint{5.700000in}{5.700000in}}%
\pgfusepath{clip}%
\pgfsetbuttcap%
\pgfsetroundjoin%
\definecolor{currentfill}{rgb}{0.283229,0.120777,0.440584}%
\pgfsetfillcolor{currentfill}%
\pgfsetfillopacity{0.800000}%
\pgfsetlinewidth{0.000000pt}%
\definecolor{currentstroke}{rgb}{0.000000,0.000000,0.000000}%
\pgfsetstrokecolor{currentstroke}%
\pgfsetdash{}{0pt}%
\pgfpathmoveto{\pgfqpoint{3.793038in}{2.249475in}}%
\pgfpathlineto{\pgfqpoint{3.806554in}{2.247513in}}%
\pgfpathlineto{\pgfqpoint{3.820076in}{2.245757in}}%
\pgfpathlineto{\pgfqpoint{3.833605in}{2.244207in}}%
\pgfpathlineto{\pgfqpoint{3.847139in}{2.242861in}}%
\pgfpathlineto{\pgfqpoint{3.855063in}{2.253228in}}%
\pgfpathlineto{\pgfqpoint{3.862982in}{2.263597in}}%
\pgfpathlineto{\pgfqpoint{3.870896in}{2.273969in}}%
\pgfpathlineto{\pgfqpoint{3.878804in}{2.284345in}}%
\pgfpathlineto{\pgfqpoint{3.865278in}{2.285729in}}%
\pgfpathlineto{\pgfqpoint{3.851758in}{2.287317in}}%
\pgfpathlineto{\pgfqpoint{3.838245in}{2.289110in}}%
\pgfpathlineto{\pgfqpoint{3.824737in}{2.291109in}}%
\pgfpathlineto{\pgfqpoint{3.816820in}{2.280684in}}%
\pgfpathlineto{\pgfqpoint{3.808898in}{2.270271in}}%
\pgfpathlineto{\pgfqpoint{3.800971in}{2.259868in}}%
\pgfpathlineto{\pgfqpoint{3.793038in}{2.249475in}}%
\pgfpathclose%
\pgfusepath{fill}%
\end{pgfscope}%
\begin{pgfscope}%
\pgfpathrectangle{\pgfqpoint{1.150000in}{0.150000in}}{\pgfqpoint{5.700000in}{5.700000in}}%
\pgfusepath{clip}%
\pgfsetbuttcap%
\pgfsetroundjoin%
\definecolor{currentfill}{rgb}{0.216210,0.351535,0.550627}%
\pgfsetfillcolor{currentfill}%
\pgfsetfillopacity{0.800000}%
\pgfsetlinewidth{0.000000pt}%
\definecolor{currentstroke}{rgb}{0.000000,0.000000,0.000000}%
\pgfsetstrokecolor{currentstroke}%
\pgfsetdash{}{0pt}%
\pgfpathmoveto{\pgfqpoint{4.821650in}{2.795172in}}%
\pgfpathlineto{\pgfqpoint{4.835526in}{2.799863in}}%
\pgfpathlineto{\pgfqpoint{4.849416in}{2.804735in}}%
\pgfpathlineto{\pgfqpoint{4.863319in}{2.809788in}}%
\pgfpathlineto{\pgfqpoint{4.877236in}{2.815021in}}%
\pgfpathlineto{\pgfqpoint{4.884799in}{2.822986in}}%
\pgfpathlineto{\pgfqpoint{4.892357in}{2.830994in}}%
\pgfpathlineto{\pgfqpoint{4.899910in}{2.839048in}}%
\pgfpathlineto{\pgfqpoint{4.907459in}{2.847154in}}%
\pgfpathlineto{\pgfqpoint{4.893557in}{2.842344in}}%
\pgfpathlineto{\pgfqpoint{4.879669in}{2.837715in}}%
\pgfpathlineto{\pgfqpoint{4.865794in}{2.833266in}}%
\pgfpathlineto{\pgfqpoint{4.851932in}{2.828998in}}%
\pgfpathlineto{\pgfqpoint{4.844368in}{2.820457in}}%
\pgfpathlineto{\pgfqpoint{4.836800in}{2.811976in}}%
\pgfpathlineto{\pgfqpoint{4.829227in}{2.803549in}}%
\pgfpathlineto{\pgfqpoint{4.821650in}{2.795172in}}%
\pgfpathclose%
\pgfusepath{fill}%
\end{pgfscope}%
\begin{pgfscope}%
\pgfpathrectangle{\pgfqpoint{1.150000in}{0.150000in}}{\pgfqpoint{5.700000in}{5.700000in}}%
\pgfusepath{clip}%
\pgfsetbuttcap%
\pgfsetroundjoin%
\definecolor{currentfill}{rgb}{0.280255,0.165693,0.476498}%
\pgfsetfillcolor{currentfill}%
\pgfsetfillopacity{0.800000}%
\pgfsetlinewidth{0.000000pt}%
\definecolor{currentstroke}{rgb}{0.000000,0.000000,0.000000}%
\pgfsetstrokecolor{currentstroke}%
\pgfsetdash{}{0pt}%
\pgfpathmoveto{\pgfqpoint{2.951953in}{2.381217in}}%
\pgfpathlineto{\pgfqpoint{2.965505in}{2.367533in}}%
\pgfpathlineto{\pgfqpoint{2.979052in}{2.354121in}}%
\pgfpathlineto{\pgfqpoint{2.992596in}{2.340981in}}%
\pgfpathlineto{\pgfqpoint{3.006134in}{2.328108in}}%
\pgfpathlineto{\pgfqpoint{3.014365in}{2.336488in}}%
\pgfpathlineto{\pgfqpoint{3.022587in}{2.344969in}}%
\pgfpathlineto{\pgfqpoint{3.030801in}{2.353550in}}%
\pgfpathlineto{\pgfqpoint{3.039008in}{2.362231in}}%
\pgfpathlineto{\pgfqpoint{3.025489in}{2.374946in}}%
\pgfpathlineto{\pgfqpoint{3.011966in}{2.387930in}}%
\pgfpathlineto{\pgfqpoint{2.998439in}{2.401184in}}%
\pgfpathlineto{\pgfqpoint{2.984908in}{2.414711in}}%
\pgfpathlineto{\pgfqpoint{2.976682in}{2.406176in}}%
\pgfpathlineto{\pgfqpoint{2.968447in}{2.397747in}}%
\pgfpathlineto{\pgfqpoint{2.960204in}{2.389427in}}%
\pgfpathlineto{\pgfqpoint{2.951953in}{2.381217in}}%
\pgfpathclose%
\pgfusepath{fill}%
\end{pgfscope}%
\begin{pgfscope}%
\pgfpathrectangle{\pgfqpoint{1.150000in}{0.150000in}}{\pgfqpoint{5.700000in}{5.700000in}}%
\pgfusepath{clip}%
\pgfsetbuttcap%
\pgfsetroundjoin%
\definecolor{currentfill}{rgb}{0.281446,0.084320,0.407414}%
\pgfsetfillcolor{currentfill}%
\pgfsetfillopacity{0.800000}%
\pgfsetlinewidth{0.000000pt}%
\definecolor{currentstroke}{rgb}{0.000000,0.000000,0.000000}%
\pgfsetstrokecolor{currentstroke}%
\pgfsetdash{}{0pt}%
\pgfpathmoveto{\pgfqpoint{3.481379in}{2.186750in}}%
\pgfpathlineto{\pgfqpoint{3.494855in}{2.181285in}}%
\pgfpathlineto{\pgfqpoint{3.508334in}{2.176042in}}%
\pgfpathlineto{\pgfqpoint{3.521817in}{2.171019in}}%
\pgfpathlineto{\pgfqpoint{3.535302in}{2.166216in}}%
\pgfpathlineto{\pgfqpoint{3.543329in}{2.176319in}}%
\pgfpathlineto{\pgfqpoint{3.551350in}{2.186452in}}%
\pgfpathlineto{\pgfqpoint{3.559365in}{2.196614in}}%
\pgfpathlineto{\pgfqpoint{3.567374in}{2.206806in}}%
\pgfpathlineto{\pgfqpoint{3.553900in}{2.211552in}}%
\pgfpathlineto{\pgfqpoint{3.540429in}{2.216517in}}%
\pgfpathlineto{\pgfqpoint{3.526962in}{2.221702in}}%
\pgfpathlineto{\pgfqpoint{3.513497in}{2.227109in}}%
\pgfpathlineto{\pgfqpoint{3.505476in}{2.216963in}}%
\pgfpathlineto{\pgfqpoint{3.497449in}{2.206855in}}%
\pgfpathlineto{\pgfqpoint{3.489417in}{2.196784in}}%
\pgfpathlineto{\pgfqpoint{3.481379in}{2.186750in}}%
\pgfpathclose%
\pgfusepath{fill}%
\end{pgfscope}%
\begin{pgfscope}%
\pgfpathrectangle{\pgfqpoint{1.150000in}{0.150000in}}{\pgfqpoint{5.700000in}{5.700000in}}%
\pgfusepath{clip}%
\pgfsetbuttcap%
\pgfsetroundjoin%
\definecolor{currentfill}{rgb}{0.282910,0.105393,0.426902}%
\pgfsetfillcolor{currentfill}%
\pgfsetfillopacity{0.800000}%
\pgfsetlinewidth{0.000000pt}%
\definecolor{currentstroke}{rgb}{0.000000,0.000000,0.000000}%
\pgfsetstrokecolor{currentstroke}%
\pgfsetdash{}{0pt}%
\pgfpathmoveto{\pgfqpoint{3.201013in}{2.229870in}}%
\pgfpathlineto{\pgfqpoint{3.214502in}{2.220472in}}%
\pgfpathlineto{\pgfqpoint{3.227990in}{2.211317in}}%
\pgfpathlineto{\pgfqpoint{3.241477in}{2.202403in}}%
\pgfpathlineto{\pgfqpoint{3.254965in}{2.193730in}}%
\pgfpathlineto{\pgfqpoint{3.263094in}{2.203048in}}%
\pgfpathlineto{\pgfqpoint{3.271216in}{2.212432in}}%
\pgfpathlineto{\pgfqpoint{3.279331in}{2.221880in}}%
\pgfpathlineto{\pgfqpoint{3.287440in}{2.231392in}}%
\pgfpathlineto{\pgfqpoint{3.273969in}{2.239944in}}%
\pgfpathlineto{\pgfqpoint{3.260497in}{2.248735in}}%
\pgfpathlineto{\pgfqpoint{3.247024in}{2.257768in}}%
\pgfpathlineto{\pgfqpoint{3.233551in}{2.267044in}}%
\pgfpathlineto{\pgfqpoint{3.225427in}{2.257642in}}%
\pgfpathlineto{\pgfqpoint{3.217296in}{2.248312in}}%
\pgfpathlineto{\pgfqpoint{3.209158in}{2.239054in}}%
\pgfpathlineto{\pgfqpoint{3.201013in}{2.229870in}}%
\pgfpathclose%
\pgfusepath{fill}%
\end{pgfscope}%
\begin{pgfscope}%
\pgfpathrectangle{\pgfqpoint{1.150000in}{0.150000in}}{\pgfqpoint{5.700000in}{5.700000in}}%
\pgfusepath{clip}%
\pgfsetbuttcap%
\pgfsetroundjoin%
\definecolor{currentfill}{rgb}{0.136408,0.541173,0.554483}%
\pgfsetfillcolor{currentfill}%
\pgfsetfillopacity{0.800000}%
\pgfsetlinewidth{0.000000pt}%
\definecolor{currentstroke}{rgb}{0.000000,0.000000,0.000000}%
\pgfsetstrokecolor{currentstroke}%
\pgfsetdash{}{0pt}%
\pgfpathmoveto{\pgfqpoint{5.709952in}{3.355879in}}%
\pgfpathlineto{\pgfqpoint{5.724178in}{3.361129in}}%
\pgfpathlineto{\pgfqpoint{5.738421in}{3.366549in}}%
\pgfpathlineto{\pgfqpoint{5.752680in}{3.372139in}}%
\pgfpathlineto{\pgfqpoint{5.766955in}{3.377897in}}%
\pgfpathlineto{\pgfqpoint{5.774200in}{3.385832in}}%
\pgfpathlineto{\pgfqpoint{5.781449in}{3.394027in}}%
\pgfpathlineto{\pgfqpoint{5.788701in}{3.402491in}}%
\pgfpathlineto{\pgfqpoint{5.774451in}{3.397314in}}%
\pgfpathlineto{\pgfqpoint{5.760217in}{3.392305in}}%
\pgfpathlineto{\pgfqpoint{5.745999in}{3.387465in}}%
\pgfpathlineto{\pgfqpoint{5.731796in}{3.382794in}}%
\pgfpathlineto{\pgfqpoint{5.724511in}{3.373549in}}%
\pgfpathlineto{\pgfqpoint{5.717230in}{3.364580in}}%
\pgfpathlineto{\pgfqpoint{5.709952in}{3.355879in}}%
\pgfpathclose%
\pgfusepath{fill}%
\end{pgfscope}%
\begin{pgfscope}%
\pgfpathrectangle{\pgfqpoint{1.150000in}{0.150000in}}{\pgfqpoint{5.700000in}{5.700000in}}%
\pgfusepath{clip}%
\pgfsetbuttcap%
\pgfsetroundjoin%
\definecolor{currentfill}{rgb}{0.206756,0.371758,0.553117}%
\pgfsetfillcolor{currentfill}%
\pgfsetfillopacity{0.800000}%
\pgfsetlinewidth{0.000000pt}%
\definecolor{currentstroke}{rgb}{0.000000,0.000000,0.000000}%
\pgfsetstrokecolor{currentstroke}%
\pgfsetdash{}{0pt}%
\pgfpathmoveto{\pgfqpoint{4.907459in}{2.847154in}}%
\pgfpathlineto{\pgfqpoint{4.921375in}{2.852143in}}%
\pgfpathlineto{\pgfqpoint{4.935304in}{2.857311in}}%
\pgfpathlineto{\pgfqpoint{4.949247in}{2.862659in}}%
\pgfpathlineto{\pgfqpoint{4.963204in}{2.868187in}}%
\pgfpathlineto{\pgfqpoint{4.970733in}{2.875905in}}%
\pgfpathlineto{\pgfqpoint{4.978257in}{2.883677in}}%
\pgfpathlineto{\pgfqpoint{4.985777in}{2.891509in}}%
\pgfpathlineto{\pgfqpoint{4.993293in}{2.899405in}}%
\pgfpathlineto{\pgfqpoint{4.979352in}{2.894335in}}%
\pgfpathlineto{\pgfqpoint{4.965425in}{2.889443in}}%
\pgfpathlineto{\pgfqpoint{4.951512in}{2.884730in}}%
\pgfpathlineto{\pgfqpoint{4.937612in}{2.880196in}}%
\pgfpathlineto{\pgfqpoint{4.930080in}{2.871832in}}%
\pgfpathlineto{\pgfqpoint{4.922544in}{2.863541in}}%
\pgfpathlineto{\pgfqpoint{4.915004in}{2.855316in}}%
\pgfpathlineto{\pgfqpoint{4.907459in}{2.847154in}}%
\pgfpathclose%
\pgfusepath{fill}%
\end{pgfscope}%
\begin{pgfscope}%
\pgfpathrectangle{\pgfqpoint{1.150000in}{0.150000in}}{\pgfqpoint{5.700000in}{5.700000in}}%
\pgfusepath{clip}%
\pgfsetbuttcap%
\pgfsetroundjoin%
\definecolor{currentfill}{rgb}{0.239346,0.300855,0.540844}%
\pgfsetfillcolor{currentfill}%
\pgfsetfillopacity{0.800000}%
\pgfsetlinewidth{0.000000pt}%
\definecolor{currentstroke}{rgb}{0.000000,0.000000,0.000000}%
\pgfsetstrokecolor{currentstroke}%
\pgfsetdash{}{0pt}%
\pgfpathmoveto{\pgfqpoint{2.679517in}{2.715728in}}%
\pgfpathlineto{\pgfqpoint{2.693220in}{2.696088in}}%
\pgfpathlineto{\pgfqpoint{2.706912in}{2.676771in}}%
\pgfpathlineto{\pgfqpoint{2.720595in}{2.657775in}}%
\pgfpathlineto{\pgfqpoint{2.734267in}{2.639097in}}%
\pgfpathlineto{\pgfqpoint{2.742613in}{2.646619in}}%
\pgfpathlineto{\pgfqpoint{2.750949in}{2.654283in}}%
\pgfpathlineto{\pgfqpoint{2.759275in}{2.662087in}}%
\pgfpathlineto{\pgfqpoint{2.767591in}{2.670030in}}%
\pgfpathlineto{\pgfqpoint{2.753944in}{2.688546in}}%
\pgfpathlineto{\pgfqpoint{2.740287in}{2.707378in}}%
\pgfpathlineto{\pgfqpoint{2.726621in}{2.726531in}}%
\pgfpathlineto{\pgfqpoint{2.712944in}{2.746007in}}%
\pgfpathlineto{\pgfqpoint{2.704603in}{2.738215in}}%
\pgfpathlineto{\pgfqpoint{2.696251in}{2.730570in}}%
\pgfpathlineto{\pgfqpoint{2.687889in}{2.723074in}}%
\pgfpathlineto{\pgfqpoint{2.679517in}{2.715728in}}%
\pgfpathclose%
\pgfusepath{fill}%
\end{pgfscope}%
\begin{pgfscope}%
\pgfpathrectangle{\pgfqpoint{1.150000in}{0.150000in}}{\pgfqpoint{5.700000in}{5.700000in}}%
\pgfusepath{clip}%
\pgfsetbuttcap%
\pgfsetroundjoin%
\definecolor{currentfill}{rgb}{0.282910,0.105393,0.426902}%
\pgfsetfillcolor{currentfill}%
\pgfsetfillopacity{0.800000}%
\pgfsetlinewidth{0.000000pt}%
\definecolor{currentstroke}{rgb}{0.000000,0.000000,0.000000}%
\pgfsetstrokecolor{currentstroke}%
\pgfsetdash{}{0pt}%
\pgfpathmoveto{\pgfqpoint{3.707213in}{2.217893in}}%
\pgfpathlineto{\pgfqpoint{3.720716in}{2.215103in}}%
\pgfpathlineto{\pgfqpoint{3.734224in}{2.212523in}}%
\pgfpathlineto{\pgfqpoint{3.747738in}{2.210152in}}%
\pgfpathlineto{\pgfqpoint{3.761257in}{2.207988in}}%
\pgfpathlineto{\pgfqpoint{3.769210in}{2.218348in}}%
\pgfpathlineto{\pgfqpoint{3.777158in}{2.228716in}}%
\pgfpathlineto{\pgfqpoint{3.785101in}{2.239091in}}%
\pgfpathlineto{\pgfqpoint{3.793038in}{2.249475in}}%
\pgfpathlineto{\pgfqpoint{3.779528in}{2.251644in}}%
\pgfpathlineto{\pgfqpoint{3.766024in}{2.254022in}}%
\pgfpathlineto{\pgfqpoint{3.752525in}{2.256608in}}%
\pgfpathlineto{\pgfqpoint{3.739031in}{2.259403in}}%
\pgfpathlineto{\pgfqpoint{3.731084in}{2.249002in}}%
\pgfpathlineto{\pgfqpoint{3.723132in}{2.238617in}}%
\pgfpathlineto{\pgfqpoint{3.715175in}{2.228247in}}%
\pgfpathlineto{\pgfqpoint{3.707213in}{2.217893in}}%
\pgfpathclose%
\pgfusepath{fill}%
\end{pgfscope}%
\begin{pgfscope}%
\pgfpathrectangle{\pgfqpoint{1.150000in}{0.150000in}}{\pgfqpoint{5.700000in}{5.700000in}}%
\pgfusepath{clip}%
\pgfsetbuttcap%
\pgfsetroundjoin%
\definecolor{currentfill}{rgb}{0.282290,0.145912,0.461510}%
\pgfsetfillcolor{currentfill}%
\pgfsetfillopacity{0.800000}%
\pgfsetlinewidth{0.000000pt}%
\definecolor{currentstroke}{rgb}{0.000000,0.000000,0.000000}%
\pgfsetstrokecolor{currentstroke}%
\pgfsetdash{}{0pt}%
\pgfpathmoveto{\pgfqpoint{3.006134in}{2.328108in}}%
\pgfpathlineto{\pgfqpoint{3.019669in}{2.315503in}}%
\pgfpathlineto{\pgfqpoint{3.033201in}{2.303161in}}%
\pgfpathlineto{\pgfqpoint{3.046729in}{2.291083in}}%
\pgfpathlineto{\pgfqpoint{3.060253in}{2.279264in}}%
\pgfpathlineto{\pgfqpoint{3.068463in}{2.287812in}}%
\pgfpathlineto{\pgfqpoint{3.076666in}{2.296453in}}%
\pgfpathlineto{\pgfqpoint{3.084861in}{2.305186in}}%
\pgfpathlineto{\pgfqpoint{3.093048in}{2.314011in}}%
\pgfpathlineto{\pgfqpoint{3.079543in}{2.325673in}}%
\pgfpathlineto{\pgfqpoint{3.066034in}{2.337596in}}%
\pgfpathlineto{\pgfqpoint{3.052523in}{2.349781in}}%
\pgfpathlineto{\pgfqpoint{3.039008in}{2.362231in}}%
\pgfpathlineto{\pgfqpoint{3.030801in}{2.353550in}}%
\pgfpathlineto{\pgfqpoint{3.022587in}{2.344969in}}%
\pgfpathlineto{\pgfqpoint{3.014365in}{2.336488in}}%
\pgfpathlineto{\pgfqpoint{3.006134in}{2.328108in}}%
\pgfpathclose%
\pgfusepath{fill}%
\end{pgfscope}%
\begin{pgfscope}%
\pgfpathrectangle{\pgfqpoint{1.150000in}{0.150000in}}{\pgfqpoint{5.700000in}{5.700000in}}%
\pgfusepath{clip}%
\pgfsetbuttcap%
\pgfsetroundjoin%
\definecolor{currentfill}{rgb}{0.197636,0.391528,0.554969}%
\pgfsetfillcolor{currentfill}%
\pgfsetfillopacity{0.800000}%
\pgfsetlinewidth{0.000000pt}%
\definecolor{currentstroke}{rgb}{0.000000,0.000000,0.000000}%
\pgfsetstrokecolor{currentstroke}%
\pgfsetdash{}{0pt}%
\pgfpathmoveto{\pgfqpoint{4.993293in}{2.899405in}}%
\pgfpathlineto{\pgfqpoint{5.007248in}{2.904654in}}%
\pgfpathlineto{\pgfqpoint{5.021216in}{2.910082in}}%
\pgfpathlineto{\pgfqpoint{5.035199in}{2.915688in}}%
\pgfpathlineto{\pgfqpoint{5.049197in}{2.921472in}}%
\pgfpathlineto{\pgfqpoint{5.056691in}{2.928961in}}%
\pgfpathlineto{\pgfqpoint{5.064182in}{2.936517in}}%
\pgfpathlineto{\pgfqpoint{5.071668in}{2.944147in}}%
\pgfpathlineto{\pgfqpoint{5.079150in}{2.951856in}}%
\pgfpathlineto{\pgfqpoint{5.065171in}{2.946562in}}%
\pgfpathlineto{\pgfqpoint{5.051205in}{2.941445in}}%
\pgfpathlineto{\pgfqpoint{5.037254in}{2.936506in}}%
\pgfpathlineto{\pgfqpoint{5.023316in}{2.931744in}}%
\pgfpathlineto{\pgfqpoint{5.015816in}{2.923535in}}%
\pgfpathlineto{\pgfqpoint{5.008312in}{2.915413in}}%
\pgfpathlineto{\pgfqpoint{5.000805in}{2.907371in}}%
\pgfpathlineto{\pgfqpoint{4.993293in}{2.899405in}}%
\pgfpathclose%
\pgfusepath{fill}%
\end{pgfscope}%
\begin{pgfscope}%
\pgfpathrectangle{\pgfqpoint{1.150000in}{0.150000in}}{\pgfqpoint{5.700000in}{5.700000in}}%
\pgfusepath{clip}%
\pgfsetbuttcap%
\pgfsetroundjoin%
\definecolor{currentfill}{rgb}{0.188923,0.410910,0.556326}%
\pgfsetfillcolor{currentfill}%
\pgfsetfillopacity{0.800000}%
\pgfsetlinewidth{0.000000pt}%
\definecolor{currentstroke}{rgb}{0.000000,0.000000,0.000000}%
\pgfsetstrokecolor{currentstroke}%
\pgfsetdash{}{0pt}%
\pgfpathmoveto{\pgfqpoint{5.079150in}{2.951856in}}%
\pgfpathlineto{\pgfqpoint{5.093144in}{2.957328in}}%
\pgfpathlineto{\pgfqpoint{5.107153in}{2.962977in}}%
\pgfpathlineto{\pgfqpoint{5.121176in}{2.968803in}}%
\pgfpathlineto{\pgfqpoint{5.135214in}{2.974806in}}%
\pgfpathlineto{\pgfqpoint{5.142674in}{2.982089in}}%
\pgfpathlineto{\pgfqpoint{5.150130in}{2.989455in}}%
\pgfpathlineto{\pgfqpoint{5.157583in}{2.996910in}}%
\pgfpathlineto{\pgfqpoint{5.165032in}{3.004459in}}%
\pgfpathlineto{\pgfqpoint{5.151013in}{2.998978in}}%
\pgfpathlineto{\pgfqpoint{5.137009in}{2.993674in}}%
\pgfpathlineto{\pgfqpoint{5.123020in}{2.988546in}}%
\pgfpathlineto{\pgfqpoint{5.109044in}{2.983594in}}%
\pgfpathlineto{\pgfqpoint{5.101576in}{2.975513in}}%
\pgfpathlineto{\pgfqpoint{5.094104in}{2.967533in}}%
\pgfpathlineto{\pgfqpoint{5.086629in}{2.959650in}}%
\pgfpathlineto{\pgfqpoint{5.079150in}{2.951856in}}%
\pgfpathclose%
\pgfusepath{fill}%
\end{pgfscope}%
\begin{pgfscope}%
\pgfpathrectangle{\pgfqpoint{1.150000in}{0.150000in}}{\pgfqpoint{5.700000in}{5.700000in}}%
\pgfusepath{clip}%
\pgfsetbuttcap%
\pgfsetroundjoin%
\definecolor{currentfill}{rgb}{0.223925,0.334994,0.548053}%
\pgfsetfillcolor{currentfill}%
\pgfsetfillopacity{0.800000}%
\pgfsetlinewidth{0.000000pt}%
\definecolor{currentstroke}{rgb}{0.000000,0.000000,0.000000}%
\pgfsetstrokecolor{currentstroke}%
\pgfsetdash{}{0pt}%
\pgfpathmoveto{\pgfqpoint{2.624596in}{2.797587in}}%
\pgfpathlineto{\pgfqpoint{2.638343in}{2.776621in}}%
\pgfpathlineto{\pgfqpoint{2.652079in}{2.755992in}}%
\pgfpathlineto{\pgfqpoint{2.665803in}{2.735695in}}%
\pgfpathlineto{\pgfqpoint{2.679517in}{2.715728in}}%
\pgfpathlineto{\pgfqpoint{2.687889in}{2.723074in}}%
\pgfpathlineto{\pgfqpoint{2.696251in}{2.730570in}}%
\pgfpathlineto{\pgfqpoint{2.704603in}{2.738215in}}%
\pgfpathlineto{\pgfqpoint{2.712944in}{2.746007in}}%
\pgfpathlineto{\pgfqpoint{2.699257in}{2.765809in}}%
\pgfpathlineto{\pgfqpoint{2.685560in}{2.785940in}}%
\pgfpathlineto{\pgfqpoint{2.671851in}{2.806404in}}%
\pgfpathlineto{\pgfqpoint{2.658131in}{2.827204in}}%
\pgfpathlineto{\pgfqpoint{2.649763in}{2.819565in}}%
\pgfpathlineto{\pgfqpoint{2.641385in}{2.812082in}}%
\pgfpathlineto{\pgfqpoint{2.632996in}{2.804756in}}%
\pgfpathlineto{\pgfqpoint{2.624596in}{2.797587in}}%
\pgfpathclose%
\pgfusepath{fill}%
\end{pgfscope}%
\begin{pgfscope}%
\pgfpathrectangle{\pgfqpoint{1.150000in}{0.150000in}}{\pgfqpoint{5.700000in}{5.700000in}}%
\pgfusepath{clip}%
\pgfsetbuttcap%
\pgfsetroundjoin%
\definecolor{currentfill}{rgb}{0.282327,0.094955,0.417331}%
\pgfsetfillcolor{currentfill}%
\pgfsetfillopacity{0.800000}%
\pgfsetlinewidth{0.000000pt}%
\definecolor{currentstroke}{rgb}{0.000000,0.000000,0.000000}%
\pgfsetstrokecolor{currentstroke}%
\pgfsetdash{}{0pt}%
\pgfpathmoveto{\pgfqpoint{3.621309in}{2.189998in}}%
\pgfpathlineto{\pgfqpoint{3.634802in}{2.186335in}}%
\pgfpathlineto{\pgfqpoint{3.648300in}{2.182885in}}%
\pgfpathlineto{\pgfqpoint{3.661803in}{2.179647in}}%
\pgfpathlineto{\pgfqpoint{3.675311in}{2.176621in}}%
\pgfpathlineto{\pgfqpoint{3.683294in}{2.186918in}}%
\pgfpathlineto{\pgfqpoint{3.691272in}{2.197229in}}%
\pgfpathlineto{\pgfqpoint{3.699245in}{2.207554in}}%
\pgfpathlineto{\pgfqpoint{3.707213in}{2.217893in}}%
\pgfpathlineto{\pgfqpoint{3.693715in}{2.220894in}}%
\pgfpathlineto{\pgfqpoint{3.680222in}{2.224105in}}%
\pgfpathlineto{\pgfqpoint{3.666734in}{2.227530in}}%
\pgfpathlineto{\pgfqpoint{3.653251in}{2.231167in}}%
\pgfpathlineto{\pgfqpoint{3.645273in}{2.220842in}}%
\pgfpathlineto{\pgfqpoint{3.637290in}{2.210539in}}%
\pgfpathlineto{\pgfqpoint{3.629302in}{2.200258in}}%
\pgfpathlineto{\pgfqpoint{3.621309in}{2.189998in}}%
\pgfpathclose%
\pgfusepath{fill}%
\end{pgfscope}%
\begin{pgfscope}%
\pgfpathrectangle{\pgfqpoint{1.150000in}{0.150000in}}{\pgfqpoint{5.700000in}{5.700000in}}%
\pgfusepath{clip}%
\pgfsetbuttcap%
\pgfsetroundjoin%
\definecolor{currentfill}{rgb}{0.180629,0.429975,0.557282}%
\pgfsetfillcolor{currentfill}%
\pgfsetfillopacity{0.800000}%
\pgfsetlinewidth{0.000000pt}%
\definecolor{currentstroke}{rgb}{0.000000,0.000000,0.000000}%
\pgfsetstrokecolor{currentstroke}%
\pgfsetdash{}{0pt}%
\pgfpathmoveto{\pgfqpoint{5.165032in}{3.004459in}}%
\pgfpathlineto{\pgfqpoint{5.179065in}{3.010116in}}%
\pgfpathlineto{\pgfqpoint{5.193113in}{3.015950in}}%
\pgfpathlineto{\pgfqpoint{5.207176in}{3.021959in}}%
\pgfpathlineto{\pgfqpoint{5.221255in}{3.028144in}}%
\pgfpathlineto{\pgfqpoint{5.228680in}{3.035251in}}%
\pgfpathlineto{\pgfqpoint{5.236102in}{3.042457in}}%
\pgfpathlineto{\pgfqpoint{5.243521in}{3.049768in}}%
\pgfpathlineto{\pgfqpoint{5.250937in}{3.057191in}}%
\pgfpathlineto{\pgfqpoint{5.236880in}{3.051560in}}%
\pgfpathlineto{\pgfqpoint{5.222838in}{3.046106in}}%
\pgfpathlineto{\pgfqpoint{5.208810in}{3.040826in}}%
\pgfpathlineto{\pgfqpoint{5.194797in}{3.035722in}}%
\pgfpathlineto{\pgfqpoint{5.187360in}{3.027734in}}%
\pgfpathlineto{\pgfqpoint{5.179920in}{3.019865in}}%
\pgfpathlineto{\pgfqpoint{5.172478in}{3.012109in}}%
\pgfpathlineto{\pgfqpoint{5.165032in}{3.004459in}}%
\pgfpathclose%
\pgfusepath{fill}%
\end{pgfscope}%
\begin{pgfscope}%
\pgfpathrectangle{\pgfqpoint{1.150000in}{0.150000in}}{\pgfqpoint{5.700000in}{5.700000in}}%
\pgfusepath{clip}%
\pgfsetbuttcap%
\pgfsetroundjoin%
\definecolor{currentfill}{rgb}{0.283187,0.125848,0.444960}%
\pgfsetfillcolor{currentfill}%
\pgfsetfillopacity{0.800000}%
\pgfsetlinewidth{0.000000pt}%
\definecolor{currentstroke}{rgb}{0.000000,0.000000,0.000000}%
\pgfsetstrokecolor{currentstroke}%
\pgfsetdash{}{0pt}%
\pgfpathmoveto{\pgfqpoint{3.060253in}{2.279264in}}%
\pgfpathlineto{\pgfqpoint{3.073775in}{2.267705in}}%
\pgfpathlineto{\pgfqpoint{3.087294in}{2.256403in}}%
\pgfpathlineto{\pgfqpoint{3.100810in}{2.245356in}}%
\pgfpathlineto{\pgfqpoint{3.114324in}{2.234562in}}%
\pgfpathlineto{\pgfqpoint{3.122516in}{2.243276in}}%
\pgfpathlineto{\pgfqpoint{3.130699in}{2.252076in}}%
\pgfpathlineto{\pgfqpoint{3.138876in}{2.260961in}}%
\pgfpathlineto{\pgfqpoint{3.147045in}{2.269928in}}%
\pgfpathlineto{\pgfqpoint{3.133549in}{2.280567in}}%
\pgfpathlineto{\pgfqpoint{3.120051in}{2.291460in}}%
\pgfpathlineto{\pgfqpoint{3.106551in}{2.302607in}}%
\pgfpathlineto{\pgfqpoint{3.093048in}{2.314011in}}%
\pgfpathlineto{\pgfqpoint{3.084861in}{2.305186in}}%
\pgfpathlineto{\pgfqpoint{3.076666in}{2.296453in}}%
\pgfpathlineto{\pgfqpoint{3.068463in}{2.287812in}}%
\pgfpathlineto{\pgfqpoint{3.060253in}{2.279264in}}%
\pgfpathclose%
\pgfusepath{fill}%
\end{pgfscope}%
\begin{pgfscope}%
\pgfpathrectangle{\pgfqpoint{1.150000in}{0.150000in}}{\pgfqpoint{5.700000in}{5.700000in}}%
\pgfusepath{clip}%
\pgfsetbuttcap%
\pgfsetroundjoin%
\definecolor{currentfill}{rgb}{0.281446,0.084320,0.407414}%
\pgfsetfillcolor{currentfill}%
\pgfsetfillopacity{0.800000}%
\pgfsetlinewidth{0.000000pt}%
\definecolor{currentstroke}{rgb}{0.000000,0.000000,0.000000}%
\pgfsetstrokecolor{currentstroke}%
\pgfsetdash{}{0pt}%
\pgfpathmoveto{\pgfqpoint{3.395236in}{2.171456in}}%
\pgfpathlineto{\pgfqpoint{3.408716in}{2.165003in}}%
\pgfpathlineto{\pgfqpoint{3.422198in}{2.158777in}}%
\pgfpathlineto{\pgfqpoint{3.435682in}{2.152776in}}%
\pgfpathlineto{\pgfqpoint{3.449168in}{2.146999in}}%
\pgfpathlineto{\pgfqpoint{3.457230in}{2.156878in}}%
\pgfpathlineto{\pgfqpoint{3.465285in}{2.166797in}}%
\pgfpathlineto{\pgfqpoint{3.473335in}{2.176754in}}%
\pgfpathlineto{\pgfqpoint{3.481379in}{2.186750in}}%
\pgfpathlineto{\pgfqpoint{3.467905in}{2.192438in}}%
\pgfpathlineto{\pgfqpoint{3.454434in}{2.198350in}}%
\pgfpathlineto{\pgfqpoint{3.440964in}{2.204487in}}%
\pgfpathlineto{\pgfqpoint{3.427497in}{2.210851in}}%
\pgfpathlineto{\pgfqpoint{3.419441in}{2.200933in}}%
\pgfpathlineto{\pgfqpoint{3.411379in}{2.191061in}}%
\pgfpathlineto{\pgfqpoint{3.403310in}{2.181235in}}%
\pgfpathlineto{\pgfqpoint{3.395236in}{2.171456in}}%
\pgfpathclose%
\pgfusepath{fill}%
\end{pgfscope}%
\begin{pgfscope}%
\pgfpathrectangle{\pgfqpoint{1.150000in}{0.150000in}}{\pgfqpoint{5.700000in}{5.700000in}}%
\pgfusepath{clip}%
\pgfsetbuttcap%
\pgfsetroundjoin%
\definecolor{currentfill}{rgb}{0.281924,0.089666,0.412415}%
\pgfsetfillcolor{currentfill}%
\pgfsetfillopacity{0.800000}%
\pgfsetlinewidth{0.000000pt}%
\definecolor{currentstroke}{rgb}{0.000000,0.000000,0.000000}%
\pgfsetstrokecolor{currentstroke}%
\pgfsetdash{}{0pt}%
\pgfpathmoveto{\pgfqpoint{3.254965in}{2.193730in}}%
\pgfpathlineto{\pgfqpoint{3.268452in}{2.185296in}}%
\pgfpathlineto{\pgfqpoint{3.281939in}{2.177099in}}%
\pgfpathlineto{\pgfqpoint{3.295427in}{2.169138in}}%
\pgfpathlineto{\pgfqpoint{3.308915in}{2.161412in}}%
\pgfpathlineto{\pgfqpoint{3.317029in}{2.170863in}}%
\pgfpathlineto{\pgfqpoint{3.325136in}{2.180372in}}%
\pgfpathlineto{\pgfqpoint{3.333237in}{2.189938in}}%
\pgfpathlineto{\pgfqpoint{3.341331in}{2.199560in}}%
\pgfpathlineto{\pgfqpoint{3.327857in}{2.207165in}}%
\pgfpathlineto{\pgfqpoint{3.314385in}{2.215005in}}%
\pgfpathlineto{\pgfqpoint{3.300912in}{2.223080in}}%
\pgfpathlineto{\pgfqpoint{3.287440in}{2.231392in}}%
\pgfpathlineto{\pgfqpoint{3.279331in}{2.221880in}}%
\pgfpathlineto{\pgfqpoint{3.271216in}{2.212432in}}%
\pgfpathlineto{\pgfqpoint{3.263094in}{2.203048in}}%
\pgfpathlineto{\pgfqpoint{3.254965in}{2.193730in}}%
\pgfpathclose%
\pgfusepath{fill}%
\end{pgfscope}%
\begin{pgfscope}%
\pgfpathrectangle{\pgfqpoint{1.150000in}{0.150000in}}{\pgfqpoint{5.700000in}{5.700000in}}%
\pgfusepath{clip}%
\pgfsetbuttcap%
\pgfsetroundjoin%
\definecolor{currentfill}{rgb}{0.172719,0.448791,0.557885}%
\pgfsetfillcolor{currentfill}%
\pgfsetfillopacity{0.800000}%
\pgfsetlinewidth{0.000000pt}%
\definecolor{currentstroke}{rgb}{0.000000,0.000000,0.000000}%
\pgfsetstrokecolor{currentstroke}%
\pgfsetdash{}{0pt}%
\pgfpathmoveto{\pgfqpoint{5.250937in}{3.057191in}}%
\pgfpathlineto{\pgfqpoint{5.265009in}{3.062996in}}%
\pgfpathlineto{\pgfqpoint{5.279097in}{3.068976in}}%
\pgfpathlineto{\pgfqpoint{5.293200in}{3.075132in}}%
\pgfpathlineto{\pgfqpoint{5.307318in}{3.081462in}}%
\pgfpathlineto{\pgfqpoint{5.314709in}{3.088427in}}%
\pgfpathlineto{\pgfqpoint{5.322098in}{3.095509in}}%
\pgfpathlineto{\pgfqpoint{5.329484in}{3.102715in}}%
\pgfpathlineto{\pgfqpoint{5.336867in}{3.110050in}}%
\pgfpathlineto{\pgfqpoint{5.322771in}{3.104307in}}%
\pgfpathlineto{\pgfqpoint{5.308691in}{3.098739in}}%
\pgfpathlineto{\pgfqpoint{5.294626in}{3.093345in}}%
\pgfpathlineto{\pgfqpoint{5.280576in}{3.088125in}}%
\pgfpathlineto{\pgfqpoint{5.273169in}{3.080192in}}%
\pgfpathlineto{\pgfqpoint{5.265761in}{3.072396in}}%
\pgfpathlineto{\pgfqpoint{5.258350in}{3.064731in}}%
\pgfpathlineto{\pgfqpoint{5.250937in}{3.057191in}}%
\pgfpathclose%
\pgfusepath{fill}%
\end{pgfscope}%
\begin{pgfscope}%
\pgfpathrectangle{\pgfqpoint{1.150000in}{0.150000in}}{\pgfqpoint{5.700000in}{5.700000in}}%
\pgfusepath{clip}%
\pgfsetbuttcap%
\pgfsetroundjoin%
\definecolor{currentfill}{rgb}{0.165117,0.467423,0.558141}%
\pgfsetfillcolor{currentfill}%
\pgfsetfillopacity{0.800000}%
\pgfsetlinewidth{0.000000pt}%
\definecolor{currentstroke}{rgb}{0.000000,0.000000,0.000000}%
\pgfsetstrokecolor{currentstroke}%
\pgfsetdash{}{0pt}%
\pgfpathmoveto{\pgfqpoint{5.336867in}{3.110050in}}%
\pgfpathlineto{\pgfqpoint{5.350978in}{3.115967in}}%
\pgfpathlineto{\pgfqpoint{5.365104in}{3.122057in}}%
\pgfpathlineto{\pgfqpoint{5.379246in}{3.128322in}}%
\pgfpathlineto{\pgfqpoint{5.393404in}{3.134761in}}%
\pgfpathlineto{\pgfqpoint{5.400762in}{3.141625in}}%
\pgfpathlineto{\pgfqpoint{5.408117in}{3.148624in}}%
\pgfpathlineto{\pgfqpoint{5.415471in}{3.155767in}}%
\pgfpathlineto{\pgfqpoint{5.422823in}{3.163060in}}%
\pgfpathlineto{\pgfqpoint{5.408689in}{3.157241in}}%
\pgfpathlineto{\pgfqpoint{5.394572in}{3.151596in}}%
\pgfpathlineto{\pgfqpoint{5.380469in}{3.146124in}}%
\pgfpathlineto{\pgfqpoint{5.366382in}{3.140825in}}%
\pgfpathlineto{\pgfqpoint{5.359006in}{3.132903in}}%
\pgfpathlineto{\pgfqpoint{5.351628in}{3.125137in}}%
\pgfpathlineto{\pgfqpoint{5.344248in}{3.117522in}}%
\pgfpathlineto{\pgfqpoint{5.336867in}{3.110050in}}%
\pgfpathclose%
\pgfusepath{fill}%
\end{pgfscope}%
\begin{pgfscope}%
\pgfpathrectangle{\pgfqpoint{1.150000in}{0.150000in}}{\pgfqpoint{5.700000in}{5.700000in}}%
\pgfusepath{clip}%
\pgfsetbuttcap%
\pgfsetroundjoin%
\definecolor{currentfill}{rgb}{0.278012,0.180367,0.486697}%
\pgfsetfillcolor{currentfill}%
\pgfsetfillopacity{0.800000}%
\pgfsetlinewidth{0.000000pt}%
\definecolor{currentstroke}{rgb}{0.000000,0.000000,0.000000}%
\pgfsetstrokecolor{currentstroke}%
\pgfsetdash{}{0pt}%
\pgfpathmoveto{\pgfqpoint{4.104579in}{2.364686in}}%
\pgfpathlineto{\pgfqpoint{4.118189in}{2.365728in}}%
\pgfpathlineto{\pgfqpoint{4.131808in}{2.366966in}}%
\pgfpathlineto{\pgfqpoint{4.145436in}{2.368399in}}%
\pgfpathlineto{\pgfqpoint{4.159073in}{2.370025in}}%
\pgfpathlineto{\pgfqpoint{4.166904in}{2.380007in}}%
\pgfpathlineto{\pgfqpoint{4.174730in}{2.389975in}}%
\pgfpathlineto{\pgfqpoint{4.182551in}{2.399931in}}%
\pgfpathlineto{\pgfqpoint{4.190366in}{2.409877in}}%
\pgfpathlineto{\pgfqpoint{4.176737in}{2.408383in}}%
\pgfpathlineto{\pgfqpoint{4.163117in}{2.407084in}}%
\pgfpathlineto{\pgfqpoint{4.149506in}{2.405980in}}%
\pgfpathlineto{\pgfqpoint{4.135903in}{2.405070in}}%
\pgfpathlineto{\pgfqpoint{4.128080in}{2.394979in}}%
\pgfpathlineto{\pgfqpoint{4.120251in}{2.384886in}}%
\pgfpathlineto{\pgfqpoint{4.112417in}{2.374789in}}%
\pgfpathlineto{\pgfqpoint{4.104579in}{2.364686in}}%
\pgfpathclose%
\pgfusepath{fill}%
\end{pgfscope}%
\begin{pgfscope}%
\pgfpathrectangle{\pgfqpoint{1.150000in}{0.150000in}}{\pgfqpoint{5.700000in}{5.700000in}}%
\pgfusepath{clip}%
\pgfsetbuttcap%
\pgfsetroundjoin%
\definecolor{currentfill}{rgb}{0.274128,0.199721,0.498911}%
\pgfsetfillcolor{currentfill}%
\pgfsetfillopacity{0.800000}%
\pgfsetlinewidth{0.000000pt}%
\definecolor{currentstroke}{rgb}{0.000000,0.000000,0.000000}%
\pgfsetstrokecolor{currentstroke}%
\pgfsetdash{}{0pt}%
\pgfpathmoveto{\pgfqpoint{4.190366in}{2.409877in}}%
\pgfpathlineto{\pgfqpoint{4.204005in}{2.411563in}}%
\pgfpathlineto{\pgfqpoint{4.217654in}{2.413443in}}%
\pgfpathlineto{\pgfqpoint{4.231312in}{2.415516in}}%
\pgfpathlineto{\pgfqpoint{4.244979in}{2.417781in}}%
\pgfpathlineto{\pgfqpoint{4.252782in}{2.427564in}}%
\pgfpathlineto{\pgfqpoint{4.260579in}{2.437334in}}%
\pgfpathlineto{\pgfqpoint{4.268371in}{2.447093in}}%
\pgfpathlineto{\pgfqpoint{4.276158in}{2.456841in}}%
\pgfpathlineto{\pgfqpoint{4.262499in}{2.454742in}}%
\pgfpathlineto{\pgfqpoint{4.248849in}{2.452835in}}%
\pgfpathlineto{\pgfqpoint{4.235209in}{2.451121in}}%
\pgfpathlineto{\pgfqpoint{4.221578in}{2.449599in}}%
\pgfpathlineto{\pgfqpoint{4.213783in}{2.439673in}}%
\pgfpathlineto{\pgfqpoint{4.205982in}{2.429745in}}%
\pgfpathlineto{\pgfqpoint{4.198177in}{2.419814in}}%
\pgfpathlineto{\pgfqpoint{4.190366in}{2.409877in}}%
\pgfpathclose%
\pgfusepath{fill}%
\end{pgfscope}%
\begin{pgfscope}%
\pgfpathrectangle{\pgfqpoint{1.150000in}{0.150000in}}{\pgfqpoint{5.700000in}{5.700000in}}%
\pgfusepath{clip}%
\pgfsetbuttcap%
\pgfsetroundjoin%
\definecolor{currentfill}{rgb}{0.280868,0.160771,0.472899}%
\pgfsetfillcolor{currentfill}%
\pgfsetfillopacity{0.800000}%
\pgfsetlinewidth{0.000000pt}%
\definecolor{currentstroke}{rgb}{0.000000,0.000000,0.000000}%
\pgfsetstrokecolor{currentstroke}%
\pgfsetdash{}{0pt}%
\pgfpathmoveto{\pgfqpoint{4.018786in}{2.321567in}}%
\pgfpathlineto{\pgfqpoint{4.032370in}{2.321924in}}%
\pgfpathlineto{\pgfqpoint{4.045962in}{2.322479in}}%
\pgfpathlineto{\pgfqpoint{4.059563in}{2.323230in}}%
\pgfpathlineto{\pgfqpoint{4.073172in}{2.324179in}}%
\pgfpathlineto{\pgfqpoint{4.081032in}{2.334323in}}%
\pgfpathlineto{\pgfqpoint{4.088886in}{2.344454in}}%
\pgfpathlineto{\pgfqpoint{4.096735in}{2.354575in}}%
\pgfpathlineto{\pgfqpoint{4.104579in}{2.364686in}}%
\pgfpathlineto{\pgfqpoint{4.090977in}{2.363839in}}%
\pgfpathlineto{\pgfqpoint{4.077384in}{2.363189in}}%
\pgfpathlineto{\pgfqpoint{4.063800in}{2.362736in}}%
\pgfpathlineto{\pgfqpoint{4.050223in}{2.362480in}}%
\pgfpathlineto{\pgfqpoint{4.042371in}{2.352256in}}%
\pgfpathlineto{\pgfqpoint{4.034515in}{2.342030in}}%
\pgfpathlineto{\pgfqpoint{4.026653in}{2.331801in}}%
\pgfpathlineto{\pgfqpoint{4.018786in}{2.321567in}}%
\pgfpathclose%
\pgfusepath{fill}%
\end{pgfscope}%
\begin{pgfscope}%
\pgfpathrectangle{\pgfqpoint{1.150000in}{0.150000in}}{\pgfqpoint{5.700000in}{5.700000in}}%
\pgfusepath{clip}%
\pgfsetbuttcap%
\pgfsetroundjoin%
\definecolor{currentfill}{rgb}{0.267968,0.223549,0.512008}%
\pgfsetfillcolor{currentfill}%
\pgfsetfillopacity{0.800000}%
\pgfsetlinewidth{0.000000pt}%
\definecolor{currentstroke}{rgb}{0.000000,0.000000,0.000000}%
\pgfsetstrokecolor{currentstroke}%
\pgfsetdash{}{0pt}%
\pgfpathmoveto{\pgfqpoint{4.276158in}{2.456841in}}%
\pgfpathlineto{\pgfqpoint{4.289828in}{2.459132in}}%
\pgfpathlineto{\pgfqpoint{4.303508in}{2.461613in}}%
\pgfpathlineto{\pgfqpoint{4.317198in}{2.464285in}}%
\pgfpathlineto{\pgfqpoint{4.330899in}{2.467148in}}%
\pgfpathlineto{\pgfqpoint{4.338672in}{2.476703in}}%
\pgfpathlineto{\pgfqpoint{4.346441in}{2.486246in}}%
\pgfpathlineto{\pgfqpoint{4.354204in}{2.495779in}}%
\pgfpathlineto{\pgfqpoint{4.361962in}{2.505305in}}%
\pgfpathlineto{\pgfqpoint{4.348270in}{2.502640in}}%
\pgfpathlineto{\pgfqpoint{4.334588in}{2.500166in}}%
\pgfpathlineto{\pgfqpoint{4.320916in}{2.497882in}}%
\pgfpathlineto{\pgfqpoint{4.307255in}{2.495788in}}%
\pgfpathlineto{\pgfqpoint{4.299489in}{2.486054in}}%
\pgfpathlineto{\pgfqpoint{4.291717in}{2.476320in}}%
\pgfpathlineto{\pgfqpoint{4.283940in}{2.466583in}}%
\pgfpathlineto{\pgfqpoint{4.276158in}{2.456841in}}%
\pgfpathclose%
\pgfusepath{fill}%
\end{pgfscope}%
\begin{pgfscope}%
\pgfpathrectangle{\pgfqpoint{1.150000in}{0.150000in}}{\pgfqpoint{5.700000in}{5.700000in}}%
\pgfusepath{clip}%
\pgfsetbuttcap%
\pgfsetroundjoin%
\definecolor{currentfill}{rgb}{0.208623,0.367752,0.552675}%
\pgfsetfillcolor{currentfill}%
\pgfsetfillopacity{0.800000}%
\pgfsetlinewidth{0.000000pt}%
\definecolor{currentstroke}{rgb}{0.000000,0.000000,0.000000}%
\pgfsetstrokecolor{currentstroke}%
\pgfsetdash{}{0pt}%
\pgfpathmoveto{\pgfqpoint{2.569483in}{2.884880in}}%
\pgfpathlineto{\pgfqpoint{2.583281in}{2.862536in}}%
\pgfpathlineto{\pgfqpoint{2.597065in}{2.840541in}}%
\pgfpathlineto{\pgfqpoint{2.610837in}{2.818893in}}%
\pgfpathlineto{\pgfqpoint{2.624596in}{2.797587in}}%
\pgfpathlineto{\pgfqpoint{2.632996in}{2.804756in}}%
\pgfpathlineto{\pgfqpoint{2.641385in}{2.812082in}}%
\pgfpathlineto{\pgfqpoint{2.649763in}{2.819565in}}%
\pgfpathlineto{\pgfqpoint{2.658131in}{2.827204in}}%
\pgfpathlineto{\pgfqpoint{2.644400in}{2.848343in}}%
\pgfpathlineto{\pgfqpoint{2.630656in}{2.869825in}}%
\pgfpathlineto{\pgfqpoint{2.616900in}{2.891652in}}%
\pgfpathlineto{\pgfqpoint{2.603131in}{2.913829in}}%
\pgfpathlineto{\pgfqpoint{2.594736in}{2.906345in}}%
\pgfpathlineto{\pgfqpoint{2.586329in}{2.899024in}}%
\pgfpathlineto{\pgfqpoint{2.577912in}{2.891869in}}%
\pgfpathlineto{\pgfqpoint{2.569483in}{2.884880in}}%
\pgfpathclose%
\pgfusepath{fill}%
\end{pgfscope}%
\begin{pgfscope}%
\pgfpathrectangle{\pgfqpoint{1.150000in}{0.150000in}}{\pgfqpoint{5.700000in}{5.700000in}}%
\pgfusepath{clip}%
\pgfsetbuttcap%
\pgfsetroundjoin%
\definecolor{currentfill}{rgb}{0.281446,0.084320,0.407414}%
\pgfsetfillcolor{currentfill}%
\pgfsetfillopacity{0.800000}%
\pgfsetlinewidth{0.000000pt}%
\definecolor{currentstroke}{rgb}{0.000000,0.000000,0.000000}%
\pgfsetstrokecolor{currentstroke}%
\pgfsetdash{}{0pt}%
\pgfpathmoveto{\pgfqpoint{3.535302in}{2.166216in}}%
\pgfpathlineto{\pgfqpoint{3.548791in}{2.161632in}}%
\pgfpathlineto{\pgfqpoint{3.562284in}{2.157265in}}%
\pgfpathlineto{\pgfqpoint{3.575780in}{2.153115in}}%
\pgfpathlineto{\pgfqpoint{3.589280in}{2.149180in}}%
\pgfpathlineto{\pgfqpoint{3.597295in}{2.159352in}}%
\pgfpathlineto{\pgfqpoint{3.605305in}{2.169545in}}%
\pgfpathlineto{\pgfqpoint{3.613310in}{2.179761in}}%
\pgfpathlineto{\pgfqpoint{3.621309in}{2.189998in}}%
\pgfpathlineto{\pgfqpoint{3.607819in}{2.193876in}}%
\pgfpathlineto{\pgfqpoint{3.594334in}{2.197970in}}%
\pgfpathlineto{\pgfqpoint{3.580852in}{2.202279in}}%
\pgfpathlineto{\pgfqpoint{3.567374in}{2.206806in}}%
\pgfpathlineto{\pgfqpoint{3.559365in}{2.196614in}}%
\pgfpathlineto{\pgfqpoint{3.551350in}{2.186452in}}%
\pgfpathlineto{\pgfqpoint{3.543329in}{2.176319in}}%
\pgfpathlineto{\pgfqpoint{3.535302in}{2.166216in}}%
\pgfpathclose%
\pgfusepath{fill}%
\end{pgfscope}%
\begin{pgfscope}%
\pgfpathrectangle{\pgfqpoint{1.150000in}{0.150000in}}{\pgfqpoint{5.700000in}{5.700000in}}%
\pgfusepath{clip}%
\pgfsetbuttcap%
\pgfsetroundjoin%
\definecolor{currentfill}{rgb}{0.157729,0.485932,0.558013}%
\pgfsetfillcolor{currentfill}%
\pgfsetfillopacity{0.800000}%
\pgfsetlinewidth{0.000000pt}%
\definecolor{currentstroke}{rgb}{0.000000,0.000000,0.000000}%
\pgfsetstrokecolor{currentstroke}%
\pgfsetdash{}{0pt}%
\pgfpathmoveto{\pgfqpoint{5.422823in}{3.163060in}}%
\pgfpathlineto{\pgfqpoint{5.436971in}{3.169051in}}%
\pgfpathlineto{\pgfqpoint{5.451136in}{3.175216in}}%
\pgfpathlineto{\pgfqpoint{5.465317in}{3.181553in}}%
\pgfpathlineto{\pgfqpoint{5.479514in}{3.188064in}}%
\pgfpathlineto{\pgfqpoint{5.486838in}{3.194873in}}%
\pgfpathlineto{\pgfqpoint{5.494162in}{3.201838in}}%
\pgfpathlineto{\pgfqpoint{5.501484in}{3.208967in}}%
\pgfpathlineto{\pgfqpoint{5.508806in}{3.216267in}}%
\pgfpathlineto{\pgfqpoint{5.494635in}{3.210409in}}%
\pgfpathlineto{\pgfqpoint{5.480481in}{3.204724in}}%
\pgfpathlineto{\pgfqpoint{5.466342in}{3.199211in}}%
\pgfpathlineto{\pgfqpoint{5.452219in}{3.193870in}}%
\pgfpathlineto{\pgfqpoint{5.444871in}{3.185907in}}%
\pgfpathlineto{\pgfqpoint{5.437522in}{3.178122in}}%
\pgfpathlineto{\pgfqpoint{5.430173in}{3.170509in}}%
\pgfpathlineto{\pgfqpoint{5.422823in}{3.163060in}}%
\pgfpathclose%
\pgfusepath{fill}%
\end{pgfscope}%
\begin{pgfscope}%
\pgfpathrectangle{\pgfqpoint{1.150000in}{0.150000in}}{\pgfqpoint{5.700000in}{5.700000in}}%
\pgfusepath{clip}%
\pgfsetbuttcap%
\pgfsetroundjoin%
\definecolor{currentfill}{rgb}{0.262138,0.242286,0.520837}%
\pgfsetfillcolor{currentfill}%
\pgfsetfillopacity{0.800000}%
\pgfsetlinewidth{0.000000pt}%
\definecolor{currentstroke}{rgb}{0.000000,0.000000,0.000000}%
\pgfsetstrokecolor{currentstroke}%
\pgfsetdash{}{0pt}%
\pgfpathmoveto{\pgfqpoint{4.361962in}{2.505305in}}%
\pgfpathlineto{\pgfqpoint{4.375665in}{2.508159in}}%
\pgfpathlineto{\pgfqpoint{4.389378in}{2.511202in}}%
\pgfpathlineto{\pgfqpoint{4.403102in}{2.514433in}}%
\pgfpathlineto{\pgfqpoint{4.416838in}{2.517854in}}%
\pgfpathlineto{\pgfqpoint{4.424582in}{2.527156in}}%
\pgfpathlineto{\pgfqpoint{4.432320in}{2.536448in}}%
\pgfpathlineto{\pgfqpoint{4.440054in}{2.545733in}}%
\pgfpathlineto{\pgfqpoint{4.447782in}{2.555014in}}%
\pgfpathlineto{\pgfqpoint{4.434055in}{2.551824in}}%
\pgfpathlineto{\pgfqpoint{4.420340in}{2.548823in}}%
\pgfpathlineto{\pgfqpoint{4.406636in}{2.546009in}}%
\pgfpathlineto{\pgfqpoint{4.392942in}{2.543385in}}%
\pgfpathlineto{\pgfqpoint{4.385204in}{2.533862in}}%
\pgfpathlineto{\pgfqpoint{4.377462in}{2.524343in}}%
\pgfpathlineto{\pgfqpoint{4.369714in}{2.514825in}}%
\pgfpathlineto{\pgfqpoint{4.361962in}{2.505305in}}%
\pgfpathclose%
\pgfusepath{fill}%
\end{pgfscope}%
\begin{pgfscope}%
\pgfpathrectangle{\pgfqpoint{1.150000in}{0.150000in}}{\pgfqpoint{5.700000in}{5.700000in}}%
\pgfusepath{clip}%
\pgfsetbuttcap%
\pgfsetroundjoin%
\definecolor{currentfill}{rgb}{0.282623,0.140926,0.457517}%
\pgfsetfillcolor{currentfill}%
\pgfsetfillopacity{0.800000}%
\pgfsetlinewidth{0.000000pt}%
\definecolor{currentstroke}{rgb}{0.000000,0.000000,0.000000}%
\pgfsetstrokecolor{currentstroke}%
\pgfsetdash{}{0pt}%
\pgfpathmoveto{\pgfqpoint{3.932977in}{2.280844in}}%
\pgfpathlineto{\pgfqpoint{3.946538in}{2.280473in}}%
\pgfpathlineto{\pgfqpoint{3.960107in}{2.280302in}}%
\pgfpathlineto{\pgfqpoint{3.973683in}{2.280331in}}%
\pgfpathlineto{\pgfqpoint{3.987267in}{2.280560in}}%
\pgfpathlineto{\pgfqpoint{3.995155in}{2.290825in}}%
\pgfpathlineto{\pgfqpoint{4.003037in}{2.301081in}}%
\pgfpathlineto{\pgfqpoint{4.010914in}{2.311328in}}%
\pgfpathlineto{\pgfqpoint{4.018786in}{2.321567in}}%
\pgfpathlineto{\pgfqpoint{4.005210in}{2.321409in}}%
\pgfpathlineto{\pgfqpoint{3.991641in}{2.321450in}}%
\pgfpathlineto{\pgfqpoint{3.978081in}{2.321690in}}%
\pgfpathlineto{\pgfqpoint{3.964528in}{2.322130in}}%
\pgfpathlineto{\pgfqpoint{3.956648in}{2.311809in}}%
\pgfpathlineto{\pgfqpoint{3.948763in}{2.301489in}}%
\pgfpathlineto{\pgfqpoint{3.940872in}{2.291167in}}%
\pgfpathlineto{\pgfqpoint{3.932977in}{2.280844in}}%
\pgfpathclose%
\pgfusepath{fill}%
\end{pgfscope}%
\begin{pgfscope}%
\pgfpathrectangle{\pgfqpoint{1.150000in}{0.150000in}}{\pgfqpoint{5.700000in}{5.700000in}}%
\pgfusepath{clip}%
\pgfsetbuttcap%
\pgfsetroundjoin%
\definecolor{currentfill}{rgb}{0.283091,0.110553,0.431554}%
\pgfsetfillcolor{currentfill}%
\pgfsetfillopacity{0.800000}%
\pgfsetlinewidth{0.000000pt}%
\definecolor{currentstroke}{rgb}{0.000000,0.000000,0.000000}%
\pgfsetstrokecolor{currentstroke}%
\pgfsetdash{}{0pt}%
\pgfpathmoveto{\pgfqpoint{3.114324in}{2.234562in}}%
\pgfpathlineto{\pgfqpoint{3.127836in}{2.224020in}}%
\pgfpathlineto{\pgfqpoint{3.141347in}{2.213727in}}%
\pgfpathlineto{\pgfqpoint{3.154855in}{2.203684in}}%
\pgfpathlineto{\pgfqpoint{3.168363in}{2.193886in}}%
\pgfpathlineto{\pgfqpoint{3.176536in}{2.202767in}}%
\pgfpathlineto{\pgfqpoint{3.184702in}{2.211725in}}%
\pgfpathlineto{\pgfqpoint{3.192861in}{2.220760in}}%
\pgfpathlineto{\pgfqpoint{3.201013in}{2.229870in}}%
\pgfpathlineto{\pgfqpoint{3.187523in}{2.239513in}}%
\pgfpathlineto{\pgfqpoint{3.174032in}{2.249403in}}%
\pgfpathlineto{\pgfqpoint{3.160539in}{2.259541in}}%
\pgfpathlineto{\pgfqpoint{3.147045in}{2.269928in}}%
\pgfpathlineto{\pgfqpoint{3.138876in}{2.260961in}}%
\pgfpathlineto{\pgfqpoint{3.130699in}{2.252076in}}%
\pgfpathlineto{\pgfqpoint{3.122516in}{2.243276in}}%
\pgfpathlineto{\pgfqpoint{3.114324in}{2.234562in}}%
\pgfpathclose%
\pgfusepath{fill}%
\end{pgfscope}%
\begin{pgfscope}%
\pgfpathrectangle{\pgfqpoint{1.150000in}{0.150000in}}{\pgfqpoint{5.700000in}{5.700000in}}%
\pgfusepath{clip}%
\pgfsetbuttcap%
\pgfsetroundjoin%
\definecolor{currentfill}{rgb}{0.253935,0.265254,0.529983}%
\pgfsetfillcolor{currentfill}%
\pgfsetfillopacity{0.800000}%
\pgfsetlinewidth{0.000000pt}%
\definecolor{currentstroke}{rgb}{0.000000,0.000000,0.000000}%
\pgfsetstrokecolor{currentstroke}%
\pgfsetdash{}{0pt}%
\pgfpathmoveto{\pgfqpoint{4.447782in}{2.555014in}}%
\pgfpathlineto{\pgfqpoint{4.461519in}{2.558392in}}%
\pgfpathlineto{\pgfqpoint{4.475268in}{2.561957in}}%
\pgfpathlineto{\pgfqpoint{4.489029in}{2.565709in}}%
\pgfpathlineto{\pgfqpoint{4.502801in}{2.569647in}}%
\pgfpathlineto{\pgfqpoint{4.510514in}{2.578677in}}%
\pgfpathlineto{\pgfqpoint{4.518222in}{2.587700in}}%
\pgfpathlineto{\pgfqpoint{4.525925in}{2.596720in}}%
\pgfpathlineto{\pgfqpoint{4.533623in}{2.605740in}}%
\pgfpathlineto{\pgfqpoint{4.519861in}{2.602065in}}%
\pgfpathlineto{\pgfqpoint{4.506110in}{2.598575in}}%
\pgfpathlineto{\pgfqpoint{4.492371in}{2.595272in}}%
\pgfpathlineto{\pgfqpoint{4.478643in}{2.592156in}}%
\pgfpathlineto{\pgfqpoint{4.470935in}{2.582862in}}%
\pgfpathlineto{\pgfqpoint{4.463223in}{2.573575in}}%
\pgfpathlineto{\pgfqpoint{4.455505in}{2.564294in}}%
\pgfpathlineto{\pgfqpoint{4.447782in}{2.555014in}}%
\pgfpathclose%
\pgfusepath{fill}%
\end{pgfscope}%
\begin{pgfscope}%
\pgfpathrectangle{\pgfqpoint{1.150000in}{0.150000in}}{\pgfqpoint{5.700000in}{5.700000in}}%
\pgfusepath{clip}%
\pgfsetbuttcap%
\pgfsetroundjoin%
\definecolor{currentfill}{rgb}{0.150476,0.504369,0.557430}%
\pgfsetfillcolor{currentfill}%
\pgfsetfillopacity{0.800000}%
\pgfsetlinewidth{0.000000pt}%
\definecolor{currentstroke}{rgb}{0.000000,0.000000,0.000000}%
\pgfsetstrokecolor{currentstroke}%
\pgfsetdash{}{0pt}%
\pgfpathmoveto{\pgfqpoint{5.508806in}{3.216267in}}%
\pgfpathlineto{\pgfqpoint{5.522992in}{3.222297in}}%
\pgfpathlineto{\pgfqpoint{5.537194in}{3.228499in}}%
\pgfpathlineto{\pgfqpoint{5.551413in}{3.234873in}}%
\pgfpathlineto{\pgfqpoint{5.565647in}{3.241419in}}%
\pgfpathlineto{\pgfqpoint{5.572941in}{3.248224in}}%
\pgfpathlineto{\pgfqpoint{5.580234in}{3.255208in}}%
\pgfpathlineto{\pgfqpoint{5.587526in}{3.262378in}}%
\pgfpathlineto{\pgfqpoint{5.594819in}{3.269742in}}%
\pgfpathlineto{\pgfqpoint{5.580613in}{3.263882in}}%
\pgfpathlineto{\pgfqpoint{5.566423in}{3.258192in}}%
\pgfpathlineto{\pgfqpoint{5.552248in}{3.252674in}}%
\pgfpathlineto{\pgfqpoint{5.538090in}{3.247327in}}%
\pgfpathlineto{\pgfqpoint{5.530768in}{3.239268in}}%
\pgfpathlineto{\pgfqpoint{5.523447in}{3.231410in}}%
\pgfpathlineto{\pgfqpoint{5.516127in}{3.223746in}}%
\pgfpathlineto{\pgfqpoint{5.508806in}{3.216267in}}%
\pgfpathclose%
\pgfusepath{fill}%
\end{pgfscope}%
\begin{pgfscope}%
\pgfpathrectangle{\pgfqpoint{1.150000in}{0.150000in}}{\pgfqpoint{5.700000in}{5.700000in}}%
\pgfusepath{clip}%
\pgfsetbuttcap%
\pgfsetroundjoin%
\definecolor{currentfill}{rgb}{0.283187,0.125848,0.444960}%
\pgfsetfillcolor{currentfill}%
\pgfsetfillopacity{0.800000}%
\pgfsetlinewidth{0.000000pt}%
\definecolor{currentstroke}{rgb}{0.000000,0.000000,0.000000}%
\pgfsetstrokecolor{currentstroke}%
\pgfsetdash{}{0pt}%
\pgfpathmoveto{\pgfqpoint{3.847139in}{2.242861in}}%
\pgfpathlineto{\pgfqpoint{3.860681in}{2.241719in}}%
\pgfpathlineto{\pgfqpoint{3.874229in}{2.240781in}}%
\pgfpathlineto{\pgfqpoint{3.887784in}{2.240045in}}%
\pgfpathlineto{\pgfqpoint{3.901346in}{2.239511in}}%
\pgfpathlineto{\pgfqpoint{3.909261in}{2.249852in}}%
\pgfpathlineto{\pgfqpoint{3.917172in}{2.260187in}}%
\pgfpathlineto{\pgfqpoint{3.925077in}{2.270517in}}%
\pgfpathlineto{\pgfqpoint{3.932977in}{2.280844in}}%
\pgfpathlineto{\pgfqpoint{3.919423in}{2.281416in}}%
\pgfpathlineto{\pgfqpoint{3.905877in}{2.282190in}}%
\pgfpathlineto{\pgfqpoint{3.892337in}{2.283166in}}%
\pgfpathlineto{\pgfqpoint{3.878804in}{2.284345in}}%
\pgfpathlineto{\pgfqpoint{3.870896in}{2.273969in}}%
\pgfpathlineto{\pgfqpoint{3.862982in}{2.263597in}}%
\pgfpathlineto{\pgfqpoint{3.855063in}{2.253228in}}%
\pgfpathlineto{\pgfqpoint{3.847139in}{2.242861in}}%
\pgfpathclose%
\pgfusepath{fill}%
\end{pgfscope}%
\begin{pgfscope}%
\pgfpathrectangle{\pgfqpoint{1.150000in}{0.150000in}}{\pgfqpoint{5.700000in}{5.700000in}}%
\pgfusepath{clip}%
\pgfsetbuttcap%
\pgfsetroundjoin%
\definecolor{currentfill}{rgb}{0.244972,0.287675,0.537260}%
\pgfsetfillcolor{currentfill}%
\pgfsetfillopacity{0.800000}%
\pgfsetlinewidth{0.000000pt}%
\definecolor{currentstroke}{rgb}{0.000000,0.000000,0.000000}%
\pgfsetstrokecolor{currentstroke}%
\pgfsetdash{}{0pt}%
\pgfpathmoveto{\pgfqpoint{4.533623in}{2.605740in}}%
\pgfpathlineto{\pgfqpoint{4.547397in}{2.609602in}}%
\pgfpathlineto{\pgfqpoint{4.561183in}{2.613650in}}%
\pgfpathlineto{\pgfqpoint{4.574980in}{2.617883in}}%
\pgfpathlineto{\pgfqpoint{4.588791in}{2.622302in}}%
\pgfpathlineto{\pgfqpoint{4.596473in}{2.631043in}}%
\pgfpathlineto{\pgfqpoint{4.604150in}{2.639784in}}%
\pgfpathlineto{\pgfqpoint{4.611821in}{2.648527in}}%
\pgfpathlineto{\pgfqpoint{4.619487in}{2.657277in}}%
\pgfpathlineto{\pgfqpoint{4.605688in}{2.653154in}}%
\pgfpathlineto{\pgfqpoint{4.591901in}{2.649215in}}%
\pgfpathlineto{\pgfqpoint{4.578125in}{2.645462in}}%
\pgfpathlineto{\pgfqpoint{4.564362in}{2.641894in}}%
\pgfpathlineto{\pgfqpoint{4.556685in}{2.632838in}}%
\pgfpathlineto{\pgfqpoint{4.549003in}{2.623796in}}%
\pgfpathlineto{\pgfqpoint{4.541315in}{2.614765in}}%
\pgfpathlineto{\pgfqpoint{4.533623in}{2.605740in}}%
\pgfpathclose%
\pgfusepath{fill}%
\end{pgfscope}%
\begin{pgfscope}%
\pgfpathrectangle{\pgfqpoint{1.150000in}{0.150000in}}{\pgfqpoint{5.700000in}{5.700000in}}%
\pgfusepath{clip}%
\pgfsetbuttcap%
\pgfsetroundjoin%
\definecolor{currentfill}{rgb}{0.143343,0.522773,0.556295}%
\pgfsetfillcolor{currentfill}%
\pgfsetfillopacity{0.800000}%
\pgfsetlinewidth{0.000000pt}%
\definecolor{currentstroke}{rgb}{0.000000,0.000000,0.000000}%
\pgfsetstrokecolor{currentstroke}%
\pgfsetdash{}{0pt}%
\pgfpathmoveto{\pgfqpoint{5.594819in}{3.269742in}}%
\pgfpathlineto{\pgfqpoint{5.609042in}{3.275774in}}%
\pgfpathlineto{\pgfqpoint{5.623281in}{3.281977in}}%
\pgfpathlineto{\pgfqpoint{5.637536in}{3.288352in}}%
\pgfpathlineto{\pgfqpoint{5.651808in}{3.294897in}}%
\pgfpathlineto{\pgfqpoint{5.659071in}{3.301757in}}%
\pgfpathlineto{\pgfqpoint{5.666336in}{3.308818in}}%
\pgfpathlineto{\pgfqpoint{5.673601in}{3.316090in}}%
\pgfpathlineto{\pgfqpoint{5.680867in}{3.323580in}}%
\pgfpathlineto{\pgfqpoint{5.666626in}{3.317752in}}%
\pgfpathlineto{\pgfqpoint{5.652401in}{3.312095in}}%
\pgfpathlineto{\pgfqpoint{5.638192in}{3.306607in}}%
\pgfpathlineto{\pgfqpoint{5.623999in}{3.301291in}}%
\pgfpathlineto{\pgfqpoint{5.616702in}{3.293074in}}%
\pgfpathlineto{\pgfqpoint{5.609407in}{3.285082in}}%
\pgfpathlineto{\pgfqpoint{5.602113in}{3.277308in}}%
\pgfpathlineto{\pgfqpoint{5.594819in}{3.269742in}}%
\pgfpathclose%
\pgfusepath{fill}%
\end{pgfscope}%
\begin{pgfscope}%
\pgfpathrectangle{\pgfqpoint{1.150000in}{0.150000in}}{\pgfqpoint{5.700000in}{5.700000in}}%
\pgfusepath{clip}%
\pgfsetbuttcap%
\pgfsetroundjoin%
\definecolor{currentfill}{rgb}{0.237441,0.305202,0.541921}%
\pgfsetfillcolor{currentfill}%
\pgfsetfillopacity{0.800000}%
\pgfsetlinewidth{0.000000pt}%
\definecolor{currentstroke}{rgb}{0.000000,0.000000,0.000000}%
\pgfsetstrokecolor{currentstroke}%
\pgfsetdash{}{0pt}%
\pgfpathmoveto{\pgfqpoint{4.619487in}{2.657277in}}%
\pgfpathlineto{\pgfqpoint{4.633299in}{2.661584in}}%
\pgfpathlineto{\pgfqpoint{4.647123in}{2.666076in}}%
\pgfpathlineto{\pgfqpoint{4.660960in}{2.670752in}}%
\pgfpathlineto{\pgfqpoint{4.674809in}{2.675611in}}%
\pgfpathlineto{\pgfqpoint{4.682459in}{2.684055in}}%
\pgfpathlineto{\pgfqpoint{4.690104in}{2.692505in}}%
\pgfpathlineto{\pgfqpoint{4.697743in}{2.700965in}}%
\pgfpathlineto{\pgfqpoint{4.705377in}{2.709439in}}%
\pgfpathlineto{\pgfqpoint{4.691540in}{2.704908in}}%
\pgfpathlineto{\pgfqpoint{4.677715in}{2.700560in}}%
\pgfpathlineto{\pgfqpoint{4.663902in}{2.696395in}}%
\pgfpathlineto{\pgfqpoint{4.650102in}{2.692414in}}%
\pgfpathlineto{\pgfqpoint{4.642456in}{2.683602in}}%
\pgfpathlineto{\pgfqpoint{4.634805in}{2.674810in}}%
\pgfpathlineto{\pgfqpoint{4.627149in}{2.666037in}}%
\pgfpathlineto{\pgfqpoint{4.619487in}{2.657277in}}%
\pgfpathclose%
\pgfusepath{fill}%
\end{pgfscope}%
\begin{pgfscope}%
\pgfpathrectangle{\pgfqpoint{1.150000in}{0.150000in}}{\pgfqpoint{5.700000in}{5.700000in}}%
\pgfusepath{clip}%
\pgfsetbuttcap%
\pgfsetroundjoin%
\definecolor{currentfill}{rgb}{0.281446,0.084320,0.407414}%
\pgfsetfillcolor{currentfill}%
\pgfsetfillopacity{0.800000}%
\pgfsetlinewidth{0.000000pt}%
\definecolor{currentstroke}{rgb}{0.000000,0.000000,0.000000}%
\pgfsetstrokecolor{currentstroke}%
\pgfsetdash{}{0pt}%
\pgfpathmoveto{\pgfqpoint{3.308915in}{2.161412in}}%
\pgfpathlineto{\pgfqpoint{3.322404in}{2.153919in}}%
\pgfpathlineto{\pgfqpoint{3.335894in}{2.146657in}}%
\pgfpathlineto{\pgfqpoint{3.349385in}{2.139626in}}%
\pgfpathlineto{\pgfqpoint{3.362877in}{2.132825in}}%
\pgfpathlineto{\pgfqpoint{3.370976in}{2.142409in}}%
\pgfpathlineto{\pgfqpoint{3.379069in}{2.152042in}}%
\pgfpathlineto{\pgfqpoint{3.387156in}{2.161725in}}%
\pgfpathlineto{\pgfqpoint{3.395236in}{2.171456in}}%
\pgfpathlineto{\pgfqpoint{3.381758in}{2.178137in}}%
\pgfpathlineto{\pgfqpoint{3.368281in}{2.185047in}}%
\pgfpathlineto{\pgfqpoint{3.354805in}{2.192188in}}%
\pgfpathlineto{\pgfqpoint{3.341331in}{2.199560in}}%
\pgfpathlineto{\pgfqpoint{3.333237in}{2.189938in}}%
\pgfpathlineto{\pgfqpoint{3.325136in}{2.180372in}}%
\pgfpathlineto{\pgfqpoint{3.317029in}{2.170863in}}%
\pgfpathlineto{\pgfqpoint{3.308915in}{2.161412in}}%
\pgfpathclose%
\pgfusepath{fill}%
\end{pgfscope}%
\begin{pgfscope}%
\pgfpathrectangle{\pgfqpoint{1.150000in}{0.150000in}}{\pgfqpoint{5.700000in}{5.700000in}}%
\pgfusepath{clip}%
\pgfsetbuttcap%
\pgfsetroundjoin%
\definecolor{currentfill}{rgb}{0.283091,0.110553,0.431554}%
\pgfsetfillcolor{currentfill}%
\pgfsetfillopacity{0.800000}%
\pgfsetlinewidth{0.000000pt}%
\definecolor{currentstroke}{rgb}{0.000000,0.000000,0.000000}%
\pgfsetstrokecolor{currentstroke}%
\pgfsetdash{}{0pt}%
\pgfpathmoveto{\pgfqpoint{3.761257in}{2.207988in}}%
\pgfpathlineto{\pgfqpoint{3.774782in}{2.206032in}}%
\pgfpathlineto{\pgfqpoint{3.788313in}{2.204283in}}%
\pgfpathlineto{\pgfqpoint{3.801850in}{2.202739in}}%
\pgfpathlineto{\pgfqpoint{3.815393in}{2.201399in}}%
\pgfpathlineto{\pgfqpoint{3.823337in}{2.211765in}}%
\pgfpathlineto{\pgfqpoint{3.831276in}{2.222130in}}%
\pgfpathlineto{\pgfqpoint{3.839211in}{2.232495in}}%
\pgfpathlineto{\pgfqpoint{3.847139in}{2.242861in}}%
\pgfpathlineto{\pgfqpoint{3.833605in}{2.244207in}}%
\pgfpathlineto{\pgfqpoint{3.820076in}{2.245757in}}%
\pgfpathlineto{\pgfqpoint{3.806554in}{2.247513in}}%
\pgfpathlineto{\pgfqpoint{3.793038in}{2.249475in}}%
\pgfpathlineto{\pgfqpoint{3.785101in}{2.239091in}}%
\pgfpathlineto{\pgfqpoint{3.777158in}{2.228716in}}%
\pgfpathlineto{\pgfqpoint{3.769210in}{2.218348in}}%
\pgfpathlineto{\pgfqpoint{3.761257in}{2.207988in}}%
\pgfpathclose%
\pgfusepath{fill}%
\end{pgfscope}%
\begin{pgfscope}%
\pgfpathrectangle{\pgfqpoint{1.150000in}{0.150000in}}{\pgfqpoint{5.700000in}{5.700000in}}%
\pgfusepath{clip}%
\pgfsetbuttcap%
\pgfsetroundjoin%
\definecolor{currentfill}{rgb}{0.270595,0.214069,0.507052}%
\pgfsetfillcolor{currentfill}%
\pgfsetfillopacity{0.800000}%
\pgfsetlinewidth{0.000000pt}%
\definecolor{currentstroke}{rgb}{0.000000,0.000000,0.000000}%
\pgfsetstrokecolor{currentstroke}%
\pgfsetdash{}{0pt}%
\pgfpathmoveto{\pgfqpoint{2.810058in}{2.470569in}}%
\pgfpathlineto{\pgfqpoint{2.823681in}{2.454432in}}%
\pgfpathlineto{\pgfqpoint{2.837297in}{2.438586in}}%
\pgfpathlineto{\pgfqpoint{2.850907in}{2.423029in}}%
\pgfpathlineto{\pgfqpoint{2.864510in}{2.407759in}}%
\pgfpathlineto{\pgfqpoint{2.872819in}{2.415316in}}%
\pgfpathlineto{\pgfqpoint{2.881119in}{2.422997in}}%
\pgfpathlineto{\pgfqpoint{2.889410in}{2.430799in}}%
\pgfpathlineto{\pgfqpoint{2.897692in}{2.438721in}}%
\pgfpathlineto{\pgfqpoint{2.884113in}{2.453801in}}%
\pgfpathlineto{\pgfqpoint{2.870527in}{2.469166in}}%
\pgfpathlineto{\pgfqpoint{2.856935in}{2.484820in}}%
\pgfpathlineto{\pgfqpoint{2.843336in}{2.500765in}}%
\pgfpathlineto{\pgfqpoint{2.835031in}{2.493023in}}%
\pgfpathlineto{\pgfqpoint{2.826716in}{2.485408in}}%
\pgfpathlineto{\pgfqpoint{2.818392in}{2.477923in}}%
\pgfpathlineto{\pgfqpoint{2.810058in}{2.470569in}}%
\pgfpathclose%
\pgfusepath{fill}%
\end{pgfscope}%
\begin{pgfscope}%
\pgfpathrectangle{\pgfqpoint{1.150000in}{0.150000in}}{\pgfqpoint{5.700000in}{5.700000in}}%
\pgfusepath{clip}%
\pgfsetbuttcap%
\pgfsetroundjoin%
\definecolor{currentfill}{rgb}{0.262138,0.242286,0.520837}%
\pgfsetfillcolor{currentfill}%
\pgfsetfillopacity{0.800000}%
\pgfsetlinewidth{0.000000pt}%
\definecolor{currentstroke}{rgb}{0.000000,0.000000,0.000000}%
\pgfsetstrokecolor{currentstroke}%
\pgfsetdash{}{0pt}%
\pgfpathmoveto{\pgfqpoint{2.755488in}{2.538078in}}%
\pgfpathlineto{\pgfqpoint{2.769143in}{2.520751in}}%
\pgfpathlineto{\pgfqpoint{2.782789in}{2.503726in}}%
\pgfpathlineto{\pgfqpoint{2.796427in}{2.486999in}}%
\pgfpathlineto{\pgfqpoint{2.810058in}{2.470569in}}%
\pgfpathlineto{\pgfqpoint{2.818392in}{2.477923in}}%
\pgfpathlineto{\pgfqpoint{2.826716in}{2.485408in}}%
\pgfpathlineto{\pgfqpoint{2.835031in}{2.493023in}}%
\pgfpathlineto{\pgfqpoint{2.843336in}{2.500765in}}%
\pgfpathlineto{\pgfqpoint{2.829730in}{2.517003in}}%
\pgfpathlineto{\pgfqpoint{2.816117in}{2.533537in}}%
\pgfpathlineto{\pgfqpoint{2.802496in}{2.550369in}}%
\pgfpathlineto{\pgfqpoint{2.788867in}{2.567502in}}%
\pgfpathlineto{\pgfqpoint{2.780537in}{2.559940in}}%
\pgfpathlineto{\pgfqpoint{2.772197in}{2.552515in}}%
\pgfpathlineto{\pgfqpoint{2.763848in}{2.545227in}}%
\pgfpathlineto{\pgfqpoint{2.755488in}{2.538078in}}%
\pgfpathclose%
\pgfusepath{fill}%
\end{pgfscope}%
\begin{pgfscope}%
\pgfpathrectangle{\pgfqpoint{1.150000in}{0.150000in}}{\pgfqpoint{5.700000in}{5.700000in}}%
\pgfusepath{clip}%
\pgfsetbuttcap%
\pgfsetroundjoin%
\definecolor{currentfill}{rgb}{0.227802,0.326594,0.546532}%
\pgfsetfillcolor{currentfill}%
\pgfsetfillopacity{0.800000}%
\pgfsetlinewidth{0.000000pt}%
\definecolor{currentstroke}{rgb}{0.000000,0.000000,0.000000}%
\pgfsetstrokecolor{currentstroke}%
\pgfsetdash{}{0pt}%
\pgfpathmoveto{\pgfqpoint{4.705377in}{2.709439in}}%
\pgfpathlineto{\pgfqpoint{4.719228in}{2.714154in}}%
\pgfpathlineto{\pgfqpoint{4.733091in}{2.719051in}}%
\pgfpathlineto{\pgfqpoint{4.746967in}{2.724131in}}%
\pgfpathlineto{\pgfqpoint{4.760857in}{2.729393in}}%
\pgfpathlineto{\pgfqpoint{4.768474in}{2.737536in}}%
\pgfpathlineto{\pgfqpoint{4.776085in}{2.745693in}}%
\pgfpathlineto{\pgfqpoint{4.783692in}{2.753869in}}%
\pgfpathlineto{\pgfqpoint{4.791293in}{2.762067in}}%
\pgfpathlineto{\pgfqpoint{4.777417in}{2.757165in}}%
\pgfpathlineto{\pgfqpoint{4.763553in}{2.752446in}}%
\pgfpathlineto{\pgfqpoint{4.749702in}{2.747908in}}%
\pgfpathlineto{\pgfqpoint{4.735864in}{2.743553in}}%
\pgfpathlineto{\pgfqpoint{4.728250in}{2.734984in}}%
\pgfpathlineto{\pgfqpoint{4.720631in}{2.726444in}}%
\pgfpathlineto{\pgfqpoint{4.713006in}{2.717931in}}%
\pgfpathlineto{\pgfqpoint{4.705377in}{2.709439in}}%
\pgfpathclose%
\pgfusepath{fill}%
\end{pgfscope}%
\begin{pgfscope}%
\pgfpathrectangle{\pgfqpoint{1.150000in}{0.150000in}}{\pgfqpoint{5.700000in}{5.700000in}}%
\pgfusepath{clip}%
\pgfsetbuttcap%
\pgfsetroundjoin%
\definecolor{currentfill}{rgb}{0.277134,0.185228,0.489898}%
\pgfsetfillcolor{currentfill}%
\pgfsetfillopacity{0.800000}%
\pgfsetlinewidth{0.000000pt}%
\definecolor{currentstroke}{rgb}{0.000000,0.000000,0.000000}%
\pgfsetstrokecolor{currentstroke}%
\pgfsetdash{}{0pt}%
\pgfpathmoveto{\pgfqpoint{2.864510in}{2.407759in}}%
\pgfpathlineto{\pgfqpoint{2.878107in}{2.392772in}}%
\pgfpathlineto{\pgfqpoint{2.891697in}{2.378067in}}%
\pgfpathlineto{\pgfqpoint{2.905283in}{2.363642in}}%
\pgfpathlineto{\pgfqpoint{2.918862in}{2.349493in}}%
\pgfpathlineto{\pgfqpoint{2.927148in}{2.357253in}}%
\pgfpathlineto{\pgfqpoint{2.935425in}{2.365128in}}%
\pgfpathlineto{\pgfqpoint{2.943693in}{2.373116in}}%
\pgfpathlineto{\pgfqpoint{2.951953in}{2.381217in}}%
\pgfpathlineto{\pgfqpoint{2.938396in}{2.395175in}}%
\pgfpathlineto{\pgfqpoint{2.924833in}{2.409411in}}%
\pgfpathlineto{\pgfqpoint{2.911266in}{2.423925in}}%
\pgfpathlineto{\pgfqpoint{2.897692in}{2.438721in}}%
\pgfpathlineto{\pgfqpoint{2.889410in}{2.430799in}}%
\pgfpathlineto{\pgfqpoint{2.881119in}{2.422997in}}%
\pgfpathlineto{\pgfqpoint{2.872819in}{2.415316in}}%
\pgfpathlineto{\pgfqpoint{2.864510in}{2.407759in}}%
\pgfpathclose%
\pgfusepath{fill}%
\end{pgfscope}%
\begin{pgfscope}%
\pgfpathrectangle{\pgfqpoint{1.150000in}{0.150000in}}{\pgfqpoint{5.700000in}{5.700000in}}%
\pgfusepath{clip}%
\pgfsetbuttcap%
\pgfsetroundjoin%
\definecolor{currentfill}{rgb}{0.136408,0.541173,0.554483}%
\pgfsetfillcolor{currentfill}%
\pgfsetfillopacity{0.800000}%
\pgfsetlinewidth{0.000000pt}%
\definecolor{currentstroke}{rgb}{0.000000,0.000000,0.000000}%
\pgfsetstrokecolor{currentstroke}%
\pgfsetdash{}{0pt}%
\pgfpathmoveto{\pgfqpoint{5.680867in}{3.323580in}}%
\pgfpathlineto{\pgfqpoint{5.695125in}{3.329577in}}%
\pgfpathlineto{\pgfqpoint{5.709399in}{3.335746in}}%
\pgfpathlineto{\pgfqpoint{5.723691in}{3.342084in}}%
\pgfpathlineto{\pgfqpoint{5.737999in}{3.348593in}}%
\pgfpathlineto{\pgfqpoint{5.745235in}{3.355571in}}%
\pgfpathlineto{\pgfqpoint{5.752472in}{3.362775in}}%
\pgfpathlineto{\pgfqpoint{5.759712in}{3.370214in}}%
\pgfpathlineto{\pgfqpoint{5.766955in}{3.377897in}}%
\pgfpathlineto{\pgfqpoint{5.752680in}{3.372139in}}%
\pgfpathlineto{\pgfqpoint{5.738421in}{3.366549in}}%
\pgfpathlineto{\pgfqpoint{5.724178in}{3.361129in}}%
\pgfpathlineto{\pgfqpoint{5.709952in}{3.355879in}}%
\pgfpathlineto{\pgfqpoint{5.702677in}{3.347436in}}%
\pgfpathlineto{\pgfqpoint{5.695405in}{3.339245in}}%
\pgfpathlineto{\pgfqpoint{5.688135in}{3.331295in}}%
\pgfpathlineto{\pgfqpoint{5.680867in}{3.323580in}}%
\pgfpathclose%
\pgfusepath{fill}%
\end{pgfscope}%
\begin{pgfscope}%
\pgfpathrectangle{\pgfqpoint{1.150000in}{0.150000in}}{\pgfqpoint{5.700000in}{5.700000in}}%
\pgfusepath{clip}%
\pgfsetbuttcap%
\pgfsetroundjoin%
\definecolor{currentfill}{rgb}{0.280894,0.078907,0.402329}%
\pgfsetfillcolor{currentfill}%
\pgfsetfillopacity{0.800000}%
\pgfsetlinewidth{0.000000pt}%
\definecolor{currentstroke}{rgb}{0.000000,0.000000,0.000000}%
\pgfsetstrokecolor{currentstroke}%
\pgfsetdash{}{0pt}%
\pgfpathmoveto{\pgfqpoint{3.449168in}{2.146999in}}%
\pgfpathlineto{\pgfqpoint{3.462657in}{2.141445in}}%
\pgfpathlineto{\pgfqpoint{3.476149in}{2.136113in}}%
\pgfpathlineto{\pgfqpoint{3.489643in}{2.131001in}}%
\pgfpathlineto{\pgfqpoint{3.503140in}{2.126110in}}%
\pgfpathlineto{\pgfqpoint{3.511189in}{2.136090in}}%
\pgfpathlineto{\pgfqpoint{3.519233in}{2.146101in}}%
\pgfpathlineto{\pgfqpoint{3.527270in}{2.156144in}}%
\pgfpathlineto{\pgfqpoint{3.535302in}{2.166216in}}%
\pgfpathlineto{\pgfqpoint{3.521817in}{2.171019in}}%
\pgfpathlineto{\pgfqpoint{3.508334in}{2.176042in}}%
\pgfpathlineto{\pgfqpoint{3.494855in}{2.181285in}}%
\pgfpathlineto{\pgfqpoint{3.481379in}{2.186750in}}%
\pgfpathlineto{\pgfqpoint{3.473335in}{2.176754in}}%
\pgfpathlineto{\pgfqpoint{3.465285in}{2.166797in}}%
\pgfpathlineto{\pgfqpoint{3.457230in}{2.156878in}}%
\pgfpathlineto{\pgfqpoint{3.449168in}{2.146999in}}%
\pgfpathclose%
\pgfusepath{fill}%
\end{pgfscope}%
\begin{pgfscope}%
\pgfpathrectangle{\pgfqpoint{1.150000in}{0.150000in}}{\pgfqpoint{5.700000in}{5.700000in}}%
\pgfusepath{clip}%
\pgfsetbuttcap%
\pgfsetroundjoin%
\definecolor{currentfill}{rgb}{0.282656,0.100196,0.422160}%
\pgfsetfillcolor{currentfill}%
\pgfsetfillopacity{0.800000}%
\pgfsetlinewidth{0.000000pt}%
\definecolor{currentstroke}{rgb}{0.000000,0.000000,0.000000}%
\pgfsetstrokecolor{currentstroke}%
\pgfsetdash{}{0pt}%
\pgfpathmoveto{\pgfqpoint{3.168363in}{2.193886in}}%
\pgfpathlineto{\pgfqpoint{3.181869in}{2.184334in}}%
\pgfpathlineto{\pgfqpoint{3.195374in}{2.175025in}}%
\pgfpathlineto{\pgfqpoint{3.208878in}{2.165959in}}%
\pgfpathlineto{\pgfqpoint{3.222382in}{2.157133in}}%
\pgfpathlineto{\pgfqpoint{3.230538in}{2.166179in}}%
\pgfpathlineto{\pgfqpoint{3.238687in}{2.175294in}}%
\pgfpathlineto{\pgfqpoint{3.246829in}{2.184479in}}%
\pgfpathlineto{\pgfqpoint{3.254965in}{2.193730in}}%
\pgfpathlineto{\pgfqpoint{3.241477in}{2.202403in}}%
\pgfpathlineto{\pgfqpoint{3.227990in}{2.211317in}}%
\pgfpathlineto{\pgfqpoint{3.214502in}{2.220472in}}%
\pgfpathlineto{\pgfqpoint{3.201013in}{2.229870in}}%
\pgfpathlineto{\pgfqpoint{3.192861in}{2.220760in}}%
\pgfpathlineto{\pgfqpoint{3.184702in}{2.211725in}}%
\pgfpathlineto{\pgfqpoint{3.176536in}{2.202767in}}%
\pgfpathlineto{\pgfqpoint{3.168363in}{2.193886in}}%
\pgfpathclose%
\pgfusepath{fill}%
\end{pgfscope}%
\begin{pgfscope}%
\pgfpathrectangle{\pgfqpoint{1.150000in}{0.150000in}}{\pgfqpoint{5.700000in}{5.700000in}}%
\pgfusepath{clip}%
\pgfsetbuttcap%
\pgfsetroundjoin%
\definecolor{currentfill}{rgb}{0.252194,0.269783,0.531579}%
\pgfsetfillcolor{currentfill}%
\pgfsetfillopacity{0.800000}%
\pgfsetlinewidth{0.000000pt}%
\definecolor{currentstroke}{rgb}{0.000000,0.000000,0.000000}%
\pgfsetstrokecolor{currentstroke}%
\pgfsetdash{}{0pt}%
\pgfpathmoveto{\pgfqpoint{2.700783in}{2.610452in}}%
\pgfpathlineto{\pgfqpoint{2.714473in}{2.591893in}}%
\pgfpathlineto{\pgfqpoint{2.728154in}{2.573646in}}%
\pgfpathlineto{\pgfqpoint{2.741826in}{2.555709in}}%
\pgfpathlineto{\pgfqpoint{2.755488in}{2.538078in}}%
\pgfpathlineto{\pgfqpoint{2.763848in}{2.545227in}}%
\pgfpathlineto{\pgfqpoint{2.772197in}{2.552515in}}%
\pgfpathlineto{\pgfqpoint{2.780537in}{2.559940in}}%
\pgfpathlineto{\pgfqpoint{2.788867in}{2.567502in}}%
\pgfpathlineto{\pgfqpoint{2.775230in}{2.584938in}}%
\pgfpathlineto{\pgfqpoint{2.761585in}{2.602681in}}%
\pgfpathlineto{\pgfqpoint{2.747931in}{2.620733in}}%
\pgfpathlineto{\pgfqpoint{2.734267in}{2.639097in}}%
\pgfpathlineto{\pgfqpoint{2.725911in}{2.631718in}}%
\pgfpathlineto{\pgfqpoint{2.717546in}{2.624483in}}%
\pgfpathlineto{\pgfqpoint{2.709170in}{2.617393in}}%
\pgfpathlineto{\pgfqpoint{2.700783in}{2.610452in}}%
\pgfpathclose%
\pgfusepath{fill}%
\end{pgfscope}%
\begin{pgfscope}%
\pgfpathrectangle{\pgfqpoint{1.150000in}{0.150000in}}{\pgfqpoint{5.700000in}{5.700000in}}%
\pgfusepath{clip}%
\pgfsetbuttcap%
\pgfsetroundjoin%
\definecolor{currentfill}{rgb}{0.218130,0.347432,0.550038}%
\pgfsetfillcolor{currentfill}%
\pgfsetfillopacity{0.800000}%
\pgfsetlinewidth{0.000000pt}%
\definecolor{currentstroke}{rgb}{0.000000,0.000000,0.000000}%
\pgfsetstrokecolor{currentstroke}%
\pgfsetdash{}{0pt}%
\pgfpathmoveto{\pgfqpoint{4.791293in}{2.762067in}}%
\pgfpathlineto{\pgfqpoint{4.805183in}{2.767150in}}%
\pgfpathlineto{\pgfqpoint{4.819087in}{2.772414in}}%
\pgfpathlineto{\pgfqpoint{4.833004in}{2.777860in}}%
\pgfpathlineto{\pgfqpoint{4.846934in}{2.783487in}}%
\pgfpathlineto{\pgfqpoint{4.854517in}{2.791331in}}%
\pgfpathlineto{\pgfqpoint{4.862095in}{2.799198in}}%
\pgfpathlineto{\pgfqpoint{4.869668in}{2.807093in}}%
\pgfpathlineto{\pgfqpoint{4.877236in}{2.815021in}}%
\pgfpathlineto{\pgfqpoint{4.863319in}{2.809788in}}%
\pgfpathlineto{\pgfqpoint{4.849416in}{2.804735in}}%
\pgfpathlineto{\pgfqpoint{4.835526in}{2.799863in}}%
\pgfpathlineto{\pgfqpoint{4.821650in}{2.795172in}}%
\pgfpathlineto{\pgfqpoint{4.814068in}{2.786840in}}%
\pgfpathlineto{\pgfqpoint{4.806481in}{2.778548in}}%
\pgfpathlineto{\pgfqpoint{4.798890in}{2.770292in}}%
\pgfpathlineto{\pgfqpoint{4.791293in}{2.762067in}}%
\pgfpathclose%
\pgfusepath{fill}%
\end{pgfscope}%
\begin{pgfscope}%
\pgfpathrectangle{\pgfqpoint{1.150000in}{0.150000in}}{\pgfqpoint{5.700000in}{5.700000in}}%
\pgfusepath{clip}%
\pgfsetbuttcap%
\pgfsetroundjoin%
\definecolor{currentfill}{rgb}{0.280255,0.165693,0.476498}%
\pgfsetfillcolor{currentfill}%
\pgfsetfillopacity{0.800000}%
\pgfsetlinewidth{0.000000pt}%
\definecolor{currentstroke}{rgb}{0.000000,0.000000,0.000000}%
\pgfsetstrokecolor{currentstroke}%
\pgfsetdash{}{0pt}%
\pgfpathmoveto{\pgfqpoint{2.918862in}{2.349493in}}%
\pgfpathlineto{\pgfqpoint{2.932437in}{2.335620in}}%
\pgfpathlineto{\pgfqpoint{2.946006in}{2.322020in}}%
\pgfpathlineto{\pgfqpoint{2.959571in}{2.308690in}}%
\pgfpathlineto{\pgfqpoint{2.973131in}{2.295629in}}%
\pgfpathlineto{\pgfqpoint{2.981395in}{2.303590in}}%
\pgfpathlineto{\pgfqpoint{2.989649in}{2.311658in}}%
\pgfpathlineto{\pgfqpoint{2.997896in}{2.319831in}}%
\pgfpathlineto{\pgfqpoint{3.006134in}{2.328108in}}%
\pgfpathlineto{\pgfqpoint{2.992596in}{2.340981in}}%
\pgfpathlineto{\pgfqpoint{2.979052in}{2.354121in}}%
\pgfpathlineto{\pgfqpoint{2.965505in}{2.367533in}}%
\pgfpathlineto{\pgfqpoint{2.951953in}{2.381217in}}%
\pgfpathlineto{\pgfqpoint{2.943693in}{2.373116in}}%
\pgfpathlineto{\pgfqpoint{2.935425in}{2.365128in}}%
\pgfpathlineto{\pgfqpoint{2.927148in}{2.357253in}}%
\pgfpathlineto{\pgfqpoint{2.918862in}{2.349493in}}%
\pgfpathclose%
\pgfusepath{fill}%
\end{pgfscope}%
\begin{pgfscope}%
\pgfpathrectangle{\pgfqpoint{1.150000in}{0.150000in}}{\pgfqpoint{5.700000in}{5.700000in}}%
\pgfusepath{clip}%
\pgfsetbuttcap%
\pgfsetroundjoin%
\definecolor{currentfill}{rgb}{0.282327,0.094955,0.417331}%
\pgfsetfillcolor{currentfill}%
\pgfsetfillopacity{0.800000}%
\pgfsetlinewidth{0.000000pt}%
\definecolor{currentstroke}{rgb}{0.000000,0.000000,0.000000}%
\pgfsetstrokecolor{currentstroke}%
\pgfsetdash{}{0pt}%
\pgfpathmoveto{\pgfqpoint{3.675311in}{2.176621in}}%
\pgfpathlineto{\pgfqpoint{3.688823in}{2.173806in}}%
\pgfpathlineto{\pgfqpoint{3.702341in}{2.171201in}}%
\pgfpathlineto{\pgfqpoint{3.715864in}{2.168804in}}%
\pgfpathlineto{\pgfqpoint{3.729392in}{2.166616in}}%
\pgfpathlineto{\pgfqpoint{3.737366in}{2.176949in}}%
\pgfpathlineto{\pgfqpoint{3.745335in}{2.187289in}}%
\pgfpathlineto{\pgfqpoint{3.753298in}{2.197635in}}%
\pgfpathlineto{\pgfqpoint{3.761257in}{2.207988in}}%
\pgfpathlineto{\pgfqpoint{3.747738in}{2.210152in}}%
\pgfpathlineto{\pgfqpoint{3.734224in}{2.212523in}}%
\pgfpathlineto{\pgfqpoint{3.720716in}{2.215103in}}%
\pgfpathlineto{\pgfqpoint{3.707213in}{2.217893in}}%
\pgfpathlineto{\pgfqpoint{3.699245in}{2.207554in}}%
\pgfpathlineto{\pgfqpoint{3.691272in}{2.197229in}}%
\pgfpathlineto{\pgfqpoint{3.683294in}{2.186918in}}%
\pgfpathlineto{\pgfqpoint{3.675311in}{2.176621in}}%
\pgfpathclose%
\pgfusepath{fill}%
\end{pgfscope}%
\begin{pgfscope}%
\pgfpathrectangle{\pgfqpoint{1.150000in}{0.150000in}}{\pgfqpoint{5.700000in}{5.700000in}}%
\pgfusepath{clip}%
\pgfsetbuttcap%
\pgfsetroundjoin%
\definecolor{currentfill}{rgb}{0.208623,0.367752,0.552675}%
\pgfsetfillcolor{currentfill}%
\pgfsetfillopacity{0.800000}%
\pgfsetlinewidth{0.000000pt}%
\definecolor{currentstroke}{rgb}{0.000000,0.000000,0.000000}%
\pgfsetstrokecolor{currentstroke}%
\pgfsetdash{}{0pt}%
\pgfpathmoveto{\pgfqpoint{4.877236in}{2.815021in}}%
\pgfpathlineto{\pgfqpoint{4.891166in}{2.820435in}}%
\pgfpathlineto{\pgfqpoint{4.905110in}{2.826029in}}%
\pgfpathlineto{\pgfqpoint{4.919068in}{2.831803in}}%
\pgfpathlineto{\pgfqpoint{4.933040in}{2.837756in}}%
\pgfpathlineto{\pgfqpoint{4.940588in}{2.845308in}}%
\pgfpathlineto{\pgfqpoint{4.948132in}{2.852893in}}%
\pgfpathlineto{\pgfqpoint{4.955670in}{2.860518in}}%
\pgfpathlineto{\pgfqpoint{4.963204in}{2.868187in}}%
\pgfpathlineto{\pgfqpoint{4.949247in}{2.862659in}}%
\pgfpathlineto{\pgfqpoint{4.935304in}{2.857311in}}%
\pgfpathlineto{\pgfqpoint{4.921375in}{2.852143in}}%
\pgfpathlineto{\pgfqpoint{4.907459in}{2.847154in}}%
\pgfpathlineto{\pgfqpoint{4.899910in}{2.839048in}}%
\pgfpathlineto{\pgfqpoint{4.892357in}{2.830994in}}%
\pgfpathlineto{\pgfqpoint{4.884799in}{2.822986in}}%
\pgfpathlineto{\pgfqpoint{4.877236in}{2.815021in}}%
\pgfpathclose%
\pgfusepath{fill}%
\end{pgfscope}%
\begin{pgfscope}%
\pgfpathrectangle{\pgfqpoint{1.150000in}{0.150000in}}{\pgfqpoint{5.700000in}{5.700000in}}%
\pgfusepath{clip}%
\pgfsetbuttcap%
\pgfsetroundjoin%
\definecolor{currentfill}{rgb}{0.239346,0.300855,0.540844}%
\pgfsetfillcolor{currentfill}%
\pgfsetfillopacity{0.800000}%
\pgfsetlinewidth{0.000000pt}%
\definecolor{currentstroke}{rgb}{0.000000,0.000000,0.000000}%
\pgfsetstrokecolor{currentstroke}%
\pgfsetdash{}{0pt}%
\pgfpathmoveto{\pgfqpoint{2.645923in}{2.687869in}}%
\pgfpathlineto{\pgfqpoint{2.659653in}{2.668032in}}%
\pgfpathlineto{\pgfqpoint{2.673373in}{2.648518in}}%
\pgfpathlineto{\pgfqpoint{2.687083in}{2.629326in}}%
\pgfpathlineto{\pgfqpoint{2.700783in}{2.610452in}}%
\pgfpathlineto{\pgfqpoint{2.709170in}{2.617393in}}%
\pgfpathlineto{\pgfqpoint{2.717546in}{2.624483in}}%
\pgfpathlineto{\pgfqpoint{2.725911in}{2.631718in}}%
\pgfpathlineto{\pgfqpoint{2.734267in}{2.639097in}}%
\pgfpathlineto{\pgfqpoint{2.720595in}{2.657775in}}%
\pgfpathlineto{\pgfqpoint{2.706912in}{2.676771in}}%
\pgfpathlineto{\pgfqpoint{2.693220in}{2.696088in}}%
\pgfpathlineto{\pgfqpoint{2.679517in}{2.715728in}}%
\pgfpathlineto{\pgfqpoint{2.671134in}{2.708533in}}%
\pgfpathlineto{\pgfqpoint{2.662741in}{2.701490in}}%
\pgfpathlineto{\pgfqpoint{2.654337in}{2.694602in}}%
\pgfpathlineto{\pgfqpoint{2.645923in}{2.687869in}}%
\pgfpathclose%
\pgfusepath{fill}%
\end{pgfscope}%
\begin{pgfscope}%
\pgfpathrectangle{\pgfqpoint{1.150000in}{0.150000in}}{\pgfqpoint{5.700000in}{5.700000in}}%
\pgfusepath{clip}%
\pgfsetbuttcap%
\pgfsetroundjoin%
\definecolor{currentfill}{rgb}{0.129933,0.559582,0.551864}%
\pgfsetfillcolor{currentfill}%
\pgfsetfillopacity{0.800000}%
\pgfsetlinewidth{0.000000pt}%
\definecolor{currentstroke}{rgb}{0.000000,0.000000,0.000000}%
\pgfsetstrokecolor{currentstroke}%
\pgfsetdash{}{0pt}%
\pgfpathmoveto{\pgfqpoint{5.766955in}{3.377897in}}%
\pgfpathlineto{\pgfqpoint{5.781247in}{3.383825in}}%
\pgfpathlineto{\pgfqpoint{5.795556in}{3.389923in}}%
\pgfpathlineto{\pgfqpoint{5.809881in}{3.396190in}}%
\pgfpathlineto{\pgfqpoint{5.824224in}{3.402627in}}%
\pgfpathlineto{\pgfqpoint{5.831436in}{3.409791in}}%
\pgfpathlineto{\pgfqpoint{5.838650in}{3.417209in}}%
\pgfpathlineto{\pgfqpoint{5.845868in}{3.424888in}}%
\pgfpathlineto{\pgfqpoint{5.831551in}{3.419036in}}%
\pgfpathlineto{\pgfqpoint{5.817251in}{3.413352in}}%
\pgfpathlineto{\pgfqpoint{5.802968in}{3.407837in}}%
\pgfpathlineto{\pgfqpoint{5.788701in}{3.402491in}}%
\pgfpathlineto{\pgfqpoint{5.781449in}{3.394027in}}%
\pgfpathlineto{\pgfqpoint{5.774200in}{3.385832in}}%
\pgfpathlineto{\pgfqpoint{5.766955in}{3.377897in}}%
\pgfpathclose%
\pgfusepath{fill}%
\end{pgfscope}%
\begin{pgfscope}%
\pgfpathrectangle{\pgfqpoint{1.150000in}{0.150000in}}{\pgfqpoint{5.700000in}{5.700000in}}%
\pgfusepath{clip}%
\pgfsetbuttcap%
\pgfsetroundjoin%
\definecolor{currentfill}{rgb}{0.282623,0.140926,0.457517}%
\pgfsetfillcolor{currentfill}%
\pgfsetfillopacity{0.800000}%
\pgfsetlinewidth{0.000000pt}%
\definecolor{currentstroke}{rgb}{0.000000,0.000000,0.000000}%
\pgfsetstrokecolor{currentstroke}%
\pgfsetdash{}{0pt}%
\pgfpathmoveto{\pgfqpoint{2.973131in}{2.295629in}}%
\pgfpathlineto{\pgfqpoint{2.986688in}{2.282835in}}%
\pgfpathlineto{\pgfqpoint{3.000240in}{2.270306in}}%
\pgfpathlineto{\pgfqpoint{3.013788in}{2.258040in}}%
\pgfpathlineto{\pgfqpoint{3.027333in}{2.246034in}}%
\pgfpathlineto{\pgfqpoint{3.035575in}{2.254195in}}%
\pgfpathlineto{\pgfqpoint{3.043809in}{2.262454in}}%
\pgfpathlineto{\pgfqpoint{3.052035in}{2.270811in}}%
\pgfpathlineto{\pgfqpoint{3.060253in}{2.279264in}}%
\pgfpathlineto{\pgfqpoint{3.046729in}{2.291083in}}%
\pgfpathlineto{\pgfqpoint{3.033201in}{2.303161in}}%
\pgfpathlineto{\pgfqpoint{3.019669in}{2.315503in}}%
\pgfpathlineto{\pgfqpoint{3.006134in}{2.328108in}}%
\pgfpathlineto{\pgfqpoint{2.997896in}{2.319831in}}%
\pgfpathlineto{\pgfqpoint{2.989649in}{2.311658in}}%
\pgfpathlineto{\pgfqpoint{2.981395in}{2.303590in}}%
\pgfpathlineto{\pgfqpoint{2.973131in}{2.295629in}}%
\pgfpathclose%
\pgfusepath{fill}%
\end{pgfscope}%
\begin{pgfscope}%
\pgfpathrectangle{\pgfqpoint{1.150000in}{0.150000in}}{\pgfqpoint{5.700000in}{5.700000in}}%
\pgfusepath{clip}%
\pgfsetbuttcap%
\pgfsetroundjoin%
\definecolor{currentfill}{rgb}{0.199430,0.387607,0.554642}%
\pgfsetfillcolor{currentfill}%
\pgfsetfillopacity{0.800000}%
\pgfsetlinewidth{0.000000pt}%
\definecolor{currentstroke}{rgb}{0.000000,0.000000,0.000000}%
\pgfsetstrokecolor{currentstroke}%
\pgfsetdash{}{0pt}%
\pgfpathmoveto{\pgfqpoint{4.963204in}{2.868187in}}%
\pgfpathlineto{\pgfqpoint{4.977175in}{2.873893in}}%
\pgfpathlineto{\pgfqpoint{4.991160in}{2.879779in}}%
\pgfpathlineto{\pgfqpoint{5.005159in}{2.885844in}}%
\pgfpathlineto{\pgfqpoint{5.019174in}{2.892087in}}%
\pgfpathlineto{\pgfqpoint{5.026686in}{2.899358in}}%
\pgfpathlineto{\pgfqpoint{5.034194in}{2.906676in}}%
\pgfpathlineto{\pgfqpoint{5.041698in}{2.914045in}}%
\pgfpathlineto{\pgfqpoint{5.049197in}{2.921472in}}%
\pgfpathlineto{\pgfqpoint{5.035199in}{2.915688in}}%
\pgfpathlineto{\pgfqpoint{5.021216in}{2.910082in}}%
\pgfpathlineto{\pgfqpoint{5.007248in}{2.904654in}}%
\pgfpathlineto{\pgfqpoint{4.993293in}{2.899405in}}%
\pgfpathlineto{\pgfqpoint{4.985777in}{2.891509in}}%
\pgfpathlineto{\pgfqpoint{4.978257in}{2.883677in}}%
\pgfpathlineto{\pgfqpoint{4.970733in}{2.875905in}}%
\pgfpathlineto{\pgfqpoint{4.963204in}{2.868187in}}%
\pgfpathclose%
\pgfusepath{fill}%
\end{pgfscope}%
\begin{pgfscope}%
\pgfpathrectangle{\pgfqpoint{1.150000in}{0.150000in}}{\pgfqpoint{5.700000in}{5.700000in}}%
\pgfusepath{clip}%
\pgfsetbuttcap%
\pgfsetroundjoin%
\definecolor{currentfill}{rgb}{0.281446,0.084320,0.407414}%
\pgfsetfillcolor{currentfill}%
\pgfsetfillopacity{0.800000}%
\pgfsetlinewidth{0.000000pt}%
\definecolor{currentstroke}{rgb}{0.000000,0.000000,0.000000}%
\pgfsetstrokecolor{currentstroke}%
\pgfsetdash{}{0pt}%
\pgfpathmoveto{\pgfqpoint{3.589280in}{2.149180in}}%
\pgfpathlineto{\pgfqpoint{3.602784in}{2.145459in}}%
\pgfpathlineto{\pgfqpoint{3.616293in}{2.141952in}}%
\pgfpathlineto{\pgfqpoint{3.629806in}{2.138658in}}%
\pgfpathlineto{\pgfqpoint{3.643323in}{2.135575in}}%
\pgfpathlineto{\pgfqpoint{3.651328in}{2.145816in}}%
\pgfpathlineto{\pgfqpoint{3.659328in}{2.156070in}}%
\pgfpathlineto{\pgfqpoint{3.667322in}{2.166339in}}%
\pgfpathlineto{\pgfqpoint{3.675311in}{2.176621in}}%
\pgfpathlineto{\pgfqpoint{3.661803in}{2.179647in}}%
\pgfpathlineto{\pgfqpoint{3.648300in}{2.182885in}}%
\pgfpathlineto{\pgfqpoint{3.634802in}{2.186335in}}%
\pgfpathlineto{\pgfqpoint{3.621309in}{2.189998in}}%
\pgfpathlineto{\pgfqpoint{3.613310in}{2.179761in}}%
\pgfpathlineto{\pgfqpoint{3.605305in}{2.169545in}}%
\pgfpathlineto{\pgfqpoint{3.597295in}{2.159352in}}%
\pgfpathlineto{\pgfqpoint{3.589280in}{2.149180in}}%
\pgfpathclose%
\pgfusepath{fill}%
\end{pgfscope}%
\begin{pgfscope}%
\pgfpathrectangle{\pgfqpoint{1.150000in}{0.150000in}}{\pgfqpoint{5.700000in}{5.700000in}}%
\pgfusepath{clip}%
\pgfsetbuttcap%
\pgfsetroundjoin%
\definecolor{currentfill}{rgb}{0.190631,0.407061,0.556089}%
\pgfsetfillcolor{currentfill}%
\pgfsetfillopacity{0.800000}%
\pgfsetlinewidth{0.000000pt}%
\definecolor{currentstroke}{rgb}{0.000000,0.000000,0.000000}%
\pgfsetstrokecolor{currentstroke}%
\pgfsetdash{}{0pt}%
\pgfpathmoveto{\pgfqpoint{5.049197in}{2.921472in}}%
\pgfpathlineto{\pgfqpoint{5.063209in}{2.927434in}}%
\pgfpathlineto{\pgfqpoint{5.077235in}{2.933574in}}%
\pgfpathlineto{\pgfqpoint{5.091276in}{2.939891in}}%
\pgfpathlineto{\pgfqpoint{5.105333in}{2.946387in}}%
\pgfpathlineto{\pgfqpoint{5.112810in}{2.953396in}}%
\pgfpathlineto{\pgfqpoint{5.120282in}{2.960465in}}%
\pgfpathlineto{\pgfqpoint{5.127750in}{2.967600in}}%
\pgfpathlineto{\pgfqpoint{5.135214in}{2.974806in}}%
\pgfpathlineto{\pgfqpoint{5.121176in}{2.968803in}}%
\pgfpathlineto{\pgfqpoint{5.107153in}{2.962977in}}%
\pgfpathlineto{\pgfqpoint{5.093144in}{2.957328in}}%
\pgfpathlineto{\pgfqpoint{5.079150in}{2.951856in}}%
\pgfpathlineto{\pgfqpoint{5.071668in}{2.944147in}}%
\pgfpathlineto{\pgfqpoint{5.064182in}{2.936517in}}%
\pgfpathlineto{\pgfqpoint{5.056691in}{2.928961in}}%
\pgfpathlineto{\pgfqpoint{5.049197in}{2.921472in}}%
\pgfpathclose%
\pgfusepath{fill}%
\end{pgfscope}%
\begin{pgfscope}%
\pgfpathrectangle{\pgfqpoint{1.150000in}{0.150000in}}{\pgfqpoint{5.700000in}{5.700000in}}%
\pgfusepath{clip}%
\pgfsetbuttcap%
\pgfsetroundjoin%
\definecolor{currentfill}{rgb}{0.280267,0.073417,0.397163}%
\pgfsetfillcolor{currentfill}%
\pgfsetfillopacity{0.800000}%
\pgfsetlinewidth{0.000000pt}%
\definecolor{currentstroke}{rgb}{0.000000,0.000000,0.000000}%
\pgfsetstrokecolor{currentstroke}%
\pgfsetdash{}{0pt}%
\pgfpathmoveto{\pgfqpoint{3.362877in}{2.132825in}}%
\pgfpathlineto{\pgfqpoint{3.376371in}{2.126251in}}%
\pgfpathlineto{\pgfqpoint{3.389867in}{2.119904in}}%
\pgfpathlineto{\pgfqpoint{3.403364in}{2.113782in}}%
\pgfpathlineto{\pgfqpoint{3.416864in}{2.107885in}}%
\pgfpathlineto{\pgfqpoint{3.424949in}{2.117601in}}%
\pgfpathlineto{\pgfqpoint{3.433028in}{2.127359in}}%
\pgfpathlineto{\pgfqpoint{3.441101in}{2.137159in}}%
\pgfpathlineto{\pgfqpoint{3.449168in}{2.146999in}}%
\pgfpathlineto{\pgfqpoint{3.435682in}{2.152776in}}%
\pgfpathlineto{\pgfqpoint{3.422198in}{2.158777in}}%
\pgfpathlineto{\pgfqpoint{3.408716in}{2.165003in}}%
\pgfpathlineto{\pgfqpoint{3.395236in}{2.171456in}}%
\pgfpathlineto{\pgfqpoint{3.387156in}{2.161725in}}%
\pgfpathlineto{\pgfqpoint{3.379069in}{2.152042in}}%
\pgfpathlineto{\pgfqpoint{3.370976in}{2.142409in}}%
\pgfpathlineto{\pgfqpoint{3.362877in}{2.132825in}}%
\pgfpathclose%
\pgfusepath{fill}%
\end{pgfscope}%
\begin{pgfscope}%
\pgfpathrectangle{\pgfqpoint{1.150000in}{0.150000in}}{\pgfqpoint{5.700000in}{5.700000in}}%
\pgfusepath{clip}%
\pgfsetbuttcap%
\pgfsetroundjoin%
\definecolor{currentfill}{rgb}{0.281446,0.084320,0.407414}%
\pgfsetfillcolor{currentfill}%
\pgfsetfillopacity{0.800000}%
\pgfsetlinewidth{0.000000pt}%
\definecolor{currentstroke}{rgb}{0.000000,0.000000,0.000000}%
\pgfsetstrokecolor{currentstroke}%
\pgfsetdash{}{0pt}%
\pgfpathmoveto{\pgfqpoint{3.222382in}{2.157133in}}%
\pgfpathlineto{\pgfqpoint{3.235886in}{2.148545in}}%
\pgfpathlineto{\pgfqpoint{3.249389in}{2.140195in}}%
\pgfpathlineto{\pgfqpoint{3.262893in}{2.132082in}}%
\pgfpathlineto{\pgfqpoint{3.276396in}{2.124203in}}%
\pgfpathlineto{\pgfqpoint{3.284536in}{2.133413in}}%
\pgfpathlineto{\pgfqpoint{3.292669in}{2.142686in}}%
\pgfpathlineto{\pgfqpoint{3.300795in}{2.152019in}}%
\pgfpathlineto{\pgfqpoint{3.308915in}{2.161412in}}%
\pgfpathlineto{\pgfqpoint{3.295427in}{2.169138in}}%
\pgfpathlineto{\pgfqpoint{3.281939in}{2.177099in}}%
\pgfpathlineto{\pgfqpoint{3.268452in}{2.185296in}}%
\pgfpathlineto{\pgfqpoint{3.254965in}{2.193730in}}%
\pgfpathlineto{\pgfqpoint{3.246829in}{2.184479in}}%
\pgfpathlineto{\pgfqpoint{3.238687in}{2.175294in}}%
\pgfpathlineto{\pgfqpoint{3.230538in}{2.166179in}}%
\pgfpathlineto{\pgfqpoint{3.222382in}{2.157133in}}%
\pgfpathclose%
\pgfusepath{fill}%
\end{pgfscope}%
\begin{pgfscope}%
\pgfpathrectangle{\pgfqpoint{1.150000in}{0.150000in}}{\pgfqpoint{5.700000in}{5.700000in}}%
\pgfusepath{clip}%
\pgfsetbuttcap%
\pgfsetroundjoin%
\definecolor{currentfill}{rgb}{0.223925,0.334994,0.548053}%
\pgfsetfillcolor{currentfill}%
\pgfsetfillopacity{0.800000}%
\pgfsetlinewidth{0.000000pt}%
\definecolor{currentstroke}{rgb}{0.000000,0.000000,0.000000}%
\pgfsetstrokecolor{currentstroke}%
\pgfsetdash{}{0pt}%
\pgfpathmoveto{\pgfqpoint{2.590887in}{2.770522in}}%
\pgfpathlineto{\pgfqpoint{2.604663in}{2.749357in}}%
\pgfpathlineto{\pgfqpoint{2.618428in}{2.728528in}}%
\pgfpathlineto{\pgfqpoint{2.632181in}{2.708034in}}%
\pgfpathlineto{\pgfqpoint{2.645923in}{2.687869in}}%
\pgfpathlineto{\pgfqpoint{2.654337in}{2.694602in}}%
\pgfpathlineto{\pgfqpoint{2.662741in}{2.701490in}}%
\pgfpathlineto{\pgfqpoint{2.671134in}{2.708533in}}%
\pgfpathlineto{\pgfqpoint{2.679517in}{2.715728in}}%
\pgfpathlineto{\pgfqpoint{2.665803in}{2.735695in}}%
\pgfpathlineto{\pgfqpoint{2.652079in}{2.755992in}}%
\pgfpathlineto{\pgfqpoint{2.638343in}{2.776621in}}%
\pgfpathlineto{\pgfqpoint{2.624596in}{2.797587in}}%
\pgfpathlineto{\pgfqpoint{2.616185in}{2.790578in}}%
\pgfpathlineto{\pgfqpoint{2.607764in}{2.783730in}}%
\pgfpathlineto{\pgfqpoint{2.599331in}{2.777044in}}%
\pgfpathlineto{\pgfqpoint{2.590887in}{2.770522in}}%
\pgfpathclose%
\pgfusepath{fill}%
\end{pgfscope}%
\begin{pgfscope}%
\pgfpathrectangle{\pgfqpoint{1.150000in}{0.150000in}}{\pgfqpoint{5.700000in}{5.700000in}}%
\pgfusepath{clip}%
\pgfsetbuttcap%
\pgfsetroundjoin%
\definecolor{currentfill}{rgb}{0.283187,0.125848,0.444960}%
\pgfsetfillcolor{currentfill}%
\pgfsetfillopacity{0.800000}%
\pgfsetlinewidth{0.000000pt}%
\definecolor{currentstroke}{rgb}{0.000000,0.000000,0.000000}%
\pgfsetstrokecolor{currentstroke}%
\pgfsetdash{}{0pt}%
\pgfpathmoveto{\pgfqpoint{3.027333in}{2.246034in}}%
\pgfpathlineto{\pgfqpoint{3.040875in}{2.234288in}}%
\pgfpathlineto{\pgfqpoint{3.054414in}{2.222798in}}%
\pgfpathlineto{\pgfqpoint{3.067950in}{2.211564in}}%
\pgfpathlineto{\pgfqpoint{3.081484in}{2.200584in}}%
\pgfpathlineto{\pgfqpoint{3.089706in}{2.208944in}}%
\pgfpathlineto{\pgfqpoint{3.097919in}{2.217394in}}%
\pgfpathlineto{\pgfqpoint{3.106126in}{2.225934in}}%
\pgfpathlineto{\pgfqpoint{3.114324in}{2.234562in}}%
\pgfpathlineto{\pgfqpoint{3.100810in}{2.245356in}}%
\pgfpathlineto{\pgfqpoint{3.087294in}{2.256403in}}%
\pgfpathlineto{\pgfqpoint{3.073775in}{2.267705in}}%
\pgfpathlineto{\pgfqpoint{3.060253in}{2.279264in}}%
\pgfpathlineto{\pgfqpoint{3.052035in}{2.270811in}}%
\pgfpathlineto{\pgfqpoint{3.043809in}{2.262454in}}%
\pgfpathlineto{\pgfqpoint{3.035575in}{2.254195in}}%
\pgfpathlineto{\pgfqpoint{3.027333in}{2.246034in}}%
\pgfpathclose%
\pgfusepath{fill}%
\end{pgfscope}%
\begin{pgfscope}%
\pgfpathrectangle{\pgfqpoint{1.150000in}{0.150000in}}{\pgfqpoint{5.700000in}{5.700000in}}%
\pgfusepath{clip}%
\pgfsetbuttcap%
\pgfsetroundjoin%
\definecolor{currentfill}{rgb}{0.182256,0.426184,0.557120}%
\pgfsetfillcolor{currentfill}%
\pgfsetfillopacity{0.800000}%
\pgfsetlinewidth{0.000000pt}%
\definecolor{currentstroke}{rgb}{0.000000,0.000000,0.000000}%
\pgfsetstrokecolor{currentstroke}%
\pgfsetdash{}{0pt}%
\pgfpathmoveto{\pgfqpoint{5.135214in}{2.974806in}}%
\pgfpathlineto{\pgfqpoint{5.149267in}{2.980986in}}%
\pgfpathlineto{\pgfqpoint{5.163335in}{2.987343in}}%
\pgfpathlineto{\pgfqpoint{5.177418in}{2.993877in}}%
\pgfpathlineto{\pgfqpoint{5.191516in}{3.000587in}}%
\pgfpathlineto{\pgfqpoint{5.198957in}{3.007358in}}%
\pgfpathlineto{\pgfqpoint{5.206393in}{3.014203in}}%
\pgfpathlineto{\pgfqpoint{5.213826in}{3.021130in}}%
\pgfpathlineto{\pgfqpoint{5.221255in}{3.028144in}}%
\pgfpathlineto{\pgfqpoint{5.207176in}{3.021959in}}%
\pgfpathlineto{\pgfqpoint{5.193113in}{3.015950in}}%
\pgfpathlineto{\pgfqpoint{5.179065in}{3.010116in}}%
\pgfpathlineto{\pgfqpoint{5.165032in}{3.004459in}}%
\pgfpathlineto{\pgfqpoint{5.157583in}{2.996910in}}%
\pgfpathlineto{\pgfqpoint{5.150130in}{2.989455in}}%
\pgfpathlineto{\pgfqpoint{5.142674in}{2.982089in}}%
\pgfpathlineto{\pgfqpoint{5.135214in}{2.974806in}}%
\pgfpathclose%
\pgfusepath{fill}%
\end{pgfscope}%
\begin{pgfscope}%
\pgfpathrectangle{\pgfqpoint{1.150000in}{0.150000in}}{\pgfqpoint{5.700000in}{5.700000in}}%
\pgfusepath{clip}%
\pgfsetbuttcap%
\pgfsetroundjoin%
\definecolor{currentfill}{rgb}{0.278826,0.175490,0.483397}%
\pgfsetfillcolor{currentfill}%
\pgfsetfillopacity{0.800000}%
\pgfsetlinewidth{0.000000pt}%
\definecolor{currentstroke}{rgb}{0.000000,0.000000,0.000000}%
\pgfsetstrokecolor{currentstroke}%
\pgfsetdash{}{0pt}%
\pgfpathmoveto{\pgfqpoint{4.073172in}{2.324179in}}%
\pgfpathlineto{\pgfqpoint{4.086790in}{2.325323in}}%
\pgfpathlineto{\pgfqpoint{4.100417in}{2.326663in}}%
\pgfpathlineto{\pgfqpoint{4.114053in}{2.328198in}}%
\pgfpathlineto{\pgfqpoint{4.127698in}{2.329927in}}%
\pgfpathlineto{\pgfqpoint{4.135549in}{2.339980in}}%
\pgfpathlineto{\pgfqpoint{4.143396in}{2.350014in}}%
\pgfpathlineto{\pgfqpoint{4.151237in}{2.360028in}}%
\pgfpathlineto{\pgfqpoint{4.159073in}{2.370025in}}%
\pgfpathlineto{\pgfqpoint{4.145436in}{2.368399in}}%
\pgfpathlineto{\pgfqpoint{4.131808in}{2.366966in}}%
\pgfpathlineto{\pgfqpoint{4.118189in}{2.365728in}}%
\pgfpathlineto{\pgfqpoint{4.104579in}{2.364686in}}%
\pgfpathlineto{\pgfqpoint{4.096735in}{2.354575in}}%
\pgfpathlineto{\pgfqpoint{4.088886in}{2.344454in}}%
\pgfpathlineto{\pgfqpoint{4.081032in}{2.334323in}}%
\pgfpathlineto{\pgfqpoint{4.073172in}{2.324179in}}%
\pgfpathclose%
\pgfusepath{fill}%
\end{pgfscope}%
\begin{pgfscope}%
\pgfpathrectangle{\pgfqpoint{1.150000in}{0.150000in}}{\pgfqpoint{5.700000in}{5.700000in}}%
\pgfusepath{clip}%
\pgfsetbuttcap%
\pgfsetroundjoin%
\definecolor{currentfill}{rgb}{0.174274,0.445044,0.557792}%
\pgfsetfillcolor{currentfill}%
\pgfsetfillopacity{0.800000}%
\pgfsetlinewidth{0.000000pt}%
\definecolor{currentstroke}{rgb}{0.000000,0.000000,0.000000}%
\pgfsetstrokecolor{currentstroke}%
\pgfsetdash{}{0pt}%
\pgfpathmoveto{\pgfqpoint{5.221255in}{3.028144in}}%
\pgfpathlineto{\pgfqpoint{5.235348in}{3.034505in}}%
\pgfpathlineto{\pgfqpoint{5.249457in}{3.041042in}}%
\pgfpathlineto{\pgfqpoint{5.263582in}{3.047755in}}%
\pgfpathlineto{\pgfqpoint{5.277722in}{3.054643in}}%
\pgfpathlineto{\pgfqpoint{5.285126in}{3.061204in}}%
\pgfpathlineto{\pgfqpoint{5.292527in}{3.067857in}}%
\pgfpathlineto{\pgfqpoint{5.299924in}{3.074607in}}%
\pgfpathlineto{\pgfqpoint{5.307318in}{3.081462in}}%
\pgfpathlineto{\pgfqpoint{5.293200in}{3.075132in}}%
\pgfpathlineto{\pgfqpoint{5.279097in}{3.068976in}}%
\pgfpathlineto{\pgfqpoint{5.265009in}{3.062996in}}%
\pgfpathlineto{\pgfqpoint{5.250937in}{3.057191in}}%
\pgfpathlineto{\pgfqpoint{5.243521in}{3.049768in}}%
\pgfpathlineto{\pgfqpoint{5.236102in}{3.042457in}}%
\pgfpathlineto{\pgfqpoint{5.228680in}{3.035251in}}%
\pgfpathlineto{\pgfqpoint{5.221255in}{3.028144in}}%
\pgfpathclose%
\pgfusepath{fill}%
\end{pgfscope}%
\begin{pgfscope}%
\pgfpathrectangle{\pgfqpoint{1.150000in}{0.150000in}}{\pgfqpoint{5.700000in}{5.700000in}}%
\pgfusepath{clip}%
\pgfsetbuttcap%
\pgfsetroundjoin%
\definecolor{currentfill}{rgb}{0.275191,0.194905,0.496005}%
\pgfsetfillcolor{currentfill}%
\pgfsetfillopacity{0.800000}%
\pgfsetlinewidth{0.000000pt}%
\definecolor{currentstroke}{rgb}{0.000000,0.000000,0.000000}%
\pgfsetstrokecolor{currentstroke}%
\pgfsetdash{}{0pt}%
\pgfpathmoveto{\pgfqpoint{4.159073in}{2.370025in}}%
\pgfpathlineto{\pgfqpoint{4.172720in}{2.371846in}}%
\pgfpathlineto{\pgfqpoint{4.186376in}{2.373860in}}%
\pgfpathlineto{\pgfqpoint{4.200042in}{2.376067in}}%
\pgfpathlineto{\pgfqpoint{4.213717in}{2.378466in}}%
\pgfpathlineto{\pgfqpoint{4.221540in}{2.388325in}}%
\pgfpathlineto{\pgfqpoint{4.229359in}{2.398163in}}%
\pgfpathlineto{\pgfqpoint{4.237172in}{2.407981in}}%
\pgfpathlineto{\pgfqpoint{4.244979in}{2.417781in}}%
\pgfpathlineto{\pgfqpoint{4.231312in}{2.415516in}}%
\pgfpathlineto{\pgfqpoint{4.217654in}{2.413443in}}%
\pgfpathlineto{\pgfqpoint{4.204005in}{2.411563in}}%
\pgfpathlineto{\pgfqpoint{4.190366in}{2.409877in}}%
\pgfpathlineto{\pgfqpoint{4.182551in}{2.399931in}}%
\pgfpathlineto{\pgfqpoint{4.174730in}{2.389975in}}%
\pgfpathlineto{\pgfqpoint{4.166904in}{2.380007in}}%
\pgfpathlineto{\pgfqpoint{4.159073in}{2.370025in}}%
\pgfpathclose%
\pgfusepath{fill}%
\end{pgfscope}%
\begin{pgfscope}%
\pgfpathrectangle{\pgfqpoint{1.150000in}{0.150000in}}{\pgfqpoint{5.700000in}{5.700000in}}%
\pgfusepath{clip}%
\pgfsetbuttcap%
\pgfsetroundjoin%
\definecolor{currentfill}{rgb}{0.281412,0.155834,0.469201}%
\pgfsetfillcolor{currentfill}%
\pgfsetfillopacity{0.800000}%
\pgfsetlinewidth{0.000000pt}%
\definecolor{currentstroke}{rgb}{0.000000,0.000000,0.000000}%
\pgfsetstrokecolor{currentstroke}%
\pgfsetdash{}{0pt}%
\pgfpathmoveto{\pgfqpoint{3.987267in}{2.280560in}}%
\pgfpathlineto{\pgfqpoint{4.000859in}{2.280986in}}%
\pgfpathlineto{\pgfqpoint{4.014459in}{2.281611in}}%
\pgfpathlineto{\pgfqpoint{4.028068in}{2.282433in}}%
\pgfpathlineto{\pgfqpoint{4.041685in}{2.283452in}}%
\pgfpathlineto{\pgfqpoint{4.049564in}{2.293659in}}%
\pgfpathlineto{\pgfqpoint{4.057439in}{2.303848in}}%
\pgfpathlineto{\pgfqpoint{4.065308in}{2.314021in}}%
\pgfpathlineto{\pgfqpoint{4.073172in}{2.324179in}}%
\pgfpathlineto{\pgfqpoint{4.059563in}{2.323230in}}%
\pgfpathlineto{\pgfqpoint{4.045962in}{2.322479in}}%
\pgfpathlineto{\pgfqpoint{4.032370in}{2.321924in}}%
\pgfpathlineto{\pgfqpoint{4.018786in}{2.321567in}}%
\pgfpathlineto{\pgfqpoint{4.010914in}{2.311328in}}%
\pgfpathlineto{\pgfqpoint{4.003037in}{2.301081in}}%
\pgfpathlineto{\pgfqpoint{3.995155in}{2.290825in}}%
\pgfpathlineto{\pgfqpoint{3.987267in}{2.280560in}}%
\pgfpathclose%
\pgfusepath{fill}%
\end{pgfscope}%
\begin{pgfscope}%
\pgfpathrectangle{\pgfqpoint{1.150000in}{0.150000in}}{\pgfqpoint{5.700000in}{5.700000in}}%
\pgfusepath{clip}%
\pgfsetbuttcap%
\pgfsetroundjoin%
\definecolor{currentfill}{rgb}{0.270595,0.214069,0.507052}%
\pgfsetfillcolor{currentfill}%
\pgfsetfillopacity{0.800000}%
\pgfsetlinewidth{0.000000pt}%
\definecolor{currentstroke}{rgb}{0.000000,0.000000,0.000000}%
\pgfsetstrokecolor{currentstroke}%
\pgfsetdash{}{0pt}%
\pgfpathmoveto{\pgfqpoint{4.244979in}{2.417781in}}%
\pgfpathlineto{\pgfqpoint{4.258657in}{2.420237in}}%
\pgfpathlineto{\pgfqpoint{4.272345in}{2.422885in}}%
\pgfpathlineto{\pgfqpoint{4.286043in}{2.425723in}}%
\pgfpathlineto{\pgfqpoint{4.299752in}{2.428752in}}%
\pgfpathlineto{\pgfqpoint{4.307547in}{2.438382in}}%
\pgfpathlineto{\pgfqpoint{4.315336in}{2.447989in}}%
\pgfpathlineto{\pgfqpoint{4.323120in}{2.457577in}}%
\pgfpathlineto{\pgfqpoint{4.330899in}{2.467148in}}%
\pgfpathlineto{\pgfqpoint{4.317198in}{2.464285in}}%
\pgfpathlineto{\pgfqpoint{4.303508in}{2.461613in}}%
\pgfpathlineto{\pgfqpoint{4.289828in}{2.459132in}}%
\pgfpathlineto{\pgfqpoint{4.276158in}{2.456841in}}%
\pgfpathlineto{\pgfqpoint{4.268371in}{2.447093in}}%
\pgfpathlineto{\pgfqpoint{4.260579in}{2.437334in}}%
\pgfpathlineto{\pgfqpoint{4.252782in}{2.427564in}}%
\pgfpathlineto{\pgfqpoint{4.244979in}{2.417781in}}%
\pgfpathclose%
\pgfusepath{fill}%
\end{pgfscope}%
\begin{pgfscope}%
\pgfpathrectangle{\pgfqpoint{1.150000in}{0.150000in}}{\pgfqpoint{5.700000in}{5.700000in}}%
\pgfusepath{clip}%
\pgfsetbuttcap%
\pgfsetroundjoin%
\definecolor{currentfill}{rgb}{0.280894,0.078907,0.402329}%
\pgfsetfillcolor{currentfill}%
\pgfsetfillopacity{0.800000}%
\pgfsetlinewidth{0.000000pt}%
\definecolor{currentstroke}{rgb}{0.000000,0.000000,0.000000}%
\pgfsetstrokecolor{currentstroke}%
\pgfsetdash{}{0pt}%
\pgfpathmoveto{\pgfqpoint{3.503140in}{2.126110in}}%
\pgfpathlineto{\pgfqpoint{3.516641in}{2.121437in}}%
\pgfpathlineto{\pgfqpoint{3.530145in}{2.116982in}}%
\pgfpathlineto{\pgfqpoint{3.543652in}{2.112743in}}%
\pgfpathlineto{\pgfqpoint{3.557164in}{2.108720in}}%
\pgfpathlineto{\pgfqpoint{3.565201in}{2.118800in}}%
\pgfpathlineto{\pgfqpoint{3.573233in}{2.128904in}}%
\pgfpathlineto{\pgfqpoint{3.581259in}{2.139030in}}%
\pgfpathlineto{\pgfqpoint{3.589280in}{2.149180in}}%
\pgfpathlineto{\pgfqpoint{3.575780in}{2.153115in}}%
\pgfpathlineto{\pgfqpoint{3.562284in}{2.157265in}}%
\pgfpathlineto{\pgfqpoint{3.548791in}{2.161632in}}%
\pgfpathlineto{\pgfqpoint{3.535302in}{2.166216in}}%
\pgfpathlineto{\pgfqpoint{3.527270in}{2.156144in}}%
\pgfpathlineto{\pgfqpoint{3.519233in}{2.146101in}}%
\pgfpathlineto{\pgfqpoint{3.511189in}{2.136090in}}%
\pgfpathlineto{\pgfqpoint{3.503140in}{2.126110in}}%
\pgfpathclose%
\pgfusepath{fill}%
\end{pgfscope}%
\begin{pgfscope}%
\pgfpathrectangle{\pgfqpoint{1.150000in}{0.150000in}}{\pgfqpoint{5.700000in}{5.700000in}}%
\pgfusepath{clip}%
\pgfsetbuttcap%
\pgfsetroundjoin%
\definecolor{currentfill}{rgb}{0.282884,0.135920,0.453427}%
\pgfsetfillcolor{currentfill}%
\pgfsetfillopacity{0.800000}%
\pgfsetlinewidth{0.000000pt}%
\definecolor{currentstroke}{rgb}{0.000000,0.000000,0.000000}%
\pgfsetstrokecolor{currentstroke}%
\pgfsetdash{}{0pt}%
\pgfpathmoveto{\pgfqpoint{3.901346in}{2.239511in}}%
\pgfpathlineto{\pgfqpoint{3.914915in}{2.239178in}}%
\pgfpathlineto{\pgfqpoint{3.928491in}{2.239046in}}%
\pgfpathlineto{\pgfqpoint{3.942075in}{2.239113in}}%
\pgfpathlineto{\pgfqpoint{3.955667in}{2.239380in}}%
\pgfpathlineto{\pgfqpoint{3.963575in}{2.249694in}}%
\pgfpathlineto{\pgfqpoint{3.971477in}{2.259995in}}%
\pgfpathlineto{\pgfqpoint{3.979375in}{2.270283in}}%
\pgfpathlineto{\pgfqpoint{3.987267in}{2.280560in}}%
\pgfpathlineto{\pgfqpoint{3.973683in}{2.280331in}}%
\pgfpathlineto{\pgfqpoint{3.960107in}{2.280302in}}%
\pgfpathlineto{\pgfqpoint{3.946538in}{2.280473in}}%
\pgfpathlineto{\pgfqpoint{3.932977in}{2.280844in}}%
\pgfpathlineto{\pgfqpoint{3.925077in}{2.270517in}}%
\pgfpathlineto{\pgfqpoint{3.917172in}{2.260187in}}%
\pgfpathlineto{\pgfqpoint{3.909261in}{2.249852in}}%
\pgfpathlineto{\pgfqpoint{3.901346in}{2.239511in}}%
\pgfpathclose%
\pgfusepath{fill}%
\end{pgfscope}%
\begin{pgfscope}%
\pgfpathrectangle{\pgfqpoint{1.150000in}{0.150000in}}{\pgfqpoint{5.700000in}{5.700000in}}%
\pgfusepath{clip}%
\pgfsetbuttcap%
\pgfsetroundjoin%
\definecolor{currentfill}{rgb}{0.263663,0.237631,0.518762}%
\pgfsetfillcolor{currentfill}%
\pgfsetfillopacity{0.800000}%
\pgfsetlinewidth{0.000000pt}%
\definecolor{currentstroke}{rgb}{0.000000,0.000000,0.000000}%
\pgfsetstrokecolor{currentstroke}%
\pgfsetdash{}{0pt}%
\pgfpathmoveto{\pgfqpoint{4.330899in}{2.467148in}}%
\pgfpathlineto{\pgfqpoint{4.344610in}{2.470200in}}%
\pgfpathlineto{\pgfqpoint{4.358332in}{2.473441in}}%
\pgfpathlineto{\pgfqpoint{4.372065in}{2.476872in}}%
\pgfpathlineto{\pgfqpoint{4.385809in}{2.480491in}}%
\pgfpathlineto{\pgfqpoint{4.393574in}{2.489860in}}%
\pgfpathlineto{\pgfqpoint{4.401334in}{2.499208in}}%
\pgfpathlineto{\pgfqpoint{4.409089in}{2.508539in}}%
\pgfpathlineto{\pgfqpoint{4.416838in}{2.517854in}}%
\pgfpathlineto{\pgfqpoint{4.403102in}{2.514433in}}%
\pgfpathlineto{\pgfqpoint{4.389378in}{2.511202in}}%
\pgfpathlineto{\pgfqpoint{4.375665in}{2.508159in}}%
\pgfpathlineto{\pgfqpoint{4.361962in}{2.505305in}}%
\pgfpathlineto{\pgfqpoint{4.354204in}{2.495779in}}%
\pgfpathlineto{\pgfqpoint{4.346441in}{2.486246in}}%
\pgfpathlineto{\pgfqpoint{4.338672in}{2.476703in}}%
\pgfpathlineto{\pgfqpoint{4.330899in}{2.467148in}}%
\pgfpathclose%
\pgfusepath{fill}%
\end{pgfscope}%
\begin{pgfscope}%
\pgfpathrectangle{\pgfqpoint{1.150000in}{0.150000in}}{\pgfqpoint{5.700000in}{5.700000in}}%
\pgfusepath{clip}%
\pgfsetbuttcap%
\pgfsetroundjoin%
\definecolor{currentfill}{rgb}{0.166617,0.463708,0.558119}%
\pgfsetfillcolor{currentfill}%
\pgfsetfillopacity{0.800000}%
\pgfsetlinewidth{0.000000pt}%
\definecolor{currentstroke}{rgb}{0.000000,0.000000,0.000000}%
\pgfsetstrokecolor{currentstroke}%
\pgfsetdash{}{0pt}%
\pgfpathmoveto{\pgfqpoint{5.307318in}{3.081462in}}%
\pgfpathlineto{\pgfqpoint{5.321452in}{3.087967in}}%
\pgfpathlineto{\pgfqpoint{5.335602in}{3.094647in}}%
\pgfpathlineto{\pgfqpoint{5.349768in}{3.101502in}}%
\pgfpathlineto{\pgfqpoint{5.363949in}{3.108532in}}%
\pgfpathlineto{\pgfqpoint{5.371317in}{3.114918in}}%
\pgfpathlineto{\pgfqpoint{5.378682in}{3.121414in}}%
\pgfpathlineto{\pgfqpoint{5.386045in}{3.128026in}}%
\pgfpathlineto{\pgfqpoint{5.393404in}{3.134761in}}%
\pgfpathlineto{\pgfqpoint{5.379246in}{3.128322in}}%
\pgfpathlineto{\pgfqpoint{5.365104in}{3.122057in}}%
\pgfpathlineto{\pgfqpoint{5.350978in}{3.115967in}}%
\pgfpathlineto{\pgfqpoint{5.336867in}{3.110050in}}%
\pgfpathlineto{\pgfqpoint{5.329484in}{3.102715in}}%
\pgfpathlineto{\pgfqpoint{5.322098in}{3.095509in}}%
\pgfpathlineto{\pgfqpoint{5.314709in}{3.088427in}}%
\pgfpathlineto{\pgfqpoint{5.307318in}{3.081462in}}%
\pgfpathclose%
\pgfusepath{fill}%
\end{pgfscope}%
\begin{pgfscope}%
\pgfpathrectangle{\pgfqpoint{1.150000in}{0.150000in}}{\pgfqpoint{5.700000in}{5.700000in}}%
\pgfusepath{clip}%
\pgfsetbuttcap%
\pgfsetroundjoin%
\definecolor{currentfill}{rgb}{0.283197,0.115680,0.436115}%
\pgfsetfillcolor{currentfill}%
\pgfsetfillopacity{0.800000}%
\pgfsetlinewidth{0.000000pt}%
\definecolor{currentstroke}{rgb}{0.000000,0.000000,0.000000}%
\pgfsetstrokecolor{currentstroke}%
\pgfsetdash{}{0pt}%
\pgfpathmoveto{\pgfqpoint{3.815393in}{2.201399in}}%
\pgfpathlineto{\pgfqpoint{3.828943in}{2.200264in}}%
\pgfpathlineto{\pgfqpoint{3.842499in}{2.199332in}}%
\pgfpathlineto{\pgfqpoint{3.856062in}{2.198603in}}%
\pgfpathlineto{\pgfqpoint{3.869633in}{2.198076in}}%
\pgfpathlineto{\pgfqpoint{3.877568in}{2.208447in}}%
\pgfpathlineto{\pgfqpoint{3.885499in}{2.218809in}}%
\pgfpathlineto{\pgfqpoint{3.893425in}{2.229163in}}%
\pgfpathlineto{\pgfqpoint{3.901346in}{2.239511in}}%
\pgfpathlineto{\pgfqpoint{3.887784in}{2.240045in}}%
\pgfpathlineto{\pgfqpoint{3.874229in}{2.240781in}}%
\pgfpathlineto{\pgfqpoint{3.860681in}{2.241719in}}%
\pgfpathlineto{\pgfqpoint{3.847139in}{2.242861in}}%
\pgfpathlineto{\pgfqpoint{3.839211in}{2.232495in}}%
\pgfpathlineto{\pgfqpoint{3.831276in}{2.222130in}}%
\pgfpathlineto{\pgfqpoint{3.823337in}{2.211765in}}%
\pgfpathlineto{\pgfqpoint{3.815393in}{2.201399in}}%
\pgfpathclose%
\pgfusepath{fill}%
\end{pgfscope}%
\begin{pgfscope}%
\pgfpathrectangle{\pgfqpoint{1.150000in}{0.150000in}}{\pgfqpoint{5.700000in}{5.700000in}}%
\pgfusepath{clip}%
\pgfsetbuttcap%
\pgfsetroundjoin%
\definecolor{currentfill}{rgb}{0.282910,0.105393,0.426902}%
\pgfsetfillcolor{currentfill}%
\pgfsetfillopacity{0.800000}%
\pgfsetlinewidth{0.000000pt}%
\definecolor{currentstroke}{rgb}{0.000000,0.000000,0.000000}%
\pgfsetstrokecolor{currentstroke}%
\pgfsetdash{}{0pt}%
\pgfpathmoveto{\pgfqpoint{3.081484in}{2.200584in}}%
\pgfpathlineto{\pgfqpoint{3.095015in}{2.189856in}}%
\pgfpathlineto{\pgfqpoint{3.108544in}{2.179378in}}%
\pgfpathlineto{\pgfqpoint{3.122072in}{2.169148in}}%
\pgfpathlineto{\pgfqpoint{3.135598in}{2.159165in}}%
\pgfpathlineto{\pgfqpoint{3.143800in}{2.167722in}}%
\pgfpathlineto{\pgfqpoint{3.151995in}{2.176363in}}%
\pgfpathlineto{\pgfqpoint{3.160182in}{2.185084in}}%
\pgfpathlineto{\pgfqpoint{3.168363in}{2.193886in}}%
\pgfpathlineto{\pgfqpoint{3.154855in}{2.203684in}}%
\pgfpathlineto{\pgfqpoint{3.141347in}{2.213727in}}%
\pgfpathlineto{\pgfqpoint{3.127836in}{2.224020in}}%
\pgfpathlineto{\pgfqpoint{3.114324in}{2.234562in}}%
\pgfpathlineto{\pgfqpoint{3.106126in}{2.225934in}}%
\pgfpathlineto{\pgfqpoint{3.097919in}{2.217394in}}%
\pgfpathlineto{\pgfqpoint{3.089706in}{2.208944in}}%
\pgfpathlineto{\pgfqpoint{3.081484in}{2.200584in}}%
\pgfpathclose%
\pgfusepath{fill}%
\end{pgfscope}%
\begin{pgfscope}%
\pgfpathrectangle{\pgfqpoint{1.150000in}{0.150000in}}{\pgfqpoint{5.700000in}{5.700000in}}%
\pgfusepath{clip}%
\pgfsetbuttcap%
\pgfsetroundjoin%
\definecolor{currentfill}{rgb}{0.255645,0.260703,0.528312}%
\pgfsetfillcolor{currentfill}%
\pgfsetfillopacity{0.800000}%
\pgfsetlinewidth{0.000000pt}%
\definecolor{currentstroke}{rgb}{0.000000,0.000000,0.000000}%
\pgfsetstrokecolor{currentstroke}%
\pgfsetdash{}{0pt}%
\pgfpathmoveto{\pgfqpoint{4.416838in}{2.517854in}}%
\pgfpathlineto{\pgfqpoint{4.430584in}{2.521462in}}%
\pgfpathlineto{\pgfqpoint{4.444343in}{2.525258in}}%
\pgfpathlineto{\pgfqpoint{4.458112in}{2.529241in}}%
\pgfpathlineto{\pgfqpoint{4.471893in}{2.533411in}}%
\pgfpathlineto{\pgfqpoint{4.479628in}{2.542494in}}%
\pgfpathlineto{\pgfqpoint{4.487358in}{2.551559in}}%
\pgfpathlineto{\pgfqpoint{4.495082in}{2.560609in}}%
\pgfpathlineto{\pgfqpoint{4.502801in}{2.569647in}}%
\pgfpathlineto{\pgfqpoint{4.489029in}{2.565709in}}%
\pgfpathlineto{\pgfqpoint{4.475268in}{2.561957in}}%
\pgfpathlineto{\pgfqpoint{4.461519in}{2.558392in}}%
\pgfpathlineto{\pgfqpoint{4.447782in}{2.555014in}}%
\pgfpathlineto{\pgfqpoint{4.440054in}{2.545733in}}%
\pgfpathlineto{\pgfqpoint{4.432320in}{2.536448in}}%
\pgfpathlineto{\pgfqpoint{4.424582in}{2.527156in}}%
\pgfpathlineto{\pgfqpoint{4.416838in}{2.517854in}}%
\pgfpathclose%
\pgfusepath{fill}%
\end{pgfscope}%
\begin{pgfscope}%
\pgfpathrectangle{\pgfqpoint{1.150000in}{0.150000in}}{\pgfqpoint{5.700000in}{5.700000in}}%
\pgfusepath{clip}%
\pgfsetbuttcap%
\pgfsetroundjoin%
\definecolor{currentfill}{rgb}{0.208623,0.367752,0.552675}%
\pgfsetfillcolor{currentfill}%
\pgfsetfillopacity{0.800000}%
\pgfsetlinewidth{0.000000pt}%
\definecolor{currentstroke}{rgb}{0.000000,0.000000,0.000000}%
\pgfsetstrokecolor{currentstroke}%
\pgfsetdash{}{0pt}%
\pgfpathmoveto{\pgfqpoint{2.535653in}{2.858618in}}%
\pgfpathlineto{\pgfqpoint{2.549481in}{2.836072in}}%
\pgfpathlineto{\pgfqpoint{2.563296in}{2.813876in}}%
\pgfpathlineto{\pgfqpoint{2.577098in}{2.792027in}}%
\pgfpathlineto{\pgfqpoint{2.590887in}{2.770522in}}%
\pgfpathlineto{\pgfqpoint{2.599331in}{2.777044in}}%
\pgfpathlineto{\pgfqpoint{2.607764in}{2.783730in}}%
\pgfpathlineto{\pgfqpoint{2.616185in}{2.790578in}}%
\pgfpathlineto{\pgfqpoint{2.624596in}{2.797587in}}%
\pgfpathlineto{\pgfqpoint{2.610837in}{2.818893in}}%
\pgfpathlineto{\pgfqpoint{2.597065in}{2.840541in}}%
\pgfpathlineto{\pgfqpoint{2.583281in}{2.862536in}}%
\pgfpathlineto{\pgfqpoint{2.569483in}{2.884880in}}%
\pgfpathlineto{\pgfqpoint{2.561043in}{2.878059in}}%
\pgfpathlineto{\pgfqpoint{2.552592in}{2.871407in}}%
\pgfpathlineto{\pgfqpoint{2.544128in}{2.864926in}}%
\pgfpathlineto{\pgfqpoint{2.535653in}{2.858618in}}%
\pgfpathclose%
\pgfusepath{fill}%
\end{pgfscope}%
\begin{pgfscope}%
\pgfpathrectangle{\pgfqpoint{1.150000in}{0.150000in}}{\pgfqpoint{5.700000in}{5.700000in}}%
\pgfusepath{clip}%
\pgfsetbuttcap%
\pgfsetroundjoin%
\definecolor{currentfill}{rgb}{0.159194,0.482237,0.558073}%
\pgfsetfillcolor{currentfill}%
\pgfsetfillopacity{0.800000}%
\pgfsetlinewidth{0.000000pt}%
\definecolor{currentstroke}{rgb}{0.000000,0.000000,0.000000}%
\pgfsetstrokecolor{currentstroke}%
\pgfsetdash{}{0pt}%
\pgfpathmoveto{\pgfqpoint{5.393404in}{3.134761in}}%
\pgfpathlineto{\pgfqpoint{5.407578in}{3.141373in}}%
\pgfpathlineto{\pgfqpoint{5.421768in}{3.148160in}}%
\pgfpathlineto{\pgfqpoint{5.435974in}{3.155120in}}%
\pgfpathlineto{\pgfqpoint{5.450197in}{3.162255in}}%
\pgfpathlineto{\pgfqpoint{5.457529in}{3.168507in}}%
\pgfpathlineto{\pgfqpoint{5.464859in}{3.174888in}}%
\pgfpathlineto{\pgfqpoint{5.472188in}{3.181405in}}%
\pgfpathlineto{\pgfqpoint{5.479514in}{3.188064in}}%
\pgfpathlineto{\pgfqpoint{5.465317in}{3.181553in}}%
\pgfpathlineto{\pgfqpoint{5.451136in}{3.175216in}}%
\pgfpathlineto{\pgfqpoint{5.436971in}{3.169051in}}%
\pgfpathlineto{\pgfqpoint{5.422823in}{3.163060in}}%
\pgfpathlineto{\pgfqpoint{5.415471in}{3.155767in}}%
\pgfpathlineto{\pgfqpoint{5.408117in}{3.148624in}}%
\pgfpathlineto{\pgfqpoint{5.400762in}{3.141625in}}%
\pgfpathlineto{\pgfqpoint{5.393404in}{3.134761in}}%
\pgfpathclose%
\pgfusepath{fill}%
\end{pgfscope}%
\begin{pgfscope}%
\pgfpathrectangle{\pgfqpoint{1.150000in}{0.150000in}}{\pgfqpoint{5.700000in}{5.700000in}}%
\pgfusepath{clip}%
\pgfsetbuttcap%
\pgfsetroundjoin%
\definecolor{currentfill}{rgb}{0.246811,0.283237,0.535941}%
\pgfsetfillcolor{currentfill}%
\pgfsetfillopacity{0.800000}%
\pgfsetlinewidth{0.000000pt}%
\definecolor{currentstroke}{rgb}{0.000000,0.000000,0.000000}%
\pgfsetstrokecolor{currentstroke}%
\pgfsetdash{}{0pt}%
\pgfpathmoveto{\pgfqpoint{4.502801in}{2.569647in}}%
\pgfpathlineto{\pgfqpoint{4.516584in}{2.573773in}}%
\pgfpathlineto{\pgfqpoint{4.530380in}{2.578084in}}%
\pgfpathlineto{\pgfqpoint{4.544188in}{2.582581in}}%
\pgfpathlineto{\pgfqpoint{4.558008in}{2.587263in}}%
\pgfpathlineto{\pgfqpoint{4.565712in}{2.596040in}}%
\pgfpathlineto{\pgfqpoint{4.573410in}{2.604804in}}%
\pgfpathlineto{\pgfqpoint{4.581103in}{2.613556in}}%
\pgfpathlineto{\pgfqpoint{4.588791in}{2.622302in}}%
\pgfpathlineto{\pgfqpoint{4.574980in}{2.617883in}}%
\pgfpathlineto{\pgfqpoint{4.561183in}{2.613650in}}%
\pgfpathlineto{\pgfqpoint{4.547397in}{2.609602in}}%
\pgfpathlineto{\pgfqpoint{4.533623in}{2.605740in}}%
\pgfpathlineto{\pgfqpoint{4.525925in}{2.596720in}}%
\pgfpathlineto{\pgfqpoint{4.518222in}{2.587700in}}%
\pgfpathlineto{\pgfqpoint{4.510514in}{2.578677in}}%
\pgfpathlineto{\pgfqpoint{4.502801in}{2.569647in}}%
\pgfpathclose%
\pgfusepath{fill}%
\end{pgfscope}%
\begin{pgfscope}%
\pgfpathrectangle{\pgfqpoint{1.150000in}{0.150000in}}{\pgfqpoint{5.700000in}{5.700000in}}%
\pgfusepath{clip}%
\pgfsetbuttcap%
\pgfsetroundjoin%
\definecolor{currentfill}{rgb}{0.282656,0.100196,0.422160}%
\pgfsetfillcolor{currentfill}%
\pgfsetfillopacity{0.800000}%
\pgfsetlinewidth{0.000000pt}%
\definecolor{currentstroke}{rgb}{0.000000,0.000000,0.000000}%
\pgfsetstrokecolor{currentstroke}%
\pgfsetdash{}{0pt}%
\pgfpathmoveto{\pgfqpoint{3.729392in}{2.166616in}}%
\pgfpathlineto{\pgfqpoint{3.742926in}{2.164634in}}%
\pgfpathlineto{\pgfqpoint{3.756466in}{2.162860in}}%
\pgfpathlineto{\pgfqpoint{3.770012in}{2.161291in}}%
\pgfpathlineto{\pgfqpoint{3.783564in}{2.159927in}}%
\pgfpathlineto{\pgfqpoint{3.791529in}{2.170297in}}%
\pgfpathlineto{\pgfqpoint{3.799489in}{2.180666in}}%
\pgfpathlineto{\pgfqpoint{3.807444in}{2.191033in}}%
\pgfpathlineto{\pgfqpoint{3.815393in}{2.201399in}}%
\pgfpathlineto{\pgfqpoint{3.801850in}{2.202739in}}%
\pgfpathlineto{\pgfqpoint{3.788313in}{2.204283in}}%
\pgfpathlineto{\pgfqpoint{3.774782in}{2.206032in}}%
\pgfpathlineto{\pgfqpoint{3.761257in}{2.207988in}}%
\pgfpathlineto{\pgfqpoint{3.753298in}{2.197635in}}%
\pgfpathlineto{\pgfqpoint{3.745335in}{2.187289in}}%
\pgfpathlineto{\pgfqpoint{3.737366in}{2.176949in}}%
\pgfpathlineto{\pgfqpoint{3.729392in}{2.166616in}}%
\pgfpathclose%
\pgfusepath{fill}%
\end{pgfscope}%
\begin{pgfscope}%
\pgfpathrectangle{\pgfqpoint{1.150000in}{0.150000in}}{\pgfqpoint{5.700000in}{5.700000in}}%
\pgfusepath{clip}%
\pgfsetbuttcap%
\pgfsetroundjoin%
\definecolor{currentfill}{rgb}{0.150476,0.504369,0.557430}%
\pgfsetfillcolor{currentfill}%
\pgfsetfillopacity{0.800000}%
\pgfsetlinewidth{0.000000pt}%
\definecolor{currentstroke}{rgb}{0.000000,0.000000,0.000000}%
\pgfsetstrokecolor{currentstroke}%
\pgfsetdash{}{0pt}%
\pgfpathmoveto{\pgfqpoint{5.479514in}{3.188064in}}%
\pgfpathlineto{\pgfqpoint{5.493727in}{3.194748in}}%
\pgfpathlineto{\pgfqpoint{5.507956in}{3.201604in}}%
\pgfpathlineto{\pgfqpoint{5.522202in}{3.208634in}}%
\pgfpathlineto{\pgfqpoint{5.536465in}{3.215836in}}%
\pgfpathlineto{\pgfqpoint{5.543763in}{3.222001in}}%
\pgfpathlineto{\pgfqpoint{5.551059in}{3.228314in}}%
\pgfpathlineto{\pgfqpoint{5.558354in}{3.234785in}}%
\pgfpathlineto{\pgfqpoint{5.565647in}{3.241419in}}%
\pgfpathlineto{\pgfqpoint{5.551413in}{3.234873in}}%
\pgfpathlineto{\pgfqpoint{5.537194in}{3.228499in}}%
\pgfpathlineto{\pgfqpoint{5.522992in}{3.222297in}}%
\pgfpathlineto{\pgfqpoint{5.508806in}{3.216267in}}%
\pgfpathlineto{\pgfqpoint{5.501484in}{3.208967in}}%
\pgfpathlineto{\pgfqpoint{5.494162in}{3.201838in}}%
\pgfpathlineto{\pgfqpoint{5.486838in}{3.194873in}}%
\pgfpathlineto{\pgfqpoint{5.479514in}{3.188064in}}%
\pgfpathclose%
\pgfusepath{fill}%
\end{pgfscope}%
\begin{pgfscope}%
\pgfpathrectangle{\pgfqpoint{1.150000in}{0.150000in}}{\pgfqpoint{5.700000in}{5.700000in}}%
\pgfusepath{clip}%
\pgfsetbuttcap%
\pgfsetroundjoin%
\definecolor{currentfill}{rgb}{0.280894,0.078907,0.402329}%
\pgfsetfillcolor{currentfill}%
\pgfsetfillopacity{0.800000}%
\pgfsetlinewidth{0.000000pt}%
\definecolor{currentstroke}{rgb}{0.000000,0.000000,0.000000}%
\pgfsetstrokecolor{currentstroke}%
\pgfsetdash{}{0pt}%
\pgfpathmoveto{\pgfqpoint{3.276396in}{2.124203in}}%
\pgfpathlineto{\pgfqpoint{3.289901in}{2.116557in}}%
\pgfpathlineto{\pgfqpoint{3.303406in}{2.109143in}}%
\pgfpathlineto{\pgfqpoint{3.316912in}{2.101960in}}%
\pgfpathlineto{\pgfqpoint{3.330419in}{2.095006in}}%
\pgfpathlineto{\pgfqpoint{3.338543in}{2.104381in}}%
\pgfpathlineto{\pgfqpoint{3.346661in}{2.113810in}}%
\pgfpathlineto{\pgfqpoint{3.354772in}{2.123291in}}%
\pgfpathlineto{\pgfqpoint{3.362877in}{2.132825in}}%
\pgfpathlineto{\pgfqpoint{3.349385in}{2.139626in}}%
\pgfpathlineto{\pgfqpoint{3.335894in}{2.146657in}}%
\pgfpathlineto{\pgfqpoint{3.322404in}{2.153919in}}%
\pgfpathlineto{\pgfqpoint{3.308915in}{2.161412in}}%
\pgfpathlineto{\pgfqpoint{3.300795in}{2.152019in}}%
\pgfpathlineto{\pgfqpoint{3.292669in}{2.142686in}}%
\pgfpathlineto{\pgfqpoint{3.284536in}{2.133413in}}%
\pgfpathlineto{\pgfqpoint{3.276396in}{2.124203in}}%
\pgfpathclose%
\pgfusepath{fill}%
\end{pgfscope}%
\begin{pgfscope}%
\pgfpathrectangle{\pgfqpoint{1.150000in}{0.150000in}}{\pgfqpoint{5.700000in}{5.700000in}}%
\pgfusepath{clip}%
\pgfsetbuttcap%
\pgfsetroundjoin%
\definecolor{currentfill}{rgb}{0.237441,0.305202,0.541921}%
\pgfsetfillcolor{currentfill}%
\pgfsetfillopacity{0.800000}%
\pgfsetlinewidth{0.000000pt}%
\definecolor{currentstroke}{rgb}{0.000000,0.000000,0.000000}%
\pgfsetstrokecolor{currentstroke}%
\pgfsetdash{}{0pt}%
\pgfpathmoveto{\pgfqpoint{4.588791in}{2.622302in}}%
\pgfpathlineto{\pgfqpoint{4.602613in}{2.626905in}}%
\pgfpathlineto{\pgfqpoint{4.616448in}{2.631693in}}%
\pgfpathlineto{\pgfqpoint{4.630295in}{2.636665in}}%
\pgfpathlineto{\pgfqpoint{4.644155in}{2.641821in}}%
\pgfpathlineto{\pgfqpoint{4.651827in}{2.650278in}}%
\pgfpathlineto{\pgfqpoint{4.659493in}{2.658726in}}%
\pgfpathlineto{\pgfqpoint{4.667154in}{2.667169in}}%
\pgfpathlineto{\pgfqpoint{4.674809in}{2.675611in}}%
\pgfpathlineto{\pgfqpoint{4.660960in}{2.670752in}}%
\pgfpathlineto{\pgfqpoint{4.647123in}{2.666076in}}%
\pgfpathlineto{\pgfqpoint{4.633299in}{2.661584in}}%
\pgfpathlineto{\pgfqpoint{4.619487in}{2.657277in}}%
\pgfpathlineto{\pgfqpoint{4.611821in}{2.648527in}}%
\pgfpathlineto{\pgfqpoint{4.604150in}{2.639784in}}%
\pgfpathlineto{\pgfqpoint{4.596473in}{2.631043in}}%
\pgfpathlineto{\pgfqpoint{4.588791in}{2.622302in}}%
\pgfpathclose%
\pgfusepath{fill}%
\end{pgfscope}%
\begin{pgfscope}%
\pgfpathrectangle{\pgfqpoint{1.150000in}{0.150000in}}{\pgfqpoint{5.700000in}{5.700000in}}%
\pgfusepath{clip}%
\pgfsetbuttcap%
\pgfsetroundjoin%
\definecolor{currentfill}{rgb}{0.280267,0.073417,0.397163}%
\pgfsetfillcolor{currentfill}%
\pgfsetfillopacity{0.800000}%
\pgfsetlinewidth{0.000000pt}%
\definecolor{currentstroke}{rgb}{0.000000,0.000000,0.000000}%
\pgfsetstrokecolor{currentstroke}%
\pgfsetdash{}{0pt}%
\pgfpathmoveto{\pgfqpoint{3.416864in}{2.107885in}}%
\pgfpathlineto{\pgfqpoint{3.430365in}{2.102211in}}%
\pgfpathlineto{\pgfqpoint{3.443870in}{2.096759in}}%
\pgfpathlineto{\pgfqpoint{3.457377in}{2.091527in}}%
\pgfpathlineto{\pgfqpoint{3.470886in}{2.086516in}}%
\pgfpathlineto{\pgfqpoint{3.478959in}{2.096364in}}%
\pgfpathlineto{\pgfqpoint{3.487025in}{2.106246in}}%
\pgfpathlineto{\pgfqpoint{3.495085in}{2.116162in}}%
\pgfpathlineto{\pgfqpoint{3.503140in}{2.126110in}}%
\pgfpathlineto{\pgfqpoint{3.489643in}{2.131001in}}%
\pgfpathlineto{\pgfqpoint{3.476149in}{2.136113in}}%
\pgfpathlineto{\pgfqpoint{3.462657in}{2.141445in}}%
\pgfpathlineto{\pgfqpoint{3.449168in}{2.146999in}}%
\pgfpathlineto{\pgfqpoint{3.441101in}{2.137159in}}%
\pgfpathlineto{\pgfqpoint{3.433028in}{2.127359in}}%
\pgfpathlineto{\pgfqpoint{3.424949in}{2.117601in}}%
\pgfpathlineto{\pgfqpoint{3.416864in}{2.107885in}}%
\pgfpathclose%
\pgfusepath{fill}%
\end{pgfscope}%
\begin{pgfscope}%
\pgfpathrectangle{\pgfqpoint{1.150000in}{0.150000in}}{\pgfqpoint{5.700000in}{5.700000in}}%
\pgfusepath{clip}%
\pgfsetbuttcap%
\pgfsetroundjoin%
\definecolor{currentfill}{rgb}{0.143343,0.522773,0.556295}%
\pgfsetfillcolor{currentfill}%
\pgfsetfillopacity{0.800000}%
\pgfsetlinewidth{0.000000pt}%
\definecolor{currentstroke}{rgb}{0.000000,0.000000,0.000000}%
\pgfsetstrokecolor{currentstroke}%
\pgfsetdash{}{0pt}%
\pgfpathmoveto{\pgfqpoint{5.565647in}{3.241419in}}%
\pgfpathlineto{\pgfqpoint{5.579899in}{3.248137in}}%
\pgfpathlineto{\pgfqpoint{5.594167in}{3.255027in}}%
\pgfpathlineto{\pgfqpoint{5.608452in}{3.262090in}}%
\pgfpathlineto{\pgfqpoint{5.622754in}{3.269324in}}%
\pgfpathlineto{\pgfqpoint{5.630018in}{3.275453in}}%
\pgfpathlineto{\pgfqpoint{5.637282in}{3.281753in}}%
\pgfpathlineto{\pgfqpoint{5.644545in}{3.288232in}}%
\pgfpathlineto{\pgfqpoint{5.651808in}{3.294897in}}%
\pgfpathlineto{\pgfqpoint{5.637536in}{3.288352in}}%
\pgfpathlineto{\pgfqpoint{5.623281in}{3.281977in}}%
\pgfpathlineto{\pgfqpoint{5.609042in}{3.275774in}}%
\pgfpathlineto{\pgfqpoint{5.594819in}{3.269742in}}%
\pgfpathlineto{\pgfqpoint{5.587526in}{3.262378in}}%
\pgfpathlineto{\pgfqpoint{5.580234in}{3.255208in}}%
\pgfpathlineto{\pgfqpoint{5.572941in}{3.248224in}}%
\pgfpathlineto{\pgfqpoint{5.565647in}{3.241419in}}%
\pgfpathclose%
\pgfusepath{fill}%
\end{pgfscope}%
\begin{pgfscope}%
\pgfpathrectangle{\pgfqpoint{1.150000in}{0.150000in}}{\pgfqpoint{5.700000in}{5.700000in}}%
\pgfusepath{clip}%
\pgfsetbuttcap%
\pgfsetroundjoin%
\definecolor{currentfill}{rgb}{0.227802,0.326594,0.546532}%
\pgfsetfillcolor{currentfill}%
\pgfsetfillopacity{0.800000}%
\pgfsetlinewidth{0.000000pt}%
\definecolor{currentstroke}{rgb}{0.000000,0.000000,0.000000}%
\pgfsetstrokecolor{currentstroke}%
\pgfsetdash{}{0pt}%
\pgfpathmoveto{\pgfqpoint{4.674809in}{2.675611in}}%
\pgfpathlineto{\pgfqpoint{4.688671in}{2.680654in}}%
\pgfpathlineto{\pgfqpoint{4.702546in}{2.685880in}}%
\pgfpathlineto{\pgfqpoint{4.716435in}{2.691289in}}%
\pgfpathlineto{\pgfqpoint{4.730336in}{2.696881in}}%
\pgfpathlineto{\pgfqpoint{4.737975in}{2.705008in}}%
\pgfpathlineto{\pgfqpoint{4.745607in}{2.713133in}}%
\pgfpathlineto{\pgfqpoint{4.753235in}{2.721260in}}%
\pgfpathlineto{\pgfqpoint{4.760857in}{2.729393in}}%
\pgfpathlineto{\pgfqpoint{4.746967in}{2.724131in}}%
\pgfpathlineto{\pgfqpoint{4.733091in}{2.719051in}}%
\pgfpathlineto{\pgfqpoint{4.719228in}{2.714154in}}%
\pgfpathlineto{\pgfqpoint{4.705377in}{2.709439in}}%
\pgfpathlineto{\pgfqpoint{4.697743in}{2.700965in}}%
\pgfpathlineto{\pgfqpoint{4.690104in}{2.692505in}}%
\pgfpathlineto{\pgfqpoint{4.682459in}{2.684055in}}%
\pgfpathlineto{\pgfqpoint{4.674809in}{2.675611in}}%
\pgfpathclose%
\pgfusepath{fill}%
\end{pgfscope}%
\begin{pgfscope}%
\pgfpathrectangle{\pgfqpoint{1.150000in}{0.150000in}}{\pgfqpoint{5.700000in}{5.700000in}}%
\pgfusepath{clip}%
\pgfsetbuttcap%
\pgfsetroundjoin%
\definecolor{currentfill}{rgb}{0.281924,0.089666,0.412415}%
\pgfsetfillcolor{currentfill}%
\pgfsetfillopacity{0.800000}%
\pgfsetlinewidth{0.000000pt}%
\definecolor{currentstroke}{rgb}{0.000000,0.000000,0.000000}%
\pgfsetstrokecolor{currentstroke}%
\pgfsetdash{}{0pt}%
\pgfpathmoveto{\pgfqpoint{3.643323in}{2.135575in}}%
\pgfpathlineto{\pgfqpoint{3.656846in}{2.132703in}}%
\pgfpathlineto{\pgfqpoint{3.670373in}{2.130041in}}%
\pgfpathlineto{\pgfqpoint{3.683906in}{2.127588in}}%
\pgfpathlineto{\pgfqpoint{3.697444in}{2.125343in}}%
\pgfpathlineto{\pgfqpoint{3.705439in}{2.135652in}}%
\pgfpathlineto{\pgfqpoint{3.713429in}{2.145967in}}%
\pgfpathlineto{\pgfqpoint{3.721413in}{2.156288in}}%
\pgfpathlineto{\pgfqpoint{3.729392in}{2.166616in}}%
\pgfpathlineto{\pgfqpoint{3.715864in}{2.168804in}}%
\pgfpathlineto{\pgfqpoint{3.702341in}{2.171201in}}%
\pgfpathlineto{\pgfqpoint{3.688823in}{2.173806in}}%
\pgfpathlineto{\pgfqpoint{3.675311in}{2.176621in}}%
\pgfpathlineto{\pgfqpoint{3.667322in}{2.166339in}}%
\pgfpathlineto{\pgfqpoint{3.659328in}{2.156070in}}%
\pgfpathlineto{\pgfqpoint{3.651328in}{2.145816in}}%
\pgfpathlineto{\pgfqpoint{3.643323in}{2.135575in}}%
\pgfpathclose%
\pgfusepath{fill}%
\end{pgfscope}%
\begin{pgfscope}%
\pgfpathrectangle{\pgfqpoint{1.150000in}{0.150000in}}{\pgfqpoint{5.700000in}{5.700000in}}%
\pgfusepath{clip}%
\pgfsetbuttcap%
\pgfsetroundjoin%
\definecolor{currentfill}{rgb}{0.270595,0.214069,0.507052}%
\pgfsetfillcolor{currentfill}%
\pgfsetfillopacity{0.800000}%
\pgfsetlinewidth{0.000000pt}%
\definecolor{currentstroke}{rgb}{0.000000,0.000000,0.000000}%
\pgfsetstrokecolor{currentstroke}%
\pgfsetdash{}{0pt}%
\pgfpathmoveto{\pgfqpoint{2.776626in}{2.442498in}}%
\pgfpathlineto{\pgfqpoint{2.790275in}{2.426136in}}%
\pgfpathlineto{\pgfqpoint{2.803917in}{2.410066in}}%
\pgfpathlineto{\pgfqpoint{2.817552in}{2.394286in}}%
\pgfpathlineto{\pgfqpoint{2.831180in}{2.378792in}}%
\pgfpathlineto{\pgfqpoint{2.839527in}{2.385840in}}%
\pgfpathlineto{\pgfqpoint{2.847864in}{2.393019in}}%
\pgfpathlineto{\pgfqpoint{2.856192in}{2.400326in}}%
\pgfpathlineto{\pgfqpoint{2.864510in}{2.407759in}}%
\pgfpathlineto{\pgfqpoint{2.850907in}{2.423029in}}%
\pgfpathlineto{\pgfqpoint{2.837297in}{2.438586in}}%
\pgfpathlineto{\pgfqpoint{2.823681in}{2.454432in}}%
\pgfpathlineto{\pgfqpoint{2.810058in}{2.470569in}}%
\pgfpathlineto{\pgfqpoint{2.801715in}{2.463348in}}%
\pgfpathlineto{\pgfqpoint{2.793361in}{2.456261in}}%
\pgfpathlineto{\pgfqpoint{2.784998in}{2.449310in}}%
\pgfpathlineto{\pgfqpoint{2.776626in}{2.442498in}}%
\pgfpathclose%
\pgfusepath{fill}%
\end{pgfscope}%
\begin{pgfscope}%
\pgfpathrectangle{\pgfqpoint{1.150000in}{0.150000in}}{\pgfqpoint{5.700000in}{5.700000in}}%
\pgfusepath{clip}%
\pgfsetbuttcap%
\pgfsetroundjoin%
\definecolor{currentfill}{rgb}{0.282327,0.094955,0.417331}%
\pgfsetfillcolor{currentfill}%
\pgfsetfillopacity{0.800000}%
\pgfsetlinewidth{0.000000pt}%
\definecolor{currentstroke}{rgb}{0.000000,0.000000,0.000000}%
\pgfsetstrokecolor{currentstroke}%
\pgfsetdash{}{0pt}%
\pgfpathmoveto{\pgfqpoint{3.135598in}{2.159165in}}%
\pgfpathlineto{\pgfqpoint{3.149122in}{2.149428in}}%
\pgfpathlineto{\pgfqpoint{3.162645in}{2.139934in}}%
\pgfpathlineto{\pgfqpoint{3.176167in}{2.130682in}}%
\pgfpathlineto{\pgfqpoint{3.189689in}{2.121671in}}%
\pgfpathlineto{\pgfqpoint{3.197873in}{2.130425in}}%
\pgfpathlineto{\pgfqpoint{3.206049in}{2.139254in}}%
\pgfpathlineto{\pgfqpoint{3.214219in}{2.148157in}}%
\pgfpathlineto{\pgfqpoint{3.222382in}{2.157133in}}%
\pgfpathlineto{\pgfqpoint{3.208878in}{2.165959in}}%
\pgfpathlineto{\pgfqpoint{3.195374in}{2.175025in}}%
\pgfpathlineto{\pgfqpoint{3.181869in}{2.184334in}}%
\pgfpathlineto{\pgfqpoint{3.168363in}{2.193886in}}%
\pgfpathlineto{\pgfqpoint{3.160182in}{2.185084in}}%
\pgfpathlineto{\pgfqpoint{3.151995in}{2.176363in}}%
\pgfpathlineto{\pgfqpoint{3.143800in}{2.167722in}}%
\pgfpathlineto{\pgfqpoint{3.135598in}{2.159165in}}%
\pgfpathclose%
\pgfusepath{fill}%
\end{pgfscope}%
\begin{pgfscope}%
\pgfpathrectangle{\pgfqpoint{1.150000in}{0.150000in}}{\pgfqpoint{5.700000in}{5.700000in}}%
\pgfusepath{clip}%
\pgfsetbuttcap%
\pgfsetroundjoin%
\definecolor{currentfill}{rgb}{0.277134,0.185228,0.489898}%
\pgfsetfillcolor{currentfill}%
\pgfsetfillopacity{0.800000}%
\pgfsetlinewidth{0.000000pt}%
\definecolor{currentstroke}{rgb}{0.000000,0.000000,0.000000}%
\pgfsetstrokecolor{currentstroke}%
\pgfsetdash{}{0pt}%
\pgfpathmoveto{\pgfqpoint{2.831180in}{2.378792in}}%
\pgfpathlineto{\pgfqpoint{2.844802in}{2.363582in}}%
\pgfpathlineto{\pgfqpoint{2.858417in}{2.348654in}}%
\pgfpathlineto{\pgfqpoint{2.872027in}{2.334007in}}%
\pgfpathlineto{\pgfqpoint{2.885630in}{2.319636in}}%
\pgfpathlineto{\pgfqpoint{2.893952in}{2.326920in}}%
\pgfpathlineto{\pgfqpoint{2.902264in}{2.334325in}}%
\pgfpathlineto{\pgfqpoint{2.910568in}{2.341850in}}%
\pgfpathlineto{\pgfqpoint{2.918862in}{2.349493in}}%
\pgfpathlineto{\pgfqpoint{2.905283in}{2.363642in}}%
\pgfpathlineto{\pgfqpoint{2.891697in}{2.378067in}}%
\pgfpathlineto{\pgfqpoint{2.878107in}{2.392772in}}%
\pgfpathlineto{\pgfqpoint{2.864510in}{2.407759in}}%
\pgfpathlineto{\pgfqpoint{2.856192in}{2.400326in}}%
\pgfpathlineto{\pgfqpoint{2.847864in}{2.393019in}}%
\pgfpathlineto{\pgfqpoint{2.839527in}{2.385840in}}%
\pgfpathlineto{\pgfqpoint{2.831180in}{2.378792in}}%
\pgfpathclose%
\pgfusepath{fill}%
\end{pgfscope}%
\begin{pgfscope}%
\pgfpathrectangle{\pgfqpoint{1.150000in}{0.150000in}}{\pgfqpoint{5.700000in}{5.700000in}}%
\pgfusepath{clip}%
\pgfsetbuttcap%
\pgfsetroundjoin%
\definecolor{currentfill}{rgb}{0.136408,0.541173,0.554483}%
\pgfsetfillcolor{currentfill}%
\pgfsetfillopacity{0.800000}%
\pgfsetlinewidth{0.000000pt}%
\definecolor{currentstroke}{rgb}{0.000000,0.000000,0.000000}%
\pgfsetstrokecolor{currentstroke}%
\pgfsetdash{}{0pt}%
\pgfpathmoveto{\pgfqpoint{5.651808in}{3.294897in}}%
\pgfpathlineto{\pgfqpoint{5.666097in}{3.301614in}}%
\pgfpathlineto{\pgfqpoint{5.680403in}{3.308502in}}%
\pgfpathlineto{\pgfqpoint{5.694725in}{3.315561in}}%
\pgfpathlineto{\pgfqpoint{5.709065in}{3.322791in}}%
\pgfpathlineto{\pgfqpoint{5.716298in}{3.328942in}}%
\pgfpathlineto{\pgfqpoint{5.723530in}{3.335287in}}%
\pgfpathlineto{\pgfqpoint{5.730764in}{3.341835in}}%
\pgfpathlineto{\pgfqpoint{5.737999in}{3.348593in}}%
\pgfpathlineto{\pgfqpoint{5.723691in}{3.342084in}}%
\pgfpathlineto{\pgfqpoint{5.709399in}{3.335746in}}%
\pgfpathlineto{\pgfqpoint{5.695125in}{3.329577in}}%
\pgfpathlineto{\pgfqpoint{5.680867in}{3.323580in}}%
\pgfpathlineto{\pgfqpoint{5.673601in}{3.316090in}}%
\pgfpathlineto{\pgfqpoint{5.666336in}{3.308818in}}%
\pgfpathlineto{\pgfqpoint{5.659071in}{3.301757in}}%
\pgfpathlineto{\pgfqpoint{5.651808in}{3.294897in}}%
\pgfpathclose%
\pgfusepath{fill}%
\end{pgfscope}%
\begin{pgfscope}%
\pgfpathrectangle{\pgfqpoint{1.150000in}{0.150000in}}{\pgfqpoint{5.700000in}{5.700000in}}%
\pgfusepath{clip}%
\pgfsetbuttcap%
\pgfsetroundjoin%
\definecolor{currentfill}{rgb}{0.262138,0.242286,0.520837}%
\pgfsetfillcolor{currentfill}%
\pgfsetfillopacity{0.800000}%
\pgfsetlinewidth{0.000000pt}%
\definecolor{currentstroke}{rgb}{0.000000,0.000000,0.000000}%
\pgfsetstrokecolor{currentstroke}%
\pgfsetdash{}{0pt}%
\pgfpathmoveto{\pgfqpoint{2.721949in}{2.510909in}}%
\pgfpathlineto{\pgfqpoint{2.735630in}{2.493356in}}%
\pgfpathlineto{\pgfqpoint{2.749304in}{2.476105in}}%
\pgfpathlineto{\pgfqpoint{2.762969in}{2.459153in}}%
\pgfpathlineto{\pgfqpoint{2.776626in}{2.442498in}}%
\pgfpathlineto{\pgfqpoint{2.784998in}{2.449310in}}%
\pgfpathlineto{\pgfqpoint{2.793361in}{2.456261in}}%
\pgfpathlineto{\pgfqpoint{2.801715in}{2.463348in}}%
\pgfpathlineto{\pgfqpoint{2.810058in}{2.470569in}}%
\pgfpathlineto{\pgfqpoint{2.796427in}{2.486999in}}%
\pgfpathlineto{\pgfqpoint{2.782789in}{2.503726in}}%
\pgfpathlineto{\pgfqpoint{2.769143in}{2.520751in}}%
\pgfpathlineto{\pgfqpoint{2.755488in}{2.538078in}}%
\pgfpathlineto{\pgfqpoint{2.747119in}{2.531070in}}%
\pgfpathlineto{\pgfqpoint{2.738739in}{2.524205in}}%
\pgfpathlineto{\pgfqpoint{2.730349in}{2.517484in}}%
\pgfpathlineto{\pgfqpoint{2.721949in}{2.510909in}}%
\pgfpathclose%
\pgfusepath{fill}%
\end{pgfscope}%
\begin{pgfscope}%
\pgfpathrectangle{\pgfqpoint{1.150000in}{0.150000in}}{\pgfqpoint{5.700000in}{5.700000in}}%
\pgfusepath{clip}%
\pgfsetbuttcap%
\pgfsetroundjoin%
\definecolor{currentfill}{rgb}{0.218130,0.347432,0.550038}%
\pgfsetfillcolor{currentfill}%
\pgfsetfillopacity{0.800000}%
\pgfsetlinewidth{0.000000pt}%
\definecolor{currentstroke}{rgb}{0.000000,0.000000,0.000000}%
\pgfsetstrokecolor{currentstroke}%
\pgfsetdash{}{0pt}%
\pgfpathmoveto{\pgfqpoint{4.760857in}{2.729393in}}%
\pgfpathlineto{\pgfqpoint{4.774760in}{2.734837in}}%
\pgfpathlineto{\pgfqpoint{4.788676in}{2.740463in}}%
\pgfpathlineto{\pgfqpoint{4.802606in}{2.746271in}}%
\pgfpathlineto{\pgfqpoint{4.816550in}{2.752261in}}%
\pgfpathlineto{\pgfqpoint{4.824154in}{2.760054in}}%
\pgfpathlineto{\pgfqpoint{4.831753in}{2.767853in}}%
\pgfpathlineto{\pgfqpoint{4.839346in}{2.775662in}}%
\pgfpathlineto{\pgfqpoint{4.846934in}{2.783487in}}%
\pgfpathlineto{\pgfqpoint{4.833004in}{2.777860in}}%
\pgfpathlineto{\pgfqpoint{4.819087in}{2.772414in}}%
\pgfpathlineto{\pgfqpoint{4.805183in}{2.767150in}}%
\pgfpathlineto{\pgfqpoint{4.791293in}{2.762067in}}%
\pgfpathlineto{\pgfqpoint{4.783692in}{2.753869in}}%
\pgfpathlineto{\pgfqpoint{4.776085in}{2.745693in}}%
\pgfpathlineto{\pgfqpoint{4.768474in}{2.737536in}}%
\pgfpathlineto{\pgfqpoint{4.760857in}{2.729393in}}%
\pgfpathclose%
\pgfusepath{fill}%
\end{pgfscope}%
\begin{pgfscope}%
\pgfpathrectangle{\pgfqpoint{1.150000in}{0.150000in}}{\pgfqpoint{5.700000in}{5.700000in}}%
\pgfusepath{clip}%
\pgfsetbuttcap%
\pgfsetroundjoin%
\definecolor{currentfill}{rgb}{0.280868,0.160771,0.472899}%
\pgfsetfillcolor{currentfill}%
\pgfsetfillopacity{0.800000}%
\pgfsetlinewidth{0.000000pt}%
\definecolor{currentstroke}{rgb}{0.000000,0.000000,0.000000}%
\pgfsetstrokecolor{currentstroke}%
\pgfsetdash{}{0pt}%
\pgfpathmoveto{\pgfqpoint{2.885630in}{2.319636in}}%
\pgfpathlineto{\pgfqpoint{2.899228in}{2.305542in}}%
\pgfpathlineto{\pgfqpoint{2.912821in}{2.291720in}}%
\pgfpathlineto{\pgfqpoint{2.926409in}{2.278170in}}%
\pgfpathlineto{\pgfqpoint{2.939993in}{2.264889in}}%
\pgfpathlineto{\pgfqpoint{2.948290in}{2.272405in}}%
\pgfpathlineto{\pgfqpoint{2.956579in}{2.280035in}}%
\pgfpathlineto{\pgfqpoint{2.964860in}{2.287777in}}%
\pgfpathlineto{\pgfqpoint{2.973131in}{2.295629in}}%
\pgfpathlineto{\pgfqpoint{2.959571in}{2.308690in}}%
\pgfpathlineto{\pgfqpoint{2.946006in}{2.322020in}}%
\pgfpathlineto{\pgfqpoint{2.932437in}{2.335620in}}%
\pgfpathlineto{\pgfqpoint{2.918862in}{2.349493in}}%
\pgfpathlineto{\pgfqpoint{2.910568in}{2.341850in}}%
\pgfpathlineto{\pgfqpoint{2.902264in}{2.334325in}}%
\pgfpathlineto{\pgfqpoint{2.893952in}{2.326920in}}%
\pgfpathlineto{\pgfqpoint{2.885630in}{2.319636in}}%
\pgfpathclose%
\pgfusepath{fill}%
\end{pgfscope}%
\begin{pgfscope}%
\pgfpathrectangle{\pgfqpoint{1.150000in}{0.150000in}}{\pgfqpoint{5.700000in}{5.700000in}}%
\pgfusepath{clip}%
\pgfsetbuttcap%
\pgfsetroundjoin%
\definecolor{currentfill}{rgb}{0.129933,0.559582,0.551864}%
\pgfsetfillcolor{currentfill}%
\pgfsetfillopacity{0.800000}%
\pgfsetlinewidth{0.000000pt}%
\definecolor{currentstroke}{rgb}{0.000000,0.000000,0.000000}%
\pgfsetstrokecolor{currentstroke}%
\pgfsetdash{}{0pt}%
\pgfpathmoveto{\pgfqpoint{5.737999in}{3.348593in}}%
\pgfpathlineto{\pgfqpoint{5.752324in}{3.355272in}}%
\pgfpathlineto{\pgfqpoint{5.766666in}{3.362122in}}%
\pgfpathlineto{\pgfqpoint{5.781025in}{3.369142in}}%
\pgfpathlineto{\pgfqpoint{5.795402in}{3.376333in}}%
\pgfpathlineto{\pgfqpoint{5.802605in}{3.382568in}}%
\pgfpathlineto{\pgfqpoint{5.809809in}{3.389024in}}%
\pgfpathlineto{\pgfqpoint{5.817016in}{3.395707in}}%
\pgfpathlineto{\pgfqpoint{5.824224in}{3.402627in}}%
\pgfpathlineto{\pgfqpoint{5.809881in}{3.396190in}}%
\pgfpathlineto{\pgfqpoint{5.795556in}{3.389923in}}%
\pgfpathlineto{\pgfqpoint{5.781247in}{3.383825in}}%
\pgfpathlineto{\pgfqpoint{5.766955in}{3.377897in}}%
\pgfpathlineto{\pgfqpoint{5.759712in}{3.370214in}}%
\pgfpathlineto{\pgfqpoint{5.752472in}{3.362775in}}%
\pgfpathlineto{\pgfqpoint{5.745235in}{3.355571in}}%
\pgfpathlineto{\pgfqpoint{5.737999in}{3.348593in}}%
\pgfpathclose%
\pgfusepath{fill}%
\end{pgfscope}%
\begin{pgfscope}%
\pgfpathrectangle{\pgfqpoint{1.150000in}{0.150000in}}{\pgfqpoint{5.700000in}{5.700000in}}%
\pgfusepath{clip}%
\pgfsetbuttcap%
\pgfsetroundjoin%
\definecolor{currentfill}{rgb}{0.252194,0.269783,0.531579}%
\pgfsetfillcolor{currentfill}%
\pgfsetfillopacity{0.800000}%
\pgfsetlinewidth{0.000000pt}%
\definecolor{currentstroke}{rgb}{0.000000,0.000000,0.000000}%
\pgfsetstrokecolor{currentstroke}%
\pgfsetdash{}{0pt}%
\pgfpathmoveto{\pgfqpoint{2.667131in}{2.584193in}}%
\pgfpathlineto{\pgfqpoint{2.680850in}{2.565406in}}%
\pgfpathlineto{\pgfqpoint{2.694559in}{2.546931in}}%
\pgfpathlineto{\pgfqpoint{2.708258in}{2.528767in}}%
\pgfpathlineto{\pgfqpoint{2.721949in}{2.510909in}}%
\pgfpathlineto{\pgfqpoint{2.730349in}{2.517484in}}%
\pgfpathlineto{\pgfqpoint{2.738739in}{2.524205in}}%
\pgfpathlineto{\pgfqpoint{2.747119in}{2.531070in}}%
\pgfpathlineto{\pgfqpoint{2.755488in}{2.538078in}}%
\pgfpathlineto{\pgfqpoint{2.741826in}{2.555709in}}%
\pgfpathlineto{\pgfqpoint{2.728154in}{2.573646in}}%
\pgfpathlineto{\pgfqpoint{2.714473in}{2.591893in}}%
\pgfpathlineto{\pgfqpoint{2.700783in}{2.610452in}}%
\pgfpathlineto{\pgfqpoint{2.692386in}{2.603659in}}%
\pgfpathlineto{\pgfqpoint{2.683979in}{2.597017in}}%
\pgfpathlineto{\pgfqpoint{2.675560in}{2.590528in}}%
\pgfpathlineto{\pgfqpoint{2.667131in}{2.584193in}}%
\pgfpathclose%
\pgfusepath{fill}%
\end{pgfscope}%
\begin{pgfscope}%
\pgfpathrectangle{\pgfqpoint{1.150000in}{0.150000in}}{\pgfqpoint{5.700000in}{5.700000in}}%
\pgfusepath{clip}%
\pgfsetbuttcap%
\pgfsetroundjoin%
\definecolor{currentfill}{rgb}{0.208623,0.367752,0.552675}%
\pgfsetfillcolor{currentfill}%
\pgfsetfillopacity{0.800000}%
\pgfsetlinewidth{0.000000pt}%
\definecolor{currentstroke}{rgb}{0.000000,0.000000,0.000000}%
\pgfsetstrokecolor{currentstroke}%
\pgfsetdash{}{0pt}%
\pgfpathmoveto{\pgfqpoint{4.846934in}{2.783487in}}%
\pgfpathlineto{\pgfqpoint{4.860879in}{2.789295in}}%
\pgfpathlineto{\pgfqpoint{4.874837in}{2.795283in}}%
\pgfpathlineto{\pgfqpoint{4.888810in}{2.801452in}}%
\pgfpathlineto{\pgfqpoint{4.902797in}{2.807802in}}%
\pgfpathlineto{\pgfqpoint{4.910365in}{2.815262in}}%
\pgfpathlineto{\pgfqpoint{4.917929in}{2.822738in}}%
\pgfpathlineto{\pgfqpoint{4.925487in}{2.830235in}}%
\pgfpathlineto{\pgfqpoint{4.933040in}{2.837756in}}%
\pgfpathlineto{\pgfqpoint{4.919068in}{2.831803in}}%
\pgfpathlineto{\pgfqpoint{4.905110in}{2.826029in}}%
\pgfpathlineto{\pgfqpoint{4.891166in}{2.820435in}}%
\pgfpathlineto{\pgfqpoint{4.877236in}{2.815021in}}%
\pgfpathlineto{\pgfqpoint{4.869668in}{2.807093in}}%
\pgfpathlineto{\pgfqpoint{4.862095in}{2.799198in}}%
\pgfpathlineto{\pgfqpoint{4.854517in}{2.791331in}}%
\pgfpathlineto{\pgfqpoint{4.846934in}{2.783487in}}%
\pgfpathclose%
\pgfusepath{fill}%
\end{pgfscope}%
\begin{pgfscope}%
\pgfpathrectangle{\pgfqpoint{1.150000in}{0.150000in}}{\pgfqpoint{5.700000in}{5.700000in}}%
\pgfusepath{clip}%
\pgfsetbuttcap%
\pgfsetroundjoin%
\definecolor{currentfill}{rgb}{0.280894,0.078907,0.402329}%
\pgfsetfillcolor{currentfill}%
\pgfsetfillopacity{0.800000}%
\pgfsetlinewidth{0.000000pt}%
\definecolor{currentstroke}{rgb}{0.000000,0.000000,0.000000}%
\pgfsetstrokecolor{currentstroke}%
\pgfsetdash{}{0pt}%
\pgfpathmoveto{\pgfqpoint{3.557164in}{2.108720in}}%
\pgfpathlineto{\pgfqpoint{3.570679in}{2.104911in}}%
\pgfpathlineto{\pgfqpoint{3.584198in}{2.101316in}}%
\pgfpathlineto{\pgfqpoint{3.597722in}{2.097933in}}%
\pgfpathlineto{\pgfqpoint{3.611250in}{2.094763in}}%
\pgfpathlineto{\pgfqpoint{3.619277in}{2.104943in}}%
\pgfpathlineto{\pgfqpoint{3.627298in}{2.115138in}}%
\pgfpathlineto{\pgfqpoint{3.635313in}{2.125349in}}%
\pgfpathlineto{\pgfqpoint{3.643323in}{2.135575in}}%
\pgfpathlineto{\pgfqpoint{3.629806in}{2.138658in}}%
\pgfpathlineto{\pgfqpoint{3.616293in}{2.141952in}}%
\pgfpathlineto{\pgfqpoint{3.602784in}{2.145459in}}%
\pgfpathlineto{\pgfqpoint{3.589280in}{2.149180in}}%
\pgfpathlineto{\pgfqpoint{3.581259in}{2.139030in}}%
\pgfpathlineto{\pgfqpoint{3.573233in}{2.128904in}}%
\pgfpathlineto{\pgfqpoint{3.565201in}{2.118800in}}%
\pgfpathlineto{\pgfqpoint{3.557164in}{2.108720in}}%
\pgfpathclose%
\pgfusepath{fill}%
\end{pgfscope}%
\begin{pgfscope}%
\pgfpathrectangle{\pgfqpoint{1.150000in}{0.150000in}}{\pgfqpoint{5.700000in}{5.700000in}}%
\pgfusepath{clip}%
\pgfsetbuttcap%
\pgfsetroundjoin%
\definecolor{currentfill}{rgb}{0.282623,0.140926,0.457517}%
\pgfsetfillcolor{currentfill}%
\pgfsetfillopacity{0.800000}%
\pgfsetlinewidth{0.000000pt}%
\definecolor{currentstroke}{rgb}{0.000000,0.000000,0.000000}%
\pgfsetstrokecolor{currentstroke}%
\pgfsetdash{}{0pt}%
\pgfpathmoveto{\pgfqpoint{2.939993in}{2.264889in}}%
\pgfpathlineto{\pgfqpoint{2.953571in}{2.251874in}}%
\pgfpathlineto{\pgfqpoint{2.967146in}{2.239125in}}%
\pgfpathlineto{\pgfqpoint{2.980717in}{2.226639in}}%
\pgfpathlineto{\pgfqpoint{2.994283in}{2.214414in}}%
\pgfpathlineto{\pgfqpoint{3.002558in}{2.222162in}}%
\pgfpathlineto{\pgfqpoint{3.010825in}{2.230016in}}%
\pgfpathlineto{\pgfqpoint{3.019083in}{2.237974in}}%
\pgfpathlineto{\pgfqpoint{3.027333in}{2.246034in}}%
\pgfpathlineto{\pgfqpoint{3.013788in}{2.258040in}}%
\pgfpathlineto{\pgfqpoint{3.000240in}{2.270306in}}%
\pgfpathlineto{\pgfqpoint{2.986688in}{2.282835in}}%
\pgfpathlineto{\pgfqpoint{2.973131in}{2.295629in}}%
\pgfpathlineto{\pgfqpoint{2.964860in}{2.287777in}}%
\pgfpathlineto{\pgfqpoint{2.956579in}{2.280035in}}%
\pgfpathlineto{\pgfqpoint{2.948290in}{2.272405in}}%
\pgfpathlineto{\pgfqpoint{2.939993in}{2.264889in}}%
\pgfpathclose%
\pgfusepath{fill}%
\end{pgfscope}%
\begin{pgfscope}%
\pgfpathrectangle{\pgfqpoint{1.150000in}{0.150000in}}{\pgfqpoint{5.700000in}{5.700000in}}%
\pgfusepath{clip}%
\pgfsetbuttcap%
\pgfsetroundjoin%
\definecolor{currentfill}{rgb}{0.239346,0.300855,0.540844}%
\pgfsetfillcolor{currentfill}%
\pgfsetfillopacity{0.800000}%
\pgfsetlinewidth{0.000000pt}%
\definecolor{currentstroke}{rgb}{0.000000,0.000000,0.000000}%
\pgfsetstrokecolor{currentstroke}%
\pgfsetdash{}{0pt}%
\pgfpathmoveto{\pgfqpoint{2.612153in}{2.662529in}}%
\pgfpathlineto{\pgfqpoint{2.625914in}{2.642461in}}%
\pgfpathlineto{\pgfqpoint{2.639664in}{2.622718in}}%
\pgfpathlineto{\pgfqpoint{2.653403in}{2.603296in}}%
\pgfpathlineto{\pgfqpoint{2.667131in}{2.584193in}}%
\pgfpathlineto{\pgfqpoint{2.675560in}{2.590528in}}%
\pgfpathlineto{\pgfqpoint{2.683979in}{2.597017in}}%
\pgfpathlineto{\pgfqpoint{2.692386in}{2.603659in}}%
\pgfpathlineto{\pgfqpoint{2.700783in}{2.610452in}}%
\pgfpathlineto{\pgfqpoint{2.687083in}{2.629326in}}%
\pgfpathlineto{\pgfqpoint{2.673373in}{2.648518in}}%
\pgfpathlineto{\pgfqpoint{2.659653in}{2.668032in}}%
\pgfpathlineto{\pgfqpoint{2.645923in}{2.687869in}}%
\pgfpathlineto{\pgfqpoint{2.637497in}{2.681293in}}%
\pgfpathlineto{\pgfqpoint{2.629060in}{2.674877in}}%
\pgfpathlineto{\pgfqpoint{2.620612in}{2.668621in}}%
\pgfpathlineto{\pgfqpoint{2.612153in}{2.662529in}}%
\pgfpathclose%
\pgfusepath{fill}%
\end{pgfscope}%
\begin{pgfscope}%
\pgfpathrectangle{\pgfqpoint{1.150000in}{0.150000in}}{\pgfqpoint{5.700000in}{5.700000in}}%
\pgfusepath{clip}%
\pgfsetbuttcap%
\pgfsetroundjoin%
\definecolor{currentfill}{rgb}{0.199430,0.387607,0.554642}%
\pgfsetfillcolor{currentfill}%
\pgfsetfillopacity{0.800000}%
\pgfsetlinewidth{0.000000pt}%
\definecolor{currentstroke}{rgb}{0.000000,0.000000,0.000000}%
\pgfsetstrokecolor{currentstroke}%
\pgfsetdash{}{0pt}%
\pgfpathmoveto{\pgfqpoint{4.933040in}{2.837756in}}%
\pgfpathlineto{\pgfqpoint{4.947027in}{2.843890in}}%
\pgfpathlineto{\pgfqpoint{4.961028in}{2.850203in}}%
\pgfpathlineto{\pgfqpoint{4.975043in}{2.856696in}}%
\pgfpathlineto{\pgfqpoint{4.989074in}{2.863368in}}%
\pgfpathlineto{\pgfqpoint{4.996606in}{2.870503in}}%
\pgfpathlineto{\pgfqpoint{5.004134in}{2.877664in}}%
\pgfpathlineto{\pgfqpoint{5.011656in}{2.884857in}}%
\pgfpathlineto{\pgfqpoint{5.019174in}{2.892087in}}%
\pgfpathlineto{\pgfqpoint{5.005159in}{2.885844in}}%
\pgfpathlineto{\pgfqpoint{4.991160in}{2.879779in}}%
\pgfpathlineto{\pgfqpoint{4.977175in}{2.873893in}}%
\pgfpathlineto{\pgfqpoint{4.963204in}{2.868187in}}%
\pgfpathlineto{\pgfqpoint{4.955670in}{2.860518in}}%
\pgfpathlineto{\pgfqpoint{4.948132in}{2.852893in}}%
\pgfpathlineto{\pgfqpoint{4.940588in}{2.845308in}}%
\pgfpathlineto{\pgfqpoint{4.933040in}{2.837756in}}%
\pgfpathclose%
\pgfusepath{fill}%
\end{pgfscope}%
\begin{pgfscope}%
\pgfpathrectangle{\pgfqpoint{1.150000in}{0.150000in}}{\pgfqpoint{5.700000in}{5.700000in}}%
\pgfusepath{clip}%
\pgfsetbuttcap%
\pgfsetroundjoin%
\definecolor{currentfill}{rgb}{0.279566,0.067836,0.391917}%
\pgfsetfillcolor{currentfill}%
\pgfsetfillopacity{0.800000}%
\pgfsetlinewidth{0.000000pt}%
\definecolor{currentstroke}{rgb}{0.000000,0.000000,0.000000}%
\pgfsetstrokecolor{currentstroke}%
\pgfsetdash{}{0pt}%
\pgfpathmoveto{\pgfqpoint{3.330419in}{2.095006in}}%
\pgfpathlineto{\pgfqpoint{3.343927in}{2.088280in}}%
\pgfpathlineto{\pgfqpoint{3.357437in}{2.081781in}}%
\pgfpathlineto{\pgfqpoint{3.370949in}{2.075508in}}%
\pgfpathlineto{\pgfqpoint{3.384462in}{2.069459in}}%
\pgfpathlineto{\pgfqpoint{3.392572in}{2.078998in}}%
\pgfpathlineto{\pgfqpoint{3.400675in}{2.088582in}}%
\pgfpathlineto{\pgfqpoint{3.408772in}{2.098211in}}%
\pgfpathlineto{\pgfqpoint{3.416864in}{2.107885in}}%
\pgfpathlineto{\pgfqpoint{3.403364in}{2.113782in}}%
\pgfpathlineto{\pgfqpoint{3.389867in}{2.119904in}}%
\pgfpathlineto{\pgfqpoint{3.376371in}{2.126251in}}%
\pgfpathlineto{\pgfqpoint{3.362877in}{2.132825in}}%
\pgfpathlineto{\pgfqpoint{3.354772in}{2.123291in}}%
\pgfpathlineto{\pgfqpoint{3.346661in}{2.113810in}}%
\pgfpathlineto{\pgfqpoint{3.338543in}{2.104381in}}%
\pgfpathlineto{\pgfqpoint{3.330419in}{2.095006in}}%
\pgfpathclose%
\pgfusepath{fill}%
\end{pgfscope}%
\begin{pgfscope}%
\pgfpathrectangle{\pgfqpoint{1.150000in}{0.150000in}}{\pgfqpoint{5.700000in}{5.700000in}}%
\pgfusepath{clip}%
\pgfsetbuttcap%
\pgfsetroundjoin%
\definecolor{currentfill}{rgb}{0.124395,0.578002,0.548287}%
\pgfsetfillcolor{currentfill}%
\pgfsetfillopacity{0.800000}%
\pgfsetlinewidth{0.000000pt}%
\definecolor{currentstroke}{rgb}{0.000000,0.000000,0.000000}%
\pgfsetstrokecolor{currentstroke}%
\pgfsetdash{}{0pt}%
\pgfpathmoveto{\pgfqpoint{5.824224in}{3.402627in}}%
\pgfpathlineto{\pgfqpoint{5.838584in}{3.409233in}}%
\pgfpathlineto{\pgfqpoint{5.852962in}{3.416008in}}%
\pgfpathlineto{\pgfqpoint{5.867357in}{3.422954in}}%
\pgfpathlineto{\pgfqpoint{5.881769in}{3.430068in}}%
\pgfpathlineto{\pgfqpoint{5.888945in}{3.436459in}}%
\pgfpathlineto{\pgfqpoint{5.896124in}{3.443095in}}%
\pgfpathlineto{\pgfqpoint{5.903306in}{3.449986in}}%
\pgfpathlineto{\pgfqpoint{5.888920in}{3.443458in}}%
\pgfpathlineto{\pgfqpoint{5.874552in}{3.437099in}}%
\pgfpathlineto{\pgfqpoint{5.860201in}{3.430909in}}%
\pgfpathlineto{\pgfqpoint{5.845868in}{3.424888in}}%
\pgfpathlineto{\pgfqpoint{5.838650in}{3.417209in}}%
\pgfpathlineto{\pgfqpoint{5.831436in}{3.409791in}}%
\pgfpathlineto{\pgfqpoint{5.824224in}{3.402627in}}%
\pgfpathclose%
\pgfusepath{fill}%
\end{pgfscope}%
\begin{pgfscope}%
\pgfpathrectangle{\pgfqpoint{1.150000in}{0.150000in}}{\pgfqpoint{5.700000in}{5.700000in}}%
\pgfusepath{clip}%
\pgfsetbuttcap%
\pgfsetroundjoin%
\definecolor{currentfill}{rgb}{0.280894,0.078907,0.402329}%
\pgfsetfillcolor{currentfill}%
\pgfsetfillopacity{0.800000}%
\pgfsetlinewidth{0.000000pt}%
\definecolor{currentstroke}{rgb}{0.000000,0.000000,0.000000}%
\pgfsetstrokecolor{currentstroke}%
\pgfsetdash{}{0pt}%
\pgfpathmoveto{\pgfqpoint{3.189689in}{2.121671in}}%
\pgfpathlineto{\pgfqpoint{3.203209in}{2.112899in}}%
\pgfpathlineto{\pgfqpoint{3.216730in}{2.104365in}}%
\pgfpathlineto{\pgfqpoint{3.230250in}{2.096067in}}%
\pgfpathlineto{\pgfqpoint{3.243771in}{2.088004in}}%
\pgfpathlineto{\pgfqpoint{3.251937in}{2.096954in}}%
\pgfpathlineto{\pgfqpoint{3.260097in}{2.105972in}}%
\pgfpathlineto{\pgfqpoint{3.268250in}{2.115055in}}%
\pgfpathlineto{\pgfqpoint{3.276396in}{2.124203in}}%
\pgfpathlineto{\pgfqpoint{3.262893in}{2.132082in}}%
\pgfpathlineto{\pgfqpoint{3.249389in}{2.140195in}}%
\pgfpathlineto{\pgfqpoint{3.235886in}{2.148545in}}%
\pgfpathlineto{\pgfqpoint{3.222382in}{2.157133in}}%
\pgfpathlineto{\pgfqpoint{3.214219in}{2.148157in}}%
\pgfpathlineto{\pgfqpoint{3.206049in}{2.139254in}}%
\pgfpathlineto{\pgfqpoint{3.197873in}{2.130425in}}%
\pgfpathlineto{\pgfqpoint{3.189689in}{2.121671in}}%
\pgfpathclose%
\pgfusepath{fill}%
\end{pgfscope}%
\begin{pgfscope}%
\pgfpathrectangle{\pgfqpoint{1.150000in}{0.150000in}}{\pgfqpoint{5.700000in}{5.700000in}}%
\pgfusepath{clip}%
\pgfsetbuttcap%
\pgfsetroundjoin%
\definecolor{currentfill}{rgb}{0.190631,0.407061,0.556089}%
\pgfsetfillcolor{currentfill}%
\pgfsetfillopacity{0.800000}%
\pgfsetlinewidth{0.000000pt}%
\definecolor{currentstroke}{rgb}{0.000000,0.000000,0.000000}%
\pgfsetstrokecolor{currentstroke}%
\pgfsetdash{}{0pt}%
\pgfpathmoveto{\pgfqpoint{5.019174in}{2.892087in}}%
\pgfpathlineto{\pgfqpoint{5.033203in}{2.898509in}}%
\pgfpathlineto{\pgfqpoint{5.047246in}{2.905109in}}%
\pgfpathlineto{\pgfqpoint{5.061305in}{2.911888in}}%
\pgfpathlineto{\pgfqpoint{5.075379in}{2.918846in}}%
\pgfpathlineto{\pgfqpoint{5.082875in}{2.925667in}}%
\pgfpathlineto{\pgfqpoint{5.090366in}{2.932528in}}%
\pgfpathlineto{\pgfqpoint{5.097852in}{2.939432in}}%
\pgfpathlineto{\pgfqpoint{5.105333in}{2.946387in}}%
\pgfpathlineto{\pgfqpoint{5.091276in}{2.939891in}}%
\pgfpathlineto{\pgfqpoint{5.077235in}{2.933574in}}%
\pgfpathlineto{\pgfqpoint{5.063209in}{2.927434in}}%
\pgfpathlineto{\pgfqpoint{5.049197in}{2.921472in}}%
\pgfpathlineto{\pgfqpoint{5.041698in}{2.914045in}}%
\pgfpathlineto{\pgfqpoint{5.034194in}{2.906676in}}%
\pgfpathlineto{\pgfqpoint{5.026686in}{2.899358in}}%
\pgfpathlineto{\pgfqpoint{5.019174in}{2.892087in}}%
\pgfpathclose%
\pgfusepath{fill}%
\end{pgfscope}%
\begin{pgfscope}%
\pgfpathrectangle{\pgfqpoint{1.150000in}{0.150000in}}{\pgfqpoint{5.700000in}{5.700000in}}%
\pgfusepath{clip}%
\pgfsetbuttcap%
\pgfsetroundjoin%
\definecolor{currentfill}{rgb}{0.280255,0.165693,0.476498}%
\pgfsetfillcolor{currentfill}%
\pgfsetfillopacity{0.800000}%
\pgfsetlinewidth{0.000000pt}%
\definecolor{currentstroke}{rgb}{0.000000,0.000000,0.000000}%
\pgfsetstrokecolor{currentstroke}%
\pgfsetdash{}{0pt}%
\pgfpathmoveto{\pgfqpoint{4.041685in}{2.283452in}}%
\pgfpathlineto{\pgfqpoint{4.055310in}{2.284666in}}%
\pgfpathlineto{\pgfqpoint{4.068944in}{2.286077in}}%
\pgfpathlineto{\pgfqpoint{4.082587in}{2.287683in}}%
\pgfpathlineto{\pgfqpoint{4.096240in}{2.289483in}}%
\pgfpathlineto{\pgfqpoint{4.104112in}{2.299631in}}%
\pgfpathlineto{\pgfqpoint{4.111979in}{2.309753in}}%
\pgfpathlineto{\pgfqpoint{4.119841in}{2.319852in}}%
\pgfpathlineto{\pgfqpoint{4.127698in}{2.329927in}}%
\pgfpathlineto{\pgfqpoint{4.114053in}{2.328198in}}%
\pgfpathlineto{\pgfqpoint{4.100417in}{2.326663in}}%
\pgfpathlineto{\pgfqpoint{4.086790in}{2.325323in}}%
\pgfpathlineto{\pgfqpoint{4.073172in}{2.324179in}}%
\pgfpathlineto{\pgfqpoint{4.065308in}{2.314021in}}%
\pgfpathlineto{\pgfqpoint{4.057439in}{2.303848in}}%
\pgfpathlineto{\pgfqpoint{4.049564in}{2.293659in}}%
\pgfpathlineto{\pgfqpoint{4.041685in}{2.283452in}}%
\pgfpathclose%
\pgfusepath{fill}%
\end{pgfscope}%
\begin{pgfscope}%
\pgfpathrectangle{\pgfqpoint{1.150000in}{0.150000in}}{\pgfqpoint{5.700000in}{5.700000in}}%
\pgfusepath{clip}%
\pgfsetbuttcap%
\pgfsetroundjoin%
\definecolor{currentfill}{rgb}{0.283229,0.120777,0.440584}%
\pgfsetfillcolor{currentfill}%
\pgfsetfillopacity{0.800000}%
\pgfsetlinewidth{0.000000pt}%
\definecolor{currentstroke}{rgb}{0.000000,0.000000,0.000000}%
\pgfsetstrokecolor{currentstroke}%
\pgfsetdash{}{0pt}%
\pgfpathmoveto{\pgfqpoint{2.994283in}{2.214414in}}%
\pgfpathlineto{\pgfqpoint{3.007847in}{2.202449in}}%
\pgfpathlineto{\pgfqpoint{3.021407in}{2.190741in}}%
\pgfpathlineto{\pgfqpoint{3.034964in}{2.179288in}}%
\pgfpathlineto{\pgfqpoint{3.048518in}{2.168090in}}%
\pgfpathlineto{\pgfqpoint{3.056772in}{2.176068in}}%
\pgfpathlineto{\pgfqpoint{3.065017in}{2.184145in}}%
\pgfpathlineto{\pgfqpoint{3.073254in}{2.192317in}}%
\pgfpathlineto{\pgfqpoint{3.081484in}{2.200584in}}%
\pgfpathlineto{\pgfqpoint{3.067950in}{2.211564in}}%
\pgfpathlineto{\pgfqpoint{3.054414in}{2.222798in}}%
\pgfpathlineto{\pgfqpoint{3.040875in}{2.234288in}}%
\pgfpathlineto{\pgfqpoint{3.027333in}{2.246034in}}%
\pgfpathlineto{\pgfqpoint{3.019083in}{2.237974in}}%
\pgfpathlineto{\pgfqpoint{3.010825in}{2.230016in}}%
\pgfpathlineto{\pgfqpoint{3.002558in}{2.222162in}}%
\pgfpathlineto{\pgfqpoint{2.994283in}{2.214414in}}%
\pgfpathclose%
\pgfusepath{fill}%
\end{pgfscope}%
\begin{pgfscope}%
\pgfpathrectangle{\pgfqpoint{1.150000in}{0.150000in}}{\pgfqpoint{5.700000in}{5.700000in}}%
\pgfusepath{clip}%
\pgfsetbuttcap%
\pgfsetroundjoin%
\definecolor{currentfill}{rgb}{0.282290,0.145912,0.461510}%
\pgfsetfillcolor{currentfill}%
\pgfsetfillopacity{0.800000}%
\pgfsetlinewidth{0.000000pt}%
\definecolor{currentstroke}{rgb}{0.000000,0.000000,0.000000}%
\pgfsetstrokecolor{currentstroke}%
\pgfsetdash{}{0pt}%
\pgfpathmoveto{\pgfqpoint{3.955667in}{2.239380in}}%
\pgfpathlineto{\pgfqpoint{3.969267in}{2.239846in}}%
\pgfpathlineto{\pgfqpoint{3.982875in}{2.240509in}}%
\pgfpathlineto{\pgfqpoint{3.996491in}{2.241370in}}%
\pgfpathlineto{\pgfqpoint{4.010115in}{2.242428in}}%
\pgfpathlineto{\pgfqpoint{4.018015in}{2.252715in}}%
\pgfpathlineto{\pgfqpoint{4.025910in}{2.262981in}}%
\pgfpathlineto{\pgfqpoint{4.033800in}{2.273226in}}%
\pgfpathlineto{\pgfqpoint{4.041685in}{2.283452in}}%
\pgfpathlineto{\pgfqpoint{4.028068in}{2.282433in}}%
\pgfpathlineto{\pgfqpoint{4.014459in}{2.281611in}}%
\pgfpathlineto{\pgfqpoint{4.000859in}{2.280986in}}%
\pgfpathlineto{\pgfqpoint{3.987267in}{2.280560in}}%
\pgfpathlineto{\pgfqpoint{3.979375in}{2.270283in}}%
\pgfpathlineto{\pgfqpoint{3.971477in}{2.259995in}}%
\pgfpathlineto{\pgfqpoint{3.963575in}{2.249694in}}%
\pgfpathlineto{\pgfqpoint{3.955667in}{2.239380in}}%
\pgfpathclose%
\pgfusepath{fill}%
\end{pgfscope}%
\begin{pgfscope}%
\pgfpathrectangle{\pgfqpoint{1.150000in}{0.150000in}}{\pgfqpoint{5.700000in}{5.700000in}}%
\pgfusepath{clip}%
\pgfsetbuttcap%
\pgfsetroundjoin%
\definecolor{currentfill}{rgb}{0.277134,0.185228,0.489898}%
\pgfsetfillcolor{currentfill}%
\pgfsetfillopacity{0.800000}%
\pgfsetlinewidth{0.000000pt}%
\definecolor{currentstroke}{rgb}{0.000000,0.000000,0.000000}%
\pgfsetstrokecolor{currentstroke}%
\pgfsetdash{}{0pt}%
\pgfpathmoveto{\pgfqpoint{4.127698in}{2.329927in}}%
\pgfpathlineto{\pgfqpoint{4.141352in}{2.331850in}}%
\pgfpathlineto{\pgfqpoint{4.155015in}{2.333967in}}%
\pgfpathlineto{\pgfqpoint{4.168689in}{2.336277in}}%
\pgfpathlineto{\pgfqpoint{4.182371in}{2.338779in}}%
\pgfpathlineto{\pgfqpoint{4.190216in}{2.348742in}}%
\pgfpathlineto{\pgfqpoint{4.198055in}{2.358676in}}%
\pgfpathlineto{\pgfqpoint{4.205889in}{2.368584in}}%
\pgfpathlineto{\pgfqpoint{4.213717in}{2.378466in}}%
\pgfpathlineto{\pgfqpoint{4.200042in}{2.376067in}}%
\pgfpathlineto{\pgfqpoint{4.186376in}{2.373860in}}%
\pgfpathlineto{\pgfqpoint{4.172720in}{2.371846in}}%
\pgfpathlineto{\pgfqpoint{4.159073in}{2.370025in}}%
\pgfpathlineto{\pgfqpoint{4.151237in}{2.360028in}}%
\pgfpathlineto{\pgfqpoint{4.143396in}{2.350014in}}%
\pgfpathlineto{\pgfqpoint{4.135549in}{2.339980in}}%
\pgfpathlineto{\pgfqpoint{4.127698in}{2.329927in}}%
\pgfpathclose%
\pgfusepath{fill}%
\end{pgfscope}%
\begin{pgfscope}%
\pgfpathrectangle{\pgfqpoint{1.150000in}{0.150000in}}{\pgfqpoint{5.700000in}{5.700000in}}%
\pgfusepath{clip}%
\pgfsetbuttcap%
\pgfsetroundjoin%
\definecolor{currentfill}{rgb}{0.283187,0.125848,0.444960}%
\pgfsetfillcolor{currentfill}%
\pgfsetfillopacity{0.800000}%
\pgfsetlinewidth{0.000000pt}%
\definecolor{currentstroke}{rgb}{0.000000,0.000000,0.000000}%
\pgfsetstrokecolor{currentstroke}%
\pgfsetdash{}{0pt}%
\pgfpathmoveto{\pgfqpoint{3.869633in}{2.198076in}}%
\pgfpathlineto{\pgfqpoint{3.883210in}{2.197750in}}%
\pgfpathlineto{\pgfqpoint{3.896794in}{2.197625in}}%
\pgfpathlineto{\pgfqpoint{3.910386in}{2.197700in}}%
\pgfpathlineto{\pgfqpoint{3.923986in}{2.197974in}}%
\pgfpathlineto{\pgfqpoint{3.931914in}{2.208349in}}%
\pgfpathlineto{\pgfqpoint{3.939837in}{2.218708in}}%
\pgfpathlineto{\pgfqpoint{3.947755in}{2.229051in}}%
\pgfpathlineto{\pgfqpoint{3.955667in}{2.239380in}}%
\pgfpathlineto{\pgfqpoint{3.942075in}{2.239113in}}%
\pgfpathlineto{\pgfqpoint{3.928491in}{2.239046in}}%
\pgfpathlineto{\pgfqpoint{3.914915in}{2.239178in}}%
\pgfpathlineto{\pgfqpoint{3.901346in}{2.239511in}}%
\pgfpathlineto{\pgfqpoint{3.893425in}{2.229163in}}%
\pgfpathlineto{\pgfqpoint{3.885499in}{2.218809in}}%
\pgfpathlineto{\pgfqpoint{3.877568in}{2.208447in}}%
\pgfpathlineto{\pgfqpoint{3.869633in}{2.198076in}}%
\pgfpathclose%
\pgfusepath{fill}%
\end{pgfscope}%
\begin{pgfscope}%
\pgfpathrectangle{\pgfqpoint{1.150000in}{0.150000in}}{\pgfqpoint{5.700000in}{5.700000in}}%
\pgfusepath{clip}%
\pgfsetbuttcap%
\pgfsetroundjoin%
\definecolor{currentfill}{rgb}{0.279566,0.067836,0.391917}%
\pgfsetfillcolor{currentfill}%
\pgfsetfillopacity{0.800000}%
\pgfsetlinewidth{0.000000pt}%
\definecolor{currentstroke}{rgb}{0.000000,0.000000,0.000000}%
\pgfsetstrokecolor{currentstroke}%
\pgfsetdash{}{0pt}%
\pgfpathmoveto{\pgfqpoint{3.470886in}{2.086516in}}%
\pgfpathlineto{\pgfqpoint{3.484399in}{2.081723in}}%
\pgfpathlineto{\pgfqpoint{3.497915in}{2.077148in}}%
\pgfpathlineto{\pgfqpoint{3.511435in}{2.072790in}}%
\pgfpathlineto{\pgfqpoint{3.524958in}{2.068647in}}%
\pgfpathlineto{\pgfqpoint{3.533018in}{2.078626in}}%
\pgfpathlineto{\pgfqpoint{3.541072in}{2.088632in}}%
\pgfpathlineto{\pgfqpoint{3.549121in}{2.098663in}}%
\pgfpathlineto{\pgfqpoint{3.557164in}{2.108720in}}%
\pgfpathlineto{\pgfqpoint{3.543652in}{2.112743in}}%
\pgfpathlineto{\pgfqpoint{3.530145in}{2.116982in}}%
\pgfpathlineto{\pgfqpoint{3.516641in}{2.121437in}}%
\pgfpathlineto{\pgfqpoint{3.503140in}{2.126110in}}%
\pgfpathlineto{\pgfqpoint{3.495085in}{2.116162in}}%
\pgfpathlineto{\pgfqpoint{3.487025in}{2.106246in}}%
\pgfpathlineto{\pgfqpoint{3.478959in}{2.096364in}}%
\pgfpathlineto{\pgfqpoint{3.470886in}{2.086516in}}%
\pgfpathclose%
\pgfusepath{fill}%
\end{pgfscope}%
\begin{pgfscope}%
\pgfpathrectangle{\pgfqpoint{1.150000in}{0.150000in}}{\pgfqpoint{5.700000in}{5.700000in}}%
\pgfusepath{clip}%
\pgfsetbuttcap%
\pgfsetroundjoin%
\definecolor{currentfill}{rgb}{0.271828,0.209303,0.504434}%
\pgfsetfillcolor{currentfill}%
\pgfsetfillopacity{0.800000}%
\pgfsetlinewidth{0.000000pt}%
\definecolor{currentstroke}{rgb}{0.000000,0.000000,0.000000}%
\pgfsetstrokecolor{currentstroke}%
\pgfsetdash{}{0pt}%
\pgfpathmoveto{\pgfqpoint{4.213717in}{2.378466in}}%
\pgfpathlineto{\pgfqpoint{4.227403in}{2.381057in}}%
\pgfpathlineto{\pgfqpoint{4.241098in}{2.383840in}}%
\pgfpathlineto{\pgfqpoint{4.254804in}{2.386814in}}%
\pgfpathlineto{\pgfqpoint{4.268521in}{2.389978in}}%
\pgfpathlineto{\pgfqpoint{4.276336in}{2.399714in}}%
\pgfpathlineto{\pgfqpoint{4.284147in}{2.409421in}}%
\pgfpathlineto{\pgfqpoint{4.291952in}{2.419099in}}%
\pgfpathlineto{\pgfqpoint{4.299752in}{2.428752in}}%
\pgfpathlineto{\pgfqpoint{4.286043in}{2.425723in}}%
\pgfpathlineto{\pgfqpoint{4.272345in}{2.422885in}}%
\pgfpathlineto{\pgfqpoint{4.258657in}{2.420237in}}%
\pgfpathlineto{\pgfqpoint{4.244979in}{2.417781in}}%
\pgfpathlineto{\pgfqpoint{4.237172in}{2.407981in}}%
\pgfpathlineto{\pgfqpoint{4.229359in}{2.398163in}}%
\pgfpathlineto{\pgfqpoint{4.221540in}{2.388325in}}%
\pgfpathlineto{\pgfqpoint{4.213717in}{2.378466in}}%
\pgfpathclose%
\pgfusepath{fill}%
\end{pgfscope}%
\begin{pgfscope}%
\pgfpathrectangle{\pgfqpoint{1.150000in}{0.150000in}}{\pgfqpoint{5.700000in}{5.700000in}}%
\pgfusepath{clip}%
\pgfsetbuttcap%
\pgfsetroundjoin%
\definecolor{currentfill}{rgb}{0.182256,0.426184,0.557120}%
\pgfsetfillcolor{currentfill}%
\pgfsetfillopacity{0.800000}%
\pgfsetlinewidth{0.000000pt}%
\definecolor{currentstroke}{rgb}{0.000000,0.000000,0.000000}%
\pgfsetstrokecolor{currentstroke}%
\pgfsetdash{}{0pt}%
\pgfpathmoveto{\pgfqpoint{5.105333in}{2.946387in}}%
\pgfpathlineto{\pgfqpoint{5.119404in}{2.953060in}}%
\pgfpathlineto{\pgfqpoint{5.133491in}{2.959910in}}%
\pgfpathlineto{\pgfqpoint{5.147593in}{2.966938in}}%
\pgfpathlineto{\pgfqpoint{5.161711in}{2.974144in}}%
\pgfpathlineto{\pgfqpoint{5.169169in}{2.980670in}}%
\pgfpathlineto{\pgfqpoint{5.176623in}{2.987249in}}%
\pgfpathlineto{\pgfqpoint{5.184072in}{2.993886in}}%
\pgfpathlineto{\pgfqpoint{5.191516in}{3.000587in}}%
\pgfpathlineto{\pgfqpoint{5.177418in}{2.993877in}}%
\pgfpathlineto{\pgfqpoint{5.163335in}{2.987343in}}%
\pgfpathlineto{\pgfqpoint{5.149267in}{2.980986in}}%
\pgfpathlineto{\pgfqpoint{5.135214in}{2.974806in}}%
\pgfpathlineto{\pgfqpoint{5.127750in}{2.967600in}}%
\pgfpathlineto{\pgfqpoint{5.120282in}{2.960465in}}%
\pgfpathlineto{\pgfqpoint{5.112810in}{2.953396in}}%
\pgfpathlineto{\pgfqpoint{5.105333in}{2.946387in}}%
\pgfpathclose%
\pgfusepath{fill}%
\end{pgfscope}%
\begin{pgfscope}%
\pgfpathrectangle{\pgfqpoint{1.150000in}{0.150000in}}{\pgfqpoint{5.700000in}{5.700000in}}%
\pgfusepath{clip}%
\pgfsetbuttcap%
\pgfsetroundjoin%
\definecolor{currentfill}{rgb}{0.223925,0.334994,0.548053}%
\pgfsetfillcolor{currentfill}%
\pgfsetfillopacity{0.800000}%
\pgfsetlinewidth{0.000000pt}%
\definecolor{currentstroke}{rgb}{0.000000,0.000000,0.000000}%
\pgfsetstrokecolor{currentstroke}%
\pgfsetdash{}{0pt}%
\pgfpathmoveto{\pgfqpoint{2.556994in}{2.746109in}}%
\pgfpathlineto{\pgfqpoint{2.570802in}{2.724711in}}%
\pgfpathlineto{\pgfqpoint{2.584598in}{2.703651in}}%
\pgfpathlineto{\pgfqpoint{2.598381in}{2.682924in}}%
\pgfpathlineto{\pgfqpoint{2.612153in}{2.662529in}}%
\pgfpathlineto{\pgfqpoint{2.620612in}{2.668621in}}%
\pgfpathlineto{\pgfqpoint{2.629060in}{2.674877in}}%
\pgfpathlineto{\pgfqpoint{2.637497in}{2.681293in}}%
\pgfpathlineto{\pgfqpoint{2.645923in}{2.687869in}}%
\pgfpathlineto{\pgfqpoint{2.632181in}{2.708034in}}%
\pgfpathlineto{\pgfqpoint{2.618428in}{2.728528in}}%
\pgfpathlineto{\pgfqpoint{2.604663in}{2.749357in}}%
\pgfpathlineto{\pgfqpoint{2.590887in}{2.770522in}}%
\pgfpathlineto{\pgfqpoint{2.582431in}{2.764166in}}%
\pgfpathlineto{\pgfqpoint{2.573964in}{2.757977in}}%
\pgfpathlineto{\pgfqpoint{2.565485in}{2.751957in}}%
\pgfpathlineto{\pgfqpoint{2.556994in}{2.746109in}}%
\pgfpathclose%
\pgfusepath{fill}%
\end{pgfscope}%
\begin{pgfscope}%
\pgfpathrectangle{\pgfqpoint{1.150000in}{0.150000in}}{\pgfqpoint{5.700000in}{5.700000in}}%
\pgfusepath{clip}%
\pgfsetbuttcap%
\pgfsetroundjoin%
\definecolor{currentfill}{rgb}{0.265145,0.232956,0.516599}%
\pgfsetfillcolor{currentfill}%
\pgfsetfillopacity{0.800000}%
\pgfsetlinewidth{0.000000pt}%
\definecolor{currentstroke}{rgb}{0.000000,0.000000,0.000000}%
\pgfsetstrokecolor{currentstroke}%
\pgfsetdash{}{0pt}%
\pgfpathmoveto{\pgfqpoint{4.299752in}{2.428752in}}%
\pgfpathlineto{\pgfqpoint{4.313471in}{2.431971in}}%
\pgfpathlineto{\pgfqpoint{4.327201in}{2.435380in}}%
\pgfpathlineto{\pgfqpoint{4.340942in}{2.438978in}}%
\pgfpathlineto{\pgfqpoint{4.354694in}{2.442765in}}%
\pgfpathlineto{\pgfqpoint{4.362481in}{2.452239in}}%
\pgfpathlineto{\pgfqpoint{4.370263in}{2.461683in}}%
\pgfpathlineto{\pgfqpoint{4.378038in}{2.471100in}}%
\pgfpathlineto{\pgfqpoint{4.385809in}{2.480491in}}%
\pgfpathlineto{\pgfqpoint{4.372065in}{2.476872in}}%
\pgfpathlineto{\pgfqpoint{4.358332in}{2.473441in}}%
\pgfpathlineto{\pgfqpoint{4.344610in}{2.470200in}}%
\pgfpathlineto{\pgfqpoint{4.330899in}{2.467148in}}%
\pgfpathlineto{\pgfqpoint{4.323120in}{2.457577in}}%
\pgfpathlineto{\pgfqpoint{4.315336in}{2.447989in}}%
\pgfpathlineto{\pgfqpoint{4.307547in}{2.438382in}}%
\pgfpathlineto{\pgfqpoint{4.299752in}{2.428752in}}%
\pgfpathclose%
\pgfusepath{fill}%
\end{pgfscope}%
\begin{pgfscope}%
\pgfpathrectangle{\pgfqpoint{1.150000in}{0.150000in}}{\pgfqpoint{5.700000in}{5.700000in}}%
\pgfusepath{clip}%
\pgfsetbuttcap%
\pgfsetroundjoin%
\definecolor{currentfill}{rgb}{0.283091,0.110553,0.431554}%
\pgfsetfillcolor{currentfill}%
\pgfsetfillopacity{0.800000}%
\pgfsetlinewidth{0.000000pt}%
\definecolor{currentstroke}{rgb}{0.000000,0.000000,0.000000}%
\pgfsetstrokecolor{currentstroke}%
\pgfsetdash{}{0pt}%
\pgfpathmoveto{\pgfqpoint{3.783564in}{2.159927in}}%
\pgfpathlineto{\pgfqpoint{3.797123in}{2.158767in}}%
\pgfpathlineto{\pgfqpoint{3.810688in}{2.157810in}}%
\pgfpathlineto{\pgfqpoint{3.824259in}{2.157057in}}%
\pgfpathlineto{\pgfqpoint{3.837838in}{2.156505in}}%
\pgfpathlineto{\pgfqpoint{3.845794in}{2.166912in}}%
\pgfpathlineto{\pgfqpoint{3.853745in}{2.177309in}}%
\pgfpathlineto{\pgfqpoint{3.861691in}{2.187697in}}%
\pgfpathlineto{\pgfqpoint{3.869633in}{2.198076in}}%
\pgfpathlineto{\pgfqpoint{3.856062in}{2.198603in}}%
\pgfpathlineto{\pgfqpoint{3.842499in}{2.199332in}}%
\pgfpathlineto{\pgfqpoint{3.828943in}{2.200264in}}%
\pgfpathlineto{\pgfqpoint{3.815393in}{2.201399in}}%
\pgfpathlineto{\pgfqpoint{3.807444in}{2.191033in}}%
\pgfpathlineto{\pgfqpoint{3.799489in}{2.180666in}}%
\pgfpathlineto{\pgfqpoint{3.791529in}{2.170297in}}%
\pgfpathlineto{\pgfqpoint{3.783564in}{2.159927in}}%
\pgfpathclose%
\pgfusepath{fill}%
\end{pgfscope}%
\begin{pgfscope}%
\pgfpathrectangle{\pgfqpoint{1.150000in}{0.150000in}}{\pgfqpoint{5.700000in}{5.700000in}}%
\pgfusepath{clip}%
\pgfsetbuttcap%
\pgfsetroundjoin%
\definecolor{currentfill}{rgb}{0.174274,0.445044,0.557792}%
\pgfsetfillcolor{currentfill}%
\pgfsetfillopacity{0.800000}%
\pgfsetlinewidth{0.000000pt}%
\definecolor{currentstroke}{rgb}{0.000000,0.000000,0.000000}%
\pgfsetstrokecolor{currentstroke}%
\pgfsetdash{}{0pt}%
\pgfpathmoveto{\pgfqpoint{5.191516in}{3.000587in}}%
\pgfpathlineto{\pgfqpoint{5.205630in}{3.007474in}}%
\pgfpathlineto{\pgfqpoint{5.219760in}{3.014538in}}%
\pgfpathlineto{\pgfqpoint{5.233905in}{3.021778in}}%
\pgfpathlineto{\pgfqpoint{5.248066in}{3.029194in}}%
\pgfpathlineto{\pgfqpoint{5.255487in}{3.035449in}}%
\pgfpathlineto{\pgfqpoint{5.262902in}{3.041771in}}%
\pgfpathlineto{\pgfqpoint{5.270314in}{3.048167in}}%
\pgfpathlineto{\pgfqpoint{5.277722in}{3.054643in}}%
\pgfpathlineto{\pgfqpoint{5.263582in}{3.047755in}}%
\pgfpathlineto{\pgfqpoint{5.249457in}{3.041042in}}%
\pgfpathlineto{\pgfqpoint{5.235348in}{3.034505in}}%
\pgfpathlineto{\pgfqpoint{5.221255in}{3.028144in}}%
\pgfpathlineto{\pgfqpoint{5.213826in}{3.021130in}}%
\pgfpathlineto{\pgfqpoint{5.206393in}{3.014203in}}%
\pgfpathlineto{\pgfqpoint{5.198957in}{3.007358in}}%
\pgfpathlineto{\pgfqpoint{5.191516in}{3.000587in}}%
\pgfpathclose%
\pgfusepath{fill}%
\end{pgfscope}%
\begin{pgfscope}%
\pgfpathrectangle{\pgfqpoint{1.150000in}{0.150000in}}{\pgfqpoint{5.700000in}{5.700000in}}%
\pgfusepath{clip}%
\pgfsetbuttcap%
\pgfsetroundjoin%
\definecolor{currentfill}{rgb}{0.257322,0.256130,0.526563}%
\pgfsetfillcolor{currentfill}%
\pgfsetfillopacity{0.800000}%
\pgfsetlinewidth{0.000000pt}%
\definecolor{currentstroke}{rgb}{0.000000,0.000000,0.000000}%
\pgfsetstrokecolor{currentstroke}%
\pgfsetdash{}{0pt}%
\pgfpathmoveto{\pgfqpoint{4.385809in}{2.480491in}}%
\pgfpathlineto{\pgfqpoint{4.399564in}{2.484299in}}%
\pgfpathlineto{\pgfqpoint{4.413331in}{2.488294in}}%
\pgfpathlineto{\pgfqpoint{4.427109in}{2.492478in}}%
\pgfpathlineto{\pgfqpoint{4.440899in}{2.496848in}}%
\pgfpathlineto{\pgfqpoint{4.448656in}{2.506029in}}%
\pgfpathlineto{\pgfqpoint{4.456407in}{2.515181in}}%
\pgfpathlineto{\pgfqpoint{4.464153in}{2.524308in}}%
\pgfpathlineto{\pgfqpoint{4.471893in}{2.533411in}}%
\pgfpathlineto{\pgfqpoint{4.458112in}{2.529241in}}%
\pgfpathlineto{\pgfqpoint{4.444343in}{2.525258in}}%
\pgfpathlineto{\pgfqpoint{4.430584in}{2.521462in}}%
\pgfpathlineto{\pgfqpoint{4.416838in}{2.517854in}}%
\pgfpathlineto{\pgfqpoint{4.409089in}{2.508539in}}%
\pgfpathlineto{\pgfqpoint{4.401334in}{2.499208in}}%
\pgfpathlineto{\pgfqpoint{4.393574in}{2.489860in}}%
\pgfpathlineto{\pgfqpoint{4.385809in}{2.480491in}}%
\pgfpathclose%
\pgfusepath{fill}%
\end{pgfscope}%
\begin{pgfscope}%
\pgfpathrectangle{\pgfqpoint{1.150000in}{0.150000in}}{\pgfqpoint{5.700000in}{5.700000in}}%
\pgfusepath{clip}%
\pgfsetbuttcap%
\pgfsetroundjoin%
\definecolor{currentfill}{rgb}{0.282910,0.105393,0.426902}%
\pgfsetfillcolor{currentfill}%
\pgfsetfillopacity{0.800000}%
\pgfsetlinewidth{0.000000pt}%
\definecolor{currentstroke}{rgb}{0.000000,0.000000,0.000000}%
\pgfsetstrokecolor{currentstroke}%
\pgfsetdash{}{0pt}%
\pgfpathmoveto{\pgfqpoint{3.048518in}{2.168090in}}%
\pgfpathlineto{\pgfqpoint{3.062070in}{2.157144in}}%
\pgfpathlineto{\pgfqpoint{3.075620in}{2.146448in}}%
\pgfpathlineto{\pgfqpoint{3.089167in}{2.136001in}}%
\pgfpathlineto{\pgfqpoint{3.102712in}{2.125801in}}%
\pgfpathlineto{\pgfqpoint{3.110945in}{2.134009in}}%
\pgfpathlineto{\pgfqpoint{3.119170in}{2.142307in}}%
\pgfpathlineto{\pgfqpoint{3.127388in}{2.150693in}}%
\pgfpathlineto{\pgfqpoint{3.135598in}{2.159165in}}%
\pgfpathlineto{\pgfqpoint{3.122072in}{2.169148in}}%
\pgfpathlineto{\pgfqpoint{3.108544in}{2.179378in}}%
\pgfpathlineto{\pgfqpoint{3.095015in}{2.189856in}}%
\pgfpathlineto{\pgfqpoint{3.081484in}{2.200584in}}%
\pgfpathlineto{\pgfqpoint{3.073254in}{2.192317in}}%
\pgfpathlineto{\pgfqpoint{3.065017in}{2.184145in}}%
\pgfpathlineto{\pgfqpoint{3.056772in}{2.176068in}}%
\pgfpathlineto{\pgfqpoint{3.048518in}{2.168090in}}%
\pgfpathclose%
\pgfusepath{fill}%
\end{pgfscope}%
\begin{pgfscope}%
\pgfpathrectangle{\pgfqpoint{1.150000in}{0.150000in}}{\pgfqpoint{5.700000in}{5.700000in}}%
\pgfusepath{clip}%
\pgfsetbuttcap%
\pgfsetroundjoin%
\definecolor{currentfill}{rgb}{0.282327,0.094955,0.417331}%
\pgfsetfillcolor{currentfill}%
\pgfsetfillopacity{0.800000}%
\pgfsetlinewidth{0.000000pt}%
\definecolor{currentstroke}{rgb}{0.000000,0.000000,0.000000}%
\pgfsetstrokecolor{currentstroke}%
\pgfsetdash{}{0pt}%
\pgfpathmoveto{\pgfqpoint{3.697444in}{2.125343in}}%
\pgfpathlineto{\pgfqpoint{3.710988in}{2.123306in}}%
\pgfpathlineto{\pgfqpoint{3.724537in}{2.121475in}}%
\pgfpathlineto{\pgfqpoint{3.738092in}{2.119850in}}%
\pgfpathlineto{\pgfqpoint{3.751653in}{2.118430in}}%
\pgfpathlineto{\pgfqpoint{3.759639in}{2.128806in}}%
\pgfpathlineto{\pgfqpoint{3.767619in}{2.139181in}}%
\pgfpathlineto{\pgfqpoint{3.775594in}{2.149555in}}%
\pgfpathlineto{\pgfqpoint{3.783564in}{2.159927in}}%
\pgfpathlineto{\pgfqpoint{3.770012in}{2.161291in}}%
\pgfpathlineto{\pgfqpoint{3.756466in}{2.162860in}}%
\pgfpathlineto{\pgfqpoint{3.742926in}{2.164634in}}%
\pgfpathlineto{\pgfqpoint{3.729392in}{2.166616in}}%
\pgfpathlineto{\pgfqpoint{3.721413in}{2.156288in}}%
\pgfpathlineto{\pgfqpoint{3.713429in}{2.145967in}}%
\pgfpathlineto{\pgfqpoint{3.705439in}{2.135652in}}%
\pgfpathlineto{\pgfqpoint{3.697444in}{2.125343in}}%
\pgfpathclose%
\pgfusepath{fill}%
\end{pgfscope}%
\begin{pgfscope}%
\pgfpathrectangle{\pgfqpoint{1.150000in}{0.150000in}}{\pgfqpoint{5.700000in}{5.700000in}}%
\pgfusepath{clip}%
\pgfsetbuttcap%
\pgfsetroundjoin%
\definecolor{currentfill}{rgb}{0.165117,0.467423,0.558141}%
\pgfsetfillcolor{currentfill}%
\pgfsetfillopacity{0.800000}%
\pgfsetlinewidth{0.000000pt}%
\definecolor{currentstroke}{rgb}{0.000000,0.000000,0.000000}%
\pgfsetstrokecolor{currentstroke}%
\pgfsetdash{}{0pt}%
\pgfpathmoveto{\pgfqpoint{5.277722in}{3.054643in}}%
\pgfpathlineto{\pgfqpoint{5.291878in}{3.061707in}}%
\pgfpathlineto{\pgfqpoint{5.306050in}{3.068947in}}%
\pgfpathlineto{\pgfqpoint{5.320239in}{3.076362in}}%
\pgfpathlineto{\pgfqpoint{5.334443in}{3.083953in}}%
\pgfpathlineto{\pgfqpoint{5.341825in}{3.089965in}}%
\pgfpathlineto{\pgfqpoint{5.349204in}{3.096061in}}%
\pgfpathlineto{\pgfqpoint{5.356578in}{3.102248in}}%
\pgfpathlineto{\pgfqpoint{5.363949in}{3.108532in}}%
\pgfpathlineto{\pgfqpoint{5.349768in}{3.101502in}}%
\pgfpathlineto{\pgfqpoint{5.335602in}{3.094647in}}%
\pgfpathlineto{\pgfqpoint{5.321452in}{3.087967in}}%
\pgfpathlineto{\pgfqpoint{5.307318in}{3.081462in}}%
\pgfpathlineto{\pgfqpoint{5.299924in}{3.074607in}}%
\pgfpathlineto{\pgfqpoint{5.292527in}{3.067857in}}%
\pgfpathlineto{\pgfqpoint{5.285126in}{3.061204in}}%
\pgfpathlineto{\pgfqpoint{5.277722in}{3.054643in}}%
\pgfpathclose%
\pgfusepath{fill}%
\end{pgfscope}%
\begin{pgfscope}%
\pgfpathrectangle{\pgfqpoint{1.150000in}{0.150000in}}{\pgfqpoint{5.700000in}{5.700000in}}%
\pgfusepath{clip}%
\pgfsetbuttcap%
\pgfsetroundjoin%
\definecolor{currentfill}{rgb}{0.248629,0.278775,0.534556}%
\pgfsetfillcolor{currentfill}%
\pgfsetfillopacity{0.800000}%
\pgfsetlinewidth{0.000000pt}%
\definecolor{currentstroke}{rgb}{0.000000,0.000000,0.000000}%
\pgfsetstrokecolor{currentstroke}%
\pgfsetdash{}{0pt}%
\pgfpathmoveto{\pgfqpoint{4.471893in}{2.533411in}}%
\pgfpathlineto{\pgfqpoint{4.485686in}{2.537768in}}%
\pgfpathlineto{\pgfqpoint{4.499491in}{2.542312in}}%
\pgfpathlineto{\pgfqpoint{4.513308in}{2.547042in}}%
\pgfpathlineto{\pgfqpoint{4.527138in}{2.551957in}}%
\pgfpathlineto{\pgfqpoint{4.534864in}{2.560819in}}%
\pgfpathlineto{\pgfqpoint{4.542584in}{2.569656in}}%
\pgfpathlineto{\pgfqpoint{4.550299in}{2.578470in}}%
\pgfpathlineto{\pgfqpoint{4.558008in}{2.587263in}}%
\pgfpathlineto{\pgfqpoint{4.544188in}{2.582581in}}%
\pgfpathlineto{\pgfqpoint{4.530380in}{2.578084in}}%
\pgfpathlineto{\pgfqpoint{4.516584in}{2.573773in}}%
\pgfpathlineto{\pgfqpoint{4.502801in}{2.569647in}}%
\pgfpathlineto{\pgfqpoint{4.495082in}{2.560609in}}%
\pgfpathlineto{\pgfqpoint{4.487358in}{2.551559in}}%
\pgfpathlineto{\pgfqpoint{4.479628in}{2.542494in}}%
\pgfpathlineto{\pgfqpoint{4.471893in}{2.533411in}}%
\pgfpathclose%
\pgfusepath{fill}%
\end{pgfscope}%
\begin{pgfscope}%
\pgfpathrectangle{\pgfqpoint{1.150000in}{0.150000in}}{\pgfqpoint{5.700000in}{5.700000in}}%
\pgfusepath{clip}%
\pgfsetbuttcap%
\pgfsetroundjoin%
\definecolor{currentfill}{rgb}{0.280267,0.073417,0.397163}%
\pgfsetfillcolor{currentfill}%
\pgfsetfillopacity{0.800000}%
\pgfsetlinewidth{0.000000pt}%
\definecolor{currentstroke}{rgb}{0.000000,0.000000,0.000000}%
\pgfsetstrokecolor{currentstroke}%
\pgfsetdash{}{0pt}%
\pgfpathmoveto{\pgfqpoint{3.243771in}{2.088004in}}%
\pgfpathlineto{\pgfqpoint{3.257291in}{2.080174in}}%
\pgfpathlineto{\pgfqpoint{3.270813in}{2.072577in}}%
\pgfpathlineto{\pgfqpoint{3.284334in}{2.065210in}}%
\pgfpathlineto{\pgfqpoint{3.297857in}{2.058072in}}%
\pgfpathlineto{\pgfqpoint{3.306008in}{2.067218in}}%
\pgfpathlineto{\pgfqpoint{3.314151in}{2.076423in}}%
\pgfpathlineto{\pgfqpoint{3.322288in}{2.085686in}}%
\pgfpathlineto{\pgfqpoint{3.330419in}{2.095006in}}%
\pgfpathlineto{\pgfqpoint{3.316912in}{2.101960in}}%
\pgfpathlineto{\pgfqpoint{3.303406in}{2.109143in}}%
\pgfpathlineto{\pgfqpoint{3.289901in}{2.116557in}}%
\pgfpathlineto{\pgfqpoint{3.276396in}{2.124203in}}%
\pgfpathlineto{\pgfqpoint{3.268250in}{2.115055in}}%
\pgfpathlineto{\pgfqpoint{3.260097in}{2.105972in}}%
\pgfpathlineto{\pgfqpoint{3.251937in}{2.096954in}}%
\pgfpathlineto{\pgfqpoint{3.243771in}{2.088004in}}%
\pgfpathclose%
\pgfusepath{fill}%
\end{pgfscope}%
\begin{pgfscope}%
\pgfpathrectangle{\pgfqpoint{1.150000in}{0.150000in}}{\pgfqpoint{5.700000in}{5.700000in}}%
\pgfusepath{clip}%
\pgfsetbuttcap%
\pgfsetroundjoin%
\definecolor{currentfill}{rgb}{0.157729,0.485932,0.558013}%
\pgfsetfillcolor{currentfill}%
\pgfsetfillopacity{0.800000}%
\pgfsetlinewidth{0.000000pt}%
\definecolor{currentstroke}{rgb}{0.000000,0.000000,0.000000}%
\pgfsetstrokecolor{currentstroke}%
\pgfsetdash{}{0pt}%
\pgfpathmoveto{\pgfqpoint{5.363949in}{3.108532in}}%
\pgfpathlineto{\pgfqpoint{5.378147in}{3.115736in}}%
\pgfpathlineto{\pgfqpoint{5.392362in}{3.123115in}}%
\pgfpathlineto{\pgfqpoint{5.406592in}{3.130669in}}%
\pgfpathlineto{\pgfqpoint{5.420840in}{3.138398in}}%
\pgfpathlineto{\pgfqpoint{5.428184in}{3.144202in}}%
\pgfpathlineto{\pgfqpoint{5.435524in}{3.150109in}}%
\pgfpathlineto{\pgfqpoint{5.442862in}{3.156124in}}%
\pgfpathlineto{\pgfqpoint{5.450197in}{3.162255in}}%
\pgfpathlineto{\pgfqpoint{5.435974in}{3.155120in}}%
\pgfpathlineto{\pgfqpoint{5.421768in}{3.148160in}}%
\pgfpathlineto{\pgfqpoint{5.407578in}{3.141373in}}%
\pgfpathlineto{\pgfqpoint{5.393404in}{3.134761in}}%
\pgfpathlineto{\pgfqpoint{5.386045in}{3.128026in}}%
\pgfpathlineto{\pgfqpoint{5.378682in}{3.121414in}}%
\pgfpathlineto{\pgfqpoint{5.371317in}{3.114918in}}%
\pgfpathlineto{\pgfqpoint{5.363949in}{3.108532in}}%
\pgfpathclose%
\pgfusepath{fill}%
\end{pgfscope}%
\begin{pgfscope}%
\pgfpathrectangle{\pgfqpoint{1.150000in}{0.150000in}}{\pgfqpoint{5.700000in}{5.700000in}}%
\pgfusepath{clip}%
\pgfsetbuttcap%
\pgfsetroundjoin%
\definecolor{currentfill}{rgb}{0.206756,0.371758,0.553117}%
\pgfsetfillcolor{currentfill}%
\pgfsetfillopacity{0.800000}%
\pgfsetlinewidth{0.000000pt}%
\definecolor{currentstroke}{rgb}{0.000000,0.000000,0.000000}%
\pgfsetstrokecolor{currentstroke}%
\pgfsetdash{}{0pt}%
\pgfpathmoveto{\pgfqpoint{2.501633in}{2.835141in}}%
\pgfpathlineto{\pgfqpoint{2.515494in}{2.812360in}}%
\pgfpathlineto{\pgfqpoint{2.529340in}{2.789930in}}%
\pgfpathlineto{\pgfqpoint{2.543174in}{2.767847in}}%
\pgfpathlineto{\pgfqpoint{2.556994in}{2.746109in}}%
\pgfpathlineto{\pgfqpoint{2.565485in}{2.751957in}}%
\pgfpathlineto{\pgfqpoint{2.573964in}{2.757977in}}%
\pgfpathlineto{\pgfqpoint{2.582431in}{2.764166in}}%
\pgfpathlineto{\pgfqpoint{2.590887in}{2.770522in}}%
\pgfpathlineto{\pgfqpoint{2.577098in}{2.792027in}}%
\pgfpathlineto{\pgfqpoint{2.563296in}{2.813876in}}%
\pgfpathlineto{\pgfqpoint{2.549481in}{2.836072in}}%
\pgfpathlineto{\pgfqpoint{2.535653in}{2.858618in}}%
\pgfpathlineto{\pgfqpoint{2.527167in}{2.852483in}}%
\pgfpathlineto{\pgfqpoint{2.518668in}{2.846524in}}%
\pgfpathlineto{\pgfqpoint{2.510156in}{2.840743in}}%
\pgfpathlineto{\pgfqpoint{2.501633in}{2.835141in}}%
\pgfpathclose%
\pgfusepath{fill}%
\end{pgfscope}%
\begin{pgfscope}%
\pgfpathrectangle{\pgfqpoint{1.150000in}{0.150000in}}{\pgfqpoint{5.700000in}{5.700000in}}%
\pgfusepath{clip}%
\pgfsetbuttcap%
\pgfsetroundjoin%
\definecolor{currentfill}{rgb}{0.278791,0.062145,0.386592}%
\pgfsetfillcolor{currentfill}%
\pgfsetfillopacity{0.800000}%
\pgfsetlinewidth{0.000000pt}%
\definecolor{currentstroke}{rgb}{0.000000,0.000000,0.000000}%
\pgfsetstrokecolor{currentstroke}%
\pgfsetdash{}{0pt}%
\pgfpathmoveto{\pgfqpoint{3.384462in}{2.069459in}}%
\pgfpathlineto{\pgfqpoint{3.397978in}{2.063633in}}%
\pgfpathlineto{\pgfqpoint{3.411496in}{2.058030in}}%
\pgfpathlineto{\pgfqpoint{3.425016in}{2.052647in}}%
\pgfpathlineto{\pgfqpoint{3.438539in}{2.047485in}}%
\pgfpathlineto{\pgfqpoint{3.446635in}{2.057186in}}%
\pgfpathlineto{\pgfqpoint{3.454725in}{2.066926in}}%
\pgfpathlineto{\pgfqpoint{3.462808in}{2.076703in}}%
\pgfpathlineto{\pgfqpoint{3.470886in}{2.086516in}}%
\pgfpathlineto{\pgfqpoint{3.457377in}{2.091527in}}%
\pgfpathlineto{\pgfqpoint{3.443870in}{2.096759in}}%
\pgfpathlineto{\pgfqpoint{3.430365in}{2.102211in}}%
\pgfpathlineto{\pgfqpoint{3.416864in}{2.107885in}}%
\pgfpathlineto{\pgfqpoint{3.408772in}{2.098211in}}%
\pgfpathlineto{\pgfqpoint{3.400675in}{2.088582in}}%
\pgfpathlineto{\pgfqpoint{3.392572in}{2.078998in}}%
\pgfpathlineto{\pgfqpoint{3.384462in}{2.069459in}}%
\pgfpathclose%
\pgfusepath{fill}%
\end{pgfscope}%
\begin{pgfscope}%
\pgfpathrectangle{\pgfqpoint{1.150000in}{0.150000in}}{\pgfqpoint{5.700000in}{5.700000in}}%
\pgfusepath{clip}%
\pgfsetbuttcap%
\pgfsetroundjoin%
\definecolor{currentfill}{rgb}{0.239346,0.300855,0.540844}%
\pgfsetfillcolor{currentfill}%
\pgfsetfillopacity{0.800000}%
\pgfsetlinewidth{0.000000pt}%
\definecolor{currentstroke}{rgb}{0.000000,0.000000,0.000000}%
\pgfsetstrokecolor{currentstroke}%
\pgfsetdash{}{0pt}%
\pgfpathmoveto{\pgfqpoint{4.558008in}{2.587263in}}%
\pgfpathlineto{\pgfqpoint{4.571840in}{2.592131in}}%
\pgfpathlineto{\pgfqpoint{4.585685in}{2.597184in}}%
\pgfpathlineto{\pgfqpoint{4.599543in}{2.602422in}}%
\pgfpathlineto{\pgfqpoint{4.613413in}{2.607844in}}%
\pgfpathlineto{\pgfqpoint{4.621107in}{2.616368in}}%
\pgfpathlineto{\pgfqpoint{4.628796in}{2.624870in}}%
\pgfpathlineto{\pgfqpoint{4.636478in}{2.633353in}}%
\pgfpathlineto{\pgfqpoint{4.644155in}{2.641821in}}%
\pgfpathlineto{\pgfqpoint{4.630295in}{2.636665in}}%
\pgfpathlineto{\pgfqpoint{4.616448in}{2.631693in}}%
\pgfpathlineto{\pgfqpoint{4.602613in}{2.626905in}}%
\pgfpathlineto{\pgfqpoint{4.588791in}{2.622302in}}%
\pgfpathlineto{\pgfqpoint{4.581103in}{2.613556in}}%
\pgfpathlineto{\pgfqpoint{4.573410in}{2.604804in}}%
\pgfpathlineto{\pgfqpoint{4.565712in}{2.596040in}}%
\pgfpathlineto{\pgfqpoint{4.558008in}{2.587263in}}%
\pgfpathclose%
\pgfusepath{fill}%
\end{pgfscope}%
\begin{pgfscope}%
\pgfpathrectangle{\pgfqpoint{1.150000in}{0.150000in}}{\pgfqpoint{5.700000in}{5.700000in}}%
\pgfusepath{clip}%
\pgfsetbuttcap%
\pgfsetroundjoin%
\definecolor{currentfill}{rgb}{0.280894,0.078907,0.402329}%
\pgfsetfillcolor{currentfill}%
\pgfsetfillopacity{0.800000}%
\pgfsetlinewidth{0.000000pt}%
\definecolor{currentstroke}{rgb}{0.000000,0.000000,0.000000}%
\pgfsetstrokecolor{currentstroke}%
\pgfsetdash{}{0pt}%
\pgfpathmoveto{\pgfqpoint{3.611250in}{2.094763in}}%
\pgfpathlineto{\pgfqpoint{3.624783in}{2.091803in}}%
\pgfpathlineto{\pgfqpoint{3.638321in}{2.089053in}}%
\pgfpathlineto{\pgfqpoint{3.651863in}{2.086512in}}%
\pgfpathlineto{\pgfqpoint{3.665411in}{2.084180in}}%
\pgfpathlineto{\pgfqpoint{3.673428in}{2.094459in}}%
\pgfpathlineto{\pgfqpoint{3.681438in}{2.104747in}}%
\pgfpathlineto{\pgfqpoint{3.689444in}{2.115041in}}%
\pgfpathlineto{\pgfqpoint{3.697444in}{2.125343in}}%
\pgfpathlineto{\pgfqpoint{3.683906in}{2.127588in}}%
\pgfpathlineto{\pgfqpoint{3.670373in}{2.130041in}}%
\pgfpathlineto{\pgfqpoint{3.656846in}{2.132703in}}%
\pgfpathlineto{\pgfqpoint{3.643323in}{2.135575in}}%
\pgfpathlineto{\pgfqpoint{3.635313in}{2.125349in}}%
\pgfpathlineto{\pgfqpoint{3.627298in}{2.115138in}}%
\pgfpathlineto{\pgfqpoint{3.619277in}{2.104943in}}%
\pgfpathlineto{\pgfqpoint{3.611250in}{2.094763in}}%
\pgfpathclose%
\pgfusepath{fill}%
\end{pgfscope}%
\begin{pgfscope}%
\pgfpathrectangle{\pgfqpoint{1.150000in}{0.150000in}}{\pgfqpoint{5.700000in}{5.700000in}}%
\pgfusepath{clip}%
\pgfsetbuttcap%
\pgfsetroundjoin%
\definecolor{currentfill}{rgb}{0.150476,0.504369,0.557430}%
\pgfsetfillcolor{currentfill}%
\pgfsetfillopacity{0.800000}%
\pgfsetlinewidth{0.000000pt}%
\definecolor{currentstroke}{rgb}{0.000000,0.000000,0.000000}%
\pgfsetstrokecolor{currentstroke}%
\pgfsetdash{}{0pt}%
\pgfpathmoveto{\pgfqpoint{5.450197in}{3.162255in}}%
\pgfpathlineto{\pgfqpoint{5.464436in}{3.169563in}}%
\pgfpathlineto{\pgfqpoint{5.478692in}{3.177045in}}%
\pgfpathlineto{\pgfqpoint{5.492965in}{3.184701in}}%
\pgfpathlineto{\pgfqpoint{5.507254in}{3.192531in}}%
\pgfpathlineto{\pgfqpoint{5.514560in}{3.198168in}}%
\pgfpathlineto{\pgfqpoint{5.521864in}{3.203927in}}%
\pgfpathlineto{\pgfqpoint{5.529166in}{3.209814in}}%
\pgfpathlineto{\pgfqpoint{5.536465in}{3.215836in}}%
\pgfpathlineto{\pgfqpoint{5.522202in}{3.208634in}}%
\pgfpathlineto{\pgfqpoint{5.507956in}{3.201604in}}%
\pgfpathlineto{\pgfqpoint{5.493727in}{3.194748in}}%
\pgfpathlineto{\pgfqpoint{5.479514in}{3.188064in}}%
\pgfpathlineto{\pgfqpoint{5.472188in}{3.181405in}}%
\pgfpathlineto{\pgfqpoint{5.464859in}{3.174888in}}%
\pgfpathlineto{\pgfqpoint{5.457529in}{3.168507in}}%
\pgfpathlineto{\pgfqpoint{5.450197in}{3.162255in}}%
\pgfpathclose%
\pgfusepath{fill}%
\end{pgfscope}%
\begin{pgfscope}%
\pgfpathrectangle{\pgfqpoint{1.150000in}{0.150000in}}{\pgfqpoint{5.700000in}{5.700000in}}%
\pgfusepath{clip}%
\pgfsetbuttcap%
\pgfsetroundjoin%
\definecolor{currentfill}{rgb}{0.229739,0.322361,0.545706}%
\pgfsetfillcolor{currentfill}%
\pgfsetfillopacity{0.800000}%
\pgfsetlinewidth{0.000000pt}%
\definecolor{currentstroke}{rgb}{0.000000,0.000000,0.000000}%
\pgfsetstrokecolor{currentstroke}%
\pgfsetdash{}{0pt}%
\pgfpathmoveto{\pgfqpoint{4.644155in}{2.641821in}}%
\pgfpathlineto{\pgfqpoint{4.658028in}{2.647162in}}%
\pgfpathlineto{\pgfqpoint{4.671915in}{2.652686in}}%
\pgfpathlineto{\pgfqpoint{4.685814in}{2.658393in}}%
\pgfpathlineto{\pgfqpoint{4.699727in}{2.664284in}}%
\pgfpathlineto{\pgfqpoint{4.707388in}{2.672454in}}%
\pgfpathlineto{\pgfqpoint{4.715043in}{2.680608in}}%
\pgfpathlineto{\pgfqpoint{4.722692in}{2.688749in}}%
\pgfpathlineto{\pgfqpoint{4.730336in}{2.696881in}}%
\pgfpathlineto{\pgfqpoint{4.716435in}{2.691289in}}%
\pgfpathlineto{\pgfqpoint{4.702546in}{2.685880in}}%
\pgfpathlineto{\pgfqpoint{4.688671in}{2.680654in}}%
\pgfpathlineto{\pgfqpoint{4.674809in}{2.675611in}}%
\pgfpathlineto{\pgfqpoint{4.667154in}{2.667169in}}%
\pgfpathlineto{\pgfqpoint{4.659493in}{2.658726in}}%
\pgfpathlineto{\pgfqpoint{4.651827in}{2.650278in}}%
\pgfpathlineto{\pgfqpoint{4.644155in}{2.641821in}}%
\pgfpathclose%
\pgfusepath{fill}%
\end{pgfscope}%
\begin{pgfscope}%
\pgfpathrectangle{\pgfqpoint{1.150000in}{0.150000in}}{\pgfqpoint{5.700000in}{5.700000in}}%
\pgfusepath{clip}%
\pgfsetbuttcap%
\pgfsetroundjoin%
\definecolor{currentfill}{rgb}{0.281924,0.089666,0.412415}%
\pgfsetfillcolor{currentfill}%
\pgfsetfillopacity{0.800000}%
\pgfsetlinewidth{0.000000pt}%
\definecolor{currentstroke}{rgb}{0.000000,0.000000,0.000000}%
\pgfsetstrokecolor{currentstroke}%
\pgfsetdash{}{0pt}%
\pgfpathmoveto{\pgfqpoint{3.102712in}{2.125801in}}%
\pgfpathlineto{\pgfqpoint{3.116256in}{2.115846in}}%
\pgfpathlineto{\pgfqpoint{3.129798in}{2.106136in}}%
\pgfpathlineto{\pgfqpoint{3.143340in}{2.096667in}}%
\pgfpathlineto{\pgfqpoint{3.156880in}{2.087440in}}%
\pgfpathlineto{\pgfqpoint{3.165093in}{2.095876in}}%
\pgfpathlineto{\pgfqpoint{3.173299in}{2.104395in}}%
\pgfpathlineto{\pgfqpoint{3.181497in}{2.112994in}}%
\pgfpathlineto{\pgfqpoint{3.189689in}{2.121671in}}%
\pgfpathlineto{\pgfqpoint{3.176167in}{2.130682in}}%
\pgfpathlineto{\pgfqpoint{3.162645in}{2.139934in}}%
\pgfpathlineto{\pgfqpoint{3.149122in}{2.149428in}}%
\pgfpathlineto{\pgfqpoint{3.135598in}{2.159165in}}%
\pgfpathlineto{\pgfqpoint{3.127388in}{2.150693in}}%
\pgfpathlineto{\pgfqpoint{3.119170in}{2.142307in}}%
\pgfpathlineto{\pgfqpoint{3.110945in}{2.134009in}}%
\pgfpathlineto{\pgfqpoint{3.102712in}{2.125801in}}%
\pgfpathclose%
\pgfusepath{fill}%
\end{pgfscope}%
\begin{pgfscope}%
\pgfpathrectangle{\pgfqpoint{1.150000in}{0.150000in}}{\pgfqpoint{5.700000in}{5.700000in}}%
\pgfusepath{clip}%
\pgfsetbuttcap%
\pgfsetroundjoin%
\definecolor{currentfill}{rgb}{0.143343,0.522773,0.556295}%
\pgfsetfillcolor{currentfill}%
\pgfsetfillopacity{0.800000}%
\pgfsetlinewidth{0.000000pt}%
\definecolor{currentstroke}{rgb}{0.000000,0.000000,0.000000}%
\pgfsetstrokecolor{currentstroke}%
\pgfsetdash{}{0pt}%
\pgfpathmoveto{\pgfqpoint{5.536465in}{3.215836in}}%
\pgfpathlineto{\pgfqpoint{5.550745in}{3.223212in}}%
\pgfpathlineto{\pgfqpoint{5.565041in}{3.230760in}}%
\pgfpathlineto{\pgfqpoint{5.579355in}{3.238482in}}%
\pgfpathlineto{\pgfqpoint{5.593686in}{3.246377in}}%
\pgfpathlineto{\pgfqpoint{5.600955in}{3.251893in}}%
\pgfpathlineto{\pgfqpoint{5.608223in}{3.257552in}}%
\pgfpathlineto{\pgfqpoint{5.615489in}{3.263360in}}%
\pgfpathlineto{\pgfqpoint{5.622754in}{3.269324in}}%
\pgfpathlineto{\pgfqpoint{5.608452in}{3.262090in}}%
\pgfpathlineto{\pgfqpoint{5.594167in}{3.255027in}}%
\pgfpathlineto{\pgfqpoint{5.579899in}{3.248137in}}%
\pgfpathlineto{\pgfqpoint{5.565647in}{3.241419in}}%
\pgfpathlineto{\pgfqpoint{5.558354in}{3.234785in}}%
\pgfpathlineto{\pgfqpoint{5.551059in}{3.228314in}}%
\pgfpathlineto{\pgfqpoint{5.543763in}{3.222001in}}%
\pgfpathlineto{\pgfqpoint{5.536465in}{3.215836in}}%
\pgfpathclose%
\pgfusepath{fill}%
\end{pgfscope}%
\begin{pgfscope}%
\pgfpathrectangle{\pgfqpoint{1.150000in}{0.150000in}}{\pgfqpoint{5.700000in}{5.700000in}}%
\pgfusepath{clip}%
\pgfsetbuttcap%
\pgfsetroundjoin%
\definecolor{currentfill}{rgb}{0.270595,0.214069,0.507052}%
\pgfsetfillcolor{currentfill}%
\pgfsetfillopacity{0.800000}%
\pgfsetlinewidth{0.000000pt}%
\definecolor{currentstroke}{rgb}{0.000000,0.000000,0.000000}%
\pgfsetstrokecolor{currentstroke}%
\pgfsetdash{}{0pt}%
\pgfpathmoveto{\pgfqpoint{2.743031in}{2.416671in}}%
\pgfpathlineto{\pgfqpoint{2.756709in}{2.400052in}}%
\pgfpathlineto{\pgfqpoint{2.770378in}{2.383725in}}%
\pgfpathlineto{\pgfqpoint{2.784040in}{2.367688in}}%
\pgfpathlineto{\pgfqpoint{2.797695in}{2.351938in}}%
\pgfpathlineto{\pgfqpoint{2.806081in}{2.358446in}}%
\pgfpathlineto{\pgfqpoint{2.814458in}{2.365093in}}%
\pgfpathlineto{\pgfqpoint{2.822824in}{2.371875in}}%
\pgfpathlineto{\pgfqpoint{2.831180in}{2.378792in}}%
\pgfpathlineto{\pgfqpoint{2.817552in}{2.394286in}}%
\pgfpathlineto{\pgfqpoint{2.803917in}{2.410066in}}%
\pgfpathlineto{\pgfqpoint{2.790275in}{2.426136in}}%
\pgfpathlineto{\pgfqpoint{2.776626in}{2.442498in}}%
\pgfpathlineto{\pgfqpoint{2.768243in}{2.435825in}}%
\pgfpathlineto{\pgfqpoint{2.759849in}{2.429295in}}%
\pgfpathlineto{\pgfqpoint{2.751446in}{2.422910in}}%
\pgfpathlineto{\pgfqpoint{2.743031in}{2.416671in}}%
\pgfpathclose%
\pgfusepath{fill}%
\end{pgfscope}%
\begin{pgfscope}%
\pgfpathrectangle{\pgfqpoint{1.150000in}{0.150000in}}{\pgfqpoint{5.700000in}{5.700000in}}%
\pgfusepath{clip}%
\pgfsetbuttcap%
\pgfsetroundjoin%
\definecolor{currentfill}{rgb}{0.277134,0.185228,0.489898}%
\pgfsetfillcolor{currentfill}%
\pgfsetfillopacity{0.800000}%
\pgfsetlinewidth{0.000000pt}%
\definecolor{currentstroke}{rgb}{0.000000,0.000000,0.000000}%
\pgfsetstrokecolor{currentstroke}%
\pgfsetdash{}{0pt}%
\pgfpathmoveto{\pgfqpoint{2.797695in}{2.351938in}}%
\pgfpathlineto{\pgfqpoint{2.811343in}{2.336473in}}%
\pgfpathlineto{\pgfqpoint{2.824985in}{2.321291in}}%
\pgfpathlineto{\pgfqpoint{2.838620in}{2.306388in}}%
\pgfpathlineto{\pgfqpoint{2.852250in}{2.291764in}}%
\pgfpathlineto{\pgfqpoint{2.860609in}{2.298539in}}%
\pgfpathlineto{\pgfqpoint{2.868959in}{2.305444in}}%
\pgfpathlineto{\pgfqpoint{2.877299in}{2.312477in}}%
\pgfpathlineto{\pgfqpoint{2.885630in}{2.319636in}}%
\pgfpathlineto{\pgfqpoint{2.872027in}{2.334007in}}%
\pgfpathlineto{\pgfqpoint{2.858417in}{2.348654in}}%
\pgfpathlineto{\pgfqpoint{2.844802in}{2.363582in}}%
\pgfpathlineto{\pgfqpoint{2.831180in}{2.378792in}}%
\pgfpathlineto{\pgfqpoint{2.822824in}{2.371875in}}%
\pgfpathlineto{\pgfqpoint{2.814458in}{2.365093in}}%
\pgfpathlineto{\pgfqpoint{2.806081in}{2.358446in}}%
\pgfpathlineto{\pgfqpoint{2.797695in}{2.351938in}}%
\pgfpathclose%
\pgfusepath{fill}%
\end{pgfscope}%
\begin{pgfscope}%
\pgfpathrectangle{\pgfqpoint{1.150000in}{0.150000in}}{\pgfqpoint{5.700000in}{5.700000in}}%
\pgfusepath{clip}%
\pgfsetbuttcap%
\pgfsetroundjoin%
\definecolor{currentfill}{rgb}{0.220057,0.343307,0.549413}%
\pgfsetfillcolor{currentfill}%
\pgfsetfillopacity{0.800000}%
\pgfsetlinewidth{0.000000pt}%
\definecolor{currentstroke}{rgb}{0.000000,0.000000,0.000000}%
\pgfsetstrokecolor{currentstroke}%
\pgfsetdash{}{0pt}%
\pgfpathmoveto{\pgfqpoint{4.730336in}{2.696881in}}%
\pgfpathlineto{\pgfqpoint{4.744251in}{2.702656in}}%
\pgfpathlineto{\pgfqpoint{4.758180in}{2.708613in}}%
\pgfpathlineto{\pgfqpoint{4.772122in}{2.714752in}}%
\pgfpathlineto{\pgfqpoint{4.786078in}{2.721074in}}%
\pgfpathlineto{\pgfqpoint{4.793705in}{2.728881in}}%
\pgfpathlineto{\pgfqpoint{4.801326in}{2.736679in}}%
\pgfpathlineto{\pgfqpoint{4.808941in}{2.744471in}}%
\pgfpathlineto{\pgfqpoint{4.816550in}{2.752261in}}%
\pgfpathlineto{\pgfqpoint{4.802606in}{2.746271in}}%
\pgfpathlineto{\pgfqpoint{4.788676in}{2.740463in}}%
\pgfpathlineto{\pgfqpoint{4.774760in}{2.734837in}}%
\pgfpathlineto{\pgfqpoint{4.760857in}{2.729393in}}%
\pgfpathlineto{\pgfqpoint{4.753235in}{2.721260in}}%
\pgfpathlineto{\pgfqpoint{4.745607in}{2.713133in}}%
\pgfpathlineto{\pgfqpoint{4.737975in}{2.705008in}}%
\pgfpathlineto{\pgfqpoint{4.730336in}{2.696881in}}%
\pgfpathclose%
\pgfusepath{fill}%
\end{pgfscope}%
\begin{pgfscope}%
\pgfpathrectangle{\pgfqpoint{1.150000in}{0.150000in}}{\pgfqpoint{5.700000in}{5.700000in}}%
\pgfusepath{clip}%
\pgfsetbuttcap%
\pgfsetroundjoin%
\definecolor{currentfill}{rgb}{0.136408,0.541173,0.554483}%
\pgfsetfillcolor{currentfill}%
\pgfsetfillopacity{0.800000}%
\pgfsetlinewidth{0.000000pt}%
\definecolor{currentstroke}{rgb}{0.000000,0.000000,0.000000}%
\pgfsetstrokecolor{currentstroke}%
\pgfsetdash{}{0pt}%
\pgfpathmoveto{\pgfqpoint{5.622754in}{3.269324in}}%
\pgfpathlineto{\pgfqpoint{5.637073in}{3.276731in}}%
\pgfpathlineto{\pgfqpoint{5.651409in}{3.284310in}}%
\pgfpathlineto{\pgfqpoint{5.665763in}{3.292061in}}%
\pgfpathlineto{\pgfqpoint{5.680134in}{3.299985in}}%
\pgfpathlineto{\pgfqpoint{5.687368in}{3.305433in}}%
\pgfpathlineto{\pgfqpoint{5.694601in}{3.311045in}}%
\pgfpathlineto{\pgfqpoint{5.701833in}{3.316828in}}%
\pgfpathlineto{\pgfqpoint{5.709065in}{3.322791in}}%
\pgfpathlineto{\pgfqpoint{5.694725in}{3.315561in}}%
\pgfpathlineto{\pgfqpoint{5.680403in}{3.308502in}}%
\pgfpathlineto{\pgfqpoint{5.666097in}{3.301614in}}%
\pgfpathlineto{\pgfqpoint{5.651808in}{3.294897in}}%
\pgfpathlineto{\pgfqpoint{5.644545in}{3.288232in}}%
\pgfpathlineto{\pgfqpoint{5.637282in}{3.281753in}}%
\pgfpathlineto{\pgfqpoint{5.630018in}{3.275453in}}%
\pgfpathlineto{\pgfqpoint{5.622754in}{3.269324in}}%
\pgfpathclose%
\pgfusepath{fill}%
\end{pgfscope}%
\begin{pgfscope}%
\pgfpathrectangle{\pgfqpoint{1.150000in}{0.150000in}}{\pgfqpoint{5.700000in}{5.700000in}}%
\pgfusepath{clip}%
\pgfsetbuttcap%
\pgfsetroundjoin%
\definecolor{currentfill}{rgb}{0.262138,0.242286,0.520837}%
\pgfsetfillcolor{currentfill}%
\pgfsetfillopacity{0.800000}%
\pgfsetlinewidth{0.000000pt}%
\definecolor{currentstroke}{rgb}{0.000000,0.000000,0.000000}%
\pgfsetstrokecolor{currentstroke}%
\pgfsetdash{}{0pt}%
\pgfpathmoveto{\pgfqpoint{2.688240in}{2.486116in}}%
\pgfpathlineto{\pgfqpoint{2.701951in}{2.468304in}}%
\pgfpathlineto{\pgfqpoint{2.715653in}{2.450794in}}%
\pgfpathlineto{\pgfqpoint{2.729346in}{2.433584in}}%
\pgfpathlineto{\pgfqpoint{2.743031in}{2.416671in}}%
\pgfpathlineto{\pgfqpoint{2.751446in}{2.422910in}}%
\pgfpathlineto{\pgfqpoint{2.759849in}{2.429295in}}%
\pgfpathlineto{\pgfqpoint{2.768243in}{2.435825in}}%
\pgfpathlineto{\pgfqpoint{2.776626in}{2.442498in}}%
\pgfpathlineto{\pgfqpoint{2.762969in}{2.459153in}}%
\pgfpathlineto{\pgfqpoint{2.749304in}{2.476105in}}%
\pgfpathlineto{\pgfqpoint{2.735630in}{2.493356in}}%
\pgfpathlineto{\pgfqpoint{2.721949in}{2.510909in}}%
\pgfpathlineto{\pgfqpoint{2.713538in}{2.504483in}}%
\pgfpathlineto{\pgfqpoint{2.705116in}{2.498207in}}%
\pgfpathlineto{\pgfqpoint{2.696684in}{2.492084in}}%
\pgfpathlineto{\pgfqpoint{2.688240in}{2.486116in}}%
\pgfpathclose%
\pgfusepath{fill}%
\end{pgfscope}%
\begin{pgfscope}%
\pgfpathrectangle{\pgfqpoint{1.150000in}{0.150000in}}{\pgfqpoint{5.700000in}{5.700000in}}%
\pgfusepath{clip}%
\pgfsetbuttcap%
\pgfsetroundjoin%
\definecolor{currentfill}{rgb}{0.279566,0.067836,0.391917}%
\pgfsetfillcolor{currentfill}%
\pgfsetfillopacity{0.800000}%
\pgfsetlinewidth{0.000000pt}%
\definecolor{currentstroke}{rgb}{0.000000,0.000000,0.000000}%
\pgfsetstrokecolor{currentstroke}%
\pgfsetdash{}{0pt}%
\pgfpathmoveto{\pgfqpoint{3.524958in}{2.068647in}}%
\pgfpathlineto{\pgfqpoint{3.538485in}{2.064719in}}%
\pgfpathlineto{\pgfqpoint{3.552016in}{2.061004in}}%
\pgfpathlineto{\pgfqpoint{3.565550in}{2.057503in}}%
\pgfpathlineto{\pgfqpoint{3.579090in}{2.054213in}}%
\pgfpathlineto{\pgfqpoint{3.587138in}{2.064323in}}%
\pgfpathlineto{\pgfqpoint{3.595181in}{2.074452in}}%
\pgfpathlineto{\pgfqpoint{3.603218in}{2.084599in}}%
\pgfpathlineto{\pgfqpoint{3.611250in}{2.094763in}}%
\pgfpathlineto{\pgfqpoint{3.597722in}{2.097933in}}%
\pgfpathlineto{\pgfqpoint{3.584198in}{2.101316in}}%
\pgfpathlineto{\pgfqpoint{3.570679in}{2.104911in}}%
\pgfpathlineto{\pgfqpoint{3.557164in}{2.108720in}}%
\pgfpathlineto{\pgfqpoint{3.549121in}{2.098663in}}%
\pgfpathlineto{\pgfqpoint{3.541072in}{2.088632in}}%
\pgfpathlineto{\pgfqpoint{3.533018in}{2.078626in}}%
\pgfpathlineto{\pgfqpoint{3.524958in}{2.068647in}}%
\pgfpathclose%
\pgfusepath{fill}%
\end{pgfscope}%
\begin{pgfscope}%
\pgfpathrectangle{\pgfqpoint{1.150000in}{0.150000in}}{\pgfqpoint{5.700000in}{5.700000in}}%
\pgfusepath{clip}%
\pgfsetbuttcap%
\pgfsetroundjoin%
\definecolor{currentfill}{rgb}{0.280868,0.160771,0.472899}%
\pgfsetfillcolor{currentfill}%
\pgfsetfillopacity{0.800000}%
\pgfsetlinewidth{0.000000pt}%
\definecolor{currentstroke}{rgb}{0.000000,0.000000,0.000000}%
\pgfsetstrokecolor{currentstroke}%
\pgfsetdash{}{0pt}%
\pgfpathmoveto{\pgfqpoint{2.852250in}{2.291764in}}%
\pgfpathlineto{\pgfqpoint{2.865873in}{2.277415in}}%
\pgfpathlineto{\pgfqpoint{2.879491in}{2.263340in}}%
\pgfpathlineto{\pgfqpoint{2.893104in}{2.249537in}}%
\pgfpathlineto{\pgfqpoint{2.906711in}{2.236003in}}%
\pgfpathlineto{\pgfqpoint{2.915045in}{2.243043in}}%
\pgfpathlineto{\pgfqpoint{2.923370in}{2.250206in}}%
\pgfpathlineto{\pgfqpoint{2.931686in}{2.257488in}}%
\pgfpathlineto{\pgfqpoint{2.939993in}{2.264889in}}%
\pgfpathlineto{\pgfqpoint{2.926409in}{2.278170in}}%
\pgfpathlineto{\pgfqpoint{2.912821in}{2.291720in}}%
\pgfpathlineto{\pgfqpoint{2.899228in}{2.305542in}}%
\pgfpathlineto{\pgfqpoint{2.885630in}{2.319636in}}%
\pgfpathlineto{\pgfqpoint{2.877299in}{2.312477in}}%
\pgfpathlineto{\pgfqpoint{2.868959in}{2.305444in}}%
\pgfpathlineto{\pgfqpoint{2.860609in}{2.298539in}}%
\pgfpathlineto{\pgfqpoint{2.852250in}{2.291764in}}%
\pgfpathclose%
\pgfusepath{fill}%
\end{pgfscope}%
\begin{pgfscope}%
\pgfpathrectangle{\pgfqpoint{1.150000in}{0.150000in}}{\pgfqpoint{5.700000in}{5.700000in}}%
\pgfusepath{clip}%
\pgfsetbuttcap%
\pgfsetroundjoin%
\definecolor{currentfill}{rgb}{0.129933,0.559582,0.551864}%
\pgfsetfillcolor{currentfill}%
\pgfsetfillopacity{0.800000}%
\pgfsetlinewidth{0.000000pt}%
\definecolor{currentstroke}{rgb}{0.000000,0.000000,0.000000}%
\pgfsetstrokecolor{currentstroke}%
\pgfsetdash{}{0pt}%
\pgfpathmoveto{\pgfqpoint{5.709065in}{3.322791in}}%
\pgfpathlineto{\pgfqpoint{5.723423in}{3.330193in}}%
\pgfpathlineto{\pgfqpoint{5.737798in}{3.337767in}}%
\pgfpathlineto{\pgfqpoint{5.752190in}{3.345511in}}%
\pgfpathlineto{\pgfqpoint{5.766601in}{3.353427in}}%
\pgfpathlineto{\pgfqpoint{5.773801in}{3.358864in}}%
\pgfpathlineto{\pgfqpoint{5.781001in}{3.364489in}}%
\pgfpathlineto{\pgfqpoint{5.788201in}{3.370309in}}%
\pgfpathlineto{\pgfqpoint{5.795402in}{3.376333in}}%
\pgfpathlineto{\pgfqpoint{5.781025in}{3.369142in}}%
\pgfpathlineto{\pgfqpoint{5.766666in}{3.362122in}}%
\pgfpathlineto{\pgfqpoint{5.752324in}{3.355272in}}%
\pgfpathlineto{\pgfqpoint{5.737999in}{3.348593in}}%
\pgfpathlineto{\pgfqpoint{5.730764in}{3.341835in}}%
\pgfpathlineto{\pgfqpoint{5.723530in}{3.335287in}}%
\pgfpathlineto{\pgfqpoint{5.716298in}{3.328942in}}%
\pgfpathlineto{\pgfqpoint{5.709065in}{3.322791in}}%
\pgfpathclose%
\pgfusepath{fill}%
\end{pgfscope}%
\begin{pgfscope}%
\pgfpathrectangle{\pgfqpoint{1.150000in}{0.150000in}}{\pgfqpoint{5.700000in}{5.700000in}}%
\pgfusepath{clip}%
\pgfsetbuttcap%
\pgfsetroundjoin%
\definecolor{currentfill}{rgb}{0.208623,0.367752,0.552675}%
\pgfsetfillcolor{currentfill}%
\pgfsetfillopacity{0.800000}%
\pgfsetlinewidth{0.000000pt}%
\definecolor{currentstroke}{rgb}{0.000000,0.000000,0.000000}%
\pgfsetstrokecolor{currentstroke}%
\pgfsetdash{}{0pt}%
\pgfpathmoveto{\pgfqpoint{4.816550in}{2.752261in}}%
\pgfpathlineto{\pgfqpoint{4.830508in}{2.758432in}}%
\pgfpathlineto{\pgfqpoint{4.844480in}{2.764785in}}%
\pgfpathlineto{\pgfqpoint{4.858466in}{2.771318in}}%
\pgfpathlineto{\pgfqpoint{4.872467in}{2.778033in}}%
\pgfpathlineto{\pgfqpoint{4.880058in}{2.785473in}}%
\pgfpathlineto{\pgfqpoint{4.887643in}{2.792912in}}%
\pgfpathlineto{\pgfqpoint{4.895222in}{2.800353in}}%
\pgfpathlineto{\pgfqpoint{4.902797in}{2.807802in}}%
\pgfpathlineto{\pgfqpoint{4.888810in}{2.801452in}}%
\pgfpathlineto{\pgfqpoint{4.874837in}{2.795283in}}%
\pgfpathlineto{\pgfqpoint{4.860879in}{2.789295in}}%
\pgfpathlineto{\pgfqpoint{4.846934in}{2.783487in}}%
\pgfpathlineto{\pgfqpoint{4.839346in}{2.775662in}}%
\pgfpathlineto{\pgfqpoint{4.831753in}{2.767853in}}%
\pgfpathlineto{\pgfqpoint{4.824154in}{2.760054in}}%
\pgfpathlineto{\pgfqpoint{4.816550in}{2.752261in}}%
\pgfpathclose%
\pgfusepath{fill}%
\end{pgfscope}%
\begin{pgfscope}%
\pgfpathrectangle{\pgfqpoint{1.150000in}{0.150000in}}{\pgfqpoint{5.700000in}{5.700000in}}%
\pgfusepath{clip}%
\pgfsetbuttcap%
\pgfsetroundjoin%
\definecolor{currentfill}{rgb}{0.250425,0.274290,0.533103}%
\pgfsetfillcolor{currentfill}%
\pgfsetfillopacity{0.800000}%
\pgfsetlinewidth{0.000000pt}%
\definecolor{currentstroke}{rgb}{0.000000,0.000000,0.000000}%
\pgfsetstrokecolor{currentstroke}%
\pgfsetdash{}{0pt}%
\pgfpathmoveto{\pgfqpoint{2.633303in}{2.560442in}}%
\pgfpathlineto{\pgfqpoint{2.647052in}{2.541393in}}%
\pgfpathlineto{\pgfqpoint{2.660791in}{2.522658in}}%
\pgfpathlineto{\pgfqpoint{2.674520in}{2.504233in}}%
\pgfpathlineto{\pgfqpoint{2.688240in}{2.486116in}}%
\pgfpathlineto{\pgfqpoint{2.696684in}{2.492084in}}%
\pgfpathlineto{\pgfqpoint{2.705116in}{2.498207in}}%
\pgfpathlineto{\pgfqpoint{2.713538in}{2.504483in}}%
\pgfpathlineto{\pgfqpoint{2.721949in}{2.510909in}}%
\pgfpathlineto{\pgfqpoint{2.708258in}{2.528767in}}%
\pgfpathlineto{\pgfqpoint{2.694559in}{2.546931in}}%
\pgfpathlineto{\pgfqpoint{2.680850in}{2.565406in}}%
\pgfpathlineto{\pgfqpoint{2.667131in}{2.584193in}}%
\pgfpathlineto{\pgfqpoint{2.658691in}{2.578015in}}%
\pgfpathlineto{\pgfqpoint{2.650240in}{2.571996in}}%
\pgfpathlineto{\pgfqpoint{2.641777in}{2.566137in}}%
\pgfpathlineto{\pgfqpoint{2.633303in}{2.560442in}}%
\pgfpathclose%
\pgfusepath{fill}%
\end{pgfscope}%
\begin{pgfscope}%
\pgfpathrectangle{\pgfqpoint{1.150000in}{0.150000in}}{\pgfqpoint{5.700000in}{5.700000in}}%
\pgfusepath{clip}%
\pgfsetbuttcap%
\pgfsetroundjoin%
\definecolor{currentfill}{rgb}{0.278791,0.062145,0.386592}%
\pgfsetfillcolor{currentfill}%
\pgfsetfillopacity{0.800000}%
\pgfsetlinewidth{0.000000pt}%
\definecolor{currentstroke}{rgb}{0.000000,0.000000,0.000000}%
\pgfsetstrokecolor{currentstroke}%
\pgfsetdash{}{0pt}%
\pgfpathmoveto{\pgfqpoint{3.297857in}{2.058072in}}%
\pgfpathlineto{\pgfqpoint{3.311381in}{2.051163in}}%
\pgfpathlineto{\pgfqpoint{3.324906in}{2.044481in}}%
\pgfpathlineto{\pgfqpoint{3.338433in}{2.038025in}}%
\pgfpathlineto{\pgfqpoint{3.351961in}{2.031793in}}%
\pgfpathlineto{\pgfqpoint{3.360096in}{2.041134in}}%
\pgfpathlineto{\pgfqpoint{3.368224in}{2.050526in}}%
\pgfpathlineto{\pgfqpoint{3.376346in}{2.059968in}}%
\pgfpathlineto{\pgfqpoint{3.384462in}{2.069459in}}%
\pgfpathlineto{\pgfqpoint{3.370949in}{2.075508in}}%
\pgfpathlineto{\pgfqpoint{3.357437in}{2.081781in}}%
\pgfpathlineto{\pgfqpoint{3.343927in}{2.088280in}}%
\pgfpathlineto{\pgfqpoint{3.330419in}{2.095006in}}%
\pgfpathlineto{\pgfqpoint{3.322288in}{2.085686in}}%
\pgfpathlineto{\pgfqpoint{3.314151in}{2.076423in}}%
\pgfpathlineto{\pgfqpoint{3.306008in}{2.067218in}}%
\pgfpathlineto{\pgfqpoint{3.297857in}{2.058072in}}%
\pgfpathclose%
\pgfusepath{fill}%
\end{pgfscope}%
\begin{pgfscope}%
\pgfpathrectangle{\pgfqpoint{1.150000in}{0.150000in}}{\pgfqpoint{5.700000in}{5.700000in}}%
\pgfusepath{clip}%
\pgfsetbuttcap%
\pgfsetroundjoin%
\definecolor{currentfill}{rgb}{0.282623,0.140926,0.457517}%
\pgfsetfillcolor{currentfill}%
\pgfsetfillopacity{0.800000}%
\pgfsetlinewidth{0.000000pt}%
\definecolor{currentstroke}{rgb}{0.000000,0.000000,0.000000}%
\pgfsetstrokecolor{currentstroke}%
\pgfsetdash{}{0pt}%
\pgfpathmoveto{\pgfqpoint{2.906711in}{2.236003in}}%
\pgfpathlineto{\pgfqpoint{2.920314in}{2.222736in}}%
\pgfpathlineto{\pgfqpoint{2.933913in}{2.209735in}}%
\pgfpathlineto{\pgfqpoint{2.947507in}{2.196998in}}%
\pgfpathlineto{\pgfqpoint{2.961097in}{2.184521in}}%
\pgfpathlineto{\pgfqpoint{2.969407in}{2.191826in}}%
\pgfpathlineto{\pgfqpoint{2.977708in}{2.199244in}}%
\pgfpathlineto{\pgfqpoint{2.986000in}{2.206774in}}%
\pgfpathlineto{\pgfqpoint{2.994283in}{2.214414in}}%
\pgfpathlineto{\pgfqpoint{2.980717in}{2.226639in}}%
\pgfpathlineto{\pgfqpoint{2.967146in}{2.239125in}}%
\pgfpathlineto{\pgfqpoint{2.953571in}{2.251874in}}%
\pgfpathlineto{\pgfqpoint{2.939993in}{2.264889in}}%
\pgfpathlineto{\pgfqpoint{2.931686in}{2.257488in}}%
\pgfpathlineto{\pgfqpoint{2.923370in}{2.250206in}}%
\pgfpathlineto{\pgfqpoint{2.915045in}{2.243043in}}%
\pgfpathlineto{\pgfqpoint{2.906711in}{2.236003in}}%
\pgfpathclose%
\pgfusepath{fill}%
\end{pgfscope}%
\begin{pgfscope}%
\pgfpathrectangle{\pgfqpoint{1.150000in}{0.150000in}}{\pgfqpoint{5.700000in}{5.700000in}}%
\pgfusepath{clip}%
\pgfsetbuttcap%
\pgfsetroundjoin%
\definecolor{currentfill}{rgb}{0.124395,0.578002,0.548287}%
\pgfsetfillcolor{currentfill}%
\pgfsetfillopacity{0.800000}%
\pgfsetlinewidth{0.000000pt}%
\definecolor{currentstroke}{rgb}{0.000000,0.000000,0.000000}%
\pgfsetstrokecolor{currentstroke}%
\pgfsetdash{}{0pt}%
\pgfpathmoveto{\pgfqpoint{5.795402in}{3.376333in}}%
\pgfpathlineto{\pgfqpoint{5.809797in}{3.383694in}}%
\pgfpathlineto{\pgfqpoint{5.824210in}{3.391226in}}%
\pgfpathlineto{\pgfqpoint{5.838640in}{3.398928in}}%
\pgfpathlineto{\pgfqpoint{5.853088in}{3.406801in}}%
\pgfpathlineto{\pgfqpoint{5.860256in}{3.412291in}}%
\pgfpathlineto{\pgfqpoint{5.867425in}{3.417993in}}%
\pgfpathlineto{\pgfqpoint{5.874596in}{3.423916in}}%
\pgfpathlineto{\pgfqpoint{5.881769in}{3.430068in}}%
\pgfpathlineto{\pgfqpoint{5.867357in}{3.422954in}}%
\pgfpathlineto{\pgfqpoint{5.852962in}{3.416008in}}%
\pgfpathlineto{\pgfqpoint{5.838584in}{3.409233in}}%
\pgfpathlineto{\pgfqpoint{5.824224in}{3.402627in}}%
\pgfpathlineto{\pgfqpoint{5.817016in}{3.395707in}}%
\pgfpathlineto{\pgfqpoint{5.809809in}{3.389024in}}%
\pgfpathlineto{\pgfqpoint{5.802605in}{3.382568in}}%
\pgfpathlineto{\pgfqpoint{5.795402in}{3.376333in}}%
\pgfpathclose%
\pgfusepath{fill}%
\end{pgfscope}%
\begin{pgfscope}%
\pgfpathrectangle{\pgfqpoint{1.150000in}{0.150000in}}{\pgfqpoint{5.700000in}{5.700000in}}%
\pgfusepath{clip}%
\pgfsetbuttcap%
\pgfsetroundjoin%
\definecolor{currentfill}{rgb}{0.199430,0.387607,0.554642}%
\pgfsetfillcolor{currentfill}%
\pgfsetfillopacity{0.800000}%
\pgfsetlinewidth{0.000000pt}%
\definecolor{currentstroke}{rgb}{0.000000,0.000000,0.000000}%
\pgfsetstrokecolor{currentstroke}%
\pgfsetdash{}{0pt}%
\pgfpathmoveto{\pgfqpoint{4.902797in}{2.807802in}}%
\pgfpathlineto{\pgfqpoint{4.916798in}{2.814332in}}%
\pgfpathlineto{\pgfqpoint{4.930814in}{2.821043in}}%
\pgfpathlineto{\pgfqpoint{4.944844in}{2.827933in}}%
\pgfpathlineto{\pgfqpoint{4.958890in}{2.835003in}}%
\pgfpathlineto{\pgfqpoint{4.966444in}{2.842078in}}%
\pgfpathlineto{\pgfqpoint{4.973993in}{2.849160in}}%
\pgfpathlineto{\pgfqpoint{4.981536in}{2.856256in}}%
\pgfpathlineto{\pgfqpoint{4.989074in}{2.863368in}}%
\pgfpathlineto{\pgfqpoint{4.975043in}{2.856696in}}%
\pgfpathlineto{\pgfqpoint{4.961028in}{2.850203in}}%
\pgfpathlineto{\pgfqpoint{4.947027in}{2.843890in}}%
\pgfpathlineto{\pgfqpoint{4.933040in}{2.837756in}}%
\pgfpathlineto{\pgfqpoint{4.925487in}{2.830235in}}%
\pgfpathlineto{\pgfqpoint{4.917929in}{2.822738in}}%
\pgfpathlineto{\pgfqpoint{4.910365in}{2.815262in}}%
\pgfpathlineto{\pgfqpoint{4.902797in}{2.807802in}}%
\pgfpathclose%
\pgfusepath{fill}%
\end{pgfscope}%
\begin{pgfscope}%
\pgfpathrectangle{\pgfqpoint{1.150000in}{0.150000in}}{\pgfqpoint{5.700000in}{5.700000in}}%
\pgfusepath{clip}%
\pgfsetbuttcap%
\pgfsetroundjoin%
\definecolor{currentfill}{rgb}{0.280868,0.160771,0.472899}%
\pgfsetfillcolor{currentfill}%
\pgfsetfillopacity{0.800000}%
\pgfsetlinewidth{0.000000pt}%
\definecolor{currentstroke}{rgb}{0.000000,0.000000,0.000000}%
\pgfsetstrokecolor{currentstroke}%
\pgfsetdash{}{0pt}%
\pgfpathmoveto{\pgfqpoint{4.010115in}{2.242428in}}%
\pgfpathlineto{\pgfqpoint{4.023748in}{2.243682in}}%
\pgfpathlineto{\pgfqpoint{4.037390in}{2.245132in}}%
\pgfpathlineto{\pgfqpoint{4.051040in}{2.246777in}}%
\pgfpathlineto{\pgfqpoint{4.064699in}{2.248617in}}%
\pgfpathlineto{\pgfqpoint{4.072592in}{2.258876in}}%
\pgfpathlineto{\pgfqpoint{4.080480in}{2.269107in}}%
\pgfpathlineto{\pgfqpoint{4.088362in}{2.279309in}}%
\pgfpathlineto{\pgfqpoint{4.096240in}{2.289483in}}%
\pgfpathlineto{\pgfqpoint{4.082587in}{2.287683in}}%
\pgfpathlineto{\pgfqpoint{4.068944in}{2.286077in}}%
\pgfpathlineto{\pgfqpoint{4.055310in}{2.284666in}}%
\pgfpathlineto{\pgfqpoint{4.041685in}{2.283452in}}%
\pgfpathlineto{\pgfqpoint{4.033800in}{2.273226in}}%
\pgfpathlineto{\pgfqpoint{4.025910in}{2.262981in}}%
\pgfpathlineto{\pgfqpoint{4.018015in}{2.252715in}}%
\pgfpathlineto{\pgfqpoint{4.010115in}{2.242428in}}%
\pgfpathclose%
\pgfusepath{fill}%
\end{pgfscope}%
\begin{pgfscope}%
\pgfpathrectangle{\pgfqpoint{1.150000in}{0.150000in}}{\pgfqpoint{5.700000in}{5.700000in}}%
\pgfusepath{clip}%
\pgfsetbuttcap%
\pgfsetroundjoin%
\definecolor{currentfill}{rgb}{0.282884,0.135920,0.453427}%
\pgfsetfillcolor{currentfill}%
\pgfsetfillopacity{0.800000}%
\pgfsetlinewidth{0.000000pt}%
\definecolor{currentstroke}{rgb}{0.000000,0.000000,0.000000}%
\pgfsetstrokecolor{currentstroke}%
\pgfsetdash{}{0pt}%
\pgfpathmoveto{\pgfqpoint{3.923986in}{2.197974in}}%
\pgfpathlineto{\pgfqpoint{3.937593in}{2.198446in}}%
\pgfpathlineto{\pgfqpoint{3.951209in}{2.199117in}}%
\pgfpathlineto{\pgfqpoint{3.964832in}{2.199986in}}%
\pgfpathlineto{\pgfqpoint{3.978464in}{2.201051in}}%
\pgfpathlineto{\pgfqpoint{3.986385in}{2.211431in}}%
\pgfpathlineto{\pgfqpoint{3.994300in}{2.221786in}}%
\pgfpathlineto{\pgfqpoint{4.002210in}{2.232118in}}%
\pgfpathlineto{\pgfqpoint{4.010115in}{2.242428in}}%
\pgfpathlineto{\pgfqpoint{3.996491in}{2.241370in}}%
\pgfpathlineto{\pgfqpoint{3.982875in}{2.240509in}}%
\pgfpathlineto{\pgfqpoint{3.969267in}{2.239846in}}%
\pgfpathlineto{\pgfqpoint{3.955667in}{2.239380in}}%
\pgfpathlineto{\pgfqpoint{3.947755in}{2.229051in}}%
\pgfpathlineto{\pgfqpoint{3.939837in}{2.218708in}}%
\pgfpathlineto{\pgfqpoint{3.931914in}{2.208349in}}%
\pgfpathlineto{\pgfqpoint{3.923986in}{2.197974in}}%
\pgfpathclose%
\pgfusepath{fill}%
\end{pgfscope}%
\begin{pgfscope}%
\pgfpathrectangle{\pgfqpoint{1.150000in}{0.150000in}}{\pgfqpoint{5.700000in}{5.700000in}}%
\pgfusepath{clip}%
\pgfsetbuttcap%
\pgfsetroundjoin%
\definecolor{currentfill}{rgb}{0.280894,0.078907,0.402329}%
\pgfsetfillcolor{currentfill}%
\pgfsetfillopacity{0.800000}%
\pgfsetlinewidth{0.000000pt}%
\definecolor{currentstroke}{rgb}{0.000000,0.000000,0.000000}%
\pgfsetstrokecolor{currentstroke}%
\pgfsetdash{}{0pt}%
\pgfpathmoveto{\pgfqpoint{3.156880in}{2.087440in}}%
\pgfpathlineto{\pgfqpoint{3.170419in}{2.078452in}}%
\pgfpathlineto{\pgfqpoint{3.183958in}{2.069702in}}%
\pgfpathlineto{\pgfqpoint{3.197496in}{2.061188in}}%
\pgfpathlineto{\pgfqpoint{3.211034in}{2.052909in}}%
\pgfpathlineto{\pgfqpoint{3.219229in}{2.061573in}}%
\pgfpathlineto{\pgfqpoint{3.227416in}{2.070312in}}%
\pgfpathlineto{\pgfqpoint{3.235597in}{2.079123in}}%
\pgfpathlineto{\pgfqpoint{3.243771in}{2.088004in}}%
\pgfpathlineto{\pgfqpoint{3.230250in}{2.096067in}}%
\pgfpathlineto{\pgfqpoint{3.216730in}{2.104365in}}%
\pgfpathlineto{\pgfqpoint{3.203209in}{2.112899in}}%
\pgfpathlineto{\pgfqpoint{3.189689in}{2.121671in}}%
\pgfpathlineto{\pgfqpoint{3.181497in}{2.112994in}}%
\pgfpathlineto{\pgfqpoint{3.173299in}{2.104395in}}%
\pgfpathlineto{\pgfqpoint{3.165093in}{2.095876in}}%
\pgfpathlineto{\pgfqpoint{3.156880in}{2.087440in}}%
\pgfpathclose%
\pgfusepath{fill}%
\end{pgfscope}%
\begin{pgfscope}%
\pgfpathrectangle{\pgfqpoint{1.150000in}{0.150000in}}{\pgfqpoint{5.700000in}{5.700000in}}%
\pgfusepath{clip}%
\pgfsetbuttcap%
\pgfsetroundjoin%
\definecolor{currentfill}{rgb}{0.278012,0.180367,0.486697}%
\pgfsetfillcolor{currentfill}%
\pgfsetfillopacity{0.800000}%
\pgfsetlinewidth{0.000000pt}%
\definecolor{currentstroke}{rgb}{0.000000,0.000000,0.000000}%
\pgfsetstrokecolor{currentstroke}%
\pgfsetdash{}{0pt}%
\pgfpathmoveto{\pgfqpoint{4.096240in}{2.289483in}}%
\pgfpathlineto{\pgfqpoint{4.109901in}{2.291477in}}%
\pgfpathlineto{\pgfqpoint{4.123572in}{2.293666in}}%
\pgfpathlineto{\pgfqpoint{4.137252in}{2.296047in}}%
\pgfpathlineto{\pgfqpoint{4.150943in}{2.298621in}}%
\pgfpathlineto{\pgfqpoint{4.158808in}{2.308709in}}%
\pgfpathlineto{\pgfqpoint{4.166667in}{2.318764in}}%
\pgfpathlineto{\pgfqpoint{4.174522in}{2.328787in}}%
\pgfpathlineto{\pgfqpoint{4.182371in}{2.338779in}}%
\pgfpathlineto{\pgfqpoint{4.168689in}{2.336277in}}%
\pgfpathlineto{\pgfqpoint{4.155015in}{2.333967in}}%
\pgfpathlineto{\pgfqpoint{4.141352in}{2.331850in}}%
\pgfpathlineto{\pgfqpoint{4.127698in}{2.329927in}}%
\pgfpathlineto{\pgfqpoint{4.119841in}{2.319852in}}%
\pgfpathlineto{\pgfqpoint{4.111979in}{2.309753in}}%
\pgfpathlineto{\pgfqpoint{4.104112in}{2.299631in}}%
\pgfpathlineto{\pgfqpoint{4.096240in}{2.289483in}}%
\pgfpathclose%
\pgfusepath{fill}%
\end{pgfscope}%
\begin{pgfscope}%
\pgfpathrectangle{\pgfqpoint{1.150000in}{0.150000in}}{\pgfqpoint{5.700000in}{5.700000in}}%
\pgfusepath{clip}%
\pgfsetbuttcap%
\pgfsetroundjoin%
\definecolor{currentfill}{rgb}{0.283229,0.120777,0.440584}%
\pgfsetfillcolor{currentfill}%
\pgfsetfillopacity{0.800000}%
\pgfsetlinewidth{0.000000pt}%
\definecolor{currentstroke}{rgb}{0.000000,0.000000,0.000000}%
\pgfsetstrokecolor{currentstroke}%
\pgfsetdash{}{0pt}%
\pgfpathmoveto{\pgfqpoint{3.837838in}{2.156505in}}%
\pgfpathlineto{\pgfqpoint{3.851423in}{2.156155in}}%
\pgfpathlineto{\pgfqpoint{3.865016in}{2.156005in}}%
\pgfpathlineto{\pgfqpoint{3.878616in}{2.156056in}}%
\pgfpathlineto{\pgfqpoint{3.892223in}{2.156306in}}%
\pgfpathlineto{\pgfqpoint{3.900172in}{2.166748in}}%
\pgfpathlineto{\pgfqpoint{3.908115in}{2.177174in}}%
\pgfpathlineto{\pgfqpoint{3.916053in}{2.187582in}}%
\pgfpathlineto{\pgfqpoint{3.923986in}{2.197974in}}%
\pgfpathlineto{\pgfqpoint{3.910386in}{2.197700in}}%
\pgfpathlineto{\pgfqpoint{3.896794in}{2.197625in}}%
\pgfpathlineto{\pgfqpoint{3.883210in}{2.197750in}}%
\pgfpathlineto{\pgfqpoint{3.869633in}{2.198076in}}%
\pgfpathlineto{\pgfqpoint{3.861691in}{2.187697in}}%
\pgfpathlineto{\pgfqpoint{3.853745in}{2.177309in}}%
\pgfpathlineto{\pgfqpoint{3.845794in}{2.166912in}}%
\pgfpathlineto{\pgfqpoint{3.837838in}{2.156505in}}%
\pgfpathclose%
\pgfusepath{fill}%
\end{pgfscope}%
\begin{pgfscope}%
\pgfpathrectangle{\pgfqpoint{1.150000in}{0.150000in}}{\pgfqpoint{5.700000in}{5.700000in}}%
\pgfusepath{clip}%
\pgfsetbuttcap%
\pgfsetroundjoin%
\definecolor{currentfill}{rgb}{0.237441,0.305202,0.541921}%
\pgfsetfillcolor{currentfill}%
\pgfsetfillopacity{0.800000}%
\pgfsetlinewidth{0.000000pt}%
\definecolor{currentstroke}{rgb}{0.000000,0.000000,0.000000}%
\pgfsetstrokecolor{currentstroke}%
\pgfsetdash{}{0pt}%
\pgfpathmoveto{\pgfqpoint{2.578200in}{2.639829in}}%
\pgfpathlineto{\pgfqpoint{2.591992in}{2.619497in}}%
\pgfpathlineto{\pgfqpoint{2.605773in}{2.599491in}}%
\pgfpathlineto{\pgfqpoint{2.619544in}{2.579807in}}%
\pgfpathlineto{\pgfqpoint{2.633303in}{2.560442in}}%
\pgfpathlineto{\pgfqpoint{2.641777in}{2.566137in}}%
\pgfpathlineto{\pgfqpoint{2.650240in}{2.571996in}}%
\pgfpathlineto{\pgfqpoint{2.658691in}{2.578015in}}%
\pgfpathlineto{\pgfqpoint{2.667131in}{2.584193in}}%
\pgfpathlineto{\pgfqpoint{2.653403in}{2.603296in}}%
\pgfpathlineto{\pgfqpoint{2.639664in}{2.622718in}}%
\pgfpathlineto{\pgfqpoint{2.625914in}{2.642461in}}%
\pgfpathlineto{\pgfqpoint{2.612153in}{2.662529in}}%
\pgfpathlineto{\pgfqpoint{2.603683in}{2.656601in}}%
\pgfpathlineto{\pgfqpoint{2.595200in}{2.650840in}}%
\pgfpathlineto{\pgfqpoint{2.586706in}{2.645249in}}%
\pgfpathlineto{\pgfqpoint{2.578200in}{2.639829in}}%
\pgfpathclose%
\pgfusepath{fill}%
\end{pgfscope}%
\begin{pgfscope}%
\pgfpathrectangle{\pgfqpoint{1.150000in}{0.150000in}}{\pgfqpoint{5.700000in}{5.700000in}}%
\pgfusepath{clip}%
\pgfsetbuttcap%
\pgfsetroundjoin%
\definecolor{currentfill}{rgb}{0.278791,0.062145,0.386592}%
\pgfsetfillcolor{currentfill}%
\pgfsetfillopacity{0.800000}%
\pgfsetlinewidth{0.000000pt}%
\definecolor{currentstroke}{rgb}{0.000000,0.000000,0.000000}%
\pgfsetstrokecolor{currentstroke}%
\pgfsetdash{}{0pt}%
\pgfpathmoveto{\pgfqpoint{3.438539in}{2.047485in}}%
\pgfpathlineto{\pgfqpoint{3.452065in}{2.042541in}}%
\pgfpathlineto{\pgfqpoint{3.465594in}{2.037815in}}%
\pgfpathlineto{\pgfqpoint{3.479126in}{2.033305in}}%
\pgfpathlineto{\pgfqpoint{3.492661in}{2.029012in}}%
\pgfpathlineto{\pgfqpoint{3.500744in}{2.038876in}}%
\pgfpathlineto{\pgfqpoint{3.508821in}{2.048771in}}%
\pgfpathlineto{\pgfqpoint{3.516892in}{2.058695in}}%
\pgfpathlineto{\pgfqpoint{3.524958in}{2.068647in}}%
\pgfpathlineto{\pgfqpoint{3.511435in}{2.072790in}}%
\pgfpathlineto{\pgfqpoint{3.497915in}{2.077148in}}%
\pgfpathlineto{\pgfqpoint{3.484399in}{2.081723in}}%
\pgfpathlineto{\pgfqpoint{3.470886in}{2.086516in}}%
\pgfpathlineto{\pgfqpoint{3.462808in}{2.076703in}}%
\pgfpathlineto{\pgfqpoint{3.454725in}{2.066926in}}%
\pgfpathlineto{\pgfqpoint{3.446635in}{2.057186in}}%
\pgfpathlineto{\pgfqpoint{3.438539in}{2.047485in}}%
\pgfpathclose%
\pgfusepath{fill}%
\end{pgfscope}%
\begin{pgfscope}%
\pgfpathrectangle{\pgfqpoint{1.150000in}{0.150000in}}{\pgfqpoint{5.700000in}{5.700000in}}%
\pgfusepath{clip}%
\pgfsetbuttcap%
\pgfsetroundjoin%
\definecolor{currentfill}{rgb}{0.273006,0.204520,0.501721}%
\pgfsetfillcolor{currentfill}%
\pgfsetfillopacity{0.800000}%
\pgfsetlinewidth{0.000000pt}%
\definecolor{currentstroke}{rgb}{0.000000,0.000000,0.000000}%
\pgfsetstrokecolor{currentstroke}%
\pgfsetdash{}{0pt}%
\pgfpathmoveto{\pgfqpoint{4.182371in}{2.338779in}}%
\pgfpathlineto{\pgfqpoint{4.196064in}{2.341474in}}%
\pgfpathlineto{\pgfqpoint{4.209767in}{2.344360in}}%
\pgfpathlineto{\pgfqpoint{4.223481in}{2.347437in}}%
\pgfpathlineto{\pgfqpoint{4.237204in}{2.350705in}}%
\pgfpathlineto{\pgfqpoint{4.245041in}{2.360576in}}%
\pgfpathlineto{\pgfqpoint{4.252873in}{2.370411in}}%
\pgfpathlineto{\pgfqpoint{4.260699in}{2.380211in}}%
\pgfpathlineto{\pgfqpoint{4.268521in}{2.389978in}}%
\pgfpathlineto{\pgfqpoint{4.254804in}{2.386814in}}%
\pgfpathlineto{\pgfqpoint{4.241098in}{2.383840in}}%
\pgfpathlineto{\pgfqpoint{4.227403in}{2.381057in}}%
\pgfpathlineto{\pgfqpoint{4.213717in}{2.378466in}}%
\pgfpathlineto{\pgfqpoint{4.205889in}{2.368584in}}%
\pgfpathlineto{\pgfqpoint{4.198055in}{2.358676in}}%
\pgfpathlineto{\pgfqpoint{4.190216in}{2.348742in}}%
\pgfpathlineto{\pgfqpoint{4.182371in}{2.338779in}}%
\pgfpathclose%
\pgfusepath{fill}%
\end{pgfscope}%
\begin{pgfscope}%
\pgfpathrectangle{\pgfqpoint{1.150000in}{0.150000in}}{\pgfqpoint{5.700000in}{5.700000in}}%
\pgfusepath{clip}%
\pgfsetbuttcap%
\pgfsetroundjoin%
\definecolor{currentfill}{rgb}{0.190631,0.407061,0.556089}%
\pgfsetfillcolor{currentfill}%
\pgfsetfillopacity{0.800000}%
\pgfsetlinewidth{0.000000pt}%
\definecolor{currentstroke}{rgb}{0.000000,0.000000,0.000000}%
\pgfsetstrokecolor{currentstroke}%
\pgfsetdash{}{0pt}%
\pgfpathmoveto{\pgfqpoint{4.989074in}{2.863368in}}%
\pgfpathlineto{\pgfqpoint{5.003119in}{2.870220in}}%
\pgfpathlineto{\pgfqpoint{5.017179in}{2.877251in}}%
\pgfpathlineto{\pgfqpoint{5.031254in}{2.884460in}}%
\pgfpathlineto{\pgfqpoint{5.045345in}{2.891849in}}%
\pgfpathlineto{\pgfqpoint{5.052861in}{2.898565in}}%
\pgfpathlineto{\pgfqpoint{5.060373in}{2.905300in}}%
\pgfpathlineto{\pgfqpoint{5.067878in}{2.912058in}}%
\pgfpathlineto{\pgfqpoint{5.075379in}{2.918846in}}%
\pgfpathlineto{\pgfqpoint{5.061305in}{2.911888in}}%
\pgfpathlineto{\pgfqpoint{5.047246in}{2.905109in}}%
\pgfpathlineto{\pgfqpoint{5.033203in}{2.898509in}}%
\pgfpathlineto{\pgfqpoint{5.019174in}{2.892087in}}%
\pgfpathlineto{\pgfqpoint{5.011656in}{2.884857in}}%
\pgfpathlineto{\pgfqpoint{5.004134in}{2.877664in}}%
\pgfpathlineto{\pgfqpoint{4.996606in}{2.870503in}}%
\pgfpathlineto{\pgfqpoint{4.989074in}{2.863368in}}%
\pgfpathclose%
\pgfusepath{fill}%
\end{pgfscope}%
\begin{pgfscope}%
\pgfpathrectangle{\pgfqpoint{1.150000in}{0.150000in}}{\pgfqpoint{5.700000in}{5.700000in}}%
\pgfusepath{clip}%
\pgfsetbuttcap%
\pgfsetroundjoin%
\definecolor{currentfill}{rgb}{0.282656,0.100196,0.422160}%
\pgfsetfillcolor{currentfill}%
\pgfsetfillopacity{0.800000}%
\pgfsetlinewidth{0.000000pt}%
\definecolor{currentstroke}{rgb}{0.000000,0.000000,0.000000}%
\pgfsetstrokecolor{currentstroke}%
\pgfsetdash{}{0pt}%
\pgfpathmoveto{\pgfqpoint{3.751653in}{2.118430in}}%
\pgfpathlineto{\pgfqpoint{3.765220in}{2.117214in}}%
\pgfpathlineto{\pgfqpoint{3.778794in}{2.116201in}}%
\pgfpathlineto{\pgfqpoint{3.792374in}{2.115392in}}%
\pgfpathlineto{\pgfqpoint{3.805961in}{2.114784in}}%
\pgfpathlineto{\pgfqpoint{3.813938in}{2.125228in}}%
\pgfpathlineto{\pgfqpoint{3.821910in}{2.135663in}}%
\pgfpathlineto{\pgfqpoint{3.829876in}{2.146089in}}%
\pgfpathlineto{\pgfqpoint{3.837838in}{2.156505in}}%
\pgfpathlineto{\pgfqpoint{3.824259in}{2.157057in}}%
\pgfpathlineto{\pgfqpoint{3.810688in}{2.157810in}}%
\pgfpathlineto{\pgfqpoint{3.797123in}{2.158767in}}%
\pgfpathlineto{\pgfqpoint{3.783564in}{2.159927in}}%
\pgfpathlineto{\pgfqpoint{3.775594in}{2.149555in}}%
\pgfpathlineto{\pgfqpoint{3.767619in}{2.139181in}}%
\pgfpathlineto{\pgfqpoint{3.759639in}{2.128806in}}%
\pgfpathlineto{\pgfqpoint{3.751653in}{2.118430in}}%
\pgfpathclose%
\pgfusepath{fill}%
\end{pgfscope}%
\begin{pgfscope}%
\pgfpathrectangle{\pgfqpoint{1.150000in}{0.150000in}}{\pgfqpoint{5.700000in}{5.700000in}}%
\pgfusepath{clip}%
\pgfsetbuttcap%
\pgfsetroundjoin%
\definecolor{currentfill}{rgb}{0.283229,0.120777,0.440584}%
\pgfsetfillcolor{currentfill}%
\pgfsetfillopacity{0.800000}%
\pgfsetlinewidth{0.000000pt}%
\definecolor{currentstroke}{rgb}{0.000000,0.000000,0.000000}%
\pgfsetstrokecolor{currentstroke}%
\pgfsetdash{}{0pt}%
\pgfpathmoveto{\pgfqpoint{2.961097in}{2.184521in}}%
\pgfpathlineto{\pgfqpoint{2.974683in}{2.172305in}}%
\pgfpathlineto{\pgfqpoint{2.988266in}{2.160346in}}%
\pgfpathlineto{\pgfqpoint{3.001846in}{2.148643in}}%
\pgfpathlineto{\pgfqpoint{3.015422in}{2.137195in}}%
\pgfpathlineto{\pgfqpoint{3.023709in}{2.144762in}}%
\pgfpathlineto{\pgfqpoint{3.031987in}{2.152434in}}%
\pgfpathlineto{\pgfqpoint{3.040257in}{2.160211in}}%
\pgfpathlineto{\pgfqpoint{3.048518in}{2.168090in}}%
\pgfpathlineto{\pgfqpoint{3.034964in}{2.179288in}}%
\pgfpathlineto{\pgfqpoint{3.021407in}{2.190741in}}%
\pgfpathlineto{\pgfqpoint{3.007847in}{2.202449in}}%
\pgfpathlineto{\pgfqpoint{2.994283in}{2.214414in}}%
\pgfpathlineto{\pgfqpoint{2.986000in}{2.206774in}}%
\pgfpathlineto{\pgfqpoint{2.977708in}{2.199244in}}%
\pgfpathlineto{\pgfqpoint{2.969407in}{2.191826in}}%
\pgfpathlineto{\pgfqpoint{2.961097in}{2.184521in}}%
\pgfpathclose%
\pgfusepath{fill}%
\end{pgfscope}%
\begin{pgfscope}%
\pgfpathrectangle{\pgfqpoint{1.150000in}{0.150000in}}{\pgfqpoint{5.700000in}{5.700000in}}%
\pgfusepath{clip}%
\pgfsetbuttcap%
\pgfsetroundjoin%
\definecolor{currentfill}{rgb}{0.266580,0.228262,0.514349}%
\pgfsetfillcolor{currentfill}%
\pgfsetfillopacity{0.800000}%
\pgfsetlinewidth{0.000000pt}%
\definecolor{currentstroke}{rgb}{0.000000,0.000000,0.000000}%
\pgfsetstrokecolor{currentstroke}%
\pgfsetdash{}{0pt}%
\pgfpathmoveto{\pgfqpoint{4.268521in}{2.389978in}}%
\pgfpathlineto{\pgfqpoint{4.282248in}{2.393333in}}%
\pgfpathlineto{\pgfqpoint{4.295985in}{2.396877in}}%
\pgfpathlineto{\pgfqpoint{4.309734in}{2.400612in}}%
\pgfpathlineto{\pgfqpoint{4.323494in}{2.404535in}}%
\pgfpathlineto{\pgfqpoint{4.331302in}{2.414147in}}%
\pgfpathlineto{\pgfqpoint{4.339105in}{2.423721in}}%
\pgfpathlineto{\pgfqpoint{4.346902in}{2.433260in}}%
\pgfpathlineto{\pgfqpoint{4.354694in}{2.442765in}}%
\pgfpathlineto{\pgfqpoint{4.340942in}{2.438978in}}%
\pgfpathlineto{\pgfqpoint{4.327201in}{2.435380in}}%
\pgfpathlineto{\pgfqpoint{4.313471in}{2.431971in}}%
\pgfpathlineto{\pgfqpoint{4.299752in}{2.428752in}}%
\pgfpathlineto{\pgfqpoint{4.291952in}{2.419099in}}%
\pgfpathlineto{\pgfqpoint{4.284147in}{2.409421in}}%
\pgfpathlineto{\pgfqpoint{4.276336in}{2.399714in}}%
\pgfpathlineto{\pgfqpoint{4.268521in}{2.389978in}}%
\pgfpathclose%
\pgfusepath{fill}%
\end{pgfscope}%
\begin{pgfscope}%
\pgfpathrectangle{\pgfqpoint{1.150000in}{0.150000in}}{\pgfqpoint{5.700000in}{5.700000in}}%
\pgfusepath{clip}%
\pgfsetbuttcap%
\pgfsetroundjoin%
\definecolor{currentfill}{rgb}{0.120565,0.596422,0.543611}%
\pgfsetfillcolor{currentfill}%
\pgfsetfillopacity{0.800000}%
\pgfsetlinewidth{0.000000pt}%
\definecolor{currentstroke}{rgb}{0.000000,0.000000,0.000000}%
\pgfsetstrokecolor{currentstroke}%
\pgfsetdash{}{0pt}%
\pgfpathmoveto{\pgfqpoint{5.881769in}{3.430068in}}%
\pgfpathlineto{\pgfqpoint{5.896200in}{3.437353in}}%
\pgfpathlineto{\pgfqpoint{5.910648in}{3.444808in}}%
\pgfpathlineto{\pgfqpoint{5.925115in}{3.452432in}}%
\pgfpathlineto{\pgfqpoint{5.939600in}{3.460226in}}%
\pgfpathlineto{\pgfqpoint{5.946738in}{3.465838in}}%
\pgfpathlineto{\pgfqpoint{5.953880in}{3.471689in}}%
\pgfpathlineto{\pgfqpoint{5.961024in}{3.477787in}}%
\pgfpathlineto{\pgfqpoint{5.946568in}{3.470583in}}%
\pgfpathlineto{\pgfqpoint{5.932129in}{3.463549in}}%
\pgfpathlineto{\pgfqpoint{5.917709in}{3.456683in}}%
\pgfpathlineto{\pgfqpoint{5.903306in}{3.449986in}}%
\pgfpathlineto{\pgfqpoint{5.896124in}{3.443095in}}%
\pgfpathlineto{\pgfqpoint{5.888945in}{3.436459in}}%
\pgfpathlineto{\pgfqpoint{5.881769in}{3.430068in}}%
\pgfpathclose%
\pgfusepath{fill}%
\end{pgfscope}%
\begin{pgfscope}%
\pgfpathrectangle{\pgfqpoint{1.150000in}{0.150000in}}{\pgfqpoint{5.700000in}{5.700000in}}%
\pgfusepath{clip}%
\pgfsetbuttcap%
\pgfsetroundjoin%
\definecolor{currentfill}{rgb}{0.182256,0.426184,0.557120}%
\pgfsetfillcolor{currentfill}%
\pgfsetfillopacity{0.800000}%
\pgfsetlinewidth{0.000000pt}%
\definecolor{currentstroke}{rgb}{0.000000,0.000000,0.000000}%
\pgfsetstrokecolor{currentstroke}%
\pgfsetdash{}{0pt}%
\pgfpathmoveto{\pgfqpoint{5.075379in}{2.918846in}}%
\pgfpathlineto{\pgfqpoint{5.089468in}{2.925981in}}%
\pgfpathlineto{\pgfqpoint{5.103573in}{2.933295in}}%
\pgfpathlineto{\pgfqpoint{5.117694in}{2.940788in}}%
\pgfpathlineto{\pgfqpoint{5.131830in}{2.948458in}}%
\pgfpathlineto{\pgfqpoint{5.139308in}{2.954827in}}%
\pgfpathlineto{\pgfqpoint{5.146780in}{2.961228in}}%
\pgfpathlineto{\pgfqpoint{5.154248in}{2.967665in}}%
\pgfpathlineto{\pgfqpoint{5.161711in}{2.974144in}}%
\pgfpathlineto{\pgfqpoint{5.147593in}{2.966938in}}%
\pgfpathlineto{\pgfqpoint{5.133491in}{2.959910in}}%
\pgfpathlineto{\pgfqpoint{5.119404in}{2.953060in}}%
\pgfpathlineto{\pgfqpoint{5.105333in}{2.946387in}}%
\pgfpathlineto{\pgfqpoint{5.097852in}{2.939432in}}%
\pgfpathlineto{\pgfqpoint{5.090366in}{2.932528in}}%
\pgfpathlineto{\pgfqpoint{5.082875in}{2.925667in}}%
\pgfpathlineto{\pgfqpoint{5.075379in}{2.918846in}}%
\pgfpathclose%
\pgfusepath{fill}%
\end{pgfscope}%
\begin{pgfscope}%
\pgfpathrectangle{\pgfqpoint{1.150000in}{0.150000in}}{\pgfqpoint{5.700000in}{5.700000in}}%
\pgfusepath{clip}%
\pgfsetbuttcap%
\pgfsetroundjoin%
\definecolor{currentfill}{rgb}{0.258965,0.251537,0.524736}%
\pgfsetfillcolor{currentfill}%
\pgfsetfillopacity{0.800000}%
\pgfsetlinewidth{0.000000pt}%
\definecolor{currentstroke}{rgb}{0.000000,0.000000,0.000000}%
\pgfsetstrokecolor{currentstroke}%
\pgfsetdash{}{0pt}%
\pgfpathmoveto{\pgfqpoint{4.354694in}{2.442765in}}%
\pgfpathlineto{\pgfqpoint{4.368458in}{2.446741in}}%
\pgfpathlineto{\pgfqpoint{4.382232in}{2.450905in}}%
\pgfpathlineto{\pgfqpoint{4.396019in}{2.455257in}}%
\pgfpathlineto{\pgfqpoint{4.409817in}{2.459796in}}%
\pgfpathlineto{\pgfqpoint{4.417595in}{2.469113in}}%
\pgfpathlineto{\pgfqpoint{4.425369in}{2.478393in}}%
\pgfpathlineto{\pgfqpoint{4.433137in}{2.487637in}}%
\pgfpathlineto{\pgfqpoint{4.440899in}{2.496848in}}%
\pgfpathlineto{\pgfqpoint{4.427109in}{2.492478in}}%
\pgfpathlineto{\pgfqpoint{4.413331in}{2.488294in}}%
\pgfpathlineto{\pgfqpoint{4.399564in}{2.484299in}}%
\pgfpathlineto{\pgfqpoint{4.385809in}{2.480491in}}%
\pgfpathlineto{\pgfqpoint{4.378038in}{2.471100in}}%
\pgfpathlineto{\pgfqpoint{4.370263in}{2.461683in}}%
\pgfpathlineto{\pgfqpoint{4.362481in}{2.452239in}}%
\pgfpathlineto{\pgfqpoint{4.354694in}{2.442765in}}%
\pgfpathclose%
\pgfusepath{fill}%
\end{pgfscope}%
\begin{pgfscope}%
\pgfpathrectangle{\pgfqpoint{1.150000in}{0.150000in}}{\pgfqpoint{5.700000in}{5.700000in}}%
\pgfusepath{clip}%
\pgfsetbuttcap%
\pgfsetroundjoin%
\definecolor{currentfill}{rgb}{0.281446,0.084320,0.407414}%
\pgfsetfillcolor{currentfill}%
\pgfsetfillopacity{0.800000}%
\pgfsetlinewidth{0.000000pt}%
\definecolor{currentstroke}{rgb}{0.000000,0.000000,0.000000}%
\pgfsetstrokecolor{currentstroke}%
\pgfsetdash{}{0pt}%
\pgfpathmoveto{\pgfqpoint{3.665411in}{2.084180in}}%
\pgfpathlineto{\pgfqpoint{3.678965in}{2.082055in}}%
\pgfpathlineto{\pgfqpoint{3.692524in}{2.080136in}}%
\pgfpathlineto{\pgfqpoint{3.706088in}{2.078423in}}%
\pgfpathlineto{\pgfqpoint{3.719659in}{2.076916in}}%
\pgfpathlineto{\pgfqpoint{3.727665in}{2.087295in}}%
\pgfpathlineto{\pgfqpoint{3.735666in}{2.097674in}}%
\pgfpathlineto{\pgfqpoint{3.743662in}{2.108052in}}%
\pgfpathlineto{\pgfqpoint{3.751653in}{2.118430in}}%
\pgfpathlineto{\pgfqpoint{3.738092in}{2.119850in}}%
\pgfpathlineto{\pgfqpoint{3.724537in}{2.121475in}}%
\pgfpathlineto{\pgfqpoint{3.710988in}{2.123306in}}%
\pgfpathlineto{\pgfqpoint{3.697444in}{2.125343in}}%
\pgfpathlineto{\pgfqpoint{3.689444in}{2.115041in}}%
\pgfpathlineto{\pgfqpoint{3.681438in}{2.104747in}}%
\pgfpathlineto{\pgfqpoint{3.673428in}{2.094459in}}%
\pgfpathlineto{\pgfqpoint{3.665411in}{2.084180in}}%
\pgfpathclose%
\pgfusepath{fill}%
\end{pgfscope}%
\begin{pgfscope}%
\pgfpathrectangle{\pgfqpoint{1.150000in}{0.150000in}}{\pgfqpoint{5.700000in}{5.700000in}}%
\pgfusepath{clip}%
\pgfsetbuttcap%
\pgfsetroundjoin%
\definecolor{currentfill}{rgb}{0.221989,0.339161,0.548752}%
\pgfsetfillcolor{currentfill}%
\pgfsetfillopacity{0.800000}%
\pgfsetlinewidth{0.000000pt}%
\definecolor{currentstroke}{rgb}{0.000000,0.000000,0.000000}%
\pgfsetstrokecolor{currentstroke}%
\pgfsetdash{}{0pt}%
\pgfpathmoveto{\pgfqpoint{2.522910in}{2.724470in}}%
\pgfpathlineto{\pgfqpoint{2.536751in}{2.702806in}}%
\pgfpathlineto{\pgfqpoint{2.550580in}{2.681480in}}%
\pgfpathlineto{\pgfqpoint{2.564396in}{2.660489in}}%
\pgfpathlineto{\pgfqpoint{2.578200in}{2.639829in}}%
\pgfpathlineto{\pgfqpoint{2.586706in}{2.645249in}}%
\pgfpathlineto{\pgfqpoint{2.595200in}{2.650840in}}%
\pgfpathlineto{\pgfqpoint{2.603683in}{2.656601in}}%
\pgfpathlineto{\pgfqpoint{2.612153in}{2.662529in}}%
\pgfpathlineto{\pgfqpoint{2.598381in}{2.682924in}}%
\pgfpathlineto{\pgfqpoint{2.584598in}{2.703651in}}%
\pgfpathlineto{\pgfqpoint{2.570802in}{2.724711in}}%
\pgfpathlineto{\pgfqpoint{2.556994in}{2.746109in}}%
\pgfpathlineto{\pgfqpoint{2.548492in}{2.740434in}}%
\pgfpathlineto{\pgfqpoint{2.539977in}{2.734934in}}%
\pgfpathlineto{\pgfqpoint{2.531450in}{2.729612in}}%
\pgfpathlineto{\pgfqpoint{2.522910in}{2.724470in}}%
\pgfpathclose%
\pgfusepath{fill}%
\end{pgfscope}%
\begin{pgfscope}%
\pgfpathrectangle{\pgfqpoint{1.150000in}{0.150000in}}{\pgfqpoint{5.700000in}{5.700000in}}%
\pgfusepath{clip}%
\pgfsetbuttcap%
\pgfsetroundjoin%
\definecolor{currentfill}{rgb}{0.172719,0.448791,0.557885}%
\pgfsetfillcolor{currentfill}%
\pgfsetfillopacity{0.800000}%
\pgfsetlinewidth{0.000000pt}%
\definecolor{currentstroke}{rgb}{0.000000,0.000000,0.000000}%
\pgfsetstrokecolor{currentstroke}%
\pgfsetdash{}{0pt}%
\pgfpathmoveto{\pgfqpoint{5.161711in}{2.974144in}}%
\pgfpathlineto{\pgfqpoint{5.175844in}{2.981527in}}%
\pgfpathlineto{\pgfqpoint{5.189994in}{2.989087in}}%
\pgfpathlineto{\pgfqpoint{5.204159in}{2.996824in}}%
\pgfpathlineto{\pgfqpoint{5.218341in}{3.004739in}}%
\pgfpathlineto{\pgfqpoint{5.225779in}{3.010780in}}%
\pgfpathlineto{\pgfqpoint{5.233213in}{3.016865in}}%
\pgfpathlineto{\pgfqpoint{5.240642in}{3.023002in}}%
\pgfpathlineto{\pgfqpoint{5.248066in}{3.029194in}}%
\pgfpathlineto{\pgfqpoint{5.233905in}{3.021778in}}%
\pgfpathlineto{\pgfqpoint{5.219760in}{3.014538in}}%
\pgfpathlineto{\pgfqpoint{5.205630in}{3.007474in}}%
\pgfpathlineto{\pgfqpoint{5.191516in}{3.000587in}}%
\pgfpathlineto{\pgfqpoint{5.184072in}{2.993886in}}%
\pgfpathlineto{\pgfqpoint{5.176623in}{2.987249in}}%
\pgfpathlineto{\pgfqpoint{5.169169in}{2.980670in}}%
\pgfpathlineto{\pgfqpoint{5.161711in}{2.974144in}}%
\pgfpathclose%
\pgfusepath{fill}%
\end{pgfscope}%
\begin{pgfscope}%
\pgfpathrectangle{\pgfqpoint{1.150000in}{0.150000in}}{\pgfqpoint{5.700000in}{5.700000in}}%
\pgfusepath{clip}%
\pgfsetbuttcap%
\pgfsetroundjoin%
\definecolor{currentfill}{rgb}{0.250425,0.274290,0.533103}%
\pgfsetfillcolor{currentfill}%
\pgfsetfillopacity{0.800000}%
\pgfsetlinewidth{0.000000pt}%
\definecolor{currentstroke}{rgb}{0.000000,0.000000,0.000000}%
\pgfsetstrokecolor{currentstroke}%
\pgfsetdash{}{0pt}%
\pgfpathmoveto{\pgfqpoint{4.440899in}{2.496848in}}%
\pgfpathlineto{\pgfqpoint{4.454700in}{2.501406in}}%
\pgfpathlineto{\pgfqpoint{4.468514in}{2.506151in}}%
\pgfpathlineto{\pgfqpoint{4.482340in}{2.511082in}}%
\pgfpathlineto{\pgfqpoint{4.496178in}{2.516199in}}%
\pgfpathlineto{\pgfqpoint{4.503926in}{2.525190in}}%
\pgfpathlineto{\pgfqpoint{4.511669in}{2.534145in}}%
\pgfpathlineto{\pgfqpoint{4.519406in}{2.543067in}}%
\pgfpathlineto{\pgfqpoint{4.527138in}{2.551957in}}%
\pgfpathlineto{\pgfqpoint{4.513308in}{2.547042in}}%
\pgfpathlineto{\pgfqpoint{4.499491in}{2.542312in}}%
\pgfpathlineto{\pgfqpoint{4.485686in}{2.537768in}}%
\pgfpathlineto{\pgfqpoint{4.471893in}{2.533411in}}%
\pgfpathlineto{\pgfqpoint{4.464153in}{2.524308in}}%
\pgfpathlineto{\pgfqpoint{4.456407in}{2.515181in}}%
\pgfpathlineto{\pgfqpoint{4.448656in}{2.506029in}}%
\pgfpathlineto{\pgfqpoint{4.440899in}{2.496848in}}%
\pgfpathclose%
\pgfusepath{fill}%
\end{pgfscope}%
\begin{pgfscope}%
\pgfpathrectangle{\pgfqpoint{1.150000in}{0.150000in}}{\pgfqpoint{5.700000in}{5.700000in}}%
\pgfusepath{clip}%
\pgfsetbuttcap%
\pgfsetroundjoin%
\definecolor{currentfill}{rgb}{0.282656,0.100196,0.422160}%
\pgfsetfillcolor{currentfill}%
\pgfsetfillopacity{0.800000}%
\pgfsetlinewidth{0.000000pt}%
\definecolor{currentstroke}{rgb}{0.000000,0.000000,0.000000}%
\pgfsetstrokecolor{currentstroke}%
\pgfsetdash{}{0pt}%
\pgfpathmoveto{\pgfqpoint{3.015422in}{2.137195in}}%
\pgfpathlineto{\pgfqpoint{3.028996in}{2.125999in}}%
\pgfpathlineto{\pgfqpoint{3.042567in}{2.115053in}}%
\pgfpathlineto{\pgfqpoint{3.056135in}{2.104357in}}%
\pgfpathlineto{\pgfqpoint{3.069702in}{2.093908in}}%
\pgfpathlineto{\pgfqpoint{3.077967in}{2.101736in}}%
\pgfpathlineto{\pgfqpoint{3.086223in}{2.109662in}}%
\pgfpathlineto{\pgfqpoint{3.094472in}{2.117684in}}%
\pgfpathlineto{\pgfqpoint{3.102712in}{2.125801in}}%
\pgfpathlineto{\pgfqpoint{3.089167in}{2.136001in}}%
\pgfpathlineto{\pgfqpoint{3.075620in}{2.146448in}}%
\pgfpathlineto{\pgfqpoint{3.062070in}{2.157144in}}%
\pgfpathlineto{\pgfqpoint{3.048518in}{2.168090in}}%
\pgfpathlineto{\pgfqpoint{3.040257in}{2.160211in}}%
\pgfpathlineto{\pgfqpoint{3.031987in}{2.152434in}}%
\pgfpathlineto{\pgfqpoint{3.023709in}{2.144762in}}%
\pgfpathlineto{\pgfqpoint{3.015422in}{2.137195in}}%
\pgfpathclose%
\pgfusepath{fill}%
\end{pgfscope}%
\begin{pgfscope}%
\pgfpathrectangle{\pgfqpoint{1.150000in}{0.150000in}}{\pgfqpoint{5.700000in}{5.700000in}}%
\pgfusepath{clip}%
\pgfsetbuttcap%
\pgfsetroundjoin%
\definecolor{currentfill}{rgb}{0.279566,0.067836,0.391917}%
\pgfsetfillcolor{currentfill}%
\pgfsetfillopacity{0.800000}%
\pgfsetlinewidth{0.000000pt}%
\definecolor{currentstroke}{rgb}{0.000000,0.000000,0.000000}%
\pgfsetstrokecolor{currentstroke}%
\pgfsetdash{}{0pt}%
\pgfpathmoveto{\pgfqpoint{3.211034in}{2.052909in}}%
\pgfpathlineto{\pgfqpoint{3.224572in}{2.044864in}}%
\pgfpathlineto{\pgfqpoint{3.238111in}{2.037051in}}%
\pgfpathlineto{\pgfqpoint{3.251649in}{2.029469in}}%
\pgfpathlineto{\pgfqpoint{3.265189in}{2.022117in}}%
\pgfpathlineto{\pgfqpoint{3.273366in}{2.031008in}}%
\pgfpathlineto{\pgfqpoint{3.281537in}{2.039966in}}%
\pgfpathlineto{\pgfqpoint{3.289700in}{2.048988in}}%
\pgfpathlineto{\pgfqpoint{3.297857in}{2.058072in}}%
\pgfpathlineto{\pgfqpoint{3.284334in}{2.065210in}}%
\pgfpathlineto{\pgfqpoint{3.270813in}{2.072577in}}%
\pgfpathlineto{\pgfqpoint{3.257291in}{2.080174in}}%
\pgfpathlineto{\pgfqpoint{3.243771in}{2.088004in}}%
\pgfpathlineto{\pgfqpoint{3.235597in}{2.079123in}}%
\pgfpathlineto{\pgfqpoint{3.227416in}{2.070312in}}%
\pgfpathlineto{\pgfqpoint{3.219229in}{2.061573in}}%
\pgfpathlineto{\pgfqpoint{3.211034in}{2.052909in}}%
\pgfpathclose%
\pgfusepath{fill}%
\end{pgfscope}%
\begin{pgfscope}%
\pgfpathrectangle{\pgfqpoint{1.150000in}{0.150000in}}{\pgfqpoint{5.700000in}{5.700000in}}%
\pgfusepath{clip}%
\pgfsetbuttcap%
\pgfsetroundjoin%
\definecolor{currentfill}{rgb}{0.277941,0.056324,0.381191}%
\pgfsetfillcolor{currentfill}%
\pgfsetfillopacity{0.800000}%
\pgfsetlinewidth{0.000000pt}%
\definecolor{currentstroke}{rgb}{0.000000,0.000000,0.000000}%
\pgfsetstrokecolor{currentstroke}%
\pgfsetdash{}{0pt}%
\pgfpathmoveto{\pgfqpoint{3.351961in}{2.031793in}}%
\pgfpathlineto{\pgfqpoint{3.365491in}{2.025784in}}%
\pgfpathlineto{\pgfqpoint{3.379024in}{2.019998in}}%
\pgfpathlineto{\pgfqpoint{3.392558in}{2.014433in}}%
\pgfpathlineto{\pgfqpoint{3.406095in}{2.009088in}}%
\pgfpathlineto{\pgfqpoint{3.414215in}{2.018623in}}%
\pgfpathlineto{\pgfqpoint{3.422329in}{2.028202in}}%
\pgfpathlineto{\pgfqpoint{3.430437in}{2.037823in}}%
\pgfpathlineto{\pgfqpoint{3.438539in}{2.047485in}}%
\pgfpathlineto{\pgfqpoint{3.425016in}{2.052647in}}%
\pgfpathlineto{\pgfqpoint{3.411496in}{2.058030in}}%
\pgfpathlineto{\pgfqpoint{3.397978in}{2.063633in}}%
\pgfpathlineto{\pgfqpoint{3.384462in}{2.069459in}}%
\pgfpathlineto{\pgfqpoint{3.376346in}{2.059968in}}%
\pgfpathlineto{\pgfqpoint{3.368224in}{2.050526in}}%
\pgfpathlineto{\pgfqpoint{3.360096in}{2.041134in}}%
\pgfpathlineto{\pgfqpoint{3.351961in}{2.031793in}}%
\pgfpathclose%
\pgfusepath{fill}%
\end{pgfscope}%
\begin{pgfscope}%
\pgfpathrectangle{\pgfqpoint{1.150000in}{0.150000in}}{\pgfqpoint{5.700000in}{5.700000in}}%
\pgfusepath{clip}%
\pgfsetbuttcap%
\pgfsetroundjoin%
\definecolor{currentfill}{rgb}{0.280267,0.073417,0.397163}%
\pgfsetfillcolor{currentfill}%
\pgfsetfillopacity{0.800000}%
\pgfsetlinewidth{0.000000pt}%
\definecolor{currentstroke}{rgb}{0.000000,0.000000,0.000000}%
\pgfsetstrokecolor{currentstroke}%
\pgfsetdash{}{0pt}%
\pgfpathmoveto{\pgfqpoint{3.579090in}{2.054213in}}%
\pgfpathlineto{\pgfqpoint{3.592633in}{2.051134in}}%
\pgfpathlineto{\pgfqpoint{3.606182in}{2.048265in}}%
\pgfpathlineto{\pgfqpoint{3.619735in}{2.045605in}}%
\pgfpathlineto{\pgfqpoint{3.633293in}{2.043153in}}%
\pgfpathlineto{\pgfqpoint{3.641331in}{2.053395in}}%
\pgfpathlineto{\pgfqpoint{3.649363in}{2.063647in}}%
\pgfpathlineto{\pgfqpoint{3.657390in}{2.073909in}}%
\pgfpathlineto{\pgfqpoint{3.665411in}{2.084180in}}%
\pgfpathlineto{\pgfqpoint{3.651863in}{2.086512in}}%
\pgfpathlineto{\pgfqpoint{3.638321in}{2.089053in}}%
\pgfpathlineto{\pgfqpoint{3.624783in}{2.091803in}}%
\pgfpathlineto{\pgfqpoint{3.611250in}{2.094763in}}%
\pgfpathlineto{\pgfqpoint{3.603218in}{2.084599in}}%
\pgfpathlineto{\pgfqpoint{3.595181in}{2.074452in}}%
\pgfpathlineto{\pgfqpoint{3.587138in}{2.064323in}}%
\pgfpathlineto{\pgfqpoint{3.579090in}{2.054213in}}%
\pgfpathclose%
\pgfusepath{fill}%
\end{pgfscope}%
\begin{pgfscope}%
\pgfpathrectangle{\pgfqpoint{1.150000in}{0.150000in}}{\pgfqpoint{5.700000in}{5.700000in}}%
\pgfusepath{clip}%
\pgfsetbuttcap%
\pgfsetroundjoin%
\definecolor{currentfill}{rgb}{0.165117,0.467423,0.558141}%
\pgfsetfillcolor{currentfill}%
\pgfsetfillopacity{0.800000}%
\pgfsetlinewidth{0.000000pt}%
\definecolor{currentstroke}{rgb}{0.000000,0.000000,0.000000}%
\pgfsetstrokecolor{currentstroke}%
\pgfsetdash{}{0pt}%
\pgfpathmoveto{\pgfqpoint{5.248066in}{3.029194in}}%
\pgfpathlineto{\pgfqpoint{5.262244in}{3.036787in}}%
\pgfpathlineto{\pgfqpoint{5.276438in}{3.044557in}}%
\pgfpathlineto{\pgfqpoint{5.290648in}{3.052503in}}%
\pgfpathlineto{\pgfqpoint{5.304874in}{3.060625in}}%
\pgfpathlineto{\pgfqpoint{5.312273in}{3.066361in}}%
\pgfpathlineto{\pgfqpoint{5.319668in}{3.072157in}}%
\pgfpathlineto{\pgfqpoint{5.327058in}{3.078019in}}%
\pgfpathlineto{\pgfqpoint{5.334443in}{3.083953in}}%
\pgfpathlineto{\pgfqpoint{5.320239in}{3.076362in}}%
\pgfpathlineto{\pgfqpoint{5.306050in}{3.068947in}}%
\pgfpathlineto{\pgfqpoint{5.291878in}{3.061707in}}%
\pgfpathlineto{\pgfqpoint{5.277722in}{3.054643in}}%
\pgfpathlineto{\pgfqpoint{5.270314in}{3.048167in}}%
\pgfpathlineto{\pgfqpoint{5.262902in}{3.041771in}}%
\pgfpathlineto{\pgfqpoint{5.255487in}{3.035449in}}%
\pgfpathlineto{\pgfqpoint{5.248066in}{3.029194in}}%
\pgfpathclose%
\pgfusepath{fill}%
\end{pgfscope}%
\begin{pgfscope}%
\pgfpathrectangle{\pgfqpoint{1.150000in}{0.150000in}}{\pgfqpoint{5.700000in}{5.700000in}}%
\pgfusepath{clip}%
\pgfsetbuttcap%
\pgfsetroundjoin%
\definecolor{currentfill}{rgb}{0.241237,0.296485,0.539709}%
\pgfsetfillcolor{currentfill}%
\pgfsetfillopacity{0.800000}%
\pgfsetlinewidth{0.000000pt}%
\definecolor{currentstroke}{rgb}{0.000000,0.000000,0.000000}%
\pgfsetstrokecolor{currentstroke}%
\pgfsetdash{}{0pt}%
\pgfpathmoveto{\pgfqpoint{4.527138in}{2.551957in}}%
\pgfpathlineto{\pgfqpoint{4.540979in}{2.557058in}}%
\pgfpathlineto{\pgfqpoint{4.554834in}{2.562345in}}%
\pgfpathlineto{\pgfqpoint{4.568701in}{2.567817in}}%
\pgfpathlineto{\pgfqpoint{4.582581in}{2.573474in}}%
\pgfpathlineto{\pgfqpoint{4.590298in}{2.582114in}}%
\pgfpathlineto{\pgfqpoint{4.598009in}{2.590720in}}%
\pgfpathlineto{\pgfqpoint{4.605714in}{2.599296in}}%
\pgfpathlineto{\pgfqpoint{4.613413in}{2.607844in}}%
\pgfpathlineto{\pgfqpoint{4.599543in}{2.602422in}}%
\pgfpathlineto{\pgfqpoint{4.585685in}{2.597184in}}%
\pgfpathlineto{\pgfqpoint{4.571840in}{2.592131in}}%
\pgfpathlineto{\pgfqpoint{4.558008in}{2.587263in}}%
\pgfpathlineto{\pgfqpoint{4.550299in}{2.578470in}}%
\pgfpathlineto{\pgfqpoint{4.542584in}{2.569656in}}%
\pgfpathlineto{\pgfqpoint{4.534864in}{2.560819in}}%
\pgfpathlineto{\pgfqpoint{4.527138in}{2.551957in}}%
\pgfpathclose%
\pgfusepath{fill}%
\end{pgfscope}%
\begin{pgfscope}%
\pgfpathrectangle{\pgfqpoint{1.150000in}{0.150000in}}{\pgfqpoint{5.700000in}{5.700000in}}%
\pgfusepath{clip}%
\pgfsetbuttcap%
\pgfsetroundjoin%
\definecolor{currentfill}{rgb}{0.157729,0.485932,0.558013}%
\pgfsetfillcolor{currentfill}%
\pgfsetfillopacity{0.800000}%
\pgfsetlinewidth{0.000000pt}%
\definecolor{currentstroke}{rgb}{0.000000,0.000000,0.000000}%
\pgfsetstrokecolor{currentstroke}%
\pgfsetdash{}{0pt}%
\pgfpathmoveto{\pgfqpoint{5.334443in}{3.083953in}}%
\pgfpathlineto{\pgfqpoint{5.348665in}{3.091719in}}%
\pgfpathlineto{\pgfqpoint{5.362902in}{3.099662in}}%
\pgfpathlineto{\pgfqpoint{5.377157in}{3.107779in}}%
\pgfpathlineto{\pgfqpoint{5.391428in}{3.116073in}}%
\pgfpathlineto{\pgfqpoint{5.398787in}{3.121533in}}%
\pgfpathlineto{\pgfqpoint{5.406142in}{3.127069in}}%
\pgfpathlineto{\pgfqpoint{5.413493in}{3.132689in}}%
\pgfpathlineto{\pgfqpoint{5.420840in}{3.138398in}}%
\pgfpathlineto{\pgfqpoint{5.406592in}{3.130669in}}%
\pgfpathlineto{\pgfqpoint{5.392362in}{3.123115in}}%
\pgfpathlineto{\pgfqpoint{5.378147in}{3.115736in}}%
\pgfpathlineto{\pgfqpoint{5.363949in}{3.108532in}}%
\pgfpathlineto{\pgfqpoint{5.356578in}{3.102248in}}%
\pgfpathlineto{\pgfqpoint{5.349204in}{3.096061in}}%
\pgfpathlineto{\pgfqpoint{5.341825in}{3.089965in}}%
\pgfpathlineto{\pgfqpoint{5.334443in}{3.083953in}}%
\pgfpathclose%
\pgfusepath{fill}%
\end{pgfscope}%
\begin{pgfscope}%
\pgfpathrectangle{\pgfqpoint{1.150000in}{0.150000in}}{\pgfqpoint{5.700000in}{5.700000in}}%
\pgfusepath{clip}%
\pgfsetbuttcap%
\pgfsetroundjoin%
\definecolor{currentfill}{rgb}{0.206756,0.371758,0.553117}%
\pgfsetfillcolor{currentfill}%
\pgfsetfillopacity{0.800000}%
\pgfsetlinewidth{0.000000pt}%
\definecolor{currentstroke}{rgb}{0.000000,0.000000,0.000000}%
\pgfsetstrokecolor{currentstroke}%
\pgfsetdash{}{0pt}%
\pgfpathmoveto{\pgfqpoint{2.467412in}{2.814575in}}%
\pgfpathlineto{\pgfqpoint{2.481307in}{2.791524in}}%
\pgfpathlineto{\pgfqpoint{2.495189in}{2.768826in}}%
\pgfpathlineto{\pgfqpoint{2.509056in}{2.746476in}}%
\pgfpathlineto{\pgfqpoint{2.522910in}{2.724470in}}%
\pgfpathlineto{\pgfqpoint{2.531450in}{2.729612in}}%
\pgfpathlineto{\pgfqpoint{2.539977in}{2.734934in}}%
\pgfpathlineto{\pgfqpoint{2.548492in}{2.740434in}}%
\pgfpathlineto{\pgfqpoint{2.556994in}{2.746109in}}%
\pgfpathlineto{\pgfqpoint{2.543174in}{2.767847in}}%
\pgfpathlineto{\pgfqpoint{2.529340in}{2.789930in}}%
\pgfpathlineto{\pgfqpoint{2.515494in}{2.812360in}}%
\pgfpathlineto{\pgfqpoint{2.501633in}{2.835141in}}%
\pgfpathlineto{\pgfqpoint{2.493097in}{2.829721in}}%
\pgfpathlineto{\pgfqpoint{2.484548in}{2.824486in}}%
\pgfpathlineto{\pgfqpoint{2.475987in}{2.819436in}}%
\pgfpathlineto{\pgfqpoint{2.467412in}{2.814575in}}%
\pgfpathclose%
\pgfusepath{fill}%
\end{pgfscope}%
\begin{pgfscope}%
\pgfpathrectangle{\pgfqpoint{1.150000in}{0.150000in}}{\pgfqpoint{5.700000in}{5.700000in}}%
\pgfusepath{clip}%
\pgfsetbuttcap%
\pgfsetroundjoin%
\definecolor{currentfill}{rgb}{0.231674,0.318106,0.544834}%
\pgfsetfillcolor{currentfill}%
\pgfsetfillopacity{0.800000}%
\pgfsetlinewidth{0.000000pt}%
\definecolor{currentstroke}{rgb}{0.000000,0.000000,0.000000}%
\pgfsetstrokecolor{currentstroke}%
\pgfsetdash{}{0pt}%
\pgfpathmoveto{\pgfqpoint{4.613413in}{2.607844in}}%
\pgfpathlineto{\pgfqpoint{4.627297in}{2.613451in}}%
\pgfpathlineto{\pgfqpoint{4.641193in}{2.619241in}}%
\pgfpathlineto{\pgfqpoint{4.655103in}{2.625216in}}%
\pgfpathlineto{\pgfqpoint{4.669027in}{2.631374in}}%
\pgfpathlineto{\pgfqpoint{4.676710in}{2.639643in}}%
\pgfpathlineto{\pgfqpoint{4.684388in}{2.647882in}}%
\pgfpathlineto{\pgfqpoint{4.692061in}{2.656094in}}%
\pgfpathlineto{\pgfqpoint{4.699727in}{2.664284in}}%
\pgfpathlineto{\pgfqpoint{4.685814in}{2.658393in}}%
\pgfpathlineto{\pgfqpoint{4.671915in}{2.652686in}}%
\pgfpathlineto{\pgfqpoint{4.658028in}{2.647162in}}%
\pgfpathlineto{\pgfqpoint{4.644155in}{2.641821in}}%
\pgfpathlineto{\pgfqpoint{4.636478in}{2.633353in}}%
\pgfpathlineto{\pgfqpoint{4.628796in}{2.624870in}}%
\pgfpathlineto{\pgfqpoint{4.621107in}{2.616368in}}%
\pgfpathlineto{\pgfqpoint{4.613413in}{2.607844in}}%
\pgfpathclose%
\pgfusepath{fill}%
\end{pgfscope}%
\begin{pgfscope}%
\pgfpathrectangle{\pgfqpoint{1.150000in}{0.150000in}}{\pgfqpoint{5.700000in}{5.700000in}}%
\pgfusepath{clip}%
\pgfsetbuttcap%
\pgfsetroundjoin%
\definecolor{currentfill}{rgb}{0.281446,0.084320,0.407414}%
\pgfsetfillcolor{currentfill}%
\pgfsetfillopacity{0.800000}%
\pgfsetlinewidth{0.000000pt}%
\definecolor{currentstroke}{rgb}{0.000000,0.000000,0.000000}%
\pgfsetstrokecolor{currentstroke}%
\pgfsetdash{}{0pt}%
\pgfpathmoveto{\pgfqpoint{3.069702in}{2.093908in}}%
\pgfpathlineto{\pgfqpoint{3.083266in}{2.083704in}}%
\pgfpathlineto{\pgfqpoint{3.096829in}{2.073745in}}%
\pgfpathlineto{\pgfqpoint{3.110390in}{2.064029in}}%
\pgfpathlineto{\pgfqpoint{3.123950in}{2.054553in}}%
\pgfpathlineto{\pgfqpoint{3.132194in}{2.062642in}}%
\pgfpathlineto{\pgfqpoint{3.140430in}{2.070820in}}%
\pgfpathlineto{\pgfqpoint{3.148659in}{2.079087in}}%
\pgfpathlineto{\pgfqpoint{3.156880in}{2.087440in}}%
\pgfpathlineto{\pgfqpoint{3.143340in}{2.096667in}}%
\pgfpathlineto{\pgfqpoint{3.129798in}{2.106136in}}%
\pgfpathlineto{\pgfqpoint{3.116256in}{2.115846in}}%
\pgfpathlineto{\pgfqpoint{3.102712in}{2.125801in}}%
\pgfpathlineto{\pgfqpoint{3.094472in}{2.117684in}}%
\pgfpathlineto{\pgfqpoint{3.086223in}{2.109662in}}%
\pgfpathlineto{\pgfqpoint{3.077967in}{2.101736in}}%
\pgfpathlineto{\pgfqpoint{3.069702in}{2.093908in}}%
\pgfpathclose%
\pgfusepath{fill}%
\end{pgfscope}%
\begin{pgfscope}%
\pgfpathrectangle{\pgfqpoint{1.150000in}{0.150000in}}{\pgfqpoint{5.700000in}{5.700000in}}%
\pgfusepath{clip}%
\pgfsetbuttcap%
\pgfsetroundjoin%
\definecolor{currentfill}{rgb}{0.150476,0.504369,0.557430}%
\pgfsetfillcolor{currentfill}%
\pgfsetfillopacity{0.800000}%
\pgfsetlinewidth{0.000000pt}%
\definecolor{currentstroke}{rgb}{0.000000,0.000000,0.000000}%
\pgfsetstrokecolor{currentstroke}%
\pgfsetdash{}{0pt}%
\pgfpathmoveto{\pgfqpoint{5.420840in}{3.138398in}}%
\pgfpathlineto{\pgfqpoint{5.435104in}{3.146301in}}%
\pgfpathlineto{\pgfqpoint{5.449385in}{3.154379in}}%
\pgfpathlineto{\pgfqpoint{5.463684in}{3.162632in}}%
\pgfpathlineto{\pgfqpoint{5.477999in}{3.171060in}}%
\pgfpathlineto{\pgfqpoint{5.485318in}{3.176279in}}%
\pgfpathlineto{\pgfqpoint{5.492633in}{3.181593in}}%
\pgfpathlineto{\pgfqpoint{5.499945in}{3.187008in}}%
\pgfpathlineto{\pgfqpoint{5.507254in}{3.192531in}}%
\pgfpathlineto{\pgfqpoint{5.492965in}{3.184701in}}%
\pgfpathlineto{\pgfqpoint{5.478692in}{3.177045in}}%
\pgfpathlineto{\pgfqpoint{5.464436in}{3.169563in}}%
\pgfpathlineto{\pgfqpoint{5.450197in}{3.162255in}}%
\pgfpathlineto{\pgfqpoint{5.442862in}{3.156124in}}%
\pgfpathlineto{\pgfqpoint{5.435524in}{3.150109in}}%
\pgfpathlineto{\pgfqpoint{5.428184in}{3.144202in}}%
\pgfpathlineto{\pgfqpoint{5.420840in}{3.138398in}}%
\pgfpathclose%
\pgfusepath{fill}%
\end{pgfscope}%
\begin{pgfscope}%
\pgfpathrectangle{\pgfqpoint{1.150000in}{0.150000in}}{\pgfqpoint{5.700000in}{5.700000in}}%
\pgfusepath{clip}%
\pgfsetbuttcap%
\pgfsetroundjoin%
\definecolor{currentfill}{rgb}{0.278791,0.062145,0.386592}%
\pgfsetfillcolor{currentfill}%
\pgfsetfillopacity{0.800000}%
\pgfsetlinewidth{0.000000pt}%
\definecolor{currentstroke}{rgb}{0.000000,0.000000,0.000000}%
\pgfsetstrokecolor{currentstroke}%
\pgfsetdash{}{0pt}%
\pgfpathmoveto{\pgfqpoint{3.492661in}{2.029012in}}%
\pgfpathlineto{\pgfqpoint{3.506200in}{2.024933in}}%
\pgfpathlineto{\pgfqpoint{3.519743in}{2.021068in}}%
\pgfpathlineto{\pgfqpoint{3.533290in}{2.017415in}}%
\pgfpathlineto{\pgfqpoint{3.546840in}{2.013975in}}%
\pgfpathlineto{\pgfqpoint{3.554911in}{2.024001in}}%
\pgfpathlineto{\pgfqpoint{3.562976in}{2.034051in}}%
\pgfpathlineto{\pgfqpoint{3.571036in}{2.044121in}}%
\pgfpathlineto{\pgfqpoint{3.579090in}{2.054213in}}%
\pgfpathlineto{\pgfqpoint{3.565550in}{2.057503in}}%
\pgfpathlineto{\pgfqpoint{3.552016in}{2.061004in}}%
\pgfpathlineto{\pgfqpoint{3.538485in}{2.064719in}}%
\pgfpathlineto{\pgfqpoint{3.524958in}{2.068647in}}%
\pgfpathlineto{\pgfqpoint{3.516892in}{2.058695in}}%
\pgfpathlineto{\pgfqpoint{3.508821in}{2.048771in}}%
\pgfpathlineto{\pgfqpoint{3.500744in}{2.038876in}}%
\pgfpathlineto{\pgfqpoint{3.492661in}{2.029012in}}%
\pgfpathclose%
\pgfusepath{fill}%
\end{pgfscope}%
\begin{pgfscope}%
\pgfpathrectangle{\pgfqpoint{1.150000in}{0.150000in}}{\pgfqpoint{5.700000in}{5.700000in}}%
\pgfusepath{clip}%
\pgfsetbuttcap%
\pgfsetroundjoin%
\definecolor{currentfill}{rgb}{0.220057,0.343307,0.549413}%
\pgfsetfillcolor{currentfill}%
\pgfsetfillopacity{0.800000}%
\pgfsetlinewidth{0.000000pt}%
\definecolor{currentstroke}{rgb}{0.000000,0.000000,0.000000}%
\pgfsetstrokecolor{currentstroke}%
\pgfsetdash{}{0pt}%
\pgfpathmoveto{\pgfqpoint{4.699727in}{2.664284in}}%
\pgfpathlineto{\pgfqpoint{4.713653in}{2.670358in}}%
\pgfpathlineto{\pgfqpoint{4.727593in}{2.676615in}}%
\pgfpathlineto{\pgfqpoint{4.741547in}{2.683055in}}%
\pgfpathlineto{\pgfqpoint{4.755515in}{2.689677in}}%
\pgfpathlineto{\pgfqpoint{4.763165in}{2.697559in}}%
\pgfpathlineto{\pgfqpoint{4.770809in}{2.705416in}}%
\pgfpathlineto{\pgfqpoint{4.778446in}{2.713254in}}%
\pgfpathlineto{\pgfqpoint{4.786078in}{2.721074in}}%
\pgfpathlineto{\pgfqpoint{4.772122in}{2.714752in}}%
\pgfpathlineto{\pgfqpoint{4.758180in}{2.708613in}}%
\pgfpathlineto{\pgfqpoint{4.744251in}{2.702656in}}%
\pgfpathlineto{\pgfqpoint{4.730336in}{2.696881in}}%
\pgfpathlineto{\pgfqpoint{4.722692in}{2.688749in}}%
\pgfpathlineto{\pgfqpoint{4.715043in}{2.680608in}}%
\pgfpathlineto{\pgfqpoint{4.707388in}{2.672454in}}%
\pgfpathlineto{\pgfqpoint{4.699727in}{2.664284in}}%
\pgfpathclose%
\pgfusepath{fill}%
\end{pgfscope}%
\begin{pgfscope}%
\pgfpathrectangle{\pgfqpoint{1.150000in}{0.150000in}}{\pgfqpoint{5.700000in}{5.700000in}}%
\pgfusepath{clip}%
\pgfsetbuttcap%
\pgfsetroundjoin%
\definecolor{currentfill}{rgb}{0.143343,0.522773,0.556295}%
\pgfsetfillcolor{currentfill}%
\pgfsetfillopacity{0.800000}%
\pgfsetlinewidth{0.000000pt}%
\definecolor{currentstroke}{rgb}{0.000000,0.000000,0.000000}%
\pgfsetstrokecolor{currentstroke}%
\pgfsetdash{}{0pt}%
\pgfpathmoveto{\pgfqpoint{5.507254in}{3.192531in}}%
\pgfpathlineto{\pgfqpoint{5.521561in}{3.200534in}}%
\pgfpathlineto{\pgfqpoint{5.535885in}{3.208712in}}%
\pgfpathlineto{\pgfqpoint{5.550227in}{3.217064in}}%
\pgfpathlineto{\pgfqpoint{5.564586in}{3.225590in}}%
\pgfpathlineto{\pgfqpoint{5.571865in}{3.230609in}}%
\pgfpathlineto{\pgfqpoint{5.579141in}{3.235741in}}%
\pgfpathlineto{\pgfqpoint{5.586414in}{3.240995in}}%
\pgfpathlineto{\pgfqpoint{5.593686in}{3.246377in}}%
\pgfpathlineto{\pgfqpoint{5.579355in}{3.238482in}}%
\pgfpathlineto{\pgfqpoint{5.565041in}{3.230760in}}%
\pgfpathlineto{\pgfqpoint{5.550745in}{3.223212in}}%
\pgfpathlineto{\pgfqpoint{5.536465in}{3.215836in}}%
\pgfpathlineto{\pgfqpoint{5.529166in}{3.209814in}}%
\pgfpathlineto{\pgfqpoint{5.521864in}{3.203927in}}%
\pgfpathlineto{\pgfqpoint{5.514560in}{3.198168in}}%
\pgfpathlineto{\pgfqpoint{5.507254in}{3.192531in}}%
\pgfpathclose%
\pgfusepath{fill}%
\end{pgfscope}%
\begin{pgfscope}%
\pgfpathrectangle{\pgfqpoint{1.150000in}{0.150000in}}{\pgfqpoint{5.700000in}{5.700000in}}%
\pgfusepath{clip}%
\pgfsetbuttcap%
\pgfsetroundjoin%
\definecolor{currentfill}{rgb}{0.270595,0.214069,0.507052}%
\pgfsetfillcolor{currentfill}%
\pgfsetfillopacity{0.800000}%
\pgfsetlinewidth{0.000000pt}%
\definecolor{currentstroke}{rgb}{0.000000,0.000000,0.000000}%
\pgfsetstrokecolor{currentstroke}%
\pgfsetdash{}{0pt}%
\pgfpathmoveto{\pgfqpoint{2.709267in}{2.393228in}}%
\pgfpathlineto{\pgfqpoint{2.722974in}{2.376319in}}%
\pgfpathlineto{\pgfqpoint{2.736672in}{2.359703in}}%
\pgfpathlineto{\pgfqpoint{2.750364in}{2.343377in}}%
\pgfpathlineto{\pgfqpoint{2.764047in}{2.327339in}}%
\pgfpathlineto{\pgfqpoint{2.772475in}{2.333269in}}%
\pgfpathlineto{\pgfqpoint{2.780892in}{2.339347in}}%
\pgfpathlineto{\pgfqpoint{2.789299in}{2.345571in}}%
\pgfpathlineto{\pgfqpoint{2.797695in}{2.351938in}}%
\pgfpathlineto{\pgfqpoint{2.784040in}{2.367688in}}%
\pgfpathlineto{\pgfqpoint{2.770378in}{2.383725in}}%
\pgfpathlineto{\pgfqpoint{2.756709in}{2.400052in}}%
\pgfpathlineto{\pgfqpoint{2.743031in}{2.416671in}}%
\pgfpathlineto{\pgfqpoint{2.734607in}{2.410580in}}%
\pgfpathlineto{\pgfqpoint{2.726171in}{2.404642in}}%
\pgfpathlineto{\pgfqpoint{2.717724in}{2.398857in}}%
\pgfpathlineto{\pgfqpoint{2.709267in}{2.393228in}}%
\pgfpathclose%
\pgfusepath{fill}%
\end{pgfscope}%
\begin{pgfscope}%
\pgfpathrectangle{\pgfqpoint{1.150000in}{0.150000in}}{\pgfqpoint{5.700000in}{5.700000in}}%
\pgfusepath{clip}%
\pgfsetbuttcap%
\pgfsetroundjoin%
\definecolor{currentfill}{rgb}{0.277134,0.185228,0.489898}%
\pgfsetfillcolor{currentfill}%
\pgfsetfillopacity{0.800000}%
\pgfsetlinewidth{0.000000pt}%
\definecolor{currentstroke}{rgb}{0.000000,0.000000,0.000000}%
\pgfsetstrokecolor{currentstroke}%
\pgfsetdash{}{0pt}%
\pgfpathmoveto{\pgfqpoint{2.764047in}{2.327339in}}%
\pgfpathlineto{\pgfqpoint{2.777724in}{2.311585in}}%
\pgfpathlineto{\pgfqpoint{2.791393in}{2.296115in}}%
\pgfpathlineto{\pgfqpoint{2.805056in}{2.280926in}}%
\pgfpathlineto{\pgfqpoint{2.818713in}{2.266015in}}%
\pgfpathlineto{\pgfqpoint{2.827112in}{2.272244in}}%
\pgfpathlineto{\pgfqpoint{2.835501in}{2.278614in}}%
\pgfpathlineto{\pgfqpoint{2.843880in}{2.285122in}}%
\pgfpathlineto{\pgfqpoint{2.852250in}{2.291764in}}%
\pgfpathlineto{\pgfqpoint{2.838620in}{2.306388in}}%
\pgfpathlineto{\pgfqpoint{2.824985in}{2.321291in}}%
\pgfpathlineto{\pgfqpoint{2.811343in}{2.336473in}}%
\pgfpathlineto{\pgfqpoint{2.797695in}{2.351938in}}%
\pgfpathlineto{\pgfqpoint{2.789299in}{2.345571in}}%
\pgfpathlineto{\pgfqpoint{2.780892in}{2.339347in}}%
\pgfpathlineto{\pgfqpoint{2.772475in}{2.333269in}}%
\pgfpathlineto{\pgfqpoint{2.764047in}{2.327339in}}%
\pgfpathclose%
\pgfusepath{fill}%
\end{pgfscope}%
\begin{pgfscope}%
\pgfpathrectangle{\pgfqpoint{1.150000in}{0.150000in}}{\pgfqpoint{5.700000in}{5.700000in}}%
\pgfusepath{clip}%
\pgfsetbuttcap%
\pgfsetroundjoin%
\definecolor{currentfill}{rgb}{0.136408,0.541173,0.554483}%
\pgfsetfillcolor{currentfill}%
\pgfsetfillopacity{0.800000}%
\pgfsetlinewidth{0.000000pt}%
\definecolor{currentstroke}{rgb}{0.000000,0.000000,0.000000}%
\pgfsetstrokecolor{currentstroke}%
\pgfsetdash{}{0pt}%
\pgfpathmoveto{\pgfqpoint{5.593686in}{3.246377in}}%
\pgfpathlineto{\pgfqpoint{5.608034in}{3.254445in}}%
\pgfpathlineto{\pgfqpoint{5.622400in}{3.262686in}}%
\pgfpathlineto{\pgfqpoint{5.636784in}{3.271100in}}%
\pgfpathlineto{\pgfqpoint{5.651185in}{3.279688in}}%
\pgfpathlineto{\pgfqpoint{5.658425in}{3.284552in}}%
\pgfpathlineto{\pgfqpoint{5.665663in}{3.289552in}}%
\pgfpathlineto{\pgfqpoint{5.672899in}{3.294694in}}%
\pgfpathlineto{\pgfqpoint{5.680134in}{3.299985in}}%
\pgfpathlineto{\pgfqpoint{5.665763in}{3.292061in}}%
\pgfpathlineto{\pgfqpoint{5.651409in}{3.284310in}}%
\pgfpathlineto{\pgfqpoint{5.637073in}{3.276731in}}%
\pgfpathlineto{\pgfqpoint{5.622754in}{3.269324in}}%
\pgfpathlineto{\pgfqpoint{5.615489in}{3.263360in}}%
\pgfpathlineto{\pgfqpoint{5.608223in}{3.257552in}}%
\pgfpathlineto{\pgfqpoint{5.600955in}{3.251893in}}%
\pgfpathlineto{\pgfqpoint{5.593686in}{3.246377in}}%
\pgfpathclose%
\pgfusepath{fill}%
\end{pgfscope}%
\begin{pgfscope}%
\pgfpathrectangle{\pgfqpoint{1.150000in}{0.150000in}}{\pgfqpoint{5.700000in}{5.700000in}}%
\pgfusepath{clip}%
\pgfsetbuttcap%
\pgfsetroundjoin%
\definecolor{currentfill}{rgb}{0.277941,0.056324,0.381191}%
\pgfsetfillcolor{currentfill}%
\pgfsetfillopacity{0.800000}%
\pgfsetlinewidth{0.000000pt}%
\definecolor{currentstroke}{rgb}{0.000000,0.000000,0.000000}%
\pgfsetstrokecolor{currentstroke}%
\pgfsetdash{}{0pt}%
\pgfpathmoveto{\pgfqpoint{3.265189in}{2.022117in}}%
\pgfpathlineto{\pgfqpoint{3.278729in}{2.014993in}}%
\pgfpathlineto{\pgfqpoint{3.292271in}{2.008096in}}%
\pgfpathlineto{\pgfqpoint{3.305813in}{2.001425in}}%
\pgfpathlineto{\pgfqpoint{3.319357in}{1.994978in}}%
\pgfpathlineto{\pgfqpoint{3.327518in}{2.004096in}}%
\pgfpathlineto{\pgfqpoint{3.335672in}{2.013273in}}%
\pgfpathlineto{\pgfqpoint{3.343820in}{2.022505in}}%
\pgfpathlineto{\pgfqpoint{3.351961in}{2.031793in}}%
\pgfpathlineto{\pgfqpoint{3.338433in}{2.038025in}}%
\pgfpathlineto{\pgfqpoint{3.324906in}{2.044481in}}%
\pgfpathlineto{\pgfqpoint{3.311381in}{2.051163in}}%
\pgfpathlineto{\pgfqpoint{3.297857in}{2.058072in}}%
\pgfpathlineto{\pgfqpoint{3.289700in}{2.048988in}}%
\pgfpathlineto{\pgfqpoint{3.281537in}{2.039966in}}%
\pgfpathlineto{\pgfqpoint{3.273366in}{2.031008in}}%
\pgfpathlineto{\pgfqpoint{3.265189in}{2.022117in}}%
\pgfpathclose%
\pgfusepath{fill}%
\end{pgfscope}%
\begin{pgfscope}%
\pgfpathrectangle{\pgfqpoint{1.150000in}{0.150000in}}{\pgfqpoint{5.700000in}{5.700000in}}%
\pgfusepath{clip}%
\pgfsetbuttcap%
\pgfsetroundjoin%
\definecolor{currentfill}{rgb}{0.262138,0.242286,0.520837}%
\pgfsetfillcolor{currentfill}%
\pgfsetfillopacity{0.800000}%
\pgfsetlinewidth{0.000000pt}%
\definecolor{currentstroke}{rgb}{0.000000,0.000000,0.000000}%
\pgfsetstrokecolor{currentstroke}%
\pgfsetdash{}{0pt}%
\pgfpathmoveto{\pgfqpoint{2.654354in}{2.463840in}}%
\pgfpathlineto{\pgfqpoint{2.668095in}{2.445735in}}%
\pgfpathlineto{\pgfqpoint{2.681828in}{2.427933in}}%
\pgfpathlineto{\pgfqpoint{2.695552in}{2.410432in}}%
\pgfpathlineto{\pgfqpoint{2.709267in}{2.393228in}}%
\pgfpathlineto{\pgfqpoint{2.717724in}{2.398857in}}%
\pgfpathlineto{\pgfqpoint{2.726171in}{2.404642in}}%
\pgfpathlineto{\pgfqpoint{2.734607in}{2.410580in}}%
\pgfpathlineto{\pgfqpoint{2.743031in}{2.416671in}}%
\pgfpathlineto{\pgfqpoint{2.729346in}{2.433584in}}%
\pgfpathlineto{\pgfqpoint{2.715653in}{2.450794in}}%
\pgfpathlineto{\pgfqpoint{2.701951in}{2.468304in}}%
\pgfpathlineto{\pgfqpoint{2.688240in}{2.486116in}}%
\pgfpathlineto{\pgfqpoint{2.679786in}{2.480305in}}%
\pgfpathlineto{\pgfqpoint{2.671320in}{2.474653in}}%
\pgfpathlineto{\pgfqpoint{2.662843in}{2.469164in}}%
\pgfpathlineto{\pgfqpoint{2.654354in}{2.463840in}}%
\pgfpathclose%
\pgfusepath{fill}%
\end{pgfscope}%
\begin{pgfscope}%
\pgfpathrectangle{\pgfqpoint{1.150000in}{0.150000in}}{\pgfqpoint{5.700000in}{5.700000in}}%
\pgfusepath{clip}%
\pgfsetbuttcap%
\pgfsetroundjoin%
\definecolor{currentfill}{rgb}{0.280868,0.160771,0.472899}%
\pgfsetfillcolor{currentfill}%
\pgfsetfillopacity{0.800000}%
\pgfsetlinewidth{0.000000pt}%
\definecolor{currentstroke}{rgb}{0.000000,0.000000,0.000000}%
\pgfsetstrokecolor{currentstroke}%
\pgfsetdash{}{0pt}%
\pgfpathmoveto{\pgfqpoint{2.818713in}{2.266015in}}%
\pgfpathlineto{\pgfqpoint{2.832363in}{2.251380in}}%
\pgfpathlineto{\pgfqpoint{2.846008in}{2.237019in}}%
\pgfpathlineto{\pgfqpoint{2.859647in}{2.222930in}}%
\pgfpathlineto{\pgfqpoint{2.873281in}{2.209111in}}%
\pgfpathlineto{\pgfqpoint{2.881653in}{2.215639in}}%
\pgfpathlineto{\pgfqpoint{2.890015in}{2.222298in}}%
\pgfpathlineto{\pgfqpoint{2.898368in}{2.229087in}}%
\pgfpathlineto{\pgfqpoint{2.906711in}{2.236003in}}%
\pgfpathlineto{\pgfqpoint{2.893104in}{2.249537in}}%
\pgfpathlineto{\pgfqpoint{2.879491in}{2.263340in}}%
\pgfpathlineto{\pgfqpoint{2.865873in}{2.277415in}}%
\pgfpathlineto{\pgfqpoint{2.852250in}{2.291764in}}%
\pgfpathlineto{\pgfqpoint{2.843880in}{2.285122in}}%
\pgfpathlineto{\pgfqpoint{2.835501in}{2.278614in}}%
\pgfpathlineto{\pgfqpoint{2.827112in}{2.272244in}}%
\pgfpathlineto{\pgfqpoint{2.818713in}{2.266015in}}%
\pgfpathclose%
\pgfusepath{fill}%
\end{pgfscope}%
\begin{pgfscope}%
\pgfpathrectangle{\pgfqpoint{1.150000in}{0.150000in}}{\pgfqpoint{5.700000in}{5.700000in}}%
\pgfusepath{clip}%
\pgfsetbuttcap%
\pgfsetroundjoin%
\definecolor{currentfill}{rgb}{0.210503,0.363727,0.552206}%
\pgfsetfillcolor{currentfill}%
\pgfsetfillopacity{0.800000}%
\pgfsetlinewidth{0.000000pt}%
\definecolor{currentstroke}{rgb}{0.000000,0.000000,0.000000}%
\pgfsetstrokecolor{currentstroke}%
\pgfsetdash{}{0pt}%
\pgfpathmoveto{\pgfqpoint{4.786078in}{2.721074in}}%
\pgfpathlineto{\pgfqpoint{4.800049in}{2.727577in}}%
\pgfpathlineto{\pgfqpoint{4.814033in}{2.734263in}}%
\pgfpathlineto{\pgfqpoint{4.828032in}{2.741130in}}%
\pgfpathlineto{\pgfqpoint{4.842046in}{2.748179in}}%
\pgfpathlineto{\pgfqpoint{4.849660in}{2.755664in}}%
\pgfpathlineto{\pgfqpoint{4.857268in}{2.763133in}}%
\pgfpathlineto{\pgfqpoint{4.864870in}{2.770588in}}%
\pgfpathlineto{\pgfqpoint{4.872467in}{2.778033in}}%
\pgfpathlineto{\pgfqpoint{4.858466in}{2.771318in}}%
\pgfpathlineto{\pgfqpoint{4.844480in}{2.764785in}}%
\pgfpathlineto{\pgfqpoint{4.830508in}{2.758432in}}%
\pgfpathlineto{\pgfqpoint{4.816550in}{2.752261in}}%
\pgfpathlineto{\pgfqpoint{4.808941in}{2.744471in}}%
\pgfpathlineto{\pgfqpoint{4.801326in}{2.736679in}}%
\pgfpathlineto{\pgfqpoint{4.793705in}{2.728881in}}%
\pgfpathlineto{\pgfqpoint{4.786078in}{2.721074in}}%
\pgfpathclose%
\pgfusepath{fill}%
\end{pgfscope}%
\begin{pgfscope}%
\pgfpathrectangle{\pgfqpoint{1.150000in}{0.150000in}}{\pgfqpoint{5.700000in}{5.700000in}}%
\pgfusepath{clip}%
\pgfsetbuttcap%
\pgfsetroundjoin%
\definecolor{currentfill}{rgb}{0.281887,0.150881,0.465405}%
\pgfsetfillcolor{currentfill}%
\pgfsetfillopacity{0.800000}%
\pgfsetlinewidth{0.000000pt}%
\definecolor{currentstroke}{rgb}{0.000000,0.000000,0.000000}%
\pgfsetstrokecolor{currentstroke}%
\pgfsetdash{}{0pt}%
\pgfpathmoveto{\pgfqpoint{3.978464in}{2.201051in}}%
\pgfpathlineto{\pgfqpoint{3.992104in}{2.202313in}}%
\pgfpathlineto{\pgfqpoint{4.005753in}{2.203771in}}%
\pgfpathlineto{\pgfqpoint{4.019411in}{2.205424in}}%
\pgfpathlineto{\pgfqpoint{4.033077in}{2.207272in}}%
\pgfpathlineto{\pgfqpoint{4.040990in}{2.217655in}}%
\pgfpathlineto{\pgfqpoint{4.048899in}{2.228007in}}%
\pgfpathlineto{\pgfqpoint{4.056802in}{2.238327in}}%
\pgfpathlineto{\pgfqpoint{4.064699in}{2.248617in}}%
\pgfpathlineto{\pgfqpoint{4.051040in}{2.246777in}}%
\pgfpathlineto{\pgfqpoint{4.037390in}{2.245132in}}%
\pgfpathlineto{\pgfqpoint{4.023748in}{2.243682in}}%
\pgfpathlineto{\pgfqpoint{4.010115in}{2.242428in}}%
\pgfpathlineto{\pgfqpoint{4.002210in}{2.232118in}}%
\pgfpathlineto{\pgfqpoint{3.994300in}{2.221786in}}%
\pgfpathlineto{\pgfqpoint{3.986385in}{2.211431in}}%
\pgfpathlineto{\pgfqpoint{3.978464in}{2.201051in}}%
\pgfpathclose%
\pgfusepath{fill}%
\end{pgfscope}%
\begin{pgfscope}%
\pgfpathrectangle{\pgfqpoint{1.150000in}{0.150000in}}{\pgfqpoint{5.700000in}{5.700000in}}%
\pgfusepath{clip}%
\pgfsetbuttcap%
\pgfsetroundjoin%
\definecolor{currentfill}{rgb}{0.283072,0.130895,0.449241}%
\pgfsetfillcolor{currentfill}%
\pgfsetfillopacity{0.800000}%
\pgfsetlinewidth{0.000000pt}%
\definecolor{currentstroke}{rgb}{0.000000,0.000000,0.000000}%
\pgfsetstrokecolor{currentstroke}%
\pgfsetdash{}{0pt}%
\pgfpathmoveto{\pgfqpoint{3.892223in}{2.156306in}}%
\pgfpathlineto{\pgfqpoint{3.905839in}{2.156754in}}%
\pgfpathlineto{\pgfqpoint{3.919462in}{2.157401in}}%
\pgfpathlineto{\pgfqpoint{3.933093in}{2.158246in}}%
\pgfpathlineto{\pgfqpoint{3.946732in}{2.159288in}}%
\pgfpathlineto{\pgfqpoint{3.954673in}{2.169766in}}%
\pgfpathlineto{\pgfqpoint{3.962608in}{2.180219in}}%
\pgfpathlineto{\pgfqpoint{3.970539in}{2.190648in}}%
\pgfpathlineto{\pgfqpoint{3.978464in}{2.201051in}}%
\pgfpathlineto{\pgfqpoint{3.964832in}{2.199986in}}%
\pgfpathlineto{\pgfqpoint{3.951209in}{2.199117in}}%
\pgfpathlineto{\pgfqpoint{3.937593in}{2.198446in}}%
\pgfpathlineto{\pgfqpoint{3.923986in}{2.197974in}}%
\pgfpathlineto{\pgfqpoint{3.916053in}{2.187582in}}%
\pgfpathlineto{\pgfqpoint{3.908115in}{2.177174in}}%
\pgfpathlineto{\pgfqpoint{3.900172in}{2.166748in}}%
\pgfpathlineto{\pgfqpoint{3.892223in}{2.156306in}}%
\pgfpathclose%
\pgfusepath{fill}%
\end{pgfscope}%
\begin{pgfscope}%
\pgfpathrectangle{\pgfqpoint{1.150000in}{0.150000in}}{\pgfqpoint{5.700000in}{5.700000in}}%
\pgfusepath{clip}%
\pgfsetbuttcap%
\pgfsetroundjoin%
\definecolor{currentfill}{rgb}{0.278826,0.175490,0.483397}%
\pgfsetfillcolor{currentfill}%
\pgfsetfillopacity{0.800000}%
\pgfsetlinewidth{0.000000pt}%
\definecolor{currentstroke}{rgb}{0.000000,0.000000,0.000000}%
\pgfsetstrokecolor{currentstroke}%
\pgfsetdash{}{0pt}%
\pgfpathmoveto{\pgfqpoint{4.064699in}{2.248617in}}%
\pgfpathlineto{\pgfqpoint{4.078368in}{2.250651in}}%
\pgfpathlineto{\pgfqpoint{4.092046in}{2.252879in}}%
\pgfpathlineto{\pgfqpoint{4.105733in}{2.255300in}}%
\pgfpathlineto{\pgfqpoint{4.119430in}{2.257914in}}%
\pgfpathlineto{\pgfqpoint{4.127316in}{2.268146in}}%
\pgfpathlineto{\pgfqpoint{4.135197in}{2.278340in}}%
\pgfpathlineto{\pgfqpoint{4.143072in}{2.288498in}}%
\pgfpathlineto{\pgfqpoint{4.150943in}{2.298621in}}%
\pgfpathlineto{\pgfqpoint{4.137252in}{2.296047in}}%
\pgfpathlineto{\pgfqpoint{4.123572in}{2.293666in}}%
\pgfpathlineto{\pgfqpoint{4.109901in}{2.291477in}}%
\pgfpathlineto{\pgfqpoint{4.096240in}{2.289483in}}%
\pgfpathlineto{\pgfqpoint{4.088362in}{2.279309in}}%
\pgfpathlineto{\pgfqpoint{4.080480in}{2.269107in}}%
\pgfpathlineto{\pgfqpoint{4.072592in}{2.258876in}}%
\pgfpathlineto{\pgfqpoint{4.064699in}{2.248617in}}%
\pgfpathclose%
\pgfusepath{fill}%
\end{pgfscope}%
\begin{pgfscope}%
\pgfpathrectangle{\pgfqpoint{1.150000in}{0.150000in}}{\pgfqpoint{5.700000in}{5.700000in}}%
\pgfusepath{clip}%
\pgfsetbuttcap%
\pgfsetroundjoin%
\definecolor{currentfill}{rgb}{0.283091,0.110553,0.431554}%
\pgfsetfillcolor{currentfill}%
\pgfsetfillopacity{0.800000}%
\pgfsetlinewidth{0.000000pt}%
\definecolor{currentstroke}{rgb}{0.000000,0.000000,0.000000}%
\pgfsetstrokecolor{currentstroke}%
\pgfsetdash{}{0pt}%
\pgfpathmoveto{\pgfqpoint{3.805961in}{2.114784in}}%
\pgfpathlineto{\pgfqpoint{3.819555in}{2.114378in}}%
\pgfpathlineto{\pgfqpoint{3.833156in}{2.114173in}}%
\pgfpathlineto{\pgfqpoint{3.846764in}{2.114168in}}%
\pgfpathlineto{\pgfqpoint{3.860379in}{2.114362in}}%
\pgfpathlineto{\pgfqpoint{3.868348in}{2.124874in}}%
\pgfpathlineto{\pgfqpoint{3.876312in}{2.135368in}}%
\pgfpathlineto{\pgfqpoint{3.884270in}{2.145846in}}%
\pgfpathlineto{\pgfqpoint{3.892223in}{2.156306in}}%
\pgfpathlineto{\pgfqpoint{3.878616in}{2.156056in}}%
\pgfpathlineto{\pgfqpoint{3.865016in}{2.156005in}}%
\pgfpathlineto{\pgfqpoint{3.851423in}{2.156155in}}%
\pgfpathlineto{\pgfqpoint{3.837838in}{2.156505in}}%
\pgfpathlineto{\pgfqpoint{3.829876in}{2.146089in}}%
\pgfpathlineto{\pgfqpoint{3.821910in}{2.135663in}}%
\pgfpathlineto{\pgfqpoint{3.813938in}{2.125228in}}%
\pgfpathlineto{\pgfqpoint{3.805961in}{2.114784in}}%
\pgfpathclose%
\pgfusepath{fill}%
\end{pgfscope}%
\begin{pgfscope}%
\pgfpathrectangle{\pgfqpoint{1.150000in}{0.150000in}}{\pgfqpoint{5.700000in}{5.700000in}}%
\pgfusepath{clip}%
\pgfsetbuttcap%
\pgfsetroundjoin%
\definecolor{currentfill}{rgb}{0.129933,0.559582,0.551864}%
\pgfsetfillcolor{currentfill}%
\pgfsetfillopacity{0.800000}%
\pgfsetlinewidth{0.000000pt}%
\definecolor{currentstroke}{rgb}{0.000000,0.000000,0.000000}%
\pgfsetstrokecolor{currentstroke}%
\pgfsetdash{}{0pt}%
\pgfpathmoveto{\pgfqpoint{5.680134in}{3.299985in}}%
\pgfpathlineto{\pgfqpoint{5.694523in}{3.308081in}}%
\pgfpathlineto{\pgfqpoint{5.708930in}{3.316349in}}%
\pgfpathlineto{\pgfqpoint{5.723355in}{3.324790in}}%
\pgfpathlineto{\pgfqpoint{5.737798in}{3.333403in}}%
\pgfpathlineto{\pgfqpoint{5.745000in}{3.338166in}}%
\pgfpathlineto{\pgfqpoint{5.752201in}{3.343086in}}%
\pgfpathlineto{\pgfqpoint{5.759401in}{3.348171in}}%
\pgfpathlineto{\pgfqpoint{5.766601in}{3.353427in}}%
\pgfpathlineto{\pgfqpoint{5.752190in}{3.345511in}}%
\pgfpathlineto{\pgfqpoint{5.737798in}{3.337767in}}%
\pgfpathlineto{\pgfqpoint{5.723423in}{3.330193in}}%
\pgfpathlineto{\pgfqpoint{5.709065in}{3.322791in}}%
\pgfpathlineto{\pgfqpoint{5.701833in}{3.316828in}}%
\pgfpathlineto{\pgfqpoint{5.694601in}{3.311045in}}%
\pgfpathlineto{\pgfqpoint{5.687368in}{3.305433in}}%
\pgfpathlineto{\pgfqpoint{5.680134in}{3.299985in}}%
\pgfpathclose%
\pgfusepath{fill}%
\end{pgfscope}%
\begin{pgfscope}%
\pgfpathrectangle{\pgfqpoint{1.150000in}{0.150000in}}{\pgfqpoint{5.700000in}{5.700000in}}%
\pgfusepath{clip}%
\pgfsetbuttcap%
\pgfsetroundjoin%
\definecolor{currentfill}{rgb}{0.275191,0.194905,0.496005}%
\pgfsetfillcolor{currentfill}%
\pgfsetfillopacity{0.800000}%
\pgfsetlinewidth{0.000000pt}%
\definecolor{currentstroke}{rgb}{0.000000,0.000000,0.000000}%
\pgfsetstrokecolor{currentstroke}%
\pgfsetdash{}{0pt}%
\pgfpathmoveto{\pgfqpoint{4.150943in}{2.298621in}}%
\pgfpathlineto{\pgfqpoint{4.164643in}{2.301387in}}%
\pgfpathlineto{\pgfqpoint{4.178353in}{2.304345in}}%
\pgfpathlineto{\pgfqpoint{4.192073in}{2.307495in}}%
\pgfpathlineto{\pgfqpoint{4.205803in}{2.310836in}}%
\pgfpathlineto{\pgfqpoint{4.213662in}{2.320864in}}%
\pgfpathlineto{\pgfqpoint{4.221514in}{2.330851in}}%
\pgfpathlineto{\pgfqpoint{4.229362in}{2.340797in}}%
\pgfpathlineto{\pgfqpoint{4.237204in}{2.350705in}}%
\pgfpathlineto{\pgfqpoint{4.223481in}{2.347437in}}%
\pgfpathlineto{\pgfqpoint{4.209767in}{2.344360in}}%
\pgfpathlineto{\pgfqpoint{4.196064in}{2.341474in}}%
\pgfpathlineto{\pgfqpoint{4.182371in}{2.338779in}}%
\pgfpathlineto{\pgfqpoint{4.174522in}{2.328787in}}%
\pgfpathlineto{\pgfqpoint{4.166667in}{2.318764in}}%
\pgfpathlineto{\pgfqpoint{4.158808in}{2.308709in}}%
\pgfpathlineto{\pgfqpoint{4.150943in}{2.298621in}}%
\pgfpathclose%
\pgfusepath{fill}%
\end{pgfscope}%
\begin{pgfscope}%
\pgfpathrectangle{\pgfqpoint{1.150000in}{0.150000in}}{\pgfqpoint{5.700000in}{5.700000in}}%
\pgfusepath{clip}%
\pgfsetbuttcap%
\pgfsetroundjoin%
\definecolor{currentfill}{rgb}{0.277941,0.056324,0.381191}%
\pgfsetfillcolor{currentfill}%
\pgfsetfillopacity{0.800000}%
\pgfsetlinewidth{0.000000pt}%
\definecolor{currentstroke}{rgb}{0.000000,0.000000,0.000000}%
\pgfsetstrokecolor{currentstroke}%
\pgfsetdash{}{0pt}%
\pgfpathmoveto{\pgfqpoint{3.406095in}{2.009088in}}%
\pgfpathlineto{\pgfqpoint{3.419635in}{2.003961in}}%
\pgfpathlineto{\pgfqpoint{3.433177in}{1.999053in}}%
\pgfpathlineto{\pgfqpoint{3.446722in}{1.994361in}}%
\pgfpathlineto{\pgfqpoint{3.460271in}{1.989885in}}%
\pgfpathlineto{\pgfqpoint{3.468377in}{1.999615in}}%
\pgfpathlineto{\pgfqpoint{3.476478in}{2.009380in}}%
\pgfpathlineto{\pgfqpoint{3.484572in}{2.019179in}}%
\pgfpathlineto{\pgfqpoint{3.492661in}{2.029012in}}%
\pgfpathlineto{\pgfqpoint{3.479126in}{2.033305in}}%
\pgfpathlineto{\pgfqpoint{3.465594in}{2.037815in}}%
\pgfpathlineto{\pgfqpoint{3.452065in}{2.042541in}}%
\pgfpathlineto{\pgfqpoint{3.438539in}{2.047485in}}%
\pgfpathlineto{\pgfqpoint{3.430437in}{2.037823in}}%
\pgfpathlineto{\pgfqpoint{3.422329in}{2.028202in}}%
\pgfpathlineto{\pgfqpoint{3.414215in}{2.018623in}}%
\pgfpathlineto{\pgfqpoint{3.406095in}{2.009088in}}%
\pgfpathclose%
\pgfusepath{fill}%
\end{pgfscope}%
\begin{pgfscope}%
\pgfpathrectangle{\pgfqpoint{1.150000in}{0.150000in}}{\pgfqpoint{5.700000in}{5.700000in}}%
\pgfusepath{clip}%
\pgfsetbuttcap%
\pgfsetroundjoin%
\definecolor{currentfill}{rgb}{0.282623,0.140926,0.457517}%
\pgfsetfillcolor{currentfill}%
\pgfsetfillopacity{0.800000}%
\pgfsetlinewidth{0.000000pt}%
\definecolor{currentstroke}{rgb}{0.000000,0.000000,0.000000}%
\pgfsetstrokecolor{currentstroke}%
\pgfsetdash{}{0pt}%
\pgfpathmoveto{\pgfqpoint{2.873281in}{2.209111in}}%
\pgfpathlineto{\pgfqpoint{2.886909in}{2.195560in}}%
\pgfpathlineto{\pgfqpoint{2.900533in}{2.182274in}}%
\pgfpathlineto{\pgfqpoint{2.914152in}{2.169253in}}%
\pgfpathlineto{\pgfqpoint{2.927767in}{2.156493in}}%
\pgfpathlineto{\pgfqpoint{2.936114in}{2.163317in}}%
\pgfpathlineto{\pgfqpoint{2.944451in}{2.170265in}}%
\pgfpathlineto{\pgfqpoint{2.952778in}{2.177334in}}%
\pgfpathlineto{\pgfqpoint{2.961097in}{2.184521in}}%
\pgfpathlineto{\pgfqpoint{2.947507in}{2.196998in}}%
\pgfpathlineto{\pgfqpoint{2.933913in}{2.209735in}}%
\pgfpathlineto{\pgfqpoint{2.920314in}{2.222736in}}%
\pgfpathlineto{\pgfqpoint{2.906711in}{2.236003in}}%
\pgfpathlineto{\pgfqpoint{2.898368in}{2.229087in}}%
\pgfpathlineto{\pgfqpoint{2.890015in}{2.222298in}}%
\pgfpathlineto{\pgfqpoint{2.881653in}{2.215639in}}%
\pgfpathlineto{\pgfqpoint{2.873281in}{2.209111in}}%
\pgfpathclose%
\pgfusepath{fill}%
\end{pgfscope}%
\begin{pgfscope}%
\pgfpathrectangle{\pgfqpoint{1.150000in}{0.150000in}}{\pgfqpoint{5.700000in}{5.700000in}}%
\pgfusepath{clip}%
\pgfsetbuttcap%
\pgfsetroundjoin%
\definecolor{currentfill}{rgb}{0.250425,0.274290,0.533103}%
\pgfsetfillcolor{currentfill}%
\pgfsetfillopacity{0.800000}%
\pgfsetlinewidth{0.000000pt}%
\definecolor{currentstroke}{rgb}{0.000000,0.000000,0.000000}%
\pgfsetstrokecolor{currentstroke}%
\pgfsetdash{}{0pt}%
\pgfpathmoveto{\pgfqpoint{2.599289in}{2.539341in}}%
\pgfpathlineto{\pgfqpoint{2.613070in}{2.519997in}}%
\pgfpathlineto{\pgfqpoint{2.626841in}{2.500968in}}%
\pgfpathlineto{\pgfqpoint{2.640602in}{2.482250in}}%
\pgfpathlineto{\pgfqpoint{2.654354in}{2.463840in}}%
\pgfpathlineto{\pgfqpoint{2.662843in}{2.469164in}}%
\pgfpathlineto{\pgfqpoint{2.671320in}{2.474653in}}%
\pgfpathlineto{\pgfqpoint{2.679786in}{2.480305in}}%
\pgfpathlineto{\pgfqpoint{2.688240in}{2.486116in}}%
\pgfpathlineto{\pgfqpoint{2.674520in}{2.504233in}}%
\pgfpathlineto{\pgfqpoint{2.660791in}{2.522658in}}%
\pgfpathlineto{\pgfqpoint{2.647052in}{2.541393in}}%
\pgfpathlineto{\pgfqpoint{2.633303in}{2.560442in}}%
\pgfpathlineto{\pgfqpoint{2.624817in}{2.554912in}}%
\pgfpathlineto{\pgfqpoint{2.616320in}{2.549550in}}%
\pgfpathlineto{\pgfqpoint{2.607810in}{2.544359in}}%
\pgfpathlineto{\pgfqpoint{2.599289in}{2.539341in}}%
\pgfpathclose%
\pgfusepath{fill}%
\end{pgfscope}%
\begin{pgfscope}%
\pgfpathrectangle{\pgfqpoint{1.150000in}{0.150000in}}{\pgfqpoint{5.700000in}{5.700000in}}%
\pgfusepath{clip}%
\pgfsetbuttcap%
\pgfsetroundjoin%
\definecolor{currentfill}{rgb}{0.280267,0.073417,0.397163}%
\pgfsetfillcolor{currentfill}%
\pgfsetfillopacity{0.800000}%
\pgfsetlinewidth{0.000000pt}%
\definecolor{currentstroke}{rgb}{0.000000,0.000000,0.000000}%
\pgfsetstrokecolor{currentstroke}%
\pgfsetdash{}{0pt}%
\pgfpathmoveto{\pgfqpoint{3.123950in}{2.054553in}}%
\pgfpathlineto{\pgfqpoint{3.137509in}{2.045317in}}%
\pgfpathlineto{\pgfqpoint{3.151067in}{2.036319in}}%
\pgfpathlineto{\pgfqpoint{3.164625in}{2.027558in}}%
\pgfpathlineto{\pgfqpoint{3.178182in}{2.019032in}}%
\pgfpathlineto{\pgfqpoint{3.186406in}{2.027380in}}%
\pgfpathlineto{\pgfqpoint{3.194623in}{2.035810in}}%
\pgfpathlineto{\pgfqpoint{3.202832in}{2.044321in}}%
\pgfpathlineto{\pgfqpoint{3.211034in}{2.052909in}}%
\pgfpathlineto{\pgfqpoint{3.197496in}{2.061188in}}%
\pgfpathlineto{\pgfqpoint{3.183958in}{2.069702in}}%
\pgfpathlineto{\pgfqpoint{3.170419in}{2.078452in}}%
\pgfpathlineto{\pgfqpoint{3.156880in}{2.087440in}}%
\pgfpathlineto{\pgfqpoint{3.148659in}{2.079087in}}%
\pgfpathlineto{\pgfqpoint{3.140430in}{2.070820in}}%
\pgfpathlineto{\pgfqpoint{3.132194in}{2.062642in}}%
\pgfpathlineto{\pgfqpoint{3.123950in}{2.054553in}}%
\pgfpathclose%
\pgfusepath{fill}%
\end{pgfscope}%
\begin{pgfscope}%
\pgfpathrectangle{\pgfqpoint{1.150000in}{0.150000in}}{\pgfqpoint{5.700000in}{5.700000in}}%
\pgfusepath{clip}%
\pgfsetbuttcap%
\pgfsetroundjoin%
\definecolor{currentfill}{rgb}{0.282327,0.094955,0.417331}%
\pgfsetfillcolor{currentfill}%
\pgfsetfillopacity{0.800000}%
\pgfsetlinewidth{0.000000pt}%
\definecolor{currentstroke}{rgb}{0.000000,0.000000,0.000000}%
\pgfsetstrokecolor{currentstroke}%
\pgfsetdash{}{0pt}%
\pgfpathmoveto{\pgfqpoint{3.719659in}{2.076916in}}%
\pgfpathlineto{\pgfqpoint{3.733235in}{2.075613in}}%
\pgfpathlineto{\pgfqpoint{3.746818in}{2.074513in}}%
\pgfpathlineto{\pgfqpoint{3.760407in}{2.073616in}}%
\pgfpathlineto{\pgfqpoint{3.774002in}{2.072921in}}%
\pgfpathlineto{\pgfqpoint{3.782000in}{2.083399in}}%
\pgfpathlineto{\pgfqpoint{3.789992in}{2.093869in}}%
\pgfpathlineto{\pgfqpoint{3.797979in}{2.104331in}}%
\pgfpathlineto{\pgfqpoint{3.805961in}{2.114784in}}%
\pgfpathlineto{\pgfqpoint{3.792374in}{2.115392in}}%
\pgfpathlineto{\pgfqpoint{3.778794in}{2.116201in}}%
\pgfpathlineto{\pgfqpoint{3.765220in}{2.117214in}}%
\pgfpathlineto{\pgfqpoint{3.751653in}{2.118430in}}%
\pgfpathlineto{\pgfqpoint{3.743662in}{2.108052in}}%
\pgfpathlineto{\pgfqpoint{3.735666in}{2.097674in}}%
\pgfpathlineto{\pgfqpoint{3.727665in}{2.087295in}}%
\pgfpathlineto{\pgfqpoint{3.719659in}{2.076916in}}%
\pgfpathclose%
\pgfusepath{fill}%
\end{pgfscope}%
\begin{pgfscope}%
\pgfpathrectangle{\pgfqpoint{1.150000in}{0.150000in}}{\pgfqpoint{5.700000in}{5.700000in}}%
\pgfusepath{clip}%
\pgfsetbuttcap%
\pgfsetroundjoin%
\definecolor{currentfill}{rgb}{0.199430,0.387607,0.554642}%
\pgfsetfillcolor{currentfill}%
\pgfsetfillopacity{0.800000}%
\pgfsetlinewidth{0.000000pt}%
\definecolor{currentstroke}{rgb}{0.000000,0.000000,0.000000}%
\pgfsetstrokecolor{currentstroke}%
\pgfsetdash{}{0pt}%
\pgfpathmoveto{\pgfqpoint{4.872467in}{2.778033in}}%
\pgfpathlineto{\pgfqpoint{4.886482in}{2.784929in}}%
\pgfpathlineto{\pgfqpoint{4.900512in}{2.792005in}}%
\pgfpathlineto{\pgfqpoint{4.914556in}{2.799262in}}%
\pgfpathlineto{\pgfqpoint{4.928616in}{2.806700in}}%
\pgfpathlineto{\pgfqpoint{4.936193in}{2.813785in}}%
\pgfpathlineto{\pgfqpoint{4.943764in}{2.820862in}}%
\pgfpathlineto{\pgfqpoint{4.951330in}{2.827933in}}%
\pgfpathlineto{\pgfqpoint{4.958890in}{2.835003in}}%
\pgfpathlineto{\pgfqpoint{4.944844in}{2.827933in}}%
\pgfpathlineto{\pgfqpoint{4.930814in}{2.821043in}}%
\pgfpathlineto{\pgfqpoint{4.916798in}{2.814332in}}%
\pgfpathlineto{\pgfqpoint{4.902797in}{2.807802in}}%
\pgfpathlineto{\pgfqpoint{4.895222in}{2.800353in}}%
\pgfpathlineto{\pgfqpoint{4.887643in}{2.792912in}}%
\pgfpathlineto{\pgfqpoint{4.880058in}{2.785473in}}%
\pgfpathlineto{\pgfqpoint{4.872467in}{2.778033in}}%
\pgfpathclose%
\pgfusepath{fill}%
\end{pgfscope}%
\begin{pgfscope}%
\pgfpathrectangle{\pgfqpoint{1.150000in}{0.150000in}}{\pgfqpoint{5.700000in}{5.700000in}}%
\pgfusepath{clip}%
\pgfsetbuttcap%
\pgfsetroundjoin%
\definecolor{currentfill}{rgb}{0.124395,0.578002,0.548287}%
\pgfsetfillcolor{currentfill}%
\pgfsetfillopacity{0.800000}%
\pgfsetlinewidth{0.000000pt}%
\definecolor{currentstroke}{rgb}{0.000000,0.000000,0.000000}%
\pgfsetstrokecolor{currentstroke}%
\pgfsetdash{}{0pt}%
\pgfpathmoveto{\pgfqpoint{5.766601in}{3.353427in}}%
\pgfpathlineto{\pgfqpoint{5.781029in}{3.361515in}}%
\pgfpathlineto{\pgfqpoint{5.795476in}{3.369775in}}%
\pgfpathlineto{\pgfqpoint{5.809941in}{3.378206in}}%
\pgfpathlineto{\pgfqpoint{5.824424in}{3.386809in}}%
\pgfpathlineto{\pgfqpoint{5.831590in}{3.391528in}}%
\pgfpathlineto{\pgfqpoint{5.838755in}{3.396428in}}%
\pgfpathlineto{\pgfqpoint{5.845921in}{3.401516in}}%
\pgfpathlineto{\pgfqpoint{5.853088in}{3.406801in}}%
\pgfpathlineto{\pgfqpoint{5.838640in}{3.398928in}}%
\pgfpathlineto{\pgfqpoint{5.824210in}{3.391226in}}%
\pgfpathlineto{\pgfqpoint{5.809797in}{3.383694in}}%
\pgfpathlineto{\pgfqpoint{5.795402in}{3.376333in}}%
\pgfpathlineto{\pgfqpoint{5.788201in}{3.370309in}}%
\pgfpathlineto{\pgfqpoint{5.781001in}{3.364489in}}%
\pgfpathlineto{\pgfqpoint{5.773801in}{3.358864in}}%
\pgfpathlineto{\pgfqpoint{5.766601in}{3.353427in}}%
\pgfpathclose%
\pgfusepath{fill}%
\end{pgfscope}%
\begin{pgfscope}%
\pgfpathrectangle{\pgfqpoint{1.150000in}{0.150000in}}{\pgfqpoint{5.700000in}{5.700000in}}%
\pgfusepath{clip}%
\pgfsetbuttcap%
\pgfsetroundjoin%
\definecolor{currentfill}{rgb}{0.269308,0.218818,0.509577}%
\pgfsetfillcolor{currentfill}%
\pgfsetfillopacity{0.800000}%
\pgfsetlinewidth{0.000000pt}%
\definecolor{currentstroke}{rgb}{0.000000,0.000000,0.000000}%
\pgfsetstrokecolor{currentstroke}%
\pgfsetdash{}{0pt}%
\pgfpathmoveto{\pgfqpoint{4.237204in}{2.350705in}}%
\pgfpathlineto{\pgfqpoint{4.250939in}{2.354164in}}%
\pgfpathlineto{\pgfqpoint{4.264684in}{2.357813in}}%
\pgfpathlineto{\pgfqpoint{4.278439in}{2.361652in}}%
\pgfpathlineto{\pgfqpoint{4.292206in}{2.365680in}}%
\pgfpathlineto{\pgfqpoint{4.300036in}{2.375458in}}%
\pgfpathlineto{\pgfqpoint{4.307861in}{2.385192in}}%
\pgfpathlineto{\pgfqpoint{4.315680in}{2.394884in}}%
\pgfpathlineto{\pgfqpoint{4.323494in}{2.404535in}}%
\pgfpathlineto{\pgfqpoint{4.309734in}{2.400612in}}%
\pgfpathlineto{\pgfqpoint{4.295985in}{2.396877in}}%
\pgfpathlineto{\pgfqpoint{4.282248in}{2.393333in}}%
\pgfpathlineto{\pgfqpoint{4.268521in}{2.389978in}}%
\pgfpathlineto{\pgfqpoint{4.260699in}{2.380211in}}%
\pgfpathlineto{\pgfqpoint{4.252873in}{2.370411in}}%
\pgfpathlineto{\pgfqpoint{4.245041in}{2.360576in}}%
\pgfpathlineto{\pgfqpoint{4.237204in}{2.350705in}}%
\pgfpathclose%
\pgfusepath{fill}%
\end{pgfscope}%
\begin{pgfscope}%
\pgfpathrectangle{\pgfqpoint{1.150000in}{0.150000in}}{\pgfqpoint{5.700000in}{5.700000in}}%
\pgfusepath{clip}%
\pgfsetbuttcap%
\pgfsetroundjoin%
\definecolor{currentfill}{rgb}{0.120565,0.596422,0.543611}%
\pgfsetfillcolor{currentfill}%
\pgfsetfillopacity{0.800000}%
\pgfsetlinewidth{0.000000pt}%
\definecolor{currentstroke}{rgb}{0.000000,0.000000,0.000000}%
\pgfsetstrokecolor{currentstroke}%
\pgfsetdash{}{0pt}%
\pgfpathmoveto{\pgfqpoint{5.853088in}{3.406801in}}%
\pgfpathlineto{\pgfqpoint{5.867555in}{3.414845in}}%
\pgfpathlineto{\pgfqpoint{5.882040in}{3.423060in}}%
\pgfpathlineto{\pgfqpoint{5.896543in}{3.431446in}}%
\pgfpathlineto{\pgfqpoint{5.911065in}{3.440003in}}%
\pgfpathlineto{\pgfqpoint{5.918197in}{3.444742in}}%
\pgfpathlineto{\pgfqpoint{5.925329in}{3.449687in}}%
\pgfpathlineto{\pgfqpoint{5.932464in}{3.454845in}}%
\pgfpathlineto{\pgfqpoint{5.939600in}{3.460226in}}%
\pgfpathlineto{\pgfqpoint{5.925115in}{3.452432in}}%
\pgfpathlineto{\pgfqpoint{5.910648in}{3.444808in}}%
\pgfpathlineto{\pgfqpoint{5.896200in}{3.437353in}}%
\pgfpathlineto{\pgfqpoint{5.881769in}{3.430068in}}%
\pgfpathlineto{\pgfqpoint{5.874596in}{3.423916in}}%
\pgfpathlineto{\pgfqpoint{5.867425in}{3.417993in}}%
\pgfpathlineto{\pgfqpoint{5.860256in}{3.412291in}}%
\pgfpathlineto{\pgfqpoint{5.853088in}{3.406801in}}%
\pgfpathclose%
\pgfusepath{fill}%
\end{pgfscope}%
\begin{pgfscope}%
\pgfpathrectangle{\pgfqpoint{1.150000in}{0.150000in}}{\pgfqpoint{5.700000in}{5.700000in}}%
\pgfusepath{clip}%
\pgfsetbuttcap%
\pgfsetroundjoin%
\definecolor{currentfill}{rgb}{0.280894,0.078907,0.402329}%
\pgfsetfillcolor{currentfill}%
\pgfsetfillopacity{0.800000}%
\pgfsetlinewidth{0.000000pt}%
\definecolor{currentstroke}{rgb}{0.000000,0.000000,0.000000}%
\pgfsetstrokecolor{currentstroke}%
\pgfsetdash{}{0pt}%
\pgfpathmoveto{\pgfqpoint{3.633293in}{2.043153in}}%
\pgfpathlineto{\pgfqpoint{3.646857in}{2.040909in}}%
\pgfpathlineto{\pgfqpoint{3.660426in}{2.038872in}}%
\pgfpathlineto{\pgfqpoint{3.674000in}{2.037040in}}%
\pgfpathlineto{\pgfqpoint{3.687580in}{2.035414in}}%
\pgfpathlineto{\pgfqpoint{3.695608in}{2.045786in}}%
\pgfpathlineto{\pgfqpoint{3.703630in}{2.056161in}}%
\pgfpathlineto{\pgfqpoint{3.711647in}{2.066538in}}%
\pgfpathlineto{\pgfqpoint{3.719659in}{2.076916in}}%
\pgfpathlineto{\pgfqpoint{3.706088in}{2.078423in}}%
\pgfpathlineto{\pgfqpoint{3.692524in}{2.080136in}}%
\pgfpathlineto{\pgfqpoint{3.678965in}{2.082055in}}%
\pgfpathlineto{\pgfqpoint{3.665411in}{2.084180in}}%
\pgfpathlineto{\pgfqpoint{3.657390in}{2.073909in}}%
\pgfpathlineto{\pgfqpoint{3.649363in}{2.063647in}}%
\pgfpathlineto{\pgfqpoint{3.641331in}{2.053395in}}%
\pgfpathlineto{\pgfqpoint{3.633293in}{2.043153in}}%
\pgfpathclose%
\pgfusepath{fill}%
\end{pgfscope}%
\begin{pgfscope}%
\pgfpathrectangle{\pgfqpoint{1.150000in}{0.150000in}}{\pgfqpoint{5.700000in}{5.700000in}}%
\pgfusepath{clip}%
\pgfsetbuttcap%
\pgfsetroundjoin%
\definecolor{currentfill}{rgb}{0.190631,0.407061,0.556089}%
\pgfsetfillcolor{currentfill}%
\pgfsetfillopacity{0.800000}%
\pgfsetlinewidth{0.000000pt}%
\definecolor{currentstroke}{rgb}{0.000000,0.000000,0.000000}%
\pgfsetstrokecolor{currentstroke}%
\pgfsetdash{}{0pt}%
\pgfpathmoveto{\pgfqpoint{4.958890in}{2.835003in}}%
\pgfpathlineto{\pgfqpoint{4.972950in}{2.842254in}}%
\pgfpathlineto{\pgfqpoint{4.987026in}{2.849684in}}%
\pgfpathlineto{\pgfqpoint{5.001117in}{2.857294in}}%
\pgfpathlineto{\pgfqpoint{5.015223in}{2.865084in}}%
\pgfpathlineto{\pgfqpoint{5.022762in}{2.871770in}}%
\pgfpathlineto{\pgfqpoint{5.030295in}{2.878457in}}%
\pgfpathlineto{\pgfqpoint{5.037823in}{2.885148in}}%
\pgfpathlineto{\pgfqpoint{5.045345in}{2.891849in}}%
\pgfpathlineto{\pgfqpoint{5.031254in}{2.884460in}}%
\pgfpathlineto{\pgfqpoint{5.017179in}{2.877251in}}%
\pgfpathlineto{\pgfqpoint{5.003119in}{2.870220in}}%
\pgfpathlineto{\pgfqpoint{4.989074in}{2.863368in}}%
\pgfpathlineto{\pgfqpoint{4.981536in}{2.856256in}}%
\pgfpathlineto{\pgfqpoint{4.973993in}{2.849160in}}%
\pgfpathlineto{\pgfqpoint{4.966444in}{2.842078in}}%
\pgfpathlineto{\pgfqpoint{4.958890in}{2.835003in}}%
\pgfpathclose%
\pgfusepath{fill}%
\end{pgfscope}%
\begin{pgfscope}%
\pgfpathrectangle{\pgfqpoint{1.150000in}{0.150000in}}{\pgfqpoint{5.700000in}{5.700000in}}%
\pgfusepath{clip}%
\pgfsetbuttcap%
\pgfsetroundjoin%
\definecolor{currentfill}{rgb}{0.283229,0.120777,0.440584}%
\pgfsetfillcolor{currentfill}%
\pgfsetfillopacity{0.800000}%
\pgfsetlinewidth{0.000000pt}%
\definecolor{currentstroke}{rgb}{0.000000,0.000000,0.000000}%
\pgfsetstrokecolor{currentstroke}%
\pgfsetdash{}{0pt}%
\pgfpathmoveto{\pgfqpoint{2.927767in}{2.156493in}}%
\pgfpathlineto{\pgfqpoint{2.941378in}{2.143993in}}%
\pgfpathlineto{\pgfqpoint{2.954985in}{2.131752in}}%
\pgfpathlineto{\pgfqpoint{2.968589in}{2.119767in}}%
\pgfpathlineto{\pgfqpoint{2.982189in}{2.108036in}}%
\pgfpathlineto{\pgfqpoint{2.990510in}{2.115155in}}%
\pgfpathlineto{\pgfqpoint{2.998823in}{2.122389in}}%
\pgfpathlineto{\pgfqpoint{3.007127in}{2.129736in}}%
\pgfpathlineto{\pgfqpoint{3.015422in}{2.137195in}}%
\pgfpathlineto{\pgfqpoint{3.001846in}{2.148643in}}%
\pgfpathlineto{\pgfqpoint{2.988266in}{2.160346in}}%
\pgfpathlineto{\pgfqpoint{2.974683in}{2.172305in}}%
\pgfpathlineto{\pgfqpoint{2.961097in}{2.184521in}}%
\pgfpathlineto{\pgfqpoint{2.952778in}{2.177334in}}%
\pgfpathlineto{\pgfqpoint{2.944451in}{2.170265in}}%
\pgfpathlineto{\pgfqpoint{2.936114in}{2.163317in}}%
\pgfpathlineto{\pgfqpoint{2.927767in}{2.156493in}}%
\pgfpathclose%
\pgfusepath{fill}%
\end{pgfscope}%
\begin{pgfscope}%
\pgfpathrectangle{\pgfqpoint{1.150000in}{0.150000in}}{\pgfqpoint{5.700000in}{5.700000in}}%
\pgfusepath{clip}%
\pgfsetbuttcap%
\pgfsetroundjoin%
\definecolor{currentfill}{rgb}{0.262138,0.242286,0.520837}%
\pgfsetfillcolor{currentfill}%
\pgfsetfillopacity{0.800000}%
\pgfsetlinewidth{0.000000pt}%
\definecolor{currentstroke}{rgb}{0.000000,0.000000,0.000000}%
\pgfsetstrokecolor{currentstroke}%
\pgfsetdash{}{0pt}%
\pgfpathmoveto{\pgfqpoint{4.323494in}{2.404535in}}%
\pgfpathlineto{\pgfqpoint{4.337264in}{2.408647in}}%
\pgfpathlineto{\pgfqpoint{4.351047in}{2.412948in}}%
\pgfpathlineto{\pgfqpoint{4.364840in}{2.417437in}}%
\pgfpathlineto{\pgfqpoint{4.378646in}{2.422114in}}%
\pgfpathlineto{\pgfqpoint{4.386447in}{2.431600in}}%
\pgfpathlineto{\pgfqpoint{4.394242in}{2.441042in}}%
\pgfpathlineto{\pgfqpoint{4.402032in}{2.450440in}}%
\pgfpathlineto{\pgfqpoint{4.409817in}{2.459796in}}%
\pgfpathlineto{\pgfqpoint{4.396019in}{2.455257in}}%
\pgfpathlineto{\pgfqpoint{4.382232in}{2.450905in}}%
\pgfpathlineto{\pgfqpoint{4.368458in}{2.446741in}}%
\pgfpathlineto{\pgfqpoint{4.354694in}{2.442765in}}%
\pgfpathlineto{\pgfqpoint{4.346902in}{2.433260in}}%
\pgfpathlineto{\pgfqpoint{4.339105in}{2.423721in}}%
\pgfpathlineto{\pgfqpoint{4.331302in}{2.414147in}}%
\pgfpathlineto{\pgfqpoint{4.323494in}{2.404535in}}%
\pgfpathclose%
\pgfusepath{fill}%
\end{pgfscope}%
\begin{pgfscope}%
\pgfpathrectangle{\pgfqpoint{1.150000in}{0.150000in}}{\pgfqpoint{5.700000in}{5.700000in}}%
\pgfusepath{clip}%
\pgfsetbuttcap%
\pgfsetroundjoin%
\definecolor{currentfill}{rgb}{0.235526,0.309527,0.542944}%
\pgfsetfillcolor{currentfill}%
\pgfsetfillopacity{0.800000}%
\pgfsetlinewidth{0.000000pt}%
\definecolor{currentstroke}{rgb}{0.000000,0.000000,0.000000}%
\pgfsetstrokecolor{currentstroke}%
\pgfsetdash{}{0pt}%
\pgfpathmoveto{\pgfqpoint{2.544052in}{2.619913in}}%
\pgfpathlineto{\pgfqpoint{2.557879in}{2.599284in}}%
\pgfpathlineto{\pgfqpoint{2.571693in}{2.578981in}}%
\pgfpathlineto{\pgfqpoint{2.585497in}{2.559001in}}%
\pgfpathlineto{\pgfqpoint{2.599289in}{2.539341in}}%
\pgfpathlineto{\pgfqpoint{2.607810in}{2.544359in}}%
\pgfpathlineto{\pgfqpoint{2.616320in}{2.549550in}}%
\pgfpathlineto{\pgfqpoint{2.624817in}{2.554912in}}%
\pgfpathlineto{\pgfqpoint{2.633303in}{2.560442in}}%
\pgfpathlineto{\pgfqpoint{2.619544in}{2.579807in}}%
\pgfpathlineto{\pgfqpoint{2.605773in}{2.599491in}}%
\pgfpathlineto{\pgfqpoint{2.591992in}{2.619497in}}%
\pgfpathlineto{\pgfqpoint{2.578200in}{2.639829in}}%
\pgfpathlineto{\pgfqpoint{2.569682in}{2.634583in}}%
\pgfpathlineto{\pgfqpoint{2.561151in}{2.629513in}}%
\pgfpathlineto{\pgfqpoint{2.552608in}{2.624622in}}%
\pgfpathlineto{\pgfqpoint{2.544052in}{2.619913in}}%
\pgfpathclose%
\pgfusepath{fill}%
\end{pgfscope}%
\begin{pgfscope}%
\pgfpathrectangle{\pgfqpoint{1.150000in}{0.150000in}}{\pgfqpoint{5.700000in}{5.700000in}}%
\pgfusepath{clip}%
\pgfsetbuttcap%
\pgfsetroundjoin%
\definecolor{currentfill}{rgb}{0.180629,0.429975,0.557282}%
\pgfsetfillcolor{currentfill}%
\pgfsetfillopacity{0.800000}%
\pgfsetlinewidth{0.000000pt}%
\definecolor{currentstroke}{rgb}{0.000000,0.000000,0.000000}%
\pgfsetstrokecolor{currentstroke}%
\pgfsetdash{}{0pt}%
\pgfpathmoveto{\pgfqpoint{5.045345in}{2.891849in}}%
\pgfpathlineto{\pgfqpoint{5.059451in}{2.899417in}}%
\pgfpathlineto{\pgfqpoint{5.073573in}{2.907164in}}%
\pgfpathlineto{\pgfqpoint{5.087710in}{2.915090in}}%
\pgfpathlineto{\pgfqpoint{5.101864in}{2.923195in}}%
\pgfpathlineto{\pgfqpoint{5.109364in}{2.929489in}}%
\pgfpathlineto{\pgfqpoint{5.116858in}{2.935794in}}%
\pgfpathlineto{\pgfqpoint{5.124346in}{2.942115in}}%
\pgfpathlineto{\pgfqpoint{5.131830in}{2.948458in}}%
\pgfpathlineto{\pgfqpoint{5.117694in}{2.940788in}}%
\pgfpathlineto{\pgfqpoint{5.103573in}{2.933295in}}%
\pgfpathlineto{\pgfqpoint{5.089468in}{2.925981in}}%
\pgfpathlineto{\pgfqpoint{5.075379in}{2.918846in}}%
\pgfpathlineto{\pgfqpoint{5.067878in}{2.912058in}}%
\pgfpathlineto{\pgfqpoint{5.060373in}{2.905300in}}%
\pgfpathlineto{\pgfqpoint{5.052861in}{2.898565in}}%
\pgfpathlineto{\pgfqpoint{5.045345in}{2.891849in}}%
\pgfpathclose%
\pgfusepath{fill}%
\end{pgfscope}%
\begin{pgfscope}%
\pgfpathrectangle{\pgfqpoint{1.150000in}{0.150000in}}{\pgfqpoint{5.700000in}{5.700000in}}%
\pgfusepath{clip}%
\pgfsetbuttcap%
\pgfsetroundjoin%
\definecolor{currentfill}{rgb}{0.252194,0.269783,0.531579}%
\pgfsetfillcolor{currentfill}%
\pgfsetfillopacity{0.800000}%
\pgfsetlinewidth{0.000000pt}%
\definecolor{currentstroke}{rgb}{0.000000,0.000000,0.000000}%
\pgfsetstrokecolor{currentstroke}%
\pgfsetdash{}{0pt}%
\pgfpathmoveto{\pgfqpoint{4.409817in}{2.459796in}}%
\pgfpathlineto{\pgfqpoint{4.423626in}{2.464523in}}%
\pgfpathlineto{\pgfqpoint{4.437448in}{2.469437in}}%
\pgfpathlineto{\pgfqpoint{4.451282in}{2.474538in}}%
\pgfpathlineto{\pgfqpoint{4.465128in}{2.479825in}}%
\pgfpathlineto{\pgfqpoint{4.472899in}{2.488984in}}%
\pgfpathlineto{\pgfqpoint{4.480664in}{2.498098in}}%
\pgfpathlineto{\pgfqpoint{4.488424in}{2.507169in}}%
\pgfpathlineto{\pgfqpoint{4.496178in}{2.516199in}}%
\pgfpathlineto{\pgfqpoint{4.482340in}{2.511082in}}%
\pgfpathlineto{\pgfqpoint{4.468514in}{2.506151in}}%
\pgfpathlineto{\pgfqpoint{4.454700in}{2.501406in}}%
\pgfpathlineto{\pgfqpoint{4.440899in}{2.496848in}}%
\pgfpathlineto{\pgfqpoint{4.433137in}{2.487637in}}%
\pgfpathlineto{\pgfqpoint{4.425369in}{2.478393in}}%
\pgfpathlineto{\pgfqpoint{4.417595in}{2.469113in}}%
\pgfpathlineto{\pgfqpoint{4.409817in}{2.459796in}}%
\pgfpathclose%
\pgfusepath{fill}%
\end{pgfscope}%
\begin{pgfscope}%
\pgfpathrectangle{\pgfqpoint{1.150000in}{0.150000in}}{\pgfqpoint{5.700000in}{5.700000in}}%
\pgfusepath{clip}%
\pgfsetbuttcap%
\pgfsetroundjoin%
\definecolor{currentfill}{rgb}{0.279566,0.067836,0.391917}%
\pgfsetfillcolor{currentfill}%
\pgfsetfillopacity{0.800000}%
\pgfsetlinewidth{0.000000pt}%
\definecolor{currentstroke}{rgb}{0.000000,0.000000,0.000000}%
\pgfsetstrokecolor{currentstroke}%
\pgfsetdash{}{0pt}%
\pgfpathmoveto{\pgfqpoint{3.546840in}{2.013975in}}%
\pgfpathlineto{\pgfqpoint{3.560396in}{2.010745in}}%
\pgfpathlineto{\pgfqpoint{3.573955in}{2.007726in}}%
\pgfpathlineto{\pgfqpoint{3.587520in}{2.004916in}}%
\pgfpathlineto{\pgfqpoint{3.601089in}{2.002314in}}%
\pgfpathlineto{\pgfqpoint{3.609148in}{2.012503in}}%
\pgfpathlineto{\pgfqpoint{3.617202in}{2.022706in}}%
\pgfpathlineto{\pgfqpoint{3.625250in}{2.032923in}}%
\pgfpathlineto{\pgfqpoint{3.633293in}{2.043153in}}%
\pgfpathlineto{\pgfqpoint{3.619735in}{2.045605in}}%
\pgfpathlineto{\pgfqpoint{3.606182in}{2.048265in}}%
\pgfpathlineto{\pgfqpoint{3.592633in}{2.051134in}}%
\pgfpathlineto{\pgfqpoint{3.579090in}{2.054213in}}%
\pgfpathlineto{\pgfqpoint{3.571036in}{2.044121in}}%
\pgfpathlineto{\pgfqpoint{3.562976in}{2.034051in}}%
\pgfpathlineto{\pgfqpoint{3.554911in}{2.024001in}}%
\pgfpathlineto{\pgfqpoint{3.546840in}{2.013975in}}%
\pgfpathclose%
\pgfusepath{fill}%
\end{pgfscope}%
\begin{pgfscope}%
\pgfpathrectangle{\pgfqpoint{1.150000in}{0.150000in}}{\pgfqpoint{5.700000in}{5.700000in}}%
\pgfusepath{clip}%
\pgfsetbuttcap%
\pgfsetroundjoin%
\definecolor{currentfill}{rgb}{0.119483,0.614817,0.537692}%
\pgfsetfillcolor{currentfill}%
\pgfsetfillopacity{0.800000}%
\pgfsetlinewidth{0.000000pt}%
\definecolor{currentstroke}{rgb}{0.000000,0.000000,0.000000}%
\pgfsetstrokecolor{currentstroke}%
\pgfsetdash{}{0pt}%
\pgfpathmoveto{\pgfqpoint{5.939600in}{3.460226in}}%
\pgfpathlineto{\pgfqpoint{5.954103in}{3.468191in}}%
\pgfpathlineto{\pgfqpoint{5.968625in}{3.476326in}}%
\pgfpathlineto{\pgfqpoint{5.983165in}{3.484630in}}%
\pgfpathlineto{\pgfqpoint{5.997725in}{3.493106in}}%
\pgfpathlineto{\pgfqpoint{6.004825in}{3.497934in}}%
\pgfpathlineto{\pgfqpoint{6.011927in}{3.502995in}}%
\pgfpathlineto{\pgfqpoint{6.019032in}{3.508297in}}%
\pgfpathlineto{\pgfqpoint{6.004503in}{3.500415in}}%
\pgfpathlineto{\pgfqpoint{5.989991in}{3.492703in}}%
\pgfpathlineto{\pgfqpoint{5.975499in}{3.485161in}}%
\pgfpathlineto{\pgfqpoint{5.961024in}{3.477787in}}%
\pgfpathlineto{\pgfqpoint{5.953880in}{3.471689in}}%
\pgfpathlineto{\pgfqpoint{5.946738in}{3.465838in}}%
\pgfpathlineto{\pgfqpoint{5.939600in}{3.460226in}}%
\pgfpathclose%
\pgfusepath{fill}%
\end{pgfscope}%
\begin{pgfscope}%
\pgfpathrectangle{\pgfqpoint{1.150000in}{0.150000in}}{\pgfqpoint{5.700000in}{5.700000in}}%
\pgfusepath{clip}%
\pgfsetbuttcap%
\pgfsetroundjoin%
\definecolor{currentfill}{rgb}{0.277018,0.050344,0.375715}%
\pgfsetfillcolor{currentfill}%
\pgfsetfillopacity{0.800000}%
\pgfsetlinewidth{0.000000pt}%
\definecolor{currentstroke}{rgb}{0.000000,0.000000,0.000000}%
\pgfsetstrokecolor{currentstroke}%
\pgfsetdash{}{0pt}%
\pgfpathmoveto{\pgfqpoint{3.319357in}{1.994978in}}%
\pgfpathlineto{\pgfqpoint{3.332903in}{1.988756in}}%
\pgfpathlineto{\pgfqpoint{3.346451in}{1.982755in}}%
\pgfpathlineto{\pgfqpoint{3.360000in}{1.976976in}}%
\pgfpathlineto{\pgfqpoint{3.373552in}{1.971417in}}%
\pgfpathlineto{\pgfqpoint{3.381697in}{1.980760in}}%
\pgfpathlineto{\pgfqpoint{3.389836in}{1.990155in}}%
\pgfpathlineto{\pgfqpoint{3.397969in}{1.999598in}}%
\pgfpathlineto{\pgfqpoint{3.406095in}{2.009088in}}%
\pgfpathlineto{\pgfqpoint{3.392558in}{2.014433in}}%
\pgfpathlineto{\pgfqpoint{3.379024in}{2.019998in}}%
\pgfpathlineto{\pgfqpoint{3.365491in}{2.025784in}}%
\pgfpathlineto{\pgfqpoint{3.351961in}{2.031793in}}%
\pgfpathlineto{\pgfqpoint{3.343820in}{2.022505in}}%
\pgfpathlineto{\pgfqpoint{3.335672in}{2.013273in}}%
\pgfpathlineto{\pgfqpoint{3.327518in}{2.004096in}}%
\pgfpathlineto{\pgfqpoint{3.319357in}{1.994978in}}%
\pgfpathclose%
\pgfusepath{fill}%
\end{pgfscope}%
\begin{pgfscope}%
\pgfpathrectangle{\pgfqpoint{1.150000in}{0.150000in}}{\pgfqpoint{5.700000in}{5.700000in}}%
\pgfusepath{clip}%
\pgfsetbuttcap%
\pgfsetroundjoin%
\definecolor{currentfill}{rgb}{0.278791,0.062145,0.386592}%
\pgfsetfillcolor{currentfill}%
\pgfsetfillopacity{0.800000}%
\pgfsetlinewidth{0.000000pt}%
\definecolor{currentstroke}{rgb}{0.000000,0.000000,0.000000}%
\pgfsetstrokecolor{currentstroke}%
\pgfsetdash{}{0pt}%
\pgfpathmoveto{\pgfqpoint{3.178182in}{2.019032in}}%
\pgfpathlineto{\pgfqpoint{3.191739in}{2.010740in}}%
\pgfpathlineto{\pgfqpoint{3.205295in}{2.002680in}}%
\pgfpathlineto{\pgfqpoint{3.218852in}{1.994851in}}%
\pgfpathlineto{\pgfqpoint{3.232410in}{1.987252in}}%
\pgfpathlineto{\pgfqpoint{3.240615in}{1.995859in}}%
\pgfpathlineto{\pgfqpoint{3.248813in}{2.004540in}}%
\pgfpathlineto{\pgfqpoint{3.257005in}{2.013293in}}%
\pgfpathlineto{\pgfqpoint{3.265189in}{2.022117in}}%
\pgfpathlineto{\pgfqpoint{3.251649in}{2.029469in}}%
\pgfpathlineto{\pgfqpoint{3.238111in}{2.037051in}}%
\pgfpathlineto{\pgfqpoint{3.224572in}{2.044864in}}%
\pgfpathlineto{\pgfqpoint{3.211034in}{2.052909in}}%
\pgfpathlineto{\pgfqpoint{3.202832in}{2.044321in}}%
\pgfpathlineto{\pgfqpoint{3.194623in}{2.035810in}}%
\pgfpathlineto{\pgfqpoint{3.186406in}{2.027380in}}%
\pgfpathlineto{\pgfqpoint{3.178182in}{2.019032in}}%
\pgfpathclose%
\pgfusepath{fill}%
\end{pgfscope}%
\begin{pgfscope}%
\pgfpathrectangle{\pgfqpoint{1.150000in}{0.150000in}}{\pgfqpoint{5.700000in}{5.700000in}}%
\pgfusepath{clip}%
\pgfsetbuttcap%
\pgfsetroundjoin%
\definecolor{currentfill}{rgb}{0.282656,0.100196,0.422160}%
\pgfsetfillcolor{currentfill}%
\pgfsetfillopacity{0.800000}%
\pgfsetlinewidth{0.000000pt}%
\definecolor{currentstroke}{rgb}{0.000000,0.000000,0.000000}%
\pgfsetstrokecolor{currentstroke}%
\pgfsetdash{}{0pt}%
\pgfpathmoveto{\pgfqpoint{2.982189in}{2.108036in}}%
\pgfpathlineto{\pgfqpoint{2.995786in}{2.096558in}}%
\pgfpathlineto{\pgfqpoint{3.009380in}{2.085331in}}%
\pgfpathlineto{\pgfqpoint{3.022971in}{2.074353in}}%
\pgfpathlineto{\pgfqpoint{3.036560in}{2.063623in}}%
\pgfpathlineto{\pgfqpoint{3.044858in}{2.071035in}}%
\pgfpathlineto{\pgfqpoint{3.053148in}{2.078555in}}%
\pgfpathlineto{\pgfqpoint{3.061429in}{2.086180in}}%
\pgfpathlineto{\pgfqpoint{3.069702in}{2.093908in}}%
\pgfpathlineto{\pgfqpoint{3.056135in}{2.104357in}}%
\pgfpathlineto{\pgfqpoint{3.042567in}{2.115053in}}%
\pgfpathlineto{\pgfqpoint{3.028996in}{2.125999in}}%
\pgfpathlineto{\pgfqpoint{3.015422in}{2.137195in}}%
\pgfpathlineto{\pgfqpoint{3.007127in}{2.129736in}}%
\pgfpathlineto{\pgfqpoint{2.998823in}{2.122389in}}%
\pgfpathlineto{\pgfqpoint{2.990510in}{2.115155in}}%
\pgfpathlineto{\pgfqpoint{2.982189in}{2.108036in}}%
\pgfpathclose%
\pgfusepath{fill}%
\end{pgfscope}%
\begin{pgfscope}%
\pgfpathrectangle{\pgfqpoint{1.150000in}{0.150000in}}{\pgfqpoint{5.700000in}{5.700000in}}%
\pgfusepath{clip}%
\pgfsetbuttcap%
\pgfsetroundjoin%
\definecolor{currentfill}{rgb}{0.172719,0.448791,0.557885}%
\pgfsetfillcolor{currentfill}%
\pgfsetfillopacity{0.800000}%
\pgfsetlinewidth{0.000000pt}%
\definecolor{currentstroke}{rgb}{0.000000,0.000000,0.000000}%
\pgfsetstrokecolor{currentstroke}%
\pgfsetdash{}{0pt}%
\pgfpathmoveto{\pgfqpoint{5.131830in}{2.948458in}}%
\pgfpathlineto{\pgfqpoint{5.145982in}{2.956306in}}%
\pgfpathlineto{\pgfqpoint{5.160150in}{2.964333in}}%
\pgfpathlineto{\pgfqpoint{5.174334in}{2.972538in}}%
\pgfpathlineto{\pgfqpoint{5.188535in}{2.980920in}}%
\pgfpathlineto{\pgfqpoint{5.195994in}{2.986834in}}%
\pgfpathlineto{\pgfqpoint{5.203448in}{2.992772in}}%
\pgfpathlineto{\pgfqpoint{5.210897in}{2.998738in}}%
\pgfpathlineto{\pgfqpoint{5.218341in}{3.004739in}}%
\pgfpathlineto{\pgfqpoint{5.204159in}{2.996824in}}%
\pgfpathlineto{\pgfqpoint{5.189994in}{2.989087in}}%
\pgfpathlineto{\pgfqpoint{5.175844in}{2.981527in}}%
\pgfpathlineto{\pgfqpoint{5.161711in}{2.974144in}}%
\pgfpathlineto{\pgfqpoint{5.154248in}{2.967665in}}%
\pgfpathlineto{\pgfqpoint{5.146780in}{2.961228in}}%
\pgfpathlineto{\pgfqpoint{5.139308in}{2.954827in}}%
\pgfpathlineto{\pgfqpoint{5.131830in}{2.948458in}}%
\pgfpathclose%
\pgfusepath{fill}%
\end{pgfscope}%
\begin{pgfscope}%
\pgfpathrectangle{\pgfqpoint{1.150000in}{0.150000in}}{\pgfqpoint{5.700000in}{5.700000in}}%
\pgfusepath{clip}%
\pgfsetbuttcap%
\pgfsetroundjoin%
\definecolor{currentfill}{rgb}{0.220057,0.343307,0.549413}%
\pgfsetfillcolor{currentfill}%
\pgfsetfillopacity{0.800000}%
\pgfsetlinewidth{0.000000pt}%
\definecolor{currentstroke}{rgb}{0.000000,0.000000,0.000000}%
\pgfsetstrokecolor{currentstroke}%
\pgfsetdash{}{0pt}%
\pgfpathmoveto{\pgfqpoint{2.488623in}{2.705751in}}%
\pgfpathlineto{\pgfqpoint{2.502500in}{2.683786in}}%
\pgfpathlineto{\pgfqpoint{2.516363in}{2.662161in}}%
\pgfpathlineto{\pgfqpoint{2.530214in}{2.640871in}}%
\pgfpathlineto{\pgfqpoint{2.544052in}{2.619913in}}%
\pgfpathlineto{\pgfqpoint{2.552608in}{2.624622in}}%
\pgfpathlineto{\pgfqpoint{2.561151in}{2.629513in}}%
\pgfpathlineto{\pgfqpoint{2.569682in}{2.634583in}}%
\pgfpathlineto{\pgfqpoint{2.578200in}{2.639829in}}%
\pgfpathlineto{\pgfqpoint{2.564396in}{2.660489in}}%
\pgfpathlineto{\pgfqpoint{2.550580in}{2.681480in}}%
\pgfpathlineto{\pgfqpoint{2.536751in}{2.702806in}}%
\pgfpathlineto{\pgfqpoint{2.522910in}{2.724470in}}%
\pgfpathlineto{\pgfqpoint{2.514358in}{2.719510in}}%
\pgfpathlineto{\pgfqpoint{2.505793in}{2.714736in}}%
\pgfpathlineto{\pgfqpoint{2.497215in}{2.710148in}}%
\pgfpathlineto{\pgfqpoint{2.488623in}{2.705751in}}%
\pgfpathclose%
\pgfusepath{fill}%
\end{pgfscope}%
\begin{pgfscope}%
\pgfpathrectangle{\pgfqpoint{1.150000in}{0.150000in}}{\pgfqpoint{5.700000in}{5.700000in}}%
\pgfusepath{clip}%
\pgfsetbuttcap%
\pgfsetroundjoin%
\definecolor{currentfill}{rgb}{0.243113,0.292092,0.538516}%
\pgfsetfillcolor{currentfill}%
\pgfsetfillopacity{0.800000}%
\pgfsetlinewidth{0.000000pt}%
\definecolor{currentstroke}{rgb}{0.000000,0.000000,0.000000}%
\pgfsetstrokecolor{currentstroke}%
\pgfsetdash{}{0pt}%
\pgfpathmoveto{\pgfqpoint{4.496178in}{2.516199in}}%
\pgfpathlineto{\pgfqpoint{4.510029in}{2.521502in}}%
\pgfpathlineto{\pgfqpoint{4.523892in}{2.526991in}}%
\pgfpathlineto{\pgfqpoint{4.537768in}{2.532665in}}%
\pgfpathlineto{\pgfqpoint{4.551657in}{2.538525in}}%
\pgfpathlineto{\pgfqpoint{4.559396in}{2.547326in}}%
\pgfpathlineto{\pgfqpoint{4.567130in}{2.556082in}}%
\pgfpathlineto{\pgfqpoint{4.574859in}{2.564797in}}%
\pgfpathlineto{\pgfqpoint{4.582581in}{2.573474in}}%
\pgfpathlineto{\pgfqpoint{4.568701in}{2.567817in}}%
\pgfpathlineto{\pgfqpoint{4.554834in}{2.562345in}}%
\pgfpathlineto{\pgfqpoint{4.540979in}{2.557058in}}%
\pgfpathlineto{\pgfqpoint{4.527138in}{2.551957in}}%
\pgfpathlineto{\pgfqpoint{4.519406in}{2.543067in}}%
\pgfpathlineto{\pgfqpoint{4.511669in}{2.534145in}}%
\pgfpathlineto{\pgfqpoint{4.503926in}{2.525190in}}%
\pgfpathlineto{\pgfqpoint{4.496178in}{2.516199in}}%
\pgfpathclose%
\pgfusepath{fill}%
\end{pgfscope}%
\begin{pgfscope}%
\pgfpathrectangle{\pgfqpoint{1.150000in}{0.150000in}}{\pgfqpoint{5.700000in}{5.700000in}}%
\pgfusepath{clip}%
\pgfsetbuttcap%
\pgfsetroundjoin%
\definecolor{currentfill}{rgb}{0.165117,0.467423,0.558141}%
\pgfsetfillcolor{currentfill}%
\pgfsetfillopacity{0.800000}%
\pgfsetlinewidth{0.000000pt}%
\definecolor{currentstroke}{rgb}{0.000000,0.000000,0.000000}%
\pgfsetstrokecolor{currentstroke}%
\pgfsetdash{}{0pt}%
\pgfpathmoveto{\pgfqpoint{5.218341in}{3.004739in}}%
\pgfpathlineto{\pgfqpoint{5.232538in}{3.012831in}}%
\pgfpathlineto{\pgfqpoint{5.246753in}{3.021100in}}%
\pgfpathlineto{\pgfqpoint{5.260984in}{3.029547in}}%
\pgfpathlineto{\pgfqpoint{5.275231in}{3.038171in}}%
\pgfpathlineto{\pgfqpoint{5.282650in}{3.043723in}}%
\pgfpathlineto{\pgfqpoint{5.290063in}{3.049312in}}%
\pgfpathlineto{\pgfqpoint{5.297471in}{3.054944in}}%
\pgfpathlineto{\pgfqpoint{5.304874in}{3.060625in}}%
\pgfpathlineto{\pgfqpoint{5.290648in}{3.052503in}}%
\pgfpathlineto{\pgfqpoint{5.276438in}{3.044557in}}%
\pgfpathlineto{\pgfqpoint{5.262244in}{3.036787in}}%
\pgfpathlineto{\pgfqpoint{5.248066in}{3.029194in}}%
\pgfpathlineto{\pgfqpoint{5.240642in}{3.023002in}}%
\pgfpathlineto{\pgfqpoint{5.233213in}{3.016865in}}%
\pgfpathlineto{\pgfqpoint{5.225779in}{3.010780in}}%
\pgfpathlineto{\pgfqpoint{5.218341in}{3.004739in}}%
\pgfpathclose%
\pgfusepath{fill}%
\end{pgfscope}%
\begin{pgfscope}%
\pgfpathrectangle{\pgfqpoint{1.150000in}{0.150000in}}{\pgfqpoint{5.700000in}{5.700000in}}%
\pgfusepath{clip}%
\pgfsetbuttcap%
\pgfsetroundjoin%
\definecolor{currentfill}{rgb}{0.231674,0.318106,0.544834}%
\pgfsetfillcolor{currentfill}%
\pgfsetfillopacity{0.800000}%
\pgfsetlinewidth{0.000000pt}%
\definecolor{currentstroke}{rgb}{0.000000,0.000000,0.000000}%
\pgfsetstrokecolor{currentstroke}%
\pgfsetdash{}{0pt}%
\pgfpathmoveto{\pgfqpoint{4.582581in}{2.573474in}}%
\pgfpathlineto{\pgfqpoint{4.596474in}{2.579315in}}%
\pgfpathlineto{\pgfqpoint{4.610380in}{2.585341in}}%
\pgfpathlineto{\pgfqpoint{4.624300in}{2.591551in}}%
\pgfpathlineto{\pgfqpoint{4.638233in}{2.597946in}}%
\pgfpathlineto{\pgfqpoint{4.645940in}{2.606362in}}%
\pgfpathlineto{\pgfqpoint{4.653642in}{2.614737in}}%
\pgfpathlineto{\pgfqpoint{4.661337in}{2.623073in}}%
\pgfpathlineto{\pgfqpoint{4.669027in}{2.631374in}}%
\pgfpathlineto{\pgfqpoint{4.655103in}{2.625216in}}%
\pgfpathlineto{\pgfqpoint{4.641193in}{2.619241in}}%
\pgfpathlineto{\pgfqpoint{4.627297in}{2.613451in}}%
\pgfpathlineto{\pgfqpoint{4.613413in}{2.607844in}}%
\pgfpathlineto{\pgfqpoint{4.605714in}{2.599296in}}%
\pgfpathlineto{\pgfqpoint{4.598009in}{2.590720in}}%
\pgfpathlineto{\pgfqpoint{4.590298in}{2.582114in}}%
\pgfpathlineto{\pgfqpoint{4.582581in}{2.573474in}}%
\pgfpathclose%
\pgfusepath{fill}%
\end{pgfscope}%
\begin{pgfscope}%
\pgfpathrectangle{\pgfqpoint{1.150000in}{0.150000in}}{\pgfqpoint{5.700000in}{5.700000in}}%
\pgfusepath{clip}%
\pgfsetbuttcap%
\pgfsetroundjoin%
\definecolor{currentfill}{rgb}{0.277941,0.056324,0.381191}%
\pgfsetfillcolor{currentfill}%
\pgfsetfillopacity{0.800000}%
\pgfsetlinewidth{0.000000pt}%
\definecolor{currentstroke}{rgb}{0.000000,0.000000,0.000000}%
\pgfsetstrokecolor{currentstroke}%
\pgfsetdash{}{0pt}%
\pgfpathmoveto{\pgfqpoint{3.460271in}{1.989885in}}%
\pgfpathlineto{\pgfqpoint{3.473823in}{1.985624in}}%
\pgfpathlineto{\pgfqpoint{3.487378in}{1.981577in}}%
\pgfpathlineto{\pgfqpoint{3.500937in}{1.977742in}}%
\pgfpathlineto{\pgfqpoint{3.514501in}{1.974120in}}%
\pgfpathlineto{\pgfqpoint{3.522594in}{1.984043in}}%
\pgfpathlineto{\pgfqpoint{3.530682in}{1.993994in}}%
\pgfpathlineto{\pgfqpoint{3.538764in}{2.003972in}}%
\pgfpathlineto{\pgfqpoint{3.546840in}{2.013975in}}%
\pgfpathlineto{\pgfqpoint{3.533290in}{2.017415in}}%
\pgfpathlineto{\pgfqpoint{3.519743in}{2.021068in}}%
\pgfpathlineto{\pgfqpoint{3.506200in}{2.024933in}}%
\pgfpathlineto{\pgfqpoint{3.492661in}{2.029012in}}%
\pgfpathlineto{\pgfqpoint{3.484572in}{2.019179in}}%
\pgfpathlineto{\pgfqpoint{3.476478in}{2.009380in}}%
\pgfpathlineto{\pgfqpoint{3.468377in}{1.999615in}}%
\pgfpathlineto{\pgfqpoint{3.460271in}{1.989885in}}%
\pgfpathclose%
\pgfusepath{fill}%
\end{pgfscope}%
\begin{pgfscope}%
\pgfpathrectangle{\pgfqpoint{1.150000in}{0.150000in}}{\pgfqpoint{5.700000in}{5.700000in}}%
\pgfusepath{clip}%
\pgfsetbuttcap%
\pgfsetroundjoin%
\definecolor{currentfill}{rgb}{0.156270,0.489624,0.557936}%
\pgfsetfillcolor{currentfill}%
\pgfsetfillopacity{0.800000}%
\pgfsetlinewidth{0.000000pt}%
\definecolor{currentstroke}{rgb}{0.000000,0.000000,0.000000}%
\pgfsetstrokecolor{currentstroke}%
\pgfsetdash{}{0pt}%
\pgfpathmoveto{\pgfqpoint{5.304874in}{3.060625in}}%
\pgfpathlineto{\pgfqpoint{5.319118in}{3.068924in}}%
\pgfpathlineto{\pgfqpoint{5.333378in}{3.077400in}}%
\pgfpathlineto{\pgfqpoint{5.347655in}{3.086051in}}%
\pgfpathlineto{\pgfqpoint{5.361950in}{3.094880in}}%
\pgfpathlineto{\pgfqpoint{5.369326in}{3.100093in}}%
\pgfpathlineto{\pgfqpoint{5.376698in}{3.105359in}}%
\pgfpathlineto{\pgfqpoint{5.384065in}{3.110684in}}%
\pgfpathlineto{\pgfqpoint{5.391428in}{3.116073in}}%
\pgfpathlineto{\pgfqpoint{5.377157in}{3.107779in}}%
\pgfpathlineto{\pgfqpoint{5.362902in}{3.099662in}}%
\pgfpathlineto{\pgfqpoint{5.348665in}{3.091719in}}%
\pgfpathlineto{\pgfqpoint{5.334443in}{3.083953in}}%
\pgfpathlineto{\pgfqpoint{5.327058in}{3.078019in}}%
\pgfpathlineto{\pgfqpoint{5.319668in}{3.072157in}}%
\pgfpathlineto{\pgfqpoint{5.312273in}{3.066361in}}%
\pgfpathlineto{\pgfqpoint{5.304874in}{3.060625in}}%
\pgfpathclose%
\pgfusepath{fill}%
\end{pgfscope}%
\begin{pgfscope}%
\pgfpathrectangle{\pgfqpoint{1.150000in}{0.150000in}}{\pgfqpoint{5.700000in}{5.700000in}}%
\pgfusepath{clip}%
\pgfsetbuttcap%
\pgfsetroundjoin%
\definecolor{currentfill}{rgb}{0.281446,0.084320,0.407414}%
\pgfsetfillcolor{currentfill}%
\pgfsetfillopacity{0.800000}%
\pgfsetlinewidth{0.000000pt}%
\definecolor{currentstroke}{rgb}{0.000000,0.000000,0.000000}%
\pgfsetstrokecolor{currentstroke}%
\pgfsetdash{}{0pt}%
\pgfpathmoveto{\pgfqpoint{3.036560in}{2.063623in}}%
\pgfpathlineto{\pgfqpoint{3.050146in}{2.053139in}}%
\pgfpathlineto{\pgfqpoint{3.063731in}{2.042900in}}%
\pgfpathlineto{\pgfqpoint{3.077314in}{2.032903in}}%
\pgfpathlineto{\pgfqpoint{3.090895in}{2.023147in}}%
\pgfpathlineto{\pgfqpoint{3.099171in}{2.030852in}}%
\pgfpathlineto{\pgfqpoint{3.107439in}{2.038656in}}%
\pgfpathlineto{\pgfqpoint{3.115699in}{2.046557in}}%
\pgfpathlineto{\pgfqpoint{3.123950in}{2.054553in}}%
\pgfpathlineto{\pgfqpoint{3.110390in}{2.064029in}}%
\pgfpathlineto{\pgfqpoint{3.096829in}{2.073745in}}%
\pgfpathlineto{\pgfqpoint{3.083266in}{2.083704in}}%
\pgfpathlineto{\pgfqpoint{3.069702in}{2.093908in}}%
\pgfpathlineto{\pgfqpoint{3.061429in}{2.086180in}}%
\pgfpathlineto{\pgfqpoint{3.053148in}{2.078555in}}%
\pgfpathlineto{\pgfqpoint{3.044858in}{2.071035in}}%
\pgfpathlineto{\pgfqpoint{3.036560in}{2.063623in}}%
\pgfpathclose%
\pgfusepath{fill}%
\end{pgfscope}%
\begin{pgfscope}%
\pgfpathrectangle{\pgfqpoint{1.150000in}{0.150000in}}{\pgfqpoint{5.700000in}{5.700000in}}%
\pgfusepath{clip}%
\pgfsetbuttcap%
\pgfsetroundjoin%
\definecolor{currentfill}{rgb}{0.221989,0.339161,0.548752}%
\pgfsetfillcolor{currentfill}%
\pgfsetfillopacity{0.800000}%
\pgfsetlinewidth{0.000000pt}%
\definecolor{currentstroke}{rgb}{0.000000,0.000000,0.000000}%
\pgfsetstrokecolor{currentstroke}%
\pgfsetdash{}{0pt}%
\pgfpathmoveto{\pgfqpoint{4.669027in}{2.631374in}}%
\pgfpathlineto{\pgfqpoint{4.682964in}{2.637716in}}%
\pgfpathlineto{\pgfqpoint{4.696914in}{2.644242in}}%
\pgfpathlineto{\pgfqpoint{4.710879in}{2.650950in}}%
\pgfpathlineto{\pgfqpoint{4.724858in}{2.657842in}}%
\pgfpathlineto{\pgfqpoint{4.732531in}{2.665853in}}%
\pgfpathlineto{\pgfqpoint{4.740198in}{2.673827in}}%
\pgfpathlineto{\pgfqpoint{4.747860in}{2.681768in}}%
\pgfpathlineto{\pgfqpoint{4.755515in}{2.689677in}}%
\pgfpathlineto{\pgfqpoint{4.741547in}{2.683055in}}%
\pgfpathlineto{\pgfqpoint{4.727593in}{2.676615in}}%
\pgfpathlineto{\pgfqpoint{4.713653in}{2.670358in}}%
\pgfpathlineto{\pgfqpoint{4.699727in}{2.664284in}}%
\pgfpathlineto{\pgfqpoint{4.692061in}{2.656094in}}%
\pgfpathlineto{\pgfqpoint{4.684388in}{2.647882in}}%
\pgfpathlineto{\pgfqpoint{4.676710in}{2.639643in}}%
\pgfpathlineto{\pgfqpoint{4.669027in}{2.631374in}}%
\pgfpathclose%
\pgfusepath{fill}%
\end{pgfscope}%
\begin{pgfscope}%
\pgfpathrectangle{\pgfqpoint{1.150000in}{0.150000in}}{\pgfqpoint{5.700000in}{5.700000in}}%
\pgfusepath{clip}%
\pgfsetbuttcap%
\pgfsetroundjoin%
\definecolor{currentfill}{rgb}{0.204903,0.375746,0.553533}%
\pgfsetfillcolor{currentfill}%
\pgfsetfillopacity{0.800000}%
\pgfsetlinewidth{0.000000pt}%
\definecolor{currentstroke}{rgb}{0.000000,0.000000,0.000000}%
\pgfsetstrokecolor{currentstroke}%
\pgfsetdash{}{0pt}%
\pgfpathmoveto{\pgfqpoint{2.432980in}{2.797064in}}%
\pgfpathlineto{\pgfqpoint{2.446912in}{2.773710in}}%
\pgfpathlineto{\pgfqpoint{2.460830in}{2.750709in}}%
\pgfpathlineto{\pgfqpoint{2.474734in}{2.728057in}}%
\pgfpathlineto{\pgfqpoint{2.488623in}{2.705751in}}%
\pgfpathlineto{\pgfqpoint{2.497215in}{2.710148in}}%
\pgfpathlineto{\pgfqpoint{2.505793in}{2.714736in}}%
\pgfpathlineto{\pgfqpoint{2.514358in}{2.719510in}}%
\pgfpathlineto{\pgfqpoint{2.522910in}{2.724470in}}%
\pgfpathlineto{\pgfqpoint{2.509056in}{2.746476in}}%
\pgfpathlineto{\pgfqpoint{2.495189in}{2.768826in}}%
\pgfpathlineto{\pgfqpoint{2.481307in}{2.791524in}}%
\pgfpathlineto{\pgfqpoint{2.467412in}{2.814575in}}%
\pgfpathlineto{\pgfqpoint{2.458824in}{2.809904in}}%
\pgfpathlineto{\pgfqpoint{2.450223in}{2.805427in}}%
\pgfpathlineto{\pgfqpoint{2.441608in}{2.801146in}}%
\pgfpathlineto{\pgfqpoint{2.432980in}{2.797064in}}%
\pgfpathclose%
\pgfusepath{fill}%
\end{pgfscope}%
\begin{pgfscope}%
\pgfpathrectangle{\pgfqpoint{1.150000in}{0.150000in}}{\pgfqpoint{5.700000in}{5.700000in}}%
\pgfusepath{clip}%
\pgfsetbuttcap%
\pgfsetroundjoin%
\definecolor{currentfill}{rgb}{0.283229,0.120777,0.440584}%
\pgfsetfillcolor{currentfill}%
\pgfsetfillopacity{0.800000}%
\pgfsetlinewidth{0.000000pt}%
\definecolor{currentstroke}{rgb}{0.000000,0.000000,0.000000}%
\pgfsetstrokecolor{currentstroke}%
\pgfsetdash{}{0pt}%
\pgfpathmoveto{\pgfqpoint{3.860379in}{2.114362in}}%
\pgfpathlineto{\pgfqpoint{3.874003in}{2.114756in}}%
\pgfpathlineto{\pgfqpoint{3.887633in}{2.115347in}}%
\pgfpathlineto{\pgfqpoint{3.901272in}{2.116136in}}%
\pgfpathlineto{\pgfqpoint{3.914919in}{2.117123in}}%
\pgfpathlineto{\pgfqpoint{3.922879in}{2.127702in}}%
\pgfpathlineto{\pgfqpoint{3.930835in}{2.138255in}}%
\pgfpathlineto{\pgfqpoint{3.938786in}{2.148784in}}%
\pgfpathlineto{\pgfqpoint{3.946732in}{2.159288in}}%
\pgfpathlineto{\pgfqpoint{3.933093in}{2.158246in}}%
\pgfpathlineto{\pgfqpoint{3.919462in}{2.157401in}}%
\pgfpathlineto{\pgfqpoint{3.905839in}{2.156754in}}%
\pgfpathlineto{\pgfqpoint{3.892223in}{2.156306in}}%
\pgfpathlineto{\pgfqpoint{3.884270in}{2.145846in}}%
\pgfpathlineto{\pgfqpoint{3.876312in}{2.135368in}}%
\pgfpathlineto{\pgfqpoint{3.868348in}{2.124874in}}%
\pgfpathlineto{\pgfqpoint{3.860379in}{2.114362in}}%
\pgfpathclose%
\pgfusepath{fill}%
\end{pgfscope}%
\begin{pgfscope}%
\pgfpathrectangle{\pgfqpoint{1.150000in}{0.150000in}}{\pgfqpoint{5.700000in}{5.700000in}}%
\pgfusepath{clip}%
\pgfsetbuttcap%
\pgfsetroundjoin%
\definecolor{currentfill}{rgb}{0.282623,0.140926,0.457517}%
\pgfsetfillcolor{currentfill}%
\pgfsetfillopacity{0.800000}%
\pgfsetlinewidth{0.000000pt}%
\definecolor{currentstroke}{rgb}{0.000000,0.000000,0.000000}%
\pgfsetstrokecolor{currentstroke}%
\pgfsetdash{}{0pt}%
\pgfpathmoveto{\pgfqpoint{3.946732in}{2.159288in}}%
\pgfpathlineto{\pgfqpoint{3.960379in}{2.160526in}}%
\pgfpathlineto{\pgfqpoint{3.974035in}{2.161960in}}%
\pgfpathlineto{\pgfqpoint{3.987700in}{2.163590in}}%
\pgfpathlineto{\pgfqpoint{4.001374in}{2.165414in}}%
\pgfpathlineto{\pgfqpoint{4.009307in}{2.175928in}}%
\pgfpathlineto{\pgfqpoint{4.017236in}{2.186409in}}%
\pgfpathlineto{\pgfqpoint{4.025159in}{2.196857in}}%
\pgfpathlineto{\pgfqpoint{4.033077in}{2.207272in}}%
\pgfpathlineto{\pgfqpoint{4.019411in}{2.205424in}}%
\pgfpathlineto{\pgfqpoint{4.005753in}{2.203771in}}%
\pgfpathlineto{\pgfqpoint{3.992104in}{2.202313in}}%
\pgfpathlineto{\pgfqpoint{3.978464in}{2.201051in}}%
\pgfpathlineto{\pgfqpoint{3.970539in}{2.190648in}}%
\pgfpathlineto{\pgfqpoint{3.962608in}{2.180219in}}%
\pgfpathlineto{\pgfqpoint{3.954673in}{2.169766in}}%
\pgfpathlineto{\pgfqpoint{3.946732in}{2.159288in}}%
\pgfpathclose%
\pgfusepath{fill}%
\end{pgfscope}%
\begin{pgfscope}%
\pgfpathrectangle{\pgfqpoint{1.150000in}{0.150000in}}{\pgfqpoint{5.700000in}{5.700000in}}%
\pgfusepath{clip}%
\pgfsetbuttcap%
\pgfsetroundjoin%
\definecolor{currentfill}{rgb}{0.149039,0.508051,0.557250}%
\pgfsetfillcolor{currentfill}%
\pgfsetfillopacity{0.800000}%
\pgfsetlinewidth{0.000000pt}%
\definecolor{currentstroke}{rgb}{0.000000,0.000000,0.000000}%
\pgfsetstrokecolor{currentstroke}%
\pgfsetdash{}{0pt}%
\pgfpathmoveto{\pgfqpoint{5.391428in}{3.116073in}}%
\pgfpathlineto{\pgfqpoint{5.405717in}{3.124542in}}%
\pgfpathlineto{\pgfqpoint{5.420023in}{3.133187in}}%
\pgfpathlineto{\pgfqpoint{5.434346in}{3.142007in}}%
\pgfpathlineto{\pgfqpoint{5.448686in}{3.151003in}}%
\pgfpathlineto{\pgfqpoint{5.456021in}{3.155907in}}%
\pgfpathlineto{\pgfqpoint{5.463351in}{3.160881in}}%
\pgfpathlineto{\pgfqpoint{5.470677in}{3.165929in}}%
\pgfpathlineto{\pgfqpoint{5.477999in}{3.171060in}}%
\pgfpathlineto{\pgfqpoint{5.463684in}{3.162632in}}%
\pgfpathlineto{\pgfqpoint{5.449385in}{3.154379in}}%
\pgfpathlineto{\pgfqpoint{5.435104in}{3.146301in}}%
\pgfpathlineto{\pgfqpoint{5.420840in}{3.138398in}}%
\pgfpathlineto{\pgfqpoint{5.413493in}{3.132689in}}%
\pgfpathlineto{\pgfqpoint{5.406142in}{3.127069in}}%
\pgfpathlineto{\pgfqpoint{5.398787in}{3.121533in}}%
\pgfpathlineto{\pgfqpoint{5.391428in}{3.116073in}}%
\pgfpathclose%
\pgfusepath{fill}%
\end{pgfscope}%
\begin{pgfscope}%
\pgfpathrectangle{\pgfqpoint{1.150000in}{0.150000in}}{\pgfqpoint{5.700000in}{5.700000in}}%
\pgfusepath{clip}%
\pgfsetbuttcap%
\pgfsetroundjoin%
\definecolor{currentfill}{rgb}{0.282910,0.105393,0.426902}%
\pgfsetfillcolor{currentfill}%
\pgfsetfillopacity{0.800000}%
\pgfsetlinewidth{0.000000pt}%
\definecolor{currentstroke}{rgb}{0.000000,0.000000,0.000000}%
\pgfsetstrokecolor{currentstroke}%
\pgfsetdash{}{0pt}%
\pgfpathmoveto{\pgfqpoint{3.774002in}{2.072921in}}%
\pgfpathlineto{\pgfqpoint{3.787605in}{2.072428in}}%
\pgfpathlineto{\pgfqpoint{3.801214in}{2.072136in}}%
\pgfpathlineto{\pgfqpoint{3.814831in}{2.072044in}}%
\pgfpathlineto{\pgfqpoint{3.828454in}{2.072151in}}%
\pgfpathlineto{\pgfqpoint{3.836443in}{2.082728in}}%
\pgfpathlineto{\pgfqpoint{3.844427in}{2.093289in}}%
\pgfpathlineto{\pgfqpoint{3.852406in}{2.103834in}}%
\pgfpathlineto{\pgfqpoint{3.860379in}{2.114362in}}%
\pgfpathlineto{\pgfqpoint{3.846764in}{2.114168in}}%
\pgfpathlineto{\pgfqpoint{3.833156in}{2.114173in}}%
\pgfpathlineto{\pgfqpoint{3.819555in}{2.114378in}}%
\pgfpathlineto{\pgfqpoint{3.805961in}{2.114784in}}%
\pgfpathlineto{\pgfqpoint{3.797979in}{2.104331in}}%
\pgfpathlineto{\pgfqpoint{3.789992in}{2.093869in}}%
\pgfpathlineto{\pgfqpoint{3.782000in}{2.083399in}}%
\pgfpathlineto{\pgfqpoint{3.774002in}{2.072921in}}%
\pgfpathclose%
\pgfusepath{fill}%
\end{pgfscope}%
\begin{pgfscope}%
\pgfpathrectangle{\pgfqpoint{1.150000in}{0.150000in}}{\pgfqpoint{5.700000in}{5.700000in}}%
\pgfusepath{clip}%
\pgfsetbuttcap%
\pgfsetroundjoin%
\definecolor{currentfill}{rgb}{0.280255,0.165693,0.476498}%
\pgfsetfillcolor{currentfill}%
\pgfsetfillopacity{0.800000}%
\pgfsetlinewidth{0.000000pt}%
\definecolor{currentstroke}{rgb}{0.000000,0.000000,0.000000}%
\pgfsetstrokecolor{currentstroke}%
\pgfsetdash{}{0pt}%
\pgfpathmoveto{\pgfqpoint{4.033077in}{2.207272in}}%
\pgfpathlineto{\pgfqpoint{4.046753in}{2.209315in}}%
\pgfpathlineto{\pgfqpoint{4.060438in}{2.211551in}}%
\pgfpathlineto{\pgfqpoint{4.074132in}{2.213981in}}%
\pgfpathlineto{\pgfqpoint{4.087836in}{2.216603in}}%
\pgfpathlineto{\pgfqpoint{4.095742in}{2.226990in}}%
\pgfpathlineto{\pgfqpoint{4.103643in}{2.237337in}}%
\pgfpathlineto{\pgfqpoint{4.111539in}{2.247645in}}%
\pgfpathlineto{\pgfqpoint{4.119430in}{2.257914in}}%
\pgfpathlineto{\pgfqpoint{4.105733in}{2.255300in}}%
\pgfpathlineto{\pgfqpoint{4.092046in}{2.252879in}}%
\pgfpathlineto{\pgfqpoint{4.078368in}{2.250651in}}%
\pgfpathlineto{\pgfqpoint{4.064699in}{2.248617in}}%
\pgfpathlineto{\pgfqpoint{4.056802in}{2.238327in}}%
\pgfpathlineto{\pgfqpoint{4.048899in}{2.228007in}}%
\pgfpathlineto{\pgfqpoint{4.040990in}{2.217655in}}%
\pgfpathlineto{\pgfqpoint{4.033077in}{2.207272in}}%
\pgfpathclose%
\pgfusepath{fill}%
\end{pgfscope}%
\begin{pgfscope}%
\pgfpathrectangle{\pgfqpoint{1.150000in}{0.150000in}}{\pgfqpoint{5.700000in}{5.700000in}}%
\pgfusepath{clip}%
\pgfsetbuttcap%
\pgfsetroundjoin%
\definecolor{currentfill}{rgb}{0.277018,0.050344,0.375715}%
\pgfsetfillcolor{currentfill}%
\pgfsetfillopacity{0.800000}%
\pgfsetlinewidth{0.000000pt}%
\definecolor{currentstroke}{rgb}{0.000000,0.000000,0.000000}%
\pgfsetstrokecolor{currentstroke}%
\pgfsetdash{}{0pt}%
\pgfpathmoveto{\pgfqpoint{3.232410in}{1.987252in}}%
\pgfpathlineto{\pgfqpoint{3.245967in}{1.979882in}}%
\pgfpathlineto{\pgfqpoint{3.259526in}{1.972739in}}%
\pgfpathlineto{\pgfqpoint{3.273086in}{1.965822in}}%
\pgfpathlineto{\pgfqpoint{3.286647in}{1.959129in}}%
\pgfpathlineto{\pgfqpoint{3.294835in}{1.967994in}}%
\pgfpathlineto{\pgfqpoint{3.303016in}{1.976925in}}%
\pgfpathlineto{\pgfqpoint{3.311190in}{1.985921in}}%
\pgfpathlineto{\pgfqpoint{3.319357in}{1.994978in}}%
\pgfpathlineto{\pgfqpoint{3.305813in}{2.001425in}}%
\pgfpathlineto{\pgfqpoint{3.292271in}{2.008096in}}%
\pgfpathlineto{\pgfqpoint{3.278729in}{2.014993in}}%
\pgfpathlineto{\pgfqpoint{3.265189in}{2.022117in}}%
\pgfpathlineto{\pgfqpoint{3.257005in}{2.013293in}}%
\pgfpathlineto{\pgfqpoint{3.248813in}{2.004540in}}%
\pgfpathlineto{\pgfqpoint{3.240615in}{1.995859in}}%
\pgfpathlineto{\pgfqpoint{3.232410in}{1.987252in}}%
\pgfpathclose%
\pgfusepath{fill}%
\end{pgfscope}%
\begin{pgfscope}%
\pgfpathrectangle{\pgfqpoint{1.150000in}{0.150000in}}{\pgfqpoint{5.700000in}{5.700000in}}%
\pgfusepath{clip}%
\pgfsetbuttcap%
\pgfsetroundjoin%
\definecolor{currentfill}{rgb}{0.276194,0.190074,0.493001}%
\pgfsetfillcolor{currentfill}%
\pgfsetfillopacity{0.800000}%
\pgfsetlinewidth{0.000000pt}%
\definecolor{currentstroke}{rgb}{0.000000,0.000000,0.000000}%
\pgfsetstrokecolor{currentstroke}%
\pgfsetdash{}{0pt}%
\pgfpathmoveto{\pgfqpoint{4.119430in}{2.257914in}}%
\pgfpathlineto{\pgfqpoint{4.133137in}{2.260721in}}%
\pgfpathlineto{\pgfqpoint{4.146854in}{2.263720in}}%
\pgfpathlineto{\pgfqpoint{4.160581in}{2.266910in}}%
\pgfpathlineto{\pgfqpoint{4.174318in}{2.270291in}}%
\pgfpathlineto{\pgfqpoint{4.182198in}{2.280494in}}%
\pgfpathlineto{\pgfqpoint{4.190071in}{2.290652in}}%
\pgfpathlineto{\pgfqpoint{4.197940in}{2.300765in}}%
\pgfpathlineto{\pgfqpoint{4.205803in}{2.310836in}}%
\pgfpathlineto{\pgfqpoint{4.192073in}{2.307495in}}%
\pgfpathlineto{\pgfqpoint{4.178353in}{2.304345in}}%
\pgfpathlineto{\pgfqpoint{4.164643in}{2.301387in}}%
\pgfpathlineto{\pgfqpoint{4.150943in}{2.298621in}}%
\pgfpathlineto{\pgfqpoint{4.143072in}{2.288498in}}%
\pgfpathlineto{\pgfqpoint{4.135197in}{2.278340in}}%
\pgfpathlineto{\pgfqpoint{4.127316in}{2.268146in}}%
\pgfpathlineto{\pgfqpoint{4.119430in}{2.257914in}}%
\pgfpathclose%
\pgfusepath{fill}%
\end{pgfscope}%
\begin{pgfscope}%
\pgfpathrectangle{\pgfqpoint{1.150000in}{0.150000in}}{\pgfqpoint{5.700000in}{5.700000in}}%
\pgfusepath{clip}%
\pgfsetbuttcap%
\pgfsetroundjoin%
\definecolor{currentfill}{rgb}{0.281446,0.084320,0.407414}%
\pgfsetfillcolor{currentfill}%
\pgfsetfillopacity{0.800000}%
\pgfsetlinewidth{0.000000pt}%
\definecolor{currentstroke}{rgb}{0.000000,0.000000,0.000000}%
\pgfsetstrokecolor{currentstroke}%
\pgfsetdash{}{0pt}%
\pgfpathmoveto{\pgfqpoint{3.687580in}{2.035414in}}%
\pgfpathlineto{\pgfqpoint{3.701166in}{2.033992in}}%
\pgfpathlineto{\pgfqpoint{3.714758in}{2.032774in}}%
\pgfpathlineto{\pgfqpoint{3.728357in}{2.031758in}}%
\pgfpathlineto{\pgfqpoint{3.741961in}{2.030945in}}%
\pgfpathlineto{\pgfqpoint{3.749979in}{2.041447in}}%
\pgfpathlineto{\pgfqpoint{3.757992in}{2.051945in}}%
\pgfpathlineto{\pgfqpoint{3.766000in}{2.062436in}}%
\pgfpathlineto{\pgfqpoint{3.774002in}{2.072921in}}%
\pgfpathlineto{\pgfqpoint{3.760407in}{2.073616in}}%
\pgfpathlineto{\pgfqpoint{3.746818in}{2.074513in}}%
\pgfpathlineto{\pgfqpoint{3.733235in}{2.075613in}}%
\pgfpathlineto{\pgfqpoint{3.719659in}{2.076916in}}%
\pgfpathlineto{\pgfqpoint{3.711647in}{2.066538in}}%
\pgfpathlineto{\pgfqpoint{3.703630in}{2.056161in}}%
\pgfpathlineto{\pgfqpoint{3.695608in}{2.045786in}}%
\pgfpathlineto{\pgfqpoint{3.687580in}{2.035414in}}%
\pgfpathclose%
\pgfusepath{fill}%
\end{pgfscope}%
\begin{pgfscope}%
\pgfpathrectangle{\pgfqpoint{1.150000in}{0.150000in}}{\pgfqpoint{5.700000in}{5.700000in}}%
\pgfusepath{clip}%
\pgfsetbuttcap%
\pgfsetroundjoin%
\definecolor{currentfill}{rgb}{0.141935,0.526453,0.555991}%
\pgfsetfillcolor{currentfill}%
\pgfsetfillopacity{0.800000}%
\pgfsetlinewidth{0.000000pt}%
\definecolor{currentstroke}{rgb}{0.000000,0.000000,0.000000}%
\pgfsetstrokecolor{currentstroke}%
\pgfsetdash{}{0pt}%
\pgfpathmoveto{\pgfqpoint{5.477999in}{3.171060in}}%
\pgfpathlineto{\pgfqpoint{5.492333in}{3.179663in}}%
\pgfpathlineto{\pgfqpoint{5.506683in}{3.188440in}}%
\pgfpathlineto{\pgfqpoint{5.521051in}{3.197393in}}%
\pgfpathlineto{\pgfqpoint{5.535438in}{3.206521in}}%
\pgfpathlineto{\pgfqpoint{5.542730in}{3.211150in}}%
\pgfpathlineto{\pgfqpoint{5.550019in}{3.215867in}}%
\pgfpathlineto{\pgfqpoint{5.557304in}{3.220678in}}%
\pgfpathlineto{\pgfqpoint{5.564586in}{3.225590in}}%
\pgfpathlineto{\pgfqpoint{5.550227in}{3.217064in}}%
\pgfpathlineto{\pgfqpoint{5.535885in}{3.208712in}}%
\pgfpathlineto{\pgfqpoint{5.521561in}{3.200534in}}%
\pgfpathlineto{\pgfqpoint{5.507254in}{3.192531in}}%
\pgfpathlineto{\pgfqpoint{5.499945in}{3.187008in}}%
\pgfpathlineto{\pgfqpoint{5.492633in}{3.181593in}}%
\pgfpathlineto{\pgfqpoint{5.485318in}{3.176279in}}%
\pgfpathlineto{\pgfqpoint{5.477999in}{3.171060in}}%
\pgfpathclose%
\pgfusepath{fill}%
\end{pgfscope}%
\begin{pgfscope}%
\pgfpathrectangle{\pgfqpoint{1.150000in}{0.150000in}}{\pgfqpoint{5.700000in}{5.700000in}}%
\pgfusepath{clip}%
\pgfsetbuttcap%
\pgfsetroundjoin%
\definecolor{currentfill}{rgb}{0.210503,0.363727,0.552206}%
\pgfsetfillcolor{currentfill}%
\pgfsetfillopacity{0.800000}%
\pgfsetlinewidth{0.000000pt}%
\definecolor{currentstroke}{rgb}{0.000000,0.000000,0.000000}%
\pgfsetstrokecolor{currentstroke}%
\pgfsetdash{}{0pt}%
\pgfpathmoveto{\pgfqpoint{4.755515in}{2.689677in}}%
\pgfpathlineto{\pgfqpoint{4.769497in}{2.696482in}}%
\pgfpathlineto{\pgfqpoint{4.783494in}{2.703469in}}%
\pgfpathlineto{\pgfqpoint{4.797504in}{2.710638in}}%
\pgfpathlineto{\pgfqpoint{4.811530in}{2.717989in}}%
\pgfpathlineto{\pgfqpoint{4.819168in}{2.725581in}}%
\pgfpathlineto{\pgfqpoint{4.826800in}{2.733141in}}%
\pgfpathlineto{\pgfqpoint{4.834426in}{2.740672in}}%
\pgfpathlineto{\pgfqpoint{4.842046in}{2.748179in}}%
\pgfpathlineto{\pgfqpoint{4.828032in}{2.741130in}}%
\pgfpathlineto{\pgfqpoint{4.814033in}{2.734263in}}%
\pgfpathlineto{\pgfqpoint{4.800049in}{2.727577in}}%
\pgfpathlineto{\pgfqpoint{4.786078in}{2.721074in}}%
\pgfpathlineto{\pgfqpoint{4.778446in}{2.713254in}}%
\pgfpathlineto{\pgfqpoint{4.770809in}{2.705416in}}%
\pgfpathlineto{\pgfqpoint{4.763165in}{2.697559in}}%
\pgfpathlineto{\pgfqpoint{4.755515in}{2.689677in}}%
\pgfpathclose%
\pgfusepath{fill}%
\end{pgfscope}%
\begin{pgfscope}%
\pgfpathrectangle{\pgfqpoint{1.150000in}{0.150000in}}{\pgfqpoint{5.700000in}{5.700000in}}%
\pgfusepath{clip}%
\pgfsetbuttcap%
\pgfsetroundjoin%
\definecolor{currentfill}{rgb}{0.277018,0.050344,0.375715}%
\pgfsetfillcolor{currentfill}%
\pgfsetfillopacity{0.800000}%
\pgfsetlinewidth{0.000000pt}%
\definecolor{currentstroke}{rgb}{0.000000,0.000000,0.000000}%
\pgfsetstrokecolor{currentstroke}%
\pgfsetdash{}{0pt}%
\pgfpathmoveto{\pgfqpoint{3.373552in}{1.971417in}}%
\pgfpathlineto{\pgfqpoint{3.387106in}{1.966076in}}%
\pgfpathlineto{\pgfqpoint{3.400663in}{1.960954in}}%
\pgfpathlineto{\pgfqpoint{3.414222in}{1.956049in}}%
\pgfpathlineto{\pgfqpoint{3.427785in}{1.951359in}}%
\pgfpathlineto{\pgfqpoint{3.435915in}{1.960928in}}%
\pgfpathlineto{\pgfqpoint{3.444040in}{1.970541in}}%
\pgfpathlineto{\pgfqpoint{3.452158in}{1.980193in}}%
\pgfpathlineto{\pgfqpoint{3.460271in}{1.989885in}}%
\pgfpathlineto{\pgfqpoint{3.446722in}{1.994361in}}%
\pgfpathlineto{\pgfqpoint{3.433177in}{1.999053in}}%
\pgfpathlineto{\pgfqpoint{3.419635in}{2.003961in}}%
\pgfpathlineto{\pgfqpoint{3.406095in}{2.009088in}}%
\pgfpathlineto{\pgfqpoint{3.397969in}{1.999598in}}%
\pgfpathlineto{\pgfqpoint{3.389836in}{1.990155in}}%
\pgfpathlineto{\pgfqpoint{3.381697in}{1.980760in}}%
\pgfpathlineto{\pgfqpoint{3.373552in}{1.971417in}}%
\pgfpathclose%
\pgfusepath{fill}%
\end{pgfscope}%
\begin{pgfscope}%
\pgfpathrectangle{\pgfqpoint{1.150000in}{0.150000in}}{\pgfqpoint{5.700000in}{5.700000in}}%
\pgfusepath{clip}%
\pgfsetbuttcap%
\pgfsetroundjoin%
\definecolor{currentfill}{rgb}{0.276194,0.190074,0.493001}%
\pgfsetfillcolor{currentfill}%
\pgfsetfillopacity{0.800000}%
\pgfsetlinewidth{0.000000pt}%
\definecolor{currentstroke}{rgb}{0.000000,0.000000,0.000000}%
\pgfsetstrokecolor{currentstroke}%
\pgfsetdash{}{0pt}%
\pgfpathmoveto{\pgfqpoint{2.730227in}{2.305155in}}%
\pgfpathlineto{\pgfqpoint{2.743933in}{2.289081in}}%
\pgfpathlineto{\pgfqpoint{2.757633in}{2.273290in}}%
\pgfpathlineto{\pgfqpoint{2.771325in}{2.257780in}}%
\pgfpathlineto{\pgfqpoint{2.785011in}{2.242550in}}%
\pgfpathlineto{\pgfqpoint{2.793452in}{2.248192in}}%
\pgfpathlineto{\pgfqpoint{2.801883in}{2.253986in}}%
\pgfpathlineto{\pgfqpoint{2.810303in}{2.259927in}}%
\pgfpathlineto{\pgfqpoint{2.818713in}{2.266015in}}%
\pgfpathlineto{\pgfqpoint{2.805056in}{2.280926in}}%
\pgfpathlineto{\pgfqpoint{2.791393in}{2.296115in}}%
\pgfpathlineto{\pgfqpoint{2.777724in}{2.311585in}}%
\pgfpathlineto{\pgfqpoint{2.764047in}{2.327339in}}%
\pgfpathlineto{\pgfqpoint{2.755609in}{2.321559in}}%
\pgfpathlineto{\pgfqpoint{2.747159in}{2.315934in}}%
\pgfpathlineto{\pgfqpoint{2.738698in}{2.310465in}}%
\pgfpathlineto{\pgfqpoint{2.730227in}{2.305155in}}%
\pgfpathclose%
\pgfusepath{fill}%
\end{pgfscope}%
\begin{pgfscope}%
\pgfpathrectangle{\pgfqpoint{1.150000in}{0.150000in}}{\pgfqpoint{5.700000in}{5.700000in}}%
\pgfusepath{clip}%
\pgfsetbuttcap%
\pgfsetroundjoin%
\definecolor{currentfill}{rgb}{0.269308,0.218818,0.509577}%
\pgfsetfillcolor{currentfill}%
\pgfsetfillopacity{0.800000}%
\pgfsetlinewidth{0.000000pt}%
\definecolor{currentstroke}{rgb}{0.000000,0.000000,0.000000}%
\pgfsetstrokecolor{currentstroke}%
\pgfsetdash{}{0pt}%
\pgfpathmoveto{\pgfqpoint{2.675322in}{2.372334in}}%
\pgfpathlineto{\pgfqpoint{2.689060in}{2.355102in}}%
\pgfpathlineto{\pgfqpoint{2.702790in}{2.338163in}}%
\pgfpathlineto{\pgfqpoint{2.716512in}{2.321515in}}%
\pgfpathlineto{\pgfqpoint{2.730227in}{2.305155in}}%
\pgfpathlineto{\pgfqpoint{2.738698in}{2.310465in}}%
\pgfpathlineto{\pgfqpoint{2.747159in}{2.315934in}}%
\pgfpathlineto{\pgfqpoint{2.755609in}{2.321559in}}%
\pgfpathlineto{\pgfqpoint{2.764047in}{2.327339in}}%
\pgfpathlineto{\pgfqpoint{2.750364in}{2.343377in}}%
\pgfpathlineto{\pgfqpoint{2.736672in}{2.359703in}}%
\pgfpathlineto{\pgfqpoint{2.722974in}{2.376319in}}%
\pgfpathlineto{\pgfqpoint{2.709267in}{2.393228in}}%
\pgfpathlineto{\pgfqpoint{2.700798in}{2.387759in}}%
\pgfpathlineto{\pgfqpoint{2.692318in}{2.382451in}}%
\pgfpathlineto{\pgfqpoint{2.683826in}{2.377309in}}%
\pgfpathlineto{\pgfqpoint{2.675322in}{2.372334in}}%
\pgfpathclose%
\pgfusepath{fill}%
\end{pgfscope}%
\begin{pgfscope}%
\pgfpathrectangle{\pgfqpoint{1.150000in}{0.150000in}}{\pgfqpoint{5.700000in}{5.700000in}}%
\pgfusepath{clip}%
\pgfsetbuttcap%
\pgfsetroundjoin%
\definecolor{currentfill}{rgb}{0.280255,0.165693,0.476498}%
\pgfsetfillcolor{currentfill}%
\pgfsetfillopacity{0.800000}%
\pgfsetlinewidth{0.000000pt}%
\definecolor{currentstroke}{rgb}{0.000000,0.000000,0.000000}%
\pgfsetstrokecolor{currentstroke}%
\pgfsetdash{}{0pt}%
\pgfpathmoveto{\pgfqpoint{2.785011in}{2.242550in}}%
\pgfpathlineto{\pgfqpoint{2.798690in}{2.227596in}}%
\pgfpathlineto{\pgfqpoint{2.812363in}{2.212917in}}%
\pgfpathlineto{\pgfqpoint{2.826030in}{2.198510in}}%
\pgfpathlineto{\pgfqpoint{2.839692in}{2.184373in}}%
\pgfpathlineto{\pgfqpoint{2.848104in}{2.190346in}}%
\pgfpathlineto{\pgfqpoint{2.856507in}{2.196462in}}%
\pgfpathlineto{\pgfqpoint{2.864899in}{2.202718in}}%
\pgfpathlineto{\pgfqpoint{2.873281in}{2.209111in}}%
\pgfpathlineto{\pgfqpoint{2.859647in}{2.222930in}}%
\pgfpathlineto{\pgfqpoint{2.846008in}{2.237019in}}%
\pgfpathlineto{\pgfqpoint{2.832363in}{2.251380in}}%
\pgfpathlineto{\pgfqpoint{2.818713in}{2.266015in}}%
\pgfpathlineto{\pgfqpoint{2.810303in}{2.259927in}}%
\pgfpathlineto{\pgfqpoint{2.801883in}{2.253986in}}%
\pgfpathlineto{\pgfqpoint{2.793452in}{2.248192in}}%
\pgfpathlineto{\pgfqpoint{2.785011in}{2.242550in}}%
\pgfpathclose%
\pgfusepath{fill}%
\end{pgfscope}%
\begin{pgfscope}%
\pgfpathrectangle{\pgfqpoint{1.150000in}{0.150000in}}{\pgfqpoint{5.700000in}{5.700000in}}%
\pgfusepath{clip}%
\pgfsetbuttcap%
\pgfsetroundjoin%
\definecolor{currentfill}{rgb}{0.270595,0.214069,0.507052}%
\pgfsetfillcolor{currentfill}%
\pgfsetfillopacity{0.800000}%
\pgfsetlinewidth{0.000000pt}%
\definecolor{currentstroke}{rgb}{0.000000,0.000000,0.000000}%
\pgfsetstrokecolor{currentstroke}%
\pgfsetdash{}{0pt}%
\pgfpathmoveto{\pgfqpoint{4.205803in}{2.310836in}}%
\pgfpathlineto{\pgfqpoint{4.219545in}{2.314367in}}%
\pgfpathlineto{\pgfqpoint{4.233296in}{2.318089in}}%
\pgfpathlineto{\pgfqpoint{4.247059in}{2.322000in}}%
\pgfpathlineto{\pgfqpoint{4.260833in}{2.326101in}}%
\pgfpathlineto{\pgfqpoint{4.268684in}{2.336069in}}%
\pgfpathlineto{\pgfqpoint{4.276530in}{2.345987in}}%
\pgfpathlineto{\pgfqpoint{4.284371in}{2.355857in}}%
\pgfpathlineto{\pgfqpoint{4.292206in}{2.365680in}}%
\pgfpathlineto{\pgfqpoint{4.278439in}{2.361652in}}%
\pgfpathlineto{\pgfqpoint{4.264684in}{2.357813in}}%
\pgfpathlineto{\pgfqpoint{4.250939in}{2.354164in}}%
\pgfpathlineto{\pgfqpoint{4.237204in}{2.350705in}}%
\pgfpathlineto{\pgfqpoint{4.229362in}{2.340797in}}%
\pgfpathlineto{\pgfqpoint{4.221514in}{2.330851in}}%
\pgfpathlineto{\pgfqpoint{4.213662in}{2.320864in}}%
\pgfpathlineto{\pgfqpoint{4.205803in}{2.310836in}}%
\pgfpathclose%
\pgfusepath{fill}%
\end{pgfscope}%
\begin{pgfscope}%
\pgfpathrectangle{\pgfqpoint{1.150000in}{0.150000in}}{\pgfqpoint{5.700000in}{5.700000in}}%
\pgfusepath{clip}%
\pgfsetbuttcap%
\pgfsetroundjoin%
\definecolor{currentfill}{rgb}{0.135066,0.544853,0.554029}%
\pgfsetfillcolor{currentfill}%
\pgfsetfillopacity{0.800000}%
\pgfsetlinewidth{0.000000pt}%
\definecolor{currentstroke}{rgb}{0.000000,0.000000,0.000000}%
\pgfsetstrokecolor{currentstroke}%
\pgfsetdash{}{0pt}%
\pgfpathmoveto{\pgfqpoint{5.564586in}{3.225590in}}%
\pgfpathlineto{\pgfqpoint{5.578962in}{3.234290in}}%
\pgfpathlineto{\pgfqpoint{5.593357in}{3.243164in}}%
\pgfpathlineto{\pgfqpoint{5.607770in}{3.252212in}}%
\pgfpathlineto{\pgfqpoint{5.622201in}{3.261435in}}%
\pgfpathlineto{\pgfqpoint{5.629451in}{3.265831in}}%
\pgfpathlineto{\pgfqpoint{5.636699in}{3.270334in}}%
\pgfpathlineto{\pgfqpoint{5.643943in}{3.274951in}}%
\pgfpathlineto{\pgfqpoint{5.651185in}{3.279688in}}%
\pgfpathlineto{\pgfqpoint{5.636784in}{3.271100in}}%
\pgfpathlineto{\pgfqpoint{5.622400in}{3.262686in}}%
\pgfpathlineto{\pgfqpoint{5.608034in}{3.254445in}}%
\pgfpathlineto{\pgfqpoint{5.593686in}{3.246377in}}%
\pgfpathlineto{\pgfqpoint{5.586414in}{3.240995in}}%
\pgfpathlineto{\pgfqpoint{5.579141in}{3.235741in}}%
\pgfpathlineto{\pgfqpoint{5.571865in}{3.230609in}}%
\pgfpathlineto{\pgfqpoint{5.564586in}{3.225590in}}%
\pgfpathclose%
\pgfusepath{fill}%
\end{pgfscope}%
\begin{pgfscope}%
\pgfpathrectangle{\pgfqpoint{1.150000in}{0.150000in}}{\pgfqpoint{5.700000in}{5.700000in}}%
\pgfusepath{clip}%
\pgfsetbuttcap%
\pgfsetroundjoin%
\definecolor{currentfill}{rgb}{0.279566,0.067836,0.391917}%
\pgfsetfillcolor{currentfill}%
\pgfsetfillopacity{0.800000}%
\pgfsetlinewidth{0.000000pt}%
\definecolor{currentstroke}{rgb}{0.000000,0.000000,0.000000}%
\pgfsetstrokecolor{currentstroke}%
\pgfsetdash{}{0pt}%
\pgfpathmoveto{\pgfqpoint{3.090895in}{2.023147in}}%
\pgfpathlineto{\pgfqpoint{3.104475in}{2.013632in}}%
\pgfpathlineto{\pgfqpoint{3.118054in}{2.004354in}}%
\pgfpathlineto{\pgfqpoint{3.131632in}{1.995314in}}%
\pgfpathlineto{\pgfqpoint{3.145209in}{1.986509in}}%
\pgfpathlineto{\pgfqpoint{3.153464in}{1.994504in}}%
\pgfpathlineto{\pgfqpoint{3.161711in}{2.002592in}}%
\pgfpathlineto{\pgfqpoint{3.169950in}{2.010768in}}%
\pgfpathlineto{\pgfqpoint{3.178182in}{2.019032in}}%
\pgfpathlineto{\pgfqpoint{3.164625in}{2.027558in}}%
\pgfpathlineto{\pgfqpoint{3.151067in}{2.036319in}}%
\pgfpathlineto{\pgfqpoint{3.137509in}{2.045317in}}%
\pgfpathlineto{\pgfqpoint{3.123950in}{2.054553in}}%
\pgfpathlineto{\pgfqpoint{3.115699in}{2.046557in}}%
\pgfpathlineto{\pgfqpoint{3.107439in}{2.038656in}}%
\pgfpathlineto{\pgfqpoint{3.099171in}{2.030852in}}%
\pgfpathlineto{\pgfqpoint{3.090895in}{2.023147in}}%
\pgfpathclose%
\pgfusepath{fill}%
\end{pgfscope}%
\begin{pgfscope}%
\pgfpathrectangle{\pgfqpoint{1.150000in}{0.150000in}}{\pgfqpoint{5.700000in}{5.700000in}}%
\pgfusepath{clip}%
\pgfsetbuttcap%
\pgfsetroundjoin%
\definecolor{currentfill}{rgb}{0.260571,0.246922,0.522828}%
\pgfsetfillcolor{currentfill}%
\pgfsetfillopacity{0.800000}%
\pgfsetlinewidth{0.000000pt}%
\definecolor{currentstroke}{rgb}{0.000000,0.000000,0.000000}%
\pgfsetstrokecolor{currentstroke}%
\pgfsetdash{}{0pt}%
\pgfpathmoveto{\pgfqpoint{2.620279in}{2.444244in}}%
\pgfpathlineto{\pgfqpoint{2.634054in}{2.425814in}}%
\pgfpathlineto{\pgfqpoint{2.647819in}{2.407687in}}%
\pgfpathlineto{\pgfqpoint{2.661575in}{2.389861in}}%
\pgfpathlineto{\pgfqpoint{2.675322in}{2.372334in}}%
\pgfpathlineto{\pgfqpoint{2.683826in}{2.377309in}}%
\pgfpathlineto{\pgfqpoint{2.692318in}{2.382451in}}%
\pgfpathlineto{\pgfqpoint{2.700798in}{2.387759in}}%
\pgfpathlineto{\pgfqpoint{2.709267in}{2.393228in}}%
\pgfpathlineto{\pgfqpoint{2.695552in}{2.410432in}}%
\pgfpathlineto{\pgfqpoint{2.681828in}{2.427933in}}%
\pgfpathlineto{\pgfqpoint{2.668095in}{2.445735in}}%
\pgfpathlineto{\pgfqpoint{2.654354in}{2.463840in}}%
\pgfpathlineto{\pgfqpoint{2.645853in}{2.458682in}}%
\pgfpathlineto{\pgfqpoint{2.637341in}{2.453696in}}%
\pgfpathlineto{\pgfqpoint{2.628816in}{2.448882in}}%
\pgfpathlineto{\pgfqpoint{2.620279in}{2.444244in}}%
\pgfpathclose%
\pgfusepath{fill}%
\end{pgfscope}%
\begin{pgfscope}%
\pgfpathrectangle{\pgfqpoint{1.150000in}{0.150000in}}{\pgfqpoint{5.700000in}{5.700000in}}%
\pgfusepath{clip}%
\pgfsetbuttcap%
\pgfsetroundjoin%
\definecolor{currentfill}{rgb}{0.279566,0.067836,0.391917}%
\pgfsetfillcolor{currentfill}%
\pgfsetfillopacity{0.800000}%
\pgfsetlinewidth{0.000000pt}%
\definecolor{currentstroke}{rgb}{0.000000,0.000000,0.000000}%
\pgfsetstrokecolor{currentstroke}%
\pgfsetdash{}{0pt}%
\pgfpathmoveto{\pgfqpoint{3.601089in}{2.002314in}}%
\pgfpathlineto{\pgfqpoint{3.614663in}{1.999919in}}%
\pgfpathlineto{\pgfqpoint{3.628242in}{1.997732in}}%
\pgfpathlineto{\pgfqpoint{3.641827in}{1.995750in}}%
\pgfpathlineto{\pgfqpoint{3.655417in}{1.993973in}}%
\pgfpathlineto{\pgfqpoint{3.663466in}{2.004324in}}%
\pgfpathlineto{\pgfqpoint{3.671509in}{2.014682in}}%
\pgfpathlineto{\pgfqpoint{3.679547in}{2.025046in}}%
\pgfpathlineto{\pgfqpoint{3.687580in}{2.035414in}}%
\pgfpathlineto{\pgfqpoint{3.674000in}{2.037040in}}%
\pgfpathlineto{\pgfqpoint{3.660426in}{2.038872in}}%
\pgfpathlineto{\pgfqpoint{3.646857in}{2.040909in}}%
\pgfpathlineto{\pgfqpoint{3.633293in}{2.043153in}}%
\pgfpathlineto{\pgfqpoint{3.625250in}{2.032923in}}%
\pgfpathlineto{\pgfqpoint{3.617202in}{2.022706in}}%
\pgfpathlineto{\pgfqpoint{3.609148in}{2.012503in}}%
\pgfpathlineto{\pgfqpoint{3.601089in}{2.002314in}}%
\pgfpathclose%
\pgfusepath{fill}%
\end{pgfscope}%
\begin{pgfscope}%
\pgfpathrectangle{\pgfqpoint{1.150000in}{0.150000in}}{\pgfqpoint{5.700000in}{5.700000in}}%
\pgfusepath{clip}%
\pgfsetbuttcap%
\pgfsetroundjoin%
\definecolor{currentfill}{rgb}{0.199430,0.387607,0.554642}%
\pgfsetfillcolor{currentfill}%
\pgfsetfillopacity{0.800000}%
\pgfsetlinewidth{0.000000pt}%
\definecolor{currentstroke}{rgb}{0.000000,0.000000,0.000000}%
\pgfsetstrokecolor{currentstroke}%
\pgfsetdash{}{0pt}%
\pgfpathmoveto{\pgfqpoint{4.842046in}{2.748179in}}%
\pgfpathlineto{\pgfqpoint{4.856074in}{2.755409in}}%
\pgfpathlineto{\pgfqpoint{4.870117in}{2.762821in}}%
\pgfpathlineto{\pgfqpoint{4.884175in}{2.770413in}}%
\pgfpathlineto{\pgfqpoint{4.898247in}{2.778187in}}%
\pgfpathlineto{\pgfqpoint{4.905849in}{2.785349in}}%
\pgfpathlineto{\pgfqpoint{4.913444in}{2.792486in}}%
\pgfpathlineto{\pgfqpoint{4.921033in}{2.799602in}}%
\pgfpathlineto{\pgfqpoint{4.928616in}{2.806700in}}%
\pgfpathlineto{\pgfqpoint{4.914556in}{2.799262in}}%
\pgfpathlineto{\pgfqpoint{4.900512in}{2.792005in}}%
\pgfpathlineto{\pgfqpoint{4.886482in}{2.784929in}}%
\pgfpathlineto{\pgfqpoint{4.872467in}{2.778033in}}%
\pgfpathlineto{\pgfqpoint{4.864870in}{2.770588in}}%
\pgfpathlineto{\pgfqpoint{4.857268in}{2.763133in}}%
\pgfpathlineto{\pgfqpoint{4.849660in}{2.755664in}}%
\pgfpathlineto{\pgfqpoint{4.842046in}{2.748179in}}%
\pgfpathclose%
\pgfusepath{fill}%
\end{pgfscope}%
\begin{pgfscope}%
\pgfpathrectangle{\pgfqpoint{1.150000in}{0.150000in}}{\pgfqpoint{5.700000in}{5.700000in}}%
\pgfusepath{clip}%
\pgfsetbuttcap%
\pgfsetroundjoin%
\definecolor{currentfill}{rgb}{0.282623,0.140926,0.457517}%
\pgfsetfillcolor{currentfill}%
\pgfsetfillopacity{0.800000}%
\pgfsetlinewidth{0.000000pt}%
\definecolor{currentstroke}{rgb}{0.000000,0.000000,0.000000}%
\pgfsetstrokecolor{currentstroke}%
\pgfsetdash{}{0pt}%
\pgfpathmoveto{\pgfqpoint{2.839692in}{2.184373in}}%
\pgfpathlineto{\pgfqpoint{2.853348in}{2.170505in}}%
\pgfpathlineto{\pgfqpoint{2.866999in}{2.156903in}}%
\pgfpathlineto{\pgfqpoint{2.880645in}{2.143565in}}%
\pgfpathlineto{\pgfqpoint{2.894286in}{2.130489in}}%
\pgfpathlineto{\pgfqpoint{2.902671in}{2.136791in}}%
\pgfpathlineto{\pgfqpoint{2.911046in}{2.143227in}}%
\pgfpathlineto{\pgfqpoint{2.919412in}{2.149796in}}%
\pgfpathlineto{\pgfqpoint{2.927767in}{2.156493in}}%
\pgfpathlineto{\pgfqpoint{2.914152in}{2.169253in}}%
\pgfpathlineto{\pgfqpoint{2.900533in}{2.182274in}}%
\pgfpathlineto{\pgfqpoint{2.886909in}{2.195560in}}%
\pgfpathlineto{\pgfqpoint{2.873281in}{2.209111in}}%
\pgfpathlineto{\pgfqpoint{2.864899in}{2.202718in}}%
\pgfpathlineto{\pgfqpoint{2.856507in}{2.196462in}}%
\pgfpathlineto{\pgfqpoint{2.848104in}{2.190346in}}%
\pgfpathlineto{\pgfqpoint{2.839692in}{2.184373in}}%
\pgfpathclose%
\pgfusepath{fill}%
\end{pgfscope}%
\begin{pgfscope}%
\pgfpathrectangle{\pgfqpoint{1.150000in}{0.150000in}}{\pgfqpoint{5.700000in}{5.700000in}}%
\pgfusepath{clip}%
\pgfsetbuttcap%
\pgfsetroundjoin%
\definecolor{currentfill}{rgb}{0.128729,0.563265,0.551229}%
\pgfsetfillcolor{currentfill}%
\pgfsetfillopacity{0.800000}%
\pgfsetlinewidth{0.000000pt}%
\definecolor{currentstroke}{rgb}{0.000000,0.000000,0.000000}%
\pgfsetstrokecolor{currentstroke}%
\pgfsetdash{}{0pt}%
\pgfpathmoveto{\pgfqpoint{5.651185in}{3.279688in}}%
\pgfpathlineto{\pgfqpoint{5.665605in}{3.288449in}}%
\pgfpathlineto{\pgfqpoint{5.680043in}{3.297383in}}%
\pgfpathlineto{\pgfqpoint{5.694499in}{3.306491in}}%
\pgfpathlineto{\pgfqpoint{5.708973in}{3.315773in}}%
\pgfpathlineto{\pgfqpoint{5.716183in}{3.319982in}}%
\pgfpathlineto{\pgfqpoint{5.723390in}{3.324318in}}%
\pgfpathlineto{\pgfqpoint{5.730595in}{3.328790in}}%
\pgfpathlineto{\pgfqpoint{5.737798in}{3.333403in}}%
\pgfpathlineto{\pgfqpoint{5.723355in}{3.324790in}}%
\pgfpathlineto{\pgfqpoint{5.708930in}{3.316349in}}%
\pgfpathlineto{\pgfqpoint{5.694523in}{3.308081in}}%
\pgfpathlineto{\pgfqpoint{5.680134in}{3.299985in}}%
\pgfpathlineto{\pgfqpoint{5.672899in}{3.294694in}}%
\pgfpathlineto{\pgfqpoint{5.665663in}{3.289552in}}%
\pgfpathlineto{\pgfqpoint{5.658425in}{3.284552in}}%
\pgfpathlineto{\pgfqpoint{5.651185in}{3.279688in}}%
\pgfpathclose%
\pgfusepath{fill}%
\end{pgfscope}%
\begin{pgfscope}%
\pgfpathrectangle{\pgfqpoint{1.150000in}{0.150000in}}{\pgfqpoint{5.700000in}{5.700000in}}%
\pgfusepath{clip}%
\pgfsetbuttcap%
\pgfsetroundjoin%
\definecolor{currentfill}{rgb}{0.263663,0.237631,0.518762}%
\pgfsetfillcolor{currentfill}%
\pgfsetfillopacity{0.800000}%
\pgfsetlinewidth{0.000000pt}%
\definecolor{currentstroke}{rgb}{0.000000,0.000000,0.000000}%
\pgfsetstrokecolor{currentstroke}%
\pgfsetdash{}{0pt}%
\pgfpathmoveto{\pgfqpoint{4.292206in}{2.365680in}}%
\pgfpathlineto{\pgfqpoint{4.305984in}{2.369897in}}%
\pgfpathlineto{\pgfqpoint{4.319774in}{2.374303in}}%
\pgfpathlineto{\pgfqpoint{4.333574in}{2.378897in}}%
\pgfpathlineto{\pgfqpoint{4.347387in}{2.383680in}}%
\pgfpathlineto{\pgfqpoint{4.355210in}{2.393365in}}%
\pgfpathlineto{\pgfqpoint{4.363028in}{2.402997in}}%
\pgfpathlineto{\pgfqpoint{4.370840in}{2.412580in}}%
\pgfpathlineto{\pgfqpoint{4.378646in}{2.422114in}}%
\pgfpathlineto{\pgfqpoint{4.364840in}{2.417437in}}%
\pgfpathlineto{\pgfqpoint{4.351047in}{2.412948in}}%
\pgfpathlineto{\pgfqpoint{4.337264in}{2.408647in}}%
\pgfpathlineto{\pgfqpoint{4.323494in}{2.404535in}}%
\pgfpathlineto{\pgfqpoint{4.315680in}{2.394884in}}%
\pgfpathlineto{\pgfqpoint{4.307861in}{2.385192in}}%
\pgfpathlineto{\pgfqpoint{4.300036in}{2.375458in}}%
\pgfpathlineto{\pgfqpoint{4.292206in}{2.365680in}}%
\pgfpathclose%
\pgfusepath{fill}%
\end{pgfscope}%
\begin{pgfscope}%
\pgfpathrectangle{\pgfqpoint{1.150000in}{0.150000in}}{\pgfqpoint{5.700000in}{5.700000in}}%
\pgfusepath{clip}%
\pgfsetbuttcap%
\pgfsetroundjoin%
\definecolor{currentfill}{rgb}{0.248629,0.278775,0.534556}%
\pgfsetfillcolor{currentfill}%
\pgfsetfillopacity{0.800000}%
\pgfsetlinewidth{0.000000pt}%
\definecolor{currentstroke}{rgb}{0.000000,0.000000,0.000000}%
\pgfsetstrokecolor{currentstroke}%
\pgfsetdash{}{0pt}%
\pgfpathmoveto{\pgfqpoint{2.565078in}{2.521055in}}%
\pgfpathlineto{\pgfqpoint{2.578894in}{2.501383in}}%
\pgfpathlineto{\pgfqpoint{2.592700in}{2.482026in}}%
\pgfpathlineto{\pgfqpoint{2.606494in}{2.462981in}}%
\pgfpathlineto{\pgfqpoint{2.620279in}{2.444244in}}%
\pgfpathlineto{\pgfqpoint{2.628816in}{2.448882in}}%
\pgfpathlineto{\pgfqpoint{2.637341in}{2.453696in}}%
\pgfpathlineto{\pgfqpoint{2.645853in}{2.458682in}}%
\pgfpathlineto{\pgfqpoint{2.654354in}{2.463840in}}%
\pgfpathlineto{\pgfqpoint{2.640602in}{2.482250in}}%
\pgfpathlineto{\pgfqpoint{2.626841in}{2.500968in}}%
\pgfpathlineto{\pgfqpoint{2.613070in}{2.519997in}}%
\pgfpathlineto{\pgfqpoint{2.599289in}{2.539341in}}%
\pgfpathlineto{\pgfqpoint{2.590755in}{2.534498in}}%
\pgfpathlineto{\pgfqpoint{2.582209in}{2.529835in}}%
\pgfpathlineto{\pgfqpoint{2.573650in}{2.525352in}}%
\pgfpathlineto{\pgfqpoint{2.565078in}{2.521055in}}%
\pgfpathclose%
\pgfusepath{fill}%
\end{pgfscope}%
\begin{pgfscope}%
\pgfpathrectangle{\pgfqpoint{1.150000in}{0.150000in}}{\pgfqpoint{5.700000in}{5.700000in}}%
\pgfusepath{clip}%
\pgfsetbuttcap%
\pgfsetroundjoin%
\definecolor{currentfill}{rgb}{0.123463,0.581687,0.547445}%
\pgfsetfillcolor{currentfill}%
\pgfsetfillopacity{0.800000}%
\pgfsetlinewidth{0.000000pt}%
\definecolor{currentstroke}{rgb}{0.000000,0.000000,0.000000}%
\pgfsetstrokecolor{currentstroke}%
\pgfsetdash{}{0pt}%
\pgfpathmoveto{\pgfqpoint{5.737798in}{3.333403in}}%
\pgfpathlineto{\pgfqpoint{5.752259in}{3.342189in}}%
\pgfpathlineto{\pgfqpoint{5.766739in}{3.351148in}}%
\pgfpathlineto{\pgfqpoint{5.781237in}{3.360279in}}%
\pgfpathlineto{\pgfqpoint{5.795755in}{3.369584in}}%
\pgfpathlineto{\pgfqpoint{5.802924in}{3.373658in}}%
\pgfpathlineto{\pgfqpoint{5.810091in}{3.377882in}}%
\pgfpathlineto{\pgfqpoint{5.817258in}{3.382263in}}%
\pgfpathlineto{\pgfqpoint{5.824424in}{3.386809in}}%
\pgfpathlineto{\pgfqpoint{5.809941in}{3.378206in}}%
\pgfpathlineto{\pgfqpoint{5.795476in}{3.369775in}}%
\pgfpathlineto{\pgfqpoint{5.781029in}{3.361515in}}%
\pgfpathlineto{\pgfqpoint{5.766601in}{3.353427in}}%
\pgfpathlineto{\pgfqpoint{5.759401in}{3.348171in}}%
\pgfpathlineto{\pgfqpoint{5.752201in}{3.343086in}}%
\pgfpathlineto{\pgfqpoint{5.745000in}{3.338166in}}%
\pgfpathlineto{\pgfqpoint{5.737798in}{3.333403in}}%
\pgfpathclose%
\pgfusepath{fill}%
\end{pgfscope}%
\begin{pgfscope}%
\pgfpathrectangle{\pgfqpoint{1.150000in}{0.150000in}}{\pgfqpoint{5.700000in}{5.700000in}}%
\pgfusepath{clip}%
\pgfsetbuttcap%
\pgfsetroundjoin%
\definecolor{currentfill}{rgb}{0.190631,0.407061,0.556089}%
\pgfsetfillcolor{currentfill}%
\pgfsetfillopacity{0.800000}%
\pgfsetlinewidth{0.000000pt}%
\definecolor{currentstroke}{rgb}{0.000000,0.000000,0.000000}%
\pgfsetstrokecolor{currentstroke}%
\pgfsetdash{}{0pt}%
\pgfpathmoveto{\pgfqpoint{4.928616in}{2.806700in}}%
\pgfpathlineto{\pgfqpoint{4.942691in}{2.814319in}}%
\pgfpathlineto{\pgfqpoint{4.956781in}{2.822118in}}%
\pgfpathlineto{\pgfqpoint{4.970886in}{2.830097in}}%
\pgfpathlineto{\pgfqpoint{4.985008in}{2.838256in}}%
\pgfpathlineto{\pgfqpoint{4.992571in}{2.844984in}}%
\pgfpathlineto{\pgfqpoint{5.000127in}{2.851695in}}%
\pgfpathlineto{\pgfqpoint{5.007678in}{2.858394in}}%
\pgfpathlineto{\pgfqpoint{5.015223in}{2.865084in}}%
\pgfpathlineto{\pgfqpoint{5.001117in}{2.857294in}}%
\pgfpathlineto{\pgfqpoint{4.987026in}{2.849684in}}%
\pgfpathlineto{\pgfqpoint{4.972950in}{2.842254in}}%
\pgfpathlineto{\pgfqpoint{4.958890in}{2.835003in}}%
\pgfpathlineto{\pgfqpoint{4.951330in}{2.827933in}}%
\pgfpathlineto{\pgfqpoint{4.943764in}{2.820862in}}%
\pgfpathlineto{\pgfqpoint{4.936193in}{2.813785in}}%
\pgfpathlineto{\pgfqpoint{4.928616in}{2.806700in}}%
\pgfpathclose%
\pgfusepath{fill}%
\end{pgfscope}%
\begin{pgfscope}%
\pgfpathrectangle{\pgfqpoint{1.150000in}{0.150000in}}{\pgfqpoint{5.700000in}{5.700000in}}%
\pgfusepath{clip}%
\pgfsetbuttcap%
\pgfsetroundjoin%
\definecolor{currentfill}{rgb}{0.283229,0.120777,0.440584}%
\pgfsetfillcolor{currentfill}%
\pgfsetfillopacity{0.800000}%
\pgfsetlinewidth{0.000000pt}%
\definecolor{currentstroke}{rgb}{0.000000,0.000000,0.000000}%
\pgfsetstrokecolor{currentstroke}%
\pgfsetdash{}{0pt}%
\pgfpathmoveto{\pgfqpoint{2.894286in}{2.130489in}}%
\pgfpathlineto{\pgfqpoint{2.907923in}{2.117674in}}%
\pgfpathlineto{\pgfqpoint{2.921556in}{2.105117in}}%
\pgfpathlineto{\pgfqpoint{2.935185in}{2.092817in}}%
\pgfpathlineto{\pgfqpoint{2.948810in}{2.080772in}}%
\pgfpathlineto{\pgfqpoint{2.957169in}{2.087401in}}%
\pgfpathlineto{\pgfqpoint{2.965518in}{2.094156in}}%
\pgfpathlineto{\pgfqpoint{2.973858in}{2.101036in}}%
\pgfpathlineto{\pgfqpoint{2.982189in}{2.108036in}}%
\pgfpathlineto{\pgfqpoint{2.968589in}{2.119767in}}%
\pgfpathlineto{\pgfqpoint{2.954985in}{2.131752in}}%
\pgfpathlineto{\pgfqpoint{2.941378in}{2.143993in}}%
\pgfpathlineto{\pgfqpoint{2.927767in}{2.156493in}}%
\pgfpathlineto{\pgfqpoint{2.919412in}{2.149796in}}%
\pgfpathlineto{\pgfqpoint{2.911046in}{2.143227in}}%
\pgfpathlineto{\pgfqpoint{2.902671in}{2.136791in}}%
\pgfpathlineto{\pgfqpoint{2.894286in}{2.130489in}}%
\pgfpathclose%
\pgfusepath{fill}%
\end{pgfscope}%
\begin{pgfscope}%
\pgfpathrectangle{\pgfqpoint{1.150000in}{0.150000in}}{\pgfqpoint{5.700000in}{5.700000in}}%
\pgfusepath{clip}%
\pgfsetbuttcap%
\pgfsetroundjoin%
\definecolor{currentfill}{rgb}{0.277941,0.056324,0.381191}%
\pgfsetfillcolor{currentfill}%
\pgfsetfillopacity{0.800000}%
\pgfsetlinewidth{0.000000pt}%
\definecolor{currentstroke}{rgb}{0.000000,0.000000,0.000000}%
\pgfsetstrokecolor{currentstroke}%
\pgfsetdash{}{0pt}%
\pgfpathmoveto{\pgfqpoint{3.514501in}{1.974120in}}%
\pgfpathlineto{\pgfqpoint{3.528068in}{1.970709in}}%
\pgfpathlineto{\pgfqpoint{3.541639in}{1.967507in}}%
\pgfpathlineto{\pgfqpoint{3.555215in}{1.964515in}}%
\pgfpathlineto{\pgfqpoint{3.568796in}{1.961731in}}%
\pgfpathlineto{\pgfqpoint{3.576877in}{1.971848in}}%
\pgfpathlineto{\pgfqpoint{3.584953in}{1.981985in}}%
\pgfpathlineto{\pgfqpoint{3.593024in}{1.992141in}}%
\pgfpathlineto{\pgfqpoint{3.601089in}{2.002314in}}%
\pgfpathlineto{\pgfqpoint{3.587520in}{2.004916in}}%
\pgfpathlineto{\pgfqpoint{3.573955in}{2.007726in}}%
\pgfpathlineto{\pgfqpoint{3.560396in}{2.010745in}}%
\pgfpathlineto{\pgfqpoint{3.546840in}{2.013975in}}%
\pgfpathlineto{\pgfqpoint{3.538764in}{2.003972in}}%
\pgfpathlineto{\pgfqpoint{3.530682in}{1.993994in}}%
\pgfpathlineto{\pgfqpoint{3.522594in}{1.984043in}}%
\pgfpathlineto{\pgfqpoint{3.514501in}{1.974120in}}%
\pgfpathclose%
\pgfusepath{fill}%
\end{pgfscope}%
\begin{pgfscope}%
\pgfpathrectangle{\pgfqpoint{1.150000in}{0.150000in}}{\pgfqpoint{5.700000in}{5.700000in}}%
\pgfusepath{clip}%
\pgfsetbuttcap%
\pgfsetroundjoin%
\definecolor{currentfill}{rgb}{0.253935,0.265254,0.529983}%
\pgfsetfillcolor{currentfill}%
\pgfsetfillopacity{0.800000}%
\pgfsetlinewidth{0.000000pt}%
\definecolor{currentstroke}{rgb}{0.000000,0.000000,0.000000}%
\pgfsetstrokecolor{currentstroke}%
\pgfsetdash{}{0pt}%
\pgfpathmoveto{\pgfqpoint{4.378646in}{2.422114in}}%
\pgfpathlineto{\pgfqpoint{4.392463in}{2.426978in}}%
\pgfpathlineto{\pgfqpoint{4.406293in}{2.432030in}}%
\pgfpathlineto{\pgfqpoint{4.420134in}{2.437269in}}%
\pgfpathlineto{\pgfqpoint{4.433988in}{2.442694in}}%
\pgfpathlineto{\pgfqpoint{4.441781in}{2.452055in}}%
\pgfpathlineto{\pgfqpoint{4.449569in}{2.461363in}}%
\pgfpathlineto{\pgfqpoint{4.457352in}{2.470619in}}%
\pgfpathlineto{\pgfqpoint{4.465128in}{2.479825in}}%
\pgfpathlineto{\pgfqpoint{4.451282in}{2.474538in}}%
\pgfpathlineto{\pgfqpoint{4.437448in}{2.469437in}}%
\pgfpathlineto{\pgfqpoint{4.423626in}{2.464523in}}%
\pgfpathlineto{\pgfqpoint{4.409817in}{2.459796in}}%
\pgfpathlineto{\pgfqpoint{4.402032in}{2.450440in}}%
\pgfpathlineto{\pgfqpoint{4.394242in}{2.441042in}}%
\pgfpathlineto{\pgfqpoint{4.386447in}{2.431600in}}%
\pgfpathlineto{\pgfqpoint{4.378646in}{2.422114in}}%
\pgfpathclose%
\pgfusepath{fill}%
\end{pgfscope}%
\begin{pgfscope}%
\pgfpathrectangle{\pgfqpoint{1.150000in}{0.150000in}}{\pgfqpoint{5.700000in}{5.700000in}}%
\pgfusepath{clip}%
\pgfsetbuttcap%
\pgfsetroundjoin%
\definecolor{currentfill}{rgb}{0.120092,0.600104,0.542530}%
\pgfsetfillcolor{currentfill}%
\pgfsetfillopacity{0.800000}%
\pgfsetlinewidth{0.000000pt}%
\definecolor{currentstroke}{rgb}{0.000000,0.000000,0.000000}%
\pgfsetstrokecolor{currentstroke}%
\pgfsetdash{}{0pt}%
\pgfpathmoveto{\pgfqpoint{5.824424in}{3.386809in}}%
\pgfpathlineto{\pgfqpoint{5.838926in}{3.395584in}}%
\pgfpathlineto{\pgfqpoint{5.853446in}{3.404531in}}%
\pgfpathlineto{\pgfqpoint{5.867986in}{3.413650in}}%
\pgfpathlineto{\pgfqpoint{5.882544in}{3.422942in}}%
\pgfpathlineto{\pgfqpoint{5.889674in}{3.426939in}}%
\pgfpathlineto{\pgfqpoint{5.896804in}{3.431109in}}%
\pgfpathlineto{\pgfqpoint{5.903934in}{3.435461in}}%
\pgfpathlineto{\pgfqpoint{5.911065in}{3.440003in}}%
\pgfpathlineto{\pgfqpoint{5.896543in}{3.431446in}}%
\pgfpathlineto{\pgfqpoint{5.882040in}{3.423060in}}%
\pgfpathlineto{\pgfqpoint{5.867555in}{3.414845in}}%
\pgfpathlineto{\pgfqpoint{5.853088in}{3.406801in}}%
\pgfpathlineto{\pgfqpoint{5.845921in}{3.401516in}}%
\pgfpathlineto{\pgfqpoint{5.838755in}{3.396428in}}%
\pgfpathlineto{\pgfqpoint{5.831590in}{3.391528in}}%
\pgfpathlineto{\pgfqpoint{5.824424in}{3.386809in}}%
\pgfpathclose%
\pgfusepath{fill}%
\end{pgfscope}%
\begin{pgfscope}%
\pgfpathrectangle{\pgfqpoint{1.150000in}{0.150000in}}{\pgfqpoint{5.700000in}{5.700000in}}%
\pgfusepath{clip}%
\pgfsetbuttcap%
\pgfsetroundjoin%
\definecolor{currentfill}{rgb}{0.276022,0.044167,0.370164}%
\pgfsetfillcolor{currentfill}%
\pgfsetfillopacity{0.800000}%
\pgfsetlinewidth{0.000000pt}%
\definecolor{currentstroke}{rgb}{0.000000,0.000000,0.000000}%
\pgfsetstrokecolor{currentstroke}%
\pgfsetdash{}{0pt}%
\pgfpathmoveto{\pgfqpoint{3.286647in}{1.959129in}}%
\pgfpathlineto{\pgfqpoint{3.300209in}{1.952661in}}%
\pgfpathlineto{\pgfqpoint{3.313773in}{1.946414in}}%
\pgfpathlineto{\pgfqpoint{3.327338in}{1.940390in}}%
\pgfpathlineto{\pgfqpoint{3.340906in}{1.934585in}}%
\pgfpathlineto{\pgfqpoint{3.349077in}{1.943707in}}%
\pgfpathlineto{\pgfqpoint{3.357242in}{1.952888in}}%
\pgfpathlineto{\pgfqpoint{3.365400in}{1.962125in}}%
\pgfpathlineto{\pgfqpoint{3.373552in}{1.971417in}}%
\pgfpathlineto{\pgfqpoint{3.360000in}{1.976976in}}%
\pgfpathlineto{\pgfqpoint{3.346451in}{1.982755in}}%
\pgfpathlineto{\pgfqpoint{3.332903in}{1.988756in}}%
\pgfpathlineto{\pgfqpoint{3.319357in}{1.994978in}}%
\pgfpathlineto{\pgfqpoint{3.311190in}{1.985921in}}%
\pgfpathlineto{\pgfqpoint{3.303016in}{1.976925in}}%
\pgfpathlineto{\pgfqpoint{3.294835in}{1.967994in}}%
\pgfpathlineto{\pgfqpoint{3.286647in}{1.959129in}}%
\pgfpathclose%
\pgfusepath{fill}%
\end{pgfscope}%
\begin{pgfscope}%
\pgfpathrectangle{\pgfqpoint{1.150000in}{0.150000in}}{\pgfqpoint{5.700000in}{5.700000in}}%
\pgfusepath{clip}%
\pgfsetbuttcap%
\pgfsetroundjoin%
\definecolor{currentfill}{rgb}{0.233603,0.313828,0.543914}%
\pgfsetfillcolor{currentfill}%
\pgfsetfillopacity{0.800000}%
\pgfsetlinewidth{0.000000pt}%
\definecolor{currentstroke}{rgb}{0.000000,0.000000,0.000000}%
\pgfsetstrokecolor{currentstroke}%
\pgfsetdash{}{0pt}%
\pgfpathmoveto{\pgfqpoint{2.509700in}{2.602948in}}%
\pgfpathlineto{\pgfqpoint{2.523562in}{2.581987in}}%
\pgfpathlineto{\pgfqpoint{2.537412in}{2.561354in}}%
\pgfpathlineto{\pgfqpoint{2.551251in}{2.541044in}}%
\pgfpathlineto{\pgfqpoint{2.565078in}{2.521055in}}%
\pgfpathlineto{\pgfqpoint{2.573650in}{2.525352in}}%
\pgfpathlineto{\pgfqpoint{2.582209in}{2.529835in}}%
\pgfpathlineto{\pgfqpoint{2.590755in}{2.534498in}}%
\pgfpathlineto{\pgfqpoint{2.599289in}{2.539341in}}%
\pgfpathlineto{\pgfqpoint{2.585497in}{2.559001in}}%
\pgfpathlineto{\pgfqpoint{2.571693in}{2.578981in}}%
\pgfpathlineto{\pgfqpoint{2.557879in}{2.599284in}}%
\pgfpathlineto{\pgfqpoint{2.544052in}{2.619913in}}%
\pgfpathlineto{\pgfqpoint{2.535484in}{2.615388in}}%
\pgfpathlineto{\pgfqpoint{2.526902in}{2.611050in}}%
\pgfpathlineto{\pgfqpoint{2.518308in}{2.606902in}}%
\pgfpathlineto{\pgfqpoint{2.509700in}{2.602948in}}%
\pgfpathclose%
\pgfusepath{fill}%
\end{pgfscope}%
\begin{pgfscope}%
\pgfpathrectangle{\pgfqpoint{1.150000in}{0.150000in}}{\pgfqpoint{5.700000in}{5.700000in}}%
\pgfusepath{clip}%
\pgfsetbuttcap%
\pgfsetroundjoin%
\definecolor{currentfill}{rgb}{0.119699,0.618490,0.536347}%
\pgfsetfillcolor{currentfill}%
\pgfsetfillopacity{0.800000}%
\pgfsetlinewidth{0.000000pt}%
\definecolor{currentstroke}{rgb}{0.000000,0.000000,0.000000}%
\pgfsetstrokecolor{currentstroke}%
\pgfsetdash{}{0pt}%
\pgfpathmoveto{\pgfqpoint{5.911065in}{3.440003in}}%
\pgfpathlineto{\pgfqpoint{5.925606in}{3.448731in}}%
\pgfpathlineto{\pgfqpoint{5.940166in}{3.457631in}}%
\pgfpathlineto{\pgfqpoint{5.954745in}{3.466702in}}%
\pgfpathlineto{\pgfqpoint{5.969343in}{3.475944in}}%
\pgfpathlineto{\pgfqpoint{5.976437in}{3.479928in}}%
\pgfpathlineto{\pgfqpoint{5.983531in}{3.484111in}}%
\pgfpathlineto{\pgfqpoint{5.990627in}{3.488501in}}%
\pgfpathlineto{\pgfqpoint{5.997725in}{3.493106in}}%
\pgfpathlineto{\pgfqpoint{5.983165in}{3.484630in}}%
\pgfpathlineto{\pgfqpoint{5.968625in}{3.476326in}}%
\pgfpathlineto{\pgfqpoint{5.954103in}{3.468191in}}%
\pgfpathlineto{\pgfqpoint{5.939600in}{3.460226in}}%
\pgfpathlineto{\pgfqpoint{5.932464in}{3.454845in}}%
\pgfpathlineto{\pgfqpoint{5.925329in}{3.449687in}}%
\pgfpathlineto{\pgfqpoint{5.918197in}{3.444742in}}%
\pgfpathlineto{\pgfqpoint{5.911065in}{3.440003in}}%
\pgfpathclose%
\pgfusepath{fill}%
\end{pgfscope}%
\begin{pgfscope}%
\pgfpathrectangle{\pgfqpoint{1.150000in}{0.150000in}}{\pgfqpoint{5.700000in}{5.700000in}}%
\pgfusepath{clip}%
\pgfsetbuttcap%
\pgfsetroundjoin%
\definecolor{currentfill}{rgb}{0.180629,0.429975,0.557282}%
\pgfsetfillcolor{currentfill}%
\pgfsetfillopacity{0.800000}%
\pgfsetlinewidth{0.000000pt}%
\definecolor{currentstroke}{rgb}{0.000000,0.000000,0.000000}%
\pgfsetstrokecolor{currentstroke}%
\pgfsetdash{}{0pt}%
\pgfpathmoveto{\pgfqpoint{5.015223in}{2.865084in}}%
\pgfpathlineto{\pgfqpoint{5.029345in}{2.873053in}}%
\pgfpathlineto{\pgfqpoint{5.043483in}{2.881202in}}%
\pgfpathlineto{\pgfqpoint{5.057637in}{2.889531in}}%
\pgfpathlineto{\pgfqpoint{5.071807in}{2.898039in}}%
\pgfpathlineto{\pgfqpoint{5.079330in}{2.904335in}}%
\pgfpathlineto{\pgfqpoint{5.086847in}{2.910622in}}%
\pgfpathlineto{\pgfqpoint{5.094358in}{2.916908in}}%
\pgfpathlineto{\pgfqpoint{5.101864in}{2.923195in}}%
\pgfpathlineto{\pgfqpoint{5.087710in}{2.915090in}}%
\pgfpathlineto{\pgfqpoint{5.073573in}{2.907164in}}%
\pgfpathlineto{\pgfqpoint{5.059451in}{2.899417in}}%
\pgfpathlineto{\pgfqpoint{5.045345in}{2.891849in}}%
\pgfpathlineto{\pgfqpoint{5.037823in}{2.885148in}}%
\pgfpathlineto{\pgfqpoint{5.030295in}{2.878457in}}%
\pgfpathlineto{\pgfqpoint{5.022762in}{2.871770in}}%
\pgfpathlineto{\pgfqpoint{5.015223in}{2.865084in}}%
\pgfpathclose%
\pgfusepath{fill}%
\end{pgfscope}%
\begin{pgfscope}%
\pgfpathrectangle{\pgfqpoint{1.150000in}{0.150000in}}{\pgfqpoint{5.700000in}{5.700000in}}%
\pgfusepath{clip}%
\pgfsetbuttcap%
\pgfsetroundjoin%
\definecolor{currentfill}{rgb}{0.277941,0.056324,0.381191}%
\pgfsetfillcolor{currentfill}%
\pgfsetfillopacity{0.800000}%
\pgfsetlinewidth{0.000000pt}%
\definecolor{currentstroke}{rgb}{0.000000,0.000000,0.000000}%
\pgfsetstrokecolor{currentstroke}%
\pgfsetdash{}{0pt}%
\pgfpathmoveto{\pgfqpoint{3.145209in}{1.986509in}}%
\pgfpathlineto{\pgfqpoint{3.158785in}{1.977938in}}%
\pgfpathlineto{\pgfqpoint{3.172362in}{1.969599in}}%
\pgfpathlineto{\pgfqpoint{3.185938in}{1.961492in}}%
\pgfpathlineto{\pgfqpoint{3.199514in}{1.953615in}}%
\pgfpathlineto{\pgfqpoint{3.207749in}{1.961901in}}%
\pgfpathlineto{\pgfqpoint{3.215977in}{1.970271in}}%
\pgfpathlineto{\pgfqpoint{3.224197in}{1.978722in}}%
\pgfpathlineto{\pgfqpoint{3.232410in}{1.987252in}}%
\pgfpathlineto{\pgfqpoint{3.218852in}{1.994851in}}%
\pgfpathlineto{\pgfqpoint{3.205295in}{2.002680in}}%
\pgfpathlineto{\pgfqpoint{3.191739in}{2.010740in}}%
\pgfpathlineto{\pgfqpoint{3.178182in}{2.019032in}}%
\pgfpathlineto{\pgfqpoint{3.169950in}{2.010768in}}%
\pgfpathlineto{\pgfqpoint{3.161711in}{2.002592in}}%
\pgfpathlineto{\pgfqpoint{3.153464in}{1.994504in}}%
\pgfpathlineto{\pgfqpoint{3.145209in}{1.986509in}}%
\pgfpathclose%
\pgfusepath{fill}%
\end{pgfscope}%
\begin{pgfscope}%
\pgfpathrectangle{\pgfqpoint{1.150000in}{0.150000in}}{\pgfqpoint{5.700000in}{5.700000in}}%
\pgfusepath{clip}%
\pgfsetbuttcap%
\pgfsetroundjoin%
\definecolor{currentfill}{rgb}{0.244972,0.287675,0.537260}%
\pgfsetfillcolor{currentfill}%
\pgfsetfillopacity{0.800000}%
\pgfsetlinewidth{0.000000pt}%
\definecolor{currentstroke}{rgb}{0.000000,0.000000,0.000000}%
\pgfsetstrokecolor{currentstroke}%
\pgfsetdash{}{0pt}%
\pgfpathmoveto{\pgfqpoint{4.465128in}{2.479825in}}%
\pgfpathlineto{\pgfqpoint{4.478987in}{2.485299in}}%
\pgfpathlineto{\pgfqpoint{4.492858in}{2.490958in}}%
\pgfpathlineto{\pgfqpoint{4.506742in}{2.496804in}}%
\pgfpathlineto{\pgfqpoint{4.520639in}{2.502835in}}%
\pgfpathlineto{\pgfqpoint{4.528402in}{2.511835in}}%
\pgfpathlineto{\pgfqpoint{4.536160in}{2.520782in}}%
\pgfpathlineto{\pgfqpoint{4.543911in}{2.529678in}}%
\pgfpathlineto{\pgfqpoint{4.551657in}{2.538525in}}%
\pgfpathlineto{\pgfqpoint{4.537768in}{2.532665in}}%
\pgfpathlineto{\pgfqpoint{4.523892in}{2.526991in}}%
\pgfpathlineto{\pgfqpoint{4.510029in}{2.521502in}}%
\pgfpathlineto{\pgfqpoint{4.496178in}{2.516199in}}%
\pgfpathlineto{\pgfqpoint{4.488424in}{2.507169in}}%
\pgfpathlineto{\pgfqpoint{4.480664in}{2.498098in}}%
\pgfpathlineto{\pgfqpoint{4.472899in}{2.488984in}}%
\pgfpathlineto{\pgfqpoint{4.465128in}{2.479825in}}%
\pgfpathclose%
\pgfusepath{fill}%
\end{pgfscope}%
\begin{pgfscope}%
\pgfpathrectangle{\pgfqpoint{1.150000in}{0.150000in}}{\pgfqpoint{5.700000in}{5.700000in}}%
\pgfusepath{clip}%
\pgfsetbuttcap%
\pgfsetroundjoin%
\definecolor{currentfill}{rgb}{0.282656,0.100196,0.422160}%
\pgfsetfillcolor{currentfill}%
\pgfsetfillopacity{0.800000}%
\pgfsetlinewidth{0.000000pt}%
\definecolor{currentstroke}{rgb}{0.000000,0.000000,0.000000}%
\pgfsetstrokecolor{currentstroke}%
\pgfsetdash{}{0pt}%
\pgfpathmoveto{\pgfqpoint{2.948810in}{2.080772in}}%
\pgfpathlineto{\pgfqpoint{2.962432in}{2.068980in}}%
\pgfpathlineto{\pgfqpoint{2.976051in}{2.057439in}}%
\pgfpathlineto{\pgfqpoint{2.989666in}{2.046148in}}%
\pgfpathlineto{\pgfqpoint{3.003279in}{2.035105in}}%
\pgfpathlineto{\pgfqpoint{3.011613in}{2.042059in}}%
\pgfpathlineto{\pgfqpoint{3.019937in}{2.049132in}}%
\pgfpathlineto{\pgfqpoint{3.028253in}{2.056321in}}%
\pgfpathlineto{\pgfqpoint{3.036560in}{2.063623in}}%
\pgfpathlineto{\pgfqpoint{3.022971in}{2.074353in}}%
\pgfpathlineto{\pgfqpoint{3.009380in}{2.085331in}}%
\pgfpathlineto{\pgfqpoint{2.995786in}{2.096558in}}%
\pgfpathlineto{\pgfqpoint{2.982189in}{2.108036in}}%
\pgfpathlineto{\pgfqpoint{2.973858in}{2.101036in}}%
\pgfpathlineto{\pgfqpoint{2.965518in}{2.094156in}}%
\pgfpathlineto{\pgfqpoint{2.957169in}{2.087401in}}%
\pgfpathlineto{\pgfqpoint{2.948810in}{2.080772in}}%
\pgfpathclose%
\pgfusepath{fill}%
\end{pgfscope}%
\begin{pgfscope}%
\pgfpathrectangle{\pgfqpoint{1.150000in}{0.150000in}}{\pgfqpoint{5.700000in}{5.700000in}}%
\pgfusepath{clip}%
\pgfsetbuttcap%
\pgfsetroundjoin%
\definecolor{currentfill}{rgb}{0.277018,0.050344,0.375715}%
\pgfsetfillcolor{currentfill}%
\pgfsetfillopacity{0.800000}%
\pgfsetlinewidth{0.000000pt}%
\definecolor{currentstroke}{rgb}{0.000000,0.000000,0.000000}%
\pgfsetstrokecolor{currentstroke}%
\pgfsetdash{}{0pt}%
\pgfpathmoveto{\pgfqpoint{3.427785in}{1.951359in}}%
\pgfpathlineto{\pgfqpoint{3.441350in}{1.946884in}}%
\pgfpathlineto{\pgfqpoint{3.454919in}{1.942624in}}%
\pgfpathlineto{\pgfqpoint{3.468492in}{1.938576in}}%
\pgfpathlineto{\pgfqpoint{3.482068in}{1.934740in}}%
\pgfpathlineto{\pgfqpoint{3.490185in}{1.944535in}}%
\pgfpathlineto{\pgfqpoint{3.498296in}{1.954364in}}%
\pgfpathlineto{\pgfqpoint{3.506401in}{1.964226in}}%
\pgfpathlineto{\pgfqpoint{3.514501in}{1.974120in}}%
\pgfpathlineto{\pgfqpoint{3.500937in}{1.977742in}}%
\pgfpathlineto{\pgfqpoint{3.487378in}{1.981577in}}%
\pgfpathlineto{\pgfqpoint{3.473823in}{1.985624in}}%
\pgfpathlineto{\pgfqpoint{3.460271in}{1.989885in}}%
\pgfpathlineto{\pgfqpoint{3.452158in}{1.980193in}}%
\pgfpathlineto{\pgfqpoint{3.444040in}{1.970541in}}%
\pgfpathlineto{\pgfqpoint{3.435915in}{1.960928in}}%
\pgfpathlineto{\pgfqpoint{3.427785in}{1.951359in}}%
\pgfpathclose%
\pgfusepath{fill}%
\end{pgfscope}%
\begin{pgfscope}%
\pgfpathrectangle{\pgfqpoint{1.150000in}{0.150000in}}{\pgfqpoint{5.700000in}{5.700000in}}%
\pgfusepath{clip}%
\pgfsetbuttcap%
\pgfsetroundjoin%
\definecolor{currentfill}{rgb}{0.172719,0.448791,0.557885}%
\pgfsetfillcolor{currentfill}%
\pgfsetfillopacity{0.800000}%
\pgfsetlinewidth{0.000000pt}%
\definecolor{currentstroke}{rgb}{0.000000,0.000000,0.000000}%
\pgfsetstrokecolor{currentstroke}%
\pgfsetdash{}{0pt}%
\pgfpathmoveto{\pgfqpoint{5.101864in}{2.923195in}}%
\pgfpathlineto{\pgfqpoint{5.116033in}{2.931478in}}%
\pgfpathlineto{\pgfqpoint{5.130219in}{2.939941in}}%
\pgfpathlineto{\pgfqpoint{5.144421in}{2.948582in}}%
\pgfpathlineto{\pgfqpoint{5.158640in}{2.957402in}}%
\pgfpathlineto{\pgfqpoint{5.166123in}{2.963271in}}%
\pgfpathlineto{\pgfqpoint{5.173599in}{2.969144in}}%
\pgfpathlineto{\pgfqpoint{5.181070in}{2.975025in}}%
\pgfpathlineto{\pgfqpoint{5.188535in}{2.980920in}}%
\pgfpathlineto{\pgfqpoint{5.174334in}{2.972538in}}%
\pgfpathlineto{\pgfqpoint{5.160150in}{2.964333in}}%
\pgfpathlineto{\pgfqpoint{5.145982in}{2.956306in}}%
\pgfpathlineto{\pgfqpoint{5.131830in}{2.948458in}}%
\pgfpathlineto{\pgfqpoint{5.124346in}{2.942115in}}%
\pgfpathlineto{\pgfqpoint{5.116858in}{2.935794in}}%
\pgfpathlineto{\pgfqpoint{5.109364in}{2.929489in}}%
\pgfpathlineto{\pgfqpoint{5.101864in}{2.923195in}}%
\pgfpathclose%
\pgfusepath{fill}%
\end{pgfscope}%
\begin{pgfscope}%
\pgfpathrectangle{\pgfqpoint{1.150000in}{0.150000in}}{\pgfqpoint{5.700000in}{5.700000in}}%
\pgfusepath{clip}%
\pgfsetbuttcap%
\pgfsetroundjoin%
\definecolor{currentfill}{rgb}{0.233603,0.313828,0.543914}%
\pgfsetfillcolor{currentfill}%
\pgfsetfillopacity{0.800000}%
\pgfsetlinewidth{0.000000pt}%
\definecolor{currentstroke}{rgb}{0.000000,0.000000,0.000000}%
\pgfsetstrokecolor{currentstroke}%
\pgfsetdash{}{0pt}%
\pgfpathmoveto{\pgfqpoint{4.551657in}{2.538525in}}%
\pgfpathlineto{\pgfqpoint{4.565559in}{2.544570in}}%
\pgfpathlineto{\pgfqpoint{4.579474in}{2.550800in}}%
\pgfpathlineto{\pgfqpoint{4.593402in}{2.557214in}}%
\pgfpathlineto{\pgfqpoint{4.607344in}{2.563813in}}%
\pgfpathlineto{\pgfqpoint{4.615075in}{2.572422in}}%
\pgfpathlineto{\pgfqpoint{4.622801in}{2.580978in}}%
\pgfpathlineto{\pgfqpoint{4.630520in}{2.589485in}}%
\pgfpathlineto{\pgfqpoint{4.638233in}{2.597946in}}%
\pgfpathlineto{\pgfqpoint{4.624300in}{2.591551in}}%
\pgfpathlineto{\pgfqpoint{4.610380in}{2.585341in}}%
\pgfpathlineto{\pgfqpoint{4.596474in}{2.579315in}}%
\pgfpathlineto{\pgfqpoint{4.582581in}{2.573474in}}%
\pgfpathlineto{\pgfqpoint{4.574859in}{2.564797in}}%
\pgfpathlineto{\pgfqpoint{4.567130in}{2.556082in}}%
\pgfpathlineto{\pgfqpoint{4.559396in}{2.547326in}}%
\pgfpathlineto{\pgfqpoint{4.551657in}{2.538525in}}%
\pgfpathclose%
\pgfusepath{fill}%
\end{pgfscope}%
\begin{pgfscope}%
\pgfpathrectangle{\pgfqpoint{1.150000in}{0.150000in}}{\pgfqpoint{5.700000in}{5.700000in}}%
\pgfusepath{clip}%
\pgfsetbuttcap%
\pgfsetroundjoin%
\definecolor{currentfill}{rgb}{0.123444,0.636809,0.528763}%
\pgfsetfillcolor{currentfill}%
\pgfsetfillopacity{0.800000}%
\pgfsetlinewidth{0.000000pt}%
\definecolor{currentstroke}{rgb}{0.000000,0.000000,0.000000}%
\pgfsetstrokecolor{currentstroke}%
\pgfsetdash{}{0pt}%
\pgfpathmoveto{\pgfqpoint{5.997725in}{3.493106in}}%
\pgfpathlineto{\pgfqpoint{6.012303in}{3.501751in}}%
\pgfpathlineto{\pgfqpoint{6.026901in}{3.510568in}}%
\pgfpathlineto{\pgfqpoint{6.041518in}{3.519555in}}%
\pgfpathlineto{\pgfqpoint{6.056154in}{3.528713in}}%
\pgfpathlineto{\pgfqpoint{6.063214in}{3.532754in}}%
\pgfpathlineto{\pgfqpoint{6.070276in}{3.537020in}}%
\pgfpathlineto{\pgfqpoint{6.077340in}{3.541520in}}%
\pgfpathlineto{\pgfqpoint{6.062735in}{3.532960in}}%
\pgfpathlineto{\pgfqpoint{6.048148in}{3.524569in}}%
\pgfpathlineto{\pgfqpoint{6.033581in}{3.516348in}}%
\pgfpathlineto{\pgfqpoint{6.019032in}{3.508297in}}%
\pgfpathlineto{\pgfqpoint{6.011927in}{3.502995in}}%
\pgfpathlineto{\pgfqpoint{6.004825in}{3.497934in}}%
\pgfpathlineto{\pgfqpoint{5.997725in}{3.493106in}}%
\pgfpathclose%
\pgfusepath{fill}%
\end{pgfscope}%
\begin{pgfscope}%
\pgfpathrectangle{\pgfqpoint{1.150000in}{0.150000in}}{\pgfqpoint{5.700000in}{5.700000in}}%
\pgfusepath{clip}%
\pgfsetbuttcap%
\pgfsetroundjoin%
\definecolor{currentfill}{rgb}{0.218130,0.347432,0.550038}%
\pgfsetfillcolor{currentfill}%
\pgfsetfillopacity{0.800000}%
\pgfsetlinewidth{0.000000pt}%
\definecolor{currentstroke}{rgb}{0.000000,0.000000,0.000000}%
\pgfsetstrokecolor{currentstroke}%
\pgfsetdash{}{0pt}%
\pgfpathmoveto{\pgfqpoint{2.454122in}{2.690118in}}%
\pgfpathlineto{\pgfqpoint{2.468037in}{2.667820in}}%
\pgfpathlineto{\pgfqpoint{2.481937in}{2.645860in}}%
\pgfpathlineto{\pgfqpoint{2.495825in}{2.624238in}}%
\pgfpathlineto{\pgfqpoint{2.509700in}{2.602948in}}%
\pgfpathlineto{\pgfqpoint{2.518308in}{2.606902in}}%
\pgfpathlineto{\pgfqpoint{2.526902in}{2.611050in}}%
\pgfpathlineto{\pgfqpoint{2.535484in}{2.615388in}}%
\pgfpathlineto{\pgfqpoint{2.544052in}{2.619913in}}%
\pgfpathlineto{\pgfqpoint{2.530214in}{2.640871in}}%
\pgfpathlineto{\pgfqpoint{2.516363in}{2.662161in}}%
\pgfpathlineto{\pgfqpoint{2.502500in}{2.683786in}}%
\pgfpathlineto{\pgfqpoint{2.488623in}{2.705751in}}%
\pgfpathlineto{\pgfqpoint{2.480019in}{2.701546in}}%
\pgfpathlineto{\pgfqpoint{2.471401in}{2.697537in}}%
\pgfpathlineto{\pgfqpoint{2.462768in}{2.693727in}}%
\pgfpathlineto{\pgfqpoint{2.454122in}{2.690118in}}%
\pgfpathclose%
\pgfusepath{fill}%
\end{pgfscope}%
\begin{pgfscope}%
\pgfpathrectangle{\pgfqpoint{1.150000in}{0.150000in}}{\pgfqpoint{5.700000in}{5.700000in}}%
\pgfusepath{clip}%
\pgfsetbuttcap%
\pgfsetroundjoin%
\definecolor{currentfill}{rgb}{0.283197,0.115680,0.436115}%
\pgfsetfillcolor{currentfill}%
\pgfsetfillopacity{0.800000}%
\pgfsetlinewidth{0.000000pt}%
\definecolor{currentstroke}{rgb}{0.000000,0.000000,0.000000}%
\pgfsetstrokecolor{currentstroke}%
\pgfsetdash{}{0pt}%
\pgfpathmoveto{\pgfqpoint{3.828454in}{2.072151in}}%
\pgfpathlineto{\pgfqpoint{3.842085in}{2.072457in}}%
\pgfpathlineto{\pgfqpoint{3.855724in}{2.072962in}}%
\pgfpathlineto{\pgfqpoint{3.869370in}{2.073664in}}%
\pgfpathlineto{\pgfqpoint{3.883024in}{2.074564in}}%
\pgfpathlineto{\pgfqpoint{3.891005in}{2.085239in}}%
\pgfpathlineto{\pgfqpoint{3.898981in}{2.095891in}}%
\pgfpathlineto{\pgfqpoint{3.906953in}{2.106519in}}%
\pgfpathlineto{\pgfqpoint{3.914919in}{2.117123in}}%
\pgfpathlineto{\pgfqpoint{3.901272in}{2.116136in}}%
\pgfpathlineto{\pgfqpoint{3.887633in}{2.115347in}}%
\pgfpathlineto{\pgfqpoint{3.874003in}{2.114756in}}%
\pgfpathlineto{\pgfqpoint{3.860379in}{2.114362in}}%
\pgfpathlineto{\pgfqpoint{3.852406in}{2.103834in}}%
\pgfpathlineto{\pgfqpoint{3.844427in}{2.093289in}}%
\pgfpathlineto{\pgfqpoint{3.836443in}{2.082728in}}%
\pgfpathlineto{\pgfqpoint{3.828454in}{2.072151in}}%
\pgfpathclose%
\pgfusepath{fill}%
\end{pgfscope}%
\begin{pgfscope}%
\pgfpathrectangle{\pgfqpoint{1.150000in}{0.150000in}}{\pgfqpoint{5.700000in}{5.700000in}}%
\pgfusepath{clip}%
\pgfsetbuttcap%
\pgfsetroundjoin%
\definecolor{currentfill}{rgb}{0.282884,0.135920,0.453427}%
\pgfsetfillcolor{currentfill}%
\pgfsetfillopacity{0.800000}%
\pgfsetlinewidth{0.000000pt}%
\definecolor{currentstroke}{rgb}{0.000000,0.000000,0.000000}%
\pgfsetstrokecolor{currentstroke}%
\pgfsetdash{}{0pt}%
\pgfpathmoveto{\pgfqpoint{3.914919in}{2.117123in}}%
\pgfpathlineto{\pgfqpoint{3.928573in}{2.118306in}}%
\pgfpathlineto{\pgfqpoint{3.942237in}{2.119685in}}%
\pgfpathlineto{\pgfqpoint{3.955908in}{2.121259in}}%
\pgfpathlineto{\pgfqpoint{3.969589in}{2.123029in}}%
\pgfpathlineto{\pgfqpoint{3.977543in}{2.133674in}}%
\pgfpathlineto{\pgfqpoint{3.985491in}{2.144287in}}%
\pgfpathlineto{\pgfqpoint{3.993435in}{2.154867in}}%
\pgfpathlineto{\pgfqpoint{4.001374in}{2.165414in}}%
\pgfpathlineto{\pgfqpoint{3.987700in}{2.163590in}}%
\pgfpathlineto{\pgfqpoint{3.974035in}{2.161960in}}%
\pgfpathlineto{\pgfqpoint{3.960379in}{2.160526in}}%
\pgfpathlineto{\pgfqpoint{3.946732in}{2.159288in}}%
\pgfpathlineto{\pgfqpoint{3.938786in}{2.148784in}}%
\pgfpathlineto{\pgfqpoint{3.930835in}{2.138255in}}%
\pgfpathlineto{\pgfqpoint{3.922879in}{2.127702in}}%
\pgfpathlineto{\pgfqpoint{3.914919in}{2.117123in}}%
\pgfpathclose%
\pgfusepath{fill}%
\end{pgfscope}%
\begin{pgfscope}%
\pgfpathrectangle{\pgfqpoint{1.150000in}{0.150000in}}{\pgfqpoint{5.700000in}{5.700000in}}%
\pgfusepath{clip}%
\pgfsetbuttcap%
\pgfsetroundjoin%
\definecolor{currentfill}{rgb}{0.163625,0.471133,0.558148}%
\pgfsetfillcolor{currentfill}%
\pgfsetfillopacity{0.800000}%
\pgfsetlinewidth{0.000000pt}%
\definecolor{currentstroke}{rgb}{0.000000,0.000000,0.000000}%
\pgfsetstrokecolor{currentstroke}%
\pgfsetdash{}{0pt}%
\pgfpathmoveto{\pgfqpoint{5.188535in}{2.980920in}}%
\pgfpathlineto{\pgfqpoint{5.202752in}{2.989481in}}%
\pgfpathlineto{\pgfqpoint{5.216986in}{2.998220in}}%
\pgfpathlineto{\pgfqpoint{5.231236in}{3.007137in}}%
\pgfpathlineto{\pgfqpoint{5.245504in}{3.016231in}}%
\pgfpathlineto{\pgfqpoint{5.252944in}{3.021687in}}%
\pgfpathlineto{\pgfqpoint{5.260379in}{3.027158in}}%
\pgfpathlineto{\pgfqpoint{5.267808in}{3.032651in}}%
\pgfpathlineto{\pgfqpoint{5.275231in}{3.038171in}}%
\pgfpathlineto{\pgfqpoint{5.260984in}{3.029547in}}%
\pgfpathlineto{\pgfqpoint{5.246753in}{3.021100in}}%
\pgfpathlineto{\pgfqpoint{5.232538in}{3.012831in}}%
\pgfpathlineto{\pgfqpoint{5.218341in}{3.004739in}}%
\pgfpathlineto{\pgfqpoint{5.210897in}{2.998738in}}%
\pgfpathlineto{\pgfqpoint{5.203448in}{2.992772in}}%
\pgfpathlineto{\pgfqpoint{5.195994in}{2.986834in}}%
\pgfpathlineto{\pgfqpoint{5.188535in}{2.980920in}}%
\pgfpathclose%
\pgfusepath{fill}%
\end{pgfscope}%
\begin{pgfscope}%
\pgfpathrectangle{\pgfqpoint{1.150000in}{0.150000in}}{\pgfqpoint{5.700000in}{5.700000in}}%
\pgfusepath{clip}%
\pgfsetbuttcap%
\pgfsetroundjoin%
\definecolor{currentfill}{rgb}{0.282327,0.094955,0.417331}%
\pgfsetfillcolor{currentfill}%
\pgfsetfillopacity{0.800000}%
\pgfsetlinewidth{0.000000pt}%
\definecolor{currentstroke}{rgb}{0.000000,0.000000,0.000000}%
\pgfsetstrokecolor{currentstroke}%
\pgfsetdash{}{0pt}%
\pgfpathmoveto{\pgfqpoint{3.741961in}{2.030945in}}%
\pgfpathlineto{\pgfqpoint{3.755572in}{2.030333in}}%
\pgfpathlineto{\pgfqpoint{3.769190in}{2.029922in}}%
\pgfpathlineto{\pgfqpoint{3.782815in}{2.029711in}}%
\pgfpathlineto{\pgfqpoint{3.796447in}{2.029700in}}%
\pgfpathlineto{\pgfqpoint{3.804457in}{2.040333in}}%
\pgfpathlineto{\pgfqpoint{3.812461in}{2.050953in}}%
\pgfpathlineto{\pgfqpoint{3.820460in}{2.061559in}}%
\pgfpathlineto{\pgfqpoint{3.828454in}{2.072151in}}%
\pgfpathlineto{\pgfqpoint{3.814831in}{2.072044in}}%
\pgfpathlineto{\pgfqpoint{3.801214in}{2.072136in}}%
\pgfpathlineto{\pgfqpoint{3.787605in}{2.072428in}}%
\pgfpathlineto{\pgfqpoint{3.774002in}{2.072921in}}%
\pgfpathlineto{\pgfqpoint{3.766000in}{2.062436in}}%
\pgfpathlineto{\pgfqpoint{3.757992in}{2.051945in}}%
\pgfpathlineto{\pgfqpoint{3.749979in}{2.041447in}}%
\pgfpathlineto{\pgfqpoint{3.741961in}{2.030945in}}%
\pgfpathclose%
\pgfusepath{fill}%
\end{pgfscope}%
\begin{pgfscope}%
\pgfpathrectangle{\pgfqpoint{1.150000in}{0.150000in}}{\pgfqpoint{5.700000in}{5.700000in}}%
\pgfusepath{clip}%
\pgfsetbuttcap%
\pgfsetroundjoin%
\definecolor{currentfill}{rgb}{0.281412,0.155834,0.469201}%
\pgfsetfillcolor{currentfill}%
\pgfsetfillopacity{0.800000}%
\pgfsetlinewidth{0.000000pt}%
\definecolor{currentstroke}{rgb}{0.000000,0.000000,0.000000}%
\pgfsetstrokecolor{currentstroke}%
\pgfsetdash{}{0pt}%
\pgfpathmoveto{\pgfqpoint{4.001374in}{2.165414in}}%
\pgfpathlineto{\pgfqpoint{4.015056in}{2.167433in}}%
\pgfpathlineto{\pgfqpoint{4.028748in}{2.169646in}}%
\pgfpathlineto{\pgfqpoint{4.042449in}{2.172053in}}%
\pgfpathlineto{\pgfqpoint{4.056159in}{2.174652in}}%
\pgfpathlineto{\pgfqpoint{4.064086in}{2.185201in}}%
\pgfpathlineto{\pgfqpoint{4.072007in}{2.195709in}}%
\pgfpathlineto{\pgfqpoint{4.079924in}{2.206176in}}%
\pgfpathlineto{\pgfqpoint{4.087836in}{2.216603in}}%
\pgfpathlineto{\pgfqpoint{4.074132in}{2.213981in}}%
\pgfpathlineto{\pgfqpoint{4.060438in}{2.211551in}}%
\pgfpathlineto{\pgfqpoint{4.046753in}{2.209315in}}%
\pgfpathlineto{\pgfqpoint{4.033077in}{2.207272in}}%
\pgfpathlineto{\pgfqpoint{4.025159in}{2.196857in}}%
\pgfpathlineto{\pgfqpoint{4.017236in}{2.186409in}}%
\pgfpathlineto{\pgfqpoint{4.009307in}{2.175928in}}%
\pgfpathlineto{\pgfqpoint{4.001374in}{2.165414in}}%
\pgfpathclose%
\pgfusepath{fill}%
\end{pgfscope}%
\begin{pgfscope}%
\pgfpathrectangle{\pgfqpoint{1.150000in}{0.150000in}}{\pgfqpoint{5.700000in}{5.700000in}}%
\pgfusepath{clip}%
\pgfsetbuttcap%
\pgfsetroundjoin%
\definecolor{currentfill}{rgb}{0.281446,0.084320,0.407414}%
\pgfsetfillcolor{currentfill}%
\pgfsetfillopacity{0.800000}%
\pgfsetlinewidth{0.000000pt}%
\definecolor{currentstroke}{rgb}{0.000000,0.000000,0.000000}%
\pgfsetstrokecolor{currentstroke}%
\pgfsetdash{}{0pt}%
\pgfpathmoveto{\pgfqpoint{3.003279in}{2.035105in}}%
\pgfpathlineto{\pgfqpoint{3.016889in}{2.024308in}}%
\pgfpathlineto{\pgfqpoint{3.030497in}{2.013756in}}%
\pgfpathlineto{\pgfqpoint{3.044103in}{2.003447in}}%
\pgfpathlineto{\pgfqpoint{3.057707in}{1.993380in}}%
\pgfpathlineto{\pgfqpoint{3.066017in}{2.000659in}}%
\pgfpathlineto{\pgfqpoint{3.074318in}{2.008048in}}%
\pgfpathlineto{\pgfqpoint{3.082611in}{2.015545in}}%
\pgfpathlineto{\pgfqpoint{3.090895in}{2.023147in}}%
\pgfpathlineto{\pgfqpoint{3.077314in}{2.032903in}}%
\pgfpathlineto{\pgfqpoint{3.063731in}{2.042900in}}%
\pgfpathlineto{\pgfqpoint{3.050146in}{2.053139in}}%
\pgfpathlineto{\pgfqpoint{3.036560in}{2.063623in}}%
\pgfpathlineto{\pgfqpoint{3.028253in}{2.056321in}}%
\pgfpathlineto{\pgfqpoint{3.019937in}{2.049132in}}%
\pgfpathlineto{\pgfqpoint{3.011613in}{2.042059in}}%
\pgfpathlineto{\pgfqpoint{3.003279in}{2.035105in}}%
\pgfpathclose%
\pgfusepath{fill}%
\end{pgfscope}%
\begin{pgfscope}%
\pgfpathrectangle{\pgfqpoint{1.150000in}{0.150000in}}{\pgfqpoint{5.700000in}{5.700000in}}%
\pgfusepath{clip}%
\pgfsetbuttcap%
\pgfsetroundjoin%
\definecolor{currentfill}{rgb}{0.221989,0.339161,0.548752}%
\pgfsetfillcolor{currentfill}%
\pgfsetfillopacity{0.800000}%
\pgfsetlinewidth{0.000000pt}%
\definecolor{currentstroke}{rgb}{0.000000,0.000000,0.000000}%
\pgfsetstrokecolor{currentstroke}%
\pgfsetdash{}{0pt}%
\pgfpathmoveto{\pgfqpoint{4.638233in}{2.597946in}}%
\pgfpathlineto{\pgfqpoint{4.652180in}{2.604524in}}%
\pgfpathlineto{\pgfqpoint{4.666140in}{2.611286in}}%
\pgfpathlineto{\pgfqpoint{4.680115in}{2.618232in}}%
\pgfpathlineto{\pgfqpoint{4.694103in}{2.625362in}}%
\pgfpathlineto{\pgfqpoint{4.701801in}{2.633553in}}%
\pgfpathlineto{\pgfqpoint{4.709493in}{2.641694in}}%
\pgfpathlineto{\pgfqpoint{4.717178in}{2.649790in}}%
\pgfpathlineto{\pgfqpoint{4.724858in}{2.657842in}}%
\pgfpathlineto{\pgfqpoint{4.710879in}{2.650950in}}%
\pgfpathlineto{\pgfqpoint{4.696914in}{2.644242in}}%
\pgfpathlineto{\pgfqpoint{4.682964in}{2.637716in}}%
\pgfpathlineto{\pgfqpoint{4.669027in}{2.631374in}}%
\pgfpathlineto{\pgfqpoint{4.661337in}{2.623073in}}%
\pgfpathlineto{\pgfqpoint{4.653642in}{2.614737in}}%
\pgfpathlineto{\pgfqpoint{4.645940in}{2.606362in}}%
\pgfpathlineto{\pgfqpoint{4.638233in}{2.597946in}}%
\pgfpathclose%
\pgfusepath{fill}%
\end{pgfscope}%
\begin{pgfscope}%
\pgfpathrectangle{\pgfqpoint{1.150000in}{0.150000in}}{\pgfqpoint{5.700000in}{5.700000in}}%
\pgfusepath{clip}%
\pgfsetbuttcap%
\pgfsetroundjoin%
\definecolor{currentfill}{rgb}{0.278012,0.180367,0.486697}%
\pgfsetfillcolor{currentfill}%
\pgfsetfillopacity{0.800000}%
\pgfsetlinewidth{0.000000pt}%
\definecolor{currentstroke}{rgb}{0.000000,0.000000,0.000000}%
\pgfsetstrokecolor{currentstroke}%
\pgfsetdash{}{0pt}%
\pgfpathmoveto{\pgfqpoint{4.087836in}{2.216603in}}%
\pgfpathlineto{\pgfqpoint{4.101549in}{2.219418in}}%
\pgfpathlineto{\pgfqpoint{4.115273in}{2.222426in}}%
\pgfpathlineto{\pgfqpoint{4.129006in}{2.225625in}}%
\pgfpathlineto{\pgfqpoint{4.142750in}{2.229015in}}%
\pgfpathlineto{\pgfqpoint{4.150650in}{2.239405in}}%
\pgfpathlineto{\pgfqpoint{4.158544in}{2.249747in}}%
\pgfpathlineto{\pgfqpoint{4.166434in}{2.260043in}}%
\pgfpathlineto{\pgfqpoint{4.174318in}{2.270291in}}%
\pgfpathlineto{\pgfqpoint{4.160581in}{2.266910in}}%
\pgfpathlineto{\pgfqpoint{4.146854in}{2.263720in}}%
\pgfpathlineto{\pgfqpoint{4.133137in}{2.260721in}}%
\pgfpathlineto{\pgfqpoint{4.119430in}{2.257914in}}%
\pgfpathlineto{\pgfqpoint{4.111539in}{2.247645in}}%
\pgfpathlineto{\pgfqpoint{4.103643in}{2.237337in}}%
\pgfpathlineto{\pgfqpoint{4.095742in}{2.226990in}}%
\pgfpathlineto{\pgfqpoint{4.087836in}{2.216603in}}%
\pgfpathclose%
\pgfusepath{fill}%
\end{pgfscope}%
\begin{pgfscope}%
\pgfpathrectangle{\pgfqpoint{1.150000in}{0.150000in}}{\pgfqpoint{5.700000in}{5.700000in}}%
\pgfusepath{clip}%
\pgfsetbuttcap%
\pgfsetroundjoin%
\definecolor{currentfill}{rgb}{0.280894,0.078907,0.402329}%
\pgfsetfillcolor{currentfill}%
\pgfsetfillopacity{0.800000}%
\pgfsetlinewidth{0.000000pt}%
\definecolor{currentstroke}{rgb}{0.000000,0.000000,0.000000}%
\pgfsetstrokecolor{currentstroke}%
\pgfsetdash{}{0pt}%
\pgfpathmoveto{\pgfqpoint{3.655417in}{1.993973in}}%
\pgfpathlineto{\pgfqpoint{3.669013in}{1.992401in}}%
\pgfpathlineto{\pgfqpoint{3.682615in}{1.991033in}}%
\pgfpathlineto{\pgfqpoint{3.696223in}{1.989867in}}%
\pgfpathlineto{\pgfqpoint{3.709837in}{1.988904in}}%
\pgfpathlineto{\pgfqpoint{3.717876in}{1.999416in}}%
\pgfpathlineto{\pgfqpoint{3.725909in}{2.009928in}}%
\pgfpathlineto{\pgfqpoint{3.733938in}{2.020438in}}%
\pgfpathlineto{\pgfqpoint{3.741961in}{2.030945in}}%
\pgfpathlineto{\pgfqpoint{3.728357in}{2.031758in}}%
\pgfpathlineto{\pgfqpoint{3.714758in}{2.032774in}}%
\pgfpathlineto{\pgfqpoint{3.701166in}{2.033992in}}%
\pgfpathlineto{\pgfqpoint{3.687580in}{2.035414in}}%
\pgfpathlineto{\pgfqpoint{3.679547in}{2.025046in}}%
\pgfpathlineto{\pgfqpoint{3.671509in}{2.014682in}}%
\pgfpathlineto{\pgfqpoint{3.663466in}{2.004324in}}%
\pgfpathlineto{\pgfqpoint{3.655417in}{1.993973in}}%
\pgfpathclose%
\pgfusepath{fill}%
\end{pgfscope}%
\begin{pgfscope}%
\pgfpathrectangle{\pgfqpoint{1.150000in}{0.150000in}}{\pgfqpoint{5.700000in}{5.700000in}}%
\pgfusepath{clip}%
\pgfsetbuttcap%
\pgfsetroundjoin%
\definecolor{currentfill}{rgb}{0.156270,0.489624,0.557936}%
\pgfsetfillcolor{currentfill}%
\pgfsetfillopacity{0.800000}%
\pgfsetlinewidth{0.000000pt}%
\definecolor{currentstroke}{rgb}{0.000000,0.000000,0.000000}%
\pgfsetstrokecolor{currentstroke}%
\pgfsetdash{}{0pt}%
\pgfpathmoveto{\pgfqpoint{5.275231in}{3.038171in}}%
\pgfpathlineto{\pgfqpoint{5.289496in}{3.046972in}}%
\pgfpathlineto{\pgfqpoint{5.303777in}{3.055950in}}%
\pgfpathlineto{\pgfqpoint{5.318076in}{3.065106in}}%
\pgfpathlineto{\pgfqpoint{5.332393in}{3.074439in}}%
\pgfpathlineto{\pgfqpoint{5.339790in}{3.079499in}}%
\pgfpathlineto{\pgfqpoint{5.347182in}{3.084588in}}%
\pgfpathlineto{\pgfqpoint{5.354568in}{3.089713in}}%
\pgfpathlineto{\pgfqpoint{5.361950in}{3.094880in}}%
\pgfpathlineto{\pgfqpoint{5.347655in}{3.086051in}}%
\pgfpathlineto{\pgfqpoint{5.333378in}{3.077400in}}%
\pgfpathlineto{\pgfqpoint{5.319118in}{3.068924in}}%
\pgfpathlineto{\pgfqpoint{5.304874in}{3.060625in}}%
\pgfpathlineto{\pgfqpoint{5.297471in}{3.054944in}}%
\pgfpathlineto{\pgfqpoint{5.290063in}{3.049312in}}%
\pgfpathlineto{\pgfqpoint{5.282650in}{3.043723in}}%
\pgfpathlineto{\pgfqpoint{5.275231in}{3.038171in}}%
\pgfpathclose%
\pgfusepath{fill}%
\end{pgfscope}%
\begin{pgfscope}%
\pgfpathrectangle{\pgfqpoint{1.150000in}{0.150000in}}{\pgfqpoint{5.700000in}{5.700000in}}%
\pgfusepath{clip}%
\pgfsetbuttcap%
\pgfsetroundjoin%
\definecolor{currentfill}{rgb}{0.277018,0.050344,0.375715}%
\pgfsetfillcolor{currentfill}%
\pgfsetfillopacity{0.800000}%
\pgfsetlinewidth{0.000000pt}%
\definecolor{currentstroke}{rgb}{0.000000,0.000000,0.000000}%
\pgfsetstrokecolor{currentstroke}%
\pgfsetdash{}{0pt}%
\pgfpathmoveto{\pgfqpoint{3.199514in}{1.953615in}}%
\pgfpathlineto{\pgfqpoint{3.213091in}{1.945966in}}%
\pgfpathlineto{\pgfqpoint{3.226668in}{1.938545in}}%
\pgfpathlineto{\pgfqpoint{3.240246in}{1.931351in}}%
\pgfpathlineto{\pgfqpoint{3.253825in}{1.924381in}}%
\pgfpathlineto{\pgfqpoint{3.262041in}{1.932957in}}%
\pgfpathlineto{\pgfqpoint{3.270250in}{1.941608in}}%
\pgfpathlineto{\pgfqpoint{3.278452in}{1.950333in}}%
\pgfpathlineto{\pgfqpoint{3.286647in}{1.959129in}}%
\pgfpathlineto{\pgfqpoint{3.273086in}{1.965822in}}%
\pgfpathlineto{\pgfqpoint{3.259526in}{1.972739in}}%
\pgfpathlineto{\pgfqpoint{3.245967in}{1.979882in}}%
\pgfpathlineto{\pgfqpoint{3.232410in}{1.987252in}}%
\pgfpathlineto{\pgfqpoint{3.224197in}{1.978722in}}%
\pgfpathlineto{\pgfqpoint{3.215977in}{1.970271in}}%
\pgfpathlineto{\pgfqpoint{3.207749in}{1.961901in}}%
\pgfpathlineto{\pgfqpoint{3.199514in}{1.953615in}}%
\pgfpathclose%
\pgfusepath{fill}%
\end{pgfscope}%
\begin{pgfscope}%
\pgfpathrectangle{\pgfqpoint{1.150000in}{0.150000in}}{\pgfqpoint{5.700000in}{5.700000in}}%
\pgfusepath{clip}%
\pgfsetbuttcap%
\pgfsetroundjoin%
\definecolor{currentfill}{rgb}{0.276022,0.044167,0.370164}%
\pgfsetfillcolor{currentfill}%
\pgfsetfillopacity{0.800000}%
\pgfsetlinewidth{0.000000pt}%
\definecolor{currentstroke}{rgb}{0.000000,0.000000,0.000000}%
\pgfsetstrokecolor{currentstroke}%
\pgfsetdash{}{0pt}%
\pgfpathmoveto{\pgfqpoint{3.340906in}{1.934585in}}%
\pgfpathlineto{\pgfqpoint{3.354475in}{1.928999in}}%
\pgfpathlineto{\pgfqpoint{3.368047in}{1.923632in}}%
\pgfpathlineto{\pgfqpoint{3.381622in}{1.918481in}}%
\pgfpathlineto{\pgfqpoint{3.395199in}{1.913546in}}%
\pgfpathlineto{\pgfqpoint{3.403355in}{1.922926in}}%
\pgfpathlineto{\pgfqpoint{3.411505in}{1.932356in}}%
\pgfpathlineto{\pgfqpoint{3.419648in}{1.941834in}}%
\pgfpathlineto{\pgfqpoint{3.427785in}{1.951359in}}%
\pgfpathlineto{\pgfqpoint{3.414222in}{1.956049in}}%
\pgfpathlineto{\pgfqpoint{3.400663in}{1.960954in}}%
\pgfpathlineto{\pgfqpoint{3.387106in}{1.966076in}}%
\pgfpathlineto{\pgfqpoint{3.373552in}{1.971417in}}%
\pgfpathlineto{\pgfqpoint{3.365400in}{1.962125in}}%
\pgfpathlineto{\pgfqpoint{3.357242in}{1.952888in}}%
\pgfpathlineto{\pgfqpoint{3.349077in}{1.943707in}}%
\pgfpathlineto{\pgfqpoint{3.340906in}{1.934585in}}%
\pgfpathclose%
\pgfusepath{fill}%
\end{pgfscope}%
\begin{pgfscope}%
\pgfpathrectangle{\pgfqpoint{1.150000in}{0.150000in}}{\pgfqpoint{5.700000in}{5.700000in}}%
\pgfusepath{clip}%
\pgfsetbuttcap%
\pgfsetroundjoin%
\definecolor{currentfill}{rgb}{0.271828,0.209303,0.504434}%
\pgfsetfillcolor{currentfill}%
\pgfsetfillopacity{0.800000}%
\pgfsetlinewidth{0.000000pt}%
\definecolor{currentstroke}{rgb}{0.000000,0.000000,0.000000}%
\pgfsetstrokecolor{currentstroke}%
\pgfsetdash{}{0pt}%
\pgfpathmoveto{\pgfqpoint{4.174318in}{2.270291in}}%
\pgfpathlineto{\pgfqpoint{4.188066in}{2.273863in}}%
\pgfpathlineto{\pgfqpoint{4.201824in}{2.277626in}}%
\pgfpathlineto{\pgfqpoint{4.215593in}{2.281579in}}%
\pgfpathlineto{\pgfqpoint{4.229373in}{2.285721in}}%
\pgfpathlineto{\pgfqpoint{4.237246in}{2.295895in}}%
\pgfpathlineto{\pgfqpoint{4.245114in}{2.306016in}}%
\pgfpathlineto{\pgfqpoint{4.252976in}{2.316084in}}%
\pgfpathlineto{\pgfqpoint{4.260833in}{2.326101in}}%
\pgfpathlineto{\pgfqpoint{4.247059in}{2.322000in}}%
\pgfpathlineto{\pgfqpoint{4.233296in}{2.318089in}}%
\pgfpathlineto{\pgfqpoint{4.219545in}{2.314367in}}%
\pgfpathlineto{\pgfqpoint{4.205803in}{2.310836in}}%
\pgfpathlineto{\pgfqpoint{4.197940in}{2.300765in}}%
\pgfpathlineto{\pgfqpoint{4.190071in}{2.290652in}}%
\pgfpathlineto{\pgfqpoint{4.182198in}{2.280494in}}%
\pgfpathlineto{\pgfqpoint{4.174318in}{2.270291in}}%
\pgfpathclose%
\pgfusepath{fill}%
\end{pgfscope}%
\begin{pgfscope}%
\pgfpathrectangle{\pgfqpoint{1.150000in}{0.150000in}}{\pgfqpoint{5.700000in}{5.700000in}}%
\pgfusepath{clip}%
\pgfsetbuttcap%
\pgfsetroundjoin%
\definecolor{currentfill}{rgb}{0.210503,0.363727,0.552206}%
\pgfsetfillcolor{currentfill}%
\pgfsetfillopacity{0.800000}%
\pgfsetlinewidth{0.000000pt}%
\definecolor{currentstroke}{rgb}{0.000000,0.000000,0.000000}%
\pgfsetstrokecolor{currentstroke}%
\pgfsetdash{}{0pt}%
\pgfpathmoveto{\pgfqpoint{4.724858in}{2.657842in}}%
\pgfpathlineto{\pgfqpoint{4.738851in}{2.664916in}}%
\pgfpathlineto{\pgfqpoint{4.752858in}{2.672174in}}%
\pgfpathlineto{\pgfqpoint{4.766880in}{2.679614in}}%
\pgfpathlineto{\pgfqpoint{4.780916in}{2.687237in}}%
\pgfpathlineto{\pgfqpoint{4.788579in}{2.694989in}}%
\pgfpathlineto{\pgfqpoint{4.796235in}{2.702697in}}%
\pgfpathlineto{\pgfqpoint{4.803886in}{2.710363in}}%
\pgfpathlineto{\pgfqpoint{4.811530in}{2.717989in}}%
\pgfpathlineto{\pgfqpoint{4.797504in}{2.710638in}}%
\pgfpathlineto{\pgfqpoint{4.783494in}{2.703469in}}%
\pgfpathlineto{\pgfqpoint{4.769497in}{2.696482in}}%
\pgfpathlineto{\pgfqpoint{4.755515in}{2.689677in}}%
\pgfpathlineto{\pgfqpoint{4.747860in}{2.681768in}}%
\pgfpathlineto{\pgfqpoint{4.740198in}{2.673827in}}%
\pgfpathlineto{\pgfqpoint{4.732531in}{2.665853in}}%
\pgfpathlineto{\pgfqpoint{4.724858in}{2.657842in}}%
\pgfpathclose%
\pgfusepath{fill}%
\end{pgfscope}%
\begin{pgfscope}%
\pgfpathrectangle{\pgfqpoint{1.150000in}{0.150000in}}{\pgfqpoint{5.700000in}{5.700000in}}%
\pgfusepath{clip}%
\pgfsetbuttcap%
\pgfsetroundjoin%
\definecolor{currentfill}{rgb}{0.147607,0.511733,0.557049}%
\pgfsetfillcolor{currentfill}%
\pgfsetfillopacity{0.800000}%
\pgfsetlinewidth{0.000000pt}%
\definecolor{currentstroke}{rgb}{0.000000,0.000000,0.000000}%
\pgfsetstrokecolor{currentstroke}%
\pgfsetdash{}{0pt}%
\pgfpathmoveto{\pgfqpoint{5.361950in}{3.094880in}}%
\pgfpathlineto{\pgfqpoint{5.376261in}{3.103885in}}%
\pgfpathlineto{\pgfqpoint{5.390591in}{3.113066in}}%
\pgfpathlineto{\pgfqpoint{5.404937in}{3.122424in}}%
\pgfpathlineto{\pgfqpoint{5.419302in}{3.131959in}}%
\pgfpathlineto{\pgfqpoint{5.426655in}{3.136646in}}%
\pgfpathlineto{\pgfqpoint{5.434004in}{3.141379in}}%
\pgfpathlineto{\pgfqpoint{5.441347in}{3.146162in}}%
\pgfpathlineto{\pgfqpoint{5.448686in}{3.151003in}}%
\pgfpathlineto{\pgfqpoint{5.434346in}{3.142007in}}%
\pgfpathlineto{\pgfqpoint{5.420023in}{3.133187in}}%
\pgfpathlineto{\pgfqpoint{5.405717in}{3.124542in}}%
\pgfpathlineto{\pgfqpoint{5.391428in}{3.116073in}}%
\pgfpathlineto{\pgfqpoint{5.384065in}{3.110684in}}%
\pgfpathlineto{\pgfqpoint{5.376698in}{3.105359in}}%
\pgfpathlineto{\pgfqpoint{5.369326in}{3.100093in}}%
\pgfpathlineto{\pgfqpoint{5.361950in}{3.094880in}}%
\pgfpathclose%
\pgfusepath{fill}%
\end{pgfscope}%
\begin{pgfscope}%
\pgfpathrectangle{\pgfqpoint{1.150000in}{0.150000in}}{\pgfqpoint{5.700000in}{5.700000in}}%
\pgfusepath{clip}%
\pgfsetbuttcap%
\pgfsetroundjoin%
\definecolor{currentfill}{rgb}{0.278791,0.062145,0.386592}%
\pgfsetfillcolor{currentfill}%
\pgfsetfillopacity{0.800000}%
\pgfsetlinewidth{0.000000pt}%
\definecolor{currentstroke}{rgb}{0.000000,0.000000,0.000000}%
\pgfsetstrokecolor{currentstroke}%
\pgfsetdash{}{0pt}%
\pgfpathmoveto{\pgfqpoint{3.568796in}{1.961731in}}%
\pgfpathlineto{\pgfqpoint{3.582381in}{1.959155in}}%
\pgfpathlineto{\pgfqpoint{3.595972in}{1.956786in}}%
\pgfpathlineto{\pgfqpoint{3.609567in}{1.954623in}}%
\pgfpathlineto{\pgfqpoint{3.623168in}{1.952664in}}%
\pgfpathlineto{\pgfqpoint{3.631238in}{1.962975in}}%
\pgfpathlineto{\pgfqpoint{3.639303in}{1.973297in}}%
\pgfpathlineto{\pgfqpoint{3.647363in}{1.983631in}}%
\pgfpathlineto{\pgfqpoint{3.655417in}{1.993973in}}%
\pgfpathlineto{\pgfqpoint{3.641827in}{1.995750in}}%
\pgfpathlineto{\pgfqpoint{3.628242in}{1.997732in}}%
\pgfpathlineto{\pgfqpoint{3.614663in}{1.999919in}}%
\pgfpathlineto{\pgfqpoint{3.601089in}{2.002314in}}%
\pgfpathlineto{\pgfqpoint{3.593024in}{1.992141in}}%
\pgfpathlineto{\pgfqpoint{3.584953in}{1.981985in}}%
\pgfpathlineto{\pgfqpoint{3.576877in}{1.971848in}}%
\pgfpathlineto{\pgfqpoint{3.568796in}{1.961731in}}%
\pgfpathclose%
\pgfusepath{fill}%
\end{pgfscope}%
\begin{pgfscope}%
\pgfpathrectangle{\pgfqpoint{1.150000in}{0.150000in}}{\pgfqpoint{5.700000in}{5.700000in}}%
\pgfusepath{clip}%
\pgfsetbuttcap%
\pgfsetroundjoin%
\definecolor{currentfill}{rgb}{0.201239,0.383670,0.554294}%
\pgfsetfillcolor{currentfill}%
\pgfsetfillopacity{0.800000}%
\pgfsetlinewidth{0.000000pt}%
\definecolor{currentstroke}{rgb}{0.000000,0.000000,0.000000}%
\pgfsetstrokecolor{currentstroke}%
\pgfsetdash{}{0pt}%
\pgfpathmoveto{\pgfqpoint{2.398324in}{2.782778in}}%
\pgfpathlineto{\pgfqpoint{2.412296in}{2.759086in}}%
\pgfpathlineto{\pgfqpoint{2.426252in}{2.735748in}}%
\pgfpathlineto{\pgfqpoint{2.440194in}{2.712760in}}%
\pgfpathlineto{\pgfqpoint{2.454122in}{2.690118in}}%
\pgfpathlineto{\pgfqpoint{2.462768in}{2.693727in}}%
\pgfpathlineto{\pgfqpoint{2.471401in}{2.697537in}}%
\pgfpathlineto{\pgfqpoint{2.480019in}{2.701546in}}%
\pgfpathlineto{\pgfqpoint{2.488623in}{2.705751in}}%
\pgfpathlineto{\pgfqpoint{2.474734in}{2.728057in}}%
\pgfpathlineto{\pgfqpoint{2.460830in}{2.750709in}}%
\pgfpathlineto{\pgfqpoint{2.446912in}{2.773710in}}%
\pgfpathlineto{\pgfqpoint{2.432980in}{2.797064in}}%
\pgfpathlineto{\pgfqpoint{2.424337in}{2.793183in}}%
\pgfpathlineto{\pgfqpoint{2.415681in}{2.789506in}}%
\pgfpathlineto{\pgfqpoint{2.407010in}{2.786037in}}%
\pgfpathlineto{\pgfqpoint{2.398324in}{2.782778in}}%
\pgfpathclose%
\pgfusepath{fill}%
\end{pgfscope}%
\begin{pgfscope}%
\pgfpathrectangle{\pgfqpoint{1.150000in}{0.150000in}}{\pgfqpoint{5.700000in}{5.700000in}}%
\pgfusepath{clip}%
\pgfsetbuttcap%
\pgfsetroundjoin%
\definecolor{currentfill}{rgb}{0.265145,0.232956,0.516599}%
\pgfsetfillcolor{currentfill}%
\pgfsetfillopacity{0.800000}%
\pgfsetlinewidth{0.000000pt}%
\definecolor{currentstroke}{rgb}{0.000000,0.000000,0.000000}%
\pgfsetstrokecolor{currentstroke}%
\pgfsetdash{}{0pt}%
\pgfpathmoveto{\pgfqpoint{4.260833in}{2.326101in}}%
\pgfpathlineto{\pgfqpoint{4.274617in}{2.330392in}}%
\pgfpathlineto{\pgfqpoint{4.288413in}{2.334871in}}%
\pgfpathlineto{\pgfqpoint{4.302221in}{2.339539in}}%
\pgfpathlineto{\pgfqpoint{4.316040in}{2.344395in}}%
\pgfpathlineto{\pgfqpoint{4.323885in}{2.354301in}}%
\pgfpathlineto{\pgfqpoint{4.331724in}{2.364149in}}%
\pgfpathlineto{\pgfqpoint{4.339558in}{2.373942in}}%
\pgfpathlineto{\pgfqpoint{4.347387in}{2.383680in}}%
\pgfpathlineto{\pgfqpoint{4.333574in}{2.378897in}}%
\pgfpathlineto{\pgfqpoint{4.319774in}{2.374303in}}%
\pgfpathlineto{\pgfqpoint{4.305984in}{2.369897in}}%
\pgfpathlineto{\pgfqpoint{4.292206in}{2.365680in}}%
\pgfpathlineto{\pgfqpoint{4.284371in}{2.355857in}}%
\pgfpathlineto{\pgfqpoint{4.276530in}{2.345987in}}%
\pgfpathlineto{\pgfqpoint{4.268684in}{2.336069in}}%
\pgfpathlineto{\pgfqpoint{4.260833in}{2.326101in}}%
\pgfpathclose%
\pgfusepath{fill}%
\end{pgfscope}%
\begin{pgfscope}%
\pgfpathrectangle{\pgfqpoint{1.150000in}{0.150000in}}{\pgfqpoint{5.700000in}{5.700000in}}%
\pgfusepath{clip}%
\pgfsetbuttcap%
\pgfsetroundjoin%
\definecolor{currentfill}{rgb}{0.279566,0.067836,0.391917}%
\pgfsetfillcolor{currentfill}%
\pgfsetfillopacity{0.800000}%
\pgfsetlinewidth{0.000000pt}%
\definecolor{currentstroke}{rgb}{0.000000,0.000000,0.000000}%
\pgfsetstrokecolor{currentstroke}%
\pgfsetdash{}{0pt}%
\pgfpathmoveto{\pgfqpoint{3.057707in}{1.993380in}}%
\pgfpathlineto{\pgfqpoint{3.071309in}{1.983552in}}%
\pgfpathlineto{\pgfqpoint{3.084910in}{1.973964in}}%
\pgfpathlineto{\pgfqpoint{3.098510in}{1.964612in}}%
\pgfpathlineto{\pgfqpoint{3.112109in}{1.955496in}}%
\pgfpathlineto{\pgfqpoint{3.120396in}{1.963098in}}%
\pgfpathlineto{\pgfqpoint{3.128675in}{1.970803in}}%
\pgfpathlineto{\pgfqpoint{3.136946in}{1.978607in}}%
\pgfpathlineto{\pgfqpoint{3.145209in}{1.986509in}}%
\pgfpathlineto{\pgfqpoint{3.131632in}{1.995314in}}%
\pgfpathlineto{\pgfqpoint{3.118054in}{2.004354in}}%
\pgfpathlineto{\pgfqpoint{3.104475in}{2.013632in}}%
\pgfpathlineto{\pgfqpoint{3.090895in}{2.023147in}}%
\pgfpathlineto{\pgfqpoint{3.082611in}{2.015545in}}%
\pgfpathlineto{\pgfqpoint{3.074318in}{2.008048in}}%
\pgfpathlineto{\pgfqpoint{3.066017in}{2.000659in}}%
\pgfpathlineto{\pgfqpoint{3.057707in}{1.993380in}}%
\pgfpathclose%
\pgfusepath{fill}%
\end{pgfscope}%
\begin{pgfscope}%
\pgfpathrectangle{\pgfqpoint{1.150000in}{0.150000in}}{\pgfqpoint{5.700000in}{5.700000in}}%
\pgfusepath{clip}%
\pgfsetbuttcap%
\pgfsetroundjoin%
\definecolor{currentfill}{rgb}{0.140536,0.530132,0.555659}%
\pgfsetfillcolor{currentfill}%
\pgfsetfillopacity{0.800000}%
\pgfsetlinewidth{0.000000pt}%
\definecolor{currentstroke}{rgb}{0.000000,0.000000,0.000000}%
\pgfsetstrokecolor{currentstroke}%
\pgfsetdash{}{0pt}%
\pgfpathmoveto{\pgfqpoint{5.448686in}{3.151003in}}%
\pgfpathlineto{\pgfqpoint{5.463045in}{3.160175in}}%
\pgfpathlineto{\pgfqpoint{5.477421in}{3.169523in}}%
\pgfpathlineto{\pgfqpoint{5.491815in}{3.179047in}}%
\pgfpathlineto{\pgfqpoint{5.506227in}{3.188747in}}%
\pgfpathlineto{\pgfqpoint{5.513536in}{3.193091in}}%
\pgfpathlineto{\pgfqpoint{5.520841in}{3.197498in}}%
\pgfpathlineto{\pgfqpoint{5.528141in}{3.201972in}}%
\pgfpathlineto{\pgfqpoint{5.535438in}{3.206521in}}%
\pgfpathlineto{\pgfqpoint{5.521051in}{3.197393in}}%
\pgfpathlineto{\pgfqpoint{5.506683in}{3.188440in}}%
\pgfpathlineto{\pgfqpoint{5.492333in}{3.179663in}}%
\pgfpathlineto{\pgfqpoint{5.477999in}{3.171060in}}%
\pgfpathlineto{\pgfqpoint{5.470677in}{3.165929in}}%
\pgfpathlineto{\pgfqpoint{5.463351in}{3.160881in}}%
\pgfpathlineto{\pgfqpoint{5.456021in}{3.155907in}}%
\pgfpathlineto{\pgfqpoint{5.448686in}{3.151003in}}%
\pgfpathclose%
\pgfusepath{fill}%
\end{pgfscope}%
\begin{pgfscope}%
\pgfpathrectangle{\pgfqpoint{1.150000in}{0.150000in}}{\pgfqpoint{5.700000in}{5.700000in}}%
\pgfusepath{clip}%
\pgfsetbuttcap%
\pgfsetroundjoin%
\definecolor{currentfill}{rgb}{0.275191,0.194905,0.496005}%
\pgfsetfillcolor{currentfill}%
\pgfsetfillopacity{0.800000}%
\pgfsetlinewidth{0.000000pt}%
\definecolor{currentstroke}{rgb}{0.000000,0.000000,0.000000}%
\pgfsetstrokecolor{currentstroke}%
\pgfsetdash{}{0pt}%
\pgfpathmoveto{\pgfqpoint{2.696223in}{2.285571in}}%
\pgfpathlineto{\pgfqpoint{2.709962in}{2.269143in}}%
\pgfpathlineto{\pgfqpoint{2.723693in}{2.252999in}}%
\pgfpathlineto{\pgfqpoint{2.737417in}{2.237136in}}%
\pgfpathlineto{\pgfqpoint{2.751134in}{2.221553in}}%
\pgfpathlineto{\pgfqpoint{2.759620in}{2.226560in}}%
\pgfpathlineto{\pgfqpoint{2.768095in}{2.231731in}}%
\pgfpathlineto{\pgfqpoint{2.776558in}{2.237062in}}%
\pgfpathlineto{\pgfqpoint{2.785011in}{2.242550in}}%
\pgfpathlineto{\pgfqpoint{2.771325in}{2.257780in}}%
\pgfpathlineto{\pgfqpoint{2.757633in}{2.273290in}}%
\pgfpathlineto{\pgfqpoint{2.743933in}{2.289081in}}%
\pgfpathlineto{\pgfqpoint{2.730227in}{2.305155in}}%
\pgfpathlineto{\pgfqpoint{2.721743in}{2.300007in}}%
\pgfpathlineto{\pgfqpoint{2.713248in}{2.295026in}}%
\pgfpathlineto{\pgfqpoint{2.704742in}{2.290212in}}%
\pgfpathlineto{\pgfqpoint{2.696223in}{2.285571in}}%
\pgfpathclose%
\pgfusepath{fill}%
\end{pgfscope}%
\begin{pgfscope}%
\pgfpathrectangle{\pgfqpoint{1.150000in}{0.150000in}}{\pgfqpoint{5.700000in}{5.700000in}}%
\pgfusepath{clip}%
\pgfsetbuttcap%
\pgfsetroundjoin%
\definecolor{currentfill}{rgb}{0.201239,0.383670,0.554294}%
\pgfsetfillcolor{currentfill}%
\pgfsetfillopacity{0.800000}%
\pgfsetlinewidth{0.000000pt}%
\definecolor{currentstroke}{rgb}{0.000000,0.000000,0.000000}%
\pgfsetstrokecolor{currentstroke}%
\pgfsetdash{}{0pt}%
\pgfpathmoveto{\pgfqpoint{4.811530in}{2.717989in}}%
\pgfpathlineto{\pgfqpoint{4.825570in}{2.725523in}}%
\pgfpathlineto{\pgfqpoint{4.839625in}{2.733238in}}%
\pgfpathlineto{\pgfqpoint{4.853695in}{2.741135in}}%
\pgfpathlineto{\pgfqpoint{4.867781in}{2.749214in}}%
\pgfpathlineto{\pgfqpoint{4.875407in}{2.756514in}}%
\pgfpathlineto{\pgfqpoint{4.883027in}{2.763773in}}%
\pgfpathlineto{\pgfqpoint{4.890640in}{2.770997in}}%
\pgfpathlineto{\pgfqpoint{4.898247in}{2.778187in}}%
\pgfpathlineto{\pgfqpoint{4.884175in}{2.770413in}}%
\pgfpathlineto{\pgfqpoint{4.870117in}{2.762821in}}%
\pgfpathlineto{\pgfqpoint{4.856074in}{2.755409in}}%
\pgfpathlineto{\pgfqpoint{4.842046in}{2.748179in}}%
\pgfpathlineto{\pgfqpoint{4.834426in}{2.740672in}}%
\pgfpathlineto{\pgfqpoint{4.826800in}{2.733141in}}%
\pgfpathlineto{\pgfqpoint{4.819168in}{2.725581in}}%
\pgfpathlineto{\pgfqpoint{4.811530in}{2.717989in}}%
\pgfpathclose%
\pgfusepath{fill}%
\end{pgfscope}%
\begin{pgfscope}%
\pgfpathrectangle{\pgfqpoint{1.150000in}{0.150000in}}{\pgfqpoint{5.700000in}{5.700000in}}%
\pgfusepath{clip}%
\pgfsetbuttcap%
\pgfsetroundjoin%
\definecolor{currentfill}{rgb}{0.280255,0.165693,0.476498}%
\pgfsetfillcolor{currentfill}%
\pgfsetfillopacity{0.800000}%
\pgfsetlinewidth{0.000000pt}%
\definecolor{currentstroke}{rgb}{0.000000,0.000000,0.000000}%
\pgfsetstrokecolor{currentstroke}%
\pgfsetdash{}{0pt}%
\pgfpathmoveto{\pgfqpoint{2.751134in}{2.221553in}}%
\pgfpathlineto{\pgfqpoint{2.764844in}{2.206247in}}%
\pgfpathlineto{\pgfqpoint{2.778547in}{2.191216in}}%
\pgfpathlineto{\pgfqpoint{2.792244in}{2.176459in}}%
\pgfpathlineto{\pgfqpoint{2.805936in}{2.161972in}}%
\pgfpathlineto{\pgfqpoint{2.814391in}{2.167342in}}%
\pgfpathlineto{\pgfqpoint{2.822835in}{2.172868in}}%
\pgfpathlineto{\pgfqpoint{2.831269in}{2.178546in}}%
\pgfpathlineto{\pgfqpoint{2.839692in}{2.184373in}}%
\pgfpathlineto{\pgfqpoint{2.826030in}{2.198510in}}%
\pgfpathlineto{\pgfqpoint{2.812363in}{2.212917in}}%
\pgfpathlineto{\pgfqpoint{2.798690in}{2.227596in}}%
\pgfpathlineto{\pgfqpoint{2.785011in}{2.242550in}}%
\pgfpathlineto{\pgfqpoint{2.776558in}{2.237062in}}%
\pgfpathlineto{\pgfqpoint{2.768095in}{2.231731in}}%
\pgfpathlineto{\pgfqpoint{2.759620in}{2.226560in}}%
\pgfpathlineto{\pgfqpoint{2.751134in}{2.221553in}}%
\pgfpathclose%
\pgfusepath{fill}%
\end{pgfscope}%
\begin{pgfscope}%
\pgfpathrectangle{\pgfqpoint{1.150000in}{0.150000in}}{\pgfqpoint{5.700000in}{5.700000in}}%
\pgfusepath{clip}%
\pgfsetbuttcap%
\pgfsetroundjoin%
\definecolor{currentfill}{rgb}{0.267968,0.223549,0.512008}%
\pgfsetfillcolor{currentfill}%
\pgfsetfillopacity{0.800000}%
\pgfsetlinewidth{0.000000pt}%
\definecolor{currentstroke}{rgb}{0.000000,0.000000,0.000000}%
\pgfsetstrokecolor{currentstroke}%
\pgfsetdash{}{0pt}%
\pgfpathmoveto{\pgfqpoint{2.641186in}{2.354173in}}%
\pgfpathlineto{\pgfqpoint{2.654958in}{2.336584in}}%
\pgfpathlineto{\pgfqpoint{2.668722in}{2.319290in}}%
\pgfpathlineto{\pgfqpoint{2.682476in}{2.302286in}}%
\pgfpathlineto{\pgfqpoint{2.696223in}{2.285571in}}%
\pgfpathlineto{\pgfqpoint{2.704742in}{2.290212in}}%
\pgfpathlineto{\pgfqpoint{2.713248in}{2.295026in}}%
\pgfpathlineto{\pgfqpoint{2.721743in}{2.300007in}}%
\pgfpathlineto{\pgfqpoint{2.730227in}{2.305155in}}%
\pgfpathlineto{\pgfqpoint{2.716512in}{2.321515in}}%
\pgfpathlineto{\pgfqpoint{2.702790in}{2.338163in}}%
\pgfpathlineto{\pgfqpoint{2.689060in}{2.355102in}}%
\pgfpathlineto{\pgfqpoint{2.675322in}{2.372334in}}%
\pgfpathlineto{\pgfqpoint{2.666806in}{2.367530in}}%
\pgfpathlineto{\pgfqpoint{2.658279in}{2.362899in}}%
\pgfpathlineto{\pgfqpoint{2.649739in}{2.358446in}}%
\pgfpathlineto{\pgfqpoint{2.641186in}{2.354173in}}%
\pgfpathclose%
\pgfusepath{fill}%
\end{pgfscope}%
\begin{pgfscope}%
\pgfpathrectangle{\pgfqpoint{1.150000in}{0.150000in}}{\pgfqpoint{5.700000in}{5.700000in}}%
\pgfusepath{clip}%
\pgfsetbuttcap%
\pgfsetroundjoin%
\definecolor{currentfill}{rgb}{0.277018,0.050344,0.375715}%
\pgfsetfillcolor{currentfill}%
\pgfsetfillopacity{0.800000}%
\pgfsetlinewidth{0.000000pt}%
\definecolor{currentstroke}{rgb}{0.000000,0.000000,0.000000}%
\pgfsetstrokecolor{currentstroke}%
\pgfsetdash{}{0pt}%
\pgfpathmoveto{\pgfqpoint{3.482068in}{1.934740in}}%
\pgfpathlineto{\pgfqpoint{3.495648in}{1.931115in}}%
\pgfpathlineto{\pgfqpoint{3.509232in}{1.927700in}}%
\pgfpathlineto{\pgfqpoint{3.522820in}{1.924495in}}%
\pgfpathlineto{\pgfqpoint{3.536413in}{1.921498in}}%
\pgfpathlineto{\pgfqpoint{3.544517in}{1.931518in}}%
\pgfpathlineto{\pgfqpoint{3.552616in}{1.941565in}}%
\pgfpathlineto{\pgfqpoint{3.560709in}{1.951636in}}%
\pgfpathlineto{\pgfqpoint{3.568796in}{1.961731in}}%
\pgfpathlineto{\pgfqpoint{3.555215in}{1.964515in}}%
\pgfpathlineto{\pgfqpoint{3.541639in}{1.967507in}}%
\pgfpathlineto{\pgfqpoint{3.528068in}{1.970709in}}%
\pgfpathlineto{\pgfqpoint{3.514501in}{1.974120in}}%
\pgfpathlineto{\pgfqpoint{3.506401in}{1.964226in}}%
\pgfpathlineto{\pgfqpoint{3.498296in}{1.954364in}}%
\pgfpathlineto{\pgfqpoint{3.490185in}{1.944535in}}%
\pgfpathlineto{\pgfqpoint{3.482068in}{1.934740in}}%
\pgfpathclose%
\pgfusepath{fill}%
\end{pgfscope}%
\begin{pgfscope}%
\pgfpathrectangle{\pgfqpoint{1.150000in}{0.150000in}}{\pgfqpoint{5.700000in}{5.700000in}}%
\pgfusepath{clip}%
\pgfsetbuttcap%
\pgfsetroundjoin%
\definecolor{currentfill}{rgb}{0.133743,0.548535,0.553541}%
\pgfsetfillcolor{currentfill}%
\pgfsetfillopacity{0.800000}%
\pgfsetlinewidth{0.000000pt}%
\definecolor{currentstroke}{rgb}{0.000000,0.000000,0.000000}%
\pgfsetstrokecolor{currentstroke}%
\pgfsetdash{}{0pt}%
\pgfpathmoveto{\pgfqpoint{5.535438in}{3.206521in}}%
\pgfpathlineto{\pgfqpoint{5.549842in}{3.215824in}}%
\pgfpathlineto{\pgfqpoint{5.564264in}{3.225302in}}%
\pgfpathlineto{\pgfqpoint{5.578705in}{3.234955in}}%
\pgfpathlineto{\pgfqpoint{5.593164in}{3.244784in}}%
\pgfpathlineto{\pgfqpoint{5.600429in}{3.248820in}}%
\pgfpathlineto{\pgfqpoint{5.607690in}{3.252936in}}%
\pgfpathlineto{\pgfqpoint{5.614947in}{3.257139in}}%
\pgfpathlineto{\pgfqpoint{5.622201in}{3.261435in}}%
\pgfpathlineto{\pgfqpoint{5.607770in}{3.252212in}}%
\pgfpathlineto{\pgfqpoint{5.593357in}{3.243164in}}%
\pgfpathlineto{\pgfqpoint{5.578962in}{3.234290in}}%
\pgfpathlineto{\pgfqpoint{5.564586in}{3.225590in}}%
\pgfpathlineto{\pgfqpoint{5.557304in}{3.220678in}}%
\pgfpathlineto{\pgfqpoint{5.550019in}{3.215867in}}%
\pgfpathlineto{\pgfqpoint{5.542730in}{3.211150in}}%
\pgfpathlineto{\pgfqpoint{5.535438in}{3.206521in}}%
\pgfpathclose%
\pgfusepath{fill}%
\end{pgfscope}%
\begin{pgfscope}%
\pgfpathrectangle{\pgfqpoint{1.150000in}{0.150000in}}{\pgfqpoint{5.700000in}{5.700000in}}%
\pgfusepath{clip}%
\pgfsetbuttcap%
\pgfsetroundjoin%
\definecolor{currentfill}{rgb}{0.255645,0.260703,0.528312}%
\pgfsetfillcolor{currentfill}%
\pgfsetfillopacity{0.800000}%
\pgfsetlinewidth{0.000000pt}%
\definecolor{currentstroke}{rgb}{0.000000,0.000000,0.000000}%
\pgfsetstrokecolor{currentstroke}%
\pgfsetdash{}{0pt}%
\pgfpathmoveto{\pgfqpoint{4.347387in}{2.383680in}}%
\pgfpathlineto{\pgfqpoint{4.361211in}{2.388650in}}%
\pgfpathlineto{\pgfqpoint{4.375048in}{2.393808in}}%
\pgfpathlineto{\pgfqpoint{4.388896in}{2.399153in}}%
\pgfpathlineto{\pgfqpoint{4.402757in}{2.404685in}}%
\pgfpathlineto{\pgfqpoint{4.410573in}{2.414276in}}%
\pgfpathlineto{\pgfqpoint{4.418384in}{2.423806in}}%
\pgfpathlineto{\pgfqpoint{4.426189in}{2.433279in}}%
\pgfpathlineto{\pgfqpoint{4.433988in}{2.442694in}}%
\pgfpathlineto{\pgfqpoint{4.420134in}{2.437269in}}%
\pgfpathlineto{\pgfqpoint{4.406293in}{2.432030in}}%
\pgfpathlineto{\pgfqpoint{4.392463in}{2.426978in}}%
\pgfpathlineto{\pgfqpoint{4.378646in}{2.422114in}}%
\pgfpathlineto{\pgfqpoint{4.370840in}{2.412580in}}%
\pgfpathlineto{\pgfqpoint{4.363028in}{2.402997in}}%
\pgfpathlineto{\pgfqpoint{4.355210in}{2.393365in}}%
\pgfpathlineto{\pgfqpoint{4.347387in}{2.383680in}}%
\pgfpathclose%
\pgfusepath{fill}%
\end{pgfscope}%
\begin{pgfscope}%
\pgfpathrectangle{\pgfqpoint{1.150000in}{0.150000in}}{\pgfqpoint{5.700000in}{5.700000in}}%
\pgfusepath{clip}%
\pgfsetbuttcap%
\pgfsetroundjoin%
\definecolor{currentfill}{rgb}{0.282290,0.145912,0.461510}%
\pgfsetfillcolor{currentfill}%
\pgfsetfillopacity{0.800000}%
\pgfsetlinewidth{0.000000pt}%
\definecolor{currentstroke}{rgb}{0.000000,0.000000,0.000000}%
\pgfsetstrokecolor{currentstroke}%
\pgfsetdash{}{0pt}%
\pgfpathmoveto{\pgfqpoint{2.805936in}{2.161972in}}%
\pgfpathlineto{\pgfqpoint{2.819621in}{2.147753in}}%
\pgfpathlineto{\pgfqpoint{2.833301in}{2.133802in}}%
\pgfpathlineto{\pgfqpoint{2.846976in}{2.120115in}}%
\pgfpathlineto{\pgfqpoint{2.860645in}{2.106691in}}%
\pgfpathlineto{\pgfqpoint{2.869071in}{2.112423in}}%
\pgfpathlineto{\pgfqpoint{2.877486in}{2.118302in}}%
\pgfpathlineto{\pgfqpoint{2.885892in}{2.124325in}}%
\pgfpathlineto{\pgfqpoint{2.894286in}{2.130489in}}%
\pgfpathlineto{\pgfqpoint{2.880645in}{2.143565in}}%
\pgfpathlineto{\pgfqpoint{2.866999in}{2.156903in}}%
\pgfpathlineto{\pgfqpoint{2.853348in}{2.170505in}}%
\pgfpathlineto{\pgfqpoint{2.839692in}{2.184373in}}%
\pgfpathlineto{\pgfqpoint{2.831269in}{2.178546in}}%
\pgfpathlineto{\pgfqpoint{2.822835in}{2.172868in}}%
\pgfpathlineto{\pgfqpoint{2.814391in}{2.167342in}}%
\pgfpathlineto{\pgfqpoint{2.805936in}{2.161972in}}%
\pgfpathclose%
\pgfusepath{fill}%
\end{pgfscope}%
\begin{pgfscope}%
\pgfpathrectangle{\pgfqpoint{1.150000in}{0.150000in}}{\pgfqpoint{5.700000in}{5.700000in}}%
\pgfusepath{clip}%
\pgfsetbuttcap%
\pgfsetroundjoin%
\definecolor{currentfill}{rgb}{0.258965,0.251537,0.524736}%
\pgfsetfillcolor{currentfill}%
\pgfsetfillopacity{0.800000}%
\pgfsetlinewidth{0.000000pt}%
\definecolor{currentstroke}{rgb}{0.000000,0.000000,0.000000}%
\pgfsetstrokecolor{currentstroke}%
\pgfsetdash{}{0pt}%
\pgfpathmoveto{\pgfqpoint{2.586005in}{2.427517in}}%
\pgfpathlineto{\pgfqpoint{2.599815in}{2.408727in}}%
\pgfpathlineto{\pgfqpoint{2.613615in}{2.390242in}}%
\pgfpathlineto{\pgfqpoint{2.627405in}{2.372058in}}%
\pgfpathlineto{\pgfqpoint{2.641186in}{2.354173in}}%
\pgfpathlineto{\pgfqpoint{2.649739in}{2.358446in}}%
\pgfpathlineto{\pgfqpoint{2.658279in}{2.362899in}}%
\pgfpathlineto{\pgfqpoint{2.666806in}{2.367530in}}%
\pgfpathlineto{\pgfqpoint{2.675322in}{2.372334in}}%
\pgfpathlineto{\pgfqpoint{2.661575in}{2.389861in}}%
\pgfpathlineto{\pgfqpoint{2.647819in}{2.407687in}}%
\pgfpathlineto{\pgfqpoint{2.634054in}{2.425814in}}%
\pgfpathlineto{\pgfqpoint{2.620279in}{2.444244in}}%
\pgfpathlineto{\pgfqpoint{2.611730in}{2.439786in}}%
\pgfpathlineto{\pgfqpoint{2.603167in}{2.435509in}}%
\pgfpathlineto{\pgfqpoint{2.594593in}{2.431419in}}%
\pgfpathlineto{\pgfqpoint{2.586005in}{2.427517in}}%
\pgfpathclose%
\pgfusepath{fill}%
\end{pgfscope}%
\begin{pgfscope}%
\pgfpathrectangle{\pgfqpoint{1.150000in}{0.150000in}}{\pgfqpoint{5.700000in}{5.700000in}}%
\pgfusepath{clip}%
\pgfsetbuttcap%
\pgfsetroundjoin%
\definecolor{currentfill}{rgb}{0.127568,0.566949,0.550556}%
\pgfsetfillcolor{currentfill}%
\pgfsetfillopacity{0.800000}%
\pgfsetlinewidth{0.000000pt}%
\definecolor{currentstroke}{rgb}{0.000000,0.000000,0.000000}%
\pgfsetstrokecolor{currentstroke}%
\pgfsetdash{}{0pt}%
\pgfpathmoveto{\pgfqpoint{5.622201in}{3.261435in}}%
\pgfpathlineto{\pgfqpoint{5.636650in}{3.270832in}}%
\pgfpathlineto{\pgfqpoint{5.651118in}{3.280404in}}%
\pgfpathlineto{\pgfqpoint{5.665604in}{3.290151in}}%
\pgfpathlineto{\pgfqpoint{5.680109in}{3.300072in}}%
\pgfpathlineto{\pgfqpoint{5.687330in}{3.303841in}}%
\pgfpathlineto{\pgfqpoint{5.694547in}{3.307709in}}%
\pgfpathlineto{\pgfqpoint{5.701762in}{3.311685in}}%
\pgfpathlineto{\pgfqpoint{5.708973in}{3.315773in}}%
\pgfpathlineto{\pgfqpoint{5.694499in}{3.306491in}}%
\pgfpathlineto{\pgfqpoint{5.680043in}{3.297383in}}%
\pgfpathlineto{\pgfqpoint{5.665605in}{3.288449in}}%
\pgfpathlineto{\pgfqpoint{5.651185in}{3.279688in}}%
\pgfpathlineto{\pgfqpoint{5.643943in}{3.274951in}}%
\pgfpathlineto{\pgfqpoint{5.636699in}{3.270334in}}%
\pgfpathlineto{\pgfqpoint{5.629451in}{3.265831in}}%
\pgfpathlineto{\pgfqpoint{5.622201in}{3.261435in}}%
\pgfpathclose%
\pgfusepath{fill}%
\end{pgfscope}%
\begin{pgfscope}%
\pgfpathrectangle{\pgfqpoint{1.150000in}{0.150000in}}{\pgfqpoint{5.700000in}{5.700000in}}%
\pgfusepath{clip}%
\pgfsetbuttcap%
\pgfsetroundjoin%
\definecolor{currentfill}{rgb}{0.190631,0.407061,0.556089}%
\pgfsetfillcolor{currentfill}%
\pgfsetfillopacity{0.800000}%
\pgfsetlinewidth{0.000000pt}%
\definecolor{currentstroke}{rgb}{0.000000,0.000000,0.000000}%
\pgfsetstrokecolor{currentstroke}%
\pgfsetdash{}{0pt}%
\pgfpathmoveto{\pgfqpoint{4.898247in}{2.778187in}}%
\pgfpathlineto{\pgfqpoint{4.912336in}{2.786143in}}%
\pgfpathlineto{\pgfqpoint{4.926439in}{2.794279in}}%
\pgfpathlineto{\pgfqpoint{4.940559in}{2.802596in}}%
\pgfpathlineto{\pgfqpoint{4.954694in}{2.811094in}}%
\pgfpathlineto{\pgfqpoint{4.962282in}{2.817930in}}%
\pgfpathlineto{\pgfqpoint{4.969863in}{2.824733in}}%
\pgfpathlineto{\pgfqpoint{4.977439in}{2.831507in}}%
\pgfpathlineto{\pgfqpoint{4.985008in}{2.838256in}}%
\pgfpathlineto{\pgfqpoint{4.970886in}{2.830097in}}%
\pgfpathlineto{\pgfqpoint{4.956781in}{2.822118in}}%
\pgfpathlineto{\pgfqpoint{4.942691in}{2.814319in}}%
\pgfpathlineto{\pgfqpoint{4.928616in}{2.806700in}}%
\pgfpathlineto{\pgfqpoint{4.921033in}{2.799602in}}%
\pgfpathlineto{\pgfqpoint{4.913444in}{2.792486in}}%
\pgfpathlineto{\pgfqpoint{4.905849in}{2.785349in}}%
\pgfpathlineto{\pgfqpoint{4.898247in}{2.778187in}}%
\pgfpathclose%
\pgfusepath{fill}%
\end{pgfscope}%
\begin{pgfscope}%
\pgfpathrectangle{\pgfqpoint{1.150000in}{0.150000in}}{\pgfqpoint{5.700000in}{5.700000in}}%
\pgfusepath{clip}%
\pgfsetbuttcap%
\pgfsetroundjoin%
\definecolor{currentfill}{rgb}{0.274952,0.037752,0.364543}%
\pgfsetfillcolor{currentfill}%
\pgfsetfillopacity{0.800000}%
\pgfsetlinewidth{0.000000pt}%
\definecolor{currentstroke}{rgb}{0.000000,0.000000,0.000000}%
\pgfsetstrokecolor{currentstroke}%
\pgfsetdash{}{0pt}%
\pgfpathmoveto{\pgfqpoint{3.253825in}{1.924381in}}%
\pgfpathlineto{\pgfqpoint{3.267404in}{1.917634in}}%
\pgfpathlineto{\pgfqpoint{3.280986in}{1.911111in}}%
\pgfpathlineto{\pgfqpoint{3.294568in}{1.904809in}}%
\pgfpathlineto{\pgfqpoint{3.308153in}{1.898727in}}%
\pgfpathlineto{\pgfqpoint{3.316351in}{1.907592in}}%
\pgfpathlineto{\pgfqpoint{3.324543in}{1.916525in}}%
\pgfpathlineto{\pgfqpoint{3.332728in}{1.925523in}}%
\pgfpathlineto{\pgfqpoint{3.340906in}{1.934585in}}%
\pgfpathlineto{\pgfqpoint{3.327338in}{1.940390in}}%
\pgfpathlineto{\pgfqpoint{3.313773in}{1.946414in}}%
\pgfpathlineto{\pgfqpoint{3.300209in}{1.952661in}}%
\pgfpathlineto{\pgfqpoint{3.286647in}{1.959129in}}%
\pgfpathlineto{\pgfqpoint{3.278452in}{1.950333in}}%
\pgfpathlineto{\pgfqpoint{3.270250in}{1.941608in}}%
\pgfpathlineto{\pgfqpoint{3.262041in}{1.932957in}}%
\pgfpathlineto{\pgfqpoint{3.253825in}{1.924381in}}%
\pgfpathclose%
\pgfusepath{fill}%
\end{pgfscope}%
\begin{pgfscope}%
\pgfpathrectangle{\pgfqpoint{1.150000in}{0.150000in}}{\pgfqpoint{5.700000in}{5.700000in}}%
\pgfusepath{clip}%
\pgfsetbuttcap%
\pgfsetroundjoin%
\definecolor{currentfill}{rgb}{0.283229,0.120777,0.440584}%
\pgfsetfillcolor{currentfill}%
\pgfsetfillopacity{0.800000}%
\pgfsetlinewidth{0.000000pt}%
\definecolor{currentstroke}{rgb}{0.000000,0.000000,0.000000}%
\pgfsetstrokecolor{currentstroke}%
\pgfsetdash{}{0pt}%
\pgfpathmoveto{\pgfqpoint{2.860645in}{2.106691in}}%
\pgfpathlineto{\pgfqpoint{2.874310in}{2.093527in}}%
\pgfpathlineto{\pgfqpoint{2.887971in}{2.080623in}}%
\pgfpathlineto{\pgfqpoint{2.901627in}{2.067976in}}%
\pgfpathlineto{\pgfqpoint{2.915279in}{2.055584in}}%
\pgfpathlineto{\pgfqpoint{2.923677in}{2.061676in}}%
\pgfpathlineto{\pgfqpoint{2.932064in}{2.067906in}}%
\pgfpathlineto{\pgfqpoint{2.940442in}{2.074273in}}%
\pgfpathlineto{\pgfqpoint{2.948810in}{2.080772in}}%
\pgfpathlineto{\pgfqpoint{2.935185in}{2.092817in}}%
\pgfpathlineto{\pgfqpoint{2.921556in}{2.105117in}}%
\pgfpathlineto{\pgfqpoint{2.907923in}{2.117674in}}%
\pgfpathlineto{\pgfqpoint{2.894286in}{2.130489in}}%
\pgfpathlineto{\pgfqpoint{2.885892in}{2.124325in}}%
\pgfpathlineto{\pgfqpoint{2.877486in}{2.118302in}}%
\pgfpathlineto{\pgfqpoint{2.869071in}{2.112423in}}%
\pgfpathlineto{\pgfqpoint{2.860645in}{2.106691in}}%
\pgfpathclose%
\pgfusepath{fill}%
\end{pgfscope}%
\begin{pgfscope}%
\pgfpathrectangle{\pgfqpoint{1.150000in}{0.150000in}}{\pgfqpoint{5.700000in}{5.700000in}}%
\pgfusepath{clip}%
\pgfsetbuttcap%
\pgfsetroundjoin%
\definecolor{currentfill}{rgb}{0.246811,0.283237,0.535941}%
\pgfsetfillcolor{currentfill}%
\pgfsetfillopacity{0.800000}%
\pgfsetlinewidth{0.000000pt}%
\definecolor{currentstroke}{rgb}{0.000000,0.000000,0.000000}%
\pgfsetstrokecolor{currentstroke}%
\pgfsetdash{}{0pt}%
\pgfpathmoveto{\pgfqpoint{4.433988in}{2.442694in}}%
\pgfpathlineto{\pgfqpoint{4.447854in}{2.448307in}}%
\pgfpathlineto{\pgfqpoint{4.461733in}{2.454105in}}%
\pgfpathlineto{\pgfqpoint{4.475624in}{2.460090in}}%
\pgfpathlineto{\pgfqpoint{4.489529in}{2.466260in}}%
\pgfpathlineto{\pgfqpoint{4.497315in}{2.475494in}}%
\pgfpathlineto{\pgfqpoint{4.505096in}{2.484666in}}%
\pgfpathlineto{\pgfqpoint{4.512871in}{2.493779in}}%
\pgfpathlineto{\pgfqpoint{4.520639in}{2.502835in}}%
\pgfpathlineto{\pgfqpoint{4.506742in}{2.496804in}}%
\pgfpathlineto{\pgfqpoint{4.492858in}{2.490958in}}%
\pgfpathlineto{\pgfqpoint{4.478987in}{2.485299in}}%
\pgfpathlineto{\pgfqpoint{4.465128in}{2.479825in}}%
\pgfpathlineto{\pgfqpoint{4.457352in}{2.470619in}}%
\pgfpathlineto{\pgfqpoint{4.449569in}{2.461363in}}%
\pgfpathlineto{\pgfqpoint{4.441781in}{2.452055in}}%
\pgfpathlineto{\pgfqpoint{4.433988in}{2.442694in}}%
\pgfpathclose%
\pgfusepath{fill}%
\end{pgfscope}%
\begin{pgfscope}%
\pgfpathrectangle{\pgfqpoint{1.150000in}{0.150000in}}{\pgfqpoint{5.700000in}{5.700000in}}%
\pgfusepath{clip}%
\pgfsetbuttcap%
\pgfsetroundjoin%
\definecolor{currentfill}{rgb}{0.246811,0.283237,0.535941}%
\pgfsetfillcolor{currentfill}%
\pgfsetfillopacity{0.800000}%
\pgfsetlinewidth{0.000000pt}%
\definecolor{currentstroke}{rgb}{0.000000,0.000000,0.000000}%
\pgfsetstrokecolor{currentstroke}%
\pgfsetdash{}{0pt}%
\pgfpathmoveto{\pgfqpoint{2.530659in}{2.505772in}}%
\pgfpathlineto{\pgfqpoint{2.544512in}{2.485738in}}%
\pgfpathlineto{\pgfqpoint{2.558354in}{2.466019in}}%
\pgfpathlineto{\pgfqpoint{2.572184in}{2.446613in}}%
\pgfpathlineto{\pgfqpoint{2.586005in}{2.427517in}}%
\pgfpathlineto{\pgfqpoint{2.594593in}{2.431419in}}%
\pgfpathlineto{\pgfqpoint{2.603167in}{2.435509in}}%
\pgfpathlineto{\pgfqpoint{2.611730in}{2.439786in}}%
\pgfpathlineto{\pgfqpoint{2.620279in}{2.444244in}}%
\pgfpathlineto{\pgfqpoint{2.606494in}{2.462981in}}%
\pgfpathlineto{\pgfqpoint{2.592700in}{2.482026in}}%
\pgfpathlineto{\pgfqpoint{2.578894in}{2.501383in}}%
\pgfpathlineto{\pgfqpoint{2.565078in}{2.521055in}}%
\pgfpathlineto{\pgfqpoint{2.556493in}{2.516945in}}%
\pgfpathlineto{\pgfqpoint{2.547896in}{2.513025in}}%
\pgfpathlineto{\pgfqpoint{2.539284in}{2.509300in}}%
\pgfpathlineto{\pgfqpoint{2.530659in}{2.505772in}}%
\pgfpathclose%
\pgfusepath{fill}%
\end{pgfscope}%
\begin{pgfscope}%
\pgfpathrectangle{\pgfqpoint{1.150000in}{0.150000in}}{\pgfqpoint{5.700000in}{5.700000in}}%
\pgfusepath{clip}%
\pgfsetbuttcap%
\pgfsetroundjoin%
\definecolor{currentfill}{rgb}{0.122606,0.585371,0.546557}%
\pgfsetfillcolor{currentfill}%
\pgfsetfillopacity{0.800000}%
\pgfsetlinewidth{0.000000pt}%
\definecolor{currentstroke}{rgb}{0.000000,0.000000,0.000000}%
\pgfsetstrokecolor{currentstroke}%
\pgfsetdash{}{0pt}%
\pgfpathmoveto{\pgfqpoint{5.708973in}{3.315773in}}%
\pgfpathlineto{\pgfqpoint{5.723467in}{3.325228in}}%
\pgfpathlineto{\pgfqpoint{5.737979in}{3.334858in}}%
\pgfpathlineto{\pgfqpoint{5.752510in}{3.344661in}}%
\pgfpathlineto{\pgfqpoint{5.767060in}{3.354638in}}%
\pgfpathlineto{\pgfqpoint{5.774237in}{3.358187in}}%
\pgfpathlineto{\pgfqpoint{5.781412in}{3.361856in}}%
\pgfpathlineto{\pgfqpoint{5.788584in}{3.365652in}}%
\pgfpathlineto{\pgfqpoint{5.795755in}{3.369584in}}%
\pgfpathlineto{\pgfqpoint{5.781237in}{3.360279in}}%
\pgfpathlineto{\pgfqpoint{5.766739in}{3.351148in}}%
\pgfpathlineto{\pgfqpoint{5.752259in}{3.342189in}}%
\pgfpathlineto{\pgfqpoint{5.737798in}{3.333403in}}%
\pgfpathlineto{\pgfqpoint{5.730595in}{3.328790in}}%
\pgfpathlineto{\pgfqpoint{5.723390in}{3.324318in}}%
\pgfpathlineto{\pgfqpoint{5.716183in}{3.319982in}}%
\pgfpathlineto{\pgfqpoint{5.708973in}{3.315773in}}%
\pgfpathclose%
\pgfusepath{fill}%
\end{pgfscope}%
\begin{pgfscope}%
\pgfpathrectangle{\pgfqpoint{1.150000in}{0.150000in}}{\pgfqpoint{5.700000in}{5.700000in}}%
\pgfusepath{clip}%
\pgfsetbuttcap%
\pgfsetroundjoin%
\definecolor{currentfill}{rgb}{0.277941,0.056324,0.381191}%
\pgfsetfillcolor{currentfill}%
\pgfsetfillopacity{0.800000}%
\pgfsetlinewidth{0.000000pt}%
\definecolor{currentstroke}{rgb}{0.000000,0.000000,0.000000}%
\pgfsetstrokecolor{currentstroke}%
\pgfsetdash{}{0pt}%
\pgfpathmoveto{\pgfqpoint{3.112109in}{1.955496in}}%
\pgfpathlineto{\pgfqpoint{3.125706in}{1.946614in}}%
\pgfpathlineto{\pgfqpoint{3.139304in}{1.937965in}}%
\pgfpathlineto{\pgfqpoint{3.152900in}{1.929548in}}%
\pgfpathlineto{\pgfqpoint{3.166497in}{1.921361in}}%
\pgfpathlineto{\pgfqpoint{3.174763in}{1.929285in}}%
\pgfpathlineto{\pgfqpoint{3.183021in}{1.937304in}}%
\pgfpathlineto{\pgfqpoint{3.191271in}{1.945415in}}%
\pgfpathlineto{\pgfqpoint{3.199514in}{1.953615in}}%
\pgfpathlineto{\pgfqpoint{3.185938in}{1.961492in}}%
\pgfpathlineto{\pgfqpoint{3.172362in}{1.969599in}}%
\pgfpathlineto{\pgfqpoint{3.158785in}{1.977938in}}%
\pgfpathlineto{\pgfqpoint{3.145209in}{1.986509in}}%
\pgfpathlineto{\pgfqpoint{3.136946in}{1.978607in}}%
\pgfpathlineto{\pgfqpoint{3.128675in}{1.970803in}}%
\pgfpathlineto{\pgfqpoint{3.120396in}{1.963098in}}%
\pgfpathlineto{\pgfqpoint{3.112109in}{1.955496in}}%
\pgfpathclose%
\pgfusepath{fill}%
\end{pgfscope}%
\begin{pgfscope}%
\pgfpathrectangle{\pgfqpoint{1.150000in}{0.150000in}}{\pgfqpoint{5.700000in}{5.700000in}}%
\pgfusepath{clip}%
\pgfsetbuttcap%
\pgfsetroundjoin%
\definecolor{currentfill}{rgb}{0.180629,0.429975,0.557282}%
\pgfsetfillcolor{currentfill}%
\pgfsetfillopacity{0.800000}%
\pgfsetlinewidth{0.000000pt}%
\definecolor{currentstroke}{rgb}{0.000000,0.000000,0.000000}%
\pgfsetstrokecolor{currentstroke}%
\pgfsetdash{}{0pt}%
\pgfpathmoveto{\pgfqpoint{4.985008in}{2.838256in}}%
\pgfpathlineto{\pgfqpoint{4.999145in}{2.846596in}}%
\pgfpathlineto{\pgfqpoint{5.013297in}{2.855117in}}%
\pgfpathlineto{\pgfqpoint{5.027466in}{2.863817in}}%
\pgfpathlineto{\pgfqpoint{5.041652in}{2.872698in}}%
\pgfpathlineto{\pgfqpoint{5.049200in}{2.879066in}}%
\pgfpathlineto{\pgfqpoint{5.056742in}{2.885409in}}%
\pgfpathlineto{\pgfqpoint{5.064277in}{2.891732in}}%
\pgfpathlineto{\pgfqpoint{5.071807in}{2.898039in}}%
\pgfpathlineto{\pgfqpoint{5.057637in}{2.889531in}}%
\pgfpathlineto{\pgfqpoint{5.043483in}{2.881202in}}%
\pgfpathlineto{\pgfqpoint{5.029345in}{2.873053in}}%
\pgfpathlineto{\pgfqpoint{5.015223in}{2.865084in}}%
\pgfpathlineto{\pgfqpoint{5.007678in}{2.858394in}}%
\pgfpathlineto{\pgfqpoint{5.000127in}{2.851695in}}%
\pgfpathlineto{\pgfqpoint{4.992571in}{2.844984in}}%
\pgfpathlineto{\pgfqpoint{4.985008in}{2.838256in}}%
\pgfpathclose%
\pgfusepath{fill}%
\end{pgfscope}%
\begin{pgfscope}%
\pgfpathrectangle{\pgfqpoint{1.150000in}{0.150000in}}{\pgfqpoint{5.700000in}{5.700000in}}%
\pgfusepath{clip}%
\pgfsetbuttcap%
\pgfsetroundjoin%
\definecolor{currentfill}{rgb}{0.274952,0.037752,0.364543}%
\pgfsetfillcolor{currentfill}%
\pgfsetfillopacity{0.800000}%
\pgfsetlinewidth{0.000000pt}%
\definecolor{currentstroke}{rgb}{0.000000,0.000000,0.000000}%
\pgfsetstrokecolor{currentstroke}%
\pgfsetdash{}{0pt}%
\pgfpathmoveto{\pgfqpoint{3.395199in}{1.913546in}}%
\pgfpathlineto{\pgfqpoint{3.408779in}{1.908827in}}%
\pgfpathlineto{\pgfqpoint{3.422363in}{1.904321in}}%
\pgfpathlineto{\pgfqpoint{3.435949in}{1.900028in}}%
\pgfpathlineto{\pgfqpoint{3.449539in}{1.895947in}}%
\pgfpathlineto{\pgfqpoint{3.457680in}{1.905583in}}%
\pgfpathlineto{\pgfqpoint{3.465816in}{1.915262in}}%
\pgfpathlineto{\pgfqpoint{3.473945in}{1.924982in}}%
\pgfpathlineto{\pgfqpoint{3.482068in}{1.934740in}}%
\pgfpathlineto{\pgfqpoint{3.468492in}{1.938576in}}%
\pgfpathlineto{\pgfqpoint{3.454919in}{1.942624in}}%
\pgfpathlineto{\pgfqpoint{3.441350in}{1.946884in}}%
\pgfpathlineto{\pgfqpoint{3.427785in}{1.951359in}}%
\pgfpathlineto{\pgfqpoint{3.419648in}{1.941834in}}%
\pgfpathlineto{\pgfqpoint{3.411505in}{1.932356in}}%
\pgfpathlineto{\pgfqpoint{3.403355in}{1.922926in}}%
\pgfpathlineto{\pgfqpoint{3.395199in}{1.913546in}}%
\pgfpathclose%
\pgfusepath{fill}%
\end{pgfscope}%
\begin{pgfscope}%
\pgfpathrectangle{\pgfqpoint{1.150000in}{0.150000in}}{\pgfqpoint{5.700000in}{5.700000in}}%
\pgfusepath{clip}%
\pgfsetbuttcap%
\pgfsetroundjoin%
\definecolor{currentfill}{rgb}{0.119738,0.603785,0.541400}%
\pgfsetfillcolor{currentfill}%
\pgfsetfillopacity{0.800000}%
\pgfsetlinewidth{0.000000pt}%
\definecolor{currentstroke}{rgb}{0.000000,0.000000,0.000000}%
\pgfsetstrokecolor{currentstroke}%
\pgfsetdash{}{0pt}%
\pgfpathmoveto{\pgfqpoint{5.795755in}{3.369584in}}%
\pgfpathlineto{\pgfqpoint{5.810291in}{3.379061in}}%
\pgfpathlineto{\pgfqpoint{5.824846in}{3.388712in}}%
\pgfpathlineto{\pgfqpoint{5.839420in}{3.398536in}}%
\pgfpathlineto{\pgfqpoint{5.854014in}{3.408534in}}%
\pgfpathlineto{\pgfqpoint{5.861149in}{3.411914in}}%
\pgfpathlineto{\pgfqpoint{5.868281in}{3.415437in}}%
\pgfpathlineto{\pgfqpoint{5.875413in}{3.419110in}}%
\pgfpathlineto{\pgfqpoint{5.882544in}{3.422942in}}%
\pgfpathlineto{\pgfqpoint{5.867986in}{3.413650in}}%
\pgfpathlineto{\pgfqpoint{5.853446in}{3.404531in}}%
\pgfpathlineto{\pgfqpoint{5.838926in}{3.395584in}}%
\pgfpathlineto{\pgfqpoint{5.824424in}{3.386809in}}%
\pgfpathlineto{\pgfqpoint{5.817258in}{3.382263in}}%
\pgfpathlineto{\pgfqpoint{5.810091in}{3.377882in}}%
\pgfpathlineto{\pgfqpoint{5.802924in}{3.373658in}}%
\pgfpathlineto{\pgfqpoint{5.795755in}{3.369584in}}%
\pgfpathclose%
\pgfusepath{fill}%
\end{pgfscope}%
\begin{pgfscope}%
\pgfpathrectangle{\pgfqpoint{1.150000in}{0.150000in}}{\pgfqpoint{5.700000in}{5.700000in}}%
\pgfusepath{clip}%
\pgfsetbuttcap%
\pgfsetroundjoin%
\definecolor{currentfill}{rgb}{0.282910,0.105393,0.426902}%
\pgfsetfillcolor{currentfill}%
\pgfsetfillopacity{0.800000}%
\pgfsetlinewidth{0.000000pt}%
\definecolor{currentstroke}{rgb}{0.000000,0.000000,0.000000}%
\pgfsetstrokecolor{currentstroke}%
\pgfsetdash{}{0pt}%
\pgfpathmoveto{\pgfqpoint{3.796447in}{2.029700in}}%
\pgfpathlineto{\pgfqpoint{3.810087in}{2.029888in}}%
\pgfpathlineto{\pgfqpoint{3.823733in}{2.030274in}}%
\pgfpathlineto{\pgfqpoint{3.837387in}{2.030858in}}%
\pgfpathlineto{\pgfqpoint{3.851049in}{2.031639in}}%
\pgfpathlineto{\pgfqpoint{3.859051in}{2.042402in}}%
\pgfpathlineto{\pgfqpoint{3.867047in}{2.053145in}}%
\pgfpathlineto{\pgfqpoint{3.875038in}{2.063865in}}%
\pgfpathlineto{\pgfqpoint{3.883024in}{2.074564in}}%
\pgfpathlineto{\pgfqpoint{3.869370in}{2.073664in}}%
\pgfpathlineto{\pgfqpoint{3.855724in}{2.072962in}}%
\pgfpathlineto{\pgfqpoint{3.842085in}{2.072457in}}%
\pgfpathlineto{\pgfqpoint{3.828454in}{2.072151in}}%
\pgfpathlineto{\pgfqpoint{3.820460in}{2.061559in}}%
\pgfpathlineto{\pgfqpoint{3.812461in}{2.050953in}}%
\pgfpathlineto{\pgfqpoint{3.804457in}{2.040333in}}%
\pgfpathlineto{\pgfqpoint{3.796447in}{2.029700in}}%
\pgfpathclose%
\pgfusepath{fill}%
\end{pgfscope}%
\begin{pgfscope}%
\pgfpathrectangle{\pgfqpoint{1.150000in}{0.150000in}}{\pgfqpoint{5.700000in}{5.700000in}}%
\pgfusepath{clip}%
\pgfsetbuttcap%
\pgfsetroundjoin%
\definecolor{currentfill}{rgb}{0.283187,0.125848,0.444960}%
\pgfsetfillcolor{currentfill}%
\pgfsetfillopacity{0.800000}%
\pgfsetlinewidth{0.000000pt}%
\definecolor{currentstroke}{rgb}{0.000000,0.000000,0.000000}%
\pgfsetstrokecolor{currentstroke}%
\pgfsetdash{}{0pt}%
\pgfpathmoveto{\pgfqpoint{3.883024in}{2.074564in}}%
\pgfpathlineto{\pgfqpoint{3.896687in}{2.075660in}}%
\pgfpathlineto{\pgfqpoint{3.910357in}{2.076952in}}%
\pgfpathlineto{\pgfqpoint{3.924036in}{2.078440in}}%
\pgfpathlineto{\pgfqpoint{3.937723in}{2.080123in}}%
\pgfpathlineto{\pgfqpoint{3.945697in}{2.090897in}}%
\pgfpathlineto{\pgfqpoint{3.953666in}{2.101639in}}%
\pgfpathlineto{\pgfqpoint{3.961630in}{2.112350in}}%
\pgfpathlineto{\pgfqpoint{3.969589in}{2.123029in}}%
\pgfpathlineto{\pgfqpoint{3.955908in}{2.121259in}}%
\pgfpathlineto{\pgfqpoint{3.942237in}{2.119685in}}%
\pgfpathlineto{\pgfqpoint{3.928573in}{2.118306in}}%
\pgfpathlineto{\pgfqpoint{3.914919in}{2.117123in}}%
\pgfpathlineto{\pgfqpoint{3.906953in}{2.106519in}}%
\pgfpathlineto{\pgfqpoint{3.898981in}{2.095891in}}%
\pgfpathlineto{\pgfqpoint{3.891005in}{2.085239in}}%
\pgfpathlineto{\pgfqpoint{3.883024in}{2.074564in}}%
\pgfpathclose%
\pgfusepath{fill}%
\end{pgfscope}%
\begin{pgfscope}%
\pgfpathrectangle{\pgfqpoint{1.150000in}{0.150000in}}{\pgfqpoint{5.700000in}{5.700000in}}%
\pgfusepath{clip}%
\pgfsetbuttcap%
\pgfsetroundjoin%
\definecolor{currentfill}{rgb}{0.235526,0.309527,0.542944}%
\pgfsetfillcolor{currentfill}%
\pgfsetfillopacity{0.800000}%
\pgfsetlinewidth{0.000000pt}%
\definecolor{currentstroke}{rgb}{0.000000,0.000000,0.000000}%
\pgfsetstrokecolor{currentstroke}%
\pgfsetdash{}{0pt}%
\pgfpathmoveto{\pgfqpoint{4.520639in}{2.502835in}}%
\pgfpathlineto{\pgfqpoint{4.534550in}{2.509051in}}%
\pgfpathlineto{\pgfqpoint{4.548473in}{2.515453in}}%
\pgfpathlineto{\pgfqpoint{4.562410in}{2.522040in}}%
\pgfpathlineto{\pgfqpoint{4.576360in}{2.528811in}}%
\pgfpathlineto{\pgfqpoint{4.584115in}{2.537651in}}%
\pgfpathlineto{\pgfqpoint{4.591864in}{2.546430in}}%
\pgfpathlineto{\pgfqpoint{4.599607in}{2.555150in}}%
\pgfpathlineto{\pgfqpoint{4.607344in}{2.563813in}}%
\pgfpathlineto{\pgfqpoint{4.593402in}{2.557214in}}%
\pgfpathlineto{\pgfqpoint{4.579474in}{2.550800in}}%
\pgfpathlineto{\pgfqpoint{4.565559in}{2.544570in}}%
\pgfpathlineto{\pgfqpoint{4.551657in}{2.538525in}}%
\pgfpathlineto{\pgfqpoint{4.543911in}{2.529678in}}%
\pgfpathlineto{\pgfqpoint{4.536160in}{2.520782in}}%
\pgfpathlineto{\pgfqpoint{4.528402in}{2.511835in}}%
\pgfpathlineto{\pgfqpoint{4.520639in}{2.502835in}}%
\pgfpathclose%
\pgfusepath{fill}%
\end{pgfscope}%
\begin{pgfscope}%
\pgfpathrectangle{\pgfqpoint{1.150000in}{0.150000in}}{\pgfqpoint{5.700000in}{5.700000in}}%
\pgfusepath{clip}%
\pgfsetbuttcap%
\pgfsetroundjoin%
\definecolor{currentfill}{rgb}{0.282656,0.100196,0.422160}%
\pgfsetfillcolor{currentfill}%
\pgfsetfillopacity{0.800000}%
\pgfsetlinewidth{0.000000pt}%
\definecolor{currentstroke}{rgb}{0.000000,0.000000,0.000000}%
\pgfsetstrokecolor{currentstroke}%
\pgfsetdash{}{0pt}%
\pgfpathmoveto{\pgfqpoint{2.915279in}{2.055584in}}%
\pgfpathlineto{\pgfqpoint{2.928927in}{2.043445in}}%
\pgfpathlineto{\pgfqpoint{2.942572in}{2.031558in}}%
\pgfpathlineto{\pgfqpoint{2.956214in}{2.019922in}}%
\pgfpathlineto{\pgfqpoint{2.969852in}{2.008533in}}%
\pgfpathlineto{\pgfqpoint{2.978223in}{2.014983in}}%
\pgfpathlineto{\pgfqpoint{2.986584in}{2.021564in}}%
\pgfpathlineto{\pgfqpoint{2.994936in}{2.028272in}}%
\pgfpathlineto{\pgfqpoint{3.003279in}{2.035105in}}%
\pgfpathlineto{\pgfqpoint{2.989666in}{2.046148in}}%
\pgfpathlineto{\pgfqpoint{2.976051in}{2.057439in}}%
\pgfpathlineto{\pgfqpoint{2.962432in}{2.068980in}}%
\pgfpathlineto{\pgfqpoint{2.948810in}{2.080772in}}%
\pgfpathlineto{\pgfqpoint{2.940442in}{2.074273in}}%
\pgfpathlineto{\pgfqpoint{2.932064in}{2.067906in}}%
\pgfpathlineto{\pgfqpoint{2.923677in}{2.061676in}}%
\pgfpathlineto{\pgfqpoint{2.915279in}{2.055584in}}%
\pgfpathclose%
\pgfusepath{fill}%
\end{pgfscope}%
\begin{pgfscope}%
\pgfpathrectangle{\pgfqpoint{1.150000in}{0.150000in}}{\pgfqpoint{5.700000in}{5.700000in}}%
\pgfusepath{clip}%
\pgfsetbuttcap%
\pgfsetroundjoin%
\definecolor{currentfill}{rgb}{0.120081,0.622161,0.534946}%
\pgfsetfillcolor{currentfill}%
\pgfsetfillopacity{0.800000}%
\pgfsetlinewidth{0.000000pt}%
\definecolor{currentstroke}{rgb}{0.000000,0.000000,0.000000}%
\pgfsetstrokecolor{currentstroke}%
\pgfsetdash{}{0pt}%
\pgfpathmoveto{\pgfqpoint{5.882544in}{3.422942in}}%
\pgfpathlineto{\pgfqpoint{5.897122in}{3.432405in}}%
\pgfpathlineto{\pgfqpoint{5.911719in}{3.442042in}}%
\pgfpathlineto{\pgfqpoint{5.926335in}{3.451851in}}%
\pgfpathlineto{\pgfqpoint{5.940971in}{3.461832in}}%
\pgfpathlineto{\pgfqpoint{5.948064in}{3.465103in}}%
\pgfpathlineto{\pgfqpoint{5.955157in}{3.468539in}}%
\pgfpathlineto{\pgfqpoint{5.962250in}{3.472151in}}%
\pgfpathlineto{\pgfqpoint{5.969343in}{3.475944in}}%
\pgfpathlineto{\pgfqpoint{5.954745in}{3.466702in}}%
\pgfpathlineto{\pgfqpoint{5.940166in}{3.457631in}}%
\pgfpathlineto{\pgfqpoint{5.925606in}{3.448731in}}%
\pgfpathlineto{\pgfqpoint{5.911065in}{3.440003in}}%
\pgfpathlineto{\pgfqpoint{5.903934in}{3.435461in}}%
\pgfpathlineto{\pgfqpoint{5.896804in}{3.431109in}}%
\pgfpathlineto{\pgfqpoint{5.889674in}{3.426939in}}%
\pgfpathlineto{\pgfqpoint{5.882544in}{3.422942in}}%
\pgfpathclose%
\pgfusepath{fill}%
\end{pgfscope}%
\begin{pgfscope}%
\pgfpathrectangle{\pgfqpoint{1.150000in}{0.150000in}}{\pgfqpoint{5.700000in}{5.700000in}}%
\pgfusepath{clip}%
\pgfsetbuttcap%
\pgfsetroundjoin%
\definecolor{currentfill}{rgb}{0.231674,0.318106,0.544834}%
\pgfsetfillcolor{currentfill}%
\pgfsetfillopacity{0.800000}%
\pgfsetlinewidth{0.000000pt}%
\definecolor{currentstroke}{rgb}{0.000000,0.000000,0.000000}%
\pgfsetstrokecolor{currentstroke}%
\pgfsetdash{}{0pt}%
\pgfpathmoveto{\pgfqpoint{2.475130in}{2.589122in}}%
\pgfpathlineto{\pgfqpoint{2.489030in}{2.567796in}}%
\pgfpathlineto{\pgfqpoint{2.502919in}{2.546798in}}%
\pgfpathlineto{\pgfqpoint{2.516795in}{2.526124in}}%
\pgfpathlineto{\pgfqpoint{2.530659in}{2.505772in}}%
\pgfpathlineto{\pgfqpoint{2.539284in}{2.509300in}}%
\pgfpathlineto{\pgfqpoint{2.547896in}{2.513025in}}%
\pgfpathlineto{\pgfqpoint{2.556493in}{2.516945in}}%
\pgfpathlineto{\pgfqpoint{2.565078in}{2.521055in}}%
\pgfpathlineto{\pgfqpoint{2.551251in}{2.541044in}}%
\pgfpathlineto{\pgfqpoint{2.537412in}{2.561354in}}%
\pgfpathlineto{\pgfqpoint{2.523562in}{2.581987in}}%
\pgfpathlineto{\pgfqpoint{2.509700in}{2.602948in}}%
\pgfpathlineto{\pgfqpoint{2.501078in}{2.599189in}}%
\pgfpathlineto{\pgfqpoint{2.492443in}{2.595630in}}%
\pgfpathlineto{\pgfqpoint{2.483793in}{2.592273in}}%
\pgfpathlineto{\pgfqpoint{2.475130in}{2.589122in}}%
\pgfpathclose%
\pgfusepath{fill}%
\end{pgfscope}%
\begin{pgfscope}%
\pgfpathrectangle{\pgfqpoint{1.150000in}{0.150000in}}{\pgfqpoint{5.700000in}{5.700000in}}%
\pgfusepath{clip}%
\pgfsetbuttcap%
\pgfsetroundjoin%
\definecolor{currentfill}{rgb}{0.281446,0.084320,0.407414}%
\pgfsetfillcolor{currentfill}%
\pgfsetfillopacity{0.800000}%
\pgfsetlinewidth{0.000000pt}%
\definecolor{currentstroke}{rgb}{0.000000,0.000000,0.000000}%
\pgfsetstrokecolor{currentstroke}%
\pgfsetdash{}{0pt}%
\pgfpathmoveto{\pgfqpoint{3.709837in}{1.988904in}}%
\pgfpathlineto{\pgfqpoint{3.723457in}{1.988142in}}%
\pgfpathlineto{\pgfqpoint{3.737084in}{1.987581in}}%
\pgfpathlineto{\pgfqpoint{3.750718in}{1.987220in}}%
\pgfpathlineto{\pgfqpoint{3.764359in}{1.987059in}}%
\pgfpathlineto{\pgfqpoint{3.772388in}{1.997733in}}%
\pgfpathlineto{\pgfqpoint{3.780413in}{2.008399in}}%
\pgfpathlineto{\pgfqpoint{3.788433in}{2.019055in}}%
\pgfpathlineto{\pgfqpoint{3.796447in}{2.029700in}}%
\pgfpathlineto{\pgfqpoint{3.782815in}{2.029711in}}%
\pgfpathlineto{\pgfqpoint{3.769190in}{2.029922in}}%
\pgfpathlineto{\pgfqpoint{3.755572in}{2.030333in}}%
\pgfpathlineto{\pgfqpoint{3.741961in}{2.030945in}}%
\pgfpathlineto{\pgfqpoint{3.733938in}{2.020438in}}%
\pgfpathlineto{\pgfqpoint{3.725909in}{2.009928in}}%
\pgfpathlineto{\pgfqpoint{3.717876in}{1.999416in}}%
\pgfpathlineto{\pgfqpoint{3.709837in}{1.988904in}}%
\pgfpathclose%
\pgfusepath{fill}%
\end{pgfscope}%
\begin{pgfscope}%
\pgfpathrectangle{\pgfqpoint{1.150000in}{0.150000in}}{\pgfqpoint{5.700000in}{5.700000in}}%
\pgfusepath{clip}%
\pgfsetbuttcap%
\pgfsetroundjoin%
\definecolor{currentfill}{rgb}{0.281887,0.150881,0.465405}%
\pgfsetfillcolor{currentfill}%
\pgfsetfillopacity{0.800000}%
\pgfsetlinewidth{0.000000pt}%
\definecolor{currentstroke}{rgb}{0.000000,0.000000,0.000000}%
\pgfsetstrokecolor{currentstroke}%
\pgfsetdash{}{0pt}%
\pgfpathmoveto{\pgfqpoint{3.969589in}{2.123029in}}%
\pgfpathlineto{\pgfqpoint{3.983278in}{2.124993in}}%
\pgfpathlineto{\pgfqpoint{3.996976in}{2.127151in}}%
\pgfpathlineto{\pgfqpoint{4.010684in}{2.129502in}}%
\pgfpathlineto{\pgfqpoint{4.024401in}{2.132047in}}%
\pgfpathlineto{\pgfqpoint{4.032348in}{2.142759in}}%
\pgfpathlineto{\pgfqpoint{4.040290in}{2.153431in}}%
\pgfpathlineto{\pgfqpoint{4.048227in}{2.164062in}}%
\pgfpathlineto{\pgfqpoint{4.056159in}{2.174652in}}%
\pgfpathlineto{\pgfqpoint{4.042449in}{2.172053in}}%
\pgfpathlineto{\pgfqpoint{4.028748in}{2.169646in}}%
\pgfpathlineto{\pgfqpoint{4.015056in}{2.167433in}}%
\pgfpathlineto{\pgfqpoint{4.001374in}{2.165414in}}%
\pgfpathlineto{\pgfqpoint{3.993435in}{2.154867in}}%
\pgfpathlineto{\pgfqpoint{3.985491in}{2.144287in}}%
\pgfpathlineto{\pgfqpoint{3.977543in}{2.133674in}}%
\pgfpathlineto{\pgfqpoint{3.969589in}{2.123029in}}%
\pgfpathclose%
\pgfusepath{fill}%
\end{pgfscope}%
\begin{pgfscope}%
\pgfpathrectangle{\pgfqpoint{1.150000in}{0.150000in}}{\pgfqpoint{5.700000in}{5.700000in}}%
\pgfusepath{clip}%
\pgfsetbuttcap%
\pgfsetroundjoin%
\definecolor{currentfill}{rgb}{0.171176,0.452530,0.557965}%
\pgfsetfillcolor{currentfill}%
\pgfsetfillopacity{0.800000}%
\pgfsetlinewidth{0.000000pt}%
\definecolor{currentstroke}{rgb}{0.000000,0.000000,0.000000}%
\pgfsetstrokecolor{currentstroke}%
\pgfsetdash{}{0pt}%
\pgfpathmoveto{\pgfqpoint{5.071807in}{2.898039in}}%
\pgfpathlineto{\pgfqpoint{5.085993in}{2.906727in}}%
\pgfpathlineto{\pgfqpoint{5.100195in}{2.915594in}}%
\pgfpathlineto{\pgfqpoint{5.114414in}{2.924641in}}%
\pgfpathlineto{\pgfqpoint{5.128650in}{2.933867in}}%
\pgfpathlineto{\pgfqpoint{5.136157in}{2.939769in}}%
\pgfpathlineto{\pgfqpoint{5.143657in}{2.945655in}}%
\pgfpathlineto{\pgfqpoint{5.151152in}{2.951531in}}%
\pgfpathlineto{\pgfqpoint{5.158640in}{2.957402in}}%
\pgfpathlineto{\pgfqpoint{5.144421in}{2.948582in}}%
\pgfpathlineto{\pgfqpoint{5.130219in}{2.939941in}}%
\pgfpathlineto{\pgfqpoint{5.116033in}{2.931478in}}%
\pgfpathlineto{\pgfqpoint{5.101864in}{2.923195in}}%
\pgfpathlineto{\pgfqpoint{5.094358in}{2.916908in}}%
\pgfpathlineto{\pgfqpoint{5.086847in}{2.910622in}}%
\pgfpathlineto{\pgfqpoint{5.079330in}{2.904335in}}%
\pgfpathlineto{\pgfqpoint{5.071807in}{2.898039in}}%
\pgfpathclose%
\pgfusepath{fill}%
\end{pgfscope}%
\begin{pgfscope}%
\pgfpathrectangle{\pgfqpoint{1.150000in}{0.150000in}}{\pgfqpoint{5.700000in}{5.700000in}}%
\pgfusepath{clip}%
\pgfsetbuttcap%
\pgfsetroundjoin%
\definecolor{currentfill}{rgb}{0.124780,0.640461,0.527068}%
\pgfsetfillcolor{currentfill}%
\pgfsetfillopacity{0.800000}%
\pgfsetlinewidth{0.000000pt}%
\definecolor{currentstroke}{rgb}{0.000000,0.000000,0.000000}%
\pgfsetstrokecolor{currentstroke}%
\pgfsetdash{}{0pt}%
\pgfpathmoveto{\pgfqpoint{5.969343in}{3.475944in}}%
\pgfpathlineto{\pgfqpoint{5.983961in}{3.485358in}}%
\pgfpathlineto{\pgfqpoint{5.998598in}{3.494944in}}%
\pgfpathlineto{\pgfqpoint{6.013255in}{3.504702in}}%
\pgfpathlineto{\pgfqpoint{6.027931in}{3.514633in}}%
\pgfpathlineto{\pgfqpoint{6.034986in}{3.517857in}}%
\pgfpathlineto{\pgfqpoint{6.042041in}{3.521273in}}%
\pgfpathlineto{\pgfqpoint{6.049097in}{3.524889in}}%
\pgfpathlineto{\pgfqpoint{6.056154in}{3.528713in}}%
\pgfpathlineto{\pgfqpoint{6.041518in}{3.519555in}}%
\pgfpathlineto{\pgfqpoint{6.026901in}{3.510568in}}%
\pgfpathlineto{\pgfqpoint{6.012303in}{3.501751in}}%
\pgfpathlineto{\pgfqpoint{5.997725in}{3.493106in}}%
\pgfpathlineto{\pgfqpoint{5.990627in}{3.488501in}}%
\pgfpathlineto{\pgfqpoint{5.983531in}{3.484111in}}%
\pgfpathlineto{\pgfqpoint{5.976437in}{3.479928in}}%
\pgfpathlineto{\pgfqpoint{5.969343in}{3.475944in}}%
\pgfpathclose%
\pgfusepath{fill}%
\end{pgfscope}%
\begin{pgfscope}%
\pgfpathrectangle{\pgfqpoint{1.150000in}{0.150000in}}{\pgfqpoint{5.700000in}{5.700000in}}%
\pgfusepath{clip}%
\pgfsetbuttcap%
\pgfsetroundjoin%
\definecolor{currentfill}{rgb}{0.278826,0.175490,0.483397}%
\pgfsetfillcolor{currentfill}%
\pgfsetfillopacity{0.800000}%
\pgfsetlinewidth{0.000000pt}%
\definecolor{currentstroke}{rgb}{0.000000,0.000000,0.000000}%
\pgfsetstrokecolor{currentstroke}%
\pgfsetdash{}{0pt}%
\pgfpathmoveto{\pgfqpoint{4.056159in}{2.174652in}}%
\pgfpathlineto{\pgfqpoint{4.069879in}{2.177444in}}%
\pgfpathlineto{\pgfqpoint{4.083609in}{2.180429in}}%
\pgfpathlineto{\pgfqpoint{4.097348in}{2.183605in}}%
\pgfpathlineto{\pgfqpoint{4.111098in}{2.186972in}}%
\pgfpathlineto{\pgfqpoint{4.119019in}{2.197556in}}%
\pgfpathlineto{\pgfqpoint{4.126934in}{2.208091in}}%
\pgfpathlineto{\pgfqpoint{4.134844in}{2.218577in}}%
\pgfpathlineto{\pgfqpoint{4.142750in}{2.229015in}}%
\pgfpathlineto{\pgfqpoint{4.129006in}{2.225625in}}%
\pgfpathlineto{\pgfqpoint{4.115273in}{2.222426in}}%
\pgfpathlineto{\pgfqpoint{4.101549in}{2.219418in}}%
\pgfpathlineto{\pgfqpoint{4.087836in}{2.216603in}}%
\pgfpathlineto{\pgfqpoint{4.079924in}{2.206176in}}%
\pgfpathlineto{\pgfqpoint{4.072007in}{2.195709in}}%
\pgfpathlineto{\pgfqpoint{4.064086in}{2.185201in}}%
\pgfpathlineto{\pgfqpoint{4.056159in}{2.174652in}}%
\pgfpathclose%
\pgfusepath{fill}%
\end{pgfscope}%
\begin{pgfscope}%
\pgfpathrectangle{\pgfqpoint{1.150000in}{0.150000in}}{\pgfqpoint{5.700000in}{5.700000in}}%
\pgfusepath{clip}%
\pgfsetbuttcap%
\pgfsetroundjoin%
\definecolor{currentfill}{rgb}{0.279566,0.067836,0.391917}%
\pgfsetfillcolor{currentfill}%
\pgfsetfillopacity{0.800000}%
\pgfsetlinewidth{0.000000pt}%
\definecolor{currentstroke}{rgb}{0.000000,0.000000,0.000000}%
\pgfsetstrokecolor{currentstroke}%
\pgfsetdash{}{0pt}%
\pgfpathmoveto{\pgfqpoint{3.623168in}{1.952664in}}%
\pgfpathlineto{\pgfqpoint{3.636774in}{1.950911in}}%
\pgfpathlineto{\pgfqpoint{3.650386in}{1.949360in}}%
\pgfpathlineto{\pgfqpoint{3.664004in}{1.948013in}}%
\pgfpathlineto{\pgfqpoint{3.677628in}{1.946868in}}%
\pgfpathlineto{\pgfqpoint{3.685688in}{1.957372in}}%
\pgfpathlineto{\pgfqpoint{3.693743in}{1.967880in}}%
\pgfpathlineto{\pgfqpoint{3.701792in}{1.978391in}}%
\pgfpathlineto{\pgfqpoint{3.709837in}{1.988904in}}%
\pgfpathlineto{\pgfqpoint{3.696223in}{1.989867in}}%
\pgfpathlineto{\pgfqpoint{3.682615in}{1.991033in}}%
\pgfpathlineto{\pgfqpoint{3.669013in}{1.992401in}}%
\pgfpathlineto{\pgfqpoint{3.655417in}{1.993973in}}%
\pgfpathlineto{\pgfqpoint{3.647363in}{1.983631in}}%
\pgfpathlineto{\pgfqpoint{3.639303in}{1.973297in}}%
\pgfpathlineto{\pgfqpoint{3.631238in}{1.962975in}}%
\pgfpathlineto{\pgfqpoint{3.623168in}{1.952664in}}%
\pgfpathclose%
\pgfusepath{fill}%
\end{pgfscope}%
\begin{pgfscope}%
\pgfpathrectangle{\pgfqpoint{1.150000in}{0.150000in}}{\pgfqpoint{5.700000in}{5.700000in}}%
\pgfusepath{clip}%
\pgfsetbuttcap%
\pgfsetroundjoin%
\definecolor{currentfill}{rgb}{0.223925,0.334994,0.548053}%
\pgfsetfillcolor{currentfill}%
\pgfsetfillopacity{0.800000}%
\pgfsetlinewidth{0.000000pt}%
\definecolor{currentstroke}{rgb}{0.000000,0.000000,0.000000}%
\pgfsetstrokecolor{currentstroke}%
\pgfsetdash{}{0pt}%
\pgfpathmoveto{\pgfqpoint{4.607344in}{2.563813in}}%
\pgfpathlineto{\pgfqpoint{4.621300in}{2.570596in}}%
\pgfpathlineto{\pgfqpoint{4.635270in}{2.577564in}}%
\pgfpathlineto{\pgfqpoint{4.649253in}{2.584715in}}%
\pgfpathlineto{\pgfqpoint{4.663251in}{2.592050in}}%
\pgfpathlineto{\pgfqpoint{4.670973in}{2.600465in}}%
\pgfpathlineto{\pgfqpoint{4.678690in}{2.608821in}}%
\pgfpathlineto{\pgfqpoint{4.686400in}{2.617119in}}%
\pgfpathlineto{\pgfqpoint{4.694103in}{2.625362in}}%
\pgfpathlineto{\pgfqpoint{4.680115in}{2.618232in}}%
\pgfpathlineto{\pgfqpoint{4.666140in}{2.611286in}}%
\pgfpathlineto{\pgfqpoint{4.652180in}{2.604524in}}%
\pgfpathlineto{\pgfqpoint{4.638233in}{2.597946in}}%
\pgfpathlineto{\pgfqpoint{4.630520in}{2.589485in}}%
\pgfpathlineto{\pgfqpoint{4.622801in}{2.580978in}}%
\pgfpathlineto{\pgfqpoint{4.615075in}{2.572422in}}%
\pgfpathlineto{\pgfqpoint{4.607344in}{2.563813in}}%
\pgfpathclose%
\pgfusepath{fill}%
\end{pgfscope}%
\begin{pgfscope}%
\pgfpathrectangle{\pgfqpoint{1.150000in}{0.150000in}}{\pgfqpoint{5.700000in}{5.700000in}}%
\pgfusepath{clip}%
\pgfsetbuttcap%
\pgfsetroundjoin%
\definecolor{currentfill}{rgb}{0.163625,0.471133,0.558148}%
\pgfsetfillcolor{currentfill}%
\pgfsetfillopacity{0.800000}%
\pgfsetlinewidth{0.000000pt}%
\definecolor{currentstroke}{rgb}{0.000000,0.000000,0.000000}%
\pgfsetstrokecolor{currentstroke}%
\pgfsetdash{}{0pt}%
\pgfpathmoveto{\pgfqpoint{5.158640in}{2.957402in}}%
\pgfpathlineto{\pgfqpoint{5.172876in}{2.966400in}}%
\pgfpathlineto{\pgfqpoint{5.187128in}{2.975578in}}%
\pgfpathlineto{\pgfqpoint{5.201397in}{2.984934in}}%
\pgfpathlineto{\pgfqpoint{5.215684in}{2.994469in}}%
\pgfpathlineto{\pgfqpoint{5.223148in}{2.999911in}}%
\pgfpathlineto{\pgfqpoint{5.230606in}{3.005349in}}%
\pgfpathlineto{\pgfqpoint{5.238058in}{3.010787in}}%
\pgfpathlineto{\pgfqpoint{5.245504in}{3.016231in}}%
\pgfpathlineto{\pgfqpoint{5.231236in}{3.007137in}}%
\pgfpathlineto{\pgfqpoint{5.216986in}{2.998220in}}%
\pgfpathlineto{\pgfqpoint{5.202752in}{2.989481in}}%
\pgfpathlineto{\pgfqpoint{5.188535in}{2.980920in}}%
\pgfpathlineto{\pgfqpoint{5.181070in}{2.975025in}}%
\pgfpathlineto{\pgfqpoint{5.173599in}{2.969144in}}%
\pgfpathlineto{\pgfqpoint{5.166123in}{2.963271in}}%
\pgfpathlineto{\pgfqpoint{5.158640in}{2.957402in}}%
\pgfpathclose%
\pgfusepath{fill}%
\end{pgfscope}%
\begin{pgfscope}%
\pgfpathrectangle{\pgfqpoint{1.150000in}{0.150000in}}{\pgfqpoint{5.700000in}{5.700000in}}%
\pgfusepath{clip}%
\pgfsetbuttcap%
\pgfsetroundjoin%
\definecolor{currentfill}{rgb}{0.274128,0.199721,0.498911}%
\pgfsetfillcolor{currentfill}%
\pgfsetfillopacity{0.800000}%
\pgfsetlinewidth{0.000000pt}%
\definecolor{currentstroke}{rgb}{0.000000,0.000000,0.000000}%
\pgfsetstrokecolor{currentstroke}%
\pgfsetdash{}{0pt}%
\pgfpathmoveto{\pgfqpoint{4.142750in}{2.229015in}}%
\pgfpathlineto{\pgfqpoint{4.156504in}{2.232596in}}%
\pgfpathlineto{\pgfqpoint{4.170268in}{2.236368in}}%
\pgfpathlineto{\pgfqpoint{4.184043in}{2.240330in}}%
\pgfpathlineto{\pgfqpoint{4.197829in}{2.244482in}}%
\pgfpathlineto{\pgfqpoint{4.205723in}{2.254875in}}%
\pgfpathlineto{\pgfqpoint{4.213612in}{2.265212in}}%
\pgfpathlineto{\pgfqpoint{4.221495in}{2.275494in}}%
\pgfpathlineto{\pgfqpoint{4.229373in}{2.285721in}}%
\pgfpathlineto{\pgfqpoint{4.215593in}{2.281579in}}%
\pgfpathlineto{\pgfqpoint{4.201824in}{2.277626in}}%
\pgfpathlineto{\pgfqpoint{4.188066in}{2.273863in}}%
\pgfpathlineto{\pgfqpoint{4.174318in}{2.270291in}}%
\pgfpathlineto{\pgfqpoint{4.166434in}{2.260043in}}%
\pgfpathlineto{\pgfqpoint{4.158544in}{2.249747in}}%
\pgfpathlineto{\pgfqpoint{4.150650in}{2.239405in}}%
\pgfpathlineto{\pgfqpoint{4.142750in}{2.229015in}}%
\pgfpathclose%
\pgfusepath{fill}%
\end{pgfscope}%
\begin{pgfscope}%
\pgfpathrectangle{\pgfqpoint{1.150000in}{0.150000in}}{\pgfqpoint{5.700000in}{5.700000in}}%
\pgfusepath{clip}%
\pgfsetbuttcap%
\pgfsetroundjoin%
\definecolor{currentfill}{rgb}{0.281446,0.084320,0.407414}%
\pgfsetfillcolor{currentfill}%
\pgfsetfillopacity{0.800000}%
\pgfsetlinewidth{0.000000pt}%
\definecolor{currentstroke}{rgb}{0.000000,0.000000,0.000000}%
\pgfsetstrokecolor{currentstroke}%
\pgfsetdash{}{0pt}%
\pgfpathmoveto{\pgfqpoint{2.969852in}{2.008533in}}%
\pgfpathlineto{\pgfqpoint{2.983487in}{1.997391in}}%
\pgfpathlineto{\pgfqpoint{2.997120in}{1.986494in}}%
\pgfpathlineto{\pgfqpoint{3.010751in}{1.975841in}}%
\pgfpathlineto{\pgfqpoint{3.024379in}{1.965430in}}%
\pgfpathlineto{\pgfqpoint{3.032725in}{1.972236in}}%
\pgfpathlineto{\pgfqpoint{3.041061in}{1.979165in}}%
\pgfpathlineto{\pgfqpoint{3.049389in}{1.986214in}}%
\pgfpathlineto{\pgfqpoint{3.057707in}{1.993380in}}%
\pgfpathlineto{\pgfqpoint{3.044103in}{2.003447in}}%
\pgfpathlineto{\pgfqpoint{3.030497in}{2.013756in}}%
\pgfpathlineto{\pgfqpoint{3.016889in}{2.024308in}}%
\pgfpathlineto{\pgfqpoint{3.003279in}{2.035105in}}%
\pgfpathlineto{\pgfqpoint{2.994936in}{2.028272in}}%
\pgfpathlineto{\pgfqpoint{2.986584in}{2.021564in}}%
\pgfpathlineto{\pgfqpoint{2.978223in}{2.014983in}}%
\pgfpathlineto{\pgfqpoint{2.969852in}{2.008533in}}%
\pgfpathclose%
\pgfusepath{fill}%
\end{pgfscope}%
\begin{pgfscope}%
\pgfpathrectangle{\pgfqpoint{1.150000in}{0.150000in}}{\pgfqpoint{5.700000in}{5.700000in}}%
\pgfusepath{clip}%
\pgfsetbuttcap%
\pgfsetroundjoin%
\definecolor{currentfill}{rgb}{0.276022,0.044167,0.370164}%
\pgfsetfillcolor{currentfill}%
\pgfsetfillopacity{0.800000}%
\pgfsetlinewidth{0.000000pt}%
\definecolor{currentstroke}{rgb}{0.000000,0.000000,0.000000}%
\pgfsetstrokecolor{currentstroke}%
\pgfsetdash{}{0pt}%
\pgfpathmoveto{\pgfqpoint{3.166497in}{1.921361in}}%
\pgfpathlineto{\pgfqpoint{3.180094in}{1.913402in}}%
\pgfpathlineto{\pgfqpoint{3.193691in}{1.905672in}}%
\pgfpathlineto{\pgfqpoint{3.207288in}{1.898167in}}%
\pgfpathlineto{\pgfqpoint{3.220886in}{1.890888in}}%
\pgfpathlineto{\pgfqpoint{3.229132in}{1.899134in}}%
\pgfpathlineto{\pgfqpoint{3.237370in}{1.907467in}}%
\pgfpathlineto{\pgfqpoint{3.245601in}{1.915883in}}%
\pgfpathlineto{\pgfqpoint{3.253825in}{1.924381in}}%
\pgfpathlineto{\pgfqpoint{3.240246in}{1.931351in}}%
\pgfpathlineto{\pgfqpoint{3.226668in}{1.938545in}}%
\pgfpathlineto{\pgfqpoint{3.213091in}{1.945966in}}%
\pgfpathlineto{\pgfqpoint{3.199514in}{1.953615in}}%
\pgfpathlineto{\pgfqpoint{3.191271in}{1.945415in}}%
\pgfpathlineto{\pgfqpoint{3.183021in}{1.937304in}}%
\pgfpathlineto{\pgfqpoint{3.174763in}{1.929285in}}%
\pgfpathlineto{\pgfqpoint{3.166497in}{1.921361in}}%
\pgfpathclose%
\pgfusepath{fill}%
\end{pgfscope}%
\begin{pgfscope}%
\pgfpathrectangle{\pgfqpoint{1.150000in}{0.150000in}}{\pgfqpoint{5.700000in}{5.700000in}}%
\pgfusepath{clip}%
\pgfsetbuttcap%
\pgfsetroundjoin%
\definecolor{currentfill}{rgb}{0.274952,0.037752,0.364543}%
\pgfsetfillcolor{currentfill}%
\pgfsetfillopacity{0.800000}%
\pgfsetlinewidth{0.000000pt}%
\definecolor{currentstroke}{rgb}{0.000000,0.000000,0.000000}%
\pgfsetstrokecolor{currentstroke}%
\pgfsetdash{}{0pt}%
\pgfpathmoveto{\pgfqpoint{3.308153in}{1.898727in}}%
\pgfpathlineto{\pgfqpoint{3.321739in}{1.892864in}}%
\pgfpathlineto{\pgfqpoint{3.335327in}{1.887220in}}%
\pgfpathlineto{\pgfqpoint{3.348918in}{1.881792in}}%
\pgfpathlineto{\pgfqpoint{3.362511in}{1.876581in}}%
\pgfpathlineto{\pgfqpoint{3.370693in}{1.885735in}}%
\pgfpathlineto{\pgfqpoint{3.378868in}{1.894949in}}%
\pgfpathlineto{\pgfqpoint{3.387037in}{1.904220in}}%
\pgfpathlineto{\pgfqpoint{3.395199in}{1.913546in}}%
\pgfpathlineto{\pgfqpoint{3.381622in}{1.918481in}}%
\pgfpathlineto{\pgfqpoint{3.368047in}{1.923632in}}%
\pgfpathlineto{\pgfqpoint{3.354475in}{1.928999in}}%
\pgfpathlineto{\pgfqpoint{3.340906in}{1.934585in}}%
\pgfpathlineto{\pgfqpoint{3.332728in}{1.925523in}}%
\pgfpathlineto{\pgfqpoint{3.324543in}{1.916525in}}%
\pgfpathlineto{\pgfqpoint{3.316351in}{1.907592in}}%
\pgfpathlineto{\pgfqpoint{3.308153in}{1.898727in}}%
\pgfpathclose%
\pgfusepath{fill}%
\end{pgfscope}%
\begin{pgfscope}%
\pgfpathrectangle{\pgfqpoint{1.150000in}{0.150000in}}{\pgfqpoint{5.700000in}{5.700000in}}%
\pgfusepath{clip}%
\pgfsetbuttcap%
\pgfsetroundjoin%
\definecolor{currentfill}{rgb}{0.277941,0.056324,0.381191}%
\pgfsetfillcolor{currentfill}%
\pgfsetfillopacity{0.800000}%
\pgfsetlinewidth{0.000000pt}%
\definecolor{currentstroke}{rgb}{0.000000,0.000000,0.000000}%
\pgfsetstrokecolor{currentstroke}%
\pgfsetdash{}{0pt}%
\pgfpathmoveto{\pgfqpoint{3.536413in}{1.921498in}}%
\pgfpathlineto{\pgfqpoint{3.550010in}{1.918709in}}%
\pgfpathlineto{\pgfqpoint{3.563612in}{1.916126in}}%
\pgfpathlineto{\pgfqpoint{3.577219in}{1.913750in}}%
\pgfpathlineto{\pgfqpoint{3.590831in}{1.911579in}}%
\pgfpathlineto{\pgfqpoint{3.598924in}{1.921823in}}%
\pgfpathlineto{\pgfqpoint{3.607011in}{1.932087in}}%
\pgfpathlineto{\pgfqpoint{3.615092in}{1.942368in}}%
\pgfpathlineto{\pgfqpoint{3.623168in}{1.952664in}}%
\pgfpathlineto{\pgfqpoint{3.609567in}{1.954623in}}%
\pgfpathlineto{\pgfqpoint{3.595972in}{1.956786in}}%
\pgfpathlineto{\pgfqpoint{3.582381in}{1.959155in}}%
\pgfpathlineto{\pgfqpoint{3.568796in}{1.961731in}}%
\pgfpathlineto{\pgfqpoint{3.560709in}{1.951636in}}%
\pgfpathlineto{\pgfqpoint{3.552616in}{1.941565in}}%
\pgfpathlineto{\pgfqpoint{3.544517in}{1.931518in}}%
\pgfpathlineto{\pgfqpoint{3.536413in}{1.921498in}}%
\pgfpathclose%
\pgfusepath{fill}%
\end{pgfscope}%
\begin{pgfscope}%
\pgfpathrectangle{\pgfqpoint{1.150000in}{0.150000in}}{\pgfqpoint{5.700000in}{5.700000in}}%
\pgfusepath{clip}%
\pgfsetbuttcap%
\pgfsetroundjoin%
\definecolor{currentfill}{rgb}{0.216210,0.351535,0.550627}%
\pgfsetfillcolor{currentfill}%
\pgfsetfillopacity{0.800000}%
\pgfsetlinewidth{0.000000pt}%
\definecolor{currentstroke}{rgb}{0.000000,0.000000,0.000000}%
\pgfsetstrokecolor{currentstroke}%
\pgfsetdash{}{0pt}%
\pgfpathmoveto{\pgfqpoint{2.419394in}{2.677764in}}%
\pgfpathlineto{\pgfqpoint{2.433349in}{2.655096in}}%
\pgfpathlineto{\pgfqpoint{2.447289in}{2.632768in}}%
\pgfpathlineto{\pgfqpoint{2.461216in}{2.610778in}}%
\pgfpathlineto{\pgfqpoint{2.475130in}{2.589122in}}%
\pgfpathlineto{\pgfqpoint{2.483793in}{2.592273in}}%
\pgfpathlineto{\pgfqpoint{2.492443in}{2.595630in}}%
\pgfpathlineto{\pgfqpoint{2.501078in}{2.599189in}}%
\pgfpathlineto{\pgfqpoint{2.509700in}{2.602948in}}%
\pgfpathlineto{\pgfqpoint{2.495825in}{2.624238in}}%
\pgfpathlineto{\pgfqpoint{2.481937in}{2.645860in}}%
\pgfpathlineto{\pgfqpoint{2.468037in}{2.667820in}}%
\pgfpathlineto{\pgfqpoint{2.454122in}{2.690118in}}%
\pgfpathlineto{\pgfqpoint{2.445462in}{2.686714in}}%
\pgfpathlineto{\pgfqpoint{2.436788in}{2.683518in}}%
\pgfpathlineto{\pgfqpoint{2.428098in}{2.680534in}}%
\pgfpathlineto{\pgfqpoint{2.419394in}{2.677764in}}%
\pgfpathclose%
\pgfusepath{fill}%
\end{pgfscope}%
\begin{pgfscope}%
\pgfpathrectangle{\pgfqpoint{1.150000in}{0.150000in}}{\pgfqpoint{5.700000in}{5.700000in}}%
\pgfusepath{clip}%
\pgfsetbuttcap%
\pgfsetroundjoin%
\definecolor{currentfill}{rgb}{0.134692,0.658636,0.517649}%
\pgfsetfillcolor{currentfill}%
\pgfsetfillopacity{0.800000}%
\pgfsetlinewidth{0.000000pt}%
\definecolor{currentstroke}{rgb}{0.000000,0.000000,0.000000}%
\pgfsetstrokecolor{currentstroke}%
\pgfsetdash{}{0pt}%
\pgfpathmoveto{\pgfqpoint{6.056154in}{3.528713in}}%
\pgfpathlineto{\pgfqpoint{6.070810in}{3.538042in}}%
\pgfpathlineto{\pgfqpoint{6.085486in}{3.547542in}}%
\pgfpathlineto{\pgfqpoint{6.100182in}{3.557214in}}%
\pgfpathlineto{\pgfqpoint{6.114898in}{3.567057in}}%
\pgfpathlineto{\pgfqpoint{6.121915in}{3.570305in}}%
\pgfpathlineto{\pgfqpoint{6.128935in}{3.573772in}}%
\pgfpathlineto{\pgfqpoint{6.135957in}{3.577465in}}%
\pgfpathlineto{\pgfqpoint{6.121273in}{3.568223in}}%
\pgfpathlineto{\pgfqpoint{6.106609in}{3.559152in}}%
\pgfpathlineto{\pgfqpoint{6.091965in}{3.550251in}}%
\pgfpathlineto{\pgfqpoint{6.077340in}{3.541520in}}%
\pgfpathlineto{\pgfqpoint{6.070276in}{3.537020in}}%
\pgfpathlineto{\pgfqpoint{6.063214in}{3.532754in}}%
\pgfpathlineto{\pgfqpoint{6.056154in}{3.528713in}}%
\pgfpathclose%
\pgfusepath{fill}%
\end{pgfscope}%
\begin{pgfscope}%
\pgfpathrectangle{\pgfqpoint{1.150000in}{0.150000in}}{\pgfqpoint{5.700000in}{5.700000in}}%
\pgfusepath{clip}%
\pgfsetbuttcap%
\pgfsetroundjoin%
\definecolor{currentfill}{rgb}{0.212395,0.359683,0.551710}%
\pgfsetfillcolor{currentfill}%
\pgfsetfillopacity{0.800000}%
\pgfsetlinewidth{0.000000pt}%
\definecolor{currentstroke}{rgb}{0.000000,0.000000,0.000000}%
\pgfsetstrokecolor{currentstroke}%
\pgfsetdash{}{0pt}%
\pgfpathmoveto{\pgfqpoint{4.694103in}{2.625362in}}%
\pgfpathlineto{\pgfqpoint{4.708106in}{2.632674in}}%
\pgfpathlineto{\pgfqpoint{4.722124in}{2.640170in}}%
\pgfpathlineto{\pgfqpoint{4.736156in}{2.647850in}}%
\pgfpathlineto{\pgfqpoint{4.750202in}{2.655712in}}%
\pgfpathlineto{\pgfqpoint{4.757890in}{2.663676in}}%
\pgfpathlineto{\pgfqpoint{4.765572in}{2.671583in}}%
\pgfpathlineto{\pgfqpoint{4.773247in}{2.679435in}}%
\pgfpathlineto{\pgfqpoint{4.780916in}{2.687237in}}%
\pgfpathlineto{\pgfqpoint{4.766880in}{2.679614in}}%
\pgfpathlineto{\pgfqpoint{4.752858in}{2.672174in}}%
\pgfpathlineto{\pgfqpoint{4.738851in}{2.664916in}}%
\pgfpathlineto{\pgfqpoint{4.724858in}{2.657842in}}%
\pgfpathlineto{\pgfqpoint{4.717178in}{2.649790in}}%
\pgfpathlineto{\pgfqpoint{4.709493in}{2.641694in}}%
\pgfpathlineto{\pgfqpoint{4.701801in}{2.633553in}}%
\pgfpathlineto{\pgfqpoint{4.694103in}{2.625362in}}%
\pgfpathclose%
\pgfusepath{fill}%
\end{pgfscope}%
\begin{pgfscope}%
\pgfpathrectangle{\pgfqpoint{1.150000in}{0.150000in}}{\pgfqpoint{5.700000in}{5.700000in}}%
\pgfusepath{clip}%
\pgfsetbuttcap%
\pgfsetroundjoin%
\definecolor{currentfill}{rgb}{0.266580,0.228262,0.514349}%
\pgfsetfillcolor{currentfill}%
\pgfsetfillopacity{0.800000}%
\pgfsetlinewidth{0.000000pt}%
\definecolor{currentstroke}{rgb}{0.000000,0.000000,0.000000}%
\pgfsetstrokecolor{currentstroke}%
\pgfsetdash{}{0pt}%
\pgfpathmoveto{\pgfqpoint{4.229373in}{2.285721in}}%
\pgfpathlineto{\pgfqpoint{4.243164in}{2.290053in}}%
\pgfpathlineto{\pgfqpoint{4.256967in}{2.294574in}}%
\pgfpathlineto{\pgfqpoint{4.270780in}{2.299283in}}%
\pgfpathlineto{\pgfqpoint{4.284606in}{2.304182in}}%
\pgfpathlineto{\pgfqpoint{4.292472in}{2.314326in}}%
\pgfpathlineto{\pgfqpoint{4.300334in}{2.324409in}}%
\pgfpathlineto{\pgfqpoint{4.308190in}{2.334432in}}%
\pgfpathlineto{\pgfqpoint{4.316040in}{2.344395in}}%
\pgfpathlineto{\pgfqpoint{4.302221in}{2.339539in}}%
\pgfpathlineto{\pgfqpoint{4.288413in}{2.334871in}}%
\pgfpathlineto{\pgfqpoint{4.274617in}{2.330392in}}%
\pgfpathlineto{\pgfqpoint{4.260833in}{2.326101in}}%
\pgfpathlineto{\pgfqpoint{4.252976in}{2.316084in}}%
\pgfpathlineto{\pgfqpoint{4.245114in}{2.306016in}}%
\pgfpathlineto{\pgfqpoint{4.237246in}{2.295895in}}%
\pgfpathlineto{\pgfqpoint{4.229373in}{2.285721in}}%
\pgfpathclose%
\pgfusepath{fill}%
\end{pgfscope}%
\begin{pgfscope}%
\pgfpathrectangle{\pgfqpoint{1.150000in}{0.150000in}}{\pgfqpoint{5.700000in}{5.700000in}}%
\pgfusepath{clip}%
\pgfsetbuttcap%
\pgfsetroundjoin%
\definecolor{currentfill}{rgb}{0.154815,0.493313,0.557840}%
\pgfsetfillcolor{currentfill}%
\pgfsetfillopacity{0.800000}%
\pgfsetlinewidth{0.000000pt}%
\definecolor{currentstroke}{rgb}{0.000000,0.000000,0.000000}%
\pgfsetstrokecolor{currentstroke}%
\pgfsetdash{}{0pt}%
\pgfpathmoveto{\pgfqpoint{5.245504in}{3.016231in}}%
\pgfpathlineto{\pgfqpoint{5.259789in}{3.025504in}}%
\pgfpathlineto{\pgfqpoint{5.274091in}{3.034955in}}%
\pgfpathlineto{\pgfqpoint{5.288410in}{3.044585in}}%
\pgfpathlineto{\pgfqpoint{5.302747in}{3.054392in}}%
\pgfpathlineto{\pgfqpoint{5.310167in}{3.059386in}}%
\pgfpathlineto{\pgfqpoint{5.317581in}{3.064388in}}%
\pgfpathlineto{\pgfqpoint{5.324990in}{3.069404in}}%
\pgfpathlineto{\pgfqpoint{5.332393in}{3.074439in}}%
\pgfpathlineto{\pgfqpoint{5.318076in}{3.065106in}}%
\pgfpathlineto{\pgfqpoint{5.303777in}{3.055950in}}%
\pgfpathlineto{\pgfqpoint{5.289496in}{3.046972in}}%
\pgfpathlineto{\pgfqpoint{5.275231in}{3.038171in}}%
\pgfpathlineto{\pgfqpoint{5.267808in}{3.032651in}}%
\pgfpathlineto{\pgfqpoint{5.260379in}{3.027158in}}%
\pgfpathlineto{\pgfqpoint{5.252944in}{3.021687in}}%
\pgfpathlineto{\pgfqpoint{5.245504in}{3.016231in}}%
\pgfpathclose%
\pgfusepath{fill}%
\end{pgfscope}%
\begin{pgfscope}%
\pgfpathrectangle{\pgfqpoint{1.150000in}{0.150000in}}{\pgfqpoint{5.700000in}{5.700000in}}%
\pgfusepath{clip}%
\pgfsetbuttcap%
\pgfsetroundjoin%
\definecolor{currentfill}{rgb}{0.276022,0.044167,0.370164}%
\pgfsetfillcolor{currentfill}%
\pgfsetfillopacity{0.800000}%
\pgfsetlinewidth{0.000000pt}%
\definecolor{currentstroke}{rgb}{0.000000,0.000000,0.000000}%
\pgfsetstrokecolor{currentstroke}%
\pgfsetdash{}{0pt}%
\pgfpathmoveto{\pgfqpoint{3.449539in}{1.895947in}}%
\pgfpathlineto{\pgfqpoint{3.463133in}{1.892078in}}%
\pgfpathlineto{\pgfqpoint{3.476730in}{1.888418in}}%
\pgfpathlineto{\pgfqpoint{3.490331in}{1.884968in}}%
\pgfpathlineto{\pgfqpoint{3.503937in}{1.881727in}}%
\pgfpathlineto{\pgfqpoint{3.512065in}{1.891619in}}%
\pgfpathlineto{\pgfqpoint{3.520187in}{1.901547in}}%
\pgfpathlineto{\pgfqpoint{3.528303in}{1.911507in}}%
\pgfpathlineto{\pgfqpoint{3.536413in}{1.921498in}}%
\pgfpathlineto{\pgfqpoint{3.522820in}{1.924495in}}%
\pgfpathlineto{\pgfqpoint{3.509232in}{1.927700in}}%
\pgfpathlineto{\pgfqpoint{3.495648in}{1.931115in}}%
\pgfpathlineto{\pgfqpoint{3.482068in}{1.934740in}}%
\pgfpathlineto{\pgfqpoint{3.473945in}{1.924982in}}%
\pgfpathlineto{\pgfqpoint{3.465816in}{1.915262in}}%
\pgfpathlineto{\pgfqpoint{3.457680in}{1.905583in}}%
\pgfpathlineto{\pgfqpoint{3.449539in}{1.895947in}}%
\pgfpathclose%
\pgfusepath{fill}%
\end{pgfscope}%
\begin{pgfscope}%
\pgfpathrectangle{\pgfqpoint{1.150000in}{0.150000in}}{\pgfqpoint{5.700000in}{5.700000in}}%
\pgfusepath{clip}%
\pgfsetbuttcap%
\pgfsetroundjoin%
\definecolor{currentfill}{rgb}{0.147607,0.511733,0.557049}%
\pgfsetfillcolor{currentfill}%
\pgfsetfillopacity{0.800000}%
\pgfsetlinewidth{0.000000pt}%
\definecolor{currentstroke}{rgb}{0.000000,0.000000,0.000000}%
\pgfsetstrokecolor{currentstroke}%
\pgfsetdash{}{0pt}%
\pgfpathmoveto{\pgfqpoint{5.332393in}{3.074439in}}%
\pgfpathlineto{\pgfqpoint{5.346726in}{3.083950in}}%
\pgfpathlineto{\pgfqpoint{5.361078in}{3.093638in}}%
\pgfpathlineto{\pgfqpoint{5.375447in}{3.103503in}}%
\pgfpathlineto{\pgfqpoint{5.389835in}{3.113546in}}%
\pgfpathlineto{\pgfqpoint{5.397210in}{3.118110in}}%
\pgfpathlineto{\pgfqpoint{5.404579in}{3.122697in}}%
\pgfpathlineto{\pgfqpoint{5.411943in}{3.127311in}}%
\pgfpathlineto{\pgfqpoint{5.419302in}{3.131959in}}%
\pgfpathlineto{\pgfqpoint{5.404937in}{3.122424in}}%
\pgfpathlineto{\pgfqpoint{5.390591in}{3.113066in}}%
\pgfpathlineto{\pgfqpoint{5.376261in}{3.103885in}}%
\pgfpathlineto{\pgfqpoint{5.361950in}{3.094880in}}%
\pgfpathlineto{\pgfqpoint{5.354568in}{3.089713in}}%
\pgfpathlineto{\pgfqpoint{5.347182in}{3.084588in}}%
\pgfpathlineto{\pgfqpoint{5.339790in}{3.079499in}}%
\pgfpathlineto{\pgfqpoint{5.332393in}{3.074439in}}%
\pgfpathclose%
\pgfusepath{fill}%
\end{pgfscope}%
\begin{pgfscope}%
\pgfpathrectangle{\pgfqpoint{1.150000in}{0.150000in}}{\pgfqpoint{5.700000in}{5.700000in}}%
\pgfusepath{clip}%
\pgfsetbuttcap%
\pgfsetroundjoin%
\definecolor{currentfill}{rgb}{0.279566,0.067836,0.391917}%
\pgfsetfillcolor{currentfill}%
\pgfsetfillopacity{0.800000}%
\pgfsetlinewidth{0.000000pt}%
\definecolor{currentstroke}{rgb}{0.000000,0.000000,0.000000}%
\pgfsetstrokecolor{currentstroke}%
\pgfsetdash{}{0pt}%
\pgfpathmoveto{\pgfqpoint{3.024379in}{1.965430in}}%
\pgfpathlineto{\pgfqpoint{3.038005in}{1.955259in}}%
\pgfpathlineto{\pgfqpoint{3.051630in}{1.945326in}}%
\pgfpathlineto{\pgfqpoint{3.065253in}{1.935631in}}%
\pgfpathlineto{\pgfqpoint{3.078874in}{1.926172in}}%
\pgfpathlineto{\pgfqpoint{3.087196in}{1.933334in}}%
\pgfpathlineto{\pgfqpoint{3.095509in}{1.940611in}}%
\pgfpathlineto{\pgfqpoint{3.103813in}{1.947999in}}%
\pgfpathlineto{\pgfqpoint{3.112109in}{1.955496in}}%
\pgfpathlineto{\pgfqpoint{3.098510in}{1.964612in}}%
\pgfpathlineto{\pgfqpoint{3.084910in}{1.973964in}}%
\pgfpathlineto{\pgfqpoint{3.071309in}{1.983552in}}%
\pgfpathlineto{\pgfqpoint{3.057707in}{1.993380in}}%
\pgfpathlineto{\pgfqpoint{3.049389in}{1.986214in}}%
\pgfpathlineto{\pgfqpoint{3.041061in}{1.979165in}}%
\pgfpathlineto{\pgfqpoint{3.032725in}{1.972236in}}%
\pgfpathlineto{\pgfqpoint{3.024379in}{1.965430in}}%
\pgfpathclose%
\pgfusepath{fill}%
\end{pgfscope}%
\begin{pgfscope}%
\pgfpathrectangle{\pgfqpoint{1.150000in}{0.150000in}}{\pgfqpoint{5.700000in}{5.700000in}}%
\pgfusepath{clip}%
\pgfsetbuttcap%
\pgfsetroundjoin%
\definecolor{currentfill}{rgb}{0.258965,0.251537,0.524736}%
\pgfsetfillcolor{currentfill}%
\pgfsetfillopacity{0.800000}%
\pgfsetlinewidth{0.000000pt}%
\definecolor{currentstroke}{rgb}{0.000000,0.000000,0.000000}%
\pgfsetstrokecolor{currentstroke}%
\pgfsetdash{}{0pt}%
\pgfpathmoveto{\pgfqpoint{4.316040in}{2.344395in}}%
\pgfpathlineto{\pgfqpoint{4.329871in}{2.349440in}}%
\pgfpathlineto{\pgfqpoint{4.343714in}{2.354672in}}%
\pgfpathlineto{\pgfqpoint{4.357568in}{2.360091in}}%
\pgfpathlineto{\pgfqpoint{4.371436in}{2.365698in}}%
\pgfpathlineto{\pgfqpoint{4.379274in}{2.375541in}}%
\pgfpathlineto{\pgfqpoint{4.387107in}{2.385319in}}%
\pgfpathlineto{\pgfqpoint{4.394935in}{2.395034in}}%
\pgfpathlineto{\pgfqpoint{4.402757in}{2.404685in}}%
\pgfpathlineto{\pgfqpoint{4.388896in}{2.399153in}}%
\pgfpathlineto{\pgfqpoint{4.375048in}{2.393808in}}%
\pgfpathlineto{\pgfqpoint{4.361211in}{2.388650in}}%
\pgfpathlineto{\pgfqpoint{4.347387in}{2.383680in}}%
\pgfpathlineto{\pgfqpoint{4.339558in}{2.373942in}}%
\pgfpathlineto{\pgfqpoint{4.331724in}{2.364149in}}%
\pgfpathlineto{\pgfqpoint{4.323885in}{2.354301in}}%
\pgfpathlineto{\pgfqpoint{4.316040in}{2.344395in}}%
\pgfpathclose%
\pgfusepath{fill}%
\end{pgfscope}%
\begin{pgfscope}%
\pgfpathrectangle{\pgfqpoint{1.150000in}{0.150000in}}{\pgfqpoint{5.700000in}{5.700000in}}%
\pgfusepath{clip}%
\pgfsetbuttcap%
\pgfsetroundjoin%
\definecolor{currentfill}{rgb}{0.201239,0.383670,0.554294}%
\pgfsetfillcolor{currentfill}%
\pgfsetfillopacity{0.800000}%
\pgfsetlinewidth{0.000000pt}%
\definecolor{currentstroke}{rgb}{0.000000,0.000000,0.000000}%
\pgfsetstrokecolor{currentstroke}%
\pgfsetdash{}{0pt}%
\pgfpathmoveto{\pgfqpoint{4.780916in}{2.687237in}}%
\pgfpathlineto{\pgfqpoint{4.794967in}{2.695042in}}%
\pgfpathlineto{\pgfqpoint{4.809034in}{2.703029in}}%
\pgfpathlineto{\pgfqpoint{4.823115in}{2.711199in}}%
\pgfpathlineto{\pgfqpoint{4.837212in}{2.719551in}}%
\pgfpathlineto{\pgfqpoint{4.844864in}{2.727044in}}%
\pgfpathlineto{\pgfqpoint{4.852509in}{2.734483in}}%
\pgfpathlineto{\pgfqpoint{4.860148in}{2.741872in}}%
\pgfpathlineto{\pgfqpoint{4.867781in}{2.749214in}}%
\pgfpathlineto{\pgfqpoint{4.853695in}{2.741135in}}%
\pgfpathlineto{\pgfqpoint{4.839625in}{2.733238in}}%
\pgfpathlineto{\pgfqpoint{4.825570in}{2.725523in}}%
\pgfpathlineto{\pgfqpoint{4.811530in}{2.717989in}}%
\pgfpathlineto{\pgfqpoint{4.803886in}{2.710363in}}%
\pgfpathlineto{\pgfqpoint{4.796235in}{2.702697in}}%
\pgfpathlineto{\pgfqpoint{4.788579in}{2.694989in}}%
\pgfpathlineto{\pgfqpoint{4.780916in}{2.687237in}}%
\pgfpathclose%
\pgfusepath{fill}%
\end{pgfscope}%
\begin{pgfscope}%
\pgfpathrectangle{\pgfqpoint{1.150000in}{0.150000in}}{\pgfqpoint{5.700000in}{5.700000in}}%
\pgfusepath{clip}%
\pgfsetbuttcap%
\pgfsetroundjoin%
\definecolor{currentfill}{rgb}{0.199430,0.387607,0.554642}%
\pgfsetfillcolor{currentfill}%
\pgfsetfillopacity{0.800000}%
\pgfsetlinewidth{0.000000pt}%
\definecolor{currentstroke}{rgb}{0.000000,0.000000,0.000000}%
\pgfsetstrokecolor{currentstroke}%
\pgfsetdash{}{0pt}%
\pgfpathmoveto{\pgfqpoint{2.363431in}{2.771909in}}%
\pgfpathlineto{\pgfqpoint{2.377445in}{2.747844in}}%
\pgfpathlineto{\pgfqpoint{2.391443in}{2.724134in}}%
\pgfpathlineto{\pgfqpoint{2.405426in}{2.700775in}}%
\pgfpathlineto{\pgfqpoint{2.419394in}{2.677764in}}%
\pgfpathlineto{\pgfqpoint{2.428098in}{2.680534in}}%
\pgfpathlineto{\pgfqpoint{2.436788in}{2.683518in}}%
\pgfpathlineto{\pgfqpoint{2.445462in}{2.686714in}}%
\pgfpathlineto{\pgfqpoint{2.454122in}{2.690118in}}%
\pgfpathlineto{\pgfqpoint{2.440194in}{2.712760in}}%
\pgfpathlineto{\pgfqpoint{2.426252in}{2.735748in}}%
\pgfpathlineto{\pgfqpoint{2.412296in}{2.759086in}}%
\pgfpathlineto{\pgfqpoint{2.398324in}{2.782778in}}%
\pgfpathlineto{\pgfqpoint{2.389624in}{2.779732in}}%
\pgfpathlineto{\pgfqpoint{2.380908in}{2.776903in}}%
\pgfpathlineto{\pgfqpoint{2.372177in}{2.774294in}}%
\pgfpathlineto{\pgfqpoint{2.363431in}{2.771909in}}%
\pgfpathclose%
\pgfusepath{fill}%
\end{pgfscope}%
\begin{pgfscope}%
\pgfpathrectangle{\pgfqpoint{1.150000in}{0.150000in}}{\pgfqpoint{5.700000in}{5.700000in}}%
\pgfusepath{clip}%
\pgfsetbuttcap%
\pgfsetroundjoin%
\definecolor{currentfill}{rgb}{0.139147,0.533812,0.555298}%
\pgfsetfillcolor{currentfill}%
\pgfsetfillopacity{0.800000}%
\pgfsetlinewidth{0.000000pt}%
\definecolor{currentstroke}{rgb}{0.000000,0.000000,0.000000}%
\pgfsetstrokecolor{currentstroke}%
\pgfsetdash{}{0pt}%
\pgfpathmoveto{\pgfqpoint{5.419302in}{3.131959in}}%
\pgfpathlineto{\pgfqpoint{5.433684in}{3.141670in}}%
\pgfpathlineto{\pgfqpoint{5.448085in}{3.151559in}}%
\pgfpathlineto{\pgfqpoint{5.462504in}{3.161624in}}%
\pgfpathlineto{\pgfqpoint{5.476941in}{3.171866in}}%
\pgfpathlineto{\pgfqpoint{5.484270in}{3.176024in}}%
\pgfpathlineto{\pgfqpoint{5.491594in}{3.180219in}}%
\pgfpathlineto{\pgfqpoint{5.498913in}{3.184458in}}%
\pgfpathlineto{\pgfqpoint{5.506227in}{3.188747in}}%
\pgfpathlineto{\pgfqpoint{5.491815in}{3.179047in}}%
\pgfpathlineto{\pgfqpoint{5.477421in}{3.169523in}}%
\pgfpathlineto{\pgfqpoint{5.463045in}{3.160175in}}%
\pgfpathlineto{\pgfqpoint{5.448686in}{3.151003in}}%
\pgfpathlineto{\pgfqpoint{5.441347in}{3.146162in}}%
\pgfpathlineto{\pgfqpoint{5.434004in}{3.141379in}}%
\pgfpathlineto{\pgfqpoint{5.426655in}{3.136646in}}%
\pgfpathlineto{\pgfqpoint{5.419302in}{3.131959in}}%
\pgfpathclose%
\pgfusepath{fill}%
\end{pgfscope}%
\begin{pgfscope}%
\pgfpathrectangle{\pgfqpoint{1.150000in}{0.150000in}}{\pgfqpoint{5.700000in}{5.700000in}}%
\pgfusepath{clip}%
\pgfsetbuttcap%
\pgfsetroundjoin%
\definecolor{currentfill}{rgb}{0.274128,0.199721,0.498911}%
\pgfsetfillcolor{currentfill}%
\pgfsetfillopacity{0.800000}%
\pgfsetlinewidth{0.000000pt}%
\definecolor{currentstroke}{rgb}{0.000000,0.000000,0.000000}%
\pgfsetstrokecolor{currentstroke}%
\pgfsetdash{}{0pt}%
\pgfpathmoveto{\pgfqpoint{2.662026in}{2.268795in}}%
\pgfpathlineto{\pgfqpoint{2.675799in}{2.251979in}}%
\pgfpathlineto{\pgfqpoint{2.689564in}{2.235448in}}%
\pgfpathlineto{\pgfqpoint{2.703321in}{2.219199in}}%
\pgfpathlineto{\pgfqpoint{2.717071in}{2.203230in}}%
\pgfpathlineto{\pgfqpoint{2.725605in}{2.207548in}}%
\pgfpathlineto{\pgfqpoint{2.734126in}{2.212043in}}%
\pgfpathlineto{\pgfqpoint{2.742636in}{2.216713in}}%
\pgfpathlineto{\pgfqpoint{2.751134in}{2.221553in}}%
\pgfpathlineto{\pgfqpoint{2.737417in}{2.237136in}}%
\pgfpathlineto{\pgfqpoint{2.723693in}{2.252999in}}%
\pgfpathlineto{\pgfqpoint{2.709962in}{2.269143in}}%
\pgfpathlineto{\pgfqpoint{2.696223in}{2.285571in}}%
\pgfpathlineto{\pgfqpoint{2.687692in}{2.281105in}}%
\pgfpathlineto{\pgfqpoint{2.679149in}{2.276818in}}%
\pgfpathlineto{\pgfqpoint{2.670594in}{2.272714in}}%
\pgfpathlineto{\pgfqpoint{2.662026in}{2.268795in}}%
\pgfpathclose%
\pgfusepath{fill}%
\end{pgfscope}%
\begin{pgfscope}%
\pgfpathrectangle{\pgfqpoint{1.150000in}{0.150000in}}{\pgfqpoint{5.700000in}{5.700000in}}%
\pgfusepath{clip}%
\pgfsetbuttcap%
\pgfsetroundjoin%
\definecolor{currentfill}{rgb}{0.278826,0.175490,0.483397}%
\pgfsetfillcolor{currentfill}%
\pgfsetfillopacity{0.800000}%
\pgfsetlinewidth{0.000000pt}%
\definecolor{currentstroke}{rgb}{0.000000,0.000000,0.000000}%
\pgfsetstrokecolor{currentstroke}%
\pgfsetdash{}{0pt}%
\pgfpathmoveto{\pgfqpoint{2.717071in}{2.203230in}}%
\pgfpathlineto{\pgfqpoint{2.730814in}{2.187539in}}%
\pgfpathlineto{\pgfqpoint{2.744550in}{2.172124in}}%
\pgfpathlineto{\pgfqpoint{2.758279in}{2.156982in}}%
\pgfpathlineto{\pgfqpoint{2.772002in}{2.142111in}}%
\pgfpathlineto{\pgfqpoint{2.780502in}{2.146826in}}%
\pgfpathlineto{\pgfqpoint{2.788992in}{2.151710in}}%
\pgfpathlineto{\pgfqpoint{2.797469in}{2.156760in}}%
\pgfpathlineto{\pgfqpoint{2.805936in}{2.161972in}}%
\pgfpathlineto{\pgfqpoint{2.792244in}{2.176459in}}%
\pgfpathlineto{\pgfqpoint{2.778547in}{2.191216in}}%
\pgfpathlineto{\pgfqpoint{2.764844in}{2.206247in}}%
\pgfpathlineto{\pgfqpoint{2.751134in}{2.221553in}}%
\pgfpathlineto{\pgfqpoint{2.742636in}{2.216713in}}%
\pgfpathlineto{\pgfqpoint{2.734126in}{2.212043in}}%
\pgfpathlineto{\pgfqpoint{2.725605in}{2.207548in}}%
\pgfpathlineto{\pgfqpoint{2.717071in}{2.203230in}}%
\pgfpathclose%
\pgfusepath{fill}%
\end{pgfscope}%
\begin{pgfscope}%
\pgfpathrectangle{\pgfqpoint{1.150000in}{0.150000in}}{\pgfqpoint{5.700000in}{5.700000in}}%
\pgfusepath{clip}%
\pgfsetbuttcap%
\pgfsetroundjoin%
\definecolor{currentfill}{rgb}{0.274952,0.037752,0.364543}%
\pgfsetfillcolor{currentfill}%
\pgfsetfillopacity{0.800000}%
\pgfsetlinewidth{0.000000pt}%
\definecolor{currentstroke}{rgb}{0.000000,0.000000,0.000000}%
\pgfsetstrokecolor{currentstroke}%
\pgfsetdash{}{0pt}%
\pgfpathmoveto{\pgfqpoint{3.220886in}{1.890888in}}%
\pgfpathlineto{\pgfqpoint{3.234485in}{1.883833in}}%
\pgfpathlineto{\pgfqpoint{3.248084in}{1.877000in}}%
\pgfpathlineto{\pgfqpoint{3.261685in}{1.870389in}}%
\pgfpathlineto{\pgfqpoint{3.275288in}{1.863998in}}%
\pgfpathlineto{\pgfqpoint{3.283515in}{1.872565in}}%
\pgfpathlineto{\pgfqpoint{3.291735in}{1.881211in}}%
\pgfpathlineto{\pgfqpoint{3.299947in}{1.889932in}}%
\pgfpathlineto{\pgfqpoint{3.308153in}{1.898727in}}%
\pgfpathlineto{\pgfqpoint{3.294568in}{1.904809in}}%
\pgfpathlineto{\pgfqpoint{3.280986in}{1.911111in}}%
\pgfpathlineto{\pgfqpoint{3.267404in}{1.917634in}}%
\pgfpathlineto{\pgfqpoint{3.253825in}{1.924381in}}%
\pgfpathlineto{\pgfqpoint{3.245601in}{1.915883in}}%
\pgfpathlineto{\pgfqpoint{3.237370in}{1.907467in}}%
\pgfpathlineto{\pgfqpoint{3.229132in}{1.899134in}}%
\pgfpathlineto{\pgfqpoint{3.220886in}{1.890888in}}%
\pgfpathclose%
\pgfusepath{fill}%
\end{pgfscope}%
\begin{pgfscope}%
\pgfpathrectangle{\pgfqpoint{1.150000in}{0.150000in}}{\pgfqpoint{5.700000in}{5.700000in}}%
\pgfusepath{clip}%
\pgfsetbuttcap%
\pgfsetroundjoin%
\definecolor{currentfill}{rgb}{0.248629,0.278775,0.534556}%
\pgfsetfillcolor{currentfill}%
\pgfsetfillopacity{0.800000}%
\pgfsetlinewidth{0.000000pt}%
\definecolor{currentstroke}{rgb}{0.000000,0.000000,0.000000}%
\pgfsetstrokecolor{currentstroke}%
\pgfsetdash{}{0pt}%
\pgfpathmoveto{\pgfqpoint{4.402757in}{2.404685in}}%
\pgfpathlineto{\pgfqpoint{4.416630in}{2.410404in}}%
\pgfpathlineto{\pgfqpoint{4.430516in}{2.416310in}}%
\pgfpathlineto{\pgfqpoint{4.444414in}{2.422402in}}%
\pgfpathlineto{\pgfqpoint{4.458325in}{2.428680in}}%
\pgfpathlineto{\pgfqpoint{4.466135in}{2.438175in}}%
\pgfpathlineto{\pgfqpoint{4.473939in}{2.447603in}}%
\pgfpathlineto{\pgfqpoint{4.481737in}{2.456964in}}%
\pgfpathlineto{\pgfqpoint{4.489529in}{2.466260in}}%
\pgfpathlineto{\pgfqpoint{4.475624in}{2.460090in}}%
\pgfpathlineto{\pgfqpoint{4.461733in}{2.454105in}}%
\pgfpathlineto{\pgfqpoint{4.447854in}{2.448307in}}%
\pgfpathlineto{\pgfqpoint{4.433988in}{2.442694in}}%
\pgfpathlineto{\pgfqpoint{4.426189in}{2.433279in}}%
\pgfpathlineto{\pgfqpoint{4.418384in}{2.423806in}}%
\pgfpathlineto{\pgfqpoint{4.410573in}{2.414276in}}%
\pgfpathlineto{\pgfqpoint{4.402757in}{2.404685in}}%
\pgfpathclose%
\pgfusepath{fill}%
\end{pgfscope}%
\begin{pgfscope}%
\pgfpathrectangle{\pgfqpoint{1.150000in}{0.150000in}}{\pgfqpoint{5.700000in}{5.700000in}}%
\pgfusepath{clip}%
\pgfsetbuttcap%
\pgfsetroundjoin%
\definecolor{currentfill}{rgb}{0.266580,0.228262,0.514349}%
\pgfsetfillcolor{currentfill}%
\pgfsetfillopacity{0.800000}%
\pgfsetlinewidth{0.000000pt}%
\definecolor{currentstroke}{rgb}{0.000000,0.000000,0.000000}%
\pgfsetstrokecolor{currentstroke}%
\pgfsetdash{}{0pt}%
\pgfpathmoveto{\pgfqpoint{2.606847in}{2.338954in}}%
\pgfpathlineto{\pgfqpoint{2.620655in}{2.320975in}}%
\pgfpathlineto{\pgfqpoint{2.634454in}{2.303290in}}%
\pgfpathlineto{\pgfqpoint{2.648244in}{2.285898in}}%
\pgfpathlineto{\pgfqpoint{2.662026in}{2.268795in}}%
\pgfpathlineto{\pgfqpoint{2.670594in}{2.272714in}}%
\pgfpathlineto{\pgfqpoint{2.679149in}{2.276818in}}%
\pgfpathlineto{\pgfqpoint{2.687692in}{2.281105in}}%
\pgfpathlineto{\pgfqpoint{2.696223in}{2.285571in}}%
\pgfpathlineto{\pgfqpoint{2.682476in}{2.302286in}}%
\pgfpathlineto{\pgfqpoint{2.668722in}{2.319290in}}%
\pgfpathlineto{\pgfqpoint{2.654958in}{2.336584in}}%
\pgfpathlineto{\pgfqpoint{2.641186in}{2.354173in}}%
\pgfpathlineto{\pgfqpoint{2.632621in}{2.350084in}}%
\pgfpathlineto{\pgfqpoint{2.624043in}{2.346182in}}%
\pgfpathlineto{\pgfqpoint{2.615452in}{2.342471in}}%
\pgfpathlineto{\pgfqpoint{2.606847in}{2.338954in}}%
\pgfpathclose%
\pgfusepath{fill}%
\end{pgfscope}%
\begin{pgfscope}%
\pgfpathrectangle{\pgfqpoint{1.150000in}{0.150000in}}{\pgfqpoint{5.700000in}{5.700000in}}%
\pgfusepath{clip}%
\pgfsetbuttcap%
\pgfsetroundjoin%
\definecolor{currentfill}{rgb}{0.190631,0.407061,0.556089}%
\pgfsetfillcolor{currentfill}%
\pgfsetfillopacity{0.800000}%
\pgfsetlinewidth{0.000000pt}%
\definecolor{currentstroke}{rgb}{0.000000,0.000000,0.000000}%
\pgfsetstrokecolor{currentstroke}%
\pgfsetdash{}{0pt}%
\pgfpathmoveto{\pgfqpoint{4.867781in}{2.749214in}}%
\pgfpathlineto{\pgfqpoint{4.881881in}{2.757475in}}%
\pgfpathlineto{\pgfqpoint{4.895997in}{2.765917in}}%
\pgfpathlineto{\pgfqpoint{4.910129in}{2.774541in}}%
\pgfpathlineto{\pgfqpoint{4.924277in}{2.783346in}}%
\pgfpathlineto{\pgfqpoint{4.931891in}{2.790351in}}%
\pgfpathlineto{\pgfqpoint{4.939498in}{2.797309in}}%
\pgfpathlineto{\pgfqpoint{4.947099in}{2.804222in}}%
\pgfpathlineto{\pgfqpoint{4.954694in}{2.811094in}}%
\pgfpathlineto{\pgfqpoint{4.940559in}{2.802596in}}%
\pgfpathlineto{\pgfqpoint{4.926439in}{2.794279in}}%
\pgfpathlineto{\pgfqpoint{4.912336in}{2.786143in}}%
\pgfpathlineto{\pgfqpoint{4.898247in}{2.778187in}}%
\pgfpathlineto{\pgfqpoint{4.890640in}{2.770997in}}%
\pgfpathlineto{\pgfqpoint{4.883027in}{2.763773in}}%
\pgfpathlineto{\pgfqpoint{4.875407in}{2.756514in}}%
\pgfpathlineto{\pgfqpoint{4.867781in}{2.749214in}}%
\pgfpathclose%
\pgfusepath{fill}%
\end{pgfscope}%
\begin{pgfscope}%
\pgfpathrectangle{\pgfqpoint{1.150000in}{0.150000in}}{\pgfqpoint{5.700000in}{5.700000in}}%
\pgfusepath{clip}%
\pgfsetbuttcap%
\pgfsetroundjoin%
\definecolor{currentfill}{rgb}{0.282290,0.145912,0.461510}%
\pgfsetfillcolor{currentfill}%
\pgfsetfillopacity{0.800000}%
\pgfsetlinewidth{0.000000pt}%
\definecolor{currentstroke}{rgb}{0.000000,0.000000,0.000000}%
\pgfsetstrokecolor{currentstroke}%
\pgfsetdash{}{0pt}%
\pgfpathmoveto{\pgfqpoint{2.772002in}{2.142111in}}%
\pgfpathlineto{\pgfqpoint{2.785718in}{2.127510in}}%
\pgfpathlineto{\pgfqpoint{2.799429in}{2.113176in}}%
\pgfpathlineto{\pgfqpoint{2.813134in}{2.099108in}}%
\pgfpathlineto{\pgfqpoint{2.826834in}{2.085302in}}%
\pgfpathlineto{\pgfqpoint{2.835304in}{2.090411in}}%
\pgfpathlineto{\pgfqpoint{2.843762in}{2.095682in}}%
\pgfpathlineto{\pgfqpoint{2.852209in}{2.101109in}}%
\pgfpathlineto{\pgfqpoint{2.860645in}{2.106691in}}%
\pgfpathlineto{\pgfqpoint{2.846976in}{2.120115in}}%
\pgfpathlineto{\pgfqpoint{2.833301in}{2.133802in}}%
\pgfpathlineto{\pgfqpoint{2.819621in}{2.147753in}}%
\pgfpathlineto{\pgfqpoint{2.805936in}{2.161972in}}%
\pgfpathlineto{\pgfqpoint{2.797469in}{2.156760in}}%
\pgfpathlineto{\pgfqpoint{2.788992in}{2.151710in}}%
\pgfpathlineto{\pgfqpoint{2.780502in}{2.146826in}}%
\pgfpathlineto{\pgfqpoint{2.772002in}{2.142111in}}%
\pgfpathclose%
\pgfusepath{fill}%
\end{pgfscope}%
\begin{pgfscope}%
\pgfpathrectangle{\pgfqpoint{1.150000in}{0.150000in}}{\pgfqpoint{5.700000in}{5.700000in}}%
\pgfusepath{clip}%
\pgfsetbuttcap%
\pgfsetroundjoin%
\definecolor{currentfill}{rgb}{0.132444,0.552216,0.553018}%
\pgfsetfillcolor{currentfill}%
\pgfsetfillopacity{0.800000}%
\pgfsetlinewidth{0.000000pt}%
\definecolor{currentstroke}{rgb}{0.000000,0.000000,0.000000}%
\pgfsetstrokecolor{currentstroke}%
\pgfsetdash{}{0pt}%
\pgfpathmoveto{\pgfqpoint{5.506227in}{3.188747in}}%
\pgfpathlineto{\pgfqpoint{5.520658in}{3.198623in}}%
\pgfpathlineto{\pgfqpoint{5.535107in}{3.208675in}}%
\pgfpathlineto{\pgfqpoint{5.549574in}{3.218904in}}%
\pgfpathlineto{\pgfqpoint{5.564061in}{3.229309in}}%
\pgfpathlineto{\pgfqpoint{5.571344in}{3.233089in}}%
\pgfpathlineto{\pgfqpoint{5.578622in}{3.236925in}}%
\pgfpathlineto{\pgfqpoint{5.585895in}{3.240821in}}%
\pgfpathlineto{\pgfqpoint{5.593164in}{3.244784in}}%
\pgfpathlineto{\pgfqpoint{5.578705in}{3.234955in}}%
\pgfpathlineto{\pgfqpoint{5.564264in}{3.225302in}}%
\pgfpathlineto{\pgfqpoint{5.549842in}{3.215824in}}%
\pgfpathlineto{\pgfqpoint{5.535438in}{3.206521in}}%
\pgfpathlineto{\pgfqpoint{5.528141in}{3.201972in}}%
\pgfpathlineto{\pgfqpoint{5.520841in}{3.197498in}}%
\pgfpathlineto{\pgfqpoint{5.513536in}{3.193091in}}%
\pgfpathlineto{\pgfqpoint{5.506227in}{3.188747in}}%
\pgfpathclose%
\pgfusepath{fill}%
\end{pgfscope}%
\begin{pgfscope}%
\pgfpathrectangle{\pgfqpoint{1.150000in}{0.150000in}}{\pgfqpoint{5.700000in}{5.700000in}}%
\pgfusepath{clip}%
\pgfsetbuttcap%
\pgfsetroundjoin%
\definecolor{currentfill}{rgb}{0.273809,0.031497,0.358853}%
\pgfsetfillcolor{currentfill}%
\pgfsetfillopacity{0.800000}%
\pgfsetlinewidth{0.000000pt}%
\definecolor{currentstroke}{rgb}{0.000000,0.000000,0.000000}%
\pgfsetstrokecolor{currentstroke}%
\pgfsetdash{}{0pt}%
\pgfpathmoveto{\pgfqpoint{3.362511in}{1.876581in}}%
\pgfpathlineto{\pgfqpoint{3.376107in}{1.871585in}}%
\pgfpathlineto{\pgfqpoint{3.389705in}{1.866802in}}%
\pgfpathlineto{\pgfqpoint{3.403307in}{1.862233in}}%
\pgfpathlineto{\pgfqpoint{3.416911in}{1.857876in}}%
\pgfpathlineto{\pgfqpoint{3.425078in}{1.867318in}}%
\pgfpathlineto{\pgfqpoint{3.433238in}{1.876812in}}%
\pgfpathlineto{\pgfqpoint{3.441392in}{1.886356in}}%
\pgfpathlineto{\pgfqpoint{3.449539in}{1.895947in}}%
\pgfpathlineto{\pgfqpoint{3.435949in}{1.900028in}}%
\pgfpathlineto{\pgfqpoint{3.422363in}{1.904321in}}%
\pgfpathlineto{\pgfqpoint{3.408779in}{1.908827in}}%
\pgfpathlineto{\pgfqpoint{3.395199in}{1.913546in}}%
\pgfpathlineto{\pgfqpoint{3.387037in}{1.904220in}}%
\pgfpathlineto{\pgfqpoint{3.378868in}{1.894949in}}%
\pgfpathlineto{\pgfqpoint{3.370693in}{1.885735in}}%
\pgfpathlineto{\pgfqpoint{3.362511in}{1.876581in}}%
\pgfpathclose%
\pgfusepath{fill}%
\end{pgfscope}%
\begin{pgfscope}%
\pgfpathrectangle{\pgfqpoint{1.150000in}{0.150000in}}{\pgfqpoint{5.700000in}{5.700000in}}%
\pgfusepath{clip}%
\pgfsetbuttcap%
\pgfsetroundjoin%
\definecolor{currentfill}{rgb}{0.282327,0.094955,0.417331}%
\pgfsetfillcolor{currentfill}%
\pgfsetfillopacity{0.800000}%
\pgfsetlinewidth{0.000000pt}%
\definecolor{currentstroke}{rgb}{0.000000,0.000000,0.000000}%
\pgfsetstrokecolor{currentstroke}%
\pgfsetdash{}{0pt}%
\pgfpathmoveto{\pgfqpoint{3.764359in}{1.987059in}}%
\pgfpathlineto{\pgfqpoint{3.778006in}{1.987096in}}%
\pgfpathlineto{\pgfqpoint{3.791661in}{1.987332in}}%
\pgfpathlineto{\pgfqpoint{3.805323in}{1.987766in}}%
\pgfpathlineto{\pgfqpoint{3.818993in}{1.988397in}}%
\pgfpathlineto{\pgfqpoint{3.827015in}{1.999234in}}%
\pgfpathlineto{\pgfqpoint{3.835031in}{2.010054in}}%
\pgfpathlineto{\pgfqpoint{3.843043in}{2.020856in}}%
\pgfpathlineto{\pgfqpoint{3.851049in}{2.031639in}}%
\pgfpathlineto{\pgfqpoint{3.837387in}{2.030858in}}%
\pgfpathlineto{\pgfqpoint{3.823733in}{2.030274in}}%
\pgfpathlineto{\pgfqpoint{3.810087in}{2.029888in}}%
\pgfpathlineto{\pgfqpoint{3.796447in}{2.029700in}}%
\pgfpathlineto{\pgfqpoint{3.788433in}{2.019055in}}%
\pgfpathlineto{\pgfqpoint{3.780413in}{2.008399in}}%
\pgfpathlineto{\pgfqpoint{3.772388in}{1.997733in}}%
\pgfpathlineto{\pgfqpoint{3.764359in}{1.987059in}}%
\pgfpathclose%
\pgfusepath{fill}%
\end{pgfscope}%
\begin{pgfscope}%
\pgfpathrectangle{\pgfqpoint{1.150000in}{0.150000in}}{\pgfqpoint{5.700000in}{5.700000in}}%
\pgfusepath{clip}%
\pgfsetbuttcap%
\pgfsetroundjoin%
\definecolor{currentfill}{rgb}{0.283229,0.120777,0.440584}%
\pgfsetfillcolor{currentfill}%
\pgfsetfillopacity{0.800000}%
\pgfsetlinewidth{0.000000pt}%
\definecolor{currentstroke}{rgb}{0.000000,0.000000,0.000000}%
\pgfsetstrokecolor{currentstroke}%
\pgfsetdash{}{0pt}%
\pgfpathmoveto{\pgfqpoint{3.851049in}{2.031639in}}%
\pgfpathlineto{\pgfqpoint{3.864719in}{2.032617in}}%
\pgfpathlineto{\pgfqpoint{3.878397in}{2.033790in}}%
\pgfpathlineto{\pgfqpoint{3.892083in}{2.035160in}}%
\pgfpathlineto{\pgfqpoint{3.905777in}{2.036724in}}%
\pgfpathlineto{\pgfqpoint{3.913771in}{2.047617in}}%
\pgfpathlineto{\pgfqpoint{3.921760in}{2.058482in}}%
\pgfpathlineto{\pgfqpoint{3.929744in}{2.069318in}}%
\pgfpathlineto{\pgfqpoint{3.937723in}{2.080123in}}%
\pgfpathlineto{\pgfqpoint{3.924036in}{2.078440in}}%
\pgfpathlineto{\pgfqpoint{3.910357in}{2.076952in}}%
\pgfpathlineto{\pgfqpoint{3.896687in}{2.075660in}}%
\pgfpathlineto{\pgfqpoint{3.883024in}{2.074564in}}%
\pgfpathlineto{\pgfqpoint{3.875038in}{2.063865in}}%
\pgfpathlineto{\pgfqpoint{3.867047in}{2.053145in}}%
\pgfpathlineto{\pgfqpoint{3.859051in}{2.042402in}}%
\pgfpathlineto{\pgfqpoint{3.851049in}{2.031639in}}%
\pgfpathclose%
\pgfusepath{fill}%
\end{pgfscope}%
\begin{pgfscope}%
\pgfpathrectangle{\pgfqpoint{1.150000in}{0.150000in}}{\pgfqpoint{5.700000in}{5.700000in}}%
\pgfusepath{clip}%
\pgfsetbuttcap%
\pgfsetroundjoin%
\definecolor{currentfill}{rgb}{0.255645,0.260703,0.528312}%
\pgfsetfillcolor{currentfill}%
\pgfsetfillopacity{0.800000}%
\pgfsetlinewidth{0.000000pt}%
\definecolor{currentstroke}{rgb}{0.000000,0.000000,0.000000}%
\pgfsetstrokecolor{currentstroke}%
\pgfsetdash{}{0pt}%
\pgfpathmoveto{\pgfqpoint{2.551518in}{2.413866in}}%
\pgfpathlineto{\pgfqpoint{2.565366in}{2.394684in}}%
\pgfpathlineto{\pgfqpoint{2.579203in}{2.375806in}}%
\pgfpathlineto{\pgfqpoint{2.593030in}{2.357230in}}%
\pgfpathlineto{\pgfqpoint{2.606847in}{2.338954in}}%
\pgfpathlineto{\pgfqpoint{2.615452in}{2.342471in}}%
\pgfpathlineto{\pgfqpoint{2.624043in}{2.346182in}}%
\pgfpathlineto{\pgfqpoint{2.632621in}{2.350084in}}%
\pgfpathlineto{\pgfqpoint{2.641186in}{2.354173in}}%
\pgfpathlineto{\pgfqpoint{2.627405in}{2.372058in}}%
\pgfpathlineto{\pgfqpoint{2.613615in}{2.390242in}}%
\pgfpathlineto{\pgfqpoint{2.599815in}{2.408727in}}%
\pgfpathlineto{\pgfqpoint{2.586005in}{2.427517in}}%
\pgfpathlineto{\pgfqpoint{2.577403in}{2.423807in}}%
\pgfpathlineto{\pgfqpoint{2.568789in}{2.420293in}}%
\pgfpathlineto{\pgfqpoint{2.560161in}{2.416978in}}%
\pgfpathlineto{\pgfqpoint{2.551518in}{2.413866in}}%
\pgfpathclose%
\pgfusepath{fill}%
\end{pgfscope}%
\begin{pgfscope}%
\pgfpathrectangle{\pgfqpoint{1.150000in}{0.150000in}}{\pgfqpoint{5.700000in}{5.700000in}}%
\pgfusepath{clip}%
\pgfsetbuttcap%
\pgfsetroundjoin%
\definecolor{currentfill}{rgb}{0.277941,0.056324,0.381191}%
\pgfsetfillcolor{currentfill}%
\pgfsetfillopacity{0.800000}%
\pgfsetlinewidth{0.000000pt}%
\definecolor{currentstroke}{rgb}{0.000000,0.000000,0.000000}%
\pgfsetstrokecolor{currentstroke}%
\pgfsetdash{}{0pt}%
\pgfpathmoveto{\pgfqpoint{3.078874in}{1.926172in}}%
\pgfpathlineto{\pgfqpoint{3.092495in}{1.916948in}}%
\pgfpathlineto{\pgfqpoint{3.106114in}{1.907957in}}%
\pgfpathlineto{\pgfqpoint{3.119733in}{1.899197in}}%
\pgfpathlineto{\pgfqpoint{3.133352in}{1.890668in}}%
\pgfpathlineto{\pgfqpoint{3.141650in}{1.898184in}}%
\pgfpathlineto{\pgfqpoint{3.149941in}{1.905807in}}%
\pgfpathlineto{\pgfqpoint{3.158223in}{1.913534in}}%
\pgfpathlineto{\pgfqpoint{3.166497in}{1.921361in}}%
\pgfpathlineto{\pgfqpoint{3.152900in}{1.929548in}}%
\pgfpathlineto{\pgfqpoint{3.139304in}{1.937965in}}%
\pgfpathlineto{\pgfqpoint{3.125706in}{1.946614in}}%
\pgfpathlineto{\pgfqpoint{3.112109in}{1.955496in}}%
\pgfpathlineto{\pgfqpoint{3.103813in}{1.947999in}}%
\pgfpathlineto{\pgfqpoint{3.095509in}{1.940611in}}%
\pgfpathlineto{\pgfqpoint{3.087196in}{1.933334in}}%
\pgfpathlineto{\pgfqpoint{3.078874in}{1.926172in}}%
\pgfpathclose%
\pgfusepath{fill}%
\end{pgfscope}%
\begin{pgfscope}%
\pgfpathrectangle{\pgfqpoint{1.150000in}{0.150000in}}{\pgfqpoint{5.700000in}{5.700000in}}%
\pgfusepath{clip}%
\pgfsetbuttcap%
\pgfsetroundjoin%
\definecolor{currentfill}{rgb}{0.282623,0.140926,0.457517}%
\pgfsetfillcolor{currentfill}%
\pgfsetfillopacity{0.800000}%
\pgfsetlinewidth{0.000000pt}%
\definecolor{currentstroke}{rgb}{0.000000,0.000000,0.000000}%
\pgfsetstrokecolor{currentstroke}%
\pgfsetdash{}{0pt}%
\pgfpathmoveto{\pgfqpoint{3.937723in}{2.080123in}}%
\pgfpathlineto{\pgfqpoint{3.951419in}{2.082000in}}%
\pgfpathlineto{\pgfqpoint{3.965124in}{2.084071in}}%
\pgfpathlineto{\pgfqpoint{3.978838in}{2.086336in}}%
\pgfpathlineto{\pgfqpoint{3.992562in}{2.088794in}}%
\pgfpathlineto{\pgfqpoint{4.000529in}{2.099667in}}%
\pgfpathlineto{\pgfqpoint{4.008491in}{2.110500in}}%
\pgfpathlineto{\pgfqpoint{4.016448in}{2.121294in}}%
\pgfpathlineto{\pgfqpoint{4.024401in}{2.132047in}}%
\pgfpathlineto{\pgfqpoint{4.010684in}{2.129502in}}%
\pgfpathlineto{\pgfqpoint{3.996976in}{2.127151in}}%
\pgfpathlineto{\pgfqpoint{3.983278in}{2.124993in}}%
\pgfpathlineto{\pgfqpoint{3.969589in}{2.123029in}}%
\pgfpathlineto{\pgfqpoint{3.961630in}{2.112350in}}%
\pgfpathlineto{\pgfqpoint{3.953666in}{2.101639in}}%
\pgfpathlineto{\pgfqpoint{3.945697in}{2.090897in}}%
\pgfpathlineto{\pgfqpoint{3.937723in}{2.080123in}}%
\pgfpathclose%
\pgfusepath{fill}%
\end{pgfscope}%
\begin{pgfscope}%
\pgfpathrectangle{\pgfqpoint{1.150000in}{0.150000in}}{\pgfqpoint{5.700000in}{5.700000in}}%
\pgfusepath{clip}%
\pgfsetbuttcap%
\pgfsetroundjoin%
\definecolor{currentfill}{rgb}{0.280894,0.078907,0.402329}%
\pgfsetfillcolor{currentfill}%
\pgfsetfillopacity{0.800000}%
\pgfsetlinewidth{0.000000pt}%
\definecolor{currentstroke}{rgb}{0.000000,0.000000,0.000000}%
\pgfsetstrokecolor{currentstroke}%
\pgfsetdash{}{0pt}%
\pgfpathmoveto{\pgfqpoint{3.677628in}{1.946868in}}%
\pgfpathlineto{\pgfqpoint{3.691258in}{1.945925in}}%
\pgfpathlineto{\pgfqpoint{3.704895in}{1.945182in}}%
\pgfpathlineto{\pgfqpoint{3.718538in}{1.944640in}}%
\pgfpathlineto{\pgfqpoint{3.732187in}{1.944297in}}%
\pgfpathlineto{\pgfqpoint{3.740238in}{1.954994in}}%
\pgfpathlineto{\pgfqpoint{3.748283in}{1.965688in}}%
\pgfpathlineto{\pgfqpoint{3.756323in}{1.976376in}}%
\pgfpathlineto{\pgfqpoint{3.764359in}{1.987059in}}%
\pgfpathlineto{\pgfqpoint{3.750718in}{1.987220in}}%
\pgfpathlineto{\pgfqpoint{3.737084in}{1.987581in}}%
\pgfpathlineto{\pgfqpoint{3.723457in}{1.988142in}}%
\pgfpathlineto{\pgfqpoint{3.709837in}{1.988904in}}%
\pgfpathlineto{\pgfqpoint{3.701792in}{1.978391in}}%
\pgfpathlineto{\pgfqpoint{3.693743in}{1.967880in}}%
\pgfpathlineto{\pgfqpoint{3.685688in}{1.957372in}}%
\pgfpathlineto{\pgfqpoint{3.677628in}{1.946868in}}%
\pgfpathclose%
\pgfusepath{fill}%
\end{pgfscope}%
\begin{pgfscope}%
\pgfpathrectangle{\pgfqpoint{1.150000in}{0.150000in}}{\pgfqpoint{5.700000in}{5.700000in}}%
\pgfusepath{clip}%
\pgfsetbuttcap%
\pgfsetroundjoin%
\definecolor{currentfill}{rgb}{0.126453,0.570633,0.549841}%
\pgfsetfillcolor{currentfill}%
\pgfsetfillopacity{0.800000}%
\pgfsetlinewidth{0.000000pt}%
\definecolor{currentstroke}{rgb}{0.000000,0.000000,0.000000}%
\pgfsetstrokecolor{currentstroke}%
\pgfsetdash{}{0pt}%
\pgfpathmoveto{\pgfqpoint{5.593164in}{3.244784in}}%
\pgfpathlineto{\pgfqpoint{5.607642in}{3.254788in}}%
\pgfpathlineto{\pgfqpoint{5.622139in}{3.264967in}}%
\pgfpathlineto{\pgfqpoint{5.636654in}{3.275323in}}%
\pgfpathlineto{\pgfqpoint{5.651189in}{3.285854in}}%
\pgfpathlineto{\pgfqpoint{5.658425in}{3.289293in}}%
\pgfpathlineto{\pgfqpoint{5.665657in}{3.292804in}}%
\pgfpathlineto{\pgfqpoint{5.672885in}{3.296395in}}%
\pgfpathlineto{\pgfqpoint{5.680109in}{3.300072in}}%
\pgfpathlineto{\pgfqpoint{5.665604in}{3.290151in}}%
\pgfpathlineto{\pgfqpoint{5.651118in}{3.280404in}}%
\pgfpathlineto{\pgfqpoint{5.636650in}{3.270832in}}%
\pgfpathlineto{\pgfqpoint{5.622201in}{3.261435in}}%
\pgfpathlineto{\pgfqpoint{5.614947in}{3.257139in}}%
\pgfpathlineto{\pgfqpoint{5.607690in}{3.252936in}}%
\pgfpathlineto{\pgfqpoint{5.600429in}{3.248820in}}%
\pgfpathlineto{\pgfqpoint{5.593164in}{3.244784in}}%
\pgfpathclose%
\pgfusepath{fill}%
\end{pgfscope}%
\begin{pgfscope}%
\pgfpathrectangle{\pgfqpoint{1.150000in}{0.150000in}}{\pgfqpoint{5.700000in}{5.700000in}}%
\pgfusepath{clip}%
\pgfsetbuttcap%
\pgfsetroundjoin%
\definecolor{currentfill}{rgb}{0.283187,0.125848,0.444960}%
\pgfsetfillcolor{currentfill}%
\pgfsetfillopacity{0.800000}%
\pgfsetlinewidth{0.000000pt}%
\definecolor{currentstroke}{rgb}{0.000000,0.000000,0.000000}%
\pgfsetstrokecolor{currentstroke}%
\pgfsetdash{}{0pt}%
\pgfpathmoveto{\pgfqpoint{2.826834in}{2.085302in}}%
\pgfpathlineto{\pgfqpoint{2.840529in}{2.071758in}}%
\pgfpathlineto{\pgfqpoint{2.854219in}{2.058473in}}%
\pgfpathlineto{\pgfqpoint{2.867904in}{2.045446in}}%
\pgfpathlineto{\pgfqpoint{2.881585in}{2.032674in}}%
\pgfpathlineto{\pgfqpoint{2.890024in}{2.038176in}}%
\pgfpathlineto{\pgfqpoint{2.898453in}{2.043831in}}%
\pgfpathlineto{\pgfqpoint{2.906871in}{2.049634in}}%
\pgfpathlineto{\pgfqpoint{2.915279in}{2.055584in}}%
\pgfpathlineto{\pgfqpoint{2.901627in}{2.067976in}}%
\pgfpathlineto{\pgfqpoint{2.887971in}{2.080623in}}%
\pgfpathlineto{\pgfqpoint{2.874310in}{2.093527in}}%
\pgfpathlineto{\pgfqpoint{2.860645in}{2.106691in}}%
\pgfpathlineto{\pgfqpoint{2.852209in}{2.101109in}}%
\pgfpathlineto{\pgfqpoint{2.843762in}{2.095682in}}%
\pgfpathlineto{\pgfqpoint{2.835304in}{2.090411in}}%
\pgfpathlineto{\pgfqpoint{2.826834in}{2.085302in}}%
\pgfpathclose%
\pgfusepath{fill}%
\end{pgfscope}%
\begin{pgfscope}%
\pgfpathrectangle{\pgfqpoint{1.150000in}{0.150000in}}{\pgfqpoint{5.700000in}{5.700000in}}%
\pgfusepath{clip}%
\pgfsetbuttcap%
\pgfsetroundjoin%
\definecolor{currentfill}{rgb}{0.237441,0.305202,0.541921}%
\pgfsetfillcolor{currentfill}%
\pgfsetfillopacity{0.800000}%
\pgfsetlinewidth{0.000000pt}%
\definecolor{currentstroke}{rgb}{0.000000,0.000000,0.000000}%
\pgfsetstrokecolor{currentstroke}%
\pgfsetdash{}{0pt}%
\pgfpathmoveto{\pgfqpoint{4.489529in}{2.466260in}}%
\pgfpathlineto{\pgfqpoint{4.503447in}{2.472617in}}%
\pgfpathlineto{\pgfqpoint{4.517377in}{2.479158in}}%
\pgfpathlineto{\pgfqpoint{4.531322in}{2.485885in}}%
\pgfpathlineto{\pgfqpoint{4.545279in}{2.492797in}}%
\pgfpathlineto{\pgfqpoint{4.553059in}{2.501903in}}%
\pgfpathlineto{\pgfqpoint{4.560832in}{2.510939in}}%
\pgfpathlineto{\pgfqpoint{4.568599in}{2.519908in}}%
\pgfpathlineto{\pgfqpoint{4.576360in}{2.528811in}}%
\pgfpathlineto{\pgfqpoint{4.562410in}{2.522040in}}%
\pgfpathlineto{\pgfqpoint{4.548473in}{2.515453in}}%
\pgfpathlineto{\pgfqpoint{4.534550in}{2.509051in}}%
\pgfpathlineto{\pgfqpoint{4.520639in}{2.502835in}}%
\pgfpathlineto{\pgfqpoint{4.512871in}{2.493779in}}%
\pgfpathlineto{\pgfqpoint{4.505096in}{2.484666in}}%
\pgfpathlineto{\pgfqpoint{4.497315in}{2.475494in}}%
\pgfpathlineto{\pgfqpoint{4.489529in}{2.466260in}}%
\pgfpathclose%
\pgfusepath{fill}%
\end{pgfscope}%
\begin{pgfscope}%
\pgfpathrectangle{\pgfqpoint{1.150000in}{0.150000in}}{\pgfqpoint{5.700000in}{5.700000in}}%
\pgfusepath{clip}%
\pgfsetbuttcap%
\pgfsetroundjoin%
\definecolor{currentfill}{rgb}{0.180629,0.429975,0.557282}%
\pgfsetfillcolor{currentfill}%
\pgfsetfillopacity{0.800000}%
\pgfsetlinewidth{0.000000pt}%
\definecolor{currentstroke}{rgb}{0.000000,0.000000,0.000000}%
\pgfsetstrokecolor{currentstroke}%
\pgfsetdash{}{0pt}%
\pgfpathmoveto{\pgfqpoint{4.954694in}{2.811094in}}%
\pgfpathlineto{\pgfqpoint{4.968844in}{2.819774in}}%
\pgfpathlineto{\pgfqpoint{4.983011in}{2.828634in}}%
\pgfpathlineto{\pgfqpoint{4.997194in}{2.837674in}}%
\pgfpathlineto{\pgfqpoint{5.011394in}{2.846896in}}%
\pgfpathlineto{\pgfqpoint{5.018968in}{2.853404in}}%
\pgfpathlineto{\pgfqpoint{5.026536in}{2.859871in}}%
\pgfpathlineto{\pgfqpoint{5.034097in}{2.866301in}}%
\pgfpathlineto{\pgfqpoint{5.041652in}{2.872698in}}%
\pgfpathlineto{\pgfqpoint{5.027466in}{2.863817in}}%
\pgfpathlineto{\pgfqpoint{5.013297in}{2.855117in}}%
\pgfpathlineto{\pgfqpoint{4.999145in}{2.846596in}}%
\pgfpathlineto{\pgfqpoint{4.985008in}{2.838256in}}%
\pgfpathlineto{\pgfqpoint{4.977439in}{2.831507in}}%
\pgfpathlineto{\pgfqpoint{4.969863in}{2.824733in}}%
\pgfpathlineto{\pgfqpoint{4.962282in}{2.817930in}}%
\pgfpathlineto{\pgfqpoint{4.954694in}{2.811094in}}%
\pgfpathclose%
\pgfusepath{fill}%
\end{pgfscope}%
\begin{pgfscope}%
\pgfpathrectangle{\pgfqpoint{1.150000in}{0.150000in}}{\pgfqpoint{5.700000in}{5.700000in}}%
\pgfusepath{clip}%
\pgfsetbuttcap%
\pgfsetroundjoin%
\definecolor{currentfill}{rgb}{0.280255,0.165693,0.476498}%
\pgfsetfillcolor{currentfill}%
\pgfsetfillopacity{0.800000}%
\pgfsetlinewidth{0.000000pt}%
\definecolor{currentstroke}{rgb}{0.000000,0.000000,0.000000}%
\pgfsetstrokecolor{currentstroke}%
\pgfsetdash{}{0pt}%
\pgfpathmoveto{\pgfqpoint{4.024401in}{2.132047in}}%
\pgfpathlineto{\pgfqpoint{4.038127in}{2.134784in}}%
\pgfpathlineto{\pgfqpoint{4.051863in}{2.137714in}}%
\pgfpathlineto{\pgfqpoint{4.065609in}{2.140835in}}%
\pgfpathlineto{\pgfqpoint{4.079364in}{2.144148in}}%
\pgfpathlineto{\pgfqpoint{4.087305in}{2.154927in}}%
\pgfpathlineto{\pgfqpoint{4.095241in}{2.165658in}}%
\pgfpathlineto{\pgfqpoint{4.103172in}{2.176340in}}%
\pgfpathlineto{\pgfqpoint{4.111098in}{2.186972in}}%
\pgfpathlineto{\pgfqpoint{4.097348in}{2.183605in}}%
\pgfpathlineto{\pgfqpoint{4.083609in}{2.180429in}}%
\pgfpathlineto{\pgfqpoint{4.069879in}{2.177444in}}%
\pgfpathlineto{\pgfqpoint{4.056159in}{2.174652in}}%
\pgfpathlineto{\pgfqpoint{4.048227in}{2.164062in}}%
\pgfpathlineto{\pgfqpoint{4.040290in}{2.153431in}}%
\pgfpathlineto{\pgfqpoint{4.032348in}{2.142759in}}%
\pgfpathlineto{\pgfqpoint{4.024401in}{2.132047in}}%
\pgfpathclose%
\pgfusepath{fill}%
\end{pgfscope}%
\begin{pgfscope}%
\pgfpathrectangle{\pgfqpoint{1.150000in}{0.150000in}}{\pgfqpoint{5.700000in}{5.700000in}}%
\pgfusepath{clip}%
\pgfsetbuttcap%
\pgfsetroundjoin%
\definecolor{currentfill}{rgb}{0.278791,0.062145,0.386592}%
\pgfsetfillcolor{currentfill}%
\pgfsetfillopacity{0.800000}%
\pgfsetlinewidth{0.000000pt}%
\definecolor{currentstroke}{rgb}{0.000000,0.000000,0.000000}%
\pgfsetstrokecolor{currentstroke}%
\pgfsetdash{}{0pt}%
\pgfpathmoveto{\pgfqpoint{3.590831in}{1.911579in}}%
\pgfpathlineto{\pgfqpoint{3.604448in}{1.909612in}}%
\pgfpathlineto{\pgfqpoint{3.618071in}{1.907848in}}%
\pgfpathlineto{\pgfqpoint{3.631700in}{1.906288in}}%
\pgfpathlineto{\pgfqpoint{3.645334in}{1.904930in}}%
\pgfpathlineto{\pgfqpoint{3.653416in}{1.915400in}}%
\pgfpathlineto{\pgfqpoint{3.661492in}{1.925880in}}%
\pgfpathlineto{\pgfqpoint{3.669563in}{1.936370in}}%
\pgfpathlineto{\pgfqpoint{3.677628in}{1.946868in}}%
\pgfpathlineto{\pgfqpoint{3.664004in}{1.948013in}}%
\pgfpathlineto{\pgfqpoint{3.650386in}{1.949360in}}%
\pgfpathlineto{\pgfqpoint{3.636774in}{1.950911in}}%
\pgfpathlineto{\pgfqpoint{3.623168in}{1.952664in}}%
\pgfpathlineto{\pgfqpoint{3.615092in}{1.942368in}}%
\pgfpathlineto{\pgfqpoint{3.607011in}{1.932087in}}%
\pgfpathlineto{\pgfqpoint{3.598924in}{1.921823in}}%
\pgfpathlineto{\pgfqpoint{3.590831in}{1.911579in}}%
\pgfpathclose%
\pgfusepath{fill}%
\end{pgfscope}%
\begin{pgfscope}%
\pgfpathrectangle{\pgfqpoint{1.150000in}{0.150000in}}{\pgfqpoint{5.700000in}{5.700000in}}%
\pgfusepath{clip}%
\pgfsetbuttcap%
\pgfsetroundjoin%
\definecolor{currentfill}{rgb}{0.121148,0.592739,0.544641}%
\pgfsetfillcolor{currentfill}%
\pgfsetfillopacity{0.800000}%
\pgfsetlinewidth{0.000000pt}%
\definecolor{currentstroke}{rgb}{0.000000,0.000000,0.000000}%
\pgfsetstrokecolor{currentstroke}%
\pgfsetdash{}{0pt}%
\pgfpathmoveto{\pgfqpoint{5.680109in}{3.300072in}}%
\pgfpathlineto{\pgfqpoint{5.694634in}{3.310168in}}%
\pgfpathlineto{\pgfqpoint{5.709177in}{3.320438in}}%
\pgfpathlineto{\pgfqpoint{5.723739in}{3.330884in}}%
\pgfpathlineto{\pgfqpoint{5.738321in}{3.341506in}}%
\pgfpathlineto{\pgfqpoint{5.745511in}{3.344644in}}%
\pgfpathlineto{\pgfqpoint{5.752697in}{3.347874in}}%
\pgfpathlineto{\pgfqpoint{5.759880in}{3.351203in}}%
\pgfpathlineto{\pgfqpoint{5.767060in}{3.354638in}}%
\pgfpathlineto{\pgfqpoint{5.752510in}{3.344661in}}%
\pgfpathlineto{\pgfqpoint{5.737979in}{3.334858in}}%
\pgfpathlineto{\pgfqpoint{5.723467in}{3.325228in}}%
\pgfpathlineto{\pgfqpoint{5.708973in}{3.315773in}}%
\pgfpathlineto{\pgfqpoint{5.701762in}{3.311685in}}%
\pgfpathlineto{\pgfqpoint{5.694547in}{3.307709in}}%
\pgfpathlineto{\pgfqpoint{5.687330in}{3.303841in}}%
\pgfpathlineto{\pgfqpoint{5.680109in}{3.300072in}}%
\pgfpathclose%
\pgfusepath{fill}%
\end{pgfscope}%
\begin{pgfscope}%
\pgfpathrectangle{\pgfqpoint{1.150000in}{0.150000in}}{\pgfqpoint{5.700000in}{5.700000in}}%
\pgfusepath{clip}%
\pgfsetbuttcap%
\pgfsetroundjoin%
\definecolor{currentfill}{rgb}{0.243113,0.292092,0.538516}%
\pgfsetfillcolor{currentfill}%
\pgfsetfillopacity{0.800000}%
\pgfsetlinewidth{0.000000pt}%
\definecolor{currentstroke}{rgb}{0.000000,0.000000,0.000000}%
\pgfsetstrokecolor{currentstroke}%
\pgfsetdash{}{0pt}%
\pgfpathmoveto{\pgfqpoint{2.496019in}{2.493703in}}%
\pgfpathlineto{\pgfqpoint{2.509911in}{2.473273in}}%
\pgfpathlineto{\pgfqpoint{2.523791in}{2.453158in}}%
\pgfpathlineto{\pgfqpoint{2.537660in}{2.433357in}}%
\pgfpathlineto{\pgfqpoint{2.551518in}{2.413866in}}%
\pgfpathlineto{\pgfqpoint{2.560161in}{2.416978in}}%
\pgfpathlineto{\pgfqpoint{2.568789in}{2.420293in}}%
\pgfpathlineto{\pgfqpoint{2.577403in}{2.423807in}}%
\pgfpathlineto{\pgfqpoint{2.586005in}{2.427517in}}%
\pgfpathlineto{\pgfqpoint{2.572184in}{2.446613in}}%
\pgfpathlineto{\pgfqpoint{2.558354in}{2.466019in}}%
\pgfpathlineto{\pgfqpoint{2.544512in}{2.485738in}}%
\pgfpathlineto{\pgfqpoint{2.530659in}{2.505772in}}%
\pgfpathlineto{\pgfqpoint{2.522020in}{2.502445in}}%
\pgfpathlineto{\pgfqpoint{2.513368in}{2.499322in}}%
\pgfpathlineto{\pgfqpoint{2.504700in}{2.496407in}}%
\pgfpathlineto{\pgfqpoint{2.496019in}{2.493703in}}%
\pgfpathclose%
\pgfusepath{fill}%
\end{pgfscope}%
\begin{pgfscope}%
\pgfpathrectangle{\pgfqpoint{1.150000in}{0.150000in}}{\pgfqpoint{5.700000in}{5.700000in}}%
\pgfusepath{clip}%
\pgfsetbuttcap%
\pgfsetroundjoin%
\definecolor{currentfill}{rgb}{0.282910,0.105393,0.426902}%
\pgfsetfillcolor{currentfill}%
\pgfsetfillopacity{0.800000}%
\pgfsetlinewidth{0.000000pt}%
\definecolor{currentstroke}{rgb}{0.000000,0.000000,0.000000}%
\pgfsetstrokecolor{currentstroke}%
\pgfsetdash{}{0pt}%
\pgfpathmoveto{\pgfqpoint{2.881585in}{2.032674in}}%
\pgfpathlineto{\pgfqpoint{2.895262in}{2.020157in}}%
\pgfpathlineto{\pgfqpoint{2.908935in}{2.007891in}}%
\pgfpathlineto{\pgfqpoint{2.922604in}{1.995876in}}%
\pgfpathlineto{\pgfqpoint{2.936270in}{1.984110in}}%
\pgfpathlineto{\pgfqpoint{2.944680in}{1.990002in}}%
\pgfpathlineto{\pgfqpoint{2.953081in}{1.996039in}}%
\pgfpathlineto{\pgfqpoint{2.961471in}{2.002217in}}%
\pgfpathlineto{\pgfqpoint{2.969852in}{2.008533in}}%
\pgfpathlineto{\pgfqpoint{2.956214in}{2.019922in}}%
\pgfpathlineto{\pgfqpoint{2.942572in}{2.031558in}}%
\pgfpathlineto{\pgfqpoint{2.928927in}{2.043445in}}%
\pgfpathlineto{\pgfqpoint{2.915279in}{2.055584in}}%
\pgfpathlineto{\pgfqpoint{2.906871in}{2.049634in}}%
\pgfpathlineto{\pgfqpoint{2.898453in}{2.043831in}}%
\pgfpathlineto{\pgfqpoint{2.890024in}{2.038176in}}%
\pgfpathlineto{\pgfqpoint{2.881585in}{2.032674in}}%
\pgfpathclose%
\pgfusepath{fill}%
\end{pgfscope}%
\begin{pgfscope}%
\pgfpathrectangle{\pgfqpoint{1.150000in}{0.150000in}}{\pgfqpoint{5.700000in}{5.700000in}}%
\pgfusepath{clip}%
\pgfsetbuttcap%
\pgfsetroundjoin%
\definecolor{currentfill}{rgb}{0.276194,0.190074,0.493001}%
\pgfsetfillcolor{currentfill}%
\pgfsetfillopacity{0.800000}%
\pgfsetlinewidth{0.000000pt}%
\definecolor{currentstroke}{rgb}{0.000000,0.000000,0.000000}%
\pgfsetstrokecolor{currentstroke}%
\pgfsetdash{}{0pt}%
\pgfpathmoveto{\pgfqpoint{4.111098in}{2.186972in}}%
\pgfpathlineto{\pgfqpoint{4.124858in}{2.190531in}}%
\pgfpathlineto{\pgfqpoint{4.138629in}{2.194280in}}%
\pgfpathlineto{\pgfqpoint{4.152410in}{2.198219in}}%
\pgfpathlineto{\pgfqpoint{4.166201in}{2.202349in}}%
\pgfpathlineto{\pgfqpoint{4.174116in}{2.212967in}}%
\pgfpathlineto{\pgfqpoint{4.182026in}{2.223529in}}%
\pgfpathlineto{\pgfqpoint{4.189930in}{2.234034in}}%
\pgfpathlineto{\pgfqpoint{4.197829in}{2.244482in}}%
\pgfpathlineto{\pgfqpoint{4.184043in}{2.240330in}}%
\pgfpathlineto{\pgfqpoint{4.170268in}{2.236368in}}%
\pgfpathlineto{\pgfqpoint{4.156504in}{2.232596in}}%
\pgfpathlineto{\pgfqpoint{4.142750in}{2.229015in}}%
\pgfpathlineto{\pgfqpoint{4.134844in}{2.218577in}}%
\pgfpathlineto{\pgfqpoint{4.126934in}{2.208091in}}%
\pgfpathlineto{\pgfqpoint{4.119019in}{2.197556in}}%
\pgfpathlineto{\pgfqpoint{4.111098in}{2.186972in}}%
\pgfpathclose%
\pgfusepath{fill}%
\end{pgfscope}%
\begin{pgfscope}%
\pgfpathrectangle{\pgfqpoint{1.150000in}{0.150000in}}{\pgfqpoint{5.700000in}{5.700000in}}%
\pgfusepath{clip}%
\pgfsetbuttcap%
\pgfsetroundjoin%
\definecolor{currentfill}{rgb}{0.119423,0.611141,0.538982}%
\pgfsetfillcolor{currentfill}%
\pgfsetfillopacity{0.800000}%
\pgfsetlinewidth{0.000000pt}%
\definecolor{currentstroke}{rgb}{0.000000,0.000000,0.000000}%
\pgfsetstrokecolor{currentstroke}%
\pgfsetdash{}{0pt}%
\pgfpathmoveto{\pgfqpoint{5.767060in}{3.354638in}}%
\pgfpathlineto{\pgfqpoint{5.781629in}{3.364790in}}%
\pgfpathlineto{\pgfqpoint{5.796218in}{3.375116in}}%
\pgfpathlineto{\pgfqpoint{5.810827in}{3.385616in}}%
\pgfpathlineto{\pgfqpoint{5.825455in}{3.396291in}}%
\pgfpathlineto{\pgfqpoint{5.832598in}{3.399175in}}%
\pgfpathlineto{\pgfqpoint{5.839739in}{3.402171in}}%
\pgfpathlineto{\pgfqpoint{5.846878in}{3.405289in}}%
\pgfpathlineto{\pgfqpoint{5.854014in}{3.408534in}}%
\pgfpathlineto{\pgfqpoint{5.839420in}{3.398536in}}%
\pgfpathlineto{\pgfqpoint{5.824846in}{3.388712in}}%
\pgfpathlineto{\pgfqpoint{5.810291in}{3.379061in}}%
\pgfpathlineto{\pgfqpoint{5.795755in}{3.369584in}}%
\pgfpathlineto{\pgfqpoint{5.788584in}{3.365652in}}%
\pgfpathlineto{\pgfqpoint{5.781412in}{3.361856in}}%
\pgfpathlineto{\pgfqpoint{5.774237in}{3.358187in}}%
\pgfpathlineto{\pgfqpoint{5.767060in}{3.354638in}}%
\pgfpathclose%
\pgfusepath{fill}%
\end{pgfscope}%
\begin{pgfscope}%
\pgfpathrectangle{\pgfqpoint{1.150000in}{0.150000in}}{\pgfqpoint{5.700000in}{5.700000in}}%
\pgfusepath{clip}%
\pgfsetbuttcap%
\pgfsetroundjoin%
\definecolor{currentfill}{rgb}{0.225863,0.330805,0.547314}%
\pgfsetfillcolor{currentfill}%
\pgfsetfillopacity{0.800000}%
\pgfsetlinewidth{0.000000pt}%
\definecolor{currentstroke}{rgb}{0.000000,0.000000,0.000000}%
\pgfsetstrokecolor{currentstroke}%
\pgfsetdash{}{0pt}%
\pgfpathmoveto{\pgfqpoint{4.576360in}{2.528811in}}%
\pgfpathlineto{\pgfqpoint{4.590324in}{2.535767in}}%
\pgfpathlineto{\pgfqpoint{4.604302in}{2.542908in}}%
\pgfpathlineto{\pgfqpoint{4.618294in}{2.550233in}}%
\pgfpathlineto{\pgfqpoint{4.632300in}{2.557742in}}%
\pgfpathlineto{\pgfqpoint{4.640047in}{2.566421in}}%
\pgfpathlineto{\pgfqpoint{4.647788in}{2.575030in}}%
\pgfpathlineto{\pgfqpoint{4.655523in}{2.583573in}}%
\pgfpathlineto{\pgfqpoint{4.663251in}{2.592050in}}%
\pgfpathlineto{\pgfqpoint{4.649253in}{2.584715in}}%
\pgfpathlineto{\pgfqpoint{4.635270in}{2.577564in}}%
\pgfpathlineto{\pgfqpoint{4.621300in}{2.570596in}}%
\pgfpathlineto{\pgfqpoint{4.607344in}{2.563813in}}%
\pgfpathlineto{\pgfqpoint{4.599607in}{2.555150in}}%
\pgfpathlineto{\pgfqpoint{4.591864in}{2.546430in}}%
\pgfpathlineto{\pgfqpoint{4.584115in}{2.537651in}}%
\pgfpathlineto{\pgfqpoint{4.576360in}{2.528811in}}%
\pgfpathclose%
\pgfusepath{fill}%
\end{pgfscope}%
\begin{pgfscope}%
\pgfpathrectangle{\pgfqpoint{1.150000in}{0.150000in}}{\pgfqpoint{5.700000in}{5.700000in}}%
\pgfusepath{clip}%
\pgfsetbuttcap%
\pgfsetroundjoin%
\definecolor{currentfill}{rgb}{0.171176,0.452530,0.557965}%
\pgfsetfillcolor{currentfill}%
\pgfsetfillopacity{0.800000}%
\pgfsetlinewidth{0.000000pt}%
\definecolor{currentstroke}{rgb}{0.000000,0.000000,0.000000}%
\pgfsetstrokecolor{currentstroke}%
\pgfsetdash{}{0pt}%
\pgfpathmoveto{\pgfqpoint{5.041652in}{2.872698in}}%
\pgfpathlineto{\pgfqpoint{5.055853in}{2.881758in}}%
\pgfpathlineto{\pgfqpoint{5.070071in}{2.891000in}}%
\pgfpathlineto{\pgfqpoint{5.084306in}{2.900421in}}%
\pgfpathlineto{\pgfqpoint{5.098558in}{2.910022in}}%
\pgfpathlineto{\pgfqpoint{5.106091in}{2.916028in}}%
\pgfpathlineto{\pgfqpoint{5.113617in}{2.922001in}}%
\pgfpathlineto{\pgfqpoint{5.121137in}{2.927946in}}%
\pgfpathlineto{\pgfqpoint{5.128650in}{2.933867in}}%
\pgfpathlineto{\pgfqpoint{5.114414in}{2.924641in}}%
\pgfpathlineto{\pgfqpoint{5.100195in}{2.915594in}}%
\pgfpathlineto{\pgfqpoint{5.085993in}{2.906727in}}%
\pgfpathlineto{\pgfqpoint{5.071807in}{2.898039in}}%
\pgfpathlineto{\pgfqpoint{5.064277in}{2.891732in}}%
\pgfpathlineto{\pgfqpoint{5.056742in}{2.885409in}}%
\pgfpathlineto{\pgfqpoint{5.049200in}{2.879066in}}%
\pgfpathlineto{\pgfqpoint{5.041652in}{2.872698in}}%
\pgfpathclose%
\pgfusepath{fill}%
\end{pgfscope}%
\begin{pgfscope}%
\pgfpathrectangle{\pgfqpoint{1.150000in}{0.150000in}}{\pgfqpoint{5.700000in}{5.700000in}}%
\pgfusepath{clip}%
\pgfsetbuttcap%
\pgfsetroundjoin%
\definecolor{currentfill}{rgb}{0.276022,0.044167,0.370164}%
\pgfsetfillcolor{currentfill}%
\pgfsetfillopacity{0.800000}%
\pgfsetlinewidth{0.000000pt}%
\definecolor{currentstroke}{rgb}{0.000000,0.000000,0.000000}%
\pgfsetstrokecolor{currentstroke}%
\pgfsetdash{}{0pt}%
\pgfpathmoveto{\pgfqpoint{3.503937in}{1.881727in}}%
\pgfpathlineto{\pgfqpoint{3.517547in}{1.878693in}}%
\pgfpathlineto{\pgfqpoint{3.531161in}{1.875865in}}%
\pgfpathlineto{\pgfqpoint{3.544780in}{1.873244in}}%
\pgfpathlineto{\pgfqpoint{3.558404in}{1.870828in}}%
\pgfpathlineto{\pgfqpoint{3.566520in}{1.880977in}}%
\pgfpathlineto{\pgfqpoint{3.574629in}{1.891154in}}%
\pgfpathlineto{\pgfqpoint{3.582733in}{1.901355in}}%
\pgfpathlineto{\pgfqpoint{3.590831in}{1.911579in}}%
\pgfpathlineto{\pgfqpoint{3.577219in}{1.913750in}}%
\pgfpathlineto{\pgfqpoint{3.563612in}{1.916126in}}%
\pgfpathlineto{\pgfqpoint{3.550010in}{1.918709in}}%
\pgfpathlineto{\pgfqpoint{3.536413in}{1.921498in}}%
\pgfpathlineto{\pgfqpoint{3.528303in}{1.911507in}}%
\pgfpathlineto{\pgfqpoint{3.520187in}{1.901547in}}%
\pgfpathlineto{\pgfqpoint{3.512065in}{1.891619in}}%
\pgfpathlineto{\pgfqpoint{3.503937in}{1.881727in}}%
\pgfpathclose%
\pgfusepath{fill}%
\end{pgfscope}%
\begin{pgfscope}%
\pgfpathrectangle{\pgfqpoint{1.150000in}{0.150000in}}{\pgfqpoint{5.700000in}{5.700000in}}%
\pgfusepath{clip}%
\pgfsetbuttcap%
\pgfsetroundjoin%
\definecolor{currentfill}{rgb}{0.121380,0.629492,0.531973}%
\pgfsetfillcolor{currentfill}%
\pgfsetfillopacity{0.800000}%
\pgfsetlinewidth{0.000000pt}%
\definecolor{currentstroke}{rgb}{0.000000,0.000000,0.000000}%
\pgfsetstrokecolor{currentstroke}%
\pgfsetdash{}{0pt}%
\pgfpathmoveto{\pgfqpoint{5.854014in}{3.408534in}}%
\pgfpathlineto{\pgfqpoint{5.868627in}{3.418705in}}%
\pgfpathlineto{\pgfqpoint{5.883260in}{3.429050in}}%
\pgfpathlineto{\pgfqpoint{5.897913in}{3.439568in}}%
\pgfpathlineto{\pgfqpoint{5.912586in}{3.450261in}}%
\pgfpathlineto{\pgfqpoint{5.919685in}{3.452943in}}%
\pgfpathlineto{\pgfqpoint{5.926782in}{3.455760in}}%
\pgfpathlineto{\pgfqpoint{5.933877in}{3.458721in}}%
\pgfpathlineto{\pgfqpoint{5.940971in}{3.461832in}}%
\pgfpathlineto{\pgfqpoint{5.926335in}{3.451851in}}%
\pgfpathlineto{\pgfqpoint{5.911719in}{3.442042in}}%
\pgfpathlineto{\pgfqpoint{5.897122in}{3.432405in}}%
\pgfpathlineto{\pgfqpoint{5.882544in}{3.422942in}}%
\pgfpathlineto{\pgfqpoint{5.875413in}{3.419110in}}%
\pgfpathlineto{\pgfqpoint{5.868281in}{3.415437in}}%
\pgfpathlineto{\pgfqpoint{5.861149in}{3.411914in}}%
\pgfpathlineto{\pgfqpoint{5.854014in}{3.408534in}}%
\pgfpathclose%
\pgfusepath{fill}%
\end{pgfscope}%
\begin{pgfscope}%
\pgfpathrectangle{\pgfqpoint{1.150000in}{0.150000in}}{\pgfqpoint{5.700000in}{5.700000in}}%
\pgfusepath{clip}%
\pgfsetbuttcap%
\pgfsetroundjoin%
\definecolor{currentfill}{rgb}{0.273809,0.031497,0.358853}%
\pgfsetfillcolor{currentfill}%
\pgfsetfillopacity{0.800000}%
\pgfsetlinewidth{0.000000pt}%
\definecolor{currentstroke}{rgb}{0.000000,0.000000,0.000000}%
\pgfsetstrokecolor{currentstroke}%
\pgfsetdash{}{0pt}%
\pgfpathmoveto{\pgfqpoint{3.275288in}{1.863998in}}%
\pgfpathlineto{\pgfqpoint{3.288892in}{1.857827in}}%
\pgfpathlineto{\pgfqpoint{3.302498in}{1.851874in}}%
\pgfpathlineto{\pgfqpoint{3.316105in}{1.846138in}}%
\pgfpathlineto{\pgfqpoint{3.329715in}{1.840619in}}%
\pgfpathlineto{\pgfqpoint{3.337925in}{1.849506in}}%
\pgfpathlineto{\pgfqpoint{3.346127in}{1.858464in}}%
\pgfpathlineto{\pgfqpoint{3.354322in}{1.867490in}}%
\pgfpathlineto{\pgfqpoint{3.362511in}{1.876581in}}%
\pgfpathlineto{\pgfqpoint{3.348918in}{1.881792in}}%
\pgfpathlineto{\pgfqpoint{3.335327in}{1.887220in}}%
\pgfpathlineto{\pgfqpoint{3.321739in}{1.892864in}}%
\pgfpathlineto{\pgfqpoint{3.308153in}{1.898727in}}%
\pgfpathlineto{\pgfqpoint{3.299947in}{1.889932in}}%
\pgfpathlineto{\pgfqpoint{3.291735in}{1.881211in}}%
\pgfpathlineto{\pgfqpoint{3.283515in}{1.872565in}}%
\pgfpathlineto{\pgfqpoint{3.275288in}{1.863998in}}%
\pgfpathclose%
\pgfusepath{fill}%
\end{pgfscope}%
\begin{pgfscope}%
\pgfpathrectangle{\pgfqpoint{1.150000in}{0.150000in}}{\pgfqpoint{5.700000in}{5.700000in}}%
\pgfusepath{clip}%
\pgfsetbuttcap%
\pgfsetroundjoin%
\definecolor{currentfill}{rgb}{0.227802,0.326594,0.546532}%
\pgfsetfillcolor{currentfill}%
\pgfsetfillopacity{0.800000}%
\pgfsetlinewidth{0.000000pt}%
\definecolor{currentstroke}{rgb}{0.000000,0.000000,0.000000}%
\pgfsetstrokecolor{currentstroke}%
\pgfsetdash{}{0pt}%
\pgfpathmoveto{\pgfqpoint{2.440328in}{2.578649in}}%
\pgfpathlineto{\pgfqpoint{2.454270in}{2.556923in}}%
\pgfpathlineto{\pgfqpoint{2.468199in}{2.535525in}}%
\pgfpathlineto{\pgfqpoint{2.482115in}{2.514453in}}%
\pgfpathlineto{\pgfqpoint{2.496019in}{2.493703in}}%
\pgfpathlineto{\pgfqpoint{2.504700in}{2.496407in}}%
\pgfpathlineto{\pgfqpoint{2.513368in}{2.499322in}}%
\pgfpathlineto{\pgfqpoint{2.522020in}{2.502445in}}%
\pgfpathlineto{\pgfqpoint{2.530659in}{2.505772in}}%
\pgfpathlineto{\pgfqpoint{2.516795in}{2.526124in}}%
\pgfpathlineto{\pgfqpoint{2.502919in}{2.546798in}}%
\pgfpathlineto{\pgfqpoint{2.489030in}{2.567796in}}%
\pgfpathlineto{\pgfqpoint{2.475130in}{2.589122in}}%
\pgfpathlineto{\pgfqpoint{2.466451in}{2.586181in}}%
\pgfpathlineto{\pgfqpoint{2.457759in}{2.583452in}}%
\pgfpathlineto{\pgfqpoint{2.449051in}{2.580940in}}%
\pgfpathlineto{\pgfqpoint{2.440328in}{2.578649in}}%
\pgfpathclose%
\pgfusepath{fill}%
\end{pgfscope}%
\begin{pgfscope}%
\pgfpathrectangle{\pgfqpoint{1.150000in}{0.150000in}}{\pgfqpoint{5.700000in}{5.700000in}}%
\pgfusepath{clip}%
\pgfsetbuttcap%
\pgfsetroundjoin%
\definecolor{currentfill}{rgb}{0.269308,0.218818,0.509577}%
\pgfsetfillcolor{currentfill}%
\pgfsetfillopacity{0.800000}%
\pgfsetlinewidth{0.000000pt}%
\definecolor{currentstroke}{rgb}{0.000000,0.000000,0.000000}%
\pgfsetstrokecolor{currentstroke}%
\pgfsetdash{}{0pt}%
\pgfpathmoveto{\pgfqpoint{4.197829in}{2.244482in}}%
\pgfpathlineto{\pgfqpoint{4.211626in}{2.248823in}}%
\pgfpathlineto{\pgfqpoint{4.225434in}{2.253354in}}%
\pgfpathlineto{\pgfqpoint{4.239254in}{2.258073in}}%
\pgfpathlineto{\pgfqpoint{4.253085in}{2.262981in}}%
\pgfpathlineto{\pgfqpoint{4.260973in}{2.273376in}}%
\pgfpathlineto{\pgfqpoint{4.268856in}{2.283708in}}%
\pgfpathlineto{\pgfqpoint{4.276733in}{2.293976in}}%
\pgfpathlineto{\pgfqpoint{4.284606in}{2.304182in}}%
\pgfpathlineto{\pgfqpoint{4.270780in}{2.299283in}}%
\pgfpathlineto{\pgfqpoint{4.256967in}{2.294574in}}%
\pgfpathlineto{\pgfqpoint{4.243164in}{2.290053in}}%
\pgfpathlineto{\pgfqpoint{4.229373in}{2.285721in}}%
\pgfpathlineto{\pgfqpoint{4.221495in}{2.275494in}}%
\pgfpathlineto{\pgfqpoint{4.213612in}{2.265212in}}%
\pgfpathlineto{\pgfqpoint{4.205723in}{2.254875in}}%
\pgfpathlineto{\pgfqpoint{4.197829in}{2.244482in}}%
\pgfpathclose%
\pgfusepath{fill}%
\end{pgfscope}%
\begin{pgfscope}%
\pgfpathrectangle{\pgfqpoint{1.150000in}{0.150000in}}{\pgfqpoint{5.700000in}{5.700000in}}%
\pgfusepath{clip}%
\pgfsetbuttcap%
\pgfsetroundjoin%
\definecolor{currentfill}{rgb}{0.276022,0.044167,0.370164}%
\pgfsetfillcolor{currentfill}%
\pgfsetfillopacity{0.800000}%
\pgfsetlinewidth{0.000000pt}%
\definecolor{currentstroke}{rgb}{0.000000,0.000000,0.000000}%
\pgfsetstrokecolor{currentstroke}%
\pgfsetdash{}{0pt}%
\pgfpathmoveto{\pgfqpoint{3.133352in}{1.890668in}}%
\pgfpathlineto{\pgfqpoint{3.146970in}{1.882368in}}%
\pgfpathlineto{\pgfqpoint{3.160588in}{1.874295in}}%
\pgfpathlineto{\pgfqpoint{3.174206in}{1.866450in}}%
\pgfpathlineto{\pgfqpoint{3.187824in}{1.858829in}}%
\pgfpathlineto{\pgfqpoint{3.196102in}{1.866699in}}%
\pgfpathlineto{\pgfqpoint{3.204371in}{1.874668in}}%
\pgfpathlineto{\pgfqpoint{3.212632in}{1.882732in}}%
\pgfpathlineto{\pgfqpoint{3.220886in}{1.890888in}}%
\pgfpathlineto{\pgfqpoint{3.207288in}{1.898167in}}%
\pgfpathlineto{\pgfqpoint{3.193691in}{1.905672in}}%
\pgfpathlineto{\pgfqpoint{3.180094in}{1.913402in}}%
\pgfpathlineto{\pgfqpoint{3.166497in}{1.921361in}}%
\pgfpathlineto{\pgfqpoint{3.158223in}{1.913534in}}%
\pgfpathlineto{\pgfqpoint{3.149941in}{1.905807in}}%
\pgfpathlineto{\pgfqpoint{3.141650in}{1.898184in}}%
\pgfpathlineto{\pgfqpoint{3.133352in}{1.890668in}}%
\pgfpathclose%
\pgfusepath{fill}%
\end{pgfscope}%
\begin{pgfscope}%
\pgfpathrectangle{\pgfqpoint{1.150000in}{0.150000in}}{\pgfqpoint{5.700000in}{5.700000in}}%
\pgfusepath{clip}%
\pgfsetbuttcap%
\pgfsetroundjoin%
\definecolor{currentfill}{rgb}{0.128087,0.647749,0.523491}%
\pgfsetfillcolor{currentfill}%
\pgfsetfillopacity{0.800000}%
\pgfsetlinewidth{0.000000pt}%
\definecolor{currentstroke}{rgb}{0.000000,0.000000,0.000000}%
\pgfsetstrokecolor{currentstroke}%
\pgfsetdash{}{0pt}%
\pgfpathmoveto{\pgfqpoint{5.940971in}{3.461832in}}%
\pgfpathlineto{\pgfqpoint{5.955627in}{3.471987in}}%
\pgfpathlineto{\pgfqpoint{5.970303in}{3.482315in}}%
\pgfpathlineto{\pgfqpoint{5.984999in}{3.492816in}}%
\pgfpathlineto{\pgfqpoint{5.999715in}{3.503490in}}%
\pgfpathlineto{\pgfqpoint{6.006770in}{3.506029in}}%
\pgfpathlineto{\pgfqpoint{6.013824in}{3.508727in}}%
\pgfpathlineto{\pgfqpoint{6.020878in}{3.511592in}}%
\pgfpathlineto{\pgfqpoint{6.027931in}{3.514633in}}%
\pgfpathlineto{\pgfqpoint{6.013255in}{3.504702in}}%
\pgfpathlineto{\pgfqpoint{5.998598in}{3.494944in}}%
\pgfpathlineto{\pgfqpoint{5.983961in}{3.485358in}}%
\pgfpathlineto{\pgfqpoint{5.969343in}{3.475944in}}%
\pgfpathlineto{\pgfqpoint{5.962250in}{3.472151in}}%
\pgfpathlineto{\pgfqpoint{5.955157in}{3.468539in}}%
\pgfpathlineto{\pgfqpoint{5.948064in}{3.465103in}}%
\pgfpathlineto{\pgfqpoint{5.940971in}{3.461832in}}%
\pgfpathclose%
\pgfusepath{fill}%
\end{pgfscope}%
\begin{pgfscope}%
\pgfpathrectangle{\pgfqpoint{1.150000in}{0.150000in}}{\pgfqpoint{5.700000in}{5.700000in}}%
\pgfusepath{clip}%
\pgfsetbuttcap%
\pgfsetroundjoin%
\definecolor{currentfill}{rgb}{0.281446,0.084320,0.407414}%
\pgfsetfillcolor{currentfill}%
\pgfsetfillopacity{0.800000}%
\pgfsetlinewidth{0.000000pt}%
\definecolor{currentstroke}{rgb}{0.000000,0.000000,0.000000}%
\pgfsetstrokecolor{currentstroke}%
\pgfsetdash{}{0pt}%
\pgfpathmoveto{\pgfqpoint{2.936270in}{1.984110in}}%
\pgfpathlineto{\pgfqpoint{2.949932in}{1.972590in}}%
\pgfpathlineto{\pgfqpoint{2.963592in}{1.961316in}}%
\pgfpathlineto{\pgfqpoint{2.977248in}{1.950286in}}%
\pgfpathlineto{\pgfqpoint{2.990902in}{1.939498in}}%
\pgfpathlineto{\pgfqpoint{2.999286in}{1.945780in}}%
\pgfpathlineto{\pgfqpoint{3.007660in}{1.952198in}}%
\pgfpathlineto{\pgfqpoint{3.016024in}{1.958749in}}%
\pgfpathlineto{\pgfqpoint{3.024379in}{1.965430in}}%
\pgfpathlineto{\pgfqpoint{3.010751in}{1.975841in}}%
\pgfpathlineto{\pgfqpoint{2.997120in}{1.986494in}}%
\pgfpathlineto{\pgfqpoint{2.983487in}{1.997391in}}%
\pgfpathlineto{\pgfqpoint{2.969852in}{2.008533in}}%
\pgfpathlineto{\pgfqpoint{2.961471in}{2.002217in}}%
\pgfpathlineto{\pgfqpoint{2.953081in}{1.996039in}}%
\pgfpathlineto{\pgfqpoint{2.944680in}{1.990002in}}%
\pgfpathlineto{\pgfqpoint{2.936270in}{1.984110in}}%
\pgfpathclose%
\pgfusepath{fill}%
\end{pgfscope}%
\begin{pgfscope}%
\pgfpathrectangle{\pgfqpoint{1.150000in}{0.150000in}}{\pgfqpoint{5.700000in}{5.700000in}}%
\pgfusepath{clip}%
\pgfsetbuttcap%
\pgfsetroundjoin%
\definecolor{currentfill}{rgb}{0.162142,0.474838,0.558140}%
\pgfsetfillcolor{currentfill}%
\pgfsetfillopacity{0.800000}%
\pgfsetlinewidth{0.000000pt}%
\definecolor{currentstroke}{rgb}{0.000000,0.000000,0.000000}%
\pgfsetstrokecolor{currentstroke}%
\pgfsetdash{}{0pt}%
\pgfpathmoveto{\pgfqpoint{5.128650in}{2.933867in}}%
\pgfpathlineto{\pgfqpoint{5.142903in}{2.943273in}}%
\pgfpathlineto{\pgfqpoint{5.157172in}{2.952859in}}%
\pgfpathlineto{\pgfqpoint{5.171459in}{2.962624in}}%
\pgfpathlineto{\pgfqpoint{5.185763in}{2.972568in}}%
\pgfpathlineto{\pgfqpoint{5.193253in}{2.978073in}}%
\pgfpathlineto{\pgfqpoint{5.200736in}{2.983555in}}%
\pgfpathlineto{\pgfqpoint{5.208213in}{2.989019in}}%
\pgfpathlineto{\pgfqpoint{5.215684in}{2.994469in}}%
\pgfpathlineto{\pgfqpoint{5.201397in}{2.984934in}}%
\pgfpathlineto{\pgfqpoint{5.187128in}{2.975578in}}%
\pgfpathlineto{\pgfqpoint{5.172876in}{2.966400in}}%
\pgfpathlineto{\pgfqpoint{5.158640in}{2.957402in}}%
\pgfpathlineto{\pgfqpoint{5.151152in}{2.951531in}}%
\pgfpathlineto{\pgfqpoint{5.143657in}{2.945655in}}%
\pgfpathlineto{\pgfqpoint{5.136157in}{2.939769in}}%
\pgfpathlineto{\pgfqpoint{5.128650in}{2.933867in}}%
\pgfpathclose%
\pgfusepath{fill}%
\end{pgfscope}%
\begin{pgfscope}%
\pgfpathrectangle{\pgfqpoint{1.150000in}{0.150000in}}{\pgfqpoint{5.700000in}{5.700000in}}%
\pgfusepath{clip}%
\pgfsetbuttcap%
\pgfsetroundjoin%
\definecolor{currentfill}{rgb}{0.212395,0.359683,0.551710}%
\pgfsetfillcolor{currentfill}%
\pgfsetfillopacity{0.800000}%
\pgfsetlinewidth{0.000000pt}%
\definecolor{currentstroke}{rgb}{0.000000,0.000000,0.000000}%
\pgfsetstrokecolor{currentstroke}%
\pgfsetdash{}{0pt}%
\pgfpathmoveto{\pgfqpoint{4.663251in}{2.592050in}}%
\pgfpathlineto{\pgfqpoint{4.677263in}{2.599569in}}%
\pgfpathlineto{\pgfqpoint{4.691290in}{2.607272in}}%
\pgfpathlineto{\pgfqpoint{4.705331in}{2.615158in}}%
\pgfpathlineto{\pgfqpoint{4.719387in}{2.623228in}}%
\pgfpathlineto{\pgfqpoint{4.727100in}{2.631448in}}%
\pgfpathlineto{\pgfqpoint{4.734807in}{2.639601in}}%
\pgfpathlineto{\pgfqpoint{4.742508in}{2.647688in}}%
\pgfpathlineto{\pgfqpoint{4.750202in}{2.655712in}}%
\pgfpathlineto{\pgfqpoint{4.736156in}{2.647850in}}%
\pgfpathlineto{\pgfqpoint{4.722124in}{2.640170in}}%
\pgfpathlineto{\pgfqpoint{4.708106in}{2.632674in}}%
\pgfpathlineto{\pgfqpoint{4.694103in}{2.625362in}}%
\pgfpathlineto{\pgfqpoint{4.686400in}{2.617119in}}%
\pgfpathlineto{\pgfqpoint{4.678690in}{2.608821in}}%
\pgfpathlineto{\pgfqpoint{4.670973in}{2.600465in}}%
\pgfpathlineto{\pgfqpoint{4.663251in}{2.592050in}}%
\pgfpathclose%
\pgfusepath{fill}%
\end{pgfscope}%
\begin{pgfscope}%
\pgfpathrectangle{\pgfqpoint{1.150000in}{0.150000in}}{\pgfqpoint{5.700000in}{5.700000in}}%
\pgfusepath{clip}%
\pgfsetbuttcap%
\pgfsetroundjoin%
\definecolor{currentfill}{rgb}{0.140210,0.665859,0.513427}%
\pgfsetfillcolor{currentfill}%
\pgfsetfillopacity{0.800000}%
\pgfsetlinewidth{0.000000pt}%
\definecolor{currentstroke}{rgb}{0.000000,0.000000,0.000000}%
\pgfsetstrokecolor{currentstroke}%
\pgfsetdash{}{0pt}%
\pgfpathmoveto{\pgfqpoint{6.027931in}{3.514633in}}%
\pgfpathlineto{\pgfqpoint{6.042628in}{3.524735in}}%
\pgfpathlineto{\pgfqpoint{6.057345in}{3.535010in}}%
\pgfpathlineto{\pgfqpoint{6.072083in}{3.545457in}}%
\pgfpathlineto{\pgfqpoint{6.086840in}{3.556078in}}%
\pgfpathlineto{\pgfqpoint{6.093854in}{3.558537in}}%
\pgfpathlineto{\pgfqpoint{6.100867in}{3.561181in}}%
\pgfpathlineto{\pgfqpoint{6.107882in}{3.564018in}}%
\pgfpathlineto{\pgfqpoint{6.114898in}{3.567057in}}%
\pgfpathlineto{\pgfqpoint{6.100182in}{3.557214in}}%
\pgfpathlineto{\pgfqpoint{6.085486in}{3.547542in}}%
\pgfpathlineto{\pgfqpoint{6.070810in}{3.538042in}}%
\pgfpathlineto{\pgfqpoint{6.056154in}{3.528713in}}%
\pgfpathlineto{\pgfqpoint{6.049097in}{3.524889in}}%
\pgfpathlineto{\pgfqpoint{6.042041in}{3.521273in}}%
\pgfpathlineto{\pgfqpoint{6.034986in}{3.517857in}}%
\pgfpathlineto{\pgfqpoint{6.027931in}{3.514633in}}%
\pgfpathclose%
\pgfusepath{fill}%
\end{pgfscope}%
\begin{pgfscope}%
\pgfpathrectangle{\pgfqpoint{1.150000in}{0.150000in}}{\pgfqpoint{5.700000in}{5.700000in}}%
\pgfusepath{clip}%
\pgfsetbuttcap%
\pgfsetroundjoin%
\definecolor{currentfill}{rgb}{0.273809,0.031497,0.358853}%
\pgfsetfillcolor{currentfill}%
\pgfsetfillopacity{0.800000}%
\pgfsetlinewidth{0.000000pt}%
\definecolor{currentstroke}{rgb}{0.000000,0.000000,0.000000}%
\pgfsetstrokecolor{currentstroke}%
\pgfsetdash{}{0pt}%
\pgfpathmoveto{\pgfqpoint{3.416911in}{1.857876in}}%
\pgfpathlineto{\pgfqpoint{3.430519in}{1.853730in}}%
\pgfpathlineto{\pgfqpoint{3.444131in}{1.849794in}}%
\pgfpathlineto{\pgfqpoint{3.457746in}{1.846068in}}%
\pgfpathlineto{\pgfqpoint{3.471366in}{1.842550in}}%
\pgfpathlineto{\pgfqpoint{3.479518in}{1.852280in}}%
\pgfpathlineto{\pgfqpoint{3.487663in}{1.862055in}}%
\pgfpathlineto{\pgfqpoint{3.495803in}{1.871871in}}%
\pgfpathlineto{\pgfqpoint{3.503937in}{1.881727in}}%
\pgfpathlineto{\pgfqpoint{3.490331in}{1.884968in}}%
\pgfpathlineto{\pgfqpoint{3.476730in}{1.888418in}}%
\pgfpathlineto{\pgfqpoint{3.463133in}{1.892078in}}%
\pgfpathlineto{\pgfqpoint{3.449539in}{1.895947in}}%
\pgfpathlineto{\pgfqpoint{3.441392in}{1.886356in}}%
\pgfpathlineto{\pgfqpoint{3.433238in}{1.876812in}}%
\pgfpathlineto{\pgfqpoint{3.425078in}{1.867318in}}%
\pgfpathlineto{\pgfqpoint{3.416911in}{1.857876in}}%
\pgfpathclose%
\pgfusepath{fill}%
\end{pgfscope}%
\begin{pgfscope}%
\pgfpathrectangle{\pgfqpoint{1.150000in}{0.150000in}}{\pgfqpoint{5.700000in}{5.700000in}}%
\pgfusepath{clip}%
\pgfsetbuttcap%
\pgfsetroundjoin%
\definecolor{currentfill}{rgb}{0.260571,0.246922,0.522828}%
\pgfsetfillcolor{currentfill}%
\pgfsetfillopacity{0.800000}%
\pgfsetlinewidth{0.000000pt}%
\definecolor{currentstroke}{rgb}{0.000000,0.000000,0.000000}%
\pgfsetstrokecolor{currentstroke}%
\pgfsetdash{}{0pt}%
\pgfpathmoveto{\pgfqpoint{4.284606in}{2.304182in}}%
\pgfpathlineto{\pgfqpoint{4.298443in}{2.309268in}}%
\pgfpathlineto{\pgfqpoint{4.312291in}{2.314542in}}%
\pgfpathlineto{\pgfqpoint{4.326152in}{2.320004in}}%
\pgfpathlineto{\pgfqpoint{4.340025in}{2.325653in}}%
\pgfpathlineto{\pgfqpoint{4.347886in}{2.335767in}}%
\pgfpathlineto{\pgfqpoint{4.355741in}{2.345812in}}%
\pgfpathlineto{\pgfqpoint{4.363591in}{2.355788in}}%
\pgfpathlineto{\pgfqpoint{4.371436in}{2.365698in}}%
\pgfpathlineto{\pgfqpoint{4.357568in}{2.360091in}}%
\pgfpathlineto{\pgfqpoint{4.343714in}{2.354672in}}%
\pgfpathlineto{\pgfqpoint{4.329871in}{2.349440in}}%
\pgfpathlineto{\pgfqpoint{4.316040in}{2.344395in}}%
\pgfpathlineto{\pgfqpoint{4.308190in}{2.334432in}}%
\pgfpathlineto{\pgfqpoint{4.300334in}{2.324409in}}%
\pgfpathlineto{\pgfqpoint{4.292472in}{2.314326in}}%
\pgfpathlineto{\pgfqpoint{4.284606in}{2.304182in}}%
\pgfpathclose%
\pgfusepath{fill}%
\end{pgfscope}%
\begin{pgfscope}%
\pgfpathrectangle{\pgfqpoint{1.150000in}{0.150000in}}{\pgfqpoint{5.700000in}{5.700000in}}%
\pgfusepath{clip}%
\pgfsetbuttcap%
\pgfsetroundjoin%
\definecolor{currentfill}{rgb}{0.153364,0.497000,0.557724}%
\pgfsetfillcolor{currentfill}%
\pgfsetfillopacity{0.800000}%
\pgfsetlinewidth{0.000000pt}%
\definecolor{currentstroke}{rgb}{0.000000,0.000000,0.000000}%
\pgfsetstrokecolor{currentstroke}%
\pgfsetdash{}{0pt}%
\pgfpathmoveto{\pgfqpoint{5.215684in}{2.994469in}}%
\pgfpathlineto{\pgfqpoint{5.229987in}{3.004184in}}%
\pgfpathlineto{\pgfqpoint{5.244309in}{3.014077in}}%
\pgfpathlineto{\pgfqpoint{5.258647in}{3.024149in}}%
\pgfpathlineto{\pgfqpoint{5.273004in}{3.034400in}}%
\pgfpathlineto{\pgfqpoint{5.280449in}{3.039410in}}%
\pgfpathlineto{\pgfqpoint{5.287888in}{3.044409in}}%
\pgfpathlineto{\pgfqpoint{5.295321in}{3.049402in}}%
\pgfpathlineto{\pgfqpoint{5.302747in}{3.054392in}}%
\pgfpathlineto{\pgfqpoint{5.288410in}{3.044585in}}%
\pgfpathlineto{\pgfqpoint{5.274091in}{3.034955in}}%
\pgfpathlineto{\pgfqpoint{5.259789in}{3.025504in}}%
\pgfpathlineto{\pgfqpoint{5.245504in}{3.016231in}}%
\pgfpathlineto{\pgfqpoint{5.238058in}{3.010787in}}%
\pgfpathlineto{\pgfqpoint{5.230606in}{3.005349in}}%
\pgfpathlineto{\pgfqpoint{5.223148in}{2.999911in}}%
\pgfpathlineto{\pgfqpoint{5.215684in}{2.994469in}}%
\pgfpathclose%
\pgfusepath{fill}%
\end{pgfscope}%
\begin{pgfscope}%
\pgfpathrectangle{\pgfqpoint{1.150000in}{0.150000in}}{\pgfqpoint{5.700000in}{5.700000in}}%
\pgfusepath{clip}%
\pgfsetbuttcap%
\pgfsetroundjoin%
\definecolor{currentfill}{rgb}{0.212395,0.359683,0.551710}%
\pgfsetfillcolor{currentfill}%
\pgfsetfillopacity{0.800000}%
\pgfsetlinewidth{0.000000pt}%
\definecolor{currentstroke}{rgb}{0.000000,0.000000,0.000000}%
\pgfsetstrokecolor{currentstroke}%
\pgfsetdash{}{0pt}%
\pgfpathmoveto{\pgfqpoint{2.384425in}{2.668901in}}%
\pgfpathlineto{\pgfqpoint{2.398422in}{2.645829in}}%
\pgfpathlineto{\pgfqpoint{2.412405in}{2.623099in}}%
\pgfpathlineto{\pgfqpoint{2.426373in}{2.600706in}}%
\pgfpathlineto{\pgfqpoint{2.440328in}{2.578649in}}%
\pgfpathlineto{\pgfqpoint{2.449051in}{2.580940in}}%
\pgfpathlineto{\pgfqpoint{2.457759in}{2.583452in}}%
\pgfpathlineto{\pgfqpoint{2.466451in}{2.586181in}}%
\pgfpathlineto{\pgfqpoint{2.475130in}{2.589122in}}%
\pgfpathlineto{\pgfqpoint{2.461216in}{2.610778in}}%
\pgfpathlineto{\pgfqpoint{2.447289in}{2.632768in}}%
\pgfpathlineto{\pgfqpoint{2.433349in}{2.655096in}}%
\pgfpathlineto{\pgfqpoint{2.419394in}{2.677764in}}%
\pgfpathlineto{\pgfqpoint{2.410675in}{2.675212in}}%
\pgfpathlineto{\pgfqpoint{2.401941in}{2.672881in}}%
\pgfpathlineto{\pgfqpoint{2.393191in}{2.670777in}}%
\pgfpathlineto{\pgfqpoint{2.384425in}{2.668901in}}%
\pgfpathclose%
\pgfusepath{fill}%
\end{pgfscope}%
\begin{pgfscope}%
\pgfpathrectangle{\pgfqpoint{1.150000in}{0.150000in}}{\pgfqpoint{5.700000in}{5.700000in}}%
\pgfusepath{clip}%
\pgfsetbuttcap%
\pgfsetroundjoin%
\definecolor{currentfill}{rgb}{0.201239,0.383670,0.554294}%
\pgfsetfillcolor{currentfill}%
\pgfsetfillopacity{0.800000}%
\pgfsetlinewidth{0.000000pt}%
\definecolor{currentstroke}{rgb}{0.000000,0.000000,0.000000}%
\pgfsetstrokecolor{currentstroke}%
\pgfsetdash{}{0pt}%
\pgfpathmoveto{\pgfqpoint{4.750202in}{2.655712in}}%
\pgfpathlineto{\pgfqpoint{4.764264in}{2.663757in}}%
\pgfpathlineto{\pgfqpoint{4.778340in}{2.671985in}}%
\pgfpathlineto{\pgfqpoint{4.792432in}{2.680396in}}%
\pgfpathlineto{\pgfqpoint{4.806539in}{2.688989in}}%
\pgfpathlineto{\pgfqpoint{4.814217in}{2.696725in}}%
\pgfpathlineto{\pgfqpoint{4.821889in}{2.704395in}}%
\pgfpathlineto{\pgfqpoint{4.829554in}{2.712003in}}%
\pgfpathlineto{\pgfqpoint{4.837212in}{2.719551in}}%
\pgfpathlineto{\pgfqpoint{4.823115in}{2.711199in}}%
\pgfpathlineto{\pgfqpoint{4.809034in}{2.703029in}}%
\pgfpathlineto{\pgfqpoint{4.794967in}{2.695042in}}%
\pgfpathlineto{\pgfqpoint{4.780916in}{2.687237in}}%
\pgfpathlineto{\pgfqpoint{4.773247in}{2.679435in}}%
\pgfpathlineto{\pgfqpoint{4.765572in}{2.671583in}}%
\pgfpathlineto{\pgfqpoint{4.757890in}{2.663676in}}%
\pgfpathlineto{\pgfqpoint{4.750202in}{2.655712in}}%
\pgfpathclose%
\pgfusepath{fill}%
\end{pgfscope}%
\begin{pgfscope}%
\pgfpathrectangle{\pgfqpoint{1.150000in}{0.150000in}}{\pgfqpoint{5.700000in}{5.700000in}}%
\pgfusepath{clip}%
\pgfsetbuttcap%
\pgfsetroundjoin%
\definecolor{currentfill}{rgb}{0.157851,0.683765,0.501686}%
\pgfsetfillcolor{currentfill}%
\pgfsetfillopacity{0.800000}%
\pgfsetlinewidth{0.000000pt}%
\definecolor{currentstroke}{rgb}{0.000000,0.000000,0.000000}%
\pgfsetstrokecolor{currentstroke}%
\pgfsetdash{}{0pt}%
\pgfpathmoveto{\pgfqpoint{6.114898in}{3.567057in}}%
\pgfpathlineto{\pgfqpoint{6.129634in}{3.577071in}}%
\pgfpathlineto{\pgfqpoint{6.144390in}{3.587258in}}%
\pgfpathlineto{\pgfqpoint{6.159167in}{3.597616in}}%
\pgfpathlineto{\pgfqpoint{6.173965in}{3.608146in}}%
\pgfpathlineto{\pgfqpoint{6.180939in}{3.610597in}}%
\pgfpathlineto{\pgfqpoint{6.187915in}{3.613259in}}%
\pgfpathlineto{\pgfqpoint{6.194893in}{3.616141in}}%
\pgfpathlineto{\pgfqpoint{6.180128in}{3.606215in}}%
\pgfpathlineto{\pgfqpoint{6.165384in}{3.596461in}}%
\pgfpathlineto{\pgfqpoint{6.150661in}{3.586878in}}%
\pgfpathlineto{\pgfqpoint{6.135957in}{3.577465in}}%
\pgfpathlineto{\pgfqpoint{6.128935in}{3.573772in}}%
\pgfpathlineto{\pgfqpoint{6.121915in}{3.570305in}}%
\pgfpathlineto{\pgfqpoint{6.114898in}{3.567057in}}%
\pgfpathclose%
\pgfusepath{fill}%
\end{pgfscope}%
\begin{pgfscope}%
\pgfpathrectangle{\pgfqpoint{1.150000in}{0.150000in}}{\pgfqpoint{5.700000in}{5.700000in}}%
\pgfusepath{clip}%
\pgfsetbuttcap%
\pgfsetroundjoin%
\definecolor{currentfill}{rgb}{0.279566,0.067836,0.391917}%
\pgfsetfillcolor{currentfill}%
\pgfsetfillopacity{0.800000}%
\pgfsetlinewidth{0.000000pt}%
\definecolor{currentstroke}{rgb}{0.000000,0.000000,0.000000}%
\pgfsetstrokecolor{currentstroke}%
\pgfsetdash{}{0pt}%
\pgfpathmoveto{\pgfqpoint{2.990902in}{1.939498in}}%
\pgfpathlineto{\pgfqpoint{3.004554in}{1.928951in}}%
\pgfpathlineto{\pgfqpoint{3.018204in}{1.918643in}}%
\pgfpathlineto{\pgfqpoint{3.031852in}{1.908573in}}%
\pgfpathlineto{\pgfqpoint{3.045498in}{1.898739in}}%
\pgfpathlineto{\pgfqpoint{3.053856in}{1.905408in}}%
\pgfpathlineto{\pgfqpoint{3.062205in}{1.912206in}}%
\pgfpathlineto{\pgfqpoint{3.070544in}{1.919129in}}%
\pgfpathlineto{\pgfqpoint{3.078874in}{1.926172in}}%
\pgfpathlineto{\pgfqpoint{3.065253in}{1.935631in}}%
\pgfpathlineto{\pgfqpoint{3.051630in}{1.945326in}}%
\pgfpathlineto{\pgfqpoint{3.038005in}{1.955259in}}%
\pgfpathlineto{\pgfqpoint{3.024379in}{1.965430in}}%
\pgfpathlineto{\pgfqpoint{3.016024in}{1.958749in}}%
\pgfpathlineto{\pgfqpoint{3.007660in}{1.952198in}}%
\pgfpathlineto{\pgfqpoint{2.999286in}{1.945780in}}%
\pgfpathlineto{\pgfqpoint{2.990902in}{1.939498in}}%
\pgfpathclose%
\pgfusepath{fill}%
\end{pgfscope}%
\begin{pgfscope}%
\pgfpathrectangle{\pgfqpoint{1.150000in}{0.150000in}}{\pgfqpoint{5.700000in}{5.700000in}}%
\pgfusepath{clip}%
\pgfsetbuttcap%
\pgfsetroundjoin%
\definecolor{currentfill}{rgb}{0.250425,0.274290,0.533103}%
\pgfsetfillcolor{currentfill}%
\pgfsetfillopacity{0.800000}%
\pgfsetlinewidth{0.000000pt}%
\definecolor{currentstroke}{rgb}{0.000000,0.000000,0.000000}%
\pgfsetstrokecolor{currentstroke}%
\pgfsetdash{}{0pt}%
\pgfpathmoveto{\pgfqpoint{4.371436in}{2.365698in}}%
\pgfpathlineto{\pgfqpoint{4.385315in}{2.371492in}}%
\pgfpathlineto{\pgfqpoint{4.399207in}{2.377472in}}%
\pgfpathlineto{\pgfqpoint{4.413112in}{2.383639in}}%
\pgfpathlineto{\pgfqpoint{4.427029in}{2.389992in}}%
\pgfpathlineto{\pgfqpoint{4.434862in}{2.399773in}}%
\pgfpathlineto{\pgfqpoint{4.442689in}{2.409480in}}%
\pgfpathlineto{\pgfqpoint{4.450510in}{2.419115in}}%
\pgfpathlineto{\pgfqpoint{4.458325in}{2.428680in}}%
\pgfpathlineto{\pgfqpoint{4.444414in}{2.422402in}}%
\pgfpathlineto{\pgfqpoint{4.430516in}{2.416310in}}%
\pgfpathlineto{\pgfqpoint{4.416630in}{2.410404in}}%
\pgfpathlineto{\pgfqpoint{4.402757in}{2.404685in}}%
\pgfpathlineto{\pgfqpoint{4.394935in}{2.395034in}}%
\pgfpathlineto{\pgfqpoint{4.387107in}{2.385319in}}%
\pgfpathlineto{\pgfqpoint{4.379274in}{2.375541in}}%
\pgfpathlineto{\pgfqpoint{4.371436in}{2.365698in}}%
\pgfpathclose%
\pgfusepath{fill}%
\end{pgfscope}%
\begin{pgfscope}%
\pgfpathrectangle{\pgfqpoint{1.150000in}{0.150000in}}{\pgfqpoint{5.700000in}{5.700000in}}%
\pgfusepath{clip}%
\pgfsetbuttcap%
\pgfsetroundjoin%
\definecolor{currentfill}{rgb}{0.146180,0.515413,0.556823}%
\pgfsetfillcolor{currentfill}%
\pgfsetfillopacity{0.800000}%
\pgfsetlinewidth{0.000000pt}%
\definecolor{currentstroke}{rgb}{0.000000,0.000000,0.000000}%
\pgfsetstrokecolor{currentstroke}%
\pgfsetdash{}{0pt}%
\pgfpathmoveto{\pgfqpoint{5.302747in}{3.054392in}}%
\pgfpathlineto{\pgfqpoint{5.317102in}{3.064378in}}%
\pgfpathlineto{\pgfqpoint{5.331474in}{3.074542in}}%
\pgfpathlineto{\pgfqpoint{5.345865in}{3.084885in}}%
\pgfpathlineto{\pgfqpoint{5.360274in}{3.095406in}}%
\pgfpathlineto{\pgfqpoint{5.367673in}{3.099934in}}%
\pgfpathlineto{\pgfqpoint{5.375067in}{3.104464in}}%
\pgfpathlineto{\pgfqpoint{5.382454in}{3.108999in}}%
\pgfpathlineto{\pgfqpoint{5.389835in}{3.113546in}}%
\pgfpathlineto{\pgfqpoint{5.375447in}{3.103503in}}%
\pgfpathlineto{\pgfqpoint{5.361078in}{3.093638in}}%
\pgfpathlineto{\pgfqpoint{5.346726in}{3.083950in}}%
\pgfpathlineto{\pgfqpoint{5.332393in}{3.074439in}}%
\pgfpathlineto{\pgfqpoint{5.324990in}{3.069404in}}%
\pgfpathlineto{\pgfqpoint{5.317581in}{3.064388in}}%
\pgfpathlineto{\pgfqpoint{5.310167in}{3.059386in}}%
\pgfpathlineto{\pgfqpoint{5.302747in}{3.054392in}}%
\pgfpathclose%
\pgfusepath{fill}%
\end{pgfscope}%
\begin{pgfscope}%
\pgfpathrectangle{\pgfqpoint{1.150000in}{0.150000in}}{\pgfqpoint{5.700000in}{5.700000in}}%
\pgfusepath{clip}%
\pgfsetbuttcap%
\pgfsetroundjoin%
\definecolor{currentfill}{rgb}{0.283091,0.110553,0.431554}%
\pgfsetfillcolor{currentfill}%
\pgfsetfillopacity{0.800000}%
\pgfsetlinewidth{0.000000pt}%
\definecolor{currentstroke}{rgb}{0.000000,0.000000,0.000000}%
\pgfsetstrokecolor{currentstroke}%
\pgfsetdash{}{0pt}%
\pgfpathmoveto{\pgfqpoint{3.818993in}{1.988397in}}%
\pgfpathlineto{\pgfqpoint{3.832671in}{1.989225in}}%
\pgfpathlineto{\pgfqpoint{3.846356in}{1.990248in}}%
\pgfpathlineto{\pgfqpoint{3.860050in}{1.991468in}}%
\pgfpathlineto{\pgfqpoint{3.873752in}{1.992882in}}%
\pgfpathlineto{\pgfqpoint{3.881766in}{2.003880in}}%
\pgfpathlineto{\pgfqpoint{3.889775in}{2.014854in}}%
\pgfpathlineto{\pgfqpoint{3.897779in}{2.025803in}}%
\pgfpathlineto{\pgfqpoint{3.905777in}{2.036724in}}%
\pgfpathlineto{\pgfqpoint{3.892083in}{2.035160in}}%
\pgfpathlineto{\pgfqpoint{3.878397in}{2.033790in}}%
\pgfpathlineto{\pgfqpoint{3.864719in}{2.032617in}}%
\pgfpathlineto{\pgfqpoint{3.851049in}{2.031639in}}%
\pgfpathlineto{\pgfqpoint{3.843043in}{2.020856in}}%
\pgfpathlineto{\pgfqpoint{3.835031in}{2.010054in}}%
\pgfpathlineto{\pgfqpoint{3.827015in}{1.999234in}}%
\pgfpathlineto{\pgfqpoint{3.818993in}{1.988397in}}%
\pgfpathclose%
\pgfusepath{fill}%
\end{pgfscope}%
\begin{pgfscope}%
\pgfpathrectangle{\pgfqpoint{1.150000in}{0.150000in}}{\pgfqpoint{5.700000in}{5.700000in}}%
\pgfusepath{clip}%
\pgfsetbuttcap%
\pgfsetroundjoin%
\definecolor{currentfill}{rgb}{0.281924,0.089666,0.412415}%
\pgfsetfillcolor{currentfill}%
\pgfsetfillopacity{0.800000}%
\pgfsetlinewidth{0.000000pt}%
\definecolor{currentstroke}{rgb}{0.000000,0.000000,0.000000}%
\pgfsetstrokecolor{currentstroke}%
\pgfsetdash{}{0pt}%
\pgfpathmoveto{\pgfqpoint{3.732187in}{1.944297in}}%
\pgfpathlineto{\pgfqpoint{3.745844in}{1.944153in}}%
\pgfpathlineto{\pgfqpoint{3.759507in}{1.944207in}}%
\pgfpathlineto{\pgfqpoint{3.773178in}{1.944460in}}%
\pgfpathlineto{\pgfqpoint{3.786856in}{1.944909in}}%
\pgfpathlineto{\pgfqpoint{3.794898in}{1.955799in}}%
\pgfpathlineto{\pgfqpoint{3.802935in}{1.966679in}}%
\pgfpathlineto{\pgfqpoint{3.810967in}{1.977545in}}%
\pgfpathlineto{\pgfqpoint{3.818993in}{1.988397in}}%
\pgfpathlineto{\pgfqpoint{3.805323in}{1.987766in}}%
\pgfpathlineto{\pgfqpoint{3.791661in}{1.987332in}}%
\pgfpathlineto{\pgfqpoint{3.778006in}{1.987096in}}%
\pgfpathlineto{\pgfqpoint{3.764359in}{1.987059in}}%
\pgfpathlineto{\pgfqpoint{3.756323in}{1.976376in}}%
\pgfpathlineto{\pgfqpoint{3.748283in}{1.965688in}}%
\pgfpathlineto{\pgfqpoint{3.740238in}{1.954994in}}%
\pgfpathlineto{\pgfqpoint{3.732187in}{1.944297in}}%
\pgfpathclose%
\pgfusepath{fill}%
\end{pgfscope}%
\begin{pgfscope}%
\pgfpathrectangle{\pgfqpoint{1.150000in}{0.150000in}}{\pgfqpoint{5.700000in}{5.700000in}}%
\pgfusepath{clip}%
\pgfsetbuttcap%
\pgfsetroundjoin%
\definecolor{currentfill}{rgb}{0.273809,0.031497,0.358853}%
\pgfsetfillcolor{currentfill}%
\pgfsetfillopacity{0.800000}%
\pgfsetlinewidth{0.000000pt}%
\definecolor{currentstroke}{rgb}{0.000000,0.000000,0.000000}%
\pgfsetstrokecolor{currentstroke}%
\pgfsetdash{}{0pt}%
\pgfpathmoveto{\pgfqpoint{3.187824in}{1.858829in}}%
\pgfpathlineto{\pgfqpoint{3.201443in}{1.851433in}}%
\pgfpathlineto{\pgfqpoint{3.215063in}{1.844259in}}%
\pgfpathlineto{\pgfqpoint{3.228684in}{1.837308in}}%
\pgfpathlineto{\pgfqpoint{3.242305in}{1.830577in}}%
\pgfpathlineto{\pgfqpoint{3.250562in}{1.838799in}}%
\pgfpathlineto{\pgfqpoint{3.258812in}{1.847112in}}%
\pgfpathlineto{\pgfqpoint{3.267053in}{1.855513in}}%
\pgfpathlineto{\pgfqpoint{3.275288in}{1.863998in}}%
\pgfpathlineto{\pgfqpoint{3.261685in}{1.870389in}}%
\pgfpathlineto{\pgfqpoint{3.248084in}{1.877000in}}%
\pgfpathlineto{\pgfqpoint{3.234485in}{1.883833in}}%
\pgfpathlineto{\pgfqpoint{3.220886in}{1.890888in}}%
\pgfpathlineto{\pgfqpoint{3.212632in}{1.882732in}}%
\pgfpathlineto{\pgfqpoint{3.204371in}{1.874668in}}%
\pgfpathlineto{\pgfqpoint{3.196102in}{1.866699in}}%
\pgfpathlineto{\pgfqpoint{3.187824in}{1.858829in}}%
\pgfpathclose%
\pgfusepath{fill}%
\end{pgfscope}%
\begin{pgfscope}%
\pgfpathrectangle{\pgfqpoint{1.150000in}{0.150000in}}{\pgfqpoint{5.700000in}{5.700000in}}%
\pgfusepath{clip}%
\pgfsetbuttcap%
\pgfsetroundjoin%
\definecolor{currentfill}{rgb}{0.283072,0.130895,0.449241}%
\pgfsetfillcolor{currentfill}%
\pgfsetfillopacity{0.800000}%
\pgfsetlinewidth{0.000000pt}%
\definecolor{currentstroke}{rgb}{0.000000,0.000000,0.000000}%
\pgfsetstrokecolor{currentstroke}%
\pgfsetdash{}{0pt}%
\pgfpathmoveto{\pgfqpoint{3.905777in}{2.036724in}}%
\pgfpathlineto{\pgfqpoint{3.919481in}{2.038483in}}%
\pgfpathlineto{\pgfqpoint{3.933192in}{2.040436in}}%
\pgfpathlineto{\pgfqpoint{3.946913in}{2.042582in}}%
\pgfpathlineto{\pgfqpoint{3.960643in}{2.044922in}}%
\pgfpathlineto{\pgfqpoint{3.968630in}{2.055946in}}%
\pgfpathlineto{\pgfqpoint{3.976612in}{2.066933in}}%
\pgfpathlineto{\pgfqpoint{3.984589in}{2.077883in}}%
\pgfpathlineto{\pgfqpoint{3.992562in}{2.088794in}}%
\pgfpathlineto{\pgfqpoint{3.978838in}{2.086336in}}%
\pgfpathlineto{\pgfqpoint{3.965124in}{2.084071in}}%
\pgfpathlineto{\pgfqpoint{3.951419in}{2.082000in}}%
\pgfpathlineto{\pgfqpoint{3.937723in}{2.080123in}}%
\pgfpathlineto{\pgfqpoint{3.929744in}{2.069318in}}%
\pgfpathlineto{\pgfqpoint{3.921760in}{2.058482in}}%
\pgfpathlineto{\pgfqpoint{3.913771in}{2.047617in}}%
\pgfpathlineto{\pgfqpoint{3.905777in}{2.036724in}}%
\pgfpathclose%
\pgfusepath{fill}%
\end{pgfscope}%
\begin{pgfscope}%
\pgfpathrectangle{\pgfqpoint{1.150000in}{0.150000in}}{\pgfqpoint{5.700000in}{5.700000in}}%
\pgfusepath{clip}%
\pgfsetbuttcap%
\pgfsetroundjoin%
\definecolor{currentfill}{rgb}{0.279566,0.067836,0.391917}%
\pgfsetfillcolor{currentfill}%
\pgfsetfillopacity{0.800000}%
\pgfsetlinewidth{0.000000pt}%
\definecolor{currentstroke}{rgb}{0.000000,0.000000,0.000000}%
\pgfsetstrokecolor{currentstroke}%
\pgfsetdash{}{0pt}%
\pgfpathmoveto{\pgfqpoint{3.645334in}{1.904930in}}%
\pgfpathlineto{\pgfqpoint{3.658974in}{1.903773in}}%
\pgfpathlineto{\pgfqpoint{3.672621in}{1.902818in}}%
\pgfpathlineto{\pgfqpoint{3.686273in}{1.902062in}}%
\pgfpathlineto{\pgfqpoint{3.699932in}{1.901506in}}%
\pgfpathlineto{\pgfqpoint{3.708004in}{1.912201in}}%
\pgfpathlineto{\pgfqpoint{3.716070in}{1.922899in}}%
\pgfpathlineto{\pgfqpoint{3.724131in}{1.933598in}}%
\pgfpathlineto{\pgfqpoint{3.732187in}{1.944297in}}%
\pgfpathlineto{\pgfqpoint{3.718538in}{1.944640in}}%
\pgfpathlineto{\pgfqpoint{3.704895in}{1.945182in}}%
\pgfpathlineto{\pgfqpoint{3.691258in}{1.945925in}}%
\pgfpathlineto{\pgfqpoint{3.677628in}{1.946868in}}%
\pgfpathlineto{\pgfqpoint{3.669563in}{1.936370in}}%
\pgfpathlineto{\pgfqpoint{3.661492in}{1.925880in}}%
\pgfpathlineto{\pgfqpoint{3.653416in}{1.915400in}}%
\pgfpathlineto{\pgfqpoint{3.645334in}{1.904930in}}%
\pgfpathclose%
\pgfusepath{fill}%
\end{pgfscope}%
\begin{pgfscope}%
\pgfpathrectangle{\pgfqpoint{1.150000in}{0.150000in}}{\pgfqpoint{5.700000in}{5.700000in}}%
\pgfusepath{clip}%
\pgfsetbuttcap%
\pgfsetroundjoin%
\definecolor{currentfill}{rgb}{0.272594,0.025563,0.353093}%
\pgfsetfillcolor{currentfill}%
\pgfsetfillopacity{0.800000}%
\pgfsetlinewidth{0.000000pt}%
\definecolor{currentstroke}{rgb}{0.000000,0.000000,0.000000}%
\pgfsetstrokecolor{currentstroke}%
\pgfsetdash{}{0pt}%
\pgfpathmoveto{\pgfqpoint{3.329715in}{1.840619in}}%
\pgfpathlineto{\pgfqpoint{3.343328in}{1.835314in}}%
\pgfpathlineto{\pgfqpoint{3.356942in}{1.830224in}}%
\pgfpathlineto{\pgfqpoint{3.370560in}{1.825346in}}%
\pgfpathlineto{\pgfqpoint{3.384180in}{1.820681in}}%
\pgfpathlineto{\pgfqpoint{3.392373in}{1.829888in}}%
\pgfpathlineto{\pgfqpoint{3.400559in}{1.839158in}}%
\pgfpathlineto{\pgfqpoint{3.408738in}{1.848489in}}%
\pgfpathlineto{\pgfqpoint{3.416911in}{1.857876in}}%
\pgfpathlineto{\pgfqpoint{3.403307in}{1.862233in}}%
\pgfpathlineto{\pgfqpoint{3.389705in}{1.866802in}}%
\pgfpathlineto{\pgfqpoint{3.376107in}{1.871585in}}%
\pgfpathlineto{\pgfqpoint{3.362511in}{1.876581in}}%
\pgfpathlineto{\pgfqpoint{3.354322in}{1.867490in}}%
\pgfpathlineto{\pgfqpoint{3.346127in}{1.858464in}}%
\pgfpathlineto{\pgfqpoint{3.337925in}{1.849506in}}%
\pgfpathlineto{\pgfqpoint{3.329715in}{1.840619in}}%
\pgfpathclose%
\pgfusepath{fill}%
\end{pgfscope}%
\begin{pgfscope}%
\pgfpathrectangle{\pgfqpoint{1.150000in}{0.150000in}}{\pgfqpoint{5.700000in}{5.700000in}}%
\pgfusepath{clip}%
\pgfsetbuttcap%
\pgfsetroundjoin%
\definecolor{currentfill}{rgb}{0.190631,0.407061,0.556089}%
\pgfsetfillcolor{currentfill}%
\pgfsetfillopacity{0.800000}%
\pgfsetlinewidth{0.000000pt}%
\definecolor{currentstroke}{rgb}{0.000000,0.000000,0.000000}%
\pgfsetstrokecolor{currentstroke}%
\pgfsetdash{}{0pt}%
\pgfpathmoveto{\pgfqpoint{4.837212in}{2.719551in}}%
\pgfpathlineto{\pgfqpoint{4.851324in}{2.728086in}}%
\pgfpathlineto{\pgfqpoint{4.865452in}{2.736802in}}%
\pgfpathlineto{\pgfqpoint{4.879595in}{2.745701in}}%
\pgfpathlineto{\pgfqpoint{4.893755in}{2.754781in}}%
\pgfpathlineto{\pgfqpoint{4.901395in}{2.762011in}}%
\pgfpathlineto{\pgfqpoint{4.909029in}{2.769180in}}%
\pgfpathlineto{\pgfqpoint{4.916656in}{2.776290in}}%
\pgfpathlineto{\pgfqpoint{4.924277in}{2.783346in}}%
\pgfpathlineto{\pgfqpoint{4.910129in}{2.774541in}}%
\pgfpathlineto{\pgfqpoint{4.895997in}{2.765917in}}%
\pgfpathlineto{\pgfqpoint{4.881881in}{2.757475in}}%
\pgfpathlineto{\pgfqpoint{4.867781in}{2.749214in}}%
\pgfpathlineto{\pgfqpoint{4.860148in}{2.741872in}}%
\pgfpathlineto{\pgfqpoint{4.852509in}{2.734483in}}%
\pgfpathlineto{\pgfqpoint{4.844864in}{2.727044in}}%
\pgfpathlineto{\pgfqpoint{4.837212in}{2.719551in}}%
\pgfpathclose%
\pgfusepath{fill}%
\end{pgfscope}%
\begin{pgfscope}%
\pgfpathrectangle{\pgfqpoint{1.150000in}{0.150000in}}{\pgfqpoint{5.700000in}{5.700000in}}%
\pgfusepath{clip}%
\pgfsetbuttcap%
\pgfsetroundjoin%
\definecolor{currentfill}{rgb}{0.137770,0.537492,0.554906}%
\pgfsetfillcolor{currentfill}%
\pgfsetfillopacity{0.800000}%
\pgfsetlinewidth{0.000000pt}%
\definecolor{currentstroke}{rgb}{0.000000,0.000000,0.000000}%
\pgfsetstrokecolor{currentstroke}%
\pgfsetdash{}{0pt}%
\pgfpathmoveto{\pgfqpoint{5.389835in}{3.113546in}}%
\pgfpathlineto{\pgfqpoint{5.404240in}{3.123767in}}%
\pgfpathlineto{\pgfqpoint{5.418664in}{3.134166in}}%
\pgfpathlineto{\pgfqpoint{5.433106in}{3.144742in}}%
\pgfpathlineto{\pgfqpoint{5.447567in}{3.155497in}}%
\pgfpathlineto{\pgfqpoint{5.454919in}{3.159561in}}%
\pgfpathlineto{\pgfqpoint{5.462266in}{3.163641in}}%
\pgfpathlineto{\pgfqpoint{5.469606in}{3.167740in}}%
\pgfpathlineto{\pgfqpoint{5.476941in}{3.171866in}}%
\pgfpathlineto{\pgfqpoint{5.462504in}{3.161624in}}%
\pgfpathlineto{\pgfqpoint{5.448085in}{3.151559in}}%
\pgfpathlineto{\pgfqpoint{5.433684in}{3.141670in}}%
\pgfpathlineto{\pgfqpoint{5.419302in}{3.131959in}}%
\pgfpathlineto{\pgfqpoint{5.411943in}{3.127311in}}%
\pgfpathlineto{\pgfqpoint{5.404579in}{3.122697in}}%
\pgfpathlineto{\pgfqpoint{5.397210in}{3.118110in}}%
\pgfpathlineto{\pgfqpoint{5.389835in}{3.113546in}}%
\pgfpathclose%
\pgfusepath{fill}%
\end{pgfscope}%
\begin{pgfscope}%
\pgfpathrectangle{\pgfqpoint{1.150000in}{0.150000in}}{\pgfqpoint{5.700000in}{5.700000in}}%
\pgfusepath{clip}%
\pgfsetbuttcap%
\pgfsetroundjoin%
\definecolor{currentfill}{rgb}{0.281412,0.155834,0.469201}%
\pgfsetfillcolor{currentfill}%
\pgfsetfillopacity{0.800000}%
\pgfsetlinewidth{0.000000pt}%
\definecolor{currentstroke}{rgb}{0.000000,0.000000,0.000000}%
\pgfsetstrokecolor{currentstroke}%
\pgfsetdash{}{0pt}%
\pgfpathmoveto{\pgfqpoint{3.992562in}{2.088794in}}%
\pgfpathlineto{\pgfqpoint{4.006294in}{2.091445in}}%
\pgfpathlineto{\pgfqpoint{4.020036in}{2.094288in}}%
\pgfpathlineto{\pgfqpoint{4.033788in}{2.097323in}}%
\pgfpathlineto{\pgfqpoint{4.047550in}{2.100549in}}%
\pgfpathlineto{\pgfqpoint{4.055511in}{2.111520in}}%
\pgfpathlineto{\pgfqpoint{4.063467in}{2.122444in}}%
\pgfpathlineto{\pgfqpoint{4.071418in}{2.133320in}}%
\pgfpathlineto{\pgfqpoint{4.079364in}{2.144148in}}%
\pgfpathlineto{\pgfqpoint{4.065609in}{2.140835in}}%
\pgfpathlineto{\pgfqpoint{4.051863in}{2.137714in}}%
\pgfpathlineto{\pgfqpoint{4.038127in}{2.134784in}}%
\pgfpathlineto{\pgfqpoint{4.024401in}{2.132047in}}%
\pgfpathlineto{\pgfqpoint{4.016448in}{2.121294in}}%
\pgfpathlineto{\pgfqpoint{4.008491in}{2.110500in}}%
\pgfpathlineto{\pgfqpoint{4.000529in}{2.099667in}}%
\pgfpathlineto{\pgfqpoint{3.992562in}{2.088794in}}%
\pgfpathclose%
\pgfusepath{fill}%
\end{pgfscope}%
\begin{pgfscope}%
\pgfpathrectangle{\pgfqpoint{1.150000in}{0.150000in}}{\pgfqpoint{5.700000in}{5.700000in}}%
\pgfusepath{clip}%
\pgfsetbuttcap%
\pgfsetroundjoin%
\definecolor{currentfill}{rgb}{0.239346,0.300855,0.540844}%
\pgfsetfillcolor{currentfill}%
\pgfsetfillopacity{0.800000}%
\pgfsetlinewidth{0.000000pt}%
\definecolor{currentstroke}{rgb}{0.000000,0.000000,0.000000}%
\pgfsetstrokecolor{currentstroke}%
\pgfsetdash{}{0pt}%
\pgfpathmoveto{\pgfqpoint{4.458325in}{2.428680in}}%
\pgfpathlineto{\pgfqpoint{4.472250in}{2.435144in}}%
\pgfpathlineto{\pgfqpoint{4.486188in}{2.441793in}}%
\pgfpathlineto{\pgfqpoint{4.500139in}{2.448628in}}%
\pgfpathlineto{\pgfqpoint{4.514103in}{2.455649in}}%
\pgfpathlineto{\pgfqpoint{4.521906in}{2.465048in}}%
\pgfpathlineto{\pgfqpoint{4.529703in}{2.474372in}}%
\pgfpathlineto{\pgfqpoint{4.537494in}{2.483621in}}%
\pgfpathlineto{\pgfqpoint{4.545279in}{2.492797in}}%
\pgfpathlineto{\pgfqpoint{4.531322in}{2.485885in}}%
\pgfpathlineto{\pgfqpoint{4.517377in}{2.479158in}}%
\pgfpathlineto{\pgfqpoint{4.503447in}{2.472617in}}%
\pgfpathlineto{\pgfqpoint{4.489529in}{2.466260in}}%
\pgfpathlineto{\pgfqpoint{4.481737in}{2.456964in}}%
\pgfpathlineto{\pgfqpoint{4.473939in}{2.447603in}}%
\pgfpathlineto{\pgfqpoint{4.466135in}{2.438175in}}%
\pgfpathlineto{\pgfqpoint{4.458325in}{2.428680in}}%
\pgfpathclose%
\pgfusepath{fill}%
\end{pgfscope}%
\begin{pgfscope}%
\pgfpathrectangle{\pgfqpoint{1.150000in}{0.150000in}}{\pgfqpoint{5.700000in}{5.700000in}}%
\pgfusepath{clip}%
\pgfsetbuttcap%
\pgfsetroundjoin%
\definecolor{currentfill}{rgb}{0.277018,0.050344,0.375715}%
\pgfsetfillcolor{currentfill}%
\pgfsetfillopacity{0.800000}%
\pgfsetlinewidth{0.000000pt}%
\definecolor{currentstroke}{rgb}{0.000000,0.000000,0.000000}%
\pgfsetstrokecolor{currentstroke}%
\pgfsetdash{}{0pt}%
\pgfpathmoveto{\pgfqpoint{3.558404in}{1.870828in}}%
\pgfpathlineto{\pgfqpoint{3.572033in}{1.868617in}}%
\pgfpathlineto{\pgfqpoint{3.585668in}{1.866609in}}%
\pgfpathlineto{\pgfqpoint{3.599307in}{1.864804in}}%
\pgfpathlineto{\pgfqpoint{3.612953in}{1.863201in}}%
\pgfpathlineto{\pgfqpoint{3.621056in}{1.873607in}}%
\pgfpathlineto{\pgfqpoint{3.629154in}{1.884031in}}%
\pgfpathlineto{\pgfqpoint{3.637247in}{1.894473in}}%
\pgfpathlineto{\pgfqpoint{3.645334in}{1.904930in}}%
\pgfpathlineto{\pgfqpoint{3.631700in}{1.906288in}}%
\pgfpathlineto{\pgfqpoint{3.618071in}{1.907848in}}%
\pgfpathlineto{\pgfqpoint{3.604448in}{1.909612in}}%
\pgfpathlineto{\pgfqpoint{3.590831in}{1.911579in}}%
\pgfpathlineto{\pgfqpoint{3.582733in}{1.901355in}}%
\pgfpathlineto{\pgfqpoint{3.574629in}{1.891154in}}%
\pgfpathlineto{\pgfqpoint{3.566520in}{1.880977in}}%
\pgfpathlineto{\pgfqpoint{3.558404in}{1.870828in}}%
\pgfpathclose%
\pgfusepath{fill}%
\end{pgfscope}%
\begin{pgfscope}%
\pgfpathrectangle{\pgfqpoint{1.150000in}{0.150000in}}{\pgfqpoint{5.700000in}{5.700000in}}%
\pgfusepath{clip}%
\pgfsetbuttcap%
\pgfsetroundjoin%
\definecolor{currentfill}{rgb}{0.278012,0.180367,0.486697}%
\pgfsetfillcolor{currentfill}%
\pgfsetfillopacity{0.800000}%
\pgfsetlinewidth{0.000000pt}%
\definecolor{currentstroke}{rgb}{0.000000,0.000000,0.000000}%
\pgfsetstrokecolor{currentstroke}%
\pgfsetdash{}{0pt}%
\pgfpathmoveto{\pgfqpoint{2.682810in}{2.187809in}}%
\pgfpathlineto{\pgfqpoint{2.696588in}{2.171700in}}%
\pgfpathlineto{\pgfqpoint{2.710358in}{2.155866in}}%
\pgfpathlineto{\pgfqpoint{2.724122in}{2.140307in}}%
\pgfpathlineto{\pgfqpoint{2.737878in}{2.125020in}}%
\pgfpathlineto{\pgfqpoint{2.746428in}{2.129020in}}%
\pgfpathlineto{\pgfqpoint{2.754965in}{2.133204in}}%
\pgfpathlineto{\pgfqpoint{2.763489in}{2.137570in}}%
\pgfpathlineto{\pgfqpoint{2.772002in}{2.142111in}}%
\pgfpathlineto{\pgfqpoint{2.758279in}{2.156982in}}%
\pgfpathlineto{\pgfqpoint{2.744550in}{2.172124in}}%
\pgfpathlineto{\pgfqpoint{2.730814in}{2.187539in}}%
\pgfpathlineto{\pgfqpoint{2.717071in}{2.203230in}}%
\pgfpathlineto{\pgfqpoint{2.708525in}{2.199093in}}%
\pgfpathlineto{\pgfqpoint{2.699966in}{2.195142in}}%
\pgfpathlineto{\pgfqpoint{2.691395in}{2.191379in}}%
\pgfpathlineto{\pgfqpoint{2.682810in}{2.187809in}}%
\pgfpathclose%
\pgfusepath{fill}%
\end{pgfscope}%
\begin{pgfscope}%
\pgfpathrectangle{\pgfqpoint{1.150000in}{0.150000in}}{\pgfqpoint{5.700000in}{5.700000in}}%
\pgfusepath{clip}%
\pgfsetbuttcap%
\pgfsetroundjoin%
\definecolor{currentfill}{rgb}{0.271828,0.209303,0.504434}%
\pgfsetfillcolor{currentfill}%
\pgfsetfillopacity{0.800000}%
\pgfsetlinewidth{0.000000pt}%
\definecolor{currentstroke}{rgb}{0.000000,0.000000,0.000000}%
\pgfsetstrokecolor{currentstroke}%
\pgfsetdash{}{0pt}%
\pgfpathmoveto{\pgfqpoint{2.627621in}{2.255056in}}%
\pgfpathlineto{\pgfqpoint{2.641431in}{2.237818in}}%
\pgfpathlineto{\pgfqpoint{2.655232in}{2.220866in}}%
\pgfpathlineto{\pgfqpoint{2.669025in}{2.204197in}}%
\pgfpathlineto{\pgfqpoint{2.682810in}{2.187809in}}%
\pgfpathlineto{\pgfqpoint{2.691395in}{2.191379in}}%
\pgfpathlineto{\pgfqpoint{2.699966in}{2.195142in}}%
\pgfpathlineto{\pgfqpoint{2.708525in}{2.199093in}}%
\pgfpathlineto{\pgfqpoint{2.717071in}{2.203230in}}%
\pgfpathlineto{\pgfqpoint{2.703321in}{2.219199in}}%
\pgfpathlineto{\pgfqpoint{2.689564in}{2.235448in}}%
\pgfpathlineto{\pgfqpoint{2.675799in}{2.251979in}}%
\pgfpathlineto{\pgfqpoint{2.662026in}{2.268795in}}%
\pgfpathlineto{\pgfqpoint{2.653444in}{2.265066in}}%
\pgfpathlineto{\pgfqpoint{2.644850in}{2.261530in}}%
\pgfpathlineto{\pgfqpoint{2.636242in}{2.258192in}}%
\pgfpathlineto{\pgfqpoint{2.627621in}{2.255056in}}%
\pgfpathclose%
\pgfusepath{fill}%
\end{pgfscope}%
\begin{pgfscope}%
\pgfpathrectangle{\pgfqpoint{1.150000in}{0.150000in}}{\pgfqpoint{5.700000in}{5.700000in}}%
\pgfusepath{clip}%
\pgfsetbuttcap%
\pgfsetroundjoin%
\definecolor{currentfill}{rgb}{0.277941,0.056324,0.381191}%
\pgfsetfillcolor{currentfill}%
\pgfsetfillopacity{0.800000}%
\pgfsetlinewidth{0.000000pt}%
\definecolor{currentstroke}{rgb}{0.000000,0.000000,0.000000}%
\pgfsetstrokecolor{currentstroke}%
\pgfsetdash{}{0pt}%
\pgfpathmoveto{\pgfqpoint{3.045498in}{1.898739in}}%
\pgfpathlineto{\pgfqpoint{3.059143in}{1.889139in}}%
\pgfpathlineto{\pgfqpoint{3.072787in}{1.879773in}}%
\pgfpathlineto{\pgfqpoint{3.086429in}{1.870639in}}%
\pgfpathlineto{\pgfqpoint{3.100071in}{1.861736in}}%
\pgfpathlineto{\pgfqpoint{3.108404in}{1.868792in}}%
\pgfpathlineto{\pgfqpoint{3.116729in}{1.875968in}}%
\pgfpathlineto{\pgfqpoint{3.125045in}{1.883262in}}%
\pgfpathlineto{\pgfqpoint{3.133352in}{1.890668in}}%
\pgfpathlineto{\pgfqpoint{3.119733in}{1.899197in}}%
\pgfpathlineto{\pgfqpoint{3.106114in}{1.907957in}}%
\pgfpathlineto{\pgfqpoint{3.092495in}{1.916948in}}%
\pgfpathlineto{\pgfqpoint{3.078874in}{1.926172in}}%
\pgfpathlineto{\pgfqpoint{3.070544in}{1.919129in}}%
\pgfpathlineto{\pgfqpoint{3.062205in}{1.912206in}}%
\pgfpathlineto{\pgfqpoint{3.053856in}{1.905408in}}%
\pgfpathlineto{\pgfqpoint{3.045498in}{1.898739in}}%
\pgfpathclose%
\pgfusepath{fill}%
\end{pgfscope}%
\begin{pgfscope}%
\pgfpathrectangle{\pgfqpoint{1.150000in}{0.150000in}}{\pgfqpoint{5.700000in}{5.700000in}}%
\pgfusepath{clip}%
\pgfsetbuttcap%
\pgfsetroundjoin%
\definecolor{currentfill}{rgb}{0.194100,0.399323,0.555565}%
\pgfsetfillcolor{currentfill}%
\pgfsetfillopacity{0.800000}%
\pgfsetlinewidth{0.000000pt}%
\definecolor{currentstroke}{rgb}{0.000000,0.000000,0.000000}%
\pgfsetstrokecolor{currentstroke}%
\pgfsetdash{}{0pt}%
\pgfpathmoveto{\pgfqpoint{2.328287in}{2.764673in}}%
\pgfpathlineto{\pgfqpoint{2.342345in}{2.740200in}}%
\pgfpathlineto{\pgfqpoint{2.356387in}{2.716083in}}%
\pgfpathlineto{\pgfqpoint{2.370413in}{2.692318in}}%
\pgfpathlineto{\pgfqpoint{2.384425in}{2.668901in}}%
\pgfpathlineto{\pgfqpoint{2.393191in}{2.670777in}}%
\pgfpathlineto{\pgfqpoint{2.401941in}{2.672881in}}%
\pgfpathlineto{\pgfqpoint{2.410675in}{2.675212in}}%
\pgfpathlineto{\pgfqpoint{2.419394in}{2.677764in}}%
\pgfpathlineto{\pgfqpoint{2.405426in}{2.700775in}}%
\pgfpathlineto{\pgfqpoint{2.391443in}{2.724134in}}%
\pgfpathlineto{\pgfqpoint{2.377445in}{2.747844in}}%
\pgfpathlineto{\pgfqpoint{2.363431in}{2.771909in}}%
\pgfpathlineto{\pgfqpoint{2.354669in}{2.769750in}}%
\pgfpathlineto{\pgfqpoint{2.345891in}{2.767822in}}%
\pgfpathlineto{\pgfqpoint{2.337097in}{2.766129in}}%
\pgfpathlineto{\pgfqpoint{2.328287in}{2.764673in}}%
\pgfpathclose%
\pgfusepath{fill}%
\end{pgfscope}%
\begin{pgfscope}%
\pgfpathrectangle{\pgfqpoint{1.150000in}{0.150000in}}{\pgfqpoint{5.700000in}{5.700000in}}%
\pgfusepath{clip}%
\pgfsetbuttcap%
\pgfsetroundjoin%
\definecolor{currentfill}{rgb}{0.281412,0.155834,0.469201}%
\pgfsetfillcolor{currentfill}%
\pgfsetfillopacity{0.800000}%
\pgfsetlinewidth{0.000000pt}%
\definecolor{currentstroke}{rgb}{0.000000,0.000000,0.000000}%
\pgfsetstrokecolor{currentstroke}%
\pgfsetdash{}{0pt}%
\pgfpathmoveto{\pgfqpoint{2.737878in}{2.125020in}}%
\pgfpathlineto{\pgfqpoint{2.751629in}{2.110002in}}%
\pgfpathlineto{\pgfqpoint{2.765372in}{2.095253in}}%
\pgfpathlineto{\pgfqpoint{2.779110in}{2.080769in}}%
\pgfpathlineto{\pgfqpoint{2.792842in}{2.066549in}}%
\pgfpathlineto{\pgfqpoint{2.801358in}{2.070977in}}%
\pgfpathlineto{\pgfqpoint{2.809862in}{2.075581in}}%
\pgfpathlineto{\pgfqpoint{2.818354in}{2.080357in}}%
\pgfpathlineto{\pgfqpoint{2.826834in}{2.085302in}}%
\pgfpathlineto{\pgfqpoint{2.813134in}{2.099108in}}%
\pgfpathlineto{\pgfqpoint{2.799429in}{2.113176in}}%
\pgfpathlineto{\pgfqpoint{2.785718in}{2.127510in}}%
\pgfpathlineto{\pgfqpoint{2.772002in}{2.142111in}}%
\pgfpathlineto{\pgfqpoint{2.763489in}{2.137570in}}%
\pgfpathlineto{\pgfqpoint{2.754965in}{2.133204in}}%
\pgfpathlineto{\pgfqpoint{2.746428in}{2.129020in}}%
\pgfpathlineto{\pgfqpoint{2.737878in}{2.125020in}}%
\pgfpathclose%
\pgfusepath{fill}%
\end{pgfscope}%
\begin{pgfscope}%
\pgfpathrectangle{\pgfqpoint{1.150000in}{0.150000in}}{\pgfqpoint{5.700000in}{5.700000in}}%
\pgfusepath{clip}%
\pgfsetbuttcap%
\pgfsetroundjoin%
\definecolor{currentfill}{rgb}{0.129933,0.559582,0.551864}%
\pgfsetfillcolor{currentfill}%
\pgfsetfillopacity{0.800000}%
\pgfsetlinewidth{0.000000pt}%
\definecolor{currentstroke}{rgb}{0.000000,0.000000,0.000000}%
\pgfsetstrokecolor{currentstroke}%
\pgfsetdash{}{0pt}%
\pgfpathmoveto{\pgfqpoint{5.476941in}{3.171866in}}%
\pgfpathlineto{\pgfqpoint{5.491397in}{3.182286in}}%
\pgfpathlineto{\pgfqpoint{5.505871in}{3.192882in}}%
\pgfpathlineto{\pgfqpoint{5.520364in}{3.203656in}}%
\pgfpathlineto{\pgfqpoint{5.534877in}{3.214607in}}%
\pgfpathlineto{\pgfqpoint{5.542181in}{3.218231in}}%
\pgfpathlineto{\pgfqpoint{5.549479in}{3.221886in}}%
\pgfpathlineto{\pgfqpoint{5.556773in}{3.225576in}}%
\pgfpathlineto{\pgfqpoint{5.564061in}{3.229309in}}%
\pgfpathlineto{\pgfqpoint{5.549574in}{3.218904in}}%
\pgfpathlineto{\pgfqpoint{5.535107in}{3.208675in}}%
\pgfpathlineto{\pgfqpoint{5.520658in}{3.198623in}}%
\pgfpathlineto{\pgfqpoint{5.506227in}{3.188747in}}%
\pgfpathlineto{\pgfqpoint{5.498913in}{3.184458in}}%
\pgfpathlineto{\pgfqpoint{5.491594in}{3.180219in}}%
\pgfpathlineto{\pgfqpoint{5.484270in}{3.176024in}}%
\pgfpathlineto{\pgfqpoint{5.476941in}{3.171866in}}%
\pgfpathclose%
\pgfusepath{fill}%
\end{pgfscope}%
\begin{pgfscope}%
\pgfpathrectangle{\pgfqpoint{1.150000in}{0.150000in}}{\pgfqpoint{5.700000in}{5.700000in}}%
\pgfusepath{clip}%
\pgfsetbuttcap%
\pgfsetroundjoin%
\definecolor{currentfill}{rgb}{0.277134,0.185228,0.489898}%
\pgfsetfillcolor{currentfill}%
\pgfsetfillopacity{0.800000}%
\pgfsetlinewidth{0.000000pt}%
\definecolor{currentstroke}{rgb}{0.000000,0.000000,0.000000}%
\pgfsetstrokecolor{currentstroke}%
\pgfsetdash{}{0pt}%
\pgfpathmoveto{\pgfqpoint{4.079364in}{2.144148in}}%
\pgfpathlineto{\pgfqpoint{4.093130in}{2.147652in}}%
\pgfpathlineto{\pgfqpoint{4.106907in}{2.151346in}}%
\pgfpathlineto{\pgfqpoint{4.120693in}{2.155231in}}%
\pgfpathlineto{\pgfqpoint{4.134490in}{2.159306in}}%
\pgfpathlineto{\pgfqpoint{4.142426in}{2.170152in}}%
\pgfpathlineto{\pgfqpoint{4.150356in}{2.180941in}}%
\pgfpathlineto{\pgfqpoint{4.158281in}{2.191673in}}%
\pgfpathlineto{\pgfqpoint{4.166201in}{2.202349in}}%
\pgfpathlineto{\pgfqpoint{4.152410in}{2.198219in}}%
\pgfpathlineto{\pgfqpoint{4.138629in}{2.194280in}}%
\pgfpathlineto{\pgfqpoint{4.124858in}{2.190531in}}%
\pgfpathlineto{\pgfqpoint{4.111098in}{2.186972in}}%
\pgfpathlineto{\pgfqpoint{4.103172in}{2.176340in}}%
\pgfpathlineto{\pgfqpoint{4.095241in}{2.165658in}}%
\pgfpathlineto{\pgfqpoint{4.087305in}{2.154927in}}%
\pgfpathlineto{\pgfqpoint{4.079364in}{2.144148in}}%
\pgfpathclose%
\pgfusepath{fill}%
\end{pgfscope}%
\begin{pgfscope}%
\pgfpathrectangle{\pgfqpoint{1.150000in}{0.150000in}}{\pgfqpoint{5.700000in}{5.700000in}}%
\pgfusepath{clip}%
\pgfsetbuttcap%
\pgfsetroundjoin%
\definecolor{currentfill}{rgb}{0.263663,0.237631,0.518762}%
\pgfsetfillcolor{currentfill}%
\pgfsetfillopacity{0.800000}%
\pgfsetlinewidth{0.000000pt}%
\definecolor{currentstroke}{rgb}{0.000000,0.000000,0.000000}%
\pgfsetstrokecolor{currentstroke}%
\pgfsetdash{}{0pt}%
\pgfpathmoveto{\pgfqpoint{2.572292in}{2.326908in}}%
\pgfpathlineto{\pgfqpoint{2.586139in}{2.308505in}}%
\pgfpathlineto{\pgfqpoint{2.599975in}{2.290397in}}%
\pgfpathlineto{\pgfqpoint{2.613803in}{2.272581in}}%
\pgfpathlineto{\pgfqpoint{2.627621in}{2.255056in}}%
\pgfpathlineto{\pgfqpoint{2.636242in}{2.258192in}}%
\pgfpathlineto{\pgfqpoint{2.644850in}{2.261530in}}%
\pgfpathlineto{\pgfqpoint{2.653444in}{2.265066in}}%
\pgfpathlineto{\pgfqpoint{2.662026in}{2.268795in}}%
\pgfpathlineto{\pgfqpoint{2.648244in}{2.285898in}}%
\pgfpathlineto{\pgfqpoint{2.634454in}{2.303290in}}%
\pgfpathlineto{\pgfqpoint{2.620655in}{2.320975in}}%
\pgfpathlineto{\pgfqpoint{2.606847in}{2.338954in}}%
\pgfpathlineto{\pgfqpoint{2.598230in}{2.335635in}}%
\pgfpathlineto{\pgfqpoint{2.589598in}{2.332519in}}%
\pgfpathlineto{\pgfqpoint{2.580952in}{2.329609in}}%
\pgfpathlineto{\pgfqpoint{2.572292in}{2.326908in}}%
\pgfpathclose%
\pgfusepath{fill}%
\end{pgfscope}%
\begin{pgfscope}%
\pgfpathrectangle{\pgfqpoint{1.150000in}{0.150000in}}{\pgfqpoint{5.700000in}{5.700000in}}%
\pgfusepath{clip}%
\pgfsetbuttcap%
\pgfsetroundjoin%
\definecolor{currentfill}{rgb}{0.180629,0.429975,0.557282}%
\pgfsetfillcolor{currentfill}%
\pgfsetfillopacity{0.800000}%
\pgfsetlinewidth{0.000000pt}%
\definecolor{currentstroke}{rgb}{0.000000,0.000000,0.000000}%
\pgfsetstrokecolor{currentstroke}%
\pgfsetdash{}{0pt}%
\pgfpathmoveto{\pgfqpoint{4.924277in}{2.783346in}}%
\pgfpathlineto{\pgfqpoint{4.938441in}{2.792333in}}%
\pgfpathlineto{\pgfqpoint{4.952621in}{2.801501in}}%
\pgfpathlineto{\pgfqpoint{4.966817in}{2.810851in}}%
\pgfpathlineto{\pgfqpoint{4.981029in}{2.820382in}}%
\pgfpathlineto{\pgfqpoint{4.988631in}{2.827091in}}%
\pgfpathlineto{\pgfqpoint{4.996225in}{2.833743in}}%
\pgfpathlineto{\pgfqpoint{5.003813in}{2.840344in}}%
\pgfpathlineto{\pgfqpoint{5.011394in}{2.846896in}}%
\pgfpathlineto{\pgfqpoint{4.997194in}{2.837674in}}%
\pgfpathlineto{\pgfqpoint{4.983011in}{2.828634in}}%
\pgfpathlineto{\pgfqpoint{4.968844in}{2.819774in}}%
\pgfpathlineto{\pgfqpoint{4.954694in}{2.811094in}}%
\pgfpathlineto{\pgfqpoint{4.947099in}{2.804222in}}%
\pgfpathlineto{\pgfqpoint{4.939498in}{2.797309in}}%
\pgfpathlineto{\pgfqpoint{4.931891in}{2.790351in}}%
\pgfpathlineto{\pgfqpoint{4.924277in}{2.783346in}}%
\pgfpathclose%
\pgfusepath{fill}%
\end{pgfscope}%
\begin{pgfscope}%
\pgfpathrectangle{\pgfqpoint{1.150000in}{0.150000in}}{\pgfqpoint{5.700000in}{5.700000in}}%
\pgfusepath{clip}%
\pgfsetbuttcap%
\pgfsetroundjoin%
\definecolor{currentfill}{rgb}{0.283072,0.130895,0.449241}%
\pgfsetfillcolor{currentfill}%
\pgfsetfillopacity{0.800000}%
\pgfsetlinewidth{0.000000pt}%
\definecolor{currentstroke}{rgb}{0.000000,0.000000,0.000000}%
\pgfsetstrokecolor{currentstroke}%
\pgfsetdash{}{0pt}%
\pgfpathmoveto{\pgfqpoint{2.792842in}{2.066549in}}%
\pgfpathlineto{\pgfqpoint{2.806569in}{2.052591in}}%
\pgfpathlineto{\pgfqpoint{2.820290in}{2.038892in}}%
\pgfpathlineto{\pgfqpoint{2.834007in}{2.025452in}}%
\pgfpathlineto{\pgfqpoint{2.847718in}{2.012268in}}%
\pgfpathlineto{\pgfqpoint{2.856202in}{2.017122in}}%
\pgfpathlineto{\pgfqpoint{2.864674in}{2.022143in}}%
\pgfpathlineto{\pgfqpoint{2.873135in}{2.027329in}}%
\pgfpathlineto{\pgfqpoint{2.881585in}{2.032674in}}%
\pgfpathlineto{\pgfqpoint{2.867904in}{2.045446in}}%
\pgfpathlineto{\pgfqpoint{2.854219in}{2.058473in}}%
\pgfpathlineto{\pgfqpoint{2.840529in}{2.071758in}}%
\pgfpathlineto{\pgfqpoint{2.826834in}{2.085302in}}%
\pgfpathlineto{\pgfqpoint{2.818354in}{2.080357in}}%
\pgfpathlineto{\pgfqpoint{2.809862in}{2.075581in}}%
\pgfpathlineto{\pgfqpoint{2.801358in}{2.070977in}}%
\pgfpathlineto{\pgfqpoint{2.792842in}{2.066549in}}%
\pgfpathclose%
\pgfusepath{fill}%
\end{pgfscope}%
\begin{pgfscope}%
\pgfpathrectangle{\pgfqpoint{1.150000in}{0.150000in}}{\pgfqpoint{5.700000in}{5.700000in}}%
\pgfusepath{clip}%
\pgfsetbuttcap%
\pgfsetroundjoin%
\definecolor{currentfill}{rgb}{0.274952,0.037752,0.364543}%
\pgfsetfillcolor{currentfill}%
\pgfsetfillopacity{0.800000}%
\pgfsetlinewidth{0.000000pt}%
\definecolor{currentstroke}{rgb}{0.000000,0.000000,0.000000}%
\pgfsetstrokecolor{currentstroke}%
\pgfsetdash{}{0pt}%
\pgfpathmoveto{\pgfqpoint{3.471366in}{1.842550in}}%
\pgfpathlineto{\pgfqpoint{3.484989in}{1.839240in}}%
\pgfpathlineto{\pgfqpoint{3.498617in}{1.836137in}}%
\pgfpathlineto{\pgfqpoint{3.512249in}{1.833239in}}%
\pgfpathlineto{\pgfqpoint{3.525885in}{1.830547in}}%
\pgfpathlineto{\pgfqpoint{3.534024in}{1.840565in}}%
\pgfpathlineto{\pgfqpoint{3.542157in}{1.850620in}}%
\pgfpathlineto{\pgfqpoint{3.550283in}{1.860708in}}%
\pgfpathlineto{\pgfqpoint{3.558404in}{1.870828in}}%
\pgfpathlineto{\pgfqpoint{3.544780in}{1.873244in}}%
\pgfpathlineto{\pgfqpoint{3.531161in}{1.875865in}}%
\pgfpathlineto{\pgfqpoint{3.517547in}{1.878693in}}%
\pgfpathlineto{\pgfqpoint{3.503937in}{1.881727in}}%
\pgfpathlineto{\pgfqpoint{3.495803in}{1.871871in}}%
\pgfpathlineto{\pgfqpoint{3.487663in}{1.862055in}}%
\pgfpathlineto{\pgfqpoint{3.479518in}{1.852280in}}%
\pgfpathlineto{\pgfqpoint{3.471366in}{1.842550in}}%
\pgfpathclose%
\pgfusepath{fill}%
\end{pgfscope}%
\begin{pgfscope}%
\pgfpathrectangle{\pgfqpoint{1.150000in}{0.150000in}}{\pgfqpoint{5.700000in}{5.700000in}}%
\pgfusepath{clip}%
\pgfsetbuttcap%
\pgfsetroundjoin%
\definecolor{currentfill}{rgb}{0.225863,0.330805,0.547314}%
\pgfsetfillcolor{currentfill}%
\pgfsetfillopacity{0.800000}%
\pgfsetlinewidth{0.000000pt}%
\definecolor{currentstroke}{rgb}{0.000000,0.000000,0.000000}%
\pgfsetstrokecolor{currentstroke}%
\pgfsetdash{}{0pt}%
\pgfpathmoveto{\pgfqpoint{4.545279in}{2.492797in}}%
\pgfpathlineto{\pgfqpoint{4.559251in}{2.499894in}}%
\pgfpathlineto{\pgfqpoint{4.573236in}{2.507176in}}%
\pgfpathlineto{\pgfqpoint{4.587236in}{2.514643in}}%
\pgfpathlineto{\pgfqpoint{4.601249in}{2.522294in}}%
\pgfpathlineto{\pgfqpoint{4.609021in}{2.531270in}}%
\pgfpathlineto{\pgfqpoint{4.616787in}{2.540169in}}%
\pgfpathlineto{\pgfqpoint{4.624546in}{2.548992in}}%
\pgfpathlineto{\pgfqpoint{4.632300in}{2.557742in}}%
\pgfpathlineto{\pgfqpoint{4.618294in}{2.550233in}}%
\pgfpathlineto{\pgfqpoint{4.604302in}{2.542908in}}%
\pgfpathlineto{\pgfqpoint{4.590324in}{2.535767in}}%
\pgfpathlineto{\pgfqpoint{4.576360in}{2.528811in}}%
\pgfpathlineto{\pgfqpoint{4.568599in}{2.519908in}}%
\pgfpathlineto{\pgfqpoint{4.560832in}{2.510939in}}%
\pgfpathlineto{\pgfqpoint{4.553059in}{2.501903in}}%
\pgfpathlineto{\pgfqpoint{4.545279in}{2.492797in}}%
\pgfpathclose%
\pgfusepath{fill}%
\end{pgfscope}%
\begin{pgfscope}%
\pgfpathrectangle{\pgfqpoint{1.150000in}{0.150000in}}{\pgfqpoint{5.700000in}{5.700000in}}%
\pgfusepath{clip}%
\pgfsetbuttcap%
\pgfsetroundjoin%
\definecolor{currentfill}{rgb}{0.124395,0.578002,0.548287}%
\pgfsetfillcolor{currentfill}%
\pgfsetfillopacity{0.800000}%
\pgfsetlinewidth{0.000000pt}%
\definecolor{currentstroke}{rgb}{0.000000,0.000000,0.000000}%
\pgfsetstrokecolor{currentstroke}%
\pgfsetdash{}{0pt}%
\pgfpathmoveto{\pgfqpoint{5.564061in}{3.229309in}}%
\pgfpathlineto{\pgfqpoint{5.578566in}{3.239890in}}%
\pgfpathlineto{\pgfqpoint{5.593090in}{3.250648in}}%
\pgfpathlineto{\pgfqpoint{5.607634in}{3.261583in}}%
\pgfpathlineto{\pgfqpoint{5.622197in}{3.272695in}}%
\pgfpathlineto{\pgfqpoint{5.629452in}{3.275908in}}%
\pgfpathlineto{\pgfqpoint{5.636703in}{3.279168in}}%
\pgfpathlineto{\pgfqpoint{5.643948in}{3.282481in}}%
\pgfpathlineto{\pgfqpoint{5.651189in}{3.285854in}}%
\pgfpathlineto{\pgfqpoint{5.636654in}{3.275323in}}%
\pgfpathlineto{\pgfqpoint{5.622139in}{3.264967in}}%
\pgfpathlineto{\pgfqpoint{5.607642in}{3.254788in}}%
\pgfpathlineto{\pgfqpoint{5.593164in}{3.244784in}}%
\pgfpathlineto{\pgfqpoint{5.585895in}{3.240821in}}%
\pgfpathlineto{\pgfqpoint{5.578622in}{3.236925in}}%
\pgfpathlineto{\pgfqpoint{5.571344in}{3.233089in}}%
\pgfpathlineto{\pgfqpoint{5.564061in}{3.229309in}}%
\pgfpathclose%
\pgfusepath{fill}%
\end{pgfscope}%
\begin{pgfscope}%
\pgfpathrectangle{\pgfqpoint{1.150000in}{0.150000in}}{\pgfqpoint{5.700000in}{5.700000in}}%
\pgfusepath{clip}%
\pgfsetbuttcap%
\pgfsetroundjoin%
\definecolor{currentfill}{rgb}{0.252194,0.269783,0.531579}%
\pgfsetfillcolor{currentfill}%
\pgfsetfillopacity{0.800000}%
\pgfsetlinewidth{0.000000pt}%
\definecolor{currentstroke}{rgb}{0.000000,0.000000,0.000000}%
\pgfsetstrokecolor{currentstroke}%
\pgfsetdash{}{0pt}%
\pgfpathmoveto{\pgfqpoint{2.516806in}{2.403527in}}%
\pgfpathlineto{\pgfqpoint{2.530694in}{2.383916in}}%
\pgfpathlineto{\pgfqpoint{2.544570in}{2.364611in}}%
\pgfpathlineto{\pgfqpoint{2.558436in}{2.345609in}}%
\pgfpathlineto{\pgfqpoint{2.572292in}{2.326908in}}%
\pgfpathlineto{\pgfqpoint{2.580952in}{2.329609in}}%
\pgfpathlineto{\pgfqpoint{2.589598in}{2.332519in}}%
\pgfpathlineto{\pgfqpoint{2.598230in}{2.335635in}}%
\pgfpathlineto{\pgfqpoint{2.606847in}{2.338954in}}%
\pgfpathlineto{\pgfqpoint{2.593030in}{2.357230in}}%
\pgfpathlineto{\pgfqpoint{2.579203in}{2.375806in}}%
\pgfpathlineto{\pgfqpoint{2.565366in}{2.394684in}}%
\pgfpathlineto{\pgfqpoint{2.551518in}{2.413866in}}%
\pgfpathlineto{\pgfqpoint{2.542862in}{2.410961in}}%
\pgfpathlineto{\pgfqpoint{2.534191in}{2.408267in}}%
\pgfpathlineto{\pgfqpoint{2.525506in}{2.405788in}}%
\pgfpathlineto{\pgfqpoint{2.516806in}{2.403527in}}%
\pgfpathclose%
\pgfusepath{fill}%
\end{pgfscope}%
\begin{pgfscope}%
\pgfpathrectangle{\pgfqpoint{1.150000in}{0.150000in}}{\pgfqpoint{5.700000in}{5.700000in}}%
\pgfusepath{clip}%
\pgfsetbuttcap%
\pgfsetroundjoin%
\definecolor{currentfill}{rgb}{0.270595,0.214069,0.507052}%
\pgfsetfillcolor{currentfill}%
\pgfsetfillopacity{0.800000}%
\pgfsetlinewidth{0.000000pt}%
\definecolor{currentstroke}{rgb}{0.000000,0.000000,0.000000}%
\pgfsetstrokecolor{currentstroke}%
\pgfsetdash{}{0pt}%
\pgfpathmoveto{\pgfqpoint{4.166201in}{2.202349in}}%
\pgfpathlineto{\pgfqpoint{4.180004in}{2.206668in}}%
\pgfpathlineto{\pgfqpoint{4.193818in}{2.211176in}}%
\pgfpathlineto{\pgfqpoint{4.207643in}{2.215873in}}%
\pgfpathlineto{\pgfqpoint{4.221479in}{2.220759in}}%
\pgfpathlineto{\pgfqpoint{4.229388in}{2.231411in}}%
\pgfpathlineto{\pgfqpoint{4.237292in}{2.241999in}}%
\pgfpathlineto{\pgfqpoint{4.245191in}{2.252522in}}%
\pgfpathlineto{\pgfqpoint{4.253085in}{2.262981in}}%
\pgfpathlineto{\pgfqpoint{4.239254in}{2.258073in}}%
\pgfpathlineto{\pgfqpoint{4.225434in}{2.253354in}}%
\pgfpathlineto{\pgfqpoint{4.211626in}{2.248823in}}%
\pgfpathlineto{\pgfqpoint{4.197829in}{2.244482in}}%
\pgfpathlineto{\pgfqpoint{4.189930in}{2.234034in}}%
\pgfpathlineto{\pgfqpoint{4.182026in}{2.223529in}}%
\pgfpathlineto{\pgfqpoint{4.174116in}{2.212967in}}%
\pgfpathlineto{\pgfqpoint{4.166201in}{2.202349in}}%
\pgfpathclose%
\pgfusepath{fill}%
\end{pgfscope}%
\begin{pgfscope}%
\pgfpathrectangle{\pgfqpoint{1.150000in}{0.150000in}}{\pgfqpoint{5.700000in}{5.700000in}}%
\pgfusepath{clip}%
\pgfsetbuttcap%
\pgfsetroundjoin%
\definecolor{currentfill}{rgb}{0.272594,0.025563,0.353093}%
\pgfsetfillcolor{currentfill}%
\pgfsetfillopacity{0.800000}%
\pgfsetlinewidth{0.000000pt}%
\definecolor{currentstroke}{rgb}{0.000000,0.000000,0.000000}%
\pgfsetstrokecolor{currentstroke}%
\pgfsetdash{}{0pt}%
\pgfpathmoveto{\pgfqpoint{3.242305in}{1.830577in}}%
\pgfpathlineto{\pgfqpoint{3.255928in}{1.824065in}}%
\pgfpathlineto{\pgfqpoint{3.269553in}{1.817772in}}%
\pgfpathlineto{\pgfqpoint{3.283179in}{1.811696in}}%
\pgfpathlineto{\pgfqpoint{3.296807in}{1.805836in}}%
\pgfpathlineto{\pgfqpoint{3.305045in}{1.814410in}}%
\pgfpathlineto{\pgfqpoint{3.313276in}{1.823068in}}%
\pgfpathlineto{\pgfqpoint{3.321499in}{1.831805in}}%
\pgfpathlineto{\pgfqpoint{3.329715in}{1.840619in}}%
\pgfpathlineto{\pgfqpoint{3.316105in}{1.846138in}}%
\pgfpathlineto{\pgfqpoint{3.302498in}{1.851874in}}%
\pgfpathlineto{\pgfqpoint{3.288892in}{1.857827in}}%
\pgfpathlineto{\pgfqpoint{3.275288in}{1.863998in}}%
\pgfpathlineto{\pgfqpoint{3.267053in}{1.855513in}}%
\pgfpathlineto{\pgfqpoint{3.258812in}{1.847112in}}%
\pgfpathlineto{\pgfqpoint{3.250562in}{1.838799in}}%
\pgfpathlineto{\pgfqpoint{3.242305in}{1.830577in}}%
\pgfpathclose%
\pgfusepath{fill}%
\end{pgfscope}%
\begin{pgfscope}%
\pgfpathrectangle{\pgfqpoint{1.150000in}{0.150000in}}{\pgfqpoint{5.700000in}{5.700000in}}%
\pgfusepath{clip}%
\pgfsetbuttcap%
\pgfsetroundjoin%
\definecolor{currentfill}{rgb}{0.283091,0.110553,0.431554}%
\pgfsetfillcolor{currentfill}%
\pgfsetfillopacity{0.800000}%
\pgfsetlinewidth{0.000000pt}%
\definecolor{currentstroke}{rgb}{0.000000,0.000000,0.000000}%
\pgfsetstrokecolor{currentstroke}%
\pgfsetdash{}{0pt}%
\pgfpathmoveto{\pgfqpoint{2.847718in}{2.012268in}}%
\pgfpathlineto{\pgfqpoint{2.861425in}{1.999339in}}%
\pgfpathlineto{\pgfqpoint{2.875128in}{1.986662in}}%
\pgfpathlineto{\pgfqpoint{2.888827in}{1.974235in}}%
\pgfpathlineto{\pgfqpoint{2.902522in}{1.962058in}}%
\pgfpathlineto{\pgfqpoint{2.910975in}{1.967336in}}%
\pgfpathlineto{\pgfqpoint{2.919417in}{1.972773in}}%
\pgfpathlineto{\pgfqpoint{2.927849in}{1.978365in}}%
\pgfpathlineto{\pgfqpoint{2.936270in}{1.984110in}}%
\pgfpathlineto{\pgfqpoint{2.922604in}{1.995876in}}%
\pgfpathlineto{\pgfqpoint{2.908935in}{2.007891in}}%
\pgfpathlineto{\pgfqpoint{2.895262in}{2.020157in}}%
\pgfpathlineto{\pgfqpoint{2.881585in}{2.032674in}}%
\pgfpathlineto{\pgfqpoint{2.873135in}{2.027329in}}%
\pgfpathlineto{\pgfqpoint{2.864674in}{2.022143in}}%
\pgfpathlineto{\pgfqpoint{2.856202in}{2.017122in}}%
\pgfpathlineto{\pgfqpoint{2.847718in}{2.012268in}}%
\pgfpathclose%
\pgfusepath{fill}%
\end{pgfscope}%
\begin{pgfscope}%
\pgfpathrectangle{\pgfqpoint{1.150000in}{0.150000in}}{\pgfqpoint{5.700000in}{5.700000in}}%
\pgfusepath{clip}%
\pgfsetbuttcap%
\pgfsetroundjoin%
\definecolor{currentfill}{rgb}{0.120565,0.596422,0.543611}%
\pgfsetfillcolor{currentfill}%
\pgfsetfillopacity{0.800000}%
\pgfsetlinewidth{0.000000pt}%
\definecolor{currentstroke}{rgb}{0.000000,0.000000,0.000000}%
\pgfsetstrokecolor{currentstroke}%
\pgfsetdash{}{0pt}%
\pgfpathmoveto{\pgfqpoint{5.651189in}{3.285854in}}%
\pgfpathlineto{\pgfqpoint{5.665743in}{3.296561in}}%
\pgfpathlineto{\pgfqpoint{5.680316in}{3.307444in}}%
\pgfpathlineto{\pgfqpoint{5.694909in}{3.318504in}}%
\pgfpathlineto{\pgfqpoint{5.709522in}{3.329739in}}%
\pgfpathlineto{\pgfqpoint{5.716728in}{3.332577in}}%
\pgfpathlineto{\pgfqpoint{5.723930in}{3.335479in}}%
\pgfpathlineto{\pgfqpoint{5.731128in}{3.338453in}}%
\pgfpathlineto{\pgfqpoint{5.738321in}{3.341506in}}%
\pgfpathlineto{\pgfqpoint{5.723739in}{3.330884in}}%
\pgfpathlineto{\pgfqpoint{5.709177in}{3.320438in}}%
\pgfpathlineto{\pgfqpoint{5.694634in}{3.310168in}}%
\pgfpathlineto{\pgfqpoint{5.680109in}{3.300072in}}%
\pgfpathlineto{\pgfqpoint{5.672885in}{3.296395in}}%
\pgfpathlineto{\pgfqpoint{5.665657in}{3.292804in}}%
\pgfpathlineto{\pgfqpoint{5.658425in}{3.289293in}}%
\pgfpathlineto{\pgfqpoint{5.651189in}{3.285854in}}%
\pgfpathclose%
\pgfusepath{fill}%
\end{pgfscope}%
\begin{pgfscope}%
\pgfpathrectangle{\pgfqpoint{1.150000in}{0.150000in}}{\pgfqpoint{5.700000in}{5.700000in}}%
\pgfusepath{clip}%
\pgfsetbuttcap%
\pgfsetroundjoin%
\definecolor{currentfill}{rgb}{0.171176,0.452530,0.557965}%
\pgfsetfillcolor{currentfill}%
\pgfsetfillopacity{0.800000}%
\pgfsetlinewidth{0.000000pt}%
\definecolor{currentstroke}{rgb}{0.000000,0.000000,0.000000}%
\pgfsetstrokecolor{currentstroke}%
\pgfsetdash{}{0pt}%
\pgfpathmoveto{\pgfqpoint{5.011394in}{2.846896in}}%
\pgfpathlineto{\pgfqpoint{5.025610in}{2.856299in}}%
\pgfpathlineto{\pgfqpoint{5.039842in}{2.865882in}}%
\pgfpathlineto{\pgfqpoint{5.054092in}{2.875646in}}%
\pgfpathlineto{\pgfqpoint{5.068358in}{2.885592in}}%
\pgfpathlineto{\pgfqpoint{5.075918in}{2.891768in}}%
\pgfpathlineto{\pgfqpoint{5.083472in}{2.897897in}}%
\pgfpathlineto{\pgfqpoint{5.091018in}{2.903980in}}%
\pgfpathlineto{\pgfqpoint{5.098558in}{2.910022in}}%
\pgfpathlineto{\pgfqpoint{5.084306in}{2.900421in}}%
\pgfpathlineto{\pgfqpoint{5.070071in}{2.891000in}}%
\pgfpathlineto{\pgfqpoint{5.055853in}{2.881758in}}%
\pgfpathlineto{\pgfqpoint{5.041652in}{2.872698in}}%
\pgfpathlineto{\pgfqpoint{5.034097in}{2.866301in}}%
\pgfpathlineto{\pgfqpoint{5.026536in}{2.859871in}}%
\pgfpathlineto{\pgfqpoint{5.018968in}{2.853404in}}%
\pgfpathlineto{\pgfqpoint{5.011394in}{2.846896in}}%
\pgfpathclose%
\pgfusepath{fill}%
\end{pgfscope}%
\begin{pgfscope}%
\pgfpathrectangle{\pgfqpoint{1.150000in}{0.150000in}}{\pgfqpoint{5.700000in}{5.700000in}}%
\pgfusepath{clip}%
\pgfsetbuttcap%
\pgfsetroundjoin%
\definecolor{currentfill}{rgb}{0.239346,0.300855,0.540844}%
\pgfsetfillcolor{currentfill}%
\pgfsetfillopacity{0.800000}%
\pgfsetlinewidth{0.000000pt}%
\definecolor{currentstroke}{rgb}{0.000000,0.000000,0.000000}%
\pgfsetstrokecolor{currentstroke}%
\pgfsetdash{}{0pt}%
\pgfpathmoveto{\pgfqpoint{2.461143in}{2.485084in}}%
\pgfpathlineto{\pgfqpoint{2.475076in}{2.464221in}}%
\pgfpathlineto{\pgfqpoint{2.488998in}{2.443677in}}%
\pgfpathlineto{\pgfqpoint{2.502908in}{2.423446in}}%
\pgfpathlineto{\pgfqpoint{2.516806in}{2.403527in}}%
\pgfpathlineto{\pgfqpoint{2.525506in}{2.405788in}}%
\pgfpathlineto{\pgfqpoint{2.534191in}{2.408267in}}%
\pgfpathlineto{\pgfqpoint{2.542862in}{2.410961in}}%
\pgfpathlineto{\pgfqpoint{2.551518in}{2.413866in}}%
\pgfpathlineto{\pgfqpoint{2.537660in}{2.433357in}}%
\pgfpathlineto{\pgfqpoint{2.523791in}{2.453158in}}%
\pgfpathlineto{\pgfqpoint{2.509911in}{2.473273in}}%
\pgfpathlineto{\pgfqpoint{2.496019in}{2.493703in}}%
\pgfpathlineto{\pgfqpoint{2.487322in}{2.491215in}}%
\pgfpathlineto{\pgfqpoint{2.478611in}{2.488947in}}%
\pgfpathlineto{\pgfqpoint{2.469885in}{2.486901in}}%
\pgfpathlineto{\pgfqpoint{2.461143in}{2.485084in}}%
\pgfpathclose%
\pgfusepath{fill}%
\end{pgfscope}%
\begin{pgfscope}%
\pgfpathrectangle{\pgfqpoint{1.150000in}{0.150000in}}{\pgfqpoint{5.700000in}{5.700000in}}%
\pgfusepath{clip}%
\pgfsetbuttcap%
\pgfsetroundjoin%
\definecolor{currentfill}{rgb}{0.276022,0.044167,0.370164}%
\pgfsetfillcolor{currentfill}%
\pgfsetfillopacity{0.800000}%
\pgfsetlinewidth{0.000000pt}%
\definecolor{currentstroke}{rgb}{0.000000,0.000000,0.000000}%
\pgfsetstrokecolor{currentstroke}%
\pgfsetdash{}{0pt}%
\pgfpathmoveto{\pgfqpoint{3.100071in}{1.861736in}}%
\pgfpathlineto{\pgfqpoint{3.113712in}{1.853062in}}%
\pgfpathlineto{\pgfqpoint{3.127353in}{1.844616in}}%
\pgfpathlineto{\pgfqpoint{3.140993in}{1.836397in}}%
\pgfpathlineto{\pgfqpoint{3.154633in}{1.828403in}}%
\pgfpathlineto{\pgfqpoint{3.162944in}{1.835845in}}%
\pgfpathlineto{\pgfqpoint{3.171246in}{1.843399in}}%
\pgfpathlineto{\pgfqpoint{3.179539in}{1.851061in}}%
\pgfpathlineto{\pgfqpoint{3.187824in}{1.858829in}}%
\pgfpathlineto{\pgfqpoint{3.174206in}{1.866450in}}%
\pgfpathlineto{\pgfqpoint{3.160588in}{1.874295in}}%
\pgfpathlineto{\pgfqpoint{3.146970in}{1.882368in}}%
\pgfpathlineto{\pgfqpoint{3.133352in}{1.890668in}}%
\pgfpathlineto{\pgfqpoint{3.125045in}{1.883262in}}%
\pgfpathlineto{\pgfqpoint{3.116729in}{1.875968in}}%
\pgfpathlineto{\pgfqpoint{3.108404in}{1.868792in}}%
\pgfpathlineto{\pgfqpoint{3.100071in}{1.861736in}}%
\pgfpathclose%
\pgfusepath{fill}%
\end{pgfscope}%
\begin{pgfscope}%
\pgfpathrectangle{\pgfqpoint{1.150000in}{0.150000in}}{\pgfqpoint{5.700000in}{5.700000in}}%
\pgfusepath{clip}%
\pgfsetbuttcap%
\pgfsetroundjoin%
\definecolor{currentfill}{rgb}{0.119483,0.614817,0.537692}%
\pgfsetfillcolor{currentfill}%
\pgfsetfillopacity{0.800000}%
\pgfsetlinewidth{0.000000pt}%
\definecolor{currentstroke}{rgb}{0.000000,0.000000,0.000000}%
\pgfsetstrokecolor{currentstroke}%
\pgfsetdash{}{0pt}%
\pgfpathmoveto{\pgfqpoint{5.738321in}{3.341506in}}%
\pgfpathlineto{\pgfqpoint{5.752923in}{3.352302in}}%
\pgfpathlineto{\pgfqpoint{5.767544in}{3.363274in}}%
\pgfpathlineto{\pgfqpoint{5.782185in}{3.374422in}}%
\pgfpathlineto{\pgfqpoint{5.796847in}{3.385745in}}%
\pgfpathlineto{\pgfqpoint{5.804004in}{3.388247in}}%
\pgfpathlineto{\pgfqpoint{5.811158in}{3.390834in}}%
\pgfpathlineto{\pgfqpoint{5.818308in}{3.393513in}}%
\pgfpathlineto{\pgfqpoint{5.825455in}{3.396291in}}%
\pgfpathlineto{\pgfqpoint{5.810827in}{3.385616in}}%
\pgfpathlineto{\pgfqpoint{5.796218in}{3.375116in}}%
\pgfpathlineto{\pgfqpoint{5.781629in}{3.364790in}}%
\pgfpathlineto{\pgfqpoint{5.767060in}{3.354638in}}%
\pgfpathlineto{\pgfqpoint{5.759880in}{3.351203in}}%
\pgfpathlineto{\pgfqpoint{5.752697in}{3.347874in}}%
\pgfpathlineto{\pgfqpoint{5.745511in}{3.344644in}}%
\pgfpathlineto{\pgfqpoint{5.738321in}{3.341506in}}%
\pgfpathclose%
\pgfusepath{fill}%
\end{pgfscope}%
\begin{pgfscope}%
\pgfpathrectangle{\pgfqpoint{1.150000in}{0.150000in}}{\pgfqpoint{5.700000in}{5.700000in}}%
\pgfusepath{clip}%
\pgfsetbuttcap%
\pgfsetroundjoin%
\definecolor{currentfill}{rgb}{0.214298,0.355619,0.551184}%
\pgfsetfillcolor{currentfill}%
\pgfsetfillopacity{0.800000}%
\pgfsetlinewidth{0.000000pt}%
\definecolor{currentstroke}{rgb}{0.000000,0.000000,0.000000}%
\pgfsetstrokecolor{currentstroke}%
\pgfsetdash{}{0pt}%
\pgfpathmoveto{\pgfqpoint{4.632300in}{2.557742in}}%
\pgfpathlineto{\pgfqpoint{4.646320in}{2.565436in}}%
\pgfpathlineto{\pgfqpoint{4.660355in}{2.573313in}}%
\pgfpathlineto{\pgfqpoint{4.674404in}{2.581375in}}%
\pgfpathlineto{\pgfqpoint{4.688469in}{2.589620in}}%
\pgfpathlineto{\pgfqpoint{4.696208in}{2.598136in}}%
\pgfpathlineto{\pgfqpoint{4.703941in}{2.606574in}}%
\pgfpathlineto{\pgfqpoint{4.711667in}{2.614938in}}%
\pgfpathlineto{\pgfqpoint{4.719387in}{2.623228in}}%
\pgfpathlineto{\pgfqpoint{4.705331in}{2.615158in}}%
\pgfpathlineto{\pgfqpoint{4.691290in}{2.607272in}}%
\pgfpathlineto{\pgfqpoint{4.677263in}{2.599569in}}%
\pgfpathlineto{\pgfqpoint{4.663251in}{2.592050in}}%
\pgfpathlineto{\pgfqpoint{4.655523in}{2.583573in}}%
\pgfpathlineto{\pgfqpoint{4.647788in}{2.575030in}}%
\pgfpathlineto{\pgfqpoint{4.640047in}{2.566421in}}%
\pgfpathlineto{\pgfqpoint{4.632300in}{2.557742in}}%
\pgfpathclose%
\pgfusepath{fill}%
\end{pgfscope}%
\begin{pgfscope}%
\pgfpathrectangle{\pgfqpoint{1.150000in}{0.150000in}}{\pgfqpoint{5.700000in}{5.700000in}}%
\pgfusepath{clip}%
\pgfsetbuttcap%
\pgfsetroundjoin%
\definecolor{currentfill}{rgb}{0.262138,0.242286,0.520837}%
\pgfsetfillcolor{currentfill}%
\pgfsetfillopacity{0.800000}%
\pgfsetlinewidth{0.000000pt}%
\definecolor{currentstroke}{rgb}{0.000000,0.000000,0.000000}%
\pgfsetstrokecolor{currentstroke}%
\pgfsetdash{}{0pt}%
\pgfpathmoveto{\pgfqpoint{4.253085in}{2.262981in}}%
\pgfpathlineto{\pgfqpoint{4.266927in}{2.268078in}}%
\pgfpathlineto{\pgfqpoint{4.280782in}{2.273362in}}%
\pgfpathlineto{\pgfqpoint{4.294648in}{2.278834in}}%
\pgfpathlineto{\pgfqpoint{4.308526in}{2.284493in}}%
\pgfpathlineto{\pgfqpoint{4.316409in}{2.294890in}}%
\pgfpathlineto{\pgfqpoint{4.324287in}{2.305215in}}%
\pgfpathlineto{\pgfqpoint{4.332159in}{2.315469in}}%
\pgfpathlineto{\pgfqpoint{4.340025in}{2.325653in}}%
\pgfpathlineto{\pgfqpoint{4.326152in}{2.320004in}}%
\pgfpathlineto{\pgfqpoint{4.312291in}{2.314542in}}%
\pgfpathlineto{\pgfqpoint{4.298443in}{2.309268in}}%
\pgfpathlineto{\pgfqpoint{4.284606in}{2.304182in}}%
\pgfpathlineto{\pgfqpoint{4.276733in}{2.293976in}}%
\pgfpathlineto{\pgfqpoint{4.268856in}{2.283708in}}%
\pgfpathlineto{\pgfqpoint{4.260973in}{2.273376in}}%
\pgfpathlineto{\pgfqpoint{4.253085in}{2.262981in}}%
\pgfpathclose%
\pgfusepath{fill}%
\end{pgfscope}%
\begin{pgfscope}%
\pgfpathrectangle{\pgfqpoint{1.150000in}{0.150000in}}{\pgfqpoint{5.700000in}{5.700000in}}%
\pgfusepath{clip}%
\pgfsetbuttcap%
\pgfsetroundjoin%
\definecolor{currentfill}{rgb}{0.272594,0.025563,0.353093}%
\pgfsetfillcolor{currentfill}%
\pgfsetfillopacity{0.800000}%
\pgfsetlinewidth{0.000000pt}%
\definecolor{currentstroke}{rgb}{0.000000,0.000000,0.000000}%
\pgfsetstrokecolor{currentstroke}%
\pgfsetdash{}{0pt}%
\pgfpathmoveto{\pgfqpoint{3.384180in}{1.820681in}}%
\pgfpathlineto{\pgfqpoint{3.397804in}{1.816227in}}%
\pgfpathlineto{\pgfqpoint{3.411431in}{1.811984in}}%
\pgfpathlineto{\pgfqpoint{3.425061in}{1.807949in}}%
\pgfpathlineto{\pgfqpoint{3.438695in}{1.804124in}}%
\pgfpathlineto{\pgfqpoint{3.446872in}{1.813651in}}%
\pgfpathlineto{\pgfqpoint{3.455043in}{1.823233in}}%
\pgfpathlineto{\pgfqpoint{3.463207in}{1.832867in}}%
\pgfpathlineto{\pgfqpoint{3.471366in}{1.842550in}}%
\pgfpathlineto{\pgfqpoint{3.457746in}{1.846068in}}%
\pgfpathlineto{\pgfqpoint{3.444131in}{1.849794in}}%
\pgfpathlineto{\pgfqpoint{3.430519in}{1.853730in}}%
\pgfpathlineto{\pgfqpoint{3.416911in}{1.857876in}}%
\pgfpathlineto{\pgfqpoint{3.408738in}{1.848489in}}%
\pgfpathlineto{\pgfqpoint{3.400559in}{1.839158in}}%
\pgfpathlineto{\pgfqpoint{3.392373in}{1.829888in}}%
\pgfpathlineto{\pgfqpoint{3.384180in}{1.820681in}}%
\pgfpathclose%
\pgfusepath{fill}%
\end{pgfscope}%
\begin{pgfscope}%
\pgfpathrectangle{\pgfqpoint{1.150000in}{0.150000in}}{\pgfqpoint{5.700000in}{5.700000in}}%
\pgfusepath{clip}%
\pgfsetbuttcap%
\pgfsetroundjoin%
\definecolor{currentfill}{rgb}{0.281924,0.089666,0.412415}%
\pgfsetfillcolor{currentfill}%
\pgfsetfillopacity{0.800000}%
\pgfsetlinewidth{0.000000pt}%
\definecolor{currentstroke}{rgb}{0.000000,0.000000,0.000000}%
\pgfsetstrokecolor{currentstroke}%
\pgfsetdash{}{0pt}%
\pgfpathmoveto{\pgfqpoint{2.902522in}{1.962058in}}%
\pgfpathlineto{\pgfqpoint{2.916213in}{1.950129in}}%
\pgfpathlineto{\pgfqpoint{2.929901in}{1.938445in}}%
\pgfpathlineto{\pgfqpoint{2.943586in}{1.927006in}}%
\pgfpathlineto{\pgfqpoint{2.957268in}{1.915809in}}%
\pgfpathlineto{\pgfqpoint{2.965692in}{1.921508in}}%
\pgfpathlineto{\pgfqpoint{2.974106in}{1.927358in}}%
\pgfpathlineto{\pgfqpoint{2.982509in}{1.933357in}}%
\pgfpathlineto{\pgfqpoint{2.990902in}{1.939498in}}%
\pgfpathlineto{\pgfqpoint{2.977248in}{1.950286in}}%
\pgfpathlineto{\pgfqpoint{2.963592in}{1.961316in}}%
\pgfpathlineto{\pgfqpoint{2.949932in}{1.972590in}}%
\pgfpathlineto{\pgfqpoint{2.936270in}{1.984110in}}%
\pgfpathlineto{\pgfqpoint{2.927849in}{1.978365in}}%
\pgfpathlineto{\pgfqpoint{2.919417in}{1.972773in}}%
\pgfpathlineto{\pgfqpoint{2.910975in}{1.967336in}}%
\pgfpathlineto{\pgfqpoint{2.902522in}{1.962058in}}%
\pgfpathclose%
\pgfusepath{fill}%
\end{pgfscope}%
\begin{pgfscope}%
\pgfpathrectangle{\pgfqpoint{1.150000in}{0.150000in}}{\pgfqpoint{5.700000in}{5.700000in}}%
\pgfusepath{clip}%
\pgfsetbuttcap%
\pgfsetroundjoin%
\definecolor{currentfill}{rgb}{0.122312,0.633153,0.530398}%
\pgfsetfillcolor{currentfill}%
\pgfsetfillopacity{0.800000}%
\pgfsetlinewidth{0.000000pt}%
\definecolor{currentstroke}{rgb}{0.000000,0.000000,0.000000}%
\pgfsetstrokecolor{currentstroke}%
\pgfsetdash{}{0pt}%
\pgfpathmoveto{\pgfqpoint{5.825455in}{3.396291in}}%
\pgfpathlineto{\pgfqpoint{5.840103in}{3.407141in}}%
\pgfpathlineto{\pgfqpoint{5.854771in}{3.418165in}}%
\pgfpathlineto{\pgfqpoint{5.869459in}{3.429365in}}%
\pgfpathlineto{\pgfqpoint{5.884167in}{3.440740in}}%
\pgfpathlineto{\pgfqpoint{5.891276in}{3.442954in}}%
\pgfpathlineto{\pgfqpoint{5.898382in}{3.445274in}}%
\pgfpathlineto{\pgfqpoint{5.905485in}{3.447707in}}%
\pgfpathlineto{\pgfqpoint{5.912586in}{3.450261in}}%
\pgfpathlineto{\pgfqpoint{5.897913in}{3.439568in}}%
\pgfpathlineto{\pgfqpoint{5.883260in}{3.429050in}}%
\pgfpathlineto{\pgfqpoint{5.868627in}{3.418705in}}%
\pgfpathlineto{\pgfqpoint{5.854014in}{3.408534in}}%
\pgfpathlineto{\pgfqpoint{5.846878in}{3.405289in}}%
\pgfpathlineto{\pgfqpoint{5.839739in}{3.402171in}}%
\pgfpathlineto{\pgfqpoint{5.832598in}{3.399175in}}%
\pgfpathlineto{\pgfqpoint{5.825455in}{3.396291in}}%
\pgfpathclose%
\pgfusepath{fill}%
\end{pgfscope}%
\begin{pgfscope}%
\pgfpathrectangle{\pgfqpoint{1.150000in}{0.150000in}}{\pgfqpoint{5.700000in}{5.700000in}}%
\pgfusepath{clip}%
\pgfsetbuttcap%
\pgfsetroundjoin%
\definecolor{currentfill}{rgb}{0.160665,0.478540,0.558115}%
\pgfsetfillcolor{currentfill}%
\pgfsetfillopacity{0.800000}%
\pgfsetlinewidth{0.000000pt}%
\definecolor{currentstroke}{rgb}{0.000000,0.000000,0.000000}%
\pgfsetstrokecolor{currentstroke}%
\pgfsetdash{}{0pt}%
\pgfpathmoveto{\pgfqpoint{5.098558in}{2.910022in}}%
\pgfpathlineto{\pgfqpoint{5.112826in}{2.919804in}}%
\pgfpathlineto{\pgfqpoint{5.127112in}{2.929766in}}%
\pgfpathlineto{\pgfqpoint{5.141415in}{2.939908in}}%
\pgfpathlineto{\pgfqpoint{5.155736in}{2.950231in}}%
\pgfpathlineto{\pgfqpoint{5.163253in}{2.955872in}}%
\pgfpathlineto{\pgfqpoint{5.170763in}{2.961472in}}%
\pgfpathlineto{\pgfqpoint{5.178267in}{2.967036in}}%
\pgfpathlineto{\pgfqpoint{5.185763in}{2.972568in}}%
\pgfpathlineto{\pgfqpoint{5.171459in}{2.962624in}}%
\pgfpathlineto{\pgfqpoint{5.157172in}{2.952859in}}%
\pgfpathlineto{\pgfqpoint{5.142903in}{2.943273in}}%
\pgfpathlineto{\pgfqpoint{5.128650in}{2.933867in}}%
\pgfpathlineto{\pgfqpoint{5.121137in}{2.927946in}}%
\pgfpathlineto{\pgfqpoint{5.113617in}{2.922001in}}%
\pgfpathlineto{\pgfqpoint{5.106091in}{2.916028in}}%
\pgfpathlineto{\pgfqpoint{5.098558in}{2.910022in}}%
\pgfpathclose%
\pgfusepath{fill}%
\end{pgfscope}%
\begin{pgfscope}%
\pgfpathrectangle{\pgfqpoint{1.150000in}{0.150000in}}{\pgfqpoint{5.700000in}{5.700000in}}%
\pgfusepath{clip}%
\pgfsetbuttcap%
\pgfsetroundjoin%
\definecolor{currentfill}{rgb}{0.130067,0.651384,0.521608}%
\pgfsetfillcolor{currentfill}%
\pgfsetfillopacity{0.800000}%
\pgfsetlinewidth{0.000000pt}%
\definecolor{currentstroke}{rgb}{0.000000,0.000000,0.000000}%
\pgfsetstrokecolor{currentstroke}%
\pgfsetdash{}{0pt}%
\pgfpathmoveto{\pgfqpoint{5.912586in}{3.450261in}}%
\pgfpathlineto{\pgfqpoint{5.927279in}{3.461128in}}%
\pgfpathlineto{\pgfqpoint{5.941993in}{3.472169in}}%
\pgfpathlineto{\pgfqpoint{5.956726in}{3.483385in}}%
\pgfpathlineto{\pgfqpoint{5.971481in}{3.494775in}}%
\pgfpathlineto{\pgfqpoint{5.978542in}{3.496753in}}%
\pgfpathlineto{\pgfqpoint{5.985601in}{3.498860in}}%
\pgfpathlineto{\pgfqpoint{5.992659in}{3.501103in}}%
\pgfpathlineto{\pgfqpoint{5.999715in}{3.503490in}}%
\pgfpathlineto{\pgfqpoint{5.984999in}{3.492816in}}%
\pgfpathlineto{\pgfqpoint{5.970303in}{3.482315in}}%
\pgfpathlineto{\pgfqpoint{5.955627in}{3.471987in}}%
\pgfpathlineto{\pgfqpoint{5.940971in}{3.461832in}}%
\pgfpathlineto{\pgfqpoint{5.933877in}{3.458721in}}%
\pgfpathlineto{\pgfqpoint{5.926782in}{3.455760in}}%
\pgfpathlineto{\pgfqpoint{5.919685in}{3.452943in}}%
\pgfpathlineto{\pgfqpoint{5.912586in}{3.450261in}}%
\pgfpathclose%
\pgfusepath{fill}%
\end{pgfscope}%
\begin{pgfscope}%
\pgfpathrectangle{\pgfqpoint{1.150000in}{0.150000in}}{\pgfqpoint{5.700000in}{5.700000in}}%
\pgfusepath{clip}%
\pgfsetbuttcap%
\pgfsetroundjoin%
\definecolor{currentfill}{rgb}{0.223925,0.334994,0.548053}%
\pgfsetfillcolor{currentfill}%
\pgfsetfillopacity{0.800000}%
\pgfsetlinewidth{0.000000pt}%
\definecolor{currentstroke}{rgb}{0.000000,0.000000,0.000000}%
\pgfsetstrokecolor{currentstroke}%
\pgfsetdash{}{0pt}%
\pgfpathmoveto{\pgfqpoint{2.405281in}{2.571764in}}%
\pgfpathlineto{\pgfqpoint{2.419266in}{2.549603in}}%
\pgfpathlineto{\pgfqpoint{2.433238in}{2.527771in}}%
\pgfpathlineto{\pgfqpoint{2.447196in}{2.506266in}}%
\pgfpathlineto{\pgfqpoint{2.461143in}{2.485084in}}%
\pgfpathlineto{\pgfqpoint{2.469885in}{2.486901in}}%
\pgfpathlineto{\pgfqpoint{2.478611in}{2.488947in}}%
\pgfpathlineto{\pgfqpoint{2.487322in}{2.491215in}}%
\pgfpathlineto{\pgfqpoint{2.496019in}{2.493703in}}%
\pgfpathlineto{\pgfqpoint{2.482115in}{2.514453in}}%
\pgfpathlineto{\pgfqpoint{2.468199in}{2.535525in}}%
\pgfpathlineto{\pgfqpoint{2.454270in}{2.556923in}}%
\pgfpathlineto{\pgfqpoint{2.440328in}{2.578649in}}%
\pgfpathlineto{\pgfqpoint{2.431590in}{2.576581in}}%
\pgfpathlineto{\pgfqpoint{2.422836in}{2.574742in}}%
\pgfpathlineto{\pgfqpoint{2.414067in}{2.573135in}}%
\pgfpathlineto{\pgfqpoint{2.405281in}{2.571764in}}%
\pgfpathclose%
\pgfusepath{fill}%
\end{pgfscope}%
\begin{pgfscope}%
\pgfpathrectangle{\pgfqpoint{1.150000in}{0.150000in}}{\pgfqpoint{5.700000in}{5.700000in}}%
\pgfusepath{clip}%
\pgfsetbuttcap%
\pgfsetroundjoin%
\definecolor{currentfill}{rgb}{0.252194,0.269783,0.531579}%
\pgfsetfillcolor{currentfill}%
\pgfsetfillopacity{0.800000}%
\pgfsetlinewidth{0.000000pt}%
\definecolor{currentstroke}{rgb}{0.000000,0.000000,0.000000}%
\pgfsetstrokecolor{currentstroke}%
\pgfsetdash{}{0pt}%
\pgfpathmoveto{\pgfqpoint{4.340025in}{2.325653in}}%
\pgfpathlineto{\pgfqpoint{4.353910in}{2.331489in}}%
\pgfpathlineto{\pgfqpoint{4.367808in}{2.337513in}}%
\pgfpathlineto{\pgfqpoint{4.381719in}{2.343723in}}%
\pgfpathlineto{\pgfqpoint{4.395642in}{2.350119in}}%
\pgfpathlineto{\pgfqpoint{4.403497in}{2.360202in}}%
\pgfpathlineto{\pgfqpoint{4.411347in}{2.370208in}}%
\pgfpathlineto{\pgfqpoint{4.419191in}{2.380138in}}%
\pgfpathlineto{\pgfqpoint{4.427029in}{2.389992in}}%
\pgfpathlineto{\pgfqpoint{4.413112in}{2.383639in}}%
\pgfpathlineto{\pgfqpoint{4.399207in}{2.377472in}}%
\pgfpathlineto{\pgfqpoint{4.385315in}{2.371492in}}%
\pgfpathlineto{\pgfqpoint{4.371436in}{2.365698in}}%
\pgfpathlineto{\pgfqpoint{4.363591in}{2.355788in}}%
\pgfpathlineto{\pgfqpoint{4.355741in}{2.345812in}}%
\pgfpathlineto{\pgfqpoint{4.347886in}{2.335767in}}%
\pgfpathlineto{\pgfqpoint{4.340025in}{2.325653in}}%
\pgfpathclose%
\pgfusepath{fill}%
\end{pgfscope}%
\begin{pgfscope}%
\pgfpathrectangle{\pgfqpoint{1.150000in}{0.150000in}}{\pgfqpoint{5.700000in}{5.700000in}}%
\pgfusepath{clip}%
\pgfsetbuttcap%
\pgfsetroundjoin%
\definecolor{currentfill}{rgb}{0.282656,0.100196,0.422160}%
\pgfsetfillcolor{currentfill}%
\pgfsetfillopacity{0.800000}%
\pgfsetlinewidth{0.000000pt}%
\definecolor{currentstroke}{rgb}{0.000000,0.000000,0.000000}%
\pgfsetstrokecolor{currentstroke}%
\pgfsetdash{}{0pt}%
\pgfpathmoveto{\pgfqpoint{3.786856in}{1.944909in}}%
\pgfpathlineto{\pgfqpoint{3.800542in}{1.945555in}}%
\pgfpathlineto{\pgfqpoint{3.814235in}{1.946396in}}%
\pgfpathlineto{\pgfqpoint{3.827936in}{1.947434in}}%
\pgfpathlineto{\pgfqpoint{3.841646in}{1.948666in}}%
\pgfpathlineto{\pgfqpoint{3.849680in}{1.959750in}}%
\pgfpathlineto{\pgfqpoint{3.857709in}{1.970815in}}%
\pgfpathlineto{\pgfqpoint{3.865733in}{1.981860in}}%
\pgfpathlineto{\pgfqpoint{3.873752in}{1.992882in}}%
\pgfpathlineto{\pgfqpoint{3.860050in}{1.991468in}}%
\pgfpathlineto{\pgfqpoint{3.846356in}{1.990248in}}%
\pgfpathlineto{\pgfqpoint{3.832671in}{1.989225in}}%
\pgfpathlineto{\pgfqpoint{3.818993in}{1.988397in}}%
\pgfpathlineto{\pgfqpoint{3.810967in}{1.977545in}}%
\pgfpathlineto{\pgfqpoint{3.802935in}{1.966679in}}%
\pgfpathlineto{\pgfqpoint{3.794898in}{1.955799in}}%
\pgfpathlineto{\pgfqpoint{3.786856in}{1.944909in}}%
\pgfpathclose%
\pgfusepath{fill}%
\end{pgfscope}%
\begin{pgfscope}%
\pgfpathrectangle{\pgfqpoint{1.150000in}{0.150000in}}{\pgfqpoint{5.700000in}{5.700000in}}%
\pgfusepath{clip}%
\pgfsetbuttcap%
\pgfsetroundjoin%
\definecolor{currentfill}{rgb}{0.280894,0.078907,0.402329}%
\pgfsetfillcolor{currentfill}%
\pgfsetfillopacity{0.800000}%
\pgfsetlinewidth{0.000000pt}%
\definecolor{currentstroke}{rgb}{0.000000,0.000000,0.000000}%
\pgfsetstrokecolor{currentstroke}%
\pgfsetdash{}{0pt}%
\pgfpathmoveto{\pgfqpoint{3.699932in}{1.901506in}}%
\pgfpathlineto{\pgfqpoint{3.713598in}{1.901149in}}%
\pgfpathlineto{\pgfqpoint{3.727271in}{1.900990in}}%
\pgfpathlineto{\pgfqpoint{3.740950in}{1.901029in}}%
\pgfpathlineto{\pgfqpoint{3.754637in}{1.901265in}}%
\pgfpathlineto{\pgfqpoint{3.762700in}{1.912185in}}%
\pgfpathlineto{\pgfqpoint{3.770757in}{1.923100in}}%
\pgfpathlineto{\pgfqpoint{3.778809in}{1.934008in}}%
\pgfpathlineto{\pgfqpoint{3.786856in}{1.944909in}}%
\pgfpathlineto{\pgfqpoint{3.773178in}{1.944460in}}%
\pgfpathlineto{\pgfqpoint{3.759507in}{1.944207in}}%
\pgfpathlineto{\pgfqpoint{3.745844in}{1.944153in}}%
\pgfpathlineto{\pgfqpoint{3.732187in}{1.944297in}}%
\pgfpathlineto{\pgfqpoint{3.724131in}{1.933598in}}%
\pgfpathlineto{\pgfqpoint{3.716070in}{1.922899in}}%
\pgfpathlineto{\pgfqpoint{3.708004in}{1.912201in}}%
\pgfpathlineto{\pgfqpoint{3.699932in}{1.901506in}}%
\pgfpathclose%
\pgfusepath{fill}%
\end{pgfscope}%
\begin{pgfscope}%
\pgfpathrectangle{\pgfqpoint{1.150000in}{0.150000in}}{\pgfqpoint{5.700000in}{5.700000in}}%
\pgfusepath{clip}%
\pgfsetbuttcap%
\pgfsetroundjoin%
\definecolor{currentfill}{rgb}{0.203063,0.379716,0.553925}%
\pgfsetfillcolor{currentfill}%
\pgfsetfillopacity{0.800000}%
\pgfsetlinewidth{0.000000pt}%
\definecolor{currentstroke}{rgb}{0.000000,0.000000,0.000000}%
\pgfsetstrokecolor{currentstroke}%
\pgfsetdash{}{0pt}%
\pgfpathmoveto{\pgfqpoint{4.719387in}{2.623228in}}%
\pgfpathlineto{\pgfqpoint{4.733458in}{2.631481in}}%
\pgfpathlineto{\pgfqpoint{4.747544in}{2.639918in}}%
\pgfpathlineto{\pgfqpoint{4.761645in}{2.648537in}}%
\pgfpathlineto{\pgfqpoint{4.775761in}{2.657340in}}%
\pgfpathlineto{\pgfqpoint{4.783466in}{2.665364in}}%
\pgfpathlineto{\pgfqpoint{4.791163in}{2.673311in}}%
\pgfpathlineto{\pgfqpoint{4.798855in}{2.681186in}}%
\pgfpathlineto{\pgfqpoint{4.806539in}{2.688989in}}%
\pgfpathlineto{\pgfqpoint{4.792432in}{2.680396in}}%
\pgfpathlineto{\pgfqpoint{4.778340in}{2.671985in}}%
\pgfpathlineto{\pgfqpoint{4.764264in}{2.663757in}}%
\pgfpathlineto{\pgfqpoint{4.750202in}{2.655712in}}%
\pgfpathlineto{\pgfqpoint{4.742508in}{2.647688in}}%
\pgfpathlineto{\pgfqpoint{4.734807in}{2.639601in}}%
\pgfpathlineto{\pgfqpoint{4.727100in}{2.631448in}}%
\pgfpathlineto{\pgfqpoint{4.719387in}{2.623228in}}%
\pgfpathclose%
\pgfusepath{fill}%
\end{pgfscope}%
\begin{pgfscope}%
\pgfpathrectangle{\pgfqpoint{1.150000in}{0.150000in}}{\pgfqpoint{5.700000in}{5.700000in}}%
\pgfusepath{clip}%
\pgfsetbuttcap%
\pgfsetroundjoin%
\definecolor{currentfill}{rgb}{0.143303,0.669459,0.511215}%
\pgfsetfillcolor{currentfill}%
\pgfsetfillopacity{0.800000}%
\pgfsetlinewidth{0.000000pt}%
\definecolor{currentstroke}{rgb}{0.000000,0.000000,0.000000}%
\pgfsetstrokecolor{currentstroke}%
\pgfsetdash{}{0pt}%
\pgfpathmoveto{\pgfqpoint{5.999715in}{3.503490in}}%
\pgfpathlineto{\pgfqpoint{6.014451in}{3.514338in}}%
\pgfpathlineto{\pgfqpoint{6.029209in}{3.525360in}}%
\pgfpathlineto{\pgfqpoint{6.043986in}{3.536555in}}%
\pgfpathlineto{\pgfqpoint{6.058785in}{3.547925in}}%
\pgfpathlineto{\pgfqpoint{6.065800in}{3.549727in}}%
\pgfpathlineto{\pgfqpoint{6.072814in}{3.551681in}}%
\pgfpathlineto{\pgfqpoint{6.079827in}{3.553795in}}%
\pgfpathlineto{\pgfqpoint{6.086840in}{3.556078in}}%
\pgfpathlineto{\pgfqpoint{6.072083in}{3.545457in}}%
\pgfpathlineto{\pgfqpoint{6.057345in}{3.535010in}}%
\pgfpathlineto{\pgfqpoint{6.042628in}{3.524735in}}%
\pgfpathlineto{\pgfqpoint{6.027931in}{3.514633in}}%
\pgfpathlineto{\pgfqpoint{6.020878in}{3.511592in}}%
\pgfpathlineto{\pgfqpoint{6.013824in}{3.508727in}}%
\pgfpathlineto{\pgfqpoint{6.006770in}{3.506029in}}%
\pgfpathlineto{\pgfqpoint{5.999715in}{3.503490in}}%
\pgfpathclose%
\pgfusepath{fill}%
\end{pgfscope}%
\begin{pgfscope}%
\pgfpathrectangle{\pgfqpoint{1.150000in}{0.150000in}}{\pgfqpoint{5.700000in}{5.700000in}}%
\pgfusepath{clip}%
\pgfsetbuttcap%
\pgfsetroundjoin%
\definecolor{currentfill}{rgb}{0.283187,0.125848,0.444960}%
\pgfsetfillcolor{currentfill}%
\pgfsetfillopacity{0.800000}%
\pgfsetlinewidth{0.000000pt}%
\definecolor{currentstroke}{rgb}{0.000000,0.000000,0.000000}%
\pgfsetstrokecolor{currentstroke}%
\pgfsetdash{}{0pt}%
\pgfpathmoveto{\pgfqpoint{3.873752in}{1.992882in}}%
\pgfpathlineto{\pgfqpoint{3.887462in}{1.994490in}}%
\pgfpathlineto{\pgfqpoint{3.901180in}{1.996293in}}%
\pgfpathlineto{\pgfqpoint{3.914908in}{1.998289in}}%
\pgfpathlineto{\pgfqpoint{3.928644in}{2.000479in}}%
\pgfpathlineto{\pgfqpoint{3.936651in}{2.011639in}}%
\pgfpathlineto{\pgfqpoint{3.944653in}{2.022768in}}%
\pgfpathlineto{\pgfqpoint{3.952650in}{2.033862in}}%
\pgfpathlineto{\pgfqpoint{3.960643in}{2.044922in}}%
\pgfpathlineto{\pgfqpoint{3.946913in}{2.042582in}}%
\pgfpathlineto{\pgfqpoint{3.933192in}{2.040436in}}%
\pgfpathlineto{\pgfqpoint{3.919481in}{2.038483in}}%
\pgfpathlineto{\pgfqpoint{3.905777in}{2.036724in}}%
\pgfpathlineto{\pgfqpoint{3.897779in}{2.025803in}}%
\pgfpathlineto{\pgfqpoint{3.889775in}{2.014854in}}%
\pgfpathlineto{\pgfqpoint{3.881766in}{2.003880in}}%
\pgfpathlineto{\pgfqpoint{3.873752in}{1.992882in}}%
\pgfpathclose%
\pgfusepath{fill}%
\end{pgfscope}%
\begin{pgfscope}%
\pgfpathrectangle{\pgfqpoint{1.150000in}{0.150000in}}{\pgfqpoint{5.700000in}{5.700000in}}%
\pgfusepath{clip}%
\pgfsetbuttcap%
\pgfsetroundjoin%
\definecolor{currentfill}{rgb}{0.151918,0.500685,0.557587}%
\pgfsetfillcolor{currentfill}%
\pgfsetfillopacity{0.800000}%
\pgfsetlinewidth{0.000000pt}%
\definecolor{currentstroke}{rgb}{0.000000,0.000000,0.000000}%
\pgfsetstrokecolor{currentstroke}%
\pgfsetdash{}{0pt}%
\pgfpathmoveto{\pgfqpoint{5.185763in}{2.972568in}}%
\pgfpathlineto{\pgfqpoint{5.200085in}{2.982692in}}%
\pgfpathlineto{\pgfqpoint{5.214424in}{2.992996in}}%
\pgfpathlineto{\pgfqpoint{5.228781in}{3.003480in}}%
\pgfpathlineto{\pgfqpoint{5.243156in}{3.014144in}}%
\pgfpathlineto{\pgfqpoint{5.250628in}{3.019249in}}%
\pgfpathlineto{\pgfqpoint{5.258093in}{3.024324in}}%
\pgfpathlineto{\pgfqpoint{5.265552in}{3.029372in}}%
\pgfpathlineto{\pgfqpoint{5.273004in}{3.034400in}}%
\pgfpathlineto{\pgfqpoint{5.258647in}{3.024149in}}%
\pgfpathlineto{\pgfqpoint{5.244309in}{3.014077in}}%
\pgfpathlineto{\pgfqpoint{5.229987in}{3.004184in}}%
\pgfpathlineto{\pgfqpoint{5.215684in}{2.994469in}}%
\pgfpathlineto{\pgfqpoint{5.208213in}{2.989019in}}%
\pgfpathlineto{\pgfqpoint{5.200736in}{2.983555in}}%
\pgfpathlineto{\pgfqpoint{5.193253in}{2.978073in}}%
\pgfpathlineto{\pgfqpoint{5.185763in}{2.972568in}}%
\pgfpathclose%
\pgfusepath{fill}%
\end{pgfscope}%
\begin{pgfscope}%
\pgfpathrectangle{\pgfqpoint{1.150000in}{0.150000in}}{\pgfqpoint{5.700000in}{5.700000in}}%
\pgfusepath{clip}%
\pgfsetbuttcap%
\pgfsetroundjoin%
\definecolor{currentfill}{rgb}{0.278791,0.062145,0.386592}%
\pgfsetfillcolor{currentfill}%
\pgfsetfillopacity{0.800000}%
\pgfsetlinewidth{0.000000pt}%
\definecolor{currentstroke}{rgb}{0.000000,0.000000,0.000000}%
\pgfsetstrokecolor{currentstroke}%
\pgfsetdash{}{0pt}%
\pgfpathmoveto{\pgfqpoint{3.612953in}{1.863201in}}%
\pgfpathlineto{\pgfqpoint{3.626604in}{1.861800in}}%
\pgfpathlineto{\pgfqpoint{3.640260in}{1.860599in}}%
\pgfpathlineto{\pgfqpoint{3.653923in}{1.859599in}}%
\pgfpathlineto{\pgfqpoint{3.667593in}{1.858798in}}%
\pgfpathlineto{\pgfqpoint{3.675686in}{1.869460in}}%
\pgfpathlineto{\pgfqpoint{3.683773in}{1.880134in}}%
\pgfpathlineto{\pgfqpoint{3.691855in}{1.890816in}}%
\pgfpathlineto{\pgfqpoint{3.699932in}{1.901506in}}%
\pgfpathlineto{\pgfqpoint{3.686273in}{1.902062in}}%
\pgfpathlineto{\pgfqpoint{3.672621in}{1.902818in}}%
\pgfpathlineto{\pgfqpoint{3.658974in}{1.903773in}}%
\pgfpathlineto{\pgfqpoint{3.645334in}{1.904930in}}%
\pgfpathlineto{\pgfqpoint{3.637247in}{1.894473in}}%
\pgfpathlineto{\pgfqpoint{3.629154in}{1.884031in}}%
\pgfpathlineto{\pgfqpoint{3.621056in}{1.873607in}}%
\pgfpathlineto{\pgfqpoint{3.612953in}{1.863201in}}%
\pgfpathclose%
\pgfusepath{fill}%
\end{pgfscope}%
\begin{pgfscope}%
\pgfpathrectangle{\pgfqpoint{1.150000in}{0.150000in}}{\pgfqpoint{5.700000in}{5.700000in}}%
\pgfusepath{clip}%
\pgfsetbuttcap%
\pgfsetroundjoin%
\definecolor{currentfill}{rgb}{0.162016,0.687316,0.499129}%
\pgfsetfillcolor{currentfill}%
\pgfsetfillopacity{0.800000}%
\pgfsetlinewidth{0.000000pt}%
\definecolor{currentstroke}{rgb}{0.000000,0.000000,0.000000}%
\pgfsetstrokecolor{currentstroke}%
\pgfsetdash{}{0pt}%
\pgfpathmoveto{\pgfqpoint{6.086840in}{3.556078in}}%
\pgfpathlineto{\pgfqpoint{6.101619in}{3.566871in}}%
\pgfpathlineto{\pgfqpoint{6.116418in}{3.577837in}}%
\pgfpathlineto{\pgfqpoint{6.131239in}{3.588977in}}%
\pgfpathlineto{\pgfqpoint{6.146080in}{3.600290in}}%
\pgfpathlineto{\pgfqpoint{6.153051in}{3.601979in}}%
\pgfpathlineto{\pgfqpoint{6.160021in}{3.603846in}}%
\pgfpathlineto{\pgfqpoint{6.166993in}{3.605899in}}%
\pgfpathlineto{\pgfqpoint{6.173965in}{3.608146in}}%
\pgfpathlineto{\pgfqpoint{6.159167in}{3.597616in}}%
\pgfpathlineto{\pgfqpoint{6.144390in}{3.587258in}}%
\pgfpathlineto{\pgfqpoint{6.129634in}{3.577071in}}%
\pgfpathlineto{\pgfqpoint{6.114898in}{3.567057in}}%
\pgfpathlineto{\pgfqpoint{6.107882in}{3.564018in}}%
\pgfpathlineto{\pgfqpoint{6.100867in}{3.561181in}}%
\pgfpathlineto{\pgfqpoint{6.093854in}{3.558537in}}%
\pgfpathlineto{\pgfqpoint{6.086840in}{3.556078in}}%
\pgfpathclose%
\pgfusepath{fill}%
\end{pgfscope}%
\begin{pgfscope}%
\pgfpathrectangle{\pgfqpoint{1.150000in}{0.150000in}}{\pgfqpoint{5.700000in}{5.700000in}}%
\pgfusepath{clip}%
\pgfsetbuttcap%
\pgfsetroundjoin%
\definecolor{currentfill}{rgb}{0.280267,0.073417,0.397163}%
\pgfsetfillcolor{currentfill}%
\pgfsetfillopacity{0.800000}%
\pgfsetlinewidth{0.000000pt}%
\definecolor{currentstroke}{rgb}{0.000000,0.000000,0.000000}%
\pgfsetstrokecolor{currentstroke}%
\pgfsetdash{}{0pt}%
\pgfpathmoveto{\pgfqpoint{2.957268in}{1.915809in}}%
\pgfpathlineto{\pgfqpoint{2.970947in}{1.904853in}}%
\pgfpathlineto{\pgfqpoint{2.984624in}{1.894137in}}%
\pgfpathlineto{\pgfqpoint{2.998299in}{1.883658in}}%
\pgfpathlineto{\pgfqpoint{3.011971in}{1.873417in}}%
\pgfpathlineto{\pgfqpoint{3.020368in}{1.879536in}}%
\pgfpathlineto{\pgfqpoint{3.028754in}{1.885799in}}%
\pgfpathlineto{\pgfqpoint{3.037131in}{1.892201in}}%
\pgfpathlineto{\pgfqpoint{3.045498in}{1.898739in}}%
\pgfpathlineto{\pgfqpoint{3.031852in}{1.908573in}}%
\pgfpathlineto{\pgfqpoint{3.018204in}{1.918643in}}%
\pgfpathlineto{\pgfqpoint{3.004554in}{1.928951in}}%
\pgfpathlineto{\pgfqpoint{2.990902in}{1.939498in}}%
\pgfpathlineto{\pgfqpoint{2.982509in}{1.933357in}}%
\pgfpathlineto{\pgfqpoint{2.974106in}{1.927358in}}%
\pgfpathlineto{\pgfqpoint{2.965692in}{1.921508in}}%
\pgfpathlineto{\pgfqpoint{2.957268in}{1.915809in}}%
\pgfpathclose%
\pgfusepath{fill}%
\end{pgfscope}%
\begin{pgfscope}%
\pgfpathrectangle{\pgfqpoint{1.150000in}{0.150000in}}{\pgfqpoint{5.700000in}{5.700000in}}%
\pgfusepath{clip}%
\pgfsetbuttcap%
\pgfsetroundjoin%
\definecolor{currentfill}{rgb}{0.281887,0.150881,0.465405}%
\pgfsetfillcolor{currentfill}%
\pgfsetfillopacity{0.800000}%
\pgfsetlinewidth{0.000000pt}%
\definecolor{currentstroke}{rgb}{0.000000,0.000000,0.000000}%
\pgfsetstrokecolor{currentstroke}%
\pgfsetdash{}{0pt}%
\pgfpathmoveto{\pgfqpoint{3.960643in}{2.044922in}}%
\pgfpathlineto{\pgfqpoint{3.974382in}{2.047454in}}%
\pgfpathlineto{\pgfqpoint{3.988130in}{2.050178in}}%
\pgfpathlineto{\pgfqpoint{4.001888in}{2.053095in}}%
\pgfpathlineto{\pgfqpoint{4.015655in}{2.056203in}}%
\pgfpathlineto{\pgfqpoint{4.023636in}{2.067357in}}%
\pgfpathlineto{\pgfqpoint{4.031612in}{2.078466in}}%
\pgfpathlineto{\pgfqpoint{4.039584in}{2.089531in}}%
\pgfpathlineto{\pgfqpoint{4.047550in}{2.100549in}}%
\pgfpathlineto{\pgfqpoint{4.033788in}{2.097323in}}%
\pgfpathlineto{\pgfqpoint{4.020036in}{2.094288in}}%
\pgfpathlineto{\pgfqpoint{4.006294in}{2.091445in}}%
\pgfpathlineto{\pgfqpoint{3.992562in}{2.088794in}}%
\pgfpathlineto{\pgfqpoint{3.984589in}{2.077883in}}%
\pgfpathlineto{\pgfqpoint{3.976612in}{2.066933in}}%
\pgfpathlineto{\pgfqpoint{3.968630in}{2.055946in}}%
\pgfpathlineto{\pgfqpoint{3.960643in}{2.044922in}}%
\pgfpathclose%
\pgfusepath{fill}%
\end{pgfscope}%
\begin{pgfscope}%
\pgfpathrectangle{\pgfqpoint{1.150000in}{0.150000in}}{\pgfqpoint{5.700000in}{5.700000in}}%
\pgfusepath{clip}%
\pgfsetbuttcap%
\pgfsetroundjoin%
\definecolor{currentfill}{rgb}{0.273809,0.031497,0.358853}%
\pgfsetfillcolor{currentfill}%
\pgfsetfillopacity{0.800000}%
\pgfsetlinewidth{0.000000pt}%
\definecolor{currentstroke}{rgb}{0.000000,0.000000,0.000000}%
\pgfsetstrokecolor{currentstroke}%
\pgfsetdash{}{0pt}%
\pgfpathmoveto{\pgfqpoint{3.154633in}{1.828403in}}%
\pgfpathlineto{\pgfqpoint{3.168274in}{1.820634in}}%
\pgfpathlineto{\pgfqpoint{3.181915in}{1.813088in}}%
\pgfpathlineto{\pgfqpoint{3.195557in}{1.805763in}}%
\pgfpathlineto{\pgfqpoint{3.209199in}{1.798660in}}%
\pgfpathlineto{\pgfqpoint{3.217488in}{1.806486in}}%
\pgfpathlineto{\pgfqpoint{3.225768in}{1.814417in}}%
\pgfpathlineto{\pgfqpoint{3.234041in}{1.822448in}}%
\pgfpathlineto{\pgfqpoint{3.242305in}{1.830577in}}%
\pgfpathlineto{\pgfqpoint{3.228684in}{1.837308in}}%
\pgfpathlineto{\pgfqpoint{3.215063in}{1.844259in}}%
\pgfpathlineto{\pgfqpoint{3.201443in}{1.851433in}}%
\pgfpathlineto{\pgfqpoint{3.187824in}{1.858829in}}%
\pgfpathlineto{\pgfqpoint{3.179539in}{1.851061in}}%
\pgfpathlineto{\pgfqpoint{3.171246in}{1.843399in}}%
\pgfpathlineto{\pgfqpoint{3.162944in}{1.835845in}}%
\pgfpathlineto{\pgfqpoint{3.154633in}{1.828403in}}%
\pgfpathclose%
\pgfusepath{fill}%
\end{pgfscope}%
\begin{pgfscope}%
\pgfpathrectangle{\pgfqpoint{1.150000in}{0.150000in}}{\pgfqpoint{5.700000in}{5.700000in}}%
\pgfusepath{clip}%
\pgfsetbuttcap%
\pgfsetroundjoin%
\definecolor{currentfill}{rgb}{0.180653,0.701402,0.488189}%
\pgfsetfillcolor{currentfill}%
\pgfsetfillopacity{0.800000}%
\pgfsetlinewidth{0.000000pt}%
\definecolor{currentstroke}{rgb}{0.000000,0.000000,0.000000}%
\pgfsetstrokecolor{currentstroke}%
\pgfsetdash{}{0pt}%
\pgfpathmoveto{\pgfqpoint{6.173965in}{3.608146in}}%
\pgfpathlineto{\pgfqpoint{6.188784in}{3.618849in}}%
\pgfpathlineto{\pgfqpoint{6.203623in}{3.629725in}}%
\pgfpathlineto{\pgfqpoint{6.218484in}{3.640773in}}%
\pgfpathlineto{\pgfqpoint{6.225424in}{3.642621in}}%
\pgfpathlineto{\pgfqpoint{6.232366in}{3.644676in}}%
\pgfpathlineto{\pgfqpoint{6.239310in}{3.646945in}}%
\pgfpathlineto{\pgfqpoint{6.224483in}{3.636505in}}%
\pgfpathlineto{\pgfqpoint{6.209678in}{3.626237in}}%
\pgfpathlineto{\pgfqpoint{6.194893in}{3.616141in}}%
\pgfpathlineto{\pgfqpoint{6.187915in}{3.613259in}}%
\pgfpathlineto{\pgfqpoint{6.180939in}{3.610597in}}%
\pgfpathlineto{\pgfqpoint{6.173965in}{3.608146in}}%
\pgfpathclose%
\pgfusepath{fill}%
\end{pgfscope}%
\begin{pgfscope}%
\pgfpathrectangle{\pgfqpoint{1.150000in}{0.150000in}}{\pgfqpoint{5.700000in}{5.700000in}}%
\pgfusepath{clip}%
\pgfsetbuttcap%
\pgfsetroundjoin%
\definecolor{currentfill}{rgb}{0.272594,0.025563,0.353093}%
\pgfsetfillcolor{currentfill}%
\pgfsetfillopacity{0.800000}%
\pgfsetlinewidth{0.000000pt}%
\definecolor{currentstroke}{rgb}{0.000000,0.000000,0.000000}%
\pgfsetstrokecolor{currentstroke}%
\pgfsetdash{}{0pt}%
\pgfpathmoveto{\pgfqpoint{3.296807in}{1.805836in}}%
\pgfpathlineto{\pgfqpoint{3.310437in}{1.800191in}}%
\pgfpathlineto{\pgfqpoint{3.324069in}{1.794761in}}%
\pgfpathlineto{\pgfqpoint{3.337704in}{1.789544in}}%
\pgfpathlineto{\pgfqpoint{3.351342in}{1.784539in}}%
\pgfpathlineto{\pgfqpoint{3.359562in}{1.793465in}}%
\pgfpathlineto{\pgfqpoint{3.367775in}{1.802467in}}%
\pgfpathlineto{\pgfqpoint{3.375981in}{1.811539in}}%
\pgfpathlineto{\pgfqpoint{3.384180in}{1.820681in}}%
\pgfpathlineto{\pgfqpoint{3.370560in}{1.825346in}}%
\pgfpathlineto{\pgfqpoint{3.356942in}{1.830224in}}%
\pgfpathlineto{\pgfqpoint{3.343328in}{1.835314in}}%
\pgfpathlineto{\pgfqpoint{3.329715in}{1.840619in}}%
\pgfpathlineto{\pgfqpoint{3.321499in}{1.831805in}}%
\pgfpathlineto{\pgfqpoint{3.313276in}{1.823068in}}%
\pgfpathlineto{\pgfqpoint{3.305045in}{1.814410in}}%
\pgfpathlineto{\pgfqpoint{3.296807in}{1.805836in}}%
\pgfpathclose%
\pgfusepath{fill}%
\end{pgfscope}%
\begin{pgfscope}%
\pgfpathrectangle{\pgfqpoint{1.150000in}{0.150000in}}{\pgfqpoint{5.700000in}{5.700000in}}%
\pgfusepath{clip}%
\pgfsetbuttcap%
\pgfsetroundjoin%
\definecolor{currentfill}{rgb}{0.276022,0.044167,0.370164}%
\pgfsetfillcolor{currentfill}%
\pgfsetfillopacity{0.800000}%
\pgfsetlinewidth{0.000000pt}%
\definecolor{currentstroke}{rgb}{0.000000,0.000000,0.000000}%
\pgfsetstrokecolor{currentstroke}%
\pgfsetdash{}{0pt}%
\pgfpathmoveto{\pgfqpoint{3.525885in}{1.830547in}}%
\pgfpathlineto{\pgfqpoint{3.539527in}{1.828059in}}%
\pgfpathlineto{\pgfqpoint{3.553173in}{1.825775in}}%
\pgfpathlineto{\pgfqpoint{3.566825in}{1.823694in}}%
\pgfpathlineto{\pgfqpoint{3.580482in}{1.821815in}}%
\pgfpathlineto{\pgfqpoint{3.588608in}{1.832121in}}%
\pgfpathlineto{\pgfqpoint{3.596729in}{1.842456in}}%
\pgfpathlineto{\pgfqpoint{3.604843in}{1.852817in}}%
\pgfpathlineto{\pgfqpoint{3.612953in}{1.863201in}}%
\pgfpathlineto{\pgfqpoint{3.599307in}{1.864804in}}%
\pgfpathlineto{\pgfqpoint{3.585668in}{1.866609in}}%
\pgfpathlineto{\pgfqpoint{3.572033in}{1.868617in}}%
\pgfpathlineto{\pgfqpoint{3.558404in}{1.870828in}}%
\pgfpathlineto{\pgfqpoint{3.550283in}{1.860708in}}%
\pgfpathlineto{\pgfqpoint{3.542157in}{1.850620in}}%
\pgfpathlineto{\pgfqpoint{3.534024in}{1.840565in}}%
\pgfpathlineto{\pgfqpoint{3.525885in}{1.830547in}}%
\pgfpathclose%
\pgfusepath{fill}%
\end{pgfscope}%
\begin{pgfscope}%
\pgfpathrectangle{\pgfqpoint{1.150000in}{0.150000in}}{\pgfqpoint{5.700000in}{5.700000in}}%
\pgfusepath{clip}%
\pgfsetbuttcap%
\pgfsetroundjoin%
\definecolor{currentfill}{rgb}{0.241237,0.296485,0.539709}%
\pgfsetfillcolor{currentfill}%
\pgfsetfillopacity{0.800000}%
\pgfsetlinewidth{0.000000pt}%
\definecolor{currentstroke}{rgb}{0.000000,0.000000,0.000000}%
\pgfsetstrokecolor{currentstroke}%
\pgfsetdash{}{0pt}%
\pgfpathmoveto{\pgfqpoint{4.427029in}{2.389992in}}%
\pgfpathlineto{\pgfqpoint{4.440960in}{2.396532in}}%
\pgfpathlineto{\pgfqpoint{4.454904in}{2.403257in}}%
\pgfpathlineto{\pgfqpoint{4.468861in}{2.410168in}}%
\pgfpathlineto{\pgfqpoint{4.482832in}{2.417265in}}%
\pgfpathlineto{\pgfqpoint{4.490658in}{2.426982in}}%
\pgfpathlineto{\pgfqpoint{4.498479in}{2.436617in}}%
\pgfpathlineto{\pgfqpoint{4.506294in}{2.446172in}}%
\pgfpathlineto{\pgfqpoint{4.514103in}{2.455649in}}%
\pgfpathlineto{\pgfqpoint{4.500139in}{2.448628in}}%
\pgfpathlineto{\pgfqpoint{4.486188in}{2.441793in}}%
\pgfpathlineto{\pgfqpoint{4.472250in}{2.435144in}}%
\pgfpathlineto{\pgfqpoint{4.458325in}{2.428680in}}%
\pgfpathlineto{\pgfqpoint{4.450510in}{2.419115in}}%
\pgfpathlineto{\pgfqpoint{4.442689in}{2.409480in}}%
\pgfpathlineto{\pgfqpoint{4.434862in}{2.399773in}}%
\pgfpathlineto{\pgfqpoint{4.427029in}{2.389992in}}%
\pgfpathclose%
\pgfusepath{fill}%
\end{pgfscope}%
\begin{pgfscope}%
\pgfpathrectangle{\pgfqpoint{1.150000in}{0.150000in}}{\pgfqpoint{5.700000in}{5.700000in}}%
\pgfusepath{clip}%
\pgfsetbuttcap%
\pgfsetroundjoin%
\definecolor{currentfill}{rgb}{0.190631,0.407061,0.556089}%
\pgfsetfillcolor{currentfill}%
\pgfsetfillopacity{0.800000}%
\pgfsetlinewidth{0.000000pt}%
\definecolor{currentstroke}{rgb}{0.000000,0.000000,0.000000}%
\pgfsetstrokecolor{currentstroke}%
\pgfsetdash{}{0pt}%
\pgfpathmoveto{\pgfqpoint{4.806539in}{2.688989in}}%
\pgfpathlineto{\pgfqpoint{4.820662in}{2.697765in}}%
\pgfpathlineto{\pgfqpoint{4.834800in}{2.706724in}}%
\pgfpathlineto{\pgfqpoint{4.848954in}{2.715865in}}%
\pgfpathlineto{\pgfqpoint{4.863124in}{2.725189in}}%
\pgfpathlineto{\pgfqpoint{4.870792in}{2.732694in}}%
\pgfpathlineto{\pgfqpoint{4.878453in}{2.740126in}}%
\pgfpathlineto{\pgfqpoint{4.886107in}{2.747487in}}%
\pgfpathlineto{\pgfqpoint{4.893755in}{2.754781in}}%
\pgfpathlineto{\pgfqpoint{4.879595in}{2.745701in}}%
\pgfpathlineto{\pgfqpoint{4.865452in}{2.736802in}}%
\pgfpathlineto{\pgfqpoint{4.851324in}{2.728086in}}%
\pgfpathlineto{\pgfqpoint{4.837212in}{2.719551in}}%
\pgfpathlineto{\pgfqpoint{4.829554in}{2.712003in}}%
\pgfpathlineto{\pgfqpoint{4.821889in}{2.704395in}}%
\pgfpathlineto{\pgfqpoint{4.814217in}{2.696725in}}%
\pgfpathlineto{\pgfqpoint{4.806539in}{2.688989in}}%
\pgfpathclose%
\pgfusepath{fill}%
\end{pgfscope}%
\begin{pgfscope}%
\pgfpathrectangle{\pgfqpoint{1.150000in}{0.150000in}}{\pgfqpoint{5.700000in}{5.700000in}}%
\pgfusepath{clip}%
\pgfsetbuttcap%
\pgfsetroundjoin%
\definecolor{currentfill}{rgb}{0.144759,0.519093,0.556572}%
\pgfsetfillcolor{currentfill}%
\pgfsetfillopacity{0.800000}%
\pgfsetlinewidth{0.000000pt}%
\definecolor{currentstroke}{rgb}{0.000000,0.000000,0.000000}%
\pgfsetstrokecolor{currentstroke}%
\pgfsetdash{}{0pt}%
\pgfpathmoveto{\pgfqpoint{5.273004in}{3.034400in}}%
\pgfpathlineto{\pgfqpoint{5.287378in}{3.044830in}}%
\pgfpathlineto{\pgfqpoint{5.301771in}{3.055439in}}%
\pgfpathlineto{\pgfqpoint{5.316182in}{3.066228in}}%
\pgfpathlineto{\pgfqpoint{5.330611in}{3.077196in}}%
\pgfpathlineto{\pgfqpoint{5.338037in}{3.081773in}}%
\pgfpathlineto{\pgfqpoint{5.345456in}{3.086330in}}%
\pgfpathlineto{\pgfqpoint{5.352868in}{3.090872in}}%
\pgfpathlineto{\pgfqpoint{5.360274in}{3.095406in}}%
\pgfpathlineto{\pgfqpoint{5.345865in}{3.084885in}}%
\pgfpathlineto{\pgfqpoint{5.331474in}{3.074542in}}%
\pgfpathlineto{\pgfqpoint{5.317102in}{3.064378in}}%
\pgfpathlineto{\pgfqpoint{5.302747in}{3.054392in}}%
\pgfpathlineto{\pgfqpoint{5.295321in}{3.049402in}}%
\pgfpathlineto{\pgfqpoint{5.287888in}{3.044409in}}%
\pgfpathlineto{\pgfqpoint{5.280449in}{3.039410in}}%
\pgfpathlineto{\pgfqpoint{5.273004in}{3.034400in}}%
\pgfpathclose%
\pgfusepath{fill}%
\end{pgfscope}%
\begin{pgfscope}%
\pgfpathrectangle{\pgfqpoint{1.150000in}{0.150000in}}{\pgfqpoint{5.700000in}{5.700000in}}%
\pgfusepath{clip}%
\pgfsetbuttcap%
\pgfsetroundjoin%
\definecolor{currentfill}{rgb}{0.278826,0.175490,0.483397}%
\pgfsetfillcolor{currentfill}%
\pgfsetfillopacity{0.800000}%
\pgfsetlinewidth{0.000000pt}%
\definecolor{currentstroke}{rgb}{0.000000,0.000000,0.000000}%
\pgfsetstrokecolor{currentstroke}%
\pgfsetdash{}{0pt}%
\pgfpathmoveto{\pgfqpoint{4.047550in}{2.100549in}}%
\pgfpathlineto{\pgfqpoint{4.061321in}{2.103966in}}%
\pgfpathlineto{\pgfqpoint{4.075103in}{2.107574in}}%
\pgfpathlineto{\pgfqpoint{4.088895in}{2.111372in}}%
\pgfpathlineto{\pgfqpoint{4.102698in}{2.115361in}}%
\pgfpathlineto{\pgfqpoint{4.110654in}{2.126430in}}%
\pgfpathlineto{\pgfqpoint{4.118604in}{2.137445in}}%
\pgfpathlineto{\pgfqpoint{4.126550in}{2.148404in}}%
\pgfpathlineto{\pgfqpoint{4.134490in}{2.159306in}}%
\pgfpathlineto{\pgfqpoint{4.120693in}{2.155231in}}%
\pgfpathlineto{\pgfqpoint{4.106907in}{2.151346in}}%
\pgfpathlineto{\pgfqpoint{4.093130in}{2.147652in}}%
\pgfpathlineto{\pgfqpoint{4.079364in}{2.144148in}}%
\pgfpathlineto{\pgfqpoint{4.071418in}{2.133320in}}%
\pgfpathlineto{\pgfqpoint{4.063467in}{2.122444in}}%
\pgfpathlineto{\pgfqpoint{4.055511in}{2.111520in}}%
\pgfpathlineto{\pgfqpoint{4.047550in}{2.100549in}}%
\pgfpathclose%
\pgfusepath{fill}%
\end{pgfscope}%
\begin{pgfscope}%
\pgfpathrectangle{\pgfqpoint{1.150000in}{0.150000in}}{\pgfqpoint{5.700000in}{5.700000in}}%
\pgfusepath{clip}%
\pgfsetbuttcap%
\pgfsetroundjoin%
\definecolor{currentfill}{rgb}{0.206756,0.371758,0.553117}%
\pgfsetfillcolor{currentfill}%
\pgfsetfillopacity{0.800000}%
\pgfsetlinewidth{0.000000pt}%
\definecolor{currentstroke}{rgb}{0.000000,0.000000,0.000000}%
\pgfsetstrokecolor{currentstroke}%
\pgfsetdash{}{0pt}%
\pgfpathmoveto{\pgfqpoint{2.349199in}{2.663768in}}%
\pgfpathlineto{\pgfqpoint{2.363241in}{2.640257in}}%
\pgfpathlineto{\pgfqpoint{2.377269in}{2.617088in}}%
\pgfpathlineto{\pgfqpoint{2.391282in}{2.594258in}}%
\pgfpathlineto{\pgfqpoint{2.405281in}{2.571764in}}%
\pgfpathlineto{\pgfqpoint{2.414067in}{2.573135in}}%
\pgfpathlineto{\pgfqpoint{2.422836in}{2.574742in}}%
\pgfpathlineto{\pgfqpoint{2.431590in}{2.576581in}}%
\pgfpathlineto{\pgfqpoint{2.440328in}{2.578649in}}%
\pgfpathlineto{\pgfqpoint{2.426373in}{2.600706in}}%
\pgfpathlineto{\pgfqpoint{2.412405in}{2.623099in}}%
\pgfpathlineto{\pgfqpoint{2.398422in}{2.645829in}}%
\pgfpathlineto{\pgfqpoint{2.384425in}{2.668901in}}%
\pgfpathlineto{\pgfqpoint{2.375643in}{2.667258in}}%
\pgfpathlineto{\pgfqpoint{2.366845in}{2.665853in}}%
\pgfpathlineto{\pgfqpoint{2.358030in}{2.664688in}}%
\pgfpathlineto{\pgfqpoint{2.349199in}{2.663768in}}%
\pgfpathclose%
\pgfusepath{fill}%
\end{pgfscope}%
\begin{pgfscope}%
\pgfpathrectangle{\pgfqpoint{1.150000in}{0.150000in}}{\pgfqpoint{5.700000in}{5.700000in}}%
\pgfusepath{clip}%
\pgfsetbuttcap%
\pgfsetroundjoin%
\definecolor{currentfill}{rgb}{0.277941,0.056324,0.381191}%
\pgfsetfillcolor{currentfill}%
\pgfsetfillopacity{0.800000}%
\pgfsetlinewidth{0.000000pt}%
\definecolor{currentstroke}{rgb}{0.000000,0.000000,0.000000}%
\pgfsetstrokecolor{currentstroke}%
\pgfsetdash{}{0pt}%
\pgfpathmoveto{\pgfqpoint{3.011971in}{1.873417in}}%
\pgfpathlineto{\pgfqpoint{3.025642in}{1.863410in}}%
\pgfpathlineto{\pgfqpoint{3.039311in}{1.853637in}}%
\pgfpathlineto{\pgfqpoint{3.052979in}{1.844096in}}%
\pgfpathlineto{\pgfqpoint{3.066646in}{1.834787in}}%
\pgfpathlineto{\pgfqpoint{3.075016in}{1.841325in}}%
\pgfpathlineto{\pgfqpoint{3.083377in}{1.847999in}}%
\pgfpathlineto{\pgfqpoint{3.091728in}{1.854804in}}%
\pgfpathlineto{\pgfqpoint{3.100071in}{1.861736in}}%
\pgfpathlineto{\pgfqpoint{3.086429in}{1.870639in}}%
\pgfpathlineto{\pgfqpoint{3.072787in}{1.879773in}}%
\pgfpathlineto{\pgfqpoint{3.059143in}{1.889139in}}%
\pgfpathlineto{\pgfqpoint{3.045498in}{1.898739in}}%
\pgfpathlineto{\pgfqpoint{3.037131in}{1.892201in}}%
\pgfpathlineto{\pgfqpoint{3.028754in}{1.885799in}}%
\pgfpathlineto{\pgfqpoint{3.020368in}{1.879536in}}%
\pgfpathlineto{\pgfqpoint{3.011971in}{1.873417in}}%
\pgfpathclose%
\pgfusepath{fill}%
\end{pgfscope}%
\begin{pgfscope}%
\pgfpathrectangle{\pgfqpoint{1.150000in}{0.150000in}}{\pgfqpoint{5.700000in}{5.700000in}}%
\pgfusepath{clip}%
\pgfsetbuttcap%
\pgfsetroundjoin%
\definecolor{currentfill}{rgb}{0.136408,0.541173,0.554483}%
\pgfsetfillcolor{currentfill}%
\pgfsetfillopacity{0.800000}%
\pgfsetlinewidth{0.000000pt}%
\definecolor{currentstroke}{rgb}{0.000000,0.000000,0.000000}%
\pgfsetstrokecolor{currentstroke}%
\pgfsetdash{}{0pt}%
\pgfpathmoveto{\pgfqpoint{5.360274in}{3.095406in}}%
\pgfpathlineto{\pgfqpoint{5.374701in}{3.106105in}}%
\pgfpathlineto{\pgfqpoint{5.389147in}{3.116983in}}%
\pgfpathlineto{\pgfqpoint{5.403612in}{3.128040in}}%
\pgfpathlineto{\pgfqpoint{5.418095in}{3.139276in}}%
\pgfpathlineto{\pgfqpoint{5.425473in}{3.143336in}}%
\pgfpathlineto{\pgfqpoint{5.432844in}{3.147390in}}%
\pgfpathlineto{\pgfqpoint{5.440209in}{3.151441in}}%
\pgfpathlineto{\pgfqpoint{5.447567in}{3.155497in}}%
\pgfpathlineto{\pgfqpoint{5.433106in}{3.144742in}}%
\pgfpathlineto{\pgfqpoint{5.418664in}{3.134166in}}%
\pgfpathlineto{\pgfqpoint{5.404240in}{3.123767in}}%
\pgfpathlineto{\pgfqpoint{5.389835in}{3.113546in}}%
\pgfpathlineto{\pgfqpoint{5.382454in}{3.108999in}}%
\pgfpathlineto{\pgfqpoint{5.375067in}{3.104464in}}%
\pgfpathlineto{\pgfqpoint{5.367673in}{3.099934in}}%
\pgfpathlineto{\pgfqpoint{5.360274in}{3.095406in}}%
\pgfpathclose%
\pgfusepath{fill}%
\end{pgfscope}%
\begin{pgfscope}%
\pgfpathrectangle{\pgfqpoint{1.150000in}{0.150000in}}{\pgfqpoint{5.700000in}{5.700000in}}%
\pgfusepath{clip}%
\pgfsetbuttcap%
\pgfsetroundjoin%
\definecolor{currentfill}{rgb}{0.273809,0.031497,0.358853}%
\pgfsetfillcolor{currentfill}%
\pgfsetfillopacity{0.800000}%
\pgfsetlinewidth{0.000000pt}%
\definecolor{currentstroke}{rgb}{0.000000,0.000000,0.000000}%
\pgfsetstrokecolor{currentstroke}%
\pgfsetdash{}{0pt}%
\pgfpathmoveto{\pgfqpoint{3.438695in}{1.804124in}}%
\pgfpathlineto{\pgfqpoint{3.452333in}{1.800506in}}%
\pgfpathlineto{\pgfqpoint{3.465975in}{1.797094in}}%
\pgfpathlineto{\pgfqpoint{3.479621in}{1.793889in}}%
\pgfpathlineto{\pgfqpoint{3.493271in}{1.790889in}}%
\pgfpathlineto{\pgfqpoint{3.501434in}{1.800736in}}%
\pgfpathlineto{\pgfqpoint{3.509590in}{1.810630in}}%
\pgfpathlineto{\pgfqpoint{3.517741in}{1.820568in}}%
\pgfpathlineto{\pgfqpoint{3.525885in}{1.830547in}}%
\pgfpathlineto{\pgfqpoint{3.512249in}{1.833239in}}%
\pgfpathlineto{\pgfqpoint{3.498617in}{1.836137in}}%
\pgfpathlineto{\pgfqpoint{3.484989in}{1.839240in}}%
\pgfpathlineto{\pgfqpoint{3.471366in}{1.842550in}}%
\pgfpathlineto{\pgfqpoint{3.463207in}{1.832867in}}%
\pgfpathlineto{\pgfqpoint{3.455043in}{1.823233in}}%
\pgfpathlineto{\pgfqpoint{3.446872in}{1.813651in}}%
\pgfpathlineto{\pgfqpoint{3.438695in}{1.804124in}}%
\pgfpathclose%
\pgfusepath{fill}%
\end{pgfscope}%
\begin{pgfscope}%
\pgfpathrectangle{\pgfqpoint{1.150000in}{0.150000in}}{\pgfqpoint{5.700000in}{5.700000in}}%
\pgfusepath{clip}%
\pgfsetbuttcap%
\pgfsetroundjoin%
\definecolor{currentfill}{rgb}{0.273006,0.204520,0.501721}%
\pgfsetfillcolor{currentfill}%
\pgfsetfillopacity{0.800000}%
\pgfsetlinewidth{0.000000pt}%
\definecolor{currentstroke}{rgb}{0.000000,0.000000,0.000000}%
\pgfsetstrokecolor{currentstroke}%
\pgfsetdash{}{0pt}%
\pgfpathmoveto{\pgfqpoint{4.134490in}{2.159306in}}%
\pgfpathlineto{\pgfqpoint{4.148299in}{2.163570in}}%
\pgfpathlineto{\pgfqpoint{4.162118in}{2.168024in}}%
\pgfpathlineto{\pgfqpoint{4.175948in}{2.172667in}}%
\pgfpathlineto{\pgfqpoint{4.189789in}{2.177498in}}%
\pgfpathlineto{\pgfqpoint{4.197719in}{2.188411in}}%
\pgfpathlineto{\pgfqpoint{4.205644in}{2.199258in}}%
\pgfpathlineto{\pgfqpoint{4.213564in}{2.210041in}}%
\pgfpathlineto{\pgfqpoint{4.221479in}{2.220759in}}%
\pgfpathlineto{\pgfqpoint{4.207643in}{2.215873in}}%
\pgfpathlineto{\pgfqpoint{4.193818in}{2.211176in}}%
\pgfpathlineto{\pgfqpoint{4.180004in}{2.206668in}}%
\pgfpathlineto{\pgfqpoint{4.166201in}{2.202349in}}%
\pgfpathlineto{\pgfqpoint{4.158281in}{2.191673in}}%
\pgfpathlineto{\pgfqpoint{4.150356in}{2.180941in}}%
\pgfpathlineto{\pgfqpoint{4.142426in}{2.170152in}}%
\pgfpathlineto{\pgfqpoint{4.134490in}{2.159306in}}%
\pgfpathclose%
\pgfusepath{fill}%
\end{pgfscope}%
\begin{pgfscope}%
\pgfpathrectangle{\pgfqpoint{1.150000in}{0.150000in}}{\pgfqpoint{5.700000in}{5.700000in}}%
\pgfusepath{clip}%
\pgfsetbuttcap%
\pgfsetroundjoin%
\definecolor{currentfill}{rgb}{0.227802,0.326594,0.546532}%
\pgfsetfillcolor{currentfill}%
\pgfsetfillopacity{0.800000}%
\pgfsetlinewidth{0.000000pt}%
\definecolor{currentstroke}{rgb}{0.000000,0.000000,0.000000}%
\pgfsetstrokecolor{currentstroke}%
\pgfsetdash{}{0pt}%
\pgfpathmoveto{\pgfqpoint{4.514103in}{2.455649in}}%
\pgfpathlineto{\pgfqpoint{4.528081in}{2.462855in}}%
\pgfpathlineto{\pgfqpoint{4.542073in}{2.470246in}}%
\pgfpathlineto{\pgfqpoint{4.556079in}{2.477821in}}%
\pgfpathlineto{\pgfqpoint{4.570100in}{2.485582in}}%
\pgfpathlineto{\pgfqpoint{4.577896in}{2.494884in}}%
\pgfpathlineto{\pgfqpoint{4.585687in}{2.504103in}}%
\pgfpathlineto{\pgfqpoint{4.593471in}{2.513239in}}%
\pgfpathlineto{\pgfqpoint{4.601249in}{2.522294in}}%
\pgfpathlineto{\pgfqpoint{4.587236in}{2.514643in}}%
\pgfpathlineto{\pgfqpoint{4.573236in}{2.507176in}}%
\pgfpathlineto{\pgfqpoint{4.559251in}{2.499894in}}%
\pgfpathlineto{\pgfqpoint{4.545279in}{2.492797in}}%
\pgfpathlineto{\pgfqpoint{4.537494in}{2.483621in}}%
\pgfpathlineto{\pgfqpoint{4.529703in}{2.474372in}}%
\pgfpathlineto{\pgfqpoint{4.521906in}{2.465048in}}%
\pgfpathlineto{\pgfqpoint{4.514103in}{2.455649in}}%
\pgfpathclose%
\pgfusepath{fill}%
\end{pgfscope}%
\begin{pgfscope}%
\pgfpathrectangle{\pgfqpoint{1.150000in}{0.150000in}}{\pgfqpoint{5.700000in}{5.700000in}}%
\pgfusepath{clip}%
\pgfsetbuttcap%
\pgfsetroundjoin%
\definecolor{currentfill}{rgb}{0.180629,0.429975,0.557282}%
\pgfsetfillcolor{currentfill}%
\pgfsetfillopacity{0.800000}%
\pgfsetlinewidth{0.000000pt}%
\definecolor{currentstroke}{rgb}{0.000000,0.000000,0.000000}%
\pgfsetstrokecolor{currentstroke}%
\pgfsetdash{}{0pt}%
\pgfpathmoveto{\pgfqpoint{4.893755in}{2.754781in}}%
\pgfpathlineto{\pgfqpoint{4.907930in}{2.764044in}}%
\pgfpathlineto{\pgfqpoint{4.922122in}{2.773489in}}%
\pgfpathlineto{\pgfqpoint{4.936330in}{2.783115in}}%
\pgfpathlineto{\pgfqpoint{4.950555in}{2.792924in}}%
\pgfpathlineto{\pgfqpoint{4.958184in}{2.799889in}}%
\pgfpathlineto{\pgfqpoint{4.965806in}{2.806785in}}%
\pgfpathlineto{\pgfqpoint{4.973421in}{2.813615in}}%
\pgfpathlineto{\pgfqpoint{4.981029in}{2.820382in}}%
\pgfpathlineto{\pgfqpoint{4.966817in}{2.810851in}}%
\pgfpathlineto{\pgfqpoint{4.952621in}{2.801501in}}%
\pgfpathlineto{\pgfqpoint{4.938441in}{2.792333in}}%
\pgfpathlineto{\pgfqpoint{4.924277in}{2.783346in}}%
\pgfpathlineto{\pgfqpoint{4.916656in}{2.776290in}}%
\pgfpathlineto{\pgfqpoint{4.909029in}{2.769180in}}%
\pgfpathlineto{\pgfqpoint{4.901395in}{2.762011in}}%
\pgfpathlineto{\pgfqpoint{4.893755in}{2.754781in}}%
\pgfpathclose%
\pgfusepath{fill}%
\end{pgfscope}%
\begin{pgfscope}%
\pgfpathrectangle{\pgfqpoint{1.150000in}{0.150000in}}{\pgfqpoint{5.700000in}{5.700000in}}%
\pgfusepath{clip}%
\pgfsetbuttcap%
\pgfsetroundjoin%
\definecolor{currentfill}{rgb}{0.276194,0.190074,0.493001}%
\pgfsetfillcolor{currentfill}%
\pgfsetfillopacity{0.800000}%
\pgfsetlinewidth{0.000000pt}%
\definecolor{currentstroke}{rgb}{0.000000,0.000000,0.000000}%
\pgfsetstrokecolor{currentstroke}%
\pgfsetdash{}{0pt}%
\pgfpathmoveto{\pgfqpoint{2.648338in}{2.175543in}}%
\pgfpathlineto{\pgfqpoint{2.662153in}{2.158981in}}%
\pgfpathlineto{\pgfqpoint{2.675961in}{2.142695in}}%
\pgfpathlineto{\pgfqpoint{2.689760in}{2.126685in}}%
\pgfpathlineto{\pgfqpoint{2.703553in}{2.110947in}}%
\pgfpathlineto{\pgfqpoint{2.712154in}{2.114168in}}%
\pgfpathlineto{\pgfqpoint{2.720742in}{2.117590in}}%
\pgfpathlineto{\pgfqpoint{2.729317in}{2.121208in}}%
\pgfpathlineto{\pgfqpoint{2.737878in}{2.125020in}}%
\pgfpathlineto{\pgfqpoint{2.724122in}{2.140307in}}%
\pgfpathlineto{\pgfqpoint{2.710358in}{2.155866in}}%
\pgfpathlineto{\pgfqpoint{2.696588in}{2.171700in}}%
\pgfpathlineto{\pgfqpoint{2.682810in}{2.187809in}}%
\pgfpathlineto{\pgfqpoint{2.674213in}{2.184437in}}%
\pgfpathlineto{\pgfqpoint{2.665602in}{2.181265in}}%
\pgfpathlineto{\pgfqpoint{2.656977in}{2.178299in}}%
\pgfpathlineto{\pgfqpoint{2.648338in}{2.175543in}}%
\pgfpathclose%
\pgfusepath{fill}%
\end{pgfscope}%
\begin{pgfscope}%
\pgfpathrectangle{\pgfqpoint{1.150000in}{0.150000in}}{\pgfqpoint{5.700000in}{5.700000in}}%
\pgfusepath{clip}%
\pgfsetbuttcap%
\pgfsetroundjoin%
\definecolor{currentfill}{rgb}{0.280868,0.160771,0.472899}%
\pgfsetfillcolor{currentfill}%
\pgfsetfillopacity{0.800000}%
\pgfsetlinewidth{0.000000pt}%
\definecolor{currentstroke}{rgb}{0.000000,0.000000,0.000000}%
\pgfsetstrokecolor{currentstroke}%
\pgfsetdash{}{0pt}%
\pgfpathmoveto{\pgfqpoint{2.703553in}{2.110947in}}%
\pgfpathlineto{\pgfqpoint{2.717339in}{2.095480in}}%
\pgfpathlineto{\pgfqpoint{2.731118in}{2.080281in}}%
\pgfpathlineto{\pgfqpoint{2.744891in}{2.065349in}}%
\pgfpathlineto{\pgfqpoint{2.758657in}{2.050681in}}%
\pgfpathlineto{\pgfqpoint{2.767222in}{2.054363in}}%
\pgfpathlineto{\pgfqpoint{2.775775in}{2.058238in}}%
\pgfpathlineto{\pgfqpoint{2.784315in}{2.062301in}}%
\pgfpathlineto{\pgfqpoint{2.792842in}{2.066549in}}%
\pgfpathlineto{\pgfqpoint{2.779110in}{2.080769in}}%
\pgfpathlineto{\pgfqpoint{2.765372in}{2.095253in}}%
\pgfpathlineto{\pgfqpoint{2.751629in}{2.110002in}}%
\pgfpathlineto{\pgfqpoint{2.737878in}{2.125020in}}%
\pgfpathlineto{\pgfqpoint{2.729317in}{2.121208in}}%
\pgfpathlineto{\pgfqpoint{2.720742in}{2.117590in}}%
\pgfpathlineto{\pgfqpoint{2.712154in}{2.114168in}}%
\pgfpathlineto{\pgfqpoint{2.703553in}{2.110947in}}%
\pgfpathclose%
\pgfusepath{fill}%
\end{pgfscope}%
\begin{pgfscope}%
\pgfpathrectangle{\pgfqpoint{1.150000in}{0.150000in}}{\pgfqpoint{5.700000in}{5.700000in}}%
\pgfusepath{clip}%
\pgfsetbuttcap%
\pgfsetroundjoin%
\definecolor{currentfill}{rgb}{0.269308,0.218818,0.509577}%
\pgfsetfillcolor{currentfill}%
\pgfsetfillopacity{0.800000}%
\pgfsetlinewidth{0.000000pt}%
\definecolor{currentstroke}{rgb}{0.000000,0.000000,0.000000}%
\pgfsetstrokecolor{currentstroke}%
\pgfsetdash{}{0pt}%
\pgfpathmoveto{\pgfqpoint{2.592996in}{2.244608in}}%
\pgfpathlineto{\pgfqpoint{2.606844in}{2.226914in}}%
\pgfpathlineto{\pgfqpoint{2.620684in}{2.209508in}}%
\pgfpathlineto{\pgfqpoint{2.634515in}{2.192385in}}%
\pgfpathlineto{\pgfqpoint{2.648338in}{2.175543in}}%
\pgfpathlineto{\pgfqpoint{2.656977in}{2.178299in}}%
\pgfpathlineto{\pgfqpoint{2.665602in}{2.181265in}}%
\pgfpathlineto{\pgfqpoint{2.674213in}{2.184437in}}%
\pgfpathlineto{\pgfqpoint{2.682810in}{2.187809in}}%
\pgfpathlineto{\pgfqpoint{2.669025in}{2.204197in}}%
\pgfpathlineto{\pgfqpoint{2.655232in}{2.220866in}}%
\pgfpathlineto{\pgfqpoint{2.641431in}{2.237818in}}%
\pgfpathlineto{\pgfqpoint{2.627621in}{2.255056in}}%
\pgfpathlineto{\pgfqpoint{2.618986in}{2.252125in}}%
\pgfpathlineto{\pgfqpoint{2.610337in}{2.249404in}}%
\pgfpathlineto{\pgfqpoint{2.601673in}{2.246896in}}%
\pgfpathlineto{\pgfqpoint{2.592996in}{2.244608in}}%
\pgfpathclose%
\pgfusepath{fill}%
\end{pgfscope}%
\begin{pgfscope}%
\pgfpathrectangle{\pgfqpoint{1.150000in}{0.150000in}}{\pgfqpoint{5.700000in}{5.700000in}}%
\pgfusepath{clip}%
\pgfsetbuttcap%
\pgfsetroundjoin%
\definecolor{currentfill}{rgb}{0.128729,0.563265,0.551229}%
\pgfsetfillcolor{currentfill}%
\pgfsetfillopacity{0.800000}%
\pgfsetlinewidth{0.000000pt}%
\definecolor{currentstroke}{rgb}{0.000000,0.000000,0.000000}%
\pgfsetstrokecolor{currentstroke}%
\pgfsetdash{}{0pt}%
\pgfpathmoveto{\pgfqpoint{5.447567in}{3.155497in}}%
\pgfpathlineto{\pgfqpoint{5.462047in}{3.166429in}}%
\pgfpathlineto{\pgfqpoint{5.476545in}{3.177540in}}%
\pgfpathlineto{\pgfqpoint{5.491063in}{3.188829in}}%
\pgfpathlineto{\pgfqpoint{5.505600in}{3.200296in}}%
\pgfpathlineto{\pgfqpoint{5.512928in}{3.203857in}}%
\pgfpathlineto{\pgfqpoint{5.520251in}{3.207426in}}%
\pgfpathlineto{\pgfqpoint{5.527567in}{3.211007in}}%
\pgfpathlineto{\pgfqpoint{5.534877in}{3.214607in}}%
\pgfpathlineto{\pgfqpoint{5.520364in}{3.203656in}}%
\pgfpathlineto{\pgfqpoint{5.505871in}{3.192882in}}%
\pgfpathlineto{\pgfqpoint{5.491397in}{3.182286in}}%
\pgfpathlineto{\pgfqpoint{5.476941in}{3.171866in}}%
\pgfpathlineto{\pgfqpoint{5.469606in}{3.167740in}}%
\pgfpathlineto{\pgfqpoint{5.462266in}{3.163641in}}%
\pgfpathlineto{\pgfqpoint{5.454919in}{3.159561in}}%
\pgfpathlineto{\pgfqpoint{5.447567in}{3.155497in}}%
\pgfpathclose%
\pgfusepath{fill}%
\end{pgfscope}%
\begin{pgfscope}%
\pgfpathrectangle{\pgfqpoint{1.150000in}{0.150000in}}{\pgfqpoint{5.700000in}{5.700000in}}%
\pgfusepath{clip}%
\pgfsetbuttcap%
\pgfsetroundjoin%
\definecolor{currentfill}{rgb}{0.272594,0.025563,0.353093}%
\pgfsetfillcolor{currentfill}%
\pgfsetfillopacity{0.800000}%
\pgfsetlinewidth{0.000000pt}%
\definecolor{currentstroke}{rgb}{0.000000,0.000000,0.000000}%
\pgfsetstrokecolor{currentstroke}%
\pgfsetdash{}{0pt}%
\pgfpathmoveto{\pgfqpoint{3.209199in}{1.798660in}}%
\pgfpathlineto{\pgfqpoint{3.222842in}{1.791776in}}%
\pgfpathlineto{\pgfqpoint{3.236487in}{1.785110in}}%
\pgfpathlineto{\pgfqpoint{3.250133in}{1.778662in}}%
\pgfpathlineto{\pgfqpoint{3.263780in}{1.772430in}}%
\pgfpathlineto{\pgfqpoint{3.272048in}{1.780641in}}%
\pgfpathlineto{\pgfqpoint{3.280309in}{1.788948in}}%
\pgfpathlineto{\pgfqpoint{3.288562in}{1.797347in}}%
\pgfpathlineto{\pgfqpoint{3.296807in}{1.805836in}}%
\pgfpathlineto{\pgfqpoint{3.283179in}{1.811696in}}%
\pgfpathlineto{\pgfqpoint{3.269553in}{1.817772in}}%
\pgfpathlineto{\pgfqpoint{3.255928in}{1.824065in}}%
\pgfpathlineto{\pgfqpoint{3.242305in}{1.830577in}}%
\pgfpathlineto{\pgfqpoint{3.234041in}{1.822448in}}%
\pgfpathlineto{\pgfqpoint{3.225768in}{1.814417in}}%
\pgfpathlineto{\pgfqpoint{3.217488in}{1.806486in}}%
\pgfpathlineto{\pgfqpoint{3.209199in}{1.798660in}}%
\pgfpathclose%
\pgfusepath{fill}%
\end{pgfscope}%
\begin{pgfscope}%
\pgfpathrectangle{\pgfqpoint{1.150000in}{0.150000in}}{\pgfqpoint{5.700000in}{5.700000in}}%
\pgfusepath{clip}%
\pgfsetbuttcap%
\pgfsetroundjoin%
\definecolor{currentfill}{rgb}{0.265145,0.232956,0.516599}%
\pgfsetfillcolor{currentfill}%
\pgfsetfillopacity{0.800000}%
\pgfsetlinewidth{0.000000pt}%
\definecolor{currentstroke}{rgb}{0.000000,0.000000,0.000000}%
\pgfsetstrokecolor{currentstroke}%
\pgfsetdash{}{0pt}%
\pgfpathmoveto{\pgfqpoint{4.221479in}{2.220759in}}%
\pgfpathlineto{\pgfqpoint{4.235327in}{2.225833in}}%
\pgfpathlineto{\pgfqpoint{4.249186in}{2.231095in}}%
\pgfpathlineto{\pgfqpoint{4.263057in}{2.236545in}}%
\pgfpathlineto{\pgfqpoint{4.276941in}{2.242182in}}%
\pgfpathlineto{\pgfqpoint{4.284845in}{2.252869in}}%
\pgfpathlineto{\pgfqpoint{4.292744in}{2.263483in}}%
\pgfpathlineto{\pgfqpoint{4.300638in}{2.274025in}}%
\pgfpathlineto{\pgfqpoint{4.308526in}{2.284493in}}%
\pgfpathlineto{\pgfqpoint{4.294648in}{2.278834in}}%
\pgfpathlineto{\pgfqpoint{4.280782in}{2.273362in}}%
\pgfpathlineto{\pgfqpoint{4.266927in}{2.268078in}}%
\pgfpathlineto{\pgfqpoint{4.253085in}{2.262981in}}%
\pgfpathlineto{\pgfqpoint{4.245191in}{2.252522in}}%
\pgfpathlineto{\pgfqpoint{4.237292in}{2.241999in}}%
\pgfpathlineto{\pgfqpoint{4.229388in}{2.231411in}}%
\pgfpathlineto{\pgfqpoint{4.221479in}{2.220759in}}%
\pgfpathclose%
\pgfusepath{fill}%
\end{pgfscope}%
\begin{pgfscope}%
\pgfpathrectangle{\pgfqpoint{1.150000in}{0.150000in}}{\pgfqpoint{5.700000in}{5.700000in}}%
\pgfusepath{clip}%
\pgfsetbuttcap%
\pgfsetroundjoin%
\definecolor{currentfill}{rgb}{0.282884,0.135920,0.453427}%
\pgfsetfillcolor{currentfill}%
\pgfsetfillopacity{0.800000}%
\pgfsetlinewidth{0.000000pt}%
\definecolor{currentstroke}{rgb}{0.000000,0.000000,0.000000}%
\pgfsetstrokecolor{currentstroke}%
\pgfsetdash{}{0pt}%
\pgfpathmoveto{\pgfqpoint{2.758657in}{2.050681in}}%
\pgfpathlineto{\pgfqpoint{2.772418in}{2.036275in}}%
\pgfpathlineto{\pgfqpoint{2.786173in}{2.022130in}}%
\pgfpathlineto{\pgfqpoint{2.799922in}{2.008244in}}%
\pgfpathlineto{\pgfqpoint{2.813667in}{1.994614in}}%
\pgfpathlineto{\pgfqpoint{2.822198in}{1.998755in}}%
\pgfpathlineto{\pgfqpoint{2.830716in}{2.003081in}}%
\pgfpathlineto{\pgfqpoint{2.839223in}{2.007586in}}%
\pgfpathlineto{\pgfqpoint{2.847718in}{2.012268in}}%
\pgfpathlineto{\pgfqpoint{2.834007in}{2.025452in}}%
\pgfpathlineto{\pgfqpoint{2.820290in}{2.038892in}}%
\pgfpathlineto{\pgfqpoint{2.806569in}{2.052591in}}%
\pgfpathlineto{\pgfqpoint{2.792842in}{2.066549in}}%
\pgfpathlineto{\pgfqpoint{2.784315in}{2.062301in}}%
\pgfpathlineto{\pgfqpoint{2.775775in}{2.058238in}}%
\pgfpathlineto{\pgfqpoint{2.767222in}{2.054363in}}%
\pgfpathlineto{\pgfqpoint{2.758657in}{2.050681in}}%
\pgfpathclose%
\pgfusepath{fill}%
\end{pgfscope}%
\begin{pgfscope}%
\pgfpathrectangle{\pgfqpoint{1.150000in}{0.150000in}}{\pgfqpoint{5.700000in}{5.700000in}}%
\pgfusepath{clip}%
\pgfsetbuttcap%
\pgfsetroundjoin%
\definecolor{currentfill}{rgb}{0.190631,0.407061,0.556089}%
\pgfsetfillcolor{currentfill}%
\pgfsetfillopacity{0.800000}%
\pgfsetlinewidth{0.000000pt}%
\definecolor{currentstroke}{rgb}{0.000000,0.000000,0.000000}%
\pgfsetstrokecolor{currentstroke}%
\pgfsetdash{}{0pt}%
\pgfpathmoveto{\pgfqpoint{2.292874in}{2.761311in}}%
\pgfpathlineto{\pgfqpoint{2.306980in}{2.736393in}}%
\pgfpathlineto{\pgfqpoint{2.321068in}{2.711833in}}%
\pgfpathlineto{\pgfqpoint{2.335141in}{2.687626in}}%
\pgfpathlineto{\pgfqpoint{2.349199in}{2.663768in}}%
\pgfpathlineto{\pgfqpoint{2.358030in}{2.664688in}}%
\pgfpathlineto{\pgfqpoint{2.366845in}{2.665853in}}%
\pgfpathlineto{\pgfqpoint{2.375643in}{2.667258in}}%
\pgfpathlineto{\pgfqpoint{2.384425in}{2.668901in}}%
\pgfpathlineto{\pgfqpoint{2.370413in}{2.692318in}}%
\pgfpathlineto{\pgfqpoint{2.356387in}{2.716083in}}%
\pgfpathlineto{\pgfqpoint{2.342345in}{2.740200in}}%
\pgfpathlineto{\pgfqpoint{2.328287in}{2.764673in}}%
\pgfpathlineto{\pgfqpoint{2.319459in}{2.763459in}}%
\pgfpathlineto{\pgfqpoint{2.310615in}{2.762492in}}%
\pgfpathlineto{\pgfqpoint{2.301753in}{2.761774in}}%
\pgfpathlineto{\pgfqpoint{2.292874in}{2.761311in}}%
\pgfpathclose%
\pgfusepath{fill}%
\end{pgfscope}%
\begin{pgfscope}%
\pgfpathrectangle{\pgfqpoint{1.150000in}{0.150000in}}{\pgfqpoint{5.700000in}{5.700000in}}%
\pgfusepath{clip}%
\pgfsetbuttcap%
\pgfsetroundjoin%
\definecolor{currentfill}{rgb}{0.260571,0.246922,0.522828}%
\pgfsetfillcolor{currentfill}%
\pgfsetfillopacity{0.800000}%
\pgfsetlinewidth{0.000000pt}%
\definecolor{currentstroke}{rgb}{0.000000,0.000000,0.000000}%
\pgfsetstrokecolor{currentstroke}%
\pgfsetdash{}{0pt}%
\pgfpathmoveto{\pgfqpoint{2.537507in}{2.318291in}}%
\pgfpathlineto{\pgfqpoint{2.551394in}{2.299428in}}%
\pgfpathlineto{\pgfqpoint{2.565271in}{2.280862in}}%
\pgfpathlineto{\pgfqpoint{2.579138in}{2.262589in}}%
\pgfpathlineto{\pgfqpoint{2.592996in}{2.244608in}}%
\pgfpathlineto{\pgfqpoint{2.601673in}{2.246896in}}%
\pgfpathlineto{\pgfqpoint{2.610337in}{2.249404in}}%
\pgfpathlineto{\pgfqpoint{2.618986in}{2.252125in}}%
\pgfpathlineto{\pgfqpoint{2.627621in}{2.255056in}}%
\pgfpathlineto{\pgfqpoint{2.613803in}{2.272581in}}%
\pgfpathlineto{\pgfqpoint{2.599975in}{2.290397in}}%
\pgfpathlineto{\pgfqpoint{2.586139in}{2.308505in}}%
\pgfpathlineto{\pgfqpoint{2.572292in}{2.326908in}}%
\pgfpathlineto{\pgfqpoint{2.563618in}{2.324422in}}%
\pgfpathlineto{\pgfqpoint{2.554929in}{2.322154in}}%
\pgfpathlineto{\pgfqpoint{2.546226in}{2.320109in}}%
\pgfpathlineto{\pgfqpoint{2.537507in}{2.318291in}}%
\pgfpathclose%
\pgfusepath{fill}%
\end{pgfscope}%
\begin{pgfscope}%
\pgfpathrectangle{\pgfqpoint{1.150000in}{0.150000in}}{\pgfqpoint{5.700000in}{5.700000in}}%
\pgfusepath{clip}%
\pgfsetbuttcap%
\pgfsetroundjoin%
\definecolor{currentfill}{rgb}{0.216210,0.351535,0.550627}%
\pgfsetfillcolor{currentfill}%
\pgfsetfillopacity{0.800000}%
\pgfsetlinewidth{0.000000pt}%
\definecolor{currentstroke}{rgb}{0.000000,0.000000,0.000000}%
\pgfsetstrokecolor{currentstroke}%
\pgfsetdash{}{0pt}%
\pgfpathmoveto{\pgfqpoint{4.601249in}{2.522294in}}%
\pgfpathlineto{\pgfqpoint{4.615277in}{2.530130in}}%
\pgfpathlineto{\pgfqpoint{4.629319in}{2.538150in}}%
\pgfpathlineto{\pgfqpoint{4.643376in}{2.546354in}}%
\pgfpathlineto{\pgfqpoint{4.657448in}{2.554742in}}%
\pgfpathlineto{\pgfqpoint{4.665213in}{2.563588in}}%
\pgfpathlineto{\pgfqpoint{4.672971in}{2.572348in}}%
\pgfpathlineto{\pgfqpoint{4.680723in}{2.581025in}}%
\pgfpathlineto{\pgfqpoint{4.688469in}{2.589620in}}%
\pgfpathlineto{\pgfqpoint{4.674404in}{2.581375in}}%
\pgfpathlineto{\pgfqpoint{4.660355in}{2.573313in}}%
\pgfpathlineto{\pgfqpoint{4.646320in}{2.565436in}}%
\pgfpathlineto{\pgfqpoint{4.632300in}{2.557742in}}%
\pgfpathlineto{\pgfqpoint{4.624546in}{2.548992in}}%
\pgfpathlineto{\pgfqpoint{4.616787in}{2.540169in}}%
\pgfpathlineto{\pgfqpoint{4.609021in}{2.531270in}}%
\pgfpathlineto{\pgfqpoint{4.601249in}{2.522294in}}%
\pgfpathclose%
\pgfusepath{fill}%
\end{pgfscope}%
\begin{pgfscope}%
\pgfpathrectangle{\pgfqpoint{1.150000in}{0.150000in}}{\pgfqpoint{5.700000in}{5.700000in}}%
\pgfusepath{clip}%
\pgfsetbuttcap%
\pgfsetroundjoin%
\definecolor{currentfill}{rgb}{0.169646,0.456262,0.558030}%
\pgfsetfillcolor{currentfill}%
\pgfsetfillopacity{0.800000}%
\pgfsetlinewidth{0.000000pt}%
\definecolor{currentstroke}{rgb}{0.000000,0.000000,0.000000}%
\pgfsetstrokecolor{currentstroke}%
\pgfsetdash{}{0pt}%
\pgfpathmoveto{\pgfqpoint{4.981029in}{2.820382in}}%
\pgfpathlineto{\pgfqpoint{4.995259in}{2.830095in}}%
\pgfpathlineto{\pgfqpoint{5.009505in}{2.839989in}}%
\pgfpathlineto{\pgfqpoint{5.023767in}{2.850065in}}%
\pgfpathlineto{\pgfqpoint{5.038047in}{2.860322in}}%
\pgfpathlineto{\pgfqpoint{5.045636in}{2.866731in}}%
\pgfpathlineto{\pgfqpoint{5.053217in}{2.873077in}}%
\pgfpathlineto{\pgfqpoint{5.060791in}{2.879362in}}%
\pgfpathlineto{\pgfqpoint{5.068358in}{2.885592in}}%
\pgfpathlineto{\pgfqpoint{5.054092in}{2.875646in}}%
\pgfpathlineto{\pgfqpoint{5.039842in}{2.865882in}}%
\pgfpathlineto{\pgfqpoint{5.025610in}{2.856299in}}%
\pgfpathlineto{\pgfqpoint{5.011394in}{2.846896in}}%
\pgfpathlineto{\pgfqpoint{5.003813in}{2.840344in}}%
\pgfpathlineto{\pgfqpoint{4.996225in}{2.833743in}}%
\pgfpathlineto{\pgfqpoint{4.988631in}{2.827091in}}%
\pgfpathlineto{\pgfqpoint{4.981029in}{2.820382in}}%
\pgfpathclose%
\pgfusepath{fill}%
\end{pgfscope}%
\begin{pgfscope}%
\pgfpathrectangle{\pgfqpoint{1.150000in}{0.150000in}}{\pgfqpoint{5.700000in}{5.700000in}}%
\pgfusepath{clip}%
\pgfsetbuttcap%
\pgfsetroundjoin%
\definecolor{currentfill}{rgb}{0.271305,0.019942,0.347269}%
\pgfsetfillcolor{currentfill}%
\pgfsetfillopacity{0.800000}%
\pgfsetlinewidth{0.000000pt}%
\definecolor{currentstroke}{rgb}{0.000000,0.000000,0.000000}%
\pgfsetstrokecolor{currentstroke}%
\pgfsetdash{}{0pt}%
\pgfpathmoveto{\pgfqpoint{3.351342in}{1.784539in}}%
\pgfpathlineto{\pgfqpoint{3.364982in}{1.779745in}}%
\pgfpathlineto{\pgfqpoint{3.378625in}{1.775162in}}%
\pgfpathlineto{\pgfqpoint{3.392271in}{1.770788in}}%
\pgfpathlineto{\pgfqpoint{3.405921in}{1.766623in}}%
\pgfpathlineto{\pgfqpoint{3.414125in}{1.775901in}}%
\pgfpathlineto{\pgfqpoint{3.422321in}{1.785246in}}%
\pgfpathlineto{\pgfqpoint{3.430511in}{1.794654in}}%
\pgfpathlineto{\pgfqpoint{3.438695in}{1.804124in}}%
\pgfpathlineto{\pgfqpoint{3.425061in}{1.807949in}}%
\pgfpathlineto{\pgfqpoint{3.411431in}{1.811984in}}%
\pgfpathlineto{\pgfqpoint{3.397804in}{1.816227in}}%
\pgfpathlineto{\pgfqpoint{3.384180in}{1.820681in}}%
\pgfpathlineto{\pgfqpoint{3.375981in}{1.811539in}}%
\pgfpathlineto{\pgfqpoint{3.367775in}{1.802467in}}%
\pgfpathlineto{\pgfqpoint{3.359562in}{1.793465in}}%
\pgfpathlineto{\pgfqpoint{3.351342in}{1.784539in}}%
\pgfpathclose%
\pgfusepath{fill}%
\end{pgfscope}%
\begin{pgfscope}%
\pgfpathrectangle{\pgfqpoint{1.150000in}{0.150000in}}{\pgfqpoint{5.700000in}{5.700000in}}%
\pgfusepath{clip}%
\pgfsetbuttcap%
\pgfsetroundjoin%
\definecolor{currentfill}{rgb}{0.122606,0.585371,0.546557}%
\pgfsetfillcolor{currentfill}%
\pgfsetfillopacity{0.800000}%
\pgfsetlinewidth{0.000000pt}%
\definecolor{currentstroke}{rgb}{0.000000,0.000000,0.000000}%
\pgfsetstrokecolor{currentstroke}%
\pgfsetdash{}{0pt}%
\pgfpathmoveto{\pgfqpoint{5.534877in}{3.214607in}}%
\pgfpathlineto{\pgfqpoint{5.549408in}{3.225736in}}%
\pgfpathlineto{\pgfqpoint{5.563959in}{3.237043in}}%
\pgfpathlineto{\pgfqpoint{5.578529in}{3.248527in}}%
\pgfpathlineto{\pgfqpoint{5.593119in}{3.260189in}}%
\pgfpathlineto{\pgfqpoint{5.600397in}{3.263276in}}%
\pgfpathlineto{\pgfqpoint{5.607669in}{3.266385in}}%
\pgfpathlineto{\pgfqpoint{5.614936in}{3.269522in}}%
\pgfpathlineto{\pgfqpoint{5.622197in}{3.272695in}}%
\pgfpathlineto{\pgfqpoint{5.607634in}{3.261583in}}%
\pgfpathlineto{\pgfqpoint{5.593090in}{3.250648in}}%
\pgfpathlineto{\pgfqpoint{5.578566in}{3.239890in}}%
\pgfpathlineto{\pgfqpoint{5.564061in}{3.229309in}}%
\pgfpathlineto{\pgfqpoint{5.556773in}{3.225576in}}%
\pgfpathlineto{\pgfqpoint{5.549479in}{3.221886in}}%
\pgfpathlineto{\pgfqpoint{5.542181in}{3.218231in}}%
\pgfpathlineto{\pgfqpoint{5.534877in}{3.214607in}}%
\pgfpathclose%
\pgfusepath{fill}%
\end{pgfscope}%
\begin{pgfscope}%
\pgfpathrectangle{\pgfqpoint{1.150000in}{0.150000in}}{\pgfqpoint{5.700000in}{5.700000in}}%
\pgfusepath{clip}%
\pgfsetbuttcap%
\pgfsetroundjoin%
\definecolor{currentfill}{rgb}{0.276022,0.044167,0.370164}%
\pgfsetfillcolor{currentfill}%
\pgfsetfillopacity{0.800000}%
\pgfsetlinewidth{0.000000pt}%
\definecolor{currentstroke}{rgb}{0.000000,0.000000,0.000000}%
\pgfsetstrokecolor{currentstroke}%
\pgfsetdash{}{0pt}%
\pgfpathmoveto{\pgfqpoint{3.066646in}{1.834787in}}%
\pgfpathlineto{\pgfqpoint{3.080312in}{1.825707in}}%
\pgfpathlineto{\pgfqpoint{3.093976in}{1.816855in}}%
\pgfpathlineto{\pgfqpoint{3.107641in}{1.808230in}}%
\pgfpathlineto{\pgfqpoint{3.121305in}{1.799831in}}%
\pgfpathlineto{\pgfqpoint{3.129650in}{1.806787in}}%
\pgfpathlineto{\pgfqpoint{3.137987in}{1.813871in}}%
\pgfpathlineto{\pgfqpoint{3.146314in}{1.821077in}}%
\pgfpathlineto{\pgfqpoint{3.154633in}{1.828403in}}%
\pgfpathlineto{\pgfqpoint{3.140993in}{1.836397in}}%
\pgfpathlineto{\pgfqpoint{3.127353in}{1.844616in}}%
\pgfpathlineto{\pgfqpoint{3.113712in}{1.853062in}}%
\pgfpathlineto{\pgfqpoint{3.100071in}{1.861736in}}%
\pgfpathlineto{\pgfqpoint{3.091728in}{1.854804in}}%
\pgfpathlineto{\pgfqpoint{3.083377in}{1.847999in}}%
\pgfpathlineto{\pgfqpoint{3.075016in}{1.841325in}}%
\pgfpathlineto{\pgfqpoint{3.066646in}{1.834787in}}%
\pgfpathclose%
\pgfusepath{fill}%
\end{pgfscope}%
\begin{pgfscope}%
\pgfpathrectangle{\pgfqpoint{1.150000in}{0.150000in}}{\pgfqpoint{5.700000in}{5.700000in}}%
\pgfusepath{clip}%
\pgfsetbuttcap%
\pgfsetroundjoin%
\definecolor{currentfill}{rgb}{0.283197,0.115680,0.436115}%
\pgfsetfillcolor{currentfill}%
\pgfsetfillopacity{0.800000}%
\pgfsetlinewidth{0.000000pt}%
\definecolor{currentstroke}{rgb}{0.000000,0.000000,0.000000}%
\pgfsetstrokecolor{currentstroke}%
\pgfsetdash{}{0pt}%
\pgfpathmoveto{\pgfqpoint{2.813667in}{1.994614in}}%
\pgfpathlineto{\pgfqpoint{2.827406in}{1.981239in}}%
\pgfpathlineto{\pgfqpoint{2.841141in}{1.968117in}}%
\pgfpathlineto{\pgfqpoint{2.854872in}{1.955247in}}%
\pgfpathlineto{\pgfqpoint{2.868598in}{1.942626in}}%
\pgfpathlineto{\pgfqpoint{2.877096in}{1.947224in}}%
\pgfpathlineto{\pgfqpoint{2.885583in}{1.951999in}}%
\pgfpathlineto{\pgfqpoint{2.894058in}{1.956945in}}%
\pgfpathlineto{\pgfqpoint{2.902522in}{1.962058in}}%
\pgfpathlineto{\pgfqpoint{2.888827in}{1.974235in}}%
\pgfpathlineto{\pgfqpoint{2.875128in}{1.986662in}}%
\pgfpathlineto{\pgfqpoint{2.861425in}{1.999339in}}%
\pgfpathlineto{\pgfqpoint{2.847718in}{2.012268in}}%
\pgfpathlineto{\pgfqpoint{2.839223in}{2.007586in}}%
\pgfpathlineto{\pgfqpoint{2.830716in}{2.003081in}}%
\pgfpathlineto{\pgfqpoint{2.822198in}{1.998755in}}%
\pgfpathlineto{\pgfqpoint{2.813667in}{1.994614in}}%
\pgfpathclose%
\pgfusepath{fill}%
\end{pgfscope}%
\begin{pgfscope}%
\pgfpathrectangle{\pgfqpoint{1.150000in}{0.150000in}}{\pgfqpoint{5.700000in}{5.700000in}}%
\pgfusepath{clip}%
\pgfsetbuttcap%
\pgfsetroundjoin%
\definecolor{currentfill}{rgb}{0.248629,0.278775,0.534556}%
\pgfsetfillcolor{currentfill}%
\pgfsetfillopacity{0.800000}%
\pgfsetlinewidth{0.000000pt}%
\definecolor{currentstroke}{rgb}{0.000000,0.000000,0.000000}%
\pgfsetstrokecolor{currentstroke}%
\pgfsetdash{}{0pt}%
\pgfpathmoveto{\pgfqpoint{2.481853in}{2.396754in}}%
\pgfpathlineto{\pgfqpoint{2.495783in}{2.376681in}}%
\pgfpathlineto{\pgfqpoint{2.509702in}{2.356914in}}%
\pgfpathlineto{\pgfqpoint{2.523610in}{2.337452in}}%
\pgfpathlineto{\pgfqpoint{2.537507in}{2.318291in}}%
\pgfpathlineto{\pgfqpoint{2.546226in}{2.320109in}}%
\pgfpathlineto{\pgfqpoint{2.554929in}{2.322154in}}%
\pgfpathlineto{\pgfqpoint{2.563618in}{2.324422in}}%
\pgfpathlineto{\pgfqpoint{2.572292in}{2.326908in}}%
\pgfpathlineto{\pgfqpoint{2.558436in}{2.345609in}}%
\pgfpathlineto{\pgfqpoint{2.544570in}{2.364611in}}%
\pgfpathlineto{\pgfqpoint{2.530694in}{2.383916in}}%
\pgfpathlineto{\pgfqpoint{2.516806in}{2.403527in}}%
\pgfpathlineto{\pgfqpoint{2.508091in}{2.401489in}}%
\pgfpathlineto{\pgfqpoint{2.499361in}{2.399678in}}%
\pgfpathlineto{\pgfqpoint{2.490615in}{2.398098in}}%
\pgfpathlineto{\pgfqpoint{2.481853in}{2.396754in}}%
\pgfpathclose%
\pgfusepath{fill}%
\end{pgfscope}%
\begin{pgfscope}%
\pgfpathrectangle{\pgfqpoint{1.150000in}{0.150000in}}{\pgfqpoint{5.700000in}{5.700000in}}%
\pgfusepath{clip}%
\pgfsetbuttcap%
\pgfsetroundjoin%
\definecolor{currentfill}{rgb}{0.281924,0.089666,0.412415}%
\pgfsetfillcolor{currentfill}%
\pgfsetfillopacity{0.800000}%
\pgfsetlinewidth{0.000000pt}%
\definecolor{currentstroke}{rgb}{0.000000,0.000000,0.000000}%
\pgfsetstrokecolor{currentstroke}%
\pgfsetdash{}{0pt}%
\pgfpathmoveto{\pgfqpoint{3.754637in}{1.901265in}}%
\pgfpathlineto{\pgfqpoint{3.768331in}{1.901697in}}%
\pgfpathlineto{\pgfqpoint{3.782033in}{1.902326in}}%
\pgfpathlineto{\pgfqpoint{3.795742in}{1.903149in}}%
\pgfpathlineto{\pgfqpoint{3.809459in}{1.904168in}}%
\pgfpathlineto{\pgfqpoint{3.817513in}{1.915313in}}%
\pgfpathlineto{\pgfqpoint{3.825562in}{1.926446in}}%
\pgfpathlineto{\pgfqpoint{3.833606in}{1.937564in}}%
\pgfpathlineto{\pgfqpoint{3.841646in}{1.948666in}}%
\pgfpathlineto{\pgfqpoint{3.827936in}{1.947434in}}%
\pgfpathlineto{\pgfqpoint{3.814235in}{1.946396in}}%
\pgfpathlineto{\pgfqpoint{3.800542in}{1.945555in}}%
\pgfpathlineto{\pgfqpoint{3.786856in}{1.944909in}}%
\pgfpathlineto{\pgfqpoint{3.778809in}{1.934008in}}%
\pgfpathlineto{\pgfqpoint{3.770757in}{1.923100in}}%
\pgfpathlineto{\pgfqpoint{3.762700in}{1.912185in}}%
\pgfpathlineto{\pgfqpoint{3.754637in}{1.901265in}}%
\pgfpathclose%
\pgfusepath{fill}%
\end{pgfscope}%
\begin{pgfscope}%
\pgfpathrectangle{\pgfqpoint{1.150000in}{0.150000in}}{\pgfqpoint{5.700000in}{5.700000in}}%
\pgfusepath{clip}%
\pgfsetbuttcap%
\pgfsetroundjoin%
\definecolor{currentfill}{rgb}{0.279566,0.067836,0.391917}%
\pgfsetfillcolor{currentfill}%
\pgfsetfillopacity{0.800000}%
\pgfsetlinewidth{0.000000pt}%
\definecolor{currentstroke}{rgb}{0.000000,0.000000,0.000000}%
\pgfsetstrokecolor{currentstroke}%
\pgfsetdash{}{0pt}%
\pgfpathmoveto{\pgfqpoint{3.667593in}{1.858798in}}%
\pgfpathlineto{\pgfqpoint{3.681268in}{1.858196in}}%
\pgfpathlineto{\pgfqpoint{3.694950in}{1.857792in}}%
\pgfpathlineto{\pgfqpoint{3.708639in}{1.857586in}}%
\pgfpathlineto{\pgfqpoint{3.722335in}{1.857577in}}%
\pgfpathlineto{\pgfqpoint{3.730419in}{1.868496in}}%
\pgfpathlineto{\pgfqpoint{3.738497in}{1.879419in}}%
\pgfpathlineto{\pgfqpoint{3.746570in}{1.890342in}}%
\pgfpathlineto{\pgfqpoint{3.754637in}{1.901265in}}%
\pgfpathlineto{\pgfqpoint{3.740950in}{1.901029in}}%
\pgfpathlineto{\pgfqpoint{3.727271in}{1.900990in}}%
\pgfpathlineto{\pgfqpoint{3.713598in}{1.901149in}}%
\pgfpathlineto{\pgfqpoint{3.699932in}{1.901506in}}%
\pgfpathlineto{\pgfqpoint{3.691855in}{1.890816in}}%
\pgfpathlineto{\pgfqpoint{3.683773in}{1.880134in}}%
\pgfpathlineto{\pgfqpoint{3.675686in}{1.869460in}}%
\pgfpathlineto{\pgfqpoint{3.667593in}{1.858798in}}%
\pgfpathclose%
\pgfusepath{fill}%
\end{pgfscope}%
\begin{pgfscope}%
\pgfpathrectangle{\pgfqpoint{1.150000in}{0.150000in}}{\pgfqpoint{5.700000in}{5.700000in}}%
\pgfusepath{clip}%
\pgfsetbuttcap%
\pgfsetroundjoin%
\definecolor{currentfill}{rgb}{0.255645,0.260703,0.528312}%
\pgfsetfillcolor{currentfill}%
\pgfsetfillopacity{0.800000}%
\pgfsetlinewidth{0.000000pt}%
\definecolor{currentstroke}{rgb}{0.000000,0.000000,0.000000}%
\pgfsetstrokecolor{currentstroke}%
\pgfsetdash{}{0pt}%
\pgfpathmoveto{\pgfqpoint{4.308526in}{2.284493in}}%
\pgfpathlineto{\pgfqpoint{4.322417in}{2.290340in}}%
\pgfpathlineto{\pgfqpoint{4.336320in}{2.296374in}}%
\pgfpathlineto{\pgfqpoint{4.350236in}{2.302595in}}%
\pgfpathlineto{\pgfqpoint{4.364164in}{2.309002in}}%
\pgfpathlineto{\pgfqpoint{4.372042in}{2.319400in}}%
\pgfpathlineto{\pgfqpoint{4.379914in}{2.329719in}}%
\pgfpathlineto{\pgfqpoint{4.387781in}{2.339958in}}%
\pgfpathlineto{\pgfqpoint{4.395642in}{2.350119in}}%
\pgfpathlineto{\pgfqpoint{4.381719in}{2.343723in}}%
\pgfpathlineto{\pgfqpoint{4.367808in}{2.337513in}}%
\pgfpathlineto{\pgfqpoint{4.353910in}{2.331489in}}%
\pgfpathlineto{\pgfqpoint{4.340025in}{2.325653in}}%
\pgfpathlineto{\pgfqpoint{4.332159in}{2.315469in}}%
\pgfpathlineto{\pgfqpoint{4.324287in}{2.305215in}}%
\pgfpathlineto{\pgfqpoint{4.316409in}{2.294890in}}%
\pgfpathlineto{\pgfqpoint{4.308526in}{2.284493in}}%
\pgfpathclose%
\pgfusepath{fill}%
\end{pgfscope}%
\begin{pgfscope}%
\pgfpathrectangle{\pgfqpoint{1.150000in}{0.150000in}}{\pgfqpoint{5.700000in}{5.700000in}}%
\pgfusepath{clip}%
\pgfsetbuttcap%
\pgfsetroundjoin%
\definecolor{currentfill}{rgb}{0.283197,0.115680,0.436115}%
\pgfsetfillcolor{currentfill}%
\pgfsetfillopacity{0.800000}%
\pgfsetlinewidth{0.000000pt}%
\definecolor{currentstroke}{rgb}{0.000000,0.000000,0.000000}%
\pgfsetstrokecolor{currentstroke}%
\pgfsetdash{}{0pt}%
\pgfpathmoveto{\pgfqpoint{3.841646in}{1.948666in}}%
\pgfpathlineto{\pgfqpoint{3.855363in}{1.950093in}}%
\pgfpathlineto{\pgfqpoint{3.869089in}{1.951714in}}%
\pgfpathlineto{\pgfqpoint{3.882823in}{1.953528in}}%
\pgfpathlineto{\pgfqpoint{3.896566in}{1.955535in}}%
\pgfpathlineto{\pgfqpoint{3.904593in}{1.966813in}}%
\pgfpathlineto{\pgfqpoint{3.912615in}{1.978064in}}%
\pgfpathlineto{\pgfqpoint{3.920632in}{1.989287in}}%
\pgfpathlineto{\pgfqpoint{3.928644in}{2.000479in}}%
\pgfpathlineto{\pgfqpoint{3.914908in}{1.998289in}}%
\pgfpathlineto{\pgfqpoint{3.901180in}{1.996293in}}%
\pgfpathlineto{\pgfqpoint{3.887462in}{1.994490in}}%
\pgfpathlineto{\pgfqpoint{3.873752in}{1.992882in}}%
\pgfpathlineto{\pgfqpoint{3.865733in}{1.981860in}}%
\pgfpathlineto{\pgfqpoint{3.857709in}{1.970815in}}%
\pgfpathlineto{\pgfqpoint{3.849680in}{1.959750in}}%
\pgfpathlineto{\pgfqpoint{3.841646in}{1.948666in}}%
\pgfpathclose%
\pgfusepath{fill}%
\end{pgfscope}%
\begin{pgfscope}%
\pgfpathrectangle{\pgfqpoint{1.150000in}{0.150000in}}{\pgfqpoint{5.700000in}{5.700000in}}%
\pgfusepath{clip}%
\pgfsetbuttcap%
\pgfsetroundjoin%
\definecolor{currentfill}{rgb}{0.119738,0.603785,0.541400}%
\pgfsetfillcolor{currentfill}%
\pgfsetfillopacity{0.800000}%
\pgfsetlinewidth{0.000000pt}%
\definecolor{currentstroke}{rgb}{0.000000,0.000000,0.000000}%
\pgfsetstrokecolor{currentstroke}%
\pgfsetdash{}{0pt}%
\pgfpathmoveto{\pgfqpoint{5.622197in}{3.272695in}}%
\pgfpathlineto{\pgfqpoint{5.636779in}{3.283984in}}%
\pgfpathlineto{\pgfqpoint{5.651381in}{3.295449in}}%
\pgfpathlineto{\pgfqpoint{5.666003in}{3.307093in}}%
\pgfpathlineto{\pgfqpoint{5.680645in}{3.318913in}}%
\pgfpathlineto{\pgfqpoint{5.687872in}{3.321554in}}%
\pgfpathlineto{\pgfqpoint{5.695094in}{3.324235in}}%
\pgfpathlineto{\pgfqpoint{5.702310in}{3.326961in}}%
\pgfpathlineto{\pgfqpoint{5.709522in}{3.329739in}}%
\pgfpathlineto{\pgfqpoint{5.694909in}{3.318504in}}%
\pgfpathlineto{\pgfqpoint{5.680316in}{3.307444in}}%
\pgfpathlineto{\pgfqpoint{5.665743in}{3.296561in}}%
\pgfpathlineto{\pgfqpoint{5.651189in}{3.285854in}}%
\pgfpathlineto{\pgfqpoint{5.643948in}{3.282481in}}%
\pgfpathlineto{\pgfqpoint{5.636703in}{3.279168in}}%
\pgfpathlineto{\pgfqpoint{5.629452in}{3.275908in}}%
\pgfpathlineto{\pgfqpoint{5.622197in}{3.272695in}}%
\pgfpathclose%
\pgfusepath{fill}%
\end{pgfscope}%
\begin{pgfscope}%
\pgfpathrectangle{\pgfqpoint{1.150000in}{0.150000in}}{\pgfqpoint{5.700000in}{5.700000in}}%
\pgfusepath{clip}%
\pgfsetbuttcap%
\pgfsetroundjoin%
\definecolor{currentfill}{rgb}{0.277018,0.050344,0.375715}%
\pgfsetfillcolor{currentfill}%
\pgfsetfillopacity{0.800000}%
\pgfsetlinewidth{0.000000pt}%
\definecolor{currentstroke}{rgb}{0.000000,0.000000,0.000000}%
\pgfsetstrokecolor{currentstroke}%
\pgfsetdash{}{0pt}%
\pgfpathmoveto{\pgfqpoint{3.580482in}{1.821815in}}%
\pgfpathlineto{\pgfqpoint{3.594144in}{1.820137in}}%
\pgfpathlineto{\pgfqpoint{3.607812in}{1.818661in}}%
\pgfpathlineto{\pgfqpoint{3.621486in}{1.817384in}}%
\pgfpathlineto{\pgfqpoint{3.635166in}{1.816307in}}%
\pgfpathlineto{\pgfqpoint{3.643281in}{1.826901in}}%
\pgfpathlineto{\pgfqpoint{3.651390in}{1.837516in}}%
\pgfpathlineto{\pgfqpoint{3.659494in}{1.848149in}}%
\pgfpathlineto{\pgfqpoint{3.667593in}{1.858798in}}%
\pgfpathlineto{\pgfqpoint{3.653923in}{1.859599in}}%
\pgfpathlineto{\pgfqpoint{3.640260in}{1.860599in}}%
\pgfpathlineto{\pgfqpoint{3.626604in}{1.861800in}}%
\pgfpathlineto{\pgfqpoint{3.612953in}{1.863201in}}%
\pgfpathlineto{\pgfqpoint{3.604843in}{1.852817in}}%
\pgfpathlineto{\pgfqpoint{3.596729in}{1.842456in}}%
\pgfpathlineto{\pgfqpoint{3.588608in}{1.832121in}}%
\pgfpathlineto{\pgfqpoint{3.580482in}{1.821815in}}%
\pgfpathclose%
\pgfusepath{fill}%
\end{pgfscope}%
\begin{pgfscope}%
\pgfpathrectangle{\pgfqpoint{1.150000in}{0.150000in}}{\pgfqpoint{5.700000in}{5.700000in}}%
\pgfusepath{clip}%
\pgfsetbuttcap%
\pgfsetroundjoin%
\definecolor{currentfill}{rgb}{0.160665,0.478540,0.558115}%
\pgfsetfillcolor{currentfill}%
\pgfsetfillopacity{0.800000}%
\pgfsetlinewidth{0.000000pt}%
\definecolor{currentstroke}{rgb}{0.000000,0.000000,0.000000}%
\pgfsetstrokecolor{currentstroke}%
\pgfsetdash{}{0pt}%
\pgfpathmoveto{\pgfqpoint{5.068358in}{2.885592in}}%
\pgfpathlineto{\pgfqpoint{5.082642in}{2.895718in}}%
\pgfpathlineto{\pgfqpoint{5.096943in}{2.906025in}}%
\pgfpathlineto{\pgfqpoint{5.111261in}{2.916513in}}%
\pgfpathlineto{\pgfqpoint{5.125597in}{2.927182in}}%
\pgfpathlineto{\pgfqpoint{5.133142in}{2.933025in}}%
\pgfpathlineto{\pgfqpoint{5.140681in}{2.938812in}}%
\pgfpathlineto{\pgfqpoint{5.148212in}{2.944546in}}%
\pgfpathlineto{\pgfqpoint{5.155736in}{2.950231in}}%
\pgfpathlineto{\pgfqpoint{5.141415in}{2.939908in}}%
\pgfpathlineto{\pgfqpoint{5.127112in}{2.929766in}}%
\pgfpathlineto{\pgfqpoint{5.112826in}{2.919804in}}%
\pgfpathlineto{\pgfqpoint{5.098558in}{2.910022in}}%
\pgfpathlineto{\pgfqpoint{5.091018in}{2.903980in}}%
\pgfpathlineto{\pgfqpoint{5.083472in}{2.897897in}}%
\pgfpathlineto{\pgfqpoint{5.075918in}{2.891768in}}%
\pgfpathlineto{\pgfqpoint{5.068358in}{2.885592in}}%
\pgfpathclose%
\pgfusepath{fill}%
\end{pgfscope}%
\begin{pgfscope}%
\pgfpathrectangle{\pgfqpoint{1.150000in}{0.150000in}}{\pgfqpoint{5.700000in}{5.700000in}}%
\pgfusepath{clip}%
\pgfsetbuttcap%
\pgfsetroundjoin%
\definecolor{currentfill}{rgb}{0.282623,0.140926,0.457517}%
\pgfsetfillcolor{currentfill}%
\pgfsetfillopacity{0.800000}%
\pgfsetlinewidth{0.000000pt}%
\definecolor{currentstroke}{rgb}{0.000000,0.000000,0.000000}%
\pgfsetstrokecolor{currentstroke}%
\pgfsetdash{}{0pt}%
\pgfpathmoveto{\pgfqpoint{3.928644in}{2.000479in}}%
\pgfpathlineto{\pgfqpoint{3.942389in}{2.002861in}}%
\pgfpathlineto{\pgfqpoint{3.956144in}{2.005435in}}%
\pgfpathlineto{\pgfqpoint{3.969907in}{2.008201in}}%
\pgfpathlineto{\pgfqpoint{3.983681in}{2.011158in}}%
\pgfpathlineto{\pgfqpoint{3.991682in}{2.022481in}}%
\pgfpathlineto{\pgfqpoint{3.999678in}{2.033764in}}%
\pgfpathlineto{\pgfqpoint{4.007669in}{2.045004in}}%
\pgfpathlineto{\pgfqpoint{4.015655in}{2.056203in}}%
\pgfpathlineto{\pgfqpoint{4.001888in}{2.053095in}}%
\pgfpathlineto{\pgfqpoint{3.988130in}{2.050178in}}%
\pgfpathlineto{\pgfqpoint{3.974382in}{2.047454in}}%
\pgfpathlineto{\pgfqpoint{3.960643in}{2.044922in}}%
\pgfpathlineto{\pgfqpoint{3.952650in}{2.033862in}}%
\pgfpathlineto{\pgfqpoint{3.944653in}{2.022768in}}%
\pgfpathlineto{\pgfqpoint{3.936651in}{2.011639in}}%
\pgfpathlineto{\pgfqpoint{3.928644in}{2.000479in}}%
\pgfpathclose%
\pgfusepath{fill}%
\end{pgfscope}%
\begin{pgfscope}%
\pgfpathrectangle{\pgfqpoint{1.150000in}{0.150000in}}{\pgfqpoint{5.700000in}{5.700000in}}%
\pgfusepath{clip}%
\pgfsetbuttcap%
\pgfsetroundjoin%
\definecolor{currentfill}{rgb}{0.282327,0.094955,0.417331}%
\pgfsetfillcolor{currentfill}%
\pgfsetfillopacity{0.800000}%
\pgfsetlinewidth{0.000000pt}%
\definecolor{currentstroke}{rgb}{0.000000,0.000000,0.000000}%
\pgfsetstrokecolor{currentstroke}%
\pgfsetdash{}{0pt}%
\pgfpathmoveto{\pgfqpoint{2.868598in}{1.942626in}}%
\pgfpathlineto{\pgfqpoint{2.882320in}{1.930254in}}%
\pgfpathlineto{\pgfqpoint{2.896038in}{1.918127in}}%
\pgfpathlineto{\pgfqpoint{2.909753in}{1.906245in}}%
\pgfpathlineto{\pgfqpoint{2.923465in}{1.894607in}}%
\pgfpathlineto{\pgfqpoint{2.931932in}{1.899660in}}%
\pgfpathlineto{\pgfqpoint{2.940388in}{1.904881in}}%
\pgfpathlineto{\pgfqpoint{2.948834in}{1.910265in}}%
\pgfpathlineto{\pgfqpoint{2.957268in}{1.915809in}}%
\pgfpathlineto{\pgfqpoint{2.943586in}{1.927006in}}%
\pgfpathlineto{\pgfqpoint{2.929901in}{1.938445in}}%
\pgfpathlineto{\pgfqpoint{2.916213in}{1.950129in}}%
\pgfpathlineto{\pgfqpoint{2.902522in}{1.962058in}}%
\pgfpathlineto{\pgfqpoint{2.894058in}{1.956945in}}%
\pgfpathlineto{\pgfqpoint{2.885583in}{1.951999in}}%
\pgfpathlineto{\pgfqpoint{2.877096in}{1.947224in}}%
\pgfpathlineto{\pgfqpoint{2.868598in}{1.942626in}}%
\pgfpathclose%
\pgfusepath{fill}%
\end{pgfscope}%
\begin{pgfscope}%
\pgfpathrectangle{\pgfqpoint{1.150000in}{0.150000in}}{\pgfqpoint{5.700000in}{5.700000in}}%
\pgfusepath{clip}%
\pgfsetbuttcap%
\pgfsetroundjoin%
\definecolor{currentfill}{rgb}{0.203063,0.379716,0.553925}%
\pgfsetfillcolor{currentfill}%
\pgfsetfillopacity{0.800000}%
\pgfsetlinewidth{0.000000pt}%
\definecolor{currentstroke}{rgb}{0.000000,0.000000,0.000000}%
\pgfsetstrokecolor{currentstroke}%
\pgfsetdash{}{0pt}%
\pgfpathmoveto{\pgfqpoint{4.688469in}{2.589620in}}%
\pgfpathlineto{\pgfqpoint{4.702548in}{2.598049in}}%
\pgfpathlineto{\pgfqpoint{4.716642in}{2.606661in}}%
\pgfpathlineto{\pgfqpoint{4.730752in}{2.615458in}}%
\pgfpathlineto{\pgfqpoint{4.744877in}{2.624437in}}%
\pgfpathlineto{\pgfqpoint{4.752608in}{2.632789in}}%
\pgfpathlineto{\pgfqpoint{4.760332in}{2.641055in}}%
\pgfpathlineto{\pgfqpoint{4.768050in}{2.649238in}}%
\pgfpathlineto{\pgfqpoint{4.775761in}{2.657340in}}%
\pgfpathlineto{\pgfqpoint{4.761645in}{2.648537in}}%
\pgfpathlineto{\pgfqpoint{4.747544in}{2.639918in}}%
\pgfpathlineto{\pgfqpoint{4.733458in}{2.631481in}}%
\pgfpathlineto{\pgfqpoint{4.719387in}{2.623228in}}%
\pgfpathlineto{\pgfqpoint{4.711667in}{2.614938in}}%
\pgfpathlineto{\pgfqpoint{4.703941in}{2.606574in}}%
\pgfpathlineto{\pgfqpoint{4.696208in}{2.598136in}}%
\pgfpathlineto{\pgfqpoint{4.688469in}{2.589620in}}%
\pgfpathclose%
\pgfusepath{fill}%
\end{pgfscope}%
\begin{pgfscope}%
\pgfpathrectangle{\pgfqpoint{1.150000in}{0.150000in}}{\pgfqpoint{5.700000in}{5.700000in}}%
\pgfusepath{clip}%
\pgfsetbuttcap%
\pgfsetroundjoin%
\definecolor{currentfill}{rgb}{0.120081,0.622161,0.534946}%
\pgfsetfillcolor{currentfill}%
\pgfsetfillopacity{0.800000}%
\pgfsetlinewidth{0.000000pt}%
\definecolor{currentstroke}{rgb}{0.000000,0.000000,0.000000}%
\pgfsetstrokecolor{currentstroke}%
\pgfsetdash{}{0pt}%
\pgfpathmoveto{\pgfqpoint{5.709522in}{3.329739in}}%
\pgfpathlineto{\pgfqpoint{5.724154in}{3.341152in}}%
\pgfpathlineto{\pgfqpoint{5.738807in}{3.352741in}}%
\pgfpathlineto{\pgfqpoint{5.753480in}{3.364506in}}%
\pgfpathlineto{\pgfqpoint{5.768173in}{3.376449in}}%
\pgfpathlineto{\pgfqpoint{5.775348in}{3.378680in}}%
\pgfpathlineto{\pgfqpoint{5.782519in}{3.380968in}}%
\pgfpathlineto{\pgfqpoint{5.789685in}{3.383321in}}%
\pgfpathlineto{\pgfqpoint{5.796847in}{3.385745in}}%
\pgfpathlineto{\pgfqpoint{5.782185in}{3.374422in}}%
\pgfpathlineto{\pgfqpoint{5.767544in}{3.363274in}}%
\pgfpathlineto{\pgfqpoint{5.752923in}{3.352302in}}%
\pgfpathlineto{\pgfqpoint{5.738321in}{3.341506in}}%
\pgfpathlineto{\pgfqpoint{5.731128in}{3.338453in}}%
\pgfpathlineto{\pgfqpoint{5.723930in}{3.335479in}}%
\pgfpathlineto{\pgfqpoint{5.716728in}{3.332577in}}%
\pgfpathlineto{\pgfqpoint{5.709522in}{3.329739in}}%
\pgfpathclose%
\pgfusepath{fill}%
\end{pgfscope}%
\begin{pgfscope}%
\pgfpathrectangle{\pgfqpoint{1.150000in}{0.150000in}}{\pgfqpoint{5.700000in}{5.700000in}}%
\pgfusepath{clip}%
\pgfsetbuttcap%
\pgfsetroundjoin%
\definecolor{currentfill}{rgb}{0.233603,0.313828,0.543914}%
\pgfsetfillcolor{currentfill}%
\pgfsetfillopacity{0.800000}%
\pgfsetlinewidth{0.000000pt}%
\definecolor{currentstroke}{rgb}{0.000000,0.000000,0.000000}%
\pgfsetstrokecolor{currentstroke}%
\pgfsetdash{}{0pt}%
\pgfpathmoveto{\pgfqpoint{2.426015in}{2.480171in}}%
\pgfpathlineto{\pgfqpoint{2.439993in}{2.458843in}}%
\pgfpathlineto{\pgfqpoint{2.453958in}{2.437832in}}%
\pgfpathlineto{\pgfqpoint{2.467912in}{2.417137in}}%
\pgfpathlineto{\pgfqpoint{2.481853in}{2.396754in}}%
\pgfpathlineto{\pgfqpoint{2.490615in}{2.398098in}}%
\pgfpathlineto{\pgfqpoint{2.499361in}{2.399678in}}%
\pgfpathlineto{\pgfqpoint{2.508091in}{2.401489in}}%
\pgfpathlineto{\pgfqpoint{2.516806in}{2.403527in}}%
\pgfpathlineto{\pgfqpoint{2.502908in}{2.423446in}}%
\pgfpathlineto{\pgfqpoint{2.488998in}{2.443677in}}%
\pgfpathlineto{\pgfqpoint{2.475076in}{2.464221in}}%
\pgfpathlineto{\pgfqpoint{2.461143in}{2.485084in}}%
\pgfpathlineto{\pgfqpoint{2.452385in}{2.483497in}}%
\pgfpathlineto{\pgfqpoint{2.443611in}{2.482147in}}%
\pgfpathlineto{\pgfqpoint{2.434821in}{2.481037in}}%
\pgfpathlineto{\pgfqpoint{2.426015in}{2.480171in}}%
\pgfpathclose%
\pgfusepath{fill}%
\end{pgfscope}%
\begin{pgfscope}%
\pgfpathrectangle{\pgfqpoint{1.150000in}{0.150000in}}{\pgfqpoint{5.700000in}{5.700000in}}%
\pgfusepath{clip}%
\pgfsetbuttcap%
\pgfsetroundjoin%
\definecolor{currentfill}{rgb}{0.280255,0.165693,0.476498}%
\pgfsetfillcolor{currentfill}%
\pgfsetfillopacity{0.800000}%
\pgfsetlinewidth{0.000000pt}%
\definecolor{currentstroke}{rgb}{0.000000,0.000000,0.000000}%
\pgfsetstrokecolor{currentstroke}%
\pgfsetdash{}{0pt}%
\pgfpathmoveto{\pgfqpoint{4.015655in}{2.056203in}}%
\pgfpathlineto{\pgfqpoint{4.029432in}{2.059501in}}%
\pgfpathlineto{\pgfqpoint{4.043220in}{2.062991in}}%
\pgfpathlineto{\pgfqpoint{4.057017in}{2.066670in}}%
\pgfpathlineto{\pgfqpoint{4.070825in}{2.070540in}}%
\pgfpathlineto{\pgfqpoint{4.078801in}{2.081825in}}%
\pgfpathlineto{\pgfqpoint{4.086771in}{2.093057in}}%
\pgfpathlineto{\pgfqpoint{4.094737in}{2.104236in}}%
\pgfpathlineto{\pgfqpoint{4.102698in}{2.115361in}}%
\pgfpathlineto{\pgfqpoint{4.088895in}{2.111372in}}%
\pgfpathlineto{\pgfqpoint{4.075103in}{2.107574in}}%
\pgfpathlineto{\pgfqpoint{4.061321in}{2.103966in}}%
\pgfpathlineto{\pgfqpoint{4.047550in}{2.100549in}}%
\pgfpathlineto{\pgfqpoint{4.039584in}{2.089531in}}%
\pgfpathlineto{\pgfqpoint{4.031612in}{2.078466in}}%
\pgfpathlineto{\pgfqpoint{4.023636in}{2.067357in}}%
\pgfpathlineto{\pgfqpoint{4.015655in}{2.056203in}}%
\pgfpathclose%
\pgfusepath{fill}%
\end{pgfscope}%
\begin{pgfscope}%
\pgfpathrectangle{\pgfqpoint{1.150000in}{0.150000in}}{\pgfqpoint{5.700000in}{5.700000in}}%
\pgfusepath{clip}%
\pgfsetbuttcap%
\pgfsetroundjoin%
\definecolor{currentfill}{rgb}{0.273809,0.031497,0.358853}%
\pgfsetfillcolor{currentfill}%
\pgfsetfillopacity{0.800000}%
\pgfsetlinewidth{0.000000pt}%
\definecolor{currentstroke}{rgb}{0.000000,0.000000,0.000000}%
\pgfsetstrokecolor{currentstroke}%
\pgfsetdash{}{0pt}%
\pgfpathmoveto{\pgfqpoint{3.493271in}{1.790889in}}%
\pgfpathlineto{\pgfqpoint{3.506926in}{1.788094in}}%
\pgfpathlineto{\pgfqpoint{3.520585in}{1.785502in}}%
\pgfpathlineto{\pgfqpoint{3.534250in}{1.783113in}}%
\pgfpathlineto{\pgfqpoint{3.547919in}{1.780926in}}%
\pgfpathlineto{\pgfqpoint{3.556069in}{1.791092in}}%
\pgfpathlineto{\pgfqpoint{3.564212in}{1.801298in}}%
\pgfpathlineto{\pgfqpoint{3.572350in}{1.811540in}}%
\pgfpathlineto{\pgfqpoint{3.580482in}{1.821815in}}%
\pgfpathlineto{\pgfqpoint{3.566825in}{1.823694in}}%
\pgfpathlineto{\pgfqpoint{3.553173in}{1.825775in}}%
\pgfpathlineto{\pgfqpoint{3.539527in}{1.828059in}}%
\pgfpathlineto{\pgfqpoint{3.525885in}{1.830547in}}%
\pgfpathlineto{\pgfqpoint{3.517741in}{1.820568in}}%
\pgfpathlineto{\pgfqpoint{3.509590in}{1.810630in}}%
\pgfpathlineto{\pgfqpoint{3.501434in}{1.800736in}}%
\pgfpathlineto{\pgfqpoint{3.493271in}{1.790889in}}%
\pgfpathclose%
\pgfusepath{fill}%
\end{pgfscope}%
\begin{pgfscope}%
\pgfpathrectangle{\pgfqpoint{1.150000in}{0.150000in}}{\pgfqpoint{5.700000in}{5.700000in}}%
\pgfusepath{clip}%
\pgfsetbuttcap%
\pgfsetroundjoin%
\definecolor{currentfill}{rgb}{0.124780,0.640461,0.527068}%
\pgfsetfillcolor{currentfill}%
\pgfsetfillopacity{0.800000}%
\pgfsetlinewidth{0.000000pt}%
\definecolor{currentstroke}{rgb}{0.000000,0.000000,0.000000}%
\pgfsetstrokecolor{currentstroke}%
\pgfsetdash{}{0pt}%
\pgfpathmoveto{\pgfqpoint{5.796847in}{3.385745in}}%
\pgfpathlineto{\pgfqpoint{5.811528in}{3.397245in}}%
\pgfpathlineto{\pgfqpoint{5.826230in}{3.408920in}}%
\pgfpathlineto{\pgfqpoint{5.840952in}{3.420772in}}%
\pgfpathlineto{\pgfqpoint{5.855695in}{3.432801in}}%
\pgfpathlineto{\pgfqpoint{5.862819in}{3.434662in}}%
\pgfpathlineto{\pgfqpoint{5.869939in}{3.436601in}}%
\pgfpathlineto{\pgfqpoint{5.877055in}{3.438625in}}%
\pgfpathlineto{\pgfqpoint{5.884167in}{3.440740in}}%
\pgfpathlineto{\pgfqpoint{5.869459in}{3.429365in}}%
\pgfpathlineto{\pgfqpoint{5.854771in}{3.418165in}}%
\pgfpathlineto{\pgfqpoint{5.840103in}{3.407141in}}%
\pgfpathlineto{\pgfqpoint{5.825455in}{3.396291in}}%
\pgfpathlineto{\pgfqpoint{5.818308in}{3.393513in}}%
\pgfpathlineto{\pgfqpoint{5.811158in}{3.390834in}}%
\pgfpathlineto{\pgfqpoint{5.804004in}{3.388247in}}%
\pgfpathlineto{\pgfqpoint{5.796847in}{3.385745in}}%
\pgfpathclose%
\pgfusepath{fill}%
\end{pgfscope}%
\begin{pgfscope}%
\pgfpathrectangle{\pgfqpoint{1.150000in}{0.150000in}}{\pgfqpoint{5.700000in}{5.700000in}}%
\pgfusepath{clip}%
\pgfsetbuttcap%
\pgfsetroundjoin%
\definecolor{currentfill}{rgb}{0.243113,0.292092,0.538516}%
\pgfsetfillcolor{currentfill}%
\pgfsetfillopacity{0.800000}%
\pgfsetlinewidth{0.000000pt}%
\definecolor{currentstroke}{rgb}{0.000000,0.000000,0.000000}%
\pgfsetstrokecolor{currentstroke}%
\pgfsetdash{}{0pt}%
\pgfpathmoveto{\pgfqpoint{4.395642in}{2.350119in}}%
\pgfpathlineto{\pgfqpoint{4.409578in}{2.356702in}}%
\pgfpathlineto{\pgfqpoint{4.423527in}{2.363471in}}%
\pgfpathlineto{\pgfqpoint{4.437490in}{2.370425in}}%
\pgfpathlineto{\pgfqpoint{4.451466in}{2.377566in}}%
\pgfpathlineto{\pgfqpoint{4.459316in}{2.387618in}}%
\pgfpathlineto{\pgfqpoint{4.467160in}{2.397584in}}%
\pgfpathlineto{\pgfqpoint{4.474999in}{2.407466in}}%
\pgfpathlineto{\pgfqpoint{4.482832in}{2.417265in}}%
\pgfpathlineto{\pgfqpoint{4.468861in}{2.410168in}}%
\pgfpathlineto{\pgfqpoint{4.454904in}{2.403257in}}%
\pgfpathlineto{\pgfqpoint{4.440960in}{2.396532in}}%
\pgfpathlineto{\pgfqpoint{4.427029in}{2.389992in}}%
\pgfpathlineto{\pgfqpoint{4.419191in}{2.380138in}}%
\pgfpathlineto{\pgfqpoint{4.411347in}{2.370208in}}%
\pgfpathlineto{\pgfqpoint{4.403497in}{2.360202in}}%
\pgfpathlineto{\pgfqpoint{4.395642in}{2.350119in}}%
\pgfpathclose%
\pgfusepath{fill}%
\end{pgfscope}%
\begin{pgfscope}%
\pgfpathrectangle{\pgfqpoint{1.150000in}{0.150000in}}{\pgfqpoint{5.700000in}{5.700000in}}%
\pgfusepath{clip}%
\pgfsetbuttcap%
\pgfsetroundjoin%
\definecolor{currentfill}{rgb}{0.271305,0.019942,0.347269}%
\pgfsetfillcolor{currentfill}%
\pgfsetfillopacity{0.800000}%
\pgfsetlinewidth{0.000000pt}%
\definecolor{currentstroke}{rgb}{0.000000,0.000000,0.000000}%
\pgfsetstrokecolor{currentstroke}%
\pgfsetdash{}{0pt}%
\pgfpathmoveto{\pgfqpoint{3.263780in}{1.772430in}}%
\pgfpathlineto{\pgfqpoint{3.277429in}{1.766414in}}%
\pgfpathlineto{\pgfqpoint{3.291081in}{1.760612in}}%
\pgfpathlineto{\pgfqpoint{3.304734in}{1.755023in}}%
\pgfpathlineto{\pgfqpoint{3.318389in}{1.749647in}}%
\pgfpathlineto{\pgfqpoint{3.326638in}{1.758241in}}%
\pgfpathlineto{\pgfqpoint{3.334880in}{1.766923in}}%
\pgfpathlineto{\pgfqpoint{3.343114in}{1.775690in}}%
\pgfpathlineto{\pgfqpoint{3.351342in}{1.784539in}}%
\pgfpathlineto{\pgfqpoint{3.337704in}{1.789544in}}%
\pgfpathlineto{\pgfqpoint{3.324069in}{1.794761in}}%
\pgfpathlineto{\pgfqpoint{3.310437in}{1.800191in}}%
\pgfpathlineto{\pgfqpoint{3.296807in}{1.805836in}}%
\pgfpathlineto{\pgfqpoint{3.288562in}{1.797347in}}%
\pgfpathlineto{\pgfqpoint{3.280309in}{1.788948in}}%
\pgfpathlineto{\pgfqpoint{3.272048in}{1.780641in}}%
\pgfpathlineto{\pgfqpoint{3.263780in}{1.772430in}}%
\pgfpathclose%
\pgfusepath{fill}%
\end{pgfscope}%
\begin{pgfscope}%
\pgfpathrectangle{\pgfqpoint{1.150000in}{0.150000in}}{\pgfqpoint{5.700000in}{5.700000in}}%
\pgfusepath{clip}%
\pgfsetbuttcap%
\pgfsetroundjoin%
\definecolor{currentfill}{rgb}{0.273809,0.031497,0.358853}%
\pgfsetfillcolor{currentfill}%
\pgfsetfillopacity{0.800000}%
\pgfsetlinewidth{0.000000pt}%
\definecolor{currentstroke}{rgb}{0.000000,0.000000,0.000000}%
\pgfsetstrokecolor{currentstroke}%
\pgfsetdash{}{0pt}%
\pgfpathmoveto{\pgfqpoint{3.121305in}{1.799831in}}%
\pgfpathlineto{\pgfqpoint{3.134969in}{1.791657in}}%
\pgfpathlineto{\pgfqpoint{3.148633in}{1.783705in}}%
\pgfpathlineto{\pgfqpoint{3.162297in}{1.775976in}}%
\pgfpathlineto{\pgfqpoint{3.175961in}{1.768468in}}%
\pgfpathlineto{\pgfqpoint{3.184284in}{1.775841in}}%
\pgfpathlineto{\pgfqpoint{3.192597in}{1.783334in}}%
\pgfpathlineto{\pgfqpoint{3.200902in}{1.790941in}}%
\pgfpathlineto{\pgfqpoint{3.209199in}{1.798660in}}%
\pgfpathlineto{\pgfqpoint{3.195557in}{1.805763in}}%
\pgfpathlineto{\pgfqpoint{3.181915in}{1.813088in}}%
\pgfpathlineto{\pgfqpoint{3.168274in}{1.820634in}}%
\pgfpathlineto{\pgfqpoint{3.154633in}{1.828403in}}%
\pgfpathlineto{\pgfqpoint{3.146314in}{1.821077in}}%
\pgfpathlineto{\pgfqpoint{3.137987in}{1.813871in}}%
\pgfpathlineto{\pgfqpoint{3.129650in}{1.806787in}}%
\pgfpathlineto{\pgfqpoint{3.121305in}{1.799831in}}%
\pgfpathclose%
\pgfusepath{fill}%
\end{pgfscope}%
\begin{pgfscope}%
\pgfpathrectangle{\pgfqpoint{1.150000in}{0.150000in}}{\pgfqpoint{5.700000in}{5.700000in}}%
\pgfusepath{clip}%
\pgfsetbuttcap%
\pgfsetroundjoin%
\definecolor{currentfill}{rgb}{0.151918,0.500685,0.557587}%
\pgfsetfillcolor{currentfill}%
\pgfsetfillopacity{0.800000}%
\pgfsetlinewidth{0.000000pt}%
\definecolor{currentstroke}{rgb}{0.000000,0.000000,0.000000}%
\pgfsetstrokecolor{currentstroke}%
\pgfsetdash{}{0pt}%
\pgfpathmoveto{\pgfqpoint{5.155736in}{2.950231in}}%
\pgfpathlineto{\pgfqpoint{5.170074in}{2.960734in}}%
\pgfpathlineto{\pgfqpoint{5.184430in}{2.971417in}}%
\pgfpathlineto{\pgfqpoint{5.198804in}{2.982282in}}%
\pgfpathlineto{\pgfqpoint{5.213196in}{2.993326in}}%
\pgfpathlineto{\pgfqpoint{5.220697in}{2.998599in}}%
\pgfpathlineto{\pgfqpoint{5.228190in}{3.003823in}}%
\pgfpathlineto{\pgfqpoint{5.235677in}{3.009003in}}%
\pgfpathlineto{\pgfqpoint{5.243156in}{3.014144in}}%
\pgfpathlineto{\pgfqpoint{5.228781in}{3.003480in}}%
\pgfpathlineto{\pgfqpoint{5.214424in}{2.992996in}}%
\pgfpathlineto{\pgfqpoint{5.200085in}{2.982692in}}%
\pgfpathlineto{\pgfqpoint{5.185763in}{2.972568in}}%
\pgfpathlineto{\pgfqpoint{5.178267in}{2.967036in}}%
\pgfpathlineto{\pgfqpoint{5.170763in}{2.961472in}}%
\pgfpathlineto{\pgfqpoint{5.163253in}{2.955872in}}%
\pgfpathlineto{\pgfqpoint{5.155736in}{2.950231in}}%
\pgfpathclose%
\pgfusepath{fill}%
\end{pgfscope}%
\begin{pgfscope}%
\pgfpathrectangle{\pgfqpoint{1.150000in}{0.150000in}}{\pgfqpoint{5.700000in}{5.700000in}}%
\pgfusepath{clip}%
\pgfsetbuttcap%
\pgfsetroundjoin%
\definecolor{currentfill}{rgb}{0.137339,0.662252,0.515571}%
\pgfsetfillcolor{currentfill}%
\pgfsetfillopacity{0.800000}%
\pgfsetlinewidth{0.000000pt}%
\definecolor{currentstroke}{rgb}{0.000000,0.000000,0.000000}%
\pgfsetstrokecolor{currentstroke}%
\pgfsetdash{}{0pt}%
\pgfpathmoveto{\pgfqpoint{5.884167in}{3.440740in}}%
\pgfpathlineto{\pgfqpoint{5.898896in}{3.452291in}}%
\pgfpathlineto{\pgfqpoint{5.913646in}{3.464017in}}%
\pgfpathlineto{\pgfqpoint{5.928417in}{3.475919in}}%
\pgfpathlineto{\pgfqpoint{5.943208in}{3.487996in}}%
\pgfpathlineto{\pgfqpoint{5.950281in}{3.489536in}}%
\pgfpathlineto{\pgfqpoint{5.957351in}{3.491173in}}%
\pgfpathlineto{\pgfqpoint{5.964417in}{3.492917in}}%
\pgfpathlineto{\pgfqpoint{5.971481in}{3.494775in}}%
\pgfpathlineto{\pgfqpoint{5.956726in}{3.483385in}}%
\pgfpathlineto{\pgfqpoint{5.941993in}{3.472169in}}%
\pgfpathlineto{\pgfqpoint{5.927279in}{3.461128in}}%
\pgfpathlineto{\pgfqpoint{5.912586in}{3.450261in}}%
\pgfpathlineto{\pgfqpoint{5.905485in}{3.447707in}}%
\pgfpathlineto{\pgfqpoint{5.898382in}{3.445274in}}%
\pgfpathlineto{\pgfqpoint{5.891276in}{3.442954in}}%
\pgfpathlineto{\pgfqpoint{5.884167in}{3.440740in}}%
\pgfpathclose%
\pgfusepath{fill}%
\end{pgfscope}%
\begin{pgfscope}%
\pgfpathrectangle{\pgfqpoint{1.150000in}{0.150000in}}{\pgfqpoint{5.700000in}{5.700000in}}%
\pgfusepath{clip}%
\pgfsetbuttcap%
\pgfsetroundjoin%
\definecolor{currentfill}{rgb}{0.280894,0.078907,0.402329}%
\pgfsetfillcolor{currentfill}%
\pgfsetfillopacity{0.800000}%
\pgfsetlinewidth{0.000000pt}%
\definecolor{currentstroke}{rgb}{0.000000,0.000000,0.000000}%
\pgfsetstrokecolor{currentstroke}%
\pgfsetdash{}{0pt}%
\pgfpathmoveto{\pgfqpoint{2.923465in}{1.894607in}}%
\pgfpathlineto{\pgfqpoint{2.937174in}{1.883209in}}%
\pgfpathlineto{\pgfqpoint{2.950880in}{1.872052in}}%
\pgfpathlineto{\pgfqpoint{2.964583in}{1.861133in}}%
\pgfpathlineto{\pgfqpoint{2.978284in}{1.850451in}}%
\pgfpathlineto{\pgfqpoint{2.986721in}{1.855957in}}%
\pgfpathlineto{\pgfqpoint{2.995148in}{1.861623in}}%
\pgfpathlineto{\pgfqpoint{3.003565in}{1.867444in}}%
\pgfpathlineto{\pgfqpoint{3.011971in}{1.873417in}}%
\pgfpathlineto{\pgfqpoint{2.998299in}{1.883658in}}%
\pgfpathlineto{\pgfqpoint{2.984624in}{1.894137in}}%
\pgfpathlineto{\pgfqpoint{2.970947in}{1.904853in}}%
\pgfpathlineto{\pgfqpoint{2.957268in}{1.915809in}}%
\pgfpathlineto{\pgfqpoint{2.948834in}{1.910265in}}%
\pgfpathlineto{\pgfqpoint{2.940388in}{1.904881in}}%
\pgfpathlineto{\pgfqpoint{2.931932in}{1.899660in}}%
\pgfpathlineto{\pgfqpoint{2.923465in}{1.894607in}}%
\pgfpathclose%
\pgfusepath{fill}%
\end{pgfscope}%
\begin{pgfscope}%
\pgfpathrectangle{\pgfqpoint{1.150000in}{0.150000in}}{\pgfqpoint{5.700000in}{5.700000in}}%
\pgfusepath{clip}%
\pgfsetbuttcap%
\pgfsetroundjoin%
\definecolor{currentfill}{rgb}{0.190631,0.407061,0.556089}%
\pgfsetfillcolor{currentfill}%
\pgfsetfillopacity{0.800000}%
\pgfsetlinewidth{0.000000pt}%
\definecolor{currentstroke}{rgb}{0.000000,0.000000,0.000000}%
\pgfsetstrokecolor{currentstroke}%
\pgfsetdash{}{0pt}%
\pgfpathmoveto{\pgfqpoint{4.775761in}{2.657340in}}%
\pgfpathlineto{\pgfqpoint{4.789893in}{2.666326in}}%
\pgfpathlineto{\pgfqpoint{4.804041in}{2.675495in}}%
\pgfpathlineto{\pgfqpoint{4.818205in}{2.684847in}}%
\pgfpathlineto{\pgfqpoint{4.832385in}{2.694382in}}%
\pgfpathlineto{\pgfqpoint{4.840080in}{2.702207in}}%
\pgfpathlineto{\pgfqpoint{4.847768in}{2.709948in}}%
\pgfpathlineto{\pgfqpoint{4.855450in}{2.717608in}}%
\pgfpathlineto{\pgfqpoint{4.863124in}{2.725189in}}%
\pgfpathlineto{\pgfqpoint{4.848954in}{2.715865in}}%
\pgfpathlineto{\pgfqpoint{4.834800in}{2.706724in}}%
\pgfpathlineto{\pgfqpoint{4.820662in}{2.697765in}}%
\pgfpathlineto{\pgfqpoint{4.806539in}{2.688989in}}%
\pgfpathlineto{\pgfqpoint{4.798855in}{2.681186in}}%
\pgfpathlineto{\pgfqpoint{4.791163in}{2.673311in}}%
\pgfpathlineto{\pgfqpoint{4.783466in}{2.665364in}}%
\pgfpathlineto{\pgfqpoint{4.775761in}{2.657340in}}%
\pgfpathclose%
\pgfusepath{fill}%
\end{pgfscope}%
\begin{pgfscope}%
\pgfpathrectangle{\pgfqpoint{1.150000in}{0.150000in}}{\pgfqpoint{5.700000in}{5.700000in}}%
\pgfusepath{clip}%
\pgfsetbuttcap%
\pgfsetroundjoin%
\definecolor{currentfill}{rgb}{0.191090,0.708366,0.482284}%
\pgfsetfillcolor{currentfill}%
\pgfsetfillopacity{0.800000}%
\pgfsetlinewidth{0.000000pt}%
\definecolor{currentstroke}{rgb}{0.000000,0.000000,0.000000}%
\pgfsetstrokecolor{currentstroke}%
\pgfsetdash{}{0pt}%
\pgfpathmoveto{\pgfqpoint{6.146080in}{3.600290in}}%
\pgfpathlineto{\pgfqpoint{6.160943in}{3.611777in}}%
\pgfpathlineto{\pgfqpoint{6.175827in}{3.623438in}}%
\pgfpathlineto{\pgfqpoint{6.190733in}{3.635273in}}%
\pgfpathlineto{\pgfqpoint{6.197670in}{3.636381in}}%
\pgfpathlineto{\pgfqpoint{6.204608in}{3.637661in}}%
\pgfpathlineto{\pgfqpoint{6.211546in}{3.639122in}}%
\pgfpathlineto{\pgfqpoint{6.218484in}{3.640773in}}%
\pgfpathlineto{\pgfqpoint{6.203623in}{3.629725in}}%
\pgfpathlineto{\pgfqpoint{6.188784in}{3.618849in}}%
\pgfpathlineto{\pgfqpoint{6.173965in}{3.608146in}}%
\pgfpathlineto{\pgfqpoint{6.166993in}{3.605899in}}%
\pgfpathlineto{\pgfqpoint{6.160021in}{3.603846in}}%
\pgfpathlineto{\pgfqpoint{6.153051in}{3.601979in}}%
\pgfpathlineto{\pgfqpoint{6.146080in}{3.600290in}}%
\pgfpathclose%
\pgfusepath{fill}%
\end{pgfscope}%
\begin{pgfscope}%
\pgfpathrectangle{\pgfqpoint{1.150000in}{0.150000in}}{\pgfqpoint{5.700000in}{5.700000in}}%
\pgfusepath{clip}%
\pgfsetbuttcap%
\pgfsetroundjoin%
\definecolor{currentfill}{rgb}{0.275191,0.194905,0.496005}%
\pgfsetfillcolor{currentfill}%
\pgfsetfillopacity{0.800000}%
\pgfsetlinewidth{0.000000pt}%
\definecolor{currentstroke}{rgb}{0.000000,0.000000,0.000000}%
\pgfsetstrokecolor{currentstroke}%
\pgfsetdash{}{0pt}%
\pgfpathmoveto{\pgfqpoint{4.102698in}{2.115361in}}%
\pgfpathlineto{\pgfqpoint{4.116511in}{2.119538in}}%
\pgfpathlineto{\pgfqpoint{4.130335in}{2.123906in}}%
\pgfpathlineto{\pgfqpoint{4.144170in}{2.128462in}}%
\pgfpathlineto{\pgfqpoint{4.158016in}{2.133207in}}%
\pgfpathlineto{\pgfqpoint{4.165967in}{2.144375in}}%
\pgfpathlineto{\pgfqpoint{4.173913in}{2.155480in}}%
\pgfpathlineto{\pgfqpoint{4.181853in}{2.166521in}}%
\pgfpathlineto{\pgfqpoint{4.189789in}{2.177498in}}%
\pgfpathlineto{\pgfqpoint{4.175948in}{2.172667in}}%
\pgfpathlineto{\pgfqpoint{4.162118in}{2.168024in}}%
\pgfpathlineto{\pgfqpoint{4.148299in}{2.163570in}}%
\pgfpathlineto{\pgfqpoint{4.134490in}{2.159306in}}%
\pgfpathlineto{\pgfqpoint{4.126550in}{2.148404in}}%
\pgfpathlineto{\pgfqpoint{4.118604in}{2.137445in}}%
\pgfpathlineto{\pgfqpoint{4.110654in}{2.126430in}}%
\pgfpathlineto{\pgfqpoint{4.102698in}{2.115361in}}%
\pgfpathclose%
\pgfusepath{fill}%
\end{pgfscope}%
\begin{pgfscope}%
\pgfpathrectangle{\pgfqpoint{1.150000in}{0.150000in}}{\pgfqpoint{5.700000in}{5.700000in}}%
\pgfusepath{clip}%
\pgfsetbuttcap%
\pgfsetroundjoin%
\definecolor{currentfill}{rgb}{0.218130,0.347432,0.550038}%
\pgfsetfillcolor{currentfill}%
\pgfsetfillopacity{0.800000}%
\pgfsetlinewidth{0.000000pt}%
\definecolor{currentstroke}{rgb}{0.000000,0.000000,0.000000}%
\pgfsetstrokecolor{currentstroke}%
\pgfsetdash{}{0pt}%
\pgfpathmoveto{\pgfqpoint{2.369971in}{2.568729in}}%
\pgfpathlineto{\pgfqpoint{2.384002in}{2.546097in}}%
\pgfpathlineto{\pgfqpoint{2.398020in}{2.523795in}}%
\pgfpathlineto{\pgfqpoint{2.412024in}{2.501821in}}%
\pgfpathlineto{\pgfqpoint{2.426015in}{2.480171in}}%
\pgfpathlineto{\pgfqpoint{2.434821in}{2.481037in}}%
\pgfpathlineto{\pgfqpoint{2.443611in}{2.482147in}}%
\pgfpathlineto{\pgfqpoint{2.452385in}{2.483497in}}%
\pgfpathlineto{\pgfqpoint{2.461143in}{2.485084in}}%
\pgfpathlineto{\pgfqpoint{2.447196in}{2.506266in}}%
\pgfpathlineto{\pgfqpoint{2.433238in}{2.527771in}}%
\pgfpathlineto{\pgfqpoint{2.419266in}{2.549603in}}%
\pgfpathlineto{\pgfqpoint{2.405281in}{2.571764in}}%
\pgfpathlineto{\pgfqpoint{2.396478in}{2.570634in}}%
\pgfpathlineto{\pgfqpoint{2.387660in}{2.569748in}}%
\pgfpathlineto{\pgfqpoint{2.378824in}{2.569112in}}%
\pgfpathlineto{\pgfqpoint{2.369971in}{2.568729in}}%
\pgfpathclose%
\pgfusepath{fill}%
\end{pgfscope}%
\begin{pgfscope}%
\pgfpathrectangle{\pgfqpoint{1.150000in}{0.150000in}}{\pgfqpoint{5.700000in}{5.700000in}}%
\pgfusepath{clip}%
\pgfsetbuttcap%
\pgfsetroundjoin%
\definecolor{currentfill}{rgb}{0.153894,0.680203,0.504172}%
\pgfsetfillcolor{currentfill}%
\pgfsetfillopacity{0.800000}%
\pgfsetlinewidth{0.000000pt}%
\definecolor{currentstroke}{rgb}{0.000000,0.000000,0.000000}%
\pgfsetstrokecolor{currentstroke}%
\pgfsetdash{}{0pt}%
\pgfpathmoveto{\pgfqpoint{5.971481in}{3.494775in}}%
\pgfpathlineto{\pgfqpoint{5.986256in}{3.506340in}}%
\pgfpathlineto{\pgfqpoint{6.001052in}{3.518081in}}%
\pgfpathlineto{\pgfqpoint{6.015870in}{3.529996in}}%
\pgfpathlineto{\pgfqpoint{6.030708in}{3.542087in}}%
\pgfpathlineto{\pgfqpoint{6.037731in}{3.543357in}}%
\pgfpathlineto{\pgfqpoint{6.044751in}{3.544748in}}%
\pgfpathlineto{\pgfqpoint{6.051769in}{3.546268in}}%
\pgfpathlineto{\pgfqpoint{6.058785in}{3.547925in}}%
\pgfpathlineto{\pgfqpoint{6.043986in}{3.536555in}}%
\pgfpathlineto{\pgfqpoint{6.029209in}{3.525360in}}%
\pgfpathlineto{\pgfqpoint{6.014451in}{3.514338in}}%
\pgfpathlineto{\pgfqpoint{5.999715in}{3.503490in}}%
\pgfpathlineto{\pgfqpoint{5.992659in}{3.501103in}}%
\pgfpathlineto{\pgfqpoint{5.985601in}{3.498860in}}%
\pgfpathlineto{\pgfqpoint{5.978542in}{3.496753in}}%
\pgfpathlineto{\pgfqpoint{5.971481in}{3.494775in}}%
\pgfpathclose%
\pgfusepath{fill}%
\end{pgfscope}%
\begin{pgfscope}%
\pgfpathrectangle{\pgfqpoint{1.150000in}{0.150000in}}{\pgfqpoint{5.700000in}{5.700000in}}%
\pgfusepath{clip}%
\pgfsetbuttcap%
\pgfsetroundjoin%
\definecolor{currentfill}{rgb}{0.272594,0.025563,0.353093}%
\pgfsetfillcolor{currentfill}%
\pgfsetfillopacity{0.800000}%
\pgfsetlinewidth{0.000000pt}%
\definecolor{currentstroke}{rgb}{0.000000,0.000000,0.000000}%
\pgfsetstrokecolor{currentstroke}%
\pgfsetdash{}{0pt}%
\pgfpathmoveto{\pgfqpoint{3.405921in}{1.766623in}}%
\pgfpathlineto{\pgfqpoint{3.419574in}{1.762666in}}%
\pgfpathlineto{\pgfqpoint{3.433231in}{1.758915in}}%
\pgfpathlineto{\pgfqpoint{3.446892in}{1.755370in}}%
\pgfpathlineto{\pgfqpoint{3.460557in}{1.752031in}}%
\pgfpathlineto{\pgfqpoint{3.468745in}{1.761660in}}%
\pgfpathlineto{\pgfqpoint{3.476927in}{1.771348in}}%
\pgfpathlineto{\pgfqpoint{3.485102in}{1.781092in}}%
\pgfpathlineto{\pgfqpoint{3.493271in}{1.790889in}}%
\pgfpathlineto{\pgfqpoint{3.479621in}{1.793889in}}%
\pgfpathlineto{\pgfqpoint{3.465975in}{1.797094in}}%
\pgfpathlineto{\pgfqpoint{3.452333in}{1.800506in}}%
\pgfpathlineto{\pgfqpoint{3.438695in}{1.804124in}}%
\pgfpathlineto{\pgfqpoint{3.430511in}{1.794654in}}%
\pgfpathlineto{\pgfqpoint{3.422321in}{1.785246in}}%
\pgfpathlineto{\pgfqpoint{3.414125in}{1.775901in}}%
\pgfpathlineto{\pgfqpoint{3.405921in}{1.766623in}}%
\pgfpathclose%
\pgfusepath{fill}%
\end{pgfscope}%
\begin{pgfscope}%
\pgfpathrectangle{\pgfqpoint{1.150000in}{0.150000in}}{\pgfqpoint{5.700000in}{5.700000in}}%
\pgfusepath{clip}%
\pgfsetbuttcap%
\pgfsetroundjoin%
\definecolor{currentfill}{rgb}{0.170948,0.694384,0.493803}%
\pgfsetfillcolor{currentfill}%
\pgfsetfillopacity{0.800000}%
\pgfsetlinewidth{0.000000pt}%
\definecolor{currentstroke}{rgb}{0.000000,0.000000,0.000000}%
\pgfsetstrokecolor{currentstroke}%
\pgfsetdash{}{0pt}%
\pgfpathmoveto{\pgfqpoint{6.058785in}{3.547925in}}%
\pgfpathlineto{\pgfqpoint{6.073605in}{3.559469in}}%
\pgfpathlineto{\pgfqpoint{6.088446in}{3.571188in}}%
\pgfpathlineto{\pgfqpoint{6.103309in}{3.583081in}}%
\pgfpathlineto{\pgfqpoint{6.118193in}{3.595150in}}%
\pgfpathlineto{\pgfqpoint{6.125166in}{3.596209in}}%
\pgfpathlineto{\pgfqpoint{6.132138in}{3.597413in}}%
\pgfpathlineto{\pgfqpoint{6.139109in}{3.598771in}}%
\pgfpathlineto{\pgfqpoint{6.146080in}{3.600290in}}%
\pgfpathlineto{\pgfqpoint{6.131239in}{3.588977in}}%
\pgfpathlineto{\pgfqpoint{6.116418in}{3.577837in}}%
\pgfpathlineto{\pgfqpoint{6.101619in}{3.566871in}}%
\pgfpathlineto{\pgfqpoint{6.086840in}{3.556078in}}%
\pgfpathlineto{\pgfqpoint{6.079827in}{3.553795in}}%
\pgfpathlineto{\pgfqpoint{6.072814in}{3.551681in}}%
\pgfpathlineto{\pgfqpoint{6.065800in}{3.549727in}}%
\pgfpathlineto{\pgfqpoint{6.058785in}{3.547925in}}%
\pgfpathclose%
\pgfusepath{fill}%
\end{pgfscope}%
\begin{pgfscope}%
\pgfpathrectangle{\pgfqpoint{1.150000in}{0.150000in}}{\pgfqpoint{5.700000in}{5.700000in}}%
\pgfusepath{clip}%
\pgfsetbuttcap%
\pgfsetroundjoin%
\definecolor{currentfill}{rgb}{0.229739,0.322361,0.545706}%
\pgfsetfillcolor{currentfill}%
\pgfsetfillopacity{0.800000}%
\pgfsetlinewidth{0.000000pt}%
\definecolor{currentstroke}{rgb}{0.000000,0.000000,0.000000}%
\pgfsetstrokecolor{currentstroke}%
\pgfsetdash{}{0pt}%
\pgfpathmoveto{\pgfqpoint{4.482832in}{2.417265in}}%
\pgfpathlineto{\pgfqpoint{4.496816in}{2.424547in}}%
\pgfpathlineto{\pgfqpoint{4.510814in}{2.432015in}}%
\pgfpathlineto{\pgfqpoint{4.524826in}{2.439667in}}%
\pgfpathlineto{\pgfqpoint{4.538852in}{2.447505in}}%
\pgfpathlineto{\pgfqpoint{4.546673in}{2.457157in}}%
\pgfpathlineto{\pgfqpoint{4.554488in}{2.466720in}}%
\pgfpathlineto{\pgfqpoint{4.562297in}{2.476194in}}%
\pgfpathlineto{\pgfqpoint{4.570100in}{2.485582in}}%
\pgfpathlineto{\pgfqpoint{4.556079in}{2.477821in}}%
\pgfpathlineto{\pgfqpoint{4.542073in}{2.470246in}}%
\pgfpathlineto{\pgfqpoint{4.528081in}{2.462855in}}%
\pgfpathlineto{\pgfqpoint{4.514103in}{2.455649in}}%
\pgfpathlineto{\pgfqpoint{4.506294in}{2.446172in}}%
\pgfpathlineto{\pgfqpoint{4.498479in}{2.436617in}}%
\pgfpathlineto{\pgfqpoint{4.490658in}{2.426982in}}%
\pgfpathlineto{\pgfqpoint{4.482832in}{2.417265in}}%
\pgfpathclose%
\pgfusepath{fill}%
\end{pgfscope}%
\begin{pgfscope}%
\pgfpathrectangle{\pgfqpoint{1.150000in}{0.150000in}}{\pgfqpoint{5.700000in}{5.700000in}}%
\pgfusepath{clip}%
\pgfsetbuttcap%
\pgfsetroundjoin%
\definecolor{currentfill}{rgb}{0.143343,0.522773,0.556295}%
\pgfsetfillcolor{currentfill}%
\pgfsetfillopacity{0.800000}%
\pgfsetlinewidth{0.000000pt}%
\definecolor{currentstroke}{rgb}{0.000000,0.000000,0.000000}%
\pgfsetstrokecolor{currentstroke}%
\pgfsetdash{}{0pt}%
\pgfpathmoveto{\pgfqpoint{5.243156in}{3.014144in}}%
\pgfpathlineto{\pgfqpoint{5.257549in}{3.024987in}}%
\pgfpathlineto{\pgfqpoint{5.271960in}{3.036011in}}%
\pgfpathlineto{\pgfqpoint{5.286390in}{3.047214in}}%
\pgfpathlineto{\pgfqpoint{5.300838in}{3.058598in}}%
\pgfpathlineto{\pgfqpoint{5.308292in}{3.063301in}}%
\pgfpathlineto{\pgfqpoint{5.315739in}{3.067965in}}%
\pgfpathlineto{\pgfqpoint{5.323179in}{3.072595in}}%
\pgfpathlineto{\pgfqpoint{5.330611in}{3.077196in}}%
\pgfpathlineto{\pgfqpoint{5.316182in}{3.066228in}}%
\pgfpathlineto{\pgfqpoint{5.301771in}{3.055439in}}%
\pgfpathlineto{\pgfqpoint{5.287378in}{3.044830in}}%
\pgfpathlineto{\pgfqpoint{5.273004in}{3.034400in}}%
\pgfpathlineto{\pgfqpoint{5.265552in}{3.029372in}}%
\pgfpathlineto{\pgfqpoint{5.258093in}{3.024324in}}%
\pgfpathlineto{\pgfqpoint{5.250628in}{3.019249in}}%
\pgfpathlineto{\pgfqpoint{5.243156in}{3.014144in}}%
\pgfpathclose%
\pgfusepath{fill}%
\end{pgfscope}%
\begin{pgfscope}%
\pgfpathrectangle{\pgfqpoint{1.150000in}{0.150000in}}{\pgfqpoint{5.700000in}{5.700000in}}%
\pgfusepath{clip}%
\pgfsetbuttcap%
\pgfsetroundjoin%
\definecolor{currentfill}{rgb}{0.267968,0.223549,0.512008}%
\pgfsetfillcolor{currentfill}%
\pgfsetfillopacity{0.800000}%
\pgfsetlinewidth{0.000000pt}%
\definecolor{currentstroke}{rgb}{0.000000,0.000000,0.000000}%
\pgfsetstrokecolor{currentstroke}%
\pgfsetdash{}{0pt}%
\pgfpathmoveto{\pgfqpoint{4.189789in}{2.177498in}}%
\pgfpathlineto{\pgfqpoint{4.203642in}{2.182518in}}%
\pgfpathlineto{\pgfqpoint{4.217506in}{2.187726in}}%
\pgfpathlineto{\pgfqpoint{4.231382in}{2.193122in}}%
\pgfpathlineto{\pgfqpoint{4.245270in}{2.198705in}}%
\pgfpathlineto{\pgfqpoint{4.253195in}{2.209684in}}%
\pgfpathlineto{\pgfqpoint{4.261116in}{2.220590in}}%
\pgfpathlineto{\pgfqpoint{4.269031in}{2.231423in}}%
\pgfpathlineto{\pgfqpoint{4.276941in}{2.242182in}}%
\pgfpathlineto{\pgfqpoint{4.263057in}{2.236545in}}%
\pgfpathlineto{\pgfqpoint{4.249186in}{2.231095in}}%
\pgfpathlineto{\pgfqpoint{4.235327in}{2.225833in}}%
\pgfpathlineto{\pgfqpoint{4.221479in}{2.220759in}}%
\pgfpathlineto{\pgfqpoint{4.213564in}{2.210041in}}%
\pgfpathlineto{\pgfqpoint{4.205644in}{2.199258in}}%
\pgfpathlineto{\pgfqpoint{4.197719in}{2.188411in}}%
\pgfpathlineto{\pgfqpoint{4.189789in}{2.177498in}}%
\pgfpathclose%
\pgfusepath{fill}%
\end{pgfscope}%
\begin{pgfscope}%
\pgfpathrectangle{\pgfqpoint{1.150000in}{0.150000in}}{\pgfqpoint{5.700000in}{5.700000in}}%
\pgfusepath{clip}%
\pgfsetbuttcap%
\pgfsetroundjoin%
\definecolor{currentfill}{rgb}{0.278791,0.062145,0.386592}%
\pgfsetfillcolor{currentfill}%
\pgfsetfillopacity{0.800000}%
\pgfsetlinewidth{0.000000pt}%
\definecolor{currentstroke}{rgb}{0.000000,0.000000,0.000000}%
\pgfsetstrokecolor{currentstroke}%
\pgfsetdash{}{0pt}%
\pgfpathmoveto{\pgfqpoint{2.978284in}{1.850451in}}%
\pgfpathlineto{\pgfqpoint{2.991982in}{1.840005in}}%
\pgfpathlineto{\pgfqpoint{3.005679in}{1.829792in}}%
\pgfpathlineto{\pgfqpoint{3.019374in}{1.819812in}}%
\pgfpathlineto{\pgfqpoint{3.033067in}{1.810064in}}%
\pgfpathlineto{\pgfqpoint{3.041477in}{1.816022in}}%
\pgfpathlineto{\pgfqpoint{3.049876in}{1.822131in}}%
\pgfpathlineto{\pgfqpoint{3.058266in}{1.828387in}}%
\pgfpathlineto{\pgfqpoint{3.066646in}{1.834787in}}%
\pgfpathlineto{\pgfqpoint{3.052979in}{1.844096in}}%
\pgfpathlineto{\pgfqpoint{3.039311in}{1.853637in}}%
\pgfpathlineto{\pgfqpoint{3.025642in}{1.863410in}}%
\pgfpathlineto{\pgfqpoint{3.011971in}{1.873417in}}%
\pgfpathlineto{\pgfqpoint{3.003565in}{1.867444in}}%
\pgfpathlineto{\pgfqpoint{2.995148in}{1.861623in}}%
\pgfpathlineto{\pgfqpoint{2.986721in}{1.855957in}}%
\pgfpathlineto{\pgfqpoint{2.978284in}{1.850451in}}%
\pgfpathclose%
\pgfusepath{fill}%
\end{pgfscope}%
\begin{pgfscope}%
\pgfpathrectangle{\pgfqpoint{1.150000in}{0.150000in}}{\pgfqpoint{5.700000in}{5.700000in}}%
\pgfusepath{clip}%
\pgfsetbuttcap%
\pgfsetroundjoin%
\definecolor{currentfill}{rgb}{0.180629,0.429975,0.557282}%
\pgfsetfillcolor{currentfill}%
\pgfsetfillopacity{0.800000}%
\pgfsetlinewidth{0.000000pt}%
\definecolor{currentstroke}{rgb}{0.000000,0.000000,0.000000}%
\pgfsetstrokecolor{currentstroke}%
\pgfsetdash{}{0pt}%
\pgfpathmoveto{\pgfqpoint{4.863124in}{2.725189in}}%
\pgfpathlineto{\pgfqpoint{4.877311in}{2.734696in}}%
\pgfpathlineto{\pgfqpoint{4.891513in}{2.744385in}}%
\pgfpathlineto{\pgfqpoint{4.905732in}{2.754256in}}%
\pgfpathlineto{\pgfqpoint{4.919968in}{2.764310in}}%
\pgfpathlineto{\pgfqpoint{4.927625in}{2.771583in}}%
\pgfpathlineto{\pgfqpoint{4.935275in}{2.778774in}}%
\pgfpathlineto{\pgfqpoint{4.942919in}{2.785887in}}%
\pgfpathlineto{\pgfqpoint{4.950555in}{2.792924in}}%
\pgfpathlineto{\pgfqpoint{4.936330in}{2.783115in}}%
\pgfpathlineto{\pgfqpoint{4.922122in}{2.773489in}}%
\pgfpathlineto{\pgfqpoint{4.907930in}{2.764044in}}%
\pgfpathlineto{\pgfqpoint{4.893755in}{2.754781in}}%
\pgfpathlineto{\pgfqpoint{4.886107in}{2.747487in}}%
\pgfpathlineto{\pgfqpoint{4.878453in}{2.740126in}}%
\pgfpathlineto{\pgfqpoint{4.870792in}{2.732694in}}%
\pgfpathlineto{\pgfqpoint{4.863124in}{2.725189in}}%
\pgfpathclose%
\pgfusepath{fill}%
\end{pgfscope}%
\begin{pgfscope}%
\pgfpathrectangle{\pgfqpoint{1.150000in}{0.150000in}}{\pgfqpoint{5.700000in}{5.700000in}}%
\pgfusepath{clip}%
\pgfsetbuttcap%
\pgfsetroundjoin%
\definecolor{currentfill}{rgb}{0.135066,0.544853,0.554029}%
\pgfsetfillcolor{currentfill}%
\pgfsetfillopacity{0.800000}%
\pgfsetlinewidth{0.000000pt}%
\definecolor{currentstroke}{rgb}{0.000000,0.000000,0.000000}%
\pgfsetstrokecolor{currentstroke}%
\pgfsetdash{}{0pt}%
\pgfpathmoveto{\pgfqpoint{5.330611in}{3.077196in}}%
\pgfpathlineto{\pgfqpoint{5.345059in}{3.088344in}}%
\pgfpathlineto{\pgfqpoint{5.359526in}{3.099671in}}%
\pgfpathlineto{\pgfqpoint{5.374011in}{3.111177in}}%
\pgfpathlineto{\pgfqpoint{5.388515in}{3.122864in}}%
\pgfpathlineto{\pgfqpoint{5.395921in}{3.127003in}}%
\pgfpathlineto{\pgfqpoint{5.403319in}{3.131115in}}%
\pgfpathlineto{\pgfqpoint{5.410710in}{3.135204in}}%
\pgfpathlineto{\pgfqpoint{5.418095in}{3.139276in}}%
\pgfpathlineto{\pgfqpoint{5.403612in}{3.128040in}}%
\pgfpathlineto{\pgfqpoint{5.389147in}{3.116983in}}%
\pgfpathlineto{\pgfqpoint{5.374701in}{3.106105in}}%
\pgfpathlineto{\pgfqpoint{5.360274in}{3.095406in}}%
\pgfpathlineto{\pgfqpoint{5.352868in}{3.090872in}}%
\pgfpathlineto{\pgfqpoint{5.345456in}{3.086330in}}%
\pgfpathlineto{\pgfqpoint{5.338037in}{3.081773in}}%
\pgfpathlineto{\pgfqpoint{5.330611in}{3.077196in}}%
\pgfpathclose%
\pgfusepath{fill}%
\end{pgfscope}%
\begin{pgfscope}%
\pgfpathrectangle{\pgfqpoint{1.150000in}{0.150000in}}{\pgfqpoint{5.700000in}{5.700000in}}%
\pgfusepath{clip}%
\pgfsetbuttcap%
\pgfsetroundjoin%
\definecolor{currentfill}{rgb}{0.271305,0.019942,0.347269}%
\pgfsetfillcolor{currentfill}%
\pgfsetfillopacity{0.800000}%
\pgfsetlinewidth{0.000000pt}%
\definecolor{currentstroke}{rgb}{0.000000,0.000000,0.000000}%
\pgfsetstrokecolor{currentstroke}%
\pgfsetdash{}{0pt}%
\pgfpathmoveto{\pgfqpoint{3.175961in}{1.768468in}}%
\pgfpathlineto{\pgfqpoint{3.189627in}{1.761180in}}%
\pgfpathlineto{\pgfqpoint{3.203293in}{1.754110in}}%
\pgfpathlineto{\pgfqpoint{3.216960in}{1.747258in}}%
\pgfpathlineto{\pgfqpoint{3.230628in}{1.740623in}}%
\pgfpathlineto{\pgfqpoint{3.238928in}{1.748412in}}%
\pgfpathlineto{\pgfqpoint{3.247220in}{1.756312in}}%
\pgfpathlineto{\pgfqpoint{3.255504in}{1.764320in}}%
\pgfpathlineto{\pgfqpoint{3.263780in}{1.772430in}}%
\pgfpathlineto{\pgfqpoint{3.250133in}{1.778662in}}%
\pgfpathlineto{\pgfqpoint{3.236487in}{1.785110in}}%
\pgfpathlineto{\pgfqpoint{3.222842in}{1.791776in}}%
\pgfpathlineto{\pgfqpoint{3.209199in}{1.798660in}}%
\pgfpathlineto{\pgfqpoint{3.200902in}{1.790941in}}%
\pgfpathlineto{\pgfqpoint{3.192597in}{1.783334in}}%
\pgfpathlineto{\pgfqpoint{3.184284in}{1.775841in}}%
\pgfpathlineto{\pgfqpoint{3.175961in}{1.768468in}}%
\pgfpathclose%
\pgfusepath{fill}%
\end{pgfscope}%
\begin{pgfscope}%
\pgfpathrectangle{\pgfqpoint{1.150000in}{0.150000in}}{\pgfqpoint{5.700000in}{5.700000in}}%
\pgfusepath{clip}%
\pgfsetbuttcap%
\pgfsetroundjoin%
\definecolor{currentfill}{rgb}{0.201239,0.383670,0.554294}%
\pgfsetfillcolor{currentfill}%
\pgfsetfillopacity{0.800000}%
\pgfsetlinewidth{0.000000pt}%
\definecolor{currentstroke}{rgb}{0.000000,0.000000,0.000000}%
\pgfsetstrokecolor{currentstroke}%
\pgfsetdash{}{0pt}%
\pgfpathmoveto{\pgfqpoint{2.313699in}{2.662629in}}%
\pgfpathlineto{\pgfqpoint{2.327789in}{2.638641in}}%
\pgfpathlineto{\pgfqpoint{2.341865in}{2.614998in}}%
\pgfpathlineto{\pgfqpoint{2.355925in}{2.591695in}}%
\pgfpathlineto{\pgfqpoint{2.369971in}{2.568729in}}%
\pgfpathlineto{\pgfqpoint{2.378824in}{2.569112in}}%
\pgfpathlineto{\pgfqpoint{2.387660in}{2.569748in}}%
\pgfpathlineto{\pgfqpoint{2.396478in}{2.570634in}}%
\pgfpathlineto{\pgfqpoint{2.405281in}{2.571764in}}%
\pgfpathlineto{\pgfqpoint{2.391282in}{2.594258in}}%
\pgfpathlineto{\pgfqpoint{2.377269in}{2.617088in}}%
\pgfpathlineto{\pgfqpoint{2.363241in}{2.640257in}}%
\pgfpathlineto{\pgfqpoint{2.349199in}{2.663768in}}%
\pgfpathlineto{\pgfqpoint{2.340350in}{2.663098in}}%
\pgfpathlineto{\pgfqpoint{2.331484in}{2.662682in}}%
\pgfpathlineto{\pgfqpoint{2.322601in}{2.662524in}}%
\pgfpathlineto{\pgfqpoint{2.313699in}{2.662629in}}%
\pgfpathclose%
\pgfusepath{fill}%
\end{pgfscope}%
\begin{pgfscope}%
\pgfpathrectangle{\pgfqpoint{1.150000in}{0.150000in}}{\pgfqpoint{5.700000in}{5.700000in}}%
\pgfusepath{clip}%
\pgfsetbuttcap%
\pgfsetroundjoin%
\definecolor{currentfill}{rgb}{0.218130,0.347432,0.550038}%
\pgfsetfillcolor{currentfill}%
\pgfsetfillopacity{0.800000}%
\pgfsetlinewidth{0.000000pt}%
\definecolor{currentstroke}{rgb}{0.000000,0.000000,0.000000}%
\pgfsetstrokecolor{currentstroke}%
\pgfsetdash{}{0pt}%
\pgfpathmoveto{\pgfqpoint{4.570100in}{2.485582in}}%
\pgfpathlineto{\pgfqpoint{4.584134in}{2.493527in}}%
\pgfpathlineto{\pgfqpoint{4.598183in}{2.501657in}}%
\pgfpathlineto{\pgfqpoint{4.612247in}{2.509972in}}%
\pgfpathlineto{\pgfqpoint{4.626325in}{2.518471in}}%
\pgfpathlineto{\pgfqpoint{4.634115in}{2.527675in}}%
\pgfpathlineto{\pgfqpoint{4.641899in}{2.536787in}}%
\pgfpathlineto{\pgfqpoint{4.649677in}{2.545809in}}%
\pgfpathlineto{\pgfqpoint{4.657448in}{2.554742in}}%
\pgfpathlineto{\pgfqpoint{4.643376in}{2.546354in}}%
\pgfpathlineto{\pgfqpoint{4.629319in}{2.538150in}}%
\pgfpathlineto{\pgfqpoint{4.615277in}{2.530130in}}%
\pgfpathlineto{\pgfqpoint{4.601249in}{2.522294in}}%
\pgfpathlineto{\pgfqpoint{4.593471in}{2.513239in}}%
\pgfpathlineto{\pgfqpoint{4.585687in}{2.504103in}}%
\pgfpathlineto{\pgfqpoint{4.577896in}{2.494884in}}%
\pgfpathlineto{\pgfqpoint{4.570100in}{2.485582in}}%
\pgfpathclose%
\pgfusepath{fill}%
\end{pgfscope}%
\begin{pgfscope}%
\pgfpathrectangle{\pgfqpoint{1.150000in}{0.150000in}}{\pgfqpoint{5.700000in}{5.700000in}}%
\pgfusepath{clip}%
\pgfsetbuttcap%
\pgfsetroundjoin%
\definecolor{currentfill}{rgb}{0.280894,0.078907,0.402329}%
\pgfsetfillcolor{currentfill}%
\pgfsetfillopacity{0.800000}%
\pgfsetlinewidth{0.000000pt}%
\definecolor{currentstroke}{rgb}{0.000000,0.000000,0.000000}%
\pgfsetstrokecolor{currentstroke}%
\pgfsetdash{}{0pt}%
\pgfpathmoveto{\pgfqpoint{3.722335in}{1.857577in}}%
\pgfpathlineto{\pgfqpoint{3.736038in}{1.857765in}}%
\pgfpathlineto{\pgfqpoint{3.749749in}{1.858148in}}%
\pgfpathlineto{\pgfqpoint{3.763466in}{1.858726in}}%
\pgfpathlineto{\pgfqpoint{3.777191in}{1.859500in}}%
\pgfpathlineto{\pgfqpoint{3.785266in}{1.870676in}}%
\pgfpathlineto{\pgfqpoint{3.793335in}{1.881847in}}%
\pgfpathlineto{\pgfqpoint{3.801400in}{1.893012in}}%
\pgfpathlineto{\pgfqpoint{3.809459in}{1.904168in}}%
\pgfpathlineto{\pgfqpoint{3.795742in}{1.903149in}}%
\pgfpathlineto{\pgfqpoint{3.782033in}{1.902326in}}%
\pgfpathlineto{\pgfqpoint{3.768331in}{1.901697in}}%
\pgfpathlineto{\pgfqpoint{3.754637in}{1.901265in}}%
\pgfpathlineto{\pgfqpoint{3.746570in}{1.890342in}}%
\pgfpathlineto{\pgfqpoint{3.738497in}{1.879419in}}%
\pgfpathlineto{\pgfqpoint{3.730419in}{1.868496in}}%
\pgfpathlineto{\pgfqpoint{3.722335in}{1.857577in}}%
\pgfpathclose%
\pgfusepath{fill}%
\end{pgfscope}%
\begin{pgfscope}%
\pgfpathrectangle{\pgfqpoint{1.150000in}{0.150000in}}{\pgfqpoint{5.700000in}{5.700000in}}%
\pgfusepath{clip}%
\pgfsetbuttcap%
\pgfsetroundjoin%
\definecolor{currentfill}{rgb}{0.271305,0.019942,0.347269}%
\pgfsetfillcolor{currentfill}%
\pgfsetfillopacity{0.800000}%
\pgfsetlinewidth{0.000000pt}%
\definecolor{currentstroke}{rgb}{0.000000,0.000000,0.000000}%
\pgfsetstrokecolor{currentstroke}%
\pgfsetdash{}{0pt}%
\pgfpathmoveto{\pgfqpoint{3.318389in}{1.749647in}}%
\pgfpathlineto{\pgfqpoint{3.332048in}{1.744482in}}%
\pgfpathlineto{\pgfqpoint{3.345708in}{1.739528in}}%
\pgfpathlineto{\pgfqpoint{3.359372in}{1.734782in}}%
\pgfpathlineto{\pgfqpoint{3.373038in}{1.730246in}}%
\pgfpathlineto{\pgfqpoint{3.381270in}{1.739223in}}%
\pgfpathlineto{\pgfqpoint{3.389494in}{1.748281in}}%
\pgfpathlineto{\pgfqpoint{3.397711in}{1.757415in}}%
\pgfpathlineto{\pgfqpoint{3.405921in}{1.766623in}}%
\pgfpathlineto{\pgfqpoint{3.392271in}{1.770788in}}%
\pgfpathlineto{\pgfqpoint{3.378625in}{1.775162in}}%
\pgfpathlineto{\pgfqpoint{3.364982in}{1.779745in}}%
\pgfpathlineto{\pgfqpoint{3.351342in}{1.784539in}}%
\pgfpathlineto{\pgfqpoint{3.343114in}{1.775690in}}%
\pgfpathlineto{\pgfqpoint{3.334880in}{1.766923in}}%
\pgfpathlineto{\pgfqpoint{3.326638in}{1.758241in}}%
\pgfpathlineto{\pgfqpoint{3.318389in}{1.749647in}}%
\pgfpathclose%
\pgfusepath{fill}%
\end{pgfscope}%
\begin{pgfscope}%
\pgfpathrectangle{\pgfqpoint{1.150000in}{0.150000in}}{\pgfqpoint{5.700000in}{5.700000in}}%
\pgfusepath{clip}%
\pgfsetbuttcap%
\pgfsetroundjoin%
\definecolor{currentfill}{rgb}{0.282910,0.105393,0.426902}%
\pgfsetfillcolor{currentfill}%
\pgfsetfillopacity{0.800000}%
\pgfsetlinewidth{0.000000pt}%
\definecolor{currentstroke}{rgb}{0.000000,0.000000,0.000000}%
\pgfsetstrokecolor{currentstroke}%
\pgfsetdash{}{0pt}%
\pgfpathmoveto{\pgfqpoint{3.809459in}{1.904168in}}%
\pgfpathlineto{\pgfqpoint{3.823184in}{1.905381in}}%
\pgfpathlineto{\pgfqpoint{3.836917in}{1.906788in}}%
\pgfpathlineto{\pgfqpoint{3.850659in}{1.908389in}}%
\pgfpathlineto{\pgfqpoint{3.864409in}{1.910182in}}%
\pgfpathlineto{\pgfqpoint{3.872455in}{1.921553in}}%
\pgfpathlineto{\pgfqpoint{3.880497in}{1.932903in}}%
\pgfpathlineto{\pgfqpoint{3.888534in}{1.944231in}}%
\pgfpathlineto{\pgfqpoint{3.896566in}{1.955535in}}%
\pgfpathlineto{\pgfqpoint{3.882823in}{1.953528in}}%
\pgfpathlineto{\pgfqpoint{3.869089in}{1.951714in}}%
\pgfpathlineto{\pgfqpoint{3.855363in}{1.950093in}}%
\pgfpathlineto{\pgfqpoint{3.841646in}{1.948666in}}%
\pgfpathlineto{\pgfqpoint{3.833606in}{1.937564in}}%
\pgfpathlineto{\pgfqpoint{3.825562in}{1.926446in}}%
\pgfpathlineto{\pgfqpoint{3.817513in}{1.915313in}}%
\pgfpathlineto{\pgfqpoint{3.809459in}{1.904168in}}%
\pgfpathclose%
\pgfusepath{fill}%
\end{pgfscope}%
\begin{pgfscope}%
\pgfpathrectangle{\pgfqpoint{1.150000in}{0.150000in}}{\pgfqpoint{5.700000in}{5.700000in}}%
\pgfusepath{clip}%
\pgfsetbuttcap%
\pgfsetroundjoin%
\definecolor{currentfill}{rgb}{0.278791,0.062145,0.386592}%
\pgfsetfillcolor{currentfill}%
\pgfsetfillopacity{0.800000}%
\pgfsetlinewidth{0.000000pt}%
\definecolor{currentstroke}{rgb}{0.000000,0.000000,0.000000}%
\pgfsetstrokecolor{currentstroke}%
\pgfsetdash{}{0pt}%
\pgfpathmoveto{\pgfqpoint{3.635166in}{1.816307in}}%
\pgfpathlineto{\pgfqpoint{3.648852in}{1.815428in}}%
\pgfpathlineto{\pgfqpoint{3.662545in}{1.814748in}}%
\pgfpathlineto{\pgfqpoint{3.676244in}{1.814265in}}%
\pgfpathlineto{\pgfqpoint{3.689949in}{1.813980in}}%
\pgfpathlineto{\pgfqpoint{3.698054in}{1.824862in}}%
\pgfpathlineto{\pgfqpoint{3.706153in}{1.835758in}}%
\pgfpathlineto{\pgfqpoint{3.714247in}{1.846664in}}%
\pgfpathlineto{\pgfqpoint{3.722335in}{1.857577in}}%
\pgfpathlineto{\pgfqpoint{3.708639in}{1.857586in}}%
\pgfpathlineto{\pgfqpoint{3.694950in}{1.857792in}}%
\pgfpathlineto{\pgfqpoint{3.681268in}{1.858196in}}%
\pgfpathlineto{\pgfqpoint{3.667593in}{1.858798in}}%
\pgfpathlineto{\pgfqpoint{3.659494in}{1.848149in}}%
\pgfpathlineto{\pgfqpoint{3.651390in}{1.837516in}}%
\pgfpathlineto{\pgfqpoint{3.643281in}{1.826901in}}%
\pgfpathlineto{\pgfqpoint{3.635166in}{1.816307in}}%
\pgfpathclose%
\pgfusepath{fill}%
\end{pgfscope}%
\begin{pgfscope}%
\pgfpathrectangle{\pgfqpoint{1.150000in}{0.150000in}}{\pgfqpoint{5.700000in}{5.700000in}}%
\pgfusepath{clip}%
\pgfsetbuttcap%
\pgfsetroundjoin%
\definecolor{currentfill}{rgb}{0.257322,0.256130,0.526563}%
\pgfsetfillcolor{currentfill}%
\pgfsetfillopacity{0.800000}%
\pgfsetlinewidth{0.000000pt}%
\definecolor{currentstroke}{rgb}{0.000000,0.000000,0.000000}%
\pgfsetstrokecolor{currentstroke}%
\pgfsetdash{}{0pt}%
\pgfpathmoveto{\pgfqpoint{4.276941in}{2.242182in}}%
\pgfpathlineto{\pgfqpoint{4.290836in}{2.248007in}}%
\pgfpathlineto{\pgfqpoint{4.304744in}{2.254019in}}%
\pgfpathlineto{\pgfqpoint{4.318664in}{2.260218in}}%
\pgfpathlineto{\pgfqpoint{4.332597in}{2.266604in}}%
\pgfpathlineto{\pgfqpoint{4.340497in}{2.277325in}}%
\pgfpathlineto{\pgfqpoint{4.348392in}{2.287964in}}%
\pgfpathlineto{\pgfqpoint{4.356281in}{2.298523in}}%
\pgfpathlineto{\pgfqpoint{4.364164in}{2.309002in}}%
\pgfpathlineto{\pgfqpoint{4.350236in}{2.302595in}}%
\pgfpathlineto{\pgfqpoint{4.336320in}{2.296374in}}%
\pgfpathlineto{\pgfqpoint{4.322417in}{2.290340in}}%
\pgfpathlineto{\pgfqpoint{4.308526in}{2.284493in}}%
\pgfpathlineto{\pgfqpoint{4.300638in}{2.274025in}}%
\pgfpathlineto{\pgfqpoint{4.292744in}{2.263483in}}%
\pgfpathlineto{\pgfqpoint{4.284845in}{2.252869in}}%
\pgfpathlineto{\pgfqpoint{4.276941in}{2.242182in}}%
\pgfpathclose%
\pgfusepath{fill}%
\end{pgfscope}%
\begin{pgfscope}%
\pgfpathrectangle{\pgfqpoint{1.150000in}{0.150000in}}{\pgfqpoint{5.700000in}{5.700000in}}%
\pgfusepath{clip}%
\pgfsetbuttcap%
\pgfsetroundjoin%
\definecolor{currentfill}{rgb}{0.127568,0.566949,0.550556}%
\pgfsetfillcolor{currentfill}%
\pgfsetfillopacity{0.800000}%
\pgfsetlinewidth{0.000000pt}%
\definecolor{currentstroke}{rgb}{0.000000,0.000000,0.000000}%
\pgfsetstrokecolor{currentstroke}%
\pgfsetdash{}{0pt}%
\pgfpathmoveto{\pgfqpoint{5.418095in}{3.139276in}}%
\pgfpathlineto{\pgfqpoint{5.432597in}{3.150691in}}%
\pgfpathlineto{\pgfqpoint{5.447119in}{3.162285in}}%
\pgfpathlineto{\pgfqpoint{5.461660in}{3.174059in}}%
\pgfpathlineto{\pgfqpoint{5.476220in}{3.186011in}}%
\pgfpathlineto{\pgfqpoint{5.483575in}{3.189599in}}%
\pgfpathlineto{\pgfqpoint{5.490923in}{3.193172in}}%
\pgfpathlineto{\pgfqpoint{5.498265in}{3.196736in}}%
\pgfpathlineto{\pgfqpoint{5.505600in}{3.200296in}}%
\pgfpathlineto{\pgfqpoint{5.491063in}{3.188829in}}%
\pgfpathlineto{\pgfqpoint{5.476545in}{3.177540in}}%
\pgfpathlineto{\pgfqpoint{5.462047in}{3.166429in}}%
\pgfpathlineto{\pgfqpoint{5.447567in}{3.155497in}}%
\pgfpathlineto{\pgfqpoint{5.440209in}{3.151441in}}%
\pgfpathlineto{\pgfqpoint{5.432844in}{3.147390in}}%
\pgfpathlineto{\pgfqpoint{5.425473in}{3.143336in}}%
\pgfpathlineto{\pgfqpoint{5.418095in}{3.139276in}}%
\pgfpathclose%
\pgfusepath{fill}%
\end{pgfscope}%
\begin{pgfscope}%
\pgfpathrectangle{\pgfqpoint{1.150000in}{0.150000in}}{\pgfqpoint{5.700000in}{5.700000in}}%
\pgfusepath{clip}%
\pgfsetbuttcap%
\pgfsetroundjoin%
\definecolor{currentfill}{rgb}{0.169646,0.456262,0.558030}%
\pgfsetfillcolor{currentfill}%
\pgfsetfillopacity{0.800000}%
\pgfsetlinewidth{0.000000pt}%
\definecolor{currentstroke}{rgb}{0.000000,0.000000,0.000000}%
\pgfsetstrokecolor{currentstroke}%
\pgfsetdash{}{0pt}%
\pgfpathmoveto{\pgfqpoint{4.950555in}{2.792924in}}%
\pgfpathlineto{\pgfqpoint{4.964796in}{2.802915in}}%
\pgfpathlineto{\pgfqpoint{4.979054in}{2.813087in}}%
\pgfpathlineto{\pgfqpoint{4.993329in}{2.823442in}}%
\pgfpathlineto{\pgfqpoint{5.007622in}{2.833979in}}%
\pgfpathlineto{\pgfqpoint{5.015239in}{2.840677in}}%
\pgfpathlineto{\pgfqpoint{5.022849in}{2.847298in}}%
\pgfpathlineto{\pgfqpoint{5.030452in}{2.853845in}}%
\pgfpathlineto{\pgfqpoint{5.038047in}{2.860322in}}%
\pgfpathlineto{\pgfqpoint{5.023767in}{2.850065in}}%
\pgfpathlineto{\pgfqpoint{5.009505in}{2.839989in}}%
\pgfpathlineto{\pgfqpoint{4.995259in}{2.830095in}}%
\pgfpathlineto{\pgfqpoint{4.981029in}{2.820382in}}%
\pgfpathlineto{\pgfqpoint{4.973421in}{2.813615in}}%
\pgfpathlineto{\pgfqpoint{4.965806in}{2.806785in}}%
\pgfpathlineto{\pgfqpoint{4.958184in}{2.799889in}}%
\pgfpathlineto{\pgfqpoint{4.950555in}{2.792924in}}%
\pgfpathclose%
\pgfusepath{fill}%
\end{pgfscope}%
\begin{pgfscope}%
\pgfpathrectangle{\pgfqpoint{1.150000in}{0.150000in}}{\pgfqpoint{5.700000in}{5.700000in}}%
\pgfusepath{clip}%
\pgfsetbuttcap%
\pgfsetroundjoin%
\definecolor{currentfill}{rgb}{0.283072,0.130895,0.449241}%
\pgfsetfillcolor{currentfill}%
\pgfsetfillopacity{0.800000}%
\pgfsetlinewidth{0.000000pt}%
\definecolor{currentstroke}{rgb}{0.000000,0.000000,0.000000}%
\pgfsetstrokecolor{currentstroke}%
\pgfsetdash{}{0pt}%
\pgfpathmoveto{\pgfqpoint{3.896566in}{1.955535in}}%
\pgfpathlineto{\pgfqpoint{3.910318in}{1.957735in}}%
\pgfpathlineto{\pgfqpoint{3.924078in}{1.960127in}}%
\pgfpathlineto{\pgfqpoint{3.937848in}{1.962711in}}%
\pgfpathlineto{\pgfqpoint{3.951628in}{1.965486in}}%
\pgfpathlineto{\pgfqpoint{3.959648in}{1.976958in}}%
\pgfpathlineto{\pgfqpoint{3.967664in}{1.988395in}}%
\pgfpathlineto{\pgfqpoint{3.975675in}{1.999796in}}%
\pgfpathlineto{\pgfqpoint{3.983681in}{2.011158in}}%
\pgfpathlineto{\pgfqpoint{3.969907in}{2.008201in}}%
\pgfpathlineto{\pgfqpoint{3.956144in}{2.005435in}}%
\pgfpathlineto{\pgfqpoint{3.942389in}{2.002861in}}%
\pgfpathlineto{\pgfqpoint{3.928644in}{2.000479in}}%
\pgfpathlineto{\pgfqpoint{3.920632in}{1.989287in}}%
\pgfpathlineto{\pgfqpoint{3.912615in}{1.978064in}}%
\pgfpathlineto{\pgfqpoint{3.904593in}{1.966813in}}%
\pgfpathlineto{\pgfqpoint{3.896566in}{1.955535in}}%
\pgfpathclose%
\pgfusepath{fill}%
\end{pgfscope}%
\begin{pgfscope}%
\pgfpathrectangle{\pgfqpoint{1.150000in}{0.150000in}}{\pgfqpoint{5.700000in}{5.700000in}}%
\pgfusepath{clip}%
\pgfsetbuttcap%
\pgfsetroundjoin%
\definecolor{currentfill}{rgb}{0.274128,0.199721,0.498911}%
\pgfsetfillcolor{currentfill}%
\pgfsetfillopacity{0.800000}%
\pgfsetlinewidth{0.000000pt}%
\definecolor{currentstroke}{rgb}{0.000000,0.000000,0.000000}%
\pgfsetstrokecolor{currentstroke}%
\pgfsetdash{}{0pt}%
\pgfpathmoveto{\pgfqpoint{2.613640in}{2.166707in}}%
\pgfpathlineto{\pgfqpoint{2.627495in}{2.149657in}}%
\pgfpathlineto{\pgfqpoint{2.641342in}{2.132886in}}%
\pgfpathlineto{\pgfqpoint{2.655180in}{2.116390in}}%
\pgfpathlineto{\pgfqpoint{2.669012in}{2.100168in}}%
\pgfpathlineto{\pgfqpoint{2.677668in}{2.102538in}}%
\pgfpathlineto{\pgfqpoint{2.686310in}{2.105128in}}%
\pgfpathlineto{\pgfqpoint{2.694939in}{2.107932in}}%
\pgfpathlineto{\pgfqpoint{2.703553in}{2.110947in}}%
\pgfpathlineto{\pgfqpoint{2.689760in}{2.126685in}}%
\pgfpathlineto{\pgfqpoint{2.675961in}{2.142695in}}%
\pgfpathlineto{\pgfqpoint{2.662153in}{2.158981in}}%
\pgfpathlineto{\pgfqpoint{2.648338in}{2.175543in}}%
\pgfpathlineto{\pgfqpoint{2.639686in}{2.173001in}}%
\pgfpathlineto{\pgfqpoint{2.631019in}{2.170678in}}%
\pgfpathlineto{\pgfqpoint{2.622337in}{2.168578in}}%
\pgfpathlineto{\pgfqpoint{2.613640in}{2.166707in}}%
\pgfpathclose%
\pgfusepath{fill}%
\end{pgfscope}%
\begin{pgfscope}%
\pgfpathrectangle{\pgfqpoint{1.150000in}{0.150000in}}{\pgfqpoint{5.700000in}{5.700000in}}%
\pgfusepath{clip}%
\pgfsetbuttcap%
\pgfsetroundjoin%
\definecolor{currentfill}{rgb}{0.279574,0.170599,0.479997}%
\pgfsetfillcolor{currentfill}%
\pgfsetfillopacity{0.800000}%
\pgfsetlinewidth{0.000000pt}%
\definecolor{currentstroke}{rgb}{0.000000,0.000000,0.000000}%
\pgfsetstrokecolor{currentstroke}%
\pgfsetdash{}{0pt}%
\pgfpathmoveto{\pgfqpoint{2.669012in}{2.100168in}}%
\pgfpathlineto{\pgfqpoint{2.682835in}{2.084216in}}%
\pgfpathlineto{\pgfqpoint{2.696652in}{2.068534in}}%
\pgfpathlineto{\pgfqpoint{2.710462in}{2.053119in}}%
\pgfpathlineto{\pgfqpoint{2.724265in}{2.037970in}}%
\pgfpathlineto{\pgfqpoint{2.732883in}{2.040836in}}%
\pgfpathlineto{\pgfqpoint{2.741488in}{2.043913in}}%
\pgfpathlineto{\pgfqpoint{2.750079in}{2.047196in}}%
\pgfpathlineto{\pgfqpoint{2.758657in}{2.050681in}}%
\pgfpathlineto{\pgfqpoint{2.744891in}{2.065349in}}%
\pgfpathlineto{\pgfqpoint{2.731118in}{2.080281in}}%
\pgfpathlineto{\pgfqpoint{2.717339in}{2.095480in}}%
\pgfpathlineto{\pgfqpoint{2.703553in}{2.110947in}}%
\pgfpathlineto{\pgfqpoint{2.694939in}{2.107932in}}%
\pgfpathlineto{\pgfqpoint{2.686310in}{2.105128in}}%
\pgfpathlineto{\pgfqpoint{2.677668in}{2.102538in}}%
\pgfpathlineto{\pgfqpoint{2.669012in}{2.100168in}}%
\pgfpathclose%
\pgfusepath{fill}%
\end{pgfscope}%
\begin{pgfscope}%
\pgfpathrectangle{\pgfqpoint{1.150000in}{0.150000in}}{\pgfqpoint{5.700000in}{5.700000in}}%
\pgfusepath{clip}%
\pgfsetbuttcap%
\pgfsetroundjoin%
\definecolor{currentfill}{rgb}{0.276022,0.044167,0.370164}%
\pgfsetfillcolor{currentfill}%
\pgfsetfillopacity{0.800000}%
\pgfsetlinewidth{0.000000pt}%
\definecolor{currentstroke}{rgb}{0.000000,0.000000,0.000000}%
\pgfsetstrokecolor{currentstroke}%
\pgfsetdash{}{0pt}%
\pgfpathmoveto{\pgfqpoint{3.547919in}{1.780926in}}%
\pgfpathlineto{\pgfqpoint{3.561594in}{1.778941in}}%
\pgfpathlineto{\pgfqpoint{3.575274in}{1.777156in}}%
\pgfpathlineto{\pgfqpoint{3.588959in}{1.775571in}}%
\pgfpathlineto{\pgfqpoint{3.602651in}{1.774186in}}%
\pgfpathlineto{\pgfqpoint{3.610788in}{1.784672in}}%
\pgfpathlineto{\pgfqpoint{3.618920in}{1.795189in}}%
\pgfpathlineto{\pgfqpoint{3.627046in}{1.805735in}}%
\pgfpathlineto{\pgfqpoint{3.635166in}{1.816307in}}%
\pgfpathlineto{\pgfqpoint{3.621486in}{1.817384in}}%
\pgfpathlineto{\pgfqpoint{3.607812in}{1.818661in}}%
\pgfpathlineto{\pgfqpoint{3.594144in}{1.820137in}}%
\pgfpathlineto{\pgfqpoint{3.580482in}{1.821815in}}%
\pgfpathlineto{\pgfqpoint{3.572350in}{1.811540in}}%
\pgfpathlineto{\pgfqpoint{3.564212in}{1.801298in}}%
\pgfpathlineto{\pgfqpoint{3.556069in}{1.791092in}}%
\pgfpathlineto{\pgfqpoint{3.547919in}{1.780926in}}%
\pgfpathclose%
\pgfusepath{fill}%
\end{pgfscope}%
\begin{pgfscope}%
\pgfpathrectangle{\pgfqpoint{1.150000in}{0.150000in}}{\pgfqpoint{5.700000in}{5.700000in}}%
\pgfusepath{clip}%
\pgfsetbuttcap%
\pgfsetroundjoin%
\definecolor{currentfill}{rgb}{0.276022,0.044167,0.370164}%
\pgfsetfillcolor{currentfill}%
\pgfsetfillopacity{0.800000}%
\pgfsetlinewidth{0.000000pt}%
\definecolor{currentstroke}{rgb}{0.000000,0.000000,0.000000}%
\pgfsetstrokecolor{currentstroke}%
\pgfsetdash{}{0pt}%
\pgfpathmoveto{\pgfqpoint{3.033067in}{1.810064in}}%
\pgfpathlineto{\pgfqpoint{3.046760in}{1.800545in}}%
\pgfpathlineto{\pgfqpoint{3.060450in}{1.791255in}}%
\pgfpathlineto{\pgfqpoint{3.074141in}{1.782192in}}%
\pgfpathlineto{\pgfqpoint{3.087830in}{1.773355in}}%
\pgfpathlineto{\pgfqpoint{3.096213in}{1.779764in}}%
\pgfpathlineto{\pgfqpoint{3.104586in}{1.786315in}}%
\pgfpathlineto{\pgfqpoint{3.112950in}{1.793006in}}%
\pgfpathlineto{\pgfqpoint{3.121305in}{1.799831in}}%
\pgfpathlineto{\pgfqpoint{3.107641in}{1.808230in}}%
\pgfpathlineto{\pgfqpoint{3.093976in}{1.816855in}}%
\pgfpathlineto{\pgfqpoint{3.080312in}{1.825707in}}%
\pgfpathlineto{\pgfqpoint{3.066646in}{1.834787in}}%
\pgfpathlineto{\pgfqpoint{3.058266in}{1.828387in}}%
\pgfpathlineto{\pgfqpoint{3.049876in}{1.822131in}}%
\pgfpathlineto{\pgfqpoint{3.041477in}{1.816022in}}%
\pgfpathlineto{\pgfqpoint{3.033067in}{1.810064in}}%
\pgfpathclose%
\pgfusepath{fill}%
\end{pgfscope}%
\begin{pgfscope}%
\pgfpathrectangle{\pgfqpoint{1.150000in}{0.150000in}}{\pgfqpoint{5.700000in}{5.700000in}}%
\pgfusepath{clip}%
\pgfsetbuttcap%
\pgfsetroundjoin%
\definecolor{currentfill}{rgb}{0.266580,0.228262,0.514349}%
\pgfsetfillcolor{currentfill}%
\pgfsetfillopacity{0.800000}%
\pgfsetlinewidth{0.000000pt}%
\definecolor{currentstroke}{rgb}{0.000000,0.000000,0.000000}%
\pgfsetstrokecolor{currentstroke}%
\pgfsetdash{}{0pt}%
\pgfpathmoveto{\pgfqpoint{2.558134in}{2.237727in}}%
\pgfpathlineto{\pgfqpoint{2.572024in}{2.219544in}}%
\pgfpathlineto{\pgfqpoint{2.585905in}{2.201647in}}%
\pgfpathlineto{\pgfqpoint{2.599777in}{2.184036in}}%
\pgfpathlineto{\pgfqpoint{2.613640in}{2.166707in}}%
\pgfpathlineto{\pgfqpoint{2.622337in}{2.168578in}}%
\pgfpathlineto{\pgfqpoint{2.631019in}{2.170678in}}%
\pgfpathlineto{\pgfqpoint{2.639686in}{2.173001in}}%
\pgfpathlineto{\pgfqpoint{2.648338in}{2.175543in}}%
\pgfpathlineto{\pgfqpoint{2.634515in}{2.192385in}}%
\pgfpathlineto{\pgfqpoint{2.620684in}{2.209508in}}%
\pgfpathlineto{\pgfqpoint{2.606844in}{2.226914in}}%
\pgfpathlineto{\pgfqpoint{2.592996in}{2.244608in}}%
\pgfpathlineto{\pgfqpoint{2.584303in}{2.242542in}}%
\pgfpathlineto{\pgfqpoint{2.575595in}{2.240703in}}%
\pgfpathlineto{\pgfqpoint{2.566872in}{2.239097in}}%
\pgfpathlineto{\pgfqpoint{2.558134in}{2.237727in}}%
\pgfpathclose%
\pgfusepath{fill}%
\end{pgfscope}%
\begin{pgfscope}%
\pgfpathrectangle{\pgfqpoint{1.150000in}{0.150000in}}{\pgfqpoint{5.700000in}{5.700000in}}%
\pgfusepath{clip}%
\pgfsetbuttcap%
\pgfsetroundjoin%
\definecolor{currentfill}{rgb}{0.282290,0.145912,0.461510}%
\pgfsetfillcolor{currentfill}%
\pgfsetfillopacity{0.800000}%
\pgfsetlinewidth{0.000000pt}%
\definecolor{currentstroke}{rgb}{0.000000,0.000000,0.000000}%
\pgfsetstrokecolor{currentstroke}%
\pgfsetdash{}{0pt}%
\pgfpathmoveto{\pgfqpoint{2.724265in}{2.037970in}}%
\pgfpathlineto{\pgfqpoint{2.738062in}{2.023083in}}%
\pgfpathlineto{\pgfqpoint{2.751853in}{2.008457in}}%
\pgfpathlineto{\pgfqpoint{2.765638in}{1.994091in}}%
\pgfpathlineto{\pgfqpoint{2.779418in}{1.979982in}}%
\pgfpathlineto{\pgfqpoint{2.787999in}{1.983341in}}%
\pgfpathlineto{\pgfqpoint{2.796568in}{1.986902in}}%
\pgfpathlineto{\pgfqpoint{2.805124in}{1.990661in}}%
\pgfpathlineto{\pgfqpoint{2.813667in}{1.994614in}}%
\pgfpathlineto{\pgfqpoint{2.799922in}{2.008244in}}%
\pgfpathlineto{\pgfqpoint{2.786173in}{2.022130in}}%
\pgfpathlineto{\pgfqpoint{2.772418in}{2.036275in}}%
\pgfpathlineto{\pgfqpoint{2.758657in}{2.050681in}}%
\pgfpathlineto{\pgfqpoint{2.750079in}{2.047196in}}%
\pgfpathlineto{\pgfqpoint{2.741488in}{2.043913in}}%
\pgfpathlineto{\pgfqpoint{2.732883in}{2.040836in}}%
\pgfpathlineto{\pgfqpoint{2.724265in}{2.037970in}}%
\pgfpathclose%
\pgfusepath{fill}%
\end{pgfscope}%
\begin{pgfscope}%
\pgfpathrectangle{\pgfqpoint{1.150000in}{0.150000in}}{\pgfqpoint{5.700000in}{5.700000in}}%
\pgfusepath{clip}%
\pgfsetbuttcap%
\pgfsetroundjoin%
\definecolor{currentfill}{rgb}{0.281412,0.155834,0.469201}%
\pgfsetfillcolor{currentfill}%
\pgfsetfillopacity{0.800000}%
\pgfsetlinewidth{0.000000pt}%
\definecolor{currentstroke}{rgb}{0.000000,0.000000,0.000000}%
\pgfsetstrokecolor{currentstroke}%
\pgfsetdash{}{0pt}%
\pgfpathmoveto{\pgfqpoint{3.983681in}{2.011158in}}%
\pgfpathlineto{\pgfqpoint{3.997464in}{2.014306in}}%
\pgfpathlineto{\pgfqpoint{4.011257in}{2.017645in}}%
\pgfpathlineto{\pgfqpoint{4.025059in}{2.021174in}}%
\pgfpathlineto{\pgfqpoint{4.038872in}{2.024893in}}%
\pgfpathlineto{\pgfqpoint{4.046868in}{2.036379in}}%
\pgfpathlineto{\pgfqpoint{4.054858in}{2.047816in}}%
\pgfpathlineto{\pgfqpoint{4.062844in}{2.059204in}}%
\pgfpathlineto{\pgfqpoint{4.070825in}{2.070540in}}%
\pgfpathlineto{\pgfqpoint{4.057017in}{2.066670in}}%
\pgfpathlineto{\pgfqpoint{4.043220in}{2.062991in}}%
\pgfpathlineto{\pgfqpoint{4.029432in}{2.059501in}}%
\pgfpathlineto{\pgfqpoint{4.015655in}{2.056203in}}%
\pgfpathlineto{\pgfqpoint{4.007669in}{2.045004in}}%
\pgfpathlineto{\pgfqpoint{3.999678in}{2.033764in}}%
\pgfpathlineto{\pgfqpoint{3.991682in}{2.022481in}}%
\pgfpathlineto{\pgfqpoint{3.983681in}{2.011158in}}%
\pgfpathclose%
\pgfusepath{fill}%
\end{pgfscope}%
\begin{pgfscope}%
\pgfpathrectangle{\pgfqpoint{1.150000in}{0.150000in}}{\pgfqpoint{5.700000in}{5.700000in}}%
\pgfusepath{clip}%
\pgfsetbuttcap%
\pgfsetroundjoin%
\definecolor{currentfill}{rgb}{0.121831,0.589055,0.545623}%
\pgfsetfillcolor{currentfill}%
\pgfsetfillopacity{0.800000}%
\pgfsetlinewidth{0.000000pt}%
\definecolor{currentstroke}{rgb}{0.000000,0.000000,0.000000}%
\pgfsetstrokecolor{currentstroke}%
\pgfsetdash{}{0pt}%
\pgfpathmoveto{\pgfqpoint{5.505600in}{3.200296in}}%
\pgfpathlineto{\pgfqpoint{5.520156in}{3.211942in}}%
\pgfpathlineto{\pgfqpoint{5.534732in}{3.223767in}}%
\pgfpathlineto{\pgfqpoint{5.549328in}{3.235770in}}%
\pgfpathlineto{\pgfqpoint{5.563943in}{3.247952in}}%
\pgfpathlineto{\pgfqpoint{5.571247in}{3.251007in}}%
\pgfpathlineto{\pgfqpoint{5.578544in}{3.254061in}}%
\pgfpathlineto{\pgfqpoint{5.585835in}{3.257119in}}%
\pgfpathlineto{\pgfqpoint{5.593119in}{3.260189in}}%
\pgfpathlineto{\pgfqpoint{5.578529in}{3.248527in}}%
\pgfpathlineto{\pgfqpoint{5.563959in}{3.237043in}}%
\pgfpathlineto{\pgfqpoint{5.549408in}{3.225736in}}%
\pgfpathlineto{\pgfqpoint{5.534877in}{3.214607in}}%
\pgfpathlineto{\pgfqpoint{5.527567in}{3.211007in}}%
\pgfpathlineto{\pgfqpoint{5.520251in}{3.207426in}}%
\pgfpathlineto{\pgfqpoint{5.512928in}{3.203857in}}%
\pgfpathlineto{\pgfqpoint{5.505600in}{3.200296in}}%
\pgfpathclose%
\pgfusepath{fill}%
\end{pgfscope}%
\begin{pgfscope}%
\pgfpathrectangle{\pgfqpoint{1.150000in}{0.150000in}}{\pgfqpoint{5.700000in}{5.700000in}}%
\pgfusepath{clip}%
\pgfsetbuttcap%
\pgfsetroundjoin%
\definecolor{currentfill}{rgb}{0.204903,0.375746,0.553533}%
\pgfsetfillcolor{currentfill}%
\pgfsetfillopacity{0.800000}%
\pgfsetlinewidth{0.000000pt}%
\definecolor{currentstroke}{rgb}{0.000000,0.000000,0.000000}%
\pgfsetstrokecolor{currentstroke}%
\pgfsetdash{}{0pt}%
\pgfpathmoveto{\pgfqpoint{4.657448in}{2.554742in}}%
\pgfpathlineto{\pgfqpoint{4.671535in}{2.563314in}}%
\pgfpathlineto{\pgfqpoint{4.685637in}{2.572071in}}%
\pgfpathlineto{\pgfqpoint{4.699754in}{2.581011in}}%
\pgfpathlineto{\pgfqpoint{4.713886in}{2.590135in}}%
\pgfpathlineto{\pgfqpoint{4.721644in}{2.598849in}}%
\pgfpathlineto{\pgfqpoint{4.729395in}{2.607469in}}%
\pgfpathlineto{\pgfqpoint{4.737139in}{2.615998in}}%
\pgfpathlineto{\pgfqpoint{4.744877in}{2.624437in}}%
\pgfpathlineto{\pgfqpoint{4.730752in}{2.615458in}}%
\pgfpathlineto{\pgfqpoint{4.716642in}{2.606661in}}%
\pgfpathlineto{\pgfqpoint{4.702548in}{2.598049in}}%
\pgfpathlineto{\pgfqpoint{4.688469in}{2.589620in}}%
\pgfpathlineto{\pgfqpoint{4.680723in}{2.581025in}}%
\pgfpathlineto{\pgfqpoint{4.672971in}{2.572348in}}%
\pgfpathlineto{\pgfqpoint{4.665213in}{2.563588in}}%
\pgfpathlineto{\pgfqpoint{4.657448in}{2.554742in}}%
\pgfpathclose%
\pgfusepath{fill}%
\end{pgfscope}%
\begin{pgfscope}%
\pgfpathrectangle{\pgfqpoint{1.150000in}{0.150000in}}{\pgfqpoint{5.700000in}{5.700000in}}%
\pgfusepath{clip}%
\pgfsetbuttcap%
\pgfsetroundjoin%
\definecolor{currentfill}{rgb}{0.255645,0.260703,0.528312}%
\pgfsetfillcolor{currentfill}%
\pgfsetfillopacity{0.800000}%
\pgfsetlinewidth{0.000000pt}%
\definecolor{currentstroke}{rgb}{0.000000,0.000000,0.000000}%
\pgfsetstrokecolor{currentstroke}%
\pgfsetdash{}{0pt}%
\pgfpathmoveto{\pgfqpoint{2.502474in}{2.313381in}}%
\pgfpathlineto{\pgfqpoint{2.516404in}{2.294024in}}%
\pgfpathlineto{\pgfqpoint{2.530324in}{2.274965in}}%
\pgfpathlineto{\pgfqpoint{2.544234in}{2.256200in}}%
\pgfpathlineto{\pgfqpoint{2.558134in}{2.237727in}}%
\pgfpathlineto{\pgfqpoint{2.566872in}{2.239097in}}%
\pgfpathlineto{\pgfqpoint{2.575595in}{2.240703in}}%
\pgfpathlineto{\pgfqpoint{2.584303in}{2.242542in}}%
\pgfpathlineto{\pgfqpoint{2.592996in}{2.244608in}}%
\pgfpathlineto{\pgfqpoint{2.579138in}{2.262589in}}%
\pgfpathlineto{\pgfqpoint{2.565271in}{2.280862in}}%
\pgfpathlineto{\pgfqpoint{2.551394in}{2.299428in}}%
\pgfpathlineto{\pgfqpoint{2.537507in}{2.318291in}}%
\pgfpathlineto{\pgfqpoint{2.528772in}{2.316704in}}%
\pgfpathlineto{\pgfqpoint{2.520022in}{2.315354in}}%
\pgfpathlineto{\pgfqpoint{2.511256in}{2.314244in}}%
\pgfpathlineto{\pgfqpoint{2.502474in}{2.313381in}}%
\pgfpathclose%
\pgfusepath{fill}%
\end{pgfscope}%
\begin{pgfscope}%
\pgfpathrectangle{\pgfqpoint{1.150000in}{0.150000in}}{\pgfqpoint{5.700000in}{5.700000in}}%
\pgfusepath{clip}%
\pgfsetbuttcap%
\pgfsetroundjoin%
\definecolor{currentfill}{rgb}{0.283229,0.120777,0.440584}%
\pgfsetfillcolor{currentfill}%
\pgfsetfillopacity{0.800000}%
\pgfsetlinewidth{0.000000pt}%
\definecolor{currentstroke}{rgb}{0.000000,0.000000,0.000000}%
\pgfsetstrokecolor{currentstroke}%
\pgfsetdash{}{0pt}%
\pgfpathmoveto{\pgfqpoint{2.779418in}{1.979982in}}%
\pgfpathlineto{\pgfqpoint{2.793192in}{1.966128in}}%
\pgfpathlineto{\pgfqpoint{2.806961in}{1.952529in}}%
\pgfpathlineto{\pgfqpoint{2.820725in}{1.939181in}}%
\pgfpathlineto{\pgfqpoint{2.834485in}{1.926083in}}%
\pgfpathlineto{\pgfqpoint{2.843032in}{1.929932in}}%
\pgfpathlineto{\pgfqpoint{2.851566in}{1.933976in}}%
\pgfpathlineto{\pgfqpoint{2.860088in}{1.938209in}}%
\pgfpathlineto{\pgfqpoint{2.868598in}{1.942626in}}%
\pgfpathlineto{\pgfqpoint{2.854872in}{1.955247in}}%
\pgfpathlineto{\pgfqpoint{2.841141in}{1.968117in}}%
\pgfpathlineto{\pgfqpoint{2.827406in}{1.981239in}}%
\pgfpathlineto{\pgfqpoint{2.813667in}{1.994614in}}%
\pgfpathlineto{\pgfqpoint{2.805124in}{1.990661in}}%
\pgfpathlineto{\pgfqpoint{2.796568in}{1.986902in}}%
\pgfpathlineto{\pgfqpoint{2.787999in}{1.983341in}}%
\pgfpathlineto{\pgfqpoint{2.779418in}{1.979982in}}%
\pgfpathclose%
\pgfusepath{fill}%
\end{pgfscope}%
\begin{pgfscope}%
\pgfpathrectangle{\pgfqpoint{1.150000in}{0.150000in}}{\pgfqpoint{5.700000in}{5.700000in}}%
\pgfusepath{clip}%
\pgfsetbuttcap%
\pgfsetroundjoin%
\definecolor{currentfill}{rgb}{0.246811,0.283237,0.535941}%
\pgfsetfillcolor{currentfill}%
\pgfsetfillopacity{0.800000}%
\pgfsetlinewidth{0.000000pt}%
\definecolor{currentstroke}{rgb}{0.000000,0.000000,0.000000}%
\pgfsetstrokecolor{currentstroke}%
\pgfsetdash{}{0pt}%
\pgfpathmoveto{\pgfqpoint{4.364164in}{2.309002in}}%
\pgfpathlineto{\pgfqpoint{4.378105in}{2.315595in}}%
\pgfpathlineto{\pgfqpoint{4.392059in}{2.322375in}}%
\pgfpathlineto{\pgfqpoint{4.406027in}{2.329341in}}%
\pgfpathlineto{\pgfqpoint{4.420008in}{2.336493in}}%
\pgfpathlineto{\pgfqpoint{4.427881in}{2.346892in}}%
\pgfpathlineto{\pgfqpoint{4.435748in}{2.357204in}}%
\pgfpathlineto{\pgfqpoint{4.443610in}{2.367428in}}%
\pgfpathlineto{\pgfqpoint{4.451466in}{2.377566in}}%
\pgfpathlineto{\pgfqpoint{4.437490in}{2.370425in}}%
\pgfpathlineto{\pgfqpoint{4.423527in}{2.363471in}}%
\pgfpathlineto{\pgfqpoint{4.409578in}{2.356702in}}%
\pgfpathlineto{\pgfqpoint{4.395642in}{2.350119in}}%
\pgfpathlineto{\pgfqpoint{4.387781in}{2.339958in}}%
\pgfpathlineto{\pgfqpoint{4.379914in}{2.329719in}}%
\pgfpathlineto{\pgfqpoint{4.372042in}{2.319400in}}%
\pgfpathlineto{\pgfqpoint{4.364164in}{2.309002in}}%
\pgfpathclose%
\pgfusepath{fill}%
\end{pgfscope}%
\begin{pgfscope}%
\pgfpathrectangle{\pgfqpoint{1.150000in}{0.150000in}}{\pgfqpoint{5.700000in}{5.700000in}}%
\pgfusepath{clip}%
\pgfsetbuttcap%
\pgfsetroundjoin%
\definecolor{currentfill}{rgb}{0.272594,0.025563,0.353093}%
\pgfsetfillcolor{currentfill}%
\pgfsetfillopacity{0.800000}%
\pgfsetlinewidth{0.000000pt}%
\definecolor{currentstroke}{rgb}{0.000000,0.000000,0.000000}%
\pgfsetstrokecolor{currentstroke}%
\pgfsetdash{}{0pt}%
\pgfpathmoveto{\pgfqpoint{3.460557in}{1.752031in}}%
\pgfpathlineto{\pgfqpoint{3.474226in}{1.748896in}}%
\pgfpathlineto{\pgfqpoint{3.487900in}{1.745965in}}%
\pgfpathlineto{\pgfqpoint{3.501578in}{1.743236in}}%
\pgfpathlineto{\pgfqpoint{3.515261in}{1.740710in}}%
\pgfpathlineto{\pgfqpoint{3.523434in}{1.750690in}}%
\pgfpathlineto{\pgfqpoint{3.531602in}{1.760722in}}%
\pgfpathlineto{\pgfqpoint{3.539764in}{1.770802in}}%
\pgfpathlineto{\pgfqpoint{3.547919in}{1.780926in}}%
\pgfpathlineto{\pgfqpoint{3.534250in}{1.783113in}}%
\pgfpathlineto{\pgfqpoint{3.520585in}{1.785502in}}%
\pgfpathlineto{\pgfqpoint{3.506926in}{1.788094in}}%
\pgfpathlineto{\pgfqpoint{3.493271in}{1.790889in}}%
\pgfpathlineto{\pgfqpoint{3.485102in}{1.781092in}}%
\pgfpathlineto{\pgfqpoint{3.476927in}{1.771348in}}%
\pgfpathlineto{\pgfqpoint{3.468745in}{1.761660in}}%
\pgfpathlineto{\pgfqpoint{3.460557in}{1.752031in}}%
\pgfpathclose%
\pgfusepath{fill}%
\end{pgfscope}%
\begin{pgfscope}%
\pgfpathrectangle{\pgfqpoint{1.150000in}{0.150000in}}{\pgfqpoint{5.700000in}{5.700000in}}%
\pgfusepath{clip}%
\pgfsetbuttcap%
\pgfsetroundjoin%
\definecolor{currentfill}{rgb}{0.185556,0.418570,0.556753}%
\pgfsetfillcolor{currentfill}%
\pgfsetfillopacity{0.800000}%
\pgfsetlinewidth{0.000000pt}%
\definecolor{currentstroke}{rgb}{0.000000,0.000000,0.000000}%
\pgfsetstrokecolor{currentstroke}%
\pgfsetdash{}{0pt}%
\pgfpathmoveto{\pgfqpoint{2.257176in}{2.762086in}}%
\pgfpathlineto{\pgfqpoint{2.271332in}{2.736688in}}%
\pgfpathlineto{\pgfqpoint{2.285470in}{2.711648in}}%
\pgfpathlineto{\pgfqpoint{2.299593in}{2.686963in}}%
\pgfpathlineto{\pgfqpoint{2.313699in}{2.662629in}}%
\pgfpathlineto{\pgfqpoint{2.322601in}{2.662524in}}%
\pgfpathlineto{\pgfqpoint{2.331484in}{2.662682in}}%
\pgfpathlineto{\pgfqpoint{2.340350in}{2.663098in}}%
\pgfpathlineto{\pgfqpoint{2.349199in}{2.663768in}}%
\pgfpathlineto{\pgfqpoint{2.335141in}{2.687626in}}%
\pgfpathlineto{\pgfqpoint{2.321068in}{2.711833in}}%
\pgfpathlineto{\pgfqpoint{2.306980in}{2.736393in}}%
\pgfpathlineto{\pgfqpoint{2.292874in}{2.761311in}}%
\pgfpathlineto{\pgfqpoint{2.283977in}{2.761106in}}%
\pgfpathlineto{\pgfqpoint{2.275062in}{2.761163in}}%
\pgfpathlineto{\pgfqpoint{2.266129in}{2.761489in}}%
\pgfpathlineto{\pgfqpoint{2.257176in}{2.762086in}}%
\pgfpathclose%
\pgfusepath{fill}%
\end{pgfscope}%
\begin{pgfscope}%
\pgfpathrectangle{\pgfqpoint{1.150000in}{0.150000in}}{\pgfqpoint{5.700000in}{5.700000in}}%
\pgfusepath{clip}%
\pgfsetbuttcap%
\pgfsetroundjoin%
\definecolor{currentfill}{rgb}{0.160665,0.478540,0.558115}%
\pgfsetfillcolor{currentfill}%
\pgfsetfillopacity{0.800000}%
\pgfsetlinewidth{0.000000pt}%
\definecolor{currentstroke}{rgb}{0.000000,0.000000,0.000000}%
\pgfsetstrokecolor{currentstroke}%
\pgfsetdash{}{0pt}%
\pgfpathmoveto{\pgfqpoint{5.038047in}{2.860322in}}%
\pgfpathlineto{\pgfqpoint{5.052345in}{2.870760in}}%
\pgfpathlineto{\pgfqpoint{5.066659in}{2.881380in}}%
\pgfpathlineto{\pgfqpoint{5.080991in}{2.892182in}}%
\pgfpathlineto{\pgfqpoint{5.095341in}{2.903166in}}%
\pgfpathlineto{\pgfqpoint{5.102916in}{2.909274in}}%
\pgfpathlineto{\pgfqpoint{5.110484in}{2.915310in}}%
\pgfpathlineto{\pgfqpoint{5.118044in}{2.921278in}}%
\pgfpathlineto{\pgfqpoint{5.125597in}{2.927182in}}%
\pgfpathlineto{\pgfqpoint{5.111261in}{2.916513in}}%
\pgfpathlineto{\pgfqpoint{5.096943in}{2.906025in}}%
\pgfpathlineto{\pgfqpoint{5.082642in}{2.895718in}}%
\pgfpathlineto{\pgfqpoint{5.068358in}{2.885592in}}%
\pgfpathlineto{\pgfqpoint{5.060791in}{2.879362in}}%
\pgfpathlineto{\pgfqpoint{5.053217in}{2.873077in}}%
\pgfpathlineto{\pgfqpoint{5.045636in}{2.866731in}}%
\pgfpathlineto{\pgfqpoint{5.038047in}{2.860322in}}%
\pgfpathclose%
\pgfusepath{fill}%
\end{pgfscope}%
\begin{pgfscope}%
\pgfpathrectangle{\pgfqpoint{1.150000in}{0.150000in}}{\pgfqpoint{5.700000in}{5.700000in}}%
\pgfusepath{clip}%
\pgfsetbuttcap%
\pgfsetroundjoin%
\definecolor{currentfill}{rgb}{0.119423,0.611141,0.538982}%
\pgfsetfillcolor{currentfill}%
\pgfsetfillopacity{0.800000}%
\pgfsetlinewidth{0.000000pt}%
\definecolor{currentstroke}{rgb}{0.000000,0.000000,0.000000}%
\pgfsetstrokecolor{currentstroke}%
\pgfsetdash{}{0pt}%
\pgfpathmoveto{\pgfqpoint{5.593119in}{3.260189in}}%
\pgfpathlineto{\pgfqpoint{5.607729in}{3.272030in}}%
\pgfpathlineto{\pgfqpoint{5.622358in}{3.284048in}}%
\pgfpathlineto{\pgfqpoint{5.637008in}{3.296245in}}%
\pgfpathlineto{\pgfqpoint{5.651678in}{3.308621in}}%
\pgfpathlineto{\pgfqpoint{5.658929in}{3.311166in}}%
\pgfpathlineto{\pgfqpoint{5.666174in}{3.313725in}}%
\pgfpathlineto{\pgfqpoint{5.673413in}{3.316306in}}%
\pgfpathlineto{\pgfqpoint{5.680645in}{3.318913in}}%
\pgfpathlineto{\pgfqpoint{5.666003in}{3.307093in}}%
\pgfpathlineto{\pgfqpoint{5.651381in}{3.295449in}}%
\pgfpathlineto{\pgfqpoint{5.636779in}{3.283984in}}%
\pgfpathlineto{\pgfqpoint{5.622197in}{3.272695in}}%
\pgfpathlineto{\pgfqpoint{5.614936in}{3.269522in}}%
\pgfpathlineto{\pgfqpoint{5.607669in}{3.266385in}}%
\pgfpathlineto{\pgfqpoint{5.600397in}{3.263276in}}%
\pgfpathlineto{\pgfqpoint{5.593119in}{3.260189in}}%
\pgfpathclose%
\pgfusepath{fill}%
\end{pgfscope}%
\begin{pgfscope}%
\pgfpathrectangle{\pgfqpoint{1.150000in}{0.150000in}}{\pgfqpoint{5.700000in}{5.700000in}}%
\pgfusepath{clip}%
\pgfsetbuttcap%
\pgfsetroundjoin%
\definecolor{currentfill}{rgb}{0.269944,0.014625,0.341379}%
\pgfsetfillcolor{currentfill}%
\pgfsetfillopacity{0.800000}%
\pgfsetlinewidth{0.000000pt}%
\definecolor{currentstroke}{rgb}{0.000000,0.000000,0.000000}%
\pgfsetstrokecolor{currentstroke}%
\pgfsetdash{}{0pt}%
\pgfpathmoveto{\pgfqpoint{3.230628in}{1.740623in}}%
\pgfpathlineto{\pgfqpoint{3.244298in}{1.734202in}}%
\pgfpathlineto{\pgfqpoint{3.257969in}{1.727997in}}%
\pgfpathlineto{\pgfqpoint{3.271642in}{1.722005in}}%
\pgfpathlineto{\pgfqpoint{3.285318in}{1.716225in}}%
\pgfpathlineto{\pgfqpoint{3.293597in}{1.724430in}}%
\pgfpathlineto{\pgfqpoint{3.301869in}{1.732738in}}%
\pgfpathlineto{\pgfqpoint{3.310133in}{1.741145in}}%
\pgfpathlineto{\pgfqpoint{3.318389in}{1.749647in}}%
\pgfpathlineto{\pgfqpoint{3.304734in}{1.755023in}}%
\pgfpathlineto{\pgfqpoint{3.291081in}{1.760612in}}%
\pgfpathlineto{\pgfqpoint{3.277429in}{1.766414in}}%
\pgfpathlineto{\pgfqpoint{3.263780in}{1.772430in}}%
\pgfpathlineto{\pgfqpoint{3.255504in}{1.764320in}}%
\pgfpathlineto{\pgfqpoint{3.247220in}{1.756312in}}%
\pgfpathlineto{\pgfqpoint{3.238928in}{1.748412in}}%
\pgfpathlineto{\pgfqpoint{3.230628in}{1.740623in}}%
\pgfpathclose%
\pgfusepath{fill}%
\end{pgfscope}%
\begin{pgfscope}%
\pgfpathrectangle{\pgfqpoint{1.150000in}{0.150000in}}{\pgfqpoint{5.700000in}{5.700000in}}%
\pgfusepath{clip}%
\pgfsetbuttcap%
\pgfsetroundjoin%
\definecolor{currentfill}{rgb}{0.277134,0.185228,0.489898}%
\pgfsetfillcolor{currentfill}%
\pgfsetfillopacity{0.800000}%
\pgfsetlinewidth{0.000000pt}%
\definecolor{currentstroke}{rgb}{0.000000,0.000000,0.000000}%
\pgfsetstrokecolor{currentstroke}%
\pgfsetdash{}{0pt}%
\pgfpathmoveto{\pgfqpoint{4.070825in}{2.070540in}}%
\pgfpathlineto{\pgfqpoint{4.084643in}{2.074599in}}%
\pgfpathlineto{\pgfqpoint{4.098472in}{2.078848in}}%
\pgfpathlineto{\pgfqpoint{4.112312in}{2.083285in}}%
\pgfpathlineto{\pgfqpoint{4.126163in}{2.087912in}}%
\pgfpathlineto{\pgfqpoint{4.134134in}{2.099327in}}%
\pgfpathlineto{\pgfqpoint{4.142100in}{2.110682in}}%
\pgfpathlineto{\pgfqpoint{4.150060in}{2.121975in}}%
\pgfpathlineto{\pgfqpoint{4.158016in}{2.133207in}}%
\pgfpathlineto{\pgfqpoint{4.144170in}{2.128462in}}%
\pgfpathlineto{\pgfqpoint{4.130335in}{2.123906in}}%
\pgfpathlineto{\pgfqpoint{4.116511in}{2.119538in}}%
\pgfpathlineto{\pgfqpoint{4.102698in}{2.115361in}}%
\pgfpathlineto{\pgfqpoint{4.094737in}{2.104236in}}%
\pgfpathlineto{\pgfqpoint{4.086771in}{2.093057in}}%
\pgfpathlineto{\pgfqpoint{4.078801in}{2.081825in}}%
\pgfpathlineto{\pgfqpoint{4.070825in}{2.070540in}}%
\pgfpathclose%
\pgfusepath{fill}%
\end{pgfscope}%
\begin{pgfscope}%
\pgfpathrectangle{\pgfqpoint{1.150000in}{0.150000in}}{\pgfqpoint{5.700000in}{5.700000in}}%
\pgfusepath{clip}%
\pgfsetbuttcap%
\pgfsetroundjoin%
\definecolor{currentfill}{rgb}{0.243113,0.292092,0.538516}%
\pgfsetfillcolor{currentfill}%
\pgfsetfillopacity{0.800000}%
\pgfsetlinewidth{0.000000pt}%
\definecolor{currentstroke}{rgb}{0.000000,0.000000,0.000000}%
\pgfsetstrokecolor{currentstroke}%
\pgfsetdash{}{0pt}%
\pgfpathmoveto{\pgfqpoint{2.446642in}{2.393830in}}%
\pgfpathlineto{\pgfqpoint{2.460617in}{2.373259in}}%
\pgfpathlineto{\pgfqpoint{2.474581in}{2.352995in}}%
\pgfpathlineto{\pgfqpoint{2.488533in}{2.333037in}}%
\pgfpathlineto{\pgfqpoint{2.502474in}{2.313381in}}%
\pgfpathlineto{\pgfqpoint{2.511256in}{2.314244in}}%
\pgfpathlineto{\pgfqpoint{2.520022in}{2.315354in}}%
\pgfpathlineto{\pgfqpoint{2.528772in}{2.316704in}}%
\pgfpathlineto{\pgfqpoint{2.537507in}{2.318291in}}%
\pgfpathlineto{\pgfqpoint{2.523610in}{2.337452in}}%
\pgfpathlineto{\pgfqpoint{2.509702in}{2.356914in}}%
\pgfpathlineto{\pgfqpoint{2.495783in}{2.376681in}}%
\pgfpathlineto{\pgfqpoint{2.481853in}{2.396754in}}%
\pgfpathlineto{\pgfqpoint{2.473075in}{2.395651in}}%
\pgfpathlineto{\pgfqpoint{2.464281in}{2.394792in}}%
\pgfpathlineto{\pgfqpoint{2.455470in}{2.394184in}}%
\pgfpathlineto{\pgfqpoint{2.446642in}{2.393830in}}%
\pgfpathclose%
\pgfusepath{fill}%
\end{pgfscope}%
\begin{pgfscope}%
\pgfpathrectangle{\pgfqpoint{1.150000in}{0.150000in}}{\pgfqpoint{5.700000in}{5.700000in}}%
\pgfusepath{clip}%
\pgfsetbuttcap%
\pgfsetroundjoin%
\definecolor{currentfill}{rgb}{0.282656,0.100196,0.422160}%
\pgfsetfillcolor{currentfill}%
\pgfsetfillopacity{0.800000}%
\pgfsetlinewidth{0.000000pt}%
\definecolor{currentstroke}{rgb}{0.000000,0.000000,0.000000}%
\pgfsetstrokecolor{currentstroke}%
\pgfsetdash{}{0pt}%
\pgfpathmoveto{\pgfqpoint{2.834485in}{1.926083in}}%
\pgfpathlineto{\pgfqpoint{2.848240in}{1.913234in}}%
\pgfpathlineto{\pgfqpoint{2.861991in}{1.900631in}}%
\pgfpathlineto{\pgfqpoint{2.875739in}{1.888274in}}%
\pgfpathlineto{\pgfqpoint{2.889482in}{1.876160in}}%
\pgfpathlineto{\pgfqpoint{2.897996in}{1.880498in}}%
\pgfpathlineto{\pgfqpoint{2.906497in}{1.885021in}}%
\pgfpathlineto{\pgfqpoint{2.914987in}{1.889726in}}%
\pgfpathlineto{\pgfqpoint{2.923465in}{1.894607in}}%
\pgfpathlineto{\pgfqpoint{2.909753in}{1.906245in}}%
\pgfpathlineto{\pgfqpoint{2.896038in}{1.918127in}}%
\pgfpathlineto{\pgfqpoint{2.882320in}{1.930254in}}%
\pgfpathlineto{\pgfqpoint{2.868598in}{1.942626in}}%
\pgfpathlineto{\pgfqpoint{2.860088in}{1.938209in}}%
\pgfpathlineto{\pgfqpoint{2.851566in}{1.933976in}}%
\pgfpathlineto{\pgfqpoint{2.843032in}{1.929932in}}%
\pgfpathlineto{\pgfqpoint{2.834485in}{1.926083in}}%
\pgfpathclose%
\pgfusepath{fill}%
\end{pgfscope}%
\begin{pgfscope}%
\pgfpathrectangle{\pgfqpoint{1.150000in}{0.150000in}}{\pgfqpoint{5.700000in}{5.700000in}}%
\pgfusepath{clip}%
\pgfsetbuttcap%
\pgfsetroundjoin%
\definecolor{currentfill}{rgb}{0.273809,0.031497,0.358853}%
\pgfsetfillcolor{currentfill}%
\pgfsetfillopacity{0.800000}%
\pgfsetlinewidth{0.000000pt}%
\definecolor{currentstroke}{rgb}{0.000000,0.000000,0.000000}%
\pgfsetstrokecolor{currentstroke}%
\pgfsetdash{}{0pt}%
\pgfpathmoveto{\pgfqpoint{3.087830in}{1.773355in}}%
\pgfpathlineto{\pgfqpoint{3.101519in}{1.764743in}}%
\pgfpathlineto{\pgfqpoint{3.115207in}{1.756355in}}%
\pgfpathlineto{\pgfqpoint{3.128896in}{1.748189in}}%
\pgfpathlineto{\pgfqpoint{3.142584in}{1.740244in}}%
\pgfpathlineto{\pgfqpoint{3.150942in}{1.747102in}}%
\pgfpathlineto{\pgfqpoint{3.159291in}{1.754094in}}%
\pgfpathlineto{\pgfqpoint{3.167630in}{1.761218in}}%
\pgfpathlineto{\pgfqpoint{3.175961in}{1.768468in}}%
\pgfpathlineto{\pgfqpoint{3.162297in}{1.775976in}}%
\pgfpathlineto{\pgfqpoint{3.148633in}{1.783705in}}%
\pgfpathlineto{\pgfqpoint{3.134969in}{1.791657in}}%
\pgfpathlineto{\pgfqpoint{3.121305in}{1.799831in}}%
\pgfpathlineto{\pgfqpoint{3.112950in}{1.793006in}}%
\pgfpathlineto{\pgfqpoint{3.104586in}{1.786315in}}%
\pgfpathlineto{\pgfqpoint{3.096213in}{1.779764in}}%
\pgfpathlineto{\pgfqpoint{3.087830in}{1.773355in}}%
\pgfpathclose%
\pgfusepath{fill}%
\end{pgfscope}%
\begin{pgfscope}%
\pgfpathrectangle{\pgfqpoint{1.150000in}{0.150000in}}{\pgfqpoint{5.700000in}{5.700000in}}%
\pgfusepath{clip}%
\pgfsetbuttcap%
\pgfsetroundjoin%
\definecolor{currentfill}{rgb}{0.192357,0.403199,0.555836}%
\pgfsetfillcolor{currentfill}%
\pgfsetfillopacity{0.800000}%
\pgfsetlinewidth{0.000000pt}%
\definecolor{currentstroke}{rgb}{0.000000,0.000000,0.000000}%
\pgfsetstrokecolor{currentstroke}%
\pgfsetdash{}{0pt}%
\pgfpathmoveto{\pgfqpoint{4.744877in}{2.624437in}}%
\pgfpathlineto{\pgfqpoint{4.759018in}{2.633600in}}%
\pgfpathlineto{\pgfqpoint{4.773174in}{2.642947in}}%
\pgfpathlineto{\pgfqpoint{4.787346in}{2.652477in}}%
\pgfpathlineto{\pgfqpoint{4.801534in}{2.662190in}}%
\pgfpathlineto{\pgfqpoint{4.809257in}{2.670376in}}%
\pgfpathlineto{\pgfqpoint{4.816973in}{2.678469in}}%
\pgfpathlineto{\pgfqpoint{4.824682in}{2.686470in}}%
\pgfpathlineto{\pgfqpoint{4.832385in}{2.694382in}}%
\pgfpathlineto{\pgfqpoint{4.818205in}{2.684847in}}%
\pgfpathlineto{\pgfqpoint{4.804041in}{2.675495in}}%
\pgfpathlineto{\pgfqpoint{4.789893in}{2.666326in}}%
\pgfpathlineto{\pgfqpoint{4.775761in}{2.657340in}}%
\pgfpathlineto{\pgfqpoint{4.768050in}{2.649238in}}%
\pgfpathlineto{\pgfqpoint{4.760332in}{2.641055in}}%
\pgfpathlineto{\pgfqpoint{4.752608in}{2.632789in}}%
\pgfpathlineto{\pgfqpoint{4.744877in}{2.624437in}}%
\pgfpathclose%
\pgfusepath{fill}%
\end{pgfscope}%
\begin{pgfscope}%
\pgfpathrectangle{\pgfqpoint{1.150000in}{0.150000in}}{\pgfqpoint{5.700000in}{5.700000in}}%
\pgfusepath{clip}%
\pgfsetbuttcap%
\pgfsetroundjoin%
\definecolor{currentfill}{rgb}{0.121380,0.629492,0.531973}%
\pgfsetfillcolor{currentfill}%
\pgfsetfillopacity{0.800000}%
\pgfsetlinewidth{0.000000pt}%
\definecolor{currentstroke}{rgb}{0.000000,0.000000,0.000000}%
\pgfsetstrokecolor{currentstroke}%
\pgfsetdash{}{0pt}%
\pgfpathmoveto{\pgfqpoint{5.680645in}{3.318913in}}%
\pgfpathlineto{\pgfqpoint{5.695308in}{3.330912in}}%
\pgfpathlineto{\pgfqpoint{5.709990in}{3.343088in}}%
\pgfpathlineto{\pgfqpoint{5.724693in}{3.355442in}}%
\pgfpathlineto{\pgfqpoint{5.739417in}{3.367975in}}%
\pgfpathlineto{\pgfqpoint{5.746614in}{3.370039in}}%
\pgfpathlineto{\pgfqpoint{5.753806in}{3.372135in}}%
\pgfpathlineto{\pgfqpoint{5.760992in}{3.374270in}}%
\pgfpathlineto{\pgfqpoint{5.768173in}{3.376449in}}%
\pgfpathlineto{\pgfqpoint{5.753480in}{3.364506in}}%
\pgfpathlineto{\pgfqpoint{5.738807in}{3.352741in}}%
\pgfpathlineto{\pgfqpoint{5.724154in}{3.341152in}}%
\pgfpathlineto{\pgfqpoint{5.709522in}{3.329739in}}%
\pgfpathlineto{\pgfqpoint{5.702310in}{3.326961in}}%
\pgfpathlineto{\pgfqpoint{5.695094in}{3.324235in}}%
\pgfpathlineto{\pgfqpoint{5.687872in}{3.321554in}}%
\pgfpathlineto{\pgfqpoint{5.680645in}{3.318913in}}%
\pgfpathclose%
\pgfusepath{fill}%
\end{pgfscope}%
\begin{pgfscope}%
\pgfpathrectangle{\pgfqpoint{1.150000in}{0.150000in}}{\pgfqpoint{5.700000in}{5.700000in}}%
\pgfusepath{clip}%
\pgfsetbuttcap%
\pgfsetroundjoin%
\definecolor{currentfill}{rgb}{0.271305,0.019942,0.347269}%
\pgfsetfillcolor{currentfill}%
\pgfsetfillopacity{0.800000}%
\pgfsetlinewidth{0.000000pt}%
\definecolor{currentstroke}{rgb}{0.000000,0.000000,0.000000}%
\pgfsetstrokecolor{currentstroke}%
\pgfsetdash{}{0pt}%
\pgfpathmoveto{\pgfqpoint{3.373038in}{1.730246in}}%
\pgfpathlineto{\pgfqpoint{3.386708in}{1.725917in}}%
\pgfpathlineto{\pgfqpoint{3.400382in}{1.721796in}}%
\pgfpathlineto{\pgfqpoint{3.414058in}{1.717880in}}%
\pgfpathlineto{\pgfqpoint{3.427739in}{1.714169in}}%
\pgfpathlineto{\pgfqpoint{3.435954in}{1.723530in}}%
\pgfpathlineto{\pgfqpoint{3.444161in}{1.732963in}}%
\pgfpathlineto{\pgfqpoint{3.452362in}{1.742464in}}%
\pgfpathlineto{\pgfqpoint{3.460557in}{1.752031in}}%
\pgfpathlineto{\pgfqpoint{3.446892in}{1.755370in}}%
\pgfpathlineto{\pgfqpoint{3.433231in}{1.758915in}}%
\pgfpathlineto{\pgfqpoint{3.419574in}{1.762666in}}%
\pgfpathlineto{\pgfqpoint{3.405921in}{1.766623in}}%
\pgfpathlineto{\pgfqpoint{3.397711in}{1.757415in}}%
\pgfpathlineto{\pgfqpoint{3.389494in}{1.748281in}}%
\pgfpathlineto{\pgfqpoint{3.381270in}{1.739223in}}%
\pgfpathlineto{\pgfqpoint{3.373038in}{1.730246in}}%
\pgfpathclose%
\pgfusepath{fill}%
\end{pgfscope}%
\begin{pgfscope}%
\pgfpathrectangle{\pgfqpoint{1.150000in}{0.150000in}}{\pgfqpoint{5.700000in}{5.700000in}}%
\pgfusepath{clip}%
\pgfsetbuttcap%
\pgfsetroundjoin%
\definecolor{currentfill}{rgb}{0.150476,0.504369,0.557430}%
\pgfsetfillcolor{currentfill}%
\pgfsetfillopacity{0.800000}%
\pgfsetlinewidth{0.000000pt}%
\definecolor{currentstroke}{rgb}{0.000000,0.000000,0.000000}%
\pgfsetstrokecolor{currentstroke}%
\pgfsetdash{}{0pt}%
\pgfpathmoveto{\pgfqpoint{5.125597in}{2.927182in}}%
\pgfpathlineto{\pgfqpoint{5.139950in}{2.938032in}}%
\pgfpathlineto{\pgfqpoint{5.154322in}{2.949063in}}%
\pgfpathlineto{\pgfqpoint{5.168711in}{2.960276in}}%
\pgfpathlineto{\pgfqpoint{5.183119in}{2.971670in}}%
\pgfpathlineto{\pgfqpoint{5.190650in}{2.977177in}}%
\pgfpathlineto{\pgfqpoint{5.198173in}{2.982619in}}%
\pgfpathlineto{\pgfqpoint{5.205688in}{2.988001in}}%
\pgfpathlineto{\pgfqpoint{5.213196in}{2.993326in}}%
\pgfpathlineto{\pgfqpoint{5.198804in}{2.982282in}}%
\pgfpathlineto{\pgfqpoint{5.184430in}{2.971417in}}%
\pgfpathlineto{\pgfqpoint{5.170074in}{2.960734in}}%
\pgfpathlineto{\pgfqpoint{5.155736in}{2.950231in}}%
\pgfpathlineto{\pgfqpoint{5.148212in}{2.944546in}}%
\pgfpathlineto{\pgfqpoint{5.140681in}{2.938812in}}%
\pgfpathlineto{\pgfqpoint{5.133142in}{2.933025in}}%
\pgfpathlineto{\pgfqpoint{5.125597in}{2.927182in}}%
\pgfpathclose%
\pgfusepath{fill}%
\end{pgfscope}%
\begin{pgfscope}%
\pgfpathrectangle{\pgfqpoint{1.150000in}{0.150000in}}{\pgfqpoint{5.700000in}{5.700000in}}%
\pgfusepath{clip}%
\pgfsetbuttcap%
\pgfsetroundjoin%
\definecolor{currentfill}{rgb}{0.233603,0.313828,0.543914}%
\pgfsetfillcolor{currentfill}%
\pgfsetfillopacity{0.800000}%
\pgfsetlinewidth{0.000000pt}%
\definecolor{currentstroke}{rgb}{0.000000,0.000000,0.000000}%
\pgfsetstrokecolor{currentstroke}%
\pgfsetdash{}{0pt}%
\pgfpathmoveto{\pgfqpoint{4.451466in}{2.377566in}}%
\pgfpathlineto{\pgfqpoint{4.465456in}{2.384892in}}%
\pgfpathlineto{\pgfqpoint{4.479459in}{2.392404in}}%
\pgfpathlineto{\pgfqpoint{4.493476in}{2.400101in}}%
\pgfpathlineto{\pgfqpoint{4.507507in}{2.407983in}}%
\pgfpathlineto{\pgfqpoint{4.515353in}{2.418002in}}%
\pgfpathlineto{\pgfqpoint{4.523192in}{2.427929in}}%
\pgfpathlineto{\pgfqpoint{4.531025in}{2.437763in}}%
\pgfpathlineto{\pgfqpoint{4.538852in}{2.447505in}}%
\pgfpathlineto{\pgfqpoint{4.524826in}{2.439667in}}%
\pgfpathlineto{\pgfqpoint{4.510814in}{2.432015in}}%
\pgfpathlineto{\pgfqpoint{4.496816in}{2.424547in}}%
\pgfpathlineto{\pgfqpoint{4.482832in}{2.417265in}}%
\pgfpathlineto{\pgfqpoint{4.474999in}{2.407466in}}%
\pgfpathlineto{\pgfqpoint{4.467160in}{2.397584in}}%
\pgfpathlineto{\pgfqpoint{4.459316in}{2.387618in}}%
\pgfpathlineto{\pgfqpoint{4.451466in}{2.377566in}}%
\pgfpathclose%
\pgfusepath{fill}%
\end{pgfscope}%
\begin{pgfscope}%
\pgfpathrectangle{\pgfqpoint{1.150000in}{0.150000in}}{\pgfqpoint{5.700000in}{5.700000in}}%
\pgfusepath{clip}%
\pgfsetbuttcap%
\pgfsetroundjoin%
\definecolor{currentfill}{rgb}{0.270595,0.214069,0.507052}%
\pgfsetfillcolor{currentfill}%
\pgfsetfillopacity{0.800000}%
\pgfsetlinewidth{0.000000pt}%
\definecolor{currentstroke}{rgb}{0.000000,0.000000,0.000000}%
\pgfsetstrokecolor{currentstroke}%
\pgfsetdash{}{0pt}%
\pgfpathmoveto{\pgfqpoint{4.158016in}{2.133207in}}%
\pgfpathlineto{\pgfqpoint{4.171874in}{2.138140in}}%
\pgfpathlineto{\pgfqpoint{4.185743in}{2.143261in}}%
\pgfpathlineto{\pgfqpoint{4.199623in}{2.148571in}}%
\pgfpathlineto{\pgfqpoint{4.213515in}{2.154067in}}%
\pgfpathlineto{\pgfqpoint{4.221462in}{2.165334in}}%
\pgfpathlineto{\pgfqpoint{4.229403in}{2.176530in}}%
\pgfpathlineto{\pgfqpoint{4.237339in}{2.187654in}}%
\pgfpathlineto{\pgfqpoint{4.245270in}{2.198705in}}%
\pgfpathlineto{\pgfqpoint{4.231382in}{2.193122in}}%
\pgfpathlineto{\pgfqpoint{4.217506in}{2.187726in}}%
\pgfpathlineto{\pgfqpoint{4.203642in}{2.182518in}}%
\pgfpathlineto{\pgfqpoint{4.189789in}{2.177498in}}%
\pgfpathlineto{\pgfqpoint{4.181853in}{2.166521in}}%
\pgfpathlineto{\pgfqpoint{4.173913in}{2.155480in}}%
\pgfpathlineto{\pgfqpoint{4.165967in}{2.144375in}}%
\pgfpathlineto{\pgfqpoint{4.158016in}{2.133207in}}%
\pgfpathclose%
\pgfusepath{fill}%
\end{pgfscope}%
\begin{pgfscope}%
\pgfpathrectangle{\pgfqpoint{1.150000in}{0.150000in}}{\pgfqpoint{5.700000in}{5.700000in}}%
\pgfusepath{clip}%
\pgfsetbuttcap%
\pgfsetroundjoin%
\definecolor{currentfill}{rgb}{0.227802,0.326594,0.546532}%
\pgfsetfillcolor{currentfill}%
\pgfsetfillopacity{0.800000}%
\pgfsetlinewidth{0.000000pt}%
\definecolor{currentstroke}{rgb}{0.000000,0.000000,0.000000}%
\pgfsetstrokecolor{currentstroke}%
\pgfsetdash{}{0pt}%
\pgfpathmoveto{\pgfqpoint{2.390618in}{2.479250in}}%
\pgfpathlineto{\pgfqpoint{2.404643in}{2.457419in}}%
\pgfpathlineto{\pgfqpoint{2.418655in}{2.435907in}}%
\pgfpathlineto{\pgfqpoint{2.432655in}{2.414712in}}%
\pgfpathlineto{\pgfqpoint{2.446642in}{2.393830in}}%
\pgfpathlineto{\pgfqpoint{2.455470in}{2.394184in}}%
\pgfpathlineto{\pgfqpoint{2.464281in}{2.394792in}}%
\pgfpathlineto{\pgfqpoint{2.473075in}{2.395651in}}%
\pgfpathlineto{\pgfqpoint{2.481853in}{2.396754in}}%
\pgfpathlineto{\pgfqpoint{2.467912in}{2.417137in}}%
\pgfpathlineto{\pgfqpoint{2.453958in}{2.437832in}}%
\pgfpathlineto{\pgfqpoint{2.439993in}{2.458843in}}%
\pgfpathlineto{\pgfqpoint{2.426015in}{2.480171in}}%
\pgfpathlineto{\pgfqpoint{2.417191in}{2.479555in}}%
\pgfpathlineto{\pgfqpoint{2.408351in}{2.479193in}}%
\pgfpathlineto{\pgfqpoint{2.399493in}{2.479090in}}%
\pgfpathlineto{\pgfqpoint{2.390618in}{2.479250in}}%
\pgfpathclose%
\pgfusepath{fill}%
\end{pgfscope}%
\begin{pgfscope}%
\pgfpathrectangle{\pgfqpoint{1.150000in}{0.150000in}}{\pgfqpoint{5.700000in}{5.700000in}}%
\pgfusepath{clip}%
\pgfsetbuttcap%
\pgfsetroundjoin%
\definecolor{currentfill}{rgb}{0.281446,0.084320,0.407414}%
\pgfsetfillcolor{currentfill}%
\pgfsetfillopacity{0.800000}%
\pgfsetlinewidth{0.000000pt}%
\definecolor{currentstroke}{rgb}{0.000000,0.000000,0.000000}%
\pgfsetstrokecolor{currentstroke}%
\pgfsetdash{}{0pt}%
\pgfpathmoveto{\pgfqpoint{2.889482in}{1.876160in}}%
\pgfpathlineto{\pgfqpoint{2.903222in}{1.864289in}}%
\pgfpathlineto{\pgfqpoint{2.916959in}{1.852657in}}%
\pgfpathlineto{\pgfqpoint{2.930693in}{1.841265in}}%
\pgfpathlineto{\pgfqpoint{2.944424in}{1.830110in}}%
\pgfpathlineto{\pgfqpoint{2.952906in}{1.834934in}}%
\pgfpathlineto{\pgfqpoint{2.961376in}{1.839935in}}%
\pgfpathlineto{\pgfqpoint{2.969835in}{1.845109in}}%
\pgfpathlineto{\pgfqpoint{2.978284in}{1.850451in}}%
\pgfpathlineto{\pgfqpoint{2.964583in}{1.861133in}}%
\pgfpathlineto{\pgfqpoint{2.950880in}{1.872052in}}%
\pgfpathlineto{\pgfqpoint{2.937174in}{1.883209in}}%
\pgfpathlineto{\pgfqpoint{2.923465in}{1.894607in}}%
\pgfpathlineto{\pgfqpoint{2.914987in}{1.889726in}}%
\pgfpathlineto{\pgfqpoint{2.906497in}{1.885021in}}%
\pgfpathlineto{\pgfqpoint{2.897996in}{1.880498in}}%
\pgfpathlineto{\pgfqpoint{2.889482in}{1.876160in}}%
\pgfpathclose%
\pgfusepath{fill}%
\end{pgfscope}%
\begin{pgfscope}%
\pgfpathrectangle{\pgfqpoint{1.150000in}{0.150000in}}{\pgfqpoint{5.700000in}{5.700000in}}%
\pgfusepath{clip}%
\pgfsetbuttcap%
\pgfsetroundjoin%
\definecolor{currentfill}{rgb}{0.130067,0.651384,0.521608}%
\pgfsetfillcolor{currentfill}%
\pgfsetfillopacity{0.800000}%
\pgfsetlinewidth{0.000000pt}%
\definecolor{currentstroke}{rgb}{0.000000,0.000000,0.000000}%
\pgfsetstrokecolor{currentstroke}%
\pgfsetdash{}{0pt}%
\pgfpathmoveto{\pgfqpoint{5.768173in}{3.376449in}}%
\pgfpathlineto{\pgfqpoint{5.782886in}{3.388569in}}%
\pgfpathlineto{\pgfqpoint{5.797621in}{3.400867in}}%
\pgfpathlineto{\pgfqpoint{5.812376in}{3.413341in}}%
\pgfpathlineto{\pgfqpoint{5.827152in}{3.425994in}}%
\pgfpathlineto{\pgfqpoint{5.834295in}{3.427613in}}%
\pgfpathlineto{\pgfqpoint{5.841434in}{3.429283in}}%
\pgfpathlineto{\pgfqpoint{5.848567in}{3.431010in}}%
\pgfpathlineto{\pgfqpoint{5.855695in}{3.432801in}}%
\pgfpathlineto{\pgfqpoint{5.840952in}{3.420772in}}%
\pgfpathlineto{\pgfqpoint{5.826230in}{3.408920in}}%
\pgfpathlineto{\pgfqpoint{5.811528in}{3.397245in}}%
\pgfpathlineto{\pgfqpoint{5.796847in}{3.385745in}}%
\pgfpathlineto{\pgfqpoint{5.789685in}{3.383321in}}%
\pgfpathlineto{\pgfqpoint{5.782519in}{3.380968in}}%
\pgfpathlineto{\pgfqpoint{5.775348in}{3.378680in}}%
\pgfpathlineto{\pgfqpoint{5.768173in}{3.376449in}}%
\pgfpathclose%
\pgfusepath{fill}%
\end{pgfscope}%
\begin{pgfscope}%
\pgfpathrectangle{\pgfqpoint{1.150000in}{0.150000in}}{\pgfqpoint{5.700000in}{5.700000in}}%
\pgfusepath{clip}%
\pgfsetbuttcap%
\pgfsetroundjoin%
\definecolor{currentfill}{rgb}{0.143303,0.669459,0.511215}%
\pgfsetfillcolor{currentfill}%
\pgfsetfillopacity{0.800000}%
\pgfsetlinewidth{0.000000pt}%
\definecolor{currentstroke}{rgb}{0.000000,0.000000,0.000000}%
\pgfsetstrokecolor{currentstroke}%
\pgfsetdash{}{0pt}%
\pgfpathmoveto{\pgfqpoint{5.855695in}{3.432801in}}%
\pgfpathlineto{\pgfqpoint{5.870459in}{3.445006in}}%
\pgfpathlineto{\pgfqpoint{5.885244in}{3.457388in}}%
\pgfpathlineto{\pgfqpoint{5.900050in}{3.469947in}}%
\pgfpathlineto{\pgfqpoint{5.914878in}{3.482684in}}%
\pgfpathlineto{\pgfqpoint{5.921967in}{3.483900in}}%
\pgfpathlineto{\pgfqpoint{5.929052in}{3.485186in}}%
\pgfpathlineto{\pgfqpoint{5.936132in}{3.486549in}}%
\pgfpathlineto{\pgfqpoint{5.943208in}{3.487996in}}%
\pgfpathlineto{\pgfqpoint{5.928417in}{3.475919in}}%
\pgfpathlineto{\pgfqpoint{5.913646in}{3.464017in}}%
\pgfpathlineto{\pgfqpoint{5.898896in}{3.452291in}}%
\pgfpathlineto{\pgfqpoint{5.884167in}{3.440740in}}%
\pgfpathlineto{\pgfqpoint{5.877055in}{3.438625in}}%
\pgfpathlineto{\pgfqpoint{5.869939in}{3.436601in}}%
\pgfpathlineto{\pgfqpoint{5.862819in}{3.434662in}}%
\pgfpathlineto{\pgfqpoint{5.855695in}{3.432801in}}%
\pgfpathclose%
\pgfusepath{fill}%
\end{pgfscope}%
\begin{pgfscope}%
\pgfpathrectangle{\pgfqpoint{1.150000in}{0.150000in}}{\pgfqpoint{5.700000in}{5.700000in}}%
\pgfusepath{clip}%
\pgfsetbuttcap%
\pgfsetroundjoin%
\definecolor{currentfill}{rgb}{0.141935,0.526453,0.555991}%
\pgfsetfillcolor{currentfill}%
\pgfsetfillopacity{0.800000}%
\pgfsetlinewidth{0.000000pt}%
\definecolor{currentstroke}{rgb}{0.000000,0.000000,0.000000}%
\pgfsetstrokecolor{currentstroke}%
\pgfsetdash{}{0pt}%
\pgfpathmoveto{\pgfqpoint{5.213196in}{2.993326in}}%
\pgfpathlineto{\pgfqpoint{5.227606in}{3.004552in}}%
\pgfpathlineto{\pgfqpoint{5.242035in}{3.015958in}}%
\pgfpathlineto{\pgfqpoint{5.256482in}{3.027545in}}%
\pgfpathlineto{\pgfqpoint{5.270948in}{3.039313in}}%
\pgfpathlineto{\pgfqpoint{5.278432in}{3.044214in}}%
\pgfpathlineto{\pgfqpoint{5.285908in}{3.049059in}}%
\pgfpathlineto{\pgfqpoint{5.293377in}{3.053853in}}%
\pgfpathlineto{\pgfqpoint{5.300838in}{3.058598in}}%
\pgfpathlineto{\pgfqpoint{5.286390in}{3.047214in}}%
\pgfpathlineto{\pgfqpoint{5.271960in}{3.036011in}}%
\pgfpathlineto{\pgfqpoint{5.257549in}{3.024987in}}%
\pgfpathlineto{\pgfqpoint{5.243156in}{3.014144in}}%
\pgfpathlineto{\pgfqpoint{5.235677in}{3.009003in}}%
\pgfpathlineto{\pgfqpoint{5.228190in}{3.003823in}}%
\pgfpathlineto{\pgfqpoint{5.220697in}{2.998599in}}%
\pgfpathlineto{\pgfqpoint{5.213196in}{2.993326in}}%
\pgfpathclose%
\pgfusepath{fill}%
\end{pgfscope}%
\begin{pgfscope}%
\pgfpathrectangle{\pgfqpoint{1.150000in}{0.150000in}}{\pgfqpoint{5.700000in}{5.700000in}}%
\pgfusepath{clip}%
\pgfsetbuttcap%
\pgfsetroundjoin%
\definecolor{currentfill}{rgb}{0.280267,0.073417,0.397163}%
\pgfsetfillcolor{currentfill}%
\pgfsetfillopacity{0.800000}%
\pgfsetlinewidth{0.000000pt}%
\definecolor{currentstroke}{rgb}{0.000000,0.000000,0.000000}%
\pgfsetstrokecolor{currentstroke}%
\pgfsetdash{}{0pt}%
\pgfpathmoveto{\pgfqpoint{3.689949in}{1.813980in}}%
\pgfpathlineto{\pgfqpoint{3.703662in}{1.813890in}}%
\pgfpathlineto{\pgfqpoint{3.717381in}{1.813997in}}%
\pgfpathlineto{\pgfqpoint{3.731108in}{1.814298in}}%
\pgfpathlineto{\pgfqpoint{3.744842in}{1.814795in}}%
\pgfpathlineto{\pgfqpoint{3.752937in}{1.825966in}}%
\pgfpathlineto{\pgfqpoint{3.761027in}{1.837143in}}%
\pgfpathlineto{\pgfqpoint{3.769112in}{1.848321in}}%
\pgfpathlineto{\pgfqpoint{3.777191in}{1.859500in}}%
\pgfpathlineto{\pgfqpoint{3.763466in}{1.858726in}}%
\pgfpathlineto{\pgfqpoint{3.749749in}{1.858148in}}%
\pgfpathlineto{\pgfqpoint{3.736038in}{1.857765in}}%
\pgfpathlineto{\pgfqpoint{3.722335in}{1.857577in}}%
\pgfpathlineto{\pgfqpoint{3.714247in}{1.846664in}}%
\pgfpathlineto{\pgfqpoint{3.706153in}{1.835758in}}%
\pgfpathlineto{\pgfqpoint{3.698054in}{1.824862in}}%
\pgfpathlineto{\pgfqpoint{3.689949in}{1.813980in}}%
\pgfpathclose%
\pgfusepath{fill}%
\end{pgfscope}%
\begin{pgfscope}%
\pgfpathrectangle{\pgfqpoint{1.150000in}{0.150000in}}{\pgfqpoint{5.700000in}{5.700000in}}%
\pgfusepath{clip}%
\pgfsetbuttcap%
\pgfsetroundjoin%
\definecolor{currentfill}{rgb}{0.180629,0.429975,0.557282}%
\pgfsetfillcolor{currentfill}%
\pgfsetfillopacity{0.800000}%
\pgfsetlinewidth{0.000000pt}%
\definecolor{currentstroke}{rgb}{0.000000,0.000000,0.000000}%
\pgfsetstrokecolor{currentstroke}%
\pgfsetdash{}{0pt}%
\pgfpathmoveto{\pgfqpoint{4.832385in}{2.694382in}}%
\pgfpathlineto{\pgfqpoint{4.846580in}{2.704099in}}%
\pgfpathlineto{\pgfqpoint{4.860793in}{2.714000in}}%
\pgfpathlineto{\pgfqpoint{4.875021in}{2.724084in}}%
\pgfpathlineto{\pgfqpoint{4.889267in}{2.734351in}}%
\pgfpathlineto{\pgfqpoint{4.896953in}{2.741977in}}%
\pgfpathlineto{\pgfqpoint{4.904632in}{2.749510in}}%
\pgfpathlineto{\pgfqpoint{4.912303in}{2.756953in}}%
\pgfpathlineto{\pgfqpoint{4.919968in}{2.764310in}}%
\pgfpathlineto{\pgfqpoint{4.905732in}{2.754256in}}%
\pgfpathlineto{\pgfqpoint{4.891513in}{2.744385in}}%
\pgfpathlineto{\pgfqpoint{4.877311in}{2.734696in}}%
\pgfpathlineto{\pgfqpoint{4.863124in}{2.725189in}}%
\pgfpathlineto{\pgfqpoint{4.855450in}{2.717608in}}%
\pgfpathlineto{\pgfqpoint{4.847768in}{2.709948in}}%
\pgfpathlineto{\pgfqpoint{4.840080in}{2.702207in}}%
\pgfpathlineto{\pgfqpoint{4.832385in}{2.694382in}}%
\pgfpathclose%
\pgfusepath{fill}%
\end{pgfscope}%
\begin{pgfscope}%
\pgfpathrectangle{\pgfqpoint{1.150000in}{0.150000in}}{\pgfqpoint{5.700000in}{5.700000in}}%
\pgfusepath{clip}%
\pgfsetbuttcap%
\pgfsetroundjoin%
\definecolor{currentfill}{rgb}{0.282327,0.094955,0.417331}%
\pgfsetfillcolor{currentfill}%
\pgfsetfillopacity{0.800000}%
\pgfsetlinewidth{0.000000pt}%
\definecolor{currentstroke}{rgb}{0.000000,0.000000,0.000000}%
\pgfsetstrokecolor{currentstroke}%
\pgfsetdash{}{0pt}%
\pgfpathmoveto{\pgfqpoint{3.777191in}{1.859500in}}%
\pgfpathlineto{\pgfqpoint{3.790924in}{1.860468in}}%
\pgfpathlineto{\pgfqpoint{3.804665in}{1.861629in}}%
\pgfpathlineto{\pgfqpoint{3.818414in}{1.862984in}}%
\pgfpathlineto{\pgfqpoint{3.832172in}{1.864532in}}%
\pgfpathlineto{\pgfqpoint{3.840238in}{1.875965in}}%
\pgfpathlineto{\pgfqpoint{3.848300in}{1.887386in}}%
\pgfpathlineto{\pgfqpoint{3.856357in}{1.898792in}}%
\pgfpathlineto{\pgfqpoint{3.864409in}{1.910182in}}%
\pgfpathlineto{\pgfqpoint{3.850659in}{1.908389in}}%
\pgfpathlineto{\pgfqpoint{3.836917in}{1.906788in}}%
\pgfpathlineto{\pgfqpoint{3.823184in}{1.905381in}}%
\pgfpathlineto{\pgfqpoint{3.809459in}{1.904168in}}%
\pgfpathlineto{\pgfqpoint{3.801400in}{1.893012in}}%
\pgfpathlineto{\pgfqpoint{3.793335in}{1.881847in}}%
\pgfpathlineto{\pgfqpoint{3.785266in}{1.870676in}}%
\pgfpathlineto{\pgfqpoint{3.777191in}{1.859500in}}%
\pgfpathclose%
\pgfusepath{fill}%
\end{pgfscope}%
\begin{pgfscope}%
\pgfpathrectangle{\pgfqpoint{1.150000in}{0.150000in}}{\pgfqpoint{5.700000in}{5.700000in}}%
\pgfusepath{clip}%
\pgfsetbuttcap%
\pgfsetroundjoin%
\definecolor{currentfill}{rgb}{0.208030,0.718701,0.472873}%
\pgfsetfillcolor{currentfill}%
\pgfsetfillopacity{0.800000}%
\pgfsetlinewidth{0.000000pt}%
\definecolor{currentstroke}{rgb}{0.000000,0.000000,0.000000}%
\pgfsetstrokecolor{currentstroke}%
\pgfsetdash{}{0pt}%
\pgfpathmoveto{\pgfqpoint{6.118193in}{3.595150in}}%
\pgfpathlineto{\pgfqpoint{6.133098in}{3.607393in}}%
\pgfpathlineto{\pgfqpoint{6.148026in}{3.619812in}}%
\pgfpathlineto{\pgfqpoint{6.162976in}{3.632406in}}%
\pgfpathlineto{\pgfqpoint{6.169917in}{3.632905in}}%
\pgfpathlineto{\pgfqpoint{6.176857in}{3.633543in}}%
\pgfpathlineto{\pgfqpoint{6.183795in}{3.634330in}}%
\pgfpathlineto{\pgfqpoint{6.190733in}{3.635273in}}%
\pgfpathlineto{\pgfqpoint{6.175827in}{3.623438in}}%
\pgfpathlineto{\pgfqpoint{6.160943in}{3.611777in}}%
\pgfpathlineto{\pgfqpoint{6.146080in}{3.600290in}}%
\pgfpathlineto{\pgfqpoint{6.139109in}{3.598771in}}%
\pgfpathlineto{\pgfqpoint{6.132138in}{3.597413in}}%
\pgfpathlineto{\pgfqpoint{6.125166in}{3.596209in}}%
\pgfpathlineto{\pgfqpoint{6.118193in}{3.595150in}}%
\pgfpathclose%
\pgfusepath{fill}%
\end{pgfscope}%
\begin{pgfscope}%
\pgfpathrectangle{\pgfqpoint{1.150000in}{0.150000in}}{\pgfqpoint{5.700000in}{5.700000in}}%
\pgfusepath{clip}%
\pgfsetbuttcap%
\pgfsetroundjoin%
\definecolor{currentfill}{rgb}{0.277018,0.050344,0.375715}%
\pgfsetfillcolor{currentfill}%
\pgfsetfillopacity{0.800000}%
\pgfsetlinewidth{0.000000pt}%
\definecolor{currentstroke}{rgb}{0.000000,0.000000,0.000000}%
\pgfsetstrokecolor{currentstroke}%
\pgfsetdash{}{0pt}%
\pgfpathmoveto{\pgfqpoint{3.602651in}{1.774186in}}%
\pgfpathlineto{\pgfqpoint{3.616348in}{1.772999in}}%
\pgfpathlineto{\pgfqpoint{3.630051in}{1.772011in}}%
\pgfpathlineto{\pgfqpoint{3.643761in}{1.771220in}}%
\pgfpathlineto{\pgfqpoint{3.657477in}{1.770626in}}%
\pgfpathlineto{\pgfqpoint{3.665603in}{1.781432in}}%
\pgfpathlineto{\pgfqpoint{3.673724in}{1.792262in}}%
\pgfpathlineto{\pgfqpoint{3.681839in}{1.803112in}}%
\pgfpathlineto{\pgfqpoint{3.689949in}{1.813980in}}%
\pgfpathlineto{\pgfqpoint{3.676244in}{1.814265in}}%
\pgfpathlineto{\pgfqpoint{3.662545in}{1.814748in}}%
\pgfpathlineto{\pgfqpoint{3.648852in}{1.815428in}}%
\pgfpathlineto{\pgfqpoint{3.635166in}{1.816307in}}%
\pgfpathlineto{\pgfqpoint{3.627046in}{1.805735in}}%
\pgfpathlineto{\pgfqpoint{3.618920in}{1.795189in}}%
\pgfpathlineto{\pgfqpoint{3.610788in}{1.784672in}}%
\pgfpathlineto{\pgfqpoint{3.602651in}{1.774186in}}%
\pgfpathclose%
\pgfusepath{fill}%
\end{pgfscope}%
\begin{pgfscope}%
\pgfpathrectangle{\pgfqpoint{1.150000in}{0.150000in}}{\pgfqpoint{5.700000in}{5.700000in}}%
\pgfusepath{clip}%
\pgfsetbuttcap%
\pgfsetroundjoin%
\definecolor{currentfill}{rgb}{0.162016,0.687316,0.499129}%
\pgfsetfillcolor{currentfill}%
\pgfsetfillopacity{0.800000}%
\pgfsetlinewidth{0.000000pt}%
\definecolor{currentstroke}{rgb}{0.000000,0.000000,0.000000}%
\pgfsetstrokecolor{currentstroke}%
\pgfsetdash{}{0pt}%
\pgfpathmoveto{\pgfqpoint{5.943208in}{3.487996in}}%
\pgfpathlineto{\pgfqpoint{5.958021in}{3.500251in}}%
\pgfpathlineto{\pgfqpoint{5.972855in}{3.512681in}}%
\pgfpathlineto{\pgfqpoint{5.987711in}{3.525288in}}%
\pgfpathlineto{\pgfqpoint{6.002588in}{3.538072in}}%
\pgfpathlineto{\pgfqpoint{6.009623in}{3.538931in}}%
\pgfpathlineto{\pgfqpoint{6.016655in}{3.539882in}}%
\pgfpathlineto{\pgfqpoint{6.023683in}{3.540931in}}%
\pgfpathlineto{\pgfqpoint{6.030708in}{3.542087in}}%
\pgfpathlineto{\pgfqpoint{6.015870in}{3.529996in}}%
\pgfpathlineto{\pgfqpoint{6.001052in}{3.518081in}}%
\pgfpathlineto{\pgfqpoint{5.986256in}{3.506340in}}%
\pgfpathlineto{\pgfqpoint{5.971481in}{3.494775in}}%
\pgfpathlineto{\pgfqpoint{5.964417in}{3.492917in}}%
\pgfpathlineto{\pgfqpoint{5.957351in}{3.491173in}}%
\pgfpathlineto{\pgfqpoint{5.950281in}{3.489536in}}%
\pgfpathlineto{\pgfqpoint{5.943208in}{3.487996in}}%
\pgfpathclose%
\pgfusepath{fill}%
\end{pgfscope}%
\begin{pgfscope}%
\pgfpathrectangle{\pgfqpoint{1.150000in}{0.150000in}}{\pgfqpoint{5.700000in}{5.700000in}}%
\pgfusepath{clip}%
\pgfsetbuttcap%
\pgfsetroundjoin%
\definecolor{currentfill}{rgb}{0.220057,0.343307,0.549413}%
\pgfsetfillcolor{currentfill}%
\pgfsetfillopacity{0.800000}%
\pgfsetlinewidth{0.000000pt}%
\definecolor{currentstroke}{rgb}{0.000000,0.000000,0.000000}%
\pgfsetstrokecolor{currentstroke}%
\pgfsetdash{}{0pt}%
\pgfpathmoveto{\pgfqpoint{4.538852in}{2.447505in}}%
\pgfpathlineto{\pgfqpoint{4.552892in}{2.455528in}}%
\pgfpathlineto{\pgfqpoint{4.566947in}{2.463735in}}%
\pgfpathlineto{\pgfqpoint{4.581017in}{2.472127in}}%
\pgfpathlineto{\pgfqpoint{4.595101in}{2.480704in}}%
\pgfpathlineto{\pgfqpoint{4.602916in}{2.490291in}}%
\pgfpathlineto{\pgfqpoint{4.610726in}{2.499780in}}%
\pgfpathlineto{\pgfqpoint{4.618529in}{2.509173in}}%
\pgfpathlineto{\pgfqpoint{4.626325in}{2.518471in}}%
\pgfpathlineto{\pgfqpoint{4.612247in}{2.509972in}}%
\pgfpathlineto{\pgfqpoint{4.598183in}{2.501657in}}%
\pgfpathlineto{\pgfqpoint{4.584134in}{2.493527in}}%
\pgfpathlineto{\pgfqpoint{4.570100in}{2.485582in}}%
\pgfpathlineto{\pgfqpoint{4.562297in}{2.476194in}}%
\pgfpathlineto{\pgfqpoint{4.554488in}{2.466720in}}%
\pgfpathlineto{\pgfqpoint{4.546673in}{2.457157in}}%
\pgfpathlineto{\pgfqpoint{4.538852in}{2.447505in}}%
\pgfpathclose%
\pgfusepath{fill}%
\end{pgfscope}%
\begin{pgfscope}%
\pgfpathrectangle{\pgfqpoint{1.150000in}{0.150000in}}{\pgfqpoint{5.700000in}{5.700000in}}%
\pgfusepath{clip}%
\pgfsetbuttcap%
\pgfsetroundjoin%
\definecolor{currentfill}{rgb}{0.260571,0.246922,0.522828}%
\pgfsetfillcolor{currentfill}%
\pgfsetfillopacity{0.800000}%
\pgfsetlinewidth{0.000000pt}%
\definecolor{currentstroke}{rgb}{0.000000,0.000000,0.000000}%
\pgfsetstrokecolor{currentstroke}%
\pgfsetdash{}{0pt}%
\pgfpathmoveto{\pgfqpoint{4.245270in}{2.198705in}}%
\pgfpathlineto{\pgfqpoint{4.259170in}{2.204476in}}%
\pgfpathlineto{\pgfqpoint{4.273082in}{2.210434in}}%
\pgfpathlineto{\pgfqpoint{4.287006in}{2.216579in}}%
\pgfpathlineto{\pgfqpoint{4.300943in}{2.222910in}}%
\pgfpathlineto{\pgfqpoint{4.308865in}{2.233955in}}%
\pgfpathlineto{\pgfqpoint{4.316781in}{2.244919in}}%
\pgfpathlineto{\pgfqpoint{4.324692in}{2.255802in}}%
\pgfpathlineto{\pgfqpoint{4.332597in}{2.266604in}}%
\pgfpathlineto{\pgfqpoint{4.318664in}{2.260218in}}%
\pgfpathlineto{\pgfqpoint{4.304744in}{2.254019in}}%
\pgfpathlineto{\pgfqpoint{4.290836in}{2.248007in}}%
\pgfpathlineto{\pgfqpoint{4.276941in}{2.242182in}}%
\pgfpathlineto{\pgfqpoint{4.269031in}{2.231423in}}%
\pgfpathlineto{\pgfqpoint{4.261116in}{2.220590in}}%
\pgfpathlineto{\pgfqpoint{4.253195in}{2.209684in}}%
\pgfpathlineto{\pgfqpoint{4.245270in}{2.198705in}}%
\pgfpathclose%
\pgfusepath{fill}%
\end{pgfscope}%
\begin{pgfscope}%
\pgfpathrectangle{\pgfqpoint{1.150000in}{0.150000in}}{\pgfqpoint{5.700000in}{5.700000in}}%
\pgfusepath{clip}%
\pgfsetbuttcap%
\pgfsetroundjoin%
\definecolor{currentfill}{rgb}{0.283229,0.120777,0.440584}%
\pgfsetfillcolor{currentfill}%
\pgfsetfillopacity{0.800000}%
\pgfsetlinewidth{0.000000pt}%
\definecolor{currentstroke}{rgb}{0.000000,0.000000,0.000000}%
\pgfsetstrokecolor{currentstroke}%
\pgfsetdash{}{0pt}%
\pgfpathmoveto{\pgfqpoint{3.864409in}{1.910182in}}%
\pgfpathlineto{\pgfqpoint{3.878167in}{1.912168in}}%
\pgfpathlineto{\pgfqpoint{3.891934in}{1.914346in}}%
\pgfpathlineto{\pgfqpoint{3.905711in}{1.916715in}}%
\pgfpathlineto{\pgfqpoint{3.919496in}{1.919276in}}%
\pgfpathlineto{\pgfqpoint{3.927536in}{1.930873in}}%
\pgfpathlineto{\pgfqpoint{3.935572in}{1.942441in}}%
\pgfpathlineto{\pgfqpoint{3.943602in}{1.953979in}}%
\pgfpathlineto{\pgfqpoint{3.951628in}{1.965486in}}%
\pgfpathlineto{\pgfqpoint{3.937848in}{1.962711in}}%
\pgfpathlineto{\pgfqpoint{3.924078in}{1.960127in}}%
\pgfpathlineto{\pgfqpoint{3.910318in}{1.957735in}}%
\pgfpathlineto{\pgfqpoint{3.896566in}{1.955535in}}%
\pgfpathlineto{\pgfqpoint{3.888534in}{1.944231in}}%
\pgfpathlineto{\pgfqpoint{3.880497in}{1.932903in}}%
\pgfpathlineto{\pgfqpoint{3.872455in}{1.921553in}}%
\pgfpathlineto{\pgfqpoint{3.864409in}{1.910182in}}%
\pgfpathclose%
\pgfusepath{fill}%
\end{pgfscope}%
\begin{pgfscope}%
\pgfpathrectangle{\pgfqpoint{1.150000in}{0.150000in}}{\pgfqpoint{5.700000in}{5.700000in}}%
\pgfusepath{clip}%
\pgfsetbuttcap%
\pgfsetroundjoin%
\definecolor{currentfill}{rgb}{0.278791,0.062145,0.386592}%
\pgfsetfillcolor{currentfill}%
\pgfsetfillopacity{0.800000}%
\pgfsetlinewidth{0.000000pt}%
\definecolor{currentstroke}{rgb}{0.000000,0.000000,0.000000}%
\pgfsetstrokecolor{currentstroke}%
\pgfsetdash{}{0pt}%
\pgfpathmoveto{\pgfqpoint{2.944424in}{1.830110in}}%
\pgfpathlineto{\pgfqpoint{2.958153in}{1.819190in}}%
\pgfpathlineto{\pgfqpoint{2.971879in}{1.808505in}}%
\pgfpathlineto{\pgfqpoint{2.985603in}{1.798054in}}%
\pgfpathlineto{\pgfqpoint{2.999325in}{1.787833in}}%
\pgfpathlineto{\pgfqpoint{3.007777in}{1.793142in}}%
\pgfpathlineto{\pgfqpoint{3.016217in}{1.798619in}}%
\pgfpathlineto{\pgfqpoint{3.024648in}{1.804262in}}%
\pgfpathlineto{\pgfqpoint{3.033067in}{1.810064in}}%
\pgfpathlineto{\pgfqpoint{3.019374in}{1.819812in}}%
\pgfpathlineto{\pgfqpoint{3.005679in}{1.829792in}}%
\pgfpathlineto{\pgfqpoint{2.991982in}{1.840005in}}%
\pgfpathlineto{\pgfqpoint{2.978284in}{1.850451in}}%
\pgfpathlineto{\pgfqpoint{2.969835in}{1.845109in}}%
\pgfpathlineto{\pgfqpoint{2.961376in}{1.839935in}}%
\pgfpathlineto{\pgfqpoint{2.952906in}{1.834934in}}%
\pgfpathlineto{\pgfqpoint{2.944424in}{1.830110in}}%
\pgfpathclose%
\pgfusepath{fill}%
\end{pgfscope}%
\begin{pgfscope}%
\pgfpathrectangle{\pgfqpoint{1.150000in}{0.150000in}}{\pgfqpoint{5.700000in}{5.700000in}}%
\pgfusepath{clip}%
\pgfsetbuttcap%
\pgfsetroundjoin%
\definecolor{currentfill}{rgb}{0.269944,0.014625,0.341379}%
\pgfsetfillcolor{currentfill}%
\pgfsetfillopacity{0.800000}%
\pgfsetlinewidth{0.000000pt}%
\definecolor{currentstroke}{rgb}{0.000000,0.000000,0.000000}%
\pgfsetstrokecolor{currentstroke}%
\pgfsetdash{}{0pt}%
\pgfpathmoveto{\pgfqpoint{3.285318in}{1.716225in}}%
\pgfpathlineto{\pgfqpoint{3.298995in}{1.710657in}}%
\pgfpathlineto{\pgfqpoint{3.312674in}{1.705299in}}%
\pgfpathlineto{\pgfqpoint{3.326357in}{1.700151in}}%
\pgfpathlineto{\pgfqpoint{3.340042in}{1.695211in}}%
\pgfpathlineto{\pgfqpoint{3.348302in}{1.703831in}}%
\pgfpathlineto{\pgfqpoint{3.356555in}{1.712546in}}%
\pgfpathlineto{\pgfqpoint{3.364800in}{1.721352in}}%
\pgfpathlineto{\pgfqpoint{3.373038in}{1.730246in}}%
\pgfpathlineto{\pgfqpoint{3.359372in}{1.734782in}}%
\pgfpathlineto{\pgfqpoint{3.345708in}{1.739528in}}%
\pgfpathlineto{\pgfqpoint{3.332048in}{1.744482in}}%
\pgfpathlineto{\pgfqpoint{3.318389in}{1.749647in}}%
\pgfpathlineto{\pgfqpoint{3.310133in}{1.741145in}}%
\pgfpathlineto{\pgfqpoint{3.301869in}{1.732738in}}%
\pgfpathlineto{\pgfqpoint{3.293597in}{1.724430in}}%
\pgfpathlineto{\pgfqpoint{3.285318in}{1.716225in}}%
\pgfpathclose%
\pgfusepath{fill}%
\end{pgfscope}%
\begin{pgfscope}%
\pgfpathrectangle{\pgfqpoint{1.150000in}{0.150000in}}{\pgfqpoint{5.700000in}{5.700000in}}%
\pgfusepath{clip}%
\pgfsetbuttcap%
\pgfsetroundjoin%
\definecolor{currentfill}{rgb}{0.271305,0.019942,0.347269}%
\pgfsetfillcolor{currentfill}%
\pgfsetfillopacity{0.800000}%
\pgfsetlinewidth{0.000000pt}%
\definecolor{currentstroke}{rgb}{0.000000,0.000000,0.000000}%
\pgfsetstrokecolor{currentstroke}%
\pgfsetdash{}{0pt}%
\pgfpathmoveto{\pgfqpoint{3.142584in}{1.740244in}}%
\pgfpathlineto{\pgfqpoint{3.156273in}{1.732519in}}%
\pgfpathlineto{\pgfqpoint{3.169962in}{1.725013in}}%
\pgfpathlineto{\pgfqpoint{3.183652in}{1.717725in}}%
\pgfpathlineto{\pgfqpoint{3.197343in}{1.710653in}}%
\pgfpathlineto{\pgfqpoint{3.205677in}{1.717959in}}%
\pgfpathlineto{\pgfqpoint{3.214003in}{1.725392in}}%
\pgfpathlineto{\pgfqpoint{3.222320in}{1.732948in}}%
\pgfpathlineto{\pgfqpoint{3.230628in}{1.740623in}}%
\pgfpathlineto{\pgfqpoint{3.216960in}{1.747258in}}%
\pgfpathlineto{\pgfqpoint{3.203293in}{1.754110in}}%
\pgfpathlineto{\pgfqpoint{3.189627in}{1.761180in}}%
\pgfpathlineto{\pgfqpoint{3.175961in}{1.768468in}}%
\pgfpathlineto{\pgfqpoint{3.167630in}{1.761218in}}%
\pgfpathlineto{\pgfqpoint{3.159291in}{1.754094in}}%
\pgfpathlineto{\pgfqpoint{3.150942in}{1.747102in}}%
\pgfpathlineto{\pgfqpoint{3.142584in}{1.740244in}}%
\pgfpathclose%
\pgfusepath{fill}%
\end{pgfscope}%
\begin{pgfscope}%
\pgfpathrectangle{\pgfqpoint{1.150000in}{0.150000in}}{\pgfqpoint{5.700000in}{5.700000in}}%
\pgfusepath{clip}%
\pgfsetbuttcap%
\pgfsetroundjoin%
\definecolor{currentfill}{rgb}{0.212395,0.359683,0.551710}%
\pgfsetfillcolor{currentfill}%
\pgfsetfillopacity{0.800000}%
\pgfsetlinewidth{0.000000pt}%
\definecolor{currentstroke}{rgb}{0.000000,0.000000,0.000000}%
\pgfsetstrokecolor{currentstroke}%
\pgfsetdash{}{0pt}%
\pgfpathmoveto{\pgfqpoint{2.334380in}{2.569829in}}%
\pgfpathlineto{\pgfqpoint{2.348461in}{2.546690in}}%
\pgfpathlineto{\pgfqpoint{2.362527in}{2.523882in}}%
\pgfpathlineto{\pgfqpoint{2.376579in}{2.501403in}}%
\pgfpathlineto{\pgfqpoint{2.390618in}{2.479250in}}%
\pgfpathlineto{\pgfqpoint{2.399493in}{2.479090in}}%
\pgfpathlineto{\pgfqpoint{2.408351in}{2.479193in}}%
\pgfpathlineto{\pgfqpoint{2.417191in}{2.479555in}}%
\pgfpathlineto{\pgfqpoint{2.426015in}{2.480171in}}%
\pgfpathlineto{\pgfqpoint{2.412024in}{2.501821in}}%
\pgfpathlineto{\pgfqpoint{2.398020in}{2.523795in}}%
\pgfpathlineto{\pgfqpoint{2.384002in}{2.546097in}}%
\pgfpathlineto{\pgfqpoint{2.369971in}{2.568729in}}%
\pgfpathlineto{\pgfqpoint{2.361100in}{2.568605in}}%
\pgfpathlineto{\pgfqpoint{2.352212in}{2.568743in}}%
\pgfpathlineto{\pgfqpoint{2.343305in}{2.569150in}}%
\pgfpathlineto{\pgfqpoint{2.334380in}{2.569829in}}%
\pgfpathclose%
\pgfusepath{fill}%
\end{pgfscope}%
\begin{pgfscope}%
\pgfpathrectangle{\pgfqpoint{1.150000in}{0.150000in}}{\pgfqpoint{5.700000in}{5.700000in}}%
\pgfusepath{clip}%
\pgfsetbuttcap%
\pgfsetroundjoin%
\definecolor{currentfill}{rgb}{0.185783,0.704891,0.485273}%
\pgfsetfillcolor{currentfill}%
\pgfsetfillopacity{0.800000}%
\pgfsetlinewidth{0.000000pt}%
\definecolor{currentstroke}{rgb}{0.000000,0.000000,0.000000}%
\pgfsetstrokecolor{currentstroke}%
\pgfsetdash{}{0pt}%
\pgfpathmoveto{\pgfqpoint{6.030708in}{3.542087in}}%
\pgfpathlineto{\pgfqpoint{6.045569in}{3.554354in}}%
\pgfpathlineto{\pgfqpoint{6.060450in}{3.566797in}}%
\pgfpathlineto{\pgfqpoint{6.075354in}{3.579416in}}%
\pgfpathlineto{\pgfqpoint{6.090279in}{3.592211in}}%
\pgfpathlineto{\pgfqpoint{6.097261in}{3.592767in}}%
\pgfpathlineto{\pgfqpoint{6.104241in}{3.593436in}}%
\pgfpathlineto{\pgfqpoint{6.111218in}{3.594228in}}%
\pgfpathlineto{\pgfqpoint{6.118193in}{3.595150in}}%
\pgfpathlineto{\pgfqpoint{6.103309in}{3.583081in}}%
\pgfpathlineto{\pgfqpoint{6.088446in}{3.571188in}}%
\pgfpathlineto{\pgfqpoint{6.073605in}{3.559469in}}%
\pgfpathlineto{\pgfqpoint{6.058785in}{3.547925in}}%
\pgfpathlineto{\pgfqpoint{6.051769in}{3.546268in}}%
\pgfpathlineto{\pgfqpoint{6.044751in}{3.544748in}}%
\pgfpathlineto{\pgfqpoint{6.037731in}{3.543357in}}%
\pgfpathlineto{\pgfqpoint{6.030708in}{3.542087in}}%
\pgfpathclose%
\pgfusepath{fill}%
\end{pgfscope}%
\begin{pgfscope}%
\pgfpathrectangle{\pgfqpoint{1.150000in}{0.150000in}}{\pgfqpoint{5.700000in}{5.700000in}}%
\pgfusepath{clip}%
\pgfsetbuttcap%
\pgfsetroundjoin%
\definecolor{currentfill}{rgb}{0.273809,0.031497,0.358853}%
\pgfsetfillcolor{currentfill}%
\pgfsetfillopacity{0.800000}%
\pgfsetlinewidth{0.000000pt}%
\definecolor{currentstroke}{rgb}{0.000000,0.000000,0.000000}%
\pgfsetstrokecolor{currentstroke}%
\pgfsetdash{}{0pt}%
\pgfpathmoveto{\pgfqpoint{3.515261in}{1.740710in}}%
\pgfpathlineto{\pgfqpoint{3.528948in}{1.738385in}}%
\pgfpathlineto{\pgfqpoint{3.542641in}{1.736261in}}%
\pgfpathlineto{\pgfqpoint{3.556340in}{1.734337in}}%
\pgfpathlineto{\pgfqpoint{3.570043in}{1.732612in}}%
\pgfpathlineto{\pgfqpoint{3.578204in}{1.742944in}}%
\pgfpathlineto{\pgfqpoint{3.586359in}{1.753319in}}%
\pgfpathlineto{\pgfqpoint{3.594508in}{1.763734in}}%
\pgfpathlineto{\pgfqpoint{3.602651in}{1.774186in}}%
\pgfpathlineto{\pgfqpoint{3.588959in}{1.775571in}}%
\pgfpathlineto{\pgfqpoint{3.575274in}{1.777156in}}%
\pgfpathlineto{\pgfqpoint{3.561594in}{1.778941in}}%
\pgfpathlineto{\pgfqpoint{3.547919in}{1.780926in}}%
\pgfpathlineto{\pgfqpoint{3.539764in}{1.770802in}}%
\pgfpathlineto{\pgfqpoint{3.531602in}{1.760722in}}%
\pgfpathlineto{\pgfqpoint{3.523434in}{1.750690in}}%
\pgfpathlineto{\pgfqpoint{3.515261in}{1.740710in}}%
\pgfpathclose%
\pgfusepath{fill}%
\end{pgfscope}%
\begin{pgfscope}%
\pgfpathrectangle{\pgfqpoint{1.150000in}{0.150000in}}{\pgfqpoint{5.700000in}{5.700000in}}%
\pgfusepath{clip}%
\pgfsetbuttcap%
\pgfsetroundjoin%
\definecolor{currentfill}{rgb}{0.132444,0.552216,0.553018}%
\pgfsetfillcolor{currentfill}%
\pgfsetfillopacity{0.800000}%
\pgfsetlinewidth{0.000000pt}%
\definecolor{currentstroke}{rgb}{0.000000,0.000000,0.000000}%
\pgfsetstrokecolor{currentstroke}%
\pgfsetdash{}{0pt}%
\pgfpathmoveto{\pgfqpoint{5.300838in}{3.058598in}}%
\pgfpathlineto{\pgfqpoint{5.315305in}{3.070162in}}%
\pgfpathlineto{\pgfqpoint{5.329791in}{3.081907in}}%
\pgfpathlineto{\pgfqpoint{5.344296in}{3.093832in}}%
\pgfpathlineto{\pgfqpoint{5.358820in}{3.105938in}}%
\pgfpathlineto{\pgfqpoint{5.366255in}{3.110234in}}%
\pgfpathlineto{\pgfqpoint{5.373683in}{3.114484in}}%
\pgfpathlineto{\pgfqpoint{5.381103in}{3.118693in}}%
\pgfpathlineto{\pgfqpoint{5.388515in}{3.122864in}}%
\pgfpathlineto{\pgfqpoint{5.374011in}{3.111177in}}%
\pgfpathlineto{\pgfqpoint{5.359526in}{3.099671in}}%
\pgfpathlineto{\pgfqpoint{5.345059in}{3.088344in}}%
\pgfpathlineto{\pgfqpoint{5.330611in}{3.077196in}}%
\pgfpathlineto{\pgfqpoint{5.323179in}{3.072595in}}%
\pgfpathlineto{\pgfqpoint{5.315739in}{3.067965in}}%
\pgfpathlineto{\pgfqpoint{5.308292in}{3.063301in}}%
\pgfpathlineto{\pgfqpoint{5.300838in}{3.058598in}}%
\pgfpathclose%
\pgfusepath{fill}%
\end{pgfscope}%
\begin{pgfscope}%
\pgfpathrectangle{\pgfqpoint{1.150000in}{0.150000in}}{\pgfqpoint{5.700000in}{5.700000in}}%
\pgfusepath{clip}%
\pgfsetbuttcap%
\pgfsetroundjoin%
\definecolor{currentfill}{rgb}{0.282290,0.145912,0.461510}%
\pgfsetfillcolor{currentfill}%
\pgfsetfillopacity{0.800000}%
\pgfsetlinewidth{0.000000pt}%
\definecolor{currentstroke}{rgb}{0.000000,0.000000,0.000000}%
\pgfsetstrokecolor{currentstroke}%
\pgfsetdash{}{0pt}%
\pgfpathmoveto{\pgfqpoint{3.951628in}{1.965486in}}%
\pgfpathlineto{\pgfqpoint{3.965416in}{1.968451in}}%
\pgfpathlineto{\pgfqpoint{3.979215in}{1.971608in}}%
\pgfpathlineto{\pgfqpoint{3.993023in}{1.974954in}}%
\pgfpathlineto{\pgfqpoint{4.006841in}{1.978491in}}%
\pgfpathlineto{\pgfqpoint{4.014856in}{1.990158in}}%
\pgfpathlineto{\pgfqpoint{4.022867in}{2.001781in}}%
\pgfpathlineto{\pgfqpoint{4.030872in}{2.013360in}}%
\pgfpathlineto{\pgfqpoint{4.038872in}{2.024893in}}%
\pgfpathlineto{\pgfqpoint{4.025059in}{2.021174in}}%
\pgfpathlineto{\pgfqpoint{4.011257in}{2.017645in}}%
\pgfpathlineto{\pgfqpoint{3.997464in}{2.014306in}}%
\pgfpathlineto{\pgfqpoint{3.983681in}{2.011158in}}%
\pgfpathlineto{\pgfqpoint{3.975675in}{1.999796in}}%
\pgfpathlineto{\pgfqpoint{3.967664in}{1.988395in}}%
\pgfpathlineto{\pgfqpoint{3.959648in}{1.976958in}}%
\pgfpathlineto{\pgfqpoint{3.951628in}{1.965486in}}%
\pgfpathclose%
\pgfusepath{fill}%
\end{pgfscope}%
\begin{pgfscope}%
\pgfpathrectangle{\pgfqpoint{1.150000in}{0.150000in}}{\pgfqpoint{5.700000in}{5.700000in}}%
\pgfusepath{clip}%
\pgfsetbuttcap%
\pgfsetroundjoin%
\definecolor{currentfill}{rgb}{0.169646,0.456262,0.558030}%
\pgfsetfillcolor{currentfill}%
\pgfsetfillopacity{0.800000}%
\pgfsetlinewidth{0.000000pt}%
\definecolor{currentstroke}{rgb}{0.000000,0.000000,0.000000}%
\pgfsetstrokecolor{currentstroke}%
\pgfsetdash{}{0pt}%
\pgfpathmoveto{\pgfqpoint{4.919968in}{2.764310in}}%
\pgfpathlineto{\pgfqpoint{4.934220in}{2.774547in}}%
\pgfpathlineto{\pgfqpoint{4.948489in}{2.784966in}}%
\pgfpathlineto{\pgfqpoint{4.962776in}{2.795567in}}%
\pgfpathlineto{\pgfqpoint{4.977079in}{2.806352in}}%
\pgfpathlineto{\pgfqpoint{4.984726in}{2.813390in}}%
\pgfpathlineto{\pgfqpoint{4.992365in}{2.820338in}}%
\pgfpathlineto{\pgfqpoint{4.999997in}{2.827200in}}%
\pgfpathlineto{\pgfqpoint{5.007622in}{2.833979in}}%
\pgfpathlineto{\pgfqpoint{4.993329in}{2.823442in}}%
\pgfpathlineto{\pgfqpoint{4.979054in}{2.813087in}}%
\pgfpathlineto{\pgfqpoint{4.964796in}{2.802915in}}%
\pgfpathlineto{\pgfqpoint{4.950555in}{2.792924in}}%
\pgfpathlineto{\pgfqpoint{4.942919in}{2.785887in}}%
\pgfpathlineto{\pgfqpoint{4.935275in}{2.778774in}}%
\pgfpathlineto{\pgfqpoint{4.927625in}{2.771583in}}%
\pgfpathlineto{\pgfqpoint{4.919968in}{2.764310in}}%
\pgfpathclose%
\pgfusepath{fill}%
\end{pgfscope}%
\begin{pgfscope}%
\pgfpathrectangle{\pgfqpoint{1.150000in}{0.150000in}}{\pgfqpoint{5.700000in}{5.700000in}}%
\pgfusepath{clip}%
\pgfsetbuttcap%
\pgfsetroundjoin%
\definecolor{currentfill}{rgb}{0.248629,0.278775,0.534556}%
\pgfsetfillcolor{currentfill}%
\pgfsetfillopacity{0.800000}%
\pgfsetlinewidth{0.000000pt}%
\definecolor{currentstroke}{rgb}{0.000000,0.000000,0.000000}%
\pgfsetstrokecolor{currentstroke}%
\pgfsetdash{}{0pt}%
\pgfpathmoveto{\pgfqpoint{4.332597in}{2.266604in}}%
\pgfpathlineto{\pgfqpoint{4.346543in}{2.273176in}}%
\pgfpathlineto{\pgfqpoint{4.360502in}{2.279934in}}%
\pgfpathlineto{\pgfqpoint{4.374473in}{2.286879in}}%
\pgfpathlineto{\pgfqpoint{4.388458in}{2.294009in}}%
\pgfpathlineto{\pgfqpoint{4.396354in}{2.304764in}}%
\pgfpathlineto{\pgfqpoint{4.404244in}{2.315429in}}%
\pgfpathlineto{\pgfqpoint{4.412129in}{2.326005in}}%
\pgfpathlineto{\pgfqpoint{4.420008in}{2.336493in}}%
\pgfpathlineto{\pgfqpoint{4.406027in}{2.329341in}}%
\pgfpathlineto{\pgfqpoint{4.392059in}{2.322375in}}%
\pgfpathlineto{\pgfqpoint{4.378105in}{2.315595in}}%
\pgfpathlineto{\pgfqpoint{4.364164in}{2.309002in}}%
\pgfpathlineto{\pgfqpoint{4.356281in}{2.298523in}}%
\pgfpathlineto{\pgfqpoint{4.348392in}{2.287964in}}%
\pgfpathlineto{\pgfqpoint{4.340497in}{2.277325in}}%
\pgfpathlineto{\pgfqpoint{4.332597in}{2.266604in}}%
\pgfpathclose%
\pgfusepath{fill}%
\end{pgfscope}%
\begin{pgfscope}%
\pgfpathrectangle{\pgfqpoint{1.150000in}{0.150000in}}{\pgfqpoint{5.700000in}{5.700000in}}%
\pgfusepath{clip}%
\pgfsetbuttcap%
\pgfsetroundjoin%
\definecolor{currentfill}{rgb}{0.206756,0.371758,0.553117}%
\pgfsetfillcolor{currentfill}%
\pgfsetfillopacity{0.800000}%
\pgfsetlinewidth{0.000000pt}%
\definecolor{currentstroke}{rgb}{0.000000,0.000000,0.000000}%
\pgfsetstrokecolor{currentstroke}%
\pgfsetdash{}{0pt}%
\pgfpathmoveto{\pgfqpoint{4.626325in}{2.518471in}}%
\pgfpathlineto{\pgfqpoint{4.640419in}{2.527154in}}%
\pgfpathlineto{\pgfqpoint{4.654527in}{2.536021in}}%
\pgfpathlineto{\pgfqpoint{4.668651in}{2.545073in}}%
\pgfpathlineto{\pgfqpoint{4.682790in}{2.554309in}}%
\pgfpathlineto{\pgfqpoint{4.690574in}{2.563414in}}%
\pgfpathlineto{\pgfqpoint{4.698351in}{2.572419in}}%
\pgfpathlineto{\pgfqpoint{4.706122in}{2.581326in}}%
\pgfpathlineto{\pgfqpoint{4.713886in}{2.590135in}}%
\pgfpathlineto{\pgfqpoint{4.699754in}{2.581011in}}%
\pgfpathlineto{\pgfqpoint{4.685637in}{2.572071in}}%
\pgfpathlineto{\pgfqpoint{4.671535in}{2.563314in}}%
\pgfpathlineto{\pgfqpoint{4.657448in}{2.554742in}}%
\pgfpathlineto{\pgfqpoint{4.649677in}{2.545809in}}%
\pgfpathlineto{\pgfqpoint{4.641899in}{2.536787in}}%
\pgfpathlineto{\pgfqpoint{4.634115in}{2.527675in}}%
\pgfpathlineto{\pgfqpoint{4.626325in}{2.518471in}}%
\pgfpathclose%
\pgfusepath{fill}%
\end{pgfscope}%
\begin{pgfscope}%
\pgfpathrectangle{\pgfqpoint{1.150000in}{0.150000in}}{\pgfqpoint{5.700000in}{5.700000in}}%
\pgfusepath{clip}%
\pgfsetbuttcap%
\pgfsetroundjoin%
\definecolor{currentfill}{rgb}{0.271305,0.019942,0.347269}%
\pgfsetfillcolor{currentfill}%
\pgfsetfillopacity{0.800000}%
\pgfsetlinewidth{0.000000pt}%
\definecolor{currentstroke}{rgb}{0.000000,0.000000,0.000000}%
\pgfsetstrokecolor{currentstroke}%
\pgfsetdash{}{0pt}%
\pgfpathmoveto{\pgfqpoint{3.427739in}{1.714169in}}%
\pgfpathlineto{\pgfqpoint{3.441424in}{1.710663in}}%
\pgfpathlineto{\pgfqpoint{3.455112in}{1.707361in}}%
\pgfpathlineto{\pgfqpoint{3.468805in}{1.704261in}}%
\pgfpathlineto{\pgfqpoint{3.482503in}{1.701364in}}%
\pgfpathlineto{\pgfqpoint{3.490702in}{1.711107in}}%
\pgfpathlineto{\pgfqpoint{3.498894in}{1.720915in}}%
\pgfpathlineto{\pgfqpoint{3.507081in}{1.730784in}}%
\pgfpathlineto{\pgfqpoint{3.515261in}{1.740710in}}%
\pgfpathlineto{\pgfqpoint{3.501578in}{1.743236in}}%
\pgfpathlineto{\pgfqpoint{3.487900in}{1.745965in}}%
\pgfpathlineto{\pgfqpoint{3.474226in}{1.748896in}}%
\pgfpathlineto{\pgfqpoint{3.460557in}{1.752031in}}%
\pgfpathlineto{\pgfqpoint{3.452362in}{1.742464in}}%
\pgfpathlineto{\pgfqpoint{3.444161in}{1.732963in}}%
\pgfpathlineto{\pgfqpoint{3.435954in}{1.723530in}}%
\pgfpathlineto{\pgfqpoint{3.427739in}{1.714169in}}%
\pgfpathclose%
\pgfusepath{fill}%
\end{pgfscope}%
\begin{pgfscope}%
\pgfpathrectangle{\pgfqpoint{1.150000in}{0.150000in}}{\pgfqpoint{5.700000in}{5.700000in}}%
\pgfusepath{clip}%
\pgfsetbuttcap%
\pgfsetroundjoin%
\definecolor{currentfill}{rgb}{0.277018,0.050344,0.375715}%
\pgfsetfillcolor{currentfill}%
\pgfsetfillopacity{0.800000}%
\pgfsetlinewidth{0.000000pt}%
\definecolor{currentstroke}{rgb}{0.000000,0.000000,0.000000}%
\pgfsetstrokecolor{currentstroke}%
\pgfsetdash{}{0pt}%
\pgfpathmoveto{\pgfqpoint{2.999325in}{1.787833in}}%
\pgfpathlineto{\pgfqpoint{3.013045in}{1.777843in}}%
\pgfpathlineto{\pgfqpoint{3.026764in}{1.768082in}}%
\pgfpathlineto{\pgfqpoint{3.040482in}{1.758549in}}%
\pgfpathlineto{\pgfqpoint{3.054198in}{1.749242in}}%
\pgfpathlineto{\pgfqpoint{3.062622in}{1.755033in}}%
\pgfpathlineto{\pgfqpoint{3.071034in}{1.760986in}}%
\pgfpathlineto{\pgfqpoint{3.079437in}{1.767095in}}%
\pgfpathlineto{\pgfqpoint{3.087830in}{1.773355in}}%
\pgfpathlineto{\pgfqpoint{3.074141in}{1.782192in}}%
\pgfpathlineto{\pgfqpoint{3.060450in}{1.791255in}}%
\pgfpathlineto{\pgfqpoint{3.046760in}{1.800545in}}%
\pgfpathlineto{\pgfqpoint{3.033067in}{1.810064in}}%
\pgfpathlineto{\pgfqpoint{3.024648in}{1.804262in}}%
\pgfpathlineto{\pgfqpoint{3.016217in}{1.798619in}}%
\pgfpathlineto{\pgfqpoint{3.007777in}{1.793142in}}%
\pgfpathlineto{\pgfqpoint{2.999325in}{1.787833in}}%
\pgfpathclose%
\pgfusepath{fill}%
\end{pgfscope}%
\begin{pgfscope}%
\pgfpathrectangle{\pgfqpoint{1.150000in}{0.150000in}}{\pgfqpoint{5.700000in}{5.700000in}}%
\pgfusepath{clip}%
\pgfsetbuttcap%
\pgfsetroundjoin%
\definecolor{currentfill}{rgb}{0.278826,0.175490,0.483397}%
\pgfsetfillcolor{currentfill}%
\pgfsetfillopacity{0.800000}%
\pgfsetlinewidth{0.000000pt}%
\definecolor{currentstroke}{rgb}{0.000000,0.000000,0.000000}%
\pgfsetstrokecolor{currentstroke}%
\pgfsetdash{}{0pt}%
\pgfpathmoveto{\pgfqpoint{4.038872in}{2.024893in}}%
\pgfpathlineto{\pgfqpoint{4.052696in}{2.028802in}}%
\pgfpathlineto{\pgfqpoint{4.066529in}{2.032900in}}%
\pgfpathlineto{\pgfqpoint{4.080374in}{2.037186in}}%
\pgfpathlineto{\pgfqpoint{4.094229in}{2.041662in}}%
\pgfpathlineto{\pgfqpoint{4.102220in}{2.053310in}}%
\pgfpathlineto{\pgfqpoint{4.110206in}{2.064902in}}%
\pgfpathlineto{\pgfqpoint{4.118187in}{2.076436in}}%
\pgfpathlineto{\pgfqpoint{4.126163in}{2.087912in}}%
\pgfpathlineto{\pgfqpoint{4.112312in}{2.083285in}}%
\pgfpathlineto{\pgfqpoint{4.098472in}{2.078848in}}%
\pgfpathlineto{\pgfqpoint{4.084643in}{2.074599in}}%
\pgfpathlineto{\pgfqpoint{4.070825in}{2.070540in}}%
\pgfpathlineto{\pgfqpoint{4.062844in}{2.059204in}}%
\pgfpathlineto{\pgfqpoint{4.054858in}{2.047816in}}%
\pgfpathlineto{\pgfqpoint{4.046868in}{2.036379in}}%
\pgfpathlineto{\pgfqpoint{4.038872in}{2.024893in}}%
\pgfpathclose%
\pgfusepath{fill}%
\end{pgfscope}%
\begin{pgfscope}%
\pgfpathrectangle{\pgfqpoint{1.150000in}{0.150000in}}{\pgfqpoint{5.700000in}{5.700000in}}%
\pgfusepath{clip}%
\pgfsetbuttcap%
\pgfsetroundjoin%
\definecolor{currentfill}{rgb}{0.125394,0.574318,0.549086}%
\pgfsetfillcolor{currentfill}%
\pgfsetfillopacity{0.800000}%
\pgfsetlinewidth{0.000000pt}%
\definecolor{currentstroke}{rgb}{0.000000,0.000000,0.000000}%
\pgfsetstrokecolor{currentstroke}%
\pgfsetdash{}{0pt}%
\pgfpathmoveto{\pgfqpoint{5.388515in}{3.122864in}}%
\pgfpathlineto{\pgfqpoint{5.403039in}{3.134730in}}%
\pgfpathlineto{\pgfqpoint{5.417582in}{3.146777in}}%
\pgfpathlineto{\pgfqpoint{5.432145in}{3.159004in}}%
\pgfpathlineto{\pgfqpoint{5.446727in}{3.171411in}}%
\pgfpathlineto{\pgfqpoint{5.454111in}{3.175108in}}%
\pgfpathlineto{\pgfqpoint{5.461488in}{3.178771in}}%
\pgfpathlineto{\pgfqpoint{5.468858in}{3.182404in}}%
\pgfpathlineto{\pgfqpoint{5.476220in}{3.186011in}}%
\pgfpathlineto{\pgfqpoint{5.461660in}{3.174059in}}%
\pgfpathlineto{\pgfqpoint{5.447119in}{3.162285in}}%
\pgfpathlineto{\pgfqpoint{5.432597in}{3.150691in}}%
\pgfpathlineto{\pgfqpoint{5.418095in}{3.139276in}}%
\pgfpathlineto{\pgfqpoint{5.410710in}{3.135204in}}%
\pgfpathlineto{\pgfqpoint{5.403319in}{3.131115in}}%
\pgfpathlineto{\pgfqpoint{5.395921in}{3.127003in}}%
\pgfpathlineto{\pgfqpoint{5.388515in}{3.122864in}}%
\pgfpathclose%
\pgfusepath{fill}%
\end{pgfscope}%
\begin{pgfscope}%
\pgfpathrectangle{\pgfqpoint{1.150000in}{0.150000in}}{\pgfqpoint{5.700000in}{5.700000in}}%
\pgfusepath{clip}%
\pgfsetbuttcap%
\pgfsetroundjoin%
\definecolor{currentfill}{rgb}{0.195860,0.395433,0.555276}%
\pgfsetfillcolor{currentfill}%
\pgfsetfillopacity{0.800000}%
\pgfsetlinewidth{0.000000pt}%
\definecolor{currentstroke}{rgb}{0.000000,0.000000,0.000000}%
\pgfsetstrokecolor{currentstroke}%
\pgfsetdash{}{0pt}%
\pgfpathmoveto{\pgfqpoint{2.277906in}{2.665769in}}%
\pgfpathlineto{\pgfqpoint{2.292048in}{2.641270in}}%
\pgfpathlineto{\pgfqpoint{2.306174in}{2.617116in}}%
\pgfpathlineto{\pgfqpoint{2.320284in}{2.593303in}}%
\pgfpathlineto{\pgfqpoint{2.334380in}{2.569829in}}%
\pgfpathlineto{\pgfqpoint{2.343305in}{2.569150in}}%
\pgfpathlineto{\pgfqpoint{2.352212in}{2.568743in}}%
\pgfpathlineto{\pgfqpoint{2.361100in}{2.568605in}}%
\pgfpathlineto{\pgfqpoint{2.369971in}{2.568729in}}%
\pgfpathlineto{\pgfqpoint{2.355925in}{2.591695in}}%
\pgfpathlineto{\pgfqpoint{2.341865in}{2.614998in}}%
\pgfpathlineto{\pgfqpoint{2.327789in}{2.638641in}}%
\pgfpathlineto{\pgfqpoint{2.313699in}{2.662629in}}%
\pgfpathlineto{\pgfqpoint{2.304779in}{2.663001in}}%
\pgfpathlineto{\pgfqpoint{2.295840in}{2.663645in}}%
\pgfpathlineto{\pgfqpoint{2.286883in}{2.664566in}}%
\pgfpathlineto{\pgfqpoint{2.277906in}{2.665769in}}%
\pgfpathclose%
\pgfusepath{fill}%
\end{pgfscope}%
\begin{pgfscope}%
\pgfpathrectangle{\pgfqpoint{1.150000in}{0.150000in}}{\pgfqpoint{5.700000in}{5.700000in}}%
\pgfusepath{clip}%
\pgfsetbuttcap%
\pgfsetroundjoin%
\definecolor{currentfill}{rgb}{0.278012,0.180367,0.486697}%
\pgfsetfillcolor{currentfill}%
\pgfsetfillopacity{0.800000}%
\pgfsetlinewidth{0.000000pt}%
\definecolor{currentstroke}{rgb}{0.000000,0.000000,0.000000}%
\pgfsetstrokecolor{currentstroke}%
\pgfsetdash{}{0pt}%
\pgfpathmoveto{\pgfqpoint{2.634238in}{2.092979in}}%
\pgfpathlineto{\pgfqpoint{2.648102in}{2.076509in}}%
\pgfpathlineto{\pgfqpoint{2.661959in}{2.060309in}}%
\pgfpathlineto{\pgfqpoint{2.675809in}{2.044378in}}%
\pgfpathlineto{\pgfqpoint{2.689651in}{2.028712in}}%
\pgfpathlineto{\pgfqpoint{2.698327in}{2.030685in}}%
\pgfpathlineto{\pgfqpoint{2.706987in}{2.032890in}}%
\pgfpathlineto{\pgfqpoint{2.715633in}{2.035319in}}%
\pgfpathlineto{\pgfqpoint{2.724265in}{2.037970in}}%
\pgfpathlineto{\pgfqpoint{2.710462in}{2.053119in}}%
\pgfpathlineto{\pgfqpoint{2.696652in}{2.068534in}}%
\pgfpathlineto{\pgfqpoint{2.682835in}{2.084216in}}%
\pgfpathlineto{\pgfqpoint{2.669012in}{2.100168in}}%
\pgfpathlineto{\pgfqpoint{2.660341in}{2.098022in}}%
\pgfpathlineto{\pgfqpoint{2.651655in}{2.096105in}}%
\pgfpathlineto{\pgfqpoint{2.642954in}{2.094422in}}%
\pgfpathlineto{\pgfqpoint{2.634238in}{2.092979in}}%
\pgfpathclose%
\pgfusepath{fill}%
\end{pgfscope}%
\begin{pgfscope}%
\pgfpathrectangle{\pgfqpoint{1.150000in}{0.150000in}}{\pgfqpoint{5.700000in}{5.700000in}}%
\pgfusepath{clip}%
\pgfsetbuttcap%
\pgfsetroundjoin%
\definecolor{currentfill}{rgb}{0.159194,0.482237,0.558073}%
\pgfsetfillcolor{currentfill}%
\pgfsetfillopacity{0.800000}%
\pgfsetlinewidth{0.000000pt}%
\definecolor{currentstroke}{rgb}{0.000000,0.000000,0.000000}%
\pgfsetstrokecolor{currentstroke}%
\pgfsetdash{}{0pt}%
\pgfpathmoveto{\pgfqpoint{5.007622in}{2.833979in}}%
\pgfpathlineto{\pgfqpoint{5.021932in}{2.844698in}}%
\pgfpathlineto{\pgfqpoint{5.036259in}{2.855599in}}%
\pgfpathlineto{\pgfqpoint{5.050604in}{2.866683in}}%
\pgfpathlineto{\pgfqpoint{5.064966in}{2.877948in}}%
\pgfpathlineto{\pgfqpoint{5.072571in}{2.884377in}}%
\pgfpathlineto{\pgfqpoint{5.080169in}{2.890721in}}%
\pgfpathlineto{\pgfqpoint{5.087759in}{2.896983in}}%
\pgfpathlineto{\pgfqpoint{5.095341in}{2.903166in}}%
\pgfpathlineto{\pgfqpoint{5.080991in}{2.892182in}}%
\pgfpathlineto{\pgfqpoint{5.066659in}{2.881380in}}%
\pgfpathlineto{\pgfqpoint{5.052345in}{2.870760in}}%
\pgfpathlineto{\pgfqpoint{5.038047in}{2.860322in}}%
\pgfpathlineto{\pgfqpoint{5.030452in}{2.853845in}}%
\pgfpathlineto{\pgfqpoint{5.022849in}{2.847298in}}%
\pgfpathlineto{\pgfqpoint{5.015239in}{2.840677in}}%
\pgfpathlineto{\pgfqpoint{5.007622in}{2.833979in}}%
\pgfpathclose%
\pgfusepath{fill}%
\end{pgfscope}%
\begin{pgfscope}%
\pgfpathrectangle{\pgfqpoint{1.150000in}{0.150000in}}{\pgfqpoint{5.700000in}{5.700000in}}%
\pgfusepath{clip}%
\pgfsetbuttcap%
\pgfsetroundjoin%
\definecolor{currentfill}{rgb}{0.271828,0.209303,0.504434}%
\pgfsetfillcolor{currentfill}%
\pgfsetfillopacity{0.800000}%
\pgfsetlinewidth{0.000000pt}%
\definecolor{currentstroke}{rgb}{0.000000,0.000000,0.000000}%
\pgfsetstrokecolor{currentstroke}%
\pgfsetdash{}{0pt}%
\pgfpathmoveto{\pgfqpoint{2.578700in}{2.161600in}}%
\pgfpathlineto{\pgfqpoint{2.592597in}{2.144029in}}%
\pgfpathlineto{\pgfqpoint{2.606486in}{2.126736in}}%
\pgfpathlineto{\pgfqpoint{2.620366in}{2.109720in}}%
\pgfpathlineto{\pgfqpoint{2.634238in}{2.092979in}}%
\pgfpathlineto{\pgfqpoint{2.642954in}{2.094422in}}%
\pgfpathlineto{\pgfqpoint{2.651655in}{2.096105in}}%
\pgfpathlineto{\pgfqpoint{2.660341in}{2.098022in}}%
\pgfpathlineto{\pgfqpoint{2.669012in}{2.100168in}}%
\pgfpathlineto{\pgfqpoint{2.655180in}{2.116390in}}%
\pgfpathlineto{\pgfqpoint{2.641342in}{2.132886in}}%
\pgfpathlineto{\pgfqpoint{2.627495in}{2.149657in}}%
\pgfpathlineto{\pgfqpoint{2.613640in}{2.166707in}}%
\pgfpathlineto{\pgfqpoint{2.604928in}{2.165068in}}%
\pgfpathlineto{\pgfqpoint{2.596201in}{2.163667in}}%
\pgfpathlineto{\pgfqpoint{2.587458in}{2.162509in}}%
\pgfpathlineto{\pgfqpoint{2.578700in}{2.161600in}}%
\pgfpathclose%
\pgfusepath{fill}%
\end{pgfscope}%
\begin{pgfscope}%
\pgfpathrectangle{\pgfqpoint{1.150000in}{0.150000in}}{\pgfqpoint{5.700000in}{5.700000in}}%
\pgfusepath{clip}%
\pgfsetbuttcap%
\pgfsetroundjoin%
\definecolor{currentfill}{rgb}{0.269944,0.014625,0.341379}%
\pgfsetfillcolor{currentfill}%
\pgfsetfillopacity{0.800000}%
\pgfsetlinewidth{0.000000pt}%
\definecolor{currentstroke}{rgb}{0.000000,0.000000,0.000000}%
\pgfsetstrokecolor{currentstroke}%
\pgfsetdash{}{0pt}%
\pgfpathmoveto{\pgfqpoint{3.197343in}{1.710653in}}%
\pgfpathlineto{\pgfqpoint{3.211035in}{1.703798in}}%
\pgfpathlineto{\pgfqpoint{3.224728in}{1.697156in}}%
\pgfpathlineto{\pgfqpoint{3.238422in}{1.690728in}}%
\pgfpathlineto{\pgfqpoint{3.252119in}{1.684513in}}%
\pgfpathlineto{\pgfqpoint{3.260431in}{1.692267in}}%
\pgfpathlineto{\pgfqpoint{3.268734in}{1.700139in}}%
\pgfpathlineto{\pgfqpoint{3.277030in}{1.708127in}}%
\pgfpathlineto{\pgfqpoint{3.285318in}{1.716225in}}%
\pgfpathlineto{\pgfqpoint{3.271642in}{1.722005in}}%
\pgfpathlineto{\pgfqpoint{3.257969in}{1.727997in}}%
\pgfpathlineto{\pgfqpoint{3.244298in}{1.734202in}}%
\pgfpathlineto{\pgfqpoint{3.230628in}{1.740623in}}%
\pgfpathlineto{\pgfqpoint{3.222320in}{1.732948in}}%
\pgfpathlineto{\pgfqpoint{3.214003in}{1.725392in}}%
\pgfpathlineto{\pgfqpoint{3.205677in}{1.717959in}}%
\pgfpathlineto{\pgfqpoint{3.197343in}{1.710653in}}%
\pgfpathclose%
\pgfusepath{fill}%
\end{pgfscope}%
\begin{pgfscope}%
\pgfpathrectangle{\pgfqpoint{1.150000in}{0.150000in}}{\pgfqpoint{5.700000in}{5.700000in}}%
\pgfusepath{clip}%
\pgfsetbuttcap%
\pgfsetroundjoin%
\definecolor{currentfill}{rgb}{0.281412,0.155834,0.469201}%
\pgfsetfillcolor{currentfill}%
\pgfsetfillopacity{0.800000}%
\pgfsetlinewidth{0.000000pt}%
\definecolor{currentstroke}{rgb}{0.000000,0.000000,0.000000}%
\pgfsetstrokecolor{currentstroke}%
\pgfsetdash{}{0pt}%
\pgfpathmoveto{\pgfqpoint{2.689651in}{2.028712in}}%
\pgfpathlineto{\pgfqpoint{2.703487in}{2.013310in}}%
\pgfpathlineto{\pgfqpoint{2.717317in}{1.998169in}}%
\pgfpathlineto{\pgfqpoint{2.731140in}{1.983289in}}%
\pgfpathlineto{\pgfqpoint{2.744957in}{1.968667in}}%
\pgfpathlineto{\pgfqpoint{2.753593in}{1.971168in}}%
\pgfpathlineto{\pgfqpoint{2.762215in}{1.973890in}}%
\pgfpathlineto{\pgfqpoint{2.770823in}{1.976830in}}%
\pgfpathlineto{\pgfqpoint{2.779418in}{1.979982in}}%
\pgfpathlineto{\pgfqpoint{2.765638in}{1.994091in}}%
\pgfpathlineto{\pgfqpoint{2.751853in}{2.008457in}}%
\pgfpathlineto{\pgfqpoint{2.738062in}{2.023083in}}%
\pgfpathlineto{\pgfqpoint{2.724265in}{2.037970in}}%
\pgfpathlineto{\pgfqpoint{2.715633in}{2.035319in}}%
\pgfpathlineto{\pgfqpoint{2.706987in}{2.032890in}}%
\pgfpathlineto{\pgfqpoint{2.698327in}{2.030685in}}%
\pgfpathlineto{\pgfqpoint{2.689651in}{2.028712in}}%
\pgfpathclose%
\pgfusepath{fill}%
\end{pgfscope}%
\begin{pgfscope}%
\pgfpathrectangle{\pgfqpoint{1.150000in}{0.150000in}}{\pgfqpoint{5.700000in}{5.700000in}}%
\pgfusepath{clip}%
\pgfsetbuttcap%
\pgfsetroundjoin%
\definecolor{currentfill}{rgb}{0.120565,0.596422,0.543611}%
\pgfsetfillcolor{currentfill}%
\pgfsetfillopacity{0.800000}%
\pgfsetlinewidth{0.000000pt}%
\definecolor{currentstroke}{rgb}{0.000000,0.000000,0.000000}%
\pgfsetstrokecolor{currentstroke}%
\pgfsetdash{}{0pt}%
\pgfpathmoveto{\pgfqpoint{5.476220in}{3.186011in}}%
\pgfpathlineto{\pgfqpoint{5.490800in}{3.198144in}}%
\pgfpathlineto{\pgfqpoint{5.505399in}{3.210456in}}%
\pgfpathlineto{\pgfqpoint{5.520019in}{3.222948in}}%
\pgfpathlineto{\pgfqpoint{5.534659in}{3.235619in}}%
\pgfpathlineto{\pgfqpoint{5.541991in}{3.238731in}}%
\pgfpathlineto{\pgfqpoint{5.549315in}{3.241820in}}%
\pgfpathlineto{\pgfqpoint{5.556633in}{3.244892in}}%
\pgfpathlineto{\pgfqpoint{5.563943in}{3.247952in}}%
\pgfpathlineto{\pgfqpoint{5.549328in}{3.235770in}}%
\pgfpathlineto{\pgfqpoint{5.534732in}{3.223767in}}%
\pgfpathlineto{\pgfqpoint{5.520156in}{3.211942in}}%
\pgfpathlineto{\pgfqpoint{5.505600in}{3.200296in}}%
\pgfpathlineto{\pgfqpoint{5.498265in}{3.196736in}}%
\pgfpathlineto{\pgfqpoint{5.490923in}{3.193172in}}%
\pgfpathlineto{\pgfqpoint{5.483575in}{3.189599in}}%
\pgfpathlineto{\pgfqpoint{5.476220in}{3.186011in}}%
\pgfpathclose%
\pgfusepath{fill}%
\end{pgfscope}%
\begin{pgfscope}%
\pgfpathrectangle{\pgfqpoint{1.150000in}{0.150000in}}{\pgfqpoint{5.700000in}{5.700000in}}%
\pgfusepath{clip}%
\pgfsetbuttcap%
\pgfsetroundjoin%
\definecolor{currentfill}{rgb}{0.271828,0.209303,0.504434}%
\pgfsetfillcolor{currentfill}%
\pgfsetfillopacity{0.800000}%
\pgfsetlinewidth{0.000000pt}%
\definecolor{currentstroke}{rgb}{0.000000,0.000000,0.000000}%
\pgfsetstrokecolor{currentstroke}%
\pgfsetdash{}{0pt}%
\pgfpathmoveto{\pgfqpoint{4.126163in}{2.087912in}}%
\pgfpathlineto{\pgfqpoint{4.140024in}{2.092726in}}%
\pgfpathlineto{\pgfqpoint{4.153898in}{2.097729in}}%
\pgfpathlineto{\pgfqpoint{4.167782in}{2.102919in}}%
\pgfpathlineto{\pgfqpoint{4.181678in}{2.108297in}}%
\pgfpathlineto{\pgfqpoint{4.189645in}{2.119844in}}%
\pgfpathlineto{\pgfqpoint{4.197607in}{2.131321in}}%
\pgfpathlineto{\pgfqpoint{4.205564in}{2.142729in}}%
\pgfpathlineto{\pgfqpoint{4.213515in}{2.154067in}}%
\pgfpathlineto{\pgfqpoint{4.199623in}{2.148571in}}%
\pgfpathlineto{\pgfqpoint{4.185743in}{2.143261in}}%
\pgfpathlineto{\pgfqpoint{4.171874in}{2.138140in}}%
\pgfpathlineto{\pgfqpoint{4.158016in}{2.133207in}}%
\pgfpathlineto{\pgfqpoint{4.150060in}{2.121975in}}%
\pgfpathlineto{\pgfqpoint{4.142100in}{2.110682in}}%
\pgfpathlineto{\pgfqpoint{4.134134in}{2.099327in}}%
\pgfpathlineto{\pgfqpoint{4.126163in}{2.087912in}}%
\pgfpathclose%
\pgfusepath{fill}%
\end{pgfscope}%
\begin{pgfscope}%
\pgfpathrectangle{\pgfqpoint{1.150000in}{0.150000in}}{\pgfqpoint{5.700000in}{5.700000in}}%
\pgfusepath{clip}%
\pgfsetbuttcap%
\pgfsetroundjoin%
\definecolor{currentfill}{rgb}{0.235526,0.309527,0.542944}%
\pgfsetfillcolor{currentfill}%
\pgfsetfillopacity{0.800000}%
\pgfsetlinewidth{0.000000pt}%
\definecolor{currentstroke}{rgb}{0.000000,0.000000,0.000000}%
\pgfsetstrokecolor{currentstroke}%
\pgfsetdash{}{0pt}%
\pgfpathmoveto{\pgfqpoint{4.420008in}{2.336493in}}%
\pgfpathlineto{\pgfqpoint{4.434002in}{2.343830in}}%
\pgfpathlineto{\pgfqpoint{4.448010in}{2.351354in}}%
\pgfpathlineto{\pgfqpoint{4.462032in}{2.359062in}}%
\pgfpathlineto{\pgfqpoint{4.476067in}{2.366956in}}%
\pgfpathlineto{\pgfqpoint{4.483936in}{2.377356in}}%
\pgfpathlineto{\pgfqpoint{4.491799in}{2.387660in}}%
\pgfpathlineto{\pgfqpoint{4.499656in}{2.397869in}}%
\pgfpathlineto{\pgfqpoint{4.507507in}{2.407983in}}%
\pgfpathlineto{\pgfqpoint{4.493476in}{2.400101in}}%
\pgfpathlineto{\pgfqpoint{4.479459in}{2.392404in}}%
\pgfpathlineto{\pgfqpoint{4.465456in}{2.384892in}}%
\pgfpathlineto{\pgfqpoint{4.451466in}{2.377566in}}%
\pgfpathlineto{\pgfqpoint{4.443610in}{2.367428in}}%
\pgfpathlineto{\pgfqpoint{4.435748in}{2.357204in}}%
\pgfpathlineto{\pgfqpoint{4.427881in}{2.346892in}}%
\pgfpathlineto{\pgfqpoint{4.420008in}{2.336493in}}%
\pgfpathclose%
\pgfusepath{fill}%
\end{pgfscope}%
\begin{pgfscope}%
\pgfpathrectangle{\pgfqpoint{1.150000in}{0.150000in}}{\pgfqpoint{5.700000in}{5.700000in}}%
\pgfusepath{clip}%
\pgfsetbuttcap%
\pgfsetroundjoin%
\definecolor{currentfill}{rgb}{0.262138,0.242286,0.520837}%
\pgfsetfillcolor{currentfill}%
\pgfsetfillopacity{0.800000}%
\pgfsetlinewidth{0.000000pt}%
\definecolor{currentstroke}{rgb}{0.000000,0.000000,0.000000}%
\pgfsetstrokecolor{currentstroke}%
\pgfsetdash{}{0pt}%
\pgfpathmoveto{\pgfqpoint{2.523019in}{2.234716in}}%
\pgfpathlineto{\pgfqpoint{2.536953in}{2.216008in}}%
\pgfpathlineto{\pgfqpoint{2.550878in}{2.197587in}}%
\pgfpathlineto{\pgfqpoint{2.564794in}{2.179452in}}%
\pgfpathlineto{\pgfqpoint{2.578700in}{2.161600in}}%
\pgfpathlineto{\pgfqpoint{2.587458in}{2.162509in}}%
\pgfpathlineto{\pgfqpoint{2.596201in}{2.163667in}}%
\pgfpathlineto{\pgfqpoint{2.604928in}{2.165068in}}%
\pgfpathlineto{\pgfqpoint{2.613640in}{2.166707in}}%
\pgfpathlineto{\pgfqpoint{2.599777in}{2.184036in}}%
\pgfpathlineto{\pgfqpoint{2.585905in}{2.201647in}}%
\pgfpathlineto{\pgfqpoint{2.572024in}{2.219544in}}%
\pgfpathlineto{\pgfqpoint{2.558134in}{2.237727in}}%
\pgfpathlineto{\pgfqpoint{2.549379in}{2.236599in}}%
\pgfpathlineto{\pgfqpoint{2.540609in}{2.235718in}}%
\pgfpathlineto{\pgfqpoint{2.531822in}{2.235089in}}%
\pgfpathlineto{\pgfqpoint{2.523019in}{2.234716in}}%
\pgfpathclose%
\pgfusepath{fill}%
\end{pgfscope}%
\begin{pgfscope}%
\pgfpathrectangle{\pgfqpoint{1.150000in}{0.150000in}}{\pgfqpoint{5.700000in}{5.700000in}}%
\pgfusepath{clip}%
\pgfsetbuttcap%
\pgfsetroundjoin%
\definecolor{currentfill}{rgb}{0.192357,0.403199,0.555836}%
\pgfsetfillcolor{currentfill}%
\pgfsetfillopacity{0.800000}%
\pgfsetlinewidth{0.000000pt}%
\definecolor{currentstroke}{rgb}{0.000000,0.000000,0.000000}%
\pgfsetstrokecolor{currentstroke}%
\pgfsetdash{}{0pt}%
\pgfpathmoveto{\pgfqpoint{4.713886in}{2.590135in}}%
\pgfpathlineto{\pgfqpoint{4.728034in}{2.599443in}}%
\pgfpathlineto{\pgfqpoint{4.742198in}{2.608934in}}%
\pgfpathlineto{\pgfqpoint{4.756378in}{2.618610in}}%
\pgfpathlineto{\pgfqpoint{4.770574in}{2.628469in}}%
\pgfpathlineto{\pgfqpoint{4.778324in}{2.637050in}}%
\pgfpathlineto{\pgfqpoint{4.786068in}{2.645529in}}%
\pgfpathlineto{\pgfqpoint{4.793805in}{2.653909in}}%
\pgfpathlineto{\pgfqpoint{4.801534in}{2.662190in}}%
\pgfpathlineto{\pgfqpoint{4.787346in}{2.652477in}}%
\pgfpathlineto{\pgfqpoint{4.773174in}{2.642947in}}%
\pgfpathlineto{\pgfqpoint{4.759018in}{2.633600in}}%
\pgfpathlineto{\pgfqpoint{4.744877in}{2.624437in}}%
\pgfpathlineto{\pgfqpoint{4.737139in}{2.615998in}}%
\pgfpathlineto{\pgfqpoint{4.729395in}{2.607469in}}%
\pgfpathlineto{\pgfqpoint{4.721644in}{2.598849in}}%
\pgfpathlineto{\pgfqpoint{4.713886in}{2.590135in}}%
\pgfpathclose%
\pgfusepath{fill}%
\end{pgfscope}%
\begin{pgfscope}%
\pgfpathrectangle{\pgfqpoint{1.150000in}{0.150000in}}{\pgfqpoint{5.700000in}{5.700000in}}%
\pgfusepath{clip}%
\pgfsetbuttcap%
\pgfsetroundjoin%
\definecolor{currentfill}{rgb}{0.283072,0.130895,0.449241}%
\pgfsetfillcolor{currentfill}%
\pgfsetfillopacity{0.800000}%
\pgfsetlinewidth{0.000000pt}%
\definecolor{currentstroke}{rgb}{0.000000,0.000000,0.000000}%
\pgfsetstrokecolor{currentstroke}%
\pgfsetdash{}{0pt}%
\pgfpathmoveto{\pgfqpoint{2.744957in}{1.968667in}}%
\pgfpathlineto{\pgfqpoint{2.758768in}{1.954301in}}%
\pgfpathlineto{\pgfqpoint{2.772574in}{1.940189in}}%
\pgfpathlineto{\pgfqpoint{2.786374in}{1.926330in}}%
\pgfpathlineto{\pgfqpoint{2.800170in}{1.912721in}}%
\pgfpathlineto{\pgfqpoint{2.808768in}{1.915747in}}%
\pgfpathlineto{\pgfqpoint{2.817354in}{1.918985in}}%
\pgfpathlineto{\pgfqpoint{2.825926in}{1.922432in}}%
\pgfpathlineto{\pgfqpoint{2.834485in}{1.926083in}}%
\pgfpathlineto{\pgfqpoint{2.820725in}{1.939181in}}%
\pgfpathlineto{\pgfqpoint{2.806961in}{1.952529in}}%
\pgfpathlineto{\pgfqpoint{2.793192in}{1.966128in}}%
\pgfpathlineto{\pgfqpoint{2.779418in}{1.979982in}}%
\pgfpathlineto{\pgfqpoint{2.770823in}{1.976830in}}%
\pgfpathlineto{\pgfqpoint{2.762215in}{1.973890in}}%
\pgfpathlineto{\pgfqpoint{2.753593in}{1.971168in}}%
\pgfpathlineto{\pgfqpoint{2.744957in}{1.968667in}}%
\pgfpathclose%
\pgfusepath{fill}%
\end{pgfscope}%
\begin{pgfscope}%
\pgfpathrectangle{\pgfqpoint{1.150000in}{0.150000in}}{\pgfqpoint{5.700000in}{5.700000in}}%
\pgfusepath{clip}%
\pgfsetbuttcap%
\pgfsetroundjoin%
\definecolor{currentfill}{rgb}{0.269944,0.014625,0.341379}%
\pgfsetfillcolor{currentfill}%
\pgfsetfillopacity{0.800000}%
\pgfsetlinewidth{0.000000pt}%
\definecolor{currentstroke}{rgb}{0.000000,0.000000,0.000000}%
\pgfsetstrokecolor{currentstroke}%
\pgfsetdash{}{0pt}%
\pgfpathmoveto{\pgfqpoint{3.340042in}{1.695211in}}%
\pgfpathlineto{\pgfqpoint{3.353729in}{1.690480in}}%
\pgfpathlineto{\pgfqpoint{3.367420in}{1.685955in}}%
\pgfpathlineto{\pgfqpoint{3.381114in}{1.681636in}}%
\pgfpathlineto{\pgfqpoint{3.394812in}{1.677523in}}%
\pgfpathlineto{\pgfqpoint{3.403054in}{1.686558in}}%
\pgfpathlineto{\pgfqpoint{3.411289in}{1.695680in}}%
\pgfpathlineto{\pgfqpoint{3.419518in}{1.704885in}}%
\pgfpathlineto{\pgfqpoint{3.427739in}{1.714169in}}%
\pgfpathlineto{\pgfqpoint{3.414058in}{1.717880in}}%
\pgfpathlineto{\pgfqpoint{3.400382in}{1.721796in}}%
\pgfpathlineto{\pgfqpoint{3.386708in}{1.725917in}}%
\pgfpathlineto{\pgfqpoint{3.373038in}{1.730246in}}%
\pgfpathlineto{\pgfqpoint{3.364800in}{1.721352in}}%
\pgfpathlineto{\pgfqpoint{3.356555in}{1.712546in}}%
\pgfpathlineto{\pgfqpoint{3.348302in}{1.703831in}}%
\pgfpathlineto{\pgfqpoint{3.340042in}{1.695211in}}%
\pgfpathclose%
\pgfusepath{fill}%
\end{pgfscope}%
\begin{pgfscope}%
\pgfpathrectangle{\pgfqpoint{1.150000in}{0.150000in}}{\pgfqpoint{5.700000in}{5.700000in}}%
\pgfusepath{clip}%
\pgfsetbuttcap%
\pgfsetroundjoin%
\definecolor{currentfill}{rgb}{0.273809,0.031497,0.358853}%
\pgfsetfillcolor{currentfill}%
\pgfsetfillopacity{0.800000}%
\pgfsetlinewidth{0.000000pt}%
\definecolor{currentstroke}{rgb}{0.000000,0.000000,0.000000}%
\pgfsetstrokecolor{currentstroke}%
\pgfsetdash{}{0pt}%
\pgfpathmoveto{\pgfqpoint{3.054198in}{1.749242in}}%
\pgfpathlineto{\pgfqpoint{3.067914in}{1.740160in}}%
\pgfpathlineto{\pgfqpoint{3.081629in}{1.731302in}}%
\pgfpathlineto{\pgfqpoint{3.095343in}{1.722666in}}%
\pgfpathlineto{\pgfqpoint{3.109058in}{1.714252in}}%
\pgfpathlineto{\pgfqpoint{3.117454in}{1.720525in}}%
\pgfpathlineto{\pgfqpoint{3.125840in}{1.726952in}}%
\pgfpathlineto{\pgfqpoint{3.134217in}{1.733526in}}%
\pgfpathlineto{\pgfqpoint{3.142584in}{1.740244in}}%
\pgfpathlineto{\pgfqpoint{3.128896in}{1.748189in}}%
\pgfpathlineto{\pgfqpoint{3.115207in}{1.756355in}}%
\pgfpathlineto{\pgfqpoint{3.101519in}{1.764743in}}%
\pgfpathlineto{\pgfqpoint{3.087830in}{1.773355in}}%
\pgfpathlineto{\pgfqpoint{3.079437in}{1.767095in}}%
\pgfpathlineto{\pgfqpoint{3.071034in}{1.760986in}}%
\pgfpathlineto{\pgfqpoint{3.062622in}{1.755033in}}%
\pgfpathlineto{\pgfqpoint{3.054198in}{1.749242in}}%
\pgfpathclose%
\pgfusepath{fill}%
\end{pgfscope}%
\begin{pgfscope}%
\pgfpathrectangle{\pgfqpoint{1.150000in}{0.150000in}}{\pgfqpoint{5.700000in}{5.700000in}}%
\pgfusepath{clip}%
\pgfsetbuttcap%
\pgfsetroundjoin%
\definecolor{currentfill}{rgb}{0.250425,0.274290,0.533103}%
\pgfsetfillcolor{currentfill}%
\pgfsetfillopacity{0.800000}%
\pgfsetlinewidth{0.000000pt}%
\definecolor{currentstroke}{rgb}{0.000000,0.000000,0.000000}%
\pgfsetstrokecolor{currentstroke}%
\pgfsetdash{}{0pt}%
\pgfpathmoveto{\pgfqpoint{2.467177in}{2.312482in}}%
\pgfpathlineto{\pgfqpoint{2.481154in}{2.292596in}}%
\pgfpathlineto{\pgfqpoint{2.495119in}{2.273008in}}%
\pgfpathlineto{\pgfqpoint{2.509074in}{2.253716in}}%
\pgfpathlineto{\pgfqpoint{2.523019in}{2.234716in}}%
\pgfpathlineto{\pgfqpoint{2.531822in}{2.235089in}}%
\pgfpathlineto{\pgfqpoint{2.540609in}{2.235718in}}%
\pgfpathlineto{\pgfqpoint{2.549379in}{2.236599in}}%
\pgfpathlineto{\pgfqpoint{2.558134in}{2.237727in}}%
\pgfpathlineto{\pgfqpoint{2.544234in}{2.256200in}}%
\pgfpathlineto{\pgfqpoint{2.530324in}{2.274965in}}%
\pgfpathlineto{\pgfqpoint{2.516404in}{2.294024in}}%
\pgfpathlineto{\pgfqpoint{2.502474in}{2.313381in}}%
\pgfpathlineto{\pgfqpoint{2.493675in}{2.312767in}}%
\pgfpathlineto{\pgfqpoint{2.484860in}{2.312410in}}%
\pgfpathlineto{\pgfqpoint{2.476027in}{2.312313in}}%
\pgfpathlineto{\pgfqpoint{2.467177in}{2.312482in}}%
\pgfpathclose%
\pgfusepath{fill}%
\end{pgfscope}%
\begin{pgfscope}%
\pgfpathrectangle{\pgfqpoint{1.150000in}{0.150000in}}{\pgfqpoint{5.700000in}{5.700000in}}%
\pgfusepath{clip}%
\pgfsetbuttcap%
\pgfsetroundjoin%
\definecolor{currentfill}{rgb}{0.119699,0.618490,0.536347}%
\pgfsetfillcolor{currentfill}%
\pgfsetfillopacity{0.800000}%
\pgfsetlinewidth{0.000000pt}%
\definecolor{currentstroke}{rgb}{0.000000,0.000000,0.000000}%
\pgfsetstrokecolor{currentstroke}%
\pgfsetdash{}{0pt}%
\pgfpathmoveto{\pgfqpoint{5.563943in}{3.247952in}}%
\pgfpathlineto{\pgfqpoint{5.578579in}{3.260314in}}%
\pgfpathlineto{\pgfqpoint{5.593235in}{3.272855in}}%
\pgfpathlineto{\pgfqpoint{5.607911in}{3.285575in}}%
\pgfpathlineto{\pgfqpoint{5.622608in}{3.298476in}}%
\pgfpathlineto{\pgfqpoint{5.629886in}{3.301019in}}%
\pgfpathlineto{\pgfqpoint{5.637157in}{3.303553in}}%
\pgfpathlineto{\pgfqpoint{5.644421in}{3.306086in}}%
\pgfpathlineto{\pgfqpoint{5.651678in}{3.308621in}}%
\pgfpathlineto{\pgfqpoint{5.637008in}{3.296245in}}%
\pgfpathlineto{\pgfqpoint{5.622358in}{3.284048in}}%
\pgfpathlineto{\pgfqpoint{5.607729in}{3.272030in}}%
\pgfpathlineto{\pgfqpoint{5.593119in}{3.260189in}}%
\pgfpathlineto{\pgfqpoint{5.585835in}{3.257119in}}%
\pgfpathlineto{\pgfqpoint{5.578544in}{3.254061in}}%
\pgfpathlineto{\pgfqpoint{5.571247in}{3.251007in}}%
\pgfpathlineto{\pgfqpoint{5.563943in}{3.247952in}}%
\pgfpathclose%
\pgfusepath{fill}%
\end{pgfscope}%
\begin{pgfscope}%
\pgfpathrectangle{\pgfqpoint{1.150000in}{0.150000in}}{\pgfqpoint{5.700000in}{5.700000in}}%
\pgfusepath{clip}%
\pgfsetbuttcap%
\pgfsetroundjoin%
\definecolor{currentfill}{rgb}{0.283091,0.110553,0.431554}%
\pgfsetfillcolor{currentfill}%
\pgfsetfillopacity{0.800000}%
\pgfsetlinewidth{0.000000pt}%
\definecolor{currentstroke}{rgb}{0.000000,0.000000,0.000000}%
\pgfsetstrokecolor{currentstroke}%
\pgfsetdash{}{0pt}%
\pgfpathmoveto{\pgfqpoint{2.800170in}{1.912721in}}%
\pgfpathlineto{\pgfqpoint{2.813960in}{1.899362in}}%
\pgfpathlineto{\pgfqpoint{2.827746in}{1.886250in}}%
\pgfpathlineto{\pgfqpoint{2.841528in}{1.873384in}}%
\pgfpathlineto{\pgfqpoint{2.855306in}{1.860762in}}%
\pgfpathlineto{\pgfqpoint{2.863869in}{1.864309in}}%
\pgfpathlineto{\pgfqpoint{2.872419in}{1.868061in}}%
\pgfpathlineto{\pgfqpoint{2.880957in}{1.872013in}}%
\pgfpathlineto{\pgfqpoint{2.889482in}{1.876160in}}%
\pgfpathlineto{\pgfqpoint{2.875739in}{1.888274in}}%
\pgfpathlineto{\pgfqpoint{2.861991in}{1.900631in}}%
\pgfpathlineto{\pgfqpoint{2.848240in}{1.913234in}}%
\pgfpathlineto{\pgfqpoint{2.834485in}{1.926083in}}%
\pgfpathlineto{\pgfqpoint{2.825926in}{1.922432in}}%
\pgfpathlineto{\pgfqpoint{2.817354in}{1.918985in}}%
\pgfpathlineto{\pgfqpoint{2.808768in}{1.915747in}}%
\pgfpathlineto{\pgfqpoint{2.800170in}{1.912721in}}%
\pgfpathclose%
\pgfusepath{fill}%
\end{pgfscope}%
\begin{pgfscope}%
\pgfpathrectangle{\pgfqpoint{1.150000in}{0.150000in}}{\pgfqpoint{5.700000in}{5.700000in}}%
\pgfusepath{clip}%
\pgfsetbuttcap%
\pgfsetroundjoin%
\definecolor{currentfill}{rgb}{0.278791,0.062145,0.386592}%
\pgfsetfillcolor{currentfill}%
\pgfsetfillopacity{0.800000}%
\pgfsetlinewidth{0.000000pt}%
\definecolor{currentstroke}{rgb}{0.000000,0.000000,0.000000}%
\pgfsetstrokecolor{currentstroke}%
\pgfsetdash{}{0pt}%
\pgfpathmoveto{\pgfqpoint{3.657477in}{1.770626in}}%
\pgfpathlineto{\pgfqpoint{3.671200in}{1.770228in}}%
\pgfpathlineto{\pgfqpoint{3.684929in}{1.770026in}}%
\pgfpathlineto{\pgfqpoint{3.698665in}{1.770019in}}%
\pgfpathlineto{\pgfqpoint{3.712408in}{1.770207in}}%
\pgfpathlineto{\pgfqpoint{3.720525in}{1.781334in}}%
\pgfpathlineto{\pgfqpoint{3.728636in}{1.792476in}}%
\pgfpathlineto{\pgfqpoint{3.736741in}{1.803630in}}%
\pgfpathlineto{\pgfqpoint{3.744842in}{1.814795in}}%
\pgfpathlineto{\pgfqpoint{3.731108in}{1.814298in}}%
\pgfpathlineto{\pgfqpoint{3.717381in}{1.813997in}}%
\pgfpathlineto{\pgfqpoint{3.703662in}{1.813890in}}%
\pgfpathlineto{\pgfqpoint{3.689949in}{1.813980in}}%
\pgfpathlineto{\pgfqpoint{3.681839in}{1.803112in}}%
\pgfpathlineto{\pgfqpoint{3.673724in}{1.792262in}}%
\pgfpathlineto{\pgfqpoint{3.665603in}{1.781432in}}%
\pgfpathlineto{\pgfqpoint{3.657477in}{1.770626in}}%
\pgfpathclose%
\pgfusepath{fill}%
\end{pgfscope}%
\begin{pgfscope}%
\pgfpathrectangle{\pgfqpoint{1.150000in}{0.150000in}}{\pgfqpoint{5.700000in}{5.700000in}}%
\pgfusepath{clip}%
\pgfsetbuttcap%
\pgfsetroundjoin%
\definecolor{currentfill}{rgb}{0.281446,0.084320,0.407414}%
\pgfsetfillcolor{currentfill}%
\pgfsetfillopacity{0.800000}%
\pgfsetlinewidth{0.000000pt}%
\definecolor{currentstroke}{rgb}{0.000000,0.000000,0.000000}%
\pgfsetstrokecolor{currentstroke}%
\pgfsetdash{}{0pt}%
\pgfpathmoveto{\pgfqpoint{3.744842in}{1.814795in}}%
\pgfpathlineto{\pgfqpoint{3.758583in}{1.815485in}}%
\pgfpathlineto{\pgfqpoint{3.772332in}{1.816370in}}%
\pgfpathlineto{\pgfqpoint{3.786089in}{1.817447in}}%
\pgfpathlineto{\pgfqpoint{3.799854in}{1.818717in}}%
\pgfpathlineto{\pgfqpoint{3.807941in}{1.830178in}}%
\pgfpathlineto{\pgfqpoint{3.816023in}{1.841636in}}%
\pgfpathlineto{\pgfqpoint{3.824100in}{1.853088in}}%
\pgfpathlineto{\pgfqpoint{3.832172in}{1.864532in}}%
\pgfpathlineto{\pgfqpoint{3.818414in}{1.862984in}}%
\pgfpathlineto{\pgfqpoint{3.804665in}{1.861629in}}%
\pgfpathlineto{\pgfqpoint{3.790924in}{1.860468in}}%
\pgfpathlineto{\pgfqpoint{3.777191in}{1.859500in}}%
\pgfpathlineto{\pgfqpoint{3.769112in}{1.848321in}}%
\pgfpathlineto{\pgfqpoint{3.761027in}{1.837143in}}%
\pgfpathlineto{\pgfqpoint{3.752937in}{1.825966in}}%
\pgfpathlineto{\pgfqpoint{3.744842in}{1.814795in}}%
\pgfpathclose%
\pgfusepath{fill}%
\end{pgfscope}%
\begin{pgfscope}%
\pgfpathrectangle{\pgfqpoint{1.150000in}{0.150000in}}{\pgfqpoint{5.700000in}{5.700000in}}%
\pgfusepath{clip}%
\pgfsetbuttcap%
\pgfsetroundjoin%
\definecolor{currentfill}{rgb}{0.149039,0.508051,0.557250}%
\pgfsetfillcolor{currentfill}%
\pgfsetfillopacity{0.800000}%
\pgfsetlinewidth{0.000000pt}%
\definecolor{currentstroke}{rgb}{0.000000,0.000000,0.000000}%
\pgfsetstrokecolor{currentstroke}%
\pgfsetdash{}{0pt}%
\pgfpathmoveto{\pgfqpoint{5.095341in}{2.903166in}}%
\pgfpathlineto{\pgfqpoint{5.109709in}{2.914331in}}%
\pgfpathlineto{\pgfqpoint{5.124095in}{2.925678in}}%
\pgfpathlineto{\pgfqpoint{5.138499in}{2.937207in}}%
\pgfpathlineto{\pgfqpoint{5.152921in}{2.948919in}}%
\pgfpathlineto{\pgfqpoint{5.160482in}{2.954722in}}%
\pgfpathlineto{\pgfqpoint{5.168036in}{2.960446in}}%
\pgfpathlineto{\pgfqpoint{5.175581in}{2.966094in}}%
\pgfpathlineto{\pgfqpoint{5.183119in}{2.971670in}}%
\pgfpathlineto{\pgfqpoint{5.168711in}{2.960276in}}%
\pgfpathlineto{\pgfqpoint{5.154322in}{2.949063in}}%
\pgfpathlineto{\pgfqpoint{5.139950in}{2.938032in}}%
\pgfpathlineto{\pgfqpoint{5.125597in}{2.927182in}}%
\pgfpathlineto{\pgfqpoint{5.118044in}{2.921278in}}%
\pgfpathlineto{\pgfqpoint{5.110484in}{2.915310in}}%
\pgfpathlineto{\pgfqpoint{5.102916in}{2.909274in}}%
\pgfpathlineto{\pgfqpoint{5.095341in}{2.903166in}}%
\pgfpathclose%
\pgfusepath{fill}%
\end{pgfscope}%
\begin{pgfscope}%
\pgfpathrectangle{\pgfqpoint{1.150000in}{0.150000in}}{\pgfqpoint{5.700000in}{5.700000in}}%
\pgfusepath{clip}%
\pgfsetbuttcap%
\pgfsetroundjoin%
\definecolor{currentfill}{rgb}{0.263663,0.237631,0.518762}%
\pgfsetfillcolor{currentfill}%
\pgfsetfillopacity{0.800000}%
\pgfsetlinewidth{0.000000pt}%
\definecolor{currentstroke}{rgb}{0.000000,0.000000,0.000000}%
\pgfsetstrokecolor{currentstroke}%
\pgfsetdash{}{0pt}%
\pgfpathmoveto{\pgfqpoint{4.213515in}{2.154067in}}%
\pgfpathlineto{\pgfqpoint{4.227419in}{2.159752in}}%
\pgfpathlineto{\pgfqpoint{4.241335in}{2.165623in}}%
\pgfpathlineto{\pgfqpoint{4.255264in}{2.171681in}}%
\pgfpathlineto{\pgfqpoint{4.269204in}{2.177926in}}%
\pgfpathlineto{\pgfqpoint{4.277147in}{2.189292in}}%
\pgfpathlineto{\pgfqpoint{4.285084in}{2.200578in}}%
\pgfpathlineto{\pgfqpoint{4.293017in}{2.211784in}}%
\pgfpathlineto{\pgfqpoint{4.300943in}{2.222910in}}%
\pgfpathlineto{\pgfqpoint{4.287006in}{2.216579in}}%
\pgfpathlineto{\pgfqpoint{4.273082in}{2.210434in}}%
\pgfpathlineto{\pgfqpoint{4.259170in}{2.204476in}}%
\pgfpathlineto{\pgfqpoint{4.245270in}{2.198705in}}%
\pgfpathlineto{\pgfqpoint{4.237339in}{2.187654in}}%
\pgfpathlineto{\pgfqpoint{4.229403in}{2.176530in}}%
\pgfpathlineto{\pgfqpoint{4.221462in}{2.165334in}}%
\pgfpathlineto{\pgfqpoint{4.213515in}{2.154067in}}%
\pgfpathclose%
\pgfusepath{fill}%
\end{pgfscope}%
\begin{pgfscope}%
\pgfpathrectangle{\pgfqpoint{1.150000in}{0.150000in}}{\pgfqpoint{5.700000in}{5.700000in}}%
\pgfusepath{clip}%
\pgfsetbuttcap%
\pgfsetroundjoin%
\definecolor{currentfill}{rgb}{0.179019,0.433756,0.557430}%
\pgfsetfillcolor{currentfill}%
\pgfsetfillopacity{0.800000}%
\pgfsetlinewidth{0.000000pt}%
\definecolor{currentstroke}{rgb}{0.000000,0.000000,0.000000}%
\pgfsetstrokecolor{currentstroke}%
\pgfsetdash{}{0pt}%
\pgfpathmoveto{\pgfqpoint{2.221173in}{2.767290in}}%
\pgfpathlineto{\pgfqpoint{2.235382in}{2.741374in}}%
\pgfpathlineto{\pgfqpoint{2.249574in}{2.715818in}}%
\pgfpathlineto{\pgfqpoint{2.263748in}{2.690617in}}%
\pgfpathlineto{\pgfqpoint{2.277906in}{2.665769in}}%
\pgfpathlineto{\pgfqpoint{2.286883in}{2.664566in}}%
\pgfpathlineto{\pgfqpoint{2.295840in}{2.663645in}}%
\pgfpathlineto{\pgfqpoint{2.304779in}{2.663001in}}%
\pgfpathlineto{\pgfqpoint{2.313699in}{2.662629in}}%
\pgfpathlineto{\pgfqpoint{2.299593in}{2.686963in}}%
\pgfpathlineto{\pgfqpoint{2.285470in}{2.711648in}}%
\pgfpathlineto{\pgfqpoint{2.271332in}{2.736688in}}%
\pgfpathlineto{\pgfqpoint{2.257176in}{2.762086in}}%
\pgfpathlineto{\pgfqpoint{2.248205in}{2.762960in}}%
\pgfpathlineto{\pgfqpoint{2.239214in}{2.764115in}}%
\pgfpathlineto{\pgfqpoint{2.230204in}{2.765557in}}%
\pgfpathlineto{\pgfqpoint{2.221173in}{2.767290in}}%
\pgfpathclose%
\pgfusepath{fill}%
\end{pgfscope}%
\begin{pgfscope}%
\pgfpathrectangle{\pgfqpoint{1.150000in}{0.150000in}}{\pgfqpoint{5.700000in}{5.700000in}}%
\pgfusepath{clip}%
\pgfsetbuttcap%
\pgfsetroundjoin%
\definecolor{currentfill}{rgb}{0.283091,0.110553,0.431554}%
\pgfsetfillcolor{currentfill}%
\pgfsetfillopacity{0.800000}%
\pgfsetlinewidth{0.000000pt}%
\definecolor{currentstroke}{rgb}{0.000000,0.000000,0.000000}%
\pgfsetstrokecolor{currentstroke}%
\pgfsetdash{}{0pt}%
\pgfpathmoveto{\pgfqpoint{3.832172in}{1.864532in}}%
\pgfpathlineto{\pgfqpoint{3.845937in}{1.866272in}}%
\pgfpathlineto{\pgfqpoint{3.859711in}{1.868204in}}%
\pgfpathlineto{\pgfqpoint{3.873494in}{1.870327in}}%
\pgfpathlineto{\pgfqpoint{3.887286in}{1.872642in}}%
\pgfpathlineto{\pgfqpoint{3.895346in}{1.884333in}}%
\pgfpathlineto{\pgfqpoint{3.903401in}{1.896004in}}%
\pgfpathlineto{\pgfqpoint{3.911451in}{1.907652in}}%
\pgfpathlineto{\pgfqpoint{3.919496in}{1.919276in}}%
\pgfpathlineto{\pgfqpoint{3.905711in}{1.916715in}}%
\pgfpathlineto{\pgfqpoint{3.891934in}{1.914346in}}%
\pgfpathlineto{\pgfqpoint{3.878167in}{1.912168in}}%
\pgfpathlineto{\pgfqpoint{3.864409in}{1.910182in}}%
\pgfpathlineto{\pgfqpoint{3.856357in}{1.898792in}}%
\pgfpathlineto{\pgfqpoint{3.848300in}{1.887386in}}%
\pgfpathlineto{\pgfqpoint{3.840238in}{1.875965in}}%
\pgfpathlineto{\pgfqpoint{3.832172in}{1.864532in}}%
\pgfpathclose%
\pgfusepath{fill}%
\end{pgfscope}%
\begin{pgfscope}%
\pgfpathrectangle{\pgfqpoint{1.150000in}{0.150000in}}{\pgfqpoint{5.700000in}{5.700000in}}%
\pgfusepath{clip}%
\pgfsetbuttcap%
\pgfsetroundjoin%
\definecolor{currentfill}{rgb}{0.274952,0.037752,0.364543}%
\pgfsetfillcolor{currentfill}%
\pgfsetfillopacity{0.800000}%
\pgfsetlinewidth{0.000000pt}%
\definecolor{currentstroke}{rgb}{0.000000,0.000000,0.000000}%
\pgfsetstrokecolor{currentstroke}%
\pgfsetdash{}{0pt}%
\pgfpathmoveto{\pgfqpoint{3.570043in}{1.732612in}}%
\pgfpathlineto{\pgfqpoint{3.583753in}{1.731085in}}%
\pgfpathlineto{\pgfqpoint{3.597468in}{1.729757in}}%
\pgfpathlineto{\pgfqpoint{3.611189in}{1.728626in}}%
\pgfpathlineto{\pgfqpoint{3.624916in}{1.727692in}}%
\pgfpathlineto{\pgfqpoint{3.633065in}{1.738376in}}%
\pgfpathlineto{\pgfqpoint{3.641208in}{1.749095in}}%
\pgfpathlineto{\pgfqpoint{3.649345in}{1.759846in}}%
\pgfpathlineto{\pgfqpoint{3.657477in}{1.770626in}}%
\pgfpathlineto{\pgfqpoint{3.643761in}{1.771220in}}%
\pgfpathlineto{\pgfqpoint{3.630051in}{1.772011in}}%
\pgfpathlineto{\pgfqpoint{3.616348in}{1.772999in}}%
\pgfpathlineto{\pgfqpoint{3.602651in}{1.774186in}}%
\pgfpathlineto{\pgfqpoint{3.594508in}{1.763734in}}%
\pgfpathlineto{\pgfqpoint{3.586359in}{1.753319in}}%
\pgfpathlineto{\pgfqpoint{3.578204in}{1.742944in}}%
\pgfpathlineto{\pgfqpoint{3.570043in}{1.732612in}}%
\pgfpathclose%
\pgfusepath{fill}%
\end{pgfscope}%
\begin{pgfscope}%
\pgfpathrectangle{\pgfqpoint{1.150000in}{0.150000in}}{\pgfqpoint{5.700000in}{5.700000in}}%
\pgfusepath{clip}%
\pgfsetbuttcap%
\pgfsetroundjoin%
\definecolor{currentfill}{rgb}{0.221989,0.339161,0.548752}%
\pgfsetfillcolor{currentfill}%
\pgfsetfillopacity{0.800000}%
\pgfsetlinewidth{0.000000pt}%
\definecolor{currentstroke}{rgb}{0.000000,0.000000,0.000000}%
\pgfsetstrokecolor{currentstroke}%
\pgfsetdash{}{0pt}%
\pgfpathmoveto{\pgfqpoint{4.507507in}{2.407983in}}%
\pgfpathlineto{\pgfqpoint{4.521553in}{2.416050in}}%
\pgfpathlineto{\pgfqpoint{4.535613in}{2.424302in}}%
\pgfpathlineto{\pgfqpoint{4.549687in}{2.432739in}}%
\pgfpathlineto{\pgfqpoint{4.563776in}{2.441361in}}%
\pgfpathlineto{\pgfqpoint{4.571617in}{2.451348in}}%
\pgfpathlineto{\pgfqpoint{4.579451in}{2.461234in}}%
\pgfpathlineto{\pgfqpoint{4.587279in}{2.471019in}}%
\pgfpathlineto{\pgfqpoint{4.595101in}{2.480704in}}%
\pgfpathlineto{\pgfqpoint{4.581017in}{2.472127in}}%
\pgfpathlineto{\pgfqpoint{4.566947in}{2.463735in}}%
\pgfpathlineto{\pgfqpoint{4.552892in}{2.455528in}}%
\pgfpathlineto{\pgfqpoint{4.538852in}{2.447505in}}%
\pgfpathlineto{\pgfqpoint{4.531025in}{2.437763in}}%
\pgfpathlineto{\pgfqpoint{4.523192in}{2.427929in}}%
\pgfpathlineto{\pgfqpoint{4.515353in}{2.418002in}}%
\pgfpathlineto{\pgfqpoint{4.507507in}{2.407983in}}%
\pgfpathclose%
\pgfusepath{fill}%
\end{pgfscope}%
\begin{pgfscope}%
\pgfpathrectangle{\pgfqpoint{1.150000in}{0.150000in}}{\pgfqpoint{5.700000in}{5.700000in}}%
\pgfusepath{clip}%
\pgfsetbuttcap%
\pgfsetroundjoin%
\definecolor{currentfill}{rgb}{0.123444,0.636809,0.528763}%
\pgfsetfillcolor{currentfill}%
\pgfsetfillopacity{0.800000}%
\pgfsetlinewidth{0.000000pt}%
\definecolor{currentstroke}{rgb}{0.000000,0.000000,0.000000}%
\pgfsetstrokecolor{currentstroke}%
\pgfsetdash{}{0pt}%
\pgfpathmoveto{\pgfqpoint{5.651678in}{3.308621in}}%
\pgfpathlineto{\pgfqpoint{5.666369in}{3.321175in}}%
\pgfpathlineto{\pgfqpoint{5.681080in}{3.333909in}}%
\pgfpathlineto{\pgfqpoint{5.695812in}{3.346821in}}%
\pgfpathlineto{\pgfqpoint{5.710565in}{3.359913in}}%
\pgfpathlineto{\pgfqpoint{5.717788in}{3.361911in}}%
\pgfpathlineto{\pgfqpoint{5.725004in}{3.363917in}}%
\pgfpathlineto{\pgfqpoint{5.732213in}{3.365936in}}%
\pgfpathlineto{\pgfqpoint{5.739417in}{3.367975in}}%
\pgfpathlineto{\pgfqpoint{5.724693in}{3.355442in}}%
\pgfpathlineto{\pgfqpoint{5.709990in}{3.343088in}}%
\pgfpathlineto{\pgfqpoint{5.695308in}{3.330912in}}%
\pgfpathlineto{\pgfqpoint{5.680645in}{3.318913in}}%
\pgfpathlineto{\pgfqpoint{5.673413in}{3.316306in}}%
\pgfpathlineto{\pgfqpoint{5.666174in}{3.313725in}}%
\pgfpathlineto{\pgfqpoint{5.658929in}{3.311166in}}%
\pgfpathlineto{\pgfqpoint{5.651678in}{3.308621in}}%
\pgfpathclose%
\pgfusepath{fill}%
\end{pgfscope}%
\begin{pgfscope}%
\pgfpathrectangle{\pgfqpoint{1.150000in}{0.150000in}}{\pgfqpoint{5.700000in}{5.700000in}}%
\pgfusepath{clip}%
\pgfsetbuttcap%
\pgfsetroundjoin%
\definecolor{currentfill}{rgb}{0.237441,0.305202,0.541921}%
\pgfsetfillcolor{currentfill}%
\pgfsetfillopacity{0.800000}%
\pgfsetlinewidth{0.000000pt}%
\definecolor{currentstroke}{rgb}{0.000000,0.000000,0.000000}%
\pgfsetstrokecolor{currentstroke}%
\pgfsetdash{}{0pt}%
\pgfpathmoveto{\pgfqpoint{2.411155in}{2.395060in}}%
\pgfpathlineto{\pgfqpoint{2.425179in}{2.373955in}}%
\pgfpathlineto{\pgfqpoint{2.439190in}{2.353158in}}%
\pgfpathlineto{\pgfqpoint{2.453189in}{2.332668in}}%
\pgfpathlineto{\pgfqpoint{2.467177in}{2.312482in}}%
\pgfpathlineto{\pgfqpoint{2.476027in}{2.312313in}}%
\pgfpathlineto{\pgfqpoint{2.484860in}{2.312410in}}%
\pgfpathlineto{\pgfqpoint{2.493675in}{2.312767in}}%
\pgfpathlineto{\pgfqpoint{2.502474in}{2.313381in}}%
\pgfpathlineto{\pgfqpoint{2.488533in}{2.333037in}}%
\pgfpathlineto{\pgfqpoint{2.474581in}{2.352995in}}%
\pgfpathlineto{\pgfqpoint{2.460617in}{2.373259in}}%
\pgfpathlineto{\pgfqpoint{2.446642in}{2.393830in}}%
\pgfpathlineto{\pgfqpoint{2.437797in}{2.393735in}}%
\pgfpathlineto{\pgfqpoint{2.428934in}{2.393905in}}%
\pgfpathlineto{\pgfqpoint{2.420054in}{2.394345in}}%
\pgfpathlineto{\pgfqpoint{2.411155in}{2.395060in}}%
\pgfpathclose%
\pgfusepath{fill}%
\end{pgfscope}%
\begin{pgfscope}%
\pgfpathrectangle{\pgfqpoint{1.150000in}{0.150000in}}{\pgfqpoint{5.700000in}{5.700000in}}%
\pgfusepath{clip}%
\pgfsetbuttcap%
\pgfsetroundjoin%
\definecolor{currentfill}{rgb}{0.180629,0.429975,0.557282}%
\pgfsetfillcolor{currentfill}%
\pgfsetfillopacity{0.800000}%
\pgfsetlinewidth{0.000000pt}%
\definecolor{currentstroke}{rgb}{0.000000,0.000000,0.000000}%
\pgfsetstrokecolor{currentstroke}%
\pgfsetdash{}{0pt}%
\pgfpathmoveto{\pgfqpoint{4.801534in}{2.662190in}}%
\pgfpathlineto{\pgfqpoint{4.815739in}{2.672087in}}%
\pgfpathlineto{\pgfqpoint{4.829960in}{2.682167in}}%
\pgfpathlineto{\pgfqpoint{4.844197in}{2.692431in}}%
\pgfpathlineto{\pgfqpoint{4.858451in}{2.702878in}}%
\pgfpathlineto{\pgfqpoint{4.866166in}{2.710897in}}%
\pgfpathlineto{\pgfqpoint{4.873873in}{2.718813in}}%
\pgfpathlineto{\pgfqpoint{4.881574in}{2.726631in}}%
\pgfpathlineto{\pgfqpoint{4.889267in}{2.734351in}}%
\pgfpathlineto{\pgfqpoint{4.875021in}{2.724084in}}%
\pgfpathlineto{\pgfqpoint{4.860793in}{2.714000in}}%
\pgfpathlineto{\pgfqpoint{4.846580in}{2.704099in}}%
\pgfpathlineto{\pgfqpoint{4.832385in}{2.694382in}}%
\pgfpathlineto{\pgfqpoint{4.824682in}{2.686470in}}%
\pgfpathlineto{\pgfqpoint{4.816973in}{2.678469in}}%
\pgfpathlineto{\pgfqpoint{4.809257in}{2.670376in}}%
\pgfpathlineto{\pgfqpoint{4.801534in}{2.662190in}}%
\pgfpathclose%
\pgfusepath{fill}%
\end{pgfscope}%
\begin{pgfscope}%
\pgfpathrectangle{\pgfqpoint{1.150000in}{0.150000in}}{\pgfqpoint{5.700000in}{5.700000in}}%
\pgfusepath{clip}%
\pgfsetbuttcap%
\pgfsetroundjoin%
\definecolor{currentfill}{rgb}{0.281924,0.089666,0.412415}%
\pgfsetfillcolor{currentfill}%
\pgfsetfillopacity{0.800000}%
\pgfsetlinewidth{0.000000pt}%
\definecolor{currentstroke}{rgb}{0.000000,0.000000,0.000000}%
\pgfsetstrokecolor{currentstroke}%
\pgfsetdash{}{0pt}%
\pgfpathmoveto{\pgfqpoint{2.855306in}{1.860762in}}%
\pgfpathlineto{\pgfqpoint{2.869080in}{1.848382in}}%
\pgfpathlineto{\pgfqpoint{2.882850in}{1.836243in}}%
\pgfpathlineto{\pgfqpoint{2.896617in}{1.824344in}}%
\pgfpathlineto{\pgfqpoint{2.910380in}{1.812682in}}%
\pgfpathlineto{\pgfqpoint{2.918909in}{1.816749in}}%
\pgfpathlineto{\pgfqpoint{2.927426in}{1.821012in}}%
\pgfpathlineto{\pgfqpoint{2.935931in}{1.825468in}}%
\pgfpathlineto{\pgfqpoint{2.944424in}{1.830110in}}%
\pgfpathlineto{\pgfqpoint{2.930693in}{1.841265in}}%
\pgfpathlineto{\pgfqpoint{2.916959in}{1.852657in}}%
\pgfpathlineto{\pgfqpoint{2.903222in}{1.864289in}}%
\pgfpathlineto{\pgfqpoint{2.889482in}{1.876160in}}%
\pgfpathlineto{\pgfqpoint{2.880957in}{1.872013in}}%
\pgfpathlineto{\pgfqpoint{2.872419in}{1.868061in}}%
\pgfpathlineto{\pgfqpoint{2.863869in}{1.864309in}}%
\pgfpathlineto{\pgfqpoint{2.855306in}{1.860762in}}%
\pgfpathclose%
\pgfusepath{fill}%
\end{pgfscope}%
\begin{pgfscope}%
\pgfpathrectangle{\pgfqpoint{1.150000in}{0.150000in}}{\pgfqpoint{5.700000in}{5.700000in}}%
\pgfusepath{clip}%
\pgfsetbuttcap%
\pgfsetroundjoin%
\definecolor{currentfill}{rgb}{0.282884,0.135920,0.453427}%
\pgfsetfillcolor{currentfill}%
\pgfsetfillopacity{0.800000}%
\pgfsetlinewidth{0.000000pt}%
\definecolor{currentstroke}{rgb}{0.000000,0.000000,0.000000}%
\pgfsetstrokecolor{currentstroke}%
\pgfsetdash{}{0pt}%
\pgfpathmoveto{\pgfqpoint{3.919496in}{1.919276in}}%
\pgfpathlineto{\pgfqpoint{3.933291in}{1.922027in}}%
\pgfpathlineto{\pgfqpoint{3.947095in}{1.924969in}}%
\pgfpathlineto{\pgfqpoint{3.960909in}{1.928101in}}%
\pgfpathlineto{\pgfqpoint{3.974732in}{1.931423in}}%
\pgfpathlineto{\pgfqpoint{3.982767in}{1.943246in}}%
\pgfpathlineto{\pgfqpoint{3.990796in}{1.955033in}}%
\pgfpathlineto{\pgfqpoint{3.998821in}{1.966782in}}%
\pgfpathlineto{\pgfqpoint{4.006841in}{1.978491in}}%
\pgfpathlineto{\pgfqpoint{3.993023in}{1.974954in}}%
\pgfpathlineto{\pgfqpoint{3.979215in}{1.971608in}}%
\pgfpathlineto{\pgfqpoint{3.965416in}{1.968451in}}%
\pgfpathlineto{\pgfqpoint{3.951628in}{1.965486in}}%
\pgfpathlineto{\pgfqpoint{3.943602in}{1.953979in}}%
\pgfpathlineto{\pgfqpoint{3.935572in}{1.942441in}}%
\pgfpathlineto{\pgfqpoint{3.927536in}{1.930873in}}%
\pgfpathlineto{\pgfqpoint{3.919496in}{1.919276in}}%
\pgfpathclose%
\pgfusepath{fill}%
\end{pgfscope}%
\begin{pgfscope}%
\pgfpathrectangle{\pgfqpoint{1.150000in}{0.150000in}}{\pgfqpoint{5.700000in}{5.700000in}}%
\pgfusepath{clip}%
\pgfsetbuttcap%
\pgfsetroundjoin%
\definecolor{currentfill}{rgb}{0.272594,0.025563,0.353093}%
\pgfsetfillcolor{currentfill}%
\pgfsetfillopacity{0.800000}%
\pgfsetlinewidth{0.000000pt}%
\definecolor{currentstroke}{rgb}{0.000000,0.000000,0.000000}%
\pgfsetstrokecolor{currentstroke}%
\pgfsetdash{}{0pt}%
\pgfpathmoveto{\pgfqpoint{3.482503in}{1.701364in}}%
\pgfpathlineto{\pgfqpoint{3.496205in}{1.698668in}}%
\pgfpathlineto{\pgfqpoint{3.509911in}{1.696172in}}%
\pgfpathlineto{\pgfqpoint{3.523623in}{1.693877in}}%
\pgfpathlineto{\pgfqpoint{3.537340in}{1.691780in}}%
\pgfpathlineto{\pgfqpoint{3.545525in}{1.701907in}}%
\pgfpathlineto{\pgfqpoint{3.553704in}{1.712090in}}%
\pgfpathlineto{\pgfqpoint{3.561877in}{1.722326in}}%
\pgfpathlineto{\pgfqpoint{3.570043in}{1.732612in}}%
\pgfpathlineto{\pgfqpoint{3.556340in}{1.734337in}}%
\pgfpathlineto{\pgfqpoint{3.542641in}{1.736261in}}%
\pgfpathlineto{\pgfqpoint{3.528948in}{1.738385in}}%
\pgfpathlineto{\pgfqpoint{3.515261in}{1.740710in}}%
\pgfpathlineto{\pgfqpoint{3.507081in}{1.730784in}}%
\pgfpathlineto{\pgfqpoint{3.498894in}{1.720915in}}%
\pgfpathlineto{\pgfqpoint{3.490702in}{1.711107in}}%
\pgfpathlineto{\pgfqpoint{3.482503in}{1.701364in}}%
\pgfpathclose%
\pgfusepath{fill}%
\end{pgfscope}%
\begin{pgfscope}%
\pgfpathrectangle{\pgfqpoint{1.150000in}{0.150000in}}{\pgfqpoint{5.700000in}{5.700000in}}%
\pgfusepath{clip}%
\pgfsetbuttcap%
\pgfsetroundjoin%
\definecolor{currentfill}{rgb}{0.268510,0.009605,0.335427}%
\pgfsetfillcolor{currentfill}%
\pgfsetfillopacity{0.800000}%
\pgfsetlinewidth{0.000000pt}%
\definecolor{currentstroke}{rgb}{0.000000,0.000000,0.000000}%
\pgfsetstrokecolor{currentstroke}%
\pgfsetdash{}{0pt}%
\pgfpathmoveto{\pgfqpoint{3.252119in}{1.684513in}}%
\pgfpathlineto{\pgfqpoint{3.265817in}{1.678510in}}%
\pgfpathlineto{\pgfqpoint{3.279517in}{1.672717in}}%
\pgfpathlineto{\pgfqpoint{3.293219in}{1.667134in}}%
\pgfpathlineto{\pgfqpoint{3.306923in}{1.661759in}}%
\pgfpathlineto{\pgfqpoint{3.315215in}{1.669960in}}%
\pgfpathlineto{\pgfqpoint{3.323498in}{1.678271in}}%
\pgfpathlineto{\pgfqpoint{3.331774in}{1.686690in}}%
\pgfpathlineto{\pgfqpoint{3.340042in}{1.695211in}}%
\pgfpathlineto{\pgfqpoint{3.326357in}{1.700151in}}%
\pgfpathlineto{\pgfqpoint{3.312674in}{1.705299in}}%
\pgfpathlineto{\pgfqpoint{3.298995in}{1.710657in}}%
\pgfpathlineto{\pgfqpoint{3.285318in}{1.716225in}}%
\pgfpathlineto{\pgfqpoint{3.277030in}{1.708127in}}%
\pgfpathlineto{\pgfqpoint{3.268734in}{1.700139in}}%
\pgfpathlineto{\pgfqpoint{3.260431in}{1.692267in}}%
\pgfpathlineto{\pgfqpoint{3.252119in}{1.684513in}}%
\pgfpathclose%
\pgfusepath{fill}%
\end{pgfscope}%
\begin{pgfscope}%
\pgfpathrectangle{\pgfqpoint{1.150000in}{0.150000in}}{\pgfqpoint{5.700000in}{5.700000in}}%
\pgfusepath{clip}%
\pgfsetbuttcap%
\pgfsetroundjoin%
\definecolor{currentfill}{rgb}{0.134692,0.658636,0.517649}%
\pgfsetfillcolor{currentfill}%
\pgfsetfillopacity{0.800000}%
\pgfsetlinewidth{0.000000pt}%
\definecolor{currentstroke}{rgb}{0.000000,0.000000,0.000000}%
\pgfsetstrokecolor{currentstroke}%
\pgfsetdash{}{0pt}%
\pgfpathmoveto{\pgfqpoint{5.739417in}{3.367975in}}%
\pgfpathlineto{\pgfqpoint{5.754161in}{3.380685in}}%
\pgfpathlineto{\pgfqpoint{5.768927in}{3.393575in}}%
\pgfpathlineto{\pgfqpoint{5.783714in}{3.406643in}}%
\pgfpathlineto{\pgfqpoint{5.798522in}{3.419890in}}%
\pgfpathlineto{\pgfqpoint{5.805688in}{3.421373in}}%
\pgfpathlineto{\pgfqpoint{5.812849in}{3.422880in}}%
\pgfpathlineto{\pgfqpoint{5.820003in}{3.424419in}}%
\pgfpathlineto{\pgfqpoint{5.827152in}{3.425994in}}%
\pgfpathlineto{\pgfqpoint{5.812376in}{3.413341in}}%
\pgfpathlineto{\pgfqpoint{5.797621in}{3.400867in}}%
\pgfpathlineto{\pgfqpoint{5.782886in}{3.388569in}}%
\pgfpathlineto{\pgfqpoint{5.768173in}{3.376449in}}%
\pgfpathlineto{\pgfqpoint{5.760992in}{3.374270in}}%
\pgfpathlineto{\pgfqpoint{5.753806in}{3.372135in}}%
\pgfpathlineto{\pgfqpoint{5.746614in}{3.370039in}}%
\pgfpathlineto{\pgfqpoint{5.739417in}{3.367975in}}%
\pgfpathclose%
\pgfusepath{fill}%
\end{pgfscope}%
\begin{pgfscope}%
\pgfpathrectangle{\pgfqpoint{1.150000in}{0.150000in}}{\pgfqpoint{5.700000in}{5.700000in}}%
\pgfusepath{clip}%
\pgfsetbuttcap%
\pgfsetroundjoin%
\definecolor{currentfill}{rgb}{0.140536,0.530132,0.555659}%
\pgfsetfillcolor{currentfill}%
\pgfsetfillopacity{0.800000}%
\pgfsetlinewidth{0.000000pt}%
\definecolor{currentstroke}{rgb}{0.000000,0.000000,0.000000}%
\pgfsetstrokecolor{currentstroke}%
\pgfsetdash{}{0pt}%
\pgfpathmoveto{\pgfqpoint{5.183119in}{2.971670in}}%
\pgfpathlineto{\pgfqpoint{5.197546in}{2.983245in}}%
\pgfpathlineto{\pgfqpoint{5.211990in}{2.995002in}}%
\pgfpathlineto{\pgfqpoint{5.226454in}{3.006941in}}%
\pgfpathlineto{\pgfqpoint{5.240936in}{3.019061in}}%
\pgfpathlineto{\pgfqpoint{5.248451in}{3.024229in}}%
\pgfpathlineto{\pgfqpoint{5.255958in}{3.029324in}}%
\pgfpathlineto{\pgfqpoint{5.263457in}{3.034351in}}%
\pgfpathlineto{\pgfqpoint{5.270948in}{3.039313in}}%
\pgfpathlineto{\pgfqpoint{5.256482in}{3.027545in}}%
\pgfpathlineto{\pgfqpoint{5.242035in}{3.015958in}}%
\pgfpathlineto{\pgfqpoint{5.227606in}{3.004552in}}%
\pgfpathlineto{\pgfqpoint{5.213196in}{2.993326in}}%
\pgfpathlineto{\pgfqpoint{5.205688in}{2.988001in}}%
\pgfpathlineto{\pgfqpoint{5.198173in}{2.982619in}}%
\pgfpathlineto{\pgfqpoint{5.190650in}{2.977177in}}%
\pgfpathlineto{\pgfqpoint{5.183119in}{2.971670in}}%
\pgfpathclose%
\pgfusepath{fill}%
\end{pgfscope}%
\begin{pgfscope}%
\pgfpathrectangle{\pgfqpoint{1.150000in}{0.150000in}}{\pgfqpoint{5.700000in}{5.700000in}}%
\pgfusepath{clip}%
\pgfsetbuttcap%
\pgfsetroundjoin%
\definecolor{currentfill}{rgb}{0.272594,0.025563,0.353093}%
\pgfsetfillcolor{currentfill}%
\pgfsetfillopacity{0.800000}%
\pgfsetlinewidth{0.000000pt}%
\definecolor{currentstroke}{rgb}{0.000000,0.000000,0.000000}%
\pgfsetstrokecolor{currentstroke}%
\pgfsetdash{}{0pt}%
\pgfpathmoveto{\pgfqpoint{3.109058in}{1.714252in}}%
\pgfpathlineto{\pgfqpoint{3.122772in}{1.706058in}}%
\pgfpathlineto{\pgfqpoint{3.136486in}{1.698083in}}%
\pgfpathlineto{\pgfqpoint{3.150200in}{1.690327in}}%
\pgfpathlineto{\pgfqpoint{3.163915in}{1.682787in}}%
\pgfpathlineto{\pgfqpoint{3.172286in}{1.689541in}}%
\pgfpathlineto{\pgfqpoint{3.180647in}{1.696440in}}%
\pgfpathlineto{\pgfqpoint{3.189000in}{1.703479in}}%
\pgfpathlineto{\pgfqpoint{3.197343in}{1.710653in}}%
\pgfpathlineto{\pgfqpoint{3.183652in}{1.717725in}}%
\pgfpathlineto{\pgfqpoint{3.169962in}{1.725013in}}%
\pgfpathlineto{\pgfqpoint{3.156273in}{1.732519in}}%
\pgfpathlineto{\pgfqpoint{3.142584in}{1.740244in}}%
\pgfpathlineto{\pgfqpoint{3.134217in}{1.733526in}}%
\pgfpathlineto{\pgfqpoint{3.125840in}{1.726952in}}%
\pgfpathlineto{\pgfqpoint{3.117454in}{1.720525in}}%
\pgfpathlineto{\pgfqpoint{3.109058in}{1.714252in}}%
\pgfpathclose%
\pgfusepath{fill}%
\end{pgfscope}%
\begin{pgfscope}%
\pgfpathrectangle{\pgfqpoint{1.150000in}{0.150000in}}{\pgfqpoint{5.700000in}{5.700000in}}%
\pgfusepath{clip}%
\pgfsetbuttcap%
\pgfsetroundjoin%
\definecolor{currentfill}{rgb}{0.252194,0.269783,0.531579}%
\pgfsetfillcolor{currentfill}%
\pgfsetfillopacity{0.800000}%
\pgfsetlinewidth{0.000000pt}%
\definecolor{currentstroke}{rgb}{0.000000,0.000000,0.000000}%
\pgfsetstrokecolor{currentstroke}%
\pgfsetdash{}{0pt}%
\pgfpathmoveto{\pgfqpoint{4.300943in}{2.222910in}}%
\pgfpathlineto{\pgfqpoint{4.314893in}{2.229428in}}%
\pgfpathlineto{\pgfqpoint{4.328856in}{2.236132in}}%
\pgfpathlineto{\pgfqpoint{4.342831in}{2.243023in}}%
\pgfpathlineto{\pgfqpoint{4.356820in}{2.250099in}}%
\pgfpathlineto{\pgfqpoint{4.364738in}{2.261211in}}%
\pgfpathlineto{\pgfqpoint{4.372650in}{2.272233in}}%
\pgfpathlineto{\pgfqpoint{4.380557in}{2.283166in}}%
\pgfpathlineto{\pgfqpoint{4.388458in}{2.294009in}}%
\pgfpathlineto{\pgfqpoint{4.374473in}{2.286879in}}%
\pgfpathlineto{\pgfqpoint{4.360502in}{2.279934in}}%
\pgfpathlineto{\pgfqpoint{4.346543in}{2.273176in}}%
\pgfpathlineto{\pgfqpoint{4.332597in}{2.266604in}}%
\pgfpathlineto{\pgfqpoint{4.324692in}{2.255802in}}%
\pgfpathlineto{\pgfqpoint{4.316781in}{2.244919in}}%
\pgfpathlineto{\pgfqpoint{4.308865in}{2.233955in}}%
\pgfpathlineto{\pgfqpoint{4.300943in}{2.222910in}}%
\pgfpathclose%
\pgfusepath{fill}%
\end{pgfscope}%
\begin{pgfscope}%
\pgfpathrectangle{\pgfqpoint{1.150000in}{0.150000in}}{\pgfqpoint{5.700000in}{5.700000in}}%
\pgfusepath{clip}%
\pgfsetbuttcap%
\pgfsetroundjoin%
\definecolor{currentfill}{rgb}{0.280255,0.165693,0.476498}%
\pgfsetfillcolor{currentfill}%
\pgfsetfillopacity{0.800000}%
\pgfsetlinewidth{0.000000pt}%
\definecolor{currentstroke}{rgb}{0.000000,0.000000,0.000000}%
\pgfsetstrokecolor{currentstroke}%
\pgfsetdash{}{0pt}%
\pgfpathmoveto{\pgfqpoint{4.006841in}{1.978491in}}%
\pgfpathlineto{\pgfqpoint{4.020669in}{1.982216in}}%
\pgfpathlineto{\pgfqpoint{4.034508in}{1.986131in}}%
\pgfpathlineto{\pgfqpoint{4.048357in}{1.990235in}}%
\pgfpathlineto{\pgfqpoint{4.062217in}{1.994527in}}%
\pgfpathlineto{\pgfqpoint{4.070227in}{2.006389in}}%
\pgfpathlineto{\pgfqpoint{4.078233in}{2.018200in}}%
\pgfpathlineto{\pgfqpoint{4.086233in}{2.029958in}}%
\pgfpathlineto{\pgfqpoint{4.094229in}{2.041662in}}%
\pgfpathlineto{\pgfqpoint{4.080374in}{2.037186in}}%
\pgfpathlineto{\pgfqpoint{4.066529in}{2.032900in}}%
\pgfpathlineto{\pgfqpoint{4.052696in}{2.028802in}}%
\pgfpathlineto{\pgfqpoint{4.038872in}{2.024893in}}%
\pgfpathlineto{\pgfqpoint{4.030872in}{2.013360in}}%
\pgfpathlineto{\pgfqpoint{4.022867in}{2.001781in}}%
\pgfpathlineto{\pgfqpoint{4.014856in}{1.990158in}}%
\pgfpathlineto{\pgfqpoint{4.006841in}{1.978491in}}%
\pgfpathclose%
\pgfusepath{fill}%
\end{pgfscope}%
\begin{pgfscope}%
\pgfpathrectangle{\pgfqpoint{1.150000in}{0.150000in}}{\pgfqpoint{5.700000in}{5.700000in}}%
\pgfusepath{clip}%
\pgfsetbuttcap%
\pgfsetroundjoin%
\definecolor{currentfill}{rgb}{0.221989,0.339161,0.548752}%
\pgfsetfillcolor{currentfill}%
\pgfsetfillopacity{0.800000}%
\pgfsetlinewidth{0.000000pt}%
\definecolor{currentstroke}{rgb}{0.000000,0.000000,0.000000}%
\pgfsetstrokecolor{currentstroke}%
\pgfsetdash{}{0pt}%
\pgfpathmoveto{\pgfqpoint{2.354933in}{2.482627in}}%
\pgfpathlineto{\pgfqpoint{2.369008in}{2.460257in}}%
\pgfpathlineto{\pgfqpoint{2.383070in}{2.438208in}}%
\pgfpathlineto{\pgfqpoint{2.397119in}{2.416477in}}%
\pgfpathlineto{\pgfqpoint{2.411155in}{2.395060in}}%
\pgfpathlineto{\pgfqpoint{2.420054in}{2.394345in}}%
\pgfpathlineto{\pgfqpoint{2.428934in}{2.393905in}}%
\pgfpathlineto{\pgfqpoint{2.437797in}{2.393735in}}%
\pgfpathlineto{\pgfqpoint{2.446642in}{2.393830in}}%
\pgfpathlineto{\pgfqpoint{2.432655in}{2.414712in}}%
\pgfpathlineto{\pgfqpoint{2.418655in}{2.435907in}}%
\pgfpathlineto{\pgfqpoint{2.404643in}{2.457419in}}%
\pgfpathlineto{\pgfqpoint{2.390618in}{2.479250in}}%
\pgfpathlineto{\pgfqpoint{2.381724in}{2.479678in}}%
\pgfpathlineto{\pgfqpoint{2.372813in}{2.480381in}}%
\pgfpathlineto{\pgfqpoint{2.363882in}{2.481362in}}%
\pgfpathlineto{\pgfqpoint{2.354933in}{2.482627in}}%
\pgfpathclose%
\pgfusepath{fill}%
\end{pgfscope}%
\begin{pgfscope}%
\pgfpathrectangle{\pgfqpoint{1.150000in}{0.150000in}}{\pgfqpoint{5.700000in}{5.700000in}}%
\pgfusepath{clip}%
\pgfsetbuttcap%
\pgfsetroundjoin%
\definecolor{currentfill}{rgb}{0.150148,0.676631,0.506589}%
\pgfsetfillcolor{currentfill}%
\pgfsetfillopacity{0.800000}%
\pgfsetlinewidth{0.000000pt}%
\definecolor{currentstroke}{rgb}{0.000000,0.000000,0.000000}%
\pgfsetstrokecolor{currentstroke}%
\pgfsetdash{}{0pt}%
\pgfpathmoveto{\pgfqpoint{5.827152in}{3.425994in}}%
\pgfpathlineto{\pgfqpoint{5.841949in}{3.438825in}}%
\pgfpathlineto{\pgfqpoint{5.856768in}{3.451834in}}%
\pgfpathlineto{\pgfqpoint{5.871608in}{3.465021in}}%
\pgfpathlineto{\pgfqpoint{5.886470in}{3.478387in}}%
\pgfpathlineto{\pgfqpoint{5.893580in}{3.479390in}}%
\pgfpathlineto{\pgfqpoint{5.900685in}{3.480436in}}%
\pgfpathlineto{\pgfqpoint{5.907784in}{3.481532in}}%
\pgfpathlineto{\pgfqpoint{5.914878in}{3.482684in}}%
\pgfpathlineto{\pgfqpoint{5.900050in}{3.469947in}}%
\pgfpathlineto{\pgfqpoint{5.885244in}{3.457388in}}%
\pgfpathlineto{\pgfqpoint{5.870459in}{3.445006in}}%
\pgfpathlineto{\pgfqpoint{5.855695in}{3.432801in}}%
\pgfpathlineto{\pgfqpoint{5.848567in}{3.431010in}}%
\pgfpathlineto{\pgfqpoint{5.841434in}{3.429283in}}%
\pgfpathlineto{\pgfqpoint{5.834295in}{3.427613in}}%
\pgfpathlineto{\pgfqpoint{5.827152in}{3.425994in}}%
\pgfpathclose%
\pgfusepath{fill}%
\end{pgfscope}%
\begin{pgfscope}%
\pgfpathrectangle{\pgfqpoint{1.150000in}{0.150000in}}{\pgfqpoint{5.700000in}{5.700000in}}%
\pgfusepath{clip}%
\pgfsetbuttcap%
\pgfsetroundjoin%
\definecolor{currentfill}{rgb}{0.280267,0.073417,0.397163}%
\pgfsetfillcolor{currentfill}%
\pgfsetfillopacity{0.800000}%
\pgfsetlinewidth{0.000000pt}%
\definecolor{currentstroke}{rgb}{0.000000,0.000000,0.000000}%
\pgfsetstrokecolor{currentstroke}%
\pgfsetdash{}{0pt}%
\pgfpathmoveto{\pgfqpoint{2.910380in}{1.812682in}}%
\pgfpathlineto{\pgfqpoint{2.924141in}{1.801257in}}%
\pgfpathlineto{\pgfqpoint{2.937899in}{1.790066in}}%
\pgfpathlineto{\pgfqpoint{2.951654in}{1.779109in}}%
\pgfpathlineto{\pgfqpoint{2.965407in}{1.768384in}}%
\pgfpathlineto{\pgfqpoint{2.973904in}{1.772969in}}%
\pgfpathlineto{\pgfqpoint{2.982389in}{1.777742in}}%
\pgfpathlineto{\pgfqpoint{2.990862in}{1.782699in}}%
\pgfpathlineto{\pgfqpoint{2.999325in}{1.787833in}}%
\pgfpathlineto{\pgfqpoint{2.985603in}{1.798054in}}%
\pgfpathlineto{\pgfqpoint{2.971879in}{1.808505in}}%
\pgfpathlineto{\pgfqpoint{2.958153in}{1.819190in}}%
\pgfpathlineto{\pgfqpoint{2.944424in}{1.830110in}}%
\pgfpathlineto{\pgfqpoint{2.935931in}{1.825468in}}%
\pgfpathlineto{\pgfqpoint{2.927426in}{1.821012in}}%
\pgfpathlineto{\pgfqpoint{2.918909in}{1.816749in}}%
\pgfpathlineto{\pgfqpoint{2.910380in}{1.812682in}}%
\pgfpathclose%
\pgfusepath{fill}%
\end{pgfscope}%
\begin{pgfscope}%
\pgfpathrectangle{\pgfqpoint{1.150000in}{0.150000in}}{\pgfqpoint{5.700000in}{5.700000in}}%
\pgfusepath{clip}%
\pgfsetbuttcap%
\pgfsetroundjoin%
\definecolor{currentfill}{rgb}{0.208623,0.367752,0.552675}%
\pgfsetfillcolor{currentfill}%
\pgfsetfillopacity{0.800000}%
\pgfsetlinewidth{0.000000pt}%
\definecolor{currentstroke}{rgb}{0.000000,0.000000,0.000000}%
\pgfsetstrokecolor{currentstroke}%
\pgfsetdash{}{0pt}%
\pgfpathmoveto{\pgfqpoint{4.595101in}{2.480704in}}%
\pgfpathlineto{\pgfqpoint{4.609200in}{2.489465in}}%
\pgfpathlineto{\pgfqpoint{4.623314in}{2.498411in}}%
\pgfpathlineto{\pgfqpoint{4.637443in}{2.507541in}}%
\pgfpathlineto{\pgfqpoint{4.651588in}{2.516856in}}%
\pgfpathlineto{\pgfqpoint{4.659398in}{2.526377in}}%
\pgfpathlineto{\pgfqpoint{4.667202in}{2.535791in}}%
\pgfpathlineto{\pgfqpoint{4.674999in}{2.545102in}}%
\pgfpathlineto{\pgfqpoint{4.682790in}{2.554309in}}%
\pgfpathlineto{\pgfqpoint{4.668651in}{2.545073in}}%
\pgfpathlineto{\pgfqpoint{4.654527in}{2.536021in}}%
\pgfpathlineto{\pgfqpoint{4.640419in}{2.527154in}}%
\pgfpathlineto{\pgfqpoint{4.626325in}{2.518471in}}%
\pgfpathlineto{\pgfqpoint{4.618529in}{2.509173in}}%
\pgfpathlineto{\pgfqpoint{4.610726in}{2.499780in}}%
\pgfpathlineto{\pgfqpoint{4.602916in}{2.490291in}}%
\pgfpathlineto{\pgfqpoint{4.595101in}{2.480704in}}%
\pgfpathclose%
\pgfusepath{fill}%
\end{pgfscope}%
\begin{pgfscope}%
\pgfpathrectangle{\pgfqpoint{1.150000in}{0.150000in}}{\pgfqpoint{5.700000in}{5.700000in}}%
\pgfusepath{clip}%
\pgfsetbuttcap%
\pgfsetroundjoin%
\definecolor{currentfill}{rgb}{0.169646,0.456262,0.558030}%
\pgfsetfillcolor{currentfill}%
\pgfsetfillopacity{0.800000}%
\pgfsetlinewidth{0.000000pt}%
\definecolor{currentstroke}{rgb}{0.000000,0.000000,0.000000}%
\pgfsetstrokecolor{currentstroke}%
\pgfsetdash{}{0pt}%
\pgfpathmoveto{\pgfqpoint{4.889267in}{2.734351in}}%
\pgfpathlineto{\pgfqpoint{4.903529in}{2.744801in}}%
\pgfpathlineto{\pgfqpoint{4.917808in}{2.755434in}}%
\pgfpathlineto{\pgfqpoint{4.932104in}{2.766250in}}%
\pgfpathlineto{\pgfqpoint{4.946418in}{2.777249in}}%
\pgfpathlineto{\pgfqpoint{4.954094in}{2.784672in}}%
\pgfpathlineto{\pgfqpoint{4.961763in}{2.791996in}}%
\pgfpathlineto{\pgfqpoint{4.969425in}{2.799221in}}%
\pgfpathlineto{\pgfqpoint{4.977079in}{2.806352in}}%
\pgfpathlineto{\pgfqpoint{4.962776in}{2.795567in}}%
\pgfpathlineto{\pgfqpoint{4.948489in}{2.784966in}}%
\pgfpathlineto{\pgfqpoint{4.934220in}{2.774547in}}%
\pgfpathlineto{\pgfqpoint{4.919968in}{2.764310in}}%
\pgfpathlineto{\pgfqpoint{4.912303in}{2.756953in}}%
\pgfpathlineto{\pgfqpoint{4.904632in}{2.749510in}}%
\pgfpathlineto{\pgfqpoint{4.896953in}{2.741977in}}%
\pgfpathlineto{\pgfqpoint{4.889267in}{2.734351in}}%
\pgfpathclose%
\pgfusepath{fill}%
\end{pgfscope}%
\begin{pgfscope}%
\pgfpathrectangle{\pgfqpoint{1.150000in}{0.150000in}}{\pgfqpoint{5.700000in}{5.700000in}}%
\pgfusepath{clip}%
\pgfsetbuttcap%
\pgfsetroundjoin%
\definecolor{currentfill}{rgb}{0.269944,0.014625,0.341379}%
\pgfsetfillcolor{currentfill}%
\pgfsetfillopacity{0.800000}%
\pgfsetlinewidth{0.000000pt}%
\definecolor{currentstroke}{rgb}{0.000000,0.000000,0.000000}%
\pgfsetstrokecolor{currentstroke}%
\pgfsetdash{}{0pt}%
\pgfpathmoveto{\pgfqpoint{3.394812in}{1.677523in}}%
\pgfpathlineto{\pgfqpoint{3.408513in}{1.673614in}}%
\pgfpathlineto{\pgfqpoint{3.422218in}{1.669908in}}%
\pgfpathlineto{\pgfqpoint{3.435927in}{1.666406in}}%
\pgfpathlineto{\pgfqpoint{3.449640in}{1.663105in}}%
\pgfpathlineto{\pgfqpoint{3.457866in}{1.672555in}}%
\pgfpathlineto{\pgfqpoint{3.466085in}{1.682084in}}%
\pgfpathlineto{\pgfqpoint{3.474297in}{1.691688in}}%
\pgfpathlineto{\pgfqpoint{3.482503in}{1.701364in}}%
\pgfpathlineto{\pgfqpoint{3.468805in}{1.704261in}}%
\pgfpathlineto{\pgfqpoint{3.455112in}{1.707361in}}%
\pgfpathlineto{\pgfqpoint{3.441424in}{1.710663in}}%
\pgfpathlineto{\pgfqpoint{3.427739in}{1.714169in}}%
\pgfpathlineto{\pgfqpoint{3.419518in}{1.704885in}}%
\pgfpathlineto{\pgfqpoint{3.411289in}{1.695680in}}%
\pgfpathlineto{\pgfqpoint{3.403054in}{1.686558in}}%
\pgfpathlineto{\pgfqpoint{3.394812in}{1.677523in}}%
\pgfpathclose%
\pgfusepath{fill}%
\end{pgfscope}%
\begin{pgfscope}%
\pgfpathrectangle{\pgfqpoint{1.150000in}{0.150000in}}{\pgfqpoint{5.700000in}{5.700000in}}%
\pgfusepath{clip}%
\pgfsetbuttcap%
\pgfsetroundjoin%
\definecolor{currentfill}{rgb}{0.175707,0.697900,0.491033}%
\pgfsetfillcolor{currentfill}%
\pgfsetfillopacity{0.800000}%
\pgfsetlinewidth{0.000000pt}%
\definecolor{currentstroke}{rgb}{0.000000,0.000000,0.000000}%
\pgfsetstrokecolor{currentstroke}%
\pgfsetdash{}{0pt}%
\pgfpathmoveto{\pgfqpoint{5.914878in}{3.482684in}}%
\pgfpathlineto{\pgfqpoint{5.929727in}{3.495598in}}%
\pgfpathlineto{\pgfqpoint{5.944597in}{3.508690in}}%
\pgfpathlineto{\pgfqpoint{5.959490in}{3.521960in}}%
\pgfpathlineto{\pgfqpoint{5.974404in}{3.535408in}}%
\pgfpathlineto{\pgfqpoint{5.981457in}{3.535972in}}%
\pgfpathlineto{\pgfqpoint{5.988505in}{3.536600in}}%
\pgfpathlineto{\pgfqpoint{5.995549in}{3.537297in}}%
\pgfpathlineto{\pgfqpoint{6.002588in}{3.538072in}}%
\pgfpathlineto{\pgfqpoint{5.987711in}{3.525288in}}%
\pgfpathlineto{\pgfqpoint{5.972855in}{3.512681in}}%
\pgfpathlineto{\pgfqpoint{5.958021in}{3.500251in}}%
\pgfpathlineto{\pgfqpoint{5.943208in}{3.487996in}}%
\pgfpathlineto{\pgfqpoint{5.936132in}{3.486549in}}%
\pgfpathlineto{\pgfqpoint{5.929052in}{3.485186in}}%
\pgfpathlineto{\pgfqpoint{5.921967in}{3.483900in}}%
\pgfpathlineto{\pgfqpoint{5.914878in}{3.482684in}}%
\pgfpathclose%
\pgfusepath{fill}%
\end{pgfscope}%
\begin{pgfscope}%
\pgfpathrectangle{\pgfqpoint{1.150000in}{0.150000in}}{\pgfqpoint{5.700000in}{5.700000in}}%
\pgfusepath{clip}%
\pgfsetbuttcap%
\pgfsetroundjoin%
\definecolor{currentfill}{rgb}{0.226397,0.728888,0.462789}%
\pgfsetfillcolor{currentfill}%
\pgfsetfillopacity{0.800000}%
\pgfsetlinewidth{0.000000pt}%
\definecolor{currentstroke}{rgb}{0.000000,0.000000,0.000000}%
\pgfsetstrokecolor{currentstroke}%
\pgfsetdash{}{0pt}%
\pgfpathmoveto{\pgfqpoint{6.090279in}{3.592211in}}%
\pgfpathlineto{\pgfqpoint{6.105227in}{3.605183in}}%
\pgfpathlineto{\pgfqpoint{6.120196in}{3.618332in}}%
\pgfpathlineto{\pgfqpoint{6.135188in}{3.631657in}}%
\pgfpathlineto{\pgfqpoint{6.142139in}{3.631673in}}%
\pgfpathlineto{\pgfqpoint{6.149087in}{3.631798in}}%
\pgfpathlineto{\pgfqpoint{6.156033in}{3.632040in}}%
\pgfpathlineto{\pgfqpoint{6.162976in}{3.632406in}}%
\pgfpathlineto{\pgfqpoint{6.148026in}{3.619812in}}%
\pgfpathlineto{\pgfqpoint{6.133098in}{3.607393in}}%
\pgfpathlineto{\pgfqpoint{6.118193in}{3.595150in}}%
\pgfpathlineto{\pgfqpoint{6.111218in}{3.594228in}}%
\pgfpathlineto{\pgfqpoint{6.104241in}{3.593436in}}%
\pgfpathlineto{\pgfqpoint{6.097261in}{3.592767in}}%
\pgfpathlineto{\pgfqpoint{6.090279in}{3.592211in}}%
\pgfpathclose%
\pgfusepath{fill}%
\end{pgfscope}%
\begin{pgfscope}%
\pgfpathrectangle{\pgfqpoint{1.150000in}{0.150000in}}{\pgfqpoint{5.700000in}{5.700000in}}%
\pgfusepath{clip}%
\pgfsetbuttcap%
\pgfsetroundjoin%
\definecolor{currentfill}{rgb}{0.131172,0.555899,0.552459}%
\pgfsetfillcolor{currentfill}%
\pgfsetfillopacity{0.800000}%
\pgfsetlinewidth{0.000000pt}%
\definecolor{currentstroke}{rgb}{0.000000,0.000000,0.000000}%
\pgfsetstrokecolor{currentstroke}%
\pgfsetdash{}{0pt}%
\pgfpathmoveto{\pgfqpoint{5.270948in}{3.039313in}}%
\pgfpathlineto{\pgfqpoint{5.285433in}{3.051262in}}%
\pgfpathlineto{\pgfqpoint{5.299937in}{3.063393in}}%
\pgfpathlineto{\pgfqpoint{5.314461in}{3.075704in}}%
\pgfpathlineto{\pgfqpoint{5.329003in}{3.088198in}}%
\pgfpathlineto{\pgfqpoint{5.336469in}{3.092725in}}%
\pgfpathlineto{\pgfqpoint{5.343927in}{3.097188in}}%
\pgfpathlineto{\pgfqpoint{5.351378in}{3.101590in}}%
\pgfpathlineto{\pgfqpoint{5.358820in}{3.105938in}}%
\pgfpathlineto{\pgfqpoint{5.344296in}{3.093832in}}%
\pgfpathlineto{\pgfqpoint{5.329791in}{3.081907in}}%
\pgfpathlineto{\pgfqpoint{5.315305in}{3.070162in}}%
\pgfpathlineto{\pgfqpoint{5.300838in}{3.058598in}}%
\pgfpathlineto{\pgfqpoint{5.293377in}{3.053853in}}%
\pgfpathlineto{\pgfqpoint{5.285908in}{3.049059in}}%
\pgfpathlineto{\pgfqpoint{5.278432in}{3.044214in}}%
\pgfpathlineto{\pgfqpoint{5.270948in}{3.039313in}}%
\pgfpathclose%
\pgfusepath{fill}%
\end{pgfscope}%
\begin{pgfscope}%
\pgfpathrectangle{\pgfqpoint{1.150000in}{0.150000in}}{\pgfqpoint{5.700000in}{5.700000in}}%
\pgfusepath{clip}%
\pgfsetbuttcap%
\pgfsetroundjoin%
\definecolor{currentfill}{rgb}{0.202219,0.715272,0.476084}%
\pgfsetfillcolor{currentfill}%
\pgfsetfillopacity{0.800000}%
\pgfsetlinewidth{0.000000pt}%
\definecolor{currentstroke}{rgb}{0.000000,0.000000,0.000000}%
\pgfsetstrokecolor{currentstroke}%
\pgfsetdash{}{0pt}%
\pgfpathmoveto{\pgfqpoint{6.002588in}{3.538072in}}%
\pgfpathlineto{\pgfqpoint{6.017487in}{3.551033in}}%
\pgfpathlineto{\pgfqpoint{6.032408in}{3.564172in}}%
\pgfpathlineto{\pgfqpoint{6.047351in}{3.577488in}}%
\pgfpathlineto{\pgfqpoint{6.062317in}{3.590981in}}%
\pgfpathlineto{\pgfqpoint{6.069313in}{3.591155in}}%
\pgfpathlineto{\pgfqpoint{6.076305in}{3.591413in}}%
\pgfpathlineto{\pgfqpoint{6.083294in}{3.591762in}}%
\pgfpathlineto{\pgfqpoint{6.090279in}{3.592211in}}%
\pgfpathlineto{\pgfqpoint{6.075354in}{3.579416in}}%
\pgfpathlineto{\pgfqpoint{6.060450in}{3.566797in}}%
\pgfpathlineto{\pgfqpoint{6.045569in}{3.554354in}}%
\pgfpathlineto{\pgfqpoint{6.030708in}{3.542087in}}%
\pgfpathlineto{\pgfqpoint{6.023683in}{3.540931in}}%
\pgfpathlineto{\pgfqpoint{6.016655in}{3.539882in}}%
\pgfpathlineto{\pgfqpoint{6.009623in}{3.538931in}}%
\pgfpathlineto{\pgfqpoint{6.002588in}{3.538072in}}%
\pgfpathclose%
\pgfusepath{fill}%
\end{pgfscope}%
\begin{pgfscope}%
\pgfpathrectangle{\pgfqpoint{1.150000in}{0.150000in}}{\pgfqpoint{5.700000in}{5.700000in}}%
\pgfusepath{clip}%
\pgfsetbuttcap%
\pgfsetroundjoin%
\definecolor{currentfill}{rgb}{0.274128,0.199721,0.498911}%
\pgfsetfillcolor{currentfill}%
\pgfsetfillopacity{0.800000}%
\pgfsetlinewidth{0.000000pt}%
\definecolor{currentstroke}{rgb}{0.000000,0.000000,0.000000}%
\pgfsetstrokecolor{currentstroke}%
\pgfsetdash{}{0pt}%
\pgfpathmoveto{\pgfqpoint{4.094229in}{2.041662in}}%
\pgfpathlineto{\pgfqpoint{4.108095in}{2.046325in}}%
\pgfpathlineto{\pgfqpoint{4.121972in}{2.051177in}}%
\pgfpathlineto{\pgfqpoint{4.135861in}{2.056216in}}%
\pgfpathlineto{\pgfqpoint{4.149761in}{2.061443in}}%
\pgfpathlineto{\pgfqpoint{4.157748in}{2.073254in}}%
\pgfpathlineto{\pgfqpoint{4.165730in}{2.085002in}}%
\pgfpathlineto{\pgfqpoint{4.173706in}{2.096683in}}%
\pgfpathlineto{\pgfqpoint{4.181678in}{2.108297in}}%
\pgfpathlineto{\pgfqpoint{4.167782in}{2.102919in}}%
\pgfpathlineto{\pgfqpoint{4.153898in}{2.097729in}}%
\pgfpathlineto{\pgfqpoint{4.140024in}{2.092726in}}%
\pgfpathlineto{\pgfqpoint{4.126163in}{2.087912in}}%
\pgfpathlineto{\pgfqpoint{4.118187in}{2.076436in}}%
\pgfpathlineto{\pgfqpoint{4.110206in}{2.064902in}}%
\pgfpathlineto{\pgfqpoint{4.102220in}{2.053310in}}%
\pgfpathlineto{\pgfqpoint{4.094229in}{2.041662in}}%
\pgfpathclose%
\pgfusepath{fill}%
\end{pgfscope}%
\begin{pgfscope}%
\pgfpathrectangle{\pgfqpoint{1.150000in}{0.150000in}}{\pgfqpoint{5.700000in}{5.700000in}}%
\pgfusepath{clip}%
\pgfsetbuttcap%
\pgfsetroundjoin%
\definecolor{currentfill}{rgb}{0.239346,0.300855,0.540844}%
\pgfsetfillcolor{currentfill}%
\pgfsetfillopacity{0.800000}%
\pgfsetlinewidth{0.000000pt}%
\definecolor{currentstroke}{rgb}{0.000000,0.000000,0.000000}%
\pgfsetstrokecolor{currentstroke}%
\pgfsetdash{}{0pt}%
\pgfpathmoveto{\pgfqpoint{4.388458in}{2.294009in}}%
\pgfpathlineto{\pgfqpoint{4.402457in}{2.301326in}}%
\pgfpathlineto{\pgfqpoint{4.416469in}{2.308828in}}%
\pgfpathlineto{\pgfqpoint{4.430494in}{2.316515in}}%
\pgfpathlineto{\pgfqpoint{4.444534in}{2.324388in}}%
\pgfpathlineto{\pgfqpoint{4.452426in}{2.335176in}}%
\pgfpathlineto{\pgfqpoint{4.460312in}{2.345866in}}%
\pgfpathlineto{\pgfqpoint{4.468193in}{2.356459in}}%
\pgfpathlineto{\pgfqpoint{4.476067in}{2.366956in}}%
\pgfpathlineto{\pgfqpoint{4.462032in}{2.359062in}}%
\pgfpathlineto{\pgfqpoint{4.448010in}{2.351354in}}%
\pgfpathlineto{\pgfqpoint{4.434002in}{2.343830in}}%
\pgfpathlineto{\pgfqpoint{4.420008in}{2.336493in}}%
\pgfpathlineto{\pgfqpoint{4.412129in}{2.326005in}}%
\pgfpathlineto{\pgfqpoint{4.404244in}{2.315429in}}%
\pgfpathlineto{\pgfqpoint{4.396354in}{2.304764in}}%
\pgfpathlineto{\pgfqpoint{4.388458in}{2.294009in}}%
\pgfpathclose%
\pgfusepath{fill}%
\end{pgfscope}%
\begin{pgfscope}%
\pgfpathrectangle{\pgfqpoint{1.150000in}{0.150000in}}{\pgfqpoint{5.700000in}{5.700000in}}%
\pgfusepath{clip}%
\pgfsetbuttcap%
\pgfsetroundjoin%
\definecolor{currentfill}{rgb}{0.277941,0.056324,0.381191}%
\pgfsetfillcolor{currentfill}%
\pgfsetfillopacity{0.800000}%
\pgfsetlinewidth{0.000000pt}%
\definecolor{currentstroke}{rgb}{0.000000,0.000000,0.000000}%
\pgfsetstrokecolor{currentstroke}%
\pgfsetdash{}{0pt}%
\pgfpathmoveto{\pgfqpoint{2.965407in}{1.768384in}}%
\pgfpathlineto{\pgfqpoint{2.979158in}{1.757890in}}%
\pgfpathlineto{\pgfqpoint{2.992906in}{1.747625in}}%
\pgfpathlineto{\pgfqpoint{3.006654in}{1.737588in}}%
\pgfpathlineto{\pgfqpoint{3.020399in}{1.727778in}}%
\pgfpathlineto{\pgfqpoint{3.028866in}{1.732879in}}%
\pgfpathlineto{\pgfqpoint{3.037321in}{1.738160in}}%
\pgfpathlineto{\pgfqpoint{3.045765in}{1.743616in}}%
\pgfpathlineto{\pgfqpoint{3.054198in}{1.749242in}}%
\pgfpathlineto{\pgfqpoint{3.040482in}{1.758549in}}%
\pgfpathlineto{\pgfqpoint{3.026764in}{1.768082in}}%
\pgfpathlineto{\pgfqpoint{3.013045in}{1.777843in}}%
\pgfpathlineto{\pgfqpoint{2.999325in}{1.787833in}}%
\pgfpathlineto{\pgfqpoint{2.990862in}{1.782699in}}%
\pgfpathlineto{\pgfqpoint{2.982389in}{1.777742in}}%
\pgfpathlineto{\pgfqpoint{2.973904in}{1.772969in}}%
\pgfpathlineto{\pgfqpoint{2.965407in}{1.768384in}}%
\pgfpathclose%
\pgfusepath{fill}%
\end{pgfscope}%
\begin{pgfscope}%
\pgfpathrectangle{\pgfqpoint{1.150000in}{0.150000in}}{\pgfqpoint{5.700000in}{5.700000in}}%
\pgfusepath{clip}%
\pgfsetbuttcap%
\pgfsetroundjoin%
\definecolor{currentfill}{rgb}{0.204903,0.375746,0.553533}%
\pgfsetfillcolor{currentfill}%
\pgfsetfillopacity{0.800000}%
\pgfsetlinewidth{0.000000pt}%
\definecolor{currentstroke}{rgb}{0.000000,0.000000,0.000000}%
\pgfsetstrokecolor{currentstroke}%
\pgfsetdash{}{0pt}%
\pgfpathmoveto{\pgfqpoint{2.298489in}{2.575374in}}%
\pgfpathlineto{\pgfqpoint{2.312622in}{2.551691in}}%
\pgfpathlineto{\pgfqpoint{2.326740in}{2.528341in}}%
\pgfpathlineto{\pgfqpoint{2.340844in}{2.505320in}}%
\pgfpathlineto{\pgfqpoint{2.354933in}{2.482627in}}%
\pgfpathlineto{\pgfqpoint{2.363882in}{2.481362in}}%
\pgfpathlineto{\pgfqpoint{2.372813in}{2.480381in}}%
\pgfpathlineto{\pgfqpoint{2.381724in}{2.479678in}}%
\pgfpathlineto{\pgfqpoint{2.390618in}{2.479250in}}%
\pgfpathlineto{\pgfqpoint{2.376579in}{2.501403in}}%
\pgfpathlineto{\pgfqpoint{2.362527in}{2.523882in}}%
\pgfpathlineto{\pgfqpoint{2.348461in}{2.546690in}}%
\pgfpathlineto{\pgfqpoint{2.334380in}{2.569829in}}%
\pgfpathlineto{\pgfqpoint{2.325436in}{2.570786in}}%
\pgfpathlineto{\pgfqpoint{2.316473in}{2.572025in}}%
\pgfpathlineto{\pgfqpoint{2.307491in}{2.573553in}}%
\pgfpathlineto{\pgfqpoint{2.298489in}{2.575374in}}%
\pgfpathclose%
\pgfusepath{fill}%
\end{pgfscope}%
\begin{pgfscope}%
\pgfpathrectangle{\pgfqpoint{1.150000in}{0.150000in}}{\pgfqpoint{5.700000in}{5.700000in}}%
\pgfusepath{clip}%
\pgfsetbuttcap%
\pgfsetroundjoin%
\definecolor{currentfill}{rgb}{0.124395,0.578002,0.548287}%
\pgfsetfillcolor{currentfill}%
\pgfsetfillopacity{0.800000}%
\pgfsetlinewidth{0.000000pt}%
\definecolor{currentstroke}{rgb}{0.000000,0.000000,0.000000}%
\pgfsetstrokecolor{currentstroke}%
\pgfsetdash{}{0pt}%
\pgfpathmoveto{\pgfqpoint{5.358820in}{3.105938in}}%
\pgfpathlineto{\pgfqpoint{5.373364in}{3.118224in}}%
\pgfpathlineto{\pgfqpoint{5.387927in}{3.130692in}}%
\pgfpathlineto{\pgfqpoint{5.402510in}{3.143341in}}%
\pgfpathlineto{\pgfqpoint{5.417113in}{3.156171in}}%
\pgfpathlineto{\pgfqpoint{5.424528in}{3.160057in}}%
\pgfpathlineto{\pgfqpoint{5.431935in}{3.163890in}}%
\pgfpathlineto{\pgfqpoint{5.439335in}{3.167673in}}%
\pgfpathlineto{\pgfqpoint{5.446727in}{3.171411in}}%
\pgfpathlineto{\pgfqpoint{5.432145in}{3.159004in}}%
\pgfpathlineto{\pgfqpoint{5.417582in}{3.146777in}}%
\pgfpathlineto{\pgfqpoint{5.403039in}{3.134730in}}%
\pgfpathlineto{\pgfqpoint{5.388515in}{3.122864in}}%
\pgfpathlineto{\pgfqpoint{5.381103in}{3.118693in}}%
\pgfpathlineto{\pgfqpoint{5.373683in}{3.114484in}}%
\pgfpathlineto{\pgfqpoint{5.366255in}{3.110234in}}%
\pgfpathlineto{\pgfqpoint{5.358820in}{3.105938in}}%
\pgfpathclose%
\pgfusepath{fill}%
\end{pgfscope}%
\begin{pgfscope}%
\pgfpathrectangle{\pgfqpoint{1.150000in}{0.150000in}}{\pgfqpoint{5.700000in}{5.700000in}}%
\pgfusepath{clip}%
\pgfsetbuttcap%
\pgfsetroundjoin%
\definecolor{currentfill}{rgb}{0.269944,0.014625,0.341379}%
\pgfsetfillcolor{currentfill}%
\pgfsetfillopacity{0.800000}%
\pgfsetlinewidth{0.000000pt}%
\definecolor{currentstroke}{rgb}{0.000000,0.000000,0.000000}%
\pgfsetstrokecolor{currentstroke}%
\pgfsetdash{}{0pt}%
\pgfpathmoveto{\pgfqpoint{3.163915in}{1.682787in}}%
\pgfpathlineto{\pgfqpoint{3.177631in}{1.675463in}}%
\pgfpathlineto{\pgfqpoint{3.191348in}{1.668354in}}%
\pgfpathlineto{\pgfqpoint{3.205065in}{1.661458in}}%
\pgfpathlineto{\pgfqpoint{3.218784in}{1.654775in}}%
\pgfpathlineto{\pgfqpoint{3.227131in}{1.662010in}}%
\pgfpathlineto{\pgfqpoint{3.235469in}{1.669380in}}%
\pgfpathlineto{\pgfqpoint{3.243798in}{1.676883in}}%
\pgfpathlineto{\pgfqpoint{3.252119in}{1.684513in}}%
\pgfpathlineto{\pgfqpoint{3.238422in}{1.690728in}}%
\pgfpathlineto{\pgfqpoint{3.224728in}{1.697156in}}%
\pgfpathlineto{\pgfqpoint{3.211035in}{1.703798in}}%
\pgfpathlineto{\pgfqpoint{3.197343in}{1.710653in}}%
\pgfpathlineto{\pgfqpoint{3.189000in}{1.703479in}}%
\pgfpathlineto{\pgfqpoint{3.180647in}{1.696440in}}%
\pgfpathlineto{\pgfqpoint{3.172286in}{1.689541in}}%
\pgfpathlineto{\pgfqpoint{3.163915in}{1.682787in}}%
\pgfpathclose%
\pgfusepath{fill}%
\end{pgfscope}%
\begin{pgfscope}%
\pgfpathrectangle{\pgfqpoint{1.150000in}{0.150000in}}{\pgfqpoint{5.700000in}{5.700000in}}%
\pgfusepath{clip}%
\pgfsetbuttcap%
\pgfsetroundjoin%
\definecolor{currentfill}{rgb}{0.159194,0.482237,0.558073}%
\pgfsetfillcolor{currentfill}%
\pgfsetfillopacity{0.800000}%
\pgfsetlinewidth{0.000000pt}%
\definecolor{currentstroke}{rgb}{0.000000,0.000000,0.000000}%
\pgfsetstrokecolor{currentstroke}%
\pgfsetdash{}{0pt}%
\pgfpathmoveto{\pgfqpoint{4.977079in}{2.806352in}}%
\pgfpathlineto{\pgfqpoint{4.991400in}{2.817319in}}%
\pgfpathlineto{\pgfqpoint{5.005739in}{2.828468in}}%
\pgfpathlineto{\pgfqpoint{5.020095in}{2.839801in}}%
\pgfpathlineto{\pgfqpoint{5.034469in}{2.851316in}}%
\pgfpathlineto{\pgfqpoint{5.042105in}{2.858118in}}%
\pgfpathlineto{\pgfqpoint{5.049733in}{2.864822in}}%
\pgfpathlineto{\pgfqpoint{5.057353in}{2.871431in}}%
\pgfpathlineto{\pgfqpoint{5.064966in}{2.877948in}}%
\pgfpathlineto{\pgfqpoint{5.050604in}{2.866683in}}%
\pgfpathlineto{\pgfqpoint{5.036259in}{2.855599in}}%
\pgfpathlineto{\pgfqpoint{5.021932in}{2.844698in}}%
\pgfpathlineto{\pgfqpoint{5.007622in}{2.833979in}}%
\pgfpathlineto{\pgfqpoint{4.999997in}{2.827200in}}%
\pgfpathlineto{\pgfqpoint{4.992365in}{2.820338in}}%
\pgfpathlineto{\pgfqpoint{4.984726in}{2.813390in}}%
\pgfpathlineto{\pgfqpoint{4.977079in}{2.806352in}}%
\pgfpathclose%
\pgfusepath{fill}%
\end{pgfscope}%
\begin{pgfscope}%
\pgfpathrectangle{\pgfqpoint{1.150000in}{0.150000in}}{\pgfqpoint{5.700000in}{5.700000in}}%
\pgfusepath{clip}%
\pgfsetbuttcap%
\pgfsetroundjoin%
\definecolor{currentfill}{rgb}{0.194100,0.399323,0.555565}%
\pgfsetfillcolor{currentfill}%
\pgfsetfillopacity{0.800000}%
\pgfsetlinewidth{0.000000pt}%
\definecolor{currentstroke}{rgb}{0.000000,0.000000,0.000000}%
\pgfsetstrokecolor{currentstroke}%
\pgfsetdash{}{0pt}%
\pgfpathmoveto{\pgfqpoint{4.682790in}{2.554309in}}%
\pgfpathlineto{\pgfqpoint{4.696944in}{2.563729in}}%
\pgfpathlineto{\pgfqpoint{4.711115in}{2.573333in}}%
\pgfpathlineto{\pgfqpoint{4.725301in}{2.583121in}}%
\pgfpathlineto{\pgfqpoint{4.739503in}{2.593093in}}%
\pgfpathlineto{\pgfqpoint{4.747281in}{2.602098in}}%
\pgfpathlineto{\pgfqpoint{4.755052in}{2.610995in}}%
\pgfpathlineto{\pgfqpoint{4.762816in}{2.619785in}}%
\pgfpathlineto{\pgfqpoint{4.770574in}{2.628469in}}%
\pgfpathlineto{\pgfqpoint{4.756378in}{2.618610in}}%
\pgfpathlineto{\pgfqpoint{4.742198in}{2.608934in}}%
\pgfpathlineto{\pgfqpoint{4.728034in}{2.599443in}}%
\pgfpathlineto{\pgfqpoint{4.713886in}{2.590135in}}%
\pgfpathlineto{\pgfqpoint{4.706122in}{2.581326in}}%
\pgfpathlineto{\pgfqpoint{4.698351in}{2.572419in}}%
\pgfpathlineto{\pgfqpoint{4.690574in}{2.563414in}}%
\pgfpathlineto{\pgfqpoint{4.682790in}{2.554309in}}%
\pgfpathclose%
\pgfusepath{fill}%
\end{pgfscope}%
\begin{pgfscope}%
\pgfpathrectangle{\pgfqpoint{1.150000in}{0.150000in}}{\pgfqpoint{5.700000in}{5.700000in}}%
\pgfusepath{clip}%
\pgfsetbuttcap%
\pgfsetroundjoin%
\definecolor{currentfill}{rgb}{0.268510,0.009605,0.335427}%
\pgfsetfillcolor{currentfill}%
\pgfsetfillopacity{0.800000}%
\pgfsetlinewidth{0.000000pt}%
\definecolor{currentstroke}{rgb}{0.000000,0.000000,0.000000}%
\pgfsetstrokecolor{currentstroke}%
\pgfsetdash{}{0pt}%
\pgfpathmoveto{\pgfqpoint{3.306923in}{1.661759in}}%
\pgfpathlineto{\pgfqpoint{3.320630in}{1.656592in}}%
\pgfpathlineto{\pgfqpoint{3.334340in}{1.651633in}}%
\pgfpathlineto{\pgfqpoint{3.348053in}{1.646879in}}%
\pgfpathlineto{\pgfqpoint{3.361769in}{1.642331in}}%
\pgfpathlineto{\pgfqpoint{3.370041in}{1.650979in}}%
\pgfpathlineto{\pgfqpoint{3.378305in}{1.659729in}}%
\pgfpathlineto{\pgfqpoint{3.386562in}{1.668579in}}%
\pgfpathlineto{\pgfqpoint{3.394812in}{1.677523in}}%
\pgfpathlineto{\pgfqpoint{3.381114in}{1.681636in}}%
\pgfpathlineto{\pgfqpoint{3.367420in}{1.685955in}}%
\pgfpathlineto{\pgfqpoint{3.353729in}{1.690480in}}%
\pgfpathlineto{\pgfqpoint{3.340042in}{1.695211in}}%
\pgfpathlineto{\pgfqpoint{3.331774in}{1.686690in}}%
\pgfpathlineto{\pgfqpoint{3.323498in}{1.678271in}}%
\pgfpathlineto{\pgfqpoint{3.315215in}{1.669960in}}%
\pgfpathlineto{\pgfqpoint{3.306923in}{1.661759in}}%
\pgfpathclose%
\pgfusepath{fill}%
\end{pgfscope}%
\begin{pgfscope}%
\pgfpathrectangle{\pgfqpoint{1.150000in}{0.150000in}}{\pgfqpoint{5.700000in}{5.700000in}}%
\pgfusepath{clip}%
\pgfsetbuttcap%
\pgfsetroundjoin%
\definecolor{currentfill}{rgb}{0.280267,0.073417,0.397163}%
\pgfsetfillcolor{currentfill}%
\pgfsetfillopacity{0.800000}%
\pgfsetlinewidth{0.000000pt}%
\definecolor{currentstroke}{rgb}{0.000000,0.000000,0.000000}%
\pgfsetstrokecolor{currentstroke}%
\pgfsetdash{}{0pt}%
\pgfpathmoveto{\pgfqpoint{3.712408in}{1.770207in}}%
\pgfpathlineto{\pgfqpoint{3.726159in}{1.770589in}}%
\pgfpathlineto{\pgfqpoint{3.739917in}{1.771164in}}%
\pgfpathlineto{\pgfqpoint{3.753683in}{1.771933in}}%
\pgfpathlineto{\pgfqpoint{3.767456in}{1.772893in}}%
\pgfpathlineto{\pgfqpoint{3.775563in}{1.784341in}}%
\pgfpathlineto{\pgfqpoint{3.783665in}{1.795796in}}%
\pgfpathlineto{\pgfqpoint{3.791762in}{1.807256in}}%
\pgfpathlineto{\pgfqpoint{3.799854in}{1.818717in}}%
\pgfpathlineto{\pgfqpoint{3.786089in}{1.817447in}}%
\pgfpathlineto{\pgfqpoint{3.772332in}{1.816370in}}%
\pgfpathlineto{\pgfqpoint{3.758583in}{1.815485in}}%
\pgfpathlineto{\pgfqpoint{3.744842in}{1.814795in}}%
\pgfpathlineto{\pgfqpoint{3.736741in}{1.803630in}}%
\pgfpathlineto{\pgfqpoint{3.728636in}{1.792476in}}%
\pgfpathlineto{\pgfqpoint{3.720525in}{1.781334in}}%
\pgfpathlineto{\pgfqpoint{3.712408in}{1.770207in}}%
\pgfpathclose%
\pgfusepath{fill}%
\end{pgfscope}%
\begin{pgfscope}%
\pgfpathrectangle{\pgfqpoint{1.150000in}{0.150000in}}{\pgfqpoint{5.700000in}{5.700000in}}%
\pgfusepath{clip}%
\pgfsetbuttcap%
\pgfsetroundjoin%
\definecolor{currentfill}{rgb}{0.277018,0.050344,0.375715}%
\pgfsetfillcolor{currentfill}%
\pgfsetfillopacity{0.800000}%
\pgfsetlinewidth{0.000000pt}%
\definecolor{currentstroke}{rgb}{0.000000,0.000000,0.000000}%
\pgfsetstrokecolor{currentstroke}%
\pgfsetdash{}{0pt}%
\pgfpathmoveto{\pgfqpoint{3.624916in}{1.727692in}}%
\pgfpathlineto{\pgfqpoint{3.638649in}{1.726954in}}%
\pgfpathlineto{\pgfqpoint{3.652389in}{1.726412in}}%
\pgfpathlineto{\pgfqpoint{3.666136in}{1.726065in}}%
\pgfpathlineto{\pgfqpoint{3.679889in}{1.725912in}}%
\pgfpathlineto{\pgfqpoint{3.688027in}{1.736948in}}%
\pgfpathlineto{\pgfqpoint{3.696160in}{1.748011in}}%
\pgfpathlineto{\pgfqpoint{3.704287in}{1.759099in}}%
\pgfpathlineto{\pgfqpoint{3.712408in}{1.770207in}}%
\pgfpathlineto{\pgfqpoint{3.698665in}{1.770019in}}%
\pgfpathlineto{\pgfqpoint{3.684929in}{1.770026in}}%
\pgfpathlineto{\pgfqpoint{3.671200in}{1.770228in}}%
\pgfpathlineto{\pgfqpoint{3.657477in}{1.770626in}}%
\pgfpathlineto{\pgfqpoint{3.649345in}{1.759846in}}%
\pgfpathlineto{\pgfqpoint{3.641208in}{1.749095in}}%
\pgfpathlineto{\pgfqpoint{3.633065in}{1.738376in}}%
\pgfpathlineto{\pgfqpoint{3.624916in}{1.727692in}}%
\pgfpathclose%
\pgfusepath{fill}%
\end{pgfscope}%
\begin{pgfscope}%
\pgfpathrectangle{\pgfqpoint{1.150000in}{0.150000in}}{\pgfqpoint{5.700000in}{5.700000in}}%
\pgfusepath{clip}%
\pgfsetbuttcap%
\pgfsetroundjoin%
\definecolor{currentfill}{rgb}{0.266580,0.228262,0.514349}%
\pgfsetfillcolor{currentfill}%
\pgfsetfillopacity{0.800000}%
\pgfsetlinewidth{0.000000pt}%
\definecolor{currentstroke}{rgb}{0.000000,0.000000,0.000000}%
\pgfsetstrokecolor{currentstroke}%
\pgfsetdash{}{0pt}%
\pgfpathmoveto{\pgfqpoint{4.181678in}{2.108297in}}%
\pgfpathlineto{\pgfqpoint{4.195586in}{2.113862in}}%
\pgfpathlineto{\pgfqpoint{4.209506in}{2.119615in}}%
\pgfpathlineto{\pgfqpoint{4.223438in}{2.125554in}}%
\pgfpathlineto{\pgfqpoint{4.237382in}{2.131680in}}%
\pgfpathlineto{\pgfqpoint{4.245345in}{2.143357in}}%
\pgfpathlineto{\pgfqpoint{4.253304in}{2.154958in}}%
\pgfpathlineto{\pgfqpoint{4.261257in}{2.166481in}}%
\pgfpathlineto{\pgfqpoint{4.269204in}{2.177926in}}%
\pgfpathlineto{\pgfqpoint{4.255264in}{2.171681in}}%
\pgfpathlineto{\pgfqpoint{4.241335in}{2.165623in}}%
\pgfpathlineto{\pgfqpoint{4.227419in}{2.159752in}}%
\pgfpathlineto{\pgfqpoint{4.213515in}{2.154067in}}%
\pgfpathlineto{\pgfqpoint{4.205564in}{2.142729in}}%
\pgfpathlineto{\pgfqpoint{4.197607in}{2.131321in}}%
\pgfpathlineto{\pgfqpoint{4.189645in}{2.119844in}}%
\pgfpathlineto{\pgfqpoint{4.181678in}{2.108297in}}%
\pgfpathclose%
\pgfusepath{fill}%
\end{pgfscope}%
\begin{pgfscope}%
\pgfpathrectangle{\pgfqpoint{1.150000in}{0.150000in}}{\pgfqpoint{5.700000in}{5.700000in}}%
\pgfusepath{clip}%
\pgfsetbuttcap%
\pgfsetroundjoin%
\definecolor{currentfill}{rgb}{0.282656,0.100196,0.422160}%
\pgfsetfillcolor{currentfill}%
\pgfsetfillopacity{0.800000}%
\pgfsetlinewidth{0.000000pt}%
\definecolor{currentstroke}{rgb}{0.000000,0.000000,0.000000}%
\pgfsetstrokecolor{currentstroke}%
\pgfsetdash{}{0pt}%
\pgfpathmoveto{\pgfqpoint{3.799854in}{1.818717in}}%
\pgfpathlineto{\pgfqpoint{3.813628in}{1.820180in}}%
\pgfpathlineto{\pgfqpoint{3.827409in}{1.821834in}}%
\pgfpathlineto{\pgfqpoint{3.841199in}{1.823680in}}%
\pgfpathlineto{\pgfqpoint{3.854998in}{1.825716in}}%
\pgfpathlineto{\pgfqpoint{3.863077in}{1.837467in}}%
\pgfpathlineto{\pgfqpoint{3.871152in}{1.849206in}}%
\pgfpathlineto{\pgfqpoint{3.879221in}{1.860932in}}%
\pgfpathlineto{\pgfqpoint{3.887286in}{1.872642in}}%
\pgfpathlineto{\pgfqpoint{3.873494in}{1.870327in}}%
\pgfpathlineto{\pgfqpoint{3.859711in}{1.868204in}}%
\pgfpathlineto{\pgfqpoint{3.845937in}{1.866272in}}%
\pgfpathlineto{\pgfqpoint{3.832172in}{1.864532in}}%
\pgfpathlineto{\pgfqpoint{3.824100in}{1.853088in}}%
\pgfpathlineto{\pgfqpoint{3.816023in}{1.841636in}}%
\pgfpathlineto{\pgfqpoint{3.807941in}{1.830178in}}%
\pgfpathlineto{\pgfqpoint{3.799854in}{1.818717in}}%
\pgfpathclose%
\pgfusepath{fill}%
\end{pgfscope}%
\begin{pgfscope}%
\pgfpathrectangle{\pgfqpoint{1.150000in}{0.150000in}}{\pgfqpoint{5.700000in}{5.700000in}}%
\pgfusepath{clip}%
\pgfsetbuttcap%
\pgfsetroundjoin%
\definecolor{currentfill}{rgb}{0.273809,0.031497,0.358853}%
\pgfsetfillcolor{currentfill}%
\pgfsetfillopacity{0.800000}%
\pgfsetlinewidth{0.000000pt}%
\definecolor{currentstroke}{rgb}{0.000000,0.000000,0.000000}%
\pgfsetstrokecolor{currentstroke}%
\pgfsetdash{}{0pt}%
\pgfpathmoveto{\pgfqpoint{3.537340in}{1.691780in}}%
\pgfpathlineto{\pgfqpoint{3.551062in}{1.689883in}}%
\pgfpathlineto{\pgfqpoint{3.564790in}{1.688183in}}%
\pgfpathlineto{\pgfqpoint{3.578524in}{1.686681in}}%
\pgfpathlineto{\pgfqpoint{3.592263in}{1.685375in}}%
\pgfpathlineto{\pgfqpoint{3.600435in}{1.695885in}}%
\pgfpathlineto{\pgfqpoint{3.608601in}{1.706443in}}%
\pgfpathlineto{\pgfqpoint{3.616761in}{1.717047in}}%
\pgfpathlineto{\pgfqpoint{3.624916in}{1.727692in}}%
\pgfpathlineto{\pgfqpoint{3.611189in}{1.728626in}}%
\pgfpathlineto{\pgfqpoint{3.597468in}{1.729757in}}%
\pgfpathlineto{\pgfqpoint{3.583753in}{1.731085in}}%
\pgfpathlineto{\pgfqpoint{3.570043in}{1.732612in}}%
\pgfpathlineto{\pgfqpoint{3.561877in}{1.722326in}}%
\pgfpathlineto{\pgfqpoint{3.553704in}{1.712090in}}%
\pgfpathlineto{\pgfqpoint{3.545525in}{1.701907in}}%
\pgfpathlineto{\pgfqpoint{3.537340in}{1.691780in}}%
\pgfpathclose%
\pgfusepath{fill}%
\end{pgfscope}%
\begin{pgfscope}%
\pgfpathrectangle{\pgfqpoint{1.150000in}{0.150000in}}{\pgfqpoint{5.700000in}{5.700000in}}%
\pgfusepath{clip}%
\pgfsetbuttcap%
\pgfsetroundjoin%
\definecolor{currentfill}{rgb}{0.120092,0.600104,0.542530}%
\pgfsetfillcolor{currentfill}%
\pgfsetfillopacity{0.800000}%
\pgfsetlinewidth{0.000000pt}%
\definecolor{currentstroke}{rgb}{0.000000,0.000000,0.000000}%
\pgfsetstrokecolor{currentstroke}%
\pgfsetdash{}{0pt}%
\pgfpathmoveto{\pgfqpoint{5.446727in}{3.171411in}}%
\pgfpathlineto{\pgfqpoint{5.461329in}{3.183998in}}%
\pgfpathlineto{\pgfqpoint{5.475951in}{3.196766in}}%
\pgfpathlineto{\pgfqpoint{5.490593in}{3.209716in}}%
\pgfpathlineto{\pgfqpoint{5.505256in}{3.222846in}}%
\pgfpathlineto{\pgfqpoint{5.512618in}{3.226099in}}%
\pgfpathlineto{\pgfqpoint{5.519973in}{3.229308in}}%
\pgfpathlineto{\pgfqpoint{5.527320in}{3.232480in}}%
\pgfpathlineto{\pgfqpoint{5.534659in}{3.235619in}}%
\pgfpathlineto{\pgfqpoint{5.520019in}{3.222948in}}%
\pgfpathlineto{\pgfqpoint{5.505399in}{3.210456in}}%
\pgfpathlineto{\pgfqpoint{5.490800in}{3.198144in}}%
\pgfpathlineto{\pgfqpoint{5.476220in}{3.186011in}}%
\pgfpathlineto{\pgfqpoint{5.468858in}{3.182404in}}%
\pgfpathlineto{\pgfqpoint{5.461488in}{3.178771in}}%
\pgfpathlineto{\pgfqpoint{5.454111in}{3.175108in}}%
\pgfpathlineto{\pgfqpoint{5.446727in}{3.171411in}}%
\pgfpathclose%
\pgfusepath{fill}%
\end{pgfscope}%
\begin{pgfscope}%
\pgfpathrectangle{\pgfqpoint{1.150000in}{0.150000in}}{\pgfqpoint{5.700000in}{5.700000in}}%
\pgfusepath{clip}%
\pgfsetbuttcap%
\pgfsetroundjoin%
\definecolor{currentfill}{rgb}{0.223925,0.334994,0.548053}%
\pgfsetfillcolor{currentfill}%
\pgfsetfillopacity{0.800000}%
\pgfsetlinewidth{0.000000pt}%
\definecolor{currentstroke}{rgb}{0.000000,0.000000,0.000000}%
\pgfsetstrokecolor{currentstroke}%
\pgfsetdash{}{0pt}%
\pgfpathmoveto{\pgfqpoint{4.476067in}{2.366956in}}%
\pgfpathlineto{\pgfqpoint{4.490117in}{2.375035in}}%
\pgfpathlineto{\pgfqpoint{4.504182in}{2.383299in}}%
\pgfpathlineto{\pgfqpoint{4.518260in}{2.391749in}}%
\pgfpathlineto{\pgfqpoint{4.532354in}{2.400383in}}%
\pgfpathlineto{\pgfqpoint{4.540218in}{2.410783in}}%
\pgfpathlineto{\pgfqpoint{4.548077in}{2.421079in}}%
\pgfpathlineto{\pgfqpoint{4.555930in}{2.431272in}}%
\pgfpathlineto{\pgfqpoint{4.563776in}{2.441361in}}%
\pgfpathlineto{\pgfqpoint{4.549687in}{2.432739in}}%
\pgfpathlineto{\pgfqpoint{4.535613in}{2.424302in}}%
\pgfpathlineto{\pgfqpoint{4.521553in}{2.416050in}}%
\pgfpathlineto{\pgfqpoint{4.507507in}{2.407983in}}%
\pgfpathlineto{\pgfqpoint{4.499656in}{2.397869in}}%
\pgfpathlineto{\pgfqpoint{4.491799in}{2.387660in}}%
\pgfpathlineto{\pgfqpoint{4.483936in}{2.377356in}}%
\pgfpathlineto{\pgfqpoint{4.476067in}{2.366956in}}%
\pgfpathclose%
\pgfusepath{fill}%
\end{pgfscope}%
\begin{pgfscope}%
\pgfpathrectangle{\pgfqpoint{1.150000in}{0.150000in}}{\pgfqpoint{5.700000in}{5.700000in}}%
\pgfusepath{clip}%
\pgfsetbuttcap%
\pgfsetroundjoin%
\definecolor{currentfill}{rgb}{0.283187,0.125848,0.444960}%
\pgfsetfillcolor{currentfill}%
\pgfsetfillopacity{0.800000}%
\pgfsetlinewidth{0.000000pt}%
\definecolor{currentstroke}{rgb}{0.000000,0.000000,0.000000}%
\pgfsetstrokecolor{currentstroke}%
\pgfsetdash{}{0pt}%
\pgfpathmoveto{\pgfqpoint{3.887286in}{1.872642in}}%
\pgfpathlineto{\pgfqpoint{3.901087in}{1.875147in}}%
\pgfpathlineto{\pgfqpoint{3.914897in}{1.877842in}}%
\pgfpathlineto{\pgfqpoint{3.928717in}{1.880728in}}%
\pgfpathlineto{\pgfqpoint{3.942546in}{1.883803in}}%
\pgfpathlineto{\pgfqpoint{3.950600in}{1.895753in}}%
\pgfpathlineto{\pgfqpoint{3.958649in}{1.907674in}}%
\pgfpathlineto{\pgfqpoint{3.966693in}{1.919565in}}%
\pgfpathlineto{\pgfqpoint{3.974732in}{1.931423in}}%
\pgfpathlineto{\pgfqpoint{3.960909in}{1.928101in}}%
\pgfpathlineto{\pgfqpoint{3.947095in}{1.924969in}}%
\pgfpathlineto{\pgfqpoint{3.933291in}{1.922027in}}%
\pgfpathlineto{\pgfqpoint{3.919496in}{1.919276in}}%
\pgfpathlineto{\pgfqpoint{3.911451in}{1.907652in}}%
\pgfpathlineto{\pgfqpoint{3.903401in}{1.896004in}}%
\pgfpathlineto{\pgfqpoint{3.895346in}{1.884333in}}%
\pgfpathlineto{\pgfqpoint{3.887286in}{1.872642in}}%
\pgfpathclose%
\pgfusepath{fill}%
\end{pgfscope}%
\begin{pgfscope}%
\pgfpathrectangle{\pgfqpoint{1.150000in}{0.150000in}}{\pgfqpoint{5.700000in}{5.700000in}}%
\pgfusepath{clip}%
\pgfsetbuttcap%
\pgfsetroundjoin%
\definecolor{currentfill}{rgb}{0.274952,0.037752,0.364543}%
\pgfsetfillcolor{currentfill}%
\pgfsetfillopacity{0.800000}%
\pgfsetlinewidth{0.000000pt}%
\definecolor{currentstroke}{rgb}{0.000000,0.000000,0.000000}%
\pgfsetstrokecolor{currentstroke}%
\pgfsetdash{}{0pt}%
\pgfpathmoveto{\pgfqpoint{3.020399in}{1.727778in}}%
\pgfpathlineto{\pgfqpoint{3.034144in}{1.718193in}}%
\pgfpathlineto{\pgfqpoint{3.047887in}{1.708833in}}%
\pgfpathlineto{\pgfqpoint{3.061630in}{1.699695in}}%
\pgfpathlineto{\pgfqpoint{3.075371in}{1.690779in}}%
\pgfpathlineto{\pgfqpoint{3.083809in}{1.696394in}}%
\pgfpathlineto{\pgfqpoint{3.092235in}{1.702182in}}%
\pgfpathlineto{\pgfqpoint{3.100652in}{1.708136in}}%
\pgfpathlineto{\pgfqpoint{3.109058in}{1.714252in}}%
\pgfpathlineto{\pgfqpoint{3.095343in}{1.722666in}}%
\pgfpathlineto{\pgfqpoint{3.081629in}{1.731302in}}%
\pgfpathlineto{\pgfqpoint{3.067914in}{1.740160in}}%
\pgfpathlineto{\pgfqpoint{3.054198in}{1.749242in}}%
\pgfpathlineto{\pgfqpoint{3.045765in}{1.743616in}}%
\pgfpathlineto{\pgfqpoint{3.037321in}{1.738160in}}%
\pgfpathlineto{\pgfqpoint{3.028866in}{1.732879in}}%
\pgfpathlineto{\pgfqpoint{3.020399in}{1.727778in}}%
\pgfpathclose%
\pgfusepath{fill}%
\end{pgfscope}%
\begin{pgfscope}%
\pgfpathrectangle{\pgfqpoint{1.150000in}{0.150000in}}{\pgfqpoint{5.700000in}{5.700000in}}%
\pgfusepath{clip}%
\pgfsetbuttcap%
\pgfsetroundjoin%
\definecolor{currentfill}{rgb}{0.275191,0.194905,0.496005}%
\pgfsetfillcolor{currentfill}%
\pgfsetfillopacity{0.800000}%
\pgfsetlinewidth{0.000000pt}%
\definecolor{currentstroke}{rgb}{0.000000,0.000000,0.000000}%
\pgfsetstrokecolor{currentstroke}%
\pgfsetdash{}{0pt}%
\pgfpathmoveto{\pgfqpoint{2.599216in}{2.089703in}}%
\pgfpathlineto{\pgfqpoint{2.613123in}{2.072680in}}%
\pgfpathlineto{\pgfqpoint{2.627023in}{2.055928in}}%
\pgfpathlineto{\pgfqpoint{2.640915in}{2.039444in}}%
\pgfpathlineto{\pgfqpoint{2.654800in}{2.023227in}}%
\pgfpathlineto{\pgfqpoint{2.663536in}{2.024226in}}%
\pgfpathlineto{\pgfqpoint{2.672256in}{2.025477in}}%
\pgfpathlineto{\pgfqpoint{2.680962in}{2.026974in}}%
\pgfpathlineto{\pgfqpoint{2.689651in}{2.028712in}}%
\pgfpathlineto{\pgfqpoint{2.675809in}{2.044378in}}%
\pgfpathlineto{\pgfqpoint{2.661959in}{2.060309in}}%
\pgfpathlineto{\pgfqpoint{2.648102in}{2.076509in}}%
\pgfpathlineto{\pgfqpoint{2.634238in}{2.092979in}}%
\pgfpathlineto{\pgfqpoint{2.625506in}{2.091780in}}%
\pgfpathlineto{\pgfqpoint{2.616759in}{2.090830in}}%
\pgfpathlineto{\pgfqpoint{2.607996in}{2.090136in}}%
\pgfpathlineto{\pgfqpoint{2.599216in}{2.089703in}}%
\pgfpathclose%
\pgfusepath{fill}%
\end{pgfscope}%
\begin{pgfscope}%
\pgfpathrectangle{\pgfqpoint{1.150000in}{0.150000in}}{\pgfqpoint{5.700000in}{5.700000in}}%
\pgfusepath{clip}%
\pgfsetbuttcap%
\pgfsetroundjoin%
\definecolor{currentfill}{rgb}{0.279574,0.170599,0.479997}%
\pgfsetfillcolor{currentfill}%
\pgfsetfillopacity{0.800000}%
\pgfsetlinewidth{0.000000pt}%
\definecolor{currentstroke}{rgb}{0.000000,0.000000,0.000000}%
\pgfsetstrokecolor{currentstroke}%
\pgfsetdash{}{0pt}%
\pgfpathmoveto{\pgfqpoint{2.654800in}{2.023227in}}%
\pgfpathlineto{\pgfqpoint{2.668677in}{2.007275in}}%
\pgfpathlineto{\pgfqpoint{2.682547in}{1.991586in}}%
\pgfpathlineto{\pgfqpoint{2.696411in}{1.976157in}}%
\pgfpathlineto{\pgfqpoint{2.710268in}{1.960987in}}%
\pgfpathlineto{\pgfqpoint{2.718962in}{1.962548in}}%
\pgfpathlineto{\pgfqpoint{2.727642in}{1.964352in}}%
\pgfpathlineto{\pgfqpoint{2.736306in}{1.966393in}}%
\pgfpathlineto{\pgfqpoint{2.744957in}{1.968667in}}%
\pgfpathlineto{\pgfqpoint{2.731140in}{1.983289in}}%
\pgfpathlineto{\pgfqpoint{2.717317in}{1.998169in}}%
\pgfpathlineto{\pgfqpoint{2.703487in}{2.013310in}}%
\pgfpathlineto{\pgfqpoint{2.689651in}{2.028712in}}%
\pgfpathlineto{\pgfqpoint{2.680962in}{2.026974in}}%
\pgfpathlineto{\pgfqpoint{2.672256in}{2.025477in}}%
\pgfpathlineto{\pgfqpoint{2.663536in}{2.024226in}}%
\pgfpathlineto{\pgfqpoint{2.654800in}{2.023227in}}%
\pgfpathclose%
\pgfusepath{fill}%
\end{pgfscope}%
\begin{pgfscope}%
\pgfpathrectangle{\pgfqpoint{1.150000in}{0.150000in}}{\pgfqpoint{5.700000in}{5.700000in}}%
\pgfusepath{clip}%
\pgfsetbuttcap%
\pgfsetroundjoin%
\definecolor{currentfill}{rgb}{0.188923,0.410910,0.556326}%
\pgfsetfillcolor{currentfill}%
\pgfsetfillopacity{0.800000}%
\pgfsetlinewidth{0.000000pt}%
\definecolor{currentstroke}{rgb}{0.000000,0.000000,0.000000}%
\pgfsetstrokecolor{currentstroke}%
\pgfsetdash{}{0pt}%
\pgfpathmoveto{\pgfqpoint{2.241800in}{2.673504in}}%
\pgfpathlineto{\pgfqpoint{2.255997in}{2.648455in}}%
\pgfpathlineto{\pgfqpoint{2.270177in}{2.623752in}}%
\pgfpathlineto{\pgfqpoint{2.284340in}{2.599393in}}%
\pgfpathlineto{\pgfqpoint{2.298489in}{2.575374in}}%
\pgfpathlineto{\pgfqpoint{2.307491in}{2.573553in}}%
\pgfpathlineto{\pgfqpoint{2.316473in}{2.572025in}}%
\pgfpathlineto{\pgfqpoint{2.325436in}{2.570786in}}%
\pgfpathlineto{\pgfqpoint{2.334380in}{2.569829in}}%
\pgfpathlineto{\pgfqpoint{2.320284in}{2.593303in}}%
\pgfpathlineto{\pgfqpoint{2.306174in}{2.617116in}}%
\pgfpathlineto{\pgfqpoint{2.292048in}{2.641270in}}%
\pgfpathlineto{\pgfqpoint{2.277906in}{2.665769in}}%
\pgfpathlineto{\pgfqpoint{2.268910in}{2.667260in}}%
\pgfpathlineto{\pgfqpoint{2.259894in}{2.669042in}}%
\pgfpathlineto{\pgfqpoint{2.250857in}{2.671121in}}%
\pgfpathlineto{\pgfqpoint{2.241800in}{2.673504in}}%
\pgfpathclose%
\pgfusepath{fill}%
\end{pgfscope}%
\begin{pgfscope}%
\pgfpathrectangle{\pgfqpoint{1.150000in}{0.150000in}}{\pgfqpoint{5.700000in}{5.700000in}}%
\pgfusepath{clip}%
\pgfsetbuttcap%
\pgfsetroundjoin%
\definecolor{currentfill}{rgb}{0.149039,0.508051,0.557250}%
\pgfsetfillcolor{currentfill}%
\pgfsetfillopacity{0.800000}%
\pgfsetlinewidth{0.000000pt}%
\definecolor{currentstroke}{rgb}{0.000000,0.000000,0.000000}%
\pgfsetstrokecolor{currentstroke}%
\pgfsetdash{}{0pt}%
\pgfpathmoveto{\pgfqpoint{5.064966in}{2.877948in}}%
\pgfpathlineto{\pgfqpoint{5.079347in}{2.889397in}}%
\pgfpathlineto{\pgfqpoint{5.093746in}{2.901027in}}%
\pgfpathlineto{\pgfqpoint{5.108163in}{2.912840in}}%
\pgfpathlineto{\pgfqpoint{5.122598in}{2.924836in}}%
\pgfpathlineto{\pgfqpoint{5.130191in}{2.930994in}}%
\pgfpathlineto{\pgfqpoint{5.137775in}{2.937058in}}%
\pgfpathlineto{\pgfqpoint{5.145352in}{2.943031in}}%
\pgfpathlineto{\pgfqpoint{5.152921in}{2.948919in}}%
\pgfpathlineto{\pgfqpoint{5.138499in}{2.937207in}}%
\pgfpathlineto{\pgfqpoint{5.124095in}{2.925678in}}%
\pgfpathlineto{\pgfqpoint{5.109709in}{2.914331in}}%
\pgfpathlineto{\pgfqpoint{5.095341in}{2.903166in}}%
\pgfpathlineto{\pgfqpoint{5.087759in}{2.896983in}}%
\pgfpathlineto{\pgfqpoint{5.080169in}{2.890721in}}%
\pgfpathlineto{\pgfqpoint{5.072571in}{2.884377in}}%
\pgfpathlineto{\pgfqpoint{5.064966in}{2.877948in}}%
\pgfpathclose%
\pgfusepath{fill}%
\end{pgfscope}%
\begin{pgfscope}%
\pgfpathrectangle{\pgfqpoint{1.150000in}{0.150000in}}{\pgfqpoint{5.700000in}{5.700000in}}%
\pgfusepath{clip}%
\pgfsetbuttcap%
\pgfsetroundjoin%
\definecolor{currentfill}{rgb}{0.267968,0.223549,0.512008}%
\pgfsetfillcolor{currentfill}%
\pgfsetfillopacity{0.800000}%
\pgfsetlinewidth{0.000000pt}%
\definecolor{currentstroke}{rgb}{0.000000,0.000000,0.000000}%
\pgfsetstrokecolor{currentstroke}%
\pgfsetdash{}{0pt}%
\pgfpathmoveto{\pgfqpoint{2.543499in}{2.160546in}}%
\pgfpathlineto{\pgfqpoint{2.557441in}{2.142418in}}%
\pgfpathlineto{\pgfqpoint{2.571375in}{2.124570in}}%
\pgfpathlineto{\pgfqpoint{2.585300in}{2.106999in}}%
\pgfpathlineto{\pgfqpoint{2.599216in}{2.089703in}}%
\pgfpathlineto{\pgfqpoint{2.607996in}{2.090136in}}%
\pgfpathlineto{\pgfqpoint{2.616759in}{2.090830in}}%
\pgfpathlineto{\pgfqpoint{2.625506in}{2.091780in}}%
\pgfpathlineto{\pgfqpoint{2.634238in}{2.092979in}}%
\pgfpathlineto{\pgfqpoint{2.620366in}{2.109720in}}%
\pgfpathlineto{\pgfqpoint{2.606486in}{2.126736in}}%
\pgfpathlineto{\pgfqpoint{2.592597in}{2.144029in}}%
\pgfpathlineto{\pgfqpoint{2.578700in}{2.161600in}}%
\pgfpathlineto{\pgfqpoint{2.569925in}{2.160943in}}%
\pgfpathlineto{\pgfqpoint{2.561133in}{2.160545in}}%
\pgfpathlineto{\pgfqpoint{2.552325in}{2.160411in}}%
\pgfpathlineto{\pgfqpoint{2.543499in}{2.160546in}}%
\pgfpathclose%
\pgfusepath{fill}%
\end{pgfscope}%
\begin{pgfscope}%
\pgfpathrectangle{\pgfqpoint{1.150000in}{0.150000in}}{\pgfqpoint{5.700000in}{5.700000in}}%
\pgfusepath{clip}%
\pgfsetbuttcap%
\pgfsetroundjoin%
\definecolor{currentfill}{rgb}{0.282290,0.145912,0.461510}%
\pgfsetfillcolor{currentfill}%
\pgfsetfillopacity{0.800000}%
\pgfsetlinewidth{0.000000pt}%
\definecolor{currentstroke}{rgb}{0.000000,0.000000,0.000000}%
\pgfsetstrokecolor{currentstroke}%
\pgfsetdash{}{0pt}%
\pgfpathmoveto{\pgfqpoint{2.710268in}{1.960987in}}%
\pgfpathlineto{\pgfqpoint{2.724119in}{1.946074in}}%
\pgfpathlineto{\pgfqpoint{2.737964in}{1.931416in}}%
\pgfpathlineto{\pgfqpoint{2.751803in}{1.917011in}}%
\pgfpathlineto{\pgfqpoint{2.765637in}{1.902858in}}%
\pgfpathlineto{\pgfqpoint{2.774291in}{1.904978in}}%
\pgfpathlineto{\pgfqpoint{2.782931in}{1.907332in}}%
\pgfpathlineto{\pgfqpoint{2.791558in}{1.909915in}}%
\pgfpathlineto{\pgfqpoint{2.800170in}{1.912721in}}%
\pgfpathlineto{\pgfqpoint{2.786374in}{1.926330in}}%
\pgfpathlineto{\pgfqpoint{2.772574in}{1.940189in}}%
\pgfpathlineto{\pgfqpoint{2.758768in}{1.954301in}}%
\pgfpathlineto{\pgfqpoint{2.744957in}{1.968667in}}%
\pgfpathlineto{\pgfqpoint{2.736306in}{1.966393in}}%
\pgfpathlineto{\pgfqpoint{2.727642in}{1.964352in}}%
\pgfpathlineto{\pgfqpoint{2.718962in}{1.962548in}}%
\pgfpathlineto{\pgfqpoint{2.710268in}{1.960987in}}%
\pgfpathclose%
\pgfusepath{fill}%
\end{pgfscope}%
\begin{pgfscope}%
\pgfpathrectangle{\pgfqpoint{1.150000in}{0.150000in}}{\pgfqpoint{5.700000in}{5.700000in}}%
\pgfusepath{clip}%
\pgfsetbuttcap%
\pgfsetroundjoin%
\definecolor{currentfill}{rgb}{0.182256,0.426184,0.557120}%
\pgfsetfillcolor{currentfill}%
\pgfsetfillopacity{0.800000}%
\pgfsetlinewidth{0.000000pt}%
\definecolor{currentstroke}{rgb}{0.000000,0.000000,0.000000}%
\pgfsetstrokecolor{currentstroke}%
\pgfsetdash{}{0pt}%
\pgfpathmoveto{\pgfqpoint{4.770574in}{2.628469in}}%
\pgfpathlineto{\pgfqpoint{4.784786in}{2.638512in}}%
\pgfpathlineto{\pgfqpoint{4.799014in}{2.648739in}}%
\pgfpathlineto{\pgfqpoint{4.813259in}{2.659149in}}%
\pgfpathlineto{\pgfqpoint{4.827521in}{2.669743in}}%
\pgfpathlineto{\pgfqpoint{4.835264in}{2.678190in}}%
\pgfpathlineto{\pgfqpoint{4.843000in}{2.686527in}}%
\pgfpathlineto{\pgfqpoint{4.850729in}{2.694755in}}%
\pgfpathlineto{\pgfqpoint{4.858451in}{2.702878in}}%
\pgfpathlineto{\pgfqpoint{4.844197in}{2.692431in}}%
\pgfpathlineto{\pgfqpoint{4.829960in}{2.682167in}}%
\pgfpathlineto{\pgfqpoint{4.815739in}{2.672087in}}%
\pgfpathlineto{\pgfqpoint{4.801534in}{2.662190in}}%
\pgfpathlineto{\pgfqpoint{4.793805in}{2.653909in}}%
\pgfpathlineto{\pgfqpoint{4.786068in}{2.645529in}}%
\pgfpathlineto{\pgfqpoint{4.778324in}{2.637050in}}%
\pgfpathlineto{\pgfqpoint{4.770574in}{2.628469in}}%
\pgfpathclose%
\pgfusepath{fill}%
\end{pgfscope}%
\begin{pgfscope}%
\pgfpathrectangle{\pgfqpoint{1.150000in}{0.150000in}}{\pgfqpoint{5.700000in}{5.700000in}}%
\pgfusepath{clip}%
\pgfsetbuttcap%
\pgfsetroundjoin%
\definecolor{currentfill}{rgb}{0.271305,0.019942,0.347269}%
\pgfsetfillcolor{currentfill}%
\pgfsetfillopacity{0.800000}%
\pgfsetlinewidth{0.000000pt}%
\definecolor{currentstroke}{rgb}{0.000000,0.000000,0.000000}%
\pgfsetstrokecolor{currentstroke}%
\pgfsetdash{}{0pt}%
\pgfpathmoveto{\pgfqpoint{3.449640in}{1.663105in}}%
\pgfpathlineto{\pgfqpoint{3.463357in}{1.660006in}}%
\pgfpathlineto{\pgfqpoint{3.477079in}{1.657108in}}%
\pgfpathlineto{\pgfqpoint{3.490806in}{1.654409in}}%
\pgfpathlineto{\pgfqpoint{3.504537in}{1.651910in}}%
\pgfpathlineto{\pgfqpoint{3.512747in}{1.661775in}}%
\pgfpathlineto{\pgfqpoint{3.520951in}{1.671711in}}%
\pgfpathlineto{\pgfqpoint{3.529149in}{1.681714in}}%
\pgfpathlineto{\pgfqpoint{3.537340in}{1.691780in}}%
\pgfpathlineto{\pgfqpoint{3.523623in}{1.693877in}}%
\pgfpathlineto{\pgfqpoint{3.509911in}{1.696172in}}%
\pgfpathlineto{\pgfqpoint{3.496205in}{1.698668in}}%
\pgfpathlineto{\pgfqpoint{3.482503in}{1.701364in}}%
\pgfpathlineto{\pgfqpoint{3.474297in}{1.691688in}}%
\pgfpathlineto{\pgfqpoint{3.466085in}{1.682084in}}%
\pgfpathlineto{\pgfqpoint{3.457866in}{1.672555in}}%
\pgfpathlineto{\pgfqpoint{3.449640in}{1.663105in}}%
\pgfpathclose%
\pgfusepath{fill}%
\end{pgfscope}%
\begin{pgfscope}%
\pgfpathrectangle{\pgfqpoint{1.150000in}{0.150000in}}{\pgfqpoint{5.700000in}{5.700000in}}%
\pgfusepath{clip}%
\pgfsetbuttcap%
\pgfsetroundjoin%
\definecolor{currentfill}{rgb}{0.257322,0.256130,0.526563}%
\pgfsetfillcolor{currentfill}%
\pgfsetfillopacity{0.800000}%
\pgfsetlinewidth{0.000000pt}%
\definecolor{currentstroke}{rgb}{0.000000,0.000000,0.000000}%
\pgfsetstrokecolor{currentstroke}%
\pgfsetdash{}{0pt}%
\pgfpathmoveto{\pgfqpoint{2.487632in}{2.235902in}}%
\pgfpathlineto{\pgfqpoint{2.501614in}{2.216632in}}%
\pgfpathlineto{\pgfqpoint{2.515585in}{2.197651in}}%
\pgfpathlineto{\pgfqpoint{2.529547in}{2.178956in}}%
\pgfpathlineto{\pgfqpoint{2.543499in}{2.160546in}}%
\pgfpathlineto{\pgfqpoint{2.552325in}{2.160411in}}%
\pgfpathlineto{\pgfqpoint{2.561133in}{2.160545in}}%
\pgfpathlineto{\pgfqpoint{2.569925in}{2.160943in}}%
\pgfpathlineto{\pgfqpoint{2.578700in}{2.161600in}}%
\pgfpathlineto{\pgfqpoint{2.564794in}{2.179452in}}%
\pgfpathlineto{\pgfqpoint{2.550878in}{2.197587in}}%
\pgfpathlineto{\pgfqpoint{2.536953in}{2.216008in}}%
\pgfpathlineto{\pgfqpoint{2.523019in}{2.234716in}}%
\pgfpathlineto{\pgfqpoint{2.514198in}{2.234606in}}%
\pgfpathlineto{\pgfqpoint{2.505361in}{2.234763in}}%
\pgfpathlineto{\pgfqpoint{2.496505in}{2.235193in}}%
\pgfpathlineto{\pgfqpoint{2.487632in}{2.235902in}}%
\pgfpathclose%
\pgfusepath{fill}%
\end{pgfscope}%
\begin{pgfscope}%
\pgfpathrectangle{\pgfqpoint{1.150000in}{0.150000in}}{\pgfqpoint{5.700000in}{5.700000in}}%
\pgfusepath{clip}%
\pgfsetbuttcap%
\pgfsetroundjoin%
\definecolor{currentfill}{rgb}{0.255645,0.260703,0.528312}%
\pgfsetfillcolor{currentfill}%
\pgfsetfillopacity{0.800000}%
\pgfsetlinewidth{0.000000pt}%
\definecolor{currentstroke}{rgb}{0.000000,0.000000,0.000000}%
\pgfsetstrokecolor{currentstroke}%
\pgfsetdash{}{0pt}%
\pgfpathmoveto{\pgfqpoint{4.269204in}{2.177926in}}%
\pgfpathlineto{\pgfqpoint{4.283158in}{2.184358in}}%
\pgfpathlineto{\pgfqpoint{4.297124in}{2.190975in}}%
\pgfpathlineto{\pgfqpoint{4.311103in}{2.197779in}}%
\pgfpathlineto{\pgfqpoint{4.325094in}{2.204769in}}%
\pgfpathlineto{\pgfqpoint{4.333034in}{2.216234in}}%
\pgfpathlineto{\pgfqpoint{4.340968in}{2.227611in}}%
\pgfpathlineto{\pgfqpoint{4.348897in}{2.238899in}}%
\pgfpathlineto{\pgfqpoint{4.356820in}{2.250099in}}%
\pgfpathlineto{\pgfqpoint{4.342831in}{2.243023in}}%
\pgfpathlineto{\pgfqpoint{4.328856in}{2.236132in}}%
\pgfpathlineto{\pgfqpoint{4.314893in}{2.229428in}}%
\pgfpathlineto{\pgfqpoint{4.300943in}{2.222910in}}%
\pgfpathlineto{\pgfqpoint{4.293017in}{2.211784in}}%
\pgfpathlineto{\pgfqpoint{4.285084in}{2.200578in}}%
\pgfpathlineto{\pgfqpoint{4.277147in}{2.189292in}}%
\pgfpathlineto{\pgfqpoint{4.269204in}{2.177926in}}%
\pgfpathclose%
\pgfusepath{fill}%
\end{pgfscope}%
\begin{pgfscope}%
\pgfpathrectangle{\pgfqpoint{1.150000in}{0.150000in}}{\pgfqpoint{5.700000in}{5.700000in}}%
\pgfusepath{clip}%
\pgfsetbuttcap%
\pgfsetroundjoin%
\definecolor{currentfill}{rgb}{0.120638,0.625828,0.533488}%
\pgfsetfillcolor{currentfill}%
\pgfsetfillopacity{0.800000}%
\pgfsetlinewidth{0.000000pt}%
\definecolor{currentstroke}{rgb}{0.000000,0.000000,0.000000}%
\pgfsetstrokecolor{currentstroke}%
\pgfsetdash{}{0pt}%
\pgfpathmoveto{\pgfqpoint{5.534659in}{3.235619in}}%
\pgfpathlineto{\pgfqpoint{5.549319in}{3.248472in}}%
\pgfpathlineto{\pgfqpoint{5.564000in}{3.261504in}}%
\pgfpathlineto{\pgfqpoint{5.578701in}{3.274717in}}%
\pgfpathlineto{\pgfqpoint{5.593423in}{3.288111in}}%
\pgfpathlineto{\pgfqpoint{5.600730in}{3.290742in}}%
\pgfpathlineto{\pgfqpoint{5.608030in}{3.293343in}}%
\pgfpathlineto{\pgfqpoint{5.615323in}{3.295919in}}%
\pgfpathlineto{\pgfqpoint{5.622608in}{3.298476in}}%
\pgfpathlineto{\pgfqpoint{5.607911in}{3.285575in}}%
\pgfpathlineto{\pgfqpoint{5.593235in}{3.272855in}}%
\pgfpathlineto{\pgfqpoint{5.578579in}{3.260314in}}%
\pgfpathlineto{\pgfqpoint{5.563943in}{3.247952in}}%
\pgfpathlineto{\pgfqpoint{5.556633in}{3.244892in}}%
\pgfpathlineto{\pgfqpoint{5.549315in}{3.241820in}}%
\pgfpathlineto{\pgfqpoint{5.541991in}{3.238731in}}%
\pgfpathlineto{\pgfqpoint{5.534659in}{3.235619in}}%
\pgfpathclose%
\pgfusepath{fill}%
\end{pgfscope}%
\begin{pgfscope}%
\pgfpathrectangle{\pgfqpoint{1.150000in}{0.150000in}}{\pgfqpoint{5.700000in}{5.700000in}}%
\pgfusepath{clip}%
\pgfsetbuttcap%
\pgfsetroundjoin%
\definecolor{currentfill}{rgb}{0.281412,0.155834,0.469201}%
\pgfsetfillcolor{currentfill}%
\pgfsetfillopacity{0.800000}%
\pgfsetlinewidth{0.000000pt}%
\definecolor{currentstroke}{rgb}{0.000000,0.000000,0.000000}%
\pgfsetstrokecolor{currentstroke}%
\pgfsetdash{}{0pt}%
\pgfpathmoveto{\pgfqpoint{3.974732in}{1.931423in}}%
\pgfpathlineto{\pgfqpoint{3.988566in}{1.934934in}}%
\pgfpathlineto{\pgfqpoint{4.002409in}{1.938634in}}%
\pgfpathlineto{\pgfqpoint{4.016263in}{1.942522in}}%
\pgfpathlineto{\pgfqpoint{4.030127in}{1.946599in}}%
\pgfpathlineto{\pgfqpoint{4.038157in}{1.958650in}}%
\pgfpathlineto{\pgfqpoint{4.046181in}{1.970656in}}%
\pgfpathlineto{\pgfqpoint{4.054202in}{1.982616in}}%
\pgfpathlineto{\pgfqpoint{4.062217in}{1.994527in}}%
\pgfpathlineto{\pgfqpoint{4.048357in}{1.990235in}}%
\pgfpathlineto{\pgfqpoint{4.034508in}{1.986131in}}%
\pgfpathlineto{\pgfqpoint{4.020669in}{1.982216in}}%
\pgfpathlineto{\pgfqpoint{4.006841in}{1.978491in}}%
\pgfpathlineto{\pgfqpoint{3.998821in}{1.966782in}}%
\pgfpathlineto{\pgfqpoint{3.990796in}{1.955033in}}%
\pgfpathlineto{\pgfqpoint{3.982767in}{1.943246in}}%
\pgfpathlineto{\pgfqpoint{3.974732in}{1.931423in}}%
\pgfpathclose%
\pgfusepath{fill}%
\end{pgfscope}%
\begin{pgfscope}%
\pgfpathrectangle{\pgfqpoint{1.150000in}{0.150000in}}{\pgfqpoint{5.700000in}{5.700000in}}%
\pgfusepath{clip}%
\pgfsetbuttcap%
\pgfsetroundjoin%
\definecolor{currentfill}{rgb}{0.283229,0.120777,0.440584}%
\pgfsetfillcolor{currentfill}%
\pgfsetfillopacity{0.800000}%
\pgfsetlinewidth{0.000000pt}%
\definecolor{currentstroke}{rgb}{0.000000,0.000000,0.000000}%
\pgfsetstrokecolor{currentstroke}%
\pgfsetdash{}{0pt}%
\pgfpathmoveto{\pgfqpoint{2.765637in}{1.902858in}}%
\pgfpathlineto{\pgfqpoint{2.779465in}{1.888955in}}%
\pgfpathlineto{\pgfqpoint{2.793289in}{1.875299in}}%
\pgfpathlineto{\pgfqpoint{2.807108in}{1.861890in}}%
\pgfpathlineto{\pgfqpoint{2.820922in}{1.848725in}}%
\pgfpathlineto{\pgfqpoint{2.829538in}{1.851401in}}%
\pgfpathlineto{\pgfqpoint{2.838141in}{1.854303in}}%
\pgfpathlineto{\pgfqpoint{2.846730in}{1.857425in}}%
\pgfpathlineto{\pgfqpoint{2.855306in}{1.860762in}}%
\pgfpathlineto{\pgfqpoint{2.841528in}{1.873384in}}%
\pgfpathlineto{\pgfqpoint{2.827746in}{1.886250in}}%
\pgfpathlineto{\pgfqpoint{2.813960in}{1.899362in}}%
\pgfpathlineto{\pgfqpoint{2.800170in}{1.912721in}}%
\pgfpathlineto{\pgfqpoint{2.791558in}{1.909915in}}%
\pgfpathlineto{\pgfqpoint{2.782931in}{1.907332in}}%
\pgfpathlineto{\pgfqpoint{2.774291in}{1.904978in}}%
\pgfpathlineto{\pgfqpoint{2.765637in}{1.902858in}}%
\pgfpathclose%
\pgfusepath{fill}%
\end{pgfscope}%
\begin{pgfscope}%
\pgfpathrectangle{\pgfqpoint{1.150000in}{0.150000in}}{\pgfqpoint{5.700000in}{5.700000in}}%
\pgfusepath{clip}%
\pgfsetbuttcap%
\pgfsetroundjoin%
\definecolor{currentfill}{rgb}{0.268510,0.009605,0.335427}%
\pgfsetfillcolor{currentfill}%
\pgfsetfillopacity{0.800000}%
\pgfsetlinewidth{0.000000pt}%
\definecolor{currentstroke}{rgb}{0.000000,0.000000,0.000000}%
\pgfsetstrokecolor{currentstroke}%
\pgfsetdash{}{0pt}%
\pgfpathmoveto{\pgfqpoint{3.218784in}{1.654775in}}%
\pgfpathlineto{\pgfqpoint{3.232505in}{1.648304in}}%
\pgfpathlineto{\pgfqpoint{3.246227in}{1.642044in}}%
\pgfpathlineto{\pgfqpoint{3.259950in}{1.635994in}}%
\pgfpathlineto{\pgfqpoint{3.273676in}{1.630152in}}%
\pgfpathlineto{\pgfqpoint{3.282001in}{1.637866in}}%
\pgfpathlineto{\pgfqpoint{3.290316in}{1.645708in}}%
\pgfpathlineto{\pgfqpoint{3.298624in}{1.653674in}}%
\pgfpathlineto{\pgfqpoint{3.306923in}{1.661759in}}%
\pgfpathlineto{\pgfqpoint{3.293219in}{1.667134in}}%
\pgfpathlineto{\pgfqpoint{3.279517in}{1.672717in}}%
\pgfpathlineto{\pgfqpoint{3.265817in}{1.678510in}}%
\pgfpathlineto{\pgfqpoint{3.252119in}{1.684513in}}%
\pgfpathlineto{\pgfqpoint{3.243798in}{1.676883in}}%
\pgfpathlineto{\pgfqpoint{3.235469in}{1.669380in}}%
\pgfpathlineto{\pgfqpoint{3.227131in}{1.662010in}}%
\pgfpathlineto{\pgfqpoint{3.218784in}{1.654775in}}%
\pgfpathclose%
\pgfusepath{fill}%
\end{pgfscope}%
\begin{pgfscope}%
\pgfpathrectangle{\pgfqpoint{1.150000in}{0.150000in}}{\pgfqpoint{5.700000in}{5.700000in}}%
\pgfusepath{clip}%
\pgfsetbuttcap%
\pgfsetroundjoin%
\definecolor{currentfill}{rgb}{0.244972,0.287675,0.537260}%
\pgfsetfillcolor{currentfill}%
\pgfsetfillopacity{0.800000}%
\pgfsetlinewidth{0.000000pt}%
\definecolor{currentstroke}{rgb}{0.000000,0.000000,0.000000}%
\pgfsetstrokecolor{currentstroke}%
\pgfsetdash{}{0pt}%
\pgfpathmoveto{\pgfqpoint{2.431596in}{2.315923in}}%
\pgfpathlineto{\pgfqpoint{2.445622in}{2.295471in}}%
\pgfpathlineto{\pgfqpoint{2.459636in}{2.275319in}}%
\pgfpathlineto{\pgfqpoint{2.473640in}{2.255463in}}%
\pgfpathlineto{\pgfqpoint{2.487632in}{2.235902in}}%
\pgfpathlineto{\pgfqpoint{2.496505in}{2.235193in}}%
\pgfpathlineto{\pgfqpoint{2.505361in}{2.234763in}}%
\pgfpathlineto{\pgfqpoint{2.514198in}{2.234606in}}%
\pgfpathlineto{\pgfqpoint{2.523019in}{2.234716in}}%
\pgfpathlineto{\pgfqpoint{2.509074in}{2.253716in}}%
\pgfpathlineto{\pgfqpoint{2.495119in}{2.273008in}}%
\pgfpathlineto{\pgfqpoint{2.481154in}{2.292596in}}%
\pgfpathlineto{\pgfqpoint{2.467177in}{2.312482in}}%
\pgfpathlineto{\pgfqpoint{2.458309in}{2.312922in}}%
\pgfpathlineto{\pgfqpoint{2.449423in}{2.313638in}}%
\pgfpathlineto{\pgfqpoint{2.440519in}{2.314637in}}%
\pgfpathlineto{\pgfqpoint{2.431596in}{2.315923in}}%
\pgfpathclose%
\pgfusepath{fill}%
\end{pgfscope}%
\begin{pgfscope}%
\pgfpathrectangle{\pgfqpoint{1.150000in}{0.150000in}}{\pgfqpoint{5.700000in}{5.700000in}}%
\pgfusepath{clip}%
\pgfsetbuttcap%
\pgfsetroundjoin%
\definecolor{currentfill}{rgb}{0.210503,0.363727,0.552206}%
\pgfsetfillcolor{currentfill}%
\pgfsetfillopacity{0.800000}%
\pgfsetlinewidth{0.000000pt}%
\definecolor{currentstroke}{rgb}{0.000000,0.000000,0.000000}%
\pgfsetstrokecolor{currentstroke}%
\pgfsetdash{}{0pt}%
\pgfpathmoveto{\pgfqpoint{4.563776in}{2.441361in}}%
\pgfpathlineto{\pgfqpoint{4.577880in}{2.450168in}}%
\pgfpathlineto{\pgfqpoint{4.591999in}{2.459159in}}%
\pgfpathlineto{\pgfqpoint{4.606134in}{2.468335in}}%
\pgfpathlineto{\pgfqpoint{4.620283in}{2.477696in}}%
\pgfpathlineto{\pgfqpoint{4.628119in}{2.487649in}}%
\pgfpathlineto{\pgfqpoint{4.635948in}{2.497494in}}%
\pgfpathlineto{\pgfqpoint{4.643771in}{2.507229in}}%
\pgfpathlineto{\pgfqpoint{4.651588in}{2.516856in}}%
\pgfpathlineto{\pgfqpoint{4.637443in}{2.507541in}}%
\pgfpathlineto{\pgfqpoint{4.623314in}{2.498411in}}%
\pgfpathlineto{\pgfqpoint{4.609200in}{2.489465in}}%
\pgfpathlineto{\pgfqpoint{4.595101in}{2.480704in}}%
\pgfpathlineto{\pgfqpoint{4.587279in}{2.471019in}}%
\pgfpathlineto{\pgfqpoint{4.579451in}{2.461234in}}%
\pgfpathlineto{\pgfqpoint{4.571617in}{2.451348in}}%
\pgfpathlineto{\pgfqpoint{4.563776in}{2.441361in}}%
\pgfpathclose%
\pgfusepath{fill}%
\end{pgfscope}%
\begin{pgfscope}%
\pgfpathrectangle{\pgfqpoint{1.150000in}{0.150000in}}{\pgfqpoint{5.700000in}{5.700000in}}%
\pgfusepath{clip}%
\pgfsetbuttcap%
\pgfsetroundjoin%
\definecolor{currentfill}{rgb}{0.128087,0.647749,0.523491}%
\pgfsetfillcolor{currentfill}%
\pgfsetfillopacity{0.800000}%
\pgfsetlinewidth{0.000000pt}%
\definecolor{currentstroke}{rgb}{0.000000,0.000000,0.000000}%
\pgfsetstrokecolor{currentstroke}%
\pgfsetdash{}{0pt}%
\pgfpathmoveto{\pgfqpoint{5.622608in}{3.298476in}}%
\pgfpathlineto{\pgfqpoint{5.637325in}{3.311556in}}%
\pgfpathlineto{\pgfqpoint{5.652064in}{3.324816in}}%
\pgfpathlineto{\pgfqpoint{5.666823in}{3.338256in}}%
\pgfpathlineto{\pgfqpoint{5.681603in}{3.351877in}}%
\pgfpathlineto{\pgfqpoint{5.688855in}{3.353904in}}%
\pgfpathlineto{\pgfqpoint{5.696099in}{3.355916in}}%
\pgfpathlineto{\pgfqpoint{5.703335in}{3.357917in}}%
\pgfpathlineto{\pgfqpoint{5.710565in}{3.359913in}}%
\pgfpathlineto{\pgfqpoint{5.695812in}{3.346821in}}%
\pgfpathlineto{\pgfqpoint{5.681080in}{3.333909in}}%
\pgfpathlineto{\pgfqpoint{5.666369in}{3.321175in}}%
\pgfpathlineto{\pgfqpoint{5.651678in}{3.308621in}}%
\pgfpathlineto{\pgfqpoint{5.644421in}{3.306086in}}%
\pgfpathlineto{\pgfqpoint{5.637157in}{3.303553in}}%
\pgfpathlineto{\pgfqpoint{5.629886in}{3.301019in}}%
\pgfpathlineto{\pgfqpoint{5.622608in}{3.298476in}}%
\pgfpathclose%
\pgfusepath{fill}%
\end{pgfscope}%
\begin{pgfscope}%
\pgfpathrectangle{\pgfqpoint{1.150000in}{0.150000in}}{\pgfqpoint{5.700000in}{5.700000in}}%
\pgfusepath{clip}%
\pgfsetbuttcap%
\pgfsetroundjoin%
\definecolor{currentfill}{rgb}{0.139147,0.533812,0.555298}%
\pgfsetfillcolor{currentfill}%
\pgfsetfillopacity{0.800000}%
\pgfsetlinewidth{0.000000pt}%
\definecolor{currentstroke}{rgb}{0.000000,0.000000,0.000000}%
\pgfsetstrokecolor{currentstroke}%
\pgfsetdash{}{0pt}%
\pgfpathmoveto{\pgfqpoint{5.152921in}{2.948919in}}%
\pgfpathlineto{\pgfqpoint{5.167362in}{2.960812in}}%
\pgfpathlineto{\pgfqpoint{5.181821in}{2.972887in}}%
\pgfpathlineto{\pgfqpoint{5.196300in}{2.985145in}}%
\pgfpathlineto{\pgfqpoint{5.210797in}{2.997586in}}%
\pgfpathlineto{\pgfqpoint{5.218344in}{3.003083in}}%
\pgfpathlineto{\pgfqpoint{5.225883in}{3.008492in}}%
\pgfpathlineto{\pgfqpoint{5.233414in}{3.013817in}}%
\pgfpathlineto{\pgfqpoint{5.240936in}{3.019061in}}%
\pgfpathlineto{\pgfqpoint{5.226454in}{3.006941in}}%
\pgfpathlineto{\pgfqpoint{5.211990in}{2.995002in}}%
\pgfpathlineto{\pgfqpoint{5.197546in}{2.983245in}}%
\pgfpathlineto{\pgfqpoint{5.183119in}{2.971670in}}%
\pgfpathlineto{\pgfqpoint{5.175581in}{2.966094in}}%
\pgfpathlineto{\pgfqpoint{5.168036in}{2.960446in}}%
\pgfpathlineto{\pgfqpoint{5.160482in}{2.954722in}}%
\pgfpathlineto{\pgfqpoint{5.152921in}{2.948919in}}%
\pgfpathclose%
\pgfusepath{fill}%
\end{pgfscope}%
\begin{pgfscope}%
\pgfpathrectangle{\pgfqpoint{1.150000in}{0.150000in}}{\pgfqpoint{5.700000in}{5.700000in}}%
\pgfusepath{clip}%
\pgfsetbuttcap%
\pgfsetroundjoin%
\definecolor{currentfill}{rgb}{0.282656,0.100196,0.422160}%
\pgfsetfillcolor{currentfill}%
\pgfsetfillopacity{0.800000}%
\pgfsetlinewidth{0.000000pt}%
\definecolor{currentstroke}{rgb}{0.000000,0.000000,0.000000}%
\pgfsetstrokecolor{currentstroke}%
\pgfsetdash{}{0pt}%
\pgfpathmoveto{\pgfqpoint{2.820922in}{1.848725in}}%
\pgfpathlineto{\pgfqpoint{2.834732in}{1.835804in}}%
\pgfpathlineto{\pgfqpoint{2.848538in}{1.823124in}}%
\pgfpathlineto{\pgfqpoint{2.862340in}{1.810684in}}%
\pgfpathlineto{\pgfqpoint{2.876138in}{1.798482in}}%
\pgfpathlineto{\pgfqpoint{2.884718in}{1.801712in}}%
\pgfpathlineto{\pgfqpoint{2.893285in}{1.805158in}}%
\pgfpathlineto{\pgfqpoint{2.901839in}{1.808817in}}%
\pgfpathlineto{\pgfqpoint{2.910380in}{1.812682in}}%
\pgfpathlineto{\pgfqpoint{2.896617in}{1.824344in}}%
\pgfpathlineto{\pgfqpoint{2.882850in}{1.836243in}}%
\pgfpathlineto{\pgfqpoint{2.869080in}{1.848382in}}%
\pgfpathlineto{\pgfqpoint{2.855306in}{1.860762in}}%
\pgfpathlineto{\pgfqpoint{2.846730in}{1.857425in}}%
\pgfpathlineto{\pgfqpoint{2.838141in}{1.854303in}}%
\pgfpathlineto{\pgfqpoint{2.829538in}{1.851401in}}%
\pgfpathlineto{\pgfqpoint{2.820922in}{1.848725in}}%
\pgfpathclose%
\pgfusepath{fill}%
\end{pgfscope}%
\begin{pgfscope}%
\pgfpathrectangle{\pgfqpoint{1.150000in}{0.150000in}}{\pgfqpoint{5.700000in}{5.700000in}}%
\pgfusepath{clip}%
\pgfsetbuttcap%
\pgfsetroundjoin%
\definecolor{currentfill}{rgb}{0.272594,0.025563,0.353093}%
\pgfsetfillcolor{currentfill}%
\pgfsetfillopacity{0.800000}%
\pgfsetlinewidth{0.000000pt}%
\definecolor{currentstroke}{rgb}{0.000000,0.000000,0.000000}%
\pgfsetstrokecolor{currentstroke}%
\pgfsetdash{}{0pt}%
\pgfpathmoveto{\pgfqpoint{3.075371in}{1.690779in}}%
\pgfpathlineto{\pgfqpoint{3.089113in}{1.682083in}}%
\pgfpathlineto{\pgfqpoint{3.102854in}{1.673607in}}%
\pgfpathlineto{\pgfqpoint{3.116595in}{1.665349in}}%
\pgfpathlineto{\pgfqpoint{3.130336in}{1.657309in}}%
\pgfpathlineto{\pgfqpoint{3.138746in}{1.663438in}}%
\pgfpathlineto{\pgfqpoint{3.147145in}{1.669730in}}%
\pgfpathlineto{\pgfqpoint{3.155535in}{1.676182in}}%
\pgfpathlineto{\pgfqpoint{3.163915in}{1.682787in}}%
\pgfpathlineto{\pgfqpoint{3.150200in}{1.690327in}}%
\pgfpathlineto{\pgfqpoint{3.136486in}{1.698083in}}%
\pgfpathlineto{\pgfqpoint{3.122772in}{1.706058in}}%
\pgfpathlineto{\pgfqpoint{3.109058in}{1.714252in}}%
\pgfpathlineto{\pgfqpoint{3.100652in}{1.708136in}}%
\pgfpathlineto{\pgfqpoint{3.092235in}{1.702182in}}%
\pgfpathlineto{\pgfqpoint{3.083809in}{1.696394in}}%
\pgfpathlineto{\pgfqpoint{3.075371in}{1.690779in}}%
\pgfpathclose%
\pgfusepath{fill}%
\end{pgfscope}%
\begin{pgfscope}%
\pgfpathrectangle{\pgfqpoint{1.150000in}{0.150000in}}{\pgfqpoint{5.700000in}{5.700000in}}%
\pgfusepath{clip}%
\pgfsetbuttcap%
\pgfsetroundjoin%
\definecolor{currentfill}{rgb}{0.268510,0.009605,0.335427}%
\pgfsetfillcolor{currentfill}%
\pgfsetfillopacity{0.800000}%
\pgfsetlinewidth{0.000000pt}%
\definecolor{currentstroke}{rgb}{0.000000,0.000000,0.000000}%
\pgfsetstrokecolor{currentstroke}%
\pgfsetdash{}{0pt}%
\pgfpathmoveto{\pgfqpoint{3.361769in}{1.642331in}}%
\pgfpathlineto{\pgfqpoint{3.375488in}{1.637987in}}%
\pgfpathlineto{\pgfqpoint{3.389211in}{1.633847in}}%
\pgfpathlineto{\pgfqpoint{3.402937in}{1.629910in}}%
\pgfpathlineto{\pgfqpoint{3.416667in}{1.626175in}}%
\pgfpathlineto{\pgfqpoint{3.424921in}{1.635269in}}%
\pgfpathlineto{\pgfqpoint{3.433168in}{1.644458in}}%
\pgfpathlineto{\pgfqpoint{3.441407in}{1.653739in}}%
\pgfpathlineto{\pgfqpoint{3.449640in}{1.663105in}}%
\pgfpathlineto{\pgfqpoint{3.435927in}{1.666406in}}%
\pgfpathlineto{\pgfqpoint{3.422218in}{1.669908in}}%
\pgfpathlineto{\pgfqpoint{3.408513in}{1.673614in}}%
\pgfpathlineto{\pgfqpoint{3.394812in}{1.677523in}}%
\pgfpathlineto{\pgfqpoint{3.386562in}{1.668579in}}%
\pgfpathlineto{\pgfqpoint{3.378305in}{1.659729in}}%
\pgfpathlineto{\pgfqpoint{3.370041in}{1.650979in}}%
\pgfpathlineto{\pgfqpoint{3.361769in}{1.642331in}}%
\pgfpathclose%
\pgfusepath{fill}%
\end{pgfscope}%
\begin{pgfscope}%
\pgfpathrectangle{\pgfqpoint{1.150000in}{0.150000in}}{\pgfqpoint{5.700000in}{5.700000in}}%
\pgfusepath{clip}%
\pgfsetbuttcap%
\pgfsetroundjoin%
\definecolor{currentfill}{rgb}{0.277134,0.185228,0.489898}%
\pgfsetfillcolor{currentfill}%
\pgfsetfillopacity{0.800000}%
\pgfsetlinewidth{0.000000pt}%
\definecolor{currentstroke}{rgb}{0.000000,0.000000,0.000000}%
\pgfsetstrokecolor{currentstroke}%
\pgfsetdash{}{0pt}%
\pgfpathmoveto{\pgfqpoint{4.062217in}{1.994527in}}%
\pgfpathlineto{\pgfqpoint{4.076087in}{1.999007in}}%
\pgfpathlineto{\pgfqpoint{4.089969in}{2.003676in}}%
\pgfpathlineto{\pgfqpoint{4.103861in}{2.008532in}}%
\pgfpathlineto{\pgfqpoint{4.117765in}{2.013575in}}%
\pgfpathlineto{\pgfqpoint{4.125771in}{2.025632in}}%
\pgfpathlineto{\pgfqpoint{4.133773in}{2.037630in}}%
\pgfpathlineto{\pgfqpoint{4.141769in}{2.049568in}}%
\pgfpathlineto{\pgfqpoint{4.149761in}{2.061443in}}%
\pgfpathlineto{\pgfqpoint{4.135861in}{2.056216in}}%
\pgfpathlineto{\pgfqpoint{4.121972in}{2.051177in}}%
\pgfpathlineto{\pgfqpoint{4.108095in}{2.046325in}}%
\pgfpathlineto{\pgfqpoint{4.094229in}{2.041662in}}%
\pgfpathlineto{\pgfqpoint{4.086233in}{2.029958in}}%
\pgfpathlineto{\pgfqpoint{4.078233in}{2.018200in}}%
\pgfpathlineto{\pgfqpoint{4.070227in}{2.006389in}}%
\pgfpathlineto{\pgfqpoint{4.062217in}{1.994527in}}%
\pgfpathclose%
\pgfusepath{fill}%
\end{pgfscope}%
\begin{pgfscope}%
\pgfpathrectangle{\pgfqpoint{1.150000in}{0.150000in}}{\pgfqpoint{5.700000in}{5.700000in}}%
\pgfusepath{clip}%
\pgfsetbuttcap%
\pgfsetroundjoin%
\definecolor{currentfill}{rgb}{0.169646,0.456262,0.558030}%
\pgfsetfillcolor{currentfill}%
\pgfsetfillopacity{0.800000}%
\pgfsetlinewidth{0.000000pt}%
\definecolor{currentstroke}{rgb}{0.000000,0.000000,0.000000}%
\pgfsetstrokecolor{currentstroke}%
\pgfsetdash{}{0pt}%
\pgfpathmoveto{\pgfqpoint{4.858451in}{2.702878in}}%
\pgfpathlineto{\pgfqpoint{4.872722in}{2.713508in}}%
\pgfpathlineto{\pgfqpoint{4.887010in}{2.724322in}}%
\pgfpathlineto{\pgfqpoint{4.901315in}{2.735320in}}%
\pgfpathlineto{\pgfqpoint{4.915637in}{2.746501in}}%
\pgfpathlineto{\pgfqpoint{4.923344in}{2.754350in}}%
\pgfpathlineto{\pgfqpoint{4.931043in}{2.762090in}}%
\pgfpathlineto{\pgfqpoint{4.938734in}{2.769722in}}%
\pgfpathlineto{\pgfqpoint{4.946418in}{2.777249in}}%
\pgfpathlineto{\pgfqpoint{4.932104in}{2.766250in}}%
\pgfpathlineto{\pgfqpoint{4.917808in}{2.755434in}}%
\pgfpathlineto{\pgfqpoint{4.903529in}{2.744801in}}%
\pgfpathlineto{\pgfqpoint{4.889267in}{2.734351in}}%
\pgfpathlineto{\pgfqpoint{4.881574in}{2.726631in}}%
\pgfpathlineto{\pgfqpoint{4.873873in}{2.718813in}}%
\pgfpathlineto{\pgfqpoint{4.866166in}{2.710897in}}%
\pgfpathlineto{\pgfqpoint{4.858451in}{2.702878in}}%
\pgfpathclose%
\pgfusepath{fill}%
\end{pgfscope}%
\begin{pgfscope}%
\pgfpathrectangle{\pgfqpoint{1.150000in}{0.150000in}}{\pgfqpoint{5.700000in}{5.700000in}}%
\pgfusepath{clip}%
\pgfsetbuttcap%
\pgfsetroundjoin%
\definecolor{currentfill}{rgb}{0.241237,0.296485,0.539709}%
\pgfsetfillcolor{currentfill}%
\pgfsetfillopacity{0.800000}%
\pgfsetlinewidth{0.000000pt}%
\definecolor{currentstroke}{rgb}{0.000000,0.000000,0.000000}%
\pgfsetstrokecolor{currentstroke}%
\pgfsetdash{}{0pt}%
\pgfpathmoveto{\pgfqpoint{4.356820in}{2.250099in}}%
\pgfpathlineto{\pgfqpoint{4.370822in}{2.257362in}}%
\pgfpathlineto{\pgfqpoint{4.384837in}{2.264810in}}%
\pgfpathlineto{\pgfqpoint{4.398866in}{2.272443in}}%
\pgfpathlineto{\pgfqpoint{4.412909in}{2.280262in}}%
\pgfpathlineto{\pgfqpoint{4.420824in}{2.291440in}}%
\pgfpathlineto{\pgfqpoint{4.428733in}{2.302520in}}%
\pgfpathlineto{\pgfqpoint{4.436636in}{2.313503in}}%
\pgfpathlineto{\pgfqpoint{4.444534in}{2.324388in}}%
\pgfpathlineto{\pgfqpoint{4.430494in}{2.316515in}}%
\pgfpathlineto{\pgfqpoint{4.416469in}{2.308828in}}%
\pgfpathlineto{\pgfqpoint{4.402457in}{2.301326in}}%
\pgfpathlineto{\pgfqpoint{4.388458in}{2.294009in}}%
\pgfpathlineto{\pgfqpoint{4.380557in}{2.283166in}}%
\pgfpathlineto{\pgfqpoint{4.372650in}{2.272233in}}%
\pgfpathlineto{\pgfqpoint{4.364738in}{2.261211in}}%
\pgfpathlineto{\pgfqpoint{4.356820in}{2.250099in}}%
\pgfpathclose%
\pgfusepath{fill}%
\end{pgfscope}%
\begin{pgfscope}%
\pgfpathrectangle{\pgfqpoint{1.150000in}{0.150000in}}{\pgfqpoint{5.700000in}{5.700000in}}%
\pgfusepath{clip}%
\pgfsetbuttcap%
\pgfsetroundjoin%
\definecolor{currentfill}{rgb}{0.172719,0.448791,0.557885}%
\pgfsetfillcolor{currentfill}%
\pgfsetfillopacity{0.800000}%
\pgfsetlinewidth{0.000000pt}%
\definecolor{currentstroke}{rgb}{0.000000,0.000000,0.000000}%
\pgfsetstrokecolor{currentstroke}%
\pgfsetdash{}{0pt}%
\pgfpathmoveto{\pgfqpoint{2.184844in}{2.777237in}}%
\pgfpathlineto{\pgfqpoint{2.199109in}{2.750766in}}%
\pgfpathlineto{\pgfqpoint{2.213357in}{2.724656in}}%
\pgfpathlineto{\pgfqpoint{2.227587in}{2.698903in}}%
\pgfpathlineto{\pgfqpoint{2.241800in}{2.673504in}}%
\pgfpathlineto{\pgfqpoint{2.250857in}{2.671121in}}%
\pgfpathlineto{\pgfqpoint{2.259894in}{2.669042in}}%
\pgfpathlineto{\pgfqpoint{2.268910in}{2.667260in}}%
\pgfpathlineto{\pgfqpoint{2.277906in}{2.665769in}}%
\pgfpathlineto{\pgfqpoint{2.263748in}{2.690617in}}%
\pgfpathlineto{\pgfqpoint{2.249574in}{2.715818in}}%
\pgfpathlineto{\pgfqpoint{2.235382in}{2.741374in}}%
\pgfpathlineto{\pgfqpoint{2.221173in}{2.767290in}}%
\pgfpathlineto{\pgfqpoint{2.212122in}{2.769319in}}%
\pgfpathlineto{\pgfqpoint{2.203050in}{2.771650in}}%
\pgfpathlineto{\pgfqpoint{2.193958in}{2.774287in}}%
\pgfpathlineto{\pgfqpoint{2.184844in}{2.777237in}}%
\pgfpathclose%
\pgfusepath{fill}%
\end{pgfscope}%
\begin{pgfscope}%
\pgfpathrectangle{\pgfqpoint{1.150000in}{0.150000in}}{\pgfqpoint{5.700000in}{5.700000in}}%
\pgfusepath{clip}%
\pgfsetbuttcap%
\pgfsetroundjoin%
\definecolor{currentfill}{rgb}{0.140210,0.665859,0.513427}%
\pgfsetfillcolor{currentfill}%
\pgfsetfillopacity{0.800000}%
\pgfsetlinewidth{0.000000pt}%
\definecolor{currentstroke}{rgb}{0.000000,0.000000,0.000000}%
\pgfsetstrokecolor{currentstroke}%
\pgfsetdash{}{0pt}%
\pgfpathmoveto{\pgfqpoint{5.710565in}{3.359913in}}%
\pgfpathlineto{\pgfqpoint{5.725339in}{3.373185in}}%
\pgfpathlineto{\pgfqpoint{5.740134in}{3.386636in}}%
\pgfpathlineto{\pgfqpoint{5.754951in}{3.400267in}}%
\pgfpathlineto{\pgfqpoint{5.769789in}{3.414079in}}%
\pgfpathlineto{\pgfqpoint{5.776982in}{3.415526in}}%
\pgfpathlineto{\pgfqpoint{5.784169in}{3.416973in}}%
\pgfpathlineto{\pgfqpoint{5.791349in}{3.418425in}}%
\pgfpathlineto{\pgfqpoint{5.798522in}{3.419890in}}%
\pgfpathlineto{\pgfqpoint{5.783714in}{3.406643in}}%
\pgfpathlineto{\pgfqpoint{5.768927in}{3.393575in}}%
\pgfpathlineto{\pgfqpoint{5.754161in}{3.380685in}}%
\pgfpathlineto{\pgfqpoint{5.739417in}{3.367975in}}%
\pgfpathlineto{\pgfqpoint{5.732213in}{3.365936in}}%
\pgfpathlineto{\pgfqpoint{5.725004in}{3.363917in}}%
\pgfpathlineto{\pgfqpoint{5.717788in}{3.361911in}}%
\pgfpathlineto{\pgfqpoint{5.710565in}{3.359913in}}%
\pgfpathclose%
\pgfusepath{fill}%
\end{pgfscope}%
\begin{pgfscope}%
\pgfpathrectangle{\pgfqpoint{1.150000in}{0.150000in}}{\pgfqpoint{5.700000in}{5.700000in}}%
\pgfusepath{clip}%
\pgfsetbuttcap%
\pgfsetroundjoin%
\definecolor{currentfill}{rgb}{0.229739,0.322361,0.545706}%
\pgfsetfillcolor{currentfill}%
\pgfsetfillopacity{0.800000}%
\pgfsetlinewidth{0.000000pt}%
\definecolor{currentstroke}{rgb}{0.000000,0.000000,0.000000}%
\pgfsetstrokecolor{currentstroke}%
\pgfsetdash{}{0pt}%
\pgfpathmoveto{\pgfqpoint{2.375372in}{2.400775in}}%
\pgfpathlineto{\pgfqpoint{2.389447in}{2.379100in}}%
\pgfpathlineto{\pgfqpoint{2.403509in}{2.357734in}}%
\pgfpathlineto{\pgfqpoint{2.417559in}{2.336676in}}%
\pgfpathlineto{\pgfqpoint{2.431596in}{2.315923in}}%
\pgfpathlineto{\pgfqpoint{2.440519in}{2.314637in}}%
\pgfpathlineto{\pgfqpoint{2.449423in}{2.313638in}}%
\pgfpathlineto{\pgfqpoint{2.458309in}{2.312922in}}%
\pgfpathlineto{\pgfqpoint{2.467177in}{2.312482in}}%
\pgfpathlineto{\pgfqpoint{2.453189in}{2.332668in}}%
\pgfpathlineto{\pgfqpoint{2.439190in}{2.353158in}}%
\pgfpathlineto{\pgfqpoint{2.425179in}{2.373955in}}%
\pgfpathlineto{\pgfqpoint{2.411155in}{2.395060in}}%
\pgfpathlineto{\pgfqpoint{2.402238in}{2.396055in}}%
\pgfpathlineto{\pgfqpoint{2.393302in}{2.397335in}}%
\pgfpathlineto{\pgfqpoint{2.384347in}{2.398907in}}%
\pgfpathlineto{\pgfqpoint{2.375372in}{2.400775in}}%
\pgfpathclose%
\pgfusepath{fill}%
\end{pgfscope}%
\begin{pgfscope}%
\pgfpathrectangle{\pgfqpoint{1.150000in}{0.150000in}}{\pgfqpoint{5.700000in}{5.700000in}}%
\pgfusepath{clip}%
\pgfsetbuttcap%
\pgfsetroundjoin%
\definecolor{currentfill}{rgb}{0.280894,0.078907,0.402329}%
\pgfsetfillcolor{currentfill}%
\pgfsetfillopacity{0.800000}%
\pgfsetlinewidth{0.000000pt}%
\definecolor{currentstroke}{rgb}{0.000000,0.000000,0.000000}%
\pgfsetstrokecolor{currentstroke}%
\pgfsetdash{}{0pt}%
\pgfpathmoveto{\pgfqpoint{2.876138in}{1.798482in}}%
\pgfpathlineto{\pgfqpoint{2.889933in}{1.786517in}}%
\pgfpathlineto{\pgfqpoint{2.903725in}{1.774788in}}%
\pgfpathlineto{\pgfqpoint{2.917513in}{1.763292in}}%
\pgfpathlineto{\pgfqpoint{2.931299in}{1.752029in}}%
\pgfpathlineto{\pgfqpoint{2.939845in}{1.755810in}}%
\pgfpathlineto{\pgfqpoint{2.948378in}{1.759800in}}%
\pgfpathlineto{\pgfqpoint{2.956898in}{1.763993in}}%
\pgfpathlineto{\pgfqpoint{2.965407in}{1.768384in}}%
\pgfpathlineto{\pgfqpoint{2.951654in}{1.779109in}}%
\pgfpathlineto{\pgfqpoint{2.937899in}{1.790066in}}%
\pgfpathlineto{\pgfqpoint{2.924141in}{1.801257in}}%
\pgfpathlineto{\pgfqpoint{2.910380in}{1.812682in}}%
\pgfpathlineto{\pgfqpoint{2.901839in}{1.808817in}}%
\pgfpathlineto{\pgfqpoint{2.893285in}{1.805158in}}%
\pgfpathlineto{\pgfqpoint{2.884718in}{1.801712in}}%
\pgfpathlineto{\pgfqpoint{2.876138in}{1.798482in}}%
\pgfpathclose%
\pgfusepath{fill}%
\end{pgfscope}%
\begin{pgfscope}%
\pgfpathrectangle{\pgfqpoint{1.150000in}{0.150000in}}{\pgfqpoint{5.700000in}{5.700000in}}%
\pgfusepath{clip}%
\pgfsetbuttcap%
\pgfsetroundjoin%
\definecolor{currentfill}{rgb}{0.129933,0.559582,0.551864}%
\pgfsetfillcolor{currentfill}%
\pgfsetfillopacity{0.800000}%
\pgfsetlinewidth{0.000000pt}%
\definecolor{currentstroke}{rgb}{0.000000,0.000000,0.000000}%
\pgfsetstrokecolor{currentstroke}%
\pgfsetdash{}{0pt}%
\pgfpathmoveto{\pgfqpoint{5.240936in}{3.019061in}}%
\pgfpathlineto{\pgfqpoint{5.255438in}{3.031364in}}%
\pgfpathlineto{\pgfqpoint{5.269958in}{3.043848in}}%
\pgfpathlineto{\pgfqpoint{5.284498in}{3.056514in}}%
\pgfpathlineto{\pgfqpoint{5.299058in}{3.069363in}}%
\pgfpathlineto{\pgfqpoint{5.306556in}{3.074189in}}%
\pgfpathlineto{\pgfqpoint{5.314047in}{3.078934in}}%
\pgfpathlineto{\pgfqpoint{5.321529in}{3.083602in}}%
\pgfpathlineto{\pgfqpoint{5.329003in}{3.088198in}}%
\pgfpathlineto{\pgfqpoint{5.314461in}{3.075704in}}%
\pgfpathlineto{\pgfqpoint{5.299937in}{3.063393in}}%
\pgfpathlineto{\pgfqpoint{5.285433in}{3.051262in}}%
\pgfpathlineto{\pgfqpoint{5.270948in}{3.039313in}}%
\pgfpathlineto{\pgfqpoint{5.263457in}{3.034351in}}%
\pgfpathlineto{\pgfqpoint{5.255958in}{3.029324in}}%
\pgfpathlineto{\pgfqpoint{5.248451in}{3.024229in}}%
\pgfpathlineto{\pgfqpoint{5.240936in}{3.019061in}}%
\pgfpathclose%
\pgfusepath{fill}%
\end{pgfscope}%
\begin{pgfscope}%
\pgfpathrectangle{\pgfqpoint{1.150000in}{0.150000in}}{\pgfqpoint{5.700000in}{5.700000in}}%
\pgfusepath{clip}%
\pgfsetbuttcap%
\pgfsetroundjoin%
\definecolor{currentfill}{rgb}{0.278791,0.062145,0.386592}%
\pgfsetfillcolor{currentfill}%
\pgfsetfillopacity{0.800000}%
\pgfsetlinewidth{0.000000pt}%
\definecolor{currentstroke}{rgb}{0.000000,0.000000,0.000000}%
\pgfsetstrokecolor{currentstroke}%
\pgfsetdash{}{0pt}%
\pgfpathmoveto{\pgfqpoint{3.679889in}{1.725912in}}%
\pgfpathlineto{\pgfqpoint{3.693650in}{1.725953in}}%
\pgfpathlineto{\pgfqpoint{3.707417in}{1.726188in}}%
\pgfpathlineto{\pgfqpoint{3.721192in}{1.726616in}}%
\pgfpathlineto{\pgfqpoint{3.734974in}{1.727236in}}%
\pgfpathlineto{\pgfqpoint{3.743103in}{1.738624in}}%
\pgfpathlineto{\pgfqpoint{3.751226in}{1.750032in}}%
\pgfpathlineto{\pgfqpoint{3.759343in}{1.761456in}}%
\pgfpathlineto{\pgfqpoint{3.767456in}{1.772893in}}%
\pgfpathlineto{\pgfqpoint{3.753683in}{1.771933in}}%
\pgfpathlineto{\pgfqpoint{3.739917in}{1.771164in}}%
\pgfpathlineto{\pgfqpoint{3.726159in}{1.770589in}}%
\pgfpathlineto{\pgfqpoint{3.712408in}{1.770207in}}%
\pgfpathlineto{\pgfqpoint{3.704287in}{1.759099in}}%
\pgfpathlineto{\pgfqpoint{3.696160in}{1.748011in}}%
\pgfpathlineto{\pgfqpoint{3.688027in}{1.736948in}}%
\pgfpathlineto{\pgfqpoint{3.679889in}{1.725912in}}%
\pgfpathclose%
\pgfusepath{fill}%
\end{pgfscope}%
\begin{pgfscope}%
\pgfpathrectangle{\pgfqpoint{1.150000in}{0.150000in}}{\pgfqpoint{5.700000in}{5.700000in}}%
\pgfusepath{clip}%
\pgfsetbuttcap%
\pgfsetroundjoin%
\definecolor{currentfill}{rgb}{0.269308,0.218818,0.509577}%
\pgfsetfillcolor{currentfill}%
\pgfsetfillopacity{0.800000}%
\pgfsetlinewidth{0.000000pt}%
\definecolor{currentstroke}{rgb}{0.000000,0.000000,0.000000}%
\pgfsetstrokecolor{currentstroke}%
\pgfsetdash{}{0pt}%
\pgfpathmoveto{\pgfqpoint{4.149761in}{2.061443in}}%
\pgfpathlineto{\pgfqpoint{4.163673in}{2.066857in}}%
\pgfpathlineto{\pgfqpoint{4.177596in}{2.072458in}}%
\pgfpathlineto{\pgfqpoint{4.191531in}{2.078245in}}%
\pgfpathlineto{\pgfqpoint{4.205478in}{2.084220in}}%
\pgfpathlineto{\pgfqpoint{4.213462in}{2.096195in}}%
\pgfpathlineto{\pgfqpoint{4.221440in}{2.108098in}}%
\pgfpathlineto{\pgfqpoint{4.229414in}{2.119926in}}%
\pgfpathlineto{\pgfqpoint{4.237382in}{2.131680in}}%
\pgfpathlineto{\pgfqpoint{4.223438in}{2.125554in}}%
\pgfpathlineto{\pgfqpoint{4.209506in}{2.119615in}}%
\pgfpathlineto{\pgfqpoint{4.195586in}{2.113862in}}%
\pgfpathlineto{\pgfqpoint{4.181678in}{2.108297in}}%
\pgfpathlineto{\pgfqpoint{4.173706in}{2.096683in}}%
\pgfpathlineto{\pgfqpoint{4.165730in}{2.085002in}}%
\pgfpathlineto{\pgfqpoint{4.157748in}{2.073254in}}%
\pgfpathlineto{\pgfqpoint{4.149761in}{2.061443in}}%
\pgfpathclose%
\pgfusepath{fill}%
\end{pgfscope}%
\begin{pgfscope}%
\pgfpathrectangle{\pgfqpoint{1.150000in}{0.150000in}}{\pgfqpoint{5.700000in}{5.700000in}}%
\pgfusepath{clip}%
\pgfsetbuttcap%
\pgfsetroundjoin%
\definecolor{currentfill}{rgb}{0.162016,0.687316,0.499129}%
\pgfsetfillcolor{currentfill}%
\pgfsetfillopacity{0.800000}%
\pgfsetlinewidth{0.000000pt}%
\definecolor{currentstroke}{rgb}{0.000000,0.000000,0.000000}%
\pgfsetstrokecolor{currentstroke}%
\pgfsetdash{}{0pt}%
\pgfpathmoveto{\pgfqpoint{5.798522in}{3.419890in}}%
\pgfpathlineto{\pgfqpoint{5.813351in}{3.433316in}}%
\pgfpathlineto{\pgfqpoint{5.828202in}{3.446922in}}%
\pgfpathlineto{\pgfqpoint{5.843075in}{3.460708in}}%
\pgfpathlineto{\pgfqpoint{5.857970in}{3.474673in}}%
\pgfpathlineto{\pgfqpoint{5.865105in}{3.475570in}}%
\pgfpathlineto{\pgfqpoint{5.872233in}{3.476483in}}%
\pgfpathlineto{\pgfqpoint{5.879355in}{3.477420in}}%
\pgfpathlineto{\pgfqpoint{5.886470in}{3.478387in}}%
\pgfpathlineto{\pgfqpoint{5.871608in}{3.465021in}}%
\pgfpathlineto{\pgfqpoint{5.856768in}{3.451834in}}%
\pgfpathlineto{\pgfqpoint{5.841949in}{3.438825in}}%
\pgfpathlineto{\pgfqpoint{5.827152in}{3.425994in}}%
\pgfpathlineto{\pgfqpoint{5.820003in}{3.424419in}}%
\pgfpathlineto{\pgfqpoint{5.812849in}{3.422880in}}%
\pgfpathlineto{\pgfqpoint{5.805688in}{3.421373in}}%
\pgfpathlineto{\pgfqpoint{5.798522in}{3.419890in}}%
\pgfpathclose%
\pgfusepath{fill}%
\end{pgfscope}%
\begin{pgfscope}%
\pgfpathrectangle{\pgfqpoint{1.150000in}{0.150000in}}{\pgfqpoint{5.700000in}{5.700000in}}%
\pgfusepath{clip}%
\pgfsetbuttcap%
\pgfsetroundjoin%
\definecolor{currentfill}{rgb}{0.281924,0.089666,0.412415}%
\pgfsetfillcolor{currentfill}%
\pgfsetfillopacity{0.800000}%
\pgfsetlinewidth{0.000000pt}%
\definecolor{currentstroke}{rgb}{0.000000,0.000000,0.000000}%
\pgfsetstrokecolor{currentstroke}%
\pgfsetdash{}{0pt}%
\pgfpathmoveto{\pgfqpoint{3.767456in}{1.772893in}}%
\pgfpathlineto{\pgfqpoint{3.781237in}{1.774047in}}%
\pgfpathlineto{\pgfqpoint{3.795026in}{1.775391in}}%
\pgfpathlineto{\pgfqpoint{3.808824in}{1.776927in}}%
\pgfpathlineto{\pgfqpoint{3.822630in}{1.778654in}}%
\pgfpathlineto{\pgfqpoint{3.830729in}{1.790423in}}%
\pgfpathlineto{\pgfqpoint{3.838824in}{1.802192in}}%
\pgfpathlineto{\pgfqpoint{3.846913in}{1.813957in}}%
\pgfpathlineto{\pgfqpoint{3.854998in}{1.825716in}}%
\pgfpathlineto{\pgfqpoint{3.841199in}{1.823680in}}%
\pgfpathlineto{\pgfqpoint{3.827409in}{1.821834in}}%
\pgfpathlineto{\pgfqpoint{3.813628in}{1.820180in}}%
\pgfpathlineto{\pgfqpoint{3.799854in}{1.818717in}}%
\pgfpathlineto{\pgfqpoint{3.791762in}{1.807256in}}%
\pgfpathlineto{\pgfqpoint{3.783665in}{1.795796in}}%
\pgfpathlineto{\pgfqpoint{3.775563in}{1.784341in}}%
\pgfpathlineto{\pgfqpoint{3.767456in}{1.772893in}}%
\pgfpathclose%
\pgfusepath{fill}%
\end{pgfscope}%
\begin{pgfscope}%
\pgfpathrectangle{\pgfqpoint{1.150000in}{0.150000in}}{\pgfqpoint{5.700000in}{5.700000in}}%
\pgfusepath{clip}%
\pgfsetbuttcap%
\pgfsetroundjoin%
\definecolor{currentfill}{rgb}{0.276022,0.044167,0.370164}%
\pgfsetfillcolor{currentfill}%
\pgfsetfillopacity{0.800000}%
\pgfsetlinewidth{0.000000pt}%
\definecolor{currentstroke}{rgb}{0.000000,0.000000,0.000000}%
\pgfsetstrokecolor{currentstroke}%
\pgfsetdash{}{0pt}%
\pgfpathmoveto{\pgfqpoint{3.592263in}{1.685375in}}%
\pgfpathlineto{\pgfqpoint{3.606008in}{1.684265in}}%
\pgfpathlineto{\pgfqpoint{3.619760in}{1.683351in}}%
\pgfpathlineto{\pgfqpoint{3.633518in}{1.682632in}}%
\pgfpathlineto{\pgfqpoint{3.647282in}{1.682107in}}%
\pgfpathlineto{\pgfqpoint{3.655442in}{1.693001in}}%
\pgfpathlineto{\pgfqpoint{3.663597in}{1.703935in}}%
\pgfpathlineto{\pgfqpoint{3.671746in}{1.714907in}}%
\pgfpathlineto{\pgfqpoint{3.679889in}{1.725912in}}%
\pgfpathlineto{\pgfqpoint{3.666136in}{1.726065in}}%
\pgfpathlineto{\pgfqpoint{3.652389in}{1.726412in}}%
\pgfpathlineto{\pgfqpoint{3.638649in}{1.726954in}}%
\pgfpathlineto{\pgfqpoint{3.624916in}{1.727692in}}%
\pgfpathlineto{\pgfqpoint{3.616761in}{1.717047in}}%
\pgfpathlineto{\pgfqpoint{3.608601in}{1.706443in}}%
\pgfpathlineto{\pgfqpoint{3.600435in}{1.695885in}}%
\pgfpathlineto{\pgfqpoint{3.592263in}{1.685375in}}%
\pgfpathclose%
\pgfusepath{fill}%
\end{pgfscope}%
\begin{pgfscope}%
\pgfpathrectangle{\pgfqpoint{1.150000in}{0.150000in}}{\pgfqpoint{5.700000in}{5.700000in}}%
\pgfusepath{clip}%
\pgfsetbuttcap%
\pgfsetroundjoin%
\definecolor{currentfill}{rgb}{0.195860,0.395433,0.555276}%
\pgfsetfillcolor{currentfill}%
\pgfsetfillopacity{0.800000}%
\pgfsetlinewidth{0.000000pt}%
\definecolor{currentstroke}{rgb}{0.000000,0.000000,0.000000}%
\pgfsetstrokecolor{currentstroke}%
\pgfsetdash{}{0pt}%
\pgfpathmoveto{\pgfqpoint{4.651588in}{2.516856in}}%
\pgfpathlineto{\pgfqpoint{4.665748in}{2.526355in}}%
\pgfpathlineto{\pgfqpoint{4.679924in}{2.536038in}}%
\pgfpathlineto{\pgfqpoint{4.694116in}{2.545906in}}%
\pgfpathlineto{\pgfqpoint{4.708324in}{2.555958in}}%
\pgfpathlineto{\pgfqpoint{4.716129in}{2.565411in}}%
\pgfpathlineto{\pgfqpoint{4.723927in}{2.574751in}}%
\pgfpathlineto{\pgfqpoint{4.731719in}{2.583978in}}%
\pgfpathlineto{\pgfqpoint{4.739503in}{2.593093in}}%
\pgfpathlineto{\pgfqpoint{4.725301in}{2.583121in}}%
\pgfpathlineto{\pgfqpoint{4.711115in}{2.573333in}}%
\pgfpathlineto{\pgfqpoint{4.696944in}{2.563729in}}%
\pgfpathlineto{\pgfqpoint{4.682790in}{2.554309in}}%
\pgfpathlineto{\pgfqpoint{4.674999in}{2.545102in}}%
\pgfpathlineto{\pgfqpoint{4.667202in}{2.535791in}}%
\pgfpathlineto{\pgfqpoint{4.659398in}{2.526377in}}%
\pgfpathlineto{\pgfqpoint{4.651588in}{2.516856in}}%
\pgfpathclose%
\pgfusepath{fill}%
\end{pgfscope}%
\begin{pgfscope}%
\pgfpathrectangle{\pgfqpoint{1.150000in}{0.150000in}}{\pgfqpoint{5.700000in}{5.700000in}}%
\pgfusepath{clip}%
\pgfsetbuttcap%
\pgfsetroundjoin%
\definecolor{currentfill}{rgb}{0.268510,0.009605,0.335427}%
\pgfsetfillcolor{currentfill}%
\pgfsetfillopacity{0.800000}%
\pgfsetlinewidth{0.000000pt}%
\definecolor{currentstroke}{rgb}{0.000000,0.000000,0.000000}%
\pgfsetstrokecolor{currentstroke}%
\pgfsetdash{}{0pt}%
\pgfpathmoveto{\pgfqpoint{3.273676in}{1.630152in}}%
\pgfpathlineto{\pgfqpoint{3.287404in}{1.624518in}}%
\pgfpathlineto{\pgfqpoint{3.301135in}{1.619092in}}%
\pgfpathlineto{\pgfqpoint{3.314868in}{1.613872in}}%
\pgfpathlineto{\pgfqpoint{3.328603in}{1.608857in}}%
\pgfpathlineto{\pgfqpoint{3.336907in}{1.617049in}}%
\pgfpathlineto{\pgfqpoint{3.345202in}{1.625362in}}%
\pgfpathlineto{\pgfqpoint{3.353489in}{1.633791in}}%
\pgfpathlineto{\pgfqpoint{3.361769in}{1.642331in}}%
\pgfpathlineto{\pgfqpoint{3.348053in}{1.646879in}}%
\pgfpathlineto{\pgfqpoint{3.334340in}{1.651633in}}%
\pgfpathlineto{\pgfqpoint{3.320630in}{1.656592in}}%
\pgfpathlineto{\pgfqpoint{3.306923in}{1.661759in}}%
\pgfpathlineto{\pgfqpoint{3.298624in}{1.653674in}}%
\pgfpathlineto{\pgfqpoint{3.290316in}{1.645708in}}%
\pgfpathlineto{\pgfqpoint{3.282001in}{1.637866in}}%
\pgfpathlineto{\pgfqpoint{3.273676in}{1.630152in}}%
\pgfpathclose%
\pgfusepath{fill}%
\end{pgfscope}%
\begin{pgfscope}%
\pgfpathrectangle{\pgfqpoint{1.150000in}{0.150000in}}{\pgfqpoint{5.700000in}{5.700000in}}%
\pgfusepath{clip}%
\pgfsetbuttcap%
\pgfsetroundjoin%
\definecolor{currentfill}{rgb}{0.159194,0.482237,0.558073}%
\pgfsetfillcolor{currentfill}%
\pgfsetfillopacity{0.800000}%
\pgfsetlinewidth{0.000000pt}%
\definecolor{currentstroke}{rgb}{0.000000,0.000000,0.000000}%
\pgfsetstrokecolor{currentstroke}%
\pgfsetdash{}{0pt}%
\pgfpathmoveto{\pgfqpoint{4.946418in}{2.777249in}}%
\pgfpathlineto{\pgfqpoint{4.960749in}{2.788431in}}%
\pgfpathlineto{\pgfqpoint{4.975098in}{2.799797in}}%
\pgfpathlineto{\pgfqpoint{4.989464in}{2.811345in}}%
\pgfpathlineto{\pgfqpoint{5.003848in}{2.823078in}}%
\pgfpathlineto{\pgfqpoint{5.011515in}{2.830298in}}%
\pgfpathlineto{\pgfqpoint{5.019174in}{2.837409in}}%
\pgfpathlineto{\pgfqpoint{5.026826in}{2.844414in}}%
\pgfpathlineto{\pgfqpoint{5.034469in}{2.851316in}}%
\pgfpathlineto{\pgfqpoint{5.020095in}{2.839801in}}%
\pgfpathlineto{\pgfqpoint{5.005739in}{2.828468in}}%
\pgfpathlineto{\pgfqpoint{4.991400in}{2.817319in}}%
\pgfpathlineto{\pgfqpoint{4.977079in}{2.806352in}}%
\pgfpathlineto{\pgfqpoint{4.969425in}{2.799221in}}%
\pgfpathlineto{\pgfqpoint{4.961763in}{2.791996in}}%
\pgfpathlineto{\pgfqpoint{4.954094in}{2.784672in}}%
\pgfpathlineto{\pgfqpoint{4.946418in}{2.777249in}}%
\pgfpathclose%
\pgfusepath{fill}%
\end{pgfscope}%
\begin{pgfscope}%
\pgfpathrectangle{\pgfqpoint{1.150000in}{0.150000in}}{\pgfqpoint{5.700000in}{5.700000in}}%
\pgfusepath{clip}%
\pgfsetbuttcap%
\pgfsetroundjoin%
\definecolor{currentfill}{rgb}{0.283197,0.115680,0.436115}%
\pgfsetfillcolor{currentfill}%
\pgfsetfillopacity{0.800000}%
\pgfsetlinewidth{0.000000pt}%
\definecolor{currentstroke}{rgb}{0.000000,0.000000,0.000000}%
\pgfsetstrokecolor{currentstroke}%
\pgfsetdash{}{0pt}%
\pgfpathmoveto{\pgfqpoint{3.854998in}{1.825716in}}%
\pgfpathlineto{\pgfqpoint{3.868805in}{1.827943in}}%
\pgfpathlineto{\pgfqpoint{3.882621in}{1.830360in}}%
\pgfpathlineto{\pgfqpoint{3.896447in}{1.832967in}}%
\pgfpathlineto{\pgfqpoint{3.910282in}{1.835763in}}%
\pgfpathlineto{\pgfqpoint{3.918355in}{1.847804in}}%
\pgfpathlineto{\pgfqpoint{3.926423in}{1.859826in}}%
\pgfpathlineto{\pgfqpoint{3.934487in}{1.871826in}}%
\pgfpathlineto{\pgfqpoint{3.942546in}{1.883803in}}%
\pgfpathlineto{\pgfqpoint{3.928717in}{1.880728in}}%
\pgfpathlineto{\pgfqpoint{3.914897in}{1.877842in}}%
\pgfpathlineto{\pgfqpoint{3.901087in}{1.875147in}}%
\pgfpathlineto{\pgfqpoint{3.887286in}{1.872642in}}%
\pgfpathlineto{\pgfqpoint{3.879221in}{1.860932in}}%
\pgfpathlineto{\pgfqpoint{3.871152in}{1.849206in}}%
\pgfpathlineto{\pgfqpoint{3.863077in}{1.837467in}}%
\pgfpathlineto{\pgfqpoint{3.854998in}{1.825716in}}%
\pgfpathclose%
\pgfusepath{fill}%
\end{pgfscope}%
\begin{pgfscope}%
\pgfpathrectangle{\pgfqpoint{1.150000in}{0.150000in}}{\pgfqpoint{5.700000in}{5.700000in}}%
\pgfusepath{clip}%
\pgfsetbuttcap%
\pgfsetroundjoin%
\definecolor{currentfill}{rgb}{0.185783,0.704891,0.485273}%
\pgfsetfillcolor{currentfill}%
\pgfsetfillopacity{0.800000}%
\pgfsetlinewidth{0.000000pt}%
\definecolor{currentstroke}{rgb}{0.000000,0.000000,0.000000}%
\pgfsetstrokecolor{currentstroke}%
\pgfsetdash{}{0pt}%
\pgfpathmoveto{\pgfqpoint{5.886470in}{3.478387in}}%
\pgfpathlineto{\pgfqpoint{5.901354in}{3.491931in}}%
\pgfpathlineto{\pgfqpoint{5.916260in}{3.505655in}}%
\pgfpathlineto{\pgfqpoint{5.931188in}{3.519558in}}%
\pgfpathlineto{\pgfqpoint{5.946138in}{3.533641in}}%
\pgfpathlineto{\pgfqpoint{5.953213in}{3.534022in}}%
\pgfpathlineto{\pgfqpoint{5.960282in}{3.534440in}}%
\pgfpathlineto{\pgfqpoint{5.967346in}{3.534899in}}%
\pgfpathlineto{\pgfqpoint{5.974404in}{3.535408in}}%
\pgfpathlineto{\pgfqpoint{5.959490in}{3.521960in}}%
\pgfpathlineto{\pgfqpoint{5.944597in}{3.508690in}}%
\pgfpathlineto{\pgfqpoint{5.929727in}{3.495598in}}%
\pgfpathlineto{\pgfqpoint{5.914878in}{3.482684in}}%
\pgfpathlineto{\pgfqpoint{5.907784in}{3.481532in}}%
\pgfpathlineto{\pgfqpoint{5.900685in}{3.480436in}}%
\pgfpathlineto{\pgfqpoint{5.893580in}{3.479390in}}%
\pgfpathlineto{\pgfqpoint{5.886470in}{3.478387in}}%
\pgfpathclose%
\pgfusepath{fill}%
\end{pgfscope}%
\begin{pgfscope}%
\pgfpathrectangle{\pgfqpoint{1.150000in}{0.150000in}}{\pgfqpoint{5.700000in}{5.700000in}}%
\pgfusepath{clip}%
\pgfsetbuttcap%
\pgfsetroundjoin%
\definecolor{currentfill}{rgb}{0.214298,0.355619,0.551184}%
\pgfsetfillcolor{currentfill}%
\pgfsetfillopacity{0.800000}%
\pgfsetlinewidth{0.000000pt}%
\definecolor{currentstroke}{rgb}{0.000000,0.000000,0.000000}%
\pgfsetstrokecolor{currentstroke}%
\pgfsetdash{}{0pt}%
\pgfpathmoveto{\pgfqpoint{2.318939in}{2.490637in}}%
\pgfpathlineto{\pgfqpoint{2.333068in}{2.467692in}}%
\pgfpathlineto{\pgfqpoint{2.347183in}{2.445068in}}%
\pgfpathlineto{\pgfqpoint{2.361284in}{2.422764in}}%
\pgfpathlineto{\pgfqpoint{2.375372in}{2.400775in}}%
\pgfpathlineto{\pgfqpoint{2.384347in}{2.398907in}}%
\pgfpathlineto{\pgfqpoint{2.393302in}{2.397335in}}%
\pgfpathlineto{\pgfqpoint{2.402238in}{2.396055in}}%
\pgfpathlineto{\pgfqpoint{2.411155in}{2.395060in}}%
\pgfpathlineto{\pgfqpoint{2.397119in}{2.416477in}}%
\pgfpathlineto{\pgfqpoint{2.383070in}{2.438208in}}%
\pgfpathlineto{\pgfqpoint{2.369008in}{2.460257in}}%
\pgfpathlineto{\pgfqpoint{2.354933in}{2.482627in}}%
\pgfpathlineto{\pgfqpoint{2.345964in}{2.484182in}}%
\pgfpathlineto{\pgfqpoint{2.336976in}{2.486031in}}%
\pgfpathlineto{\pgfqpoint{2.327968in}{2.488181in}}%
\pgfpathlineto{\pgfqpoint{2.318939in}{2.490637in}}%
\pgfpathclose%
\pgfusepath{fill}%
\end{pgfscope}%
\begin{pgfscope}%
\pgfpathrectangle{\pgfqpoint{1.150000in}{0.150000in}}{\pgfqpoint{5.700000in}{5.700000in}}%
\pgfusepath{clip}%
\pgfsetbuttcap%
\pgfsetroundjoin%
\definecolor{currentfill}{rgb}{0.271305,0.019942,0.347269}%
\pgfsetfillcolor{currentfill}%
\pgfsetfillopacity{0.800000}%
\pgfsetlinewidth{0.000000pt}%
\definecolor{currentstroke}{rgb}{0.000000,0.000000,0.000000}%
\pgfsetstrokecolor{currentstroke}%
\pgfsetdash{}{0pt}%
\pgfpathmoveto{\pgfqpoint{3.130336in}{1.657309in}}%
\pgfpathlineto{\pgfqpoint{3.144077in}{1.649484in}}%
\pgfpathlineto{\pgfqpoint{3.157819in}{1.641875in}}%
\pgfpathlineto{\pgfqpoint{3.171562in}{1.634479in}}%
\pgfpathlineto{\pgfqpoint{3.185305in}{1.627296in}}%
\pgfpathlineto{\pgfqpoint{3.193689in}{1.633938in}}%
\pgfpathlineto{\pgfqpoint{3.202063in}{1.640735in}}%
\pgfpathlineto{\pgfqpoint{3.210428in}{1.647682in}}%
\pgfpathlineto{\pgfqpoint{3.218784in}{1.654775in}}%
\pgfpathlineto{\pgfqpoint{3.205065in}{1.661458in}}%
\pgfpathlineto{\pgfqpoint{3.191348in}{1.668354in}}%
\pgfpathlineto{\pgfqpoint{3.177631in}{1.675463in}}%
\pgfpathlineto{\pgfqpoint{3.163915in}{1.682787in}}%
\pgfpathlineto{\pgfqpoint{3.155535in}{1.676182in}}%
\pgfpathlineto{\pgfqpoint{3.147145in}{1.669730in}}%
\pgfpathlineto{\pgfqpoint{3.138746in}{1.663438in}}%
\pgfpathlineto{\pgfqpoint{3.130336in}{1.657309in}}%
\pgfpathclose%
\pgfusepath{fill}%
\end{pgfscope}%
\begin{pgfscope}%
\pgfpathrectangle{\pgfqpoint{1.150000in}{0.150000in}}{\pgfqpoint{5.700000in}{5.700000in}}%
\pgfusepath{clip}%
\pgfsetbuttcap%
\pgfsetroundjoin%
\definecolor{currentfill}{rgb}{0.272594,0.025563,0.353093}%
\pgfsetfillcolor{currentfill}%
\pgfsetfillopacity{0.800000}%
\pgfsetlinewidth{0.000000pt}%
\definecolor{currentstroke}{rgb}{0.000000,0.000000,0.000000}%
\pgfsetstrokecolor{currentstroke}%
\pgfsetdash{}{0pt}%
\pgfpathmoveto{\pgfqpoint{3.504537in}{1.651910in}}%
\pgfpathlineto{\pgfqpoint{3.518273in}{1.649609in}}%
\pgfpathlineto{\pgfqpoint{3.532015in}{1.647506in}}%
\pgfpathlineto{\pgfqpoint{3.545762in}{1.645601in}}%
\pgfpathlineto{\pgfqpoint{3.559514in}{1.643892in}}%
\pgfpathlineto{\pgfqpoint{3.567711in}{1.654172in}}%
\pgfpathlineto{\pgfqpoint{3.575901in}{1.664514in}}%
\pgfpathlineto{\pgfqpoint{3.584085in}{1.674917in}}%
\pgfpathlineto{\pgfqpoint{3.592263in}{1.685375in}}%
\pgfpathlineto{\pgfqpoint{3.578524in}{1.686681in}}%
\pgfpathlineto{\pgfqpoint{3.564790in}{1.688183in}}%
\pgfpathlineto{\pgfqpoint{3.551062in}{1.689883in}}%
\pgfpathlineto{\pgfqpoint{3.537340in}{1.691780in}}%
\pgfpathlineto{\pgfqpoint{3.529149in}{1.681714in}}%
\pgfpathlineto{\pgfqpoint{3.520951in}{1.671711in}}%
\pgfpathlineto{\pgfqpoint{3.512747in}{1.661775in}}%
\pgfpathlineto{\pgfqpoint{3.504537in}{1.651910in}}%
\pgfpathclose%
\pgfusepath{fill}%
\end{pgfscope}%
\begin{pgfscope}%
\pgfpathrectangle{\pgfqpoint{1.150000in}{0.150000in}}{\pgfqpoint{5.700000in}{5.700000in}}%
\pgfusepath{clip}%
\pgfsetbuttcap%
\pgfsetroundjoin%
\definecolor{currentfill}{rgb}{0.246070,0.738910,0.452024}%
\pgfsetfillcolor{currentfill}%
\pgfsetfillopacity{0.800000}%
\pgfsetlinewidth{0.000000pt}%
\definecolor{currentstroke}{rgb}{0.000000,0.000000,0.000000}%
\pgfsetstrokecolor{currentstroke}%
\pgfsetdash{}{0pt}%
\pgfpathmoveto{\pgfqpoint{6.062317in}{3.590981in}}%
\pgfpathlineto{\pgfqpoint{6.077305in}{3.604653in}}%
\pgfpathlineto{\pgfqpoint{6.092315in}{3.618503in}}%
\pgfpathlineto{\pgfqpoint{6.107349in}{3.632532in}}%
\pgfpathlineto{\pgfqpoint{6.114315in}{3.632188in}}%
\pgfpathlineto{\pgfqpoint{6.121276in}{3.631922in}}%
\pgfpathlineto{\pgfqpoint{6.128234in}{3.631743in}}%
\pgfpathlineto{\pgfqpoint{6.135188in}{3.631657in}}%
\pgfpathlineto{\pgfqpoint{6.120196in}{3.618332in}}%
\pgfpathlineto{\pgfqpoint{6.105227in}{3.605183in}}%
\pgfpathlineto{\pgfqpoint{6.090279in}{3.592211in}}%
\pgfpathlineto{\pgfqpoint{6.083294in}{3.591762in}}%
\pgfpathlineto{\pgfqpoint{6.076305in}{3.591413in}}%
\pgfpathlineto{\pgfqpoint{6.069313in}{3.591155in}}%
\pgfpathlineto{\pgfqpoint{6.062317in}{3.590981in}}%
\pgfpathclose%
\pgfusepath{fill}%
\end{pgfscope}%
\begin{pgfscope}%
\pgfpathrectangle{\pgfqpoint{1.150000in}{0.150000in}}{\pgfqpoint{5.700000in}{5.700000in}}%
\pgfusepath{clip}%
\pgfsetbuttcap%
\pgfsetroundjoin%
\definecolor{currentfill}{rgb}{0.278791,0.062145,0.386592}%
\pgfsetfillcolor{currentfill}%
\pgfsetfillopacity{0.800000}%
\pgfsetlinewidth{0.000000pt}%
\definecolor{currentstroke}{rgb}{0.000000,0.000000,0.000000}%
\pgfsetstrokecolor{currentstroke}%
\pgfsetdash{}{0pt}%
\pgfpathmoveto{\pgfqpoint{2.931299in}{1.752029in}}%
\pgfpathlineto{\pgfqpoint{2.945083in}{1.740998in}}%
\pgfpathlineto{\pgfqpoint{2.958864in}{1.730195in}}%
\pgfpathlineto{\pgfqpoint{2.972643in}{1.719622in}}%
\pgfpathlineto{\pgfqpoint{2.986420in}{1.709275in}}%
\pgfpathlineto{\pgfqpoint{2.994933in}{1.713605in}}%
\pgfpathlineto{\pgfqpoint{3.003433in}{1.718136in}}%
\pgfpathlineto{\pgfqpoint{3.011922in}{1.722862in}}%
\pgfpathlineto{\pgfqpoint{3.020399in}{1.727778in}}%
\pgfpathlineto{\pgfqpoint{3.006654in}{1.737588in}}%
\pgfpathlineto{\pgfqpoint{2.992906in}{1.747625in}}%
\pgfpathlineto{\pgfqpoint{2.979158in}{1.757890in}}%
\pgfpathlineto{\pgfqpoint{2.965407in}{1.768384in}}%
\pgfpathlineto{\pgfqpoint{2.956898in}{1.763993in}}%
\pgfpathlineto{\pgfqpoint{2.948378in}{1.759800in}}%
\pgfpathlineto{\pgfqpoint{2.939845in}{1.755810in}}%
\pgfpathlineto{\pgfqpoint{2.931299in}{1.752029in}}%
\pgfpathclose%
\pgfusepath{fill}%
\end{pgfscope}%
\begin{pgfscope}%
\pgfpathrectangle{\pgfqpoint{1.150000in}{0.150000in}}{\pgfqpoint{5.700000in}{5.700000in}}%
\pgfusepath{clip}%
\pgfsetbuttcap%
\pgfsetroundjoin%
\definecolor{currentfill}{rgb}{0.227802,0.326594,0.546532}%
\pgfsetfillcolor{currentfill}%
\pgfsetfillopacity{0.800000}%
\pgfsetlinewidth{0.000000pt}%
\definecolor{currentstroke}{rgb}{0.000000,0.000000,0.000000}%
\pgfsetstrokecolor{currentstroke}%
\pgfsetdash{}{0pt}%
\pgfpathmoveto{\pgfqpoint{4.444534in}{2.324388in}}%
\pgfpathlineto{\pgfqpoint{4.458588in}{2.332446in}}%
\pgfpathlineto{\pgfqpoint{4.472656in}{2.340689in}}%
\pgfpathlineto{\pgfqpoint{4.486738in}{2.349118in}}%
\pgfpathlineto{\pgfqpoint{4.500835in}{2.357731in}}%
\pgfpathlineto{\pgfqpoint{4.508723in}{2.368552in}}%
\pgfpathlineto{\pgfqpoint{4.516606in}{2.379268in}}%
\pgfpathlineto{\pgfqpoint{4.524483in}{2.389878in}}%
\pgfpathlineto{\pgfqpoint{4.532354in}{2.400383in}}%
\pgfpathlineto{\pgfqpoint{4.518260in}{2.391749in}}%
\pgfpathlineto{\pgfqpoint{4.504182in}{2.383299in}}%
\pgfpathlineto{\pgfqpoint{4.490117in}{2.375035in}}%
\pgfpathlineto{\pgfqpoint{4.476067in}{2.366956in}}%
\pgfpathlineto{\pgfqpoint{4.468193in}{2.356459in}}%
\pgfpathlineto{\pgfqpoint{4.460312in}{2.345866in}}%
\pgfpathlineto{\pgfqpoint{4.452426in}{2.335176in}}%
\pgfpathlineto{\pgfqpoint{4.444534in}{2.324388in}}%
\pgfpathclose%
\pgfusepath{fill}%
\end{pgfscope}%
\begin{pgfscope}%
\pgfpathrectangle{\pgfqpoint{1.150000in}{0.150000in}}{\pgfqpoint{5.700000in}{5.700000in}}%
\pgfusepath{clip}%
\pgfsetbuttcap%
\pgfsetroundjoin%
\definecolor{currentfill}{rgb}{0.122606,0.585371,0.546557}%
\pgfsetfillcolor{currentfill}%
\pgfsetfillopacity{0.800000}%
\pgfsetlinewidth{0.000000pt}%
\definecolor{currentstroke}{rgb}{0.000000,0.000000,0.000000}%
\pgfsetstrokecolor{currentstroke}%
\pgfsetdash{}{0pt}%
\pgfpathmoveto{\pgfqpoint{5.329003in}{3.088198in}}%
\pgfpathlineto{\pgfqpoint{5.343565in}{3.100873in}}%
\pgfpathlineto{\pgfqpoint{5.358147in}{3.113729in}}%
\pgfpathlineto{\pgfqpoint{5.372749in}{3.126768in}}%
\pgfpathlineto{\pgfqpoint{5.387371in}{3.139989in}}%
\pgfpathlineto{\pgfqpoint{5.394819in}{3.144138in}}%
\pgfpathlineto{\pgfqpoint{5.402258in}{3.148216in}}%
\pgfpathlineto{\pgfqpoint{5.409690in}{3.152225in}}%
\pgfpathlineto{\pgfqpoint{5.417113in}{3.156171in}}%
\pgfpathlineto{\pgfqpoint{5.402510in}{3.143341in}}%
\pgfpathlineto{\pgfqpoint{5.387927in}{3.130692in}}%
\pgfpathlineto{\pgfqpoint{5.373364in}{3.118224in}}%
\pgfpathlineto{\pgfqpoint{5.358820in}{3.105938in}}%
\pgfpathlineto{\pgfqpoint{5.351378in}{3.101590in}}%
\pgfpathlineto{\pgfqpoint{5.343927in}{3.097188in}}%
\pgfpathlineto{\pgfqpoint{5.336469in}{3.092725in}}%
\pgfpathlineto{\pgfqpoint{5.329003in}{3.088198in}}%
\pgfpathclose%
\pgfusepath{fill}%
\end{pgfscope}%
\begin{pgfscope}%
\pgfpathrectangle{\pgfqpoint{1.150000in}{0.150000in}}{\pgfqpoint{5.700000in}{5.700000in}}%
\pgfusepath{clip}%
\pgfsetbuttcap%
\pgfsetroundjoin%
\definecolor{currentfill}{rgb}{0.220124,0.725509,0.466226}%
\pgfsetfillcolor{currentfill}%
\pgfsetfillopacity{0.800000}%
\pgfsetlinewidth{0.000000pt}%
\definecolor{currentstroke}{rgb}{0.000000,0.000000,0.000000}%
\pgfsetstrokecolor{currentstroke}%
\pgfsetdash{}{0pt}%
\pgfpathmoveto{\pgfqpoint{5.974404in}{3.535408in}}%
\pgfpathlineto{\pgfqpoint{5.989341in}{3.549034in}}%
\pgfpathlineto{\pgfqpoint{6.004300in}{3.562839in}}%
\pgfpathlineto{\pgfqpoint{6.019281in}{3.576823in}}%
\pgfpathlineto{\pgfqpoint{6.034285in}{3.590987in}}%
\pgfpathlineto{\pgfqpoint{6.041301in}{3.590895in}}%
\pgfpathlineto{\pgfqpoint{6.048311in}{3.590859in}}%
\pgfpathlineto{\pgfqpoint{6.055316in}{3.590885in}}%
\pgfpathlineto{\pgfqpoint{6.062317in}{3.590981in}}%
\pgfpathlineto{\pgfqpoint{6.047351in}{3.577488in}}%
\pgfpathlineto{\pgfqpoint{6.032408in}{3.564172in}}%
\pgfpathlineto{\pgfqpoint{6.017487in}{3.551033in}}%
\pgfpathlineto{\pgfqpoint{6.002588in}{3.538072in}}%
\pgfpathlineto{\pgfqpoint{5.995549in}{3.537297in}}%
\pgfpathlineto{\pgfqpoint{5.988505in}{3.536600in}}%
\pgfpathlineto{\pgfqpoint{5.981457in}{3.535972in}}%
\pgfpathlineto{\pgfqpoint{5.974404in}{3.535408in}}%
\pgfpathclose%
\pgfusepath{fill}%
\end{pgfscope}%
\begin{pgfscope}%
\pgfpathrectangle{\pgfqpoint{1.150000in}{0.150000in}}{\pgfqpoint{5.700000in}{5.700000in}}%
\pgfusepath{clip}%
\pgfsetbuttcap%
\pgfsetroundjoin%
\definecolor{currentfill}{rgb}{0.282290,0.145912,0.461510}%
\pgfsetfillcolor{currentfill}%
\pgfsetfillopacity{0.800000}%
\pgfsetlinewidth{0.000000pt}%
\definecolor{currentstroke}{rgb}{0.000000,0.000000,0.000000}%
\pgfsetstrokecolor{currentstroke}%
\pgfsetdash{}{0pt}%
\pgfpathmoveto{\pgfqpoint{3.942546in}{1.883803in}}%
\pgfpathlineto{\pgfqpoint{3.956384in}{1.887067in}}%
\pgfpathlineto{\pgfqpoint{3.970233in}{1.890520in}}%
\pgfpathlineto{\pgfqpoint{3.984092in}{1.894161in}}%
\pgfpathlineto{\pgfqpoint{3.997960in}{1.897990in}}%
\pgfpathlineto{\pgfqpoint{4.006009in}{1.910200in}}%
\pgfpathlineto{\pgfqpoint{4.014053in}{1.922372in}}%
\pgfpathlineto{\pgfqpoint{4.022092in}{1.934506in}}%
\pgfpathlineto{\pgfqpoint{4.030127in}{1.946599in}}%
\pgfpathlineto{\pgfqpoint{4.016263in}{1.942522in}}%
\pgfpathlineto{\pgfqpoint{4.002409in}{1.938634in}}%
\pgfpathlineto{\pgfqpoint{3.988566in}{1.934934in}}%
\pgfpathlineto{\pgfqpoint{3.974732in}{1.931423in}}%
\pgfpathlineto{\pgfqpoint{3.966693in}{1.919565in}}%
\pgfpathlineto{\pgfqpoint{3.958649in}{1.907674in}}%
\pgfpathlineto{\pgfqpoint{3.950600in}{1.895753in}}%
\pgfpathlineto{\pgfqpoint{3.942546in}{1.883803in}}%
\pgfpathclose%
\pgfusepath{fill}%
\end{pgfscope}%
\begin{pgfscope}%
\pgfpathrectangle{\pgfqpoint{1.150000in}{0.150000in}}{\pgfqpoint{5.700000in}{5.700000in}}%
\pgfusepath{clip}%
\pgfsetbuttcap%
\pgfsetroundjoin%
\definecolor{currentfill}{rgb}{0.258965,0.251537,0.524736}%
\pgfsetfillcolor{currentfill}%
\pgfsetfillopacity{0.800000}%
\pgfsetlinewidth{0.000000pt}%
\definecolor{currentstroke}{rgb}{0.000000,0.000000,0.000000}%
\pgfsetstrokecolor{currentstroke}%
\pgfsetdash{}{0pt}%
\pgfpathmoveto{\pgfqpoint{4.237382in}{2.131680in}}%
\pgfpathlineto{\pgfqpoint{4.251339in}{2.137992in}}%
\pgfpathlineto{\pgfqpoint{4.265308in}{2.144491in}}%
\pgfpathlineto{\pgfqpoint{4.279290in}{2.151175in}}%
\pgfpathlineto{\pgfqpoint{4.293284in}{2.158046in}}%
\pgfpathlineto{\pgfqpoint{4.301245in}{2.169855in}}%
\pgfpathlineto{\pgfqpoint{4.309200in}{2.181579in}}%
\pgfpathlineto{\pgfqpoint{4.317150in}{2.193217in}}%
\pgfpathlineto{\pgfqpoint{4.325094in}{2.204769in}}%
\pgfpathlineto{\pgfqpoint{4.311103in}{2.197779in}}%
\pgfpathlineto{\pgfqpoint{4.297124in}{2.190975in}}%
\pgfpathlineto{\pgfqpoint{4.283158in}{2.184358in}}%
\pgfpathlineto{\pgfqpoint{4.269204in}{2.177926in}}%
\pgfpathlineto{\pgfqpoint{4.261257in}{2.166481in}}%
\pgfpathlineto{\pgfqpoint{4.253304in}{2.154958in}}%
\pgfpathlineto{\pgfqpoint{4.245345in}{2.143357in}}%
\pgfpathlineto{\pgfqpoint{4.237382in}{2.131680in}}%
\pgfpathclose%
\pgfusepath{fill}%
\end{pgfscope}%
\begin{pgfscope}%
\pgfpathrectangle{\pgfqpoint{1.150000in}{0.150000in}}{\pgfqpoint{5.700000in}{5.700000in}}%
\pgfusepath{clip}%
\pgfsetbuttcap%
\pgfsetroundjoin%
\definecolor{currentfill}{rgb}{0.269944,0.014625,0.341379}%
\pgfsetfillcolor{currentfill}%
\pgfsetfillopacity{0.800000}%
\pgfsetlinewidth{0.000000pt}%
\definecolor{currentstroke}{rgb}{0.000000,0.000000,0.000000}%
\pgfsetstrokecolor{currentstroke}%
\pgfsetdash{}{0pt}%
\pgfpathmoveto{\pgfqpoint{3.416667in}{1.626175in}}%
\pgfpathlineto{\pgfqpoint{3.430401in}{1.622641in}}%
\pgfpathlineto{\pgfqpoint{3.444139in}{1.619308in}}%
\pgfpathlineto{\pgfqpoint{3.457882in}{1.616175in}}%
\pgfpathlineto{\pgfqpoint{3.471629in}{1.613240in}}%
\pgfpathlineto{\pgfqpoint{3.479866in}{1.622781in}}%
\pgfpathlineto{\pgfqpoint{3.488096in}{1.632409in}}%
\pgfpathlineto{\pgfqpoint{3.496320in}{1.642120in}}%
\pgfpathlineto{\pgfqpoint{3.504537in}{1.651910in}}%
\pgfpathlineto{\pgfqpoint{3.490806in}{1.654409in}}%
\pgfpathlineto{\pgfqpoint{3.477079in}{1.657108in}}%
\pgfpathlineto{\pgfqpoint{3.463357in}{1.660006in}}%
\pgfpathlineto{\pgfqpoint{3.449640in}{1.663105in}}%
\pgfpathlineto{\pgfqpoint{3.441407in}{1.653739in}}%
\pgfpathlineto{\pgfqpoint{3.433168in}{1.644458in}}%
\pgfpathlineto{\pgfqpoint{3.424921in}{1.635269in}}%
\pgfpathlineto{\pgfqpoint{3.416667in}{1.626175in}}%
\pgfpathclose%
\pgfusepath{fill}%
\end{pgfscope}%
\begin{pgfscope}%
\pgfpathrectangle{\pgfqpoint{1.150000in}{0.150000in}}{\pgfqpoint{5.700000in}{5.700000in}}%
\pgfusepath{clip}%
\pgfsetbuttcap%
\pgfsetroundjoin%
\definecolor{currentfill}{rgb}{0.183898,0.422383,0.556944}%
\pgfsetfillcolor{currentfill}%
\pgfsetfillopacity{0.800000}%
\pgfsetlinewidth{0.000000pt}%
\definecolor{currentstroke}{rgb}{0.000000,0.000000,0.000000}%
\pgfsetstrokecolor{currentstroke}%
\pgfsetdash{}{0pt}%
\pgfpathmoveto{\pgfqpoint{4.739503in}{2.593093in}}%
\pgfpathlineto{\pgfqpoint{4.753722in}{2.603249in}}%
\pgfpathlineto{\pgfqpoint{4.767957in}{2.613589in}}%
\pgfpathlineto{\pgfqpoint{4.782208in}{2.624113in}}%
\pgfpathlineto{\pgfqpoint{4.796476in}{2.634822in}}%
\pgfpathlineto{\pgfqpoint{4.804248in}{2.643726in}}%
\pgfpathlineto{\pgfqpoint{4.812013in}{2.652513in}}%
\pgfpathlineto{\pgfqpoint{4.819770in}{2.661185in}}%
\pgfpathlineto{\pgfqpoint{4.827521in}{2.669743in}}%
\pgfpathlineto{\pgfqpoint{4.813259in}{2.659149in}}%
\pgfpathlineto{\pgfqpoint{4.799014in}{2.648739in}}%
\pgfpathlineto{\pgfqpoint{4.784786in}{2.638512in}}%
\pgfpathlineto{\pgfqpoint{4.770574in}{2.628469in}}%
\pgfpathlineto{\pgfqpoint{4.762816in}{2.619785in}}%
\pgfpathlineto{\pgfqpoint{4.755052in}{2.610995in}}%
\pgfpathlineto{\pgfqpoint{4.747281in}{2.602098in}}%
\pgfpathlineto{\pgfqpoint{4.739503in}{2.593093in}}%
\pgfpathclose%
\pgfusepath{fill}%
\end{pgfscope}%
\begin{pgfscope}%
\pgfpathrectangle{\pgfqpoint{1.150000in}{0.150000in}}{\pgfqpoint{5.700000in}{5.700000in}}%
\pgfusepath{clip}%
\pgfsetbuttcap%
\pgfsetroundjoin%
\definecolor{currentfill}{rgb}{0.149039,0.508051,0.557250}%
\pgfsetfillcolor{currentfill}%
\pgfsetfillopacity{0.800000}%
\pgfsetlinewidth{0.000000pt}%
\definecolor{currentstroke}{rgb}{0.000000,0.000000,0.000000}%
\pgfsetstrokecolor{currentstroke}%
\pgfsetdash{}{0pt}%
\pgfpathmoveto{\pgfqpoint{5.034469in}{2.851316in}}%
\pgfpathlineto{\pgfqpoint{5.048861in}{2.863015in}}%
\pgfpathlineto{\pgfqpoint{5.063272in}{2.874896in}}%
\pgfpathlineto{\pgfqpoint{5.077700in}{2.886961in}}%
\pgfpathlineto{\pgfqpoint{5.092148in}{2.899208in}}%
\pgfpathlineto{\pgfqpoint{5.099772in}{2.905771in}}%
\pgfpathlineto{\pgfqpoint{5.107389in}{2.912228in}}%
\pgfpathlineto{\pgfqpoint{5.114997in}{2.918582in}}%
\pgfpathlineto{\pgfqpoint{5.122598in}{2.924836in}}%
\pgfpathlineto{\pgfqpoint{5.108163in}{2.912840in}}%
\pgfpathlineto{\pgfqpoint{5.093746in}{2.901027in}}%
\pgfpathlineto{\pgfqpoint{5.079347in}{2.889397in}}%
\pgfpathlineto{\pgfqpoint{5.064966in}{2.877948in}}%
\pgfpathlineto{\pgfqpoint{5.057353in}{2.871431in}}%
\pgfpathlineto{\pgfqpoint{5.049733in}{2.864822in}}%
\pgfpathlineto{\pgfqpoint{5.042105in}{2.858118in}}%
\pgfpathlineto{\pgfqpoint{5.034469in}{2.851316in}}%
\pgfpathclose%
\pgfusepath{fill}%
\end{pgfscope}%
\begin{pgfscope}%
\pgfpathrectangle{\pgfqpoint{1.150000in}{0.150000in}}{\pgfqpoint{5.700000in}{5.700000in}}%
\pgfusepath{clip}%
\pgfsetbuttcap%
\pgfsetroundjoin%
\definecolor{currentfill}{rgb}{0.197636,0.391528,0.554969}%
\pgfsetfillcolor{currentfill}%
\pgfsetfillopacity{0.800000}%
\pgfsetlinewidth{0.000000pt}%
\definecolor{currentstroke}{rgb}{0.000000,0.000000,0.000000}%
\pgfsetstrokecolor{currentstroke}%
\pgfsetdash{}{0pt}%
\pgfpathmoveto{\pgfqpoint{2.262276in}{2.585701in}}%
\pgfpathlineto{\pgfqpoint{2.276464in}{2.561436in}}%
\pgfpathlineto{\pgfqpoint{2.290638in}{2.537506in}}%
\pgfpathlineto{\pgfqpoint{2.304796in}{2.513908in}}%
\pgfpathlineto{\pgfqpoint{2.318939in}{2.490637in}}%
\pgfpathlineto{\pgfqpoint{2.327968in}{2.488181in}}%
\pgfpathlineto{\pgfqpoint{2.336976in}{2.486031in}}%
\pgfpathlineto{\pgfqpoint{2.345964in}{2.484182in}}%
\pgfpathlineto{\pgfqpoint{2.354933in}{2.482627in}}%
\pgfpathlineto{\pgfqpoint{2.340844in}{2.505320in}}%
\pgfpathlineto{\pgfqpoint{2.326740in}{2.528341in}}%
\pgfpathlineto{\pgfqpoint{2.312622in}{2.551691in}}%
\pgfpathlineto{\pgfqpoint{2.298489in}{2.575374in}}%
\pgfpathlineto{\pgfqpoint{2.289467in}{2.577493in}}%
\pgfpathlineto{\pgfqpoint{2.280424in}{2.579917in}}%
\pgfpathlineto{\pgfqpoint{2.271360in}{2.582651in}}%
\pgfpathlineto{\pgfqpoint{2.262276in}{2.585701in}}%
\pgfpathclose%
\pgfusepath{fill}%
\end{pgfscope}%
\begin{pgfscope}%
\pgfpathrectangle{\pgfqpoint{1.150000in}{0.150000in}}{\pgfqpoint{5.700000in}{5.700000in}}%
\pgfusepath{clip}%
\pgfsetbuttcap%
\pgfsetroundjoin%
\definecolor{currentfill}{rgb}{0.276022,0.044167,0.370164}%
\pgfsetfillcolor{currentfill}%
\pgfsetfillopacity{0.800000}%
\pgfsetlinewidth{0.000000pt}%
\definecolor{currentstroke}{rgb}{0.000000,0.000000,0.000000}%
\pgfsetstrokecolor{currentstroke}%
\pgfsetdash{}{0pt}%
\pgfpathmoveto{\pgfqpoint{2.986420in}{1.709275in}}%
\pgfpathlineto{\pgfqpoint{3.000196in}{1.699154in}}%
\pgfpathlineto{\pgfqpoint{3.013970in}{1.689258in}}%
\pgfpathlineto{\pgfqpoint{3.027742in}{1.679585in}}%
\pgfpathlineto{\pgfqpoint{3.041514in}{1.670134in}}%
\pgfpathlineto{\pgfqpoint{3.049995in}{1.675012in}}%
\pgfpathlineto{\pgfqpoint{3.058465in}{1.680083in}}%
\pgfpathlineto{\pgfqpoint{3.066924in}{1.685340in}}%
\pgfpathlineto{\pgfqpoint{3.075371in}{1.690779in}}%
\pgfpathlineto{\pgfqpoint{3.061630in}{1.699695in}}%
\pgfpathlineto{\pgfqpoint{3.047887in}{1.708833in}}%
\pgfpathlineto{\pgfqpoint{3.034144in}{1.718193in}}%
\pgfpathlineto{\pgfqpoint{3.020399in}{1.727778in}}%
\pgfpathlineto{\pgfqpoint{3.011922in}{1.722862in}}%
\pgfpathlineto{\pgfqpoint{3.003433in}{1.718136in}}%
\pgfpathlineto{\pgfqpoint{2.994933in}{1.713605in}}%
\pgfpathlineto{\pgfqpoint{2.986420in}{1.709275in}}%
\pgfpathclose%
\pgfusepath{fill}%
\end{pgfscope}%
\begin{pgfscope}%
\pgfpathrectangle{\pgfqpoint{1.150000in}{0.150000in}}{\pgfqpoint{5.700000in}{5.700000in}}%
\pgfusepath{clip}%
\pgfsetbuttcap%
\pgfsetroundjoin%
\definecolor{currentfill}{rgb}{0.119512,0.607464,0.540218}%
\pgfsetfillcolor{currentfill}%
\pgfsetfillopacity{0.800000}%
\pgfsetlinewidth{0.000000pt}%
\definecolor{currentstroke}{rgb}{0.000000,0.000000,0.000000}%
\pgfsetstrokecolor{currentstroke}%
\pgfsetdash{}{0pt}%
\pgfpathmoveto{\pgfqpoint{5.417113in}{3.156171in}}%
\pgfpathlineto{\pgfqpoint{5.431736in}{3.169182in}}%
\pgfpathlineto{\pgfqpoint{5.446379in}{3.182375in}}%
\pgfpathlineto{\pgfqpoint{5.461042in}{3.195750in}}%
\pgfpathlineto{\pgfqpoint{5.475726in}{3.209306in}}%
\pgfpathlineto{\pgfqpoint{5.483121in}{3.212780in}}%
\pgfpathlineto{\pgfqpoint{5.490507in}{3.216191in}}%
\pgfpathlineto{\pgfqpoint{5.497886in}{3.219545in}}%
\pgfpathlineto{\pgfqpoint{5.505256in}{3.222846in}}%
\pgfpathlineto{\pgfqpoint{5.490593in}{3.209716in}}%
\pgfpathlineto{\pgfqpoint{5.475951in}{3.196766in}}%
\pgfpathlineto{\pgfqpoint{5.461329in}{3.183998in}}%
\pgfpathlineto{\pgfqpoint{5.446727in}{3.171411in}}%
\pgfpathlineto{\pgfqpoint{5.439335in}{3.167673in}}%
\pgfpathlineto{\pgfqpoint{5.431935in}{3.163890in}}%
\pgfpathlineto{\pgfqpoint{5.424528in}{3.160057in}}%
\pgfpathlineto{\pgfqpoint{5.417113in}{3.156171in}}%
\pgfpathclose%
\pgfusepath{fill}%
\end{pgfscope}%
\begin{pgfscope}%
\pgfpathrectangle{\pgfqpoint{1.150000in}{0.150000in}}{\pgfqpoint{5.700000in}{5.700000in}}%
\pgfusepath{clip}%
\pgfsetbuttcap%
\pgfsetroundjoin%
\definecolor{currentfill}{rgb}{0.278826,0.175490,0.483397}%
\pgfsetfillcolor{currentfill}%
\pgfsetfillopacity{0.800000}%
\pgfsetlinewidth{0.000000pt}%
\definecolor{currentstroke}{rgb}{0.000000,0.000000,0.000000}%
\pgfsetstrokecolor{currentstroke}%
\pgfsetdash{}{0pt}%
\pgfpathmoveto{\pgfqpoint{4.030127in}{1.946599in}}%
\pgfpathlineto{\pgfqpoint{4.044002in}{1.950864in}}%
\pgfpathlineto{\pgfqpoint{4.057887in}{1.955317in}}%
\pgfpathlineto{\pgfqpoint{4.071783in}{1.959957in}}%
\pgfpathlineto{\pgfqpoint{4.085691in}{1.964784in}}%
\pgfpathlineto{\pgfqpoint{4.093716in}{1.977063in}}%
\pgfpathlineto{\pgfqpoint{4.101737in}{1.989288in}}%
\pgfpathlineto{\pgfqpoint{4.109753in}{2.001460in}}%
\pgfpathlineto{\pgfqpoint{4.117765in}{2.013575in}}%
\pgfpathlineto{\pgfqpoint{4.103861in}{2.008532in}}%
\pgfpathlineto{\pgfqpoint{4.089969in}{2.003676in}}%
\pgfpathlineto{\pgfqpoint{4.076087in}{1.999007in}}%
\pgfpathlineto{\pgfqpoint{4.062217in}{1.994527in}}%
\pgfpathlineto{\pgfqpoint{4.054202in}{1.982616in}}%
\pgfpathlineto{\pgfqpoint{4.046181in}{1.970656in}}%
\pgfpathlineto{\pgfqpoint{4.038157in}{1.958650in}}%
\pgfpathlineto{\pgfqpoint{4.030127in}{1.946599in}}%
\pgfpathclose%
\pgfusepath{fill}%
\end{pgfscope}%
\begin{pgfscope}%
\pgfpathrectangle{\pgfqpoint{1.150000in}{0.150000in}}{\pgfqpoint{5.700000in}{5.700000in}}%
\pgfusepath{clip}%
\pgfsetbuttcap%
\pgfsetroundjoin%
\definecolor{currentfill}{rgb}{0.212395,0.359683,0.551710}%
\pgfsetfillcolor{currentfill}%
\pgfsetfillopacity{0.800000}%
\pgfsetlinewidth{0.000000pt}%
\definecolor{currentstroke}{rgb}{0.000000,0.000000,0.000000}%
\pgfsetstrokecolor{currentstroke}%
\pgfsetdash{}{0pt}%
\pgfpathmoveto{\pgfqpoint{4.532354in}{2.400383in}}%
\pgfpathlineto{\pgfqpoint{4.546462in}{2.409202in}}%
\pgfpathlineto{\pgfqpoint{4.560585in}{2.418206in}}%
\pgfpathlineto{\pgfqpoint{4.574723in}{2.427394in}}%
\pgfpathlineto{\pgfqpoint{4.588876in}{2.436767in}}%
\pgfpathlineto{\pgfqpoint{4.596737in}{2.447168in}}%
\pgfpathlineto{\pgfqpoint{4.604592in}{2.457455in}}%
\pgfpathlineto{\pgfqpoint{4.612441in}{2.467631in}}%
\pgfpathlineto{\pgfqpoint{4.620283in}{2.477696in}}%
\pgfpathlineto{\pgfqpoint{4.606134in}{2.468335in}}%
\pgfpathlineto{\pgfqpoint{4.591999in}{2.459159in}}%
\pgfpathlineto{\pgfqpoint{4.577880in}{2.450168in}}%
\pgfpathlineto{\pgfqpoint{4.563776in}{2.441361in}}%
\pgfpathlineto{\pgfqpoint{4.555930in}{2.431272in}}%
\pgfpathlineto{\pgfqpoint{4.548077in}{2.421079in}}%
\pgfpathlineto{\pgfqpoint{4.540218in}{2.410783in}}%
\pgfpathlineto{\pgfqpoint{4.532354in}{2.400383in}}%
\pgfpathclose%
\pgfusepath{fill}%
\end{pgfscope}%
\begin{pgfscope}%
\pgfpathrectangle{\pgfqpoint{1.150000in}{0.150000in}}{\pgfqpoint{5.700000in}{5.700000in}}%
\pgfusepath{clip}%
\pgfsetbuttcap%
\pgfsetroundjoin%
\definecolor{currentfill}{rgb}{0.268510,0.009605,0.335427}%
\pgfsetfillcolor{currentfill}%
\pgfsetfillopacity{0.800000}%
\pgfsetlinewidth{0.000000pt}%
\definecolor{currentstroke}{rgb}{0.000000,0.000000,0.000000}%
\pgfsetstrokecolor{currentstroke}%
\pgfsetdash{}{0pt}%
\pgfpathmoveto{\pgfqpoint{3.185305in}{1.627296in}}%
\pgfpathlineto{\pgfqpoint{3.199050in}{1.620325in}}%
\pgfpathlineto{\pgfqpoint{3.212795in}{1.613565in}}%
\pgfpathlineto{\pgfqpoint{3.226543in}{1.607015in}}%
\pgfpathlineto{\pgfqpoint{3.240291in}{1.600674in}}%
\pgfpathlineto{\pgfqpoint{3.248651in}{1.607828in}}%
\pgfpathlineto{\pgfqpoint{3.257002in}{1.615128in}}%
\pgfpathlineto{\pgfqpoint{3.265343in}{1.622571in}}%
\pgfpathlineto{\pgfqpoint{3.273676in}{1.630152in}}%
\pgfpathlineto{\pgfqpoint{3.259950in}{1.635994in}}%
\pgfpathlineto{\pgfqpoint{3.246227in}{1.642044in}}%
\pgfpathlineto{\pgfqpoint{3.232505in}{1.648304in}}%
\pgfpathlineto{\pgfqpoint{3.218784in}{1.654775in}}%
\pgfpathlineto{\pgfqpoint{3.210428in}{1.647682in}}%
\pgfpathlineto{\pgfqpoint{3.202063in}{1.640735in}}%
\pgfpathlineto{\pgfqpoint{3.193689in}{1.633938in}}%
\pgfpathlineto{\pgfqpoint{3.185305in}{1.627296in}}%
\pgfpathclose%
\pgfusepath{fill}%
\end{pgfscope}%
\begin{pgfscope}%
\pgfpathrectangle{\pgfqpoint{1.150000in}{0.150000in}}{\pgfqpoint{5.700000in}{5.700000in}}%
\pgfusepath{clip}%
\pgfsetbuttcap%
\pgfsetroundjoin%
\definecolor{currentfill}{rgb}{0.244972,0.287675,0.537260}%
\pgfsetfillcolor{currentfill}%
\pgfsetfillopacity{0.800000}%
\pgfsetlinewidth{0.000000pt}%
\definecolor{currentstroke}{rgb}{0.000000,0.000000,0.000000}%
\pgfsetstrokecolor{currentstroke}%
\pgfsetdash{}{0pt}%
\pgfpathmoveto{\pgfqpoint{4.325094in}{2.204769in}}%
\pgfpathlineto{\pgfqpoint{4.339100in}{2.211945in}}%
\pgfpathlineto{\pgfqpoint{4.353118in}{2.219306in}}%
\pgfpathlineto{\pgfqpoint{4.367150in}{2.226853in}}%
\pgfpathlineto{\pgfqpoint{4.381195in}{2.234585in}}%
\pgfpathlineto{\pgfqpoint{4.389132in}{2.246149in}}%
\pgfpathlineto{\pgfqpoint{4.397063in}{2.257616in}}%
\pgfpathlineto{\pgfqpoint{4.404989in}{2.268988in}}%
\pgfpathlineto{\pgfqpoint{4.412909in}{2.280262in}}%
\pgfpathlineto{\pgfqpoint{4.398866in}{2.272443in}}%
\pgfpathlineto{\pgfqpoint{4.384837in}{2.264810in}}%
\pgfpathlineto{\pgfqpoint{4.370822in}{2.257362in}}%
\pgfpathlineto{\pgfqpoint{4.356820in}{2.250099in}}%
\pgfpathlineto{\pgfqpoint{4.348897in}{2.238899in}}%
\pgfpathlineto{\pgfqpoint{4.340968in}{2.227611in}}%
\pgfpathlineto{\pgfqpoint{4.333034in}{2.216234in}}%
\pgfpathlineto{\pgfqpoint{4.325094in}{2.204769in}}%
\pgfpathclose%
\pgfusepath{fill}%
\end{pgfscope}%
\begin{pgfscope}%
\pgfpathrectangle{\pgfqpoint{1.150000in}{0.150000in}}{\pgfqpoint{5.700000in}{5.700000in}}%
\pgfusepath{clip}%
\pgfsetbuttcap%
\pgfsetroundjoin%
\definecolor{currentfill}{rgb}{0.268510,0.009605,0.335427}%
\pgfsetfillcolor{currentfill}%
\pgfsetfillopacity{0.800000}%
\pgfsetlinewidth{0.000000pt}%
\definecolor{currentstroke}{rgb}{0.000000,0.000000,0.000000}%
\pgfsetstrokecolor{currentstroke}%
\pgfsetdash{}{0pt}%
\pgfpathmoveto{\pgfqpoint{3.328603in}{1.608857in}}%
\pgfpathlineto{\pgfqpoint{3.342342in}{1.604046in}}%
\pgfpathlineto{\pgfqpoint{3.356084in}{1.599439in}}%
\pgfpathlineto{\pgfqpoint{3.369829in}{1.595035in}}%
\pgfpathlineto{\pgfqpoint{3.383577in}{1.590834in}}%
\pgfpathlineto{\pgfqpoint{3.391861in}{1.599505in}}%
\pgfpathlineto{\pgfqpoint{3.400137in}{1.608288in}}%
\pgfpathlineto{\pgfqpoint{3.408406in}{1.617180in}}%
\pgfpathlineto{\pgfqpoint{3.416667in}{1.626175in}}%
\pgfpathlineto{\pgfqpoint{3.402937in}{1.629910in}}%
\pgfpathlineto{\pgfqpoint{3.389211in}{1.633847in}}%
\pgfpathlineto{\pgfqpoint{3.375488in}{1.637987in}}%
\pgfpathlineto{\pgfqpoint{3.361769in}{1.642331in}}%
\pgfpathlineto{\pgfqpoint{3.353489in}{1.633791in}}%
\pgfpathlineto{\pgfqpoint{3.345202in}{1.625362in}}%
\pgfpathlineto{\pgfqpoint{3.336907in}{1.617049in}}%
\pgfpathlineto{\pgfqpoint{3.328603in}{1.608857in}}%
\pgfpathclose%
\pgfusepath{fill}%
\end{pgfscope}%
\begin{pgfscope}%
\pgfpathrectangle{\pgfqpoint{1.150000in}{0.150000in}}{\pgfqpoint{5.700000in}{5.700000in}}%
\pgfusepath{clip}%
\pgfsetbuttcap%
\pgfsetroundjoin%
\definecolor{currentfill}{rgb}{0.277134,0.185228,0.489898}%
\pgfsetfillcolor{currentfill}%
\pgfsetfillopacity{0.800000}%
\pgfsetlinewidth{0.000000pt}%
\definecolor{currentstroke}{rgb}{0.000000,0.000000,0.000000}%
\pgfsetstrokecolor{currentstroke}%
\pgfsetdash{}{0pt}%
\pgfpathmoveto{\pgfqpoint{2.619691in}{2.021862in}}%
\pgfpathlineto{\pgfqpoint{2.633613in}{2.005325in}}%
\pgfpathlineto{\pgfqpoint{2.647527in}{1.989051in}}%
\pgfpathlineto{\pgfqpoint{2.661434in}{1.973039in}}%
\pgfpathlineto{\pgfqpoint{2.675334in}{1.957286in}}%
\pgfpathlineto{\pgfqpoint{2.684092in}{1.957819in}}%
\pgfpathlineto{\pgfqpoint{2.692833in}{1.958617in}}%
\pgfpathlineto{\pgfqpoint{2.701558in}{1.959675in}}%
\pgfpathlineto{\pgfqpoint{2.710268in}{1.960987in}}%
\pgfpathlineto{\pgfqpoint{2.696411in}{1.976157in}}%
\pgfpathlineto{\pgfqpoint{2.682547in}{1.991586in}}%
\pgfpathlineto{\pgfqpoint{2.668677in}{2.007275in}}%
\pgfpathlineto{\pgfqpoint{2.654800in}{2.023227in}}%
\pgfpathlineto{\pgfqpoint{2.646047in}{2.022486in}}%
\pgfpathlineto{\pgfqpoint{2.637279in}{2.022007in}}%
\pgfpathlineto{\pgfqpoint{2.628494in}{2.021797in}}%
\pgfpathlineto{\pgfqpoint{2.619691in}{2.021862in}}%
\pgfpathclose%
\pgfusepath{fill}%
\end{pgfscope}%
\begin{pgfscope}%
\pgfpathrectangle{\pgfqpoint{1.150000in}{0.150000in}}{\pgfqpoint{5.700000in}{5.700000in}}%
\pgfusepath{clip}%
\pgfsetbuttcap%
\pgfsetroundjoin%
\definecolor{currentfill}{rgb}{0.271828,0.209303,0.504434}%
\pgfsetfillcolor{currentfill}%
\pgfsetfillopacity{0.800000}%
\pgfsetlinewidth{0.000000pt}%
\definecolor{currentstroke}{rgb}{0.000000,0.000000,0.000000}%
\pgfsetstrokecolor{currentstroke}%
\pgfsetdash{}{0pt}%
\pgfpathmoveto{\pgfqpoint{2.563925in}{2.090688in}}%
\pgfpathlineto{\pgfqpoint{2.577879in}{2.073076in}}%
\pgfpathlineto{\pgfqpoint{2.591825in}{2.055735in}}%
\pgfpathlineto{\pgfqpoint{2.605762in}{2.038665in}}%
\pgfpathlineto{\pgfqpoint{2.619691in}{2.021862in}}%
\pgfpathlineto{\pgfqpoint{2.628494in}{2.021797in}}%
\pgfpathlineto{\pgfqpoint{2.637279in}{2.022007in}}%
\pgfpathlineto{\pgfqpoint{2.646047in}{2.022486in}}%
\pgfpathlineto{\pgfqpoint{2.654800in}{2.023227in}}%
\pgfpathlineto{\pgfqpoint{2.640915in}{2.039444in}}%
\pgfpathlineto{\pgfqpoint{2.627023in}{2.055928in}}%
\pgfpathlineto{\pgfqpoint{2.613123in}{2.072680in}}%
\pgfpathlineto{\pgfqpoint{2.599216in}{2.089703in}}%
\pgfpathlineto{\pgfqpoint{2.590419in}{2.089535in}}%
\pgfpathlineto{\pgfqpoint{2.581605in}{2.089640in}}%
\pgfpathlineto{\pgfqpoint{2.572774in}{2.090022in}}%
\pgfpathlineto{\pgfqpoint{2.563925in}{2.090688in}}%
\pgfpathclose%
\pgfusepath{fill}%
\end{pgfscope}%
\begin{pgfscope}%
\pgfpathrectangle{\pgfqpoint{1.150000in}{0.150000in}}{\pgfqpoint{5.700000in}{5.700000in}}%
\pgfusepath{clip}%
\pgfsetbuttcap%
\pgfsetroundjoin%
\definecolor{currentfill}{rgb}{0.121380,0.629492,0.531973}%
\pgfsetfillcolor{currentfill}%
\pgfsetfillopacity{0.800000}%
\pgfsetlinewidth{0.000000pt}%
\definecolor{currentstroke}{rgb}{0.000000,0.000000,0.000000}%
\pgfsetstrokecolor{currentstroke}%
\pgfsetdash{}{0pt}%
\pgfpathmoveto{\pgfqpoint{5.505256in}{3.222846in}}%
\pgfpathlineto{\pgfqpoint{5.519939in}{3.236157in}}%
\pgfpathlineto{\pgfqpoint{5.534643in}{3.249650in}}%
\pgfpathlineto{\pgfqpoint{5.549367in}{3.263324in}}%
\pgfpathlineto{\pgfqpoint{5.564113in}{3.277180in}}%
\pgfpathlineto{\pgfqpoint{5.571452in}{3.279984in}}%
\pgfpathlineto{\pgfqpoint{5.578784in}{3.282737in}}%
\pgfpathlineto{\pgfqpoint{5.586107in}{3.285444in}}%
\pgfpathlineto{\pgfqpoint{5.593423in}{3.288111in}}%
\pgfpathlineto{\pgfqpoint{5.578701in}{3.274717in}}%
\pgfpathlineto{\pgfqpoint{5.564000in}{3.261504in}}%
\pgfpathlineto{\pgfqpoint{5.549319in}{3.248472in}}%
\pgfpathlineto{\pgfqpoint{5.534659in}{3.235619in}}%
\pgfpathlineto{\pgfqpoint{5.527320in}{3.232480in}}%
\pgfpathlineto{\pgfqpoint{5.519973in}{3.229308in}}%
\pgfpathlineto{\pgfqpoint{5.512618in}{3.226099in}}%
\pgfpathlineto{\pgfqpoint{5.505256in}{3.222846in}}%
\pgfpathclose%
\pgfusepath{fill}%
\end{pgfscope}%
\begin{pgfscope}%
\pgfpathrectangle{\pgfqpoint{1.150000in}{0.150000in}}{\pgfqpoint{5.700000in}{5.700000in}}%
\pgfusepath{clip}%
\pgfsetbuttcap%
\pgfsetroundjoin%
\definecolor{currentfill}{rgb}{0.281412,0.155834,0.469201}%
\pgfsetfillcolor{currentfill}%
\pgfsetfillopacity{0.800000}%
\pgfsetlinewidth{0.000000pt}%
\definecolor{currentstroke}{rgb}{0.000000,0.000000,0.000000}%
\pgfsetstrokecolor{currentstroke}%
\pgfsetdash{}{0pt}%
\pgfpathmoveto{\pgfqpoint{2.675334in}{1.957286in}}%
\pgfpathlineto{\pgfqpoint{2.689228in}{1.941791in}}%
\pgfpathlineto{\pgfqpoint{2.703115in}{1.926552in}}%
\pgfpathlineto{\pgfqpoint{2.716995in}{1.911567in}}%
\pgfpathlineto{\pgfqpoint{2.730870in}{1.896835in}}%
\pgfpathlineto{\pgfqpoint{2.739585in}{1.897961in}}%
\pgfpathlineto{\pgfqpoint{2.748284in}{1.899344in}}%
\pgfpathlineto{\pgfqpoint{2.756968in}{1.900978in}}%
\pgfpathlineto{\pgfqpoint{2.765637in}{1.902858in}}%
\pgfpathlineto{\pgfqpoint{2.751803in}{1.917011in}}%
\pgfpathlineto{\pgfqpoint{2.737964in}{1.931416in}}%
\pgfpathlineto{\pgfqpoint{2.724119in}{1.946074in}}%
\pgfpathlineto{\pgfqpoint{2.710268in}{1.960987in}}%
\pgfpathlineto{\pgfqpoint{2.701558in}{1.959675in}}%
\pgfpathlineto{\pgfqpoint{2.692833in}{1.958617in}}%
\pgfpathlineto{\pgfqpoint{2.684092in}{1.957819in}}%
\pgfpathlineto{\pgfqpoint{2.675334in}{1.957286in}}%
\pgfpathclose%
\pgfusepath{fill}%
\end{pgfscope}%
\begin{pgfscope}%
\pgfpathrectangle{\pgfqpoint{1.150000in}{0.150000in}}{\pgfqpoint{5.700000in}{5.700000in}}%
\pgfusepath{clip}%
\pgfsetbuttcap%
\pgfsetroundjoin%
\definecolor{currentfill}{rgb}{0.171176,0.452530,0.557965}%
\pgfsetfillcolor{currentfill}%
\pgfsetfillopacity{0.800000}%
\pgfsetlinewidth{0.000000pt}%
\definecolor{currentstroke}{rgb}{0.000000,0.000000,0.000000}%
\pgfsetstrokecolor{currentstroke}%
\pgfsetdash{}{0pt}%
\pgfpathmoveto{\pgfqpoint{4.827521in}{2.669743in}}%
\pgfpathlineto{\pgfqpoint{4.841799in}{2.680521in}}%
\pgfpathlineto{\pgfqpoint{4.856095in}{2.691483in}}%
\pgfpathlineto{\pgfqpoint{4.870407in}{2.702629in}}%
\pgfpathlineto{\pgfqpoint{4.884737in}{2.713959in}}%
\pgfpathlineto{\pgfqpoint{4.892474in}{2.722270in}}%
\pgfpathlineto{\pgfqpoint{4.900202in}{2.730462in}}%
\pgfpathlineto{\pgfqpoint{4.907924in}{2.738539in}}%
\pgfpathlineto{\pgfqpoint{4.915637in}{2.746501in}}%
\pgfpathlineto{\pgfqpoint{4.901315in}{2.735320in}}%
\pgfpathlineto{\pgfqpoint{4.887010in}{2.724322in}}%
\pgfpathlineto{\pgfqpoint{4.872722in}{2.713508in}}%
\pgfpathlineto{\pgfqpoint{4.858451in}{2.702878in}}%
\pgfpathlineto{\pgfqpoint{4.850729in}{2.694755in}}%
\pgfpathlineto{\pgfqpoint{4.843000in}{2.686527in}}%
\pgfpathlineto{\pgfqpoint{4.835264in}{2.678190in}}%
\pgfpathlineto{\pgfqpoint{4.827521in}{2.669743in}}%
\pgfpathclose%
\pgfusepath{fill}%
\end{pgfscope}%
\begin{pgfscope}%
\pgfpathrectangle{\pgfqpoint{1.150000in}{0.150000in}}{\pgfqpoint{5.700000in}{5.700000in}}%
\pgfusepath{clip}%
\pgfsetbuttcap%
\pgfsetroundjoin%
\definecolor{currentfill}{rgb}{0.137770,0.537492,0.554906}%
\pgfsetfillcolor{currentfill}%
\pgfsetfillopacity{0.800000}%
\pgfsetlinewidth{0.000000pt}%
\definecolor{currentstroke}{rgb}{0.000000,0.000000,0.000000}%
\pgfsetstrokecolor{currentstroke}%
\pgfsetdash{}{0pt}%
\pgfpathmoveto{\pgfqpoint{5.122598in}{2.924836in}}%
\pgfpathlineto{\pgfqpoint{5.137052in}{2.937015in}}%
\pgfpathlineto{\pgfqpoint{5.151525in}{2.949376in}}%
\pgfpathlineto{\pgfqpoint{5.166017in}{2.961921in}}%
\pgfpathlineto{\pgfqpoint{5.180528in}{2.974649in}}%
\pgfpathlineto{\pgfqpoint{5.188107in}{2.980532in}}%
\pgfpathlineto{\pgfqpoint{5.195679in}{2.986314in}}%
\pgfpathlineto{\pgfqpoint{5.203242in}{2.991997in}}%
\pgfpathlineto{\pgfqpoint{5.210797in}{2.997586in}}%
\pgfpathlineto{\pgfqpoint{5.196300in}{2.985145in}}%
\pgfpathlineto{\pgfqpoint{5.181821in}{2.972887in}}%
\pgfpathlineto{\pgfqpoint{5.167362in}{2.960812in}}%
\pgfpathlineto{\pgfqpoint{5.152921in}{2.948919in}}%
\pgfpathlineto{\pgfqpoint{5.145352in}{2.943031in}}%
\pgfpathlineto{\pgfqpoint{5.137775in}{2.937058in}}%
\pgfpathlineto{\pgfqpoint{5.130191in}{2.930994in}}%
\pgfpathlineto{\pgfqpoint{5.122598in}{2.924836in}}%
\pgfpathclose%
\pgfusepath{fill}%
\end{pgfscope}%
\begin{pgfscope}%
\pgfpathrectangle{\pgfqpoint{1.150000in}{0.150000in}}{\pgfqpoint{5.700000in}{5.700000in}}%
\pgfusepath{clip}%
\pgfsetbuttcap%
\pgfsetroundjoin%
\definecolor{currentfill}{rgb}{0.262138,0.242286,0.520837}%
\pgfsetfillcolor{currentfill}%
\pgfsetfillopacity{0.800000}%
\pgfsetlinewidth{0.000000pt}%
\definecolor{currentstroke}{rgb}{0.000000,0.000000,0.000000}%
\pgfsetstrokecolor{currentstroke}%
\pgfsetdash{}{0pt}%
\pgfpathmoveto{\pgfqpoint{2.508018in}{2.163897in}}%
\pgfpathlineto{\pgfqpoint{2.522009in}{2.145176in}}%
\pgfpathlineto{\pgfqpoint{2.535990in}{2.126735in}}%
\pgfpathlineto{\pgfqpoint{2.549962in}{2.108573in}}%
\pgfpathlineto{\pgfqpoint{2.563925in}{2.090688in}}%
\pgfpathlineto{\pgfqpoint{2.572774in}{2.090022in}}%
\pgfpathlineto{\pgfqpoint{2.581605in}{2.089640in}}%
\pgfpathlineto{\pgfqpoint{2.590419in}{2.089535in}}%
\pgfpathlineto{\pgfqpoint{2.599216in}{2.089703in}}%
\pgfpathlineto{\pgfqpoint{2.585300in}{2.106999in}}%
\pgfpathlineto{\pgfqpoint{2.571375in}{2.124570in}}%
\pgfpathlineto{\pgfqpoint{2.557441in}{2.142418in}}%
\pgfpathlineto{\pgfqpoint{2.543499in}{2.160546in}}%
\pgfpathlineto{\pgfqpoint{2.534656in}{2.160957in}}%
\pgfpathlineto{\pgfqpoint{2.525795in}{2.161648in}}%
\pgfpathlineto{\pgfqpoint{2.516916in}{2.162626in}}%
\pgfpathlineto{\pgfqpoint{2.508018in}{2.163897in}}%
\pgfpathclose%
\pgfusepath{fill}%
\end{pgfscope}%
\begin{pgfscope}%
\pgfpathrectangle{\pgfqpoint{1.150000in}{0.150000in}}{\pgfqpoint{5.700000in}{5.700000in}}%
\pgfusepath{clip}%
\pgfsetbuttcap%
\pgfsetroundjoin%
\definecolor{currentfill}{rgb}{0.271828,0.209303,0.504434}%
\pgfsetfillcolor{currentfill}%
\pgfsetfillopacity{0.800000}%
\pgfsetlinewidth{0.000000pt}%
\definecolor{currentstroke}{rgb}{0.000000,0.000000,0.000000}%
\pgfsetstrokecolor{currentstroke}%
\pgfsetdash{}{0pt}%
\pgfpathmoveto{\pgfqpoint{4.117765in}{2.013575in}}%
\pgfpathlineto{\pgfqpoint{4.131680in}{2.018805in}}%
\pgfpathlineto{\pgfqpoint{4.145606in}{2.024222in}}%
\pgfpathlineto{\pgfqpoint{4.159545in}{2.029826in}}%
\pgfpathlineto{\pgfqpoint{4.173495in}{2.035616in}}%
\pgfpathlineto{\pgfqpoint{4.181498in}{2.047870in}}%
\pgfpathlineto{\pgfqpoint{4.189497in}{2.060056in}}%
\pgfpathlineto{\pgfqpoint{4.197490in}{2.072173in}}%
\pgfpathlineto{\pgfqpoint{4.205478in}{2.084220in}}%
\pgfpathlineto{\pgfqpoint{4.191531in}{2.078245in}}%
\pgfpathlineto{\pgfqpoint{4.177596in}{2.072458in}}%
\pgfpathlineto{\pgfqpoint{4.163673in}{2.066857in}}%
\pgfpathlineto{\pgfqpoint{4.149761in}{2.061443in}}%
\pgfpathlineto{\pgfqpoint{4.141769in}{2.049568in}}%
\pgfpathlineto{\pgfqpoint{4.133773in}{2.037630in}}%
\pgfpathlineto{\pgfqpoint{4.125771in}{2.025632in}}%
\pgfpathlineto{\pgfqpoint{4.117765in}{2.013575in}}%
\pgfpathclose%
\pgfusepath{fill}%
\end{pgfscope}%
\begin{pgfscope}%
\pgfpathrectangle{\pgfqpoint{1.150000in}{0.150000in}}{\pgfqpoint{5.700000in}{5.700000in}}%
\pgfusepath{clip}%
\pgfsetbuttcap%
\pgfsetroundjoin%
\definecolor{currentfill}{rgb}{0.277941,0.056324,0.381191}%
\pgfsetfillcolor{currentfill}%
\pgfsetfillopacity{0.800000}%
\pgfsetlinewidth{0.000000pt}%
\definecolor{currentstroke}{rgb}{0.000000,0.000000,0.000000}%
\pgfsetstrokecolor{currentstroke}%
\pgfsetdash{}{0pt}%
\pgfpathmoveto{\pgfqpoint{3.647282in}{1.682107in}}%
\pgfpathlineto{\pgfqpoint{3.661053in}{1.681776in}}%
\pgfpathlineto{\pgfqpoint{3.674831in}{1.681638in}}%
\pgfpathlineto{\pgfqpoint{3.688615in}{1.681693in}}%
\pgfpathlineto{\pgfqpoint{3.702407in}{1.681941in}}%
\pgfpathlineto{\pgfqpoint{3.710557in}{1.693219in}}%
\pgfpathlineto{\pgfqpoint{3.718702in}{1.704530in}}%
\pgfpathlineto{\pgfqpoint{3.726841in}{1.715870in}}%
\pgfpathlineto{\pgfqpoint{3.734974in}{1.727236in}}%
\pgfpathlineto{\pgfqpoint{3.721192in}{1.726616in}}%
\pgfpathlineto{\pgfqpoint{3.707417in}{1.726188in}}%
\pgfpathlineto{\pgfqpoint{3.693650in}{1.725953in}}%
\pgfpathlineto{\pgfqpoint{3.679889in}{1.725912in}}%
\pgfpathlineto{\pgfqpoint{3.671746in}{1.714907in}}%
\pgfpathlineto{\pgfqpoint{3.663597in}{1.703935in}}%
\pgfpathlineto{\pgfqpoint{3.655442in}{1.693001in}}%
\pgfpathlineto{\pgfqpoint{3.647282in}{1.682107in}}%
\pgfpathclose%
\pgfusepath{fill}%
\end{pgfscope}%
\begin{pgfscope}%
\pgfpathrectangle{\pgfqpoint{1.150000in}{0.150000in}}{\pgfqpoint{5.700000in}{5.700000in}}%
\pgfusepath{clip}%
\pgfsetbuttcap%
\pgfsetroundjoin%
\definecolor{currentfill}{rgb}{0.283072,0.130895,0.449241}%
\pgfsetfillcolor{currentfill}%
\pgfsetfillopacity{0.800000}%
\pgfsetlinewidth{0.000000pt}%
\definecolor{currentstroke}{rgb}{0.000000,0.000000,0.000000}%
\pgfsetstrokecolor{currentstroke}%
\pgfsetdash{}{0pt}%
\pgfpathmoveto{\pgfqpoint{2.730870in}{1.896835in}}%
\pgfpathlineto{\pgfqpoint{2.744739in}{1.882352in}}%
\pgfpathlineto{\pgfqpoint{2.758603in}{1.868119in}}%
\pgfpathlineto{\pgfqpoint{2.772461in}{1.854132in}}%
\pgfpathlineto{\pgfqpoint{2.786314in}{1.840391in}}%
\pgfpathlineto{\pgfqpoint{2.794988in}{1.842108in}}%
\pgfpathlineto{\pgfqpoint{2.803647in}{1.844073in}}%
\pgfpathlineto{\pgfqpoint{2.812292in}{1.846281in}}%
\pgfpathlineto{\pgfqpoint{2.820922in}{1.848725in}}%
\pgfpathlineto{\pgfqpoint{2.807108in}{1.861890in}}%
\pgfpathlineto{\pgfqpoint{2.793289in}{1.875299in}}%
\pgfpathlineto{\pgfqpoint{2.779465in}{1.888955in}}%
\pgfpathlineto{\pgfqpoint{2.765637in}{1.902858in}}%
\pgfpathlineto{\pgfqpoint{2.756968in}{1.900978in}}%
\pgfpathlineto{\pgfqpoint{2.748284in}{1.899344in}}%
\pgfpathlineto{\pgfqpoint{2.739585in}{1.897961in}}%
\pgfpathlineto{\pgfqpoint{2.730870in}{1.896835in}}%
\pgfpathclose%
\pgfusepath{fill}%
\end{pgfscope}%
\begin{pgfscope}%
\pgfpathrectangle{\pgfqpoint{1.150000in}{0.150000in}}{\pgfqpoint{5.700000in}{5.700000in}}%
\pgfusepath{clip}%
\pgfsetbuttcap%
\pgfsetroundjoin%
\definecolor{currentfill}{rgb}{0.273809,0.031497,0.358853}%
\pgfsetfillcolor{currentfill}%
\pgfsetfillopacity{0.800000}%
\pgfsetlinewidth{0.000000pt}%
\definecolor{currentstroke}{rgb}{0.000000,0.000000,0.000000}%
\pgfsetstrokecolor{currentstroke}%
\pgfsetdash{}{0pt}%
\pgfpathmoveto{\pgfqpoint{3.041514in}{1.670134in}}%
\pgfpathlineto{\pgfqpoint{3.055284in}{1.660904in}}%
\pgfpathlineto{\pgfqpoint{3.069054in}{1.651894in}}%
\pgfpathlineto{\pgfqpoint{3.082824in}{1.643102in}}%
\pgfpathlineto{\pgfqpoint{3.096593in}{1.634528in}}%
\pgfpathlineto{\pgfqpoint{3.105045in}{1.639953in}}%
\pgfpathlineto{\pgfqpoint{3.113486in}{1.645561in}}%
\pgfpathlineto{\pgfqpoint{3.121916in}{1.651348in}}%
\pgfpathlineto{\pgfqpoint{3.130336in}{1.657309in}}%
\pgfpathlineto{\pgfqpoint{3.116595in}{1.665349in}}%
\pgfpathlineto{\pgfqpoint{3.102854in}{1.673607in}}%
\pgfpathlineto{\pgfqpoint{3.089113in}{1.682083in}}%
\pgfpathlineto{\pgfqpoint{3.075371in}{1.690779in}}%
\pgfpathlineto{\pgfqpoint{3.066924in}{1.685340in}}%
\pgfpathlineto{\pgfqpoint{3.058465in}{1.680083in}}%
\pgfpathlineto{\pgfqpoint{3.049995in}{1.675012in}}%
\pgfpathlineto{\pgfqpoint{3.041514in}{1.670134in}}%
\pgfpathclose%
\pgfusepath{fill}%
\end{pgfscope}%
\begin{pgfscope}%
\pgfpathrectangle{\pgfqpoint{1.150000in}{0.150000in}}{\pgfqpoint{5.700000in}{5.700000in}}%
\pgfusepath{clip}%
\pgfsetbuttcap%
\pgfsetroundjoin%
\definecolor{currentfill}{rgb}{0.280894,0.078907,0.402329}%
\pgfsetfillcolor{currentfill}%
\pgfsetfillopacity{0.800000}%
\pgfsetlinewidth{0.000000pt}%
\definecolor{currentstroke}{rgb}{0.000000,0.000000,0.000000}%
\pgfsetstrokecolor{currentstroke}%
\pgfsetdash{}{0pt}%
\pgfpathmoveto{\pgfqpoint{3.734974in}{1.727236in}}%
\pgfpathlineto{\pgfqpoint{3.748764in}{1.728048in}}%
\pgfpathlineto{\pgfqpoint{3.762562in}{1.729051in}}%
\pgfpathlineto{\pgfqpoint{3.776367in}{1.730245in}}%
\pgfpathlineto{\pgfqpoint{3.790181in}{1.731630in}}%
\pgfpathlineto{\pgfqpoint{3.798301in}{1.743372in}}%
\pgfpathlineto{\pgfqpoint{3.806416in}{1.755125in}}%
\pgfpathlineto{\pgfqpoint{3.814525in}{1.766887in}}%
\pgfpathlineto{\pgfqpoint{3.822630in}{1.778654in}}%
\pgfpathlineto{\pgfqpoint{3.808824in}{1.776927in}}%
\pgfpathlineto{\pgfqpoint{3.795026in}{1.775391in}}%
\pgfpathlineto{\pgfqpoint{3.781237in}{1.774047in}}%
\pgfpathlineto{\pgfqpoint{3.767456in}{1.772893in}}%
\pgfpathlineto{\pgfqpoint{3.759343in}{1.761456in}}%
\pgfpathlineto{\pgfqpoint{3.751226in}{1.750032in}}%
\pgfpathlineto{\pgfqpoint{3.743103in}{1.738624in}}%
\pgfpathlineto{\pgfqpoint{3.734974in}{1.727236in}}%
\pgfpathclose%
\pgfusepath{fill}%
\end{pgfscope}%
\begin{pgfscope}%
\pgfpathrectangle{\pgfqpoint{1.150000in}{0.150000in}}{\pgfqpoint{5.700000in}{5.700000in}}%
\pgfusepath{clip}%
\pgfsetbuttcap%
\pgfsetroundjoin%
\definecolor{currentfill}{rgb}{0.180629,0.429975,0.557282}%
\pgfsetfillcolor{currentfill}%
\pgfsetfillopacity{0.800000}%
\pgfsetlinewidth{0.000000pt}%
\definecolor{currentstroke}{rgb}{0.000000,0.000000,0.000000}%
\pgfsetstrokecolor{currentstroke}%
\pgfsetdash{}{0pt}%
\pgfpathmoveto{\pgfqpoint{2.205359in}{2.686172in}}%
\pgfpathlineto{\pgfqpoint{2.219613in}{2.660535in}}%
\pgfpathlineto{\pgfqpoint{2.233850in}{2.635247in}}%
\pgfpathlineto{\pgfqpoint{2.248071in}{2.610303in}}%
\pgfpathlineto{\pgfqpoint{2.262276in}{2.585701in}}%
\pgfpathlineto{\pgfqpoint{2.271360in}{2.582651in}}%
\pgfpathlineto{\pgfqpoint{2.280424in}{2.579917in}}%
\pgfpathlineto{\pgfqpoint{2.289467in}{2.577493in}}%
\pgfpathlineto{\pgfqpoint{2.298489in}{2.575374in}}%
\pgfpathlineto{\pgfqpoint{2.284340in}{2.599393in}}%
\pgfpathlineto{\pgfqpoint{2.270177in}{2.623752in}}%
\pgfpathlineto{\pgfqpoint{2.255997in}{2.648455in}}%
\pgfpathlineto{\pgfqpoint{2.241800in}{2.673504in}}%
\pgfpathlineto{\pgfqpoint{2.232722in}{2.676194in}}%
\pgfpathlineto{\pgfqpoint{2.223623in}{2.679199in}}%
\pgfpathlineto{\pgfqpoint{2.214502in}{2.682523in}}%
\pgfpathlineto{\pgfqpoint{2.205359in}{2.686172in}}%
\pgfpathclose%
\pgfusepath{fill}%
\end{pgfscope}%
\begin{pgfscope}%
\pgfpathrectangle{\pgfqpoint{1.150000in}{0.150000in}}{\pgfqpoint{5.700000in}{5.700000in}}%
\pgfusepath{clip}%
\pgfsetbuttcap%
\pgfsetroundjoin%
\definecolor{currentfill}{rgb}{0.273809,0.031497,0.358853}%
\pgfsetfillcolor{currentfill}%
\pgfsetfillopacity{0.800000}%
\pgfsetlinewidth{0.000000pt}%
\definecolor{currentstroke}{rgb}{0.000000,0.000000,0.000000}%
\pgfsetstrokecolor{currentstroke}%
\pgfsetdash{}{0pt}%
\pgfpathmoveto{\pgfqpoint{3.559514in}{1.643892in}}%
\pgfpathlineto{\pgfqpoint{3.573272in}{1.642379in}}%
\pgfpathlineto{\pgfqpoint{3.587036in}{1.641061in}}%
\pgfpathlineto{\pgfqpoint{3.600806in}{1.639938in}}%
\pgfpathlineto{\pgfqpoint{3.614582in}{1.639009in}}%
\pgfpathlineto{\pgfqpoint{3.622766in}{1.649705in}}%
\pgfpathlineto{\pgfqpoint{3.630944in}{1.660455in}}%
\pgfpathlineto{\pgfqpoint{3.639116in}{1.671257in}}%
\pgfpathlineto{\pgfqpoint{3.647282in}{1.682107in}}%
\pgfpathlineto{\pgfqpoint{3.633518in}{1.682632in}}%
\pgfpathlineto{\pgfqpoint{3.619760in}{1.683351in}}%
\pgfpathlineto{\pgfqpoint{3.606008in}{1.684265in}}%
\pgfpathlineto{\pgfqpoint{3.592263in}{1.685375in}}%
\pgfpathlineto{\pgfqpoint{3.584085in}{1.674917in}}%
\pgfpathlineto{\pgfqpoint{3.575901in}{1.664514in}}%
\pgfpathlineto{\pgfqpoint{3.567711in}{1.654172in}}%
\pgfpathlineto{\pgfqpoint{3.559514in}{1.643892in}}%
\pgfpathclose%
\pgfusepath{fill}%
\end{pgfscope}%
\begin{pgfscope}%
\pgfpathrectangle{\pgfqpoint{1.150000in}{0.150000in}}{\pgfqpoint{5.700000in}{5.700000in}}%
\pgfusepath{clip}%
\pgfsetbuttcap%
\pgfsetroundjoin%
\definecolor{currentfill}{rgb}{0.250425,0.274290,0.533103}%
\pgfsetfillcolor{currentfill}%
\pgfsetfillopacity{0.800000}%
\pgfsetlinewidth{0.000000pt}%
\definecolor{currentstroke}{rgb}{0.000000,0.000000,0.000000}%
\pgfsetstrokecolor{currentstroke}%
\pgfsetdash{}{0pt}%
\pgfpathmoveto{\pgfqpoint{2.451953in}{2.241636in}}%
\pgfpathlineto{\pgfqpoint{2.465985in}{2.221768in}}%
\pgfpathlineto{\pgfqpoint{2.480007in}{2.202191in}}%
\pgfpathlineto{\pgfqpoint{2.494018in}{2.182901in}}%
\pgfpathlineto{\pgfqpoint{2.508018in}{2.163897in}}%
\pgfpathlineto{\pgfqpoint{2.516916in}{2.162626in}}%
\pgfpathlineto{\pgfqpoint{2.525795in}{2.161648in}}%
\pgfpathlineto{\pgfqpoint{2.534656in}{2.160957in}}%
\pgfpathlineto{\pgfqpoint{2.543499in}{2.160546in}}%
\pgfpathlineto{\pgfqpoint{2.529547in}{2.178956in}}%
\pgfpathlineto{\pgfqpoint{2.515585in}{2.197651in}}%
\pgfpathlineto{\pgfqpoint{2.501614in}{2.216632in}}%
\pgfpathlineto{\pgfqpoint{2.487632in}{2.235902in}}%
\pgfpathlineto{\pgfqpoint{2.478741in}{2.236894in}}%
\pgfpathlineto{\pgfqpoint{2.469830in}{2.238177in}}%
\pgfpathlineto{\pgfqpoint{2.460901in}{2.239755in}}%
\pgfpathlineto{\pgfqpoint{2.451953in}{2.241636in}}%
\pgfpathclose%
\pgfusepath{fill}%
\end{pgfscope}%
\begin{pgfscope}%
\pgfpathrectangle{\pgfqpoint{1.150000in}{0.150000in}}{\pgfqpoint{5.700000in}{5.700000in}}%
\pgfusepath{clip}%
\pgfsetbuttcap%
\pgfsetroundjoin%
\definecolor{currentfill}{rgb}{0.282910,0.105393,0.426902}%
\pgfsetfillcolor{currentfill}%
\pgfsetfillopacity{0.800000}%
\pgfsetlinewidth{0.000000pt}%
\definecolor{currentstroke}{rgb}{0.000000,0.000000,0.000000}%
\pgfsetstrokecolor{currentstroke}%
\pgfsetdash{}{0pt}%
\pgfpathmoveto{\pgfqpoint{3.822630in}{1.778654in}}%
\pgfpathlineto{\pgfqpoint{3.836444in}{1.780571in}}%
\pgfpathlineto{\pgfqpoint{3.850267in}{1.782678in}}%
\pgfpathlineto{\pgfqpoint{3.864099in}{1.784974in}}%
\pgfpathlineto{\pgfqpoint{3.877940in}{1.787460in}}%
\pgfpathlineto{\pgfqpoint{3.886033in}{1.799551in}}%
\pgfpathlineto{\pgfqpoint{3.894121in}{1.811634in}}%
\pgfpathlineto{\pgfqpoint{3.902204in}{1.823706in}}%
\pgfpathlineto{\pgfqpoint{3.910282in}{1.835763in}}%
\pgfpathlineto{\pgfqpoint{3.896447in}{1.832967in}}%
\pgfpathlineto{\pgfqpoint{3.882621in}{1.830360in}}%
\pgfpathlineto{\pgfqpoint{3.868805in}{1.827943in}}%
\pgfpathlineto{\pgfqpoint{3.854998in}{1.825716in}}%
\pgfpathlineto{\pgfqpoint{3.846913in}{1.813957in}}%
\pgfpathlineto{\pgfqpoint{3.838824in}{1.802192in}}%
\pgfpathlineto{\pgfqpoint{3.830729in}{1.790423in}}%
\pgfpathlineto{\pgfqpoint{3.822630in}{1.778654in}}%
\pgfpathclose%
\pgfusepath{fill}%
\end{pgfscope}%
\begin{pgfscope}%
\pgfpathrectangle{\pgfqpoint{1.150000in}{0.150000in}}{\pgfqpoint{5.700000in}{5.700000in}}%
\pgfusepath{clip}%
\pgfsetbuttcap%
\pgfsetroundjoin%
\definecolor{currentfill}{rgb}{0.197636,0.391528,0.554969}%
\pgfsetfillcolor{currentfill}%
\pgfsetfillopacity{0.800000}%
\pgfsetlinewidth{0.000000pt}%
\definecolor{currentstroke}{rgb}{0.000000,0.000000,0.000000}%
\pgfsetstrokecolor{currentstroke}%
\pgfsetdash{}{0pt}%
\pgfpathmoveto{\pgfqpoint{4.620283in}{2.477696in}}%
\pgfpathlineto{\pgfqpoint{4.634448in}{2.487241in}}%
\pgfpathlineto{\pgfqpoint{4.648629in}{2.496970in}}%
\pgfpathlineto{\pgfqpoint{4.662825in}{2.506884in}}%
\pgfpathlineto{\pgfqpoint{4.677037in}{2.516983in}}%
\pgfpathlineto{\pgfqpoint{4.684869in}{2.526903in}}%
\pgfpathlineto{\pgfqpoint{4.692694in}{2.536705in}}%
\pgfpathlineto{\pgfqpoint{4.700512in}{2.546389in}}%
\pgfpathlineto{\pgfqpoint{4.708324in}{2.555958in}}%
\pgfpathlineto{\pgfqpoint{4.694116in}{2.545906in}}%
\pgfpathlineto{\pgfqpoint{4.679924in}{2.536038in}}%
\pgfpathlineto{\pgfqpoint{4.665748in}{2.526355in}}%
\pgfpathlineto{\pgfqpoint{4.651588in}{2.516856in}}%
\pgfpathlineto{\pgfqpoint{4.643771in}{2.507229in}}%
\pgfpathlineto{\pgfqpoint{4.635948in}{2.497494in}}%
\pgfpathlineto{\pgfqpoint{4.628119in}{2.487649in}}%
\pgfpathlineto{\pgfqpoint{4.620283in}{2.477696in}}%
\pgfpathclose%
\pgfusepath{fill}%
\end{pgfscope}%
\begin{pgfscope}%
\pgfpathrectangle{\pgfqpoint{1.150000in}{0.150000in}}{\pgfqpoint{5.700000in}{5.700000in}}%
\pgfusepath{clip}%
\pgfsetbuttcap%
\pgfsetroundjoin%
\definecolor{currentfill}{rgb}{0.132268,0.655014,0.519661}%
\pgfsetfillcolor{currentfill}%
\pgfsetfillopacity{0.800000}%
\pgfsetlinewidth{0.000000pt}%
\definecolor{currentstroke}{rgb}{0.000000,0.000000,0.000000}%
\pgfsetstrokecolor{currentstroke}%
\pgfsetdash{}{0pt}%
\pgfpathmoveto{\pgfqpoint{5.593423in}{3.288111in}}%
\pgfpathlineto{\pgfqpoint{5.608165in}{3.301686in}}%
\pgfpathlineto{\pgfqpoint{5.622929in}{3.315442in}}%
\pgfpathlineto{\pgfqpoint{5.637715in}{3.329379in}}%
\pgfpathlineto{\pgfqpoint{5.652521in}{3.343498in}}%
\pgfpathlineto{\pgfqpoint{5.659804in}{3.345644in}}%
\pgfpathlineto{\pgfqpoint{5.667078in}{3.347753in}}%
\pgfpathlineto{\pgfqpoint{5.674345in}{3.349829in}}%
\pgfpathlineto{\pgfqpoint{5.681603in}{3.351877in}}%
\pgfpathlineto{\pgfqpoint{5.666823in}{3.338256in}}%
\pgfpathlineto{\pgfqpoint{5.652064in}{3.324816in}}%
\pgfpathlineto{\pgfqpoint{5.637325in}{3.311556in}}%
\pgfpathlineto{\pgfqpoint{5.622608in}{3.298476in}}%
\pgfpathlineto{\pgfqpoint{5.615323in}{3.295919in}}%
\pgfpathlineto{\pgfqpoint{5.608030in}{3.293343in}}%
\pgfpathlineto{\pgfqpoint{5.600730in}{3.290742in}}%
\pgfpathlineto{\pgfqpoint{5.593423in}{3.288111in}}%
\pgfpathclose%
\pgfusepath{fill}%
\end{pgfscope}%
\begin{pgfscope}%
\pgfpathrectangle{\pgfqpoint{1.150000in}{0.150000in}}{\pgfqpoint{5.700000in}{5.700000in}}%
\pgfusepath{clip}%
\pgfsetbuttcap%
\pgfsetroundjoin%
\definecolor{currentfill}{rgb}{0.283091,0.110553,0.431554}%
\pgfsetfillcolor{currentfill}%
\pgfsetfillopacity{0.800000}%
\pgfsetlinewidth{0.000000pt}%
\definecolor{currentstroke}{rgb}{0.000000,0.000000,0.000000}%
\pgfsetstrokecolor{currentstroke}%
\pgfsetdash{}{0pt}%
\pgfpathmoveto{\pgfqpoint{2.786314in}{1.840391in}}%
\pgfpathlineto{\pgfqpoint{2.800163in}{1.826894in}}%
\pgfpathlineto{\pgfqpoint{2.814007in}{1.813638in}}%
\pgfpathlineto{\pgfqpoint{2.827847in}{1.800624in}}%
\pgfpathlineto{\pgfqpoint{2.841682in}{1.787848in}}%
\pgfpathlineto{\pgfqpoint{2.850317in}{1.790153in}}%
\pgfpathlineto{\pgfqpoint{2.858938in}{1.792697in}}%
\pgfpathlineto{\pgfqpoint{2.867545in}{1.795476in}}%
\pgfpathlineto{\pgfqpoint{2.876138in}{1.798482in}}%
\pgfpathlineto{\pgfqpoint{2.862340in}{1.810684in}}%
\pgfpathlineto{\pgfqpoint{2.848538in}{1.823124in}}%
\pgfpathlineto{\pgfqpoint{2.834732in}{1.835804in}}%
\pgfpathlineto{\pgfqpoint{2.820922in}{1.848725in}}%
\pgfpathlineto{\pgfqpoint{2.812292in}{1.846281in}}%
\pgfpathlineto{\pgfqpoint{2.803647in}{1.844073in}}%
\pgfpathlineto{\pgfqpoint{2.794988in}{1.842108in}}%
\pgfpathlineto{\pgfqpoint{2.786314in}{1.840391in}}%
\pgfpathclose%
\pgfusepath{fill}%
\end{pgfscope}%
\begin{pgfscope}%
\pgfpathrectangle{\pgfqpoint{1.150000in}{0.150000in}}{\pgfqpoint{5.700000in}{5.700000in}}%
\pgfusepath{clip}%
\pgfsetbuttcap%
\pgfsetroundjoin%
\definecolor{currentfill}{rgb}{0.231674,0.318106,0.544834}%
\pgfsetfillcolor{currentfill}%
\pgfsetfillopacity{0.800000}%
\pgfsetlinewidth{0.000000pt}%
\definecolor{currentstroke}{rgb}{0.000000,0.000000,0.000000}%
\pgfsetstrokecolor{currentstroke}%
\pgfsetdash{}{0pt}%
\pgfpathmoveto{\pgfqpoint{4.412909in}{2.280262in}}%
\pgfpathlineto{\pgfqpoint{4.426966in}{2.288267in}}%
\pgfpathlineto{\pgfqpoint{4.441037in}{2.296456in}}%
\pgfpathlineto{\pgfqpoint{4.455122in}{2.304831in}}%
\pgfpathlineto{\pgfqpoint{4.469222in}{2.313390in}}%
\pgfpathlineto{\pgfqpoint{4.477134in}{2.324634in}}%
\pgfpathlineto{\pgfqpoint{4.485040in}{2.335772in}}%
\pgfpathlineto{\pgfqpoint{4.492940in}{2.346804in}}%
\pgfpathlineto{\pgfqpoint{4.500835in}{2.357731in}}%
\pgfpathlineto{\pgfqpoint{4.486738in}{2.349118in}}%
\pgfpathlineto{\pgfqpoint{4.472656in}{2.340689in}}%
\pgfpathlineto{\pgfqpoint{4.458588in}{2.332446in}}%
\pgfpathlineto{\pgfqpoint{4.444534in}{2.324388in}}%
\pgfpathlineto{\pgfqpoint{4.436636in}{2.313503in}}%
\pgfpathlineto{\pgfqpoint{4.428733in}{2.302520in}}%
\pgfpathlineto{\pgfqpoint{4.420824in}{2.291440in}}%
\pgfpathlineto{\pgfqpoint{4.412909in}{2.280262in}}%
\pgfpathclose%
\pgfusepath{fill}%
\end{pgfscope}%
\begin{pgfscope}%
\pgfpathrectangle{\pgfqpoint{1.150000in}{0.150000in}}{\pgfqpoint{5.700000in}{5.700000in}}%
\pgfusepath{clip}%
\pgfsetbuttcap%
\pgfsetroundjoin%
\definecolor{currentfill}{rgb}{0.269944,0.014625,0.341379}%
\pgfsetfillcolor{currentfill}%
\pgfsetfillopacity{0.800000}%
\pgfsetlinewidth{0.000000pt}%
\definecolor{currentstroke}{rgb}{0.000000,0.000000,0.000000}%
\pgfsetstrokecolor{currentstroke}%
\pgfsetdash{}{0pt}%
\pgfpathmoveto{\pgfqpoint{3.471629in}{1.613240in}}%
\pgfpathlineto{\pgfqpoint{3.485380in}{1.610505in}}%
\pgfpathlineto{\pgfqpoint{3.499137in}{1.607967in}}%
\pgfpathlineto{\pgfqpoint{3.512898in}{1.605626in}}%
\pgfpathlineto{\pgfqpoint{3.526665in}{1.603482in}}%
\pgfpathlineto{\pgfqpoint{3.534887in}{1.613470in}}%
\pgfpathlineto{\pgfqpoint{3.543102in}{1.623537in}}%
\pgfpathlineto{\pgfqpoint{3.551311in}{1.633679in}}%
\pgfpathlineto{\pgfqpoint{3.559514in}{1.643892in}}%
\pgfpathlineto{\pgfqpoint{3.545762in}{1.645601in}}%
\pgfpathlineto{\pgfqpoint{3.532015in}{1.647506in}}%
\pgfpathlineto{\pgfqpoint{3.518273in}{1.649609in}}%
\pgfpathlineto{\pgfqpoint{3.504537in}{1.651910in}}%
\pgfpathlineto{\pgfqpoint{3.496320in}{1.642120in}}%
\pgfpathlineto{\pgfqpoint{3.488096in}{1.632409in}}%
\pgfpathlineto{\pgfqpoint{3.479866in}{1.622781in}}%
\pgfpathlineto{\pgfqpoint{3.471629in}{1.613240in}}%
\pgfpathclose%
\pgfusepath{fill}%
\end{pgfscope}%
\begin{pgfscope}%
\pgfpathrectangle{\pgfqpoint{1.150000in}{0.150000in}}{\pgfqpoint{5.700000in}{5.700000in}}%
\pgfusepath{clip}%
\pgfsetbuttcap%
\pgfsetroundjoin%
\definecolor{currentfill}{rgb}{0.282884,0.135920,0.453427}%
\pgfsetfillcolor{currentfill}%
\pgfsetfillopacity{0.800000}%
\pgfsetlinewidth{0.000000pt}%
\definecolor{currentstroke}{rgb}{0.000000,0.000000,0.000000}%
\pgfsetstrokecolor{currentstroke}%
\pgfsetdash{}{0pt}%
\pgfpathmoveto{\pgfqpoint{3.910282in}{1.835763in}}%
\pgfpathlineto{\pgfqpoint{3.924126in}{1.838749in}}%
\pgfpathlineto{\pgfqpoint{3.937980in}{1.841922in}}%
\pgfpathlineto{\pgfqpoint{3.951844in}{1.845285in}}%
\pgfpathlineto{\pgfqpoint{3.965717in}{1.848835in}}%
\pgfpathlineto{\pgfqpoint{3.973785in}{1.861167in}}%
\pgfpathlineto{\pgfqpoint{3.981848in}{1.873472in}}%
\pgfpathlineto{\pgfqpoint{3.989907in}{1.885747in}}%
\pgfpathlineto{\pgfqpoint{3.997960in}{1.897990in}}%
\pgfpathlineto{\pgfqpoint{3.984092in}{1.894161in}}%
\pgfpathlineto{\pgfqpoint{3.970233in}{1.890520in}}%
\pgfpathlineto{\pgfqpoint{3.956384in}{1.887067in}}%
\pgfpathlineto{\pgfqpoint{3.942546in}{1.883803in}}%
\pgfpathlineto{\pgfqpoint{3.934487in}{1.871826in}}%
\pgfpathlineto{\pgfqpoint{3.926423in}{1.859826in}}%
\pgfpathlineto{\pgfqpoint{3.918355in}{1.847804in}}%
\pgfpathlineto{\pgfqpoint{3.910282in}{1.835763in}}%
\pgfpathclose%
\pgfusepath{fill}%
\end{pgfscope}%
\begin{pgfscope}%
\pgfpathrectangle{\pgfqpoint{1.150000in}{0.150000in}}{\pgfqpoint{5.700000in}{5.700000in}}%
\pgfusepath{clip}%
\pgfsetbuttcap%
\pgfsetroundjoin%
\definecolor{currentfill}{rgb}{0.128729,0.563265,0.551229}%
\pgfsetfillcolor{currentfill}%
\pgfsetfillopacity{0.800000}%
\pgfsetlinewidth{0.000000pt}%
\definecolor{currentstroke}{rgb}{0.000000,0.000000,0.000000}%
\pgfsetstrokecolor{currentstroke}%
\pgfsetdash{}{0pt}%
\pgfpathmoveto{\pgfqpoint{5.210797in}{2.997586in}}%
\pgfpathlineto{\pgfqpoint{5.225314in}{3.010209in}}%
\pgfpathlineto{\pgfqpoint{5.239849in}{3.023014in}}%
\pgfpathlineto{\pgfqpoint{5.254405in}{3.036003in}}%
\pgfpathlineto{\pgfqpoint{5.268980in}{3.049175in}}%
\pgfpathlineto{\pgfqpoint{5.276512in}{3.054362in}}%
\pgfpathlineto{\pgfqpoint{5.284036in}{3.059453in}}%
\pgfpathlineto{\pgfqpoint{5.291551in}{3.064452in}}%
\pgfpathlineto{\pgfqpoint{5.299058in}{3.069363in}}%
\pgfpathlineto{\pgfqpoint{5.284498in}{3.056514in}}%
\pgfpathlineto{\pgfqpoint{5.269958in}{3.043848in}}%
\pgfpathlineto{\pgfqpoint{5.255438in}{3.031364in}}%
\pgfpathlineto{\pgfqpoint{5.240936in}{3.019061in}}%
\pgfpathlineto{\pgfqpoint{5.233414in}{3.013817in}}%
\pgfpathlineto{\pgfqpoint{5.225883in}{3.008492in}}%
\pgfpathlineto{\pgfqpoint{5.218344in}{3.003083in}}%
\pgfpathlineto{\pgfqpoint{5.210797in}{2.997586in}}%
\pgfpathclose%
\pgfusepath{fill}%
\end{pgfscope}%
\begin{pgfscope}%
\pgfpathrectangle{\pgfqpoint{1.150000in}{0.150000in}}{\pgfqpoint{5.700000in}{5.700000in}}%
\pgfusepath{clip}%
\pgfsetbuttcap%
\pgfsetroundjoin%
\definecolor{currentfill}{rgb}{0.237441,0.305202,0.541921}%
\pgfsetfillcolor{currentfill}%
\pgfsetfillopacity{0.800000}%
\pgfsetlinewidth{0.000000pt}%
\definecolor{currentstroke}{rgb}{0.000000,0.000000,0.000000}%
\pgfsetstrokecolor{currentstroke}%
\pgfsetdash{}{0pt}%
\pgfpathmoveto{\pgfqpoint{2.395711in}{2.324059in}}%
\pgfpathlineto{\pgfqpoint{2.409789in}{2.303005in}}%
\pgfpathlineto{\pgfqpoint{2.423855in}{2.282251in}}%
\pgfpathlineto{\pgfqpoint{2.437910in}{2.261796in}}%
\pgfpathlineto{\pgfqpoint{2.451953in}{2.241636in}}%
\pgfpathlineto{\pgfqpoint{2.460901in}{2.239755in}}%
\pgfpathlineto{\pgfqpoint{2.469830in}{2.238177in}}%
\pgfpathlineto{\pgfqpoint{2.478741in}{2.236894in}}%
\pgfpathlineto{\pgfqpoint{2.487632in}{2.235902in}}%
\pgfpathlineto{\pgfqpoint{2.473640in}{2.255463in}}%
\pgfpathlineto{\pgfqpoint{2.459636in}{2.275319in}}%
\pgfpathlineto{\pgfqpoint{2.445622in}{2.295471in}}%
\pgfpathlineto{\pgfqpoint{2.431596in}{2.315923in}}%
\pgfpathlineto{\pgfqpoint{2.422654in}{2.317502in}}%
\pgfpathlineto{\pgfqpoint{2.413693in}{2.319381in}}%
\pgfpathlineto{\pgfqpoint{2.404712in}{2.321564in}}%
\pgfpathlineto{\pgfqpoint{2.395711in}{2.324059in}}%
\pgfpathclose%
\pgfusepath{fill}%
\end{pgfscope}%
\begin{pgfscope}%
\pgfpathrectangle{\pgfqpoint{1.150000in}{0.150000in}}{\pgfqpoint{5.700000in}{5.700000in}}%
\pgfusepath{clip}%
\pgfsetbuttcap%
\pgfsetroundjoin%
\definecolor{currentfill}{rgb}{0.262138,0.242286,0.520837}%
\pgfsetfillcolor{currentfill}%
\pgfsetfillopacity{0.800000}%
\pgfsetlinewidth{0.000000pt}%
\definecolor{currentstroke}{rgb}{0.000000,0.000000,0.000000}%
\pgfsetstrokecolor{currentstroke}%
\pgfsetdash{}{0pt}%
\pgfpathmoveto{\pgfqpoint{4.205478in}{2.084220in}}%
\pgfpathlineto{\pgfqpoint{4.219438in}{2.090380in}}%
\pgfpathlineto{\pgfqpoint{4.233410in}{2.096727in}}%
\pgfpathlineto{\pgfqpoint{4.247394in}{2.103260in}}%
\pgfpathlineto{\pgfqpoint{4.261391in}{2.109978in}}%
\pgfpathlineto{\pgfqpoint{4.269372in}{2.122118in}}%
\pgfpathlineto{\pgfqpoint{4.277348in}{2.134176in}}%
\pgfpathlineto{\pgfqpoint{4.285319in}{2.146152in}}%
\pgfpathlineto{\pgfqpoint{4.293284in}{2.158046in}}%
\pgfpathlineto{\pgfqpoint{4.279290in}{2.151175in}}%
\pgfpathlineto{\pgfqpoint{4.265308in}{2.144491in}}%
\pgfpathlineto{\pgfqpoint{4.251339in}{2.137992in}}%
\pgfpathlineto{\pgfqpoint{4.237382in}{2.131680in}}%
\pgfpathlineto{\pgfqpoint{4.229414in}{2.119926in}}%
\pgfpathlineto{\pgfqpoint{4.221440in}{2.108098in}}%
\pgfpathlineto{\pgfqpoint{4.213462in}{2.096195in}}%
\pgfpathlineto{\pgfqpoint{4.205478in}{2.084220in}}%
\pgfpathclose%
\pgfusepath{fill}%
\end{pgfscope}%
\begin{pgfscope}%
\pgfpathrectangle{\pgfqpoint{1.150000in}{0.150000in}}{\pgfqpoint{5.700000in}{5.700000in}}%
\pgfusepath{clip}%
\pgfsetbuttcap%
\pgfsetroundjoin%
\definecolor{currentfill}{rgb}{0.268510,0.009605,0.335427}%
\pgfsetfillcolor{currentfill}%
\pgfsetfillopacity{0.800000}%
\pgfsetlinewidth{0.000000pt}%
\definecolor{currentstroke}{rgb}{0.000000,0.000000,0.000000}%
\pgfsetstrokecolor{currentstroke}%
\pgfsetdash{}{0pt}%
\pgfpathmoveto{\pgfqpoint{3.240291in}{1.600674in}}%
\pgfpathlineto{\pgfqpoint{3.254042in}{1.594542in}}%
\pgfpathlineto{\pgfqpoint{3.267795in}{1.588616in}}%
\pgfpathlineto{\pgfqpoint{3.281549in}{1.582897in}}%
\pgfpathlineto{\pgfqpoint{3.295307in}{1.577383in}}%
\pgfpathlineto{\pgfqpoint{3.303644in}{1.585047in}}%
\pgfpathlineto{\pgfqpoint{3.311972in}{1.592851in}}%
\pgfpathlineto{\pgfqpoint{3.320292in}{1.600789in}}%
\pgfpathlineto{\pgfqpoint{3.328603in}{1.608857in}}%
\pgfpathlineto{\pgfqpoint{3.314868in}{1.613872in}}%
\pgfpathlineto{\pgfqpoint{3.301135in}{1.619092in}}%
\pgfpathlineto{\pgfqpoint{3.287404in}{1.624518in}}%
\pgfpathlineto{\pgfqpoint{3.273676in}{1.630152in}}%
\pgfpathlineto{\pgfqpoint{3.265343in}{1.622571in}}%
\pgfpathlineto{\pgfqpoint{3.257002in}{1.615128in}}%
\pgfpathlineto{\pgfqpoint{3.248651in}{1.607828in}}%
\pgfpathlineto{\pgfqpoint{3.240291in}{1.600674in}}%
\pgfpathclose%
\pgfusepath{fill}%
\end{pgfscope}%
\begin{pgfscope}%
\pgfpathrectangle{\pgfqpoint{1.150000in}{0.150000in}}{\pgfqpoint{5.700000in}{5.700000in}}%
\pgfusepath{clip}%
\pgfsetbuttcap%
\pgfsetroundjoin%
\definecolor{currentfill}{rgb}{0.150148,0.676631,0.506589}%
\pgfsetfillcolor{currentfill}%
\pgfsetfillopacity{0.800000}%
\pgfsetlinewidth{0.000000pt}%
\definecolor{currentstroke}{rgb}{0.000000,0.000000,0.000000}%
\pgfsetstrokecolor{currentstroke}%
\pgfsetdash{}{0pt}%
\pgfpathmoveto{\pgfqpoint{5.681603in}{3.351877in}}%
\pgfpathlineto{\pgfqpoint{5.696405in}{3.365679in}}%
\pgfpathlineto{\pgfqpoint{5.711229in}{3.379662in}}%
\pgfpathlineto{\pgfqpoint{5.726074in}{3.393825in}}%
\pgfpathlineto{\pgfqpoint{5.740941in}{3.408171in}}%
\pgfpathlineto{\pgfqpoint{5.748164in}{3.409677in}}%
\pgfpathlineto{\pgfqpoint{5.755380in}{3.411161in}}%
\pgfpathlineto{\pgfqpoint{5.762588in}{3.412626in}}%
\pgfpathlineto{\pgfqpoint{5.769789in}{3.414079in}}%
\pgfpathlineto{\pgfqpoint{5.754951in}{3.400267in}}%
\pgfpathlineto{\pgfqpoint{5.740134in}{3.386636in}}%
\pgfpathlineto{\pgfqpoint{5.725339in}{3.373185in}}%
\pgfpathlineto{\pgfqpoint{5.710565in}{3.359913in}}%
\pgfpathlineto{\pgfqpoint{5.703335in}{3.357917in}}%
\pgfpathlineto{\pgfqpoint{5.696099in}{3.355916in}}%
\pgfpathlineto{\pgfqpoint{5.688855in}{3.353904in}}%
\pgfpathlineto{\pgfqpoint{5.681603in}{3.351877in}}%
\pgfpathclose%
\pgfusepath{fill}%
\end{pgfscope}%
\begin{pgfscope}%
\pgfpathrectangle{\pgfqpoint{1.150000in}{0.150000in}}{\pgfqpoint{5.700000in}{5.700000in}}%
\pgfusepath{clip}%
\pgfsetbuttcap%
\pgfsetroundjoin%
\definecolor{currentfill}{rgb}{0.159194,0.482237,0.558073}%
\pgfsetfillcolor{currentfill}%
\pgfsetfillopacity{0.800000}%
\pgfsetlinewidth{0.000000pt}%
\definecolor{currentstroke}{rgb}{0.000000,0.000000,0.000000}%
\pgfsetstrokecolor{currentstroke}%
\pgfsetdash{}{0pt}%
\pgfpathmoveto{\pgfqpoint{4.915637in}{2.746501in}}%
\pgfpathlineto{\pgfqpoint{4.929977in}{2.757865in}}%
\pgfpathlineto{\pgfqpoint{4.944335in}{2.769413in}}%
\pgfpathlineto{\pgfqpoint{4.958710in}{2.781145in}}%
\pgfpathlineto{\pgfqpoint{4.973103in}{2.793061in}}%
\pgfpathlineto{\pgfqpoint{4.980801in}{2.800740in}}%
\pgfpathlineto{\pgfqpoint{4.988491in}{2.808301in}}%
\pgfpathlineto{\pgfqpoint{4.996174in}{2.815746in}}%
\pgfpathlineto{\pgfqpoint{5.003848in}{2.823078in}}%
\pgfpathlineto{\pgfqpoint{4.989464in}{2.811345in}}%
\pgfpathlineto{\pgfqpoint{4.975098in}{2.799797in}}%
\pgfpathlineto{\pgfqpoint{4.960749in}{2.788431in}}%
\pgfpathlineto{\pgfqpoint{4.946418in}{2.777249in}}%
\pgfpathlineto{\pgfqpoint{4.938734in}{2.769722in}}%
\pgfpathlineto{\pgfqpoint{4.931043in}{2.762090in}}%
\pgfpathlineto{\pgfqpoint{4.923344in}{2.754350in}}%
\pgfpathlineto{\pgfqpoint{4.915637in}{2.746501in}}%
\pgfpathclose%
\pgfusepath{fill}%
\end{pgfscope}%
\begin{pgfscope}%
\pgfpathrectangle{\pgfqpoint{1.150000in}{0.150000in}}{\pgfqpoint{5.700000in}{5.700000in}}%
\pgfusepath{clip}%
\pgfsetbuttcap%
\pgfsetroundjoin%
\definecolor{currentfill}{rgb}{0.281924,0.089666,0.412415}%
\pgfsetfillcolor{currentfill}%
\pgfsetfillopacity{0.800000}%
\pgfsetlinewidth{0.000000pt}%
\definecolor{currentstroke}{rgb}{0.000000,0.000000,0.000000}%
\pgfsetstrokecolor{currentstroke}%
\pgfsetdash{}{0pt}%
\pgfpathmoveto{\pgfqpoint{2.841682in}{1.787848in}}%
\pgfpathlineto{\pgfqpoint{2.855514in}{1.775309in}}%
\pgfpathlineto{\pgfqpoint{2.869342in}{1.763007in}}%
\pgfpathlineto{\pgfqpoint{2.883167in}{1.750939in}}%
\pgfpathlineto{\pgfqpoint{2.896989in}{1.739104in}}%
\pgfpathlineto{\pgfqpoint{2.905586in}{1.741995in}}%
\pgfpathlineto{\pgfqpoint{2.914170in}{1.745116in}}%
\pgfpathlineto{\pgfqpoint{2.922741in}{1.748463in}}%
\pgfpathlineto{\pgfqpoint{2.931299in}{1.752029in}}%
\pgfpathlineto{\pgfqpoint{2.917513in}{1.763292in}}%
\pgfpathlineto{\pgfqpoint{2.903725in}{1.774788in}}%
\pgfpathlineto{\pgfqpoint{2.889933in}{1.786517in}}%
\pgfpathlineto{\pgfqpoint{2.876138in}{1.798482in}}%
\pgfpathlineto{\pgfqpoint{2.867545in}{1.795476in}}%
\pgfpathlineto{\pgfqpoint{2.858938in}{1.792697in}}%
\pgfpathlineto{\pgfqpoint{2.850317in}{1.790153in}}%
\pgfpathlineto{\pgfqpoint{2.841682in}{1.787848in}}%
\pgfpathclose%
\pgfusepath{fill}%
\end{pgfscope}%
\begin{pgfscope}%
\pgfpathrectangle{\pgfqpoint{1.150000in}{0.150000in}}{\pgfqpoint{5.700000in}{5.700000in}}%
\pgfusepath{clip}%
\pgfsetbuttcap%
\pgfsetroundjoin%
\definecolor{currentfill}{rgb}{0.271305,0.019942,0.347269}%
\pgfsetfillcolor{currentfill}%
\pgfsetfillopacity{0.800000}%
\pgfsetlinewidth{0.000000pt}%
\definecolor{currentstroke}{rgb}{0.000000,0.000000,0.000000}%
\pgfsetstrokecolor{currentstroke}%
\pgfsetdash{}{0pt}%
\pgfpathmoveto{\pgfqpoint{3.096593in}{1.634528in}}%
\pgfpathlineto{\pgfqpoint{3.110362in}{1.626170in}}%
\pgfpathlineto{\pgfqpoint{3.124131in}{1.618027in}}%
\pgfpathlineto{\pgfqpoint{3.137901in}{1.610099in}}%
\pgfpathlineto{\pgfqpoint{3.151670in}{1.602383in}}%
\pgfpathlineto{\pgfqpoint{3.160094in}{1.608353in}}%
\pgfpathlineto{\pgfqpoint{3.168508in}{1.614499in}}%
\pgfpathlineto{\pgfqpoint{3.176911in}{1.620815in}}%
\pgfpathlineto{\pgfqpoint{3.185305in}{1.627296in}}%
\pgfpathlineto{\pgfqpoint{3.171562in}{1.634479in}}%
\pgfpathlineto{\pgfqpoint{3.157819in}{1.641875in}}%
\pgfpathlineto{\pgfqpoint{3.144077in}{1.649484in}}%
\pgfpathlineto{\pgfqpoint{3.130336in}{1.657309in}}%
\pgfpathlineto{\pgfqpoint{3.121916in}{1.651348in}}%
\pgfpathlineto{\pgfqpoint{3.113486in}{1.645561in}}%
\pgfpathlineto{\pgfqpoint{3.105045in}{1.639953in}}%
\pgfpathlineto{\pgfqpoint{3.096593in}{1.634528in}}%
\pgfpathclose%
\pgfusepath{fill}%
\end{pgfscope}%
\begin{pgfscope}%
\pgfpathrectangle{\pgfqpoint{1.150000in}{0.150000in}}{\pgfqpoint{5.700000in}{5.700000in}}%
\pgfusepath{clip}%
\pgfsetbuttcap%
\pgfsetroundjoin%
\definecolor{currentfill}{rgb}{0.268510,0.009605,0.335427}%
\pgfsetfillcolor{currentfill}%
\pgfsetfillopacity{0.800000}%
\pgfsetlinewidth{0.000000pt}%
\definecolor{currentstroke}{rgb}{0.000000,0.000000,0.000000}%
\pgfsetstrokecolor{currentstroke}%
\pgfsetdash{}{0pt}%
\pgfpathmoveto{\pgfqpoint{3.383577in}{1.590834in}}%
\pgfpathlineto{\pgfqpoint{3.397329in}{1.586833in}}%
\pgfpathlineto{\pgfqpoint{3.411085in}{1.583033in}}%
\pgfpathlineto{\pgfqpoint{3.424845in}{1.579433in}}%
\pgfpathlineto{\pgfqpoint{3.438609in}{1.576032in}}%
\pgfpathlineto{\pgfqpoint{3.446874in}{1.585182in}}%
\pgfpathlineto{\pgfqpoint{3.455133in}{1.594437in}}%
\pgfpathlineto{\pgfqpoint{3.463384in}{1.603791in}}%
\pgfpathlineto{\pgfqpoint{3.471629in}{1.613240in}}%
\pgfpathlineto{\pgfqpoint{3.457882in}{1.616175in}}%
\pgfpathlineto{\pgfqpoint{3.444139in}{1.619308in}}%
\pgfpathlineto{\pgfqpoint{3.430401in}{1.622641in}}%
\pgfpathlineto{\pgfqpoint{3.416667in}{1.626175in}}%
\pgfpathlineto{\pgfqpoint{3.408406in}{1.617180in}}%
\pgfpathlineto{\pgfqpoint{3.400137in}{1.608288in}}%
\pgfpathlineto{\pgfqpoint{3.391861in}{1.599505in}}%
\pgfpathlineto{\pgfqpoint{3.383577in}{1.590834in}}%
\pgfpathclose%
\pgfusepath{fill}%
\end{pgfscope}%
\begin{pgfscope}%
\pgfpathrectangle{\pgfqpoint{1.150000in}{0.150000in}}{\pgfqpoint{5.700000in}{5.700000in}}%
\pgfusepath{clip}%
\pgfsetbuttcap%
\pgfsetroundjoin%
\definecolor{currentfill}{rgb}{0.175707,0.697900,0.491033}%
\pgfsetfillcolor{currentfill}%
\pgfsetfillopacity{0.800000}%
\pgfsetlinewidth{0.000000pt}%
\definecolor{currentstroke}{rgb}{0.000000,0.000000,0.000000}%
\pgfsetstrokecolor{currentstroke}%
\pgfsetdash{}{0pt}%
\pgfpathmoveto{\pgfqpoint{5.769789in}{3.414079in}}%
\pgfpathlineto{\pgfqpoint{5.784649in}{3.428071in}}%
\pgfpathlineto{\pgfqpoint{5.799531in}{3.442244in}}%
\pgfpathlineto{\pgfqpoint{5.814435in}{3.456597in}}%
\pgfpathlineto{\pgfqpoint{5.829361in}{3.471132in}}%
\pgfpathlineto{\pgfqpoint{5.836524in}{3.472023in}}%
\pgfpathlineto{\pgfqpoint{5.843680in}{3.472906in}}%
\pgfpathlineto{\pgfqpoint{5.850828in}{3.473787in}}%
\pgfpathlineto{\pgfqpoint{5.857970in}{3.474673in}}%
\pgfpathlineto{\pgfqpoint{5.843075in}{3.460708in}}%
\pgfpathlineto{\pgfqpoint{5.828202in}{3.446922in}}%
\pgfpathlineto{\pgfqpoint{5.813351in}{3.433316in}}%
\pgfpathlineto{\pgfqpoint{5.798522in}{3.419890in}}%
\pgfpathlineto{\pgfqpoint{5.791349in}{3.418425in}}%
\pgfpathlineto{\pgfqpoint{5.784169in}{3.416973in}}%
\pgfpathlineto{\pgfqpoint{5.776982in}{3.415526in}}%
\pgfpathlineto{\pgfqpoint{5.769789in}{3.414079in}}%
\pgfpathclose%
\pgfusepath{fill}%
\end{pgfscope}%
\begin{pgfscope}%
\pgfpathrectangle{\pgfqpoint{1.150000in}{0.150000in}}{\pgfqpoint{5.700000in}{5.700000in}}%
\pgfusepath{clip}%
\pgfsetbuttcap%
\pgfsetroundjoin%
\definecolor{currentfill}{rgb}{0.280255,0.165693,0.476498}%
\pgfsetfillcolor{currentfill}%
\pgfsetfillopacity{0.800000}%
\pgfsetlinewidth{0.000000pt}%
\definecolor{currentstroke}{rgb}{0.000000,0.000000,0.000000}%
\pgfsetstrokecolor{currentstroke}%
\pgfsetdash{}{0pt}%
\pgfpathmoveto{\pgfqpoint{3.997960in}{1.897990in}}%
\pgfpathlineto{\pgfqpoint{4.011839in}{1.902008in}}%
\pgfpathlineto{\pgfqpoint{4.025729in}{1.906213in}}%
\pgfpathlineto{\pgfqpoint{4.039629in}{1.910605in}}%
\pgfpathlineto{\pgfqpoint{4.053541in}{1.915184in}}%
\pgfpathlineto{\pgfqpoint{4.061585in}{1.927653in}}%
\pgfpathlineto{\pgfqpoint{4.069625in}{1.940078in}}%
\pgfpathlineto{\pgfqpoint{4.077660in}{1.952455in}}%
\pgfpathlineto{\pgfqpoint{4.085691in}{1.964784in}}%
\pgfpathlineto{\pgfqpoint{4.071783in}{1.959957in}}%
\pgfpathlineto{\pgfqpoint{4.057887in}{1.955317in}}%
\pgfpathlineto{\pgfqpoint{4.044002in}{1.950864in}}%
\pgfpathlineto{\pgfqpoint{4.030127in}{1.946599in}}%
\pgfpathlineto{\pgfqpoint{4.022092in}{1.934506in}}%
\pgfpathlineto{\pgfqpoint{4.014053in}{1.922372in}}%
\pgfpathlineto{\pgfqpoint{4.006009in}{1.910200in}}%
\pgfpathlineto{\pgfqpoint{3.997960in}{1.897990in}}%
\pgfpathclose%
\pgfusepath{fill}%
\end{pgfscope}%
\begin{pgfscope}%
\pgfpathrectangle{\pgfqpoint{1.150000in}{0.150000in}}{\pgfqpoint{5.700000in}{5.700000in}}%
\pgfusepath{clip}%
\pgfsetbuttcap%
\pgfsetroundjoin%
\definecolor{currentfill}{rgb}{0.183898,0.422383,0.556944}%
\pgfsetfillcolor{currentfill}%
\pgfsetfillopacity{0.800000}%
\pgfsetlinewidth{0.000000pt}%
\definecolor{currentstroke}{rgb}{0.000000,0.000000,0.000000}%
\pgfsetstrokecolor{currentstroke}%
\pgfsetdash{}{0pt}%
\pgfpathmoveto{\pgfqpoint{4.708324in}{2.555958in}}%
\pgfpathlineto{\pgfqpoint{4.722548in}{2.566194in}}%
\pgfpathlineto{\pgfqpoint{4.736788in}{2.576615in}}%
\pgfpathlineto{\pgfqpoint{4.751045in}{2.587219in}}%
\pgfpathlineto{\pgfqpoint{4.765319in}{2.598009in}}%
\pgfpathlineto{\pgfqpoint{4.773119in}{2.607394in}}%
\pgfpathlineto{\pgfqpoint{4.780911in}{2.616657in}}%
\pgfpathlineto{\pgfqpoint{4.788697in}{2.625799in}}%
\pgfpathlineto{\pgfqpoint{4.796476in}{2.634822in}}%
\pgfpathlineto{\pgfqpoint{4.782208in}{2.624113in}}%
\pgfpathlineto{\pgfqpoint{4.767957in}{2.613589in}}%
\pgfpathlineto{\pgfqpoint{4.753722in}{2.603249in}}%
\pgfpathlineto{\pgfqpoint{4.739503in}{2.593093in}}%
\pgfpathlineto{\pgfqpoint{4.731719in}{2.583978in}}%
\pgfpathlineto{\pgfqpoint{4.723927in}{2.574751in}}%
\pgfpathlineto{\pgfqpoint{4.716129in}{2.565411in}}%
\pgfpathlineto{\pgfqpoint{4.708324in}{2.555958in}}%
\pgfpathclose%
\pgfusepath{fill}%
\end{pgfscope}%
\begin{pgfscope}%
\pgfpathrectangle{\pgfqpoint{1.150000in}{0.150000in}}{\pgfqpoint{5.700000in}{5.700000in}}%
\pgfusepath{clip}%
\pgfsetbuttcap%
\pgfsetroundjoin%
\definecolor{currentfill}{rgb}{0.221989,0.339161,0.548752}%
\pgfsetfillcolor{currentfill}%
\pgfsetfillopacity{0.800000}%
\pgfsetlinewidth{0.000000pt}%
\definecolor{currentstroke}{rgb}{0.000000,0.000000,0.000000}%
\pgfsetstrokecolor{currentstroke}%
\pgfsetdash{}{0pt}%
\pgfpathmoveto{\pgfqpoint{2.339271in}{2.411334in}}%
\pgfpathlineto{\pgfqpoint{2.353401in}{2.389051in}}%
\pgfpathlineto{\pgfqpoint{2.367517in}{2.367079in}}%
\pgfpathlineto{\pgfqpoint{2.381620in}{2.345416in}}%
\pgfpathlineto{\pgfqpoint{2.395711in}{2.324059in}}%
\pgfpathlineto{\pgfqpoint{2.404712in}{2.321564in}}%
\pgfpathlineto{\pgfqpoint{2.413693in}{2.319381in}}%
\pgfpathlineto{\pgfqpoint{2.422654in}{2.317502in}}%
\pgfpathlineto{\pgfqpoint{2.431596in}{2.315923in}}%
\pgfpathlineto{\pgfqpoint{2.417559in}{2.336676in}}%
\pgfpathlineto{\pgfqpoint{2.403509in}{2.357734in}}%
\pgfpathlineto{\pgfqpoint{2.389447in}{2.379100in}}%
\pgfpathlineto{\pgfqpoint{2.375372in}{2.400775in}}%
\pgfpathlineto{\pgfqpoint{2.366378in}{2.402946in}}%
\pgfpathlineto{\pgfqpoint{2.357363in}{2.405426in}}%
\pgfpathlineto{\pgfqpoint{2.348328in}{2.408220in}}%
\pgfpathlineto{\pgfqpoint{2.339271in}{2.411334in}}%
\pgfpathclose%
\pgfusepath{fill}%
\end{pgfscope}%
\begin{pgfscope}%
\pgfpathrectangle{\pgfqpoint{1.150000in}{0.150000in}}{\pgfqpoint{5.700000in}{5.700000in}}%
\pgfusepath{clip}%
\pgfsetbuttcap%
\pgfsetroundjoin%
\definecolor{currentfill}{rgb}{0.165117,0.467423,0.558141}%
\pgfsetfillcolor{currentfill}%
\pgfsetfillopacity{0.800000}%
\pgfsetlinewidth{0.000000pt}%
\definecolor{currentstroke}{rgb}{0.000000,0.000000,0.000000}%
\pgfsetstrokecolor{currentstroke}%
\pgfsetdash{}{0pt}%
\pgfpathmoveto{\pgfqpoint{2.148165in}{2.792272in}}%
\pgfpathlineto{\pgfqpoint{2.162491in}{2.765206in}}%
\pgfpathlineto{\pgfqpoint{2.176798in}{2.738504in}}%
\pgfpathlineto{\pgfqpoint{2.191087in}{2.712160in}}%
\pgfpathlineto{\pgfqpoint{2.205359in}{2.686172in}}%
\pgfpathlineto{\pgfqpoint{2.214502in}{2.682523in}}%
\pgfpathlineto{\pgfqpoint{2.223623in}{2.679199in}}%
\pgfpathlineto{\pgfqpoint{2.232722in}{2.676194in}}%
\pgfpathlineto{\pgfqpoint{2.241800in}{2.673504in}}%
\pgfpathlineto{\pgfqpoint{2.227587in}{2.698903in}}%
\pgfpathlineto{\pgfqpoint{2.213357in}{2.724656in}}%
\pgfpathlineto{\pgfqpoint{2.199109in}{2.750766in}}%
\pgfpathlineto{\pgfqpoint{2.184844in}{2.777237in}}%
\pgfpathlineto{\pgfqpoint{2.175708in}{2.780505in}}%
\pgfpathlineto{\pgfqpoint{2.166549in}{2.784096in}}%
\pgfpathlineto{\pgfqpoint{2.157368in}{2.788017in}}%
\pgfpathlineto{\pgfqpoint{2.148165in}{2.792272in}}%
\pgfpathclose%
\pgfusepath{fill}%
\end{pgfscope}%
\begin{pgfscope}%
\pgfpathrectangle{\pgfqpoint{1.150000in}{0.150000in}}{\pgfqpoint{5.700000in}{5.700000in}}%
\pgfusepath{clip}%
\pgfsetbuttcap%
\pgfsetroundjoin%
\definecolor{currentfill}{rgb}{0.280267,0.073417,0.397163}%
\pgfsetfillcolor{currentfill}%
\pgfsetfillopacity{0.800000}%
\pgfsetlinewidth{0.000000pt}%
\definecolor{currentstroke}{rgb}{0.000000,0.000000,0.000000}%
\pgfsetstrokecolor{currentstroke}%
\pgfsetdash{}{0pt}%
\pgfpathmoveto{\pgfqpoint{2.896989in}{1.739104in}}%
\pgfpathlineto{\pgfqpoint{2.910807in}{1.727501in}}%
\pgfpathlineto{\pgfqpoint{2.924623in}{1.716128in}}%
\pgfpathlineto{\pgfqpoint{2.938436in}{1.704984in}}%
\pgfpathlineto{\pgfqpoint{2.952247in}{1.694068in}}%
\pgfpathlineto{\pgfqpoint{2.960809in}{1.697541in}}%
\pgfpathlineto{\pgfqpoint{2.969359in}{1.701238in}}%
\pgfpathlineto{\pgfqpoint{2.977896in}{1.705151in}}%
\pgfpathlineto{\pgfqpoint{2.986420in}{1.709275in}}%
\pgfpathlineto{\pgfqpoint{2.972643in}{1.719622in}}%
\pgfpathlineto{\pgfqpoint{2.958864in}{1.730195in}}%
\pgfpathlineto{\pgfqpoint{2.945083in}{1.740998in}}%
\pgfpathlineto{\pgfqpoint{2.931299in}{1.752029in}}%
\pgfpathlineto{\pgfqpoint{2.922741in}{1.748463in}}%
\pgfpathlineto{\pgfqpoint{2.914170in}{1.745116in}}%
\pgfpathlineto{\pgfqpoint{2.905586in}{1.741995in}}%
\pgfpathlineto{\pgfqpoint{2.896989in}{1.739104in}}%
\pgfpathclose%
\pgfusepath{fill}%
\end{pgfscope}%
\begin{pgfscope}%
\pgfpathrectangle{\pgfqpoint{1.150000in}{0.150000in}}{\pgfqpoint{5.700000in}{5.700000in}}%
\pgfusepath{clip}%
\pgfsetbuttcap%
\pgfsetroundjoin%
\definecolor{currentfill}{rgb}{0.121831,0.589055,0.545623}%
\pgfsetfillcolor{currentfill}%
\pgfsetfillopacity{0.800000}%
\pgfsetlinewidth{0.000000pt}%
\definecolor{currentstroke}{rgb}{0.000000,0.000000,0.000000}%
\pgfsetstrokecolor{currentstroke}%
\pgfsetdash{}{0pt}%
\pgfpathmoveto{\pgfqpoint{5.299058in}{3.069363in}}%
\pgfpathlineto{\pgfqpoint{5.313637in}{3.082394in}}%
\pgfpathlineto{\pgfqpoint{5.328236in}{3.095608in}}%
\pgfpathlineto{\pgfqpoint{5.342855in}{3.109005in}}%
\pgfpathlineto{\pgfqpoint{5.357494in}{3.122584in}}%
\pgfpathlineto{\pgfqpoint{5.364976in}{3.127065in}}%
\pgfpathlineto{\pgfqpoint{5.372450in}{3.131456in}}%
\pgfpathlineto{\pgfqpoint{5.379914in}{3.135763in}}%
\pgfpathlineto{\pgfqpoint{5.387371in}{3.139989in}}%
\pgfpathlineto{\pgfqpoint{5.372749in}{3.126768in}}%
\pgfpathlineto{\pgfqpoint{5.358147in}{3.113729in}}%
\pgfpathlineto{\pgfqpoint{5.343565in}{3.100873in}}%
\pgfpathlineto{\pgfqpoint{5.329003in}{3.088198in}}%
\pgfpathlineto{\pgfqpoint{5.321529in}{3.083602in}}%
\pgfpathlineto{\pgfqpoint{5.314047in}{3.078934in}}%
\pgfpathlineto{\pgfqpoint{5.306556in}{3.074189in}}%
\pgfpathlineto{\pgfqpoint{5.299058in}{3.069363in}}%
\pgfpathclose%
\pgfusepath{fill}%
\end{pgfscope}%
\begin{pgfscope}%
\pgfpathrectangle{\pgfqpoint{1.150000in}{0.150000in}}{\pgfqpoint{5.700000in}{5.700000in}}%
\pgfusepath{clip}%
\pgfsetbuttcap%
\pgfsetroundjoin%
\definecolor{currentfill}{rgb}{0.216210,0.351535,0.550627}%
\pgfsetfillcolor{currentfill}%
\pgfsetfillopacity{0.800000}%
\pgfsetlinewidth{0.000000pt}%
\definecolor{currentstroke}{rgb}{0.000000,0.000000,0.000000}%
\pgfsetstrokecolor{currentstroke}%
\pgfsetdash{}{0pt}%
\pgfpathmoveto{\pgfqpoint{4.500835in}{2.357731in}}%
\pgfpathlineto{\pgfqpoint{4.514946in}{2.366529in}}%
\pgfpathlineto{\pgfqpoint{4.529073in}{2.375513in}}%
\pgfpathlineto{\pgfqpoint{4.543214in}{2.384681in}}%
\pgfpathlineto{\pgfqpoint{4.557370in}{2.394034in}}%
\pgfpathlineto{\pgfqpoint{4.565256in}{2.404888in}}%
\pgfpathlineto{\pgfqpoint{4.573136in}{2.415628in}}%
\pgfpathlineto{\pgfqpoint{4.581009in}{2.426254in}}%
\pgfpathlineto{\pgfqpoint{4.588876in}{2.436767in}}%
\pgfpathlineto{\pgfqpoint{4.574723in}{2.427394in}}%
\pgfpathlineto{\pgfqpoint{4.560585in}{2.418206in}}%
\pgfpathlineto{\pgfqpoint{4.546462in}{2.409202in}}%
\pgfpathlineto{\pgfqpoint{4.532354in}{2.400383in}}%
\pgfpathlineto{\pgfqpoint{4.524483in}{2.389878in}}%
\pgfpathlineto{\pgfqpoint{4.516606in}{2.379268in}}%
\pgfpathlineto{\pgfqpoint{4.508723in}{2.368552in}}%
\pgfpathlineto{\pgfqpoint{4.500835in}{2.357731in}}%
\pgfpathclose%
\pgfusepath{fill}%
\end{pgfscope}%
\begin{pgfscope}%
\pgfpathrectangle{\pgfqpoint{1.150000in}{0.150000in}}{\pgfqpoint{5.700000in}{5.700000in}}%
\pgfusepath{clip}%
\pgfsetbuttcap%
\pgfsetroundjoin%
\definecolor{currentfill}{rgb}{0.202219,0.715272,0.476084}%
\pgfsetfillcolor{currentfill}%
\pgfsetfillopacity{0.800000}%
\pgfsetlinewidth{0.000000pt}%
\definecolor{currentstroke}{rgb}{0.000000,0.000000,0.000000}%
\pgfsetstrokecolor{currentstroke}%
\pgfsetdash{}{0pt}%
\pgfpathmoveto{\pgfqpoint{5.857970in}{3.474673in}}%
\pgfpathlineto{\pgfqpoint{5.872887in}{3.488819in}}%
\pgfpathlineto{\pgfqpoint{5.887826in}{3.503145in}}%
\pgfpathlineto{\pgfqpoint{5.902788in}{3.517651in}}%
\pgfpathlineto{\pgfqpoint{5.917773in}{3.532339in}}%
\pgfpathlineto{\pgfqpoint{5.924874in}{3.532644in}}%
\pgfpathlineto{\pgfqpoint{5.931969in}{3.532958in}}%
\pgfpathlineto{\pgfqpoint{5.939057in}{3.533288in}}%
\pgfpathlineto{\pgfqpoint{5.946138in}{3.533641in}}%
\pgfpathlineto{\pgfqpoint{5.931188in}{3.519558in}}%
\pgfpathlineto{\pgfqpoint{5.916260in}{3.505655in}}%
\pgfpathlineto{\pgfqpoint{5.901354in}{3.491931in}}%
\pgfpathlineto{\pgfqpoint{5.886470in}{3.478387in}}%
\pgfpathlineto{\pgfqpoint{5.879355in}{3.477420in}}%
\pgfpathlineto{\pgfqpoint{5.872233in}{3.476483in}}%
\pgfpathlineto{\pgfqpoint{5.865105in}{3.475570in}}%
\pgfpathlineto{\pgfqpoint{5.857970in}{3.474673in}}%
\pgfpathclose%
\pgfusepath{fill}%
\end{pgfscope}%
\begin{pgfscope}%
\pgfpathrectangle{\pgfqpoint{1.150000in}{0.150000in}}{\pgfqpoint{5.700000in}{5.700000in}}%
\pgfusepath{clip}%
\pgfsetbuttcap%
\pgfsetroundjoin%
\definecolor{currentfill}{rgb}{0.248629,0.278775,0.534556}%
\pgfsetfillcolor{currentfill}%
\pgfsetfillopacity{0.800000}%
\pgfsetlinewidth{0.000000pt}%
\definecolor{currentstroke}{rgb}{0.000000,0.000000,0.000000}%
\pgfsetstrokecolor{currentstroke}%
\pgfsetdash{}{0pt}%
\pgfpathmoveto{\pgfqpoint{4.293284in}{2.158046in}}%
\pgfpathlineto{\pgfqpoint{4.307292in}{2.165102in}}%
\pgfpathlineto{\pgfqpoint{4.321313in}{2.172344in}}%
\pgfpathlineto{\pgfqpoint{4.335347in}{2.179771in}}%
\pgfpathlineto{\pgfqpoint{4.349395in}{2.187384in}}%
\pgfpathlineto{\pgfqpoint{4.357353in}{2.199325in}}%
\pgfpathlineto{\pgfqpoint{4.365306in}{2.211172in}}%
\pgfpathlineto{\pgfqpoint{4.373253in}{2.222926in}}%
\pgfpathlineto{\pgfqpoint{4.381195in}{2.234585in}}%
\pgfpathlineto{\pgfqpoint{4.367150in}{2.226853in}}%
\pgfpathlineto{\pgfqpoint{4.353118in}{2.219306in}}%
\pgfpathlineto{\pgfqpoint{4.339100in}{2.211945in}}%
\pgfpathlineto{\pgfqpoint{4.325094in}{2.204769in}}%
\pgfpathlineto{\pgfqpoint{4.317150in}{2.193217in}}%
\pgfpathlineto{\pgfqpoint{4.309200in}{2.181579in}}%
\pgfpathlineto{\pgfqpoint{4.301245in}{2.169855in}}%
\pgfpathlineto{\pgfqpoint{4.293284in}{2.158046in}}%
\pgfpathclose%
\pgfusepath{fill}%
\end{pgfscope}%
\begin{pgfscope}%
\pgfpathrectangle{\pgfqpoint{1.150000in}{0.150000in}}{\pgfqpoint{5.700000in}{5.700000in}}%
\pgfusepath{clip}%
\pgfsetbuttcap%
\pgfsetroundjoin%
\definecolor{currentfill}{rgb}{0.149039,0.508051,0.557250}%
\pgfsetfillcolor{currentfill}%
\pgfsetfillopacity{0.800000}%
\pgfsetlinewidth{0.000000pt}%
\definecolor{currentstroke}{rgb}{0.000000,0.000000,0.000000}%
\pgfsetstrokecolor{currentstroke}%
\pgfsetdash{}{0pt}%
\pgfpathmoveto{\pgfqpoint{5.003848in}{2.823078in}}%
\pgfpathlineto{\pgfqpoint{5.018251in}{2.834993in}}%
\pgfpathlineto{\pgfqpoint{5.032672in}{2.847092in}}%
\pgfpathlineto{\pgfqpoint{5.047111in}{2.859375in}}%
\pgfpathlineto{\pgfqpoint{5.061568in}{2.871842in}}%
\pgfpathlineto{\pgfqpoint{5.069225in}{2.878856in}}%
\pgfpathlineto{\pgfqpoint{5.076874in}{2.885754in}}%
\pgfpathlineto{\pgfqpoint{5.084515in}{2.892537in}}%
\pgfpathlineto{\pgfqpoint{5.092148in}{2.899208in}}%
\pgfpathlineto{\pgfqpoint{5.077700in}{2.886961in}}%
\pgfpathlineto{\pgfqpoint{5.063272in}{2.874896in}}%
\pgfpathlineto{\pgfqpoint{5.048861in}{2.863015in}}%
\pgfpathlineto{\pgfqpoint{5.034469in}{2.851316in}}%
\pgfpathlineto{\pgfqpoint{5.026826in}{2.844414in}}%
\pgfpathlineto{\pgfqpoint{5.019174in}{2.837409in}}%
\pgfpathlineto{\pgfqpoint{5.011515in}{2.830298in}}%
\pgfpathlineto{\pgfqpoint{5.003848in}{2.823078in}}%
\pgfpathclose%
\pgfusepath{fill}%
\end{pgfscope}%
\begin{pgfscope}%
\pgfpathrectangle{\pgfqpoint{1.150000in}{0.150000in}}{\pgfqpoint{5.700000in}{5.700000in}}%
\pgfusepath{clip}%
\pgfsetbuttcap%
\pgfsetroundjoin%
\definecolor{currentfill}{rgb}{0.266941,0.748751,0.440573}%
\pgfsetfillcolor{currentfill}%
\pgfsetfillopacity{0.800000}%
\pgfsetlinewidth{0.000000pt}%
\definecolor{currentstroke}{rgb}{0.000000,0.000000,0.000000}%
\pgfsetstrokecolor{currentstroke}%
\pgfsetdash{}{0pt}%
\pgfpathmoveto{\pgfqpoint{6.034285in}{3.590987in}}%
\pgfpathlineto{\pgfqpoint{6.049312in}{3.605330in}}%
\pgfpathlineto{\pgfqpoint{6.064362in}{3.619852in}}%
\pgfpathlineto{\pgfqpoint{6.079435in}{3.634555in}}%
\pgfpathlineto{\pgfqpoint{6.086422in}{3.633967in}}%
\pgfpathlineto{\pgfqpoint{6.093402in}{3.633429in}}%
\pgfpathlineto{\pgfqpoint{6.100378in}{3.632948in}}%
\pgfpathlineto{\pgfqpoint{6.107349in}{3.632532in}}%
\pgfpathlineto{\pgfqpoint{6.092315in}{3.618503in}}%
\pgfpathlineto{\pgfqpoint{6.077305in}{3.604653in}}%
\pgfpathlineto{\pgfqpoint{6.062317in}{3.590981in}}%
\pgfpathlineto{\pgfqpoint{6.055316in}{3.590885in}}%
\pgfpathlineto{\pgfqpoint{6.048311in}{3.590859in}}%
\pgfpathlineto{\pgfqpoint{6.041301in}{3.590895in}}%
\pgfpathlineto{\pgfqpoint{6.034285in}{3.590987in}}%
\pgfpathclose%
\pgfusepath{fill}%
\end{pgfscope}%
\begin{pgfscope}%
\pgfpathrectangle{\pgfqpoint{1.150000in}{0.150000in}}{\pgfqpoint{5.700000in}{5.700000in}}%
\pgfusepath{clip}%
\pgfsetbuttcap%
\pgfsetroundjoin%
\definecolor{currentfill}{rgb}{0.274128,0.199721,0.498911}%
\pgfsetfillcolor{currentfill}%
\pgfsetfillopacity{0.800000}%
\pgfsetlinewidth{0.000000pt}%
\definecolor{currentstroke}{rgb}{0.000000,0.000000,0.000000}%
\pgfsetstrokecolor{currentstroke}%
\pgfsetdash{}{0pt}%
\pgfpathmoveto{\pgfqpoint{4.085691in}{1.964784in}}%
\pgfpathlineto{\pgfqpoint{4.099609in}{1.969799in}}%
\pgfpathlineto{\pgfqpoint{4.113539in}{1.975000in}}%
\pgfpathlineto{\pgfqpoint{4.127481in}{1.980387in}}%
\pgfpathlineto{\pgfqpoint{4.141434in}{1.985961in}}%
\pgfpathlineto{\pgfqpoint{4.149456in}{1.998467in}}%
\pgfpathlineto{\pgfqpoint{4.157474in}{2.010913in}}%
\pgfpathlineto{\pgfqpoint{4.165487in}{2.023297in}}%
\pgfpathlineto{\pgfqpoint{4.173495in}{2.035616in}}%
\pgfpathlineto{\pgfqpoint{4.159545in}{2.029826in}}%
\pgfpathlineto{\pgfqpoint{4.145606in}{2.024222in}}%
\pgfpathlineto{\pgfqpoint{4.131680in}{2.018805in}}%
\pgfpathlineto{\pgfqpoint{4.117765in}{2.013575in}}%
\pgfpathlineto{\pgfqpoint{4.109753in}{2.001460in}}%
\pgfpathlineto{\pgfqpoint{4.101737in}{1.989288in}}%
\pgfpathlineto{\pgfqpoint{4.093716in}{1.977063in}}%
\pgfpathlineto{\pgfqpoint{4.085691in}{1.964784in}}%
\pgfpathclose%
\pgfusepath{fill}%
\end{pgfscope}%
\begin{pgfscope}%
\pgfpathrectangle{\pgfqpoint{1.150000in}{0.150000in}}{\pgfqpoint{5.700000in}{5.700000in}}%
\pgfusepath{clip}%
\pgfsetbuttcap%
\pgfsetroundjoin%
\definecolor{currentfill}{rgb}{0.239374,0.735588,0.455688}%
\pgfsetfillcolor{currentfill}%
\pgfsetfillopacity{0.800000}%
\pgfsetlinewidth{0.000000pt}%
\definecolor{currentstroke}{rgb}{0.000000,0.000000,0.000000}%
\pgfsetstrokecolor{currentstroke}%
\pgfsetdash{}{0pt}%
\pgfpathmoveto{\pgfqpoint{5.946138in}{3.533641in}}%
\pgfpathlineto{\pgfqpoint{5.961111in}{3.547903in}}%
\pgfpathlineto{\pgfqpoint{5.976106in}{3.562346in}}%
\pgfpathlineto{\pgfqpoint{5.991124in}{3.576969in}}%
\pgfpathlineto{\pgfqpoint{6.006166in}{3.591773in}}%
\pgfpathlineto{\pgfqpoint{6.013205in}{3.591527in}}%
\pgfpathlineto{\pgfqpoint{6.020238in}{3.591310in}}%
\pgfpathlineto{\pgfqpoint{6.027264in}{3.591128in}}%
\pgfpathlineto{\pgfqpoint{6.034285in}{3.590987in}}%
\pgfpathlineto{\pgfqpoint{6.019281in}{3.576823in}}%
\pgfpathlineto{\pgfqpoint{6.004300in}{3.562839in}}%
\pgfpathlineto{\pgfqpoint{5.989341in}{3.549034in}}%
\pgfpathlineto{\pgfqpoint{5.974404in}{3.535408in}}%
\pgfpathlineto{\pgfqpoint{5.967346in}{3.534899in}}%
\pgfpathlineto{\pgfqpoint{5.960282in}{3.534440in}}%
\pgfpathlineto{\pgfqpoint{5.953213in}{3.534022in}}%
\pgfpathlineto{\pgfqpoint{5.946138in}{3.533641in}}%
\pgfpathclose%
\pgfusepath{fill}%
\end{pgfscope}%
\begin{pgfscope}%
\pgfpathrectangle{\pgfqpoint{1.150000in}{0.150000in}}{\pgfqpoint{5.700000in}{5.700000in}}%
\pgfusepath{clip}%
\pgfsetbuttcap%
\pgfsetroundjoin%
\definecolor{currentfill}{rgb}{0.204903,0.375746,0.553533}%
\pgfsetfillcolor{currentfill}%
\pgfsetfillopacity{0.800000}%
\pgfsetlinewidth{0.000000pt}%
\definecolor{currentstroke}{rgb}{0.000000,0.000000,0.000000}%
\pgfsetstrokecolor{currentstroke}%
\pgfsetdash{}{0pt}%
\pgfpathmoveto{\pgfqpoint{2.282614in}{2.503641in}}%
\pgfpathlineto{\pgfqpoint{2.296800in}{2.480082in}}%
\pgfpathlineto{\pgfqpoint{2.310971in}{2.456847in}}%
\pgfpathlineto{\pgfqpoint{2.325128in}{2.433932in}}%
\pgfpathlineto{\pgfqpoint{2.339271in}{2.411334in}}%
\pgfpathlineto{\pgfqpoint{2.348328in}{2.408220in}}%
\pgfpathlineto{\pgfqpoint{2.357363in}{2.405426in}}%
\pgfpathlineto{\pgfqpoint{2.366378in}{2.402946in}}%
\pgfpathlineto{\pgfqpoint{2.375372in}{2.400775in}}%
\pgfpathlineto{\pgfqpoint{2.361284in}{2.422764in}}%
\pgfpathlineto{\pgfqpoint{2.347183in}{2.445068in}}%
\pgfpathlineto{\pgfqpoint{2.333068in}{2.467692in}}%
\pgfpathlineto{\pgfqpoint{2.318939in}{2.490637in}}%
\pgfpathlineto{\pgfqpoint{2.309890in}{2.493405in}}%
\pgfpathlineto{\pgfqpoint{2.300820in}{2.496491in}}%
\pgfpathlineto{\pgfqpoint{2.291728in}{2.499901in}}%
\pgfpathlineto{\pgfqpoint{2.282614in}{2.503641in}}%
\pgfpathclose%
\pgfusepath{fill}%
\end{pgfscope}%
\begin{pgfscope}%
\pgfpathrectangle{\pgfqpoint{1.150000in}{0.150000in}}{\pgfqpoint{5.700000in}{5.700000in}}%
\pgfusepath{clip}%
\pgfsetbuttcap%
\pgfsetroundjoin%
\definecolor{currentfill}{rgb}{0.279566,0.067836,0.391917}%
\pgfsetfillcolor{currentfill}%
\pgfsetfillopacity{0.800000}%
\pgfsetlinewidth{0.000000pt}%
\definecolor{currentstroke}{rgb}{0.000000,0.000000,0.000000}%
\pgfsetstrokecolor{currentstroke}%
\pgfsetdash{}{0pt}%
\pgfpathmoveto{\pgfqpoint{3.702407in}{1.681941in}}%
\pgfpathlineto{\pgfqpoint{3.716207in}{1.682380in}}%
\pgfpathlineto{\pgfqpoint{3.730013in}{1.683010in}}%
\pgfpathlineto{\pgfqpoint{3.743828in}{1.683831in}}%
\pgfpathlineto{\pgfqpoint{3.757650in}{1.684843in}}%
\pgfpathlineto{\pgfqpoint{3.765791in}{1.696506in}}%
\pgfpathlineto{\pgfqpoint{3.773926in}{1.708194in}}%
\pgfpathlineto{\pgfqpoint{3.782056in}{1.719903in}}%
\pgfpathlineto{\pgfqpoint{3.790181in}{1.731630in}}%
\pgfpathlineto{\pgfqpoint{3.776367in}{1.730245in}}%
\pgfpathlineto{\pgfqpoint{3.762562in}{1.729051in}}%
\pgfpathlineto{\pgfqpoint{3.748764in}{1.728048in}}%
\pgfpathlineto{\pgfqpoint{3.734974in}{1.727236in}}%
\pgfpathlineto{\pgfqpoint{3.726841in}{1.715870in}}%
\pgfpathlineto{\pgfqpoint{3.718702in}{1.704530in}}%
\pgfpathlineto{\pgfqpoint{3.710557in}{1.693219in}}%
\pgfpathlineto{\pgfqpoint{3.702407in}{1.681941in}}%
\pgfpathclose%
\pgfusepath{fill}%
\end{pgfscope}%
\begin{pgfscope}%
\pgfpathrectangle{\pgfqpoint{1.150000in}{0.150000in}}{\pgfqpoint{5.700000in}{5.700000in}}%
\pgfusepath{clip}%
\pgfsetbuttcap%
\pgfsetroundjoin%
\definecolor{currentfill}{rgb}{0.277941,0.056324,0.381191}%
\pgfsetfillcolor{currentfill}%
\pgfsetfillopacity{0.800000}%
\pgfsetlinewidth{0.000000pt}%
\definecolor{currentstroke}{rgb}{0.000000,0.000000,0.000000}%
\pgfsetstrokecolor{currentstroke}%
\pgfsetdash{}{0pt}%
\pgfpathmoveto{\pgfqpoint{2.952247in}{1.694068in}}%
\pgfpathlineto{\pgfqpoint{2.966056in}{1.683377in}}%
\pgfpathlineto{\pgfqpoint{2.979863in}{1.672912in}}%
\pgfpathlineto{\pgfqpoint{2.993668in}{1.662671in}}%
\pgfpathlineto{\pgfqpoint{3.007471in}{1.652651in}}%
\pgfpathlineto{\pgfqpoint{3.016000in}{1.656707in}}%
\pgfpathlineto{\pgfqpoint{3.024516in}{1.660976in}}%
\pgfpathlineto{\pgfqpoint{3.033021in}{1.665454in}}%
\pgfpathlineto{\pgfqpoint{3.041514in}{1.670134in}}%
\pgfpathlineto{\pgfqpoint{3.027742in}{1.679585in}}%
\pgfpathlineto{\pgfqpoint{3.013970in}{1.689258in}}%
\pgfpathlineto{\pgfqpoint{3.000196in}{1.699154in}}%
\pgfpathlineto{\pgfqpoint{2.986420in}{1.709275in}}%
\pgfpathlineto{\pgfqpoint{2.977896in}{1.705151in}}%
\pgfpathlineto{\pgfqpoint{2.969359in}{1.701238in}}%
\pgfpathlineto{\pgfqpoint{2.960809in}{1.697541in}}%
\pgfpathlineto{\pgfqpoint{2.952247in}{1.694068in}}%
\pgfpathclose%
\pgfusepath{fill}%
\end{pgfscope}%
\begin{pgfscope}%
\pgfpathrectangle{\pgfqpoint{1.150000in}{0.150000in}}{\pgfqpoint{5.700000in}{5.700000in}}%
\pgfusepath{clip}%
\pgfsetbuttcap%
\pgfsetroundjoin%
\definecolor{currentfill}{rgb}{0.276022,0.044167,0.370164}%
\pgfsetfillcolor{currentfill}%
\pgfsetfillopacity{0.800000}%
\pgfsetlinewidth{0.000000pt}%
\definecolor{currentstroke}{rgb}{0.000000,0.000000,0.000000}%
\pgfsetstrokecolor{currentstroke}%
\pgfsetdash{}{0pt}%
\pgfpathmoveto{\pgfqpoint{3.614582in}{1.639009in}}%
\pgfpathlineto{\pgfqpoint{3.628365in}{1.638274in}}%
\pgfpathlineto{\pgfqpoint{3.642154in}{1.637732in}}%
\pgfpathlineto{\pgfqpoint{3.655950in}{1.637383in}}%
\pgfpathlineto{\pgfqpoint{3.669752in}{1.637226in}}%
\pgfpathlineto{\pgfqpoint{3.677925in}{1.648337in}}%
\pgfpathlineto{\pgfqpoint{3.686091in}{1.659496in}}%
\pgfpathlineto{\pgfqpoint{3.694252in}{1.670698in}}%
\pgfpathlineto{\pgfqpoint{3.702407in}{1.681941in}}%
\pgfpathlineto{\pgfqpoint{3.688615in}{1.681693in}}%
\pgfpathlineto{\pgfqpoint{3.674831in}{1.681638in}}%
\pgfpathlineto{\pgfqpoint{3.661053in}{1.681776in}}%
\pgfpathlineto{\pgfqpoint{3.647282in}{1.682107in}}%
\pgfpathlineto{\pgfqpoint{3.639116in}{1.671257in}}%
\pgfpathlineto{\pgfqpoint{3.630944in}{1.660455in}}%
\pgfpathlineto{\pgfqpoint{3.622766in}{1.649705in}}%
\pgfpathlineto{\pgfqpoint{3.614582in}{1.639009in}}%
\pgfpathclose%
\pgfusepath{fill}%
\end{pgfscope}%
\begin{pgfscope}%
\pgfpathrectangle{\pgfqpoint{1.150000in}{0.150000in}}{\pgfqpoint{5.700000in}{5.700000in}}%
\pgfusepath{clip}%
\pgfsetbuttcap%
\pgfsetroundjoin%
\definecolor{currentfill}{rgb}{0.119483,0.614817,0.537692}%
\pgfsetfillcolor{currentfill}%
\pgfsetfillopacity{0.800000}%
\pgfsetlinewidth{0.000000pt}%
\definecolor{currentstroke}{rgb}{0.000000,0.000000,0.000000}%
\pgfsetstrokecolor{currentstroke}%
\pgfsetdash{}{0pt}%
\pgfpathmoveto{\pgfqpoint{5.387371in}{3.139989in}}%
\pgfpathlineto{\pgfqpoint{5.402013in}{3.153392in}}%
\pgfpathlineto{\pgfqpoint{5.416675in}{3.166978in}}%
\pgfpathlineto{\pgfqpoint{5.431358in}{3.180746in}}%
\pgfpathlineto{\pgfqpoint{5.446061in}{3.194698in}}%
\pgfpathlineto{\pgfqpoint{5.453490in}{3.198466in}}%
\pgfpathlineto{\pgfqpoint{5.460911in}{3.202154in}}%
\pgfpathlineto{\pgfqpoint{5.468322in}{3.205766in}}%
\pgfpathlineto{\pgfqpoint{5.475726in}{3.209306in}}%
\pgfpathlineto{\pgfqpoint{5.461042in}{3.195750in}}%
\pgfpathlineto{\pgfqpoint{5.446379in}{3.182375in}}%
\pgfpathlineto{\pgfqpoint{5.431736in}{3.169182in}}%
\pgfpathlineto{\pgfqpoint{5.417113in}{3.156171in}}%
\pgfpathlineto{\pgfqpoint{5.409690in}{3.152225in}}%
\pgfpathlineto{\pgfqpoint{5.402258in}{3.148216in}}%
\pgfpathlineto{\pgfqpoint{5.394819in}{3.144138in}}%
\pgfpathlineto{\pgfqpoint{5.387371in}{3.139989in}}%
\pgfpathclose%
\pgfusepath{fill}%
\end{pgfscope}%
\begin{pgfscope}%
\pgfpathrectangle{\pgfqpoint{1.150000in}{0.150000in}}{\pgfqpoint{5.700000in}{5.700000in}}%
\pgfusepath{clip}%
\pgfsetbuttcap%
\pgfsetroundjoin%
\definecolor{currentfill}{rgb}{0.267004,0.004874,0.329415}%
\pgfsetfillcolor{currentfill}%
\pgfsetfillopacity{0.800000}%
\pgfsetlinewidth{0.000000pt}%
\definecolor{currentstroke}{rgb}{0.000000,0.000000,0.000000}%
\pgfsetstrokecolor{currentstroke}%
\pgfsetdash{}{0pt}%
\pgfpathmoveto{\pgfqpoint{3.295307in}{1.577383in}}%
\pgfpathlineto{\pgfqpoint{3.309066in}{1.572073in}}%
\pgfpathlineto{\pgfqpoint{3.322829in}{1.566968in}}%
\pgfpathlineto{\pgfqpoint{3.336594in}{1.562065in}}%
\pgfpathlineto{\pgfqpoint{3.350362in}{1.557365in}}%
\pgfpathlineto{\pgfqpoint{3.358678in}{1.565540in}}%
\pgfpathlineto{\pgfqpoint{3.366986in}{1.573846in}}%
\pgfpathlineto{\pgfqpoint{3.375285in}{1.582279in}}%
\pgfpathlineto{\pgfqpoint{3.383577in}{1.590834in}}%
\pgfpathlineto{\pgfqpoint{3.369829in}{1.595035in}}%
\pgfpathlineto{\pgfqpoint{3.356084in}{1.599439in}}%
\pgfpathlineto{\pgfqpoint{3.342342in}{1.604046in}}%
\pgfpathlineto{\pgfqpoint{3.328603in}{1.608857in}}%
\pgfpathlineto{\pgfqpoint{3.320292in}{1.600789in}}%
\pgfpathlineto{\pgfqpoint{3.311972in}{1.592851in}}%
\pgfpathlineto{\pgfqpoint{3.303644in}{1.585047in}}%
\pgfpathlineto{\pgfqpoint{3.295307in}{1.577383in}}%
\pgfpathclose%
\pgfusepath{fill}%
\end{pgfscope}%
\begin{pgfscope}%
\pgfpathrectangle{\pgfqpoint{1.150000in}{0.150000in}}{\pgfqpoint{5.700000in}{5.700000in}}%
\pgfusepath{clip}%
\pgfsetbuttcap%
\pgfsetroundjoin%
\definecolor{currentfill}{rgb}{0.269944,0.014625,0.341379}%
\pgfsetfillcolor{currentfill}%
\pgfsetfillopacity{0.800000}%
\pgfsetlinewidth{0.000000pt}%
\definecolor{currentstroke}{rgb}{0.000000,0.000000,0.000000}%
\pgfsetstrokecolor{currentstroke}%
\pgfsetdash{}{0pt}%
\pgfpathmoveto{\pgfqpoint{3.151670in}{1.602383in}}%
\pgfpathlineto{\pgfqpoint{3.165441in}{1.594880in}}%
\pgfpathlineto{\pgfqpoint{3.179212in}{1.587588in}}%
\pgfpathlineto{\pgfqpoint{3.192985in}{1.580506in}}%
\pgfpathlineto{\pgfqpoint{3.206759in}{1.573633in}}%
\pgfpathlineto{\pgfqpoint{3.215156in}{1.580147in}}%
\pgfpathlineto{\pgfqpoint{3.223544in}{1.586829in}}%
\pgfpathlineto{\pgfqpoint{3.231922in}{1.593673in}}%
\pgfpathlineto{\pgfqpoint{3.240291in}{1.600674in}}%
\pgfpathlineto{\pgfqpoint{3.226543in}{1.607015in}}%
\pgfpathlineto{\pgfqpoint{3.212795in}{1.613565in}}%
\pgfpathlineto{\pgfqpoint{3.199050in}{1.620325in}}%
\pgfpathlineto{\pgfqpoint{3.185305in}{1.627296in}}%
\pgfpathlineto{\pgfqpoint{3.176911in}{1.620815in}}%
\pgfpathlineto{\pgfqpoint{3.168508in}{1.614499in}}%
\pgfpathlineto{\pgfqpoint{3.160094in}{1.608353in}}%
\pgfpathlineto{\pgfqpoint{3.151670in}{1.602383in}}%
\pgfpathclose%
\pgfusepath{fill}%
\end{pgfscope}%
\begin{pgfscope}%
\pgfpathrectangle{\pgfqpoint{1.150000in}{0.150000in}}{\pgfqpoint{5.700000in}{5.700000in}}%
\pgfusepath{clip}%
\pgfsetbuttcap%
\pgfsetroundjoin%
\definecolor{currentfill}{rgb}{0.171176,0.452530,0.557965}%
\pgfsetfillcolor{currentfill}%
\pgfsetfillopacity{0.800000}%
\pgfsetlinewidth{0.000000pt}%
\definecolor{currentstroke}{rgb}{0.000000,0.000000,0.000000}%
\pgfsetstrokecolor{currentstroke}%
\pgfsetdash{}{0pt}%
\pgfpathmoveto{\pgfqpoint{4.796476in}{2.634822in}}%
\pgfpathlineto{\pgfqpoint{4.810761in}{2.645714in}}%
\pgfpathlineto{\pgfqpoint{4.825063in}{2.656791in}}%
\pgfpathlineto{\pgfqpoint{4.839382in}{2.668052in}}%
\pgfpathlineto{\pgfqpoint{4.853719in}{2.679497in}}%
\pgfpathlineto{\pgfqpoint{4.861484in}{2.688298in}}%
\pgfpathlineto{\pgfqpoint{4.869243in}{2.696975in}}%
\pgfpathlineto{\pgfqpoint{4.876994in}{2.705528in}}%
\pgfpathlineto{\pgfqpoint{4.884737in}{2.713959in}}%
\pgfpathlineto{\pgfqpoint{4.870407in}{2.702629in}}%
\pgfpathlineto{\pgfqpoint{4.856095in}{2.691483in}}%
\pgfpathlineto{\pgfqpoint{4.841799in}{2.680521in}}%
\pgfpathlineto{\pgfqpoint{4.827521in}{2.669743in}}%
\pgfpathlineto{\pgfqpoint{4.819770in}{2.661185in}}%
\pgfpathlineto{\pgfqpoint{4.812013in}{2.652513in}}%
\pgfpathlineto{\pgfqpoint{4.804248in}{2.643726in}}%
\pgfpathlineto{\pgfqpoint{4.796476in}{2.634822in}}%
\pgfpathclose%
\pgfusepath{fill}%
\end{pgfscope}%
\begin{pgfscope}%
\pgfpathrectangle{\pgfqpoint{1.150000in}{0.150000in}}{\pgfqpoint{5.700000in}{5.700000in}}%
\pgfusepath{clip}%
\pgfsetbuttcap%
\pgfsetroundjoin%
\definecolor{currentfill}{rgb}{0.282327,0.094955,0.417331}%
\pgfsetfillcolor{currentfill}%
\pgfsetfillopacity{0.800000}%
\pgfsetlinewidth{0.000000pt}%
\definecolor{currentstroke}{rgb}{0.000000,0.000000,0.000000}%
\pgfsetstrokecolor{currentstroke}%
\pgfsetdash{}{0pt}%
\pgfpathmoveto{\pgfqpoint{3.790181in}{1.731630in}}%
\pgfpathlineto{\pgfqpoint{3.804003in}{1.733206in}}%
\pgfpathlineto{\pgfqpoint{3.817834in}{1.734970in}}%
\pgfpathlineto{\pgfqpoint{3.831673in}{1.736925in}}%
\pgfpathlineto{\pgfqpoint{3.845520in}{1.739068in}}%
\pgfpathlineto{\pgfqpoint{3.853633in}{1.751164in}}%
\pgfpathlineto{\pgfqpoint{3.861740in}{1.763263in}}%
\pgfpathlineto{\pgfqpoint{3.869843in}{1.775363in}}%
\pgfpathlineto{\pgfqpoint{3.877940in}{1.787460in}}%
\pgfpathlineto{\pgfqpoint{3.864099in}{1.784974in}}%
\pgfpathlineto{\pgfqpoint{3.850267in}{1.782678in}}%
\pgfpathlineto{\pgfqpoint{3.836444in}{1.780571in}}%
\pgfpathlineto{\pgfqpoint{3.822630in}{1.778654in}}%
\pgfpathlineto{\pgfqpoint{3.814525in}{1.766887in}}%
\pgfpathlineto{\pgfqpoint{3.806416in}{1.755125in}}%
\pgfpathlineto{\pgfqpoint{3.798301in}{1.743372in}}%
\pgfpathlineto{\pgfqpoint{3.790181in}{1.731630in}}%
\pgfpathclose%
\pgfusepath{fill}%
\end{pgfscope}%
\begin{pgfscope}%
\pgfpathrectangle{\pgfqpoint{1.150000in}{0.150000in}}{\pgfqpoint{5.700000in}{5.700000in}}%
\pgfusepath{clip}%
\pgfsetbuttcap%
\pgfsetroundjoin%
\definecolor{currentfill}{rgb}{0.272594,0.025563,0.353093}%
\pgfsetfillcolor{currentfill}%
\pgfsetfillopacity{0.800000}%
\pgfsetlinewidth{0.000000pt}%
\definecolor{currentstroke}{rgb}{0.000000,0.000000,0.000000}%
\pgfsetstrokecolor{currentstroke}%
\pgfsetdash{}{0pt}%
\pgfpathmoveto{\pgfqpoint{3.526665in}{1.603482in}}%
\pgfpathlineto{\pgfqpoint{3.540437in}{1.601534in}}%
\pgfpathlineto{\pgfqpoint{3.554214in}{1.599781in}}%
\pgfpathlineto{\pgfqpoint{3.567998in}{1.598222in}}%
\pgfpathlineto{\pgfqpoint{3.581787in}{1.596858in}}%
\pgfpathlineto{\pgfqpoint{3.589995in}{1.607293in}}%
\pgfpathlineto{\pgfqpoint{3.598197in}{1.617800in}}%
\pgfpathlineto{\pgfqpoint{3.606393in}{1.628373in}}%
\pgfpathlineto{\pgfqpoint{3.614582in}{1.639009in}}%
\pgfpathlineto{\pgfqpoint{3.600806in}{1.639938in}}%
\pgfpathlineto{\pgfqpoint{3.587036in}{1.641061in}}%
\pgfpathlineto{\pgfqpoint{3.573272in}{1.642379in}}%
\pgfpathlineto{\pgfqpoint{3.559514in}{1.643892in}}%
\pgfpathlineto{\pgfqpoint{3.551311in}{1.633679in}}%
\pgfpathlineto{\pgfqpoint{3.543102in}{1.623537in}}%
\pgfpathlineto{\pgfqpoint{3.534887in}{1.613470in}}%
\pgfpathlineto{\pgfqpoint{3.526665in}{1.603482in}}%
\pgfpathclose%
\pgfusepath{fill}%
\end{pgfscope}%
\begin{pgfscope}%
\pgfpathrectangle{\pgfqpoint{1.150000in}{0.150000in}}{\pgfqpoint{5.700000in}{5.700000in}}%
\pgfusepath{clip}%
\pgfsetbuttcap%
\pgfsetroundjoin%
\definecolor{currentfill}{rgb}{0.201239,0.383670,0.554294}%
\pgfsetfillcolor{currentfill}%
\pgfsetfillopacity{0.800000}%
\pgfsetlinewidth{0.000000pt}%
\definecolor{currentstroke}{rgb}{0.000000,0.000000,0.000000}%
\pgfsetstrokecolor{currentstroke}%
\pgfsetdash{}{0pt}%
\pgfpathmoveto{\pgfqpoint{4.588876in}{2.436767in}}%
\pgfpathlineto{\pgfqpoint{4.603045in}{2.446325in}}%
\pgfpathlineto{\pgfqpoint{4.617230in}{2.456068in}}%
\pgfpathlineto{\pgfqpoint{4.631430in}{2.465995in}}%
\pgfpathlineto{\pgfqpoint{4.645646in}{2.476107in}}%
\pgfpathlineto{\pgfqpoint{4.653503in}{2.486506in}}%
\pgfpathlineto{\pgfqpoint{4.661355in}{2.496785in}}%
\pgfpathlineto{\pgfqpoint{4.669199in}{2.506944in}}%
\pgfpathlineto{\pgfqpoint{4.677037in}{2.516983in}}%
\pgfpathlineto{\pgfqpoint{4.662825in}{2.506884in}}%
\pgfpathlineto{\pgfqpoint{4.648629in}{2.496970in}}%
\pgfpathlineto{\pgfqpoint{4.634448in}{2.487241in}}%
\pgfpathlineto{\pgfqpoint{4.620283in}{2.477696in}}%
\pgfpathlineto{\pgfqpoint{4.612441in}{2.467631in}}%
\pgfpathlineto{\pgfqpoint{4.604592in}{2.457455in}}%
\pgfpathlineto{\pgfqpoint{4.596737in}{2.447168in}}%
\pgfpathlineto{\pgfqpoint{4.588876in}{2.436767in}}%
\pgfpathclose%
\pgfusepath{fill}%
\end{pgfscope}%
\begin{pgfscope}%
\pgfpathrectangle{\pgfqpoint{1.150000in}{0.150000in}}{\pgfqpoint{5.700000in}{5.700000in}}%
\pgfusepath{clip}%
\pgfsetbuttcap%
\pgfsetroundjoin%
\definecolor{currentfill}{rgb}{0.137770,0.537492,0.554906}%
\pgfsetfillcolor{currentfill}%
\pgfsetfillopacity{0.800000}%
\pgfsetlinewidth{0.000000pt}%
\definecolor{currentstroke}{rgb}{0.000000,0.000000,0.000000}%
\pgfsetstrokecolor{currentstroke}%
\pgfsetdash{}{0pt}%
\pgfpathmoveto{\pgfqpoint{5.092148in}{2.899208in}}%
\pgfpathlineto{\pgfqpoint{5.106614in}{2.911640in}}%
\pgfpathlineto{\pgfqpoint{5.121098in}{2.924254in}}%
\pgfpathlineto{\pgfqpoint{5.135602in}{2.937053in}}%
\pgfpathlineto{\pgfqpoint{5.150125in}{2.950035in}}%
\pgfpathlineto{\pgfqpoint{5.157738in}{2.956356in}}%
\pgfpathlineto{\pgfqpoint{5.165343in}{2.962564in}}%
\pgfpathlineto{\pgfqpoint{5.172940in}{2.968660in}}%
\pgfpathlineto{\pgfqpoint{5.180528in}{2.974649in}}%
\pgfpathlineto{\pgfqpoint{5.166017in}{2.961921in}}%
\pgfpathlineto{\pgfqpoint{5.151525in}{2.949376in}}%
\pgfpathlineto{\pgfqpoint{5.137052in}{2.937015in}}%
\pgfpathlineto{\pgfqpoint{5.122598in}{2.924836in}}%
\pgfpathlineto{\pgfqpoint{5.114997in}{2.918582in}}%
\pgfpathlineto{\pgfqpoint{5.107389in}{2.912228in}}%
\pgfpathlineto{\pgfqpoint{5.099772in}{2.905771in}}%
\pgfpathlineto{\pgfqpoint{5.092148in}{2.899208in}}%
\pgfpathclose%
\pgfusepath{fill}%
\end{pgfscope}%
\begin{pgfscope}%
\pgfpathrectangle{\pgfqpoint{1.150000in}{0.150000in}}{\pgfqpoint{5.700000in}{5.700000in}}%
\pgfusepath{clip}%
\pgfsetbuttcap%
\pgfsetroundjoin%
\definecolor{currentfill}{rgb}{0.233603,0.313828,0.543914}%
\pgfsetfillcolor{currentfill}%
\pgfsetfillopacity{0.800000}%
\pgfsetlinewidth{0.000000pt}%
\definecolor{currentstroke}{rgb}{0.000000,0.000000,0.000000}%
\pgfsetstrokecolor{currentstroke}%
\pgfsetdash{}{0pt}%
\pgfpathmoveto{\pgfqpoint{4.381195in}{2.234585in}}%
\pgfpathlineto{\pgfqpoint{4.395255in}{2.242503in}}%
\pgfpathlineto{\pgfqpoint{4.409328in}{2.250605in}}%
\pgfpathlineto{\pgfqpoint{4.423416in}{2.258893in}}%
\pgfpathlineto{\pgfqpoint{4.437517in}{2.267366in}}%
\pgfpathlineto{\pgfqpoint{4.445452in}{2.279029in}}%
\pgfpathlineto{\pgfqpoint{4.453381in}{2.290587in}}%
\pgfpathlineto{\pgfqpoint{4.461304in}{2.302041in}}%
\pgfpathlineto{\pgfqpoint{4.469222in}{2.313390in}}%
\pgfpathlineto{\pgfqpoint{4.455122in}{2.304831in}}%
\pgfpathlineto{\pgfqpoint{4.441037in}{2.296456in}}%
\pgfpathlineto{\pgfqpoint{4.426966in}{2.288267in}}%
\pgfpathlineto{\pgfqpoint{4.412909in}{2.280262in}}%
\pgfpathlineto{\pgfqpoint{4.404989in}{2.268988in}}%
\pgfpathlineto{\pgfqpoint{4.397063in}{2.257616in}}%
\pgfpathlineto{\pgfqpoint{4.389132in}{2.246149in}}%
\pgfpathlineto{\pgfqpoint{4.381195in}{2.234585in}}%
\pgfpathclose%
\pgfusepath{fill}%
\end{pgfscope}%
\begin{pgfscope}%
\pgfpathrectangle{\pgfqpoint{1.150000in}{0.150000in}}{\pgfqpoint{5.700000in}{5.700000in}}%
\pgfusepath{clip}%
\pgfsetbuttcap%
\pgfsetroundjoin%
\definecolor{currentfill}{rgb}{0.283229,0.120777,0.440584}%
\pgfsetfillcolor{currentfill}%
\pgfsetfillopacity{0.800000}%
\pgfsetlinewidth{0.000000pt}%
\definecolor{currentstroke}{rgb}{0.000000,0.000000,0.000000}%
\pgfsetstrokecolor{currentstroke}%
\pgfsetdash{}{0pt}%
\pgfpathmoveto{\pgfqpoint{3.877940in}{1.787460in}}%
\pgfpathlineto{\pgfqpoint{3.891791in}{1.790134in}}%
\pgfpathlineto{\pgfqpoint{3.905650in}{1.792997in}}%
\pgfpathlineto{\pgfqpoint{3.919519in}{1.796048in}}%
\pgfpathlineto{\pgfqpoint{3.933398in}{1.799287in}}%
\pgfpathlineto{\pgfqpoint{3.941485in}{1.811701in}}%
\pgfpathlineto{\pgfqpoint{3.949567in}{1.824099in}}%
\pgfpathlineto{\pgfqpoint{3.957645in}{1.836478in}}%
\pgfpathlineto{\pgfqpoint{3.965717in}{1.848835in}}%
\pgfpathlineto{\pgfqpoint{3.951844in}{1.845285in}}%
\pgfpathlineto{\pgfqpoint{3.937980in}{1.841922in}}%
\pgfpathlineto{\pgfqpoint{3.924126in}{1.838749in}}%
\pgfpathlineto{\pgfqpoint{3.910282in}{1.835763in}}%
\pgfpathlineto{\pgfqpoint{3.902204in}{1.823706in}}%
\pgfpathlineto{\pgfqpoint{3.894121in}{1.811634in}}%
\pgfpathlineto{\pgfqpoint{3.886033in}{1.799551in}}%
\pgfpathlineto{\pgfqpoint{3.877940in}{1.787460in}}%
\pgfpathclose%
\pgfusepath{fill}%
\end{pgfscope}%
\begin{pgfscope}%
\pgfpathrectangle{\pgfqpoint{1.150000in}{0.150000in}}{\pgfqpoint{5.700000in}{5.700000in}}%
\pgfusepath{clip}%
\pgfsetbuttcap%
\pgfsetroundjoin%
\definecolor{currentfill}{rgb}{0.265145,0.232956,0.516599}%
\pgfsetfillcolor{currentfill}%
\pgfsetfillopacity{0.800000}%
\pgfsetlinewidth{0.000000pt}%
\definecolor{currentstroke}{rgb}{0.000000,0.000000,0.000000}%
\pgfsetstrokecolor{currentstroke}%
\pgfsetdash{}{0pt}%
\pgfpathmoveto{\pgfqpoint{4.173495in}{2.035616in}}%
\pgfpathlineto{\pgfqpoint{4.187457in}{2.041592in}}%
\pgfpathlineto{\pgfqpoint{4.201432in}{2.047755in}}%
\pgfpathlineto{\pgfqpoint{4.215419in}{2.054103in}}%
\pgfpathlineto{\pgfqpoint{4.229418in}{2.060637in}}%
\pgfpathlineto{\pgfqpoint{4.237419in}{2.073087in}}%
\pgfpathlineto{\pgfqpoint{4.245415in}{2.085462in}}%
\pgfpathlineto{\pgfqpoint{4.253406in}{2.097759in}}%
\pgfpathlineto{\pgfqpoint{4.261391in}{2.109978in}}%
\pgfpathlineto{\pgfqpoint{4.247394in}{2.103260in}}%
\pgfpathlineto{\pgfqpoint{4.233410in}{2.096727in}}%
\pgfpathlineto{\pgfqpoint{4.219438in}{2.090380in}}%
\pgfpathlineto{\pgfqpoint{4.205478in}{2.084220in}}%
\pgfpathlineto{\pgfqpoint{4.197490in}{2.072173in}}%
\pgfpathlineto{\pgfqpoint{4.189497in}{2.060056in}}%
\pgfpathlineto{\pgfqpoint{4.181498in}{2.047870in}}%
\pgfpathlineto{\pgfqpoint{4.173495in}{2.035616in}}%
\pgfpathclose%
\pgfusepath{fill}%
\end{pgfscope}%
\begin{pgfscope}%
\pgfpathrectangle{\pgfqpoint{1.150000in}{0.150000in}}{\pgfqpoint{5.700000in}{5.700000in}}%
\pgfusepath{clip}%
\pgfsetbuttcap%
\pgfsetroundjoin%
\definecolor{currentfill}{rgb}{0.123444,0.636809,0.528763}%
\pgfsetfillcolor{currentfill}%
\pgfsetfillopacity{0.800000}%
\pgfsetlinewidth{0.000000pt}%
\definecolor{currentstroke}{rgb}{0.000000,0.000000,0.000000}%
\pgfsetstrokecolor{currentstroke}%
\pgfsetdash{}{0pt}%
\pgfpathmoveto{\pgfqpoint{5.475726in}{3.209306in}}%
\pgfpathlineto{\pgfqpoint{5.490430in}{3.223045in}}%
\pgfpathlineto{\pgfqpoint{5.505156in}{3.236967in}}%
\pgfpathlineto{\pgfqpoint{5.519902in}{3.251071in}}%
\pgfpathlineto{\pgfqpoint{5.534670in}{3.265357in}}%
\pgfpathlineto{\pgfqpoint{5.542043in}{3.268414in}}%
\pgfpathlineto{\pgfqpoint{5.549408in}{3.271400in}}%
\pgfpathlineto{\pgfqpoint{5.556765in}{3.274320in}}%
\pgfpathlineto{\pgfqpoint{5.564113in}{3.277180in}}%
\pgfpathlineto{\pgfqpoint{5.549367in}{3.263324in}}%
\pgfpathlineto{\pgfqpoint{5.534643in}{3.249650in}}%
\pgfpathlineto{\pgfqpoint{5.519939in}{3.236157in}}%
\pgfpathlineto{\pgfqpoint{5.505256in}{3.222846in}}%
\pgfpathlineto{\pgfqpoint{5.497886in}{3.219545in}}%
\pgfpathlineto{\pgfqpoint{5.490507in}{3.216191in}}%
\pgfpathlineto{\pgfqpoint{5.483121in}{3.212780in}}%
\pgfpathlineto{\pgfqpoint{5.475726in}{3.209306in}}%
\pgfpathclose%
\pgfusepath{fill}%
\end{pgfscope}%
\begin{pgfscope}%
\pgfpathrectangle{\pgfqpoint{1.150000in}{0.150000in}}{\pgfqpoint{5.700000in}{5.700000in}}%
\pgfusepath{clip}%
\pgfsetbuttcap%
\pgfsetroundjoin%
\definecolor{currentfill}{rgb}{0.268510,0.009605,0.335427}%
\pgfsetfillcolor{currentfill}%
\pgfsetfillopacity{0.800000}%
\pgfsetlinewidth{0.000000pt}%
\definecolor{currentstroke}{rgb}{0.000000,0.000000,0.000000}%
\pgfsetstrokecolor{currentstroke}%
\pgfsetdash{}{0pt}%
\pgfpathmoveto{\pgfqpoint{3.438609in}{1.576032in}}%
\pgfpathlineto{\pgfqpoint{3.452377in}{1.572830in}}%
\pgfpathlineto{\pgfqpoint{3.466150in}{1.569825in}}%
\pgfpathlineto{\pgfqpoint{3.479927in}{1.567018in}}%
\pgfpathlineto{\pgfqpoint{3.493709in}{1.564407in}}%
\pgfpathlineto{\pgfqpoint{3.501959in}{1.574035in}}%
\pgfpathlineto{\pgfqpoint{3.510201in}{1.583760in}}%
\pgfpathlineto{\pgfqpoint{3.518436in}{1.593577in}}%
\pgfpathlineto{\pgfqpoint{3.526665in}{1.603482in}}%
\pgfpathlineto{\pgfqpoint{3.512898in}{1.605626in}}%
\pgfpathlineto{\pgfqpoint{3.499137in}{1.607967in}}%
\pgfpathlineto{\pgfqpoint{3.485380in}{1.610505in}}%
\pgfpathlineto{\pgfqpoint{3.471629in}{1.613240in}}%
\pgfpathlineto{\pgfqpoint{3.463384in}{1.603791in}}%
\pgfpathlineto{\pgfqpoint{3.455133in}{1.594437in}}%
\pgfpathlineto{\pgfqpoint{3.446874in}{1.585182in}}%
\pgfpathlineto{\pgfqpoint{3.438609in}{1.576032in}}%
\pgfpathclose%
\pgfusepath{fill}%
\end{pgfscope}%
\begin{pgfscope}%
\pgfpathrectangle{\pgfqpoint{1.150000in}{0.150000in}}{\pgfqpoint{5.700000in}{5.700000in}}%
\pgfusepath{clip}%
\pgfsetbuttcap%
\pgfsetroundjoin%
\definecolor{currentfill}{rgb}{0.274952,0.037752,0.364543}%
\pgfsetfillcolor{currentfill}%
\pgfsetfillopacity{0.800000}%
\pgfsetlinewidth{0.000000pt}%
\definecolor{currentstroke}{rgb}{0.000000,0.000000,0.000000}%
\pgfsetstrokecolor{currentstroke}%
\pgfsetdash{}{0pt}%
\pgfpathmoveto{\pgfqpoint{3.007471in}{1.652651in}}%
\pgfpathlineto{\pgfqpoint{3.021273in}{1.642854in}}%
\pgfpathlineto{\pgfqpoint{3.035074in}{1.633276in}}%
\pgfpathlineto{\pgfqpoint{3.048875in}{1.623917in}}%
\pgfpathlineto{\pgfqpoint{3.062674in}{1.614776in}}%
\pgfpathlineto{\pgfqpoint{3.071171in}{1.619411in}}%
\pgfpathlineto{\pgfqpoint{3.079656in}{1.624252in}}%
\pgfpathlineto{\pgfqpoint{3.088130in}{1.629293in}}%
\pgfpathlineto{\pgfqpoint{3.096593in}{1.634528in}}%
\pgfpathlineto{\pgfqpoint{3.082824in}{1.643102in}}%
\pgfpathlineto{\pgfqpoint{3.069054in}{1.651894in}}%
\pgfpathlineto{\pgfqpoint{3.055284in}{1.660904in}}%
\pgfpathlineto{\pgfqpoint{3.041514in}{1.670134in}}%
\pgfpathlineto{\pgfqpoint{3.033021in}{1.665454in}}%
\pgfpathlineto{\pgfqpoint{3.024516in}{1.660976in}}%
\pgfpathlineto{\pgfqpoint{3.016000in}{1.656707in}}%
\pgfpathlineto{\pgfqpoint{3.007471in}{1.652651in}}%
\pgfpathclose%
\pgfusepath{fill}%
\end{pgfscope}%
\begin{pgfscope}%
\pgfpathrectangle{\pgfqpoint{1.150000in}{0.150000in}}{\pgfqpoint{5.700000in}{5.700000in}}%
\pgfusepath{clip}%
\pgfsetbuttcap%
\pgfsetroundjoin%
\definecolor{currentfill}{rgb}{0.188923,0.410910,0.556326}%
\pgfsetfillcolor{currentfill}%
\pgfsetfillopacity{0.800000}%
\pgfsetlinewidth{0.000000pt}%
\definecolor{currentstroke}{rgb}{0.000000,0.000000,0.000000}%
\pgfsetstrokecolor{currentstroke}%
\pgfsetdash{}{0pt}%
\pgfpathmoveto{\pgfqpoint{2.225717in}{2.601174in}}%
\pgfpathlineto{\pgfqpoint{2.239965in}{2.576290in}}%
\pgfpathlineto{\pgfqpoint{2.254197in}{2.551742in}}%
\pgfpathlineto{\pgfqpoint{2.268413in}{2.527527in}}%
\pgfpathlineto{\pgfqpoint{2.282614in}{2.503641in}}%
\pgfpathlineto{\pgfqpoint{2.291728in}{2.499901in}}%
\pgfpathlineto{\pgfqpoint{2.300820in}{2.496491in}}%
\pgfpathlineto{\pgfqpoint{2.309890in}{2.493405in}}%
\pgfpathlineto{\pgfqpoint{2.318939in}{2.490637in}}%
\pgfpathlineto{\pgfqpoint{2.304796in}{2.513908in}}%
\pgfpathlineto{\pgfqpoint{2.290638in}{2.537506in}}%
\pgfpathlineto{\pgfqpoint{2.276464in}{2.561436in}}%
\pgfpathlineto{\pgfqpoint{2.262276in}{2.585701in}}%
\pgfpathlineto{\pgfqpoint{2.253169in}{2.589072in}}%
\pgfpathlineto{\pgfqpoint{2.244041in}{2.592770in}}%
\pgfpathlineto{\pgfqpoint{2.234891in}{2.596802in}}%
\pgfpathlineto{\pgfqpoint{2.225717in}{2.601174in}}%
\pgfpathclose%
\pgfusepath{fill}%
\end{pgfscope}%
\begin{pgfscope}%
\pgfpathrectangle{\pgfqpoint{1.150000in}{0.150000in}}{\pgfqpoint{5.700000in}{5.700000in}}%
\pgfusepath{clip}%
\pgfsetbuttcap%
\pgfsetroundjoin%
\definecolor{currentfill}{rgb}{0.281412,0.155834,0.469201}%
\pgfsetfillcolor{currentfill}%
\pgfsetfillopacity{0.800000}%
\pgfsetlinewidth{0.000000pt}%
\definecolor{currentstroke}{rgb}{0.000000,0.000000,0.000000}%
\pgfsetstrokecolor{currentstroke}%
\pgfsetdash{}{0pt}%
\pgfpathmoveto{\pgfqpoint{3.965717in}{1.848835in}}%
\pgfpathlineto{\pgfqpoint{3.979601in}{1.852572in}}%
\pgfpathlineto{\pgfqpoint{3.993495in}{1.856498in}}%
\pgfpathlineto{\pgfqpoint{4.007400in}{1.860610in}}%
\pgfpathlineto{\pgfqpoint{4.021315in}{1.864909in}}%
\pgfpathlineto{\pgfqpoint{4.029378in}{1.877533in}}%
\pgfpathlineto{\pgfqpoint{4.037437in}{1.890122in}}%
\pgfpathlineto{\pgfqpoint{4.045491in}{1.902673in}}%
\pgfpathlineto{\pgfqpoint{4.053541in}{1.915184in}}%
\pgfpathlineto{\pgfqpoint{4.039629in}{1.910605in}}%
\pgfpathlineto{\pgfqpoint{4.025729in}{1.906213in}}%
\pgfpathlineto{\pgfqpoint{4.011839in}{1.902008in}}%
\pgfpathlineto{\pgfqpoint{3.997960in}{1.897990in}}%
\pgfpathlineto{\pgfqpoint{3.989907in}{1.885747in}}%
\pgfpathlineto{\pgfqpoint{3.981848in}{1.873472in}}%
\pgfpathlineto{\pgfqpoint{3.973785in}{1.861167in}}%
\pgfpathlineto{\pgfqpoint{3.965717in}{1.848835in}}%
\pgfpathclose%
\pgfusepath{fill}%
\end{pgfscope}%
\begin{pgfscope}%
\pgfpathrectangle{\pgfqpoint{1.150000in}{0.150000in}}{\pgfqpoint{5.700000in}{5.700000in}}%
\pgfusepath{clip}%
\pgfsetbuttcap%
\pgfsetroundjoin%
\definecolor{currentfill}{rgb}{0.274128,0.199721,0.498911}%
\pgfsetfillcolor{currentfill}%
\pgfsetfillopacity{0.800000}%
\pgfsetlinewidth{0.000000pt}%
\definecolor{currentstroke}{rgb}{0.000000,0.000000,0.000000}%
\pgfsetstrokecolor{currentstroke}%
\pgfsetdash{}{0pt}%
\pgfpathmoveto{\pgfqpoint{2.584307in}{2.024986in}}%
\pgfpathlineto{\pgfqpoint{2.598276in}{2.007828in}}%
\pgfpathlineto{\pgfqpoint{2.612237in}{1.990934in}}%
\pgfpathlineto{\pgfqpoint{2.626191in}{1.974303in}}%
\pgfpathlineto{\pgfqpoint{2.640136in}{1.957932in}}%
\pgfpathlineto{\pgfqpoint{2.648962in}{1.957342in}}%
\pgfpathlineto{\pgfqpoint{2.657770in}{1.957042in}}%
\pgfpathlineto{\pgfqpoint{2.666560in}{1.957025in}}%
\pgfpathlineto{\pgfqpoint{2.675334in}{1.957286in}}%
\pgfpathlineto{\pgfqpoint{2.661434in}{1.973039in}}%
\pgfpathlineto{\pgfqpoint{2.647527in}{1.989051in}}%
\pgfpathlineto{\pgfqpoint{2.633613in}{2.005325in}}%
\pgfpathlineto{\pgfqpoint{2.619691in}{2.021862in}}%
\pgfpathlineto{\pgfqpoint{2.610872in}{2.022207in}}%
\pgfpathlineto{\pgfqpoint{2.602035in}{2.022839in}}%
\pgfpathlineto{\pgfqpoint{2.593180in}{2.023763in}}%
\pgfpathlineto{\pgfqpoint{2.584307in}{2.024986in}}%
\pgfpathclose%
\pgfusepath{fill}%
\end{pgfscope}%
\begin{pgfscope}%
\pgfpathrectangle{\pgfqpoint{1.150000in}{0.150000in}}{\pgfqpoint{5.700000in}{5.700000in}}%
\pgfusepath{clip}%
\pgfsetbuttcap%
\pgfsetroundjoin%
\definecolor{currentfill}{rgb}{0.279574,0.170599,0.479997}%
\pgfsetfillcolor{currentfill}%
\pgfsetfillopacity{0.800000}%
\pgfsetlinewidth{0.000000pt}%
\definecolor{currentstroke}{rgb}{0.000000,0.000000,0.000000}%
\pgfsetstrokecolor{currentstroke}%
\pgfsetdash{}{0pt}%
\pgfpathmoveto{\pgfqpoint{2.640136in}{1.957932in}}%
\pgfpathlineto{\pgfqpoint{2.654075in}{1.941820in}}%
\pgfpathlineto{\pgfqpoint{2.668007in}{1.925965in}}%
\pgfpathlineto{\pgfqpoint{2.681932in}{1.910364in}}%
\pgfpathlineto{\pgfqpoint{2.695851in}{1.895017in}}%
\pgfpathlineto{\pgfqpoint{2.704630in}{1.895056in}}%
\pgfpathlineto{\pgfqpoint{2.713393in}{1.895376in}}%
\pgfpathlineto{\pgfqpoint{2.722140in}{1.895971in}}%
\pgfpathlineto{\pgfqpoint{2.730870in}{1.896835in}}%
\pgfpathlineto{\pgfqpoint{2.716995in}{1.911567in}}%
\pgfpathlineto{\pgfqpoint{2.703115in}{1.926552in}}%
\pgfpathlineto{\pgfqpoint{2.689228in}{1.941791in}}%
\pgfpathlineto{\pgfqpoint{2.675334in}{1.957286in}}%
\pgfpathlineto{\pgfqpoint{2.666560in}{1.957025in}}%
\pgfpathlineto{\pgfqpoint{2.657770in}{1.957042in}}%
\pgfpathlineto{\pgfqpoint{2.648962in}{1.957342in}}%
\pgfpathlineto{\pgfqpoint{2.640136in}{1.957932in}}%
\pgfpathclose%
\pgfusepath{fill}%
\end{pgfscope}%
\begin{pgfscope}%
\pgfpathrectangle{\pgfqpoint{1.150000in}{0.150000in}}{\pgfqpoint{5.700000in}{5.700000in}}%
\pgfusepath{clip}%
\pgfsetbuttcap%
\pgfsetroundjoin%
\definecolor{currentfill}{rgb}{0.160665,0.478540,0.558115}%
\pgfsetfillcolor{currentfill}%
\pgfsetfillopacity{0.800000}%
\pgfsetlinewidth{0.000000pt}%
\definecolor{currentstroke}{rgb}{0.000000,0.000000,0.000000}%
\pgfsetstrokecolor{currentstroke}%
\pgfsetdash{}{0pt}%
\pgfpathmoveto{\pgfqpoint{4.884737in}{2.713959in}}%
\pgfpathlineto{\pgfqpoint{4.899085in}{2.725472in}}%
\pgfpathlineto{\pgfqpoint{4.913450in}{2.737170in}}%
\pgfpathlineto{\pgfqpoint{4.927833in}{2.749052in}}%
\pgfpathlineto{\pgfqpoint{4.942234in}{2.761118in}}%
\pgfpathlineto{\pgfqpoint{4.949963in}{2.769292in}}%
\pgfpathlineto{\pgfqpoint{4.957684in}{2.777339in}}%
\pgfpathlineto{\pgfqpoint{4.965398in}{2.785261in}}%
\pgfpathlineto{\pgfqpoint{4.973103in}{2.793061in}}%
\pgfpathlineto{\pgfqpoint{4.958710in}{2.781145in}}%
\pgfpathlineto{\pgfqpoint{4.944335in}{2.769413in}}%
\pgfpathlineto{\pgfqpoint{4.929977in}{2.757865in}}%
\pgfpathlineto{\pgfqpoint{4.915637in}{2.746501in}}%
\pgfpathlineto{\pgfqpoint{4.907924in}{2.738539in}}%
\pgfpathlineto{\pgfqpoint{4.900202in}{2.730462in}}%
\pgfpathlineto{\pgfqpoint{4.892474in}{2.722270in}}%
\pgfpathlineto{\pgfqpoint{4.884737in}{2.713959in}}%
\pgfpathclose%
\pgfusepath{fill}%
\end{pgfscope}%
\begin{pgfscope}%
\pgfpathrectangle{\pgfqpoint{1.150000in}{0.150000in}}{\pgfqpoint{5.700000in}{5.700000in}}%
\pgfusepath{clip}%
\pgfsetbuttcap%
\pgfsetroundjoin%
\definecolor{currentfill}{rgb}{0.266580,0.228262,0.514349}%
\pgfsetfillcolor{currentfill}%
\pgfsetfillopacity{0.800000}%
\pgfsetlinewidth{0.000000pt}%
\definecolor{currentstroke}{rgb}{0.000000,0.000000,0.000000}%
\pgfsetstrokecolor{currentstroke}%
\pgfsetdash{}{0pt}%
\pgfpathmoveto{\pgfqpoint{2.528346in}{2.096306in}}%
\pgfpathlineto{\pgfqpoint{2.542350in}{2.078069in}}%
\pgfpathlineto{\pgfqpoint{2.556344in}{2.060105in}}%
\pgfpathlineto{\pgfqpoint{2.570330in}{2.042411in}}%
\pgfpathlineto{\pgfqpoint{2.584307in}{2.024986in}}%
\pgfpathlineto{\pgfqpoint{2.593180in}{2.023763in}}%
\pgfpathlineto{\pgfqpoint{2.602035in}{2.022839in}}%
\pgfpathlineto{\pgfqpoint{2.610872in}{2.022207in}}%
\pgfpathlineto{\pgfqpoint{2.619691in}{2.021862in}}%
\pgfpathlineto{\pgfqpoint{2.605762in}{2.038665in}}%
\pgfpathlineto{\pgfqpoint{2.591825in}{2.055735in}}%
\pgfpathlineto{\pgfqpoint{2.577879in}{2.073076in}}%
\pgfpathlineto{\pgfqpoint{2.563925in}{2.090688in}}%
\pgfpathlineto{\pgfqpoint{2.555058in}{2.091643in}}%
\pgfpathlineto{\pgfqpoint{2.546173in}{2.092893in}}%
\pgfpathlineto{\pgfqpoint{2.537269in}{2.094446in}}%
\pgfpathlineto{\pgfqpoint{2.528346in}{2.096306in}}%
\pgfpathclose%
\pgfusepath{fill}%
\end{pgfscope}%
\begin{pgfscope}%
\pgfpathrectangle{\pgfqpoint{1.150000in}{0.150000in}}{\pgfqpoint{5.700000in}{5.700000in}}%
\pgfusepath{clip}%
\pgfsetbuttcap%
\pgfsetroundjoin%
\definecolor{currentfill}{rgb}{0.268510,0.009605,0.335427}%
\pgfsetfillcolor{currentfill}%
\pgfsetfillopacity{0.800000}%
\pgfsetlinewidth{0.000000pt}%
\definecolor{currentstroke}{rgb}{0.000000,0.000000,0.000000}%
\pgfsetstrokecolor{currentstroke}%
\pgfsetdash{}{0pt}%
\pgfpathmoveto{\pgfqpoint{3.206759in}{1.573633in}}%
\pgfpathlineto{\pgfqpoint{3.220534in}{1.566969in}}%
\pgfpathlineto{\pgfqpoint{3.234310in}{1.560512in}}%
\pgfpathlineto{\pgfqpoint{3.248089in}{1.554261in}}%
\pgfpathlineto{\pgfqpoint{3.261869in}{1.548216in}}%
\pgfpathlineto{\pgfqpoint{3.270242in}{1.555274in}}%
\pgfpathlineto{\pgfqpoint{3.278606in}{1.562491in}}%
\pgfpathlineto{\pgfqpoint{3.286961in}{1.569862in}}%
\pgfpathlineto{\pgfqpoint{3.295307in}{1.577383in}}%
\pgfpathlineto{\pgfqpoint{3.281549in}{1.582897in}}%
\pgfpathlineto{\pgfqpoint{3.267795in}{1.588616in}}%
\pgfpathlineto{\pgfqpoint{3.254042in}{1.594542in}}%
\pgfpathlineto{\pgfqpoint{3.240291in}{1.600674in}}%
\pgfpathlineto{\pgfqpoint{3.231922in}{1.593673in}}%
\pgfpathlineto{\pgfqpoint{3.223544in}{1.586829in}}%
\pgfpathlineto{\pgfqpoint{3.215156in}{1.580147in}}%
\pgfpathlineto{\pgfqpoint{3.206759in}{1.573633in}}%
\pgfpathclose%
\pgfusepath{fill}%
\end{pgfscope}%
\begin{pgfscope}%
\pgfpathrectangle{\pgfqpoint{1.150000in}{0.150000in}}{\pgfqpoint{5.700000in}{5.700000in}}%
\pgfusepath{clip}%
\pgfsetbuttcap%
\pgfsetroundjoin%
\definecolor{currentfill}{rgb}{0.282290,0.145912,0.461510}%
\pgfsetfillcolor{currentfill}%
\pgfsetfillopacity{0.800000}%
\pgfsetlinewidth{0.000000pt}%
\definecolor{currentstroke}{rgb}{0.000000,0.000000,0.000000}%
\pgfsetstrokecolor{currentstroke}%
\pgfsetdash{}{0pt}%
\pgfpathmoveto{\pgfqpoint{2.695851in}{1.895017in}}%
\pgfpathlineto{\pgfqpoint{2.709763in}{1.879921in}}%
\pgfpathlineto{\pgfqpoint{2.723670in}{1.865074in}}%
\pgfpathlineto{\pgfqpoint{2.737570in}{1.850475in}}%
\pgfpathlineto{\pgfqpoint{2.751466in}{1.836123in}}%
\pgfpathlineto{\pgfqpoint{2.760201in}{1.836788in}}%
\pgfpathlineto{\pgfqpoint{2.768921in}{1.837725in}}%
\pgfpathlineto{\pgfqpoint{2.777625in}{1.838928in}}%
\pgfpathlineto{\pgfqpoint{2.786314in}{1.840391in}}%
\pgfpathlineto{\pgfqpoint{2.772461in}{1.854132in}}%
\pgfpathlineto{\pgfqpoint{2.758603in}{1.868119in}}%
\pgfpathlineto{\pgfqpoint{2.744739in}{1.882352in}}%
\pgfpathlineto{\pgfqpoint{2.730870in}{1.896835in}}%
\pgfpathlineto{\pgfqpoint{2.722140in}{1.895971in}}%
\pgfpathlineto{\pgfqpoint{2.713393in}{1.895376in}}%
\pgfpathlineto{\pgfqpoint{2.704630in}{1.895056in}}%
\pgfpathlineto{\pgfqpoint{2.695851in}{1.895017in}}%
\pgfpathclose%
\pgfusepath{fill}%
\end{pgfscope}%
\begin{pgfscope}%
\pgfpathrectangle{\pgfqpoint{1.150000in}{0.150000in}}{\pgfqpoint{5.700000in}{5.700000in}}%
\pgfusepath{clip}%
\pgfsetbuttcap%
\pgfsetroundjoin%
\definecolor{currentfill}{rgb}{0.137339,0.662252,0.515571}%
\pgfsetfillcolor{currentfill}%
\pgfsetfillopacity{0.800000}%
\pgfsetlinewidth{0.000000pt}%
\definecolor{currentstroke}{rgb}{0.000000,0.000000,0.000000}%
\pgfsetstrokecolor{currentstroke}%
\pgfsetdash{}{0pt}%
\pgfpathmoveto{\pgfqpoint{5.564113in}{3.277180in}}%
\pgfpathlineto{\pgfqpoint{5.578880in}{3.291218in}}%
\pgfpathlineto{\pgfqpoint{5.593668in}{3.305439in}}%
\pgfpathlineto{\pgfqpoint{5.608477in}{3.319842in}}%
\pgfpathlineto{\pgfqpoint{5.623308in}{3.334427in}}%
\pgfpathlineto{\pgfqpoint{5.630624in}{3.336778in}}%
\pgfpathlineto{\pgfqpoint{5.637932in}{3.339070in}}%
\pgfpathlineto{\pgfqpoint{5.645231in}{3.341308in}}%
\pgfpathlineto{\pgfqpoint{5.652521in}{3.343498in}}%
\pgfpathlineto{\pgfqpoint{5.637715in}{3.329379in}}%
\pgfpathlineto{\pgfqpoint{5.622929in}{3.315442in}}%
\pgfpathlineto{\pgfqpoint{5.608165in}{3.301686in}}%
\pgfpathlineto{\pgfqpoint{5.593423in}{3.288111in}}%
\pgfpathlineto{\pgfqpoint{5.586107in}{3.285444in}}%
\pgfpathlineto{\pgfqpoint{5.578784in}{3.282737in}}%
\pgfpathlineto{\pgfqpoint{5.571452in}{3.279984in}}%
\pgfpathlineto{\pgfqpoint{5.564113in}{3.277180in}}%
\pgfpathclose%
\pgfusepath{fill}%
\end{pgfscope}%
\begin{pgfscope}%
\pgfpathrectangle{\pgfqpoint{1.150000in}{0.150000in}}{\pgfqpoint{5.700000in}{5.700000in}}%
\pgfusepath{clip}%
\pgfsetbuttcap%
\pgfsetroundjoin%
\definecolor{currentfill}{rgb}{0.128729,0.563265,0.551229}%
\pgfsetfillcolor{currentfill}%
\pgfsetfillopacity{0.800000}%
\pgfsetlinewidth{0.000000pt}%
\definecolor{currentstroke}{rgb}{0.000000,0.000000,0.000000}%
\pgfsetstrokecolor{currentstroke}%
\pgfsetdash{}{0pt}%
\pgfpathmoveto{\pgfqpoint{5.180528in}{2.974649in}}%
\pgfpathlineto{\pgfqpoint{5.195058in}{2.987560in}}%
\pgfpathlineto{\pgfqpoint{5.209607in}{3.000654in}}%
\pgfpathlineto{\pgfqpoint{5.224177in}{3.013932in}}%
\pgfpathlineto{\pgfqpoint{5.238766in}{3.027394in}}%
\pgfpathlineto{\pgfqpoint{5.246332in}{3.033001in}}%
\pgfpathlineto{\pgfqpoint{5.253890in}{3.038497in}}%
\pgfpathlineto{\pgfqpoint{5.261439in}{3.043888in}}%
\pgfpathlineto{\pgfqpoint{5.268980in}{3.049175in}}%
\pgfpathlineto{\pgfqpoint{5.254405in}{3.036003in}}%
\pgfpathlineto{\pgfqpoint{5.239849in}{3.023014in}}%
\pgfpathlineto{\pgfqpoint{5.225314in}{3.010209in}}%
\pgfpathlineto{\pgfqpoint{5.210797in}{2.997586in}}%
\pgfpathlineto{\pgfqpoint{5.203242in}{2.991997in}}%
\pgfpathlineto{\pgfqpoint{5.195679in}{2.986314in}}%
\pgfpathlineto{\pgfqpoint{5.188107in}{2.980532in}}%
\pgfpathlineto{\pgfqpoint{5.180528in}{2.974649in}}%
\pgfpathclose%
\pgfusepath{fill}%
\end{pgfscope}%
\begin{pgfscope}%
\pgfpathrectangle{\pgfqpoint{1.150000in}{0.150000in}}{\pgfqpoint{5.700000in}{5.700000in}}%
\pgfusepath{clip}%
\pgfsetbuttcap%
\pgfsetroundjoin%
\definecolor{currentfill}{rgb}{0.187231,0.414746,0.556547}%
\pgfsetfillcolor{currentfill}%
\pgfsetfillopacity{0.800000}%
\pgfsetlinewidth{0.000000pt}%
\definecolor{currentstroke}{rgb}{0.000000,0.000000,0.000000}%
\pgfsetstrokecolor{currentstroke}%
\pgfsetdash{}{0pt}%
\pgfpathmoveto{\pgfqpoint{4.677037in}{2.516983in}}%
\pgfpathlineto{\pgfqpoint{4.691266in}{2.527266in}}%
\pgfpathlineto{\pgfqpoint{4.705511in}{2.537733in}}%
\pgfpathlineto{\pgfqpoint{4.719772in}{2.548385in}}%
\pgfpathlineto{\pgfqpoint{4.734050in}{2.559222in}}%
\pgfpathlineto{\pgfqpoint{4.741877in}{2.569107in}}%
\pgfpathlineto{\pgfqpoint{4.749698in}{2.578866in}}%
\pgfpathlineto{\pgfqpoint{4.757512in}{2.588500in}}%
\pgfpathlineto{\pgfqpoint{4.765319in}{2.598009in}}%
\pgfpathlineto{\pgfqpoint{4.751045in}{2.587219in}}%
\pgfpathlineto{\pgfqpoint{4.736788in}{2.576615in}}%
\pgfpathlineto{\pgfqpoint{4.722548in}{2.566194in}}%
\pgfpathlineto{\pgfqpoint{4.708324in}{2.555958in}}%
\pgfpathlineto{\pgfqpoint{4.700512in}{2.546389in}}%
\pgfpathlineto{\pgfqpoint{4.692694in}{2.536705in}}%
\pgfpathlineto{\pgfqpoint{4.684869in}{2.526903in}}%
\pgfpathlineto{\pgfqpoint{4.677037in}{2.516983in}}%
\pgfpathclose%
\pgfusepath{fill}%
\end{pgfscope}%
\begin{pgfscope}%
\pgfpathrectangle{\pgfqpoint{1.150000in}{0.150000in}}{\pgfqpoint{5.700000in}{5.700000in}}%
\pgfusepath{clip}%
\pgfsetbuttcap%
\pgfsetroundjoin%
\definecolor{currentfill}{rgb}{0.255645,0.260703,0.528312}%
\pgfsetfillcolor{currentfill}%
\pgfsetfillopacity{0.800000}%
\pgfsetlinewidth{0.000000pt}%
\definecolor{currentstroke}{rgb}{0.000000,0.000000,0.000000}%
\pgfsetstrokecolor{currentstroke}%
\pgfsetdash{}{0pt}%
\pgfpathmoveto{\pgfqpoint{2.472237in}{2.172028in}}%
\pgfpathlineto{\pgfqpoint{2.486279in}{2.152677in}}%
\pgfpathlineto{\pgfqpoint{2.500311in}{2.133608in}}%
\pgfpathlineto{\pgfqpoint{2.514334in}{2.114818in}}%
\pgfpathlineto{\pgfqpoint{2.528346in}{2.096306in}}%
\pgfpathlineto{\pgfqpoint{2.537269in}{2.094446in}}%
\pgfpathlineto{\pgfqpoint{2.546173in}{2.092893in}}%
\pgfpathlineto{\pgfqpoint{2.555058in}{2.091643in}}%
\pgfpathlineto{\pgfqpoint{2.563925in}{2.090688in}}%
\pgfpathlineto{\pgfqpoint{2.549962in}{2.108573in}}%
\pgfpathlineto{\pgfqpoint{2.535990in}{2.126735in}}%
\pgfpathlineto{\pgfqpoint{2.522009in}{2.145176in}}%
\pgfpathlineto{\pgfqpoint{2.508018in}{2.163897in}}%
\pgfpathlineto{\pgfqpoint{2.499102in}{2.165466in}}%
\pgfpathlineto{\pgfqpoint{2.490167in}{2.167340in}}%
\pgfpathlineto{\pgfqpoint{2.481211in}{2.169525in}}%
\pgfpathlineto{\pgfqpoint{2.472237in}{2.172028in}}%
\pgfpathclose%
\pgfusepath{fill}%
\end{pgfscope}%
\begin{pgfscope}%
\pgfpathrectangle{\pgfqpoint{1.150000in}{0.150000in}}{\pgfqpoint{5.700000in}{5.700000in}}%
\pgfusepath{clip}%
\pgfsetbuttcap%
\pgfsetroundjoin%
\definecolor{currentfill}{rgb}{0.218130,0.347432,0.550038}%
\pgfsetfillcolor{currentfill}%
\pgfsetfillopacity{0.800000}%
\pgfsetlinewidth{0.000000pt}%
\definecolor{currentstroke}{rgb}{0.000000,0.000000,0.000000}%
\pgfsetstrokecolor{currentstroke}%
\pgfsetdash{}{0pt}%
\pgfpathmoveto{\pgfqpoint{4.469222in}{2.313390in}}%
\pgfpathlineto{\pgfqpoint{4.483336in}{2.322135in}}%
\pgfpathlineto{\pgfqpoint{4.497465in}{2.331065in}}%
\pgfpathlineto{\pgfqpoint{4.511609in}{2.340179in}}%
\pgfpathlineto{\pgfqpoint{4.525768in}{2.349478in}}%
\pgfpathlineto{\pgfqpoint{4.533677in}{2.360788in}}%
\pgfpathlineto{\pgfqpoint{4.541581in}{2.371984in}}%
\pgfpathlineto{\pgfqpoint{4.549479in}{2.383066in}}%
\pgfpathlineto{\pgfqpoint{4.557370in}{2.394034in}}%
\pgfpathlineto{\pgfqpoint{4.543214in}{2.384681in}}%
\pgfpathlineto{\pgfqpoint{4.529073in}{2.375513in}}%
\pgfpathlineto{\pgfqpoint{4.514946in}{2.366529in}}%
\pgfpathlineto{\pgfqpoint{4.500835in}{2.357731in}}%
\pgfpathlineto{\pgfqpoint{4.492940in}{2.346804in}}%
\pgfpathlineto{\pgfqpoint{4.485040in}{2.335772in}}%
\pgfpathlineto{\pgfqpoint{4.477134in}{2.324634in}}%
\pgfpathlineto{\pgfqpoint{4.469222in}{2.313390in}}%
\pgfpathclose%
\pgfusepath{fill}%
\end{pgfscope}%
\begin{pgfscope}%
\pgfpathrectangle{\pgfqpoint{1.150000in}{0.150000in}}{\pgfqpoint{5.700000in}{5.700000in}}%
\pgfusepath{clip}%
\pgfsetbuttcap%
\pgfsetroundjoin%
\definecolor{currentfill}{rgb}{0.267004,0.004874,0.329415}%
\pgfsetfillcolor{currentfill}%
\pgfsetfillopacity{0.800000}%
\pgfsetlinewidth{0.000000pt}%
\definecolor{currentstroke}{rgb}{0.000000,0.000000,0.000000}%
\pgfsetstrokecolor{currentstroke}%
\pgfsetdash{}{0pt}%
\pgfpathmoveto{\pgfqpoint{3.350362in}{1.557365in}}%
\pgfpathlineto{\pgfqpoint{3.364134in}{1.552865in}}%
\pgfpathlineto{\pgfqpoint{3.377909in}{1.548567in}}%
\pgfpathlineto{\pgfqpoint{3.391688in}{1.544468in}}%
\pgfpathlineto{\pgfqpoint{3.405470in}{1.540569in}}%
\pgfpathlineto{\pgfqpoint{3.413766in}{1.549255in}}%
\pgfpathlineto{\pgfqpoint{3.422055in}{1.558064in}}%
\pgfpathlineto{\pgfqpoint{3.430336in}{1.566991in}}%
\pgfpathlineto{\pgfqpoint{3.438609in}{1.576032in}}%
\pgfpathlineto{\pgfqpoint{3.424845in}{1.579433in}}%
\pgfpathlineto{\pgfqpoint{3.411085in}{1.583033in}}%
\pgfpathlineto{\pgfqpoint{3.397329in}{1.586833in}}%
\pgfpathlineto{\pgfqpoint{3.383577in}{1.590834in}}%
\pgfpathlineto{\pgfqpoint{3.375285in}{1.582279in}}%
\pgfpathlineto{\pgfqpoint{3.366986in}{1.573846in}}%
\pgfpathlineto{\pgfqpoint{3.358678in}{1.565540in}}%
\pgfpathlineto{\pgfqpoint{3.350362in}{1.557365in}}%
\pgfpathclose%
\pgfusepath{fill}%
\end{pgfscope}%
\begin{pgfscope}%
\pgfpathrectangle{\pgfqpoint{1.150000in}{0.150000in}}{\pgfqpoint{5.700000in}{5.700000in}}%
\pgfusepath{clip}%
\pgfsetbuttcap%
\pgfsetroundjoin%
\definecolor{currentfill}{rgb}{0.253935,0.265254,0.529983}%
\pgfsetfillcolor{currentfill}%
\pgfsetfillopacity{0.800000}%
\pgfsetlinewidth{0.000000pt}%
\definecolor{currentstroke}{rgb}{0.000000,0.000000,0.000000}%
\pgfsetstrokecolor{currentstroke}%
\pgfsetdash{}{0pt}%
\pgfpathmoveto{\pgfqpoint{4.261391in}{2.109978in}}%
\pgfpathlineto{\pgfqpoint{4.275402in}{2.116882in}}%
\pgfpathlineto{\pgfqpoint{4.289425in}{2.123972in}}%
\pgfpathlineto{\pgfqpoint{4.303461in}{2.131247in}}%
\pgfpathlineto{\pgfqpoint{4.317510in}{2.138707in}}%
\pgfpathlineto{\pgfqpoint{4.325489in}{2.151011in}}%
\pgfpathlineto{\pgfqpoint{4.333463in}{2.163226in}}%
\pgfpathlineto{\pgfqpoint{4.341432in}{2.175351in}}%
\pgfpathlineto{\pgfqpoint{4.349395in}{2.187384in}}%
\pgfpathlineto{\pgfqpoint{4.335347in}{2.179771in}}%
\pgfpathlineto{\pgfqpoint{4.321313in}{2.172344in}}%
\pgfpathlineto{\pgfqpoint{4.307292in}{2.165102in}}%
\pgfpathlineto{\pgfqpoint{4.293284in}{2.158046in}}%
\pgfpathlineto{\pgfqpoint{4.285319in}{2.146152in}}%
\pgfpathlineto{\pgfqpoint{4.277348in}{2.134176in}}%
\pgfpathlineto{\pgfqpoint{4.269372in}{2.122118in}}%
\pgfpathlineto{\pgfqpoint{4.261391in}{2.109978in}}%
\pgfpathclose%
\pgfusepath{fill}%
\end{pgfscope}%
\begin{pgfscope}%
\pgfpathrectangle{\pgfqpoint{1.150000in}{0.150000in}}{\pgfqpoint{5.700000in}{5.700000in}}%
\pgfusepath{clip}%
\pgfsetbuttcap%
\pgfsetroundjoin%
\definecolor{currentfill}{rgb}{0.283187,0.125848,0.444960}%
\pgfsetfillcolor{currentfill}%
\pgfsetfillopacity{0.800000}%
\pgfsetlinewidth{0.000000pt}%
\definecolor{currentstroke}{rgb}{0.000000,0.000000,0.000000}%
\pgfsetstrokecolor{currentstroke}%
\pgfsetdash{}{0pt}%
\pgfpathmoveto{\pgfqpoint{2.751466in}{1.836123in}}%
\pgfpathlineto{\pgfqpoint{2.765356in}{1.822015in}}%
\pgfpathlineto{\pgfqpoint{2.779241in}{1.808149in}}%
\pgfpathlineto{\pgfqpoint{2.793121in}{1.794525in}}%
\pgfpathlineto{\pgfqpoint{2.806997in}{1.781141in}}%
\pgfpathlineto{\pgfqpoint{2.815691in}{1.782429in}}%
\pgfpathlineto{\pgfqpoint{2.824369in}{1.783980in}}%
\pgfpathlineto{\pgfqpoint{2.833033in}{1.785788in}}%
\pgfpathlineto{\pgfqpoint{2.841682in}{1.787848in}}%
\pgfpathlineto{\pgfqpoint{2.827847in}{1.800624in}}%
\pgfpathlineto{\pgfqpoint{2.814007in}{1.813638in}}%
\pgfpathlineto{\pgfqpoint{2.800163in}{1.826894in}}%
\pgfpathlineto{\pgfqpoint{2.786314in}{1.840391in}}%
\pgfpathlineto{\pgfqpoint{2.777625in}{1.838928in}}%
\pgfpathlineto{\pgfqpoint{2.768921in}{1.837725in}}%
\pgfpathlineto{\pgfqpoint{2.760201in}{1.836788in}}%
\pgfpathlineto{\pgfqpoint{2.751466in}{1.836123in}}%
\pgfpathclose%
\pgfusepath{fill}%
\end{pgfscope}%
\begin{pgfscope}%
\pgfpathrectangle{\pgfqpoint{1.150000in}{0.150000in}}{\pgfqpoint{5.700000in}{5.700000in}}%
\pgfusepath{clip}%
\pgfsetbuttcap%
\pgfsetroundjoin%
\definecolor{currentfill}{rgb}{0.272594,0.025563,0.353093}%
\pgfsetfillcolor{currentfill}%
\pgfsetfillopacity{0.800000}%
\pgfsetlinewidth{0.000000pt}%
\definecolor{currentstroke}{rgb}{0.000000,0.000000,0.000000}%
\pgfsetstrokecolor{currentstroke}%
\pgfsetdash{}{0pt}%
\pgfpathmoveto{\pgfqpoint{3.062674in}{1.614776in}}%
\pgfpathlineto{\pgfqpoint{3.076473in}{1.605852in}}%
\pgfpathlineto{\pgfqpoint{3.090272in}{1.597143in}}%
\pgfpathlineto{\pgfqpoint{3.104070in}{1.588649in}}%
\pgfpathlineto{\pgfqpoint{3.117869in}{1.580368in}}%
\pgfpathlineto{\pgfqpoint{3.126335in}{1.585581in}}%
\pgfpathlineto{\pgfqpoint{3.134791in}{1.590992in}}%
\pgfpathlineto{\pgfqpoint{3.143236in}{1.596594in}}%
\pgfpathlineto{\pgfqpoint{3.151670in}{1.602383in}}%
\pgfpathlineto{\pgfqpoint{3.137901in}{1.610099in}}%
\pgfpathlineto{\pgfqpoint{3.124131in}{1.618027in}}%
\pgfpathlineto{\pgfqpoint{3.110362in}{1.626170in}}%
\pgfpathlineto{\pgfqpoint{3.096593in}{1.634528in}}%
\pgfpathlineto{\pgfqpoint{3.088130in}{1.629293in}}%
\pgfpathlineto{\pgfqpoint{3.079656in}{1.624252in}}%
\pgfpathlineto{\pgfqpoint{3.071171in}{1.619411in}}%
\pgfpathlineto{\pgfqpoint{3.062674in}{1.614776in}}%
\pgfpathclose%
\pgfusepath{fill}%
\end{pgfscope}%
\begin{pgfscope}%
\pgfpathrectangle{\pgfqpoint{1.150000in}{0.150000in}}{\pgfqpoint{5.700000in}{5.700000in}}%
\pgfusepath{clip}%
\pgfsetbuttcap%
\pgfsetroundjoin%
\definecolor{currentfill}{rgb}{0.277134,0.185228,0.489898}%
\pgfsetfillcolor{currentfill}%
\pgfsetfillopacity{0.800000}%
\pgfsetlinewidth{0.000000pt}%
\definecolor{currentstroke}{rgb}{0.000000,0.000000,0.000000}%
\pgfsetstrokecolor{currentstroke}%
\pgfsetdash{}{0pt}%
\pgfpathmoveto{\pgfqpoint{4.053541in}{1.915184in}}%
\pgfpathlineto{\pgfqpoint{4.067463in}{1.919950in}}%
\pgfpathlineto{\pgfqpoint{4.081396in}{1.924903in}}%
\pgfpathlineto{\pgfqpoint{4.095340in}{1.930042in}}%
\pgfpathlineto{\pgfqpoint{4.109296in}{1.935366in}}%
\pgfpathlineto{\pgfqpoint{4.117338in}{1.948096in}}%
\pgfpathlineto{\pgfqpoint{4.125375in}{1.960773in}}%
\pgfpathlineto{\pgfqpoint{4.133407in}{1.973395in}}%
\pgfpathlineto{\pgfqpoint{4.141434in}{1.985961in}}%
\pgfpathlineto{\pgfqpoint{4.127481in}{1.980387in}}%
\pgfpathlineto{\pgfqpoint{4.113539in}{1.975000in}}%
\pgfpathlineto{\pgfqpoint{4.099609in}{1.969799in}}%
\pgfpathlineto{\pgfqpoint{4.085691in}{1.964784in}}%
\pgfpathlineto{\pgfqpoint{4.077660in}{1.952455in}}%
\pgfpathlineto{\pgfqpoint{4.069625in}{1.940078in}}%
\pgfpathlineto{\pgfqpoint{4.061585in}{1.927653in}}%
\pgfpathlineto{\pgfqpoint{4.053541in}{1.915184in}}%
\pgfpathclose%
\pgfusepath{fill}%
\end{pgfscope}%
\begin{pgfscope}%
\pgfpathrectangle{\pgfqpoint{1.150000in}{0.150000in}}{\pgfqpoint{5.700000in}{5.700000in}}%
\pgfusepath{clip}%
\pgfsetbuttcap%
\pgfsetroundjoin%
\definecolor{currentfill}{rgb}{0.243113,0.292092,0.538516}%
\pgfsetfillcolor{currentfill}%
\pgfsetfillopacity{0.800000}%
\pgfsetlinewidth{0.000000pt}%
\definecolor{currentstroke}{rgb}{0.000000,0.000000,0.000000}%
\pgfsetstrokecolor{currentstroke}%
\pgfsetdash{}{0pt}%
\pgfpathmoveto{\pgfqpoint{2.415960in}{2.252297in}}%
\pgfpathlineto{\pgfqpoint{2.430046in}{2.231795in}}%
\pgfpathlineto{\pgfqpoint{2.444120in}{2.211584in}}%
\pgfpathlineto{\pgfqpoint{2.458184in}{2.191663in}}%
\pgfpathlineto{\pgfqpoint{2.472237in}{2.172028in}}%
\pgfpathlineto{\pgfqpoint{2.481211in}{2.169525in}}%
\pgfpathlineto{\pgfqpoint{2.490167in}{2.167340in}}%
\pgfpathlineto{\pgfqpoint{2.499102in}{2.165466in}}%
\pgfpathlineto{\pgfqpoint{2.508018in}{2.163897in}}%
\pgfpathlineto{\pgfqpoint{2.494018in}{2.182901in}}%
\pgfpathlineto{\pgfqpoint{2.480007in}{2.202191in}}%
\pgfpathlineto{\pgfqpoint{2.465985in}{2.221768in}}%
\pgfpathlineto{\pgfqpoint{2.451953in}{2.241636in}}%
\pgfpathlineto{\pgfqpoint{2.442985in}{2.243824in}}%
\pgfpathlineto{\pgfqpoint{2.433997in}{2.246326in}}%
\pgfpathlineto{\pgfqpoint{2.424989in}{2.249148in}}%
\pgfpathlineto{\pgfqpoint{2.415960in}{2.252297in}}%
\pgfpathclose%
\pgfusepath{fill}%
\end{pgfscope}%
\begin{pgfscope}%
\pgfpathrectangle{\pgfqpoint{1.150000in}{0.150000in}}{\pgfqpoint{5.700000in}{5.700000in}}%
\pgfusepath{clip}%
\pgfsetbuttcap%
\pgfsetroundjoin%
\definecolor{currentfill}{rgb}{0.157851,0.683765,0.501686}%
\pgfsetfillcolor{currentfill}%
\pgfsetfillopacity{0.800000}%
\pgfsetlinewidth{0.000000pt}%
\definecolor{currentstroke}{rgb}{0.000000,0.000000,0.000000}%
\pgfsetstrokecolor{currentstroke}%
\pgfsetdash{}{0pt}%
\pgfpathmoveto{\pgfqpoint{5.652521in}{3.343498in}}%
\pgfpathlineto{\pgfqpoint{5.667350in}{3.357799in}}%
\pgfpathlineto{\pgfqpoint{5.682200in}{3.372282in}}%
\pgfpathlineto{\pgfqpoint{5.697072in}{3.386947in}}%
\pgfpathlineto{\pgfqpoint{5.711966in}{3.401795in}}%
\pgfpathlineto{\pgfqpoint{5.719222in}{3.403452in}}%
\pgfpathlineto{\pgfqpoint{5.726470in}{3.405064in}}%
\pgfpathlineto{\pgfqpoint{5.733710in}{3.406635in}}%
\pgfpathlineto{\pgfqpoint{5.740941in}{3.408171in}}%
\pgfpathlineto{\pgfqpoint{5.726074in}{3.393825in}}%
\pgfpathlineto{\pgfqpoint{5.711229in}{3.379662in}}%
\pgfpathlineto{\pgfqpoint{5.696405in}{3.365679in}}%
\pgfpathlineto{\pgfqpoint{5.681603in}{3.351877in}}%
\pgfpathlineto{\pgfqpoint{5.674345in}{3.349829in}}%
\pgfpathlineto{\pgfqpoint{5.667078in}{3.347753in}}%
\pgfpathlineto{\pgfqpoint{5.659804in}{3.345644in}}%
\pgfpathlineto{\pgfqpoint{5.652521in}{3.343498in}}%
\pgfpathclose%
\pgfusepath{fill}%
\end{pgfscope}%
\begin{pgfscope}%
\pgfpathrectangle{\pgfqpoint{1.150000in}{0.150000in}}{\pgfqpoint{5.700000in}{5.700000in}}%
\pgfusepath{clip}%
\pgfsetbuttcap%
\pgfsetroundjoin%
\definecolor{currentfill}{rgb}{0.282910,0.105393,0.426902}%
\pgfsetfillcolor{currentfill}%
\pgfsetfillopacity{0.800000}%
\pgfsetlinewidth{0.000000pt}%
\definecolor{currentstroke}{rgb}{0.000000,0.000000,0.000000}%
\pgfsetstrokecolor{currentstroke}%
\pgfsetdash{}{0pt}%
\pgfpathmoveto{\pgfqpoint{2.806997in}{1.781141in}}%
\pgfpathlineto{\pgfqpoint{2.820868in}{1.767994in}}%
\pgfpathlineto{\pgfqpoint{2.834735in}{1.755085in}}%
\pgfpathlineto{\pgfqpoint{2.848598in}{1.742410in}}%
\pgfpathlineto{\pgfqpoint{2.862458in}{1.729969in}}%
\pgfpathlineto{\pgfqpoint{2.871112in}{1.731877in}}%
\pgfpathlineto{\pgfqpoint{2.879752in}{1.734039in}}%
\pgfpathlineto{\pgfqpoint{2.888377in}{1.736451in}}%
\pgfpathlineto{\pgfqpoint{2.896989in}{1.739104in}}%
\pgfpathlineto{\pgfqpoint{2.883167in}{1.750939in}}%
\pgfpathlineto{\pgfqpoint{2.869342in}{1.763007in}}%
\pgfpathlineto{\pgfqpoint{2.855514in}{1.775309in}}%
\pgfpathlineto{\pgfqpoint{2.841682in}{1.787848in}}%
\pgfpathlineto{\pgfqpoint{2.833033in}{1.785788in}}%
\pgfpathlineto{\pgfqpoint{2.824369in}{1.783980in}}%
\pgfpathlineto{\pgfqpoint{2.815691in}{1.782429in}}%
\pgfpathlineto{\pgfqpoint{2.806997in}{1.781141in}}%
\pgfpathclose%
\pgfusepath{fill}%
\end{pgfscope}%
\begin{pgfscope}%
\pgfpathrectangle{\pgfqpoint{1.150000in}{0.150000in}}{\pgfqpoint{5.700000in}{5.700000in}}%
\pgfusepath{clip}%
\pgfsetbuttcap%
\pgfsetroundjoin%
\definecolor{currentfill}{rgb}{0.172719,0.448791,0.557885}%
\pgfsetfillcolor{currentfill}%
\pgfsetfillopacity{0.800000}%
\pgfsetlinewidth{0.000000pt}%
\definecolor{currentstroke}{rgb}{0.000000,0.000000,0.000000}%
\pgfsetstrokecolor{currentstroke}%
\pgfsetdash{}{0pt}%
\pgfpathmoveto{\pgfqpoint{2.168558in}{2.704140in}}%
\pgfpathlineto{\pgfqpoint{2.182874in}{2.677877in}}%
\pgfpathlineto{\pgfqpoint{2.197172in}{2.651964in}}%
\pgfpathlineto{\pgfqpoint{2.211453in}{2.626398in}}%
\pgfpathlineto{\pgfqpoint{2.225717in}{2.601174in}}%
\pgfpathlineto{\pgfqpoint{2.234891in}{2.596802in}}%
\pgfpathlineto{\pgfqpoint{2.244041in}{2.592770in}}%
\pgfpathlineto{\pgfqpoint{2.253169in}{2.589072in}}%
\pgfpathlineto{\pgfqpoint{2.262276in}{2.585701in}}%
\pgfpathlineto{\pgfqpoint{2.248071in}{2.610303in}}%
\pgfpathlineto{\pgfqpoint{2.233850in}{2.635247in}}%
\pgfpathlineto{\pgfqpoint{2.219613in}{2.660535in}}%
\pgfpathlineto{\pgfqpoint{2.205359in}{2.686172in}}%
\pgfpathlineto{\pgfqpoint{2.196193in}{2.690152in}}%
\pgfpathlineto{\pgfqpoint{2.187005in}{2.694469in}}%
\pgfpathlineto{\pgfqpoint{2.177793in}{2.699130in}}%
\pgfpathlineto{\pgfqpoint{2.168558in}{2.704140in}}%
\pgfpathclose%
\pgfusepath{fill}%
\end{pgfscope}%
\begin{pgfscope}%
\pgfpathrectangle{\pgfqpoint{1.150000in}{0.150000in}}{\pgfqpoint{5.700000in}{5.700000in}}%
\pgfusepath{clip}%
\pgfsetbuttcap%
\pgfsetroundjoin%
\definecolor{currentfill}{rgb}{0.277941,0.056324,0.381191}%
\pgfsetfillcolor{currentfill}%
\pgfsetfillopacity{0.800000}%
\pgfsetlinewidth{0.000000pt}%
\definecolor{currentstroke}{rgb}{0.000000,0.000000,0.000000}%
\pgfsetstrokecolor{currentstroke}%
\pgfsetdash{}{0pt}%
\pgfpathmoveto{\pgfqpoint{3.669752in}{1.637226in}}%
\pgfpathlineto{\pgfqpoint{3.683562in}{1.637260in}}%
\pgfpathlineto{\pgfqpoint{3.697379in}{1.637486in}}%
\pgfpathlineto{\pgfqpoint{3.711203in}{1.637902in}}%
\pgfpathlineto{\pgfqpoint{3.725034in}{1.638509in}}%
\pgfpathlineto{\pgfqpoint{3.733197in}{1.650037in}}%
\pgfpathlineto{\pgfqpoint{3.741353in}{1.661604in}}%
\pgfpathlineto{\pgfqpoint{3.749504in}{1.673208in}}%
\pgfpathlineto{\pgfqpoint{3.757650in}{1.684843in}}%
\pgfpathlineto{\pgfqpoint{3.743828in}{1.683831in}}%
\pgfpathlineto{\pgfqpoint{3.730013in}{1.683010in}}%
\pgfpathlineto{\pgfqpoint{3.716207in}{1.682380in}}%
\pgfpathlineto{\pgfqpoint{3.702407in}{1.681941in}}%
\pgfpathlineto{\pgfqpoint{3.694252in}{1.670698in}}%
\pgfpathlineto{\pgfqpoint{3.686091in}{1.659496in}}%
\pgfpathlineto{\pgfqpoint{3.677925in}{1.648337in}}%
\pgfpathlineto{\pgfqpoint{3.669752in}{1.637226in}}%
\pgfpathclose%
\pgfusepath{fill}%
\end{pgfscope}%
\begin{pgfscope}%
\pgfpathrectangle{\pgfqpoint{1.150000in}{0.150000in}}{\pgfqpoint{5.700000in}{5.700000in}}%
\pgfusepath{clip}%
\pgfsetbuttcap%
\pgfsetroundjoin%
\definecolor{currentfill}{rgb}{0.149039,0.508051,0.557250}%
\pgfsetfillcolor{currentfill}%
\pgfsetfillopacity{0.800000}%
\pgfsetlinewidth{0.000000pt}%
\definecolor{currentstroke}{rgb}{0.000000,0.000000,0.000000}%
\pgfsetstrokecolor{currentstroke}%
\pgfsetdash{}{0pt}%
\pgfpathmoveto{\pgfqpoint{4.973103in}{2.793061in}}%
\pgfpathlineto{\pgfqpoint{4.987515in}{2.805161in}}%
\pgfpathlineto{\pgfqpoint{5.001944in}{2.817445in}}%
\pgfpathlineto{\pgfqpoint{5.016393in}{2.829913in}}%
\pgfpathlineto{\pgfqpoint{5.030859in}{2.842565in}}%
\pgfpathlineto{\pgfqpoint{5.038549in}{2.850072in}}%
\pgfpathlineto{\pgfqpoint{5.046230in}{2.857452in}}%
\pgfpathlineto{\pgfqpoint{5.053903in}{2.864708in}}%
\pgfpathlineto{\pgfqpoint{5.061568in}{2.871842in}}%
\pgfpathlineto{\pgfqpoint{5.047111in}{2.859375in}}%
\pgfpathlineto{\pgfqpoint{5.032672in}{2.847092in}}%
\pgfpathlineto{\pgfqpoint{5.018251in}{2.834993in}}%
\pgfpathlineto{\pgfqpoint{5.003848in}{2.823078in}}%
\pgfpathlineto{\pgfqpoint{4.996174in}{2.815746in}}%
\pgfpathlineto{\pgfqpoint{4.988491in}{2.808301in}}%
\pgfpathlineto{\pgfqpoint{4.980801in}{2.800740in}}%
\pgfpathlineto{\pgfqpoint{4.973103in}{2.793061in}}%
\pgfpathclose%
\pgfusepath{fill}%
\end{pgfscope}%
\begin{pgfscope}%
\pgfpathrectangle{\pgfqpoint{1.150000in}{0.150000in}}{\pgfqpoint{5.700000in}{5.700000in}}%
\pgfusepath{clip}%
\pgfsetbuttcap%
\pgfsetroundjoin%
\definecolor{currentfill}{rgb}{0.273809,0.031497,0.358853}%
\pgfsetfillcolor{currentfill}%
\pgfsetfillopacity{0.800000}%
\pgfsetlinewidth{0.000000pt}%
\definecolor{currentstroke}{rgb}{0.000000,0.000000,0.000000}%
\pgfsetstrokecolor{currentstroke}%
\pgfsetdash{}{0pt}%
\pgfpathmoveto{\pgfqpoint{3.581787in}{1.596858in}}%
\pgfpathlineto{\pgfqpoint{3.595582in}{1.595687in}}%
\pgfpathlineto{\pgfqpoint{3.609383in}{1.594710in}}%
\pgfpathlineto{\pgfqpoint{3.623191in}{1.593924in}}%
\pgfpathlineto{\pgfqpoint{3.637005in}{1.593331in}}%
\pgfpathlineto{\pgfqpoint{3.645201in}{1.604214in}}%
\pgfpathlineto{\pgfqpoint{3.653391in}{1.615160in}}%
\pgfpathlineto{\pgfqpoint{3.661574in}{1.626166in}}%
\pgfpathlineto{\pgfqpoint{3.669752in}{1.637226in}}%
\pgfpathlineto{\pgfqpoint{3.655950in}{1.637383in}}%
\pgfpathlineto{\pgfqpoint{3.642154in}{1.637732in}}%
\pgfpathlineto{\pgfqpoint{3.628365in}{1.638274in}}%
\pgfpathlineto{\pgfqpoint{3.614582in}{1.639009in}}%
\pgfpathlineto{\pgfqpoint{3.606393in}{1.628373in}}%
\pgfpathlineto{\pgfqpoint{3.598197in}{1.617800in}}%
\pgfpathlineto{\pgfqpoint{3.589995in}{1.607293in}}%
\pgfpathlineto{\pgfqpoint{3.581787in}{1.596858in}}%
\pgfpathclose%
\pgfusepath{fill}%
\end{pgfscope}%
\begin{pgfscope}%
\pgfpathrectangle{\pgfqpoint{1.150000in}{0.150000in}}{\pgfqpoint{5.700000in}{5.700000in}}%
\pgfusepath{clip}%
\pgfsetbuttcap%
\pgfsetroundjoin%
\definecolor{currentfill}{rgb}{0.121148,0.592739,0.544641}%
\pgfsetfillcolor{currentfill}%
\pgfsetfillopacity{0.800000}%
\pgfsetlinewidth{0.000000pt}%
\definecolor{currentstroke}{rgb}{0.000000,0.000000,0.000000}%
\pgfsetstrokecolor{currentstroke}%
\pgfsetdash{}{0pt}%
\pgfpathmoveto{\pgfqpoint{5.268980in}{3.049175in}}%
\pgfpathlineto{\pgfqpoint{5.283575in}{3.062529in}}%
\pgfpathlineto{\pgfqpoint{5.298189in}{3.076068in}}%
\pgfpathlineto{\pgfqpoint{5.312824in}{3.089790in}}%
\pgfpathlineto{\pgfqpoint{5.327479in}{3.103695in}}%
\pgfpathlineto{\pgfqpoint{5.334996in}{3.108570in}}%
\pgfpathlineto{\pgfqpoint{5.342505in}{3.113341in}}%
\pgfpathlineto{\pgfqpoint{5.350004in}{3.118011in}}%
\pgfpathlineto{\pgfqpoint{5.357494in}{3.122584in}}%
\pgfpathlineto{\pgfqpoint{5.342855in}{3.109005in}}%
\pgfpathlineto{\pgfqpoint{5.328236in}{3.095608in}}%
\pgfpathlineto{\pgfqpoint{5.313637in}{3.082394in}}%
\pgfpathlineto{\pgfqpoint{5.299058in}{3.069363in}}%
\pgfpathlineto{\pgfqpoint{5.291551in}{3.064452in}}%
\pgfpathlineto{\pgfqpoint{5.284036in}{3.059453in}}%
\pgfpathlineto{\pgfqpoint{5.276512in}{3.054362in}}%
\pgfpathlineto{\pgfqpoint{5.268980in}{3.049175in}}%
\pgfpathclose%
\pgfusepath{fill}%
\end{pgfscope}%
\begin{pgfscope}%
\pgfpathrectangle{\pgfqpoint{1.150000in}{0.150000in}}{\pgfqpoint{5.700000in}{5.700000in}}%
\pgfusepath{clip}%
\pgfsetbuttcap%
\pgfsetroundjoin%
\definecolor{currentfill}{rgb}{0.281446,0.084320,0.407414}%
\pgfsetfillcolor{currentfill}%
\pgfsetfillopacity{0.800000}%
\pgfsetlinewidth{0.000000pt}%
\definecolor{currentstroke}{rgb}{0.000000,0.000000,0.000000}%
\pgfsetstrokecolor{currentstroke}%
\pgfsetdash{}{0pt}%
\pgfpathmoveto{\pgfqpoint{3.757650in}{1.684843in}}%
\pgfpathlineto{\pgfqpoint{3.771480in}{1.686044in}}%
\pgfpathlineto{\pgfqpoint{3.785319in}{1.687435in}}%
\pgfpathlineto{\pgfqpoint{3.799166in}{1.689015in}}%
\pgfpathlineto{\pgfqpoint{3.813021in}{1.690784in}}%
\pgfpathlineto{\pgfqpoint{3.821153in}{1.702833in}}%
\pgfpathlineto{\pgfqpoint{3.829281in}{1.714899in}}%
\pgfpathlineto{\pgfqpoint{3.837403in}{1.726979in}}%
\pgfpathlineto{\pgfqpoint{3.845520in}{1.739068in}}%
\pgfpathlineto{\pgfqpoint{3.831673in}{1.736925in}}%
\pgfpathlineto{\pgfqpoint{3.817834in}{1.734970in}}%
\pgfpathlineto{\pgfqpoint{3.804003in}{1.733206in}}%
\pgfpathlineto{\pgfqpoint{3.790181in}{1.731630in}}%
\pgfpathlineto{\pgfqpoint{3.782056in}{1.719903in}}%
\pgfpathlineto{\pgfqpoint{3.773926in}{1.708194in}}%
\pgfpathlineto{\pgfqpoint{3.765791in}{1.696506in}}%
\pgfpathlineto{\pgfqpoint{3.757650in}{1.684843in}}%
\pgfpathclose%
\pgfusepath{fill}%
\end{pgfscope}%
\begin{pgfscope}%
\pgfpathrectangle{\pgfqpoint{1.150000in}{0.150000in}}{\pgfqpoint{5.700000in}{5.700000in}}%
\pgfusepath{clip}%
\pgfsetbuttcap%
\pgfsetroundjoin%
\definecolor{currentfill}{rgb}{0.227802,0.326594,0.546532}%
\pgfsetfillcolor{currentfill}%
\pgfsetfillopacity{0.800000}%
\pgfsetlinewidth{0.000000pt}%
\definecolor{currentstroke}{rgb}{0.000000,0.000000,0.000000}%
\pgfsetstrokecolor{currentstroke}%
\pgfsetdash{}{0pt}%
\pgfpathmoveto{\pgfqpoint{2.359497in}{2.337272in}}%
\pgfpathlineto{\pgfqpoint{2.373631in}{2.315578in}}%
\pgfpathlineto{\pgfqpoint{2.387753in}{2.294186in}}%
\pgfpathlineto{\pgfqpoint{2.401862in}{2.273093in}}%
\pgfpathlineto{\pgfqpoint{2.415960in}{2.252297in}}%
\pgfpathlineto{\pgfqpoint{2.424989in}{2.249148in}}%
\pgfpathlineto{\pgfqpoint{2.433997in}{2.246326in}}%
\pgfpathlineto{\pgfqpoint{2.442985in}{2.243824in}}%
\pgfpathlineto{\pgfqpoint{2.451953in}{2.241636in}}%
\pgfpathlineto{\pgfqpoint{2.437910in}{2.261796in}}%
\pgfpathlineto{\pgfqpoint{2.423855in}{2.282251in}}%
\pgfpathlineto{\pgfqpoint{2.409789in}{2.303005in}}%
\pgfpathlineto{\pgfqpoint{2.395711in}{2.324059in}}%
\pgfpathlineto{\pgfqpoint{2.386689in}{2.326871in}}%
\pgfpathlineto{\pgfqpoint{2.377646in}{2.330006in}}%
\pgfpathlineto{\pgfqpoint{2.368582in}{2.333471in}}%
\pgfpathlineto{\pgfqpoint{2.359497in}{2.337272in}}%
\pgfpathclose%
\pgfusepath{fill}%
\end{pgfscope}%
\begin{pgfscope}%
\pgfpathrectangle{\pgfqpoint{1.150000in}{0.150000in}}{\pgfqpoint{5.700000in}{5.700000in}}%
\pgfusepath{clip}%
\pgfsetbuttcap%
\pgfsetroundjoin%
\definecolor{currentfill}{rgb}{0.185783,0.704891,0.485273}%
\pgfsetfillcolor{currentfill}%
\pgfsetfillopacity{0.800000}%
\pgfsetlinewidth{0.000000pt}%
\definecolor{currentstroke}{rgb}{0.000000,0.000000,0.000000}%
\pgfsetstrokecolor{currentstroke}%
\pgfsetdash{}{0pt}%
\pgfpathmoveto{\pgfqpoint{5.740941in}{3.408171in}}%
\pgfpathlineto{\pgfqpoint{5.755830in}{3.422698in}}%
\pgfpathlineto{\pgfqpoint{5.770741in}{3.437406in}}%
\pgfpathlineto{\pgfqpoint{5.785675in}{3.452297in}}%
\pgfpathlineto{\pgfqpoint{5.800631in}{3.467371in}}%
\pgfpathlineto{\pgfqpoint{5.807826in}{3.468352in}}%
\pgfpathlineto{\pgfqpoint{5.815012in}{3.469303in}}%
\pgfpathlineto{\pgfqpoint{5.822191in}{3.470227in}}%
\pgfpathlineto{\pgfqpoint{5.829361in}{3.471132in}}%
\pgfpathlineto{\pgfqpoint{5.814435in}{3.456597in}}%
\pgfpathlineto{\pgfqpoint{5.799531in}{3.442244in}}%
\pgfpathlineto{\pgfqpoint{5.784649in}{3.428071in}}%
\pgfpathlineto{\pgfqpoint{5.769789in}{3.414079in}}%
\pgfpathlineto{\pgfqpoint{5.762588in}{3.412626in}}%
\pgfpathlineto{\pgfqpoint{5.755380in}{3.411161in}}%
\pgfpathlineto{\pgfqpoint{5.748164in}{3.409677in}}%
\pgfpathlineto{\pgfqpoint{5.740941in}{3.408171in}}%
\pgfpathclose%
\pgfusepath{fill}%
\end{pgfscope}%
\begin{pgfscope}%
\pgfpathrectangle{\pgfqpoint{1.150000in}{0.150000in}}{\pgfqpoint{5.700000in}{5.700000in}}%
\pgfusepath{clip}%
\pgfsetbuttcap%
\pgfsetroundjoin%
\definecolor{currentfill}{rgb}{0.281446,0.084320,0.407414}%
\pgfsetfillcolor{currentfill}%
\pgfsetfillopacity{0.800000}%
\pgfsetlinewidth{0.000000pt}%
\definecolor{currentstroke}{rgb}{0.000000,0.000000,0.000000}%
\pgfsetstrokecolor{currentstroke}%
\pgfsetdash{}{0pt}%
\pgfpathmoveto{\pgfqpoint{2.862458in}{1.729969in}}%
\pgfpathlineto{\pgfqpoint{2.876314in}{1.717761in}}%
\pgfpathlineto{\pgfqpoint{2.890167in}{1.705783in}}%
\pgfpathlineto{\pgfqpoint{2.904017in}{1.694034in}}%
\pgfpathlineto{\pgfqpoint{2.917864in}{1.682514in}}%
\pgfpathlineto{\pgfqpoint{2.926481in}{1.685039in}}%
\pgfpathlineto{\pgfqpoint{2.935083in}{1.687811in}}%
\pgfpathlineto{\pgfqpoint{2.943672in}{1.690822in}}%
\pgfpathlineto{\pgfqpoint{2.952247in}{1.694068in}}%
\pgfpathlineto{\pgfqpoint{2.938436in}{1.704984in}}%
\pgfpathlineto{\pgfqpoint{2.924623in}{1.716128in}}%
\pgfpathlineto{\pgfqpoint{2.910807in}{1.727501in}}%
\pgfpathlineto{\pgfqpoint{2.896989in}{1.739104in}}%
\pgfpathlineto{\pgfqpoint{2.888377in}{1.736451in}}%
\pgfpathlineto{\pgfqpoint{2.879752in}{1.734039in}}%
\pgfpathlineto{\pgfqpoint{2.871112in}{1.731877in}}%
\pgfpathlineto{\pgfqpoint{2.862458in}{1.729969in}}%
\pgfpathclose%
\pgfusepath{fill}%
\end{pgfscope}%
\begin{pgfscope}%
\pgfpathrectangle{\pgfqpoint{1.150000in}{0.150000in}}{\pgfqpoint{5.700000in}{5.700000in}}%
\pgfusepath{clip}%
\pgfsetbuttcap%
\pgfsetroundjoin%
\definecolor{currentfill}{rgb}{0.269944,0.014625,0.341379}%
\pgfsetfillcolor{currentfill}%
\pgfsetfillopacity{0.800000}%
\pgfsetlinewidth{0.000000pt}%
\definecolor{currentstroke}{rgb}{0.000000,0.000000,0.000000}%
\pgfsetstrokecolor{currentstroke}%
\pgfsetdash{}{0pt}%
\pgfpathmoveto{\pgfqpoint{3.493709in}{1.564407in}}%
\pgfpathlineto{\pgfqpoint{3.507497in}{1.561991in}}%
\pgfpathlineto{\pgfqpoint{3.521289in}{1.559771in}}%
\pgfpathlineto{\pgfqpoint{3.535087in}{1.557746in}}%
\pgfpathlineto{\pgfqpoint{3.548890in}{1.555914in}}%
\pgfpathlineto{\pgfqpoint{3.557124in}{1.566022in}}%
\pgfpathlineto{\pgfqpoint{3.565352in}{1.576218in}}%
\pgfpathlineto{\pgfqpoint{3.573572in}{1.586498in}}%
\pgfpathlineto{\pgfqpoint{3.581787in}{1.596858in}}%
\pgfpathlineto{\pgfqpoint{3.567998in}{1.598222in}}%
\pgfpathlineto{\pgfqpoint{3.554214in}{1.599781in}}%
\pgfpathlineto{\pgfqpoint{3.540437in}{1.601534in}}%
\pgfpathlineto{\pgfqpoint{3.526665in}{1.603482in}}%
\pgfpathlineto{\pgfqpoint{3.518436in}{1.593577in}}%
\pgfpathlineto{\pgfqpoint{3.510201in}{1.583760in}}%
\pgfpathlineto{\pgfqpoint{3.501959in}{1.574035in}}%
\pgfpathlineto{\pgfqpoint{3.493709in}{1.564407in}}%
\pgfpathclose%
\pgfusepath{fill}%
\end{pgfscope}%
\begin{pgfscope}%
\pgfpathrectangle{\pgfqpoint{1.150000in}{0.150000in}}{\pgfqpoint{5.700000in}{5.700000in}}%
\pgfusepath{clip}%
\pgfsetbuttcap%
\pgfsetroundjoin%
\definecolor{currentfill}{rgb}{0.239346,0.300855,0.540844}%
\pgfsetfillcolor{currentfill}%
\pgfsetfillopacity{0.800000}%
\pgfsetlinewidth{0.000000pt}%
\definecolor{currentstroke}{rgb}{0.000000,0.000000,0.000000}%
\pgfsetstrokecolor{currentstroke}%
\pgfsetdash{}{0pt}%
\pgfpathmoveto{\pgfqpoint{4.349395in}{2.187384in}}%
\pgfpathlineto{\pgfqpoint{4.363456in}{2.195182in}}%
\pgfpathlineto{\pgfqpoint{4.377532in}{2.203165in}}%
\pgfpathlineto{\pgfqpoint{4.391621in}{2.211333in}}%
\pgfpathlineto{\pgfqpoint{4.405724in}{2.219686in}}%
\pgfpathlineto{\pgfqpoint{4.413681in}{2.231759in}}%
\pgfpathlineto{\pgfqpoint{4.421632in}{2.243730in}}%
\pgfpathlineto{\pgfqpoint{4.429577in}{2.255600in}}%
\pgfpathlineto{\pgfqpoint{4.437517in}{2.267366in}}%
\pgfpathlineto{\pgfqpoint{4.423416in}{2.258893in}}%
\pgfpathlineto{\pgfqpoint{4.409328in}{2.250605in}}%
\pgfpathlineto{\pgfqpoint{4.395255in}{2.242503in}}%
\pgfpathlineto{\pgfqpoint{4.381195in}{2.234585in}}%
\pgfpathlineto{\pgfqpoint{4.373253in}{2.222926in}}%
\pgfpathlineto{\pgfqpoint{4.365306in}{2.211172in}}%
\pgfpathlineto{\pgfqpoint{4.357353in}{2.199325in}}%
\pgfpathlineto{\pgfqpoint{4.349395in}{2.187384in}}%
\pgfpathclose%
\pgfusepath{fill}%
\end{pgfscope}%
\begin{pgfscope}%
\pgfpathrectangle{\pgfqpoint{1.150000in}{0.150000in}}{\pgfqpoint{5.700000in}{5.700000in}}%
\pgfusepath{clip}%
\pgfsetbuttcap%
\pgfsetroundjoin%
\definecolor{currentfill}{rgb}{0.283091,0.110553,0.431554}%
\pgfsetfillcolor{currentfill}%
\pgfsetfillopacity{0.800000}%
\pgfsetlinewidth{0.000000pt}%
\definecolor{currentstroke}{rgb}{0.000000,0.000000,0.000000}%
\pgfsetstrokecolor{currentstroke}%
\pgfsetdash{}{0pt}%
\pgfpathmoveto{\pgfqpoint{3.845520in}{1.739068in}}%
\pgfpathlineto{\pgfqpoint{3.859377in}{1.741399in}}%
\pgfpathlineto{\pgfqpoint{3.873243in}{1.743919in}}%
\pgfpathlineto{\pgfqpoint{3.887118in}{1.746627in}}%
\pgfpathlineto{\pgfqpoint{3.901003in}{1.749522in}}%
\pgfpathlineto{\pgfqpoint{3.909109in}{1.761973in}}%
\pgfpathlineto{\pgfqpoint{3.917210in}{1.774420in}}%
\pgfpathlineto{\pgfqpoint{3.925307in}{1.786859in}}%
\pgfpathlineto{\pgfqpoint{3.933398in}{1.799287in}}%
\pgfpathlineto{\pgfqpoint{3.919519in}{1.796048in}}%
\pgfpathlineto{\pgfqpoint{3.905650in}{1.792997in}}%
\pgfpathlineto{\pgfqpoint{3.891791in}{1.790134in}}%
\pgfpathlineto{\pgfqpoint{3.877940in}{1.787460in}}%
\pgfpathlineto{\pgfqpoint{3.869843in}{1.775363in}}%
\pgfpathlineto{\pgfqpoint{3.861740in}{1.763263in}}%
\pgfpathlineto{\pgfqpoint{3.853633in}{1.751164in}}%
\pgfpathlineto{\pgfqpoint{3.845520in}{1.739068in}}%
\pgfpathclose%
\pgfusepath{fill}%
\end{pgfscope}%
\begin{pgfscope}%
\pgfpathrectangle{\pgfqpoint{1.150000in}{0.150000in}}{\pgfqpoint{5.700000in}{5.700000in}}%
\pgfusepath{clip}%
\pgfsetbuttcap%
\pgfsetroundjoin%
\definecolor{currentfill}{rgb}{0.172719,0.448791,0.557885}%
\pgfsetfillcolor{currentfill}%
\pgfsetfillopacity{0.800000}%
\pgfsetlinewidth{0.000000pt}%
\definecolor{currentstroke}{rgb}{0.000000,0.000000,0.000000}%
\pgfsetstrokecolor{currentstroke}%
\pgfsetdash{}{0pt}%
\pgfpathmoveto{\pgfqpoint{4.765319in}{2.598009in}}%
\pgfpathlineto{\pgfqpoint{4.779609in}{2.608982in}}%
\pgfpathlineto{\pgfqpoint{4.793916in}{2.620140in}}%
\pgfpathlineto{\pgfqpoint{4.808241in}{2.631483in}}%
\pgfpathlineto{\pgfqpoint{4.822583in}{2.643009in}}%
\pgfpathlineto{\pgfqpoint{4.830377in}{2.652326in}}%
\pgfpathlineto{\pgfqpoint{4.838165in}{2.661512in}}%
\pgfpathlineto{\pgfqpoint{4.845946in}{2.670568in}}%
\pgfpathlineto{\pgfqpoint{4.853719in}{2.679497in}}%
\pgfpathlineto{\pgfqpoint{4.839382in}{2.668052in}}%
\pgfpathlineto{\pgfqpoint{4.825063in}{2.656791in}}%
\pgfpathlineto{\pgfqpoint{4.810761in}{2.645714in}}%
\pgfpathlineto{\pgfqpoint{4.796476in}{2.634822in}}%
\pgfpathlineto{\pgfqpoint{4.788697in}{2.625799in}}%
\pgfpathlineto{\pgfqpoint{4.780911in}{2.616657in}}%
\pgfpathlineto{\pgfqpoint{4.773119in}{2.607394in}}%
\pgfpathlineto{\pgfqpoint{4.765319in}{2.598009in}}%
\pgfpathclose%
\pgfusepath{fill}%
\end{pgfscope}%
\begin{pgfscope}%
\pgfpathrectangle{\pgfqpoint{1.150000in}{0.150000in}}{\pgfqpoint{5.700000in}{5.700000in}}%
\pgfusepath{clip}%
\pgfsetbuttcap%
\pgfsetroundjoin%
\definecolor{currentfill}{rgb}{0.269308,0.218818,0.509577}%
\pgfsetfillcolor{currentfill}%
\pgfsetfillopacity{0.800000}%
\pgfsetlinewidth{0.000000pt}%
\definecolor{currentstroke}{rgb}{0.000000,0.000000,0.000000}%
\pgfsetstrokecolor{currentstroke}%
\pgfsetdash{}{0pt}%
\pgfpathmoveto{\pgfqpoint{4.141434in}{1.985961in}}%
\pgfpathlineto{\pgfqpoint{4.155399in}{1.991720in}}%
\pgfpathlineto{\pgfqpoint{4.169376in}{1.997666in}}%
\pgfpathlineto{\pgfqpoint{4.183365in}{2.003797in}}%
\pgfpathlineto{\pgfqpoint{4.197366in}{2.010114in}}%
\pgfpathlineto{\pgfqpoint{4.205387in}{2.022850in}}%
\pgfpathlineto{\pgfqpoint{4.213402in}{2.035517in}}%
\pgfpathlineto{\pgfqpoint{4.221413in}{2.048113in}}%
\pgfpathlineto{\pgfqpoint{4.229418in}{2.060637in}}%
\pgfpathlineto{\pgfqpoint{4.215419in}{2.054103in}}%
\pgfpathlineto{\pgfqpoint{4.201432in}{2.047755in}}%
\pgfpathlineto{\pgfqpoint{4.187457in}{2.041592in}}%
\pgfpathlineto{\pgfqpoint{4.173495in}{2.035616in}}%
\pgfpathlineto{\pgfqpoint{4.165487in}{2.023297in}}%
\pgfpathlineto{\pgfqpoint{4.157474in}{2.010913in}}%
\pgfpathlineto{\pgfqpoint{4.149456in}{1.998467in}}%
\pgfpathlineto{\pgfqpoint{4.141434in}{1.985961in}}%
\pgfpathclose%
\pgfusepath{fill}%
\end{pgfscope}%
\begin{pgfscope}%
\pgfpathrectangle{\pgfqpoint{1.150000in}{0.150000in}}{\pgfqpoint{5.700000in}{5.700000in}}%
\pgfusepath{clip}%
\pgfsetbuttcap%
\pgfsetroundjoin%
\definecolor{currentfill}{rgb}{0.267004,0.004874,0.329415}%
\pgfsetfillcolor{currentfill}%
\pgfsetfillopacity{0.800000}%
\pgfsetlinewidth{0.000000pt}%
\definecolor{currentstroke}{rgb}{0.000000,0.000000,0.000000}%
\pgfsetstrokecolor{currentstroke}%
\pgfsetdash{}{0pt}%
\pgfpathmoveto{\pgfqpoint{3.261869in}{1.548216in}}%
\pgfpathlineto{\pgfqpoint{3.275652in}{1.542375in}}%
\pgfpathlineto{\pgfqpoint{3.289437in}{1.536738in}}%
\pgfpathlineto{\pgfqpoint{3.303224in}{1.531305in}}%
\pgfpathlineto{\pgfqpoint{3.317015in}{1.526073in}}%
\pgfpathlineto{\pgfqpoint{3.325365in}{1.533675in}}%
\pgfpathlineto{\pgfqpoint{3.333706in}{1.541427in}}%
\pgfpathlineto{\pgfqpoint{3.342038in}{1.549325in}}%
\pgfpathlineto{\pgfqpoint{3.350362in}{1.557365in}}%
\pgfpathlineto{\pgfqpoint{3.336594in}{1.562065in}}%
\pgfpathlineto{\pgfqpoint{3.322829in}{1.566968in}}%
\pgfpathlineto{\pgfqpoint{3.309066in}{1.572073in}}%
\pgfpathlineto{\pgfqpoint{3.295307in}{1.577383in}}%
\pgfpathlineto{\pgfqpoint{3.286961in}{1.569862in}}%
\pgfpathlineto{\pgfqpoint{3.278606in}{1.562491in}}%
\pgfpathlineto{\pgfqpoint{3.270242in}{1.555274in}}%
\pgfpathlineto{\pgfqpoint{3.261869in}{1.548216in}}%
\pgfpathclose%
\pgfusepath{fill}%
\end{pgfscope}%
\begin{pgfscope}%
\pgfpathrectangle{\pgfqpoint{1.150000in}{0.150000in}}{\pgfqpoint{5.700000in}{5.700000in}}%
\pgfusepath{clip}%
\pgfsetbuttcap%
\pgfsetroundjoin%
\definecolor{currentfill}{rgb}{0.203063,0.379716,0.553925}%
\pgfsetfillcolor{currentfill}%
\pgfsetfillopacity{0.800000}%
\pgfsetlinewidth{0.000000pt}%
\definecolor{currentstroke}{rgb}{0.000000,0.000000,0.000000}%
\pgfsetstrokecolor{currentstroke}%
\pgfsetdash{}{0pt}%
\pgfpathmoveto{\pgfqpoint{4.557370in}{2.394034in}}%
\pgfpathlineto{\pgfqpoint{4.571542in}{2.403571in}}%
\pgfpathlineto{\pgfqpoint{4.585730in}{2.413293in}}%
\pgfpathlineto{\pgfqpoint{4.599933in}{2.423200in}}%
\pgfpathlineto{\pgfqpoint{4.614151in}{2.433292in}}%
\pgfpathlineto{\pgfqpoint{4.622034in}{2.444179in}}%
\pgfpathlineto{\pgfqpoint{4.629911in}{2.454943in}}%
\pgfpathlineto{\pgfqpoint{4.637782in}{2.465586in}}%
\pgfpathlineto{\pgfqpoint{4.645646in}{2.476107in}}%
\pgfpathlineto{\pgfqpoint{4.631430in}{2.465995in}}%
\pgfpathlineto{\pgfqpoint{4.617230in}{2.456068in}}%
\pgfpathlineto{\pgfqpoint{4.603045in}{2.446325in}}%
\pgfpathlineto{\pgfqpoint{4.588876in}{2.436767in}}%
\pgfpathlineto{\pgfqpoint{4.581009in}{2.426254in}}%
\pgfpathlineto{\pgfqpoint{4.573136in}{2.415628in}}%
\pgfpathlineto{\pgfqpoint{4.565256in}{2.404888in}}%
\pgfpathlineto{\pgfqpoint{4.557370in}{2.394034in}}%
\pgfpathclose%
\pgfusepath{fill}%
\end{pgfscope}%
\begin{pgfscope}%
\pgfpathrectangle{\pgfqpoint{1.150000in}{0.150000in}}{\pgfqpoint{5.700000in}{5.700000in}}%
\pgfusepath{clip}%
\pgfsetbuttcap%
\pgfsetroundjoin%
\definecolor{currentfill}{rgb}{0.269944,0.014625,0.341379}%
\pgfsetfillcolor{currentfill}%
\pgfsetfillopacity{0.800000}%
\pgfsetlinewidth{0.000000pt}%
\definecolor{currentstroke}{rgb}{0.000000,0.000000,0.000000}%
\pgfsetstrokecolor{currentstroke}%
\pgfsetdash{}{0pt}%
\pgfpathmoveto{\pgfqpoint{3.117869in}{1.580368in}}%
\pgfpathlineto{\pgfqpoint{3.131667in}{1.572299in}}%
\pgfpathlineto{\pgfqpoint{3.145467in}{1.564442in}}%
\pgfpathlineto{\pgfqpoint{3.159266in}{1.556795in}}%
\pgfpathlineto{\pgfqpoint{3.173067in}{1.549358in}}%
\pgfpathlineto{\pgfqpoint{3.181505in}{1.555149in}}%
\pgfpathlineto{\pgfqpoint{3.189933in}{1.561128in}}%
\pgfpathlineto{\pgfqpoint{3.198351in}{1.567292in}}%
\pgfpathlineto{\pgfqpoint{3.206759in}{1.573633in}}%
\pgfpathlineto{\pgfqpoint{3.192985in}{1.580506in}}%
\pgfpathlineto{\pgfqpoint{3.179212in}{1.587588in}}%
\pgfpathlineto{\pgfqpoint{3.165441in}{1.594880in}}%
\pgfpathlineto{\pgfqpoint{3.151670in}{1.602383in}}%
\pgfpathlineto{\pgfqpoint{3.143236in}{1.596594in}}%
\pgfpathlineto{\pgfqpoint{3.134791in}{1.590992in}}%
\pgfpathlineto{\pgfqpoint{3.126335in}{1.585581in}}%
\pgfpathlineto{\pgfqpoint{3.117869in}{1.580368in}}%
\pgfpathclose%
\pgfusepath{fill}%
\end{pgfscope}%
\begin{pgfscope}%
\pgfpathrectangle{\pgfqpoint{1.150000in}{0.150000in}}{\pgfqpoint{5.700000in}{5.700000in}}%
\pgfusepath{clip}%
\pgfsetbuttcap%
\pgfsetroundjoin%
\definecolor{currentfill}{rgb}{0.220124,0.725509,0.466226}%
\pgfsetfillcolor{currentfill}%
\pgfsetfillopacity{0.800000}%
\pgfsetlinewidth{0.000000pt}%
\definecolor{currentstroke}{rgb}{0.000000,0.000000,0.000000}%
\pgfsetstrokecolor{currentstroke}%
\pgfsetdash{}{0pt}%
\pgfpathmoveto{\pgfqpoint{5.829361in}{3.471132in}}%
\pgfpathlineto{\pgfqpoint{5.844310in}{3.485848in}}%
\pgfpathlineto{\pgfqpoint{5.859281in}{3.500746in}}%
\pgfpathlineto{\pgfqpoint{5.874276in}{3.515826in}}%
\pgfpathlineto{\pgfqpoint{5.889293in}{3.531088in}}%
\pgfpathlineto{\pgfqpoint{5.896424in}{3.531418in}}%
\pgfpathlineto{\pgfqpoint{5.903548in}{3.531732in}}%
\pgfpathlineto{\pgfqpoint{5.910664in}{3.532037in}}%
\pgfpathlineto{\pgfqpoint{5.917773in}{3.532339in}}%
\pgfpathlineto{\pgfqpoint{5.902788in}{3.517651in}}%
\pgfpathlineto{\pgfqpoint{5.887826in}{3.503145in}}%
\pgfpathlineto{\pgfqpoint{5.872887in}{3.488819in}}%
\pgfpathlineto{\pgfqpoint{5.857970in}{3.474673in}}%
\pgfpathlineto{\pgfqpoint{5.850828in}{3.473787in}}%
\pgfpathlineto{\pgfqpoint{5.843680in}{3.472906in}}%
\pgfpathlineto{\pgfqpoint{5.836524in}{3.472023in}}%
\pgfpathlineto{\pgfqpoint{5.829361in}{3.471132in}}%
\pgfpathclose%
\pgfusepath{fill}%
\end{pgfscope}%
\begin{pgfscope}%
\pgfpathrectangle{\pgfqpoint{1.150000in}{0.150000in}}{\pgfqpoint{5.700000in}{5.700000in}}%
\pgfusepath{clip}%
\pgfsetbuttcap%
\pgfsetroundjoin%
\definecolor{currentfill}{rgb}{0.119699,0.618490,0.536347}%
\pgfsetfillcolor{currentfill}%
\pgfsetfillopacity{0.800000}%
\pgfsetlinewidth{0.000000pt}%
\definecolor{currentstroke}{rgb}{0.000000,0.000000,0.000000}%
\pgfsetstrokecolor{currentstroke}%
\pgfsetdash{}{0pt}%
\pgfpathmoveto{\pgfqpoint{5.357494in}{3.122584in}}%
\pgfpathlineto{\pgfqpoint{5.372154in}{3.136347in}}%
\pgfpathlineto{\pgfqpoint{5.386834in}{3.150293in}}%
\pgfpathlineto{\pgfqpoint{5.401535in}{3.164423in}}%
\pgfpathlineto{\pgfqpoint{5.416257in}{3.178736in}}%
\pgfpathlineto{\pgfqpoint{5.423721in}{3.182868in}}%
\pgfpathlineto{\pgfqpoint{5.431177in}{3.186903in}}%
\pgfpathlineto{\pgfqpoint{5.438624in}{3.190845in}}%
\pgfpathlineto{\pgfqpoint{5.446061in}{3.194698in}}%
\pgfpathlineto{\pgfqpoint{5.431358in}{3.180746in}}%
\pgfpathlineto{\pgfqpoint{5.416675in}{3.166978in}}%
\pgfpathlineto{\pgfqpoint{5.402013in}{3.153392in}}%
\pgfpathlineto{\pgfqpoint{5.387371in}{3.139989in}}%
\pgfpathlineto{\pgfqpoint{5.379914in}{3.135763in}}%
\pgfpathlineto{\pgfqpoint{5.372450in}{3.131456in}}%
\pgfpathlineto{\pgfqpoint{5.364976in}{3.127065in}}%
\pgfpathlineto{\pgfqpoint{5.357494in}{3.122584in}}%
\pgfpathclose%
\pgfusepath{fill}%
\end{pgfscope}%
\begin{pgfscope}%
\pgfpathrectangle{\pgfqpoint{1.150000in}{0.150000in}}{\pgfqpoint{5.700000in}{5.700000in}}%
\pgfusepath{clip}%
\pgfsetbuttcap%
\pgfsetroundjoin%
\definecolor{currentfill}{rgb}{0.282623,0.140926,0.457517}%
\pgfsetfillcolor{currentfill}%
\pgfsetfillopacity{0.800000}%
\pgfsetlinewidth{0.000000pt}%
\definecolor{currentstroke}{rgb}{0.000000,0.000000,0.000000}%
\pgfsetstrokecolor{currentstroke}%
\pgfsetdash{}{0pt}%
\pgfpathmoveto{\pgfqpoint{3.933398in}{1.799287in}}%
\pgfpathlineto{\pgfqpoint{3.947287in}{1.802713in}}%
\pgfpathlineto{\pgfqpoint{3.961186in}{1.806326in}}%
\pgfpathlineto{\pgfqpoint{3.975095in}{1.810126in}}%
\pgfpathlineto{\pgfqpoint{3.989014in}{1.814113in}}%
\pgfpathlineto{\pgfqpoint{3.997096in}{1.826851in}}%
\pgfpathlineto{\pgfqpoint{4.005174in}{1.839565in}}%
\pgfpathlineto{\pgfqpoint{4.013246in}{1.852252in}}%
\pgfpathlineto{\pgfqpoint{4.021315in}{1.864909in}}%
\pgfpathlineto{\pgfqpoint{4.007400in}{1.860610in}}%
\pgfpathlineto{\pgfqpoint{3.993495in}{1.856498in}}%
\pgfpathlineto{\pgfqpoint{3.979601in}{1.852572in}}%
\pgfpathlineto{\pgfqpoint{3.965717in}{1.848835in}}%
\pgfpathlineto{\pgfqpoint{3.957645in}{1.836478in}}%
\pgfpathlineto{\pgfqpoint{3.949567in}{1.824099in}}%
\pgfpathlineto{\pgfqpoint{3.941485in}{1.811701in}}%
\pgfpathlineto{\pgfqpoint{3.933398in}{1.799287in}}%
\pgfpathclose%
\pgfusepath{fill}%
\end{pgfscope}%
\begin{pgfscope}%
\pgfpathrectangle{\pgfqpoint{1.150000in}{0.150000in}}{\pgfqpoint{5.700000in}{5.700000in}}%
\pgfusepath{clip}%
\pgfsetbuttcap%
\pgfsetroundjoin%
\definecolor{currentfill}{rgb}{0.212395,0.359683,0.551710}%
\pgfsetfillcolor{currentfill}%
\pgfsetfillopacity{0.800000}%
\pgfsetlinewidth{0.000000pt}%
\definecolor{currentstroke}{rgb}{0.000000,0.000000,0.000000}%
\pgfsetstrokecolor{currentstroke}%
\pgfsetdash{}{0pt}%
\pgfpathmoveto{\pgfqpoint{2.302828in}{2.427121in}}%
\pgfpathlineto{\pgfqpoint{2.317016in}{2.404192in}}%
\pgfpathlineto{\pgfqpoint{2.331190in}{2.381576in}}%
\pgfpathlineto{\pgfqpoint{2.345350in}{2.359270in}}%
\pgfpathlineto{\pgfqpoint{2.359497in}{2.337272in}}%
\pgfpathlineto{\pgfqpoint{2.368582in}{2.333471in}}%
\pgfpathlineto{\pgfqpoint{2.377646in}{2.330006in}}%
\pgfpathlineto{\pgfqpoint{2.386689in}{2.326871in}}%
\pgfpathlineto{\pgfqpoint{2.395711in}{2.324059in}}%
\pgfpathlineto{\pgfqpoint{2.381620in}{2.345416in}}%
\pgfpathlineto{\pgfqpoint{2.367517in}{2.367079in}}%
\pgfpathlineto{\pgfqpoint{2.353401in}{2.389051in}}%
\pgfpathlineto{\pgfqpoint{2.339271in}{2.411334in}}%
\pgfpathlineto{\pgfqpoint{2.330193in}{2.414775in}}%
\pgfpathlineto{\pgfqpoint{2.321094in}{2.418549in}}%
\pgfpathlineto{\pgfqpoint{2.311972in}{2.422662in}}%
\pgfpathlineto{\pgfqpoint{2.302828in}{2.427121in}}%
\pgfpathclose%
\pgfusepath{fill}%
\end{pgfscope}%
\begin{pgfscope}%
\pgfpathrectangle{\pgfqpoint{1.150000in}{0.150000in}}{\pgfqpoint{5.700000in}{5.700000in}}%
\pgfusepath{clip}%
\pgfsetbuttcap%
\pgfsetroundjoin%
\definecolor{currentfill}{rgb}{0.279566,0.067836,0.391917}%
\pgfsetfillcolor{currentfill}%
\pgfsetfillopacity{0.800000}%
\pgfsetlinewidth{0.000000pt}%
\definecolor{currentstroke}{rgb}{0.000000,0.000000,0.000000}%
\pgfsetstrokecolor{currentstroke}%
\pgfsetdash{}{0pt}%
\pgfpathmoveto{\pgfqpoint{2.917864in}{1.682514in}}%
\pgfpathlineto{\pgfqpoint{2.931709in}{1.671220in}}%
\pgfpathlineto{\pgfqpoint{2.945551in}{1.660152in}}%
\pgfpathlineto{\pgfqpoint{2.959391in}{1.649308in}}%
\pgfpathlineto{\pgfqpoint{2.973229in}{1.638687in}}%
\pgfpathlineto{\pgfqpoint{2.981809in}{1.641828in}}%
\pgfpathlineto{\pgfqpoint{2.990376in}{1.645206in}}%
\pgfpathlineto{\pgfqpoint{2.998930in}{1.648816in}}%
\pgfpathlineto{\pgfqpoint{3.007471in}{1.652651in}}%
\pgfpathlineto{\pgfqpoint{2.993668in}{1.662671in}}%
\pgfpathlineto{\pgfqpoint{2.979863in}{1.672912in}}%
\pgfpathlineto{\pgfqpoint{2.966056in}{1.683377in}}%
\pgfpathlineto{\pgfqpoint{2.952247in}{1.694068in}}%
\pgfpathlineto{\pgfqpoint{2.943672in}{1.690822in}}%
\pgfpathlineto{\pgfqpoint{2.935083in}{1.687811in}}%
\pgfpathlineto{\pgfqpoint{2.926481in}{1.685039in}}%
\pgfpathlineto{\pgfqpoint{2.917864in}{1.682514in}}%
\pgfpathclose%
\pgfusepath{fill}%
\end{pgfscope}%
\begin{pgfscope}%
\pgfpathrectangle{\pgfqpoint{1.150000in}{0.150000in}}{\pgfqpoint{5.700000in}{5.700000in}}%
\pgfusepath{clip}%
\pgfsetbuttcap%
\pgfsetroundjoin%
\definecolor{currentfill}{rgb}{0.268510,0.009605,0.335427}%
\pgfsetfillcolor{currentfill}%
\pgfsetfillopacity{0.800000}%
\pgfsetlinewidth{0.000000pt}%
\definecolor{currentstroke}{rgb}{0.000000,0.000000,0.000000}%
\pgfsetstrokecolor{currentstroke}%
\pgfsetdash{}{0pt}%
\pgfpathmoveto{\pgfqpoint{3.405470in}{1.540569in}}%
\pgfpathlineto{\pgfqpoint{3.419257in}{1.536868in}}%
\pgfpathlineto{\pgfqpoint{3.433047in}{1.533364in}}%
\pgfpathlineto{\pgfqpoint{3.446842in}{1.530058in}}%
\pgfpathlineto{\pgfqpoint{3.460641in}{1.526948in}}%
\pgfpathlineto{\pgfqpoint{3.468919in}{1.536145in}}%
\pgfpathlineto{\pgfqpoint{3.477190in}{1.545457in}}%
\pgfpathlineto{\pgfqpoint{3.485453in}{1.554879in}}%
\pgfpathlineto{\pgfqpoint{3.493709in}{1.564407in}}%
\pgfpathlineto{\pgfqpoint{3.479927in}{1.567018in}}%
\pgfpathlineto{\pgfqpoint{3.466150in}{1.569825in}}%
\pgfpathlineto{\pgfqpoint{3.452377in}{1.572830in}}%
\pgfpathlineto{\pgfqpoint{3.438609in}{1.576032in}}%
\pgfpathlineto{\pgfqpoint{3.430336in}{1.566991in}}%
\pgfpathlineto{\pgfqpoint{3.422055in}{1.558064in}}%
\pgfpathlineto{\pgfqpoint{3.413766in}{1.549255in}}%
\pgfpathlineto{\pgfqpoint{3.405470in}{1.540569in}}%
\pgfpathclose%
\pgfusepath{fill}%
\end{pgfscope}%
\begin{pgfscope}%
\pgfpathrectangle{\pgfqpoint{1.150000in}{0.150000in}}{\pgfqpoint{5.700000in}{5.700000in}}%
\pgfusepath{clip}%
\pgfsetbuttcap%
\pgfsetroundjoin%
\definecolor{currentfill}{rgb}{0.288921,0.758394,0.428426}%
\pgfsetfillcolor{currentfill}%
\pgfsetfillopacity{0.800000}%
\pgfsetlinewidth{0.000000pt}%
\definecolor{currentstroke}{rgb}{0.000000,0.000000,0.000000}%
\pgfsetstrokecolor{currentstroke}%
\pgfsetdash{}{0pt}%
\pgfpathmoveto{\pgfqpoint{6.006166in}{3.591773in}}%
\pgfpathlineto{\pgfqpoint{6.021230in}{3.606757in}}%
\pgfpathlineto{\pgfqpoint{6.036318in}{3.621923in}}%
\pgfpathlineto{\pgfqpoint{6.051429in}{3.637271in}}%
\pgfpathlineto{\pgfqpoint{6.058440in}{3.636551in}}%
\pgfpathlineto{\pgfqpoint{6.065445in}{3.635854in}}%
\pgfpathlineto{\pgfqpoint{6.072443in}{3.635187in}}%
\pgfpathlineto{\pgfqpoint{6.079435in}{3.634555in}}%
\pgfpathlineto{\pgfqpoint{6.064362in}{3.619852in}}%
\pgfpathlineto{\pgfqpoint{6.049312in}{3.605330in}}%
\pgfpathlineto{\pgfqpoint{6.034285in}{3.590987in}}%
\pgfpathlineto{\pgfqpoint{6.027264in}{3.591128in}}%
\pgfpathlineto{\pgfqpoint{6.020238in}{3.591310in}}%
\pgfpathlineto{\pgfqpoint{6.013205in}{3.591527in}}%
\pgfpathlineto{\pgfqpoint{6.006166in}{3.591773in}}%
\pgfpathclose%
\pgfusepath{fill}%
\end{pgfscope}%
\begin{pgfscope}%
\pgfpathrectangle{\pgfqpoint{1.150000in}{0.150000in}}{\pgfqpoint{5.700000in}{5.700000in}}%
\pgfusepath{clip}%
\pgfsetbuttcap%
\pgfsetroundjoin%
\definecolor{currentfill}{rgb}{0.137770,0.537492,0.554906}%
\pgfsetfillcolor{currentfill}%
\pgfsetfillopacity{0.800000}%
\pgfsetlinewidth{0.000000pt}%
\definecolor{currentstroke}{rgb}{0.000000,0.000000,0.000000}%
\pgfsetstrokecolor{currentstroke}%
\pgfsetdash{}{0pt}%
\pgfpathmoveto{\pgfqpoint{5.061568in}{2.871842in}}%
\pgfpathlineto{\pgfqpoint{5.076045in}{2.884493in}}%
\pgfpathlineto{\pgfqpoint{5.090540in}{2.897327in}}%
\pgfpathlineto{\pgfqpoint{5.105055in}{2.910346in}}%
\pgfpathlineto{\pgfqpoint{5.119588in}{2.923549in}}%
\pgfpathlineto{\pgfqpoint{5.127235in}{2.930356in}}%
\pgfpathlineto{\pgfqpoint{5.134874in}{2.937037in}}%
\pgfpathlineto{\pgfqpoint{5.142504in}{2.943596in}}%
\pgfpathlineto{\pgfqpoint{5.150125in}{2.950035in}}%
\pgfpathlineto{\pgfqpoint{5.135602in}{2.937053in}}%
\pgfpathlineto{\pgfqpoint{5.121098in}{2.924254in}}%
\pgfpathlineto{\pgfqpoint{5.106614in}{2.911640in}}%
\pgfpathlineto{\pgfqpoint{5.092148in}{2.899208in}}%
\pgfpathlineto{\pgfqpoint{5.084515in}{2.892537in}}%
\pgfpathlineto{\pgfqpoint{5.076874in}{2.885754in}}%
\pgfpathlineto{\pgfqpoint{5.069225in}{2.878856in}}%
\pgfpathlineto{\pgfqpoint{5.061568in}{2.871842in}}%
\pgfpathclose%
\pgfusepath{fill}%
\end{pgfscope}%
\begin{pgfscope}%
\pgfpathrectangle{\pgfqpoint{1.150000in}{0.150000in}}{\pgfqpoint{5.700000in}{5.700000in}}%
\pgfusepath{clip}%
\pgfsetbuttcap%
\pgfsetroundjoin%
\definecolor{currentfill}{rgb}{0.157729,0.485932,0.558013}%
\pgfsetfillcolor{currentfill}%
\pgfsetfillopacity{0.800000}%
\pgfsetlinewidth{0.000000pt}%
\definecolor{currentstroke}{rgb}{0.000000,0.000000,0.000000}%
\pgfsetstrokecolor{currentstroke}%
\pgfsetdash{}{0pt}%
\pgfpathmoveto{\pgfqpoint{2.111111in}{2.812764in}}%
\pgfpathlineto{\pgfqpoint{2.125501in}{2.785065in}}%
\pgfpathlineto{\pgfqpoint{2.139872in}{2.757731in}}%
\pgfpathlineto{\pgfqpoint{2.154224in}{2.730757in}}%
\pgfpathlineto{\pgfqpoint{2.168558in}{2.704140in}}%
\pgfpathlineto{\pgfqpoint{2.177793in}{2.699130in}}%
\pgfpathlineto{\pgfqpoint{2.187005in}{2.694469in}}%
\pgfpathlineto{\pgfqpoint{2.196193in}{2.690152in}}%
\pgfpathlineto{\pgfqpoint{2.205359in}{2.686172in}}%
\pgfpathlineto{\pgfqpoint{2.191087in}{2.712160in}}%
\pgfpathlineto{\pgfqpoint{2.176798in}{2.738504in}}%
\pgfpathlineto{\pgfqpoint{2.162491in}{2.765206in}}%
\pgfpathlineto{\pgfqpoint{2.148165in}{2.792272in}}%
\pgfpathlineto{\pgfqpoint{2.138937in}{2.796868in}}%
\pgfpathlineto{\pgfqpoint{2.129686in}{2.801812in}}%
\pgfpathlineto{\pgfqpoint{2.120411in}{2.807108in}}%
\pgfpathlineto{\pgfqpoint{2.111111in}{2.812764in}}%
\pgfpathclose%
\pgfusepath{fill}%
\end{pgfscope}%
\begin{pgfscope}%
\pgfpathrectangle{\pgfqpoint{1.150000in}{0.150000in}}{\pgfqpoint{5.700000in}{5.700000in}}%
\pgfusepath{clip}%
\pgfsetbuttcap%
\pgfsetroundjoin%
\definecolor{currentfill}{rgb}{0.259857,0.745492,0.444467}%
\pgfsetfillcolor{currentfill}%
\pgfsetfillopacity{0.800000}%
\pgfsetlinewidth{0.000000pt}%
\definecolor{currentstroke}{rgb}{0.000000,0.000000,0.000000}%
\pgfsetstrokecolor{currentstroke}%
\pgfsetdash{}{0pt}%
\pgfpathmoveto{\pgfqpoint{5.917773in}{3.532339in}}%
\pgfpathlineto{\pgfqpoint{5.932780in}{3.547208in}}%
\pgfpathlineto{\pgfqpoint{5.947810in}{3.562258in}}%
\pgfpathlineto{\pgfqpoint{5.962864in}{3.577490in}}%
\pgfpathlineto{\pgfqpoint{5.977940in}{3.592904in}}%
\pgfpathlineto{\pgfqpoint{5.985008in}{3.592612in}}%
\pgfpathlineto{\pgfqpoint{5.992067in}{3.592322in}}%
\pgfpathlineto{\pgfqpoint{5.999120in}{3.592040in}}%
\pgfpathlineto{\pgfqpoint{6.006166in}{3.591773in}}%
\pgfpathlineto{\pgfqpoint{5.991124in}{3.576969in}}%
\pgfpathlineto{\pgfqpoint{5.976106in}{3.562346in}}%
\pgfpathlineto{\pgfqpoint{5.961111in}{3.547903in}}%
\pgfpathlineto{\pgfqpoint{5.946138in}{3.533641in}}%
\pgfpathlineto{\pgfqpoint{5.939057in}{3.533288in}}%
\pgfpathlineto{\pgfqpoint{5.931969in}{3.532958in}}%
\pgfpathlineto{\pgfqpoint{5.924874in}{3.532644in}}%
\pgfpathlineto{\pgfqpoint{5.917773in}{3.532339in}}%
\pgfpathclose%
\pgfusepath{fill}%
\end{pgfscope}%
\begin{pgfscope}%
\pgfpathrectangle{\pgfqpoint{1.150000in}{0.150000in}}{\pgfqpoint{5.700000in}{5.700000in}}%
\pgfusepath{clip}%
\pgfsetbuttcap%
\pgfsetroundjoin%
\definecolor{currentfill}{rgb}{0.257322,0.256130,0.526563}%
\pgfsetfillcolor{currentfill}%
\pgfsetfillopacity{0.800000}%
\pgfsetlinewidth{0.000000pt}%
\definecolor{currentstroke}{rgb}{0.000000,0.000000,0.000000}%
\pgfsetstrokecolor{currentstroke}%
\pgfsetdash{}{0pt}%
\pgfpathmoveto{\pgfqpoint{4.229418in}{2.060637in}}%
\pgfpathlineto{\pgfqpoint{4.243430in}{2.067356in}}%
\pgfpathlineto{\pgfqpoint{4.257455in}{2.074261in}}%
\pgfpathlineto{\pgfqpoint{4.271493in}{2.081351in}}%
\pgfpathlineto{\pgfqpoint{4.285544in}{2.088626in}}%
\pgfpathlineto{\pgfqpoint{4.293543in}{2.101273in}}%
\pgfpathlineto{\pgfqpoint{4.301537in}{2.113837in}}%
\pgfpathlineto{\pgfqpoint{4.309526in}{2.126315in}}%
\pgfpathlineto{\pgfqpoint{4.317510in}{2.138707in}}%
\pgfpathlineto{\pgfqpoint{4.303461in}{2.131247in}}%
\pgfpathlineto{\pgfqpoint{4.289425in}{2.123972in}}%
\pgfpathlineto{\pgfqpoint{4.275402in}{2.116882in}}%
\pgfpathlineto{\pgfqpoint{4.261391in}{2.109978in}}%
\pgfpathlineto{\pgfqpoint{4.253406in}{2.097759in}}%
\pgfpathlineto{\pgfqpoint{4.245415in}{2.085462in}}%
\pgfpathlineto{\pgfqpoint{4.237419in}{2.073087in}}%
\pgfpathlineto{\pgfqpoint{4.229418in}{2.060637in}}%
\pgfpathclose%
\pgfusepath{fill}%
\end{pgfscope}%
\begin{pgfscope}%
\pgfpathrectangle{\pgfqpoint{1.150000in}{0.150000in}}{\pgfqpoint{5.700000in}{5.700000in}}%
\pgfusepath{clip}%
\pgfsetbuttcap%
\pgfsetroundjoin%
\definecolor{currentfill}{rgb}{0.221989,0.339161,0.548752}%
\pgfsetfillcolor{currentfill}%
\pgfsetfillopacity{0.800000}%
\pgfsetlinewidth{0.000000pt}%
\definecolor{currentstroke}{rgb}{0.000000,0.000000,0.000000}%
\pgfsetstrokecolor{currentstroke}%
\pgfsetdash{}{0pt}%
\pgfpathmoveto{\pgfqpoint{4.437517in}{2.267366in}}%
\pgfpathlineto{\pgfqpoint{4.451634in}{2.276024in}}%
\pgfpathlineto{\pgfqpoint{4.465764in}{2.284866in}}%
\pgfpathlineto{\pgfqpoint{4.479910in}{2.293894in}}%
\pgfpathlineto{\pgfqpoint{4.494071in}{2.303106in}}%
\pgfpathlineto{\pgfqpoint{4.502004in}{2.314868in}}%
\pgfpathlineto{\pgfqpoint{4.509931in}{2.326518in}}%
\pgfpathlineto{\pgfqpoint{4.517852in}{2.338055in}}%
\pgfpathlineto{\pgfqpoint{4.525768in}{2.349478in}}%
\pgfpathlineto{\pgfqpoint{4.511609in}{2.340179in}}%
\pgfpathlineto{\pgfqpoint{4.497465in}{2.331065in}}%
\pgfpathlineto{\pgfqpoint{4.483336in}{2.322135in}}%
\pgfpathlineto{\pgfqpoint{4.469222in}{2.313390in}}%
\pgfpathlineto{\pgfqpoint{4.461304in}{2.302041in}}%
\pgfpathlineto{\pgfqpoint{4.453381in}{2.290587in}}%
\pgfpathlineto{\pgfqpoint{4.445452in}{2.279029in}}%
\pgfpathlineto{\pgfqpoint{4.437517in}{2.267366in}}%
\pgfpathclose%
\pgfusepath{fill}%
\end{pgfscope}%
\begin{pgfscope}%
\pgfpathrectangle{\pgfqpoint{1.150000in}{0.150000in}}{\pgfqpoint{5.700000in}{5.700000in}}%
\pgfusepath{clip}%
\pgfsetbuttcap%
\pgfsetroundjoin%
\definecolor{currentfill}{rgb}{0.160665,0.478540,0.558115}%
\pgfsetfillcolor{currentfill}%
\pgfsetfillopacity{0.800000}%
\pgfsetlinewidth{0.000000pt}%
\definecolor{currentstroke}{rgb}{0.000000,0.000000,0.000000}%
\pgfsetstrokecolor{currentstroke}%
\pgfsetdash{}{0pt}%
\pgfpathmoveto{\pgfqpoint{4.853719in}{2.679497in}}%
\pgfpathlineto{\pgfqpoint{4.868073in}{2.691126in}}%
\pgfpathlineto{\pgfqpoint{4.882444in}{2.702940in}}%
\pgfpathlineto{\pgfqpoint{4.896834in}{2.714938in}}%
\pgfpathlineto{\pgfqpoint{4.911241in}{2.727121in}}%
\pgfpathlineto{\pgfqpoint{4.919001in}{2.735819in}}%
\pgfpathlineto{\pgfqpoint{4.926753in}{2.744384in}}%
\pgfpathlineto{\pgfqpoint{4.934497in}{2.752816in}}%
\pgfpathlineto{\pgfqpoint{4.942234in}{2.761118in}}%
\pgfpathlineto{\pgfqpoint{4.927833in}{2.749052in}}%
\pgfpathlineto{\pgfqpoint{4.913450in}{2.737170in}}%
\pgfpathlineto{\pgfqpoint{4.899085in}{2.725472in}}%
\pgfpathlineto{\pgfqpoint{4.884737in}{2.713959in}}%
\pgfpathlineto{\pgfqpoint{4.876994in}{2.705528in}}%
\pgfpathlineto{\pgfqpoint{4.869243in}{2.696975in}}%
\pgfpathlineto{\pgfqpoint{4.861484in}{2.688298in}}%
\pgfpathlineto{\pgfqpoint{4.853719in}{2.679497in}}%
\pgfpathclose%
\pgfusepath{fill}%
\end{pgfscope}%
\begin{pgfscope}%
\pgfpathrectangle{\pgfqpoint{1.150000in}{0.150000in}}{\pgfqpoint{5.700000in}{5.700000in}}%
\pgfusepath{clip}%
\pgfsetbuttcap%
\pgfsetroundjoin%
\definecolor{currentfill}{rgb}{0.278826,0.175490,0.483397}%
\pgfsetfillcolor{currentfill}%
\pgfsetfillopacity{0.800000}%
\pgfsetlinewidth{0.000000pt}%
\definecolor{currentstroke}{rgb}{0.000000,0.000000,0.000000}%
\pgfsetstrokecolor{currentstroke}%
\pgfsetdash{}{0pt}%
\pgfpathmoveto{\pgfqpoint{4.021315in}{1.864909in}}%
\pgfpathlineto{\pgfqpoint{4.035240in}{1.869394in}}%
\pgfpathlineto{\pgfqpoint{4.049177in}{1.874066in}}%
\pgfpathlineto{\pgfqpoint{4.063125in}{1.878924in}}%
\pgfpathlineto{\pgfqpoint{4.077084in}{1.883967in}}%
\pgfpathlineto{\pgfqpoint{4.085144in}{1.896884in}}%
\pgfpathlineto{\pgfqpoint{4.093199in}{1.909758in}}%
\pgfpathlineto{\pgfqpoint{4.101250in}{1.922586in}}%
\pgfpathlineto{\pgfqpoint{4.109296in}{1.935366in}}%
\pgfpathlineto{\pgfqpoint{4.095340in}{1.930042in}}%
\pgfpathlineto{\pgfqpoint{4.081396in}{1.924903in}}%
\pgfpathlineto{\pgfqpoint{4.067463in}{1.919950in}}%
\pgfpathlineto{\pgfqpoint{4.053541in}{1.915184in}}%
\pgfpathlineto{\pgfqpoint{4.045491in}{1.902673in}}%
\pgfpathlineto{\pgfqpoint{4.037437in}{1.890122in}}%
\pgfpathlineto{\pgfqpoint{4.029378in}{1.877533in}}%
\pgfpathlineto{\pgfqpoint{4.021315in}{1.864909in}}%
\pgfpathclose%
\pgfusepath{fill}%
\end{pgfscope}%
\begin{pgfscope}%
\pgfpathrectangle{\pgfqpoint{1.150000in}{0.150000in}}{\pgfqpoint{5.700000in}{5.700000in}}%
\pgfusepath{clip}%
\pgfsetbuttcap%
\pgfsetroundjoin%
\definecolor{currentfill}{rgb}{0.188923,0.410910,0.556326}%
\pgfsetfillcolor{currentfill}%
\pgfsetfillopacity{0.800000}%
\pgfsetlinewidth{0.000000pt}%
\definecolor{currentstroke}{rgb}{0.000000,0.000000,0.000000}%
\pgfsetstrokecolor{currentstroke}%
\pgfsetdash{}{0pt}%
\pgfpathmoveto{\pgfqpoint{4.645646in}{2.476107in}}%
\pgfpathlineto{\pgfqpoint{4.659878in}{2.486403in}}%
\pgfpathlineto{\pgfqpoint{4.674126in}{2.496884in}}%
\pgfpathlineto{\pgfqpoint{4.688391in}{2.507550in}}%
\pgfpathlineto{\pgfqpoint{4.702672in}{2.518400in}}%
\pgfpathlineto{\pgfqpoint{4.710527in}{2.528799in}}%
\pgfpathlineto{\pgfqpoint{4.718375in}{2.539068in}}%
\pgfpathlineto{\pgfqpoint{4.726216in}{2.549209in}}%
\pgfpathlineto{\pgfqpoint{4.734050in}{2.559222in}}%
\pgfpathlineto{\pgfqpoint{4.719772in}{2.548385in}}%
\pgfpathlineto{\pgfqpoint{4.705511in}{2.537733in}}%
\pgfpathlineto{\pgfqpoint{4.691266in}{2.527266in}}%
\pgfpathlineto{\pgfqpoint{4.677037in}{2.516983in}}%
\pgfpathlineto{\pgfqpoint{4.669199in}{2.506944in}}%
\pgfpathlineto{\pgfqpoint{4.661355in}{2.496785in}}%
\pgfpathlineto{\pgfqpoint{4.653503in}{2.486506in}}%
\pgfpathlineto{\pgfqpoint{4.645646in}{2.476107in}}%
\pgfpathclose%
\pgfusepath{fill}%
\end{pgfscope}%
\begin{pgfscope}%
\pgfpathrectangle{\pgfqpoint{1.150000in}{0.150000in}}{\pgfqpoint{5.700000in}{5.700000in}}%
\pgfusepath{clip}%
\pgfsetbuttcap%
\pgfsetroundjoin%
\definecolor{currentfill}{rgb}{0.126326,0.644107,0.525311}%
\pgfsetfillcolor{currentfill}%
\pgfsetfillopacity{0.800000}%
\pgfsetlinewidth{0.000000pt}%
\definecolor{currentstroke}{rgb}{0.000000,0.000000,0.000000}%
\pgfsetstrokecolor{currentstroke}%
\pgfsetdash{}{0pt}%
\pgfpathmoveto{\pgfqpoint{5.446061in}{3.194698in}}%
\pgfpathlineto{\pgfqpoint{5.460786in}{3.208832in}}%
\pgfpathlineto{\pgfqpoint{5.475531in}{3.223150in}}%
\pgfpathlineto{\pgfqpoint{5.490298in}{3.237651in}}%
\pgfpathlineto{\pgfqpoint{5.505086in}{3.252336in}}%
\pgfpathlineto{\pgfqpoint{5.512495in}{3.255720in}}%
\pgfpathlineto{\pgfqpoint{5.519896in}{3.259015in}}%
\pgfpathlineto{\pgfqpoint{5.527287in}{3.262226in}}%
\pgfpathlineto{\pgfqpoint{5.534670in}{3.265357in}}%
\pgfpathlineto{\pgfqpoint{5.519902in}{3.251071in}}%
\pgfpathlineto{\pgfqpoint{5.505156in}{3.236967in}}%
\pgfpathlineto{\pgfqpoint{5.490430in}{3.223045in}}%
\pgfpathlineto{\pgfqpoint{5.475726in}{3.209306in}}%
\pgfpathlineto{\pgfqpoint{5.468322in}{3.205766in}}%
\pgfpathlineto{\pgfqpoint{5.460911in}{3.202154in}}%
\pgfpathlineto{\pgfqpoint{5.453490in}{3.198466in}}%
\pgfpathlineto{\pgfqpoint{5.446061in}{3.194698in}}%
\pgfpathclose%
\pgfusepath{fill}%
\end{pgfscope}%
\begin{pgfscope}%
\pgfpathrectangle{\pgfqpoint{1.150000in}{0.150000in}}{\pgfqpoint{5.700000in}{5.700000in}}%
\pgfusepath{clip}%
\pgfsetbuttcap%
\pgfsetroundjoin%
\definecolor{currentfill}{rgb}{0.277018,0.050344,0.375715}%
\pgfsetfillcolor{currentfill}%
\pgfsetfillopacity{0.800000}%
\pgfsetlinewidth{0.000000pt}%
\definecolor{currentstroke}{rgb}{0.000000,0.000000,0.000000}%
\pgfsetstrokecolor{currentstroke}%
\pgfsetdash{}{0pt}%
\pgfpathmoveto{\pgfqpoint{2.973229in}{1.638687in}}%
\pgfpathlineto{\pgfqpoint{2.987065in}{1.628288in}}%
\pgfpathlineto{\pgfqpoint{3.000900in}{1.618110in}}%
\pgfpathlineto{\pgfqpoint{3.014733in}{1.608150in}}%
\pgfpathlineto{\pgfqpoint{3.028565in}{1.598409in}}%
\pgfpathlineto{\pgfqpoint{3.037111in}{1.602163in}}%
\pgfpathlineto{\pgfqpoint{3.045644in}{1.606146in}}%
\pgfpathlineto{\pgfqpoint{3.054165in}{1.610353in}}%
\pgfpathlineto{\pgfqpoint{3.062674in}{1.614776in}}%
\pgfpathlineto{\pgfqpoint{3.048875in}{1.623917in}}%
\pgfpathlineto{\pgfqpoint{3.035074in}{1.633276in}}%
\pgfpathlineto{\pgfqpoint{3.021273in}{1.642854in}}%
\pgfpathlineto{\pgfqpoint{3.007471in}{1.652651in}}%
\pgfpathlineto{\pgfqpoint{2.998930in}{1.648816in}}%
\pgfpathlineto{\pgfqpoint{2.990376in}{1.645206in}}%
\pgfpathlineto{\pgfqpoint{2.981809in}{1.641828in}}%
\pgfpathlineto{\pgfqpoint{2.973229in}{1.638687in}}%
\pgfpathclose%
\pgfusepath{fill}%
\end{pgfscope}%
\begin{pgfscope}%
\pgfpathrectangle{\pgfqpoint{1.150000in}{0.150000in}}{\pgfqpoint{5.700000in}{5.700000in}}%
\pgfusepath{clip}%
\pgfsetbuttcap%
\pgfsetroundjoin%
\definecolor{currentfill}{rgb}{0.195860,0.395433,0.555276}%
\pgfsetfillcolor{currentfill}%
\pgfsetfillopacity{0.800000}%
\pgfsetlinewidth{0.000000pt}%
\definecolor{currentstroke}{rgb}{0.000000,0.000000,0.000000}%
\pgfsetstrokecolor{currentstroke}%
\pgfsetdash{}{0pt}%
\pgfpathmoveto{\pgfqpoint{2.245933in}{2.522027in}}%
\pgfpathlineto{\pgfqpoint{2.260179in}{2.497816in}}%
\pgfpathlineto{\pgfqpoint{2.274410in}{2.473930in}}%
\pgfpathlineto{\pgfqpoint{2.288627in}{2.450366in}}%
\pgfpathlineto{\pgfqpoint{2.302828in}{2.427121in}}%
\pgfpathlineto{\pgfqpoint{2.311972in}{2.422662in}}%
\pgfpathlineto{\pgfqpoint{2.321094in}{2.418549in}}%
\pgfpathlineto{\pgfqpoint{2.330193in}{2.414775in}}%
\pgfpathlineto{\pgfqpoint{2.339271in}{2.411334in}}%
\pgfpathlineto{\pgfqpoint{2.325128in}{2.433932in}}%
\pgfpathlineto{\pgfqpoint{2.310971in}{2.456847in}}%
\pgfpathlineto{\pgfqpoint{2.296800in}{2.480082in}}%
\pgfpathlineto{\pgfqpoint{2.282614in}{2.503641in}}%
\pgfpathlineto{\pgfqpoint{2.273478in}{2.507717in}}%
\pgfpathlineto{\pgfqpoint{2.264319in}{2.512136in}}%
\pgfpathlineto{\pgfqpoint{2.255138in}{2.516904in}}%
\pgfpathlineto{\pgfqpoint{2.245933in}{2.522027in}}%
\pgfpathclose%
\pgfusepath{fill}%
\end{pgfscope}%
\begin{pgfscope}%
\pgfpathrectangle{\pgfqpoint{1.150000in}{0.150000in}}{\pgfqpoint{5.700000in}{5.700000in}}%
\pgfusepath{clip}%
\pgfsetbuttcap%
\pgfsetroundjoin%
\definecolor{currentfill}{rgb}{0.268510,0.009605,0.335427}%
\pgfsetfillcolor{currentfill}%
\pgfsetfillopacity{0.800000}%
\pgfsetlinewidth{0.000000pt}%
\definecolor{currentstroke}{rgb}{0.000000,0.000000,0.000000}%
\pgfsetstrokecolor{currentstroke}%
\pgfsetdash{}{0pt}%
\pgfpathmoveto{\pgfqpoint{3.173067in}{1.549358in}}%
\pgfpathlineto{\pgfqpoint{3.186869in}{1.542129in}}%
\pgfpathlineto{\pgfqpoint{3.200671in}{1.535108in}}%
\pgfpathlineto{\pgfqpoint{3.214476in}{1.528293in}}%
\pgfpathlineto{\pgfqpoint{3.228281in}{1.521684in}}%
\pgfpathlineto{\pgfqpoint{3.236693in}{1.528051in}}%
\pgfpathlineto{\pgfqpoint{3.245095in}{1.534599in}}%
\pgfpathlineto{\pgfqpoint{3.253487in}{1.541322in}}%
\pgfpathlineto{\pgfqpoint{3.261869in}{1.548216in}}%
\pgfpathlineto{\pgfqpoint{3.248089in}{1.554261in}}%
\pgfpathlineto{\pgfqpoint{3.234310in}{1.560512in}}%
\pgfpathlineto{\pgfqpoint{3.220534in}{1.566969in}}%
\pgfpathlineto{\pgfqpoint{3.206759in}{1.573633in}}%
\pgfpathlineto{\pgfqpoint{3.198351in}{1.567292in}}%
\pgfpathlineto{\pgfqpoint{3.189933in}{1.561128in}}%
\pgfpathlineto{\pgfqpoint{3.181505in}{1.555149in}}%
\pgfpathlineto{\pgfqpoint{3.173067in}{1.549358in}}%
\pgfpathclose%
\pgfusepath{fill}%
\end{pgfscope}%
\begin{pgfscope}%
\pgfpathrectangle{\pgfqpoint{1.150000in}{0.150000in}}{\pgfqpoint{5.700000in}{5.700000in}}%
\pgfusepath{clip}%
\pgfsetbuttcap%
\pgfsetroundjoin%
\definecolor{currentfill}{rgb}{0.267004,0.004874,0.329415}%
\pgfsetfillcolor{currentfill}%
\pgfsetfillopacity{0.800000}%
\pgfsetlinewidth{0.000000pt}%
\definecolor{currentstroke}{rgb}{0.000000,0.000000,0.000000}%
\pgfsetstrokecolor{currentstroke}%
\pgfsetdash{}{0pt}%
\pgfpathmoveto{\pgfqpoint{3.317015in}{1.526073in}}%
\pgfpathlineto{\pgfqpoint{3.330807in}{1.521043in}}%
\pgfpathlineto{\pgfqpoint{3.344603in}{1.516214in}}%
\pgfpathlineto{\pgfqpoint{3.358403in}{1.511585in}}%
\pgfpathlineto{\pgfqpoint{3.372205in}{1.507154in}}%
\pgfpathlineto{\pgfqpoint{3.380534in}{1.515298in}}%
\pgfpathlineto{\pgfqpoint{3.388854in}{1.523586in}}%
\pgfpathlineto{\pgfqpoint{3.397166in}{1.532011in}}%
\pgfpathlineto{\pgfqpoint{3.405470in}{1.540569in}}%
\pgfpathlineto{\pgfqpoint{3.391688in}{1.544468in}}%
\pgfpathlineto{\pgfqpoint{3.377909in}{1.548567in}}%
\pgfpathlineto{\pgfqpoint{3.364134in}{1.552865in}}%
\pgfpathlineto{\pgfqpoint{3.350362in}{1.557365in}}%
\pgfpathlineto{\pgfqpoint{3.342038in}{1.549325in}}%
\pgfpathlineto{\pgfqpoint{3.333706in}{1.541427in}}%
\pgfpathlineto{\pgfqpoint{3.325365in}{1.533675in}}%
\pgfpathlineto{\pgfqpoint{3.317015in}{1.526073in}}%
\pgfpathclose%
\pgfusepath{fill}%
\end{pgfscope}%
\begin{pgfscope}%
\pgfpathrectangle{\pgfqpoint{1.150000in}{0.150000in}}{\pgfqpoint{5.700000in}{5.700000in}}%
\pgfusepath{clip}%
\pgfsetbuttcap%
\pgfsetroundjoin%
\definecolor{currentfill}{rgb}{0.127568,0.566949,0.550556}%
\pgfsetfillcolor{currentfill}%
\pgfsetfillopacity{0.800000}%
\pgfsetlinewidth{0.000000pt}%
\definecolor{currentstroke}{rgb}{0.000000,0.000000,0.000000}%
\pgfsetstrokecolor{currentstroke}%
\pgfsetdash{}{0pt}%
\pgfpathmoveto{\pgfqpoint{5.150125in}{2.950035in}}%
\pgfpathlineto{\pgfqpoint{5.164668in}{2.963201in}}%
\pgfpathlineto{\pgfqpoint{5.179230in}{2.976551in}}%
\pgfpathlineto{\pgfqpoint{5.193811in}{2.990085in}}%
\pgfpathlineto{\pgfqpoint{5.208412in}{3.003804in}}%
\pgfpathlineto{\pgfqpoint{5.216014in}{3.009882in}}%
\pgfpathlineto{\pgfqpoint{5.223606in}{3.015838in}}%
\pgfpathlineto{\pgfqpoint{5.231190in}{3.021674in}}%
\pgfpathlineto{\pgfqpoint{5.238766in}{3.027394in}}%
\pgfpathlineto{\pgfqpoint{5.224177in}{3.013932in}}%
\pgfpathlineto{\pgfqpoint{5.209607in}{3.000654in}}%
\pgfpathlineto{\pgfqpoint{5.195058in}{2.987560in}}%
\pgfpathlineto{\pgfqpoint{5.180528in}{2.974649in}}%
\pgfpathlineto{\pgfqpoint{5.172940in}{2.968660in}}%
\pgfpathlineto{\pgfqpoint{5.165343in}{2.962564in}}%
\pgfpathlineto{\pgfqpoint{5.157738in}{2.956356in}}%
\pgfpathlineto{\pgfqpoint{5.150125in}{2.950035in}}%
\pgfpathclose%
\pgfusepath{fill}%
\end{pgfscope}%
\begin{pgfscope}%
\pgfpathrectangle{\pgfqpoint{1.150000in}{0.150000in}}{\pgfqpoint{5.700000in}{5.700000in}}%
\pgfusepath{clip}%
\pgfsetbuttcap%
\pgfsetroundjoin%
\definecolor{currentfill}{rgb}{0.276022,0.044167,0.370164}%
\pgfsetfillcolor{currentfill}%
\pgfsetfillopacity{0.800000}%
\pgfsetlinewidth{0.000000pt}%
\definecolor{currentstroke}{rgb}{0.000000,0.000000,0.000000}%
\pgfsetstrokecolor{currentstroke}%
\pgfsetdash{}{0pt}%
\pgfpathmoveto{\pgfqpoint{3.637005in}{1.593331in}}%
\pgfpathlineto{\pgfqpoint{3.650826in}{1.592929in}}%
\pgfpathlineto{\pgfqpoint{3.664654in}{1.592718in}}%
\pgfpathlineto{\pgfqpoint{3.678489in}{1.592698in}}%
\pgfpathlineto{\pgfqpoint{3.692331in}{1.592867in}}%
\pgfpathlineto{\pgfqpoint{3.700515in}{1.604199in}}%
\pgfpathlineto{\pgfqpoint{3.708694in}{1.615585in}}%
\pgfpathlineto{\pgfqpoint{3.716867in}{1.627023in}}%
\pgfpathlineto{\pgfqpoint{3.725034in}{1.638509in}}%
\pgfpathlineto{\pgfqpoint{3.711203in}{1.637902in}}%
\pgfpathlineto{\pgfqpoint{3.697379in}{1.637486in}}%
\pgfpathlineto{\pgfqpoint{3.683562in}{1.637260in}}%
\pgfpathlineto{\pgfqpoint{3.669752in}{1.637226in}}%
\pgfpathlineto{\pgfqpoint{3.661574in}{1.626166in}}%
\pgfpathlineto{\pgfqpoint{3.653391in}{1.615160in}}%
\pgfpathlineto{\pgfqpoint{3.645201in}{1.604214in}}%
\pgfpathlineto{\pgfqpoint{3.637005in}{1.593331in}}%
\pgfpathclose%
\pgfusepath{fill}%
\end{pgfscope}%
\begin{pgfscope}%
\pgfpathrectangle{\pgfqpoint{1.150000in}{0.150000in}}{\pgfqpoint{5.700000in}{5.700000in}}%
\pgfusepath{clip}%
\pgfsetbuttcap%
\pgfsetroundjoin%
\definecolor{currentfill}{rgb}{0.280267,0.073417,0.397163}%
\pgfsetfillcolor{currentfill}%
\pgfsetfillopacity{0.800000}%
\pgfsetlinewidth{0.000000pt}%
\definecolor{currentstroke}{rgb}{0.000000,0.000000,0.000000}%
\pgfsetstrokecolor{currentstroke}%
\pgfsetdash{}{0pt}%
\pgfpathmoveto{\pgfqpoint{3.725034in}{1.638509in}}%
\pgfpathlineto{\pgfqpoint{3.738874in}{1.639305in}}%
\pgfpathlineto{\pgfqpoint{3.752721in}{1.640290in}}%
\pgfpathlineto{\pgfqpoint{3.766576in}{1.641464in}}%
\pgfpathlineto{\pgfqpoint{3.780439in}{1.642827in}}%
\pgfpathlineto{\pgfqpoint{3.788593in}{1.654773in}}%
\pgfpathlineto{\pgfqpoint{3.796740in}{1.666750in}}%
\pgfpathlineto{\pgfqpoint{3.804883in}{1.678755in}}%
\pgfpathlineto{\pgfqpoint{3.813021in}{1.690784in}}%
\pgfpathlineto{\pgfqpoint{3.799166in}{1.689015in}}%
\pgfpathlineto{\pgfqpoint{3.785319in}{1.687435in}}%
\pgfpathlineto{\pgfqpoint{3.771480in}{1.686044in}}%
\pgfpathlineto{\pgfqpoint{3.757650in}{1.684843in}}%
\pgfpathlineto{\pgfqpoint{3.749504in}{1.673208in}}%
\pgfpathlineto{\pgfqpoint{3.741353in}{1.661604in}}%
\pgfpathlineto{\pgfqpoint{3.733197in}{1.650037in}}%
\pgfpathlineto{\pgfqpoint{3.725034in}{1.638509in}}%
\pgfpathclose%
\pgfusepath{fill}%
\end{pgfscope}%
\begin{pgfscope}%
\pgfpathrectangle{\pgfqpoint{1.150000in}{0.150000in}}{\pgfqpoint{5.700000in}{5.700000in}}%
\pgfusepath{clip}%
\pgfsetbuttcap%
\pgfsetroundjoin%
\definecolor{currentfill}{rgb}{0.272594,0.025563,0.353093}%
\pgfsetfillcolor{currentfill}%
\pgfsetfillopacity{0.800000}%
\pgfsetlinewidth{0.000000pt}%
\definecolor{currentstroke}{rgb}{0.000000,0.000000,0.000000}%
\pgfsetstrokecolor{currentstroke}%
\pgfsetdash{}{0pt}%
\pgfpathmoveto{\pgfqpoint{3.548890in}{1.555914in}}%
\pgfpathlineto{\pgfqpoint{3.562699in}{1.554276in}}%
\pgfpathlineto{\pgfqpoint{3.576514in}{1.552831in}}%
\pgfpathlineto{\pgfqpoint{3.590335in}{1.551578in}}%
\pgfpathlineto{\pgfqpoint{3.604162in}{1.550516in}}%
\pgfpathlineto{\pgfqpoint{3.612382in}{1.561104in}}%
\pgfpathlineto{\pgfqpoint{3.620596in}{1.571772in}}%
\pgfpathlineto{\pgfqpoint{3.628804in}{1.582516in}}%
\pgfpathlineto{\pgfqpoint{3.637005in}{1.593331in}}%
\pgfpathlineto{\pgfqpoint{3.623191in}{1.593924in}}%
\pgfpathlineto{\pgfqpoint{3.609383in}{1.594710in}}%
\pgfpathlineto{\pgfqpoint{3.595582in}{1.595687in}}%
\pgfpathlineto{\pgfqpoint{3.581787in}{1.596858in}}%
\pgfpathlineto{\pgfqpoint{3.573572in}{1.586498in}}%
\pgfpathlineto{\pgfqpoint{3.565352in}{1.576218in}}%
\pgfpathlineto{\pgfqpoint{3.557124in}{1.566022in}}%
\pgfpathlineto{\pgfqpoint{3.548890in}{1.555914in}}%
\pgfpathclose%
\pgfusepath{fill}%
\end{pgfscope}%
\begin{pgfscope}%
\pgfpathrectangle{\pgfqpoint{1.150000in}{0.150000in}}{\pgfqpoint{5.700000in}{5.700000in}}%
\pgfusepath{clip}%
\pgfsetbuttcap%
\pgfsetroundjoin%
\definecolor{currentfill}{rgb}{0.143303,0.669459,0.511215}%
\pgfsetfillcolor{currentfill}%
\pgfsetfillopacity{0.800000}%
\pgfsetlinewidth{0.000000pt}%
\definecolor{currentstroke}{rgb}{0.000000,0.000000,0.000000}%
\pgfsetstrokecolor{currentstroke}%
\pgfsetdash{}{0pt}%
\pgfpathmoveto{\pgfqpoint{5.534670in}{3.265357in}}%
\pgfpathlineto{\pgfqpoint{5.549459in}{3.279827in}}%
\pgfpathlineto{\pgfqpoint{5.564269in}{3.294480in}}%
\pgfpathlineto{\pgfqpoint{5.579101in}{3.309316in}}%
\pgfpathlineto{\pgfqpoint{5.593955in}{3.324337in}}%
\pgfpathlineto{\pgfqpoint{5.601307in}{3.326972in}}%
\pgfpathlineto{\pgfqpoint{5.608650in}{3.329529in}}%
\pgfpathlineto{\pgfqpoint{5.615984in}{3.332012in}}%
\pgfpathlineto{\pgfqpoint{5.623308in}{3.334427in}}%
\pgfpathlineto{\pgfqpoint{5.608477in}{3.319842in}}%
\pgfpathlineto{\pgfqpoint{5.593668in}{3.305439in}}%
\pgfpathlineto{\pgfqpoint{5.578880in}{3.291218in}}%
\pgfpathlineto{\pgfqpoint{5.564113in}{3.277180in}}%
\pgfpathlineto{\pgfqpoint{5.556765in}{3.274320in}}%
\pgfpathlineto{\pgfqpoint{5.549408in}{3.271400in}}%
\pgfpathlineto{\pgfqpoint{5.542043in}{3.268414in}}%
\pgfpathlineto{\pgfqpoint{5.534670in}{3.265357in}}%
\pgfpathclose%
\pgfusepath{fill}%
\end{pgfscope}%
\begin{pgfscope}%
\pgfpathrectangle{\pgfqpoint{1.150000in}{0.150000in}}{\pgfqpoint{5.700000in}{5.700000in}}%
\pgfusepath{clip}%
\pgfsetbuttcap%
\pgfsetroundjoin%
\definecolor{currentfill}{rgb}{0.271828,0.209303,0.504434}%
\pgfsetfillcolor{currentfill}%
\pgfsetfillopacity{0.800000}%
\pgfsetlinewidth{0.000000pt}%
\definecolor{currentstroke}{rgb}{0.000000,0.000000,0.000000}%
\pgfsetstrokecolor{currentstroke}%
\pgfsetdash{}{0pt}%
\pgfpathmoveto{\pgfqpoint{4.109296in}{1.935366in}}%
\pgfpathlineto{\pgfqpoint{4.123264in}{1.940877in}}%
\pgfpathlineto{\pgfqpoint{4.137243in}{1.946573in}}%
\pgfpathlineto{\pgfqpoint{4.151235in}{1.952455in}}%
\pgfpathlineto{\pgfqpoint{4.165238in}{1.958522in}}%
\pgfpathlineto{\pgfqpoint{4.173277in}{1.971513in}}%
\pgfpathlineto{\pgfqpoint{4.181312in}{1.984443in}}%
\pgfpathlineto{\pgfqpoint{4.189341in}{1.997311in}}%
\pgfpathlineto{\pgfqpoint{4.197366in}{2.010114in}}%
\pgfpathlineto{\pgfqpoint{4.183365in}{2.003797in}}%
\pgfpathlineto{\pgfqpoint{4.169376in}{1.997666in}}%
\pgfpathlineto{\pgfqpoint{4.155399in}{1.991720in}}%
\pgfpathlineto{\pgfqpoint{4.141434in}{1.985961in}}%
\pgfpathlineto{\pgfqpoint{4.133407in}{1.973395in}}%
\pgfpathlineto{\pgfqpoint{4.125375in}{1.960773in}}%
\pgfpathlineto{\pgfqpoint{4.117338in}{1.948096in}}%
\pgfpathlineto{\pgfqpoint{4.109296in}{1.935366in}}%
\pgfpathclose%
\pgfusepath{fill}%
\end{pgfscope}%
\begin{pgfscope}%
\pgfpathrectangle{\pgfqpoint{1.150000in}{0.150000in}}{\pgfqpoint{5.700000in}{5.700000in}}%
\pgfusepath{clip}%
\pgfsetbuttcap%
\pgfsetroundjoin%
\definecolor{currentfill}{rgb}{0.282656,0.100196,0.422160}%
\pgfsetfillcolor{currentfill}%
\pgfsetfillopacity{0.800000}%
\pgfsetlinewidth{0.000000pt}%
\definecolor{currentstroke}{rgb}{0.000000,0.000000,0.000000}%
\pgfsetstrokecolor{currentstroke}%
\pgfsetdash{}{0pt}%
\pgfpathmoveto{\pgfqpoint{3.813021in}{1.690784in}}%
\pgfpathlineto{\pgfqpoint{3.826885in}{1.692741in}}%
\pgfpathlineto{\pgfqpoint{3.840757in}{1.694887in}}%
\pgfpathlineto{\pgfqpoint{3.854639in}{1.697219in}}%
\pgfpathlineto{\pgfqpoint{3.868530in}{1.699739in}}%
\pgfpathlineto{\pgfqpoint{3.876655in}{1.712175in}}%
\pgfpathlineto{\pgfqpoint{3.884776in}{1.724620in}}%
\pgfpathlineto{\pgfqpoint{3.892892in}{1.737070in}}%
\pgfpathlineto{\pgfqpoint{3.901003in}{1.749522in}}%
\pgfpathlineto{\pgfqpoint{3.887118in}{1.746627in}}%
\pgfpathlineto{\pgfqpoint{3.873243in}{1.743919in}}%
\pgfpathlineto{\pgfqpoint{3.859377in}{1.741399in}}%
\pgfpathlineto{\pgfqpoint{3.845520in}{1.739068in}}%
\pgfpathlineto{\pgfqpoint{3.837403in}{1.726979in}}%
\pgfpathlineto{\pgfqpoint{3.829281in}{1.714899in}}%
\pgfpathlineto{\pgfqpoint{3.821153in}{1.702833in}}%
\pgfpathlineto{\pgfqpoint{3.813021in}{1.690784in}}%
\pgfpathclose%
\pgfusepath{fill}%
\end{pgfscope}%
\begin{pgfscope}%
\pgfpathrectangle{\pgfqpoint{1.150000in}{0.150000in}}{\pgfqpoint{5.700000in}{5.700000in}}%
\pgfusepath{clip}%
\pgfsetbuttcap%
\pgfsetroundjoin%
\definecolor{currentfill}{rgb}{0.243113,0.292092,0.538516}%
\pgfsetfillcolor{currentfill}%
\pgfsetfillopacity{0.800000}%
\pgfsetlinewidth{0.000000pt}%
\definecolor{currentstroke}{rgb}{0.000000,0.000000,0.000000}%
\pgfsetstrokecolor{currentstroke}%
\pgfsetdash{}{0pt}%
\pgfpathmoveto{\pgfqpoint{4.317510in}{2.138707in}}%
\pgfpathlineto{\pgfqpoint{4.331574in}{2.146353in}}%
\pgfpathlineto{\pgfqpoint{4.345650in}{2.154183in}}%
\pgfpathlineto{\pgfqpoint{4.359741in}{2.162198in}}%
\pgfpathlineto{\pgfqpoint{4.373845in}{2.170398in}}%
\pgfpathlineto{\pgfqpoint{4.381823in}{2.182867in}}%
\pgfpathlineto{\pgfqpoint{4.389795in}{2.195239in}}%
\pgfpathlineto{\pgfqpoint{4.397763in}{2.207512in}}%
\pgfpathlineto{\pgfqpoint{4.405724in}{2.219686in}}%
\pgfpathlineto{\pgfqpoint{4.391621in}{2.211333in}}%
\pgfpathlineto{\pgfqpoint{4.377532in}{2.203165in}}%
\pgfpathlineto{\pgfqpoint{4.363456in}{2.195182in}}%
\pgfpathlineto{\pgfqpoint{4.349395in}{2.187384in}}%
\pgfpathlineto{\pgfqpoint{4.341432in}{2.175351in}}%
\pgfpathlineto{\pgfqpoint{4.333463in}{2.163226in}}%
\pgfpathlineto{\pgfqpoint{4.325489in}{2.151011in}}%
\pgfpathlineto{\pgfqpoint{4.317510in}{2.138707in}}%
\pgfpathclose%
\pgfusepath{fill}%
\end{pgfscope}%
\begin{pgfscope}%
\pgfpathrectangle{\pgfqpoint{1.150000in}{0.150000in}}{\pgfqpoint{5.700000in}{5.700000in}}%
\pgfusepath{clip}%
\pgfsetbuttcap%
\pgfsetroundjoin%
\definecolor{currentfill}{rgb}{0.149039,0.508051,0.557250}%
\pgfsetfillcolor{currentfill}%
\pgfsetfillopacity{0.800000}%
\pgfsetlinewidth{0.000000pt}%
\definecolor{currentstroke}{rgb}{0.000000,0.000000,0.000000}%
\pgfsetstrokecolor{currentstroke}%
\pgfsetdash{}{0pt}%
\pgfpathmoveto{\pgfqpoint{4.942234in}{2.761118in}}%
\pgfpathlineto{\pgfqpoint{4.956653in}{2.773369in}}%
\pgfpathlineto{\pgfqpoint{4.971091in}{2.785804in}}%
\pgfpathlineto{\pgfqpoint{4.985546in}{2.798423in}}%
\pgfpathlineto{\pgfqpoint{5.000021in}{2.811227in}}%
\pgfpathlineto{\pgfqpoint{5.007743in}{2.819262in}}%
\pgfpathlineto{\pgfqpoint{5.015456in}{2.827162in}}%
\pgfpathlineto{\pgfqpoint{5.023162in}{2.834929in}}%
\pgfpathlineto{\pgfqpoint{5.030859in}{2.842565in}}%
\pgfpathlineto{\pgfqpoint{5.016393in}{2.829913in}}%
\pgfpathlineto{\pgfqpoint{5.001944in}{2.817445in}}%
\pgfpathlineto{\pgfqpoint{4.987515in}{2.805161in}}%
\pgfpathlineto{\pgfqpoint{4.973103in}{2.793061in}}%
\pgfpathlineto{\pgfqpoint{4.965398in}{2.785261in}}%
\pgfpathlineto{\pgfqpoint{4.957684in}{2.777339in}}%
\pgfpathlineto{\pgfqpoint{4.949963in}{2.769292in}}%
\pgfpathlineto{\pgfqpoint{4.942234in}{2.761118in}}%
\pgfpathclose%
\pgfusepath{fill}%
\end{pgfscope}%
\begin{pgfscope}%
\pgfpathrectangle{\pgfqpoint{1.150000in}{0.150000in}}{\pgfqpoint{5.700000in}{5.700000in}}%
\pgfusepath{clip}%
\pgfsetbuttcap%
\pgfsetroundjoin%
\definecolor{currentfill}{rgb}{0.273809,0.031497,0.358853}%
\pgfsetfillcolor{currentfill}%
\pgfsetfillopacity{0.800000}%
\pgfsetlinewidth{0.000000pt}%
\definecolor{currentstroke}{rgb}{0.000000,0.000000,0.000000}%
\pgfsetstrokecolor{currentstroke}%
\pgfsetdash{}{0pt}%
\pgfpathmoveto{\pgfqpoint{3.028565in}{1.598409in}}%
\pgfpathlineto{\pgfqpoint{3.042397in}{1.588885in}}%
\pgfpathlineto{\pgfqpoint{3.056227in}{1.579577in}}%
\pgfpathlineto{\pgfqpoint{3.070057in}{1.570483in}}%
\pgfpathlineto{\pgfqpoint{3.083886in}{1.561604in}}%
\pgfpathlineto{\pgfqpoint{3.092399in}{1.565970in}}%
\pgfpathlineto{\pgfqpoint{3.100901in}{1.570556in}}%
\pgfpathlineto{\pgfqpoint{3.109390in}{1.575358in}}%
\pgfpathlineto{\pgfqpoint{3.117869in}{1.580368in}}%
\pgfpathlineto{\pgfqpoint{3.104070in}{1.588649in}}%
\pgfpathlineto{\pgfqpoint{3.090272in}{1.597143in}}%
\pgfpathlineto{\pgfqpoint{3.076473in}{1.605852in}}%
\pgfpathlineto{\pgfqpoint{3.062674in}{1.614776in}}%
\pgfpathlineto{\pgfqpoint{3.054165in}{1.610353in}}%
\pgfpathlineto{\pgfqpoint{3.045644in}{1.606146in}}%
\pgfpathlineto{\pgfqpoint{3.037111in}{1.602163in}}%
\pgfpathlineto{\pgfqpoint{3.028565in}{1.598409in}}%
\pgfpathclose%
\pgfusepath{fill}%
\end{pgfscope}%
\begin{pgfscope}%
\pgfpathrectangle{\pgfqpoint{1.150000in}{0.150000in}}{\pgfqpoint{5.700000in}{5.700000in}}%
\pgfusepath{clip}%
\pgfsetbuttcap%
\pgfsetroundjoin%
\definecolor{currentfill}{rgb}{0.276194,0.190074,0.493001}%
\pgfsetfillcolor{currentfill}%
\pgfsetfillopacity{0.800000}%
\pgfsetlinewidth{0.000000pt}%
\definecolor{currentstroke}{rgb}{0.000000,0.000000,0.000000}%
\pgfsetstrokecolor{currentstroke}%
\pgfsetdash{}{0pt}%
\pgfpathmoveto{\pgfqpoint{2.604654in}{1.963320in}}%
\pgfpathlineto{\pgfqpoint{2.618641in}{1.946555in}}%
\pgfpathlineto{\pgfqpoint{2.632621in}{1.930047in}}%
\pgfpathlineto{\pgfqpoint{2.646593in}{1.913795in}}%
\pgfpathlineto{\pgfqpoint{2.660559in}{1.897797in}}%
\pgfpathlineto{\pgfqpoint{2.669408in}{1.896649in}}%
\pgfpathlineto{\pgfqpoint{2.678240in}{1.895807in}}%
\pgfpathlineto{\pgfqpoint{2.687054in}{1.895265in}}%
\pgfpathlineto{\pgfqpoint{2.695851in}{1.895017in}}%
\pgfpathlineto{\pgfqpoint{2.681932in}{1.910364in}}%
\pgfpathlineto{\pgfqpoint{2.668007in}{1.925965in}}%
\pgfpathlineto{\pgfqpoint{2.654075in}{1.941820in}}%
\pgfpathlineto{\pgfqpoint{2.640136in}{1.957932in}}%
\pgfpathlineto{\pgfqpoint{2.631293in}{1.958819in}}%
\pgfpathlineto{\pgfqpoint{2.622432in}{1.960008in}}%
\pgfpathlineto{\pgfqpoint{2.613552in}{1.961506in}}%
\pgfpathlineto{\pgfqpoint{2.604654in}{1.963320in}}%
\pgfpathclose%
\pgfusepath{fill}%
\end{pgfscope}%
\begin{pgfscope}%
\pgfpathrectangle{\pgfqpoint{1.150000in}{0.150000in}}{\pgfqpoint{5.700000in}{5.700000in}}%
\pgfusepath{clip}%
\pgfsetbuttcap%
\pgfsetroundjoin%
\definecolor{currentfill}{rgb}{0.206756,0.371758,0.553117}%
\pgfsetfillcolor{currentfill}%
\pgfsetfillopacity{0.800000}%
\pgfsetlinewidth{0.000000pt}%
\definecolor{currentstroke}{rgb}{0.000000,0.000000,0.000000}%
\pgfsetstrokecolor{currentstroke}%
\pgfsetdash{}{0pt}%
\pgfpathmoveto{\pgfqpoint{4.525768in}{2.349478in}}%
\pgfpathlineto{\pgfqpoint{4.539942in}{2.358962in}}%
\pgfpathlineto{\pgfqpoint{4.554131in}{2.368631in}}%
\pgfpathlineto{\pgfqpoint{4.568336in}{2.378484in}}%
\pgfpathlineto{\pgfqpoint{4.582557in}{2.388522in}}%
\pgfpathlineto{\pgfqpoint{4.590465in}{2.399898in}}%
\pgfpathlineto{\pgfqpoint{4.598366in}{2.411152in}}%
\pgfpathlineto{\pgfqpoint{4.606262in}{2.422283in}}%
\pgfpathlineto{\pgfqpoint{4.614151in}{2.433292in}}%
\pgfpathlineto{\pgfqpoint{4.599933in}{2.423200in}}%
\pgfpathlineto{\pgfqpoint{4.585730in}{2.413293in}}%
\pgfpathlineto{\pgfqpoint{4.571542in}{2.403571in}}%
\pgfpathlineto{\pgfqpoint{4.557370in}{2.394034in}}%
\pgfpathlineto{\pgfqpoint{4.549479in}{2.383066in}}%
\pgfpathlineto{\pgfqpoint{4.541581in}{2.371984in}}%
\pgfpathlineto{\pgfqpoint{4.533677in}{2.360788in}}%
\pgfpathlineto{\pgfqpoint{4.525768in}{2.349478in}}%
\pgfpathclose%
\pgfusepath{fill}%
\end{pgfscope}%
\begin{pgfscope}%
\pgfpathrectangle{\pgfqpoint{1.150000in}{0.150000in}}{\pgfqpoint{5.700000in}{5.700000in}}%
\pgfusepath{clip}%
\pgfsetbuttcap%
\pgfsetroundjoin%
\definecolor{currentfill}{rgb}{0.269308,0.218818,0.509577}%
\pgfsetfillcolor{currentfill}%
\pgfsetfillopacity{0.800000}%
\pgfsetlinewidth{0.000000pt}%
\definecolor{currentstroke}{rgb}{0.000000,0.000000,0.000000}%
\pgfsetstrokecolor{currentstroke}%
\pgfsetdash{}{0pt}%
\pgfpathmoveto{\pgfqpoint{2.548625in}{2.032998in}}%
\pgfpathlineto{\pgfqpoint{2.562645in}{2.015182in}}%
\pgfpathlineto{\pgfqpoint{2.576656in}{1.997632in}}%
\pgfpathlineto{\pgfqpoint{2.590659in}{1.980345in}}%
\pgfpathlineto{\pgfqpoint{2.604654in}{1.963320in}}%
\pgfpathlineto{\pgfqpoint{2.613552in}{1.961506in}}%
\pgfpathlineto{\pgfqpoint{2.622432in}{1.960008in}}%
\pgfpathlineto{\pgfqpoint{2.631293in}{1.958819in}}%
\pgfpathlineto{\pgfqpoint{2.640136in}{1.957932in}}%
\pgfpathlineto{\pgfqpoint{2.626191in}{1.974303in}}%
\pgfpathlineto{\pgfqpoint{2.612237in}{1.990934in}}%
\pgfpathlineto{\pgfqpoint{2.598276in}{2.007828in}}%
\pgfpathlineto{\pgfqpoint{2.584307in}{2.024986in}}%
\pgfpathlineto{\pgfqpoint{2.575415in}{2.026515in}}%
\pgfpathlineto{\pgfqpoint{2.566505in}{2.028355in}}%
\pgfpathlineto{\pgfqpoint{2.557575in}{2.030514in}}%
\pgfpathlineto{\pgfqpoint{2.548625in}{2.032998in}}%
\pgfpathclose%
\pgfusepath{fill}%
\end{pgfscope}%
\begin{pgfscope}%
\pgfpathrectangle{\pgfqpoint{1.150000in}{0.150000in}}{\pgfqpoint{5.700000in}{5.700000in}}%
\pgfusepath{clip}%
\pgfsetbuttcap%
\pgfsetroundjoin%
\definecolor{currentfill}{rgb}{0.280255,0.165693,0.476498}%
\pgfsetfillcolor{currentfill}%
\pgfsetfillopacity{0.800000}%
\pgfsetlinewidth{0.000000pt}%
\definecolor{currentstroke}{rgb}{0.000000,0.000000,0.000000}%
\pgfsetstrokecolor{currentstroke}%
\pgfsetdash{}{0pt}%
\pgfpathmoveto{\pgfqpoint{2.660559in}{1.897797in}}%
\pgfpathlineto{\pgfqpoint{2.674518in}{1.882052in}}%
\pgfpathlineto{\pgfqpoint{2.688470in}{1.866556in}}%
\pgfpathlineto{\pgfqpoint{2.702416in}{1.851310in}}%
\pgfpathlineto{\pgfqpoint{2.716356in}{1.836310in}}%
\pgfpathlineto{\pgfqpoint{2.725159in}{1.835823in}}%
\pgfpathlineto{\pgfqpoint{2.733945in}{1.835634in}}%
\pgfpathlineto{\pgfqpoint{2.742713in}{1.835736in}}%
\pgfpathlineto{\pgfqpoint{2.751466in}{1.836123in}}%
\pgfpathlineto{\pgfqpoint{2.737570in}{1.850475in}}%
\pgfpathlineto{\pgfqpoint{2.723670in}{1.865074in}}%
\pgfpathlineto{\pgfqpoint{2.709763in}{1.879921in}}%
\pgfpathlineto{\pgfqpoint{2.695851in}{1.895017in}}%
\pgfpathlineto{\pgfqpoint{2.687054in}{1.895265in}}%
\pgfpathlineto{\pgfqpoint{2.678240in}{1.895807in}}%
\pgfpathlineto{\pgfqpoint{2.669408in}{1.896649in}}%
\pgfpathlineto{\pgfqpoint{2.660559in}{1.897797in}}%
\pgfpathclose%
\pgfusepath{fill}%
\end{pgfscope}%
\begin{pgfscope}%
\pgfpathrectangle{\pgfqpoint{1.150000in}{0.150000in}}{\pgfqpoint{5.700000in}{5.700000in}}%
\pgfusepath{clip}%
\pgfsetbuttcap%
\pgfsetroundjoin%
\definecolor{currentfill}{rgb}{0.268510,0.009605,0.335427}%
\pgfsetfillcolor{currentfill}%
\pgfsetfillopacity{0.800000}%
\pgfsetlinewidth{0.000000pt}%
\definecolor{currentstroke}{rgb}{0.000000,0.000000,0.000000}%
\pgfsetstrokecolor{currentstroke}%
\pgfsetdash{}{0pt}%
\pgfpathmoveto{\pgfqpoint{3.460641in}{1.526948in}}%
\pgfpathlineto{\pgfqpoint{3.474445in}{1.524034in}}%
\pgfpathlineto{\pgfqpoint{3.488254in}{1.521315in}}%
\pgfpathlineto{\pgfqpoint{3.502068in}{1.518790in}}%
\pgfpathlineto{\pgfqpoint{3.515887in}{1.516459in}}%
\pgfpathlineto{\pgfqpoint{3.524148in}{1.526167in}}%
\pgfpathlineto{\pgfqpoint{3.532402in}{1.535982in}}%
\pgfpathlineto{\pgfqpoint{3.540650in}{1.545899in}}%
\pgfpathlineto{\pgfqpoint{3.548890in}{1.555914in}}%
\pgfpathlineto{\pgfqpoint{3.535087in}{1.557746in}}%
\pgfpathlineto{\pgfqpoint{3.521289in}{1.559771in}}%
\pgfpathlineto{\pgfqpoint{3.507497in}{1.561991in}}%
\pgfpathlineto{\pgfqpoint{3.493709in}{1.564407in}}%
\pgfpathlineto{\pgfqpoint{3.485453in}{1.554879in}}%
\pgfpathlineto{\pgfqpoint{3.477190in}{1.545457in}}%
\pgfpathlineto{\pgfqpoint{3.468919in}{1.536145in}}%
\pgfpathlineto{\pgfqpoint{3.460641in}{1.526948in}}%
\pgfpathclose%
\pgfusepath{fill}%
\end{pgfscope}%
\begin{pgfscope}%
\pgfpathrectangle{\pgfqpoint{1.150000in}{0.150000in}}{\pgfqpoint{5.700000in}{5.700000in}}%
\pgfusepath{clip}%
\pgfsetbuttcap%
\pgfsetroundjoin%
\definecolor{currentfill}{rgb}{0.174274,0.445044,0.557792}%
\pgfsetfillcolor{currentfill}%
\pgfsetfillopacity{0.800000}%
\pgfsetlinewidth{0.000000pt}%
\definecolor{currentstroke}{rgb}{0.000000,0.000000,0.000000}%
\pgfsetstrokecolor{currentstroke}%
\pgfsetdash{}{0pt}%
\pgfpathmoveto{\pgfqpoint{4.734050in}{2.559222in}}%
\pgfpathlineto{\pgfqpoint{4.748345in}{2.570243in}}%
\pgfpathlineto{\pgfqpoint{4.762656in}{2.581449in}}%
\pgfpathlineto{\pgfqpoint{4.776985in}{2.592839in}}%
\pgfpathlineto{\pgfqpoint{4.791331in}{2.604414in}}%
\pgfpathlineto{\pgfqpoint{4.799154in}{2.614264in}}%
\pgfpathlineto{\pgfqpoint{4.806971in}{2.623980in}}%
\pgfpathlineto{\pgfqpoint{4.814780in}{2.633561in}}%
\pgfpathlineto{\pgfqpoint{4.822583in}{2.643009in}}%
\pgfpathlineto{\pgfqpoint{4.808241in}{2.631483in}}%
\pgfpathlineto{\pgfqpoint{4.793916in}{2.620140in}}%
\pgfpathlineto{\pgfqpoint{4.779609in}{2.608982in}}%
\pgfpathlineto{\pgfqpoint{4.765319in}{2.598009in}}%
\pgfpathlineto{\pgfqpoint{4.757512in}{2.588500in}}%
\pgfpathlineto{\pgfqpoint{4.749698in}{2.578866in}}%
\pgfpathlineto{\pgfqpoint{4.741877in}{2.569107in}}%
\pgfpathlineto{\pgfqpoint{4.734050in}{2.559222in}}%
\pgfpathclose%
\pgfusepath{fill}%
\end{pgfscope}%
\begin{pgfscope}%
\pgfpathrectangle{\pgfqpoint{1.150000in}{0.150000in}}{\pgfqpoint{5.700000in}{5.700000in}}%
\pgfusepath{clip}%
\pgfsetbuttcap%
\pgfsetroundjoin%
\definecolor{currentfill}{rgb}{0.180629,0.429975,0.557282}%
\pgfsetfillcolor{currentfill}%
\pgfsetfillopacity{0.800000}%
\pgfsetlinewidth{0.000000pt}%
\definecolor{currentstroke}{rgb}{0.000000,0.000000,0.000000}%
\pgfsetstrokecolor{currentstroke}%
\pgfsetdash{}{0pt}%
\pgfpathmoveto{\pgfqpoint{2.188788in}{2.622185in}}%
\pgfpathlineto{\pgfqpoint{2.203099in}{2.596642in}}%
\pgfpathlineto{\pgfqpoint{2.217393in}{2.571437in}}%
\pgfpathlineto{\pgfqpoint{2.231671in}{2.546566in}}%
\pgfpathlineto{\pgfqpoint{2.245933in}{2.522027in}}%
\pgfpathlineto{\pgfqpoint{2.255138in}{2.516904in}}%
\pgfpathlineto{\pgfqpoint{2.264319in}{2.512136in}}%
\pgfpathlineto{\pgfqpoint{2.273478in}{2.507717in}}%
\pgfpathlineto{\pgfqpoint{2.282614in}{2.503641in}}%
\pgfpathlineto{\pgfqpoint{2.268413in}{2.527527in}}%
\pgfpathlineto{\pgfqpoint{2.254197in}{2.551742in}}%
\pgfpathlineto{\pgfqpoint{2.239965in}{2.576290in}}%
\pgfpathlineto{\pgfqpoint{2.225717in}{2.601174in}}%
\pgfpathlineto{\pgfqpoint{2.216521in}{2.605892in}}%
\pgfpathlineto{\pgfqpoint{2.207301in}{2.610962in}}%
\pgfpathlineto{\pgfqpoint{2.198056in}{2.616391in}}%
\pgfpathlineto{\pgfqpoint{2.188788in}{2.622185in}}%
\pgfpathclose%
\pgfusepath{fill}%
\end{pgfscope}%
\begin{pgfscope}%
\pgfpathrectangle{\pgfqpoint{1.150000in}{0.150000in}}{\pgfqpoint{5.700000in}{5.700000in}}%
\pgfusepath{clip}%
\pgfsetbuttcap%
\pgfsetroundjoin%
\definecolor{currentfill}{rgb}{0.258965,0.251537,0.524736}%
\pgfsetfillcolor{currentfill}%
\pgfsetfillopacity{0.800000}%
\pgfsetlinewidth{0.000000pt}%
\definecolor{currentstroke}{rgb}{0.000000,0.000000,0.000000}%
\pgfsetstrokecolor{currentstroke}%
\pgfsetdash{}{0pt}%
\pgfpathmoveto{\pgfqpoint{2.492456in}{2.106959in}}%
\pgfpathlineto{\pgfqpoint{2.506513in}{2.088060in}}%
\pgfpathlineto{\pgfqpoint{2.520559in}{2.069435in}}%
\pgfpathlineto{\pgfqpoint{2.534597in}{2.051081in}}%
\pgfpathlineto{\pgfqpoint{2.548625in}{2.032998in}}%
\pgfpathlineto{\pgfqpoint{2.557575in}{2.030514in}}%
\pgfpathlineto{\pgfqpoint{2.566505in}{2.028355in}}%
\pgfpathlineto{\pgfqpoint{2.575415in}{2.026515in}}%
\pgfpathlineto{\pgfqpoint{2.584307in}{2.024986in}}%
\pgfpathlineto{\pgfqpoint{2.570330in}{2.042411in}}%
\pgfpathlineto{\pgfqpoint{2.556344in}{2.060105in}}%
\pgfpathlineto{\pgfqpoint{2.542350in}{2.078069in}}%
\pgfpathlineto{\pgfqpoint{2.528346in}{2.096306in}}%
\pgfpathlineto{\pgfqpoint{2.519404in}{2.098481in}}%
\pgfpathlineto{\pgfqpoint{2.510442in}{2.100977in}}%
\pgfpathlineto{\pgfqpoint{2.501459in}{2.103801in}}%
\pgfpathlineto{\pgfqpoint{2.492456in}{2.106959in}}%
\pgfpathclose%
\pgfusepath{fill}%
\end{pgfscope}%
\begin{pgfscope}%
\pgfpathrectangle{\pgfqpoint{1.150000in}{0.150000in}}{\pgfqpoint{5.700000in}{5.700000in}}%
\pgfusepath{clip}%
\pgfsetbuttcap%
\pgfsetroundjoin%
\definecolor{currentfill}{rgb}{0.282623,0.140926,0.457517}%
\pgfsetfillcolor{currentfill}%
\pgfsetfillopacity{0.800000}%
\pgfsetlinewidth{0.000000pt}%
\definecolor{currentstroke}{rgb}{0.000000,0.000000,0.000000}%
\pgfsetstrokecolor{currentstroke}%
\pgfsetdash{}{0pt}%
\pgfpathmoveto{\pgfqpoint{2.716356in}{1.836310in}}%
\pgfpathlineto{\pgfqpoint{2.730291in}{1.821556in}}%
\pgfpathlineto{\pgfqpoint{2.744220in}{1.807046in}}%
\pgfpathlineto{\pgfqpoint{2.758143in}{1.792777in}}%
\pgfpathlineto{\pgfqpoint{2.772062in}{1.778749in}}%
\pgfpathlineto{\pgfqpoint{2.780820in}{1.778920in}}%
\pgfpathlineto{\pgfqpoint{2.789562in}{1.779380in}}%
\pgfpathlineto{\pgfqpoint{2.798287in}{1.780123in}}%
\pgfpathlineto{\pgfqpoint{2.806997in}{1.781141in}}%
\pgfpathlineto{\pgfqpoint{2.793121in}{1.794525in}}%
\pgfpathlineto{\pgfqpoint{2.779241in}{1.808149in}}%
\pgfpathlineto{\pgfqpoint{2.765356in}{1.822015in}}%
\pgfpathlineto{\pgfqpoint{2.751466in}{1.836123in}}%
\pgfpathlineto{\pgfqpoint{2.742713in}{1.835736in}}%
\pgfpathlineto{\pgfqpoint{2.733945in}{1.835634in}}%
\pgfpathlineto{\pgfqpoint{2.725159in}{1.835823in}}%
\pgfpathlineto{\pgfqpoint{2.716356in}{1.836310in}}%
\pgfpathclose%
\pgfusepath{fill}%
\end{pgfscope}%
\begin{pgfscope}%
\pgfpathrectangle{\pgfqpoint{1.150000in}{0.150000in}}{\pgfqpoint{5.700000in}{5.700000in}}%
\pgfusepath{clip}%
\pgfsetbuttcap%
\pgfsetroundjoin%
\definecolor{currentfill}{rgb}{0.283072,0.130895,0.449241}%
\pgfsetfillcolor{currentfill}%
\pgfsetfillopacity{0.800000}%
\pgfsetlinewidth{0.000000pt}%
\definecolor{currentstroke}{rgb}{0.000000,0.000000,0.000000}%
\pgfsetstrokecolor{currentstroke}%
\pgfsetdash{}{0pt}%
\pgfpathmoveto{\pgfqpoint{3.901003in}{1.749522in}}%
\pgfpathlineto{\pgfqpoint{3.914897in}{1.752605in}}%
\pgfpathlineto{\pgfqpoint{3.928801in}{1.755874in}}%
\pgfpathlineto{\pgfqpoint{3.942714in}{1.759330in}}%
\pgfpathlineto{\pgfqpoint{3.956638in}{1.762972in}}%
\pgfpathlineto{\pgfqpoint{3.964739in}{1.775779in}}%
\pgfpathlineto{\pgfqpoint{3.972835in}{1.788574in}}%
\pgfpathlineto{\pgfqpoint{3.980927in}{1.801352in}}%
\pgfpathlineto{\pgfqpoint{3.989014in}{1.814113in}}%
\pgfpathlineto{\pgfqpoint{3.975095in}{1.810126in}}%
\pgfpathlineto{\pgfqpoint{3.961186in}{1.806326in}}%
\pgfpathlineto{\pgfqpoint{3.947287in}{1.802713in}}%
\pgfpathlineto{\pgfqpoint{3.933398in}{1.799287in}}%
\pgfpathlineto{\pgfqpoint{3.925307in}{1.786859in}}%
\pgfpathlineto{\pgfqpoint{3.917210in}{1.774420in}}%
\pgfpathlineto{\pgfqpoint{3.909109in}{1.761973in}}%
\pgfpathlineto{\pgfqpoint{3.901003in}{1.749522in}}%
\pgfpathclose%
\pgfusepath{fill}%
\end{pgfscope}%
\begin{pgfscope}%
\pgfpathrectangle{\pgfqpoint{1.150000in}{0.150000in}}{\pgfqpoint{5.700000in}{5.700000in}}%
\pgfusepath{clip}%
\pgfsetbuttcap%
\pgfsetroundjoin%
\definecolor{currentfill}{rgb}{0.120565,0.596422,0.543611}%
\pgfsetfillcolor{currentfill}%
\pgfsetfillopacity{0.800000}%
\pgfsetlinewidth{0.000000pt}%
\definecolor{currentstroke}{rgb}{0.000000,0.000000,0.000000}%
\pgfsetstrokecolor{currentstroke}%
\pgfsetdash{}{0pt}%
\pgfpathmoveto{\pgfqpoint{5.238766in}{3.027394in}}%
\pgfpathlineto{\pgfqpoint{5.253374in}{3.041040in}}%
\pgfpathlineto{\pgfqpoint{5.268003in}{3.054869in}}%
\pgfpathlineto{\pgfqpoint{5.282653in}{3.068883in}}%
\pgfpathlineto{\pgfqpoint{5.297322in}{3.083082in}}%
\pgfpathlineto{\pgfqpoint{5.304875in}{3.088409in}}%
\pgfpathlineto{\pgfqpoint{5.312419in}{3.093618in}}%
\pgfpathlineto{\pgfqpoint{5.319954in}{3.098712in}}%
\pgfpathlineto{\pgfqpoint{5.327479in}{3.103695in}}%
\pgfpathlineto{\pgfqpoint{5.312824in}{3.089790in}}%
\pgfpathlineto{\pgfqpoint{5.298189in}{3.076068in}}%
\pgfpathlineto{\pgfqpoint{5.283575in}{3.062529in}}%
\pgfpathlineto{\pgfqpoint{5.268980in}{3.049175in}}%
\pgfpathlineto{\pgfqpoint{5.261439in}{3.043888in}}%
\pgfpathlineto{\pgfqpoint{5.253890in}{3.038497in}}%
\pgfpathlineto{\pgfqpoint{5.246332in}{3.033001in}}%
\pgfpathlineto{\pgfqpoint{5.238766in}{3.027394in}}%
\pgfpathclose%
\pgfusepath{fill}%
\end{pgfscope}%
\begin{pgfscope}%
\pgfpathrectangle{\pgfqpoint{1.150000in}{0.150000in}}{\pgfqpoint{5.700000in}{5.700000in}}%
\pgfusepath{clip}%
\pgfsetbuttcap%
\pgfsetroundjoin%
\definecolor{currentfill}{rgb}{0.166383,0.690856,0.496502}%
\pgfsetfillcolor{currentfill}%
\pgfsetfillopacity{0.800000}%
\pgfsetlinewidth{0.000000pt}%
\definecolor{currentstroke}{rgb}{0.000000,0.000000,0.000000}%
\pgfsetstrokecolor{currentstroke}%
\pgfsetdash{}{0pt}%
\pgfpathmoveto{\pgfqpoint{5.623308in}{3.334427in}}%
\pgfpathlineto{\pgfqpoint{5.638161in}{3.349196in}}%
\pgfpathlineto{\pgfqpoint{5.653036in}{3.364147in}}%
\pgfpathlineto{\pgfqpoint{5.667934in}{3.379283in}}%
\pgfpathlineto{\pgfqpoint{5.682853in}{3.394601in}}%
\pgfpathlineto{\pgfqpoint{5.690145in}{3.396495in}}%
\pgfpathlineto{\pgfqpoint{5.697427in}{3.398321in}}%
\pgfpathlineto{\pgfqpoint{5.704701in}{3.400086in}}%
\pgfpathlineto{\pgfqpoint{5.711966in}{3.401795in}}%
\pgfpathlineto{\pgfqpoint{5.697072in}{3.386947in}}%
\pgfpathlineto{\pgfqpoint{5.682200in}{3.372282in}}%
\pgfpathlineto{\pgfqpoint{5.667350in}{3.357799in}}%
\pgfpathlineto{\pgfqpoint{5.652521in}{3.343498in}}%
\pgfpathlineto{\pgfqpoint{5.645231in}{3.341308in}}%
\pgfpathlineto{\pgfqpoint{5.637932in}{3.339070in}}%
\pgfpathlineto{\pgfqpoint{5.630624in}{3.336778in}}%
\pgfpathlineto{\pgfqpoint{5.623308in}{3.334427in}}%
\pgfpathclose%
\pgfusepath{fill}%
\end{pgfscope}%
\begin{pgfscope}%
\pgfpathrectangle{\pgfqpoint{1.150000in}{0.150000in}}{\pgfqpoint{5.700000in}{5.700000in}}%
\pgfusepath{clip}%
\pgfsetbuttcap%
\pgfsetroundjoin%
\definecolor{currentfill}{rgb}{0.267004,0.004874,0.329415}%
\pgfsetfillcolor{currentfill}%
\pgfsetfillopacity{0.800000}%
\pgfsetlinewidth{0.000000pt}%
\definecolor{currentstroke}{rgb}{0.000000,0.000000,0.000000}%
\pgfsetstrokecolor{currentstroke}%
\pgfsetdash{}{0pt}%
\pgfpathmoveto{\pgfqpoint{3.228281in}{1.521684in}}%
\pgfpathlineto{\pgfqpoint{3.242089in}{1.515280in}}%
\pgfpathlineto{\pgfqpoint{3.255898in}{1.509080in}}%
\pgfpathlineto{\pgfqpoint{3.269709in}{1.503083in}}%
\pgfpathlineto{\pgfqpoint{3.283523in}{1.497288in}}%
\pgfpathlineto{\pgfqpoint{3.291910in}{1.504231in}}%
\pgfpathlineto{\pgfqpoint{3.300287in}{1.511346in}}%
\pgfpathlineto{\pgfqpoint{3.308656in}{1.518629in}}%
\pgfpathlineto{\pgfqpoint{3.317015in}{1.526073in}}%
\pgfpathlineto{\pgfqpoint{3.303224in}{1.531305in}}%
\pgfpathlineto{\pgfqpoint{3.289437in}{1.536738in}}%
\pgfpathlineto{\pgfqpoint{3.275652in}{1.542375in}}%
\pgfpathlineto{\pgfqpoint{3.261869in}{1.548216in}}%
\pgfpathlineto{\pgfqpoint{3.253487in}{1.541322in}}%
\pgfpathlineto{\pgfqpoint{3.245095in}{1.534599in}}%
\pgfpathlineto{\pgfqpoint{3.236693in}{1.528051in}}%
\pgfpathlineto{\pgfqpoint{3.228281in}{1.521684in}}%
\pgfpathclose%
\pgfusepath{fill}%
\end{pgfscope}%
\begin{pgfscope}%
\pgfpathrectangle{\pgfqpoint{1.150000in}{0.150000in}}{\pgfqpoint{5.700000in}{5.700000in}}%
\pgfusepath{clip}%
\pgfsetbuttcap%
\pgfsetroundjoin%
\definecolor{currentfill}{rgb}{0.248629,0.278775,0.534556}%
\pgfsetfillcolor{currentfill}%
\pgfsetfillopacity{0.800000}%
\pgfsetlinewidth{0.000000pt}%
\definecolor{currentstroke}{rgb}{0.000000,0.000000,0.000000}%
\pgfsetstrokecolor{currentstroke}%
\pgfsetdash{}{0pt}%
\pgfpathmoveto{\pgfqpoint{2.436130in}{2.185341in}}%
\pgfpathlineto{\pgfqpoint{2.450228in}{2.165323in}}%
\pgfpathlineto{\pgfqpoint{2.464314in}{2.145588in}}%
\pgfpathlineto{\pgfqpoint{2.478390in}{2.126135in}}%
\pgfpathlineto{\pgfqpoint{2.492456in}{2.106959in}}%
\pgfpathlineto{\pgfqpoint{2.501459in}{2.103801in}}%
\pgfpathlineto{\pgfqpoint{2.510442in}{2.100977in}}%
\pgfpathlineto{\pgfqpoint{2.519404in}{2.098481in}}%
\pgfpathlineto{\pgfqpoint{2.528346in}{2.096306in}}%
\pgfpathlineto{\pgfqpoint{2.514334in}{2.114818in}}%
\pgfpathlineto{\pgfqpoint{2.500311in}{2.133608in}}%
\pgfpathlineto{\pgfqpoint{2.486279in}{2.152677in}}%
\pgfpathlineto{\pgfqpoint{2.472237in}{2.172028in}}%
\pgfpathlineto{\pgfqpoint{2.463241in}{2.174854in}}%
\pgfpathlineto{\pgfqpoint{2.454225in}{2.178010in}}%
\pgfpathlineto{\pgfqpoint{2.445189in}{2.181504in}}%
\pgfpathlineto{\pgfqpoint{2.436130in}{2.185341in}}%
\pgfpathclose%
\pgfusepath{fill}%
\end{pgfscope}%
\begin{pgfscope}%
\pgfpathrectangle{\pgfqpoint{1.150000in}{0.150000in}}{\pgfqpoint{5.700000in}{5.700000in}}%
\pgfusepath{clip}%
\pgfsetbuttcap%
\pgfsetroundjoin%
\definecolor{currentfill}{rgb}{0.283197,0.115680,0.436115}%
\pgfsetfillcolor{currentfill}%
\pgfsetfillopacity{0.800000}%
\pgfsetlinewidth{0.000000pt}%
\definecolor{currentstroke}{rgb}{0.000000,0.000000,0.000000}%
\pgfsetstrokecolor{currentstroke}%
\pgfsetdash{}{0pt}%
\pgfpathmoveto{\pgfqpoint{2.772062in}{1.778749in}}%
\pgfpathlineto{\pgfqpoint{2.785976in}{1.764960in}}%
\pgfpathlineto{\pgfqpoint{2.799885in}{1.751408in}}%
\pgfpathlineto{\pgfqpoint{2.813790in}{1.738092in}}%
\pgfpathlineto{\pgfqpoint{2.827690in}{1.725010in}}%
\pgfpathlineto{\pgfqpoint{2.836405in}{1.725836in}}%
\pgfpathlineto{\pgfqpoint{2.845105in}{1.726942in}}%
\pgfpathlineto{\pgfqpoint{2.853789in}{1.728322in}}%
\pgfpathlineto{\pgfqpoint{2.862458in}{1.729969in}}%
\pgfpathlineto{\pgfqpoint{2.848598in}{1.742410in}}%
\pgfpathlineto{\pgfqpoint{2.834735in}{1.755085in}}%
\pgfpathlineto{\pgfqpoint{2.820868in}{1.767994in}}%
\pgfpathlineto{\pgfqpoint{2.806997in}{1.781141in}}%
\pgfpathlineto{\pgfqpoint{2.798287in}{1.780123in}}%
\pgfpathlineto{\pgfqpoint{2.789562in}{1.779380in}}%
\pgfpathlineto{\pgfqpoint{2.780820in}{1.778920in}}%
\pgfpathlineto{\pgfqpoint{2.772062in}{1.778749in}}%
\pgfpathclose%
\pgfusepath{fill}%
\end{pgfscope}%
\begin{pgfscope}%
\pgfpathrectangle{\pgfqpoint{1.150000in}{0.150000in}}{\pgfqpoint{5.700000in}{5.700000in}}%
\pgfusepath{clip}%
\pgfsetbuttcap%
\pgfsetroundjoin%
\definecolor{currentfill}{rgb}{0.260571,0.246922,0.522828}%
\pgfsetfillcolor{currentfill}%
\pgfsetfillopacity{0.800000}%
\pgfsetlinewidth{0.000000pt}%
\definecolor{currentstroke}{rgb}{0.000000,0.000000,0.000000}%
\pgfsetstrokecolor{currentstroke}%
\pgfsetdash{}{0pt}%
\pgfpathmoveto{\pgfqpoint{4.197366in}{2.010114in}}%
\pgfpathlineto{\pgfqpoint{4.211380in}{2.016616in}}%
\pgfpathlineto{\pgfqpoint{4.225407in}{2.023303in}}%
\pgfpathlineto{\pgfqpoint{4.239446in}{2.030175in}}%
\pgfpathlineto{\pgfqpoint{4.253499in}{2.037232in}}%
\pgfpathlineto{\pgfqpoint{4.261517in}{2.050197in}}%
\pgfpathlineto{\pgfqpoint{4.269531in}{2.063086in}}%
\pgfpathlineto{\pgfqpoint{4.277540in}{2.075896in}}%
\pgfpathlineto{\pgfqpoint{4.285544in}{2.088626in}}%
\pgfpathlineto{\pgfqpoint{4.271493in}{2.081351in}}%
\pgfpathlineto{\pgfqpoint{4.257455in}{2.074261in}}%
\pgfpathlineto{\pgfqpoint{4.243430in}{2.067356in}}%
\pgfpathlineto{\pgfqpoint{4.229418in}{2.060637in}}%
\pgfpathlineto{\pgfqpoint{4.221413in}{2.048113in}}%
\pgfpathlineto{\pgfqpoint{4.213402in}{2.035517in}}%
\pgfpathlineto{\pgfqpoint{4.205387in}{2.022850in}}%
\pgfpathlineto{\pgfqpoint{4.197366in}{2.010114in}}%
\pgfpathclose%
\pgfusepath{fill}%
\end{pgfscope}%
\begin{pgfscope}%
\pgfpathrectangle{\pgfqpoint{1.150000in}{0.150000in}}{\pgfqpoint{5.700000in}{5.700000in}}%
\pgfusepath{clip}%
\pgfsetbuttcap%
\pgfsetroundjoin%
\definecolor{currentfill}{rgb}{0.267004,0.004874,0.329415}%
\pgfsetfillcolor{currentfill}%
\pgfsetfillopacity{0.800000}%
\pgfsetlinewidth{0.000000pt}%
\definecolor{currentstroke}{rgb}{0.000000,0.000000,0.000000}%
\pgfsetstrokecolor{currentstroke}%
\pgfsetdash{}{0pt}%
\pgfpathmoveto{\pgfqpoint{3.372205in}{1.507154in}}%
\pgfpathlineto{\pgfqpoint{3.386011in}{1.502923in}}%
\pgfpathlineto{\pgfqpoint{3.399821in}{1.498888in}}%
\pgfpathlineto{\pgfqpoint{3.413635in}{1.495051in}}%
\pgfpathlineto{\pgfqpoint{3.427453in}{1.491410in}}%
\pgfpathlineto{\pgfqpoint{3.435762in}{1.500097in}}%
\pgfpathlineto{\pgfqpoint{3.444063in}{1.508919in}}%
\pgfpathlineto{\pgfqpoint{3.452356in}{1.517871in}}%
\pgfpathlineto{\pgfqpoint{3.460641in}{1.526948in}}%
\pgfpathlineto{\pgfqpoint{3.446842in}{1.530058in}}%
\pgfpathlineto{\pgfqpoint{3.433047in}{1.533364in}}%
\pgfpathlineto{\pgfqpoint{3.419257in}{1.536868in}}%
\pgfpathlineto{\pgfqpoint{3.405470in}{1.540569in}}%
\pgfpathlineto{\pgfqpoint{3.397166in}{1.532011in}}%
\pgfpathlineto{\pgfqpoint{3.388854in}{1.523586in}}%
\pgfpathlineto{\pgfqpoint{3.380534in}{1.515298in}}%
\pgfpathlineto{\pgfqpoint{3.372205in}{1.507154in}}%
\pgfpathclose%
\pgfusepath{fill}%
\end{pgfscope}%
\begin{pgfscope}%
\pgfpathrectangle{\pgfqpoint{1.150000in}{0.150000in}}{\pgfqpoint{5.700000in}{5.700000in}}%
\pgfusepath{clip}%
\pgfsetbuttcap%
\pgfsetroundjoin%
\definecolor{currentfill}{rgb}{0.280868,0.160771,0.472899}%
\pgfsetfillcolor{currentfill}%
\pgfsetfillopacity{0.800000}%
\pgfsetlinewidth{0.000000pt}%
\definecolor{currentstroke}{rgb}{0.000000,0.000000,0.000000}%
\pgfsetstrokecolor{currentstroke}%
\pgfsetdash{}{0pt}%
\pgfpathmoveto{\pgfqpoint{3.989014in}{1.814113in}}%
\pgfpathlineto{\pgfqpoint{4.002943in}{1.818285in}}%
\pgfpathlineto{\pgfqpoint{4.016884in}{1.822644in}}%
\pgfpathlineto{\pgfqpoint{4.030835in}{1.827189in}}%
\pgfpathlineto{\pgfqpoint{4.044797in}{1.831919in}}%
\pgfpathlineto{\pgfqpoint{4.052876in}{1.844983in}}%
\pgfpathlineto{\pgfqpoint{4.060950in}{1.858014in}}%
\pgfpathlineto{\pgfqpoint{4.069019in}{1.871010in}}%
\pgfpathlineto{\pgfqpoint{4.077084in}{1.883967in}}%
\pgfpathlineto{\pgfqpoint{4.063125in}{1.878924in}}%
\pgfpathlineto{\pgfqpoint{4.049177in}{1.874066in}}%
\pgfpathlineto{\pgfqpoint{4.035240in}{1.869394in}}%
\pgfpathlineto{\pgfqpoint{4.021315in}{1.864909in}}%
\pgfpathlineto{\pgfqpoint{4.013246in}{1.852252in}}%
\pgfpathlineto{\pgfqpoint{4.005174in}{1.839565in}}%
\pgfpathlineto{\pgfqpoint{3.997096in}{1.826851in}}%
\pgfpathlineto{\pgfqpoint{3.989014in}{1.814113in}}%
\pgfpathclose%
\pgfusepath{fill}%
\end{pgfscope}%
\begin{pgfscope}%
\pgfpathrectangle{\pgfqpoint{1.150000in}{0.150000in}}{\pgfqpoint{5.700000in}{5.700000in}}%
\pgfusepath{clip}%
\pgfsetbuttcap%
\pgfsetroundjoin%
\definecolor{currentfill}{rgb}{0.227802,0.326594,0.546532}%
\pgfsetfillcolor{currentfill}%
\pgfsetfillopacity{0.800000}%
\pgfsetlinewidth{0.000000pt}%
\definecolor{currentstroke}{rgb}{0.000000,0.000000,0.000000}%
\pgfsetstrokecolor{currentstroke}%
\pgfsetdash{}{0pt}%
\pgfpathmoveto{\pgfqpoint{4.405724in}{2.219686in}}%
\pgfpathlineto{\pgfqpoint{4.419842in}{2.228223in}}%
\pgfpathlineto{\pgfqpoint{4.433974in}{2.236946in}}%
\pgfpathlineto{\pgfqpoint{4.448121in}{2.245853in}}%
\pgfpathlineto{\pgfqpoint{4.462283in}{2.254945in}}%
\pgfpathlineto{\pgfqpoint{4.470238in}{2.267151in}}%
\pgfpathlineto{\pgfqpoint{4.478188in}{2.279247in}}%
\pgfpathlineto{\pgfqpoint{4.486132in}{2.291232in}}%
\pgfpathlineto{\pgfqpoint{4.494071in}{2.303106in}}%
\pgfpathlineto{\pgfqpoint{4.479910in}{2.293894in}}%
\pgfpathlineto{\pgfqpoint{4.465764in}{2.284866in}}%
\pgfpathlineto{\pgfqpoint{4.451634in}{2.276024in}}%
\pgfpathlineto{\pgfqpoint{4.437517in}{2.267366in}}%
\pgfpathlineto{\pgfqpoint{4.429577in}{2.255600in}}%
\pgfpathlineto{\pgfqpoint{4.421632in}{2.243730in}}%
\pgfpathlineto{\pgfqpoint{4.413681in}{2.231759in}}%
\pgfpathlineto{\pgfqpoint{4.405724in}{2.219686in}}%
\pgfpathclose%
\pgfusepath{fill}%
\end{pgfscope}%
\begin{pgfscope}%
\pgfpathrectangle{\pgfqpoint{1.150000in}{0.150000in}}{\pgfqpoint{5.700000in}{5.700000in}}%
\pgfusepath{clip}%
\pgfsetbuttcap%
\pgfsetroundjoin%
\definecolor{currentfill}{rgb}{0.271305,0.019942,0.347269}%
\pgfsetfillcolor{currentfill}%
\pgfsetfillopacity{0.800000}%
\pgfsetlinewidth{0.000000pt}%
\definecolor{currentstroke}{rgb}{0.000000,0.000000,0.000000}%
\pgfsetstrokecolor{currentstroke}%
\pgfsetdash{}{0pt}%
\pgfpathmoveto{\pgfqpoint{3.083886in}{1.561604in}}%
\pgfpathlineto{\pgfqpoint{3.097715in}{1.552937in}}%
\pgfpathlineto{\pgfqpoint{3.111544in}{1.544481in}}%
\pgfpathlineto{\pgfqpoint{3.125374in}{1.536237in}}%
\pgfpathlineto{\pgfqpoint{3.139204in}{1.528202in}}%
\pgfpathlineto{\pgfqpoint{3.147686in}{1.533178in}}%
\pgfpathlineto{\pgfqpoint{3.156158in}{1.538367in}}%
\pgfpathlineto{\pgfqpoint{3.164618in}{1.543762in}}%
\pgfpathlineto{\pgfqpoint{3.173067in}{1.549358in}}%
\pgfpathlineto{\pgfqpoint{3.159266in}{1.556795in}}%
\pgfpathlineto{\pgfqpoint{3.145467in}{1.564442in}}%
\pgfpathlineto{\pgfqpoint{3.131667in}{1.572299in}}%
\pgfpathlineto{\pgfqpoint{3.117869in}{1.580368in}}%
\pgfpathlineto{\pgfqpoint{3.109390in}{1.575358in}}%
\pgfpathlineto{\pgfqpoint{3.100901in}{1.570556in}}%
\pgfpathlineto{\pgfqpoint{3.092399in}{1.565970in}}%
\pgfpathlineto{\pgfqpoint{3.083886in}{1.561604in}}%
\pgfpathclose%
\pgfusepath{fill}%
\end{pgfscope}%
\begin{pgfscope}%
\pgfpathrectangle{\pgfqpoint{1.150000in}{0.150000in}}{\pgfqpoint{5.700000in}{5.700000in}}%
\pgfusepath{clip}%
\pgfsetbuttcap%
\pgfsetroundjoin%
\definecolor{currentfill}{rgb}{0.202219,0.715272,0.476084}%
\pgfsetfillcolor{currentfill}%
\pgfsetfillopacity{0.800000}%
\pgfsetlinewidth{0.000000pt}%
\definecolor{currentstroke}{rgb}{0.000000,0.000000,0.000000}%
\pgfsetstrokecolor{currentstroke}%
\pgfsetdash{}{0pt}%
\pgfpathmoveto{\pgfqpoint{5.711966in}{3.401795in}}%
\pgfpathlineto{\pgfqpoint{5.726882in}{3.416826in}}%
\pgfpathlineto{\pgfqpoint{5.741821in}{3.432039in}}%
\pgfpathlineto{\pgfqpoint{5.756783in}{3.447437in}}%
\pgfpathlineto{\pgfqpoint{5.771767in}{3.463017in}}%
\pgfpathlineto{\pgfqpoint{5.778996in}{3.464181in}}%
\pgfpathlineto{\pgfqpoint{5.786216in}{3.465291in}}%
\pgfpathlineto{\pgfqpoint{5.793428in}{3.466352in}}%
\pgfpathlineto{\pgfqpoint{5.800631in}{3.467371in}}%
\pgfpathlineto{\pgfqpoint{5.785675in}{3.452297in}}%
\pgfpathlineto{\pgfqpoint{5.770741in}{3.437406in}}%
\pgfpathlineto{\pgfqpoint{5.755830in}{3.422698in}}%
\pgfpathlineto{\pgfqpoint{5.740941in}{3.408171in}}%
\pgfpathlineto{\pgfqpoint{5.733710in}{3.406635in}}%
\pgfpathlineto{\pgfqpoint{5.726470in}{3.405064in}}%
\pgfpathlineto{\pgfqpoint{5.719222in}{3.403452in}}%
\pgfpathlineto{\pgfqpoint{5.711966in}{3.401795in}}%
\pgfpathclose%
\pgfusepath{fill}%
\end{pgfscope}%
\begin{pgfscope}%
\pgfpathrectangle{\pgfqpoint{1.150000in}{0.150000in}}{\pgfqpoint{5.700000in}{5.700000in}}%
\pgfusepath{clip}%
\pgfsetbuttcap%
\pgfsetroundjoin%
\definecolor{currentfill}{rgb}{0.282327,0.094955,0.417331}%
\pgfsetfillcolor{currentfill}%
\pgfsetfillopacity{0.800000}%
\pgfsetlinewidth{0.000000pt}%
\definecolor{currentstroke}{rgb}{0.000000,0.000000,0.000000}%
\pgfsetstrokecolor{currentstroke}%
\pgfsetdash{}{0pt}%
\pgfpathmoveto{\pgfqpoint{2.827690in}{1.725010in}}%
\pgfpathlineto{\pgfqpoint{2.841587in}{1.712161in}}%
\pgfpathlineto{\pgfqpoint{2.855480in}{1.699544in}}%
\pgfpathlineto{\pgfqpoint{2.869369in}{1.687156in}}%
\pgfpathlineto{\pgfqpoint{2.883255in}{1.674998in}}%
\pgfpathlineto{\pgfqpoint{2.891930in}{1.676476in}}%
\pgfpathlineto{\pgfqpoint{2.900589in}{1.678226in}}%
\pgfpathlineto{\pgfqpoint{2.909234in}{1.680241in}}%
\pgfpathlineto{\pgfqpoint{2.917864in}{1.682514in}}%
\pgfpathlineto{\pgfqpoint{2.904017in}{1.694034in}}%
\pgfpathlineto{\pgfqpoint{2.890167in}{1.705783in}}%
\pgfpathlineto{\pgfqpoint{2.876314in}{1.717761in}}%
\pgfpathlineto{\pgfqpoint{2.862458in}{1.729969in}}%
\pgfpathlineto{\pgfqpoint{2.853789in}{1.728322in}}%
\pgfpathlineto{\pgfqpoint{2.845105in}{1.726942in}}%
\pgfpathlineto{\pgfqpoint{2.836405in}{1.725836in}}%
\pgfpathlineto{\pgfqpoint{2.827690in}{1.725010in}}%
\pgfpathclose%
\pgfusepath{fill}%
\end{pgfscope}%
\begin{pgfscope}%
\pgfpathrectangle{\pgfqpoint{1.150000in}{0.150000in}}{\pgfqpoint{5.700000in}{5.700000in}}%
\pgfusepath{clip}%
\pgfsetbuttcap%
\pgfsetroundjoin%
\definecolor{currentfill}{rgb}{0.137770,0.537492,0.554906}%
\pgfsetfillcolor{currentfill}%
\pgfsetfillopacity{0.800000}%
\pgfsetlinewidth{0.000000pt}%
\definecolor{currentstroke}{rgb}{0.000000,0.000000,0.000000}%
\pgfsetstrokecolor{currentstroke}%
\pgfsetdash{}{0pt}%
\pgfpathmoveto{\pgfqpoint{5.030859in}{2.842565in}}%
\pgfpathlineto{\pgfqpoint{5.045345in}{2.855402in}}%
\pgfpathlineto{\pgfqpoint{5.059850in}{2.868423in}}%
\pgfpathlineto{\pgfqpoint{5.074373in}{2.881629in}}%
\pgfpathlineto{\pgfqpoint{5.088916in}{2.895019in}}%
\pgfpathlineto{\pgfqpoint{5.096597in}{2.902352in}}%
\pgfpathlineto{\pgfqpoint{5.104269in}{2.909550in}}%
\pgfpathlineto{\pgfqpoint{5.111933in}{2.916615in}}%
\pgfpathlineto{\pgfqpoint{5.119588in}{2.923549in}}%
\pgfpathlineto{\pgfqpoint{5.105055in}{2.910346in}}%
\pgfpathlineto{\pgfqpoint{5.090540in}{2.897327in}}%
\pgfpathlineto{\pgfqpoint{5.076045in}{2.884493in}}%
\pgfpathlineto{\pgfqpoint{5.061568in}{2.871842in}}%
\pgfpathlineto{\pgfqpoint{5.053903in}{2.864708in}}%
\pgfpathlineto{\pgfqpoint{5.046230in}{2.857452in}}%
\pgfpathlineto{\pgfqpoint{5.038549in}{2.850072in}}%
\pgfpathlineto{\pgfqpoint{5.030859in}{2.842565in}}%
\pgfpathclose%
\pgfusepath{fill}%
\end{pgfscope}%
\begin{pgfscope}%
\pgfpathrectangle{\pgfqpoint{1.150000in}{0.150000in}}{\pgfqpoint{5.700000in}{5.700000in}}%
\pgfusepath{clip}%
\pgfsetbuttcap%
\pgfsetroundjoin%
\definecolor{currentfill}{rgb}{0.233603,0.313828,0.543914}%
\pgfsetfillcolor{currentfill}%
\pgfsetfillopacity{0.800000}%
\pgfsetlinewidth{0.000000pt}%
\definecolor{currentstroke}{rgb}{0.000000,0.000000,0.000000}%
\pgfsetstrokecolor{currentstroke}%
\pgfsetdash{}{0pt}%
\pgfpathmoveto{\pgfqpoint{2.379628in}{2.268292in}}%
\pgfpathlineto{\pgfqpoint{2.393771in}{2.247118in}}%
\pgfpathlineto{\pgfqpoint{2.407902in}{2.226236in}}%
\pgfpathlineto{\pgfqpoint{2.422022in}{2.205645in}}%
\pgfpathlineto{\pgfqpoint{2.436130in}{2.185341in}}%
\pgfpathlineto{\pgfqpoint{2.445189in}{2.181504in}}%
\pgfpathlineto{\pgfqpoint{2.454225in}{2.178010in}}%
\pgfpathlineto{\pgfqpoint{2.463241in}{2.174854in}}%
\pgfpathlineto{\pgfqpoint{2.472237in}{2.172028in}}%
\pgfpathlineto{\pgfqpoint{2.458184in}{2.191663in}}%
\pgfpathlineto{\pgfqpoint{2.444120in}{2.211584in}}%
\pgfpathlineto{\pgfqpoint{2.430046in}{2.231795in}}%
\pgfpathlineto{\pgfqpoint{2.415960in}{2.252297in}}%
\pgfpathlineto{\pgfqpoint{2.406910in}{2.255779in}}%
\pgfpathlineto{\pgfqpoint{2.397838in}{2.259601in}}%
\pgfpathlineto{\pgfqpoint{2.388744in}{2.263770in}}%
\pgfpathlineto{\pgfqpoint{2.379628in}{2.268292in}}%
\pgfpathclose%
\pgfusepath{fill}%
\end{pgfscope}%
\begin{pgfscope}%
\pgfpathrectangle{\pgfqpoint{1.150000in}{0.150000in}}{\pgfqpoint{5.700000in}{5.700000in}}%
\pgfusepath{clip}%
\pgfsetbuttcap%
\pgfsetroundjoin%
\definecolor{currentfill}{rgb}{0.190631,0.407061,0.556089}%
\pgfsetfillcolor{currentfill}%
\pgfsetfillopacity{0.800000}%
\pgfsetlinewidth{0.000000pt}%
\definecolor{currentstroke}{rgb}{0.000000,0.000000,0.000000}%
\pgfsetstrokecolor{currentstroke}%
\pgfsetdash{}{0pt}%
\pgfpathmoveto{\pgfqpoint{4.614151in}{2.433292in}}%
\pgfpathlineto{\pgfqpoint{4.628386in}{2.443568in}}%
\pgfpathlineto{\pgfqpoint{4.642637in}{2.454030in}}%
\pgfpathlineto{\pgfqpoint{4.656904in}{2.464675in}}%
\pgfpathlineto{\pgfqpoint{4.671188in}{2.475506in}}%
\pgfpathlineto{\pgfqpoint{4.679069in}{2.486425in}}%
\pgfpathlineto{\pgfqpoint{4.686943in}{2.497214in}}%
\pgfpathlineto{\pgfqpoint{4.694811in}{2.507872in}}%
\pgfpathlineto{\pgfqpoint{4.702672in}{2.518400in}}%
\pgfpathlineto{\pgfqpoint{4.688391in}{2.507550in}}%
\pgfpathlineto{\pgfqpoint{4.674126in}{2.496884in}}%
\pgfpathlineto{\pgfqpoint{4.659878in}{2.486403in}}%
\pgfpathlineto{\pgfqpoint{4.645646in}{2.476107in}}%
\pgfpathlineto{\pgfqpoint{4.637782in}{2.465586in}}%
\pgfpathlineto{\pgfqpoint{4.629911in}{2.454943in}}%
\pgfpathlineto{\pgfqpoint{4.622034in}{2.444179in}}%
\pgfpathlineto{\pgfqpoint{4.614151in}{2.433292in}}%
\pgfpathclose%
\pgfusepath{fill}%
\end{pgfscope}%
\begin{pgfscope}%
\pgfpathrectangle{\pgfqpoint{1.150000in}{0.150000in}}{\pgfqpoint{5.700000in}{5.700000in}}%
\pgfusepath{clip}%
\pgfsetbuttcap%
\pgfsetroundjoin%
\definecolor{currentfill}{rgb}{0.120081,0.622161,0.534946}%
\pgfsetfillcolor{currentfill}%
\pgfsetfillopacity{0.800000}%
\pgfsetlinewidth{0.000000pt}%
\definecolor{currentstroke}{rgb}{0.000000,0.000000,0.000000}%
\pgfsetstrokecolor{currentstroke}%
\pgfsetdash{}{0pt}%
\pgfpathmoveto{\pgfqpoint{5.327479in}{3.103695in}}%
\pgfpathlineto{\pgfqpoint{5.342155in}{3.117785in}}%
\pgfpathlineto{\pgfqpoint{5.356852in}{3.132058in}}%
\pgfpathlineto{\pgfqpoint{5.371569in}{3.146517in}}%
\pgfpathlineto{\pgfqpoint{5.386307in}{3.161159in}}%
\pgfpathlineto{\pgfqpoint{5.393808in}{3.165718in}}%
\pgfpathlineto{\pgfqpoint{5.401300in}{3.170165in}}%
\pgfpathlineto{\pgfqpoint{5.408783in}{3.174503in}}%
\pgfpathlineto{\pgfqpoint{5.416257in}{3.178736in}}%
\pgfpathlineto{\pgfqpoint{5.401535in}{3.164423in}}%
\pgfpathlineto{\pgfqpoint{5.386834in}{3.150293in}}%
\pgfpathlineto{\pgfqpoint{5.372154in}{3.136347in}}%
\pgfpathlineto{\pgfqpoint{5.357494in}{3.122584in}}%
\pgfpathlineto{\pgfqpoint{5.350004in}{3.118011in}}%
\pgfpathlineto{\pgfqpoint{5.342505in}{3.113341in}}%
\pgfpathlineto{\pgfqpoint{5.334996in}{3.108570in}}%
\pgfpathlineto{\pgfqpoint{5.327479in}{3.103695in}}%
\pgfpathclose%
\pgfusepath{fill}%
\end{pgfscope}%
\begin{pgfscope}%
\pgfpathrectangle{\pgfqpoint{1.150000in}{0.150000in}}{\pgfqpoint{5.700000in}{5.700000in}}%
\pgfusepath{clip}%
\pgfsetbuttcap%
\pgfsetroundjoin%
\definecolor{currentfill}{rgb}{0.162142,0.474838,0.558140}%
\pgfsetfillcolor{currentfill}%
\pgfsetfillopacity{0.800000}%
\pgfsetlinewidth{0.000000pt}%
\definecolor{currentstroke}{rgb}{0.000000,0.000000,0.000000}%
\pgfsetstrokecolor{currentstroke}%
\pgfsetdash{}{0pt}%
\pgfpathmoveto{\pgfqpoint{4.822583in}{2.643009in}}%
\pgfpathlineto{\pgfqpoint{4.836942in}{2.654721in}}%
\pgfpathlineto{\pgfqpoint{4.851319in}{2.666617in}}%
\pgfpathlineto{\pgfqpoint{4.865713in}{2.678698in}}%
\pgfpathlineto{\pgfqpoint{4.880126in}{2.690964in}}%
\pgfpathlineto{\pgfqpoint{4.887916in}{2.700211in}}%
\pgfpathlineto{\pgfqpoint{4.895699in}{2.709318in}}%
\pgfpathlineto{\pgfqpoint{4.903474in}{2.718288in}}%
\pgfpathlineto{\pgfqpoint{4.911241in}{2.727121in}}%
\pgfpathlineto{\pgfqpoint{4.896834in}{2.714938in}}%
\pgfpathlineto{\pgfqpoint{4.882444in}{2.702940in}}%
\pgfpathlineto{\pgfqpoint{4.868073in}{2.691126in}}%
\pgfpathlineto{\pgfqpoint{4.853719in}{2.679497in}}%
\pgfpathlineto{\pgfqpoint{4.845946in}{2.670568in}}%
\pgfpathlineto{\pgfqpoint{4.838165in}{2.661512in}}%
\pgfpathlineto{\pgfqpoint{4.830377in}{2.652326in}}%
\pgfpathlineto{\pgfqpoint{4.822583in}{2.643009in}}%
\pgfpathclose%
\pgfusepath{fill}%
\end{pgfscope}%
\begin{pgfscope}%
\pgfpathrectangle{\pgfqpoint{1.150000in}{0.150000in}}{\pgfqpoint{5.700000in}{5.700000in}}%
\pgfusepath{clip}%
\pgfsetbuttcap%
\pgfsetroundjoin%
\definecolor{currentfill}{rgb}{0.165117,0.467423,0.558141}%
\pgfsetfillcolor{currentfill}%
\pgfsetfillopacity{0.800000}%
\pgfsetlinewidth{0.000000pt}%
\definecolor{currentstroke}{rgb}{0.000000,0.000000,0.000000}%
\pgfsetstrokecolor{currentstroke}%
\pgfsetdash{}{0pt}%
\pgfpathmoveto{\pgfqpoint{2.131370in}{2.727805in}}%
\pgfpathlineto{\pgfqpoint{2.145752in}{2.700876in}}%
\pgfpathlineto{\pgfqpoint{2.160115in}{2.674298in}}%
\pgfpathlineto{\pgfqpoint{2.174460in}{2.648069in}}%
\pgfpathlineto{\pgfqpoint{2.188788in}{2.622185in}}%
\pgfpathlineto{\pgfqpoint{2.198056in}{2.616391in}}%
\pgfpathlineto{\pgfqpoint{2.207301in}{2.610962in}}%
\pgfpathlineto{\pgfqpoint{2.216521in}{2.605892in}}%
\pgfpathlineto{\pgfqpoint{2.225717in}{2.601174in}}%
\pgfpathlineto{\pgfqpoint{2.211453in}{2.626398in}}%
\pgfpathlineto{\pgfqpoint{2.197172in}{2.651964in}}%
\pgfpathlineto{\pgfqpoint{2.182874in}{2.677877in}}%
\pgfpathlineto{\pgfqpoint{2.168558in}{2.704140in}}%
\pgfpathlineto{\pgfqpoint{2.159298in}{2.709506in}}%
\pgfpathlineto{\pgfqpoint{2.150014in}{2.715235in}}%
\pgfpathlineto{\pgfqpoint{2.140705in}{2.721332in}}%
\pgfpathlineto{\pgfqpoint{2.131370in}{2.727805in}}%
\pgfpathclose%
\pgfusepath{fill}%
\end{pgfscope}%
\begin{pgfscope}%
\pgfpathrectangle{\pgfqpoint{1.150000in}{0.150000in}}{\pgfqpoint{5.700000in}{5.700000in}}%
\pgfusepath{clip}%
\pgfsetbuttcap%
\pgfsetroundjoin%
\definecolor{currentfill}{rgb}{0.239374,0.735588,0.455688}%
\pgfsetfillcolor{currentfill}%
\pgfsetfillopacity{0.800000}%
\pgfsetlinewidth{0.000000pt}%
\definecolor{currentstroke}{rgb}{0.000000,0.000000,0.000000}%
\pgfsetstrokecolor{currentstroke}%
\pgfsetdash{}{0pt}%
\pgfpathmoveto{\pgfqpoint{5.800631in}{3.467371in}}%
\pgfpathlineto{\pgfqpoint{5.815610in}{3.482627in}}%
\pgfpathlineto{\pgfqpoint{5.830612in}{3.498066in}}%
\pgfpathlineto{\pgfqpoint{5.845637in}{3.513688in}}%
\pgfpathlineto{\pgfqpoint{5.860685in}{3.529494in}}%
\pgfpathlineto{\pgfqpoint{5.867849in}{3.529946in}}%
\pgfpathlineto{\pgfqpoint{5.875006in}{3.530358in}}%
\pgfpathlineto{\pgfqpoint{5.882153in}{3.530737in}}%
\pgfpathlineto{\pgfqpoint{5.889293in}{3.531088in}}%
\pgfpathlineto{\pgfqpoint{5.874276in}{3.515826in}}%
\pgfpathlineto{\pgfqpoint{5.859281in}{3.500746in}}%
\pgfpathlineto{\pgfqpoint{5.844310in}{3.485848in}}%
\pgfpathlineto{\pgfqpoint{5.829361in}{3.471132in}}%
\pgfpathlineto{\pgfqpoint{5.822191in}{3.470227in}}%
\pgfpathlineto{\pgfqpoint{5.815012in}{3.469303in}}%
\pgfpathlineto{\pgfqpoint{5.807826in}{3.468352in}}%
\pgfpathlineto{\pgfqpoint{5.800631in}{3.467371in}}%
\pgfpathclose%
\pgfusepath{fill}%
\end{pgfscope}%
\begin{pgfscope}%
\pgfpathrectangle{\pgfqpoint{1.150000in}{0.150000in}}{\pgfqpoint{5.700000in}{5.700000in}}%
\pgfusepath{clip}%
\pgfsetbuttcap%
\pgfsetroundjoin%
\definecolor{currentfill}{rgb}{0.280894,0.078907,0.402329}%
\pgfsetfillcolor{currentfill}%
\pgfsetfillopacity{0.800000}%
\pgfsetlinewidth{0.000000pt}%
\definecolor{currentstroke}{rgb}{0.000000,0.000000,0.000000}%
\pgfsetstrokecolor{currentstroke}%
\pgfsetdash{}{0pt}%
\pgfpathmoveto{\pgfqpoint{2.883255in}{1.674998in}}%
\pgfpathlineto{\pgfqpoint{2.897138in}{1.663066in}}%
\pgfpathlineto{\pgfqpoint{2.911019in}{1.651361in}}%
\pgfpathlineto{\pgfqpoint{2.924896in}{1.639881in}}%
\pgfpathlineto{\pgfqpoint{2.938771in}{1.628624in}}%
\pgfpathlineto{\pgfqpoint{2.947407in}{1.630752in}}%
\pgfpathlineto{\pgfqpoint{2.956028in}{1.633143in}}%
\pgfpathlineto{\pgfqpoint{2.964636in}{1.635790in}}%
\pgfpathlineto{\pgfqpoint{2.973229in}{1.638687in}}%
\pgfpathlineto{\pgfqpoint{2.959391in}{1.649308in}}%
\pgfpathlineto{\pgfqpoint{2.945551in}{1.660152in}}%
\pgfpathlineto{\pgfqpoint{2.931709in}{1.671220in}}%
\pgfpathlineto{\pgfqpoint{2.917864in}{1.682514in}}%
\pgfpathlineto{\pgfqpoint{2.909234in}{1.680241in}}%
\pgfpathlineto{\pgfqpoint{2.900589in}{1.678226in}}%
\pgfpathlineto{\pgfqpoint{2.891930in}{1.676476in}}%
\pgfpathlineto{\pgfqpoint{2.883255in}{1.674998in}}%
\pgfpathclose%
\pgfusepath{fill}%
\end{pgfscope}%
\begin{pgfscope}%
\pgfpathrectangle{\pgfqpoint{1.150000in}{0.150000in}}{\pgfqpoint{5.700000in}{5.700000in}}%
\pgfusepath{clip}%
\pgfsetbuttcap%
\pgfsetroundjoin%
\definecolor{currentfill}{rgb}{0.278791,0.062145,0.386592}%
\pgfsetfillcolor{currentfill}%
\pgfsetfillopacity{0.800000}%
\pgfsetlinewidth{0.000000pt}%
\definecolor{currentstroke}{rgb}{0.000000,0.000000,0.000000}%
\pgfsetstrokecolor{currentstroke}%
\pgfsetdash{}{0pt}%
\pgfpathmoveto{\pgfqpoint{3.692331in}{1.592867in}}%
\pgfpathlineto{\pgfqpoint{3.706180in}{1.593226in}}%
\pgfpathlineto{\pgfqpoint{3.720037in}{1.593774in}}%
\pgfpathlineto{\pgfqpoint{3.733901in}{1.594511in}}%
\pgfpathlineto{\pgfqpoint{3.747774in}{1.595436in}}%
\pgfpathlineto{\pgfqpoint{3.755948in}{1.607217in}}%
\pgfpathlineto{\pgfqpoint{3.764117in}{1.619045in}}%
\pgfpathlineto{\pgfqpoint{3.772281in}{1.630916in}}%
\pgfpathlineto{\pgfqpoint{3.780439in}{1.642827in}}%
\pgfpathlineto{\pgfqpoint{3.766576in}{1.641464in}}%
\pgfpathlineto{\pgfqpoint{3.752721in}{1.640290in}}%
\pgfpathlineto{\pgfqpoint{3.738874in}{1.639305in}}%
\pgfpathlineto{\pgfqpoint{3.725034in}{1.638509in}}%
\pgfpathlineto{\pgfqpoint{3.716867in}{1.627023in}}%
\pgfpathlineto{\pgfqpoint{3.708694in}{1.615585in}}%
\pgfpathlineto{\pgfqpoint{3.700515in}{1.604199in}}%
\pgfpathlineto{\pgfqpoint{3.692331in}{1.592867in}}%
\pgfpathclose%
\pgfusepath{fill}%
\end{pgfscope}%
\begin{pgfscope}%
\pgfpathrectangle{\pgfqpoint{1.150000in}{0.150000in}}{\pgfqpoint{5.700000in}{5.700000in}}%
\pgfusepath{clip}%
\pgfsetbuttcap%
\pgfsetroundjoin%
\definecolor{currentfill}{rgb}{0.274952,0.037752,0.364543}%
\pgfsetfillcolor{currentfill}%
\pgfsetfillopacity{0.800000}%
\pgfsetlinewidth{0.000000pt}%
\definecolor{currentstroke}{rgb}{0.000000,0.000000,0.000000}%
\pgfsetstrokecolor{currentstroke}%
\pgfsetdash{}{0pt}%
\pgfpathmoveto{\pgfqpoint{3.604162in}{1.550516in}}%
\pgfpathlineto{\pgfqpoint{3.617995in}{1.549646in}}%
\pgfpathlineto{\pgfqpoint{3.631835in}{1.548967in}}%
\pgfpathlineto{\pgfqpoint{3.645682in}{1.548478in}}%
\pgfpathlineto{\pgfqpoint{3.659535in}{1.548179in}}%
\pgfpathlineto{\pgfqpoint{3.667743in}{1.559247in}}%
\pgfpathlineto{\pgfqpoint{3.675945in}{1.570387in}}%
\pgfpathlineto{\pgfqpoint{3.684141in}{1.581595in}}%
\pgfpathlineto{\pgfqpoint{3.692331in}{1.592867in}}%
\pgfpathlineto{\pgfqpoint{3.678489in}{1.592698in}}%
\pgfpathlineto{\pgfqpoint{3.664654in}{1.592718in}}%
\pgfpathlineto{\pgfqpoint{3.650826in}{1.592929in}}%
\pgfpathlineto{\pgfqpoint{3.637005in}{1.593331in}}%
\pgfpathlineto{\pgfqpoint{3.628804in}{1.582516in}}%
\pgfpathlineto{\pgfqpoint{3.620596in}{1.571772in}}%
\pgfpathlineto{\pgfqpoint{3.612382in}{1.561104in}}%
\pgfpathlineto{\pgfqpoint{3.604162in}{1.550516in}}%
\pgfpathclose%
\pgfusepath{fill}%
\end{pgfscope}%
\begin{pgfscope}%
\pgfpathrectangle{\pgfqpoint{1.150000in}{0.150000in}}{\pgfqpoint{5.700000in}{5.700000in}}%
\pgfusepath{clip}%
\pgfsetbuttcap%
\pgfsetroundjoin%
\definecolor{currentfill}{rgb}{0.218130,0.347432,0.550038}%
\pgfsetfillcolor{currentfill}%
\pgfsetfillopacity{0.800000}%
\pgfsetlinewidth{0.000000pt}%
\definecolor{currentstroke}{rgb}{0.000000,0.000000,0.000000}%
\pgfsetstrokecolor{currentstroke}%
\pgfsetdash{}{0pt}%
\pgfpathmoveto{\pgfqpoint{2.322930in}{2.355972in}}%
\pgfpathlineto{\pgfqpoint{2.337124in}{2.333599in}}%
\pgfpathlineto{\pgfqpoint{2.351305in}{2.311530in}}%
\pgfpathlineto{\pgfqpoint{2.365473in}{2.289762in}}%
\pgfpathlineto{\pgfqpoint{2.379628in}{2.268292in}}%
\pgfpathlineto{\pgfqpoint{2.388744in}{2.263770in}}%
\pgfpathlineto{\pgfqpoint{2.397838in}{2.259601in}}%
\pgfpathlineto{\pgfqpoint{2.406910in}{2.255779in}}%
\pgfpathlineto{\pgfqpoint{2.415960in}{2.252297in}}%
\pgfpathlineto{\pgfqpoint{2.401862in}{2.273093in}}%
\pgfpathlineto{\pgfqpoint{2.387753in}{2.294186in}}%
\pgfpathlineto{\pgfqpoint{2.373631in}{2.315578in}}%
\pgfpathlineto{\pgfqpoint{2.359497in}{2.337272in}}%
\pgfpathlineto{\pgfqpoint{2.350390in}{2.341416in}}%
\pgfpathlineto{\pgfqpoint{2.341260in}{2.345909in}}%
\pgfpathlineto{\pgfqpoint{2.332107in}{2.350759in}}%
\pgfpathlineto{\pgfqpoint{2.322930in}{2.355972in}}%
\pgfpathclose%
\pgfusepath{fill}%
\end{pgfscope}%
\begin{pgfscope}%
\pgfpathrectangle{\pgfqpoint{1.150000in}{0.150000in}}{\pgfqpoint{5.700000in}{5.700000in}}%
\pgfusepath{clip}%
\pgfsetbuttcap%
\pgfsetroundjoin%
\definecolor{currentfill}{rgb}{0.275191,0.194905,0.496005}%
\pgfsetfillcolor{currentfill}%
\pgfsetfillopacity{0.800000}%
\pgfsetlinewidth{0.000000pt}%
\definecolor{currentstroke}{rgb}{0.000000,0.000000,0.000000}%
\pgfsetstrokecolor{currentstroke}%
\pgfsetdash{}{0pt}%
\pgfpathmoveto{\pgfqpoint{4.077084in}{1.883967in}}%
\pgfpathlineto{\pgfqpoint{4.091054in}{1.889197in}}%
\pgfpathlineto{\pgfqpoint{4.105036in}{1.894611in}}%
\pgfpathlineto{\pgfqpoint{4.119030in}{1.900211in}}%
\pgfpathlineto{\pgfqpoint{4.133035in}{1.905995in}}%
\pgfpathlineto{\pgfqpoint{4.141093in}{1.919206in}}%
\pgfpathlineto{\pgfqpoint{4.149146in}{1.932366in}}%
\pgfpathlineto{\pgfqpoint{4.157194in}{1.945472in}}%
\pgfpathlineto{\pgfqpoint{4.165238in}{1.958522in}}%
\pgfpathlineto{\pgfqpoint{4.151235in}{1.952455in}}%
\pgfpathlineto{\pgfqpoint{4.137243in}{1.946573in}}%
\pgfpathlineto{\pgfqpoint{4.123264in}{1.940877in}}%
\pgfpathlineto{\pgfqpoint{4.109296in}{1.935366in}}%
\pgfpathlineto{\pgfqpoint{4.101250in}{1.922586in}}%
\pgfpathlineto{\pgfqpoint{4.093199in}{1.909758in}}%
\pgfpathlineto{\pgfqpoint{4.085144in}{1.896884in}}%
\pgfpathlineto{\pgfqpoint{4.077084in}{1.883967in}}%
\pgfpathclose%
\pgfusepath{fill}%
\end{pgfscope}%
\begin{pgfscope}%
\pgfpathrectangle{\pgfqpoint{1.150000in}{0.150000in}}{\pgfqpoint{5.700000in}{5.700000in}}%
\pgfusepath{clip}%
\pgfsetbuttcap%
\pgfsetroundjoin%
\definecolor{currentfill}{rgb}{0.246811,0.283237,0.535941}%
\pgfsetfillcolor{currentfill}%
\pgfsetfillopacity{0.800000}%
\pgfsetlinewidth{0.000000pt}%
\definecolor{currentstroke}{rgb}{0.000000,0.000000,0.000000}%
\pgfsetstrokecolor{currentstroke}%
\pgfsetdash{}{0pt}%
\pgfpathmoveto{\pgfqpoint{4.285544in}{2.088626in}}%
\pgfpathlineto{\pgfqpoint{4.299609in}{2.096086in}}%
\pgfpathlineto{\pgfqpoint{4.313686in}{2.103730in}}%
\pgfpathlineto{\pgfqpoint{4.327778in}{2.111560in}}%
\pgfpathlineto{\pgfqpoint{4.341883in}{2.119574in}}%
\pgfpathlineto{\pgfqpoint{4.349882in}{2.132419in}}%
\pgfpathlineto{\pgfqpoint{4.357875in}{2.145173in}}%
\pgfpathlineto{\pgfqpoint{4.365863in}{2.157833in}}%
\pgfpathlineto{\pgfqpoint{4.373845in}{2.170398in}}%
\pgfpathlineto{\pgfqpoint{4.359741in}{2.162198in}}%
\pgfpathlineto{\pgfqpoint{4.345650in}{2.154183in}}%
\pgfpathlineto{\pgfqpoint{4.331574in}{2.146353in}}%
\pgfpathlineto{\pgfqpoint{4.317510in}{2.138707in}}%
\pgfpathlineto{\pgfqpoint{4.309526in}{2.126315in}}%
\pgfpathlineto{\pgfqpoint{4.301537in}{2.113837in}}%
\pgfpathlineto{\pgfqpoint{4.293543in}{2.101273in}}%
\pgfpathlineto{\pgfqpoint{4.285544in}{2.088626in}}%
\pgfpathclose%
\pgfusepath{fill}%
\end{pgfscope}%
\begin{pgfscope}%
\pgfpathrectangle{\pgfqpoint{1.150000in}{0.150000in}}{\pgfqpoint{5.700000in}{5.700000in}}%
\pgfusepath{clip}%
\pgfsetbuttcap%
\pgfsetroundjoin%
\definecolor{currentfill}{rgb}{0.281446,0.084320,0.407414}%
\pgfsetfillcolor{currentfill}%
\pgfsetfillopacity{0.800000}%
\pgfsetlinewidth{0.000000pt}%
\definecolor{currentstroke}{rgb}{0.000000,0.000000,0.000000}%
\pgfsetstrokecolor{currentstroke}%
\pgfsetdash{}{0pt}%
\pgfpathmoveto{\pgfqpoint{3.780439in}{1.642827in}}%
\pgfpathlineto{\pgfqpoint{3.794311in}{1.644378in}}%
\pgfpathlineto{\pgfqpoint{3.808191in}{1.646117in}}%
\pgfpathlineto{\pgfqpoint{3.822080in}{1.648043in}}%
\pgfpathlineto{\pgfqpoint{3.835978in}{1.650155in}}%
\pgfpathlineto{\pgfqpoint{3.844123in}{1.662520in}}%
\pgfpathlineto{\pgfqpoint{3.852264in}{1.674908in}}%
\pgfpathlineto{\pgfqpoint{3.860399in}{1.687316in}}%
\pgfpathlineto{\pgfqpoint{3.868530in}{1.699739in}}%
\pgfpathlineto{\pgfqpoint{3.854639in}{1.697219in}}%
\pgfpathlineto{\pgfqpoint{3.840757in}{1.694887in}}%
\pgfpathlineto{\pgfqpoint{3.826885in}{1.692741in}}%
\pgfpathlineto{\pgfqpoint{3.813021in}{1.690784in}}%
\pgfpathlineto{\pgfqpoint{3.804883in}{1.678755in}}%
\pgfpathlineto{\pgfqpoint{3.796740in}{1.666750in}}%
\pgfpathlineto{\pgfqpoint{3.788593in}{1.654773in}}%
\pgfpathlineto{\pgfqpoint{3.780439in}{1.642827in}}%
\pgfpathclose%
\pgfusepath{fill}%
\end{pgfscope}%
\begin{pgfscope}%
\pgfpathrectangle{\pgfqpoint{1.150000in}{0.150000in}}{\pgfqpoint{5.700000in}{5.700000in}}%
\pgfusepath{clip}%
\pgfsetbuttcap%
\pgfsetroundjoin%
\definecolor{currentfill}{rgb}{0.267004,0.004874,0.329415}%
\pgfsetfillcolor{currentfill}%
\pgfsetfillopacity{0.800000}%
\pgfsetlinewidth{0.000000pt}%
\definecolor{currentstroke}{rgb}{0.000000,0.000000,0.000000}%
\pgfsetstrokecolor{currentstroke}%
\pgfsetdash{}{0pt}%
\pgfpathmoveto{\pgfqpoint{3.283523in}{1.497288in}}%
\pgfpathlineto{\pgfqpoint{3.297339in}{1.491695in}}%
\pgfpathlineto{\pgfqpoint{3.311158in}{1.486302in}}%
\pgfpathlineto{\pgfqpoint{3.324980in}{1.481110in}}%
\pgfpathlineto{\pgfqpoint{3.338804in}{1.476116in}}%
\pgfpathlineto{\pgfqpoint{3.347168in}{1.483634in}}%
\pgfpathlineto{\pgfqpoint{3.355522in}{1.491317in}}%
\pgfpathlineto{\pgfqpoint{3.363868in}{1.499159in}}%
\pgfpathlineto{\pgfqpoint{3.372205in}{1.507154in}}%
\pgfpathlineto{\pgfqpoint{3.358403in}{1.511585in}}%
\pgfpathlineto{\pgfqpoint{3.344603in}{1.516214in}}%
\pgfpathlineto{\pgfqpoint{3.330807in}{1.521043in}}%
\pgfpathlineto{\pgfqpoint{3.317015in}{1.526073in}}%
\pgfpathlineto{\pgfqpoint{3.308656in}{1.518629in}}%
\pgfpathlineto{\pgfqpoint{3.300287in}{1.511346in}}%
\pgfpathlineto{\pgfqpoint{3.291910in}{1.504231in}}%
\pgfpathlineto{\pgfqpoint{3.283523in}{1.497288in}}%
\pgfpathclose%
\pgfusepath{fill}%
\end{pgfscope}%
\begin{pgfscope}%
\pgfpathrectangle{\pgfqpoint{1.150000in}{0.150000in}}{\pgfqpoint{5.700000in}{5.700000in}}%
\pgfusepath{clip}%
\pgfsetbuttcap%
\pgfsetroundjoin%
\definecolor{currentfill}{rgb}{0.319809,0.770914,0.411152}%
\pgfsetfillcolor{currentfill}%
\pgfsetfillopacity{0.800000}%
\pgfsetlinewidth{0.000000pt}%
\definecolor{currentstroke}{rgb}{0.000000,0.000000,0.000000}%
\pgfsetstrokecolor{currentstroke}%
\pgfsetdash{}{0pt}%
\pgfpathmoveto{\pgfqpoint{5.977940in}{3.592904in}}%
\pgfpathlineto{\pgfqpoint{5.993041in}{3.608501in}}%
\pgfpathlineto{\pgfqpoint{6.008165in}{3.624280in}}%
\pgfpathlineto{\pgfqpoint{6.023313in}{3.640242in}}%
\pgfpathlineto{\pgfqpoint{6.030353in}{3.639498in}}%
\pgfpathlineto{\pgfqpoint{6.037386in}{3.638751in}}%
\pgfpathlineto{\pgfqpoint{6.044411in}{3.638006in}}%
\pgfpathlineto{\pgfqpoint{6.051429in}{3.637271in}}%
\pgfpathlineto{\pgfqpoint{6.036318in}{3.621923in}}%
\pgfpathlineto{\pgfqpoint{6.021230in}{3.606757in}}%
\pgfpathlineto{\pgfqpoint{6.006166in}{3.591773in}}%
\pgfpathlineto{\pgfqpoint{5.999120in}{3.592040in}}%
\pgfpathlineto{\pgfqpoint{5.992067in}{3.592322in}}%
\pgfpathlineto{\pgfqpoint{5.985008in}{3.592612in}}%
\pgfpathlineto{\pgfqpoint{5.977940in}{3.592904in}}%
\pgfpathclose%
\pgfusepath{fill}%
\end{pgfscope}%
\begin{pgfscope}%
\pgfpathrectangle{\pgfqpoint{1.150000in}{0.150000in}}{\pgfqpoint{5.700000in}{5.700000in}}%
\pgfusepath{clip}%
\pgfsetbuttcap%
\pgfsetroundjoin%
\definecolor{currentfill}{rgb}{0.269944,0.014625,0.341379}%
\pgfsetfillcolor{currentfill}%
\pgfsetfillopacity{0.800000}%
\pgfsetlinewidth{0.000000pt}%
\definecolor{currentstroke}{rgb}{0.000000,0.000000,0.000000}%
\pgfsetstrokecolor{currentstroke}%
\pgfsetdash{}{0pt}%
\pgfpathmoveto{\pgfqpoint{3.515887in}{1.516459in}}%
\pgfpathlineto{\pgfqpoint{3.529711in}{1.514322in}}%
\pgfpathlineto{\pgfqpoint{3.543540in}{1.512377in}}%
\pgfpathlineto{\pgfqpoint{3.557376in}{1.510625in}}%
\pgfpathlineto{\pgfqpoint{3.571217in}{1.509064in}}%
\pgfpathlineto{\pgfqpoint{3.579463in}{1.519283in}}%
\pgfpathlineto{\pgfqpoint{3.587702in}{1.529601in}}%
\pgfpathlineto{\pgfqpoint{3.595935in}{1.540014in}}%
\pgfpathlineto{\pgfqpoint{3.604162in}{1.550516in}}%
\pgfpathlineto{\pgfqpoint{3.590335in}{1.551578in}}%
\pgfpathlineto{\pgfqpoint{3.576514in}{1.552831in}}%
\pgfpathlineto{\pgfqpoint{3.562699in}{1.554276in}}%
\pgfpathlineto{\pgfqpoint{3.548890in}{1.555914in}}%
\pgfpathlineto{\pgfqpoint{3.540650in}{1.545899in}}%
\pgfpathlineto{\pgfqpoint{3.532402in}{1.535982in}}%
\pgfpathlineto{\pgfqpoint{3.524148in}{1.526167in}}%
\pgfpathlineto{\pgfqpoint{3.515887in}{1.516459in}}%
\pgfpathclose%
\pgfusepath{fill}%
\end{pgfscope}%
\begin{pgfscope}%
\pgfpathrectangle{\pgfqpoint{1.150000in}{0.150000in}}{\pgfqpoint{5.700000in}{5.700000in}}%
\pgfusepath{clip}%
\pgfsetbuttcap%
\pgfsetroundjoin%
\definecolor{currentfill}{rgb}{0.281477,0.755203,0.432552}%
\pgfsetfillcolor{currentfill}%
\pgfsetfillopacity{0.800000}%
\pgfsetlinewidth{0.000000pt}%
\definecolor{currentstroke}{rgb}{0.000000,0.000000,0.000000}%
\pgfsetstrokecolor{currentstroke}%
\pgfsetdash{}{0pt}%
\pgfpathmoveto{\pgfqpoint{5.889293in}{3.531088in}}%
\pgfpathlineto{\pgfqpoint{5.904333in}{3.546533in}}%
\pgfpathlineto{\pgfqpoint{5.919397in}{3.562160in}}%
\pgfpathlineto{\pgfqpoint{5.934484in}{3.577971in}}%
\pgfpathlineto{\pgfqpoint{5.949594in}{3.593965in}}%
\pgfpathlineto{\pgfqpoint{5.956693in}{3.593729in}}%
\pgfpathlineto{\pgfqpoint{5.963783in}{3.593469in}}%
\pgfpathlineto{\pgfqpoint{5.970866in}{3.593192in}}%
\pgfpathlineto{\pgfqpoint{5.977940in}{3.592904in}}%
\pgfpathlineto{\pgfqpoint{5.962864in}{3.577490in}}%
\pgfpathlineto{\pgfqpoint{5.947810in}{3.562258in}}%
\pgfpathlineto{\pgfqpoint{5.932780in}{3.547208in}}%
\pgfpathlineto{\pgfqpoint{5.917773in}{3.532339in}}%
\pgfpathlineto{\pgfqpoint{5.910664in}{3.532037in}}%
\pgfpathlineto{\pgfqpoint{5.903548in}{3.531732in}}%
\pgfpathlineto{\pgfqpoint{5.896424in}{3.531418in}}%
\pgfpathlineto{\pgfqpoint{5.889293in}{3.531088in}}%
\pgfpathclose%
\pgfusepath{fill}%
\end{pgfscope}%
\begin{pgfscope}%
\pgfpathrectangle{\pgfqpoint{1.150000in}{0.150000in}}{\pgfqpoint{5.700000in}{5.700000in}}%
\pgfusepath{clip}%
\pgfsetbuttcap%
\pgfsetroundjoin%
\definecolor{currentfill}{rgb}{0.128087,0.647749,0.523491}%
\pgfsetfillcolor{currentfill}%
\pgfsetfillopacity{0.800000}%
\pgfsetlinewidth{0.000000pt}%
\definecolor{currentstroke}{rgb}{0.000000,0.000000,0.000000}%
\pgfsetstrokecolor{currentstroke}%
\pgfsetdash{}{0pt}%
\pgfpathmoveto{\pgfqpoint{5.416257in}{3.178736in}}%
\pgfpathlineto{\pgfqpoint{5.430999in}{3.193234in}}%
\pgfpathlineto{\pgfqpoint{5.445763in}{3.207915in}}%
\pgfpathlineto{\pgfqpoint{5.460549in}{3.222781in}}%
\pgfpathlineto{\pgfqpoint{5.475355in}{3.237832in}}%
\pgfpathlineto{\pgfqpoint{5.482802in}{3.241612in}}%
\pgfpathlineto{\pgfqpoint{5.490239in}{3.245286in}}%
\pgfpathlineto{\pgfqpoint{5.497667in}{3.248860in}}%
\pgfpathlineto{\pgfqpoint{5.505086in}{3.252336in}}%
\pgfpathlineto{\pgfqpoint{5.490298in}{3.237651in}}%
\pgfpathlineto{\pgfqpoint{5.475531in}{3.223150in}}%
\pgfpathlineto{\pgfqpoint{5.460786in}{3.208832in}}%
\pgfpathlineto{\pgfqpoint{5.446061in}{3.194698in}}%
\pgfpathlineto{\pgfqpoint{5.438624in}{3.190845in}}%
\pgfpathlineto{\pgfqpoint{5.431177in}{3.186903in}}%
\pgfpathlineto{\pgfqpoint{5.423721in}{3.182868in}}%
\pgfpathlineto{\pgfqpoint{5.416257in}{3.178736in}}%
\pgfpathclose%
\pgfusepath{fill}%
\end{pgfscope}%
\begin{pgfscope}%
\pgfpathrectangle{\pgfqpoint{1.150000in}{0.150000in}}{\pgfqpoint{5.700000in}{5.700000in}}%
\pgfusepath{clip}%
\pgfsetbuttcap%
\pgfsetroundjoin%
\definecolor{currentfill}{rgb}{0.269944,0.014625,0.341379}%
\pgfsetfillcolor{currentfill}%
\pgfsetfillopacity{0.800000}%
\pgfsetlinewidth{0.000000pt}%
\definecolor{currentstroke}{rgb}{0.000000,0.000000,0.000000}%
\pgfsetstrokecolor{currentstroke}%
\pgfsetdash{}{0pt}%
\pgfpathmoveto{\pgfqpoint{3.139204in}{1.528202in}}%
\pgfpathlineto{\pgfqpoint{3.153034in}{1.520376in}}%
\pgfpathlineto{\pgfqpoint{3.166865in}{1.512757in}}%
\pgfpathlineto{\pgfqpoint{3.180697in}{1.505346in}}%
\pgfpathlineto{\pgfqpoint{3.194530in}{1.498140in}}%
\pgfpathlineto{\pgfqpoint{3.202984in}{1.503726in}}%
\pgfpathlineto{\pgfqpoint{3.211427in}{1.509516in}}%
\pgfpathlineto{\pgfqpoint{3.219859in}{1.515504in}}%
\pgfpathlineto{\pgfqpoint{3.228281in}{1.521684in}}%
\pgfpathlineto{\pgfqpoint{3.214476in}{1.528293in}}%
\pgfpathlineto{\pgfqpoint{3.200671in}{1.535108in}}%
\pgfpathlineto{\pgfqpoint{3.186869in}{1.542129in}}%
\pgfpathlineto{\pgfqpoint{3.173067in}{1.549358in}}%
\pgfpathlineto{\pgfqpoint{3.164618in}{1.543762in}}%
\pgfpathlineto{\pgfqpoint{3.156158in}{1.538367in}}%
\pgfpathlineto{\pgfqpoint{3.147686in}{1.533178in}}%
\pgfpathlineto{\pgfqpoint{3.139204in}{1.528202in}}%
\pgfpathclose%
\pgfusepath{fill}%
\end{pgfscope}%
\begin{pgfscope}%
\pgfpathrectangle{\pgfqpoint{1.150000in}{0.150000in}}{\pgfqpoint{5.700000in}{5.700000in}}%
\pgfusepath{clip}%
\pgfsetbuttcap%
\pgfsetroundjoin%
\definecolor{currentfill}{rgb}{0.127568,0.566949,0.550556}%
\pgfsetfillcolor{currentfill}%
\pgfsetfillopacity{0.800000}%
\pgfsetlinewidth{0.000000pt}%
\definecolor{currentstroke}{rgb}{0.000000,0.000000,0.000000}%
\pgfsetstrokecolor{currentstroke}%
\pgfsetdash{}{0pt}%
\pgfpathmoveto{\pgfqpoint{5.119588in}{2.923549in}}%
\pgfpathlineto{\pgfqpoint{5.134142in}{2.936937in}}%
\pgfpathlineto{\pgfqpoint{5.148714in}{2.950509in}}%
\pgfpathlineto{\pgfqpoint{5.163307in}{2.964266in}}%
\pgfpathlineto{\pgfqpoint{5.177919in}{2.978208in}}%
\pgfpathlineto{\pgfqpoint{5.185556in}{2.984805in}}%
\pgfpathlineto{\pgfqpoint{5.193183in}{2.991268in}}%
\pgfpathlineto{\pgfqpoint{5.200802in}{2.997600in}}%
\pgfpathlineto{\pgfqpoint{5.208412in}{3.003804in}}%
\pgfpathlineto{\pgfqpoint{5.193811in}{2.990085in}}%
\pgfpathlineto{\pgfqpoint{5.179230in}{2.976551in}}%
\pgfpathlineto{\pgfqpoint{5.164668in}{2.963201in}}%
\pgfpathlineto{\pgfqpoint{5.150125in}{2.950035in}}%
\pgfpathlineto{\pgfqpoint{5.142504in}{2.943596in}}%
\pgfpathlineto{\pgfqpoint{5.134874in}{2.937037in}}%
\pgfpathlineto{\pgfqpoint{5.127235in}{2.930356in}}%
\pgfpathlineto{\pgfqpoint{5.119588in}{2.923549in}}%
\pgfpathclose%
\pgfusepath{fill}%
\end{pgfscope}%
\begin{pgfscope}%
\pgfpathrectangle{\pgfqpoint{1.150000in}{0.150000in}}{\pgfqpoint{5.700000in}{5.700000in}}%
\pgfusepath{clip}%
\pgfsetbuttcap%
\pgfsetroundjoin%
\definecolor{currentfill}{rgb}{0.210503,0.363727,0.552206}%
\pgfsetfillcolor{currentfill}%
\pgfsetfillopacity{0.800000}%
\pgfsetlinewidth{0.000000pt}%
\definecolor{currentstroke}{rgb}{0.000000,0.000000,0.000000}%
\pgfsetstrokecolor{currentstroke}%
\pgfsetdash{}{0pt}%
\pgfpathmoveto{\pgfqpoint{4.494071in}{2.303106in}}%
\pgfpathlineto{\pgfqpoint{4.508246in}{2.312503in}}%
\pgfpathlineto{\pgfqpoint{4.522437in}{2.322085in}}%
\pgfpathlineto{\pgfqpoint{4.536644in}{2.331851in}}%
\pgfpathlineto{\pgfqpoint{4.550865in}{2.341802in}}%
\pgfpathlineto{\pgfqpoint{4.558797in}{2.353663in}}%
\pgfpathlineto{\pgfqpoint{4.566723in}{2.365404in}}%
\pgfpathlineto{\pgfqpoint{4.574643in}{2.377024in}}%
\pgfpathlineto{\pgfqpoint{4.582557in}{2.388522in}}%
\pgfpathlineto{\pgfqpoint{4.568336in}{2.378484in}}%
\pgfpathlineto{\pgfqpoint{4.554131in}{2.368631in}}%
\pgfpathlineto{\pgfqpoint{4.539942in}{2.358962in}}%
\pgfpathlineto{\pgfqpoint{4.525768in}{2.349478in}}%
\pgfpathlineto{\pgfqpoint{4.517852in}{2.338055in}}%
\pgfpathlineto{\pgfqpoint{4.509931in}{2.326518in}}%
\pgfpathlineto{\pgfqpoint{4.502004in}{2.314868in}}%
\pgfpathlineto{\pgfqpoint{4.494071in}{2.303106in}}%
\pgfpathclose%
\pgfusepath{fill}%
\end{pgfscope}%
\begin{pgfscope}%
\pgfpathrectangle{\pgfqpoint{1.150000in}{0.150000in}}{\pgfqpoint{5.700000in}{5.700000in}}%
\pgfusepath{clip}%
\pgfsetbuttcap%
\pgfsetroundjoin%
\definecolor{currentfill}{rgb}{0.283197,0.115680,0.436115}%
\pgfsetfillcolor{currentfill}%
\pgfsetfillopacity{0.800000}%
\pgfsetlinewidth{0.000000pt}%
\definecolor{currentstroke}{rgb}{0.000000,0.000000,0.000000}%
\pgfsetstrokecolor{currentstroke}%
\pgfsetdash{}{0pt}%
\pgfpathmoveto{\pgfqpoint{3.868530in}{1.699739in}}%
\pgfpathlineto{\pgfqpoint{3.882430in}{1.702446in}}%
\pgfpathlineto{\pgfqpoint{3.896339in}{1.705340in}}%
\pgfpathlineto{\pgfqpoint{3.910258in}{1.708420in}}%
\pgfpathlineto{\pgfqpoint{3.924187in}{1.711685in}}%
\pgfpathlineto{\pgfqpoint{3.932307in}{1.724509in}}%
\pgfpathlineto{\pgfqpoint{3.940422in}{1.737334in}}%
\pgfpathlineto{\pgfqpoint{3.948532in}{1.750156in}}%
\pgfpathlineto{\pgfqpoint{3.956638in}{1.762972in}}%
\pgfpathlineto{\pgfqpoint{3.942714in}{1.759330in}}%
\pgfpathlineto{\pgfqpoint{3.928801in}{1.755874in}}%
\pgfpathlineto{\pgfqpoint{3.914897in}{1.752605in}}%
\pgfpathlineto{\pgfqpoint{3.901003in}{1.749522in}}%
\pgfpathlineto{\pgfqpoint{3.892892in}{1.737070in}}%
\pgfpathlineto{\pgfqpoint{3.884776in}{1.724620in}}%
\pgfpathlineto{\pgfqpoint{3.876655in}{1.712175in}}%
\pgfpathlineto{\pgfqpoint{3.868530in}{1.699739in}}%
\pgfpathclose%
\pgfusepath{fill}%
\end{pgfscope}%
\begin{pgfscope}%
\pgfpathrectangle{\pgfqpoint{1.150000in}{0.150000in}}{\pgfqpoint{5.700000in}{5.700000in}}%
\pgfusepath{clip}%
\pgfsetbuttcap%
\pgfsetroundjoin%
\definecolor{currentfill}{rgb}{0.278791,0.062145,0.386592}%
\pgfsetfillcolor{currentfill}%
\pgfsetfillopacity{0.800000}%
\pgfsetlinewidth{0.000000pt}%
\definecolor{currentstroke}{rgb}{0.000000,0.000000,0.000000}%
\pgfsetstrokecolor{currentstroke}%
\pgfsetdash{}{0pt}%
\pgfpathmoveto{\pgfqpoint{2.938771in}{1.628624in}}%
\pgfpathlineto{\pgfqpoint{2.952644in}{1.617589in}}%
\pgfpathlineto{\pgfqpoint{2.966515in}{1.606775in}}%
\pgfpathlineto{\pgfqpoint{2.980384in}{1.596181in}}%
\pgfpathlineto{\pgfqpoint{2.994251in}{1.585806in}}%
\pgfpathlineto{\pgfqpoint{3.002850in}{1.588582in}}%
\pgfpathlineto{\pgfqpoint{3.011435in}{1.591612in}}%
\pgfpathlineto{\pgfqpoint{3.020007in}{1.594890in}}%
\pgfpathlineto{\pgfqpoint{3.028565in}{1.598409in}}%
\pgfpathlineto{\pgfqpoint{3.014733in}{1.608150in}}%
\pgfpathlineto{\pgfqpoint{3.000900in}{1.618110in}}%
\pgfpathlineto{\pgfqpoint{2.987065in}{1.628288in}}%
\pgfpathlineto{\pgfqpoint{2.973229in}{1.638687in}}%
\pgfpathlineto{\pgfqpoint{2.964636in}{1.635790in}}%
\pgfpathlineto{\pgfqpoint{2.956028in}{1.633143in}}%
\pgfpathlineto{\pgfqpoint{2.947407in}{1.630752in}}%
\pgfpathlineto{\pgfqpoint{2.938771in}{1.628624in}}%
\pgfpathclose%
\pgfusepath{fill}%
\end{pgfscope}%
\begin{pgfscope}%
\pgfpathrectangle{\pgfqpoint{1.150000in}{0.150000in}}{\pgfqpoint{5.700000in}{5.700000in}}%
\pgfusepath{clip}%
\pgfsetbuttcap%
\pgfsetroundjoin%
\definecolor{currentfill}{rgb}{0.177423,0.437527,0.557565}%
\pgfsetfillcolor{currentfill}%
\pgfsetfillopacity{0.800000}%
\pgfsetlinewidth{0.000000pt}%
\definecolor{currentstroke}{rgb}{0.000000,0.000000,0.000000}%
\pgfsetstrokecolor{currentstroke}%
\pgfsetdash{}{0pt}%
\pgfpathmoveto{\pgfqpoint{4.702672in}{2.518400in}}%
\pgfpathlineto{\pgfqpoint{4.716970in}{2.529435in}}%
\pgfpathlineto{\pgfqpoint{4.731285in}{2.540655in}}%
\pgfpathlineto{\pgfqpoint{4.745617in}{2.552060in}}%
\pgfpathlineto{\pgfqpoint{4.759966in}{2.563649in}}%
\pgfpathlineto{\pgfqpoint{4.767818in}{2.574046in}}%
\pgfpathlineto{\pgfqpoint{4.775662in}{2.584306in}}%
\pgfpathlineto{\pgfqpoint{4.783500in}{2.594428in}}%
\pgfpathlineto{\pgfqpoint{4.791331in}{2.604414in}}%
\pgfpathlineto{\pgfqpoint{4.776985in}{2.592839in}}%
\pgfpathlineto{\pgfqpoint{4.762656in}{2.581449in}}%
\pgfpathlineto{\pgfqpoint{4.748345in}{2.570243in}}%
\pgfpathlineto{\pgfqpoint{4.734050in}{2.559222in}}%
\pgfpathlineto{\pgfqpoint{4.726216in}{2.549209in}}%
\pgfpathlineto{\pgfqpoint{4.718375in}{2.539068in}}%
\pgfpathlineto{\pgfqpoint{4.710527in}{2.528799in}}%
\pgfpathlineto{\pgfqpoint{4.702672in}{2.518400in}}%
\pgfpathclose%
\pgfusepath{fill}%
\end{pgfscope}%
\begin{pgfscope}%
\pgfpathrectangle{\pgfqpoint{1.150000in}{0.150000in}}{\pgfqpoint{5.700000in}{5.700000in}}%
\pgfusepath{clip}%
\pgfsetbuttcap%
\pgfsetroundjoin%
\definecolor{currentfill}{rgb}{0.203063,0.379716,0.553925}%
\pgfsetfillcolor{currentfill}%
\pgfsetfillopacity{0.800000}%
\pgfsetlinewidth{0.000000pt}%
\definecolor{currentstroke}{rgb}{0.000000,0.000000,0.000000}%
\pgfsetstrokecolor{currentstroke}%
\pgfsetdash{}{0pt}%
\pgfpathmoveto{\pgfqpoint{2.266018in}{2.448550in}}%
\pgfpathlineto{\pgfqpoint{2.280267in}{2.424936in}}%
\pgfpathlineto{\pgfqpoint{2.294502in}{2.401637in}}%
\pgfpathlineto{\pgfqpoint{2.308723in}{2.378650in}}%
\pgfpathlineto{\pgfqpoint{2.322930in}{2.355972in}}%
\pgfpathlineto{\pgfqpoint{2.332107in}{2.350759in}}%
\pgfpathlineto{\pgfqpoint{2.341260in}{2.345909in}}%
\pgfpathlineto{\pgfqpoint{2.350390in}{2.341416in}}%
\pgfpathlineto{\pgfqpoint{2.359497in}{2.337272in}}%
\pgfpathlineto{\pgfqpoint{2.345350in}{2.359270in}}%
\pgfpathlineto{\pgfqpoint{2.331190in}{2.381576in}}%
\pgfpathlineto{\pgfqpoint{2.317016in}{2.404192in}}%
\pgfpathlineto{\pgfqpoint{2.302828in}{2.427121in}}%
\pgfpathlineto{\pgfqpoint{2.293661in}{2.431933in}}%
\pgfpathlineto{\pgfqpoint{2.284471in}{2.437104in}}%
\pgfpathlineto{\pgfqpoint{2.275256in}{2.442640in}}%
\pgfpathlineto{\pgfqpoint{2.266018in}{2.448550in}}%
\pgfpathclose%
\pgfusepath{fill}%
\end{pgfscope}%
\begin{pgfscope}%
\pgfpathrectangle{\pgfqpoint{1.150000in}{0.150000in}}{\pgfqpoint{5.700000in}{5.700000in}}%
\pgfusepath{clip}%
\pgfsetbuttcap%
\pgfsetroundjoin%
\definecolor{currentfill}{rgb}{0.149039,0.508051,0.557250}%
\pgfsetfillcolor{currentfill}%
\pgfsetfillopacity{0.800000}%
\pgfsetlinewidth{0.000000pt}%
\definecolor{currentstroke}{rgb}{0.000000,0.000000,0.000000}%
\pgfsetstrokecolor{currentstroke}%
\pgfsetdash{}{0pt}%
\pgfpathmoveto{\pgfqpoint{4.911241in}{2.727121in}}%
\pgfpathlineto{\pgfqpoint{4.925667in}{2.739489in}}%
\pgfpathlineto{\pgfqpoint{4.940110in}{2.752041in}}%
\pgfpathlineto{\pgfqpoint{4.954573in}{2.764778in}}%
\pgfpathlineto{\pgfqpoint{4.969054in}{2.777700in}}%
\pgfpathlineto{\pgfqpoint{4.976808in}{2.786294in}}%
\pgfpathlineto{\pgfqpoint{4.984553in}{2.794745in}}%
\pgfpathlineto{\pgfqpoint{4.992291in}{2.803055in}}%
\pgfpathlineto{\pgfqpoint{5.000021in}{2.811227in}}%
\pgfpathlineto{\pgfqpoint{4.985546in}{2.798423in}}%
\pgfpathlineto{\pgfqpoint{4.971091in}{2.785804in}}%
\pgfpathlineto{\pgfqpoint{4.956653in}{2.773369in}}%
\pgfpathlineto{\pgfqpoint{4.942234in}{2.761118in}}%
\pgfpathlineto{\pgfqpoint{4.934497in}{2.752816in}}%
\pgfpathlineto{\pgfqpoint{4.926753in}{2.744384in}}%
\pgfpathlineto{\pgfqpoint{4.919001in}{2.735819in}}%
\pgfpathlineto{\pgfqpoint{4.911241in}{2.727121in}}%
\pgfpathclose%
\pgfusepath{fill}%
\end{pgfscope}%
\begin{pgfscope}%
\pgfpathrectangle{\pgfqpoint{1.150000in}{0.150000in}}{\pgfqpoint{5.700000in}{5.700000in}}%
\pgfusepath{clip}%
\pgfsetbuttcap%
\pgfsetroundjoin%
\definecolor{currentfill}{rgb}{0.267004,0.004874,0.329415}%
\pgfsetfillcolor{currentfill}%
\pgfsetfillopacity{0.800000}%
\pgfsetlinewidth{0.000000pt}%
\definecolor{currentstroke}{rgb}{0.000000,0.000000,0.000000}%
\pgfsetstrokecolor{currentstroke}%
\pgfsetdash{}{0pt}%
\pgfpathmoveto{\pgfqpoint{3.427453in}{1.491410in}}%
\pgfpathlineto{\pgfqpoint{3.441275in}{1.487965in}}%
\pgfpathlineto{\pgfqpoint{3.455102in}{1.484715in}}%
\pgfpathlineto{\pgfqpoint{3.468933in}{1.481660in}}%
\pgfpathlineto{\pgfqpoint{3.482769in}{1.478798in}}%
\pgfpathlineto{\pgfqpoint{3.491059in}{1.488028in}}%
\pgfpathlineto{\pgfqpoint{3.499342in}{1.497385in}}%
\pgfpathlineto{\pgfqpoint{3.507618in}{1.506864in}}%
\pgfpathlineto{\pgfqpoint{3.515887in}{1.516459in}}%
\pgfpathlineto{\pgfqpoint{3.502068in}{1.518790in}}%
\pgfpathlineto{\pgfqpoint{3.488254in}{1.521315in}}%
\pgfpathlineto{\pgfqpoint{3.474445in}{1.524034in}}%
\pgfpathlineto{\pgfqpoint{3.460641in}{1.526948in}}%
\pgfpathlineto{\pgfqpoint{3.452356in}{1.517871in}}%
\pgfpathlineto{\pgfqpoint{3.444063in}{1.508919in}}%
\pgfpathlineto{\pgfqpoint{3.435762in}{1.500097in}}%
\pgfpathlineto{\pgfqpoint{3.427453in}{1.491410in}}%
\pgfpathclose%
\pgfusepath{fill}%
\end{pgfscope}%
\begin{pgfscope}%
\pgfpathrectangle{\pgfqpoint{1.150000in}{0.150000in}}{\pgfqpoint{5.700000in}{5.700000in}}%
\pgfusepath{clip}%
\pgfsetbuttcap%
\pgfsetroundjoin%
\definecolor{currentfill}{rgb}{0.265145,0.232956,0.516599}%
\pgfsetfillcolor{currentfill}%
\pgfsetfillopacity{0.800000}%
\pgfsetlinewidth{0.000000pt}%
\definecolor{currentstroke}{rgb}{0.000000,0.000000,0.000000}%
\pgfsetstrokecolor{currentstroke}%
\pgfsetdash{}{0pt}%
\pgfpathmoveto{\pgfqpoint{4.165238in}{1.958522in}}%
\pgfpathlineto{\pgfqpoint{4.179254in}{1.964774in}}%
\pgfpathlineto{\pgfqpoint{4.193282in}{1.971210in}}%
\pgfpathlineto{\pgfqpoint{4.207323in}{1.977832in}}%
\pgfpathlineto{\pgfqpoint{4.221376in}{1.984638in}}%
\pgfpathlineto{\pgfqpoint{4.229414in}{1.997892in}}%
\pgfpathlineto{\pgfqpoint{4.237447in}{2.011077in}}%
\pgfpathlineto{\pgfqpoint{4.245475in}{2.024191in}}%
\pgfpathlineto{\pgfqpoint{4.253499in}{2.037232in}}%
\pgfpathlineto{\pgfqpoint{4.239446in}{2.030175in}}%
\pgfpathlineto{\pgfqpoint{4.225407in}{2.023303in}}%
\pgfpathlineto{\pgfqpoint{4.211380in}{2.016616in}}%
\pgfpathlineto{\pgfqpoint{4.197366in}{2.010114in}}%
\pgfpathlineto{\pgfqpoint{4.189341in}{1.997311in}}%
\pgfpathlineto{\pgfqpoint{4.181312in}{1.984443in}}%
\pgfpathlineto{\pgfqpoint{4.173277in}{1.971513in}}%
\pgfpathlineto{\pgfqpoint{4.165238in}{1.958522in}}%
\pgfpathclose%
\pgfusepath{fill}%
\end{pgfscope}%
\begin{pgfscope}%
\pgfpathrectangle{\pgfqpoint{1.150000in}{0.150000in}}{\pgfqpoint{5.700000in}{5.700000in}}%
\pgfusepath{clip}%
\pgfsetbuttcap%
\pgfsetroundjoin%
\definecolor{currentfill}{rgb}{0.149039,0.508051,0.557250}%
\pgfsetfillcolor{currentfill}%
\pgfsetfillopacity{0.800000}%
\pgfsetlinewidth{0.000000pt}%
\definecolor{currentstroke}{rgb}{0.000000,0.000000,0.000000}%
\pgfsetstrokecolor{currentstroke}%
\pgfsetdash{}{0pt}%
\pgfpathmoveto{\pgfqpoint{2.073655in}{2.839113in}}%
\pgfpathlineto{\pgfqpoint{2.088113in}{2.810740in}}%
\pgfpathlineto{\pgfqpoint{2.102552in}{2.782733in}}%
\pgfpathlineto{\pgfqpoint{2.116970in}{2.755090in}}%
\pgfpathlineto{\pgfqpoint{2.131370in}{2.727805in}}%
\pgfpathlineto{\pgfqpoint{2.140705in}{2.721332in}}%
\pgfpathlineto{\pgfqpoint{2.150014in}{2.715235in}}%
\pgfpathlineto{\pgfqpoint{2.159298in}{2.709506in}}%
\pgfpathlineto{\pgfqpoint{2.168558in}{2.704140in}}%
\pgfpathlineto{\pgfqpoint{2.154224in}{2.730757in}}%
\pgfpathlineto{\pgfqpoint{2.139872in}{2.757731in}}%
\pgfpathlineto{\pgfqpoint{2.125501in}{2.785065in}}%
\pgfpathlineto{\pgfqpoint{2.111111in}{2.812764in}}%
\pgfpathlineto{\pgfqpoint{2.101786in}{2.818786in}}%
\pgfpathlineto{\pgfqpoint{2.092435in}{2.825181in}}%
\pgfpathlineto{\pgfqpoint{2.083059in}{2.831954in}}%
\pgfpathlineto{\pgfqpoint{2.073655in}{2.839113in}}%
\pgfpathclose%
\pgfusepath{fill}%
\end{pgfscope}%
\begin{pgfscope}%
\pgfpathrectangle{\pgfqpoint{1.150000in}{0.150000in}}{\pgfqpoint{5.700000in}{5.700000in}}%
\pgfusepath{clip}%
\pgfsetbuttcap%
\pgfsetroundjoin%
\definecolor{currentfill}{rgb}{0.281887,0.150881,0.465405}%
\pgfsetfillcolor{currentfill}%
\pgfsetfillopacity{0.800000}%
\pgfsetlinewidth{0.000000pt}%
\definecolor{currentstroke}{rgb}{0.000000,0.000000,0.000000}%
\pgfsetstrokecolor{currentstroke}%
\pgfsetdash{}{0pt}%
\pgfpathmoveto{\pgfqpoint{3.956638in}{1.762972in}}%
\pgfpathlineto{\pgfqpoint{3.970572in}{1.766800in}}%
\pgfpathlineto{\pgfqpoint{3.984516in}{1.770814in}}%
\pgfpathlineto{\pgfqpoint{3.998471in}{1.775014in}}%
\pgfpathlineto{\pgfqpoint{4.012437in}{1.779398in}}%
\pgfpathlineto{\pgfqpoint{4.020534in}{1.792562in}}%
\pgfpathlineto{\pgfqpoint{4.028626in}{1.805706in}}%
\pgfpathlineto{\pgfqpoint{4.036714in}{1.818826in}}%
\pgfpathlineto{\pgfqpoint{4.044797in}{1.831919in}}%
\pgfpathlineto{\pgfqpoint{4.030835in}{1.827189in}}%
\pgfpathlineto{\pgfqpoint{4.016884in}{1.822644in}}%
\pgfpathlineto{\pgfqpoint{4.002943in}{1.818285in}}%
\pgfpathlineto{\pgfqpoint{3.989014in}{1.814113in}}%
\pgfpathlineto{\pgfqpoint{3.980927in}{1.801352in}}%
\pgfpathlineto{\pgfqpoint{3.972835in}{1.788574in}}%
\pgfpathlineto{\pgfqpoint{3.964739in}{1.775779in}}%
\pgfpathlineto{\pgfqpoint{3.956638in}{1.762972in}}%
\pgfpathclose%
\pgfusepath{fill}%
\end{pgfscope}%
\begin{pgfscope}%
\pgfpathrectangle{\pgfqpoint{1.150000in}{0.150000in}}{\pgfqpoint{5.700000in}{5.700000in}}%
\pgfusepath{clip}%
\pgfsetbuttcap%
\pgfsetroundjoin%
\definecolor{currentfill}{rgb}{0.150148,0.676631,0.506589}%
\pgfsetfillcolor{currentfill}%
\pgfsetfillopacity{0.800000}%
\pgfsetlinewidth{0.000000pt}%
\definecolor{currentstroke}{rgb}{0.000000,0.000000,0.000000}%
\pgfsetstrokecolor{currentstroke}%
\pgfsetdash{}{0pt}%
\pgfpathmoveto{\pgfqpoint{5.505086in}{3.252336in}}%
\pgfpathlineto{\pgfqpoint{5.519896in}{3.267205in}}%
\pgfpathlineto{\pgfqpoint{5.534727in}{3.282258in}}%
\pgfpathlineto{\pgfqpoint{5.549580in}{3.297496in}}%
\pgfpathlineto{\pgfqpoint{5.564455in}{3.312919in}}%
\pgfpathlineto{\pgfqpoint{5.571844in}{3.315914in}}%
\pgfpathlineto{\pgfqpoint{5.579224in}{3.318812in}}%
\pgfpathlineto{\pgfqpoint{5.586594in}{3.321618in}}%
\pgfpathlineto{\pgfqpoint{5.593955in}{3.324337in}}%
\pgfpathlineto{\pgfqpoint{5.579101in}{3.309316in}}%
\pgfpathlineto{\pgfqpoint{5.564269in}{3.294480in}}%
\pgfpathlineto{\pgfqpoint{5.549459in}{3.279827in}}%
\pgfpathlineto{\pgfqpoint{5.534670in}{3.265357in}}%
\pgfpathlineto{\pgfqpoint{5.527287in}{3.262226in}}%
\pgfpathlineto{\pgfqpoint{5.519896in}{3.259015in}}%
\pgfpathlineto{\pgfqpoint{5.512495in}{3.255720in}}%
\pgfpathlineto{\pgfqpoint{5.505086in}{3.252336in}}%
\pgfpathclose%
\pgfusepath{fill}%
\end{pgfscope}%
\begin{pgfscope}%
\pgfpathrectangle{\pgfqpoint{1.150000in}{0.150000in}}{\pgfqpoint{5.700000in}{5.700000in}}%
\pgfusepath{clip}%
\pgfsetbuttcap%
\pgfsetroundjoin%
\definecolor{currentfill}{rgb}{0.231674,0.318106,0.544834}%
\pgfsetfillcolor{currentfill}%
\pgfsetfillopacity{0.800000}%
\pgfsetlinewidth{0.000000pt}%
\definecolor{currentstroke}{rgb}{0.000000,0.000000,0.000000}%
\pgfsetstrokecolor{currentstroke}%
\pgfsetdash{}{0pt}%
\pgfpathmoveto{\pgfqpoint{4.373845in}{2.170398in}}%
\pgfpathlineto{\pgfqpoint{4.387964in}{2.178783in}}%
\pgfpathlineto{\pgfqpoint{4.402097in}{2.187352in}}%
\pgfpathlineto{\pgfqpoint{4.416245in}{2.196106in}}%
\pgfpathlineto{\pgfqpoint{4.430407in}{2.205044in}}%
\pgfpathlineto{\pgfqpoint{4.438384in}{2.217679in}}%
\pgfpathlineto{\pgfqpoint{4.446356in}{2.230208in}}%
\pgfpathlineto{\pgfqpoint{4.454322in}{2.242631in}}%
\pgfpathlineto{\pgfqpoint{4.462283in}{2.254945in}}%
\pgfpathlineto{\pgfqpoint{4.448121in}{2.245853in}}%
\pgfpathlineto{\pgfqpoint{4.433974in}{2.236946in}}%
\pgfpathlineto{\pgfqpoint{4.419842in}{2.228223in}}%
\pgfpathlineto{\pgfqpoint{4.405724in}{2.219686in}}%
\pgfpathlineto{\pgfqpoint{4.397763in}{2.207512in}}%
\pgfpathlineto{\pgfqpoint{4.389795in}{2.195239in}}%
\pgfpathlineto{\pgfqpoint{4.381823in}{2.182867in}}%
\pgfpathlineto{\pgfqpoint{4.373845in}{2.170398in}}%
\pgfpathclose%
\pgfusepath{fill}%
\end{pgfscope}%
\begin{pgfscope}%
\pgfpathrectangle{\pgfqpoint{1.150000in}{0.150000in}}{\pgfqpoint{5.700000in}{5.700000in}}%
\pgfusepath{clip}%
\pgfsetbuttcap%
\pgfsetroundjoin%
\definecolor{currentfill}{rgb}{0.276022,0.044167,0.370164}%
\pgfsetfillcolor{currentfill}%
\pgfsetfillopacity{0.800000}%
\pgfsetlinewidth{0.000000pt}%
\definecolor{currentstroke}{rgb}{0.000000,0.000000,0.000000}%
\pgfsetstrokecolor{currentstroke}%
\pgfsetdash{}{0pt}%
\pgfpathmoveto{\pgfqpoint{2.994251in}{1.585806in}}%
\pgfpathlineto{\pgfqpoint{3.008117in}{1.575648in}}%
\pgfpathlineto{\pgfqpoint{3.021982in}{1.565707in}}%
\pgfpathlineto{\pgfqpoint{3.035845in}{1.555981in}}%
\pgfpathlineto{\pgfqpoint{3.049708in}{1.546469in}}%
\pgfpathlineto{\pgfqpoint{3.058272in}{1.549891in}}%
\pgfpathlineto{\pgfqpoint{3.066823in}{1.553558in}}%
\pgfpathlineto{\pgfqpoint{3.075361in}{1.557464in}}%
\pgfpathlineto{\pgfqpoint{3.083886in}{1.561604in}}%
\pgfpathlineto{\pgfqpoint{3.070057in}{1.570483in}}%
\pgfpathlineto{\pgfqpoint{3.056227in}{1.579577in}}%
\pgfpathlineto{\pgfqpoint{3.042397in}{1.588885in}}%
\pgfpathlineto{\pgfqpoint{3.028565in}{1.598409in}}%
\pgfpathlineto{\pgfqpoint{3.020007in}{1.594890in}}%
\pgfpathlineto{\pgfqpoint{3.011435in}{1.591612in}}%
\pgfpathlineto{\pgfqpoint{3.002850in}{1.588582in}}%
\pgfpathlineto{\pgfqpoint{2.994251in}{1.585806in}}%
\pgfpathclose%
\pgfusepath{fill}%
\end{pgfscope}%
\begin{pgfscope}%
\pgfpathrectangle{\pgfqpoint{1.150000in}{0.150000in}}{\pgfqpoint{5.700000in}{5.700000in}}%
\pgfusepath{clip}%
\pgfsetbuttcap%
\pgfsetroundjoin%
\definecolor{currentfill}{rgb}{0.120565,0.596422,0.543611}%
\pgfsetfillcolor{currentfill}%
\pgfsetfillopacity{0.800000}%
\pgfsetlinewidth{0.000000pt}%
\definecolor{currentstroke}{rgb}{0.000000,0.000000,0.000000}%
\pgfsetstrokecolor{currentstroke}%
\pgfsetdash{}{0pt}%
\pgfpathmoveto{\pgfqpoint{5.208412in}{3.003804in}}%
\pgfpathlineto{\pgfqpoint{5.223034in}{3.017707in}}%
\pgfpathlineto{\pgfqpoint{5.237675in}{3.031795in}}%
\pgfpathlineto{\pgfqpoint{5.252337in}{3.046068in}}%
\pgfpathlineto{\pgfqpoint{5.267020in}{3.060526in}}%
\pgfpathlineto{\pgfqpoint{5.274609in}{3.066358in}}%
\pgfpathlineto{\pgfqpoint{5.282189in}{3.072060in}}%
\pgfpathlineto{\pgfqpoint{5.289760in}{3.077633in}}%
\pgfpathlineto{\pgfqpoint{5.297322in}{3.083082in}}%
\pgfpathlineto{\pgfqpoint{5.282653in}{3.068883in}}%
\pgfpathlineto{\pgfqpoint{5.268003in}{3.054869in}}%
\pgfpathlineto{\pgfqpoint{5.253374in}{3.041040in}}%
\pgfpathlineto{\pgfqpoint{5.238766in}{3.027394in}}%
\pgfpathlineto{\pgfqpoint{5.231190in}{3.021674in}}%
\pgfpathlineto{\pgfqpoint{5.223606in}{3.015838in}}%
\pgfpathlineto{\pgfqpoint{5.216014in}{3.009882in}}%
\pgfpathlineto{\pgfqpoint{5.208412in}{3.003804in}}%
\pgfpathclose%
\pgfusepath{fill}%
\end{pgfscope}%
\begin{pgfscope}%
\pgfpathrectangle{\pgfqpoint{1.150000in}{0.150000in}}{\pgfqpoint{5.700000in}{5.700000in}}%
\pgfusepath{clip}%
\pgfsetbuttcap%
\pgfsetroundjoin%
\definecolor{currentfill}{rgb}{0.268510,0.009605,0.335427}%
\pgfsetfillcolor{currentfill}%
\pgfsetfillopacity{0.800000}%
\pgfsetlinewidth{0.000000pt}%
\definecolor{currentstroke}{rgb}{0.000000,0.000000,0.000000}%
\pgfsetstrokecolor{currentstroke}%
\pgfsetdash{}{0pt}%
\pgfpathmoveto{\pgfqpoint{3.194530in}{1.498140in}}%
\pgfpathlineto{\pgfqpoint{3.208364in}{1.491139in}}%
\pgfpathlineto{\pgfqpoint{3.222200in}{1.484343in}}%
\pgfpathlineto{\pgfqpoint{3.236038in}{1.477750in}}%
\pgfpathlineto{\pgfqpoint{3.249877in}{1.471359in}}%
\pgfpathlineto{\pgfqpoint{3.258304in}{1.477553in}}%
\pgfpathlineto{\pgfqpoint{3.266720in}{1.483944in}}%
\pgfpathlineto{\pgfqpoint{3.275127in}{1.490524in}}%
\pgfpathlineto{\pgfqpoint{3.283523in}{1.497288in}}%
\pgfpathlineto{\pgfqpoint{3.269709in}{1.503083in}}%
\pgfpathlineto{\pgfqpoint{3.255898in}{1.509080in}}%
\pgfpathlineto{\pgfqpoint{3.242089in}{1.515280in}}%
\pgfpathlineto{\pgfqpoint{3.228281in}{1.521684in}}%
\pgfpathlineto{\pgfqpoint{3.219859in}{1.515504in}}%
\pgfpathlineto{\pgfqpoint{3.211427in}{1.509516in}}%
\pgfpathlineto{\pgfqpoint{3.202984in}{1.503726in}}%
\pgfpathlineto{\pgfqpoint{3.194530in}{1.498140in}}%
\pgfpathclose%
\pgfusepath{fill}%
\end{pgfscope}%
\begin{pgfscope}%
\pgfpathrectangle{\pgfqpoint{1.150000in}{0.150000in}}{\pgfqpoint{5.700000in}{5.700000in}}%
\pgfusepath{clip}%
\pgfsetbuttcap%
\pgfsetroundjoin%
\definecolor{currentfill}{rgb}{0.194100,0.399323,0.555565}%
\pgfsetfillcolor{currentfill}%
\pgfsetfillopacity{0.800000}%
\pgfsetlinewidth{0.000000pt}%
\definecolor{currentstroke}{rgb}{0.000000,0.000000,0.000000}%
\pgfsetstrokecolor{currentstroke}%
\pgfsetdash{}{0pt}%
\pgfpathmoveto{\pgfqpoint{4.582557in}{2.388522in}}%
\pgfpathlineto{\pgfqpoint{4.596794in}{2.398745in}}%
\pgfpathlineto{\pgfqpoint{4.611046in}{2.409152in}}%
\pgfpathlineto{\pgfqpoint{4.625315in}{2.419744in}}%
\pgfpathlineto{\pgfqpoint{4.639600in}{2.430521in}}%
\pgfpathlineto{\pgfqpoint{4.647507in}{2.441963in}}%
\pgfpathlineto{\pgfqpoint{4.655407in}{2.453275in}}%
\pgfpathlineto{\pgfqpoint{4.663301in}{2.464456in}}%
\pgfpathlineto{\pgfqpoint{4.671188in}{2.475506in}}%
\pgfpathlineto{\pgfqpoint{4.656904in}{2.464675in}}%
\pgfpathlineto{\pgfqpoint{4.642637in}{2.454030in}}%
\pgfpathlineto{\pgfqpoint{4.628386in}{2.443568in}}%
\pgfpathlineto{\pgfqpoint{4.614151in}{2.433292in}}%
\pgfpathlineto{\pgfqpoint{4.606262in}{2.422283in}}%
\pgfpathlineto{\pgfqpoint{4.598366in}{2.411152in}}%
\pgfpathlineto{\pgfqpoint{4.590465in}{2.399898in}}%
\pgfpathlineto{\pgfqpoint{4.582557in}{2.388522in}}%
\pgfpathclose%
\pgfusepath{fill}%
\end{pgfscope}%
\begin{pgfscope}%
\pgfpathrectangle{\pgfqpoint{1.150000in}{0.150000in}}{\pgfqpoint{5.700000in}{5.700000in}}%
\pgfusepath{clip}%
\pgfsetbuttcap%
\pgfsetroundjoin%
\definecolor{currentfill}{rgb}{0.185556,0.418570,0.556753}%
\pgfsetfillcolor{currentfill}%
\pgfsetfillopacity{0.800000}%
\pgfsetlinewidth{0.000000pt}%
\definecolor{currentstroke}{rgb}{0.000000,0.000000,0.000000}%
\pgfsetstrokecolor{currentstroke}%
\pgfsetdash{}{0pt}%
\pgfpathmoveto{\pgfqpoint{2.208868in}{2.546212in}}%
\pgfpathlineto{\pgfqpoint{2.223179in}{2.521309in}}%
\pgfpathlineto{\pgfqpoint{2.237474in}{2.496734in}}%
\pgfpathlineto{\pgfqpoint{2.251753in}{2.472482in}}%
\pgfpathlineto{\pgfqpoint{2.266018in}{2.448550in}}%
\pgfpathlineto{\pgfqpoint{2.275256in}{2.442640in}}%
\pgfpathlineto{\pgfqpoint{2.284471in}{2.437104in}}%
\pgfpathlineto{\pgfqpoint{2.293661in}{2.431933in}}%
\pgfpathlineto{\pgfqpoint{2.302828in}{2.427121in}}%
\pgfpathlineto{\pgfqpoint{2.288627in}{2.450366in}}%
\pgfpathlineto{\pgfqpoint{2.274410in}{2.473930in}}%
\pgfpathlineto{\pgfqpoint{2.260179in}{2.497816in}}%
\pgfpathlineto{\pgfqpoint{2.245933in}{2.522027in}}%
\pgfpathlineto{\pgfqpoint{2.236703in}{2.527513in}}%
\pgfpathlineto{\pgfqpoint{2.227450in}{2.533367in}}%
\pgfpathlineto{\pgfqpoint{2.218172in}{2.539598in}}%
\pgfpathlineto{\pgfqpoint{2.208868in}{2.546212in}}%
\pgfpathclose%
\pgfusepath{fill}%
\end{pgfscope}%
\begin{pgfscope}%
\pgfpathrectangle{\pgfqpoint{1.150000in}{0.150000in}}{\pgfqpoint{5.700000in}{5.700000in}}%
\pgfusepath{clip}%
\pgfsetbuttcap%
\pgfsetroundjoin%
\definecolor{currentfill}{rgb}{0.267004,0.004874,0.329415}%
\pgfsetfillcolor{currentfill}%
\pgfsetfillopacity{0.800000}%
\pgfsetlinewidth{0.000000pt}%
\definecolor{currentstroke}{rgb}{0.000000,0.000000,0.000000}%
\pgfsetstrokecolor{currentstroke}%
\pgfsetdash{}{0pt}%
\pgfpathmoveto{\pgfqpoint{3.338804in}{1.476116in}}%
\pgfpathlineto{\pgfqpoint{3.352632in}{1.471321in}}%
\pgfpathlineto{\pgfqpoint{3.366463in}{1.466724in}}%
\pgfpathlineto{\pgfqpoint{3.380298in}{1.462324in}}%
\pgfpathlineto{\pgfqpoint{3.394136in}{1.458121in}}%
\pgfpathlineto{\pgfqpoint{3.402478in}{1.466214in}}%
\pgfpathlineto{\pgfqpoint{3.410811in}{1.474463in}}%
\pgfpathlineto{\pgfqpoint{3.419136in}{1.482864in}}%
\pgfpathlineto{\pgfqpoint{3.427453in}{1.491410in}}%
\pgfpathlineto{\pgfqpoint{3.413635in}{1.495051in}}%
\pgfpathlineto{\pgfqpoint{3.399821in}{1.498888in}}%
\pgfpathlineto{\pgfqpoint{3.386011in}{1.502923in}}%
\pgfpathlineto{\pgfqpoint{3.372205in}{1.507154in}}%
\pgfpathlineto{\pgfqpoint{3.363868in}{1.499159in}}%
\pgfpathlineto{\pgfqpoint{3.355522in}{1.491317in}}%
\pgfpathlineto{\pgfqpoint{3.347168in}{1.483634in}}%
\pgfpathlineto{\pgfqpoint{3.338804in}{1.476116in}}%
\pgfpathclose%
\pgfusepath{fill}%
\end{pgfscope}%
\begin{pgfscope}%
\pgfpathrectangle{\pgfqpoint{1.150000in}{0.150000in}}{\pgfqpoint{5.700000in}{5.700000in}}%
\pgfusepath{clip}%
\pgfsetbuttcap%
\pgfsetroundjoin%
\definecolor{currentfill}{rgb}{0.180653,0.701402,0.488189}%
\pgfsetfillcolor{currentfill}%
\pgfsetfillopacity{0.800000}%
\pgfsetlinewidth{0.000000pt}%
\definecolor{currentstroke}{rgb}{0.000000,0.000000,0.000000}%
\pgfsetstrokecolor{currentstroke}%
\pgfsetdash{}{0pt}%
\pgfpathmoveto{\pgfqpoint{5.593955in}{3.324337in}}%
\pgfpathlineto{\pgfqpoint{5.608832in}{3.339541in}}%
\pgfpathlineto{\pgfqpoint{5.623730in}{3.354929in}}%
\pgfpathlineto{\pgfqpoint{5.638651in}{3.370502in}}%
\pgfpathlineto{\pgfqpoint{5.653594in}{3.386260in}}%
\pgfpathlineto{\pgfqpoint{5.660923in}{3.388470in}}%
\pgfpathlineto{\pgfqpoint{5.668242in}{3.390594in}}%
\pgfpathlineto{\pgfqpoint{5.675552in}{3.392636in}}%
\pgfpathlineto{\pgfqpoint{5.682853in}{3.394601in}}%
\pgfpathlineto{\pgfqpoint{5.667934in}{3.379283in}}%
\pgfpathlineto{\pgfqpoint{5.653036in}{3.364147in}}%
\pgfpathlineto{\pgfqpoint{5.638161in}{3.349196in}}%
\pgfpathlineto{\pgfqpoint{5.623308in}{3.334427in}}%
\pgfpathlineto{\pgfqpoint{5.615984in}{3.332012in}}%
\pgfpathlineto{\pgfqpoint{5.608650in}{3.329529in}}%
\pgfpathlineto{\pgfqpoint{5.601307in}{3.326972in}}%
\pgfpathlineto{\pgfqpoint{5.593955in}{3.324337in}}%
\pgfpathclose%
\pgfusepath{fill}%
\end{pgfscope}%
\begin{pgfscope}%
\pgfpathrectangle{\pgfqpoint{1.150000in}{0.150000in}}{\pgfqpoint{5.700000in}{5.700000in}}%
\pgfusepath{clip}%
\pgfsetbuttcap%
\pgfsetroundjoin%
\definecolor{currentfill}{rgb}{0.137770,0.537492,0.554906}%
\pgfsetfillcolor{currentfill}%
\pgfsetfillopacity{0.800000}%
\pgfsetlinewidth{0.000000pt}%
\definecolor{currentstroke}{rgb}{0.000000,0.000000,0.000000}%
\pgfsetstrokecolor{currentstroke}%
\pgfsetdash{}{0pt}%
\pgfpathmoveto{\pgfqpoint{5.000021in}{2.811227in}}%
\pgfpathlineto{\pgfqpoint{5.014514in}{2.824216in}}%
\pgfpathlineto{\pgfqpoint{5.029027in}{2.837390in}}%
\pgfpathlineto{\pgfqpoint{5.043558in}{2.850749in}}%
\pgfpathlineto{\pgfqpoint{5.058109in}{2.864293in}}%
\pgfpathlineto{\pgfqpoint{5.065823in}{2.872188in}}%
\pgfpathlineto{\pgfqpoint{5.073530in}{2.879939in}}%
\pgfpathlineto{\pgfqpoint{5.081227in}{2.887549in}}%
\pgfpathlineto{\pgfqpoint{5.088916in}{2.895019in}}%
\pgfpathlineto{\pgfqpoint{5.074373in}{2.881629in}}%
\pgfpathlineto{\pgfqpoint{5.059850in}{2.868423in}}%
\pgfpathlineto{\pgfqpoint{5.045345in}{2.855402in}}%
\pgfpathlineto{\pgfqpoint{5.030859in}{2.842565in}}%
\pgfpathlineto{\pgfqpoint{5.023162in}{2.834929in}}%
\pgfpathlineto{\pgfqpoint{5.015456in}{2.827162in}}%
\pgfpathlineto{\pgfqpoint{5.007743in}{2.819262in}}%
\pgfpathlineto{\pgfqpoint{5.000021in}{2.811227in}}%
\pgfpathclose%
\pgfusepath{fill}%
\end{pgfscope}%
\begin{pgfscope}%
\pgfpathrectangle{\pgfqpoint{1.150000in}{0.150000in}}{\pgfqpoint{5.700000in}{5.700000in}}%
\pgfusepath{clip}%
\pgfsetbuttcap%
\pgfsetroundjoin%
\definecolor{currentfill}{rgb}{0.277134,0.185228,0.489898}%
\pgfsetfillcolor{currentfill}%
\pgfsetfillopacity{0.800000}%
\pgfsetlinewidth{0.000000pt}%
\definecolor{currentstroke}{rgb}{0.000000,0.000000,0.000000}%
\pgfsetstrokecolor{currentstroke}%
\pgfsetdash{}{0pt}%
\pgfpathmoveto{\pgfqpoint{4.044797in}{1.831919in}}%
\pgfpathlineto{\pgfqpoint{4.058770in}{1.836835in}}%
\pgfpathlineto{\pgfqpoint{4.072755in}{1.841935in}}%
\pgfpathlineto{\pgfqpoint{4.086751in}{1.847220in}}%
\pgfpathlineto{\pgfqpoint{4.100758in}{1.852690in}}%
\pgfpathlineto{\pgfqpoint{4.108834in}{1.866080in}}%
\pgfpathlineto{\pgfqpoint{4.116906in}{1.879430in}}%
\pgfpathlineto{\pgfqpoint{4.124973in}{1.892735in}}%
\pgfpathlineto{\pgfqpoint{4.133035in}{1.905995in}}%
\pgfpathlineto{\pgfqpoint{4.119030in}{1.900211in}}%
\pgfpathlineto{\pgfqpoint{4.105036in}{1.894611in}}%
\pgfpathlineto{\pgfqpoint{4.091054in}{1.889197in}}%
\pgfpathlineto{\pgfqpoint{4.077084in}{1.883967in}}%
\pgfpathlineto{\pgfqpoint{4.069019in}{1.871010in}}%
\pgfpathlineto{\pgfqpoint{4.060950in}{1.858014in}}%
\pgfpathlineto{\pgfqpoint{4.052876in}{1.844983in}}%
\pgfpathlineto{\pgfqpoint{4.044797in}{1.831919in}}%
\pgfpathclose%
\pgfusepath{fill}%
\end{pgfscope}%
\begin{pgfscope}%
\pgfpathrectangle{\pgfqpoint{1.150000in}{0.150000in}}{\pgfqpoint{5.700000in}{5.700000in}}%
\pgfusepath{clip}%
\pgfsetbuttcap%
\pgfsetroundjoin%
\definecolor{currentfill}{rgb}{0.163625,0.471133,0.558148}%
\pgfsetfillcolor{currentfill}%
\pgfsetfillopacity{0.800000}%
\pgfsetlinewidth{0.000000pt}%
\definecolor{currentstroke}{rgb}{0.000000,0.000000,0.000000}%
\pgfsetstrokecolor{currentstroke}%
\pgfsetdash{}{0pt}%
\pgfpathmoveto{\pgfqpoint{4.791331in}{2.604414in}}%
\pgfpathlineto{\pgfqpoint{4.805694in}{2.616174in}}%
\pgfpathlineto{\pgfqpoint{4.820075in}{2.628119in}}%
\pgfpathlineto{\pgfqpoint{4.834474in}{2.640248in}}%
\pgfpathlineto{\pgfqpoint{4.848890in}{2.652563in}}%
\pgfpathlineto{\pgfqpoint{4.856710in}{2.662378in}}%
\pgfpathlineto{\pgfqpoint{4.864523in}{2.672048in}}%
\pgfpathlineto{\pgfqpoint{4.872328in}{2.681577in}}%
\pgfpathlineto{\pgfqpoint{4.880126in}{2.690964in}}%
\pgfpathlineto{\pgfqpoint{4.865713in}{2.678698in}}%
\pgfpathlineto{\pgfqpoint{4.851319in}{2.666617in}}%
\pgfpathlineto{\pgfqpoint{4.836942in}{2.654721in}}%
\pgfpathlineto{\pgfqpoint{4.822583in}{2.643009in}}%
\pgfpathlineto{\pgfqpoint{4.814780in}{2.633561in}}%
\pgfpathlineto{\pgfqpoint{4.806971in}{2.623980in}}%
\pgfpathlineto{\pgfqpoint{4.799154in}{2.614264in}}%
\pgfpathlineto{\pgfqpoint{4.791331in}{2.604414in}}%
\pgfpathclose%
\pgfusepath{fill}%
\end{pgfscope}%
\begin{pgfscope}%
\pgfpathrectangle{\pgfqpoint{1.150000in}{0.150000in}}{\pgfqpoint{5.700000in}{5.700000in}}%
\pgfusepath{clip}%
\pgfsetbuttcap%
\pgfsetroundjoin%
\definecolor{currentfill}{rgb}{0.252194,0.269783,0.531579}%
\pgfsetfillcolor{currentfill}%
\pgfsetfillopacity{0.800000}%
\pgfsetlinewidth{0.000000pt}%
\definecolor{currentstroke}{rgb}{0.000000,0.000000,0.000000}%
\pgfsetstrokecolor{currentstroke}%
\pgfsetdash{}{0pt}%
\pgfpathmoveto{\pgfqpoint{4.253499in}{2.037232in}}%
\pgfpathlineto{\pgfqpoint{4.267564in}{2.044473in}}%
\pgfpathlineto{\pgfqpoint{4.281643in}{2.051899in}}%
\pgfpathlineto{\pgfqpoint{4.295735in}{2.059510in}}%
\pgfpathlineto{\pgfqpoint{4.309841in}{2.067305in}}%
\pgfpathlineto{\pgfqpoint{4.317859in}{2.080502in}}%
\pgfpathlineto{\pgfqpoint{4.325872in}{2.093613in}}%
\pgfpathlineto{\pgfqpoint{4.333880in}{2.106638in}}%
\pgfpathlineto{\pgfqpoint{4.341883in}{2.119574in}}%
\pgfpathlineto{\pgfqpoint{4.327778in}{2.111560in}}%
\pgfpathlineto{\pgfqpoint{4.313686in}{2.103730in}}%
\pgfpathlineto{\pgfqpoint{4.299609in}{2.096086in}}%
\pgfpathlineto{\pgfqpoint{4.285544in}{2.088626in}}%
\pgfpathlineto{\pgfqpoint{4.277540in}{2.075896in}}%
\pgfpathlineto{\pgfqpoint{4.269531in}{2.063086in}}%
\pgfpathlineto{\pgfqpoint{4.261517in}{2.050197in}}%
\pgfpathlineto{\pgfqpoint{4.253499in}{2.037232in}}%
\pgfpathclose%
\pgfusepath{fill}%
\end{pgfscope}%
\begin{pgfscope}%
\pgfpathrectangle{\pgfqpoint{1.150000in}{0.150000in}}{\pgfqpoint{5.700000in}{5.700000in}}%
\pgfusepath{clip}%
\pgfsetbuttcap%
\pgfsetroundjoin%
\definecolor{currentfill}{rgb}{0.277018,0.050344,0.375715}%
\pgfsetfillcolor{currentfill}%
\pgfsetfillopacity{0.800000}%
\pgfsetlinewidth{0.000000pt}%
\definecolor{currentstroke}{rgb}{0.000000,0.000000,0.000000}%
\pgfsetstrokecolor{currentstroke}%
\pgfsetdash{}{0pt}%
\pgfpathmoveto{\pgfqpoint{3.659535in}{1.548179in}}%
\pgfpathlineto{\pgfqpoint{3.673395in}{1.548069in}}%
\pgfpathlineto{\pgfqpoint{3.687263in}{1.548148in}}%
\pgfpathlineto{\pgfqpoint{3.701138in}{1.548416in}}%
\pgfpathlineto{\pgfqpoint{3.715020in}{1.548871in}}%
\pgfpathlineto{\pgfqpoint{3.723217in}{1.560420in}}%
\pgfpathlineto{\pgfqpoint{3.731408in}{1.572033in}}%
\pgfpathlineto{\pgfqpoint{3.739594in}{1.583707in}}%
\pgfpathlineto{\pgfqpoint{3.747774in}{1.595436in}}%
\pgfpathlineto{\pgfqpoint{3.733901in}{1.594511in}}%
\pgfpathlineto{\pgfqpoint{3.720037in}{1.593774in}}%
\pgfpathlineto{\pgfqpoint{3.706180in}{1.593226in}}%
\pgfpathlineto{\pgfqpoint{3.692331in}{1.592867in}}%
\pgfpathlineto{\pgfqpoint{3.684141in}{1.581595in}}%
\pgfpathlineto{\pgfqpoint{3.675945in}{1.570387in}}%
\pgfpathlineto{\pgfqpoint{3.667743in}{1.559247in}}%
\pgfpathlineto{\pgfqpoint{3.659535in}{1.548179in}}%
\pgfpathclose%
\pgfusepath{fill}%
\end{pgfscope}%
\begin{pgfscope}%
\pgfpathrectangle{\pgfqpoint{1.150000in}{0.150000in}}{\pgfqpoint{5.700000in}{5.700000in}}%
\pgfusepath{clip}%
\pgfsetbuttcap%
\pgfsetroundjoin%
\definecolor{currentfill}{rgb}{0.277134,0.185228,0.489898}%
\pgfsetfillcolor{currentfill}%
\pgfsetfillopacity{0.800000}%
\pgfsetlinewidth{0.000000pt}%
\definecolor{currentstroke}{rgb}{0.000000,0.000000,0.000000}%
\pgfsetstrokecolor{currentstroke}%
\pgfsetdash{}{0pt}%
\pgfpathmoveto{\pgfqpoint{2.624972in}{1.905595in}}%
\pgfpathlineto{\pgfqpoint{2.638981in}{1.889163in}}%
\pgfpathlineto{\pgfqpoint{2.652982in}{1.872983in}}%
\pgfpathlineto{\pgfqpoint{2.666977in}{1.857052in}}%
\pgfpathlineto{\pgfqpoint{2.680965in}{1.841370in}}%
\pgfpathlineto{\pgfqpoint{2.689841in}{1.839625in}}%
\pgfpathlineto{\pgfqpoint{2.698697in}{1.838204in}}%
\pgfpathlineto{\pgfqpoint{2.707536in}{1.837102in}}%
\pgfpathlineto{\pgfqpoint{2.716356in}{1.836310in}}%
\pgfpathlineto{\pgfqpoint{2.702416in}{1.851310in}}%
\pgfpathlineto{\pgfqpoint{2.688470in}{1.866556in}}%
\pgfpathlineto{\pgfqpoint{2.674518in}{1.882052in}}%
\pgfpathlineto{\pgfqpoint{2.660559in}{1.897797in}}%
\pgfpathlineto{\pgfqpoint{2.651691in}{1.899259in}}%
\pgfpathlineto{\pgfqpoint{2.642804in}{1.901042in}}%
\pgfpathlineto{\pgfqpoint{2.633898in}{1.903151in}}%
\pgfpathlineto{\pgfqpoint{2.624972in}{1.905595in}}%
\pgfpathclose%
\pgfusepath{fill}%
\end{pgfscope}%
\begin{pgfscope}%
\pgfpathrectangle{\pgfqpoint{1.150000in}{0.150000in}}{\pgfqpoint{5.700000in}{5.700000in}}%
\pgfusepath{clip}%
\pgfsetbuttcap%
\pgfsetroundjoin%
\definecolor{currentfill}{rgb}{0.270595,0.214069,0.507052}%
\pgfsetfillcolor{currentfill}%
\pgfsetfillopacity{0.800000}%
\pgfsetlinewidth{0.000000pt}%
\definecolor{currentstroke}{rgb}{0.000000,0.000000,0.000000}%
\pgfsetstrokecolor{currentstroke}%
\pgfsetdash{}{0pt}%
\pgfpathmoveto{\pgfqpoint{2.568863in}{1.973872in}}%
\pgfpathlineto{\pgfqpoint{2.582902in}{1.956416in}}%
\pgfpathlineto{\pgfqpoint{2.596933in}{1.939219in}}%
\pgfpathlineto{\pgfqpoint{2.610956in}{1.922279in}}%
\pgfpathlineto{\pgfqpoint{2.624972in}{1.905595in}}%
\pgfpathlineto{\pgfqpoint{2.633898in}{1.903151in}}%
\pgfpathlineto{\pgfqpoint{2.642804in}{1.901042in}}%
\pgfpathlineto{\pgfqpoint{2.651691in}{1.899259in}}%
\pgfpathlineto{\pgfqpoint{2.660559in}{1.897797in}}%
\pgfpathlineto{\pgfqpoint{2.646593in}{1.913795in}}%
\pgfpathlineto{\pgfqpoint{2.632621in}{1.930047in}}%
\pgfpathlineto{\pgfqpoint{2.618641in}{1.946555in}}%
\pgfpathlineto{\pgfqpoint{2.604654in}{1.963320in}}%
\pgfpathlineto{\pgfqpoint{2.595736in}{1.965457in}}%
\pgfpathlineto{\pgfqpoint{2.586798in}{1.967923in}}%
\pgfpathlineto{\pgfqpoint{2.577841in}{1.970725in}}%
\pgfpathlineto{\pgfqpoint{2.568863in}{1.973872in}}%
\pgfpathclose%
\pgfusepath{fill}%
\end{pgfscope}%
\begin{pgfscope}%
\pgfpathrectangle{\pgfqpoint{1.150000in}{0.150000in}}{\pgfqpoint{5.700000in}{5.700000in}}%
\pgfusepath{clip}%
\pgfsetbuttcap%
\pgfsetroundjoin%
\definecolor{currentfill}{rgb}{0.280267,0.073417,0.397163}%
\pgfsetfillcolor{currentfill}%
\pgfsetfillopacity{0.800000}%
\pgfsetlinewidth{0.000000pt}%
\definecolor{currentstroke}{rgb}{0.000000,0.000000,0.000000}%
\pgfsetstrokecolor{currentstroke}%
\pgfsetdash{}{0pt}%
\pgfpathmoveto{\pgfqpoint{3.747774in}{1.595436in}}%
\pgfpathlineto{\pgfqpoint{3.761654in}{1.596549in}}%
\pgfpathlineto{\pgfqpoint{3.775542in}{1.597849in}}%
\pgfpathlineto{\pgfqpoint{3.789439in}{1.599336in}}%
\pgfpathlineto{\pgfqpoint{3.803344in}{1.601010in}}%
\pgfpathlineto{\pgfqpoint{3.811510in}{1.613242in}}%
\pgfpathlineto{\pgfqpoint{3.819671in}{1.625512in}}%
\pgfpathlineto{\pgfqpoint{3.827827in}{1.637818in}}%
\pgfpathlineto{\pgfqpoint{3.835978in}{1.650155in}}%
\pgfpathlineto{\pgfqpoint{3.822080in}{1.648043in}}%
\pgfpathlineto{\pgfqpoint{3.808191in}{1.646117in}}%
\pgfpathlineto{\pgfqpoint{3.794311in}{1.644378in}}%
\pgfpathlineto{\pgfqpoint{3.780439in}{1.642827in}}%
\pgfpathlineto{\pgfqpoint{3.772281in}{1.630916in}}%
\pgfpathlineto{\pgfqpoint{3.764117in}{1.619045in}}%
\pgfpathlineto{\pgfqpoint{3.755948in}{1.607217in}}%
\pgfpathlineto{\pgfqpoint{3.747774in}{1.595436in}}%
\pgfpathclose%
\pgfusepath{fill}%
\end{pgfscope}%
\begin{pgfscope}%
\pgfpathrectangle{\pgfqpoint{1.150000in}{0.150000in}}{\pgfqpoint{5.700000in}{5.700000in}}%
\pgfusepath{clip}%
\pgfsetbuttcap%
\pgfsetroundjoin%
\definecolor{currentfill}{rgb}{0.272594,0.025563,0.353093}%
\pgfsetfillcolor{currentfill}%
\pgfsetfillopacity{0.800000}%
\pgfsetlinewidth{0.000000pt}%
\definecolor{currentstroke}{rgb}{0.000000,0.000000,0.000000}%
\pgfsetstrokecolor{currentstroke}%
\pgfsetdash{}{0pt}%
\pgfpathmoveto{\pgfqpoint{3.571217in}{1.509064in}}%
\pgfpathlineto{\pgfqpoint{3.585064in}{1.507694in}}%
\pgfpathlineto{\pgfqpoint{3.598917in}{1.506515in}}%
\pgfpathlineto{\pgfqpoint{3.612776in}{1.505526in}}%
\pgfpathlineto{\pgfqpoint{3.626642in}{1.504726in}}%
\pgfpathlineto{\pgfqpoint{3.634875in}{1.515457in}}%
\pgfpathlineto{\pgfqpoint{3.643101in}{1.526280in}}%
\pgfpathlineto{\pgfqpoint{3.651321in}{1.537189in}}%
\pgfpathlineto{\pgfqpoint{3.659535in}{1.548179in}}%
\pgfpathlineto{\pgfqpoint{3.645682in}{1.548478in}}%
\pgfpathlineto{\pgfqpoint{3.631835in}{1.548967in}}%
\pgfpathlineto{\pgfqpoint{3.617995in}{1.549646in}}%
\pgfpathlineto{\pgfqpoint{3.604162in}{1.550516in}}%
\pgfpathlineto{\pgfqpoint{3.595935in}{1.540014in}}%
\pgfpathlineto{\pgfqpoint{3.587702in}{1.529601in}}%
\pgfpathlineto{\pgfqpoint{3.579463in}{1.519283in}}%
\pgfpathlineto{\pgfqpoint{3.571217in}{1.509064in}}%
\pgfpathclose%
\pgfusepath{fill}%
\end{pgfscope}%
\begin{pgfscope}%
\pgfpathrectangle{\pgfqpoint{1.150000in}{0.150000in}}{\pgfqpoint{5.700000in}{5.700000in}}%
\pgfusepath{clip}%
\pgfsetbuttcap%
\pgfsetroundjoin%
\definecolor{currentfill}{rgb}{0.280868,0.160771,0.472899}%
\pgfsetfillcolor{currentfill}%
\pgfsetfillopacity{0.800000}%
\pgfsetlinewidth{0.000000pt}%
\definecolor{currentstroke}{rgb}{0.000000,0.000000,0.000000}%
\pgfsetstrokecolor{currentstroke}%
\pgfsetdash{}{0pt}%
\pgfpathmoveto{\pgfqpoint{2.680965in}{1.841370in}}%
\pgfpathlineto{\pgfqpoint{2.694947in}{1.825934in}}%
\pgfpathlineto{\pgfqpoint{2.708923in}{1.810742in}}%
\pgfpathlineto{\pgfqpoint{2.722894in}{1.795793in}}%
\pgfpathlineto{\pgfqpoint{2.736858in}{1.781086in}}%
\pgfpathlineto{\pgfqpoint{2.745686in}{1.780034in}}%
\pgfpathlineto{\pgfqpoint{2.754495in}{1.779299in}}%
\pgfpathlineto{\pgfqpoint{2.763287in}{1.778873in}}%
\pgfpathlineto{\pgfqpoint{2.772062in}{1.778749in}}%
\pgfpathlineto{\pgfqpoint{2.758143in}{1.792777in}}%
\pgfpathlineto{\pgfqpoint{2.744220in}{1.807046in}}%
\pgfpathlineto{\pgfqpoint{2.730291in}{1.821556in}}%
\pgfpathlineto{\pgfqpoint{2.716356in}{1.836310in}}%
\pgfpathlineto{\pgfqpoint{2.707536in}{1.837102in}}%
\pgfpathlineto{\pgfqpoint{2.698697in}{1.838204in}}%
\pgfpathlineto{\pgfqpoint{2.689841in}{1.839625in}}%
\pgfpathlineto{\pgfqpoint{2.680965in}{1.841370in}}%
\pgfpathclose%
\pgfusepath{fill}%
\end{pgfscope}%
\begin{pgfscope}%
\pgfpathrectangle{\pgfqpoint{1.150000in}{0.150000in}}{\pgfqpoint{5.700000in}{5.700000in}}%
\pgfusepath{clip}%
\pgfsetbuttcap%
\pgfsetroundjoin%
\definecolor{currentfill}{rgb}{0.262138,0.242286,0.520837}%
\pgfsetfillcolor{currentfill}%
\pgfsetfillopacity{0.800000}%
\pgfsetlinewidth{0.000000pt}%
\definecolor{currentstroke}{rgb}{0.000000,0.000000,0.000000}%
\pgfsetstrokecolor{currentstroke}%
\pgfsetdash{}{0pt}%
\pgfpathmoveto{\pgfqpoint{2.512622in}{2.046321in}}%
\pgfpathlineto{\pgfqpoint{2.526695in}{2.027811in}}%
\pgfpathlineto{\pgfqpoint{2.540760in}{2.009567in}}%
\pgfpathlineto{\pgfqpoint{2.554816in}{1.991588in}}%
\pgfpathlineto{\pgfqpoint{2.568863in}{1.973872in}}%
\pgfpathlineto{\pgfqpoint{2.577841in}{1.970725in}}%
\pgfpathlineto{\pgfqpoint{2.586798in}{1.967923in}}%
\pgfpathlineto{\pgfqpoint{2.595736in}{1.965457in}}%
\pgfpathlineto{\pgfqpoint{2.604654in}{1.963320in}}%
\pgfpathlineto{\pgfqpoint{2.590659in}{1.980345in}}%
\pgfpathlineto{\pgfqpoint{2.576656in}{1.997632in}}%
\pgfpathlineto{\pgfqpoint{2.562645in}{2.015182in}}%
\pgfpathlineto{\pgfqpoint{2.548625in}{2.032998in}}%
\pgfpathlineto{\pgfqpoint{2.539655in}{2.035814in}}%
\pgfpathlineto{\pgfqpoint{2.530665in}{2.038968in}}%
\pgfpathlineto{\pgfqpoint{2.521654in}{2.042468in}}%
\pgfpathlineto{\pgfqpoint{2.512622in}{2.046321in}}%
\pgfpathclose%
\pgfusepath{fill}%
\end{pgfscope}%
\begin{pgfscope}%
\pgfpathrectangle{\pgfqpoint{1.150000in}{0.150000in}}{\pgfqpoint{5.700000in}{5.700000in}}%
\pgfusepath{clip}%
\pgfsetbuttcap%
\pgfsetroundjoin%
\definecolor{currentfill}{rgb}{0.273809,0.031497,0.358853}%
\pgfsetfillcolor{currentfill}%
\pgfsetfillopacity{0.800000}%
\pgfsetlinewidth{0.000000pt}%
\definecolor{currentstroke}{rgb}{0.000000,0.000000,0.000000}%
\pgfsetstrokecolor{currentstroke}%
\pgfsetdash{}{0pt}%
\pgfpathmoveto{\pgfqpoint{3.049708in}{1.546469in}}%
\pgfpathlineto{\pgfqpoint{3.063570in}{1.537170in}}%
\pgfpathlineto{\pgfqpoint{3.077432in}{1.528083in}}%
\pgfpathlineto{\pgfqpoint{3.091293in}{1.519207in}}%
\pgfpathlineto{\pgfqpoint{3.105155in}{1.510541in}}%
\pgfpathlineto{\pgfqpoint{3.113685in}{1.514607in}}%
\pgfpathlineto{\pgfqpoint{3.122203in}{1.518910in}}%
\pgfpathlineto{\pgfqpoint{3.130709in}{1.523444in}}%
\pgfpathlineto{\pgfqpoint{3.139204in}{1.528202in}}%
\pgfpathlineto{\pgfqpoint{3.125374in}{1.536237in}}%
\pgfpathlineto{\pgfqpoint{3.111544in}{1.544481in}}%
\pgfpathlineto{\pgfqpoint{3.097715in}{1.552937in}}%
\pgfpathlineto{\pgfqpoint{3.083886in}{1.561604in}}%
\pgfpathlineto{\pgfqpoint{3.075361in}{1.557464in}}%
\pgfpathlineto{\pgfqpoint{3.066823in}{1.553558in}}%
\pgfpathlineto{\pgfqpoint{3.058272in}{1.549891in}}%
\pgfpathlineto{\pgfqpoint{3.049708in}{1.546469in}}%
\pgfpathclose%
\pgfusepath{fill}%
\end{pgfscope}%
\begin{pgfscope}%
\pgfpathrectangle{\pgfqpoint{1.150000in}{0.150000in}}{\pgfqpoint{5.700000in}{5.700000in}}%
\pgfusepath{clip}%
\pgfsetbuttcap%
\pgfsetroundjoin%
\definecolor{currentfill}{rgb}{0.120638,0.625828,0.533488}%
\pgfsetfillcolor{currentfill}%
\pgfsetfillopacity{0.800000}%
\pgfsetlinewidth{0.000000pt}%
\definecolor{currentstroke}{rgb}{0.000000,0.000000,0.000000}%
\pgfsetstrokecolor{currentstroke}%
\pgfsetdash{}{0pt}%
\pgfpathmoveto{\pgfqpoint{5.297322in}{3.083082in}}%
\pgfpathlineto{\pgfqpoint{5.312012in}{3.097465in}}%
\pgfpathlineto{\pgfqpoint{5.326723in}{3.112033in}}%
\pgfpathlineto{\pgfqpoint{5.341455in}{3.126787in}}%
\pgfpathlineto{\pgfqpoint{5.356208in}{3.141725in}}%
\pgfpathlineto{\pgfqpoint{5.363747in}{3.146770in}}%
\pgfpathlineto{\pgfqpoint{5.371276in}{3.151689in}}%
\pgfpathlineto{\pgfqpoint{5.378796in}{3.156484in}}%
\pgfpathlineto{\pgfqpoint{5.386307in}{3.161159in}}%
\pgfpathlineto{\pgfqpoint{5.371569in}{3.146517in}}%
\pgfpathlineto{\pgfqpoint{5.356852in}{3.132058in}}%
\pgfpathlineto{\pgfqpoint{5.342155in}{3.117785in}}%
\pgfpathlineto{\pgfqpoint{5.327479in}{3.103695in}}%
\pgfpathlineto{\pgfqpoint{5.319954in}{3.098712in}}%
\pgfpathlineto{\pgfqpoint{5.312419in}{3.093618in}}%
\pgfpathlineto{\pgfqpoint{5.304875in}{3.088409in}}%
\pgfpathlineto{\pgfqpoint{5.297322in}{3.083082in}}%
\pgfpathclose%
\pgfusepath{fill}%
\end{pgfscope}%
\begin{pgfscope}%
\pgfpathrectangle{\pgfqpoint{1.150000in}{0.150000in}}{\pgfqpoint{5.700000in}{5.700000in}}%
\pgfusepath{clip}%
\pgfsetbuttcap%
\pgfsetroundjoin%
\definecolor{currentfill}{rgb}{0.282884,0.135920,0.453427}%
\pgfsetfillcolor{currentfill}%
\pgfsetfillopacity{0.800000}%
\pgfsetlinewidth{0.000000pt}%
\definecolor{currentstroke}{rgb}{0.000000,0.000000,0.000000}%
\pgfsetstrokecolor{currentstroke}%
\pgfsetdash{}{0pt}%
\pgfpathmoveto{\pgfqpoint{2.736858in}{1.781086in}}%
\pgfpathlineto{\pgfqpoint{2.750817in}{1.766618in}}%
\pgfpathlineto{\pgfqpoint{2.764772in}{1.752388in}}%
\pgfpathlineto{\pgfqpoint{2.778721in}{1.738395in}}%
\pgfpathlineto{\pgfqpoint{2.792665in}{1.724637in}}%
\pgfpathlineto{\pgfqpoint{2.801447in}{1.724277in}}%
\pgfpathlineto{\pgfqpoint{2.810211in}{1.724223in}}%
\pgfpathlineto{\pgfqpoint{2.818959in}{1.724470in}}%
\pgfpathlineto{\pgfqpoint{2.827690in}{1.725010in}}%
\pgfpathlineto{\pgfqpoint{2.813790in}{1.738092in}}%
\pgfpathlineto{\pgfqpoint{2.799885in}{1.751408in}}%
\pgfpathlineto{\pgfqpoint{2.785976in}{1.764960in}}%
\pgfpathlineto{\pgfqpoint{2.772062in}{1.778749in}}%
\pgfpathlineto{\pgfqpoint{2.763287in}{1.778873in}}%
\pgfpathlineto{\pgfqpoint{2.754495in}{1.779299in}}%
\pgfpathlineto{\pgfqpoint{2.745686in}{1.780034in}}%
\pgfpathlineto{\pgfqpoint{2.736858in}{1.781086in}}%
\pgfpathclose%
\pgfusepath{fill}%
\end{pgfscope}%
\begin{pgfscope}%
\pgfpathrectangle{\pgfqpoint{1.150000in}{0.150000in}}{\pgfqpoint{5.700000in}{5.700000in}}%
\pgfusepath{clip}%
\pgfsetbuttcap%
\pgfsetroundjoin%
\definecolor{currentfill}{rgb}{0.214298,0.355619,0.551184}%
\pgfsetfillcolor{currentfill}%
\pgfsetfillopacity{0.800000}%
\pgfsetlinewidth{0.000000pt}%
\definecolor{currentstroke}{rgb}{0.000000,0.000000,0.000000}%
\pgfsetstrokecolor{currentstroke}%
\pgfsetdash{}{0pt}%
\pgfpathmoveto{\pgfqpoint{4.462283in}{2.254945in}}%
\pgfpathlineto{\pgfqpoint{4.476459in}{2.264222in}}%
\pgfpathlineto{\pgfqpoint{4.490651in}{2.273683in}}%
\pgfpathlineto{\pgfqpoint{4.504858in}{2.283328in}}%
\pgfpathlineto{\pgfqpoint{4.519080in}{2.293159in}}%
\pgfpathlineto{\pgfqpoint{4.527035in}{2.305497in}}%
\pgfpathlineto{\pgfqpoint{4.534984in}{2.317718in}}%
\pgfpathlineto{\pgfqpoint{4.542928in}{2.329820in}}%
\pgfpathlineto{\pgfqpoint{4.550865in}{2.341802in}}%
\pgfpathlineto{\pgfqpoint{4.536644in}{2.331851in}}%
\pgfpathlineto{\pgfqpoint{4.522437in}{2.322085in}}%
\pgfpathlineto{\pgfqpoint{4.508246in}{2.312503in}}%
\pgfpathlineto{\pgfqpoint{4.494071in}{2.303106in}}%
\pgfpathlineto{\pgfqpoint{4.486132in}{2.291232in}}%
\pgfpathlineto{\pgfqpoint{4.478188in}{2.279247in}}%
\pgfpathlineto{\pgfqpoint{4.470238in}{2.267151in}}%
\pgfpathlineto{\pgfqpoint{4.462283in}{2.254945in}}%
\pgfpathclose%
\pgfusepath{fill}%
\end{pgfscope}%
\begin{pgfscope}%
\pgfpathrectangle{\pgfqpoint{1.150000in}{0.150000in}}{\pgfqpoint{5.700000in}{5.700000in}}%
\pgfusepath{clip}%
\pgfsetbuttcap%
\pgfsetroundjoin%
\definecolor{currentfill}{rgb}{0.250425,0.274290,0.533103}%
\pgfsetfillcolor{currentfill}%
\pgfsetfillopacity{0.800000}%
\pgfsetlinewidth{0.000000pt}%
\definecolor{currentstroke}{rgb}{0.000000,0.000000,0.000000}%
\pgfsetstrokecolor{currentstroke}%
\pgfsetdash{}{0pt}%
\pgfpathmoveto{\pgfqpoint{2.456231in}{2.123074in}}%
\pgfpathlineto{\pgfqpoint{2.470344in}{2.103475in}}%
\pgfpathlineto{\pgfqpoint{2.484446in}{2.084152in}}%
\pgfpathlineto{\pgfqpoint{2.498539in}{2.065101in}}%
\pgfpathlineto{\pgfqpoint{2.512622in}{2.046321in}}%
\pgfpathlineto{\pgfqpoint{2.521654in}{2.042468in}}%
\pgfpathlineto{\pgfqpoint{2.530665in}{2.038968in}}%
\pgfpathlineto{\pgfqpoint{2.539655in}{2.035814in}}%
\pgfpathlineto{\pgfqpoint{2.548625in}{2.032998in}}%
\pgfpathlineto{\pgfqpoint{2.534597in}{2.051081in}}%
\pgfpathlineto{\pgfqpoint{2.520559in}{2.069435in}}%
\pgfpathlineto{\pgfqpoint{2.506513in}{2.088060in}}%
\pgfpathlineto{\pgfqpoint{2.492456in}{2.106959in}}%
\pgfpathlineto{\pgfqpoint{2.483433in}{2.110458in}}%
\pgfpathlineto{\pgfqpoint{2.474387in}{2.114306in}}%
\pgfpathlineto{\pgfqpoint{2.465320in}{2.118509in}}%
\pgfpathlineto{\pgfqpoint{2.456231in}{2.123074in}}%
\pgfpathclose%
\pgfusepath{fill}%
\end{pgfscope}%
\begin{pgfscope}%
\pgfpathrectangle{\pgfqpoint{1.150000in}{0.150000in}}{\pgfqpoint{5.700000in}{5.700000in}}%
\pgfusepath{clip}%
\pgfsetbuttcap%
\pgfsetroundjoin%
\definecolor{currentfill}{rgb}{0.282910,0.105393,0.426902}%
\pgfsetfillcolor{currentfill}%
\pgfsetfillopacity{0.800000}%
\pgfsetlinewidth{0.000000pt}%
\definecolor{currentstroke}{rgb}{0.000000,0.000000,0.000000}%
\pgfsetstrokecolor{currentstroke}%
\pgfsetdash{}{0pt}%
\pgfpathmoveto{\pgfqpoint{3.835978in}{1.650155in}}%
\pgfpathlineto{\pgfqpoint{3.849884in}{1.652455in}}%
\pgfpathlineto{\pgfqpoint{3.863800in}{1.654941in}}%
\pgfpathlineto{\pgfqpoint{3.877725in}{1.657613in}}%
\pgfpathlineto{\pgfqpoint{3.891659in}{1.660470in}}%
\pgfpathlineto{\pgfqpoint{3.899798in}{1.673254in}}%
\pgfpathlineto{\pgfqpoint{3.907933in}{1.686054in}}%
\pgfpathlineto{\pgfqpoint{3.916062in}{1.698866in}}%
\pgfpathlineto{\pgfqpoint{3.924187in}{1.711685in}}%
\pgfpathlineto{\pgfqpoint{3.910258in}{1.708420in}}%
\pgfpathlineto{\pgfqpoint{3.896339in}{1.705340in}}%
\pgfpathlineto{\pgfqpoint{3.882430in}{1.702446in}}%
\pgfpathlineto{\pgfqpoint{3.868530in}{1.699739in}}%
\pgfpathlineto{\pgfqpoint{3.860399in}{1.687316in}}%
\pgfpathlineto{\pgfqpoint{3.852264in}{1.674908in}}%
\pgfpathlineto{\pgfqpoint{3.844123in}{1.662520in}}%
\pgfpathlineto{\pgfqpoint{3.835978in}{1.650155in}}%
\pgfpathclose%
\pgfusepath{fill}%
\end{pgfscope}%
\begin{pgfscope}%
\pgfpathrectangle{\pgfqpoint{1.150000in}{0.150000in}}{\pgfqpoint{5.700000in}{5.700000in}}%
\pgfusepath{clip}%
\pgfsetbuttcap%
\pgfsetroundjoin%
\definecolor{currentfill}{rgb}{0.268510,0.009605,0.335427}%
\pgfsetfillcolor{currentfill}%
\pgfsetfillopacity{0.800000}%
\pgfsetlinewidth{0.000000pt}%
\definecolor{currentstroke}{rgb}{0.000000,0.000000,0.000000}%
\pgfsetstrokecolor{currentstroke}%
\pgfsetdash{}{0pt}%
\pgfpathmoveto{\pgfqpoint{3.482769in}{1.478798in}}%
\pgfpathlineto{\pgfqpoint{3.496610in}{1.476129in}}%
\pgfpathlineto{\pgfqpoint{3.510455in}{1.473653in}}%
\pgfpathlineto{\pgfqpoint{3.524307in}{1.471369in}}%
\pgfpathlineto{\pgfqpoint{3.538163in}{1.469277in}}%
\pgfpathlineto{\pgfqpoint{3.546437in}{1.479050in}}%
\pgfpathlineto{\pgfqpoint{3.554704in}{1.488942in}}%
\pgfpathlineto{\pgfqpoint{3.562964in}{1.498949in}}%
\pgfpathlineto{\pgfqpoint{3.571217in}{1.509064in}}%
\pgfpathlineto{\pgfqpoint{3.557376in}{1.510625in}}%
\pgfpathlineto{\pgfqpoint{3.543540in}{1.512377in}}%
\pgfpathlineto{\pgfqpoint{3.529711in}{1.514322in}}%
\pgfpathlineto{\pgfqpoint{3.515887in}{1.516459in}}%
\pgfpathlineto{\pgfqpoint{3.507618in}{1.506864in}}%
\pgfpathlineto{\pgfqpoint{3.499342in}{1.497385in}}%
\pgfpathlineto{\pgfqpoint{3.491059in}{1.488028in}}%
\pgfpathlineto{\pgfqpoint{3.482769in}{1.478798in}}%
\pgfpathclose%
\pgfusepath{fill}%
\end{pgfscope}%
\begin{pgfscope}%
\pgfpathrectangle{\pgfqpoint{1.150000in}{0.150000in}}{\pgfqpoint{5.700000in}{5.700000in}}%
\pgfusepath{clip}%
\pgfsetbuttcap%
\pgfsetroundjoin%
\definecolor{currentfill}{rgb}{0.214000,0.722114,0.469588}%
\pgfsetfillcolor{currentfill}%
\pgfsetfillopacity{0.800000}%
\pgfsetlinewidth{0.000000pt}%
\definecolor{currentstroke}{rgb}{0.000000,0.000000,0.000000}%
\pgfsetstrokecolor{currentstroke}%
\pgfsetdash{}{0pt}%
\pgfpathmoveto{\pgfqpoint{5.682853in}{3.394601in}}%
\pgfpathlineto{\pgfqpoint{5.697795in}{3.410104in}}%
\pgfpathlineto{\pgfqpoint{5.712760in}{3.425791in}}%
\pgfpathlineto{\pgfqpoint{5.727748in}{3.441663in}}%
\pgfpathlineto{\pgfqpoint{5.742758in}{3.457719in}}%
\pgfpathlineto{\pgfqpoint{5.750025in}{3.459151in}}%
\pgfpathlineto{\pgfqpoint{5.757281in}{3.460508in}}%
\pgfpathlineto{\pgfqpoint{5.764529in}{3.461795in}}%
\pgfpathlineto{\pgfqpoint{5.771767in}{3.463017in}}%
\pgfpathlineto{\pgfqpoint{5.756783in}{3.447437in}}%
\pgfpathlineto{\pgfqpoint{5.741821in}{3.432039in}}%
\pgfpathlineto{\pgfqpoint{5.726882in}{3.416826in}}%
\pgfpathlineto{\pgfqpoint{5.711966in}{3.401795in}}%
\pgfpathlineto{\pgfqpoint{5.704701in}{3.400086in}}%
\pgfpathlineto{\pgfqpoint{5.697427in}{3.398321in}}%
\pgfpathlineto{\pgfqpoint{5.690145in}{3.396495in}}%
\pgfpathlineto{\pgfqpoint{5.682853in}{3.394601in}}%
\pgfpathclose%
\pgfusepath{fill}%
\end{pgfscope}%
\begin{pgfscope}%
\pgfpathrectangle{\pgfqpoint{1.150000in}{0.150000in}}{\pgfqpoint{5.700000in}{5.700000in}}%
\pgfusepath{clip}%
\pgfsetbuttcap%
\pgfsetroundjoin%
\definecolor{currentfill}{rgb}{0.171176,0.452530,0.557965}%
\pgfsetfillcolor{currentfill}%
\pgfsetfillopacity{0.800000}%
\pgfsetlinewidth{0.000000pt}%
\definecolor{currentstroke}{rgb}{0.000000,0.000000,0.000000}%
\pgfsetstrokecolor{currentstroke}%
\pgfsetdash{}{0pt}%
\pgfpathmoveto{\pgfqpoint{2.151459in}{2.649155in}}%
\pgfpathlineto{\pgfqpoint{2.165837in}{2.622913in}}%
\pgfpathlineto{\pgfqpoint{2.180198in}{2.597010in}}%
\pgfpathlineto{\pgfqpoint{2.194541in}{2.571444in}}%
\pgfpathlineto{\pgfqpoint{2.208868in}{2.546212in}}%
\pgfpathlineto{\pgfqpoint{2.218172in}{2.539598in}}%
\pgfpathlineto{\pgfqpoint{2.227450in}{2.533367in}}%
\pgfpathlineto{\pgfqpoint{2.236703in}{2.527513in}}%
\pgfpathlineto{\pgfqpoint{2.245933in}{2.522027in}}%
\pgfpathlineto{\pgfqpoint{2.231671in}{2.546566in}}%
\pgfpathlineto{\pgfqpoint{2.217393in}{2.571437in}}%
\pgfpathlineto{\pgfqpoint{2.203099in}{2.596642in}}%
\pgfpathlineto{\pgfqpoint{2.188788in}{2.622185in}}%
\pgfpathlineto{\pgfqpoint{2.179494in}{2.628352in}}%
\pgfpathlineto{\pgfqpoint{2.170175in}{2.634897in}}%
\pgfpathlineto{\pgfqpoint{2.160830in}{2.641829in}}%
\pgfpathlineto{\pgfqpoint{2.151459in}{2.649155in}}%
\pgfpathclose%
\pgfusepath{fill}%
\end{pgfscope}%
\begin{pgfscope}%
\pgfpathrectangle{\pgfqpoint{1.150000in}{0.150000in}}{\pgfqpoint{5.700000in}{5.700000in}}%
\pgfusepath{clip}%
\pgfsetbuttcap%
\pgfsetroundjoin%
\definecolor{currentfill}{rgb}{0.283197,0.115680,0.436115}%
\pgfsetfillcolor{currentfill}%
\pgfsetfillopacity{0.800000}%
\pgfsetlinewidth{0.000000pt}%
\definecolor{currentstroke}{rgb}{0.000000,0.000000,0.000000}%
\pgfsetstrokecolor{currentstroke}%
\pgfsetdash{}{0pt}%
\pgfpathmoveto{\pgfqpoint{2.792665in}{1.724637in}}%
\pgfpathlineto{\pgfqpoint{2.806605in}{1.711113in}}%
\pgfpathlineto{\pgfqpoint{2.820541in}{1.697821in}}%
\pgfpathlineto{\pgfqpoint{2.834473in}{1.684759in}}%
\pgfpathlineto{\pgfqpoint{2.848401in}{1.671927in}}%
\pgfpathlineto{\pgfqpoint{2.857139in}{1.672255in}}%
\pgfpathlineto{\pgfqpoint{2.865860in}{1.672880in}}%
\pgfpathlineto{\pgfqpoint{2.874566in}{1.673797in}}%
\pgfpathlineto{\pgfqpoint{2.883255in}{1.674998in}}%
\pgfpathlineto{\pgfqpoint{2.869369in}{1.687156in}}%
\pgfpathlineto{\pgfqpoint{2.855480in}{1.699544in}}%
\pgfpathlineto{\pgfqpoint{2.841587in}{1.712161in}}%
\pgfpathlineto{\pgfqpoint{2.827690in}{1.725010in}}%
\pgfpathlineto{\pgfqpoint{2.818959in}{1.724470in}}%
\pgfpathlineto{\pgfqpoint{2.810211in}{1.724223in}}%
\pgfpathlineto{\pgfqpoint{2.801447in}{1.724277in}}%
\pgfpathlineto{\pgfqpoint{2.792665in}{1.724637in}}%
\pgfpathclose%
\pgfusepath{fill}%
\end{pgfscope}%
\begin{pgfscope}%
\pgfpathrectangle{\pgfqpoint{1.150000in}{0.150000in}}{\pgfqpoint{5.700000in}{5.700000in}}%
\pgfusepath{clip}%
\pgfsetbuttcap%
\pgfsetroundjoin%
\definecolor{currentfill}{rgb}{0.269308,0.218818,0.509577}%
\pgfsetfillcolor{currentfill}%
\pgfsetfillopacity{0.800000}%
\pgfsetlinewidth{0.000000pt}%
\definecolor{currentstroke}{rgb}{0.000000,0.000000,0.000000}%
\pgfsetstrokecolor{currentstroke}%
\pgfsetdash{}{0pt}%
\pgfpathmoveto{\pgfqpoint{4.133035in}{1.905995in}}%
\pgfpathlineto{\pgfqpoint{4.147052in}{1.911965in}}%
\pgfpathlineto{\pgfqpoint{4.161082in}{1.918119in}}%
\pgfpathlineto{\pgfqpoint{4.175124in}{1.924457in}}%
\pgfpathlineto{\pgfqpoint{4.189179in}{1.930980in}}%
\pgfpathlineto{\pgfqpoint{4.197235in}{1.944486in}}%
\pgfpathlineto{\pgfqpoint{4.205287in}{1.957933in}}%
\pgfpathlineto{\pgfqpoint{4.213334in}{1.971317in}}%
\pgfpathlineto{\pgfqpoint{4.221376in}{1.984638in}}%
\pgfpathlineto{\pgfqpoint{4.207323in}{1.977832in}}%
\pgfpathlineto{\pgfqpoint{4.193282in}{1.971210in}}%
\pgfpathlineto{\pgfqpoint{4.179254in}{1.964774in}}%
\pgfpathlineto{\pgfqpoint{4.165238in}{1.958522in}}%
\pgfpathlineto{\pgfqpoint{4.157194in}{1.945472in}}%
\pgfpathlineto{\pgfqpoint{4.149146in}{1.932366in}}%
\pgfpathlineto{\pgfqpoint{4.141093in}{1.919206in}}%
\pgfpathlineto{\pgfqpoint{4.133035in}{1.905995in}}%
\pgfpathclose%
\pgfusepath{fill}%
\end{pgfscope}%
\begin{pgfscope}%
\pgfpathrectangle{\pgfqpoint{1.150000in}{0.150000in}}{\pgfqpoint{5.700000in}{5.700000in}}%
\pgfusepath{clip}%
\pgfsetbuttcap%
\pgfsetroundjoin%
\definecolor{currentfill}{rgb}{0.267004,0.004874,0.329415}%
\pgfsetfillcolor{currentfill}%
\pgfsetfillopacity{0.800000}%
\pgfsetlinewidth{0.000000pt}%
\definecolor{currentstroke}{rgb}{0.000000,0.000000,0.000000}%
\pgfsetstrokecolor{currentstroke}%
\pgfsetdash{}{0pt}%
\pgfpathmoveto{\pgfqpoint{3.249877in}{1.471359in}}%
\pgfpathlineto{\pgfqpoint{3.263719in}{1.465170in}}%
\pgfpathlineto{\pgfqpoint{3.277562in}{1.459182in}}%
\pgfpathlineto{\pgfqpoint{3.291408in}{1.453393in}}%
\pgfpathlineto{\pgfqpoint{3.305257in}{1.447805in}}%
\pgfpathlineto{\pgfqpoint{3.313658in}{1.454607in}}%
\pgfpathlineto{\pgfqpoint{3.322050in}{1.461597in}}%
\pgfpathlineto{\pgfqpoint{3.330432in}{1.468769in}}%
\pgfpathlineto{\pgfqpoint{3.338804in}{1.476116in}}%
\pgfpathlineto{\pgfqpoint{3.324980in}{1.481110in}}%
\pgfpathlineto{\pgfqpoint{3.311158in}{1.486302in}}%
\pgfpathlineto{\pgfqpoint{3.297339in}{1.491695in}}%
\pgfpathlineto{\pgfqpoint{3.283523in}{1.497288in}}%
\pgfpathlineto{\pgfqpoint{3.275127in}{1.490524in}}%
\pgfpathlineto{\pgfqpoint{3.266720in}{1.483944in}}%
\pgfpathlineto{\pgfqpoint{3.258304in}{1.477553in}}%
\pgfpathlineto{\pgfqpoint{3.249877in}{1.471359in}}%
\pgfpathclose%
\pgfusepath{fill}%
\end{pgfscope}%
\begin{pgfscope}%
\pgfpathrectangle{\pgfqpoint{1.150000in}{0.150000in}}{\pgfqpoint{5.700000in}{5.700000in}}%
\pgfusepath{clip}%
\pgfsetbuttcap%
\pgfsetroundjoin%
\definecolor{currentfill}{rgb}{0.237441,0.305202,0.541921}%
\pgfsetfillcolor{currentfill}%
\pgfsetfillopacity{0.800000}%
\pgfsetlinewidth{0.000000pt}%
\definecolor{currentstroke}{rgb}{0.000000,0.000000,0.000000}%
\pgfsetstrokecolor{currentstroke}%
\pgfsetdash{}{0pt}%
\pgfpathmoveto{\pgfqpoint{2.399674in}{2.204269in}}%
\pgfpathlineto{\pgfqpoint{2.413830in}{2.183546in}}%
\pgfpathlineto{\pgfqpoint{2.427974in}{2.163107in}}%
\pgfpathlineto{\pgfqpoint{2.442108in}{2.142951in}}%
\pgfpathlineto{\pgfqpoint{2.456231in}{2.123074in}}%
\pgfpathlineto{\pgfqpoint{2.465320in}{2.118509in}}%
\pgfpathlineto{\pgfqpoint{2.474387in}{2.114306in}}%
\pgfpathlineto{\pgfqpoint{2.483433in}{2.110458in}}%
\pgfpathlineto{\pgfqpoint{2.492456in}{2.106959in}}%
\pgfpathlineto{\pgfqpoint{2.478390in}{2.126135in}}%
\pgfpathlineto{\pgfqpoint{2.464314in}{2.145588in}}%
\pgfpathlineto{\pgfqpoint{2.450228in}{2.165323in}}%
\pgfpathlineto{\pgfqpoint{2.436130in}{2.185341in}}%
\pgfpathlineto{\pgfqpoint{2.427050in}{2.189529in}}%
\pgfpathlineto{\pgfqpoint{2.417947in}{2.194075in}}%
\pgfpathlineto{\pgfqpoint{2.408822in}{2.198986in}}%
\pgfpathlineto{\pgfqpoint{2.399674in}{2.204269in}}%
\pgfpathclose%
\pgfusepath{fill}%
\end{pgfscope}%
\begin{pgfscope}%
\pgfpathrectangle{\pgfqpoint{1.150000in}{0.150000in}}{\pgfqpoint{5.700000in}{5.700000in}}%
\pgfusepath{clip}%
\pgfsetbuttcap%
\pgfsetroundjoin%
\definecolor{currentfill}{rgb}{0.179019,0.433756,0.557430}%
\pgfsetfillcolor{currentfill}%
\pgfsetfillopacity{0.800000}%
\pgfsetlinewidth{0.000000pt}%
\definecolor{currentstroke}{rgb}{0.000000,0.000000,0.000000}%
\pgfsetstrokecolor{currentstroke}%
\pgfsetdash{}{0pt}%
\pgfpathmoveto{\pgfqpoint{4.671188in}{2.475506in}}%
\pgfpathlineto{\pgfqpoint{4.685488in}{2.486521in}}%
\pgfpathlineto{\pgfqpoint{4.699805in}{2.497721in}}%
\pgfpathlineto{\pgfqpoint{4.714139in}{2.509106in}}%
\pgfpathlineto{\pgfqpoint{4.728490in}{2.520676in}}%
\pgfpathlineto{\pgfqpoint{4.736369in}{2.531628in}}%
\pgfpathlineto{\pgfqpoint{4.744242in}{2.542440in}}%
\pgfpathlineto{\pgfqpoint{4.752107in}{2.553114in}}%
\pgfpathlineto{\pgfqpoint{4.759966in}{2.563649in}}%
\pgfpathlineto{\pgfqpoint{4.745617in}{2.552060in}}%
\pgfpathlineto{\pgfqpoint{4.731285in}{2.540655in}}%
\pgfpathlineto{\pgfqpoint{4.716970in}{2.529435in}}%
\pgfpathlineto{\pgfqpoint{4.702672in}{2.518400in}}%
\pgfpathlineto{\pgfqpoint{4.694811in}{2.507872in}}%
\pgfpathlineto{\pgfqpoint{4.686943in}{2.497214in}}%
\pgfpathlineto{\pgfqpoint{4.679069in}{2.486425in}}%
\pgfpathlineto{\pgfqpoint{4.671188in}{2.475506in}}%
\pgfpathclose%
\pgfusepath{fill}%
\end{pgfscope}%
\begin{pgfscope}%
\pgfpathrectangle{\pgfqpoint{1.150000in}{0.150000in}}{\pgfqpoint{5.700000in}{5.700000in}}%
\pgfusepath{clip}%
\pgfsetbuttcap%
\pgfsetroundjoin%
\definecolor{currentfill}{rgb}{0.126453,0.570633,0.549841}%
\pgfsetfillcolor{currentfill}%
\pgfsetfillopacity{0.800000}%
\pgfsetlinewidth{0.000000pt}%
\definecolor{currentstroke}{rgb}{0.000000,0.000000,0.000000}%
\pgfsetstrokecolor{currentstroke}%
\pgfsetdash{}{0pt}%
\pgfpathmoveto{\pgfqpoint{5.088916in}{2.895019in}}%
\pgfpathlineto{\pgfqpoint{5.103479in}{2.908595in}}%
\pgfpathlineto{\pgfqpoint{5.118061in}{2.922356in}}%
\pgfpathlineto{\pgfqpoint{5.132663in}{2.936302in}}%
\pgfpathlineto{\pgfqpoint{5.147284in}{2.950433in}}%
\pgfpathlineto{\pgfqpoint{5.154956in}{2.957590in}}%
\pgfpathlineto{\pgfqpoint{5.162619in}{2.964603in}}%
\pgfpathlineto{\pgfqpoint{5.170274in}{2.971475in}}%
\pgfpathlineto{\pgfqpoint{5.177919in}{2.978208in}}%
\pgfpathlineto{\pgfqpoint{5.163307in}{2.964266in}}%
\pgfpathlineto{\pgfqpoint{5.148714in}{2.950509in}}%
\pgfpathlineto{\pgfqpoint{5.134142in}{2.936937in}}%
\pgfpathlineto{\pgfqpoint{5.119588in}{2.923549in}}%
\pgfpathlineto{\pgfqpoint{5.111933in}{2.916615in}}%
\pgfpathlineto{\pgfqpoint{5.104269in}{2.909550in}}%
\pgfpathlineto{\pgfqpoint{5.096597in}{2.902352in}}%
\pgfpathlineto{\pgfqpoint{5.088916in}{2.895019in}}%
\pgfpathclose%
\pgfusepath{fill}%
\end{pgfscope}%
\begin{pgfscope}%
\pgfpathrectangle{\pgfqpoint{1.150000in}{0.150000in}}{\pgfqpoint{5.700000in}{5.700000in}}%
\pgfusepath{clip}%
\pgfsetbuttcap%
\pgfsetroundjoin%
\definecolor{currentfill}{rgb}{0.282884,0.135920,0.453427}%
\pgfsetfillcolor{currentfill}%
\pgfsetfillopacity{0.800000}%
\pgfsetlinewidth{0.000000pt}%
\definecolor{currentstroke}{rgb}{0.000000,0.000000,0.000000}%
\pgfsetstrokecolor{currentstroke}%
\pgfsetdash{}{0pt}%
\pgfpathmoveto{\pgfqpoint{3.924187in}{1.711685in}}%
\pgfpathlineto{\pgfqpoint{3.938125in}{1.715137in}}%
\pgfpathlineto{\pgfqpoint{3.952074in}{1.718774in}}%
\pgfpathlineto{\pgfqpoint{3.966033in}{1.722596in}}%
\pgfpathlineto{\pgfqpoint{3.980003in}{1.726603in}}%
\pgfpathlineto{\pgfqpoint{3.988118in}{1.739816in}}%
\pgfpathlineto{\pgfqpoint{3.996229in}{1.753022in}}%
\pgfpathlineto{\pgfqpoint{4.004335in}{1.766217in}}%
\pgfpathlineto{\pgfqpoint{4.012437in}{1.779398in}}%
\pgfpathlineto{\pgfqpoint{3.998471in}{1.775014in}}%
\pgfpathlineto{\pgfqpoint{3.984516in}{1.770814in}}%
\pgfpathlineto{\pgfqpoint{3.970572in}{1.766800in}}%
\pgfpathlineto{\pgfqpoint{3.956638in}{1.762972in}}%
\pgfpathlineto{\pgfqpoint{3.948532in}{1.750156in}}%
\pgfpathlineto{\pgfqpoint{3.940422in}{1.737334in}}%
\pgfpathlineto{\pgfqpoint{3.932307in}{1.724509in}}%
\pgfpathlineto{\pgfqpoint{3.924187in}{1.711685in}}%
\pgfpathclose%
\pgfusepath{fill}%
\end{pgfscope}%
\begin{pgfscope}%
\pgfpathrectangle{\pgfqpoint{1.150000in}{0.150000in}}{\pgfqpoint{5.700000in}{5.700000in}}%
\pgfusepath{clip}%
\pgfsetbuttcap%
\pgfsetroundjoin%
\definecolor{currentfill}{rgb}{0.150476,0.504369,0.557430}%
\pgfsetfillcolor{currentfill}%
\pgfsetfillopacity{0.800000}%
\pgfsetlinewidth{0.000000pt}%
\definecolor{currentstroke}{rgb}{0.000000,0.000000,0.000000}%
\pgfsetstrokecolor{currentstroke}%
\pgfsetdash{}{0pt}%
\pgfpathmoveto{\pgfqpoint{4.880126in}{2.690964in}}%
\pgfpathlineto{\pgfqpoint{4.894557in}{2.703415in}}%
\pgfpathlineto{\pgfqpoint{4.909006in}{2.716051in}}%
\pgfpathlineto{\pgfqpoint{4.923473in}{2.728872in}}%
\pgfpathlineto{\pgfqpoint{4.937959in}{2.741878in}}%
\pgfpathlineto{\pgfqpoint{4.945745in}{2.751054in}}%
\pgfpathlineto{\pgfqpoint{4.953522in}{2.760082in}}%
\pgfpathlineto{\pgfqpoint{4.961292in}{2.768964in}}%
\pgfpathlineto{\pgfqpoint{4.969054in}{2.777700in}}%
\pgfpathlineto{\pgfqpoint{4.954573in}{2.764778in}}%
\pgfpathlineto{\pgfqpoint{4.940110in}{2.752041in}}%
\pgfpathlineto{\pgfqpoint{4.925667in}{2.739489in}}%
\pgfpathlineto{\pgfqpoint{4.911241in}{2.727121in}}%
\pgfpathlineto{\pgfqpoint{4.903474in}{2.718288in}}%
\pgfpathlineto{\pgfqpoint{4.895699in}{2.709318in}}%
\pgfpathlineto{\pgfqpoint{4.887916in}{2.700211in}}%
\pgfpathlineto{\pgfqpoint{4.880126in}{2.690964in}}%
\pgfpathclose%
\pgfusepath{fill}%
\end{pgfscope}%
\begin{pgfscope}%
\pgfpathrectangle{\pgfqpoint{1.150000in}{0.150000in}}{\pgfqpoint{5.700000in}{5.700000in}}%
\pgfusepath{clip}%
\pgfsetbuttcap%
\pgfsetroundjoin%
\definecolor{currentfill}{rgb}{0.235526,0.309527,0.542944}%
\pgfsetfillcolor{currentfill}%
\pgfsetfillopacity{0.800000}%
\pgfsetlinewidth{0.000000pt}%
\definecolor{currentstroke}{rgb}{0.000000,0.000000,0.000000}%
\pgfsetstrokecolor{currentstroke}%
\pgfsetdash{}{0pt}%
\pgfpathmoveto{\pgfqpoint{4.341883in}{2.119574in}}%
\pgfpathlineto{\pgfqpoint{4.356003in}{2.127772in}}%
\pgfpathlineto{\pgfqpoint{4.370136in}{2.136155in}}%
\pgfpathlineto{\pgfqpoint{4.384284in}{2.144722in}}%
\pgfpathlineto{\pgfqpoint{4.398446in}{2.153474in}}%
\pgfpathlineto{\pgfqpoint{4.406444in}{2.166518in}}%
\pgfpathlineto{\pgfqpoint{4.414437in}{2.179462in}}%
\pgfpathlineto{\pgfqpoint{4.422424in}{2.192305in}}%
\pgfpathlineto{\pgfqpoint{4.430407in}{2.205044in}}%
\pgfpathlineto{\pgfqpoint{4.416245in}{2.196106in}}%
\pgfpathlineto{\pgfqpoint{4.402097in}{2.187352in}}%
\pgfpathlineto{\pgfqpoint{4.387964in}{2.178783in}}%
\pgfpathlineto{\pgfqpoint{4.373845in}{2.170398in}}%
\pgfpathlineto{\pgfqpoint{4.365863in}{2.157833in}}%
\pgfpathlineto{\pgfqpoint{4.357875in}{2.145173in}}%
\pgfpathlineto{\pgfqpoint{4.349882in}{2.132419in}}%
\pgfpathlineto{\pgfqpoint{4.341883in}{2.119574in}}%
\pgfpathclose%
\pgfusepath{fill}%
\end{pgfscope}%
\begin{pgfscope}%
\pgfpathrectangle{\pgfqpoint{1.150000in}{0.150000in}}{\pgfqpoint{5.700000in}{5.700000in}}%
\pgfusepath{clip}%
\pgfsetbuttcap%
\pgfsetroundjoin%
\definecolor{currentfill}{rgb}{0.259857,0.745492,0.444467}%
\pgfsetfillcolor{currentfill}%
\pgfsetfillopacity{0.800000}%
\pgfsetlinewidth{0.000000pt}%
\definecolor{currentstroke}{rgb}{0.000000,0.000000,0.000000}%
\pgfsetstrokecolor{currentstroke}%
\pgfsetdash{}{0pt}%
\pgfpathmoveto{\pgfqpoint{5.771767in}{3.463017in}}%
\pgfpathlineto{\pgfqpoint{5.786774in}{3.478782in}}%
\pgfpathlineto{\pgfqpoint{5.801805in}{3.494731in}}%
\pgfpathlineto{\pgfqpoint{5.816858in}{3.510864in}}%
\pgfpathlineto{\pgfqpoint{5.831936in}{3.527183in}}%
\pgfpathlineto{\pgfqpoint{5.839137in}{3.527848in}}%
\pgfpathlineto{\pgfqpoint{5.846328in}{3.528451in}}%
\pgfpathlineto{\pgfqpoint{5.853511in}{3.528998in}}%
\pgfpathlineto{\pgfqpoint{5.860685in}{3.529494in}}%
\pgfpathlineto{\pgfqpoint{5.845637in}{3.513688in}}%
\pgfpathlineto{\pgfqpoint{5.830612in}{3.498066in}}%
\pgfpathlineto{\pgfqpoint{5.815610in}{3.482627in}}%
\pgfpathlineto{\pgfqpoint{5.800631in}{3.467371in}}%
\pgfpathlineto{\pgfqpoint{5.793428in}{3.466352in}}%
\pgfpathlineto{\pgfqpoint{5.786216in}{3.465291in}}%
\pgfpathlineto{\pgfqpoint{5.778996in}{3.464181in}}%
\pgfpathlineto{\pgfqpoint{5.771767in}{3.463017in}}%
\pgfpathclose%
\pgfusepath{fill}%
\end{pgfscope}%
\begin{pgfscope}%
\pgfpathrectangle{\pgfqpoint{1.150000in}{0.150000in}}{\pgfqpoint{5.700000in}{5.700000in}}%
\pgfusepath{clip}%
\pgfsetbuttcap%
\pgfsetroundjoin%
\definecolor{currentfill}{rgb}{0.267004,0.004874,0.329415}%
\pgfsetfillcolor{currentfill}%
\pgfsetfillopacity{0.800000}%
\pgfsetlinewidth{0.000000pt}%
\definecolor{currentstroke}{rgb}{0.000000,0.000000,0.000000}%
\pgfsetstrokecolor{currentstroke}%
\pgfsetdash{}{0pt}%
\pgfpathmoveto{\pgfqpoint{3.394136in}{1.458121in}}%
\pgfpathlineto{\pgfqpoint{3.407978in}{1.454112in}}%
\pgfpathlineto{\pgfqpoint{3.421824in}{1.450299in}}%
\pgfpathlineto{\pgfqpoint{3.435674in}{1.446681in}}%
\pgfpathlineto{\pgfqpoint{3.449529in}{1.443256in}}%
\pgfpathlineto{\pgfqpoint{3.457851in}{1.451924in}}%
\pgfpathlineto{\pgfqpoint{3.466165in}{1.460741in}}%
\pgfpathlineto{\pgfqpoint{3.474471in}{1.469701in}}%
\pgfpathlineto{\pgfqpoint{3.482769in}{1.478798in}}%
\pgfpathlineto{\pgfqpoint{3.468933in}{1.481660in}}%
\pgfpathlineto{\pgfqpoint{3.455102in}{1.484715in}}%
\pgfpathlineto{\pgfqpoint{3.441275in}{1.487965in}}%
\pgfpathlineto{\pgfqpoint{3.427453in}{1.491410in}}%
\pgfpathlineto{\pgfqpoint{3.419136in}{1.482864in}}%
\pgfpathlineto{\pgfqpoint{3.410811in}{1.474463in}}%
\pgfpathlineto{\pgfqpoint{3.402478in}{1.466214in}}%
\pgfpathlineto{\pgfqpoint{3.394136in}{1.458121in}}%
\pgfpathclose%
\pgfusepath{fill}%
\end{pgfscope}%
\begin{pgfscope}%
\pgfpathrectangle{\pgfqpoint{1.150000in}{0.150000in}}{\pgfqpoint{5.700000in}{5.700000in}}%
\pgfusepath{clip}%
\pgfsetbuttcap%
\pgfsetroundjoin%
\definecolor{currentfill}{rgb}{0.282327,0.094955,0.417331}%
\pgfsetfillcolor{currentfill}%
\pgfsetfillopacity{0.800000}%
\pgfsetlinewidth{0.000000pt}%
\definecolor{currentstroke}{rgb}{0.000000,0.000000,0.000000}%
\pgfsetstrokecolor{currentstroke}%
\pgfsetdash{}{0pt}%
\pgfpathmoveto{\pgfqpoint{2.848401in}{1.671927in}}%
\pgfpathlineto{\pgfqpoint{2.862326in}{1.659324in}}%
\pgfpathlineto{\pgfqpoint{2.876247in}{1.646946in}}%
\pgfpathlineto{\pgfqpoint{2.890165in}{1.634795in}}%
\pgfpathlineto{\pgfqpoint{2.904080in}{1.622867in}}%
\pgfpathlineto{\pgfqpoint{2.912776in}{1.623879in}}%
\pgfpathlineto{\pgfqpoint{2.921456in}{1.625181in}}%
\pgfpathlineto{\pgfqpoint{2.930121in}{1.626764in}}%
\pgfpathlineto{\pgfqpoint{2.938771in}{1.628624in}}%
\pgfpathlineto{\pgfqpoint{2.924896in}{1.639881in}}%
\pgfpathlineto{\pgfqpoint{2.911019in}{1.651361in}}%
\pgfpathlineto{\pgfqpoint{2.897138in}{1.663066in}}%
\pgfpathlineto{\pgfqpoint{2.883255in}{1.674998in}}%
\pgfpathlineto{\pgfqpoint{2.874566in}{1.673797in}}%
\pgfpathlineto{\pgfqpoint{2.865860in}{1.672880in}}%
\pgfpathlineto{\pgfqpoint{2.857139in}{1.672255in}}%
\pgfpathlineto{\pgfqpoint{2.848401in}{1.671927in}}%
\pgfpathclose%
\pgfusepath{fill}%
\end{pgfscope}%
\begin{pgfscope}%
\pgfpathrectangle{\pgfqpoint{1.150000in}{0.150000in}}{\pgfqpoint{5.700000in}{5.700000in}}%
\pgfusepath{clip}%
\pgfsetbuttcap%
\pgfsetroundjoin%
\definecolor{currentfill}{rgb}{0.271305,0.019942,0.347269}%
\pgfsetfillcolor{currentfill}%
\pgfsetfillopacity{0.800000}%
\pgfsetlinewidth{0.000000pt}%
\definecolor{currentstroke}{rgb}{0.000000,0.000000,0.000000}%
\pgfsetstrokecolor{currentstroke}%
\pgfsetdash{}{0pt}%
\pgfpathmoveto{\pgfqpoint{3.105155in}{1.510541in}}%
\pgfpathlineto{\pgfqpoint{3.119016in}{1.502084in}}%
\pgfpathlineto{\pgfqpoint{3.132878in}{1.493835in}}%
\pgfpathlineto{\pgfqpoint{3.146740in}{1.485794in}}%
\pgfpathlineto{\pgfqpoint{3.160603in}{1.477958in}}%
\pgfpathlineto{\pgfqpoint{3.169102in}{1.482667in}}%
\pgfpathlineto{\pgfqpoint{3.177589in}{1.487604in}}%
\pgfpathlineto{\pgfqpoint{3.186065in}{1.492764in}}%
\pgfpathlineto{\pgfqpoint{3.194530in}{1.498140in}}%
\pgfpathlineto{\pgfqpoint{3.180697in}{1.505346in}}%
\pgfpathlineto{\pgfqpoint{3.166865in}{1.512757in}}%
\pgfpathlineto{\pgfqpoint{3.153034in}{1.520376in}}%
\pgfpathlineto{\pgfqpoint{3.139204in}{1.528202in}}%
\pgfpathlineto{\pgfqpoint{3.130709in}{1.523444in}}%
\pgfpathlineto{\pgfqpoint{3.122203in}{1.518910in}}%
\pgfpathlineto{\pgfqpoint{3.113685in}{1.514607in}}%
\pgfpathlineto{\pgfqpoint{3.105155in}{1.510541in}}%
\pgfpathclose%
\pgfusepath{fill}%
\end{pgfscope}%
\begin{pgfscope}%
\pgfpathrectangle{\pgfqpoint{1.150000in}{0.150000in}}{\pgfqpoint{5.700000in}{5.700000in}}%
\pgfusepath{clip}%
\pgfsetbuttcap%
\pgfsetroundjoin%
\definecolor{currentfill}{rgb}{0.132268,0.655014,0.519661}%
\pgfsetfillcolor{currentfill}%
\pgfsetfillopacity{0.800000}%
\pgfsetlinewidth{0.000000pt}%
\definecolor{currentstroke}{rgb}{0.000000,0.000000,0.000000}%
\pgfsetstrokecolor{currentstroke}%
\pgfsetdash{}{0pt}%
\pgfpathmoveto{\pgfqpoint{5.386307in}{3.161159in}}%
\pgfpathlineto{\pgfqpoint{5.401066in}{3.175987in}}%
\pgfpathlineto{\pgfqpoint{5.415847in}{3.190999in}}%
\pgfpathlineto{\pgfqpoint{5.430649in}{3.206197in}}%
\pgfpathlineto{\pgfqpoint{5.445473in}{3.221580in}}%
\pgfpathlineto{\pgfqpoint{5.452958in}{3.225820in}}%
\pgfpathlineto{\pgfqpoint{5.460433in}{3.229940in}}%
\pgfpathlineto{\pgfqpoint{5.467899in}{3.233943in}}%
\pgfpathlineto{\pgfqpoint{5.475355in}{3.237832in}}%
\pgfpathlineto{\pgfqpoint{5.460549in}{3.222781in}}%
\pgfpathlineto{\pgfqpoint{5.445763in}{3.207915in}}%
\pgfpathlineto{\pgfqpoint{5.430999in}{3.193234in}}%
\pgfpathlineto{\pgfqpoint{5.416257in}{3.178736in}}%
\pgfpathlineto{\pgfqpoint{5.408783in}{3.174503in}}%
\pgfpathlineto{\pgfqpoint{5.401300in}{3.170165in}}%
\pgfpathlineto{\pgfqpoint{5.393808in}{3.165718in}}%
\pgfpathlineto{\pgfqpoint{5.386307in}{3.161159in}}%
\pgfpathclose%
\pgfusepath{fill}%
\end{pgfscope}%
\begin{pgfscope}%
\pgfpathrectangle{\pgfqpoint{1.150000in}{0.150000in}}{\pgfqpoint{5.700000in}{5.700000in}}%
\pgfusepath{clip}%
\pgfsetbuttcap%
\pgfsetroundjoin%
\definecolor{currentfill}{rgb}{0.223925,0.334994,0.548053}%
\pgfsetfillcolor{currentfill}%
\pgfsetfillopacity{0.800000}%
\pgfsetlinewidth{0.000000pt}%
\definecolor{currentstroke}{rgb}{0.000000,0.000000,0.000000}%
\pgfsetstrokecolor{currentstroke}%
\pgfsetdash{}{0pt}%
\pgfpathmoveto{\pgfqpoint{2.342931in}{2.290056in}}%
\pgfpathlineto{\pgfqpoint{2.357135in}{2.268170in}}%
\pgfpathlineto{\pgfqpoint{2.371327in}{2.246579in}}%
\pgfpathlineto{\pgfqpoint{2.385506in}{2.225279in}}%
\pgfpathlineto{\pgfqpoint{2.399674in}{2.204269in}}%
\pgfpathlineto{\pgfqpoint{2.408822in}{2.198986in}}%
\pgfpathlineto{\pgfqpoint{2.417947in}{2.194075in}}%
\pgfpathlineto{\pgfqpoint{2.427050in}{2.189529in}}%
\pgfpathlineto{\pgfqpoint{2.436130in}{2.185341in}}%
\pgfpathlineto{\pgfqpoint{2.422022in}{2.205645in}}%
\pgfpathlineto{\pgfqpoint{2.407902in}{2.226236in}}%
\pgfpathlineto{\pgfqpoint{2.393771in}{2.247118in}}%
\pgfpathlineto{\pgfqpoint{2.379628in}{2.268292in}}%
\pgfpathlineto{\pgfqpoint{2.370489in}{2.273175in}}%
\pgfpathlineto{\pgfqpoint{2.361327in}{2.278425in}}%
\pgfpathlineto{\pgfqpoint{2.352141in}{2.284050in}}%
\pgfpathlineto{\pgfqpoint{2.342931in}{2.290056in}}%
\pgfpathclose%
\pgfusepath{fill}%
\end{pgfscope}%
\begin{pgfscope}%
\pgfpathrectangle{\pgfqpoint{1.150000in}{0.150000in}}{\pgfqpoint{5.700000in}{5.700000in}}%
\pgfusepath{clip}%
\pgfsetbuttcap%
\pgfsetroundjoin%
\definecolor{currentfill}{rgb}{0.197636,0.391528,0.554969}%
\pgfsetfillcolor{currentfill}%
\pgfsetfillopacity{0.800000}%
\pgfsetlinewidth{0.000000pt}%
\definecolor{currentstroke}{rgb}{0.000000,0.000000,0.000000}%
\pgfsetstrokecolor{currentstroke}%
\pgfsetdash{}{0pt}%
\pgfpathmoveto{\pgfqpoint{4.550865in}{2.341802in}}%
\pgfpathlineto{\pgfqpoint{4.565103in}{2.351937in}}%
\pgfpathlineto{\pgfqpoint{4.579357in}{2.362258in}}%
\pgfpathlineto{\pgfqpoint{4.593626in}{2.372763in}}%
\pgfpathlineto{\pgfqpoint{4.607912in}{2.383452in}}%
\pgfpathlineto{\pgfqpoint{4.615843in}{2.395414in}}%
\pgfpathlineto{\pgfqpoint{4.623768in}{2.407246in}}%
\pgfpathlineto{\pgfqpoint{4.631687in}{2.418949in}}%
\pgfpathlineto{\pgfqpoint{4.639600in}{2.430521in}}%
\pgfpathlineto{\pgfqpoint{4.625315in}{2.419744in}}%
\pgfpathlineto{\pgfqpoint{4.611046in}{2.409152in}}%
\pgfpathlineto{\pgfqpoint{4.596794in}{2.398745in}}%
\pgfpathlineto{\pgfqpoint{4.582557in}{2.388522in}}%
\pgfpathlineto{\pgfqpoint{4.574643in}{2.377024in}}%
\pgfpathlineto{\pgfqpoint{4.566723in}{2.365404in}}%
\pgfpathlineto{\pgfqpoint{4.558797in}{2.353663in}}%
\pgfpathlineto{\pgfqpoint{4.550865in}{2.341802in}}%
\pgfpathclose%
\pgfusepath{fill}%
\end{pgfscope}%
\begin{pgfscope}%
\pgfpathrectangle{\pgfqpoint{1.150000in}{0.150000in}}{\pgfqpoint{5.700000in}{5.700000in}}%
\pgfusepath{clip}%
\pgfsetbuttcap%
\pgfsetroundjoin%
\definecolor{currentfill}{rgb}{0.344074,0.780029,0.397381}%
\pgfsetfillcolor{currentfill}%
\pgfsetfillopacity{0.800000}%
\pgfsetlinewidth{0.000000pt}%
\definecolor{currentstroke}{rgb}{0.000000,0.000000,0.000000}%
\pgfsetstrokecolor{currentstroke}%
\pgfsetdash{}{0pt}%
\pgfpathmoveto{\pgfqpoint{5.949594in}{3.593965in}}%
\pgfpathlineto{\pgfqpoint{5.964729in}{3.610143in}}%
\pgfpathlineto{\pgfqpoint{5.979887in}{3.626505in}}%
\pgfpathlineto{\pgfqpoint{5.995070in}{3.643052in}}%
\pgfpathlineto{\pgfqpoint{6.002143in}{3.642386in}}%
\pgfpathlineto{\pgfqpoint{6.009208in}{3.641692in}}%
\pgfpathlineto{\pgfqpoint{6.016264in}{3.640975in}}%
\pgfpathlineto{\pgfqpoint{6.023313in}{3.640242in}}%
\pgfpathlineto{\pgfqpoint{6.008165in}{3.624280in}}%
\pgfpathlineto{\pgfqpoint{5.993041in}{3.608501in}}%
\pgfpathlineto{\pgfqpoint{5.977940in}{3.592904in}}%
\pgfpathlineto{\pgfqpoint{5.970866in}{3.593192in}}%
\pgfpathlineto{\pgfqpoint{5.963783in}{3.593469in}}%
\pgfpathlineto{\pgfqpoint{5.956693in}{3.593729in}}%
\pgfpathlineto{\pgfqpoint{5.949594in}{3.593965in}}%
\pgfpathclose%
\pgfusepath{fill}%
\end{pgfscope}%
\begin{pgfscope}%
\pgfpathrectangle{\pgfqpoint{1.150000in}{0.150000in}}{\pgfqpoint{5.700000in}{5.700000in}}%
\pgfusepath{clip}%
\pgfsetbuttcap%
\pgfsetroundjoin%
\definecolor{currentfill}{rgb}{0.304148,0.764704,0.419943}%
\pgfsetfillcolor{currentfill}%
\pgfsetfillopacity{0.800000}%
\pgfsetlinewidth{0.000000pt}%
\definecolor{currentstroke}{rgb}{0.000000,0.000000,0.000000}%
\pgfsetstrokecolor{currentstroke}%
\pgfsetdash{}{0pt}%
\pgfpathmoveto{\pgfqpoint{5.860685in}{3.529494in}}%
\pgfpathlineto{\pgfqpoint{5.875756in}{3.545484in}}%
\pgfpathlineto{\pgfqpoint{5.890851in}{3.561658in}}%
\pgfpathlineto{\pgfqpoint{5.905970in}{3.578016in}}%
\pgfpathlineto{\pgfqpoint{5.921113in}{3.594560in}}%
\pgfpathlineto{\pgfqpoint{5.928246in}{3.594476in}}%
\pgfpathlineto{\pgfqpoint{5.935371in}{3.594345in}}%
\pgfpathlineto{\pgfqpoint{5.942487in}{3.594173in}}%
\pgfpathlineto{\pgfqpoint{5.949594in}{3.593965in}}%
\pgfpathlineto{\pgfqpoint{5.934484in}{3.577971in}}%
\pgfpathlineto{\pgfqpoint{5.919397in}{3.562160in}}%
\pgfpathlineto{\pgfqpoint{5.904333in}{3.546533in}}%
\pgfpathlineto{\pgfqpoint{5.889293in}{3.531088in}}%
\pgfpathlineto{\pgfqpoint{5.882153in}{3.530737in}}%
\pgfpathlineto{\pgfqpoint{5.875006in}{3.530358in}}%
\pgfpathlineto{\pgfqpoint{5.867849in}{3.529946in}}%
\pgfpathlineto{\pgfqpoint{5.860685in}{3.529494in}}%
\pgfpathclose%
\pgfusepath{fill}%
\end{pgfscope}%
\begin{pgfscope}%
\pgfpathrectangle{\pgfqpoint{1.150000in}{0.150000in}}{\pgfqpoint{5.700000in}{5.700000in}}%
\pgfusepath{clip}%
\pgfsetbuttcap%
\pgfsetroundjoin%
\definecolor{currentfill}{rgb}{0.279574,0.170599,0.479997}%
\pgfsetfillcolor{currentfill}%
\pgfsetfillopacity{0.800000}%
\pgfsetlinewidth{0.000000pt}%
\definecolor{currentstroke}{rgb}{0.000000,0.000000,0.000000}%
\pgfsetstrokecolor{currentstroke}%
\pgfsetdash{}{0pt}%
\pgfpathmoveto{\pgfqpoint{4.012437in}{1.779398in}}%
\pgfpathlineto{\pgfqpoint{4.026413in}{1.783968in}}%
\pgfpathlineto{\pgfqpoint{4.040400in}{1.788722in}}%
\pgfpathlineto{\pgfqpoint{4.054399in}{1.793661in}}%
\pgfpathlineto{\pgfqpoint{4.068409in}{1.798784in}}%
\pgfpathlineto{\pgfqpoint{4.076503in}{1.812307in}}%
\pgfpathlineto{\pgfqpoint{4.084593in}{1.825800in}}%
\pgfpathlineto{\pgfqpoint{4.092678in}{1.839263in}}%
\pgfpathlineto{\pgfqpoint{4.100758in}{1.852690in}}%
\pgfpathlineto{\pgfqpoint{4.086751in}{1.847220in}}%
\pgfpathlineto{\pgfqpoint{4.072755in}{1.841935in}}%
\pgfpathlineto{\pgfqpoint{4.058770in}{1.836835in}}%
\pgfpathlineto{\pgfqpoint{4.044797in}{1.831919in}}%
\pgfpathlineto{\pgfqpoint{4.036714in}{1.818826in}}%
\pgfpathlineto{\pgfqpoint{4.028626in}{1.805706in}}%
\pgfpathlineto{\pgfqpoint{4.020534in}{1.792562in}}%
\pgfpathlineto{\pgfqpoint{4.012437in}{1.779398in}}%
\pgfpathclose%
\pgfusepath{fill}%
\end{pgfscope}%
\begin{pgfscope}%
\pgfpathrectangle{\pgfqpoint{1.150000in}{0.150000in}}{\pgfqpoint{5.700000in}{5.700000in}}%
\pgfusepath{clip}%
\pgfsetbuttcap%
\pgfsetroundjoin%
\definecolor{currentfill}{rgb}{0.280267,0.073417,0.397163}%
\pgfsetfillcolor{currentfill}%
\pgfsetfillopacity{0.800000}%
\pgfsetlinewidth{0.000000pt}%
\definecolor{currentstroke}{rgb}{0.000000,0.000000,0.000000}%
\pgfsetstrokecolor{currentstroke}%
\pgfsetdash{}{0pt}%
\pgfpathmoveto{\pgfqpoint{2.904080in}{1.622867in}}%
\pgfpathlineto{\pgfqpoint{2.917992in}{1.611162in}}%
\pgfpathlineto{\pgfqpoint{2.931902in}{1.599679in}}%
\pgfpathlineto{\pgfqpoint{2.945810in}{1.588416in}}%
\pgfpathlineto{\pgfqpoint{2.959715in}{1.577372in}}%
\pgfpathlineto{\pgfqpoint{2.968371in}{1.579067in}}%
\pgfpathlineto{\pgfqpoint{2.977012in}{1.581042in}}%
\pgfpathlineto{\pgfqpoint{2.985639in}{1.583291in}}%
\pgfpathlineto{\pgfqpoint{2.994251in}{1.585806in}}%
\pgfpathlineto{\pgfqpoint{2.980384in}{1.596181in}}%
\pgfpathlineto{\pgfqpoint{2.966515in}{1.606775in}}%
\pgfpathlineto{\pgfqpoint{2.952644in}{1.617589in}}%
\pgfpathlineto{\pgfqpoint{2.938771in}{1.628624in}}%
\pgfpathlineto{\pgfqpoint{2.930121in}{1.626764in}}%
\pgfpathlineto{\pgfqpoint{2.921456in}{1.625181in}}%
\pgfpathlineto{\pgfqpoint{2.912776in}{1.623879in}}%
\pgfpathlineto{\pgfqpoint{2.904080in}{1.622867in}}%
\pgfpathclose%
\pgfusepath{fill}%
\end{pgfscope}%
\begin{pgfscope}%
\pgfpathrectangle{\pgfqpoint{1.150000in}{0.150000in}}{\pgfqpoint{5.700000in}{5.700000in}}%
\pgfusepath{clip}%
\pgfsetbuttcap%
\pgfsetroundjoin%
\definecolor{currentfill}{rgb}{0.257322,0.256130,0.526563}%
\pgfsetfillcolor{currentfill}%
\pgfsetfillopacity{0.800000}%
\pgfsetlinewidth{0.000000pt}%
\definecolor{currentstroke}{rgb}{0.000000,0.000000,0.000000}%
\pgfsetstrokecolor{currentstroke}%
\pgfsetdash{}{0pt}%
\pgfpathmoveto{\pgfqpoint{4.221376in}{1.984638in}}%
\pgfpathlineto{\pgfqpoint{4.235442in}{1.991628in}}%
\pgfpathlineto{\pgfqpoint{4.249522in}{1.998803in}}%
\pgfpathlineto{\pgfqpoint{4.263614in}{2.006162in}}%
\pgfpathlineto{\pgfqpoint{4.277720in}{2.013706in}}%
\pgfpathlineto{\pgfqpoint{4.285758in}{2.027223in}}%
\pgfpathlineto{\pgfqpoint{4.293790in}{2.040664in}}%
\pgfpathlineto{\pgfqpoint{4.301818in}{2.054025in}}%
\pgfpathlineto{\pgfqpoint{4.309841in}{2.067305in}}%
\pgfpathlineto{\pgfqpoint{4.295735in}{2.059510in}}%
\pgfpathlineto{\pgfqpoint{4.281643in}{2.051899in}}%
\pgfpathlineto{\pgfqpoint{4.267564in}{2.044473in}}%
\pgfpathlineto{\pgfqpoint{4.253499in}{2.037232in}}%
\pgfpathlineto{\pgfqpoint{4.245475in}{2.024191in}}%
\pgfpathlineto{\pgfqpoint{4.237447in}{2.011077in}}%
\pgfpathlineto{\pgfqpoint{4.229414in}{1.997892in}}%
\pgfpathlineto{\pgfqpoint{4.221376in}{1.984638in}}%
\pgfpathclose%
\pgfusepath{fill}%
\end{pgfscope}%
\begin{pgfscope}%
\pgfpathrectangle{\pgfqpoint{1.150000in}{0.150000in}}{\pgfqpoint{5.700000in}{5.700000in}}%
\pgfusepath{clip}%
\pgfsetbuttcap%
\pgfsetroundjoin%
\definecolor{currentfill}{rgb}{0.154815,0.493313,0.557840}%
\pgfsetfillcolor{currentfill}%
\pgfsetfillopacity{0.800000}%
\pgfsetlinewidth{0.000000pt}%
\definecolor{currentstroke}{rgb}{0.000000,0.000000,0.000000}%
\pgfsetstrokecolor{currentstroke}%
\pgfsetdash{}{0pt}%
\pgfpathmoveto{\pgfqpoint{2.093768in}{2.757591in}}%
\pgfpathlineto{\pgfqpoint{2.108219in}{2.729954in}}%
\pgfpathlineto{\pgfqpoint{2.122650in}{2.702672in}}%
\pgfpathlineto{\pgfqpoint{2.137064in}{2.675740in}}%
\pgfpathlineto{\pgfqpoint{2.151459in}{2.649155in}}%
\pgfpathlineto{\pgfqpoint{2.160830in}{2.641829in}}%
\pgfpathlineto{\pgfqpoint{2.170175in}{2.634897in}}%
\pgfpathlineto{\pgfqpoint{2.179494in}{2.628352in}}%
\pgfpathlineto{\pgfqpoint{2.188788in}{2.622185in}}%
\pgfpathlineto{\pgfqpoint{2.174460in}{2.648069in}}%
\pgfpathlineto{\pgfqpoint{2.160115in}{2.674298in}}%
\pgfpathlineto{\pgfqpoint{2.145752in}{2.700876in}}%
\pgfpathlineto{\pgfqpoint{2.131370in}{2.727805in}}%
\pgfpathlineto{\pgfqpoint{2.122010in}{2.734660in}}%
\pgfpathlineto{\pgfqpoint{2.112623in}{2.741905in}}%
\pgfpathlineto{\pgfqpoint{2.103209in}{2.749546in}}%
\pgfpathlineto{\pgfqpoint{2.093768in}{2.757591in}}%
\pgfpathclose%
\pgfusepath{fill}%
\end{pgfscope}%
\begin{pgfscope}%
\pgfpathrectangle{\pgfqpoint{1.150000in}{0.150000in}}{\pgfqpoint{5.700000in}{5.700000in}}%
\pgfusepath{clip}%
\pgfsetbuttcap%
\pgfsetroundjoin%
\definecolor{currentfill}{rgb}{0.206756,0.371758,0.553117}%
\pgfsetfillcolor{currentfill}%
\pgfsetfillopacity{0.800000}%
\pgfsetlinewidth{0.000000pt}%
\definecolor{currentstroke}{rgb}{0.000000,0.000000,0.000000}%
\pgfsetstrokecolor{currentstroke}%
\pgfsetdash{}{0pt}%
\pgfpathmoveto{\pgfqpoint{2.285984in}{2.380597in}}%
\pgfpathlineto{\pgfqpoint{2.300241in}{2.357507in}}%
\pgfpathlineto{\pgfqpoint{2.314484in}{2.334722in}}%
\pgfpathlineto{\pgfqpoint{2.328714in}{2.312239in}}%
\pgfpathlineto{\pgfqpoint{2.342931in}{2.290056in}}%
\pgfpathlineto{\pgfqpoint{2.352141in}{2.284050in}}%
\pgfpathlineto{\pgfqpoint{2.361327in}{2.278425in}}%
\pgfpathlineto{\pgfqpoint{2.370489in}{2.273175in}}%
\pgfpathlineto{\pgfqpoint{2.379628in}{2.268292in}}%
\pgfpathlineto{\pgfqpoint{2.365473in}{2.289762in}}%
\pgfpathlineto{\pgfqpoint{2.351305in}{2.311530in}}%
\pgfpathlineto{\pgfqpoint{2.337124in}{2.333599in}}%
\pgfpathlineto{\pgfqpoint{2.322930in}{2.355972in}}%
\pgfpathlineto{\pgfqpoint{2.313731in}{2.361555in}}%
\pgfpathlineto{\pgfqpoint{2.304506in}{2.367515in}}%
\pgfpathlineto{\pgfqpoint{2.295258in}{2.373860in}}%
\pgfpathlineto{\pgfqpoint{2.285984in}{2.380597in}}%
\pgfpathclose%
\pgfusepath{fill}%
\end{pgfscope}%
\begin{pgfscope}%
\pgfpathrectangle{\pgfqpoint{1.150000in}{0.150000in}}{\pgfqpoint{5.700000in}{5.700000in}}%
\pgfusepath{clip}%
\pgfsetbuttcap%
\pgfsetroundjoin%
\definecolor{currentfill}{rgb}{0.120092,0.600104,0.542530}%
\pgfsetfillcolor{currentfill}%
\pgfsetfillopacity{0.800000}%
\pgfsetlinewidth{0.000000pt}%
\definecolor{currentstroke}{rgb}{0.000000,0.000000,0.000000}%
\pgfsetstrokecolor{currentstroke}%
\pgfsetdash{}{0pt}%
\pgfpathmoveto{\pgfqpoint{5.177919in}{2.978208in}}%
\pgfpathlineto{\pgfqpoint{5.192551in}{2.992335in}}%
\pgfpathlineto{\pgfqpoint{5.207204in}{3.006648in}}%
\pgfpathlineto{\pgfqpoint{5.221877in}{3.021146in}}%
\pgfpathlineto{\pgfqpoint{5.236571in}{3.035830in}}%
\pgfpathlineto{\pgfqpoint{5.244197in}{3.042214in}}%
\pgfpathlineto{\pgfqpoint{5.251814in}{3.048457in}}%
\pgfpathlineto{\pgfqpoint{5.259421in}{3.054560in}}%
\pgfpathlineto{\pgfqpoint{5.267020in}{3.060526in}}%
\pgfpathlineto{\pgfqpoint{5.252337in}{3.046068in}}%
\pgfpathlineto{\pgfqpoint{5.237675in}{3.031795in}}%
\pgfpathlineto{\pgfqpoint{5.223034in}{3.017707in}}%
\pgfpathlineto{\pgfqpoint{5.208412in}{3.003804in}}%
\pgfpathlineto{\pgfqpoint{5.200802in}{2.997600in}}%
\pgfpathlineto{\pgfqpoint{5.193183in}{2.991268in}}%
\pgfpathlineto{\pgfqpoint{5.185556in}{2.984805in}}%
\pgfpathlineto{\pgfqpoint{5.177919in}{2.978208in}}%
\pgfpathclose%
\pgfusepath{fill}%
\end{pgfscope}%
\begin{pgfscope}%
\pgfpathrectangle{\pgfqpoint{1.150000in}{0.150000in}}{\pgfqpoint{5.700000in}{5.700000in}}%
\pgfusepath{clip}%
\pgfsetbuttcap%
\pgfsetroundjoin%
\definecolor{currentfill}{rgb}{0.274952,0.037752,0.364543}%
\pgfsetfillcolor{currentfill}%
\pgfsetfillopacity{0.800000}%
\pgfsetlinewidth{0.000000pt}%
\definecolor{currentstroke}{rgb}{0.000000,0.000000,0.000000}%
\pgfsetstrokecolor{currentstroke}%
\pgfsetdash{}{0pt}%
\pgfpathmoveto{\pgfqpoint{3.626642in}{1.504726in}}%
\pgfpathlineto{\pgfqpoint{3.640515in}{1.504116in}}%
\pgfpathlineto{\pgfqpoint{3.654394in}{1.503694in}}%
\pgfpathlineto{\pgfqpoint{3.668281in}{1.503460in}}%
\pgfpathlineto{\pgfqpoint{3.682174in}{1.503414in}}%
\pgfpathlineto{\pgfqpoint{3.690395in}{1.514658in}}%
\pgfpathlineto{\pgfqpoint{3.698609in}{1.525986in}}%
\pgfpathlineto{\pgfqpoint{3.706817in}{1.537392in}}%
\pgfpathlineto{\pgfqpoint{3.715020in}{1.548871in}}%
\pgfpathlineto{\pgfqpoint{3.701138in}{1.548416in}}%
\pgfpathlineto{\pgfqpoint{3.687263in}{1.548148in}}%
\pgfpathlineto{\pgfqpoint{3.673395in}{1.548069in}}%
\pgfpathlineto{\pgfqpoint{3.659535in}{1.548179in}}%
\pgfpathlineto{\pgfqpoint{3.651321in}{1.537189in}}%
\pgfpathlineto{\pgfqpoint{3.643101in}{1.526280in}}%
\pgfpathlineto{\pgfqpoint{3.634875in}{1.515457in}}%
\pgfpathlineto{\pgfqpoint{3.626642in}{1.504726in}}%
\pgfpathclose%
\pgfusepath{fill}%
\end{pgfscope}%
\begin{pgfscope}%
\pgfpathrectangle{\pgfqpoint{1.150000in}{0.150000in}}{\pgfqpoint{5.700000in}{5.700000in}}%
\pgfusepath{clip}%
\pgfsetbuttcap%
\pgfsetroundjoin%
\definecolor{currentfill}{rgb}{0.278791,0.062145,0.386592}%
\pgfsetfillcolor{currentfill}%
\pgfsetfillopacity{0.800000}%
\pgfsetlinewidth{0.000000pt}%
\definecolor{currentstroke}{rgb}{0.000000,0.000000,0.000000}%
\pgfsetstrokecolor{currentstroke}%
\pgfsetdash{}{0pt}%
\pgfpathmoveto{\pgfqpoint{3.715020in}{1.548871in}}%
\pgfpathlineto{\pgfqpoint{3.728910in}{1.549514in}}%
\pgfpathlineto{\pgfqpoint{3.742808in}{1.550344in}}%
\pgfpathlineto{\pgfqpoint{3.756713in}{1.551361in}}%
\pgfpathlineto{\pgfqpoint{3.770627in}{1.552565in}}%
\pgfpathlineto{\pgfqpoint{3.778815in}{1.564595in}}%
\pgfpathlineto{\pgfqpoint{3.786997in}{1.576683in}}%
\pgfpathlineto{\pgfqpoint{3.795173in}{1.588823in}}%
\pgfpathlineto{\pgfqpoint{3.803344in}{1.601010in}}%
\pgfpathlineto{\pgfqpoint{3.789439in}{1.599336in}}%
\pgfpathlineto{\pgfqpoint{3.775542in}{1.597849in}}%
\pgfpathlineto{\pgfqpoint{3.761654in}{1.596549in}}%
\pgfpathlineto{\pgfqpoint{3.747774in}{1.595436in}}%
\pgfpathlineto{\pgfqpoint{3.739594in}{1.583707in}}%
\pgfpathlineto{\pgfqpoint{3.731408in}{1.572033in}}%
\pgfpathlineto{\pgfqpoint{3.723217in}{1.560420in}}%
\pgfpathlineto{\pgfqpoint{3.715020in}{1.548871in}}%
\pgfpathclose%
\pgfusepath{fill}%
\end{pgfscope}%
\begin{pgfscope}%
\pgfpathrectangle{\pgfqpoint{1.150000in}{0.150000in}}{\pgfqpoint{5.700000in}{5.700000in}}%
\pgfusepath{clip}%
\pgfsetbuttcap%
\pgfsetroundjoin%
\definecolor{currentfill}{rgb}{0.165117,0.467423,0.558141}%
\pgfsetfillcolor{currentfill}%
\pgfsetfillopacity{0.800000}%
\pgfsetlinewidth{0.000000pt}%
\definecolor{currentstroke}{rgb}{0.000000,0.000000,0.000000}%
\pgfsetstrokecolor{currentstroke}%
\pgfsetdash{}{0pt}%
\pgfpathmoveto{\pgfqpoint{4.759966in}{2.563649in}}%
\pgfpathlineto{\pgfqpoint{4.774332in}{2.575424in}}%
\pgfpathlineto{\pgfqpoint{4.788716in}{2.587383in}}%
\pgfpathlineto{\pgfqpoint{4.803118in}{2.599528in}}%
\pgfpathlineto{\pgfqpoint{4.817537in}{2.611857in}}%
\pgfpathlineto{\pgfqpoint{4.825386in}{2.622253in}}%
\pgfpathlineto{\pgfqpoint{4.833228in}{2.632501in}}%
\pgfpathlineto{\pgfqpoint{4.841063in}{2.642605in}}%
\pgfpathlineto{\pgfqpoint{4.848890in}{2.652563in}}%
\pgfpathlineto{\pgfqpoint{4.834474in}{2.640248in}}%
\pgfpathlineto{\pgfqpoint{4.820075in}{2.628119in}}%
\pgfpathlineto{\pgfqpoint{4.805694in}{2.616174in}}%
\pgfpathlineto{\pgfqpoint{4.791331in}{2.604414in}}%
\pgfpathlineto{\pgfqpoint{4.783500in}{2.594428in}}%
\pgfpathlineto{\pgfqpoint{4.775662in}{2.584306in}}%
\pgfpathlineto{\pgfqpoint{4.767818in}{2.574046in}}%
\pgfpathlineto{\pgfqpoint{4.759966in}{2.563649in}}%
\pgfpathclose%
\pgfusepath{fill}%
\end{pgfscope}%
\begin{pgfscope}%
\pgfpathrectangle{\pgfqpoint{1.150000in}{0.150000in}}{\pgfqpoint{5.700000in}{5.700000in}}%
\pgfusepath{clip}%
\pgfsetbuttcap%
\pgfsetroundjoin%
\definecolor{currentfill}{rgb}{0.153894,0.680203,0.504172}%
\pgfsetfillcolor{currentfill}%
\pgfsetfillopacity{0.800000}%
\pgfsetlinewidth{0.000000pt}%
\definecolor{currentstroke}{rgb}{0.000000,0.000000,0.000000}%
\pgfsetstrokecolor{currentstroke}%
\pgfsetdash{}{0pt}%
\pgfpathmoveto{\pgfqpoint{5.475355in}{3.237832in}}%
\pgfpathlineto{\pgfqpoint{5.490184in}{3.253067in}}%
\pgfpathlineto{\pgfqpoint{5.505034in}{3.268488in}}%
\pgfpathlineto{\pgfqpoint{5.519907in}{3.284094in}}%
\pgfpathlineto{\pgfqpoint{5.534801in}{3.299886in}}%
\pgfpathlineto{\pgfqpoint{5.542229in}{3.303310in}}%
\pgfpathlineto{\pgfqpoint{5.549648in}{3.306621in}}%
\pgfpathlineto{\pgfqpoint{5.557056in}{3.309823in}}%
\pgfpathlineto{\pgfqpoint{5.564455in}{3.312919in}}%
\pgfpathlineto{\pgfqpoint{5.549580in}{3.297496in}}%
\pgfpathlineto{\pgfqpoint{5.534727in}{3.282258in}}%
\pgfpathlineto{\pgfqpoint{5.519896in}{3.267205in}}%
\pgfpathlineto{\pgfqpoint{5.505086in}{3.252336in}}%
\pgfpathlineto{\pgfqpoint{5.497667in}{3.248860in}}%
\pgfpathlineto{\pgfqpoint{5.490239in}{3.245286in}}%
\pgfpathlineto{\pgfqpoint{5.482802in}{3.241612in}}%
\pgfpathlineto{\pgfqpoint{5.475355in}{3.237832in}}%
\pgfpathclose%
\pgfusepath{fill}%
\end{pgfscope}%
\begin{pgfscope}%
\pgfpathrectangle{\pgfqpoint{1.150000in}{0.150000in}}{\pgfqpoint{5.700000in}{5.700000in}}%
\pgfusepath{clip}%
\pgfsetbuttcap%
\pgfsetroundjoin%
\definecolor{currentfill}{rgb}{0.267004,0.004874,0.329415}%
\pgfsetfillcolor{currentfill}%
\pgfsetfillopacity{0.800000}%
\pgfsetlinewidth{0.000000pt}%
\definecolor{currentstroke}{rgb}{0.000000,0.000000,0.000000}%
\pgfsetstrokecolor{currentstroke}%
\pgfsetdash{}{0pt}%
\pgfpathmoveto{\pgfqpoint{3.305257in}{1.447805in}}%
\pgfpathlineto{\pgfqpoint{3.319108in}{1.442414in}}%
\pgfpathlineto{\pgfqpoint{3.332962in}{1.437222in}}%
\pgfpathlineto{\pgfqpoint{3.346819in}{1.432226in}}%
\pgfpathlineto{\pgfqpoint{3.360680in}{1.427427in}}%
\pgfpathlineto{\pgfqpoint{3.369057in}{1.434837in}}%
\pgfpathlineto{\pgfqpoint{3.377426in}{1.442427in}}%
\pgfpathlineto{\pgfqpoint{3.385785in}{1.450190in}}%
\pgfpathlineto{\pgfqpoint{3.394136in}{1.458121in}}%
\pgfpathlineto{\pgfqpoint{3.380298in}{1.462324in}}%
\pgfpathlineto{\pgfqpoint{3.366463in}{1.466724in}}%
\pgfpathlineto{\pgfqpoint{3.352632in}{1.471321in}}%
\pgfpathlineto{\pgfqpoint{3.338804in}{1.476116in}}%
\pgfpathlineto{\pgfqpoint{3.330432in}{1.468769in}}%
\pgfpathlineto{\pgfqpoint{3.322050in}{1.461597in}}%
\pgfpathlineto{\pgfqpoint{3.313658in}{1.454607in}}%
\pgfpathlineto{\pgfqpoint{3.305257in}{1.447805in}}%
\pgfpathclose%
\pgfusepath{fill}%
\end{pgfscope}%
\begin{pgfscope}%
\pgfpathrectangle{\pgfqpoint{1.150000in}{0.150000in}}{\pgfqpoint{5.700000in}{5.700000in}}%
\pgfusepath{clip}%
\pgfsetbuttcap%
\pgfsetroundjoin%
\definecolor{currentfill}{rgb}{0.137770,0.537492,0.554906}%
\pgfsetfillcolor{currentfill}%
\pgfsetfillopacity{0.800000}%
\pgfsetlinewidth{0.000000pt}%
\definecolor{currentstroke}{rgb}{0.000000,0.000000,0.000000}%
\pgfsetstrokecolor{currentstroke}%
\pgfsetdash{}{0pt}%
\pgfpathmoveto{\pgfqpoint{4.969054in}{2.777700in}}%
\pgfpathlineto{\pgfqpoint{4.983554in}{2.790808in}}%
\pgfpathlineto{\pgfqpoint{4.998072in}{2.804101in}}%
\pgfpathlineto{\pgfqpoint{5.012610in}{2.817579in}}%
\pgfpathlineto{\pgfqpoint{5.027167in}{2.831242in}}%
\pgfpathlineto{\pgfqpoint{5.034915in}{2.839729in}}%
\pgfpathlineto{\pgfqpoint{5.042655in}{2.848066in}}%
\pgfpathlineto{\pgfqpoint{5.050386in}{2.856253in}}%
\pgfpathlineto{\pgfqpoint{5.058109in}{2.864293in}}%
\pgfpathlineto{\pgfqpoint{5.043558in}{2.850749in}}%
\pgfpathlineto{\pgfqpoint{5.029027in}{2.837390in}}%
\pgfpathlineto{\pgfqpoint{5.014514in}{2.824216in}}%
\pgfpathlineto{\pgfqpoint{5.000021in}{2.811227in}}%
\pgfpathlineto{\pgfqpoint{4.992291in}{2.803055in}}%
\pgfpathlineto{\pgfqpoint{4.984553in}{2.794745in}}%
\pgfpathlineto{\pgfqpoint{4.976808in}{2.786294in}}%
\pgfpathlineto{\pgfqpoint{4.969054in}{2.777700in}}%
\pgfpathclose%
\pgfusepath{fill}%
\end{pgfscope}%
\begin{pgfscope}%
\pgfpathrectangle{\pgfqpoint{1.150000in}{0.150000in}}{\pgfqpoint{5.700000in}{5.700000in}}%
\pgfusepath{clip}%
\pgfsetbuttcap%
\pgfsetroundjoin%
\definecolor{currentfill}{rgb}{0.271305,0.019942,0.347269}%
\pgfsetfillcolor{currentfill}%
\pgfsetfillopacity{0.800000}%
\pgfsetlinewidth{0.000000pt}%
\definecolor{currentstroke}{rgb}{0.000000,0.000000,0.000000}%
\pgfsetstrokecolor{currentstroke}%
\pgfsetdash{}{0pt}%
\pgfpathmoveto{\pgfqpoint{3.538163in}{1.469277in}}%
\pgfpathlineto{\pgfqpoint{3.552025in}{1.467376in}}%
\pgfpathlineto{\pgfqpoint{3.565893in}{1.465665in}}%
\pgfpathlineto{\pgfqpoint{3.579767in}{1.464143in}}%
\pgfpathlineto{\pgfqpoint{3.593647in}{1.462812in}}%
\pgfpathlineto{\pgfqpoint{3.601906in}{1.473129in}}%
\pgfpathlineto{\pgfqpoint{3.610158in}{1.483557in}}%
\pgfpathlineto{\pgfqpoint{3.618403in}{1.494091in}}%
\pgfpathlineto{\pgfqpoint{3.626642in}{1.504726in}}%
\pgfpathlineto{\pgfqpoint{3.612776in}{1.505526in}}%
\pgfpathlineto{\pgfqpoint{3.598917in}{1.506515in}}%
\pgfpathlineto{\pgfqpoint{3.585064in}{1.507694in}}%
\pgfpathlineto{\pgfqpoint{3.571217in}{1.509064in}}%
\pgfpathlineto{\pgfqpoint{3.562964in}{1.498949in}}%
\pgfpathlineto{\pgfqpoint{3.554704in}{1.488942in}}%
\pgfpathlineto{\pgfqpoint{3.546437in}{1.479050in}}%
\pgfpathlineto{\pgfqpoint{3.538163in}{1.469277in}}%
\pgfpathclose%
\pgfusepath{fill}%
\end{pgfscope}%
\begin{pgfscope}%
\pgfpathrectangle{\pgfqpoint{1.150000in}{0.150000in}}{\pgfqpoint{5.700000in}{5.700000in}}%
\pgfusepath{clip}%
\pgfsetbuttcap%
\pgfsetroundjoin%
\definecolor{currentfill}{rgb}{0.269944,0.014625,0.341379}%
\pgfsetfillcolor{currentfill}%
\pgfsetfillopacity{0.800000}%
\pgfsetlinewidth{0.000000pt}%
\definecolor{currentstroke}{rgb}{0.000000,0.000000,0.000000}%
\pgfsetstrokecolor{currentstroke}%
\pgfsetdash{}{0pt}%
\pgfpathmoveto{\pgfqpoint{3.160603in}{1.477958in}}%
\pgfpathlineto{\pgfqpoint{3.174466in}{1.470328in}}%
\pgfpathlineto{\pgfqpoint{3.188331in}{1.462902in}}%
\pgfpathlineto{\pgfqpoint{3.202197in}{1.455680in}}%
\pgfpathlineto{\pgfqpoint{3.216064in}{1.448660in}}%
\pgfpathlineto{\pgfqpoint{3.224534in}{1.454011in}}%
\pgfpathlineto{\pgfqpoint{3.232992in}{1.459581in}}%
\pgfpathlineto{\pgfqpoint{3.241440in}{1.465366in}}%
\pgfpathlineto{\pgfqpoint{3.249877in}{1.471359in}}%
\pgfpathlineto{\pgfqpoint{3.236038in}{1.477750in}}%
\pgfpathlineto{\pgfqpoint{3.222200in}{1.484343in}}%
\pgfpathlineto{\pgfqpoint{3.208364in}{1.491139in}}%
\pgfpathlineto{\pgfqpoint{3.194530in}{1.498140in}}%
\pgfpathlineto{\pgfqpoint{3.186065in}{1.492764in}}%
\pgfpathlineto{\pgfqpoint{3.177589in}{1.487604in}}%
\pgfpathlineto{\pgfqpoint{3.169102in}{1.482667in}}%
\pgfpathlineto{\pgfqpoint{3.160603in}{1.477958in}}%
\pgfpathclose%
\pgfusepath{fill}%
\end{pgfscope}%
\begin{pgfscope}%
\pgfpathrectangle{\pgfqpoint{1.150000in}{0.150000in}}{\pgfqpoint{5.700000in}{5.700000in}}%
\pgfusepath{clip}%
\pgfsetbuttcap%
\pgfsetroundjoin%
\definecolor{currentfill}{rgb}{0.218130,0.347432,0.550038}%
\pgfsetfillcolor{currentfill}%
\pgfsetfillopacity{0.800000}%
\pgfsetlinewidth{0.000000pt}%
\definecolor{currentstroke}{rgb}{0.000000,0.000000,0.000000}%
\pgfsetstrokecolor{currentstroke}%
\pgfsetdash{}{0pt}%
\pgfpathmoveto{\pgfqpoint{4.430407in}{2.205044in}}%
\pgfpathlineto{\pgfqpoint{4.444584in}{2.214167in}}%
\pgfpathlineto{\pgfqpoint{4.458775in}{2.223474in}}%
\pgfpathlineto{\pgfqpoint{4.472982in}{2.232966in}}%
\pgfpathlineto{\pgfqpoint{4.487205in}{2.242642in}}%
\pgfpathlineto{\pgfqpoint{4.495182in}{2.255444in}}%
\pgfpathlineto{\pgfqpoint{4.503154in}{2.268131in}}%
\pgfpathlineto{\pgfqpoint{4.511120in}{2.280703in}}%
\pgfpathlineto{\pgfqpoint{4.519080in}{2.293159in}}%
\pgfpathlineto{\pgfqpoint{4.504858in}{2.283328in}}%
\pgfpathlineto{\pgfqpoint{4.490651in}{2.273683in}}%
\pgfpathlineto{\pgfqpoint{4.476459in}{2.264222in}}%
\pgfpathlineto{\pgfqpoint{4.462283in}{2.254945in}}%
\pgfpathlineto{\pgfqpoint{4.454322in}{2.242631in}}%
\pgfpathlineto{\pgfqpoint{4.446356in}{2.230208in}}%
\pgfpathlineto{\pgfqpoint{4.438384in}{2.217679in}}%
\pgfpathlineto{\pgfqpoint{4.430407in}{2.205044in}}%
\pgfpathclose%
\pgfusepath{fill}%
\end{pgfscope}%
\begin{pgfscope}%
\pgfpathrectangle{\pgfqpoint{1.150000in}{0.150000in}}{\pgfqpoint{5.700000in}{5.700000in}}%
\pgfusepath{clip}%
\pgfsetbuttcap%
\pgfsetroundjoin%
\definecolor{currentfill}{rgb}{0.281924,0.089666,0.412415}%
\pgfsetfillcolor{currentfill}%
\pgfsetfillopacity{0.800000}%
\pgfsetlinewidth{0.000000pt}%
\definecolor{currentstroke}{rgb}{0.000000,0.000000,0.000000}%
\pgfsetstrokecolor{currentstroke}%
\pgfsetdash{}{0pt}%
\pgfpathmoveto{\pgfqpoint{3.803344in}{1.601010in}}%
\pgfpathlineto{\pgfqpoint{3.817258in}{1.602871in}}%
\pgfpathlineto{\pgfqpoint{3.831181in}{1.604917in}}%
\pgfpathlineto{\pgfqpoint{3.845113in}{1.607149in}}%
\pgfpathlineto{\pgfqpoint{3.859053in}{1.609567in}}%
\pgfpathlineto{\pgfqpoint{3.867212in}{1.622249in}}%
\pgfpathlineto{\pgfqpoint{3.875366in}{1.634964in}}%
\pgfpathlineto{\pgfqpoint{3.883515in}{1.647705in}}%
\pgfpathlineto{\pgfqpoint{3.891659in}{1.660470in}}%
\pgfpathlineto{\pgfqpoint{3.877725in}{1.657613in}}%
\pgfpathlineto{\pgfqpoint{3.863800in}{1.654941in}}%
\pgfpathlineto{\pgfqpoint{3.849884in}{1.652455in}}%
\pgfpathlineto{\pgfqpoint{3.835978in}{1.650155in}}%
\pgfpathlineto{\pgfqpoint{3.827827in}{1.637818in}}%
\pgfpathlineto{\pgfqpoint{3.819671in}{1.625512in}}%
\pgfpathlineto{\pgfqpoint{3.811510in}{1.613242in}}%
\pgfpathlineto{\pgfqpoint{3.803344in}{1.601010in}}%
\pgfpathclose%
\pgfusepath{fill}%
\end{pgfscope}%
\begin{pgfscope}%
\pgfpathrectangle{\pgfqpoint{1.150000in}{0.150000in}}{\pgfqpoint{5.700000in}{5.700000in}}%
\pgfusepath{clip}%
\pgfsetbuttcap%
\pgfsetroundjoin%
\definecolor{currentfill}{rgb}{0.277941,0.056324,0.381191}%
\pgfsetfillcolor{currentfill}%
\pgfsetfillopacity{0.800000}%
\pgfsetlinewidth{0.000000pt}%
\definecolor{currentstroke}{rgb}{0.000000,0.000000,0.000000}%
\pgfsetstrokecolor{currentstroke}%
\pgfsetdash{}{0pt}%
\pgfpathmoveto{\pgfqpoint{2.959715in}{1.577372in}}%
\pgfpathlineto{\pgfqpoint{2.973618in}{1.566547in}}%
\pgfpathlineto{\pgfqpoint{2.987520in}{1.555938in}}%
\pgfpathlineto{\pgfqpoint{3.001420in}{1.545545in}}%
\pgfpathlineto{\pgfqpoint{3.015319in}{1.535366in}}%
\pgfpathlineto{\pgfqpoint{3.023937in}{1.537741in}}%
\pgfpathlineto{\pgfqpoint{3.032541in}{1.540387in}}%
\pgfpathlineto{\pgfqpoint{3.041131in}{1.543299in}}%
\pgfpathlineto{\pgfqpoint{3.049708in}{1.546469in}}%
\pgfpathlineto{\pgfqpoint{3.035845in}{1.555981in}}%
\pgfpathlineto{\pgfqpoint{3.021982in}{1.565707in}}%
\pgfpathlineto{\pgfqpoint{3.008117in}{1.575648in}}%
\pgfpathlineto{\pgfqpoint{2.994251in}{1.585806in}}%
\pgfpathlineto{\pgfqpoint{2.985639in}{1.583291in}}%
\pgfpathlineto{\pgfqpoint{2.977012in}{1.581042in}}%
\pgfpathlineto{\pgfqpoint{2.968371in}{1.579067in}}%
\pgfpathlineto{\pgfqpoint{2.959715in}{1.577372in}}%
\pgfpathclose%
\pgfusepath{fill}%
\end{pgfscope}%
\begin{pgfscope}%
\pgfpathrectangle{\pgfqpoint{1.150000in}{0.150000in}}{\pgfqpoint{5.700000in}{5.700000in}}%
\pgfusepath{clip}%
\pgfsetbuttcap%
\pgfsetroundjoin%
\definecolor{currentfill}{rgb}{0.273006,0.204520,0.501721}%
\pgfsetfillcolor{currentfill}%
\pgfsetfillopacity{0.800000}%
\pgfsetlinewidth{0.000000pt}%
\definecolor{currentstroke}{rgb}{0.000000,0.000000,0.000000}%
\pgfsetstrokecolor{currentstroke}%
\pgfsetdash{}{0pt}%
\pgfpathmoveto{\pgfqpoint{4.100758in}{1.852690in}}%
\pgfpathlineto{\pgfqpoint{4.114778in}{1.858345in}}%
\pgfpathlineto{\pgfqpoint{4.128809in}{1.864184in}}%
\pgfpathlineto{\pgfqpoint{4.142852in}{1.870207in}}%
\pgfpathlineto{\pgfqpoint{4.156908in}{1.876413in}}%
\pgfpathlineto{\pgfqpoint{4.164982in}{1.890131in}}%
\pgfpathlineto{\pgfqpoint{4.173052in}{1.903800in}}%
\pgfpathlineto{\pgfqpoint{4.181118in}{1.917417in}}%
\pgfpathlineto{\pgfqpoint{4.189179in}{1.930980in}}%
\pgfpathlineto{\pgfqpoint{4.175124in}{1.924457in}}%
\pgfpathlineto{\pgfqpoint{4.161082in}{1.918119in}}%
\pgfpathlineto{\pgfqpoint{4.147052in}{1.911965in}}%
\pgfpathlineto{\pgfqpoint{4.133035in}{1.905995in}}%
\pgfpathlineto{\pgfqpoint{4.124973in}{1.892735in}}%
\pgfpathlineto{\pgfqpoint{4.116906in}{1.879430in}}%
\pgfpathlineto{\pgfqpoint{4.108834in}{1.866080in}}%
\pgfpathlineto{\pgfqpoint{4.100758in}{1.852690in}}%
\pgfpathclose%
\pgfusepath{fill}%
\end{pgfscope}%
\begin{pgfscope}%
\pgfpathrectangle{\pgfqpoint{1.150000in}{0.150000in}}{\pgfqpoint{5.700000in}{5.700000in}}%
\pgfusepath{clip}%
\pgfsetbuttcap%
\pgfsetroundjoin%
\definecolor{currentfill}{rgb}{0.267004,0.004874,0.329415}%
\pgfsetfillcolor{currentfill}%
\pgfsetfillopacity{0.800000}%
\pgfsetlinewidth{0.000000pt}%
\definecolor{currentstroke}{rgb}{0.000000,0.000000,0.000000}%
\pgfsetstrokecolor{currentstroke}%
\pgfsetdash{}{0pt}%
\pgfpathmoveto{\pgfqpoint{3.449529in}{1.443256in}}%
\pgfpathlineto{\pgfqpoint{3.463388in}{1.440024in}}%
\pgfpathlineto{\pgfqpoint{3.477252in}{1.436985in}}%
\pgfpathlineto{\pgfqpoint{3.491120in}{1.434137in}}%
\pgfpathlineto{\pgfqpoint{3.504994in}{1.431482in}}%
\pgfpathlineto{\pgfqpoint{3.513298in}{1.440725in}}%
\pgfpathlineto{\pgfqpoint{3.521594in}{1.450109in}}%
\pgfpathlineto{\pgfqpoint{3.529882in}{1.459628in}}%
\pgfpathlineto{\pgfqpoint{3.538163in}{1.469277in}}%
\pgfpathlineto{\pgfqpoint{3.524307in}{1.471369in}}%
\pgfpathlineto{\pgfqpoint{3.510455in}{1.473653in}}%
\pgfpathlineto{\pgfqpoint{3.496610in}{1.476129in}}%
\pgfpathlineto{\pgfqpoint{3.482769in}{1.478798in}}%
\pgfpathlineto{\pgfqpoint{3.474471in}{1.469701in}}%
\pgfpathlineto{\pgfqpoint{3.466165in}{1.460741in}}%
\pgfpathlineto{\pgfqpoint{3.457851in}{1.451924in}}%
\pgfpathlineto{\pgfqpoint{3.449529in}{1.443256in}}%
\pgfpathclose%
\pgfusepath{fill}%
\end{pgfscope}%
\begin{pgfscope}%
\pgfpathrectangle{\pgfqpoint{1.150000in}{0.150000in}}{\pgfqpoint{5.700000in}{5.700000in}}%
\pgfusepath{clip}%
\pgfsetbuttcap%
\pgfsetroundjoin%
\definecolor{currentfill}{rgb}{0.283229,0.120777,0.440584}%
\pgfsetfillcolor{currentfill}%
\pgfsetfillopacity{0.800000}%
\pgfsetlinewidth{0.000000pt}%
\definecolor{currentstroke}{rgb}{0.000000,0.000000,0.000000}%
\pgfsetstrokecolor{currentstroke}%
\pgfsetdash{}{0pt}%
\pgfpathmoveto{\pgfqpoint{3.891659in}{1.660470in}}%
\pgfpathlineto{\pgfqpoint{3.905603in}{1.663513in}}%
\pgfpathlineto{\pgfqpoint{3.919557in}{1.666741in}}%
\pgfpathlineto{\pgfqpoint{3.933521in}{1.670154in}}%
\pgfpathlineto{\pgfqpoint{3.947494in}{1.673751in}}%
\pgfpathlineto{\pgfqpoint{3.955628in}{1.686956in}}%
\pgfpathlineto{\pgfqpoint{3.963758in}{1.700169in}}%
\pgfpathlineto{\pgfqpoint{3.971883in}{1.713386in}}%
\pgfpathlineto{\pgfqpoint{3.980003in}{1.726603in}}%
\pgfpathlineto{\pgfqpoint{3.966033in}{1.722596in}}%
\pgfpathlineto{\pgfqpoint{3.952074in}{1.718774in}}%
\pgfpathlineto{\pgfqpoint{3.938125in}{1.715137in}}%
\pgfpathlineto{\pgfqpoint{3.924187in}{1.711685in}}%
\pgfpathlineto{\pgfqpoint{3.916062in}{1.698866in}}%
\pgfpathlineto{\pgfqpoint{3.907933in}{1.686054in}}%
\pgfpathlineto{\pgfqpoint{3.899798in}{1.673254in}}%
\pgfpathlineto{\pgfqpoint{3.891659in}{1.660470in}}%
\pgfpathclose%
\pgfusepath{fill}%
\end{pgfscope}%
\begin{pgfscope}%
\pgfpathrectangle{\pgfqpoint{1.150000in}{0.150000in}}{\pgfqpoint{5.700000in}{5.700000in}}%
\pgfusepath{clip}%
\pgfsetbuttcap%
\pgfsetroundjoin%
\definecolor{currentfill}{rgb}{0.190631,0.407061,0.556089}%
\pgfsetfillcolor{currentfill}%
\pgfsetfillopacity{0.800000}%
\pgfsetlinewidth{0.000000pt}%
\definecolor{currentstroke}{rgb}{0.000000,0.000000,0.000000}%
\pgfsetstrokecolor{currentstroke}%
\pgfsetdash{}{0pt}%
\pgfpathmoveto{\pgfqpoint{2.228811in}{2.476063in}}%
\pgfpathlineto{\pgfqpoint{2.243126in}{2.451725in}}%
\pgfpathlineto{\pgfqpoint{2.257427in}{2.427703in}}%
\pgfpathlineto{\pgfqpoint{2.271712in}{2.403994in}}%
\pgfpathlineto{\pgfqpoint{2.285984in}{2.380597in}}%
\pgfpathlineto{\pgfqpoint{2.295258in}{2.373860in}}%
\pgfpathlineto{\pgfqpoint{2.304506in}{2.367515in}}%
\pgfpathlineto{\pgfqpoint{2.313731in}{2.361555in}}%
\pgfpathlineto{\pgfqpoint{2.322930in}{2.355972in}}%
\pgfpathlineto{\pgfqpoint{2.308723in}{2.378650in}}%
\pgfpathlineto{\pgfqpoint{2.294502in}{2.401637in}}%
\pgfpathlineto{\pgfqpoint{2.280267in}{2.424936in}}%
\pgfpathlineto{\pgfqpoint{2.266018in}{2.448550in}}%
\pgfpathlineto{\pgfqpoint{2.256754in}{2.454840in}}%
\pgfpathlineto{\pgfqpoint{2.247466in}{2.461517in}}%
\pgfpathlineto{\pgfqpoint{2.238151in}{2.468589in}}%
\pgfpathlineto{\pgfqpoint{2.228811in}{2.476063in}}%
\pgfpathclose%
\pgfusepath{fill}%
\end{pgfscope}%
\begin{pgfscope}%
\pgfpathrectangle{\pgfqpoint{1.150000in}{0.150000in}}{\pgfqpoint{5.700000in}{5.700000in}}%
\pgfusepath{clip}%
\pgfsetbuttcap%
\pgfsetroundjoin%
\definecolor{currentfill}{rgb}{0.191090,0.708366,0.482284}%
\pgfsetfillcolor{currentfill}%
\pgfsetfillopacity{0.800000}%
\pgfsetlinewidth{0.000000pt}%
\definecolor{currentstroke}{rgb}{0.000000,0.000000,0.000000}%
\pgfsetstrokecolor{currentstroke}%
\pgfsetdash{}{0pt}%
\pgfpathmoveto{\pgfqpoint{5.564455in}{3.312919in}}%
\pgfpathlineto{\pgfqpoint{5.579353in}{3.328526in}}%
\pgfpathlineto{\pgfqpoint{5.594272in}{3.344319in}}%
\pgfpathlineto{\pgfqpoint{5.609215in}{3.360297in}}%
\pgfpathlineto{\pgfqpoint{5.624180in}{3.376461in}}%
\pgfpathlineto{\pgfqpoint{5.631548in}{3.379064in}}%
\pgfpathlineto{\pgfqpoint{5.638907in}{3.381561in}}%
\pgfpathlineto{\pgfqpoint{5.646255in}{3.383959in}}%
\pgfpathlineto{\pgfqpoint{5.653594in}{3.386260in}}%
\pgfpathlineto{\pgfqpoint{5.638651in}{3.370502in}}%
\pgfpathlineto{\pgfqpoint{5.623730in}{3.354929in}}%
\pgfpathlineto{\pgfqpoint{5.608832in}{3.339541in}}%
\pgfpathlineto{\pgfqpoint{5.593955in}{3.324337in}}%
\pgfpathlineto{\pgfqpoint{5.586594in}{3.321618in}}%
\pgfpathlineto{\pgfqpoint{5.579224in}{3.318812in}}%
\pgfpathlineto{\pgfqpoint{5.571844in}{3.315914in}}%
\pgfpathlineto{\pgfqpoint{5.564455in}{3.312919in}}%
\pgfpathclose%
\pgfusepath{fill}%
\end{pgfscope}%
\begin{pgfscope}%
\pgfpathrectangle{\pgfqpoint{1.150000in}{0.150000in}}{\pgfqpoint{5.700000in}{5.700000in}}%
\pgfusepath{clip}%
\pgfsetbuttcap%
\pgfsetroundjoin%
\definecolor{currentfill}{rgb}{0.182256,0.426184,0.557120}%
\pgfsetfillcolor{currentfill}%
\pgfsetfillopacity{0.800000}%
\pgfsetlinewidth{0.000000pt}%
\definecolor{currentstroke}{rgb}{0.000000,0.000000,0.000000}%
\pgfsetstrokecolor{currentstroke}%
\pgfsetdash{}{0pt}%
\pgfpathmoveto{\pgfqpoint{4.639600in}{2.430521in}}%
\pgfpathlineto{\pgfqpoint{4.653902in}{2.441483in}}%
\pgfpathlineto{\pgfqpoint{4.668220in}{2.452630in}}%
\pgfpathlineto{\pgfqpoint{4.682555in}{2.463961in}}%
\pgfpathlineto{\pgfqpoint{4.696908in}{2.475478in}}%
\pgfpathlineto{\pgfqpoint{4.704813in}{2.486986in}}%
\pgfpathlineto{\pgfqpoint{4.712712in}{2.498355in}}%
\pgfpathlineto{\pgfqpoint{4.720605in}{2.509585in}}%
\pgfpathlineto{\pgfqpoint{4.728490in}{2.520676in}}%
\pgfpathlineto{\pgfqpoint{4.714139in}{2.509106in}}%
\pgfpathlineto{\pgfqpoint{4.699805in}{2.497721in}}%
\pgfpathlineto{\pgfqpoint{4.685488in}{2.486521in}}%
\pgfpathlineto{\pgfqpoint{4.671188in}{2.475506in}}%
\pgfpathlineto{\pgfqpoint{4.663301in}{2.464456in}}%
\pgfpathlineto{\pgfqpoint{4.655407in}{2.453275in}}%
\pgfpathlineto{\pgfqpoint{4.647507in}{2.441963in}}%
\pgfpathlineto{\pgfqpoint{4.639600in}{2.430521in}}%
\pgfpathclose%
\pgfusepath{fill}%
\end{pgfscope}%
\begin{pgfscope}%
\pgfpathrectangle{\pgfqpoint{1.150000in}{0.150000in}}{\pgfqpoint{5.700000in}{5.700000in}}%
\pgfusepath{clip}%
\pgfsetbuttcap%
\pgfsetroundjoin%
\definecolor{currentfill}{rgb}{0.241237,0.296485,0.539709}%
\pgfsetfillcolor{currentfill}%
\pgfsetfillopacity{0.800000}%
\pgfsetlinewidth{0.000000pt}%
\definecolor{currentstroke}{rgb}{0.000000,0.000000,0.000000}%
\pgfsetstrokecolor{currentstroke}%
\pgfsetdash{}{0pt}%
\pgfpathmoveto{\pgfqpoint{4.309841in}{2.067305in}}%
\pgfpathlineto{\pgfqpoint{4.323960in}{2.075284in}}%
\pgfpathlineto{\pgfqpoint{4.338094in}{2.083448in}}%
\pgfpathlineto{\pgfqpoint{4.352241in}{2.091795in}}%
\pgfpathlineto{\pgfqpoint{4.366403in}{2.100327in}}%
\pgfpathlineto{\pgfqpoint{4.374421in}{2.113756in}}%
\pgfpathlineto{\pgfqpoint{4.382434in}{2.127091in}}%
\pgfpathlineto{\pgfqpoint{4.390443in}{2.140331in}}%
\pgfpathlineto{\pgfqpoint{4.398446in}{2.153474in}}%
\pgfpathlineto{\pgfqpoint{4.384284in}{2.144722in}}%
\pgfpathlineto{\pgfqpoint{4.370136in}{2.136155in}}%
\pgfpathlineto{\pgfqpoint{4.356003in}{2.127772in}}%
\pgfpathlineto{\pgfqpoint{4.341883in}{2.119574in}}%
\pgfpathlineto{\pgfqpoint{4.333880in}{2.106638in}}%
\pgfpathlineto{\pgfqpoint{4.325872in}{2.093613in}}%
\pgfpathlineto{\pgfqpoint{4.317859in}{2.080502in}}%
\pgfpathlineto{\pgfqpoint{4.309841in}{2.067305in}}%
\pgfpathclose%
\pgfusepath{fill}%
\end{pgfscope}%
\begin{pgfscope}%
\pgfpathrectangle{\pgfqpoint{1.150000in}{0.150000in}}{\pgfqpoint{5.700000in}{5.700000in}}%
\pgfusepath{clip}%
\pgfsetbuttcap%
\pgfsetroundjoin%
\definecolor{currentfill}{rgb}{0.121380,0.629492,0.531973}%
\pgfsetfillcolor{currentfill}%
\pgfsetfillopacity{0.800000}%
\pgfsetlinewidth{0.000000pt}%
\definecolor{currentstroke}{rgb}{0.000000,0.000000,0.000000}%
\pgfsetstrokecolor{currentstroke}%
\pgfsetdash{}{0pt}%
\pgfpathmoveto{\pgfqpoint{5.267020in}{3.060526in}}%
\pgfpathlineto{\pgfqpoint{5.281723in}{3.075169in}}%
\pgfpathlineto{\pgfqpoint{5.296447in}{3.089998in}}%
\pgfpathlineto{\pgfqpoint{5.311192in}{3.105013in}}%
\pgfpathlineto{\pgfqpoint{5.325958in}{3.120214in}}%
\pgfpathlineto{\pgfqpoint{5.333535in}{3.125797in}}%
\pgfpathlineto{\pgfqpoint{5.341102in}{3.131242in}}%
\pgfpathlineto{\pgfqpoint{5.348660in}{3.136550in}}%
\pgfpathlineto{\pgfqpoint{5.356208in}{3.141725in}}%
\pgfpathlineto{\pgfqpoint{5.341455in}{3.126787in}}%
\pgfpathlineto{\pgfqpoint{5.326723in}{3.112033in}}%
\pgfpathlineto{\pgfqpoint{5.312012in}{3.097465in}}%
\pgfpathlineto{\pgfqpoint{5.297322in}{3.083082in}}%
\pgfpathlineto{\pgfqpoint{5.289760in}{3.077633in}}%
\pgfpathlineto{\pgfqpoint{5.282189in}{3.072060in}}%
\pgfpathlineto{\pgfqpoint{5.274609in}{3.066358in}}%
\pgfpathlineto{\pgfqpoint{5.267020in}{3.060526in}}%
\pgfpathclose%
\pgfusepath{fill}%
\end{pgfscope}%
\begin{pgfscope}%
\pgfpathrectangle{\pgfqpoint{1.150000in}{0.150000in}}{\pgfqpoint{5.700000in}{5.700000in}}%
\pgfusepath{clip}%
\pgfsetbuttcap%
\pgfsetroundjoin%
\definecolor{currentfill}{rgb}{0.139147,0.533812,0.555298}%
\pgfsetfillcolor{currentfill}%
\pgfsetfillopacity{0.800000}%
\pgfsetlinewidth{0.000000pt}%
\definecolor{currentstroke}{rgb}{0.000000,0.000000,0.000000}%
\pgfsetstrokecolor{currentstroke}%
\pgfsetdash{}{0pt}%
\pgfpathmoveto{\pgfqpoint{2.035768in}{2.871749in}}%
\pgfpathlineto{\pgfqpoint{2.050298in}{2.842660in}}%
\pgfpathlineto{\pgfqpoint{2.064808in}{2.813940in}}%
\pgfpathlineto{\pgfqpoint{2.079298in}{2.785584in}}%
\pgfpathlineto{\pgfqpoint{2.093768in}{2.757591in}}%
\pgfpathlineto{\pgfqpoint{2.103209in}{2.749546in}}%
\pgfpathlineto{\pgfqpoint{2.112623in}{2.741905in}}%
\pgfpathlineto{\pgfqpoint{2.122010in}{2.734660in}}%
\pgfpathlineto{\pgfqpoint{2.131370in}{2.727805in}}%
\pgfpathlineto{\pgfqpoint{2.116970in}{2.755090in}}%
\pgfpathlineto{\pgfqpoint{2.102552in}{2.782733in}}%
\pgfpathlineto{\pgfqpoint{2.088113in}{2.810740in}}%
\pgfpathlineto{\pgfqpoint{2.073655in}{2.839113in}}%
\pgfpathlineto{\pgfqpoint{2.064225in}{2.846666in}}%
\pgfpathlineto{\pgfqpoint{2.054768in}{2.854617in}}%
\pgfpathlineto{\pgfqpoint{2.045282in}{2.862976in}}%
\pgfpathlineto{\pgfqpoint{2.035768in}{2.871749in}}%
\pgfpathclose%
\pgfusepath{fill}%
\end{pgfscope}%
\begin{pgfscope}%
\pgfpathrectangle{\pgfqpoint{1.150000in}{0.150000in}}{\pgfqpoint{5.700000in}{5.700000in}}%
\pgfusepath{clip}%
\pgfsetbuttcap%
\pgfsetroundjoin%
\definecolor{currentfill}{rgb}{0.276022,0.044167,0.370164}%
\pgfsetfillcolor{currentfill}%
\pgfsetfillopacity{0.800000}%
\pgfsetlinewidth{0.000000pt}%
\definecolor{currentstroke}{rgb}{0.000000,0.000000,0.000000}%
\pgfsetstrokecolor{currentstroke}%
\pgfsetdash{}{0pt}%
\pgfpathmoveto{\pgfqpoint{3.015319in}{1.535366in}}%
\pgfpathlineto{\pgfqpoint{3.029216in}{1.525401in}}%
\pgfpathlineto{\pgfqpoint{3.043113in}{1.515649in}}%
\pgfpathlineto{\pgfqpoint{3.057009in}{1.506108in}}%
\pgfpathlineto{\pgfqpoint{3.070904in}{1.496777in}}%
\pgfpathlineto{\pgfqpoint{3.079487in}{1.499830in}}%
\pgfpathlineto{\pgfqpoint{3.088056in}{1.503146in}}%
\pgfpathlineto{\pgfqpoint{3.096612in}{1.506719in}}%
\pgfpathlineto{\pgfqpoint{3.105155in}{1.510541in}}%
\pgfpathlineto{\pgfqpoint{3.091293in}{1.519207in}}%
\pgfpathlineto{\pgfqpoint{3.077432in}{1.528083in}}%
\pgfpathlineto{\pgfqpoint{3.063570in}{1.537170in}}%
\pgfpathlineto{\pgfqpoint{3.049708in}{1.546469in}}%
\pgfpathlineto{\pgfqpoint{3.041131in}{1.543299in}}%
\pgfpathlineto{\pgfqpoint{3.032541in}{1.540387in}}%
\pgfpathlineto{\pgfqpoint{3.023937in}{1.537741in}}%
\pgfpathlineto{\pgfqpoint{3.015319in}{1.535366in}}%
\pgfpathclose%
\pgfusepath{fill}%
\end{pgfscope}%
\begin{pgfscope}%
\pgfpathrectangle{\pgfqpoint{1.150000in}{0.150000in}}{\pgfqpoint{5.700000in}{5.700000in}}%
\pgfusepath{clip}%
\pgfsetbuttcap%
\pgfsetroundjoin%
\definecolor{currentfill}{rgb}{0.151918,0.500685,0.557587}%
\pgfsetfillcolor{currentfill}%
\pgfsetfillopacity{0.800000}%
\pgfsetlinewidth{0.000000pt}%
\definecolor{currentstroke}{rgb}{0.000000,0.000000,0.000000}%
\pgfsetstrokecolor{currentstroke}%
\pgfsetdash{}{0pt}%
\pgfpathmoveto{\pgfqpoint{4.848890in}{2.652563in}}%
\pgfpathlineto{\pgfqpoint{4.863325in}{2.665063in}}%
\pgfpathlineto{\pgfqpoint{4.877778in}{2.677749in}}%
\pgfpathlineto{\pgfqpoint{4.892249in}{2.690619in}}%
\pgfpathlineto{\pgfqpoint{4.906739in}{2.703675in}}%
\pgfpathlineto{\pgfqpoint{4.914556in}{2.713453in}}%
\pgfpathlineto{\pgfqpoint{4.922365in}{2.723079in}}%
\pgfpathlineto{\pgfqpoint{4.930166in}{2.732553in}}%
\pgfpathlineto{\pgfqpoint{4.937959in}{2.741878in}}%
\pgfpathlineto{\pgfqpoint{4.923473in}{2.728872in}}%
\pgfpathlineto{\pgfqpoint{4.909006in}{2.716051in}}%
\pgfpathlineto{\pgfqpoint{4.894557in}{2.703415in}}%
\pgfpathlineto{\pgfqpoint{4.880126in}{2.690964in}}%
\pgfpathlineto{\pgfqpoint{4.872328in}{2.681577in}}%
\pgfpathlineto{\pgfqpoint{4.864523in}{2.672048in}}%
\pgfpathlineto{\pgfqpoint{4.856710in}{2.662378in}}%
\pgfpathlineto{\pgfqpoint{4.848890in}{2.652563in}}%
\pgfpathclose%
\pgfusepath{fill}%
\end{pgfscope}%
\begin{pgfscope}%
\pgfpathrectangle{\pgfqpoint{1.150000in}{0.150000in}}{\pgfqpoint{5.700000in}{5.700000in}}%
\pgfusepath{clip}%
\pgfsetbuttcap%
\pgfsetroundjoin%
\definecolor{currentfill}{rgb}{0.127568,0.566949,0.550556}%
\pgfsetfillcolor{currentfill}%
\pgfsetfillopacity{0.800000}%
\pgfsetlinewidth{0.000000pt}%
\definecolor{currentstroke}{rgb}{0.000000,0.000000,0.000000}%
\pgfsetstrokecolor{currentstroke}%
\pgfsetdash{}{0pt}%
\pgfpathmoveto{\pgfqpoint{5.058109in}{2.864293in}}%
\pgfpathlineto{\pgfqpoint{5.072679in}{2.878023in}}%
\pgfpathlineto{\pgfqpoint{5.087269in}{2.891938in}}%
\pgfpathlineto{\pgfqpoint{5.101879in}{2.906039in}}%
\pgfpathlineto{\pgfqpoint{5.116509in}{2.920326in}}%
\pgfpathlineto{\pgfqpoint{5.124216in}{2.928079in}}%
\pgfpathlineto{\pgfqpoint{5.131914in}{2.935680in}}%
\pgfpathlineto{\pgfqpoint{5.139604in}{2.943130in}}%
\pgfpathlineto{\pgfqpoint{5.147284in}{2.950433in}}%
\pgfpathlineto{\pgfqpoint{5.132663in}{2.936302in}}%
\pgfpathlineto{\pgfqpoint{5.118061in}{2.922356in}}%
\pgfpathlineto{\pgfqpoint{5.103479in}{2.908595in}}%
\pgfpathlineto{\pgfqpoint{5.088916in}{2.895019in}}%
\pgfpathlineto{\pgfqpoint{5.081227in}{2.887549in}}%
\pgfpathlineto{\pgfqpoint{5.073530in}{2.879939in}}%
\pgfpathlineto{\pgfqpoint{5.065823in}{2.872188in}}%
\pgfpathlineto{\pgfqpoint{5.058109in}{2.864293in}}%
\pgfpathclose%
\pgfusepath{fill}%
\end{pgfscope}%
\begin{pgfscope}%
\pgfpathrectangle{\pgfqpoint{1.150000in}{0.150000in}}{\pgfqpoint{5.700000in}{5.700000in}}%
\pgfusepath{clip}%
\pgfsetbuttcap%
\pgfsetroundjoin%
\definecolor{currentfill}{rgb}{0.267004,0.004874,0.329415}%
\pgfsetfillcolor{currentfill}%
\pgfsetfillopacity{0.800000}%
\pgfsetlinewidth{0.000000pt}%
\definecolor{currentstroke}{rgb}{0.000000,0.000000,0.000000}%
\pgfsetstrokecolor{currentstroke}%
\pgfsetdash{}{0pt}%
\pgfpathmoveto{\pgfqpoint{3.216064in}{1.448660in}}%
\pgfpathlineto{\pgfqpoint{3.229933in}{1.441842in}}%
\pgfpathlineto{\pgfqpoint{3.243804in}{1.435226in}}%
\pgfpathlineto{\pgfqpoint{3.257676in}{1.428809in}}%
\pgfpathlineto{\pgfqpoint{3.271551in}{1.422592in}}%
\pgfpathlineto{\pgfqpoint{3.279993in}{1.428583in}}%
\pgfpathlineto{\pgfqpoint{3.288424in}{1.434787in}}%
\pgfpathlineto{\pgfqpoint{3.296846in}{1.441196in}}%
\pgfpathlineto{\pgfqpoint{3.305257in}{1.447805in}}%
\pgfpathlineto{\pgfqpoint{3.291408in}{1.453393in}}%
\pgfpathlineto{\pgfqpoint{3.277562in}{1.459182in}}%
\pgfpathlineto{\pgfqpoint{3.263719in}{1.465170in}}%
\pgfpathlineto{\pgfqpoint{3.249877in}{1.471359in}}%
\pgfpathlineto{\pgfqpoint{3.241440in}{1.465366in}}%
\pgfpathlineto{\pgfqpoint{3.232992in}{1.459581in}}%
\pgfpathlineto{\pgfqpoint{3.224534in}{1.454011in}}%
\pgfpathlineto{\pgfqpoint{3.216064in}{1.448660in}}%
\pgfpathclose%
\pgfusepath{fill}%
\end{pgfscope}%
\begin{pgfscope}%
\pgfpathrectangle{\pgfqpoint{1.150000in}{0.150000in}}{\pgfqpoint{5.700000in}{5.700000in}}%
\pgfusepath{clip}%
\pgfsetbuttcap%
\pgfsetroundjoin%
\definecolor{currentfill}{rgb}{0.281412,0.155834,0.469201}%
\pgfsetfillcolor{currentfill}%
\pgfsetfillopacity{0.800000}%
\pgfsetlinewidth{0.000000pt}%
\definecolor{currentstroke}{rgb}{0.000000,0.000000,0.000000}%
\pgfsetstrokecolor{currentstroke}%
\pgfsetdash{}{0pt}%
\pgfpathmoveto{\pgfqpoint{3.980003in}{1.726603in}}%
\pgfpathlineto{\pgfqpoint{3.993983in}{1.730794in}}%
\pgfpathlineto{\pgfqpoint{4.007973in}{1.735170in}}%
\pgfpathlineto{\pgfqpoint{4.021975in}{1.739730in}}%
\pgfpathlineto{\pgfqpoint{4.035987in}{1.744474in}}%
\pgfpathlineto{\pgfqpoint{4.044100in}{1.758078in}}%
\pgfpathlineto{\pgfqpoint{4.052207in}{1.771666in}}%
\pgfpathlineto{\pgfqpoint{4.060310in}{1.785236in}}%
\pgfpathlineto{\pgfqpoint{4.068409in}{1.798784in}}%
\pgfpathlineto{\pgfqpoint{4.054399in}{1.793661in}}%
\pgfpathlineto{\pgfqpoint{4.040400in}{1.788722in}}%
\pgfpathlineto{\pgfqpoint{4.026413in}{1.783968in}}%
\pgfpathlineto{\pgfqpoint{4.012437in}{1.779398in}}%
\pgfpathlineto{\pgfqpoint{4.004335in}{1.766217in}}%
\pgfpathlineto{\pgfqpoint{3.996229in}{1.753022in}}%
\pgfpathlineto{\pgfqpoint{3.988118in}{1.739816in}}%
\pgfpathlineto{\pgfqpoint{3.980003in}{1.726603in}}%
\pgfpathclose%
\pgfusepath{fill}%
\end{pgfscope}%
\begin{pgfscope}%
\pgfpathrectangle{\pgfqpoint{1.150000in}{0.150000in}}{\pgfqpoint{5.700000in}{5.700000in}}%
\pgfusepath{clip}%
\pgfsetbuttcap%
\pgfsetroundjoin%
\definecolor{currentfill}{rgb}{0.201239,0.383670,0.554294}%
\pgfsetfillcolor{currentfill}%
\pgfsetfillopacity{0.800000}%
\pgfsetlinewidth{0.000000pt}%
\definecolor{currentstroke}{rgb}{0.000000,0.000000,0.000000}%
\pgfsetstrokecolor{currentstroke}%
\pgfsetdash{}{0pt}%
\pgfpathmoveto{\pgfqpoint{4.519080in}{2.293159in}}%
\pgfpathlineto{\pgfqpoint{4.533318in}{2.303174in}}%
\pgfpathlineto{\pgfqpoint{4.547572in}{2.313373in}}%
\pgfpathlineto{\pgfqpoint{4.561842in}{2.323757in}}%
\pgfpathlineto{\pgfqpoint{4.576127in}{2.334326in}}%
\pgfpathlineto{\pgfqpoint{4.584082in}{2.346798in}}%
\pgfpathlineto{\pgfqpoint{4.592032in}{2.359144in}}%
\pgfpathlineto{\pgfqpoint{4.599975in}{2.371362in}}%
\pgfpathlineto{\pgfqpoint{4.607912in}{2.383452in}}%
\pgfpathlineto{\pgfqpoint{4.593626in}{2.372763in}}%
\pgfpathlineto{\pgfqpoint{4.579357in}{2.362258in}}%
\pgfpathlineto{\pgfqpoint{4.565103in}{2.351937in}}%
\pgfpathlineto{\pgfqpoint{4.550865in}{2.341802in}}%
\pgfpathlineto{\pgfqpoint{4.542928in}{2.329820in}}%
\pgfpathlineto{\pgfqpoint{4.534984in}{2.317718in}}%
\pgfpathlineto{\pgfqpoint{4.527035in}{2.305497in}}%
\pgfpathlineto{\pgfqpoint{4.519080in}{2.293159in}}%
\pgfpathclose%
\pgfusepath{fill}%
\end{pgfscope}%
\begin{pgfscope}%
\pgfpathrectangle{\pgfqpoint{1.150000in}{0.150000in}}{\pgfqpoint{5.700000in}{5.700000in}}%
\pgfusepath{clip}%
\pgfsetbuttcap%
\pgfsetroundjoin%
\definecolor{currentfill}{rgb}{0.260571,0.246922,0.522828}%
\pgfsetfillcolor{currentfill}%
\pgfsetfillopacity{0.800000}%
\pgfsetlinewidth{0.000000pt}%
\definecolor{currentstroke}{rgb}{0.000000,0.000000,0.000000}%
\pgfsetstrokecolor{currentstroke}%
\pgfsetdash{}{0pt}%
\pgfpathmoveto{\pgfqpoint{4.189179in}{1.930980in}}%
\pgfpathlineto{\pgfqpoint{4.203246in}{1.937687in}}%
\pgfpathlineto{\pgfqpoint{4.217326in}{1.944578in}}%
\pgfpathlineto{\pgfqpoint{4.231419in}{1.951652in}}%
\pgfpathlineto{\pgfqpoint{4.245525in}{1.958911in}}%
\pgfpathlineto{\pgfqpoint{4.253581in}{1.972714in}}%
\pgfpathlineto{\pgfqpoint{4.261632in}{1.986448in}}%
\pgfpathlineto{\pgfqpoint{4.269678in}{2.000113in}}%
\pgfpathlineto{\pgfqpoint{4.277720in}{2.013706in}}%
\pgfpathlineto{\pgfqpoint{4.263614in}{2.006162in}}%
\pgfpathlineto{\pgfqpoint{4.249522in}{1.998803in}}%
\pgfpathlineto{\pgfqpoint{4.235442in}{1.991628in}}%
\pgfpathlineto{\pgfqpoint{4.221376in}{1.984638in}}%
\pgfpathlineto{\pgfqpoint{4.213334in}{1.971317in}}%
\pgfpathlineto{\pgfqpoint{4.205287in}{1.957933in}}%
\pgfpathlineto{\pgfqpoint{4.197235in}{1.944486in}}%
\pgfpathlineto{\pgfqpoint{4.189179in}{1.930980in}}%
\pgfpathclose%
\pgfusepath{fill}%
\end{pgfscope}%
\begin{pgfscope}%
\pgfpathrectangle{\pgfqpoint{1.150000in}{0.150000in}}{\pgfqpoint{5.700000in}{5.700000in}}%
\pgfusepath{clip}%
\pgfsetbuttcap%
\pgfsetroundjoin%
\definecolor{currentfill}{rgb}{0.267004,0.004874,0.329415}%
\pgfsetfillcolor{currentfill}%
\pgfsetfillopacity{0.800000}%
\pgfsetlinewidth{0.000000pt}%
\definecolor{currentstroke}{rgb}{0.000000,0.000000,0.000000}%
\pgfsetstrokecolor{currentstroke}%
\pgfsetdash{}{0pt}%
\pgfpathmoveto{\pgfqpoint{3.360680in}{1.427427in}}%
\pgfpathlineto{\pgfqpoint{3.374544in}{1.422824in}}%
\pgfpathlineto{\pgfqpoint{3.388411in}{1.418415in}}%
\pgfpathlineto{\pgfqpoint{3.402282in}{1.414201in}}%
\pgfpathlineto{\pgfqpoint{3.416158in}{1.410181in}}%
\pgfpathlineto{\pgfqpoint{3.424513in}{1.418199in}}%
\pgfpathlineto{\pgfqpoint{3.432860in}{1.426387in}}%
\pgfpathlineto{\pgfqpoint{3.441199in}{1.434742in}}%
\pgfpathlineto{\pgfqpoint{3.449529in}{1.443256in}}%
\pgfpathlineto{\pgfqpoint{3.435674in}{1.446681in}}%
\pgfpathlineto{\pgfqpoint{3.421824in}{1.450299in}}%
\pgfpathlineto{\pgfqpoint{3.407978in}{1.454112in}}%
\pgfpathlineto{\pgfqpoint{3.394136in}{1.458121in}}%
\pgfpathlineto{\pgfqpoint{3.385785in}{1.450190in}}%
\pgfpathlineto{\pgfqpoint{3.377426in}{1.442427in}}%
\pgfpathlineto{\pgfqpoint{3.369057in}{1.434837in}}%
\pgfpathlineto{\pgfqpoint{3.360680in}{1.427427in}}%
\pgfpathclose%
\pgfusepath{fill}%
\end{pgfscope}%
\begin{pgfscope}%
\pgfpathrectangle{\pgfqpoint{1.150000in}{0.150000in}}{\pgfqpoint{5.700000in}{5.700000in}}%
\pgfusepath{clip}%
\pgfsetbuttcap%
\pgfsetroundjoin%
\definecolor{currentfill}{rgb}{0.232815,0.732247,0.459277}%
\pgfsetfillcolor{currentfill}%
\pgfsetfillopacity{0.800000}%
\pgfsetlinewidth{0.000000pt}%
\definecolor{currentstroke}{rgb}{0.000000,0.000000,0.000000}%
\pgfsetstrokecolor{currentstroke}%
\pgfsetdash{}{0pt}%
\pgfpathmoveto{\pgfqpoint{5.653594in}{3.386260in}}%
\pgfpathlineto{\pgfqpoint{5.668560in}{3.402203in}}%
\pgfpathlineto{\pgfqpoint{5.683549in}{3.418331in}}%
\pgfpathlineto{\pgfqpoint{5.698561in}{3.434645in}}%
\pgfpathlineto{\pgfqpoint{5.713596in}{3.451145in}}%
\pgfpathlineto{\pgfqpoint{5.720901in}{3.452925in}}%
\pgfpathlineto{\pgfqpoint{5.728197in}{3.454612in}}%
\pgfpathlineto{\pgfqpoint{5.735482in}{3.456208in}}%
\pgfpathlineto{\pgfqpoint{5.742758in}{3.457719in}}%
\pgfpathlineto{\pgfqpoint{5.727748in}{3.441663in}}%
\pgfpathlineto{\pgfqpoint{5.712760in}{3.425791in}}%
\pgfpathlineto{\pgfqpoint{5.697795in}{3.410104in}}%
\pgfpathlineto{\pgfqpoint{5.682853in}{3.394601in}}%
\pgfpathlineto{\pgfqpoint{5.675552in}{3.392636in}}%
\pgfpathlineto{\pgfqpoint{5.668242in}{3.390594in}}%
\pgfpathlineto{\pgfqpoint{5.660923in}{3.388470in}}%
\pgfpathlineto{\pgfqpoint{5.653594in}{3.386260in}}%
\pgfpathclose%
\pgfusepath{fill}%
\end{pgfscope}%
\begin{pgfscope}%
\pgfpathrectangle{\pgfqpoint{1.150000in}{0.150000in}}{\pgfqpoint{5.700000in}{5.700000in}}%
\pgfusepath{clip}%
\pgfsetbuttcap%
\pgfsetroundjoin%
\definecolor{currentfill}{rgb}{0.271828,0.209303,0.504434}%
\pgfsetfillcolor{currentfill}%
\pgfsetfillopacity{0.800000}%
\pgfsetlinewidth{0.000000pt}%
\definecolor{currentstroke}{rgb}{0.000000,0.000000,0.000000}%
\pgfsetstrokecolor{currentstroke}%
\pgfsetdash{}{0pt}%
\pgfpathmoveto{\pgfqpoint{2.589067in}{1.918856in}}%
\pgfpathlineto{\pgfqpoint{2.603129in}{1.901701in}}%
\pgfpathlineto{\pgfqpoint{2.617183in}{1.884799in}}%
\pgfpathlineto{\pgfqpoint{2.631230in}{1.868147in}}%
\pgfpathlineto{\pgfqpoint{2.645270in}{1.851745in}}%
\pgfpathlineto{\pgfqpoint{2.654224in}{1.848628in}}%
\pgfpathlineto{\pgfqpoint{2.663157in}{1.845864in}}%
\pgfpathlineto{\pgfqpoint{2.672071in}{1.843447in}}%
\pgfpathlineto{\pgfqpoint{2.680965in}{1.841370in}}%
\pgfpathlineto{\pgfqpoint{2.666977in}{1.857052in}}%
\pgfpathlineto{\pgfqpoint{2.652982in}{1.872983in}}%
\pgfpathlineto{\pgfqpoint{2.638981in}{1.889163in}}%
\pgfpathlineto{\pgfqpoint{2.624972in}{1.905595in}}%
\pgfpathlineto{\pgfqpoint{2.616027in}{1.908380in}}%
\pgfpathlineto{\pgfqpoint{2.607061in}{1.911513in}}%
\pgfpathlineto{\pgfqpoint{2.598075in}{1.915003in}}%
\pgfpathlineto{\pgfqpoint{2.589067in}{1.918856in}}%
\pgfpathclose%
\pgfusepath{fill}%
\end{pgfscope}%
\begin{pgfscope}%
\pgfpathrectangle{\pgfqpoint{1.150000in}{0.150000in}}{\pgfqpoint{5.700000in}{5.700000in}}%
\pgfusepath{clip}%
\pgfsetbuttcap%
\pgfsetroundjoin%
\definecolor{currentfill}{rgb}{0.278012,0.180367,0.486697}%
\pgfsetfillcolor{currentfill}%
\pgfsetfillopacity{0.800000}%
\pgfsetlinewidth{0.000000pt}%
\definecolor{currentstroke}{rgb}{0.000000,0.000000,0.000000}%
\pgfsetstrokecolor{currentstroke}%
\pgfsetdash{}{0pt}%
\pgfpathmoveto{\pgfqpoint{2.645270in}{1.851745in}}%
\pgfpathlineto{\pgfqpoint{2.659303in}{1.835590in}}%
\pgfpathlineto{\pgfqpoint{2.673329in}{1.819680in}}%
\pgfpathlineto{\pgfqpoint{2.687349in}{1.804015in}}%
\pgfpathlineto{\pgfqpoint{2.701363in}{1.788591in}}%
\pgfpathlineto{\pgfqpoint{2.710266in}{1.786205in}}%
\pgfpathlineto{\pgfqpoint{2.719149in}{1.784164in}}%
\pgfpathlineto{\pgfqpoint{2.728013in}{1.782460in}}%
\pgfpathlineto{\pgfqpoint{2.736858in}{1.781086in}}%
\pgfpathlineto{\pgfqpoint{2.722894in}{1.795793in}}%
\pgfpathlineto{\pgfqpoint{2.708923in}{1.810742in}}%
\pgfpathlineto{\pgfqpoint{2.694947in}{1.825934in}}%
\pgfpathlineto{\pgfqpoint{2.680965in}{1.841370in}}%
\pgfpathlineto{\pgfqpoint{2.672071in}{1.843447in}}%
\pgfpathlineto{\pgfqpoint{2.663157in}{1.845864in}}%
\pgfpathlineto{\pgfqpoint{2.654224in}{1.848628in}}%
\pgfpathlineto{\pgfqpoint{2.645270in}{1.851745in}}%
\pgfpathclose%
\pgfusepath{fill}%
\end{pgfscope}%
\begin{pgfscope}%
\pgfpathrectangle{\pgfqpoint{1.150000in}{0.150000in}}{\pgfqpoint{5.700000in}{5.700000in}}%
\pgfusepath{clip}%
\pgfsetbuttcap%
\pgfsetroundjoin%
\definecolor{currentfill}{rgb}{0.175841,0.441290,0.557685}%
\pgfsetfillcolor{currentfill}%
\pgfsetfillopacity{0.800000}%
\pgfsetlinewidth{0.000000pt}%
\definecolor{currentstroke}{rgb}{0.000000,0.000000,0.000000}%
\pgfsetstrokecolor{currentstroke}%
\pgfsetdash{}{0pt}%
\pgfpathmoveto{\pgfqpoint{2.171391in}{2.576643in}}%
\pgfpathlineto{\pgfqpoint{2.185771in}{2.551008in}}%
\pgfpathlineto{\pgfqpoint{2.200133in}{2.525702in}}%
\pgfpathlineto{\pgfqpoint{2.214480in}{2.500721in}}%
\pgfpathlineto{\pgfqpoint{2.228811in}{2.476063in}}%
\pgfpathlineto{\pgfqpoint{2.238151in}{2.468589in}}%
\pgfpathlineto{\pgfqpoint{2.247466in}{2.461517in}}%
\pgfpathlineto{\pgfqpoint{2.256754in}{2.454840in}}%
\pgfpathlineto{\pgfqpoint{2.266018in}{2.448550in}}%
\pgfpathlineto{\pgfqpoint{2.251753in}{2.472482in}}%
\pgfpathlineto{\pgfqpoint{2.237474in}{2.496734in}}%
\pgfpathlineto{\pgfqpoint{2.223179in}{2.521309in}}%
\pgfpathlineto{\pgfqpoint{2.208868in}{2.546212in}}%
\pgfpathlineto{\pgfqpoint{2.199539in}{2.553216in}}%
\pgfpathlineto{\pgfqpoint{2.190183in}{2.560617in}}%
\pgfpathlineto{\pgfqpoint{2.180801in}{2.568424in}}%
\pgfpathlineto{\pgfqpoint{2.171391in}{2.576643in}}%
\pgfpathclose%
\pgfusepath{fill}%
\end{pgfscope}%
\begin{pgfscope}%
\pgfpathrectangle{\pgfqpoint{1.150000in}{0.150000in}}{\pgfqpoint{5.700000in}{5.700000in}}%
\pgfusepath{clip}%
\pgfsetbuttcap%
\pgfsetroundjoin%
\definecolor{currentfill}{rgb}{0.263663,0.237631,0.518762}%
\pgfsetfillcolor{currentfill}%
\pgfsetfillopacity{0.800000}%
\pgfsetlinewidth{0.000000pt}%
\definecolor{currentstroke}{rgb}{0.000000,0.000000,0.000000}%
\pgfsetstrokecolor{currentstroke}%
\pgfsetdash{}{0pt}%
\pgfpathmoveto{\pgfqpoint{2.532740in}{1.990037in}}%
\pgfpathlineto{\pgfqpoint{2.546834in}{1.971853in}}%
\pgfpathlineto{\pgfqpoint{2.560920in}{1.953930in}}%
\pgfpathlineto{\pgfqpoint{2.574998in}{1.936265in}}%
\pgfpathlineto{\pgfqpoint{2.589067in}{1.918856in}}%
\pgfpathlineto{\pgfqpoint{2.598075in}{1.915003in}}%
\pgfpathlineto{\pgfqpoint{2.607061in}{1.911513in}}%
\pgfpathlineto{\pgfqpoint{2.616027in}{1.908380in}}%
\pgfpathlineto{\pgfqpoint{2.624972in}{1.905595in}}%
\pgfpathlineto{\pgfqpoint{2.610956in}{1.922279in}}%
\pgfpathlineto{\pgfqpoint{2.596933in}{1.939219in}}%
\pgfpathlineto{\pgfqpoint{2.582902in}{1.956416in}}%
\pgfpathlineto{\pgfqpoint{2.568863in}{1.973872in}}%
\pgfpathlineto{\pgfqpoint{2.559864in}{1.977368in}}%
\pgfpathlineto{\pgfqpoint{2.550845in}{1.981223in}}%
\pgfpathlineto{\pgfqpoint{2.541803in}{1.985443in}}%
\pgfpathlineto{\pgfqpoint{2.532740in}{1.990037in}}%
\pgfpathclose%
\pgfusepath{fill}%
\end{pgfscope}%
\begin{pgfscope}%
\pgfpathrectangle{\pgfqpoint{1.150000in}{0.150000in}}{\pgfqpoint{5.700000in}{5.700000in}}%
\pgfusepath{clip}%
\pgfsetbuttcap%
\pgfsetroundjoin%
\definecolor{currentfill}{rgb}{0.281412,0.155834,0.469201}%
\pgfsetfillcolor{currentfill}%
\pgfsetfillopacity{0.800000}%
\pgfsetlinewidth{0.000000pt}%
\definecolor{currentstroke}{rgb}{0.000000,0.000000,0.000000}%
\pgfsetstrokecolor{currentstroke}%
\pgfsetdash{}{0pt}%
\pgfpathmoveto{\pgfqpoint{2.701363in}{1.788591in}}%
\pgfpathlineto{\pgfqpoint{2.715371in}{1.773409in}}%
\pgfpathlineto{\pgfqpoint{2.729374in}{1.758465in}}%
\pgfpathlineto{\pgfqpoint{2.743371in}{1.743759in}}%
\pgfpathlineto{\pgfqpoint{2.757362in}{1.729289in}}%
\pgfpathlineto{\pgfqpoint{2.766215in}{1.727630in}}%
\pgfpathlineto{\pgfqpoint{2.775050in}{1.726306in}}%
\pgfpathlineto{\pgfqpoint{2.783867in}{1.725311in}}%
\pgfpathlineto{\pgfqpoint{2.792665in}{1.724637in}}%
\pgfpathlineto{\pgfqpoint{2.778721in}{1.738395in}}%
\pgfpathlineto{\pgfqpoint{2.764772in}{1.752388in}}%
\pgfpathlineto{\pgfqpoint{2.750817in}{1.766618in}}%
\pgfpathlineto{\pgfqpoint{2.736858in}{1.781086in}}%
\pgfpathlineto{\pgfqpoint{2.728013in}{1.782460in}}%
\pgfpathlineto{\pgfqpoint{2.719149in}{1.784164in}}%
\pgfpathlineto{\pgfqpoint{2.710266in}{1.786205in}}%
\pgfpathlineto{\pgfqpoint{2.701363in}{1.788591in}}%
\pgfpathclose%
\pgfusepath{fill}%
\end{pgfscope}%
\begin{pgfscope}%
\pgfpathrectangle{\pgfqpoint{1.150000in}{0.150000in}}{\pgfqpoint{5.700000in}{5.700000in}}%
\pgfusepath{clip}%
\pgfsetbuttcap%
\pgfsetroundjoin%
\definecolor{currentfill}{rgb}{0.253935,0.265254,0.529983}%
\pgfsetfillcolor{currentfill}%
\pgfsetfillopacity{0.800000}%
\pgfsetlinewidth{0.000000pt}%
\definecolor{currentstroke}{rgb}{0.000000,0.000000,0.000000}%
\pgfsetstrokecolor{currentstroke}%
\pgfsetdash{}{0pt}%
\pgfpathmoveto{\pgfqpoint{2.476271in}{2.065410in}}%
\pgfpathlineto{\pgfqpoint{2.490402in}{2.046167in}}%
\pgfpathlineto{\pgfqpoint{2.504524in}{2.027191in}}%
\pgfpathlineto{\pgfqpoint{2.518636in}{2.008482in}}%
\pgfpathlineto{\pgfqpoint{2.532740in}{1.990037in}}%
\pgfpathlineto{\pgfqpoint{2.541803in}{1.985443in}}%
\pgfpathlineto{\pgfqpoint{2.550845in}{1.981223in}}%
\pgfpathlineto{\pgfqpoint{2.559864in}{1.977368in}}%
\pgfpathlineto{\pgfqpoint{2.568863in}{1.973872in}}%
\pgfpathlineto{\pgfqpoint{2.554816in}{1.991588in}}%
\pgfpathlineto{\pgfqpoint{2.540760in}{2.009567in}}%
\pgfpathlineto{\pgfqpoint{2.526695in}{2.027811in}}%
\pgfpathlineto{\pgfqpoint{2.512622in}{2.046321in}}%
\pgfpathlineto{\pgfqpoint{2.503568in}{2.050535in}}%
\pgfpathlineto{\pgfqpoint{2.494491in}{2.055116in}}%
\pgfpathlineto{\pgfqpoint{2.485393in}{2.060072in}}%
\pgfpathlineto{\pgfqpoint{2.476271in}{2.065410in}}%
\pgfpathclose%
\pgfusepath{fill}%
\end{pgfscope}%
\begin{pgfscope}%
\pgfpathrectangle{\pgfqpoint{1.150000in}{0.150000in}}{\pgfqpoint{5.700000in}{5.700000in}}%
\pgfusepath{clip}%
\pgfsetbuttcap%
\pgfsetroundjoin%
\definecolor{currentfill}{rgb}{0.134692,0.658636,0.517649}%
\pgfsetfillcolor{currentfill}%
\pgfsetfillopacity{0.800000}%
\pgfsetlinewidth{0.000000pt}%
\definecolor{currentstroke}{rgb}{0.000000,0.000000,0.000000}%
\pgfsetstrokecolor{currentstroke}%
\pgfsetdash{}{0pt}%
\pgfpathmoveto{\pgfqpoint{5.356208in}{3.141725in}}%
\pgfpathlineto{\pgfqpoint{5.370983in}{3.156849in}}%
\pgfpathlineto{\pgfqpoint{5.385778in}{3.172159in}}%
\pgfpathlineto{\pgfqpoint{5.400596in}{3.187655in}}%
\pgfpathlineto{\pgfqpoint{5.415435in}{3.203337in}}%
\pgfpathlineto{\pgfqpoint{5.422959in}{3.208097in}}%
\pgfpathlineto{\pgfqpoint{5.430474in}{3.212722in}}%
\pgfpathlineto{\pgfqpoint{5.437978in}{3.217215in}}%
\pgfpathlineto{\pgfqpoint{5.445473in}{3.221580in}}%
\pgfpathlineto{\pgfqpoint{5.430649in}{3.206197in}}%
\pgfpathlineto{\pgfqpoint{5.415847in}{3.190999in}}%
\pgfpathlineto{\pgfqpoint{5.401066in}{3.175987in}}%
\pgfpathlineto{\pgfqpoint{5.386307in}{3.161159in}}%
\pgfpathlineto{\pgfqpoint{5.378796in}{3.156484in}}%
\pgfpathlineto{\pgfqpoint{5.371276in}{3.151689in}}%
\pgfpathlineto{\pgfqpoint{5.363747in}{3.146770in}}%
\pgfpathlineto{\pgfqpoint{5.356208in}{3.141725in}}%
\pgfpathclose%
\pgfusepath{fill}%
\end{pgfscope}%
\begin{pgfscope}%
\pgfpathrectangle{\pgfqpoint{1.150000in}{0.150000in}}{\pgfqpoint{5.700000in}{5.700000in}}%
\pgfusepath{clip}%
\pgfsetbuttcap%
\pgfsetroundjoin%
\definecolor{currentfill}{rgb}{0.277018,0.050344,0.375715}%
\pgfsetfillcolor{currentfill}%
\pgfsetfillopacity{0.800000}%
\pgfsetlinewidth{0.000000pt}%
\definecolor{currentstroke}{rgb}{0.000000,0.000000,0.000000}%
\pgfsetstrokecolor{currentstroke}%
\pgfsetdash{}{0pt}%
\pgfpathmoveto{\pgfqpoint{3.682174in}{1.503414in}}%
\pgfpathlineto{\pgfqpoint{3.696075in}{1.503556in}}%
\pgfpathlineto{\pgfqpoint{3.709983in}{1.503884in}}%
\pgfpathlineto{\pgfqpoint{3.723899in}{1.504399in}}%
\pgfpathlineto{\pgfqpoint{3.737823in}{1.505100in}}%
\pgfpathlineto{\pgfqpoint{3.746032in}{1.516858in}}%
\pgfpathlineto{\pgfqpoint{3.754236in}{1.528691in}}%
\pgfpathlineto{\pgfqpoint{3.762435in}{1.540595in}}%
\pgfpathlineto{\pgfqpoint{3.770627in}{1.552565in}}%
\pgfpathlineto{\pgfqpoint{3.756713in}{1.551361in}}%
\pgfpathlineto{\pgfqpoint{3.742808in}{1.550344in}}%
\pgfpathlineto{\pgfqpoint{3.728910in}{1.549514in}}%
\pgfpathlineto{\pgfqpoint{3.715020in}{1.548871in}}%
\pgfpathlineto{\pgfqpoint{3.706817in}{1.537392in}}%
\pgfpathlineto{\pgfqpoint{3.698609in}{1.525986in}}%
\pgfpathlineto{\pgfqpoint{3.690395in}{1.514658in}}%
\pgfpathlineto{\pgfqpoint{3.682174in}{1.503414in}}%
\pgfpathclose%
\pgfusepath{fill}%
\end{pgfscope}%
\begin{pgfscope}%
\pgfpathrectangle{\pgfqpoint{1.150000in}{0.150000in}}{\pgfqpoint{5.700000in}{5.700000in}}%
\pgfusepath{clip}%
\pgfsetbuttcap%
\pgfsetroundjoin%
\definecolor{currentfill}{rgb}{0.273809,0.031497,0.358853}%
\pgfsetfillcolor{currentfill}%
\pgfsetfillopacity{0.800000}%
\pgfsetlinewidth{0.000000pt}%
\definecolor{currentstroke}{rgb}{0.000000,0.000000,0.000000}%
\pgfsetstrokecolor{currentstroke}%
\pgfsetdash{}{0pt}%
\pgfpathmoveto{\pgfqpoint{3.070904in}{1.496777in}}%
\pgfpathlineto{\pgfqpoint{3.084799in}{1.487656in}}%
\pgfpathlineto{\pgfqpoint{3.098694in}{1.478743in}}%
\pgfpathlineto{\pgfqpoint{3.112589in}{1.470038in}}%
\pgfpathlineto{\pgfqpoint{3.126484in}{1.461539in}}%
\pgfpathlineto{\pgfqpoint{3.135032in}{1.465268in}}%
\pgfpathlineto{\pgfqpoint{3.143568in}{1.469252in}}%
\pgfpathlineto{\pgfqpoint{3.152091in}{1.473484in}}%
\pgfpathlineto{\pgfqpoint{3.160603in}{1.477958in}}%
\pgfpathlineto{\pgfqpoint{3.146740in}{1.485794in}}%
\pgfpathlineto{\pgfqpoint{3.132878in}{1.493835in}}%
\pgfpathlineto{\pgfqpoint{3.119016in}{1.502084in}}%
\pgfpathlineto{\pgfqpoint{3.105155in}{1.510541in}}%
\pgfpathlineto{\pgfqpoint{3.096612in}{1.506719in}}%
\pgfpathlineto{\pgfqpoint{3.088056in}{1.503146in}}%
\pgfpathlineto{\pgfqpoint{3.079487in}{1.499830in}}%
\pgfpathlineto{\pgfqpoint{3.070904in}{1.496777in}}%
\pgfpathclose%
\pgfusepath{fill}%
\end{pgfscope}%
\begin{pgfscope}%
\pgfpathrectangle{\pgfqpoint{1.150000in}{0.150000in}}{\pgfqpoint{5.700000in}{5.700000in}}%
\pgfusepath{clip}%
\pgfsetbuttcap%
\pgfsetroundjoin%
\definecolor{currentfill}{rgb}{0.223925,0.334994,0.548053}%
\pgfsetfillcolor{currentfill}%
\pgfsetfillopacity{0.800000}%
\pgfsetlinewidth{0.000000pt}%
\definecolor{currentstroke}{rgb}{0.000000,0.000000,0.000000}%
\pgfsetstrokecolor{currentstroke}%
\pgfsetdash{}{0pt}%
\pgfpathmoveto{\pgfqpoint{4.398446in}{2.153474in}}%
\pgfpathlineto{\pgfqpoint{4.412623in}{2.162410in}}%
\pgfpathlineto{\pgfqpoint{4.426814in}{2.171530in}}%
\pgfpathlineto{\pgfqpoint{4.441020in}{2.180835in}}%
\pgfpathlineto{\pgfqpoint{4.455242in}{2.190323in}}%
\pgfpathlineto{\pgfqpoint{4.463241in}{2.203567in}}%
\pgfpathlineto{\pgfqpoint{4.471234in}{2.216703in}}%
\pgfpathlineto{\pgfqpoint{4.479222in}{2.229728in}}%
\pgfpathlineto{\pgfqpoint{4.487205in}{2.242642in}}%
\pgfpathlineto{\pgfqpoint{4.472982in}{2.232966in}}%
\pgfpathlineto{\pgfqpoint{4.458775in}{2.223474in}}%
\pgfpathlineto{\pgfqpoint{4.444584in}{2.214167in}}%
\pgfpathlineto{\pgfqpoint{4.430407in}{2.205044in}}%
\pgfpathlineto{\pgfqpoint{4.422424in}{2.192305in}}%
\pgfpathlineto{\pgfqpoint{4.414437in}{2.179462in}}%
\pgfpathlineto{\pgfqpoint{4.406444in}{2.166518in}}%
\pgfpathlineto{\pgfqpoint{4.398446in}{2.153474in}}%
\pgfpathclose%
\pgfusepath{fill}%
\end{pgfscope}%
\begin{pgfscope}%
\pgfpathrectangle{\pgfqpoint{1.150000in}{0.150000in}}{\pgfqpoint{5.700000in}{5.700000in}}%
\pgfusepath{clip}%
\pgfsetbuttcap%
\pgfsetroundjoin%
\definecolor{currentfill}{rgb}{0.272594,0.025563,0.353093}%
\pgfsetfillcolor{currentfill}%
\pgfsetfillopacity{0.800000}%
\pgfsetlinewidth{0.000000pt}%
\definecolor{currentstroke}{rgb}{0.000000,0.000000,0.000000}%
\pgfsetstrokecolor{currentstroke}%
\pgfsetdash{}{0pt}%
\pgfpathmoveto{\pgfqpoint{3.593647in}{1.462812in}}%
\pgfpathlineto{\pgfqpoint{3.607533in}{1.461669in}}%
\pgfpathlineto{\pgfqpoint{3.621426in}{1.460714in}}%
\pgfpathlineto{\pgfqpoint{3.635325in}{1.459948in}}%
\pgfpathlineto{\pgfqpoint{3.649231in}{1.459369in}}%
\pgfpathlineto{\pgfqpoint{3.657476in}{1.470231in}}%
\pgfpathlineto{\pgfqpoint{3.665715in}{1.481196in}}%
\pgfpathlineto{\pgfqpoint{3.673948in}{1.492258in}}%
\pgfpathlineto{\pgfqpoint{3.682174in}{1.503414in}}%
\pgfpathlineto{\pgfqpoint{3.668281in}{1.503460in}}%
\pgfpathlineto{\pgfqpoint{3.654394in}{1.503694in}}%
\pgfpathlineto{\pgfqpoint{3.640515in}{1.504116in}}%
\pgfpathlineto{\pgfqpoint{3.626642in}{1.504726in}}%
\pgfpathlineto{\pgfqpoint{3.618403in}{1.494091in}}%
\pgfpathlineto{\pgfqpoint{3.610158in}{1.483557in}}%
\pgfpathlineto{\pgfqpoint{3.601906in}{1.473129in}}%
\pgfpathlineto{\pgfqpoint{3.593647in}{1.462812in}}%
\pgfpathclose%
\pgfusepath{fill}%
\end{pgfscope}%
\begin{pgfscope}%
\pgfpathrectangle{\pgfqpoint{1.150000in}{0.150000in}}{\pgfqpoint{5.700000in}{5.700000in}}%
\pgfusepath{clip}%
\pgfsetbuttcap%
\pgfsetroundjoin%
\definecolor{currentfill}{rgb}{0.283072,0.130895,0.449241}%
\pgfsetfillcolor{currentfill}%
\pgfsetfillopacity{0.800000}%
\pgfsetlinewidth{0.000000pt}%
\definecolor{currentstroke}{rgb}{0.000000,0.000000,0.000000}%
\pgfsetstrokecolor{currentstroke}%
\pgfsetdash{}{0pt}%
\pgfpathmoveto{\pgfqpoint{2.757362in}{1.729289in}}%
\pgfpathlineto{\pgfqpoint{2.771349in}{1.715053in}}%
\pgfpathlineto{\pgfqpoint{2.785331in}{1.701050in}}%
\pgfpathlineto{\pgfqpoint{2.799309in}{1.687279in}}%
\pgfpathlineto{\pgfqpoint{2.813282in}{1.673739in}}%
\pgfpathlineto{\pgfqpoint{2.822088in}{1.672803in}}%
\pgfpathlineto{\pgfqpoint{2.830876in}{1.672195in}}%
\pgfpathlineto{\pgfqpoint{2.839647in}{1.671905in}}%
\pgfpathlineto{\pgfqpoint{2.848401in}{1.671927in}}%
\pgfpathlineto{\pgfqpoint{2.834473in}{1.684759in}}%
\pgfpathlineto{\pgfqpoint{2.820541in}{1.697821in}}%
\pgfpathlineto{\pgfqpoint{2.806605in}{1.711113in}}%
\pgfpathlineto{\pgfqpoint{2.792665in}{1.724637in}}%
\pgfpathlineto{\pgfqpoint{2.783867in}{1.725311in}}%
\pgfpathlineto{\pgfqpoint{2.775050in}{1.726306in}}%
\pgfpathlineto{\pgfqpoint{2.766215in}{1.727630in}}%
\pgfpathlineto{\pgfqpoint{2.757362in}{1.729289in}}%
\pgfpathclose%
\pgfusepath{fill}%
\end{pgfscope}%
\begin{pgfscope}%
\pgfpathrectangle{\pgfqpoint{1.150000in}{0.150000in}}{\pgfqpoint{5.700000in}{5.700000in}}%
\pgfusepath{clip}%
\pgfsetbuttcap%
\pgfsetroundjoin%
\definecolor{currentfill}{rgb}{0.168126,0.459988,0.558082}%
\pgfsetfillcolor{currentfill}%
\pgfsetfillopacity{0.800000}%
\pgfsetlinewidth{0.000000pt}%
\definecolor{currentstroke}{rgb}{0.000000,0.000000,0.000000}%
\pgfsetstrokecolor{currentstroke}%
\pgfsetdash{}{0pt}%
\pgfpathmoveto{\pgfqpoint{4.728490in}{2.520676in}}%
\pgfpathlineto{\pgfqpoint{4.742859in}{2.532431in}}%
\pgfpathlineto{\pgfqpoint{4.757245in}{2.544371in}}%
\pgfpathlineto{\pgfqpoint{4.771648in}{2.556496in}}%
\pgfpathlineto{\pgfqpoint{4.786069in}{2.568807in}}%
\pgfpathlineto{\pgfqpoint{4.793947in}{2.579791in}}%
\pgfpathlineto{\pgfqpoint{4.801817in}{2.590627in}}%
\pgfpathlineto{\pgfqpoint{4.809681in}{2.601316in}}%
\pgfpathlineto{\pgfqpoint{4.817537in}{2.611857in}}%
\pgfpathlineto{\pgfqpoint{4.803118in}{2.599528in}}%
\pgfpathlineto{\pgfqpoint{4.788716in}{2.587383in}}%
\pgfpathlineto{\pgfqpoint{4.774332in}{2.575424in}}%
\pgfpathlineto{\pgfqpoint{4.759966in}{2.563649in}}%
\pgfpathlineto{\pgfqpoint{4.752107in}{2.553114in}}%
\pgfpathlineto{\pgfqpoint{4.744242in}{2.542440in}}%
\pgfpathlineto{\pgfqpoint{4.736369in}{2.531628in}}%
\pgfpathlineto{\pgfqpoint{4.728490in}{2.520676in}}%
\pgfpathclose%
\pgfusepath{fill}%
\end{pgfscope}%
\begin{pgfscope}%
\pgfpathrectangle{\pgfqpoint{1.150000in}{0.150000in}}{\pgfqpoint{5.700000in}{5.700000in}}%
\pgfusepath{clip}%
\pgfsetbuttcap%
\pgfsetroundjoin%
\definecolor{currentfill}{rgb}{0.280894,0.078907,0.402329}%
\pgfsetfillcolor{currentfill}%
\pgfsetfillopacity{0.800000}%
\pgfsetlinewidth{0.000000pt}%
\definecolor{currentstroke}{rgb}{0.000000,0.000000,0.000000}%
\pgfsetstrokecolor{currentstroke}%
\pgfsetdash{}{0pt}%
\pgfpathmoveto{\pgfqpoint{3.770627in}{1.552565in}}%
\pgfpathlineto{\pgfqpoint{3.784550in}{1.553954in}}%
\pgfpathlineto{\pgfqpoint{3.798480in}{1.555529in}}%
\pgfpathlineto{\pgfqpoint{3.812419in}{1.557290in}}%
\pgfpathlineto{\pgfqpoint{3.826367in}{1.559235in}}%
\pgfpathlineto{\pgfqpoint{3.834547in}{1.571749in}}%
\pgfpathlineto{\pgfqpoint{3.842721in}{1.584312in}}%
\pgfpathlineto{\pgfqpoint{3.850890in}{1.596919in}}%
\pgfpathlineto{\pgfqpoint{3.859053in}{1.609567in}}%
\pgfpathlineto{\pgfqpoint{3.845113in}{1.607149in}}%
\pgfpathlineto{\pgfqpoint{3.831181in}{1.604917in}}%
\pgfpathlineto{\pgfqpoint{3.817258in}{1.602871in}}%
\pgfpathlineto{\pgfqpoint{3.803344in}{1.601010in}}%
\pgfpathlineto{\pgfqpoint{3.795173in}{1.588823in}}%
\pgfpathlineto{\pgfqpoint{3.786997in}{1.576683in}}%
\pgfpathlineto{\pgfqpoint{3.778815in}{1.564595in}}%
\pgfpathlineto{\pgfqpoint{3.770627in}{1.552565in}}%
\pgfpathclose%
\pgfusepath{fill}%
\end{pgfscope}%
\begin{pgfscope}%
\pgfpathrectangle{\pgfqpoint{1.150000in}{0.150000in}}{\pgfqpoint{5.700000in}{5.700000in}}%
\pgfusepath{clip}%
\pgfsetbuttcap%
\pgfsetroundjoin%
\definecolor{currentfill}{rgb}{0.281477,0.755203,0.432552}%
\pgfsetfillcolor{currentfill}%
\pgfsetfillopacity{0.800000}%
\pgfsetlinewidth{0.000000pt}%
\definecolor{currentstroke}{rgb}{0.000000,0.000000,0.000000}%
\pgfsetstrokecolor{currentstroke}%
\pgfsetdash{}{0pt}%
\pgfpathmoveto{\pgfqpoint{5.742758in}{3.457719in}}%
\pgfpathlineto{\pgfqpoint{5.757792in}{3.473961in}}%
\pgfpathlineto{\pgfqpoint{5.772849in}{3.490388in}}%
\pgfpathlineto{\pgfqpoint{5.787930in}{3.507001in}}%
\pgfpathlineto{\pgfqpoint{5.803035in}{3.523799in}}%
\pgfpathlineto{\pgfqpoint{5.810275in}{3.524764in}}%
\pgfpathlineto{\pgfqpoint{5.817505in}{3.525646in}}%
\pgfpathlineto{\pgfqpoint{5.824725in}{3.526451in}}%
\pgfpathlineto{\pgfqpoint{5.831936in}{3.527183in}}%
\pgfpathlineto{\pgfqpoint{5.816858in}{3.510864in}}%
\pgfpathlineto{\pgfqpoint{5.801805in}{3.494731in}}%
\pgfpathlineto{\pgfqpoint{5.786774in}{3.478782in}}%
\pgfpathlineto{\pgfqpoint{5.771767in}{3.463017in}}%
\pgfpathlineto{\pgfqpoint{5.764529in}{3.461795in}}%
\pgfpathlineto{\pgfqpoint{5.757281in}{3.460508in}}%
\pgfpathlineto{\pgfqpoint{5.750025in}{3.459151in}}%
\pgfpathlineto{\pgfqpoint{5.742758in}{3.457719in}}%
\pgfpathclose%
\pgfusepath{fill}%
\end{pgfscope}%
\begin{pgfscope}%
\pgfpathrectangle{\pgfqpoint{1.150000in}{0.150000in}}{\pgfqpoint{5.700000in}{5.700000in}}%
\pgfusepath{clip}%
\pgfsetbuttcap%
\pgfsetroundjoin%
\definecolor{currentfill}{rgb}{0.275191,0.194905,0.496005}%
\pgfsetfillcolor{currentfill}%
\pgfsetfillopacity{0.800000}%
\pgfsetlinewidth{0.000000pt}%
\definecolor{currentstroke}{rgb}{0.000000,0.000000,0.000000}%
\pgfsetstrokecolor{currentstroke}%
\pgfsetdash{}{0pt}%
\pgfpathmoveto{\pgfqpoint{4.068409in}{1.798784in}}%
\pgfpathlineto{\pgfqpoint{4.082430in}{1.804091in}}%
\pgfpathlineto{\pgfqpoint{4.096463in}{1.809583in}}%
\pgfpathlineto{\pgfqpoint{4.110508in}{1.815258in}}%
\pgfpathlineto{\pgfqpoint{4.124565in}{1.821116in}}%
\pgfpathlineto{\pgfqpoint{4.132657in}{1.834999in}}%
\pgfpathlineto{\pgfqpoint{4.140745in}{1.848844in}}%
\pgfpathlineto{\pgfqpoint{4.148829in}{1.862650in}}%
\pgfpathlineto{\pgfqpoint{4.156908in}{1.876413in}}%
\pgfpathlineto{\pgfqpoint{4.142852in}{1.870207in}}%
\pgfpathlineto{\pgfqpoint{4.128809in}{1.864184in}}%
\pgfpathlineto{\pgfqpoint{4.114778in}{1.858345in}}%
\pgfpathlineto{\pgfqpoint{4.100758in}{1.852690in}}%
\pgfpathlineto{\pgfqpoint{4.092678in}{1.839263in}}%
\pgfpathlineto{\pgfqpoint{4.084593in}{1.825800in}}%
\pgfpathlineto{\pgfqpoint{4.076503in}{1.812307in}}%
\pgfpathlineto{\pgfqpoint{4.068409in}{1.798784in}}%
\pgfpathclose%
\pgfusepath{fill}%
\end{pgfscope}%
\begin{pgfscope}%
\pgfpathrectangle{\pgfqpoint{1.150000in}{0.150000in}}{\pgfqpoint{5.700000in}{5.700000in}}%
\pgfusepath{clip}%
\pgfsetbuttcap%
\pgfsetroundjoin%
\definecolor{currentfill}{rgb}{0.241237,0.296485,0.539709}%
\pgfsetfillcolor{currentfill}%
\pgfsetfillopacity{0.800000}%
\pgfsetlinewidth{0.000000pt}%
\definecolor{currentstroke}{rgb}{0.000000,0.000000,0.000000}%
\pgfsetstrokecolor{currentstroke}%
\pgfsetdash{}{0pt}%
\pgfpathmoveto{\pgfqpoint{2.419643in}{2.145109in}}%
\pgfpathlineto{\pgfqpoint{2.433816in}{2.124771in}}%
\pgfpathlineto{\pgfqpoint{2.447978in}{2.104710in}}%
\pgfpathlineto{\pgfqpoint{2.462130in}{2.084924in}}%
\pgfpathlineto{\pgfqpoint{2.476271in}{2.065410in}}%
\pgfpathlineto{\pgfqpoint{2.485393in}{2.060072in}}%
\pgfpathlineto{\pgfqpoint{2.494491in}{2.055116in}}%
\pgfpathlineto{\pgfqpoint{2.503568in}{2.050535in}}%
\pgfpathlineto{\pgfqpoint{2.512622in}{2.046321in}}%
\pgfpathlineto{\pgfqpoint{2.498539in}{2.065101in}}%
\pgfpathlineto{\pgfqpoint{2.484446in}{2.084152in}}%
\pgfpathlineto{\pgfqpoint{2.470344in}{2.103475in}}%
\pgfpathlineto{\pgfqpoint{2.456231in}{2.123074in}}%
\pgfpathlineto{\pgfqpoint{2.447119in}{2.128009in}}%
\pgfpathlineto{\pgfqpoint{2.437984in}{2.133322in}}%
\pgfpathlineto{\pgfqpoint{2.428826in}{2.139019in}}%
\pgfpathlineto{\pgfqpoint{2.419643in}{2.145109in}}%
\pgfpathclose%
\pgfusepath{fill}%
\end{pgfscope}%
\begin{pgfscope}%
\pgfpathrectangle{\pgfqpoint{1.150000in}{0.150000in}}{\pgfqpoint{5.700000in}{5.700000in}}%
\pgfusepath{clip}%
\pgfsetbuttcap%
\pgfsetroundjoin%
\definecolor{currentfill}{rgb}{0.269944,0.014625,0.341379}%
\pgfsetfillcolor{currentfill}%
\pgfsetfillopacity{0.800000}%
\pgfsetlinewidth{0.000000pt}%
\definecolor{currentstroke}{rgb}{0.000000,0.000000,0.000000}%
\pgfsetstrokecolor{currentstroke}%
\pgfsetdash{}{0pt}%
\pgfpathmoveto{\pgfqpoint{3.504994in}{1.431482in}}%
\pgfpathlineto{\pgfqpoint{3.518873in}{1.429016in}}%
\pgfpathlineto{\pgfqpoint{3.532757in}{1.426742in}}%
\pgfpathlineto{\pgfqpoint{3.546647in}{1.424657in}}%
\pgfpathlineto{\pgfqpoint{3.560543in}{1.422761in}}%
\pgfpathlineto{\pgfqpoint{3.568829in}{1.432580in}}%
\pgfpathlineto{\pgfqpoint{3.577109in}{1.442532in}}%
\pgfpathlineto{\pgfqpoint{3.585382in}{1.452611in}}%
\pgfpathlineto{\pgfqpoint{3.593647in}{1.462812in}}%
\pgfpathlineto{\pgfqpoint{3.579767in}{1.464143in}}%
\pgfpathlineto{\pgfqpoint{3.565893in}{1.465665in}}%
\pgfpathlineto{\pgfqpoint{3.552025in}{1.467376in}}%
\pgfpathlineto{\pgfqpoint{3.538163in}{1.469277in}}%
\pgfpathlineto{\pgfqpoint{3.529882in}{1.459628in}}%
\pgfpathlineto{\pgfqpoint{3.521594in}{1.450109in}}%
\pgfpathlineto{\pgfqpoint{3.513298in}{1.440725in}}%
\pgfpathlineto{\pgfqpoint{3.504994in}{1.431482in}}%
\pgfpathclose%
\pgfusepath{fill}%
\end{pgfscope}%
\begin{pgfscope}%
\pgfpathrectangle{\pgfqpoint{1.150000in}{0.150000in}}{\pgfqpoint{5.700000in}{5.700000in}}%
\pgfusepath{clip}%
\pgfsetbuttcap%
\pgfsetroundjoin%
\definecolor{currentfill}{rgb}{0.283091,0.110553,0.431554}%
\pgfsetfillcolor{currentfill}%
\pgfsetfillopacity{0.800000}%
\pgfsetlinewidth{0.000000pt}%
\definecolor{currentstroke}{rgb}{0.000000,0.000000,0.000000}%
\pgfsetstrokecolor{currentstroke}%
\pgfsetdash{}{0pt}%
\pgfpathmoveto{\pgfqpoint{2.813282in}{1.673739in}}%
\pgfpathlineto{\pgfqpoint{2.827251in}{1.660427in}}%
\pgfpathlineto{\pgfqpoint{2.841216in}{1.647342in}}%
\pgfpathlineto{\pgfqpoint{2.855177in}{1.634484in}}%
\pgfpathlineto{\pgfqpoint{2.869135in}{1.621850in}}%
\pgfpathlineto{\pgfqpoint{2.877896in}{1.621635in}}%
\pgfpathlineto{\pgfqpoint{2.886641in}{1.621738in}}%
\pgfpathlineto{\pgfqpoint{2.895368in}{1.622151in}}%
\pgfpathlineto{\pgfqpoint{2.904080in}{1.622867in}}%
\pgfpathlineto{\pgfqpoint{2.890165in}{1.634795in}}%
\pgfpathlineto{\pgfqpoint{2.876247in}{1.646946in}}%
\pgfpathlineto{\pgfqpoint{2.862326in}{1.659324in}}%
\pgfpathlineto{\pgfqpoint{2.848401in}{1.671927in}}%
\pgfpathlineto{\pgfqpoint{2.839647in}{1.671905in}}%
\pgfpathlineto{\pgfqpoint{2.830876in}{1.672195in}}%
\pgfpathlineto{\pgfqpoint{2.822088in}{1.672803in}}%
\pgfpathlineto{\pgfqpoint{2.813282in}{1.673739in}}%
\pgfpathclose%
\pgfusepath{fill}%
\end{pgfscope}%
\begin{pgfscope}%
\pgfpathrectangle{\pgfqpoint{1.150000in}{0.150000in}}{\pgfqpoint{5.700000in}{5.700000in}}%
\pgfusepath{clip}%
\pgfsetbuttcap%
\pgfsetroundjoin%
\definecolor{currentfill}{rgb}{0.120092,0.600104,0.542530}%
\pgfsetfillcolor{currentfill}%
\pgfsetfillopacity{0.800000}%
\pgfsetlinewidth{0.000000pt}%
\definecolor{currentstroke}{rgb}{0.000000,0.000000,0.000000}%
\pgfsetstrokecolor{currentstroke}%
\pgfsetdash{}{0pt}%
\pgfpathmoveto{\pgfqpoint{5.147284in}{2.950433in}}%
\pgfpathlineto{\pgfqpoint{5.161926in}{2.964750in}}%
\pgfpathlineto{\pgfqpoint{5.176588in}{2.979253in}}%
\pgfpathlineto{\pgfqpoint{5.191271in}{2.993942in}}%
\pgfpathlineto{\pgfqpoint{5.205974in}{3.008818in}}%
\pgfpathlineto{\pgfqpoint{5.213637in}{3.015797in}}%
\pgfpathlineto{\pgfqpoint{5.221291in}{3.022623in}}%
\pgfpathlineto{\pgfqpoint{5.228936in}{3.029300in}}%
\pgfpathlineto{\pgfqpoint{5.236571in}{3.035830in}}%
\pgfpathlineto{\pgfqpoint{5.221877in}{3.021146in}}%
\pgfpathlineto{\pgfqpoint{5.207204in}{3.006648in}}%
\pgfpathlineto{\pgfqpoint{5.192551in}{2.992335in}}%
\pgfpathlineto{\pgfqpoint{5.177919in}{2.978208in}}%
\pgfpathlineto{\pgfqpoint{5.170274in}{2.971475in}}%
\pgfpathlineto{\pgfqpoint{5.162619in}{2.964603in}}%
\pgfpathlineto{\pgfqpoint{5.154956in}{2.957590in}}%
\pgfpathlineto{\pgfqpoint{5.147284in}{2.950433in}}%
\pgfpathclose%
\pgfusepath{fill}%
\end{pgfscope}%
\begin{pgfscope}%
\pgfpathrectangle{\pgfqpoint{1.150000in}{0.150000in}}{\pgfqpoint{5.700000in}{5.700000in}}%
\pgfusepath{clip}%
\pgfsetbuttcap%
\pgfsetroundjoin%
\definecolor{currentfill}{rgb}{0.139147,0.533812,0.555298}%
\pgfsetfillcolor{currentfill}%
\pgfsetfillopacity{0.800000}%
\pgfsetlinewidth{0.000000pt}%
\definecolor{currentstroke}{rgb}{0.000000,0.000000,0.000000}%
\pgfsetstrokecolor{currentstroke}%
\pgfsetdash{}{0pt}%
\pgfpathmoveto{\pgfqpoint{4.937959in}{2.741878in}}%
\pgfpathlineto{\pgfqpoint{4.952464in}{2.755070in}}%
\pgfpathlineto{\pgfqpoint{4.966988in}{2.768447in}}%
\pgfpathlineto{\pgfqpoint{4.981531in}{2.782010in}}%
\pgfpathlineto{\pgfqpoint{4.996093in}{2.795759in}}%
\pgfpathlineto{\pgfqpoint{5.003874in}{2.804863in}}%
\pgfpathlineto{\pgfqpoint{5.011647in}{2.813811in}}%
\pgfpathlineto{\pgfqpoint{5.019411in}{2.822603in}}%
\pgfpathlineto{\pgfqpoint{5.027167in}{2.831242in}}%
\pgfpathlineto{\pgfqpoint{5.012610in}{2.817579in}}%
\pgfpathlineto{\pgfqpoint{4.998072in}{2.804101in}}%
\pgfpathlineto{\pgfqpoint{4.983554in}{2.790808in}}%
\pgfpathlineto{\pgfqpoint{4.969054in}{2.777700in}}%
\pgfpathlineto{\pgfqpoint{4.961292in}{2.768964in}}%
\pgfpathlineto{\pgfqpoint{4.953522in}{2.760082in}}%
\pgfpathlineto{\pgfqpoint{4.945745in}{2.751054in}}%
\pgfpathlineto{\pgfqpoint{4.937959in}{2.741878in}}%
\pgfpathclose%
\pgfusepath{fill}%
\end{pgfscope}%
\begin{pgfscope}%
\pgfpathrectangle{\pgfqpoint{1.150000in}{0.150000in}}{\pgfqpoint{5.700000in}{5.700000in}}%
\pgfusepath{clip}%
\pgfsetbuttcap%
\pgfsetroundjoin%
\definecolor{currentfill}{rgb}{0.283091,0.110553,0.431554}%
\pgfsetfillcolor{currentfill}%
\pgfsetfillopacity{0.800000}%
\pgfsetlinewidth{0.000000pt}%
\definecolor{currentstroke}{rgb}{0.000000,0.000000,0.000000}%
\pgfsetstrokecolor{currentstroke}%
\pgfsetdash{}{0pt}%
\pgfpathmoveto{\pgfqpoint{3.859053in}{1.609567in}}%
\pgfpathlineto{\pgfqpoint{3.873004in}{1.612169in}}%
\pgfpathlineto{\pgfqpoint{3.886963in}{1.614956in}}%
\pgfpathlineto{\pgfqpoint{3.900932in}{1.617928in}}%
\pgfpathlineto{\pgfqpoint{3.914911in}{1.621084in}}%
\pgfpathlineto{\pgfqpoint{3.923064in}{1.634219in}}%
\pgfpathlineto{\pgfqpoint{3.931212in}{1.647379in}}%
\pgfpathlineto{\pgfqpoint{3.939356in}{1.660557in}}%
\pgfpathlineto{\pgfqpoint{3.947494in}{1.673751in}}%
\pgfpathlineto{\pgfqpoint{3.933521in}{1.670154in}}%
\pgfpathlineto{\pgfqpoint{3.919557in}{1.666741in}}%
\pgfpathlineto{\pgfqpoint{3.905603in}{1.663513in}}%
\pgfpathlineto{\pgfqpoint{3.891659in}{1.660470in}}%
\pgfpathlineto{\pgfqpoint{3.883515in}{1.647705in}}%
\pgfpathlineto{\pgfqpoint{3.875366in}{1.634964in}}%
\pgfpathlineto{\pgfqpoint{3.867212in}{1.622249in}}%
\pgfpathlineto{\pgfqpoint{3.859053in}{1.609567in}}%
\pgfpathclose%
\pgfusepath{fill}%
\end{pgfscope}%
\begin{pgfscope}%
\pgfpathrectangle{\pgfqpoint{1.150000in}{0.150000in}}{\pgfqpoint{5.700000in}{5.700000in}}%
\pgfusepath{clip}%
\pgfsetbuttcap%
\pgfsetroundjoin%
\definecolor{currentfill}{rgb}{0.246811,0.283237,0.535941}%
\pgfsetfillcolor{currentfill}%
\pgfsetfillopacity{0.800000}%
\pgfsetlinewidth{0.000000pt}%
\definecolor{currentstroke}{rgb}{0.000000,0.000000,0.000000}%
\pgfsetstrokecolor{currentstroke}%
\pgfsetdash{}{0pt}%
\pgfpathmoveto{\pgfqpoint{4.277720in}{2.013706in}}%
\pgfpathlineto{\pgfqpoint{4.291840in}{2.021433in}}%
\pgfpathlineto{\pgfqpoint{4.305973in}{2.029344in}}%
\pgfpathlineto{\pgfqpoint{4.320120in}{2.037439in}}%
\pgfpathlineto{\pgfqpoint{4.334281in}{2.045718in}}%
\pgfpathlineto{\pgfqpoint{4.342319in}{2.059500in}}%
\pgfpathlineto{\pgfqpoint{4.350352in}{2.073197in}}%
\pgfpathlineto{\pgfqpoint{4.358380in}{2.086807in}}%
\pgfpathlineto{\pgfqpoint{4.366403in}{2.100327in}}%
\pgfpathlineto{\pgfqpoint{4.352241in}{2.091795in}}%
\pgfpathlineto{\pgfqpoint{4.338094in}{2.083448in}}%
\pgfpathlineto{\pgfqpoint{4.323960in}{2.075284in}}%
\pgfpathlineto{\pgfqpoint{4.309841in}{2.067305in}}%
\pgfpathlineto{\pgfqpoint{4.301818in}{2.054025in}}%
\pgfpathlineto{\pgfqpoint{4.293790in}{2.040664in}}%
\pgfpathlineto{\pgfqpoint{4.285758in}{2.027223in}}%
\pgfpathlineto{\pgfqpoint{4.277720in}{2.013706in}}%
\pgfpathclose%
\pgfusepath{fill}%
\end{pgfscope}%
\begin{pgfscope}%
\pgfpathrectangle{\pgfqpoint{1.150000in}{0.150000in}}{\pgfqpoint{5.700000in}{5.700000in}}%
\pgfusepath{clip}%
\pgfsetbuttcap%
\pgfsetroundjoin%
\definecolor{currentfill}{rgb}{0.267004,0.004874,0.329415}%
\pgfsetfillcolor{currentfill}%
\pgfsetfillopacity{0.800000}%
\pgfsetlinewidth{0.000000pt}%
\definecolor{currentstroke}{rgb}{0.000000,0.000000,0.000000}%
\pgfsetstrokecolor{currentstroke}%
\pgfsetdash{}{0pt}%
\pgfpathmoveto{\pgfqpoint{3.271551in}{1.422592in}}%
\pgfpathlineto{\pgfqpoint{3.285428in}{1.416573in}}%
\pgfpathlineto{\pgfqpoint{3.299307in}{1.410753in}}%
\pgfpathlineto{\pgfqpoint{3.313189in}{1.405129in}}%
\pgfpathlineto{\pgfqpoint{3.327074in}{1.399702in}}%
\pgfpathlineto{\pgfqpoint{3.335490in}{1.406334in}}%
\pgfpathlineto{\pgfqpoint{3.343896in}{1.413169in}}%
\pgfpathlineto{\pgfqpoint{3.352293in}{1.420203in}}%
\pgfpathlineto{\pgfqpoint{3.360680in}{1.427427in}}%
\pgfpathlineto{\pgfqpoint{3.346819in}{1.432226in}}%
\pgfpathlineto{\pgfqpoint{3.332962in}{1.437222in}}%
\pgfpathlineto{\pgfqpoint{3.319108in}{1.442414in}}%
\pgfpathlineto{\pgfqpoint{3.305257in}{1.447805in}}%
\pgfpathlineto{\pgfqpoint{3.296846in}{1.441196in}}%
\pgfpathlineto{\pgfqpoint{3.288424in}{1.434787in}}%
\pgfpathlineto{\pgfqpoint{3.279993in}{1.428583in}}%
\pgfpathlineto{\pgfqpoint{3.271551in}{1.422592in}}%
\pgfpathclose%
\pgfusepath{fill}%
\end{pgfscope}%
\begin{pgfscope}%
\pgfpathrectangle{\pgfqpoint{1.150000in}{0.150000in}}{\pgfqpoint{5.700000in}{5.700000in}}%
\pgfusepath{clip}%
\pgfsetbuttcap%
\pgfsetroundjoin%
\definecolor{currentfill}{rgb}{0.225863,0.330805,0.547314}%
\pgfsetfillcolor{currentfill}%
\pgfsetfillopacity{0.800000}%
\pgfsetlinewidth{0.000000pt}%
\definecolor{currentstroke}{rgb}{0.000000,0.000000,0.000000}%
\pgfsetstrokecolor{currentstroke}%
\pgfsetdash{}{0pt}%
\pgfpathmoveto{\pgfqpoint{2.362840in}{2.229273in}}%
\pgfpathlineto{\pgfqpoint{2.377058in}{2.207805in}}%
\pgfpathlineto{\pgfqpoint{2.391265in}{2.186623in}}%
\pgfpathlineto{\pgfqpoint{2.405460in}{2.165725in}}%
\pgfpathlineto{\pgfqpoint{2.419643in}{2.145109in}}%
\pgfpathlineto{\pgfqpoint{2.428826in}{2.139019in}}%
\pgfpathlineto{\pgfqpoint{2.437984in}{2.133322in}}%
\pgfpathlineto{\pgfqpoint{2.447119in}{2.128009in}}%
\pgfpathlineto{\pgfqpoint{2.456231in}{2.123074in}}%
\pgfpathlineto{\pgfqpoint{2.442108in}{2.142951in}}%
\pgfpathlineto{\pgfqpoint{2.427974in}{2.163107in}}%
\pgfpathlineto{\pgfqpoint{2.413830in}{2.183546in}}%
\pgfpathlineto{\pgfqpoint{2.399674in}{2.204269in}}%
\pgfpathlineto{\pgfqpoint{2.390502in}{2.209932in}}%
\pgfpathlineto{\pgfqpoint{2.381306in}{2.215981in}}%
\pgfpathlineto{\pgfqpoint{2.372085in}{2.222426in}}%
\pgfpathlineto{\pgfqpoint{2.362840in}{2.229273in}}%
\pgfpathclose%
\pgfusepath{fill}%
\end{pgfscope}%
\begin{pgfscope}%
\pgfpathrectangle{\pgfqpoint{1.150000in}{0.150000in}}{\pgfqpoint{5.700000in}{5.700000in}}%
\pgfusepath{clip}%
\pgfsetbuttcap%
\pgfsetroundjoin%
\definecolor{currentfill}{rgb}{0.185556,0.418570,0.556753}%
\pgfsetfillcolor{currentfill}%
\pgfsetfillopacity{0.800000}%
\pgfsetlinewidth{0.000000pt}%
\definecolor{currentstroke}{rgb}{0.000000,0.000000,0.000000}%
\pgfsetstrokecolor{currentstroke}%
\pgfsetdash{}{0pt}%
\pgfpathmoveto{\pgfqpoint{4.607912in}{2.383452in}}%
\pgfpathlineto{\pgfqpoint{4.622214in}{2.394327in}}%
\pgfpathlineto{\pgfqpoint{4.636533in}{2.405386in}}%
\pgfpathlineto{\pgfqpoint{4.650869in}{2.416630in}}%
\pgfpathlineto{\pgfqpoint{4.665221in}{2.428059in}}%
\pgfpathlineto{\pgfqpoint{4.673152in}{2.440120in}}%
\pgfpathlineto{\pgfqpoint{4.681077in}{2.452044in}}%
\pgfpathlineto{\pgfqpoint{4.688996in}{2.463830in}}%
\pgfpathlineto{\pgfqpoint{4.696908in}{2.475478in}}%
\pgfpathlineto{\pgfqpoint{4.682555in}{2.463961in}}%
\pgfpathlineto{\pgfqpoint{4.668220in}{2.452630in}}%
\pgfpathlineto{\pgfqpoint{4.653902in}{2.441483in}}%
\pgfpathlineto{\pgfqpoint{4.639600in}{2.430521in}}%
\pgfpathlineto{\pgfqpoint{4.631687in}{2.418949in}}%
\pgfpathlineto{\pgfqpoint{4.623768in}{2.407246in}}%
\pgfpathlineto{\pgfqpoint{4.615843in}{2.395414in}}%
\pgfpathlineto{\pgfqpoint{4.607912in}{2.383452in}}%
\pgfpathclose%
\pgfusepath{fill}%
\end{pgfscope}%
\begin{pgfscope}%
\pgfpathrectangle{\pgfqpoint{1.150000in}{0.150000in}}{\pgfqpoint{5.700000in}{5.700000in}}%
\pgfusepath{clip}%
\pgfsetbuttcap%
\pgfsetroundjoin%
\definecolor{currentfill}{rgb}{0.160665,0.478540,0.558115}%
\pgfsetfillcolor{currentfill}%
\pgfsetfillopacity{0.800000}%
\pgfsetlinewidth{0.000000pt}%
\definecolor{currentstroke}{rgb}{0.000000,0.000000,0.000000}%
\pgfsetstrokecolor{currentstroke}%
\pgfsetdash{}{0pt}%
\pgfpathmoveto{\pgfqpoint{2.113702in}{2.682534in}}%
\pgfpathlineto{\pgfqpoint{2.128151in}{2.655552in}}%
\pgfpathlineto{\pgfqpoint{2.142582in}{2.628911in}}%
\pgfpathlineto{\pgfqpoint{2.156995in}{2.602609in}}%
\pgfpathlineto{\pgfqpoint{2.171391in}{2.576643in}}%
\pgfpathlineto{\pgfqpoint{2.180801in}{2.568424in}}%
\pgfpathlineto{\pgfqpoint{2.190183in}{2.560617in}}%
\pgfpathlineto{\pgfqpoint{2.199539in}{2.553216in}}%
\pgfpathlineto{\pgfqpoint{2.208868in}{2.546212in}}%
\pgfpathlineto{\pgfqpoint{2.194541in}{2.571444in}}%
\pgfpathlineto{\pgfqpoint{2.180198in}{2.597010in}}%
\pgfpathlineto{\pgfqpoint{2.165837in}{2.622913in}}%
\pgfpathlineto{\pgfqpoint{2.151459in}{2.649155in}}%
\pgfpathlineto{\pgfqpoint{2.142061in}{2.656880in}}%
\pgfpathlineto{\pgfqpoint{2.132636in}{2.665014in}}%
\pgfpathlineto{\pgfqpoint{2.123183in}{2.673562in}}%
\pgfpathlineto{\pgfqpoint{2.113702in}{2.682534in}}%
\pgfpathclose%
\pgfusepath{fill}%
\end{pgfscope}%
\begin{pgfscope}%
\pgfpathrectangle{\pgfqpoint{1.150000in}{0.150000in}}{\pgfqpoint{5.700000in}{5.700000in}}%
\pgfusepath{clip}%
\pgfsetbuttcap%
\pgfsetroundjoin%
\definecolor{currentfill}{rgb}{0.327796,0.773980,0.406640}%
\pgfsetfillcolor{currentfill}%
\pgfsetfillopacity{0.800000}%
\pgfsetlinewidth{0.000000pt}%
\definecolor{currentstroke}{rgb}{0.000000,0.000000,0.000000}%
\pgfsetstrokecolor{currentstroke}%
\pgfsetdash{}{0pt}%
\pgfpathmoveto{\pgfqpoint{5.831936in}{3.527183in}}%
\pgfpathlineto{\pgfqpoint{5.847037in}{3.543686in}}%
\pgfpathlineto{\pgfqpoint{5.862161in}{3.560375in}}%
\pgfpathlineto{\pgfqpoint{5.877310in}{3.577250in}}%
\pgfpathlineto{\pgfqpoint{5.892483in}{3.594311in}}%
\pgfpathlineto{\pgfqpoint{5.899655in}{3.594472in}}%
\pgfpathlineto{\pgfqpoint{5.906817in}{3.594563in}}%
\pgfpathlineto{\pgfqpoint{5.913970in}{3.594591in}}%
\pgfpathlineto{\pgfqpoint{5.921113in}{3.594560in}}%
\pgfpathlineto{\pgfqpoint{5.905970in}{3.578016in}}%
\pgfpathlineto{\pgfqpoint{5.890851in}{3.561658in}}%
\pgfpathlineto{\pgfqpoint{5.875756in}{3.545484in}}%
\pgfpathlineto{\pgfqpoint{5.860685in}{3.529494in}}%
\pgfpathlineto{\pgfqpoint{5.853511in}{3.528998in}}%
\pgfpathlineto{\pgfqpoint{5.846328in}{3.528451in}}%
\pgfpathlineto{\pgfqpoint{5.839137in}{3.527848in}}%
\pgfpathlineto{\pgfqpoint{5.831936in}{3.527183in}}%
\pgfpathclose%
\pgfusepath{fill}%
\end{pgfscope}%
\begin{pgfscope}%
\pgfpathrectangle{\pgfqpoint{1.150000in}{0.150000in}}{\pgfqpoint{5.700000in}{5.700000in}}%
\pgfusepath{clip}%
\pgfsetbuttcap%
\pgfsetroundjoin%
\definecolor{currentfill}{rgb}{0.162016,0.687316,0.499129}%
\pgfsetfillcolor{currentfill}%
\pgfsetfillopacity{0.800000}%
\pgfsetlinewidth{0.000000pt}%
\definecolor{currentstroke}{rgb}{0.000000,0.000000,0.000000}%
\pgfsetstrokecolor{currentstroke}%
\pgfsetdash{}{0pt}%
\pgfpathmoveto{\pgfqpoint{5.445473in}{3.221580in}}%
\pgfpathlineto{\pgfqpoint{5.460319in}{3.237149in}}%
\pgfpathlineto{\pgfqpoint{5.475186in}{3.252904in}}%
\pgfpathlineto{\pgfqpoint{5.490076in}{3.268845in}}%
\pgfpathlineto{\pgfqpoint{5.504988in}{3.284973in}}%
\pgfpathlineto{\pgfqpoint{5.512457in}{3.288891in}}%
\pgfpathlineto{\pgfqpoint{5.519915in}{3.292680in}}%
\pgfpathlineto{\pgfqpoint{5.527363in}{3.296344in}}%
\pgfpathlineto{\pgfqpoint{5.534801in}{3.299886in}}%
\pgfpathlineto{\pgfqpoint{5.519907in}{3.284094in}}%
\pgfpathlineto{\pgfqpoint{5.505034in}{3.268488in}}%
\pgfpathlineto{\pgfqpoint{5.490184in}{3.253067in}}%
\pgfpathlineto{\pgfqpoint{5.475355in}{3.237832in}}%
\pgfpathlineto{\pgfqpoint{5.467899in}{3.233943in}}%
\pgfpathlineto{\pgfqpoint{5.460433in}{3.229940in}}%
\pgfpathlineto{\pgfqpoint{5.452958in}{3.225820in}}%
\pgfpathlineto{\pgfqpoint{5.445473in}{3.221580in}}%
\pgfpathclose%
\pgfusepath{fill}%
\end{pgfscope}%
\begin{pgfscope}%
\pgfpathrectangle{\pgfqpoint{1.150000in}{0.150000in}}{\pgfqpoint{5.700000in}{5.700000in}}%
\pgfusepath{clip}%
\pgfsetbuttcap%
\pgfsetroundjoin%
\definecolor{currentfill}{rgb}{0.369214,0.788888,0.382914}%
\pgfsetfillcolor{currentfill}%
\pgfsetfillopacity{0.800000}%
\pgfsetlinewidth{0.000000pt}%
\definecolor{currentstroke}{rgb}{0.000000,0.000000,0.000000}%
\pgfsetstrokecolor{currentstroke}%
\pgfsetdash{}{0pt}%
\pgfpathmoveto{\pgfqpoint{5.921113in}{3.594560in}}%
\pgfpathlineto{\pgfqpoint{5.936280in}{3.611288in}}%
\pgfpathlineto{\pgfqpoint{5.951471in}{3.628202in}}%
\pgfpathlineto{\pgfqpoint{5.966686in}{3.645301in}}%
\pgfpathlineto{\pgfqpoint{5.973796in}{3.644812in}}%
\pgfpathlineto{\pgfqpoint{5.980897in}{3.644270in}}%
\pgfpathlineto{\pgfqpoint{5.987988in}{3.643682in}}%
\pgfpathlineto{\pgfqpoint{5.995070in}{3.643052in}}%
\pgfpathlineto{\pgfqpoint{5.979887in}{3.626505in}}%
\pgfpathlineto{\pgfqpoint{5.964729in}{3.610143in}}%
\pgfpathlineto{\pgfqpoint{5.949594in}{3.593965in}}%
\pgfpathlineto{\pgfqpoint{5.942487in}{3.594173in}}%
\pgfpathlineto{\pgfqpoint{5.935371in}{3.594345in}}%
\pgfpathlineto{\pgfqpoint{5.928246in}{3.594476in}}%
\pgfpathlineto{\pgfqpoint{5.921113in}{3.594560in}}%
\pgfpathclose%
\pgfusepath{fill}%
\end{pgfscope}%
\begin{pgfscope}%
\pgfpathrectangle{\pgfqpoint{1.150000in}{0.150000in}}{\pgfqpoint{5.700000in}{5.700000in}}%
\pgfusepath{clip}%
\pgfsetbuttcap%
\pgfsetroundjoin%
\definecolor{currentfill}{rgb}{0.281924,0.089666,0.412415}%
\pgfsetfillcolor{currentfill}%
\pgfsetfillopacity{0.800000}%
\pgfsetlinewidth{0.000000pt}%
\definecolor{currentstroke}{rgb}{0.000000,0.000000,0.000000}%
\pgfsetstrokecolor{currentstroke}%
\pgfsetdash{}{0pt}%
\pgfpathmoveto{\pgfqpoint{2.869135in}{1.621850in}}%
\pgfpathlineto{\pgfqpoint{2.883090in}{1.609440in}}%
\pgfpathlineto{\pgfqpoint{2.897042in}{1.597252in}}%
\pgfpathlineto{\pgfqpoint{2.910991in}{1.585285in}}%
\pgfpathlineto{\pgfqpoint{2.924937in}{1.573538in}}%
\pgfpathlineto{\pgfqpoint{2.933655in}{1.574041in}}%
\pgfpathlineto{\pgfqpoint{2.942357in}{1.574852in}}%
\pgfpathlineto{\pgfqpoint{2.951044in}{1.575965in}}%
\pgfpathlineto{\pgfqpoint{2.959715in}{1.577372in}}%
\pgfpathlineto{\pgfqpoint{2.945810in}{1.588416in}}%
\pgfpathlineto{\pgfqpoint{2.931902in}{1.599679in}}%
\pgfpathlineto{\pgfqpoint{2.917992in}{1.611162in}}%
\pgfpathlineto{\pgfqpoint{2.904080in}{1.622867in}}%
\pgfpathlineto{\pgfqpoint{2.895368in}{1.622151in}}%
\pgfpathlineto{\pgfqpoint{2.886641in}{1.621738in}}%
\pgfpathlineto{\pgfqpoint{2.877896in}{1.621635in}}%
\pgfpathlineto{\pgfqpoint{2.869135in}{1.621850in}}%
\pgfpathclose%
\pgfusepath{fill}%
\end{pgfscope}%
\begin{pgfscope}%
\pgfpathrectangle{\pgfqpoint{1.150000in}{0.150000in}}{\pgfqpoint{5.700000in}{5.700000in}}%
\pgfusepath{clip}%
\pgfsetbuttcap%
\pgfsetroundjoin%
\definecolor{currentfill}{rgb}{0.271305,0.019942,0.347269}%
\pgfsetfillcolor{currentfill}%
\pgfsetfillopacity{0.800000}%
\pgfsetlinewidth{0.000000pt}%
\definecolor{currentstroke}{rgb}{0.000000,0.000000,0.000000}%
\pgfsetstrokecolor{currentstroke}%
\pgfsetdash{}{0pt}%
\pgfpathmoveto{\pgfqpoint{3.126484in}{1.461539in}}%
\pgfpathlineto{\pgfqpoint{3.140379in}{1.453245in}}%
\pgfpathlineto{\pgfqpoint{3.154275in}{1.445157in}}%
\pgfpathlineto{\pgfqpoint{3.168172in}{1.437272in}}%
\pgfpathlineto{\pgfqpoint{3.182070in}{1.429590in}}%
\pgfpathlineto{\pgfqpoint{3.190586in}{1.433995in}}%
\pgfpathlineto{\pgfqpoint{3.199090in}{1.438646in}}%
\pgfpathlineto{\pgfqpoint{3.207583in}{1.443536in}}%
\pgfpathlineto{\pgfqpoint{3.216064in}{1.448660in}}%
\pgfpathlineto{\pgfqpoint{3.202197in}{1.455680in}}%
\pgfpathlineto{\pgfqpoint{3.188331in}{1.462902in}}%
\pgfpathlineto{\pgfqpoint{3.174466in}{1.470328in}}%
\pgfpathlineto{\pgfqpoint{3.160603in}{1.477958in}}%
\pgfpathlineto{\pgfqpoint{3.152091in}{1.473484in}}%
\pgfpathlineto{\pgfqpoint{3.143568in}{1.469252in}}%
\pgfpathlineto{\pgfqpoint{3.135032in}{1.465268in}}%
\pgfpathlineto{\pgfqpoint{3.126484in}{1.461539in}}%
\pgfpathclose%
\pgfusepath{fill}%
\end{pgfscope}%
\begin{pgfscope}%
\pgfpathrectangle{\pgfqpoint{1.150000in}{0.150000in}}{\pgfqpoint{5.700000in}{5.700000in}}%
\pgfusepath{clip}%
\pgfsetbuttcap%
\pgfsetroundjoin%
\definecolor{currentfill}{rgb}{0.267004,0.004874,0.329415}%
\pgfsetfillcolor{currentfill}%
\pgfsetfillopacity{0.800000}%
\pgfsetlinewidth{0.000000pt}%
\definecolor{currentstroke}{rgb}{0.000000,0.000000,0.000000}%
\pgfsetstrokecolor{currentstroke}%
\pgfsetdash{}{0pt}%
\pgfpathmoveto{\pgfqpoint{3.416158in}{1.410181in}}%
\pgfpathlineto{\pgfqpoint{3.430037in}{1.406354in}}%
\pgfpathlineto{\pgfqpoint{3.443920in}{1.402719in}}%
\pgfpathlineto{\pgfqpoint{3.457808in}{1.399277in}}%
\pgfpathlineto{\pgfqpoint{3.471701in}{1.396025in}}%
\pgfpathlineto{\pgfqpoint{3.480036in}{1.404650in}}%
\pgfpathlineto{\pgfqpoint{3.488364in}{1.413438in}}%
\pgfpathlineto{\pgfqpoint{3.496683in}{1.422384in}}%
\pgfpathlineto{\pgfqpoint{3.504994in}{1.431482in}}%
\pgfpathlineto{\pgfqpoint{3.491120in}{1.434137in}}%
\pgfpathlineto{\pgfqpoint{3.477252in}{1.436985in}}%
\pgfpathlineto{\pgfqpoint{3.463388in}{1.440024in}}%
\pgfpathlineto{\pgfqpoint{3.449529in}{1.443256in}}%
\pgfpathlineto{\pgfqpoint{3.441199in}{1.434742in}}%
\pgfpathlineto{\pgfqpoint{3.432860in}{1.426387in}}%
\pgfpathlineto{\pgfqpoint{3.424513in}{1.418199in}}%
\pgfpathlineto{\pgfqpoint{3.416158in}{1.410181in}}%
\pgfpathclose%
\pgfusepath{fill}%
\end{pgfscope}%
\begin{pgfscope}%
\pgfpathrectangle{\pgfqpoint{1.150000in}{0.150000in}}{\pgfqpoint{5.700000in}{5.700000in}}%
\pgfusepath{clip}%
\pgfsetbuttcap%
\pgfsetroundjoin%
\definecolor{currentfill}{rgb}{0.282623,0.140926,0.457517}%
\pgfsetfillcolor{currentfill}%
\pgfsetfillopacity{0.800000}%
\pgfsetlinewidth{0.000000pt}%
\definecolor{currentstroke}{rgb}{0.000000,0.000000,0.000000}%
\pgfsetstrokecolor{currentstroke}%
\pgfsetdash{}{0pt}%
\pgfpathmoveto{\pgfqpoint{3.947494in}{1.673751in}}%
\pgfpathlineto{\pgfqpoint{3.961479in}{1.677533in}}%
\pgfpathlineto{\pgfqpoint{3.975473in}{1.681498in}}%
\pgfpathlineto{\pgfqpoint{3.989478in}{1.685648in}}%
\pgfpathlineto{\pgfqpoint{4.003494in}{1.689981in}}%
\pgfpathlineto{\pgfqpoint{4.011624in}{1.703609in}}%
\pgfpathlineto{\pgfqpoint{4.019750in}{1.717236in}}%
\pgfpathlineto{\pgfqpoint{4.027871in}{1.730859in}}%
\pgfpathlineto{\pgfqpoint{4.035987in}{1.744474in}}%
\pgfpathlineto{\pgfqpoint{4.021975in}{1.739730in}}%
\pgfpathlineto{\pgfqpoint{4.007973in}{1.735170in}}%
\pgfpathlineto{\pgfqpoint{3.993983in}{1.730794in}}%
\pgfpathlineto{\pgfqpoint{3.980003in}{1.726603in}}%
\pgfpathlineto{\pgfqpoint{3.971883in}{1.713386in}}%
\pgfpathlineto{\pgfqpoint{3.963758in}{1.700169in}}%
\pgfpathlineto{\pgfqpoint{3.955628in}{1.686956in}}%
\pgfpathlineto{\pgfqpoint{3.947494in}{1.673751in}}%
\pgfpathclose%
\pgfusepath{fill}%
\end{pgfscope}%
\begin{pgfscope}%
\pgfpathrectangle{\pgfqpoint{1.150000in}{0.150000in}}{\pgfqpoint{5.700000in}{5.700000in}}%
\pgfusepath{clip}%
\pgfsetbuttcap%
\pgfsetroundjoin%
\definecolor{currentfill}{rgb}{0.265145,0.232956,0.516599}%
\pgfsetfillcolor{currentfill}%
\pgfsetfillopacity{0.800000}%
\pgfsetlinewidth{0.000000pt}%
\definecolor{currentstroke}{rgb}{0.000000,0.000000,0.000000}%
\pgfsetstrokecolor{currentstroke}%
\pgfsetdash{}{0pt}%
\pgfpathmoveto{\pgfqpoint{4.156908in}{1.876413in}}%
\pgfpathlineto{\pgfqpoint{4.170976in}{1.882804in}}%
\pgfpathlineto{\pgfqpoint{4.185056in}{1.889378in}}%
\pgfpathlineto{\pgfqpoint{4.199150in}{1.896136in}}%
\pgfpathlineto{\pgfqpoint{4.213256in}{1.903078in}}%
\pgfpathlineto{\pgfqpoint{4.221330in}{1.917124in}}%
\pgfpathlineto{\pgfqpoint{4.229399in}{1.931114in}}%
\pgfpathlineto{\pgfqpoint{4.237464in}{1.945044in}}%
\pgfpathlineto{\pgfqpoint{4.245525in}{1.958911in}}%
\pgfpathlineto{\pgfqpoint{4.231419in}{1.951652in}}%
\pgfpathlineto{\pgfqpoint{4.217326in}{1.944578in}}%
\pgfpathlineto{\pgfqpoint{4.203246in}{1.937687in}}%
\pgfpathlineto{\pgfqpoint{4.189179in}{1.930980in}}%
\pgfpathlineto{\pgfqpoint{4.181118in}{1.917417in}}%
\pgfpathlineto{\pgfqpoint{4.173052in}{1.903800in}}%
\pgfpathlineto{\pgfqpoint{4.164982in}{1.890131in}}%
\pgfpathlineto{\pgfqpoint{4.156908in}{1.876413in}}%
\pgfpathclose%
\pgfusepath{fill}%
\end{pgfscope}%
\begin{pgfscope}%
\pgfpathrectangle{\pgfqpoint{1.150000in}{0.150000in}}{\pgfqpoint{5.700000in}{5.700000in}}%
\pgfusepath{clip}%
\pgfsetbuttcap%
\pgfsetroundjoin%
\definecolor{currentfill}{rgb}{0.206756,0.371758,0.553117}%
\pgfsetfillcolor{currentfill}%
\pgfsetfillopacity{0.800000}%
\pgfsetlinewidth{0.000000pt}%
\definecolor{currentstroke}{rgb}{0.000000,0.000000,0.000000}%
\pgfsetstrokecolor{currentstroke}%
\pgfsetdash{}{0pt}%
\pgfpathmoveto{\pgfqpoint{4.487205in}{2.242642in}}%
\pgfpathlineto{\pgfqpoint{4.501442in}{2.252503in}}%
\pgfpathlineto{\pgfqpoint{4.515696in}{2.262548in}}%
\pgfpathlineto{\pgfqpoint{4.529965in}{2.272777in}}%
\pgfpathlineto{\pgfqpoint{4.544249in}{2.283191in}}%
\pgfpathlineto{\pgfqpoint{4.552227in}{2.296159in}}%
\pgfpathlineto{\pgfqpoint{4.560200in}{2.309005in}}%
\pgfpathlineto{\pgfqpoint{4.568166in}{2.321728in}}%
\pgfpathlineto{\pgfqpoint{4.576127in}{2.334326in}}%
\pgfpathlineto{\pgfqpoint{4.561842in}{2.323757in}}%
\pgfpathlineto{\pgfqpoint{4.547572in}{2.313373in}}%
\pgfpathlineto{\pgfqpoint{4.533318in}{2.303174in}}%
\pgfpathlineto{\pgfqpoint{4.519080in}{2.293159in}}%
\pgfpathlineto{\pgfqpoint{4.511120in}{2.280703in}}%
\pgfpathlineto{\pgfqpoint{4.503154in}{2.268131in}}%
\pgfpathlineto{\pgfqpoint{4.495182in}{2.255444in}}%
\pgfpathlineto{\pgfqpoint{4.487205in}{2.242642in}}%
\pgfpathclose%
\pgfusepath{fill}%
\end{pgfscope}%
\begin{pgfscope}%
\pgfpathrectangle{\pgfqpoint{1.150000in}{0.150000in}}{\pgfqpoint{5.700000in}{5.700000in}}%
\pgfusepath{clip}%
\pgfsetbuttcap%
\pgfsetroundjoin%
\definecolor{currentfill}{rgb}{0.210503,0.363727,0.552206}%
\pgfsetfillcolor{currentfill}%
\pgfsetfillopacity{0.800000}%
\pgfsetlinewidth{0.000000pt}%
\definecolor{currentstroke}{rgb}{0.000000,0.000000,0.000000}%
\pgfsetstrokecolor{currentstroke}%
\pgfsetdash{}{0pt}%
\pgfpathmoveto{\pgfqpoint{2.305841in}{2.318054in}}%
\pgfpathlineto{\pgfqpoint{2.320110in}{2.295417in}}%
\pgfpathlineto{\pgfqpoint{2.334366in}{2.273076in}}%
\pgfpathlineto{\pgfqpoint{2.348609in}{2.251029in}}%
\pgfpathlineto{\pgfqpoint{2.362840in}{2.229273in}}%
\pgfpathlineto{\pgfqpoint{2.372085in}{2.222426in}}%
\pgfpathlineto{\pgfqpoint{2.381306in}{2.215981in}}%
\pgfpathlineto{\pgfqpoint{2.390502in}{2.209932in}}%
\pgfpathlineto{\pgfqpoint{2.399674in}{2.204269in}}%
\pgfpathlineto{\pgfqpoint{2.385506in}{2.225279in}}%
\pgfpathlineto{\pgfqpoint{2.371327in}{2.246579in}}%
\pgfpathlineto{\pgfqpoint{2.357135in}{2.268170in}}%
\pgfpathlineto{\pgfqpoint{2.342931in}{2.290056in}}%
\pgfpathlineto{\pgfqpoint{2.333697in}{2.296453in}}%
\pgfpathlineto{\pgfqpoint{2.324437in}{2.303246in}}%
\pgfpathlineto{\pgfqpoint{2.315152in}{2.310444in}}%
\pgfpathlineto{\pgfqpoint{2.305841in}{2.318054in}}%
\pgfpathclose%
\pgfusepath{fill}%
\end{pgfscope}%
\begin{pgfscope}%
\pgfpathrectangle{\pgfqpoint{1.150000in}{0.150000in}}{\pgfqpoint{5.700000in}{5.700000in}}%
\pgfusepath{clip}%
\pgfsetbuttcap%
\pgfsetroundjoin%
\definecolor{currentfill}{rgb}{0.153364,0.497000,0.557724}%
\pgfsetfillcolor{currentfill}%
\pgfsetfillopacity{0.800000}%
\pgfsetlinewidth{0.000000pt}%
\definecolor{currentstroke}{rgb}{0.000000,0.000000,0.000000}%
\pgfsetstrokecolor{currentstroke}%
\pgfsetdash{}{0pt}%
\pgfpathmoveto{\pgfqpoint{4.817537in}{2.611857in}}%
\pgfpathlineto{\pgfqpoint{4.831975in}{2.624372in}}%
\pgfpathlineto{\pgfqpoint{4.846430in}{2.637073in}}%
\pgfpathlineto{\pgfqpoint{4.860904in}{2.649959in}}%
\pgfpathlineto{\pgfqpoint{4.875397in}{2.663031in}}%
\pgfpathlineto{\pgfqpoint{4.883244in}{2.673423in}}%
\pgfpathlineto{\pgfqpoint{4.891083in}{2.683661in}}%
\pgfpathlineto{\pgfqpoint{4.898915in}{2.693745in}}%
\pgfpathlineto{\pgfqpoint{4.906739in}{2.703675in}}%
\pgfpathlineto{\pgfqpoint{4.892249in}{2.690619in}}%
\pgfpathlineto{\pgfqpoint{4.877778in}{2.677749in}}%
\pgfpathlineto{\pgfqpoint{4.863325in}{2.665063in}}%
\pgfpathlineto{\pgfqpoint{4.848890in}{2.652563in}}%
\pgfpathlineto{\pgfqpoint{4.841063in}{2.642605in}}%
\pgfpathlineto{\pgfqpoint{4.833228in}{2.632501in}}%
\pgfpathlineto{\pgfqpoint{4.825386in}{2.622253in}}%
\pgfpathlineto{\pgfqpoint{4.817537in}{2.611857in}}%
\pgfpathclose%
\pgfusepath{fill}%
\end{pgfscope}%
\begin{pgfscope}%
\pgfpathrectangle{\pgfqpoint{1.150000in}{0.150000in}}{\pgfqpoint{5.700000in}{5.700000in}}%
\pgfusepath{clip}%
\pgfsetbuttcap%
\pgfsetroundjoin%
\definecolor{currentfill}{rgb}{0.280267,0.073417,0.397163}%
\pgfsetfillcolor{currentfill}%
\pgfsetfillopacity{0.800000}%
\pgfsetlinewidth{0.000000pt}%
\definecolor{currentstroke}{rgb}{0.000000,0.000000,0.000000}%
\pgfsetstrokecolor{currentstroke}%
\pgfsetdash{}{0pt}%
\pgfpathmoveto{\pgfqpoint{2.924937in}{1.573538in}}%
\pgfpathlineto{\pgfqpoint{2.938881in}{1.562010in}}%
\pgfpathlineto{\pgfqpoint{2.952822in}{1.550699in}}%
\pgfpathlineto{\pgfqpoint{2.966762in}{1.539604in}}%
\pgfpathlineto{\pgfqpoint{2.980700in}{1.528724in}}%
\pgfpathlineto{\pgfqpoint{2.989377in}{1.529942in}}%
\pgfpathlineto{\pgfqpoint{2.998039in}{1.531460in}}%
\pgfpathlineto{\pgfqpoint{3.006686in}{1.533270in}}%
\pgfpathlineto{\pgfqpoint{3.015319in}{1.535366in}}%
\pgfpathlineto{\pgfqpoint{3.001420in}{1.545545in}}%
\pgfpathlineto{\pgfqpoint{2.987520in}{1.555938in}}%
\pgfpathlineto{\pgfqpoint{2.973618in}{1.566547in}}%
\pgfpathlineto{\pgfqpoint{2.959715in}{1.577372in}}%
\pgfpathlineto{\pgfqpoint{2.951044in}{1.575965in}}%
\pgfpathlineto{\pgfqpoint{2.942357in}{1.574852in}}%
\pgfpathlineto{\pgfqpoint{2.933655in}{1.574041in}}%
\pgfpathlineto{\pgfqpoint{2.924937in}{1.573538in}}%
\pgfpathclose%
\pgfusepath{fill}%
\end{pgfscope}%
\begin{pgfscope}%
\pgfpathrectangle{\pgfqpoint{1.150000in}{0.150000in}}{\pgfqpoint{5.700000in}{5.700000in}}%
\pgfusepath{clip}%
\pgfsetbuttcap%
\pgfsetroundjoin%
\definecolor{currentfill}{rgb}{0.122312,0.633153,0.530398}%
\pgfsetfillcolor{currentfill}%
\pgfsetfillopacity{0.800000}%
\pgfsetlinewidth{0.000000pt}%
\definecolor{currentstroke}{rgb}{0.000000,0.000000,0.000000}%
\pgfsetstrokecolor{currentstroke}%
\pgfsetdash{}{0pt}%
\pgfpathmoveto{\pgfqpoint{5.236571in}{3.035830in}}%
\pgfpathlineto{\pgfqpoint{5.251285in}{3.050699in}}%
\pgfpathlineto{\pgfqpoint{5.266021in}{3.065755in}}%
\pgfpathlineto{\pgfqpoint{5.280777in}{3.080997in}}%
\pgfpathlineto{\pgfqpoint{5.295555in}{3.096426in}}%
\pgfpathlineto{\pgfqpoint{5.303170in}{3.102597in}}%
\pgfpathlineto{\pgfqpoint{5.310776in}{3.108616in}}%
\pgfpathlineto{\pgfqpoint{5.318372in}{3.114487in}}%
\pgfpathlineto{\pgfqpoint{5.325958in}{3.120214in}}%
\pgfpathlineto{\pgfqpoint{5.311192in}{3.105013in}}%
\pgfpathlineto{\pgfqpoint{5.296447in}{3.089998in}}%
\pgfpathlineto{\pgfqpoint{5.281723in}{3.075169in}}%
\pgfpathlineto{\pgfqpoint{5.267020in}{3.060526in}}%
\pgfpathlineto{\pgfqpoint{5.259421in}{3.054560in}}%
\pgfpathlineto{\pgfqpoint{5.251814in}{3.048457in}}%
\pgfpathlineto{\pgfqpoint{5.244197in}{3.042214in}}%
\pgfpathlineto{\pgfqpoint{5.236571in}{3.035830in}}%
\pgfpathclose%
\pgfusepath{fill}%
\end{pgfscope}%
\begin{pgfscope}%
\pgfpathrectangle{\pgfqpoint{1.150000in}{0.150000in}}{\pgfqpoint{5.700000in}{5.700000in}}%
\pgfusepath{clip}%
\pgfsetbuttcap%
\pgfsetroundjoin%
\definecolor{currentfill}{rgb}{0.127568,0.566949,0.550556}%
\pgfsetfillcolor{currentfill}%
\pgfsetfillopacity{0.800000}%
\pgfsetlinewidth{0.000000pt}%
\definecolor{currentstroke}{rgb}{0.000000,0.000000,0.000000}%
\pgfsetstrokecolor{currentstroke}%
\pgfsetdash{}{0pt}%
\pgfpathmoveto{\pgfqpoint{5.027167in}{2.831242in}}%
\pgfpathlineto{\pgfqpoint{5.041744in}{2.845092in}}%
\pgfpathlineto{\pgfqpoint{5.056340in}{2.859127in}}%
\pgfpathlineto{\pgfqpoint{5.070957in}{2.873349in}}%
\pgfpathlineto{\pgfqpoint{5.085593in}{2.887757in}}%
\pgfpathlineto{\pgfqpoint{5.093335in}{2.896136in}}%
\pgfpathlineto{\pgfqpoint{5.101068in}{2.904357in}}%
\pgfpathlineto{\pgfqpoint{5.108793in}{2.912419in}}%
\pgfpathlineto{\pgfqpoint{5.116509in}{2.920326in}}%
\pgfpathlineto{\pgfqpoint{5.101879in}{2.906039in}}%
\pgfpathlineto{\pgfqpoint{5.087269in}{2.891938in}}%
\pgfpathlineto{\pgfqpoint{5.072679in}{2.878023in}}%
\pgfpathlineto{\pgfqpoint{5.058109in}{2.864293in}}%
\pgfpathlineto{\pgfqpoint{5.050386in}{2.856253in}}%
\pgfpathlineto{\pgfqpoint{5.042655in}{2.848066in}}%
\pgfpathlineto{\pgfqpoint{5.034915in}{2.839729in}}%
\pgfpathlineto{\pgfqpoint{5.027167in}{2.831242in}}%
\pgfpathclose%
\pgfusepath{fill}%
\end{pgfscope}%
\begin{pgfscope}%
\pgfpathrectangle{\pgfqpoint{1.150000in}{0.150000in}}{\pgfqpoint{5.700000in}{5.700000in}}%
\pgfusepath{clip}%
\pgfsetbuttcap%
\pgfsetroundjoin%
\definecolor{currentfill}{rgb}{0.202219,0.715272,0.476084}%
\pgfsetfillcolor{currentfill}%
\pgfsetfillopacity{0.800000}%
\pgfsetlinewidth{0.000000pt}%
\definecolor{currentstroke}{rgb}{0.000000,0.000000,0.000000}%
\pgfsetstrokecolor{currentstroke}%
\pgfsetdash{}{0pt}%
\pgfpathmoveto{\pgfqpoint{5.534801in}{3.299886in}}%
\pgfpathlineto{\pgfqpoint{5.549718in}{3.315864in}}%
\pgfpathlineto{\pgfqpoint{5.564658in}{3.332028in}}%
\pgfpathlineto{\pgfqpoint{5.579620in}{3.348378in}}%
\pgfpathlineto{\pgfqpoint{5.594605in}{3.364915in}}%
\pgfpathlineto{\pgfqpoint{5.602014in}{3.367981in}}%
\pgfpathlineto{\pgfqpoint{5.609413in}{3.370924in}}%
\pgfpathlineto{\pgfqpoint{5.616802in}{3.373749in}}%
\pgfpathlineto{\pgfqpoint{5.624180in}{3.376461in}}%
\pgfpathlineto{\pgfqpoint{5.609215in}{3.360297in}}%
\pgfpathlineto{\pgfqpoint{5.594272in}{3.344319in}}%
\pgfpathlineto{\pgfqpoint{5.579353in}{3.328526in}}%
\pgfpathlineto{\pgfqpoint{5.564455in}{3.312919in}}%
\pgfpathlineto{\pgfqpoint{5.557056in}{3.309823in}}%
\pgfpathlineto{\pgfqpoint{5.549648in}{3.306621in}}%
\pgfpathlineto{\pgfqpoint{5.542229in}{3.303310in}}%
\pgfpathlineto{\pgfqpoint{5.534801in}{3.299886in}}%
\pgfpathclose%
\pgfusepath{fill}%
\end{pgfscope}%
\begin{pgfscope}%
\pgfpathrectangle{\pgfqpoint{1.150000in}{0.150000in}}{\pgfqpoint{5.700000in}{5.700000in}}%
\pgfusepath{clip}%
\pgfsetbuttcap%
\pgfsetroundjoin%
\definecolor{currentfill}{rgb}{0.229739,0.322361,0.545706}%
\pgfsetfillcolor{currentfill}%
\pgfsetfillopacity{0.800000}%
\pgfsetlinewidth{0.000000pt}%
\definecolor{currentstroke}{rgb}{0.000000,0.000000,0.000000}%
\pgfsetstrokecolor{currentstroke}%
\pgfsetdash{}{0pt}%
\pgfpathmoveto{\pgfqpoint{4.366403in}{2.100327in}}%
\pgfpathlineto{\pgfqpoint{4.380579in}{2.109043in}}%
\pgfpathlineto{\pgfqpoint{4.394770in}{2.117943in}}%
\pgfpathlineto{\pgfqpoint{4.408975in}{2.127027in}}%
\pgfpathlineto{\pgfqpoint{4.423196in}{2.136294in}}%
\pgfpathlineto{\pgfqpoint{4.431215in}{2.149956in}}%
\pgfpathlineto{\pgfqpoint{4.439229in}{2.163516in}}%
\pgfpathlineto{\pgfqpoint{4.447238in}{2.176972in}}%
\pgfpathlineto{\pgfqpoint{4.455242in}{2.190323in}}%
\pgfpathlineto{\pgfqpoint{4.441020in}{2.180835in}}%
\pgfpathlineto{\pgfqpoint{4.426814in}{2.171530in}}%
\pgfpathlineto{\pgfqpoint{4.412623in}{2.162410in}}%
\pgfpathlineto{\pgfqpoint{4.398446in}{2.153474in}}%
\pgfpathlineto{\pgfqpoint{4.390443in}{2.140331in}}%
\pgfpathlineto{\pgfqpoint{4.382434in}{2.127091in}}%
\pgfpathlineto{\pgfqpoint{4.374421in}{2.113756in}}%
\pgfpathlineto{\pgfqpoint{4.366403in}{2.100327in}}%
\pgfpathclose%
\pgfusepath{fill}%
\end{pgfscope}%
\begin{pgfscope}%
\pgfpathrectangle{\pgfqpoint{1.150000in}{0.150000in}}{\pgfqpoint{5.700000in}{5.700000in}}%
\pgfusepath{clip}%
\pgfsetbuttcap%
\pgfsetroundjoin%
\definecolor{currentfill}{rgb}{0.278012,0.180367,0.486697}%
\pgfsetfillcolor{currentfill}%
\pgfsetfillopacity{0.800000}%
\pgfsetlinewidth{0.000000pt}%
\definecolor{currentstroke}{rgb}{0.000000,0.000000,0.000000}%
\pgfsetstrokecolor{currentstroke}%
\pgfsetdash{}{0pt}%
\pgfpathmoveto{\pgfqpoint{4.035987in}{1.744474in}}%
\pgfpathlineto{\pgfqpoint{4.050011in}{1.749402in}}%
\pgfpathlineto{\pgfqpoint{4.064046in}{1.754514in}}%
\pgfpathlineto{\pgfqpoint{4.078093in}{1.759809in}}%
\pgfpathlineto{\pgfqpoint{4.092151in}{1.765287in}}%
\pgfpathlineto{\pgfqpoint{4.100261in}{1.779282in}}%
\pgfpathlineto{\pgfqpoint{4.108367in}{1.793255in}}%
\pgfpathlineto{\pgfqpoint{4.116468in}{1.807201in}}%
\pgfpathlineto{\pgfqpoint{4.124565in}{1.821116in}}%
\pgfpathlineto{\pgfqpoint{4.110508in}{1.815258in}}%
\pgfpathlineto{\pgfqpoint{4.096463in}{1.809583in}}%
\pgfpathlineto{\pgfqpoint{4.082430in}{1.804091in}}%
\pgfpathlineto{\pgfqpoint{4.068409in}{1.798784in}}%
\pgfpathlineto{\pgfqpoint{4.060310in}{1.785236in}}%
\pgfpathlineto{\pgfqpoint{4.052207in}{1.771666in}}%
\pgfpathlineto{\pgfqpoint{4.044100in}{1.758078in}}%
\pgfpathlineto{\pgfqpoint{4.035987in}{1.744474in}}%
\pgfpathclose%
\pgfusepath{fill}%
\end{pgfscope}%
\begin{pgfscope}%
\pgfpathrectangle{\pgfqpoint{1.150000in}{0.150000in}}{\pgfqpoint{5.700000in}{5.700000in}}%
\pgfusepath{clip}%
\pgfsetbuttcap%
\pgfsetroundjoin%
\definecolor{currentfill}{rgb}{0.274952,0.037752,0.364543}%
\pgfsetfillcolor{currentfill}%
\pgfsetfillopacity{0.800000}%
\pgfsetlinewidth{0.000000pt}%
\definecolor{currentstroke}{rgb}{0.000000,0.000000,0.000000}%
\pgfsetstrokecolor{currentstroke}%
\pgfsetdash{}{0pt}%
\pgfpathmoveto{\pgfqpoint{3.649231in}{1.459369in}}%
\pgfpathlineto{\pgfqpoint{3.663144in}{1.458977in}}%
\pgfpathlineto{\pgfqpoint{3.677064in}{1.458772in}}%
\pgfpathlineto{\pgfqpoint{3.690991in}{1.458754in}}%
\pgfpathlineto{\pgfqpoint{3.704926in}{1.458921in}}%
\pgfpathlineto{\pgfqpoint{3.713159in}{1.470328in}}%
\pgfpathlineto{\pgfqpoint{3.721386in}{1.481830in}}%
\pgfpathlineto{\pgfqpoint{3.729607in}{1.493423in}}%
\pgfpathlineto{\pgfqpoint{3.737823in}{1.505100in}}%
\pgfpathlineto{\pgfqpoint{3.723899in}{1.504399in}}%
\pgfpathlineto{\pgfqpoint{3.709983in}{1.503884in}}%
\pgfpathlineto{\pgfqpoint{3.696075in}{1.503556in}}%
\pgfpathlineto{\pgfqpoint{3.682174in}{1.503414in}}%
\pgfpathlineto{\pgfqpoint{3.673948in}{1.492258in}}%
\pgfpathlineto{\pgfqpoint{3.665715in}{1.481196in}}%
\pgfpathlineto{\pgfqpoint{3.657476in}{1.470231in}}%
\pgfpathlineto{\pgfqpoint{3.649231in}{1.459369in}}%
\pgfpathclose%
\pgfusepath{fill}%
\end{pgfscope}%
\begin{pgfscope}%
\pgfpathrectangle{\pgfqpoint{1.150000in}{0.150000in}}{\pgfqpoint{5.700000in}{5.700000in}}%
\pgfusepath{clip}%
\pgfsetbuttcap%
\pgfsetroundjoin%
\definecolor{currentfill}{rgb}{0.267004,0.004874,0.329415}%
\pgfsetfillcolor{currentfill}%
\pgfsetfillopacity{0.800000}%
\pgfsetlinewidth{0.000000pt}%
\definecolor{currentstroke}{rgb}{0.000000,0.000000,0.000000}%
\pgfsetstrokecolor{currentstroke}%
\pgfsetdash{}{0pt}%
\pgfpathmoveto{\pgfqpoint{3.327074in}{1.399702in}}%
\pgfpathlineto{\pgfqpoint{3.340962in}{1.394471in}}%
\pgfpathlineto{\pgfqpoint{3.354853in}{1.389435in}}%
\pgfpathlineto{\pgfqpoint{3.368747in}{1.384593in}}%
\pgfpathlineto{\pgfqpoint{3.382645in}{1.379945in}}%
\pgfpathlineto{\pgfqpoint{3.391037in}{1.387217in}}%
\pgfpathlineto{\pgfqpoint{3.399419in}{1.394684in}}%
\pgfpathlineto{\pgfqpoint{3.407793in}{1.402341in}}%
\pgfpathlineto{\pgfqpoint{3.416158in}{1.410181in}}%
\pgfpathlineto{\pgfqpoint{3.402282in}{1.414201in}}%
\pgfpathlineto{\pgfqpoint{3.388411in}{1.418415in}}%
\pgfpathlineto{\pgfqpoint{3.374544in}{1.422824in}}%
\pgfpathlineto{\pgfqpoint{3.360680in}{1.427427in}}%
\pgfpathlineto{\pgfqpoint{3.352293in}{1.420203in}}%
\pgfpathlineto{\pgfqpoint{3.343896in}{1.413169in}}%
\pgfpathlineto{\pgfqpoint{3.335490in}{1.406334in}}%
\pgfpathlineto{\pgfqpoint{3.327074in}{1.399702in}}%
\pgfpathclose%
\pgfusepath{fill}%
\end{pgfscope}%
\begin{pgfscope}%
\pgfpathrectangle{\pgfqpoint{1.150000in}{0.150000in}}{\pgfqpoint{5.700000in}{5.700000in}}%
\pgfusepath{clip}%
\pgfsetbuttcap%
\pgfsetroundjoin%
\definecolor{currentfill}{rgb}{0.144759,0.519093,0.556572}%
\pgfsetfillcolor{currentfill}%
\pgfsetfillopacity{0.800000}%
\pgfsetlinewidth{0.000000pt}%
\definecolor{currentstroke}{rgb}{0.000000,0.000000,0.000000}%
\pgfsetstrokecolor{currentstroke}%
\pgfsetdash{}{0pt}%
\pgfpathmoveto{\pgfqpoint{2.055719in}{2.793953in}}%
\pgfpathlineto{\pgfqpoint{2.070244in}{2.765568in}}%
\pgfpathlineto{\pgfqpoint{2.084749in}{2.737539in}}%
\pgfpathlineto{\pgfqpoint{2.099235in}{2.709862in}}%
\pgfpathlineto{\pgfqpoint{2.113702in}{2.682534in}}%
\pgfpathlineto{\pgfqpoint{2.123183in}{2.673562in}}%
\pgfpathlineto{\pgfqpoint{2.132636in}{2.665014in}}%
\pgfpathlineto{\pgfqpoint{2.142061in}{2.656880in}}%
\pgfpathlineto{\pgfqpoint{2.151459in}{2.649155in}}%
\pgfpathlineto{\pgfqpoint{2.137064in}{2.675740in}}%
\pgfpathlineto{\pgfqpoint{2.122650in}{2.702672in}}%
\pgfpathlineto{\pgfqpoint{2.108219in}{2.729954in}}%
\pgfpathlineto{\pgfqpoint{2.093768in}{2.757591in}}%
\pgfpathlineto{\pgfqpoint{2.084299in}{2.766046in}}%
\pgfpathlineto{\pgfqpoint{2.074801in}{2.774920in}}%
\pgfpathlineto{\pgfqpoint{2.065275in}{2.784220in}}%
\pgfpathlineto{\pgfqpoint{2.055719in}{2.793953in}}%
\pgfpathclose%
\pgfusepath{fill}%
\end{pgfscope}%
\begin{pgfscope}%
\pgfpathrectangle{\pgfqpoint{1.150000in}{0.150000in}}{\pgfqpoint{5.700000in}{5.700000in}}%
\pgfusepath{clip}%
\pgfsetbuttcap%
\pgfsetroundjoin%
\definecolor{currentfill}{rgb}{0.279566,0.067836,0.391917}%
\pgfsetfillcolor{currentfill}%
\pgfsetfillopacity{0.800000}%
\pgfsetlinewidth{0.000000pt}%
\definecolor{currentstroke}{rgb}{0.000000,0.000000,0.000000}%
\pgfsetstrokecolor{currentstroke}%
\pgfsetdash{}{0pt}%
\pgfpathmoveto{\pgfqpoint{3.737823in}{1.505100in}}%
\pgfpathlineto{\pgfqpoint{3.751754in}{1.505987in}}%
\pgfpathlineto{\pgfqpoint{3.765694in}{1.507059in}}%
\pgfpathlineto{\pgfqpoint{3.779642in}{1.508317in}}%
\pgfpathlineto{\pgfqpoint{3.793598in}{1.509759in}}%
\pgfpathlineto{\pgfqpoint{3.801798in}{1.522032in}}%
\pgfpathlineto{\pgfqpoint{3.809993in}{1.534372in}}%
\pgfpathlineto{\pgfqpoint{3.818183in}{1.546775in}}%
\pgfpathlineto{\pgfqpoint{3.826367in}{1.559235in}}%
\pgfpathlineto{\pgfqpoint{3.812419in}{1.557290in}}%
\pgfpathlineto{\pgfqpoint{3.798480in}{1.555529in}}%
\pgfpathlineto{\pgfqpoint{3.784550in}{1.553954in}}%
\pgfpathlineto{\pgfqpoint{3.770627in}{1.552565in}}%
\pgfpathlineto{\pgfqpoint{3.762435in}{1.540595in}}%
\pgfpathlineto{\pgfqpoint{3.754236in}{1.528691in}}%
\pgfpathlineto{\pgfqpoint{3.746032in}{1.516858in}}%
\pgfpathlineto{\pgfqpoint{3.737823in}{1.505100in}}%
\pgfpathclose%
\pgfusepath{fill}%
\end{pgfscope}%
\begin{pgfscope}%
\pgfpathrectangle{\pgfqpoint{1.150000in}{0.150000in}}{\pgfqpoint{5.700000in}{5.700000in}}%
\pgfusepath{clip}%
\pgfsetbuttcap%
\pgfsetroundjoin%
\definecolor{currentfill}{rgb}{0.195860,0.395433,0.555276}%
\pgfsetfillcolor{currentfill}%
\pgfsetfillopacity{0.800000}%
\pgfsetlinewidth{0.000000pt}%
\definecolor{currentstroke}{rgb}{0.000000,0.000000,0.000000}%
\pgfsetstrokecolor{currentstroke}%
\pgfsetdash{}{0pt}%
\pgfpathmoveto{\pgfqpoint{2.248627in}{2.411616in}}%
\pgfpathlineto{\pgfqpoint{2.262952in}{2.387768in}}%
\pgfpathlineto{\pgfqpoint{2.277262in}{2.364227in}}%
\pgfpathlineto{\pgfqpoint{2.291558in}{2.340990in}}%
\pgfpathlineto{\pgfqpoint{2.305841in}{2.318054in}}%
\pgfpathlineto{\pgfqpoint{2.315152in}{2.310444in}}%
\pgfpathlineto{\pgfqpoint{2.324437in}{2.303246in}}%
\pgfpathlineto{\pgfqpoint{2.333697in}{2.296453in}}%
\pgfpathlineto{\pgfqpoint{2.342931in}{2.290056in}}%
\pgfpathlineto{\pgfqpoint{2.328714in}{2.312239in}}%
\pgfpathlineto{\pgfqpoint{2.314484in}{2.334722in}}%
\pgfpathlineto{\pgfqpoint{2.300241in}{2.357507in}}%
\pgfpathlineto{\pgfqpoint{2.285984in}{2.380597in}}%
\pgfpathlineto{\pgfqpoint{2.276684in}{2.387733in}}%
\pgfpathlineto{\pgfqpoint{2.267358in}{2.395277in}}%
\pgfpathlineto{\pgfqpoint{2.258006in}{2.403235in}}%
\pgfpathlineto{\pgfqpoint{2.248627in}{2.411616in}}%
\pgfpathclose%
\pgfusepath{fill}%
\end{pgfscope}%
\begin{pgfscope}%
\pgfpathrectangle{\pgfqpoint{1.150000in}{0.150000in}}{\pgfqpoint{5.700000in}{5.700000in}}%
\pgfusepath{clip}%
\pgfsetbuttcap%
\pgfsetroundjoin%
\definecolor{currentfill}{rgb}{0.268510,0.009605,0.335427}%
\pgfsetfillcolor{currentfill}%
\pgfsetfillopacity{0.800000}%
\pgfsetlinewidth{0.000000pt}%
\definecolor{currentstroke}{rgb}{0.000000,0.000000,0.000000}%
\pgfsetstrokecolor{currentstroke}%
\pgfsetdash{}{0pt}%
\pgfpathmoveto{\pgfqpoint{3.182070in}{1.429590in}}%
\pgfpathlineto{\pgfqpoint{3.195968in}{1.422110in}}%
\pgfpathlineto{\pgfqpoint{3.209868in}{1.414831in}}%
\pgfpathlineto{\pgfqpoint{3.223770in}{1.407753in}}%
\pgfpathlineto{\pgfqpoint{3.237673in}{1.400875in}}%
\pgfpathlineto{\pgfqpoint{3.246159in}{1.405954in}}%
\pgfpathlineto{\pgfqpoint{3.254634in}{1.411271in}}%
\pgfpathlineto{\pgfqpoint{3.263098in}{1.416819in}}%
\pgfpathlineto{\pgfqpoint{3.271551in}{1.422592in}}%
\pgfpathlineto{\pgfqpoint{3.257676in}{1.428809in}}%
\pgfpathlineto{\pgfqpoint{3.243804in}{1.435226in}}%
\pgfpathlineto{\pgfqpoint{3.229933in}{1.441842in}}%
\pgfpathlineto{\pgfqpoint{3.216064in}{1.448660in}}%
\pgfpathlineto{\pgfqpoint{3.207583in}{1.443536in}}%
\pgfpathlineto{\pgfqpoint{3.199090in}{1.438646in}}%
\pgfpathlineto{\pgfqpoint{3.190586in}{1.433995in}}%
\pgfpathlineto{\pgfqpoint{3.182070in}{1.429590in}}%
\pgfpathclose%
\pgfusepath{fill}%
\end{pgfscope}%
\begin{pgfscope}%
\pgfpathrectangle{\pgfqpoint{1.150000in}{0.150000in}}{\pgfqpoint{5.700000in}{5.700000in}}%
\pgfusepath{clip}%
\pgfsetbuttcap%
\pgfsetroundjoin%
\definecolor{currentfill}{rgb}{0.171176,0.452530,0.557965}%
\pgfsetfillcolor{currentfill}%
\pgfsetfillopacity{0.800000}%
\pgfsetlinewidth{0.000000pt}%
\definecolor{currentstroke}{rgb}{0.000000,0.000000,0.000000}%
\pgfsetstrokecolor{currentstroke}%
\pgfsetdash{}{0pt}%
\pgfpathmoveto{\pgfqpoint{4.696908in}{2.475478in}}%
\pgfpathlineto{\pgfqpoint{4.711277in}{2.487179in}}%
\pgfpathlineto{\pgfqpoint{4.725664in}{2.499066in}}%
\pgfpathlineto{\pgfqpoint{4.740068in}{2.511137in}}%
\pgfpathlineto{\pgfqpoint{4.754490in}{2.523394in}}%
\pgfpathlineto{\pgfqpoint{4.762395in}{2.534969in}}%
\pgfpathlineto{\pgfqpoint{4.770293in}{2.546396in}}%
\pgfpathlineto{\pgfqpoint{4.778185in}{2.557675in}}%
\pgfpathlineto{\pgfqpoint{4.786069in}{2.568807in}}%
\pgfpathlineto{\pgfqpoint{4.771648in}{2.556496in}}%
\pgfpathlineto{\pgfqpoint{4.757245in}{2.544371in}}%
\pgfpathlineto{\pgfqpoint{4.742859in}{2.532431in}}%
\pgfpathlineto{\pgfqpoint{4.728490in}{2.520676in}}%
\pgfpathlineto{\pgfqpoint{4.720605in}{2.509585in}}%
\pgfpathlineto{\pgfqpoint{4.712712in}{2.498355in}}%
\pgfpathlineto{\pgfqpoint{4.704813in}{2.486986in}}%
\pgfpathlineto{\pgfqpoint{4.696908in}{2.475478in}}%
\pgfpathclose%
\pgfusepath{fill}%
\end{pgfscope}%
\begin{pgfscope}%
\pgfpathrectangle{\pgfqpoint{1.150000in}{0.150000in}}{\pgfqpoint{5.700000in}{5.700000in}}%
\pgfusepath{clip}%
\pgfsetbuttcap%
\pgfsetroundjoin%
\definecolor{currentfill}{rgb}{0.271305,0.019942,0.347269}%
\pgfsetfillcolor{currentfill}%
\pgfsetfillopacity{0.800000}%
\pgfsetlinewidth{0.000000pt}%
\definecolor{currentstroke}{rgb}{0.000000,0.000000,0.000000}%
\pgfsetstrokecolor{currentstroke}%
\pgfsetdash{}{0pt}%
\pgfpathmoveto{\pgfqpoint{3.560543in}{1.422761in}}%
\pgfpathlineto{\pgfqpoint{3.574444in}{1.421054in}}%
\pgfpathlineto{\pgfqpoint{3.588351in}{1.419535in}}%
\pgfpathlineto{\pgfqpoint{3.602265in}{1.418204in}}%
\pgfpathlineto{\pgfqpoint{3.616185in}{1.417061in}}%
\pgfpathlineto{\pgfqpoint{3.624456in}{1.427456in}}%
\pgfpathlineto{\pgfqpoint{3.632721in}{1.437977in}}%
\pgfpathlineto{\pgfqpoint{3.640979in}{1.448616in}}%
\pgfpathlineto{\pgfqpoint{3.649231in}{1.459369in}}%
\pgfpathlineto{\pgfqpoint{3.635325in}{1.459948in}}%
\pgfpathlineto{\pgfqpoint{3.621426in}{1.460714in}}%
\pgfpathlineto{\pgfqpoint{3.607533in}{1.461669in}}%
\pgfpathlineto{\pgfqpoint{3.593647in}{1.462812in}}%
\pgfpathlineto{\pgfqpoint{3.585382in}{1.452611in}}%
\pgfpathlineto{\pgfqpoint{3.577109in}{1.442532in}}%
\pgfpathlineto{\pgfqpoint{3.568829in}{1.432580in}}%
\pgfpathlineto{\pgfqpoint{3.560543in}{1.422761in}}%
\pgfpathclose%
\pgfusepath{fill}%
\end{pgfscope}%
\begin{pgfscope}%
\pgfpathrectangle{\pgfqpoint{1.150000in}{0.150000in}}{\pgfqpoint{5.700000in}{5.700000in}}%
\pgfusepath{clip}%
\pgfsetbuttcap%
\pgfsetroundjoin%
\definecolor{currentfill}{rgb}{0.277941,0.056324,0.381191}%
\pgfsetfillcolor{currentfill}%
\pgfsetfillopacity{0.800000}%
\pgfsetlinewidth{0.000000pt}%
\definecolor{currentstroke}{rgb}{0.000000,0.000000,0.000000}%
\pgfsetstrokecolor{currentstroke}%
\pgfsetdash{}{0pt}%
\pgfpathmoveto{\pgfqpoint{2.980700in}{1.528724in}}%
\pgfpathlineto{\pgfqpoint{2.994636in}{1.518059in}}%
\pgfpathlineto{\pgfqpoint{3.008570in}{1.507606in}}%
\pgfpathlineto{\pgfqpoint{3.022504in}{1.497366in}}%
\pgfpathlineto{\pgfqpoint{3.036436in}{1.487336in}}%
\pgfpathlineto{\pgfqpoint{3.045074in}{1.489267in}}%
\pgfpathlineto{\pgfqpoint{3.053699in}{1.491488in}}%
\pgfpathlineto{\pgfqpoint{3.062308in}{1.493994in}}%
\pgfpathlineto{\pgfqpoint{3.070904in}{1.496777in}}%
\pgfpathlineto{\pgfqpoint{3.057009in}{1.506108in}}%
\pgfpathlineto{\pgfqpoint{3.043113in}{1.515649in}}%
\pgfpathlineto{\pgfqpoint{3.029216in}{1.525401in}}%
\pgfpathlineto{\pgfqpoint{3.015319in}{1.535366in}}%
\pgfpathlineto{\pgfqpoint{3.006686in}{1.533270in}}%
\pgfpathlineto{\pgfqpoint{2.998039in}{1.531460in}}%
\pgfpathlineto{\pgfqpoint{2.989377in}{1.529942in}}%
\pgfpathlineto{\pgfqpoint{2.980700in}{1.528724in}}%
\pgfpathclose%
\pgfusepath{fill}%
\end{pgfscope}%
\begin{pgfscope}%
\pgfpathrectangle{\pgfqpoint{1.150000in}{0.150000in}}{\pgfqpoint{5.700000in}{5.700000in}}%
\pgfusepath{clip}%
\pgfsetbuttcap%
\pgfsetroundjoin%
\definecolor{currentfill}{rgb}{0.252194,0.269783,0.531579}%
\pgfsetfillcolor{currentfill}%
\pgfsetfillopacity{0.800000}%
\pgfsetlinewidth{0.000000pt}%
\definecolor{currentstroke}{rgb}{0.000000,0.000000,0.000000}%
\pgfsetstrokecolor{currentstroke}%
\pgfsetdash{}{0pt}%
\pgfpathmoveto{\pgfqpoint{4.245525in}{1.958911in}}%
\pgfpathlineto{\pgfqpoint{4.259644in}{1.966353in}}%
\pgfpathlineto{\pgfqpoint{4.273777in}{1.973979in}}%
\pgfpathlineto{\pgfqpoint{4.287923in}{1.981789in}}%
\pgfpathlineto{\pgfqpoint{4.302084in}{1.989782in}}%
\pgfpathlineto{\pgfqpoint{4.310140in}{2.003882in}}%
\pgfpathlineto{\pgfqpoint{4.318192in}{2.017906in}}%
\pgfpathlineto{\pgfqpoint{4.326239in}{2.031852in}}%
\pgfpathlineto{\pgfqpoint{4.334281in}{2.045718in}}%
\pgfpathlineto{\pgfqpoint{4.320120in}{2.037439in}}%
\pgfpathlineto{\pgfqpoint{4.305973in}{2.029344in}}%
\pgfpathlineto{\pgfqpoint{4.291840in}{2.021433in}}%
\pgfpathlineto{\pgfqpoint{4.277720in}{2.013706in}}%
\pgfpathlineto{\pgfqpoint{4.269678in}{2.000113in}}%
\pgfpathlineto{\pgfqpoint{4.261632in}{1.986448in}}%
\pgfpathlineto{\pgfqpoint{4.253581in}{1.972714in}}%
\pgfpathlineto{\pgfqpoint{4.245525in}{1.958911in}}%
\pgfpathclose%
\pgfusepath{fill}%
\end{pgfscope}%
\begin{pgfscope}%
\pgfpathrectangle{\pgfqpoint{1.150000in}{0.150000in}}{\pgfqpoint{5.700000in}{5.700000in}}%
\pgfusepath{clip}%
\pgfsetbuttcap%
\pgfsetroundjoin%
\definecolor{currentfill}{rgb}{0.282327,0.094955,0.417331}%
\pgfsetfillcolor{currentfill}%
\pgfsetfillopacity{0.800000}%
\pgfsetlinewidth{0.000000pt}%
\definecolor{currentstroke}{rgb}{0.000000,0.000000,0.000000}%
\pgfsetstrokecolor{currentstroke}%
\pgfsetdash{}{0pt}%
\pgfpathmoveto{\pgfqpoint{3.826367in}{1.559235in}}%
\pgfpathlineto{\pgfqpoint{3.840324in}{1.561366in}}%
\pgfpathlineto{\pgfqpoint{3.854291in}{1.563680in}}%
\pgfpathlineto{\pgfqpoint{3.868266in}{1.566179in}}%
\pgfpathlineto{\pgfqpoint{3.882251in}{1.568862in}}%
\pgfpathlineto{\pgfqpoint{3.890423in}{1.581860in}}%
\pgfpathlineto{\pgfqpoint{3.898591in}{1.594900in}}%
\pgfpathlineto{\pgfqpoint{3.906753in}{1.607976in}}%
\pgfpathlineto{\pgfqpoint{3.914911in}{1.621084in}}%
\pgfpathlineto{\pgfqpoint{3.900932in}{1.617928in}}%
\pgfpathlineto{\pgfqpoint{3.886963in}{1.614956in}}%
\pgfpathlineto{\pgfqpoint{3.873004in}{1.612169in}}%
\pgfpathlineto{\pgfqpoint{3.859053in}{1.609567in}}%
\pgfpathlineto{\pgfqpoint{3.850890in}{1.596919in}}%
\pgfpathlineto{\pgfqpoint{3.842721in}{1.584312in}}%
\pgfpathlineto{\pgfqpoint{3.834547in}{1.571749in}}%
\pgfpathlineto{\pgfqpoint{3.826367in}{1.559235in}}%
\pgfpathclose%
\pgfusepath{fill}%
\end{pgfscope}%
\begin{pgfscope}%
\pgfpathrectangle{\pgfqpoint{1.150000in}{0.150000in}}{\pgfqpoint{5.700000in}{5.700000in}}%
\pgfusepath{clip}%
\pgfsetbuttcap%
\pgfsetroundjoin%
\definecolor{currentfill}{rgb}{0.246070,0.738910,0.452024}%
\pgfsetfillcolor{currentfill}%
\pgfsetfillopacity{0.800000}%
\pgfsetlinewidth{0.000000pt}%
\definecolor{currentstroke}{rgb}{0.000000,0.000000,0.000000}%
\pgfsetstrokecolor{currentstroke}%
\pgfsetdash{}{0pt}%
\pgfpathmoveto{\pgfqpoint{5.624180in}{3.376461in}}%
\pgfpathlineto{\pgfqpoint{5.639168in}{3.392811in}}%
\pgfpathlineto{\pgfqpoint{5.654179in}{3.409348in}}%
\pgfpathlineto{\pgfqpoint{5.669214in}{3.426071in}}%
\pgfpathlineto{\pgfqpoint{5.684272in}{3.442981in}}%
\pgfpathlineto{\pgfqpoint{5.691618in}{3.445187in}}%
\pgfpathlineto{\pgfqpoint{5.698954in}{3.447280in}}%
\pgfpathlineto{\pgfqpoint{5.706280in}{3.449264in}}%
\pgfpathlineto{\pgfqpoint{5.713596in}{3.451145in}}%
\pgfpathlineto{\pgfqpoint{5.698561in}{3.434645in}}%
\pgfpathlineto{\pgfqpoint{5.683549in}{3.418331in}}%
\pgfpathlineto{\pgfqpoint{5.668560in}{3.402203in}}%
\pgfpathlineto{\pgfqpoint{5.653594in}{3.386260in}}%
\pgfpathlineto{\pgfqpoint{5.646255in}{3.383959in}}%
\pgfpathlineto{\pgfqpoint{5.638907in}{3.381561in}}%
\pgfpathlineto{\pgfqpoint{5.631548in}{3.379064in}}%
\pgfpathlineto{\pgfqpoint{5.624180in}{3.376461in}}%
\pgfpathclose%
\pgfusepath{fill}%
\end{pgfscope}%
\begin{pgfscope}%
\pgfpathrectangle{\pgfqpoint{1.150000in}{0.150000in}}{\pgfqpoint{5.700000in}{5.700000in}}%
\pgfusepath{clip}%
\pgfsetbuttcap%
\pgfsetroundjoin%
\definecolor{currentfill}{rgb}{0.137339,0.662252,0.515571}%
\pgfsetfillcolor{currentfill}%
\pgfsetfillopacity{0.800000}%
\pgfsetlinewidth{0.000000pt}%
\definecolor{currentstroke}{rgb}{0.000000,0.000000,0.000000}%
\pgfsetstrokecolor{currentstroke}%
\pgfsetdash{}{0pt}%
\pgfpathmoveto{\pgfqpoint{5.325958in}{3.120214in}}%
\pgfpathlineto{\pgfqpoint{5.340746in}{3.135601in}}%
\pgfpathlineto{\pgfqpoint{5.355555in}{3.151174in}}%
\pgfpathlineto{\pgfqpoint{5.370386in}{3.166934in}}%
\pgfpathlineto{\pgfqpoint{5.385238in}{3.182882in}}%
\pgfpathlineto{\pgfqpoint{5.392802in}{3.188214in}}%
\pgfpathlineto{\pgfqpoint{5.400357in}{3.193399in}}%
\pgfpathlineto{\pgfqpoint{5.407901in}{3.198439in}}%
\pgfpathlineto{\pgfqpoint{5.415435in}{3.203337in}}%
\pgfpathlineto{\pgfqpoint{5.400596in}{3.187655in}}%
\pgfpathlineto{\pgfqpoint{5.385778in}{3.172159in}}%
\pgfpathlineto{\pgfqpoint{5.370983in}{3.156849in}}%
\pgfpathlineto{\pgfqpoint{5.356208in}{3.141725in}}%
\pgfpathlineto{\pgfqpoint{5.348660in}{3.136550in}}%
\pgfpathlineto{\pgfqpoint{5.341102in}{3.131242in}}%
\pgfpathlineto{\pgfqpoint{5.333535in}{3.125797in}}%
\pgfpathlineto{\pgfqpoint{5.325958in}{3.120214in}}%
\pgfpathclose%
\pgfusepath{fill}%
\end{pgfscope}%
\begin{pgfscope}%
\pgfpathrectangle{\pgfqpoint{1.150000in}{0.150000in}}{\pgfqpoint{5.700000in}{5.700000in}}%
\pgfusepath{clip}%
\pgfsetbuttcap%
\pgfsetroundjoin%
\definecolor{currentfill}{rgb}{0.268510,0.009605,0.335427}%
\pgfsetfillcolor{currentfill}%
\pgfsetfillopacity{0.800000}%
\pgfsetlinewidth{0.000000pt}%
\definecolor{currentstroke}{rgb}{0.000000,0.000000,0.000000}%
\pgfsetstrokecolor{currentstroke}%
\pgfsetdash{}{0pt}%
\pgfpathmoveto{\pgfqpoint{3.471701in}{1.396025in}}%
\pgfpathlineto{\pgfqpoint{3.485598in}{1.392964in}}%
\pgfpathlineto{\pgfqpoint{3.499501in}{1.390094in}}%
\pgfpathlineto{\pgfqpoint{3.513408in}{1.387413in}}%
\pgfpathlineto{\pgfqpoint{3.527321in}{1.384921in}}%
\pgfpathlineto{\pgfqpoint{3.535638in}{1.394154in}}%
\pgfpathlineto{\pgfqpoint{3.543947in}{1.403542in}}%
\pgfpathlineto{\pgfqpoint{3.552248in}{1.413080in}}%
\pgfpathlineto{\pgfqpoint{3.560543in}{1.422761in}}%
\pgfpathlineto{\pgfqpoint{3.546647in}{1.424657in}}%
\pgfpathlineto{\pgfqpoint{3.532757in}{1.426742in}}%
\pgfpathlineto{\pgfqpoint{3.518873in}{1.429016in}}%
\pgfpathlineto{\pgfqpoint{3.504994in}{1.431482in}}%
\pgfpathlineto{\pgfqpoint{3.496683in}{1.422384in}}%
\pgfpathlineto{\pgfqpoint{3.488364in}{1.413438in}}%
\pgfpathlineto{\pgfqpoint{3.480036in}{1.404650in}}%
\pgfpathlineto{\pgfqpoint{3.471701in}{1.396025in}}%
\pgfpathclose%
\pgfusepath{fill}%
\end{pgfscope}%
\begin{pgfscope}%
\pgfpathrectangle{\pgfqpoint{1.150000in}{0.150000in}}{\pgfqpoint{5.700000in}{5.700000in}}%
\pgfusepath{clip}%
\pgfsetbuttcap%
\pgfsetroundjoin%
\definecolor{currentfill}{rgb}{0.188923,0.410910,0.556326}%
\pgfsetfillcolor{currentfill}%
\pgfsetfillopacity{0.800000}%
\pgfsetlinewidth{0.000000pt}%
\definecolor{currentstroke}{rgb}{0.000000,0.000000,0.000000}%
\pgfsetstrokecolor{currentstroke}%
\pgfsetdash{}{0pt}%
\pgfpathmoveto{\pgfqpoint{4.576127in}{2.334326in}}%
\pgfpathlineto{\pgfqpoint{4.590429in}{2.345079in}}%
\pgfpathlineto{\pgfqpoint{4.604748in}{2.356017in}}%
\pgfpathlineto{\pgfqpoint{4.619083in}{2.367139in}}%
\pgfpathlineto{\pgfqpoint{4.633434in}{2.378447in}}%
\pgfpathlineto{\pgfqpoint{4.641390in}{2.391053in}}%
\pgfpathlineto{\pgfqpoint{4.649340in}{2.403524in}}%
\pgfpathlineto{\pgfqpoint{4.657284in}{2.415860in}}%
\pgfpathlineto{\pgfqpoint{4.665221in}{2.428059in}}%
\pgfpathlineto{\pgfqpoint{4.650869in}{2.416630in}}%
\pgfpathlineto{\pgfqpoint{4.636533in}{2.405386in}}%
\pgfpathlineto{\pgfqpoint{4.622214in}{2.394327in}}%
\pgfpathlineto{\pgfqpoint{4.607912in}{2.383452in}}%
\pgfpathlineto{\pgfqpoint{4.599975in}{2.371362in}}%
\pgfpathlineto{\pgfqpoint{4.592032in}{2.359144in}}%
\pgfpathlineto{\pgfqpoint{4.584082in}{2.346798in}}%
\pgfpathlineto{\pgfqpoint{4.576127in}{2.334326in}}%
\pgfpathclose%
\pgfusepath{fill}%
\end{pgfscope}%
\begin{pgfscope}%
\pgfpathrectangle{\pgfqpoint{1.150000in}{0.150000in}}{\pgfqpoint{5.700000in}{5.700000in}}%
\pgfusepath{clip}%
\pgfsetbuttcap%
\pgfsetroundjoin%
\definecolor{currentfill}{rgb}{0.140536,0.530132,0.555659}%
\pgfsetfillcolor{currentfill}%
\pgfsetfillopacity{0.800000}%
\pgfsetlinewidth{0.000000pt}%
\definecolor{currentstroke}{rgb}{0.000000,0.000000,0.000000}%
\pgfsetstrokecolor{currentstroke}%
\pgfsetdash{}{0pt}%
\pgfpathmoveto{\pgfqpoint{4.906739in}{2.703675in}}%
\pgfpathlineto{\pgfqpoint{4.921248in}{2.716917in}}%
\pgfpathlineto{\pgfqpoint{4.935776in}{2.730345in}}%
\pgfpathlineto{\pgfqpoint{4.950322in}{2.743958in}}%
\pgfpathlineto{\pgfqpoint{4.964888in}{2.757758in}}%
\pgfpathlineto{\pgfqpoint{4.972702in}{2.767498in}}%
\pgfpathlineto{\pgfqpoint{4.980507in}{2.777078in}}%
\pgfpathlineto{\pgfqpoint{4.988304in}{2.786498in}}%
\pgfpathlineto{\pgfqpoint{4.996093in}{2.795759in}}%
\pgfpathlineto{\pgfqpoint{4.981531in}{2.782010in}}%
\pgfpathlineto{\pgfqpoint{4.966988in}{2.768447in}}%
\pgfpathlineto{\pgfqpoint{4.952464in}{2.755070in}}%
\pgfpathlineto{\pgfqpoint{4.937959in}{2.741878in}}%
\pgfpathlineto{\pgfqpoint{4.930166in}{2.732553in}}%
\pgfpathlineto{\pgfqpoint{4.922365in}{2.723079in}}%
\pgfpathlineto{\pgfqpoint{4.914556in}{2.713453in}}%
\pgfpathlineto{\pgfqpoint{4.906739in}{2.703675in}}%
\pgfpathclose%
\pgfusepath{fill}%
\end{pgfscope}%
\begin{pgfscope}%
\pgfpathrectangle{\pgfqpoint{1.150000in}{0.150000in}}{\pgfqpoint{5.700000in}{5.700000in}}%
\pgfusepath{clip}%
\pgfsetbuttcap%
\pgfsetroundjoin%
\definecolor{currentfill}{rgb}{0.269308,0.218818,0.509577}%
\pgfsetfillcolor{currentfill}%
\pgfsetfillopacity{0.800000}%
\pgfsetlinewidth{0.000000pt}%
\definecolor{currentstroke}{rgb}{0.000000,0.000000,0.000000}%
\pgfsetstrokecolor{currentstroke}%
\pgfsetdash{}{0pt}%
\pgfpathmoveto{\pgfqpoint{4.124565in}{1.821116in}}%
\pgfpathlineto{\pgfqpoint{4.138634in}{1.827158in}}%
\pgfpathlineto{\pgfqpoint{4.152715in}{1.833384in}}%
\pgfpathlineto{\pgfqpoint{4.166809in}{1.839792in}}%
\pgfpathlineto{\pgfqpoint{4.180915in}{1.846384in}}%
\pgfpathlineto{\pgfqpoint{4.189007in}{1.860628in}}%
\pgfpathlineto{\pgfqpoint{4.197094in}{1.874827in}}%
\pgfpathlineto{\pgfqpoint{4.205177in}{1.888978in}}%
\pgfpathlineto{\pgfqpoint{4.213256in}{1.903078in}}%
\pgfpathlineto{\pgfqpoint{4.199150in}{1.896136in}}%
\pgfpathlineto{\pgfqpoint{4.185056in}{1.889378in}}%
\pgfpathlineto{\pgfqpoint{4.170976in}{1.882804in}}%
\pgfpathlineto{\pgfqpoint{4.156908in}{1.876413in}}%
\pgfpathlineto{\pgfqpoint{4.148829in}{1.862650in}}%
\pgfpathlineto{\pgfqpoint{4.140745in}{1.848844in}}%
\pgfpathlineto{\pgfqpoint{4.132657in}{1.834999in}}%
\pgfpathlineto{\pgfqpoint{4.124565in}{1.821116in}}%
\pgfpathclose%
\pgfusepath{fill}%
\end{pgfscope}%
\begin{pgfscope}%
\pgfpathrectangle{\pgfqpoint{1.150000in}{0.150000in}}{\pgfqpoint{5.700000in}{5.700000in}}%
\pgfusepath{clip}%
\pgfsetbuttcap%
\pgfsetroundjoin%
\definecolor{currentfill}{rgb}{0.283187,0.125848,0.444960}%
\pgfsetfillcolor{currentfill}%
\pgfsetfillopacity{0.800000}%
\pgfsetlinewidth{0.000000pt}%
\definecolor{currentstroke}{rgb}{0.000000,0.000000,0.000000}%
\pgfsetstrokecolor{currentstroke}%
\pgfsetdash{}{0pt}%
\pgfpathmoveto{\pgfqpoint{3.914911in}{1.621084in}}%
\pgfpathlineto{\pgfqpoint{3.928900in}{1.624424in}}%
\pgfpathlineto{\pgfqpoint{3.942899in}{1.627947in}}%
\pgfpathlineto{\pgfqpoint{3.956908in}{1.631654in}}%
\pgfpathlineto{\pgfqpoint{3.970927in}{1.635544in}}%
\pgfpathlineto{\pgfqpoint{3.979076in}{1.649134in}}%
\pgfpathlineto{\pgfqpoint{3.987220in}{1.662740in}}%
\pgfpathlineto{\pgfqpoint{3.995359in}{1.676357in}}%
\pgfpathlineto{\pgfqpoint{4.003494in}{1.689981in}}%
\pgfpathlineto{\pgfqpoint{3.989478in}{1.685648in}}%
\pgfpathlineto{\pgfqpoint{3.975473in}{1.681498in}}%
\pgfpathlineto{\pgfqpoint{3.961479in}{1.677533in}}%
\pgfpathlineto{\pgfqpoint{3.947494in}{1.673751in}}%
\pgfpathlineto{\pgfqpoint{3.939356in}{1.660557in}}%
\pgfpathlineto{\pgfqpoint{3.931212in}{1.647379in}}%
\pgfpathlineto{\pgfqpoint{3.923064in}{1.634219in}}%
\pgfpathlineto{\pgfqpoint{3.914911in}{1.621084in}}%
\pgfpathclose%
\pgfusepath{fill}%
\end{pgfscope}%
\begin{pgfscope}%
\pgfpathrectangle{\pgfqpoint{1.150000in}{0.150000in}}{\pgfqpoint{5.700000in}{5.700000in}}%
\pgfusepath{clip}%
\pgfsetbuttcap%
\pgfsetroundjoin%
\definecolor{currentfill}{rgb}{0.120092,0.600104,0.542530}%
\pgfsetfillcolor{currentfill}%
\pgfsetfillopacity{0.800000}%
\pgfsetlinewidth{0.000000pt}%
\definecolor{currentstroke}{rgb}{0.000000,0.000000,0.000000}%
\pgfsetstrokecolor{currentstroke}%
\pgfsetdash{}{0pt}%
\pgfpathmoveto{\pgfqpoint{5.116509in}{2.920326in}}%
\pgfpathlineto{\pgfqpoint{5.131158in}{2.934799in}}%
\pgfpathlineto{\pgfqpoint{5.145829in}{2.949458in}}%
\pgfpathlineto{\pgfqpoint{5.160519in}{2.964304in}}%
\pgfpathlineto{\pgfqpoint{5.175231in}{2.979337in}}%
\pgfpathlineto{\pgfqpoint{5.182930in}{2.986946in}}%
\pgfpathlineto{\pgfqpoint{5.190621in}{2.994395in}}%
\pgfpathlineto{\pgfqpoint{5.198302in}{3.001685in}}%
\pgfpathlineto{\pgfqpoint{5.205974in}{3.008818in}}%
\pgfpathlineto{\pgfqpoint{5.191271in}{2.993942in}}%
\pgfpathlineto{\pgfqpoint{5.176588in}{2.979253in}}%
\pgfpathlineto{\pgfqpoint{5.161926in}{2.964750in}}%
\pgfpathlineto{\pgfqpoint{5.147284in}{2.950433in}}%
\pgfpathlineto{\pgfqpoint{5.139604in}{2.943130in}}%
\pgfpathlineto{\pgfqpoint{5.131914in}{2.935680in}}%
\pgfpathlineto{\pgfqpoint{5.124216in}{2.928079in}}%
\pgfpathlineto{\pgfqpoint{5.116509in}{2.920326in}}%
\pgfpathclose%
\pgfusepath{fill}%
\end{pgfscope}%
\begin{pgfscope}%
\pgfpathrectangle{\pgfqpoint{1.150000in}{0.150000in}}{\pgfqpoint{5.700000in}{5.700000in}}%
\pgfusepath{clip}%
\pgfsetbuttcap%
\pgfsetroundjoin%
\definecolor{currentfill}{rgb}{0.179019,0.433756,0.557430}%
\pgfsetfillcolor{currentfill}%
\pgfsetfillopacity{0.800000}%
\pgfsetlinewidth{0.000000pt}%
\definecolor{currentstroke}{rgb}{0.000000,0.000000,0.000000}%
\pgfsetstrokecolor{currentstroke}%
\pgfsetdash{}{0pt}%
\pgfpathmoveto{\pgfqpoint{2.191177in}{2.510134in}}%
\pgfpathlineto{\pgfqpoint{2.205563in}{2.485030in}}%
\pgfpathlineto{\pgfqpoint{2.219933in}{2.460244in}}%
\pgfpathlineto{\pgfqpoint{2.234287in}{2.435774in}}%
\pgfpathlineto{\pgfqpoint{2.248627in}{2.411616in}}%
\pgfpathlineto{\pgfqpoint{2.258006in}{2.403235in}}%
\pgfpathlineto{\pgfqpoint{2.267358in}{2.395277in}}%
\pgfpathlineto{\pgfqpoint{2.276684in}{2.387733in}}%
\pgfpathlineto{\pgfqpoint{2.285984in}{2.380597in}}%
\pgfpathlineto{\pgfqpoint{2.271712in}{2.403994in}}%
\pgfpathlineto{\pgfqpoint{2.257427in}{2.427703in}}%
\pgfpathlineto{\pgfqpoint{2.243126in}{2.451725in}}%
\pgfpathlineto{\pgfqpoint{2.228811in}{2.476063in}}%
\pgfpathlineto{\pgfqpoint{2.219444in}{2.483947in}}%
\pgfpathlineto{\pgfqpoint{2.210049in}{2.492248in}}%
\pgfpathlineto{\pgfqpoint{2.200628in}{2.500975in}}%
\pgfpathlineto{\pgfqpoint{2.191177in}{2.510134in}}%
\pgfpathclose%
\pgfusepath{fill}%
\end{pgfscope}%
\begin{pgfscope}%
\pgfpathrectangle{\pgfqpoint{1.150000in}{0.150000in}}{\pgfqpoint{5.700000in}{5.700000in}}%
\pgfusepath{clip}%
\pgfsetbuttcap%
\pgfsetroundjoin%
\definecolor{currentfill}{rgb}{0.276022,0.044167,0.370164}%
\pgfsetfillcolor{currentfill}%
\pgfsetfillopacity{0.800000}%
\pgfsetlinewidth{0.000000pt}%
\definecolor{currentstroke}{rgb}{0.000000,0.000000,0.000000}%
\pgfsetstrokecolor{currentstroke}%
\pgfsetdash{}{0pt}%
\pgfpathmoveto{\pgfqpoint{3.036436in}{1.487336in}}%
\pgfpathlineto{\pgfqpoint{3.050367in}{1.477516in}}%
\pgfpathlineto{\pgfqpoint{3.064298in}{1.467905in}}%
\pgfpathlineto{\pgfqpoint{3.078228in}{1.458502in}}%
\pgfpathlineto{\pgfqpoint{3.092158in}{1.449305in}}%
\pgfpathlineto{\pgfqpoint{3.100760in}{1.451947in}}%
\pgfpathlineto{\pgfqpoint{3.109348in}{1.454871in}}%
\pgfpathlineto{\pgfqpoint{3.117923in}{1.458071in}}%
\pgfpathlineto{\pgfqpoint{3.126484in}{1.461539in}}%
\pgfpathlineto{\pgfqpoint{3.112589in}{1.470038in}}%
\pgfpathlineto{\pgfqpoint{3.098694in}{1.478743in}}%
\pgfpathlineto{\pgfqpoint{3.084799in}{1.487656in}}%
\pgfpathlineto{\pgfqpoint{3.070904in}{1.496777in}}%
\pgfpathlineto{\pgfqpoint{3.062308in}{1.493994in}}%
\pgfpathlineto{\pgfqpoint{3.053699in}{1.491488in}}%
\pgfpathlineto{\pgfqpoint{3.045074in}{1.489267in}}%
\pgfpathlineto{\pgfqpoint{3.036436in}{1.487336in}}%
\pgfpathclose%
\pgfusepath{fill}%
\end{pgfscope}%
\begin{pgfscope}%
\pgfpathrectangle{\pgfqpoint{1.150000in}{0.150000in}}{\pgfqpoint{5.700000in}{5.700000in}}%
\pgfusepath{clip}%
\pgfsetbuttcap%
\pgfsetroundjoin%
\definecolor{currentfill}{rgb}{0.210503,0.363727,0.552206}%
\pgfsetfillcolor{currentfill}%
\pgfsetfillopacity{0.800000}%
\pgfsetlinewidth{0.000000pt}%
\definecolor{currentstroke}{rgb}{0.000000,0.000000,0.000000}%
\pgfsetstrokecolor{currentstroke}%
\pgfsetdash{}{0pt}%
\pgfpathmoveto{\pgfqpoint{4.455242in}{2.190323in}}%
\pgfpathlineto{\pgfqpoint{4.469479in}{2.199996in}}%
\pgfpathlineto{\pgfqpoint{4.483731in}{2.209853in}}%
\pgfpathlineto{\pgfqpoint{4.497999in}{2.219894in}}%
\pgfpathlineto{\pgfqpoint{4.512282in}{2.230120in}}%
\pgfpathlineto{\pgfqpoint{4.520282in}{2.243564in}}%
\pgfpathlineto{\pgfqpoint{4.528277in}{2.256892in}}%
\pgfpathlineto{\pgfqpoint{4.536266in}{2.270101in}}%
\pgfpathlineto{\pgfqpoint{4.544249in}{2.283191in}}%
\pgfpathlineto{\pgfqpoint{4.529965in}{2.272777in}}%
\pgfpathlineto{\pgfqpoint{4.515696in}{2.262548in}}%
\pgfpathlineto{\pgfqpoint{4.501442in}{2.252503in}}%
\pgfpathlineto{\pgfqpoint{4.487205in}{2.242642in}}%
\pgfpathlineto{\pgfqpoint{4.479222in}{2.229728in}}%
\pgfpathlineto{\pgfqpoint{4.471234in}{2.216703in}}%
\pgfpathlineto{\pgfqpoint{4.463241in}{2.203567in}}%
\pgfpathlineto{\pgfqpoint{4.455242in}{2.190323in}}%
\pgfpathclose%
\pgfusepath{fill}%
\end{pgfscope}%
\begin{pgfscope}%
\pgfpathrectangle{\pgfqpoint{1.150000in}{0.150000in}}{\pgfqpoint{5.700000in}{5.700000in}}%
\pgfusepath{clip}%
\pgfsetbuttcap%
\pgfsetroundjoin%
\definecolor{currentfill}{rgb}{0.268510,0.009605,0.335427}%
\pgfsetfillcolor{currentfill}%
\pgfsetfillopacity{0.800000}%
\pgfsetlinewidth{0.000000pt}%
\definecolor{currentstroke}{rgb}{0.000000,0.000000,0.000000}%
\pgfsetstrokecolor{currentstroke}%
\pgfsetdash{}{0pt}%
\pgfpathmoveto{\pgfqpoint{3.237673in}{1.400875in}}%
\pgfpathlineto{\pgfqpoint{3.251578in}{1.394195in}}%
\pgfpathlineto{\pgfqpoint{3.265485in}{1.387713in}}%
\pgfpathlineto{\pgfqpoint{3.279394in}{1.381429in}}%
\pgfpathlineto{\pgfqpoint{3.293305in}{1.375341in}}%
\pgfpathlineto{\pgfqpoint{3.301763in}{1.381094in}}%
\pgfpathlineto{\pgfqpoint{3.310211in}{1.387076in}}%
\pgfpathlineto{\pgfqpoint{3.318647in}{1.393281in}}%
\pgfpathlineto{\pgfqpoint{3.327074in}{1.399702in}}%
\pgfpathlineto{\pgfqpoint{3.313189in}{1.405129in}}%
\pgfpathlineto{\pgfqpoint{3.299307in}{1.410753in}}%
\pgfpathlineto{\pgfqpoint{3.285428in}{1.416573in}}%
\pgfpathlineto{\pgfqpoint{3.271551in}{1.422592in}}%
\pgfpathlineto{\pgfqpoint{3.263098in}{1.416819in}}%
\pgfpathlineto{\pgfqpoint{3.254634in}{1.411271in}}%
\pgfpathlineto{\pgfqpoint{3.246159in}{1.405954in}}%
\pgfpathlineto{\pgfqpoint{3.237673in}{1.400875in}}%
\pgfpathclose%
\pgfusepath{fill}%
\end{pgfscope}%
\begin{pgfscope}%
\pgfpathrectangle{\pgfqpoint{1.150000in}{0.150000in}}{\pgfqpoint{5.700000in}{5.700000in}}%
\pgfusepath{clip}%
\pgfsetbuttcap%
\pgfsetroundjoin%
\definecolor{currentfill}{rgb}{0.296479,0.761561,0.424223}%
\pgfsetfillcolor{currentfill}%
\pgfsetfillopacity{0.800000}%
\pgfsetlinewidth{0.000000pt}%
\definecolor{currentstroke}{rgb}{0.000000,0.000000,0.000000}%
\pgfsetstrokecolor{currentstroke}%
\pgfsetdash{}{0pt}%
\pgfpathmoveto{\pgfqpoint{5.713596in}{3.451145in}}%
\pgfpathlineto{\pgfqpoint{5.728655in}{3.467831in}}%
\pgfpathlineto{\pgfqpoint{5.743737in}{3.484703in}}%
\pgfpathlineto{\pgfqpoint{5.758843in}{3.501763in}}%
\pgfpathlineto{\pgfqpoint{5.773973in}{3.519009in}}%
\pgfpathlineto{\pgfqpoint{5.781254in}{3.520356in}}%
\pgfpathlineto{\pgfqpoint{5.788525in}{3.521600in}}%
\pgfpathlineto{\pgfqpoint{5.795785in}{3.522746in}}%
\pgfpathlineto{\pgfqpoint{5.803035in}{3.523799in}}%
\pgfpathlineto{\pgfqpoint{5.787930in}{3.507001in}}%
\pgfpathlineto{\pgfqpoint{5.772849in}{3.490388in}}%
\pgfpathlineto{\pgfqpoint{5.757792in}{3.473961in}}%
\pgfpathlineto{\pgfqpoint{5.742758in}{3.457719in}}%
\pgfpathlineto{\pgfqpoint{5.735482in}{3.456208in}}%
\pgfpathlineto{\pgfqpoint{5.728197in}{3.454612in}}%
\pgfpathlineto{\pgfqpoint{5.720901in}{3.452925in}}%
\pgfpathlineto{\pgfqpoint{5.713596in}{3.451145in}}%
\pgfpathclose%
\pgfusepath{fill}%
\end{pgfscope}%
\begin{pgfscope}%
\pgfpathrectangle{\pgfqpoint{1.150000in}{0.150000in}}{\pgfqpoint{5.700000in}{5.700000in}}%
\pgfusepath{clip}%
\pgfsetbuttcap%
\pgfsetroundjoin%
\definecolor{currentfill}{rgb}{0.273006,0.204520,0.501721}%
\pgfsetfillcolor{currentfill}%
\pgfsetfillopacity{0.800000}%
\pgfsetlinewidth{0.000000pt}%
\definecolor{currentstroke}{rgb}{0.000000,0.000000,0.000000}%
\pgfsetstrokecolor{currentstroke}%
\pgfsetdash{}{0pt}%
\pgfpathmoveto{\pgfqpoint{2.609245in}{1.867908in}}%
\pgfpathlineto{\pgfqpoint{2.623332in}{1.850996in}}%
\pgfpathlineto{\pgfqpoint{2.637413in}{1.834332in}}%
\pgfpathlineto{\pgfqpoint{2.651486in}{1.817912in}}%
\pgfpathlineto{\pgfqpoint{2.665553in}{1.801735in}}%
\pgfpathlineto{\pgfqpoint{2.674536in}{1.797894in}}%
\pgfpathlineto{\pgfqpoint{2.683499in}{1.794428in}}%
\pgfpathlineto{\pgfqpoint{2.692441in}{1.791330in}}%
\pgfpathlineto{\pgfqpoint{2.701363in}{1.788591in}}%
\pgfpathlineto{\pgfqpoint{2.687349in}{1.804015in}}%
\pgfpathlineto{\pgfqpoint{2.673329in}{1.819680in}}%
\pgfpathlineto{\pgfqpoint{2.659303in}{1.835590in}}%
\pgfpathlineto{\pgfqpoint{2.645270in}{1.851745in}}%
\pgfpathlineto{\pgfqpoint{2.636296in}{1.855224in}}%
\pgfpathlineto{\pgfqpoint{2.627300in}{1.859072in}}%
\pgfpathlineto{\pgfqpoint{2.618283in}{1.863298in}}%
\pgfpathlineto{\pgfqpoint{2.609245in}{1.867908in}}%
\pgfpathclose%
\pgfusepath{fill}%
\end{pgfscope}%
\begin{pgfscope}%
\pgfpathrectangle{\pgfqpoint{1.150000in}{0.150000in}}{\pgfqpoint{5.700000in}{5.700000in}}%
\pgfusepath{clip}%
\pgfsetbuttcap%
\pgfsetroundjoin%
\definecolor{currentfill}{rgb}{0.267004,0.004874,0.329415}%
\pgfsetfillcolor{currentfill}%
\pgfsetfillopacity{0.800000}%
\pgfsetlinewidth{0.000000pt}%
\definecolor{currentstroke}{rgb}{0.000000,0.000000,0.000000}%
\pgfsetstrokecolor{currentstroke}%
\pgfsetdash{}{0pt}%
\pgfpathmoveto{\pgfqpoint{3.382645in}{1.379945in}}%
\pgfpathlineto{\pgfqpoint{3.396546in}{1.375490in}}%
\pgfpathlineto{\pgfqpoint{3.410451in}{1.371228in}}%
\pgfpathlineto{\pgfqpoint{3.424360in}{1.367157in}}%
\pgfpathlineto{\pgfqpoint{3.438274in}{1.363278in}}%
\pgfpathlineto{\pgfqpoint{3.446644in}{1.371190in}}%
\pgfpathlineto{\pgfqpoint{3.455005in}{1.379289in}}%
\pgfpathlineto{\pgfqpoint{3.463357in}{1.387569in}}%
\pgfpathlineto{\pgfqpoint{3.471701in}{1.396025in}}%
\pgfpathlineto{\pgfqpoint{3.457808in}{1.399277in}}%
\pgfpathlineto{\pgfqpoint{3.443920in}{1.402719in}}%
\pgfpathlineto{\pgfqpoint{3.430037in}{1.406354in}}%
\pgfpathlineto{\pgfqpoint{3.416158in}{1.410181in}}%
\pgfpathlineto{\pgfqpoint{3.407793in}{1.402341in}}%
\pgfpathlineto{\pgfqpoint{3.399419in}{1.394684in}}%
\pgfpathlineto{\pgfqpoint{3.391037in}{1.387217in}}%
\pgfpathlineto{\pgfqpoint{3.382645in}{1.379945in}}%
\pgfpathclose%
\pgfusepath{fill}%
\end{pgfscope}%
\begin{pgfscope}%
\pgfpathrectangle{\pgfqpoint{1.150000in}{0.150000in}}{\pgfqpoint{5.700000in}{5.700000in}}%
\pgfusepath{clip}%
\pgfsetbuttcap%
\pgfsetroundjoin%
\definecolor{currentfill}{rgb}{0.265145,0.232956,0.516599}%
\pgfsetfillcolor{currentfill}%
\pgfsetfillopacity{0.800000}%
\pgfsetlinewidth{0.000000pt}%
\definecolor{currentstroke}{rgb}{0.000000,0.000000,0.000000}%
\pgfsetstrokecolor{currentstroke}%
\pgfsetdash{}{0pt}%
\pgfpathmoveto{\pgfqpoint{2.552818in}{1.938056in}}%
\pgfpathlineto{\pgfqpoint{2.566937in}{1.920140in}}%
\pgfpathlineto{\pgfqpoint{2.581047in}{1.902478in}}%
\pgfpathlineto{\pgfqpoint{2.595150in}{1.885068in}}%
\pgfpathlineto{\pgfqpoint{2.609245in}{1.867908in}}%
\pgfpathlineto{\pgfqpoint{2.618283in}{1.863298in}}%
\pgfpathlineto{\pgfqpoint{2.627300in}{1.859072in}}%
\pgfpathlineto{\pgfqpoint{2.636296in}{1.855224in}}%
\pgfpathlineto{\pgfqpoint{2.645270in}{1.851745in}}%
\pgfpathlineto{\pgfqpoint{2.631230in}{1.868147in}}%
\pgfpathlineto{\pgfqpoint{2.617183in}{1.884799in}}%
\pgfpathlineto{\pgfqpoint{2.603129in}{1.901701in}}%
\pgfpathlineto{\pgfqpoint{2.589067in}{1.918856in}}%
\pgfpathlineto{\pgfqpoint{2.580038in}{1.923080in}}%
\pgfpathlineto{\pgfqpoint{2.570987in}{1.927682in}}%
\pgfpathlineto{\pgfqpoint{2.561914in}{1.932672in}}%
\pgfpathlineto{\pgfqpoint{2.552818in}{1.938056in}}%
\pgfpathclose%
\pgfusepath{fill}%
\end{pgfscope}%
\begin{pgfscope}%
\pgfpathrectangle{\pgfqpoint{1.150000in}{0.150000in}}{\pgfqpoint{5.700000in}{5.700000in}}%
\pgfusepath{clip}%
\pgfsetbuttcap%
\pgfsetroundjoin%
\definecolor{currentfill}{rgb}{0.278012,0.180367,0.486697}%
\pgfsetfillcolor{currentfill}%
\pgfsetfillopacity{0.800000}%
\pgfsetlinewidth{0.000000pt}%
\definecolor{currentstroke}{rgb}{0.000000,0.000000,0.000000}%
\pgfsetstrokecolor{currentstroke}%
\pgfsetdash{}{0pt}%
\pgfpathmoveto{\pgfqpoint{2.665553in}{1.801735in}}%
\pgfpathlineto{\pgfqpoint{2.679613in}{1.785801in}}%
\pgfpathlineto{\pgfqpoint{2.693667in}{1.770106in}}%
\pgfpathlineto{\pgfqpoint{2.707715in}{1.754650in}}%
\pgfpathlineto{\pgfqpoint{2.721758in}{1.739431in}}%
\pgfpathlineto{\pgfqpoint{2.730689in}{1.736354in}}%
\pgfpathlineto{\pgfqpoint{2.739599in}{1.733643in}}%
\pgfpathlineto{\pgfqpoint{2.748490in}{1.731291in}}%
\pgfpathlineto{\pgfqpoint{2.757362in}{1.729289in}}%
\pgfpathlineto{\pgfqpoint{2.743371in}{1.743759in}}%
\pgfpathlineto{\pgfqpoint{2.729374in}{1.758465in}}%
\pgfpathlineto{\pgfqpoint{2.715371in}{1.773409in}}%
\pgfpathlineto{\pgfqpoint{2.701363in}{1.788591in}}%
\pgfpathlineto{\pgfqpoint{2.692441in}{1.791330in}}%
\pgfpathlineto{\pgfqpoint{2.683499in}{1.794428in}}%
\pgfpathlineto{\pgfqpoint{2.674536in}{1.797894in}}%
\pgfpathlineto{\pgfqpoint{2.665553in}{1.801735in}}%
\pgfpathclose%
\pgfusepath{fill}%
\end{pgfscope}%
\begin{pgfscope}%
\pgfpathrectangle{\pgfqpoint{1.150000in}{0.150000in}}{\pgfqpoint{5.700000in}{5.700000in}}%
\pgfusepath{clip}%
\pgfsetbuttcap%
\pgfsetroundjoin%
\definecolor{currentfill}{rgb}{0.129933,0.559582,0.551864}%
\pgfsetfillcolor{currentfill}%
\pgfsetfillopacity{0.800000}%
\pgfsetlinewidth{0.000000pt}%
\definecolor{currentstroke}{rgb}{0.000000,0.000000,0.000000}%
\pgfsetstrokecolor{currentstroke}%
\pgfsetdash{}{0pt}%
\pgfpathmoveto{\pgfqpoint{1.997417in}{2.911130in}}%
\pgfpathlineto{\pgfqpoint{2.012024in}{2.881283in}}%
\pgfpathlineto{\pgfqpoint{2.026609in}{2.851807in}}%
\pgfpathlineto{\pgfqpoint{2.041174in}{2.822698in}}%
\pgfpathlineto{\pgfqpoint{2.055719in}{2.793953in}}%
\pgfpathlineto{\pgfqpoint{2.065275in}{2.784220in}}%
\pgfpathlineto{\pgfqpoint{2.074801in}{2.774920in}}%
\pgfpathlineto{\pgfqpoint{2.084299in}{2.766046in}}%
\pgfpathlineto{\pgfqpoint{2.093768in}{2.757591in}}%
\pgfpathlineto{\pgfqpoint{2.079298in}{2.785584in}}%
\pgfpathlineto{\pgfqpoint{2.064808in}{2.813940in}}%
\pgfpathlineto{\pgfqpoint{2.050298in}{2.842660in}}%
\pgfpathlineto{\pgfqpoint{2.035768in}{2.871749in}}%
\pgfpathlineto{\pgfqpoint{2.026225in}{2.880943in}}%
\pgfpathlineto{\pgfqpoint{2.016653in}{2.890566in}}%
\pgfpathlineto{\pgfqpoint{2.007050in}{2.900626in}}%
\pgfpathlineto{\pgfqpoint{1.997417in}{2.911130in}}%
\pgfpathclose%
\pgfusepath{fill}%
\end{pgfscope}%
\begin{pgfscope}%
\pgfpathrectangle{\pgfqpoint{1.150000in}{0.150000in}}{\pgfqpoint{5.700000in}{5.700000in}}%
\pgfusepath{clip}%
\pgfsetbuttcap%
\pgfsetroundjoin%
\definecolor{currentfill}{rgb}{0.156270,0.489624,0.557936}%
\pgfsetfillcolor{currentfill}%
\pgfsetfillopacity{0.800000}%
\pgfsetlinewidth{0.000000pt}%
\definecolor{currentstroke}{rgb}{0.000000,0.000000,0.000000}%
\pgfsetstrokecolor{currentstroke}%
\pgfsetdash{}{0pt}%
\pgfpathmoveto{\pgfqpoint{4.786069in}{2.568807in}}%
\pgfpathlineto{\pgfqpoint{4.800508in}{2.581303in}}%
\pgfpathlineto{\pgfqpoint{4.814966in}{2.593984in}}%
\pgfpathlineto{\pgfqpoint{4.829441in}{2.606851in}}%
\pgfpathlineto{\pgfqpoint{4.843935in}{2.619904in}}%
\pgfpathlineto{\pgfqpoint{4.851812in}{2.630920in}}%
\pgfpathlineto{\pgfqpoint{4.859681in}{2.641779in}}%
\pgfpathlineto{\pgfqpoint{4.867543in}{2.652483in}}%
\pgfpathlineto{\pgfqpoint{4.875397in}{2.663031in}}%
\pgfpathlineto{\pgfqpoint{4.860904in}{2.649959in}}%
\pgfpathlineto{\pgfqpoint{4.846430in}{2.637073in}}%
\pgfpathlineto{\pgfqpoint{4.831975in}{2.624372in}}%
\pgfpathlineto{\pgfqpoint{4.817537in}{2.611857in}}%
\pgfpathlineto{\pgfqpoint{4.809681in}{2.601316in}}%
\pgfpathlineto{\pgfqpoint{4.801817in}{2.590627in}}%
\pgfpathlineto{\pgfqpoint{4.793947in}{2.579791in}}%
\pgfpathlineto{\pgfqpoint{4.786069in}{2.568807in}}%
\pgfpathclose%
\pgfusepath{fill}%
\end{pgfscope}%
\begin{pgfscope}%
\pgfpathrectangle{\pgfqpoint{1.150000in}{0.150000in}}{\pgfqpoint{5.700000in}{5.700000in}}%
\pgfusepath{clip}%
\pgfsetbuttcap%
\pgfsetroundjoin%
\definecolor{currentfill}{rgb}{0.235526,0.309527,0.542944}%
\pgfsetfillcolor{currentfill}%
\pgfsetfillopacity{0.800000}%
\pgfsetlinewidth{0.000000pt}%
\definecolor{currentstroke}{rgb}{0.000000,0.000000,0.000000}%
\pgfsetstrokecolor{currentstroke}%
\pgfsetdash{}{0pt}%
\pgfpathmoveto{\pgfqpoint{4.334281in}{2.045718in}}%
\pgfpathlineto{\pgfqpoint{4.348457in}{2.054180in}}%
\pgfpathlineto{\pgfqpoint{4.362646in}{2.062827in}}%
\pgfpathlineto{\pgfqpoint{4.376850in}{2.071657in}}%
\pgfpathlineto{\pgfqpoint{4.391069in}{2.080670in}}%
\pgfpathlineto{\pgfqpoint{4.399108in}{2.094719in}}%
\pgfpathlineto{\pgfqpoint{4.407142in}{2.108674in}}%
\pgfpathlineto{\pgfqpoint{4.415171in}{2.122533in}}%
\pgfpathlineto{\pgfqpoint{4.423196in}{2.136294in}}%
\pgfpathlineto{\pgfqpoint{4.408975in}{2.127027in}}%
\pgfpathlineto{\pgfqpoint{4.394770in}{2.117943in}}%
\pgfpathlineto{\pgfqpoint{4.380579in}{2.109043in}}%
\pgfpathlineto{\pgfqpoint{4.366403in}{2.100327in}}%
\pgfpathlineto{\pgfqpoint{4.358380in}{2.086807in}}%
\pgfpathlineto{\pgfqpoint{4.350352in}{2.073197in}}%
\pgfpathlineto{\pgfqpoint{4.342319in}{2.059500in}}%
\pgfpathlineto{\pgfqpoint{4.334281in}{2.045718in}}%
\pgfpathclose%
\pgfusepath{fill}%
\end{pgfscope}%
\begin{pgfscope}%
\pgfpathrectangle{\pgfqpoint{1.150000in}{0.150000in}}{\pgfqpoint{5.700000in}{5.700000in}}%
\pgfusepath{clip}%
\pgfsetbuttcap%
\pgfsetroundjoin%
\definecolor{currentfill}{rgb}{0.255645,0.260703,0.528312}%
\pgfsetfillcolor{currentfill}%
\pgfsetfillopacity{0.800000}%
\pgfsetlinewidth{0.000000pt}%
\definecolor{currentstroke}{rgb}{0.000000,0.000000,0.000000}%
\pgfsetstrokecolor{currentstroke}%
\pgfsetdash{}{0pt}%
\pgfpathmoveto{\pgfqpoint{2.496257in}{2.012294in}}%
\pgfpathlineto{\pgfqpoint{2.510411in}{1.993345in}}%
\pgfpathlineto{\pgfqpoint{2.524555in}{1.974656in}}%
\pgfpathlineto{\pgfqpoint{2.538691in}{1.956227in}}%
\pgfpathlineto{\pgfqpoint{2.552818in}{1.938056in}}%
\pgfpathlineto{\pgfqpoint{2.561914in}{1.932672in}}%
\pgfpathlineto{\pgfqpoint{2.570987in}{1.927682in}}%
\pgfpathlineto{\pgfqpoint{2.580038in}{1.923080in}}%
\pgfpathlineto{\pgfqpoint{2.589067in}{1.918856in}}%
\pgfpathlineto{\pgfqpoint{2.574998in}{1.936265in}}%
\pgfpathlineto{\pgfqpoint{2.560920in}{1.953930in}}%
\pgfpathlineto{\pgfqpoint{2.546834in}{1.971853in}}%
\pgfpathlineto{\pgfqpoint{2.532740in}{1.990037in}}%
\pgfpathlineto{\pgfqpoint{2.523654in}{1.995010in}}%
\pgfpathlineto{\pgfqpoint{2.514545in}{2.000373in}}%
\pgfpathlineto{\pgfqpoint{2.505413in}{2.006131in}}%
\pgfpathlineto{\pgfqpoint{2.496257in}{2.012294in}}%
\pgfpathclose%
\pgfusepath{fill}%
\end{pgfscope}%
\begin{pgfscope}%
\pgfpathrectangle{\pgfqpoint{1.150000in}{0.150000in}}{\pgfqpoint{5.700000in}{5.700000in}}%
\pgfusepath{clip}%
\pgfsetbuttcap%
\pgfsetroundjoin%
\definecolor{currentfill}{rgb}{0.281412,0.155834,0.469201}%
\pgfsetfillcolor{currentfill}%
\pgfsetfillopacity{0.800000}%
\pgfsetlinewidth{0.000000pt}%
\definecolor{currentstroke}{rgb}{0.000000,0.000000,0.000000}%
\pgfsetstrokecolor{currentstroke}%
\pgfsetdash{}{0pt}%
\pgfpathmoveto{\pgfqpoint{2.721758in}{1.739431in}}%
\pgfpathlineto{\pgfqpoint{2.735795in}{1.724447in}}%
\pgfpathlineto{\pgfqpoint{2.749826in}{1.709697in}}%
\pgfpathlineto{\pgfqpoint{2.763853in}{1.695180in}}%
\pgfpathlineto{\pgfqpoint{2.777874in}{1.680894in}}%
\pgfpathlineto{\pgfqpoint{2.786754in}{1.678578in}}%
\pgfpathlineto{\pgfqpoint{2.795615in}{1.676618in}}%
\pgfpathlineto{\pgfqpoint{2.804458in}{1.675008in}}%
\pgfpathlineto{\pgfqpoint{2.813282in}{1.673739in}}%
\pgfpathlineto{\pgfqpoint{2.799309in}{1.687279in}}%
\pgfpathlineto{\pgfqpoint{2.785331in}{1.701050in}}%
\pgfpathlineto{\pgfqpoint{2.771349in}{1.715053in}}%
\pgfpathlineto{\pgfqpoint{2.757362in}{1.729289in}}%
\pgfpathlineto{\pgfqpoint{2.748490in}{1.731291in}}%
\pgfpathlineto{\pgfqpoint{2.739599in}{1.733643in}}%
\pgfpathlineto{\pgfqpoint{2.730689in}{1.736354in}}%
\pgfpathlineto{\pgfqpoint{2.721758in}{1.739431in}}%
\pgfpathclose%
\pgfusepath{fill}%
\end{pgfscope}%
\begin{pgfscope}%
\pgfpathrectangle{\pgfqpoint{1.150000in}{0.150000in}}{\pgfqpoint{5.700000in}{5.700000in}}%
\pgfusepath{clip}%
\pgfsetbuttcap%
\pgfsetroundjoin%
\definecolor{currentfill}{rgb}{0.280255,0.165693,0.476498}%
\pgfsetfillcolor{currentfill}%
\pgfsetfillopacity{0.800000}%
\pgfsetlinewidth{0.000000pt}%
\definecolor{currentstroke}{rgb}{0.000000,0.000000,0.000000}%
\pgfsetstrokecolor{currentstroke}%
\pgfsetdash{}{0pt}%
\pgfpathmoveto{\pgfqpoint{4.003494in}{1.689981in}}%
\pgfpathlineto{\pgfqpoint{4.017520in}{1.694498in}}%
\pgfpathlineto{\pgfqpoint{4.031558in}{1.699197in}}%
\pgfpathlineto{\pgfqpoint{4.045607in}{1.704080in}}%
\pgfpathlineto{\pgfqpoint{4.059667in}{1.709145in}}%
\pgfpathlineto{\pgfqpoint{4.067795in}{1.723197in}}%
\pgfpathlineto{\pgfqpoint{4.075918in}{1.737241in}}%
\pgfpathlineto{\pgfqpoint{4.084037in}{1.751272in}}%
\pgfpathlineto{\pgfqpoint{4.092151in}{1.765287in}}%
\pgfpathlineto{\pgfqpoint{4.078093in}{1.759809in}}%
\pgfpathlineto{\pgfqpoint{4.064046in}{1.754514in}}%
\pgfpathlineto{\pgfqpoint{4.050011in}{1.749402in}}%
\pgfpathlineto{\pgfqpoint{4.035987in}{1.744474in}}%
\pgfpathlineto{\pgfqpoint{4.027871in}{1.730859in}}%
\pgfpathlineto{\pgfqpoint{4.019750in}{1.717236in}}%
\pgfpathlineto{\pgfqpoint{4.011624in}{1.703609in}}%
\pgfpathlineto{\pgfqpoint{4.003494in}{1.689981in}}%
\pgfpathclose%
\pgfusepath{fill}%
\end{pgfscope}%
\begin{pgfscope}%
\pgfpathrectangle{\pgfqpoint{1.150000in}{0.150000in}}{\pgfqpoint{5.700000in}{5.700000in}}%
\pgfusepath{clip}%
\pgfsetbuttcap%
\pgfsetroundjoin%
\definecolor{currentfill}{rgb}{0.166383,0.690856,0.496502}%
\pgfsetfillcolor{currentfill}%
\pgfsetfillopacity{0.800000}%
\pgfsetlinewidth{0.000000pt}%
\definecolor{currentstroke}{rgb}{0.000000,0.000000,0.000000}%
\pgfsetstrokecolor{currentstroke}%
\pgfsetdash{}{0pt}%
\pgfpathmoveto{\pgfqpoint{5.415435in}{3.203337in}}%
\pgfpathlineto{\pgfqpoint{5.430296in}{3.219206in}}%
\pgfpathlineto{\pgfqpoint{5.445179in}{3.235261in}}%
\pgfpathlineto{\pgfqpoint{5.460085in}{3.251504in}}%
\pgfpathlineto{\pgfqpoint{5.475013in}{3.267934in}}%
\pgfpathlineto{\pgfqpoint{5.482522in}{3.272406in}}%
\pgfpathlineto{\pgfqpoint{5.490021in}{3.276734in}}%
\pgfpathlineto{\pgfqpoint{5.497510in}{3.280922in}}%
\pgfpathlineto{\pgfqpoint{5.504988in}{3.284973in}}%
\pgfpathlineto{\pgfqpoint{5.490076in}{3.268845in}}%
\pgfpathlineto{\pgfqpoint{5.475186in}{3.252904in}}%
\pgfpathlineto{\pgfqpoint{5.460319in}{3.237149in}}%
\pgfpathlineto{\pgfqpoint{5.445473in}{3.221580in}}%
\pgfpathlineto{\pgfqpoint{5.437978in}{3.217215in}}%
\pgfpathlineto{\pgfqpoint{5.430474in}{3.212722in}}%
\pgfpathlineto{\pgfqpoint{5.422959in}{3.208097in}}%
\pgfpathlineto{\pgfqpoint{5.415435in}{3.203337in}}%
\pgfpathclose%
\pgfusepath{fill}%
\end{pgfscope}%
\begin{pgfscope}%
\pgfpathrectangle{\pgfqpoint{1.150000in}{0.150000in}}{\pgfqpoint{5.700000in}{5.700000in}}%
\pgfusepath{clip}%
\pgfsetbuttcap%
\pgfsetroundjoin%
\definecolor{currentfill}{rgb}{0.243113,0.292092,0.538516}%
\pgfsetfillcolor{currentfill}%
\pgfsetfillopacity{0.800000}%
\pgfsetlinewidth{0.000000pt}%
\definecolor{currentstroke}{rgb}{0.000000,0.000000,0.000000}%
\pgfsetstrokecolor{currentstroke}%
\pgfsetdash{}{0pt}%
\pgfpathmoveto{\pgfqpoint{2.439545in}{2.090748in}}%
\pgfpathlineto{\pgfqpoint{2.453738in}{2.070732in}}%
\pgfpathlineto{\pgfqpoint{2.467921in}{2.050986in}}%
\pgfpathlineto{\pgfqpoint{2.482094in}{2.031508in}}%
\pgfpathlineto{\pgfqpoint{2.496257in}{2.012294in}}%
\pgfpathlineto{\pgfqpoint{2.505413in}{2.006131in}}%
\pgfpathlineto{\pgfqpoint{2.514545in}{2.000373in}}%
\pgfpathlineto{\pgfqpoint{2.523654in}{1.995010in}}%
\pgfpathlineto{\pgfqpoint{2.532740in}{1.990037in}}%
\pgfpathlineto{\pgfqpoint{2.518636in}{2.008482in}}%
\pgfpathlineto{\pgfqpoint{2.504524in}{2.027191in}}%
\pgfpathlineto{\pgfqpoint{2.490402in}{2.046167in}}%
\pgfpathlineto{\pgfqpoint{2.476271in}{2.065410in}}%
\pgfpathlineto{\pgfqpoint{2.467126in}{2.071139in}}%
\pgfpathlineto{\pgfqpoint{2.457957in}{2.077267in}}%
\pgfpathlineto{\pgfqpoint{2.448763in}{2.083800in}}%
\pgfpathlineto{\pgfqpoint{2.439545in}{2.090748in}}%
\pgfpathclose%
\pgfusepath{fill}%
\end{pgfscope}%
\begin{pgfscope}%
\pgfpathrectangle{\pgfqpoint{1.150000in}{0.150000in}}{\pgfqpoint{5.700000in}{5.700000in}}%
\pgfusepath{clip}%
\pgfsetbuttcap%
\pgfsetroundjoin%
\definecolor{currentfill}{rgb}{0.283072,0.130895,0.449241}%
\pgfsetfillcolor{currentfill}%
\pgfsetfillopacity{0.800000}%
\pgfsetlinewidth{0.000000pt}%
\definecolor{currentstroke}{rgb}{0.000000,0.000000,0.000000}%
\pgfsetstrokecolor{currentstroke}%
\pgfsetdash{}{0pt}%
\pgfpathmoveto{\pgfqpoint{2.777874in}{1.680894in}}%
\pgfpathlineto{\pgfqpoint{2.791891in}{1.666838in}}%
\pgfpathlineto{\pgfqpoint{2.805904in}{1.653009in}}%
\pgfpathlineto{\pgfqpoint{2.819912in}{1.639408in}}%
\pgfpathlineto{\pgfqpoint{2.833916in}{1.626033in}}%
\pgfpathlineto{\pgfqpoint{2.842748in}{1.624473in}}%
\pgfpathlineto{\pgfqpoint{2.851562in}{1.623262in}}%
\pgfpathlineto{\pgfqpoint{2.860357in}{1.622390in}}%
\pgfpathlineto{\pgfqpoint{2.869135in}{1.621850in}}%
\pgfpathlineto{\pgfqpoint{2.855177in}{1.634484in}}%
\pgfpathlineto{\pgfqpoint{2.841216in}{1.647342in}}%
\pgfpathlineto{\pgfqpoint{2.827251in}{1.660427in}}%
\pgfpathlineto{\pgfqpoint{2.813282in}{1.673739in}}%
\pgfpathlineto{\pgfqpoint{2.804458in}{1.675008in}}%
\pgfpathlineto{\pgfqpoint{2.795615in}{1.676618in}}%
\pgfpathlineto{\pgfqpoint{2.786754in}{1.678578in}}%
\pgfpathlineto{\pgfqpoint{2.777874in}{1.680894in}}%
\pgfpathclose%
\pgfusepath{fill}%
\end{pgfscope}%
\begin{pgfscope}%
\pgfpathrectangle{\pgfqpoint{1.150000in}{0.150000in}}{\pgfqpoint{5.700000in}{5.700000in}}%
\pgfusepath{clip}%
\pgfsetbuttcap%
\pgfsetroundjoin%
\definecolor{currentfill}{rgb}{0.273809,0.031497,0.358853}%
\pgfsetfillcolor{currentfill}%
\pgfsetfillopacity{0.800000}%
\pgfsetlinewidth{0.000000pt}%
\definecolor{currentstroke}{rgb}{0.000000,0.000000,0.000000}%
\pgfsetstrokecolor{currentstroke}%
\pgfsetdash{}{0pt}%
\pgfpathmoveto{\pgfqpoint{3.092158in}{1.449305in}}%
\pgfpathlineto{\pgfqpoint{3.106088in}{1.440315in}}%
\pgfpathlineto{\pgfqpoint{3.120018in}{1.431530in}}%
\pgfpathlineto{\pgfqpoint{3.133948in}{1.422949in}}%
\pgfpathlineto{\pgfqpoint{3.147878in}{1.414571in}}%
\pgfpathlineto{\pgfqpoint{3.156446in}{1.417922in}}%
\pgfpathlineto{\pgfqpoint{3.165000in}{1.421547in}}%
\pgfpathlineto{\pgfqpoint{3.173541in}{1.425438in}}%
\pgfpathlineto{\pgfqpoint{3.182070in}{1.429590in}}%
\pgfpathlineto{\pgfqpoint{3.168172in}{1.437272in}}%
\pgfpathlineto{\pgfqpoint{3.154275in}{1.445157in}}%
\pgfpathlineto{\pgfqpoint{3.140379in}{1.453245in}}%
\pgfpathlineto{\pgfqpoint{3.126484in}{1.461539in}}%
\pgfpathlineto{\pgfqpoint{3.117923in}{1.458071in}}%
\pgfpathlineto{\pgfqpoint{3.109348in}{1.454871in}}%
\pgfpathlineto{\pgfqpoint{3.100760in}{1.451947in}}%
\pgfpathlineto{\pgfqpoint{3.092158in}{1.449305in}}%
\pgfpathclose%
\pgfusepath{fill}%
\end{pgfscope}%
\begin{pgfscope}%
\pgfpathrectangle{\pgfqpoint{1.150000in}{0.150000in}}{\pgfqpoint{5.700000in}{5.700000in}}%
\pgfusepath{clip}%
\pgfsetbuttcap%
\pgfsetroundjoin%
\definecolor{currentfill}{rgb}{0.165117,0.467423,0.558141}%
\pgfsetfillcolor{currentfill}%
\pgfsetfillopacity{0.800000}%
\pgfsetlinewidth{0.000000pt}%
\definecolor{currentstroke}{rgb}{0.000000,0.000000,0.000000}%
\pgfsetstrokecolor{currentstroke}%
\pgfsetdash{}{0pt}%
\pgfpathmoveto{\pgfqpoint{2.133470in}{2.613797in}}%
\pgfpathlineto{\pgfqpoint{2.147922in}{2.587388in}}%
\pgfpathlineto{\pgfqpoint{2.162357in}{2.561310in}}%
\pgfpathlineto{\pgfqpoint{2.176776in}{2.535560in}}%
\pgfpathlineto{\pgfqpoint{2.191177in}{2.510134in}}%
\pgfpathlineto{\pgfqpoint{2.200628in}{2.500975in}}%
\pgfpathlineto{\pgfqpoint{2.210049in}{2.492248in}}%
\pgfpathlineto{\pgfqpoint{2.219444in}{2.483947in}}%
\pgfpathlineto{\pgfqpoint{2.228811in}{2.476063in}}%
\pgfpathlineto{\pgfqpoint{2.214480in}{2.500721in}}%
\pgfpathlineto{\pgfqpoint{2.200133in}{2.525702in}}%
\pgfpathlineto{\pgfqpoint{2.185771in}{2.551008in}}%
\pgfpathlineto{\pgfqpoint{2.171391in}{2.576643in}}%
\pgfpathlineto{\pgfqpoint{2.161954in}{2.585281in}}%
\pgfpathlineto{\pgfqpoint{2.152488in}{2.594348in}}%
\pgfpathlineto{\pgfqpoint{2.142994in}{2.603851in}}%
\pgfpathlineto{\pgfqpoint{2.133470in}{2.613797in}}%
\pgfpathclose%
\pgfusepath{fill}%
\end{pgfscope}%
\begin{pgfscope}%
\pgfpathrectangle{\pgfqpoint{1.150000in}{0.150000in}}{\pgfqpoint{5.700000in}{5.700000in}}%
\pgfusepath{clip}%
\pgfsetbuttcap%
\pgfsetroundjoin%
\definecolor{currentfill}{rgb}{0.257322,0.256130,0.526563}%
\pgfsetfillcolor{currentfill}%
\pgfsetfillopacity{0.800000}%
\pgfsetlinewidth{0.000000pt}%
\definecolor{currentstroke}{rgb}{0.000000,0.000000,0.000000}%
\pgfsetstrokecolor{currentstroke}%
\pgfsetdash{}{0pt}%
\pgfpathmoveto{\pgfqpoint{4.213256in}{1.903078in}}%
\pgfpathlineto{\pgfqpoint{4.227375in}{1.910203in}}%
\pgfpathlineto{\pgfqpoint{4.241507in}{1.917511in}}%
\pgfpathlineto{\pgfqpoint{4.255653in}{1.925002in}}%
\pgfpathlineto{\pgfqpoint{4.269812in}{1.932676in}}%
\pgfpathlineto{\pgfqpoint{4.277887in}{1.947053in}}%
\pgfpathlineto{\pgfqpoint{4.285957in}{1.961364in}}%
\pgfpathlineto{\pgfqpoint{4.294022in}{1.975608in}}%
\pgfpathlineto{\pgfqpoint{4.302084in}{1.989782in}}%
\pgfpathlineto{\pgfqpoint{4.287923in}{1.981789in}}%
\pgfpathlineto{\pgfqpoint{4.273777in}{1.973979in}}%
\pgfpathlineto{\pgfqpoint{4.259644in}{1.966353in}}%
\pgfpathlineto{\pgfqpoint{4.245525in}{1.958911in}}%
\pgfpathlineto{\pgfqpoint{4.237464in}{1.945044in}}%
\pgfpathlineto{\pgfqpoint{4.229399in}{1.931114in}}%
\pgfpathlineto{\pgfqpoint{4.221330in}{1.917124in}}%
\pgfpathlineto{\pgfqpoint{4.213256in}{1.903078in}}%
\pgfpathclose%
\pgfusepath{fill}%
\end{pgfscope}%
\begin{pgfscope}%
\pgfpathrectangle{\pgfqpoint{1.150000in}{0.150000in}}{\pgfqpoint{5.700000in}{5.700000in}}%
\pgfusepath{clip}%
\pgfsetbuttcap%
\pgfsetroundjoin%
\definecolor{currentfill}{rgb}{0.277941,0.056324,0.381191}%
\pgfsetfillcolor{currentfill}%
\pgfsetfillopacity{0.800000}%
\pgfsetlinewidth{0.000000pt}%
\definecolor{currentstroke}{rgb}{0.000000,0.000000,0.000000}%
\pgfsetstrokecolor{currentstroke}%
\pgfsetdash{}{0pt}%
\pgfpathmoveto{\pgfqpoint{3.704926in}{1.458921in}}%
\pgfpathlineto{\pgfqpoint{3.718868in}{1.459273in}}%
\pgfpathlineto{\pgfqpoint{3.732818in}{1.459811in}}%
\pgfpathlineto{\pgfqpoint{3.746775in}{1.460533in}}%
\pgfpathlineto{\pgfqpoint{3.760741in}{1.461440in}}%
\pgfpathlineto{\pgfqpoint{3.768964in}{1.473394in}}%
\pgfpathlineto{\pgfqpoint{3.777181in}{1.485435in}}%
\pgfpathlineto{\pgfqpoint{3.785392in}{1.497558in}}%
\pgfpathlineto{\pgfqpoint{3.793598in}{1.509759in}}%
\pgfpathlineto{\pgfqpoint{3.779642in}{1.508317in}}%
\pgfpathlineto{\pgfqpoint{3.765694in}{1.507059in}}%
\pgfpathlineto{\pgfqpoint{3.751754in}{1.505987in}}%
\pgfpathlineto{\pgfqpoint{3.737823in}{1.505100in}}%
\pgfpathlineto{\pgfqpoint{3.729607in}{1.493423in}}%
\pgfpathlineto{\pgfqpoint{3.721386in}{1.481830in}}%
\pgfpathlineto{\pgfqpoint{3.713159in}{1.470328in}}%
\pgfpathlineto{\pgfqpoint{3.704926in}{1.458921in}}%
\pgfpathclose%
\pgfusepath{fill}%
\end{pgfscope}%
\begin{pgfscope}%
\pgfpathrectangle{\pgfqpoint{1.150000in}{0.150000in}}{\pgfqpoint{5.700000in}{5.700000in}}%
\pgfusepath{clip}%
\pgfsetbuttcap%
\pgfsetroundjoin%
\definecolor{currentfill}{rgb}{0.128729,0.563265,0.551229}%
\pgfsetfillcolor{currentfill}%
\pgfsetfillopacity{0.800000}%
\pgfsetlinewidth{0.000000pt}%
\definecolor{currentstroke}{rgb}{0.000000,0.000000,0.000000}%
\pgfsetstrokecolor{currentstroke}%
\pgfsetdash{}{0pt}%
\pgfpathmoveto{\pgfqpoint{4.996093in}{2.795759in}}%
\pgfpathlineto{\pgfqpoint{5.010675in}{2.809694in}}%
\pgfpathlineto{\pgfqpoint{5.025276in}{2.823815in}}%
\pgfpathlineto{\pgfqpoint{5.039897in}{2.838123in}}%
\pgfpathlineto{\pgfqpoint{5.054538in}{2.852617in}}%
\pgfpathlineto{\pgfqpoint{5.062315in}{2.861648in}}%
\pgfpathlineto{\pgfqpoint{5.070083in}{2.870514in}}%
\pgfpathlineto{\pgfqpoint{5.077842in}{2.879217in}}%
\pgfpathlineto{\pgfqpoint{5.085593in}{2.887757in}}%
\pgfpathlineto{\pgfqpoint{5.070957in}{2.873349in}}%
\pgfpathlineto{\pgfqpoint{5.056340in}{2.859127in}}%
\pgfpathlineto{\pgfqpoint{5.041744in}{2.845092in}}%
\pgfpathlineto{\pgfqpoint{5.027167in}{2.831242in}}%
\pgfpathlineto{\pgfqpoint{5.019411in}{2.822603in}}%
\pgfpathlineto{\pgfqpoint{5.011647in}{2.813811in}}%
\pgfpathlineto{\pgfqpoint{5.003874in}{2.804863in}}%
\pgfpathlineto{\pgfqpoint{4.996093in}{2.795759in}}%
\pgfpathclose%
\pgfusepath{fill}%
\end{pgfscope}%
\begin{pgfscope}%
\pgfpathrectangle{\pgfqpoint{1.150000in}{0.150000in}}{\pgfqpoint{5.700000in}{5.700000in}}%
\pgfusepath{clip}%
\pgfsetbuttcap%
\pgfsetroundjoin%
\definecolor{currentfill}{rgb}{0.352360,0.783011,0.392636}%
\pgfsetfillcolor{currentfill}%
\pgfsetfillopacity{0.800000}%
\pgfsetlinewidth{0.000000pt}%
\definecolor{currentstroke}{rgb}{0.000000,0.000000,0.000000}%
\pgfsetstrokecolor{currentstroke}%
\pgfsetdash{}{0pt}%
\pgfpathmoveto{\pgfqpoint{5.803035in}{3.523799in}}%
\pgfpathlineto{\pgfqpoint{5.818163in}{3.540784in}}%
\pgfpathlineto{\pgfqpoint{5.833316in}{3.557956in}}%
\pgfpathlineto{\pgfqpoint{5.848493in}{3.575315in}}%
\pgfpathlineto{\pgfqpoint{5.863694in}{3.592861in}}%
\pgfpathlineto{\pgfqpoint{5.870907in}{3.593355in}}%
\pgfpathlineto{\pgfqpoint{5.878109in}{3.593757in}}%
\pgfpathlineto{\pgfqpoint{5.885301in}{3.594074in}}%
\pgfpathlineto{\pgfqpoint{5.892483in}{3.594311in}}%
\pgfpathlineto{\pgfqpoint{5.877310in}{3.577250in}}%
\pgfpathlineto{\pgfqpoint{5.862161in}{3.560375in}}%
\pgfpathlineto{\pgfqpoint{5.847037in}{3.543686in}}%
\pgfpathlineto{\pgfqpoint{5.831936in}{3.527183in}}%
\pgfpathlineto{\pgfqpoint{5.824725in}{3.526451in}}%
\pgfpathlineto{\pgfqpoint{5.817505in}{3.525646in}}%
\pgfpathlineto{\pgfqpoint{5.810275in}{3.524764in}}%
\pgfpathlineto{\pgfqpoint{5.803035in}{3.523799in}}%
\pgfpathclose%
\pgfusepath{fill}%
\end{pgfscope}%
\begin{pgfscope}%
\pgfpathrectangle{\pgfqpoint{1.150000in}{0.150000in}}{\pgfqpoint{5.700000in}{5.700000in}}%
\pgfusepath{clip}%
\pgfsetbuttcap%
\pgfsetroundjoin%
\definecolor{currentfill}{rgb}{0.273809,0.031497,0.358853}%
\pgfsetfillcolor{currentfill}%
\pgfsetfillopacity{0.800000}%
\pgfsetlinewidth{0.000000pt}%
\definecolor{currentstroke}{rgb}{0.000000,0.000000,0.000000}%
\pgfsetstrokecolor{currentstroke}%
\pgfsetdash{}{0pt}%
\pgfpathmoveto{\pgfqpoint{3.616185in}{1.417061in}}%
\pgfpathlineto{\pgfqpoint{3.630111in}{1.416104in}}%
\pgfpathlineto{\pgfqpoint{3.644044in}{1.415333in}}%
\pgfpathlineto{\pgfqpoint{3.657984in}{1.414749in}}%
\pgfpathlineto{\pgfqpoint{3.671931in}{1.414350in}}%
\pgfpathlineto{\pgfqpoint{3.680189in}{1.425324in}}%
\pgfpathlineto{\pgfqpoint{3.688441in}{1.436413in}}%
\pgfpathlineto{\pgfqpoint{3.696686in}{1.447614in}}%
\pgfpathlineto{\pgfqpoint{3.704926in}{1.458921in}}%
\pgfpathlineto{\pgfqpoint{3.690991in}{1.458754in}}%
\pgfpathlineto{\pgfqpoint{3.677064in}{1.458772in}}%
\pgfpathlineto{\pgfqpoint{3.663144in}{1.458977in}}%
\pgfpathlineto{\pgfqpoint{3.649231in}{1.459369in}}%
\pgfpathlineto{\pgfqpoint{3.640979in}{1.448616in}}%
\pgfpathlineto{\pgfqpoint{3.632721in}{1.437977in}}%
\pgfpathlineto{\pgfqpoint{3.624456in}{1.427456in}}%
\pgfpathlineto{\pgfqpoint{3.616185in}{1.417061in}}%
\pgfpathclose%
\pgfusepath{fill}%
\end{pgfscope}%
\begin{pgfscope}%
\pgfpathrectangle{\pgfqpoint{1.150000in}{0.150000in}}{\pgfqpoint{5.700000in}{5.700000in}}%
\pgfusepath{clip}%
\pgfsetbuttcap%
\pgfsetroundjoin%
\definecolor{currentfill}{rgb}{0.174274,0.445044,0.557792}%
\pgfsetfillcolor{currentfill}%
\pgfsetfillopacity{0.800000}%
\pgfsetlinewidth{0.000000pt}%
\definecolor{currentstroke}{rgb}{0.000000,0.000000,0.000000}%
\pgfsetstrokecolor{currentstroke}%
\pgfsetdash{}{0pt}%
\pgfpathmoveto{\pgfqpoint{4.665221in}{2.428059in}}%
\pgfpathlineto{\pgfqpoint{4.679590in}{2.439673in}}%
\pgfpathlineto{\pgfqpoint{4.693977in}{2.451472in}}%
\pgfpathlineto{\pgfqpoint{4.708381in}{2.463456in}}%
\pgfpathlineto{\pgfqpoint{4.722803in}{2.475625in}}%
\pgfpathlineto{\pgfqpoint{4.730734in}{2.487787in}}%
\pgfpathlineto{\pgfqpoint{4.738659in}{2.499803in}}%
\pgfpathlineto{\pgfqpoint{4.746578in}{2.511672in}}%
\pgfpathlineto{\pgfqpoint{4.754490in}{2.523394in}}%
\pgfpathlineto{\pgfqpoint{4.740068in}{2.511137in}}%
\pgfpathlineto{\pgfqpoint{4.725664in}{2.499066in}}%
\pgfpathlineto{\pgfqpoint{4.711277in}{2.487179in}}%
\pgfpathlineto{\pgfqpoint{4.696908in}{2.475478in}}%
\pgfpathlineto{\pgfqpoint{4.688996in}{2.463830in}}%
\pgfpathlineto{\pgfqpoint{4.681077in}{2.452044in}}%
\pgfpathlineto{\pgfqpoint{4.673152in}{2.440120in}}%
\pgfpathlineto{\pgfqpoint{4.665221in}{2.428059in}}%
\pgfpathclose%
\pgfusepath{fill}%
\end{pgfscope}%
\begin{pgfscope}%
\pgfpathrectangle{\pgfqpoint{1.150000in}{0.150000in}}{\pgfqpoint{5.700000in}{5.700000in}}%
\pgfusepath{clip}%
\pgfsetbuttcap%
\pgfsetroundjoin%
\definecolor{currentfill}{rgb}{0.122312,0.633153,0.530398}%
\pgfsetfillcolor{currentfill}%
\pgfsetfillopacity{0.800000}%
\pgfsetlinewidth{0.000000pt}%
\definecolor{currentstroke}{rgb}{0.000000,0.000000,0.000000}%
\pgfsetstrokecolor{currentstroke}%
\pgfsetdash{}{0pt}%
\pgfpathmoveto{\pgfqpoint{5.205974in}{3.008818in}}%
\pgfpathlineto{\pgfqpoint{5.220698in}{3.023880in}}%
\pgfpathlineto{\pgfqpoint{5.235443in}{3.039128in}}%
\pgfpathlineto{\pgfqpoint{5.250210in}{3.054564in}}%
\pgfpathlineto{\pgfqpoint{5.264997in}{3.070186in}}%
\pgfpathlineto{\pgfqpoint{5.272651in}{3.076985in}}%
\pgfpathlineto{\pgfqpoint{5.280295in}{3.083623in}}%
\pgfpathlineto{\pgfqpoint{5.287930in}{3.090103in}}%
\pgfpathlineto{\pgfqpoint{5.295555in}{3.096426in}}%
\pgfpathlineto{\pgfqpoint{5.280777in}{3.080997in}}%
\pgfpathlineto{\pgfqpoint{5.266021in}{3.065755in}}%
\pgfpathlineto{\pgfqpoint{5.251285in}{3.050699in}}%
\pgfpathlineto{\pgfqpoint{5.236571in}{3.035830in}}%
\pgfpathlineto{\pgfqpoint{5.228936in}{3.029300in}}%
\pgfpathlineto{\pgfqpoint{5.221291in}{3.022623in}}%
\pgfpathlineto{\pgfqpoint{5.213637in}{3.015797in}}%
\pgfpathlineto{\pgfqpoint{5.205974in}{3.008818in}}%
\pgfpathclose%
\pgfusepath{fill}%
\end{pgfscope}%
\begin{pgfscope}%
\pgfpathrectangle{\pgfqpoint{1.150000in}{0.150000in}}{\pgfqpoint{5.700000in}{5.700000in}}%
\pgfusepath{clip}%
\pgfsetbuttcap%
\pgfsetroundjoin%
\definecolor{currentfill}{rgb}{0.395174,0.797475,0.367757}%
\pgfsetfillcolor{currentfill}%
\pgfsetfillopacity{0.800000}%
\pgfsetlinewidth{0.000000pt}%
\definecolor{currentstroke}{rgb}{0.000000,0.000000,0.000000}%
\pgfsetstrokecolor{currentstroke}%
\pgfsetdash{}{0pt}%
\pgfpathmoveto{\pgfqpoint{5.892483in}{3.594311in}}%
\pgfpathlineto{\pgfqpoint{5.907680in}{3.611558in}}%
\pgfpathlineto{\pgfqpoint{5.922902in}{3.628992in}}%
\pgfpathlineto{\pgfqpoint{5.938149in}{3.646613in}}%
\pgfpathlineto{\pgfqpoint{5.945298in}{3.646393in}}%
\pgfpathlineto{\pgfqpoint{5.952438in}{3.646097in}}%
\pgfpathlineto{\pgfqpoint{5.959567in}{3.645732in}}%
\pgfpathlineto{\pgfqpoint{5.966686in}{3.645301in}}%
\pgfpathlineto{\pgfqpoint{5.951471in}{3.628202in}}%
\pgfpathlineto{\pgfqpoint{5.936280in}{3.611288in}}%
\pgfpathlineto{\pgfqpoint{5.921113in}{3.594560in}}%
\pgfpathlineto{\pgfqpoint{5.913970in}{3.594591in}}%
\pgfpathlineto{\pgfqpoint{5.906817in}{3.594563in}}%
\pgfpathlineto{\pgfqpoint{5.899655in}{3.594472in}}%
\pgfpathlineto{\pgfqpoint{5.892483in}{3.594311in}}%
\pgfpathclose%
\pgfusepath{fill}%
\end{pgfscope}%
\begin{pgfscope}%
\pgfpathrectangle{\pgfqpoint{1.150000in}{0.150000in}}{\pgfqpoint{5.700000in}{5.700000in}}%
\pgfusepath{clip}%
\pgfsetbuttcap%
\pgfsetroundjoin%
\definecolor{currentfill}{rgb}{0.281446,0.084320,0.407414}%
\pgfsetfillcolor{currentfill}%
\pgfsetfillopacity{0.800000}%
\pgfsetlinewidth{0.000000pt}%
\definecolor{currentstroke}{rgb}{0.000000,0.000000,0.000000}%
\pgfsetstrokecolor{currentstroke}%
\pgfsetdash{}{0pt}%
\pgfpathmoveto{\pgfqpoint{3.793598in}{1.509759in}}%
\pgfpathlineto{\pgfqpoint{3.807563in}{1.511385in}}%
\pgfpathlineto{\pgfqpoint{3.821536in}{1.513196in}}%
\pgfpathlineto{\pgfqpoint{3.835519in}{1.515190in}}%
\pgfpathlineto{\pgfqpoint{3.849510in}{1.517368in}}%
\pgfpathlineto{\pgfqpoint{3.857703in}{1.530157in}}%
\pgfpathlineto{\pgfqpoint{3.865891in}{1.543005in}}%
\pgfpathlineto{\pgfqpoint{3.874073in}{1.555909in}}%
\pgfpathlineto{\pgfqpoint{3.882251in}{1.568862in}}%
\pgfpathlineto{\pgfqpoint{3.868266in}{1.566179in}}%
\pgfpathlineto{\pgfqpoint{3.854291in}{1.563680in}}%
\pgfpathlineto{\pgfqpoint{3.840324in}{1.561366in}}%
\pgfpathlineto{\pgfqpoint{3.826367in}{1.559235in}}%
\pgfpathlineto{\pgfqpoint{3.818183in}{1.546775in}}%
\pgfpathlineto{\pgfqpoint{3.809993in}{1.534372in}}%
\pgfpathlineto{\pgfqpoint{3.801798in}{1.522032in}}%
\pgfpathlineto{\pgfqpoint{3.793598in}{1.509759in}}%
\pgfpathclose%
\pgfusepath{fill}%
\end{pgfscope}%
\begin{pgfscope}%
\pgfpathrectangle{\pgfqpoint{1.150000in}{0.150000in}}{\pgfqpoint{5.700000in}{5.700000in}}%
\pgfusepath{clip}%
\pgfsetbuttcap%
\pgfsetroundjoin%
\definecolor{currentfill}{rgb}{0.283091,0.110553,0.431554}%
\pgfsetfillcolor{currentfill}%
\pgfsetfillopacity{0.800000}%
\pgfsetlinewidth{0.000000pt}%
\definecolor{currentstroke}{rgb}{0.000000,0.000000,0.000000}%
\pgfsetstrokecolor{currentstroke}%
\pgfsetdash{}{0pt}%
\pgfpathmoveto{\pgfqpoint{2.833916in}{1.626033in}}%
\pgfpathlineto{\pgfqpoint{2.847917in}{1.612881in}}%
\pgfpathlineto{\pgfqpoint{2.861913in}{1.599953in}}%
\pgfpathlineto{\pgfqpoint{2.875907in}{1.587247in}}%
\pgfpathlineto{\pgfqpoint{2.889897in}{1.574761in}}%
\pgfpathlineto{\pgfqpoint{2.898683in}{1.573955in}}%
\pgfpathlineto{\pgfqpoint{2.907451in}{1.573488in}}%
\pgfpathlineto{\pgfqpoint{2.916202in}{1.573351in}}%
\pgfpathlineto{\pgfqpoint{2.924937in}{1.573538in}}%
\pgfpathlineto{\pgfqpoint{2.910991in}{1.585285in}}%
\pgfpathlineto{\pgfqpoint{2.897042in}{1.597252in}}%
\pgfpathlineto{\pgfqpoint{2.883090in}{1.609440in}}%
\pgfpathlineto{\pgfqpoint{2.869135in}{1.621850in}}%
\pgfpathlineto{\pgfqpoint{2.860357in}{1.622390in}}%
\pgfpathlineto{\pgfqpoint{2.851562in}{1.623262in}}%
\pgfpathlineto{\pgfqpoint{2.842748in}{1.624473in}}%
\pgfpathlineto{\pgfqpoint{2.833916in}{1.626033in}}%
\pgfpathclose%
\pgfusepath{fill}%
\end{pgfscope}%
\begin{pgfscope}%
\pgfpathrectangle{\pgfqpoint{1.150000in}{0.150000in}}{\pgfqpoint{5.700000in}{5.700000in}}%
\pgfusepath{clip}%
\pgfsetbuttcap%
\pgfsetroundjoin%
\definecolor{currentfill}{rgb}{0.229739,0.322361,0.545706}%
\pgfsetfillcolor{currentfill}%
\pgfsetfillopacity{0.800000}%
\pgfsetlinewidth{0.000000pt}%
\definecolor{currentstroke}{rgb}{0.000000,0.000000,0.000000}%
\pgfsetstrokecolor{currentstroke}%
\pgfsetdash{}{0pt}%
\pgfpathmoveto{\pgfqpoint{2.382664in}{2.173550in}}%
\pgfpathlineto{\pgfqpoint{2.396901in}{2.152434in}}%
\pgfpathlineto{\pgfqpoint{2.411127in}{2.131597in}}%
\pgfpathlineto{\pgfqpoint{2.425341in}{2.111035in}}%
\pgfpathlineto{\pgfqpoint{2.439545in}{2.090748in}}%
\pgfpathlineto{\pgfqpoint{2.448763in}{2.083800in}}%
\pgfpathlineto{\pgfqpoint{2.457957in}{2.077267in}}%
\pgfpathlineto{\pgfqpoint{2.467126in}{2.071139in}}%
\pgfpathlineto{\pgfqpoint{2.476271in}{2.065410in}}%
\pgfpathlineto{\pgfqpoint{2.462130in}{2.084924in}}%
\pgfpathlineto{\pgfqpoint{2.447978in}{2.104710in}}%
\pgfpathlineto{\pgfqpoint{2.433816in}{2.124771in}}%
\pgfpathlineto{\pgfqpoint{2.419643in}{2.145109in}}%
\pgfpathlineto{\pgfqpoint{2.410437in}{2.151599in}}%
\pgfpathlineto{\pgfqpoint{2.401205in}{2.158497in}}%
\pgfpathlineto{\pgfqpoint{2.391948in}{2.165811in}}%
\pgfpathlineto{\pgfqpoint{2.382664in}{2.173550in}}%
\pgfpathclose%
\pgfusepath{fill}%
\end{pgfscope}%
\begin{pgfscope}%
\pgfpathrectangle{\pgfqpoint{1.150000in}{0.150000in}}{\pgfqpoint{5.700000in}{5.700000in}}%
\pgfusepath{clip}%
\pgfsetbuttcap%
\pgfsetroundjoin%
\definecolor{currentfill}{rgb}{0.269944,0.014625,0.341379}%
\pgfsetfillcolor{currentfill}%
\pgfsetfillopacity{0.800000}%
\pgfsetlinewidth{0.000000pt}%
\definecolor{currentstroke}{rgb}{0.000000,0.000000,0.000000}%
\pgfsetstrokecolor{currentstroke}%
\pgfsetdash{}{0pt}%
\pgfpathmoveto{\pgfqpoint{3.527321in}{1.384921in}}%
\pgfpathlineto{\pgfqpoint{3.541239in}{1.382618in}}%
\pgfpathlineto{\pgfqpoint{3.555163in}{1.380503in}}%
\pgfpathlineto{\pgfqpoint{3.569092in}{1.378575in}}%
\pgfpathlineto{\pgfqpoint{3.583027in}{1.376835in}}%
\pgfpathlineto{\pgfqpoint{3.591328in}{1.386676in}}%
\pgfpathlineto{\pgfqpoint{3.599620in}{1.396665in}}%
\pgfpathlineto{\pgfqpoint{3.607906in}{1.406795in}}%
\pgfpathlineto{\pgfqpoint{3.616185in}{1.417061in}}%
\pgfpathlineto{\pgfqpoint{3.602265in}{1.418204in}}%
\pgfpathlineto{\pgfqpoint{3.588351in}{1.419535in}}%
\pgfpathlineto{\pgfqpoint{3.574444in}{1.421054in}}%
\pgfpathlineto{\pgfqpoint{3.560543in}{1.422761in}}%
\pgfpathlineto{\pgfqpoint{3.552248in}{1.413080in}}%
\pgfpathlineto{\pgfqpoint{3.543947in}{1.403542in}}%
\pgfpathlineto{\pgfqpoint{3.535638in}{1.394154in}}%
\pgfpathlineto{\pgfqpoint{3.527321in}{1.384921in}}%
\pgfpathclose%
\pgfusepath{fill}%
\end{pgfscope}%
\begin{pgfscope}%
\pgfpathrectangle{\pgfqpoint{1.150000in}{0.150000in}}{\pgfqpoint{5.700000in}{5.700000in}}%
\pgfusepath{clip}%
\pgfsetbuttcap%
\pgfsetroundjoin%
\definecolor{currentfill}{rgb}{0.194100,0.399323,0.555565}%
\pgfsetfillcolor{currentfill}%
\pgfsetfillopacity{0.800000}%
\pgfsetlinewidth{0.000000pt}%
\definecolor{currentstroke}{rgb}{0.000000,0.000000,0.000000}%
\pgfsetstrokecolor{currentstroke}%
\pgfsetdash{}{0pt}%
\pgfpathmoveto{\pgfqpoint{4.544249in}{2.283191in}}%
\pgfpathlineto{\pgfqpoint{4.558550in}{2.293789in}}%
\pgfpathlineto{\pgfqpoint{4.572868in}{2.304572in}}%
\pgfpathlineto{\pgfqpoint{4.587201in}{2.315539in}}%
\pgfpathlineto{\pgfqpoint{4.601552in}{2.326691in}}%
\pgfpathlineto{\pgfqpoint{4.609531in}{2.339828in}}%
\pgfpathlineto{\pgfqpoint{4.617505in}{2.352833in}}%
\pgfpathlineto{\pgfqpoint{4.625473in}{2.365707in}}%
\pgfpathlineto{\pgfqpoint{4.633434in}{2.378447in}}%
\pgfpathlineto{\pgfqpoint{4.619083in}{2.367139in}}%
\pgfpathlineto{\pgfqpoint{4.604748in}{2.356017in}}%
\pgfpathlineto{\pgfqpoint{4.590429in}{2.345079in}}%
\pgfpathlineto{\pgfqpoint{4.576127in}{2.334326in}}%
\pgfpathlineto{\pgfqpoint{4.568166in}{2.321728in}}%
\pgfpathlineto{\pgfqpoint{4.560200in}{2.309005in}}%
\pgfpathlineto{\pgfqpoint{4.552227in}{2.296159in}}%
\pgfpathlineto{\pgfqpoint{4.544249in}{2.283191in}}%
\pgfpathclose%
\pgfusepath{fill}%
\end{pgfscope}%
\begin{pgfscope}%
\pgfpathrectangle{\pgfqpoint{1.150000in}{0.150000in}}{\pgfqpoint{5.700000in}{5.700000in}}%
\pgfusepath{clip}%
\pgfsetbuttcap%
\pgfsetroundjoin%
\definecolor{currentfill}{rgb}{0.267004,0.004874,0.329415}%
\pgfsetfillcolor{currentfill}%
\pgfsetfillopacity{0.800000}%
\pgfsetlinewidth{0.000000pt}%
\definecolor{currentstroke}{rgb}{0.000000,0.000000,0.000000}%
\pgfsetstrokecolor{currentstroke}%
\pgfsetdash{}{0pt}%
\pgfpathmoveto{\pgfqpoint{3.293305in}{1.375341in}}%
\pgfpathlineto{\pgfqpoint{3.307219in}{1.369449in}}%
\pgfpathlineto{\pgfqpoint{3.321136in}{1.363753in}}%
\pgfpathlineto{\pgfqpoint{3.335055in}{1.358250in}}%
\pgfpathlineto{\pgfqpoint{3.348978in}{1.352942in}}%
\pgfpathlineto{\pgfqpoint{3.357410in}{1.359367in}}%
\pgfpathlineto{\pgfqpoint{3.365831in}{1.366014in}}%
\pgfpathlineto{\pgfqpoint{3.374243in}{1.372875in}}%
\pgfpathlineto{\pgfqpoint{3.382645in}{1.379945in}}%
\pgfpathlineto{\pgfqpoint{3.368747in}{1.384593in}}%
\pgfpathlineto{\pgfqpoint{3.354853in}{1.389435in}}%
\pgfpathlineto{\pgfqpoint{3.340962in}{1.394471in}}%
\pgfpathlineto{\pgfqpoint{3.327074in}{1.399702in}}%
\pgfpathlineto{\pgfqpoint{3.318647in}{1.393281in}}%
\pgfpathlineto{\pgfqpoint{3.310211in}{1.387076in}}%
\pgfpathlineto{\pgfqpoint{3.301763in}{1.381094in}}%
\pgfpathlineto{\pgfqpoint{3.293305in}{1.375341in}}%
\pgfpathclose%
\pgfusepath{fill}%
\end{pgfscope}%
\begin{pgfscope}%
\pgfpathrectangle{\pgfqpoint{1.150000in}{0.150000in}}{\pgfqpoint{5.700000in}{5.700000in}}%
\pgfusepath{clip}%
\pgfsetbuttcap%
\pgfsetroundjoin%
\definecolor{currentfill}{rgb}{0.274128,0.199721,0.498911}%
\pgfsetfillcolor{currentfill}%
\pgfsetfillopacity{0.800000}%
\pgfsetlinewidth{0.000000pt}%
\definecolor{currentstroke}{rgb}{0.000000,0.000000,0.000000}%
\pgfsetstrokecolor{currentstroke}%
\pgfsetdash{}{0pt}%
\pgfpathmoveto{\pgfqpoint{4.092151in}{1.765287in}}%
\pgfpathlineto{\pgfqpoint{4.106222in}{1.770948in}}%
\pgfpathlineto{\pgfqpoint{4.120304in}{1.776792in}}%
\pgfpathlineto{\pgfqpoint{4.134398in}{1.782819in}}%
\pgfpathlineto{\pgfqpoint{4.148505in}{1.789029in}}%
\pgfpathlineto{\pgfqpoint{4.156614in}{1.803418in}}%
\pgfpathlineto{\pgfqpoint{4.164719in}{1.817776in}}%
\pgfpathlineto{\pgfqpoint{4.172819in}{1.832099in}}%
\pgfpathlineto{\pgfqpoint{4.180915in}{1.846384in}}%
\pgfpathlineto{\pgfqpoint{4.166809in}{1.839792in}}%
\pgfpathlineto{\pgfqpoint{4.152715in}{1.833384in}}%
\pgfpathlineto{\pgfqpoint{4.138634in}{1.827158in}}%
\pgfpathlineto{\pgfqpoint{4.124565in}{1.821116in}}%
\pgfpathlineto{\pgfqpoint{4.116468in}{1.807201in}}%
\pgfpathlineto{\pgfqpoint{4.108367in}{1.793255in}}%
\pgfpathlineto{\pgfqpoint{4.100261in}{1.779282in}}%
\pgfpathlineto{\pgfqpoint{4.092151in}{1.765287in}}%
\pgfpathclose%
\pgfusepath{fill}%
\end{pgfscope}%
\begin{pgfscope}%
\pgfpathrectangle{\pgfqpoint{1.150000in}{0.150000in}}{\pgfqpoint{5.700000in}{5.700000in}}%
\pgfusepath{clip}%
\pgfsetbuttcap%
\pgfsetroundjoin%
\definecolor{currentfill}{rgb}{0.208030,0.718701,0.472873}%
\pgfsetfillcolor{currentfill}%
\pgfsetfillopacity{0.800000}%
\pgfsetlinewidth{0.000000pt}%
\definecolor{currentstroke}{rgb}{0.000000,0.000000,0.000000}%
\pgfsetstrokecolor{currentstroke}%
\pgfsetdash{}{0pt}%
\pgfpathmoveto{\pgfqpoint{5.504988in}{3.284973in}}%
\pgfpathlineto{\pgfqpoint{5.519923in}{3.301287in}}%
\pgfpathlineto{\pgfqpoint{5.534881in}{3.317789in}}%
\pgfpathlineto{\pgfqpoint{5.549861in}{3.334478in}}%
\pgfpathlineto{\pgfqpoint{5.564865in}{3.351355in}}%
\pgfpathlineto{\pgfqpoint{5.572316in}{3.354948in}}%
\pgfpathlineto{\pgfqpoint{5.579756in}{3.358403in}}%
\pgfpathlineto{\pgfqpoint{5.587186in}{3.361724in}}%
\pgfpathlineto{\pgfqpoint{5.594605in}{3.364915in}}%
\pgfpathlineto{\pgfqpoint{5.579620in}{3.348378in}}%
\pgfpathlineto{\pgfqpoint{5.564658in}{3.332028in}}%
\pgfpathlineto{\pgfqpoint{5.549718in}{3.315864in}}%
\pgfpathlineto{\pgfqpoint{5.534801in}{3.299886in}}%
\pgfpathlineto{\pgfqpoint{5.527363in}{3.296344in}}%
\pgfpathlineto{\pgfqpoint{5.519915in}{3.292680in}}%
\pgfpathlineto{\pgfqpoint{5.512457in}{3.288891in}}%
\pgfpathlineto{\pgfqpoint{5.504988in}{3.284973in}}%
\pgfpathclose%
\pgfusepath{fill}%
\end{pgfscope}%
\begin{pgfscope}%
\pgfpathrectangle{\pgfqpoint{1.150000in}{0.150000in}}{\pgfqpoint{5.700000in}{5.700000in}}%
\pgfusepath{clip}%
\pgfsetbuttcap%
\pgfsetroundjoin%
\definecolor{currentfill}{rgb}{0.283197,0.115680,0.436115}%
\pgfsetfillcolor{currentfill}%
\pgfsetfillopacity{0.800000}%
\pgfsetlinewidth{0.000000pt}%
\definecolor{currentstroke}{rgb}{0.000000,0.000000,0.000000}%
\pgfsetstrokecolor{currentstroke}%
\pgfsetdash{}{0pt}%
\pgfpathmoveto{\pgfqpoint{3.882251in}{1.568862in}}%
\pgfpathlineto{\pgfqpoint{3.896245in}{1.571728in}}%
\pgfpathlineto{\pgfqpoint{3.910249in}{1.574778in}}%
\pgfpathlineto{\pgfqpoint{3.924263in}{1.578010in}}%
\pgfpathlineto{\pgfqpoint{3.938287in}{1.581425in}}%
\pgfpathlineto{\pgfqpoint{3.946454in}{1.594910in}}%
\pgfpathlineto{\pgfqpoint{3.954616in}{1.608428in}}%
\pgfpathlineto{\pgfqpoint{3.962774in}{1.621974in}}%
\pgfpathlineto{\pgfqpoint{3.970927in}{1.635544in}}%
\pgfpathlineto{\pgfqpoint{3.956908in}{1.631654in}}%
\pgfpathlineto{\pgfqpoint{3.942899in}{1.627947in}}%
\pgfpathlineto{\pgfqpoint{3.928900in}{1.624424in}}%
\pgfpathlineto{\pgfqpoint{3.914911in}{1.621084in}}%
\pgfpathlineto{\pgfqpoint{3.906753in}{1.607976in}}%
\pgfpathlineto{\pgfqpoint{3.898591in}{1.594900in}}%
\pgfpathlineto{\pgfqpoint{3.890423in}{1.581860in}}%
\pgfpathlineto{\pgfqpoint{3.882251in}{1.568862in}}%
\pgfpathclose%
\pgfusepath{fill}%
\end{pgfscope}%
\begin{pgfscope}%
\pgfpathrectangle{\pgfqpoint{1.150000in}{0.150000in}}{\pgfqpoint{5.700000in}{5.700000in}}%
\pgfusepath{clip}%
\pgfsetbuttcap%
\pgfsetroundjoin%
\definecolor{currentfill}{rgb}{0.281924,0.089666,0.412415}%
\pgfsetfillcolor{currentfill}%
\pgfsetfillopacity{0.800000}%
\pgfsetlinewidth{0.000000pt}%
\definecolor{currentstroke}{rgb}{0.000000,0.000000,0.000000}%
\pgfsetstrokecolor{currentstroke}%
\pgfsetdash{}{0pt}%
\pgfpathmoveto{\pgfqpoint{2.889897in}{1.574761in}}%
\pgfpathlineto{\pgfqpoint{2.903885in}{1.562494in}}%
\pgfpathlineto{\pgfqpoint{2.917869in}{1.550445in}}%
\pgfpathlineto{\pgfqpoint{2.931851in}{1.538613in}}%
\pgfpathlineto{\pgfqpoint{2.945831in}{1.526998in}}%
\pgfpathlineto{\pgfqpoint{2.954573in}{1.526943in}}%
\pgfpathlineto{\pgfqpoint{2.963298in}{1.527217in}}%
\pgfpathlineto{\pgfqpoint{2.972007in}{1.527813in}}%
\pgfpathlineto{\pgfqpoint{2.980700in}{1.528724in}}%
\pgfpathlineto{\pgfqpoint{2.966762in}{1.539604in}}%
\pgfpathlineto{\pgfqpoint{2.952822in}{1.550699in}}%
\pgfpathlineto{\pgfqpoint{2.938881in}{1.562010in}}%
\pgfpathlineto{\pgfqpoint{2.924937in}{1.573538in}}%
\pgfpathlineto{\pgfqpoint{2.916202in}{1.573351in}}%
\pgfpathlineto{\pgfqpoint{2.907451in}{1.573488in}}%
\pgfpathlineto{\pgfqpoint{2.898683in}{1.573955in}}%
\pgfpathlineto{\pgfqpoint{2.889897in}{1.574761in}}%
\pgfpathclose%
\pgfusepath{fill}%
\end{pgfscope}%
\begin{pgfscope}%
\pgfpathrectangle{\pgfqpoint{1.150000in}{0.150000in}}{\pgfqpoint{5.700000in}{5.700000in}}%
\pgfusepath{clip}%
\pgfsetbuttcap%
\pgfsetroundjoin%
\definecolor{currentfill}{rgb}{0.141935,0.526453,0.555991}%
\pgfsetfillcolor{currentfill}%
\pgfsetfillopacity{0.800000}%
\pgfsetlinewidth{0.000000pt}%
\definecolor{currentstroke}{rgb}{0.000000,0.000000,0.000000}%
\pgfsetstrokecolor{currentstroke}%
\pgfsetdash{}{0pt}%
\pgfpathmoveto{\pgfqpoint{4.875397in}{2.663031in}}%
\pgfpathlineto{\pgfqpoint{4.889908in}{2.676288in}}%
\pgfpathlineto{\pgfqpoint{4.904438in}{2.689732in}}%
\pgfpathlineto{\pgfqpoint{4.918987in}{2.703361in}}%
\pgfpathlineto{\pgfqpoint{4.933556in}{2.717177in}}%
\pgfpathlineto{\pgfqpoint{4.941401in}{2.727567in}}%
\pgfpathlineto{\pgfqpoint{4.949238in}{2.737793in}}%
\pgfpathlineto{\pgfqpoint{4.957067in}{2.747857in}}%
\pgfpathlineto{\pgfqpoint{4.964888in}{2.757758in}}%
\pgfpathlineto{\pgfqpoint{4.950322in}{2.743958in}}%
\pgfpathlineto{\pgfqpoint{4.935776in}{2.730345in}}%
\pgfpathlineto{\pgfqpoint{4.921248in}{2.716917in}}%
\pgfpathlineto{\pgfqpoint{4.906739in}{2.703675in}}%
\pgfpathlineto{\pgfqpoint{4.898915in}{2.693745in}}%
\pgfpathlineto{\pgfqpoint{4.891083in}{2.683661in}}%
\pgfpathlineto{\pgfqpoint{4.883244in}{2.673423in}}%
\pgfpathlineto{\pgfqpoint{4.875397in}{2.663031in}}%
\pgfpathclose%
\pgfusepath{fill}%
\end{pgfscope}%
\begin{pgfscope}%
\pgfpathrectangle{\pgfqpoint{1.150000in}{0.150000in}}{\pgfqpoint{5.700000in}{5.700000in}}%
\pgfusepath{clip}%
\pgfsetbuttcap%
\pgfsetroundjoin%
\definecolor{currentfill}{rgb}{0.216210,0.351535,0.550627}%
\pgfsetfillcolor{currentfill}%
\pgfsetfillopacity{0.800000}%
\pgfsetlinewidth{0.000000pt}%
\definecolor{currentstroke}{rgb}{0.000000,0.000000,0.000000}%
\pgfsetstrokecolor{currentstroke}%
\pgfsetdash{}{0pt}%
\pgfpathmoveto{\pgfqpoint{4.423196in}{2.136294in}}%
\pgfpathlineto{\pgfqpoint{4.437431in}{2.145746in}}%
\pgfpathlineto{\pgfqpoint{4.451682in}{2.155381in}}%
\pgfpathlineto{\pgfqpoint{4.465948in}{2.165201in}}%
\pgfpathlineto{\pgfqpoint{4.480229in}{2.175204in}}%
\pgfpathlineto{\pgfqpoint{4.488250in}{2.189100in}}%
\pgfpathlineto{\pgfqpoint{4.496266in}{2.202886in}}%
\pgfpathlineto{\pgfqpoint{4.504277in}{2.216559in}}%
\pgfpathlineto{\pgfqpoint{4.512282in}{2.230120in}}%
\pgfpathlineto{\pgfqpoint{4.497999in}{2.219894in}}%
\pgfpathlineto{\pgfqpoint{4.483731in}{2.209853in}}%
\pgfpathlineto{\pgfqpoint{4.469479in}{2.199996in}}%
\pgfpathlineto{\pgfqpoint{4.455242in}{2.190323in}}%
\pgfpathlineto{\pgfqpoint{4.447238in}{2.176972in}}%
\pgfpathlineto{\pgfqpoint{4.439229in}{2.163516in}}%
\pgfpathlineto{\pgfqpoint{4.431215in}{2.149956in}}%
\pgfpathlineto{\pgfqpoint{4.423196in}{2.136294in}}%
\pgfpathclose%
\pgfusepath{fill}%
\end{pgfscope}%
\begin{pgfscope}%
\pgfpathrectangle{\pgfqpoint{1.150000in}{0.150000in}}{\pgfqpoint{5.700000in}{5.700000in}}%
\pgfusepath{clip}%
\pgfsetbuttcap%
\pgfsetroundjoin%
\definecolor{currentfill}{rgb}{0.214298,0.355619,0.551184}%
\pgfsetfillcolor{currentfill}%
\pgfsetfillopacity{0.800000}%
\pgfsetlinewidth{0.000000pt}%
\definecolor{currentstroke}{rgb}{0.000000,0.000000,0.000000}%
\pgfsetstrokecolor{currentstroke}%
\pgfsetdash{}{0pt}%
\pgfpathmoveto{\pgfqpoint{2.325598in}{2.260843in}}%
\pgfpathlineto{\pgfqpoint{2.339883in}{2.238590in}}%
\pgfpathlineto{\pgfqpoint{2.354156in}{2.216626in}}%
\pgfpathlineto{\pgfqpoint{2.368416in}{2.194946in}}%
\pgfpathlineto{\pgfqpoint{2.382664in}{2.173550in}}%
\pgfpathlineto{\pgfqpoint{2.391948in}{2.165811in}}%
\pgfpathlineto{\pgfqpoint{2.401205in}{2.158497in}}%
\pgfpathlineto{\pgfqpoint{2.410437in}{2.151599in}}%
\pgfpathlineto{\pgfqpoint{2.419643in}{2.145109in}}%
\pgfpathlineto{\pgfqpoint{2.405460in}{2.165725in}}%
\pgfpathlineto{\pgfqpoint{2.391265in}{2.186623in}}%
\pgfpathlineto{\pgfqpoint{2.377058in}{2.207805in}}%
\pgfpathlineto{\pgfqpoint{2.362840in}{2.229273in}}%
\pgfpathlineto{\pgfqpoint{2.353569in}{2.236530in}}%
\pgfpathlineto{\pgfqpoint{2.344272in}{2.244205in}}%
\pgfpathlineto{\pgfqpoint{2.334948in}{2.252307in}}%
\pgfpathlineto{\pgfqpoint{2.325598in}{2.260843in}}%
\pgfpathclose%
\pgfusepath{fill}%
\end{pgfscope}%
\begin{pgfscope}%
\pgfpathrectangle{\pgfqpoint{1.150000in}{0.150000in}}{\pgfqpoint{5.700000in}{5.700000in}}%
\pgfusepath{clip}%
\pgfsetbuttcap%
\pgfsetroundjoin%
\definecolor{currentfill}{rgb}{0.271305,0.019942,0.347269}%
\pgfsetfillcolor{currentfill}%
\pgfsetfillopacity{0.800000}%
\pgfsetlinewidth{0.000000pt}%
\definecolor{currentstroke}{rgb}{0.000000,0.000000,0.000000}%
\pgfsetstrokecolor{currentstroke}%
\pgfsetdash{}{0pt}%
\pgfpathmoveto{\pgfqpoint{3.147878in}{1.414571in}}%
\pgfpathlineto{\pgfqpoint{3.161810in}{1.406395in}}%
\pgfpathlineto{\pgfqpoint{3.175741in}{1.398421in}}%
\pgfpathlineto{\pgfqpoint{3.189674in}{1.390648in}}%
\pgfpathlineto{\pgfqpoint{3.203609in}{1.383074in}}%
\pgfpathlineto{\pgfqpoint{3.212143in}{1.387133in}}%
\pgfpathlineto{\pgfqpoint{3.220665in}{1.391458in}}%
\pgfpathlineto{\pgfqpoint{3.229175in}{1.396040in}}%
\pgfpathlineto{\pgfqpoint{3.237673in}{1.400875in}}%
\pgfpathlineto{\pgfqpoint{3.223770in}{1.407753in}}%
\pgfpathlineto{\pgfqpoint{3.209868in}{1.414831in}}%
\pgfpathlineto{\pgfqpoint{3.195968in}{1.422110in}}%
\pgfpathlineto{\pgfqpoint{3.182070in}{1.429590in}}%
\pgfpathlineto{\pgfqpoint{3.173541in}{1.425438in}}%
\pgfpathlineto{\pgfqpoint{3.165000in}{1.421547in}}%
\pgfpathlineto{\pgfqpoint{3.156446in}{1.417922in}}%
\pgfpathlineto{\pgfqpoint{3.147878in}{1.414571in}}%
\pgfpathclose%
\pgfusepath{fill}%
\end{pgfscope}%
\begin{pgfscope}%
\pgfpathrectangle{\pgfqpoint{1.150000in}{0.150000in}}{\pgfqpoint{5.700000in}{5.700000in}}%
\pgfusepath{clip}%
\pgfsetbuttcap%
\pgfsetroundjoin%
\definecolor{currentfill}{rgb}{0.267004,0.004874,0.329415}%
\pgfsetfillcolor{currentfill}%
\pgfsetfillopacity{0.800000}%
\pgfsetlinewidth{0.000000pt}%
\definecolor{currentstroke}{rgb}{0.000000,0.000000,0.000000}%
\pgfsetstrokecolor{currentstroke}%
\pgfsetdash{}{0pt}%
\pgfpathmoveto{\pgfqpoint{3.438274in}{1.363278in}}%
\pgfpathlineto{\pgfqpoint{3.452192in}{1.359589in}}%
\pgfpathlineto{\pgfqpoint{3.466114in}{1.356091in}}%
\pgfpathlineto{\pgfqpoint{3.480041in}{1.352782in}}%
\pgfpathlineto{\pgfqpoint{3.493972in}{1.349662in}}%
\pgfpathlineto{\pgfqpoint{3.502322in}{1.358214in}}%
\pgfpathlineto{\pgfqpoint{3.510663in}{1.366945in}}%
\pgfpathlineto{\pgfqpoint{3.518996in}{1.375849in}}%
\pgfpathlineto{\pgfqpoint{3.527321in}{1.384921in}}%
\pgfpathlineto{\pgfqpoint{3.513408in}{1.387413in}}%
\pgfpathlineto{\pgfqpoint{3.499501in}{1.390094in}}%
\pgfpathlineto{\pgfqpoint{3.485598in}{1.392964in}}%
\pgfpathlineto{\pgfqpoint{3.471701in}{1.396025in}}%
\pgfpathlineto{\pgfqpoint{3.463357in}{1.387569in}}%
\pgfpathlineto{\pgfqpoint{3.455005in}{1.379289in}}%
\pgfpathlineto{\pgfqpoint{3.446644in}{1.371190in}}%
\pgfpathlineto{\pgfqpoint{3.438274in}{1.363278in}}%
\pgfpathclose%
\pgfusepath{fill}%
\end{pgfscope}%
\begin{pgfscope}%
\pgfpathrectangle{\pgfqpoint{1.150000in}{0.150000in}}{\pgfqpoint{5.700000in}{5.700000in}}%
\pgfusepath{clip}%
\pgfsetbuttcap%
\pgfsetroundjoin%
\definecolor{currentfill}{rgb}{0.149039,0.508051,0.557250}%
\pgfsetfillcolor{currentfill}%
\pgfsetfillopacity{0.800000}%
\pgfsetlinewidth{0.000000pt}%
\definecolor{currentstroke}{rgb}{0.000000,0.000000,0.000000}%
\pgfsetstrokecolor{currentstroke}%
\pgfsetdash{}{0pt}%
\pgfpathmoveto{\pgfqpoint{2.075483in}{2.722806in}}%
\pgfpathlineto{\pgfqpoint{2.090007in}{2.695041in}}%
\pgfpathlineto{\pgfqpoint{2.104513in}{2.667620in}}%
\pgfpathlineto{\pgfqpoint{2.119001in}{2.640540in}}%
\pgfpathlineto{\pgfqpoint{2.133470in}{2.613797in}}%
\pgfpathlineto{\pgfqpoint{2.142994in}{2.603851in}}%
\pgfpathlineto{\pgfqpoint{2.152488in}{2.594348in}}%
\pgfpathlineto{\pgfqpoint{2.161954in}{2.585281in}}%
\pgfpathlineto{\pgfqpoint{2.171391in}{2.576643in}}%
\pgfpathlineto{\pgfqpoint{2.156995in}{2.602609in}}%
\pgfpathlineto{\pgfqpoint{2.142582in}{2.628911in}}%
\pgfpathlineto{\pgfqpoint{2.128151in}{2.655552in}}%
\pgfpathlineto{\pgfqpoint{2.113702in}{2.682534in}}%
\pgfpathlineto{\pgfqpoint{2.104192in}{2.691936in}}%
\pgfpathlineto{\pgfqpoint{2.094652in}{2.701777in}}%
\pgfpathlineto{\pgfqpoint{2.085083in}{2.712064in}}%
\pgfpathlineto{\pgfqpoint{2.075483in}{2.722806in}}%
\pgfpathclose%
\pgfusepath{fill}%
\end{pgfscope}%
\begin{pgfscope}%
\pgfpathrectangle{\pgfqpoint{1.150000in}{0.150000in}}{\pgfqpoint{5.700000in}{5.700000in}}%
\pgfusepath{clip}%
\pgfsetbuttcap%
\pgfsetroundjoin%
\definecolor{currentfill}{rgb}{0.241237,0.296485,0.539709}%
\pgfsetfillcolor{currentfill}%
\pgfsetfillopacity{0.800000}%
\pgfsetlinewidth{0.000000pt}%
\definecolor{currentstroke}{rgb}{0.000000,0.000000,0.000000}%
\pgfsetstrokecolor{currentstroke}%
\pgfsetdash{}{0pt}%
\pgfpathmoveto{\pgfqpoint{4.302084in}{1.989782in}}%
\pgfpathlineto{\pgfqpoint{4.316258in}{1.997958in}}%
\pgfpathlineto{\pgfqpoint{4.330446in}{2.006318in}}%
\pgfpathlineto{\pgfqpoint{4.344648in}{2.014861in}}%
\pgfpathlineto{\pgfqpoint{4.358865in}{2.023587in}}%
\pgfpathlineto{\pgfqpoint{4.366923in}{2.037986in}}%
\pgfpathlineto{\pgfqpoint{4.374977in}{2.052302in}}%
\pgfpathlineto{\pgfqpoint{4.383025in}{2.066530in}}%
\pgfpathlineto{\pgfqpoint{4.391069in}{2.080670in}}%
\pgfpathlineto{\pgfqpoint{4.376850in}{2.071657in}}%
\pgfpathlineto{\pgfqpoint{4.362646in}{2.062827in}}%
\pgfpathlineto{\pgfqpoint{4.348457in}{2.054180in}}%
\pgfpathlineto{\pgfqpoint{4.334281in}{2.045718in}}%
\pgfpathlineto{\pgfqpoint{4.326239in}{2.031852in}}%
\pgfpathlineto{\pgfqpoint{4.318192in}{2.017906in}}%
\pgfpathlineto{\pgfqpoint{4.310140in}{2.003882in}}%
\pgfpathlineto{\pgfqpoint{4.302084in}{1.989782in}}%
\pgfpathclose%
\pgfusepath{fill}%
\end{pgfscope}%
\begin{pgfscope}%
\pgfpathrectangle{\pgfqpoint{1.150000in}{0.150000in}}{\pgfqpoint{5.700000in}{5.700000in}}%
\pgfusepath{clip}%
\pgfsetbuttcap%
\pgfsetroundjoin%
\definecolor{currentfill}{rgb}{0.140210,0.665859,0.513427}%
\pgfsetfillcolor{currentfill}%
\pgfsetfillopacity{0.800000}%
\pgfsetlinewidth{0.000000pt}%
\definecolor{currentstroke}{rgb}{0.000000,0.000000,0.000000}%
\pgfsetstrokecolor{currentstroke}%
\pgfsetdash{}{0pt}%
\pgfpathmoveto{\pgfqpoint{5.295555in}{3.096426in}}%
\pgfpathlineto{\pgfqpoint{5.310354in}{3.112042in}}%
\pgfpathlineto{\pgfqpoint{5.325175in}{3.127845in}}%
\pgfpathlineto{\pgfqpoint{5.340017in}{3.143835in}}%
\pgfpathlineto{\pgfqpoint{5.354882in}{3.160013in}}%
\pgfpathlineto{\pgfqpoint{5.362486in}{3.165967in}}%
\pgfpathlineto{\pgfqpoint{5.370080in}{3.171761in}}%
\pgfpathlineto{\pgfqpoint{5.377664in}{3.177398in}}%
\pgfpathlineto{\pgfqpoint{5.385238in}{3.182882in}}%
\pgfpathlineto{\pgfqpoint{5.370386in}{3.166934in}}%
\pgfpathlineto{\pgfqpoint{5.355555in}{3.151174in}}%
\pgfpathlineto{\pgfqpoint{5.340746in}{3.135601in}}%
\pgfpathlineto{\pgfqpoint{5.325958in}{3.120214in}}%
\pgfpathlineto{\pgfqpoint{5.318372in}{3.114487in}}%
\pgfpathlineto{\pgfqpoint{5.310776in}{3.108616in}}%
\pgfpathlineto{\pgfqpoint{5.303170in}{3.102597in}}%
\pgfpathlineto{\pgfqpoint{5.295555in}{3.096426in}}%
\pgfpathclose%
\pgfusepath{fill}%
\end{pgfscope}%
\begin{pgfscope}%
\pgfpathrectangle{\pgfqpoint{1.150000in}{0.150000in}}{\pgfqpoint{5.700000in}{5.700000in}}%
\pgfusepath{clip}%
\pgfsetbuttcap%
\pgfsetroundjoin%
\definecolor{currentfill}{rgb}{0.120092,0.600104,0.542530}%
\pgfsetfillcolor{currentfill}%
\pgfsetfillopacity{0.800000}%
\pgfsetlinewidth{0.000000pt}%
\definecolor{currentstroke}{rgb}{0.000000,0.000000,0.000000}%
\pgfsetstrokecolor{currentstroke}%
\pgfsetdash{}{0pt}%
\pgfpathmoveto{\pgfqpoint{5.085593in}{2.887757in}}%
\pgfpathlineto{\pgfqpoint{5.100249in}{2.902351in}}%
\pgfpathlineto{\pgfqpoint{5.114926in}{2.917132in}}%
\pgfpathlineto{\pgfqpoint{5.129623in}{2.932100in}}%
\pgfpathlineto{\pgfqpoint{5.144341in}{2.947255in}}%
\pgfpathlineto{\pgfqpoint{5.152077in}{2.955526in}}%
\pgfpathlineto{\pgfqpoint{5.159804in}{2.963629in}}%
\pgfpathlineto{\pgfqpoint{5.167522in}{2.971565in}}%
\pgfpathlineto{\pgfqpoint{5.175231in}{2.979337in}}%
\pgfpathlineto{\pgfqpoint{5.160519in}{2.964304in}}%
\pgfpathlineto{\pgfqpoint{5.145829in}{2.949458in}}%
\pgfpathlineto{\pgfqpoint{5.131158in}{2.934799in}}%
\pgfpathlineto{\pgfqpoint{5.116509in}{2.920326in}}%
\pgfpathlineto{\pgfqpoint{5.108793in}{2.912419in}}%
\pgfpathlineto{\pgfqpoint{5.101068in}{2.904357in}}%
\pgfpathlineto{\pgfqpoint{5.093335in}{2.896136in}}%
\pgfpathlineto{\pgfqpoint{5.085593in}{2.887757in}}%
\pgfpathclose%
\pgfusepath{fill}%
\end{pgfscope}%
\begin{pgfscope}%
\pgfpathrectangle{\pgfqpoint{1.150000in}{0.150000in}}{\pgfqpoint{5.700000in}{5.700000in}}%
\pgfusepath{clip}%
\pgfsetbuttcap%
\pgfsetroundjoin%
\definecolor{currentfill}{rgb}{0.281887,0.150881,0.465405}%
\pgfsetfillcolor{currentfill}%
\pgfsetfillopacity{0.800000}%
\pgfsetlinewidth{0.000000pt}%
\definecolor{currentstroke}{rgb}{0.000000,0.000000,0.000000}%
\pgfsetstrokecolor{currentstroke}%
\pgfsetdash{}{0pt}%
\pgfpathmoveto{\pgfqpoint{3.970927in}{1.635544in}}%
\pgfpathlineto{\pgfqpoint{3.984957in}{1.639617in}}%
\pgfpathlineto{\pgfqpoint{3.998998in}{1.643873in}}%
\pgfpathlineto{\pgfqpoint{4.013050in}{1.648312in}}%
\pgfpathlineto{\pgfqpoint{4.027113in}{1.652933in}}%
\pgfpathlineto{\pgfqpoint{4.035258in}{1.666978in}}%
\pgfpathlineto{\pgfqpoint{4.043399in}{1.681032in}}%
\pgfpathlineto{\pgfqpoint{4.051535in}{1.695089in}}%
\pgfpathlineto{\pgfqpoint{4.059667in}{1.709145in}}%
\pgfpathlineto{\pgfqpoint{4.045607in}{1.704080in}}%
\pgfpathlineto{\pgfqpoint{4.031558in}{1.699197in}}%
\pgfpathlineto{\pgfqpoint{4.017520in}{1.694498in}}%
\pgfpathlineto{\pgfqpoint{4.003494in}{1.689981in}}%
\pgfpathlineto{\pgfqpoint{3.995359in}{1.676357in}}%
\pgfpathlineto{\pgfqpoint{3.987220in}{1.662740in}}%
\pgfpathlineto{\pgfqpoint{3.979076in}{1.649134in}}%
\pgfpathlineto{\pgfqpoint{3.970927in}{1.635544in}}%
\pgfpathclose%
\pgfusepath{fill}%
\end{pgfscope}%
\begin{pgfscope}%
\pgfpathrectangle{\pgfqpoint{1.150000in}{0.150000in}}{\pgfqpoint{5.700000in}{5.700000in}}%
\pgfusepath{clip}%
\pgfsetbuttcap%
\pgfsetroundjoin%
\definecolor{currentfill}{rgb}{0.280267,0.073417,0.397163}%
\pgfsetfillcolor{currentfill}%
\pgfsetfillopacity{0.800000}%
\pgfsetlinewidth{0.000000pt}%
\definecolor{currentstroke}{rgb}{0.000000,0.000000,0.000000}%
\pgfsetstrokecolor{currentstroke}%
\pgfsetdash{}{0pt}%
\pgfpathmoveto{\pgfqpoint{2.945831in}{1.526998in}}%
\pgfpathlineto{\pgfqpoint{2.959808in}{1.515596in}}%
\pgfpathlineto{\pgfqpoint{2.973784in}{1.504409in}}%
\pgfpathlineto{\pgfqpoint{2.987758in}{1.493434in}}%
\pgfpathlineto{\pgfqpoint{3.001730in}{1.482670in}}%
\pgfpathlineto{\pgfqpoint{3.010430in}{1.483363in}}%
\pgfpathlineto{\pgfqpoint{3.019114in}{1.484377in}}%
\pgfpathlineto{\pgfqpoint{3.027782in}{1.485703in}}%
\pgfpathlineto{\pgfqpoint{3.036436in}{1.487336in}}%
\pgfpathlineto{\pgfqpoint{3.022504in}{1.497366in}}%
\pgfpathlineto{\pgfqpoint{3.008570in}{1.507606in}}%
\pgfpathlineto{\pgfqpoint{2.994636in}{1.518059in}}%
\pgfpathlineto{\pgfqpoint{2.980700in}{1.528724in}}%
\pgfpathlineto{\pgfqpoint{2.972007in}{1.527813in}}%
\pgfpathlineto{\pgfqpoint{2.963298in}{1.527217in}}%
\pgfpathlineto{\pgfqpoint{2.954573in}{1.526943in}}%
\pgfpathlineto{\pgfqpoint{2.945831in}{1.526998in}}%
\pgfpathclose%
\pgfusepath{fill}%
\end{pgfscope}%
\begin{pgfscope}%
\pgfpathrectangle{\pgfqpoint{1.150000in}{0.150000in}}{\pgfqpoint{5.700000in}{5.700000in}}%
\pgfusepath{clip}%
\pgfsetbuttcap%
\pgfsetroundjoin%
\definecolor{currentfill}{rgb}{0.159194,0.482237,0.558073}%
\pgfsetfillcolor{currentfill}%
\pgfsetfillopacity{0.800000}%
\pgfsetlinewidth{0.000000pt}%
\definecolor{currentstroke}{rgb}{0.000000,0.000000,0.000000}%
\pgfsetstrokecolor{currentstroke}%
\pgfsetdash{}{0pt}%
\pgfpathmoveto{\pgfqpoint{4.754490in}{2.523394in}}%
\pgfpathlineto{\pgfqpoint{4.768930in}{2.535837in}}%
\pgfpathlineto{\pgfqpoint{4.783387in}{2.548465in}}%
\pgfpathlineto{\pgfqpoint{4.797863in}{2.561278in}}%
\pgfpathlineto{\pgfqpoint{4.812357in}{2.574278in}}%
\pgfpathlineto{\pgfqpoint{4.820262in}{2.585919in}}%
\pgfpathlineto{\pgfqpoint{4.828160in}{2.597403in}}%
\pgfpathlineto{\pgfqpoint{4.836051in}{2.608732in}}%
\pgfpathlineto{\pgfqpoint{4.843935in}{2.619904in}}%
\pgfpathlineto{\pgfqpoint{4.829441in}{2.606851in}}%
\pgfpathlineto{\pgfqpoint{4.814966in}{2.593984in}}%
\pgfpathlineto{\pgfqpoint{4.800508in}{2.581303in}}%
\pgfpathlineto{\pgfqpoint{4.786069in}{2.568807in}}%
\pgfpathlineto{\pgfqpoint{4.778185in}{2.557675in}}%
\pgfpathlineto{\pgfqpoint{4.770293in}{2.546396in}}%
\pgfpathlineto{\pgfqpoint{4.762395in}{2.534969in}}%
\pgfpathlineto{\pgfqpoint{4.754490in}{2.523394in}}%
\pgfpathclose%
\pgfusepath{fill}%
\end{pgfscope}%
\begin{pgfscope}%
\pgfpathrectangle{\pgfqpoint{1.150000in}{0.150000in}}{\pgfqpoint{5.700000in}{5.700000in}}%
\pgfusepath{clip}%
\pgfsetbuttcap%
\pgfsetroundjoin%
\definecolor{currentfill}{rgb}{0.259857,0.745492,0.444467}%
\pgfsetfillcolor{currentfill}%
\pgfsetfillopacity{0.800000}%
\pgfsetlinewidth{0.000000pt}%
\definecolor{currentstroke}{rgb}{0.000000,0.000000,0.000000}%
\pgfsetstrokecolor{currentstroke}%
\pgfsetdash{}{0pt}%
\pgfpathmoveto{\pgfqpoint{5.594605in}{3.364915in}}%
\pgfpathlineto{\pgfqpoint{5.609614in}{3.381640in}}%
\pgfpathlineto{\pgfqpoint{5.624645in}{3.398551in}}%
\pgfpathlineto{\pgfqpoint{5.639700in}{3.415651in}}%
\pgfpathlineto{\pgfqpoint{5.654779in}{3.432938in}}%
\pgfpathlineto{\pgfqpoint{5.662168in}{3.435640in}}%
\pgfpathlineto{\pgfqpoint{5.669547in}{3.438212in}}%
\pgfpathlineto{\pgfqpoint{5.676915in}{3.440658in}}%
\pgfpathlineto{\pgfqpoint{5.684272in}{3.442981in}}%
\pgfpathlineto{\pgfqpoint{5.669214in}{3.426071in}}%
\pgfpathlineto{\pgfqpoint{5.654179in}{3.409348in}}%
\pgfpathlineto{\pgfqpoint{5.639168in}{3.392811in}}%
\pgfpathlineto{\pgfqpoint{5.624180in}{3.376461in}}%
\pgfpathlineto{\pgfqpoint{5.616802in}{3.373749in}}%
\pgfpathlineto{\pgfqpoint{5.609413in}{3.370924in}}%
\pgfpathlineto{\pgfqpoint{5.602014in}{3.367981in}}%
\pgfpathlineto{\pgfqpoint{5.594605in}{3.364915in}}%
\pgfpathclose%
\pgfusepath{fill}%
\end{pgfscope}%
\begin{pgfscope}%
\pgfpathrectangle{\pgfqpoint{1.150000in}{0.150000in}}{\pgfqpoint{5.700000in}{5.700000in}}%
\pgfusepath{clip}%
\pgfsetbuttcap%
\pgfsetroundjoin%
\definecolor{currentfill}{rgb}{0.197636,0.391528,0.554969}%
\pgfsetfillcolor{currentfill}%
\pgfsetfillopacity{0.800000}%
\pgfsetlinewidth{0.000000pt}%
\definecolor{currentstroke}{rgb}{0.000000,0.000000,0.000000}%
\pgfsetstrokecolor{currentstroke}%
\pgfsetdash{}{0pt}%
\pgfpathmoveto{\pgfqpoint{2.268326in}{2.352782in}}%
\pgfpathlineto{\pgfqpoint{2.282664in}{2.329353in}}%
\pgfpathlineto{\pgfqpoint{2.296989in}{2.306222in}}%
\pgfpathlineto{\pgfqpoint{2.311300in}{2.283386in}}%
\pgfpathlineto{\pgfqpoint{2.325598in}{2.260843in}}%
\pgfpathlineto{\pgfqpoint{2.334948in}{2.252307in}}%
\pgfpathlineto{\pgfqpoint{2.344272in}{2.244205in}}%
\pgfpathlineto{\pgfqpoint{2.353569in}{2.236530in}}%
\pgfpathlineto{\pgfqpoint{2.362840in}{2.229273in}}%
\pgfpathlineto{\pgfqpoint{2.348609in}{2.251029in}}%
\pgfpathlineto{\pgfqpoint{2.334366in}{2.273076in}}%
\pgfpathlineto{\pgfqpoint{2.320110in}{2.295417in}}%
\pgfpathlineto{\pgfqpoint{2.305841in}{2.318054in}}%
\pgfpathlineto{\pgfqpoint{2.296503in}{2.326085in}}%
\pgfpathlineto{\pgfqpoint{2.287138in}{2.334544in}}%
\pgfpathlineto{\pgfqpoint{2.277746in}{2.343441in}}%
\pgfpathlineto{\pgfqpoint{2.268326in}{2.352782in}}%
\pgfpathclose%
\pgfusepath{fill}%
\end{pgfscope}%
\begin{pgfscope}%
\pgfpathrectangle{\pgfqpoint{1.150000in}{0.150000in}}{\pgfqpoint{5.700000in}{5.700000in}}%
\pgfusepath{clip}%
\pgfsetbuttcap%
\pgfsetroundjoin%
\definecolor{currentfill}{rgb}{0.262138,0.242286,0.520837}%
\pgfsetfillcolor{currentfill}%
\pgfsetfillopacity{0.800000}%
\pgfsetlinewidth{0.000000pt}%
\definecolor{currentstroke}{rgb}{0.000000,0.000000,0.000000}%
\pgfsetstrokecolor{currentstroke}%
\pgfsetdash{}{0pt}%
\pgfpathmoveto{\pgfqpoint{4.180915in}{1.846384in}}%
\pgfpathlineto{\pgfqpoint{4.195034in}{1.853159in}}%
\pgfpathlineto{\pgfqpoint{4.209166in}{1.860116in}}%
\pgfpathlineto{\pgfqpoint{4.223311in}{1.867256in}}%
\pgfpathlineto{\pgfqpoint{4.237470in}{1.874579in}}%
\pgfpathlineto{\pgfqpoint{4.245562in}{1.889186in}}%
\pgfpathlineto{\pgfqpoint{4.253650in}{1.903740in}}%
\pgfpathlineto{\pgfqpoint{4.261733in}{1.918237in}}%
\pgfpathlineto{\pgfqpoint{4.269812in}{1.932676in}}%
\pgfpathlineto{\pgfqpoint{4.255653in}{1.925002in}}%
\pgfpathlineto{\pgfqpoint{4.241507in}{1.917511in}}%
\pgfpathlineto{\pgfqpoint{4.227375in}{1.910203in}}%
\pgfpathlineto{\pgfqpoint{4.213256in}{1.903078in}}%
\pgfpathlineto{\pgfqpoint{4.205177in}{1.888978in}}%
\pgfpathlineto{\pgfqpoint{4.197094in}{1.874827in}}%
\pgfpathlineto{\pgfqpoint{4.189007in}{1.860628in}}%
\pgfpathlineto{\pgfqpoint{4.180915in}{1.846384in}}%
\pgfpathclose%
\pgfusepath{fill}%
\end{pgfscope}%
\begin{pgfscope}%
\pgfpathrectangle{\pgfqpoint{1.150000in}{0.150000in}}{\pgfqpoint{5.700000in}{5.700000in}}%
\pgfusepath{clip}%
\pgfsetbuttcap%
\pgfsetroundjoin%
\definecolor{currentfill}{rgb}{0.267004,0.004874,0.329415}%
\pgfsetfillcolor{currentfill}%
\pgfsetfillopacity{0.800000}%
\pgfsetlinewidth{0.000000pt}%
\definecolor{currentstroke}{rgb}{0.000000,0.000000,0.000000}%
\pgfsetstrokecolor{currentstroke}%
\pgfsetdash{}{0pt}%
\pgfpathmoveto{\pgfqpoint{3.348978in}{1.352942in}}%
\pgfpathlineto{\pgfqpoint{3.362903in}{1.347827in}}%
\pgfpathlineto{\pgfqpoint{3.376833in}{1.342904in}}%
\pgfpathlineto{\pgfqpoint{3.390765in}{1.338173in}}%
\pgfpathlineto{\pgfqpoint{3.404701in}{1.333633in}}%
\pgfpathlineto{\pgfqpoint{3.413109in}{1.340731in}}%
\pgfpathlineto{\pgfqpoint{3.421507in}{1.348042in}}%
\pgfpathlineto{\pgfqpoint{3.429895in}{1.355560in}}%
\pgfpathlineto{\pgfqpoint{3.438274in}{1.363278in}}%
\pgfpathlineto{\pgfqpoint{3.424360in}{1.367157in}}%
\pgfpathlineto{\pgfqpoint{3.410451in}{1.371228in}}%
\pgfpathlineto{\pgfqpoint{3.396546in}{1.375490in}}%
\pgfpathlineto{\pgfqpoint{3.382645in}{1.379945in}}%
\pgfpathlineto{\pgfqpoint{3.374243in}{1.372875in}}%
\pgfpathlineto{\pgfqpoint{3.365831in}{1.366014in}}%
\pgfpathlineto{\pgfqpoint{3.357410in}{1.359367in}}%
\pgfpathlineto{\pgfqpoint{3.348978in}{1.352942in}}%
\pgfpathclose%
\pgfusepath{fill}%
\end{pgfscope}%
\begin{pgfscope}%
\pgfpathrectangle{\pgfqpoint{1.150000in}{0.150000in}}{\pgfqpoint{5.700000in}{5.700000in}}%
\pgfusepath{clip}%
\pgfsetbuttcap%
\pgfsetroundjoin%
\definecolor{currentfill}{rgb}{0.177423,0.437527,0.557565}%
\pgfsetfillcolor{currentfill}%
\pgfsetfillopacity{0.800000}%
\pgfsetlinewidth{0.000000pt}%
\definecolor{currentstroke}{rgb}{0.000000,0.000000,0.000000}%
\pgfsetstrokecolor{currentstroke}%
\pgfsetdash{}{0pt}%
\pgfpathmoveto{\pgfqpoint{4.633434in}{2.378447in}}%
\pgfpathlineto{\pgfqpoint{4.647803in}{2.389939in}}%
\pgfpathlineto{\pgfqpoint{4.662189in}{2.401616in}}%
\pgfpathlineto{\pgfqpoint{4.676592in}{2.413479in}}%
\pgfpathlineto{\pgfqpoint{4.691012in}{2.425526in}}%
\pgfpathlineto{\pgfqpoint{4.698969in}{2.438267in}}%
\pgfpathlineto{\pgfqpoint{4.706920in}{2.450864in}}%
\pgfpathlineto{\pgfqpoint{4.714865in}{2.463317in}}%
\pgfpathlineto{\pgfqpoint{4.722803in}{2.475625in}}%
\pgfpathlineto{\pgfqpoint{4.708381in}{2.463456in}}%
\pgfpathlineto{\pgfqpoint{4.693977in}{2.451472in}}%
\pgfpathlineto{\pgfqpoint{4.679590in}{2.439673in}}%
\pgfpathlineto{\pgfqpoint{4.665221in}{2.428059in}}%
\pgfpathlineto{\pgfqpoint{4.657284in}{2.415860in}}%
\pgfpathlineto{\pgfqpoint{4.649340in}{2.403524in}}%
\pgfpathlineto{\pgfqpoint{4.641390in}{2.391053in}}%
\pgfpathlineto{\pgfqpoint{4.633434in}{2.378447in}}%
\pgfpathclose%
\pgfusepath{fill}%
\end{pgfscope}%
\begin{pgfscope}%
\pgfpathrectangle{\pgfqpoint{1.150000in}{0.150000in}}{\pgfqpoint{5.700000in}{5.700000in}}%
\pgfusepath{clip}%
\pgfsetbuttcap%
\pgfsetroundjoin%
\definecolor{currentfill}{rgb}{0.276022,0.044167,0.370164}%
\pgfsetfillcolor{currentfill}%
\pgfsetfillopacity{0.800000}%
\pgfsetlinewidth{0.000000pt}%
\definecolor{currentstroke}{rgb}{0.000000,0.000000,0.000000}%
\pgfsetstrokecolor{currentstroke}%
\pgfsetdash{}{0pt}%
\pgfpathmoveto{\pgfqpoint{3.671931in}{1.414350in}}%
\pgfpathlineto{\pgfqpoint{3.685885in}{1.414137in}}%
\pgfpathlineto{\pgfqpoint{3.699846in}{1.414108in}}%
\pgfpathlineto{\pgfqpoint{3.713815in}{1.414264in}}%
\pgfpathlineto{\pgfqpoint{3.727791in}{1.414604in}}%
\pgfpathlineto{\pgfqpoint{3.736038in}{1.426155in}}%
\pgfpathlineto{\pgfqpoint{3.744278in}{1.437816in}}%
\pgfpathlineto{\pgfqpoint{3.752512in}{1.449579in}}%
\pgfpathlineto{\pgfqpoint{3.760741in}{1.461440in}}%
\pgfpathlineto{\pgfqpoint{3.746775in}{1.460533in}}%
\pgfpathlineto{\pgfqpoint{3.732818in}{1.459811in}}%
\pgfpathlineto{\pgfqpoint{3.718868in}{1.459273in}}%
\pgfpathlineto{\pgfqpoint{3.704926in}{1.458921in}}%
\pgfpathlineto{\pgfqpoint{3.696686in}{1.447614in}}%
\pgfpathlineto{\pgfqpoint{3.688441in}{1.436413in}}%
\pgfpathlineto{\pgfqpoint{3.680189in}{1.425324in}}%
\pgfpathlineto{\pgfqpoint{3.671931in}{1.414350in}}%
\pgfpathclose%
\pgfusepath{fill}%
\end{pgfscope}%
\begin{pgfscope}%
\pgfpathrectangle{\pgfqpoint{1.150000in}{0.150000in}}{\pgfqpoint{5.700000in}{5.700000in}}%
\pgfusepath{clip}%
\pgfsetbuttcap%
\pgfsetroundjoin%
\definecolor{currentfill}{rgb}{0.269944,0.014625,0.341379}%
\pgfsetfillcolor{currentfill}%
\pgfsetfillopacity{0.800000}%
\pgfsetlinewidth{0.000000pt}%
\definecolor{currentstroke}{rgb}{0.000000,0.000000,0.000000}%
\pgfsetstrokecolor{currentstroke}%
\pgfsetdash{}{0pt}%
\pgfpathmoveto{\pgfqpoint{3.203609in}{1.383074in}}%
\pgfpathlineto{\pgfqpoint{3.217544in}{1.375700in}}%
\pgfpathlineto{\pgfqpoint{3.231481in}{1.368524in}}%
\pgfpathlineto{\pgfqpoint{3.245419in}{1.361545in}}%
\pgfpathlineto{\pgfqpoint{3.259360in}{1.354763in}}%
\pgfpathlineto{\pgfqpoint{3.267864in}{1.359529in}}%
\pgfpathlineto{\pgfqpoint{3.276356in}{1.364552in}}%
\pgfpathlineto{\pgfqpoint{3.284836in}{1.369825in}}%
\pgfpathlineto{\pgfqpoint{3.293305in}{1.375341in}}%
\pgfpathlineto{\pgfqpoint{3.279394in}{1.381429in}}%
\pgfpathlineto{\pgfqpoint{3.265485in}{1.387713in}}%
\pgfpathlineto{\pgfqpoint{3.251578in}{1.394195in}}%
\pgfpathlineto{\pgfqpoint{3.237673in}{1.400875in}}%
\pgfpathlineto{\pgfqpoint{3.229175in}{1.396040in}}%
\pgfpathlineto{\pgfqpoint{3.220665in}{1.391458in}}%
\pgfpathlineto{\pgfqpoint{3.212143in}{1.387133in}}%
\pgfpathlineto{\pgfqpoint{3.203609in}{1.383074in}}%
\pgfpathclose%
\pgfusepath{fill}%
\end{pgfscope}%
\begin{pgfscope}%
\pgfpathrectangle{\pgfqpoint{1.150000in}{0.150000in}}{\pgfqpoint{5.700000in}{5.700000in}}%
\pgfusepath{clip}%
\pgfsetbuttcap%
\pgfsetroundjoin%
\definecolor{currentfill}{rgb}{0.279566,0.067836,0.391917}%
\pgfsetfillcolor{currentfill}%
\pgfsetfillopacity{0.800000}%
\pgfsetlinewidth{0.000000pt}%
\definecolor{currentstroke}{rgb}{0.000000,0.000000,0.000000}%
\pgfsetstrokecolor{currentstroke}%
\pgfsetdash{}{0pt}%
\pgfpathmoveto{\pgfqpoint{3.760741in}{1.461440in}}%
\pgfpathlineto{\pgfqpoint{3.774715in}{1.462531in}}%
\pgfpathlineto{\pgfqpoint{3.788697in}{1.463805in}}%
\pgfpathlineto{\pgfqpoint{3.802688in}{1.465263in}}%
\pgfpathlineto{\pgfqpoint{3.816687in}{1.466904in}}%
\pgfpathlineto{\pgfqpoint{3.824901in}{1.479406in}}%
\pgfpathlineto{\pgfqpoint{3.833109in}{1.491987in}}%
\pgfpathlineto{\pgfqpoint{3.841313in}{1.504643in}}%
\pgfpathlineto{\pgfqpoint{3.849510in}{1.517368in}}%
\pgfpathlineto{\pgfqpoint{3.835519in}{1.515190in}}%
\pgfpathlineto{\pgfqpoint{3.821536in}{1.513196in}}%
\pgfpathlineto{\pgfqpoint{3.807563in}{1.511385in}}%
\pgfpathlineto{\pgfqpoint{3.793598in}{1.509759in}}%
\pgfpathlineto{\pgfqpoint{3.785392in}{1.497558in}}%
\pgfpathlineto{\pgfqpoint{3.777181in}{1.485435in}}%
\pgfpathlineto{\pgfqpoint{3.768964in}{1.473394in}}%
\pgfpathlineto{\pgfqpoint{3.760741in}{1.461440in}}%
\pgfpathclose%
\pgfusepath{fill}%
\end{pgfscope}%
\begin{pgfscope}%
\pgfpathrectangle{\pgfqpoint{1.150000in}{0.150000in}}{\pgfqpoint{5.700000in}{5.700000in}}%
\pgfusepath{clip}%
\pgfsetbuttcap%
\pgfsetroundjoin%
\definecolor{currentfill}{rgb}{0.271305,0.019942,0.347269}%
\pgfsetfillcolor{currentfill}%
\pgfsetfillopacity{0.800000}%
\pgfsetlinewidth{0.000000pt}%
\definecolor{currentstroke}{rgb}{0.000000,0.000000,0.000000}%
\pgfsetstrokecolor{currentstroke}%
\pgfsetdash{}{0pt}%
\pgfpathmoveto{\pgfqpoint{3.583027in}{1.376835in}}%
\pgfpathlineto{\pgfqpoint{3.596969in}{1.375281in}}%
\pgfpathlineto{\pgfqpoint{3.610917in}{1.373914in}}%
\pgfpathlineto{\pgfqpoint{3.624871in}{1.372732in}}%
\pgfpathlineto{\pgfqpoint{3.638831in}{1.371736in}}%
\pgfpathlineto{\pgfqpoint{3.647117in}{1.382187in}}%
\pgfpathlineto{\pgfqpoint{3.655395in}{1.392776in}}%
\pgfpathlineto{\pgfqpoint{3.663666in}{1.403500in}}%
\pgfpathlineto{\pgfqpoint{3.671931in}{1.414350in}}%
\pgfpathlineto{\pgfqpoint{3.657984in}{1.414749in}}%
\pgfpathlineto{\pgfqpoint{3.644044in}{1.415333in}}%
\pgfpathlineto{\pgfqpoint{3.630111in}{1.416104in}}%
\pgfpathlineto{\pgfqpoint{3.616185in}{1.417061in}}%
\pgfpathlineto{\pgfqpoint{3.607906in}{1.406795in}}%
\pgfpathlineto{\pgfqpoint{3.599620in}{1.396665in}}%
\pgfpathlineto{\pgfqpoint{3.591328in}{1.386676in}}%
\pgfpathlineto{\pgfqpoint{3.583027in}{1.376835in}}%
\pgfpathclose%
\pgfusepath{fill}%
\end{pgfscope}%
\begin{pgfscope}%
\pgfpathrectangle{\pgfqpoint{1.150000in}{0.150000in}}{\pgfqpoint{5.700000in}{5.700000in}}%
\pgfusepath{clip}%
\pgfsetbuttcap%
\pgfsetroundjoin%
\definecolor{currentfill}{rgb}{0.277941,0.056324,0.381191}%
\pgfsetfillcolor{currentfill}%
\pgfsetfillopacity{0.800000}%
\pgfsetlinewidth{0.000000pt}%
\definecolor{currentstroke}{rgb}{0.000000,0.000000,0.000000}%
\pgfsetstrokecolor{currentstroke}%
\pgfsetdash{}{0pt}%
\pgfpathmoveto{\pgfqpoint{3.001730in}{1.482670in}}%
\pgfpathlineto{\pgfqpoint{3.015701in}{1.472116in}}%
\pgfpathlineto{\pgfqpoint{3.029670in}{1.461772in}}%
\pgfpathlineto{\pgfqpoint{3.043639in}{1.451637in}}%
\pgfpathlineto{\pgfqpoint{3.057606in}{1.441708in}}%
\pgfpathlineto{\pgfqpoint{3.066267in}{1.443147in}}%
\pgfpathlineto{\pgfqpoint{3.074912in}{1.444898in}}%
\pgfpathlineto{\pgfqpoint{3.083542in}{1.446953in}}%
\pgfpathlineto{\pgfqpoint{3.092158in}{1.449305in}}%
\pgfpathlineto{\pgfqpoint{3.078228in}{1.458502in}}%
\pgfpathlineto{\pgfqpoint{3.064298in}{1.467905in}}%
\pgfpathlineto{\pgfqpoint{3.050367in}{1.477516in}}%
\pgfpathlineto{\pgfqpoint{3.036436in}{1.487336in}}%
\pgfpathlineto{\pgfqpoint{3.027782in}{1.485703in}}%
\pgfpathlineto{\pgfqpoint{3.019114in}{1.484377in}}%
\pgfpathlineto{\pgfqpoint{3.010430in}{1.483363in}}%
\pgfpathlineto{\pgfqpoint{3.001730in}{1.482670in}}%
\pgfpathclose%
\pgfusepath{fill}%
\end{pgfscope}%
\begin{pgfscope}%
\pgfpathrectangle{\pgfqpoint{1.150000in}{0.150000in}}{\pgfqpoint{5.700000in}{5.700000in}}%
\pgfusepath{clip}%
\pgfsetbuttcap%
\pgfsetroundjoin%
\definecolor{currentfill}{rgb}{0.277134,0.185228,0.489898}%
\pgfsetfillcolor{currentfill}%
\pgfsetfillopacity{0.800000}%
\pgfsetlinewidth{0.000000pt}%
\definecolor{currentstroke}{rgb}{0.000000,0.000000,0.000000}%
\pgfsetstrokecolor{currentstroke}%
\pgfsetdash{}{0pt}%
\pgfpathmoveto{\pgfqpoint{4.059667in}{1.709145in}}%
\pgfpathlineto{\pgfqpoint{4.073739in}{1.714394in}}%
\pgfpathlineto{\pgfqpoint{4.087823in}{1.719824in}}%
\pgfpathlineto{\pgfqpoint{4.101918in}{1.725437in}}%
\pgfpathlineto{\pgfqpoint{4.116025in}{1.731232in}}%
\pgfpathlineto{\pgfqpoint{4.124151in}{1.745710in}}%
\pgfpathlineto{\pgfqpoint{4.132273in}{1.760171in}}%
\pgfpathlineto{\pgfqpoint{4.140391in}{1.774612in}}%
\pgfpathlineto{\pgfqpoint{4.148505in}{1.789029in}}%
\pgfpathlineto{\pgfqpoint{4.134398in}{1.782819in}}%
\pgfpathlineto{\pgfqpoint{4.120304in}{1.776792in}}%
\pgfpathlineto{\pgfqpoint{4.106222in}{1.770948in}}%
\pgfpathlineto{\pgfqpoint{4.092151in}{1.765287in}}%
\pgfpathlineto{\pgfqpoint{4.084037in}{1.751272in}}%
\pgfpathlineto{\pgfqpoint{4.075918in}{1.737241in}}%
\pgfpathlineto{\pgfqpoint{4.067795in}{1.723197in}}%
\pgfpathlineto{\pgfqpoint{4.059667in}{1.709145in}}%
\pgfpathclose%
\pgfusepath{fill}%
\end{pgfscope}%
\begin{pgfscope}%
\pgfpathrectangle{\pgfqpoint{1.150000in}{0.150000in}}{\pgfqpoint{5.700000in}{5.700000in}}%
\pgfusepath{clip}%
\pgfsetbuttcap%
\pgfsetroundjoin%
\definecolor{currentfill}{rgb}{0.129933,0.559582,0.551864}%
\pgfsetfillcolor{currentfill}%
\pgfsetfillopacity{0.800000}%
\pgfsetlinewidth{0.000000pt}%
\definecolor{currentstroke}{rgb}{0.000000,0.000000,0.000000}%
\pgfsetstrokecolor{currentstroke}%
\pgfsetdash{}{0pt}%
\pgfpathmoveto{\pgfqpoint{4.964888in}{2.757758in}}%
\pgfpathlineto{\pgfqpoint{4.979474in}{2.771744in}}%
\pgfpathlineto{\pgfqpoint{4.994079in}{2.785917in}}%
\pgfpathlineto{\pgfqpoint{5.008703in}{2.800276in}}%
\pgfpathlineto{\pgfqpoint{5.023348in}{2.814822in}}%
\pgfpathlineto{\pgfqpoint{5.031158in}{2.824523in}}%
\pgfpathlineto{\pgfqpoint{5.038960in}{2.834056in}}%
\pgfpathlineto{\pgfqpoint{5.046753in}{2.843420in}}%
\pgfpathlineto{\pgfqpoint{5.054538in}{2.852617in}}%
\pgfpathlineto{\pgfqpoint{5.039897in}{2.838123in}}%
\pgfpathlineto{\pgfqpoint{5.025276in}{2.823815in}}%
\pgfpathlineto{\pgfqpoint{5.010675in}{2.809694in}}%
\pgfpathlineto{\pgfqpoint{4.996093in}{2.795759in}}%
\pgfpathlineto{\pgfqpoint{4.988304in}{2.786498in}}%
\pgfpathlineto{\pgfqpoint{4.980507in}{2.777078in}}%
\pgfpathlineto{\pgfqpoint{4.972702in}{2.767498in}}%
\pgfpathlineto{\pgfqpoint{4.964888in}{2.757758in}}%
\pgfpathclose%
\pgfusepath{fill}%
\end{pgfscope}%
\begin{pgfscope}%
\pgfpathrectangle{\pgfqpoint{1.150000in}{0.150000in}}{\pgfqpoint{5.700000in}{5.700000in}}%
\pgfusepath{clip}%
\pgfsetbuttcap%
\pgfsetroundjoin%
\definecolor{currentfill}{rgb}{0.199430,0.387607,0.554642}%
\pgfsetfillcolor{currentfill}%
\pgfsetfillopacity{0.800000}%
\pgfsetlinewidth{0.000000pt}%
\definecolor{currentstroke}{rgb}{0.000000,0.000000,0.000000}%
\pgfsetstrokecolor{currentstroke}%
\pgfsetdash{}{0pt}%
\pgfpathmoveto{\pgfqpoint{4.512282in}{2.230120in}}%
\pgfpathlineto{\pgfqpoint{4.526582in}{2.240529in}}%
\pgfpathlineto{\pgfqpoint{4.540897in}{2.251123in}}%
\pgfpathlineto{\pgfqpoint{4.555229in}{2.261901in}}%
\pgfpathlineto{\pgfqpoint{4.569577in}{2.272864in}}%
\pgfpathlineto{\pgfqpoint{4.577579in}{2.286510in}}%
\pgfpathlineto{\pgfqpoint{4.585575in}{2.300031in}}%
\pgfpathlineto{\pgfqpoint{4.593566in}{2.313426in}}%
\pgfpathlineto{\pgfqpoint{4.601552in}{2.326691in}}%
\pgfpathlineto{\pgfqpoint{4.587201in}{2.315539in}}%
\pgfpathlineto{\pgfqpoint{4.572868in}{2.304572in}}%
\pgfpathlineto{\pgfqpoint{4.558550in}{2.293789in}}%
\pgfpathlineto{\pgfqpoint{4.544249in}{2.283191in}}%
\pgfpathlineto{\pgfqpoint{4.536266in}{2.270101in}}%
\pgfpathlineto{\pgfqpoint{4.528277in}{2.256892in}}%
\pgfpathlineto{\pgfqpoint{4.520282in}{2.243564in}}%
\pgfpathlineto{\pgfqpoint{4.512282in}{2.230120in}}%
\pgfpathclose%
\pgfusepath{fill}%
\end{pgfscope}%
\begin{pgfscope}%
\pgfpathrectangle{\pgfqpoint{1.150000in}{0.150000in}}{\pgfqpoint{5.700000in}{5.700000in}}%
\pgfusepath{clip}%
\pgfsetbuttcap%
\pgfsetroundjoin%
\definecolor{currentfill}{rgb}{0.182256,0.426184,0.557120}%
\pgfsetfillcolor{currentfill}%
\pgfsetfillopacity{0.800000}%
\pgfsetlinewidth{0.000000pt}%
\definecolor{currentstroke}{rgb}{0.000000,0.000000,0.000000}%
\pgfsetstrokecolor{currentstroke}%
\pgfsetdash{}{0pt}%
\pgfpathmoveto{\pgfqpoint{2.210828in}{2.449531in}}%
\pgfpathlineto{\pgfqpoint{2.225225in}{2.424883in}}%
\pgfpathlineto{\pgfqpoint{2.239606in}{2.400544in}}%
\pgfpathlineto{\pgfqpoint{2.253973in}{2.376511in}}%
\pgfpathlineto{\pgfqpoint{2.268326in}{2.352782in}}%
\pgfpathlineto{\pgfqpoint{2.277746in}{2.343441in}}%
\pgfpathlineto{\pgfqpoint{2.287138in}{2.334544in}}%
\pgfpathlineto{\pgfqpoint{2.296503in}{2.326085in}}%
\pgfpathlineto{\pgfqpoint{2.305841in}{2.318054in}}%
\pgfpathlineto{\pgfqpoint{2.291558in}{2.340990in}}%
\pgfpathlineto{\pgfqpoint{2.277262in}{2.364227in}}%
\pgfpathlineto{\pgfqpoint{2.262952in}{2.387768in}}%
\pgfpathlineto{\pgfqpoint{2.248627in}{2.411616in}}%
\pgfpathlineto{\pgfqpoint{2.239220in}{2.420428in}}%
\pgfpathlineto{\pgfqpoint{2.229785in}{2.429679in}}%
\pgfpathlineto{\pgfqpoint{2.220321in}{2.439377in}}%
\pgfpathlineto{\pgfqpoint{2.210828in}{2.449531in}}%
\pgfpathclose%
\pgfusepath{fill}%
\end{pgfscope}%
\begin{pgfscope}%
\pgfpathrectangle{\pgfqpoint{1.150000in}{0.150000in}}{\pgfqpoint{5.700000in}{5.700000in}}%
\pgfusepath{clip}%
\pgfsetbuttcap%
\pgfsetroundjoin%
\definecolor{currentfill}{rgb}{0.170948,0.694384,0.493803}%
\pgfsetfillcolor{currentfill}%
\pgfsetfillopacity{0.800000}%
\pgfsetlinewidth{0.000000pt}%
\definecolor{currentstroke}{rgb}{0.000000,0.000000,0.000000}%
\pgfsetstrokecolor{currentstroke}%
\pgfsetdash{}{0pt}%
\pgfpathmoveto{\pgfqpoint{5.385238in}{3.182882in}}%
\pgfpathlineto{\pgfqpoint{5.400113in}{3.199016in}}%
\pgfpathlineto{\pgfqpoint{5.415010in}{3.215338in}}%
\pgfpathlineto{\pgfqpoint{5.429930in}{3.231848in}}%
\pgfpathlineto{\pgfqpoint{5.444872in}{3.248546in}}%
\pgfpathlineto{\pgfqpoint{5.452423in}{3.253625in}}%
\pgfpathlineto{\pgfqpoint{5.459963in}{3.258547in}}%
\pgfpathlineto{\pgfqpoint{5.467493in}{3.263316in}}%
\pgfpathlineto{\pgfqpoint{5.475013in}{3.267934in}}%
\pgfpathlineto{\pgfqpoint{5.460085in}{3.251504in}}%
\pgfpathlineto{\pgfqpoint{5.445179in}{3.235261in}}%
\pgfpathlineto{\pgfqpoint{5.430296in}{3.219206in}}%
\pgfpathlineto{\pgfqpoint{5.415435in}{3.203337in}}%
\pgfpathlineto{\pgfqpoint{5.407901in}{3.198439in}}%
\pgfpathlineto{\pgfqpoint{5.400357in}{3.193399in}}%
\pgfpathlineto{\pgfqpoint{5.392802in}{3.188214in}}%
\pgfpathlineto{\pgfqpoint{5.385238in}{3.182882in}}%
\pgfpathclose%
\pgfusepath{fill}%
\end{pgfscope}%
\begin{pgfscope}%
\pgfpathrectangle{\pgfqpoint{1.150000in}{0.150000in}}{\pgfqpoint{5.700000in}{5.700000in}}%
\pgfusepath{clip}%
\pgfsetbuttcap%
\pgfsetroundjoin%
\definecolor{currentfill}{rgb}{0.319809,0.770914,0.411152}%
\pgfsetfillcolor{currentfill}%
\pgfsetfillopacity{0.800000}%
\pgfsetlinewidth{0.000000pt}%
\definecolor{currentstroke}{rgb}{0.000000,0.000000,0.000000}%
\pgfsetstrokecolor{currentstroke}%
\pgfsetdash{}{0pt}%
\pgfpathmoveto{\pgfqpoint{5.684272in}{3.442981in}}%
\pgfpathlineto{\pgfqpoint{5.699353in}{3.460079in}}%
\pgfpathlineto{\pgfqpoint{5.714459in}{3.477364in}}%
\pgfpathlineto{\pgfqpoint{5.729588in}{3.494837in}}%
\pgfpathlineto{\pgfqpoint{5.744741in}{3.512499in}}%
\pgfpathlineto{\pgfqpoint{5.752066in}{3.514304in}}%
\pgfpathlineto{\pgfqpoint{5.759379in}{3.515988in}}%
\pgfpathlineto{\pgfqpoint{5.766681in}{3.517555in}}%
\pgfpathlineto{\pgfqpoint{5.773973in}{3.519009in}}%
\pgfpathlineto{\pgfqpoint{5.758843in}{3.501763in}}%
\pgfpathlineto{\pgfqpoint{5.743737in}{3.484703in}}%
\pgfpathlineto{\pgfqpoint{5.728655in}{3.467831in}}%
\pgfpathlineto{\pgfqpoint{5.713596in}{3.451145in}}%
\pgfpathlineto{\pgfqpoint{5.706280in}{3.449264in}}%
\pgfpathlineto{\pgfqpoint{5.698954in}{3.447280in}}%
\pgfpathlineto{\pgfqpoint{5.691618in}{3.445187in}}%
\pgfpathlineto{\pgfqpoint{5.684272in}{3.442981in}}%
\pgfpathclose%
\pgfusepath{fill}%
\end{pgfscope}%
\begin{pgfscope}%
\pgfpathrectangle{\pgfqpoint{1.150000in}{0.150000in}}{\pgfqpoint{5.700000in}{5.700000in}}%
\pgfusepath{clip}%
\pgfsetbuttcap%
\pgfsetroundjoin%
\definecolor{currentfill}{rgb}{0.133743,0.548535,0.553541}%
\pgfsetfillcolor{currentfill}%
\pgfsetfillopacity{0.800000}%
\pgfsetlinewidth{0.000000pt}%
\definecolor{currentstroke}{rgb}{0.000000,0.000000,0.000000}%
\pgfsetstrokecolor{currentstroke}%
\pgfsetdash{}{0pt}%
\pgfpathmoveto{\pgfqpoint{2.017190in}{2.837380in}}%
\pgfpathlineto{\pgfqpoint{2.031793in}{2.808202in}}%
\pgfpathlineto{\pgfqpoint{2.046376in}{2.779383in}}%
\pgfpathlineto{\pgfqpoint{2.060939in}{2.750919in}}%
\pgfpathlineto{\pgfqpoint{2.075483in}{2.722806in}}%
\pgfpathlineto{\pgfqpoint{2.085083in}{2.712064in}}%
\pgfpathlineto{\pgfqpoint{2.094652in}{2.701777in}}%
\pgfpathlineto{\pgfqpoint{2.104192in}{2.691936in}}%
\pgfpathlineto{\pgfqpoint{2.113702in}{2.682534in}}%
\pgfpathlineto{\pgfqpoint{2.099235in}{2.709862in}}%
\pgfpathlineto{\pgfqpoint{2.084749in}{2.737539in}}%
\pgfpathlineto{\pgfqpoint{2.070244in}{2.765568in}}%
\pgfpathlineto{\pgfqpoint{2.055719in}{2.793953in}}%
\pgfpathlineto{\pgfqpoint{2.046133in}{2.804128in}}%
\pgfpathlineto{\pgfqpoint{2.036517in}{2.814752in}}%
\pgfpathlineto{\pgfqpoint{2.026869in}{2.825833in}}%
\pgfpathlineto{\pgfqpoint{2.017190in}{2.837380in}}%
\pgfpathclose%
\pgfusepath{fill}%
\end{pgfscope}%
\begin{pgfscope}%
\pgfpathrectangle{\pgfqpoint{1.150000in}{0.150000in}}{\pgfqpoint{5.700000in}{5.700000in}}%
\pgfusepath{clip}%
\pgfsetbuttcap%
\pgfsetroundjoin%
\definecolor{currentfill}{rgb}{0.282656,0.100196,0.422160}%
\pgfsetfillcolor{currentfill}%
\pgfsetfillopacity{0.800000}%
\pgfsetlinewidth{0.000000pt}%
\definecolor{currentstroke}{rgb}{0.000000,0.000000,0.000000}%
\pgfsetstrokecolor{currentstroke}%
\pgfsetdash{}{0pt}%
\pgfpathmoveto{\pgfqpoint{3.849510in}{1.517368in}}%
\pgfpathlineto{\pgfqpoint{3.863511in}{1.519728in}}%
\pgfpathlineto{\pgfqpoint{3.877521in}{1.522272in}}%
\pgfpathlineto{\pgfqpoint{3.891541in}{1.524999in}}%
\pgfpathlineto{\pgfqpoint{3.905570in}{1.527907in}}%
\pgfpathlineto{\pgfqpoint{3.913757in}{1.541214in}}%
\pgfpathlineto{\pgfqpoint{3.921938in}{1.554573in}}%
\pgfpathlineto{\pgfqpoint{3.930115in}{1.567978in}}%
\pgfpathlineto{\pgfqpoint{3.938287in}{1.581425in}}%
\pgfpathlineto{\pgfqpoint{3.924263in}{1.578010in}}%
\pgfpathlineto{\pgfqpoint{3.910249in}{1.574778in}}%
\pgfpathlineto{\pgfqpoint{3.896245in}{1.571728in}}%
\pgfpathlineto{\pgfqpoint{3.882251in}{1.568862in}}%
\pgfpathlineto{\pgfqpoint{3.874073in}{1.555909in}}%
\pgfpathlineto{\pgfqpoint{3.865891in}{1.543005in}}%
\pgfpathlineto{\pgfqpoint{3.857703in}{1.530157in}}%
\pgfpathlineto{\pgfqpoint{3.849510in}{1.517368in}}%
\pgfpathclose%
\pgfusepath{fill}%
\end{pgfscope}%
\begin{pgfscope}%
\pgfpathrectangle{\pgfqpoint{1.150000in}{0.150000in}}{\pgfqpoint{5.700000in}{5.700000in}}%
\pgfusepath{clip}%
\pgfsetbuttcap%
\pgfsetroundjoin%
\definecolor{currentfill}{rgb}{0.221989,0.339161,0.548752}%
\pgfsetfillcolor{currentfill}%
\pgfsetfillopacity{0.800000}%
\pgfsetlinewidth{0.000000pt}%
\definecolor{currentstroke}{rgb}{0.000000,0.000000,0.000000}%
\pgfsetstrokecolor{currentstroke}%
\pgfsetdash{}{0pt}%
\pgfpathmoveto{\pgfqpoint{4.391069in}{2.080670in}}%
\pgfpathlineto{\pgfqpoint{4.405303in}{2.089867in}}%
\pgfpathlineto{\pgfqpoint{4.419551in}{2.099248in}}%
\pgfpathlineto{\pgfqpoint{4.433815in}{2.108812in}}%
\pgfpathlineto{\pgfqpoint{4.448094in}{2.118560in}}%
\pgfpathlineto{\pgfqpoint{4.456135in}{2.132876in}}%
\pgfpathlineto{\pgfqpoint{4.464172in}{2.147090in}}%
\pgfpathlineto{\pgfqpoint{4.472203in}{2.161200in}}%
\pgfpathlineto{\pgfqpoint{4.480229in}{2.175204in}}%
\pgfpathlineto{\pgfqpoint{4.465948in}{2.165201in}}%
\pgfpathlineto{\pgfqpoint{4.451682in}{2.155381in}}%
\pgfpathlineto{\pgfqpoint{4.437431in}{2.145746in}}%
\pgfpathlineto{\pgfqpoint{4.423196in}{2.136294in}}%
\pgfpathlineto{\pgfqpoint{4.415171in}{2.122533in}}%
\pgfpathlineto{\pgfqpoint{4.407142in}{2.108674in}}%
\pgfpathlineto{\pgfqpoint{4.399108in}{2.094719in}}%
\pgfpathlineto{\pgfqpoint{4.391069in}{2.080670in}}%
\pgfpathclose%
\pgfusepath{fill}%
\end{pgfscope}%
\begin{pgfscope}%
\pgfpathrectangle{\pgfqpoint{1.150000in}{0.150000in}}{\pgfqpoint{5.700000in}{5.700000in}}%
\pgfusepath{clip}%
\pgfsetbuttcap%
\pgfsetroundjoin%
\definecolor{currentfill}{rgb}{0.122312,0.633153,0.530398}%
\pgfsetfillcolor{currentfill}%
\pgfsetfillopacity{0.800000}%
\pgfsetlinewidth{0.000000pt}%
\definecolor{currentstroke}{rgb}{0.000000,0.000000,0.000000}%
\pgfsetstrokecolor{currentstroke}%
\pgfsetdash{}{0pt}%
\pgfpathmoveto{\pgfqpoint{5.175231in}{2.979337in}}%
\pgfpathlineto{\pgfqpoint{5.189963in}{2.994556in}}%
\pgfpathlineto{\pgfqpoint{5.204716in}{3.009963in}}%
\pgfpathlineto{\pgfqpoint{5.219490in}{3.025557in}}%
\pgfpathlineto{\pgfqpoint{5.234286in}{3.041339in}}%
\pgfpathlineto{\pgfqpoint{5.241978in}{3.048803in}}%
\pgfpathlineto{\pgfqpoint{5.249661in}{3.056098in}}%
\pgfpathlineto{\pgfqpoint{5.257334in}{3.063225in}}%
\pgfpathlineto{\pgfqpoint{5.264997in}{3.070186in}}%
\pgfpathlineto{\pgfqpoint{5.250210in}{3.054564in}}%
\pgfpathlineto{\pgfqpoint{5.235443in}{3.039128in}}%
\pgfpathlineto{\pgfqpoint{5.220698in}{3.023880in}}%
\pgfpathlineto{\pgfqpoint{5.205974in}{3.008818in}}%
\pgfpathlineto{\pgfqpoint{5.198302in}{3.001685in}}%
\pgfpathlineto{\pgfqpoint{5.190621in}{2.994395in}}%
\pgfpathlineto{\pgfqpoint{5.182930in}{2.986946in}}%
\pgfpathlineto{\pgfqpoint{5.175231in}{2.979337in}}%
\pgfpathclose%
\pgfusepath{fill}%
\end{pgfscope}%
\begin{pgfscope}%
\pgfpathrectangle{\pgfqpoint{1.150000in}{0.150000in}}{\pgfqpoint{5.700000in}{5.700000in}}%
\pgfusepath{clip}%
\pgfsetbuttcap%
\pgfsetroundjoin%
\definecolor{currentfill}{rgb}{0.268510,0.009605,0.335427}%
\pgfsetfillcolor{currentfill}%
\pgfsetfillopacity{0.800000}%
\pgfsetlinewidth{0.000000pt}%
\definecolor{currentstroke}{rgb}{0.000000,0.000000,0.000000}%
\pgfsetstrokecolor{currentstroke}%
\pgfsetdash{}{0pt}%
\pgfpathmoveto{\pgfqpoint{3.493972in}{1.349662in}}%
\pgfpathlineto{\pgfqpoint{3.507909in}{1.346731in}}%
\pgfpathlineto{\pgfqpoint{3.521851in}{1.343987in}}%
\pgfpathlineto{\pgfqpoint{3.535798in}{1.341431in}}%
\pgfpathlineto{\pgfqpoint{3.549751in}{1.339062in}}%
\pgfpathlineto{\pgfqpoint{3.558082in}{1.348254in}}%
\pgfpathlineto{\pgfqpoint{3.566405in}{1.357618in}}%
\pgfpathlineto{\pgfqpoint{3.574720in}{1.367147in}}%
\pgfpathlineto{\pgfqpoint{3.583027in}{1.376835in}}%
\pgfpathlineto{\pgfqpoint{3.569092in}{1.378575in}}%
\pgfpathlineto{\pgfqpoint{3.555163in}{1.380503in}}%
\pgfpathlineto{\pgfqpoint{3.541239in}{1.382618in}}%
\pgfpathlineto{\pgfqpoint{3.527321in}{1.384921in}}%
\pgfpathlineto{\pgfqpoint{3.518996in}{1.375849in}}%
\pgfpathlineto{\pgfqpoint{3.510663in}{1.366945in}}%
\pgfpathlineto{\pgfqpoint{3.502322in}{1.358214in}}%
\pgfpathlineto{\pgfqpoint{3.493972in}{1.349662in}}%
\pgfpathclose%
\pgfusepath{fill}%
\end{pgfscope}%
\begin{pgfscope}%
\pgfpathrectangle{\pgfqpoint{1.150000in}{0.150000in}}{\pgfqpoint{5.700000in}{5.700000in}}%
\pgfusepath{clip}%
\pgfsetbuttcap%
\pgfsetroundjoin%
\definecolor{currentfill}{rgb}{0.246811,0.283237,0.535941}%
\pgfsetfillcolor{currentfill}%
\pgfsetfillopacity{0.800000}%
\pgfsetlinewidth{0.000000pt}%
\definecolor{currentstroke}{rgb}{0.000000,0.000000,0.000000}%
\pgfsetstrokecolor{currentstroke}%
\pgfsetdash{}{0pt}%
\pgfpathmoveto{\pgfqpoint{4.269812in}{1.932676in}}%
\pgfpathlineto{\pgfqpoint{4.283985in}{1.940533in}}%
\pgfpathlineto{\pgfqpoint{4.298172in}{1.948574in}}%
\pgfpathlineto{\pgfqpoint{4.312373in}{1.956797in}}%
\pgfpathlineto{\pgfqpoint{4.326588in}{1.965203in}}%
\pgfpathlineto{\pgfqpoint{4.334664in}{1.979911in}}%
\pgfpathlineto{\pgfqpoint{4.342736in}{1.994547in}}%
\pgfpathlineto{\pgfqpoint{4.350803in}{2.009106in}}%
\pgfpathlineto{\pgfqpoint{4.358865in}{2.023587in}}%
\pgfpathlineto{\pgfqpoint{4.344648in}{2.014861in}}%
\pgfpathlineto{\pgfqpoint{4.330446in}{2.006318in}}%
\pgfpathlineto{\pgfqpoint{4.316258in}{1.997958in}}%
\pgfpathlineto{\pgfqpoint{4.302084in}{1.989782in}}%
\pgfpathlineto{\pgfqpoint{4.294022in}{1.975608in}}%
\pgfpathlineto{\pgfqpoint{4.285957in}{1.961364in}}%
\pgfpathlineto{\pgfqpoint{4.277887in}{1.947053in}}%
\pgfpathlineto{\pgfqpoint{4.269812in}{1.932676in}}%
\pgfpathclose%
\pgfusepath{fill}%
\end{pgfscope}%
\begin{pgfscope}%
\pgfpathrectangle{\pgfqpoint{1.150000in}{0.150000in}}{\pgfqpoint{5.700000in}{5.700000in}}%
\pgfusepath{clip}%
\pgfsetbuttcap%
\pgfsetroundjoin%
\definecolor{currentfill}{rgb}{0.144759,0.519093,0.556572}%
\pgfsetfillcolor{currentfill}%
\pgfsetfillopacity{0.800000}%
\pgfsetlinewidth{0.000000pt}%
\definecolor{currentstroke}{rgb}{0.000000,0.000000,0.000000}%
\pgfsetstrokecolor{currentstroke}%
\pgfsetdash{}{0pt}%
\pgfpathmoveto{\pgfqpoint{4.843935in}{2.619904in}}%
\pgfpathlineto{\pgfqpoint{4.858448in}{2.633142in}}%
\pgfpathlineto{\pgfqpoint{4.872979in}{2.646567in}}%
\pgfpathlineto{\pgfqpoint{4.887529in}{2.660178in}}%
\pgfpathlineto{\pgfqpoint{4.902099in}{2.673975in}}%
\pgfpathlineto{\pgfqpoint{4.909974in}{2.685023in}}%
\pgfpathlineto{\pgfqpoint{4.917842in}{2.695905in}}%
\pgfpathlineto{\pgfqpoint{4.925703in}{2.706623in}}%
\pgfpathlineto{\pgfqpoint{4.933556in}{2.717177in}}%
\pgfpathlineto{\pgfqpoint{4.918987in}{2.703361in}}%
\pgfpathlineto{\pgfqpoint{4.904438in}{2.689732in}}%
\pgfpathlineto{\pgfqpoint{4.889908in}{2.676288in}}%
\pgfpathlineto{\pgfqpoint{4.875397in}{2.663031in}}%
\pgfpathlineto{\pgfqpoint{4.867543in}{2.652483in}}%
\pgfpathlineto{\pgfqpoint{4.859681in}{2.641779in}}%
\pgfpathlineto{\pgfqpoint{4.851812in}{2.630920in}}%
\pgfpathlineto{\pgfqpoint{4.843935in}{2.619904in}}%
\pgfpathclose%
\pgfusepath{fill}%
\end{pgfscope}%
\begin{pgfscope}%
\pgfpathrectangle{\pgfqpoint{1.150000in}{0.150000in}}{\pgfqpoint{5.700000in}{5.700000in}}%
\pgfusepath{clip}%
\pgfsetbuttcap%
\pgfsetroundjoin%
\definecolor{currentfill}{rgb}{0.276022,0.044167,0.370164}%
\pgfsetfillcolor{currentfill}%
\pgfsetfillopacity{0.800000}%
\pgfsetlinewidth{0.000000pt}%
\definecolor{currentstroke}{rgb}{0.000000,0.000000,0.000000}%
\pgfsetstrokecolor{currentstroke}%
\pgfsetdash{}{0pt}%
\pgfpathmoveto{\pgfqpoint{3.057606in}{1.441708in}}%
\pgfpathlineto{\pgfqpoint{3.071574in}{1.431987in}}%
\pgfpathlineto{\pgfqpoint{3.085540in}{1.422470in}}%
\pgfpathlineto{\pgfqpoint{3.099507in}{1.413158in}}%
\pgfpathlineto{\pgfqpoint{3.113473in}{1.404050in}}%
\pgfpathlineto{\pgfqpoint{3.122095in}{1.406233in}}%
\pgfpathlineto{\pgfqpoint{3.130704in}{1.408719in}}%
\pgfpathlineto{\pgfqpoint{3.139298in}{1.411501in}}%
\pgfpathlineto{\pgfqpoint{3.147878in}{1.414571in}}%
\pgfpathlineto{\pgfqpoint{3.133948in}{1.422949in}}%
\pgfpathlineto{\pgfqpoint{3.120018in}{1.431530in}}%
\pgfpathlineto{\pgfqpoint{3.106088in}{1.440315in}}%
\pgfpathlineto{\pgfqpoint{3.092158in}{1.449305in}}%
\pgfpathlineto{\pgfqpoint{3.083542in}{1.446953in}}%
\pgfpathlineto{\pgfqpoint{3.074912in}{1.444898in}}%
\pgfpathlineto{\pgfqpoint{3.066267in}{1.443147in}}%
\pgfpathlineto{\pgfqpoint{3.057606in}{1.441708in}}%
\pgfpathclose%
\pgfusepath{fill}%
\end{pgfscope}%
\begin{pgfscope}%
\pgfpathrectangle{\pgfqpoint{1.150000in}{0.150000in}}{\pgfqpoint{5.700000in}{5.700000in}}%
\pgfusepath{clip}%
\pgfsetbuttcap%
\pgfsetroundjoin%
\definecolor{currentfill}{rgb}{0.282884,0.135920,0.453427}%
\pgfsetfillcolor{currentfill}%
\pgfsetfillopacity{0.800000}%
\pgfsetlinewidth{0.000000pt}%
\definecolor{currentstroke}{rgb}{0.000000,0.000000,0.000000}%
\pgfsetstrokecolor{currentstroke}%
\pgfsetdash{}{0pt}%
\pgfpathmoveto{\pgfqpoint{3.938287in}{1.581425in}}%
\pgfpathlineto{\pgfqpoint{3.952321in}{1.585023in}}%
\pgfpathlineto{\pgfqpoint{3.966366in}{1.588804in}}%
\pgfpathlineto{\pgfqpoint{3.980421in}{1.592766in}}%
\pgfpathlineto{\pgfqpoint{3.994487in}{1.596910in}}%
\pgfpathlineto{\pgfqpoint{4.002650in}{1.610883in}}%
\pgfpathlineto{\pgfqpoint{4.010809in}{1.624880in}}%
\pgfpathlineto{\pgfqpoint{4.018963in}{1.638898in}}%
\pgfpathlineto{\pgfqpoint{4.027113in}{1.652933in}}%
\pgfpathlineto{\pgfqpoint{4.013050in}{1.648312in}}%
\pgfpathlineto{\pgfqpoint{3.998998in}{1.643873in}}%
\pgfpathlineto{\pgfqpoint{3.984957in}{1.639617in}}%
\pgfpathlineto{\pgfqpoint{3.970927in}{1.635544in}}%
\pgfpathlineto{\pgfqpoint{3.962774in}{1.621974in}}%
\pgfpathlineto{\pgfqpoint{3.954616in}{1.608428in}}%
\pgfpathlineto{\pgfqpoint{3.946454in}{1.594910in}}%
\pgfpathlineto{\pgfqpoint{3.938287in}{1.581425in}}%
\pgfpathclose%
\pgfusepath{fill}%
\end{pgfscope}%
\begin{pgfscope}%
\pgfpathrectangle{\pgfqpoint{1.150000in}{0.150000in}}{\pgfqpoint{5.700000in}{5.700000in}}%
\pgfusepath{clip}%
\pgfsetbuttcap%
\pgfsetroundjoin%
\definecolor{currentfill}{rgb}{0.377779,0.791781,0.377939}%
\pgfsetfillcolor{currentfill}%
\pgfsetfillopacity{0.800000}%
\pgfsetlinewidth{0.000000pt}%
\definecolor{currentstroke}{rgb}{0.000000,0.000000,0.000000}%
\pgfsetstrokecolor{currentstroke}%
\pgfsetdash{}{0pt}%
\pgfpathmoveto{\pgfqpoint{5.773973in}{3.519009in}}%
\pgfpathlineto{\pgfqpoint{5.789127in}{3.536444in}}%
\pgfpathlineto{\pgfqpoint{5.804306in}{3.554066in}}%
\pgfpathlineto{\pgfqpoint{5.819509in}{3.571876in}}%
\pgfpathlineto{\pgfqpoint{5.834736in}{3.589875in}}%
\pgfpathlineto{\pgfqpoint{5.841992in}{3.590783in}}%
\pgfpathlineto{\pgfqpoint{5.849237in}{3.591581in}}%
\pgfpathlineto{\pgfqpoint{5.856471in}{3.592271in}}%
\pgfpathlineto{\pgfqpoint{5.863694in}{3.592861in}}%
\pgfpathlineto{\pgfqpoint{5.848493in}{3.575315in}}%
\pgfpathlineto{\pgfqpoint{5.833316in}{3.557956in}}%
\pgfpathlineto{\pgfqpoint{5.818163in}{3.540784in}}%
\pgfpathlineto{\pgfqpoint{5.803035in}{3.523799in}}%
\pgfpathlineto{\pgfqpoint{5.795785in}{3.522746in}}%
\pgfpathlineto{\pgfqpoint{5.788525in}{3.521600in}}%
\pgfpathlineto{\pgfqpoint{5.781254in}{3.520356in}}%
\pgfpathlineto{\pgfqpoint{5.773973in}{3.519009in}}%
\pgfpathclose%
\pgfusepath{fill}%
\end{pgfscope}%
\begin{pgfscope}%
\pgfpathrectangle{\pgfqpoint{1.150000in}{0.150000in}}{\pgfqpoint{5.700000in}{5.700000in}}%
\pgfusepath{clip}%
\pgfsetbuttcap%
\pgfsetroundjoin%
\definecolor{currentfill}{rgb}{0.268510,0.009605,0.335427}%
\pgfsetfillcolor{currentfill}%
\pgfsetfillopacity{0.800000}%
\pgfsetlinewidth{0.000000pt}%
\definecolor{currentstroke}{rgb}{0.000000,0.000000,0.000000}%
\pgfsetstrokecolor{currentstroke}%
\pgfsetdash{}{0pt}%
\pgfpathmoveto{\pgfqpoint{3.259360in}{1.354763in}}%
\pgfpathlineto{\pgfqpoint{3.273302in}{1.348178in}}%
\pgfpathlineto{\pgfqpoint{3.287247in}{1.341787in}}%
\pgfpathlineto{\pgfqpoint{3.301194in}{1.335591in}}%
\pgfpathlineto{\pgfqpoint{3.315144in}{1.329590in}}%
\pgfpathlineto{\pgfqpoint{3.323619in}{1.335062in}}%
\pgfpathlineto{\pgfqpoint{3.332083in}{1.340782in}}%
\pgfpathlineto{\pgfqpoint{3.340535in}{1.346745in}}%
\pgfpathlineto{\pgfqpoint{3.348978in}{1.352942in}}%
\pgfpathlineto{\pgfqpoint{3.335055in}{1.358250in}}%
\pgfpathlineto{\pgfqpoint{3.321136in}{1.363753in}}%
\pgfpathlineto{\pgfqpoint{3.307219in}{1.369449in}}%
\pgfpathlineto{\pgfqpoint{3.293305in}{1.375341in}}%
\pgfpathlineto{\pgfqpoint{3.284836in}{1.369825in}}%
\pgfpathlineto{\pgfqpoint{3.276356in}{1.364552in}}%
\pgfpathlineto{\pgfqpoint{3.267864in}{1.359529in}}%
\pgfpathlineto{\pgfqpoint{3.259360in}{1.354763in}}%
\pgfpathclose%
\pgfusepath{fill}%
\end{pgfscope}%
\begin{pgfscope}%
\pgfpathrectangle{\pgfqpoint{1.150000in}{0.150000in}}{\pgfqpoint{5.700000in}{5.700000in}}%
\pgfusepath{clip}%
\pgfsetbuttcap%
\pgfsetroundjoin%
\definecolor{currentfill}{rgb}{0.273006,0.204520,0.501721}%
\pgfsetfillcolor{currentfill}%
\pgfsetfillopacity{0.800000}%
\pgfsetlinewidth{0.000000pt}%
\definecolor{currentstroke}{rgb}{0.000000,0.000000,0.000000}%
\pgfsetstrokecolor{currentstroke}%
\pgfsetdash{}{0pt}%
\pgfpathmoveto{\pgfqpoint{2.629400in}{1.821016in}}%
\pgfpathlineto{\pgfqpoint{2.643517in}{1.804291in}}%
\pgfpathlineto{\pgfqpoint{2.657626in}{1.787807in}}%
\pgfpathlineto{\pgfqpoint{2.671729in}{1.771563in}}%
\pgfpathlineto{\pgfqpoint{2.685826in}{1.755557in}}%
\pgfpathlineto{\pgfqpoint{2.694841in}{1.750936in}}%
\pgfpathlineto{\pgfqpoint{2.703834in}{1.746714in}}%
\pgfpathlineto{\pgfqpoint{2.712807in}{1.742881in}}%
\pgfpathlineto{\pgfqpoint{2.721758in}{1.739431in}}%
\pgfpathlineto{\pgfqpoint{2.707715in}{1.754650in}}%
\pgfpathlineto{\pgfqpoint{2.693667in}{1.770106in}}%
\pgfpathlineto{\pgfqpoint{2.679613in}{1.785801in}}%
\pgfpathlineto{\pgfqpoint{2.665553in}{1.801735in}}%
\pgfpathlineto{\pgfqpoint{2.656548in}{1.805960in}}%
\pgfpathlineto{\pgfqpoint{2.647521in}{1.810576in}}%
\pgfpathlineto{\pgfqpoint{2.638472in}{1.815592in}}%
\pgfpathlineto{\pgfqpoint{2.629400in}{1.821016in}}%
\pgfpathclose%
\pgfusepath{fill}%
\end{pgfscope}%
\begin{pgfscope}%
\pgfpathrectangle{\pgfqpoint{1.150000in}{0.150000in}}{\pgfqpoint{5.700000in}{5.700000in}}%
\pgfusepath{clip}%
\pgfsetbuttcap%
\pgfsetroundjoin%
\definecolor{currentfill}{rgb}{0.265145,0.232956,0.516599}%
\pgfsetfillcolor{currentfill}%
\pgfsetfillopacity{0.800000}%
\pgfsetlinewidth{0.000000pt}%
\definecolor{currentstroke}{rgb}{0.000000,0.000000,0.000000}%
\pgfsetstrokecolor{currentstroke}%
\pgfsetdash{}{0pt}%
\pgfpathmoveto{\pgfqpoint{2.572862in}{1.890361in}}%
\pgfpathlineto{\pgfqpoint{2.587008in}{1.872654in}}%
\pgfpathlineto{\pgfqpoint{2.601146in}{1.855196in}}%
\pgfpathlineto{\pgfqpoint{2.615277in}{1.837983in}}%
\pgfpathlineto{\pgfqpoint{2.629400in}{1.821016in}}%
\pgfpathlineto{\pgfqpoint{2.638472in}{1.815592in}}%
\pgfpathlineto{\pgfqpoint{2.647521in}{1.810576in}}%
\pgfpathlineto{\pgfqpoint{2.656548in}{1.805960in}}%
\pgfpathlineto{\pgfqpoint{2.665553in}{1.801735in}}%
\pgfpathlineto{\pgfqpoint{2.651486in}{1.817912in}}%
\pgfpathlineto{\pgfqpoint{2.637413in}{1.834332in}}%
\pgfpathlineto{\pgfqpoint{2.623332in}{1.850996in}}%
\pgfpathlineto{\pgfqpoint{2.609245in}{1.867908in}}%
\pgfpathlineto{\pgfqpoint{2.600184in}{1.872911in}}%
\pgfpathlineto{\pgfqpoint{2.591100in}{1.878315in}}%
\pgfpathlineto{\pgfqpoint{2.581993in}{1.884129in}}%
\pgfpathlineto{\pgfqpoint{2.572862in}{1.890361in}}%
\pgfpathclose%
\pgfusepath{fill}%
\end{pgfscope}%
\begin{pgfscope}%
\pgfpathrectangle{\pgfqpoint{1.150000in}{0.150000in}}{\pgfqpoint{5.700000in}{5.700000in}}%
\pgfusepath{clip}%
\pgfsetbuttcap%
\pgfsetroundjoin%
\definecolor{currentfill}{rgb}{0.266580,0.228262,0.514349}%
\pgfsetfillcolor{currentfill}%
\pgfsetfillopacity{0.800000}%
\pgfsetlinewidth{0.000000pt}%
\definecolor{currentstroke}{rgb}{0.000000,0.000000,0.000000}%
\pgfsetstrokecolor{currentstroke}%
\pgfsetdash{}{0pt}%
\pgfpathmoveto{\pgfqpoint{4.148505in}{1.789029in}}%
\pgfpathlineto{\pgfqpoint{4.162624in}{1.795421in}}%
\pgfpathlineto{\pgfqpoint{4.176755in}{1.801995in}}%
\pgfpathlineto{\pgfqpoint{4.190900in}{1.808752in}}%
\pgfpathlineto{\pgfqpoint{4.205057in}{1.815691in}}%
\pgfpathlineto{\pgfqpoint{4.213167in}{1.830476in}}%
\pgfpathlineto{\pgfqpoint{4.221272in}{1.845221in}}%
\pgfpathlineto{\pgfqpoint{4.229373in}{1.859923in}}%
\pgfpathlineto{\pgfqpoint{4.237470in}{1.874579in}}%
\pgfpathlineto{\pgfqpoint{4.223311in}{1.867256in}}%
\pgfpathlineto{\pgfqpoint{4.209166in}{1.860116in}}%
\pgfpathlineto{\pgfqpoint{4.195034in}{1.853159in}}%
\pgfpathlineto{\pgfqpoint{4.180915in}{1.846384in}}%
\pgfpathlineto{\pgfqpoint{4.172819in}{1.832099in}}%
\pgfpathlineto{\pgfqpoint{4.164719in}{1.817776in}}%
\pgfpathlineto{\pgfqpoint{4.156614in}{1.803418in}}%
\pgfpathlineto{\pgfqpoint{4.148505in}{1.789029in}}%
\pgfpathclose%
\pgfusepath{fill}%
\end{pgfscope}%
\begin{pgfscope}%
\pgfpathrectangle{\pgfqpoint{1.150000in}{0.150000in}}{\pgfqpoint{5.700000in}{5.700000in}}%
\pgfusepath{clip}%
\pgfsetbuttcap%
\pgfsetroundjoin%
\definecolor{currentfill}{rgb}{0.168126,0.459988,0.558082}%
\pgfsetfillcolor{currentfill}%
\pgfsetfillopacity{0.800000}%
\pgfsetlinewidth{0.000000pt}%
\definecolor{currentstroke}{rgb}{0.000000,0.000000,0.000000}%
\pgfsetstrokecolor{currentstroke}%
\pgfsetdash{}{0pt}%
\pgfpathmoveto{\pgfqpoint{2.153084in}{2.551269in}}%
\pgfpathlineto{\pgfqpoint{2.167544in}{2.525357in}}%
\pgfpathlineto{\pgfqpoint{2.181988in}{2.499765in}}%
\pgfpathlineto{\pgfqpoint{2.196416in}{2.474491in}}%
\pgfpathlineto{\pgfqpoint{2.210828in}{2.449531in}}%
\pgfpathlineto{\pgfqpoint{2.220321in}{2.439377in}}%
\pgfpathlineto{\pgfqpoint{2.229785in}{2.429679in}}%
\pgfpathlineto{\pgfqpoint{2.239220in}{2.420428in}}%
\pgfpathlineto{\pgfqpoint{2.248627in}{2.411616in}}%
\pgfpathlineto{\pgfqpoint{2.234287in}{2.435774in}}%
\pgfpathlineto{\pgfqpoint{2.219933in}{2.460244in}}%
\pgfpathlineto{\pgfqpoint{2.205563in}{2.485030in}}%
\pgfpathlineto{\pgfqpoint{2.191177in}{2.510134in}}%
\pgfpathlineto{\pgfqpoint{2.181698in}{2.519735in}}%
\pgfpathlineto{\pgfqpoint{2.172190in}{2.529786in}}%
\pgfpathlineto{\pgfqpoint{2.162652in}{2.540294in}}%
\pgfpathlineto{\pgfqpoint{2.153084in}{2.551269in}}%
\pgfpathclose%
\pgfusepath{fill}%
\end{pgfscope}%
\begin{pgfscope}%
\pgfpathrectangle{\pgfqpoint{1.150000in}{0.150000in}}{\pgfqpoint{5.700000in}{5.700000in}}%
\pgfusepath{clip}%
\pgfsetbuttcap%
\pgfsetroundjoin%
\definecolor{currentfill}{rgb}{0.278012,0.180367,0.486697}%
\pgfsetfillcolor{currentfill}%
\pgfsetfillopacity{0.800000}%
\pgfsetlinewidth{0.000000pt}%
\definecolor{currentstroke}{rgb}{0.000000,0.000000,0.000000}%
\pgfsetstrokecolor{currentstroke}%
\pgfsetdash{}{0pt}%
\pgfpathmoveto{\pgfqpoint{2.685826in}{1.755557in}}%
\pgfpathlineto{\pgfqpoint{2.699917in}{1.739788in}}%
\pgfpathlineto{\pgfqpoint{2.714001in}{1.724254in}}%
\pgfpathlineto{\pgfqpoint{2.728080in}{1.708953in}}%
\pgfpathlineto{\pgfqpoint{2.742154in}{1.693885in}}%
\pgfpathlineto{\pgfqpoint{2.751115in}{1.690062in}}%
\pgfpathlineto{\pgfqpoint{2.760055in}{1.686628in}}%
\pgfpathlineto{\pgfqpoint{2.768974in}{1.683575in}}%
\pgfpathlineto{\pgfqpoint{2.777874in}{1.680894in}}%
\pgfpathlineto{\pgfqpoint{2.763853in}{1.695180in}}%
\pgfpathlineto{\pgfqpoint{2.749826in}{1.709697in}}%
\pgfpathlineto{\pgfqpoint{2.735795in}{1.724447in}}%
\pgfpathlineto{\pgfqpoint{2.721758in}{1.739431in}}%
\pgfpathlineto{\pgfqpoint{2.712807in}{1.742881in}}%
\pgfpathlineto{\pgfqpoint{2.703834in}{1.746714in}}%
\pgfpathlineto{\pgfqpoint{2.694841in}{1.750936in}}%
\pgfpathlineto{\pgfqpoint{2.685826in}{1.755557in}}%
\pgfpathclose%
\pgfusepath{fill}%
\end{pgfscope}%
\begin{pgfscope}%
\pgfpathrectangle{\pgfqpoint{1.150000in}{0.150000in}}{\pgfqpoint{5.700000in}{5.700000in}}%
\pgfusepath{clip}%
\pgfsetbuttcap%
\pgfsetroundjoin%
\definecolor{currentfill}{rgb}{0.267004,0.004874,0.329415}%
\pgfsetfillcolor{currentfill}%
\pgfsetfillopacity{0.800000}%
\pgfsetlinewidth{0.000000pt}%
\definecolor{currentstroke}{rgb}{0.000000,0.000000,0.000000}%
\pgfsetstrokecolor{currentstroke}%
\pgfsetdash{}{0pt}%
\pgfpathmoveto{\pgfqpoint{3.404701in}{1.333633in}}%
\pgfpathlineto{\pgfqpoint{3.418642in}{1.329284in}}%
\pgfpathlineto{\pgfqpoint{3.432586in}{1.325125in}}%
\pgfpathlineto{\pgfqpoint{3.446534in}{1.321156in}}%
\pgfpathlineto{\pgfqpoint{3.460487in}{1.317376in}}%
\pgfpathlineto{\pgfqpoint{3.468872in}{1.325146in}}%
\pgfpathlineto{\pgfqpoint{3.477248in}{1.333122in}}%
\pgfpathlineto{\pgfqpoint{3.485614in}{1.341296in}}%
\pgfpathlineto{\pgfqpoint{3.493972in}{1.349662in}}%
\pgfpathlineto{\pgfqpoint{3.480041in}{1.352782in}}%
\pgfpathlineto{\pgfqpoint{3.466114in}{1.356091in}}%
\pgfpathlineto{\pgfqpoint{3.452192in}{1.359589in}}%
\pgfpathlineto{\pgfqpoint{3.438274in}{1.363278in}}%
\pgfpathlineto{\pgfqpoint{3.429895in}{1.355560in}}%
\pgfpathlineto{\pgfqpoint{3.421507in}{1.348042in}}%
\pgfpathlineto{\pgfqpoint{3.413109in}{1.340731in}}%
\pgfpathlineto{\pgfqpoint{3.404701in}{1.333633in}}%
\pgfpathclose%
\pgfusepath{fill}%
\end{pgfscope}%
\begin{pgfscope}%
\pgfpathrectangle{\pgfqpoint{1.150000in}{0.150000in}}{\pgfqpoint{5.700000in}{5.700000in}}%
\pgfusepath{clip}%
\pgfsetbuttcap%
\pgfsetroundjoin%
\definecolor{currentfill}{rgb}{0.430983,0.808473,0.346476}%
\pgfsetfillcolor{currentfill}%
\pgfsetfillopacity{0.800000}%
\pgfsetlinewidth{0.000000pt}%
\definecolor{currentstroke}{rgb}{0.000000,0.000000,0.000000}%
\pgfsetstrokecolor{currentstroke}%
\pgfsetdash{}{0pt}%
\pgfpathmoveto{\pgfqpoint{5.863694in}{3.592861in}}%
\pgfpathlineto{\pgfqpoint{5.878920in}{3.610595in}}%
\pgfpathlineto{\pgfqpoint{5.894171in}{3.628517in}}%
\pgfpathlineto{\pgfqpoint{5.909447in}{3.646628in}}%
\pgfpathlineto{\pgfqpoint{5.916638in}{3.646764in}}%
\pgfpathlineto{\pgfqpoint{5.923819in}{3.646804in}}%
\pgfpathlineto{\pgfqpoint{5.930989in}{3.646752in}}%
\pgfpathlineto{\pgfqpoint{5.938149in}{3.646613in}}%
\pgfpathlineto{\pgfqpoint{5.922902in}{3.628992in}}%
\pgfpathlineto{\pgfqpoint{5.907680in}{3.611558in}}%
\pgfpathlineto{\pgfqpoint{5.892483in}{3.594311in}}%
\pgfpathlineto{\pgfqpoint{5.885301in}{3.594074in}}%
\pgfpathlineto{\pgfqpoint{5.878109in}{3.593757in}}%
\pgfpathlineto{\pgfqpoint{5.870907in}{3.593355in}}%
\pgfpathlineto{\pgfqpoint{5.863694in}{3.592861in}}%
\pgfpathclose%
\pgfusepath{fill}%
\end{pgfscope}%
\begin{pgfscope}%
\pgfpathrectangle{\pgfqpoint{1.150000in}{0.150000in}}{\pgfqpoint{5.700000in}{5.700000in}}%
\pgfusepath{clip}%
\pgfsetbuttcap%
\pgfsetroundjoin%
\definecolor{currentfill}{rgb}{0.255645,0.260703,0.528312}%
\pgfsetfillcolor{currentfill}%
\pgfsetfillopacity{0.800000}%
\pgfsetlinewidth{0.000000pt}%
\definecolor{currentstroke}{rgb}{0.000000,0.000000,0.000000}%
\pgfsetstrokecolor{currentstroke}%
\pgfsetdash{}{0pt}%
\pgfpathmoveto{\pgfqpoint{2.516195in}{1.963701in}}%
\pgfpathlineto{\pgfqpoint{2.530375in}{1.944985in}}%
\pgfpathlineto{\pgfqpoint{2.544546in}{1.926524in}}%
\pgfpathlineto{\pgfqpoint{2.558708in}{1.908317in}}%
\pgfpathlineto{\pgfqpoint{2.572862in}{1.890361in}}%
\pgfpathlineto{\pgfqpoint{2.581993in}{1.884129in}}%
\pgfpathlineto{\pgfqpoint{2.591100in}{1.878315in}}%
\pgfpathlineto{\pgfqpoint{2.600184in}{1.872911in}}%
\pgfpathlineto{\pgfqpoint{2.609245in}{1.867908in}}%
\pgfpathlineto{\pgfqpoint{2.595150in}{1.885068in}}%
\pgfpathlineto{\pgfqpoint{2.581047in}{1.902478in}}%
\pgfpathlineto{\pgfqpoint{2.566937in}{1.920140in}}%
\pgfpathlineto{\pgfqpoint{2.552818in}{1.938056in}}%
\pgfpathlineto{\pgfqpoint{2.543698in}{1.943843in}}%
\pgfpathlineto{\pgfqpoint{2.534555in}{1.950040in}}%
\pgfpathlineto{\pgfqpoint{2.525388in}{1.956657in}}%
\pgfpathlineto{\pgfqpoint{2.516195in}{1.963701in}}%
\pgfpathclose%
\pgfusepath{fill}%
\end{pgfscope}%
\begin{pgfscope}%
\pgfpathrectangle{\pgfqpoint{1.150000in}{0.150000in}}{\pgfqpoint{5.700000in}{5.700000in}}%
\pgfusepath{clip}%
\pgfsetbuttcap%
\pgfsetroundjoin%
\definecolor{currentfill}{rgb}{0.162142,0.474838,0.558140}%
\pgfsetfillcolor{currentfill}%
\pgfsetfillopacity{0.800000}%
\pgfsetlinewidth{0.000000pt}%
\definecolor{currentstroke}{rgb}{0.000000,0.000000,0.000000}%
\pgfsetstrokecolor{currentstroke}%
\pgfsetdash{}{0pt}%
\pgfpathmoveto{\pgfqpoint{4.722803in}{2.475625in}}%
\pgfpathlineto{\pgfqpoint{4.737242in}{2.487980in}}%
\pgfpathlineto{\pgfqpoint{4.751699in}{2.500520in}}%
\pgfpathlineto{\pgfqpoint{4.766174in}{2.513245in}}%
\pgfpathlineto{\pgfqpoint{4.780668in}{2.526157in}}%
\pgfpathlineto{\pgfqpoint{4.788600in}{2.538420in}}%
\pgfpathlineto{\pgfqpoint{4.796526in}{2.550528in}}%
\pgfpathlineto{\pgfqpoint{4.804445in}{2.562481in}}%
\pgfpathlineto{\pgfqpoint{4.812357in}{2.574278in}}%
\pgfpathlineto{\pgfqpoint{4.797863in}{2.561278in}}%
\pgfpathlineto{\pgfqpoint{4.783387in}{2.548465in}}%
\pgfpathlineto{\pgfqpoint{4.768930in}{2.535837in}}%
\pgfpathlineto{\pgfqpoint{4.754490in}{2.523394in}}%
\pgfpathlineto{\pgfqpoint{4.746578in}{2.511672in}}%
\pgfpathlineto{\pgfqpoint{4.738659in}{2.499803in}}%
\pgfpathlineto{\pgfqpoint{4.730734in}{2.487787in}}%
\pgfpathlineto{\pgfqpoint{4.722803in}{2.475625in}}%
\pgfpathclose%
\pgfusepath{fill}%
\end{pgfscope}%
\begin{pgfscope}%
\pgfpathrectangle{\pgfqpoint{1.150000in}{0.150000in}}{\pgfqpoint{5.700000in}{5.700000in}}%
\pgfusepath{clip}%
\pgfsetbuttcap%
\pgfsetroundjoin%
\definecolor{currentfill}{rgb}{0.220124,0.725509,0.466226}%
\pgfsetfillcolor{currentfill}%
\pgfsetfillopacity{0.800000}%
\pgfsetlinewidth{0.000000pt}%
\definecolor{currentstroke}{rgb}{0.000000,0.000000,0.000000}%
\pgfsetstrokecolor{currentstroke}%
\pgfsetdash{}{0pt}%
\pgfpathmoveto{\pgfqpoint{5.475013in}{3.267934in}}%
\pgfpathlineto{\pgfqpoint{5.489964in}{3.284552in}}%
\pgfpathlineto{\pgfqpoint{5.504937in}{3.301357in}}%
\pgfpathlineto{\pgfqpoint{5.519934in}{3.318351in}}%
\pgfpathlineto{\pgfqpoint{5.534953in}{3.335533in}}%
\pgfpathlineto{\pgfqpoint{5.542447in}{3.339714in}}%
\pgfpathlineto{\pgfqpoint{5.549930in}{3.343742in}}%
\pgfpathlineto{\pgfqpoint{5.557403in}{3.347621in}}%
\pgfpathlineto{\pgfqpoint{5.564865in}{3.351355in}}%
\pgfpathlineto{\pgfqpoint{5.549861in}{3.334478in}}%
\pgfpathlineto{\pgfqpoint{5.534881in}{3.317789in}}%
\pgfpathlineto{\pgfqpoint{5.519923in}{3.301287in}}%
\pgfpathlineto{\pgfqpoint{5.504988in}{3.284973in}}%
\pgfpathlineto{\pgfqpoint{5.497510in}{3.280922in}}%
\pgfpathlineto{\pgfqpoint{5.490021in}{3.276734in}}%
\pgfpathlineto{\pgfqpoint{5.482522in}{3.272406in}}%
\pgfpathlineto{\pgfqpoint{5.475013in}{3.267934in}}%
\pgfpathclose%
\pgfusepath{fill}%
\end{pgfscope}%
\begin{pgfscope}%
\pgfpathrectangle{\pgfqpoint{1.150000in}{0.150000in}}{\pgfqpoint{5.700000in}{5.700000in}}%
\pgfusepath{clip}%
\pgfsetbuttcap%
\pgfsetroundjoin%
\definecolor{currentfill}{rgb}{0.281412,0.155834,0.469201}%
\pgfsetfillcolor{currentfill}%
\pgfsetfillopacity{0.800000}%
\pgfsetlinewidth{0.000000pt}%
\definecolor{currentstroke}{rgb}{0.000000,0.000000,0.000000}%
\pgfsetstrokecolor{currentstroke}%
\pgfsetdash{}{0pt}%
\pgfpathmoveto{\pgfqpoint{2.742154in}{1.693885in}}%
\pgfpathlineto{\pgfqpoint{2.756222in}{1.679047in}}%
\pgfpathlineto{\pgfqpoint{2.770286in}{1.664438in}}%
\pgfpathlineto{\pgfqpoint{2.784344in}{1.650057in}}%
\pgfpathlineto{\pgfqpoint{2.798398in}{1.635903in}}%
\pgfpathlineto{\pgfqpoint{2.807307in}{1.632874in}}%
\pgfpathlineto{\pgfqpoint{2.816196in}{1.630225in}}%
\pgfpathlineto{\pgfqpoint{2.825066in}{1.627947in}}%
\pgfpathlineto{\pgfqpoint{2.833916in}{1.626033in}}%
\pgfpathlineto{\pgfqpoint{2.819912in}{1.639408in}}%
\pgfpathlineto{\pgfqpoint{2.805904in}{1.653009in}}%
\pgfpathlineto{\pgfqpoint{2.791891in}{1.666838in}}%
\pgfpathlineto{\pgfqpoint{2.777874in}{1.680894in}}%
\pgfpathlineto{\pgfqpoint{2.768974in}{1.683575in}}%
\pgfpathlineto{\pgfqpoint{2.760055in}{1.686628in}}%
\pgfpathlineto{\pgfqpoint{2.751115in}{1.690062in}}%
\pgfpathlineto{\pgfqpoint{2.742154in}{1.693885in}}%
\pgfpathclose%
\pgfusepath{fill}%
\end{pgfscope}%
\begin{pgfscope}%
\pgfpathrectangle{\pgfqpoint{1.150000in}{0.150000in}}{\pgfqpoint{5.700000in}{5.700000in}}%
\pgfusepath{clip}%
\pgfsetbuttcap%
\pgfsetroundjoin%
\definecolor{currentfill}{rgb}{0.120565,0.596422,0.543611}%
\pgfsetfillcolor{currentfill}%
\pgfsetfillopacity{0.800000}%
\pgfsetlinewidth{0.000000pt}%
\definecolor{currentstroke}{rgb}{0.000000,0.000000,0.000000}%
\pgfsetstrokecolor{currentstroke}%
\pgfsetdash{}{0pt}%
\pgfpathmoveto{\pgfqpoint{5.054538in}{2.852617in}}%
\pgfpathlineto{\pgfqpoint{5.069200in}{2.867298in}}%
\pgfpathlineto{\pgfqpoint{5.083881in}{2.882167in}}%
\pgfpathlineto{\pgfqpoint{5.098583in}{2.897222in}}%
\pgfpathlineto{\pgfqpoint{5.113306in}{2.912465in}}%
\pgfpathlineto{\pgfqpoint{5.121078in}{2.921421in}}%
\pgfpathlineto{\pgfqpoint{5.128841in}{2.930205in}}%
\pgfpathlineto{\pgfqpoint{5.136595in}{2.938816in}}%
\pgfpathlineto{\pgfqpoint{5.144341in}{2.947255in}}%
\pgfpathlineto{\pgfqpoint{5.129623in}{2.932100in}}%
\pgfpathlineto{\pgfqpoint{5.114926in}{2.917132in}}%
\pgfpathlineto{\pgfqpoint{5.100249in}{2.902351in}}%
\pgfpathlineto{\pgfqpoint{5.085593in}{2.887757in}}%
\pgfpathlineto{\pgfqpoint{5.077842in}{2.879217in}}%
\pgfpathlineto{\pgfqpoint{5.070083in}{2.870514in}}%
\pgfpathlineto{\pgfqpoint{5.062315in}{2.861648in}}%
\pgfpathlineto{\pgfqpoint{5.054538in}{2.852617in}}%
\pgfpathclose%
\pgfusepath{fill}%
\end{pgfscope}%
\begin{pgfscope}%
\pgfpathrectangle{\pgfqpoint{1.150000in}{0.150000in}}{\pgfqpoint{5.700000in}{5.700000in}}%
\pgfusepath{clip}%
\pgfsetbuttcap%
\pgfsetroundjoin%
\definecolor{currentfill}{rgb}{0.243113,0.292092,0.538516}%
\pgfsetfillcolor{currentfill}%
\pgfsetfillopacity{0.800000}%
\pgfsetlinewidth{0.000000pt}%
\definecolor{currentstroke}{rgb}{0.000000,0.000000,0.000000}%
\pgfsetstrokecolor{currentstroke}%
\pgfsetdash{}{0pt}%
\pgfpathmoveto{\pgfqpoint{2.459385in}{2.041156in}}%
\pgfpathlineto{\pgfqpoint{2.473602in}{2.021400in}}%
\pgfpathlineto{\pgfqpoint{2.487809in}{2.001907in}}%
\pgfpathlineto{\pgfqpoint{2.502007in}{1.982675in}}%
\pgfpathlineto{\pgfqpoint{2.516195in}{1.963701in}}%
\pgfpathlineto{\pgfqpoint{2.525388in}{1.956657in}}%
\pgfpathlineto{\pgfqpoint{2.534555in}{1.950040in}}%
\pgfpathlineto{\pgfqpoint{2.543698in}{1.943843in}}%
\pgfpathlineto{\pgfqpoint{2.552818in}{1.938056in}}%
\pgfpathlineto{\pgfqpoint{2.538691in}{1.956227in}}%
\pgfpathlineto{\pgfqpoint{2.524555in}{1.974656in}}%
\pgfpathlineto{\pgfqpoint{2.510411in}{1.993345in}}%
\pgfpathlineto{\pgfqpoint{2.496257in}{2.012294in}}%
\pgfpathlineto{\pgfqpoint{2.487076in}{2.018870in}}%
\pgfpathlineto{\pgfqpoint{2.477871in}{2.025867in}}%
\pgfpathlineto{\pgfqpoint{2.468641in}{2.033292in}}%
\pgfpathlineto{\pgfqpoint{2.459385in}{2.041156in}}%
\pgfpathclose%
\pgfusepath{fill}%
\end{pgfscope}%
\begin{pgfscope}%
\pgfpathrectangle{\pgfqpoint{1.150000in}{0.150000in}}{\pgfqpoint{5.700000in}{5.700000in}}%
\pgfusepath{clip}%
\pgfsetbuttcap%
\pgfsetroundjoin%
\definecolor{currentfill}{rgb}{0.180629,0.429975,0.557282}%
\pgfsetfillcolor{currentfill}%
\pgfsetfillopacity{0.800000}%
\pgfsetlinewidth{0.000000pt}%
\definecolor{currentstroke}{rgb}{0.000000,0.000000,0.000000}%
\pgfsetstrokecolor{currentstroke}%
\pgfsetdash{}{0pt}%
\pgfpathmoveto{\pgfqpoint{4.601552in}{2.326691in}}%
\pgfpathlineto{\pgfqpoint{4.615919in}{2.338028in}}%
\pgfpathlineto{\pgfqpoint{4.630302in}{2.349549in}}%
\pgfpathlineto{\pgfqpoint{4.644703in}{2.361256in}}%
\pgfpathlineto{\pgfqpoint{4.659122in}{2.373147in}}%
\pgfpathlineto{\pgfqpoint{4.667103in}{2.386452in}}%
\pgfpathlineto{\pgfqpoint{4.675079in}{2.399617in}}%
\pgfpathlineto{\pgfqpoint{4.683048in}{2.412642in}}%
\pgfpathlineto{\pgfqpoint{4.691012in}{2.425526in}}%
\pgfpathlineto{\pgfqpoint{4.676592in}{2.413479in}}%
\pgfpathlineto{\pgfqpoint{4.662189in}{2.401616in}}%
\pgfpathlineto{\pgfqpoint{4.647803in}{2.389939in}}%
\pgfpathlineto{\pgfqpoint{4.633434in}{2.378447in}}%
\pgfpathlineto{\pgfqpoint{4.625473in}{2.365707in}}%
\pgfpathlineto{\pgfqpoint{4.617505in}{2.352833in}}%
\pgfpathlineto{\pgfqpoint{4.609531in}{2.339828in}}%
\pgfpathlineto{\pgfqpoint{4.601552in}{2.326691in}}%
\pgfpathclose%
\pgfusepath{fill}%
\end{pgfscope}%
\begin{pgfscope}%
\pgfpathrectangle{\pgfqpoint{1.150000in}{0.150000in}}{\pgfqpoint{5.700000in}{5.700000in}}%
\pgfusepath{clip}%
\pgfsetbuttcap%
\pgfsetroundjoin%
\definecolor{currentfill}{rgb}{0.140210,0.665859,0.513427}%
\pgfsetfillcolor{currentfill}%
\pgfsetfillopacity{0.800000}%
\pgfsetlinewidth{0.000000pt}%
\definecolor{currentstroke}{rgb}{0.000000,0.000000,0.000000}%
\pgfsetstrokecolor{currentstroke}%
\pgfsetdash{}{0pt}%
\pgfpathmoveto{\pgfqpoint{5.264997in}{3.070186in}}%
\pgfpathlineto{\pgfqpoint{5.279806in}{3.085997in}}%
\pgfpathlineto{\pgfqpoint{5.294637in}{3.101994in}}%
\pgfpathlineto{\pgfqpoint{5.309489in}{3.118180in}}%
\pgfpathlineto{\pgfqpoint{5.324364in}{3.134554in}}%
\pgfpathlineto{\pgfqpoint{5.332009in}{3.141171in}}%
\pgfpathlineto{\pgfqpoint{5.339643in}{3.147618in}}%
\pgfpathlineto{\pgfqpoint{5.347267in}{3.153898in}}%
\pgfpathlineto{\pgfqpoint{5.354882in}{3.160013in}}%
\pgfpathlineto{\pgfqpoint{5.340017in}{3.143835in}}%
\pgfpathlineto{\pgfqpoint{5.325175in}{3.127845in}}%
\pgfpathlineto{\pgfqpoint{5.310354in}{3.112042in}}%
\pgfpathlineto{\pgfqpoint{5.295555in}{3.096426in}}%
\pgfpathlineto{\pgfqpoint{5.287930in}{3.090103in}}%
\pgfpathlineto{\pgfqpoint{5.280295in}{3.083623in}}%
\pgfpathlineto{\pgfqpoint{5.272651in}{3.076985in}}%
\pgfpathlineto{\pgfqpoint{5.264997in}{3.070186in}}%
\pgfpathclose%
\pgfusepath{fill}%
\end{pgfscope}%
\begin{pgfscope}%
\pgfpathrectangle{\pgfqpoint{1.150000in}{0.150000in}}{\pgfqpoint{5.700000in}{5.700000in}}%
\pgfusepath{clip}%
\pgfsetbuttcap%
\pgfsetroundjoin%
\definecolor{currentfill}{rgb}{0.283072,0.130895,0.449241}%
\pgfsetfillcolor{currentfill}%
\pgfsetfillopacity{0.800000}%
\pgfsetlinewidth{0.000000pt}%
\definecolor{currentstroke}{rgb}{0.000000,0.000000,0.000000}%
\pgfsetstrokecolor{currentstroke}%
\pgfsetdash{}{0pt}%
\pgfpathmoveto{\pgfqpoint{2.798398in}{1.635903in}}%
\pgfpathlineto{\pgfqpoint{2.812448in}{1.621974in}}%
\pgfpathlineto{\pgfqpoint{2.826493in}{1.608268in}}%
\pgfpathlineto{\pgfqpoint{2.840535in}{1.594786in}}%
\pgfpathlineto{\pgfqpoint{2.854573in}{1.581524in}}%
\pgfpathlineto{\pgfqpoint{2.863432in}{1.579286in}}%
\pgfpathlineto{\pgfqpoint{2.872272in}{1.577418in}}%
\pgfpathlineto{\pgfqpoint{2.881094in}{1.575912in}}%
\pgfpathlineto{\pgfqpoint{2.889897in}{1.574761in}}%
\pgfpathlineto{\pgfqpoint{2.875907in}{1.587247in}}%
\pgfpathlineto{\pgfqpoint{2.861913in}{1.599953in}}%
\pgfpathlineto{\pgfqpoint{2.847917in}{1.612881in}}%
\pgfpathlineto{\pgfqpoint{2.833916in}{1.626033in}}%
\pgfpathlineto{\pgfqpoint{2.825066in}{1.627947in}}%
\pgfpathlineto{\pgfqpoint{2.816196in}{1.630225in}}%
\pgfpathlineto{\pgfqpoint{2.807307in}{1.632874in}}%
\pgfpathlineto{\pgfqpoint{2.798398in}{1.635903in}}%
\pgfpathclose%
\pgfusepath{fill}%
\end{pgfscope}%
\begin{pgfscope}%
\pgfpathrectangle{\pgfqpoint{1.150000in}{0.150000in}}{\pgfqpoint{5.700000in}{5.700000in}}%
\pgfusepath{clip}%
\pgfsetbuttcap%
\pgfsetroundjoin%
\definecolor{currentfill}{rgb}{0.273809,0.031497,0.358853}%
\pgfsetfillcolor{currentfill}%
\pgfsetfillopacity{0.800000}%
\pgfsetlinewidth{0.000000pt}%
\definecolor{currentstroke}{rgb}{0.000000,0.000000,0.000000}%
\pgfsetstrokecolor{currentstroke}%
\pgfsetdash{}{0pt}%
\pgfpathmoveto{\pgfqpoint{3.113473in}{1.404050in}}%
\pgfpathlineto{\pgfqpoint{3.127439in}{1.395145in}}%
\pgfpathlineto{\pgfqpoint{3.141406in}{1.386441in}}%
\pgfpathlineto{\pgfqpoint{3.155373in}{1.377939in}}%
\pgfpathlineto{\pgfqpoint{3.169341in}{1.369637in}}%
\pgfpathlineto{\pgfqpoint{3.177928in}{1.372562in}}%
\pgfpathlineto{\pgfqpoint{3.186501in}{1.375781in}}%
\pgfpathlineto{\pgfqpoint{3.195061in}{1.379288in}}%
\pgfpathlineto{\pgfqpoint{3.203609in}{1.383074in}}%
\pgfpathlineto{\pgfqpoint{3.189674in}{1.390648in}}%
\pgfpathlineto{\pgfqpoint{3.175741in}{1.398421in}}%
\pgfpathlineto{\pgfqpoint{3.161810in}{1.406395in}}%
\pgfpathlineto{\pgfqpoint{3.147878in}{1.414571in}}%
\pgfpathlineto{\pgfqpoint{3.139298in}{1.411501in}}%
\pgfpathlineto{\pgfqpoint{3.130704in}{1.408719in}}%
\pgfpathlineto{\pgfqpoint{3.122095in}{1.406233in}}%
\pgfpathlineto{\pgfqpoint{3.113473in}{1.404050in}}%
\pgfpathclose%
\pgfusepath{fill}%
\end{pgfscope}%
\begin{pgfscope}%
\pgfpathrectangle{\pgfqpoint{1.150000in}{0.150000in}}{\pgfqpoint{5.700000in}{5.700000in}}%
\pgfusepath{clip}%
\pgfsetbuttcap%
\pgfsetroundjoin%
\definecolor{currentfill}{rgb}{0.279574,0.170599,0.479997}%
\pgfsetfillcolor{currentfill}%
\pgfsetfillopacity{0.800000}%
\pgfsetlinewidth{0.000000pt}%
\definecolor{currentstroke}{rgb}{0.000000,0.000000,0.000000}%
\pgfsetstrokecolor{currentstroke}%
\pgfsetdash{}{0pt}%
\pgfpathmoveto{\pgfqpoint{4.027113in}{1.652933in}}%
\pgfpathlineto{\pgfqpoint{4.041186in}{1.657736in}}%
\pgfpathlineto{\pgfqpoint{4.055272in}{1.662721in}}%
\pgfpathlineto{\pgfqpoint{4.069368in}{1.667888in}}%
\pgfpathlineto{\pgfqpoint{4.083477in}{1.673237in}}%
\pgfpathlineto{\pgfqpoint{4.091620in}{1.687740in}}%
\pgfpathlineto{\pgfqpoint{4.099760in}{1.702244in}}%
\pgfpathlineto{\pgfqpoint{4.107894in}{1.716742in}}%
\pgfpathlineto{\pgfqpoint{4.116025in}{1.731232in}}%
\pgfpathlineto{\pgfqpoint{4.101918in}{1.725437in}}%
\pgfpathlineto{\pgfqpoint{4.087823in}{1.719824in}}%
\pgfpathlineto{\pgfqpoint{4.073739in}{1.714394in}}%
\pgfpathlineto{\pgfqpoint{4.059667in}{1.709145in}}%
\pgfpathlineto{\pgfqpoint{4.051535in}{1.695089in}}%
\pgfpathlineto{\pgfqpoint{4.043399in}{1.681032in}}%
\pgfpathlineto{\pgfqpoint{4.035258in}{1.666978in}}%
\pgfpathlineto{\pgfqpoint{4.027113in}{1.652933in}}%
\pgfpathclose%
\pgfusepath{fill}%
\end{pgfscope}%
\begin{pgfscope}%
\pgfpathrectangle{\pgfqpoint{1.150000in}{0.150000in}}{\pgfqpoint{5.700000in}{5.700000in}}%
\pgfusepath{clip}%
\pgfsetbuttcap%
\pgfsetroundjoin%
\definecolor{currentfill}{rgb}{0.121831,0.589055,0.545623}%
\pgfsetfillcolor{currentfill}%
\pgfsetfillopacity{0.800000}%
\pgfsetlinewidth{0.000000pt}%
\definecolor{currentstroke}{rgb}{0.000000,0.000000,0.000000}%
\pgfsetstrokecolor{currentstroke}%
\pgfsetdash{}{0pt}%
\pgfpathmoveto{\pgfqpoint{1.958566in}{2.957752in}}%
\pgfpathlineto{\pgfqpoint{1.973255in}{2.927102in}}%
\pgfpathlineto{\pgfqpoint{1.987921in}{2.896826in}}%
\pgfpathlineto{\pgfqpoint{2.002566in}{2.866920in}}%
\pgfpathlineto{\pgfqpoint{2.017190in}{2.837380in}}%
\pgfpathlineto{\pgfqpoint{2.026869in}{2.825833in}}%
\pgfpathlineto{\pgfqpoint{2.036517in}{2.814752in}}%
\pgfpathlineto{\pgfqpoint{2.046133in}{2.804128in}}%
\pgfpathlineto{\pgfqpoint{2.055719in}{2.793953in}}%
\pgfpathlineto{\pgfqpoint{2.041174in}{2.822698in}}%
\pgfpathlineto{\pgfqpoint{2.026609in}{2.851807in}}%
\pgfpathlineto{\pgfqpoint{2.012024in}{2.881283in}}%
\pgfpathlineto{\pgfqpoint{1.997417in}{2.911130in}}%
\pgfpathlineto{\pgfqpoint{1.987752in}{2.922087in}}%
\pgfpathlineto{\pgfqpoint{1.978056in}{2.933504in}}%
\pgfpathlineto{\pgfqpoint{1.968328in}{2.945390in}}%
\pgfpathlineto{\pgfqpoint{1.958566in}{2.957752in}}%
\pgfpathclose%
\pgfusepath{fill}%
\end{pgfscope}%
\begin{pgfscope}%
\pgfpathrectangle{\pgfqpoint{1.150000in}{0.150000in}}{\pgfqpoint{5.700000in}{5.700000in}}%
\pgfusepath{clip}%
\pgfsetbuttcap%
\pgfsetroundjoin%
\definecolor{currentfill}{rgb}{0.203063,0.379716,0.553925}%
\pgfsetfillcolor{currentfill}%
\pgfsetfillopacity{0.800000}%
\pgfsetlinewidth{0.000000pt}%
\definecolor{currentstroke}{rgb}{0.000000,0.000000,0.000000}%
\pgfsetstrokecolor{currentstroke}%
\pgfsetdash{}{0pt}%
\pgfpathmoveto{\pgfqpoint{4.480229in}{2.175204in}}%
\pgfpathlineto{\pgfqpoint{4.494526in}{2.185392in}}%
\pgfpathlineto{\pgfqpoint{4.508839in}{2.195763in}}%
\pgfpathlineto{\pgfqpoint{4.523168in}{2.206318in}}%
\pgfpathlineto{\pgfqpoint{4.537514in}{2.217058in}}%
\pgfpathlineto{\pgfqpoint{4.545537in}{2.231189in}}%
\pgfpathlineto{\pgfqpoint{4.553556in}{2.245202in}}%
\pgfpathlineto{\pgfqpoint{4.561569in}{2.259094in}}%
\pgfpathlineto{\pgfqpoint{4.569577in}{2.272864in}}%
\pgfpathlineto{\pgfqpoint{4.555229in}{2.261901in}}%
\pgfpathlineto{\pgfqpoint{4.540897in}{2.251123in}}%
\pgfpathlineto{\pgfqpoint{4.526582in}{2.240529in}}%
\pgfpathlineto{\pgfqpoint{4.512282in}{2.230120in}}%
\pgfpathlineto{\pgfqpoint{4.504277in}{2.216559in}}%
\pgfpathlineto{\pgfqpoint{4.496266in}{2.202886in}}%
\pgfpathlineto{\pgfqpoint{4.488250in}{2.189100in}}%
\pgfpathlineto{\pgfqpoint{4.480229in}{2.175204in}}%
\pgfpathclose%
\pgfusepath{fill}%
\end{pgfscope}%
\begin{pgfscope}%
\pgfpathrectangle{\pgfqpoint{1.150000in}{0.150000in}}{\pgfqpoint{5.700000in}{5.700000in}}%
\pgfusepath{clip}%
\pgfsetbuttcap%
\pgfsetroundjoin%
\definecolor{currentfill}{rgb}{0.277941,0.056324,0.381191}%
\pgfsetfillcolor{currentfill}%
\pgfsetfillopacity{0.800000}%
\pgfsetlinewidth{0.000000pt}%
\definecolor{currentstroke}{rgb}{0.000000,0.000000,0.000000}%
\pgfsetstrokecolor{currentstroke}%
\pgfsetdash{}{0pt}%
\pgfpathmoveto{\pgfqpoint{3.727791in}{1.414604in}}%
\pgfpathlineto{\pgfqpoint{3.741776in}{1.415128in}}%
\pgfpathlineto{\pgfqpoint{3.755768in}{1.415835in}}%
\pgfpathlineto{\pgfqpoint{3.769768in}{1.416725in}}%
\pgfpathlineto{\pgfqpoint{3.783777in}{1.417797in}}%
\pgfpathlineto{\pgfqpoint{3.792013in}{1.429928in}}%
\pgfpathlineto{\pgfqpoint{3.800243in}{1.442160in}}%
\pgfpathlineto{\pgfqpoint{3.808468in}{1.454487in}}%
\pgfpathlineto{\pgfqpoint{3.816687in}{1.466904in}}%
\pgfpathlineto{\pgfqpoint{3.802688in}{1.465263in}}%
\pgfpathlineto{\pgfqpoint{3.788697in}{1.463805in}}%
\pgfpathlineto{\pgfqpoint{3.774715in}{1.462531in}}%
\pgfpathlineto{\pgfqpoint{3.760741in}{1.461440in}}%
\pgfpathlineto{\pgfqpoint{3.752512in}{1.449579in}}%
\pgfpathlineto{\pgfqpoint{3.744278in}{1.437816in}}%
\pgfpathlineto{\pgfqpoint{3.736038in}{1.426155in}}%
\pgfpathlineto{\pgfqpoint{3.727791in}{1.414604in}}%
\pgfpathclose%
\pgfusepath{fill}%
\end{pgfscope}%
\begin{pgfscope}%
\pgfpathrectangle{\pgfqpoint{1.150000in}{0.150000in}}{\pgfqpoint{5.700000in}{5.700000in}}%
\pgfusepath{clip}%
\pgfsetbuttcap%
\pgfsetroundjoin%
\definecolor{currentfill}{rgb}{0.273809,0.031497,0.358853}%
\pgfsetfillcolor{currentfill}%
\pgfsetfillopacity{0.800000}%
\pgfsetlinewidth{0.000000pt}%
\definecolor{currentstroke}{rgb}{0.000000,0.000000,0.000000}%
\pgfsetstrokecolor{currentstroke}%
\pgfsetdash{}{0pt}%
\pgfpathmoveto{\pgfqpoint{3.638831in}{1.371736in}}%
\pgfpathlineto{\pgfqpoint{3.652799in}{1.370925in}}%
\pgfpathlineto{\pgfqpoint{3.666773in}{1.370298in}}%
\pgfpathlineto{\pgfqpoint{3.680755in}{1.369856in}}%
\pgfpathlineto{\pgfqpoint{3.694743in}{1.369597in}}%
\pgfpathlineto{\pgfqpoint{3.703015in}{1.380657in}}%
\pgfpathlineto{\pgfqpoint{3.711280in}{1.391849in}}%
\pgfpathlineto{\pgfqpoint{3.719539in}{1.403167in}}%
\pgfpathlineto{\pgfqpoint{3.727791in}{1.414604in}}%
\pgfpathlineto{\pgfqpoint{3.713815in}{1.414264in}}%
\pgfpathlineto{\pgfqpoint{3.699846in}{1.414108in}}%
\pgfpathlineto{\pgfqpoint{3.685885in}{1.414137in}}%
\pgfpathlineto{\pgfqpoint{3.671931in}{1.414350in}}%
\pgfpathlineto{\pgfqpoint{3.663666in}{1.403500in}}%
\pgfpathlineto{\pgfqpoint{3.655395in}{1.392776in}}%
\pgfpathlineto{\pgfqpoint{3.647117in}{1.382187in}}%
\pgfpathlineto{\pgfqpoint{3.638831in}{1.371736in}}%
\pgfpathclose%
\pgfusepath{fill}%
\end{pgfscope}%
\begin{pgfscope}%
\pgfpathrectangle{\pgfqpoint{1.150000in}{0.150000in}}{\pgfqpoint{5.700000in}{5.700000in}}%
\pgfusepath{clip}%
\pgfsetbuttcap%
\pgfsetroundjoin%
\definecolor{currentfill}{rgb}{0.229739,0.322361,0.545706}%
\pgfsetfillcolor{currentfill}%
\pgfsetfillopacity{0.800000}%
\pgfsetlinewidth{0.000000pt}%
\definecolor{currentstroke}{rgb}{0.000000,0.000000,0.000000}%
\pgfsetstrokecolor{currentstroke}%
\pgfsetdash{}{0pt}%
\pgfpathmoveto{\pgfqpoint{2.402413in}{2.122849in}}%
\pgfpathlineto{\pgfqpoint{2.416672in}{2.102021in}}%
\pgfpathlineto{\pgfqpoint{2.430920in}{2.081464in}}%
\pgfpathlineto{\pgfqpoint{2.445157in}{2.061176in}}%
\pgfpathlineto{\pgfqpoint{2.459385in}{2.041156in}}%
\pgfpathlineto{\pgfqpoint{2.468641in}{2.033292in}}%
\pgfpathlineto{\pgfqpoint{2.477871in}{2.025867in}}%
\pgfpathlineto{\pgfqpoint{2.487076in}{2.018870in}}%
\pgfpathlineto{\pgfqpoint{2.496257in}{2.012294in}}%
\pgfpathlineto{\pgfqpoint{2.482094in}{2.031508in}}%
\pgfpathlineto{\pgfqpoint{2.467921in}{2.050986in}}%
\pgfpathlineto{\pgfqpoint{2.453738in}{2.070732in}}%
\pgfpathlineto{\pgfqpoint{2.439545in}{2.090748in}}%
\pgfpathlineto{\pgfqpoint{2.430301in}{2.098118in}}%
\pgfpathlineto{\pgfqpoint{2.421032in}{2.105920in}}%
\pgfpathlineto{\pgfqpoint{2.411736in}{2.114160in}}%
\pgfpathlineto{\pgfqpoint{2.402413in}{2.122849in}}%
\pgfpathclose%
\pgfusepath{fill}%
\end{pgfscope}%
\begin{pgfscope}%
\pgfpathrectangle{\pgfqpoint{1.150000in}{0.150000in}}{\pgfqpoint{5.700000in}{5.700000in}}%
\pgfusepath{clip}%
\pgfsetbuttcap%
\pgfsetroundjoin%
\definecolor{currentfill}{rgb}{0.229739,0.322361,0.545706}%
\pgfsetfillcolor{currentfill}%
\pgfsetfillopacity{0.800000}%
\pgfsetlinewidth{0.000000pt}%
\definecolor{currentstroke}{rgb}{0.000000,0.000000,0.000000}%
\pgfsetstrokecolor{currentstroke}%
\pgfsetdash{}{0pt}%
\pgfpathmoveto{\pgfqpoint{4.358865in}{2.023587in}}%
\pgfpathlineto{\pgfqpoint{4.373097in}{2.032497in}}%
\pgfpathlineto{\pgfqpoint{4.387343in}{2.041589in}}%
\pgfpathlineto{\pgfqpoint{4.401604in}{2.050865in}}%
\pgfpathlineto{\pgfqpoint{4.415881in}{2.060324in}}%
\pgfpathlineto{\pgfqpoint{4.423941in}{2.075024in}}%
\pgfpathlineto{\pgfqpoint{4.431997in}{2.089632in}}%
\pgfpathlineto{\pgfqpoint{4.440048in}{2.104145in}}%
\pgfpathlineto{\pgfqpoint{4.448094in}{2.118560in}}%
\pgfpathlineto{\pgfqpoint{4.433815in}{2.108812in}}%
\pgfpathlineto{\pgfqpoint{4.419551in}{2.099248in}}%
\pgfpathlineto{\pgfqpoint{4.405303in}{2.089867in}}%
\pgfpathlineto{\pgfqpoint{4.391069in}{2.080670in}}%
\pgfpathlineto{\pgfqpoint{4.383025in}{2.066530in}}%
\pgfpathlineto{\pgfqpoint{4.374977in}{2.052302in}}%
\pgfpathlineto{\pgfqpoint{4.366923in}{2.037986in}}%
\pgfpathlineto{\pgfqpoint{4.358865in}{2.023587in}}%
\pgfpathclose%
\pgfusepath{fill}%
\end{pgfscope}%
\begin{pgfscope}%
\pgfpathrectangle{\pgfqpoint{1.150000in}{0.150000in}}{\pgfqpoint{5.700000in}{5.700000in}}%
\pgfusepath{clip}%
\pgfsetbuttcap%
\pgfsetroundjoin%
\definecolor{currentfill}{rgb}{0.283091,0.110553,0.431554}%
\pgfsetfillcolor{currentfill}%
\pgfsetfillopacity{0.800000}%
\pgfsetlinewidth{0.000000pt}%
\definecolor{currentstroke}{rgb}{0.000000,0.000000,0.000000}%
\pgfsetstrokecolor{currentstroke}%
\pgfsetdash{}{0pt}%
\pgfpathmoveto{\pgfqpoint{2.854573in}{1.581524in}}%
\pgfpathlineto{\pgfqpoint{2.868607in}{1.568483in}}%
\pgfpathlineto{\pgfqpoint{2.882638in}{1.555661in}}%
\pgfpathlineto{\pgfqpoint{2.896666in}{1.543056in}}%
\pgfpathlineto{\pgfqpoint{2.910691in}{1.530668in}}%
\pgfpathlineto{\pgfqpoint{2.919503in}{1.529217in}}%
\pgfpathlineto{\pgfqpoint{2.928296in}{1.528127in}}%
\pgfpathlineto{\pgfqpoint{2.937072in}{1.527390in}}%
\pgfpathlineto{\pgfqpoint{2.945831in}{1.526998in}}%
\pgfpathlineto{\pgfqpoint{2.931851in}{1.538613in}}%
\pgfpathlineto{\pgfqpoint{2.917869in}{1.550445in}}%
\pgfpathlineto{\pgfqpoint{2.903885in}{1.562494in}}%
\pgfpathlineto{\pgfqpoint{2.889897in}{1.574761in}}%
\pgfpathlineto{\pgfqpoint{2.881094in}{1.575912in}}%
\pgfpathlineto{\pgfqpoint{2.872272in}{1.577418in}}%
\pgfpathlineto{\pgfqpoint{2.863432in}{1.579286in}}%
\pgfpathlineto{\pgfqpoint{2.854573in}{1.581524in}}%
\pgfpathclose%
\pgfusepath{fill}%
\end{pgfscope}%
\begin{pgfscope}%
\pgfpathrectangle{\pgfqpoint{1.150000in}{0.150000in}}{\pgfqpoint{5.700000in}{5.700000in}}%
\pgfusepath{clip}%
\pgfsetbuttcap%
\pgfsetroundjoin%
\definecolor{currentfill}{rgb}{0.281446,0.084320,0.407414}%
\pgfsetfillcolor{currentfill}%
\pgfsetfillopacity{0.800000}%
\pgfsetlinewidth{0.000000pt}%
\definecolor{currentstroke}{rgb}{0.000000,0.000000,0.000000}%
\pgfsetstrokecolor{currentstroke}%
\pgfsetdash{}{0pt}%
\pgfpathmoveto{\pgfqpoint{3.816687in}{1.466904in}}%
\pgfpathlineto{\pgfqpoint{3.830696in}{1.468728in}}%
\pgfpathlineto{\pgfqpoint{3.844713in}{1.470735in}}%
\pgfpathlineto{\pgfqpoint{3.858739in}{1.472923in}}%
\pgfpathlineto{\pgfqpoint{3.872775in}{1.475294in}}%
\pgfpathlineto{\pgfqpoint{3.880981in}{1.488345in}}%
\pgfpathlineto{\pgfqpoint{3.889183in}{1.501468in}}%
\pgfpathlineto{\pgfqpoint{3.897379in}{1.514657in}}%
\pgfpathlineto{\pgfqpoint{3.905570in}{1.527907in}}%
\pgfpathlineto{\pgfqpoint{3.891541in}{1.524999in}}%
\pgfpathlineto{\pgfqpoint{3.877521in}{1.522272in}}%
\pgfpathlineto{\pgfqpoint{3.863511in}{1.519728in}}%
\pgfpathlineto{\pgfqpoint{3.849510in}{1.517368in}}%
\pgfpathlineto{\pgfqpoint{3.841313in}{1.504643in}}%
\pgfpathlineto{\pgfqpoint{3.833109in}{1.491987in}}%
\pgfpathlineto{\pgfqpoint{3.824901in}{1.479406in}}%
\pgfpathlineto{\pgfqpoint{3.816687in}{1.466904in}}%
\pgfpathclose%
\pgfusepath{fill}%
\end{pgfscope}%
\begin{pgfscope}%
\pgfpathrectangle{\pgfqpoint{1.150000in}{0.150000in}}{\pgfqpoint{5.700000in}{5.700000in}}%
\pgfusepath{clip}%
\pgfsetbuttcap%
\pgfsetroundjoin%
\definecolor{currentfill}{rgb}{0.269944,0.014625,0.341379}%
\pgfsetfillcolor{currentfill}%
\pgfsetfillopacity{0.800000}%
\pgfsetlinewidth{0.000000pt}%
\definecolor{currentstroke}{rgb}{0.000000,0.000000,0.000000}%
\pgfsetstrokecolor{currentstroke}%
\pgfsetdash{}{0pt}%
\pgfpathmoveto{\pgfqpoint{3.549751in}{1.339062in}}%
\pgfpathlineto{\pgfqpoint{3.563709in}{1.336880in}}%
\pgfpathlineto{\pgfqpoint{3.577674in}{1.334883in}}%
\pgfpathlineto{\pgfqpoint{3.591644in}{1.333072in}}%
\pgfpathlineto{\pgfqpoint{3.605620in}{1.331447in}}%
\pgfpathlineto{\pgfqpoint{3.613934in}{1.341280in}}%
\pgfpathlineto{\pgfqpoint{3.622240in}{1.351277in}}%
\pgfpathlineto{\pgfqpoint{3.630539in}{1.361431in}}%
\pgfpathlineto{\pgfqpoint{3.638831in}{1.371736in}}%
\pgfpathlineto{\pgfqpoint{3.624871in}{1.372732in}}%
\pgfpathlineto{\pgfqpoint{3.610917in}{1.373914in}}%
\pgfpathlineto{\pgfqpoint{3.596969in}{1.375281in}}%
\pgfpathlineto{\pgfqpoint{3.583027in}{1.376835in}}%
\pgfpathlineto{\pgfqpoint{3.574720in}{1.367147in}}%
\pgfpathlineto{\pgfqpoint{3.566405in}{1.357618in}}%
\pgfpathlineto{\pgfqpoint{3.558082in}{1.348254in}}%
\pgfpathlineto{\pgfqpoint{3.549751in}{1.339062in}}%
\pgfpathclose%
\pgfusepath{fill}%
\end{pgfscope}%
\begin{pgfscope}%
\pgfpathrectangle{\pgfqpoint{1.150000in}{0.150000in}}{\pgfqpoint{5.700000in}{5.700000in}}%
\pgfusepath{clip}%
\pgfsetbuttcap%
\pgfsetroundjoin%
\definecolor{currentfill}{rgb}{0.131172,0.555899,0.552459}%
\pgfsetfillcolor{currentfill}%
\pgfsetfillopacity{0.800000}%
\pgfsetlinewidth{0.000000pt}%
\definecolor{currentstroke}{rgb}{0.000000,0.000000,0.000000}%
\pgfsetstrokecolor{currentstroke}%
\pgfsetdash{}{0pt}%
\pgfpathmoveto{\pgfqpoint{4.933556in}{2.717177in}}%
\pgfpathlineto{\pgfqpoint{4.948143in}{2.731179in}}%
\pgfpathlineto{\pgfqpoint{4.962750in}{2.745368in}}%
\pgfpathlineto{\pgfqpoint{4.977377in}{2.759744in}}%
\pgfpathlineto{\pgfqpoint{4.992024in}{2.774306in}}%
\pgfpathlineto{\pgfqpoint{4.999867in}{2.784693in}}%
\pgfpathlineto{\pgfqpoint{5.007702in}{2.794907in}}%
\pgfpathlineto{\pgfqpoint{5.015529in}{2.804950in}}%
\pgfpathlineto{\pgfqpoint{5.023348in}{2.814822in}}%
\pgfpathlineto{\pgfqpoint{5.008703in}{2.800276in}}%
\pgfpathlineto{\pgfqpoint{4.994079in}{2.785917in}}%
\pgfpathlineto{\pgfqpoint{4.979474in}{2.771744in}}%
\pgfpathlineto{\pgfqpoint{4.964888in}{2.757758in}}%
\pgfpathlineto{\pgfqpoint{4.957067in}{2.747857in}}%
\pgfpathlineto{\pgfqpoint{4.949238in}{2.737793in}}%
\pgfpathlineto{\pgfqpoint{4.941401in}{2.727567in}}%
\pgfpathlineto{\pgfqpoint{4.933556in}{2.717177in}}%
\pgfpathclose%
\pgfusepath{fill}%
\end{pgfscope}%
\begin{pgfscope}%
\pgfpathrectangle{\pgfqpoint{1.150000in}{0.150000in}}{\pgfqpoint{5.700000in}{5.700000in}}%
\pgfusepath{clip}%
\pgfsetbuttcap%
\pgfsetroundjoin%
\definecolor{currentfill}{rgb}{0.153364,0.497000,0.557724}%
\pgfsetfillcolor{currentfill}%
\pgfsetfillopacity{0.800000}%
\pgfsetlinewidth{0.000000pt}%
\definecolor{currentstroke}{rgb}{0.000000,0.000000,0.000000}%
\pgfsetstrokecolor{currentstroke}%
\pgfsetdash{}{0pt}%
\pgfpathmoveto{\pgfqpoint{2.095071in}{2.658186in}}%
\pgfpathlineto{\pgfqpoint{2.109601in}{2.630960in}}%
\pgfpathlineto{\pgfqpoint{2.124112in}{2.604068in}}%
\pgfpathlineto{\pgfqpoint{2.138607in}{2.577505in}}%
\pgfpathlineto{\pgfqpoint{2.153084in}{2.551269in}}%
\pgfpathlineto{\pgfqpoint{2.162652in}{2.540294in}}%
\pgfpathlineto{\pgfqpoint{2.172190in}{2.529786in}}%
\pgfpathlineto{\pgfqpoint{2.181698in}{2.519735in}}%
\pgfpathlineto{\pgfqpoint{2.191177in}{2.510134in}}%
\pgfpathlineto{\pgfqpoint{2.176776in}{2.535560in}}%
\pgfpathlineto{\pgfqpoint{2.162357in}{2.561310in}}%
\pgfpathlineto{\pgfqpoint{2.147922in}{2.587388in}}%
\pgfpathlineto{\pgfqpoint{2.133470in}{2.613797in}}%
\pgfpathlineto{\pgfqpoint{2.123917in}{2.624195in}}%
\pgfpathlineto{\pgfqpoint{2.114333in}{2.635054in}}%
\pgfpathlineto{\pgfqpoint{2.104718in}{2.646381in}}%
\pgfpathlineto{\pgfqpoint{2.095071in}{2.658186in}}%
\pgfpathclose%
\pgfusepath{fill}%
\end{pgfscope}%
\begin{pgfscope}%
\pgfpathrectangle{\pgfqpoint{1.150000in}{0.150000in}}{\pgfqpoint{5.700000in}{5.700000in}}%
\pgfusepath{clip}%
\pgfsetbuttcap%
\pgfsetroundjoin%
\definecolor{currentfill}{rgb}{0.274149,0.751988,0.436601}%
\pgfsetfillcolor{currentfill}%
\pgfsetfillopacity{0.800000}%
\pgfsetlinewidth{0.000000pt}%
\definecolor{currentstroke}{rgb}{0.000000,0.000000,0.000000}%
\pgfsetstrokecolor{currentstroke}%
\pgfsetdash{}{0pt}%
\pgfpathmoveto{\pgfqpoint{5.564865in}{3.351355in}}%
\pgfpathlineto{\pgfqpoint{5.579891in}{3.368420in}}%
\pgfpathlineto{\pgfqpoint{5.594941in}{3.385673in}}%
\pgfpathlineto{\pgfqpoint{5.610015in}{3.403115in}}%
\pgfpathlineto{\pgfqpoint{5.625112in}{3.420746in}}%
\pgfpathlineto{\pgfqpoint{5.632546in}{3.424009in}}%
\pgfpathlineto{\pgfqpoint{5.639968in}{3.427127in}}%
\pgfpathlineto{\pgfqpoint{5.647379in}{3.430102in}}%
\pgfpathlineto{\pgfqpoint{5.654779in}{3.432938in}}%
\pgfpathlineto{\pgfqpoint{5.639700in}{3.415651in}}%
\pgfpathlineto{\pgfqpoint{5.624645in}{3.398551in}}%
\pgfpathlineto{\pgfqpoint{5.609614in}{3.381640in}}%
\pgfpathlineto{\pgfqpoint{5.594605in}{3.364915in}}%
\pgfpathlineto{\pgfqpoint{5.587186in}{3.361724in}}%
\pgfpathlineto{\pgfqpoint{5.579756in}{3.358403in}}%
\pgfpathlineto{\pgfqpoint{5.572316in}{3.354948in}}%
\pgfpathlineto{\pgfqpoint{5.564865in}{3.351355in}}%
\pgfpathclose%
\pgfusepath{fill}%
\end{pgfscope}%
\begin{pgfscope}%
\pgfpathrectangle{\pgfqpoint{1.150000in}{0.150000in}}{\pgfqpoint{5.700000in}{5.700000in}}%
\pgfusepath{clip}%
\pgfsetbuttcap%
\pgfsetroundjoin%
\definecolor{currentfill}{rgb}{0.252194,0.269783,0.531579}%
\pgfsetfillcolor{currentfill}%
\pgfsetfillopacity{0.800000}%
\pgfsetlinewidth{0.000000pt}%
\definecolor{currentstroke}{rgb}{0.000000,0.000000,0.000000}%
\pgfsetstrokecolor{currentstroke}%
\pgfsetdash{}{0pt}%
\pgfpathmoveto{\pgfqpoint{4.237470in}{1.874579in}}%
\pgfpathlineto{\pgfqpoint{4.251641in}{1.882085in}}%
\pgfpathlineto{\pgfqpoint{4.265827in}{1.889773in}}%
\pgfpathlineto{\pgfqpoint{4.280025in}{1.897643in}}%
\pgfpathlineto{\pgfqpoint{4.294238in}{1.905696in}}%
\pgfpathlineto{\pgfqpoint{4.302332in}{1.920667in}}%
\pgfpathlineto{\pgfqpoint{4.310422in}{1.935578in}}%
\pgfpathlineto{\pgfqpoint{4.318507in}{1.950424in}}%
\pgfpathlineto{\pgfqpoint{4.326588in}{1.965203in}}%
\pgfpathlineto{\pgfqpoint{4.312373in}{1.956797in}}%
\pgfpathlineto{\pgfqpoint{4.298172in}{1.948574in}}%
\pgfpathlineto{\pgfqpoint{4.283985in}{1.940533in}}%
\pgfpathlineto{\pgfqpoint{4.269812in}{1.932676in}}%
\pgfpathlineto{\pgfqpoint{4.261733in}{1.918237in}}%
\pgfpathlineto{\pgfqpoint{4.253650in}{1.903740in}}%
\pgfpathlineto{\pgfqpoint{4.245562in}{1.889186in}}%
\pgfpathlineto{\pgfqpoint{4.237470in}{1.874579in}}%
\pgfpathclose%
\pgfusepath{fill}%
\end{pgfscope}%
\begin{pgfscope}%
\pgfpathrectangle{\pgfqpoint{1.150000in}{0.150000in}}{\pgfqpoint{5.700000in}{5.700000in}}%
\pgfusepath{clip}%
\pgfsetbuttcap%
\pgfsetroundjoin%
\definecolor{currentfill}{rgb}{0.267004,0.004874,0.329415}%
\pgfsetfillcolor{currentfill}%
\pgfsetfillopacity{0.800000}%
\pgfsetlinewidth{0.000000pt}%
\definecolor{currentstroke}{rgb}{0.000000,0.000000,0.000000}%
\pgfsetstrokecolor{currentstroke}%
\pgfsetdash{}{0pt}%
\pgfpathmoveto{\pgfqpoint{3.315144in}{1.329590in}}%
\pgfpathlineto{\pgfqpoint{3.329096in}{1.323781in}}%
\pgfpathlineto{\pgfqpoint{3.343052in}{1.318165in}}%
\pgfpathlineto{\pgfqpoint{3.357010in}{1.312741in}}%
\pgfpathlineto{\pgfqpoint{3.370971in}{1.307508in}}%
\pgfpathlineto{\pgfqpoint{3.379419in}{1.313686in}}%
\pgfpathlineto{\pgfqpoint{3.387857in}{1.320104in}}%
\pgfpathlineto{\pgfqpoint{3.396284in}{1.326755in}}%
\pgfpathlineto{\pgfqpoint{3.404701in}{1.333633in}}%
\pgfpathlineto{\pgfqpoint{3.390765in}{1.338173in}}%
\pgfpathlineto{\pgfqpoint{3.376833in}{1.342904in}}%
\pgfpathlineto{\pgfqpoint{3.362903in}{1.347827in}}%
\pgfpathlineto{\pgfqpoint{3.348978in}{1.352942in}}%
\pgfpathlineto{\pgfqpoint{3.340535in}{1.346745in}}%
\pgfpathlineto{\pgfqpoint{3.332083in}{1.340782in}}%
\pgfpathlineto{\pgfqpoint{3.323619in}{1.335062in}}%
\pgfpathlineto{\pgfqpoint{3.315144in}{1.329590in}}%
\pgfpathclose%
\pgfusepath{fill}%
\end{pgfscope}%
\begin{pgfscope}%
\pgfpathrectangle{\pgfqpoint{1.150000in}{0.150000in}}{\pgfqpoint{5.700000in}{5.700000in}}%
\pgfusepath{clip}%
\pgfsetbuttcap%
\pgfsetroundjoin%
\definecolor{currentfill}{rgb}{0.214298,0.355619,0.551184}%
\pgfsetfillcolor{currentfill}%
\pgfsetfillopacity{0.800000}%
\pgfsetlinewidth{0.000000pt}%
\definecolor{currentstroke}{rgb}{0.000000,0.000000,0.000000}%
\pgfsetstrokecolor{currentstroke}%
\pgfsetdash{}{0pt}%
\pgfpathmoveto{\pgfqpoint{2.345263in}{2.208917in}}%
\pgfpathlineto{\pgfqpoint{2.359568in}{2.186982in}}%
\pgfpathlineto{\pgfqpoint{2.373861in}{2.165327in}}%
\pgfpathlineto{\pgfqpoint{2.388143in}{2.143950in}}%
\pgfpathlineto{\pgfqpoint{2.402413in}{2.122849in}}%
\pgfpathlineto{\pgfqpoint{2.411736in}{2.114160in}}%
\pgfpathlineto{\pgfqpoint{2.421032in}{2.105920in}}%
\pgfpathlineto{\pgfqpoint{2.430301in}{2.098118in}}%
\pgfpathlineto{\pgfqpoint{2.439545in}{2.090748in}}%
\pgfpathlineto{\pgfqpoint{2.425341in}{2.111035in}}%
\pgfpathlineto{\pgfqpoint{2.411127in}{2.131597in}}%
\pgfpathlineto{\pgfqpoint{2.396901in}{2.152434in}}%
\pgfpathlineto{\pgfqpoint{2.382664in}{2.173550in}}%
\pgfpathlineto{\pgfqpoint{2.373355in}{2.181721in}}%
\pgfpathlineto{\pgfqpoint{2.364018in}{2.190334in}}%
\pgfpathlineto{\pgfqpoint{2.354654in}{2.199396in}}%
\pgfpathlineto{\pgfqpoint{2.345263in}{2.208917in}}%
\pgfpathclose%
\pgfusepath{fill}%
\end{pgfscope}%
\begin{pgfscope}%
\pgfpathrectangle{\pgfqpoint{1.150000in}{0.150000in}}{\pgfqpoint{5.700000in}{5.700000in}}%
\pgfusepath{clip}%
\pgfsetbuttcap%
\pgfsetroundjoin%
\definecolor{currentfill}{rgb}{0.282327,0.094955,0.417331}%
\pgfsetfillcolor{currentfill}%
\pgfsetfillopacity{0.800000}%
\pgfsetlinewidth{0.000000pt}%
\definecolor{currentstroke}{rgb}{0.000000,0.000000,0.000000}%
\pgfsetstrokecolor{currentstroke}%
\pgfsetdash{}{0pt}%
\pgfpathmoveto{\pgfqpoint{2.910691in}{1.530668in}}%
\pgfpathlineto{\pgfqpoint{2.924713in}{1.518495in}}%
\pgfpathlineto{\pgfqpoint{2.938733in}{1.506537in}}%
\pgfpathlineto{\pgfqpoint{2.952750in}{1.494791in}}%
\pgfpathlineto{\pgfqpoint{2.966766in}{1.483258in}}%
\pgfpathlineto{\pgfqpoint{2.975532in}{1.482591in}}%
\pgfpathlineto{\pgfqpoint{2.984281in}{1.482276in}}%
\pgfpathlineto{\pgfqpoint{2.993014in}{1.482305in}}%
\pgfpathlineto{\pgfqpoint{3.001730in}{1.482670in}}%
\pgfpathlineto{\pgfqpoint{2.987758in}{1.493434in}}%
\pgfpathlineto{\pgfqpoint{2.973784in}{1.504409in}}%
\pgfpathlineto{\pgfqpoint{2.959808in}{1.515596in}}%
\pgfpathlineto{\pgfqpoint{2.945831in}{1.526998in}}%
\pgfpathlineto{\pgfqpoint{2.937072in}{1.527390in}}%
\pgfpathlineto{\pgfqpoint{2.928296in}{1.528127in}}%
\pgfpathlineto{\pgfqpoint{2.919503in}{1.529217in}}%
\pgfpathlineto{\pgfqpoint{2.910691in}{1.530668in}}%
\pgfpathclose%
\pgfusepath{fill}%
\end{pgfscope}%
\begin{pgfscope}%
\pgfpathrectangle{\pgfqpoint{1.150000in}{0.150000in}}{\pgfqpoint{5.700000in}{5.700000in}}%
\pgfusepath{clip}%
\pgfsetbuttcap%
\pgfsetroundjoin%
\definecolor{currentfill}{rgb}{0.283229,0.120777,0.440584}%
\pgfsetfillcolor{currentfill}%
\pgfsetfillopacity{0.800000}%
\pgfsetlinewidth{0.000000pt}%
\definecolor{currentstroke}{rgb}{0.000000,0.000000,0.000000}%
\pgfsetstrokecolor{currentstroke}%
\pgfsetdash{}{0pt}%
\pgfpathmoveto{\pgfqpoint{3.905570in}{1.527907in}}%
\pgfpathlineto{\pgfqpoint{3.919610in}{1.530998in}}%
\pgfpathlineto{\pgfqpoint{3.933659in}{1.534271in}}%
\pgfpathlineto{\pgfqpoint{3.947718in}{1.537726in}}%
\pgfpathlineto{\pgfqpoint{3.961788in}{1.541362in}}%
\pgfpathlineto{\pgfqpoint{3.969969in}{1.555188in}}%
\pgfpathlineto{\pgfqpoint{3.978146in}{1.569058in}}%
\pgfpathlineto{\pgfqpoint{3.986319in}{1.582967in}}%
\pgfpathlineto{\pgfqpoint{3.994487in}{1.596910in}}%
\pgfpathlineto{\pgfqpoint{3.980421in}{1.592766in}}%
\pgfpathlineto{\pgfqpoint{3.966366in}{1.588804in}}%
\pgfpathlineto{\pgfqpoint{3.952321in}{1.585023in}}%
\pgfpathlineto{\pgfqpoint{3.938287in}{1.581425in}}%
\pgfpathlineto{\pgfqpoint{3.930115in}{1.567978in}}%
\pgfpathlineto{\pgfqpoint{3.921938in}{1.554573in}}%
\pgfpathlineto{\pgfqpoint{3.913757in}{1.541214in}}%
\pgfpathlineto{\pgfqpoint{3.905570in}{1.527907in}}%
\pgfpathclose%
\pgfusepath{fill}%
\end{pgfscope}%
\begin{pgfscope}%
\pgfpathrectangle{\pgfqpoint{1.150000in}{0.150000in}}{\pgfqpoint{5.700000in}{5.700000in}}%
\pgfusepath{clip}%
\pgfsetbuttcap%
\pgfsetroundjoin%
\definecolor{currentfill}{rgb}{0.271305,0.019942,0.347269}%
\pgfsetfillcolor{currentfill}%
\pgfsetfillopacity{0.800000}%
\pgfsetlinewidth{0.000000pt}%
\definecolor{currentstroke}{rgb}{0.000000,0.000000,0.000000}%
\pgfsetstrokecolor{currentstroke}%
\pgfsetdash{}{0pt}%
\pgfpathmoveto{\pgfqpoint{3.169341in}{1.369637in}}%
\pgfpathlineto{\pgfqpoint{3.183310in}{1.361534in}}%
\pgfpathlineto{\pgfqpoint{3.197279in}{1.353629in}}%
\pgfpathlineto{\pgfqpoint{3.211250in}{1.345923in}}%
\pgfpathlineto{\pgfqpoint{3.225222in}{1.338413in}}%
\pgfpathlineto{\pgfqpoint{3.233776in}{1.342079in}}%
\pgfpathlineto{\pgfqpoint{3.242316in}{1.346031in}}%
\pgfpathlineto{\pgfqpoint{3.250844in}{1.350262in}}%
\pgfpathlineto{\pgfqpoint{3.259360in}{1.354763in}}%
\pgfpathlineto{\pgfqpoint{3.245419in}{1.361545in}}%
\pgfpathlineto{\pgfqpoint{3.231481in}{1.368524in}}%
\pgfpathlineto{\pgfqpoint{3.217544in}{1.375700in}}%
\pgfpathlineto{\pgfqpoint{3.203609in}{1.383074in}}%
\pgfpathlineto{\pgfqpoint{3.195061in}{1.379288in}}%
\pgfpathlineto{\pgfqpoint{3.186501in}{1.375781in}}%
\pgfpathlineto{\pgfqpoint{3.177928in}{1.372562in}}%
\pgfpathlineto{\pgfqpoint{3.169341in}{1.369637in}}%
\pgfpathclose%
\pgfusepath{fill}%
\end{pgfscope}%
\begin{pgfscope}%
\pgfpathrectangle{\pgfqpoint{1.150000in}{0.150000in}}{\pgfqpoint{5.700000in}{5.700000in}}%
\pgfusepath{clip}%
\pgfsetbuttcap%
\pgfsetroundjoin%
\definecolor{currentfill}{rgb}{0.271828,0.209303,0.504434}%
\pgfsetfillcolor{currentfill}%
\pgfsetfillopacity{0.800000}%
\pgfsetlinewidth{0.000000pt}%
\definecolor{currentstroke}{rgb}{0.000000,0.000000,0.000000}%
\pgfsetstrokecolor{currentstroke}%
\pgfsetdash{}{0pt}%
\pgfpathmoveto{\pgfqpoint{4.116025in}{1.731232in}}%
\pgfpathlineto{\pgfqpoint{4.130145in}{1.737210in}}%
\pgfpathlineto{\pgfqpoint{4.144276in}{1.743369in}}%
\pgfpathlineto{\pgfqpoint{4.158421in}{1.749710in}}%
\pgfpathlineto{\pgfqpoint{4.172577in}{1.756233in}}%
\pgfpathlineto{\pgfqpoint{4.180704in}{1.771138in}}%
\pgfpathlineto{\pgfqpoint{4.188826in}{1.786019in}}%
\pgfpathlineto{\pgfqpoint{4.196944in}{1.800871in}}%
\pgfpathlineto{\pgfqpoint{4.205057in}{1.815691in}}%
\pgfpathlineto{\pgfqpoint{4.190900in}{1.808752in}}%
\pgfpathlineto{\pgfqpoint{4.176755in}{1.801995in}}%
\pgfpathlineto{\pgfqpoint{4.162624in}{1.795421in}}%
\pgfpathlineto{\pgfqpoint{4.148505in}{1.789029in}}%
\pgfpathlineto{\pgfqpoint{4.140391in}{1.774612in}}%
\pgfpathlineto{\pgfqpoint{4.132273in}{1.760171in}}%
\pgfpathlineto{\pgfqpoint{4.124151in}{1.745710in}}%
\pgfpathlineto{\pgfqpoint{4.116025in}{1.731232in}}%
\pgfpathclose%
\pgfusepath{fill}%
\end{pgfscope}%
\begin{pgfscope}%
\pgfpathrectangle{\pgfqpoint{1.150000in}{0.150000in}}{\pgfqpoint{5.700000in}{5.700000in}}%
\pgfusepath{clip}%
\pgfsetbuttcap%
\pgfsetroundjoin%
\definecolor{currentfill}{rgb}{0.267004,0.004874,0.329415}%
\pgfsetfillcolor{currentfill}%
\pgfsetfillopacity{0.800000}%
\pgfsetlinewidth{0.000000pt}%
\definecolor{currentstroke}{rgb}{0.000000,0.000000,0.000000}%
\pgfsetstrokecolor{currentstroke}%
\pgfsetdash{}{0pt}%
\pgfpathmoveto{\pgfqpoint{3.460487in}{1.317376in}}%
\pgfpathlineto{\pgfqpoint{3.474445in}{1.313784in}}%
\pgfpathlineto{\pgfqpoint{3.488407in}{1.310380in}}%
\pgfpathlineto{\pgfqpoint{3.502374in}{1.307163in}}%
\pgfpathlineto{\pgfqpoint{3.516346in}{1.304133in}}%
\pgfpathlineto{\pgfqpoint{3.524710in}{1.312576in}}%
\pgfpathlineto{\pgfqpoint{3.533065in}{1.321217in}}%
\pgfpathlineto{\pgfqpoint{3.541412in}{1.330047in}}%
\pgfpathlineto{\pgfqpoint{3.549751in}{1.339062in}}%
\pgfpathlineto{\pgfqpoint{3.535798in}{1.341431in}}%
\pgfpathlineto{\pgfqpoint{3.521851in}{1.343987in}}%
\pgfpathlineto{\pgfqpoint{3.507909in}{1.346731in}}%
\pgfpathlineto{\pgfqpoint{3.493972in}{1.349662in}}%
\pgfpathlineto{\pgfqpoint{3.485614in}{1.341296in}}%
\pgfpathlineto{\pgfqpoint{3.477248in}{1.333122in}}%
\pgfpathlineto{\pgfqpoint{3.468872in}{1.325146in}}%
\pgfpathlineto{\pgfqpoint{3.460487in}{1.317376in}}%
\pgfpathclose%
\pgfusepath{fill}%
\end{pgfscope}%
\begin{pgfscope}%
\pgfpathrectangle{\pgfqpoint{1.150000in}{0.150000in}}{\pgfqpoint{5.700000in}{5.700000in}}%
\pgfusepath{clip}%
\pgfsetbuttcap%
\pgfsetroundjoin%
\definecolor{currentfill}{rgb}{0.147607,0.511733,0.557049}%
\pgfsetfillcolor{currentfill}%
\pgfsetfillopacity{0.800000}%
\pgfsetlinewidth{0.000000pt}%
\definecolor{currentstroke}{rgb}{0.000000,0.000000,0.000000}%
\pgfsetstrokecolor{currentstroke}%
\pgfsetdash{}{0pt}%
\pgfpathmoveto{\pgfqpoint{4.812357in}{2.574278in}}%
\pgfpathlineto{\pgfqpoint{4.826870in}{2.587463in}}%
\pgfpathlineto{\pgfqpoint{4.841401in}{2.600834in}}%
\pgfpathlineto{\pgfqpoint{4.855952in}{2.614391in}}%
\pgfpathlineto{\pgfqpoint{4.870521in}{2.628135in}}%
\pgfpathlineto{\pgfqpoint{4.878426in}{2.639843in}}%
\pgfpathlineto{\pgfqpoint{4.886325in}{2.651385in}}%
\pgfpathlineto{\pgfqpoint{4.894215in}{2.662763in}}%
\pgfpathlineto{\pgfqpoint{4.902099in}{2.673975in}}%
\pgfpathlineto{\pgfqpoint{4.887529in}{2.660178in}}%
\pgfpathlineto{\pgfqpoint{4.872979in}{2.646567in}}%
\pgfpathlineto{\pgfqpoint{4.858448in}{2.633142in}}%
\pgfpathlineto{\pgfqpoint{4.843935in}{2.619904in}}%
\pgfpathlineto{\pgfqpoint{4.836051in}{2.608732in}}%
\pgfpathlineto{\pgfqpoint{4.828160in}{2.597403in}}%
\pgfpathlineto{\pgfqpoint{4.820262in}{2.585919in}}%
\pgfpathlineto{\pgfqpoint{4.812357in}{2.574278in}}%
\pgfpathclose%
\pgfusepath{fill}%
\end{pgfscope}%
\begin{pgfscope}%
\pgfpathrectangle{\pgfqpoint{1.150000in}{0.150000in}}{\pgfqpoint{5.700000in}{5.700000in}}%
\pgfusepath{clip}%
\pgfsetbuttcap%
\pgfsetroundjoin%
\definecolor{currentfill}{rgb}{0.122312,0.633153,0.530398}%
\pgfsetfillcolor{currentfill}%
\pgfsetfillopacity{0.800000}%
\pgfsetlinewidth{0.000000pt}%
\definecolor{currentstroke}{rgb}{0.000000,0.000000,0.000000}%
\pgfsetstrokecolor{currentstroke}%
\pgfsetdash{}{0pt}%
\pgfpathmoveto{\pgfqpoint{5.144341in}{2.947255in}}%
\pgfpathlineto{\pgfqpoint{5.159079in}{2.962598in}}%
\pgfpathlineto{\pgfqpoint{5.173839in}{2.978128in}}%
\pgfpathlineto{\pgfqpoint{5.188620in}{2.993846in}}%
\pgfpathlineto{\pgfqpoint{5.203422in}{3.009752in}}%
\pgfpathlineto{\pgfqpoint{5.211152in}{3.017912in}}%
\pgfpathlineto{\pgfqpoint{5.218873in}{3.025895in}}%
\pgfpathlineto{\pgfqpoint{5.226584in}{3.033704in}}%
\pgfpathlineto{\pgfqpoint{5.234286in}{3.041339in}}%
\pgfpathlineto{\pgfqpoint{5.219490in}{3.025557in}}%
\pgfpathlineto{\pgfqpoint{5.204716in}{3.009963in}}%
\pgfpathlineto{\pgfqpoint{5.189963in}{2.994556in}}%
\pgfpathlineto{\pgfqpoint{5.175231in}{2.979337in}}%
\pgfpathlineto{\pgfqpoint{5.167522in}{2.971565in}}%
\pgfpathlineto{\pgfqpoint{5.159804in}{2.963629in}}%
\pgfpathlineto{\pgfqpoint{5.152077in}{2.955526in}}%
\pgfpathlineto{\pgfqpoint{5.144341in}{2.947255in}}%
\pgfpathclose%
\pgfusepath{fill}%
\end{pgfscope}%
\begin{pgfscope}%
\pgfpathrectangle{\pgfqpoint{1.150000in}{0.150000in}}{\pgfqpoint{5.700000in}{5.700000in}}%
\pgfusepath{clip}%
\pgfsetbuttcap%
\pgfsetroundjoin%
\definecolor{currentfill}{rgb}{0.175707,0.697900,0.491033}%
\pgfsetfillcolor{currentfill}%
\pgfsetfillopacity{0.800000}%
\pgfsetlinewidth{0.000000pt}%
\definecolor{currentstroke}{rgb}{0.000000,0.000000,0.000000}%
\pgfsetstrokecolor{currentstroke}%
\pgfsetdash{}{0pt}%
\pgfpathmoveto{\pgfqpoint{5.354882in}{3.160013in}}%
\pgfpathlineto{\pgfqpoint{5.369768in}{3.176379in}}%
\pgfpathlineto{\pgfqpoint{5.384677in}{3.192934in}}%
\pgfpathlineto{\pgfqpoint{5.399609in}{3.209676in}}%
\pgfpathlineto{\pgfqpoint{5.414563in}{3.226608in}}%
\pgfpathlineto{\pgfqpoint{5.422156in}{3.232341in}}%
\pgfpathlineto{\pgfqpoint{5.429738in}{3.237907in}}%
\pgfpathlineto{\pgfqpoint{5.437310in}{3.243308in}}%
\pgfpathlineto{\pgfqpoint{5.444872in}{3.248546in}}%
\pgfpathlineto{\pgfqpoint{5.429930in}{3.231848in}}%
\pgfpathlineto{\pgfqpoint{5.415010in}{3.215338in}}%
\pgfpathlineto{\pgfqpoint{5.400113in}{3.199016in}}%
\pgfpathlineto{\pgfqpoint{5.385238in}{3.182882in}}%
\pgfpathlineto{\pgfqpoint{5.377664in}{3.177398in}}%
\pgfpathlineto{\pgfqpoint{5.370080in}{3.171761in}}%
\pgfpathlineto{\pgfqpoint{5.362486in}{3.165967in}}%
\pgfpathlineto{\pgfqpoint{5.354882in}{3.160013in}}%
\pgfpathclose%
\pgfusepath{fill}%
\end{pgfscope}%
\begin{pgfscope}%
\pgfpathrectangle{\pgfqpoint{1.150000in}{0.150000in}}{\pgfqpoint{5.700000in}{5.700000in}}%
\pgfusepath{clip}%
\pgfsetbuttcap%
\pgfsetroundjoin%
\definecolor{currentfill}{rgb}{0.199430,0.387607,0.554642}%
\pgfsetfillcolor{currentfill}%
\pgfsetfillopacity{0.800000}%
\pgfsetlinewidth{0.000000pt}%
\definecolor{currentstroke}{rgb}{0.000000,0.000000,0.000000}%
\pgfsetstrokecolor{currentstroke}%
\pgfsetdash{}{0pt}%
\pgfpathmoveto{\pgfqpoint{2.287915in}{2.299504in}}%
\pgfpathlineto{\pgfqpoint{2.302272in}{2.276425in}}%
\pgfpathlineto{\pgfqpoint{2.316615in}{2.253636in}}%
\pgfpathlineto{\pgfqpoint{2.330945in}{2.231134in}}%
\pgfpathlineto{\pgfqpoint{2.345263in}{2.208917in}}%
\pgfpathlineto{\pgfqpoint{2.354654in}{2.199396in}}%
\pgfpathlineto{\pgfqpoint{2.364018in}{2.190334in}}%
\pgfpathlineto{\pgfqpoint{2.373355in}{2.181721in}}%
\pgfpathlineto{\pgfqpoint{2.382664in}{2.173550in}}%
\pgfpathlineto{\pgfqpoint{2.368416in}{2.194946in}}%
\pgfpathlineto{\pgfqpoint{2.354156in}{2.216626in}}%
\pgfpathlineto{\pgfqpoint{2.339883in}{2.238590in}}%
\pgfpathlineto{\pgfqpoint{2.325598in}{2.260843in}}%
\pgfpathlineto{\pgfqpoint{2.316220in}{2.269822in}}%
\pgfpathlineto{\pgfqpoint{2.306814in}{2.279253in}}%
\pgfpathlineto{\pgfqpoint{2.297379in}{2.289144in}}%
\pgfpathlineto{\pgfqpoint{2.287915in}{2.299504in}}%
\pgfpathclose%
\pgfusepath{fill}%
\end{pgfscope}%
\begin{pgfscope}%
\pgfpathrectangle{\pgfqpoint{1.150000in}{0.150000in}}{\pgfqpoint{5.700000in}{5.700000in}}%
\pgfusepath{clip}%
\pgfsetbuttcap%
\pgfsetroundjoin%
\definecolor{currentfill}{rgb}{0.165117,0.467423,0.558141}%
\pgfsetfillcolor{currentfill}%
\pgfsetfillopacity{0.800000}%
\pgfsetlinewidth{0.000000pt}%
\definecolor{currentstroke}{rgb}{0.000000,0.000000,0.000000}%
\pgfsetstrokecolor{currentstroke}%
\pgfsetdash{}{0pt}%
\pgfpathmoveto{\pgfqpoint{4.691012in}{2.425526in}}%
\pgfpathlineto{\pgfqpoint{4.705450in}{2.437758in}}%
\pgfpathlineto{\pgfqpoint{4.719905in}{2.450176in}}%
\pgfpathlineto{\pgfqpoint{4.734379in}{2.462780in}}%
\pgfpathlineto{\pgfqpoint{4.748870in}{2.475568in}}%
\pgfpathlineto{\pgfqpoint{4.756830in}{2.488444in}}%
\pgfpathlineto{\pgfqpoint{4.764782in}{2.501168in}}%
\pgfpathlineto{\pgfqpoint{4.772728in}{2.513739in}}%
\pgfpathlineto{\pgfqpoint{4.780668in}{2.526157in}}%
\pgfpathlineto{\pgfqpoint{4.766174in}{2.513245in}}%
\pgfpathlineto{\pgfqpoint{4.751699in}{2.500520in}}%
\pgfpathlineto{\pgfqpoint{4.737242in}{2.487980in}}%
\pgfpathlineto{\pgfqpoint{4.722803in}{2.475625in}}%
\pgfpathlineto{\pgfqpoint{4.714865in}{2.463317in}}%
\pgfpathlineto{\pgfqpoint{4.706920in}{2.450864in}}%
\pgfpathlineto{\pgfqpoint{4.698969in}{2.438267in}}%
\pgfpathlineto{\pgfqpoint{4.691012in}{2.425526in}}%
\pgfpathclose%
\pgfusepath{fill}%
\end{pgfscope}%
\begin{pgfscope}%
\pgfpathrectangle{\pgfqpoint{1.150000in}{0.150000in}}{\pgfqpoint{5.700000in}{5.700000in}}%
\pgfusepath{clip}%
\pgfsetbuttcap%
\pgfsetroundjoin%
\definecolor{currentfill}{rgb}{0.335885,0.777018,0.402049}%
\pgfsetfillcolor{currentfill}%
\pgfsetfillopacity{0.800000}%
\pgfsetlinewidth{0.000000pt}%
\definecolor{currentstroke}{rgb}{0.000000,0.000000,0.000000}%
\pgfsetstrokecolor{currentstroke}%
\pgfsetdash{}{0pt}%
\pgfpathmoveto{\pgfqpoint{5.654779in}{3.432938in}}%
\pgfpathlineto{\pgfqpoint{5.669882in}{3.450414in}}%
\pgfpathlineto{\pgfqpoint{5.685008in}{3.468079in}}%
\pgfpathlineto{\pgfqpoint{5.700159in}{3.485932in}}%
\pgfpathlineto{\pgfqpoint{5.715334in}{3.503975in}}%
\pgfpathlineto{\pgfqpoint{5.722702in}{3.506310in}}%
\pgfpathlineto{\pgfqpoint{5.730060in}{3.508506in}}%
\pgfpathlineto{\pgfqpoint{5.737406in}{3.510568in}}%
\pgfpathlineto{\pgfqpoint{5.744741in}{3.512499in}}%
\pgfpathlineto{\pgfqpoint{5.729588in}{3.494837in}}%
\pgfpathlineto{\pgfqpoint{5.714459in}{3.477364in}}%
\pgfpathlineto{\pgfqpoint{5.699353in}{3.460079in}}%
\pgfpathlineto{\pgfqpoint{5.684272in}{3.442981in}}%
\pgfpathlineto{\pgfqpoint{5.676915in}{3.440658in}}%
\pgfpathlineto{\pgfqpoint{5.669547in}{3.438212in}}%
\pgfpathlineto{\pgfqpoint{5.662168in}{3.435640in}}%
\pgfpathlineto{\pgfqpoint{5.654779in}{3.432938in}}%
\pgfpathclose%
\pgfusepath{fill}%
\end{pgfscope}%
\begin{pgfscope}%
\pgfpathrectangle{\pgfqpoint{1.150000in}{0.150000in}}{\pgfqpoint{5.700000in}{5.700000in}}%
\pgfusepath{clip}%
\pgfsetbuttcap%
\pgfsetroundjoin%
\definecolor{currentfill}{rgb}{0.280894,0.078907,0.402329}%
\pgfsetfillcolor{currentfill}%
\pgfsetfillopacity{0.800000}%
\pgfsetlinewidth{0.000000pt}%
\definecolor{currentstroke}{rgb}{0.000000,0.000000,0.000000}%
\pgfsetstrokecolor{currentstroke}%
\pgfsetdash{}{0pt}%
\pgfpathmoveto{\pgfqpoint{2.966766in}{1.483258in}}%
\pgfpathlineto{\pgfqpoint{2.980779in}{1.471936in}}%
\pgfpathlineto{\pgfqpoint{2.994790in}{1.460823in}}%
\pgfpathlineto{\pgfqpoint{3.008801in}{1.449920in}}%
\pgfpathlineto{\pgfqpoint{3.022809in}{1.439225in}}%
\pgfpathlineto{\pgfqpoint{3.031533in}{1.439339in}}%
\pgfpathlineto{\pgfqpoint{3.040240in}{1.439797in}}%
\pgfpathlineto{\pgfqpoint{3.048931in}{1.440589in}}%
\pgfpathlineto{\pgfqpoint{3.057606in}{1.441708in}}%
\pgfpathlineto{\pgfqpoint{3.043639in}{1.451637in}}%
\pgfpathlineto{\pgfqpoint{3.029670in}{1.461772in}}%
\pgfpathlineto{\pgfqpoint{3.015701in}{1.472116in}}%
\pgfpathlineto{\pgfqpoint{3.001730in}{1.482670in}}%
\pgfpathlineto{\pgfqpoint{2.993014in}{1.482305in}}%
\pgfpathlineto{\pgfqpoint{2.984281in}{1.482276in}}%
\pgfpathlineto{\pgfqpoint{2.975532in}{1.482591in}}%
\pgfpathlineto{\pgfqpoint{2.966766in}{1.483258in}}%
\pgfpathclose%
\pgfusepath{fill}%
\end{pgfscope}%
\begin{pgfscope}%
\pgfpathrectangle{\pgfqpoint{1.150000in}{0.150000in}}{\pgfqpoint{5.700000in}{5.700000in}}%
\pgfusepath{clip}%
\pgfsetbuttcap%
\pgfsetroundjoin%
\definecolor{currentfill}{rgb}{0.185556,0.418570,0.556753}%
\pgfsetfillcolor{currentfill}%
\pgfsetfillopacity{0.800000}%
\pgfsetlinewidth{0.000000pt}%
\definecolor{currentstroke}{rgb}{0.000000,0.000000,0.000000}%
\pgfsetstrokecolor{currentstroke}%
\pgfsetdash{}{0pt}%
\pgfpathmoveto{\pgfqpoint{4.569577in}{2.272864in}}%
\pgfpathlineto{\pgfqpoint{4.583941in}{2.284011in}}%
\pgfpathlineto{\pgfqpoint{4.598323in}{2.295343in}}%
\pgfpathlineto{\pgfqpoint{4.612721in}{2.306859in}}%
\pgfpathlineto{\pgfqpoint{4.627136in}{2.318559in}}%
\pgfpathlineto{\pgfqpoint{4.635141in}{2.332409in}}%
\pgfpathlineto{\pgfqpoint{4.643140in}{2.346124in}}%
\pgfpathlineto{\pgfqpoint{4.651134in}{2.359704in}}%
\pgfpathlineto{\pgfqpoint{4.659122in}{2.373147in}}%
\pgfpathlineto{\pgfqpoint{4.644703in}{2.361256in}}%
\pgfpathlineto{\pgfqpoint{4.630302in}{2.349549in}}%
\pgfpathlineto{\pgfqpoint{4.615919in}{2.338028in}}%
\pgfpathlineto{\pgfqpoint{4.601552in}{2.326691in}}%
\pgfpathlineto{\pgfqpoint{4.593566in}{2.313426in}}%
\pgfpathlineto{\pgfqpoint{4.585575in}{2.300031in}}%
\pgfpathlineto{\pgfqpoint{4.577579in}{2.286510in}}%
\pgfpathlineto{\pgfqpoint{4.569577in}{2.272864in}}%
\pgfpathclose%
\pgfusepath{fill}%
\end{pgfscope}%
\begin{pgfscope}%
\pgfpathrectangle{\pgfqpoint{1.150000in}{0.150000in}}{\pgfqpoint{5.700000in}{5.700000in}}%
\pgfusepath{clip}%
\pgfsetbuttcap%
\pgfsetroundjoin%
\definecolor{currentfill}{rgb}{0.281412,0.155834,0.469201}%
\pgfsetfillcolor{currentfill}%
\pgfsetfillopacity{0.800000}%
\pgfsetlinewidth{0.000000pt}%
\definecolor{currentstroke}{rgb}{0.000000,0.000000,0.000000}%
\pgfsetstrokecolor{currentstroke}%
\pgfsetdash{}{0pt}%
\pgfpathmoveto{\pgfqpoint{3.994487in}{1.596910in}}%
\pgfpathlineto{\pgfqpoint{4.008563in}{1.601237in}}%
\pgfpathlineto{\pgfqpoint{4.022651in}{1.605744in}}%
\pgfpathlineto{\pgfqpoint{4.036750in}{1.610433in}}%
\pgfpathlineto{\pgfqpoint{4.050860in}{1.615304in}}%
\pgfpathlineto{\pgfqpoint{4.059021in}{1.629766in}}%
\pgfpathlineto{\pgfqpoint{4.067177in}{1.644245in}}%
\pgfpathlineto{\pgfqpoint{4.075329in}{1.658737in}}%
\pgfpathlineto{\pgfqpoint{4.083477in}{1.673237in}}%
\pgfpathlineto{\pgfqpoint{4.069368in}{1.667888in}}%
\pgfpathlineto{\pgfqpoint{4.055272in}{1.662721in}}%
\pgfpathlineto{\pgfqpoint{4.041186in}{1.657736in}}%
\pgfpathlineto{\pgfqpoint{4.027113in}{1.652933in}}%
\pgfpathlineto{\pgfqpoint{4.018963in}{1.638898in}}%
\pgfpathlineto{\pgfqpoint{4.010809in}{1.624880in}}%
\pgfpathlineto{\pgfqpoint{4.002650in}{1.610883in}}%
\pgfpathlineto{\pgfqpoint{3.994487in}{1.596910in}}%
\pgfpathclose%
\pgfusepath{fill}%
\end{pgfscope}%
\begin{pgfscope}%
\pgfpathrectangle{\pgfqpoint{1.150000in}{0.150000in}}{\pgfqpoint{5.700000in}{5.700000in}}%
\pgfusepath{clip}%
\pgfsetbuttcap%
\pgfsetroundjoin%
\definecolor{currentfill}{rgb}{0.210503,0.363727,0.552206}%
\pgfsetfillcolor{currentfill}%
\pgfsetfillopacity{0.800000}%
\pgfsetlinewidth{0.000000pt}%
\definecolor{currentstroke}{rgb}{0.000000,0.000000,0.000000}%
\pgfsetstrokecolor{currentstroke}%
\pgfsetdash{}{0pt}%
\pgfpathmoveto{\pgfqpoint{4.448094in}{2.118560in}}%
\pgfpathlineto{\pgfqpoint{4.462389in}{2.128492in}}%
\pgfpathlineto{\pgfqpoint{4.476699in}{2.138607in}}%
\pgfpathlineto{\pgfqpoint{4.491025in}{2.148906in}}%
\pgfpathlineto{\pgfqpoint{4.505367in}{2.159389in}}%
\pgfpathlineto{\pgfqpoint{4.513411in}{2.173974in}}%
\pgfpathlineto{\pgfqpoint{4.521450in}{2.188448in}}%
\pgfpathlineto{\pgfqpoint{4.529485in}{2.202810in}}%
\pgfpathlineto{\pgfqpoint{4.537514in}{2.217058in}}%
\pgfpathlineto{\pgfqpoint{4.523168in}{2.206318in}}%
\pgfpathlineto{\pgfqpoint{4.508839in}{2.195763in}}%
\pgfpathlineto{\pgfqpoint{4.494526in}{2.185392in}}%
\pgfpathlineto{\pgfqpoint{4.480229in}{2.175204in}}%
\pgfpathlineto{\pgfqpoint{4.472203in}{2.161200in}}%
\pgfpathlineto{\pgfqpoint{4.464172in}{2.147090in}}%
\pgfpathlineto{\pgfqpoint{4.456135in}{2.132876in}}%
\pgfpathlineto{\pgfqpoint{4.448094in}{2.118560in}}%
\pgfpathclose%
\pgfusepath{fill}%
\end{pgfscope}%
\begin{pgfscope}%
\pgfpathrectangle{\pgfqpoint{1.150000in}{0.150000in}}{\pgfqpoint{5.700000in}{5.700000in}}%
\pgfusepath{clip}%
\pgfsetbuttcap%
\pgfsetroundjoin%
\definecolor{currentfill}{rgb}{0.137770,0.537492,0.554906}%
\pgfsetfillcolor{currentfill}%
\pgfsetfillopacity{0.800000}%
\pgfsetlinewidth{0.000000pt}%
\definecolor{currentstroke}{rgb}{0.000000,0.000000,0.000000}%
\pgfsetstrokecolor{currentstroke}%
\pgfsetdash{}{0pt}%
\pgfpathmoveto{\pgfqpoint{2.036766in}{2.770487in}}%
\pgfpathlineto{\pgfqpoint{2.051371in}{2.741895in}}%
\pgfpathlineto{\pgfqpoint{2.065957in}{2.713650in}}%
\pgfpathlineto{\pgfqpoint{2.080523in}{2.685748in}}%
\pgfpathlineto{\pgfqpoint{2.095071in}{2.658186in}}%
\pgfpathlineto{\pgfqpoint{2.104718in}{2.646381in}}%
\pgfpathlineto{\pgfqpoint{2.114333in}{2.635054in}}%
\pgfpathlineto{\pgfqpoint{2.123917in}{2.624195in}}%
\pgfpathlineto{\pgfqpoint{2.133470in}{2.613797in}}%
\pgfpathlineto{\pgfqpoint{2.119001in}{2.640540in}}%
\pgfpathlineto{\pgfqpoint{2.104513in}{2.667620in}}%
\pgfpathlineto{\pgfqpoint{2.090007in}{2.695041in}}%
\pgfpathlineto{\pgfqpoint{2.075483in}{2.722806in}}%
\pgfpathlineto{\pgfqpoint{2.065852in}{2.734011in}}%
\pgfpathlineto{\pgfqpoint{2.056189in}{2.745687in}}%
\pgfpathlineto{\pgfqpoint{2.046494in}{2.757843in}}%
\pgfpathlineto{\pgfqpoint{2.036766in}{2.770487in}}%
\pgfpathclose%
\pgfusepath{fill}%
\end{pgfscope}%
\begin{pgfscope}%
\pgfpathrectangle{\pgfqpoint{1.150000in}{0.150000in}}{\pgfqpoint{5.700000in}{5.700000in}}%
\pgfusepath{clip}%
\pgfsetbuttcap%
\pgfsetroundjoin%
\definecolor{currentfill}{rgb}{0.235526,0.309527,0.542944}%
\pgfsetfillcolor{currentfill}%
\pgfsetfillopacity{0.800000}%
\pgfsetlinewidth{0.000000pt}%
\definecolor{currentstroke}{rgb}{0.000000,0.000000,0.000000}%
\pgfsetstrokecolor{currentstroke}%
\pgfsetdash{}{0pt}%
\pgfpathmoveto{\pgfqpoint{4.326588in}{1.965203in}}%
\pgfpathlineto{\pgfqpoint{4.340817in}{1.973791in}}%
\pgfpathlineto{\pgfqpoint{4.355061in}{1.982563in}}%
\pgfpathlineto{\pgfqpoint{4.369319in}{1.991517in}}%
\pgfpathlineto{\pgfqpoint{4.383592in}{2.000655in}}%
\pgfpathlineto{\pgfqpoint{4.391671in}{2.015697in}}%
\pgfpathlineto{\pgfqpoint{4.399746in}{2.030658in}}%
\pgfpathlineto{\pgfqpoint{4.407815in}{2.045535in}}%
\pgfpathlineto{\pgfqpoint{4.415881in}{2.060324in}}%
\pgfpathlineto{\pgfqpoint{4.401604in}{2.050865in}}%
\pgfpathlineto{\pgfqpoint{4.387343in}{2.041589in}}%
\pgfpathlineto{\pgfqpoint{4.373097in}{2.032497in}}%
\pgfpathlineto{\pgfqpoint{4.358865in}{2.023587in}}%
\pgfpathlineto{\pgfqpoint{4.350803in}{2.009106in}}%
\pgfpathlineto{\pgfqpoint{4.342736in}{1.994547in}}%
\pgfpathlineto{\pgfqpoint{4.334664in}{1.979911in}}%
\pgfpathlineto{\pgfqpoint{4.326588in}{1.965203in}}%
\pgfpathclose%
\pgfusepath{fill}%
\end{pgfscope}%
\begin{pgfscope}%
\pgfpathrectangle{\pgfqpoint{1.150000in}{0.150000in}}{\pgfqpoint{5.700000in}{5.700000in}}%
\pgfusepath{clip}%
\pgfsetbuttcap%
\pgfsetroundjoin%
\definecolor{currentfill}{rgb}{0.121148,0.592739,0.544641}%
\pgfsetfillcolor{currentfill}%
\pgfsetfillopacity{0.800000}%
\pgfsetlinewidth{0.000000pt}%
\definecolor{currentstroke}{rgb}{0.000000,0.000000,0.000000}%
\pgfsetstrokecolor{currentstroke}%
\pgfsetdash{}{0pt}%
\pgfpathmoveto{\pgfqpoint{5.023348in}{2.814822in}}%
\pgfpathlineto{\pgfqpoint{5.038013in}{2.829555in}}%
\pgfpathlineto{\pgfqpoint{5.052698in}{2.844475in}}%
\pgfpathlineto{\pgfqpoint{5.067403in}{2.859582in}}%
\pgfpathlineto{\pgfqpoint{5.082129in}{2.874878in}}%
\pgfpathlineto{\pgfqpoint{5.089936in}{2.884540in}}%
\pgfpathlineto{\pgfqpoint{5.097735in}{2.894025in}}%
\pgfpathlineto{\pgfqpoint{5.105525in}{2.903333in}}%
\pgfpathlineto{\pgfqpoint{5.113306in}{2.912465in}}%
\pgfpathlineto{\pgfqpoint{5.098583in}{2.897222in}}%
\pgfpathlineto{\pgfqpoint{5.083881in}{2.882167in}}%
\pgfpathlineto{\pgfqpoint{5.069200in}{2.867298in}}%
\pgfpathlineto{\pgfqpoint{5.054538in}{2.852617in}}%
\pgfpathlineto{\pgfqpoint{5.046753in}{2.843420in}}%
\pgfpathlineto{\pgfqpoint{5.038960in}{2.834056in}}%
\pgfpathlineto{\pgfqpoint{5.031158in}{2.824523in}}%
\pgfpathlineto{\pgfqpoint{5.023348in}{2.814822in}}%
\pgfpathclose%
\pgfusepath{fill}%
\end{pgfscope}%
\begin{pgfscope}%
\pgfpathrectangle{\pgfqpoint{1.150000in}{0.150000in}}{\pgfqpoint{5.700000in}{5.700000in}}%
\pgfusepath{clip}%
\pgfsetbuttcap%
\pgfsetroundjoin%
\definecolor{currentfill}{rgb}{0.267004,0.004874,0.329415}%
\pgfsetfillcolor{currentfill}%
\pgfsetfillopacity{0.800000}%
\pgfsetlinewidth{0.000000pt}%
\definecolor{currentstroke}{rgb}{0.000000,0.000000,0.000000}%
\pgfsetstrokecolor{currentstroke}%
\pgfsetdash{}{0pt}%
\pgfpathmoveto{\pgfqpoint{3.370971in}{1.307508in}}%
\pgfpathlineto{\pgfqpoint{3.384936in}{1.302466in}}%
\pgfpathlineto{\pgfqpoint{3.398905in}{1.297614in}}%
\pgfpathlineto{\pgfqpoint{3.412877in}{1.292952in}}%
\pgfpathlineto{\pgfqpoint{3.426853in}{1.288478in}}%
\pgfpathlineto{\pgfqpoint{3.435277in}{1.295361in}}%
\pgfpathlineto{\pgfqpoint{3.443690in}{1.302476in}}%
\pgfpathlineto{\pgfqpoint{3.452093in}{1.309817in}}%
\pgfpathlineto{\pgfqpoint{3.460487in}{1.317376in}}%
\pgfpathlineto{\pgfqpoint{3.446534in}{1.321156in}}%
\pgfpathlineto{\pgfqpoint{3.432586in}{1.325125in}}%
\pgfpathlineto{\pgfqpoint{3.418642in}{1.329284in}}%
\pgfpathlineto{\pgfqpoint{3.404701in}{1.333633in}}%
\pgfpathlineto{\pgfqpoint{3.396284in}{1.326755in}}%
\pgfpathlineto{\pgfqpoint{3.387857in}{1.320104in}}%
\pgfpathlineto{\pgfqpoint{3.379419in}{1.313686in}}%
\pgfpathlineto{\pgfqpoint{3.370971in}{1.307508in}}%
\pgfpathclose%
\pgfusepath{fill}%
\end{pgfscope}%
\begin{pgfscope}%
\pgfpathrectangle{\pgfqpoint{1.150000in}{0.150000in}}{\pgfqpoint{5.700000in}{5.700000in}}%
\pgfusepath{clip}%
\pgfsetbuttcap%
\pgfsetroundjoin%
\definecolor{currentfill}{rgb}{0.183898,0.422383,0.556944}%
\pgfsetfillcolor{currentfill}%
\pgfsetfillopacity{0.800000}%
\pgfsetlinewidth{0.000000pt}%
\definecolor{currentstroke}{rgb}{0.000000,0.000000,0.000000}%
\pgfsetstrokecolor{currentstroke}%
\pgfsetdash{}{0pt}%
\pgfpathmoveto{\pgfqpoint{2.230352in}{2.394768in}}%
\pgfpathlineto{\pgfqpoint{2.244764in}{2.370505in}}%
\pgfpathlineto{\pgfqpoint{2.259162in}{2.346541in}}%
\pgfpathlineto{\pgfqpoint{2.273545in}{2.322875in}}%
\pgfpathlineto{\pgfqpoint{2.287915in}{2.299504in}}%
\pgfpathlineto{\pgfqpoint{2.297379in}{2.289144in}}%
\pgfpathlineto{\pgfqpoint{2.306814in}{2.279253in}}%
\pgfpathlineto{\pgfqpoint{2.316220in}{2.269822in}}%
\pgfpathlineto{\pgfqpoint{2.325598in}{2.260843in}}%
\pgfpathlineto{\pgfqpoint{2.311300in}{2.283386in}}%
\pgfpathlineto{\pgfqpoint{2.296989in}{2.306222in}}%
\pgfpathlineto{\pgfqpoint{2.282664in}{2.329353in}}%
\pgfpathlineto{\pgfqpoint{2.268326in}{2.352782in}}%
\pgfpathlineto{\pgfqpoint{2.258876in}{2.362576in}}%
\pgfpathlineto{\pgfqpoint{2.249398in}{2.372833in}}%
\pgfpathlineto{\pgfqpoint{2.239890in}{2.383561in}}%
\pgfpathlineto{\pgfqpoint{2.230352in}{2.394768in}}%
\pgfpathclose%
\pgfusepath{fill}%
\end{pgfscope}%
\begin{pgfscope}%
\pgfpathrectangle{\pgfqpoint{1.150000in}{0.150000in}}{\pgfqpoint{5.700000in}{5.700000in}}%
\pgfusepath{clip}%
\pgfsetbuttcap%
\pgfsetroundjoin%
\definecolor{currentfill}{rgb}{0.276022,0.044167,0.370164}%
\pgfsetfillcolor{currentfill}%
\pgfsetfillopacity{0.800000}%
\pgfsetlinewidth{0.000000pt}%
\definecolor{currentstroke}{rgb}{0.000000,0.000000,0.000000}%
\pgfsetstrokecolor{currentstroke}%
\pgfsetdash{}{0pt}%
\pgfpathmoveto{\pgfqpoint{3.694743in}{1.369597in}}%
\pgfpathlineto{\pgfqpoint{3.708739in}{1.369522in}}%
\pgfpathlineto{\pgfqpoint{3.722743in}{1.369629in}}%
\pgfpathlineto{\pgfqpoint{3.736754in}{1.369920in}}%
\pgfpathlineto{\pgfqpoint{3.750773in}{1.370393in}}%
\pgfpathlineto{\pgfqpoint{3.759033in}{1.382065in}}%
\pgfpathlineto{\pgfqpoint{3.767287in}{1.393860in}}%
\pgfpathlineto{\pgfqpoint{3.775535in}{1.405773in}}%
\pgfpathlineto{\pgfqpoint{3.783777in}{1.417797in}}%
\pgfpathlineto{\pgfqpoint{3.769768in}{1.416725in}}%
\pgfpathlineto{\pgfqpoint{3.755768in}{1.415835in}}%
\pgfpathlineto{\pgfqpoint{3.741776in}{1.415128in}}%
\pgfpathlineto{\pgfqpoint{3.727791in}{1.414604in}}%
\pgfpathlineto{\pgfqpoint{3.719539in}{1.403167in}}%
\pgfpathlineto{\pgfqpoint{3.711280in}{1.391849in}}%
\pgfpathlineto{\pgfqpoint{3.703015in}{1.380657in}}%
\pgfpathlineto{\pgfqpoint{3.694743in}{1.369597in}}%
\pgfpathclose%
\pgfusepath{fill}%
\end{pgfscope}%
\begin{pgfscope}%
\pgfpathrectangle{\pgfqpoint{1.150000in}{0.150000in}}{\pgfqpoint{5.700000in}{5.700000in}}%
\pgfusepath{clip}%
\pgfsetbuttcap%
\pgfsetroundjoin%
\definecolor{currentfill}{rgb}{0.269944,0.014625,0.341379}%
\pgfsetfillcolor{currentfill}%
\pgfsetfillopacity{0.800000}%
\pgfsetlinewidth{0.000000pt}%
\definecolor{currentstroke}{rgb}{0.000000,0.000000,0.000000}%
\pgfsetstrokecolor{currentstroke}%
\pgfsetdash{}{0pt}%
\pgfpathmoveto{\pgfqpoint{3.225222in}{1.338413in}}%
\pgfpathlineto{\pgfqpoint{3.239196in}{1.331100in}}%
\pgfpathlineto{\pgfqpoint{3.253172in}{1.323982in}}%
\pgfpathlineto{\pgfqpoint{3.267149in}{1.317059in}}%
\pgfpathlineto{\pgfqpoint{3.281128in}{1.310331in}}%
\pgfpathlineto{\pgfqpoint{3.289650in}{1.314736in}}%
\pgfpathlineto{\pgfqpoint{3.298160in}{1.319420in}}%
\pgfpathlineto{\pgfqpoint{3.306658in}{1.324373in}}%
\pgfpathlineto{\pgfqpoint{3.315144in}{1.329590in}}%
\pgfpathlineto{\pgfqpoint{3.301194in}{1.335591in}}%
\pgfpathlineto{\pgfqpoint{3.287247in}{1.341787in}}%
\pgfpathlineto{\pgfqpoint{3.273302in}{1.348178in}}%
\pgfpathlineto{\pgfqpoint{3.259360in}{1.354763in}}%
\pgfpathlineto{\pgfqpoint{3.250844in}{1.350262in}}%
\pgfpathlineto{\pgfqpoint{3.242316in}{1.346031in}}%
\pgfpathlineto{\pgfqpoint{3.233776in}{1.342079in}}%
\pgfpathlineto{\pgfqpoint{3.225222in}{1.338413in}}%
\pgfpathclose%
\pgfusepath{fill}%
\end{pgfscope}%
\begin{pgfscope}%
\pgfpathrectangle{\pgfqpoint{1.150000in}{0.150000in}}{\pgfqpoint{5.700000in}{5.700000in}}%
\pgfusepath{clip}%
\pgfsetbuttcap%
\pgfsetroundjoin%
\definecolor{currentfill}{rgb}{0.258965,0.251537,0.524736}%
\pgfsetfillcolor{currentfill}%
\pgfsetfillopacity{0.800000}%
\pgfsetlinewidth{0.000000pt}%
\definecolor{currentstroke}{rgb}{0.000000,0.000000,0.000000}%
\pgfsetstrokecolor{currentstroke}%
\pgfsetdash{}{0pt}%
\pgfpathmoveto{\pgfqpoint{4.205057in}{1.815691in}}%
\pgfpathlineto{\pgfqpoint{4.219228in}{1.822812in}}%
\pgfpathlineto{\pgfqpoint{4.233412in}{1.830115in}}%
\pgfpathlineto{\pgfqpoint{4.247609in}{1.837601in}}%
\pgfpathlineto{\pgfqpoint{4.261820in}{1.845268in}}%
\pgfpathlineto{\pgfqpoint{4.269931in}{1.860449in}}%
\pgfpathlineto{\pgfqpoint{4.278038in}{1.875584in}}%
\pgfpathlineto{\pgfqpoint{4.286140in}{1.890667in}}%
\pgfpathlineto{\pgfqpoint{4.294238in}{1.905696in}}%
\pgfpathlineto{\pgfqpoint{4.280025in}{1.897643in}}%
\pgfpathlineto{\pgfqpoint{4.265827in}{1.889773in}}%
\pgfpathlineto{\pgfqpoint{4.251641in}{1.882085in}}%
\pgfpathlineto{\pgfqpoint{4.237470in}{1.874579in}}%
\pgfpathlineto{\pgfqpoint{4.229373in}{1.859923in}}%
\pgfpathlineto{\pgfqpoint{4.221272in}{1.845221in}}%
\pgfpathlineto{\pgfqpoint{4.213167in}{1.830476in}}%
\pgfpathlineto{\pgfqpoint{4.205057in}{1.815691in}}%
\pgfpathclose%
\pgfusepath{fill}%
\end{pgfscope}%
\begin{pgfscope}%
\pgfpathrectangle{\pgfqpoint{1.150000in}{0.150000in}}{\pgfqpoint{5.700000in}{5.700000in}}%
\pgfusepath{clip}%
\pgfsetbuttcap%
\pgfsetroundjoin%
\definecolor{currentfill}{rgb}{0.272594,0.025563,0.353093}%
\pgfsetfillcolor{currentfill}%
\pgfsetfillopacity{0.800000}%
\pgfsetlinewidth{0.000000pt}%
\definecolor{currentstroke}{rgb}{0.000000,0.000000,0.000000}%
\pgfsetstrokecolor{currentstroke}%
\pgfsetdash{}{0pt}%
\pgfpathmoveto{\pgfqpoint{3.605620in}{1.331447in}}%
\pgfpathlineto{\pgfqpoint{3.619603in}{1.330006in}}%
\pgfpathlineto{\pgfqpoint{3.633592in}{1.328749in}}%
\pgfpathlineto{\pgfqpoint{3.647587in}{1.327677in}}%
\pgfpathlineto{\pgfqpoint{3.661590in}{1.326787in}}%
\pgfpathlineto{\pgfqpoint{3.669889in}{1.337263in}}%
\pgfpathlineto{\pgfqpoint{3.678180in}{1.347894in}}%
\pgfpathlineto{\pgfqpoint{3.686465in}{1.358674in}}%
\pgfpathlineto{\pgfqpoint{3.694743in}{1.369597in}}%
\pgfpathlineto{\pgfqpoint{3.680755in}{1.369856in}}%
\pgfpathlineto{\pgfqpoint{3.666773in}{1.370298in}}%
\pgfpathlineto{\pgfqpoint{3.652799in}{1.370925in}}%
\pgfpathlineto{\pgfqpoint{3.638831in}{1.371736in}}%
\pgfpathlineto{\pgfqpoint{3.630539in}{1.361431in}}%
\pgfpathlineto{\pgfqpoint{3.622240in}{1.351277in}}%
\pgfpathlineto{\pgfqpoint{3.613934in}{1.341280in}}%
\pgfpathlineto{\pgfqpoint{3.605620in}{1.331447in}}%
\pgfpathclose%
\pgfusepath{fill}%
\end{pgfscope}%
\begin{pgfscope}%
\pgfpathrectangle{\pgfqpoint{1.150000in}{0.150000in}}{\pgfqpoint{5.700000in}{5.700000in}}%
\pgfusepath{clip}%
\pgfsetbuttcap%
\pgfsetroundjoin%
\definecolor{currentfill}{rgb}{0.280267,0.073417,0.397163}%
\pgfsetfillcolor{currentfill}%
\pgfsetfillopacity{0.800000}%
\pgfsetlinewidth{0.000000pt}%
\definecolor{currentstroke}{rgb}{0.000000,0.000000,0.000000}%
\pgfsetstrokecolor{currentstroke}%
\pgfsetdash{}{0pt}%
\pgfpathmoveto{\pgfqpoint{3.783777in}{1.417797in}}%
\pgfpathlineto{\pgfqpoint{3.797794in}{1.419053in}}%
\pgfpathlineto{\pgfqpoint{3.811819in}{1.420490in}}%
\pgfpathlineto{\pgfqpoint{3.825854in}{1.422109in}}%
\pgfpathlineto{\pgfqpoint{3.839897in}{1.423910in}}%
\pgfpathlineto{\pgfqpoint{3.848124in}{1.436622in}}%
\pgfpathlineto{\pgfqpoint{3.856346in}{1.449427in}}%
\pgfpathlineto{\pgfqpoint{3.864563in}{1.462320in}}%
\pgfpathlineto{\pgfqpoint{3.872775in}{1.475294in}}%
\pgfpathlineto{\pgfqpoint{3.858739in}{1.472923in}}%
\pgfpathlineto{\pgfqpoint{3.844713in}{1.470735in}}%
\pgfpathlineto{\pgfqpoint{3.830696in}{1.468728in}}%
\pgfpathlineto{\pgfqpoint{3.816687in}{1.466904in}}%
\pgfpathlineto{\pgfqpoint{3.808468in}{1.454487in}}%
\pgfpathlineto{\pgfqpoint{3.800243in}{1.442160in}}%
\pgfpathlineto{\pgfqpoint{3.792013in}{1.429928in}}%
\pgfpathlineto{\pgfqpoint{3.783777in}{1.417797in}}%
\pgfpathclose%
\pgfusepath{fill}%
\end{pgfscope}%
\begin{pgfscope}%
\pgfpathrectangle{\pgfqpoint{1.150000in}{0.150000in}}{\pgfqpoint{5.700000in}{5.700000in}}%
\pgfusepath{clip}%
\pgfsetbuttcap%
\pgfsetroundjoin%
\definecolor{currentfill}{rgb}{0.278791,0.062145,0.386592}%
\pgfsetfillcolor{currentfill}%
\pgfsetfillopacity{0.800000}%
\pgfsetlinewidth{0.000000pt}%
\definecolor{currentstroke}{rgb}{0.000000,0.000000,0.000000}%
\pgfsetstrokecolor{currentstroke}%
\pgfsetdash{}{0pt}%
\pgfpathmoveto{\pgfqpoint{3.022809in}{1.439225in}}%
\pgfpathlineto{\pgfqpoint{3.036817in}{1.428736in}}%
\pgfpathlineto{\pgfqpoint{3.050823in}{1.418454in}}%
\pgfpathlineto{\pgfqpoint{3.064829in}{1.408376in}}%
\pgfpathlineto{\pgfqpoint{3.078834in}{1.398503in}}%
\pgfpathlineto{\pgfqpoint{3.087517in}{1.399397in}}%
\pgfpathlineto{\pgfqpoint{3.096184in}{1.400624in}}%
\pgfpathlineto{\pgfqpoint{3.104836in}{1.402178in}}%
\pgfpathlineto{\pgfqpoint{3.113473in}{1.404050in}}%
\pgfpathlineto{\pgfqpoint{3.099507in}{1.413158in}}%
\pgfpathlineto{\pgfqpoint{3.085540in}{1.422470in}}%
\pgfpathlineto{\pgfqpoint{3.071574in}{1.431987in}}%
\pgfpathlineto{\pgfqpoint{3.057606in}{1.441708in}}%
\pgfpathlineto{\pgfqpoint{3.048931in}{1.440589in}}%
\pgfpathlineto{\pgfqpoint{3.040240in}{1.439797in}}%
\pgfpathlineto{\pgfqpoint{3.031533in}{1.439339in}}%
\pgfpathlineto{\pgfqpoint{3.022809in}{1.439225in}}%
\pgfpathclose%
\pgfusepath{fill}%
\end{pgfscope}%
\begin{pgfscope}%
\pgfpathrectangle{\pgfqpoint{1.150000in}{0.150000in}}{\pgfqpoint{5.700000in}{5.700000in}}%
\pgfusepath{clip}%
\pgfsetbuttcap%
\pgfsetroundjoin%
\definecolor{currentfill}{rgb}{0.226397,0.728888,0.462789}%
\pgfsetfillcolor{currentfill}%
\pgfsetfillopacity{0.800000}%
\pgfsetlinewidth{0.000000pt}%
\definecolor{currentstroke}{rgb}{0.000000,0.000000,0.000000}%
\pgfsetstrokecolor{currentstroke}%
\pgfsetdash{}{0pt}%
\pgfpathmoveto{\pgfqpoint{5.444872in}{3.248546in}}%
\pgfpathlineto{\pgfqpoint{5.459836in}{3.265433in}}%
\pgfpathlineto{\pgfqpoint{5.474824in}{3.282508in}}%
\pgfpathlineto{\pgfqpoint{5.489835in}{3.299772in}}%
\pgfpathlineto{\pgfqpoint{5.504869in}{3.317225in}}%
\pgfpathlineto{\pgfqpoint{5.512406in}{3.322047in}}%
\pgfpathlineto{\pgfqpoint{5.519933in}{3.326703in}}%
\pgfpathlineto{\pgfqpoint{5.527449in}{3.331198in}}%
\pgfpathlineto{\pgfqpoint{5.534953in}{3.335533in}}%
\pgfpathlineto{\pgfqpoint{5.519934in}{3.318351in}}%
\pgfpathlineto{\pgfqpoint{5.504937in}{3.301357in}}%
\pgfpathlineto{\pgfqpoint{5.489964in}{3.284552in}}%
\pgfpathlineto{\pgfqpoint{5.475013in}{3.267934in}}%
\pgfpathlineto{\pgfqpoint{5.467493in}{3.263316in}}%
\pgfpathlineto{\pgfqpoint{5.459963in}{3.258547in}}%
\pgfpathlineto{\pgfqpoint{5.452423in}{3.253625in}}%
\pgfpathlineto{\pgfqpoint{5.444872in}{3.248546in}}%
\pgfpathclose%
\pgfusepath{fill}%
\end{pgfscope}%
\begin{pgfscope}%
\pgfpathrectangle{\pgfqpoint{1.150000in}{0.150000in}}{\pgfqpoint{5.700000in}{5.700000in}}%
\pgfusepath{clip}%
\pgfsetbuttcap%
\pgfsetroundjoin%
\definecolor{currentfill}{rgb}{0.395174,0.797475,0.367757}%
\pgfsetfillcolor{currentfill}%
\pgfsetfillopacity{0.800000}%
\pgfsetlinewidth{0.000000pt}%
\definecolor{currentstroke}{rgb}{0.000000,0.000000,0.000000}%
\pgfsetstrokecolor{currentstroke}%
\pgfsetdash{}{0pt}%
\pgfpathmoveto{\pgfqpoint{5.744741in}{3.512499in}}%
\pgfpathlineto{\pgfqpoint{5.759919in}{3.530349in}}%
\pgfpathlineto{\pgfqpoint{5.775122in}{3.548389in}}%
\pgfpathlineto{\pgfqpoint{5.790349in}{3.566617in}}%
\pgfpathlineto{\pgfqpoint{5.805601in}{3.585036in}}%
\pgfpathlineto{\pgfqpoint{5.812902in}{3.586436in}}%
\pgfpathlineto{\pgfqpoint{5.820191in}{3.587706in}}%
\pgfpathlineto{\pgfqpoint{5.827469in}{3.588851in}}%
\pgfpathlineto{\pgfqpoint{5.834736in}{3.589875in}}%
\pgfpathlineto{\pgfqpoint{5.819509in}{3.571876in}}%
\pgfpathlineto{\pgfqpoint{5.804306in}{3.554066in}}%
\pgfpathlineto{\pgfqpoint{5.789127in}{3.536444in}}%
\pgfpathlineto{\pgfqpoint{5.773973in}{3.519009in}}%
\pgfpathlineto{\pgfqpoint{5.766681in}{3.517555in}}%
\pgfpathlineto{\pgfqpoint{5.759379in}{3.515988in}}%
\pgfpathlineto{\pgfqpoint{5.752066in}{3.514304in}}%
\pgfpathlineto{\pgfqpoint{5.744741in}{3.512499in}}%
\pgfpathclose%
\pgfusepath{fill}%
\end{pgfscope}%
\begin{pgfscope}%
\pgfpathrectangle{\pgfqpoint{1.150000in}{0.150000in}}{\pgfqpoint{5.700000in}{5.700000in}}%
\pgfusepath{clip}%
\pgfsetbuttcap%
\pgfsetroundjoin%
\definecolor{currentfill}{rgb}{0.140210,0.665859,0.513427}%
\pgfsetfillcolor{currentfill}%
\pgfsetfillopacity{0.800000}%
\pgfsetlinewidth{0.000000pt}%
\definecolor{currentstroke}{rgb}{0.000000,0.000000,0.000000}%
\pgfsetstrokecolor{currentstroke}%
\pgfsetdash{}{0pt}%
\pgfpathmoveto{\pgfqpoint{5.234286in}{3.041339in}}%
\pgfpathlineto{\pgfqpoint{5.249103in}{3.057309in}}%
\pgfpathlineto{\pgfqpoint{5.263942in}{3.073467in}}%
\pgfpathlineto{\pgfqpoint{5.278803in}{3.089813in}}%
\pgfpathlineto{\pgfqpoint{5.293686in}{3.106348in}}%
\pgfpathlineto{\pgfqpoint{5.301370in}{3.113665in}}%
\pgfpathlineto{\pgfqpoint{5.309045in}{3.120804in}}%
\pgfpathlineto{\pgfqpoint{5.316709in}{3.127766in}}%
\pgfpathlineto{\pgfqpoint{5.324364in}{3.134554in}}%
\pgfpathlineto{\pgfqpoint{5.309489in}{3.118180in}}%
\pgfpathlineto{\pgfqpoint{5.294637in}{3.101994in}}%
\pgfpathlineto{\pgfqpoint{5.279806in}{3.085997in}}%
\pgfpathlineto{\pgfqpoint{5.264997in}{3.070186in}}%
\pgfpathlineto{\pgfqpoint{5.257334in}{3.063225in}}%
\pgfpathlineto{\pgfqpoint{5.249661in}{3.056098in}}%
\pgfpathlineto{\pgfqpoint{5.241978in}{3.048803in}}%
\pgfpathlineto{\pgfqpoint{5.234286in}{3.041339in}}%
\pgfpathclose%
\pgfusepath{fill}%
\end{pgfscope}%
\begin{pgfscope}%
\pgfpathrectangle{\pgfqpoint{1.150000in}{0.150000in}}{\pgfqpoint{5.700000in}{5.700000in}}%
\pgfusepath{clip}%
\pgfsetbuttcap%
\pgfsetroundjoin%
\definecolor{currentfill}{rgb}{0.132444,0.552216,0.553018}%
\pgfsetfillcolor{currentfill}%
\pgfsetfillopacity{0.800000}%
\pgfsetlinewidth{0.000000pt}%
\definecolor{currentstroke}{rgb}{0.000000,0.000000,0.000000}%
\pgfsetstrokecolor{currentstroke}%
\pgfsetdash{}{0pt}%
\pgfpathmoveto{\pgfqpoint{4.902099in}{2.673975in}}%
\pgfpathlineto{\pgfqpoint{4.916687in}{2.687959in}}%
\pgfpathlineto{\pgfqpoint{4.931295in}{2.702129in}}%
\pgfpathlineto{\pgfqpoint{4.945923in}{2.716487in}}%
\pgfpathlineto{\pgfqpoint{4.960570in}{2.731031in}}%
\pgfpathlineto{\pgfqpoint{4.968446in}{2.742110in}}%
\pgfpathlineto{\pgfqpoint{4.976313in}{2.753015in}}%
\pgfpathlineto{\pgfqpoint{4.984173in}{2.763747in}}%
\pgfpathlineto{\pgfqpoint{4.992024in}{2.774306in}}%
\pgfpathlineto{\pgfqpoint{4.977377in}{2.759744in}}%
\pgfpathlineto{\pgfqpoint{4.962750in}{2.745368in}}%
\pgfpathlineto{\pgfqpoint{4.948143in}{2.731179in}}%
\pgfpathlineto{\pgfqpoint{4.933556in}{2.717177in}}%
\pgfpathlineto{\pgfqpoint{4.925703in}{2.706623in}}%
\pgfpathlineto{\pgfqpoint{4.917842in}{2.695905in}}%
\pgfpathlineto{\pgfqpoint{4.909974in}{2.685023in}}%
\pgfpathlineto{\pgfqpoint{4.902099in}{2.673975in}}%
\pgfpathclose%
\pgfusepath{fill}%
\end{pgfscope}%
\begin{pgfscope}%
\pgfpathrectangle{\pgfqpoint{1.150000in}{0.150000in}}{\pgfqpoint{5.700000in}{5.700000in}}%
\pgfusepath{clip}%
\pgfsetbuttcap%
\pgfsetroundjoin%
\definecolor{currentfill}{rgb}{0.275191,0.194905,0.496005}%
\pgfsetfillcolor{currentfill}%
\pgfsetfillopacity{0.800000}%
\pgfsetlinewidth{0.000000pt}%
\definecolor{currentstroke}{rgb}{0.000000,0.000000,0.000000}%
\pgfsetstrokecolor{currentstroke}%
\pgfsetdash{}{0pt}%
\pgfpathmoveto{\pgfqpoint{4.083477in}{1.673237in}}%
\pgfpathlineto{\pgfqpoint{4.097597in}{1.678767in}}%
\pgfpathlineto{\pgfqpoint{4.111729in}{1.684479in}}%
\pgfpathlineto{\pgfqpoint{4.125874in}{1.690372in}}%
\pgfpathlineto{\pgfqpoint{4.140030in}{1.696446in}}%
\pgfpathlineto{\pgfqpoint{4.148173in}{1.711410in}}%
\pgfpathlineto{\pgfqpoint{4.156312in}{1.726365in}}%
\pgfpathlineto{\pgfqpoint{4.164447in}{1.741307in}}%
\pgfpathlineto{\pgfqpoint{4.172577in}{1.756233in}}%
\pgfpathlineto{\pgfqpoint{4.158421in}{1.749710in}}%
\pgfpathlineto{\pgfqpoint{4.144276in}{1.743369in}}%
\pgfpathlineto{\pgfqpoint{4.130145in}{1.737210in}}%
\pgfpathlineto{\pgfqpoint{4.116025in}{1.731232in}}%
\pgfpathlineto{\pgfqpoint{4.107894in}{1.716742in}}%
\pgfpathlineto{\pgfqpoint{4.099760in}{1.702244in}}%
\pgfpathlineto{\pgfqpoint{4.091620in}{1.687740in}}%
\pgfpathlineto{\pgfqpoint{4.083477in}{1.673237in}}%
\pgfpathclose%
\pgfusepath{fill}%
\end{pgfscope}%
\begin{pgfscope}%
\pgfpathrectangle{\pgfqpoint{1.150000in}{0.150000in}}{\pgfqpoint{5.700000in}{5.700000in}}%
\pgfusepath{clip}%
\pgfsetbuttcap%
\pgfsetroundjoin%
\definecolor{currentfill}{rgb}{0.282910,0.105393,0.426902}%
\pgfsetfillcolor{currentfill}%
\pgfsetfillopacity{0.800000}%
\pgfsetlinewidth{0.000000pt}%
\definecolor{currentstroke}{rgb}{0.000000,0.000000,0.000000}%
\pgfsetstrokecolor{currentstroke}%
\pgfsetdash{}{0pt}%
\pgfpathmoveto{\pgfqpoint{3.872775in}{1.475294in}}%
\pgfpathlineto{\pgfqpoint{3.886820in}{1.477846in}}%
\pgfpathlineto{\pgfqpoint{3.900875in}{1.480580in}}%
\pgfpathlineto{\pgfqpoint{3.914939in}{1.483495in}}%
\pgfpathlineto{\pgfqpoint{3.929014in}{1.486591in}}%
\pgfpathlineto{\pgfqpoint{3.937215in}{1.500194in}}%
\pgfpathlineto{\pgfqpoint{3.945410in}{1.513860in}}%
\pgfpathlineto{\pgfqpoint{3.953601in}{1.527584in}}%
\pgfpathlineto{\pgfqpoint{3.961788in}{1.541362in}}%
\pgfpathlineto{\pgfqpoint{3.947718in}{1.537726in}}%
\pgfpathlineto{\pgfqpoint{3.933659in}{1.534271in}}%
\pgfpathlineto{\pgfqpoint{3.919610in}{1.530998in}}%
\pgfpathlineto{\pgfqpoint{3.905570in}{1.527907in}}%
\pgfpathlineto{\pgfqpoint{3.897379in}{1.514657in}}%
\pgfpathlineto{\pgfqpoint{3.889183in}{1.501468in}}%
\pgfpathlineto{\pgfqpoint{3.880981in}{1.488345in}}%
\pgfpathlineto{\pgfqpoint{3.872775in}{1.475294in}}%
\pgfpathclose%
\pgfusepath{fill}%
\end{pgfscope}%
\begin{pgfscope}%
\pgfpathrectangle{\pgfqpoint{1.150000in}{0.150000in}}{\pgfqpoint{5.700000in}{5.700000in}}%
\pgfusepath{clip}%
\pgfsetbuttcap%
\pgfsetroundjoin%
\definecolor{currentfill}{rgb}{0.268510,0.009605,0.335427}%
\pgfsetfillcolor{currentfill}%
\pgfsetfillopacity{0.800000}%
\pgfsetlinewidth{0.000000pt}%
\definecolor{currentstroke}{rgb}{0.000000,0.000000,0.000000}%
\pgfsetstrokecolor{currentstroke}%
\pgfsetdash{}{0pt}%
\pgfpathmoveto{\pgfqpoint{3.516346in}{1.304133in}}%
\pgfpathlineto{\pgfqpoint{3.530323in}{1.301290in}}%
\pgfpathlineto{\pgfqpoint{3.544305in}{1.298632in}}%
\pgfpathlineto{\pgfqpoint{3.558294in}{1.296160in}}%
\pgfpathlineto{\pgfqpoint{3.572287in}{1.293873in}}%
\pgfpathlineto{\pgfqpoint{3.580632in}{1.302990in}}%
\pgfpathlineto{\pgfqpoint{3.588969in}{1.312295in}}%
\pgfpathlineto{\pgfqpoint{3.597299in}{1.321783in}}%
\pgfpathlineto{\pgfqpoint{3.605620in}{1.331447in}}%
\pgfpathlineto{\pgfqpoint{3.591644in}{1.333072in}}%
\pgfpathlineto{\pgfqpoint{3.577674in}{1.334883in}}%
\pgfpathlineto{\pgfqpoint{3.563709in}{1.336880in}}%
\pgfpathlineto{\pgfqpoint{3.549751in}{1.339062in}}%
\pgfpathlineto{\pgfqpoint{3.541412in}{1.330047in}}%
\pgfpathlineto{\pgfqpoint{3.533065in}{1.321217in}}%
\pgfpathlineto{\pgfqpoint{3.524710in}{1.312576in}}%
\pgfpathlineto{\pgfqpoint{3.516346in}{1.304133in}}%
\pgfpathclose%
\pgfusepath{fill}%
\end{pgfscope}%
\begin{pgfscope}%
\pgfpathrectangle{\pgfqpoint{1.150000in}{0.150000in}}{\pgfqpoint{5.700000in}{5.700000in}}%
\pgfusepath{clip}%
\pgfsetbuttcap%
\pgfsetroundjoin%
\definecolor{currentfill}{rgb}{0.458674,0.816363,0.329727}%
\pgfsetfillcolor{currentfill}%
\pgfsetfillopacity{0.800000}%
\pgfsetlinewidth{0.000000pt}%
\definecolor{currentstroke}{rgb}{0.000000,0.000000,0.000000}%
\pgfsetstrokecolor{currentstroke}%
\pgfsetdash{}{0pt}%
\pgfpathmoveto{\pgfqpoint{5.834736in}{3.589875in}}%
\pgfpathlineto{\pgfqpoint{5.849989in}{3.608063in}}%
\pgfpathlineto{\pgfqpoint{5.865266in}{3.626440in}}%
\pgfpathlineto{\pgfqpoint{5.880569in}{3.645007in}}%
\pgfpathlineto{\pgfqpoint{5.887805in}{3.645583in}}%
\pgfpathlineto{\pgfqpoint{5.895030in}{3.646042in}}%
\pgfpathlineto{\pgfqpoint{5.902244in}{3.646389in}}%
\pgfpathlineto{\pgfqpoint{5.909447in}{3.646628in}}%
\pgfpathlineto{\pgfqpoint{5.894171in}{3.628517in}}%
\pgfpathlineto{\pgfqpoint{5.878920in}{3.610595in}}%
\pgfpathlineto{\pgfqpoint{5.863694in}{3.592861in}}%
\pgfpathlineto{\pgfqpoint{5.856471in}{3.592271in}}%
\pgfpathlineto{\pgfqpoint{5.849237in}{3.591581in}}%
\pgfpathlineto{\pgfqpoint{5.841992in}{3.590783in}}%
\pgfpathlineto{\pgfqpoint{5.834736in}{3.589875in}}%
\pgfpathclose%
\pgfusepath{fill}%
\end{pgfscope}%
\begin{pgfscope}%
\pgfpathrectangle{\pgfqpoint{1.150000in}{0.150000in}}{\pgfqpoint{5.700000in}{5.700000in}}%
\pgfusepath{clip}%
\pgfsetbuttcap%
\pgfsetroundjoin%
\definecolor{currentfill}{rgb}{0.150476,0.504369,0.557430}%
\pgfsetfillcolor{currentfill}%
\pgfsetfillopacity{0.800000}%
\pgfsetlinewidth{0.000000pt}%
\definecolor{currentstroke}{rgb}{0.000000,0.000000,0.000000}%
\pgfsetstrokecolor{currentstroke}%
\pgfsetdash{}{0pt}%
\pgfpathmoveto{\pgfqpoint{4.780668in}{2.526157in}}%
\pgfpathlineto{\pgfqpoint{4.795179in}{2.539254in}}%
\pgfpathlineto{\pgfqpoint{4.809710in}{2.552537in}}%
\pgfpathlineto{\pgfqpoint{4.824259in}{2.566006in}}%
\pgfpathlineto{\pgfqpoint{4.838827in}{2.579662in}}%
\pgfpathlineto{\pgfqpoint{4.846761in}{2.592026in}}%
\pgfpathlineto{\pgfqpoint{4.854688in}{2.604226in}}%
\pgfpathlineto{\pgfqpoint{4.862608in}{2.616263in}}%
\pgfpathlineto{\pgfqpoint{4.870521in}{2.628135in}}%
\pgfpathlineto{\pgfqpoint{4.855952in}{2.614391in}}%
\pgfpathlineto{\pgfqpoint{4.841401in}{2.600834in}}%
\pgfpathlineto{\pgfqpoint{4.826870in}{2.587463in}}%
\pgfpathlineto{\pgfqpoint{4.812357in}{2.574278in}}%
\pgfpathlineto{\pgfqpoint{4.804445in}{2.562481in}}%
\pgfpathlineto{\pgfqpoint{4.796526in}{2.550528in}}%
\pgfpathlineto{\pgfqpoint{4.788600in}{2.538420in}}%
\pgfpathlineto{\pgfqpoint{4.780668in}{2.526157in}}%
\pgfpathclose%
\pgfusepath{fill}%
\end{pgfscope}%
\begin{pgfscope}%
\pgfpathrectangle{\pgfqpoint{1.150000in}{0.150000in}}{\pgfqpoint{5.700000in}{5.700000in}}%
\pgfusepath{clip}%
\pgfsetbuttcap%
\pgfsetroundjoin%
\definecolor{currentfill}{rgb}{0.169646,0.456262,0.558030}%
\pgfsetfillcolor{currentfill}%
\pgfsetfillopacity{0.800000}%
\pgfsetlinewidth{0.000000pt}%
\definecolor{currentstroke}{rgb}{0.000000,0.000000,0.000000}%
\pgfsetstrokecolor{currentstroke}%
\pgfsetdash{}{0pt}%
\pgfpathmoveto{\pgfqpoint{2.172552in}{2.494875in}}%
\pgfpathlineto{\pgfqpoint{2.187025in}{2.469385in}}%
\pgfpathlineto{\pgfqpoint{2.201483in}{2.444205in}}%
\pgfpathlineto{\pgfqpoint{2.215925in}{2.419334in}}%
\pgfpathlineto{\pgfqpoint{2.230352in}{2.394768in}}%
\pgfpathlineto{\pgfqpoint{2.239890in}{2.383561in}}%
\pgfpathlineto{\pgfqpoint{2.249398in}{2.372833in}}%
\pgfpathlineto{\pgfqpoint{2.258876in}{2.362576in}}%
\pgfpathlineto{\pgfqpoint{2.268326in}{2.352782in}}%
\pgfpathlineto{\pgfqpoint{2.253973in}{2.376511in}}%
\pgfpathlineto{\pgfqpoint{2.239606in}{2.400544in}}%
\pgfpathlineto{\pgfqpoint{2.225225in}{2.424883in}}%
\pgfpathlineto{\pgfqpoint{2.210828in}{2.449531in}}%
\pgfpathlineto{\pgfqpoint{2.201305in}{2.460149in}}%
\pgfpathlineto{\pgfqpoint{2.191752in}{2.471240in}}%
\pgfpathlineto{\pgfqpoint{2.182168in}{2.482812in}}%
\pgfpathlineto{\pgfqpoint{2.172552in}{2.494875in}}%
\pgfpathclose%
\pgfusepath{fill}%
\end{pgfscope}%
\begin{pgfscope}%
\pgfpathrectangle{\pgfqpoint{1.150000in}{0.150000in}}{\pgfqpoint{5.700000in}{5.700000in}}%
\pgfusepath{clip}%
\pgfsetbuttcap%
\pgfsetroundjoin%
\definecolor{currentfill}{rgb}{0.277018,0.050344,0.375715}%
\pgfsetfillcolor{currentfill}%
\pgfsetfillopacity{0.800000}%
\pgfsetlinewidth{0.000000pt}%
\definecolor{currentstroke}{rgb}{0.000000,0.000000,0.000000}%
\pgfsetstrokecolor{currentstroke}%
\pgfsetdash{}{0pt}%
\pgfpathmoveto{\pgfqpoint{3.078834in}{1.398503in}}%
\pgfpathlineto{\pgfqpoint{3.092839in}{1.388833in}}%
\pgfpathlineto{\pgfqpoint{3.106843in}{1.379365in}}%
\pgfpathlineto{\pgfqpoint{3.120848in}{1.370099in}}%
\pgfpathlineto{\pgfqpoint{3.134852in}{1.361033in}}%
\pgfpathlineto{\pgfqpoint{3.143496in}{1.362704in}}%
\pgfpathlineto{\pgfqpoint{3.152126in}{1.364700in}}%
\pgfpathlineto{\pgfqpoint{3.160740in}{1.367013in}}%
\pgfpathlineto{\pgfqpoint{3.169341in}{1.369637in}}%
\pgfpathlineto{\pgfqpoint{3.155373in}{1.377939in}}%
\pgfpathlineto{\pgfqpoint{3.141406in}{1.386441in}}%
\pgfpathlineto{\pgfqpoint{3.127439in}{1.395145in}}%
\pgfpathlineto{\pgfqpoint{3.113473in}{1.404050in}}%
\pgfpathlineto{\pgfqpoint{3.104836in}{1.402178in}}%
\pgfpathlineto{\pgfqpoint{3.096184in}{1.400624in}}%
\pgfpathlineto{\pgfqpoint{3.087517in}{1.399397in}}%
\pgfpathlineto{\pgfqpoint{3.078834in}{1.398503in}}%
\pgfpathclose%
\pgfusepath{fill}%
\end{pgfscope}%
\begin{pgfscope}%
\pgfpathrectangle{\pgfqpoint{1.150000in}{0.150000in}}{\pgfqpoint{5.700000in}{5.700000in}}%
\pgfusepath{clip}%
\pgfsetbuttcap%
\pgfsetroundjoin%
\definecolor{currentfill}{rgb}{0.169646,0.456262,0.558030}%
\pgfsetfillcolor{currentfill}%
\pgfsetfillopacity{0.800000}%
\pgfsetlinewidth{0.000000pt}%
\definecolor{currentstroke}{rgb}{0.000000,0.000000,0.000000}%
\pgfsetstrokecolor{currentstroke}%
\pgfsetdash{}{0pt}%
\pgfpathmoveto{\pgfqpoint{4.659122in}{2.373147in}}%
\pgfpathlineto{\pgfqpoint{4.673557in}{2.385223in}}%
\pgfpathlineto{\pgfqpoint{4.688010in}{2.397484in}}%
\pgfpathlineto{\pgfqpoint{4.702481in}{2.409931in}}%
\pgfpathlineto{\pgfqpoint{4.716970in}{2.422563in}}%
\pgfpathlineto{\pgfqpoint{4.724954in}{2.436037in}}%
\pgfpathlineto{\pgfqpoint{4.732933in}{2.449364in}}%
\pgfpathlineto{\pgfqpoint{4.740905in}{2.462541in}}%
\pgfpathlineto{\pgfqpoint{4.748870in}{2.475568in}}%
\pgfpathlineto{\pgfqpoint{4.734379in}{2.462780in}}%
\pgfpathlineto{\pgfqpoint{4.719905in}{2.450176in}}%
\pgfpathlineto{\pgfqpoint{4.705450in}{2.437758in}}%
\pgfpathlineto{\pgfqpoint{4.691012in}{2.425526in}}%
\pgfpathlineto{\pgfqpoint{4.683048in}{2.412642in}}%
\pgfpathlineto{\pgfqpoint{4.675079in}{2.399617in}}%
\pgfpathlineto{\pgfqpoint{4.667103in}{2.386452in}}%
\pgfpathlineto{\pgfqpoint{4.659122in}{2.373147in}}%
\pgfpathclose%
\pgfusepath{fill}%
\end{pgfscope}%
\begin{pgfscope}%
\pgfpathrectangle{\pgfqpoint{1.150000in}{0.150000in}}{\pgfqpoint{5.700000in}{5.700000in}}%
\pgfusepath{clip}%
\pgfsetbuttcap%
\pgfsetroundjoin%
\definecolor{currentfill}{rgb}{0.124395,0.578002,0.548287}%
\pgfsetfillcolor{currentfill}%
\pgfsetfillopacity{0.800000}%
\pgfsetlinewidth{0.000000pt}%
\definecolor{currentstroke}{rgb}{0.000000,0.000000,0.000000}%
\pgfsetstrokecolor{currentstroke}%
\pgfsetdash{}{0pt}%
\pgfpathmoveto{\pgfqpoint{1.978144in}{2.888393in}}%
\pgfpathlineto{\pgfqpoint{1.992831in}{2.858379in}}%
\pgfpathlineto{\pgfqpoint{2.007496in}{2.828725in}}%
\pgfpathlineto{\pgfqpoint{2.022141in}{2.799429in}}%
\pgfpathlineto{\pgfqpoint{2.036766in}{2.770487in}}%
\pgfpathlineto{\pgfqpoint{2.046494in}{2.757843in}}%
\pgfpathlineto{\pgfqpoint{2.056189in}{2.745687in}}%
\pgfpathlineto{\pgfqpoint{2.065852in}{2.734011in}}%
\pgfpathlineto{\pgfqpoint{2.075483in}{2.722806in}}%
\pgfpathlineto{\pgfqpoint{2.060939in}{2.750919in}}%
\pgfpathlineto{\pgfqpoint{2.046376in}{2.779383in}}%
\pgfpathlineto{\pgfqpoint{2.031793in}{2.808202in}}%
\pgfpathlineto{\pgfqpoint{2.017190in}{2.837380in}}%
\pgfpathlineto{\pgfqpoint{2.007478in}{2.849401in}}%
\pgfpathlineto{\pgfqpoint{1.997734in}{2.861904in}}%
\pgfpathlineto{\pgfqpoint{1.987956in}{2.874899in}}%
\pgfpathlineto{\pgfqpoint{1.978144in}{2.888393in}}%
\pgfpathclose%
\pgfusepath{fill}%
\end{pgfscope}%
\begin{pgfscope}%
\pgfpathrectangle{\pgfqpoint{1.150000in}{0.150000in}}{\pgfqpoint{5.700000in}{5.700000in}}%
\pgfusepath{clip}%
\pgfsetbuttcap%
\pgfsetroundjoin%
\definecolor{currentfill}{rgb}{0.190631,0.407061,0.556089}%
\pgfsetfillcolor{currentfill}%
\pgfsetfillopacity{0.800000}%
\pgfsetlinewidth{0.000000pt}%
\definecolor{currentstroke}{rgb}{0.000000,0.000000,0.000000}%
\pgfsetstrokecolor{currentstroke}%
\pgfsetdash{}{0pt}%
\pgfpathmoveto{\pgfqpoint{4.537514in}{2.217058in}}%
\pgfpathlineto{\pgfqpoint{4.551875in}{2.227981in}}%
\pgfpathlineto{\pgfqpoint{4.566253in}{2.239089in}}%
\pgfpathlineto{\pgfqpoint{4.580648in}{2.250381in}}%
\pgfpathlineto{\pgfqpoint{4.595060in}{2.261858in}}%
\pgfpathlineto{\pgfqpoint{4.603087in}{2.276225in}}%
\pgfpathlineto{\pgfqpoint{4.611109in}{2.290466in}}%
\pgfpathlineto{\pgfqpoint{4.619125in}{2.304578in}}%
\pgfpathlineto{\pgfqpoint{4.627136in}{2.318559in}}%
\pgfpathlineto{\pgfqpoint{4.612721in}{2.306859in}}%
\pgfpathlineto{\pgfqpoint{4.598323in}{2.295343in}}%
\pgfpathlineto{\pgfqpoint{4.583941in}{2.284011in}}%
\pgfpathlineto{\pgfqpoint{4.569577in}{2.272864in}}%
\pgfpathlineto{\pgfqpoint{4.561569in}{2.259094in}}%
\pgfpathlineto{\pgfqpoint{4.553556in}{2.245202in}}%
\pgfpathlineto{\pgfqpoint{4.545537in}{2.231189in}}%
\pgfpathlineto{\pgfqpoint{4.537514in}{2.217058in}}%
\pgfpathclose%
\pgfusepath{fill}%
\end{pgfscope}%
\begin{pgfscope}%
\pgfpathrectangle{\pgfqpoint{1.150000in}{0.150000in}}{\pgfqpoint{5.700000in}{5.700000in}}%
\pgfusepath{clip}%
\pgfsetbuttcap%
\pgfsetroundjoin%
\definecolor{currentfill}{rgb}{0.268510,0.009605,0.335427}%
\pgfsetfillcolor{currentfill}%
\pgfsetfillopacity{0.800000}%
\pgfsetlinewidth{0.000000pt}%
\definecolor{currentstroke}{rgb}{0.000000,0.000000,0.000000}%
\pgfsetstrokecolor{currentstroke}%
\pgfsetdash{}{0pt}%
\pgfpathmoveto{\pgfqpoint{3.281128in}{1.310331in}}%
\pgfpathlineto{\pgfqpoint{3.295110in}{1.303795in}}%
\pgfpathlineto{\pgfqpoint{3.309094in}{1.297453in}}%
\pgfpathlineto{\pgfqpoint{3.323080in}{1.291302in}}%
\pgfpathlineto{\pgfqpoint{3.337070in}{1.285343in}}%
\pgfpathlineto{\pgfqpoint{3.345562in}{1.290488in}}%
\pgfpathlineto{\pgfqpoint{3.354043in}{1.295902in}}%
\pgfpathlineto{\pgfqpoint{3.362513in}{1.301578in}}%
\pgfpathlineto{\pgfqpoint{3.370971in}{1.307508in}}%
\pgfpathlineto{\pgfqpoint{3.357010in}{1.312741in}}%
\pgfpathlineto{\pgfqpoint{3.343052in}{1.318165in}}%
\pgfpathlineto{\pgfqpoint{3.329096in}{1.323781in}}%
\pgfpathlineto{\pgfqpoint{3.315144in}{1.329590in}}%
\pgfpathlineto{\pgfqpoint{3.306658in}{1.324373in}}%
\pgfpathlineto{\pgfqpoint{3.298160in}{1.319420in}}%
\pgfpathlineto{\pgfqpoint{3.289650in}{1.314736in}}%
\pgfpathlineto{\pgfqpoint{3.281128in}{1.310331in}}%
\pgfpathclose%
\pgfusepath{fill}%
\end{pgfscope}%
\begin{pgfscope}%
\pgfpathrectangle{\pgfqpoint{1.150000in}{0.150000in}}{\pgfqpoint{5.700000in}{5.700000in}}%
\pgfusepath{clip}%
\pgfsetbuttcap%
\pgfsetroundjoin%
\definecolor{currentfill}{rgb}{0.265145,0.232956,0.516599}%
\pgfsetfillcolor{currentfill}%
\pgfsetfillopacity{0.800000}%
\pgfsetlinewidth{0.000000pt}%
\definecolor{currentstroke}{rgb}{0.000000,0.000000,0.000000}%
\pgfsetstrokecolor{currentstroke}%
\pgfsetdash{}{0pt}%
\pgfpathmoveto{\pgfqpoint{2.592877in}{1.846961in}}%
\pgfpathlineto{\pgfqpoint{2.607054in}{1.829407in}}%
\pgfpathlineto{\pgfqpoint{2.621223in}{1.812096in}}%
\pgfpathlineto{\pgfqpoint{2.635385in}{1.795025in}}%
\pgfpathlineto{\pgfqpoint{2.649540in}{1.778194in}}%
\pgfpathlineto{\pgfqpoint{2.658646in}{1.771895in}}%
\pgfpathlineto{\pgfqpoint{2.667729in}{1.766028in}}%
\pgfpathlineto{\pgfqpoint{2.676789in}{1.760585in}}%
\pgfpathlineto{\pgfqpoint{2.685826in}{1.755557in}}%
\pgfpathlineto{\pgfqpoint{2.671729in}{1.771563in}}%
\pgfpathlineto{\pgfqpoint{2.657626in}{1.787807in}}%
\pgfpathlineto{\pgfqpoint{2.643517in}{1.804291in}}%
\pgfpathlineto{\pgfqpoint{2.629400in}{1.821016in}}%
\pgfpathlineto{\pgfqpoint{2.620305in}{1.826856in}}%
\pgfpathlineto{\pgfqpoint{2.611187in}{1.833121in}}%
\pgfpathlineto{\pgfqpoint{2.602044in}{1.839820in}}%
\pgfpathlineto{\pgfqpoint{2.592877in}{1.846961in}}%
\pgfpathclose%
\pgfusepath{fill}%
\end{pgfscope}%
\begin{pgfscope}%
\pgfpathrectangle{\pgfqpoint{1.150000in}{0.150000in}}{\pgfqpoint{5.700000in}{5.700000in}}%
\pgfusepath{clip}%
\pgfsetbuttcap%
\pgfsetroundjoin%
\definecolor{currentfill}{rgb}{0.273006,0.204520,0.501721}%
\pgfsetfillcolor{currentfill}%
\pgfsetfillopacity{0.800000}%
\pgfsetlinewidth{0.000000pt}%
\definecolor{currentstroke}{rgb}{0.000000,0.000000,0.000000}%
\pgfsetstrokecolor{currentstroke}%
\pgfsetdash{}{0pt}%
\pgfpathmoveto{\pgfqpoint{2.649540in}{1.778194in}}%
\pgfpathlineto{\pgfqpoint{2.663688in}{1.761601in}}%
\pgfpathlineto{\pgfqpoint{2.677830in}{1.745243in}}%
\pgfpathlineto{\pgfqpoint{2.691965in}{1.729121in}}%
\pgfpathlineto{\pgfqpoint{2.706095in}{1.713232in}}%
\pgfpathlineto{\pgfqpoint{2.715143in}{1.707769in}}%
\pgfpathlineto{\pgfqpoint{2.724168in}{1.702730in}}%
\pgfpathlineto{\pgfqpoint{2.733172in}{1.698104in}}%
\pgfpathlineto{\pgfqpoint{2.742154in}{1.693885in}}%
\pgfpathlineto{\pgfqpoint{2.728080in}{1.708953in}}%
\pgfpathlineto{\pgfqpoint{2.714001in}{1.724254in}}%
\pgfpathlineto{\pgfqpoint{2.699917in}{1.739788in}}%
\pgfpathlineto{\pgfqpoint{2.685826in}{1.755557in}}%
\pgfpathlineto{\pgfqpoint{2.676789in}{1.760585in}}%
\pgfpathlineto{\pgfqpoint{2.667729in}{1.766028in}}%
\pgfpathlineto{\pgfqpoint{2.658646in}{1.771895in}}%
\pgfpathlineto{\pgfqpoint{2.649540in}{1.778194in}}%
\pgfpathclose%
\pgfusepath{fill}%
\end{pgfscope}%
\begin{pgfscope}%
\pgfpathrectangle{\pgfqpoint{1.150000in}{0.150000in}}{\pgfqpoint{5.700000in}{5.700000in}}%
\pgfusepath{clip}%
\pgfsetbuttcap%
\pgfsetroundjoin%
\definecolor{currentfill}{rgb}{0.282623,0.140926,0.457517}%
\pgfsetfillcolor{currentfill}%
\pgfsetfillopacity{0.800000}%
\pgfsetlinewidth{0.000000pt}%
\definecolor{currentstroke}{rgb}{0.000000,0.000000,0.000000}%
\pgfsetstrokecolor{currentstroke}%
\pgfsetdash{}{0pt}%
\pgfpathmoveto{\pgfqpoint{3.961788in}{1.541362in}}%
\pgfpathlineto{\pgfqpoint{3.975868in}{1.545179in}}%
\pgfpathlineto{\pgfqpoint{3.989959in}{1.549178in}}%
\pgfpathlineto{\pgfqpoint{4.004060in}{1.553357in}}%
\pgfpathlineto{\pgfqpoint{4.018173in}{1.557717in}}%
\pgfpathlineto{\pgfqpoint{4.026351in}{1.572065in}}%
\pgfpathlineto{\pgfqpoint{4.034525in}{1.586449in}}%
\pgfpathlineto{\pgfqpoint{4.042695in}{1.600863in}}%
\pgfpathlineto{\pgfqpoint{4.050860in}{1.615304in}}%
\pgfpathlineto{\pgfqpoint{4.036750in}{1.610433in}}%
\pgfpathlineto{\pgfqpoint{4.022651in}{1.605744in}}%
\pgfpathlineto{\pgfqpoint{4.008563in}{1.601237in}}%
\pgfpathlineto{\pgfqpoint{3.994487in}{1.596910in}}%
\pgfpathlineto{\pgfqpoint{3.986319in}{1.582967in}}%
\pgfpathlineto{\pgfqpoint{3.978146in}{1.569058in}}%
\pgfpathlineto{\pgfqpoint{3.969969in}{1.555188in}}%
\pgfpathlineto{\pgfqpoint{3.961788in}{1.541362in}}%
\pgfpathclose%
\pgfusepath{fill}%
\end{pgfscope}%
\begin{pgfscope}%
\pgfpathrectangle{\pgfqpoint{1.150000in}{0.150000in}}{\pgfqpoint{5.700000in}{5.700000in}}%
\pgfusepath{clip}%
\pgfsetbuttcap%
\pgfsetroundjoin%
\definecolor{currentfill}{rgb}{0.216210,0.351535,0.550627}%
\pgfsetfillcolor{currentfill}%
\pgfsetfillopacity{0.800000}%
\pgfsetlinewidth{0.000000pt}%
\definecolor{currentstroke}{rgb}{0.000000,0.000000,0.000000}%
\pgfsetstrokecolor{currentstroke}%
\pgfsetdash{}{0pt}%
\pgfpathmoveto{\pgfqpoint{4.415881in}{2.060324in}}%
\pgfpathlineto{\pgfqpoint{4.430172in}{2.069967in}}%
\pgfpathlineto{\pgfqpoint{4.444479in}{2.079792in}}%
\pgfpathlineto{\pgfqpoint{4.458801in}{2.089801in}}%
\pgfpathlineto{\pgfqpoint{4.473140in}{2.099994in}}%
\pgfpathlineto{\pgfqpoint{4.481204in}{2.114996in}}%
\pgfpathlineto{\pgfqpoint{4.489263in}{2.129897in}}%
\pgfpathlineto{\pgfqpoint{4.497317in}{2.144696in}}%
\pgfpathlineto{\pgfqpoint{4.505367in}{2.159389in}}%
\pgfpathlineto{\pgfqpoint{4.491025in}{2.148906in}}%
\pgfpathlineto{\pgfqpoint{4.476699in}{2.138607in}}%
\pgfpathlineto{\pgfqpoint{4.462389in}{2.128492in}}%
\pgfpathlineto{\pgfqpoint{4.448094in}{2.118560in}}%
\pgfpathlineto{\pgfqpoint{4.440048in}{2.104145in}}%
\pgfpathlineto{\pgfqpoint{4.431997in}{2.089632in}}%
\pgfpathlineto{\pgfqpoint{4.423941in}{2.075024in}}%
\pgfpathlineto{\pgfqpoint{4.415881in}{2.060324in}}%
\pgfpathclose%
\pgfusepath{fill}%
\end{pgfscope}%
\begin{pgfscope}%
\pgfpathrectangle{\pgfqpoint{1.150000in}{0.150000in}}{\pgfqpoint{5.700000in}{5.700000in}}%
\pgfusepath{clip}%
\pgfsetbuttcap%
\pgfsetroundjoin%
\definecolor{currentfill}{rgb}{0.121380,0.629492,0.531973}%
\pgfsetfillcolor{currentfill}%
\pgfsetfillopacity{0.800000}%
\pgfsetlinewidth{0.000000pt}%
\definecolor{currentstroke}{rgb}{0.000000,0.000000,0.000000}%
\pgfsetstrokecolor{currentstroke}%
\pgfsetdash{}{0pt}%
\pgfpathmoveto{\pgfqpoint{5.113306in}{2.912465in}}%
\pgfpathlineto{\pgfqpoint{5.128049in}{2.927895in}}%
\pgfpathlineto{\pgfqpoint{5.142814in}{2.943513in}}%
\pgfpathlineto{\pgfqpoint{5.157599in}{2.959320in}}%
\pgfpathlineto{\pgfqpoint{5.172406in}{2.975315in}}%
\pgfpathlineto{\pgfqpoint{5.180174in}{2.984196in}}%
\pgfpathlineto{\pgfqpoint{5.187933in}{2.992896in}}%
\pgfpathlineto{\pgfqpoint{5.195682in}{3.001414in}}%
\pgfpathlineto{\pgfqpoint{5.203422in}{3.009752in}}%
\pgfpathlineto{\pgfqpoint{5.188620in}{2.993846in}}%
\pgfpathlineto{\pgfqpoint{5.173839in}{2.978128in}}%
\pgfpathlineto{\pgfqpoint{5.159079in}{2.962598in}}%
\pgfpathlineto{\pgfqpoint{5.144341in}{2.947255in}}%
\pgfpathlineto{\pgfqpoint{5.136595in}{2.938816in}}%
\pgfpathlineto{\pgfqpoint{5.128841in}{2.930205in}}%
\pgfpathlineto{\pgfqpoint{5.121078in}{2.921421in}}%
\pgfpathlineto{\pgfqpoint{5.113306in}{2.912465in}}%
\pgfpathclose%
\pgfusepath{fill}%
\end{pgfscope}%
\begin{pgfscope}%
\pgfpathrectangle{\pgfqpoint{1.150000in}{0.150000in}}{\pgfqpoint{5.700000in}{5.700000in}}%
\pgfusepath{clip}%
\pgfsetbuttcap%
\pgfsetroundjoin%
\definecolor{currentfill}{rgb}{0.281477,0.755203,0.432552}%
\pgfsetfillcolor{currentfill}%
\pgfsetfillopacity{0.800000}%
\pgfsetlinewidth{0.000000pt}%
\definecolor{currentstroke}{rgb}{0.000000,0.000000,0.000000}%
\pgfsetstrokecolor{currentstroke}%
\pgfsetdash{}{0pt}%
\pgfpathmoveto{\pgfqpoint{5.534953in}{3.335533in}}%
\pgfpathlineto{\pgfqpoint{5.549996in}{3.352905in}}%
\pgfpathlineto{\pgfqpoint{5.565063in}{3.370465in}}%
\pgfpathlineto{\pgfqpoint{5.580154in}{3.388215in}}%
\pgfpathlineto{\pgfqpoint{5.595268in}{3.406155in}}%
\pgfpathlineto{\pgfqpoint{5.602746in}{3.410040in}}%
\pgfpathlineto{\pgfqpoint{5.610213in}{3.413765in}}%
\pgfpathlineto{\pgfqpoint{5.617668in}{3.417332in}}%
\pgfpathlineto{\pgfqpoint{5.625112in}{3.420746in}}%
\pgfpathlineto{\pgfqpoint{5.610015in}{3.403115in}}%
\pgfpathlineto{\pgfqpoint{5.594941in}{3.385673in}}%
\pgfpathlineto{\pgfqpoint{5.579891in}{3.368420in}}%
\pgfpathlineto{\pgfqpoint{5.564865in}{3.351355in}}%
\pgfpathlineto{\pgfqpoint{5.557403in}{3.347621in}}%
\pgfpathlineto{\pgfqpoint{5.549930in}{3.343742in}}%
\pgfpathlineto{\pgfqpoint{5.542447in}{3.339714in}}%
\pgfpathlineto{\pgfqpoint{5.534953in}{3.335533in}}%
\pgfpathclose%
\pgfusepath{fill}%
\end{pgfscope}%
\begin{pgfscope}%
\pgfpathrectangle{\pgfqpoint{1.150000in}{0.150000in}}{\pgfqpoint{5.700000in}{5.700000in}}%
\pgfusepath{clip}%
\pgfsetbuttcap%
\pgfsetroundjoin%
\definecolor{currentfill}{rgb}{0.278012,0.180367,0.486697}%
\pgfsetfillcolor{currentfill}%
\pgfsetfillopacity{0.800000}%
\pgfsetlinewidth{0.000000pt}%
\definecolor{currentstroke}{rgb}{0.000000,0.000000,0.000000}%
\pgfsetstrokecolor{currentstroke}%
\pgfsetdash{}{0pt}%
\pgfpathmoveto{\pgfqpoint{2.706095in}{1.713232in}}%
\pgfpathlineto{\pgfqpoint{2.720218in}{1.697574in}}%
\pgfpathlineto{\pgfqpoint{2.734336in}{1.682147in}}%
\pgfpathlineto{\pgfqpoint{2.748449in}{1.666949in}}%
\pgfpathlineto{\pgfqpoint{2.762556in}{1.651978in}}%
\pgfpathlineto{\pgfqpoint{2.771549in}{1.647348in}}%
\pgfpathlineto{\pgfqpoint{2.780520in}{1.643131in}}%
\pgfpathlineto{\pgfqpoint{2.789469in}{1.639319in}}%
\pgfpathlineto{\pgfqpoint{2.798398in}{1.635903in}}%
\pgfpathlineto{\pgfqpoint{2.784344in}{1.650057in}}%
\pgfpathlineto{\pgfqpoint{2.770286in}{1.664438in}}%
\pgfpathlineto{\pgfqpoint{2.756222in}{1.679047in}}%
\pgfpathlineto{\pgfqpoint{2.742154in}{1.693885in}}%
\pgfpathlineto{\pgfqpoint{2.733172in}{1.698104in}}%
\pgfpathlineto{\pgfqpoint{2.724168in}{1.702730in}}%
\pgfpathlineto{\pgfqpoint{2.715143in}{1.707769in}}%
\pgfpathlineto{\pgfqpoint{2.706095in}{1.713232in}}%
\pgfpathclose%
\pgfusepath{fill}%
\end{pgfscope}%
\begin{pgfscope}%
\pgfpathrectangle{\pgfqpoint{1.150000in}{0.150000in}}{\pgfqpoint{5.700000in}{5.700000in}}%
\pgfusepath{clip}%
\pgfsetbuttcap%
\pgfsetroundjoin%
\definecolor{currentfill}{rgb}{0.255645,0.260703,0.528312}%
\pgfsetfillcolor{currentfill}%
\pgfsetfillopacity{0.800000}%
\pgfsetlinewidth{0.000000pt}%
\definecolor{currentstroke}{rgb}{0.000000,0.000000,0.000000}%
\pgfsetstrokecolor{currentstroke}%
\pgfsetdash{}{0pt}%
\pgfpathmoveto{\pgfqpoint{2.536092in}{1.919636in}}%
\pgfpathlineto{\pgfqpoint{2.550301in}{1.901095in}}%
\pgfpathlineto{\pgfqpoint{2.564501in}{1.882804in}}%
\pgfpathlineto{\pgfqpoint{2.578693in}{1.864759in}}%
\pgfpathlineto{\pgfqpoint{2.592877in}{1.846961in}}%
\pgfpathlineto{\pgfqpoint{2.602044in}{1.839820in}}%
\pgfpathlineto{\pgfqpoint{2.611187in}{1.833121in}}%
\pgfpathlineto{\pgfqpoint{2.620305in}{1.826856in}}%
\pgfpathlineto{\pgfqpoint{2.629400in}{1.821016in}}%
\pgfpathlineto{\pgfqpoint{2.615277in}{1.837983in}}%
\pgfpathlineto{\pgfqpoint{2.601146in}{1.855196in}}%
\pgfpathlineto{\pgfqpoint{2.587008in}{1.872654in}}%
\pgfpathlineto{\pgfqpoint{2.572862in}{1.890361in}}%
\pgfpathlineto{\pgfqpoint{2.563707in}{1.897018in}}%
\pgfpathlineto{\pgfqpoint{2.554527in}{1.904111in}}%
\pgfpathlineto{\pgfqpoint{2.545323in}{1.911647in}}%
\pgfpathlineto{\pgfqpoint{2.536092in}{1.919636in}}%
\pgfpathclose%
\pgfusepath{fill}%
\end{pgfscope}%
\begin{pgfscope}%
\pgfpathrectangle{\pgfqpoint{1.150000in}{0.150000in}}{\pgfqpoint{5.700000in}{5.700000in}}%
\pgfusepath{clip}%
\pgfsetbuttcap%
\pgfsetroundjoin%
\definecolor{currentfill}{rgb}{0.267004,0.004874,0.329415}%
\pgfsetfillcolor{currentfill}%
\pgfsetfillopacity{0.800000}%
\pgfsetlinewidth{0.000000pt}%
\definecolor{currentstroke}{rgb}{0.000000,0.000000,0.000000}%
\pgfsetstrokecolor{currentstroke}%
\pgfsetdash{}{0pt}%
\pgfpathmoveto{\pgfqpoint{3.426853in}{1.288478in}}%
\pgfpathlineto{\pgfqpoint{3.440834in}{1.284193in}}%
\pgfpathlineto{\pgfqpoint{3.454818in}{1.280096in}}%
\pgfpathlineto{\pgfqpoint{3.468807in}{1.276186in}}%
\pgfpathlineto{\pgfqpoint{3.482800in}{1.272463in}}%
\pgfpathlineto{\pgfqpoint{3.491201in}{1.280051in}}%
\pgfpathlineto{\pgfqpoint{3.499591in}{1.287864in}}%
\pgfpathlineto{\pgfqpoint{3.507973in}{1.295893in}}%
\pgfpathlineto{\pgfqpoint{3.516346in}{1.304133in}}%
\pgfpathlineto{\pgfqpoint{3.502374in}{1.307163in}}%
\pgfpathlineto{\pgfqpoint{3.488407in}{1.310380in}}%
\pgfpathlineto{\pgfqpoint{3.474445in}{1.313784in}}%
\pgfpathlineto{\pgfqpoint{3.460487in}{1.317376in}}%
\pgfpathlineto{\pgfqpoint{3.452093in}{1.309817in}}%
\pgfpathlineto{\pgfqpoint{3.443690in}{1.302476in}}%
\pgfpathlineto{\pgfqpoint{3.435277in}{1.295361in}}%
\pgfpathlineto{\pgfqpoint{3.426853in}{1.288478in}}%
\pgfpathclose%
\pgfusepath{fill}%
\end{pgfscope}%
\begin{pgfscope}%
\pgfpathrectangle{\pgfqpoint{1.150000in}{0.150000in}}{\pgfqpoint{5.700000in}{5.700000in}}%
\pgfusepath{clip}%
\pgfsetbuttcap%
\pgfsetroundjoin%
\definecolor{currentfill}{rgb}{0.241237,0.296485,0.539709}%
\pgfsetfillcolor{currentfill}%
\pgfsetfillopacity{0.800000}%
\pgfsetlinewidth{0.000000pt}%
\definecolor{currentstroke}{rgb}{0.000000,0.000000,0.000000}%
\pgfsetstrokecolor{currentstroke}%
\pgfsetdash{}{0pt}%
\pgfpathmoveto{\pgfqpoint{4.294238in}{1.905696in}}%
\pgfpathlineto{\pgfqpoint{4.308465in}{1.913931in}}%
\pgfpathlineto{\pgfqpoint{4.322706in}{1.922349in}}%
\pgfpathlineto{\pgfqpoint{4.336962in}{1.930949in}}%
\pgfpathlineto{\pgfqpoint{4.351232in}{1.939731in}}%
\pgfpathlineto{\pgfqpoint{4.359329in}{1.955069in}}%
\pgfpathlineto{\pgfqpoint{4.367421in}{1.970338in}}%
\pgfpathlineto{\pgfqpoint{4.375509in}{1.985534in}}%
\pgfpathlineto{\pgfqpoint{4.383592in}{2.000655in}}%
\pgfpathlineto{\pgfqpoint{4.369319in}{1.991517in}}%
\pgfpathlineto{\pgfqpoint{4.355061in}{1.982563in}}%
\pgfpathlineto{\pgfqpoint{4.340817in}{1.973791in}}%
\pgfpathlineto{\pgfqpoint{4.326588in}{1.965203in}}%
\pgfpathlineto{\pgfqpoint{4.318507in}{1.950424in}}%
\pgfpathlineto{\pgfqpoint{4.310422in}{1.935578in}}%
\pgfpathlineto{\pgfqpoint{4.302332in}{1.920667in}}%
\pgfpathlineto{\pgfqpoint{4.294238in}{1.905696in}}%
\pgfpathclose%
\pgfusepath{fill}%
\end{pgfscope}%
\begin{pgfscope}%
\pgfpathrectangle{\pgfqpoint{1.150000in}{0.150000in}}{\pgfqpoint{5.700000in}{5.700000in}}%
\pgfusepath{clip}%
\pgfsetbuttcap%
\pgfsetroundjoin%
\definecolor{currentfill}{rgb}{0.281412,0.155834,0.469201}%
\pgfsetfillcolor{currentfill}%
\pgfsetfillopacity{0.800000}%
\pgfsetlinewidth{0.000000pt}%
\definecolor{currentstroke}{rgb}{0.000000,0.000000,0.000000}%
\pgfsetstrokecolor{currentstroke}%
\pgfsetdash{}{0pt}%
\pgfpathmoveto{\pgfqpoint{2.762556in}{1.651978in}}%
\pgfpathlineto{\pgfqpoint{2.776659in}{1.637233in}}%
\pgfpathlineto{\pgfqpoint{2.790757in}{1.622714in}}%
\pgfpathlineto{\pgfqpoint{2.804850in}{1.608418in}}%
\pgfpathlineto{\pgfqpoint{2.818939in}{1.594344in}}%
\pgfpathlineto{\pgfqpoint{2.827878in}{1.590542in}}%
\pgfpathlineto{\pgfqpoint{2.836796in}{1.587144in}}%
\pgfpathlineto{\pgfqpoint{2.845694in}{1.584141in}}%
\pgfpathlineto{\pgfqpoint{2.854573in}{1.581524in}}%
\pgfpathlineto{\pgfqpoint{2.840535in}{1.594786in}}%
\pgfpathlineto{\pgfqpoint{2.826493in}{1.608268in}}%
\pgfpathlineto{\pgfqpoint{2.812448in}{1.621974in}}%
\pgfpathlineto{\pgfqpoint{2.798398in}{1.635903in}}%
\pgfpathlineto{\pgfqpoint{2.789469in}{1.639319in}}%
\pgfpathlineto{\pgfqpoint{2.780520in}{1.643131in}}%
\pgfpathlineto{\pgfqpoint{2.771549in}{1.647348in}}%
\pgfpathlineto{\pgfqpoint{2.762556in}{1.651978in}}%
\pgfpathclose%
\pgfusepath{fill}%
\end{pgfscope}%
\begin{pgfscope}%
\pgfpathrectangle{\pgfqpoint{1.150000in}{0.150000in}}{\pgfqpoint{5.700000in}{5.700000in}}%
\pgfusepath{clip}%
\pgfsetbuttcap%
\pgfsetroundjoin%
\definecolor{currentfill}{rgb}{0.243113,0.292092,0.538516}%
\pgfsetfillcolor{currentfill}%
\pgfsetfillopacity{0.800000}%
\pgfsetlinewidth{0.000000pt}%
\definecolor{currentstroke}{rgb}{0.000000,0.000000,0.000000}%
\pgfsetstrokecolor{currentstroke}%
\pgfsetdash{}{0pt}%
\pgfpathmoveto{\pgfqpoint{2.479169in}{1.996330in}}%
\pgfpathlineto{\pgfqpoint{2.493413in}{1.976774in}}%
\pgfpathlineto{\pgfqpoint{2.507649in}{1.957474in}}%
\pgfpathlineto{\pgfqpoint{2.521875in}{1.938429in}}%
\pgfpathlineto{\pgfqpoint{2.536092in}{1.919636in}}%
\pgfpathlineto{\pgfqpoint{2.545323in}{1.911647in}}%
\pgfpathlineto{\pgfqpoint{2.554527in}{1.904111in}}%
\pgfpathlineto{\pgfqpoint{2.563707in}{1.897018in}}%
\pgfpathlineto{\pgfqpoint{2.572862in}{1.890361in}}%
\pgfpathlineto{\pgfqpoint{2.558708in}{1.908317in}}%
\pgfpathlineto{\pgfqpoint{2.544546in}{1.926524in}}%
\pgfpathlineto{\pgfqpoint{2.530375in}{1.944985in}}%
\pgfpathlineto{\pgfqpoint{2.516195in}{1.963701in}}%
\pgfpathlineto{\pgfqpoint{2.506978in}{1.971182in}}%
\pgfpathlineto{\pgfqpoint{2.497735in}{1.979108in}}%
\pgfpathlineto{\pgfqpoint{2.488465in}{1.987488in}}%
\pgfpathlineto{\pgfqpoint{2.479169in}{1.996330in}}%
\pgfpathclose%
\pgfusepath{fill}%
\end{pgfscope}%
\begin{pgfscope}%
\pgfpathrectangle{\pgfqpoint{1.150000in}{0.150000in}}{\pgfqpoint{5.700000in}{5.700000in}}%
\pgfusepath{clip}%
\pgfsetbuttcap%
\pgfsetroundjoin%
\definecolor{currentfill}{rgb}{0.180653,0.701402,0.488189}%
\pgfsetfillcolor{currentfill}%
\pgfsetfillopacity{0.800000}%
\pgfsetlinewidth{0.000000pt}%
\definecolor{currentstroke}{rgb}{0.000000,0.000000,0.000000}%
\pgfsetstrokecolor{currentstroke}%
\pgfsetdash{}{0pt}%
\pgfpathmoveto{\pgfqpoint{5.324364in}{3.134554in}}%
\pgfpathlineto{\pgfqpoint{5.339261in}{3.151117in}}%
\pgfpathlineto{\pgfqpoint{5.354180in}{3.167868in}}%
\pgfpathlineto{\pgfqpoint{5.369121in}{3.184809in}}%
\pgfpathlineto{\pgfqpoint{5.384086in}{3.201939in}}%
\pgfpathlineto{\pgfqpoint{5.391721in}{3.208371in}}%
\pgfpathlineto{\pgfqpoint{5.399345in}{3.214625in}}%
\pgfpathlineto{\pgfqpoint{5.406959in}{3.220703in}}%
\pgfpathlineto{\pgfqpoint{5.414563in}{3.226608in}}%
\pgfpathlineto{\pgfqpoint{5.399609in}{3.209676in}}%
\pgfpathlineto{\pgfqpoint{5.384677in}{3.192934in}}%
\pgfpathlineto{\pgfqpoint{5.369768in}{3.176379in}}%
\pgfpathlineto{\pgfqpoint{5.354882in}{3.160013in}}%
\pgfpathlineto{\pgfqpoint{5.347267in}{3.153898in}}%
\pgfpathlineto{\pgfqpoint{5.339643in}{3.147618in}}%
\pgfpathlineto{\pgfqpoint{5.332009in}{3.141171in}}%
\pgfpathlineto{\pgfqpoint{5.324364in}{3.134554in}}%
\pgfpathclose%
\pgfusepath{fill}%
\end{pgfscope}%
\begin{pgfscope}%
\pgfpathrectangle{\pgfqpoint{1.150000in}{0.150000in}}{\pgfqpoint{5.700000in}{5.700000in}}%
\pgfusepath{clip}%
\pgfsetbuttcap%
\pgfsetroundjoin%
\definecolor{currentfill}{rgb}{0.263663,0.237631,0.518762}%
\pgfsetfillcolor{currentfill}%
\pgfsetfillopacity{0.800000}%
\pgfsetlinewidth{0.000000pt}%
\definecolor{currentstroke}{rgb}{0.000000,0.000000,0.000000}%
\pgfsetstrokecolor{currentstroke}%
\pgfsetdash{}{0pt}%
\pgfpathmoveto{\pgfqpoint{4.172577in}{1.756233in}}%
\pgfpathlineto{\pgfqpoint{4.186747in}{1.762937in}}%
\pgfpathlineto{\pgfqpoint{4.200930in}{1.769823in}}%
\pgfpathlineto{\pgfqpoint{4.215126in}{1.776891in}}%
\pgfpathlineto{\pgfqpoint{4.229335in}{1.784140in}}%
\pgfpathlineto{\pgfqpoint{4.237462in}{1.799474in}}%
\pgfpathlineto{\pgfqpoint{4.245586in}{1.814776in}}%
\pgfpathlineto{\pgfqpoint{4.253705in}{1.830042in}}%
\pgfpathlineto{\pgfqpoint{4.261820in}{1.845268in}}%
\pgfpathlineto{\pgfqpoint{4.247609in}{1.837601in}}%
\pgfpathlineto{\pgfqpoint{4.233412in}{1.830115in}}%
\pgfpathlineto{\pgfqpoint{4.219228in}{1.822812in}}%
\pgfpathlineto{\pgfqpoint{4.205057in}{1.815691in}}%
\pgfpathlineto{\pgfqpoint{4.196944in}{1.800871in}}%
\pgfpathlineto{\pgfqpoint{4.188826in}{1.786019in}}%
\pgfpathlineto{\pgfqpoint{4.180704in}{1.771138in}}%
\pgfpathlineto{\pgfqpoint{4.172577in}{1.756233in}}%
\pgfpathclose%
\pgfusepath{fill}%
\end{pgfscope}%
\begin{pgfscope}%
\pgfpathrectangle{\pgfqpoint{1.150000in}{0.150000in}}{\pgfqpoint{5.700000in}{5.700000in}}%
\pgfusepath{clip}%
\pgfsetbuttcap%
\pgfsetroundjoin%
\definecolor{currentfill}{rgb}{0.282884,0.135920,0.453427}%
\pgfsetfillcolor{currentfill}%
\pgfsetfillopacity{0.800000}%
\pgfsetlinewidth{0.000000pt}%
\definecolor{currentstroke}{rgb}{0.000000,0.000000,0.000000}%
\pgfsetstrokecolor{currentstroke}%
\pgfsetdash{}{0pt}%
\pgfpathmoveto{\pgfqpoint{2.818939in}{1.594344in}}%
\pgfpathlineto{\pgfqpoint{2.833024in}{1.580491in}}%
\pgfpathlineto{\pgfqpoint{2.847105in}{1.566858in}}%
\pgfpathlineto{\pgfqpoint{2.861182in}{1.553443in}}%
\pgfpathlineto{\pgfqpoint{2.875256in}{1.540246in}}%
\pgfpathlineto{\pgfqpoint{2.884144in}{1.537269in}}%
\pgfpathlineto{\pgfqpoint{2.893012in}{1.534685in}}%
\pgfpathlineto{\pgfqpoint{2.901861in}{1.532488in}}%
\pgfpathlineto{\pgfqpoint{2.910691in}{1.530668in}}%
\pgfpathlineto{\pgfqpoint{2.896666in}{1.543056in}}%
\pgfpathlineto{\pgfqpoint{2.882638in}{1.555661in}}%
\pgfpathlineto{\pgfqpoint{2.868607in}{1.568483in}}%
\pgfpathlineto{\pgfqpoint{2.854573in}{1.581524in}}%
\pgfpathlineto{\pgfqpoint{2.845694in}{1.584141in}}%
\pgfpathlineto{\pgfqpoint{2.836796in}{1.587144in}}%
\pgfpathlineto{\pgfqpoint{2.827878in}{1.590542in}}%
\pgfpathlineto{\pgfqpoint{2.818939in}{1.594344in}}%
\pgfpathclose%
\pgfusepath{fill}%
\end{pgfscope}%
\begin{pgfscope}%
\pgfpathrectangle{\pgfqpoint{1.150000in}{0.150000in}}{\pgfqpoint{5.700000in}{5.700000in}}%
\pgfusepath{clip}%
\pgfsetbuttcap%
\pgfsetroundjoin%
\definecolor{currentfill}{rgb}{0.274952,0.037752,0.364543}%
\pgfsetfillcolor{currentfill}%
\pgfsetfillopacity{0.800000}%
\pgfsetlinewidth{0.000000pt}%
\definecolor{currentstroke}{rgb}{0.000000,0.000000,0.000000}%
\pgfsetstrokecolor{currentstroke}%
\pgfsetdash{}{0pt}%
\pgfpathmoveto{\pgfqpoint{3.134852in}{1.361033in}}%
\pgfpathlineto{\pgfqpoint{3.148857in}{1.352167in}}%
\pgfpathlineto{\pgfqpoint{3.162862in}{1.343501in}}%
\pgfpathlineto{\pgfqpoint{3.176868in}{1.335032in}}%
\pgfpathlineto{\pgfqpoint{3.190875in}{1.326760in}}%
\pgfpathlineto{\pgfqpoint{3.199483in}{1.329206in}}%
\pgfpathlineto{\pgfqpoint{3.208076in}{1.331969in}}%
\pgfpathlineto{\pgfqpoint{3.216656in}{1.335041in}}%
\pgfpathlineto{\pgfqpoint{3.225222in}{1.338413in}}%
\pgfpathlineto{\pgfqpoint{3.211250in}{1.345923in}}%
\pgfpathlineto{\pgfqpoint{3.197279in}{1.353629in}}%
\pgfpathlineto{\pgfqpoint{3.183310in}{1.361534in}}%
\pgfpathlineto{\pgfqpoint{3.169341in}{1.369637in}}%
\pgfpathlineto{\pgfqpoint{3.160740in}{1.367013in}}%
\pgfpathlineto{\pgfqpoint{3.152126in}{1.364700in}}%
\pgfpathlineto{\pgfqpoint{3.143496in}{1.362704in}}%
\pgfpathlineto{\pgfqpoint{3.134852in}{1.361033in}}%
\pgfpathclose%
\pgfusepath{fill}%
\end{pgfscope}%
\begin{pgfscope}%
\pgfpathrectangle{\pgfqpoint{1.150000in}{0.150000in}}{\pgfqpoint{5.700000in}{5.700000in}}%
\pgfusepath{clip}%
\pgfsetbuttcap%
\pgfsetroundjoin%
\definecolor{currentfill}{rgb}{0.154815,0.493313,0.557840}%
\pgfsetfillcolor{currentfill}%
\pgfsetfillopacity{0.800000}%
\pgfsetlinewidth{0.000000pt}%
\definecolor{currentstroke}{rgb}{0.000000,0.000000,0.000000}%
\pgfsetstrokecolor{currentstroke}%
\pgfsetdash{}{0pt}%
\pgfpathmoveto{\pgfqpoint{2.114494in}{2.600006in}}%
\pgfpathlineto{\pgfqpoint{2.129034in}{2.573242in}}%
\pgfpathlineto{\pgfqpoint{2.143557in}{2.546801in}}%
\pgfpathlineto{\pgfqpoint{2.158063in}{2.520680in}}%
\pgfpathlineto{\pgfqpoint{2.172552in}{2.494875in}}%
\pgfpathlineto{\pgfqpoint{2.182168in}{2.482812in}}%
\pgfpathlineto{\pgfqpoint{2.191752in}{2.471240in}}%
\pgfpathlineto{\pgfqpoint{2.201305in}{2.460149in}}%
\pgfpathlineto{\pgfqpoint{2.210828in}{2.449531in}}%
\pgfpathlineto{\pgfqpoint{2.196416in}{2.474491in}}%
\pgfpathlineto{\pgfqpoint{2.181988in}{2.499765in}}%
\pgfpathlineto{\pgfqpoint{2.167544in}{2.525357in}}%
\pgfpathlineto{\pgfqpoint{2.153084in}{2.551269in}}%
\pgfpathlineto{\pgfqpoint{2.143484in}{2.562719in}}%
\pgfpathlineto{\pgfqpoint{2.133853in}{2.574652in}}%
\pgfpathlineto{\pgfqpoint{2.124190in}{2.587078in}}%
\pgfpathlineto{\pgfqpoint{2.114494in}{2.600006in}}%
\pgfpathclose%
\pgfusepath{fill}%
\end{pgfscope}%
\begin{pgfscope}%
\pgfpathrectangle{\pgfqpoint{1.150000in}{0.150000in}}{\pgfqpoint{5.700000in}{5.700000in}}%
\pgfusepath{clip}%
\pgfsetbuttcap%
\pgfsetroundjoin%
\definecolor{currentfill}{rgb}{0.229739,0.322361,0.545706}%
\pgfsetfillcolor{currentfill}%
\pgfsetfillopacity{0.800000}%
\pgfsetlinewidth{0.000000pt}%
\definecolor{currentstroke}{rgb}{0.000000,0.000000,0.000000}%
\pgfsetstrokecolor{currentstroke}%
\pgfsetdash{}{0pt}%
\pgfpathmoveto{\pgfqpoint{2.422091in}{2.077162in}}%
\pgfpathlineto{\pgfqpoint{2.436376in}{2.056560in}}%
\pgfpathlineto{\pgfqpoint{2.450650in}{2.036221in}}%
\pgfpathlineto{\pgfqpoint{2.464914in}{2.016146in}}%
\pgfpathlineto{\pgfqpoint{2.479169in}{1.996330in}}%
\pgfpathlineto{\pgfqpoint{2.488465in}{1.987488in}}%
\pgfpathlineto{\pgfqpoint{2.497735in}{1.979108in}}%
\pgfpathlineto{\pgfqpoint{2.506978in}{1.971182in}}%
\pgfpathlineto{\pgfqpoint{2.516195in}{1.963701in}}%
\pgfpathlineto{\pgfqpoint{2.502007in}{1.982675in}}%
\pgfpathlineto{\pgfqpoint{2.487809in}{2.001907in}}%
\pgfpathlineto{\pgfqpoint{2.473602in}{2.021400in}}%
\pgfpathlineto{\pgfqpoint{2.459385in}{2.041156in}}%
\pgfpathlineto{\pgfqpoint{2.450102in}{2.049465in}}%
\pgfpathlineto{\pgfqpoint{2.440793in}{2.058231in}}%
\pgfpathlineto{\pgfqpoint{2.431456in}{2.067460in}}%
\pgfpathlineto{\pgfqpoint{2.422091in}{2.077162in}}%
\pgfpathclose%
\pgfusepath{fill}%
\end{pgfscope}%
\begin{pgfscope}%
\pgfpathrectangle{\pgfqpoint{1.150000in}{0.150000in}}{\pgfqpoint{5.700000in}{5.700000in}}%
\pgfusepath{clip}%
\pgfsetbuttcap%
\pgfsetroundjoin%
\definecolor{currentfill}{rgb}{0.121831,0.589055,0.545623}%
\pgfsetfillcolor{currentfill}%
\pgfsetfillopacity{0.800000}%
\pgfsetlinewidth{0.000000pt}%
\definecolor{currentstroke}{rgb}{0.000000,0.000000,0.000000}%
\pgfsetstrokecolor{currentstroke}%
\pgfsetdash{}{0pt}%
\pgfpathmoveto{\pgfqpoint{4.992024in}{2.774306in}}%
\pgfpathlineto{\pgfqpoint{5.006691in}{2.789056in}}%
\pgfpathlineto{\pgfqpoint{5.021378in}{2.803993in}}%
\pgfpathlineto{\pgfqpoint{5.036085in}{2.819118in}}%
\pgfpathlineto{\pgfqpoint{5.050813in}{2.834431in}}%
\pgfpathlineto{\pgfqpoint{5.058655in}{2.844813in}}%
\pgfpathlineto{\pgfqpoint{5.066488in}{2.855015in}}%
\pgfpathlineto{\pgfqpoint{5.074313in}{2.865036in}}%
\pgfpathlineto{\pgfqpoint{5.082129in}{2.874878in}}%
\pgfpathlineto{\pgfqpoint{5.067403in}{2.859582in}}%
\pgfpathlineto{\pgfqpoint{5.052698in}{2.844475in}}%
\pgfpathlineto{\pgfqpoint{5.038013in}{2.829555in}}%
\pgfpathlineto{\pgfqpoint{5.023348in}{2.814822in}}%
\pgfpathlineto{\pgfqpoint{5.015529in}{2.804950in}}%
\pgfpathlineto{\pgfqpoint{5.007702in}{2.794907in}}%
\pgfpathlineto{\pgfqpoint{4.999867in}{2.784693in}}%
\pgfpathlineto{\pgfqpoint{4.992024in}{2.774306in}}%
\pgfpathclose%
\pgfusepath{fill}%
\end{pgfscope}%
\begin{pgfscope}%
\pgfpathrectangle{\pgfqpoint{1.150000in}{0.150000in}}{\pgfqpoint{5.700000in}{5.700000in}}%
\pgfusepath{clip}%
\pgfsetbuttcap%
\pgfsetroundjoin%
\definecolor{currentfill}{rgb}{0.274952,0.037752,0.364543}%
\pgfsetfillcolor{currentfill}%
\pgfsetfillopacity{0.800000}%
\pgfsetlinewidth{0.000000pt}%
\definecolor{currentstroke}{rgb}{0.000000,0.000000,0.000000}%
\pgfsetstrokecolor{currentstroke}%
\pgfsetdash{}{0pt}%
\pgfpathmoveto{\pgfqpoint{3.661590in}{1.326787in}}%
\pgfpathlineto{\pgfqpoint{3.675599in}{1.326081in}}%
\pgfpathlineto{\pgfqpoint{3.689616in}{1.325558in}}%
\pgfpathlineto{\pgfqpoint{3.703640in}{1.325217in}}%
\pgfpathlineto{\pgfqpoint{3.717671in}{1.325059in}}%
\pgfpathlineto{\pgfqpoint{3.725956in}{1.336177in}}%
\pgfpathlineto{\pgfqpoint{3.734235in}{1.347443in}}%
\pgfpathlineto{\pgfqpoint{3.742507in}{1.358850in}}%
\pgfpathlineto{\pgfqpoint{3.750773in}{1.370393in}}%
\pgfpathlineto{\pgfqpoint{3.736754in}{1.369920in}}%
\pgfpathlineto{\pgfqpoint{3.722743in}{1.369629in}}%
\pgfpathlineto{\pgfqpoint{3.708739in}{1.369522in}}%
\pgfpathlineto{\pgfqpoint{3.694743in}{1.369597in}}%
\pgfpathlineto{\pgfqpoint{3.686465in}{1.358674in}}%
\pgfpathlineto{\pgfqpoint{3.678180in}{1.347894in}}%
\pgfpathlineto{\pgfqpoint{3.669889in}{1.337263in}}%
\pgfpathlineto{\pgfqpoint{3.661590in}{1.326787in}}%
\pgfpathclose%
\pgfusepath{fill}%
\end{pgfscope}%
\begin{pgfscope}%
\pgfpathrectangle{\pgfqpoint{1.150000in}{0.150000in}}{\pgfqpoint{5.700000in}{5.700000in}}%
\pgfusepath{clip}%
\pgfsetbuttcap%
\pgfsetroundjoin%
\definecolor{currentfill}{rgb}{0.278791,0.062145,0.386592}%
\pgfsetfillcolor{currentfill}%
\pgfsetfillopacity{0.800000}%
\pgfsetlinewidth{0.000000pt}%
\definecolor{currentstroke}{rgb}{0.000000,0.000000,0.000000}%
\pgfsetstrokecolor{currentstroke}%
\pgfsetdash{}{0pt}%
\pgfpathmoveto{\pgfqpoint{3.750773in}{1.370393in}}%
\pgfpathlineto{\pgfqpoint{3.764800in}{1.371048in}}%
\pgfpathlineto{\pgfqpoint{3.778836in}{1.371885in}}%
\pgfpathlineto{\pgfqpoint{3.792879in}{1.372903in}}%
\pgfpathlineto{\pgfqpoint{3.806931in}{1.374102in}}%
\pgfpathlineto{\pgfqpoint{3.815181in}{1.386387in}}%
\pgfpathlineto{\pgfqpoint{3.823425in}{1.398787in}}%
\pgfpathlineto{\pgfqpoint{3.831664in}{1.411296in}}%
\pgfpathlineto{\pgfqpoint{3.839897in}{1.423910in}}%
\pgfpathlineto{\pgfqpoint{3.825854in}{1.422109in}}%
\pgfpathlineto{\pgfqpoint{3.811819in}{1.420490in}}%
\pgfpathlineto{\pgfqpoint{3.797794in}{1.419053in}}%
\pgfpathlineto{\pgfqpoint{3.783777in}{1.417797in}}%
\pgfpathlineto{\pgfqpoint{3.775535in}{1.405773in}}%
\pgfpathlineto{\pgfqpoint{3.767287in}{1.393860in}}%
\pgfpathlineto{\pgfqpoint{3.759033in}{1.382065in}}%
\pgfpathlineto{\pgfqpoint{3.750773in}{1.370393in}}%
\pgfpathclose%
\pgfusepath{fill}%
\end{pgfscope}%
\begin{pgfscope}%
\pgfpathrectangle{\pgfqpoint{1.150000in}{0.150000in}}{\pgfqpoint{5.700000in}{5.700000in}}%
\pgfusepath{clip}%
\pgfsetbuttcap%
\pgfsetroundjoin%
\definecolor{currentfill}{rgb}{0.278826,0.175490,0.483397}%
\pgfsetfillcolor{currentfill}%
\pgfsetfillopacity{0.800000}%
\pgfsetlinewidth{0.000000pt}%
\definecolor{currentstroke}{rgb}{0.000000,0.000000,0.000000}%
\pgfsetstrokecolor{currentstroke}%
\pgfsetdash{}{0pt}%
\pgfpathmoveto{\pgfqpoint{4.050860in}{1.615304in}}%
\pgfpathlineto{\pgfqpoint{4.064981in}{1.620355in}}%
\pgfpathlineto{\pgfqpoint{4.079115in}{1.625588in}}%
\pgfpathlineto{\pgfqpoint{4.093260in}{1.631001in}}%
\pgfpathlineto{\pgfqpoint{4.107417in}{1.636595in}}%
\pgfpathlineto{\pgfqpoint{4.115576in}{1.651549in}}%
\pgfpathlineto{\pgfqpoint{4.123732in}{1.666511in}}%
\pgfpathlineto{\pgfqpoint{4.131883in}{1.681479in}}%
\pgfpathlineto{\pgfqpoint{4.140030in}{1.696446in}}%
\pgfpathlineto{\pgfqpoint{4.125874in}{1.690372in}}%
\pgfpathlineto{\pgfqpoint{4.111729in}{1.684479in}}%
\pgfpathlineto{\pgfqpoint{4.097597in}{1.678767in}}%
\pgfpathlineto{\pgfqpoint{4.083477in}{1.673237in}}%
\pgfpathlineto{\pgfqpoint{4.075329in}{1.658737in}}%
\pgfpathlineto{\pgfqpoint{4.067177in}{1.644245in}}%
\pgfpathlineto{\pgfqpoint{4.059021in}{1.629766in}}%
\pgfpathlineto{\pgfqpoint{4.050860in}{1.615304in}}%
\pgfpathclose%
\pgfusepath{fill}%
\end{pgfscope}%
\begin{pgfscope}%
\pgfpathrectangle{\pgfqpoint{1.150000in}{0.150000in}}{\pgfqpoint{5.700000in}{5.700000in}}%
\pgfusepath{clip}%
\pgfsetbuttcap%
\pgfsetroundjoin%
\definecolor{currentfill}{rgb}{0.271305,0.019942,0.347269}%
\pgfsetfillcolor{currentfill}%
\pgfsetfillopacity{0.800000}%
\pgfsetlinewidth{0.000000pt}%
\definecolor{currentstroke}{rgb}{0.000000,0.000000,0.000000}%
\pgfsetstrokecolor{currentstroke}%
\pgfsetdash{}{0pt}%
\pgfpathmoveto{\pgfqpoint{3.572287in}{1.293873in}}%
\pgfpathlineto{\pgfqpoint{3.586287in}{1.291770in}}%
\pgfpathlineto{\pgfqpoint{3.600293in}{1.289852in}}%
\pgfpathlineto{\pgfqpoint{3.614304in}{1.288117in}}%
\pgfpathlineto{\pgfqpoint{3.628323in}{1.286566in}}%
\pgfpathlineto{\pgfqpoint{3.636651in}{1.296356in}}%
\pgfpathlineto{\pgfqpoint{3.644971in}{1.306328in}}%
\pgfpathlineto{\pgfqpoint{3.653284in}{1.316474in}}%
\pgfpathlineto{\pgfqpoint{3.661590in}{1.326787in}}%
\pgfpathlineto{\pgfqpoint{3.647587in}{1.327677in}}%
\pgfpathlineto{\pgfqpoint{3.633592in}{1.328749in}}%
\pgfpathlineto{\pgfqpoint{3.619603in}{1.330006in}}%
\pgfpathlineto{\pgfqpoint{3.605620in}{1.331447in}}%
\pgfpathlineto{\pgfqpoint{3.597299in}{1.321783in}}%
\pgfpathlineto{\pgfqpoint{3.588969in}{1.312295in}}%
\pgfpathlineto{\pgfqpoint{3.580632in}{1.302990in}}%
\pgfpathlineto{\pgfqpoint{3.572287in}{1.293873in}}%
\pgfpathclose%
\pgfusepath{fill}%
\end{pgfscope}%
\begin{pgfscope}%
\pgfpathrectangle{\pgfqpoint{1.150000in}{0.150000in}}{\pgfqpoint{5.700000in}{5.700000in}}%
\pgfusepath{clip}%
\pgfsetbuttcap%
\pgfsetroundjoin%
\definecolor{currentfill}{rgb}{0.283197,0.115680,0.436115}%
\pgfsetfillcolor{currentfill}%
\pgfsetfillopacity{0.800000}%
\pgfsetlinewidth{0.000000pt}%
\definecolor{currentstroke}{rgb}{0.000000,0.000000,0.000000}%
\pgfsetstrokecolor{currentstroke}%
\pgfsetdash{}{0pt}%
\pgfpathmoveto{\pgfqpoint{2.875256in}{1.540246in}}%
\pgfpathlineto{\pgfqpoint{2.889326in}{1.527265in}}%
\pgfpathlineto{\pgfqpoint{2.903394in}{1.514499in}}%
\pgfpathlineto{\pgfqpoint{2.917458in}{1.501947in}}%
\pgfpathlineto{\pgfqpoint{2.931520in}{1.489608in}}%
\pgfpathlineto{\pgfqpoint{2.940359in}{1.487452in}}%
\pgfpathlineto{\pgfqpoint{2.949180in}{1.485680in}}%
\pgfpathlineto{\pgfqpoint{2.957981in}{1.484285in}}%
\pgfpathlineto{\pgfqpoint{2.966766in}{1.483258in}}%
\pgfpathlineto{\pgfqpoint{2.952750in}{1.494791in}}%
\pgfpathlineto{\pgfqpoint{2.938733in}{1.506537in}}%
\pgfpathlineto{\pgfqpoint{2.924713in}{1.518495in}}%
\pgfpathlineto{\pgfqpoint{2.910691in}{1.530668in}}%
\pgfpathlineto{\pgfqpoint{2.901861in}{1.532488in}}%
\pgfpathlineto{\pgfqpoint{2.893012in}{1.534685in}}%
\pgfpathlineto{\pgfqpoint{2.884144in}{1.537269in}}%
\pgfpathlineto{\pgfqpoint{2.875256in}{1.540246in}}%
\pgfpathclose%
\pgfusepath{fill}%
\end{pgfscope}%
\begin{pgfscope}%
\pgfpathrectangle{\pgfqpoint{1.150000in}{0.150000in}}{\pgfqpoint{5.700000in}{5.700000in}}%
\pgfusepath{clip}%
\pgfsetbuttcap%
\pgfsetroundjoin%
\definecolor{currentfill}{rgb}{0.352360,0.783011,0.392636}%
\pgfsetfillcolor{currentfill}%
\pgfsetfillopacity{0.800000}%
\pgfsetlinewidth{0.000000pt}%
\definecolor{currentstroke}{rgb}{0.000000,0.000000,0.000000}%
\pgfsetstrokecolor{currentstroke}%
\pgfsetdash{}{0pt}%
\pgfpathmoveto{\pgfqpoint{5.625112in}{3.420746in}}%
\pgfpathlineto{\pgfqpoint{5.640234in}{3.438566in}}%
\pgfpathlineto{\pgfqpoint{5.655380in}{3.456576in}}%
\pgfpathlineto{\pgfqpoint{5.670549in}{3.474776in}}%
\pgfpathlineto{\pgfqpoint{5.685744in}{3.493166in}}%
\pgfpathlineto{\pgfqpoint{5.693159in}{3.496096in}}%
\pgfpathlineto{\pgfqpoint{5.700562in}{3.498872in}}%
\pgfpathlineto{\pgfqpoint{5.707953in}{3.501497in}}%
\pgfpathlineto{\pgfqpoint{5.715334in}{3.503975in}}%
\pgfpathlineto{\pgfqpoint{5.700159in}{3.485932in}}%
\pgfpathlineto{\pgfqpoint{5.685008in}{3.468079in}}%
\pgfpathlineto{\pgfqpoint{5.669882in}{3.450414in}}%
\pgfpathlineto{\pgfqpoint{5.654779in}{3.432938in}}%
\pgfpathlineto{\pgfqpoint{5.647379in}{3.430102in}}%
\pgfpathlineto{\pgfqpoint{5.639968in}{3.427127in}}%
\pgfpathlineto{\pgfqpoint{5.632546in}{3.424009in}}%
\pgfpathlineto{\pgfqpoint{5.625112in}{3.420746in}}%
\pgfpathclose%
\pgfusepath{fill}%
\end{pgfscope}%
\begin{pgfscope}%
\pgfpathrectangle{\pgfqpoint{1.150000in}{0.150000in}}{\pgfqpoint{5.700000in}{5.700000in}}%
\pgfusepath{clip}%
\pgfsetbuttcap%
\pgfsetroundjoin%
\definecolor{currentfill}{rgb}{0.281924,0.089666,0.412415}%
\pgfsetfillcolor{currentfill}%
\pgfsetfillopacity{0.800000}%
\pgfsetlinewidth{0.000000pt}%
\definecolor{currentstroke}{rgb}{0.000000,0.000000,0.000000}%
\pgfsetstrokecolor{currentstroke}%
\pgfsetdash{}{0pt}%
\pgfpathmoveto{\pgfqpoint{3.839897in}{1.423910in}}%
\pgfpathlineto{\pgfqpoint{3.853949in}{1.425892in}}%
\pgfpathlineto{\pgfqpoint{3.868011in}{1.428055in}}%
\pgfpathlineto{\pgfqpoint{3.882082in}{1.430399in}}%
\pgfpathlineto{\pgfqpoint{3.896162in}{1.432924in}}%
\pgfpathlineto{\pgfqpoint{3.904383in}{1.446219in}}%
\pgfpathlineto{\pgfqpoint{3.912598in}{1.459599in}}%
\pgfpathlineto{\pgfqpoint{3.920808in}{1.473058in}}%
\pgfpathlineto{\pgfqpoint{3.929014in}{1.486591in}}%
\pgfpathlineto{\pgfqpoint{3.914939in}{1.483495in}}%
\pgfpathlineto{\pgfqpoint{3.900875in}{1.480580in}}%
\pgfpathlineto{\pgfqpoint{3.886820in}{1.477846in}}%
\pgfpathlineto{\pgfqpoint{3.872775in}{1.475294in}}%
\pgfpathlineto{\pgfqpoint{3.864563in}{1.462320in}}%
\pgfpathlineto{\pgfqpoint{3.856346in}{1.449427in}}%
\pgfpathlineto{\pgfqpoint{3.848124in}{1.436622in}}%
\pgfpathlineto{\pgfqpoint{3.839897in}{1.423910in}}%
\pgfpathclose%
\pgfusepath{fill}%
\end{pgfscope}%
\begin{pgfscope}%
\pgfpathrectangle{\pgfqpoint{1.150000in}{0.150000in}}{\pgfqpoint{5.700000in}{5.700000in}}%
\pgfusepath{clip}%
\pgfsetbuttcap%
\pgfsetroundjoin%
\definecolor{currentfill}{rgb}{0.216210,0.351535,0.550627}%
\pgfsetfillcolor{currentfill}%
\pgfsetfillopacity{0.800000}%
\pgfsetlinewidth{0.000000pt}%
\definecolor{currentstroke}{rgb}{0.000000,0.000000,0.000000}%
\pgfsetstrokecolor{currentstroke}%
\pgfsetdash{}{0pt}%
\pgfpathmoveto{\pgfqpoint{2.364842in}{2.162260in}}%
\pgfpathlineto{\pgfqpoint{2.379171in}{2.140579in}}%
\pgfpathlineto{\pgfqpoint{2.393489in}{2.119170in}}%
\pgfpathlineto{\pgfqpoint{2.407795in}{2.098032in}}%
\pgfpathlineto{\pgfqpoint{2.422091in}{2.077162in}}%
\pgfpathlineto{\pgfqpoint{2.431456in}{2.067460in}}%
\pgfpathlineto{\pgfqpoint{2.440793in}{2.058231in}}%
\pgfpathlineto{\pgfqpoint{2.450102in}{2.049465in}}%
\pgfpathlineto{\pgfqpoint{2.459385in}{2.041156in}}%
\pgfpathlineto{\pgfqpoint{2.445157in}{2.061176in}}%
\pgfpathlineto{\pgfqpoint{2.430920in}{2.081464in}}%
\pgfpathlineto{\pgfqpoint{2.416672in}{2.102021in}}%
\pgfpathlineto{\pgfqpoint{2.402413in}{2.122849in}}%
\pgfpathlineto{\pgfqpoint{2.393063in}{2.131994in}}%
\pgfpathlineto{\pgfqpoint{2.383684in}{2.141605in}}%
\pgfpathlineto{\pgfqpoint{2.374278in}{2.151691in}}%
\pgfpathlineto{\pgfqpoint{2.364842in}{2.162260in}}%
\pgfpathclose%
\pgfusepath{fill}%
\end{pgfscope}%
\begin{pgfscope}%
\pgfpathrectangle{\pgfqpoint{1.150000in}{0.150000in}}{\pgfqpoint{5.700000in}{5.700000in}}%
\pgfusepath{clip}%
\pgfsetbuttcap%
\pgfsetroundjoin%
\definecolor{currentfill}{rgb}{0.135066,0.544853,0.554029}%
\pgfsetfillcolor{currentfill}%
\pgfsetfillopacity{0.800000}%
\pgfsetlinewidth{0.000000pt}%
\definecolor{currentstroke}{rgb}{0.000000,0.000000,0.000000}%
\pgfsetstrokecolor{currentstroke}%
\pgfsetdash{}{0pt}%
\pgfpathmoveto{\pgfqpoint{4.870521in}{2.628135in}}%
\pgfpathlineto{\pgfqpoint{4.885109in}{2.642065in}}%
\pgfpathlineto{\pgfqpoint{4.899717in}{2.656182in}}%
\pgfpathlineto{\pgfqpoint{4.914344in}{2.670486in}}%
\pgfpathlineto{\pgfqpoint{4.928991in}{2.684977in}}%
\pgfpathlineto{\pgfqpoint{4.936898in}{2.696751in}}%
\pgfpathlineto{\pgfqpoint{4.944796in}{2.708352in}}%
\pgfpathlineto{\pgfqpoint{4.952687in}{2.719778in}}%
\pgfpathlineto{\pgfqpoint{4.960570in}{2.731031in}}%
\pgfpathlineto{\pgfqpoint{4.945923in}{2.716487in}}%
\pgfpathlineto{\pgfqpoint{4.931295in}{2.702129in}}%
\pgfpathlineto{\pgfqpoint{4.916687in}{2.687959in}}%
\pgfpathlineto{\pgfqpoint{4.902099in}{2.673975in}}%
\pgfpathlineto{\pgfqpoint{4.894215in}{2.662763in}}%
\pgfpathlineto{\pgfqpoint{4.886325in}{2.651385in}}%
\pgfpathlineto{\pgfqpoint{4.878426in}{2.639843in}}%
\pgfpathlineto{\pgfqpoint{4.870521in}{2.628135in}}%
\pgfpathclose%
\pgfusepath{fill}%
\end{pgfscope}%
\begin{pgfscope}%
\pgfpathrectangle{\pgfqpoint{1.150000in}{0.150000in}}{\pgfqpoint{5.700000in}{5.700000in}}%
\pgfusepath{clip}%
\pgfsetbuttcap%
\pgfsetroundjoin%
\definecolor{currentfill}{rgb}{0.268510,0.009605,0.335427}%
\pgfsetfillcolor{currentfill}%
\pgfsetfillopacity{0.800000}%
\pgfsetlinewidth{0.000000pt}%
\definecolor{currentstroke}{rgb}{0.000000,0.000000,0.000000}%
\pgfsetstrokecolor{currentstroke}%
\pgfsetdash{}{0pt}%
\pgfpathmoveto{\pgfqpoint{3.337070in}{1.285343in}}%
\pgfpathlineto{\pgfqpoint{3.351062in}{1.279575in}}%
\pgfpathlineto{\pgfqpoint{3.365057in}{1.273997in}}%
\pgfpathlineto{\pgfqpoint{3.379056in}{1.268608in}}%
\pgfpathlineto{\pgfqpoint{3.393058in}{1.263409in}}%
\pgfpathlineto{\pgfqpoint{3.401523in}{1.269292in}}%
\pgfpathlineto{\pgfqpoint{3.409977in}{1.275437in}}%
\pgfpathlineto{\pgfqpoint{3.418420in}{1.281834in}}%
\pgfpathlineto{\pgfqpoint{3.426853in}{1.288478in}}%
\pgfpathlineto{\pgfqpoint{3.412877in}{1.292952in}}%
\pgfpathlineto{\pgfqpoint{3.398905in}{1.297614in}}%
\pgfpathlineto{\pgfqpoint{3.384936in}{1.302466in}}%
\pgfpathlineto{\pgfqpoint{3.370971in}{1.307508in}}%
\pgfpathlineto{\pgfqpoint{3.362513in}{1.301578in}}%
\pgfpathlineto{\pgfqpoint{3.354043in}{1.295902in}}%
\pgfpathlineto{\pgfqpoint{3.345562in}{1.290488in}}%
\pgfpathlineto{\pgfqpoint{3.337070in}{1.285343in}}%
\pgfpathclose%
\pgfusepath{fill}%
\end{pgfscope}%
\begin{pgfscope}%
\pgfpathrectangle{\pgfqpoint{1.150000in}{0.150000in}}{\pgfqpoint{5.700000in}{5.700000in}}%
\pgfusepath{clip}%
\pgfsetbuttcap%
\pgfsetroundjoin%
\definecolor{currentfill}{rgb}{0.120081,0.622161,0.534946}%
\pgfsetfillcolor{currentfill}%
\pgfsetfillopacity{0.800000}%
\pgfsetlinewidth{0.000000pt}%
\definecolor{currentstroke}{rgb}{0.000000,0.000000,0.000000}%
\pgfsetstrokecolor{currentstroke}%
\pgfsetdash{}{0pt}%
\pgfpathmoveto{\pgfqpoint{1.919179in}{3.012141in}}%
\pgfpathlineto{\pgfqpoint{1.933954in}{2.980643in}}%
\pgfpathlineto{\pgfqpoint{1.948706in}{2.949521in}}%
\pgfpathlineto{\pgfqpoint{1.963436in}{2.918773in}}%
\pgfpathlineto{\pgfqpoint{1.978144in}{2.888393in}}%
\pgfpathlineto{\pgfqpoint{1.987956in}{2.874899in}}%
\pgfpathlineto{\pgfqpoint{1.997734in}{2.861904in}}%
\pgfpathlineto{\pgfqpoint{2.007478in}{2.849401in}}%
\pgfpathlineto{\pgfqpoint{2.017190in}{2.837380in}}%
\pgfpathlineto{\pgfqpoint{2.002566in}{2.866920in}}%
\pgfpathlineto{\pgfqpoint{1.987921in}{2.896826in}}%
\pgfpathlineto{\pgfqpoint{1.973255in}{2.927102in}}%
\pgfpathlineto{\pgfqpoint{1.958566in}{2.957752in}}%
\pgfpathlineto{\pgfqpoint{1.948771in}{2.970599in}}%
\pgfpathlineto{\pgfqpoint{1.938942in}{2.983941in}}%
\pgfpathlineto{\pgfqpoint{1.929078in}{2.997785in}}%
\pgfpathlineto{\pgfqpoint{1.919179in}{3.012141in}}%
\pgfpathclose%
\pgfusepath{fill}%
\end{pgfscope}%
\begin{pgfscope}%
\pgfpathrectangle{\pgfqpoint{1.150000in}{0.150000in}}{\pgfqpoint{5.700000in}{5.700000in}}%
\pgfusepath{clip}%
\pgfsetbuttcap%
\pgfsetroundjoin%
\definecolor{currentfill}{rgb}{0.153364,0.497000,0.557724}%
\pgfsetfillcolor{currentfill}%
\pgfsetfillopacity{0.800000}%
\pgfsetlinewidth{0.000000pt}%
\definecolor{currentstroke}{rgb}{0.000000,0.000000,0.000000}%
\pgfsetstrokecolor{currentstroke}%
\pgfsetdash{}{0pt}%
\pgfpathmoveto{\pgfqpoint{4.748870in}{2.475568in}}%
\pgfpathlineto{\pgfqpoint{4.763380in}{2.488543in}}%
\pgfpathlineto{\pgfqpoint{4.777908in}{2.501703in}}%
\pgfpathlineto{\pgfqpoint{4.792455in}{2.515050in}}%
\pgfpathlineto{\pgfqpoint{4.807021in}{2.528582in}}%
\pgfpathlineto{\pgfqpoint{4.814982in}{2.541594in}}%
\pgfpathlineto{\pgfqpoint{4.822937in}{2.554445in}}%
\pgfpathlineto{\pgfqpoint{4.830885in}{2.567135in}}%
\pgfpathlineto{\pgfqpoint{4.838827in}{2.579662in}}%
\pgfpathlineto{\pgfqpoint{4.824259in}{2.566006in}}%
\pgfpathlineto{\pgfqpoint{4.809710in}{2.552537in}}%
\pgfpathlineto{\pgfqpoint{4.795179in}{2.539254in}}%
\pgfpathlineto{\pgfqpoint{4.780668in}{2.526157in}}%
\pgfpathlineto{\pgfqpoint{4.772728in}{2.513739in}}%
\pgfpathlineto{\pgfqpoint{4.764782in}{2.501168in}}%
\pgfpathlineto{\pgfqpoint{4.756830in}{2.488444in}}%
\pgfpathlineto{\pgfqpoint{4.748870in}{2.475568in}}%
\pgfpathclose%
\pgfusepath{fill}%
\end{pgfscope}%
\begin{pgfscope}%
\pgfpathrectangle{\pgfqpoint{1.150000in}{0.150000in}}{\pgfqpoint{5.700000in}{5.700000in}}%
\pgfusepath{clip}%
\pgfsetbuttcap%
\pgfsetroundjoin%
\definecolor{currentfill}{rgb}{0.140210,0.665859,0.513427}%
\pgfsetfillcolor{currentfill}%
\pgfsetfillopacity{0.800000}%
\pgfsetlinewidth{0.000000pt}%
\definecolor{currentstroke}{rgb}{0.000000,0.000000,0.000000}%
\pgfsetstrokecolor{currentstroke}%
\pgfsetdash{}{0pt}%
\pgfpathmoveto{\pgfqpoint{5.203422in}{3.009752in}}%
\pgfpathlineto{\pgfqpoint{5.218245in}{3.025846in}}%
\pgfpathlineto{\pgfqpoint{5.233091in}{3.042129in}}%
\pgfpathlineto{\pgfqpoint{5.247958in}{3.058601in}}%
\pgfpathlineto{\pgfqpoint{5.262847in}{3.075262in}}%
\pgfpathlineto{\pgfqpoint{5.270572in}{3.083310in}}%
\pgfpathlineto{\pgfqpoint{5.278286in}{3.091173in}}%
\pgfpathlineto{\pgfqpoint{5.285991in}{3.098851in}}%
\pgfpathlineto{\pgfqpoint{5.293686in}{3.106348in}}%
\pgfpathlineto{\pgfqpoint{5.278803in}{3.089813in}}%
\pgfpathlineto{\pgfqpoint{5.263942in}{3.073467in}}%
\pgfpathlineto{\pgfqpoint{5.249103in}{3.057309in}}%
\pgfpathlineto{\pgfqpoint{5.234286in}{3.041339in}}%
\pgfpathlineto{\pgfqpoint{5.226584in}{3.033704in}}%
\pgfpathlineto{\pgfqpoint{5.218873in}{3.025895in}}%
\pgfpathlineto{\pgfqpoint{5.211152in}{3.017912in}}%
\pgfpathlineto{\pgfqpoint{5.203422in}{3.009752in}}%
\pgfpathclose%
\pgfusepath{fill}%
\end{pgfscope}%
\begin{pgfscope}%
\pgfpathrectangle{\pgfqpoint{1.150000in}{0.150000in}}{\pgfqpoint{5.700000in}{5.700000in}}%
\pgfusepath{clip}%
\pgfsetbuttcap%
\pgfsetroundjoin%
\definecolor{currentfill}{rgb}{0.174274,0.445044,0.557792}%
\pgfsetfillcolor{currentfill}%
\pgfsetfillopacity{0.800000}%
\pgfsetlinewidth{0.000000pt}%
\definecolor{currentstroke}{rgb}{0.000000,0.000000,0.000000}%
\pgfsetstrokecolor{currentstroke}%
\pgfsetdash{}{0pt}%
\pgfpathmoveto{\pgfqpoint{4.627136in}{2.318559in}}%
\pgfpathlineto{\pgfqpoint{4.641568in}{2.330445in}}%
\pgfpathlineto{\pgfqpoint{4.656018in}{2.342515in}}%
\pgfpathlineto{\pgfqpoint{4.670486in}{2.354771in}}%
\pgfpathlineto{\pgfqpoint{4.684971in}{2.367211in}}%
\pgfpathlineto{\pgfqpoint{4.692980in}{2.381264in}}%
\pgfpathlineto{\pgfqpoint{4.700982in}{2.395175in}}%
\pgfpathlineto{\pgfqpoint{4.708979in}{2.408941in}}%
\pgfpathlineto{\pgfqpoint{4.716970in}{2.422563in}}%
\pgfpathlineto{\pgfqpoint{4.702481in}{2.409931in}}%
\pgfpathlineto{\pgfqpoint{4.688010in}{2.397484in}}%
\pgfpathlineto{\pgfqpoint{4.673557in}{2.385223in}}%
\pgfpathlineto{\pgfqpoint{4.659122in}{2.373147in}}%
\pgfpathlineto{\pgfqpoint{4.651134in}{2.359704in}}%
\pgfpathlineto{\pgfqpoint{4.643140in}{2.346124in}}%
\pgfpathlineto{\pgfqpoint{4.635141in}{2.332409in}}%
\pgfpathlineto{\pgfqpoint{4.627136in}{2.318559in}}%
\pgfpathclose%
\pgfusepath{fill}%
\end{pgfscope}%
\begin{pgfscope}%
\pgfpathrectangle{\pgfqpoint{1.150000in}{0.150000in}}{\pgfqpoint{5.700000in}{5.700000in}}%
\pgfusepath{clip}%
\pgfsetbuttcap%
\pgfsetroundjoin%
\definecolor{currentfill}{rgb}{0.197636,0.391528,0.554969}%
\pgfsetfillcolor{currentfill}%
\pgfsetfillopacity{0.800000}%
\pgfsetlinewidth{0.000000pt}%
\definecolor{currentstroke}{rgb}{0.000000,0.000000,0.000000}%
\pgfsetstrokecolor{currentstroke}%
\pgfsetdash{}{0pt}%
\pgfpathmoveto{\pgfqpoint{4.505367in}{2.159389in}}%
\pgfpathlineto{\pgfqpoint{4.519725in}{2.170055in}}%
\pgfpathlineto{\pgfqpoint{4.534099in}{2.180905in}}%
\pgfpathlineto{\pgfqpoint{4.548490in}{2.191939in}}%
\pgfpathlineto{\pgfqpoint{4.562897in}{2.203158in}}%
\pgfpathlineto{\pgfqpoint{4.570946in}{2.218013in}}%
\pgfpathlineto{\pgfqpoint{4.578989in}{2.232749in}}%
\pgfpathlineto{\pgfqpoint{4.587027in}{2.247365in}}%
\pgfpathlineto{\pgfqpoint{4.595060in}{2.261858in}}%
\pgfpathlineto{\pgfqpoint{4.580648in}{2.250381in}}%
\pgfpathlineto{\pgfqpoint{4.566253in}{2.239089in}}%
\pgfpathlineto{\pgfqpoint{4.551875in}{2.227981in}}%
\pgfpathlineto{\pgfqpoint{4.537514in}{2.217058in}}%
\pgfpathlineto{\pgfqpoint{4.529485in}{2.202810in}}%
\pgfpathlineto{\pgfqpoint{4.521450in}{2.188448in}}%
\pgfpathlineto{\pgfqpoint{4.513411in}{2.173974in}}%
\pgfpathlineto{\pgfqpoint{4.505367in}{2.159389in}}%
\pgfpathclose%
\pgfusepath{fill}%
\end{pgfscope}%
\begin{pgfscope}%
\pgfpathrectangle{\pgfqpoint{1.150000in}{0.150000in}}{\pgfqpoint{5.700000in}{5.700000in}}%
\pgfusepath{clip}%
\pgfsetbuttcap%
\pgfsetroundjoin%
\definecolor{currentfill}{rgb}{0.282656,0.100196,0.422160}%
\pgfsetfillcolor{currentfill}%
\pgfsetfillopacity{0.800000}%
\pgfsetlinewidth{0.000000pt}%
\definecolor{currentstroke}{rgb}{0.000000,0.000000,0.000000}%
\pgfsetstrokecolor{currentstroke}%
\pgfsetdash{}{0pt}%
\pgfpathmoveto{\pgfqpoint{2.931520in}{1.489608in}}%
\pgfpathlineto{\pgfqpoint{2.945580in}{1.477481in}}%
\pgfpathlineto{\pgfqpoint{2.959637in}{1.465564in}}%
\pgfpathlineto{\pgfqpoint{2.973692in}{1.453857in}}%
\pgfpathlineto{\pgfqpoint{2.987745in}{1.442358in}}%
\pgfpathlineto{\pgfqpoint{2.996537in}{1.441020in}}%
\pgfpathlineto{\pgfqpoint{3.005312in}{1.440057in}}%
\pgfpathlineto{\pgfqpoint{3.014069in}{1.439461in}}%
\pgfpathlineto{\pgfqpoint{3.022809in}{1.439225in}}%
\pgfpathlineto{\pgfqpoint{3.008801in}{1.449920in}}%
\pgfpathlineto{\pgfqpoint{2.994790in}{1.460823in}}%
\pgfpathlineto{\pgfqpoint{2.980779in}{1.471936in}}%
\pgfpathlineto{\pgfqpoint{2.966766in}{1.483258in}}%
\pgfpathlineto{\pgfqpoint{2.957981in}{1.484285in}}%
\pgfpathlineto{\pgfqpoint{2.949180in}{1.485680in}}%
\pgfpathlineto{\pgfqpoint{2.940359in}{1.487452in}}%
\pgfpathlineto{\pgfqpoint{2.931520in}{1.489608in}}%
\pgfpathclose%
\pgfusepath{fill}%
\end{pgfscope}%
\begin{pgfscope}%
\pgfpathrectangle{\pgfqpoint{1.150000in}{0.150000in}}{\pgfqpoint{5.700000in}{5.700000in}}%
\pgfusepath{clip}%
\pgfsetbuttcap%
\pgfsetroundjoin%
\definecolor{currentfill}{rgb}{0.221989,0.339161,0.548752}%
\pgfsetfillcolor{currentfill}%
\pgfsetfillopacity{0.800000}%
\pgfsetlinewidth{0.000000pt}%
\definecolor{currentstroke}{rgb}{0.000000,0.000000,0.000000}%
\pgfsetstrokecolor{currentstroke}%
\pgfsetdash{}{0pt}%
\pgfpathmoveto{\pgfqpoint{4.383592in}{2.000655in}}%
\pgfpathlineto{\pgfqpoint{4.397880in}{2.009975in}}%
\pgfpathlineto{\pgfqpoint{4.412184in}{2.019478in}}%
\pgfpathlineto{\pgfqpoint{4.426502in}{2.029163in}}%
\pgfpathlineto{\pgfqpoint{4.440836in}{2.039032in}}%
\pgfpathlineto{\pgfqpoint{4.448919in}{2.054410in}}%
\pgfpathlineto{\pgfqpoint{4.456997in}{2.069698in}}%
\pgfpathlineto{\pgfqpoint{4.465071in}{2.084893in}}%
\pgfpathlineto{\pgfqpoint{4.473140in}{2.099994in}}%
\pgfpathlineto{\pgfqpoint{4.458801in}{2.089801in}}%
\pgfpathlineto{\pgfqpoint{4.444479in}{2.079792in}}%
\pgfpathlineto{\pgfqpoint{4.430172in}{2.069967in}}%
\pgfpathlineto{\pgfqpoint{4.415881in}{2.060324in}}%
\pgfpathlineto{\pgfqpoint{4.407815in}{2.045535in}}%
\pgfpathlineto{\pgfqpoint{4.399746in}{2.030658in}}%
\pgfpathlineto{\pgfqpoint{4.391671in}{2.015697in}}%
\pgfpathlineto{\pgfqpoint{4.383592in}{2.000655in}}%
\pgfpathclose%
\pgfusepath{fill}%
\end{pgfscope}%
\begin{pgfscope}%
\pgfpathrectangle{\pgfqpoint{1.150000in}{0.150000in}}{\pgfqpoint{5.700000in}{5.700000in}}%
\pgfusepath{clip}%
\pgfsetbuttcap%
\pgfsetroundjoin%
\definecolor{currentfill}{rgb}{0.232815,0.732247,0.459277}%
\pgfsetfillcolor{currentfill}%
\pgfsetfillopacity{0.800000}%
\pgfsetlinewidth{0.000000pt}%
\definecolor{currentstroke}{rgb}{0.000000,0.000000,0.000000}%
\pgfsetstrokecolor{currentstroke}%
\pgfsetdash{}{0pt}%
\pgfpathmoveto{\pgfqpoint{5.414563in}{3.226608in}}%
\pgfpathlineto{\pgfqpoint{5.429540in}{3.243728in}}%
\pgfpathlineto{\pgfqpoint{5.444540in}{3.261038in}}%
\pgfpathlineto{\pgfqpoint{5.459563in}{3.278537in}}%
\pgfpathlineto{\pgfqpoint{5.474610in}{3.296227in}}%
\pgfpathlineto{\pgfqpoint{5.482191in}{3.301739in}}%
\pgfpathlineto{\pgfqpoint{5.489761in}{3.307074in}}%
\pgfpathlineto{\pgfqpoint{5.497321in}{3.312235in}}%
\pgfpathlineto{\pgfqpoint{5.504869in}{3.317225in}}%
\pgfpathlineto{\pgfqpoint{5.489835in}{3.299772in}}%
\pgfpathlineto{\pgfqpoint{5.474824in}{3.282508in}}%
\pgfpathlineto{\pgfqpoint{5.459836in}{3.265433in}}%
\pgfpathlineto{\pgfqpoint{5.444872in}{3.248546in}}%
\pgfpathlineto{\pgfqpoint{5.437310in}{3.243308in}}%
\pgfpathlineto{\pgfqpoint{5.429738in}{3.237907in}}%
\pgfpathlineto{\pgfqpoint{5.422156in}{3.232341in}}%
\pgfpathlineto{\pgfqpoint{5.414563in}{3.226608in}}%
\pgfpathclose%
\pgfusepath{fill}%
\end{pgfscope}%
\begin{pgfscope}%
\pgfpathrectangle{\pgfqpoint{1.150000in}{0.150000in}}{\pgfqpoint{5.700000in}{5.700000in}}%
\pgfusepath{clip}%
\pgfsetbuttcap%
\pgfsetroundjoin%
\definecolor{currentfill}{rgb}{0.268510,0.009605,0.335427}%
\pgfsetfillcolor{currentfill}%
\pgfsetfillopacity{0.800000}%
\pgfsetlinewidth{0.000000pt}%
\definecolor{currentstroke}{rgb}{0.000000,0.000000,0.000000}%
\pgfsetstrokecolor{currentstroke}%
\pgfsetdash{}{0pt}%
\pgfpathmoveto{\pgfqpoint{3.482800in}{1.272463in}}%
\pgfpathlineto{\pgfqpoint{3.496799in}{1.268926in}}%
\pgfpathlineto{\pgfqpoint{3.510802in}{1.265575in}}%
\pgfpathlineto{\pgfqpoint{3.524810in}{1.262410in}}%
\pgfpathlineto{\pgfqpoint{3.538823in}{1.259429in}}%
\pgfpathlineto{\pgfqpoint{3.547202in}{1.267723in}}%
\pgfpathlineto{\pgfqpoint{3.555572in}{1.276233in}}%
\pgfpathlineto{\pgfqpoint{3.563934in}{1.284952in}}%
\pgfpathlineto{\pgfqpoint{3.572287in}{1.293873in}}%
\pgfpathlineto{\pgfqpoint{3.558294in}{1.296160in}}%
\pgfpathlineto{\pgfqpoint{3.544305in}{1.298632in}}%
\pgfpathlineto{\pgfqpoint{3.530323in}{1.301290in}}%
\pgfpathlineto{\pgfqpoint{3.516346in}{1.304133in}}%
\pgfpathlineto{\pgfqpoint{3.507973in}{1.295893in}}%
\pgfpathlineto{\pgfqpoint{3.499591in}{1.287864in}}%
\pgfpathlineto{\pgfqpoint{3.491201in}{1.280051in}}%
\pgfpathlineto{\pgfqpoint{3.482800in}{1.272463in}}%
\pgfpathclose%
\pgfusepath{fill}%
\end{pgfscope}%
\begin{pgfscope}%
\pgfpathrectangle{\pgfqpoint{1.150000in}{0.150000in}}{\pgfqpoint{5.700000in}{5.700000in}}%
\pgfusepath{clip}%
\pgfsetbuttcap%
\pgfsetroundjoin%
\definecolor{currentfill}{rgb}{0.283187,0.125848,0.444960}%
\pgfsetfillcolor{currentfill}%
\pgfsetfillopacity{0.800000}%
\pgfsetlinewidth{0.000000pt}%
\definecolor{currentstroke}{rgb}{0.000000,0.000000,0.000000}%
\pgfsetstrokecolor{currentstroke}%
\pgfsetdash{}{0pt}%
\pgfpathmoveto{\pgfqpoint{3.929014in}{1.486591in}}%
\pgfpathlineto{\pgfqpoint{3.943099in}{1.489868in}}%
\pgfpathlineto{\pgfqpoint{3.957194in}{1.493326in}}%
\pgfpathlineto{\pgfqpoint{3.971299in}{1.496964in}}%
\pgfpathlineto{\pgfqpoint{3.985415in}{1.500782in}}%
\pgfpathlineto{\pgfqpoint{3.993611in}{1.514938in}}%
\pgfpathlineto{\pgfqpoint{4.001803in}{1.529149in}}%
\pgfpathlineto{\pgfqpoint{4.009990in}{1.543410in}}%
\pgfpathlineto{\pgfqpoint{4.018173in}{1.557717in}}%
\pgfpathlineto{\pgfqpoint{4.004060in}{1.553357in}}%
\pgfpathlineto{\pgfqpoint{3.989959in}{1.549178in}}%
\pgfpathlineto{\pgfqpoint{3.975868in}{1.545179in}}%
\pgfpathlineto{\pgfqpoint{3.961788in}{1.541362in}}%
\pgfpathlineto{\pgfqpoint{3.953601in}{1.527584in}}%
\pgfpathlineto{\pgfqpoint{3.945410in}{1.513860in}}%
\pgfpathlineto{\pgfqpoint{3.937215in}{1.500194in}}%
\pgfpathlineto{\pgfqpoint{3.929014in}{1.486591in}}%
\pgfpathclose%
\pgfusepath{fill}%
\end{pgfscope}%
\begin{pgfscope}%
\pgfpathrectangle{\pgfqpoint{1.150000in}{0.150000in}}{\pgfqpoint{5.700000in}{5.700000in}}%
\pgfusepath{clip}%
\pgfsetbuttcap%
\pgfsetroundjoin%
\definecolor{currentfill}{rgb}{0.272594,0.025563,0.353093}%
\pgfsetfillcolor{currentfill}%
\pgfsetfillopacity{0.800000}%
\pgfsetlinewidth{0.000000pt}%
\definecolor{currentstroke}{rgb}{0.000000,0.000000,0.000000}%
\pgfsetstrokecolor{currentstroke}%
\pgfsetdash{}{0pt}%
\pgfpathmoveto{\pgfqpoint{3.190875in}{1.326760in}}%
\pgfpathlineto{\pgfqpoint{3.204883in}{1.318685in}}%
\pgfpathlineto{\pgfqpoint{3.218892in}{1.310806in}}%
\pgfpathlineto{\pgfqpoint{3.232902in}{1.303123in}}%
\pgfpathlineto{\pgfqpoint{3.246914in}{1.295633in}}%
\pgfpathlineto{\pgfqpoint{3.255487in}{1.298853in}}%
\pgfpathlineto{\pgfqpoint{3.264047in}{1.302381in}}%
\pgfpathlineto{\pgfqpoint{3.272594in}{1.306210in}}%
\pgfpathlineto{\pgfqpoint{3.281128in}{1.310331in}}%
\pgfpathlineto{\pgfqpoint{3.267149in}{1.317059in}}%
\pgfpathlineto{\pgfqpoint{3.253172in}{1.323982in}}%
\pgfpathlineto{\pgfqpoint{3.239196in}{1.331100in}}%
\pgfpathlineto{\pgfqpoint{3.225222in}{1.338413in}}%
\pgfpathlineto{\pgfqpoint{3.216656in}{1.335041in}}%
\pgfpathlineto{\pgfqpoint{3.208076in}{1.331969in}}%
\pgfpathlineto{\pgfqpoint{3.199483in}{1.329206in}}%
\pgfpathlineto{\pgfqpoint{3.190875in}{1.326760in}}%
\pgfpathclose%
\pgfusepath{fill}%
\end{pgfscope}%
\begin{pgfscope}%
\pgfpathrectangle{\pgfqpoint{1.150000in}{0.150000in}}{\pgfqpoint{5.700000in}{5.700000in}}%
\pgfusepath{clip}%
\pgfsetbuttcap%
\pgfsetroundjoin%
\definecolor{currentfill}{rgb}{0.248629,0.278775,0.534556}%
\pgfsetfillcolor{currentfill}%
\pgfsetfillopacity{0.800000}%
\pgfsetlinewidth{0.000000pt}%
\definecolor{currentstroke}{rgb}{0.000000,0.000000,0.000000}%
\pgfsetstrokecolor{currentstroke}%
\pgfsetdash{}{0pt}%
\pgfpathmoveto{\pgfqpoint{4.261820in}{1.845268in}}%
\pgfpathlineto{\pgfqpoint{4.276045in}{1.853117in}}%
\pgfpathlineto{\pgfqpoint{4.290283in}{1.861148in}}%
\pgfpathlineto{\pgfqpoint{4.304536in}{1.869360in}}%
\pgfpathlineto{\pgfqpoint{4.318803in}{1.877755in}}%
\pgfpathlineto{\pgfqpoint{4.326916in}{1.893336in}}%
\pgfpathlineto{\pgfqpoint{4.335026in}{1.908861in}}%
\pgfpathlineto{\pgfqpoint{4.343131in}{1.924327in}}%
\pgfpathlineto{\pgfqpoint{4.351232in}{1.939731in}}%
\pgfpathlineto{\pgfqpoint{4.336962in}{1.930949in}}%
\pgfpathlineto{\pgfqpoint{4.322706in}{1.922349in}}%
\pgfpathlineto{\pgfqpoint{4.308465in}{1.913931in}}%
\pgfpathlineto{\pgfqpoint{4.294238in}{1.905696in}}%
\pgfpathlineto{\pgfqpoint{4.286140in}{1.890667in}}%
\pgfpathlineto{\pgfqpoint{4.278038in}{1.875584in}}%
\pgfpathlineto{\pgfqpoint{4.269931in}{1.860449in}}%
\pgfpathlineto{\pgfqpoint{4.261820in}{1.845268in}}%
\pgfpathclose%
\pgfusepath{fill}%
\end{pgfscope}%
\begin{pgfscope}%
\pgfpathrectangle{\pgfqpoint{1.150000in}{0.150000in}}{\pgfqpoint{5.700000in}{5.700000in}}%
\pgfusepath{clip}%
\pgfsetbuttcap%
\pgfsetroundjoin%
\definecolor{currentfill}{rgb}{0.201239,0.383670,0.554294}%
\pgfsetfillcolor{currentfill}%
\pgfsetfillopacity{0.800000}%
\pgfsetlinewidth{0.000000pt}%
\definecolor{currentstroke}{rgb}{0.000000,0.000000,0.000000}%
\pgfsetstrokecolor{currentstroke}%
\pgfsetdash{}{0pt}%
\pgfpathmoveto{\pgfqpoint{2.307403in}{2.251762in}}%
\pgfpathlineto{\pgfqpoint{2.321782in}{2.228966in}}%
\pgfpathlineto{\pgfqpoint{2.336147in}{2.206452in}}%
\pgfpathlineto{\pgfqpoint{2.350501in}{2.184217in}}%
\pgfpathlineto{\pgfqpoint{2.364842in}{2.162260in}}%
\pgfpathlineto{\pgfqpoint{2.374278in}{2.151691in}}%
\pgfpathlineto{\pgfqpoint{2.383684in}{2.141605in}}%
\pgfpathlineto{\pgfqpoint{2.393063in}{2.131994in}}%
\pgfpathlineto{\pgfqpoint{2.402413in}{2.122849in}}%
\pgfpathlineto{\pgfqpoint{2.388143in}{2.143950in}}%
\pgfpathlineto{\pgfqpoint{2.373861in}{2.165327in}}%
\pgfpathlineto{\pgfqpoint{2.359568in}{2.186982in}}%
\pgfpathlineto{\pgfqpoint{2.345263in}{2.208917in}}%
\pgfpathlineto{\pgfqpoint{2.335842in}{2.218904in}}%
\pgfpathlineto{\pgfqpoint{2.326392in}{2.229369in}}%
\pgfpathlineto{\pgfqpoint{2.316913in}{2.240318in}}%
\pgfpathlineto{\pgfqpoint{2.307403in}{2.251762in}}%
\pgfpathclose%
\pgfusepath{fill}%
\end{pgfscope}%
\begin{pgfscope}%
\pgfpathrectangle{\pgfqpoint{1.150000in}{0.150000in}}{\pgfqpoint{5.700000in}{5.700000in}}%
\pgfusepath{clip}%
\pgfsetbuttcap%
\pgfsetroundjoin%
\definecolor{currentfill}{rgb}{0.140536,0.530132,0.555659}%
\pgfsetfillcolor{currentfill}%
\pgfsetfillopacity{0.800000}%
\pgfsetlinewidth{0.000000pt}%
\definecolor{currentstroke}{rgb}{0.000000,0.000000,0.000000}%
\pgfsetstrokecolor{currentstroke}%
\pgfsetdash{}{0pt}%
\pgfpathmoveto{\pgfqpoint{2.056156in}{2.710355in}}%
\pgfpathlineto{\pgfqpoint{2.070768in}{2.682268in}}%
\pgfpathlineto{\pgfqpoint{2.085362in}{2.654516in}}%
\pgfpathlineto{\pgfqpoint{2.099937in}{2.627097in}}%
\pgfpathlineto{\pgfqpoint{2.114494in}{2.600006in}}%
\pgfpathlineto{\pgfqpoint{2.124190in}{2.587078in}}%
\pgfpathlineto{\pgfqpoint{2.133853in}{2.574652in}}%
\pgfpathlineto{\pgfqpoint{2.143484in}{2.562719in}}%
\pgfpathlineto{\pgfqpoint{2.153084in}{2.551269in}}%
\pgfpathlineto{\pgfqpoint{2.138607in}{2.577505in}}%
\pgfpathlineto{\pgfqpoint{2.124112in}{2.604068in}}%
\pgfpathlineto{\pgfqpoint{2.109601in}{2.630960in}}%
\pgfpathlineto{\pgfqpoint{2.095071in}{2.658186in}}%
\pgfpathlineto{\pgfqpoint{2.085392in}{2.670477in}}%
\pgfpathlineto{\pgfqpoint{2.075680in}{2.683262in}}%
\pgfpathlineto{\pgfqpoint{2.065935in}{2.696552in}}%
\pgfpathlineto{\pgfqpoint{2.056156in}{2.710355in}}%
\pgfpathclose%
\pgfusepath{fill}%
\end{pgfscope}%
\begin{pgfscope}%
\pgfpathrectangle{\pgfqpoint{1.150000in}{0.150000in}}{\pgfqpoint{5.700000in}{5.700000in}}%
\pgfusepath{clip}%
\pgfsetbuttcap%
\pgfsetroundjoin%
\definecolor{currentfill}{rgb}{0.269308,0.218818,0.509577}%
\pgfsetfillcolor{currentfill}%
\pgfsetfillopacity{0.800000}%
\pgfsetlinewidth{0.000000pt}%
\definecolor{currentstroke}{rgb}{0.000000,0.000000,0.000000}%
\pgfsetstrokecolor{currentstroke}%
\pgfsetdash{}{0pt}%
\pgfpathmoveto{\pgfqpoint{4.140030in}{1.696446in}}%
\pgfpathlineto{\pgfqpoint{4.154200in}{1.702702in}}%
\pgfpathlineto{\pgfqpoint{4.168381in}{1.709138in}}%
\pgfpathlineto{\pgfqpoint{4.182576in}{1.715756in}}%
\pgfpathlineto{\pgfqpoint{4.196783in}{1.722554in}}%
\pgfpathlineto{\pgfqpoint{4.204927in}{1.737979in}}%
\pgfpathlineto{\pgfqpoint{4.213067in}{1.753388in}}%
\pgfpathlineto{\pgfqpoint{4.221203in}{1.768776in}}%
\pgfpathlineto{\pgfqpoint{4.229335in}{1.784140in}}%
\pgfpathlineto{\pgfqpoint{4.215126in}{1.776891in}}%
\pgfpathlineto{\pgfqpoint{4.200930in}{1.769823in}}%
\pgfpathlineto{\pgfqpoint{4.186747in}{1.762937in}}%
\pgfpathlineto{\pgfqpoint{4.172577in}{1.756233in}}%
\pgfpathlineto{\pgfqpoint{4.164447in}{1.741307in}}%
\pgfpathlineto{\pgfqpoint{4.156312in}{1.726365in}}%
\pgfpathlineto{\pgfqpoint{4.148173in}{1.711410in}}%
\pgfpathlineto{\pgfqpoint{4.140030in}{1.696446in}}%
\pgfpathclose%
\pgfusepath{fill}%
\end{pgfscope}%
\begin{pgfscope}%
\pgfpathrectangle{\pgfqpoint{1.150000in}{0.150000in}}{\pgfqpoint{5.700000in}{5.700000in}}%
\pgfusepath{clip}%
\pgfsetbuttcap%
\pgfsetroundjoin%
\definecolor{currentfill}{rgb}{0.421908,0.805774,0.351910}%
\pgfsetfillcolor{currentfill}%
\pgfsetfillopacity{0.800000}%
\pgfsetlinewidth{0.000000pt}%
\definecolor{currentstroke}{rgb}{0.000000,0.000000,0.000000}%
\pgfsetstrokecolor{currentstroke}%
\pgfsetdash{}{0pt}%
\pgfpathmoveto{\pgfqpoint{5.715334in}{3.503975in}}%
\pgfpathlineto{\pgfqpoint{5.730533in}{3.522208in}}%
\pgfpathlineto{\pgfqpoint{5.745757in}{3.540631in}}%
\pgfpathlineto{\pgfqpoint{5.761006in}{3.559244in}}%
\pgfpathlineto{\pgfqpoint{5.776281in}{3.578048in}}%
\pgfpathlineto{\pgfqpoint{5.783628in}{3.580012in}}%
\pgfpathlineto{\pgfqpoint{5.790964in}{3.581828in}}%
\pgfpathlineto{\pgfqpoint{5.798288in}{3.583501in}}%
\pgfpathlineto{\pgfqpoint{5.805601in}{3.585036in}}%
\pgfpathlineto{\pgfqpoint{5.790349in}{3.566617in}}%
\pgfpathlineto{\pgfqpoint{5.775122in}{3.548389in}}%
\pgfpathlineto{\pgfqpoint{5.759919in}{3.530349in}}%
\pgfpathlineto{\pgfqpoint{5.744741in}{3.512499in}}%
\pgfpathlineto{\pgfqpoint{5.737406in}{3.510568in}}%
\pgfpathlineto{\pgfqpoint{5.730060in}{3.508506in}}%
\pgfpathlineto{\pgfqpoint{5.722702in}{3.506310in}}%
\pgfpathlineto{\pgfqpoint{5.715334in}{3.503975in}}%
\pgfpathclose%
\pgfusepath{fill}%
\end{pgfscope}%
\begin{pgfscope}%
\pgfpathrectangle{\pgfqpoint{1.150000in}{0.150000in}}{\pgfqpoint{5.700000in}{5.700000in}}%
\pgfusepath{clip}%
\pgfsetbuttcap%
\pgfsetroundjoin%
\definecolor{currentfill}{rgb}{0.280894,0.078907,0.402329}%
\pgfsetfillcolor{currentfill}%
\pgfsetfillopacity{0.800000}%
\pgfsetlinewidth{0.000000pt}%
\definecolor{currentstroke}{rgb}{0.000000,0.000000,0.000000}%
\pgfsetstrokecolor{currentstroke}%
\pgfsetdash{}{0pt}%
\pgfpathmoveto{\pgfqpoint{2.987745in}{1.442358in}}%
\pgfpathlineto{\pgfqpoint{3.001796in}{1.431067in}}%
\pgfpathlineto{\pgfqpoint{3.015846in}{1.419983in}}%
\pgfpathlineto{\pgfqpoint{3.029894in}{1.409104in}}%
\pgfpathlineto{\pgfqpoint{3.043941in}{1.398431in}}%
\pgfpathlineto{\pgfqpoint{3.052690in}{1.397907in}}%
\pgfpathlineto{\pgfqpoint{3.061421in}{1.397750in}}%
\pgfpathlineto{\pgfqpoint{3.070136in}{1.397951in}}%
\pgfpathlineto{\pgfqpoint{3.078834in}{1.398503in}}%
\pgfpathlineto{\pgfqpoint{3.064829in}{1.408376in}}%
\pgfpathlineto{\pgfqpoint{3.050823in}{1.418454in}}%
\pgfpathlineto{\pgfqpoint{3.036817in}{1.428736in}}%
\pgfpathlineto{\pgfqpoint{3.022809in}{1.439225in}}%
\pgfpathlineto{\pgfqpoint{3.014069in}{1.439461in}}%
\pgfpathlineto{\pgfqpoint{3.005312in}{1.440057in}}%
\pgfpathlineto{\pgfqpoint{2.996537in}{1.441020in}}%
\pgfpathlineto{\pgfqpoint{2.987745in}{1.442358in}}%
\pgfpathclose%
\pgfusepath{fill}%
\end{pgfscope}%
\begin{pgfscope}%
\pgfpathrectangle{\pgfqpoint{1.150000in}{0.150000in}}{\pgfqpoint{5.700000in}{5.700000in}}%
\pgfusepath{clip}%
\pgfsetbuttcap%
\pgfsetroundjoin%
\definecolor{currentfill}{rgb}{0.120638,0.625828,0.533488}%
\pgfsetfillcolor{currentfill}%
\pgfsetfillopacity{0.800000}%
\pgfsetlinewidth{0.000000pt}%
\definecolor{currentstroke}{rgb}{0.000000,0.000000,0.000000}%
\pgfsetstrokecolor{currentstroke}%
\pgfsetdash{}{0pt}%
\pgfpathmoveto{\pgfqpoint{5.082129in}{2.874878in}}%
\pgfpathlineto{\pgfqpoint{5.096876in}{2.890361in}}%
\pgfpathlineto{\pgfqpoint{5.111644in}{2.906032in}}%
\pgfpathlineto{\pgfqpoint{5.126432in}{2.921892in}}%
\pgfpathlineto{\pgfqpoint{5.141243in}{2.937941in}}%
\pgfpathlineto{\pgfqpoint{5.149047in}{2.947564in}}%
\pgfpathlineto{\pgfqpoint{5.156843in}{2.957000in}}%
\pgfpathlineto{\pgfqpoint{5.164629in}{2.966250in}}%
\pgfpathlineto{\pgfqpoint{5.172406in}{2.975315in}}%
\pgfpathlineto{\pgfqpoint{5.157599in}{2.959320in}}%
\pgfpathlineto{\pgfqpoint{5.142814in}{2.943513in}}%
\pgfpathlineto{\pgfqpoint{5.128049in}{2.927895in}}%
\pgfpathlineto{\pgfqpoint{5.113306in}{2.912465in}}%
\pgfpathlineto{\pgfqpoint{5.105525in}{2.903333in}}%
\pgfpathlineto{\pgfqpoint{5.097735in}{2.894025in}}%
\pgfpathlineto{\pgfqpoint{5.089936in}{2.884540in}}%
\pgfpathlineto{\pgfqpoint{5.082129in}{2.874878in}}%
\pgfpathclose%
\pgfusepath{fill}%
\end{pgfscope}%
\begin{pgfscope}%
\pgfpathrectangle{\pgfqpoint{1.150000in}{0.150000in}}{\pgfqpoint{5.700000in}{5.700000in}}%
\pgfusepath{clip}%
\pgfsetbuttcap%
\pgfsetroundjoin%
\definecolor{currentfill}{rgb}{0.185556,0.418570,0.556753}%
\pgfsetfillcolor{currentfill}%
\pgfsetfillopacity{0.800000}%
\pgfsetlinewidth{0.000000pt}%
\definecolor{currentstroke}{rgb}{0.000000,0.000000,0.000000}%
\pgfsetstrokecolor{currentstroke}%
\pgfsetdash{}{0pt}%
\pgfpathmoveto{\pgfqpoint{2.249757in}{2.345814in}}%
\pgfpathlineto{\pgfqpoint{2.264189in}{2.321866in}}%
\pgfpathlineto{\pgfqpoint{2.278607in}{2.298210in}}%
\pgfpathlineto{\pgfqpoint{2.293012in}{2.274842in}}%
\pgfpathlineto{\pgfqpoint{2.307403in}{2.251762in}}%
\pgfpathlineto{\pgfqpoint{2.316913in}{2.240318in}}%
\pgfpathlineto{\pgfqpoint{2.326392in}{2.229369in}}%
\pgfpathlineto{\pgfqpoint{2.335842in}{2.218904in}}%
\pgfpathlineto{\pgfqpoint{2.345263in}{2.208917in}}%
\pgfpathlineto{\pgfqpoint{2.330945in}{2.231134in}}%
\pgfpathlineto{\pgfqpoint{2.316615in}{2.253636in}}%
\pgfpathlineto{\pgfqpoint{2.302272in}{2.276425in}}%
\pgfpathlineto{\pgfqpoint{2.287915in}{2.299504in}}%
\pgfpathlineto{\pgfqpoint{2.278422in}{2.310342in}}%
\pgfpathlineto{\pgfqpoint{2.268898in}{2.321667in}}%
\pgfpathlineto{\pgfqpoint{2.259343in}{2.333488in}}%
\pgfpathlineto{\pgfqpoint{2.249757in}{2.345814in}}%
\pgfpathclose%
\pgfusepath{fill}%
\end{pgfscope}%
\begin{pgfscope}%
\pgfpathrectangle{\pgfqpoint{1.150000in}{0.150000in}}{\pgfqpoint{5.700000in}{5.700000in}}%
\pgfusepath{clip}%
\pgfsetbuttcap%
\pgfsetroundjoin%
\definecolor{currentfill}{rgb}{0.280868,0.160771,0.472899}%
\pgfsetfillcolor{currentfill}%
\pgfsetfillopacity{0.800000}%
\pgfsetlinewidth{0.000000pt}%
\definecolor{currentstroke}{rgb}{0.000000,0.000000,0.000000}%
\pgfsetstrokecolor{currentstroke}%
\pgfsetdash{}{0pt}%
\pgfpathmoveto{\pgfqpoint{4.018173in}{1.557717in}}%
\pgfpathlineto{\pgfqpoint{4.032297in}{1.562258in}}%
\pgfpathlineto{\pgfqpoint{4.046432in}{1.566979in}}%
\pgfpathlineto{\pgfqpoint{4.060578in}{1.571881in}}%
\pgfpathlineto{\pgfqpoint{4.074736in}{1.576963in}}%
\pgfpathlineto{\pgfqpoint{4.082913in}{1.591834in}}%
\pgfpathlineto{\pgfqpoint{4.091085in}{1.606733in}}%
\pgfpathlineto{\pgfqpoint{4.099253in}{1.621655in}}%
\pgfpathlineto{\pgfqpoint{4.107417in}{1.636595in}}%
\pgfpathlineto{\pgfqpoint{4.093260in}{1.631001in}}%
\pgfpathlineto{\pgfqpoint{4.079115in}{1.625588in}}%
\pgfpathlineto{\pgfqpoint{4.064981in}{1.620355in}}%
\pgfpathlineto{\pgfqpoint{4.050860in}{1.615304in}}%
\pgfpathlineto{\pgfqpoint{4.042695in}{1.600863in}}%
\pgfpathlineto{\pgfqpoint{4.034525in}{1.586449in}}%
\pgfpathlineto{\pgfqpoint{4.026351in}{1.572065in}}%
\pgfpathlineto{\pgfqpoint{4.018173in}{1.557717in}}%
\pgfpathclose%
\pgfusepath{fill}%
\end{pgfscope}%
\begin{pgfscope}%
\pgfpathrectangle{\pgfqpoint{1.150000in}{0.150000in}}{\pgfqpoint{5.700000in}{5.700000in}}%
\pgfusepath{clip}%
\pgfsetbuttcap%
\pgfsetroundjoin%
\definecolor{currentfill}{rgb}{0.477504,0.821444,0.318195}%
\pgfsetfillcolor{currentfill}%
\pgfsetfillopacity{0.800000}%
\pgfsetlinewidth{0.000000pt}%
\definecolor{currentstroke}{rgb}{0.000000,0.000000,0.000000}%
\pgfsetstrokecolor{currentstroke}%
\pgfsetdash{}{0pt}%
\pgfpathmoveto{\pgfqpoint{5.805601in}{3.585036in}}%
\pgfpathlineto{\pgfqpoint{5.820878in}{3.603644in}}%
\pgfpathlineto{\pgfqpoint{5.836180in}{3.622443in}}%
\pgfpathlineto{\pgfqpoint{5.851508in}{3.641433in}}%
\pgfpathlineto{\pgfqpoint{5.858791in}{3.642526in}}%
\pgfpathlineto{\pgfqpoint{5.866062in}{3.643483in}}%
\pgfpathlineto{\pgfqpoint{5.873321in}{3.644309in}}%
\pgfpathlineto{\pgfqpoint{5.880569in}{3.645007in}}%
\pgfpathlineto{\pgfqpoint{5.865266in}{3.626440in}}%
\pgfpathlineto{\pgfqpoint{5.849989in}{3.608063in}}%
\pgfpathlineto{\pgfqpoint{5.834736in}{3.589875in}}%
\pgfpathlineto{\pgfqpoint{5.827469in}{3.588851in}}%
\pgfpathlineto{\pgfqpoint{5.820191in}{3.587706in}}%
\pgfpathlineto{\pgfqpoint{5.812902in}{3.586436in}}%
\pgfpathlineto{\pgfqpoint{5.805601in}{3.585036in}}%
\pgfpathclose%
\pgfusepath{fill}%
\end{pgfscope}%
\begin{pgfscope}%
\pgfpathrectangle{\pgfqpoint{1.150000in}{0.150000in}}{\pgfqpoint{5.700000in}{5.700000in}}%
\pgfusepath{clip}%
\pgfsetbuttcap%
\pgfsetroundjoin%
\definecolor{currentfill}{rgb}{0.268510,0.009605,0.335427}%
\pgfsetfillcolor{currentfill}%
\pgfsetfillopacity{0.800000}%
\pgfsetlinewidth{0.000000pt}%
\definecolor{currentstroke}{rgb}{0.000000,0.000000,0.000000}%
\pgfsetstrokecolor{currentstroke}%
\pgfsetdash{}{0pt}%
\pgfpathmoveto{\pgfqpoint{3.393058in}{1.263409in}}%
\pgfpathlineto{\pgfqpoint{3.407063in}{1.258398in}}%
\pgfpathlineto{\pgfqpoint{3.421072in}{1.253575in}}%
\pgfpathlineto{\pgfqpoint{3.435086in}{1.248939in}}%
\pgfpathlineto{\pgfqpoint{3.449103in}{1.244490in}}%
\pgfpathlineto{\pgfqpoint{3.457542in}{1.251112in}}%
\pgfpathlineto{\pgfqpoint{3.465972in}{1.257986in}}%
\pgfpathlineto{\pgfqpoint{3.474391in}{1.265105in}}%
\pgfpathlineto{\pgfqpoint{3.482800in}{1.272463in}}%
\pgfpathlineto{\pgfqpoint{3.468807in}{1.276186in}}%
\pgfpathlineto{\pgfqpoint{3.454818in}{1.280096in}}%
\pgfpathlineto{\pgfqpoint{3.440834in}{1.284193in}}%
\pgfpathlineto{\pgfqpoint{3.426853in}{1.288478in}}%
\pgfpathlineto{\pgfqpoint{3.418420in}{1.281834in}}%
\pgfpathlineto{\pgfqpoint{3.409977in}{1.275437in}}%
\pgfpathlineto{\pgfqpoint{3.401523in}{1.269292in}}%
\pgfpathlineto{\pgfqpoint{3.393058in}{1.263409in}}%
\pgfpathclose%
\pgfusepath{fill}%
\end{pgfscope}%
\begin{pgfscope}%
\pgfpathrectangle{\pgfqpoint{1.150000in}{0.150000in}}{\pgfqpoint{5.700000in}{5.700000in}}%
\pgfusepath{clip}%
\pgfsetbuttcap%
\pgfsetroundjoin%
\definecolor{currentfill}{rgb}{0.277018,0.050344,0.375715}%
\pgfsetfillcolor{currentfill}%
\pgfsetfillopacity{0.800000}%
\pgfsetlinewidth{0.000000pt}%
\definecolor{currentstroke}{rgb}{0.000000,0.000000,0.000000}%
\pgfsetstrokecolor{currentstroke}%
\pgfsetdash{}{0pt}%
\pgfpathmoveto{\pgfqpoint{3.717671in}{1.325059in}}%
\pgfpathlineto{\pgfqpoint{3.731710in}{1.325082in}}%
\pgfpathlineto{\pgfqpoint{3.745756in}{1.325286in}}%
\pgfpathlineto{\pgfqpoint{3.759811in}{1.325672in}}%
\pgfpathlineto{\pgfqpoint{3.773873in}{1.326238in}}%
\pgfpathlineto{\pgfqpoint{3.782147in}{1.338001in}}%
\pgfpathlineto{\pgfqpoint{3.790414in}{1.349903in}}%
\pgfpathlineto{\pgfqpoint{3.798676in}{1.361939in}}%
\pgfpathlineto{\pgfqpoint{3.806931in}{1.374102in}}%
\pgfpathlineto{\pgfqpoint{3.792879in}{1.372903in}}%
\pgfpathlineto{\pgfqpoint{3.778836in}{1.371885in}}%
\pgfpathlineto{\pgfqpoint{3.764800in}{1.371048in}}%
\pgfpathlineto{\pgfqpoint{3.750773in}{1.370393in}}%
\pgfpathlineto{\pgfqpoint{3.742507in}{1.358850in}}%
\pgfpathlineto{\pgfqpoint{3.734235in}{1.347443in}}%
\pgfpathlineto{\pgfqpoint{3.725956in}{1.336177in}}%
\pgfpathlineto{\pgfqpoint{3.717671in}{1.325059in}}%
\pgfpathclose%
\pgfusepath{fill}%
\end{pgfscope}%
\begin{pgfscope}%
\pgfpathrectangle{\pgfqpoint{1.150000in}{0.150000in}}{\pgfqpoint{5.700000in}{5.700000in}}%
\pgfusepath{clip}%
\pgfsetbuttcap%
\pgfsetroundjoin%
\definecolor{currentfill}{rgb}{0.271305,0.019942,0.347269}%
\pgfsetfillcolor{currentfill}%
\pgfsetfillopacity{0.800000}%
\pgfsetlinewidth{0.000000pt}%
\definecolor{currentstroke}{rgb}{0.000000,0.000000,0.000000}%
\pgfsetstrokecolor{currentstroke}%
\pgfsetdash{}{0pt}%
\pgfpathmoveto{\pgfqpoint{3.246914in}{1.295633in}}%
\pgfpathlineto{\pgfqpoint{3.260928in}{1.288337in}}%
\pgfpathlineto{\pgfqpoint{3.274943in}{1.281234in}}%
\pgfpathlineto{\pgfqpoint{3.288961in}{1.274324in}}%
\pgfpathlineto{\pgfqpoint{3.302981in}{1.267605in}}%
\pgfpathlineto{\pgfqpoint{3.311521in}{1.271598in}}%
\pgfpathlineto{\pgfqpoint{3.320050in}{1.275890in}}%
\pgfpathlineto{\pgfqpoint{3.328566in}{1.280475in}}%
\pgfpathlineto{\pgfqpoint{3.337070in}{1.285343in}}%
\pgfpathlineto{\pgfqpoint{3.323080in}{1.291302in}}%
\pgfpathlineto{\pgfqpoint{3.309094in}{1.297453in}}%
\pgfpathlineto{\pgfqpoint{3.295110in}{1.303795in}}%
\pgfpathlineto{\pgfqpoint{3.281128in}{1.310331in}}%
\pgfpathlineto{\pgfqpoint{3.272594in}{1.306210in}}%
\pgfpathlineto{\pgfqpoint{3.264047in}{1.302381in}}%
\pgfpathlineto{\pgfqpoint{3.255487in}{1.298853in}}%
\pgfpathlineto{\pgfqpoint{3.246914in}{1.295633in}}%
\pgfpathclose%
\pgfusepath{fill}%
\end{pgfscope}%
\begin{pgfscope}%
\pgfpathrectangle{\pgfqpoint{1.150000in}{0.150000in}}{\pgfqpoint{5.700000in}{5.700000in}}%
\pgfusepath{clip}%
\pgfsetbuttcap%
\pgfsetroundjoin%
\definecolor{currentfill}{rgb}{0.123463,0.581687,0.547445}%
\pgfsetfillcolor{currentfill}%
\pgfsetfillopacity{0.800000}%
\pgfsetlinewidth{0.000000pt}%
\definecolor{currentstroke}{rgb}{0.000000,0.000000,0.000000}%
\pgfsetstrokecolor{currentstroke}%
\pgfsetdash{}{0pt}%
\pgfpathmoveto{\pgfqpoint{4.960570in}{2.731031in}}%
\pgfpathlineto{\pgfqpoint{4.975238in}{2.745763in}}%
\pgfpathlineto{\pgfqpoint{4.989925in}{2.760682in}}%
\pgfpathlineto{\pgfqpoint{5.004633in}{2.775788in}}%
\pgfpathlineto{\pgfqpoint{5.019362in}{2.791083in}}%
\pgfpathlineto{\pgfqpoint{5.027237in}{2.802193in}}%
\pgfpathlineto{\pgfqpoint{5.035104in}{2.813121in}}%
\pgfpathlineto{\pgfqpoint{5.042963in}{2.823867in}}%
\pgfpathlineto{\pgfqpoint{5.050813in}{2.834431in}}%
\pgfpathlineto{\pgfqpoint{5.036085in}{2.819118in}}%
\pgfpathlineto{\pgfqpoint{5.021378in}{2.803993in}}%
\pgfpathlineto{\pgfqpoint{5.006691in}{2.789056in}}%
\pgfpathlineto{\pgfqpoint{4.992024in}{2.774306in}}%
\pgfpathlineto{\pgfqpoint{4.984173in}{2.763747in}}%
\pgfpathlineto{\pgfqpoint{4.976313in}{2.753015in}}%
\pgfpathlineto{\pgfqpoint{4.968446in}{2.742110in}}%
\pgfpathlineto{\pgfqpoint{4.960570in}{2.731031in}}%
\pgfpathclose%
\pgfusepath{fill}%
\end{pgfscope}%
\begin{pgfscope}%
\pgfpathrectangle{\pgfqpoint{1.150000in}{0.150000in}}{\pgfqpoint{5.700000in}{5.700000in}}%
\pgfusepath{clip}%
\pgfsetbuttcap%
\pgfsetroundjoin%
\definecolor{currentfill}{rgb}{0.296479,0.761561,0.424223}%
\pgfsetfillcolor{currentfill}%
\pgfsetfillopacity{0.800000}%
\pgfsetlinewidth{0.000000pt}%
\definecolor{currentstroke}{rgb}{0.000000,0.000000,0.000000}%
\pgfsetstrokecolor{currentstroke}%
\pgfsetdash{}{0pt}%
\pgfpathmoveto{\pgfqpoint{5.504869in}{3.317225in}}%
\pgfpathlineto{\pgfqpoint{5.519927in}{3.334868in}}%
\pgfpathlineto{\pgfqpoint{5.535008in}{3.352701in}}%
\pgfpathlineto{\pgfqpoint{5.550113in}{3.370725in}}%
\pgfpathlineto{\pgfqpoint{5.565242in}{3.388939in}}%
\pgfpathlineto{\pgfqpoint{5.572766in}{3.393501in}}%
\pgfpathlineto{\pgfqpoint{5.580278in}{3.397888in}}%
\pgfpathlineto{\pgfqpoint{5.587778in}{3.402105in}}%
\pgfpathlineto{\pgfqpoint{5.595268in}{3.406155in}}%
\pgfpathlineto{\pgfqpoint{5.580154in}{3.388215in}}%
\pgfpathlineto{\pgfqpoint{5.565063in}{3.370465in}}%
\pgfpathlineto{\pgfqpoint{5.549996in}{3.352905in}}%
\pgfpathlineto{\pgfqpoint{5.534953in}{3.335533in}}%
\pgfpathlineto{\pgfqpoint{5.527449in}{3.331198in}}%
\pgfpathlineto{\pgfqpoint{5.519933in}{3.326703in}}%
\pgfpathlineto{\pgfqpoint{5.512406in}{3.322047in}}%
\pgfpathlineto{\pgfqpoint{5.504869in}{3.317225in}}%
\pgfpathclose%
\pgfusepath{fill}%
\end{pgfscope}%
\begin{pgfscope}%
\pgfpathrectangle{\pgfqpoint{1.150000in}{0.150000in}}{\pgfqpoint{5.700000in}{5.700000in}}%
\pgfusepath{clip}%
\pgfsetbuttcap%
\pgfsetroundjoin%
\definecolor{currentfill}{rgb}{0.180653,0.701402,0.488189}%
\pgfsetfillcolor{currentfill}%
\pgfsetfillopacity{0.800000}%
\pgfsetlinewidth{0.000000pt}%
\definecolor{currentstroke}{rgb}{0.000000,0.000000,0.000000}%
\pgfsetstrokecolor{currentstroke}%
\pgfsetdash{}{0pt}%
\pgfpathmoveto{\pgfqpoint{5.293686in}{3.106348in}}%
\pgfpathlineto{\pgfqpoint{5.308591in}{3.123072in}}%
\pgfpathlineto{\pgfqpoint{5.323518in}{3.139986in}}%
\pgfpathlineto{\pgfqpoint{5.338468in}{3.157089in}}%
\pgfpathlineto{\pgfqpoint{5.353441in}{3.174382in}}%
\pgfpathlineto{\pgfqpoint{5.361118in}{3.181549in}}%
\pgfpathlineto{\pgfqpoint{5.368784in}{3.188530in}}%
\pgfpathlineto{\pgfqpoint{5.376440in}{3.195326in}}%
\pgfpathlineto{\pgfqpoint{5.384086in}{3.201939in}}%
\pgfpathlineto{\pgfqpoint{5.369121in}{3.184809in}}%
\pgfpathlineto{\pgfqpoint{5.354180in}{3.167868in}}%
\pgfpathlineto{\pgfqpoint{5.339261in}{3.151117in}}%
\pgfpathlineto{\pgfqpoint{5.324364in}{3.134554in}}%
\pgfpathlineto{\pgfqpoint{5.316709in}{3.127766in}}%
\pgfpathlineto{\pgfqpoint{5.309045in}{3.120804in}}%
\pgfpathlineto{\pgfqpoint{5.301370in}{3.113665in}}%
\pgfpathlineto{\pgfqpoint{5.293686in}{3.106348in}}%
\pgfpathclose%
\pgfusepath{fill}%
\end{pgfscope}%
\begin{pgfscope}%
\pgfpathrectangle{\pgfqpoint{1.150000in}{0.150000in}}{\pgfqpoint{5.700000in}{5.700000in}}%
\pgfusepath{clip}%
\pgfsetbuttcap%
\pgfsetroundjoin%
\definecolor{currentfill}{rgb}{0.272594,0.025563,0.353093}%
\pgfsetfillcolor{currentfill}%
\pgfsetfillopacity{0.800000}%
\pgfsetlinewidth{0.000000pt}%
\definecolor{currentstroke}{rgb}{0.000000,0.000000,0.000000}%
\pgfsetstrokecolor{currentstroke}%
\pgfsetdash{}{0pt}%
\pgfpathmoveto{\pgfqpoint{3.628323in}{1.286566in}}%
\pgfpathlineto{\pgfqpoint{3.642347in}{1.285197in}}%
\pgfpathlineto{\pgfqpoint{3.656378in}{1.284011in}}%
\pgfpathlineto{\pgfqpoint{3.670417in}{1.283007in}}%
\pgfpathlineto{\pgfqpoint{3.684462in}{1.282185in}}%
\pgfpathlineto{\pgfqpoint{3.692774in}{1.292651in}}%
\pgfpathlineto{\pgfqpoint{3.701080in}{1.303289in}}%
\pgfpathlineto{\pgfqpoint{3.709379in}{1.314094in}}%
\pgfpathlineto{\pgfqpoint{3.717671in}{1.325059in}}%
\pgfpathlineto{\pgfqpoint{3.703640in}{1.325217in}}%
\pgfpathlineto{\pgfqpoint{3.689616in}{1.325558in}}%
\pgfpathlineto{\pgfqpoint{3.675599in}{1.326081in}}%
\pgfpathlineto{\pgfqpoint{3.661590in}{1.326787in}}%
\pgfpathlineto{\pgfqpoint{3.653284in}{1.316474in}}%
\pgfpathlineto{\pgfqpoint{3.644971in}{1.306328in}}%
\pgfpathlineto{\pgfqpoint{3.636651in}{1.296356in}}%
\pgfpathlineto{\pgfqpoint{3.628323in}{1.286566in}}%
\pgfpathclose%
\pgfusepath{fill}%
\end{pgfscope}%
\begin{pgfscope}%
\pgfpathrectangle{\pgfqpoint{1.150000in}{0.150000in}}{\pgfqpoint{5.700000in}{5.700000in}}%
\pgfusepath{clip}%
\pgfsetbuttcap%
\pgfsetroundjoin%
\definecolor{currentfill}{rgb}{0.280894,0.078907,0.402329}%
\pgfsetfillcolor{currentfill}%
\pgfsetfillopacity{0.800000}%
\pgfsetlinewidth{0.000000pt}%
\definecolor{currentstroke}{rgb}{0.000000,0.000000,0.000000}%
\pgfsetstrokecolor{currentstroke}%
\pgfsetdash{}{0pt}%
\pgfpathmoveto{\pgfqpoint{3.806931in}{1.374102in}}%
\pgfpathlineto{\pgfqpoint{3.820992in}{1.375482in}}%
\pgfpathlineto{\pgfqpoint{3.835062in}{1.377043in}}%
\pgfpathlineto{\pgfqpoint{3.849140in}{1.378785in}}%
\pgfpathlineto{\pgfqpoint{3.863228in}{1.380706in}}%
\pgfpathlineto{\pgfqpoint{3.871470in}{1.393605in}}%
\pgfpathlineto{\pgfqpoint{3.879706in}{1.406611in}}%
\pgfpathlineto{\pgfqpoint{3.887936in}{1.419720in}}%
\pgfpathlineto{\pgfqpoint{3.896162in}{1.432924in}}%
\pgfpathlineto{\pgfqpoint{3.882082in}{1.430399in}}%
\pgfpathlineto{\pgfqpoint{3.868011in}{1.428055in}}%
\pgfpathlineto{\pgfqpoint{3.853949in}{1.425892in}}%
\pgfpathlineto{\pgfqpoint{3.839897in}{1.423910in}}%
\pgfpathlineto{\pgfqpoint{3.831664in}{1.411296in}}%
\pgfpathlineto{\pgfqpoint{3.823425in}{1.398787in}}%
\pgfpathlineto{\pgfqpoint{3.815181in}{1.386387in}}%
\pgfpathlineto{\pgfqpoint{3.806931in}{1.374102in}}%
\pgfpathclose%
\pgfusepath{fill}%
\end{pgfscope}%
\begin{pgfscope}%
\pgfpathrectangle{\pgfqpoint{1.150000in}{0.150000in}}{\pgfqpoint{5.700000in}{5.700000in}}%
\pgfusepath{clip}%
\pgfsetbuttcap%
\pgfsetroundjoin%
\definecolor{currentfill}{rgb}{0.279566,0.067836,0.391917}%
\pgfsetfillcolor{currentfill}%
\pgfsetfillopacity{0.800000}%
\pgfsetlinewidth{0.000000pt}%
\definecolor{currentstroke}{rgb}{0.000000,0.000000,0.000000}%
\pgfsetstrokecolor{currentstroke}%
\pgfsetdash{}{0pt}%
\pgfpathmoveto{\pgfqpoint{3.043941in}{1.398431in}}%
\pgfpathlineto{\pgfqpoint{3.057987in}{1.387960in}}%
\pgfpathlineto{\pgfqpoint{3.072033in}{1.377693in}}%
\pgfpathlineto{\pgfqpoint{3.086078in}{1.367628in}}%
\pgfpathlineto{\pgfqpoint{3.100122in}{1.357764in}}%
\pgfpathlineto{\pgfqpoint{3.108829in}{1.358053in}}%
\pgfpathlineto{\pgfqpoint{3.117519in}{1.358700in}}%
\pgfpathlineto{\pgfqpoint{3.126193in}{1.359696in}}%
\pgfpathlineto{\pgfqpoint{3.134852in}{1.361033in}}%
\pgfpathlineto{\pgfqpoint{3.120848in}{1.370099in}}%
\pgfpathlineto{\pgfqpoint{3.106843in}{1.379365in}}%
\pgfpathlineto{\pgfqpoint{3.092839in}{1.388833in}}%
\pgfpathlineto{\pgfqpoint{3.078834in}{1.398503in}}%
\pgfpathlineto{\pgfqpoint{3.070136in}{1.397951in}}%
\pgfpathlineto{\pgfqpoint{3.061421in}{1.397750in}}%
\pgfpathlineto{\pgfqpoint{3.052690in}{1.397907in}}%
\pgfpathlineto{\pgfqpoint{3.043941in}{1.398431in}}%
\pgfpathclose%
\pgfusepath{fill}%
\end{pgfscope}%
\begin{pgfscope}%
\pgfpathrectangle{\pgfqpoint{1.150000in}{0.150000in}}{\pgfqpoint{5.700000in}{5.700000in}}%
\pgfusepath{clip}%
\pgfsetbuttcap%
\pgfsetroundjoin%
\definecolor{currentfill}{rgb}{0.137770,0.537492,0.554906}%
\pgfsetfillcolor{currentfill}%
\pgfsetfillopacity{0.800000}%
\pgfsetlinewidth{0.000000pt}%
\definecolor{currentstroke}{rgb}{0.000000,0.000000,0.000000}%
\pgfsetstrokecolor{currentstroke}%
\pgfsetdash{}{0pt}%
\pgfpathmoveto{\pgfqpoint{4.838827in}{2.579662in}}%
\pgfpathlineto{\pgfqpoint{4.853414in}{2.593504in}}%
\pgfpathlineto{\pgfqpoint{4.868020in}{2.607532in}}%
\pgfpathlineto{\pgfqpoint{4.882645in}{2.621748in}}%
\pgfpathlineto{\pgfqpoint{4.897290in}{2.636150in}}%
\pgfpathlineto{\pgfqpoint{4.905227in}{2.648616in}}%
\pgfpathlineto{\pgfqpoint{4.913156in}{2.660909in}}%
\pgfpathlineto{\pgfqpoint{4.921077in}{2.673030in}}%
\pgfpathlineto{\pgfqpoint{4.928991in}{2.684977in}}%
\pgfpathlineto{\pgfqpoint{4.914344in}{2.670486in}}%
\pgfpathlineto{\pgfqpoint{4.899717in}{2.656182in}}%
\pgfpathlineto{\pgfqpoint{4.885109in}{2.642065in}}%
\pgfpathlineto{\pgfqpoint{4.870521in}{2.628135in}}%
\pgfpathlineto{\pgfqpoint{4.862608in}{2.616263in}}%
\pgfpathlineto{\pgfqpoint{4.854688in}{2.604226in}}%
\pgfpathlineto{\pgfqpoint{4.846761in}{2.592026in}}%
\pgfpathlineto{\pgfqpoint{4.838827in}{2.579662in}}%
\pgfpathclose%
\pgfusepath{fill}%
\end{pgfscope}%
\begin{pgfscope}%
\pgfpathrectangle{\pgfqpoint{1.150000in}{0.150000in}}{\pgfqpoint{5.700000in}{5.700000in}}%
\pgfusepath{clip}%
\pgfsetbuttcap%
\pgfsetroundjoin%
\definecolor{currentfill}{rgb}{0.229739,0.322361,0.545706}%
\pgfsetfillcolor{currentfill}%
\pgfsetfillopacity{0.800000}%
\pgfsetlinewidth{0.000000pt}%
\definecolor{currentstroke}{rgb}{0.000000,0.000000,0.000000}%
\pgfsetstrokecolor{currentstroke}%
\pgfsetdash{}{0pt}%
\pgfpathmoveto{\pgfqpoint{4.351232in}{1.939731in}}%
\pgfpathlineto{\pgfqpoint{4.365517in}{1.948696in}}%
\pgfpathlineto{\pgfqpoint{4.379816in}{1.957843in}}%
\pgfpathlineto{\pgfqpoint{4.394131in}{1.967172in}}%
\pgfpathlineto{\pgfqpoint{4.408461in}{1.976683in}}%
\pgfpathlineto{\pgfqpoint{4.416561in}{1.992390in}}%
\pgfpathlineto{\pgfqpoint{4.424657in}{2.008019in}}%
\pgfpathlineto{\pgfqpoint{4.432749in}{2.023567in}}%
\pgfpathlineto{\pgfqpoint{4.440836in}{2.039032in}}%
\pgfpathlineto{\pgfqpoint{4.426502in}{2.029163in}}%
\pgfpathlineto{\pgfqpoint{4.412184in}{2.019478in}}%
\pgfpathlineto{\pgfqpoint{4.397880in}{2.009975in}}%
\pgfpathlineto{\pgfqpoint{4.383592in}{2.000655in}}%
\pgfpathlineto{\pgfqpoint{4.375509in}{1.985534in}}%
\pgfpathlineto{\pgfqpoint{4.367421in}{1.970338in}}%
\pgfpathlineto{\pgfqpoint{4.359329in}{1.955069in}}%
\pgfpathlineto{\pgfqpoint{4.351232in}{1.939731in}}%
\pgfpathclose%
\pgfusepath{fill}%
\end{pgfscope}%
\begin{pgfscope}%
\pgfpathrectangle{\pgfqpoint{1.150000in}{0.150000in}}{\pgfqpoint{5.700000in}{5.700000in}}%
\pgfusepath{clip}%
\pgfsetbuttcap%
\pgfsetroundjoin%
\definecolor{currentfill}{rgb}{0.126453,0.570633,0.549841}%
\pgfsetfillcolor{currentfill}%
\pgfsetfillopacity{0.800000}%
\pgfsetlinewidth{0.000000pt}%
\definecolor{currentstroke}{rgb}{0.000000,0.000000,0.000000}%
\pgfsetstrokecolor{currentstroke}%
\pgfsetdash{}{0pt}%
\pgfpathmoveto{\pgfqpoint{1.997514in}{2.826128in}}%
\pgfpathlineto{\pgfqpoint{2.012204in}{2.796665in}}%
\pgfpathlineto{\pgfqpoint{2.026874in}{2.767550in}}%
\pgfpathlineto{\pgfqpoint{2.041525in}{2.738781in}}%
\pgfpathlineto{\pgfqpoint{2.056156in}{2.710355in}}%
\pgfpathlineto{\pgfqpoint{2.065935in}{2.696552in}}%
\pgfpathlineto{\pgfqpoint{2.075680in}{2.683262in}}%
\pgfpathlineto{\pgfqpoint{2.085392in}{2.670477in}}%
\pgfpathlineto{\pgfqpoint{2.095071in}{2.658186in}}%
\pgfpathlineto{\pgfqpoint{2.080523in}{2.685748in}}%
\pgfpathlineto{\pgfqpoint{2.065957in}{2.713650in}}%
\pgfpathlineto{\pgfqpoint{2.051371in}{2.741895in}}%
\pgfpathlineto{\pgfqpoint{2.036766in}{2.770487in}}%
\pgfpathlineto{\pgfqpoint{2.027005in}{2.783629in}}%
\pgfpathlineto{\pgfqpoint{2.017209in}{2.797277in}}%
\pgfpathlineto{\pgfqpoint{2.007379in}{2.811440in}}%
\pgfpathlineto{\pgfqpoint{1.997514in}{2.826128in}}%
\pgfpathclose%
\pgfusepath{fill}%
\end{pgfscope}%
\begin{pgfscope}%
\pgfpathrectangle{\pgfqpoint{1.150000in}{0.150000in}}{\pgfqpoint{5.700000in}{5.700000in}}%
\pgfusepath{clip}%
\pgfsetbuttcap%
\pgfsetroundjoin%
\definecolor{currentfill}{rgb}{0.203063,0.379716,0.553925}%
\pgfsetfillcolor{currentfill}%
\pgfsetfillopacity{0.800000}%
\pgfsetlinewidth{0.000000pt}%
\definecolor{currentstroke}{rgb}{0.000000,0.000000,0.000000}%
\pgfsetstrokecolor{currentstroke}%
\pgfsetdash{}{0pt}%
\pgfpathmoveto{\pgfqpoint{4.473140in}{2.099994in}}%
\pgfpathlineto{\pgfqpoint{4.487494in}{2.110369in}}%
\pgfpathlineto{\pgfqpoint{4.501864in}{2.120928in}}%
\pgfpathlineto{\pgfqpoint{4.516250in}{2.131671in}}%
\pgfpathlineto{\pgfqpoint{4.530653in}{2.142597in}}%
\pgfpathlineto{\pgfqpoint{4.538721in}{2.157903in}}%
\pgfpathlineto{\pgfqpoint{4.546785in}{2.173100in}}%
\pgfpathlineto{\pgfqpoint{4.554844in}{2.188186in}}%
\pgfpathlineto{\pgfqpoint{4.562897in}{2.203158in}}%
\pgfpathlineto{\pgfqpoint{4.548490in}{2.191939in}}%
\pgfpathlineto{\pgfqpoint{4.534099in}{2.180905in}}%
\pgfpathlineto{\pgfqpoint{4.519725in}{2.170055in}}%
\pgfpathlineto{\pgfqpoint{4.505367in}{2.159389in}}%
\pgfpathlineto{\pgfqpoint{4.497317in}{2.144696in}}%
\pgfpathlineto{\pgfqpoint{4.489263in}{2.129897in}}%
\pgfpathlineto{\pgfqpoint{4.481204in}{2.114996in}}%
\pgfpathlineto{\pgfqpoint{4.473140in}{2.099994in}}%
\pgfpathclose%
\pgfusepath{fill}%
\end{pgfscope}%
\begin{pgfscope}%
\pgfpathrectangle{\pgfqpoint{1.150000in}{0.150000in}}{\pgfqpoint{5.700000in}{5.700000in}}%
\pgfusepath{clip}%
\pgfsetbuttcap%
\pgfsetroundjoin%
\definecolor{currentfill}{rgb}{0.255645,0.260703,0.528312}%
\pgfsetfillcolor{currentfill}%
\pgfsetfillopacity{0.800000}%
\pgfsetlinewidth{0.000000pt}%
\definecolor{currentstroke}{rgb}{0.000000,0.000000,0.000000}%
\pgfsetstrokecolor{currentstroke}%
\pgfsetdash{}{0pt}%
\pgfpathmoveto{\pgfqpoint{4.229335in}{1.784140in}}%
\pgfpathlineto{\pgfqpoint{4.243557in}{1.791570in}}%
\pgfpathlineto{\pgfqpoint{4.257793in}{1.799181in}}%
\pgfpathlineto{\pgfqpoint{4.272043in}{1.806974in}}%
\pgfpathlineto{\pgfqpoint{4.286307in}{1.814949in}}%
\pgfpathlineto{\pgfqpoint{4.294437in}{1.830715in}}%
\pgfpathlineto{\pgfqpoint{4.302563in}{1.846441in}}%
\pgfpathlineto{\pgfqpoint{4.310685in}{1.862122in}}%
\pgfpathlineto{\pgfqpoint{4.318803in}{1.877755in}}%
\pgfpathlineto{\pgfqpoint{4.304536in}{1.869360in}}%
\pgfpathlineto{\pgfqpoint{4.290283in}{1.861148in}}%
\pgfpathlineto{\pgfqpoint{4.276045in}{1.853117in}}%
\pgfpathlineto{\pgfqpoint{4.261820in}{1.845268in}}%
\pgfpathlineto{\pgfqpoint{4.253705in}{1.830042in}}%
\pgfpathlineto{\pgfqpoint{4.245586in}{1.814776in}}%
\pgfpathlineto{\pgfqpoint{4.237462in}{1.799474in}}%
\pgfpathlineto{\pgfqpoint{4.229335in}{1.784140in}}%
\pgfpathclose%
\pgfusepath{fill}%
\end{pgfscope}%
\begin{pgfscope}%
\pgfpathrectangle{\pgfqpoint{1.150000in}{0.150000in}}{\pgfqpoint{5.700000in}{5.700000in}}%
\pgfusepath{clip}%
\pgfsetbuttcap%
\pgfsetroundjoin%
\definecolor{currentfill}{rgb}{0.179019,0.433756,0.557430}%
\pgfsetfillcolor{currentfill}%
\pgfsetfillopacity{0.800000}%
\pgfsetlinewidth{0.000000pt}%
\definecolor{currentstroke}{rgb}{0.000000,0.000000,0.000000}%
\pgfsetstrokecolor{currentstroke}%
\pgfsetdash{}{0pt}%
\pgfpathmoveto{\pgfqpoint{4.595060in}{2.261858in}}%
\pgfpathlineto{\pgfqpoint{4.609488in}{2.273519in}}%
\pgfpathlineto{\pgfqpoint{4.623934in}{2.285364in}}%
\pgfpathlineto{\pgfqpoint{4.638397in}{2.297394in}}%
\pgfpathlineto{\pgfqpoint{4.652878in}{2.309609in}}%
\pgfpathlineto{\pgfqpoint{4.660910in}{2.324214in}}%
\pgfpathlineto{\pgfqpoint{4.668936in}{2.338685in}}%
\pgfpathlineto{\pgfqpoint{4.676956in}{2.353018in}}%
\pgfpathlineto{\pgfqpoint{4.684971in}{2.367211in}}%
\pgfpathlineto{\pgfqpoint{4.670486in}{2.354771in}}%
\pgfpathlineto{\pgfqpoint{4.656018in}{2.342515in}}%
\pgfpathlineto{\pgfqpoint{4.641568in}{2.330445in}}%
\pgfpathlineto{\pgfqpoint{4.627136in}{2.318559in}}%
\pgfpathlineto{\pgfqpoint{4.619125in}{2.304578in}}%
\pgfpathlineto{\pgfqpoint{4.611109in}{2.290466in}}%
\pgfpathlineto{\pgfqpoint{4.603087in}{2.276225in}}%
\pgfpathlineto{\pgfqpoint{4.595060in}{2.261858in}}%
\pgfpathclose%
\pgfusepath{fill}%
\end{pgfscope}%
\begin{pgfscope}%
\pgfpathrectangle{\pgfqpoint{1.150000in}{0.150000in}}{\pgfqpoint{5.700000in}{5.700000in}}%
\pgfusepath{clip}%
\pgfsetbuttcap%
\pgfsetroundjoin%
\definecolor{currentfill}{rgb}{0.157729,0.485932,0.558013}%
\pgfsetfillcolor{currentfill}%
\pgfsetfillopacity{0.800000}%
\pgfsetlinewidth{0.000000pt}%
\definecolor{currentstroke}{rgb}{0.000000,0.000000,0.000000}%
\pgfsetstrokecolor{currentstroke}%
\pgfsetdash{}{0pt}%
\pgfpathmoveto{\pgfqpoint{4.716970in}{2.422563in}}%
\pgfpathlineto{\pgfqpoint{4.731477in}{2.435380in}}%
\pgfpathlineto{\pgfqpoint{4.746002in}{2.448383in}}%
\pgfpathlineto{\pgfqpoint{4.760545in}{2.461572in}}%
\pgfpathlineto{\pgfqpoint{4.775107in}{2.474946in}}%
\pgfpathlineto{\pgfqpoint{4.783095in}{2.488591in}}%
\pgfpathlineto{\pgfqpoint{4.791077in}{2.502080in}}%
\pgfpathlineto{\pgfqpoint{4.799052in}{2.515410in}}%
\pgfpathlineto{\pgfqpoint{4.807021in}{2.528582in}}%
\pgfpathlineto{\pgfqpoint{4.792455in}{2.515050in}}%
\pgfpathlineto{\pgfqpoint{4.777908in}{2.501703in}}%
\pgfpathlineto{\pgfqpoint{4.763380in}{2.488543in}}%
\pgfpathlineto{\pgfqpoint{4.748870in}{2.475568in}}%
\pgfpathlineto{\pgfqpoint{4.740905in}{2.462541in}}%
\pgfpathlineto{\pgfqpoint{4.732933in}{2.449364in}}%
\pgfpathlineto{\pgfqpoint{4.724954in}{2.436037in}}%
\pgfpathlineto{\pgfqpoint{4.716970in}{2.422563in}}%
\pgfpathclose%
\pgfusepath{fill}%
\end{pgfscope}%
\begin{pgfscope}%
\pgfpathrectangle{\pgfqpoint{1.150000in}{0.150000in}}{\pgfqpoint{5.700000in}{5.700000in}}%
\pgfusepath{clip}%
\pgfsetbuttcap%
\pgfsetroundjoin%
\definecolor{currentfill}{rgb}{0.269944,0.014625,0.341379}%
\pgfsetfillcolor{currentfill}%
\pgfsetfillopacity{0.800000}%
\pgfsetlinewidth{0.000000pt}%
\definecolor{currentstroke}{rgb}{0.000000,0.000000,0.000000}%
\pgfsetstrokecolor{currentstroke}%
\pgfsetdash{}{0pt}%
\pgfpathmoveto{\pgfqpoint{3.538823in}{1.259429in}}%
\pgfpathlineto{\pgfqpoint{3.552842in}{1.256632in}}%
\pgfpathlineto{\pgfqpoint{3.566866in}{1.254020in}}%
\pgfpathlineto{\pgfqpoint{3.580896in}{1.251591in}}%
\pgfpathlineto{\pgfqpoint{3.594932in}{1.249345in}}%
\pgfpathlineto{\pgfqpoint{3.603292in}{1.258345in}}%
\pgfpathlineto{\pgfqpoint{3.611643in}{1.267554in}}%
\pgfpathlineto{\pgfqpoint{3.619987in}{1.276963in}}%
\pgfpathlineto{\pgfqpoint{3.628323in}{1.286566in}}%
\pgfpathlineto{\pgfqpoint{3.614304in}{1.288117in}}%
\pgfpathlineto{\pgfqpoint{3.600293in}{1.289852in}}%
\pgfpathlineto{\pgfqpoint{3.586287in}{1.291770in}}%
\pgfpathlineto{\pgfqpoint{3.572287in}{1.293873in}}%
\pgfpathlineto{\pgfqpoint{3.563934in}{1.284952in}}%
\pgfpathlineto{\pgfqpoint{3.555572in}{1.276233in}}%
\pgfpathlineto{\pgfqpoint{3.547202in}{1.267723in}}%
\pgfpathlineto{\pgfqpoint{3.538823in}{1.259429in}}%
\pgfpathclose%
\pgfusepath{fill}%
\end{pgfscope}%
\begin{pgfscope}%
\pgfpathrectangle{\pgfqpoint{1.150000in}{0.150000in}}{\pgfqpoint{5.700000in}{5.700000in}}%
\pgfusepath{clip}%
\pgfsetbuttcap%
\pgfsetroundjoin%
\definecolor{currentfill}{rgb}{0.171176,0.452530,0.557965}%
\pgfsetfillcolor{currentfill}%
\pgfsetfillopacity{0.800000}%
\pgfsetlinewidth{0.000000pt}%
\definecolor{currentstroke}{rgb}{0.000000,0.000000,0.000000}%
\pgfsetstrokecolor{currentstroke}%
\pgfsetdash{}{0pt}%
\pgfpathmoveto{\pgfqpoint{2.191883in}{2.444577in}}%
\pgfpathlineto{\pgfqpoint{2.206374in}{2.419435in}}%
\pgfpathlineto{\pgfqpoint{2.220850in}{2.394596in}}%
\pgfpathlineto{\pgfqpoint{2.235311in}{2.370057in}}%
\pgfpathlineto{\pgfqpoint{2.249757in}{2.345814in}}%
\pgfpathlineto{\pgfqpoint{2.259343in}{2.333488in}}%
\pgfpathlineto{\pgfqpoint{2.268898in}{2.321667in}}%
\pgfpathlineto{\pgfqpoint{2.278422in}{2.310342in}}%
\pgfpathlineto{\pgfqpoint{2.287915in}{2.299504in}}%
\pgfpathlineto{\pgfqpoint{2.273545in}{2.322875in}}%
\pgfpathlineto{\pgfqpoint{2.259162in}{2.346541in}}%
\pgfpathlineto{\pgfqpoint{2.244764in}{2.370505in}}%
\pgfpathlineto{\pgfqpoint{2.230352in}{2.394768in}}%
\pgfpathlineto{\pgfqpoint{2.220783in}{2.406464in}}%
\pgfpathlineto{\pgfqpoint{2.211182in}{2.418658in}}%
\pgfpathlineto{\pgfqpoint{2.201549in}{2.431359in}}%
\pgfpathlineto{\pgfqpoint{2.191883in}{2.444577in}}%
\pgfpathclose%
\pgfusepath{fill}%
\end{pgfscope}%
\begin{pgfscope}%
\pgfpathrectangle{\pgfqpoint{1.150000in}{0.150000in}}{\pgfqpoint{5.700000in}{5.700000in}}%
\pgfusepath{clip}%
\pgfsetbuttcap%
\pgfsetroundjoin%
\definecolor{currentfill}{rgb}{0.283091,0.110553,0.431554}%
\pgfsetfillcolor{currentfill}%
\pgfsetfillopacity{0.800000}%
\pgfsetlinewidth{0.000000pt}%
\definecolor{currentstroke}{rgb}{0.000000,0.000000,0.000000}%
\pgfsetstrokecolor{currentstroke}%
\pgfsetdash{}{0pt}%
\pgfpathmoveto{\pgfqpoint{3.896162in}{1.432924in}}%
\pgfpathlineto{\pgfqpoint{3.910252in}{1.435629in}}%
\pgfpathlineto{\pgfqpoint{3.924352in}{1.438514in}}%
\pgfpathlineto{\pgfqpoint{3.938462in}{1.441579in}}%
\pgfpathlineto{\pgfqpoint{3.952583in}{1.444824in}}%
\pgfpathlineto{\pgfqpoint{3.960798in}{1.458703in}}%
\pgfpathlineto{\pgfqpoint{3.969008in}{1.472660in}}%
\pgfpathlineto{\pgfqpoint{3.977214in}{1.486688in}}%
\pgfpathlineto{\pgfqpoint{3.985415in}{1.500782in}}%
\pgfpathlineto{\pgfqpoint{3.971299in}{1.496964in}}%
\pgfpathlineto{\pgfqpoint{3.957194in}{1.493326in}}%
\pgfpathlineto{\pgfqpoint{3.943099in}{1.489868in}}%
\pgfpathlineto{\pgfqpoint{3.929014in}{1.486591in}}%
\pgfpathlineto{\pgfqpoint{3.920808in}{1.473058in}}%
\pgfpathlineto{\pgfqpoint{3.912598in}{1.459599in}}%
\pgfpathlineto{\pgfqpoint{3.904383in}{1.446219in}}%
\pgfpathlineto{\pgfqpoint{3.896162in}{1.432924in}}%
\pgfpathclose%
\pgfusepath{fill}%
\end{pgfscope}%
\begin{pgfscope}%
\pgfpathrectangle{\pgfqpoint{1.150000in}{0.150000in}}{\pgfqpoint{5.700000in}{5.700000in}}%
\pgfusepath{clip}%
\pgfsetbuttcap%
\pgfsetroundjoin%
\definecolor{currentfill}{rgb}{0.274128,0.199721,0.498911}%
\pgfsetfillcolor{currentfill}%
\pgfsetfillopacity{0.800000}%
\pgfsetlinewidth{0.000000pt}%
\definecolor{currentstroke}{rgb}{0.000000,0.000000,0.000000}%
\pgfsetstrokecolor{currentstroke}%
\pgfsetdash{}{0pt}%
\pgfpathmoveto{\pgfqpoint{4.107417in}{1.636595in}}%
\pgfpathlineto{\pgfqpoint{4.121586in}{1.642369in}}%
\pgfpathlineto{\pgfqpoint{4.135767in}{1.648324in}}%
\pgfpathlineto{\pgfqpoint{4.149961in}{1.654460in}}%
\pgfpathlineto{\pgfqpoint{4.164168in}{1.660776in}}%
\pgfpathlineto{\pgfqpoint{4.172328in}{1.676223in}}%
\pgfpathlineto{\pgfqpoint{4.180484in}{1.691672in}}%
\pgfpathlineto{\pgfqpoint{4.188636in}{1.707117in}}%
\pgfpathlineto{\pgfqpoint{4.196783in}{1.722554in}}%
\pgfpathlineto{\pgfqpoint{4.182576in}{1.715756in}}%
\pgfpathlineto{\pgfqpoint{4.168381in}{1.709138in}}%
\pgfpathlineto{\pgfqpoint{4.154200in}{1.702702in}}%
\pgfpathlineto{\pgfqpoint{4.140030in}{1.696446in}}%
\pgfpathlineto{\pgfqpoint{4.131883in}{1.681479in}}%
\pgfpathlineto{\pgfqpoint{4.123732in}{1.666511in}}%
\pgfpathlineto{\pgfqpoint{4.115576in}{1.651549in}}%
\pgfpathlineto{\pgfqpoint{4.107417in}{1.636595in}}%
\pgfpathclose%
\pgfusepath{fill}%
\end{pgfscope}%
\begin{pgfscope}%
\pgfpathrectangle{\pgfqpoint{1.150000in}{0.150000in}}{\pgfqpoint{5.700000in}{5.700000in}}%
\pgfusepath{clip}%
\pgfsetbuttcap%
\pgfsetroundjoin%
\definecolor{currentfill}{rgb}{0.137339,0.662252,0.515571}%
\pgfsetfillcolor{currentfill}%
\pgfsetfillopacity{0.800000}%
\pgfsetlinewidth{0.000000pt}%
\definecolor{currentstroke}{rgb}{0.000000,0.000000,0.000000}%
\pgfsetstrokecolor{currentstroke}%
\pgfsetdash{}{0pt}%
\pgfpathmoveto{\pgfqpoint{5.172406in}{2.975315in}}%
\pgfpathlineto{\pgfqpoint{5.187235in}{2.991498in}}%
\pgfpathlineto{\pgfqpoint{5.202085in}{3.007871in}}%
\pgfpathlineto{\pgfqpoint{5.216957in}{3.024433in}}%
\pgfpathlineto{\pgfqpoint{5.231851in}{3.041184in}}%
\pgfpathlineto{\pgfqpoint{5.239615in}{3.049989in}}%
\pgfpathlineto{\pgfqpoint{5.247369in}{3.058603in}}%
\pgfpathlineto{\pgfqpoint{5.255113in}{3.067027in}}%
\pgfpathlineto{\pgfqpoint{5.262847in}{3.075262in}}%
\pgfpathlineto{\pgfqpoint{5.247958in}{3.058601in}}%
\pgfpathlineto{\pgfqpoint{5.233091in}{3.042129in}}%
\pgfpathlineto{\pgfqpoint{5.218245in}{3.025846in}}%
\pgfpathlineto{\pgfqpoint{5.203422in}{3.009752in}}%
\pgfpathlineto{\pgfqpoint{5.195682in}{3.001414in}}%
\pgfpathlineto{\pgfqpoint{5.187933in}{2.992896in}}%
\pgfpathlineto{\pgfqpoint{5.180174in}{2.984196in}}%
\pgfpathlineto{\pgfqpoint{5.172406in}{2.975315in}}%
\pgfpathclose%
\pgfusepath{fill}%
\end{pgfscope}%
\begin{pgfscope}%
\pgfpathrectangle{\pgfqpoint{1.150000in}{0.150000in}}{\pgfqpoint{5.700000in}{5.700000in}}%
\pgfusepath{clip}%
\pgfsetbuttcap%
\pgfsetroundjoin%
\definecolor{currentfill}{rgb}{0.277018,0.050344,0.375715}%
\pgfsetfillcolor{currentfill}%
\pgfsetfillopacity{0.800000}%
\pgfsetlinewidth{0.000000pt}%
\definecolor{currentstroke}{rgb}{0.000000,0.000000,0.000000}%
\pgfsetstrokecolor{currentstroke}%
\pgfsetdash{}{0pt}%
\pgfpathmoveto{\pgfqpoint{3.100122in}{1.357764in}}%
\pgfpathlineto{\pgfqpoint{3.114166in}{1.348100in}}%
\pgfpathlineto{\pgfqpoint{3.128210in}{1.338635in}}%
\pgfpathlineto{\pgfqpoint{3.142255in}{1.329369in}}%
\pgfpathlineto{\pgfqpoint{3.156299in}{1.320301in}}%
\pgfpathlineto{\pgfqpoint{3.164966in}{1.321401in}}%
\pgfpathlineto{\pgfqpoint{3.173617in}{1.322849in}}%
\pgfpathlineto{\pgfqpoint{3.182253in}{1.324639in}}%
\pgfpathlineto{\pgfqpoint{3.190875in}{1.326760in}}%
\pgfpathlineto{\pgfqpoint{3.176868in}{1.335032in}}%
\pgfpathlineto{\pgfqpoint{3.162862in}{1.343501in}}%
\pgfpathlineto{\pgfqpoint{3.148857in}{1.352167in}}%
\pgfpathlineto{\pgfqpoint{3.134852in}{1.361033in}}%
\pgfpathlineto{\pgfqpoint{3.126193in}{1.359696in}}%
\pgfpathlineto{\pgfqpoint{3.117519in}{1.358700in}}%
\pgfpathlineto{\pgfqpoint{3.108829in}{1.358053in}}%
\pgfpathlineto{\pgfqpoint{3.100122in}{1.357764in}}%
\pgfpathclose%
\pgfusepath{fill}%
\end{pgfscope}%
\begin{pgfscope}%
\pgfpathrectangle{\pgfqpoint{1.150000in}{0.150000in}}{\pgfqpoint{5.700000in}{5.700000in}}%
\pgfusepath{clip}%
\pgfsetbuttcap%
\pgfsetroundjoin%
\definecolor{currentfill}{rgb}{0.360741,0.785964,0.387814}%
\pgfsetfillcolor{currentfill}%
\pgfsetfillopacity{0.800000}%
\pgfsetlinewidth{0.000000pt}%
\definecolor{currentstroke}{rgb}{0.000000,0.000000,0.000000}%
\pgfsetstrokecolor{currentstroke}%
\pgfsetdash{}{0pt}%
\pgfpathmoveto{\pgfqpoint{5.595268in}{3.406155in}}%
\pgfpathlineto{\pgfqpoint{5.610406in}{3.424285in}}%
\pgfpathlineto{\pgfqpoint{5.625569in}{3.442606in}}%
\pgfpathlineto{\pgfqpoint{5.640756in}{3.461117in}}%
\pgfpathlineto{\pgfqpoint{5.655968in}{3.479820in}}%
\pgfpathlineto{\pgfqpoint{5.663430in}{3.483407in}}%
\pgfpathlineto{\pgfqpoint{5.670879in}{3.486825in}}%
\pgfpathlineto{\pgfqpoint{5.678317in}{3.490076in}}%
\pgfpathlineto{\pgfqpoint{5.685744in}{3.493166in}}%
\pgfpathlineto{\pgfqpoint{5.670549in}{3.474776in}}%
\pgfpathlineto{\pgfqpoint{5.655380in}{3.456576in}}%
\pgfpathlineto{\pgfqpoint{5.640234in}{3.438566in}}%
\pgfpathlineto{\pgfqpoint{5.625112in}{3.420746in}}%
\pgfpathlineto{\pgfqpoint{5.617668in}{3.417332in}}%
\pgfpathlineto{\pgfqpoint{5.610213in}{3.413765in}}%
\pgfpathlineto{\pgfqpoint{5.602746in}{3.410040in}}%
\pgfpathlineto{\pgfqpoint{5.595268in}{3.406155in}}%
\pgfpathclose%
\pgfusepath{fill}%
\end{pgfscope}%
\begin{pgfscope}%
\pgfpathrectangle{\pgfqpoint{1.150000in}{0.150000in}}{\pgfqpoint{5.700000in}{5.700000in}}%
\pgfusepath{clip}%
\pgfsetbuttcap%
\pgfsetroundjoin%
\definecolor{currentfill}{rgb}{0.269944,0.014625,0.341379}%
\pgfsetfillcolor{currentfill}%
\pgfsetfillopacity{0.800000}%
\pgfsetlinewidth{0.000000pt}%
\definecolor{currentstroke}{rgb}{0.000000,0.000000,0.000000}%
\pgfsetstrokecolor{currentstroke}%
\pgfsetdash{}{0pt}%
\pgfpathmoveto{\pgfqpoint{3.302981in}{1.267605in}}%
\pgfpathlineto{\pgfqpoint{3.317003in}{1.261077in}}%
\pgfpathlineto{\pgfqpoint{3.331028in}{1.254739in}}%
\pgfpathlineto{\pgfqpoint{3.345055in}{1.248591in}}%
\pgfpathlineto{\pgfqpoint{3.359085in}{1.242632in}}%
\pgfpathlineto{\pgfqpoint{3.367596in}{1.247398in}}%
\pgfpathlineto{\pgfqpoint{3.376095in}{1.252454in}}%
\pgfpathlineto{\pgfqpoint{3.384582in}{1.257794in}}%
\pgfpathlineto{\pgfqpoint{3.393058in}{1.263409in}}%
\pgfpathlineto{\pgfqpoint{3.379056in}{1.268608in}}%
\pgfpathlineto{\pgfqpoint{3.365057in}{1.273997in}}%
\pgfpathlineto{\pgfqpoint{3.351062in}{1.279575in}}%
\pgfpathlineto{\pgfqpoint{3.337070in}{1.285343in}}%
\pgfpathlineto{\pgfqpoint{3.328566in}{1.280475in}}%
\pgfpathlineto{\pgfqpoint{3.320050in}{1.275890in}}%
\pgfpathlineto{\pgfqpoint{3.311521in}{1.271598in}}%
\pgfpathlineto{\pgfqpoint{3.302981in}{1.267605in}}%
\pgfpathclose%
\pgfusepath{fill}%
\end{pgfscope}%
\begin{pgfscope}%
\pgfpathrectangle{\pgfqpoint{1.150000in}{0.150000in}}{\pgfqpoint{5.700000in}{5.700000in}}%
\pgfusepath{clip}%
\pgfsetbuttcap%
\pgfsetroundjoin%
\definecolor{currentfill}{rgb}{0.282290,0.145912,0.461510}%
\pgfsetfillcolor{currentfill}%
\pgfsetfillopacity{0.800000}%
\pgfsetlinewidth{0.000000pt}%
\definecolor{currentstroke}{rgb}{0.000000,0.000000,0.000000}%
\pgfsetstrokecolor{currentstroke}%
\pgfsetdash{}{0pt}%
\pgfpathmoveto{\pgfqpoint{3.985415in}{1.500782in}}%
\pgfpathlineto{\pgfqpoint{3.999541in}{1.504781in}}%
\pgfpathlineto{\pgfqpoint{4.013679in}{1.508959in}}%
\pgfpathlineto{\pgfqpoint{4.027827in}{1.513317in}}%
\pgfpathlineto{\pgfqpoint{4.041987in}{1.517855in}}%
\pgfpathlineto{\pgfqpoint{4.050181in}{1.532565in}}%
\pgfpathlineto{\pgfqpoint{4.058370in}{1.547323in}}%
\pgfpathlineto{\pgfqpoint{4.066555in}{1.562124in}}%
\pgfpathlineto{\pgfqpoint{4.074736in}{1.576963in}}%
\pgfpathlineto{\pgfqpoint{4.060578in}{1.571881in}}%
\pgfpathlineto{\pgfqpoint{4.046432in}{1.566979in}}%
\pgfpathlineto{\pgfqpoint{4.032297in}{1.562258in}}%
\pgfpathlineto{\pgfqpoint{4.018173in}{1.557717in}}%
\pgfpathlineto{\pgfqpoint{4.009990in}{1.543410in}}%
\pgfpathlineto{\pgfqpoint{4.001803in}{1.529149in}}%
\pgfpathlineto{\pgfqpoint{3.993611in}{1.514938in}}%
\pgfpathlineto{\pgfqpoint{3.985415in}{1.500782in}}%
\pgfpathclose%
\pgfusepath{fill}%
\end{pgfscope}%
\begin{pgfscope}%
\pgfpathrectangle{\pgfqpoint{1.150000in}{0.150000in}}{\pgfqpoint{5.700000in}{5.700000in}}%
\pgfusepath{clip}%
\pgfsetbuttcap%
\pgfsetroundjoin%
\definecolor{currentfill}{rgb}{0.268510,0.009605,0.335427}%
\pgfsetfillcolor{currentfill}%
\pgfsetfillopacity{0.800000}%
\pgfsetlinewidth{0.000000pt}%
\definecolor{currentstroke}{rgb}{0.000000,0.000000,0.000000}%
\pgfsetstrokecolor{currentstroke}%
\pgfsetdash{}{0pt}%
\pgfpathmoveto{\pgfqpoint{3.449103in}{1.244490in}}%
\pgfpathlineto{\pgfqpoint{3.463124in}{1.240227in}}%
\pgfpathlineto{\pgfqpoint{3.477150in}{1.236150in}}%
\pgfpathlineto{\pgfqpoint{3.491181in}{1.232258in}}%
\pgfpathlineto{\pgfqpoint{3.505216in}{1.228551in}}%
\pgfpathlineto{\pgfqpoint{3.513632in}{1.235912in}}%
\pgfpathlineto{\pgfqpoint{3.522038in}{1.243516in}}%
\pgfpathlineto{\pgfqpoint{3.530435in}{1.251357in}}%
\pgfpathlineto{\pgfqpoint{3.538823in}{1.259429in}}%
\pgfpathlineto{\pgfqpoint{3.524810in}{1.262410in}}%
\pgfpathlineto{\pgfqpoint{3.510802in}{1.265575in}}%
\pgfpathlineto{\pgfqpoint{3.496799in}{1.268926in}}%
\pgfpathlineto{\pgfqpoint{3.482800in}{1.272463in}}%
\pgfpathlineto{\pgfqpoint{3.474391in}{1.265105in}}%
\pgfpathlineto{\pgfqpoint{3.465972in}{1.257986in}}%
\pgfpathlineto{\pgfqpoint{3.457542in}{1.251112in}}%
\pgfpathlineto{\pgfqpoint{3.449103in}{1.244490in}}%
\pgfpathclose%
\pgfusepath{fill}%
\end{pgfscope}%
\begin{pgfscope}%
\pgfpathrectangle{\pgfqpoint{1.150000in}{0.150000in}}{\pgfqpoint{5.700000in}{5.700000in}}%
\pgfusepath{clip}%
\pgfsetbuttcap%
\pgfsetroundjoin%
\definecolor{currentfill}{rgb}{0.232815,0.732247,0.459277}%
\pgfsetfillcolor{currentfill}%
\pgfsetfillopacity{0.800000}%
\pgfsetlinewidth{0.000000pt}%
\definecolor{currentstroke}{rgb}{0.000000,0.000000,0.000000}%
\pgfsetstrokecolor{currentstroke}%
\pgfsetdash{}{0pt}%
\pgfpathmoveto{\pgfqpoint{5.384086in}{3.201939in}}%
\pgfpathlineto{\pgfqpoint{5.399073in}{3.219258in}}%
\pgfpathlineto{\pgfqpoint{5.414084in}{3.236768in}}%
\pgfpathlineto{\pgfqpoint{5.429117in}{3.254467in}}%
\pgfpathlineto{\pgfqpoint{5.444174in}{3.272358in}}%
\pgfpathlineto{\pgfqpoint{5.451800in}{3.278603in}}%
\pgfpathlineto{\pgfqpoint{5.459414in}{3.284662in}}%
\pgfpathlineto{\pgfqpoint{5.467017in}{3.290535in}}%
\pgfpathlineto{\pgfqpoint{5.474610in}{3.296227in}}%
\pgfpathlineto{\pgfqpoint{5.459563in}{3.278537in}}%
\pgfpathlineto{\pgfqpoint{5.444540in}{3.261038in}}%
\pgfpathlineto{\pgfqpoint{5.429540in}{3.243728in}}%
\pgfpathlineto{\pgfqpoint{5.414563in}{3.226608in}}%
\pgfpathlineto{\pgfqpoint{5.406959in}{3.220703in}}%
\pgfpathlineto{\pgfqpoint{5.399345in}{3.214625in}}%
\pgfpathlineto{\pgfqpoint{5.391721in}{3.208371in}}%
\pgfpathlineto{\pgfqpoint{5.384086in}{3.201939in}}%
\pgfpathclose%
\pgfusepath{fill}%
\end{pgfscope}%
\begin{pgfscope}%
\pgfpathrectangle{\pgfqpoint{1.150000in}{0.150000in}}{\pgfqpoint{5.700000in}{5.700000in}}%
\pgfusepath{clip}%
\pgfsetbuttcap%
\pgfsetroundjoin%
\definecolor{currentfill}{rgb}{0.156270,0.489624,0.557936}%
\pgfsetfillcolor{currentfill}%
\pgfsetfillopacity{0.800000}%
\pgfsetlinewidth{0.000000pt}%
\definecolor{currentstroke}{rgb}{0.000000,0.000000,0.000000}%
\pgfsetstrokecolor{currentstroke}%
\pgfsetdash{}{0pt}%
\pgfpathmoveto{\pgfqpoint{2.133761in}{2.548218in}}%
\pgfpathlineto{\pgfqpoint{2.148316in}{2.521841in}}%
\pgfpathlineto{\pgfqpoint{2.162854in}{2.495776in}}%
\pgfpathlineto{\pgfqpoint{2.177377in}{2.470023in}}%
\pgfpathlineto{\pgfqpoint{2.191883in}{2.444577in}}%
\pgfpathlineto{\pgfqpoint{2.201549in}{2.431359in}}%
\pgfpathlineto{\pgfqpoint{2.211182in}{2.418658in}}%
\pgfpathlineto{\pgfqpoint{2.220783in}{2.406464in}}%
\pgfpathlineto{\pgfqpoint{2.230352in}{2.394768in}}%
\pgfpathlineto{\pgfqpoint{2.215925in}{2.419334in}}%
\pgfpathlineto{\pgfqpoint{2.201483in}{2.444205in}}%
\pgfpathlineto{\pgfqpoint{2.187025in}{2.469385in}}%
\pgfpathlineto{\pgfqpoint{2.172552in}{2.494875in}}%
\pgfpathlineto{\pgfqpoint{2.162904in}{2.507438in}}%
\pgfpathlineto{\pgfqpoint{2.153223in}{2.520510in}}%
\pgfpathlineto{\pgfqpoint{2.143509in}{2.534100in}}%
\pgfpathlineto{\pgfqpoint{2.133761in}{2.548218in}}%
\pgfpathclose%
\pgfusepath{fill}%
\end{pgfscope}%
\begin{pgfscope}%
\pgfpathrectangle{\pgfqpoint{1.150000in}{0.150000in}}{\pgfqpoint{5.700000in}{5.700000in}}%
\pgfusepath{clip}%
\pgfsetbuttcap%
\pgfsetroundjoin%
\definecolor{currentfill}{rgb}{0.120081,0.622161,0.534946}%
\pgfsetfillcolor{currentfill}%
\pgfsetfillopacity{0.800000}%
\pgfsetlinewidth{0.000000pt}%
\definecolor{currentstroke}{rgb}{0.000000,0.000000,0.000000}%
\pgfsetstrokecolor{currentstroke}%
\pgfsetdash{}{0pt}%
\pgfpathmoveto{\pgfqpoint{5.050813in}{2.834431in}}%
\pgfpathlineto{\pgfqpoint{5.065562in}{2.849932in}}%
\pgfpathlineto{\pgfqpoint{5.080331in}{2.865621in}}%
\pgfpathlineto{\pgfqpoint{5.095122in}{2.881499in}}%
\pgfpathlineto{\pgfqpoint{5.109934in}{2.897565in}}%
\pgfpathlineto{\pgfqpoint{5.117774in}{2.907943in}}%
\pgfpathlineto{\pgfqpoint{5.125606in}{2.918131in}}%
\pgfpathlineto{\pgfqpoint{5.133429in}{2.928130in}}%
\pgfpathlineto{\pgfqpoint{5.141243in}{2.937941in}}%
\pgfpathlineto{\pgfqpoint{5.126432in}{2.921892in}}%
\pgfpathlineto{\pgfqpoint{5.111644in}{2.906032in}}%
\pgfpathlineto{\pgfqpoint{5.096876in}{2.890361in}}%
\pgfpathlineto{\pgfqpoint{5.082129in}{2.874878in}}%
\pgfpathlineto{\pgfqpoint{5.074313in}{2.865036in}}%
\pgfpathlineto{\pgfqpoint{5.066488in}{2.855015in}}%
\pgfpathlineto{\pgfqpoint{5.058655in}{2.844813in}}%
\pgfpathlineto{\pgfqpoint{5.050813in}{2.834431in}}%
\pgfpathclose%
\pgfusepath{fill}%
\end{pgfscope}%
\begin{pgfscope}%
\pgfpathrectangle{\pgfqpoint{1.150000in}{0.150000in}}{\pgfqpoint{5.700000in}{5.700000in}}%
\pgfusepath{clip}%
\pgfsetbuttcap%
\pgfsetroundjoin%
\definecolor{currentfill}{rgb}{0.237441,0.305202,0.541921}%
\pgfsetfillcolor{currentfill}%
\pgfsetfillopacity{0.800000}%
\pgfsetlinewidth{0.000000pt}%
\definecolor{currentstroke}{rgb}{0.000000,0.000000,0.000000}%
\pgfsetstrokecolor{currentstroke}%
\pgfsetdash{}{0pt}%
\pgfpathmoveto{\pgfqpoint{4.318803in}{1.877755in}}%
\pgfpathlineto{\pgfqpoint{4.333084in}{1.886331in}}%
\pgfpathlineto{\pgfqpoint{4.347380in}{1.895089in}}%
\pgfpathlineto{\pgfqpoint{4.361690in}{1.904029in}}%
\pgfpathlineto{\pgfqpoint{4.376016in}{1.913151in}}%
\pgfpathlineto{\pgfqpoint{4.384133in}{1.929133in}}%
\pgfpathlineto{\pgfqpoint{4.392247in}{1.945052in}}%
\pgfpathlineto{\pgfqpoint{4.400356in}{1.960903in}}%
\pgfpathlineto{\pgfqpoint{4.408461in}{1.976683in}}%
\pgfpathlineto{\pgfqpoint{4.394131in}{1.967172in}}%
\pgfpathlineto{\pgfqpoint{4.379816in}{1.957843in}}%
\pgfpathlineto{\pgfqpoint{4.365517in}{1.948696in}}%
\pgfpathlineto{\pgfqpoint{4.351232in}{1.939731in}}%
\pgfpathlineto{\pgfqpoint{4.343131in}{1.924327in}}%
\pgfpathlineto{\pgfqpoint{4.335026in}{1.908861in}}%
\pgfpathlineto{\pgfqpoint{4.326916in}{1.893336in}}%
\pgfpathlineto{\pgfqpoint{4.318803in}{1.877755in}}%
\pgfpathclose%
\pgfusepath{fill}%
\end{pgfscope}%
\begin{pgfscope}%
\pgfpathrectangle{\pgfqpoint{1.150000in}{0.150000in}}{\pgfqpoint{5.700000in}{5.700000in}}%
\pgfusepath{clip}%
\pgfsetbuttcap%
\pgfsetroundjoin%
\definecolor{currentfill}{rgb}{0.119483,0.614817,0.537692}%
\pgfsetfillcolor{currentfill}%
\pgfsetfillopacity{0.800000}%
\pgfsetlinewidth{0.000000pt}%
\definecolor{currentstroke}{rgb}{0.000000,0.000000,0.000000}%
\pgfsetstrokecolor{currentstroke}%
\pgfsetdash{}{0pt}%
\pgfpathmoveto{\pgfqpoint{1.938542in}{2.947551in}}%
\pgfpathlineto{\pgfqpoint{1.953317in}{2.916653in}}%
\pgfpathlineto{\pgfqpoint{1.968070in}{2.886119in}}%
\pgfpathlineto{\pgfqpoint{1.982802in}{2.855945in}}%
\pgfpathlineto{\pgfqpoint{1.997514in}{2.826128in}}%
\pgfpathlineto{\pgfqpoint{2.007379in}{2.811440in}}%
\pgfpathlineto{\pgfqpoint{2.017209in}{2.797277in}}%
\pgfpathlineto{\pgfqpoint{2.027005in}{2.783629in}}%
\pgfpathlineto{\pgfqpoint{2.036766in}{2.770487in}}%
\pgfpathlineto{\pgfqpoint{2.022141in}{2.799429in}}%
\pgfpathlineto{\pgfqpoint{2.007496in}{2.828725in}}%
\pgfpathlineto{\pgfqpoint{1.992831in}{2.858379in}}%
\pgfpathlineto{\pgfqpoint{1.978144in}{2.888393in}}%
\pgfpathlineto{\pgfqpoint{1.968297in}{2.902396in}}%
\pgfpathlineto{\pgfqpoint{1.958415in}{2.916918in}}%
\pgfpathlineto{\pgfqpoint{1.948497in}{2.931966in}}%
\pgfpathlineto{\pgfqpoint{1.938542in}{2.947551in}}%
\pgfpathclose%
\pgfusepath{fill}%
\end{pgfscope}%
\begin{pgfscope}%
\pgfpathrectangle{\pgfqpoint{1.150000in}{0.150000in}}{\pgfqpoint{5.700000in}{5.700000in}}%
\pgfusepath{clip}%
\pgfsetbuttcap%
\pgfsetroundjoin%
\definecolor{currentfill}{rgb}{0.210503,0.363727,0.552206}%
\pgfsetfillcolor{currentfill}%
\pgfsetfillopacity{0.800000}%
\pgfsetlinewidth{0.000000pt}%
\definecolor{currentstroke}{rgb}{0.000000,0.000000,0.000000}%
\pgfsetstrokecolor{currentstroke}%
\pgfsetdash{}{0pt}%
\pgfpathmoveto{\pgfqpoint{4.440836in}{2.039032in}}%
\pgfpathlineto{\pgfqpoint{4.455186in}{2.049083in}}%
\pgfpathlineto{\pgfqpoint{4.469552in}{2.059317in}}%
\pgfpathlineto{\pgfqpoint{4.483933in}{2.069735in}}%
\pgfpathlineto{\pgfqpoint{4.498331in}{2.080335in}}%
\pgfpathlineto{\pgfqpoint{4.506418in}{2.096050in}}%
\pgfpathlineto{\pgfqpoint{4.514501in}{2.111668in}}%
\pgfpathlineto{\pgfqpoint{4.522579in}{2.127184in}}%
\pgfpathlineto{\pgfqpoint{4.530653in}{2.142597in}}%
\pgfpathlineto{\pgfqpoint{4.516250in}{2.131671in}}%
\pgfpathlineto{\pgfqpoint{4.501864in}{2.120928in}}%
\pgfpathlineto{\pgfqpoint{4.487494in}{2.110369in}}%
\pgfpathlineto{\pgfqpoint{4.473140in}{2.099994in}}%
\pgfpathlineto{\pgfqpoint{4.465071in}{2.084893in}}%
\pgfpathlineto{\pgfqpoint{4.456997in}{2.069698in}}%
\pgfpathlineto{\pgfqpoint{4.448919in}{2.054410in}}%
\pgfpathlineto{\pgfqpoint{4.440836in}{2.039032in}}%
\pgfpathclose%
\pgfusepath{fill}%
\end{pgfscope}%
\begin{pgfscope}%
\pgfpathrectangle{\pgfqpoint{1.150000in}{0.150000in}}{\pgfqpoint{5.700000in}{5.700000in}}%
\pgfusepath{clip}%
\pgfsetbuttcap%
\pgfsetroundjoin%
\definecolor{currentfill}{rgb}{0.262138,0.242286,0.520837}%
\pgfsetfillcolor{currentfill}%
\pgfsetfillopacity{0.800000}%
\pgfsetlinewidth{0.000000pt}%
\definecolor{currentstroke}{rgb}{0.000000,0.000000,0.000000}%
\pgfsetstrokecolor{currentstroke}%
\pgfsetdash{}{0pt}%
\pgfpathmoveto{\pgfqpoint{4.196783in}{1.722554in}}%
\pgfpathlineto{\pgfqpoint{4.211004in}{1.729533in}}%
\pgfpathlineto{\pgfqpoint{4.225238in}{1.736693in}}%
\pgfpathlineto{\pgfqpoint{4.239485in}{1.744034in}}%
\pgfpathlineto{\pgfqpoint{4.253746in}{1.751556in}}%
\pgfpathlineto{\pgfqpoint{4.261892in}{1.767445in}}%
\pgfpathlineto{\pgfqpoint{4.270034in}{1.783309in}}%
\pgfpathlineto{\pgfqpoint{4.278173in}{1.799145in}}%
\pgfpathlineto{\pgfqpoint{4.286307in}{1.814949in}}%
\pgfpathlineto{\pgfqpoint{4.272043in}{1.806974in}}%
\pgfpathlineto{\pgfqpoint{4.257793in}{1.799181in}}%
\pgfpathlineto{\pgfqpoint{4.243557in}{1.791570in}}%
\pgfpathlineto{\pgfqpoint{4.229335in}{1.784140in}}%
\pgfpathlineto{\pgfqpoint{4.221203in}{1.768776in}}%
\pgfpathlineto{\pgfqpoint{4.213067in}{1.753388in}}%
\pgfpathlineto{\pgfqpoint{4.204927in}{1.737979in}}%
\pgfpathlineto{\pgfqpoint{4.196783in}{1.722554in}}%
\pgfpathclose%
\pgfusepath{fill}%
\end{pgfscope}%
\begin{pgfscope}%
\pgfpathrectangle{\pgfqpoint{1.150000in}{0.150000in}}{\pgfqpoint{5.700000in}{5.700000in}}%
\pgfusepath{clip}%
\pgfsetbuttcap%
\pgfsetroundjoin%
\definecolor{currentfill}{rgb}{0.274952,0.037752,0.364543}%
\pgfsetfillcolor{currentfill}%
\pgfsetfillopacity{0.800000}%
\pgfsetlinewidth{0.000000pt}%
\definecolor{currentstroke}{rgb}{0.000000,0.000000,0.000000}%
\pgfsetstrokecolor{currentstroke}%
\pgfsetdash{}{0pt}%
\pgfpathmoveto{\pgfqpoint{3.156299in}{1.320301in}}%
\pgfpathlineto{\pgfqpoint{3.170344in}{1.311430in}}%
\pgfpathlineto{\pgfqpoint{3.184390in}{1.302755in}}%
\pgfpathlineto{\pgfqpoint{3.198436in}{1.294276in}}%
\pgfpathlineto{\pgfqpoint{3.212484in}{1.285991in}}%
\pgfpathlineto{\pgfqpoint{3.221113in}{1.287899in}}%
\pgfpathlineto{\pgfqpoint{3.229727in}{1.290148in}}%
\pgfpathlineto{\pgfqpoint{3.238328in}{1.292729in}}%
\pgfpathlineto{\pgfqpoint{3.246914in}{1.295633in}}%
\pgfpathlineto{\pgfqpoint{3.232902in}{1.303123in}}%
\pgfpathlineto{\pgfqpoint{3.218892in}{1.310806in}}%
\pgfpathlineto{\pgfqpoint{3.204883in}{1.318685in}}%
\pgfpathlineto{\pgfqpoint{3.190875in}{1.326760in}}%
\pgfpathlineto{\pgfqpoint{3.182253in}{1.324639in}}%
\pgfpathlineto{\pgfqpoint{3.173617in}{1.322849in}}%
\pgfpathlineto{\pgfqpoint{3.164966in}{1.321401in}}%
\pgfpathlineto{\pgfqpoint{3.156299in}{1.320301in}}%
\pgfpathclose%
\pgfusepath{fill}%
\end{pgfscope}%
\begin{pgfscope}%
\pgfpathrectangle{\pgfqpoint{1.150000in}{0.150000in}}{\pgfqpoint{5.700000in}{5.700000in}}%
\pgfusepath{clip}%
\pgfsetbuttcap%
\pgfsetroundjoin%
\definecolor{currentfill}{rgb}{0.274952,0.037752,0.364543}%
\pgfsetfillcolor{currentfill}%
\pgfsetfillopacity{0.800000}%
\pgfsetlinewidth{0.000000pt}%
\definecolor{currentstroke}{rgb}{0.000000,0.000000,0.000000}%
\pgfsetstrokecolor{currentstroke}%
\pgfsetdash{}{0pt}%
\pgfpathmoveto{\pgfqpoint{3.684462in}{1.282185in}}%
\pgfpathlineto{\pgfqpoint{3.698514in}{1.281545in}}%
\pgfpathlineto{\pgfqpoint{3.712573in}{1.281085in}}%
\pgfpathlineto{\pgfqpoint{3.726640in}{1.280806in}}%
\pgfpathlineto{\pgfqpoint{3.740715in}{1.280708in}}%
\pgfpathlineto{\pgfqpoint{3.749014in}{1.291850in}}%
\pgfpathlineto{\pgfqpoint{3.757307in}{1.303156in}}%
\pgfpathlineto{\pgfqpoint{3.765593in}{1.314621in}}%
\pgfpathlineto{\pgfqpoint{3.773873in}{1.326238in}}%
\pgfpathlineto{\pgfqpoint{3.759811in}{1.325672in}}%
\pgfpathlineto{\pgfqpoint{3.745756in}{1.325286in}}%
\pgfpathlineto{\pgfqpoint{3.731710in}{1.325082in}}%
\pgfpathlineto{\pgfqpoint{3.717671in}{1.325059in}}%
\pgfpathlineto{\pgfqpoint{3.709379in}{1.314094in}}%
\pgfpathlineto{\pgfqpoint{3.701080in}{1.303289in}}%
\pgfpathlineto{\pgfqpoint{3.692774in}{1.292651in}}%
\pgfpathlineto{\pgfqpoint{3.684462in}{1.282185in}}%
\pgfpathclose%
\pgfusepath{fill}%
\end{pgfscope}%
\begin{pgfscope}%
\pgfpathrectangle{\pgfqpoint{1.150000in}{0.150000in}}{\pgfqpoint{5.700000in}{5.700000in}}%
\pgfusepath{clip}%
\pgfsetbuttcap%
\pgfsetroundjoin%
\definecolor{currentfill}{rgb}{0.124395,0.578002,0.548287}%
\pgfsetfillcolor{currentfill}%
\pgfsetfillopacity{0.800000}%
\pgfsetlinewidth{0.000000pt}%
\definecolor{currentstroke}{rgb}{0.000000,0.000000,0.000000}%
\pgfsetstrokecolor{currentstroke}%
\pgfsetdash{}{0pt}%
\pgfpathmoveto{\pgfqpoint{4.928991in}{2.684977in}}%
\pgfpathlineto{\pgfqpoint{4.943658in}{2.699656in}}%
\pgfpathlineto{\pgfqpoint{4.958345in}{2.714521in}}%
\pgfpathlineto{\pgfqpoint{4.973052in}{2.729575in}}%
\pgfpathlineto{\pgfqpoint{4.987779in}{2.744816in}}%
\pgfpathlineto{\pgfqpoint{4.995687in}{2.756657in}}%
\pgfpathlineto{\pgfqpoint{5.003587in}{2.768315in}}%
\pgfpathlineto{\pgfqpoint{5.011478in}{2.779790in}}%
\pgfpathlineto{\pgfqpoint{5.019362in}{2.791083in}}%
\pgfpathlineto{\pgfqpoint{5.004633in}{2.775788in}}%
\pgfpathlineto{\pgfqpoint{4.989925in}{2.760682in}}%
\pgfpathlineto{\pgfqpoint{4.975238in}{2.745763in}}%
\pgfpathlineto{\pgfqpoint{4.960570in}{2.731031in}}%
\pgfpathlineto{\pgfqpoint{4.952687in}{2.719778in}}%
\pgfpathlineto{\pgfqpoint{4.944796in}{2.708352in}}%
\pgfpathlineto{\pgfqpoint{4.936898in}{2.696751in}}%
\pgfpathlineto{\pgfqpoint{4.928991in}{2.684977in}}%
\pgfpathclose%
\pgfusepath{fill}%
\end{pgfscope}%
\begin{pgfscope}%
\pgfpathrectangle{\pgfqpoint{1.150000in}{0.150000in}}{\pgfqpoint{5.700000in}{5.700000in}}%
\pgfusepath{clip}%
\pgfsetbuttcap%
\pgfsetroundjoin%
\definecolor{currentfill}{rgb}{0.278791,0.062145,0.386592}%
\pgfsetfillcolor{currentfill}%
\pgfsetfillopacity{0.800000}%
\pgfsetlinewidth{0.000000pt}%
\definecolor{currentstroke}{rgb}{0.000000,0.000000,0.000000}%
\pgfsetstrokecolor{currentstroke}%
\pgfsetdash{}{0pt}%
\pgfpathmoveto{\pgfqpoint{3.773873in}{1.326238in}}%
\pgfpathlineto{\pgfqpoint{3.787944in}{1.326985in}}%
\pgfpathlineto{\pgfqpoint{3.802023in}{1.327912in}}%
\pgfpathlineto{\pgfqpoint{3.816110in}{1.329019in}}%
\pgfpathlineto{\pgfqpoint{3.830207in}{1.330306in}}%
\pgfpathlineto{\pgfqpoint{3.838471in}{1.342715in}}%
\pgfpathlineto{\pgfqpoint{3.846729in}{1.355255in}}%
\pgfpathlineto{\pgfqpoint{3.854981in}{1.367921in}}%
\pgfpathlineto{\pgfqpoint{3.863228in}{1.380706in}}%
\pgfpathlineto{\pgfqpoint{3.849140in}{1.378785in}}%
\pgfpathlineto{\pgfqpoint{3.835062in}{1.377043in}}%
\pgfpathlineto{\pgfqpoint{3.820992in}{1.375482in}}%
\pgfpathlineto{\pgfqpoint{3.806931in}{1.374102in}}%
\pgfpathlineto{\pgfqpoint{3.798676in}{1.361939in}}%
\pgfpathlineto{\pgfqpoint{3.790414in}{1.349903in}}%
\pgfpathlineto{\pgfqpoint{3.782147in}{1.338001in}}%
\pgfpathlineto{\pgfqpoint{3.773873in}{1.326238in}}%
\pgfpathclose%
\pgfusepath{fill}%
\end{pgfscope}%
\begin{pgfscope}%
\pgfpathrectangle{\pgfqpoint{1.150000in}{0.150000in}}{\pgfqpoint{5.700000in}{5.700000in}}%
\pgfusepath{clip}%
\pgfsetbuttcap%
\pgfsetroundjoin%
\definecolor{currentfill}{rgb}{0.183898,0.422383,0.556944}%
\pgfsetfillcolor{currentfill}%
\pgfsetfillopacity{0.800000}%
\pgfsetlinewidth{0.000000pt}%
\definecolor{currentstroke}{rgb}{0.000000,0.000000,0.000000}%
\pgfsetstrokecolor{currentstroke}%
\pgfsetdash{}{0pt}%
\pgfpathmoveto{\pgfqpoint{4.562897in}{2.203158in}}%
\pgfpathlineto{\pgfqpoint{4.577321in}{2.214560in}}%
\pgfpathlineto{\pgfqpoint{4.591763in}{2.226146in}}%
\pgfpathlineto{\pgfqpoint{4.606221in}{2.237917in}}%
\pgfpathlineto{\pgfqpoint{4.620696in}{2.249871in}}%
\pgfpathlineto{\pgfqpoint{4.628750in}{2.264999in}}%
\pgfpathlineto{\pgfqpoint{4.636798in}{2.279999in}}%
\pgfpathlineto{\pgfqpoint{4.644841in}{2.294870in}}%
\pgfpathlineto{\pgfqpoint{4.652878in}{2.309609in}}%
\pgfpathlineto{\pgfqpoint{4.638397in}{2.297394in}}%
\pgfpathlineto{\pgfqpoint{4.623934in}{2.285364in}}%
\pgfpathlineto{\pgfqpoint{4.609488in}{2.273519in}}%
\pgfpathlineto{\pgfqpoint{4.595060in}{2.261858in}}%
\pgfpathlineto{\pgfqpoint{4.587027in}{2.247365in}}%
\pgfpathlineto{\pgfqpoint{4.578989in}{2.232749in}}%
\pgfpathlineto{\pgfqpoint{4.570946in}{2.218013in}}%
\pgfpathlineto{\pgfqpoint{4.562897in}{2.203158in}}%
\pgfpathclose%
\pgfusepath{fill}%
\end{pgfscope}%
\begin{pgfscope}%
\pgfpathrectangle{\pgfqpoint{1.150000in}{0.150000in}}{\pgfqpoint{5.700000in}{5.700000in}}%
\pgfusepath{clip}%
\pgfsetbuttcap%
\pgfsetroundjoin%
\definecolor{currentfill}{rgb}{0.162142,0.474838,0.558140}%
\pgfsetfillcolor{currentfill}%
\pgfsetfillopacity{0.800000}%
\pgfsetlinewidth{0.000000pt}%
\definecolor{currentstroke}{rgb}{0.000000,0.000000,0.000000}%
\pgfsetstrokecolor{currentstroke}%
\pgfsetdash{}{0pt}%
\pgfpathmoveto{\pgfqpoint{4.684971in}{2.367211in}}%
\pgfpathlineto{\pgfqpoint{4.699474in}{2.379837in}}%
\pgfpathlineto{\pgfqpoint{4.713995in}{2.392648in}}%
\pgfpathlineto{\pgfqpoint{4.728534in}{2.405645in}}%
\pgfpathlineto{\pgfqpoint{4.743092in}{2.418827in}}%
\pgfpathlineto{\pgfqpoint{4.751105in}{2.433085in}}%
\pgfpathlineto{\pgfqpoint{4.759112in}{2.447192in}}%
\pgfpathlineto{\pgfqpoint{4.767113in}{2.461146in}}%
\pgfpathlineto{\pgfqpoint{4.775107in}{2.474946in}}%
\pgfpathlineto{\pgfqpoint{4.760545in}{2.461572in}}%
\pgfpathlineto{\pgfqpoint{4.746002in}{2.448383in}}%
\pgfpathlineto{\pgfqpoint{4.731477in}{2.435380in}}%
\pgfpathlineto{\pgfqpoint{4.716970in}{2.422563in}}%
\pgfpathlineto{\pgfqpoint{4.708979in}{2.408941in}}%
\pgfpathlineto{\pgfqpoint{4.700982in}{2.395175in}}%
\pgfpathlineto{\pgfqpoint{4.692980in}{2.381264in}}%
\pgfpathlineto{\pgfqpoint{4.684971in}{2.367211in}}%
\pgfpathclose%
\pgfusepath{fill}%
\end{pgfscope}%
\begin{pgfscope}%
\pgfpathrectangle{\pgfqpoint{1.150000in}{0.150000in}}{\pgfqpoint{5.700000in}{5.700000in}}%
\pgfusepath{clip}%
\pgfsetbuttcap%
\pgfsetroundjoin%
\definecolor{currentfill}{rgb}{0.141935,0.526453,0.555991}%
\pgfsetfillcolor{currentfill}%
\pgfsetfillopacity{0.800000}%
\pgfsetlinewidth{0.000000pt}%
\definecolor{currentstroke}{rgb}{0.000000,0.000000,0.000000}%
\pgfsetstrokecolor{currentstroke}%
\pgfsetdash{}{0pt}%
\pgfpathmoveto{\pgfqpoint{4.807021in}{2.528582in}}%
\pgfpathlineto{\pgfqpoint{4.821605in}{2.542301in}}%
\pgfpathlineto{\pgfqpoint{4.836208in}{2.556206in}}%
\pgfpathlineto{\pgfqpoint{4.850831in}{2.570298in}}%
\pgfpathlineto{\pgfqpoint{4.865473in}{2.584577in}}%
\pgfpathlineto{\pgfqpoint{4.873438in}{2.597725in}}%
\pgfpathlineto{\pgfqpoint{4.881396in}{2.610704in}}%
\pgfpathlineto{\pgfqpoint{4.889347in}{2.623513in}}%
\pgfpathlineto{\pgfqpoint{4.897290in}{2.636150in}}%
\pgfpathlineto{\pgfqpoint{4.882645in}{2.621748in}}%
\pgfpathlineto{\pgfqpoint{4.868020in}{2.607532in}}%
\pgfpathlineto{\pgfqpoint{4.853414in}{2.593504in}}%
\pgfpathlineto{\pgfqpoint{4.838827in}{2.579662in}}%
\pgfpathlineto{\pgfqpoint{4.830885in}{2.567135in}}%
\pgfpathlineto{\pgfqpoint{4.822937in}{2.554445in}}%
\pgfpathlineto{\pgfqpoint{4.814982in}{2.541594in}}%
\pgfpathlineto{\pgfqpoint{4.807021in}{2.528582in}}%
\pgfpathclose%
\pgfusepath{fill}%
\end{pgfscope}%
\begin{pgfscope}%
\pgfpathrectangle{\pgfqpoint{1.150000in}{0.150000in}}{\pgfqpoint{5.700000in}{5.700000in}}%
\pgfusepath{clip}%
\pgfsetbuttcap%
\pgfsetroundjoin%
\definecolor{currentfill}{rgb}{0.271305,0.019942,0.347269}%
\pgfsetfillcolor{currentfill}%
\pgfsetfillopacity{0.800000}%
\pgfsetlinewidth{0.000000pt}%
\definecolor{currentstroke}{rgb}{0.000000,0.000000,0.000000}%
\pgfsetstrokecolor{currentstroke}%
\pgfsetdash{}{0pt}%
\pgfpathmoveto{\pgfqpoint{3.594932in}{1.249345in}}%
\pgfpathlineto{\pgfqpoint{3.608974in}{1.247282in}}%
\pgfpathlineto{\pgfqpoint{3.623022in}{1.245401in}}%
\pgfpathlineto{\pgfqpoint{3.637076in}{1.243702in}}%
\pgfpathlineto{\pgfqpoint{3.651137in}{1.242185in}}%
\pgfpathlineto{\pgfqpoint{3.659479in}{1.251892in}}%
\pgfpathlineto{\pgfqpoint{3.667814in}{1.261799in}}%
\pgfpathlineto{\pgfqpoint{3.676142in}{1.271899in}}%
\pgfpathlineto{\pgfqpoint{3.684462in}{1.282185in}}%
\pgfpathlineto{\pgfqpoint{3.670417in}{1.283007in}}%
\pgfpathlineto{\pgfqpoint{3.656378in}{1.284011in}}%
\pgfpathlineto{\pgfqpoint{3.642347in}{1.285197in}}%
\pgfpathlineto{\pgfqpoint{3.628323in}{1.286566in}}%
\pgfpathlineto{\pgfqpoint{3.619987in}{1.276963in}}%
\pgfpathlineto{\pgfqpoint{3.611643in}{1.267554in}}%
\pgfpathlineto{\pgfqpoint{3.603292in}{1.258345in}}%
\pgfpathlineto{\pgfqpoint{3.594932in}{1.249345in}}%
\pgfpathclose%
\pgfusepath{fill}%
\end{pgfscope}%
\begin{pgfscope}%
\pgfpathrectangle{\pgfqpoint{1.150000in}{0.150000in}}{\pgfqpoint{5.700000in}{5.700000in}}%
\pgfusepath{clip}%
\pgfsetbuttcap%
\pgfsetroundjoin%
\definecolor{currentfill}{rgb}{0.277134,0.185228,0.489898}%
\pgfsetfillcolor{currentfill}%
\pgfsetfillopacity{0.800000}%
\pgfsetlinewidth{0.000000pt}%
\definecolor{currentstroke}{rgb}{0.000000,0.000000,0.000000}%
\pgfsetstrokecolor{currentstroke}%
\pgfsetdash{}{0pt}%
\pgfpathmoveto{\pgfqpoint{4.074736in}{1.576963in}}%
\pgfpathlineto{\pgfqpoint{4.088906in}{1.582224in}}%
\pgfpathlineto{\pgfqpoint{4.103087in}{1.587666in}}%
\pgfpathlineto{\pgfqpoint{4.117281in}{1.593287in}}%
\pgfpathlineto{\pgfqpoint{4.131487in}{1.599089in}}%
\pgfpathlineto{\pgfqpoint{4.139663in}{1.614486in}}%
\pgfpathlineto{\pgfqpoint{4.147835in}{1.629902in}}%
\pgfpathlineto{\pgfqpoint{4.156003in}{1.645334in}}%
\pgfpathlineto{\pgfqpoint{4.164168in}{1.660776in}}%
\pgfpathlineto{\pgfqpoint{4.149961in}{1.654460in}}%
\pgfpathlineto{\pgfqpoint{4.135767in}{1.648324in}}%
\pgfpathlineto{\pgfqpoint{4.121586in}{1.642369in}}%
\pgfpathlineto{\pgfqpoint{4.107417in}{1.636595in}}%
\pgfpathlineto{\pgfqpoint{4.099253in}{1.621655in}}%
\pgfpathlineto{\pgfqpoint{4.091085in}{1.606733in}}%
\pgfpathlineto{\pgfqpoint{4.082913in}{1.591834in}}%
\pgfpathlineto{\pgfqpoint{4.074736in}{1.576963in}}%
\pgfpathclose%
\pgfusepath{fill}%
\end{pgfscope}%
\begin{pgfscope}%
\pgfpathrectangle{\pgfqpoint{1.150000in}{0.150000in}}{\pgfqpoint{5.700000in}{5.700000in}}%
\pgfusepath{clip}%
\pgfsetbuttcap%
\pgfsetroundjoin%
\definecolor{currentfill}{rgb}{0.440137,0.811138,0.340967}%
\pgfsetfillcolor{currentfill}%
\pgfsetfillopacity{0.800000}%
\pgfsetlinewidth{0.000000pt}%
\definecolor{currentstroke}{rgb}{0.000000,0.000000,0.000000}%
\pgfsetstrokecolor{currentstroke}%
\pgfsetdash{}{0pt}%
\pgfpathmoveto{\pgfqpoint{5.685744in}{3.493166in}}%
\pgfpathlineto{\pgfqpoint{5.700963in}{3.511747in}}%
\pgfpathlineto{\pgfqpoint{5.716207in}{3.530519in}}%
\pgfpathlineto{\pgfqpoint{5.731476in}{3.549482in}}%
\pgfpathlineto{\pgfqpoint{5.746770in}{3.568637in}}%
\pgfpathlineto{\pgfqpoint{5.754166in}{3.571231in}}%
\pgfpathlineto{\pgfqpoint{5.761550in}{3.573662in}}%
\pgfpathlineto{\pgfqpoint{5.768921in}{3.575933in}}%
\pgfpathlineto{\pgfqpoint{5.776281in}{3.578048in}}%
\pgfpathlineto{\pgfqpoint{5.761006in}{3.559244in}}%
\pgfpathlineto{\pgfqpoint{5.745757in}{3.540631in}}%
\pgfpathlineto{\pgfqpoint{5.730533in}{3.522208in}}%
\pgfpathlineto{\pgfqpoint{5.715334in}{3.503975in}}%
\pgfpathlineto{\pgfqpoint{5.707953in}{3.501497in}}%
\pgfpathlineto{\pgfqpoint{5.700562in}{3.498872in}}%
\pgfpathlineto{\pgfqpoint{5.693159in}{3.496096in}}%
\pgfpathlineto{\pgfqpoint{5.685744in}{3.493166in}}%
\pgfpathclose%
\pgfusepath{fill}%
\end{pgfscope}%
\begin{pgfscope}%
\pgfpathrectangle{\pgfqpoint{1.150000in}{0.150000in}}{\pgfqpoint{5.700000in}{5.700000in}}%
\pgfusepath{clip}%
\pgfsetbuttcap%
\pgfsetroundjoin%
\definecolor{currentfill}{rgb}{0.282327,0.094955,0.417331}%
\pgfsetfillcolor{currentfill}%
\pgfsetfillopacity{0.800000}%
\pgfsetlinewidth{0.000000pt}%
\definecolor{currentstroke}{rgb}{0.000000,0.000000,0.000000}%
\pgfsetstrokecolor{currentstroke}%
\pgfsetdash{}{0pt}%
\pgfpathmoveto{\pgfqpoint{3.863228in}{1.380706in}}%
\pgfpathlineto{\pgfqpoint{3.877325in}{1.382808in}}%
\pgfpathlineto{\pgfqpoint{3.891431in}{1.385089in}}%
\pgfpathlineto{\pgfqpoint{3.905547in}{1.387549in}}%
\pgfpathlineto{\pgfqpoint{3.919673in}{1.390189in}}%
\pgfpathlineto{\pgfqpoint{3.927908in}{1.403704in}}%
\pgfpathlineto{\pgfqpoint{3.936138in}{1.417319in}}%
\pgfpathlineto{\pgfqpoint{3.944363in}{1.431027in}}%
\pgfpathlineto{\pgfqpoint{3.952583in}{1.444824in}}%
\pgfpathlineto{\pgfqpoint{3.938462in}{1.441579in}}%
\pgfpathlineto{\pgfqpoint{3.924352in}{1.438514in}}%
\pgfpathlineto{\pgfqpoint{3.910252in}{1.435629in}}%
\pgfpathlineto{\pgfqpoint{3.896162in}{1.432924in}}%
\pgfpathlineto{\pgfqpoint{3.887936in}{1.419720in}}%
\pgfpathlineto{\pgfqpoint{3.879706in}{1.406611in}}%
\pgfpathlineto{\pgfqpoint{3.871470in}{1.393605in}}%
\pgfpathlineto{\pgfqpoint{3.863228in}{1.380706in}}%
\pgfpathclose%
\pgfusepath{fill}%
\end{pgfscope}%
\begin{pgfscope}%
\pgfpathrectangle{\pgfqpoint{1.150000in}{0.150000in}}{\pgfqpoint{5.700000in}{5.700000in}}%
\pgfusepath{clip}%
\pgfsetbuttcap%
\pgfsetroundjoin%
\definecolor{currentfill}{rgb}{0.180653,0.701402,0.488189}%
\pgfsetfillcolor{currentfill}%
\pgfsetfillopacity{0.800000}%
\pgfsetlinewidth{0.000000pt}%
\definecolor{currentstroke}{rgb}{0.000000,0.000000,0.000000}%
\pgfsetstrokecolor{currentstroke}%
\pgfsetdash{}{0pt}%
\pgfpathmoveto{\pgfqpoint{5.262847in}{3.075262in}}%
\pgfpathlineto{\pgfqpoint{5.277759in}{3.092112in}}%
\pgfpathlineto{\pgfqpoint{5.292693in}{3.109152in}}%
\pgfpathlineto{\pgfqpoint{5.307650in}{3.126382in}}%
\pgfpathlineto{\pgfqpoint{5.322629in}{3.143803in}}%
\pgfpathlineto{\pgfqpoint{5.330347in}{3.151737in}}%
\pgfpathlineto{\pgfqpoint{5.338056in}{3.159478in}}%
\pgfpathlineto{\pgfqpoint{5.345753in}{3.167025in}}%
\pgfpathlineto{\pgfqpoint{5.353441in}{3.174382in}}%
\pgfpathlineto{\pgfqpoint{5.338468in}{3.157089in}}%
\pgfpathlineto{\pgfqpoint{5.323518in}{3.139986in}}%
\pgfpathlineto{\pgfqpoint{5.308591in}{3.123072in}}%
\pgfpathlineto{\pgfqpoint{5.293686in}{3.106348in}}%
\pgfpathlineto{\pgfqpoint{5.285991in}{3.098851in}}%
\pgfpathlineto{\pgfqpoint{5.278286in}{3.091173in}}%
\pgfpathlineto{\pgfqpoint{5.270572in}{3.083310in}}%
\pgfpathlineto{\pgfqpoint{5.262847in}{3.075262in}}%
\pgfpathclose%
\pgfusepath{fill}%
\end{pgfscope}%
\begin{pgfscope}%
\pgfpathrectangle{\pgfqpoint{1.150000in}{0.150000in}}{\pgfqpoint{5.700000in}{5.700000in}}%
\pgfusepath{clip}%
\pgfsetbuttcap%
\pgfsetroundjoin%
\definecolor{currentfill}{rgb}{0.268510,0.009605,0.335427}%
\pgfsetfillcolor{currentfill}%
\pgfsetfillopacity{0.800000}%
\pgfsetlinewidth{0.000000pt}%
\definecolor{currentstroke}{rgb}{0.000000,0.000000,0.000000}%
\pgfsetstrokecolor{currentstroke}%
\pgfsetdash{}{0pt}%
\pgfpathmoveto{\pgfqpoint{3.359085in}{1.242632in}}%
\pgfpathlineto{\pgfqpoint{3.373119in}{1.236862in}}%
\pgfpathlineto{\pgfqpoint{3.387155in}{1.231280in}}%
\pgfpathlineto{\pgfqpoint{3.401195in}{1.225885in}}%
\pgfpathlineto{\pgfqpoint{3.415239in}{1.220677in}}%
\pgfpathlineto{\pgfqpoint{3.423721in}{1.226214in}}%
\pgfpathlineto{\pgfqpoint{3.432193in}{1.232033in}}%
\pgfpathlineto{\pgfqpoint{3.440653in}{1.238128in}}%
\pgfpathlineto{\pgfqpoint{3.449103in}{1.244490in}}%
\pgfpathlineto{\pgfqpoint{3.435086in}{1.248939in}}%
\pgfpathlineto{\pgfqpoint{3.421072in}{1.253575in}}%
\pgfpathlineto{\pgfqpoint{3.407063in}{1.258398in}}%
\pgfpathlineto{\pgfqpoint{3.393058in}{1.263409in}}%
\pgfpathlineto{\pgfqpoint{3.384582in}{1.257794in}}%
\pgfpathlineto{\pgfqpoint{3.376095in}{1.252454in}}%
\pgfpathlineto{\pgfqpoint{3.367596in}{1.247398in}}%
\pgfpathlineto{\pgfqpoint{3.359085in}{1.242632in}}%
\pgfpathclose%
\pgfusepath{fill}%
\end{pgfscope}%
\begin{pgfscope}%
\pgfpathrectangle{\pgfqpoint{1.150000in}{0.150000in}}{\pgfqpoint{5.700000in}{5.700000in}}%
\pgfusepath{clip}%
\pgfsetbuttcap%
\pgfsetroundjoin%
\definecolor{currentfill}{rgb}{0.141935,0.526453,0.555991}%
\pgfsetfillcolor{currentfill}%
\pgfsetfillopacity{0.800000}%
\pgfsetlinewidth{0.000000pt}%
\definecolor{currentstroke}{rgb}{0.000000,0.000000,0.000000}%
\pgfsetstrokecolor{currentstroke}%
\pgfsetdash{}{0pt}%
\pgfpathmoveto{\pgfqpoint{2.075369in}{2.656923in}}%
\pgfpathlineto{\pgfqpoint{2.089994in}{2.629262in}}%
\pgfpathlineto{\pgfqpoint{2.104600in}{2.601926in}}%
\pgfpathlineto{\pgfqpoint{2.119189in}{2.574912in}}%
\pgfpathlineto{\pgfqpoint{2.133761in}{2.548218in}}%
\pgfpathlineto{\pgfqpoint{2.143509in}{2.534100in}}%
\pgfpathlineto{\pgfqpoint{2.153223in}{2.520510in}}%
\pgfpathlineto{\pgfqpoint{2.162904in}{2.507438in}}%
\pgfpathlineto{\pgfqpoint{2.172552in}{2.494875in}}%
\pgfpathlineto{\pgfqpoint{2.158063in}{2.520680in}}%
\pgfpathlineto{\pgfqpoint{2.143557in}{2.546801in}}%
\pgfpathlineto{\pgfqpoint{2.129034in}{2.573242in}}%
\pgfpathlineto{\pgfqpoint{2.114494in}{2.600006in}}%
\pgfpathlineto{\pgfqpoint{2.104765in}{2.613445in}}%
\pgfpathlineto{\pgfqpoint{2.095001in}{2.627405in}}%
\pgfpathlineto{\pgfqpoint{2.085203in}{2.641894in}}%
\pgfpathlineto{\pgfqpoint{2.075369in}{2.656923in}}%
\pgfpathclose%
\pgfusepath{fill}%
\end{pgfscope}%
\begin{pgfscope}%
\pgfpathrectangle{\pgfqpoint{1.150000in}{0.150000in}}{\pgfqpoint{5.700000in}{5.700000in}}%
\pgfusepath{clip}%
\pgfsetbuttcap%
\pgfsetroundjoin%
\definecolor{currentfill}{rgb}{0.304148,0.764704,0.419943}%
\pgfsetfillcolor{currentfill}%
\pgfsetfillopacity{0.800000}%
\pgfsetlinewidth{0.000000pt}%
\definecolor{currentstroke}{rgb}{0.000000,0.000000,0.000000}%
\pgfsetstrokecolor{currentstroke}%
\pgfsetdash{}{0pt}%
\pgfpathmoveto{\pgfqpoint{5.474610in}{3.296227in}}%
\pgfpathlineto{\pgfqpoint{5.489680in}{3.314107in}}%
\pgfpathlineto{\pgfqpoint{5.504774in}{3.332178in}}%
\pgfpathlineto{\pgfqpoint{5.519891in}{3.350439in}}%
\pgfpathlineto{\pgfqpoint{5.535033in}{3.368892in}}%
\pgfpathlineto{\pgfqpoint{5.542603in}{3.374180in}}%
\pgfpathlineto{\pgfqpoint{5.550161in}{3.379281in}}%
\pgfpathlineto{\pgfqpoint{5.557707in}{3.384200in}}%
\pgfpathlineto{\pgfqpoint{5.565242in}{3.388939in}}%
\pgfpathlineto{\pgfqpoint{5.550113in}{3.370725in}}%
\pgfpathlineto{\pgfqpoint{5.535008in}{3.352701in}}%
\pgfpathlineto{\pgfqpoint{5.519927in}{3.334868in}}%
\pgfpathlineto{\pgfqpoint{5.504869in}{3.317225in}}%
\pgfpathlineto{\pgfqpoint{5.497321in}{3.312235in}}%
\pgfpathlineto{\pgfqpoint{5.489761in}{3.307074in}}%
\pgfpathlineto{\pgfqpoint{5.482191in}{3.301739in}}%
\pgfpathlineto{\pgfqpoint{5.474610in}{3.296227in}}%
\pgfpathclose%
\pgfusepath{fill}%
\end{pgfscope}%
\begin{pgfscope}%
\pgfpathrectangle{\pgfqpoint{1.150000in}{0.150000in}}{\pgfqpoint{5.700000in}{5.700000in}}%
\pgfusepath{clip}%
\pgfsetbuttcap%
\pgfsetroundjoin%
\definecolor{currentfill}{rgb}{0.506271,0.828786,0.300362}%
\pgfsetfillcolor{currentfill}%
\pgfsetfillopacity{0.800000}%
\pgfsetlinewidth{0.000000pt}%
\definecolor{currentstroke}{rgb}{0.000000,0.000000,0.000000}%
\pgfsetstrokecolor{currentstroke}%
\pgfsetdash{}{0pt}%
\pgfpathmoveto{\pgfqpoint{5.776281in}{3.578048in}}%
\pgfpathlineto{\pgfqpoint{5.791580in}{3.597043in}}%
\pgfpathlineto{\pgfqpoint{5.806905in}{3.616230in}}%
\pgfpathlineto{\pgfqpoint{5.822256in}{3.635609in}}%
\pgfpathlineto{\pgfqpoint{5.829587in}{3.637292in}}%
\pgfpathlineto{\pgfqpoint{5.836906in}{3.638820in}}%
\pgfpathlineto{\pgfqpoint{5.844213in}{3.640199in}}%
\pgfpathlineto{\pgfqpoint{5.851508in}{3.641433in}}%
\pgfpathlineto{\pgfqpoint{5.836180in}{3.622443in}}%
\pgfpathlineto{\pgfqpoint{5.820878in}{3.603644in}}%
\pgfpathlineto{\pgfqpoint{5.805601in}{3.585036in}}%
\pgfpathlineto{\pgfqpoint{5.798288in}{3.583501in}}%
\pgfpathlineto{\pgfqpoint{5.790964in}{3.581828in}}%
\pgfpathlineto{\pgfqpoint{5.783628in}{3.580012in}}%
\pgfpathlineto{\pgfqpoint{5.776281in}{3.578048in}}%
\pgfpathclose%
\pgfusepath{fill}%
\end{pgfscope}%
\begin{pgfscope}%
\pgfpathrectangle{\pgfqpoint{1.150000in}{0.150000in}}{\pgfqpoint{5.700000in}{5.700000in}}%
\pgfusepath{clip}%
\pgfsetbuttcap%
\pgfsetroundjoin%
\definecolor{currentfill}{rgb}{0.268510,0.009605,0.335427}%
\pgfsetfillcolor{currentfill}%
\pgfsetfillopacity{0.800000}%
\pgfsetlinewidth{0.000000pt}%
\definecolor{currentstroke}{rgb}{0.000000,0.000000,0.000000}%
\pgfsetstrokecolor{currentstroke}%
\pgfsetdash{}{0pt}%
\pgfpathmoveto{\pgfqpoint{3.505216in}{1.228551in}}%
\pgfpathlineto{\pgfqpoint{3.519256in}{1.225029in}}%
\pgfpathlineto{\pgfqpoint{3.533301in}{1.221690in}}%
\pgfpathlineto{\pgfqpoint{3.547351in}{1.218534in}}%
\pgfpathlineto{\pgfqpoint{3.561407in}{1.215562in}}%
\pgfpathlineto{\pgfqpoint{3.569801in}{1.223661in}}%
\pgfpathlineto{\pgfqpoint{3.578187in}{1.231996in}}%
\pgfpathlineto{\pgfqpoint{3.586564in}{1.240560in}}%
\pgfpathlineto{\pgfqpoint{3.594932in}{1.249345in}}%
\pgfpathlineto{\pgfqpoint{3.580896in}{1.251591in}}%
\pgfpathlineto{\pgfqpoint{3.566866in}{1.254020in}}%
\pgfpathlineto{\pgfqpoint{3.552842in}{1.256632in}}%
\pgfpathlineto{\pgfqpoint{3.538823in}{1.259429in}}%
\pgfpathlineto{\pgfqpoint{3.530435in}{1.251357in}}%
\pgfpathlineto{\pgfqpoint{3.522038in}{1.243516in}}%
\pgfpathlineto{\pgfqpoint{3.513632in}{1.235912in}}%
\pgfpathlineto{\pgfqpoint{3.505216in}{1.228551in}}%
\pgfpathclose%
\pgfusepath{fill}%
\end{pgfscope}%
\begin{pgfscope}%
\pgfpathrectangle{\pgfqpoint{1.150000in}{0.150000in}}{\pgfqpoint{5.700000in}{5.700000in}}%
\pgfusepath{clip}%
\pgfsetbuttcap%
\pgfsetroundjoin%
\definecolor{currentfill}{rgb}{0.273809,0.031497,0.358853}%
\pgfsetfillcolor{currentfill}%
\pgfsetfillopacity{0.800000}%
\pgfsetlinewidth{0.000000pt}%
\definecolor{currentstroke}{rgb}{0.000000,0.000000,0.000000}%
\pgfsetstrokecolor{currentstroke}%
\pgfsetdash{}{0pt}%
\pgfpathmoveto{\pgfqpoint{3.212484in}{1.285991in}}%
\pgfpathlineto{\pgfqpoint{3.226532in}{1.277900in}}%
\pgfpathlineto{\pgfqpoint{3.240582in}{1.270003in}}%
\pgfpathlineto{\pgfqpoint{3.254634in}{1.262298in}}%
\pgfpathlineto{\pgfqpoint{3.268687in}{1.254785in}}%
\pgfpathlineto{\pgfqpoint{3.277281in}{1.257501in}}%
\pgfpathlineto{\pgfqpoint{3.285861in}{1.260548in}}%
\pgfpathlineto{\pgfqpoint{3.294427in}{1.263919in}}%
\pgfpathlineto{\pgfqpoint{3.302981in}{1.267605in}}%
\pgfpathlineto{\pgfqpoint{3.288961in}{1.274324in}}%
\pgfpathlineto{\pgfqpoint{3.274943in}{1.281234in}}%
\pgfpathlineto{\pgfqpoint{3.260928in}{1.288337in}}%
\pgfpathlineto{\pgfqpoint{3.246914in}{1.295633in}}%
\pgfpathlineto{\pgfqpoint{3.238328in}{1.292729in}}%
\pgfpathlineto{\pgfqpoint{3.229727in}{1.290148in}}%
\pgfpathlineto{\pgfqpoint{3.221113in}{1.287899in}}%
\pgfpathlineto{\pgfqpoint{3.212484in}{1.285991in}}%
\pgfpathclose%
\pgfusepath{fill}%
\end{pgfscope}%
\begin{pgfscope}%
\pgfpathrectangle{\pgfqpoint{1.150000in}{0.150000in}}{\pgfqpoint{5.700000in}{5.700000in}}%
\pgfusepath{clip}%
\pgfsetbuttcap%
\pgfsetroundjoin%
\definecolor{currentfill}{rgb}{0.283072,0.130895,0.449241}%
\pgfsetfillcolor{currentfill}%
\pgfsetfillopacity{0.800000}%
\pgfsetlinewidth{0.000000pt}%
\definecolor{currentstroke}{rgb}{0.000000,0.000000,0.000000}%
\pgfsetstrokecolor{currentstroke}%
\pgfsetdash{}{0pt}%
\pgfpathmoveto{\pgfqpoint{3.952583in}{1.444824in}}%
\pgfpathlineto{\pgfqpoint{3.966713in}{1.448248in}}%
\pgfpathlineto{\pgfqpoint{3.980854in}{1.451852in}}%
\pgfpathlineto{\pgfqpoint{3.995006in}{1.455635in}}%
\pgfpathlineto{\pgfqpoint{4.009168in}{1.459597in}}%
\pgfpathlineto{\pgfqpoint{4.017380in}{1.474063in}}%
\pgfpathlineto{\pgfqpoint{4.025587in}{1.488598in}}%
\pgfpathlineto{\pgfqpoint{4.033789in}{1.503197in}}%
\pgfpathlineto{\pgfqpoint{4.041987in}{1.517855in}}%
\pgfpathlineto{\pgfqpoint{4.027827in}{1.513317in}}%
\pgfpathlineto{\pgfqpoint{4.013679in}{1.508959in}}%
\pgfpathlineto{\pgfqpoint{3.999541in}{1.504781in}}%
\pgfpathlineto{\pgfqpoint{3.985415in}{1.500782in}}%
\pgfpathlineto{\pgfqpoint{3.977214in}{1.486688in}}%
\pgfpathlineto{\pgfqpoint{3.969008in}{1.472660in}}%
\pgfpathlineto{\pgfqpoint{3.960798in}{1.458703in}}%
\pgfpathlineto{\pgfqpoint{3.952583in}{1.444824in}}%
\pgfpathclose%
\pgfusepath{fill}%
\end{pgfscope}%
\begin{pgfscope}%
\pgfpathrectangle{\pgfqpoint{1.150000in}{0.150000in}}{\pgfqpoint{5.700000in}{5.700000in}}%
\pgfusepath{clip}%
\pgfsetbuttcap%
\pgfsetroundjoin%
\definecolor{currentfill}{rgb}{0.137339,0.662252,0.515571}%
\pgfsetfillcolor{currentfill}%
\pgfsetfillopacity{0.800000}%
\pgfsetlinewidth{0.000000pt}%
\definecolor{currentstroke}{rgb}{0.000000,0.000000,0.000000}%
\pgfsetstrokecolor{currentstroke}%
\pgfsetdash{}{0pt}%
\pgfpathmoveto{\pgfqpoint{5.141243in}{2.937941in}}%
\pgfpathlineto{\pgfqpoint{5.156074in}{2.954178in}}%
\pgfpathlineto{\pgfqpoint{5.170928in}{2.970605in}}%
\pgfpathlineto{\pgfqpoint{5.185803in}{2.987221in}}%
\pgfpathlineto{\pgfqpoint{5.200700in}{3.004027in}}%
\pgfpathlineto{\pgfqpoint{5.208502in}{3.013609in}}%
\pgfpathlineto{\pgfqpoint{5.216295in}{3.022995in}}%
\pgfpathlineto{\pgfqpoint{5.224078in}{3.032186in}}%
\pgfpathlineto{\pgfqpoint{5.231851in}{3.041184in}}%
\pgfpathlineto{\pgfqpoint{5.216957in}{3.024433in}}%
\pgfpathlineto{\pgfqpoint{5.202085in}{3.007871in}}%
\pgfpathlineto{\pgfqpoint{5.187235in}{2.991498in}}%
\pgfpathlineto{\pgfqpoint{5.172406in}{2.975315in}}%
\pgfpathlineto{\pgfqpoint{5.164629in}{2.966250in}}%
\pgfpathlineto{\pgfqpoint{5.156843in}{2.957000in}}%
\pgfpathlineto{\pgfqpoint{5.149047in}{2.947564in}}%
\pgfpathlineto{\pgfqpoint{5.141243in}{2.937941in}}%
\pgfpathclose%
\pgfusepath{fill}%
\end{pgfscope}%
\begin{pgfscope}%
\pgfpathrectangle{\pgfqpoint{1.150000in}{0.150000in}}{\pgfqpoint{5.700000in}{5.700000in}}%
\pgfusepath{clip}%
\pgfsetbuttcap%
\pgfsetroundjoin%
\definecolor{currentfill}{rgb}{0.244972,0.287675,0.537260}%
\pgfsetfillcolor{currentfill}%
\pgfsetfillopacity{0.800000}%
\pgfsetlinewidth{0.000000pt}%
\definecolor{currentstroke}{rgb}{0.000000,0.000000,0.000000}%
\pgfsetstrokecolor{currentstroke}%
\pgfsetdash{}{0pt}%
\pgfpathmoveto{\pgfqpoint{4.286307in}{1.814949in}}%
\pgfpathlineto{\pgfqpoint{4.300585in}{1.823104in}}%
\pgfpathlineto{\pgfqpoint{4.314877in}{1.831440in}}%
\pgfpathlineto{\pgfqpoint{4.329183in}{1.839958in}}%
\pgfpathlineto{\pgfqpoint{4.343504in}{1.848657in}}%
\pgfpathlineto{\pgfqpoint{4.351638in}{1.864858in}}%
\pgfpathlineto{\pgfqpoint{4.359768in}{1.881009in}}%
\pgfpathlineto{\pgfqpoint{4.367894in}{1.897108in}}%
\pgfpathlineto{\pgfqpoint{4.376016in}{1.913151in}}%
\pgfpathlineto{\pgfqpoint{4.361690in}{1.904029in}}%
\pgfpathlineto{\pgfqpoint{4.347380in}{1.895089in}}%
\pgfpathlineto{\pgfqpoint{4.333084in}{1.886331in}}%
\pgfpathlineto{\pgfqpoint{4.318803in}{1.877755in}}%
\pgfpathlineto{\pgfqpoint{4.310685in}{1.862122in}}%
\pgfpathlineto{\pgfqpoint{4.302563in}{1.846441in}}%
\pgfpathlineto{\pgfqpoint{4.294437in}{1.830715in}}%
\pgfpathlineto{\pgfqpoint{4.286307in}{1.814949in}}%
\pgfpathclose%
\pgfusepath{fill}%
\end{pgfscope}%
\begin{pgfscope}%
\pgfpathrectangle{\pgfqpoint{1.150000in}{0.150000in}}{\pgfqpoint{5.700000in}{5.700000in}}%
\pgfusepath{clip}%
\pgfsetbuttcap%
\pgfsetroundjoin%
\definecolor{currentfill}{rgb}{0.216210,0.351535,0.550627}%
\pgfsetfillcolor{currentfill}%
\pgfsetfillopacity{0.800000}%
\pgfsetlinewidth{0.000000pt}%
\definecolor{currentstroke}{rgb}{0.000000,0.000000,0.000000}%
\pgfsetstrokecolor{currentstroke}%
\pgfsetdash{}{0pt}%
\pgfpathmoveto{\pgfqpoint{4.408461in}{1.976683in}}%
\pgfpathlineto{\pgfqpoint{4.422806in}{1.986377in}}%
\pgfpathlineto{\pgfqpoint{4.437166in}{1.996254in}}%
\pgfpathlineto{\pgfqpoint{4.451543in}{2.006313in}}%
\pgfpathlineto{\pgfqpoint{4.465935in}{2.016554in}}%
\pgfpathlineto{\pgfqpoint{4.474041in}{2.032631in}}%
\pgfpathlineto{\pgfqpoint{4.482142in}{2.048622in}}%
\pgfpathlineto{\pgfqpoint{4.490238in}{2.064525in}}%
\pgfpathlineto{\pgfqpoint{4.498331in}{2.080335in}}%
\pgfpathlineto{\pgfqpoint{4.483933in}{2.069735in}}%
\pgfpathlineto{\pgfqpoint{4.469552in}{2.059317in}}%
\pgfpathlineto{\pgfqpoint{4.455186in}{2.049083in}}%
\pgfpathlineto{\pgfqpoint{4.440836in}{2.039032in}}%
\pgfpathlineto{\pgfqpoint{4.432749in}{2.023567in}}%
\pgfpathlineto{\pgfqpoint{4.424657in}{2.008019in}}%
\pgfpathlineto{\pgfqpoint{4.416561in}{1.992390in}}%
\pgfpathlineto{\pgfqpoint{4.408461in}{1.976683in}}%
\pgfpathclose%
\pgfusepath{fill}%
\end{pgfscope}%
\begin{pgfscope}%
\pgfpathrectangle{\pgfqpoint{1.150000in}{0.150000in}}{\pgfqpoint{5.700000in}{5.700000in}}%
\pgfusepath{clip}%
\pgfsetbuttcap%
\pgfsetroundjoin%
\definecolor{currentfill}{rgb}{0.266580,0.228262,0.514349}%
\pgfsetfillcolor{currentfill}%
\pgfsetfillopacity{0.800000}%
\pgfsetlinewidth{0.000000pt}%
\definecolor{currentstroke}{rgb}{0.000000,0.000000,0.000000}%
\pgfsetstrokecolor{currentstroke}%
\pgfsetdash{}{0pt}%
\pgfpathmoveto{\pgfqpoint{4.164168in}{1.660776in}}%
\pgfpathlineto{\pgfqpoint{4.178387in}{1.667272in}}%
\pgfpathlineto{\pgfqpoint{4.192619in}{1.673948in}}%
\pgfpathlineto{\pgfqpoint{4.206864in}{1.680804in}}%
\pgfpathlineto{\pgfqpoint{4.221122in}{1.687841in}}%
\pgfpathlineto{\pgfqpoint{4.229284in}{1.703784in}}%
\pgfpathlineto{\pgfqpoint{4.237442in}{1.719721in}}%
\pgfpathlineto{\pgfqpoint{4.245596in}{1.735646in}}%
\pgfpathlineto{\pgfqpoint{4.253746in}{1.751556in}}%
\pgfpathlineto{\pgfqpoint{4.239485in}{1.744034in}}%
\pgfpathlineto{\pgfqpoint{4.225238in}{1.736693in}}%
\pgfpathlineto{\pgfqpoint{4.211004in}{1.729533in}}%
\pgfpathlineto{\pgfqpoint{4.196783in}{1.722554in}}%
\pgfpathlineto{\pgfqpoint{4.188636in}{1.707117in}}%
\pgfpathlineto{\pgfqpoint{4.180484in}{1.691672in}}%
\pgfpathlineto{\pgfqpoint{4.172328in}{1.676223in}}%
\pgfpathlineto{\pgfqpoint{4.164168in}{1.660776in}}%
\pgfpathclose%
\pgfusepath{fill}%
\end{pgfscope}%
\begin{pgfscope}%
\pgfpathrectangle{\pgfqpoint{1.150000in}{0.150000in}}{\pgfqpoint{5.700000in}{5.700000in}}%
\pgfusepath{clip}%
\pgfsetbuttcap%
\pgfsetroundjoin%
\definecolor{currentfill}{rgb}{0.134692,0.658636,0.517649}%
\pgfsetfillcolor{currentfill}%
\pgfsetfillopacity{0.800000}%
\pgfsetlinewidth{0.000000pt}%
\definecolor{currentstroke}{rgb}{0.000000,0.000000,0.000000}%
\pgfsetstrokecolor{currentstroke}%
\pgfsetdash{}{0pt}%
\pgfpathmoveto{\pgfqpoint{1.879214in}{3.074861in}}%
\pgfpathlineto{\pgfqpoint{1.894081in}{3.042468in}}%
\pgfpathlineto{\pgfqpoint{1.908924in}{3.010454in}}%
\pgfpathlineto{\pgfqpoint{1.923744in}{2.978816in}}%
\pgfpathlineto{\pgfqpoint{1.938542in}{2.947551in}}%
\pgfpathlineto{\pgfqpoint{1.948497in}{2.931966in}}%
\pgfpathlineto{\pgfqpoint{1.958415in}{2.916918in}}%
\pgfpathlineto{\pgfqpoint{1.968297in}{2.902396in}}%
\pgfpathlineto{\pgfqpoint{1.978144in}{2.888393in}}%
\pgfpathlineto{\pgfqpoint{1.963436in}{2.918773in}}%
\pgfpathlineto{\pgfqpoint{1.948706in}{2.949521in}}%
\pgfpathlineto{\pgfqpoint{1.933954in}{2.980643in}}%
\pgfpathlineto{\pgfqpoint{1.919179in}{3.012141in}}%
\pgfpathlineto{\pgfqpoint{1.909244in}{3.027017in}}%
\pgfpathlineto{\pgfqpoint{1.899271in}{3.042423in}}%
\pgfpathlineto{\pgfqpoint{1.889262in}{3.058368in}}%
\pgfpathlineto{\pgfqpoint{1.879214in}{3.074861in}}%
\pgfpathclose%
\pgfusepath{fill}%
\end{pgfscope}%
\begin{pgfscope}%
\pgfpathrectangle{\pgfqpoint{1.150000in}{0.150000in}}{\pgfqpoint{5.700000in}{5.700000in}}%
\pgfusepath{clip}%
\pgfsetbuttcap%
\pgfsetroundjoin%
\definecolor{currentfill}{rgb}{0.190631,0.407061,0.556089}%
\pgfsetfillcolor{currentfill}%
\pgfsetfillopacity{0.800000}%
\pgfsetlinewidth{0.000000pt}%
\definecolor{currentstroke}{rgb}{0.000000,0.000000,0.000000}%
\pgfsetstrokecolor{currentstroke}%
\pgfsetdash{}{0pt}%
\pgfpathmoveto{\pgfqpoint{4.530653in}{2.142597in}}%
\pgfpathlineto{\pgfqpoint{4.545072in}{2.153706in}}%
\pgfpathlineto{\pgfqpoint{4.559508in}{2.165000in}}%
\pgfpathlineto{\pgfqpoint{4.573961in}{2.176477in}}%
\pgfpathlineto{\pgfqpoint{4.588431in}{2.188138in}}%
\pgfpathlineto{\pgfqpoint{4.596505in}{2.203750in}}%
\pgfpathlineto{\pgfqpoint{4.604574in}{2.219245in}}%
\pgfpathlineto{\pgfqpoint{4.612638in}{2.234619in}}%
\pgfpathlineto{\pgfqpoint{4.620696in}{2.249871in}}%
\pgfpathlineto{\pgfqpoint{4.606221in}{2.237917in}}%
\pgfpathlineto{\pgfqpoint{4.591763in}{2.226146in}}%
\pgfpathlineto{\pgfqpoint{4.577321in}{2.214560in}}%
\pgfpathlineto{\pgfqpoint{4.562897in}{2.203158in}}%
\pgfpathlineto{\pgfqpoint{4.554844in}{2.188186in}}%
\pgfpathlineto{\pgfqpoint{4.546785in}{2.173100in}}%
\pgfpathlineto{\pgfqpoint{4.538721in}{2.157903in}}%
\pgfpathlineto{\pgfqpoint{4.530653in}{2.142597in}}%
\pgfpathclose%
\pgfusepath{fill}%
\end{pgfscope}%
\begin{pgfscope}%
\pgfpathrectangle{\pgfqpoint{1.150000in}{0.150000in}}{\pgfqpoint{5.700000in}{5.700000in}}%
\pgfusepath{clip}%
\pgfsetbuttcap%
\pgfsetroundjoin%
\definecolor{currentfill}{rgb}{0.119483,0.614817,0.537692}%
\pgfsetfillcolor{currentfill}%
\pgfsetfillopacity{0.800000}%
\pgfsetlinewidth{0.000000pt}%
\definecolor{currentstroke}{rgb}{0.000000,0.000000,0.000000}%
\pgfsetstrokecolor{currentstroke}%
\pgfsetdash{}{0pt}%
\pgfpathmoveto{\pgfqpoint{5.019362in}{2.791083in}}%
\pgfpathlineto{\pgfqpoint{5.034111in}{2.806566in}}%
\pgfpathlineto{\pgfqpoint{5.048880in}{2.822237in}}%
\pgfpathlineto{\pgfqpoint{5.063671in}{2.838097in}}%
\pgfpathlineto{\pgfqpoint{5.078483in}{2.854146in}}%
\pgfpathlineto{\pgfqpoint{5.086359in}{2.865287in}}%
\pgfpathlineto{\pgfqpoint{5.094226in}{2.876238in}}%
\pgfpathlineto{\pgfqpoint{5.102084in}{2.886997in}}%
\pgfpathlineto{\pgfqpoint{5.109934in}{2.897565in}}%
\pgfpathlineto{\pgfqpoint{5.095122in}{2.881499in}}%
\pgfpathlineto{\pgfqpoint{5.080331in}{2.865621in}}%
\pgfpathlineto{\pgfqpoint{5.065562in}{2.849932in}}%
\pgfpathlineto{\pgfqpoint{5.050813in}{2.834431in}}%
\pgfpathlineto{\pgfqpoint{5.042963in}{2.823867in}}%
\pgfpathlineto{\pgfqpoint{5.035104in}{2.813121in}}%
\pgfpathlineto{\pgfqpoint{5.027237in}{2.802193in}}%
\pgfpathlineto{\pgfqpoint{5.019362in}{2.791083in}}%
\pgfpathclose%
\pgfusepath{fill}%
\end{pgfscope}%
\begin{pgfscope}%
\pgfpathrectangle{\pgfqpoint{1.150000in}{0.150000in}}{\pgfqpoint{5.700000in}{5.700000in}}%
\pgfusepath{clip}%
\pgfsetbuttcap%
\pgfsetroundjoin%
\definecolor{currentfill}{rgb}{0.166617,0.463708,0.558119}%
\pgfsetfillcolor{currentfill}%
\pgfsetfillopacity{0.800000}%
\pgfsetlinewidth{0.000000pt}%
\definecolor{currentstroke}{rgb}{0.000000,0.000000,0.000000}%
\pgfsetstrokecolor{currentstroke}%
\pgfsetdash{}{0pt}%
\pgfpathmoveto{\pgfqpoint{4.652878in}{2.309609in}}%
\pgfpathlineto{\pgfqpoint{4.667376in}{2.322008in}}%
\pgfpathlineto{\pgfqpoint{4.681893in}{2.334593in}}%
\pgfpathlineto{\pgfqpoint{4.696427in}{2.347363in}}%
\pgfpathlineto{\pgfqpoint{4.710979in}{2.360318in}}%
\pgfpathlineto{\pgfqpoint{4.719016in}{2.375163in}}%
\pgfpathlineto{\pgfqpoint{4.727047in}{2.389864in}}%
\pgfpathlineto{\pgfqpoint{4.735072in}{2.404419in}}%
\pgfpathlineto{\pgfqpoint{4.743092in}{2.418827in}}%
\pgfpathlineto{\pgfqpoint{4.728534in}{2.405645in}}%
\pgfpathlineto{\pgfqpoint{4.713995in}{2.392648in}}%
\pgfpathlineto{\pgfqpoint{4.699474in}{2.379837in}}%
\pgfpathlineto{\pgfqpoint{4.684971in}{2.367211in}}%
\pgfpathlineto{\pgfqpoint{4.676956in}{2.353018in}}%
\pgfpathlineto{\pgfqpoint{4.668936in}{2.338685in}}%
\pgfpathlineto{\pgfqpoint{4.660910in}{2.324214in}}%
\pgfpathlineto{\pgfqpoint{4.652878in}{2.309609in}}%
\pgfpathclose%
\pgfusepath{fill}%
\end{pgfscope}%
\begin{pgfscope}%
\pgfpathrectangle{\pgfqpoint{1.150000in}{0.150000in}}{\pgfqpoint{5.700000in}{5.700000in}}%
\pgfusepath{clip}%
\pgfsetbuttcap%
\pgfsetroundjoin%
\definecolor{currentfill}{rgb}{0.280255,0.165693,0.476498}%
\pgfsetfillcolor{currentfill}%
\pgfsetfillopacity{0.800000}%
\pgfsetlinewidth{0.000000pt}%
\definecolor{currentstroke}{rgb}{0.000000,0.000000,0.000000}%
\pgfsetstrokecolor{currentstroke}%
\pgfsetdash{}{0pt}%
\pgfpathmoveto{\pgfqpoint{4.041987in}{1.517855in}}%
\pgfpathlineto{\pgfqpoint{4.056158in}{1.522572in}}%
\pgfpathlineto{\pgfqpoint{4.070341in}{1.527468in}}%
\pgfpathlineto{\pgfqpoint{4.084535in}{1.532544in}}%
\pgfpathlineto{\pgfqpoint{4.098741in}{1.537799in}}%
\pgfpathlineto{\pgfqpoint{4.106934in}{1.553067in}}%
\pgfpathlineto{\pgfqpoint{4.115122in}{1.568374in}}%
\pgfpathlineto{\pgfqpoint{4.123307in}{1.583717in}}%
\pgfpathlineto{\pgfqpoint{4.131487in}{1.599089in}}%
\pgfpathlineto{\pgfqpoint{4.117281in}{1.593287in}}%
\pgfpathlineto{\pgfqpoint{4.103087in}{1.587666in}}%
\pgfpathlineto{\pgfqpoint{4.088906in}{1.582224in}}%
\pgfpathlineto{\pgfqpoint{4.074736in}{1.576963in}}%
\pgfpathlineto{\pgfqpoint{4.066555in}{1.562124in}}%
\pgfpathlineto{\pgfqpoint{4.058370in}{1.547323in}}%
\pgfpathlineto{\pgfqpoint{4.050181in}{1.532565in}}%
\pgfpathlineto{\pgfqpoint{4.041987in}{1.517855in}}%
\pgfpathclose%
\pgfusepath{fill}%
\end{pgfscope}%
\begin{pgfscope}%
\pgfpathrectangle{\pgfqpoint{1.150000in}{0.150000in}}{\pgfqpoint{5.700000in}{5.700000in}}%
\pgfusepath{clip}%
\pgfsetbuttcap%
\pgfsetroundjoin%
\definecolor{currentfill}{rgb}{0.268510,0.009605,0.335427}%
\pgfsetfillcolor{currentfill}%
\pgfsetfillopacity{0.800000}%
\pgfsetlinewidth{0.000000pt}%
\definecolor{currentstroke}{rgb}{0.000000,0.000000,0.000000}%
\pgfsetstrokecolor{currentstroke}%
\pgfsetdash{}{0pt}%
\pgfpathmoveto{\pgfqpoint{3.415239in}{1.220677in}}%
\pgfpathlineto{\pgfqpoint{3.429286in}{1.215655in}}%
\pgfpathlineto{\pgfqpoint{3.443337in}{1.210819in}}%
\pgfpathlineto{\pgfqpoint{3.457393in}{1.206168in}}%
\pgfpathlineto{\pgfqpoint{3.471452in}{1.201702in}}%
\pgfpathlineto{\pgfqpoint{3.479908in}{1.208011in}}%
\pgfpathlineto{\pgfqpoint{3.488354in}{1.214594in}}%
\pgfpathlineto{\pgfqpoint{3.496790in}{1.221443in}}%
\pgfpathlineto{\pgfqpoint{3.505216in}{1.228551in}}%
\pgfpathlineto{\pgfqpoint{3.491181in}{1.232258in}}%
\pgfpathlineto{\pgfqpoint{3.477150in}{1.236150in}}%
\pgfpathlineto{\pgfqpoint{3.463124in}{1.240227in}}%
\pgfpathlineto{\pgfqpoint{3.449103in}{1.244490in}}%
\pgfpathlineto{\pgfqpoint{3.440653in}{1.238128in}}%
\pgfpathlineto{\pgfqpoint{3.432193in}{1.232033in}}%
\pgfpathlineto{\pgfqpoint{3.423721in}{1.226214in}}%
\pgfpathlineto{\pgfqpoint{3.415239in}{1.220677in}}%
\pgfpathclose%
\pgfusepath{fill}%
\end{pgfscope}%
\begin{pgfscope}%
\pgfpathrectangle{\pgfqpoint{1.150000in}{0.150000in}}{\pgfqpoint{5.700000in}{5.700000in}}%
\pgfusepath{clip}%
\pgfsetbuttcap%
\pgfsetroundjoin%
\definecolor{currentfill}{rgb}{0.127568,0.566949,0.550556}%
\pgfsetfillcolor{currentfill}%
\pgfsetfillopacity{0.800000}%
\pgfsetlinewidth{0.000000pt}%
\definecolor{currentstroke}{rgb}{0.000000,0.000000,0.000000}%
\pgfsetstrokecolor{currentstroke}%
\pgfsetdash{}{0pt}%
\pgfpathmoveto{\pgfqpoint{4.897290in}{2.636150in}}%
\pgfpathlineto{\pgfqpoint{4.911955in}{2.650740in}}%
\pgfpathlineto{\pgfqpoint{4.926640in}{2.665517in}}%
\pgfpathlineto{\pgfqpoint{4.941345in}{2.680482in}}%
\pgfpathlineto{\pgfqpoint{4.956070in}{2.695634in}}%
\pgfpathlineto{\pgfqpoint{4.964009in}{2.708202in}}%
\pgfpathlineto{\pgfqpoint{4.971940in}{2.720588in}}%
\pgfpathlineto{\pgfqpoint{4.979863in}{2.732793in}}%
\pgfpathlineto{\pgfqpoint{4.987779in}{2.744816in}}%
\pgfpathlineto{\pgfqpoint{4.973052in}{2.729575in}}%
\pgfpathlineto{\pgfqpoint{4.958345in}{2.714521in}}%
\pgfpathlineto{\pgfqpoint{4.943658in}{2.699656in}}%
\pgfpathlineto{\pgfqpoint{4.928991in}{2.684977in}}%
\pgfpathlineto{\pgfqpoint{4.921077in}{2.673030in}}%
\pgfpathlineto{\pgfqpoint{4.913156in}{2.660909in}}%
\pgfpathlineto{\pgfqpoint{4.905227in}{2.648616in}}%
\pgfpathlineto{\pgfqpoint{4.897290in}{2.636150in}}%
\pgfpathclose%
\pgfusepath{fill}%
\end{pgfscope}%
\begin{pgfscope}%
\pgfpathrectangle{\pgfqpoint{1.150000in}{0.150000in}}{\pgfqpoint{5.700000in}{5.700000in}}%
\pgfusepath{clip}%
\pgfsetbuttcap%
\pgfsetroundjoin%
\definecolor{currentfill}{rgb}{0.127568,0.566949,0.550556}%
\pgfsetfillcolor{currentfill}%
\pgfsetfillopacity{0.800000}%
\pgfsetlinewidth{0.000000pt}%
\definecolor{currentstroke}{rgb}{0.000000,0.000000,0.000000}%
\pgfsetstrokecolor{currentstroke}%
\pgfsetdash{}{0pt}%
\pgfpathmoveto{\pgfqpoint{2.016685in}{2.770886in}}%
\pgfpathlineto{\pgfqpoint{2.031385in}{2.741891in}}%
\pgfpathlineto{\pgfqpoint{2.046065in}{2.713234in}}%
\pgfpathlineto{\pgfqpoint{2.060727in}{2.684912in}}%
\pgfpathlineto{\pgfqpoint{2.075369in}{2.656923in}}%
\pgfpathlineto{\pgfqpoint{2.085203in}{2.641894in}}%
\pgfpathlineto{\pgfqpoint{2.095001in}{2.627405in}}%
\pgfpathlineto{\pgfqpoint{2.104765in}{2.613445in}}%
\pgfpathlineto{\pgfqpoint{2.114494in}{2.600006in}}%
\pgfpathlineto{\pgfqpoint{2.099937in}{2.627097in}}%
\pgfpathlineto{\pgfqpoint{2.085362in}{2.654516in}}%
\pgfpathlineto{\pgfqpoint{2.070768in}{2.682268in}}%
\pgfpathlineto{\pgfqpoint{2.056156in}{2.710355in}}%
\pgfpathlineto{\pgfqpoint{2.046342in}{2.724680in}}%
\pgfpathlineto{\pgfqpoint{2.036493in}{2.739537in}}%
\pgfpathlineto{\pgfqpoint{2.026607in}{2.754936in}}%
\pgfpathlineto{\pgfqpoint{2.016685in}{2.770886in}}%
\pgfpathclose%
\pgfusepath{fill}%
\end{pgfscope}%
\begin{pgfscope}%
\pgfpathrectangle{\pgfqpoint{1.150000in}{0.150000in}}{\pgfqpoint{5.700000in}{5.700000in}}%
\pgfusepath{clip}%
\pgfsetbuttcap%
\pgfsetroundjoin%
\definecolor{currentfill}{rgb}{0.146180,0.515413,0.556823}%
\pgfsetfillcolor{currentfill}%
\pgfsetfillopacity{0.800000}%
\pgfsetlinewidth{0.000000pt}%
\definecolor{currentstroke}{rgb}{0.000000,0.000000,0.000000}%
\pgfsetstrokecolor{currentstroke}%
\pgfsetdash{}{0pt}%
\pgfpathmoveto{\pgfqpoint{4.775107in}{2.474946in}}%
\pgfpathlineto{\pgfqpoint{4.789688in}{2.488507in}}%
\pgfpathlineto{\pgfqpoint{4.804288in}{2.502254in}}%
\pgfpathlineto{\pgfqpoint{4.818906in}{2.516188in}}%
\pgfpathlineto{\pgfqpoint{4.833544in}{2.530308in}}%
\pgfpathlineto{\pgfqpoint{4.841537in}{2.544124in}}%
\pgfpathlineto{\pgfqpoint{4.849522in}{2.557775in}}%
\pgfpathlineto{\pgfqpoint{4.857501in}{2.571260in}}%
\pgfpathlineto{\pgfqpoint{4.865473in}{2.584577in}}%
\pgfpathlineto{\pgfqpoint{4.850831in}{2.570298in}}%
\pgfpathlineto{\pgfqpoint{4.836208in}{2.556206in}}%
\pgfpathlineto{\pgfqpoint{4.821605in}{2.542301in}}%
\pgfpathlineto{\pgfqpoint{4.807021in}{2.528582in}}%
\pgfpathlineto{\pgfqpoint{4.799052in}{2.515410in}}%
\pgfpathlineto{\pgfqpoint{4.791077in}{2.502080in}}%
\pgfpathlineto{\pgfqpoint{4.783095in}{2.488591in}}%
\pgfpathlineto{\pgfqpoint{4.775107in}{2.474946in}}%
\pgfpathclose%
\pgfusepath{fill}%
\end{pgfscope}%
\begin{pgfscope}%
\pgfpathrectangle{\pgfqpoint{1.150000in}{0.150000in}}{\pgfqpoint{5.700000in}{5.700000in}}%
\pgfusepath{clip}%
\pgfsetbuttcap%
\pgfsetroundjoin%
\definecolor{currentfill}{rgb}{0.239374,0.735588,0.455688}%
\pgfsetfillcolor{currentfill}%
\pgfsetfillopacity{0.800000}%
\pgfsetlinewidth{0.000000pt}%
\definecolor{currentstroke}{rgb}{0.000000,0.000000,0.000000}%
\pgfsetstrokecolor{currentstroke}%
\pgfsetdash{}{0pt}%
\pgfpathmoveto{\pgfqpoint{5.353441in}{3.174382in}}%
\pgfpathlineto{\pgfqpoint{5.368437in}{3.191865in}}%
\pgfpathlineto{\pgfqpoint{5.383455in}{3.209539in}}%
\pgfpathlineto{\pgfqpoint{5.398498in}{3.227403in}}%
\pgfpathlineto{\pgfqpoint{5.413563in}{3.245459in}}%
\pgfpathlineto{\pgfqpoint{5.421232in}{3.252475in}}%
\pgfpathlineto{\pgfqpoint{5.428891in}{3.259296in}}%
\pgfpathlineto{\pgfqpoint{5.436538in}{3.265923in}}%
\pgfpathlineto{\pgfqpoint{5.444174in}{3.272358in}}%
\pgfpathlineto{\pgfqpoint{5.429117in}{3.254467in}}%
\pgfpathlineto{\pgfqpoint{5.414084in}{3.236768in}}%
\pgfpathlineto{\pgfqpoint{5.399073in}{3.219258in}}%
\pgfpathlineto{\pgfqpoint{5.384086in}{3.201939in}}%
\pgfpathlineto{\pgfqpoint{5.376440in}{3.195326in}}%
\pgfpathlineto{\pgfqpoint{5.368784in}{3.188530in}}%
\pgfpathlineto{\pgfqpoint{5.361118in}{3.181549in}}%
\pgfpathlineto{\pgfqpoint{5.353441in}{3.174382in}}%
\pgfpathclose%
\pgfusepath{fill}%
\end{pgfscope}%
\begin{pgfscope}%
\pgfpathrectangle{\pgfqpoint{1.150000in}{0.150000in}}{\pgfqpoint{5.700000in}{5.700000in}}%
\pgfusepath{clip}%
\pgfsetbuttcap%
\pgfsetroundjoin%
\definecolor{currentfill}{rgb}{0.277018,0.050344,0.375715}%
\pgfsetfillcolor{currentfill}%
\pgfsetfillopacity{0.800000}%
\pgfsetlinewidth{0.000000pt}%
\definecolor{currentstroke}{rgb}{0.000000,0.000000,0.000000}%
\pgfsetstrokecolor{currentstroke}%
\pgfsetdash{}{0pt}%
\pgfpathmoveto{\pgfqpoint{3.740715in}{1.280708in}}%
\pgfpathlineto{\pgfqpoint{3.754797in}{1.280790in}}%
\pgfpathlineto{\pgfqpoint{3.768887in}{1.281052in}}%
\pgfpathlineto{\pgfqpoint{3.782986in}{1.281493in}}%
\pgfpathlineto{\pgfqpoint{3.797092in}{1.282114in}}%
\pgfpathlineto{\pgfqpoint{3.805380in}{1.293933in}}%
\pgfpathlineto{\pgfqpoint{3.813662in}{1.305909in}}%
\pgfpathlineto{\pgfqpoint{3.821937in}{1.318036in}}%
\pgfpathlineto{\pgfqpoint{3.830207in}{1.330306in}}%
\pgfpathlineto{\pgfqpoint{3.816110in}{1.329019in}}%
\pgfpathlineto{\pgfqpoint{3.802023in}{1.327912in}}%
\pgfpathlineto{\pgfqpoint{3.787944in}{1.326985in}}%
\pgfpathlineto{\pgfqpoint{3.773873in}{1.326238in}}%
\pgfpathlineto{\pgfqpoint{3.765593in}{1.314621in}}%
\pgfpathlineto{\pgfqpoint{3.757307in}{1.303156in}}%
\pgfpathlineto{\pgfqpoint{3.749014in}{1.291850in}}%
\pgfpathlineto{\pgfqpoint{3.740715in}{1.280708in}}%
\pgfpathclose%
\pgfusepath{fill}%
\end{pgfscope}%
\begin{pgfscope}%
\pgfpathrectangle{\pgfqpoint{1.150000in}{0.150000in}}{\pgfqpoint{5.700000in}{5.700000in}}%
\pgfusepath{clip}%
\pgfsetbuttcap%
\pgfsetroundjoin%
\definecolor{currentfill}{rgb}{0.377779,0.791781,0.377939}%
\pgfsetfillcolor{currentfill}%
\pgfsetfillopacity{0.800000}%
\pgfsetlinewidth{0.000000pt}%
\definecolor{currentstroke}{rgb}{0.000000,0.000000,0.000000}%
\pgfsetstrokecolor{currentstroke}%
\pgfsetdash{}{0pt}%
\pgfpathmoveto{\pgfqpoint{5.565242in}{3.388939in}}%
\pgfpathlineto{\pgfqpoint{5.580395in}{3.407344in}}%
\pgfpathlineto{\pgfqpoint{5.595573in}{3.425940in}}%
\pgfpathlineto{\pgfqpoint{5.610775in}{3.444729in}}%
\pgfpathlineto{\pgfqpoint{5.626003in}{3.463709in}}%
\pgfpathlineto{\pgfqpoint{5.633512in}{3.468008in}}%
\pgfpathlineto{\pgfqpoint{5.641009in}{3.472124in}}%
\pgfpathlineto{\pgfqpoint{5.648494in}{3.476060in}}%
\pgfpathlineto{\pgfqpoint{5.655968in}{3.479820in}}%
\pgfpathlineto{\pgfqpoint{5.640756in}{3.461117in}}%
\pgfpathlineto{\pgfqpoint{5.625569in}{3.442606in}}%
\pgfpathlineto{\pgfqpoint{5.610406in}{3.424285in}}%
\pgfpathlineto{\pgfqpoint{5.595268in}{3.406155in}}%
\pgfpathlineto{\pgfqpoint{5.587778in}{3.402105in}}%
\pgfpathlineto{\pgfqpoint{5.580278in}{3.397888in}}%
\pgfpathlineto{\pgfqpoint{5.572766in}{3.393501in}}%
\pgfpathlineto{\pgfqpoint{5.565242in}{3.388939in}}%
\pgfpathclose%
\pgfusepath{fill}%
\end{pgfscope}%
\begin{pgfscope}%
\pgfpathrectangle{\pgfqpoint{1.150000in}{0.150000in}}{\pgfqpoint{5.700000in}{5.700000in}}%
\pgfusepath{clip}%
\pgfsetbuttcap%
\pgfsetroundjoin%
\definecolor{currentfill}{rgb}{0.272594,0.025563,0.353093}%
\pgfsetfillcolor{currentfill}%
\pgfsetfillopacity{0.800000}%
\pgfsetlinewidth{0.000000pt}%
\definecolor{currentstroke}{rgb}{0.000000,0.000000,0.000000}%
\pgfsetstrokecolor{currentstroke}%
\pgfsetdash{}{0pt}%
\pgfpathmoveto{\pgfqpoint{3.268687in}{1.254785in}}%
\pgfpathlineto{\pgfqpoint{3.282742in}{1.247463in}}%
\pgfpathlineto{\pgfqpoint{3.296799in}{1.240332in}}%
\pgfpathlineto{\pgfqpoint{3.310858in}{1.233390in}}%
\pgfpathlineto{\pgfqpoint{3.324920in}{1.226638in}}%
\pgfpathlineto{\pgfqpoint{3.333480in}{1.230161in}}%
\pgfpathlineto{\pgfqpoint{3.342028in}{1.234006in}}%
\pgfpathlineto{\pgfqpoint{3.350563in}{1.238166in}}%
\pgfpathlineto{\pgfqpoint{3.359085in}{1.242632in}}%
\pgfpathlineto{\pgfqpoint{3.345055in}{1.248591in}}%
\pgfpathlineto{\pgfqpoint{3.331028in}{1.254739in}}%
\pgfpathlineto{\pgfqpoint{3.317003in}{1.261077in}}%
\pgfpathlineto{\pgfqpoint{3.302981in}{1.267605in}}%
\pgfpathlineto{\pgfqpoint{3.294427in}{1.263919in}}%
\pgfpathlineto{\pgfqpoint{3.285861in}{1.260548in}}%
\pgfpathlineto{\pgfqpoint{3.277281in}{1.257501in}}%
\pgfpathlineto{\pgfqpoint{3.268687in}{1.254785in}}%
\pgfpathclose%
\pgfusepath{fill}%
\end{pgfscope}%
\begin{pgfscope}%
\pgfpathrectangle{\pgfqpoint{1.150000in}{0.150000in}}{\pgfqpoint{5.700000in}{5.700000in}}%
\pgfusepath{clip}%
\pgfsetbuttcap%
\pgfsetroundjoin%
\definecolor{currentfill}{rgb}{0.273809,0.031497,0.358853}%
\pgfsetfillcolor{currentfill}%
\pgfsetfillopacity{0.800000}%
\pgfsetlinewidth{0.000000pt}%
\definecolor{currentstroke}{rgb}{0.000000,0.000000,0.000000}%
\pgfsetstrokecolor{currentstroke}%
\pgfsetdash{}{0pt}%
\pgfpathmoveto{\pgfqpoint{3.651137in}{1.242185in}}%
\pgfpathlineto{\pgfqpoint{3.665204in}{1.240849in}}%
\pgfpathlineto{\pgfqpoint{3.679279in}{1.239693in}}%
\pgfpathlineto{\pgfqpoint{3.693360in}{1.238719in}}%
\pgfpathlineto{\pgfqpoint{3.707448in}{1.237924in}}%
\pgfpathlineto{\pgfqpoint{3.715776in}{1.248339in}}%
\pgfpathlineto{\pgfqpoint{3.724096in}{1.258946in}}%
\pgfpathlineto{\pgfqpoint{3.732409in}{1.269738in}}%
\pgfpathlineto{\pgfqpoint{3.740715in}{1.280708in}}%
\pgfpathlineto{\pgfqpoint{3.726640in}{1.280806in}}%
\pgfpathlineto{\pgfqpoint{3.712573in}{1.281085in}}%
\pgfpathlineto{\pgfqpoint{3.698514in}{1.281545in}}%
\pgfpathlineto{\pgfqpoint{3.684462in}{1.282185in}}%
\pgfpathlineto{\pgfqpoint{3.676142in}{1.271899in}}%
\pgfpathlineto{\pgfqpoint{3.667814in}{1.261799in}}%
\pgfpathlineto{\pgfqpoint{3.659479in}{1.251892in}}%
\pgfpathlineto{\pgfqpoint{3.651137in}{1.242185in}}%
\pgfpathclose%
\pgfusepath{fill}%
\end{pgfscope}%
\begin{pgfscope}%
\pgfpathrectangle{\pgfqpoint{1.150000in}{0.150000in}}{\pgfqpoint{5.700000in}{5.700000in}}%
\pgfusepath{clip}%
\pgfsetbuttcap%
\pgfsetroundjoin%
\definecolor{currentfill}{rgb}{0.280894,0.078907,0.402329}%
\pgfsetfillcolor{currentfill}%
\pgfsetfillopacity{0.800000}%
\pgfsetlinewidth{0.000000pt}%
\definecolor{currentstroke}{rgb}{0.000000,0.000000,0.000000}%
\pgfsetstrokecolor{currentstroke}%
\pgfsetdash{}{0pt}%
\pgfpathmoveto{\pgfqpoint{3.830207in}{1.330306in}}%
\pgfpathlineto{\pgfqpoint{3.844312in}{1.331773in}}%
\pgfpathlineto{\pgfqpoint{3.858426in}{1.333418in}}%
\pgfpathlineto{\pgfqpoint{3.872549in}{1.335243in}}%
\pgfpathlineto{\pgfqpoint{3.886682in}{1.337246in}}%
\pgfpathlineto{\pgfqpoint{3.894938in}{1.350302in}}%
\pgfpathlineto{\pgfqpoint{3.903188in}{1.363482in}}%
\pgfpathlineto{\pgfqpoint{3.911433in}{1.376780in}}%
\pgfpathlineto{\pgfqpoint{3.919673in}{1.390189in}}%
\pgfpathlineto{\pgfqpoint{3.905547in}{1.387549in}}%
\pgfpathlineto{\pgfqpoint{3.891431in}{1.385089in}}%
\pgfpathlineto{\pgfqpoint{3.877325in}{1.382808in}}%
\pgfpathlineto{\pgfqpoint{3.863228in}{1.380706in}}%
\pgfpathlineto{\pgfqpoint{3.854981in}{1.367921in}}%
\pgfpathlineto{\pgfqpoint{3.846729in}{1.355255in}}%
\pgfpathlineto{\pgfqpoint{3.838471in}{1.342715in}}%
\pgfpathlineto{\pgfqpoint{3.830207in}{1.330306in}}%
\pgfpathclose%
\pgfusepath{fill}%
\end{pgfscope}%
\begin{pgfscope}%
\pgfpathrectangle{\pgfqpoint{1.150000in}{0.150000in}}{\pgfqpoint{5.700000in}{5.700000in}}%
\pgfusepath{clip}%
\pgfsetbuttcap%
\pgfsetroundjoin%
\definecolor{currentfill}{rgb}{0.269944,0.014625,0.341379}%
\pgfsetfillcolor{currentfill}%
\pgfsetfillopacity{0.800000}%
\pgfsetlinewidth{0.000000pt}%
\definecolor{currentstroke}{rgb}{0.000000,0.000000,0.000000}%
\pgfsetstrokecolor{currentstroke}%
\pgfsetdash{}{0pt}%
\pgfpathmoveto{\pgfqpoint{3.561407in}{1.215562in}}%
\pgfpathlineto{\pgfqpoint{3.575468in}{1.212772in}}%
\pgfpathlineto{\pgfqpoint{3.589535in}{1.210164in}}%
\pgfpathlineto{\pgfqpoint{3.603608in}{1.207738in}}%
\pgfpathlineto{\pgfqpoint{3.617686in}{1.205493in}}%
\pgfpathlineto{\pgfqpoint{3.626061in}{1.214331in}}%
\pgfpathlineto{\pgfqpoint{3.634428in}{1.223397in}}%
\pgfpathlineto{\pgfqpoint{3.642786in}{1.232684in}}%
\pgfpathlineto{\pgfqpoint{3.651137in}{1.242185in}}%
\pgfpathlineto{\pgfqpoint{3.637076in}{1.243702in}}%
\pgfpathlineto{\pgfqpoint{3.623022in}{1.245401in}}%
\pgfpathlineto{\pgfqpoint{3.608974in}{1.247282in}}%
\pgfpathlineto{\pgfqpoint{3.594932in}{1.249345in}}%
\pgfpathlineto{\pgfqpoint{3.586564in}{1.240560in}}%
\pgfpathlineto{\pgfqpoint{3.578187in}{1.231996in}}%
\pgfpathlineto{\pgfqpoint{3.569801in}{1.223661in}}%
\pgfpathlineto{\pgfqpoint{3.561407in}{1.215562in}}%
\pgfpathclose%
\pgfusepath{fill}%
\end{pgfscope}%
\begin{pgfscope}%
\pgfpathrectangle{\pgfqpoint{1.150000in}{0.150000in}}{\pgfqpoint{5.700000in}{5.700000in}}%
\pgfusepath{clip}%
\pgfsetbuttcap%
\pgfsetroundjoin%
\definecolor{currentfill}{rgb}{0.283197,0.115680,0.436115}%
\pgfsetfillcolor{currentfill}%
\pgfsetfillopacity{0.800000}%
\pgfsetlinewidth{0.000000pt}%
\definecolor{currentstroke}{rgb}{0.000000,0.000000,0.000000}%
\pgfsetstrokecolor{currentstroke}%
\pgfsetdash{}{0pt}%
\pgfpathmoveto{\pgfqpoint{3.919673in}{1.390189in}}%
\pgfpathlineto{\pgfqpoint{3.933809in}{1.393008in}}%
\pgfpathlineto{\pgfqpoint{3.947955in}{1.396006in}}%
\pgfpathlineto{\pgfqpoint{3.962110in}{1.399182in}}%
\pgfpathlineto{\pgfqpoint{3.976277in}{1.402537in}}%
\pgfpathlineto{\pgfqpoint{3.984507in}{1.416670in}}%
\pgfpathlineto{\pgfqpoint{3.992732in}{1.430895in}}%
\pgfpathlineto{\pgfqpoint{4.000952in}{1.445206in}}%
\pgfpathlineto{\pgfqpoint{4.009168in}{1.459597in}}%
\pgfpathlineto{\pgfqpoint{3.995006in}{1.455635in}}%
\pgfpathlineto{\pgfqpoint{3.980854in}{1.451852in}}%
\pgfpathlineto{\pgfqpoint{3.966713in}{1.448248in}}%
\pgfpathlineto{\pgfqpoint{3.952583in}{1.444824in}}%
\pgfpathlineto{\pgfqpoint{3.944363in}{1.431027in}}%
\pgfpathlineto{\pgfqpoint{3.936138in}{1.417319in}}%
\pgfpathlineto{\pgfqpoint{3.927908in}{1.403704in}}%
\pgfpathlineto{\pgfqpoint{3.919673in}{1.390189in}}%
\pgfpathclose%
\pgfusepath{fill}%
\end{pgfscope}%
\begin{pgfscope}%
\pgfpathrectangle{\pgfqpoint{1.150000in}{0.150000in}}{\pgfqpoint{5.700000in}{5.700000in}}%
\pgfusepath{clip}%
\pgfsetbuttcap%
\pgfsetroundjoin%
\definecolor{currentfill}{rgb}{0.252194,0.269783,0.531579}%
\pgfsetfillcolor{currentfill}%
\pgfsetfillopacity{0.800000}%
\pgfsetlinewidth{0.000000pt}%
\definecolor{currentstroke}{rgb}{0.000000,0.000000,0.000000}%
\pgfsetstrokecolor{currentstroke}%
\pgfsetdash{}{0pt}%
\pgfpathmoveto{\pgfqpoint{4.253746in}{1.751556in}}%
\pgfpathlineto{\pgfqpoint{4.268021in}{1.759258in}}%
\pgfpathlineto{\pgfqpoint{4.282309in}{1.767140in}}%
\pgfpathlineto{\pgfqpoint{4.296612in}{1.775203in}}%
\pgfpathlineto{\pgfqpoint{4.310929in}{1.783447in}}%
\pgfpathlineto{\pgfqpoint{4.319079in}{1.799803in}}%
\pgfpathlineto{\pgfqpoint{4.327224in}{1.816126in}}%
\pgfpathlineto{\pgfqpoint{4.335366in}{1.832412in}}%
\pgfpathlineto{\pgfqpoint{4.343504in}{1.848657in}}%
\pgfpathlineto{\pgfqpoint{4.329183in}{1.839958in}}%
\pgfpathlineto{\pgfqpoint{4.314877in}{1.831440in}}%
\pgfpathlineto{\pgfqpoint{4.300585in}{1.823104in}}%
\pgfpathlineto{\pgfqpoint{4.286307in}{1.814949in}}%
\pgfpathlineto{\pgfqpoint{4.278173in}{1.799145in}}%
\pgfpathlineto{\pgfqpoint{4.270034in}{1.783309in}}%
\pgfpathlineto{\pgfqpoint{4.261892in}{1.767445in}}%
\pgfpathlineto{\pgfqpoint{4.253746in}{1.751556in}}%
\pgfpathclose%
\pgfusepath{fill}%
\end{pgfscope}%
\begin{pgfscope}%
\pgfpathrectangle{\pgfqpoint{1.150000in}{0.150000in}}{\pgfqpoint{5.700000in}{5.700000in}}%
\pgfusepath{clip}%
\pgfsetbuttcap%
\pgfsetroundjoin%
\definecolor{currentfill}{rgb}{0.175707,0.697900,0.491033}%
\pgfsetfillcolor{currentfill}%
\pgfsetfillopacity{0.800000}%
\pgfsetlinewidth{0.000000pt}%
\definecolor{currentstroke}{rgb}{0.000000,0.000000,0.000000}%
\pgfsetstrokecolor{currentstroke}%
\pgfsetdash{}{0pt}%
\pgfpathmoveto{\pgfqpoint{5.231851in}{3.041184in}}%
\pgfpathlineto{\pgfqpoint{5.246768in}{3.058125in}}%
\pgfpathlineto{\pgfqpoint{5.261707in}{3.075256in}}%
\pgfpathlineto{\pgfqpoint{5.276668in}{3.092577in}}%
\pgfpathlineto{\pgfqpoint{5.291652in}{3.110090in}}%
\pgfpathlineto{\pgfqpoint{5.299412in}{3.118817in}}%
\pgfpathlineto{\pgfqpoint{5.307161in}{3.127344in}}%
\pgfpathlineto{\pgfqpoint{5.314900in}{3.135672in}}%
\pgfpathlineto{\pgfqpoint{5.322629in}{3.143803in}}%
\pgfpathlineto{\pgfqpoint{5.307650in}{3.126382in}}%
\pgfpathlineto{\pgfqpoint{5.292693in}{3.109152in}}%
\pgfpathlineto{\pgfqpoint{5.277759in}{3.092112in}}%
\pgfpathlineto{\pgfqpoint{5.262847in}{3.075262in}}%
\pgfpathlineto{\pgfqpoint{5.255113in}{3.067027in}}%
\pgfpathlineto{\pgfqpoint{5.247369in}{3.058603in}}%
\pgfpathlineto{\pgfqpoint{5.239615in}{3.049989in}}%
\pgfpathlineto{\pgfqpoint{5.231851in}{3.041184in}}%
\pgfpathclose%
\pgfusepath{fill}%
\end{pgfscope}%
\begin{pgfscope}%
\pgfpathrectangle{\pgfqpoint{1.150000in}{0.150000in}}{\pgfqpoint{5.700000in}{5.700000in}}%
\pgfusepath{clip}%
\pgfsetbuttcap%
\pgfsetroundjoin%
\definecolor{currentfill}{rgb}{0.223925,0.334994,0.548053}%
\pgfsetfillcolor{currentfill}%
\pgfsetfillopacity{0.800000}%
\pgfsetlinewidth{0.000000pt}%
\definecolor{currentstroke}{rgb}{0.000000,0.000000,0.000000}%
\pgfsetstrokecolor{currentstroke}%
\pgfsetdash{}{0pt}%
\pgfpathmoveto{\pgfqpoint{4.376016in}{1.913151in}}%
\pgfpathlineto{\pgfqpoint{4.390356in}{1.922454in}}%
\pgfpathlineto{\pgfqpoint{4.404712in}{1.931940in}}%
\pgfpathlineto{\pgfqpoint{4.419083in}{1.941607in}}%
\pgfpathlineto{\pgfqpoint{4.433469in}{1.951456in}}%
\pgfpathlineto{\pgfqpoint{4.441592in}{1.967842in}}%
\pgfpathlineto{\pgfqpoint{4.449711in}{1.984156in}}%
\pgfpathlineto{\pgfqpoint{4.457825in}{2.000395in}}%
\pgfpathlineto{\pgfqpoint{4.465935in}{2.016554in}}%
\pgfpathlineto{\pgfqpoint{4.451543in}{2.006313in}}%
\pgfpathlineto{\pgfqpoint{4.437166in}{1.996254in}}%
\pgfpathlineto{\pgfqpoint{4.422806in}{1.986377in}}%
\pgfpathlineto{\pgfqpoint{4.408461in}{1.976683in}}%
\pgfpathlineto{\pgfqpoint{4.400356in}{1.960903in}}%
\pgfpathlineto{\pgfqpoint{4.392247in}{1.945052in}}%
\pgfpathlineto{\pgfqpoint{4.384133in}{1.929133in}}%
\pgfpathlineto{\pgfqpoint{4.376016in}{1.913151in}}%
\pgfpathclose%
\pgfusepath{fill}%
\end{pgfscope}%
\begin{pgfscope}%
\pgfpathrectangle{\pgfqpoint{1.150000in}{0.150000in}}{\pgfqpoint{5.700000in}{5.700000in}}%
\pgfusepath{clip}%
\pgfsetbuttcap%
\pgfsetroundjoin%
\definecolor{currentfill}{rgb}{0.271828,0.209303,0.504434}%
\pgfsetfillcolor{currentfill}%
\pgfsetfillopacity{0.800000}%
\pgfsetlinewidth{0.000000pt}%
\definecolor{currentstroke}{rgb}{0.000000,0.000000,0.000000}%
\pgfsetstrokecolor{currentstroke}%
\pgfsetdash{}{0pt}%
\pgfpathmoveto{\pgfqpoint{4.131487in}{1.599089in}}%
\pgfpathlineto{\pgfqpoint{4.145705in}{1.605070in}}%
\pgfpathlineto{\pgfqpoint{4.159936in}{1.611230in}}%
\pgfpathlineto{\pgfqpoint{4.174180in}{1.617570in}}%
\pgfpathlineto{\pgfqpoint{4.188436in}{1.624089in}}%
\pgfpathlineto{\pgfqpoint{4.196614in}{1.640014in}}%
\pgfpathlineto{\pgfqpoint{4.204787in}{1.655951in}}%
\pgfpathlineto{\pgfqpoint{4.212957in}{1.671895in}}%
\pgfpathlineto{\pgfqpoint{4.221122in}{1.687841in}}%
\pgfpathlineto{\pgfqpoint{4.206864in}{1.680804in}}%
\pgfpathlineto{\pgfqpoint{4.192619in}{1.673948in}}%
\pgfpathlineto{\pgfqpoint{4.178387in}{1.667272in}}%
\pgfpathlineto{\pgfqpoint{4.164168in}{1.660776in}}%
\pgfpathlineto{\pgfqpoint{4.156003in}{1.645334in}}%
\pgfpathlineto{\pgfqpoint{4.147835in}{1.629902in}}%
\pgfpathlineto{\pgfqpoint{4.139663in}{1.614486in}}%
\pgfpathlineto{\pgfqpoint{4.131487in}{1.599089in}}%
\pgfpathclose%
\pgfusepath{fill}%
\end{pgfscope}%
\begin{pgfscope}%
\pgfpathrectangle{\pgfqpoint{1.150000in}{0.150000in}}{\pgfqpoint{5.700000in}{5.700000in}}%
\pgfusepath{clip}%
\pgfsetbuttcap%
\pgfsetroundjoin%
\definecolor{currentfill}{rgb}{0.197636,0.391528,0.554969}%
\pgfsetfillcolor{currentfill}%
\pgfsetfillopacity{0.800000}%
\pgfsetlinewidth{0.000000pt}%
\definecolor{currentstroke}{rgb}{0.000000,0.000000,0.000000}%
\pgfsetstrokecolor{currentstroke}%
\pgfsetdash{}{0pt}%
\pgfpathmoveto{\pgfqpoint{4.498331in}{2.080335in}}%
\pgfpathlineto{\pgfqpoint{4.512745in}{2.091119in}}%
\pgfpathlineto{\pgfqpoint{4.527175in}{2.102085in}}%
\pgfpathlineto{\pgfqpoint{4.541622in}{2.113236in}}%
\pgfpathlineto{\pgfqpoint{4.556085in}{2.124569in}}%
\pgfpathlineto{\pgfqpoint{4.564179in}{2.140624in}}%
\pgfpathlineto{\pgfqpoint{4.572268in}{2.156572in}}%
\pgfpathlineto{\pgfqpoint{4.580352in}{2.172411in}}%
\pgfpathlineto{\pgfqpoint{4.588431in}{2.188138in}}%
\pgfpathlineto{\pgfqpoint{4.573961in}{2.176477in}}%
\pgfpathlineto{\pgfqpoint{4.559508in}{2.165000in}}%
\pgfpathlineto{\pgfqpoint{4.545072in}{2.153706in}}%
\pgfpathlineto{\pgfqpoint{4.530653in}{2.142597in}}%
\pgfpathlineto{\pgfqpoint{4.522579in}{2.127184in}}%
\pgfpathlineto{\pgfqpoint{4.514501in}{2.111668in}}%
\pgfpathlineto{\pgfqpoint{4.506418in}{2.096050in}}%
\pgfpathlineto{\pgfqpoint{4.498331in}{2.080335in}}%
\pgfpathclose%
\pgfusepath{fill}%
\end{pgfscope}%
\begin{pgfscope}%
\pgfpathrectangle{\pgfqpoint{1.150000in}{0.150000in}}{\pgfqpoint{5.700000in}{5.700000in}}%
\pgfusepath{clip}%
\pgfsetbuttcap%
\pgfsetroundjoin%
\definecolor{currentfill}{rgb}{0.458674,0.816363,0.329727}%
\pgfsetfillcolor{currentfill}%
\pgfsetfillopacity{0.800000}%
\pgfsetlinewidth{0.000000pt}%
\definecolor{currentstroke}{rgb}{0.000000,0.000000,0.000000}%
\pgfsetstrokecolor{currentstroke}%
\pgfsetdash{}{0pt}%
\pgfpathmoveto{\pgfqpoint{5.655968in}{3.479820in}}%
\pgfpathlineto{\pgfqpoint{5.671204in}{3.498714in}}%
\pgfpathlineto{\pgfqpoint{5.686466in}{3.517801in}}%
\pgfpathlineto{\pgfqpoint{5.701753in}{3.537080in}}%
\pgfpathlineto{\pgfqpoint{5.717065in}{3.556551in}}%
\pgfpathlineto{\pgfqpoint{5.724510in}{3.559837in}}%
\pgfpathlineto{\pgfqpoint{5.731942in}{3.562944in}}%
\pgfpathlineto{\pgfqpoint{5.739362in}{3.565876in}}%
\pgfpathlineto{\pgfqpoint{5.746770in}{3.568637in}}%
\pgfpathlineto{\pgfqpoint{5.731476in}{3.549482in}}%
\pgfpathlineto{\pgfqpoint{5.716207in}{3.530519in}}%
\pgfpathlineto{\pgfqpoint{5.700963in}{3.511747in}}%
\pgfpathlineto{\pgfqpoint{5.685744in}{3.493166in}}%
\pgfpathlineto{\pgfqpoint{5.678317in}{3.490076in}}%
\pgfpathlineto{\pgfqpoint{5.670879in}{3.486825in}}%
\pgfpathlineto{\pgfqpoint{5.663430in}{3.483407in}}%
\pgfpathlineto{\pgfqpoint{5.655968in}{3.479820in}}%
\pgfpathclose%
\pgfusepath{fill}%
\end{pgfscope}%
\begin{pgfscope}%
\pgfpathrectangle{\pgfqpoint{1.150000in}{0.150000in}}{\pgfqpoint{5.700000in}{5.700000in}}%
\pgfusepath{clip}%
\pgfsetbuttcap%
\pgfsetroundjoin%
\definecolor{currentfill}{rgb}{0.271305,0.019942,0.347269}%
\pgfsetfillcolor{currentfill}%
\pgfsetfillopacity{0.800000}%
\pgfsetlinewidth{0.000000pt}%
\definecolor{currentstroke}{rgb}{0.000000,0.000000,0.000000}%
\pgfsetstrokecolor{currentstroke}%
\pgfsetdash{}{0pt}%
\pgfpathmoveto{\pgfqpoint{3.324920in}{1.226638in}}%
\pgfpathlineto{\pgfqpoint{3.338984in}{1.220075in}}%
\pgfpathlineto{\pgfqpoint{3.353051in}{1.213700in}}%
\pgfpathlineto{\pgfqpoint{3.367120in}{1.207512in}}%
\pgfpathlineto{\pgfqpoint{3.381193in}{1.201512in}}%
\pgfpathlineto{\pgfqpoint{3.389722in}{1.205840in}}%
\pgfpathlineto{\pgfqpoint{3.398240in}{1.210482in}}%
\pgfpathlineto{\pgfqpoint{3.406745in}{1.215430in}}%
\pgfpathlineto{\pgfqpoint{3.415239in}{1.220677in}}%
\pgfpathlineto{\pgfqpoint{3.401195in}{1.225885in}}%
\pgfpathlineto{\pgfqpoint{3.387155in}{1.231280in}}%
\pgfpathlineto{\pgfqpoint{3.373119in}{1.236862in}}%
\pgfpathlineto{\pgfqpoint{3.359085in}{1.242632in}}%
\pgfpathlineto{\pgfqpoint{3.350563in}{1.238166in}}%
\pgfpathlineto{\pgfqpoint{3.342028in}{1.234006in}}%
\pgfpathlineto{\pgfqpoint{3.333480in}{1.230161in}}%
\pgfpathlineto{\pgfqpoint{3.324920in}{1.226638in}}%
\pgfpathclose%
\pgfusepath{fill}%
\end{pgfscope}%
\begin{pgfscope}%
\pgfpathrectangle{\pgfqpoint{1.150000in}{0.150000in}}{\pgfqpoint{5.700000in}{5.700000in}}%
\pgfusepath{clip}%
\pgfsetbuttcap%
\pgfsetroundjoin%
\definecolor{currentfill}{rgb}{0.304148,0.764704,0.419943}%
\pgfsetfillcolor{currentfill}%
\pgfsetfillopacity{0.800000}%
\pgfsetlinewidth{0.000000pt}%
\definecolor{currentstroke}{rgb}{0.000000,0.000000,0.000000}%
\pgfsetstrokecolor{currentstroke}%
\pgfsetdash{}{0pt}%
\pgfpathmoveto{\pgfqpoint{5.444174in}{3.272358in}}%
\pgfpathlineto{\pgfqpoint{5.459255in}{3.290439in}}%
\pgfpathlineto{\pgfqpoint{5.474359in}{3.308712in}}%
\pgfpathlineto{\pgfqpoint{5.489488in}{3.327176in}}%
\pgfpathlineto{\pgfqpoint{5.504641in}{3.345833in}}%
\pgfpathlineto{\pgfqpoint{5.512256in}{3.351889in}}%
\pgfpathlineto{\pgfqpoint{5.519860in}{3.357750in}}%
\pgfpathlineto{\pgfqpoint{5.527452in}{3.363417in}}%
\pgfpathlineto{\pgfqpoint{5.535033in}{3.368892in}}%
\pgfpathlineto{\pgfqpoint{5.519891in}{3.350439in}}%
\pgfpathlineto{\pgfqpoint{5.504774in}{3.332178in}}%
\pgfpathlineto{\pgfqpoint{5.489680in}{3.314107in}}%
\pgfpathlineto{\pgfqpoint{5.474610in}{3.296227in}}%
\pgfpathlineto{\pgfqpoint{5.467017in}{3.290535in}}%
\pgfpathlineto{\pgfqpoint{5.459414in}{3.284662in}}%
\pgfpathlineto{\pgfqpoint{5.451800in}{3.278603in}}%
\pgfpathlineto{\pgfqpoint{5.444174in}{3.272358in}}%
\pgfpathclose%
\pgfusepath{fill}%
\end{pgfscope}%
\begin{pgfscope}%
\pgfpathrectangle{\pgfqpoint{1.150000in}{0.150000in}}{\pgfqpoint{5.700000in}{5.700000in}}%
\pgfusepath{clip}%
\pgfsetbuttcap%
\pgfsetroundjoin%
\definecolor{currentfill}{rgb}{0.119512,0.607464,0.540218}%
\pgfsetfillcolor{currentfill}%
\pgfsetfillopacity{0.800000}%
\pgfsetlinewidth{0.000000pt}%
\definecolor{currentstroke}{rgb}{0.000000,0.000000,0.000000}%
\pgfsetstrokecolor{currentstroke}%
\pgfsetdash{}{0pt}%
\pgfpathmoveto{\pgfqpoint{1.957684in}{2.890319in}}%
\pgfpathlineto{\pgfqpoint{1.972465in}{2.859935in}}%
\pgfpathlineto{\pgfqpoint{1.987225in}{2.829905in}}%
\pgfpathlineto{\pgfqpoint{2.001965in}{2.800223in}}%
\pgfpathlineto{\pgfqpoint{2.016685in}{2.770886in}}%
\pgfpathlineto{\pgfqpoint{2.026607in}{2.754936in}}%
\pgfpathlineto{\pgfqpoint{2.036493in}{2.739537in}}%
\pgfpathlineto{\pgfqpoint{2.046342in}{2.724680in}}%
\pgfpathlineto{\pgfqpoint{2.056156in}{2.710355in}}%
\pgfpathlineto{\pgfqpoint{2.041525in}{2.738781in}}%
\pgfpathlineto{\pgfqpoint{2.026874in}{2.767550in}}%
\pgfpathlineto{\pgfqpoint{2.012204in}{2.796665in}}%
\pgfpathlineto{\pgfqpoint{1.997514in}{2.826128in}}%
\pgfpathlineto{\pgfqpoint{1.987612in}{2.841351in}}%
\pgfpathlineto{\pgfqpoint{1.977673in}{2.857117in}}%
\pgfpathlineto{\pgfqpoint{1.967698in}{2.873436in}}%
\pgfpathlineto{\pgfqpoint{1.957684in}{2.890319in}}%
\pgfpathclose%
\pgfusepath{fill}%
\end{pgfscope}%
\begin{pgfscope}%
\pgfpathrectangle{\pgfqpoint{1.150000in}{0.150000in}}{\pgfqpoint{5.700000in}{5.700000in}}%
\pgfusepath{clip}%
\pgfsetbuttcap%
\pgfsetroundjoin%
\definecolor{currentfill}{rgb}{0.269944,0.014625,0.341379}%
\pgfsetfillcolor{currentfill}%
\pgfsetfillopacity{0.800000}%
\pgfsetlinewidth{0.000000pt}%
\definecolor{currentstroke}{rgb}{0.000000,0.000000,0.000000}%
\pgfsetstrokecolor{currentstroke}%
\pgfsetdash{}{0pt}%
\pgfpathmoveto{\pgfqpoint{3.471452in}{1.201702in}}%
\pgfpathlineto{\pgfqpoint{3.485516in}{1.197421in}}%
\pgfpathlineto{\pgfqpoint{3.499584in}{1.193323in}}%
\pgfpathlineto{\pgfqpoint{3.513657in}{1.189408in}}%
\pgfpathlineto{\pgfqpoint{3.527735in}{1.185676in}}%
\pgfpathlineto{\pgfqpoint{3.536168in}{1.192756in}}%
\pgfpathlineto{\pgfqpoint{3.544590in}{1.200102in}}%
\pgfpathlineto{\pgfqpoint{3.553003in}{1.207707in}}%
\pgfpathlineto{\pgfqpoint{3.561407in}{1.215562in}}%
\pgfpathlineto{\pgfqpoint{3.547351in}{1.218534in}}%
\pgfpathlineto{\pgfqpoint{3.533301in}{1.221690in}}%
\pgfpathlineto{\pgfqpoint{3.519256in}{1.225029in}}%
\pgfpathlineto{\pgfqpoint{3.505216in}{1.228551in}}%
\pgfpathlineto{\pgfqpoint{3.496790in}{1.221443in}}%
\pgfpathlineto{\pgfqpoint{3.488354in}{1.214594in}}%
\pgfpathlineto{\pgfqpoint{3.479908in}{1.208011in}}%
\pgfpathlineto{\pgfqpoint{3.471452in}{1.201702in}}%
\pgfpathclose%
\pgfusepath{fill}%
\end{pgfscope}%
\begin{pgfscope}%
\pgfpathrectangle{\pgfqpoint{1.150000in}{0.150000in}}{\pgfqpoint{5.700000in}{5.700000in}}%
\pgfusepath{clip}%
\pgfsetbuttcap%
\pgfsetroundjoin%
\definecolor{currentfill}{rgb}{0.132268,0.655014,0.519661}%
\pgfsetfillcolor{currentfill}%
\pgfsetfillopacity{0.800000}%
\pgfsetlinewidth{0.000000pt}%
\definecolor{currentstroke}{rgb}{0.000000,0.000000,0.000000}%
\pgfsetstrokecolor{currentstroke}%
\pgfsetdash{}{0pt}%
\pgfpathmoveto{\pgfqpoint{5.109934in}{2.897565in}}%
\pgfpathlineto{\pgfqpoint{5.124767in}{2.913821in}}%
\pgfpathlineto{\pgfqpoint{5.139622in}{2.930265in}}%
\pgfpathlineto{\pgfqpoint{5.154499in}{2.946900in}}%
\pgfpathlineto{\pgfqpoint{5.169397in}{2.963724in}}%
\pgfpathlineto{\pgfqpoint{5.177237in}{2.974097in}}%
\pgfpathlineto{\pgfqpoint{5.185067in}{2.984272in}}%
\pgfpathlineto{\pgfqpoint{5.192889in}{2.994248in}}%
\pgfpathlineto{\pgfqpoint{5.200700in}{3.004027in}}%
\pgfpathlineto{\pgfqpoint{5.185803in}{2.987221in}}%
\pgfpathlineto{\pgfqpoint{5.170928in}{2.970605in}}%
\pgfpathlineto{\pgfqpoint{5.156074in}{2.954178in}}%
\pgfpathlineto{\pgfqpoint{5.141243in}{2.937941in}}%
\pgfpathlineto{\pgfqpoint{5.133429in}{2.928130in}}%
\pgfpathlineto{\pgfqpoint{5.125606in}{2.918131in}}%
\pgfpathlineto{\pgfqpoint{5.117774in}{2.907943in}}%
\pgfpathlineto{\pgfqpoint{5.109934in}{2.897565in}}%
\pgfpathclose%
\pgfusepath{fill}%
\end{pgfscope}%
\begin{pgfscope}%
\pgfpathrectangle{\pgfqpoint{1.150000in}{0.150000in}}{\pgfqpoint{5.700000in}{5.700000in}}%
\pgfusepath{clip}%
\pgfsetbuttcap%
\pgfsetroundjoin%
\definecolor{currentfill}{rgb}{0.171176,0.452530,0.557965}%
\pgfsetfillcolor{currentfill}%
\pgfsetfillopacity{0.800000}%
\pgfsetlinewidth{0.000000pt}%
\definecolor{currentstroke}{rgb}{0.000000,0.000000,0.000000}%
\pgfsetstrokecolor{currentstroke}%
\pgfsetdash{}{0pt}%
\pgfpathmoveto{\pgfqpoint{4.620696in}{2.249871in}}%
\pgfpathlineto{\pgfqpoint{4.635189in}{2.262011in}}%
\pgfpathlineto{\pgfqpoint{4.649700in}{2.274335in}}%
\pgfpathlineto{\pgfqpoint{4.664228in}{2.286844in}}%
\pgfpathlineto{\pgfqpoint{4.678775in}{2.299537in}}%
\pgfpathlineto{\pgfqpoint{4.686834in}{2.314938in}}%
\pgfpathlineto{\pgfqpoint{4.694888in}{2.330203in}}%
\pgfpathlineto{\pgfqpoint{4.702937in}{2.345331in}}%
\pgfpathlineto{\pgfqpoint{4.710979in}{2.360318in}}%
\pgfpathlineto{\pgfqpoint{4.696427in}{2.347363in}}%
\pgfpathlineto{\pgfqpoint{4.681893in}{2.334593in}}%
\pgfpathlineto{\pgfqpoint{4.667376in}{2.322008in}}%
\pgfpathlineto{\pgfqpoint{4.652878in}{2.309609in}}%
\pgfpathlineto{\pgfqpoint{4.644841in}{2.294870in}}%
\pgfpathlineto{\pgfqpoint{4.636798in}{2.279999in}}%
\pgfpathlineto{\pgfqpoint{4.628750in}{2.264999in}}%
\pgfpathlineto{\pgfqpoint{4.620696in}{2.249871in}}%
\pgfpathclose%
\pgfusepath{fill}%
\end{pgfscope}%
\begin{pgfscope}%
\pgfpathrectangle{\pgfqpoint{1.150000in}{0.150000in}}{\pgfqpoint{5.700000in}{5.700000in}}%
\pgfusepath{clip}%
\pgfsetbuttcap%
\pgfsetroundjoin%
\definecolor{currentfill}{rgb}{0.281887,0.150881,0.465405}%
\pgfsetfillcolor{currentfill}%
\pgfsetfillopacity{0.800000}%
\pgfsetlinewidth{0.000000pt}%
\definecolor{currentstroke}{rgb}{0.000000,0.000000,0.000000}%
\pgfsetstrokecolor{currentstroke}%
\pgfsetdash{}{0pt}%
\pgfpathmoveto{\pgfqpoint{4.009168in}{1.459597in}}%
\pgfpathlineto{\pgfqpoint{4.023342in}{1.463738in}}%
\pgfpathlineto{\pgfqpoint{4.037526in}{1.468058in}}%
\pgfpathlineto{\pgfqpoint{4.051722in}{1.472556in}}%
\pgfpathlineto{\pgfqpoint{4.065930in}{1.477233in}}%
\pgfpathlineto{\pgfqpoint{4.074139in}{1.492288in}}%
\pgfpathlineto{\pgfqpoint{4.082344in}{1.507405in}}%
\pgfpathlineto{\pgfqpoint{4.090545in}{1.522577in}}%
\pgfpathlineto{\pgfqpoint{4.098741in}{1.537799in}}%
\pgfpathlineto{\pgfqpoint{4.084535in}{1.532544in}}%
\pgfpathlineto{\pgfqpoint{4.070341in}{1.527468in}}%
\pgfpathlineto{\pgfqpoint{4.056158in}{1.522572in}}%
\pgfpathlineto{\pgfqpoint{4.041987in}{1.517855in}}%
\pgfpathlineto{\pgfqpoint{4.033789in}{1.503197in}}%
\pgfpathlineto{\pgfqpoint{4.025587in}{1.488598in}}%
\pgfpathlineto{\pgfqpoint{4.017380in}{1.474063in}}%
\pgfpathlineto{\pgfqpoint{4.009168in}{1.459597in}}%
\pgfpathclose%
\pgfusepath{fill}%
\end{pgfscope}%
\begin{pgfscope}%
\pgfpathrectangle{\pgfqpoint{1.150000in}{0.150000in}}{\pgfqpoint{5.700000in}{5.700000in}}%
\pgfusepath{clip}%
\pgfsetbuttcap%
\pgfsetroundjoin%
\definecolor{currentfill}{rgb}{0.150476,0.504369,0.557430}%
\pgfsetfillcolor{currentfill}%
\pgfsetfillopacity{0.800000}%
\pgfsetlinewidth{0.000000pt}%
\definecolor{currentstroke}{rgb}{0.000000,0.000000,0.000000}%
\pgfsetstrokecolor{currentstroke}%
\pgfsetdash{}{0pt}%
\pgfpathmoveto{\pgfqpoint{4.743092in}{2.418827in}}%
\pgfpathlineto{\pgfqpoint{4.757668in}{2.432195in}}%
\pgfpathlineto{\pgfqpoint{4.772263in}{2.445749in}}%
\pgfpathlineto{\pgfqpoint{4.786877in}{2.459489in}}%
\pgfpathlineto{\pgfqpoint{4.801510in}{2.473415in}}%
\pgfpathlineto{\pgfqpoint{4.809528in}{2.487879in}}%
\pgfpathlineto{\pgfqpoint{4.817540in}{2.502184in}}%
\pgfpathlineto{\pgfqpoint{4.825545in}{2.516327in}}%
\pgfpathlineto{\pgfqpoint{4.833544in}{2.530308in}}%
\pgfpathlineto{\pgfqpoint{4.818906in}{2.516188in}}%
\pgfpathlineto{\pgfqpoint{4.804288in}{2.502254in}}%
\pgfpathlineto{\pgfqpoint{4.789688in}{2.488507in}}%
\pgfpathlineto{\pgfqpoint{4.775107in}{2.474946in}}%
\pgfpathlineto{\pgfqpoint{4.767113in}{2.461146in}}%
\pgfpathlineto{\pgfqpoint{4.759112in}{2.447192in}}%
\pgfpathlineto{\pgfqpoint{4.751105in}{2.433085in}}%
\pgfpathlineto{\pgfqpoint{4.743092in}{2.418827in}}%
\pgfpathclose%
\pgfusepath{fill}%
\end{pgfscope}%
\begin{pgfscope}%
\pgfpathrectangle{\pgfqpoint{1.150000in}{0.150000in}}{\pgfqpoint{5.700000in}{5.700000in}}%
\pgfusepath{clip}%
\pgfsetbuttcap%
\pgfsetroundjoin%
\definecolor{currentfill}{rgb}{0.119423,0.611141,0.538982}%
\pgfsetfillcolor{currentfill}%
\pgfsetfillopacity{0.800000}%
\pgfsetlinewidth{0.000000pt}%
\definecolor{currentstroke}{rgb}{0.000000,0.000000,0.000000}%
\pgfsetstrokecolor{currentstroke}%
\pgfsetdash{}{0pt}%
\pgfpathmoveto{\pgfqpoint{4.987779in}{2.744816in}}%
\pgfpathlineto{\pgfqpoint{5.002527in}{2.760245in}}%
\pgfpathlineto{\pgfqpoint{5.017296in}{2.775863in}}%
\pgfpathlineto{\pgfqpoint{5.032085in}{2.791670in}}%
\pgfpathlineto{\pgfqpoint{5.046896in}{2.807665in}}%
\pgfpathlineto{\pgfqpoint{5.054805in}{2.819573in}}%
\pgfpathlineto{\pgfqpoint{5.062706in}{2.831289in}}%
\pgfpathlineto{\pgfqpoint{5.070599in}{2.842813in}}%
\pgfpathlineto{\pgfqpoint{5.078483in}{2.854146in}}%
\pgfpathlineto{\pgfqpoint{5.063671in}{2.838097in}}%
\pgfpathlineto{\pgfqpoint{5.048880in}{2.822237in}}%
\pgfpathlineto{\pgfqpoint{5.034111in}{2.806566in}}%
\pgfpathlineto{\pgfqpoint{5.019362in}{2.791083in}}%
\pgfpathlineto{\pgfqpoint{5.011478in}{2.779790in}}%
\pgfpathlineto{\pgfqpoint{5.003587in}{2.768315in}}%
\pgfpathlineto{\pgfqpoint{4.995687in}{2.756657in}}%
\pgfpathlineto{\pgfqpoint{4.987779in}{2.744816in}}%
\pgfpathclose%
\pgfusepath{fill}%
\end{pgfscope}%
\begin{pgfscope}%
\pgfpathrectangle{\pgfqpoint{1.150000in}{0.150000in}}{\pgfqpoint{5.700000in}{5.700000in}}%
\pgfusepath{clip}%
\pgfsetbuttcap%
\pgfsetroundjoin%
\definecolor{currentfill}{rgb}{0.129933,0.559582,0.551864}%
\pgfsetfillcolor{currentfill}%
\pgfsetfillopacity{0.800000}%
\pgfsetlinewidth{0.000000pt}%
\definecolor{currentstroke}{rgb}{0.000000,0.000000,0.000000}%
\pgfsetstrokecolor{currentstroke}%
\pgfsetdash{}{0pt}%
\pgfpathmoveto{\pgfqpoint{4.865473in}{2.584577in}}%
\pgfpathlineto{\pgfqpoint{4.880135in}{2.599043in}}%
\pgfpathlineto{\pgfqpoint{4.894816in}{2.613696in}}%
\pgfpathlineto{\pgfqpoint{4.909518in}{2.628536in}}%
\pgfpathlineto{\pgfqpoint{4.924239in}{2.643565in}}%
\pgfpathlineto{\pgfqpoint{4.932208in}{2.656850in}}%
\pgfpathlineto{\pgfqpoint{4.940169in}{2.669958in}}%
\pgfpathlineto{\pgfqpoint{4.948123in}{2.682886in}}%
\pgfpathlineto{\pgfqpoint{4.956070in}{2.695634in}}%
\pgfpathlineto{\pgfqpoint{4.941345in}{2.680482in}}%
\pgfpathlineto{\pgfqpoint{4.926640in}{2.665517in}}%
\pgfpathlineto{\pgfqpoint{4.911955in}{2.650740in}}%
\pgfpathlineto{\pgfqpoint{4.897290in}{2.636150in}}%
\pgfpathlineto{\pgfqpoint{4.889347in}{2.623513in}}%
\pgfpathlineto{\pgfqpoint{4.881396in}{2.610704in}}%
\pgfpathlineto{\pgfqpoint{4.873438in}{2.597725in}}%
\pgfpathlineto{\pgfqpoint{4.865473in}{2.584577in}}%
\pgfpathclose%
\pgfusepath{fill}%
\end{pgfscope}%
\begin{pgfscope}%
\pgfpathrectangle{\pgfqpoint{1.150000in}{0.150000in}}{\pgfqpoint{5.700000in}{5.700000in}}%
\pgfusepath{clip}%
\pgfsetbuttcap%
\pgfsetroundjoin%
\definecolor{currentfill}{rgb}{0.525776,0.833491,0.288127}%
\pgfsetfillcolor{currentfill}%
\pgfsetfillopacity{0.800000}%
\pgfsetlinewidth{0.000000pt}%
\definecolor{currentstroke}{rgb}{0.000000,0.000000,0.000000}%
\pgfsetstrokecolor{currentstroke}%
\pgfsetdash{}{0pt}%
\pgfpathmoveto{\pgfqpoint{5.746770in}{3.568637in}}%
\pgfpathlineto{\pgfqpoint{5.762090in}{3.587985in}}%
\pgfpathlineto{\pgfqpoint{5.777435in}{3.607525in}}%
\pgfpathlineto{\pgfqpoint{5.792807in}{3.627259in}}%
\pgfpathlineto{\pgfqpoint{5.800188in}{3.629597in}}%
\pgfpathlineto{\pgfqpoint{5.807556in}{3.631766in}}%
\pgfpathlineto{\pgfqpoint{5.814912in}{3.633769in}}%
\pgfpathlineto{\pgfqpoint{5.822256in}{3.635609in}}%
\pgfpathlineto{\pgfqpoint{5.806905in}{3.616230in}}%
\pgfpathlineto{\pgfqpoint{5.791580in}{3.597043in}}%
\pgfpathlineto{\pgfqpoint{5.776281in}{3.578048in}}%
\pgfpathlineto{\pgfqpoint{5.768921in}{3.575933in}}%
\pgfpathlineto{\pgfqpoint{5.761550in}{3.573662in}}%
\pgfpathlineto{\pgfqpoint{5.754166in}{3.571231in}}%
\pgfpathlineto{\pgfqpoint{5.746770in}{3.568637in}}%
\pgfpathclose%
\pgfusepath{fill}%
\end{pgfscope}%
\begin{pgfscope}%
\pgfpathrectangle{\pgfqpoint{1.150000in}{0.150000in}}{\pgfqpoint{5.700000in}{5.700000in}}%
\pgfusepath{clip}%
\pgfsetbuttcap%
\pgfsetroundjoin%
\definecolor{currentfill}{rgb}{0.276022,0.044167,0.370164}%
\pgfsetfillcolor{currentfill}%
\pgfsetfillopacity{0.800000}%
\pgfsetlinewidth{0.000000pt}%
\definecolor{currentstroke}{rgb}{0.000000,0.000000,0.000000}%
\pgfsetstrokecolor{currentstroke}%
\pgfsetdash{}{0pt}%
\pgfpathmoveto{\pgfqpoint{3.707448in}{1.237924in}}%
\pgfpathlineto{\pgfqpoint{3.721544in}{1.237309in}}%
\pgfpathlineto{\pgfqpoint{3.735647in}{1.236874in}}%
\pgfpathlineto{\pgfqpoint{3.749758in}{1.236618in}}%
\pgfpathlineto{\pgfqpoint{3.763877in}{1.236540in}}%
\pgfpathlineto{\pgfqpoint{3.772190in}{1.247665in}}%
\pgfpathlineto{\pgfqpoint{3.780497in}{1.258973in}}%
\pgfpathlineto{\pgfqpoint{3.788798in}{1.270458in}}%
\pgfpathlineto{\pgfqpoint{3.797092in}{1.282114in}}%
\pgfpathlineto{\pgfqpoint{3.782986in}{1.281493in}}%
\pgfpathlineto{\pgfqpoint{3.768887in}{1.281052in}}%
\pgfpathlineto{\pgfqpoint{3.754797in}{1.280790in}}%
\pgfpathlineto{\pgfqpoint{3.740715in}{1.280708in}}%
\pgfpathlineto{\pgfqpoint{3.732409in}{1.269738in}}%
\pgfpathlineto{\pgfqpoint{3.724096in}{1.258946in}}%
\pgfpathlineto{\pgfqpoint{3.715776in}{1.248339in}}%
\pgfpathlineto{\pgfqpoint{3.707448in}{1.237924in}}%
\pgfpathclose%
\pgfusepath{fill}%
\end{pgfscope}%
\begin{pgfscope}%
\pgfpathrectangle{\pgfqpoint{1.150000in}{0.150000in}}{\pgfqpoint{5.700000in}{5.700000in}}%
\pgfusepath{clip}%
\pgfsetbuttcap%
\pgfsetroundjoin%
\definecolor{currentfill}{rgb}{0.279566,0.067836,0.391917}%
\pgfsetfillcolor{currentfill}%
\pgfsetfillopacity{0.800000}%
\pgfsetlinewidth{0.000000pt}%
\definecolor{currentstroke}{rgb}{0.000000,0.000000,0.000000}%
\pgfsetstrokecolor{currentstroke}%
\pgfsetdash{}{0pt}%
\pgfpathmoveto{\pgfqpoint{3.797092in}{1.282114in}}%
\pgfpathlineto{\pgfqpoint{3.811207in}{1.282914in}}%
\pgfpathlineto{\pgfqpoint{3.825330in}{1.283892in}}%
\pgfpathlineto{\pgfqpoint{3.839463in}{1.285049in}}%
\pgfpathlineto{\pgfqpoint{3.853604in}{1.286385in}}%
\pgfpathlineto{\pgfqpoint{3.861882in}{1.298883in}}%
\pgfpathlineto{\pgfqpoint{3.870154in}{1.311530in}}%
\pgfpathlineto{\pgfqpoint{3.878421in}{1.324320in}}%
\pgfpathlineto{\pgfqpoint{3.886682in}{1.337246in}}%
\pgfpathlineto{\pgfqpoint{3.872549in}{1.335243in}}%
\pgfpathlineto{\pgfqpoint{3.858426in}{1.333418in}}%
\pgfpathlineto{\pgfqpoint{3.844312in}{1.331773in}}%
\pgfpathlineto{\pgfqpoint{3.830207in}{1.330306in}}%
\pgfpathlineto{\pgfqpoint{3.821937in}{1.318036in}}%
\pgfpathlineto{\pgfqpoint{3.813662in}{1.305909in}}%
\pgfpathlineto{\pgfqpoint{3.805380in}{1.293933in}}%
\pgfpathlineto{\pgfqpoint{3.797092in}{1.282114in}}%
\pgfpathclose%
\pgfusepath{fill}%
\end{pgfscope}%
\begin{pgfscope}%
\pgfpathrectangle{\pgfqpoint{1.150000in}{0.150000in}}{\pgfqpoint{5.700000in}{5.700000in}}%
\pgfusepath{clip}%
\pgfsetbuttcap%
\pgfsetroundjoin%
\definecolor{currentfill}{rgb}{0.258965,0.251537,0.524736}%
\pgfsetfillcolor{currentfill}%
\pgfsetfillopacity{0.800000}%
\pgfsetlinewidth{0.000000pt}%
\definecolor{currentstroke}{rgb}{0.000000,0.000000,0.000000}%
\pgfsetstrokecolor{currentstroke}%
\pgfsetdash{}{0pt}%
\pgfpathmoveto{\pgfqpoint{4.221122in}{1.687841in}}%
\pgfpathlineto{\pgfqpoint{4.235394in}{1.695057in}}%
\pgfpathlineto{\pgfqpoint{4.249680in}{1.702454in}}%
\pgfpathlineto{\pgfqpoint{4.263979in}{1.710030in}}%
\pgfpathlineto{\pgfqpoint{4.278292in}{1.717786in}}%
\pgfpathlineto{\pgfqpoint{4.286457in}{1.734229in}}%
\pgfpathlineto{\pgfqpoint{4.294618in}{1.750656in}}%
\pgfpathlineto{\pgfqpoint{4.302775in}{1.767064in}}%
\pgfpathlineto{\pgfqpoint{4.310929in}{1.783447in}}%
\pgfpathlineto{\pgfqpoint{4.296612in}{1.775203in}}%
\pgfpathlineto{\pgfqpoint{4.282309in}{1.767140in}}%
\pgfpathlineto{\pgfqpoint{4.268021in}{1.759258in}}%
\pgfpathlineto{\pgfqpoint{4.253746in}{1.751556in}}%
\pgfpathlineto{\pgfqpoint{4.245596in}{1.735646in}}%
\pgfpathlineto{\pgfqpoint{4.237442in}{1.719721in}}%
\pgfpathlineto{\pgfqpoint{4.229284in}{1.703784in}}%
\pgfpathlineto{\pgfqpoint{4.221122in}{1.687841in}}%
\pgfpathclose%
\pgfusepath{fill}%
\end{pgfscope}%
\begin{pgfscope}%
\pgfpathrectangle{\pgfqpoint{1.150000in}{0.150000in}}{\pgfqpoint{5.700000in}{5.700000in}}%
\pgfusepath{clip}%
\pgfsetbuttcap%
\pgfsetroundjoin%
\definecolor{currentfill}{rgb}{0.272594,0.025563,0.353093}%
\pgfsetfillcolor{currentfill}%
\pgfsetfillopacity{0.800000}%
\pgfsetlinewidth{0.000000pt}%
\definecolor{currentstroke}{rgb}{0.000000,0.000000,0.000000}%
\pgfsetstrokecolor{currentstroke}%
\pgfsetdash{}{0pt}%
\pgfpathmoveto{\pgfqpoint{3.617686in}{1.205493in}}%
\pgfpathlineto{\pgfqpoint{3.631771in}{1.203430in}}%
\pgfpathlineto{\pgfqpoint{3.645863in}{1.201546in}}%
\pgfpathlineto{\pgfqpoint{3.659960in}{1.199843in}}%
\pgfpathlineto{\pgfqpoint{3.674065in}{1.198320in}}%
\pgfpathlineto{\pgfqpoint{3.682422in}{1.207898in}}%
\pgfpathlineto{\pgfqpoint{3.690772in}{1.217696in}}%
\pgfpathlineto{\pgfqpoint{3.699114in}{1.227707in}}%
\pgfpathlineto{\pgfqpoint{3.707448in}{1.237924in}}%
\pgfpathlineto{\pgfqpoint{3.693360in}{1.238719in}}%
\pgfpathlineto{\pgfqpoint{3.679279in}{1.239693in}}%
\pgfpathlineto{\pgfqpoint{3.665204in}{1.240849in}}%
\pgfpathlineto{\pgfqpoint{3.651137in}{1.242185in}}%
\pgfpathlineto{\pgfqpoint{3.642786in}{1.232684in}}%
\pgfpathlineto{\pgfqpoint{3.634428in}{1.223397in}}%
\pgfpathlineto{\pgfqpoint{3.626061in}{1.214331in}}%
\pgfpathlineto{\pgfqpoint{3.617686in}{1.205493in}}%
\pgfpathclose%
\pgfusepath{fill}%
\end{pgfscope}%
\begin{pgfscope}%
\pgfpathrectangle{\pgfqpoint{1.150000in}{0.150000in}}{\pgfqpoint{5.700000in}{5.700000in}}%
\pgfusepath{clip}%
\pgfsetbuttcap%
\pgfsetroundjoin%
\definecolor{currentfill}{rgb}{0.233603,0.313828,0.543914}%
\pgfsetfillcolor{currentfill}%
\pgfsetfillopacity{0.800000}%
\pgfsetlinewidth{0.000000pt}%
\definecolor{currentstroke}{rgb}{0.000000,0.000000,0.000000}%
\pgfsetstrokecolor{currentstroke}%
\pgfsetdash{}{0pt}%
\pgfpathmoveto{\pgfqpoint{4.343504in}{1.848657in}}%
\pgfpathlineto{\pgfqpoint{4.357840in}{1.857538in}}%
\pgfpathlineto{\pgfqpoint{4.372191in}{1.866599in}}%
\pgfpathlineto{\pgfqpoint{4.386556in}{1.875842in}}%
\pgfpathlineto{\pgfqpoint{4.400937in}{1.885266in}}%
\pgfpathlineto{\pgfqpoint{4.409076in}{1.901903in}}%
\pgfpathlineto{\pgfqpoint{4.417211in}{1.918483in}}%
\pgfpathlineto{\pgfqpoint{4.425342in}{1.935002in}}%
\pgfpathlineto{\pgfqpoint{4.433469in}{1.951456in}}%
\pgfpathlineto{\pgfqpoint{4.419083in}{1.941607in}}%
\pgfpathlineto{\pgfqpoint{4.404712in}{1.931940in}}%
\pgfpathlineto{\pgfqpoint{4.390356in}{1.922454in}}%
\pgfpathlineto{\pgfqpoint{4.376016in}{1.913151in}}%
\pgfpathlineto{\pgfqpoint{4.367894in}{1.897108in}}%
\pgfpathlineto{\pgfqpoint{4.359768in}{1.881009in}}%
\pgfpathlineto{\pgfqpoint{4.351638in}{1.864858in}}%
\pgfpathlineto{\pgfqpoint{4.343504in}{1.848657in}}%
\pgfpathclose%
\pgfusepath{fill}%
\end{pgfscope}%
\begin{pgfscope}%
\pgfpathrectangle{\pgfqpoint{1.150000in}{0.150000in}}{\pgfqpoint{5.700000in}{5.700000in}}%
\pgfusepath{clip}%
\pgfsetbuttcap%
\pgfsetroundjoin%
\definecolor{currentfill}{rgb}{0.239374,0.735588,0.455688}%
\pgfsetfillcolor{currentfill}%
\pgfsetfillopacity{0.800000}%
\pgfsetlinewidth{0.000000pt}%
\definecolor{currentstroke}{rgb}{0.000000,0.000000,0.000000}%
\pgfsetstrokecolor{currentstroke}%
\pgfsetdash{}{0pt}%
\pgfpathmoveto{\pgfqpoint{5.322629in}{3.143803in}}%
\pgfpathlineto{\pgfqpoint{5.337631in}{3.161414in}}%
\pgfpathlineto{\pgfqpoint{5.352657in}{3.179216in}}%
\pgfpathlineto{\pgfqpoint{5.367705in}{3.197209in}}%
\pgfpathlineto{\pgfqpoint{5.382778in}{3.215394in}}%
\pgfpathlineto{\pgfqpoint{5.390490in}{3.223214in}}%
\pgfpathlineto{\pgfqpoint{5.398192in}{3.230830in}}%
\pgfpathlineto{\pgfqpoint{5.405883in}{3.238244in}}%
\pgfpathlineto{\pgfqpoint{5.413563in}{3.245459in}}%
\pgfpathlineto{\pgfqpoint{5.398498in}{3.227403in}}%
\pgfpathlineto{\pgfqpoint{5.383455in}{3.209539in}}%
\pgfpathlineto{\pgfqpoint{5.368437in}{3.191865in}}%
\pgfpathlineto{\pgfqpoint{5.353441in}{3.174382in}}%
\pgfpathlineto{\pgfqpoint{5.345753in}{3.167025in}}%
\pgfpathlineto{\pgfqpoint{5.338056in}{3.159478in}}%
\pgfpathlineto{\pgfqpoint{5.330347in}{3.151737in}}%
\pgfpathlineto{\pgfqpoint{5.322629in}{3.143803in}}%
\pgfpathclose%
\pgfusepath{fill}%
\end{pgfscope}%
\begin{pgfscope}%
\pgfpathrectangle{\pgfqpoint{1.150000in}{0.150000in}}{\pgfqpoint{5.700000in}{5.700000in}}%
\pgfusepath{clip}%
\pgfsetbuttcap%
\pgfsetroundjoin%
\definecolor{currentfill}{rgb}{0.282656,0.100196,0.422160}%
\pgfsetfillcolor{currentfill}%
\pgfsetfillopacity{0.800000}%
\pgfsetlinewidth{0.000000pt}%
\definecolor{currentstroke}{rgb}{0.000000,0.000000,0.000000}%
\pgfsetstrokecolor{currentstroke}%
\pgfsetdash{}{0pt}%
\pgfpathmoveto{\pgfqpoint{3.886682in}{1.337246in}}%
\pgfpathlineto{\pgfqpoint{3.900824in}{1.339428in}}%
\pgfpathlineto{\pgfqpoint{3.914976in}{1.341788in}}%
\pgfpathlineto{\pgfqpoint{3.929137in}{1.344327in}}%
\pgfpathlineto{\pgfqpoint{3.943309in}{1.347043in}}%
\pgfpathlineto{\pgfqpoint{3.951558in}{1.360749in}}%
\pgfpathlineto{\pgfqpoint{3.959803in}{1.374571in}}%
\pgfpathlineto{\pgfqpoint{3.968042in}{1.388502in}}%
\pgfpathlineto{\pgfqpoint{3.976277in}{1.402537in}}%
\pgfpathlineto{\pgfqpoint{3.962110in}{1.399182in}}%
\pgfpathlineto{\pgfqpoint{3.947955in}{1.396006in}}%
\pgfpathlineto{\pgfqpoint{3.933809in}{1.393008in}}%
\pgfpathlineto{\pgfqpoint{3.919673in}{1.390189in}}%
\pgfpathlineto{\pgfqpoint{3.911433in}{1.376780in}}%
\pgfpathlineto{\pgfqpoint{3.903188in}{1.363482in}}%
\pgfpathlineto{\pgfqpoint{3.894938in}{1.350302in}}%
\pgfpathlineto{\pgfqpoint{3.886682in}{1.337246in}}%
\pgfpathclose%
\pgfusepath{fill}%
\end{pgfscope}%
\begin{pgfscope}%
\pgfpathrectangle{\pgfqpoint{1.150000in}{0.150000in}}{\pgfqpoint{5.700000in}{5.700000in}}%
\pgfusepath{clip}%
\pgfsetbuttcap%
\pgfsetroundjoin%
\definecolor{currentfill}{rgb}{0.276194,0.190074,0.493001}%
\pgfsetfillcolor{currentfill}%
\pgfsetfillopacity{0.800000}%
\pgfsetlinewidth{0.000000pt}%
\definecolor{currentstroke}{rgb}{0.000000,0.000000,0.000000}%
\pgfsetstrokecolor{currentstroke}%
\pgfsetdash{}{0pt}%
\pgfpathmoveto{\pgfqpoint{4.098741in}{1.537799in}}%
\pgfpathlineto{\pgfqpoint{4.112960in}{1.543233in}}%
\pgfpathlineto{\pgfqpoint{4.127190in}{1.548846in}}%
\pgfpathlineto{\pgfqpoint{4.141433in}{1.554638in}}%
\pgfpathlineto{\pgfqpoint{4.155688in}{1.560608in}}%
\pgfpathlineto{\pgfqpoint{4.163881in}{1.576436in}}%
\pgfpathlineto{\pgfqpoint{4.172070in}{1.592295in}}%
\pgfpathlineto{\pgfqpoint{4.180255in}{1.608181in}}%
\pgfpathlineto{\pgfqpoint{4.188436in}{1.624089in}}%
\pgfpathlineto{\pgfqpoint{4.174180in}{1.617570in}}%
\pgfpathlineto{\pgfqpoint{4.159936in}{1.611230in}}%
\pgfpathlineto{\pgfqpoint{4.145705in}{1.605070in}}%
\pgfpathlineto{\pgfqpoint{4.131487in}{1.599089in}}%
\pgfpathlineto{\pgfqpoint{4.123307in}{1.583717in}}%
\pgfpathlineto{\pgfqpoint{4.115122in}{1.568374in}}%
\pgfpathlineto{\pgfqpoint{4.106934in}{1.553067in}}%
\pgfpathlineto{\pgfqpoint{4.098741in}{1.537799in}}%
\pgfpathclose%
\pgfusepath{fill}%
\end{pgfscope}%
\begin{pgfscope}%
\pgfpathrectangle{\pgfqpoint{1.150000in}{0.150000in}}{\pgfqpoint{5.700000in}{5.700000in}}%
\pgfusepath{clip}%
\pgfsetbuttcap%
\pgfsetroundjoin%
\definecolor{currentfill}{rgb}{0.269944,0.014625,0.341379}%
\pgfsetfillcolor{currentfill}%
\pgfsetfillopacity{0.800000}%
\pgfsetlinewidth{0.000000pt}%
\definecolor{currentstroke}{rgb}{0.000000,0.000000,0.000000}%
\pgfsetstrokecolor{currentstroke}%
\pgfsetdash{}{0pt}%
\pgfpathmoveto{\pgfqpoint{3.381193in}{1.201512in}}%
\pgfpathlineto{\pgfqpoint{3.395269in}{1.195697in}}%
\pgfpathlineto{\pgfqpoint{3.409348in}{1.190069in}}%
\pgfpathlineto{\pgfqpoint{3.423431in}{1.184626in}}%
\pgfpathlineto{\pgfqpoint{3.437517in}{1.179368in}}%
\pgfpathlineto{\pgfqpoint{3.446018in}{1.184501in}}%
\pgfpathlineto{\pgfqpoint{3.454507in}{1.189939in}}%
\pgfpathlineto{\pgfqpoint{3.462985in}{1.195676in}}%
\pgfpathlineto{\pgfqpoint{3.471452in}{1.201702in}}%
\pgfpathlineto{\pgfqpoint{3.457393in}{1.206168in}}%
\pgfpathlineto{\pgfqpoint{3.443337in}{1.210819in}}%
\pgfpathlineto{\pgfqpoint{3.429286in}{1.215655in}}%
\pgfpathlineto{\pgfqpoint{3.415239in}{1.220677in}}%
\pgfpathlineto{\pgfqpoint{3.406745in}{1.215430in}}%
\pgfpathlineto{\pgfqpoint{3.398240in}{1.210482in}}%
\pgfpathlineto{\pgfqpoint{3.389722in}{1.205840in}}%
\pgfpathlineto{\pgfqpoint{3.381193in}{1.201512in}}%
\pgfpathclose%
\pgfusepath{fill}%
\end{pgfscope}%
\begin{pgfscope}%
\pgfpathrectangle{\pgfqpoint{1.150000in}{0.150000in}}{\pgfqpoint{5.700000in}{5.700000in}}%
\pgfusepath{clip}%
\pgfsetbuttcap%
\pgfsetroundjoin%
\definecolor{currentfill}{rgb}{0.203063,0.379716,0.553925}%
\pgfsetfillcolor{currentfill}%
\pgfsetfillopacity{0.800000}%
\pgfsetlinewidth{0.000000pt}%
\definecolor{currentstroke}{rgb}{0.000000,0.000000,0.000000}%
\pgfsetstrokecolor{currentstroke}%
\pgfsetdash{}{0pt}%
\pgfpathmoveto{\pgfqpoint{4.465935in}{2.016554in}}%
\pgfpathlineto{\pgfqpoint{4.480343in}{2.026978in}}%
\pgfpathlineto{\pgfqpoint{4.494767in}{2.037585in}}%
\pgfpathlineto{\pgfqpoint{4.509208in}{2.048374in}}%
\pgfpathlineto{\pgfqpoint{4.523665in}{2.059346in}}%
\pgfpathlineto{\pgfqpoint{4.531777in}{2.075796in}}%
\pgfpathlineto{\pgfqpoint{4.539884in}{2.092152in}}%
\pgfpathlineto{\pgfqpoint{4.547987in}{2.108411in}}%
\pgfpathlineto{\pgfqpoint{4.556085in}{2.124569in}}%
\pgfpathlineto{\pgfqpoint{4.541622in}{2.113236in}}%
\pgfpathlineto{\pgfqpoint{4.527175in}{2.102085in}}%
\pgfpathlineto{\pgfqpoint{4.512745in}{2.091119in}}%
\pgfpathlineto{\pgfqpoint{4.498331in}{2.080335in}}%
\pgfpathlineto{\pgfqpoint{4.490238in}{2.064525in}}%
\pgfpathlineto{\pgfqpoint{4.482142in}{2.048622in}}%
\pgfpathlineto{\pgfqpoint{4.474041in}{2.032631in}}%
\pgfpathlineto{\pgfqpoint{4.465935in}{2.016554in}}%
\pgfpathclose%
\pgfusepath{fill}%
\end{pgfscope}%
\begin{pgfscope}%
\pgfpathrectangle{\pgfqpoint{1.150000in}{0.150000in}}{\pgfqpoint{5.700000in}{5.700000in}}%
\pgfusepath{clip}%
\pgfsetbuttcap%
\pgfsetroundjoin%
\definecolor{currentfill}{rgb}{0.386433,0.794644,0.372886}%
\pgfsetfillcolor{currentfill}%
\pgfsetfillopacity{0.800000}%
\pgfsetlinewidth{0.000000pt}%
\definecolor{currentstroke}{rgb}{0.000000,0.000000,0.000000}%
\pgfsetstrokecolor{currentstroke}%
\pgfsetdash{}{0pt}%
\pgfpathmoveto{\pgfqpoint{5.535033in}{3.368892in}}%
\pgfpathlineto{\pgfqpoint{5.550200in}{3.387537in}}%
\pgfpathlineto{\pgfqpoint{5.565390in}{3.406374in}}%
\pgfpathlineto{\pgfqpoint{5.580606in}{3.425404in}}%
\pgfpathlineto{\pgfqpoint{5.595846in}{3.444626in}}%
\pgfpathlineto{\pgfqpoint{5.603403in}{3.449686in}}%
\pgfpathlineto{\pgfqpoint{5.610948in}{3.454551in}}%
\pgfpathlineto{\pgfqpoint{5.618481in}{3.459224in}}%
\pgfpathlineto{\pgfqpoint{5.626003in}{3.463709in}}%
\pgfpathlineto{\pgfqpoint{5.610775in}{3.444729in}}%
\pgfpathlineto{\pgfqpoint{5.595573in}{3.425940in}}%
\pgfpathlineto{\pgfqpoint{5.580395in}{3.407344in}}%
\pgfpathlineto{\pgfqpoint{5.565242in}{3.388939in}}%
\pgfpathlineto{\pgfqpoint{5.557707in}{3.384200in}}%
\pgfpathlineto{\pgfqpoint{5.550161in}{3.379281in}}%
\pgfpathlineto{\pgfqpoint{5.542603in}{3.374180in}}%
\pgfpathlineto{\pgfqpoint{5.535033in}{3.368892in}}%
\pgfpathclose%
\pgfusepath{fill}%
\end{pgfscope}%
\begin{pgfscope}%
\pgfpathrectangle{\pgfqpoint{1.150000in}{0.150000in}}{\pgfqpoint{5.700000in}{5.700000in}}%
\pgfusepath{clip}%
\pgfsetbuttcap%
\pgfsetroundjoin%
\definecolor{currentfill}{rgb}{0.177423,0.437527,0.557565}%
\pgfsetfillcolor{currentfill}%
\pgfsetfillopacity{0.800000}%
\pgfsetlinewidth{0.000000pt}%
\definecolor{currentstroke}{rgb}{0.000000,0.000000,0.000000}%
\pgfsetstrokecolor{currentstroke}%
\pgfsetdash{}{0pt}%
\pgfpathmoveto{\pgfqpoint{4.588431in}{2.188138in}}%
\pgfpathlineto{\pgfqpoint{4.602918in}{2.199983in}}%
\pgfpathlineto{\pgfqpoint{4.617422in}{2.212012in}}%
\pgfpathlineto{\pgfqpoint{4.631944in}{2.224226in}}%
\pgfpathlineto{\pgfqpoint{4.646484in}{2.236624in}}%
\pgfpathlineto{\pgfqpoint{4.654564in}{2.252544in}}%
\pgfpathlineto{\pgfqpoint{4.662640in}{2.268338in}}%
\pgfpathlineto{\pgfqpoint{4.670710in}{2.284003in}}%
\pgfpathlineto{\pgfqpoint{4.678775in}{2.299537in}}%
\pgfpathlineto{\pgfqpoint{4.664228in}{2.286844in}}%
\pgfpathlineto{\pgfqpoint{4.649700in}{2.274335in}}%
\pgfpathlineto{\pgfqpoint{4.635189in}{2.262011in}}%
\pgfpathlineto{\pgfqpoint{4.620696in}{2.249871in}}%
\pgfpathlineto{\pgfqpoint{4.612638in}{2.234619in}}%
\pgfpathlineto{\pgfqpoint{4.604574in}{2.219245in}}%
\pgfpathlineto{\pgfqpoint{4.596505in}{2.203750in}}%
\pgfpathlineto{\pgfqpoint{4.588431in}{2.188138in}}%
\pgfpathclose%
\pgfusepath{fill}%
\end{pgfscope}%
\begin{pgfscope}%
\pgfpathrectangle{\pgfqpoint{1.150000in}{0.150000in}}{\pgfqpoint{5.700000in}{5.700000in}}%
\pgfusepath{clip}%
\pgfsetbuttcap%
\pgfsetroundjoin%
\definecolor{currentfill}{rgb}{0.269944,0.014625,0.341379}%
\pgfsetfillcolor{currentfill}%
\pgfsetfillopacity{0.800000}%
\pgfsetlinewidth{0.000000pt}%
\definecolor{currentstroke}{rgb}{0.000000,0.000000,0.000000}%
\pgfsetstrokecolor{currentstroke}%
\pgfsetdash{}{0pt}%
\pgfpathmoveto{\pgfqpoint{3.527735in}{1.185676in}}%
\pgfpathlineto{\pgfqpoint{3.541818in}{1.182127in}}%
\pgfpathlineto{\pgfqpoint{3.555906in}{1.178760in}}%
\pgfpathlineto{\pgfqpoint{3.570000in}{1.175574in}}%
\pgfpathlineto{\pgfqpoint{3.584099in}{1.172569in}}%
\pgfpathlineto{\pgfqpoint{3.592509in}{1.180421in}}%
\pgfpathlineto{\pgfqpoint{3.600911in}{1.188531in}}%
\pgfpathlineto{\pgfqpoint{3.609303in}{1.196891in}}%
\pgfpathlineto{\pgfqpoint{3.617686in}{1.205493in}}%
\pgfpathlineto{\pgfqpoint{3.603608in}{1.207738in}}%
\pgfpathlineto{\pgfqpoint{3.589535in}{1.210164in}}%
\pgfpathlineto{\pgfqpoint{3.575468in}{1.212772in}}%
\pgfpathlineto{\pgfqpoint{3.561407in}{1.215562in}}%
\pgfpathlineto{\pgfqpoint{3.553003in}{1.207707in}}%
\pgfpathlineto{\pgfqpoint{3.544590in}{1.200102in}}%
\pgfpathlineto{\pgfqpoint{3.536168in}{1.192756in}}%
\pgfpathlineto{\pgfqpoint{3.527735in}{1.185676in}}%
\pgfpathclose%
\pgfusepath{fill}%
\end{pgfscope}%
\begin{pgfscope}%
\pgfpathrectangle{\pgfqpoint{1.150000in}{0.150000in}}{\pgfqpoint{5.700000in}{5.700000in}}%
\pgfusepath{clip}%
\pgfsetbuttcap%
\pgfsetroundjoin%
\definecolor{currentfill}{rgb}{0.130067,0.651384,0.521608}%
\pgfsetfillcolor{currentfill}%
\pgfsetfillopacity{0.800000}%
\pgfsetlinewidth{0.000000pt}%
\definecolor{currentstroke}{rgb}{0.000000,0.000000,0.000000}%
\pgfsetstrokecolor{currentstroke}%
\pgfsetdash{}{0pt}%
\pgfpathmoveto{\pgfqpoint{1.898340in}{3.015447in}}%
\pgfpathlineto{\pgfqpoint{1.913209in}{2.983618in}}%
\pgfpathlineto{\pgfqpoint{1.928056in}{2.952156in}}%
\pgfpathlineto{\pgfqpoint{1.942881in}{2.921058in}}%
\pgfpathlineto{\pgfqpoint{1.957684in}{2.890319in}}%
\pgfpathlineto{\pgfqpoint{1.967698in}{2.873436in}}%
\pgfpathlineto{\pgfqpoint{1.977673in}{2.857117in}}%
\pgfpathlineto{\pgfqpoint{1.987612in}{2.841351in}}%
\pgfpathlineto{\pgfqpoint{1.997514in}{2.826128in}}%
\pgfpathlineto{\pgfqpoint{1.982802in}{2.855945in}}%
\pgfpathlineto{\pgfqpoint{1.968070in}{2.886119in}}%
\pgfpathlineto{\pgfqpoint{1.953317in}{2.916653in}}%
\pgfpathlineto{\pgfqpoint{1.938542in}{2.947551in}}%
\pgfpathlineto{\pgfqpoint{1.928549in}{2.963681in}}%
\pgfpathlineto{\pgfqpoint{1.918519in}{2.980368in}}%
\pgfpathlineto{\pgfqpoint{1.908449in}{2.997620in}}%
\pgfpathlineto{\pgfqpoint{1.898340in}{3.015447in}}%
\pgfpathclose%
\pgfusepath{fill}%
\end{pgfscope}%
\begin{pgfscope}%
\pgfpathrectangle{\pgfqpoint{1.150000in}{0.150000in}}{\pgfqpoint{5.700000in}{5.700000in}}%
\pgfusepath{clip}%
\pgfsetbuttcap%
\pgfsetroundjoin%
\definecolor{currentfill}{rgb}{0.283072,0.130895,0.449241}%
\pgfsetfillcolor{currentfill}%
\pgfsetfillopacity{0.800000}%
\pgfsetlinewidth{0.000000pt}%
\definecolor{currentstroke}{rgb}{0.000000,0.000000,0.000000}%
\pgfsetstrokecolor{currentstroke}%
\pgfsetdash{}{0pt}%
\pgfpathmoveto{\pgfqpoint{3.976277in}{1.402537in}}%
\pgfpathlineto{\pgfqpoint{3.990454in}{1.406071in}}%
\pgfpathlineto{\pgfqpoint{4.004641in}{1.409782in}}%
\pgfpathlineto{\pgfqpoint{4.018840in}{1.413672in}}%
\pgfpathlineto{\pgfqpoint{4.033049in}{1.417739in}}%
\pgfpathlineto{\pgfqpoint{4.041276in}{1.432493in}}%
\pgfpathlineto{\pgfqpoint{4.049498in}{1.447330in}}%
\pgfpathlineto{\pgfqpoint{4.057716in}{1.462245in}}%
\pgfpathlineto{\pgfqpoint{4.065930in}{1.477233in}}%
\pgfpathlineto{\pgfqpoint{4.051722in}{1.472556in}}%
\pgfpathlineto{\pgfqpoint{4.037526in}{1.468058in}}%
\pgfpathlineto{\pgfqpoint{4.023342in}{1.463738in}}%
\pgfpathlineto{\pgfqpoint{4.009168in}{1.459597in}}%
\pgfpathlineto{\pgfqpoint{4.000952in}{1.445206in}}%
\pgfpathlineto{\pgfqpoint{3.992732in}{1.430895in}}%
\pgfpathlineto{\pgfqpoint{3.984507in}{1.416670in}}%
\pgfpathlineto{\pgfqpoint{3.976277in}{1.402537in}}%
\pgfpathclose%
\pgfusepath{fill}%
\end{pgfscope}%
\begin{pgfscope}%
\pgfpathrectangle{\pgfqpoint{1.150000in}{0.150000in}}{\pgfqpoint{5.700000in}{5.700000in}}%
\pgfusepath{clip}%
\pgfsetbuttcap%
\pgfsetroundjoin%
\definecolor{currentfill}{rgb}{0.170948,0.694384,0.493803}%
\pgfsetfillcolor{currentfill}%
\pgfsetfillopacity{0.800000}%
\pgfsetlinewidth{0.000000pt}%
\definecolor{currentstroke}{rgb}{0.000000,0.000000,0.000000}%
\pgfsetstrokecolor{currentstroke}%
\pgfsetdash{}{0pt}%
\pgfpathmoveto{\pgfqpoint{5.200700in}{3.004027in}}%
\pgfpathlineto{\pgfqpoint{5.215620in}{3.021022in}}%
\pgfpathlineto{\pgfqpoint{5.230561in}{3.038208in}}%
\pgfpathlineto{\pgfqpoint{5.245526in}{3.055585in}}%
\pgfpathlineto{\pgfqpoint{5.260513in}{3.073153in}}%
\pgfpathlineto{\pgfqpoint{5.268313in}{3.082693in}}%
\pgfpathlineto{\pgfqpoint{5.276103in}{3.092029in}}%
\pgfpathlineto{\pgfqpoint{5.283882in}{3.101161in}}%
\pgfpathlineto{\pgfqpoint{5.291652in}{3.110090in}}%
\pgfpathlineto{\pgfqpoint{5.276668in}{3.092577in}}%
\pgfpathlineto{\pgfqpoint{5.261707in}{3.075256in}}%
\pgfpathlineto{\pgfqpoint{5.246768in}{3.058125in}}%
\pgfpathlineto{\pgfqpoint{5.231851in}{3.041184in}}%
\pgfpathlineto{\pgfqpoint{5.224078in}{3.032186in}}%
\pgfpathlineto{\pgfqpoint{5.216295in}{3.022995in}}%
\pgfpathlineto{\pgfqpoint{5.208502in}{3.013609in}}%
\pgfpathlineto{\pgfqpoint{5.200700in}{3.004027in}}%
\pgfpathclose%
\pgfusepath{fill}%
\end{pgfscope}%
\begin{pgfscope}%
\pgfpathrectangle{\pgfqpoint{1.150000in}{0.150000in}}{\pgfqpoint{5.700000in}{5.700000in}}%
\pgfusepath{clip}%
\pgfsetbuttcap%
\pgfsetroundjoin%
\definecolor{currentfill}{rgb}{0.154815,0.493313,0.557840}%
\pgfsetfillcolor{currentfill}%
\pgfsetfillopacity{0.800000}%
\pgfsetlinewidth{0.000000pt}%
\definecolor{currentstroke}{rgb}{0.000000,0.000000,0.000000}%
\pgfsetstrokecolor{currentstroke}%
\pgfsetdash{}{0pt}%
\pgfpathmoveto{\pgfqpoint{4.710979in}{2.360318in}}%
\pgfpathlineto{\pgfqpoint{4.725550in}{2.373459in}}%
\pgfpathlineto{\pgfqpoint{4.740139in}{2.386785in}}%
\pgfpathlineto{\pgfqpoint{4.754747in}{2.400297in}}%
\pgfpathlineto{\pgfqpoint{4.769374in}{2.413995in}}%
\pgfpathlineto{\pgfqpoint{4.777417in}{2.429081in}}%
\pgfpathlineto{\pgfqpoint{4.785454in}{2.444014in}}%
\pgfpathlineto{\pgfqpoint{4.793485in}{2.458793in}}%
\pgfpathlineto{\pgfqpoint{4.801510in}{2.473415in}}%
\pgfpathlineto{\pgfqpoint{4.786877in}{2.459489in}}%
\pgfpathlineto{\pgfqpoint{4.772263in}{2.445749in}}%
\pgfpathlineto{\pgfqpoint{4.757668in}{2.432195in}}%
\pgfpathlineto{\pgfqpoint{4.743092in}{2.418827in}}%
\pgfpathlineto{\pgfqpoint{4.735072in}{2.404419in}}%
\pgfpathlineto{\pgfqpoint{4.727047in}{2.389864in}}%
\pgfpathlineto{\pgfqpoint{4.719016in}{2.375163in}}%
\pgfpathlineto{\pgfqpoint{4.710979in}{2.360318in}}%
\pgfpathclose%
\pgfusepath{fill}%
\end{pgfscope}%
\begin{pgfscope}%
\pgfpathrectangle{\pgfqpoint{1.150000in}{0.150000in}}{\pgfqpoint{5.700000in}{5.700000in}}%
\pgfusepath{clip}%
\pgfsetbuttcap%
\pgfsetroundjoin%
\definecolor{currentfill}{rgb}{0.130067,0.651384,0.521608}%
\pgfsetfillcolor{currentfill}%
\pgfsetfillopacity{0.800000}%
\pgfsetlinewidth{0.000000pt}%
\definecolor{currentstroke}{rgb}{0.000000,0.000000,0.000000}%
\pgfsetstrokecolor{currentstroke}%
\pgfsetdash{}{0pt}%
\pgfpathmoveto{\pgfqpoint{5.078483in}{2.854146in}}%
\pgfpathlineto{\pgfqpoint{5.093317in}{2.870384in}}%
\pgfpathlineto{\pgfqpoint{5.108171in}{2.886811in}}%
\pgfpathlineto{\pgfqpoint{5.123048in}{2.903429in}}%
\pgfpathlineto{\pgfqpoint{5.137947in}{2.920236in}}%
\pgfpathlineto{\pgfqpoint{5.145823in}{2.931408in}}%
\pgfpathlineto{\pgfqpoint{5.153690in}{2.942380in}}%
\pgfpathlineto{\pgfqpoint{5.161548in}{2.953152in}}%
\pgfpathlineto{\pgfqpoint{5.169397in}{2.963724in}}%
\pgfpathlineto{\pgfqpoint{5.154499in}{2.946900in}}%
\pgfpathlineto{\pgfqpoint{5.139622in}{2.930265in}}%
\pgfpathlineto{\pgfqpoint{5.124767in}{2.913821in}}%
\pgfpathlineto{\pgfqpoint{5.109934in}{2.897565in}}%
\pgfpathlineto{\pgfqpoint{5.102084in}{2.886997in}}%
\pgfpathlineto{\pgfqpoint{5.094226in}{2.876238in}}%
\pgfpathlineto{\pgfqpoint{5.086359in}{2.865287in}}%
\pgfpathlineto{\pgfqpoint{5.078483in}{2.854146in}}%
\pgfpathclose%
\pgfusepath{fill}%
\end{pgfscope}%
\begin{pgfscope}%
\pgfpathrectangle{\pgfqpoint{1.150000in}{0.150000in}}{\pgfqpoint{5.700000in}{5.700000in}}%
\pgfusepath{clip}%
\pgfsetbuttcap%
\pgfsetroundjoin%
\definecolor{currentfill}{rgb}{0.133743,0.548535,0.553541}%
\pgfsetfillcolor{currentfill}%
\pgfsetfillopacity{0.800000}%
\pgfsetlinewidth{0.000000pt}%
\definecolor{currentstroke}{rgb}{0.000000,0.000000,0.000000}%
\pgfsetstrokecolor{currentstroke}%
\pgfsetdash{}{0pt}%
\pgfpathmoveto{\pgfqpoint{4.833544in}{2.530308in}}%
\pgfpathlineto{\pgfqpoint{4.848202in}{2.544615in}}%
\pgfpathlineto{\pgfqpoint{4.862879in}{2.559109in}}%
\pgfpathlineto{\pgfqpoint{4.877576in}{2.573790in}}%
\pgfpathlineto{\pgfqpoint{4.892292in}{2.588658in}}%
\pgfpathlineto{\pgfqpoint{4.900290in}{2.602647in}}%
\pgfpathlineto{\pgfqpoint{4.908280in}{2.616462in}}%
\pgfpathlineto{\pgfqpoint{4.916263in}{2.630102in}}%
\pgfpathlineto{\pgfqpoint{4.924239in}{2.643565in}}%
\pgfpathlineto{\pgfqpoint{4.909518in}{2.628536in}}%
\pgfpathlineto{\pgfqpoint{4.894816in}{2.613696in}}%
\pgfpathlineto{\pgfqpoint{4.880135in}{2.599043in}}%
\pgfpathlineto{\pgfqpoint{4.865473in}{2.584577in}}%
\pgfpathlineto{\pgfqpoint{4.857501in}{2.571260in}}%
\pgfpathlineto{\pgfqpoint{4.849522in}{2.557775in}}%
\pgfpathlineto{\pgfqpoint{4.841537in}{2.544124in}}%
\pgfpathlineto{\pgfqpoint{4.833544in}{2.530308in}}%
\pgfpathclose%
\pgfusepath{fill}%
\end{pgfscope}%
\begin{pgfscope}%
\pgfpathrectangle{\pgfqpoint{1.150000in}{0.150000in}}{\pgfqpoint{5.700000in}{5.700000in}}%
\pgfusepath{clip}%
\pgfsetbuttcap%
\pgfsetroundjoin%
\definecolor{currentfill}{rgb}{0.120092,0.600104,0.542530}%
\pgfsetfillcolor{currentfill}%
\pgfsetfillopacity{0.800000}%
\pgfsetlinewidth{0.000000pt}%
\definecolor{currentstroke}{rgb}{0.000000,0.000000,0.000000}%
\pgfsetstrokecolor{currentstroke}%
\pgfsetdash{}{0pt}%
\pgfpathmoveto{\pgfqpoint{4.956070in}{2.695634in}}%
\pgfpathlineto{\pgfqpoint{4.970815in}{2.710975in}}%
\pgfpathlineto{\pgfqpoint{4.985582in}{2.726503in}}%
\pgfpathlineto{\pgfqpoint{5.000368in}{2.742221in}}%
\pgfpathlineto{\pgfqpoint{5.015176in}{2.758127in}}%
\pgfpathlineto{\pgfqpoint{5.023118in}{2.770797in}}%
\pgfpathlineto{\pgfqpoint{5.031052in}{2.783277in}}%
\pgfpathlineto{\pgfqpoint{5.038978in}{2.795567in}}%
\pgfpathlineto{\pgfqpoint{5.046896in}{2.807665in}}%
\pgfpathlineto{\pgfqpoint{5.032085in}{2.791670in}}%
\pgfpathlineto{\pgfqpoint{5.017296in}{2.775863in}}%
\pgfpathlineto{\pgfqpoint{5.002527in}{2.760245in}}%
\pgfpathlineto{\pgfqpoint{4.987779in}{2.744816in}}%
\pgfpathlineto{\pgfqpoint{4.979863in}{2.732793in}}%
\pgfpathlineto{\pgfqpoint{4.971940in}{2.720588in}}%
\pgfpathlineto{\pgfqpoint{4.964009in}{2.708202in}}%
\pgfpathlineto{\pgfqpoint{4.956070in}{2.695634in}}%
\pgfpathclose%
\pgfusepath{fill}%
\end{pgfscope}%
\begin{pgfscope}%
\pgfpathrectangle{\pgfqpoint{1.150000in}{0.150000in}}{\pgfqpoint{5.700000in}{5.700000in}}%
\pgfusepath{clip}%
\pgfsetbuttcap%
\pgfsetroundjoin%
\definecolor{currentfill}{rgb}{0.265145,0.232956,0.516599}%
\pgfsetfillcolor{currentfill}%
\pgfsetfillopacity{0.800000}%
\pgfsetlinewidth{0.000000pt}%
\definecolor{currentstroke}{rgb}{0.000000,0.000000,0.000000}%
\pgfsetstrokecolor{currentstroke}%
\pgfsetdash{}{0pt}%
\pgfpathmoveto{\pgfqpoint{4.188436in}{1.624089in}}%
\pgfpathlineto{\pgfqpoint{4.202706in}{1.630788in}}%
\pgfpathlineto{\pgfqpoint{4.216988in}{1.637666in}}%
\pgfpathlineto{\pgfqpoint{4.231284in}{1.644724in}}%
\pgfpathlineto{\pgfqpoint{4.245594in}{1.651961in}}%
\pgfpathlineto{\pgfqpoint{4.253774in}{1.668416in}}%
\pgfpathlineto{\pgfqpoint{4.261950in}{1.684875in}}%
\pgfpathlineto{\pgfqpoint{4.270123in}{1.701334in}}%
\pgfpathlineto{\pgfqpoint{4.278292in}{1.717786in}}%
\pgfpathlineto{\pgfqpoint{4.263979in}{1.710030in}}%
\pgfpathlineto{\pgfqpoint{4.249680in}{1.702454in}}%
\pgfpathlineto{\pgfqpoint{4.235394in}{1.695057in}}%
\pgfpathlineto{\pgfqpoint{4.221122in}{1.687841in}}%
\pgfpathlineto{\pgfqpoint{4.212957in}{1.671895in}}%
\pgfpathlineto{\pgfqpoint{4.204787in}{1.655951in}}%
\pgfpathlineto{\pgfqpoint{4.196614in}{1.640014in}}%
\pgfpathlineto{\pgfqpoint{4.188436in}{1.624089in}}%
\pgfpathclose%
\pgfusepath{fill}%
\end{pgfscope}%
\begin{pgfscope}%
\pgfpathrectangle{\pgfqpoint{1.150000in}{0.150000in}}{\pgfqpoint{5.700000in}{5.700000in}}%
\pgfusepath{clip}%
\pgfsetbuttcap%
\pgfsetroundjoin%
\definecolor{currentfill}{rgb}{0.241237,0.296485,0.539709}%
\pgfsetfillcolor{currentfill}%
\pgfsetfillopacity{0.800000}%
\pgfsetlinewidth{0.000000pt}%
\definecolor{currentstroke}{rgb}{0.000000,0.000000,0.000000}%
\pgfsetstrokecolor{currentstroke}%
\pgfsetdash{}{0pt}%
\pgfpathmoveto{\pgfqpoint{4.310929in}{1.783447in}}%
\pgfpathlineto{\pgfqpoint{4.325260in}{1.791872in}}%
\pgfpathlineto{\pgfqpoint{4.339606in}{1.800477in}}%
\pgfpathlineto{\pgfqpoint{4.353966in}{1.809263in}}%
\pgfpathlineto{\pgfqpoint{4.368342in}{1.818229in}}%
\pgfpathlineto{\pgfqpoint{4.376496in}{1.835054in}}%
\pgfpathlineto{\pgfqpoint{4.384647in}{1.851837in}}%
\pgfpathlineto{\pgfqpoint{4.392794in}{1.868576in}}%
\pgfpathlineto{\pgfqpoint{4.400937in}{1.885266in}}%
\pgfpathlineto{\pgfqpoint{4.386556in}{1.875842in}}%
\pgfpathlineto{\pgfqpoint{4.372191in}{1.866599in}}%
\pgfpathlineto{\pgfqpoint{4.357840in}{1.857538in}}%
\pgfpathlineto{\pgfqpoint{4.343504in}{1.848657in}}%
\pgfpathlineto{\pgfqpoint{4.335366in}{1.832412in}}%
\pgfpathlineto{\pgfqpoint{4.327224in}{1.816126in}}%
\pgfpathlineto{\pgfqpoint{4.319079in}{1.799803in}}%
\pgfpathlineto{\pgfqpoint{4.310929in}{1.783447in}}%
\pgfpathclose%
\pgfusepath{fill}%
\end{pgfscope}%
\begin{pgfscope}%
\pgfpathrectangle{\pgfqpoint{1.150000in}{0.150000in}}{\pgfqpoint{5.700000in}{5.700000in}}%
\pgfusepath{clip}%
\pgfsetbuttcap%
\pgfsetroundjoin%
\definecolor{currentfill}{rgb}{0.311925,0.767822,0.415586}%
\pgfsetfillcolor{currentfill}%
\pgfsetfillopacity{0.800000}%
\pgfsetlinewidth{0.000000pt}%
\definecolor{currentstroke}{rgb}{0.000000,0.000000,0.000000}%
\pgfsetstrokecolor{currentstroke}%
\pgfsetdash{}{0pt}%
\pgfpathmoveto{\pgfqpoint{5.413563in}{3.245459in}}%
\pgfpathlineto{\pgfqpoint{5.428653in}{3.263706in}}%
\pgfpathlineto{\pgfqpoint{5.443766in}{3.282145in}}%
\pgfpathlineto{\pgfqpoint{5.458903in}{3.300777in}}%
\pgfpathlineto{\pgfqpoint{5.474064in}{3.319601in}}%
\pgfpathlineto{\pgfqpoint{5.481726in}{3.326464in}}%
\pgfpathlineto{\pgfqpoint{5.489375in}{3.333122in}}%
\pgfpathlineto{\pgfqpoint{5.497014in}{3.339578in}}%
\pgfpathlineto{\pgfqpoint{5.504641in}{3.345833in}}%
\pgfpathlineto{\pgfqpoint{5.489488in}{3.327176in}}%
\pgfpathlineto{\pgfqpoint{5.474359in}{3.308712in}}%
\pgfpathlineto{\pgfqpoint{5.459255in}{3.290439in}}%
\pgfpathlineto{\pgfqpoint{5.444174in}{3.272358in}}%
\pgfpathlineto{\pgfqpoint{5.436538in}{3.265923in}}%
\pgfpathlineto{\pgfqpoint{5.428891in}{3.259296in}}%
\pgfpathlineto{\pgfqpoint{5.421232in}{3.252475in}}%
\pgfpathlineto{\pgfqpoint{5.413563in}{3.245459in}}%
\pgfpathclose%
\pgfusepath{fill}%
\end{pgfscope}%
\begin{pgfscope}%
\pgfpathrectangle{\pgfqpoint{1.150000in}{0.150000in}}{\pgfqpoint{5.700000in}{5.700000in}}%
\pgfusepath{clip}%
\pgfsetbuttcap%
\pgfsetroundjoin%
\definecolor{currentfill}{rgb}{0.269944,0.014625,0.341379}%
\pgfsetfillcolor{currentfill}%
\pgfsetfillopacity{0.800000}%
\pgfsetlinewidth{0.000000pt}%
\definecolor{currentstroke}{rgb}{0.000000,0.000000,0.000000}%
\pgfsetstrokecolor{currentstroke}%
\pgfsetdash{}{0pt}%
\pgfpathmoveto{\pgfqpoint{3.437517in}{1.179368in}}%
\pgfpathlineto{\pgfqpoint{3.451608in}{1.174294in}}%
\pgfpathlineto{\pgfqpoint{3.465702in}{1.169404in}}%
\pgfpathlineto{\pgfqpoint{3.479800in}{1.164697in}}%
\pgfpathlineto{\pgfqpoint{3.493903in}{1.160173in}}%
\pgfpathlineto{\pgfqpoint{3.502377in}{1.166110in}}%
\pgfpathlineto{\pgfqpoint{3.510840in}{1.172345in}}%
\pgfpathlineto{\pgfqpoint{3.519293in}{1.178870in}}%
\pgfpathlineto{\pgfqpoint{3.527735in}{1.185676in}}%
\pgfpathlineto{\pgfqpoint{3.513657in}{1.189408in}}%
\pgfpathlineto{\pgfqpoint{3.499584in}{1.193323in}}%
\pgfpathlineto{\pgfqpoint{3.485516in}{1.197421in}}%
\pgfpathlineto{\pgfqpoint{3.471452in}{1.201702in}}%
\pgfpathlineto{\pgfqpoint{3.462985in}{1.195676in}}%
\pgfpathlineto{\pgfqpoint{3.454507in}{1.189939in}}%
\pgfpathlineto{\pgfqpoint{3.446018in}{1.184501in}}%
\pgfpathlineto{\pgfqpoint{3.437517in}{1.179368in}}%
\pgfpathclose%
\pgfusepath{fill}%
\end{pgfscope}%
\begin{pgfscope}%
\pgfpathrectangle{\pgfqpoint{1.150000in}{0.150000in}}{\pgfqpoint{5.700000in}{5.700000in}}%
\pgfusepath{clip}%
\pgfsetbuttcap%
\pgfsetroundjoin%
\definecolor{currentfill}{rgb}{0.279574,0.170599,0.479997}%
\pgfsetfillcolor{currentfill}%
\pgfsetfillopacity{0.800000}%
\pgfsetlinewidth{0.000000pt}%
\definecolor{currentstroke}{rgb}{0.000000,0.000000,0.000000}%
\pgfsetstrokecolor{currentstroke}%
\pgfsetdash{}{0pt}%
\pgfpathmoveto{\pgfqpoint{4.065930in}{1.477233in}}%
\pgfpathlineto{\pgfqpoint{4.080148in}{1.482089in}}%
\pgfpathlineto{\pgfqpoint{4.094379in}{1.487123in}}%
\pgfpathlineto{\pgfqpoint{4.108622in}{1.492335in}}%
\pgfpathlineto{\pgfqpoint{4.122876in}{1.497725in}}%
\pgfpathlineto{\pgfqpoint{4.131085in}{1.513371in}}%
\pgfpathlineto{\pgfqpoint{4.139290in}{1.529071in}}%
\pgfpathlineto{\pgfqpoint{4.147491in}{1.544818in}}%
\pgfpathlineto{\pgfqpoint{4.155688in}{1.560608in}}%
\pgfpathlineto{\pgfqpoint{4.141433in}{1.554638in}}%
\pgfpathlineto{\pgfqpoint{4.127190in}{1.548846in}}%
\pgfpathlineto{\pgfqpoint{4.112960in}{1.543233in}}%
\pgfpathlineto{\pgfqpoint{4.098741in}{1.537799in}}%
\pgfpathlineto{\pgfqpoint{4.090545in}{1.522577in}}%
\pgfpathlineto{\pgfqpoint{4.082344in}{1.507405in}}%
\pgfpathlineto{\pgfqpoint{4.074139in}{1.492288in}}%
\pgfpathlineto{\pgfqpoint{4.065930in}{1.477233in}}%
\pgfpathclose%
\pgfusepath{fill}%
\end{pgfscope}%
\begin{pgfscope}%
\pgfpathrectangle{\pgfqpoint{1.150000in}{0.150000in}}{\pgfqpoint{5.700000in}{5.700000in}}%
\pgfusepath{clip}%
\pgfsetbuttcap%
\pgfsetroundjoin%
\definecolor{currentfill}{rgb}{0.277941,0.056324,0.381191}%
\pgfsetfillcolor{currentfill}%
\pgfsetfillopacity{0.800000}%
\pgfsetlinewidth{0.000000pt}%
\definecolor{currentstroke}{rgb}{0.000000,0.000000,0.000000}%
\pgfsetstrokecolor{currentstroke}%
\pgfsetdash{}{0pt}%
\pgfpathmoveto{\pgfqpoint{3.763877in}{1.236540in}}%
\pgfpathlineto{\pgfqpoint{3.778003in}{1.236642in}}%
\pgfpathlineto{\pgfqpoint{3.792138in}{1.236922in}}%
\pgfpathlineto{\pgfqpoint{3.806280in}{1.237380in}}%
\pgfpathlineto{\pgfqpoint{3.820432in}{1.238016in}}%
\pgfpathlineto{\pgfqpoint{3.828734in}{1.249851in}}%
\pgfpathlineto{\pgfqpoint{3.837030in}{1.261862in}}%
\pgfpathlineto{\pgfqpoint{3.845320in}{1.274042in}}%
\pgfpathlineto{\pgfqpoint{3.853604in}{1.286385in}}%
\pgfpathlineto{\pgfqpoint{3.839463in}{1.285049in}}%
\pgfpathlineto{\pgfqpoint{3.825330in}{1.283892in}}%
\pgfpathlineto{\pgfqpoint{3.811207in}{1.282914in}}%
\pgfpathlineto{\pgfqpoint{3.797092in}{1.282114in}}%
\pgfpathlineto{\pgfqpoint{3.788798in}{1.270458in}}%
\pgfpathlineto{\pgfqpoint{3.780497in}{1.258973in}}%
\pgfpathlineto{\pgfqpoint{3.772190in}{1.247665in}}%
\pgfpathlineto{\pgfqpoint{3.763877in}{1.236540in}}%
\pgfpathclose%
\pgfusepath{fill}%
\end{pgfscope}%
\begin{pgfscope}%
\pgfpathrectangle{\pgfqpoint{1.150000in}{0.150000in}}{\pgfqpoint{5.700000in}{5.700000in}}%
\pgfusepath{clip}%
\pgfsetbuttcap%
\pgfsetroundjoin%
\definecolor{currentfill}{rgb}{0.212395,0.359683,0.551710}%
\pgfsetfillcolor{currentfill}%
\pgfsetfillopacity{0.800000}%
\pgfsetlinewidth{0.000000pt}%
\definecolor{currentstroke}{rgb}{0.000000,0.000000,0.000000}%
\pgfsetstrokecolor{currentstroke}%
\pgfsetdash{}{0pt}%
\pgfpathmoveto{\pgfqpoint{4.433469in}{1.951456in}}%
\pgfpathlineto{\pgfqpoint{4.447872in}{1.961488in}}%
\pgfpathlineto{\pgfqpoint{4.462290in}{1.971701in}}%
\pgfpathlineto{\pgfqpoint{4.476724in}{1.982096in}}%
\pgfpathlineto{\pgfqpoint{4.491174in}{1.992674in}}%
\pgfpathlineto{\pgfqpoint{4.499303in}{2.009466in}}%
\pgfpathlineto{\pgfqpoint{4.507428in}{2.026178in}}%
\pgfpathlineto{\pgfqpoint{4.515549in}{2.042806in}}%
\pgfpathlineto{\pgfqpoint{4.523665in}{2.059346in}}%
\pgfpathlineto{\pgfqpoint{4.509208in}{2.048374in}}%
\pgfpathlineto{\pgfqpoint{4.494767in}{2.037585in}}%
\pgfpathlineto{\pgfqpoint{4.480343in}{2.026978in}}%
\pgfpathlineto{\pgfqpoint{4.465935in}{2.016554in}}%
\pgfpathlineto{\pgfqpoint{4.457825in}{2.000395in}}%
\pgfpathlineto{\pgfqpoint{4.449711in}{1.984156in}}%
\pgfpathlineto{\pgfqpoint{4.441592in}{1.967842in}}%
\pgfpathlineto{\pgfqpoint{4.433469in}{1.951456in}}%
\pgfpathclose%
\pgfusepath{fill}%
\end{pgfscope}%
\begin{pgfscope}%
\pgfpathrectangle{\pgfqpoint{1.150000in}{0.150000in}}{\pgfqpoint{5.700000in}{5.700000in}}%
\pgfusepath{clip}%
\pgfsetbuttcap%
\pgfsetroundjoin%
\definecolor{currentfill}{rgb}{0.468053,0.818921,0.323998}%
\pgfsetfillcolor{currentfill}%
\pgfsetfillopacity{0.800000}%
\pgfsetlinewidth{0.000000pt}%
\definecolor{currentstroke}{rgb}{0.000000,0.000000,0.000000}%
\pgfsetstrokecolor{currentstroke}%
\pgfsetdash{}{0pt}%
\pgfpathmoveto{\pgfqpoint{5.626003in}{3.463709in}}%
\pgfpathlineto{\pgfqpoint{5.641255in}{3.482882in}}%
\pgfpathlineto{\pgfqpoint{5.656532in}{3.502247in}}%
\pgfpathlineto{\pgfqpoint{5.671834in}{3.521806in}}%
\pgfpathlineto{\pgfqpoint{5.687162in}{3.541559in}}%
\pgfpathlineto{\pgfqpoint{5.694657in}{3.545591in}}%
\pgfpathlineto{\pgfqpoint{5.702139in}{3.549432in}}%
\pgfpathlineto{\pgfqpoint{5.709608in}{3.553084in}}%
\pgfpathlineto{\pgfqpoint{5.717065in}{3.556551in}}%
\pgfpathlineto{\pgfqpoint{5.701753in}{3.537080in}}%
\pgfpathlineto{\pgfqpoint{5.686466in}{3.517801in}}%
\pgfpathlineto{\pgfqpoint{5.671204in}{3.498714in}}%
\pgfpathlineto{\pgfqpoint{5.655968in}{3.479820in}}%
\pgfpathlineto{\pgfqpoint{5.648494in}{3.476060in}}%
\pgfpathlineto{\pgfqpoint{5.641009in}{3.472124in}}%
\pgfpathlineto{\pgfqpoint{5.633512in}{3.468008in}}%
\pgfpathlineto{\pgfqpoint{5.626003in}{3.463709in}}%
\pgfpathclose%
\pgfusepath{fill}%
\end{pgfscope}%
\begin{pgfscope}%
\pgfpathrectangle{\pgfqpoint{1.150000in}{0.150000in}}{\pgfqpoint{5.700000in}{5.700000in}}%
\pgfusepath{clip}%
\pgfsetbuttcap%
\pgfsetroundjoin%
\definecolor{currentfill}{rgb}{0.273809,0.031497,0.358853}%
\pgfsetfillcolor{currentfill}%
\pgfsetfillopacity{0.800000}%
\pgfsetlinewidth{0.000000pt}%
\definecolor{currentstroke}{rgb}{0.000000,0.000000,0.000000}%
\pgfsetstrokecolor{currentstroke}%
\pgfsetdash{}{0pt}%
\pgfpathmoveto{\pgfqpoint{3.674065in}{1.198320in}}%
\pgfpathlineto{\pgfqpoint{3.688176in}{1.196977in}}%
\pgfpathlineto{\pgfqpoint{3.702294in}{1.195813in}}%
\pgfpathlineto{\pgfqpoint{3.716419in}{1.194827in}}%
\pgfpathlineto{\pgfqpoint{3.730552in}{1.194020in}}%
\pgfpathlineto{\pgfqpoint{3.738894in}{1.204340in}}%
\pgfpathlineto{\pgfqpoint{3.747229in}{1.214871in}}%
\pgfpathlineto{\pgfqpoint{3.755556in}{1.225607in}}%
\pgfpathlineto{\pgfqpoint{3.763877in}{1.236540in}}%
\pgfpathlineto{\pgfqpoint{3.749758in}{1.236618in}}%
\pgfpathlineto{\pgfqpoint{3.735647in}{1.236874in}}%
\pgfpathlineto{\pgfqpoint{3.721544in}{1.237309in}}%
\pgfpathlineto{\pgfqpoint{3.707448in}{1.237924in}}%
\pgfpathlineto{\pgfqpoint{3.699114in}{1.227707in}}%
\pgfpathlineto{\pgfqpoint{3.690772in}{1.217696in}}%
\pgfpathlineto{\pgfqpoint{3.682422in}{1.207898in}}%
\pgfpathlineto{\pgfqpoint{3.674065in}{1.198320in}}%
\pgfpathclose%
\pgfusepath{fill}%
\end{pgfscope}%
\begin{pgfscope}%
\pgfpathrectangle{\pgfqpoint{1.150000in}{0.150000in}}{\pgfqpoint{5.700000in}{5.700000in}}%
\pgfusepath{clip}%
\pgfsetbuttcap%
\pgfsetroundjoin%
\definecolor{currentfill}{rgb}{0.281446,0.084320,0.407414}%
\pgfsetfillcolor{currentfill}%
\pgfsetfillopacity{0.800000}%
\pgfsetlinewidth{0.000000pt}%
\definecolor{currentstroke}{rgb}{0.000000,0.000000,0.000000}%
\pgfsetstrokecolor{currentstroke}%
\pgfsetdash{}{0pt}%
\pgfpathmoveto{\pgfqpoint{3.853604in}{1.286385in}}%
\pgfpathlineto{\pgfqpoint{3.867754in}{1.287898in}}%
\pgfpathlineto{\pgfqpoint{3.881913in}{1.289589in}}%
\pgfpathlineto{\pgfqpoint{3.896082in}{1.291459in}}%
\pgfpathlineto{\pgfqpoint{3.910260in}{1.293505in}}%
\pgfpathlineto{\pgfqpoint{3.918530in}{1.306684in}}%
\pgfpathlineto{\pgfqpoint{3.926795in}{1.320005in}}%
\pgfpathlineto{\pgfqpoint{3.935054in}{1.333460in}}%
\pgfpathlineto{\pgfqpoint{3.943309in}{1.347043in}}%
\pgfpathlineto{\pgfqpoint{3.929137in}{1.344327in}}%
\pgfpathlineto{\pgfqpoint{3.914976in}{1.341788in}}%
\pgfpathlineto{\pgfqpoint{3.900824in}{1.339428in}}%
\pgfpathlineto{\pgfqpoint{3.886682in}{1.337246in}}%
\pgfpathlineto{\pgfqpoint{3.878421in}{1.324320in}}%
\pgfpathlineto{\pgfqpoint{3.870154in}{1.311530in}}%
\pgfpathlineto{\pgfqpoint{3.861882in}{1.298883in}}%
\pgfpathlineto{\pgfqpoint{3.853604in}{1.286385in}}%
\pgfpathclose%
\pgfusepath{fill}%
\end{pgfscope}%
\begin{pgfscope}%
\pgfpathrectangle{\pgfqpoint{1.150000in}{0.150000in}}{\pgfqpoint{5.700000in}{5.700000in}}%
\pgfusepath{clip}%
\pgfsetbuttcap%
\pgfsetroundjoin%
\definecolor{currentfill}{rgb}{0.183898,0.422383,0.556944}%
\pgfsetfillcolor{currentfill}%
\pgfsetfillopacity{0.800000}%
\pgfsetlinewidth{0.000000pt}%
\definecolor{currentstroke}{rgb}{0.000000,0.000000,0.000000}%
\pgfsetstrokecolor{currentstroke}%
\pgfsetdash{}{0pt}%
\pgfpathmoveto{\pgfqpoint{4.556085in}{2.124569in}}%
\pgfpathlineto{\pgfqpoint{4.570566in}{2.136086in}}%
\pgfpathlineto{\pgfqpoint{4.585064in}{2.147787in}}%
\pgfpathlineto{\pgfqpoint{4.599579in}{2.159671in}}%
\pgfpathlineto{\pgfqpoint{4.614111in}{2.171739in}}%
\pgfpathlineto{\pgfqpoint{4.622212in}{2.188135in}}%
\pgfpathlineto{\pgfqpoint{4.630307in}{2.204417in}}%
\pgfpathlineto{\pgfqpoint{4.638398in}{2.220581in}}%
\pgfpathlineto{\pgfqpoint{4.646484in}{2.236624in}}%
\pgfpathlineto{\pgfqpoint{4.631944in}{2.224226in}}%
\pgfpathlineto{\pgfqpoint{4.617422in}{2.212012in}}%
\pgfpathlineto{\pgfqpoint{4.602918in}{2.199983in}}%
\pgfpathlineto{\pgfqpoint{4.588431in}{2.188138in}}%
\pgfpathlineto{\pgfqpoint{4.580352in}{2.172411in}}%
\pgfpathlineto{\pgfqpoint{4.572268in}{2.156572in}}%
\pgfpathlineto{\pgfqpoint{4.564179in}{2.140624in}}%
\pgfpathlineto{\pgfqpoint{4.556085in}{2.124569in}}%
\pgfpathclose%
\pgfusepath{fill}%
\end{pgfscope}%
\begin{pgfscope}%
\pgfpathrectangle{\pgfqpoint{1.150000in}{0.150000in}}{\pgfqpoint{5.700000in}{5.700000in}}%
\pgfusepath{clip}%
\pgfsetbuttcap%
\pgfsetroundjoin%
\definecolor{currentfill}{rgb}{0.271305,0.019942,0.347269}%
\pgfsetfillcolor{currentfill}%
\pgfsetfillopacity{0.800000}%
\pgfsetlinewidth{0.000000pt}%
\definecolor{currentstroke}{rgb}{0.000000,0.000000,0.000000}%
\pgfsetstrokecolor{currentstroke}%
\pgfsetdash{}{0pt}%
\pgfpathmoveto{\pgfqpoint{3.584099in}{1.172569in}}%
\pgfpathlineto{\pgfqpoint{3.598203in}{1.169746in}}%
\pgfpathlineto{\pgfqpoint{3.612314in}{1.167102in}}%
\pgfpathlineto{\pgfqpoint{3.626430in}{1.164639in}}%
\pgfpathlineto{\pgfqpoint{3.640553in}{1.162355in}}%
\pgfpathlineto{\pgfqpoint{3.648944in}{1.170980in}}%
\pgfpathlineto{\pgfqpoint{3.657326in}{1.179854in}}%
\pgfpathlineto{\pgfqpoint{3.665699in}{1.188970in}}%
\pgfpathlineto{\pgfqpoint{3.674065in}{1.198320in}}%
\pgfpathlineto{\pgfqpoint{3.659960in}{1.199843in}}%
\pgfpathlineto{\pgfqpoint{3.645863in}{1.201546in}}%
\pgfpathlineto{\pgfqpoint{3.631771in}{1.203430in}}%
\pgfpathlineto{\pgfqpoint{3.617686in}{1.205493in}}%
\pgfpathlineto{\pgfqpoint{3.609303in}{1.196891in}}%
\pgfpathlineto{\pgfqpoint{3.600911in}{1.188531in}}%
\pgfpathlineto{\pgfqpoint{3.592509in}{1.180421in}}%
\pgfpathlineto{\pgfqpoint{3.584099in}{1.172569in}}%
\pgfpathclose%
\pgfusepath{fill}%
\end{pgfscope}%
\begin{pgfscope}%
\pgfpathrectangle{\pgfqpoint{1.150000in}{0.150000in}}{\pgfqpoint{5.700000in}{5.700000in}}%
\pgfusepath{clip}%
\pgfsetbuttcap%
\pgfsetroundjoin%
\definecolor{currentfill}{rgb}{0.175707,0.697900,0.491033}%
\pgfsetfillcolor{currentfill}%
\pgfsetfillopacity{0.800000}%
\pgfsetlinewidth{0.000000pt}%
\definecolor{currentstroke}{rgb}{0.000000,0.000000,0.000000}%
\pgfsetstrokecolor{currentstroke}%
\pgfsetdash{}{0pt}%
\pgfpathmoveto{\pgfqpoint{1.838626in}{3.146516in}}%
\pgfpathlineto{\pgfqpoint{1.853591in}{3.113178in}}%
\pgfpathlineto{\pgfqpoint{1.868531in}{3.080223in}}%
\pgfpathlineto{\pgfqpoint{1.883447in}{3.047648in}}%
\pgfpathlineto{\pgfqpoint{1.898340in}{3.015447in}}%
\pgfpathlineto{\pgfqpoint{1.908449in}{2.997620in}}%
\pgfpathlineto{\pgfqpoint{1.918519in}{2.980368in}}%
\pgfpathlineto{\pgfqpoint{1.928549in}{2.963681in}}%
\pgfpathlineto{\pgfqpoint{1.938542in}{2.947551in}}%
\pgfpathlineto{\pgfqpoint{1.923744in}{2.978816in}}%
\pgfpathlineto{\pgfqpoint{1.908924in}{3.010454in}}%
\pgfpathlineto{\pgfqpoint{1.894081in}{3.042468in}}%
\pgfpathlineto{\pgfqpoint{1.879214in}{3.074861in}}%
\pgfpathlineto{\pgfqpoint{1.869127in}{3.091913in}}%
\pgfpathlineto{\pgfqpoint{1.859001in}{3.109533in}}%
\pgfpathlineto{\pgfqpoint{1.848834in}{3.127730in}}%
\pgfpathlineto{\pgfqpoint{1.838626in}{3.146516in}}%
\pgfpathclose%
\pgfusepath{fill}%
\end{pgfscope}%
\begin{pgfscope}%
\pgfpathrectangle{\pgfqpoint{1.150000in}{0.150000in}}{\pgfqpoint{5.700000in}{5.700000in}}%
\pgfusepath{clip}%
\pgfsetbuttcap%
\pgfsetroundjoin%
\definecolor{currentfill}{rgb}{0.232815,0.732247,0.459277}%
\pgfsetfillcolor{currentfill}%
\pgfsetfillopacity{0.800000}%
\pgfsetlinewidth{0.000000pt}%
\definecolor{currentstroke}{rgb}{0.000000,0.000000,0.000000}%
\pgfsetstrokecolor{currentstroke}%
\pgfsetdash{}{0pt}%
\pgfpathmoveto{\pgfqpoint{5.291652in}{3.110090in}}%
\pgfpathlineto{\pgfqpoint{5.306659in}{3.127793in}}%
\pgfpathlineto{\pgfqpoint{5.321689in}{3.145687in}}%
\pgfpathlineto{\pgfqpoint{5.336743in}{3.163773in}}%
\pgfpathlineto{\pgfqpoint{5.351820in}{3.182051in}}%
\pgfpathlineto{\pgfqpoint{5.359575in}{3.190700in}}%
\pgfpathlineto{\pgfqpoint{5.367320in}{3.199139in}}%
\pgfpathlineto{\pgfqpoint{5.375054in}{3.207370in}}%
\pgfpathlineto{\pgfqpoint{5.382778in}{3.215394in}}%
\pgfpathlineto{\pgfqpoint{5.367705in}{3.197209in}}%
\pgfpathlineto{\pgfqpoint{5.352657in}{3.179216in}}%
\pgfpathlineto{\pgfqpoint{5.337631in}{3.161414in}}%
\pgfpathlineto{\pgfqpoint{5.322629in}{3.143803in}}%
\pgfpathlineto{\pgfqpoint{5.314900in}{3.135672in}}%
\pgfpathlineto{\pgfqpoint{5.307161in}{3.127344in}}%
\pgfpathlineto{\pgfqpoint{5.299412in}{3.118817in}}%
\pgfpathlineto{\pgfqpoint{5.291652in}{3.110090in}}%
\pgfpathclose%
\pgfusepath{fill}%
\end{pgfscope}%
\begin{pgfscope}%
\pgfpathrectangle{\pgfqpoint{1.150000in}{0.150000in}}{\pgfqpoint{5.700000in}{5.700000in}}%
\pgfusepath{clip}%
\pgfsetbuttcap%
\pgfsetroundjoin%
\definecolor{currentfill}{rgb}{0.283197,0.115680,0.436115}%
\pgfsetfillcolor{currentfill}%
\pgfsetfillopacity{0.800000}%
\pgfsetlinewidth{0.000000pt}%
\definecolor{currentstroke}{rgb}{0.000000,0.000000,0.000000}%
\pgfsetstrokecolor{currentstroke}%
\pgfsetdash{}{0pt}%
\pgfpathmoveto{\pgfqpoint{3.943309in}{1.347043in}}%
\pgfpathlineto{\pgfqpoint{3.957490in}{1.349938in}}%
\pgfpathlineto{\pgfqpoint{3.971682in}{1.353010in}}%
\pgfpathlineto{\pgfqpoint{3.985885in}{1.356259in}}%
\pgfpathlineto{\pgfqpoint{4.000097in}{1.359686in}}%
\pgfpathlineto{\pgfqpoint{4.008342in}{1.374043in}}%
\pgfpathlineto{\pgfqpoint{4.016583in}{1.388509in}}%
\pgfpathlineto{\pgfqpoint{4.024818in}{1.403076in}}%
\pgfpathlineto{\pgfqpoint{4.033049in}{1.417739in}}%
\pgfpathlineto{\pgfqpoint{4.018840in}{1.413672in}}%
\pgfpathlineto{\pgfqpoint{4.004641in}{1.409782in}}%
\pgfpathlineto{\pgfqpoint{3.990454in}{1.406071in}}%
\pgfpathlineto{\pgfqpoint{3.976277in}{1.402537in}}%
\pgfpathlineto{\pgfqpoint{3.968042in}{1.388502in}}%
\pgfpathlineto{\pgfqpoint{3.959803in}{1.374571in}}%
\pgfpathlineto{\pgfqpoint{3.951558in}{1.360749in}}%
\pgfpathlineto{\pgfqpoint{3.943309in}{1.347043in}}%
\pgfpathclose%
\pgfusepath{fill}%
\end{pgfscope}%
\begin{pgfscope}%
\pgfpathrectangle{\pgfqpoint{1.150000in}{0.150000in}}{\pgfqpoint{5.700000in}{5.700000in}}%
\pgfusepath{clip}%
\pgfsetbuttcap%
\pgfsetroundjoin%
\definecolor{currentfill}{rgb}{0.545524,0.838039,0.275626}%
\pgfsetfillcolor{currentfill}%
\pgfsetfillopacity{0.800000}%
\pgfsetlinewidth{0.000000pt}%
\definecolor{currentstroke}{rgb}{0.000000,0.000000,0.000000}%
\pgfsetstrokecolor{currentstroke}%
\pgfsetdash{}{0pt}%
\pgfpathmoveto{\pgfqpoint{5.717065in}{3.556551in}}%
\pgfpathlineto{\pgfqpoint{5.732403in}{3.576216in}}%
\pgfpathlineto{\pgfqpoint{5.747767in}{3.596075in}}%
\pgfpathlineto{\pgfqpoint{5.763156in}{3.616127in}}%
\pgfpathlineto{\pgfqpoint{5.770588in}{3.619184in}}%
\pgfpathlineto{\pgfqpoint{5.778007in}{3.622056in}}%
\pgfpathlineto{\pgfqpoint{5.785413in}{3.624746in}}%
\pgfpathlineto{\pgfqpoint{5.792807in}{3.627259in}}%
\pgfpathlineto{\pgfqpoint{5.777435in}{3.607525in}}%
\pgfpathlineto{\pgfqpoint{5.762090in}{3.587985in}}%
\pgfpathlineto{\pgfqpoint{5.746770in}{3.568637in}}%
\pgfpathlineto{\pgfqpoint{5.739362in}{3.565876in}}%
\pgfpathlineto{\pgfqpoint{5.731942in}{3.562944in}}%
\pgfpathlineto{\pgfqpoint{5.724510in}{3.559837in}}%
\pgfpathlineto{\pgfqpoint{5.717065in}{3.556551in}}%
\pgfpathclose%
\pgfusepath{fill}%
\end{pgfscope}%
\begin{pgfscope}%
\pgfpathrectangle{\pgfqpoint{1.150000in}{0.150000in}}{\pgfqpoint{5.700000in}{5.700000in}}%
\pgfusepath{clip}%
\pgfsetbuttcap%
\pgfsetroundjoin%
\definecolor{currentfill}{rgb}{0.159194,0.482237,0.558073}%
\pgfsetfillcolor{currentfill}%
\pgfsetfillopacity{0.800000}%
\pgfsetlinewidth{0.000000pt}%
\definecolor{currentstroke}{rgb}{0.000000,0.000000,0.000000}%
\pgfsetstrokecolor{currentstroke}%
\pgfsetdash{}{0pt}%
\pgfpathmoveto{\pgfqpoint{4.678775in}{2.299537in}}%
\pgfpathlineto{\pgfqpoint{4.693339in}{2.312416in}}%
\pgfpathlineto{\pgfqpoint{4.707922in}{2.325480in}}%
\pgfpathlineto{\pgfqpoint{4.722523in}{2.338729in}}%
\pgfpathlineto{\pgfqpoint{4.737143in}{2.352164in}}%
\pgfpathlineto{\pgfqpoint{4.745209in}{2.367840in}}%
\pgfpathlineto{\pgfqpoint{4.753270in}{2.383372in}}%
\pgfpathlineto{\pgfqpoint{4.761325in}{2.398758in}}%
\pgfpathlineto{\pgfqpoint{4.769374in}{2.413995in}}%
\pgfpathlineto{\pgfqpoint{4.754747in}{2.400297in}}%
\pgfpathlineto{\pgfqpoint{4.740139in}{2.386785in}}%
\pgfpathlineto{\pgfqpoint{4.725550in}{2.373459in}}%
\pgfpathlineto{\pgfqpoint{4.710979in}{2.360318in}}%
\pgfpathlineto{\pgfqpoint{4.702937in}{2.345331in}}%
\pgfpathlineto{\pgfqpoint{4.694888in}{2.330203in}}%
\pgfpathlineto{\pgfqpoint{4.686834in}{2.314938in}}%
\pgfpathlineto{\pgfqpoint{4.678775in}{2.299537in}}%
\pgfpathclose%
\pgfusepath{fill}%
\end{pgfscope}%
\begin{pgfscope}%
\pgfpathrectangle{\pgfqpoint{1.150000in}{0.150000in}}{\pgfqpoint{5.700000in}{5.700000in}}%
\pgfusepath{clip}%
\pgfsetbuttcap%
\pgfsetroundjoin%
\definecolor{currentfill}{rgb}{0.248629,0.278775,0.534556}%
\pgfsetfillcolor{currentfill}%
\pgfsetfillopacity{0.800000}%
\pgfsetlinewidth{0.000000pt}%
\definecolor{currentstroke}{rgb}{0.000000,0.000000,0.000000}%
\pgfsetstrokecolor{currentstroke}%
\pgfsetdash{}{0pt}%
\pgfpathmoveto{\pgfqpoint{4.278292in}{1.717786in}}%
\pgfpathlineto{\pgfqpoint{4.292619in}{1.725723in}}%
\pgfpathlineto{\pgfqpoint{4.306960in}{1.733839in}}%
\pgfpathlineto{\pgfqpoint{4.321315in}{1.742135in}}%
\pgfpathlineto{\pgfqpoint{4.335685in}{1.750611in}}%
\pgfpathlineto{\pgfqpoint{4.343855in}{1.767555in}}%
\pgfpathlineto{\pgfqpoint{4.352021in}{1.784475in}}%
\pgfpathlineto{\pgfqpoint{4.360183in}{1.801368in}}%
\pgfpathlineto{\pgfqpoint{4.368342in}{1.818229in}}%
\pgfpathlineto{\pgfqpoint{4.353966in}{1.809263in}}%
\pgfpathlineto{\pgfqpoint{4.339606in}{1.800477in}}%
\pgfpathlineto{\pgfqpoint{4.325260in}{1.791872in}}%
\pgfpathlineto{\pgfqpoint{4.310929in}{1.783447in}}%
\pgfpathlineto{\pgfqpoint{4.302775in}{1.767064in}}%
\pgfpathlineto{\pgfqpoint{4.294618in}{1.750656in}}%
\pgfpathlineto{\pgfqpoint{4.286457in}{1.734229in}}%
\pgfpathlineto{\pgfqpoint{4.278292in}{1.717786in}}%
\pgfpathclose%
\pgfusepath{fill}%
\end{pgfscope}%
\begin{pgfscope}%
\pgfpathrectangle{\pgfqpoint{1.150000in}{0.150000in}}{\pgfqpoint{5.700000in}{5.700000in}}%
\pgfusepath{clip}%
\pgfsetbuttcap%
\pgfsetroundjoin%
\definecolor{currentfill}{rgb}{0.270595,0.214069,0.507052}%
\pgfsetfillcolor{currentfill}%
\pgfsetfillopacity{0.800000}%
\pgfsetlinewidth{0.000000pt}%
\definecolor{currentstroke}{rgb}{0.000000,0.000000,0.000000}%
\pgfsetstrokecolor{currentstroke}%
\pgfsetdash{}{0pt}%
\pgfpathmoveto{\pgfqpoint{4.155688in}{1.560608in}}%
\pgfpathlineto{\pgfqpoint{4.169955in}{1.566758in}}%
\pgfpathlineto{\pgfqpoint{4.184236in}{1.573086in}}%
\pgfpathlineto{\pgfqpoint{4.198530in}{1.579593in}}%
\pgfpathlineto{\pgfqpoint{4.212836in}{1.586278in}}%
\pgfpathlineto{\pgfqpoint{4.221031in}{1.602668in}}%
\pgfpathlineto{\pgfqpoint{4.229222in}{1.619082in}}%
\pgfpathlineto{\pgfqpoint{4.237410in}{1.635514in}}%
\pgfpathlineto{\pgfqpoint{4.245594in}{1.651961in}}%
\pgfpathlineto{\pgfqpoint{4.231284in}{1.644724in}}%
\pgfpathlineto{\pgfqpoint{4.216988in}{1.637666in}}%
\pgfpathlineto{\pgfqpoint{4.202706in}{1.630788in}}%
\pgfpathlineto{\pgfqpoint{4.188436in}{1.624089in}}%
\pgfpathlineto{\pgfqpoint{4.180255in}{1.608181in}}%
\pgfpathlineto{\pgfqpoint{4.172070in}{1.592295in}}%
\pgfpathlineto{\pgfqpoint{4.163881in}{1.576436in}}%
\pgfpathlineto{\pgfqpoint{4.155688in}{1.560608in}}%
\pgfpathclose%
\pgfusepath{fill}%
\end{pgfscope}%
\begin{pgfscope}%
\pgfpathrectangle{\pgfqpoint{1.150000in}{0.150000in}}{\pgfqpoint{5.700000in}{5.700000in}}%
\pgfusepath{clip}%
\pgfsetbuttcap%
\pgfsetroundjoin%
\definecolor{currentfill}{rgb}{0.137770,0.537492,0.554906}%
\pgfsetfillcolor{currentfill}%
\pgfsetfillopacity{0.800000}%
\pgfsetlinewidth{0.000000pt}%
\definecolor{currentstroke}{rgb}{0.000000,0.000000,0.000000}%
\pgfsetstrokecolor{currentstroke}%
\pgfsetdash{}{0pt}%
\pgfpathmoveto{\pgfqpoint{4.801510in}{2.473415in}}%
\pgfpathlineto{\pgfqpoint{4.816162in}{2.487528in}}%
\pgfpathlineto{\pgfqpoint{4.830833in}{2.501828in}}%
\pgfpathlineto{\pgfqpoint{4.845524in}{2.516314in}}%
\pgfpathlineto{\pgfqpoint{4.860235in}{2.530988in}}%
\pgfpathlineto{\pgfqpoint{4.868260in}{2.545660in}}%
\pgfpathlineto{\pgfqpoint{4.876277in}{2.560163in}}%
\pgfpathlineto{\pgfqpoint{4.884288in}{2.574497in}}%
\pgfpathlineto{\pgfqpoint{4.892292in}{2.588658in}}%
\pgfpathlineto{\pgfqpoint{4.877576in}{2.573790in}}%
\pgfpathlineto{\pgfqpoint{4.862879in}{2.559109in}}%
\pgfpathlineto{\pgfqpoint{4.848202in}{2.544615in}}%
\pgfpathlineto{\pgfqpoint{4.833544in}{2.530308in}}%
\pgfpathlineto{\pgfqpoint{4.825545in}{2.516327in}}%
\pgfpathlineto{\pgfqpoint{4.817540in}{2.502184in}}%
\pgfpathlineto{\pgfqpoint{4.809528in}{2.487879in}}%
\pgfpathlineto{\pgfqpoint{4.801510in}{2.473415in}}%
\pgfpathclose%
\pgfusepath{fill}%
\end{pgfscope}%
\begin{pgfscope}%
\pgfpathrectangle{\pgfqpoint{1.150000in}{0.150000in}}{\pgfqpoint{5.700000in}{5.700000in}}%
\pgfusepath{clip}%
\pgfsetbuttcap%
\pgfsetroundjoin%
\definecolor{currentfill}{rgb}{0.166383,0.690856,0.496502}%
\pgfsetfillcolor{currentfill}%
\pgfsetfillopacity{0.800000}%
\pgfsetlinewidth{0.000000pt}%
\definecolor{currentstroke}{rgb}{0.000000,0.000000,0.000000}%
\pgfsetstrokecolor{currentstroke}%
\pgfsetdash{}{0pt}%
\pgfpathmoveto{\pgfqpoint{5.169397in}{2.963724in}}%
\pgfpathlineto{\pgfqpoint{5.184318in}{2.980739in}}%
\pgfpathlineto{\pgfqpoint{5.199261in}{2.997944in}}%
\pgfpathlineto{\pgfqpoint{5.214227in}{3.015340in}}%
\pgfpathlineto{\pgfqpoint{5.229215in}{3.032927in}}%
\pgfpathlineto{\pgfqpoint{5.237054in}{3.043295in}}%
\pgfpathlineto{\pgfqpoint{5.244884in}{3.053454in}}%
\pgfpathlineto{\pgfqpoint{5.252703in}{3.063407in}}%
\pgfpathlineto{\pgfqpoint{5.260513in}{3.073153in}}%
\pgfpathlineto{\pgfqpoint{5.245526in}{3.055585in}}%
\pgfpathlineto{\pgfqpoint{5.230561in}{3.038208in}}%
\pgfpathlineto{\pgfqpoint{5.215620in}{3.021022in}}%
\pgfpathlineto{\pgfqpoint{5.200700in}{3.004027in}}%
\pgfpathlineto{\pgfqpoint{5.192889in}{2.994248in}}%
\pgfpathlineto{\pgfqpoint{5.185067in}{2.984272in}}%
\pgfpathlineto{\pgfqpoint{5.177237in}{2.974097in}}%
\pgfpathlineto{\pgfqpoint{5.169397in}{2.963724in}}%
\pgfpathclose%
\pgfusepath{fill}%
\end{pgfscope}%
\begin{pgfscope}%
\pgfpathrectangle{\pgfqpoint{1.150000in}{0.150000in}}{\pgfqpoint{5.700000in}{5.700000in}}%
\pgfusepath{clip}%
\pgfsetbuttcap%
\pgfsetroundjoin%
\definecolor{currentfill}{rgb}{0.395174,0.797475,0.367757}%
\pgfsetfillcolor{currentfill}%
\pgfsetfillopacity{0.800000}%
\pgfsetlinewidth{0.000000pt}%
\definecolor{currentstroke}{rgb}{0.000000,0.000000,0.000000}%
\pgfsetstrokecolor{currentstroke}%
\pgfsetdash{}{0pt}%
\pgfpathmoveto{\pgfqpoint{5.504641in}{3.345833in}}%
\pgfpathlineto{\pgfqpoint{5.519818in}{3.364682in}}%
\pgfpathlineto{\pgfqpoint{5.535019in}{3.383724in}}%
\pgfpathlineto{\pgfqpoint{5.550246in}{3.402958in}}%
\pgfpathlineto{\pgfqpoint{5.565497in}{3.422387in}}%
\pgfpathlineto{\pgfqpoint{5.573102in}{3.428252in}}%
\pgfpathlineto{\pgfqpoint{5.580695in}{3.433911in}}%
\pgfpathlineto{\pgfqpoint{5.588277in}{3.439369in}}%
\pgfpathlineto{\pgfqpoint{5.595846in}{3.444626in}}%
\pgfpathlineto{\pgfqpoint{5.580606in}{3.425404in}}%
\pgfpathlineto{\pgfqpoint{5.565390in}{3.406374in}}%
\pgfpathlineto{\pgfqpoint{5.550200in}{3.387537in}}%
\pgfpathlineto{\pgfqpoint{5.535033in}{3.368892in}}%
\pgfpathlineto{\pgfqpoint{5.527452in}{3.363417in}}%
\pgfpathlineto{\pgfqpoint{5.519860in}{3.357750in}}%
\pgfpathlineto{\pgfqpoint{5.512256in}{3.351889in}}%
\pgfpathlineto{\pgfqpoint{5.504641in}{3.345833in}}%
\pgfpathclose%
\pgfusepath{fill}%
\end{pgfscope}%
\begin{pgfscope}%
\pgfpathrectangle{\pgfqpoint{1.150000in}{0.150000in}}{\pgfqpoint{5.700000in}{5.700000in}}%
\pgfusepath{clip}%
\pgfsetbuttcap%
\pgfsetroundjoin%
\definecolor{currentfill}{rgb}{0.220057,0.343307,0.549413}%
\pgfsetfillcolor{currentfill}%
\pgfsetfillopacity{0.800000}%
\pgfsetlinewidth{0.000000pt}%
\definecolor{currentstroke}{rgb}{0.000000,0.000000,0.000000}%
\pgfsetstrokecolor{currentstroke}%
\pgfsetdash{}{0pt}%
\pgfpathmoveto{\pgfqpoint{4.400937in}{1.885266in}}%
\pgfpathlineto{\pgfqpoint{4.415334in}{1.894872in}}%
\pgfpathlineto{\pgfqpoint{4.429745in}{1.904659in}}%
\pgfpathlineto{\pgfqpoint{4.444173in}{1.914627in}}%
\pgfpathlineto{\pgfqpoint{4.458616in}{1.924777in}}%
\pgfpathlineto{\pgfqpoint{4.466762in}{1.941853in}}%
\pgfpathlineto{\pgfqpoint{4.474903in}{1.958864in}}%
\pgfpathlineto{\pgfqpoint{4.483041in}{1.975805in}}%
\pgfpathlineto{\pgfqpoint{4.491174in}{1.992674in}}%
\pgfpathlineto{\pgfqpoint{4.476724in}{1.982096in}}%
\pgfpathlineto{\pgfqpoint{4.462290in}{1.971701in}}%
\pgfpathlineto{\pgfqpoint{4.447872in}{1.961488in}}%
\pgfpathlineto{\pgfqpoint{4.433469in}{1.951456in}}%
\pgfpathlineto{\pgfqpoint{4.425342in}{1.935002in}}%
\pgfpathlineto{\pgfqpoint{4.417211in}{1.918483in}}%
\pgfpathlineto{\pgfqpoint{4.409076in}{1.901903in}}%
\pgfpathlineto{\pgfqpoint{4.400937in}{1.885266in}}%
\pgfpathclose%
\pgfusepath{fill}%
\end{pgfscope}%
\begin{pgfscope}%
\pgfpathrectangle{\pgfqpoint{1.150000in}{0.150000in}}{\pgfqpoint{5.700000in}{5.700000in}}%
\pgfusepath{clip}%
\pgfsetbuttcap%
\pgfsetroundjoin%
\definecolor{currentfill}{rgb}{0.121148,0.592739,0.544641}%
\pgfsetfillcolor{currentfill}%
\pgfsetfillopacity{0.800000}%
\pgfsetlinewidth{0.000000pt}%
\definecolor{currentstroke}{rgb}{0.000000,0.000000,0.000000}%
\pgfsetstrokecolor{currentstroke}%
\pgfsetdash{}{0pt}%
\pgfpathmoveto{\pgfqpoint{4.924239in}{2.643565in}}%
\pgfpathlineto{\pgfqpoint{4.938981in}{2.658781in}}%
\pgfpathlineto{\pgfqpoint{4.953743in}{2.674185in}}%
\pgfpathlineto{\pgfqpoint{4.968526in}{2.689778in}}%
\pgfpathlineto{\pgfqpoint{4.983330in}{2.705559in}}%
\pgfpathlineto{\pgfqpoint{4.991303in}{2.718983in}}%
\pgfpathlineto{\pgfqpoint{4.999269in}{2.732219in}}%
\pgfpathlineto{\pgfqpoint{5.007226in}{2.745268in}}%
\pgfpathlineto{\pgfqpoint{5.015176in}{2.758127in}}%
\pgfpathlineto{\pgfqpoint{5.000368in}{2.742221in}}%
\pgfpathlineto{\pgfqpoint{4.985582in}{2.726503in}}%
\pgfpathlineto{\pgfqpoint{4.970815in}{2.710975in}}%
\pgfpathlineto{\pgfqpoint{4.956070in}{2.695634in}}%
\pgfpathlineto{\pgfqpoint{4.948123in}{2.682886in}}%
\pgfpathlineto{\pgfqpoint{4.940169in}{2.669958in}}%
\pgfpathlineto{\pgfqpoint{4.932208in}{2.656850in}}%
\pgfpathlineto{\pgfqpoint{4.924239in}{2.643565in}}%
\pgfpathclose%
\pgfusepath{fill}%
\end{pgfscope}%
\begin{pgfscope}%
\pgfpathrectangle{\pgfqpoint{1.150000in}{0.150000in}}{\pgfqpoint{5.700000in}{5.700000in}}%
\pgfusepath{clip}%
\pgfsetbuttcap%
\pgfsetroundjoin%
\definecolor{currentfill}{rgb}{0.126326,0.644107,0.525311}%
\pgfsetfillcolor{currentfill}%
\pgfsetfillopacity{0.800000}%
\pgfsetlinewidth{0.000000pt}%
\definecolor{currentstroke}{rgb}{0.000000,0.000000,0.000000}%
\pgfsetstrokecolor{currentstroke}%
\pgfsetdash{}{0pt}%
\pgfpathmoveto{\pgfqpoint{5.046896in}{2.807665in}}%
\pgfpathlineto{\pgfqpoint{5.061728in}{2.823850in}}%
\pgfpathlineto{\pgfqpoint{5.076581in}{2.840224in}}%
\pgfpathlineto{\pgfqpoint{5.091456in}{2.856788in}}%
\pgfpathlineto{\pgfqpoint{5.106353in}{2.873541in}}%
\pgfpathlineto{\pgfqpoint{5.114264in}{2.885516in}}%
\pgfpathlineto{\pgfqpoint{5.122167in}{2.897289in}}%
\pgfpathlineto{\pgfqpoint{5.130061in}{2.908863in}}%
\pgfpathlineto{\pgfqpoint{5.137947in}{2.920236in}}%
\pgfpathlineto{\pgfqpoint{5.123048in}{2.903429in}}%
\pgfpathlineto{\pgfqpoint{5.108171in}{2.886811in}}%
\pgfpathlineto{\pgfqpoint{5.093317in}{2.870384in}}%
\pgfpathlineto{\pgfqpoint{5.078483in}{2.854146in}}%
\pgfpathlineto{\pgfqpoint{5.070599in}{2.842813in}}%
\pgfpathlineto{\pgfqpoint{5.062706in}{2.831289in}}%
\pgfpathlineto{\pgfqpoint{5.054805in}{2.819573in}}%
\pgfpathlineto{\pgfqpoint{5.046896in}{2.807665in}}%
\pgfpathclose%
\pgfusepath{fill}%
\end{pgfscope}%
\begin{pgfscope}%
\pgfpathrectangle{\pgfqpoint{1.150000in}{0.150000in}}{\pgfqpoint{5.700000in}{5.700000in}}%
\pgfusepath{clip}%
\pgfsetbuttcap%
\pgfsetroundjoin%
\definecolor{currentfill}{rgb}{0.269944,0.014625,0.341379}%
\pgfsetfillcolor{currentfill}%
\pgfsetfillopacity{0.800000}%
\pgfsetlinewidth{0.000000pt}%
\definecolor{currentstroke}{rgb}{0.000000,0.000000,0.000000}%
\pgfsetstrokecolor{currentstroke}%
\pgfsetdash{}{0pt}%
\pgfpathmoveto{\pgfqpoint{3.493903in}{1.160173in}}%
\pgfpathlineto{\pgfqpoint{3.508010in}{1.155831in}}%
\pgfpathlineto{\pgfqpoint{3.522122in}{1.151672in}}%
\pgfpathlineto{\pgfqpoint{3.536239in}{1.147693in}}%
\pgfpathlineto{\pgfqpoint{3.550361in}{1.143896in}}%
\pgfpathlineto{\pgfqpoint{3.558810in}{1.150639in}}%
\pgfpathlineto{\pgfqpoint{3.567250in}{1.157670in}}%
\pgfpathlineto{\pgfqpoint{3.575679in}{1.164983in}}%
\pgfpathlineto{\pgfqpoint{3.584099in}{1.172569in}}%
\pgfpathlineto{\pgfqpoint{3.570000in}{1.175574in}}%
\pgfpathlineto{\pgfqpoint{3.555906in}{1.178760in}}%
\pgfpathlineto{\pgfqpoint{3.541818in}{1.182127in}}%
\pgfpathlineto{\pgfqpoint{3.527735in}{1.185676in}}%
\pgfpathlineto{\pgfqpoint{3.519293in}{1.178870in}}%
\pgfpathlineto{\pgfqpoint{3.510840in}{1.172345in}}%
\pgfpathlineto{\pgfqpoint{3.502377in}{1.166110in}}%
\pgfpathlineto{\pgfqpoint{3.493903in}{1.160173in}}%
\pgfpathclose%
\pgfusepath{fill}%
\end{pgfscope}%
\begin{pgfscope}%
\pgfpathrectangle{\pgfqpoint{1.150000in}{0.150000in}}{\pgfqpoint{5.700000in}{5.700000in}}%
\pgfusepath{clip}%
\pgfsetbuttcap%
\pgfsetroundjoin%
\definecolor{currentfill}{rgb}{0.281412,0.155834,0.469201}%
\pgfsetfillcolor{currentfill}%
\pgfsetfillopacity{0.800000}%
\pgfsetlinewidth{0.000000pt}%
\definecolor{currentstroke}{rgb}{0.000000,0.000000,0.000000}%
\pgfsetstrokecolor{currentstroke}%
\pgfsetdash{}{0pt}%
\pgfpathmoveto{\pgfqpoint{4.033049in}{1.417739in}}%
\pgfpathlineto{\pgfqpoint{4.047270in}{1.421985in}}%
\pgfpathlineto{\pgfqpoint{4.061502in}{1.426408in}}%
\pgfpathlineto{\pgfqpoint{4.075745in}{1.431009in}}%
\pgfpathlineto{\pgfqpoint{4.090000in}{1.435787in}}%
\pgfpathlineto{\pgfqpoint{4.098226in}{1.451163in}}%
\pgfpathlineto{\pgfqpoint{4.106447in}{1.466615in}}%
\pgfpathlineto{\pgfqpoint{4.114663in}{1.482138in}}%
\pgfpathlineto{\pgfqpoint{4.122876in}{1.497725in}}%
\pgfpathlineto{\pgfqpoint{4.108622in}{1.492335in}}%
\pgfpathlineto{\pgfqpoint{4.094379in}{1.487123in}}%
\pgfpathlineto{\pgfqpoint{4.080148in}{1.482089in}}%
\pgfpathlineto{\pgfqpoint{4.065930in}{1.477233in}}%
\pgfpathlineto{\pgfqpoint{4.057716in}{1.462245in}}%
\pgfpathlineto{\pgfqpoint{4.049498in}{1.447330in}}%
\pgfpathlineto{\pgfqpoint{4.041276in}{1.432493in}}%
\pgfpathlineto{\pgfqpoint{4.033049in}{1.417739in}}%
\pgfpathclose%
\pgfusepath{fill}%
\end{pgfscope}%
\begin{pgfscope}%
\pgfpathrectangle{\pgfqpoint{1.150000in}{0.150000in}}{\pgfqpoint{5.700000in}{5.700000in}}%
\pgfusepath{clip}%
\pgfsetbuttcap%
\pgfsetroundjoin%
\definecolor{currentfill}{rgb}{0.190631,0.407061,0.556089}%
\pgfsetfillcolor{currentfill}%
\pgfsetfillopacity{0.800000}%
\pgfsetlinewidth{0.000000pt}%
\definecolor{currentstroke}{rgb}{0.000000,0.000000,0.000000}%
\pgfsetstrokecolor{currentstroke}%
\pgfsetdash{}{0pt}%
\pgfpathmoveto{\pgfqpoint{4.523665in}{2.059346in}}%
\pgfpathlineto{\pgfqpoint{4.538139in}{2.070502in}}%
\pgfpathlineto{\pgfqpoint{4.552629in}{2.081840in}}%
\pgfpathlineto{\pgfqpoint{4.567137in}{2.093361in}}%
\pgfpathlineto{\pgfqpoint{4.581662in}{2.105066in}}%
\pgfpathlineto{\pgfqpoint{4.589781in}{2.121891in}}%
\pgfpathlineto{\pgfqpoint{4.597896in}{2.138614in}}%
\pgfpathlineto{\pgfqpoint{4.606006in}{2.155231in}}%
\pgfpathlineto{\pgfqpoint{4.614111in}{2.171739in}}%
\pgfpathlineto{\pgfqpoint{4.599579in}{2.159671in}}%
\pgfpathlineto{\pgfqpoint{4.585064in}{2.147787in}}%
\pgfpathlineto{\pgfqpoint{4.570566in}{2.136086in}}%
\pgfpathlineto{\pgfqpoint{4.556085in}{2.124569in}}%
\pgfpathlineto{\pgfqpoint{4.547987in}{2.108411in}}%
\pgfpathlineto{\pgfqpoint{4.539884in}{2.092152in}}%
\pgfpathlineto{\pgfqpoint{4.531777in}{2.075796in}}%
\pgfpathlineto{\pgfqpoint{4.523665in}{2.059346in}}%
\pgfpathclose%
\pgfusepath{fill}%
\end{pgfscope}%
\begin{pgfscope}%
\pgfpathrectangle{\pgfqpoint{1.150000in}{0.150000in}}{\pgfqpoint{5.700000in}{5.700000in}}%
\pgfusepath{clip}%
\pgfsetbuttcap%
\pgfsetroundjoin%
\definecolor{currentfill}{rgb}{0.276022,0.044167,0.370164}%
\pgfsetfillcolor{currentfill}%
\pgfsetfillopacity{0.800000}%
\pgfsetlinewidth{0.000000pt}%
\definecolor{currentstroke}{rgb}{0.000000,0.000000,0.000000}%
\pgfsetstrokecolor{currentstroke}%
\pgfsetdash{}{0pt}%
\pgfpathmoveto{\pgfqpoint{3.730552in}{1.194020in}}%
\pgfpathlineto{\pgfqpoint{3.744692in}{1.193392in}}%
\pgfpathlineto{\pgfqpoint{3.758839in}{1.192942in}}%
\pgfpathlineto{\pgfqpoint{3.772995in}{1.192669in}}%
\pgfpathlineto{\pgfqpoint{3.787158in}{1.192574in}}%
\pgfpathlineto{\pgfqpoint{3.795486in}{1.203635in}}%
\pgfpathlineto{\pgfqpoint{3.803808in}{1.214901in}}%
\pgfpathlineto{\pgfqpoint{3.812123in}{1.226364in}}%
\pgfpathlineto{\pgfqpoint{3.820432in}{1.238016in}}%
\pgfpathlineto{\pgfqpoint{3.806280in}{1.237380in}}%
\pgfpathlineto{\pgfqpoint{3.792138in}{1.236922in}}%
\pgfpathlineto{\pgfqpoint{3.778003in}{1.236642in}}%
\pgfpathlineto{\pgfqpoint{3.763877in}{1.236540in}}%
\pgfpathlineto{\pgfqpoint{3.755556in}{1.225607in}}%
\pgfpathlineto{\pgfqpoint{3.747229in}{1.214871in}}%
\pgfpathlineto{\pgfqpoint{3.738894in}{1.204340in}}%
\pgfpathlineto{\pgfqpoint{3.730552in}{1.194020in}}%
\pgfpathclose%
\pgfusepath{fill}%
\end{pgfscope}%
\begin{pgfscope}%
\pgfpathrectangle{\pgfqpoint{1.150000in}{0.150000in}}{\pgfqpoint{5.700000in}{5.700000in}}%
\pgfusepath{clip}%
\pgfsetbuttcap%
\pgfsetroundjoin%
\definecolor{currentfill}{rgb}{0.280267,0.073417,0.397163}%
\pgfsetfillcolor{currentfill}%
\pgfsetfillopacity{0.800000}%
\pgfsetlinewidth{0.000000pt}%
\definecolor{currentstroke}{rgb}{0.000000,0.000000,0.000000}%
\pgfsetstrokecolor{currentstroke}%
\pgfsetdash{}{0pt}%
\pgfpathmoveto{\pgfqpoint{3.820432in}{1.238016in}}%
\pgfpathlineto{\pgfqpoint{3.834591in}{1.238829in}}%
\pgfpathlineto{\pgfqpoint{3.848760in}{1.239821in}}%
\pgfpathlineto{\pgfqpoint{3.862937in}{1.240989in}}%
\pgfpathlineto{\pgfqpoint{3.877124in}{1.242334in}}%
\pgfpathlineto{\pgfqpoint{3.885416in}{1.254882in}}%
\pgfpathlineto{\pgfqpoint{3.893703in}{1.267597in}}%
\pgfpathlineto{\pgfqpoint{3.901984in}{1.280474in}}%
\pgfpathlineto{\pgfqpoint{3.910260in}{1.293505in}}%
\pgfpathlineto{\pgfqpoint{3.896082in}{1.291459in}}%
\pgfpathlineto{\pgfqpoint{3.881913in}{1.289589in}}%
\pgfpathlineto{\pgfqpoint{3.867754in}{1.287898in}}%
\pgfpathlineto{\pgfqpoint{3.853604in}{1.286385in}}%
\pgfpathlineto{\pgfqpoint{3.845320in}{1.274042in}}%
\pgfpathlineto{\pgfqpoint{3.837030in}{1.261862in}}%
\pgfpathlineto{\pgfqpoint{3.828734in}{1.249851in}}%
\pgfpathlineto{\pgfqpoint{3.820432in}{1.238016in}}%
\pgfpathclose%
\pgfusepath{fill}%
\end{pgfscope}%
\begin{pgfscope}%
\pgfpathrectangle{\pgfqpoint{1.150000in}{0.150000in}}{\pgfqpoint{5.700000in}{5.700000in}}%
\pgfusepath{clip}%
\pgfsetbuttcap%
\pgfsetroundjoin%
\definecolor{currentfill}{rgb}{0.165117,0.467423,0.558141}%
\pgfsetfillcolor{currentfill}%
\pgfsetfillopacity{0.800000}%
\pgfsetlinewidth{0.000000pt}%
\definecolor{currentstroke}{rgb}{0.000000,0.000000,0.000000}%
\pgfsetstrokecolor{currentstroke}%
\pgfsetdash{}{0pt}%
\pgfpathmoveto{\pgfqpoint{4.646484in}{2.236624in}}%
\pgfpathlineto{\pgfqpoint{4.661041in}{2.249206in}}%
\pgfpathlineto{\pgfqpoint{4.675617in}{2.261974in}}%
\pgfpathlineto{\pgfqpoint{4.690211in}{2.274926in}}%
\pgfpathlineto{\pgfqpoint{4.704823in}{2.288063in}}%
\pgfpathlineto{\pgfqpoint{4.712911in}{2.304292in}}%
\pgfpathlineto{\pgfqpoint{4.720994in}{2.320387in}}%
\pgfpathlineto{\pgfqpoint{4.729071in}{2.336345in}}%
\pgfpathlineto{\pgfqpoint{4.737143in}{2.352164in}}%
\pgfpathlineto{\pgfqpoint{4.722523in}{2.338729in}}%
\pgfpathlineto{\pgfqpoint{4.707922in}{2.325480in}}%
\pgfpathlineto{\pgfqpoint{4.693339in}{2.312416in}}%
\pgfpathlineto{\pgfqpoint{4.678775in}{2.299537in}}%
\pgfpathlineto{\pgfqpoint{4.670710in}{2.284003in}}%
\pgfpathlineto{\pgfqpoint{4.662640in}{2.268338in}}%
\pgfpathlineto{\pgfqpoint{4.654564in}{2.252544in}}%
\pgfpathlineto{\pgfqpoint{4.646484in}{2.236624in}}%
\pgfpathclose%
\pgfusepath{fill}%
\end{pgfscope}%
\begin{pgfscope}%
\pgfpathrectangle{\pgfqpoint{1.150000in}{0.150000in}}{\pgfqpoint{5.700000in}{5.700000in}}%
\pgfusepath{clip}%
\pgfsetbuttcap%
\pgfsetroundjoin%
\definecolor{currentfill}{rgb}{0.272594,0.025563,0.353093}%
\pgfsetfillcolor{currentfill}%
\pgfsetfillopacity{0.800000}%
\pgfsetlinewidth{0.000000pt}%
\definecolor{currentstroke}{rgb}{0.000000,0.000000,0.000000}%
\pgfsetstrokecolor{currentstroke}%
\pgfsetdash{}{0pt}%
\pgfpathmoveto{\pgfqpoint{3.640553in}{1.162355in}}%
\pgfpathlineto{\pgfqpoint{3.654682in}{1.160251in}}%
\pgfpathlineto{\pgfqpoint{3.668817in}{1.158326in}}%
\pgfpathlineto{\pgfqpoint{3.682959in}{1.156579in}}%
\pgfpathlineto{\pgfqpoint{3.697108in}{1.155011in}}%
\pgfpathlineto{\pgfqpoint{3.705481in}{1.164408in}}%
\pgfpathlineto{\pgfqpoint{3.713846in}{1.174047in}}%
\pgfpathlineto{\pgfqpoint{3.722203in}{1.183921in}}%
\pgfpathlineto{\pgfqpoint{3.730552in}{1.194020in}}%
\pgfpathlineto{\pgfqpoint{3.716419in}{1.194827in}}%
\pgfpathlineto{\pgfqpoint{3.702294in}{1.195813in}}%
\pgfpathlineto{\pgfqpoint{3.688176in}{1.196977in}}%
\pgfpathlineto{\pgfqpoint{3.674065in}{1.198320in}}%
\pgfpathlineto{\pgfqpoint{3.665699in}{1.188970in}}%
\pgfpathlineto{\pgfqpoint{3.657326in}{1.179854in}}%
\pgfpathlineto{\pgfqpoint{3.648944in}{1.170980in}}%
\pgfpathlineto{\pgfqpoint{3.640553in}{1.162355in}}%
\pgfpathclose%
\pgfusepath{fill}%
\end{pgfscope}%
\begin{pgfscope}%
\pgfpathrectangle{\pgfqpoint{1.150000in}{0.150000in}}{\pgfqpoint{5.700000in}{5.700000in}}%
\pgfusepath{clip}%
\pgfsetbuttcap%
\pgfsetroundjoin%
\definecolor{currentfill}{rgb}{0.311925,0.767822,0.415586}%
\pgfsetfillcolor{currentfill}%
\pgfsetfillopacity{0.800000}%
\pgfsetlinewidth{0.000000pt}%
\definecolor{currentstroke}{rgb}{0.000000,0.000000,0.000000}%
\pgfsetstrokecolor{currentstroke}%
\pgfsetdash{}{0pt}%
\pgfpathmoveto{\pgfqpoint{5.382778in}{3.215394in}}%
\pgfpathlineto{\pgfqpoint{5.397874in}{3.233771in}}%
\pgfpathlineto{\pgfqpoint{5.412994in}{3.252341in}}%
\pgfpathlineto{\pgfqpoint{5.428137in}{3.271103in}}%
\pgfpathlineto{\pgfqpoint{5.443306in}{3.290058in}}%
\pgfpathlineto{\pgfqpoint{5.451012in}{3.297760in}}%
\pgfpathlineto{\pgfqpoint{5.458708in}{3.305251in}}%
\pgfpathlineto{\pgfqpoint{5.466392in}{3.312530in}}%
\pgfpathlineto{\pgfqpoint{5.474064in}{3.319601in}}%
\pgfpathlineto{\pgfqpoint{5.458903in}{3.300777in}}%
\pgfpathlineto{\pgfqpoint{5.443766in}{3.282145in}}%
\pgfpathlineto{\pgfqpoint{5.428653in}{3.263706in}}%
\pgfpathlineto{\pgfqpoint{5.413563in}{3.245459in}}%
\pgfpathlineto{\pgfqpoint{5.405883in}{3.238244in}}%
\pgfpathlineto{\pgfqpoint{5.398192in}{3.230830in}}%
\pgfpathlineto{\pgfqpoint{5.390490in}{3.223214in}}%
\pgfpathlineto{\pgfqpoint{5.382778in}{3.215394in}}%
\pgfpathclose%
\pgfusepath{fill}%
\end{pgfscope}%
\begin{pgfscope}%
\pgfpathrectangle{\pgfqpoint{1.150000in}{0.150000in}}{\pgfqpoint{5.700000in}{5.700000in}}%
\pgfusepath{clip}%
\pgfsetbuttcap%
\pgfsetroundjoin%
\definecolor{currentfill}{rgb}{0.282656,0.100196,0.422160}%
\pgfsetfillcolor{currentfill}%
\pgfsetfillopacity{0.800000}%
\pgfsetlinewidth{0.000000pt}%
\definecolor{currentstroke}{rgb}{0.000000,0.000000,0.000000}%
\pgfsetstrokecolor{currentstroke}%
\pgfsetdash{}{0pt}%
\pgfpathmoveto{\pgfqpoint{3.910260in}{1.293505in}}%
\pgfpathlineto{\pgfqpoint{3.924447in}{1.295729in}}%
\pgfpathlineto{\pgfqpoint{3.938645in}{1.298130in}}%
\pgfpathlineto{\pgfqpoint{3.952853in}{1.300708in}}%
\pgfpathlineto{\pgfqpoint{3.967070in}{1.303463in}}%
\pgfpathlineto{\pgfqpoint{3.975334in}{1.317325in}}%
\pgfpathlineto{\pgfqpoint{3.983594in}{1.331320in}}%
\pgfpathlineto{\pgfqpoint{3.991848in}{1.345443in}}%
\pgfpathlineto{\pgfqpoint{4.000097in}{1.359686in}}%
\pgfpathlineto{\pgfqpoint{3.985885in}{1.356259in}}%
\pgfpathlineto{\pgfqpoint{3.971682in}{1.353010in}}%
\pgfpathlineto{\pgfqpoint{3.957490in}{1.349938in}}%
\pgfpathlineto{\pgfqpoint{3.943309in}{1.347043in}}%
\pgfpathlineto{\pgfqpoint{3.935054in}{1.333460in}}%
\pgfpathlineto{\pgfqpoint{3.926795in}{1.320005in}}%
\pgfpathlineto{\pgfqpoint{3.918530in}{1.306684in}}%
\pgfpathlineto{\pgfqpoint{3.910260in}{1.293505in}}%
\pgfpathclose%
\pgfusepath{fill}%
\end{pgfscope}%
\begin{pgfscope}%
\pgfpathrectangle{\pgfqpoint{1.150000in}{0.150000in}}{\pgfqpoint{5.700000in}{5.700000in}}%
\pgfusepath{clip}%
\pgfsetbuttcap%
\pgfsetroundjoin%
\definecolor{currentfill}{rgb}{0.255645,0.260703,0.528312}%
\pgfsetfillcolor{currentfill}%
\pgfsetfillopacity{0.800000}%
\pgfsetlinewidth{0.000000pt}%
\definecolor{currentstroke}{rgb}{0.000000,0.000000,0.000000}%
\pgfsetstrokecolor{currentstroke}%
\pgfsetdash{}{0pt}%
\pgfpathmoveto{\pgfqpoint{4.245594in}{1.651961in}}%
\pgfpathlineto{\pgfqpoint{4.259917in}{1.659377in}}%
\pgfpathlineto{\pgfqpoint{4.274254in}{1.666972in}}%
\pgfpathlineto{\pgfqpoint{4.288605in}{1.674747in}}%
\pgfpathlineto{\pgfqpoint{4.302970in}{1.682701in}}%
\pgfpathlineto{\pgfqpoint{4.311154in}{1.699689in}}%
\pgfpathlineto{\pgfqpoint{4.319335in}{1.716674in}}%
\pgfpathlineto{\pgfqpoint{4.327512in}{1.733650in}}%
\pgfpathlineto{\pgfqpoint{4.335685in}{1.750611in}}%
\pgfpathlineto{\pgfqpoint{4.321315in}{1.742135in}}%
\pgfpathlineto{\pgfqpoint{4.306960in}{1.733839in}}%
\pgfpathlineto{\pgfqpoint{4.292619in}{1.725723in}}%
\pgfpathlineto{\pgfqpoint{4.278292in}{1.717786in}}%
\pgfpathlineto{\pgfqpoint{4.270123in}{1.701334in}}%
\pgfpathlineto{\pgfqpoint{4.261950in}{1.684875in}}%
\pgfpathlineto{\pgfqpoint{4.253774in}{1.668416in}}%
\pgfpathlineto{\pgfqpoint{4.245594in}{1.651961in}}%
\pgfpathclose%
\pgfusepath{fill}%
\end{pgfscope}%
\begin{pgfscope}%
\pgfpathrectangle{\pgfqpoint{1.150000in}{0.150000in}}{\pgfqpoint{5.700000in}{5.700000in}}%
\pgfusepath{clip}%
\pgfsetbuttcap%
\pgfsetroundjoin%
\definecolor{currentfill}{rgb}{0.275191,0.194905,0.496005}%
\pgfsetfillcolor{currentfill}%
\pgfsetfillopacity{0.800000}%
\pgfsetlinewidth{0.000000pt}%
\definecolor{currentstroke}{rgb}{0.000000,0.000000,0.000000}%
\pgfsetstrokecolor{currentstroke}%
\pgfsetdash{}{0pt}%
\pgfpathmoveto{\pgfqpoint{4.122876in}{1.497725in}}%
\pgfpathlineto{\pgfqpoint{4.137143in}{1.503293in}}%
\pgfpathlineto{\pgfqpoint{4.151422in}{1.509040in}}%
\pgfpathlineto{\pgfqpoint{4.165714in}{1.514964in}}%
\pgfpathlineto{\pgfqpoint{4.180019in}{1.521066in}}%
\pgfpathlineto{\pgfqpoint{4.188229in}{1.537306in}}%
\pgfpathlineto{\pgfqpoint{4.196435in}{1.553592in}}%
\pgfpathlineto{\pgfqpoint{4.204637in}{1.569918in}}%
\pgfpathlineto{\pgfqpoint{4.212836in}{1.586278in}}%
\pgfpathlineto{\pgfqpoint{4.198530in}{1.579593in}}%
\pgfpathlineto{\pgfqpoint{4.184236in}{1.573086in}}%
\pgfpathlineto{\pgfqpoint{4.169955in}{1.566758in}}%
\pgfpathlineto{\pgfqpoint{4.155688in}{1.560608in}}%
\pgfpathlineto{\pgfqpoint{4.147491in}{1.544818in}}%
\pgfpathlineto{\pgfqpoint{4.139290in}{1.529071in}}%
\pgfpathlineto{\pgfqpoint{4.131085in}{1.513371in}}%
\pgfpathlineto{\pgfqpoint{4.122876in}{1.497725in}}%
\pgfpathclose%
\pgfusepath{fill}%
\end{pgfscope}%
\begin{pgfscope}%
\pgfpathrectangle{\pgfqpoint{1.150000in}{0.150000in}}{\pgfqpoint{5.700000in}{5.700000in}}%
\pgfusepath{clip}%
\pgfsetbuttcap%
\pgfsetroundjoin%
\definecolor{currentfill}{rgb}{0.227802,0.326594,0.546532}%
\pgfsetfillcolor{currentfill}%
\pgfsetfillopacity{0.800000}%
\pgfsetlinewidth{0.000000pt}%
\definecolor{currentstroke}{rgb}{0.000000,0.000000,0.000000}%
\pgfsetstrokecolor{currentstroke}%
\pgfsetdash{}{0pt}%
\pgfpathmoveto{\pgfqpoint{4.368342in}{1.818229in}}%
\pgfpathlineto{\pgfqpoint{4.382732in}{1.827376in}}%
\pgfpathlineto{\pgfqpoint{4.397138in}{1.836704in}}%
\pgfpathlineto{\pgfqpoint{4.411559in}{1.846213in}}%
\pgfpathlineto{\pgfqpoint{4.425995in}{1.855902in}}%
\pgfpathlineto{\pgfqpoint{4.434156in}{1.873199in}}%
\pgfpathlineto{\pgfqpoint{4.442313in}{1.890446in}}%
\pgfpathlineto{\pgfqpoint{4.450467in}{1.907640in}}%
\pgfpathlineto{\pgfqpoint{4.458616in}{1.924777in}}%
\pgfpathlineto{\pgfqpoint{4.444173in}{1.914627in}}%
\pgfpathlineto{\pgfqpoint{4.429745in}{1.904659in}}%
\pgfpathlineto{\pgfqpoint{4.415334in}{1.894872in}}%
\pgfpathlineto{\pgfqpoint{4.400937in}{1.885266in}}%
\pgfpathlineto{\pgfqpoint{4.392794in}{1.868576in}}%
\pgfpathlineto{\pgfqpoint{4.384647in}{1.851837in}}%
\pgfpathlineto{\pgfqpoint{4.376496in}{1.835054in}}%
\pgfpathlineto{\pgfqpoint{4.368342in}{1.818229in}}%
\pgfpathclose%
\pgfusepath{fill}%
\end{pgfscope}%
\begin{pgfscope}%
\pgfpathrectangle{\pgfqpoint{1.150000in}{0.150000in}}{\pgfqpoint{5.700000in}{5.700000in}}%
\pgfusepath{clip}%
\pgfsetbuttcap%
\pgfsetroundjoin%
\definecolor{currentfill}{rgb}{0.487026,0.823929,0.312321}%
\pgfsetfillcolor{currentfill}%
\pgfsetfillopacity{0.800000}%
\pgfsetlinewidth{0.000000pt}%
\definecolor{currentstroke}{rgb}{0.000000,0.000000,0.000000}%
\pgfsetstrokecolor{currentstroke}%
\pgfsetdash{}{0pt}%
\pgfpathmoveto{\pgfqpoint{5.595846in}{3.444626in}}%
\pgfpathlineto{\pgfqpoint{5.611111in}{3.464041in}}%
\pgfpathlineto{\pgfqpoint{5.626402in}{3.483651in}}%
\pgfpathlineto{\pgfqpoint{5.641717in}{3.503454in}}%
\pgfpathlineto{\pgfqpoint{5.657059in}{3.523452in}}%
\pgfpathlineto{\pgfqpoint{5.664604in}{3.528281in}}%
\pgfpathlineto{\pgfqpoint{5.672136in}{3.532907in}}%
\pgfpathlineto{\pgfqpoint{5.679655in}{3.537332in}}%
\pgfpathlineto{\pgfqpoint{5.687162in}{3.541559in}}%
\pgfpathlineto{\pgfqpoint{5.671834in}{3.521806in}}%
\pgfpathlineto{\pgfqpoint{5.656532in}{3.502247in}}%
\pgfpathlineto{\pgfqpoint{5.641255in}{3.482882in}}%
\pgfpathlineto{\pgfqpoint{5.626003in}{3.463709in}}%
\pgfpathlineto{\pgfqpoint{5.618481in}{3.459224in}}%
\pgfpathlineto{\pgfqpoint{5.610948in}{3.454551in}}%
\pgfpathlineto{\pgfqpoint{5.603403in}{3.449686in}}%
\pgfpathlineto{\pgfqpoint{5.595846in}{3.444626in}}%
\pgfpathclose%
\pgfusepath{fill}%
\end{pgfscope}%
\begin{pgfscope}%
\pgfpathrectangle{\pgfqpoint{1.150000in}{0.150000in}}{\pgfqpoint{5.700000in}{5.700000in}}%
\pgfusepath{clip}%
\pgfsetbuttcap%
\pgfsetroundjoin%
\definecolor{currentfill}{rgb}{0.141935,0.526453,0.555991}%
\pgfsetfillcolor{currentfill}%
\pgfsetfillopacity{0.800000}%
\pgfsetlinewidth{0.000000pt}%
\definecolor{currentstroke}{rgb}{0.000000,0.000000,0.000000}%
\pgfsetstrokecolor{currentstroke}%
\pgfsetdash{}{0pt}%
\pgfpathmoveto{\pgfqpoint{4.769374in}{2.413995in}}%
\pgfpathlineto{\pgfqpoint{4.784020in}{2.427879in}}%
\pgfpathlineto{\pgfqpoint{4.798685in}{2.441949in}}%
\pgfpathlineto{\pgfqpoint{4.813369in}{2.456206in}}%
\pgfpathlineto{\pgfqpoint{4.828073in}{2.470650in}}%
\pgfpathlineto{\pgfqpoint{4.836123in}{2.485978in}}%
\pgfpathlineto{\pgfqpoint{4.844167in}{2.501146in}}%
\pgfpathlineto{\pgfqpoint{4.852204in}{2.516150in}}%
\pgfpathlineto{\pgfqpoint{4.860235in}{2.530988in}}%
\pgfpathlineto{\pgfqpoint{4.845524in}{2.516314in}}%
\pgfpathlineto{\pgfqpoint{4.830833in}{2.501828in}}%
\pgfpathlineto{\pgfqpoint{4.816162in}{2.487528in}}%
\pgfpathlineto{\pgfqpoint{4.801510in}{2.473415in}}%
\pgfpathlineto{\pgfqpoint{4.793485in}{2.458793in}}%
\pgfpathlineto{\pgfqpoint{4.785454in}{2.444014in}}%
\pgfpathlineto{\pgfqpoint{4.777417in}{2.429081in}}%
\pgfpathlineto{\pgfqpoint{4.769374in}{2.413995in}}%
\pgfpathclose%
\pgfusepath{fill}%
\end{pgfscope}%
\begin{pgfscope}%
\pgfpathrectangle{\pgfqpoint{1.150000in}{0.150000in}}{\pgfqpoint{5.700000in}{5.700000in}}%
\pgfusepath{clip}%
\pgfsetbuttcap%
\pgfsetroundjoin%
\definecolor{currentfill}{rgb}{0.226397,0.728888,0.462789}%
\pgfsetfillcolor{currentfill}%
\pgfsetfillopacity{0.800000}%
\pgfsetlinewidth{0.000000pt}%
\definecolor{currentstroke}{rgb}{0.000000,0.000000,0.000000}%
\pgfsetstrokecolor{currentstroke}%
\pgfsetdash{}{0pt}%
\pgfpathmoveto{\pgfqpoint{5.260513in}{3.073153in}}%
\pgfpathlineto{\pgfqpoint{5.275523in}{3.090912in}}%
\pgfpathlineto{\pgfqpoint{5.290556in}{3.108862in}}%
\pgfpathlineto{\pgfqpoint{5.305612in}{3.127005in}}%
\pgfpathlineto{\pgfqpoint{5.320692in}{3.145340in}}%
\pgfpathlineto{\pgfqpoint{5.328490in}{3.154837in}}%
\pgfpathlineto{\pgfqpoint{5.336277in}{3.164121in}}%
\pgfpathlineto{\pgfqpoint{5.344054in}{3.173192in}}%
\pgfpathlineto{\pgfqpoint{5.351820in}{3.182051in}}%
\pgfpathlineto{\pgfqpoint{5.336743in}{3.163773in}}%
\pgfpathlineto{\pgfqpoint{5.321689in}{3.145687in}}%
\pgfpathlineto{\pgfqpoint{5.306659in}{3.127793in}}%
\pgfpathlineto{\pgfqpoint{5.291652in}{3.110090in}}%
\pgfpathlineto{\pgfqpoint{5.283882in}{3.101161in}}%
\pgfpathlineto{\pgfqpoint{5.276103in}{3.092029in}}%
\pgfpathlineto{\pgfqpoint{5.268313in}{3.082693in}}%
\pgfpathlineto{\pgfqpoint{5.260513in}{3.073153in}}%
\pgfpathclose%
\pgfusepath{fill}%
\end{pgfscope}%
\begin{pgfscope}%
\pgfpathrectangle{\pgfqpoint{1.150000in}{0.150000in}}{\pgfqpoint{5.700000in}{5.700000in}}%
\pgfusepath{clip}%
\pgfsetbuttcap%
\pgfsetroundjoin%
\definecolor{currentfill}{rgb}{0.271305,0.019942,0.347269}%
\pgfsetfillcolor{currentfill}%
\pgfsetfillopacity{0.800000}%
\pgfsetlinewidth{0.000000pt}%
\definecolor{currentstroke}{rgb}{0.000000,0.000000,0.000000}%
\pgfsetstrokecolor{currentstroke}%
\pgfsetdash{}{0pt}%
\pgfpathmoveto{\pgfqpoint{3.550361in}{1.143896in}}%
\pgfpathlineto{\pgfqpoint{3.564488in}{1.140280in}}%
\pgfpathlineto{\pgfqpoint{3.578620in}{1.136844in}}%
\pgfpathlineto{\pgfqpoint{3.592758in}{1.133588in}}%
\pgfpathlineto{\pgfqpoint{3.606901in}{1.130511in}}%
\pgfpathlineto{\pgfqpoint{3.615328in}{1.138059in}}%
\pgfpathlineto{\pgfqpoint{3.623745in}{1.145887in}}%
\pgfpathlineto{\pgfqpoint{3.632154in}{1.153989in}}%
\pgfpathlineto{\pgfqpoint{3.640553in}{1.162355in}}%
\pgfpathlineto{\pgfqpoint{3.626430in}{1.164639in}}%
\pgfpathlineto{\pgfqpoint{3.612314in}{1.167102in}}%
\pgfpathlineto{\pgfqpoint{3.598203in}{1.169746in}}%
\pgfpathlineto{\pgfqpoint{3.584099in}{1.172569in}}%
\pgfpathlineto{\pgfqpoint{3.575679in}{1.164983in}}%
\pgfpathlineto{\pgfqpoint{3.567250in}{1.157670in}}%
\pgfpathlineto{\pgfqpoint{3.558810in}{1.150639in}}%
\pgfpathlineto{\pgfqpoint{3.550361in}{1.143896in}}%
\pgfpathclose%
\pgfusepath{fill}%
\end{pgfscope}%
\begin{pgfscope}%
\pgfpathrectangle{\pgfqpoint{1.150000in}{0.150000in}}{\pgfqpoint{5.700000in}{5.700000in}}%
\pgfusepath{clip}%
\pgfsetbuttcap%
\pgfsetroundjoin%
\definecolor{currentfill}{rgb}{0.282884,0.135920,0.453427}%
\pgfsetfillcolor{currentfill}%
\pgfsetfillopacity{0.800000}%
\pgfsetlinewidth{0.000000pt}%
\definecolor{currentstroke}{rgb}{0.000000,0.000000,0.000000}%
\pgfsetstrokecolor{currentstroke}%
\pgfsetdash{}{0pt}%
\pgfpathmoveto{\pgfqpoint{4.000097in}{1.359686in}}%
\pgfpathlineto{\pgfqpoint{4.014321in}{1.363290in}}%
\pgfpathlineto{\pgfqpoint{4.028556in}{1.367071in}}%
\pgfpathlineto{\pgfqpoint{4.042801in}{1.371029in}}%
\pgfpathlineto{\pgfqpoint{4.057058in}{1.375164in}}%
\pgfpathlineto{\pgfqpoint{4.065300in}{1.390176in}}%
\pgfpathlineto{\pgfqpoint{4.073538in}{1.405288in}}%
\pgfpathlineto{\pgfqpoint{4.081771in}{1.420493in}}%
\pgfpathlineto{\pgfqpoint{4.090000in}{1.435787in}}%
\pgfpathlineto{\pgfqpoint{4.075745in}{1.431009in}}%
\pgfpathlineto{\pgfqpoint{4.061502in}{1.426408in}}%
\pgfpathlineto{\pgfqpoint{4.047270in}{1.421985in}}%
\pgfpathlineto{\pgfqpoint{4.033049in}{1.417739in}}%
\pgfpathlineto{\pgfqpoint{4.024818in}{1.403076in}}%
\pgfpathlineto{\pgfqpoint{4.016583in}{1.388509in}}%
\pgfpathlineto{\pgfqpoint{4.008342in}{1.374043in}}%
\pgfpathlineto{\pgfqpoint{4.000097in}{1.359686in}}%
\pgfpathclose%
\pgfusepath{fill}%
\end{pgfscope}%
\begin{pgfscope}%
\pgfpathrectangle{\pgfqpoint{1.150000in}{0.150000in}}{\pgfqpoint{5.700000in}{5.700000in}}%
\pgfusepath{clip}%
\pgfsetbuttcap%
\pgfsetroundjoin%
\definecolor{currentfill}{rgb}{0.123463,0.581687,0.547445}%
\pgfsetfillcolor{currentfill}%
\pgfsetfillopacity{0.800000}%
\pgfsetlinewidth{0.000000pt}%
\definecolor{currentstroke}{rgb}{0.000000,0.000000,0.000000}%
\pgfsetstrokecolor{currentstroke}%
\pgfsetdash{}{0pt}%
\pgfpathmoveto{\pgfqpoint{4.892292in}{2.588658in}}%
\pgfpathlineto{\pgfqpoint{4.907029in}{2.603715in}}%
\pgfpathlineto{\pgfqpoint{4.921786in}{2.618959in}}%
\pgfpathlineto{\pgfqpoint{4.936564in}{2.634391in}}%
\pgfpathlineto{\pgfqpoint{4.951362in}{2.650012in}}%
\pgfpathlineto{\pgfqpoint{4.959365in}{2.664175in}}%
\pgfpathlineto{\pgfqpoint{4.967361in}{2.678154in}}%
\pgfpathlineto{\pgfqpoint{4.975349in}{2.691949in}}%
\pgfpathlineto{\pgfqpoint{4.983330in}{2.705559in}}%
\pgfpathlineto{\pgfqpoint{4.968526in}{2.689778in}}%
\pgfpathlineto{\pgfqpoint{4.953743in}{2.674185in}}%
\pgfpathlineto{\pgfqpoint{4.938981in}{2.658781in}}%
\pgfpathlineto{\pgfqpoint{4.924239in}{2.643565in}}%
\pgfpathlineto{\pgfqpoint{4.916263in}{2.630102in}}%
\pgfpathlineto{\pgfqpoint{4.908280in}{2.616462in}}%
\pgfpathlineto{\pgfqpoint{4.900290in}{2.602647in}}%
\pgfpathlineto{\pgfqpoint{4.892292in}{2.588658in}}%
\pgfpathclose%
\pgfusepath{fill}%
\end{pgfscope}%
\begin{pgfscope}%
\pgfpathrectangle{\pgfqpoint{1.150000in}{0.150000in}}{\pgfqpoint{5.700000in}{5.700000in}}%
\pgfusepath{clip}%
\pgfsetbuttcap%
\pgfsetroundjoin%
\definecolor{currentfill}{rgb}{0.197636,0.391528,0.554969}%
\pgfsetfillcolor{currentfill}%
\pgfsetfillopacity{0.800000}%
\pgfsetlinewidth{0.000000pt}%
\definecolor{currentstroke}{rgb}{0.000000,0.000000,0.000000}%
\pgfsetstrokecolor{currentstroke}%
\pgfsetdash{}{0pt}%
\pgfpathmoveto{\pgfqpoint{4.491174in}{1.992674in}}%
\pgfpathlineto{\pgfqpoint{4.505641in}{2.003434in}}%
\pgfpathlineto{\pgfqpoint{4.520124in}{2.014377in}}%
\pgfpathlineto{\pgfqpoint{4.534624in}{2.025502in}}%
\pgfpathlineto{\pgfqpoint{4.549141in}{2.036809in}}%
\pgfpathlineto{\pgfqpoint{4.557277in}{2.054010in}}%
\pgfpathlineto{\pgfqpoint{4.565410in}{2.071122in}}%
\pgfpathlineto{\pgfqpoint{4.573538in}{2.088142in}}%
\pgfpathlineto{\pgfqpoint{4.581662in}{2.105066in}}%
\pgfpathlineto{\pgfqpoint{4.567137in}{2.093361in}}%
\pgfpathlineto{\pgfqpoint{4.552629in}{2.081840in}}%
\pgfpathlineto{\pgfqpoint{4.538139in}{2.070502in}}%
\pgfpathlineto{\pgfqpoint{4.523665in}{2.059346in}}%
\pgfpathlineto{\pgfqpoint{4.515549in}{2.042806in}}%
\pgfpathlineto{\pgfqpoint{4.507428in}{2.026178in}}%
\pgfpathlineto{\pgfqpoint{4.499303in}{2.009466in}}%
\pgfpathlineto{\pgfqpoint{4.491174in}{1.992674in}}%
\pgfpathclose%
\pgfusepath{fill}%
\end{pgfscope}%
\begin{pgfscope}%
\pgfpathrectangle{\pgfqpoint{1.150000in}{0.150000in}}{\pgfqpoint{5.700000in}{5.700000in}}%
\pgfusepath{clip}%
\pgfsetbuttcap%
\pgfsetroundjoin%
\definecolor{currentfill}{rgb}{0.162016,0.687316,0.499129}%
\pgfsetfillcolor{currentfill}%
\pgfsetfillopacity{0.800000}%
\pgfsetlinewidth{0.000000pt}%
\definecolor{currentstroke}{rgb}{0.000000,0.000000,0.000000}%
\pgfsetstrokecolor{currentstroke}%
\pgfsetdash{}{0pt}%
\pgfpathmoveto{\pgfqpoint{5.137947in}{2.920236in}}%
\pgfpathlineto{\pgfqpoint{5.152867in}{2.937233in}}%
\pgfpathlineto{\pgfqpoint{5.167810in}{2.954421in}}%
\pgfpathlineto{\pgfqpoint{5.182775in}{2.971800in}}%
\pgfpathlineto{\pgfqpoint{5.197763in}{2.989370in}}%
\pgfpathlineto{\pgfqpoint{5.205640in}{3.000573in}}%
\pgfpathlineto{\pgfqpoint{5.213508in}{3.011567in}}%
\pgfpathlineto{\pgfqpoint{5.221366in}{3.022351in}}%
\pgfpathlineto{\pgfqpoint{5.229215in}{3.032927in}}%
\pgfpathlineto{\pgfqpoint{5.214227in}{3.015340in}}%
\pgfpathlineto{\pgfqpoint{5.199261in}{2.997944in}}%
\pgfpathlineto{\pgfqpoint{5.184318in}{2.980739in}}%
\pgfpathlineto{\pgfqpoint{5.169397in}{2.963724in}}%
\pgfpathlineto{\pgfqpoint{5.161548in}{2.953152in}}%
\pgfpathlineto{\pgfqpoint{5.153690in}{2.942380in}}%
\pgfpathlineto{\pgfqpoint{5.145823in}{2.931408in}}%
\pgfpathlineto{\pgfqpoint{5.137947in}{2.920236in}}%
\pgfpathclose%
\pgfusepath{fill}%
\end{pgfscope}%
\begin{pgfscope}%
\pgfpathrectangle{\pgfqpoint{1.150000in}{0.150000in}}{\pgfqpoint{5.700000in}{5.700000in}}%
\pgfusepath{clip}%
\pgfsetbuttcap%
\pgfsetroundjoin%
\definecolor{currentfill}{rgb}{0.123444,0.636809,0.528763}%
\pgfsetfillcolor{currentfill}%
\pgfsetfillopacity{0.800000}%
\pgfsetlinewidth{0.000000pt}%
\definecolor{currentstroke}{rgb}{0.000000,0.000000,0.000000}%
\pgfsetstrokecolor{currentstroke}%
\pgfsetdash{}{0pt}%
\pgfpathmoveto{\pgfqpoint{5.015176in}{2.758127in}}%
\pgfpathlineto{\pgfqpoint{5.030005in}{2.774223in}}%
\pgfpathlineto{\pgfqpoint{5.044855in}{2.790507in}}%
\pgfpathlineto{\pgfqpoint{5.059727in}{2.806982in}}%
\pgfpathlineto{\pgfqpoint{5.074621in}{2.823646in}}%
\pgfpathlineto{\pgfqpoint{5.082566in}{2.836419in}}%
\pgfpathlineto{\pgfqpoint{5.090504in}{2.848993in}}%
\pgfpathlineto{\pgfqpoint{5.098432in}{2.861367in}}%
\pgfpathlineto{\pgfqpoint{5.106353in}{2.873541in}}%
\pgfpathlineto{\pgfqpoint{5.091456in}{2.856788in}}%
\pgfpathlineto{\pgfqpoint{5.076581in}{2.840224in}}%
\pgfpathlineto{\pgfqpoint{5.061728in}{2.823850in}}%
\pgfpathlineto{\pgfqpoint{5.046896in}{2.807665in}}%
\pgfpathlineto{\pgfqpoint{5.038978in}{2.795567in}}%
\pgfpathlineto{\pgfqpoint{5.031052in}{2.783277in}}%
\pgfpathlineto{\pgfqpoint{5.023118in}{2.770797in}}%
\pgfpathlineto{\pgfqpoint{5.015176in}{2.758127in}}%
\pgfpathclose%
\pgfusepath{fill}%
\end{pgfscope}%
\begin{pgfscope}%
\pgfpathrectangle{\pgfqpoint{1.150000in}{0.150000in}}{\pgfqpoint{5.700000in}{5.700000in}}%
\pgfusepath{clip}%
\pgfsetbuttcap%
\pgfsetroundjoin%
\definecolor{currentfill}{rgb}{0.565498,0.842430,0.262877}%
\pgfsetfillcolor{currentfill}%
\pgfsetfillopacity{0.800000}%
\pgfsetlinewidth{0.000000pt}%
\definecolor{currentstroke}{rgb}{0.000000,0.000000,0.000000}%
\pgfsetstrokecolor{currentstroke}%
\pgfsetdash{}{0pt}%
\pgfpathmoveto{\pgfqpoint{5.687162in}{3.541559in}}%
\pgfpathlineto{\pgfqpoint{5.702516in}{3.561506in}}%
\pgfpathlineto{\pgfqpoint{5.717895in}{3.581647in}}%
\pgfpathlineto{\pgfqpoint{5.733301in}{3.601983in}}%
\pgfpathlineto{\pgfqpoint{5.740784in}{3.605813in}}%
\pgfpathlineto{\pgfqpoint{5.748254in}{3.609445in}}%
\pgfpathlineto{\pgfqpoint{5.755712in}{3.612882in}}%
\pgfpathlineto{\pgfqpoint{5.763156in}{3.616127in}}%
\pgfpathlineto{\pgfqpoint{5.747767in}{3.596075in}}%
\pgfpathlineto{\pgfqpoint{5.732403in}{3.576216in}}%
\pgfpathlineto{\pgfqpoint{5.717065in}{3.556551in}}%
\pgfpathlineto{\pgfqpoint{5.709608in}{3.553084in}}%
\pgfpathlineto{\pgfqpoint{5.702139in}{3.549432in}}%
\pgfpathlineto{\pgfqpoint{5.694657in}{3.545591in}}%
\pgfpathlineto{\pgfqpoint{5.687162in}{3.541559in}}%
\pgfpathclose%
\pgfusepath{fill}%
\end{pgfscope}%
\begin{pgfscope}%
\pgfpathrectangle{\pgfqpoint{1.150000in}{0.150000in}}{\pgfqpoint{5.700000in}{5.700000in}}%
\pgfusepath{clip}%
\pgfsetbuttcap%
\pgfsetroundjoin%
\definecolor{currentfill}{rgb}{0.171176,0.452530,0.557965}%
\pgfsetfillcolor{currentfill}%
\pgfsetfillopacity{0.800000}%
\pgfsetlinewidth{0.000000pt}%
\definecolor{currentstroke}{rgb}{0.000000,0.000000,0.000000}%
\pgfsetstrokecolor{currentstroke}%
\pgfsetdash{}{0pt}%
\pgfpathmoveto{\pgfqpoint{4.614111in}{2.171739in}}%
\pgfpathlineto{\pgfqpoint{4.628661in}{2.183991in}}%
\pgfpathlineto{\pgfqpoint{4.643229in}{2.196428in}}%
\pgfpathlineto{\pgfqpoint{4.657815in}{2.209048in}}%
\pgfpathlineto{\pgfqpoint{4.672419in}{2.221853in}}%
\pgfpathlineto{\pgfqpoint{4.680527in}{2.238594in}}%
\pgfpathlineto{\pgfqpoint{4.688631in}{2.255210in}}%
\pgfpathlineto{\pgfqpoint{4.696729in}{2.271701in}}%
\pgfpathlineto{\pgfqpoint{4.704823in}{2.288063in}}%
\pgfpathlineto{\pgfqpoint{4.690211in}{2.274926in}}%
\pgfpathlineto{\pgfqpoint{4.675617in}{2.261974in}}%
\pgfpathlineto{\pgfqpoint{4.661041in}{2.249206in}}%
\pgfpathlineto{\pgfqpoint{4.646484in}{2.236624in}}%
\pgfpathlineto{\pgfqpoint{4.638398in}{2.220581in}}%
\pgfpathlineto{\pgfqpoint{4.630307in}{2.204417in}}%
\pgfpathlineto{\pgfqpoint{4.622212in}{2.188135in}}%
\pgfpathlineto{\pgfqpoint{4.614111in}{2.171739in}}%
\pgfpathclose%
\pgfusepath{fill}%
\end{pgfscope}%
\begin{pgfscope}%
\pgfpathrectangle{\pgfqpoint{1.150000in}{0.150000in}}{\pgfqpoint{5.700000in}{5.700000in}}%
\pgfusepath{clip}%
\pgfsetbuttcap%
\pgfsetroundjoin%
\definecolor{currentfill}{rgb}{0.262138,0.242286,0.520837}%
\pgfsetfillcolor{currentfill}%
\pgfsetfillopacity{0.800000}%
\pgfsetlinewidth{0.000000pt}%
\definecolor{currentstroke}{rgb}{0.000000,0.000000,0.000000}%
\pgfsetstrokecolor{currentstroke}%
\pgfsetdash{}{0pt}%
\pgfpathmoveto{\pgfqpoint{4.212836in}{1.586278in}}%
\pgfpathlineto{\pgfqpoint{4.227156in}{1.593142in}}%
\pgfpathlineto{\pgfqpoint{4.241489in}{1.600184in}}%
\pgfpathlineto{\pgfqpoint{4.255836in}{1.607405in}}%
\pgfpathlineto{\pgfqpoint{4.270197in}{1.614805in}}%
\pgfpathlineto{\pgfqpoint{4.278396in}{1.631760in}}%
\pgfpathlineto{\pgfqpoint{4.286591in}{1.648731in}}%
\pgfpathlineto{\pgfqpoint{4.294782in}{1.665713in}}%
\pgfpathlineto{\pgfqpoint{4.302970in}{1.682701in}}%
\pgfpathlineto{\pgfqpoint{4.288605in}{1.674747in}}%
\pgfpathlineto{\pgfqpoint{4.274254in}{1.666972in}}%
\pgfpathlineto{\pgfqpoint{4.259917in}{1.659377in}}%
\pgfpathlineto{\pgfqpoint{4.245594in}{1.651961in}}%
\pgfpathlineto{\pgfqpoint{4.237410in}{1.635514in}}%
\pgfpathlineto{\pgfqpoint{4.229222in}{1.619082in}}%
\pgfpathlineto{\pgfqpoint{4.221031in}{1.602668in}}%
\pgfpathlineto{\pgfqpoint{4.212836in}{1.586278in}}%
\pgfpathclose%
\pgfusepath{fill}%
\end{pgfscope}%
\begin{pgfscope}%
\pgfpathrectangle{\pgfqpoint{1.150000in}{0.150000in}}{\pgfqpoint{5.700000in}{5.700000in}}%
\pgfusepath{clip}%
\pgfsetbuttcap%
\pgfsetroundjoin%
\definecolor{currentfill}{rgb}{0.395174,0.797475,0.367757}%
\pgfsetfillcolor{currentfill}%
\pgfsetfillopacity{0.800000}%
\pgfsetlinewidth{0.000000pt}%
\definecolor{currentstroke}{rgb}{0.000000,0.000000,0.000000}%
\pgfsetstrokecolor{currentstroke}%
\pgfsetdash{}{0pt}%
\pgfpathmoveto{\pgfqpoint{5.474064in}{3.319601in}}%
\pgfpathlineto{\pgfqpoint{5.489250in}{3.338617in}}%
\pgfpathlineto{\pgfqpoint{5.504461in}{3.357828in}}%
\pgfpathlineto{\pgfqpoint{5.519696in}{3.377232in}}%
\pgfpathlineto{\pgfqpoint{5.534956in}{3.396830in}}%
\pgfpathlineto{\pgfqpoint{5.542609in}{3.403538in}}%
\pgfpathlineto{\pgfqpoint{5.550250in}{3.410032in}}%
\pgfpathlineto{\pgfqpoint{5.557880in}{3.416314in}}%
\pgfpathlineto{\pgfqpoint{5.565497in}{3.422387in}}%
\pgfpathlineto{\pgfqpoint{5.550246in}{3.402958in}}%
\pgfpathlineto{\pgfqpoint{5.535019in}{3.383724in}}%
\pgfpathlineto{\pgfqpoint{5.519818in}{3.364682in}}%
\pgfpathlineto{\pgfqpoint{5.504641in}{3.345833in}}%
\pgfpathlineto{\pgfqpoint{5.497014in}{3.339578in}}%
\pgfpathlineto{\pgfqpoint{5.489375in}{3.333122in}}%
\pgfpathlineto{\pgfqpoint{5.481726in}{3.326464in}}%
\pgfpathlineto{\pgfqpoint{5.474064in}{3.319601in}}%
\pgfpathclose%
\pgfusepath{fill}%
\end{pgfscope}%
\begin{pgfscope}%
\pgfpathrectangle{\pgfqpoint{1.150000in}{0.150000in}}{\pgfqpoint{5.700000in}{5.700000in}}%
\pgfusepath{clip}%
\pgfsetbuttcap%
\pgfsetroundjoin%
\definecolor{currentfill}{rgb}{0.278791,0.062145,0.386592}%
\pgfsetfillcolor{currentfill}%
\pgfsetfillopacity{0.800000}%
\pgfsetlinewidth{0.000000pt}%
\definecolor{currentstroke}{rgb}{0.000000,0.000000,0.000000}%
\pgfsetstrokecolor{currentstroke}%
\pgfsetdash{}{0pt}%
\pgfpathmoveto{\pgfqpoint{3.787158in}{1.192574in}}%
\pgfpathlineto{\pgfqpoint{3.801329in}{1.192656in}}%
\pgfpathlineto{\pgfqpoint{3.815509in}{1.192915in}}%
\pgfpathlineto{\pgfqpoint{3.829697in}{1.193351in}}%
\pgfpathlineto{\pgfqpoint{3.843893in}{1.193963in}}%
\pgfpathlineto{\pgfqpoint{3.852210in}{1.205769in}}%
\pgfpathlineto{\pgfqpoint{3.860521in}{1.217771in}}%
\pgfpathlineto{\pgfqpoint{3.868825in}{1.229961in}}%
\pgfpathlineto{\pgfqpoint{3.877124in}{1.242334in}}%
\pgfpathlineto{\pgfqpoint{3.862937in}{1.240989in}}%
\pgfpathlineto{\pgfqpoint{3.848760in}{1.239821in}}%
\pgfpathlineto{\pgfqpoint{3.834591in}{1.238829in}}%
\pgfpathlineto{\pgfqpoint{3.820432in}{1.238016in}}%
\pgfpathlineto{\pgfqpoint{3.812123in}{1.226364in}}%
\pgfpathlineto{\pgfqpoint{3.803808in}{1.214901in}}%
\pgfpathlineto{\pgfqpoint{3.795486in}{1.203635in}}%
\pgfpathlineto{\pgfqpoint{3.787158in}{1.192574in}}%
\pgfpathclose%
\pgfusepath{fill}%
\end{pgfscope}%
\begin{pgfscope}%
\pgfpathrectangle{\pgfqpoint{1.150000in}{0.150000in}}{\pgfqpoint{5.700000in}{5.700000in}}%
\pgfusepath{clip}%
\pgfsetbuttcap%
\pgfsetroundjoin%
\definecolor{currentfill}{rgb}{0.278826,0.175490,0.483397}%
\pgfsetfillcolor{currentfill}%
\pgfsetfillopacity{0.800000}%
\pgfsetlinewidth{0.000000pt}%
\definecolor{currentstroke}{rgb}{0.000000,0.000000,0.000000}%
\pgfsetstrokecolor{currentstroke}%
\pgfsetdash{}{0pt}%
\pgfpathmoveto{\pgfqpoint{4.090000in}{1.435787in}}%
\pgfpathlineto{\pgfqpoint{4.104267in}{1.440743in}}%
\pgfpathlineto{\pgfqpoint{4.118546in}{1.445876in}}%
\pgfpathlineto{\pgfqpoint{4.132837in}{1.451186in}}%
\pgfpathlineto{\pgfqpoint{4.147140in}{1.456674in}}%
\pgfpathlineto{\pgfqpoint{4.155366in}{1.472675in}}%
\pgfpathlineto{\pgfqpoint{4.163587in}{1.488745in}}%
\pgfpathlineto{\pgfqpoint{4.171805in}{1.504877in}}%
\pgfpathlineto{\pgfqpoint{4.180019in}{1.521066in}}%
\pgfpathlineto{\pgfqpoint{4.165714in}{1.514964in}}%
\pgfpathlineto{\pgfqpoint{4.151422in}{1.509040in}}%
\pgfpathlineto{\pgfqpoint{4.137143in}{1.503293in}}%
\pgfpathlineto{\pgfqpoint{4.122876in}{1.497725in}}%
\pgfpathlineto{\pgfqpoint{4.114663in}{1.482138in}}%
\pgfpathlineto{\pgfqpoint{4.106447in}{1.466615in}}%
\pgfpathlineto{\pgfqpoint{4.098226in}{1.451163in}}%
\pgfpathlineto{\pgfqpoint{4.090000in}{1.435787in}}%
\pgfpathclose%
\pgfusepath{fill}%
\end{pgfscope}%
\begin{pgfscope}%
\pgfpathrectangle{\pgfqpoint{1.150000in}{0.150000in}}{\pgfqpoint{5.700000in}{5.700000in}}%
\pgfusepath{clip}%
\pgfsetbuttcap%
\pgfsetroundjoin%
\definecolor{currentfill}{rgb}{0.237441,0.305202,0.541921}%
\pgfsetfillcolor{currentfill}%
\pgfsetfillopacity{0.800000}%
\pgfsetlinewidth{0.000000pt}%
\definecolor{currentstroke}{rgb}{0.000000,0.000000,0.000000}%
\pgfsetstrokecolor{currentstroke}%
\pgfsetdash{}{0pt}%
\pgfpathmoveto{\pgfqpoint{4.335685in}{1.750611in}}%
\pgfpathlineto{\pgfqpoint{4.350070in}{1.759268in}}%
\pgfpathlineto{\pgfqpoint{4.364470in}{1.768104in}}%
\pgfpathlineto{\pgfqpoint{4.378884in}{1.777120in}}%
\pgfpathlineto{\pgfqpoint{4.393314in}{1.786317in}}%
\pgfpathlineto{\pgfqpoint{4.401490in}{1.803765in}}%
\pgfpathlineto{\pgfqpoint{4.409662in}{1.821181in}}%
\pgfpathlineto{\pgfqpoint{4.417830in}{1.838562in}}%
\pgfpathlineto{\pgfqpoint{4.425995in}{1.855902in}}%
\pgfpathlineto{\pgfqpoint{4.411559in}{1.846213in}}%
\pgfpathlineto{\pgfqpoint{4.397138in}{1.836704in}}%
\pgfpathlineto{\pgfqpoint{4.382732in}{1.827376in}}%
\pgfpathlineto{\pgfqpoint{4.368342in}{1.818229in}}%
\pgfpathlineto{\pgfqpoint{4.360183in}{1.801368in}}%
\pgfpathlineto{\pgfqpoint{4.352021in}{1.784475in}}%
\pgfpathlineto{\pgfqpoint{4.343855in}{1.767555in}}%
\pgfpathlineto{\pgfqpoint{4.335685in}{1.750611in}}%
\pgfpathclose%
\pgfusepath{fill}%
\end{pgfscope}%
\begin{pgfscope}%
\pgfpathrectangle{\pgfqpoint{1.150000in}{0.150000in}}{\pgfqpoint{5.700000in}{5.700000in}}%
\pgfusepath{clip}%
\pgfsetbuttcap%
\pgfsetroundjoin%
\definecolor{currentfill}{rgb}{0.274952,0.037752,0.364543}%
\pgfsetfillcolor{currentfill}%
\pgfsetfillopacity{0.800000}%
\pgfsetlinewidth{0.000000pt}%
\definecolor{currentstroke}{rgb}{0.000000,0.000000,0.000000}%
\pgfsetstrokecolor{currentstroke}%
\pgfsetdash{}{0pt}%
\pgfpathmoveto{\pgfqpoint{3.697108in}{1.155011in}}%
\pgfpathlineto{\pgfqpoint{3.711264in}{1.153621in}}%
\pgfpathlineto{\pgfqpoint{3.725426in}{1.152408in}}%
\pgfpathlineto{\pgfqpoint{3.739596in}{1.151373in}}%
\pgfpathlineto{\pgfqpoint{3.753774in}{1.150515in}}%
\pgfpathlineto{\pgfqpoint{3.762131in}{1.160687in}}%
\pgfpathlineto{\pgfqpoint{3.770480in}{1.171092in}}%
\pgfpathlineto{\pgfqpoint{3.778823in}{1.181724in}}%
\pgfpathlineto{\pgfqpoint{3.787158in}{1.192574in}}%
\pgfpathlineto{\pgfqpoint{3.772995in}{1.192669in}}%
\pgfpathlineto{\pgfqpoint{3.758839in}{1.192942in}}%
\pgfpathlineto{\pgfqpoint{3.744692in}{1.193392in}}%
\pgfpathlineto{\pgfqpoint{3.730552in}{1.194020in}}%
\pgfpathlineto{\pgfqpoint{3.722203in}{1.183921in}}%
\pgfpathlineto{\pgfqpoint{3.713846in}{1.174047in}}%
\pgfpathlineto{\pgfqpoint{3.705481in}{1.164408in}}%
\pgfpathlineto{\pgfqpoint{3.697108in}{1.155011in}}%
\pgfpathclose%
\pgfusepath{fill}%
\end{pgfscope}%
\begin{pgfscope}%
\pgfpathrectangle{\pgfqpoint{1.150000in}{0.150000in}}{\pgfqpoint{5.700000in}{5.700000in}}%
\pgfusepath{clip}%
\pgfsetbuttcap%
\pgfsetroundjoin%
\definecolor{currentfill}{rgb}{0.281924,0.089666,0.412415}%
\pgfsetfillcolor{currentfill}%
\pgfsetfillopacity{0.800000}%
\pgfsetlinewidth{0.000000pt}%
\definecolor{currentstroke}{rgb}{0.000000,0.000000,0.000000}%
\pgfsetstrokecolor{currentstroke}%
\pgfsetdash{}{0pt}%
\pgfpathmoveto{\pgfqpoint{3.877124in}{1.242334in}}%
\pgfpathlineto{\pgfqpoint{3.891319in}{1.243856in}}%
\pgfpathlineto{\pgfqpoint{3.905524in}{1.245555in}}%
\pgfpathlineto{\pgfqpoint{3.919738in}{1.247430in}}%
\pgfpathlineto{\pgfqpoint{3.933962in}{1.249481in}}%
\pgfpathlineto{\pgfqpoint{3.942247in}{1.262743in}}%
\pgfpathlineto{\pgfqpoint{3.950527in}{1.276165in}}%
\pgfpathlineto{\pgfqpoint{3.958801in}{1.289740in}}%
\pgfpathlineto{\pgfqpoint{3.967070in}{1.303463in}}%
\pgfpathlineto{\pgfqpoint{3.952853in}{1.300708in}}%
\pgfpathlineto{\pgfqpoint{3.938645in}{1.298130in}}%
\pgfpathlineto{\pgfqpoint{3.924447in}{1.295729in}}%
\pgfpathlineto{\pgfqpoint{3.910260in}{1.293505in}}%
\pgfpathlineto{\pgfqpoint{3.901984in}{1.280474in}}%
\pgfpathlineto{\pgfqpoint{3.893703in}{1.267597in}}%
\pgfpathlineto{\pgfqpoint{3.885416in}{1.254882in}}%
\pgfpathlineto{\pgfqpoint{3.877124in}{1.242334in}}%
\pgfpathclose%
\pgfusepath{fill}%
\end{pgfscope}%
\begin{pgfscope}%
\pgfpathrectangle{\pgfqpoint{1.150000in}{0.150000in}}{\pgfqpoint{5.700000in}{5.700000in}}%
\pgfusepath{clip}%
\pgfsetbuttcap%
\pgfsetroundjoin%
\definecolor{currentfill}{rgb}{0.147607,0.511733,0.557049}%
\pgfsetfillcolor{currentfill}%
\pgfsetfillopacity{0.800000}%
\pgfsetlinewidth{0.000000pt}%
\definecolor{currentstroke}{rgb}{0.000000,0.000000,0.000000}%
\pgfsetstrokecolor{currentstroke}%
\pgfsetdash{}{0pt}%
\pgfpathmoveto{\pgfqpoint{4.737143in}{2.352164in}}%
\pgfpathlineto{\pgfqpoint{4.751782in}{2.365784in}}%
\pgfpathlineto{\pgfqpoint{4.766440in}{2.379591in}}%
\pgfpathlineto{\pgfqpoint{4.781116in}{2.393583in}}%
\pgfpathlineto{\pgfqpoint{4.795812in}{2.407762in}}%
\pgfpathlineto{\pgfqpoint{4.803887in}{2.423715in}}%
\pgfpathlineto{\pgfqpoint{4.811955in}{2.439516in}}%
\pgfpathlineto{\pgfqpoint{4.820017in}{2.455162in}}%
\pgfpathlineto{\pgfqpoint{4.828073in}{2.470650in}}%
\pgfpathlineto{\pgfqpoint{4.813369in}{2.456206in}}%
\pgfpathlineto{\pgfqpoint{4.798685in}{2.441949in}}%
\pgfpathlineto{\pgfqpoint{4.784020in}{2.427879in}}%
\pgfpathlineto{\pgfqpoint{4.769374in}{2.413995in}}%
\pgfpathlineto{\pgfqpoint{4.761325in}{2.398758in}}%
\pgfpathlineto{\pgfqpoint{4.753270in}{2.383372in}}%
\pgfpathlineto{\pgfqpoint{4.745209in}{2.367840in}}%
\pgfpathlineto{\pgfqpoint{4.737143in}{2.352164in}}%
\pgfpathclose%
\pgfusepath{fill}%
\end{pgfscope}%
\begin{pgfscope}%
\pgfpathrectangle{\pgfqpoint{1.150000in}{0.150000in}}{\pgfqpoint{5.700000in}{5.700000in}}%
\pgfusepath{clip}%
\pgfsetbuttcap%
\pgfsetroundjoin%
\definecolor{currentfill}{rgb}{0.206756,0.371758,0.553117}%
\pgfsetfillcolor{currentfill}%
\pgfsetfillopacity{0.800000}%
\pgfsetlinewidth{0.000000pt}%
\definecolor{currentstroke}{rgb}{0.000000,0.000000,0.000000}%
\pgfsetstrokecolor{currentstroke}%
\pgfsetdash{}{0pt}%
\pgfpathmoveto{\pgfqpoint{4.458616in}{1.924777in}}%
\pgfpathlineto{\pgfqpoint{4.473076in}{1.935109in}}%
\pgfpathlineto{\pgfqpoint{4.487551in}{1.945623in}}%
\pgfpathlineto{\pgfqpoint{4.502043in}{1.956318in}}%
\pgfpathlineto{\pgfqpoint{4.516552in}{1.967195in}}%
\pgfpathlineto{\pgfqpoint{4.524705in}{1.984713in}}%
\pgfpathlineto{\pgfqpoint{4.532854in}{2.002157in}}%
\pgfpathlineto{\pgfqpoint{4.541000in}{2.019524in}}%
\pgfpathlineto{\pgfqpoint{4.549141in}{2.036809in}}%
\pgfpathlineto{\pgfqpoint{4.534624in}{2.025502in}}%
\pgfpathlineto{\pgfqpoint{4.520124in}{2.014377in}}%
\pgfpathlineto{\pgfqpoint{4.505641in}{2.003434in}}%
\pgfpathlineto{\pgfqpoint{4.491174in}{1.992674in}}%
\pgfpathlineto{\pgfqpoint{4.483041in}{1.975805in}}%
\pgfpathlineto{\pgfqpoint{4.474903in}{1.958864in}}%
\pgfpathlineto{\pgfqpoint{4.466762in}{1.941853in}}%
\pgfpathlineto{\pgfqpoint{4.458616in}{1.924777in}}%
\pgfpathclose%
\pgfusepath{fill}%
\end{pgfscope}%
\begin{pgfscope}%
\pgfpathrectangle{\pgfqpoint{1.150000in}{0.150000in}}{\pgfqpoint{5.700000in}{5.700000in}}%
\pgfusepath{clip}%
\pgfsetbuttcap%
\pgfsetroundjoin%
\definecolor{currentfill}{rgb}{0.272594,0.025563,0.353093}%
\pgfsetfillcolor{currentfill}%
\pgfsetfillopacity{0.800000}%
\pgfsetlinewidth{0.000000pt}%
\definecolor{currentstroke}{rgb}{0.000000,0.000000,0.000000}%
\pgfsetstrokecolor{currentstroke}%
\pgfsetdash{}{0pt}%
\pgfpathmoveto{\pgfqpoint{3.606901in}{1.130511in}}%
\pgfpathlineto{\pgfqpoint{3.621050in}{1.127614in}}%
\pgfpathlineto{\pgfqpoint{3.635205in}{1.124895in}}%
\pgfpathlineto{\pgfqpoint{3.649366in}{1.122355in}}%
\pgfpathlineto{\pgfqpoint{3.663533in}{1.119993in}}%
\pgfpathlineto{\pgfqpoint{3.671940in}{1.128346in}}%
\pgfpathlineto{\pgfqpoint{3.680338in}{1.136972in}}%
\pgfpathlineto{\pgfqpoint{3.688727in}{1.145863in}}%
\pgfpathlineto{\pgfqpoint{3.697108in}{1.155011in}}%
\pgfpathlineto{\pgfqpoint{3.682959in}{1.156579in}}%
\pgfpathlineto{\pgfqpoint{3.668817in}{1.158326in}}%
\pgfpathlineto{\pgfqpoint{3.654682in}{1.160251in}}%
\pgfpathlineto{\pgfqpoint{3.640553in}{1.162355in}}%
\pgfpathlineto{\pgfqpoint{3.632154in}{1.153989in}}%
\pgfpathlineto{\pgfqpoint{3.623745in}{1.145887in}}%
\pgfpathlineto{\pgfqpoint{3.615328in}{1.138059in}}%
\pgfpathlineto{\pgfqpoint{3.606901in}{1.130511in}}%
\pgfpathclose%
\pgfusepath{fill}%
\end{pgfscope}%
\begin{pgfscope}%
\pgfpathrectangle{\pgfqpoint{1.150000in}{0.150000in}}{\pgfqpoint{5.700000in}{5.700000in}}%
\pgfusepath{clip}%
\pgfsetbuttcap%
\pgfsetroundjoin%
\definecolor{currentfill}{rgb}{0.304148,0.764704,0.419943}%
\pgfsetfillcolor{currentfill}%
\pgfsetfillopacity{0.800000}%
\pgfsetlinewidth{0.000000pt}%
\definecolor{currentstroke}{rgb}{0.000000,0.000000,0.000000}%
\pgfsetstrokecolor{currentstroke}%
\pgfsetdash{}{0pt}%
\pgfpathmoveto{\pgfqpoint{5.351820in}{3.182051in}}%
\pgfpathlineto{\pgfqpoint{5.366920in}{3.200522in}}%
\pgfpathlineto{\pgfqpoint{5.382045in}{3.219185in}}%
\pgfpathlineto{\pgfqpoint{5.397194in}{3.238041in}}%
\pgfpathlineto{\pgfqpoint{5.412366in}{3.257091in}}%
\pgfpathlineto{\pgfqpoint{5.420118in}{3.265659in}}%
\pgfpathlineto{\pgfqpoint{5.427858in}{3.274009in}}%
\pgfpathlineto{\pgfqpoint{5.435588in}{3.282141in}}%
\pgfpathlineto{\pgfqpoint{5.443306in}{3.290058in}}%
\pgfpathlineto{\pgfqpoint{5.428137in}{3.271103in}}%
\pgfpathlineto{\pgfqpoint{5.412994in}{3.252341in}}%
\pgfpathlineto{\pgfqpoint{5.397874in}{3.233771in}}%
\pgfpathlineto{\pgfqpoint{5.382778in}{3.215394in}}%
\pgfpathlineto{\pgfqpoint{5.375054in}{3.207370in}}%
\pgfpathlineto{\pgfqpoint{5.367320in}{3.199139in}}%
\pgfpathlineto{\pgfqpoint{5.359575in}{3.190700in}}%
\pgfpathlineto{\pgfqpoint{5.351820in}{3.182051in}}%
\pgfpathclose%
\pgfusepath{fill}%
\end{pgfscope}%
\begin{pgfscope}%
\pgfpathrectangle{\pgfqpoint{1.150000in}{0.150000in}}{\pgfqpoint{5.700000in}{5.700000in}}%
\pgfusepath{clip}%
\pgfsetbuttcap%
\pgfsetroundjoin%
\definecolor{currentfill}{rgb}{0.283229,0.120777,0.440584}%
\pgfsetfillcolor{currentfill}%
\pgfsetfillopacity{0.800000}%
\pgfsetlinewidth{0.000000pt}%
\definecolor{currentstroke}{rgb}{0.000000,0.000000,0.000000}%
\pgfsetstrokecolor{currentstroke}%
\pgfsetdash{}{0pt}%
\pgfpathmoveto{\pgfqpoint{3.967070in}{1.303463in}}%
\pgfpathlineto{\pgfqpoint{3.981298in}{1.306394in}}%
\pgfpathlineto{\pgfqpoint{3.995536in}{1.309502in}}%
\pgfpathlineto{\pgfqpoint{4.009785in}{1.312786in}}%
\pgfpathlineto{\pgfqpoint{4.024045in}{1.316247in}}%
\pgfpathlineto{\pgfqpoint{4.032305in}{1.330794in}}%
\pgfpathlineto{\pgfqpoint{4.040560in}{1.345467in}}%
\pgfpathlineto{\pgfqpoint{4.048811in}{1.360259in}}%
\pgfpathlineto{\pgfqpoint{4.057058in}{1.375164in}}%
\pgfpathlineto{\pgfqpoint{4.042801in}{1.371029in}}%
\pgfpathlineto{\pgfqpoint{4.028556in}{1.367071in}}%
\pgfpathlineto{\pgfqpoint{4.014321in}{1.363290in}}%
\pgfpathlineto{\pgfqpoint{4.000097in}{1.359686in}}%
\pgfpathlineto{\pgfqpoint{3.991848in}{1.345443in}}%
\pgfpathlineto{\pgfqpoint{3.983594in}{1.331320in}}%
\pgfpathlineto{\pgfqpoint{3.975334in}{1.317325in}}%
\pgfpathlineto{\pgfqpoint{3.967070in}{1.303463in}}%
\pgfpathclose%
\pgfusepath{fill}%
\end{pgfscope}%
\begin{pgfscope}%
\pgfpathrectangle{\pgfqpoint{1.150000in}{0.150000in}}{\pgfqpoint{5.700000in}{5.700000in}}%
\pgfusepath{clip}%
\pgfsetbuttcap%
\pgfsetroundjoin%
\definecolor{currentfill}{rgb}{0.126453,0.570633,0.549841}%
\pgfsetfillcolor{currentfill}%
\pgfsetfillopacity{0.800000}%
\pgfsetlinewidth{0.000000pt}%
\definecolor{currentstroke}{rgb}{0.000000,0.000000,0.000000}%
\pgfsetstrokecolor{currentstroke}%
\pgfsetdash{}{0pt}%
\pgfpathmoveto{\pgfqpoint{4.860235in}{2.530988in}}%
\pgfpathlineto{\pgfqpoint{4.874966in}{2.545849in}}%
\pgfpathlineto{\pgfqpoint{4.889717in}{2.560898in}}%
\pgfpathlineto{\pgfqpoint{4.904488in}{2.576135in}}%
\pgfpathlineto{\pgfqpoint{4.919280in}{2.591559in}}%
\pgfpathlineto{\pgfqpoint{4.927311in}{2.606440in}}%
\pgfpathlineto{\pgfqpoint{4.935335in}{2.621143in}}%
\pgfpathlineto{\pgfqpoint{4.943352in}{2.635668in}}%
\pgfpathlineto{\pgfqpoint{4.951362in}{2.650012in}}%
\pgfpathlineto{\pgfqpoint{4.936564in}{2.634391in}}%
\pgfpathlineto{\pgfqpoint{4.921786in}{2.618959in}}%
\pgfpathlineto{\pgfqpoint{4.907029in}{2.603715in}}%
\pgfpathlineto{\pgfqpoint{4.892292in}{2.588658in}}%
\pgfpathlineto{\pgfqpoint{4.884288in}{2.574497in}}%
\pgfpathlineto{\pgfqpoint{4.876277in}{2.560163in}}%
\pgfpathlineto{\pgfqpoint{4.868260in}{2.545660in}}%
\pgfpathlineto{\pgfqpoint{4.860235in}{2.530988in}}%
\pgfpathclose%
\pgfusepath{fill}%
\end{pgfscope}%
\begin{pgfscope}%
\pgfpathrectangle{\pgfqpoint{1.150000in}{0.150000in}}{\pgfqpoint{5.700000in}{5.700000in}}%
\pgfusepath{clip}%
\pgfsetbuttcap%
\pgfsetroundjoin%
\definecolor{currentfill}{rgb}{0.220124,0.725509,0.466226}%
\pgfsetfillcolor{currentfill}%
\pgfsetfillopacity{0.800000}%
\pgfsetlinewidth{0.000000pt}%
\definecolor{currentstroke}{rgb}{0.000000,0.000000,0.000000}%
\pgfsetstrokecolor{currentstroke}%
\pgfsetdash{}{0pt}%
\pgfpathmoveto{\pgfqpoint{5.229215in}{3.032927in}}%
\pgfpathlineto{\pgfqpoint{5.244226in}{3.050706in}}%
\pgfpathlineto{\pgfqpoint{5.259260in}{3.068676in}}%
\pgfpathlineto{\pgfqpoint{5.274318in}{3.086838in}}%
\pgfpathlineto{\pgfqpoint{5.289399in}{3.105193in}}%
\pgfpathlineto{\pgfqpoint{5.297237in}{3.115555in}}%
\pgfpathlineto{\pgfqpoint{5.305066in}{3.125699in}}%
\pgfpathlineto{\pgfqpoint{5.312884in}{3.135627in}}%
\pgfpathlineto{\pgfqpoint{5.320692in}{3.145340in}}%
\pgfpathlineto{\pgfqpoint{5.305612in}{3.127005in}}%
\pgfpathlineto{\pgfqpoint{5.290556in}{3.108862in}}%
\pgfpathlineto{\pgfqpoint{5.275523in}{3.090912in}}%
\pgfpathlineto{\pgfqpoint{5.260513in}{3.073153in}}%
\pgfpathlineto{\pgfqpoint{5.252703in}{3.063407in}}%
\pgfpathlineto{\pgfqpoint{5.244884in}{3.053454in}}%
\pgfpathlineto{\pgfqpoint{5.237054in}{3.043295in}}%
\pgfpathlineto{\pgfqpoint{5.229215in}{3.032927in}}%
\pgfpathclose%
\pgfusepath{fill}%
\end{pgfscope}%
\begin{pgfscope}%
\pgfpathrectangle{\pgfqpoint{1.150000in}{0.150000in}}{\pgfqpoint{5.700000in}{5.700000in}}%
\pgfusepath{clip}%
\pgfsetbuttcap%
\pgfsetroundjoin%
\definecolor{currentfill}{rgb}{0.179019,0.433756,0.557430}%
\pgfsetfillcolor{currentfill}%
\pgfsetfillopacity{0.800000}%
\pgfsetlinewidth{0.000000pt}%
\definecolor{currentstroke}{rgb}{0.000000,0.000000,0.000000}%
\pgfsetstrokecolor{currentstroke}%
\pgfsetdash{}{0pt}%
\pgfpathmoveto{\pgfqpoint{4.581662in}{2.105066in}}%
\pgfpathlineto{\pgfqpoint{4.596204in}{2.116954in}}%
\pgfpathlineto{\pgfqpoint{4.610763in}{2.129025in}}%
\pgfpathlineto{\pgfqpoint{4.625341in}{2.141281in}}%
\pgfpathlineto{\pgfqpoint{4.639936in}{2.153720in}}%
\pgfpathlineto{\pgfqpoint{4.648064in}{2.170923in}}%
\pgfpathlineto{\pgfqpoint{4.656187in}{2.188015in}}%
\pgfpathlineto{\pgfqpoint{4.664305in}{2.204993in}}%
\pgfpathlineto{\pgfqpoint{4.672419in}{2.221853in}}%
\pgfpathlineto{\pgfqpoint{4.657815in}{2.209048in}}%
\pgfpathlineto{\pgfqpoint{4.643229in}{2.196428in}}%
\pgfpathlineto{\pgfqpoint{4.628661in}{2.183991in}}%
\pgfpathlineto{\pgfqpoint{4.614111in}{2.171739in}}%
\pgfpathlineto{\pgfqpoint{4.606006in}{2.155231in}}%
\pgfpathlineto{\pgfqpoint{4.597896in}{2.138614in}}%
\pgfpathlineto{\pgfqpoint{4.589781in}{2.121891in}}%
\pgfpathlineto{\pgfqpoint{4.581662in}{2.105066in}}%
\pgfpathclose%
\pgfusepath{fill}%
\end{pgfscope}%
\begin{pgfscope}%
\pgfpathrectangle{\pgfqpoint{1.150000in}{0.150000in}}{\pgfqpoint{5.700000in}{5.700000in}}%
\pgfusepath{clip}%
\pgfsetbuttcap%
\pgfsetroundjoin%
\definecolor{currentfill}{rgb}{0.120638,0.625828,0.533488}%
\pgfsetfillcolor{currentfill}%
\pgfsetfillopacity{0.800000}%
\pgfsetlinewidth{0.000000pt}%
\definecolor{currentstroke}{rgb}{0.000000,0.000000,0.000000}%
\pgfsetstrokecolor{currentstroke}%
\pgfsetdash{}{0pt}%
\pgfpathmoveto{\pgfqpoint{4.983330in}{2.705559in}}%
\pgfpathlineto{\pgfqpoint{4.998155in}{2.721530in}}%
\pgfpathlineto{\pgfqpoint{5.013000in}{2.737689in}}%
\pgfpathlineto{\pgfqpoint{5.027867in}{2.754038in}}%
\pgfpathlineto{\pgfqpoint{5.042756in}{2.770577in}}%
\pgfpathlineto{\pgfqpoint{5.050734in}{2.784139in}}%
\pgfpathlineto{\pgfqpoint{5.058705in}{2.797505in}}%
\pgfpathlineto{\pgfqpoint{5.066667in}{2.810674in}}%
\pgfpathlineto{\pgfqpoint{5.074621in}{2.823646in}}%
\pgfpathlineto{\pgfqpoint{5.059727in}{2.806982in}}%
\pgfpathlineto{\pgfqpoint{5.044855in}{2.790507in}}%
\pgfpathlineto{\pgfqpoint{5.030005in}{2.774223in}}%
\pgfpathlineto{\pgfqpoint{5.015176in}{2.758127in}}%
\pgfpathlineto{\pgfqpoint{5.007226in}{2.745268in}}%
\pgfpathlineto{\pgfqpoint{4.999269in}{2.732219in}}%
\pgfpathlineto{\pgfqpoint{4.991303in}{2.718983in}}%
\pgfpathlineto{\pgfqpoint{4.983330in}{2.705559in}}%
\pgfpathclose%
\pgfusepath{fill}%
\end{pgfscope}%
\begin{pgfscope}%
\pgfpathrectangle{\pgfqpoint{1.150000in}{0.150000in}}{\pgfqpoint{5.700000in}{5.700000in}}%
\pgfusepath{clip}%
\pgfsetbuttcap%
\pgfsetroundjoin%
\definecolor{currentfill}{rgb}{0.153894,0.680203,0.504172}%
\pgfsetfillcolor{currentfill}%
\pgfsetfillopacity{0.800000}%
\pgfsetlinewidth{0.000000pt}%
\definecolor{currentstroke}{rgb}{0.000000,0.000000,0.000000}%
\pgfsetstrokecolor{currentstroke}%
\pgfsetdash{}{0pt}%
\pgfpathmoveto{\pgfqpoint{5.106353in}{2.873541in}}%
\pgfpathlineto{\pgfqpoint{5.121271in}{2.890485in}}%
\pgfpathlineto{\pgfqpoint{5.136212in}{2.907620in}}%
\pgfpathlineto{\pgfqpoint{5.151175in}{2.924945in}}%
\pgfpathlineto{\pgfqpoint{5.166161in}{2.942462in}}%
\pgfpathlineto{\pgfqpoint{5.174075in}{2.954503in}}%
\pgfpathlineto{\pgfqpoint{5.181980in}{2.966335in}}%
\pgfpathlineto{\pgfqpoint{5.189876in}{2.977957in}}%
\pgfpathlineto{\pgfqpoint{5.197763in}{2.989370in}}%
\pgfpathlineto{\pgfqpoint{5.182775in}{2.971800in}}%
\pgfpathlineto{\pgfqpoint{5.167810in}{2.954421in}}%
\pgfpathlineto{\pgfqpoint{5.152867in}{2.937233in}}%
\pgfpathlineto{\pgfqpoint{5.137947in}{2.920236in}}%
\pgfpathlineto{\pgfqpoint{5.130061in}{2.908863in}}%
\pgfpathlineto{\pgfqpoint{5.122167in}{2.897289in}}%
\pgfpathlineto{\pgfqpoint{5.114264in}{2.885516in}}%
\pgfpathlineto{\pgfqpoint{5.106353in}{2.873541in}}%
\pgfpathclose%
\pgfusepath{fill}%
\end{pgfscope}%
\begin{pgfscope}%
\pgfpathrectangle{\pgfqpoint{1.150000in}{0.150000in}}{\pgfqpoint{5.700000in}{5.700000in}}%
\pgfusepath{clip}%
\pgfsetbuttcap%
\pgfsetroundjoin%
\definecolor{currentfill}{rgb}{0.269308,0.218818,0.509577}%
\pgfsetfillcolor{currentfill}%
\pgfsetfillopacity{0.800000}%
\pgfsetlinewidth{0.000000pt}%
\definecolor{currentstroke}{rgb}{0.000000,0.000000,0.000000}%
\pgfsetstrokecolor{currentstroke}%
\pgfsetdash{}{0pt}%
\pgfpathmoveto{\pgfqpoint{4.180019in}{1.521066in}}%
\pgfpathlineto{\pgfqpoint{4.194336in}{1.527346in}}%
\pgfpathlineto{\pgfqpoint{4.208666in}{1.533804in}}%
\pgfpathlineto{\pgfqpoint{4.223010in}{1.540440in}}%
\pgfpathlineto{\pgfqpoint{4.237367in}{1.547253in}}%
\pgfpathlineto{\pgfqpoint{4.245580in}{1.564090in}}%
\pgfpathlineto{\pgfqpoint{4.253789in}{1.580965in}}%
\pgfpathlineto{\pgfqpoint{4.261995in}{1.597872in}}%
\pgfpathlineto{\pgfqpoint{4.270197in}{1.614805in}}%
\pgfpathlineto{\pgfqpoint{4.255836in}{1.607405in}}%
\pgfpathlineto{\pgfqpoint{4.241489in}{1.600184in}}%
\pgfpathlineto{\pgfqpoint{4.227156in}{1.593142in}}%
\pgfpathlineto{\pgfqpoint{4.212836in}{1.586278in}}%
\pgfpathlineto{\pgfqpoint{4.204637in}{1.569918in}}%
\pgfpathlineto{\pgfqpoint{4.196435in}{1.553592in}}%
\pgfpathlineto{\pgfqpoint{4.188229in}{1.537306in}}%
\pgfpathlineto{\pgfqpoint{4.180019in}{1.521066in}}%
\pgfpathclose%
\pgfusepath{fill}%
\end{pgfscope}%
\begin{pgfscope}%
\pgfpathrectangle{\pgfqpoint{1.150000in}{0.150000in}}{\pgfqpoint{5.700000in}{5.700000in}}%
\pgfusepath{clip}%
\pgfsetbuttcap%
\pgfsetroundjoin%
\definecolor{currentfill}{rgb}{0.244972,0.287675,0.537260}%
\pgfsetfillcolor{currentfill}%
\pgfsetfillopacity{0.800000}%
\pgfsetlinewidth{0.000000pt}%
\definecolor{currentstroke}{rgb}{0.000000,0.000000,0.000000}%
\pgfsetstrokecolor{currentstroke}%
\pgfsetdash{}{0pt}%
\pgfpathmoveto{\pgfqpoint{4.302970in}{1.682701in}}%
\pgfpathlineto{\pgfqpoint{4.317349in}{1.690834in}}%
\pgfpathlineto{\pgfqpoint{4.331743in}{1.699146in}}%
\pgfpathlineto{\pgfqpoint{4.346152in}{1.707638in}}%
\pgfpathlineto{\pgfqpoint{4.360575in}{1.716310in}}%
\pgfpathlineto{\pgfqpoint{4.368765in}{1.733834in}}%
\pgfpathlineto{\pgfqpoint{4.376952in}{1.751347in}}%
\pgfpathlineto{\pgfqpoint{4.385135in}{1.768843in}}%
\pgfpathlineto{\pgfqpoint{4.393314in}{1.786317in}}%
\pgfpathlineto{\pgfqpoint{4.378884in}{1.777120in}}%
\pgfpathlineto{\pgfqpoint{4.364470in}{1.768104in}}%
\pgfpathlineto{\pgfqpoint{4.350070in}{1.759268in}}%
\pgfpathlineto{\pgfqpoint{4.335685in}{1.750611in}}%
\pgfpathlineto{\pgfqpoint{4.327512in}{1.733650in}}%
\pgfpathlineto{\pgfqpoint{4.319335in}{1.716674in}}%
\pgfpathlineto{\pgfqpoint{4.311154in}{1.699689in}}%
\pgfpathlineto{\pgfqpoint{4.302970in}{1.682701in}}%
\pgfpathclose%
\pgfusepath{fill}%
\end{pgfscope}%
\begin{pgfscope}%
\pgfpathrectangle{\pgfqpoint{1.150000in}{0.150000in}}{\pgfqpoint{5.700000in}{5.700000in}}%
\pgfusepath{clip}%
\pgfsetbuttcap%
\pgfsetroundjoin%
\definecolor{currentfill}{rgb}{0.487026,0.823929,0.312321}%
\pgfsetfillcolor{currentfill}%
\pgfsetfillopacity{0.800000}%
\pgfsetlinewidth{0.000000pt}%
\definecolor{currentstroke}{rgb}{0.000000,0.000000,0.000000}%
\pgfsetstrokecolor{currentstroke}%
\pgfsetdash{}{0pt}%
\pgfpathmoveto{\pgfqpoint{5.565497in}{3.422387in}}%
\pgfpathlineto{\pgfqpoint{5.580773in}{3.442009in}}%
\pgfpathlineto{\pgfqpoint{5.596075in}{3.461826in}}%
\pgfpathlineto{\pgfqpoint{5.611402in}{3.481837in}}%
\pgfpathlineto{\pgfqpoint{5.626755in}{3.502043in}}%
\pgfpathlineto{\pgfqpoint{5.634350in}{3.507714in}}%
\pgfpathlineto{\pgfqpoint{5.641932in}{3.513171in}}%
\pgfpathlineto{\pgfqpoint{5.649502in}{3.518416in}}%
\pgfpathlineto{\pgfqpoint{5.657059in}{3.523452in}}%
\pgfpathlineto{\pgfqpoint{5.641717in}{3.503454in}}%
\pgfpathlineto{\pgfqpoint{5.626402in}{3.483651in}}%
\pgfpathlineto{\pgfqpoint{5.611111in}{3.464041in}}%
\pgfpathlineto{\pgfqpoint{5.595846in}{3.444626in}}%
\pgfpathlineto{\pgfqpoint{5.588277in}{3.439369in}}%
\pgfpathlineto{\pgfqpoint{5.580695in}{3.433911in}}%
\pgfpathlineto{\pgfqpoint{5.573102in}{3.428252in}}%
\pgfpathlineto{\pgfqpoint{5.565497in}{3.422387in}}%
\pgfpathclose%
\pgfusepath{fill}%
\end{pgfscope}%
\begin{pgfscope}%
\pgfpathrectangle{\pgfqpoint{1.150000in}{0.150000in}}{\pgfqpoint{5.700000in}{5.700000in}}%
\pgfusepath{clip}%
\pgfsetbuttcap%
\pgfsetroundjoin%
\definecolor{currentfill}{rgb}{0.280868,0.160771,0.472899}%
\pgfsetfillcolor{currentfill}%
\pgfsetfillopacity{0.800000}%
\pgfsetlinewidth{0.000000pt}%
\definecolor{currentstroke}{rgb}{0.000000,0.000000,0.000000}%
\pgfsetstrokecolor{currentstroke}%
\pgfsetdash{}{0pt}%
\pgfpathmoveto{\pgfqpoint{4.057058in}{1.375164in}}%
\pgfpathlineto{\pgfqpoint{4.071326in}{1.379476in}}%
\pgfpathlineto{\pgfqpoint{4.085606in}{1.383965in}}%
\pgfpathlineto{\pgfqpoint{4.099897in}{1.388630in}}%
\pgfpathlineto{\pgfqpoint{4.114200in}{1.393471in}}%
\pgfpathlineto{\pgfqpoint{4.122441in}{1.409140in}}%
\pgfpathlineto{\pgfqpoint{4.130678in}{1.424900in}}%
\pgfpathlineto{\pgfqpoint{4.138911in}{1.440747in}}%
\pgfpathlineto{\pgfqpoint{4.147140in}{1.456674in}}%
\pgfpathlineto{\pgfqpoint{4.132837in}{1.451186in}}%
\pgfpathlineto{\pgfqpoint{4.118546in}{1.445876in}}%
\pgfpathlineto{\pgfqpoint{4.104267in}{1.440743in}}%
\pgfpathlineto{\pgfqpoint{4.090000in}{1.435787in}}%
\pgfpathlineto{\pgfqpoint{4.081771in}{1.420493in}}%
\pgfpathlineto{\pgfqpoint{4.073538in}{1.405288in}}%
\pgfpathlineto{\pgfqpoint{4.065300in}{1.390176in}}%
\pgfpathlineto{\pgfqpoint{4.057058in}{1.375164in}}%
\pgfpathclose%
\pgfusepath{fill}%
\end{pgfscope}%
\begin{pgfscope}%
\pgfpathrectangle{\pgfqpoint{1.150000in}{0.150000in}}{\pgfqpoint{5.700000in}{5.700000in}}%
\pgfusepath{clip}%
\pgfsetbuttcap%
\pgfsetroundjoin%
\definecolor{currentfill}{rgb}{0.153364,0.497000,0.557724}%
\pgfsetfillcolor{currentfill}%
\pgfsetfillopacity{0.800000}%
\pgfsetlinewidth{0.000000pt}%
\definecolor{currentstroke}{rgb}{0.000000,0.000000,0.000000}%
\pgfsetstrokecolor{currentstroke}%
\pgfsetdash{}{0pt}%
\pgfpathmoveto{\pgfqpoint{4.704823in}{2.288063in}}%
\pgfpathlineto{\pgfqpoint{4.719454in}{2.301385in}}%
\pgfpathlineto{\pgfqpoint{4.734103in}{2.314892in}}%
\pgfpathlineto{\pgfqpoint{4.748772in}{2.328586in}}%
\pgfpathlineto{\pgfqpoint{4.763459in}{2.342465in}}%
\pgfpathlineto{\pgfqpoint{4.771556in}{2.359006in}}%
\pgfpathlineto{\pgfqpoint{4.779647in}{2.375405in}}%
\pgfpathlineto{\pgfqpoint{4.787733in}{2.391657in}}%
\pgfpathlineto{\pgfqpoint{4.795812in}{2.407762in}}%
\pgfpathlineto{\pgfqpoint{4.781116in}{2.393583in}}%
\pgfpathlineto{\pgfqpoint{4.766440in}{2.379591in}}%
\pgfpathlineto{\pgfqpoint{4.751782in}{2.365784in}}%
\pgfpathlineto{\pgfqpoint{4.737143in}{2.352164in}}%
\pgfpathlineto{\pgfqpoint{4.729071in}{2.336345in}}%
\pgfpathlineto{\pgfqpoint{4.720994in}{2.320387in}}%
\pgfpathlineto{\pgfqpoint{4.712911in}{2.304292in}}%
\pgfpathlineto{\pgfqpoint{4.704823in}{2.288063in}}%
\pgfpathclose%
\pgfusepath{fill}%
\end{pgfscope}%
\begin{pgfscope}%
\pgfpathrectangle{\pgfqpoint{1.150000in}{0.150000in}}{\pgfqpoint{5.700000in}{5.700000in}}%
\pgfusepath{clip}%
\pgfsetbuttcap%
\pgfsetroundjoin%
\definecolor{currentfill}{rgb}{0.214298,0.355619,0.551184}%
\pgfsetfillcolor{currentfill}%
\pgfsetfillopacity{0.800000}%
\pgfsetlinewidth{0.000000pt}%
\definecolor{currentstroke}{rgb}{0.000000,0.000000,0.000000}%
\pgfsetstrokecolor{currentstroke}%
\pgfsetdash{}{0pt}%
\pgfpathmoveto{\pgfqpoint{4.425995in}{1.855902in}}%
\pgfpathlineto{\pgfqpoint{4.440447in}{1.865773in}}%
\pgfpathlineto{\pgfqpoint{4.454915in}{1.875825in}}%
\pgfpathlineto{\pgfqpoint{4.469400in}{1.886057in}}%
\pgfpathlineto{\pgfqpoint{4.483900in}{1.896471in}}%
\pgfpathlineto{\pgfqpoint{4.492069in}{1.914242in}}%
\pgfpathlineto{\pgfqpoint{4.500234in}{1.931956in}}%
\pgfpathlineto{\pgfqpoint{4.508395in}{1.949608in}}%
\pgfpathlineto{\pgfqpoint{4.516552in}{1.967195in}}%
\pgfpathlineto{\pgfqpoint{4.502043in}{1.956318in}}%
\pgfpathlineto{\pgfqpoint{4.487551in}{1.945623in}}%
\pgfpathlineto{\pgfqpoint{4.473076in}{1.935109in}}%
\pgfpathlineto{\pgfqpoint{4.458616in}{1.924777in}}%
\pgfpathlineto{\pgfqpoint{4.450467in}{1.907640in}}%
\pgfpathlineto{\pgfqpoint{4.442313in}{1.890446in}}%
\pgfpathlineto{\pgfqpoint{4.434156in}{1.873199in}}%
\pgfpathlineto{\pgfqpoint{4.425995in}{1.855902in}}%
\pgfpathclose%
\pgfusepath{fill}%
\end{pgfscope}%
\begin{pgfscope}%
\pgfpathrectangle{\pgfqpoint{1.150000in}{0.150000in}}{\pgfqpoint{5.700000in}{5.700000in}}%
\pgfusepath{clip}%
\pgfsetbuttcap%
\pgfsetroundjoin%
\definecolor{currentfill}{rgb}{0.277018,0.050344,0.375715}%
\pgfsetfillcolor{currentfill}%
\pgfsetfillopacity{0.800000}%
\pgfsetlinewidth{0.000000pt}%
\definecolor{currentstroke}{rgb}{0.000000,0.000000,0.000000}%
\pgfsetstrokecolor{currentstroke}%
\pgfsetdash{}{0pt}%
\pgfpathmoveto{\pgfqpoint{3.753774in}{1.150515in}}%
\pgfpathlineto{\pgfqpoint{3.767959in}{1.149834in}}%
\pgfpathlineto{\pgfqpoint{3.782151in}{1.149329in}}%
\pgfpathlineto{\pgfqpoint{3.796352in}{1.149001in}}%
\pgfpathlineto{\pgfqpoint{3.810560in}{1.148849in}}%
\pgfpathlineto{\pgfqpoint{3.818904in}{1.159796in}}%
\pgfpathlineto{\pgfqpoint{3.827240in}{1.170970in}}%
\pgfpathlineto{\pgfqpoint{3.835570in}{1.182361in}}%
\pgfpathlineto{\pgfqpoint{3.843893in}{1.193963in}}%
\pgfpathlineto{\pgfqpoint{3.829697in}{1.193351in}}%
\pgfpathlineto{\pgfqpoint{3.815509in}{1.192915in}}%
\pgfpathlineto{\pgfqpoint{3.801329in}{1.192656in}}%
\pgfpathlineto{\pgfqpoint{3.787158in}{1.192574in}}%
\pgfpathlineto{\pgfqpoint{3.778823in}{1.181724in}}%
\pgfpathlineto{\pgfqpoint{3.770480in}{1.171092in}}%
\pgfpathlineto{\pgfqpoint{3.762131in}{1.160687in}}%
\pgfpathlineto{\pgfqpoint{3.753774in}{1.150515in}}%
\pgfpathclose%
\pgfusepath{fill}%
\end{pgfscope}%
\begin{pgfscope}%
\pgfpathrectangle{\pgfqpoint{1.150000in}{0.150000in}}{\pgfqpoint{5.700000in}{5.700000in}}%
\pgfusepath{clip}%
\pgfsetbuttcap%
\pgfsetroundjoin%
\definecolor{currentfill}{rgb}{0.280267,0.073417,0.397163}%
\pgfsetfillcolor{currentfill}%
\pgfsetfillopacity{0.800000}%
\pgfsetlinewidth{0.000000pt}%
\definecolor{currentstroke}{rgb}{0.000000,0.000000,0.000000}%
\pgfsetstrokecolor{currentstroke}%
\pgfsetdash{}{0pt}%
\pgfpathmoveto{\pgfqpoint{3.843893in}{1.193963in}}%
\pgfpathlineto{\pgfqpoint{3.858098in}{1.194752in}}%
\pgfpathlineto{\pgfqpoint{3.872312in}{1.195717in}}%
\pgfpathlineto{\pgfqpoint{3.886535in}{1.196858in}}%
\pgfpathlineto{\pgfqpoint{3.900767in}{1.198174in}}%
\pgfpathlineto{\pgfqpoint{3.909074in}{1.210726in}}%
\pgfpathlineto{\pgfqpoint{3.917376in}{1.223465in}}%
\pgfpathlineto{\pgfqpoint{3.925672in}{1.236386in}}%
\pgfpathlineto{\pgfqpoint{3.933962in}{1.249481in}}%
\pgfpathlineto{\pgfqpoint{3.919738in}{1.247430in}}%
\pgfpathlineto{\pgfqpoint{3.905524in}{1.245555in}}%
\pgfpathlineto{\pgfqpoint{3.891319in}{1.243856in}}%
\pgfpathlineto{\pgfqpoint{3.877124in}{1.242334in}}%
\pgfpathlineto{\pgfqpoint{3.868825in}{1.229961in}}%
\pgfpathlineto{\pgfqpoint{3.860521in}{1.217771in}}%
\pgfpathlineto{\pgfqpoint{3.852210in}{1.205769in}}%
\pgfpathlineto{\pgfqpoint{3.843893in}{1.193963in}}%
\pgfpathclose%
\pgfusepath{fill}%
\end{pgfscope}%
\begin{pgfscope}%
\pgfpathrectangle{\pgfqpoint{1.150000in}{0.150000in}}{\pgfqpoint{5.700000in}{5.700000in}}%
\pgfusepath{clip}%
\pgfsetbuttcap%
\pgfsetroundjoin%
\definecolor{currentfill}{rgb}{0.131172,0.555899,0.552459}%
\pgfsetfillcolor{currentfill}%
\pgfsetfillopacity{0.800000}%
\pgfsetlinewidth{0.000000pt}%
\definecolor{currentstroke}{rgb}{0.000000,0.000000,0.000000}%
\pgfsetstrokecolor{currentstroke}%
\pgfsetdash{}{0pt}%
\pgfpathmoveto{\pgfqpoint{4.828073in}{2.470650in}}%
\pgfpathlineto{\pgfqpoint{4.842797in}{2.485281in}}%
\pgfpathlineto{\pgfqpoint{4.857540in}{2.500098in}}%
\pgfpathlineto{\pgfqpoint{4.872304in}{2.515104in}}%
\pgfpathlineto{\pgfqpoint{4.887088in}{2.530297in}}%
\pgfpathlineto{\pgfqpoint{4.895146in}{2.545870in}}%
\pgfpathlineto{\pgfqpoint{4.903197in}{2.561272in}}%
\pgfpathlineto{\pgfqpoint{4.911242in}{2.576503in}}%
\pgfpathlineto{\pgfqpoint{4.919280in}{2.591559in}}%
\pgfpathlineto{\pgfqpoint{4.904488in}{2.576135in}}%
\pgfpathlineto{\pgfqpoint{4.889717in}{2.560898in}}%
\pgfpathlineto{\pgfqpoint{4.874966in}{2.545849in}}%
\pgfpathlineto{\pgfqpoint{4.860235in}{2.530988in}}%
\pgfpathlineto{\pgfqpoint{4.852204in}{2.516150in}}%
\pgfpathlineto{\pgfqpoint{4.844167in}{2.501146in}}%
\pgfpathlineto{\pgfqpoint{4.836123in}{2.485978in}}%
\pgfpathlineto{\pgfqpoint{4.828073in}{2.470650in}}%
\pgfpathclose%
\pgfusepath{fill}%
\end{pgfscope}%
\begin{pgfscope}%
\pgfpathrectangle{\pgfqpoint{1.150000in}{0.150000in}}{\pgfqpoint{5.700000in}{5.700000in}}%
\pgfusepath{clip}%
\pgfsetbuttcap%
\pgfsetroundjoin%
\definecolor{currentfill}{rgb}{0.395174,0.797475,0.367757}%
\pgfsetfillcolor{currentfill}%
\pgfsetfillopacity{0.800000}%
\pgfsetlinewidth{0.000000pt}%
\definecolor{currentstroke}{rgb}{0.000000,0.000000,0.000000}%
\pgfsetstrokecolor{currentstroke}%
\pgfsetdash{}{0pt}%
\pgfpathmoveto{\pgfqpoint{5.443306in}{3.290058in}}%
\pgfpathlineto{\pgfqpoint{5.458498in}{3.309206in}}%
\pgfpathlineto{\pgfqpoint{5.473715in}{3.328549in}}%
\pgfpathlineto{\pgfqpoint{5.488957in}{3.348085in}}%
\pgfpathlineto{\pgfqpoint{5.504224in}{3.367816in}}%
\pgfpathlineto{\pgfqpoint{5.511925in}{3.375400in}}%
\pgfpathlineto{\pgfqpoint{5.519614in}{3.382763in}}%
\pgfpathlineto{\pgfqpoint{5.527291in}{3.389905in}}%
\pgfpathlineto{\pgfqpoint{5.534956in}{3.396830in}}%
\pgfpathlineto{\pgfqpoint{5.519696in}{3.377232in}}%
\pgfpathlineto{\pgfqpoint{5.504461in}{3.357828in}}%
\pgfpathlineto{\pgfqpoint{5.489250in}{3.338617in}}%
\pgfpathlineto{\pgfqpoint{5.474064in}{3.319601in}}%
\pgfpathlineto{\pgfqpoint{5.466392in}{3.312530in}}%
\pgfpathlineto{\pgfqpoint{5.458708in}{3.305251in}}%
\pgfpathlineto{\pgfqpoint{5.451012in}{3.297760in}}%
\pgfpathlineto{\pgfqpoint{5.443306in}{3.290058in}}%
\pgfpathclose%
\pgfusepath{fill}%
\end{pgfscope}%
\begin{pgfscope}%
\pgfpathrectangle{\pgfqpoint{1.150000in}{0.150000in}}{\pgfqpoint{5.700000in}{5.700000in}}%
\pgfusepath{clip}%
\pgfsetbuttcap%
\pgfsetroundjoin%
\definecolor{currentfill}{rgb}{0.273809,0.031497,0.358853}%
\pgfsetfillcolor{currentfill}%
\pgfsetfillopacity{0.800000}%
\pgfsetlinewidth{0.000000pt}%
\definecolor{currentstroke}{rgb}{0.000000,0.000000,0.000000}%
\pgfsetstrokecolor{currentstroke}%
\pgfsetdash{}{0pt}%
\pgfpathmoveto{\pgfqpoint{3.663533in}{1.119993in}}%
\pgfpathlineto{\pgfqpoint{3.677707in}{1.117808in}}%
\pgfpathlineto{\pgfqpoint{3.691887in}{1.115801in}}%
\pgfpathlineto{\pgfqpoint{3.706074in}{1.113971in}}%
\pgfpathlineto{\pgfqpoint{3.720268in}{1.112319in}}%
\pgfpathlineto{\pgfqpoint{3.728656in}{1.121478in}}%
\pgfpathlineto{\pgfqpoint{3.737037in}{1.130903in}}%
\pgfpathlineto{\pgfqpoint{3.745409in}{1.140584in}}%
\pgfpathlineto{\pgfqpoint{3.753774in}{1.150515in}}%
\pgfpathlineto{\pgfqpoint{3.739596in}{1.151373in}}%
\pgfpathlineto{\pgfqpoint{3.725426in}{1.152408in}}%
\pgfpathlineto{\pgfqpoint{3.711264in}{1.153621in}}%
\pgfpathlineto{\pgfqpoint{3.697108in}{1.155011in}}%
\pgfpathlineto{\pgfqpoint{3.688727in}{1.145863in}}%
\pgfpathlineto{\pgfqpoint{3.680338in}{1.136972in}}%
\pgfpathlineto{\pgfqpoint{3.671940in}{1.128346in}}%
\pgfpathlineto{\pgfqpoint{3.663533in}{1.119993in}}%
\pgfpathclose%
\pgfusepath{fill}%
\end{pgfscope}%
\begin{pgfscope}%
\pgfpathrectangle{\pgfqpoint{1.150000in}{0.150000in}}{\pgfqpoint{5.700000in}{5.700000in}}%
\pgfusepath{clip}%
\pgfsetbuttcap%
\pgfsetroundjoin%
\definecolor{currentfill}{rgb}{0.575563,0.844566,0.256415}%
\pgfsetfillcolor{currentfill}%
\pgfsetfillopacity{0.800000}%
\pgfsetlinewidth{0.000000pt}%
\definecolor{currentstroke}{rgb}{0.000000,0.000000,0.000000}%
\pgfsetstrokecolor{currentstroke}%
\pgfsetdash{}{0pt}%
\pgfpathmoveto{\pgfqpoint{5.657059in}{3.523452in}}%
\pgfpathlineto{\pgfqpoint{5.672426in}{3.543644in}}%
\pgfpathlineto{\pgfqpoint{5.687819in}{3.564032in}}%
\pgfpathlineto{\pgfqpoint{5.703239in}{3.584616in}}%
\pgfpathlineto{\pgfqpoint{5.710774in}{3.589270in}}%
\pgfpathlineto{\pgfqpoint{5.718296in}{3.593714in}}%
\pgfpathlineto{\pgfqpoint{5.725805in}{3.597951in}}%
\pgfpathlineto{\pgfqpoint{5.733301in}{3.601983in}}%
\pgfpathlineto{\pgfqpoint{5.717895in}{3.581647in}}%
\pgfpathlineto{\pgfqpoint{5.702516in}{3.561506in}}%
\pgfpathlineto{\pgfqpoint{5.687162in}{3.541559in}}%
\pgfpathlineto{\pgfqpoint{5.679655in}{3.537332in}}%
\pgfpathlineto{\pgfqpoint{5.672136in}{3.532907in}}%
\pgfpathlineto{\pgfqpoint{5.664604in}{3.528281in}}%
\pgfpathlineto{\pgfqpoint{5.657059in}{3.523452in}}%
\pgfpathclose%
\pgfusepath{fill}%
\end{pgfscope}%
\begin{pgfscope}%
\pgfpathrectangle{\pgfqpoint{1.150000in}{0.150000in}}{\pgfqpoint{5.700000in}{5.700000in}}%
\pgfusepath{clip}%
\pgfsetbuttcap%
\pgfsetroundjoin%
\definecolor{currentfill}{rgb}{0.282910,0.105393,0.426902}%
\pgfsetfillcolor{currentfill}%
\pgfsetfillopacity{0.800000}%
\pgfsetlinewidth{0.000000pt}%
\definecolor{currentstroke}{rgb}{0.000000,0.000000,0.000000}%
\pgfsetstrokecolor{currentstroke}%
\pgfsetdash{}{0pt}%
\pgfpathmoveto{\pgfqpoint{3.933962in}{1.249481in}}%
\pgfpathlineto{\pgfqpoint{3.948196in}{1.251708in}}%
\pgfpathlineto{\pgfqpoint{3.962440in}{1.254112in}}%
\pgfpathlineto{\pgfqpoint{3.976694in}{1.256691in}}%
\pgfpathlineto{\pgfqpoint{3.990958in}{1.259446in}}%
\pgfpathlineto{\pgfqpoint{3.999237in}{1.273424in}}%
\pgfpathlineto{\pgfqpoint{4.007511in}{1.287555in}}%
\pgfpathlineto{\pgfqpoint{4.015780in}{1.301831in}}%
\pgfpathlineto{\pgfqpoint{4.024045in}{1.316247in}}%
\pgfpathlineto{\pgfqpoint{4.009785in}{1.312786in}}%
\pgfpathlineto{\pgfqpoint{3.995536in}{1.309502in}}%
\pgfpathlineto{\pgfqpoint{3.981298in}{1.306394in}}%
\pgfpathlineto{\pgfqpoint{3.967070in}{1.303463in}}%
\pgfpathlineto{\pgfqpoint{3.958801in}{1.289740in}}%
\pgfpathlineto{\pgfqpoint{3.950527in}{1.276165in}}%
\pgfpathlineto{\pgfqpoint{3.942247in}{1.262743in}}%
\pgfpathlineto{\pgfqpoint{3.933962in}{1.249481in}}%
\pgfpathclose%
\pgfusepath{fill}%
\end{pgfscope}%
\begin{pgfscope}%
\pgfpathrectangle{\pgfqpoint{1.150000in}{0.150000in}}{\pgfqpoint{5.700000in}{5.700000in}}%
\pgfusepath{clip}%
\pgfsetbuttcap%
\pgfsetroundjoin%
\definecolor{currentfill}{rgb}{0.185556,0.418570,0.556753}%
\pgfsetfillcolor{currentfill}%
\pgfsetfillopacity{0.800000}%
\pgfsetlinewidth{0.000000pt}%
\definecolor{currentstroke}{rgb}{0.000000,0.000000,0.000000}%
\pgfsetstrokecolor{currentstroke}%
\pgfsetdash{}{0pt}%
\pgfpathmoveto{\pgfqpoint{4.549141in}{2.036809in}}%
\pgfpathlineto{\pgfqpoint{4.563674in}{2.048299in}}%
\pgfpathlineto{\pgfqpoint{4.578225in}{2.059973in}}%
\pgfpathlineto{\pgfqpoint{4.592794in}{2.071829in}}%
\pgfpathlineto{\pgfqpoint{4.607379in}{2.083868in}}%
\pgfpathlineto{\pgfqpoint{4.615525in}{2.101480in}}%
\pgfpathlineto{\pgfqpoint{4.623666in}{2.118995in}}%
\pgfpathlineto{\pgfqpoint{4.631803in}{2.136410in}}%
\pgfpathlineto{\pgfqpoint{4.639936in}{2.153720in}}%
\pgfpathlineto{\pgfqpoint{4.625341in}{2.141281in}}%
\pgfpathlineto{\pgfqpoint{4.610763in}{2.129025in}}%
\pgfpathlineto{\pgfqpoint{4.596204in}{2.116954in}}%
\pgfpathlineto{\pgfqpoint{4.581662in}{2.105066in}}%
\pgfpathlineto{\pgfqpoint{4.573538in}{2.088142in}}%
\pgfpathlineto{\pgfqpoint{4.565410in}{2.071122in}}%
\pgfpathlineto{\pgfqpoint{4.557277in}{2.054010in}}%
\pgfpathlineto{\pgfqpoint{4.549141in}{2.036809in}}%
\pgfpathclose%
\pgfusepath{fill}%
\end{pgfscope}%
\begin{pgfscope}%
\pgfpathrectangle{\pgfqpoint{1.150000in}{0.150000in}}{\pgfqpoint{5.700000in}{5.700000in}}%
\pgfusepath{clip}%
\pgfsetbuttcap%
\pgfsetroundjoin%
\definecolor{currentfill}{rgb}{0.274128,0.199721,0.498911}%
\pgfsetfillcolor{currentfill}%
\pgfsetfillopacity{0.800000}%
\pgfsetlinewidth{0.000000pt}%
\definecolor{currentstroke}{rgb}{0.000000,0.000000,0.000000}%
\pgfsetstrokecolor{currentstroke}%
\pgfsetdash{}{0pt}%
\pgfpathmoveto{\pgfqpoint{4.147140in}{1.456674in}}%
\pgfpathlineto{\pgfqpoint{4.161456in}{1.462339in}}%
\pgfpathlineto{\pgfqpoint{4.175784in}{1.468181in}}%
\pgfpathlineto{\pgfqpoint{4.190126in}{1.474200in}}%
\pgfpathlineto{\pgfqpoint{4.204480in}{1.480396in}}%
\pgfpathlineto{\pgfqpoint{4.212707in}{1.497025in}}%
\pgfpathlineto{\pgfqpoint{4.220930in}{1.513715in}}%
\pgfpathlineto{\pgfqpoint{4.229150in}{1.530460in}}%
\pgfpathlineto{\pgfqpoint{4.237367in}{1.547253in}}%
\pgfpathlineto{\pgfqpoint{4.223010in}{1.540440in}}%
\pgfpathlineto{\pgfqpoint{4.208666in}{1.533804in}}%
\pgfpathlineto{\pgfqpoint{4.194336in}{1.527346in}}%
\pgfpathlineto{\pgfqpoint{4.180019in}{1.521066in}}%
\pgfpathlineto{\pgfqpoint{4.171805in}{1.504877in}}%
\pgfpathlineto{\pgfqpoint{4.163587in}{1.488745in}}%
\pgfpathlineto{\pgfqpoint{4.155366in}{1.472675in}}%
\pgfpathlineto{\pgfqpoint{4.147140in}{1.456674in}}%
\pgfpathclose%
\pgfusepath{fill}%
\end{pgfscope}%
\begin{pgfscope}%
\pgfpathrectangle{\pgfqpoint{1.150000in}{0.150000in}}{\pgfqpoint{5.700000in}{5.700000in}}%
\pgfusepath{clip}%
\pgfsetbuttcap%
\pgfsetroundjoin%
\definecolor{currentfill}{rgb}{0.253935,0.265254,0.529983}%
\pgfsetfillcolor{currentfill}%
\pgfsetfillopacity{0.800000}%
\pgfsetlinewidth{0.000000pt}%
\definecolor{currentstroke}{rgb}{0.000000,0.000000,0.000000}%
\pgfsetstrokecolor{currentstroke}%
\pgfsetdash{}{0pt}%
\pgfpathmoveto{\pgfqpoint{4.270197in}{1.614805in}}%
\pgfpathlineto{\pgfqpoint{4.284571in}{1.622383in}}%
\pgfpathlineto{\pgfqpoint{4.298960in}{1.630140in}}%
\pgfpathlineto{\pgfqpoint{4.313363in}{1.638075in}}%
\pgfpathlineto{\pgfqpoint{4.327780in}{1.646189in}}%
\pgfpathlineto{\pgfqpoint{4.335984in}{1.663712in}}%
\pgfpathlineto{\pgfqpoint{4.344185in}{1.681243in}}%
\pgfpathlineto{\pgfqpoint{4.352382in}{1.698778in}}%
\pgfpathlineto{\pgfqpoint{4.360575in}{1.716310in}}%
\pgfpathlineto{\pgfqpoint{4.346152in}{1.707638in}}%
\pgfpathlineto{\pgfqpoint{4.331743in}{1.699146in}}%
\pgfpathlineto{\pgfqpoint{4.317349in}{1.690834in}}%
\pgfpathlineto{\pgfqpoint{4.302970in}{1.682701in}}%
\pgfpathlineto{\pgfqpoint{4.294782in}{1.665713in}}%
\pgfpathlineto{\pgfqpoint{4.286591in}{1.648731in}}%
\pgfpathlineto{\pgfqpoint{4.278396in}{1.631760in}}%
\pgfpathlineto{\pgfqpoint{4.270197in}{1.614805in}}%
\pgfpathclose%
\pgfusepath{fill}%
\end{pgfscope}%
\begin{pgfscope}%
\pgfpathrectangle{\pgfqpoint{1.150000in}{0.150000in}}{\pgfqpoint{5.700000in}{5.700000in}}%
\pgfusepath{clip}%
\pgfsetbuttcap%
\pgfsetroundjoin%
\definecolor{currentfill}{rgb}{0.119483,0.614817,0.537692}%
\pgfsetfillcolor{currentfill}%
\pgfsetfillopacity{0.800000}%
\pgfsetlinewidth{0.000000pt}%
\definecolor{currentstroke}{rgb}{0.000000,0.000000,0.000000}%
\pgfsetstrokecolor{currentstroke}%
\pgfsetdash{}{0pt}%
\pgfpathmoveto{\pgfqpoint{4.951362in}{2.650012in}}%
\pgfpathlineto{\pgfqpoint{4.966181in}{2.665822in}}%
\pgfpathlineto{\pgfqpoint{4.981021in}{2.681820in}}%
\pgfpathlineto{\pgfqpoint{4.995883in}{2.698007in}}%
\pgfpathlineto{\pgfqpoint{5.010765in}{2.714384in}}%
\pgfpathlineto{\pgfqpoint{5.018774in}{2.728722in}}%
\pgfpathlineto{\pgfqpoint{5.026776in}{2.742867in}}%
\pgfpathlineto{\pgfqpoint{5.034770in}{2.756819in}}%
\pgfpathlineto{\pgfqpoint{5.042756in}{2.770577in}}%
\pgfpathlineto{\pgfqpoint{5.027867in}{2.754038in}}%
\pgfpathlineto{\pgfqpoint{5.013000in}{2.737689in}}%
\pgfpathlineto{\pgfqpoint{4.998155in}{2.721530in}}%
\pgfpathlineto{\pgfqpoint{4.983330in}{2.705559in}}%
\pgfpathlineto{\pgfqpoint{4.975349in}{2.691949in}}%
\pgfpathlineto{\pgfqpoint{4.967361in}{2.678154in}}%
\pgfpathlineto{\pgfqpoint{4.959365in}{2.664175in}}%
\pgfpathlineto{\pgfqpoint{4.951362in}{2.650012in}}%
\pgfpathclose%
\pgfusepath{fill}%
\end{pgfscope}%
\begin{pgfscope}%
\pgfpathrectangle{\pgfqpoint{1.150000in}{0.150000in}}{\pgfqpoint{5.700000in}{5.700000in}}%
\pgfusepath{clip}%
\pgfsetbuttcap%
\pgfsetroundjoin%
\definecolor{currentfill}{rgb}{0.304148,0.764704,0.419943}%
\pgfsetfillcolor{currentfill}%
\pgfsetfillopacity{0.800000}%
\pgfsetlinewidth{0.000000pt}%
\definecolor{currentstroke}{rgb}{0.000000,0.000000,0.000000}%
\pgfsetstrokecolor{currentstroke}%
\pgfsetdash{}{0pt}%
\pgfpathmoveto{\pgfqpoint{5.320692in}{3.145340in}}%
\pgfpathlineto{\pgfqpoint{5.335796in}{3.163867in}}%
\pgfpathlineto{\pgfqpoint{5.350923in}{3.182588in}}%
\pgfpathlineto{\pgfqpoint{5.366074in}{3.201501in}}%
\pgfpathlineto{\pgfqpoint{5.381250in}{3.220609in}}%
\pgfpathlineto{\pgfqpoint{5.389045in}{3.230063in}}%
\pgfpathlineto{\pgfqpoint{5.396830in}{3.239294in}}%
\pgfpathlineto{\pgfqpoint{5.404604in}{3.248303in}}%
\pgfpathlineto{\pgfqpoint{5.412366in}{3.257091in}}%
\pgfpathlineto{\pgfqpoint{5.397194in}{3.238041in}}%
\pgfpathlineto{\pgfqpoint{5.382045in}{3.219185in}}%
\pgfpathlineto{\pgfqpoint{5.366920in}{3.200522in}}%
\pgfpathlineto{\pgfqpoint{5.351820in}{3.182051in}}%
\pgfpathlineto{\pgfqpoint{5.344054in}{3.173192in}}%
\pgfpathlineto{\pgfqpoint{5.336277in}{3.164121in}}%
\pgfpathlineto{\pgfqpoint{5.328490in}{3.154837in}}%
\pgfpathlineto{\pgfqpoint{5.320692in}{3.145340in}}%
\pgfpathclose%
\pgfusepath{fill}%
\end{pgfscope}%
\begin{pgfscope}%
\pgfpathrectangle{\pgfqpoint{1.150000in}{0.150000in}}{\pgfqpoint{5.700000in}{5.700000in}}%
\pgfusepath{clip}%
\pgfsetbuttcap%
\pgfsetroundjoin%
\definecolor{currentfill}{rgb}{0.223925,0.334994,0.548053}%
\pgfsetfillcolor{currentfill}%
\pgfsetfillopacity{0.800000}%
\pgfsetlinewidth{0.000000pt}%
\definecolor{currentstroke}{rgb}{0.000000,0.000000,0.000000}%
\pgfsetstrokecolor{currentstroke}%
\pgfsetdash{}{0pt}%
\pgfpathmoveto{\pgfqpoint{4.393314in}{1.786317in}}%
\pgfpathlineto{\pgfqpoint{4.407759in}{1.795694in}}%
\pgfpathlineto{\pgfqpoint{4.422220in}{1.805251in}}%
\pgfpathlineto{\pgfqpoint{4.436696in}{1.814988in}}%
\pgfpathlineto{\pgfqpoint{4.451188in}{1.824906in}}%
\pgfpathlineto{\pgfqpoint{4.459371in}{1.842861in}}%
\pgfpathlineto{\pgfqpoint{4.467551in}{1.860776in}}%
\pgfpathlineto{\pgfqpoint{4.475727in}{1.878648in}}%
\pgfpathlineto{\pgfqpoint{4.483900in}{1.896471in}}%
\pgfpathlineto{\pgfqpoint{4.469400in}{1.886057in}}%
\pgfpathlineto{\pgfqpoint{4.454915in}{1.875825in}}%
\pgfpathlineto{\pgfqpoint{4.440447in}{1.865773in}}%
\pgfpathlineto{\pgfqpoint{4.425995in}{1.855902in}}%
\pgfpathlineto{\pgfqpoint{4.417830in}{1.838562in}}%
\pgfpathlineto{\pgfqpoint{4.409662in}{1.821181in}}%
\pgfpathlineto{\pgfqpoint{4.401490in}{1.803765in}}%
\pgfpathlineto{\pgfqpoint{4.393314in}{1.786317in}}%
\pgfpathclose%
\pgfusepath{fill}%
\end{pgfscope}%
\begin{pgfscope}%
\pgfpathrectangle{\pgfqpoint{1.150000in}{0.150000in}}{\pgfqpoint{5.700000in}{5.700000in}}%
\pgfusepath{clip}%
\pgfsetbuttcap%
\pgfsetroundjoin%
\definecolor{currentfill}{rgb}{0.143303,0.669459,0.511215}%
\pgfsetfillcolor{currentfill}%
\pgfsetfillopacity{0.800000}%
\pgfsetlinewidth{0.000000pt}%
\definecolor{currentstroke}{rgb}{0.000000,0.000000,0.000000}%
\pgfsetstrokecolor{currentstroke}%
\pgfsetdash{}{0pt}%
\pgfpathmoveto{\pgfqpoint{5.074621in}{2.823646in}}%
\pgfpathlineto{\pgfqpoint{5.089536in}{2.840500in}}%
\pgfpathlineto{\pgfqpoint{5.104473in}{2.857545in}}%
\pgfpathlineto{\pgfqpoint{5.119433in}{2.874781in}}%
\pgfpathlineto{\pgfqpoint{5.134414in}{2.892207in}}%
\pgfpathlineto{\pgfqpoint{5.142364in}{2.905084in}}%
\pgfpathlineto{\pgfqpoint{5.150305in}{2.917752in}}%
\pgfpathlineto{\pgfqpoint{5.158237in}{2.930211in}}%
\pgfpathlineto{\pgfqpoint{5.166161in}{2.942462in}}%
\pgfpathlineto{\pgfqpoint{5.151175in}{2.924945in}}%
\pgfpathlineto{\pgfqpoint{5.136212in}{2.907620in}}%
\pgfpathlineto{\pgfqpoint{5.121271in}{2.890485in}}%
\pgfpathlineto{\pgfqpoint{5.106353in}{2.873541in}}%
\pgfpathlineto{\pgfqpoint{5.098432in}{2.861367in}}%
\pgfpathlineto{\pgfqpoint{5.090504in}{2.848993in}}%
\pgfpathlineto{\pgfqpoint{5.082566in}{2.836419in}}%
\pgfpathlineto{\pgfqpoint{5.074621in}{2.823646in}}%
\pgfpathclose%
\pgfusepath{fill}%
\end{pgfscope}%
\begin{pgfscope}%
\pgfpathrectangle{\pgfqpoint{1.150000in}{0.150000in}}{\pgfqpoint{5.700000in}{5.700000in}}%
\pgfusepath{clip}%
\pgfsetbuttcap%
\pgfsetroundjoin%
\definecolor{currentfill}{rgb}{0.282623,0.140926,0.457517}%
\pgfsetfillcolor{currentfill}%
\pgfsetfillopacity{0.800000}%
\pgfsetlinewidth{0.000000pt}%
\definecolor{currentstroke}{rgb}{0.000000,0.000000,0.000000}%
\pgfsetstrokecolor{currentstroke}%
\pgfsetdash{}{0pt}%
\pgfpathmoveto{\pgfqpoint{4.024045in}{1.316247in}}%
\pgfpathlineto{\pgfqpoint{4.038316in}{1.319883in}}%
\pgfpathlineto{\pgfqpoint{4.052597in}{1.323696in}}%
\pgfpathlineto{\pgfqpoint{4.066890in}{1.327685in}}%
\pgfpathlineto{\pgfqpoint{4.081194in}{1.331849in}}%
\pgfpathlineto{\pgfqpoint{4.089452in}{1.347084in}}%
\pgfpathlineto{\pgfqpoint{4.097705in}{1.362437in}}%
\pgfpathlineto{\pgfqpoint{4.105955in}{1.377902in}}%
\pgfpathlineto{\pgfqpoint{4.114200in}{1.393471in}}%
\pgfpathlineto{\pgfqpoint{4.099897in}{1.388630in}}%
\pgfpathlineto{\pgfqpoint{4.085606in}{1.383965in}}%
\pgfpathlineto{\pgfqpoint{4.071326in}{1.379476in}}%
\pgfpathlineto{\pgfqpoint{4.057058in}{1.375164in}}%
\pgfpathlineto{\pgfqpoint{4.048811in}{1.360259in}}%
\pgfpathlineto{\pgfqpoint{4.040560in}{1.345467in}}%
\pgfpathlineto{\pgfqpoint{4.032305in}{1.330794in}}%
\pgfpathlineto{\pgfqpoint{4.024045in}{1.316247in}}%
\pgfpathclose%
\pgfusepath{fill}%
\end{pgfscope}%
\begin{pgfscope}%
\pgfpathrectangle{\pgfqpoint{1.150000in}{0.150000in}}{\pgfqpoint{5.700000in}{5.700000in}}%
\pgfusepath{clip}%
\pgfsetbuttcap%
\pgfsetroundjoin%
\definecolor{currentfill}{rgb}{0.159194,0.482237,0.558073}%
\pgfsetfillcolor{currentfill}%
\pgfsetfillopacity{0.800000}%
\pgfsetlinewidth{0.000000pt}%
\definecolor{currentstroke}{rgb}{0.000000,0.000000,0.000000}%
\pgfsetstrokecolor{currentstroke}%
\pgfsetdash{}{0pt}%
\pgfpathmoveto{\pgfqpoint{4.672419in}{2.221853in}}%
\pgfpathlineto{\pgfqpoint{4.687041in}{2.234843in}}%
\pgfpathlineto{\pgfqpoint{4.701681in}{2.248017in}}%
\pgfpathlineto{\pgfqpoint{4.716341in}{2.261377in}}%
\pgfpathlineto{\pgfqpoint{4.731019in}{2.274922in}}%
\pgfpathlineto{\pgfqpoint{4.739137in}{2.292008in}}%
\pgfpathlineto{\pgfqpoint{4.747249in}{2.308963in}}%
\pgfpathlineto{\pgfqpoint{4.755357in}{2.325783in}}%
\pgfpathlineto{\pgfqpoint{4.763459in}{2.342465in}}%
\pgfpathlineto{\pgfqpoint{4.748772in}{2.328586in}}%
\pgfpathlineto{\pgfqpoint{4.734103in}{2.314892in}}%
\pgfpathlineto{\pgfqpoint{4.719454in}{2.301385in}}%
\pgfpathlineto{\pgfqpoint{4.704823in}{2.288063in}}%
\pgfpathlineto{\pgfqpoint{4.696729in}{2.271701in}}%
\pgfpathlineto{\pgfqpoint{4.688631in}{2.255210in}}%
\pgfpathlineto{\pgfqpoint{4.680527in}{2.238594in}}%
\pgfpathlineto{\pgfqpoint{4.672419in}{2.221853in}}%
\pgfpathclose%
\pgfusepath{fill}%
\end{pgfscope}%
\begin{pgfscope}%
\pgfpathrectangle{\pgfqpoint{1.150000in}{0.150000in}}{\pgfqpoint{5.700000in}{5.700000in}}%
\pgfusepath{clip}%
\pgfsetbuttcap%
\pgfsetroundjoin%
\definecolor{currentfill}{rgb}{0.214000,0.722114,0.469588}%
\pgfsetfillcolor{currentfill}%
\pgfsetfillopacity{0.800000}%
\pgfsetlinewidth{0.000000pt}%
\definecolor{currentstroke}{rgb}{0.000000,0.000000,0.000000}%
\pgfsetstrokecolor{currentstroke}%
\pgfsetdash{}{0pt}%
\pgfpathmoveto{\pgfqpoint{5.197763in}{2.989370in}}%
\pgfpathlineto{\pgfqpoint{5.212773in}{3.007132in}}%
\pgfpathlineto{\pgfqpoint{5.227807in}{3.025085in}}%
\pgfpathlineto{\pgfqpoint{5.242864in}{3.043231in}}%
\pgfpathlineto{\pgfqpoint{5.257944in}{3.061569in}}%
\pgfpathlineto{\pgfqpoint{5.265822in}{3.072803in}}%
\pgfpathlineto{\pgfqpoint{5.273691in}{3.083818in}}%
\pgfpathlineto{\pgfqpoint{5.281550in}{3.094614in}}%
\pgfpathlineto{\pgfqpoint{5.289399in}{3.105193in}}%
\pgfpathlineto{\pgfqpoint{5.274318in}{3.086838in}}%
\pgfpathlineto{\pgfqpoint{5.259260in}{3.068676in}}%
\pgfpathlineto{\pgfqpoint{5.244226in}{3.050706in}}%
\pgfpathlineto{\pgfqpoint{5.229215in}{3.032927in}}%
\pgfpathlineto{\pgfqpoint{5.221366in}{3.022351in}}%
\pgfpathlineto{\pgfqpoint{5.213508in}{3.011567in}}%
\pgfpathlineto{\pgfqpoint{5.205640in}{3.000573in}}%
\pgfpathlineto{\pgfqpoint{5.197763in}{2.989370in}}%
\pgfpathclose%
\pgfusepath{fill}%
\end{pgfscope}%
\begin{pgfscope}%
\pgfpathrectangle{\pgfqpoint{1.150000in}{0.150000in}}{\pgfqpoint{5.700000in}{5.700000in}}%
\pgfusepath{clip}%
\pgfsetbuttcap%
\pgfsetroundjoin%
\definecolor{currentfill}{rgb}{0.194100,0.399323,0.555565}%
\pgfsetfillcolor{currentfill}%
\pgfsetfillopacity{0.800000}%
\pgfsetlinewidth{0.000000pt}%
\definecolor{currentstroke}{rgb}{0.000000,0.000000,0.000000}%
\pgfsetstrokecolor{currentstroke}%
\pgfsetdash{}{0pt}%
\pgfpathmoveto{\pgfqpoint{4.516552in}{1.967195in}}%
\pgfpathlineto{\pgfqpoint{4.531077in}{1.978254in}}%
\pgfpathlineto{\pgfqpoint{4.545619in}{1.989496in}}%
\pgfpathlineto{\pgfqpoint{4.560178in}{2.000919in}}%
\pgfpathlineto{\pgfqpoint{4.574755in}{2.012526in}}%
\pgfpathlineto{\pgfqpoint{4.582917in}{2.030488in}}%
\pgfpathlineto{\pgfqpoint{4.591075in}{2.048368in}}%
\pgfpathlineto{\pgfqpoint{4.599229in}{2.066163in}}%
\pgfpathlineto{\pgfqpoint{4.607379in}{2.083868in}}%
\pgfpathlineto{\pgfqpoint{4.592794in}{2.071829in}}%
\pgfpathlineto{\pgfqpoint{4.578225in}{2.059973in}}%
\pgfpathlineto{\pgfqpoint{4.563674in}{2.048299in}}%
\pgfpathlineto{\pgfqpoint{4.549141in}{2.036809in}}%
\pgfpathlineto{\pgfqpoint{4.541000in}{2.019524in}}%
\pgfpathlineto{\pgfqpoint{4.532854in}{2.002157in}}%
\pgfpathlineto{\pgfqpoint{4.524705in}{1.984713in}}%
\pgfpathlineto{\pgfqpoint{4.516552in}{1.967195in}}%
\pgfpathclose%
\pgfusepath{fill}%
\end{pgfscope}%
\begin{pgfscope}%
\pgfpathrectangle{\pgfqpoint{1.150000in}{0.150000in}}{\pgfqpoint{5.700000in}{5.700000in}}%
\pgfusepath{clip}%
\pgfsetbuttcap%
\pgfsetroundjoin%
\definecolor{currentfill}{rgb}{0.135066,0.544853,0.554029}%
\pgfsetfillcolor{currentfill}%
\pgfsetfillopacity{0.800000}%
\pgfsetlinewidth{0.000000pt}%
\definecolor{currentstroke}{rgb}{0.000000,0.000000,0.000000}%
\pgfsetstrokecolor{currentstroke}%
\pgfsetdash{}{0pt}%
\pgfpathmoveto{\pgfqpoint{4.795812in}{2.407762in}}%
\pgfpathlineto{\pgfqpoint{4.810528in}{2.422127in}}%
\pgfpathlineto{\pgfqpoint{4.825263in}{2.436679in}}%
\pgfpathlineto{\pgfqpoint{4.840018in}{2.451418in}}%
\pgfpathlineto{\pgfqpoint{4.854793in}{2.466344in}}%
\pgfpathlineto{\pgfqpoint{4.862876in}{2.482577in}}%
\pgfpathlineto{\pgfqpoint{4.870953in}{2.498648in}}%
\pgfpathlineto{\pgfqpoint{4.879024in}{2.514556in}}%
\pgfpathlineto{\pgfqpoint{4.887088in}{2.530297in}}%
\pgfpathlineto{\pgfqpoint{4.872304in}{2.515104in}}%
\pgfpathlineto{\pgfqpoint{4.857540in}{2.500098in}}%
\pgfpathlineto{\pgfqpoint{4.842797in}{2.485281in}}%
\pgfpathlineto{\pgfqpoint{4.828073in}{2.470650in}}%
\pgfpathlineto{\pgfqpoint{4.820017in}{2.455162in}}%
\pgfpathlineto{\pgfqpoint{4.811955in}{2.439516in}}%
\pgfpathlineto{\pgfqpoint{4.803887in}{2.423715in}}%
\pgfpathlineto{\pgfqpoint{4.795812in}{2.407762in}}%
\pgfpathclose%
\pgfusepath{fill}%
\end{pgfscope}%
\begin{pgfscope}%
\pgfpathrectangle{\pgfqpoint{1.150000in}{0.150000in}}{\pgfqpoint{5.700000in}{5.700000in}}%
\pgfusepath{clip}%
\pgfsetbuttcap%
\pgfsetroundjoin%
\definecolor{currentfill}{rgb}{0.496615,0.826376,0.306377}%
\pgfsetfillcolor{currentfill}%
\pgfsetfillopacity{0.800000}%
\pgfsetlinewidth{0.000000pt}%
\definecolor{currentstroke}{rgb}{0.000000,0.000000,0.000000}%
\pgfsetstrokecolor{currentstroke}%
\pgfsetdash{}{0pt}%
\pgfpathmoveto{\pgfqpoint{5.534956in}{3.396830in}}%
\pgfpathlineto{\pgfqpoint{5.550241in}{3.416622in}}%
\pgfpathlineto{\pgfqpoint{5.565552in}{3.436610in}}%
\pgfpathlineto{\pgfqpoint{5.580888in}{3.456792in}}%
\pgfpathlineto{\pgfqpoint{5.596250in}{3.477171in}}%
\pgfpathlineto{\pgfqpoint{5.603895in}{3.483722in}}%
\pgfpathlineto{\pgfqpoint{5.611527in}{3.490049in}}%
\pgfpathlineto{\pgfqpoint{5.619147in}{3.496156in}}%
\pgfpathlineto{\pgfqpoint{5.626755in}{3.502043in}}%
\pgfpathlineto{\pgfqpoint{5.611402in}{3.481837in}}%
\pgfpathlineto{\pgfqpoint{5.596075in}{3.461826in}}%
\pgfpathlineto{\pgfqpoint{5.580773in}{3.442009in}}%
\pgfpathlineto{\pgfqpoint{5.565497in}{3.422387in}}%
\pgfpathlineto{\pgfqpoint{5.557880in}{3.416314in}}%
\pgfpathlineto{\pgfqpoint{5.550250in}{3.410032in}}%
\pgfpathlineto{\pgfqpoint{5.542609in}{3.403538in}}%
\pgfpathlineto{\pgfqpoint{5.534956in}{3.396830in}}%
\pgfpathclose%
\pgfusepath{fill}%
\end{pgfscope}%
\begin{pgfscope}%
\pgfpathrectangle{\pgfqpoint{1.150000in}{0.150000in}}{\pgfqpoint{5.700000in}{5.700000in}}%
\pgfusepath{clip}%
\pgfsetbuttcap%
\pgfsetroundjoin%
\definecolor{currentfill}{rgb}{0.278791,0.062145,0.386592}%
\pgfsetfillcolor{currentfill}%
\pgfsetfillopacity{0.800000}%
\pgfsetlinewidth{0.000000pt}%
\definecolor{currentstroke}{rgb}{0.000000,0.000000,0.000000}%
\pgfsetstrokecolor{currentstroke}%
\pgfsetdash{}{0pt}%
\pgfpathmoveto{\pgfqpoint{3.810560in}{1.148849in}}%
\pgfpathlineto{\pgfqpoint{3.824777in}{1.148873in}}%
\pgfpathlineto{\pgfqpoint{3.839002in}{1.149072in}}%
\pgfpathlineto{\pgfqpoint{3.853235in}{1.149448in}}%
\pgfpathlineto{\pgfqpoint{3.867478in}{1.149998in}}%
\pgfpathlineto{\pgfqpoint{3.875809in}{1.161723in}}%
\pgfpathlineto{\pgfqpoint{3.884135in}{1.173665in}}%
\pgfpathlineto{\pgfqpoint{3.892454in}{1.185818in}}%
\pgfpathlineto{\pgfqpoint{3.900767in}{1.198174in}}%
\pgfpathlineto{\pgfqpoint{3.886535in}{1.196858in}}%
\pgfpathlineto{\pgfqpoint{3.872312in}{1.195717in}}%
\pgfpathlineto{\pgfqpoint{3.858098in}{1.194752in}}%
\pgfpathlineto{\pgfqpoint{3.843893in}{1.193963in}}%
\pgfpathlineto{\pgfqpoint{3.835570in}{1.182361in}}%
\pgfpathlineto{\pgfqpoint{3.827240in}{1.170970in}}%
\pgfpathlineto{\pgfqpoint{3.818904in}{1.159796in}}%
\pgfpathlineto{\pgfqpoint{3.810560in}{1.148849in}}%
\pgfpathclose%
\pgfusepath{fill}%
\end{pgfscope}%
\begin{pgfscope}%
\pgfpathrectangle{\pgfqpoint{1.150000in}{0.150000in}}{\pgfqpoint{5.700000in}{5.700000in}}%
\pgfusepath{clip}%
\pgfsetbuttcap%
\pgfsetroundjoin%
\definecolor{currentfill}{rgb}{0.260571,0.246922,0.522828}%
\pgfsetfillcolor{currentfill}%
\pgfsetfillopacity{0.800000}%
\pgfsetlinewidth{0.000000pt}%
\definecolor{currentstroke}{rgb}{0.000000,0.000000,0.000000}%
\pgfsetstrokecolor{currentstroke}%
\pgfsetdash{}{0pt}%
\pgfpathmoveto{\pgfqpoint{4.237367in}{1.547253in}}%
\pgfpathlineto{\pgfqpoint{4.251737in}{1.554245in}}%
\pgfpathlineto{\pgfqpoint{4.266121in}{1.561414in}}%
\pgfpathlineto{\pgfqpoint{4.280519in}{1.568761in}}%
\pgfpathlineto{\pgfqpoint{4.294931in}{1.576286in}}%
\pgfpathlineto{\pgfqpoint{4.303148in}{1.593722in}}%
\pgfpathlineto{\pgfqpoint{4.311362in}{1.611189in}}%
\pgfpathlineto{\pgfqpoint{4.319573in}{1.628680in}}%
\pgfpathlineto{\pgfqpoint{4.327780in}{1.646189in}}%
\pgfpathlineto{\pgfqpoint{4.313363in}{1.638075in}}%
\pgfpathlineto{\pgfqpoint{4.298960in}{1.630140in}}%
\pgfpathlineto{\pgfqpoint{4.284571in}{1.622383in}}%
\pgfpathlineto{\pgfqpoint{4.270197in}{1.614805in}}%
\pgfpathlineto{\pgfqpoint{4.261995in}{1.597872in}}%
\pgfpathlineto{\pgfqpoint{4.253789in}{1.580965in}}%
\pgfpathlineto{\pgfqpoint{4.245580in}{1.564090in}}%
\pgfpathlineto{\pgfqpoint{4.237367in}{1.547253in}}%
\pgfpathclose%
\pgfusepath{fill}%
\end{pgfscope}%
\begin{pgfscope}%
\pgfpathrectangle{\pgfqpoint{1.150000in}{0.150000in}}{\pgfqpoint{5.700000in}{5.700000in}}%
\pgfusepath{clip}%
\pgfsetbuttcap%
\pgfsetroundjoin%
\definecolor{currentfill}{rgb}{0.278012,0.180367,0.486697}%
\pgfsetfillcolor{currentfill}%
\pgfsetfillopacity{0.800000}%
\pgfsetlinewidth{0.000000pt}%
\definecolor{currentstroke}{rgb}{0.000000,0.000000,0.000000}%
\pgfsetstrokecolor{currentstroke}%
\pgfsetdash{}{0pt}%
\pgfpathmoveto{\pgfqpoint{4.114200in}{1.393471in}}%
\pgfpathlineto{\pgfqpoint{4.128515in}{1.398489in}}%
\pgfpathlineto{\pgfqpoint{4.142842in}{1.403684in}}%
\pgfpathlineto{\pgfqpoint{4.157182in}{1.409055in}}%
\pgfpathlineto{\pgfqpoint{4.171534in}{1.414602in}}%
\pgfpathlineto{\pgfqpoint{4.179776in}{1.430930in}}%
\pgfpathlineto{\pgfqpoint{4.188014in}{1.447342in}}%
\pgfpathlineto{\pgfqpoint{4.196249in}{1.463833in}}%
\pgfpathlineto{\pgfqpoint{4.204480in}{1.480396in}}%
\pgfpathlineto{\pgfqpoint{4.190126in}{1.474200in}}%
\pgfpathlineto{\pgfqpoint{4.175784in}{1.468181in}}%
\pgfpathlineto{\pgfqpoint{4.161456in}{1.462339in}}%
\pgfpathlineto{\pgfqpoint{4.147140in}{1.456674in}}%
\pgfpathlineto{\pgfqpoint{4.138911in}{1.440747in}}%
\pgfpathlineto{\pgfqpoint{4.130678in}{1.424900in}}%
\pgfpathlineto{\pgfqpoint{4.122441in}{1.409140in}}%
\pgfpathlineto{\pgfqpoint{4.114200in}{1.393471in}}%
\pgfpathclose%
\pgfusepath{fill}%
\end{pgfscope}%
\begin{pgfscope}%
\pgfpathrectangle{\pgfqpoint{1.150000in}{0.150000in}}{\pgfqpoint{5.700000in}{5.700000in}}%
\pgfusepath{clip}%
\pgfsetbuttcap%
\pgfsetroundjoin%
\definecolor{currentfill}{rgb}{0.276022,0.044167,0.370164}%
\pgfsetfillcolor{currentfill}%
\pgfsetfillopacity{0.800000}%
\pgfsetlinewidth{0.000000pt}%
\definecolor{currentstroke}{rgb}{0.000000,0.000000,0.000000}%
\pgfsetstrokecolor{currentstroke}%
\pgfsetdash{}{0pt}%
\pgfpathmoveto{\pgfqpoint{3.720268in}{1.112319in}}%
\pgfpathlineto{\pgfqpoint{3.734469in}{1.110842in}}%
\pgfpathlineto{\pgfqpoint{3.748677in}{1.109542in}}%
\pgfpathlineto{\pgfqpoint{3.762892in}{1.108418in}}%
\pgfpathlineto{\pgfqpoint{3.777115in}{1.107470in}}%
\pgfpathlineto{\pgfqpoint{3.785487in}{1.117437in}}%
\pgfpathlineto{\pgfqpoint{3.793852in}{1.127662in}}%
\pgfpathlineto{\pgfqpoint{3.802210in}{1.138135in}}%
\pgfpathlineto{\pgfqpoint{3.810560in}{1.148849in}}%
\pgfpathlineto{\pgfqpoint{3.796352in}{1.149001in}}%
\pgfpathlineto{\pgfqpoint{3.782151in}{1.149329in}}%
\pgfpathlineto{\pgfqpoint{3.767959in}{1.149834in}}%
\pgfpathlineto{\pgfqpoint{3.753774in}{1.150515in}}%
\pgfpathlineto{\pgfqpoint{3.745409in}{1.140584in}}%
\pgfpathlineto{\pgfqpoint{3.737037in}{1.130903in}}%
\pgfpathlineto{\pgfqpoint{3.728656in}{1.121478in}}%
\pgfpathlineto{\pgfqpoint{3.720268in}{1.112319in}}%
\pgfpathclose%
\pgfusepath{fill}%
\end{pgfscope}%
\begin{pgfscope}%
\pgfpathrectangle{\pgfqpoint{1.150000in}{0.150000in}}{\pgfqpoint{5.700000in}{5.700000in}}%
\pgfusepath{clip}%
\pgfsetbuttcap%
\pgfsetroundjoin%
\definecolor{currentfill}{rgb}{0.281924,0.089666,0.412415}%
\pgfsetfillcolor{currentfill}%
\pgfsetfillopacity{0.800000}%
\pgfsetlinewidth{0.000000pt}%
\definecolor{currentstroke}{rgb}{0.000000,0.000000,0.000000}%
\pgfsetstrokecolor{currentstroke}%
\pgfsetdash{}{0pt}%
\pgfpathmoveto{\pgfqpoint{3.900767in}{1.198174in}}%
\pgfpathlineto{\pgfqpoint{3.915009in}{1.199666in}}%
\pgfpathlineto{\pgfqpoint{3.929259in}{1.201334in}}%
\pgfpathlineto{\pgfqpoint{3.943520in}{1.203176in}}%
\pgfpathlineto{\pgfqpoint{3.957790in}{1.205194in}}%
\pgfpathlineto{\pgfqpoint{3.966090in}{1.218494in}}%
\pgfpathlineto{\pgfqpoint{3.974384in}{1.231973in}}%
\pgfpathlineto{\pgfqpoint{3.982674in}{1.245626in}}%
\pgfpathlineto{\pgfqpoint{3.990958in}{1.259446in}}%
\pgfpathlineto{\pgfqpoint{3.976694in}{1.256691in}}%
\pgfpathlineto{\pgfqpoint{3.962440in}{1.254112in}}%
\pgfpathlineto{\pgfqpoint{3.948196in}{1.251708in}}%
\pgfpathlineto{\pgfqpoint{3.933962in}{1.249481in}}%
\pgfpathlineto{\pgfqpoint{3.925672in}{1.236386in}}%
\pgfpathlineto{\pgfqpoint{3.917376in}{1.223465in}}%
\pgfpathlineto{\pgfqpoint{3.909074in}{1.210726in}}%
\pgfpathlineto{\pgfqpoint{3.900767in}{1.198174in}}%
\pgfpathclose%
\pgfusepath{fill}%
\end{pgfscope}%
\begin{pgfscope}%
\pgfpathrectangle{\pgfqpoint{1.150000in}{0.150000in}}{\pgfqpoint{5.700000in}{5.700000in}}%
\pgfusepath{clip}%
\pgfsetbuttcap%
\pgfsetroundjoin%
\definecolor{currentfill}{rgb}{0.233603,0.313828,0.543914}%
\pgfsetfillcolor{currentfill}%
\pgfsetfillopacity{0.800000}%
\pgfsetlinewidth{0.000000pt}%
\definecolor{currentstroke}{rgb}{0.000000,0.000000,0.000000}%
\pgfsetstrokecolor{currentstroke}%
\pgfsetdash{}{0pt}%
\pgfpathmoveto{\pgfqpoint{4.360575in}{1.716310in}}%
\pgfpathlineto{\pgfqpoint{4.375013in}{1.725160in}}%
\pgfpathlineto{\pgfqpoint{4.389467in}{1.734190in}}%
\pgfpathlineto{\pgfqpoint{4.403935in}{1.743400in}}%
\pgfpathlineto{\pgfqpoint{4.418419in}{1.752790in}}%
\pgfpathlineto{\pgfqpoint{4.426617in}{1.770854in}}%
\pgfpathlineto{\pgfqpoint{4.434811in}{1.788898in}}%
\pgfpathlineto{\pgfqpoint{4.443001in}{1.806917in}}%
\pgfpathlineto{\pgfqpoint{4.451188in}{1.824906in}}%
\pgfpathlineto{\pgfqpoint{4.436696in}{1.814988in}}%
\pgfpathlineto{\pgfqpoint{4.422220in}{1.805251in}}%
\pgfpathlineto{\pgfqpoint{4.407759in}{1.795694in}}%
\pgfpathlineto{\pgfqpoint{4.393314in}{1.786317in}}%
\pgfpathlineto{\pgfqpoint{4.385135in}{1.768843in}}%
\pgfpathlineto{\pgfqpoint{4.376952in}{1.751347in}}%
\pgfpathlineto{\pgfqpoint{4.368765in}{1.733834in}}%
\pgfpathlineto{\pgfqpoint{4.360575in}{1.716310in}}%
\pgfpathclose%
\pgfusepath{fill}%
\end{pgfscope}%
\begin{pgfscope}%
\pgfpathrectangle{\pgfqpoint{1.150000in}{0.150000in}}{\pgfqpoint{5.700000in}{5.700000in}}%
\pgfusepath{clip}%
\pgfsetbuttcap%
\pgfsetroundjoin%
\definecolor{currentfill}{rgb}{0.119738,0.603785,0.541400}%
\pgfsetfillcolor{currentfill}%
\pgfsetfillopacity{0.800000}%
\pgfsetlinewidth{0.000000pt}%
\definecolor{currentstroke}{rgb}{0.000000,0.000000,0.000000}%
\pgfsetstrokecolor{currentstroke}%
\pgfsetdash{}{0pt}%
\pgfpathmoveto{\pgfqpoint{4.919280in}{2.591559in}}%
\pgfpathlineto{\pgfqpoint{4.934092in}{2.607172in}}%
\pgfpathlineto{\pgfqpoint{4.948925in}{2.622974in}}%
\pgfpathlineto{\pgfqpoint{4.963779in}{2.638964in}}%
\pgfpathlineto{\pgfqpoint{4.978654in}{2.655143in}}%
\pgfpathlineto{\pgfqpoint{4.986693in}{2.670235in}}%
\pgfpathlineto{\pgfqpoint{4.994724in}{2.685139in}}%
\pgfpathlineto{\pgfqpoint{5.002748in}{2.699857in}}%
\pgfpathlineto{\pgfqpoint{5.010765in}{2.714384in}}%
\pgfpathlineto{\pgfqpoint{4.995883in}{2.698007in}}%
\pgfpathlineto{\pgfqpoint{4.981021in}{2.681820in}}%
\pgfpathlineto{\pgfqpoint{4.966181in}{2.665822in}}%
\pgfpathlineto{\pgfqpoint{4.951362in}{2.650012in}}%
\pgfpathlineto{\pgfqpoint{4.943352in}{2.635668in}}%
\pgfpathlineto{\pgfqpoint{4.935335in}{2.621143in}}%
\pgfpathlineto{\pgfqpoint{4.927311in}{2.606440in}}%
\pgfpathlineto{\pgfqpoint{4.919280in}{2.591559in}}%
\pgfpathclose%
\pgfusepath{fill}%
\end{pgfscope}%
\begin{pgfscope}%
\pgfpathrectangle{\pgfqpoint{1.150000in}{0.150000in}}{\pgfqpoint{5.700000in}{5.700000in}}%
\pgfusepath{clip}%
\pgfsetbuttcap%
\pgfsetroundjoin%
\definecolor{currentfill}{rgb}{0.166617,0.463708,0.558119}%
\pgfsetfillcolor{currentfill}%
\pgfsetfillopacity{0.800000}%
\pgfsetlinewidth{0.000000pt}%
\definecolor{currentstroke}{rgb}{0.000000,0.000000,0.000000}%
\pgfsetstrokecolor{currentstroke}%
\pgfsetdash{}{0pt}%
\pgfpathmoveto{\pgfqpoint{4.639936in}{2.153720in}}%
\pgfpathlineto{\pgfqpoint{4.654549in}{2.166343in}}%
\pgfpathlineto{\pgfqpoint{4.669180in}{2.179150in}}%
\pgfpathlineto{\pgfqpoint{4.683830in}{2.192142in}}%
\pgfpathlineto{\pgfqpoint{4.698498in}{2.205318in}}%
\pgfpathlineto{\pgfqpoint{4.706635in}{2.222901in}}%
\pgfpathlineto{\pgfqpoint{4.714768in}{2.240365in}}%
\pgfpathlineto{\pgfqpoint{4.722896in}{2.257706in}}%
\pgfpathlineto{\pgfqpoint{4.731019in}{2.274922in}}%
\pgfpathlineto{\pgfqpoint{4.716341in}{2.261377in}}%
\pgfpathlineto{\pgfqpoint{4.701681in}{2.248017in}}%
\pgfpathlineto{\pgfqpoint{4.687041in}{2.234843in}}%
\pgfpathlineto{\pgfqpoint{4.672419in}{2.221853in}}%
\pgfpathlineto{\pgfqpoint{4.664305in}{2.204993in}}%
\pgfpathlineto{\pgfqpoint{4.656187in}{2.188015in}}%
\pgfpathlineto{\pgfqpoint{4.648064in}{2.170923in}}%
\pgfpathlineto{\pgfqpoint{4.639936in}{2.153720in}}%
\pgfpathclose%
\pgfusepath{fill}%
\end{pgfscope}%
\begin{pgfscope}%
\pgfpathrectangle{\pgfqpoint{1.150000in}{0.150000in}}{\pgfqpoint{5.700000in}{5.700000in}}%
\pgfusepath{clip}%
\pgfsetbuttcap%
\pgfsetroundjoin%
\definecolor{currentfill}{rgb}{0.395174,0.797475,0.367757}%
\pgfsetfillcolor{currentfill}%
\pgfsetfillopacity{0.800000}%
\pgfsetlinewidth{0.000000pt}%
\definecolor{currentstroke}{rgb}{0.000000,0.000000,0.000000}%
\pgfsetstrokecolor{currentstroke}%
\pgfsetdash{}{0pt}%
\pgfpathmoveto{\pgfqpoint{5.412366in}{3.257091in}}%
\pgfpathlineto{\pgfqpoint{5.427564in}{3.276334in}}%
\pgfpathlineto{\pgfqpoint{5.442785in}{3.295772in}}%
\pgfpathlineto{\pgfqpoint{5.458032in}{3.315404in}}%
\pgfpathlineto{\pgfqpoint{5.473304in}{3.335231in}}%
\pgfpathlineto{\pgfqpoint{5.481051in}{3.343718in}}%
\pgfpathlineto{\pgfqpoint{5.488787in}{3.351977in}}%
\pgfpathlineto{\pgfqpoint{5.496512in}{3.360009in}}%
\pgfpathlineto{\pgfqpoint{5.504224in}{3.367816in}}%
\pgfpathlineto{\pgfqpoint{5.488957in}{3.348085in}}%
\pgfpathlineto{\pgfqpoint{5.473715in}{3.328549in}}%
\pgfpathlineto{\pgfqpoint{5.458498in}{3.309206in}}%
\pgfpathlineto{\pgfqpoint{5.443306in}{3.290058in}}%
\pgfpathlineto{\pgfqpoint{5.435588in}{3.282141in}}%
\pgfpathlineto{\pgfqpoint{5.427858in}{3.274009in}}%
\pgfpathlineto{\pgfqpoint{5.420118in}{3.265659in}}%
\pgfpathlineto{\pgfqpoint{5.412366in}{3.257091in}}%
\pgfpathclose%
\pgfusepath{fill}%
\end{pgfscope}%
\begin{pgfscope}%
\pgfpathrectangle{\pgfqpoint{1.150000in}{0.150000in}}{\pgfqpoint{5.700000in}{5.700000in}}%
\pgfusepath{clip}%
\pgfsetbuttcap%
\pgfsetroundjoin%
\definecolor{currentfill}{rgb}{0.283187,0.125848,0.444960}%
\pgfsetfillcolor{currentfill}%
\pgfsetfillopacity{0.800000}%
\pgfsetlinewidth{0.000000pt}%
\definecolor{currentstroke}{rgb}{0.000000,0.000000,0.000000}%
\pgfsetstrokecolor{currentstroke}%
\pgfsetdash{}{0pt}%
\pgfpathmoveto{\pgfqpoint{3.990958in}{1.259446in}}%
\pgfpathlineto{\pgfqpoint{4.005232in}{1.262376in}}%
\pgfpathlineto{\pgfqpoint{4.019517in}{1.265481in}}%
\pgfpathlineto{\pgfqpoint{4.033813in}{1.268762in}}%
\pgfpathlineto{\pgfqpoint{4.048120in}{1.272219in}}%
\pgfpathlineto{\pgfqpoint{4.056395in}{1.286916in}}%
\pgfpathlineto{\pgfqpoint{4.064666in}{1.301758in}}%
\pgfpathlineto{\pgfqpoint{4.072932in}{1.316738in}}%
\pgfpathlineto{\pgfqpoint{4.081194in}{1.331849in}}%
\pgfpathlineto{\pgfqpoint{4.066890in}{1.327685in}}%
\pgfpathlineto{\pgfqpoint{4.052597in}{1.323696in}}%
\pgfpathlineto{\pgfqpoint{4.038316in}{1.319883in}}%
\pgfpathlineto{\pgfqpoint{4.024045in}{1.316247in}}%
\pgfpathlineto{\pgfqpoint{4.015780in}{1.301831in}}%
\pgfpathlineto{\pgfqpoint{4.007511in}{1.287555in}}%
\pgfpathlineto{\pgfqpoint{3.999237in}{1.273424in}}%
\pgfpathlineto{\pgfqpoint{3.990958in}{1.259446in}}%
\pgfpathclose%
\pgfusepath{fill}%
\end{pgfscope}%
\begin{pgfscope}%
\pgfpathrectangle{\pgfqpoint{1.150000in}{0.150000in}}{\pgfqpoint{5.700000in}{5.700000in}}%
\pgfusepath{clip}%
\pgfsetbuttcap%
\pgfsetroundjoin%
\definecolor{currentfill}{rgb}{0.137339,0.662252,0.515571}%
\pgfsetfillcolor{currentfill}%
\pgfsetfillopacity{0.800000}%
\pgfsetlinewidth{0.000000pt}%
\definecolor{currentstroke}{rgb}{0.000000,0.000000,0.000000}%
\pgfsetstrokecolor{currentstroke}%
\pgfsetdash{}{0pt}%
\pgfpathmoveto{\pgfqpoint{5.042756in}{2.770577in}}%
\pgfpathlineto{\pgfqpoint{5.057667in}{2.787305in}}%
\pgfpathlineto{\pgfqpoint{5.072599in}{2.804224in}}%
\pgfpathlineto{\pgfqpoint{5.087553in}{2.821334in}}%
\pgfpathlineto{\pgfqpoint{5.102530in}{2.838634in}}%
\pgfpathlineto{\pgfqpoint{5.110513in}{2.852336in}}%
\pgfpathlineto{\pgfqpoint{5.118489in}{2.865833in}}%
\pgfpathlineto{\pgfqpoint{5.126456in}{2.879124in}}%
\pgfpathlineto{\pgfqpoint{5.134414in}{2.892207in}}%
\pgfpathlineto{\pgfqpoint{5.119433in}{2.874781in}}%
\pgfpathlineto{\pgfqpoint{5.104473in}{2.857545in}}%
\pgfpathlineto{\pgfqpoint{5.089536in}{2.840500in}}%
\pgfpathlineto{\pgfqpoint{5.074621in}{2.823646in}}%
\pgfpathlineto{\pgfqpoint{5.066667in}{2.810674in}}%
\pgfpathlineto{\pgfqpoint{5.058705in}{2.797505in}}%
\pgfpathlineto{\pgfqpoint{5.050734in}{2.784139in}}%
\pgfpathlineto{\pgfqpoint{5.042756in}{2.770577in}}%
\pgfpathclose%
\pgfusepath{fill}%
\end{pgfscope}%
\begin{pgfscope}%
\pgfpathrectangle{\pgfqpoint{1.150000in}{0.150000in}}{\pgfqpoint{5.700000in}{5.700000in}}%
\pgfusepath{clip}%
\pgfsetbuttcap%
\pgfsetroundjoin%
\definecolor{currentfill}{rgb}{0.201239,0.383670,0.554294}%
\pgfsetfillcolor{currentfill}%
\pgfsetfillopacity{0.800000}%
\pgfsetlinewidth{0.000000pt}%
\definecolor{currentstroke}{rgb}{0.000000,0.000000,0.000000}%
\pgfsetstrokecolor{currentstroke}%
\pgfsetdash{}{0pt}%
\pgfpathmoveto{\pgfqpoint{4.483900in}{1.896471in}}%
\pgfpathlineto{\pgfqpoint{4.498416in}{1.907066in}}%
\pgfpathlineto{\pgfqpoint{4.512950in}{1.917843in}}%
\pgfpathlineto{\pgfqpoint{4.527499in}{1.928801in}}%
\pgfpathlineto{\pgfqpoint{4.542066in}{1.939941in}}%
\pgfpathlineto{\pgfqpoint{4.550244in}{1.958189in}}%
\pgfpathlineto{\pgfqpoint{4.558418in}{1.976372in}}%
\pgfpathlineto{\pgfqpoint{4.566588in}{1.994486in}}%
\pgfpathlineto{\pgfqpoint{4.574755in}{2.012526in}}%
\pgfpathlineto{\pgfqpoint{4.560178in}{2.000919in}}%
\pgfpathlineto{\pgfqpoint{4.545619in}{1.989496in}}%
\pgfpathlineto{\pgfqpoint{4.531077in}{1.978254in}}%
\pgfpathlineto{\pgfqpoint{4.516552in}{1.967195in}}%
\pgfpathlineto{\pgfqpoint{4.508395in}{1.949608in}}%
\pgfpathlineto{\pgfqpoint{4.500234in}{1.931956in}}%
\pgfpathlineto{\pgfqpoint{4.492069in}{1.914242in}}%
\pgfpathlineto{\pgfqpoint{4.483900in}{1.896471in}}%
\pgfpathclose%
\pgfusepath{fill}%
\end{pgfscope}%
\begin{pgfscope}%
\pgfpathrectangle{\pgfqpoint{1.150000in}{0.150000in}}{\pgfqpoint{5.700000in}{5.700000in}}%
\pgfusepath{clip}%
\pgfsetbuttcap%
\pgfsetroundjoin%
\definecolor{currentfill}{rgb}{0.585678,0.846661,0.249897}%
\pgfsetfillcolor{currentfill}%
\pgfsetfillopacity{0.800000}%
\pgfsetlinewidth{0.000000pt}%
\definecolor{currentstroke}{rgb}{0.000000,0.000000,0.000000}%
\pgfsetstrokecolor{currentstroke}%
\pgfsetdash{}{0pt}%
\pgfpathmoveto{\pgfqpoint{5.626755in}{3.502043in}}%
\pgfpathlineto{\pgfqpoint{5.642133in}{3.522445in}}%
\pgfpathlineto{\pgfqpoint{5.657538in}{3.543043in}}%
\pgfpathlineto{\pgfqpoint{5.672969in}{3.563838in}}%
\pgfpathlineto{\pgfqpoint{5.680556in}{3.569361in}}%
\pgfpathlineto{\pgfqpoint{5.688130in}{3.574664in}}%
\pgfpathlineto{\pgfqpoint{5.695691in}{3.579748in}}%
\pgfpathlineto{\pgfqpoint{5.703239in}{3.584616in}}%
\pgfpathlineto{\pgfqpoint{5.687819in}{3.564032in}}%
\pgfpathlineto{\pgfqpoint{5.672426in}{3.543644in}}%
\pgfpathlineto{\pgfqpoint{5.657059in}{3.523452in}}%
\pgfpathlineto{\pgfqpoint{5.649502in}{3.518416in}}%
\pgfpathlineto{\pgfqpoint{5.641932in}{3.513171in}}%
\pgfpathlineto{\pgfqpoint{5.634350in}{3.507714in}}%
\pgfpathlineto{\pgfqpoint{5.626755in}{3.502043in}}%
\pgfpathclose%
\pgfusepath{fill}%
\end{pgfscope}%
\begin{pgfscope}%
\pgfpathrectangle{\pgfqpoint{1.150000in}{0.150000in}}{\pgfqpoint{5.700000in}{5.700000in}}%
\pgfusepath{clip}%
\pgfsetbuttcap%
\pgfsetroundjoin%
\definecolor{currentfill}{rgb}{0.288921,0.758394,0.428426}%
\pgfsetfillcolor{currentfill}%
\pgfsetfillopacity{0.800000}%
\pgfsetlinewidth{0.000000pt}%
\definecolor{currentstroke}{rgb}{0.000000,0.000000,0.000000}%
\pgfsetstrokecolor{currentstroke}%
\pgfsetdash{}{0pt}%
\pgfpathmoveto{\pgfqpoint{5.289399in}{3.105193in}}%
\pgfpathlineto{\pgfqpoint{5.304503in}{3.123741in}}%
\pgfpathlineto{\pgfqpoint{5.319631in}{3.142482in}}%
\pgfpathlineto{\pgfqpoint{5.334783in}{3.161416in}}%
\pgfpathlineto{\pgfqpoint{5.349960in}{3.180544in}}%
\pgfpathlineto{\pgfqpoint{5.357798in}{3.190899in}}%
\pgfpathlineto{\pgfqpoint{5.365626in}{3.201028in}}%
\pgfpathlineto{\pgfqpoint{5.373443in}{3.210931in}}%
\pgfpathlineto{\pgfqpoint{5.381250in}{3.220609in}}%
\pgfpathlineto{\pgfqpoint{5.366074in}{3.201501in}}%
\pgfpathlineto{\pgfqpoint{5.350923in}{3.182588in}}%
\pgfpathlineto{\pgfqpoint{5.335796in}{3.163867in}}%
\pgfpathlineto{\pgfqpoint{5.320692in}{3.145340in}}%
\pgfpathlineto{\pgfqpoint{5.312884in}{3.135627in}}%
\pgfpathlineto{\pgfqpoint{5.305066in}{3.125699in}}%
\pgfpathlineto{\pgfqpoint{5.297237in}{3.115555in}}%
\pgfpathlineto{\pgfqpoint{5.289399in}{3.105193in}}%
\pgfpathclose%
\pgfusepath{fill}%
\end{pgfscope}%
\begin{pgfscope}%
\pgfpathrectangle{\pgfqpoint{1.150000in}{0.150000in}}{\pgfqpoint{5.700000in}{5.700000in}}%
\pgfusepath{clip}%
\pgfsetbuttcap%
\pgfsetroundjoin%
\definecolor{currentfill}{rgb}{0.202219,0.715272,0.476084}%
\pgfsetfillcolor{currentfill}%
\pgfsetfillopacity{0.800000}%
\pgfsetlinewidth{0.000000pt}%
\definecolor{currentstroke}{rgb}{0.000000,0.000000,0.000000}%
\pgfsetstrokecolor{currentstroke}%
\pgfsetdash{}{0pt}%
\pgfpathmoveto{\pgfqpoint{5.166161in}{2.942462in}}%
\pgfpathlineto{\pgfqpoint{5.181169in}{2.960170in}}%
\pgfpathlineto{\pgfqpoint{5.196200in}{2.978070in}}%
\pgfpathlineto{\pgfqpoint{5.211254in}{2.996162in}}%
\pgfpathlineto{\pgfqpoint{5.226332in}{3.014447in}}%
\pgfpathlineto{\pgfqpoint{5.234249in}{3.026556in}}%
\pgfpathlineto{\pgfqpoint{5.242157in}{3.038446in}}%
\pgfpathlineto{\pgfqpoint{5.250055in}{3.050117in}}%
\pgfpathlineto{\pgfqpoint{5.257944in}{3.061569in}}%
\pgfpathlineto{\pgfqpoint{5.242864in}{3.043231in}}%
\pgfpathlineto{\pgfqpoint{5.227807in}{3.025085in}}%
\pgfpathlineto{\pgfqpoint{5.212773in}{3.007132in}}%
\pgfpathlineto{\pgfqpoint{5.197763in}{2.989370in}}%
\pgfpathlineto{\pgfqpoint{5.189876in}{2.977957in}}%
\pgfpathlineto{\pgfqpoint{5.181980in}{2.966335in}}%
\pgfpathlineto{\pgfqpoint{5.174075in}{2.954503in}}%
\pgfpathlineto{\pgfqpoint{5.166161in}{2.942462in}}%
\pgfpathclose%
\pgfusepath{fill}%
\end{pgfscope}%
\begin{pgfscope}%
\pgfpathrectangle{\pgfqpoint{1.150000in}{0.150000in}}{\pgfqpoint{5.700000in}{5.700000in}}%
\pgfusepath{clip}%
\pgfsetbuttcap%
\pgfsetroundjoin%
\definecolor{currentfill}{rgb}{0.140536,0.530132,0.555659}%
\pgfsetfillcolor{currentfill}%
\pgfsetfillopacity{0.800000}%
\pgfsetlinewidth{0.000000pt}%
\definecolor{currentstroke}{rgb}{0.000000,0.000000,0.000000}%
\pgfsetstrokecolor{currentstroke}%
\pgfsetdash{}{0pt}%
\pgfpathmoveto{\pgfqpoint{4.763459in}{2.342465in}}%
\pgfpathlineto{\pgfqpoint{4.778166in}{2.356530in}}%
\pgfpathlineto{\pgfqpoint{4.792892in}{2.370781in}}%
\pgfpathlineto{\pgfqpoint{4.807637in}{2.385218in}}%
\pgfpathlineto{\pgfqpoint{4.822403in}{2.399842in}}%
\pgfpathlineto{\pgfqpoint{4.830509in}{2.416698in}}%
\pgfpathlineto{\pgfqpoint{4.838610in}{2.433402in}}%
\pgfpathlineto{\pgfqpoint{4.846705in}{2.449951in}}%
\pgfpathlineto{\pgfqpoint{4.854793in}{2.466344in}}%
\pgfpathlineto{\pgfqpoint{4.840018in}{2.451418in}}%
\pgfpathlineto{\pgfqpoint{4.825263in}{2.436679in}}%
\pgfpathlineto{\pgfqpoint{4.810528in}{2.422127in}}%
\pgfpathlineto{\pgfqpoint{4.795812in}{2.407762in}}%
\pgfpathlineto{\pgfqpoint{4.787733in}{2.391657in}}%
\pgfpathlineto{\pgfqpoint{4.779647in}{2.375405in}}%
\pgfpathlineto{\pgfqpoint{4.771556in}{2.359006in}}%
\pgfpathlineto{\pgfqpoint{4.763459in}{2.342465in}}%
\pgfpathclose%
\pgfusepath{fill}%
\end{pgfscope}%
\begin{pgfscope}%
\pgfpathrectangle{\pgfqpoint{1.150000in}{0.150000in}}{\pgfqpoint{5.700000in}{5.700000in}}%
\pgfusepath{clip}%
\pgfsetbuttcap%
\pgfsetroundjoin%
\definecolor{currentfill}{rgb}{0.267968,0.223549,0.512008}%
\pgfsetfillcolor{currentfill}%
\pgfsetfillopacity{0.800000}%
\pgfsetlinewidth{0.000000pt}%
\definecolor{currentstroke}{rgb}{0.000000,0.000000,0.000000}%
\pgfsetstrokecolor{currentstroke}%
\pgfsetdash{}{0pt}%
\pgfpathmoveto{\pgfqpoint{4.204480in}{1.480396in}}%
\pgfpathlineto{\pgfqpoint{4.218847in}{1.486769in}}%
\pgfpathlineto{\pgfqpoint{4.233227in}{1.493319in}}%
\pgfpathlineto{\pgfqpoint{4.247621in}{1.500046in}}%
\pgfpathlineto{\pgfqpoint{4.262028in}{1.506950in}}%
\pgfpathlineto{\pgfqpoint{4.270259in}{1.524211in}}%
\pgfpathlineto{\pgfqpoint{4.278486in}{1.541524in}}%
\pgfpathlineto{\pgfqpoint{4.286710in}{1.558885in}}%
\pgfpathlineto{\pgfqpoint{4.294931in}{1.576286in}}%
\pgfpathlineto{\pgfqpoint{4.280519in}{1.568761in}}%
\pgfpathlineto{\pgfqpoint{4.266121in}{1.561414in}}%
\pgfpathlineto{\pgfqpoint{4.251737in}{1.554245in}}%
\pgfpathlineto{\pgfqpoint{4.237367in}{1.547253in}}%
\pgfpathlineto{\pgfqpoint{4.229150in}{1.530460in}}%
\pgfpathlineto{\pgfqpoint{4.220930in}{1.513715in}}%
\pgfpathlineto{\pgfqpoint{4.212707in}{1.497025in}}%
\pgfpathlineto{\pgfqpoint{4.204480in}{1.480396in}}%
\pgfpathclose%
\pgfusepath{fill}%
\end{pgfscope}%
\begin{pgfscope}%
\pgfpathrectangle{\pgfqpoint{1.150000in}{0.150000in}}{\pgfqpoint{5.700000in}{5.700000in}}%
\pgfusepath{clip}%
\pgfsetbuttcap%
\pgfsetroundjoin%
\definecolor{currentfill}{rgb}{0.280868,0.160771,0.472899}%
\pgfsetfillcolor{currentfill}%
\pgfsetfillopacity{0.800000}%
\pgfsetlinewidth{0.000000pt}%
\definecolor{currentstroke}{rgb}{0.000000,0.000000,0.000000}%
\pgfsetstrokecolor{currentstroke}%
\pgfsetdash{}{0pt}%
\pgfpathmoveto{\pgfqpoint{4.081194in}{1.331849in}}%
\pgfpathlineto{\pgfqpoint{4.095510in}{1.336189in}}%
\pgfpathlineto{\pgfqpoint{4.109838in}{1.340705in}}%
\pgfpathlineto{\pgfqpoint{4.124177in}{1.345397in}}%
\pgfpathlineto{\pgfqpoint{4.138528in}{1.350264in}}%
\pgfpathlineto{\pgfqpoint{4.146785in}{1.366190in}}%
\pgfpathlineto{\pgfqpoint{4.155039in}{1.382226in}}%
\pgfpathlineto{\pgfqpoint{4.163288in}{1.398365in}}%
\pgfpathlineto{\pgfqpoint{4.171534in}{1.414602in}}%
\pgfpathlineto{\pgfqpoint{4.157182in}{1.409055in}}%
\pgfpathlineto{\pgfqpoint{4.142842in}{1.403684in}}%
\pgfpathlineto{\pgfqpoint{4.128515in}{1.398489in}}%
\pgfpathlineto{\pgfqpoint{4.114200in}{1.393471in}}%
\pgfpathlineto{\pgfqpoint{4.105955in}{1.377902in}}%
\pgfpathlineto{\pgfqpoint{4.097705in}{1.362437in}}%
\pgfpathlineto{\pgfqpoint{4.089452in}{1.347084in}}%
\pgfpathlineto{\pgfqpoint{4.081194in}{1.331849in}}%
\pgfpathclose%
\pgfusepath{fill}%
\end{pgfscope}%
\begin{pgfscope}%
\pgfpathrectangle{\pgfqpoint{1.150000in}{0.150000in}}{\pgfqpoint{5.700000in}{5.700000in}}%
\pgfusepath{clip}%
\pgfsetbuttcap%
\pgfsetroundjoin%
\definecolor{currentfill}{rgb}{0.243113,0.292092,0.538516}%
\pgfsetfillcolor{currentfill}%
\pgfsetfillopacity{0.800000}%
\pgfsetlinewidth{0.000000pt}%
\definecolor{currentstroke}{rgb}{0.000000,0.000000,0.000000}%
\pgfsetstrokecolor{currentstroke}%
\pgfsetdash{}{0pt}%
\pgfpathmoveto{\pgfqpoint{4.327780in}{1.646189in}}%
\pgfpathlineto{\pgfqpoint{4.342212in}{1.654482in}}%
\pgfpathlineto{\pgfqpoint{4.356659in}{1.662953in}}%
\pgfpathlineto{\pgfqpoint{4.371120in}{1.671603in}}%
\pgfpathlineto{\pgfqpoint{4.385597in}{1.680432in}}%
\pgfpathlineto{\pgfqpoint{4.393807in}{1.698526in}}%
\pgfpathlineto{\pgfqpoint{4.402015in}{1.716621in}}%
\pgfpathlineto{\pgfqpoint{4.410219in}{1.734710in}}%
\pgfpathlineto{\pgfqpoint{4.418419in}{1.752790in}}%
\pgfpathlineto{\pgfqpoint{4.403935in}{1.743400in}}%
\pgfpathlineto{\pgfqpoint{4.389467in}{1.734190in}}%
\pgfpathlineto{\pgfqpoint{4.375013in}{1.725160in}}%
\pgfpathlineto{\pgfqpoint{4.360575in}{1.716310in}}%
\pgfpathlineto{\pgfqpoint{4.352382in}{1.698778in}}%
\pgfpathlineto{\pgfqpoint{4.344185in}{1.681243in}}%
\pgfpathlineto{\pgfqpoint{4.335984in}{1.663712in}}%
\pgfpathlineto{\pgfqpoint{4.327780in}{1.646189in}}%
\pgfpathclose%
\pgfusepath{fill}%
\end{pgfscope}%
\begin{pgfscope}%
\pgfpathrectangle{\pgfqpoint{1.150000in}{0.150000in}}{\pgfqpoint{5.700000in}{5.700000in}}%
\pgfusepath{clip}%
\pgfsetbuttcap%
\pgfsetroundjoin%
\definecolor{currentfill}{rgb}{0.172719,0.448791,0.557885}%
\pgfsetfillcolor{currentfill}%
\pgfsetfillopacity{0.800000}%
\pgfsetlinewidth{0.000000pt}%
\definecolor{currentstroke}{rgb}{0.000000,0.000000,0.000000}%
\pgfsetstrokecolor{currentstroke}%
\pgfsetdash{}{0pt}%
\pgfpathmoveto{\pgfqpoint{4.607379in}{2.083868in}}%
\pgfpathlineto{\pgfqpoint{4.621983in}{2.096091in}}%
\pgfpathlineto{\pgfqpoint{4.636604in}{2.108497in}}%
\pgfpathlineto{\pgfqpoint{4.651244in}{2.121087in}}%
\pgfpathlineto{\pgfqpoint{4.665901in}{2.133860in}}%
\pgfpathlineto{\pgfqpoint{4.674057in}{2.151886in}}%
\pgfpathlineto{\pgfqpoint{4.682208in}{2.169807in}}%
\pgfpathlineto{\pgfqpoint{4.690355in}{2.187619in}}%
\pgfpathlineto{\pgfqpoint{4.698498in}{2.205318in}}%
\pgfpathlineto{\pgfqpoint{4.683830in}{2.192142in}}%
\pgfpathlineto{\pgfqpoint{4.669180in}{2.179150in}}%
\pgfpathlineto{\pgfqpoint{4.654549in}{2.166343in}}%
\pgfpathlineto{\pgfqpoint{4.639936in}{2.153720in}}%
\pgfpathlineto{\pgfqpoint{4.631803in}{2.136410in}}%
\pgfpathlineto{\pgfqpoint{4.623666in}{2.118995in}}%
\pgfpathlineto{\pgfqpoint{4.615525in}{2.101480in}}%
\pgfpathlineto{\pgfqpoint{4.607379in}{2.083868in}}%
\pgfpathclose%
\pgfusepath{fill}%
\end{pgfscope}%
\begin{pgfscope}%
\pgfpathrectangle{\pgfqpoint{1.150000in}{0.150000in}}{\pgfqpoint{5.700000in}{5.700000in}}%
\pgfusepath{clip}%
\pgfsetbuttcap%
\pgfsetroundjoin%
\definecolor{currentfill}{rgb}{0.121831,0.589055,0.545623}%
\pgfsetfillcolor{currentfill}%
\pgfsetfillopacity{0.800000}%
\pgfsetlinewidth{0.000000pt}%
\definecolor{currentstroke}{rgb}{0.000000,0.000000,0.000000}%
\pgfsetstrokecolor{currentstroke}%
\pgfsetdash{}{0pt}%
\pgfpathmoveto{\pgfqpoint{4.887088in}{2.530297in}}%
\pgfpathlineto{\pgfqpoint{4.901892in}{2.545678in}}%
\pgfpathlineto{\pgfqpoint{4.916717in}{2.561247in}}%
\pgfpathlineto{\pgfqpoint{4.931563in}{2.577004in}}%
\pgfpathlineto{\pgfqpoint{4.946429in}{2.592950in}}%
\pgfpathlineto{\pgfqpoint{4.954496in}{2.608769in}}%
\pgfpathlineto{\pgfqpoint{4.962555in}{2.624409in}}%
\pgfpathlineto{\pgfqpoint{4.970608in}{2.639868in}}%
\pgfpathlineto{\pgfqpoint{4.978654in}{2.655143in}}%
\pgfpathlineto{\pgfqpoint{4.963779in}{2.638964in}}%
\pgfpathlineto{\pgfqpoint{4.948925in}{2.622974in}}%
\pgfpathlineto{\pgfqpoint{4.934092in}{2.607172in}}%
\pgfpathlineto{\pgfqpoint{4.919280in}{2.591559in}}%
\pgfpathlineto{\pgfqpoint{4.911242in}{2.576503in}}%
\pgfpathlineto{\pgfqpoint{4.903197in}{2.561272in}}%
\pgfpathlineto{\pgfqpoint{4.895146in}{2.545870in}}%
\pgfpathlineto{\pgfqpoint{4.887088in}{2.530297in}}%
\pgfpathclose%
\pgfusepath{fill}%
\end{pgfscope}%
\begin{pgfscope}%
\pgfpathrectangle{\pgfqpoint{1.150000in}{0.150000in}}{\pgfqpoint{5.700000in}{5.700000in}}%
\pgfusepath{clip}%
\pgfsetbuttcap%
\pgfsetroundjoin%
\definecolor{currentfill}{rgb}{0.277941,0.056324,0.381191}%
\pgfsetfillcolor{currentfill}%
\pgfsetfillopacity{0.800000}%
\pgfsetlinewidth{0.000000pt}%
\definecolor{currentstroke}{rgb}{0.000000,0.000000,0.000000}%
\pgfsetstrokecolor{currentstroke}%
\pgfsetdash{}{0pt}%
\pgfpathmoveto{\pgfqpoint{3.777115in}{1.107470in}}%
\pgfpathlineto{\pgfqpoint{3.791345in}{1.106697in}}%
\pgfpathlineto{\pgfqpoint{3.805584in}{1.106100in}}%
\pgfpathlineto{\pgfqpoint{3.819830in}{1.105677in}}%
\pgfpathlineto{\pgfqpoint{3.834084in}{1.105430in}}%
\pgfpathlineto{\pgfqpoint{3.842443in}{1.116207in}}%
\pgfpathlineto{\pgfqpoint{3.850795in}{1.127232in}}%
\pgfpathlineto{\pgfqpoint{3.859139in}{1.138498in}}%
\pgfpathlineto{\pgfqpoint{3.867478in}{1.149998in}}%
\pgfpathlineto{\pgfqpoint{3.853235in}{1.149448in}}%
\pgfpathlineto{\pgfqpoint{3.839002in}{1.149072in}}%
\pgfpathlineto{\pgfqpoint{3.824777in}{1.148873in}}%
\pgfpathlineto{\pgfqpoint{3.810560in}{1.148849in}}%
\pgfpathlineto{\pgfqpoint{3.802210in}{1.138135in}}%
\pgfpathlineto{\pgfqpoint{3.793852in}{1.127662in}}%
\pgfpathlineto{\pgfqpoint{3.785487in}{1.117437in}}%
\pgfpathlineto{\pgfqpoint{3.777115in}{1.107470in}}%
\pgfpathclose%
\pgfusepath{fill}%
\end{pgfscope}%
\begin{pgfscope}%
\pgfpathrectangle{\pgfqpoint{1.150000in}{0.150000in}}{\pgfqpoint{5.700000in}{5.700000in}}%
\pgfusepath{clip}%
\pgfsetbuttcap%
\pgfsetroundjoin%
\definecolor{currentfill}{rgb}{0.280894,0.078907,0.402329}%
\pgfsetfillcolor{currentfill}%
\pgfsetfillopacity{0.800000}%
\pgfsetlinewidth{0.000000pt}%
\definecolor{currentstroke}{rgb}{0.000000,0.000000,0.000000}%
\pgfsetstrokecolor{currentstroke}%
\pgfsetdash{}{0pt}%
\pgfpathmoveto{\pgfqpoint{3.867478in}{1.149998in}}%
\pgfpathlineto{\pgfqpoint{3.881728in}{1.150723in}}%
\pgfpathlineto{\pgfqpoint{3.895988in}{1.151623in}}%
\pgfpathlineto{\pgfqpoint{3.910257in}{1.152698in}}%
\pgfpathlineto{\pgfqpoint{3.924536in}{1.153948in}}%
\pgfpathlineto{\pgfqpoint{3.932858in}{1.166452in}}%
\pgfpathlineto{\pgfqpoint{3.941174in}{1.179166in}}%
\pgfpathlineto{\pgfqpoint{3.949485in}{1.192083in}}%
\pgfpathlineto{\pgfqpoint{3.957790in}{1.205194in}}%
\pgfpathlineto{\pgfqpoint{3.943520in}{1.203176in}}%
\pgfpathlineto{\pgfqpoint{3.929259in}{1.201334in}}%
\pgfpathlineto{\pgfqpoint{3.915009in}{1.199666in}}%
\pgfpathlineto{\pgfqpoint{3.900767in}{1.198174in}}%
\pgfpathlineto{\pgfqpoint{3.892454in}{1.185818in}}%
\pgfpathlineto{\pgfqpoint{3.884135in}{1.173665in}}%
\pgfpathlineto{\pgfqpoint{3.875809in}{1.161723in}}%
\pgfpathlineto{\pgfqpoint{3.867478in}{1.149998in}}%
\pgfpathclose%
\pgfusepath{fill}%
\end{pgfscope}%
\begin{pgfscope}%
\pgfpathrectangle{\pgfqpoint{1.150000in}{0.150000in}}{\pgfqpoint{5.700000in}{5.700000in}}%
\pgfusepath{clip}%
\pgfsetbuttcap%
\pgfsetroundjoin%
\definecolor{currentfill}{rgb}{0.210503,0.363727,0.552206}%
\pgfsetfillcolor{currentfill}%
\pgfsetfillopacity{0.800000}%
\pgfsetlinewidth{0.000000pt}%
\definecolor{currentstroke}{rgb}{0.000000,0.000000,0.000000}%
\pgfsetstrokecolor{currentstroke}%
\pgfsetdash{}{0pt}%
\pgfpathmoveto{\pgfqpoint{4.451188in}{1.824906in}}%
\pgfpathlineto{\pgfqpoint{4.465696in}{1.835005in}}%
\pgfpathlineto{\pgfqpoint{4.480220in}{1.845284in}}%
\pgfpathlineto{\pgfqpoint{4.494761in}{1.855743in}}%
\pgfpathlineto{\pgfqpoint{4.509318in}{1.866384in}}%
\pgfpathlineto{\pgfqpoint{4.517510in}{1.884849in}}%
\pgfpathlineto{\pgfqpoint{4.525699in}{1.903266in}}%
\pgfpathlineto{\pgfqpoint{4.533884in}{1.921632in}}%
\pgfpathlineto{\pgfqpoint{4.542066in}{1.939941in}}%
\pgfpathlineto{\pgfqpoint{4.527499in}{1.928801in}}%
\pgfpathlineto{\pgfqpoint{4.512950in}{1.917843in}}%
\pgfpathlineto{\pgfqpoint{4.498416in}{1.907066in}}%
\pgfpathlineto{\pgfqpoint{4.483900in}{1.896471in}}%
\pgfpathlineto{\pgfqpoint{4.475727in}{1.878648in}}%
\pgfpathlineto{\pgfqpoint{4.467551in}{1.860776in}}%
\pgfpathlineto{\pgfqpoint{4.459371in}{1.842861in}}%
\pgfpathlineto{\pgfqpoint{4.451188in}{1.824906in}}%
\pgfpathclose%
\pgfusepath{fill}%
\end{pgfscope}%
\begin{pgfscope}%
\pgfpathrectangle{\pgfqpoint{1.150000in}{0.150000in}}{\pgfqpoint{5.700000in}{5.700000in}}%
\pgfusepath{clip}%
\pgfsetbuttcap%
\pgfsetroundjoin%
\definecolor{currentfill}{rgb}{0.496615,0.826376,0.306377}%
\pgfsetfillcolor{currentfill}%
\pgfsetfillopacity{0.800000}%
\pgfsetlinewidth{0.000000pt}%
\definecolor{currentstroke}{rgb}{0.000000,0.000000,0.000000}%
\pgfsetstrokecolor{currentstroke}%
\pgfsetdash{}{0pt}%
\pgfpathmoveto{\pgfqpoint{5.504224in}{3.367816in}}%
\pgfpathlineto{\pgfqpoint{5.519516in}{3.387742in}}%
\pgfpathlineto{\pgfqpoint{5.534834in}{3.407864in}}%
\pgfpathlineto{\pgfqpoint{5.550177in}{3.428181in}}%
\pgfpathlineto{\pgfqpoint{5.565546in}{3.448695in}}%
\pgfpathlineto{\pgfqpoint{5.573240in}{3.456158in}}%
\pgfpathlineto{\pgfqpoint{5.580923in}{3.463391in}}%
\pgfpathlineto{\pgfqpoint{5.588592in}{3.470395in}}%
\pgfpathlineto{\pgfqpoint{5.596250in}{3.477171in}}%
\pgfpathlineto{\pgfqpoint{5.580888in}{3.456792in}}%
\pgfpathlineto{\pgfqpoint{5.565552in}{3.436610in}}%
\pgfpathlineto{\pgfqpoint{5.550241in}{3.416622in}}%
\pgfpathlineto{\pgfqpoint{5.534956in}{3.396830in}}%
\pgfpathlineto{\pgfqpoint{5.527291in}{3.389905in}}%
\pgfpathlineto{\pgfqpoint{5.519614in}{3.382763in}}%
\pgfpathlineto{\pgfqpoint{5.511925in}{3.375400in}}%
\pgfpathlineto{\pgfqpoint{5.504224in}{3.367816in}}%
\pgfpathclose%
\pgfusepath{fill}%
\end{pgfscope}%
\begin{pgfscope}%
\pgfpathrectangle{\pgfqpoint{1.150000in}{0.150000in}}{\pgfqpoint{5.700000in}{5.700000in}}%
\pgfusepath{clip}%
\pgfsetbuttcap%
\pgfsetroundjoin%
\definecolor{currentfill}{rgb}{0.283091,0.110553,0.431554}%
\pgfsetfillcolor{currentfill}%
\pgfsetfillopacity{0.800000}%
\pgfsetlinewidth{0.000000pt}%
\definecolor{currentstroke}{rgb}{0.000000,0.000000,0.000000}%
\pgfsetstrokecolor{currentstroke}%
\pgfsetdash{}{0pt}%
\pgfpathmoveto{\pgfqpoint{3.957790in}{1.205194in}}%
\pgfpathlineto{\pgfqpoint{3.972070in}{1.207387in}}%
\pgfpathlineto{\pgfqpoint{3.986361in}{1.209754in}}%
\pgfpathlineto{\pgfqpoint{4.000661in}{1.212297in}}%
\pgfpathlineto{\pgfqpoint{4.014972in}{1.215013in}}%
\pgfpathlineto{\pgfqpoint{4.023266in}{1.229063in}}%
\pgfpathlineto{\pgfqpoint{4.031556in}{1.243285in}}%
\pgfpathlineto{\pgfqpoint{4.039840in}{1.257672in}}%
\pgfpathlineto{\pgfqpoint{4.048120in}{1.272219in}}%
\pgfpathlineto{\pgfqpoint{4.033813in}{1.268762in}}%
\pgfpathlineto{\pgfqpoint{4.019517in}{1.265481in}}%
\pgfpathlineto{\pgfqpoint{4.005232in}{1.262376in}}%
\pgfpathlineto{\pgfqpoint{3.990958in}{1.259446in}}%
\pgfpathlineto{\pgfqpoint{3.982674in}{1.245626in}}%
\pgfpathlineto{\pgfqpoint{3.974384in}{1.231973in}}%
\pgfpathlineto{\pgfqpoint{3.966090in}{1.218494in}}%
\pgfpathlineto{\pgfqpoint{3.957790in}{1.205194in}}%
\pgfpathclose%
\pgfusepath{fill}%
\end{pgfscope}%
\begin{pgfscope}%
\pgfpathrectangle{\pgfqpoint{1.150000in}{0.150000in}}{\pgfqpoint{5.700000in}{5.700000in}}%
\pgfusepath{clip}%
\pgfsetbuttcap%
\pgfsetroundjoin%
\definecolor{currentfill}{rgb}{0.130067,0.651384,0.521608}%
\pgfsetfillcolor{currentfill}%
\pgfsetfillopacity{0.800000}%
\pgfsetlinewidth{0.000000pt}%
\definecolor{currentstroke}{rgb}{0.000000,0.000000,0.000000}%
\pgfsetstrokecolor{currentstroke}%
\pgfsetdash{}{0pt}%
\pgfpathmoveto{\pgfqpoint{5.010765in}{2.714384in}}%
\pgfpathlineto{\pgfqpoint{5.025669in}{2.730951in}}%
\pgfpathlineto{\pgfqpoint{5.040595in}{2.747708in}}%
\pgfpathlineto{\pgfqpoint{5.055543in}{2.764655in}}%
\pgfpathlineto{\pgfqpoint{5.070513in}{2.781793in}}%
\pgfpathlineto{\pgfqpoint{5.078529in}{2.796306in}}%
\pgfpathlineto{\pgfqpoint{5.086537in}{2.810618in}}%
\pgfpathlineto{\pgfqpoint{5.094537in}{2.824728in}}%
\pgfpathlineto{\pgfqpoint{5.102530in}{2.838634in}}%
\pgfpathlineto{\pgfqpoint{5.087553in}{2.821334in}}%
\pgfpathlineto{\pgfqpoint{5.072599in}{2.804224in}}%
\pgfpathlineto{\pgfqpoint{5.057667in}{2.787305in}}%
\pgfpathlineto{\pgfqpoint{5.042756in}{2.770577in}}%
\pgfpathlineto{\pgfqpoint{5.034770in}{2.756819in}}%
\pgfpathlineto{\pgfqpoint{5.026776in}{2.742867in}}%
\pgfpathlineto{\pgfqpoint{5.018774in}{2.728722in}}%
\pgfpathlineto{\pgfqpoint{5.010765in}{2.714384in}}%
\pgfpathclose%
\pgfusepath{fill}%
\end{pgfscope}%
\begin{pgfscope}%
\pgfpathrectangle{\pgfqpoint{1.150000in}{0.150000in}}{\pgfqpoint{5.700000in}{5.700000in}}%
\pgfusepath{clip}%
\pgfsetbuttcap%
\pgfsetroundjoin%
\definecolor{currentfill}{rgb}{0.147607,0.511733,0.557049}%
\pgfsetfillcolor{currentfill}%
\pgfsetfillopacity{0.800000}%
\pgfsetlinewidth{0.000000pt}%
\definecolor{currentstroke}{rgb}{0.000000,0.000000,0.000000}%
\pgfsetstrokecolor{currentstroke}%
\pgfsetdash{}{0pt}%
\pgfpathmoveto{\pgfqpoint{4.731019in}{2.274922in}}%
\pgfpathlineto{\pgfqpoint{4.745716in}{2.288652in}}%
\pgfpathlineto{\pgfqpoint{4.760432in}{2.302567in}}%
\pgfpathlineto{\pgfqpoint{4.775167in}{2.316669in}}%
\pgfpathlineto{\pgfqpoint{4.789922in}{2.330956in}}%
\pgfpathlineto{\pgfqpoint{4.798051in}{2.348391in}}%
\pgfpathlineto{\pgfqpoint{4.806173in}{2.365686in}}%
\pgfpathlineto{\pgfqpoint{4.814291in}{2.382837in}}%
\pgfpathlineto{\pgfqpoint{4.822403in}{2.399842in}}%
\pgfpathlineto{\pgfqpoint{4.807637in}{2.385218in}}%
\pgfpathlineto{\pgfqpoint{4.792892in}{2.370781in}}%
\pgfpathlineto{\pgfqpoint{4.778166in}{2.356530in}}%
\pgfpathlineto{\pgfqpoint{4.763459in}{2.342465in}}%
\pgfpathlineto{\pgfqpoint{4.755357in}{2.325783in}}%
\pgfpathlineto{\pgfqpoint{4.747249in}{2.308963in}}%
\pgfpathlineto{\pgfqpoint{4.739137in}{2.292008in}}%
\pgfpathlineto{\pgfqpoint{4.731019in}{2.274922in}}%
\pgfpathclose%
\pgfusepath{fill}%
\end{pgfscope}%
\begin{pgfscope}%
\pgfpathrectangle{\pgfqpoint{1.150000in}{0.150000in}}{\pgfqpoint{5.700000in}{5.700000in}}%
\pgfusepath{clip}%
\pgfsetbuttcap%
\pgfsetroundjoin%
\definecolor{currentfill}{rgb}{0.273006,0.204520,0.501721}%
\pgfsetfillcolor{currentfill}%
\pgfsetfillopacity{0.800000}%
\pgfsetlinewidth{0.000000pt}%
\definecolor{currentstroke}{rgb}{0.000000,0.000000,0.000000}%
\pgfsetstrokecolor{currentstroke}%
\pgfsetdash{}{0pt}%
\pgfpathmoveto{\pgfqpoint{4.171534in}{1.414602in}}%
\pgfpathlineto{\pgfqpoint{4.185899in}{1.420325in}}%
\pgfpathlineto{\pgfqpoint{4.200276in}{1.426225in}}%
\pgfpathlineto{\pgfqpoint{4.214666in}{1.432301in}}%
\pgfpathlineto{\pgfqpoint{4.229070in}{1.438553in}}%
\pgfpathlineto{\pgfqpoint{4.237315in}{1.455543in}}%
\pgfpathlineto{\pgfqpoint{4.245556in}{1.472610in}}%
\pgfpathlineto{\pgfqpoint{4.253793in}{1.489748in}}%
\pgfpathlineto{\pgfqpoint{4.262028in}{1.506950in}}%
\pgfpathlineto{\pgfqpoint{4.247621in}{1.500046in}}%
\pgfpathlineto{\pgfqpoint{4.233227in}{1.493319in}}%
\pgfpathlineto{\pgfqpoint{4.218847in}{1.486769in}}%
\pgfpathlineto{\pgfqpoint{4.204480in}{1.480396in}}%
\pgfpathlineto{\pgfqpoint{4.196249in}{1.463833in}}%
\pgfpathlineto{\pgfqpoint{4.188014in}{1.447342in}}%
\pgfpathlineto{\pgfqpoint{4.179776in}{1.430930in}}%
\pgfpathlineto{\pgfqpoint{4.171534in}{1.414602in}}%
\pgfpathclose%
\pgfusepath{fill}%
\end{pgfscope}%
\begin{pgfscope}%
\pgfpathrectangle{\pgfqpoint{1.150000in}{0.150000in}}{\pgfqpoint{5.700000in}{5.700000in}}%
\pgfusepath{clip}%
\pgfsetbuttcap%
\pgfsetroundjoin%
\definecolor{currentfill}{rgb}{0.250425,0.274290,0.533103}%
\pgfsetfillcolor{currentfill}%
\pgfsetfillopacity{0.800000}%
\pgfsetlinewidth{0.000000pt}%
\definecolor{currentstroke}{rgb}{0.000000,0.000000,0.000000}%
\pgfsetstrokecolor{currentstroke}%
\pgfsetdash{}{0pt}%
\pgfpathmoveto{\pgfqpoint{4.294931in}{1.576286in}}%
\pgfpathlineto{\pgfqpoint{4.309357in}{1.583989in}}%
\pgfpathlineto{\pgfqpoint{4.323797in}{1.591869in}}%
\pgfpathlineto{\pgfqpoint{4.338252in}{1.599928in}}%
\pgfpathlineto{\pgfqpoint{4.352722in}{1.608164in}}%
\pgfpathlineto{\pgfqpoint{4.360945in}{1.626204in}}%
\pgfpathlineto{\pgfqpoint{4.369166in}{1.644265in}}%
\pgfpathlineto{\pgfqpoint{4.377383in}{1.662343in}}%
\pgfpathlineto{\pgfqpoint{4.385597in}{1.680432in}}%
\pgfpathlineto{\pgfqpoint{4.371120in}{1.671603in}}%
\pgfpathlineto{\pgfqpoint{4.356659in}{1.662953in}}%
\pgfpathlineto{\pgfqpoint{4.342212in}{1.654482in}}%
\pgfpathlineto{\pgfqpoint{4.327780in}{1.646189in}}%
\pgfpathlineto{\pgfqpoint{4.319573in}{1.628680in}}%
\pgfpathlineto{\pgfqpoint{4.311362in}{1.611189in}}%
\pgfpathlineto{\pgfqpoint{4.303148in}{1.593722in}}%
\pgfpathlineto{\pgfqpoint{4.294931in}{1.576286in}}%
\pgfpathclose%
\pgfusepath{fill}%
\end{pgfscope}%
\begin{pgfscope}%
\pgfpathrectangle{\pgfqpoint{1.150000in}{0.150000in}}{\pgfqpoint{5.700000in}{5.700000in}}%
\pgfusepath{clip}%
\pgfsetbuttcap%
\pgfsetroundjoin%
\definecolor{currentfill}{rgb}{0.386433,0.794644,0.372886}%
\pgfsetfillcolor{currentfill}%
\pgfsetfillopacity{0.800000}%
\pgfsetlinewidth{0.000000pt}%
\definecolor{currentstroke}{rgb}{0.000000,0.000000,0.000000}%
\pgfsetstrokecolor{currentstroke}%
\pgfsetdash{}{0pt}%
\pgfpathmoveto{\pgfqpoint{5.381250in}{3.220609in}}%
\pgfpathlineto{\pgfqpoint{5.396450in}{3.239910in}}%
\pgfpathlineto{\pgfqpoint{5.411674in}{3.259406in}}%
\pgfpathlineto{\pgfqpoint{5.426924in}{3.279097in}}%
\pgfpathlineto{\pgfqpoint{5.442198in}{3.298983in}}%
\pgfpathlineto{\pgfqpoint{5.449992in}{3.308393in}}%
\pgfpathlineto{\pgfqpoint{5.457774in}{3.317570in}}%
\pgfpathlineto{\pgfqpoint{5.465545in}{3.326516in}}%
\pgfpathlineto{\pgfqpoint{5.473304in}{3.335231in}}%
\pgfpathlineto{\pgfqpoint{5.458032in}{3.315404in}}%
\pgfpathlineto{\pgfqpoint{5.442785in}{3.295772in}}%
\pgfpathlineto{\pgfqpoint{5.427564in}{3.276334in}}%
\pgfpathlineto{\pgfqpoint{5.412366in}{3.257091in}}%
\pgfpathlineto{\pgfqpoint{5.404604in}{3.248303in}}%
\pgfpathlineto{\pgfqpoint{5.396830in}{3.239294in}}%
\pgfpathlineto{\pgfqpoint{5.389045in}{3.230063in}}%
\pgfpathlineto{\pgfqpoint{5.381250in}{3.220609in}}%
\pgfpathclose%
\pgfusepath{fill}%
\end{pgfscope}%
\begin{pgfscope}%
\pgfpathrectangle{\pgfqpoint{1.150000in}{0.150000in}}{\pgfqpoint{5.700000in}{5.700000in}}%
\pgfusepath{clip}%
\pgfsetbuttcap%
\pgfsetroundjoin%
\definecolor{currentfill}{rgb}{0.185783,0.704891,0.485273}%
\pgfsetfillcolor{currentfill}%
\pgfsetfillopacity{0.800000}%
\pgfsetlinewidth{0.000000pt}%
\definecolor{currentstroke}{rgb}{0.000000,0.000000,0.000000}%
\pgfsetstrokecolor{currentstroke}%
\pgfsetdash{}{0pt}%
\pgfpathmoveto{\pgfqpoint{5.134414in}{2.892207in}}%
\pgfpathlineto{\pgfqpoint{5.149419in}{2.909826in}}%
\pgfpathlineto{\pgfqpoint{5.164446in}{2.927636in}}%
\pgfpathlineto{\pgfqpoint{5.179496in}{2.945638in}}%
\pgfpathlineto{\pgfqpoint{5.194569in}{2.963832in}}%
\pgfpathlineto{\pgfqpoint{5.202524in}{2.976812in}}%
\pgfpathlineto{\pgfqpoint{5.210469in}{2.989575in}}%
\pgfpathlineto{\pgfqpoint{5.218405in}{3.002120in}}%
\pgfpathlineto{\pgfqpoint{5.226332in}{3.014447in}}%
\pgfpathlineto{\pgfqpoint{5.211254in}{2.996162in}}%
\pgfpathlineto{\pgfqpoint{5.196200in}{2.978070in}}%
\pgfpathlineto{\pgfqpoint{5.181169in}{2.960170in}}%
\pgfpathlineto{\pgfqpoint{5.166161in}{2.942462in}}%
\pgfpathlineto{\pgfqpoint{5.158237in}{2.930211in}}%
\pgfpathlineto{\pgfqpoint{5.150305in}{2.917752in}}%
\pgfpathlineto{\pgfqpoint{5.142364in}{2.905084in}}%
\pgfpathlineto{\pgfqpoint{5.134414in}{2.892207in}}%
\pgfpathclose%
\pgfusepath{fill}%
\end{pgfscope}%
\begin{pgfscope}%
\pgfpathrectangle{\pgfqpoint{1.150000in}{0.150000in}}{\pgfqpoint{5.700000in}{5.700000in}}%
\pgfusepath{clip}%
\pgfsetbuttcap%
\pgfsetroundjoin%
\definecolor{currentfill}{rgb}{0.180629,0.429975,0.557282}%
\pgfsetfillcolor{currentfill}%
\pgfsetfillopacity{0.800000}%
\pgfsetlinewidth{0.000000pt}%
\definecolor{currentstroke}{rgb}{0.000000,0.000000,0.000000}%
\pgfsetstrokecolor{currentstroke}%
\pgfsetdash{}{0pt}%
\pgfpathmoveto{\pgfqpoint{4.574755in}{2.012526in}}%
\pgfpathlineto{\pgfqpoint{4.589348in}{2.024314in}}%
\pgfpathlineto{\pgfqpoint{4.603959in}{2.036286in}}%
\pgfpathlineto{\pgfqpoint{4.618588in}{2.048440in}}%
\pgfpathlineto{\pgfqpoint{4.633235in}{2.060778in}}%
\pgfpathlineto{\pgfqpoint{4.641408in}{2.079187in}}%
\pgfpathlineto{\pgfqpoint{4.649577in}{2.097507in}}%
\pgfpathlineto{\pgfqpoint{4.657741in}{2.115733in}}%
\pgfpathlineto{\pgfqpoint{4.665901in}{2.133860in}}%
\pgfpathlineto{\pgfqpoint{4.651244in}{2.121087in}}%
\pgfpathlineto{\pgfqpoint{4.636604in}{2.108497in}}%
\pgfpathlineto{\pgfqpoint{4.621983in}{2.096091in}}%
\pgfpathlineto{\pgfqpoint{4.607379in}{2.083868in}}%
\pgfpathlineto{\pgfqpoint{4.599229in}{2.066163in}}%
\pgfpathlineto{\pgfqpoint{4.591075in}{2.048368in}}%
\pgfpathlineto{\pgfqpoint{4.582917in}{2.030488in}}%
\pgfpathlineto{\pgfqpoint{4.574755in}{2.012526in}}%
\pgfpathclose%
\pgfusepath{fill}%
\end{pgfscope}%
\begin{pgfscope}%
\pgfpathrectangle{\pgfqpoint{1.150000in}{0.150000in}}{\pgfqpoint{5.700000in}{5.700000in}}%
\pgfusepath{clip}%
\pgfsetbuttcap%
\pgfsetroundjoin%
\definecolor{currentfill}{rgb}{0.281477,0.755203,0.432552}%
\pgfsetfillcolor{currentfill}%
\pgfsetfillopacity{0.800000}%
\pgfsetlinewidth{0.000000pt}%
\definecolor{currentstroke}{rgb}{0.000000,0.000000,0.000000}%
\pgfsetstrokecolor{currentstroke}%
\pgfsetdash{}{0pt}%
\pgfpathmoveto{\pgfqpoint{5.257944in}{3.061569in}}%
\pgfpathlineto{\pgfqpoint{5.273047in}{3.080100in}}%
\pgfpathlineto{\pgfqpoint{5.288174in}{3.098824in}}%
\pgfpathlineto{\pgfqpoint{5.303325in}{3.117742in}}%
\pgfpathlineto{\pgfqpoint{5.318501in}{3.136854in}}%
\pgfpathlineto{\pgfqpoint{5.326381in}{3.148118in}}%
\pgfpathlineto{\pgfqpoint{5.334251in}{3.159154in}}%
\pgfpathlineto{\pgfqpoint{5.342111in}{3.169963in}}%
\pgfpathlineto{\pgfqpoint{5.349960in}{3.180544in}}%
\pgfpathlineto{\pgfqpoint{5.334783in}{3.161416in}}%
\pgfpathlineto{\pgfqpoint{5.319631in}{3.142482in}}%
\pgfpathlineto{\pgfqpoint{5.304503in}{3.123741in}}%
\pgfpathlineto{\pgfqpoint{5.289399in}{3.105193in}}%
\pgfpathlineto{\pgfqpoint{5.281550in}{3.094614in}}%
\pgfpathlineto{\pgfqpoint{5.273691in}{3.083818in}}%
\pgfpathlineto{\pgfqpoint{5.265822in}{3.072803in}}%
\pgfpathlineto{\pgfqpoint{5.257944in}{3.061569in}}%
\pgfpathclose%
\pgfusepath{fill}%
\end{pgfscope}%
\begin{pgfscope}%
\pgfpathrectangle{\pgfqpoint{1.150000in}{0.150000in}}{\pgfqpoint{5.700000in}{5.700000in}}%
\pgfusepath{clip}%
\pgfsetbuttcap%
\pgfsetroundjoin%
\definecolor{currentfill}{rgb}{0.282290,0.145912,0.461510}%
\pgfsetfillcolor{currentfill}%
\pgfsetfillopacity{0.800000}%
\pgfsetlinewidth{0.000000pt}%
\definecolor{currentstroke}{rgb}{0.000000,0.000000,0.000000}%
\pgfsetstrokecolor{currentstroke}%
\pgfsetdash{}{0pt}%
\pgfpathmoveto{\pgfqpoint{4.048120in}{1.272219in}}%
\pgfpathlineto{\pgfqpoint{4.062438in}{1.275850in}}%
\pgfpathlineto{\pgfqpoint{4.076767in}{1.279656in}}%
\pgfpathlineto{\pgfqpoint{4.091107in}{1.283637in}}%
\pgfpathlineto{\pgfqpoint{4.105459in}{1.287793in}}%
\pgfpathlineto{\pgfqpoint{4.113733in}{1.303212in}}%
\pgfpathlineto{\pgfqpoint{4.122002in}{1.318768in}}%
\pgfpathlineto{\pgfqpoint{4.130267in}{1.334455in}}%
\pgfpathlineto{\pgfqpoint{4.138528in}{1.350264in}}%
\pgfpathlineto{\pgfqpoint{4.124177in}{1.345397in}}%
\pgfpathlineto{\pgfqpoint{4.109838in}{1.340705in}}%
\pgfpathlineto{\pgfqpoint{4.095510in}{1.336189in}}%
\pgfpathlineto{\pgfqpoint{4.081194in}{1.331849in}}%
\pgfpathlineto{\pgfqpoint{4.072932in}{1.316738in}}%
\pgfpathlineto{\pgfqpoint{4.064666in}{1.301758in}}%
\pgfpathlineto{\pgfqpoint{4.056395in}{1.286916in}}%
\pgfpathlineto{\pgfqpoint{4.048120in}{1.272219in}}%
\pgfpathclose%
\pgfusepath{fill}%
\end{pgfscope}%
\begin{pgfscope}%
\pgfpathrectangle{\pgfqpoint{1.150000in}{0.150000in}}{\pgfqpoint{5.700000in}{5.700000in}}%
\pgfusepath{clip}%
\pgfsetbuttcap%
\pgfsetroundjoin%
\definecolor{currentfill}{rgb}{0.220057,0.343307,0.549413}%
\pgfsetfillcolor{currentfill}%
\pgfsetfillopacity{0.800000}%
\pgfsetlinewidth{0.000000pt}%
\definecolor{currentstroke}{rgb}{0.000000,0.000000,0.000000}%
\pgfsetstrokecolor{currentstroke}%
\pgfsetdash{}{0pt}%
\pgfpathmoveto{\pgfqpoint{4.418419in}{1.752790in}}%
\pgfpathlineto{\pgfqpoint{4.432919in}{1.762359in}}%
\pgfpathlineto{\pgfqpoint{4.447434in}{1.772108in}}%
\pgfpathlineto{\pgfqpoint{4.461966in}{1.782037in}}%
\pgfpathlineto{\pgfqpoint{4.476513in}{1.792145in}}%
\pgfpathlineto{\pgfqpoint{4.484720in}{1.810752in}}%
\pgfpathlineto{\pgfqpoint{4.492922in}{1.829331in}}%
\pgfpathlineto{\pgfqpoint{4.501122in}{1.847876in}}%
\pgfpathlineto{\pgfqpoint{4.509318in}{1.866384in}}%
\pgfpathlineto{\pgfqpoint{4.494761in}{1.855743in}}%
\pgfpathlineto{\pgfqpoint{4.480220in}{1.845284in}}%
\pgfpathlineto{\pgfqpoint{4.465696in}{1.835005in}}%
\pgfpathlineto{\pgfqpoint{4.451188in}{1.824906in}}%
\pgfpathlineto{\pgfqpoint{4.443001in}{1.806917in}}%
\pgfpathlineto{\pgfqpoint{4.434811in}{1.788898in}}%
\pgfpathlineto{\pgfqpoint{4.426617in}{1.770854in}}%
\pgfpathlineto{\pgfqpoint{4.418419in}{1.752790in}}%
\pgfpathclose%
\pgfusepath{fill}%
\end{pgfscope}%
\begin{pgfscope}%
\pgfpathrectangle{\pgfqpoint{1.150000in}{0.150000in}}{\pgfqpoint{5.700000in}{5.700000in}}%
\pgfusepath{clip}%
\pgfsetbuttcap%
\pgfsetroundjoin%
\definecolor{currentfill}{rgb}{0.124395,0.578002,0.548287}%
\pgfsetfillcolor{currentfill}%
\pgfsetfillopacity{0.800000}%
\pgfsetlinewidth{0.000000pt}%
\definecolor{currentstroke}{rgb}{0.000000,0.000000,0.000000}%
\pgfsetstrokecolor{currentstroke}%
\pgfsetdash{}{0pt}%
\pgfpathmoveto{\pgfqpoint{4.854793in}{2.466344in}}%
\pgfpathlineto{\pgfqpoint{4.869589in}{2.481457in}}%
\pgfpathlineto{\pgfqpoint{4.884404in}{2.496759in}}%
\pgfpathlineto{\pgfqpoint{4.899241in}{2.512248in}}%
\pgfpathlineto{\pgfqpoint{4.914097in}{2.527925in}}%
\pgfpathlineto{\pgfqpoint{4.922190in}{2.544439in}}%
\pgfpathlineto{\pgfqpoint{4.930276in}{2.560783in}}%
\pgfpathlineto{\pgfqpoint{4.938356in}{2.576954in}}%
\pgfpathlineto{\pgfqpoint{4.946429in}{2.592950in}}%
\pgfpathlineto{\pgfqpoint{4.931563in}{2.577004in}}%
\pgfpathlineto{\pgfqpoint{4.916717in}{2.561247in}}%
\pgfpathlineto{\pgfqpoint{4.901892in}{2.545678in}}%
\pgfpathlineto{\pgfqpoint{4.887088in}{2.530297in}}%
\pgfpathlineto{\pgfqpoint{4.879024in}{2.514556in}}%
\pgfpathlineto{\pgfqpoint{4.870953in}{2.498648in}}%
\pgfpathlineto{\pgfqpoint{4.862876in}{2.482577in}}%
\pgfpathlineto{\pgfqpoint{4.854793in}{2.466344in}}%
\pgfpathclose%
\pgfusepath{fill}%
\end{pgfscope}%
\begin{pgfscope}%
\pgfpathrectangle{\pgfqpoint{1.150000in}{0.150000in}}{\pgfqpoint{5.700000in}{5.700000in}}%
\pgfusepath{clip}%
\pgfsetbuttcap%
\pgfsetroundjoin%
\definecolor{currentfill}{rgb}{0.595839,0.848717,0.243329}%
\pgfsetfillcolor{currentfill}%
\pgfsetfillopacity{0.800000}%
\pgfsetlinewidth{0.000000pt}%
\definecolor{currentstroke}{rgb}{0.000000,0.000000,0.000000}%
\pgfsetstrokecolor{currentstroke}%
\pgfsetdash{}{0pt}%
\pgfpathmoveto{\pgfqpoint{5.596250in}{3.477171in}}%
\pgfpathlineto{\pgfqpoint{5.611638in}{3.497746in}}%
\pgfpathlineto{\pgfqpoint{5.627052in}{3.518517in}}%
\pgfpathlineto{\pgfqpoint{5.642492in}{3.539485in}}%
\pgfpathlineto{\pgfqpoint{5.650131in}{3.545917in}}%
\pgfpathlineto{\pgfqpoint{5.657756in}{3.552118in}}%
\pgfpathlineto{\pgfqpoint{5.665369in}{3.558091in}}%
\pgfpathlineto{\pgfqpoint{5.672969in}{3.563838in}}%
\pgfpathlineto{\pgfqpoint{5.657538in}{3.543043in}}%
\pgfpathlineto{\pgfqpoint{5.642133in}{3.522445in}}%
\pgfpathlineto{\pgfqpoint{5.626755in}{3.502043in}}%
\pgfpathlineto{\pgfqpoint{5.619147in}{3.496156in}}%
\pgfpathlineto{\pgfqpoint{5.611527in}{3.490049in}}%
\pgfpathlineto{\pgfqpoint{5.603895in}{3.483722in}}%
\pgfpathlineto{\pgfqpoint{5.596250in}{3.477171in}}%
\pgfpathclose%
\pgfusepath{fill}%
\end{pgfscope}%
\begin{pgfscope}%
\pgfpathrectangle{\pgfqpoint{1.150000in}{0.150000in}}{\pgfqpoint{5.700000in}{5.700000in}}%
\pgfusepath{clip}%
\pgfsetbuttcap%
\pgfsetroundjoin%
\definecolor{currentfill}{rgb}{0.153364,0.497000,0.557724}%
\pgfsetfillcolor{currentfill}%
\pgfsetfillopacity{0.800000}%
\pgfsetlinewidth{0.000000pt}%
\definecolor{currentstroke}{rgb}{0.000000,0.000000,0.000000}%
\pgfsetstrokecolor{currentstroke}%
\pgfsetdash{}{0pt}%
\pgfpathmoveto{\pgfqpoint{4.698498in}{2.205318in}}%
\pgfpathlineto{\pgfqpoint{4.713184in}{2.218679in}}%
\pgfpathlineto{\pgfqpoint{4.727890in}{2.232224in}}%
\pgfpathlineto{\pgfqpoint{4.742614in}{2.245955in}}%
\pgfpathlineto{\pgfqpoint{4.757358in}{2.259871in}}%
\pgfpathlineto{\pgfqpoint{4.765507in}{2.277838in}}%
\pgfpathlineto{\pgfqpoint{4.773650in}{2.295676in}}%
\pgfpathlineto{\pgfqpoint{4.781789in}{2.313383in}}%
\pgfpathlineto{\pgfqpoint{4.789922in}{2.330956in}}%
\pgfpathlineto{\pgfqpoint{4.775167in}{2.316669in}}%
\pgfpathlineto{\pgfqpoint{4.760432in}{2.302567in}}%
\pgfpathlineto{\pgfqpoint{4.745716in}{2.288652in}}%
\pgfpathlineto{\pgfqpoint{4.731019in}{2.274922in}}%
\pgfpathlineto{\pgfqpoint{4.722896in}{2.257706in}}%
\pgfpathlineto{\pgfqpoint{4.714768in}{2.240365in}}%
\pgfpathlineto{\pgfqpoint{4.706635in}{2.222901in}}%
\pgfpathlineto{\pgfqpoint{4.698498in}{2.205318in}}%
\pgfpathclose%
\pgfusepath{fill}%
\end{pgfscope}%
\begin{pgfscope}%
\pgfpathrectangle{\pgfqpoint{1.150000in}{0.150000in}}{\pgfqpoint{5.700000in}{5.700000in}}%
\pgfusepath{clip}%
\pgfsetbuttcap%
\pgfsetroundjoin%
\definecolor{currentfill}{rgb}{0.279566,0.067836,0.391917}%
\pgfsetfillcolor{currentfill}%
\pgfsetfillopacity{0.800000}%
\pgfsetlinewidth{0.000000pt}%
\definecolor{currentstroke}{rgb}{0.000000,0.000000,0.000000}%
\pgfsetstrokecolor{currentstroke}%
\pgfsetdash{}{0pt}%
\pgfpathmoveto{\pgfqpoint{3.834084in}{1.105430in}}%
\pgfpathlineto{\pgfqpoint{3.848347in}{1.105357in}}%
\pgfpathlineto{\pgfqpoint{3.862618in}{1.105459in}}%
\pgfpathlineto{\pgfqpoint{3.876898in}{1.105734in}}%
\pgfpathlineto{\pgfqpoint{3.891186in}{1.106184in}}%
\pgfpathlineto{\pgfqpoint{3.899533in}{1.117772in}}%
\pgfpathlineto{\pgfqpoint{3.907873in}{1.129600in}}%
\pgfpathlineto{\pgfqpoint{3.916208in}{1.141661in}}%
\pgfpathlineto{\pgfqpoint{3.924536in}{1.153948in}}%
\pgfpathlineto{\pgfqpoint{3.910257in}{1.152698in}}%
\pgfpathlineto{\pgfqpoint{3.895988in}{1.151623in}}%
\pgfpathlineto{\pgfqpoint{3.881728in}{1.150723in}}%
\pgfpathlineto{\pgfqpoint{3.867478in}{1.149998in}}%
\pgfpathlineto{\pgfqpoint{3.859139in}{1.138498in}}%
\pgfpathlineto{\pgfqpoint{3.850795in}{1.127232in}}%
\pgfpathlineto{\pgfqpoint{3.842443in}{1.116207in}}%
\pgfpathlineto{\pgfqpoint{3.834084in}{1.105430in}}%
\pgfpathclose%
\pgfusepath{fill}%
\end{pgfscope}%
\begin{pgfscope}%
\pgfpathrectangle{\pgfqpoint{1.150000in}{0.150000in}}{\pgfqpoint{5.700000in}{5.700000in}}%
\pgfusepath{clip}%
\pgfsetbuttcap%
\pgfsetroundjoin%
\definecolor{currentfill}{rgb}{0.277134,0.185228,0.489898}%
\pgfsetfillcolor{currentfill}%
\pgfsetfillopacity{0.800000}%
\pgfsetlinewidth{0.000000pt}%
\definecolor{currentstroke}{rgb}{0.000000,0.000000,0.000000}%
\pgfsetstrokecolor{currentstroke}%
\pgfsetdash{}{0pt}%
\pgfpathmoveto{\pgfqpoint{4.138528in}{1.350264in}}%
\pgfpathlineto{\pgfqpoint{4.152892in}{1.355306in}}%
\pgfpathlineto{\pgfqpoint{4.167267in}{1.360525in}}%
\pgfpathlineto{\pgfqpoint{4.181656in}{1.365918in}}%
\pgfpathlineto{\pgfqpoint{4.196057in}{1.371487in}}%
\pgfpathlineto{\pgfqpoint{4.204315in}{1.388107in}}%
\pgfpathlineto{\pgfqpoint{4.212570in}{1.404829in}}%
\pgfpathlineto{\pgfqpoint{4.220822in}{1.421646in}}%
\pgfpathlineto{\pgfqpoint{4.229070in}{1.438553in}}%
\pgfpathlineto{\pgfqpoint{4.214666in}{1.432301in}}%
\pgfpathlineto{\pgfqpoint{4.200276in}{1.426225in}}%
\pgfpathlineto{\pgfqpoint{4.185899in}{1.420325in}}%
\pgfpathlineto{\pgfqpoint{4.171534in}{1.414602in}}%
\pgfpathlineto{\pgfqpoint{4.163288in}{1.398365in}}%
\pgfpathlineto{\pgfqpoint{4.155039in}{1.382226in}}%
\pgfpathlineto{\pgfqpoint{4.146785in}{1.366190in}}%
\pgfpathlineto{\pgfqpoint{4.138528in}{1.350264in}}%
\pgfpathclose%
\pgfusepath{fill}%
\end{pgfscope}%
\begin{pgfscope}%
\pgfpathrectangle{\pgfqpoint{1.150000in}{0.150000in}}{\pgfqpoint{5.700000in}{5.700000in}}%
\pgfusepath{clip}%
\pgfsetbuttcap%
\pgfsetroundjoin%
\definecolor{currentfill}{rgb}{0.258965,0.251537,0.524736}%
\pgfsetfillcolor{currentfill}%
\pgfsetfillopacity{0.800000}%
\pgfsetlinewidth{0.000000pt}%
\definecolor{currentstroke}{rgb}{0.000000,0.000000,0.000000}%
\pgfsetstrokecolor{currentstroke}%
\pgfsetdash{}{0pt}%
\pgfpathmoveto{\pgfqpoint{4.262028in}{1.506950in}}%
\pgfpathlineto{\pgfqpoint{4.276449in}{1.514031in}}%
\pgfpathlineto{\pgfqpoint{4.290883in}{1.521290in}}%
\pgfpathlineto{\pgfqpoint{4.305332in}{1.528725in}}%
\pgfpathlineto{\pgfqpoint{4.319795in}{1.536337in}}%
\pgfpathlineto{\pgfqpoint{4.328032in}{1.554233in}}%
\pgfpathlineto{\pgfqpoint{4.336265in}{1.572173in}}%
\pgfpathlineto{\pgfqpoint{4.344495in}{1.590152in}}%
\pgfpathlineto{\pgfqpoint{4.352722in}{1.608164in}}%
\pgfpathlineto{\pgfqpoint{4.338252in}{1.599928in}}%
\pgfpathlineto{\pgfqpoint{4.323797in}{1.591869in}}%
\pgfpathlineto{\pgfqpoint{4.309357in}{1.583989in}}%
\pgfpathlineto{\pgfqpoint{4.294931in}{1.576286in}}%
\pgfpathlineto{\pgfqpoint{4.286710in}{1.558885in}}%
\pgfpathlineto{\pgfqpoint{4.278486in}{1.541524in}}%
\pgfpathlineto{\pgfqpoint{4.270259in}{1.524211in}}%
\pgfpathlineto{\pgfqpoint{4.262028in}{1.506950in}}%
\pgfpathclose%
\pgfusepath{fill}%
\end{pgfscope}%
\begin{pgfscope}%
\pgfpathrectangle{\pgfqpoint{1.150000in}{0.150000in}}{\pgfqpoint{5.700000in}{5.700000in}}%
\pgfusepath{clip}%
\pgfsetbuttcap%
\pgfsetroundjoin%
\definecolor{currentfill}{rgb}{0.124780,0.640461,0.527068}%
\pgfsetfillcolor{currentfill}%
\pgfsetfillopacity{0.800000}%
\pgfsetlinewidth{0.000000pt}%
\definecolor{currentstroke}{rgb}{0.000000,0.000000,0.000000}%
\pgfsetstrokecolor{currentstroke}%
\pgfsetdash{}{0pt}%
\pgfpathmoveto{\pgfqpoint{4.978654in}{2.655143in}}%
\pgfpathlineto{\pgfqpoint{4.993551in}{2.671512in}}%
\pgfpathlineto{\pgfqpoint{5.008469in}{2.688071in}}%
\pgfpathlineto{\pgfqpoint{5.023408in}{2.704819in}}%
\pgfpathlineto{\pgfqpoint{5.038370in}{2.721758in}}%
\pgfpathlineto{\pgfqpoint{5.046417in}{2.737061in}}%
\pgfpathlineto{\pgfqpoint{5.054457in}{2.752169in}}%
\pgfpathlineto{\pgfqpoint{5.062488in}{2.767080in}}%
\pgfpathlineto{\pgfqpoint{5.070513in}{2.781793in}}%
\pgfpathlineto{\pgfqpoint{5.055543in}{2.764655in}}%
\pgfpathlineto{\pgfqpoint{5.040595in}{2.747708in}}%
\pgfpathlineto{\pgfqpoint{5.025669in}{2.730951in}}%
\pgfpathlineto{\pgfqpoint{5.010765in}{2.714384in}}%
\pgfpathlineto{\pgfqpoint{5.002748in}{2.699857in}}%
\pgfpathlineto{\pgfqpoint{4.994724in}{2.685139in}}%
\pgfpathlineto{\pgfqpoint{4.986693in}{2.670235in}}%
\pgfpathlineto{\pgfqpoint{4.978654in}{2.655143in}}%
\pgfpathclose%
\pgfusepath{fill}%
\end{pgfscope}%
\begin{pgfscope}%
\pgfpathrectangle{\pgfqpoint{1.150000in}{0.150000in}}{\pgfqpoint{5.700000in}{5.700000in}}%
\pgfusepath{clip}%
\pgfsetbuttcap%
\pgfsetroundjoin%
\definecolor{currentfill}{rgb}{0.282327,0.094955,0.417331}%
\pgfsetfillcolor{currentfill}%
\pgfsetfillopacity{0.800000}%
\pgfsetlinewidth{0.000000pt}%
\definecolor{currentstroke}{rgb}{0.000000,0.000000,0.000000}%
\pgfsetstrokecolor{currentstroke}%
\pgfsetdash{}{0pt}%
\pgfpathmoveto{\pgfqpoint{3.924536in}{1.153948in}}%
\pgfpathlineto{\pgfqpoint{3.938823in}{1.155372in}}%
\pgfpathlineto{\pgfqpoint{3.953121in}{1.156970in}}%
\pgfpathlineto{\pgfqpoint{3.967428in}{1.158742in}}%
\pgfpathlineto{\pgfqpoint{3.981745in}{1.160688in}}%
\pgfpathlineto{\pgfqpoint{3.990059in}{1.173974in}}%
\pgfpathlineto{\pgfqpoint{3.998369in}{1.187461in}}%
\pgfpathlineto{\pgfqpoint{4.006673in}{1.201144in}}%
\pgfpathlineto{\pgfqpoint{4.014972in}{1.215013in}}%
\pgfpathlineto{\pgfqpoint{4.000661in}{1.212297in}}%
\pgfpathlineto{\pgfqpoint{3.986361in}{1.209754in}}%
\pgfpathlineto{\pgfqpoint{3.972070in}{1.207387in}}%
\pgfpathlineto{\pgfqpoint{3.957790in}{1.205194in}}%
\pgfpathlineto{\pgfqpoint{3.949485in}{1.192083in}}%
\pgfpathlineto{\pgfqpoint{3.941174in}{1.179166in}}%
\pgfpathlineto{\pgfqpoint{3.932858in}{1.166452in}}%
\pgfpathlineto{\pgfqpoint{3.924536in}{1.153948in}}%
\pgfpathclose%
\pgfusepath{fill}%
\end{pgfscope}%
\begin{pgfscope}%
\pgfpathrectangle{\pgfqpoint{1.150000in}{0.150000in}}{\pgfqpoint{5.700000in}{5.700000in}}%
\pgfusepath{clip}%
\pgfsetbuttcap%
\pgfsetroundjoin%
\definecolor{currentfill}{rgb}{0.188923,0.410910,0.556326}%
\pgfsetfillcolor{currentfill}%
\pgfsetfillopacity{0.800000}%
\pgfsetlinewidth{0.000000pt}%
\definecolor{currentstroke}{rgb}{0.000000,0.000000,0.000000}%
\pgfsetstrokecolor{currentstroke}%
\pgfsetdash{}{0pt}%
\pgfpathmoveto{\pgfqpoint{4.542066in}{1.939941in}}%
\pgfpathlineto{\pgfqpoint{4.556650in}{1.951263in}}%
\pgfpathlineto{\pgfqpoint{4.571250in}{1.962766in}}%
\pgfpathlineto{\pgfqpoint{4.585869in}{1.974452in}}%
\pgfpathlineto{\pgfqpoint{4.600504in}{1.986319in}}%
\pgfpathlineto{\pgfqpoint{4.608693in}{2.005048in}}%
\pgfpathlineto{\pgfqpoint{4.616878in}{2.023704in}}%
\pgfpathlineto{\pgfqpoint{4.625058in}{2.042282in}}%
\pgfpathlineto{\pgfqpoint{4.633235in}{2.060778in}}%
\pgfpathlineto{\pgfqpoint{4.618588in}{2.048440in}}%
\pgfpathlineto{\pgfqpoint{4.603959in}{2.036286in}}%
\pgfpathlineto{\pgfqpoint{4.589348in}{2.024314in}}%
\pgfpathlineto{\pgfqpoint{4.574755in}{2.012526in}}%
\pgfpathlineto{\pgfqpoint{4.566588in}{1.994486in}}%
\pgfpathlineto{\pgfqpoint{4.558418in}{1.976372in}}%
\pgfpathlineto{\pgfqpoint{4.550244in}{1.958189in}}%
\pgfpathlineto{\pgfqpoint{4.542066in}{1.939941in}}%
\pgfpathclose%
\pgfusepath{fill}%
\end{pgfscope}%
\begin{pgfscope}%
\pgfpathrectangle{\pgfqpoint{1.150000in}{0.150000in}}{\pgfqpoint{5.700000in}{5.700000in}}%
\pgfusepath{clip}%
\pgfsetbuttcap%
\pgfsetroundjoin%
\definecolor{currentfill}{rgb}{0.229739,0.322361,0.545706}%
\pgfsetfillcolor{currentfill}%
\pgfsetfillopacity{0.800000}%
\pgfsetlinewidth{0.000000pt}%
\definecolor{currentstroke}{rgb}{0.000000,0.000000,0.000000}%
\pgfsetstrokecolor{currentstroke}%
\pgfsetdash{}{0pt}%
\pgfpathmoveto{\pgfqpoint{4.385597in}{1.680432in}}%
\pgfpathlineto{\pgfqpoint{4.400088in}{1.689440in}}%
\pgfpathlineto{\pgfqpoint{4.414595in}{1.698626in}}%
\pgfpathlineto{\pgfqpoint{4.429118in}{1.707992in}}%
\pgfpathlineto{\pgfqpoint{4.443656in}{1.717537in}}%
\pgfpathlineto{\pgfqpoint{4.451875in}{1.736206in}}%
\pgfpathlineto{\pgfqpoint{4.460091in}{1.754867in}}%
\pgfpathlineto{\pgfqpoint{4.468304in}{1.773515in}}%
\pgfpathlineto{\pgfqpoint{4.476513in}{1.792145in}}%
\pgfpathlineto{\pgfqpoint{4.461966in}{1.782037in}}%
\pgfpathlineto{\pgfqpoint{4.447434in}{1.772108in}}%
\pgfpathlineto{\pgfqpoint{4.432919in}{1.762359in}}%
\pgfpathlineto{\pgfqpoint{4.418419in}{1.752790in}}%
\pgfpathlineto{\pgfqpoint{4.410219in}{1.734710in}}%
\pgfpathlineto{\pgfqpoint{4.402015in}{1.716621in}}%
\pgfpathlineto{\pgfqpoint{4.393807in}{1.698526in}}%
\pgfpathlineto{\pgfqpoint{4.385597in}{1.680432in}}%
\pgfpathclose%
\pgfusepath{fill}%
\end{pgfscope}%
\begin{pgfscope}%
\pgfpathrectangle{\pgfqpoint{1.150000in}{0.150000in}}{\pgfqpoint{5.700000in}{5.700000in}}%
\pgfusepath{clip}%
\pgfsetbuttcap%
\pgfsetroundjoin%
\definecolor{currentfill}{rgb}{0.283187,0.125848,0.444960}%
\pgfsetfillcolor{currentfill}%
\pgfsetfillopacity{0.800000}%
\pgfsetlinewidth{0.000000pt}%
\definecolor{currentstroke}{rgb}{0.000000,0.000000,0.000000}%
\pgfsetstrokecolor{currentstroke}%
\pgfsetdash{}{0pt}%
\pgfpathmoveto{\pgfqpoint{4.014972in}{1.215013in}}%
\pgfpathlineto{\pgfqpoint{4.029294in}{1.217904in}}%
\pgfpathlineto{\pgfqpoint{4.043626in}{1.220970in}}%
\pgfpathlineto{\pgfqpoint{4.057969in}{1.224210in}}%
\pgfpathlineto{\pgfqpoint{4.072323in}{1.227624in}}%
\pgfpathlineto{\pgfqpoint{4.080614in}{1.242426in}}%
\pgfpathlineto{\pgfqpoint{4.088900in}{1.257393in}}%
\pgfpathlineto{\pgfqpoint{4.097182in}{1.272518in}}%
\pgfpathlineto{\pgfqpoint{4.105459in}{1.287793in}}%
\pgfpathlineto{\pgfqpoint{4.091107in}{1.283637in}}%
\pgfpathlineto{\pgfqpoint{4.076767in}{1.279656in}}%
\pgfpathlineto{\pgfqpoint{4.062438in}{1.275850in}}%
\pgfpathlineto{\pgfqpoint{4.048120in}{1.272219in}}%
\pgfpathlineto{\pgfqpoint{4.039840in}{1.257672in}}%
\pgfpathlineto{\pgfqpoint{4.031556in}{1.243285in}}%
\pgfpathlineto{\pgfqpoint{4.023266in}{1.229063in}}%
\pgfpathlineto{\pgfqpoint{4.014972in}{1.215013in}}%
\pgfpathclose%
\pgfusepath{fill}%
\end{pgfscope}%
\begin{pgfscope}%
\pgfpathrectangle{\pgfqpoint{1.150000in}{0.150000in}}{\pgfqpoint{5.700000in}{5.700000in}}%
\pgfusepath{clip}%
\pgfsetbuttcap%
\pgfsetroundjoin%
\definecolor{currentfill}{rgb}{0.175707,0.697900,0.491033}%
\pgfsetfillcolor{currentfill}%
\pgfsetfillopacity{0.800000}%
\pgfsetlinewidth{0.000000pt}%
\definecolor{currentstroke}{rgb}{0.000000,0.000000,0.000000}%
\pgfsetstrokecolor{currentstroke}%
\pgfsetdash{}{0pt}%
\pgfpathmoveto{\pgfqpoint{5.102530in}{2.838634in}}%
\pgfpathlineto{\pgfqpoint{5.117528in}{2.856126in}}%
\pgfpathlineto{\pgfqpoint{5.132550in}{2.873809in}}%
\pgfpathlineto{\pgfqpoint{5.147594in}{2.891684in}}%
\pgfpathlineto{\pgfqpoint{5.162662in}{2.909752in}}%
\pgfpathlineto{\pgfqpoint{5.170652in}{2.923594in}}%
\pgfpathlineto{\pgfqpoint{5.178633in}{2.937222in}}%
\pgfpathlineto{\pgfqpoint{5.186606in}{2.950635in}}%
\pgfpathlineto{\pgfqpoint{5.194569in}{2.963832in}}%
\pgfpathlineto{\pgfqpoint{5.179496in}{2.945638in}}%
\pgfpathlineto{\pgfqpoint{5.164446in}{2.927636in}}%
\pgfpathlineto{\pgfqpoint{5.149419in}{2.909826in}}%
\pgfpathlineto{\pgfqpoint{5.134414in}{2.892207in}}%
\pgfpathlineto{\pgfqpoint{5.126456in}{2.879124in}}%
\pgfpathlineto{\pgfqpoint{5.118489in}{2.865833in}}%
\pgfpathlineto{\pgfqpoint{5.110513in}{2.852336in}}%
\pgfpathlineto{\pgfqpoint{5.102530in}{2.838634in}}%
\pgfpathclose%
\pgfusepath{fill}%
\end{pgfscope}%
\begin{pgfscope}%
\pgfpathrectangle{\pgfqpoint{1.150000in}{0.150000in}}{\pgfqpoint{5.700000in}{5.700000in}}%
\pgfusepath{clip}%
\pgfsetbuttcap%
\pgfsetroundjoin%
\definecolor{currentfill}{rgb}{0.496615,0.826376,0.306377}%
\pgfsetfillcolor{currentfill}%
\pgfsetfillopacity{0.800000}%
\pgfsetlinewidth{0.000000pt}%
\definecolor{currentstroke}{rgb}{0.000000,0.000000,0.000000}%
\pgfsetstrokecolor{currentstroke}%
\pgfsetdash{}{0pt}%
\pgfpathmoveto{\pgfqpoint{5.473304in}{3.335231in}}%
\pgfpathlineto{\pgfqpoint{5.488601in}{3.355254in}}%
\pgfpathlineto{\pgfqpoint{5.503923in}{3.375472in}}%
\pgfpathlineto{\pgfqpoint{5.519270in}{3.395887in}}%
\pgfpathlineto{\pgfqpoint{5.534644in}{3.416498in}}%
\pgfpathlineto{\pgfqpoint{5.542388in}{3.424902in}}%
\pgfpathlineto{\pgfqpoint{5.550119in}{3.433068in}}%
\pgfpathlineto{\pgfqpoint{5.557839in}{3.440998in}}%
\pgfpathlineto{\pgfqpoint{5.565546in}{3.448695in}}%
\pgfpathlineto{\pgfqpoint{5.550177in}{3.428181in}}%
\pgfpathlineto{\pgfqpoint{5.534834in}{3.407864in}}%
\pgfpathlineto{\pgfqpoint{5.519516in}{3.387742in}}%
\pgfpathlineto{\pgfqpoint{5.504224in}{3.367816in}}%
\pgfpathlineto{\pgfqpoint{5.496512in}{3.360009in}}%
\pgfpathlineto{\pgfqpoint{5.488787in}{3.351977in}}%
\pgfpathlineto{\pgfqpoint{5.481051in}{3.343718in}}%
\pgfpathlineto{\pgfqpoint{5.473304in}{3.335231in}}%
\pgfpathclose%
\pgfusepath{fill}%
\end{pgfscope}%
\begin{pgfscope}%
\pgfpathrectangle{\pgfqpoint{1.150000in}{0.150000in}}{\pgfqpoint{5.700000in}{5.700000in}}%
\pgfusepath{clip}%
\pgfsetbuttcap%
\pgfsetroundjoin%
\definecolor{currentfill}{rgb}{0.129933,0.559582,0.551864}%
\pgfsetfillcolor{currentfill}%
\pgfsetfillopacity{0.800000}%
\pgfsetlinewidth{0.000000pt}%
\definecolor{currentstroke}{rgb}{0.000000,0.000000,0.000000}%
\pgfsetstrokecolor{currentstroke}%
\pgfsetdash{}{0pt}%
\pgfpathmoveto{\pgfqpoint{4.822403in}{2.399842in}}%
\pgfpathlineto{\pgfqpoint{4.837188in}{2.414653in}}%
\pgfpathlineto{\pgfqpoint{4.851994in}{2.429651in}}%
\pgfpathlineto{\pgfqpoint{4.866819in}{2.444837in}}%
\pgfpathlineto{\pgfqpoint{4.881666in}{2.460210in}}%
\pgfpathlineto{\pgfqpoint{4.889783in}{2.477382in}}%
\pgfpathlineto{\pgfqpoint{4.897894in}{2.494394in}}%
\pgfpathlineto{\pgfqpoint{4.905999in}{2.511242in}}%
\pgfpathlineto{\pgfqpoint{4.914097in}{2.527925in}}%
\pgfpathlineto{\pgfqpoint{4.899241in}{2.512248in}}%
\pgfpathlineto{\pgfqpoint{4.884404in}{2.496759in}}%
\pgfpathlineto{\pgfqpoint{4.869589in}{2.481457in}}%
\pgfpathlineto{\pgfqpoint{4.854793in}{2.466344in}}%
\pgfpathlineto{\pgfqpoint{4.846705in}{2.449951in}}%
\pgfpathlineto{\pgfqpoint{4.838610in}{2.433402in}}%
\pgfpathlineto{\pgfqpoint{4.830509in}{2.416698in}}%
\pgfpathlineto{\pgfqpoint{4.822403in}{2.399842in}}%
\pgfpathclose%
\pgfusepath{fill}%
\end{pgfscope}%
\begin{pgfscope}%
\pgfpathrectangle{\pgfqpoint{1.150000in}{0.150000in}}{\pgfqpoint{5.700000in}{5.700000in}}%
\pgfusepath{clip}%
\pgfsetbuttcap%
\pgfsetroundjoin%
\definecolor{currentfill}{rgb}{0.266941,0.748751,0.440573}%
\pgfsetfillcolor{currentfill}%
\pgfsetfillopacity{0.800000}%
\pgfsetlinewidth{0.000000pt}%
\definecolor{currentstroke}{rgb}{0.000000,0.000000,0.000000}%
\pgfsetstrokecolor{currentstroke}%
\pgfsetdash{}{0pt}%
\pgfpathmoveto{\pgfqpoint{5.226332in}{3.014447in}}%
\pgfpathlineto{\pgfqpoint{5.241433in}{3.032925in}}%
\pgfpathlineto{\pgfqpoint{5.256557in}{3.051596in}}%
\pgfpathlineto{\pgfqpoint{5.271705in}{3.070460in}}%
\pgfpathlineto{\pgfqpoint{5.286878in}{3.089519in}}%
\pgfpathlineto{\pgfqpoint{5.294798in}{3.101695in}}%
\pgfpathlineto{\pgfqpoint{5.302709in}{3.113643in}}%
\pgfpathlineto{\pgfqpoint{5.310610in}{3.125363in}}%
\pgfpathlineto{\pgfqpoint{5.318501in}{3.136854in}}%
\pgfpathlineto{\pgfqpoint{5.303325in}{3.117742in}}%
\pgfpathlineto{\pgfqpoint{5.288174in}{3.098824in}}%
\pgfpathlineto{\pgfqpoint{5.273047in}{3.080100in}}%
\pgfpathlineto{\pgfqpoint{5.257944in}{3.061569in}}%
\pgfpathlineto{\pgfqpoint{5.250055in}{3.050117in}}%
\pgfpathlineto{\pgfqpoint{5.242157in}{3.038446in}}%
\pgfpathlineto{\pgfqpoint{5.234249in}{3.026556in}}%
\pgfpathlineto{\pgfqpoint{5.226332in}{3.014447in}}%
\pgfpathclose%
\pgfusepath{fill}%
\end{pgfscope}%
\begin{pgfscope}%
\pgfpathrectangle{\pgfqpoint{1.150000in}{0.150000in}}{\pgfqpoint{5.700000in}{5.700000in}}%
\pgfusepath{clip}%
\pgfsetbuttcap%
\pgfsetroundjoin%
\definecolor{currentfill}{rgb}{0.377779,0.791781,0.377939}%
\pgfsetfillcolor{currentfill}%
\pgfsetfillopacity{0.800000}%
\pgfsetlinewidth{0.000000pt}%
\definecolor{currentstroke}{rgb}{0.000000,0.000000,0.000000}%
\pgfsetstrokecolor{currentstroke}%
\pgfsetdash{}{0pt}%
\pgfpathmoveto{\pgfqpoint{5.349960in}{3.180544in}}%
\pgfpathlineto{\pgfqpoint{5.365160in}{3.199867in}}%
\pgfpathlineto{\pgfqpoint{5.380386in}{3.219384in}}%
\pgfpathlineto{\pgfqpoint{5.395636in}{3.239096in}}%
\pgfpathlineto{\pgfqpoint{5.410910in}{3.259004in}}%
\pgfpathlineto{\pgfqpoint{5.418749in}{3.269352in}}%
\pgfpathlineto{\pgfqpoint{5.426577in}{3.279463in}}%
\pgfpathlineto{\pgfqpoint{5.434393in}{3.289340in}}%
\pgfpathlineto{\pgfqpoint{5.442198in}{3.298983in}}%
\pgfpathlineto{\pgfqpoint{5.426924in}{3.279097in}}%
\pgfpathlineto{\pgfqpoint{5.411674in}{3.259406in}}%
\pgfpathlineto{\pgfqpoint{5.396450in}{3.239910in}}%
\pgfpathlineto{\pgfqpoint{5.381250in}{3.220609in}}%
\pgfpathlineto{\pgfqpoint{5.373443in}{3.210931in}}%
\pgfpathlineto{\pgfqpoint{5.365626in}{3.201028in}}%
\pgfpathlineto{\pgfqpoint{5.357798in}{3.190899in}}%
\pgfpathlineto{\pgfqpoint{5.349960in}{3.180544in}}%
\pgfpathclose%
\pgfusepath{fill}%
\end{pgfscope}%
\begin{pgfscope}%
\pgfpathrectangle{\pgfqpoint{1.150000in}{0.150000in}}{\pgfqpoint{5.700000in}{5.700000in}}%
\pgfusepath{clip}%
\pgfsetbuttcap%
\pgfsetroundjoin%
\definecolor{currentfill}{rgb}{0.160665,0.478540,0.558115}%
\pgfsetfillcolor{currentfill}%
\pgfsetfillopacity{0.800000}%
\pgfsetlinewidth{0.000000pt}%
\definecolor{currentstroke}{rgb}{0.000000,0.000000,0.000000}%
\pgfsetstrokecolor{currentstroke}%
\pgfsetdash{}{0pt}%
\pgfpathmoveto{\pgfqpoint{4.665901in}{2.133860in}}%
\pgfpathlineto{\pgfqpoint{4.680577in}{2.146818in}}%
\pgfpathlineto{\pgfqpoint{4.695272in}{2.159960in}}%
\pgfpathlineto{\pgfqpoint{4.709985in}{2.173286in}}%
\pgfpathlineto{\pgfqpoint{4.724717in}{2.186796in}}%
\pgfpathlineto{\pgfqpoint{4.732884in}{2.205240in}}%
\pgfpathlineto{\pgfqpoint{4.741047in}{2.223569in}}%
\pgfpathlineto{\pgfqpoint{4.749205in}{2.241781in}}%
\pgfpathlineto{\pgfqpoint{4.757358in}{2.259871in}}%
\pgfpathlineto{\pgfqpoint{4.742614in}{2.245955in}}%
\pgfpathlineto{\pgfqpoint{4.727890in}{2.232224in}}%
\pgfpathlineto{\pgfqpoint{4.713184in}{2.218679in}}%
\pgfpathlineto{\pgfqpoint{4.698498in}{2.205318in}}%
\pgfpathlineto{\pgfqpoint{4.690355in}{2.187619in}}%
\pgfpathlineto{\pgfqpoint{4.682208in}{2.169807in}}%
\pgfpathlineto{\pgfqpoint{4.674057in}{2.151886in}}%
\pgfpathlineto{\pgfqpoint{4.665901in}{2.133860in}}%
\pgfpathclose%
\pgfusepath{fill}%
\end{pgfscope}%
\begin{pgfscope}%
\pgfpathrectangle{\pgfqpoint{1.150000in}{0.150000in}}{\pgfqpoint{5.700000in}{5.700000in}}%
\pgfusepath{clip}%
\pgfsetbuttcap%
\pgfsetroundjoin%
\definecolor{currentfill}{rgb}{0.197636,0.391528,0.554969}%
\pgfsetfillcolor{currentfill}%
\pgfsetfillopacity{0.800000}%
\pgfsetlinewidth{0.000000pt}%
\definecolor{currentstroke}{rgb}{0.000000,0.000000,0.000000}%
\pgfsetstrokecolor{currentstroke}%
\pgfsetdash{}{0pt}%
\pgfpathmoveto{\pgfqpoint{4.509318in}{1.866384in}}%
\pgfpathlineto{\pgfqpoint{4.523891in}{1.877205in}}%
\pgfpathlineto{\pgfqpoint{4.538482in}{1.888208in}}%
\pgfpathlineto{\pgfqpoint{4.553089in}{1.899392in}}%
\pgfpathlineto{\pgfqpoint{4.567714in}{1.910757in}}%
\pgfpathlineto{\pgfqpoint{4.575917in}{1.929736in}}%
\pgfpathlineto{\pgfqpoint{4.584116in}{1.948659in}}%
\pgfpathlineto{\pgfqpoint{4.592312in}{1.967521in}}%
\pgfpathlineto{\pgfqpoint{4.600504in}{1.986319in}}%
\pgfpathlineto{\pgfqpoint{4.585869in}{1.974452in}}%
\pgfpathlineto{\pgfqpoint{4.571250in}{1.962766in}}%
\pgfpathlineto{\pgfqpoint{4.556650in}{1.951263in}}%
\pgfpathlineto{\pgfqpoint{4.542066in}{1.939941in}}%
\pgfpathlineto{\pgfqpoint{4.533884in}{1.921632in}}%
\pgfpathlineto{\pgfqpoint{4.525699in}{1.903266in}}%
\pgfpathlineto{\pgfqpoint{4.517510in}{1.884849in}}%
\pgfpathlineto{\pgfqpoint{4.509318in}{1.866384in}}%
\pgfpathclose%
\pgfusepath{fill}%
\end{pgfscope}%
\begin{pgfscope}%
\pgfpathrectangle{\pgfqpoint{1.150000in}{0.150000in}}{\pgfqpoint{5.700000in}{5.700000in}}%
\pgfusepath{clip}%
\pgfsetbuttcap%
\pgfsetroundjoin%
\definecolor{currentfill}{rgb}{0.266580,0.228262,0.514349}%
\pgfsetfillcolor{currentfill}%
\pgfsetfillopacity{0.800000}%
\pgfsetlinewidth{0.000000pt}%
\definecolor{currentstroke}{rgb}{0.000000,0.000000,0.000000}%
\pgfsetstrokecolor{currentstroke}%
\pgfsetdash{}{0pt}%
\pgfpathmoveto{\pgfqpoint{4.229070in}{1.438553in}}%
\pgfpathlineto{\pgfqpoint{4.243487in}{1.444982in}}%
\pgfpathlineto{\pgfqpoint{4.257917in}{1.451586in}}%
\pgfpathlineto{\pgfqpoint{4.272361in}{1.458367in}}%
\pgfpathlineto{\pgfqpoint{4.286818in}{1.465324in}}%
\pgfpathlineto{\pgfqpoint{4.295067in}{1.482980in}}%
\pgfpathlineto{\pgfqpoint{4.303313in}{1.500705in}}%
\pgfpathlineto{\pgfqpoint{4.311556in}{1.518493in}}%
\pgfpathlineto{\pgfqpoint{4.319795in}{1.536337in}}%
\pgfpathlineto{\pgfqpoint{4.305332in}{1.528725in}}%
\pgfpathlineto{\pgfqpoint{4.290883in}{1.521290in}}%
\pgfpathlineto{\pgfqpoint{4.276449in}{1.514031in}}%
\pgfpathlineto{\pgfqpoint{4.262028in}{1.506950in}}%
\pgfpathlineto{\pgfqpoint{4.253793in}{1.489748in}}%
\pgfpathlineto{\pgfqpoint{4.245556in}{1.472610in}}%
\pgfpathlineto{\pgfqpoint{4.237315in}{1.455543in}}%
\pgfpathlineto{\pgfqpoint{4.229070in}{1.438553in}}%
\pgfpathclose%
\pgfusepath{fill}%
\end{pgfscope}%
\begin{pgfscope}%
\pgfpathrectangle{\pgfqpoint{1.150000in}{0.150000in}}{\pgfqpoint{5.700000in}{5.700000in}}%
\pgfusepath{clip}%
\pgfsetbuttcap%
\pgfsetroundjoin%
\definecolor{currentfill}{rgb}{0.280255,0.165693,0.476498}%
\pgfsetfillcolor{currentfill}%
\pgfsetfillopacity{0.800000}%
\pgfsetlinewidth{0.000000pt}%
\definecolor{currentstroke}{rgb}{0.000000,0.000000,0.000000}%
\pgfsetstrokecolor{currentstroke}%
\pgfsetdash{}{0pt}%
\pgfpathmoveto{\pgfqpoint{4.105459in}{1.287793in}}%
\pgfpathlineto{\pgfqpoint{4.119823in}{1.292124in}}%
\pgfpathlineto{\pgfqpoint{4.134198in}{1.296630in}}%
\pgfpathlineto{\pgfqpoint{4.148586in}{1.301310in}}%
\pgfpathlineto{\pgfqpoint{4.162985in}{1.306165in}}%
\pgfpathlineto{\pgfqpoint{4.171259in}{1.322308in}}%
\pgfpathlineto{\pgfqpoint{4.179528in}{1.338581in}}%
\pgfpathlineto{\pgfqpoint{4.187794in}{1.354977in}}%
\pgfpathlineto{\pgfqpoint{4.196057in}{1.371487in}}%
\pgfpathlineto{\pgfqpoint{4.181656in}{1.365918in}}%
\pgfpathlineto{\pgfqpoint{4.167267in}{1.360525in}}%
\pgfpathlineto{\pgfqpoint{4.152892in}{1.355306in}}%
\pgfpathlineto{\pgfqpoint{4.138528in}{1.350264in}}%
\pgfpathlineto{\pgfqpoint{4.130267in}{1.334455in}}%
\pgfpathlineto{\pgfqpoint{4.122002in}{1.318768in}}%
\pgfpathlineto{\pgfqpoint{4.113733in}{1.303212in}}%
\pgfpathlineto{\pgfqpoint{4.105459in}{1.287793in}}%
\pgfpathclose%
\pgfusepath{fill}%
\end{pgfscope}%
\begin{pgfscope}%
\pgfpathrectangle{\pgfqpoint{1.150000in}{0.150000in}}{\pgfqpoint{5.700000in}{5.700000in}}%
\pgfusepath{clip}%
\pgfsetbuttcap%
\pgfsetroundjoin%
\definecolor{currentfill}{rgb}{0.120638,0.625828,0.533488}%
\pgfsetfillcolor{currentfill}%
\pgfsetfillopacity{0.800000}%
\pgfsetlinewidth{0.000000pt}%
\definecolor{currentstroke}{rgb}{0.000000,0.000000,0.000000}%
\pgfsetstrokecolor{currentstroke}%
\pgfsetdash{}{0pt}%
\pgfpathmoveto{\pgfqpoint{4.946429in}{2.592950in}}%
\pgfpathlineto{\pgfqpoint{4.961317in}{2.609085in}}%
\pgfpathlineto{\pgfqpoint{4.976226in}{2.625410in}}%
\pgfpathlineto{\pgfqpoint{4.991156in}{2.641924in}}%
\pgfpathlineto{\pgfqpoint{5.006109in}{2.658628in}}%
\pgfpathlineto{\pgfqpoint{5.014185in}{2.674694in}}%
\pgfpathlineto{\pgfqpoint{5.022254in}{2.690573in}}%
\pgfpathlineto{\pgfqpoint{5.030315in}{2.706261in}}%
\pgfpathlineto{\pgfqpoint{5.038370in}{2.721758in}}%
\pgfpathlineto{\pgfqpoint{5.023408in}{2.704819in}}%
\pgfpathlineto{\pgfqpoint{5.008469in}{2.688071in}}%
\pgfpathlineto{\pgfqpoint{4.993551in}{2.671512in}}%
\pgfpathlineto{\pgfqpoint{4.978654in}{2.655143in}}%
\pgfpathlineto{\pgfqpoint{4.970608in}{2.639868in}}%
\pgfpathlineto{\pgfqpoint{4.962555in}{2.624409in}}%
\pgfpathlineto{\pgfqpoint{4.954496in}{2.608769in}}%
\pgfpathlineto{\pgfqpoint{4.946429in}{2.592950in}}%
\pgfpathclose%
\pgfusepath{fill}%
\end{pgfscope}%
\begin{pgfscope}%
\pgfpathrectangle{\pgfqpoint{1.150000in}{0.150000in}}{\pgfqpoint{5.700000in}{5.700000in}}%
\pgfusepath{clip}%
\pgfsetbuttcap%
\pgfsetroundjoin%
\definecolor{currentfill}{rgb}{0.239346,0.300855,0.540844}%
\pgfsetfillcolor{currentfill}%
\pgfsetfillopacity{0.800000}%
\pgfsetlinewidth{0.000000pt}%
\definecolor{currentstroke}{rgb}{0.000000,0.000000,0.000000}%
\pgfsetstrokecolor{currentstroke}%
\pgfsetdash{}{0pt}%
\pgfpathmoveto{\pgfqpoint{4.352722in}{1.608164in}}%
\pgfpathlineto{\pgfqpoint{4.367206in}{1.616578in}}%
\pgfpathlineto{\pgfqpoint{4.381705in}{1.625171in}}%
\pgfpathlineto{\pgfqpoint{4.396219in}{1.633942in}}%
\pgfpathlineto{\pgfqpoint{4.410749in}{1.642891in}}%
\pgfpathlineto{\pgfqpoint{4.418980in}{1.661537in}}%
\pgfpathlineto{\pgfqpoint{4.427209in}{1.680197in}}%
\pgfpathlineto{\pgfqpoint{4.435434in}{1.698866in}}%
\pgfpathlineto{\pgfqpoint{4.443656in}{1.717537in}}%
\pgfpathlineto{\pgfqpoint{4.429118in}{1.707992in}}%
\pgfpathlineto{\pgfqpoint{4.414595in}{1.698626in}}%
\pgfpathlineto{\pgfqpoint{4.400088in}{1.689440in}}%
\pgfpathlineto{\pgfqpoint{4.385597in}{1.680432in}}%
\pgfpathlineto{\pgfqpoint{4.377383in}{1.662343in}}%
\pgfpathlineto{\pgfqpoint{4.369166in}{1.644265in}}%
\pgfpathlineto{\pgfqpoint{4.360945in}{1.626204in}}%
\pgfpathlineto{\pgfqpoint{4.352722in}{1.608164in}}%
\pgfpathclose%
\pgfusepath{fill}%
\end{pgfscope}%
\begin{pgfscope}%
\pgfpathrectangle{\pgfqpoint{1.150000in}{0.150000in}}{\pgfqpoint{5.700000in}{5.700000in}}%
\pgfusepath{clip}%
\pgfsetbuttcap%
\pgfsetroundjoin%
\definecolor{currentfill}{rgb}{0.281446,0.084320,0.407414}%
\pgfsetfillcolor{currentfill}%
\pgfsetfillopacity{0.800000}%
\pgfsetlinewidth{0.000000pt}%
\definecolor{currentstroke}{rgb}{0.000000,0.000000,0.000000}%
\pgfsetstrokecolor{currentstroke}%
\pgfsetdash{}{0pt}%
\pgfpathmoveto{\pgfqpoint{3.891186in}{1.106184in}}%
\pgfpathlineto{\pgfqpoint{3.905483in}{1.106808in}}%
\pgfpathlineto{\pgfqpoint{3.919790in}{1.107605in}}%
\pgfpathlineto{\pgfqpoint{3.934105in}{1.108576in}}%
\pgfpathlineto{\pgfqpoint{3.948430in}{1.109720in}}%
\pgfpathlineto{\pgfqpoint{3.956767in}{1.122121in}}%
\pgfpathlineto{\pgfqpoint{3.965099in}{1.134754in}}%
\pgfpathlineto{\pgfqpoint{3.973424in}{1.147612in}}%
\pgfpathlineto{\pgfqpoint{3.981745in}{1.160688in}}%
\pgfpathlineto{\pgfqpoint{3.967428in}{1.158742in}}%
\pgfpathlineto{\pgfqpoint{3.953121in}{1.156970in}}%
\pgfpathlineto{\pgfqpoint{3.938823in}{1.155372in}}%
\pgfpathlineto{\pgfqpoint{3.924536in}{1.153948in}}%
\pgfpathlineto{\pgfqpoint{3.916208in}{1.141661in}}%
\pgfpathlineto{\pgfqpoint{3.907873in}{1.129600in}}%
\pgfpathlineto{\pgfqpoint{3.899533in}{1.117772in}}%
\pgfpathlineto{\pgfqpoint{3.891186in}{1.106184in}}%
\pgfpathclose%
\pgfusepath{fill}%
\end{pgfscope}%
\begin{pgfscope}%
\pgfpathrectangle{\pgfqpoint{1.150000in}{0.150000in}}{\pgfqpoint{5.700000in}{5.700000in}}%
\pgfusepath{clip}%
\pgfsetbuttcap%
\pgfsetroundjoin%
\definecolor{currentfill}{rgb}{0.595839,0.848717,0.243329}%
\pgfsetfillcolor{currentfill}%
\pgfsetfillopacity{0.800000}%
\pgfsetlinewidth{0.000000pt}%
\definecolor{currentstroke}{rgb}{0.000000,0.000000,0.000000}%
\pgfsetstrokecolor{currentstroke}%
\pgfsetdash{}{0pt}%
\pgfpathmoveto{\pgfqpoint{5.565546in}{3.448695in}}%
\pgfpathlineto{\pgfqpoint{5.580940in}{3.469405in}}%
\pgfpathlineto{\pgfqpoint{5.596361in}{3.490312in}}%
\pgfpathlineto{\pgfqpoint{5.611808in}{3.511417in}}%
\pgfpathlineto{\pgfqpoint{5.619498in}{3.518790in}}%
\pgfpathlineto{\pgfqpoint{5.627176in}{3.525924in}}%
\pgfpathlineto{\pgfqpoint{5.634840in}{3.532822in}}%
\pgfpathlineto{\pgfqpoint{5.642492in}{3.539485in}}%
\pgfpathlineto{\pgfqpoint{5.627052in}{3.518517in}}%
\pgfpathlineto{\pgfqpoint{5.611638in}{3.497746in}}%
\pgfpathlineto{\pgfqpoint{5.596250in}{3.477171in}}%
\pgfpathlineto{\pgfqpoint{5.588592in}{3.470395in}}%
\pgfpathlineto{\pgfqpoint{5.580923in}{3.463391in}}%
\pgfpathlineto{\pgfqpoint{5.573240in}{3.456158in}}%
\pgfpathlineto{\pgfqpoint{5.565546in}{3.448695in}}%
\pgfpathclose%
\pgfusepath{fill}%
\end{pgfscope}%
\begin{pgfscope}%
\pgfpathrectangle{\pgfqpoint{1.150000in}{0.150000in}}{\pgfqpoint{5.700000in}{5.700000in}}%
\pgfusepath{clip}%
\pgfsetbuttcap%
\pgfsetroundjoin%
\definecolor{currentfill}{rgb}{0.135066,0.544853,0.554029}%
\pgfsetfillcolor{currentfill}%
\pgfsetfillopacity{0.800000}%
\pgfsetlinewidth{0.000000pt}%
\definecolor{currentstroke}{rgb}{0.000000,0.000000,0.000000}%
\pgfsetstrokecolor{currentstroke}%
\pgfsetdash{}{0pt}%
\pgfpathmoveto{\pgfqpoint{4.789922in}{2.330956in}}%
\pgfpathlineto{\pgfqpoint{4.804697in}{2.345430in}}%
\pgfpathlineto{\pgfqpoint{4.819491in}{2.360090in}}%
\pgfpathlineto{\pgfqpoint{4.834306in}{2.374937in}}%
\pgfpathlineto{\pgfqpoint{4.849140in}{2.389971in}}%
\pgfpathlineto{\pgfqpoint{4.857280in}{2.407757in}}%
\pgfpathlineto{\pgfqpoint{4.865414in}{2.425395in}}%
\pgfpathlineto{\pgfqpoint{4.873543in}{2.442880in}}%
\pgfpathlineto{\pgfqpoint{4.881666in}{2.460210in}}%
\pgfpathlineto{\pgfqpoint{4.866819in}{2.444837in}}%
\pgfpathlineto{\pgfqpoint{4.851994in}{2.429651in}}%
\pgfpathlineto{\pgfqpoint{4.837188in}{2.414653in}}%
\pgfpathlineto{\pgfqpoint{4.822403in}{2.399842in}}%
\pgfpathlineto{\pgfqpoint{4.814291in}{2.382837in}}%
\pgfpathlineto{\pgfqpoint{4.806173in}{2.365686in}}%
\pgfpathlineto{\pgfqpoint{4.798051in}{2.348391in}}%
\pgfpathlineto{\pgfqpoint{4.789922in}{2.330956in}}%
\pgfpathclose%
\pgfusepath{fill}%
\end{pgfscope}%
\begin{pgfscope}%
\pgfpathrectangle{\pgfqpoint{1.150000in}{0.150000in}}{\pgfqpoint{5.700000in}{5.700000in}}%
\pgfusepath{clip}%
\pgfsetbuttcap%
\pgfsetroundjoin%
\definecolor{currentfill}{rgb}{0.168126,0.459988,0.558082}%
\pgfsetfillcolor{currentfill}%
\pgfsetfillopacity{0.800000}%
\pgfsetlinewidth{0.000000pt}%
\definecolor{currentstroke}{rgb}{0.000000,0.000000,0.000000}%
\pgfsetstrokecolor{currentstroke}%
\pgfsetdash{}{0pt}%
\pgfpathmoveto{\pgfqpoint{4.633235in}{2.060778in}}%
\pgfpathlineto{\pgfqpoint{4.647900in}{2.073298in}}%
\pgfpathlineto{\pgfqpoint{4.662583in}{2.086002in}}%
\pgfpathlineto{\pgfqpoint{4.677284in}{2.098890in}}%
\pgfpathlineto{\pgfqpoint{4.692004in}{2.111961in}}%
\pgfpathlineto{\pgfqpoint{4.700189in}{2.130822in}}%
\pgfpathlineto{\pgfqpoint{4.708369in}{2.149584in}}%
\pgfpathlineto{\pgfqpoint{4.716545in}{2.168243in}}%
\pgfpathlineto{\pgfqpoint{4.724717in}{2.186796in}}%
\pgfpathlineto{\pgfqpoint{4.709985in}{2.173286in}}%
\pgfpathlineto{\pgfqpoint{4.695272in}{2.159960in}}%
\pgfpathlineto{\pgfqpoint{4.680577in}{2.146818in}}%
\pgfpathlineto{\pgfqpoint{4.665901in}{2.133860in}}%
\pgfpathlineto{\pgfqpoint{4.657741in}{2.115733in}}%
\pgfpathlineto{\pgfqpoint{4.649577in}{2.097507in}}%
\pgfpathlineto{\pgfqpoint{4.641408in}{2.079187in}}%
\pgfpathlineto{\pgfqpoint{4.633235in}{2.060778in}}%
\pgfpathclose%
\pgfusepath{fill}%
\end{pgfscope}%
\begin{pgfscope}%
\pgfpathrectangle{\pgfqpoint{1.150000in}{0.150000in}}{\pgfqpoint{5.700000in}{5.700000in}}%
\pgfusepath{clip}%
\pgfsetbuttcap%
\pgfsetroundjoin%
\definecolor{currentfill}{rgb}{0.283091,0.110553,0.431554}%
\pgfsetfillcolor{currentfill}%
\pgfsetfillopacity{0.800000}%
\pgfsetlinewidth{0.000000pt}%
\definecolor{currentstroke}{rgb}{0.000000,0.000000,0.000000}%
\pgfsetstrokecolor{currentstroke}%
\pgfsetdash{}{0pt}%
\pgfpathmoveto{\pgfqpoint{3.981745in}{1.160688in}}%
\pgfpathlineto{\pgfqpoint{3.996072in}{1.162808in}}%
\pgfpathlineto{\pgfqpoint{4.010409in}{1.165102in}}%
\pgfpathlineto{\pgfqpoint{4.024756in}{1.167569in}}%
\pgfpathlineto{\pgfqpoint{4.039114in}{1.170210in}}%
\pgfpathlineto{\pgfqpoint{4.047423in}{1.184279in}}%
\pgfpathlineto{\pgfqpoint{4.055728in}{1.198543in}}%
\pgfpathlineto{\pgfqpoint{4.064028in}{1.212993in}}%
\pgfpathlineto{\pgfqpoint{4.072323in}{1.227624in}}%
\pgfpathlineto{\pgfqpoint{4.057969in}{1.224210in}}%
\pgfpathlineto{\pgfqpoint{4.043626in}{1.220970in}}%
\pgfpathlineto{\pgfqpoint{4.029294in}{1.217904in}}%
\pgfpathlineto{\pgfqpoint{4.014972in}{1.215013in}}%
\pgfpathlineto{\pgfqpoint{4.006673in}{1.201144in}}%
\pgfpathlineto{\pgfqpoint{3.998369in}{1.187461in}}%
\pgfpathlineto{\pgfqpoint{3.990059in}{1.173974in}}%
\pgfpathlineto{\pgfqpoint{3.981745in}{1.160688in}}%
\pgfpathclose%
\pgfusepath{fill}%
\end{pgfscope}%
\begin{pgfscope}%
\pgfpathrectangle{\pgfqpoint{1.150000in}{0.150000in}}{\pgfqpoint{5.700000in}{5.700000in}}%
\pgfusepath{clip}%
\pgfsetbuttcap%
\pgfsetroundjoin%
\definecolor{currentfill}{rgb}{0.162016,0.687316,0.499129}%
\pgfsetfillcolor{currentfill}%
\pgfsetfillopacity{0.800000}%
\pgfsetlinewidth{0.000000pt}%
\definecolor{currentstroke}{rgb}{0.000000,0.000000,0.000000}%
\pgfsetstrokecolor{currentstroke}%
\pgfsetdash{}{0pt}%
\pgfpathmoveto{\pgfqpoint{5.070513in}{2.781793in}}%
\pgfpathlineto{\pgfqpoint{5.085505in}{2.799122in}}%
\pgfpathlineto{\pgfqpoint{5.100519in}{2.816642in}}%
\pgfpathlineto{\pgfqpoint{5.115556in}{2.834353in}}%
\pgfpathlineto{\pgfqpoint{5.130616in}{2.852257in}}%
\pgfpathlineto{\pgfqpoint{5.138640in}{2.866947in}}%
\pgfpathlineto{\pgfqpoint{5.146656in}{2.881427in}}%
\pgfpathlineto{\pgfqpoint{5.154663in}{2.895696in}}%
\pgfpathlineto{\pgfqpoint{5.162662in}{2.909752in}}%
\pgfpathlineto{\pgfqpoint{5.147594in}{2.891684in}}%
\pgfpathlineto{\pgfqpoint{5.132550in}{2.873809in}}%
\pgfpathlineto{\pgfqpoint{5.117528in}{2.856126in}}%
\pgfpathlineto{\pgfqpoint{5.102530in}{2.838634in}}%
\pgfpathlineto{\pgfqpoint{5.094537in}{2.824728in}}%
\pgfpathlineto{\pgfqpoint{5.086537in}{2.810618in}}%
\pgfpathlineto{\pgfqpoint{5.078529in}{2.796306in}}%
\pgfpathlineto{\pgfqpoint{5.070513in}{2.781793in}}%
\pgfpathclose%
\pgfusepath{fill}%
\end{pgfscope}%
\begin{pgfscope}%
\pgfpathrectangle{\pgfqpoint{1.150000in}{0.150000in}}{\pgfqpoint{5.700000in}{5.700000in}}%
\pgfusepath{clip}%
\pgfsetbuttcap%
\pgfsetroundjoin%
\definecolor{currentfill}{rgb}{0.206756,0.371758,0.553117}%
\pgfsetfillcolor{currentfill}%
\pgfsetfillopacity{0.800000}%
\pgfsetlinewidth{0.000000pt}%
\definecolor{currentstroke}{rgb}{0.000000,0.000000,0.000000}%
\pgfsetstrokecolor{currentstroke}%
\pgfsetdash{}{0pt}%
\pgfpathmoveto{\pgfqpoint{4.476513in}{1.792145in}}%
\pgfpathlineto{\pgfqpoint{4.491077in}{1.802434in}}%
\pgfpathlineto{\pgfqpoint{4.505657in}{1.812904in}}%
\pgfpathlineto{\pgfqpoint{4.520254in}{1.823553in}}%
\pgfpathlineto{\pgfqpoint{4.534868in}{1.834383in}}%
\pgfpathlineto{\pgfqpoint{4.543084in}{1.853536in}}%
\pgfpathlineto{\pgfqpoint{4.551297in}{1.872653in}}%
\pgfpathlineto{\pgfqpoint{4.559507in}{1.891728in}}%
\pgfpathlineto{\pgfqpoint{4.567714in}{1.910757in}}%
\pgfpathlineto{\pgfqpoint{4.553089in}{1.899392in}}%
\pgfpathlineto{\pgfqpoint{4.538482in}{1.888208in}}%
\pgfpathlineto{\pgfqpoint{4.523891in}{1.877205in}}%
\pgfpathlineto{\pgfqpoint{4.509318in}{1.866384in}}%
\pgfpathlineto{\pgfqpoint{4.501122in}{1.847876in}}%
\pgfpathlineto{\pgfqpoint{4.492922in}{1.829331in}}%
\pgfpathlineto{\pgfqpoint{4.484720in}{1.810752in}}%
\pgfpathlineto{\pgfqpoint{4.476513in}{1.792145in}}%
\pgfpathclose%
\pgfusepath{fill}%
\end{pgfscope}%
\begin{pgfscope}%
\pgfpathrectangle{\pgfqpoint{1.150000in}{0.150000in}}{\pgfqpoint{5.700000in}{5.700000in}}%
\pgfusepath{clip}%
\pgfsetbuttcap%
\pgfsetroundjoin%
\definecolor{currentfill}{rgb}{0.271828,0.209303,0.504434}%
\pgfsetfillcolor{currentfill}%
\pgfsetfillopacity{0.800000}%
\pgfsetlinewidth{0.000000pt}%
\definecolor{currentstroke}{rgb}{0.000000,0.000000,0.000000}%
\pgfsetstrokecolor{currentstroke}%
\pgfsetdash{}{0pt}%
\pgfpathmoveto{\pgfqpoint{4.196057in}{1.371487in}}%
\pgfpathlineto{\pgfqpoint{4.210470in}{1.377232in}}%
\pgfpathlineto{\pgfqpoint{4.224897in}{1.383152in}}%
\pgfpathlineto{\pgfqpoint{4.239337in}{1.389247in}}%
\pgfpathlineto{\pgfqpoint{4.253790in}{1.395518in}}%
\pgfpathlineto{\pgfqpoint{4.262052in}{1.412834in}}%
\pgfpathlineto{\pgfqpoint{4.270310in}{1.430245in}}%
\pgfpathlineto{\pgfqpoint{4.278566in}{1.447744in}}%
\pgfpathlineto{\pgfqpoint{4.286818in}{1.465324in}}%
\pgfpathlineto{\pgfqpoint{4.272361in}{1.458367in}}%
\pgfpathlineto{\pgfqpoint{4.257917in}{1.451586in}}%
\pgfpathlineto{\pgfqpoint{4.243487in}{1.444982in}}%
\pgfpathlineto{\pgfqpoint{4.229070in}{1.438553in}}%
\pgfpathlineto{\pgfqpoint{4.220822in}{1.421646in}}%
\pgfpathlineto{\pgfqpoint{4.212570in}{1.404829in}}%
\pgfpathlineto{\pgfqpoint{4.204315in}{1.388107in}}%
\pgfpathlineto{\pgfqpoint{4.196057in}{1.371487in}}%
\pgfpathclose%
\pgfusepath{fill}%
\end{pgfscope}%
\begin{pgfscope}%
\pgfpathrectangle{\pgfqpoint{1.150000in}{0.150000in}}{\pgfqpoint{5.700000in}{5.700000in}}%
\pgfusepath{clip}%
\pgfsetbuttcap%
\pgfsetroundjoin%
\definecolor{currentfill}{rgb}{0.252899,0.742211,0.448284}%
\pgfsetfillcolor{currentfill}%
\pgfsetfillopacity{0.800000}%
\pgfsetlinewidth{0.000000pt}%
\definecolor{currentstroke}{rgb}{0.000000,0.000000,0.000000}%
\pgfsetstrokecolor{currentstroke}%
\pgfsetdash{}{0pt}%
\pgfpathmoveto{\pgfqpoint{5.194569in}{2.963832in}}%
\pgfpathlineto{\pgfqpoint{5.209666in}{2.982220in}}%
\pgfpathlineto{\pgfqpoint{5.224786in}{3.000800in}}%
\pgfpathlineto{\pgfqpoint{5.239929in}{3.019574in}}%
\pgfpathlineto{\pgfqpoint{5.255097in}{3.038542in}}%
\pgfpathlineto{\pgfqpoint{5.263056in}{3.051626in}}%
\pgfpathlineto{\pgfqpoint{5.271007in}{3.064484in}}%
\pgfpathlineto{\pgfqpoint{5.278947in}{3.077115in}}%
\pgfpathlineto{\pgfqpoint{5.286878in}{3.089519in}}%
\pgfpathlineto{\pgfqpoint{5.271705in}{3.070460in}}%
\pgfpathlineto{\pgfqpoint{5.256557in}{3.051596in}}%
\pgfpathlineto{\pgfqpoint{5.241433in}{3.032925in}}%
\pgfpathlineto{\pgfqpoint{5.226332in}{3.014447in}}%
\pgfpathlineto{\pgfqpoint{5.218405in}{3.002120in}}%
\pgfpathlineto{\pgfqpoint{5.210469in}{2.989575in}}%
\pgfpathlineto{\pgfqpoint{5.202524in}{2.976812in}}%
\pgfpathlineto{\pgfqpoint{5.194569in}{2.963832in}}%
\pgfpathclose%
\pgfusepath{fill}%
\end{pgfscope}%
\begin{pgfscope}%
\pgfpathrectangle{\pgfqpoint{1.150000in}{0.150000in}}{\pgfqpoint{5.700000in}{5.700000in}}%
\pgfusepath{clip}%
\pgfsetbuttcap%
\pgfsetroundjoin%
\definecolor{currentfill}{rgb}{0.248629,0.278775,0.534556}%
\pgfsetfillcolor{currentfill}%
\pgfsetfillopacity{0.800000}%
\pgfsetlinewidth{0.000000pt}%
\definecolor{currentstroke}{rgb}{0.000000,0.000000,0.000000}%
\pgfsetstrokecolor{currentstroke}%
\pgfsetdash{}{0pt}%
\pgfpathmoveto{\pgfqpoint{4.319795in}{1.536337in}}%
\pgfpathlineto{\pgfqpoint{4.334273in}{1.544127in}}%
\pgfpathlineto{\pgfqpoint{4.348765in}{1.552094in}}%
\pgfpathlineto{\pgfqpoint{4.363271in}{1.560238in}}%
\pgfpathlineto{\pgfqpoint{4.377793in}{1.568560in}}%
\pgfpathlineto{\pgfqpoint{4.386036in}{1.587093in}}%
\pgfpathlineto{\pgfqpoint{4.394277in}{1.605663in}}%
\pgfpathlineto{\pgfqpoint{4.402514in}{1.624264in}}%
\pgfpathlineto{\pgfqpoint{4.410749in}{1.642891in}}%
\pgfpathlineto{\pgfqpoint{4.396219in}{1.633942in}}%
\pgfpathlineto{\pgfqpoint{4.381705in}{1.625171in}}%
\pgfpathlineto{\pgfqpoint{4.367206in}{1.616578in}}%
\pgfpathlineto{\pgfqpoint{4.352722in}{1.608164in}}%
\pgfpathlineto{\pgfqpoint{4.344495in}{1.590152in}}%
\pgfpathlineto{\pgfqpoint{4.336265in}{1.572173in}}%
\pgfpathlineto{\pgfqpoint{4.328032in}{1.554233in}}%
\pgfpathlineto{\pgfqpoint{4.319795in}{1.536337in}}%
\pgfpathclose%
\pgfusepath{fill}%
\end{pgfscope}%
\begin{pgfscope}%
\pgfpathrectangle{\pgfqpoint{1.150000in}{0.150000in}}{\pgfqpoint{5.700000in}{5.700000in}}%
\pgfusepath{clip}%
\pgfsetbuttcap%
\pgfsetroundjoin%
\definecolor{currentfill}{rgb}{0.487026,0.823929,0.312321}%
\pgfsetfillcolor{currentfill}%
\pgfsetfillopacity{0.800000}%
\pgfsetlinewidth{0.000000pt}%
\definecolor{currentstroke}{rgb}{0.000000,0.000000,0.000000}%
\pgfsetstrokecolor{currentstroke}%
\pgfsetdash{}{0pt}%
\pgfpathmoveto{\pgfqpoint{5.442198in}{3.298983in}}%
\pgfpathlineto{\pgfqpoint{5.457497in}{3.319065in}}%
\pgfpathlineto{\pgfqpoint{5.472822in}{3.339343in}}%
\pgfpathlineto{\pgfqpoint{5.488172in}{3.359818in}}%
\pgfpathlineto{\pgfqpoint{5.503548in}{3.380489in}}%
\pgfpathlineto{\pgfqpoint{5.511340in}{3.389853in}}%
\pgfpathlineto{\pgfqpoint{5.519120in}{3.398975in}}%
\pgfpathlineto{\pgfqpoint{5.526888in}{3.407857in}}%
\pgfpathlineto{\pgfqpoint{5.534644in}{3.416498in}}%
\pgfpathlineto{\pgfqpoint{5.519270in}{3.395887in}}%
\pgfpathlineto{\pgfqpoint{5.503923in}{3.375472in}}%
\pgfpathlineto{\pgfqpoint{5.488601in}{3.355254in}}%
\pgfpathlineto{\pgfqpoint{5.473304in}{3.335231in}}%
\pgfpathlineto{\pgfqpoint{5.465545in}{3.326516in}}%
\pgfpathlineto{\pgfqpoint{5.457774in}{3.317570in}}%
\pgfpathlineto{\pgfqpoint{5.449992in}{3.308393in}}%
\pgfpathlineto{\pgfqpoint{5.442198in}{3.298983in}}%
\pgfpathclose%
\pgfusepath{fill}%
\end{pgfscope}%
\begin{pgfscope}%
\pgfpathrectangle{\pgfqpoint{1.150000in}{0.150000in}}{\pgfqpoint{5.700000in}{5.700000in}}%
\pgfusepath{clip}%
\pgfsetbuttcap%
\pgfsetroundjoin%
\definecolor{currentfill}{rgb}{0.282290,0.145912,0.461510}%
\pgfsetfillcolor{currentfill}%
\pgfsetfillopacity{0.800000}%
\pgfsetlinewidth{0.000000pt}%
\definecolor{currentstroke}{rgb}{0.000000,0.000000,0.000000}%
\pgfsetstrokecolor{currentstroke}%
\pgfsetdash{}{0pt}%
\pgfpathmoveto{\pgfqpoint{4.072323in}{1.227624in}}%
\pgfpathlineto{\pgfqpoint{4.086688in}{1.231211in}}%
\pgfpathlineto{\pgfqpoint{4.101065in}{1.234973in}}%
\pgfpathlineto{\pgfqpoint{4.115453in}{1.238909in}}%
\pgfpathlineto{\pgfqpoint{4.129853in}{1.243019in}}%
\pgfpathlineto{\pgfqpoint{4.138142in}{1.258577in}}%
\pgfpathlineto{\pgfqpoint{4.146427in}{1.274292in}}%
\pgfpathlineto{\pgfqpoint{4.154708in}{1.290157in}}%
\pgfpathlineto{\pgfqpoint{4.162985in}{1.306165in}}%
\pgfpathlineto{\pgfqpoint{4.148586in}{1.301310in}}%
\pgfpathlineto{\pgfqpoint{4.134198in}{1.296630in}}%
\pgfpathlineto{\pgfqpoint{4.119823in}{1.292124in}}%
\pgfpathlineto{\pgfqpoint{4.105459in}{1.287793in}}%
\pgfpathlineto{\pgfqpoint{4.097182in}{1.272518in}}%
\pgfpathlineto{\pgfqpoint{4.088900in}{1.257393in}}%
\pgfpathlineto{\pgfqpoint{4.080614in}{1.242426in}}%
\pgfpathlineto{\pgfqpoint{4.072323in}{1.227624in}}%
\pgfpathclose%
\pgfusepath{fill}%
\end{pgfscope}%
\begin{pgfscope}%
\pgfpathrectangle{\pgfqpoint{1.150000in}{0.150000in}}{\pgfqpoint{5.700000in}{5.700000in}}%
\pgfusepath{clip}%
\pgfsetbuttcap%
\pgfsetroundjoin%
\definecolor{currentfill}{rgb}{0.369214,0.788888,0.382914}%
\pgfsetfillcolor{currentfill}%
\pgfsetfillopacity{0.800000}%
\pgfsetlinewidth{0.000000pt}%
\definecolor{currentstroke}{rgb}{0.000000,0.000000,0.000000}%
\pgfsetstrokecolor{currentstroke}%
\pgfsetdash{}{0pt}%
\pgfpathmoveto{\pgfqpoint{5.318501in}{3.136854in}}%
\pgfpathlineto{\pgfqpoint{5.333700in}{3.156161in}}%
\pgfpathlineto{\pgfqpoint{5.348924in}{3.175662in}}%
\pgfpathlineto{\pgfqpoint{5.364173in}{3.195358in}}%
\pgfpathlineto{\pgfqpoint{5.379446in}{3.215250in}}%
\pgfpathlineto{\pgfqpoint{5.387328in}{3.226544in}}%
\pgfpathlineto{\pgfqpoint{5.395200in}{3.237601in}}%
\pgfpathlineto{\pgfqpoint{5.403061in}{3.248421in}}%
\pgfpathlineto{\pgfqpoint{5.410910in}{3.259004in}}%
\pgfpathlineto{\pgfqpoint{5.395636in}{3.239096in}}%
\pgfpathlineto{\pgfqpoint{5.380386in}{3.219384in}}%
\pgfpathlineto{\pgfqpoint{5.365160in}{3.199867in}}%
\pgfpathlineto{\pgfqpoint{5.349960in}{3.180544in}}%
\pgfpathlineto{\pgfqpoint{5.342111in}{3.169963in}}%
\pgfpathlineto{\pgfqpoint{5.334251in}{3.159154in}}%
\pgfpathlineto{\pgfqpoint{5.326381in}{3.148118in}}%
\pgfpathlineto{\pgfqpoint{5.318501in}{3.136854in}}%
\pgfpathclose%
\pgfusepath{fill}%
\end{pgfscope}%
\begin{pgfscope}%
\pgfpathrectangle{\pgfqpoint{1.150000in}{0.150000in}}{\pgfqpoint{5.700000in}{5.700000in}}%
\pgfusepath{clip}%
\pgfsetbuttcap%
\pgfsetroundjoin%
\definecolor{currentfill}{rgb}{0.119423,0.611141,0.538982}%
\pgfsetfillcolor{currentfill}%
\pgfsetfillopacity{0.800000}%
\pgfsetlinewidth{0.000000pt}%
\definecolor{currentstroke}{rgb}{0.000000,0.000000,0.000000}%
\pgfsetstrokecolor{currentstroke}%
\pgfsetdash{}{0pt}%
\pgfpathmoveto{\pgfqpoint{4.914097in}{2.527925in}}%
\pgfpathlineto{\pgfqpoint{4.928975in}{2.543791in}}%
\pgfpathlineto{\pgfqpoint{4.943874in}{2.559845in}}%
\pgfpathlineto{\pgfqpoint{4.958794in}{2.576089in}}%
\pgfpathlineto{\pgfqpoint{4.973736in}{2.592522in}}%
\pgfpathlineto{\pgfqpoint{4.981839in}{2.609320in}}%
\pgfpathlineto{\pgfqpoint{4.989936in}{2.625938in}}%
\pgfpathlineto{\pgfqpoint{4.998026in}{2.642375in}}%
\pgfpathlineto{\pgfqpoint{5.006109in}{2.658628in}}%
\pgfpathlineto{\pgfqpoint{4.991156in}{2.641924in}}%
\pgfpathlineto{\pgfqpoint{4.976226in}{2.625410in}}%
\pgfpathlineto{\pgfqpoint{4.961317in}{2.609085in}}%
\pgfpathlineto{\pgfqpoint{4.946429in}{2.592950in}}%
\pgfpathlineto{\pgfqpoint{4.938356in}{2.576954in}}%
\pgfpathlineto{\pgfqpoint{4.930276in}{2.560783in}}%
\pgfpathlineto{\pgfqpoint{4.922190in}{2.544439in}}%
\pgfpathlineto{\pgfqpoint{4.914097in}{2.527925in}}%
\pgfpathclose%
\pgfusepath{fill}%
\end{pgfscope}%
\begin{pgfscope}%
\pgfpathrectangle{\pgfqpoint{1.150000in}{0.150000in}}{\pgfqpoint{5.700000in}{5.700000in}}%
\pgfusepath{clip}%
\pgfsetbuttcap%
\pgfsetroundjoin%
\definecolor{currentfill}{rgb}{0.141935,0.526453,0.555991}%
\pgfsetfillcolor{currentfill}%
\pgfsetfillopacity{0.800000}%
\pgfsetlinewidth{0.000000pt}%
\definecolor{currentstroke}{rgb}{0.000000,0.000000,0.000000}%
\pgfsetstrokecolor{currentstroke}%
\pgfsetdash{}{0pt}%
\pgfpathmoveto{\pgfqpoint{4.757358in}{2.259871in}}%
\pgfpathlineto{\pgfqpoint{4.772121in}{2.273973in}}%
\pgfpathlineto{\pgfqpoint{4.786904in}{2.288261in}}%
\pgfpathlineto{\pgfqpoint{4.801707in}{2.302734in}}%
\pgfpathlineto{\pgfqpoint{4.816529in}{2.317394in}}%
\pgfpathlineto{\pgfqpoint{4.824689in}{2.335746in}}%
\pgfpathlineto{\pgfqpoint{4.832845in}{2.353962in}}%
\pgfpathlineto{\pgfqpoint{4.840995in}{2.372038in}}%
\pgfpathlineto{\pgfqpoint{4.849140in}{2.389971in}}%
\pgfpathlineto{\pgfqpoint{4.834306in}{2.374937in}}%
\pgfpathlineto{\pgfqpoint{4.819491in}{2.360090in}}%
\pgfpathlineto{\pgfqpoint{4.804697in}{2.345430in}}%
\pgfpathlineto{\pgfqpoint{4.789922in}{2.330956in}}%
\pgfpathlineto{\pgfqpoint{4.781789in}{2.313383in}}%
\pgfpathlineto{\pgfqpoint{4.773650in}{2.295676in}}%
\pgfpathlineto{\pgfqpoint{4.765507in}{2.277838in}}%
\pgfpathlineto{\pgfqpoint{4.757358in}{2.259871in}}%
\pgfpathclose%
\pgfusepath{fill}%
\end{pgfscope}%
\begin{pgfscope}%
\pgfpathrectangle{\pgfqpoint{1.150000in}{0.150000in}}{\pgfqpoint{5.700000in}{5.700000in}}%
\pgfusepath{clip}%
\pgfsetbuttcap%
\pgfsetroundjoin%
\definecolor{currentfill}{rgb}{0.175841,0.441290,0.557685}%
\pgfsetfillcolor{currentfill}%
\pgfsetfillopacity{0.800000}%
\pgfsetlinewidth{0.000000pt}%
\definecolor{currentstroke}{rgb}{0.000000,0.000000,0.000000}%
\pgfsetstrokecolor{currentstroke}%
\pgfsetdash{}{0pt}%
\pgfpathmoveto{\pgfqpoint{4.600504in}{1.986319in}}%
\pgfpathlineto{\pgfqpoint{4.615158in}{1.998370in}}%
\pgfpathlineto{\pgfqpoint{4.629829in}{2.010603in}}%
\pgfpathlineto{\pgfqpoint{4.644518in}{2.023018in}}%
\pgfpathlineto{\pgfqpoint{4.659226in}{2.035617in}}%
\pgfpathlineto{\pgfqpoint{4.667426in}{2.054830in}}%
\pgfpathlineto{\pgfqpoint{4.675623in}{2.073961in}}%
\pgfpathlineto{\pgfqpoint{4.683816in}{2.093006in}}%
\pgfpathlineto{\pgfqpoint{4.692004in}{2.111961in}}%
\pgfpathlineto{\pgfqpoint{4.677284in}{2.098890in}}%
\pgfpathlineto{\pgfqpoint{4.662583in}{2.086002in}}%
\pgfpathlineto{\pgfqpoint{4.647900in}{2.073298in}}%
\pgfpathlineto{\pgfqpoint{4.633235in}{2.060778in}}%
\pgfpathlineto{\pgfqpoint{4.625058in}{2.042282in}}%
\pgfpathlineto{\pgfqpoint{4.616878in}{2.023704in}}%
\pgfpathlineto{\pgfqpoint{4.608693in}{2.005048in}}%
\pgfpathlineto{\pgfqpoint{4.600504in}{1.986319in}}%
\pgfpathclose%
\pgfusepath{fill}%
\end{pgfscope}%
\begin{pgfscope}%
\pgfpathrectangle{\pgfqpoint{1.150000in}{0.150000in}}{\pgfqpoint{5.700000in}{5.700000in}}%
\pgfusepath{clip}%
\pgfsetbuttcap%
\pgfsetroundjoin%
\definecolor{currentfill}{rgb}{0.216210,0.351535,0.550627}%
\pgfsetfillcolor{currentfill}%
\pgfsetfillopacity{0.800000}%
\pgfsetlinewidth{0.000000pt}%
\definecolor{currentstroke}{rgb}{0.000000,0.000000,0.000000}%
\pgfsetstrokecolor{currentstroke}%
\pgfsetdash{}{0pt}%
\pgfpathmoveto{\pgfqpoint{4.443656in}{1.717537in}}%
\pgfpathlineto{\pgfqpoint{4.458210in}{1.727261in}}%
\pgfpathlineto{\pgfqpoint{4.472781in}{1.737165in}}%
\pgfpathlineto{\pgfqpoint{4.487367in}{1.747247in}}%
\pgfpathlineto{\pgfqpoint{4.501970in}{1.757510in}}%
\pgfpathlineto{\pgfqpoint{4.510199in}{1.776757in}}%
\pgfpathlineto{\pgfqpoint{4.518425in}{1.795989in}}%
\pgfpathlineto{\pgfqpoint{4.526648in}{1.815199in}}%
\pgfpathlineto{\pgfqpoint{4.534868in}{1.834383in}}%
\pgfpathlineto{\pgfqpoint{4.520254in}{1.823553in}}%
\pgfpathlineto{\pgfqpoint{4.505657in}{1.812904in}}%
\pgfpathlineto{\pgfqpoint{4.491077in}{1.802434in}}%
\pgfpathlineto{\pgfqpoint{4.476513in}{1.792145in}}%
\pgfpathlineto{\pgfqpoint{4.468304in}{1.773515in}}%
\pgfpathlineto{\pgfqpoint{4.460091in}{1.754867in}}%
\pgfpathlineto{\pgfqpoint{4.451875in}{1.736206in}}%
\pgfpathlineto{\pgfqpoint{4.443656in}{1.717537in}}%
\pgfpathclose%
\pgfusepath{fill}%
\end{pgfscope}%
\begin{pgfscope}%
\pgfpathrectangle{\pgfqpoint{1.150000in}{0.150000in}}{\pgfqpoint{5.700000in}{5.700000in}}%
\pgfusepath{clip}%
\pgfsetbuttcap%
\pgfsetroundjoin%
\definecolor{currentfill}{rgb}{0.146616,0.673050,0.508936}%
\pgfsetfillcolor{currentfill}%
\pgfsetfillopacity{0.800000}%
\pgfsetlinewidth{0.000000pt}%
\definecolor{currentstroke}{rgb}{0.000000,0.000000,0.000000}%
\pgfsetstrokecolor{currentstroke}%
\pgfsetdash{}{0pt}%
\pgfpathmoveto{\pgfqpoint{5.038370in}{2.721758in}}%
\pgfpathlineto{\pgfqpoint{5.053354in}{2.738888in}}%
\pgfpathlineto{\pgfqpoint{5.068359in}{2.756208in}}%
\pgfpathlineto{\pgfqpoint{5.083388in}{2.773720in}}%
\pgfpathlineto{\pgfqpoint{5.098439in}{2.791423in}}%
\pgfpathlineto{\pgfqpoint{5.106495in}{2.806939in}}%
\pgfpathlineto{\pgfqpoint{5.114543in}{2.822252in}}%
\pgfpathlineto{\pgfqpoint{5.122584in}{2.837358in}}%
\pgfpathlineto{\pgfqpoint{5.130616in}{2.852257in}}%
\pgfpathlineto{\pgfqpoint{5.115556in}{2.834353in}}%
\pgfpathlineto{\pgfqpoint{5.100519in}{2.816642in}}%
\pgfpathlineto{\pgfqpoint{5.085505in}{2.799122in}}%
\pgfpathlineto{\pgfqpoint{5.070513in}{2.781793in}}%
\pgfpathlineto{\pgfqpoint{5.062488in}{2.767080in}}%
\pgfpathlineto{\pgfqpoint{5.054457in}{2.752169in}}%
\pgfpathlineto{\pgfqpoint{5.046417in}{2.737061in}}%
\pgfpathlineto{\pgfqpoint{5.038370in}{2.721758in}}%
\pgfpathclose%
\pgfusepath{fill}%
\end{pgfscope}%
\begin{pgfscope}%
\pgfpathrectangle{\pgfqpoint{1.150000in}{0.150000in}}{\pgfqpoint{5.700000in}{5.700000in}}%
\pgfusepath{clip}%
\pgfsetbuttcap%
\pgfsetroundjoin%
\definecolor{currentfill}{rgb}{0.276194,0.190074,0.493001}%
\pgfsetfillcolor{currentfill}%
\pgfsetfillopacity{0.800000}%
\pgfsetlinewidth{0.000000pt}%
\definecolor{currentstroke}{rgb}{0.000000,0.000000,0.000000}%
\pgfsetstrokecolor{currentstroke}%
\pgfsetdash{}{0pt}%
\pgfpathmoveto{\pgfqpoint{4.162985in}{1.306165in}}%
\pgfpathlineto{\pgfqpoint{4.177397in}{1.311194in}}%
\pgfpathlineto{\pgfqpoint{4.191821in}{1.316398in}}%
\pgfpathlineto{\pgfqpoint{4.206258in}{1.321777in}}%
\pgfpathlineto{\pgfqpoint{4.220708in}{1.327330in}}%
\pgfpathlineto{\pgfqpoint{4.228984in}{1.344202in}}%
\pgfpathlineto{\pgfqpoint{4.237256in}{1.361195in}}%
\pgfpathlineto{\pgfqpoint{4.245524in}{1.378302in}}%
\pgfpathlineto{\pgfqpoint{4.253790in}{1.395518in}}%
\pgfpathlineto{\pgfqpoint{4.239337in}{1.389247in}}%
\pgfpathlineto{\pgfqpoint{4.224897in}{1.383152in}}%
\pgfpathlineto{\pgfqpoint{4.210470in}{1.377232in}}%
\pgfpathlineto{\pgfqpoint{4.196057in}{1.371487in}}%
\pgfpathlineto{\pgfqpoint{4.187794in}{1.354977in}}%
\pgfpathlineto{\pgfqpoint{4.179528in}{1.338581in}}%
\pgfpathlineto{\pgfqpoint{4.171259in}{1.322308in}}%
\pgfpathlineto{\pgfqpoint{4.162985in}{1.306165in}}%
\pgfpathclose%
\pgfusepath{fill}%
\end{pgfscope}%
\begin{pgfscope}%
\pgfpathrectangle{\pgfqpoint{1.150000in}{0.150000in}}{\pgfqpoint{5.700000in}{5.700000in}}%
\pgfusepath{clip}%
\pgfsetbuttcap%
\pgfsetroundjoin%
\definecolor{currentfill}{rgb}{0.282656,0.100196,0.422160}%
\pgfsetfillcolor{currentfill}%
\pgfsetfillopacity{0.800000}%
\pgfsetlinewidth{0.000000pt}%
\definecolor{currentstroke}{rgb}{0.000000,0.000000,0.000000}%
\pgfsetstrokecolor{currentstroke}%
\pgfsetdash{}{0pt}%
\pgfpathmoveto{\pgfqpoint{3.948430in}{1.109720in}}%
\pgfpathlineto{\pgfqpoint{3.962765in}{1.111038in}}%
\pgfpathlineto{\pgfqpoint{3.977109in}{1.112528in}}%
\pgfpathlineto{\pgfqpoint{3.991463in}{1.114192in}}%
\pgfpathlineto{\pgfqpoint{4.005826in}{1.116029in}}%
\pgfpathlineto{\pgfqpoint{4.014156in}{1.129244in}}%
\pgfpathlineto{\pgfqpoint{4.022481in}{1.142684in}}%
\pgfpathlineto{\pgfqpoint{4.030800in}{1.156342in}}%
\pgfpathlineto{\pgfqpoint{4.039114in}{1.170210in}}%
\pgfpathlineto{\pgfqpoint{4.024756in}{1.167569in}}%
\pgfpathlineto{\pgfqpoint{4.010409in}{1.165102in}}%
\pgfpathlineto{\pgfqpoint{3.996072in}{1.162808in}}%
\pgfpathlineto{\pgfqpoint{3.981745in}{1.160688in}}%
\pgfpathlineto{\pgfqpoint{3.973424in}{1.147612in}}%
\pgfpathlineto{\pgfqpoint{3.965099in}{1.134754in}}%
\pgfpathlineto{\pgfqpoint{3.956767in}{1.122121in}}%
\pgfpathlineto{\pgfqpoint{3.948430in}{1.109720in}}%
\pgfpathclose%
\pgfusepath{fill}%
\end{pgfscope}%
\begin{pgfscope}%
\pgfpathrectangle{\pgfqpoint{1.150000in}{0.150000in}}{\pgfqpoint{5.700000in}{5.700000in}}%
\pgfusepath{clip}%
\pgfsetbuttcap%
\pgfsetroundjoin%
\definecolor{currentfill}{rgb}{0.257322,0.256130,0.526563}%
\pgfsetfillcolor{currentfill}%
\pgfsetfillopacity{0.800000}%
\pgfsetlinewidth{0.000000pt}%
\definecolor{currentstroke}{rgb}{0.000000,0.000000,0.000000}%
\pgfsetstrokecolor{currentstroke}%
\pgfsetdash{}{0pt}%
\pgfpathmoveto{\pgfqpoint{4.286818in}{1.465324in}}%
\pgfpathlineto{\pgfqpoint{4.301290in}{1.472458in}}%
\pgfpathlineto{\pgfqpoint{4.315775in}{1.479768in}}%
\pgfpathlineto{\pgfqpoint{4.330275in}{1.487254in}}%
\pgfpathlineto{\pgfqpoint{4.344789in}{1.494916in}}%
\pgfpathlineto{\pgfqpoint{4.353044in}{1.513241in}}%
\pgfpathlineto{\pgfqpoint{4.361297in}{1.531628in}}%
\pgfpathlineto{\pgfqpoint{4.369546in}{1.550069in}}%
\pgfpathlineto{\pgfqpoint{4.377793in}{1.568560in}}%
\pgfpathlineto{\pgfqpoint{4.363271in}{1.560238in}}%
\pgfpathlineto{\pgfqpoint{4.348765in}{1.552094in}}%
\pgfpathlineto{\pgfqpoint{4.334273in}{1.544127in}}%
\pgfpathlineto{\pgfqpoint{4.319795in}{1.536337in}}%
\pgfpathlineto{\pgfqpoint{4.311556in}{1.518493in}}%
\pgfpathlineto{\pgfqpoint{4.303313in}{1.500705in}}%
\pgfpathlineto{\pgfqpoint{4.295067in}{1.482980in}}%
\pgfpathlineto{\pgfqpoint{4.286818in}{1.465324in}}%
\pgfpathclose%
\pgfusepath{fill}%
\end{pgfscope}%
\begin{pgfscope}%
\pgfpathrectangle{\pgfqpoint{1.150000in}{0.150000in}}{\pgfqpoint{5.700000in}{5.700000in}}%
\pgfusepath{clip}%
\pgfsetbuttcap%
\pgfsetroundjoin%
\definecolor{currentfill}{rgb}{0.595839,0.848717,0.243329}%
\pgfsetfillcolor{currentfill}%
\pgfsetfillopacity{0.800000}%
\pgfsetlinewidth{0.000000pt}%
\definecolor{currentstroke}{rgb}{0.000000,0.000000,0.000000}%
\pgfsetstrokecolor{currentstroke}%
\pgfsetdash{}{0pt}%
\pgfpathmoveto{\pgfqpoint{5.534644in}{3.416498in}}%
\pgfpathlineto{\pgfqpoint{5.550043in}{3.437307in}}%
\pgfpathlineto{\pgfqpoint{5.565469in}{3.458313in}}%
\pgfpathlineto{\pgfqpoint{5.580921in}{3.479516in}}%
\pgfpathlineto{\pgfqpoint{5.588662in}{3.487857in}}%
\pgfpathlineto{\pgfqpoint{5.596390in}{3.495952in}}%
\pgfpathlineto{\pgfqpoint{5.604105in}{3.503806in}}%
\pgfpathlineto{\pgfqpoint{5.611808in}{3.511417in}}%
\pgfpathlineto{\pgfqpoint{5.596361in}{3.490312in}}%
\pgfpathlineto{\pgfqpoint{5.580940in}{3.469405in}}%
\pgfpathlineto{\pgfqpoint{5.565546in}{3.448695in}}%
\pgfpathlineto{\pgfqpoint{5.557839in}{3.440998in}}%
\pgfpathlineto{\pgfqpoint{5.550119in}{3.433068in}}%
\pgfpathlineto{\pgfqpoint{5.542388in}{3.424902in}}%
\pgfpathlineto{\pgfqpoint{5.534644in}{3.416498in}}%
\pgfpathclose%
\pgfusepath{fill}%
\end{pgfscope}%
\begin{pgfscope}%
\pgfpathrectangle{\pgfqpoint{1.150000in}{0.150000in}}{\pgfqpoint{5.700000in}{5.700000in}}%
\pgfusepath{clip}%
\pgfsetbuttcap%
\pgfsetroundjoin%
\definecolor{currentfill}{rgb}{0.120565,0.596422,0.543611}%
\pgfsetfillcolor{currentfill}%
\pgfsetfillopacity{0.800000}%
\pgfsetlinewidth{0.000000pt}%
\definecolor{currentstroke}{rgb}{0.000000,0.000000,0.000000}%
\pgfsetstrokecolor{currentstroke}%
\pgfsetdash{}{0pt}%
\pgfpathmoveto{\pgfqpoint{4.881666in}{2.460210in}}%
\pgfpathlineto{\pgfqpoint{4.896532in}{2.475771in}}%
\pgfpathlineto{\pgfqpoint{4.911420in}{2.491520in}}%
\pgfpathlineto{\pgfqpoint{4.926329in}{2.507458in}}%
\pgfpathlineto{\pgfqpoint{4.941259in}{2.523584in}}%
\pgfpathlineto{\pgfqpoint{4.949387in}{2.541075in}}%
\pgfpathlineto{\pgfqpoint{4.957510in}{2.558397in}}%
\pgfpathlineto{\pgfqpoint{4.965626in}{2.575547in}}%
\pgfpathlineto{\pgfqpoint{4.973736in}{2.592522in}}%
\pgfpathlineto{\pgfqpoint{4.958794in}{2.576089in}}%
\pgfpathlineto{\pgfqpoint{4.943874in}{2.559845in}}%
\pgfpathlineto{\pgfqpoint{4.928975in}{2.543791in}}%
\pgfpathlineto{\pgfqpoint{4.914097in}{2.527925in}}%
\pgfpathlineto{\pgfqpoint{4.905999in}{2.511242in}}%
\pgfpathlineto{\pgfqpoint{4.897894in}{2.494394in}}%
\pgfpathlineto{\pgfqpoint{4.889783in}{2.477382in}}%
\pgfpathlineto{\pgfqpoint{4.881666in}{2.460210in}}%
\pgfpathclose%
\pgfusepath{fill}%
\end{pgfscope}%
\begin{pgfscope}%
\pgfpathrectangle{\pgfqpoint{1.150000in}{0.150000in}}{\pgfqpoint{5.700000in}{5.700000in}}%
\pgfusepath{clip}%
\pgfsetbuttcap%
\pgfsetroundjoin%
\definecolor{currentfill}{rgb}{0.185556,0.418570,0.556753}%
\pgfsetfillcolor{currentfill}%
\pgfsetfillopacity{0.800000}%
\pgfsetlinewidth{0.000000pt}%
\definecolor{currentstroke}{rgb}{0.000000,0.000000,0.000000}%
\pgfsetstrokecolor{currentstroke}%
\pgfsetdash{}{0pt}%
\pgfpathmoveto{\pgfqpoint{4.567714in}{1.910757in}}%
\pgfpathlineto{\pgfqpoint{4.582356in}{1.922304in}}%
\pgfpathlineto{\pgfqpoint{4.597015in}{1.934033in}}%
\pgfpathlineto{\pgfqpoint{4.611692in}{1.945943in}}%
\pgfpathlineto{\pgfqpoint{4.626387in}{1.958035in}}%
\pgfpathlineto{\pgfqpoint{4.634602in}{1.977531in}}%
\pgfpathlineto{\pgfqpoint{4.642814in}{1.996963in}}%
\pgfpathlineto{\pgfqpoint{4.651022in}{2.016326in}}%
\pgfpathlineto{\pgfqpoint{4.659226in}{2.035617in}}%
\pgfpathlineto{\pgfqpoint{4.644518in}{2.023018in}}%
\pgfpathlineto{\pgfqpoint{4.629829in}{2.010603in}}%
\pgfpathlineto{\pgfqpoint{4.615158in}{1.998370in}}%
\pgfpathlineto{\pgfqpoint{4.600504in}{1.986319in}}%
\pgfpathlineto{\pgfqpoint{4.592312in}{1.967521in}}%
\pgfpathlineto{\pgfqpoint{4.584116in}{1.948659in}}%
\pgfpathlineto{\pgfqpoint{4.575917in}{1.929736in}}%
\pgfpathlineto{\pgfqpoint{4.567714in}{1.910757in}}%
\pgfpathclose%
\pgfusepath{fill}%
\end{pgfscope}%
\begin{pgfscope}%
\pgfpathrectangle{\pgfqpoint{1.150000in}{0.150000in}}{\pgfqpoint{5.700000in}{5.700000in}}%
\pgfusepath{clip}%
\pgfsetbuttcap%
\pgfsetroundjoin%
\definecolor{currentfill}{rgb}{0.149039,0.508051,0.557250}%
\pgfsetfillcolor{currentfill}%
\pgfsetfillopacity{0.800000}%
\pgfsetlinewidth{0.000000pt}%
\definecolor{currentstroke}{rgb}{0.000000,0.000000,0.000000}%
\pgfsetstrokecolor{currentstroke}%
\pgfsetdash{}{0pt}%
\pgfpathmoveto{\pgfqpoint{4.724717in}{2.186796in}}%
\pgfpathlineto{\pgfqpoint{4.739468in}{2.200492in}}%
\pgfpathlineto{\pgfqpoint{4.754238in}{2.214373in}}%
\pgfpathlineto{\pgfqpoint{4.769028in}{2.228438in}}%
\pgfpathlineto{\pgfqpoint{4.783837in}{2.242689in}}%
\pgfpathlineto{\pgfqpoint{4.792017in}{2.261553in}}%
\pgfpathlineto{\pgfqpoint{4.800193in}{2.280294in}}%
\pgfpathlineto{\pgfqpoint{4.808363in}{2.298909in}}%
\pgfpathlineto{\pgfqpoint{4.816529in}{2.317394in}}%
\pgfpathlineto{\pgfqpoint{4.801707in}{2.302734in}}%
\pgfpathlineto{\pgfqpoint{4.786904in}{2.288261in}}%
\pgfpathlineto{\pgfqpoint{4.772121in}{2.273973in}}%
\pgfpathlineto{\pgfqpoint{4.757358in}{2.259871in}}%
\pgfpathlineto{\pgfqpoint{4.749205in}{2.241781in}}%
\pgfpathlineto{\pgfqpoint{4.741047in}{2.223569in}}%
\pgfpathlineto{\pgfqpoint{4.732884in}{2.205240in}}%
\pgfpathlineto{\pgfqpoint{4.724717in}{2.186796in}}%
\pgfpathclose%
\pgfusepath{fill}%
\end{pgfscope}%
\begin{pgfscope}%
\pgfpathrectangle{\pgfqpoint{1.150000in}{0.150000in}}{\pgfqpoint{5.700000in}{5.700000in}}%
\pgfusepath{clip}%
\pgfsetbuttcap%
\pgfsetroundjoin%
\definecolor{currentfill}{rgb}{0.232815,0.732247,0.459277}%
\pgfsetfillcolor{currentfill}%
\pgfsetfillopacity{0.800000}%
\pgfsetlinewidth{0.000000pt}%
\definecolor{currentstroke}{rgb}{0.000000,0.000000,0.000000}%
\pgfsetstrokecolor{currentstroke}%
\pgfsetdash{}{0pt}%
\pgfpathmoveto{\pgfqpoint{5.162662in}{2.909752in}}%
\pgfpathlineto{\pgfqpoint{5.177752in}{2.928012in}}%
\pgfpathlineto{\pgfqpoint{5.192866in}{2.946465in}}%
\pgfpathlineto{\pgfqpoint{5.208003in}{2.965111in}}%
\pgfpathlineto{\pgfqpoint{5.223164in}{2.983951in}}%
\pgfpathlineto{\pgfqpoint{5.231161in}{2.997935in}}%
\pgfpathlineto{\pgfqpoint{5.239149in}{3.011695in}}%
\pgfpathlineto{\pgfqpoint{5.247127in}{3.025231in}}%
\pgfpathlineto{\pgfqpoint{5.255097in}{3.038542in}}%
\pgfpathlineto{\pgfqpoint{5.239929in}{3.019574in}}%
\pgfpathlineto{\pgfqpoint{5.224786in}{3.000800in}}%
\pgfpathlineto{\pgfqpoint{5.209666in}{2.982220in}}%
\pgfpathlineto{\pgfqpoint{5.194569in}{2.963832in}}%
\pgfpathlineto{\pgfqpoint{5.186606in}{2.950635in}}%
\pgfpathlineto{\pgfqpoint{5.178633in}{2.937222in}}%
\pgfpathlineto{\pgfqpoint{5.170652in}{2.923594in}}%
\pgfpathlineto{\pgfqpoint{5.162662in}{2.909752in}}%
\pgfpathclose%
\pgfusepath{fill}%
\end{pgfscope}%
\begin{pgfscope}%
\pgfpathrectangle{\pgfqpoint{1.150000in}{0.150000in}}{\pgfqpoint{5.700000in}{5.700000in}}%
\pgfusepath{clip}%
\pgfsetbuttcap%
\pgfsetroundjoin%
\definecolor{currentfill}{rgb}{0.283072,0.130895,0.449241}%
\pgfsetfillcolor{currentfill}%
\pgfsetfillopacity{0.800000}%
\pgfsetlinewidth{0.000000pt}%
\definecolor{currentstroke}{rgb}{0.000000,0.000000,0.000000}%
\pgfsetstrokecolor{currentstroke}%
\pgfsetdash{}{0pt}%
\pgfpathmoveto{\pgfqpoint{4.039114in}{1.170210in}}%
\pgfpathlineto{\pgfqpoint{4.053483in}{1.173024in}}%
\pgfpathlineto{\pgfqpoint{4.067862in}{1.176011in}}%
\pgfpathlineto{\pgfqpoint{4.082253in}{1.179172in}}%
\pgfpathlineto{\pgfqpoint{4.096654in}{1.182506in}}%
\pgfpathlineto{\pgfqpoint{4.104960in}{1.197362in}}%
\pgfpathlineto{\pgfqpoint{4.113262in}{1.212404in}}%
\pgfpathlineto{\pgfqpoint{4.121559in}{1.227626in}}%
\pgfpathlineto{\pgfqpoint{4.129853in}{1.243019in}}%
\pgfpathlineto{\pgfqpoint{4.115453in}{1.238909in}}%
\pgfpathlineto{\pgfqpoint{4.101065in}{1.234973in}}%
\pgfpathlineto{\pgfqpoint{4.086688in}{1.231211in}}%
\pgfpathlineto{\pgfqpoint{4.072323in}{1.227624in}}%
\pgfpathlineto{\pgfqpoint{4.064028in}{1.212993in}}%
\pgfpathlineto{\pgfqpoint{4.055728in}{1.198543in}}%
\pgfpathlineto{\pgfqpoint{4.047423in}{1.184279in}}%
\pgfpathlineto{\pgfqpoint{4.039114in}{1.170210in}}%
\pgfpathclose%
\pgfusepath{fill}%
\end{pgfscope}%
\begin{pgfscope}%
\pgfpathrectangle{\pgfqpoint{1.150000in}{0.150000in}}{\pgfqpoint{5.700000in}{5.700000in}}%
\pgfusepath{clip}%
\pgfsetbuttcap%
\pgfsetroundjoin%
\definecolor{currentfill}{rgb}{0.227802,0.326594,0.546532}%
\pgfsetfillcolor{currentfill}%
\pgfsetfillopacity{0.800000}%
\pgfsetlinewidth{0.000000pt}%
\definecolor{currentstroke}{rgb}{0.000000,0.000000,0.000000}%
\pgfsetstrokecolor{currentstroke}%
\pgfsetdash{}{0pt}%
\pgfpathmoveto{\pgfqpoint{4.410749in}{1.642891in}}%
\pgfpathlineto{\pgfqpoint{4.425294in}{1.652018in}}%
\pgfpathlineto{\pgfqpoint{4.439854in}{1.661324in}}%
\pgfpathlineto{\pgfqpoint{4.454431in}{1.670808in}}%
\pgfpathlineto{\pgfqpoint{4.469023in}{1.680471in}}%
\pgfpathlineto{\pgfqpoint{4.477264in}{1.699727in}}%
\pgfpathlineto{\pgfqpoint{4.485502in}{1.718989in}}%
\pgfpathlineto{\pgfqpoint{4.493737in}{1.738252in}}%
\pgfpathlineto{\pgfqpoint{4.501970in}{1.757510in}}%
\pgfpathlineto{\pgfqpoint{4.487367in}{1.747247in}}%
\pgfpathlineto{\pgfqpoint{4.472781in}{1.737165in}}%
\pgfpathlineto{\pgfqpoint{4.458210in}{1.727261in}}%
\pgfpathlineto{\pgfqpoint{4.443656in}{1.717537in}}%
\pgfpathlineto{\pgfqpoint{4.435434in}{1.698866in}}%
\pgfpathlineto{\pgfqpoint{4.427209in}{1.680197in}}%
\pgfpathlineto{\pgfqpoint{4.418980in}{1.661537in}}%
\pgfpathlineto{\pgfqpoint{4.410749in}{1.642891in}}%
\pgfpathclose%
\pgfusepath{fill}%
\end{pgfscope}%
\begin{pgfscope}%
\pgfpathrectangle{\pgfqpoint{1.150000in}{0.150000in}}{\pgfqpoint{5.700000in}{5.700000in}}%
\pgfusepath{clip}%
\pgfsetbuttcap%
\pgfsetroundjoin%
\definecolor{currentfill}{rgb}{0.352360,0.783011,0.392636}%
\pgfsetfillcolor{currentfill}%
\pgfsetfillopacity{0.800000}%
\pgfsetlinewidth{0.000000pt}%
\definecolor{currentstroke}{rgb}{0.000000,0.000000,0.000000}%
\pgfsetstrokecolor{currentstroke}%
\pgfsetdash{}{0pt}%
\pgfpathmoveto{\pgfqpoint{5.286878in}{3.089519in}}%
\pgfpathlineto{\pgfqpoint{5.302074in}{3.108772in}}%
\pgfpathlineto{\pgfqpoint{5.317295in}{3.128219in}}%
\pgfpathlineto{\pgfqpoint{5.332540in}{3.147862in}}%
\pgfpathlineto{\pgfqpoint{5.347810in}{3.167700in}}%
\pgfpathlineto{\pgfqpoint{5.355735in}{3.179944in}}%
\pgfpathlineto{\pgfqpoint{5.363649in}{3.191950in}}%
\pgfpathlineto{\pgfqpoint{5.371553in}{3.203719in}}%
\pgfpathlineto{\pgfqpoint{5.379446in}{3.215250in}}%
\pgfpathlineto{\pgfqpoint{5.364173in}{3.195358in}}%
\pgfpathlineto{\pgfqpoint{5.348924in}{3.175662in}}%
\pgfpathlineto{\pgfqpoint{5.333700in}{3.156161in}}%
\pgfpathlineto{\pgfqpoint{5.318501in}{3.136854in}}%
\pgfpathlineto{\pgfqpoint{5.310610in}{3.125363in}}%
\pgfpathlineto{\pgfqpoint{5.302709in}{3.113643in}}%
\pgfpathlineto{\pgfqpoint{5.294798in}{3.101695in}}%
\pgfpathlineto{\pgfqpoint{5.286878in}{3.089519in}}%
\pgfpathclose%
\pgfusepath{fill}%
\end{pgfscope}%
\begin{pgfscope}%
\pgfpathrectangle{\pgfqpoint{1.150000in}{0.150000in}}{\pgfqpoint{5.700000in}{5.700000in}}%
\pgfusepath{clip}%
\pgfsetbuttcap%
\pgfsetroundjoin%
\definecolor{currentfill}{rgb}{0.477504,0.821444,0.318195}%
\pgfsetfillcolor{currentfill}%
\pgfsetfillopacity{0.800000}%
\pgfsetlinewidth{0.000000pt}%
\definecolor{currentstroke}{rgb}{0.000000,0.000000,0.000000}%
\pgfsetstrokecolor{currentstroke}%
\pgfsetdash{}{0pt}%
\pgfpathmoveto{\pgfqpoint{5.410910in}{3.259004in}}%
\pgfpathlineto{\pgfqpoint{5.426210in}{3.279108in}}%
\pgfpathlineto{\pgfqpoint{5.441535in}{3.299408in}}%
\pgfpathlineto{\pgfqpoint{5.456886in}{3.319904in}}%
\pgfpathlineto{\pgfqpoint{5.472263in}{3.340598in}}%
\pgfpathlineto{\pgfqpoint{5.480102in}{3.350938in}}%
\pgfpathlineto{\pgfqpoint{5.487929in}{3.361032in}}%
\pgfpathlineto{\pgfqpoint{5.495744in}{3.370882in}}%
\pgfpathlineto{\pgfqpoint{5.503548in}{3.380489in}}%
\pgfpathlineto{\pgfqpoint{5.488172in}{3.359818in}}%
\pgfpathlineto{\pgfqpoint{5.472822in}{3.339343in}}%
\pgfpathlineto{\pgfqpoint{5.457497in}{3.319065in}}%
\pgfpathlineto{\pgfqpoint{5.442198in}{3.298983in}}%
\pgfpathlineto{\pgfqpoint{5.434393in}{3.289340in}}%
\pgfpathlineto{\pgfqpoint{5.426577in}{3.279463in}}%
\pgfpathlineto{\pgfqpoint{5.418749in}{3.269352in}}%
\pgfpathlineto{\pgfqpoint{5.410910in}{3.259004in}}%
\pgfpathclose%
\pgfusepath{fill}%
\end{pgfscope}%
\begin{pgfscope}%
\pgfpathrectangle{\pgfqpoint{1.150000in}{0.150000in}}{\pgfqpoint{5.700000in}{5.700000in}}%
\pgfusepath{clip}%
\pgfsetbuttcap%
\pgfsetroundjoin%
\definecolor{currentfill}{rgb}{0.265145,0.232956,0.516599}%
\pgfsetfillcolor{currentfill}%
\pgfsetfillopacity{0.800000}%
\pgfsetlinewidth{0.000000pt}%
\definecolor{currentstroke}{rgb}{0.000000,0.000000,0.000000}%
\pgfsetstrokecolor{currentstroke}%
\pgfsetdash{}{0pt}%
\pgfpathmoveto{\pgfqpoint{4.253790in}{1.395518in}}%
\pgfpathlineto{\pgfqpoint{4.268256in}{1.401964in}}%
\pgfpathlineto{\pgfqpoint{4.282736in}{1.408585in}}%
\pgfpathlineto{\pgfqpoint{4.297230in}{1.415382in}}%
\pgfpathlineto{\pgfqpoint{4.311738in}{1.422355in}}%
\pgfpathlineto{\pgfqpoint{4.320005in}{1.440371in}}%
\pgfpathlineto{\pgfqpoint{4.328269in}{1.458475in}}%
\pgfpathlineto{\pgfqpoint{4.336531in}{1.476659in}}%
\pgfpathlineto{\pgfqpoint{4.344789in}{1.494916in}}%
\pgfpathlineto{\pgfqpoint{4.330275in}{1.487254in}}%
\pgfpathlineto{\pgfqpoint{4.315775in}{1.479768in}}%
\pgfpathlineto{\pgfqpoint{4.301290in}{1.472458in}}%
\pgfpathlineto{\pgfqpoint{4.286818in}{1.465324in}}%
\pgfpathlineto{\pgfqpoint{4.278566in}{1.447744in}}%
\pgfpathlineto{\pgfqpoint{4.270310in}{1.430245in}}%
\pgfpathlineto{\pgfqpoint{4.262052in}{1.412834in}}%
\pgfpathlineto{\pgfqpoint{4.253790in}{1.395518in}}%
\pgfpathclose%
\pgfusepath{fill}%
\end{pgfscope}%
\begin{pgfscope}%
\pgfpathrectangle{\pgfqpoint{1.150000in}{0.150000in}}{\pgfqpoint{5.700000in}{5.700000in}}%
\pgfusepath{clip}%
\pgfsetbuttcap%
\pgfsetroundjoin%
\definecolor{currentfill}{rgb}{0.137339,0.662252,0.515571}%
\pgfsetfillcolor{currentfill}%
\pgfsetfillopacity{0.800000}%
\pgfsetlinewidth{0.000000pt}%
\definecolor{currentstroke}{rgb}{0.000000,0.000000,0.000000}%
\pgfsetstrokecolor{currentstroke}%
\pgfsetdash{}{0pt}%
\pgfpathmoveto{\pgfqpoint{5.006109in}{2.658628in}}%
\pgfpathlineto{\pgfqpoint{5.021083in}{2.675522in}}%
\pgfpathlineto{\pgfqpoint{5.036079in}{2.692606in}}%
\pgfpathlineto{\pgfqpoint{5.051097in}{2.709881in}}%
\pgfpathlineto{\pgfqpoint{5.066138in}{2.727348in}}%
\pgfpathlineto{\pgfqpoint{5.074224in}{2.743664in}}%
\pgfpathlineto{\pgfqpoint{5.082303in}{2.759783in}}%
\pgfpathlineto{\pgfqpoint{5.090375in}{2.775704in}}%
\pgfpathlineto{\pgfqpoint{5.098439in}{2.791423in}}%
\pgfpathlineto{\pgfqpoint{5.083388in}{2.773720in}}%
\pgfpathlineto{\pgfqpoint{5.068359in}{2.756208in}}%
\pgfpathlineto{\pgfqpoint{5.053354in}{2.738888in}}%
\pgfpathlineto{\pgfqpoint{5.038370in}{2.721758in}}%
\pgfpathlineto{\pgfqpoint{5.030315in}{2.706261in}}%
\pgfpathlineto{\pgfqpoint{5.022254in}{2.690573in}}%
\pgfpathlineto{\pgfqpoint{5.014185in}{2.674694in}}%
\pgfpathlineto{\pgfqpoint{5.006109in}{2.658628in}}%
\pgfpathclose%
\pgfusepath{fill}%
\end{pgfscope}%
\begin{pgfscope}%
\pgfpathrectangle{\pgfqpoint{1.150000in}{0.150000in}}{\pgfqpoint{5.700000in}{5.700000in}}%
\pgfusepath{clip}%
\pgfsetbuttcap%
\pgfsetroundjoin%
\definecolor{currentfill}{rgb}{0.279574,0.170599,0.479997}%
\pgfsetfillcolor{currentfill}%
\pgfsetfillopacity{0.800000}%
\pgfsetlinewidth{0.000000pt}%
\definecolor{currentstroke}{rgb}{0.000000,0.000000,0.000000}%
\pgfsetstrokecolor{currentstroke}%
\pgfsetdash{}{0pt}%
\pgfpathmoveto{\pgfqpoint{4.129853in}{1.243019in}}%
\pgfpathlineto{\pgfqpoint{4.144264in}{1.247303in}}%
\pgfpathlineto{\pgfqpoint{4.158688in}{1.251760in}}%
\pgfpathlineto{\pgfqpoint{4.173123in}{1.256391in}}%
\pgfpathlineto{\pgfqpoint{4.187571in}{1.261196in}}%
\pgfpathlineto{\pgfqpoint{4.195861in}{1.277513in}}%
\pgfpathlineto{\pgfqpoint{4.204147in}{1.293979in}}%
\pgfpathlineto{\pgfqpoint{4.212429in}{1.310587in}}%
\pgfpathlineto{\pgfqpoint{4.220708in}{1.327330in}}%
\pgfpathlineto{\pgfqpoint{4.206258in}{1.321777in}}%
\pgfpathlineto{\pgfqpoint{4.191821in}{1.316398in}}%
\pgfpathlineto{\pgfqpoint{4.177397in}{1.311194in}}%
\pgfpathlineto{\pgfqpoint{4.162985in}{1.306165in}}%
\pgfpathlineto{\pgfqpoint{4.154708in}{1.290157in}}%
\pgfpathlineto{\pgfqpoint{4.146427in}{1.274292in}}%
\pgfpathlineto{\pgfqpoint{4.138142in}{1.258577in}}%
\pgfpathlineto{\pgfqpoint{4.129853in}{1.243019in}}%
\pgfpathclose%
\pgfusepath{fill}%
\end{pgfscope}%
\begin{pgfscope}%
\pgfpathrectangle{\pgfqpoint{1.150000in}{0.150000in}}{\pgfqpoint{5.700000in}{5.700000in}}%
\pgfusepath{clip}%
\pgfsetbuttcap%
\pgfsetroundjoin%
\definecolor{currentfill}{rgb}{0.194100,0.399323,0.555565}%
\pgfsetfillcolor{currentfill}%
\pgfsetfillopacity{0.800000}%
\pgfsetlinewidth{0.000000pt}%
\definecolor{currentstroke}{rgb}{0.000000,0.000000,0.000000}%
\pgfsetstrokecolor{currentstroke}%
\pgfsetdash{}{0pt}%
\pgfpathmoveto{\pgfqpoint{4.534868in}{1.834383in}}%
\pgfpathlineto{\pgfqpoint{4.549498in}{1.845394in}}%
\pgfpathlineto{\pgfqpoint{4.564146in}{1.856585in}}%
\pgfpathlineto{\pgfqpoint{4.578811in}{1.867958in}}%
\pgfpathlineto{\pgfqpoint{4.593493in}{1.879511in}}%
\pgfpathlineto{\pgfqpoint{4.601721in}{1.899214in}}%
\pgfpathlineto{\pgfqpoint{4.609947in}{1.918872in}}%
\pgfpathlineto{\pgfqpoint{4.618168in}{1.938481in}}%
\pgfpathlineto{\pgfqpoint{4.626387in}{1.958035in}}%
\pgfpathlineto{\pgfqpoint{4.611692in}{1.945943in}}%
\pgfpathlineto{\pgfqpoint{4.597015in}{1.934033in}}%
\pgfpathlineto{\pgfqpoint{4.582356in}{1.922304in}}%
\pgfpathlineto{\pgfqpoint{4.567714in}{1.910757in}}%
\pgfpathlineto{\pgfqpoint{4.559507in}{1.891728in}}%
\pgfpathlineto{\pgfqpoint{4.551297in}{1.872653in}}%
\pgfpathlineto{\pgfqpoint{4.543084in}{1.853536in}}%
\pgfpathlineto{\pgfqpoint{4.534868in}{1.834383in}}%
\pgfpathclose%
\pgfusepath{fill}%
\end{pgfscope}%
\begin{pgfscope}%
\pgfpathrectangle{\pgfqpoint{1.150000in}{0.150000in}}{\pgfqpoint{5.700000in}{5.700000in}}%
\pgfusepath{clip}%
\pgfsetbuttcap%
\pgfsetroundjoin%
\definecolor{currentfill}{rgb}{0.156270,0.489624,0.557936}%
\pgfsetfillcolor{currentfill}%
\pgfsetfillopacity{0.800000}%
\pgfsetlinewidth{0.000000pt}%
\definecolor{currentstroke}{rgb}{0.000000,0.000000,0.000000}%
\pgfsetstrokecolor{currentstroke}%
\pgfsetdash{}{0pt}%
\pgfpathmoveto{\pgfqpoint{4.692004in}{2.111961in}}%
\pgfpathlineto{\pgfqpoint{4.706743in}{2.125216in}}%
\pgfpathlineto{\pgfqpoint{4.721500in}{2.138656in}}%
\pgfpathlineto{\pgfqpoint{4.736277in}{2.152280in}}%
\pgfpathlineto{\pgfqpoint{4.751073in}{2.166088in}}%
\pgfpathlineto{\pgfqpoint{4.759270in}{2.185403in}}%
\pgfpathlineto{\pgfqpoint{4.767464in}{2.204611in}}%
\pgfpathlineto{\pgfqpoint{4.775653in}{2.223707in}}%
\pgfpathlineto{\pgfqpoint{4.783837in}{2.242689in}}%
\pgfpathlineto{\pgfqpoint{4.769028in}{2.228438in}}%
\pgfpathlineto{\pgfqpoint{4.754238in}{2.214373in}}%
\pgfpathlineto{\pgfqpoint{4.739468in}{2.200492in}}%
\pgfpathlineto{\pgfqpoint{4.724717in}{2.186796in}}%
\pgfpathlineto{\pgfqpoint{4.716545in}{2.168243in}}%
\pgfpathlineto{\pgfqpoint{4.708369in}{2.149584in}}%
\pgfpathlineto{\pgfqpoint{4.700189in}{2.130822in}}%
\pgfpathlineto{\pgfqpoint{4.692004in}{2.111961in}}%
\pgfpathclose%
\pgfusepath{fill}%
\end{pgfscope}%
\begin{pgfscope}%
\pgfpathrectangle{\pgfqpoint{1.150000in}{0.150000in}}{\pgfqpoint{5.700000in}{5.700000in}}%
\pgfusepath{clip}%
\pgfsetbuttcap%
\pgfsetroundjoin%
\definecolor{currentfill}{rgb}{0.124395,0.578002,0.548287}%
\pgfsetfillcolor{currentfill}%
\pgfsetfillopacity{0.800000}%
\pgfsetlinewidth{0.000000pt}%
\definecolor{currentstroke}{rgb}{0.000000,0.000000,0.000000}%
\pgfsetstrokecolor{currentstroke}%
\pgfsetdash{}{0pt}%
\pgfpathmoveto{\pgfqpoint{4.849140in}{2.389971in}}%
\pgfpathlineto{\pgfqpoint{4.863995in}{2.405192in}}%
\pgfpathlineto{\pgfqpoint{4.878871in}{2.420600in}}%
\pgfpathlineto{\pgfqpoint{4.893767in}{2.436196in}}%
\pgfpathlineto{\pgfqpoint{4.908684in}{2.451981in}}%
\pgfpathlineto{\pgfqpoint{4.916837in}{2.470122in}}%
\pgfpathlineto{\pgfqpoint{4.924983in}{2.488104in}}%
\pgfpathlineto{\pgfqpoint{4.933124in}{2.505926in}}%
\pgfpathlineto{\pgfqpoint{4.941259in}{2.523584in}}%
\pgfpathlineto{\pgfqpoint{4.926329in}{2.507458in}}%
\pgfpathlineto{\pgfqpoint{4.911420in}{2.491520in}}%
\pgfpathlineto{\pgfqpoint{4.896532in}{2.475771in}}%
\pgfpathlineto{\pgfqpoint{4.881666in}{2.460210in}}%
\pgfpathlineto{\pgfqpoint{4.873543in}{2.442880in}}%
\pgfpathlineto{\pgfqpoint{4.865414in}{2.425395in}}%
\pgfpathlineto{\pgfqpoint{4.857280in}{2.407757in}}%
\pgfpathlineto{\pgfqpoint{4.849140in}{2.389971in}}%
\pgfpathclose%
\pgfusepath{fill}%
\end{pgfscope}%
\begin{pgfscope}%
\pgfpathrectangle{\pgfqpoint{1.150000in}{0.150000in}}{\pgfqpoint{5.700000in}{5.700000in}}%
\pgfusepath{clip}%
\pgfsetbuttcap%
\pgfsetroundjoin%
\definecolor{currentfill}{rgb}{0.237441,0.305202,0.541921}%
\pgfsetfillcolor{currentfill}%
\pgfsetfillopacity{0.800000}%
\pgfsetlinewidth{0.000000pt}%
\definecolor{currentstroke}{rgb}{0.000000,0.000000,0.000000}%
\pgfsetstrokecolor{currentstroke}%
\pgfsetdash{}{0pt}%
\pgfpathmoveto{\pgfqpoint{4.377793in}{1.568560in}}%
\pgfpathlineto{\pgfqpoint{4.392329in}{1.577058in}}%
\pgfpathlineto{\pgfqpoint{4.406881in}{1.585735in}}%
\pgfpathlineto{\pgfqpoint{4.421447in}{1.594589in}}%
\pgfpathlineto{\pgfqpoint{4.436030in}{1.603620in}}%
\pgfpathlineto{\pgfqpoint{4.444282in}{1.622795in}}%
\pgfpathlineto{\pgfqpoint{4.452532in}{1.641999in}}%
\pgfpathlineto{\pgfqpoint{4.460779in}{1.661226in}}%
\pgfpathlineto{\pgfqpoint{4.469023in}{1.680471in}}%
\pgfpathlineto{\pgfqpoint{4.454431in}{1.670808in}}%
\pgfpathlineto{\pgfqpoint{4.439854in}{1.661324in}}%
\pgfpathlineto{\pgfqpoint{4.425294in}{1.652018in}}%
\pgfpathlineto{\pgfqpoint{4.410749in}{1.642891in}}%
\pgfpathlineto{\pgfqpoint{4.402514in}{1.624264in}}%
\pgfpathlineto{\pgfqpoint{4.394277in}{1.605663in}}%
\pgfpathlineto{\pgfqpoint{4.386036in}{1.587093in}}%
\pgfpathlineto{\pgfqpoint{4.377793in}{1.568560in}}%
\pgfpathclose%
\pgfusepath{fill}%
\end{pgfscope}%
\begin{pgfscope}%
\pgfpathrectangle{\pgfqpoint{1.150000in}{0.150000in}}{\pgfqpoint{5.700000in}{5.700000in}}%
\pgfusepath{clip}%
\pgfsetbuttcap%
\pgfsetroundjoin%
\definecolor{currentfill}{rgb}{0.214000,0.722114,0.469588}%
\pgfsetfillcolor{currentfill}%
\pgfsetfillopacity{0.800000}%
\pgfsetlinewidth{0.000000pt}%
\definecolor{currentstroke}{rgb}{0.000000,0.000000,0.000000}%
\pgfsetstrokecolor{currentstroke}%
\pgfsetdash{}{0pt}%
\pgfpathmoveto{\pgfqpoint{5.130616in}{2.852257in}}%
\pgfpathlineto{\pgfqpoint{5.145699in}{2.870353in}}%
\pgfpathlineto{\pgfqpoint{5.160805in}{2.888641in}}%
\pgfpathlineto{\pgfqpoint{5.175934in}{2.907123in}}%
\pgfpathlineto{\pgfqpoint{5.191087in}{2.925798in}}%
\pgfpathlineto{\pgfqpoint{5.199119in}{2.940666in}}%
\pgfpathlineto{\pgfqpoint{5.207143in}{2.955316in}}%
\pgfpathlineto{\pgfqpoint{5.215158in}{2.969744in}}%
\pgfpathlineto{\pgfqpoint{5.223164in}{2.983951in}}%
\pgfpathlineto{\pgfqpoint{5.208003in}{2.965111in}}%
\pgfpathlineto{\pgfqpoint{5.192866in}{2.946465in}}%
\pgfpathlineto{\pgfqpoint{5.177752in}{2.928012in}}%
\pgfpathlineto{\pgfqpoint{5.162662in}{2.909752in}}%
\pgfpathlineto{\pgfqpoint{5.154663in}{2.895696in}}%
\pgfpathlineto{\pgfqpoint{5.146656in}{2.881427in}}%
\pgfpathlineto{\pgfqpoint{5.138640in}{2.866947in}}%
\pgfpathlineto{\pgfqpoint{5.130616in}{2.852257in}}%
\pgfpathclose%
\pgfusepath{fill}%
\end{pgfscope}%
\begin{pgfscope}%
\pgfpathrectangle{\pgfqpoint{1.150000in}{0.150000in}}{\pgfqpoint{5.700000in}{5.700000in}}%
\pgfusepath{clip}%
\pgfsetbuttcap%
\pgfsetroundjoin%
\definecolor{currentfill}{rgb}{0.283197,0.115680,0.436115}%
\pgfsetfillcolor{currentfill}%
\pgfsetfillopacity{0.800000}%
\pgfsetlinewidth{0.000000pt}%
\definecolor{currentstroke}{rgb}{0.000000,0.000000,0.000000}%
\pgfsetstrokecolor{currentstroke}%
\pgfsetdash{}{0pt}%
\pgfpathmoveto{\pgfqpoint{4.005826in}{1.116029in}}%
\pgfpathlineto{\pgfqpoint{4.020200in}{1.118038in}}%
\pgfpathlineto{\pgfqpoint{4.034585in}{1.120220in}}%
\pgfpathlineto{\pgfqpoint{4.048979in}{1.122574in}}%
\pgfpathlineto{\pgfqpoint{4.063385in}{1.125101in}}%
\pgfpathlineto{\pgfqpoint{4.071709in}{1.139134in}}%
\pgfpathlineto{\pgfqpoint{4.080029in}{1.153384in}}%
\pgfpathlineto{\pgfqpoint{4.088344in}{1.167844in}}%
\pgfpathlineto{\pgfqpoint{4.096654in}{1.182506in}}%
\pgfpathlineto{\pgfqpoint{4.082253in}{1.179172in}}%
\pgfpathlineto{\pgfqpoint{4.067862in}{1.176011in}}%
\pgfpathlineto{\pgfqpoint{4.053483in}{1.173024in}}%
\pgfpathlineto{\pgfqpoint{4.039114in}{1.170210in}}%
\pgfpathlineto{\pgfqpoint{4.030800in}{1.156342in}}%
\pgfpathlineto{\pgfqpoint{4.022481in}{1.142684in}}%
\pgfpathlineto{\pgfqpoint{4.014156in}{1.129244in}}%
\pgfpathlineto{\pgfqpoint{4.005826in}{1.116029in}}%
\pgfpathclose%
\pgfusepath{fill}%
\end{pgfscope}%
\begin{pgfscope}%
\pgfpathrectangle{\pgfqpoint{1.150000in}{0.150000in}}{\pgfqpoint{5.700000in}{5.700000in}}%
\pgfusepath{clip}%
\pgfsetbuttcap%
\pgfsetroundjoin%
\definecolor{currentfill}{rgb}{0.585678,0.846661,0.249897}%
\pgfsetfillcolor{currentfill}%
\pgfsetfillopacity{0.800000}%
\pgfsetlinewidth{0.000000pt}%
\definecolor{currentstroke}{rgb}{0.000000,0.000000,0.000000}%
\pgfsetstrokecolor{currentstroke}%
\pgfsetdash{}{0pt}%
\pgfpathmoveto{\pgfqpoint{5.503548in}{3.380489in}}%
\pgfpathlineto{\pgfqpoint{5.518950in}{3.401358in}}%
\pgfpathlineto{\pgfqpoint{5.534378in}{3.422424in}}%
\pgfpathlineto{\pgfqpoint{5.549832in}{3.443689in}}%
\pgfpathlineto{\pgfqpoint{5.557623in}{3.453018in}}%
\pgfpathlineto{\pgfqpoint{5.565402in}{3.462099in}}%
\pgfpathlineto{\pgfqpoint{5.573167in}{3.470931in}}%
\pgfpathlineto{\pgfqpoint{5.580921in}{3.479516in}}%
\pgfpathlineto{\pgfqpoint{5.565469in}{3.458313in}}%
\pgfpathlineto{\pgfqpoint{5.550043in}{3.437307in}}%
\pgfpathlineto{\pgfqpoint{5.534644in}{3.416498in}}%
\pgfpathlineto{\pgfqpoint{5.526888in}{3.407857in}}%
\pgfpathlineto{\pgfqpoint{5.519120in}{3.398975in}}%
\pgfpathlineto{\pgfqpoint{5.511340in}{3.389853in}}%
\pgfpathlineto{\pgfqpoint{5.503548in}{3.380489in}}%
\pgfpathclose%
\pgfusepath{fill}%
\end{pgfscope}%
\begin{pgfscope}%
\pgfpathrectangle{\pgfqpoint{1.150000in}{0.150000in}}{\pgfqpoint{5.700000in}{5.700000in}}%
\pgfusepath{clip}%
\pgfsetbuttcap%
\pgfsetroundjoin%
\definecolor{currentfill}{rgb}{0.270595,0.214069,0.507052}%
\pgfsetfillcolor{currentfill}%
\pgfsetfillopacity{0.800000}%
\pgfsetlinewidth{0.000000pt}%
\definecolor{currentstroke}{rgb}{0.000000,0.000000,0.000000}%
\pgfsetstrokecolor{currentstroke}%
\pgfsetdash{}{0pt}%
\pgfpathmoveto{\pgfqpoint{4.220708in}{1.327330in}}%
\pgfpathlineto{\pgfqpoint{4.235171in}{1.333058in}}%
\pgfpathlineto{\pgfqpoint{4.249647in}{1.338960in}}%
\pgfpathlineto{\pgfqpoint{4.264136in}{1.345037in}}%
\pgfpathlineto{\pgfqpoint{4.278638in}{1.351289in}}%
\pgfpathlineto{\pgfqpoint{4.286918in}{1.368892in}}%
\pgfpathlineto{\pgfqpoint{4.295194in}{1.386608in}}%
\pgfpathlineto{\pgfqpoint{4.303467in}{1.404432in}}%
\pgfpathlineto{\pgfqpoint{4.311738in}{1.422355in}}%
\pgfpathlineto{\pgfqpoint{4.297230in}{1.415382in}}%
\pgfpathlineto{\pgfqpoint{4.282736in}{1.408585in}}%
\pgfpathlineto{\pgfqpoint{4.268256in}{1.401964in}}%
\pgfpathlineto{\pgfqpoint{4.253790in}{1.395518in}}%
\pgfpathlineto{\pgfqpoint{4.245524in}{1.378302in}}%
\pgfpathlineto{\pgfqpoint{4.237256in}{1.361195in}}%
\pgfpathlineto{\pgfqpoint{4.228984in}{1.344202in}}%
\pgfpathlineto{\pgfqpoint{4.220708in}{1.327330in}}%
\pgfpathclose%
\pgfusepath{fill}%
\end{pgfscope}%
\begin{pgfscope}%
\pgfpathrectangle{\pgfqpoint{1.150000in}{0.150000in}}{\pgfqpoint{5.700000in}{5.700000in}}%
\pgfusepath{clip}%
\pgfsetbuttcap%
\pgfsetroundjoin%
\definecolor{currentfill}{rgb}{0.203063,0.379716,0.553925}%
\pgfsetfillcolor{currentfill}%
\pgfsetfillopacity{0.800000}%
\pgfsetlinewidth{0.000000pt}%
\definecolor{currentstroke}{rgb}{0.000000,0.000000,0.000000}%
\pgfsetstrokecolor{currentstroke}%
\pgfsetdash{}{0pt}%
\pgfpathmoveto{\pgfqpoint{4.501970in}{1.757510in}}%
\pgfpathlineto{\pgfqpoint{4.516589in}{1.767952in}}%
\pgfpathlineto{\pgfqpoint{4.531225in}{1.778574in}}%
\pgfpathlineto{\pgfqpoint{4.545878in}{1.789376in}}%
\pgfpathlineto{\pgfqpoint{4.560547in}{1.800358in}}%
\pgfpathlineto{\pgfqpoint{4.568788in}{1.820187in}}%
\pgfpathlineto{\pgfqpoint{4.577026in}{1.839993in}}%
\pgfpathlineto{\pgfqpoint{4.585261in}{1.859769in}}%
\pgfpathlineto{\pgfqpoint{4.593493in}{1.879511in}}%
\pgfpathlineto{\pgfqpoint{4.578811in}{1.867958in}}%
\pgfpathlineto{\pgfqpoint{4.564146in}{1.856585in}}%
\pgfpathlineto{\pgfqpoint{4.549498in}{1.845394in}}%
\pgfpathlineto{\pgfqpoint{4.534868in}{1.834383in}}%
\pgfpathlineto{\pgfqpoint{4.526648in}{1.815199in}}%
\pgfpathlineto{\pgfqpoint{4.518425in}{1.795989in}}%
\pgfpathlineto{\pgfqpoint{4.510199in}{1.776757in}}%
\pgfpathlineto{\pgfqpoint{4.501970in}{1.757510in}}%
\pgfpathclose%
\pgfusepath{fill}%
\end{pgfscope}%
\begin{pgfscope}%
\pgfpathrectangle{\pgfqpoint{1.150000in}{0.150000in}}{\pgfqpoint{5.700000in}{5.700000in}}%
\pgfusepath{clip}%
\pgfsetbuttcap%
\pgfsetroundjoin%
\definecolor{currentfill}{rgb}{0.163625,0.471133,0.558148}%
\pgfsetfillcolor{currentfill}%
\pgfsetfillopacity{0.800000}%
\pgfsetlinewidth{0.000000pt}%
\definecolor{currentstroke}{rgb}{0.000000,0.000000,0.000000}%
\pgfsetstrokecolor{currentstroke}%
\pgfsetdash{}{0pt}%
\pgfpathmoveto{\pgfqpoint{4.659226in}{2.035617in}}%
\pgfpathlineto{\pgfqpoint{4.673952in}{2.048398in}}%
\pgfpathlineto{\pgfqpoint{4.688696in}{2.061363in}}%
\pgfpathlineto{\pgfqpoint{4.703459in}{2.074511in}}%
\pgfpathlineto{\pgfqpoint{4.718241in}{2.087844in}}%
\pgfpathlineto{\pgfqpoint{4.726455in}{2.107544in}}%
\pgfpathlineto{\pgfqpoint{4.734665in}{2.127155in}}%
\pgfpathlineto{\pgfqpoint{4.742871in}{2.146671in}}%
\pgfpathlineto{\pgfqpoint{4.751073in}{2.166088in}}%
\pgfpathlineto{\pgfqpoint{4.736277in}{2.152280in}}%
\pgfpathlineto{\pgfqpoint{4.721500in}{2.138656in}}%
\pgfpathlineto{\pgfqpoint{4.706743in}{2.125216in}}%
\pgfpathlineto{\pgfqpoint{4.692004in}{2.111961in}}%
\pgfpathlineto{\pgfqpoint{4.683816in}{2.093006in}}%
\pgfpathlineto{\pgfqpoint{4.675623in}{2.073961in}}%
\pgfpathlineto{\pgfqpoint{4.667426in}{2.054830in}}%
\pgfpathlineto{\pgfqpoint{4.659226in}{2.035617in}}%
\pgfpathclose%
\pgfusepath{fill}%
\end{pgfscope}%
\begin{pgfscope}%
\pgfpathrectangle{\pgfqpoint{1.150000in}{0.150000in}}{\pgfqpoint{5.700000in}{5.700000in}}%
\pgfusepath{clip}%
\pgfsetbuttcap%
\pgfsetroundjoin%
\definecolor{currentfill}{rgb}{0.128087,0.647749,0.523491}%
\pgfsetfillcolor{currentfill}%
\pgfsetfillopacity{0.800000}%
\pgfsetlinewidth{0.000000pt}%
\definecolor{currentstroke}{rgb}{0.000000,0.000000,0.000000}%
\pgfsetstrokecolor{currentstroke}%
\pgfsetdash{}{0pt}%
\pgfpathmoveto{\pgfqpoint{4.973736in}{2.592522in}}%
\pgfpathlineto{\pgfqpoint{4.988699in}{2.609144in}}%
\pgfpathlineto{\pgfqpoint{5.003684in}{2.625957in}}%
\pgfpathlineto{\pgfqpoint{5.018691in}{2.642959in}}%
\pgfpathlineto{\pgfqpoint{5.033720in}{2.660153in}}%
\pgfpathlineto{\pgfqpoint{5.041835in}{2.677236in}}%
\pgfpathlineto{\pgfqpoint{5.049943in}{2.694132in}}%
\pgfpathlineto{\pgfqpoint{5.058044in}{2.710836in}}%
\pgfpathlineto{\pgfqpoint{5.066138in}{2.727348in}}%
\pgfpathlineto{\pgfqpoint{5.051097in}{2.709881in}}%
\pgfpathlineto{\pgfqpoint{5.036079in}{2.692606in}}%
\pgfpathlineto{\pgfqpoint{5.021083in}{2.675522in}}%
\pgfpathlineto{\pgfqpoint{5.006109in}{2.658628in}}%
\pgfpathlineto{\pgfqpoint{4.998026in}{2.642375in}}%
\pgfpathlineto{\pgfqpoint{4.989936in}{2.625938in}}%
\pgfpathlineto{\pgfqpoint{4.981839in}{2.609320in}}%
\pgfpathlineto{\pgfqpoint{4.973736in}{2.592522in}}%
\pgfpathclose%
\pgfusepath{fill}%
\end{pgfscope}%
\begin{pgfscope}%
\pgfpathrectangle{\pgfqpoint{1.150000in}{0.150000in}}{\pgfqpoint{5.700000in}{5.700000in}}%
\pgfusepath{clip}%
\pgfsetbuttcap%
\pgfsetroundjoin%
\definecolor{currentfill}{rgb}{0.327796,0.773980,0.406640}%
\pgfsetfillcolor{currentfill}%
\pgfsetfillopacity{0.800000}%
\pgfsetlinewidth{0.000000pt}%
\definecolor{currentstroke}{rgb}{0.000000,0.000000,0.000000}%
\pgfsetstrokecolor{currentstroke}%
\pgfsetdash{}{0pt}%
\pgfpathmoveto{\pgfqpoint{5.255097in}{3.038542in}}%
\pgfpathlineto{\pgfqpoint{5.270288in}{3.057704in}}%
\pgfpathlineto{\pgfqpoint{5.285504in}{3.077060in}}%
\pgfpathlineto{\pgfqpoint{5.300744in}{3.096612in}}%
\pgfpathlineto{\pgfqpoint{5.316009in}{3.116359in}}%
\pgfpathlineto{\pgfqpoint{5.323974in}{3.129549in}}%
\pgfpathlineto{\pgfqpoint{5.331930in}{3.142503in}}%
\pgfpathlineto{\pgfqpoint{5.339875in}{3.155220in}}%
\pgfpathlineto{\pgfqpoint{5.347810in}{3.167700in}}%
\pgfpathlineto{\pgfqpoint{5.332540in}{3.147862in}}%
\pgfpathlineto{\pgfqpoint{5.317295in}{3.128219in}}%
\pgfpathlineto{\pgfqpoint{5.302074in}{3.108772in}}%
\pgfpathlineto{\pgfqpoint{5.286878in}{3.089519in}}%
\pgfpathlineto{\pgfqpoint{5.278947in}{3.077115in}}%
\pgfpathlineto{\pgfqpoint{5.271007in}{3.064484in}}%
\pgfpathlineto{\pgfqpoint{5.263056in}{3.051626in}}%
\pgfpathlineto{\pgfqpoint{5.255097in}{3.038542in}}%
\pgfpathclose%
\pgfusepath{fill}%
\end{pgfscope}%
\begin{pgfscope}%
\pgfpathrectangle{\pgfqpoint{1.150000in}{0.150000in}}{\pgfqpoint{5.700000in}{5.700000in}}%
\pgfusepath{clip}%
\pgfsetbuttcap%
\pgfsetroundjoin%
\definecolor{currentfill}{rgb}{0.129933,0.559582,0.551864}%
\pgfsetfillcolor{currentfill}%
\pgfsetfillopacity{0.800000}%
\pgfsetlinewidth{0.000000pt}%
\definecolor{currentstroke}{rgb}{0.000000,0.000000,0.000000}%
\pgfsetstrokecolor{currentstroke}%
\pgfsetdash{}{0pt}%
\pgfpathmoveto{\pgfqpoint{4.816529in}{2.317394in}}%
\pgfpathlineto{\pgfqpoint{4.831371in}{2.332240in}}%
\pgfpathlineto{\pgfqpoint{4.846234in}{2.347273in}}%
\pgfpathlineto{\pgfqpoint{4.861117in}{2.362493in}}%
\pgfpathlineto{\pgfqpoint{4.876020in}{2.377901in}}%
\pgfpathlineto{\pgfqpoint{4.884194in}{2.396642in}}%
\pgfpathlineto{\pgfqpoint{4.892363in}{2.415238in}}%
\pgfpathlineto{\pgfqpoint{4.900527in}{2.433685in}}%
\pgfpathlineto{\pgfqpoint{4.908684in}{2.451981in}}%
\pgfpathlineto{\pgfqpoint{4.893767in}{2.436196in}}%
\pgfpathlineto{\pgfqpoint{4.878871in}{2.420600in}}%
\pgfpathlineto{\pgfqpoint{4.863995in}{2.405192in}}%
\pgfpathlineto{\pgfqpoint{4.849140in}{2.389971in}}%
\pgfpathlineto{\pgfqpoint{4.840995in}{2.372038in}}%
\pgfpathlineto{\pgfqpoint{4.832845in}{2.353962in}}%
\pgfpathlineto{\pgfqpoint{4.824689in}{2.335746in}}%
\pgfpathlineto{\pgfqpoint{4.816529in}{2.317394in}}%
\pgfpathclose%
\pgfusepath{fill}%
\end{pgfscope}%
\begin{pgfscope}%
\pgfpathrectangle{\pgfqpoint{1.150000in}{0.150000in}}{\pgfqpoint{5.700000in}{5.700000in}}%
\pgfusepath{clip}%
\pgfsetbuttcap%
\pgfsetroundjoin%
\definecolor{currentfill}{rgb}{0.246811,0.283237,0.535941}%
\pgfsetfillcolor{currentfill}%
\pgfsetfillopacity{0.800000}%
\pgfsetlinewidth{0.000000pt}%
\definecolor{currentstroke}{rgb}{0.000000,0.000000,0.000000}%
\pgfsetstrokecolor{currentstroke}%
\pgfsetdash{}{0pt}%
\pgfpathmoveto{\pgfqpoint{4.344789in}{1.494916in}}%
\pgfpathlineto{\pgfqpoint{4.359317in}{1.502755in}}%
\pgfpathlineto{\pgfqpoint{4.373861in}{1.510771in}}%
\pgfpathlineto{\pgfqpoint{4.388419in}{1.518963in}}%
\pgfpathlineto{\pgfqpoint{4.402992in}{1.527332in}}%
\pgfpathlineto{\pgfqpoint{4.411256in}{1.546330in}}%
\pgfpathlineto{\pgfqpoint{4.419517in}{1.565382in}}%
\pgfpathlineto{\pgfqpoint{4.427775in}{1.584480in}}%
\pgfpathlineto{\pgfqpoint{4.436030in}{1.603620in}}%
\pgfpathlineto{\pgfqpoint{4.421447in}{1.594589in}}%
\pgfpathlineto{\pgfqpoint{4.406881in}{1.585735in}}%
\pgfpathlineto{\pgfqpoint{4.392329in}{1.577058in}}%
\pgfpathlineto{\pgfqpoint{4.377793in}{1.568560in}}%
\pgfpathlineto{\pgfqpoint{4.369546in}{1.550069in}}%
\pgfpathlineto{\pgfqpoint{4.361297in}{1.531628in}}%
\pgfpathlineto{\pgfqpoint{4.353044in}{1.513241in}}%
\pgfpathlineto{\pgfqpoint{4.344789in}{1.494916in}}%
\pgfpathclose%
\pgfusepath{fill}%
\end{pgfscope}%
\begin{pgfscope}%
\pgfpathrectangle{\pgfqpoint{1.150000in}{0.150000in}}{\pgfqpoint{5.700000in}{5.700000in}}%
\pgfusepath{clip}%
\pgfsetbuttcap%
\pgfsetroundjoin%
\definecolor{currentfill}{rgb}{0.281887,0.150881,0.465405}%
\pgfsetfillcolor{currentfill}%
\pgfsetfillopacity{0.800000}%
\pgfsetlinewidth{0.000000pt}%
\definecolor{currentstroke}{rgb}{0.000000,0.000000,0.000000}%
\pgfsetstrokecolor{currentstroke}%
\pgfsetdash{}{0pt}%
\pgfpathmoveto{\pgfqpoint{4.096654in}{1.182506in}}%
\pgfpathlineto{\pgfqpoint{4.111067in}{1.186013in}}%
\pgfpathlineto{\pgfqpoint{4.125492in}{1.189693in}}%
\pgfpathlineto{\pgfqpoint{4.139927in}{1.193546in}}%
\pgfpathlineto{\pgfqpoint{4.154375in}{1.197572in}}%
\pgfpathlineto{\pgfqpoint{4.162680in}{1.213217in}}%
\pgfpathlineto{\pgfqpoint{4.170981in}{1.229041in}}%
\pgfpathlineto{\pgfqpoint{4.179278in}{1.245037in}}%
\pgfpathlineto{\pgfqpoint{4.187571in}{1.261196in}}%
\pgfpathlineto{\pgfqpoint{4.173123in}{1.256391in}}%
\pgfpathlineto{\pgfqpoint{4.158688in}{1.251760in}}%
\pgfpathlineto{\pgfqpoint{4.144264in}{1.247303in}}%
\pgfpathlineto{\pgfqpoint{4.129853in}{1.243019in}}%
\pgfpathlineto{\pgfqpoint{4.121559in}{1.227626in}}%
\pgfpathlineto{\pgfqpoint{4.113262in}{1.212404in}}%
\pgfpathlineto{\pgfqpoint{4.104960in}{1.197362in}}%
\pgfpathlineto{\pgfqpoint{4.096654in}{1.182506in}}%
\pgfpathclose%
\pgfusepath{fill}%
\end{pgfscope}%
\begin{pgfscope}%
\pgfpathrectangle{\pgfqpoint{1.150000in}{0.150000in}}{\pgfqpoint{5.700000in}{5.700000in}}%
\pgfusepath{clip}%
\pgfsetbuttcap%
\pgfsetroundjoin%
\definecolor{currentfill}{rgb}{0.468053,0.818921,0.323998}%
\pgfsetfillcolor{currentfill}%
\pgfsetfillopacity{0.800000}%
\pgfsetlinewidth{0.000000pt}%
\definecolor{currentstroke}{rgb}{0.000000,0.000000,0.000000}%
\pgfsetstrokecolor{currentstroke}%
\pgfsetdash{}{0pt}%
\pgfpathmoveto{\pgfqpoint{5.379446in}{3.215250in}}%
\pgfpathlineto{\pgfqpoint{5.394744in}{3.235338in}}%
\pgfpathlineto{\pgfqpoint{5.410068in}{3.255622in}}%
\pgfpathlineto{\pgfqpoint{5.425417in}{3.276103in}}%
\pgfpathlineto{\pgfqpoint{5.440792in}{3.296781in}}%
\pgfpathlineto{\pgfqpoint{5.448676in}{3.308105in}}%
\pgfpathlineto{\pgfqpoint{5.456550in}{3.319182in}}%
\pgfpathlineto{\pgfqpoint{5.464412in}{3.330013in}}%
\pgfpathlineto{\pgfqpoint{5.472263in}{3.340598in}}%
\pgfpathlineto{\pgfqpoint{5.456886in}{3.319904in}}%
\pgfpathlineto{\pgfqpoint{5.441535in}{3.299408in}}%
\pgfpathlineto{\pgfqpoint{5.426210in}{3.279108in}}%
\pgfpathlineto{\pgfqpoint{5.410910in}{3.259004in}}%
\pgfpathlineto{\pgfqpoint{5.403061in}{3.248421in}}%
\pgfpathlineto{\pgfqpoint{5.395200in}{3.237601in}}%
\pgfpathlineto{\pgfqpoint{5.387328in}{3.226544in}}%
\pgfpathlineto{\pgfqpoint{5.379446in}{3.215250in}}%
\pgfpathclose%
\pgfusepath{fill}%
\end{pgfscope}%
\begin{pgfscope}%
\pgfpathrectangle{\pgfqpoint{1.150000in}{0.150000in}}{\pgfqpoint{5.700000in}{5.700000in}}%
\pgfusepath{clip}%
\pgfsetbuttcap%
\pgfsetroundjoin%
\definecolor{currentfill}{rgb}{0.214298,0.355619,0.551184}%
\pgfsetfillcolor{currentfill}%
\pgfsetfillopacity{0.800000}%
\pgfsetlinewidth{0.000000pt}%
\definecolor{currentstroke}{rgb}{0.000000,0.000000,0.000000}%
\pgfsetstrokecolor{currentstroke}%
\pgfsetdash{}{0pt}%
\pgfpathmoveto{\pgfqpoint{4.469023in}{1.680471in}}%
\pgfpathlineto{\pgfqpoint{4.483631in}{1.690313in}}%
\pgfpathlineto{\pgfqpoint{4.498256in}{1.700333in}}%
\pgfpathlineto{\pgfqpoint{4.512897in}{1.710533in}}%
\pgfpathlineto{\pgfqpoint{4.527555in}{1.720911in}}%
\pgfpathlineto{\pgfqpoint{4.535807in}{1.740781in}}%
\pgfpathlineto{\pgfqpoint{4.544057in}{1.760650in}}%
\pgfpathlineto{\pgfqpoint{4.552303in}{1.780510in}}%
\pgfpathlineto{\pgfqpoint{4.560547in}{1.800358in}}%
\pgfpathlineto{\pgfqpoint{4.545878in}{1.789376in}}%
\pgfpathlineto{\pgfqpoint{4.531225in}{1.778574in}}%
\pgfpathlineto{\pgfqpoint{4.516589in}{1.767952in}}%
\pgfpathlineto{\pgfqpoint{4.501970in}{1.757510in}}%
\pgfpathlineto{\pgfqpoint{4.493737in}{1.738252in}}%
\pgfpathlineto{\pgfqpoint{4.485502in}{1.718989in}}%
\pgfpathlineto{\pgfqpoint{4.477264in}{1.699727in}}%
\pgfpathlineto{\pgfqpoint{4.469023in}{1.680471in}}%
\pgfpathclose%
\pgfusepath{fill}%
\end{pgfscope}%
\begin{pgfscope}%
\pgfpathrectangle{\pgfqpoint{1.150000in}{0.150000in}}{\pgfqpoint{5.700000in}{5.700000in}}%
\pgfusepath{clip}%
\pgfsetbuttcap%
\pgfsetroundjoin%
\definecolor{currentfill}{rgb}{0.191090,0.708366,0.482284}%
\pgfsetfillcolor{currentfill}%
\pgfsetfillopacity{0.800000}%
\pgfsetlinewidth{0.000000pt}%
\definecolor{currentstroke}{rgb}{0.000000,0.000000,0.000000}%
\pgfsetstrokecolor{currentstroke}%
\pgfsetdash{}{0pt}%
\pgfpathmoveto{\pgfqpoint{5.098439in}{2.791423in}}%
\pgfpathlineto{\pgfqpoint{5.113512in}{2.809318in}}%
\pgfpathlineto{\pgfqpoint{5.128609in}{2.827405in}}%
\pgfpathlineto{\pgfqpoint{5.143729in}{2.845685in}}%
\pgfpathlineto{\pgfqpoint{5.158872in}{2.864158in}}%
\pgfpathlineto{\pgfqpoint{5.166938in}{2.879890in}}%
\pgfpathlineto{\pgfqpoint{5.174996in}{2.895408in}}%
\pgfpathlineto{\pgfqpoint{5.183046in}{2.910712in}}%
\pgfpathlineto{\pgfqpoint{5.191087in}{2.925798in}}%
\pgfpathlineto{\pgfqpoint{5.175934in}{2.907123in}}%
\pgfpathlineto{\pgfqpoint{5.160805in}{2.888641in}}%
\pgfpathlineto{\pgfqpoint{5.145699in}{2.870353in}}%
\pgfpathlineto{\pgfqpoint{5.130616in}{2.852257in}}%
\pgfpathlineto{\pgfqpoint{5.122584in}{2.837358in}}%
\pgfpathlineto{\pgfqpoint{5.114543in}{2.822252in}}%
\pgfpathlineto{\pgfqpoint{5.106495in}{2.806939in}}%
\pgfpathlineto{\pgfqpoint{5.098439in}{2.791423in}}%
\pgfpathclose%
\pgfusepath{fill}%
\end{pgfscope}%
\begin{pgfscope}%
\pgfpathrectangle{\pgfqpoint{1.150000in}{0.150000in}}{\pgfqpoint{5.700000in}{5.700000in}}%
\pgfusepath{clip}%
\pgfsetbuttcap%
\pgfsetroundjoin%
\definecolor{currentfill}{rgb}{0.172719,0.448791,0.557885}%
\pgfsetfillcolor{currentfill}%
\pgfsetfillopacity{0.800000}%
\pgfsetlinewidth{0.000000pt}%
\definecolor{currentstroke}{rgb}{0.000000,0.000000,0.000000}%
\pgfsetstrokecolor{currentstroke}%
\pgfsetdash{}{0pt}%
\pgfpathmoveto{\pgfqpoint{4.626387in}{1.958035in}}%
\pgfpathlineto{\pgfqpoint{4.641100in}{1.970310in}}%
\pgfpathlineto{\pgfqpoint{4.655831in}{1.982767in}}%
\pgfpathlineto{\pgfqpoint{4.670580in}{1.995407in}}%
\pgfpathlineto{\pgfqpoint{4.685348in}{2.008229in}}%
\pgfpathlineto{\pgfqpoint{4.693577in}{2.028245in}}%
\pgfpathlineto{\pgfqpoint{4.701802in}{2.048190in}}%
\pgfpathlineto{\pgfqpoint{4.710023in}{2.068057in}}%
\pgfpathlineto{\pgfqpoint{4.718241in}{2.087844in}}%
\pgfpathlineto{\pgfqpoint{4.703459in}{2.074511in}}%
\pgfpathlineto{\pgfqpoint{4.688696in}{2.061363in}}%
\pgfpathlineto{\pgfqpoint{4.673952in}{2.048398in}}%
\pgfpathlineto{\pgfqpoint{4.659226in}{2.035617in}}%
\pgfpathlineto{\pgfqpoint{4.651022in}{2.016326in}}%
\pgfpathlineto{\pgfqpoint{4.642814in}{1.996963in}}%
\pgfpathlineto{\pgfqpoint{4.634602in}{1.977531in}}%
\pgfpathlineto{\pgfqpoint{4.626387in}{1.958035in}}%
\pgfpathclose%
\pgfusepath{fill}%
\end{pgfscope}%
\begin{pgfscope}%
\pgfpathrectangle{\pgfqpoint{1.150000in}{0.150000in}}{\pgfqpoint{5.700000in}{5.700000in}}%
\pgfusepath{clip}%
\pgfsetbuttcap%
\pgfsetroundjoin%
\definecolor{currentfill}{rgb}{0.276194,0.190074,0.493001}%
\pgfsetfillcolor{currentfill}%
\pgfsetfillopacity{0.800000}%
\pgfsetlinewidth{0.000000pt}%
\definecolor{currentstroke}{rgb}{0.000000,0.000000,0.000000}%
\pgfsetstrokecolor{currentstroke}%
\pgfsetdash{}{0pt}%
\pgfpathmoveto{\pgfqpoint{4.187571in}{1.261196in}}%
\pgfpathlineto{\pgfqpoint{4.202032in}{1.266175in}}%
\pgfpathlineto{\pgfqpoint{4.216505in}{1.271328in}}%
\pgfpathlineto{\pgfqpoint{4.230990in}{1.276654in}}%
\pgfpathlineto{\pgfqpoint{4.245489in}{1.282154in}}%
\pgfpathlineto{\pgfqpoint{4.253781in}{1.299232in}}%
\pgfpathlineto{\pgfqpoint{4.262070in}{1.316452in}}%
\pgfpathlineto{\pgfqpoint{4.270356in}{1.333807in}}%
\pgfpathlineto{\pgfqpoint{4.278638in}{1.351289in}}%
\pgfpathlineto{\pgfqpoint{4.264136in}{1.345037in}}%
\pgfpathlineto{\pgfqpoint{4.249647in}{1.338960in}}%
\pgfpathlineto{\pgfqpoint{4.235171in}{1.333058in}}%
\pgfpathlineto{\pgfqpoint{4.220708in}{1.327330in}}%
\pgfpathlineto{\pgfqpoint{4.212429in}{1.310587in}}%
\pgfpathlineto{\pgfqpoint{4.204147in}{1.293979in}}%
\pgfpathlineto{\pgfqpoint{4.195861in}{1.277513in}}%
\pgfpathlineto{\pgfqpoint{4.187571in}{1.261196in}}%
\pgfpathclose%
\pgfusepath{fill}%
\end{pgfscope}%
\begin{pgfscope}%
\pgfpathrectangle{\pgfqpoint{1.150000in}{0.150000in}}{\pgfqpoint{5.700000in}{5.700000in}}%
\pgfusepath{clip}%
\pgfsetbuttcap%
\pgfsetroundjoin%
\definecolor{currentfill}{rgb}{0.136408,0.541173,0.554483}%
\pgfsetfillcolor{currentfill}%
\pgfsetfillopacity{0.800000}%
\pgfsetlinewidth{0.000000pt}%
\definecolor{currentstroke}{rgb}{0.000000,0.000000,0.000000}%
\pgfsetstrokecolor{currentstroke}%
\pgfsetdash{}{0pt}%
\pgfpathmoveto{\pgfqpoint{4.783837in}{2.242689in}}%
\pgfpathlineto{\pgfqpoint{4.798667in}{2.257126in}}%
\pgfpathlineto{\pgfqpoint{4.813516in}{2.271749in}}%
\pgfpathlineto{\pgfqpoint{4.828385in}{2.286558in}}%
\pgfpathlineto{\pgfqpoint{4.843274in}{2.301554in}}%
\pgfpathlineto{\pgfqpoint{4.851468in}{2.320841in}}%
\pgfpathlineto{\pgfqpoint{4.859657in}{2.339996in}}%
\pgfpathlineto{\pgfqpoint{4.867841in}{2.359018in}}%
\pgfpathlineto{\pgfqpoint{4.876020in}{2.377901in}}%
\pgfpathlineto{\pgfqpoint{4.861117in}{2.362493in}}%
\pgfpathlineto{\pgfqpoint{4.846234in}{2.347273in}}%
\pgfpathlineto{\pgfqpoint{4.831371in}{2.332240in}}%
\pgfpathlineto{\pgfqpoint{4.816529in}{2.317394in}}%
\pgfpathlineto{\pgfqpoint{4.808363in}{2.298909in}}%
\pgfpathlineto{\pgfqpoint{4.800193in}{2.280294in}}%
\pgfpathlineto{\pgfqpoint{4.792017in}{2.261553in}}%
\pgfpathlineto{\pgfqpoint{4.783837in}{2.242689in}}%
\pgfpathclose%
\pgfusepath{fill}%
\end{pgfscope}%
\begin{pgfscope}%
\pgfpathrectangle{\pgfqpoint{1.150000in}{0.150000in}}{\pgfqpoint{5.700000in}{5.700000in}}%
\pgfusepath{clip}%
\pgfsetbuttcap%
\pgfsetroundjoin%
\definecolor{currentfill}{rgb}{0.255645,0.260703,0.528312}%
\pgfsetfillcolor{currentfill}%
\pgfsetfillopacity{0.800000}%
\pgfsetlinewidth{0.000000pt}%
\definecolor{currentstroke}{rgb}{0.000000,0.000000,0.000000}%
\pgfsetstrokecolor{currentstroke}%
\pgfsetdash{}{0pt}%
\pgfpathmoveto{\pgfqpoint{4.311738in}{1.422355in}}%
\pgfpathlineto{\pgfqpoint{4.326260in}{1.429503in}}%
\pgfpathlineto{\pgfqpoint{4.340796in}{1.436827in}}%
\pgfpathlineto{\pgfqpoint{4.355346in}{1.444326in}}%
\pgfpathlineto{\pgfqpoint{4.369911in}{1.452001in}}%
\pgfpathlineto{\pgfqpoint{4.378185in}{1.470722in}}%
\pgfpathlineto{\pgfqpoint{4.386457in}{1.489522in}}%
\pgfpathlineto{\pgfqpoint{4.394726in}{1.508394in}}%
\pgfpathlineto{\pgfqpoint{4.402992in}{1.527332in}}%
\pgfpathlineto{\pgfqpoint{4.388419in}{1.518963in}}%
\pgfpathlineto{\pgfqpoint{4.373861in}{1.510771in}}%
\pgfpathlineto{\pgfqpoint{4.359317in}{1.502755in}}%
\pgfpathlineto{\pgfqpoint{4.344789in}{1.494916in}}%
\pgfpathlineto{\pgfqpoint{4.336531in}{1.476659in}}%
\pgfpathlineto{\pgfqpoint{4.328269in}{1.458475in}}%
\pgfpathlineto{\pgfqpoint{4.320005in}{1.440371in}}%
\pgfpathlineto{\pgfqpoint{4.311738in}{1.422355in}}%
\pgfpathclose%
\pgfusepath{fill}%
\end{pgfscope}%
\begin{pgfscope}%
\pgfpathrectangle{\pgfqpoint{1.150000in}{0.150000in}}{\pgfqpoint{5.700000in}{5.700000in}}%
\pgfusepath{clip}%
\pgfsetbuttcap%
\pgfsetroundjoin%
\definecolor{currentfill}{rgb}{0.121380,0.629492,0.531973}%
\pgfsetfillcolor{currentfill}%
\pgfsetfillopacity{0.800000}%
\pgfsetlinewidth{0.000000pt}%
\definecolor{currentstroke}{rgb}{0.000000,0.000000,0.000000}%
\pgfsetstrokecolor{currentstroke}%
\pgfsetdash{}{0pt}%
\pgfpathmoveto{\pgfqpoint{4.941259in}{2.523584in}}%
\pgfpathlineto{\pgfqpoint{4.956210in}{2.539899in}}%
\pgfpathlineto{\pgfqpoint{4.971183in}{2.556404in}}%
\pgfpathlineto{\pgfqpoint{4.986177in}{2.573098in}}%
\pgfpathlineto{\pgfqpoint{5.001193in}{2.589983in}}%
\pgfpathlineto{\pgfqpoint{5.009335in}{2.607795in}}%
\pgfpathlineto{\pgfqpoint{5.017470in}{2.625430in}}%
\pgfpathlineto{\pgfqpoint{5.025598in}{2.642883in}}%
\pgfpathlineto{\pgfqpoint{5.033720in}{2.660153in}}%
\pgfpathlineto{\pgfqpoint{5.018691in}{2.642959in}}%
\pgfpathlineto{\pgfqpoint{5.003684in}{2.625957in}}%
\pgfpathlineto{\pgfqpoint{4.988699in}{2.609144in}}%
\pgfpathlineto{\pgfqpoint{4.973736in}{2.592522in}}%
\pgfpathlineto{\pgfqpoint{4.965626in}{2.575547in}}%
\pgfpathlineto{\pgfqpoint{4.957510in}{2.558397in}}%
\pgfpathlineto{\pgfqpoint{4.949387in}{2.541075in}}%
\pgfpathlineto{\pgfqpoint{4.941259in}{2.523584in}}%
\pgfpathclose%
\pgfusepath{fill}%
\end{pgfscope}%
\begin{pgfscope}%
\pgfpathrectangle{\pgfqpoint{1.150000in}{0.150000in}}{\pgfqpoint{5.700000in}{5.700000in}}%
\pgfusepath{clip}%
\pgfsetbuttcap%
\pgfsetroundjoin%
\definecolor{currentfill}{rgb}{0.282884,0.135920,0.453427}%
\pgfsetfillcolor{currentfill}%
\pgfsetfillopacity{0.800000}%
\pgfsetlinewidth{0.000000pt}%
\definecolor{currentstroke}{rgb}{0.000000,0.000000,0.000000}%
\pgfsetstrokecolor{currentstroke}%
\pgfsetdash{}{0pt}%
\pgfpathmoveto{\pgfqpoint{4.063385in}{1.125101in}}%
\pgfpathlineto{\pgfqpoint{4.077801in}{1.127801in}}%
\pgfpathlineto{\pgfqpoint{4.092228in}{1.130672in}}%
\pgfpathlineto{\pgfqpoint{4.106666in}{1.133716in}}%
\pgfpathlineto{\pgfqpoint{4.121115in}{1.136932in}}%
\pgfpathlineto{\pgfqpoint{4.129436in}{1.151785in}}%
\pgfpathlineto{\pgfqpoint{4.137753in}{1.166848in}}%
\pgfpathlineto{\pgfqpoint{4.146066in}{1.182113in}}%
\pgfpathlineto{\pgfqpoint{4.154375in}{1.197572in}}%
\pgfpathlineto{\pgfqpoint{4.139927in}{1.193546in}}%
\pgfpathlineto{\pgfqpoint{4.125492in}{1.189693in}}%
\pgfpathlineto{\pgfqpoint{4.111067in}{1.186013in}}%
\pgfpathlineto{\pgfqpoint{4.096654in}{1.182506in}}%
\pgfpathlineto{\pgfqpoint{4.088344in}{1.167844in}}%
\pgfpathlineto{\pgfqpoint{4.080029in}{1.153384in}}%
\pgfpathlineto{\pgfqpoint{4.071709in}{1.139134in}}%
\pgfpathlineto{\pgfqpoint{4.063385in}{1.125101in}}%
\pgfpathclose%
\pgfusepath{fill}%
\end{pgfscope}%
\begin{pgfscope}%
\pgfpathrectangle{\pgfqpoint{1.150000in}{0.150000in}}{\pgfqpoint{5.700000in}{5.700000in}}%
\pgfusepath{clip}%
\pgfsetbuttcap%
\pgfsetroundjoin%
\definecolor{currentfill}{rgb}{0.311925,0.767822,0.415586}%
\pgfsetfillcolor{currentfill}%
\pgfsetfillopacity{0.800000}%
\pgfsetlinewidth{0.000000pt}%
\definecolor{currentstroke}{rgb}{0.000000,0.000000,0.000000}%
\pgfsetstrokecolor{currentstroke}%
\pgfsetdash{}{0pt}%
\pgfpathmoveto{\pgfqpoint{5.223164in}{2.983951in}}%
\pgfpathlineto{\pgfqpoint{5.238349in}{3.002985in}}%
\pgfpathlineto{\pgfqpoint{5.253558in}{3.022213in}}%
\pgfpathlineto{\pgfqpoint{5.268791in}{3.041637in}}%
\pgfpathlineto{\pgfqpoint{5.284049in}{3.061255in}}%
\pgfpathlineto{\pgfqpoint{5.292053in}{3.075382in}}%
\pgfpathlineto{\pgfqpoint{5.300048in}{3.089275in}}%
\pgfpathlineto{\pgfqpoint{5.308033in}{3.102935in}}%
\pgfpathlineto{\pgfqpoint{5.316009in}{3.116359in}}%
\pgfpathlineto{\pgfqpoint{5.300744in}{3.096612in}}%
\pgfpathlineto{\pgfqpoint{5.285504in}{3.077060in}}%
\pgfpathlineto{\pgfqpoint{5.270288in}{3.057704in}}%
\pgfpathlineto{\pgfqpoint{5.255097in}{3.038542in}}%
\pgfpathlineto{\pgfqpoint{5.247127in}{3.025231in}}%
\pgfpathlineto{\pgfqpoint{5.239149in}{3.011695in}}%
\pgfpathlineto{\pgfqpoint{5.231161in}{2.997935in}}%
\pgfpathlineto{\pgfqpoint{5.223164in}{2.983951in}}%
\pgfpathclose%
\pgfusepath{fill}%
\end{pgfscope}%
\begin{pgfscope}%
\pgfpathrectangle{\pgfqpoint{1.150000in}{0.150000in}}{\pgfqpoint{5.700000in}{5.700000in}}%
\pgfusepath{clip}%
\pgfsetbuttcap%
\pgfsetroundjoin%
\definecolor{currentfill}{rgb}{0.575563,0.844566,0.256415}%
\pgfsetfillcolor{currentfill}%
\pgfsetfillopacity{0.800000}%
\pgfsetlinewidth{0.000000pt}%
\definecolor{currentstroke}{rgb}{0.000000,0.000000,0.000000}%
\pgfsetstrokecolor{currentstroke}%
\pgfsetdash{}{0pt}%
\pgfpathmoveto{\pgfqpoint{5.472263in}{3.340598in}}%
\pgfpathlineto{\pgfqpoint{5.487665in}{3.361489in}}%
\pgfpathlineto{\pgfqpoint{5.503093in}{3.382579in}}%
\pgfpathlineto{\pgfqpoint{5.518548in}{3.403867in}}%
\pgfpathlineto{\pgfqpoint{5.526387in}{3.414200in}}%
\pgfpathlineto{\pgfqpoint{5.534214in}{3.424281in}}%
\pgfpathlineto{\pgfqpoint{5.542030in}{3.434110in}}%
\pgfpathlineto{\pgfqpoint{5.549832in}{3.443689in}}%
\pgfpathlineto{\pgfqpoint{5.534378in}{3.422424in}}%
\pgfpathlineto{\pgfqpoint{5.518950in}{3.401358in}}%
\pgfpathlineto{\pgfqpoint{5.503548in}{3.380489in}}%
\pgfpathlineto{\pgfqpoint{5.495744in}{3.370882in}}%
\pgfpathlineto{\pgfqpoint{5.487929in}{3.361032in}}%
\pgfpathlineto{\pgfqpoint{5.480102in}{3.350938in}}%
\pgfpathlineto{\pgfqpoint{5.472263in}{3.340598in}}%
\pgfpathclose%
\pgfusepath{fill}%
\end{pgfscope}%
\begin{pgfscope}%
\pgfpathrectangle{\pgfqpoint{1.150000in}{0.150000in}}{\pgfqpoint{5.700000in}{5.700000in}}%
\pgfusepath{clip}%
\pgfsetbuttcap%
\pgfsetroundjoin%
\definecolor{currentfill}{rgb}{0.223925,0.334994,0.548053}%
\pgfsetfillcolor{currentfill}%
\pgfsetfillopacity{0.800000}%
\pgfsetlinewidth{0.000000pt}%
\definecolor{currentstroke}{rgb}{0.000000,0.000000,0.000000}%
\pgfsetstrokecolor{currentstroke}%
\pgfsetdash{}{0pt}%
\pgfpathmoveto{\pgfqpoint{4.436030in}{1.603620in}}%
\pgfpathlineto{\pgfqpoint{4.450628in}{1.612830in}}%
\pgfpathlineto{\pgfqpoint{4.465242in}{1.622217in}}%
\pgfpathlineto{\pgfqpoint{4.479872in}{1.631782in}}%
\pgfpathlineto{\pgfqpoint{4.494518in}{1.641526in}}%
\pgfpathlineto{\pgfqpoint{4.502781in}{1.661346in}}%
\pgfpathlineto{\pgfqpoint{4.511041in}{1.681188in}}%
\pgfpathlineto{\pgfqpoint{4.519299in}{1.701045in}}%
\pgfpathlineto{\pgfqpoint{4.527555in}{1.720911in}}%
\pgfpathlineto{\pgfqpoint{4.512897in}{1.710533in}}%
\pgfpathlineto{\pgfqpoint{4.498256in}{1.700333in}}%
\pgfpathlineto{\pgfqpoint{4.483631in}{1.690313in}}%
\pgfpathlineto{\pgfqpoint{4.469023in}{1.680471in}}%
\pgfpathlineto{\pgfqpoint{4.460779in}{1.661226in}}%
\pgfpathlineto{\pgfqpoint{4.452532in}{1.641999in}}%
\pgfpathlineto{\pgfqpoint{4.444282in}{1.622795in}}%
\pgfpathlineto{\pgfqpoint{4.436030in}{1.603620in}}%
\pgfpathclose%
\pgfusepath{fill}%
\end{pgfscope}%
\begin{pgfscope}%
\pgfpathrectangle{\pgfqpoint{1.150000in}{0.150000in}}{\pgfqpoint{5.700000in}{5.700000in}}%
\pgfusepath{clip}%
\pgfsetbuttcap%
\pgfsetroundjoin%
\definecolor{currentfill}{rgb}{0.180629,0.429975,0.557282}%
\pgfsetfillcolor{currentfill}%
\pgfsetfillopacity{0.800000}%
\pgfsetlinewidth{0.000000pt}%
\definecolor{currentstroke}{rgb}{0.000000,0.000000,0.000000}%
\pgfsetstrokecolor{currentstroke}%
\pgfsetdash{}{0pt}%
\pgfpathmoveto{\pgfqpoint{4.593493in}{1.879511in}}%
\pgfpathlineto{\pgfqpoint{4.608193in}{1.891246in}}%
\pgfpathlineto{\pgfqpoint{4.622910in}{1.903162in}}%
\pgfpathlineto{\pgfqpoint{4.637646in}{1.915260in}}%
\pgfpathlineto{\pgfqpoint{4.652400in}{1.927540in}}%
\pgfpathlineto{\pgfqpoint{4.660642in}{1.947796in}}%
\pgfpathlineto{\pgfqpoint{4.668880in}{1.967999in}}%
\pgfpathlineto{\pgfqpoint{4.677116in}{1.988145in}}%
\pgfpathlineto{\pgfqpoint{4.685348in}{2.008229in}}%
\pgfpathlineto{\pgfqpoint{4.670580in}{1.995407in}}%
\pgfpathlineto{\pgfqpoint{4.655831in}{1.982767in}}%
\pgfpathlineto{\pgfqpoint{4.641100in}{1.970310in}}%
\pgfpathlineto{\pgfqpoint{4.626387in}{1.958035in}}%
\pgfpathlineto{\pgfqpoint{4.618168in}{1.938481in}}%
\pgfpathlineto{\pgfqpoint{4.609947in}{1.918872in}}%
\pgfpathlineto{\pgfqpoint{4.601721in}{1.899214in}}%
\pgfpathlineto{\pgfqpoint{4.593493in}{1.879511in}}%
\pgfpathclose%
\pgfusepath{fill}%
\end{pgfscope}%
\begin{pgfscope}%
\pgfpathrectangle{\pgfqpoint{1.150000in}{0.150000in}}{\pgfqpoint{5.700000in}{5.700000in}}%
\pgfusepath{clip}%
\pgfsetbuttcap%
\pgfsetroundjoin%
\definecolor{currentfill}{rgb}{0.449368,0.813768,0.335384}%
\pgfsetfillcolor{currentfill}%
\pgfsetfillopacity{0.800000}%
\pgfsetlinewidth{0.000000pt}%
\definecolor{currentstroke}{rgb}{0.000000,0.000000,0.000000}%
\pgfsetstrokecolor{currentstroke}%
\pgfsetdash{}{0pt}%
\pgfpathmoveto{\pgfqpoint{5.347810in}{3.167700in}}%
\pgfpathlineto{\pgfqpoint{5.363105in}{3.187735in}}%
\pgfpathlineto{\pgfqpoint{5.378425in}{3.207965in}}%
\pgfpathlineto{\pgfqpoint{5.393770in}{3.228393in}}%
\pgfpathlineto{\pgfqpoint{5.409141in}{3.249017in}}%
\pgfpathlineto{\pgfqpoint{5.417070in}{3.261328in}}%
\pgfpathlineto{\pgfqpoint{5.424989in}{3.273393in}}%
\pgfpathlineto{\pgfqpoint{5.432896in}{3.285210in}}%
\pgfpathlineto{\pgfqpoint{5.440792in}{3.296781in}}%
\pgfpathlineto{\pgfqpoint{5.425417in}{3.276103in}}%
\pgfpathlineto{\pgfqpoint{5.410068in}{3.255622in}}%
\pgfpathlineto{\pgfqpoint{5.394744in}{3.235338in}}%
\pgfpathlineto{\pgfqpoint{5.379446in}{3.215250in}}%
\pgfpathlineto{\pgfqpoint{5.371553in}{3.203719in}}%
\pgfpathlineto{\pgfqpoint{5.363649in}{3.191950in}}%
\pgfpathlineto{\pgfqpoint{5.355735in}{3.179944in}}%
\pgfpathlineto{\pgfqpoint{5.347810in}{3.167700in}}%
\pgfpathclose%
\pgfusepath{fill}%
\end{pgfscope}%
\begin{pgfscope}%
\pgfpathrectangle{\pgfqpoint{1.150000in}{0.150000in}}{\pgfqpoint{5.700000in}{5.700000in}}%
\pgfusepath{clip}%
\pgfsetbuttcap%
\pgfsetroundjoin%
\definecolor{currentfill}{rgb}{0.143343,0.522773,0.556295}%
\pgfsetfillcolor{currentfill}%
\pgfsetfillopacity{0.800000}%
\pgfsetlinewidth{0.000000pt}%
\definecolor{currentstroke}{rgb}{0.000000,0.000000,0.000000}%
\pgfsetstrokecolor{currentstroke}%
\pgfsetdash{}{0pt}%
\pgfpathmoveto{\pgfqpoint{4.751073in}{2.166088in}}%
\pgfpathlineto{\pgfqpoint{4.765888in}{2.180082in}}%
\pgfpathlineto{\pgfqpoint{4.780723in}{2.194260in}}%
\pgfpathlineto{\pgfqpoint{4.795577in}{2.208624in}}%
\pgfpathlineto{\pgfqpoint{4.810452in}{2.223174in}}%
\pgfpathlineto{\pgfqpoint{4.818664in}{2.242946in}}%
\pgfpathlineto{\pgfqpoint{4.826872in}{2.262603in}}%
\pgfpathlineto{\pgfqpoint{4.835075in}{2.282140in}}%
\pgfpathlineto{\pgfqpoint{4.843274in}{2.301554in}}%
\pgfpathlineto{\pgfqpoint{4.828385in}{2.286558in}}%
\pgfpathlineto{\pgfqpoint{4.813516in}{2.271749in}}%
\pgfpathlineto{\pgfqpoint{4.798667in}{2.257126in}}%
\pgfpathlineto{\pgfqpoint{4.783837in}{2.242689in}}%
\pgfpathlineto{\pgfqpoint{4.775653in}{2.223707in}}%
\pgfpathlineto{\pgfqpoint{4.767464in}{2.204611in}}%
\pgfpathlineto{\pgfqpoint{4.759270in}{2.185403in}}%
\pgfpathlineto{\pgfqpoint{4.751073in}{2.166088in}}%
\pgfpathclose%
\pgfusepath{fill}%
\end{pgfscope}%
\begin{pgfscope}%
\pgfpathrectangle{\pgfqpoint{1.150000in}{0.150000in}}{\pgfqpoint{5.700000in}{5.700000in}}%
\pgfusepath{clip}%
\pgfsetbuttcap%
\pgfsetroundjoin%
\definecolor{currentfill}{rgb}{0.175707,0.697900,0.491033}%
\pgfsetfillcolor{currentfill}%
\pgfsetfillopacity{0.800000}%
\pgfsetlinewidth{0.000000pt}%
\definecolor{currentstroke}{rgb}{0.000000,0.000000,0.000000}%
\pgfsetstrokecolor{currentstroke}%
\pgfsetdash{}{0pt}%
\pgfpathmoveto{\pgfqpoint{5.066138in}{2.727348in}}%
\pgfpathlineto{\pgfqpoint{5.081201in}{2.745005in}}%
\pgfpathlineto{\pgfqpoint{5.096287in}{2.762855in}}%
\pgfpathlineto{\pgfqpoint{5.111396in}{2.780897in}}%
\pgfpathlineto{\pgfqpoint{5.126528in}{2.799131in}}%
\pgfpathlineto{\pgfqpoint{5.134625in}{2.815699in}}%
\pgfpathlineto{\pgfqpoint{5.142716in}{2.832061in}}%
\pgfpathlineto{\pgfqpoint{5.150798in}{2.848215in}}%
\pgfpathlineto{\pgfqpoint{5.158872in}{2.864158in}}%
\pgfpathlineto{\pgfqpoint{5.143729in}{2.845685in}}%
\pgfpathlineto{\pgfqpoint{5.128609in}{2.827405in}}%
\pgfpathlineto{\pgfqpoint{5.113512in}{2.809318in}}%
\pgfpathlineto{\pgfqpoint{5.098439in}{2.791423in}}%
\pgfpathlineto{\pgfqpoint{5.090375in}{2.775704in}}%
\pgfpathlineto{\pgfqpoint{5.082303in}{2.759783in}}%
\pgfpathlineto{\pgfqpoint{5.074224in}{2.743664in}}%
\pgfpathlineto{\pgfqpoint{5.066138in}{2.727348in}}%
\pgfpathclose%
\pgfusepath{fill}%
\end{pgfscope}%
\begin{pgfscope}%
\pgfpathrectangle{\pgfqpoint{1.150000in}{0.150000in}}{\pgfqpoint{5.700000in}{5.700000in}}%
\pgfusepath{clip}%
\pgfsetbuttcap%
\pgfsetroundjoin%
\definecolor{currentfill}{rgb}{0.263663,0.237631,0.518762}%
\pgfsetfillcolor{currentfill}%
\pgfsetfillopacity{0.800000}%
\pgfsetlinewidth{0.000000pt}%
\definecolor{currentstroke}{rgb}{0.000000,0.000000,0.000000}%
\pgfsetstrokecolor{currentstroke}%
\pgfsetdash{}{0pt}%
\pgfpathmoveto{\pgfqpoint{4.278638in}{1.351289in}}%
\pgfpathlineto{\pgfqpoint{4.293154in}{1.357715in}}%
\pgfpathlineto{\pgfqpoint{4.307684in}{1.364316in}}%
\pgfpathlineto{\pgfqpoint{4.322228in}{1.371092in}}%
\pgfpathlineto{\pgfqpoint{4.336786in}{1.378042in}}%
\pgfpathlineto{\pgfqpoint{4.345071in}{1.396380in}}%
\pgfpathlineto{\pgfqpoint{4.353354in}{1.414824in}}%
\pgfpathlineto{\pgfqpoint{4.361634in}{1.433366in}}%
\pgfpathlineto{\pgfqpoint{4.369911in}{1.452001in}}%
\pgfpathlineto{\pgfqpoint{4.355346in}{1.444326in}}%
\pgfpathlineto{\pgfqpoint{4.340796in}{1.436827in}}%
\pgfpathlineto{\pgfqpoint{4.326260in}{1.429503in}}%
\pgfpathlineto{\pgfqpoint{4.311738in}{1.422355in}}%
\pgfpathlineto{\pgfqpoint{4.303467in}{1.404432in}}%
\pgfpathlineto{\pgfqpoint{4.295194in}{1.386608in}}%
\pgfpathlineto{\pgfqpoint{4.286918in}{1.368892in}}%
\pgfpathlineto{\pgfqpoint{4.278638in}{1.351289in}}%
\pgfpathclose%
\pgfusepath{fill}%
\end{pgfscope}%
\begin{pgfscope}%
\pgfpathrectangle{\pgfqpoint{1.150000in}{0.150000in}}{\pgfqpoint{5.700000in}{5.700000in}}%
\pgfusepath{clip}%
\pgfsetbuttcap%
\pgfsetroundjoin%
\definecolor{currentfill}{rgb}{0.119423,0.611141,0.538982}%
\pgfsetfillcolor{currentfill}%
\pgfsetfillopacity{0.800000}%
\pgfsetlinewidth{0.000000pt}%
\definecolor{currentstroke}{rgb}{0.000000,0.000000,0.000000}%
\pgfsetstrokecolor{currentstroke}%
\pgfsetdash{}{0pt}%
\pgfpathmoveto{\pgfqpoint{4.908684in}{2.451981in}}%
\pgfpathlineto{\pgfqpoint{4.923623in}{2.467954in}}%
\pgfpathlineto{\pgfqpoint{4.938582in}{2.484115in}}%
\pgfpathlineto{\pgfqpoint{4.953563in}{2.500465in}}%
\pgfpathlineto{\pgfqpoint{4.968566in}{2.517004in}}%
\pgfpathlineto{\pgfqpoint{4.976732in}{2.535502in}}%
\pgfpathlineto{\pgfqpoint{4.984892in}{2.553833in}}%
\pgfpathlineto{\pgfqpoint{4.993046in}{2.571994in}}%
\pgfpathlineto{\pgfqpoint{5.001193in}{2.589983in}}%
\pgfpathlineto{\pgfqpoint{4.986177in}{2.573098in}}%
\pgfpathlineto{\pgfqpoint{4.971183in}{2.556404in}}%
\pgfpathlineto{\pgfqpoint{4.956210in}{2.539899in}}%
\pgfpathlineto{\pgfqpoint{4.941259in}{2.523584in}}%
\pgfpathlineto{\pgfqpoint{4.933124in}{2.505926in}}%
\pgfpathlineto{\pgfqpoint{4.924983in}{2.488104in}}%
\pgfpathlineto{\pgfqpoint{4.916837in}{2.470122in}}%
\pgfpathlineto{\pgfqpoint{4.908684in}{2.451981in}}%
\pgfpathclose%
\pgfusepath{fill}%
\end{pgfscope}%
\begin{pgfscope}%
\pgfpathrectangle{\pgfqpoint{1.150000in}{0.150000in}}{\pgfqpoint{5.700000in}{5.700000in}}%
\pgfusepath{clip}%
\pgfsetbuttcap%
\pgfsetroundjoin%
\definecolor{currentfill}{rgb}{0.279574,0.170599,0.479997}%
\pgfsetfillcolor{currentfill}%
\pgfsetfillopacity{0.800000}%
\pgfsetlinewidth{0.000000pt}%
\definecolor{currentstroke}{rgb}{0.000000,0.000000,0.000000}%
\pgfsetstrokecolor{currentstroke}%
\pgfsetdash{}{0pt}%
\pgfpathmoveto{\pgfqpoint{4.154375in}{1.197572in}}%
\pgfpathlineto{\pgfqpoint{4.168835in}{1.201771in}}%
\pgfpathlineto{\pgfqpoint{4.183306in}{1.206142in}}%
\pgfpathlineto{\pgfqpoint{4.197790in}{1.210687in}}%
\pgfpathlineto{\pgfqpoint{4.212287in}{1.215405in}}%
\pgfpathlineto{\pgfqpoint{4.220592in}{1.231843in}}%
\pgfpathlineto{\pgfqpoint{4.228895in}{1.248452in}}%
\pgfpathlineto{\pgfqpoint{4.237193in}{1.265225in}}%
\pgfpathlineto{\pgfqpoint{4.245489in}{1.282154in}}%
\pgfpathlineto{\pgfqpoint{4.230990in}{1.276654in}}%
\pgfpathlineto{\pgfqpoint{4.216505in}{1.271328in}}%
\pgfpathlineto{\pgfqpoint{4.202032in}{1.266175in}}%
\pgfpathlineto{\pgfqpoint{4.187571in}{1.261196in}}%
\pgfpathlineto{\pgfqpoint{4.179278in}{1.245037in}}%
\pgfpathlineto{\pgfqpoint{4.170981in}{1.229041in}}%
\pgfpathlineto{\pgfqpoint{4.162680in}{1.213217in}}%
\pgfpathlineto{\pgfqpoint{4.154375in}{1.197572in}}%
\pgfpathclose%
\pgfusepath{fill}%
\end{pgfscope}%
\begin{pgfscope}%
\pgfpathrectangle{\pgfqpoint{1.150000in}{0.150000in}}{\pgfqpoint{5.700000in}{5.700000in}}%
\pgfusepath{clip}%
\pgfsetbuttcap%
\pgfsetroundjoin%
\definecolor{currentfill}{rgb}{0.190631,0.407061,0.556089}%
\pgfsetfillcolor{currentfill}%
\pgfsetfillopacity{0.800000}%
\pgfsetlinewidth{0.000000pt}%
\definecolor{currentstroke}{rgb}{0.000000,0.000000,0.000000}%
\pgfsetstrokecolor{currentstroke}%
\pgfsetdash{}{0pt}%
\pgfpathmoveto{\pgfqpoint{4.560547in}{1.800358in}}%
\pgfpathlineto{\pgfqpoint{4.575234in}{1.811521in}}%
\pgfpathlineto{\pgfqpoint{4.589939in}{1.822863in}}%
\pgfpathlineto{\pgfqpoint{4.604661in}{1.834387in}}%
\pgfpathlineto{\pgfqpoint{4.619400in}{1.846091in}}%
\pgfpathlineto{\pgfqpoint{4.627655in}{1.866506in}}%
\pgfpathlineto{\pgfqpoint{4.635906in}{1.886890in}}%
\pgfpathlineto{\pgfqpoint{4.644154in}{1.907236in}}%
\pgfpathlineto{\pgfqpoint{4.652400in}{1.927540in}}%
\pgfpathlineto{\pgfqpoint{4.637646in}{1.915260in}}%
\pgfpathlineto{\pgfqpoint{4.622910in}{1.903162in}}%
\pgfpathlineto{\pgfqpoint{4.608193in}{1.891246in}}%
\pgfpathlineto{\pgfqpoint{4.593493in}{1.879511in}}%
\pgfpathlineto{\pgfqpoint{4.585261in}{1.859769in}}%
\pgfpathlineto{\pgfqpoint{4.577026in}{1.839993in}}%
\pgfpathlineto{\pgfqpoint{4.568788in}{1.820187in}}%
\pgfpathlineto{\pgfqpoint{4.560547in}{1.800358in}}%
\pgfpathclose%
\pgfusepath{fill}%
\end{pgfscope}%
\begin{pgfscope}%
\pgfpathrectangle{\pgfqpoint{1.150000in}{0.150000in}}{\pgfqpoint{5.700000in}{5.700000in}}%
\pgfusepath{clip}%
\pgfsetbuttcap%
\pgfsetroundjoin%
\definecolor{currentfill}{rgb}{0.235526,0.309527,0.542944}%
\pgfsetfillcolor{currentfill}%
\pgfsetfillopacity{0.800000}%
\pgfsetlinewidth{0.000000pt}%
\definecolor{currentstroke}{rgb}{0.000000,0.000000,0.000000}%
\pgfsetstrokecolor{currentstroke}%
\pgfsetdash{}{0pt}%
\pgfpathmoveto{\pgfqpoint{4.402992in}{1.527332in}}%
\pgfpathlineto{\pgfqpoint{4.417581in}{1.535878in}}%
\pgfpathlineto{\pgfqpoint{4.432184in}{1.544601in}}%
\pgfpathlineto{\pgfqpoint{4.446804in}{1.553500in}}%
\pgfpathlineto{\pgfqpoint{4.461439in}{1.562577in}}%
\pgfpathlineto{\pgfqpoint{4.469712in}{1.582252in}}%
\pgfpathlineto{\pgfqpoint{4.477983in}{1.601972in}}%
\pgfpathlineto{\pgfqpoint{4.486252in}{1.621732in}}%
\pgfpathlineto{\pgfqpoint{4.494518in}{1.641526in}}%
\pgfpathlineto{\pgfqpoint{4.479872in}{1.631782in}}%
\pgfpathlineto{\pgfqpoint{4.465242in}{1.622217in}}%
\pgfpathlineto{\pgfqpoint{4.450628in}{1.612830in}}%
\pgfpathlineto{\pgfqpoint{4.436030in}{1.603620in}}%
\pgfpathlineto{\pgfqpoint{4.427775in}{1.584480in}}%
\pgfpathlineto{\pgfqpoint{4.419517in}{1.565382in}}%
\pgfpathlineto{\pgfqpoint{4.411256in}{1.546330in}}%
\pgfpathlineto{\pgfqpoint{4.402992in}{1.527332in}}%
\pgfpathclose%
\pgfusepath{fill}%
\end{pgfscope}%
\begin{pgfscope}%
\pgfpathrectangle{\pgfqpoint{1.150000in}{0.150000in}}{\pgfqpoint{5.700000in}{5.700000in}}%
\pgfusepath{clip}%
\pgfsetbuttcap%
\pgfsetroundjoin%
\definecolor{currentfill}{rgb}{0.281477,0.755203,0.432552}%
\pgfsetfillcolor{currentfill}%
\pgfsetfillopacity{0.800000}%
\pgfsetlinewidth{0.000000pt}%
\definecolor{currentstroke}{rgb}{0.000000,0.000000,0.000000}%
\pgfsetstrokecolor{currentstroke}%
\pgfsetdash{}{0pt}%
\pgfpathmoveto{\pgfqpoint{5.191087in}{2.925798in}}%
\pgfpathlineto{\pgfqpoint{5.206263in}{2.944667in}}%
\pgfpathlineto{\pgfqpoint{5.221464in}{2.963730in}}%
\pgfpathlineto{\pgfqpoint{5.236688in}{2.982987in}}%
\pgfpathlineto{\pgfqpoint{5.251937in}{3.002439in}}%
\pgfpathlineto{\pgfqpoint{5.259979in}{3.017487in}}%
\pgfpathlineto{\pgfqpoint{5.268012in}{3.032307in}}%
\pgfpathlineto{\pgfqpoint{5.276035in}{3.046897in}}%
\pgfpathlineto{\pgfqpoint{5.284049in}{3.061255in}}%
\pgfpathlineto{\pgfqpoint{5.268791in}{3.041637in}}%
\pgfpathlineto{\pgfqpoint{5.253558in}{3.022213in}}%
\pgfpathlineto{\pgfqpoint{5.238349in}{3.002985in}}%
\pgfpathlineto{\pgfqpoint{5.223164in}{2.983951in}}%
\pgfpathlineto{\pgfqpoint{5.215158in}{2.969744in}}%
\pgfpathlineto{\pgfqpoint{5.207143in}{2.955316in}}%
\pgfpathlineto{\pgfqpoint{5.199119in}{2.940666in}}%
\pgfpathlineto{\pgfqpoint{5.191087in}{2.925798in}}%
\pgfpathclose%
\pgfusepath{fill}%
\end{pgfscope}%
\begin{pgfscope}%
\pgfpathrectangle{\pgfqpoint{1.150000in}{0.150000in}}{\pgfqpoint{5.700000in}{5.700000in}}%
\pgfusepath{clip}%
\pgfsetbuttcap%
\pgfsetroundjoin%
\definecolor{currentfill}{rgb}{0.151918,0.500685,0.557587}%
\pgfsetfillcolor{currentfill}%
\pgfsetfillopacity{0.800000}%
\pgfsetlinewidth{0.000000pt}%
\definecolor{currentstroke}{rgb}{0.000000,0.000000,0.000000}%
\pgfsetstrokecolor{currentstroke}%
\pgfsetdash{}{0pt}%
\pgfpathmoveto{\pgfqpoint{4.718241in}{2.087844in}}%
\pgfpathlineto{\pgfqpoint{4.733042in}{2.101360in}}%
\pgfpathlineto{\pgfqpoint{4.747862in}{2.115060in}}%
\pgfpathlineto{\pgfqpoint{4.762702in}{2.128945in}}%
\pgfpathlineto{\pgfqpoint{4.777561in}{2.143014in}}%
\pgfpathlineto{\pgfqpoint{4.785790in}{2.163206in}}%
\pgfpathlineto{\pgfqpoint{4.794015in}{2.183300in}}%
\pgfpathlineto{\pgfqpoint{4.802235in}{2.203290in}}%
\pgfpathlineto{\pgfqpoint{4.810452in}{2.223174in}}%
\pgfpathlineto{\pgfqpoint{4.795577in}{2.208624in}}%
\pgfpathlineto{\pgfqpoint{4.780723in}{2.194260in}}%
\pgfpathlineto{\pgfqpoint{4.765888in}{2.180082in}}%
\pgfpathlineto{\pgfqpoint{4.751073in}{2.166088in}}%
\pgfpathlineto{\pgfqpoint{4.742871in}{2.146671in}}%
\pgfpathlineto{\pgfqpoint{4.734665in}{2.127155in}}%
\pgfpathlineto{\pgfqpoint{4.726455in}{2.107544in}}%
\pgfpathlineto{\pgfqpoint{4.718241in}{2.087844in}}%
\pgfpathclose%
\pgfusepath{fill}%
\end{pgfscope}%
\begin{pgfscope}%
\pgfpathrectangle{\pgfqpoint{1.150000in}{0.150000in}}{\pgfqpoint{5.700000in}{5.700000in}}%
\pgfusepath{clip}%
\pgfsetbuttcap%
\pgfsetroundjoin%
\definecolor{currentfill}{rgb}{0.121148,0.592739,0.544641}%
\pgfsetfillcolor{currentfill}%
\pgfsetfillopacity{0.800000}%
\pgfsetlinewidth{0.000000pt}%
\definecolor{currentstroke}{rgb}{0.000000,0.000000,0.000000}%
\pgfsetstrokecolor{currentstroke}%
\pgfsetdash{}{0pt}%
\pgfpathmoveto{\pgfqpoint{4.876020in}{2.377901in}}%
\pgfpathlineto{\pgfqpoint{4.890945in}{2.393495in}}%
\pgfpathlineto{\pgfqpoint{4.905890in}{2.409278in}}%
\pgfpathlineto{\pgfqpoint{4.920857in}{2.425249in}}%
\pgfpathlineto{\pgfqpoint{4.935844in}{2.441408in}}%
\pgfpathlineto{\pgfqpoint{4.944033in}{2.460542in}}%
\pgfpathlineto{\pgfqpoint{4.952216in}{2.479521in}}%
\pgfpathlineto{\pgfqpoint{4.960394in}{2.498343in}}%
\pgfpathlineto{\pgfqpoint{4.968566in}{2.517004in}}%
\pgfpathlineto{\pgfqpoint{4.953563in}{2.500465in}}%
\pgfpathlineto{\pgfqpoint{4.938582in}{2.484115in}}%
\pgfpathlineto{\pgfqpoint{4.923623in}{2.467954in}}%
\pgfpathlineto{\pgfqpoint{4.908684in}{2.451981in}}%
\pgfpathlineto{\pgfqpoint{4.900527in}{2.433685in}}%
\pgfpathlineto{\pgfqpoint{4.892363in}{2.415238in}}%
\pgfpathlineto{\pgfqpoint{4.884194in}{2.396642in}}%
\pgfpathlineto{\pgfqpoint{4.876020in}{2.377901in}}%
\pgfpathclose%
\pgfusepath{fill}%
\end{pgfscope}%
\begin{pgfscope}%
\pgfpathrectangle{\pgfqpoint{1.150000in}{0.150000in}}{\pgfqpoint{5.700000in}{5.700000in}}%
\pgfusepath{clip}%
\pgfsetbuttcap%
\pgfsetroundjoin%
\definecolor{currentfill}{rgb}{0.270595,0.214069,0.507052}%
\pgfsetfillcolor{currentfill}%
\pgfsetfillopacity{0.800000}%
\pgfsetlinewidth{0.000000pt}%
\definecolor{currentstroke}{rgb}{0.000000,0.000000,0.000000}%
\pgfsetstrokecolor{currentstroke}%
\pgfsetdash{}{0pt}%
\pgfpathmoveto{\pgfqpoint{4.245489in}{1.282154in}}%
\pgfpathlineto{\pgfqpoint{4.260001in}{1.287827in}}%
\pgfpathlineto{\pgfqpoint{4.274526in}{1.293675in}}%
\pgfpathlineto{\pgfqpoint{4.289064in}{1.299696in}}%
\pgfpathlineto{\pgfqpoint{4.303616in}{1.305891in}}%
\pgfpathlineto{\pgfqpoint{4.311912in}{1.323735in}}%
\pgfpathlineto{\pgfqpoint{4.320206in}{1.341713in}}%
\pgfpathlineto{\pgfqpoint{4.328497in}{1.359818in}}%
\pgfpathlineto{\pgfqpoint{4.336786in}{1.378042in}}%
\pgfpathlineto{\pgfqpoint{4.322228in}{1.371092in}}%
\pgfpathlineto{\pgfqpoint{4.307684in}{1.364316in}}%
\pgfpathlineto{\pgfqpoint{4.293154in}{1.357715in}}%
\pgfpathlineto{\pgfqpoint{4.278638in}{1.351289in}}%
\pgfpathlineto{\pgfqpoint{4.270356in}{1.333807in}}%
\pgfpathlineto{\pgfqpoint{4.262070in}{1.316452in}}%
\pgfpathlineto{\pgfqpoint{4.253781in}{1.299232in}}%
\pgfpathlineto{\pgfqpoint{4.245489in}{1.282154in}}%
\pgfpathclose%
\pgfusepath{fill}%
\end{pgfscope}%
\begin{pgfscope}%
\pgfpathrectangle{\pgfqpoint{1.150000in}{0.150000in}}{\pgfqpoint{5.700000in}{5.700000in}}%
\pgfusepath{clip}%
\pgfsetbuttcap%
\pgfsetroundjoin%
\definecolor{currentfill}{rgb}{0.153894,0.680203,0.504172}%
\pgfsetfillcolor{currentfill}%
\pgfsetfillopacity{0.800000}%
\pgfsetlinewidth{0.000000pt}%
\definecolor{currentstroke}{rgb}{0.000000,0.000000,0.000000}%
\pgfsetstrokecolor{currentstroke}%
\pgfsetdash{}{0pt}%
\pgfpathmoveto{\pgfqpoint{5.033720in}{2.660153in}}%
\pgfpathlineto{\pgfqpoint{5.048771in}{2.677537in}}%
\pgfpathlineto{\pgfqpoint{5.063845in}{2.695112in}}%
\pgfpathlineto{\pgfqpoint{5.078942in}{2.712879in}}%
\pgfpathlineto{\pgfqpoint{5.094061in}{2.730838in}}%
\pgfpathlineto{\pgfqpoint{5.102189in}{2.748210in}}%
\pgfpathlineto{\pgfqpoint{5.110309in}{2.765384in}}%
\pgfpathlineto{\pgfqpoint{5.118422in}{2.782359in}}%
\pgfpathlineto{\pgfqpoint{5.126528in}{2.799131in}}%
\pgfpathlineto{\pgfqpoint{5.111396in}{2.780897in}}%
\pgfpathlineto{\pgfqpoint{5.096287in}{2.762855in}}%
\pgfpathlineto{\pgfqpoint{5.081201in}{2.745005in}}%
\pgfpathlineto{\pgfqpoint{5.066138in}{2.727348in}}%
\pgfpathlineto{\pgfqpoint{5.058044in}{2.710836in}}%
\pgfpathlineto{\pgfqpoint{5.049943in}{2.694132in}}%
\pgfpathlineto{\pgfqpoint{5.041835in}{2.677236in}}%
\pgfpathlineto{\pgfqpoint{5.033720in}{2.660153in}}%
\pgfpathclose%
\pgfusepath{fill}%
\end{pgfscope}%
\begin{pgfscope}%
\pgfpathrectangle{\pgfqpoint{1.150000in}{0.150000in}}{\pgfqpoint{5.700000in}{5.700000in}}%
\pgfusepath{clip}%
\pgfsetbuttcap%
\pgfsetroundjoin%
\definecolor{currentfill}{rgb}{0.555484,0.840254,0.269281}%
\pgfsetfillcolor{currentfill}%
\pgfsetfillopacity{0.800000}%
\pgfsetlinewidth{0.000000pt}%
\definecolor{currentstroke}{rgb}{0.000000,0.000000,0.000000}%
\pgfsetstrokecolor{currentstroke}%
\pgfsetdash{}{0pt}%
\pgfpathmoveto{\pgfqpoint{5.440792in}{3.296781in}}%
\pgfpathlineto{\pgfqpoint{5.456192in}{3.317657in}}%
\pgfpathlineto{\pgfqpoint{5.471619in}{3.338731in}}%
\pgfpathlineto{\pgfqpoint{5.487071in}{3.360004in}}%
\pgfpathlineto{\pgfqpoint{5.494958in}{3.371350in}}%
\pgfpathlineto{\pgfqpoint{5.502833in}{3.382442in}}%
\pgfpathlineto{\pgfqpoint{5.510696in}{3.393281in}}%
\pgfpathlineto{\pgfqpoint{5.518548in}{3.403867in}}%
\pgfpathlineto{\pgfqpoint{5.503093in}{3.382579in}}%
\pgfpathlineto{\pgfqpoint{5.487665in}{3.361489in}}%
\pgfpathlineto{\pgfqpoint{5.472263in}{3.340598in}}%
\pgfpathlineto{\pgfqpoint{5.464412in}{3.330013in}}%
\pgfpathlineto{\pgfqpoint{5.456550in}{3.319182in}}%
\pgfpathlineto{\pgfqpoint{5.448676in}{3.308105in}}%
\pgfpathlineto{\pgfqpoint{5.440792in}{3.296781in}}%
\pgfpathclose%
\pgfusepath{fill}%
\end{pgfscope}%
\begin{pgfscope}%
\pgfpathrectangle{\pgfqpoint{1.150000in}{0.150000in}}{\pgfqpoint{5.700000in}{5.700000in}}%
\pgfusepath{clip}%
\pgfsetbuttcap%
\pgfsetroundjoin%
\definecolor{currentfill}{rgb}{0.201239,0.383670,0.554294}%
\pgfsetfillcolor{currentfill}%
\pgfsetfillopacity{0.800000}%
\pgfsetlinewidth{0.000000pt}%
\definecolor{currentstroke}{rgb}{0.000000,0.000000,0.000000}%
\pgfsetstrokecolor{currentstroke}%
\pgfsetdash{}{0pt}%
\pgfpathmoveto{\pgfqpoint{4.527555in}{1.720911in}}%
\pgfpathlineto{\pgfqpoint{4.542229in}{1.731469in}}%
\pgfpathlineto{\pgfqpoint{4.556920in}{1.742207in}}%
\pgfpathlineto{\pgfqpoint{4.571629in}{1.753123in}}%
\pgfpathlineto{\pgfqpoint{4.586355in}{1.764220in}}%
\pgfpathlineto{\pgfqpoint{4.594620in}{1.784708in}}%
\pgfpathlineto{\pgfqpoint{4.602883in}{1.805187in}}%
\pgfpathlineto{\pgfqpoint{4.611143in}{1.825649in}}%
\pgfpathlineto{\pgfqpoint{4.619400in}{1.846091in}}%
\pgfpathlineto{\pgfqpoint{4.604661in}{1.834387in}}%
\pgfpathlineto{\pgfqpoint{4.589939in}{1.822863in}}%
\pgfpathlineto{\pgfqpoint{4.575234in}{1.811521in}}%
\pgfpathlineto{\pgfqpoint{4.560547in}{1.800358in}}%
\pgfpathlineto{\pgfqpoint{4.552303in}{1.780510in}}%
\pgfpathlineto{\pgfqpoint{4.544057in}{1.760650in}}%
\pgfpathlineto{\pgfqpoint{4.535807in}{1.740781in}}%
\pgfpathlineto{\pgfqpoint{4.527555in}{1.720911in}}%
\pgfpathclose%
\pgfusepath{fill}%
\end{pgfscope}%
\begin{pgfscope}%
\pgfpathrectangle{\pgfqpoint{1.150000in}{0.150000in}}{\pgfqpoint{5.700000in}{5.700000in}}%
\pgfusepath{clip}%
\pgfsetbuttcap%
\pgfsetroundjoin%
\definecolor{currentfill}{rgb}{0.421908,0.805774,0.351910}%
\pgfsetfillcolor{currentfill}%
\pgfsetfillopacity{0.800000}%
\pgfsetlinewidth{0.000000pt}%
\definecolor{currentstroke}{rgb}{0.000000,0.000000,0.000000}%
\pgfsetstrokecolor{currentstroke}%
\pgfsetdash{}{0pt}%
\pgfpathmoveto{\pgfqpoint{5.316009in}{3.116359in}}%
\pgfpathlineto{\pgfqpoint{5.331298in}{3.136302in}}%
\pgfpathlineto{\pgfqpoint{5.346613in}{3.156442in}}%
\pgfpathlineto{\pgfqpoint{5.361953in}{3.176778in}}%
\pgfpathlineto{\pgfqpoint{5.377318in}{3.197311in}}%
\pgfpathlineto{\pgfqpoint{5.385290in}{3.210606in}}%
\pgfpathlineto{\pgfqpoint{5.393251in}{3.223656in}}%
\pgfpathlineto{\pgfqpoint{5.401201in}{3.236460in}}%
\pgfpathlineto{\pgfqpoint{5.409141in}{3.249017in}}%
\pgfpathlineto{\pgfqpoint{5.393770in}{3.228393in}}%
\pgfpathlineto{\pgfqpoint{5.378425in}{3.207965in}}%
\pgfpathlineto{\pgfqpoint{5.363105in}{3.187735in}}%
\pgfpathlineto{\pgfqpoint{5.347810in}{3.167700in}}%
\pgfpathlineto{\pgfqpoint{5.339875in}{3.155220in}}%
\pgfpathlineto{\pgfqpoint{5.331930in}{3.142503in}}%
\pgfpathlineto{\pgfqpoint{5.323974in}{3.129549in}}%
\pgfpathlineto{\pgfqpoint{5.316009in}{3.116359in}}%
\pgfpathclose%
\pgfusepath{fill}%
\end{pgfscope}%
\begin{pgfscope}%
\pgfpathrectangle{\pgfqpoint{1.150000in}{0.150000in}}{\pgfqpoint{5.700000in}{5.700000in}}%
\pgfusepath{clip}%
\pgfsetbuttcap%
\pgfsetroundjoin%
\definecolor{currentfill}{rgb}{0.281412,0.155834,0.469201}%
\pgfsetfillcolor{currentfill}%
\pgfsetfillopacity{0.800000}%
\pgfsetlinewidth{0.000000pt}%
\definecolor{currentstroke}{rgb}{0.000000,0.000000,0.000000}%
\pgfsetstrokecolor{currentstroke}%
\pgfsetdash{}{0pt}%
\pgfpathmoveto{\pgfqpoint{4.121115in}{1.136932in}}%
\pgfpathlineto{\pgfqpoint{4.135576in}{1.140321in}}%
\pgfpathlineto{\pgfqpoint{4.150048in}{1.143881in}}%
\pgfpathlineto{\pgfqpoint{4.164532in}{1.147614in}}%
\pgfpathlineto{\pgfqpoint{4.179027in}{1.151518in}}%
\pgfpathlineto{\pgfqpoint{4.187348in}{1.167195in}}%
\pgfpathlineto{\pgfqpoint{4.195664in}{1.183073in}}%
\pgfpathlineto{\pgfqpoint{4.203977in}{1.199146in}}%
\pgfpathlineto{\pgfqpoint{4.212287in}{1.215405in}}%
\pgfpathlineto{\pgfqpoint{4.197790in}{1.210687in}}%
\pgfpathlineto{\pgfqpoint{4.183306in}{1.206142in}}%
\pgfpathlineto{\pgfqpoint{4.168835in}{1.201771in}}%
\pgfpathlineto{\pgfqpoint{4.154375in}{1.197572in}}%
\pgfpathlineto{\pgfqpoint{4.146066in}{1.182113in}}%
\pgfpathlineto{\pgfqpoint{4.137753in}{1.166848in}}%
\pgfpathlineto{\pgfqpoint{4.129436in}{1.151785in}}%
\pgfpathlineto{\pgfqpoint{4.121115in}{1.136932in}}%
\pgfpathclose%
\pgfusepath{fill}%
\end{pgfscope}%
\begin{pgfscope}%
\pgfpathrectangle{\pgfqpoint{1.150000in}{0.150000in}}{\pgfqpoint{5.700000in}{5.700000in}}%
\pgfusepath{clip}%
\pgfsetbuttcap%
\pgfsetroundjoin%
\definecolor{currentfill}{rgb}{0.244972,0.287675,0.537260}%
\pgfsetfillcolor{currentfill}%
\pgfsetfillopacity{0.800000}%
\pgfsetlinewidth{0.000000pt}%
\definecolor{currentstroke}{rgb}{0.000000,0.000000,0.000000}%
\pgfsetstrokecolor{currentstroke}%
\pgfsetdash{}{0pt}%
\pgfpathmoveto{\pgfqpoint{4.369911in}{1.452001in}}%
\pgfpathlineto{\pgfqpoint{4.384491in}{1.459852in}}%
\pgfpathlineto{\pgfqpoint{4.399085in}{1.467879in}}%
\pgfpathlineto{\pgfqpoint{4.413695in}{1.476082in}}%
\pgfpathlineto{\pgfqpoint{4.428319in}{1.484461in}}%
\pgfpathlineto{\pgfqpoint{4.436603in}{1.503889in}}%
\pgfpathlineto{\pgfqpoint{4.444884in}{1.523389in}}%
\pgfpathlineto{\pgfqpoint{4.453163in}{1.542954in}}%
\pgfpathlineto{\pgfqpoint{4.461439in}{1.562577in}}%
\pgfpathlineto{\pgfqpoint{4.446804in}{1.553500in}}%
\pgfpathlineto{\pgfqpoint{4.432184in}{1.544601in}}%
\pgfpathlineto{\pgfqpoint{4.417581in}{1.535878in}}%
\pgfpathlineto{\pgfqpoint{4.402992in}{1.527332in}}%
\pgfpathlineto{\pgfqpoint{4.394726in}{1.508394in}}%
\pgfpathlineto{\pgfqpoint{4.386457in}{1.489522in}}%
\pgfpathlineto{\pgfqpoint{4.378185in}{1.470722in}}%
\pgfpathlineto{\pgfqpoint{4.369911in}{1.452001in}}%
\pgfpathclose%
\pgfusepath{fill}%
\end{pgfscope}%
\begin{pgfscope}%
\pgfpathrectangle{\pgfqpoint{1.150000in}{0.150000in}}{\pgfqpoint{5.700000in}{5.700000in}}%
\pgfusepath{clip}%
\pgfsetbuttcap%
\pgfsetroundjoin%
\definecolor{currentfill}{rgb}{0.160665,0.478540,0.558115}%
\pgfsetfillcolor{currentfill}%
\pgfsetfillopacity{0.800000}%
\pgfsetlinewidth{0.000000pt}%
\definecolor{currentstroke}{rgb}{0.000000,0.000000,0.000000}%
\pgfsetstrokecolor{currentstroke}%
\pgfsetdash{}{0pt}%
\pgfpathmoveto{\pgfqpoint{4.685348in}{2.008229in}}%
\pgfpathlineto{\pgfqpoint{4.700135in}{2.021234in}}%
\pgfpathlineto{\pgfqpoint{4.714940in}{2.034423in}}%
\pgfpathlineto{\pgfqpoint{4.729764in}{2.047795in}}%
\pgfpathlineto{\pgfqpoint{4.744608in}{2.061351in}}%
\pgfpathlineto{\pgfqpoint{4.752852in}{2.081892in}}%
\pgfpathlineto{\pgfqpoint{4.761092in}{2.102352in}}%
\pgfpathlineto{\pgfqpoint{4.769328in}{2.122728in}}%
\pgfpathlineto{\pgfqpoint{4.777561in}{2.143014in}}%
\pgfpathlineto{\pgfqpoint{4.762702in}{2.128945in}}%
\pgfpathlineto{\pgfqpoint{4.747862in}{2.115060in}}%
\pgfpathlineto{\pgfqpoint{4.733042in}{2.101360in}}%
\pgfpathlineto{\pgfqpoint{4.718241in}{2.087844in}}%
\pgfpathlineto{\pgfqpoint{4.710023in}{2.068057in}}%
\pgfpathlineto{\pgfqpoint{4.701802in}{2.048190in}}%
\pgfpathlineto{\pgfqpoint{4.693577in}{2.028245in}}%
\pgfpathlineto{\pgfqpoint{4.685348in}{2.008229in}}%
\pgfpathclose%
\pgfusepath{fill}%
\end{pgfscope}%
\begin{pgfscope}%
\pgfpathrectangle{\pgfqpoint{1.150000in}{0.150000in}}{\pgfqpoint{5.700000in}{5.700000in}}%
\pgfusepath{clip}%
\pgfsetbuttcap%
\pgfsetroundjoin%
\definecolor{currentfill}{rgb}{0.125394,0.574318,0.549086}%
\pgfsetfillcolor{currentfill}%
\pgfsetfillopacity{0.800000}%
\pgfsetlinewidth{0.000000pt}%
\definecolor{currentstroke}{rgb}{0.000000,0.000000,0.000000}%
\pgfsetstrokecolor{currentstroke}%
\pgfsetdash{}{0pt}%
\pgfpathmoveto{\pgfqpoint{4.843274in}{2.301554in}}%
\pgfpathlineto{\pgfqpoint{4.858184in}{2.316736in}}%
\pgfpathlineto{\pgfqpoint{4.873114in}{2.332106in}}%
\pgfpathlineto{\pgfqpoint{4.888065in}{2.347662in}}%
\pgfpathlineto{\pgfqpoint{4.903038in}{2.363407in}}%
\pgfpathlineto{\pgfqpoint{4.911247in}{2.383120in}}%
\pgfpathlineto{\pgfqpoint{4.919451in}{2.402694in}}%
\pgfpathlineto{\pgfqpoint{4.927650in}{2.422125in}}%
\pgfpathlineto{\pgfqpoint{4.935844in}{2.441408in}}%
\pgfpathlineto{\pgfqpoint{4.920857in}{2.425249in}}%
\pgfpathlineto{\pgfqpoint{4.905890in}{2.409278in}}%
\pgfpathlineto{\pgfqpoint{4.890945in}{2.393495in}}%
\pgfpathlineto{\pgfqpoint{4.876020in}{2.377901in}}%
\pgfpathlineto{\pgfqpoint{4.867841in}{2.359018in}}%
\pgfpathlineto{\pgfqpoint{4.859657in}{2.339996in}}%
\pgfpathlineto{\pgfqpoint{4.851468in}{2.320841in}}%
\pgfpathlineto{\pgfqpoint{4.843274in}{2.301554in}}%
\pgfpathclose%
\pgfusepath{fill}%
\end{pgfscope}%
\begin{pgfscope}%
\pgfpathrectangle{\pgfqpoint{1.150000in}{0.150000in}}{\pgfqpoint{5.700000in}{5.700000in}}%
\pgfusepath{clip}%
\pgfsetbuttcap%
\pgfsetroundjoin%
\definecolor{currentfill}{rgb}{0.259857,0.745492,0.444467}%
\pgfsetfillcolor{currentfill}%
\pgfsetfillopacity{0.800000}%
\pgfsetlinewidth{0.000000pt}%
\definecolor{currentstroke}{rgb}{0.000000,0.000000,0.000000}%
\pgfsetstrokecolor{currentstroke}%
\pgfsetdash{}{0pt}%
\pgfpathmoveto{\pgfqpoint{5.158872in}{2.864158in}}%
\pgfpathlineto{\pgfqpoint{5.174039in}{2.882825in}}%
\pgfpathlineto{\pgfqpoint{5.189229in}{2.901685in}}%
\pgfpathlineto{\pgfqpoint{5.204444in}{2.920739in}}%
\pgfpathlineto{\pgfqpoint{5.219682in}{2.939988in}}%
\pgfpathlineto{\pgfqpoint{5.227759in}{2.955937in}}%
\pgfpathlineto{\pgfqpoint{5.235827in}{2.971662in}}%
\pgfpathlineto{\pgfqpoint{5.243887in}{2.987164in}}%
\pgfpathlineto{\pgfqpoint{5.251937in}{3.002439in}}%
\pgfpathlineto{\pgfqpoint{5.236688in}{2.982987in}}%
\pgfpathlineto{\pgfqpoint{5.221464in}{2.963730in}}%
\pgfpathlineto{\pgfqpoint{5.206263in}{2.944667in}}%
\pgfpathlineto{\pgfqpoint{5.191087in}{2.925798in}}%
\pgfpathlineto{\pgfqpoint{5.183046in}{2.910712in}}%
\pgfpathlineto{\pgfqpoint{5.174996in}{2.895408in}}%
\pgfpathlineto{\pgfqpoint{5.166938in}{2.879890in}}%
\pgfpathlineto{\pgfqpoint{5.158872in}{2.864158in}}%
\pgfpathclose%
\pgfusepath{fill}%
\end{pgfscope}%
\begin{pgfscope}%
\pgfpathrectangle{\pgfqpoint{1.150000in}{0.150000in}}{\pgfqpoint{5.700000in}{5.700000in}}%
\pgfusepath{clip}%
\pgfsetbuttcap%
\pgfsetroundjoin%
\definecolor{currentfill}{rgb}{0.212395,0.359683,0.551710}%
\pgfsetfillcolor{currentfill}%
\pgfsetfillopacity{0.800000}%
\pgfsetlinewidth{0.000000pt}%
\definecolor{currentstroke}{rgb}{0.000000,0.000000,0.000000}%
\pgfsetstrokecolor{currentstroke}%
\pgfsetdash{}{0pt}%
\pgfpathmoveto{\pgfqpoint{4.494518in}{1.641526in}}%
\pgfpathlineto{\pgfqpoint{4.509180in}{1.651447in}}%
\pgfpathlineto{\pgfqpoint{4.523859in}{1.661547in}}%
\pgfpathlineto{\pgfqpoint{4.538554in}{1.671826in}}%
\pgfpathlineto{\pgfqpoint{4.553266in}{1.682283in}}%
\pgfpathlineto{\pgfqpoint{4.561542in}{1.702753in}}%
\pgfpathlineto{\pgfqpoint{4.569816in}{1.723237in}}%
\pgfpathlineto{\pgfqpoint{4.578086in}{1.743728in}}%
\pgfpathlineto{\pgfqpoint{4.586355in}{1.764220in}}%
\pgfpathlineto{\pgfqpoint{4.571629in}{1.753123in}}%
\pgfpathlineto{\pgfqpoint{4.556920in}{1.742207in}}%
\pgfpathlineto{\pgfqpoint{4.542229in}{1.731469in}}%
\pgfpathlineto{\pgfqpoint{4.527555in}{1.720911in}}%
\pgfpathlineto{\pgfqpoint{4.519299in}{1.701045in}}%
\pgfpathlineto{\pgfqpoint{4.511041in}{1.681188in}}%
\pgfpathlineto{\pgfqpoint{4.502781in}{1.661346in}}%
\pgfpathlineto{\pgfqpoint{4.494518in}{1.641526in}}%
\pgfpathclose%
\pgfusepath{fill}%
\end{pgfscope}%
\begin{pgfscope}%
\pgfpathrectangle{\pgfqpoint{1.150000in}{0.150000in}}{\pgfqpoint{5.700000in}{5.700000in}}%
\pgfusepath{clip}%
\pgfsetbuttcap%
\pgfsetroundjoin%
\definecolor{currentfill}{rgb}{0.140210,0.665859,0.513427}%
\pgfsetfillcolor{currentfill}%
\pgfsetfillopacity{0.800000}%
\pgfsetlinewidth{0.000000pt}%
\definecolor{currentstroke}{rgb}{0.000000,0.000000,0.000000}%
\pgfsetstrokecolor{currentstroke}%
\pgfsetdash{}{0pt}%
\pgfpathmoveto{\pgfqpoint{5.001193in}{2.589983in}}%
\pgfpathlineto{\pgfqpoint{5.016232in}{2.607057in}}%
\pgfpathlineto{\pgfqpoint{5.031292in}{2.624322in}}%
\pgfpathlineto{\pgfqpoint{5.046375in}{2.641778in}}%
\pgfpathlineto{\pgfqpoint{5.061481in}{2.659425in}}%
\pgfpathlineto{\pgfqpoint{5.069636in}{2.677562in}}%
\pgfpathlineto{\pgfqpoint{5.077785in}{2.695512in}}%
\pgfpathlineto{\pgfqpoint{5.085927in}{2.713271in}}%
\pgfpathlineto{\pgfqpoint{5.094061in}{2.730838in}}%
\pgfpathlineto{\pgfqpoint{5.078942in}{2.712879in}}%
\pgfpathlineto{\pgfqpoint{5.063845in}{2.695112in}}%
\pgfpathlineto{\pgfqpoint{5.048771in}{2.677537in}}%
\pgfpathlineto{\pgfqpoint{5.033720in}{2.660153in}}%
\pgfpathlineto{\pgfqpoint{5.025598in}{2.642883in}}%
\pgfpathlineto{\pgfqpoint{5.017470in}{2.625430in}}%
\pgfpathlineto{\pgfqpoint{5.009335in}{2.607795in}}%
\pgfpathlineto{\pgfqpoint{5.001193in}{2.589983in}}%
\pgfpathclose%
\pgfusepath{fill}%
\end{pgfscope}%
\begin{pgfscope}%
\pgfpathrectangle{\pgfqpoint{1.150000in}{0.150000in}}{\pgfqpoint{5.700000in}{5.700000in}}%
\pgfusepath{clip}%
\pgfsetbuttcap%
\pgfsetroundjoin%
\definecolor{currentfill}{rgb}{0.275191,0.194905,0.496005}%
\pgfsetfillcolor{currentfill}%
\pgfsetfillopacity{0.800000}%
\pgfsetlinewidth{0.000000pt}%
\definecolor{currentstroke}{rgb}{0.000000,0.000000,0.000000}%
\pgfsetstrokecolor{currentstroke}%
\pgfsetdash{}{0pt}%
\pgfpathmoveto{\pgfqpoint{4.212287in}{1.215405in}}%
\pgfpathlineto{\pgfqpoint{4.226795in}{1.220295in}}%
\pgfpathlineto{\pgfqpoint{4.241317in}{1.225358in}}%
\pgfpathlineto{\pgfqpoint{4.255851in}{1.230595in}}%
\pgfpathlineto{\pgfqpoint{4.270399in}{1.236004in}}%
\pgfpathlineto{\pgfqpoint{4.278707in}{1.253238in}}%
\pgfpathlineto{\pgfqpoint{4.287013in}{1.270635in}}%
\pgfpathlineto{\pgfqpoint{4.295316in}{1.288189in}}%
\pgfpathlineto{\pgfqpoint{4.303616in}{1.305891in}}%
\pgfpathlineto{\pgfqpoint{4.289064in}{1.299696in}}%
\pgfpathlineto{\pgfqpoint{4.274526in}{1.293675in}}%
\pgfpathlineto{\pgfqpoint{4.260001in}{1.287827in}}%
\pgfpathlineto{\pgfqpoint{4.245489in}{1.282154in}}%
\pgfpathlineto{\pgfqpoint{4.237193in}{1.265225in}}%
\pgfpathlineto{\pgfqpoint{4.228895in}{1.248452in}}%
\pgfpathlineto{\pgfqpoint{4.220592in}{1.231843in}}%
\pgfpathlineto{\pgfqpoint{4.212287in}{1.215405in}}%
\pgfpathclose%
\pgfusepath{fill}%
\end{pgfscope}%
\begin{pgfscope}%
\pgfpathrectangle{\pgfqpoint{1.150000in}{0.150000in}}{\pgfqpoint{5.700000in}{5.700000in}}%
\pgfusepath{clip}%
\pgfsetbuttcap%
\pgfsetroundjoin%
\definecolor{currentfill}{rgb}{0.253935,0.265254,0.529983}%
\pgfsetfillcolor{currentfill}%
\pgfsetfillopacity{0.800000}%
\pgfsetlinewidth{0.000000pt}%
\definecolor{currentstroke}{rgb}{0.000000,0.000000,0.000000}%
\pgfsetstrokecolor{currentstroke}%
\pgfsetdash{}{0pt}%
\pgfpathmoveto{\pgfqpoint{4.336786in}{1.378042in}}%
\pgfpathlineto{\pgfqpoint{4.351358in}{1.385168in}}%
\pgfpathlineto{\pgfqpoint{4.365944in}{1.392468in}}%
\pgfpathlineto{\pgfqpoint{4.380545in}{1.399943in}}%
\pgfpathlineto{\pgfqpoint{4.395161in}{1.407594in}}%
\pgfpathlineto{\pgfqpoint{4.403454in}{1.426670in}}%
\pgfpathlineto{\pgfqpoint{4.411745in}{1.445844in}}%
\pgfpathlineto{\pgfqpoint{4.420033in}{1.465110in}}%
\pgfpathlineto{\pgfqpoint{4.428319in}{1.484461in}}%
\pgfpathlineto{\pgfqpoint{4.413695in}{1.476082in}}%
\pgfpathlineto{\pgfqpoint{4.399085in}{1.467879in}}%
\pgfpathlineto{\pgfqpoint{4.384491in}{1.459852in}}%
\pgfpathlineto{\pgfqpoint{4.369911in}{1.452001in}}%
\pgfpathlineto{\pgfqpoint{4.361634in}{1.433366in}}%
\pgfpathlineto{\pgfqpoint{4.353354in}{1.414824in}}%
\pgfpathlineto{\pgfqpoint{4.345071in}{1.396380in}}%
\pgfpathlineto{\pgfqpoint{4.336786in}{1.378042in}}%
\pgfpathclose%
\pgfusepath{fill}%
\end{pgfscope}%
\begin{pgfscope}%
\pgfpathrectangle{\pgfqpoint{1.150000in}{0.150000in}}{\pgfqpoint{5.700000in}{5.700000in}}%
\pgfusepath{clip}%
\pgfsetbuttcap%
\pgfsetroundjoin%
\definecolor{currentfill}{rgb}{0.168126,0.459988,0.558082}%
\pgfsetfillcolor{currentfill}%
\pgfsetfillopacity{0.800000}%
\pgfsetlinewidth{0.000000pt}%
\definecolor{currentstroke}{rgb}{0.000000,0.000000,0.000000}%
\pgfsetstrokecolor{currentstroke}%
\pgfsetdash{}{0pt}%
\pgfpathmoveto{\pgfqpoint{4.652400in}{1.927540in}}%
\pgfpathlineto{\pgfqpoint{4.667171in}{1.940001in}}%
\pgfpathlineto{\pgfqpoint{4.681962in}{1.952645in}}%
\pgfpathlineto{\pgfqpoint{4.696771in}{1.965471in}}%
\pgfpathlineto{\pgfqpoint{4.711599in}{1.978480in}}%
\pgfpathlineto{\pgfqpoint{4.719856in}{1.999294in}}%
\pgfpathlineto{\pgfqpoint{4.728110in}{2.020047in}}%
\pgfpathlineto{\pgfqpoint{4.736361in}{2.040734in}}%
\pgfpathlineto{\pgfqpoint{4.744608in}{2.061351in}}%
\pgfpathlineto{\pgfqpoint{4.729764in}{2.047795in}}%
\pgfpathlineto{\pgfqpoint{4.714940in}{2.034423in}}%
\pgfpathlineto{\pgfqpoint{4.700135in}{2.021234in}}%
\pgfpathlineto{\pgfqpoint{4.685348in}{2.008229in}}%
\pgfpathlineto{\pgfqpoint{4.677116in}{1.988145in}}%
\pgfpathlineto{\pgfqpoint{4.668880in}{1.967999in}}%
\pgfpathlineto{\pgfqpoint{4.660642in}{1.947796in}}%
\pgfpathlineto{\pgfqpoint{4.652400in}{1.927540in}}%
\pgfpathclose%
\pgfusepath{fill}%
\end{pgfscope}%
\begin{pgfscope}%
\pgfpathrectangle{\pgfqpoint{1.150000in}{0.150000in}}{\pgfqpoint{5.700000in}{5.700000in}}%
\pgfusepath{clip}%
\pgfsetbuttcap%
\pgfsetroundjoin%
\definecolor{currentfill}{rgb}{0.395174,0.797475,0.367757}%
\pgfsetfillcolor{currentfill}%
\pgfsetfillopacity{0.800000}%
\pgfsetlinewidth{0.000000pt}%
\definecolor{currentstroke}{rgb}{0.000000,0.000000,0.000000}%
\pgfsetstrokecolor{currentstroke}%
\pgfsetdash{}{0pt}%
\pgfpathmoveto{\pgfqpoint{5.284049in}{3.061255in}}%
\pgfpathlineto{\pgfqpoint{5.299331in}{3.081069in}}%
\pgfpathlineto{\pgfqpoint{5.314638in}{3.101079in}}%
\pgfpathlineto{\pgfqpoint{5.329971in}{3.121286in}}%
\pgfpathlineto{\pgfqpoint{5.345328in}{3.141690in}}%
\pgfpathlineto{\pgfqpoint{5.353341in}{3.155959in}}%
\pgfpathlineto{\pgfqpoint{5.361343in}{3.169987in}}%
\pgfpathlineto{\pgfqpoint{5.369336in}{3.183771in}}%
\pgfpathlineto{\pgfqpoint{5.377318in}{3.197311in}}%
\pgfpathlineto{\pgfqpoint{5.361953in}{3.176778in}}%
\pgfpathlineto{\pgfqpoint{5.346613in}{3.156442in}}%
\pgfpathlineto{\pgfqpoint{5.331298in}{3.136302in}}%
\pgfpathlineto{\pgfqpoint{5.316009in}{3.116359in}}%
\pgfpathlineto{\pgfqpoint{5.308033in}{3.102935in}}%
\pgfpathlineto{\pgfqpoint{5.300048in}{3.089275in}}%
\pgfpathlineto{\pgfqpoint{5.292053in}{3.075382in}}%
\pgfpathlineto{\pgfqpoint{5.284049in}{3.061255in}}%
\pgfpathclose%
\pgfusepath{fill}%
\end{pgfscope}%
\begin{pgfscope}%
\pgfpathrectangle{\pgfqpoint{1.150000in}{0.150000in}}{\pgfqpoint{5.700000in}{5.700000in}}%
\pgfusepath{clip}%
\pgfsetbuttcap%
\pgfsetroundjoin%
\definecolor{currentfill}{rgb}{0.535621,0.835785,0.281908}%
\pgfsetfillcolor{currentfill}%
\pgfsetfillopacity{0.800000}%
\pgfsetlinewidth{0.000000pt}%
\definecolor{currentstroke}{rgb}{0.000000,0.000000,0.000000}%
\pgfsetstrokecolor{currentstroke}%
\pgfsetdash{}{0pt}%
\pgfpathmoveto{\pgfqpoint{5.409141in}{3.249017in}}%
\pgfpathlineto{\pgfqpoint{5.424538in}{3.269840in}}%
\pgfpathlineto{\pgfqpoint{5.439960in}{3.290860in}}%
\pgfpathlineto{\pgfqpoint{5.455409in}{3.312079in}}%
\pgfpathlineto{\pgfqpoint{5.463342in}{3.324441in}}%
\pgfpathlineto{\pgfqpoint{5.471263in}{3.336549in}}%
\pgfpathlineto{\pgfqpoint{5.479173in}{3.348403in}}%
\pgfpathlineto{\pgfqpoint{5.487071in}{3.360004in}}%
\pgfpathlineto{\pgfqpoint{5.471619in}{3.338731in}}%
\pgfpathlineto{\pgfqpoint{5.456192in}{3.317657in}}%
\pgfpathlineto{\pgfqpoint{5.440792in}{3.296781in}}%
\pgfpathlineto{\pgfqpoint{5.432896in}{3.285210in}}%
\pgfpathlineto{\pgfqpoint{5.424989in}{3.273393in}}%
\pgfpathlineto{\pgfqpoint{5.417070in}{3.261328in}}%
\pgfpathlineto{\pgfqpoint{5.409141in}{3.249017in}}%
\pgfpathclose%
\pgfusepath{fill}%
\end{pgfscope}%
\begin{pgfscope}%
\pgfpathrectangle{\pgfqpoint{1.150000in}{0.150000in}}{\pgfqpoint{5.700000in}{5.700000in}}%
\pgfusepath{clip}%
\pgfsetbuttcap%
\pgfsetroundjoin%
\definecolor{currentfill}{rgb}{0.131172,0.555899,0.552459}%
\pgfsetfillcolor{currentfill}%
\pgfsetfillopacity{0.800000}%
\pgfsetlinewidth{0.000000pt}%
\definecolor{currentstroke}{rgb}{0.000000,0.000000,0.000000}%
\pgfsetstrokecolor{currentstroke}%
\pgfsetdash{}{0pt}%
\pgfpathmoveto{\pgfqpoint{4.810452in}{2.223174in}}%
\pgfpathlineto{\pgfqpoint{4.825347in}{2.237909in}}%
\pgfpathlineto{\pgfqpoint{4.840261in}{2.252831in}}%
\pgfpathlineto{\pgfqpoint{4.855197in}{2.267938in}}%
\pgfpathlineto{\pgfqpoint{4.870153in}{2.283233in}}%
\pgfpathlineto{\pgfqpoint{4.878381in}{2.303466in}}%
\pgfpathlineto{\pgfqpoint{4.886605in}{2.323576in}}%
\pgfpathlineto{\pgfqpoint{4.894824in}{2.343557in}}%
\pgfpathlineto{\pgfqpoint{4.903038in}{2.363407in}}%
\pgfpathlineto{\pgfqpoint{4.888065in}{2.347662in}}%
\pgfpathlineto{\pgfqpoint{4.873114in}{2.332106in}}%
\pgfpathlineto{\pgfqpoint{4.858184in}{2.316736in}}%
\pgfpathlineto{\pgfqpoint{4.843274in}{2.301554in}}%
\pgfpathlineto{\pgfqpoint{4.835075in}{2.282140in}}%
\pgfpathlineto{\pgfqpoint{4.826872in}{2.262603in}}%
\pgfpathlineto{\pgfqpoint{4.818664in}{2.242946in}}%
\pgfpathlineto{\pgfqpoint{4.810452in}{2.223174in}}%
\pgfpathclose%
\pgfusepath{fill}%
\end{pgfscope}%
\begin{pgfscope}%
\pgfpathrectangle{\pgfqpoint{1.150000in}{0.150000in}}{\pgfqpoint{5.700000in}{5.700000in}}%
\pgfusepath{clip}%
\pgfsetbuttcap%
\pgfsetroundjoin%
\definecolor{currentfill}{rgb}{0.221989,0.339161,0.548752}%
\pgfsetfillcolor{currentfill}%
\pgfsetfillopacity{0.800000}%
\pgfsetlinewidth{0.000000pt}%
\definecolor{currentstroke}{rgb}{0.000000,0.000000,0.000000}%
\pgfsetstrokecolor{currentstroke}%
\pgfsetdash{}{0pt}%
\pgfpathmoveto{\pgfqpoint{4.461439in}{1.562577in}}%
\pgfpathlineto{\pgfqpoint{4.476090in}{1.571831in}}%
\pgfpathlineto{\pgfqpoint{4.490756in}{1.581262in}}%
\pgfpathlineto{\pgfqpoint{4.505439in}{1.590871in}}%
\pgfpathlineto{\pgfqpoint{4.520139in}{1.600657in}}%
\pgfpathlineto{\pgfqpoint{4.528424in}{1.621013in}}%
\pgfpathlineto{\pgfqpoint{4.536707in}{1.641407in}}%
\pgfpathlineto{\pgfqpoint{4.544988in}{1.661833in}}%
\pgfpathlineto{\pgfqpoint{4.553266in}{1.682283in}}%
\pgfpathlineto{\pgfqpoint{4.538554in}{1.671826in}}%
\pgfpathlineto{\pgfqpoint{4.523859in}{1.661547in}}%
\pgfpathlineto{\pgfqpoint{4.509180in}{1.651447in}}%
\pgfpathlineto{\pgfqpoint{4.494518in}{1.641526in}}%
\pgfpathlineto{\pgfqpoint{4.486252in}{1.621732in}}%
\pgfpathlineto{\pgfqpoint{4.477983in}{1.601972in}}%
\pgfpathlineto{\pgfqpoint{4.469712in}{1.582252in}}%
\pgfpathlineto{\pgfqpoint{4.461439in}{1.562577in}}%
\pgfpathclose%
\pgfusepath{fill}%
\end{pgfscope}%
\begin{pgfscope}%
\pgfpathrectangle{\pgfqpoint{1.150000in}{0.150000in}}{\pgfqpoint{5.700000in}{5.700000in}}%
\pgfusepath{clip}%
\pgfsetbuttcap%
\pgfsetroundjoin%
\definecolor{currentfill}{rgb}{0.128087,0.647749,0.523491}%
\pgfsetfillcolor{currentfill}%
\pgfsetfillopacity{0.800000}%
\pgfsetlinewidth{0.000000pt}%
\definecolor{currentstroke}{rgb}{0.000000,0.000000,0.000000}%
\pgfsetstrokecolor{currentstroke}%
\pgfsetdash{}{0pt}%
\pgfpathmoveto{\pgfqpoint{4.968566in}{2.517004in}}%
\pgfpathlineto{\pgfqpoint{4.983590in}{2.533733in}}%
\pgfpathlineto{\pgfqpoint{4.998636in}{2.550652in}}%
\pgfpathlineto{\pgfqpoint{5.013705in}{2.567761in}}%
\pgfpathlineto{\pgfqpoint{5.028795in}{2.585061in}}%
\pgfpathlineto{\pgfqpoint{5.036976in}{2.603919in}}%
\pgfpathlineto{\pgfqpoint{5.045151in}{2.622601in}}%
\pgfpathlineto{\pgfqpoint{5.053319in}{2.641104in}}%
\pgfpathlineto{\pgfqpoint{5.061481in}{2.659425in}}%
\pgfpathlineto{\pgfqpoint{5.046375in}{2.641778in}}%
\pgfpathlineto{\pgfqpoint{5.031292in}{2.624322in}}%
\pgfpathlineto{\pgfqpoint{5.016232in}{2.607057in}}%
\pgfpathlineto{\pgfqpoint{5.001193in}{2.589983in}}%
\pgfpathlineto{\pgfqpoint{4.993046in}{2.571994in}}%
\pgfpathlineto{\pgfqpoint{4.984892in}{2.553833in}}%
\pgfpathlineto{\pgfqpoint{4.976732in}{2.535502in}}%
\pgfpathlineto{\pgfqpoint{4.968566in}{2.517004in}}%
\pgfpathclose%
\pgfusepath{fill}%
\end{pgfscope}%
\begin{pgfscope}%
\pgfpathrectangle{\pgfqpoint{1.150000in}{0.150000in}}{\pgfqpoint{5.700000in}{5.700000in}}%
\pgfusepath{clip}%
\pgfsetbuttcap%
\pgfsetroundjoin%
\definecolor{currentfill}{rgb}{0.177423,0.437527,0.557565}%
\pgfsetfillcolor{currentfill}%
\pgfsetfillopacity{0.800000}%
\pgfsetlinewidth{0.000000pt}%
\definecolor{currentstroke}{rgb}{0.000000,0.000000,0.000000}%
\pgfsetstrokecolor{currentstroke}%
\pgfsetdash{}{0pt}%
\pgfpathmoveto{\pgfqpoint{4.619400in}{1.846091in}}%
\pgfpathlineto{\pgfqpoint{4.634158in}{1.857976in}}%
\pgfpathlineto{\pgfqpoint{4.648933in}{1.870043in}}%
\pgfpathlineto{\pgfqpoint{4.663727in}{1.882291in}}%
\pgfpathlineto{\pgfqpoint{4.678539in}{1.894720in}}%
\pgfpathlineto{\pgfqpoint{4.686808in}{1.915725in}}%
\pgfpathlineto{\pgfqpoint{4.695075in}{1.936691in}}%
\pgfpathlineto{\pgfqpoint{4.703338in}{1.957611in}}%
\pgfpathlineto{\pgfqpoint{4.711599in}{1.978480in}}%
\pgfpathlineto{\pgfqpoint{4.696771in}{1.965471in}}%
\pgfpathlineto{\pgfqpoint{4.681962in}{1.952645in}}%
\pgfpathlineto{\pgfqpoint{4.667171in}{1.940001in}}%
\pgfpathlineto{\pgfqpoint{4.652400in}{1.927540in}}%
\pgfpathlineto{\pgfqpoint{4.644154in}{1.907236in}}%
\pgfpathlineto{\pgfqpoint{4.635906in}{1.886890in}}%
\pgfpathlineto{\pgfqpoint{4.627655in}{1.866506in}}%
\pgfpathlineto{\pgfqpoint{4.619400in}{1.846091in}}%
\pgfpathclose%
\pgfusepath{fill}%
\end{pgfscope}%
\begin{pgfscope}%
\pgfpathrectangle{\pgfqpoint{1.150000in}{0.150000in}}{\pgfqpoint{5.700000in}{5.700000in}}%
\pgfusepath{clip}%
\pgfsetbuttcap%
\pgfsetroundjoin%
\definecolor{currentfill}{rgb}{0.232815,0.732247,0.459277}%
\pgfsetfillcolor{currentfill}%
\pgfsetfillopacity{0.800000}%
\pgfsetlinewidth{0.000000pt}%
\definecolor{currentstroke}{rgb}{0.000000,0.000000,0.000000}%
\pgfsetstrokecolor{currentstroke}%
\pgfsetdash{}{0pt}%
\pgfpathmoveto{\pgfqpoint{5.126528in}{2.799131in}}%
\pgfpathlineto{\pgfqpoint{5.141683in}{2.817558in}}%
\pgfpathlineto{\pgfqpoint{5.156862in}{2.836179in}}%
\pgfpathlineto{\pgfqpoint{5.172064in}{2.854993in}}%
\pgfpathlineto{\pgfqpoint{5.187291in}{2.874001in}}%
\pgfpathlineto{\pgfqpoint{5.195401in}{2.890823in}}%
\pgfpathlineto{\pgfqpoint{5.203503in}{2.907429in}}%
\pgfpathlineto{\pgfqpoint{5.211597in}{2.923818in}}%
\pgfpathlineto{\pgfqpoint{5.219682in}{2.939988in}}%
\pgfpathlineto{\pgfqpoint{5.204444in}{2.920739in}}%
\pgfpathlineto{\pgfqpoint{5.189229in}{2.901685in}}%
\pgfpathlineto{\pgfqpoint{5.174039in}{2.882825in}}%
\pgfpathlineto{\pgfqpoint{5.158872in}{2.864158in}}%
\pgfpathlineto{\pgfqpoint{5.150798in}{2.848215in}}%
\pgfpathlineto{\pgfqpoint{5.142716in}{2.832061in}}%
\pgfpathlineto{\pgfqpoint{5.134625in}{2.815699in}}%
\pgfpathlineto{\pgfqpoint{5.126528in}{2.799131in}}%
\pgfpathclose%
\pgfusepath{fill}%
\end{pgfscope}%
\begin{pgfscope}%
\pgfpathrectangle{\pgfqpoint{1.150000in}{0.150000in}}{\pgfqpoint{5.700000in}{5.700000in}}%
\pgfusepath{clip}%
\pgfsetbuttcap%
\pgfsetroundjoin%
\definecolor{currentfill}{rgb}{0.262138,0.242286,0.520837}%
\pgfsetfillcolor{currentfill}%
\pgfsetfillopacity{0.800000}%
\pgfsetlinewidth{0.000000pt}%
\definecolor{currentstroke}{rgb}{0.000000,0.000000,0.000000}%
\pgfsetstrokecolor{currentstroke}%
\pgfsetdash{}{0pt}%
\pgfpathmoveto{\pgfqpoint{4.303616in}{1.305891in}}%
\pgfpathlineto{\pgfqpoint{4.318181in}{1.312260in}}%
\pgfpathlineto{\pgfqpoint{4.332761in}{1.318803in}}%
\pgfpathlineto{\pgfqpoint{4.347354in}{1.325520in}}%
\pgfpathlineto{\pgfqpoint{4.361962in}{1.332411in}}%
\pgfpathlineto{\pgfqpoint{4.370265in}{1.351024in}}%
\pgfpathlineto{\pgfqpoint{4.378566in}{1.369764in}}%
\pgfpathlineto{\pgfqpoint{4.386865in}{1.388622in}}%
\pgfpathlineto{\pgfqpoint{4.395161in}{1.407594in}}%
\pgfpathlineto{\pgfqpoint{4.380545in}{1.399943in}}%
\pgfpathlineto{\pgfqpoint{4.365944in}{1.392468in}}%
\pgfpathlineto{\pgfqpoint{4.351358in}{1.385168in}}%
\pgfpathlineto{\pgfqpoint{4.336786in}{1.378042in}}%
\pgfpathlineto{\pgfqpoint{4.328497in}{1.359818in}}%
\pgfpathlineto{\pgfqpoint{4.320206in}{1.341713in}}%
\pgfpathlineto{\pgfqpoint{4.311912in}{1.323735in}}%
\pgfpathlineto{\pgfqpoint{4.303616in}{1.305891in}}%
\pgfpathclose%
\pgfusepath{fill}%
\end{pgfscope}%
\begin{pgfscope}%
\pgfpathrectangle{\pgfqpoint{1.150000in}{0.150000in}}{\pgfqpoint{5.700000in}{5.700000in}}%
\pgfusepath{clip}%
\pgfsetbuttcap%
\pgfsetroundjoin%
\definecolor{currentfill}{rgb}{0.278826,0.175490,0.483397}%
\pgfsetfillcolor{currentfill}%
\pgfsetfillopacity{0.800000}%
\pgfsetlinewidth{0.000000pt}%
\definecolor{currentstroke}{rgb}{0.000000,0.000000,0.000000}%
\pgfsetstrokecolor{currentstroke}%
\pgfsetdash{}{0pt}%
\pgfpathmoveto{\pgfqpoint{4.179027in}{1.151518in}}%
\pgfpathlineto{\pgfqpoint{4.193535in}{1.155595in}}%
\pgfpathlineto{\pgfqpoint{4.208055in}{1.159844in}}%
\pgfpathlineto{\pgfqpoint{4.222587in}{1.164264in}}%
\pgfpathlineto{\pgfqpoint{4.237131in}{1.168857in}}%
\pgfpathlineto{\pgfqpoint{4.245453in}{1.185360in}}%
\pgfpathlineto{\pgfqpoint{4.253772in}{1.202057in}}%
\pgfpathlineto{\pgfqpoint{4.262087in}{1.218941in}}%
\pgfpathlineto{\pgfqpoint{4.270399in}{1.236004in}}%
\pgfpathlineto{\pgfqpoint{4.255851in}{1.230595in}}%
\pgfpathlineto{\pgfqpoint{4.241317in}{1.225358in}}%
\pgfpathlineto{\pgfqpoint{4.226795in}{1.220295in}}%
\pgfpathlineto{\pgfqpoint{4.212287in}{1.215405in}}%
\pgfpathlineto{\pgfqpoint{4.203977in}{1.199146in}}%
\pgfpathlineto{\pgfqpoint{4.195664in}{1.183073in}}%
\pgfpathlineto{\pgfqpoint{4.187348in}{1.167195in}}%
\pgfpathlineto{\pgfqpoint{4.179027in}{1.151518in}}%
\pgfpathclose%
\pgfusepath{fill}%
\end{pgfscope}%
\begin{pgfscope}%
\pgfpathrectangle{\pgfqpoint{1.150000in}{0.150000in}}{\pgfqpoint{5.700000in}{5.700000in}}%
\pgfusepath{clip}%
\pgfsetbuttcap%
\pgfsetroundjoin%
\definecolor{currentfill}{rgb}{0.139147,0.533812,0.555298}%
\pgfsetfillcolor{currentfill}%
\pgfsetfillopacity{0.800000}%
\pgfsetlinewidth{0.000000pt}%
\definecolor{currentstroke}{rgb}{0.000000,0.000000,0.000000}%
\pgfsetstrokecolor{currentstroke}%
\pgfsetdash{}{0pt}%
\pgfpathmoveto{\pgfqpoint{4.777561in}{2.143014in}}%
\pgfpathlineto{\pgfqpoint{4.792440in}{2.157268in}}%
\pgfpathlineto{\pgfqpoint{4.807339in}{2.171708in}}%
\pgfpathlineto{\pgfqpoint{4.822258in}{2.186333in}}%
\pgfpathlineto{\pgfqpoint{4.837197in}{2.201143in}}%
\pgfpathlineto{\pgfqpoint{4.845442in}{2.221831in}}%
\pgfpathlineto{\pgfqpoint{4.853683in}{2.242411in}}%
\pgfpathlineto{\pgfqpoint{4.861920in}{2.262880in}}%
\pgfpathlineto{\pgfqpoint{4.870153in}{2.283233in}}%
\pgfpathlineto{\pgfqpoint{4.855197in}{2.267938in}}%
\pgfpathlineto{\pgfqpoint{4.840261in}{2.252831in}}%
\pgfpathlineto{\pgfqpoint{4.825347in}{2.237909in}}%
\pgfpathlineto{\pgfqpoint{4.810452in}{2.223174in}}%
\pgfpathlineto{\pgfqpoint{4.802235in}{2.203290in}}%
\pgfpathlineto{\pgfqpoint{4.794015in}{2.183300in}}%
\pgfpathlineto{\pgfqpoint{4.785790in}{2.163206in}}%
\pgfpathlineto{\pgfqpoint{4.777561in}{2.143014in}}%
\pgfpathclose%
\pgfusepath{fill}%
\end{pgfscope}%
\begin{pgfscope}%
\pgfpathrectangle{\pgfqpoint{1.150000in}{0.150000in}}{\pgfqpoint{5.700000in}{5.700000in}}%
\pgfusepath{clip}%
\pgfsetbuttcap%
\pgfsetroundjoin%
\definecolor{currentfill}{rgb}{0.233603,0.313828,0.543914}%
\pgfsetfillcolor{currentfill}%
\pgfsetfillopacity{0.800000}%
\pgfsetlinewidth{0.000000pt}%
\definecolor{currentstroke}{rgb}{0.000000,0.000000,0.000000}%
\pgfsetstrokecolor{currentstroke}%
\pgfsetdash{}{0pt}%
\pgfpathmoveto{\pgfqpoint{4.428319in}{1.484461in}}%
\pgfpathlineto{\pgfqpoint{4.442960in}{1.493016in}}%
\pgfpathlineto{\pgfqpoint{4.457615in}{1.501747in}}%
\pgfpathlineto{\pgfqpoint{4.472287in}{1.510655in}}%
\pgfpathlineto{\pgfqpoint{4.486974in}{1.519740in}}%
\pgfpathlineto{\pgfqpoint{4.495268in}{1.539880in}}%
\pgfpathlineto{\pgfqpoint{4.503561in}{1.560084in}}%
\pgfpathlineto{\pgfqpoint{4.511851in}{1.580346in}}%
\pgfpathlineto{\pgfqpoint{4.520139in}{1.600657in}}%
\pgfpathlineto{\pgfqpoint{4.505439in}{1.590871in}}%
\pgfpathlineto{\pgfqpoint{4.490756in}{1.581262in}}%
\pgfpathlineto{\pgfqpoint{4.476090in}{1.571831in}}%
\pgfpathlineto{\pgfqpoint{4.461439in}{1.562577in}}%
\pgfpathlineto{\pgfqpoint{4.453163in}{1.542954in}}%
\pgfpathlineto{\pgfqpoint{4.444884in}{1.523389in}}%
\pgfpathlineto{\pgfqpoint{4.436603in}{1.503889in}}%
\pgfpathlineto{\pgfqpoint{4.428319in}{1.484461in}}%
\pgfpathclose%
\pgfusepath{fill}%
\end{pgfscope}%
\begin{pgfscope}%
\pgfpathrectangle{\pgfqpoint{1.150000in}{0.150000in}}{\pgfqpoint{5.700000in}{5.700000in}}%
\pgfusepath{clip}%
\pgfsetbuttcap%
\pgfsetroundjoin%
\definecolor{currentfill}{rgb}{0.187231,0.414746,0.556547}%
\pgfsetfillcolor{currentfill}%
\pgfsetfillopacity{0.800000}%
\pgfsetlinewidth{0.000000pt}%
\definecolor{currentstroke}{rgb}{0.000000,0.000000,0.000000}%
\pgfsetstrokecolor{currentstroke}%
\pgfsetdash{}{0pt}%
\pgfpathmoveto{\pgfqpoint{4.586355in}{1.764220in}}%
\pgfpathlineto{\pgfqpoint{4.601098in}{1.775497in}}%
\pgfpathlineto{\pgfqpoint{4.615858in}{1.786954in}}%
\pgfpathlineto{\pgfqpoint{4.630637in}{1.798591in}}%
\pgfpathlineto{\pgfqpoint{4.645433in}{1.810408in}}%
\pgfpathlineto{\pgfqpoint{4.653714in}{1.831519in}}%
\pgfpathlineto{\pgfqpoint{4.661991in}{1.852611in}}%
\pgfpathlineto{\pgfqpoint{4.670267in}{1.873680in}}%
\pgfpathlineto{\pgfqpoint{4.678539in}{1.894720in}}%
\pgfpathlineto{\pgfqpoint{4.663727in}{1.882291in}}%
\pgfpathlineto{\pgfqpoint{4.648933in}{1.870043in}}%
\pgfpathlineto{\pgfqpoint{4.634158in}{1.857976in}}%
\pgfpathlineto{\pgfqpoint{4.619400in}{1.846091in}}%
\pgfpathlineto{\pgfqpoint{4.611143in}{1.825649in}}%
\pgfpathlineto{\pgfqpoint{4.602883in}{1.805187in}}%
\pgfpathlineto{\pgfqpoint{4.594620in}{1.784708in}}%
\pgfpathlineto{\pgfqpoint{4.586355in}{1.764220in}}%
\pgfpathclose%
\pgfusepath{fill}%
\end{pgfscope}%
\begin{pgfscope}%
\pgfpathrectangle{\pgfqpoint{1.150000in}{0.150000in}}{\pgfqpoint{5.700000in}{5.700000in}}%
\pgfusepath{clip}%
\pgfsetbuttcap%
\pgfsetroundjoin%
\definecolor{currentfill}{rgb}{0.369214,0.788888,0.382914}%
\pgfsetfillcolor{currentfill}%
\pgfsetfillopacity{0.800000}%
\pgfsetlinewidth{0.000000pt}%
\definecolor{currentstroke}{rgb}{0.000000,0.000000,0.000000}%
\pgfsetstrokecolor{currentstroke}%
\pgfsetdash{}{0pt}%
\pgfpathmoveto{\pgfqpoint{5.251937in}{3.002439in}}%
\pgfpathlineto{\pgfqpoint{5.267211in}{3.022087in}}%
\pgfpathlineto{\pgfqpoint{5.282509in}{3.041930in}}%
\pgfpathlineto{\pgfqpoint{5.297832in}{3.061970in}}%
\pgfpathlineto{\pgfqpoint{5.313180in}{3.082206in}}%
\pgfpathlineto{\pgfqpoint{5.321231in}{3.097435in}}%
\pgfpathlineto{\pgfqpoint{5.329273in}{3.112426in}}%
\pgfpathlineto{\pgfqpoint{5.337306in}{3.127178in}}%
\pgfpathlineto{\pgfqpoint{5.345328in}{3.141690in}}%
\pgfpathlineto{\pgfqpoint{5.329971in}{3.121286in}}%
\pgfpathlineto{\pgfqpoint{5.314638in}{3.101079in}}%
\pgfpathlineto{\pgfqpoint{5.299331in}{3.081069in}}%
\pgfpathlineto{\pgfqpoint{5.284049in}{3.061255in}}%
\pgfpathlineto{\pgfqpoint{5.276035in}{3.046897in}}%
\pgfpathlineto{\pgfqpoint{5.268012in}{3.032307in}}%
\pgfpathlineto{\pgfqpoint{5.259979in}{3.017487in}}%
\pgfpathlineto{\pgfqpoint{5.251937in}{3.002439in}}%
\pgfpathclose%
\pgfusepath{fill}%
\end{pgfscope}%
\begin{pgfscope}%
\pgfpathrectangle{\pgfqpoint{1.150000in}{0.150000in}}{\pgfqpoint{5.700000in}{5.700000in}}%
\pgfusepath{clip}%
\pgfsetbuttcap%
\pgfsetroundjoin%
\definecolor{currentfill}{rgb}{0.121380,0.629492,0.531973}%
\pgfsetfillcolor{currentfill}%
\pgfsetfillopacity{0.800000}%
\pgfsetlinewidth{0.000000pt}%
\definecolor{currentstroke}{rgb}{0.000000,0.000000,0.000000}%
\pgfsetstrokecolor{currentstroke}%
\pgfsetdash{}{0pt}%
\pgfpathmoveto{\pgfqpoint{4.935844in}{2.441408in}}%
\pgfpathlineto{\pgfqpoint{4.950854in}{2.457757in}}%
\pgfpathlineto{\pgfqpoint{4.965885in}{2.474294in}}%
\pgfpathlineto{\pgfqpoint{4.980937in}{2.491020in}}%
\pgfpathlineto{\pgfqpoint{4.996012in}{2.507937in}}%
\pgfpathlineto{\pgfqpoint{5.004217in}{2.527465in}}%
\pgfpathlineto{\pgfqpoint{5.012415in}{2.546831in}}%
\pgfpathlineto{\pgfqpoint{5.020608in}{2.566031in}}%
\pgfpathlineto{\pgfqpoint{5.028795in}{2.585061in}}%
\pgfpathlineto{\pgfqpoint{5.013705in}{2.567761in}}%
\pgfpathlineto{\pgfqpoint{4.998636in}{2.550652in}}%
\pgfpathlineto{\pgfqpoint{4.983590in}{2.533733in}}%
\pgfpathlineto{\pgfqpoint{4.968566in}{2.517004in}}%
\pgfpathlineto{\pgfqpoint{4.960394in}{2.498343in}}%
\pgfpathlineto{\pgfqpoint{4.952216in}{2.479521in}}%
\pgfpathlineto{\pgfqpoint{4.944033in}{2.460542in}}%
\pgfpathlineto{\pgfqpoint{4.935844in}{2.441408in}}%
\pgfpathclose%
\pgfusepath{fill}%
\end{pgfscope}%
\begin{pgfscope}%
\pgfpathrectangle{\pgfqpoint{1.150000in}{0.150000in}}{\pgfqpoint{5.700000in}{5.700000in}}%
\pgfusepath{clip}%
\pgfsetbuttcap%
\pgfsetroundjoin%
\definecolor{currentfill}{rgb}{0.515992,0.831158,0.294279}%
\pgfsetfillcolor{currentfill}%
\pgfsetfillopacity{0.800000}%
\pgfsetlinewidth{0.000000pt}%
\definecolor{currentstroke}{rgb}{0.000000,0.000000,0.000000}%
\pgfsetstrokecolor{currentstroke}%
\pgfsetdash{}{0pt}%
\pgfpathmoveto{\pgfqpoint{5.377318in}{3.197311in}}%
\pgfpathlineto{\pgfqpoint{5.392709in}{3.218041in}}%
\pgfpathlineto{\pgfqpoint{5.408125in}{3.238970in}}%
\pgfpathlineto{\pgfqpoint{5.423567in}{3.260097in}}%
\pgfpathlineto{\pgfqpoint{5.431544in}{3.273472in}}%
\pgfpathlineto{\pgfqpoint{5.439510in}{3.286594in}}%
\pgfpathlineto{\pgfqpoint{5.447465in}{3.299463in}}%
\pgfpathlineto{\pgfqpoint{5.455409in}{3.312079in}}%
\pgfpathlineto{\pgfqpoint{5.439960in}{3.290860in}}%
\pgfpathlineto{\pgfqpoint{5.424538in}{3.269840in}}%
\pgfpathlineto{\pgfqpoint{5.409141in}{3.249017in}}%
\pgfpathlineto{\pgfqpoint{5.401201in}{3.236460in}}%
\pgfpathlineto{\pgfqpoint{5.393251in}{3.223656in}}%
\pgfpathlineto{\pgfqpoint{5.385290in}{3.210606in}}%
\pgfpathlineto{\pgfqpoint{5.377318in}{3.197311in}}%
\pgfpathclose%
\pgfusepath{fill}%
\end{pgfscope}%
\begin{pgfscope}%
\pgfpathrectangle{\pgfqpoint{1.150000in}{0.150000in}}{\pgfqpoint{5.700000in}{5.700000in}}%
\pgfusepath{clip}%
\pgfsetbuttcap%
\pgfsetroundjoin%
\definecolor{currentfill}{rgb}{0.269308,0.218818,0.509577}%
\pgfsetfillcolor{currentfill}%
\pgfsetfillopacity{0.800000}%
\pgfsetlinewidth{0.000000pt}%
\definecolor{currentstroke}{rgb}{0.000000,0.000000,0.000000}%
\pgfsetstrokecolor{currentstroke}%
\pgfsetdash{}{0pt}%
\pgfpathmoveto{\pgfqpoint{4.270399in}{1.236004in}}%
\pgfpathlineto{\pgfqpoint{4.284959in}{1.241586in}}%
\pgfpathlineto{\pgfqpoint{4.299533in}{1.247341in}}%
\pgfpathlineto{\pgfqpoint{4.314121in}{1.253269in}}%
\pgfpathlineto{\pgfqpoint{4.328722in}{1.259370in}}%
\pgfpathlineto{\pgfqpoint{4.337036in}{1.277404in}}%
\pgfpathlineto{\pgfqpoint{4.345347in}{1.295594in}}%
\pgfpathlineto{\pgfqpoint{4.353656in}{1.313932in}}%
\pgfpathlineto{\pgfqpoint{4.361962in}{1.332411in}}%
\pgfpathlineto{\pgfqpoint{4.347354in}{1.325520in}}%
\pgfpathlineto{\pgfqpoint{4.332761in}{1.318803in}}%
\pgfpathlineto{\pgfqpoint{4.318181in}{1.312260in}}%
\pgfpathlineto{\pgfqpoint{4.303616in}{1.305891in}}%
\pgfpathlineto{\pgfqpoint{4.295316in}{1.288189in}}%
\pgfpathlineto{\pgfqpoint{4.287013in}{1.270635in}}%
\pgfpathlineto{\pgfqpoint{4.278707in}{1.253238in}}%
\pgfpathlineto{\pgfqpoint{4.270399in}{1.236004in}}%
\pgfpathclose%
\pgfusepath{fill}%
\end{pgfscope}%
\begin{pgfscope}%
\pgfpathrectangle{\pgfqpoint{1.150000in}{0.150000in}}{\pgfqpoint{5.700000in}{5.700000in}}%
\pgfusepath{clip}%
\pgfsetbuttcap%
\pgfsetroundjoin%
\definecolor{currentfill}{rgb}{0.208030,0.718701,0.472873}%
\pgfsetfillcolor{currentfill}%
\pgfsetfillopacity{0.800000}%
\pgfsetlinewidth{0.000000pt}%
\definecolor{currentstroke}{rgb}{0.000000,0.000000,0.000000}%
\pgfsetstrokecolor{currentstroke}%
\pgfsetdash{}{0pt}%
\pgfpathmoveto{\pgfqpoint{5.094061in}{2.730838in}}%
\pgfpathlineto{\pgfqpoint{5.109204in}{2.748989in}}%
\pgfpathlineto{\pgfqpoint{5.124370in}{2.767333in}}%
\pgfpathlineto{\pgfqpoint{5.139559in}{2.785870in}}%
\pgfpathlineto{\pgfqpoint{5.154772in}{2.804601in}}%
\pgfpathlineto{\pgfqpoint{5.162913in}{2.822263in}}%
\pgfpathlineto{\pgfqpoint{5.171047in}{2.839719in}}%
\pgfpathlineto{\pgfqpoint{5.179173in}{2.856965in}}%
\pgfpathlineto{\pgfqpoint{5.187291in}{2.874001in}}%
\pgfpathlineto{\pgfqpoint{5.172064in}{2.854993in}}%
\pgfpathlineto{\pgfqpoint{5.156862in}{2.836179in}}%
\pgfpathlineto{\pgfqpoint{5.141683in}{2.817558in}}%
\pgfpathlineto{\pgfqpoint{5.126528in}{2.799131in}}%
\pgfpathlineto{\pgfqpoint{5.118422in}{2.782359in}}%
\pgfpathlineto{\pgfqpoint{5.110309in}{2.765384in}}%
\pgfpathlineto{\pgfqpoint{5.102189in}{2.748210in}}%
\pgfpathlineto{\pgfqpoint{5.094061in}{2.730838in}}%
\pgfpathclose%
\pgfusepath{fill}%
\end{pgfscope}%
\begin{pgfscope}%
\pgfpathrectangle{\pgfqpoint{1.150000in}{0.150000in}}{\pgfqpoint{5.700000in}{5.700000in}}%
\pgfusepath{clip}%
\pgfsetbuttcap%
\pgfsetroundjoin%
\definecolor{currentfill}{rgb}{0.147607,0.511733,0.557049}%
\pgfsetfillcolor{currentfill}%
\pgfsetfillopacity{0.800000}%
\pgfsetlinewidth{0.000000pt}%
\definecolor{currentstroke}{rgb}{0.000000,0.000000,0.000000}%
\pgfsetstrokecolor{currentstroke}%
\pgfsetdash{}{0pt}%
\pgfpathmoveto{\pgfqpoint{4.744608in}{2.061351in}}%
\pgfpathlineto{\pgfqpoint{4.759471in}{2.075090in}}%
\pgfpathlineto{\pgfqpoint{4.774353in}{2.089014in}}%
\pgfpathlineto{\pgfqpoint{4.789256in}{2.103123in}}%
\pgfpathlineto{\pgfqpoint{4.804178in}{2.117415in}}%
\pgfpathlineto{\pgfqpoint{4.812438in}{2.138485in}}%
\pgfpathlineto{\pgfqpoint{4.820695in}{2.159466in}}%
\pgfpathlineto{\pgfqpoint{4.828948in}{2.180354in}}%
\pgfpathlineto{\pgfqpoint{4.837197in}{2.201143in}}%
\pgfpathlineto{\pgfqpoint{4.822258in}{2.186333in}}%
\pgfpathlineto{\pgfqpoint{4.807339in}{2.171708in}}%
\pgfpathlineto{\pgfqpoint{4.792440in}{2.157268in}}%
\pgfpathlineto{\pgfqpoint{4.777561in}{2.143014in}}%
\pgfpathlineto{\pgfqpoint{4.769328in}{2.122728in}}%
\pgfpathlineto{\pgfqpoint{4.761092in}{2.102352in}}%
\pgfpathlineto{\pgfqpoint{4.752852in}{2.081892in}}%
\pgfpathlineto{\pgfqpoint{4.744608in}{2.061351in}}%
\pgfpathclose%
\pgfusepath{fill}%
\end{pgfscope}%
\begin{pgfscope}%
\pgfpathrectangle{\pgfqpoint{1.150000in}{0.150000in}}{\pgfqpoint{5.700000in}{5.700000in}}%
\pgfusepath{clip}%
\pgfsetbuttcap%
\pgfsetroundjoin%
\definecolor{currentfill}{rgb}{0.244972,0.287675,0.537260}%
\pgfsetfillcolor{currentfill}%
\pgfsetfillopacity{0.800000}%
\pgfsetlinewidth{0.000000pt}%
\definecolor{currentstroke}{rgb}{0.000000,0.000000,0.000000}%
\pgfsetstrokecolor{currentstroke}%
\pgfsetdash{}{0pt}%
\pgfpathmoveto{\pgfqpoint{4.395161in}{1.407594in}}%
\pgfpathlineto{\pgfqpoint{4.409791in}{1.415419in}}%
\pgfpathlineto{\pgfqpoint{4.424437in}{1.423420in}}%
\pgfpathlineto{\pgfqpoint{4.439097in}{1.431596in}}%
\pgfpathlineto{\pgfqpoint{4.453773in}{1.439947in}}%
\pgfpathlineto{\pgfqpoint{4.462077in}{1.459766in}}%
\pgfpathlineto{\pgfqpoint{4.470378in}{1.479676in}}%
\pgfpathlineto{\pgfqpoint{4.478677in}{1.499669in}}%
\pgfpathlineto{\pgfqpoint{4.486974in}{1.519740in}}%
\pgfpathlineto{\pgfqpoint{4.472287in}{1.510655in}}%
\pgfpathlineto{\pgfqpoint{4.457615in}{1.501747in}}%
\pgfpathlineto{\pgfqpoint{4.442960in}{1.493016in}}%
\pgfpathlineto{\pgfqpoint{4.428319in}{1.484461in}}%
\pgfpathlineto{\pgfqpoint{4.420033in}{1.465110in}}%
\pgfpathlineto{\pgfqpoint{4.411745in}{1.445844in}}%
\pgfpathlineto{\pgfqpoint{4.403454in}{1.426670in}}%
\pgfpathlineto{\pgfqpoint{4.395161in}{1.407594in}}%
\pgfpathclose%
\pgfusepath{fill}%
\end{pgfscope}%
\begin{pgfscope}%
\pgfpathrectangle{\pgfqpoint{1.150000in}{0.150000in}}{\pgfqpoint{5.700000in}{5.700000in}}%
\pgfusepath{clip}%
\pgfsetbuttcap%
\pgfsetroundjoin%
\definecolor{currentfill}{rgb}{0.197636,0.391528,0.554969}%
\pgfsetfillcolor{currentfill}%
\pgfsetfillopacity{0.800000}%
\pgfsetlinewidth{0.000000pt}%
\definecolor{currentstroke}{rgb}{0.000000,0.000000,0.000000}%
\pgfsetstrokecolor{currentstroke}%
\pgfsetdash{}{0pt}%
\pgfpathmoveto{\pgfqpoint{4.553266in}{1.682283in}}%
\pgfpathlineto{\pgfqpoint{4.567996in}{1.692919in}}%
\pgfpathlineto{\pgfqpoint{4.582742in}{1.703735in}}%
\pgfpathlineto{\pgfqpoint{4.597506in}{1.714729in}}%
\pgfpathlineto{\pgfqpoint{4.612287in}{1.725903in}}%
\pgfpathlineto{\pgfqpoint{4.620577in}{1.747027in}}%
\pgfpathlineto{\pgfqpoint{4.628865in}{1.768156in}}%
\pgfpathlineto{\pgfqpoint{4.637150in}{1.789285in}}%
\pgfpathlineto{\pgfqpoint{4.645433in}{1.810408in}}%
\pgfpathlineto{\pgfqpoint{4.630637in}{1.798591in}}%
\pgfpathlineto{\pgfqpoint{4.615858in}{1.786954in}}%
\pgfpathlineto{\pgfqpoint{4.601098in}{1.775497in}}%
\pgfpathlineto{\pgfqpoint{4.586355in}{1.764220in}}%
\pgfpathlineto{\pgfqpoint{4.578086in}{1.743728in}}%
\pgfpathlineto{\pgfqpoint{4.569816in}{1.723237in}}%
\pgfpathlineto{\pgfqpoint{4.561542in}{1.702753in}}%
\pgfpathlineto{\pgfqpoint{4.553266in}{1.682283in}}%
\pgfpathclose%
\pgfusepath{fill}%
\end{pgfscope}%
\begin{pgfscope}%
\pgfpathrectangle{\pgfqpoint{1.150000in}{0.150000in}}{\pgfqpoint{5.700000in}{5.700000in}}%
\pgfusepath{clip}%
\pgfsetbuttcap%
\pgfsetroundjoin%
\definecolor{currentfill}{rgb}{0.119512,0.607464,0.540218}%
\pgfsetfillcolor{currentfill}%
\pgfsetfillopacity{0.800000}%
\pgfsetlinewidth{0.000000pt}%
\definecolor{currentstroke}{rgb}{0.000000,0.000000,0.000000}%
\pgfsetstrokecolor{currentstroke}%
\pgfsetdash{}{0pt}%
\pgfpathmoveto{\pgfqpoint{4.903038in}{2.363407in}}%
\pgfpathlineto{\pgfqpoint{4.918031in}{2.379339in}}%
\pgfpathlineto{\pgfqpoint{4.933046in}{2.395459in}}%
\pgfpathlineto{\pgfqpoint{4.948082in}{2.411768in}}%
\pgfpathlineto{\pgfqpoint{4.963140in}{2.428266in}}%
\pgfpathlineto{\pgfqpoint{4.971366in}{2.448410in}}%
\pgfpathlineto{\pgfqpoint{4.979587in}{2.468405in}}%
\pgfpathlineto{\pgfqpoint{4.987802in}{2.488249in}}%
\pgfpathlineto{\pgfqpoint{4.996012in}{2.507937in}}%
\pgfpathlineto{\pgfqpoint{4.980937in}{2.491020in}}%
\pgfpathlineto{\pgfqpoint{4.965885in}{2.474294in}}%
\pgfpathlineto{\pgfqpoint{4.950854in}{2.457757in}}%
\pgfpathlineto{\pgfqpoint{4.935844in}{2.441408in}}%
\pgfpathlineto{\pgfqpoint{4.927650in}{2.422125in}}%
\pgfpathlineto{\pgfqpoint{4.919451in}{2.402694in}}%
\pgfpathlineto{\pgfqpoint{4.911247in}{2.383120in}}%
\pgfpathlineto{\pgfqpoint{4.903038in}{2.363407in}}%
\pgfpathclose%
\pgfusepath{fill}%
\end{pgfscope}%
\begin{pgfscope}%
\pgfpathrectangle{\pgfqpoint{1.150000in}{0.150000in}}{\pgfqpoint{5.700000in}{5.700000in}}%
\pgfusepath{clip}%
\pgfsetbuttcap%
\pgfsetroundjoin%
\definecolor{currentfill}{rgb}{0.156270,0.489624,0.557936}%
\pgfsetfillcolor{currentfill}%
\pgfsetfillopacity{0.800000}%
\pgfsetlinewidth{0.000000pt}%
\definecolor{currentstroke}{rgb}{0.000000,0.000000,0.000000}%
\pgfsetstrokecolor{currentstroke}%
\pgfsetdash{}{0pt}%
\pgfpathmoveto{\pgfqpoint{4.711599in}{1.978480in}}%
\pgfpathlineto{\pgfqpoint{4.726445in}{1.991672in}}%
\pgfpathlineto{\pgfqpoint{4.741311in}{2.005047in}}%
\pgfpathlineto{\pgfqpoint{4.756197in}{2.018605in}}%
\pgfpathlineto{\pgfqpoint{4.771101in}{2.032347in}}%
\pgfpathlineto{\pgfqpoint{4.779375in}{2.053723in}}%
\pgfpathlineto{\pgfqpoint{4.787646in}{2.075029in}}%
\pgfpathlineto{\pgfqpoint{4.795914in}{2.096262in}}%
\pgfpathlineto{\pgfqpoint{4.804178in}{2.117415in}}%
\pgfpathlineto{\pgfqpoint{4.789256in}{2.103123in}}%
\pgfpathlineto{\pgfqpoint{4.774353in}{2.089014in}}%
\pgfpathlineto{\pgfqpoint{4.759471in}{2.075090in}}%
\pgfpathlineto{\pgfqpoint{4.744608in}{2.061351in}}%
\pgfpathlineto{\pgfqpoint{4.736361in}{2.040734in}}%
\pgfpathlineto{\pgfqpoint{4.728110in}{2.020047in}}%
\pgfpathlineto{\pgfqpoint{4.719856in}{1.999294in}}%
\pgfpathlineto{\pgfqpoint{4.711599in}{1.978480in}}%
\pgfpathclose%
\pgfusepath{fill}%
\end{pgfscope}%
\begin{pgfscope}%
\pgfpathrectangle{\pgfqpoint{1.150000in}{0.150000in}}{\pgfqpoint{5.700000in}{5.700000in}}%
\pgfusepath{clip}%
\pgfsetbuttcap%
\pgfsetroundjoin%
\definecolor{currentfill}{rgb}{0.344074,0.780029,0.397381}%
\pgfsetfillcolor{currentfill}%
\pgfsetfillopacity{0.800000}%
\pgfsetlinewidth{0.000000pt}%
\definecolor{currentstroke}{rgb}{0.000000,0.000000,0.000000}%
\pgfsetstrokecolor{currentstroke}%
\pgfsetdash{}{0pt}%
\pgfpathmoveto{\pgfqpoint{5.219682in}{2.939988in}}%
\pgfpathlineto{\pgfqpoint{5.234945in}{2.959432in}}%
\pgfpathlineto{\pgfqpoint{5.250232in}{2.979071in}}%
\pgfpathlineto{\pgfqpoint{5.265544in}{2.998905in}}%
\pgfpathlineto{\pgfqpoint{5.280880in}{3.018936in}}%
\pgfpathlineto{\pgfqpoint{5.288969in}{3.035103in}}%
\pgfpathlineto{\pgfqpoint{5.297048in}{3.051039in}}%
\pgfpathlineto{\pgfqpoint{5.305119in}{3.066740in}}%
\pgfpathlineto{\pgfqpoint{5.313180in}{3.082206in}}%
\pgfpathlineto{\pgfqpoint{5.297832in}{3.061970in}}%
\pgfpathlineto{\pgfqpoint{5.282509in}{3.041930in}}%
\pgfpathlineto{\pgfqpoint{5.267211in}{3.022087in}}%
\pgfpathlineto{\pgfqpoint{5.251937in}{3.002439in}}%
\pgfpathlineto{\pgfqpoint{5.243887in}{2.987164in}}%
\pgfpathlineto{\pgfqpoint{5.235827in}{2.971662in}}%
\pgfpathlineto{\pgfqpoint{5.227759in}{2.955937in}}%
\pgfpathlineto{\pgfqpoint{5.219682in}{2.939988in}}%
\pgfpathclose%
\pgfusepath{fill}%
\end{pgfscope}%
\begin{pgfscope}%
\pgfpathrectangle{\pgfqpoint{1.150000in}{0.150000in}}{\pgfqpoint{5.700000in}{5.700000in}}%
\pgfusepath{clip}%
\pgfsetbuttcap%
\pgfsetroundjoin%
\definecolor{currentfill}{rgb}{0.180653,0.701402,0.488189}%
\pgfsetfillcolor{currentfill}%
\pgfsetfillopacity{0.800000}%
\pgfsetlinewidth{0.000000pt}%
\definecolor{currentstroke}{rgb}{0.000000,0.000000,0.000000}%
\pgfsetstrokecolor{currentstroke}%
\pgfsetdash{}{0pt}%
\pgfpathmoveto{\pgfqpoint{5.061481in}{2.659425in}}%
\pgfpathlineto{\pgfqpoint{5.076610in}{2.677264in}}%
\pgfpathlineto{\pgfqpoint{5.091761in}{2.695295in}}%
\pgfpathlineto{\pgfqpoint{5.106936in}{2.713518in}}%
\pgfpathlineto{\pgfqpoint{5.122134in}{2.731934in}}%
\pgfpathlineto{\pgfqpoint{5.130304in}{2.750398in}}%
\pgfpathlineto{\pgfqpoint{5.138467in}{2.768666in}}%
\pgfpathlineto{\pgfqpoint{5.146623in}{2.786734in}}%
\pgfpathlineto{\pgfqpoint{5.154772in}{2.804601in}}%
\pgfpathlineto{\pgfqpoint{5.139559in}{2.785870in}}%
\pgfpathlineto{\pgfqpoint{5.124370in}{2.767333in}}%
\pgfpathlineto{\pgfqpoint{5.109204in}{2.748989in}}%
\pgfpathlineto{\pgfqpoint{5.094061in}{2.730838in}}%
\pgfpathlineto{\pgfqpoint{5.085927in}{2.713271in}}%
\pgfpathlineto{\pgfqpoint{5.077785in}{2.695512in}}%
\pgfpathlineto{\pgfqpoint{5.069636in}{2.677562in}}%
\pgfpathlineto{\pgfqpoint{5.061481in}{2.659425in}}%
\pgfpathclose%
\pgfusepath{fill}%
\end{pgfscope}%
\begin{pgfscope}%
\pgfpathrectangle{\pgfqpoint{1.150000in}{0.150000in}}{\pgfqpoint{5.700000in}{5.700000in}}%
\pgfusepath{clip}%
\pgfsetbuttcap%
\pgfsetroundjoin%
\definecolor{currentfill}{rgb}{0.275191,0.194905,0.496005}%
\pgfsetfillcolor{currentfill}%
\pgfsetfillopacity{0.800000}%
\pgfsetlinewidth{0.000000pt}%
\definecolor{currentstroke}{rgb}{0.000000,0.000000,0.000000}%
\pgfsetstrokecolor{currentstroke}%
\pgfsetdash{}{0pt}%
\pgfpathmoveto{\pgfqpoint{4.237131in}{1.168857in}}%
\pgfpathlineto{\pgfqpoint{4.251689in}{1.173622in}}%
\pgfpathlineto{\pgfqpoint{4.266259in}{1.178559in}}%
\pgfpathlineto{\pgfqpoint{4.280842in}{1.183668in}}%
\pgfpathlineto{\pgfqpoint{4.295438in}{1.188949in}}%
\pgfpathlineto{\pgfqpoint{4.303763in}{1.206281in}}%
\pgfpathlineto{\pgfqpoint{4.312085in}{1.223801in}}%
\pgfpathlineto{\pgfqpoint{4.320405in}{1.241500in}}%
\pgfpathlineto{\pgfqpoint{4.328722in}{1.259370in}}%
\pgfpathlineto{\pgfqpoint{4.314121in}{1.253269in}}%
\pgfpathlineto{\pgfqpoint{4.299533in}{1.247341in}}%
\pgfpathlineto{\pgfqpoint{4.284959in}{1.241586in}}%
\pgfpathlineto{\pgfqpoint{4.270399in}{1.236004in}}%
\pgfpathlineto{\pgfqpoint{4.262087in}{1.218941in}}%
\pgfpathlineto{\pgfqpoint{4.253772in}{1.202057in}}%
\pgfpathlineto{\pgfqpoint{4.245453in}{1.185360in}}%
\pgfpathlineto{\pgfqpoint{4.237131in}{1.168857in}}%
\pgfpathclose%
\pgfusepath{fill}%
\end{pgfscope}%
\begin{pgfscope}%
\pgfpathrectangle{\pgfqpoint{1.150000in}{0.150000in}}{\pgfqpoint{5.700000in}{5.700000in}}%
\pgfusepath{clip}%
\pgfsetbuttcap%
\pgfsetroundjoin%
\definecolor{currentfill}{rgb}{0.208623,0.367752,0.552675}%
\pgfsetfillcolor{currentfill}%
\pgfsetfillopacity{0.800000}%
\pgfsetlinewidth{0.000000pt}%
\definecolor{currentstroke}{rgb}{0.000000,0.000000,0.000000}%
\pgfsetstrokecolor{currentstroke}%
\pgfsetdash{}{0pt}%
\pgfpathmoveto{\pgfqpoint{4.520139in}{1.600657in}}%
\pgfpathlineto{\pgfqpoint{4.534855in}{1.610622in}}%
\pgfpathlineto{\pgfqpoint{4.549587in}{1.620764in}}%
\pgfpathlineto{\pgfqpoint{4.564336in}{1.631084in}}%
\pgfpathlineto{\pgfqpoint{4.579103in}{1.641582in}}%
\pgfpathlineto{\pgfqpoint{4.587402in}{1.662624in}}%
\pgfpathlineto{\pgfqpoint{4.595699in}{1.683695in}}%
\pgfpathlineto{\pgfqpoint{4.603994in}{1.704790in}}%
\pgfpathlineto{\pgfqpoint{4.612287in}{1.725903in}}%
\pgfpathlineto{\pgfqpoint{4.597506in}{1.714729in}}%
\pgfpathlineto{\pgfqpoint{4.582742in}{1.703735in}}%
\pgfpathlineto{\pgfqpoint{4.567996in}{1.692919in}}%
\pgfpathlineto{\pgfqpoint{4.553266in}{1.682283in}}%
\pgfpathlineto{\pgfqpoint{4.544988in}{1.661833in}}%
\pgfpathlineto{\pgfqpoint{4.536707in}{1.641407in}}%
\pgfpathlineto{\pgfqpoint{4.528424in}{1.621013in}}%
\pgfpathlineto{\pgfqpoint{4.520139in}{1.600657in}}%
\pgfpathclose%
\pgfusepath{fill}%
\end{pgfscope}%
\begin{pgfscope}%
\pgfpathrectangle{\pgfqpoint{1.150000in}{0.150000in}}{\pgfqpoint{5.700000in}{5.700000in}}%
\pgfusepath{clip}%
\pgfsetbuttcap%
\pgfsetroundjoin%
\definecolor{currentfill}{rgb}{0.253935,0.265254,0.529983}%
\pgfsetfillcolor{currentfill}%
\pgfsetfillopacity{0.800000}%
\pgfsetlinewidth{0.000000pt}%
\definecolor{currentstroke}{rgb}{0.000000,0.000000,0.000000}%
\pgfsetstrokecolor{currentstroke}%
\pgfsetdash{}{0pt}%
\pgfpathmoveto{\pgfqpoint{4.361962in}{1.332411in}}%
\pgfpathlineto{\pgfqpoint{4.376584in}{1.339476in}}%
\pgfpathlineto{\pgfqpoint{4.391220in}{1.346715in}}%
\pgfpathlineto{\pgfqpoint{4.405871in}{1.354129in}}%
\pgfpathlineto{\pgfqpoint{4.420537in}{1.361717in}}%
\pgfpathlineto{\pgfqpoint{4.428850in}{1.381103in}}%
\pgfpathlineto{\pgfqpoint{4.437160in}{1.400609in}}%
\pgfpathlineto{\pgfqpoint{4.445468in}{1.420225in}}%
\pgfpathlineto{\pgfqpoint{4.453773in}{1.439947in}}%
\pgfpathlineto{\pgfqpoint{4.439097in}{1.431596in}}%
\pgfpathlineto{\pgfqpoint{4.424437in}{1.423420in}}%
\pgfpathlineto{\pgfqpoint{4.409791in}{1.415419in}}%
\pgfpathlineto{\pgfqpoint{4.395161in}{1.407594in}}%
\pgfpathlineto{\pgfqpoint{4.386865in}{1.388622in}}%
\pgfpathlineto{\pgfqpoint{4.378566in}{1.369764in}}%
\pgfpathlineto{\pgfqpoint{4.370265in}{1.351024in}}%
\pgfpathlineto{\pgfqpoint{4.361962in}{1.332411in}}%
\pgfpathclose%
\pgfusepath{fill}%
\end{pgfscope}%
\begin{pgfscope}%
\pgfpathrectangle{\pgfqpoint{1.150000in}{0.150000in}}{\pgfqpoint{5.700000in}{5.700000in}}%
\pgfusepath{clip}%
\pgfsetbuttcap%
\pgfsetroundjoin%
\definecolor{currentfill}{rgb}{0.487026,0.823929,0.312321}%
\pgfsetfillcolor{currentfill}%
\pgfsetfillopacity{0.800000}%
\pgfsetlinewidth{0.000000pt}%
\definecolor{currentstroke}{rgb}{0.000000,0.000000,0.000000}%
\pgfsetstrokecolor{currentstroke}%
\pgfsetdash{}{0pt}%
\pgfpathmoveto{\pgfqpoint{5.345328in}{3.141690in}}%
\pgfpathlineto{\pgfqpoint{5.360711in}{3.162290in}}%
\pgfpathlineto{\pgfqpoint{5.376120in}{3.183089in}}%
\pgfpathlineto{\pgfqpoint{5.391554in}{3.204086in}}%
\pgfpathlineto{\pgfqpoint{5.399573in}{3.218464in}}%
\pgfpathlineto{\pgfqpoint{5.407582in}{3.232592in}}%
\pgfpathlineto{\pgfqpoint{5.415580in}{3.246470in}}%
\pgfpathlineto{\pgfqpoint{5.423567in}{3.260097in}}%
\pgfpathlineto{\pgfqpoint{5.408125in}{3.238970in}}%
\pgfpathlineto{\pgfqpoint{5.392709in}{3.218041in}}%
\pgfpathlineto{\pgfqpoint{5.377318in}{3.197311in}}%
\pgfpathlineto{\pgfqpoint{5.369336in}{3.183771in}}%
\pgfpathlineto{\pgfqpoint{5.361343in}{3.169987in}}%
\pgfpathlineto{\pgfqpoint{5.353341in}{3.155959in}}%
\pgfpathlineto{\pgfqpoint{5.345328in}{3.141690in}}%
\pgfpathclose%
\pgfusepath{fill}%
\end{pgfscope}%
\begin{pgfscope}%
\pgfpathrectangle{\pgfqpoint{1.150000in}{0.150000in}}{\pgfqpoint{5.700000in}{5.700000in}}%
\pgfusepath{clip}%
\pgfsetbuttcap%
\pgfsetroundjoin%
\definecolor{currentfill}{rgb}{0.121831,0.589055,0.545623}%
\pgfsetfillcolor{currentfill}%
\pgfsetfillopacity{0.800000}%
\pgfsetlinewidth{0.000000pt}%
\definecolor{currentstroke}{rgb}{0.000000,0.000000,0.000000}%
\pgfsetstrokecolor{currentstroke}%
\pgfsetdash{}{0pt}%
\pgfpathmoveto{\pgfqpoint{4.870153in}{2.283233in}}%
\pgfpathlineto{\pgfqpoint{4.885130in}{2.298714in}}%
\pgfpathlineto{\pgfqpoint{4.900127in}{2.314383in}}%
\pgfpathlineto{\pgfqpoint{4.915146in}{2.330240in}}%
\pgfpathlineto{\pgfqpoint{4.930187in}{2.346284in}}%
\pgfpathlineto{\pgfqpoint{4.938432in}{2.366982in}}%
\pgfpathlineto{\pgfqpoint{4.946673in}{2.387548in}}%
\pgfpathlineto{\pgfqpoint{4.954909in}{2.407977in}}%
\pgfpathlineto{\pgfqpoint{4.963140in}{2.428266in}}%
\pgfpathlineto{\pgfqpoint{4.948082in}{2.411768in}}%
\pgfpathlineto{\pgfqpoint{4.933046in}{2.395459in}}%
\pgfpathlineto{\pgfqpoint{4.918031in}{2.379339in}}%
\pgfpathlineto{\pgfqpoint{4.903038in}{2.363407in}}%
\pgfpathlineto{\pgfqpoint{4.894824in}{2.343557in}}%
\pgfpathlineto{\pgfqpoint{4.886605in}{2.323576in}}%
\pgfpathlineto{\pgfqpoint{4.878381in}{2.303466in}}%
\pgfpathlineto{\pgfqpoint{4.870153in}{2.283233in}}%
\pgfpathclose%
\pgfusepath{fill}%
\end{pgfscope}%
\begin{pgfscope}%
\pgfpathrectangle{\pgfqpoint{1.150000in}{0.150000in}}{\pgfqpoint{5.700000in}{5.700000in}}%
\pgfusepath{clip}%
\pgfsetbuttcap%
\pgfsetroundjoin%
\definecolor{currentfill}{rgb}{0.165117,0.467423,0.558141}%
\pgfsetfillcolor{currentfill}%
\pgfsetfillopacity{0.800000}%
\pgfsetlinewidth{0.000000pt}%
\definecolor{currentstroke}{rgb}{0.000000,0.000000,0.000000}%
\pgfsetstrokecolor{currentstroke}%
\pgfsetdash{}{0pt}%
\pgfpathmoveto{\pgfqpoint{4.678539in}{1.894720in}}%
\pgfpathlineto{\pgfqpoint{4.693369in}{1.907331in}}%
\pgfpathlineto{\pgfqpoint{4.708219in}{1.920125in}}%
\pgfpathlineto{\pgfqpoint{4.723087in}{1.933100in}}%
\pgfpathlineto{\pgfqpoint{4.737974in}{1.946258in}}%
\pgfpathlineto{\pgfqpoint{4.746260in}{1.967858in}}%
\pgfpathlineto{\pgfqpoint{4.754544in}{1.989410in}}%
\pgfpathlineto{\pgfqpoint{4.762824in}{2.010908in}}%
\pgfpathlineto{\pgfqpoint{4.771101in}{2.032347in}}%
\pgfpathlineto{\pgfqpoint{4.756197in}{2.018605in}}%
\pgfpathlineto{\pgfqpoint{4.741311in}{2.005047in}}%
\pgfpathlineto{\pgfqpoint{4.726445in}{1.991672in}}%
\pgfpathlineto{\pgfqpoint{4.711599in}{1.978480in}}%
\pgfpathlineto{\pgfqpoint{4.703338in}{1.957611in}}%
\pgfpathlineto{\pgfqpoint{4.695075in}{1.936691in}}%
\pgfpathlineto{\pgfqpoint{4.686808in}{1.915725in}}%
\pgfpathlineto{\pgfqpoint{4.678539in}{1.894720in}}%
\pgfpathclose%
\pgfusepath{fill}%
\end{pgfscope}%
\begin{pgfscope}%
\pgfpathrectangle{\pgfqpoint{1.150000in}{0.150000in}}{\pgfqpoint{5.700000in}{5.700000in}}%
\pgfusepath{clip}%
\pgfsetbuttcap%
\pgfsetroundjoin%
\definecolor{currentfill}{rgb}{0.220057,0.343307,0.549413}%
\pgfsetfillcolor{currentfill}%
\pgfsetfillopacity{0.800000}%
\pgfsetlinewidth{0.000000pt}%
\definecolor{currentstroke}{rgb}{0.000000,0.000000,0.000000}%
\pgfsetstrokecolor{currentstroke}%
\pgfsetdash{}{0pt}%
\pgfpathmoveto{\pgfqpoint{4.486974in}{1.519740in}}%
\pgfpathlineto{\pgfqpoint{4.501677in}{1.529001in}}%
\pgfpathlineto{\pgfqpoint{4.516397in}{1.538438in}}%
\pgfpathlineto{\pgfqpoint{4.531132in}{1.548053in}}%
\pgfpathlineto{\pgfqpoint{4.545885in}{1.557845in}}%
\pgfpathlineto{\pgfqpoint{4.554192in}{1.578702in}}%
\pgfpathlineto{\pgfqpoint{4.562498in}{1.599615in}}%
\pgfpathlineto{\pgfqpoint{4.570801in}{1.620577in}}%
\pgfpathlineto{\pgfqpoint{4.579103in}{1.641582in}}%
\pgfpathlineto{\pgfqpoint{4.564336in}{1.631084in}}%
\pgfpathlineto{\pgfqpoint{4.549587in}{1.620764in}}%
\pgfpathlineto{\pgfqpoint{4.534855in}{1.610622in}}%
\pgfpathlineto{\pgfqpoint{4.520139in}{1.600657in}}%
\pgfpathlineto{\pgfqpoint{4.511851in}{1.580346in}}%
\pgfpathlineto{\pgfqpoint{4.503561in}{1.560084in}}%
\pgfpathlineto{\pgfqpoint{4.495268in}{1.539880in}}%
\pgfpathlineto{\pgfqpoint{4.486974in}{1.519740in}}%
\pgfpathclose%
\pgfusepath{fill}%
\end{pgfscope}%
\begin{pgfscope}%
\pgfpathrectangle{\pgfqpoint{1.150000in}{0.150000in}}{\pgfqpoint{5.700000in}{5.700000in}}%
\pgfusepath{clip}%
\pgfsetbuttcap%
\pgfsetroundjoin%
\definecolor{currentfill}{rgb}{0.157851,0.683765,0.501686}%
\pgfsetfillcolor{currentfill}%
\pgfsetfillopacity{0.800000}%
\pgfsetlinewidth{0.000000pt}%
\definecolor{currentstroke}{rgb}{0.000000,0.000000,0.000000}%
\pgfsetstrokecolor{currentstroke}%
\pgfsetdash{}{0pt}%
\pgfpathmoveto{\pgfqpoint{5.028795in}{2.585061in}}%
\pgfpathlineto{\pgfqpoint{5.043909in}{2.602552in}}%
\pgfpathlineto{\pgfqpoint{5.059045in}{2.620233in}}%
\pgfpathlineto{\pgfqpoint{5.074204in}{2.638107in}}%
\pgfpathlineto{\pgfqpoint{5.089386in}{2.656172in}}%
\pgfpathlineto{\pgfqpoint{5.097583in}{2.675393in}}%
\pgfpathlineto{\pgfqpoint{5.105773in}{2.694429in}}%
\pgfpathlineto{\pgfqpoint{5.113957in}{2.713277in}}%
\pgfpathlineto{\pgfqpoint{5.122134in}{2.731934in}}%
\pgfpathlineto{\pgfqpoint{5.106936in}{2.713518in}}%
\pgfpathlineto{\pgfqpoint{5.091761in}{2.695295in}}%
\pgfpathlineto{\pgfqpoint{5.076610in}{2.677264in}}%
\pgfpathlineto{\pgfqpoint{5.061481in}{2.659425in}}%
\pgfpathlineto{\pgfqpoint{5.053319in}{2.641104in}}%
\pgfpathlineto{\pgfqpoint{5.045151in}{2.622601in}}%
\pgfpathlineto{\pgfqpoint{5.036976in}{2.603919in}}%
\pgfpathlineto{\pgfqpoint{5.028795in}{2.585061in}}%
\pgfpathclose%
\pgfusepath{fill}%
\end{pgfscope}%
\begin{pgfscope}%
\pgfpathrectangle{\pgfqpoint{1.150000in}{0.150000in}}{\pgfqpoint{5.700000in}{5.700000in}}%
\pgfusepath{clip}%
\pgfsetbuttcap%
\pgfsetroundjoin%
\definecolor{currentfill}{rgb}{0.311925,0.767822,0.415586}%
\pgfsetfillcolor{currentfill}%
\pgfsetfillopacity{0.800000}%
\pgfsetlinewidth{0.000000pt}%
\definecolor{currentstroke}{rgb}{0.000000,0.000000,0.000000}%
\pgfsetstrokecolor{currentstroke}%
\pgfsetdash{}{0pt}%
\pgfpathmoveto{\pgfqpoint{5.187291in}{2.874001in}}%
\pgfpathlineto{\pgfqpoint{5.202541in}{2.893203in}}%
\pgfpathlineto{\pgfqpoint{5.217816in}{2.912600in}}%
\pgfpathlineto{\pgfqpoint{5.233115in}{2.932193in}}%
\pgfpathlineto{\pgfqpoint{5.248439in}{2.951981in}}%
\pgfpathlineto{\pgfqpoint{5.256562in}{2.969059in}}%
\pgfpathlineto{\pgfqpoint{5.264677in}{2.985912in}}%
\pgfpathlineto{\pgfqpoint{5.272783in}{3.002538in}}%
\pgfpathlineto{\pgfqpoint{5.280880in}{3.018936in}}%
\pgfpathlineto{\pgfqpoint{5.265544in}{2.998905in}}%
\pgfpathlineto{\pgfqpoint{5.250232in}{2.979071in}}%
\pgfpathlineto{\pgfqpoint{5.234945in}{2.959432in}}%
\pgfpathlineto{\pgfqpoint{5.219682in}{2.939988in}}%
\pgfpathlineto{\pgfqpoint{5.211597in}{2.923818in}}%
\pgfpathlineto{\pgfqpoint{5.203503in}{2.907429in}}%
\pgfpathlineto{\pgfqpoint{5.195401in}{2.890823in}}%
\pgfpathlineto{\pgfqpoint{5.187291in}{2.874001in}}%
\pgfpathclose%
\pgfusepath{fill}%
\end{pgfscope}%
\begin{pgfscope}%
\pgfpathrectangle{\pgfqpoint{1.150000in}{0.150000in}}{\pgfqpoint{5.700000in}{5.700000in}}%
\pgfusepath{clip}%
\pgfsetbuttcap%
\pgfsetroundjoin%
\definecolor{currentfill}{rgb}{0.262138,0.242286,0.520837}%
\pgfsetfillcolor{currentfill}%
\pgfsetfillopacity{0.800000}%
\pgfsetlinewidth{0.000000pt}%
\definecolor{currentstroke}{rgb}{0.000000,0.000000,0.000000}%
\pgfsetstrokecolor{currentstroke}%
\pgfsetdash{}{0pt}%
\pgfpathmoveto{\pgfqpoint{4.328722in}{1.259370in}}%
\pgfpathlineto{\pgfqpoint{4.343336in}{1.265644in}}%
\pgfpathlineto{\pgfqpoint{4.357965in}{1.272092in}}%
\pgfpathlineto{\pgfqpoint{4.372608in}{1.278713in}}%
\pgfpathlineto{\pgfqpoint{4.387266in}{1.285507in}}%
\pgfpathlineto{\pgfqpoint{4.395587in}{1.304344in}}%
\pgfpathlineto{\pgfqpoint{4.403906in}{1.323330in}}%
\pgfpathlineto{\pgfqpoint{4.412223in}{1.342457in}}%
\pgfpathlineto{\pgfqpoint{4.420537in}{1.361717in}}%
\pgfpathlineto{\pgfqpoint{4.405871in}{1.354129in}}%
\pgfpathlineto{\pgfqpoint{4.391220in}{1.346715in}}%
\pgfpathlineto{\pgfqpoint{4.376584in}{1.339476in}}%
\pgfpathlineto{\pgfqpoint{4.361962in}{1.332411in}}%
\pgfpathlineto{\pgfqpoint{4.353656in}{1.313932in}}%
\pgfpathlineto{\pgfqpoint{4.345347in}{1.295594in}}%
\pgfpathlineto{\pgfqpoint{4.337036in}{1.277404in}}%
\pgfpathlineto{\pgfqpoint{4.328722in}{1.259370in}}%
\pgfpathclose%
\pgfusepath{fill}%
\end{pgfscope}%
\begin{pgfscope}%
\pgfpathrectangle{\pgfqpoint{1.150000in}{0.150000in}}{\pgfqpoint{5.700000in}{5.700000in}}%
\pgfusepath{clip}%
\pgfsetbuttcap%
\pgfsetroundjoin%
\definecolor{currentfill}{rgb}{0.127568,0.566949,0.550556}%
\pgfsetfillcolor{currentfill}%
\pgfsetfillopacity{0.800000}%
\pgfsetlinewidth{0.000000pt}%
\definecolor{currentstroke}{rgb}{0.000000,0.000000,0.000000}%
\pgfsetstrokecolor{currentstroke}%
\pgfsetdash{}{0pt}%
\pgfpathmoveto{\pgfqpoint{4.837197in}{2.201143in}}%
\pgfpathlineto{\pgfqpoint{4.852157in}{2.216140in}}%
\pgfpathlineto{\pgfqpoint{4.867137in}{2.231323in}}%
\pgfpathlineto{\pgfqpoint{4.882138in}{2.246692in}}%
\pgfpathlineto{\pgfqpoint{4.897161in}{2.262249in}}%
\pgfpathlineto{\pgfqpoint{4.905424in}{2.283435in}}%
\pgfpathlineto{\pgfqpoint{4.913682in}{2.304506in}}%
\pgfpathlineto{\pgfqpoint{4.921937in}{2.325457in}}%
\pgfpathlineto{\pgfqpoint{4.930187in}{2.346284in}}%
\pgfpathlineto{\pgfqpoint{4.915146in}{2.330240in}}%
\pgfpathlineto{\pgfqpoint{4.900127in}{2.314383in}}%
\pgfpathlineto{\pgfqpoint{4.885130in}{2.298714in}}%
\pgfpathlineto{\pgfqpoint{4.870153in}{2.283233in}}%
\pgfpathlineto{\pgfqpoint{4.861920in}{2.262880in}}%
\pgfpathlineto{\pgfqpoint{4.853683in}{2.242411in}}%
\pgfpathlineto{\pgfqpoint{4.845442in}{2.221831in}}%
\pgfpathlineto{\pgfqpoint{4.837197in}{2.201143in}}%
\pgfpathclose%
\pgfusepath{fill}%
\end{pgfscope}%
\begin{pgfscope}%
\pgfpathrectangle{\pgfqpoint{1.150000in}{0.150000in}}{\pgfqpoint{5.700000in}{5.700000in}}%
\pgfusepath{clip}%
\pgfsetbuttcap%
\pgfsetroundjoin%
\definecolor{currentfill}{rgb}{0.175841,0.441290,0.557685}%
\pgfsetfillcolor{currentfill}%
\pgfsetfillopacity{0.800000}%
\pgfsetlinewidth{0.000000pt}%
\definecolor{currentstroke}{rgb}{0.000000,0.000000,0.000000}%
\pgfsetstrokecolor{currentstroke}%
\pgfsetdash{}{0pt}%
\pgfpathmoveto{\pgfqpoint{4.645433in}{1.810408in}}%
\pgfpathlineto{\pgfqpoint{4.660248in}{1.822406in}}%
\pgfpathlineto{\pgfqpoint{4.675081in}{1.834585in}}%
\pgfpathlineto{\pgfqpoint{4.689932in}{1.846946in}}%
\pgfpathlineto{\pgfqpoint{4.704802in}{1.859487in}}%
\pgfpathlineto{\pgfqpoint{4.713099in}{1.881225in}}%
\pgfpathlineto{\pgfqpoint{4.721393in}{1.902936in}}%
\pgfpathlineto{\pgfqpoint{4.729685in}{1.924615in}}%
\pgfpathlineto{\pgfqpoint{4.737974in}{1.946258in}}%
\pgfpathlineto{\pgfqpoint{4.723087in}{1.933100in}}%
\pgfpathlineto{\pgfqpoint{4.708219in}{1.920125in}}%
\pgfpathlineto{\pgfqpoint{4.693369in}{1.907331in}}%
\pgfpathlineto{\pgfqpoint{4.678539in}{1.894720in}}%
\pgfpathlineto{\pgfqpoint{4.670267in}{1.873680in}}%
\pgfpathlineto{\pgfqpoint{4.661991in}{1.852611in}}%
\pgfpathlineto{\pgfqpoint{4.653714in}{1.831519in}}%
\pgfpathlineto{\pgfqpoint{4.645433in}{1.810408in}}%
\pgfpathclose%
\pgfusepath{fill}%
\end{pgfscope}%
\begin{pgfscope}%
\pgfpathrectangle{\pgfqpoint{1.150000in}{0.150000in}}{\pgfqpoint{5.700000in}{5.700000in}}%
\pgfusepath{clip}%
\pgfsetbuttcap%
\pgfsetroundjoin%
\definecolor{currentfill}{rgb}{0.458674,0.816363,0.329727}%
\pgfsetfillcolor{currentfill}%
\pgfsetfillopacity{0.800000}%
\pgfsetlinewidth{0.000000pt}%
\definecolor{currentstroke}{rgb}{0.000000,0.000000,0.000000}%
\pgfsetstrokecolor{currentstroke}%
\pgfsetdash{}{0pt}%
\pgfpathmoveto{\pgfqpoint{5.313180in}{3.082206in}}%
\pgfpathlineto{\pgfqpoint{5.328553in}{3.102639in}}%
\pgfpathlineto{\pgfqpoint{5.343952in}{3.123269in}}%
\pgfpathlineto{\pgfqpoint{5.359376in}{3.144098in}}%
\pgfpathlineto{\pgfqpoint{5.367435in}{3.159464in}}%
\pgfpathlineto{\pgfqpoint{5.375485in}{3.174585in}}%
\pgfpathlineto{\pgfqpoint{5.383524in}{3.189459in}}%
\pgfpathlineto{\pgfqpoint{5.391554in}{3.204086in}}%
\pgfpathlineto{\pgfqpoint{5.376120in}{3.183089in}}%
\pgfpathlineto{\pgfqpoint{5.360711in}{3.162290in}}%
\pgfpathlineto{\pgfqpoint{5.345328in}{3.141690in}}%
\pgfpathlineto{\pgfqpoint{5.337306in}{3.127178in}}%
\pgfpathlineto{\pgfqpoint{5.329273in}{3.112426in}}%
\pgfpathlineto{\pgfqpoint{5.321231in}{3.097435in}}%
\pgfpathlineto{\pgfqpoint{5.313180in}{3.082206in}}%
\pgfpathclose%
\pgfusepath{fill}%
\end{pgfscope}%
\begin{pgfscope}%
\pgfpathrectangle{\pgfqpoint{1.150000in}{0.150000in}}{\pgfqpoint{5.700000in}{5.700000in}}%
\pgfusepath{clip}%
\pgfsetbuttcap%
\pgfsetroundjoin%
\definecolor{currentfill}{rgb}{0.231674,0.318106,0.544834}%
\pgfsetfillcolor{currentfill}%
\pgfsetfillopacity{0.800000}%
\pgfsetlinewidth{0.000000pt}%
\definecolor{currentstroke}{rgb}{0.000000,0.000000,0.000000}%
\pgfsetstrokecolor{currentstroke}%
\pgfsetdash{}{0pt}%
\pgfpathmoveto{\pgfqpoint{4.453773in}{1.439947in}}%
\pgfpathlineto{\pgfqpoint{4.468465in}{1.448474in}}%
\pgfpathlineto{\pgfqpoint{4.483172in}{1.457177in}}%
\pgfpathlineto{\pgfqpoint{4.497895in}{1.466055in}}%
\pgfpathlineto{\pgfqpoint{4.512634in}{1.475110in}}%
\pgfpathlineto{\pgfqpoint{4.520950in}{1.495676in}}%
\pgfpathlineto{\pgfqpoint{4.529263in}{1.516325in}}%
\pgfpathlineto{\pgfqpoint{4.537575in}{1.537051in}}%
\pgfpathlineto{\pgfqpoint{4.545885in}{1.557845in}}%
\pgfpathlineto{\pgfqpoint{4.531132in}{1.548053in}}%
\pgfpathlineto{\pgfqpoint{4.516397in}{1.538438in}}%
\pgfpathlineto{\pgfqpoint{4.501677in}{1.529001in}}%
\pgfpathlineto{\pgfqpoint{4.486974in}{1.519740in}}%
\pgfpathlineto{\pgfqpoint{4.478677in}{1.499669in}}%
\pgfpathlineto{\pgfqpoint{4.470378in}{1.479676in}}%
\pgfpathlineto{\pgfqpoint{4.462077in}{1.459766in}}%
\pgfpathlineto{\pgfqpoint{4.453773in}{1.439947in}}%
\pgfpathclose%
\pgfusepath{fill}%
\end{pgfscope}%
\begin{pgfscope}%
\pgfpathrectangle{\pgfqpoint{1.150000in}{0.150000in}}{\pgfqpoint{5.700000in}{5.700000in}}%
\pgfusepath{clip}%
\pgfsetbuttcap%
\pgfsetroundjoin%
\definecolor{currentfill}{rgb}{0.140210,0.665859,0.513427}%
\pgfsetfillcolor{currentfill}%
\pgfsetfillopacity{0.800000}%
\pgfsetlinewidth{0.000000pt}%
\definecolor{currentstroke}{rgb}{0.000000,0.000000,0.000000}%
\pgfsetstrokecolor{currentstroke}%
\pgfsetdash{}{0pt}%
\pgfpathmoveto{\pgfqpoint{4.996012in}{2.507937in}}%
\pgfpathlineto{\pgfqpoint{5.011109in}{2.525043in}}%
\pgfpathlineto{\pgfqpoint{5.026229in}{2.542340in}}%
\pgfpathlineto{\pgfqpoint{5.041371in}{2.559827in}}%
\pgfpathlineto{\pgfqpoint{5.056536in}{2.577506in}}%
\pgfpathlineto{\pgfqpoint{5.064757in}{2.597433in}}%
\pgfpathlineto{\pgfqpoint{5.072973in}{2.617189in}}%
\pgfpathlineto{\pgfqpoint{5.081182in}{2.636770in}}%
\pgfpathlineto{\pgfqpoint{5.089386in}{2.656172in}}%
\pgfpathlineto{\pgfqpoint{5.074204in}{2.638107in}}%
\pgfpathlineto{\pgfqpoint{5.059045in}{2.620233in}}%
\pgfpathlineto{\pgfqpoint{5.043909in}{2.602552in}}%
\pgfpathlineto{\pgfqpoint{5.028795in}{2.585061in}}%
\pgfpathlineto{\pgfqpoint{5.020608in}{2.566031in}}%
\pgfpathlineto{\pgfqpoint{5.012415in}{2.546831in}}%
\pgfpathlineto{\pgfqpoint{5.004217in}{2.527465in}}%
\pgfpathlineto{\pgfqpoint{4.996012in}{2.507937in}}%
\pgfpathclose%
\pgfusepath{fill}%
\end{pgfscope}%
\begin{pgfscope}%
\pgfpathrectangle{\pgfqpoint{1.150000in}{0.150000in}}{\pgfqpoint{5.700000in}{5.700000in}}%
\pgfusepath{clip}%
\pgfsetbuttcap%
\pgfsetroundjoin%
\definecolor{currentfill}{rgb}{0.135066,0.544853,0.554029}%
\pgfsetfillcolor{currentfill}%
\pgfsetfillopacity{0.800000}%
\pgfsetlinewidth{0.000000pt}%
\definecolor{currentstroke}{rgb}{0.000000,0.000000,0.000000}%
\pgfsetstrokecolor{currentstroke}%
\pgfsetdash{}{0pt}%
\pgfpathmoveto{\pgfqpoint{4.804178in}{2.117415in}}%
\pgfpathlineto{\pgfqpoint{4.819120in}{2.131893in}}%
\pgfpathlineto{\pgfqpoint{4.834082in}{2.146556in}}%
\pgfpathlineto{\pgfqpoint{4.849065in}{2.161405in}}%
\pgfpathlineto{\pgfqpoint{4.864069in}{2.176439in}}%
\pgfpathlineto{\pgfqpoint{4.872347in}{2.198042in}}%
\pgfpathlineto{\pgfqpoint{4.880622in}{2.219548in}}%
\pgfpathlineto{\pgfqpoint{4.888893in}{2.240951in}}%
\pgfpathlineto{\pgfqpoint{4.897161in}{2.262249in}}%
\pgfpathlineto{\pgfqpoint{4.882138in}{2.246692in}}%
\pgfpathlineto{\pgfqpoint{4.867137in}{2.231323in}}%
\pgfpathlineto{\pgfqpoint{4.852157in}{2.216140in}}%
\pgfpathlineto{\pgfqpoint{4.837197in}{2.201143in}}%
\pgfpathlineto{\pgfqpoint{4.828948in}{2.180354in}}%
\pgfpathlineto{\pgfqpoint{4.820695in}{2.159466in}}%
\pgfpathlineto{\pgfqpoint{4.812438in}{2.138485in}}%
\pgfpathlineto{\pgfqpoint{4.804178in}{2.117415in}}%
\pgfpathclose%
\pgfusepath{fill}%
\end{pgfscope}%
\begin{pgfscope}%
\pgfpathrectangle{\pgfqpoint{1.150000in}{0.150000in}}{\pgfqpoint{5.700000in}{5.700000in}}%
\pgfusepath{clip}%
\pgfsetbuttcap%
\pgfsetroundjoin%
\definecolor{currentfill}{rgb}{0.185556,0.418570,0.556753}%
\pgfsetfillcolor{currentfill}%
\pgfsetfillopacity{0.800000}%
\pgfsetlinewidth{0.000000pt}%
\definecolor{currentstroke}{rgb}{0.000000,0.000000,0.000000}%
\pgfsetstrokecolor{currentstroke}%
\pgfsetdash{}{0pt}%
\pgfpathmoveto{\pgfqpoint{4.612287in}{1.725903in}}%
\pgfpathlineto{\pgfqpoint{4.627086in}{1.737256in}}%
\pgfpathlineto{\pgfqpoint{4.641902in}{1.748789in}}%
\pgfpathlineto{\pgfqpoint{4.656737in}{1.760502in}}%
\pgfpathlineto{\pgfqpoint{4.671590in}{1.772395in}}%
\pgfpathlineto{\pgfqpoint{4.679896in}{1.794178in}}%
\pgfpathlineto{\pgfqpoint{4.688200in}{1.815958in}}%
\pgfpathlineto{\pgfqpoint{4.696502in}{1.837730in}}%
\pgfpathlineto{\pgfqpoint{4.704802in}{1.859487in}}%
\pgfpathlineto{\pgfqpoint{4.689932in}{1.846946in}}%
\pgfpathlineto{\pgfqpoint{4.675081in}{1.834585in}}%
\pgfpathlineto{\pgfqpoint{4.660248in}{1.822406in}}%
\pgfpathlineto{\pgfqpoint{4.645433in}{1.810408in}}%
\pgfpathlineto{\pgfqpoint{4.637150in}{1.789285in}}%
\pgfpathlineto{\pgfqpoint{4.628865in}{1.768156in}}%
\pgfpathlineto{\pgfqpoint{4.620577in}{1.747027in}}%
\pgfpathlineto{\pgfqpoint{4.612287in}{1.725903in}}%
\pgfpathclose%
\pgfusepath{fill}%
\end{pgfscope}%
\begin{pgfscope}%
\pgfpathrectangle{\pgfqpoint{1.150000in}{0.150000in}}{\pgfqpoint{5.700000in}{5.700000in}}%
\pgfusepath{clip}%
\pgfsetbuttcap%
\pgfsetroundjoin%
\definecolor{currentfill}{rgb}{0.269308,0.218818,0.509577}%
\pgfsetfillcolor{currentfill}%
\pgfsetfillopacity{0.800000}%
\pgfsetlinewidth{0.000000pt}%
\definecolor{currentstroke}{rgb}{0.000000,0.000000,0.000000}%
\pgfsetstrokecolor{currentstroke}%
\pgfsetdash{}{0pt}%
\pgfpathmoveto{\pgfqpoint{4.295438in}{1.188949in}}%
\pgfpathlineto{\pgfqpoint{4.310047in}{1.194402in}}%
\pgfpathlineto{\pgfqpoint{4.324670in}{1.200027in}}%
\pgfpathlineto{\pgfqpoint{4.339306in}{1.205825in}}%
\pgfpathlineto{\pgfqpoint{4.353956in}{1.211794in}}%
\pgfpathlineto{\pgfqpoint{4.362287in}{1.229961in}}%
\pgfpathlineto{\pgfqpoint{4.370616in}{1.248308in}}%
\pgfpathlineto{\pgfqpoint{4.378942in}{1.266825in}}%
\pgfpathlineto{\pgfqpoint{4.387266in}{1.285507in}}%
\pgfpathlineto{\pgfqpoint{4.372608in}{1.278713in}}%
\pgfpathlineto{\pgfqpoint{4.357965in}{1.272092in}}%
\pgfpathlineto{\pgfqpoint{4.343336in}{1.265644in}}%
\pgfpathlineto{\pgfqpoint{4.328722in}{1.259370in}}%
\pgfpathlineto{\pgfqpoint{4.320405in}{1.241500in}}%
\pgfpathlineto{\pgfqpoint{4.312085in}{1.223801in}}%
\pgfpathlineto{\pgfqpoint{4.303763in}{1.206281in}}%
\pgfpathlineto{\pgfqpoint{4.295438in}{1.188949in}}%
\pgfpathclose%
\pgfusepath{fill}%
\end{pgfscope}%
\begin{pgfscope}%
\pgfpathrectangle{\pgfqpoint{1.150000in}{0.150000in}}{\pgfqpoint{5.700000in}{5.700000in}}%
\pgfusepath{clip}%
\pgfsetbuttcap%
\pgfsetroundjoin%
\definecolor{currentfill}{rgb}{0.274149,0.751988,0.436601}%
\pgfsetfillcolor{currentfill}%
\pgfsetfillopacity{0.800000}%
\pgfsetlinewidth{0.000000pt}%
\definecolor{currentstroke}{rgb}{0.000000,0.000000,0.000000}%
\pgfsetstrokecolor{currentstroke}%
\pgfsetdash{}{0pt}%
\pgfpathmoveto{\pgfqpoint{5.154772in}{2.804601in}}%
\pgfpathlineto{\pgfqpoint{5.170009in}{2.823525in}}%
\pgfpathlineto{\pgfqpoint{5.185269in}{2.842643in}}%
\pgfpathlineto{\pgfqpoint{5.200554in}{2.861956in}}%
\pgfpathlineto{\pgfqpoint{5.215864in}{2.881464in}}%
\pgfpathlineto{\pgfqpoint{5.224019in}{2.899419in}}%
\pgfpathlineto{\pgfqpoint{5.232167in}{2.917159in}}%
\pgfpathlineto{\pgfqpoint{5.240307in}{2.934680in}}%
\pgfpathlineto{\pgfqpoint{5.248439in}{2.951981in}}%
\pgfpathlineto{\pgfqpoint{5.233115in}{2.932193in}}%
\pgfpathlineto{\pgfqpoint{5.217816in}{2.912600in}}%
\pgfpathlineto{\pgfqpoint{5.202541in}{2.893203in}}%
\pgfpathlineto{\pgfqpoint{5.187291in}{2.874001in}}%
\pgfpathlineto{\pgfqpoint{5.179173in}{2.856965in}}%
\pgfpathlineto{\pgfqpoint{5.171047in}{2.839719in}}%
\pgfpathlineto{\pgfqpoint{5.162913in}{2.822263in}}%
\pgfpathlineto{\pgfqpoint{5.154772in}{2.804601in}}%
\pgfpathclose%
\pgfusepath{fill}%
\end{pgfscope}%
\begin{pgfscope}%
\pgfpathrectangle{\pgfqpoint{1.150000in}{0.150000in}}{\pgfqpoint{5.700000in}{5.700000in}}%
\pgfusepath{clip}%
\pgfsetbuttcap%
\pgfsetroundjoin%
\definecolor{currentfill}{rgb}{0.243113,0.292092,0.538516}%
\pgfsetfillcolor{currentfill}%
\pgfsetfillopacity{0.800000}%
\pgfsetlinewidth{0.000000pt}%
\definecolor{currentstroke}{rgb}{0.000000,0.000000,0.000000}%
\pgfsetstrokecolor{currentstroke}%
\pgfsetdash{}{0pt}%
\pgfpathmoveto{\pgfqpoint{4.420537in}{1.361717in}}%
\pgfpathlineto{\pgfqpoint{4.435218in}{1.369479in}}%
\pgfpathlineto{\pgfqpoint{4.449914in}{1.377416in}}%
\pgfpathlineto{\pgfqpoint{4.464626in}{1.385528in}}%
\pgfpathlineto{\pgfqpoint{4.479353in}{1.393814in}}%
\pgfpathlineto{\pgfqpoint{4.487676in}{1.413979in}}%
\pgfpathlineto{\pgfqpoint{4.495997in}{1.434254in}}%
\pgfpathlineto{\pgfqpoint{4.504317in}{1.454633in}}%
\pgfpathlineto{\pgfqpoint{4.512634in}{1.475110in}}%
\pgfpathlineto{\pgfqpoint{4.497895in}{1.466055in}}%
\pgfpathlineto{\pgfqpoint{4.483172in}{1.457177in}}%
\pgfpathlineto{\pgfqpoint{4.468465in}{1.448474in}}%
\pgfpathlineto{\pgfqpoint{4.453773in}{1.439947in}}%
\pgfpathlineto{\pgfqpoint{4.445468in}{1.420225in}}%
\pgfpathlineto{\pgfqpoint{4.437160in}{1.400609in}}%
\pgfpathlineto{\pgfqpoint{4.428850in}{1.381103in}}%
\pgfpathlineto{\pgfqpoint{4.420537in}{1.361717in}}%
\pgfpathclose%
\pgfusepath{fill}%
\end{pgfscope}%
\begin{pgfscope}%
\pgfpathrectangle{\pgfqpoint{1.150000in}{0.150000in}}{\pgfqpoint{5.700000in}{5.700000in}}%
\pgfusepath{clip}%
\pgfsetbuttcap%
\pgfsetroundjoin%
\definecolor{currentfill}{rgb}{0.144759,0.519093,0.556572}%
\pgfsetfillcolor{currentfill}%
\pgfsetfillopacity{0.800000}%
\pgfsetlinewidth{0.000000pt}%
\definecolor{currentstroke}{rgb}{0.000000,0.000000,0.000000}%
\pgfsetstrokecolor{currentstroke}%
\pgfsetdash{}{0pt}%
\pgfpathmoveto{\pgfqpoint{4.771101in}{2.032347in}}%
\pgfpathlineto{\pgfqpoint{4.786026in}{2.046273in}}%
\pgfpathlineto{\pgfqpoint{4.800970in}{2.060383in}}%
\pgfpathlineto{\pgfqpoint{4.815934in}{2.074677in}}%
\pgfpathlineto{\pgfqpoint{4.830919in}{2.089156in}}%
\pgfpathlineto{\pgfqpoint{4.839211in}{2.111098in}}%
\pgfpathlineto{\pgfqpoint{4.847501in}{2.132962in}}%
\pgfpathlineto{\pgfqpoint{4.855786in}{2.154745in}}%
\pgfpathlineto{\pgfqpoint{4.864069in}{2.176439in}}%
\pgfpathlineto{\pgfqpoint{4.849065in}{2.161405in}}%
\pgfpathlineto{\pgfqpoint{4.834082in}{2.146556in}}%
\pgfpathlineto{\pgfqpoint{4.819120in}{2.131893in}}%
\pgfpathlineto{\pgfqpoint{4.804178in}{2.117415in}}%
\pgfpathlineto{\pgfqpoint{4.795914in}{2.096262in}}%
\pgfpathlineto{\pgfqpoint{4.787646in}{2.075029in}}%
\pgfpathlineto{\pgfqpoint{4.779375in}{2.053723in}}%
\pgfpathlineto{\pgfqpoint{4.771101in}{2.032347in}}%
\pgfpathclose%
\pgfusepath{fill}%
\end{pgfscope}%
\begin{pgfscope}%
\pgfpathrectangle{\pgfqpoint{1.150000in}{0.150000in}}{\pgfqpoint{5.700000in}{5.700000in}}%
\pgfusepath{clip}%
\pgfsetbuttcap%
\pgfsetroundjoin%
\definecolor{currentfill}{rgb}{0.195860,0.395433,0.555276}%
\pgfsetfillcolor{currentfill}%
\pgfsetfillopacity{0.800000}%
\pgfsetlinewidth{0.000000pt}%
\definecolor{currentstroke}{rgb}{0.000000,0.000000,0.000000}%
\pgfsetstrokecolor{currentstroke}%
\pgfsetdash{}{0pt}%
\pgfpathmoveto{\pgfqpoint{4.579103in}{1.641582in}}%
\pgfpathlineto{\pgfqpoint{4.593886in}{1.652259in}}%
\pgfpathlineto{\pgfqpoint{4.608687in}{1.663114in}}%
\pgfpathlineto{\pgfqpoint{4.623506in}{1.674148in}}%
\pgfpathlineto{\pgfqpoint{4.638342in}{1.685362in}}%
\pgfpathlineto{\pgfqpoint{4.646657in}{1.707093in}}%
\pgfpathlineto{\pgfqpoint{4.654970in}{1.728847in}}%
\pgfpathlineto{\pgfqpoint{4.663281in}{1.750616in}}%
\pgfpathlineto{\pgfqpoint{4.671590in}{1.772395in}}%
\pgfpathlineto{\pgfqpoint{4.656737in}{1.760502in}}%
\pgfpathlineto{\pgfqpoint{4.641902in}{1.748789in}}%
\pgfpathlineto{\pgfqpoint{4.627086in}{1.737256in}}%
\pgfpathlineto{\pgfqpoint{4.612287in}{1.725903in}}%
\pgfpathlineto{\pgfqpoint{4.603994in}{1.704790in}}%
\pgfpathlineto{\pgfqpoint{4.595699in}{1.683695in}}%
\pgfpathlineto{\pgfqpoint{4.587402in}{1.662624in}}%
\pgfpathlineto{\pgfqpoint{4.579103in}{1.641582in}}%
\pgfpathclose%
\pgfusepath{fill}%
\end{pgfscope}%
\begin{pgfscope}%
\pgfpathrectangle{\pgfqpoint{1.150000in}{0.150000in}}{\pgfqpoint{5.700000in}{5.700000in}}%
\pgfusepath{clip}%
\pgfsetbuttcap%
\pgfsetroundjoin%
\definecolor{currentfill}{rgb}{0.126326,0.644107,0.525311}%
\pgfsetfillcolor{currentfill}%
\pgfsetfillopacity{0.800000}%
\pgfsetlinewidth{0.000000pt}%
\definecolor{currentstroke}{rgb}{0.000000,0.000000,0.000000}%
\pgfsetstrokecolor{currentstroke}%
\pgfsetdash{}{0pt}%
\pgfpathmoveto{\pgfqpoint{4.963140in}{2.428266in}}%
\pgfpathlineto{\pgfqpoint{4.978220in}{2.444952in}}%
\pgfpathlineto{\pgfqpoint{4.993322in}{2.461829in}}%
\pgfpathlineto{\pgfqpoint{5.008446in}{2.478895in}}%
\pgfpathlineto{\pgfqpoint{5.023593in}{2.496151in}}%
\pgfpathlineto{\pgfqpoint{5.031837in}{2.516729in}}%
\pgfpathlineto{\pgfqpoint{5.040075in}{2.537150in}}%
\pgfpathlineto{\pgfqpoint{5.048308in}{2.557410in}}%
\pgfpathlineto{\pgfqpoint{5.056536in}{2.577506in}}%
\pgfpathlineto{\pgfqpoint{5.041371in}{2.559827in}}%
\pgfpathlineto{\pgfqpoint{5.026229in}{2.542340in}}%
\pgfpathlineto{\pgfqpoint{5.011109in}{2.525043in}}%
\pgfpathlineto{\pgfqpoint{4.996012in}{2.507937in}}%
\pgfpathlineto{\pgfqpoint{4.987802in}{2.488249in}}%
\pgfpathlineto{\pgfqpoint{4.979587in}{2.468405in}}%
\pgfpathlineto{\pgfqpoint{4.971366in}{2.448410in}}%
\pgfpathlineto{\pgfqpoint{4.963140in}{2.428266in}}%
\pgfpathclose%
\pgfusepath{fill}%
\end{pgfscope}%
\begin{pgfscope}%
\pgfpathrectangle{\pgfqpoint{1.150000in}{0.150000in}}{\pgfqpoint{5.700000in}{5.700000in}}%
\pgfusepath{clip}%
\pgfsetbuttcap%
\pgfsetroundjoin%
\definecolor{currentfill}{rgb}{0.421908,0.805774,0.351910}%
\pgfsetfillcolor{currentfill}%
\pgfsetfillopacity{0.800000}%
\pgfsetlinewidth{0.000000pt}%
\definecolor{currentstroke}{rgb}{0.000000,0.000000,0.000000}%
\pgfsetstrokecolor{currentstroke}%
\pgfsetdash{}{0pt}%
\pgfpathmoveto{\pgfqpoint{5.280880in}{3.018936in}}%
\pgfpathlineto{\pgfqpoint{5.296242in}{3.039164in}}%
\pgfpathlineto{\pgfqpoint{5.311629in}{3.059588in}}%
\pgfpathlineto{\pgfqpoint{5.327042in}{3.080210in}}%
\pgfpathlineto{\pgfqpoint{5.335139in}{3.096542in}}%
\pgfpathlineto{\pgfqpoint{5.343228in}{3.112636in}}%
\pgfpathlineto{\pgfqpoint{5.351307in}{3.128488in}}%
\pgfpathlineto{\pgfqpoint{5.359376in}{3.144098in}}%
\pgfpathlineto{\pgfqpoint{5.343952in}{3.123269in}}%
\pgfpathlineto{\pgfqpoint{5.328553in}{3.102639in}}%
\pgfpathlineto{\pgfqpoint{5.313180in}{3.082206in}}%
\pgfpathlineto{\pgfqpoint{5.305119in}{3.066740in}}%
\pgfpathlineto{\pgfqpoint{5.297048in}{3.051039in}}%
\pgfpathlineto{\pgfqpoint{5.288969in}{3.035103in}}%
\pgfpathlineto{\pgfqpoint{5.280880in}{3.018936in}}%
\pgfpathclose%
\pgfusepath{fill}%
\end{pgfscope}%
\begin{pgfscope}%
\pgfpathrectangle{\pgfqpoint{1.150000in}{0.150000in}}{\pgfqpoint{5.700000in}{5.700000in}}%
\pgfusepath{clip}%
\pgfsetbuttcap%
\pgfsetroundjoin%
\definecolor{currentfill}{rgb}{0.239374,0.735588,0.455688}%
\pgfsetfillcolor{currentfill}%
\pgfsetfillopacity{0.800000}%
\pgfsetlinewidth{0.000000pt}%
\definecolor{currentstroke}{rgb}{0.000000,0.000000,0.000000}%
\pgfsetstrokecolor{currentstroke}%
\pgfsetdash{}{0pt}%
\pgfpathmoveto{\pgfqpoint{5.122134in}{2.731934in}}%
\pgfpathlineto{\pgfqpoint{5.137356in}{2.750544in}}%
\pgfpathlineto{\pgfqpoint{5.152601in}{2.769346in}}%
\pgfpathlineto{\pgfqpoint{5.167870in}{2.788343in}}%
\pgfpathlineto{\pgfqpoint{5.183164in}{2.807533in}}%
\pgfpathlineto{\pgfqpoint{5.191350in}{2.826327in}}%
\pgfpathlineto{\pgfqpoint{5.199529in}{2.844915in}}%
\pgfpathlineto{\pgfqpoint{5.207700in}{2.863295in}}%
\pgfpathlineto{\pgfqpoint{5.215864in}{2.881464in}}%
\pgfpathlineto{\pgfqpoint{5.200554in}{2.861956in}}%
\pgfpathlineto{\pgfqpoint{5.185269in}{2.842643in}}%
\pgfpathlineto{\pgfqpoint{5.170009in}{2.823525in}}%
\pgfpathlineto{\pgfqpoint{5.154772in}{2.804601in}}%
\pgfpathlineto{\pgfqpoint{5.146623in}{2.786734in}}%
\pgfpathlineto{\pgfqpoint{5.138467in}{2.768666in}}%
\pgfpathlineto{\pgfqpoint{5.130304in}{2.750398in}}%
\pgfpathlineto{\pgfqpoint{5.122134in}{2.731934in}}%
\pgfpathclose%
\pgfusepath{fill}%
\end{pgfscope}%
\begin{pgfscope}%
\pgfpathrectangle{\pgfqpoint{1.150000in}{0.150000in}}{\pgfqpoint{5.700000in}{5.700000in}}%
\pgfusepath{clip}%
\pgfsetbuttcap%
\pgfsetroundjoin%
\definecolor{currentfill}{rgb}{0.153364,0.497000,0.557724}%
\pgfsetfillcolor{currentfill}%
\pgfsetfillopacity{0.800000}%
\pgfsetlinewidth{0.000000pt}%
\definecolor{currentstroke}{rgb}{0.000000,0.000000,0.000000}%
\pgfsetstrokecolor{currentstroke}%
\pgfsetdash{}{0pt}%
\pgfpathmoveto{\pgfqpoint{4.737974in}{1.946258in}}%
\pgfpathlineto{\pgfqpoint{4.752881in}{1.959599in}}%
\pgfpathlineto{\pgfqpoint{4.767807in}{1.973122in}}%
\pgfpathlineto{\pgfqpoint{4.782752in}{1.986829in}}%
\pgfpathlineto{\pgfqpoint{4.797717in}{2.000720in}}%
\pgfpathlineto{\pgfqpoint{4.806022in}{2.022919in}}%
\pgfpathlineto{\pgfqpoint{4.814324in}{2.045061in}}%
\pgfpathlineto{\pgfqpoint{4.822623in}{2.067142in}}%
\pgfpathlineto{\pgfqpoint{4.830919in}{2.089156in}}%
\pgfpathlineto{\pgfqpoint{4.815934in}{2.074677in}}%
\pgfpathlineto{\pgfqpoint{4.800970in}{2.060383in}}%
\pgfpathlineto{\pgfqpoint{4.786026in}{2.046273in}}%
\pgfpathlineto{\pgfqpoint{4.771101in}{2.032347in}}%
\pgfpathlineto{\pgfqpoint{4.762824in}{2.010908in}}%
\pgfpathlineto{\pgfqpoint{4.754544in}{1.989410in}}%
\pgfpathlineto{\pgfqpoint{4.746260in}{1.967858in}}%
\pgfpathlineto{\pgfqpoint{4.737974in}{1.946258in}}%
\pgfpathclose%
\pgfusepath{fill}%
\end{pgfscope}%
\begin{pgfscope}%
\pgfpathrectangle{\pgfqpoint{1.150000in}{0.150000in}}{\pgfqpoint{5.700000in}{5.700000in}}%
\pgfusepath{clip}%
\pgfsetbuttcap%
\pgfsetroundjoin%
\definecolor{currentfill}{rgb}{0.252194,0.269783,0.531579}%
\pgfsetfillcolor{currentfill}%
\pgfsetfillopacity{0.800000}%
\pgfsetlinewidth{0.000000pt}%
\definecolor{currentstroke}{rgb}{0.000000,0.000000,0.000000}%
\pgfsetstrokecolor{currentstroke}%
\pgfsetdash{}{0pt}%
\pgfpathmoveto{\pgfqpoint{4.387266in}{1.285507in}}%
\pgfpathlineto{\pgfqpoint{4.401937in}{1.292474in}}%
\pgfpathlineto{\pgfqpoint{4.416624in}{1.299615in}}%
\pgfpathlineto{\pgfqpoint{4.431325in}{1.306930in}}%
\pgfpathlineto{\pgfqpoint{4.446041in}{1.314418in}}%
\pgfpathlineto{\pgfqpoint{4.454372in}{1.334063in}}%
\pgfpathlineto{\pgfqpoint{4.462701in}{1.353849in}}%
\pgfpathlineto{\pgfqpoint{4.471028in}{1.373769in}}%
\pgfpathlineto{\pgfqpoint{4.479353in}{1.393814in}}%
\pgfpathlineto{\pgfqpoint{4.464626in}{1.385528in}}%
\pgfpathlineto{\pgfqpoint{4.449914in}{1.377416in}}%
\pgfpathlineto{\pgfqpoint{4.435218in}{1.369479in}}%
\pgfpathlineto{\pgfqpoint{4.420537in}{1.361717in}}%
\pgfpathlineto{\pgfqpoint{4.412223in}{1.342457in}}%
\pgfpathlineto{\pgfqpoint{4.403906in}{1.323330in}}%
\pgfpathlineto{\pgfqpoint{4.395587in}{1.304344in}}%
\pgfpathlineto{\pgfqpoint{4.387266in}{1.285507in}}%
\pgfpathclose%
\pgfusepath{fill}%
\end{pgfscope}%
\begin{pgfscope}%
\pgfpathrectangle{\pgfqpoint{1.150000in}{0.150000in}}{\pgfqpoint{5.700000in}{5.700000in}}%
\pgfusepath{clip}%
\pgfsetbuttcap%
\pgfsetroundjoin%
\definecolor{currentfill}{rgb}{0.208623,0.367752,0.552675}%
\pgfsetfillcolor{currentfill}%
\pgfsetfillopacity{0.800000}%
\pgfsetlinewidth{0.000000pt}%
\definecolor{currentstroke}{rgb}{0.000000,0.000000,0.000000}%
\pgfsetstrokecolor{currentstroke}%
\pgfsetdash{}{0pt}%
\pgfpathmoveto{\pgfqpoint{4.545885in}{1.557845in}}%
\pgfpathlineto{\pgfqpoint{4.560654in}{1.567814in}}%
\pgfpathlineto{\pgfqpoint{4.575439in}{1.577961in}}%
\pgfpathlineto{\pgfqpoint{4.590242in}{1.588285in}}%
\pgfpathlineto{\pgfqpoint{4.605062in}{1.598787in}}%
\pgfpathlineto{\pgfqpoint{4.613385in}{1.620365in}}%
\pgfpathlineto{\pgfqpoint{4.621706in}{1.641991in}}%
\pgfpathlineto{\pgfqpoint{4.630025in}{1.663659in}}%
\pgfpathlineto{\pgfqpoint{4.638342in}{1.685362in}}%
\pgfpathlineto{\pgfqpoint{4.623506in}{1.674148in}}%
\pgfpathlineto{\pgfqpoint{4.608687in}{1.663114in}}%
\pgfpathlineto{\pgfqpoint{4.593886in}{1.652259in}}%
\pgfpathlineto{\pgfqpoint{4.579103in}{1.641582in}}%
\pgfpathlineto{\pgfqpoint{4.570801in}{1.620577in}}%
\pgfpathlineto{\pgfqpoint{4.562498in}{1.599615in}}%
\pgfpathlineto{\pgfqpoint{4.554192in}{1.578702in}}%
\pgfpathlineto{\pgfqpoint{4.545885in}{1.557845in}}%
\pgfpathclose%
\pgfusepath{fill}%
\end{pgfscope}%
\begin{pgfscope}%
\pgfpathrectangle{\pgfqpoint{1.150000in}{0.150000in}}{\pgfqpoint{5.700000in}{5.700000in}}%
\pgfusepath{clip}%
\pgfsetbuttcap%
\pgfsetroundjoin%
\definecolor{currentfill}{rgb}{0.120081,0.622161,0.534946}%
\pgfsetfillcolor{currentfill}%
\pgfsetfillopacity{0.800000}%
\pgfsetlinewidth{0.000000pt}%
\definecolor{currentstroke}{rgb}{0.000000,0.000000,0.000000}%
\pgfsetstrokecolor{currentstroke}%
\pgfsetdash{}{0pt}%
\pgfpathmoveto{\pgfqpoint{4.930187in}{2.346284in}}%
\pgfpathlineto{\pgfqpoint{4.945249in}{2.362516in}}%
\pgfpathlineto{\pgfqpoint{4.960332in}{2.378937in}}%
\pgfpathlineto{\pgfqpoint{4.975438in}{2.395547in}}%
\pgfpathlineto{\pgfqpoint{4.990566in}{2.412346in}}%
\pgfpathlineto{\pgfqpoint{4.998830in}{2.433513in}}%
\pgfpathlineto{\pgfqpoint{5.007089in}{2.454539in}}%
\pgfpathlineto{\pgfqpoint{5.015343in}{2.475420in}}%
\pgfpathlineto{\pgfqpoint{5.023593in}{2.496151in}}%
\pgfpathlineto{\pgfqpoint{5.008446in}{2.478895in}}%
\pgfpathlineto{\pgfqpoint{4.993322in}{2.461829in}}%
\pgfpathlineto{\pgfqpoint{4.978220in}{2.444952in}}%
\pgfpathlineto{\pgfqpoint{4.963140in}{2.428266in}}%
\pgfpathlineto{\pgfqpoint{4.954909in}{2.407977in}}%
\pgfpathlineto{\pgfqpoint{4.946673in}{2.387548in}}%
\pgfpathlineto{\pgfqpoint{4.938432in}{2.366982in}}%
\pgfpathlineto{\pgfqpoint{4.930187in}{2.346284in}}%
\pgfpathclose%
\pgfusepath{fill}%
\end{pgfscope}%
\begin{pgfscope}%
\pgfpathrectangle{\pgfqpoint{1.150000in}{0.150000in}}{\pgfqpoint{5.700000in}{5.700000in}}%
\pgfusepath{clip}%
\pgfsetbuttcap%
\pgfsetroundjoin%
\definecolor{currentfill}{rgb}{0.163625,0.471133,0.558148}%
\pgfsetfillcolor{currentfill}%
\pgfsetfillopacity{0.800000}%
\pgfsetlinewidth{0.000000pt}%
\definecolor{currentstroke}{rgb}{0.000000,0.000000,0.000000}%
\pgfsetstrokecolor{currentstroke}%
\pgfsetdash{}{0pt}%
\pgfpathmoveto{\pgfqpoint{4.704802in}{1.859487in}}%
\pgfpathlineto{\pgfqpoint{4.719691in}{1.872210in}}%
\pgfpathlineto{\pgfqpoint{4.734598in}{1.885115in}}%
\pgfpathlineto{\pgfqpoint{4.749525in}{1.898202in}}%
\pgfpathlineto{\pgfqpoint{4.764471in}{1.911472in}}%
\pgfpathlineto{\pgfqpoint{4.772787in}{1.933840in}}%
\pgfpathlineto{\pgfqpoint{4.781099in}{1.956175in}}%
\pgfpathlineto{\pgfqpoint{4.789410in}{1.978470in}}%
\pgfpathlineto{\pgfqpoint{4.797717in}{2.000720in}}%
\pgfpathlineto{\pgfqpoint{4.782752in}{1.986829in}}%
\pgfpathlineto{\pgfqpoint{4.767807in}{1.973122in}}%
\pgfpathlineto{\pgfqpoint{4.752881in}{1.959599in}}%
\pgfpathlineto{\pgfqpoint{4.737974in}{1.946258in}}%
\pgfpathlineto{\pgfqpoint{4.729685in}{1.924615in}}%
\pgfpathlineto{\pgfqpoint{4.721393in}{1.902936in}}%
\pgfpathlineto{\pgfqpoint{4.713099in}{1.881225in}}%
\pgfpathlineto{\pgfqpoint{4.704802in}{1.859487in}}%
\pgfpathclose%
\pgfusepath{fill}%
\end{pgfscope}%
\begin{pgfscope}%
\pgfpathrectangle{\pgfqpoint{1.150000in}{0.150000in}}{\pgfqpoint{5.700000in}{5.700000in}}%
\pgfusepath{clip}%
\pgfsetbuttcap%
\pgfsetroundjoin%
\definecolor{currentfill}{rgb}{0.386433,0.794644,0.372886}%
\pgfsetfillcolor{currentfill}%
\pgfsetfillopacity{0.800000}%
\pgfsetlinewidth{0.000000pt}%
\definecolor{currentstroke}{rgb}{0.000000,0.000000,0.000000}%
\pgfsetstrokecolor{currentstroke}%
\pgfsetdash{}{0pt}%
\pgfpathmoveto{\pgfqpoint{5.248439in}{2.951981in}}%
\pgfpathlineto{\pgfqpoint{5.263788in}{2.971965in}}%
\pgfpathlineto{\pgfqpoint{5.279161in}{2.992145in}}%
\pgfpathlineto{\pgfqpoint{5.294560in}{3.012523in}}%
\pgfpathlineto{\pgfqpoint{5.302694in}{3.029794in}}%
\pgfpathlineto{\pgfqpoint{5.310819in}{3.046834in}}%
\pgfpathlineto{\pgfqpoint{5.318935in}{3.063640in}}%
\pgfpathlineto{\pgfqpoint{5.327042in}{3.080210in}}%
\pgfpathlineto{\pgfqpoint{5.311629in}{3.059588in}}%
\pgfpathlineto{\pgfqpoint{5.296242in}{3.039164in}}%
\pgfpathlineto{\pgfqpoint{5.280880in}{3.018936in}}%
\pgfpathlineto{\pgfqpoint{5.272783in}{3.002538in}}%
\pgfpathlineto{\pgfqpoint{5.264677in}{2.985912in}}%
\pgfpathlineto{\pgfqpoint{5.256562in}{2.969059in}}%
\pgfpathlineto{\pgfqpoint{5.248439in}{2.951981in}}%
\pgfpathclose%
\pgfusepath{fill}%
\end{pgfscope}%
\begin{pgfscope}%
\pgfpathrectangle{\pgfqpoint{1.150000in}{0.150000in}}{\pgfqpoint{5.700000in}{5.700000in}}%
\pgfusepath{clip}%
\pgfsetbuttcap%
\pgfsetroundjoin%
\definecolor{currentfill}{rgb}{0.220057,0.343307,0.549413}%
\pgfsetfillcolor{currentfill}%
\pgfsetfillopacity{0.800000}%
\pgfsetlinewidth{0.000000pt}%
\definecolor{currentstroke}{rgb}{0.000000,0.000000,0.000000}%
\pgfsetstrokecolor{currentstroke}%
\pgfsetdash{}{0pt}%
\pgfpathmoveto{\pgfqpoint{4.512634in}{1.475110in}}%
\pgfpathlineto{\pgfqpoint{4.527389in}{1.484340in}}%
\pgfpathlineto{\pgfqpoint{4.542161in}{1.493747in}}%
\pgfpathlineto{\pgfqpoint{4.556949in}{1.503330in}}%
\pgfpathlineto{\pgfqpoint{4.571754in}{1.513090in}}%
\pgfpathlineto{\pgfqpoint{4.580084in}{1.534408in}}%
\pgfpathlineto{\pgfqpoint{4.588412in}{1.555801in}}%
\pgfpathlineto{\pgfqpoint{4.596738in}{1.577263in}}%
\pgfpathlineto{\pgfqpoint{4.605062in}{1.598787in}}%
\pgfpathlineto{\pgfqpoint{4.590242in}{1.588285in}}%
\pgfpathlineto{\pgfqpoint{4.575439in}{1.577961in}}%
\pgfpathlineto{\pgfqpoint{4.560654in}{1.567814in}}%
\pgfpathlineto{\pgfqpoint{4.545885in}{1.557845in}}%
\pgfpathlineto{\pgfqpoint{4.537575in}{1.537051in}}%
\pgfpathlineto{\pgfqpoint{4.529263in}{1.516325in}}%
\pgfpathlineto{\pgfqpoint{4.520950in}{1.495676in}}%
\pgfpathlineto{\pgfqpoint{4.512634in}{1.475110in}}%
\pgfpathclose%
\pgfusepath{fill}%
\end{pgfscope}%
\begin{pgfscope}%
\pgfpathrectangle{\pgfqpoint{1.150000in}{0.150000in}}{\pgfqpoint{5.700000in}{5.700000in}}%
\pgfusepath{clip}%
\pgfsetbuttcap%
\pgfsetroundjoin%
\definecolor{currentfill}{rgb}{0.208030,0.718701,0.472873}%
\pgfsetfillcolor{currentfill}%
\pgfsetfillopacity{0.800000}%
\pgfsetlinewidth{0.000000pt}%
\definecolor{currentstroke}{rgb}{0.000000,0.000000,0.000000}%
\pgfsetstrokecolor{currentstroke}%
\pgfsetdash{}{0pt}%
\pgfpathmoveto{\pgfqpoint{5.089386in}{2.656172in}}%
\pgfpathlineto{\pgfqpoint{5.104591in}{2.674430in}}%
\pgfpathlineto{\pgfqpoint{5.119820in}{2.692880in}}%
\pgfpathlineto{\pgfqpoint{5.135072in}{2.711523in}}%
\pgfpathlineto{\pgfqpoint{5.150348in}{2.730360in}}%
\pgfpathlineto{\pgfqpoint{5.158562in}{2.749947in}}%
\pgfpathlineto{\pgfqpoint{5.166770in}{2.769341in}}%
\pgfpathlineto{\pgfqpoint{5.174970in}{2.788537in}}%
\pgfpathlineto{\pgfqpoint{5.183164in}{2.807533in}}%
\pgfpathlineto{\pgfqpoint{5.167870in}{2.788343in}}%
\pgfpathlineto{\pgfqpoint{5.152601in}{2.769346in}}%
\pgfpathlineto{\pgfqpoint{5.137356in}{2.750544in}}%
\pgfpathlineto{\pgfqpoint{5.122134in}{2.731934in}}%
\pgfpathlineto{\pgfqpoint{5.113957in}{2.713277in}}%
\pgfpathlineto{\pgfqpoint{5.105773in}{2.694429in}}%
\pgfpathlineto{\pgfqpoint{5.097583in}{2.675393in}}%
\pgfpathlineto{\pgfqpoint{5.089386in}{2.656172in}}%
\pgfpathclose%
\pgfusepath{fill}%
\end{pgfscope}%
\begin{pgfscope}%
\pgfpathrectangle{\pgfqpoint{1.150000in}{0.150000in}}{\pgfqpoint{5.700000in}{5.700000in}}%
\pgfusepath{clip}%
\pgfsetbuttcap%
\pgfsetroundjoin%
\definecolor{currentfill}{rgb}{0.120092,0.600104,0.542530}%
\pgfsetfillcolor{currentfill}%
\pgfsetfillopacity{0.800000}%
\pgfsetlinewidth{0.000000pt}%
\definecolor{currentstroke}{rgb}{0.000000,0.000000,0.000000}%
\pgfsetstrokecolor{currentstroke}%
\pgfsetdash{}{0pt}%
\pgfpathmoveto{\pgfqpoint{4.897161in}{2.262249in}}%
\pgfpathlineto{\pgfqpoint{4.912204in}{2.277992in}}%
\pgfpathlineto{\pgfqpoint{4.927268in}{2.293923in}}%
\pgfpathlineto{\pgfqpoint{4.942355in}{2.310042in}}%
\pgfpathlineto{\pgfqpoint{4.957463in}{2.326349in}}%
\pgfpathlineto{\pgfqpoint{4.965745in}{2.348038in}}%
\pgfpathlineto{\pgfqpoint{4.974023in}{2.369604in}}%
\pgfpathlineto{\pgfqpoint{4.982297in}{2.391041in}}%
\pgfpathlineto{\pgfqpoint{4.990566in}{2.412346in}}%
\pgfpathlineto{\pgfqpoint{4.975438in}{2.395547in}}%
\pgfpathlineto{\pgfqpoint{4.960332in}{2.378937in}}%
\pgfpathlineto{\pgfqpoint{4.945249in}{2.362516in}}%
\pgfpathlineto{\pgfqpoint{4.930187in}{2.346284in}}%
\pgfpathlineto{\pgfqpoint{4.921937in}{2.325457in}}%
\pgfpathlineto{\pgfqpoint{4.913682in}{2.304506in}}%
\pgfpathlineto{\pgfqpoint{4.905424in}{2.283435in}}%
\pgfpathlineto{\pgfqpoint{4.897161in}{2.262249in}}%
\pgfpathclose%
\pgfusepath{fill}%
\end{pgfscope}%
\begin{pgfscope}%
\pgfpathrectangle{\pgfqpoint{1.150000in}{0.150000in}}{\pgfqpoint{5.700000in}{5.700000in}}%
\pgfusepath{clip}%
\pgfsetbuttcap%
\pgfsetroundjoin%
\definecolor{currentfill}{rgb}{0.262138,0.242286,0.520837}%
\pgfsetfillcolor{currentfill}%
\pgfsetfillopacity{0.800000}%
\pgfsetlinewidth{0.000000pt}%
\definecolor{currentstroke}{rgb}{0.000000,0.000000,0.000000}%
\pgfsetstrokecolor{currentstroke}%
\pgfsetdash{}{0pt}%
\pgfpathmoveto{\pgfqpoint{4.353956in}{1.211794in}}%
\pgfpathlineto{\pgfqpoint{4.368620in}{1.217937in}}%
\pgfpathlineto{\pgfqpoint{4.383298in}{1.224251in}}%
\pgfpathlineto{\pgfqpoint{4.397991in}{1.230739in}}%
\pgfpathlineto{\pgfqpoint{4.412697in}{1.237399in}}%
\pgfpathlineto{\pgfqpoint{4.421036in}{1.256404in}}%
\pgfpathlineto{\pgfqpoint{4.429373in}{1.275580in}}%
\pgfpathlineto{\pgfqpoint{4.437708in}{1.294921in}}%
\pgfpathlineto{\pgfqpoint{4.446041in}{1.314418in}}%
\pgfpathlineto{\pgfqpoint{4.431325in}{1.306930in}}%
\pgfpathlineto{\pgfqpoint{4.416624in}{1.299615in}}%
\pgfpathlineto{\pgfqpoint{4.401937in}{1.292474in}}%
\pgfpathlineto{\pgfqpoint{4.387266in}{1.285507in}}%
\pgfpathlineto{\pgfqpoint{4.378942in}{1.266825in}}%
\pgfpathlineto{\pgfqpoint{4.370616in}{1.248308in}}%
\pgfpathlineto{\pgfqpoint{4.362287in}{1.229961in}}%
\pgfpathlineto{\pgfqpoint{4.353956in}{1.211794in}}%
\pgfpathclose%
\pgfusepath{fill}%
\end{pgfscope}%
\begin{pgfscope}%
\pgfpathrectangle{\pgfqpoint{1.150000in}{0.150000in}}{\pgfqpoint{5.700000in}{5.700000in}}%
\pgfusepath{clip}%
\pgfsetbuttcap%
\pgfsetroundjoin%
\definecolor{currentfill}{rgb}{0.172719,0.448791,0.557885}%
\pgfsetfillcolor{currentfill}%
\pgfsetfillopacity{0.800000}%
\pgfsetlinewidth{0.000000pt}%
\definecolor{currentstroke}{rgb}{0.000000,0.000000,0.000000}%
\pgfsetstrokecolor{currentstroke}%
\pgfsetdash{}{0pt}%
\pgfpathmoveto{\pgfqpoint{4.671590in}{1.772395in}}%
\pgfpathlineto{\pgfqpoint{4.686461in}{1.784469in}}%
\pgfpathlineto{\pgfqpoint{4.701350in}{1.796723in}}%
\pgfpathlineto{\pgfqpoint{4.716259in}{1.809158in}}%
\pgfpathlineto{\pgfqpoint{4.731186in}{1.821774in}}%
\pgfpathlineto{\pgfqpoint{4.739510in}{1.844220in}}%
\pgfpathlineto{\pgfqpoint{4.747833in}{1.866655in}}%
\pgfpathlineto{\pgfqpoint{4.756153in}{1.889075in}}%
\pgfpathlineto{\pgfqpoint{4.764471in}{1.911472in}}%
\pgfpathlineto{\pgfqpoint{4.749525in}{1.898202in}}%
\pgfpathlineto{\pgfqpoint{4.734598in}{1.885115in}}%
\pgfpathlineto{\pgfqpoint{4.719691in}{1.872210in}}%
\pgfpathlineto{\pgfqpoint{4.704802in}{1.859487in}}%
\pgfpathlineto{\pgfqpoint{4.696502in}{1.837730in}}%
\pgfpathlineto{\pgfqpoint{4.688200in}{1.815958in}}%
\pgfpathlineto{\pgfqpoint{4.679896in}{1.794178in}}%
\pgfpathlineto{\pgfqpoint{4.671590in}{1.772395in}}%
\pgfpathclose%
\pgfusepath{fill}%
\end{pgfscope}%
\begin{pgfscope}%
\pgfpathrectangle{\pgfqpoint{1.150000in}{0.150000in}}{\pgfqpoint{5.700000in}{5.700000in}}%
\pgfusepath{clip}%
\pgfsetbuttcap%
\pgfsetroundjoin%
\definecolor{currentfill}{rgb}{0.231674,0.318106,0.544834}%
\pgfsetfillcolor{currentfill}%
\pgfsetfillopacity{0.800000}%
\pgfsetlinewidth{0.000000pt}%
\definecolor{currentstroke}{rgb}{0.000000,0.000000,0.000000}%
\pgfsetstrokecolor{currentstroke}%
\pgfsetdash{}{0pt}%
\pgfpathmoveto{\pgfqpoint{4.479353in}{1.393814in}}%
\pgfpathlineto{\pgfqpoint{4.494095in}{1.402276in}}%
\pgfpathlineto{\pgfqpoint{4.508854in}{1.410912in}}%
\pgfpathlineto{\pgfqpoint{4.523628in}{1.419724in}}%
\pgfpathlineto{\pgfqpoint{4.538419in}{1.428711in}}%
\pgfpathlineto{\pgfqpoint{4.546755in}{1.449658in}}%
\pgfpathlineto{\pgfqpoint{4.555090in}{1.470708in}}%
\pgfpathlineto{\pgfqpoint{4.563423in}{1.491854in}}%
\pgfpathlineto{\pgfqpoint{4.571754in}{1.513090in}}%
\pgfpathlineto{\pgfqpoint{4.556949in}{1.503330in}}%
\pgfpathlineto{\pgfqpoint{4.542161in}{1.493747in}}%
\pgfpathlineto{\pgfqpoint{4.527389in}{1.484340in}}%
\pgfpathlineto{\pgfqpoint{4.512634in}{1.475110in}}%
\pgfpathlineto{\pgfqpoint{4.504317in}{1.454633in}}%
\pgfpathlineto{\pgfqpoint{4.495997in}{1.434254in}}%
\pgfpathlineto{\pgfqpoint{4.487676in}{1.413979in}}%
\pgfpathlineto{\pgfqpoint{4.479353in}{1.393814in}}%
\pgfpathclose%
\pgfusepath{fill}%
\end{pgfscope}%
\begin{pgfscope}%
\pgfpathrectangle{\pgfqpoint{1.150000in}{0.150000in}}{\pgfqpoint{5.700000in}{5.700000in}}%
\pgfusepath{clip}%
\pgfsetbuttcap%
\pgfsetroundjoin%
\definecolor{currentfill}{rgb}{0.124395,0.578002,0.548287}%
\pgfsetfillcolor{currentfill}%
\pgfsetfillopacity{0.800000}%
\pgfsetlinewidth{0.000000pt}%
\definecolor{currentstroke}{rgb}{0.000000,0.000000,0.000000}%
\pgfsetstrokecolor{currentstroke}%
\pgfsetdash{}{0pt}%
\pgfpathmoveto{\pgfqpoint{4.864069in}{2.176439in}}%
\pgfpathlineto{\pgfqpoint{4.879093in}{2.191660in}}%
\pgfpathlineto{\pgfqpoint{4.894138in}{2.207067in}}%
\pgfpathlineto{\pgfqpoint{4.909204in}{2.222660in}}%
\pgfpathlineto{\pgfqpoint{4.924292in}{2.238441in}}%
\pgfpathlineto{\pgfqpoint{4.932591in}{2.260581in}}%
\pgfpathlineto{\pgfqpoint{4.940885in}{2.282615in}}%
\pgfpathlineto{\pgfqpoint{4.949176in}{2.304539in}}%
\pgfpathlineto{\pgfqpoint{4.957463in}{2.326349in}}%
\pgfpathlineto{\pgfqpoint{4.942355in}{2.310042in}}%
\pgfpathlineto{\pgfqpoint{4.927268in}{2.293923in}}%
\pgfpathlineto{\pgfqpoint{4.912204in}{2.277992in}}%
\pgfpathlineto{\pgfqpoint{4.897161in}{2.262249in}}%
\pgfpathlineto{\pgfqpoint{4.888893in}{2.240951in}}%
\pgfpathlineto{\pgfqpoint{4.880622in}{2.219548in}}%
\pgfpathlineto{\pgfqpoint{4.872347in}{2.198042in}}%
\pgfpathlineto{\pgfqpoint{4.864069in}{2.176439in}}%
\pgfpathclose%
\pgfusepath{fill}%
\end{pgfscope}%
\begin{pgfscope}%
\pgfpathrectangle{\pgfqpoint{1.150000in}{0.150000in}}{\pgfqpoint{5.700000in}{5.700000in}}%
\pgfusepath{clip}%
\pgfsetbuttcap%
\pgfsetroundjoin%
\definecolor{currentfill}{rgb}{0.180653,0.701402,0.488189}%
\pgfsetfillcolor{currentfill}%
\pgfsetfillopacity{0.800000}%
\pgfsetlinewidth{0.000000pt}%
\definecolor{currentstroke}{rgb}{0.000000,0.000000,0.000000}%
\pgfsetstrokecolor{currentstroke}%
\pgfsetdash{}{0pt}%
\pgfpathmoveto{\pgfqpoint{5.056536in}{2.577506in}}%
\pgfpathlineto{\pgfqpoint{5.071723in}{2.595376in}}%
\pgfpathlineto{\pgfqpoint{5.086934in}{2.613438in}}%
\pgfpathlineto{\pgfqpoint{5.102169in}{2.631692in}}%
\pgfpathlineto{\pgfqpoint{5.117426in}{2.650139in}}%
\pgfpathlineto{\pgfqpoint{5.125666in}{2.670469in}}%
\pgfpathlineto{\pgfqpoint{5.133900in}{2.690618in}}%
\pgfpathlineto{\pgfqpoint{5.142127in}{2.710583in}}%
\pgfpathlineto{\pgfqpoint{5.150348in}{2.730360in}}%
\pgfpathlineto{\pgfqpoint{5.135072in}{2.711523in}}%
\pgfpathlineto{\pgfqpoint{5.119820in}{2.692880in}}%
\pgfpathlineto{\pgfqpoint{5.104591in}{2.674430in}}%
\pgfpathlineto{\pgfqpoint{5.089386in}{2.656172in}}%
\pgfpathlineto{\pgfqpoint{5.081182in}{2.636770in}}%
\pgfpathlineto{\pgfqpoint{5.072973in}{2.617189in}}%
\pgfpathlineto{\pgfqpoint{5.064757in}{2.597433in}}%
\pgfpathlineto{\pgfqpoint{5.056536in}{2.577506in}}%
\pgfpathclose%
\pgfusepath{fill}%
\end{pgfscope}%
\begin{pgfscope}%
\pgfpathrectangle{\pgfqpoint{1.150000in}{0.150000in}}{\pgfqpoint{5.700000in}{5.700000in}}%
\pgfusepath{clip}%
\pgfsetbuttcap%
\pgfsetroundjoin%
\definecolor{currentfill}{rgb}{0.352360,0.783011,0.392636}%
\pgfsetfillcolor{currentfill}%
\pgfsetfillopacity{0.800000}%
\pgfsetlinewidth{0.000000pt}%
\definecolor{currentstroke}{rgb}{0.000000,0.000000,0.000000}%
\pgfsetstrokecolor{currentstroke}%
\pgfsetdash{}{0pt}%
\pgfpathmoveto{\pgfqpoint{5.215864in}{2.881464in}}%
\pgfpathlineto{\pgfqpoint{5.231198in}{2.901167in}}%
\pgfpathlineto{\pgfqpoint{5.246556in}{2.921066in}}%
\pgfpathlineto{\pgfqpoint{5.261940in}{2.941162in}}%
\pgfpathlineto{\pgfqpoint{5.270107in}{2.959339in}}%
\pgfpathlineto{\pgfqpoint{5.278267in}{2.977293in}}%
\pgfpathlineto{\pgfqpoint{5.286418in}{2.995022in}}%
\pgfpathlineto{\pgfqpoint{5.294560in}{3.012523in}}%
\pgfpathlineto{\pgfqpoint{5.279161in}{2.992145in}}%
\pgfpathlineto{\pgfqpoint{5.263788in}{2.971965in}}%
\pgfpathlineto{\pgfqpoint{5.248439in}{2.951981in}}%
\pgfpathlineto{\pgfqpoint{5.240307in}{2.934680in}}%
\pgfpathlineto{\pgfqpoint{5.232167in}{2.917159in}}%
\pgfpathlineto{\pgfqpoint{5.224019in}{2.899419in}}%
\pgfpathlineto{\pgfqpoint{5.215864in}{2.881464in}}%
\pgfpathclose%
\pgfusepath{fill}%
\end{pgfscope}%
\begin{pgfscope}%
\pgfpathrectangle{\pgfqpoint{1.150000in}{0.150000in}}{\pgfqpoint{5.700000in}{5.700000in}}%
\pgfusepath{clip}%
\pgfsetbuttcap%
\pgfsetroundjoin%
\definecolor{currentfill}{rgb}{0.183898,0.422383,0.556944}%
\pgfsetfillcolor{currentfill}%
\pgfsetfillopacity{0.800000}%
\pgfsetlinewidth{0.000000pt}%
\definecolor{currentstroke}{rgb}{0.000000,0.000000,0.000000}%
\pgfsetstrokecolor{currentstroke}%
\pgfsetdash{}{0pt}%
\pgfpathmoveto{\pgfqpoint{4.638342in}{1.685362in}}%
\pgfpathlineto{\pgfqpoint{4.653196in}{1.696754in}}%
\pgfpathlineto{\pgfqpoint{4.668068in}{1.708325in}}%
\pgfpathlineto{\pgfqpoint{4.682958in}{1.720077in}}%
\pgfpathlineto{\pgfqpoint{4.697866in}{1.732008in}}%
\pgfpathlineto{\pgfqpoint{4.706199in}{1.754434in}}%
\pgfpathlineto{\pgfqpoint{4.714530in}{1.776875in}}%
\pgfpathlineto{\pgfqpoint{4.722859in}{1.799323in}}%
\pgfpathlineto{\pgfqpoint{4.731186in}{1.821774in}}%
\pgfpathlineto{\pgfqpoint{4.716259in}{1.809158in}}%
\pgfpathlineto{\pgfqpoint{4.701350in}{1.796723in}}%
\pgfpathlineto{\pgfqpoint{4.686461in}{1.784469in}}%
\pgfpathlineto{\pgfqpoint{4.671590in}{1.772395in}}%
\pgfpathlineto{\pgfqpoint{4.663281in}{1.750616in}}%
\pgfpathlineto{\pgfqpoint{4.654970in}{1.728847in}}%
\pgfpathlineto{\pgfqpoint{4.646657in}{1.707093in}}%
\pgfpathlineto{\pgfqpoint{4.638342in}{1.685362in}}%
\pgfpathclose%
\pgfusepath{fill}%
\end{pgfscope}%
\begin{pgfscope}%
\pgfpathrectangle{\pgfqpoint{1.150000in}{0.150000in}}{\pgfqpoint{5.700000in}{5.700000in}}%
\pgfusepath{clip}%
\pgfsetbuttcap%
\pgfsetroundjoin%
\definecolor{currentfill}{rgb}{0.132444,0.552216,0.553018}%
\pgfsetfillcolor{currentfill}%
\pgfsetfillopacity{0.800000}%
\pgfsetlinewidth{0.000000pt}%
\definecolor{currentstroke}{rgb}{0.000000,0.000000,0.000000}%
\pgfsetstrokecolor{currentstroke}%
\pgfsetdash{}{0pt}%
\pgfpathmoveto{\pgfqpoint{4.830919in}{2.089156in}}%
\pgfpathlineto{\pgfqpoint{4.845923in}{2.103820in}}%
\pgfpathlineto{\pgfqpoint{4.860949in}{2.118669in}}%
\pgfpathlineto{\pgfqpoint{4.875995in}{2.133704in}}%
\pgfpathlineto{\pgfqpoint{4.891062in}{2.148924in}}%
\pgfpathlineto{\pgfqpoint{4.899375in}{2.171437in}}%
\pgfpathlineto{\pgfqpoint{4.907684in}{2.193864in}}%
\pgfpathlineto{\pgfqpoint{4.915990in}{2.216200in}}%
\pgfpathlineto{\pgfqpoint{4.924292in}{2.238441in}}%
\pgfpathlineto{\pgfqpoint{4.909204in}{2.222660in}}%
\pgfpathlineto{\pgfqpoint{4.894138in}{2.207067in}}%
\pgfpathlineto{\pgfqpoint{4.879093in}{2.191660in}}%
\pgfpathlineto{\pgfqpoint{4.864069in}{2.176439in}}%
\pgfpathlineto{\pgfqpoint{4.855786in}{2.154745in}}%
\pgfpathlineto{\pgfqpoint{4.847501in}{2.132962in}}%
\pgfpathlineto{\pgfqpoint{4.839211in}{2.111098in}}%
\pgfpathlineto{\pgfqpoint{4.830919in}{2.089156in}}%
\pgfpathclose%
\pgfusepath{fill}%
\end{pgfscope}%
\begin{pgfscope}%
\pgfpathrectangle{\pgfqpoint{1.150000in}{0.150000in}}{\pgfqpoint{5.700000in}{5.700000in}}%
\pgfusepath{clip}%
\pgfsetbuttcap%
\pgfsetroundjoin%
\definecolor{currentfill}{rgb}{0.243113,0.292092,0.538516}%
\pgfsetfillcolor{currentfill}%
\pgfsetfillopacity{0.800000}%
\pgfsetlinewidth{0.000000pt}%
\definecolor{currentstroke}{rgb}{0.000000,0.000000,0.000000}%
\pgfsetstrokecolor{currentstroke}%
\pgfsetdash{}{0pt}%
\pgfpathmoveto{\pgfqpoint{4.446041in}{1.314418in}}%
\pgfpathlineto{\pgfqpoint{4.460772in}{1.322080in}}%
\pgfpathlineto{\pgfqpoint{4.475518in}{1.329916in}}%
\pgfpathlineto{\pgfqpoint{4.490280in}{1.337926in}}%
\pgfpathlineto{\pgfqpoint{4.505058in}{1.346110in}}%
\pgfpathlineto{\pgfqpoint{4.513400in}{1.366568in}}%
\pgfpathlineto{\pgfqpoint{4.521741in}{1.387159in}}%
\pgfpathlineto{\pgfqpoint{4.530081in}{1.407876in}}%
\pgfpathlineto{\pgfqpoint{4.538419in}{1.428711in}}%
\pgfpathlineto{\pgfqpoint{4.523628in}{1.419724in}}%
\pgfpathlineto{\pgfqpoint{4.508854in}{1.410912in}}%
\pgfpathlineto{\pgfqpoint{4.494095in}{1.402276in}}%
\pgfpathlineto{\pgfqpoint{4.479353in}{1.393814in}}%
\pgfpathlineto{\pgfqpoint{4.471028in}{1.373769in}}%
\pgfpathlineto{\pgfqpoint{4.462701in}{1.353849in}}%
\pgfpathlineto{\pgfqpoint{4.454372in}{1.334063in}}%
\pgfpathlineto{\pgfqpoint{4.446041in}{1.314418in}}%
\pgfpathclose%
\pgfusepath{fill}%
\end{pgfscope}%
\begin{pgfscope}%
\pgfpathrectangle{\pgfqpoint{1.150000in}{0.150000in}}{\pgfqpoint{5.700000in}{5.700000in}}%
\pgfusepath{clip}%
\pgfsetbuttcap%
\pgfsetroundjoin%
\definecolor{currentfill}{rgb}{0.153894,0.680203,0.504172}%
\pgfsetfillcolor{currentfill}%
\pgfsetfillopacity{0.800000}%
\pgfsetlinewidth{0.000000pt}%
\definecolor{currentstroke}{rgb}{0.000000,0.000000,0.000000}%
\pgfsetstrokecolor{currentstroke}%
\pgfsetdash{}{0pt}%
\pgfpathmoveto{\pgfqpoint{5.023593in}{2.496151in}}%
\pgfpathlineto{\pgfqpoint{5.038762in}{2.513598in}}%
\pgfpathlineto{\pgfqpoint{5.053954in}{2.531236in}}%
\pgfpathlineto{\pgfqpoint{5.069169in}{2.549065in}}%
\pgfpathlineto{\pgfqpoint{5.084407in}{2.567086in}}%
\pgfpathlineto{\pgfqpoint{5.092671in}{2.588102in}}%
\pgfpathlineto{\pgfqpoint{5.100928in}{2.608952in}}%
\pgfpathlineto{\pgfqpoint{5.109180in}{2.629632in}}%
\pgfpathlineto{\pgfqpoint{5.117426in}{2.650139in}}%
\pgfpathlineto{\pgfqpoint{5.102169in}{2.631692in}}%
\pgfpathlineto{\pgfqpoint{5.086934in}{2.613438in}}%
\pgfpathlineto{\pgfqpoint{5.071723in}{2.595376in}}%
\pgfpathlineto{\pgfqpoint{5.056536in}{2.577506in}}%
\pgfpathlineto{\pgfqpoint{5.048308in}{2.557410in}}%
\pgfpathlineto{\pgfqpoint{5.040075in}{2.537150in}}%
\pgfpathlineto{\pgfqpoint{5.031837in}{2.516729in}}%
\pgfpathlineto{\pgfqpoint{5.023593in}{2.496151in}}%
\pgfpathclose%
\pgfusepath{fill}%
\end{pgfscope}%
\begin{pgfscope}%
\pgfpathrectangle{\pgfqpoint{1.150000in}{0.150000in}}{\pgfqpoint{5.700000in}{5.700000in}}%
\pgfusepath{clip}%
\pgfsetbuttcap%
\pgfsetroundjoin%
\definecolor{currentfill}{rgb}{0.194100,0.399323,0.555565}%
\pgfsetfillcolor{currentfill}%
\pgfsetfillopacity{0.800000}%
\pgfsetlinewidth{0.000000pt}%
\definecolor{currentstroke}{rgb}{0.000000,0.000000,0.000000}%
\pgfsetstrokecolor{currentstroke}%
\pgfsetdash{}{0pt}%
\pgfpathmoveto{\pgfqpoint{4.605062in}{1.598787in}}%
\pgfpathlineto{\pgfqpoint{4.619900in}{1.609466in}}%
\pgfpathlineto{\pgfqpoint{4.634755in}{1.620324in}}%
\pgfpathlineto{\pgfqpoint{4.649627in}{1.631361in}}%
\pgfpathlineto{\pgfqpoint{4.664518in}{1.642575in}}%
\pgfpathlineto{\pgfqpoint{4.672857in}{1.664879in}}%
\pgfpathlineto{\pgfqpoint{4.681196in}{1.687224in}}%
\pgfpathlineto{\pgfqpoint{4.689532in}{1.709602in}}%
\pgfpathlineto{\pgfqpoint{4.697866in}{1.732008in}}%
\pgfpathlineto{\pgfqpoint{4.682958in}{1.720077in}}%
\pgfpathlineto{\pgfqpoint{4.668068in}{1.708325in}}%
\pgfpathlineto{\pgfqpoint{4.653196in}{1.696754in}}%
\pgfpathlineto{\pgfqpoint{4.638342in}{1.685362in}}%
\pgfpathlineto{\pgfqpoint{4.630025in}{1.663659in}}%
\pgfpathlineto{\pgfqpoint{4.621706in}{1.641991in}}%
\pgfpathlineto{\pgfqpoint{4.613385in}{1.620365in}}%
\pgfpathlineto{\pgfqpoint{4.605062in}{1.598787in}}%
\pgfpathclose%
\pgfusepath{fill}%
\end{pgfscope}%
\begin{pgfscope}%
\pgfpathrectangle{\pgfqpoint{1.150000in}{0.150000in}}{\pgfqpoint{5.700000in}{5.700000in}}%
\pgfusepath{clip}%
\pgfsetbuttcap%
\pgfsetroundjoin%
\definecolor{currentfill}{rgb}{0.311925,0.767822,0.415586}%
\pgfsetfillcolor{currentfill}%
\pgfsetfillopacity{0.800000}%
\pgfsetlinewidth{0.000000pt}%
\definecolor{currentstroke}{rgb}{0.000000,0.000000,0.000000}%
\pgfsetstrokecolor{currentstroke}%
\pgfsetdash{}{0pt}%
\pgfpathmoveto{\pgfqpoint{5.183164in}{2.807533in}}%
\pgfpathlineto{\pgfqpoint{5.198481in}{2.826919in}}%
\pgfpathlineto{\pgfqpoint{5.213824in}{2.846499in}}%
\pgfpathlineto{\pgfqpoint{5.229191in}{2.866276in}}%
\pgfpathlineto{\pgfqpoint{5.237390in}{2.885319in}}%
\pgfpathlineto{\pgfqpoint{5.245581in}{2.904149in}}%
\pgfpathlineto{\pgfqpoint{5.253764in}{2.922765in}}%
\pgfpathlineto{\pgfqpoint{5.261940in}{2.941162in}}%
\pgfpathlineto{\pgfqpoint{5.246556in}{2.921066in}}%
\pgfpathlineto{\pgfqpoint{5.231198in}{2.901167in}}%
\pgfpathlineto{\pgfqpoint{5.215864in}{2.881464in}}%
\pgfpathlineto{\pgfqpoint{5.207700in}{2.863295in}}%
\pgfpathlineto{\pgfqpoint{5.199529in}{2.844915in}}%
\pgfpathlineto{\pgfqpoint{5.191350in}{2.826327in}}%
\pgfpathlineto{\pgfqpoint{5.183164in}{2.807533in}}%
\pgfpathclose%
\pgfusepath{fill}%
\end{pgfscope}%
\begin{pgfscope}%
\pgfpathrectangle{\pgfqpoint{1.150000in}{0.150000in}}{\pgfqpoint{5.700000in}{5.700000in}}%
\pgfusepath{clip}%
\pgfsetbuttcap%
\pgfsetroundjoin%
\definecolor{currentfill}{rgb}{0.141935,0.526453,0.555991}%
\pgfsetfillcolor{currentfill}%
\pgfsetfillopacity{0.800000}%
\pgfsetlinewidth{0.000000pt}%
\definecolor{currentstroke}{rgb}{0.000000,0.000000,0.000000}%
\pgfsetstrokecolor{currentstroke}%
\pgfsetdash{}{0pt}%
\pgfpathmoveto{\pgfqpoint{4.797717in}{2.000720in}}%
\pgfpathlineto{\pgfqpoint{4.812703in}{2.014794in}}%
\pgfpathlineto{\pgfqpoint{4.827708in}{2.029052in}}%
\pgfpathlineto{\pgfqpoint{4.842733in}{2.043494in}}%
\pgfpathlineto{\pgfqpoint{4.857780in}{2.058122in}}%
\pgfpathlineto{\pgfqpoint{4.866105in}{2.080924in}}%
\pgfpathlineto{\pgfqpoint{4.874427in}{2.103663in}}%
\pgfpathlineto{\pgfqpoint{4.882746in}{2.126331in}}%
\pgfpathlineto{\pgfqpoint{4.891062in}{2.148924in}}%
\pgfpathlineto{\pgfqpoint{4.875995in}{2.133704in}}%
\pgfpathlineto{\pgfqpoint{4.860949in}{2.118669in}}%
\pgfpathlineto{\pgfqpoint{4.845923in}{2.103820in}}%
\pgfpathlineto{\pgfqpoint{4.830919in}{2.089156in}}%
\pgfpathlineto{\pgfqpoint{4.822623in}{2.067142in}}%
\pgfpathlineto{\pgfqpoint{4.814324in}{2.045061in}}%
\pgfpathlineto{\pgfqpoint{4.806022in}{2.022919in}}%
\pgfpathlineto{\pgfqpoint{4.797717in}{2.000720in}}%
\pgfpathclose%
\pgfusepath{fill}%
\end{pgfscope}%
\begin{pgfscope}%
\pgfpathrectangle{\pgfqpoint{1.150000in}{0.150000in}}{\pgfqpoint{5.700000in}{5.700000in}}%
\pgfusepath{clip}%
\pgfsetbuttcap%
\pgfsetroundjoin%
\definecolor{currentfill}{rgb}{0.252194,0.269783,0.531579}%
\pgfsetfillcolor{currentfill}%
\pgfsetfillopacity{0.800000}%
\pgfsetlinewidth{0.000000pt}%
\definecolor{currentstroke}{rgb}{0.000000,0.000000,0.000000}%
\pgfsetstrokecolor{currentstroke}%
\pgfsetdash{}{0pt}%
\pgfpathmoveto{\pgfqpoint{4.412697in}{1.237399in}}%
\pgfpathlineto{\pgfqpoint{4.427418in}{1.244231in}}%
\pgfpathlineto{\pgfqpoint{4.442154in}{1.251236in}}%
\pgfpathlineto{\pgfqpoint{4.456905in}{1.258415in}}%
\pgfpathlineto{\pgfqpoint{4.471671in}{1.265766in}}%
\pgfpathlineto{\pgfqpoint{4.480020in}{1.285613in}}%
\pgfpathlineto{\pgfqpoint{4.488367in}{1.305625in}}%
\pgfpathlineto{\pgfqpoint{4.496713in}{1.325793in}}%
\pgfpathlineto{\pgfqpoint{4.505058in}{1.346110in}}%
\pgfpathlineto{\pgfqpoint{4.490280in}{1.337926in}}%
\pgfpathlineto{\pgfqpoint{4.475518in}{1.329916in}}%
\pgfpathlineto{\pgfqpoint{4.460772in}{1.322080in}}%
\pgfpathlineto{\pgfqpoint{4.446041in}{1.314418in}}%
\pgfpathlineto{\pgfqpoint{4.437708in}{1.294921in}}%
\pgfpathlineto{\pgfqpoint{4.429373in}{1.275580in}}%
\pgfpathlineto{\pgfqpoint{4.421036in}{1.256404in}}%
\pgfpathlineto{\pgfqpoint{4.412697in}{1.237399in}}%
\pgfpathclose%
\pgfusepath{fill}%
\end{pgfscope}%
\begin{pgfscope}%
\pgfpathrectangle{\pgfqpoint{1.150000in}{0.150000in}}{\pgfqpoint{5.700000in}{5.700000in}}%
\pgfusepath{clip}%
\pgfsetbuttcap%
\pgfsetroundjoin%
\definecolor{currentfill}{rgb}{0.134692,0.658636,0.517649}%
\pgfsetfillcolor{currentfill}%
\pgfsetfillopacity{0.800000}%
\pgfsetlinewidth{0.000000pt}%
\definecolor{currentstroke}{rgb}{0.000000,0.000000,0.000000}%
\pgfsetstrokecolor{currentstroke}%
\pgfsetdash{}{0pt}%
\pgfpathmoveto{\pgfqpoint{4.990566in}{2.412346in}}%
\pgfpathlineto{\pgfqpoint{5.005715in}{2.429334in}}%
\pgfpathlineto{\pgfqpoint{5.020888in}{2.446513in}}%
\pgfpathlineto{\pgfqpoint{5.036083in}{2.463881in}}%
\pgfpathlineto{\pgfqpoint{5.051301in}{2.481441in}}%
\pgfpathlineto{\pgfqpoint{5.059585in}{2.503081in}}%
\pgfpathlineto{\pgfqpoint{5.067864in}{2.524571in}}%
\pgfpathlineto{\pgfqpoint{5.076139in}{2.545908in}}%
\pgfpathlineto{\pgfqpoint{5.084407in}{2.567086in}}%
\pgfpathlineto{\pgfqpoint{5.069169in}{2.549065in}}%
\pgfpathlineto{\pgfqpoint{5.053954in}{2.531236in}}%
\pgfpathlineto{\pgfqpoint{5.038762in}{2.513598in}}%
\pgfpathlineto{\pgfqpoint{5.023593in}{2.496151in}}%
\pgfpathlineto{\pgfqpoint{5.015343in}{2.475420in}}%
\pgfpathlineto{\pgfqpoint{5.007089in}{2.454539in}}%
\pgfpathlineto{\pgfqpoint{4.998830in}{2.433513in}}%
\pgfpathlineto{\pgfqpoint{4.990566in}{2.412346in}}%
\pgfpathclose%
\pgfusepath{fill}%
\end{pgfscope}%
\begin{pgfscope}%
\pgfpathrectangle{\pgfqpoint{1.150000in}{0.150000in}}{\pgfqpoint{5.700000in}{5.700000in}}%
\pgfusepath{clip}%
\pgfsetbuttcap%
\pgfsetroundjoin%
\definecolor{currentfill}{rgb}{0.206756,0.371758,0.553117}%
\pgfsetfillcolor{currentfill}%
\pgfsetfillopacity{0.800000}%
\pgfsetlinewidth{0.000000pt}%
\definecolor{currentstroke}{rgb}{0.000000,0.000000,0.000000}%
\pgfsetstrokecolor{currentstroke}%
\pgfsetdash{}{0pt}%
\pgfpathmoveto{\pgfqpoint{4.571754in}{1.513090in}}%
\pgfpathlineto{\pgfqpoint{4.586576in}{1.523026in}}%
\pgfpathlineto{\pgfqpoint{4.601414in}{1.533140in}}%
\pgfpathlineto{\pgfqpoint{4.616270in}{1.543430in}}%
\pgfpathlineto{\pgfqpoint{4.631143in}{1.553898in}}%
\pgfpathlineto{\pgfqpoint{4.639489in}{1.575973in}}%
\pgfpathlineto{\pgfqpoint{4.647833in}{1.598115in}}%
\pgfpathlineto{\pgfqpoint{4.656176in}{1.620318in}}%
\pgfpathlineto{\pgfqpoint{4.664518in}{1.642575in}}%
\pgfpathlineto{\pgfqpoint{4.649627in}{1.631361in}}%
\pgfpathlineto{\pgfqpoint{4.634755in}{1.620324in}}%
\pgfpathlineto{\pgfqpoint{4.619900in}{1.609466in}}%
\pgfpathlineto{\pgfqpoint{4.605062in}{1.598787in}}%
\pgfpathlineto{\pgfqpoint{4.596738in}{1.577263in}}%
\pgfpathlineto{\pgfqpoint{4.588412in}{1.555801in}}%
\pgfpathlineto{\pgfqpoint{4.580084in}{1.534408in}}%
\pgfpathlineto{\pgfqpoint{4.571754in}{1.513090in}}%
\pgfpathclose%
\pgfusepath{fill}%
\end{pgfscope}%
\begin{pgfscope}%
\pgfpathrectangle{\pgfqpoint{1.150000in}{0.150000in}}{\pgfqpoint{5.700000in}{5.700000in}}%
\pgfusepath{clip}%
\pgfsetbuttcap%
\pgfsetroundjoin%
\definecolor{currentfill}{rgb}{0.150476,0.504369,0.557430}%
\pgfsetfillcolor{currentfill}%
\pgfsetfillopacity{0.800000}%
\pgfsetlinewidth{0.000000pt}%
\definecolor{currentstroke}{rgb}{0.000000,0.000000,0.000000}%
\pgfsetstrokecolor{currentstroke}%
\pgfsetdash{}{0pt}%
\pgfpathmoveto{\pgfqpoint{4.764471in}{1.911472in}}%
\pgfpathlineto{\pgfqpoint{4.779437in}{1.924923in}}%
\pgfpathlineto{\pgfqpoint{4.794422in}{1.938558in}}%
\pgfpathlineto{\pgfqpoint{4.809427in}{1.952376in}}%
\pgfpathlineto{\pgfqpoint{4.824452in}{1.966377in}}%
\pgfpathlineto{\pgfqpoint{4.832788in}{1.989382in}}%
\pgfpathlineto{\pgfqpoint{4.841121in}{2.012345in}}%
\pgfpathlineto{\pgfqpoint{4.849452in}{2.035260in}}%
\pgfpathlineto{\pgfqpoint{4.857780in}{2.058122in}}%
\pgfpathlineto{\pgfqpoint{4.842733in}{2.043494in}}%
\pgfpathlineto{\pgfqpoint{4.827708in}{2.029052in}}%
\pgfpathlineto{\pgfqpoint{4.812703in}{2.014794in}}%
\pgfpathlineto{\pgfqpoint{4.797717in}{2.000720in}}%
\pgfpathlineto{\pgfqpoint{4.789410in}{1.978470in}}%
\pgfpathlineto{\pgfqpoint{4.781099in}{1.956175in}}%
\pgfpathlineto{\pgfqpoint{4.772787in}{1.933840in}}%
\pgfpathlineto{\pgfqpoint{4.764471in}{1.911472in}}%
\pgfpathclose%
\pgfusepath{fill}%
\end{pgfscope}%
\begin{pgfscope}%
\pgfpathrectangle{\pgfqpoint{1.150000in}{0.150000in}}{\pgfqpoint{5.700000in}{5.700000in}}%
\pgfusepath{clip}%
\pgfsetbuttcap%
\pgfsetroundjoin%
\definecolor{currentfill}{rgb}{0.266941,0.748751,0.440573}%
\pgfsetfillcolor{currentfill}%
\pgfsetfillopacity{0.800000}%
\pgfsetlinewidth{0.000000pt}%
\definecolor{currentstroke}{rgb}{0.000000,0.000000,0.000000}%
\pgfsetstrokecolor{currentstroke}%
\pgfsetdash{}{0pt}%
\pgfpathmoveto{\pgfqpoint{5.150348in}{2.730360in}}%
\pgfpathlineto{\pgfqpoint{5.165648in}{2.749391in}}%
\pgfpathlineto{\pgfqpoint{5.180973in}{2.768616in}}%
\pgfpathlineto{\pgfqpoint{5.196322in}{2.788036in}}%
\pgfpathlineto{\pgfqpoint{5.204550in}{2.807900in}}%
\pgfpathlineto{\pgfqpoint{5.212771in}{2.827563in}}%
\pgfpathlineto{\pgfqpoint{5.220984in}{2.847023in}}%
\pgfpathlineto{\pgfqpoint{5.229191in}{2.866276in}}%
\pgfpathlineto{\pgfqpoint{5.213824in}{2.846499in}}%
\pgfpathlineto{\pgfqpoint{5.198481in}{2.826919in}}%
\pgfpathlineto{\pgfqpoint{5.183164in}{2.807533in}}%
\pgfpathlineto{\pgfqpoint{5.174970in}{2.788537in}}%
\pgfpathlineto{\pgfqpoint{5.166770in}{2.769341in}}%
\pgfpathlineto{\pgfqpoint{5.158562in}{2.749947in}}%
\pgfpathlineto{\pgfqpoint{5.150348in}{2.730360in}}%
\pgfpathclose%
\pgfusepath{fill}%
\end{pgfscope}%
\begin{pgfscope}%
\pgfpathrectangle{\pgfqpoint{1.150000in}{0.150000in}}{\pgfqpoint{5.700000in}{5.700000in}}%
\pgfusepath{clip}%
\pgfsetbuttcap%
\pgfsetroundjoin%
\definecolor{currentfill}{rgb}{0.123444,0.636809,0.528763}%
\pgfsetfillcolor{currentfill}%
\pgfsetfillopacity{0.800000}%
\pgfsetlinewidth{0.000000pt}%
\definecolor{currentstroke}{rgb}{0.000000,0.000000,0.000000}%
\pgfsetstrokecolor{currentstroke}%
\pgfsetdash{}{0pt}%
\pgfpathmoveto{\pgfqpoint{4.957463in}{2.326349in}}%
\pgfpathlineto{\pgfqpoint{4.972592in}{2.342844in}}%
\pgfpathlineto{\pgfqpoint{4.987744in}{2.359528in}}%
\pgfpathlineto{\pgfqpoint{5.002918in}{2.376401in}}%
\pgfpathlineto{\pgfqpoint{5.018115in}{2.393464in}}%
\pgfpathlineto{\pgfqpoint{5.026418in}{2.415662in}}%
\pgfpathlineto{\pgfqpoint{5.034717in}{2.437727in}}%
\pgfpathlineto{\pgfqpoint{5.043011in}{2.459654in}}%
\pgfpathlineto{\pgfqpoint{5.051301in}{2.481441in}}%
\pgfpathlineto{\pgfqpoint{5.036083in}{2.463881in}}%
\pgfpathlineto{\pgfqpoint{5.020888in}{2.446513in}}%
\pgfpathlineto{\pgfqpoint{5.005715in}{2.429334in}}%
\pgfpathlineto{\pgfqpoint{4.990566in}{2.412346in}}%
\pgfpathlineto{\pgfqpoint{4.982297in}{2.391041in}}%
\pgfpathlineto{\pgfqpoint{4.974023in}{2.369604in}}%
\pgfpathlineto{\pgfqpoint{4.965745in}{2.348038in}}%
\pgfpathlineto{\pgfqpoint{4.957463in}{2.326349in}}%
\pgfpathclose%
\pgfusepath{fill}%
\end{pgfscope}%
\begin{pgfscope}%
\pgfpathrectangle{\pgfqpoint{1.150000in}{0.150000in}}{\pgfqpoint{5.700000in}{5.700000in}}%
\pgfusepath{clip}%
\pgfsetbuttcap%
\pgfsetroundjoin%
\definecolor{currentfill}{rgb}{0.218130,0.347432,0.550038}%
\pgfsetfillcolor{currentfill}%
\pgfsetfillopacity{0.800000}%
\pgfsetlinewidth{0.000000pt}%
\definecolor{currentstroke}{rgb}{0.000000,0.000000,0.000000}%
\pgfsetstrokecolor{currentstroke}%
\pgfsetdash{}{0pt}%
\pgfpathmoveto{\pgfqpoint{4.538419in}{1.428711in}}%
\pgfpathlineto{\pgfqpoint{4.553225in}{1.437874in}}%
\pgfpathlineto{\pgfqpoint{4.568049in}{1.447212in}}%
\pgfpathlineto{\pgfqpoint{4.582889in}{1.456726in}}%
\pgfpathlineto{\pgfqpoint{4.597745in}{1.466416in}}%
\pgfpathlineto{\pgfqpoint{4.606097in}{1.488150in}}%
\pgfpathlineto{\pgfqpoint{4.614447in}{1.509979in}}%
\pgfpathlineto{\pgfqpoint{4.622796in}{1.531898in}}%
\pgfpathlineto{\pgfqpoint{4.631143in}{1.553898in}}%
\pgfpathlineto{\pgfqpoint{4.616270in}{1.543430in}}%
\pgfpathlineto{\pgfqpoint{4.601414in}{1.533140in}}%
\pgfpathlineto{\pgfqpoint{4.586576in}{1.523026in}}%
\pgfpathlineto{\pgfqpoint{4.571754in}{1.513090in}}%
\pgfpathlineto{\pgfqpoint{4.563423in}{1.491854in}}%
\pgfpathlineto{\pgfqpoint{4.555090in}{1.470708in}}%
\pgfpathlineto{\pgfqpoint{4.546755in}{1.449658in}}%
\pgfpathlineto{\pgfqpoint{4.538419in}{1.428711in}}%
\pgfpathclose%
\pgfusepath{fill}%
\end{pgfscope}%
\begin{pgfscope}%
\pgfpathrectangle{\pgfqpoint{1.150000in}{0.150000in}}{\pgfqpoint{5.700000in}{5.700000in}}%
\pgfusepath{clip}%
\pgfsetbuttcap%
\pgfsetroundjoin%
\definecolor{currentfill}{rgb}{0.160665,0.478540,0.558115}%
\pgfsetfillcolor{currentfill}%
\pgfsetfillopacity{0.800000}%
\pgfsetlinewidth{0.000000pt}%
\definecolor{currentstroke}{rgb}{0.000000,0.000000,0.000000}%
\pgfsetstrokecolor{currentstroke}%
\pgfsetdash{}{0pt}%
\pgfpathmoveto{\pgfqpoint{4.731186in}{1.821774in}}%
\pgfpathlineto{\pgfqpoint{4.746132in}{1.834571in}}%
\pgfpathlineto{\pgfqpoint{4.761097in}{1.847550in}}%
\pgfpathlineto{\pgfqpoint{4.776082in}{1.860710in}}%
\pgfpathlineto{\pgfqpoint{4.791086in}{1.874053in}}%
\pgfpathlineto{\pgfqpoint{4.799431in}{1.897168in}}%
\pgfpathlineto{\pgfqpoint{4.807774in}{1.920264in}}%
\pgfpathlineto{\pgfqpoint{4.816114in}{1.943335in}}%
\pgfpathlineto{\pgfqpoint{4.824452in}{1.966377in}}%
\pgfpathlineto{\pgfqpoint{4.809427in}{1.952376in}}%
\pgfpathlineto{\pgfqpoint{4.794422in}{1.938558in}}%
\pgfpathlineto{\pgfqpoint{4.779437in}{1.924923in}}%
\pgfpathlineto{\pgfqpoint{4.764471in}{1.911472in}}%
\pgfpathlineto{\pgfqpoint{4.756153in}{1.889075in}}%
\pgfpathlineto{\pgfqpoint{4.747833in}{1.866655in}}%
\pgfpathlineto{\pgfqpoint{4.739510in}{1.844220in}}%
\pgfpathlineto{\pgfqpoint{4.731186in}{1.821774in}}%
\pgfpathclose%
\pgfusepath{fill}%
\end{pgfscope}%
\begin{pgfscope}%
\pgfpathrectangle{\pgfqpoint{1.150000in}{0.150000in}}{\pgfqpoint{5.700000in}{5.700000in}}%
\pgfusepath{clip}%
\pgfsetbuttcap%
\pgfsetroundjoin%
\definecolor{currentfill}{rgb}{0.232815,0.732247,0.459277}%
\pgfsetfillcolor{currentfill}%
\pgfsetfillopacity{0.800000}%
\pgfsetlinewidth{0.000000pt}%
\definecolor{currentstroke}{rgb}{0.000000,0.000000,0.000000}%
\pgfsetstrokecolor{currentstroke}%
\pgfsetdash{}{0pt}%
\pgfpathmoveto{\pgfqpoint{5.117426in}{2.650139in}}%
\pgfpathlineto{\pgfqpoint{5.132708in}{2.668779in}}%
\pgfpathlineto{\pgfqpoint{5.148013in}{2.687612in}}%
\pgfpathlineto{\pgfqpoint{5.163343in}{2.706639in}}%
\pgfpathlineto{\pgfqpoint{5.171597in}{2.727273in}}%
\pgfpathlineto{\pgfqpoint{5.179845in}{2.747719in}}%
\pgfpathlineto{\pgfqpoint{5.188087in}{2.767975in}}%
\pgfpathlineto{\pgfqpoint{5.196322in}{2.788036in}}%
\pgfpathlineto{\pgfqpoint{5.180973in}{2.768616in}}%
\pgfpathlineto{\pgfqpoint{5.165648in}{2.749391in}}%
\pgfpathlineto{\pgfqpoint{5.150348in}{2.730360in}}%
\pgfpathlineto{\pgfqpoint{5.142127in}{2.710583in}}%
\pgfpathlineto{\pgfqpoint{5.133900in}{2.690618in}}%
\pgfpathlineto{\pgfqpoint{5.125666in}{2.670469in}}%
\pgfpathlineto{\pgfqpoint{5.117426in}{2.650139in}}%
\pgfpathclose%
\pgfusepath{fill}%
\end{pgfscope}%
\begin{pgfscope}%
\pgfpathrectangle{\pgfqpoint{1.150000in}{0.150000in}}{\pgfqpoint{5.700000in}{5.700000in}}%
\pgfusepath{clip}%
\pgfsetbuttcap%
\pgfsetroundjoin%
\definecolor{currentfill}{rgb}{0.119423,0.611141,0.538982}%
\pgfsetfillcolor{currentfill}%
\pgfsetfillopacity{0.800000}%
\pgfsetlinewidth{0.000000pt}%
\definecolor{currentstroke}{rgb}{0.000000,0.000000,0.000000}%
\pgfsetstrokecolor{currentstroke}%
\pgfsetdash{}{0pt}%
\pgfpathmoveto{\pgfqpoint{4.924292in}{2.238441in}}%
\pgfpathlineto{\pgfqpoint{4.939401in}{2.254409in}}%
\pgfpathlineto{\pgfqpoint{4.954532in}{2.270564in}}%
\pgfpathlineto{\pgfqpoint{4.969685in}{2.286908in}}%
\pgfpathlineto{\pgfqpoint{4.984859in}{2.303439in}}%
\pgfpathlineto{\pgfqpoint{4.993179in}{2.326121in}}%
\pgfpathlineto{\pgfqpoint{5.001495in}{2.348689in}}%
\pgfpathlineto{\pgfqpoint{5.009807in}{2.371138in}}%
\pgfpathlineto{\pgfqpoint{5.018115in}{2.393464in}}%
\pgfpathlineto{\pgfqpoint{5.002918in}{2.376401in}}%
\pgfpathlineto{\pgfqpoint{4.987744in}{2.359528in}}%
\pgfpathlineto{\pgfqpoint{4.972592in}{2.342844in}}%
\pgfpathlineto{\pgfqpoint{4.957463in}{2.326349in}}%
\pgfpathlineto{\pgfqpoint{4.949176in}{2.304539in}}%
\pgfpathlineto{\pgfqpoint{4.940885in}{2.282615in}}%
\pgfpathlineto{\pgfqpoint{4.932591in}{2.260581in}}%
\pgfpathlineto{\pgfqpoint{4.924292in}{2.238441in}}%
\pgfpathclose%
\pgfusepath{fill}%
\end{pgfscope}%
\begin{pgfscope}%
\pgfpathrectangle{\pgfqpoint{1.150000in}{0.150000in}}{\pgfqpoint{5.700000in}{5.700000in}}%
\pgfusepath{clip}%
\pgfsetbuttcap%
\pgfsetroundjoin%
\definecolor{currentfill}{rgb}{0.171176,0.452530,0.557965}%
\pgfsetfillcolor{currentfill}%
\pgfsetfillopacity{0.800000}%
\pgfsetlinewidth{0.000000pt}%
\definecolor{currentstroke}{rgb}{0.000000,0.000000,0.000000}%
\pgfsetstrokecolor{currentstroke}%
\pgfsetdash{}{0pt}%
\pgfpathmoveto{\pgfqpoint{4.697866in}{1.732008in}}%
\pgfpathlineto{\pgfqpoint{4.712794in}{1.744119in}}%
\pgfpathlineto{\pgfqpoint{4.727739in}{1.756410in}}%
\pgfpathlineto{\pgfqpoint{4.742704in}{1.768882in}}%
\pgfpathlineto{\pgfqpoint{4.757688in}{1.781534in}}%
\pgfpathlineto{\pgfqpoint{4.766040in}{1.804661in}}%
\pgfpathlineto{\pgfqpoint{4.774391in}{1.827793in}}%
\pgfpathlineto{\pgfqpoint{4.782739in}{1.850926in}}%
\pgfpathlineto{\pgfqpoint{4.791086in}{1.874053in}}%
\pgfpathlineto{\pgfqpoint{4.776082in}{1.860710in}}%
\pgfpathlineto{\pgfqpoint{4.761097in}{1.847550in}}%
\pgfpathlineto{\pgfqpoint{4.746132in}{1.834571in}}%
\pgfpathlineto{\pgfqpoint{4.731186in}{1.821774in}}%
\pgfpathlineto{\pgfqpoint{4.722859in}{1.799323in}}%
\pgfpathlineto{\pgfqpoint{4.714530in}{1.776875in}}%
\pgfpathlineto{\pgfqpoint{4.706199in}{1.754434in}}%
\pgfpathlineto{\pgfqpoint{4.697866in}{1.732008in}}%
\pgfpathclose%
\pgfusepath{fill}%
\end{pgfscope}%
\begin{pgfscope}%
\pgfpathrectangle{\pgfqpoint{1.150000in}{0.150000in}}{\pgfqpoint{5.700000in}{5.700000in}}%
\pgfusepath{clip}%
\pgfsetbuttcap%
\pgfsetroundjoin%
\definecolor{currentfill}{rgb}{0.229739,0.322361,0.545706}%
\pgfsetfillcolor{currentfill}%
\pgfsetfillopacity{0.800000}%
\pgfsetlinewidth{0.000000pt}%
\definecolor{currentstroke}{rgb}{0.000000,0.000000,0.000000}%
\pgfsetstrokecolor{currentstroke}%
\pgfsetdash{}{0pt}%
\pgfpathmoveto{\pgfqpoint{4.505058in}{1.346110in}}%
\pgfpathlineto{\pgfqpoint{4.519851in}{1.354468in}}%
\pgfpathlineto{\pgfqpoint{4.534660in}{1.363001in}}%
\pgfpathlineto{\pgfqpoint{4.549485in}{1.371709in}}%
\pgfpathlineto{\pgfqpoint{4.564326in}{1.380591in}}%
\pgfpathlineto{\pgfqpoint{4.572683in}{1.401866in}}%
\pgfpathlineto{\pgfqpoint{4.581038in}{1.423267in}}%
\pgfpathlineto{\pgfqpoint{4.589393in}{1.444786in}}%
\pgfpathlineto{\pgfqpoint{4.597745in}{1.466416in}}%
\pgfpathlineto{\pgfqpoint{4.582889in}{1.456726in}}%
\pgfpathlineto{\pgfqpoint{4.568049in}{1.447212in}}%
\pgfpathlineto{\pgfqpoint{4.553225in}{1.437874in}}%
\pgfpathlineto{\pgfqpoint{4.538419in}{1.428711in}}%
\pgfpathlineto{\pgfqpoint{4.530081in}{1.407876in}}%
\pgfpathlineto{\pgfqpoint{4.521741in}{1.387159in}}%
\pgfpathlineto{\pgfqpoint{4.513400in}{1.366568in}}%
\pgfpathlineto{\pgfqpoint{4.505058in}{1.346110in}}%
\pgfpathclose%
\pgfusepath{fill}%
\end{pgfscope}%
\begin{pgfscope}%
\pgfpathrectangle{\pgfqpoint{1.150000in}{0.150000in}}{\pgfqpoint{5.700000in}{5.700000in}}%
\pgfusepath{clip}%
\pgfsetbuttcap%
\pgfsetroundjoin%
\definecolor{currentfill}{rgb}{0.122606,0.585371,0.546557}%
\pgfsetfillcolor{currentfill}%
\pgfsetfillopacity{0.800000}%
\pgfsetlinewidth{0.000000pt}%
\definecolor{currentstroke}{rgb}{0.000000,0.000000,0.000000}%
\pgfsetstrokecolor{currentstroke}%
\pgfsetdash{}{0pt}%
\pgfpathmoveto{\pgfqpoint{4.891062in}{2.148924in}}%
\pgfpathlineto{\pgfqpoint{4.906150in}{2.164331in}}%
\pgfpathlineto{\pgfqpoint{4.921259in}{2.179924in}}%
\pgfpathlineto{\pgfqpoint{4.936390in}{2.195703in}}%
\pgfpathlineto{\pgfqpoint{4.951542in}{2.211670in}}%
\pgfpathlineto{\pgfqpoint{4.959877in}{2.234759in}}%
\pgfpathlineto{\pgfqpoint{4.968208in}{2.257753in}}%
\pgfpathlineto{\pgfqpoint{4.976536in}{2.280648in}}%
\pgfpathlineto{\pgfqpoint{4.984859in}{2.303439in}}%
\pgfpathlineto{\pgfqpoint{4.969685in}{2.286908in}}%
\pgfpathlineto{\pgfqpoint{4.954532in}{2.270564in}}%
\pgfpathlineto{\pgfqpoint{4.939401in}{2.254409in}}%
\pgfpathlineto{\pgfqpoint{4.924292in}{2.238441in}}%
\pgfpathlineto{\pgfqpoint{4.915990in}{2.216200in}}%
\pgfpathlineto{\pgfqpoint{4.907684in}{2.193864in}}%
\pgfpathlineto{\pgfqpoint{4.899375in}{2.171437in}}%
\pgfpathlineto{\pgfqpoint{4.891062in}{2.148924in}}%
\pgfpathclose%
\pgfusepath{fill}%
\end{pgfscope}%
\begin{pgfscope}%
\pgfpathrectangle{\pgfqpoint{1.150000in}{0.150000in}}{\pgfqpoint{5.700000in}{5.700000in}}%
\pgfusepath{clip}%
\pgfsetbuttcap%
\pgfsetroundjoin%
\definecolor{currentfill}{rgb}{0.182256,0.426184,0.557120}%
\pgfsetfillcolor{currentfill}%
\pgfsetfillopacity{0.800000}%
\pgfsetlinewidth{0.000000pt}%
\definecolor{currentstroke}{rgb}{0.000000,0.000000,0.000000}%
\pgfsetstrokecolor{currentstroke}%
\pgfsetdash{}{0pt}%
\pgfpathmoveto{\pgfqpoint{4.664518in}{1.642575in}}%
\pgfpathlineto{\pgfqpoint{4.679426in}{1.653969in}}%
\pgfpathlineto{\pgfqpoint{4.694353in}{1.665541in}}%
\pgfpathlineto{\pgfqpoint{4.709298in}{1.677293in}}%
\pgfpathlineto{\pgfqpoint{4.724262in}{1.689224in}}%
\pgfpathlineto{\pgfqpoint{4.732620in}{1.712259in}}%
\pgfpathlineto{\pgfqpoint{4.740978in}{1.735327in}}%
\pgfpathlineto{\pgfqpoint{4.749334in}{1.758421in}}%
\pgfpathlineto{\pgfqpoint{4.757688in}{1.781534in}}%
\pgfpathlineto{\pgfqpoint{4.742704in}{1.768882in}}%
\pgfpathlineto{\pgfqpoint{4.727739in}{1.756410in}}%
\pgfpathlineto{\pgfqpoint{4.712794in}{1.744119in}}%
\pgfpathlineto{\pgfqpoint{4.697866in}{1.732008in}}%
\pgfpathlineto{\pgfqpoint{4.689532in}{1.709602in}}%
\pgfpathlineto{\pgfqpoint{4.681196in}{1.687224in}}%
\pgfpathlineto{\pgfqpoint{4.672857in}{1.664879in}}%
\pgfpathlineto{\pgfqpoint{4.664518in}{1.642575in}}%
\pgfpathclose%
\pgfusepath{fill}%
\end{pgfscope}%
\begin{pgfscope}%
\pgfpathrectangle{\pgfqpoint{1.150000in}{0.150000in}}{\pgfqpoint{5.700000in}{5.700000in}}%
\pgfusepath{clip}%
\pgfsetbuttcap%
\pgfsetroundjoin%
\definecolor{currentfill}{rgb}{0.196571,0.711827,0.479221}%
\pgfsetfillcolor{currentfill}%
\pgfsetfillopacity{0.800000}%
\pgfsetlinewidth{0.000000pt}%
\definecolor{currentstroke}{rgb}{0.000000,0.000000,0.000000}%
\pgfsetstrokecolor{currentstroke}%
\pgfsetdash{}{0pt}%
\pgfpathmoveto{\pgfqpoint{5.084407in}{2.567086in}}%
\pgfpathlineto{\pgfqpoint{5.099669in}{2.585299in}}%
\pgfpathlineto{\pgfqpoint{5.114954in}{2.603704in}}%
\pgfpathlineto{\pgfqpoint{5.130263in}{2.622302in}}%
\pgfpathlineto{\pgfqpoint{5.138542in}{2.643649in}}%
\pgfpathlineto{\pgfqpoint{5.146815in}{2.664823in}}%
\pgfpathlineto{\pgfqpoint{5.155082in}{2.685821in}}%
\pgfpathlineto{\pgfqpoint{5.163343in}{2.706639in}}%
\pgfpathlineto{\pgfqpoint{5.148013in}{2.687612in}}%
\pgfpathlineto{\pgfqpoint{5.132708in}{2.668779in}}%
\pgfpathlineto{\pgfqpoint{5.117426in}{2.650139in}}%
\pgfpathlineto{\pgfqpoint{5.109180in}{2.629632in}}%
\pgfpathlineto{\pgfqpoint{5.100928in}{2.608952in}}%
\pgfpathlineto{\pgfqpoint{5.092671in}{2.588102in}}%
\pgfpathlineto{\pgfqpoint{5.084407in}{2.567086in}}%
\pgfpathclose%
\pgfusepath{fill}%
\end{pgfscope}%
\begin{pgfscope}%
\pgfpathrectangle{\pgfqpoint{1.150000in}{0.150000in}}{\pgfqpoint{5.700000in}{5.700000in}}%
\pgfusepath{clip}%
\pgfsetbuttcap%
\pgfsetroundjoin%
\definecolor{currentfill}{rgb}{0.241237,0.296485,0.539709}%
\pgfsetfillcolor{currentfill}%
\pgfsetfillopacity{0.800000}%
\pgfsetlinewidth{0.000000pt}%
\definecolor{currentstroke}{rgb}{0.000000,0.000000,0.000000}%
\pgfsetstrokecolor{currentstroke}%
\pgfsetdash{}{0pt}%
\pgfpathmoveto{\pgfqpoint{4.471671in}{1.265766in}}%
\pgfpathlineto{\pgfqpoint{4.486452in}{1.273290in}}%
\pgfpathlineto{\pgfqpoint{4.501248in}{1.280988in}}%
\pgfpathlineto{\pgfqpoint{4.516060in}{1.288859in}}%
\pgfpathlineto{\pgfqpoint{4.530887in}{1.296903in}}%
\pgfpathlineto{\pgfqpoint{4.539249in}{1.317598in}}%
\pgfpathlineto{\pgfqpoint{4.547609in}{1.338449in}}%
\pgfpathlineto{\pgfqpoint{4.555969in}{1.359450in}}%
\pgfpathlineto{\pgfqpoint{4.564326in}{1.380591in}}%
\pgfpathlineto{\pgfqpoint{4.549485in}{1.371709in}}%
\pgfpathlineto{\pgfqpoint{4.534660in}{1.363001in}}%
\pgfpathlineto{\pgfqpoint{4.519851in}{1.354468in}}%
\pgfpathlineto{\pgfqpoint{4.505058in}{1.346110in}}%
\pgfpathlineto{\pgfqpoint{4.496713in}{1.325793in}}%
\pgfpathlineto{\pgfqpoint{4.488367in}{1.305625in}}%
\pgfpathlineto{\pgfqpoint{4.480020in}{1.285613in}}%
\pgfpathlineto{\pgfqpoint{4.471671in}{1.265766in}}%
\pgfpathclose%
\pgfusepath{fill}%
\end{pgfscope}%
\begin{pgfscope}%
\pgfpathrectangle{\pgfqpoint{1.150000in}{0.150000in}}{\pgfqpoint{5.700000in}{5.700000in}}%
\pgfusepath{clip}%
\pgfsetbuttcap%
\pgfsetroundjoin%
\definecolor{currentfill}{rgb}{0.129933,0.559582,0.551864}%
\pgfsetfillcolor{currentfill}%
\pgfsetfillopacity{0.800000}%
\pgfsetlinewidth{0.000000pt}%
\definecolor{currentstroke}{rgb}{0.000000,0.000000,0.000000}%
\pgfsetstrokecolor{currentstroke}%
\pgfsetdash{}{0pt}%
\pgfpathmoveto{\pgfqpoint{4.857780in}{2.058122in}}%
\pgfpathlineto{\pgfqpoint{4.872846in}{2.072934in}}%
\pgfpathlineto{\pgfqpoint{4.887934in}{2.087931in}}%
\pgfpathlineto{\pgfqpoint{4.903042in}{2.103113in}}%
\pgfpathlineto{\pgfqpoint{4.918172in}{2.118482in}}%
\pgfpathlineto{\pgfqpoint{4.926519in}{2.141894in}}%
\pgfpathlineto{\pgfqpoint{4.934863in}{2.165232in}}%
\pgfpathlineto{\pgfqpoint{4.943205in}{2.188493in}}%
\pgfpathlineto{\pgfqpoint{4.951542in}{2.211670in}}%
\pgfpathlineto{\pgfqpoint{4.936390in}{2.195703in}}%
\pgfpathlineto{\pgfqpoint{4.921259in}{2.179924in}}%
\pgfpathlineto{\pgfqpoint{4.906150in}{2.164331in}}%
\pgfpathlineto{\pgfqpoint{4.891062in}{2.148924in}}%
\pgfpathlineto{\pgfqpoint{4.882746in}{2.126331in}}%
\pgfpathlineto{\pgfqpoint{4.874427in}{2.103663in}}%
\pgfpathlineto{\pgfqpoint{4.866105in}{2.080924in}}%
\pgfpathlineto{\pgfqpoint{4.857780in}{2.058122in}}%
\pgfpathclose%
\pgfusepath{fill}%
\end{pgfscope}%
\begin{pgfscope}%
\pgfpathrectangle{\pgfqpoint{1.150000in}{0.150000in}}{\pgfqpoint{5.700000in}{5.700000in}}%
\pgfusepath{clip}%
\pgfsetbuttcap%
\pgfsetroundjoin%
\definecolor{currentfill}{rgb}{0.194100,0.399323,0.555565}%
\pgfsetfillcolor{currentfill}%
\pgfsetfillopacity{0.800000}%
\pgfsetlinewidth{0.000000pt}%
\definecolor{currentstroke}{rgb}{0.000000,0.000000,0.000000}%
\pgfsetstrokecolor{currentstroke}%
\pgfsetdash{}{0pt}%
\pgfpathmoveto{\pgfqpoint{4.631143in}{1.553898in}}%
\pgfpathlineto{\pgfqpoint{4.646034in}{1.564543in}}%
\pgfpathlineto{\pgfqpoint{4.660942in}{1.575366in}}%
\pgfpathlineto{\pgfqpoint{4.675868in}{1.586366in}}%
\pgfpathlineto{\pgfqpoint{4.690812in}{1.597545in}}%
\pgfpathlineto{\pgfqpoint{4.699177in}{1.620382in}}%
\pgfpathlineto{\pgfqpoint{4.707540in}{1.643279in}}%
\pgfpathlineto{\pgfqpoint{4.715901in}{1.666228in}}%
\pgfpathlineto{\pgfqpoint{4.724262in}{1.689224in}}%
\pgfpathlineto{\pgfqpoint{4.709298in}{1.677293in}}%
\pgfpathlineto{\pgfqpoint{4.694353in}{1.665541in}}%
\pgfpathlineto{\pgfqpoint{4.679426in}{1.653969in}}%
\pgfpathlineto{\pgfqpoint{4.664518in}{1.642575in}}%
\pgfpathlineto{\pgfqpoint{4.656176in}{1.620318in}}%
\pgfpathlineto{\pgfqpoint{4.647833in}{1.598115in}}%
\pgfpathlineto{\pgfqpoint{4.639489in}{1.575973in}}%
\pgfpathlineto{\pgfqpoint{4.631143in}{1.553898in}}%
\pgfpathclose%
\pgfusepath{fill}%
\end{pgfscope}%
\begin{pgfscope}%
\pgfpathrectangle{\pgfqpoint{1.150000in}{0.150000in}}{\pgfqpoint{5.700000in}{5.700000in}}%
\pgfusepath{clip}%
\pgfsetbuttcap%
\pgfsetroundjoin%
\definecolor{currentfill}{rgb}{0.166383,0.690856,0.496502}%
\pgfsetfillcolor{currentfill}%
\pgfsetfillopacity{0.800000}%
\pgfsetlinewidth{0.000000pt}%
\definecolor{currentstroke}{rgb}{0.000000,0.000000,0.000000}%
\pgfsetstrokecolor{currentstroke}%
\pgfsetdash{}{0pt}%
\pgfpathmoveto{\pgfqpoint{5.051301in}{2.481441in}}%
\pgfpathlineto{\pgfqpoint{5.066541in}{2.499191in}}%
\pgfpathlineto{\pgfqpoint{5.081805in}{2.517132in}}%
\pgfpathlineto{\pgfqpoint{5.097093in}{2.535266in}}%
\pgfpathlineto{\pgfqpoint{5.105393in}{2.557264in}}%
\pgfpathlineto{\pgfqpoint{5.113689in}{2.579105in}}%
\pgfpathlineto{\pgfqpoint{5.121979in}{2.600786in}}%
\pgfpathlineto{\pgfqpoint{5.130263in}{2.622302in}}%
\pgfpathlineto{\pgfqpoint{5.114954in}{2.603704in}}%
\pgfpathlineto{\pgfqpoint{5.099669in}{2.585299in}}%
\pgfpathlineto{\pgfqpoint{5.084407in}{2.567086in}}%
\pgfpathlineto{\pgfqpoint{5.076139in}{2.545908in}}%
\pgfpathlineto{\pgfqpoint{5.067864in}{2.524571in}}%
\pgfpathlineto{\pgfqpoint{5.059585in}{2.503081in}}%
\pgfpathlineto{\pgfqpoint{5.051301in}{2.481441in}}%
\pgfpathclose%
\pgfusepath{fill}%
\end{pgfscope}%
\begin{pgfscope}%
\pgfpathrectangle{\pgfqpoint{1.150000in}{0.150000in}}{\pgfqpoint{5.700000in}{5.700000in}}%
\pgfusepath{clip}%
\pgfsetbuttcap%
\pgfsetroundjoin%
\definecolor{currentfill}{rgb}{0.139147,0.533812,0.555298}%
\pgfsetfillcolor{currentfill}%
\pgfsetfillopacity{0.800000}%
\pgfsetlinewidth{0.000000pt}%
\definecolor{currentstroke}{rgb}{0.000000,0.000000,0.000000}%
\pgfsetstrokecolor{currentstroke}%
\pgfsetdash{}{0pt}%
\pgfpathmoveto{\pgfqpoint{4.824452in}{1.966377in}}%
\pgfpathlineto{\pgfqpoint{4.839498in}{1.980561in}}%
\pgfpathlineto{\pgfqpoint{4.854563in}{1.994930in}}%
\pgfpathlineto{\pgfqpoint{4.869649in}{2.009483in}}%
\pgfpathlineto{\pgfqpoint{4.884756in}{2.024220in}}%
\pgfpathlineto{\pgfqpoint{4.893114in}{2.047867in}}%
\pgfpathlineto{\pgfqpoint{4.901470in}{2.071463in}}%
\pgfpathlineto{\pgfqpoint{4.909822in}{2.095003in}}%
\pgfpathlineto{\pgfqpoint{4.918172in}{2.118482in}}%
\pgfpathlineto{\pgfqpoint{4.903042in}{2.103113in}}%
\pgfpathlineto{\pgfqpoint{4.887934in}{2.087931in}}%
\pgfpathlineto{\pgfqpoint{4.872846in}{2.072934in}}%
\pgfpathlineto{\pgfqpoint{4.857780in}{2.058122in}}%
\pgfpathlineto{\pgfqpoint{4.849452in}{2.035260in}}%
\pgfpathlineto{\pgfqpoint{4.841121in}{2.012345in}}%
\pgfpathlineto{\pgfqpoint{4.832788in}{1.989382in}}%
\pgfpathlineto{\pgfqpoint{4.824452in}{1.966377in}}%
\pgfpathclose%
\pgfusepath{fill}%
\end{pgfscope}%
\begin{pgfscope}%
\pgfpathrectangle{\pgfqpoint{1.150000in}{0.150000in}}{\pgfqpoint{5.700000in}{5.700000in}}%
\pgfusepath{clip}%
\pgfsetbuttcap%
\pgfsetroundjoin%
\definecolor{currentfill}{rgb}{0.204903,0.375746,0.553533}%
\pgfsetfillcolor{currentfill}%
\pgfsetfillopacity{0.800000}%
\pgfsetlinewidth{0.000000pt}%
\definecolor{currentstroke}{rgb}{0.000000,0.000000,0.000000}%
\pgfsetstrokecolor{currentstroke}%
\pgfsetdash{}{0pt}%
\pgfpathmoveto{\pgfqpoint{4.597745in}{1.466416in}}%
\pgfpathlineto{\pgfqpoint{4.612619in}{1.476282in}}%
\pgfpathlineto{\pgfqpoint{4.627510in}{1.486325in}}%
\pgfpathlineto{\pgfqpoint{4.642418in}{1.496544in}}%
\pgfpathlineto{\pgfqpoint{4.657344in}{1.506940in}}%
\pgfpathlineto{\pgfqpoint{4.665712in}{1.529466in}}%
\pgfpathlineto{\pgfqpoint{4.674080in}{1.552080in}}%
\pgfpathlineto{\pgfqpoint{4.682447in}{1.574776in}}%
\pgfpathlineto{\pgfqpoint{4.690812in}{1.597545in}}%
\pgfpathlineto{\pgfqpoint{4.675868in}{1.586366in}}%
\pgfpathlineto{\pgfqpoint{4.660942in}{1.575366in}}%
\pgfpathlineto{\pgfqpoint{4.646034in}{1.564543in}}%
\pgfpathlineto{\pgfqpoint{4.631143in}{1.553898in}}%
\pgfpathlineto{\pgfqpoint{4.622796in}{1.531898in}}%
\pgfpathlineto{\pgfqpoint{4.614447in}{1.509979in}}%
\pgfpathlineto{\pgfqpoint{4.606097in}{1.488150in}}%
\pgfpathlineto{\pgfqpoint{4.597745in}{1.466416in}}%
\pgfpathclose%
\pgfusepath{fill}%
\end{pgfscope}%
\begin{pgfscope}%
\pgfpathrectangle{\pgfqpoint{1.150000in}{0.150000in}}{\pgfqpoint{5.700000in}{5.700000in}}%
\pgfusepath{clip}%
\pgfsetbuttcap%
\pgfsetroundjoin%
\definecolor{currentfill}{rgb}{0.140210,0.665859,0.513427}%
\pgfsetfillcolor{currentfill}%
\pgfsetfillopacity{0.800000}%
\pgfsetlinewidth{0.000000pt}%
\definecolor{currentstroke}{rgb}{0.000000,0.000000,0.000000}%
\pgfsetstrokecolor{currentstroke}%
\pgfsetdash{}{0pt}%
\pgfpathmoveto{\pgfqpoint{5.018115in}{2.393464in}}%
\pgfpathlineto{\pgfqpoint{5.033334in}{2.410717in}}%
\pgfpathlineto{\pgfqpoint{5.048576in}{2.428160in}}%
\pgfpathlineto{\pgfqpoint{5.063841in}{2.445793in}}%
\pgfpathlineto{\pgfqpoint{5.072161in}{2.468374in}}%
\pgfpathlineto{\pgfqpoint{5.080477in}{2.490817in}}%
\pgfpathlineto{\pgfqpoint{5.088787in}{2.513115in}}%
\pgfpathlineto{\pgfqpoint{5.097093in}{2.535266in}}%
\pgfpathlineto{\pgfqpoint{5.081805in}{2.517132in}}%
\pgfpathlineto{\pgfqpoint{5.066541in}{2.499191in}}%
\pgfpathlineto{\pgfqpoint{5.051301in}{2.481441in}}%
\pgfpathlineto{\pgfqpoint{5.043011in}{2.459654in}}%
\pgfpathlineto{\pgfqpoint{5.034717in}{2.437727in}}%
\pgfpathlineto{\pgfqpoint{5.026418in}{2.415662in}}%
\pgfpathlineto{\pgfqpoint{5.018115in}{2.393464in}}%
\pgfpathclose%
\pgfusepath{fill}%
\end{pgfscope}%
\begin{pgfscope}%
\pgfpathrectangle{\pgfqpoint{1.150000in}{0.150000in}}{\pgfqpoint{5.700000in}{5.700000in}}%
\pgfusepath{clip}%
\pgfsetbuttcap%
\pgfsetroundjoin%
\definecolor{currentfill}{rgb}{0.149039,0.508051,0.557250}%
\pgfsetfillcolor{currentfill}%
\pgfsetfillopacity{0.800000}%
\pgfsetlinewidth{0.000000pt}%
\definecolor{currentstroke}{rgb}{0.000000,0.000000,0.000000}%
\pgfsetstrokecolor{currentstroke}%
\pgfsetdash{}{0pt}%
\pgfpathmoveto{\pgfqpoint{4.791086in}{1.874053in}}%
\pgfpathlineto{\pgfqpoint{4.806110in}{1.887578in}}%
\pgfpathlineto{\pgfqpoint{4.821154in}{1.901286in}}%
\pgfpathlineto{\pgfqpoint{4.836218in}{1.915176in}}%
\pgfpathlineto{\pgfqpoint{4.851302in}{1.929250in}}%
\pgfpathlineto{\pgfqpoint{4.859669in}{1.953038in}}%
\pgfpathlineto{\pgfqpoint{4.868034in}{1.976800in}}%
\pgfpathlineto{\pgfqpoint{4.876396in}{2.000529in}}%
\pgfpathlineto{\pgfqpoint{4.884756in}{2.024220in}}%
\pgfpathlineto{\pgfqpoint{4.869649in}{2.009483in}}%
\pgfpathlineto{\pgfqpoint{4.854563in}{1.994930in}}%
\pgfpathlineto{\pgfqpoint{4.839498in}{1.980561in}}%
\pgfpathlineto{\pgfqpoint{4.824452in}{1.966377in}}%
\pgfpathlineto{\pgfqpoint{4.816114in}{1.943335in}}%
\pgfpathlineto{\pgfqpoint{4.807774in}{1.920264in}}%
\pgfpathlineto{\pgfqpoint{4.799431in}{1.897168in}}%
\pgfpathlineto{\pgfqpoint{4.791086in}{1.874053in}}%
\pgfpathclose%
\pgfusepath{fill}%
\end{pgfscope}%
\begin{pgfscope}%
\pgfpathrectangle{\pgfqpoint{1.150000in}{0.150000in}}{\pgfqpoint{5.700000in}{5.700000in}}%
\pgfusepath{clip}%
\pgfsetbuttcap%
\pgfsetroundjoin%
\definecolor{currentfill}{rgb}{0.218130,0.347432,0.550038}%
\pgfsetfillcolor{currentfill}%
\pgfsetfillopacity{0.800000}%
\pgfsetlinewidth{0.000000pt}%
\definecolor{currentstroke}{rgb}{0.000000,0.000000,0.000000}%
\pgfsetstrokecolor{currentstroke}%
\pgfsetdash{}{0pt}%
\pgfpathmoveto{\pgfqpoint{4.564326in}{1.380591in}}%
\pgfpathlineto{\pgfqpoint{4.579184in}{1.389648in}}%
\pgfpathlineto{\pgfqpoint{4.594059in}{1.398881in}}%
\pgfpathlineto{\pgfqpoint{4.608950in}{1.408288in}}%
\pgfpathlineto{\pgfqpoint{4.623858in}{1.417871in}}%
\pgfpathlineto{\pgfqpoint{4.632231in}{1.439968in}}%
\pgfpathlineto{\pgfqpoint{4.640603in}{1.462184in}}%
\pgfpathlineto{\pgfqpoint{4.648974in}{1.484510in}}%
\pgfpathlineto{\pgfqpoint{4.657344in}{1.506940in}}%
\pgfpathlineto{\pgfqpoint{4.642418in}{1.496544in}}%
\pgfpathlineto{\pgfqpoint{4.627510in}{1.486325in}}%
\pgfpathlineto{\pgfqpoint{4.612619in}{1.476282in}}%
\pgfpathlineto{\pgfqpoint{4.597745in}{1.466416in}}%
\pgfpathlineto{\pgfqpoint{4.589393in}{1.444786in}}%
\pgfpathlineto{\pgfqpoint{4.581038in}{1.423267in}}%
\pgfpathlineto{\pgfqpoint{4.572683in}{1.401866in}}%
\pgfpathlineto{\pgfqpoint{4.564326in}{1.380591in}}%
\pgfpathclose%
\pgfusepath{fill}%
\end{pgfscope}%
\begin{pgfscope}%
\pgfpathrectangle{\pgfqpoint{1.150000in}{0.150000in}}{\pgfqpoint{5.700000in}{5.700000in}}%
\pgfusepath{clip}%
\pgfsetbuttcap%
\pgfsetroundjoin%
\definecolor{currentfill}{rgb}{0.124780,0.640461,0.527068}%
\pgfsetfillcolor{currentfill}%
\pgfsetfillopacity{0.800000}%
\pgfsetlinewidth{0.000000pt}%
\definecolor{currentstroke}{rgb}{0.000000,0.000000,0.000000}%
\pgfsetstrokecolor{currentstroke}%
\pgfsetdash{}{0pt}%
\pgfpathmoveto{\pgfqpoint{4.984859in}{2.303439in}}%
\pgfpathlineto{\pgfqpoint{5.000056in}{2.320160in}}%
\pgfpathlineto{\pgfqpoint{5.015276in}{2.337069in}}%
\pgfpathlineto{\pgfqpoint{5.030518in}{2.354168in}}%
\pgfpathlineto{\pgfqpoint{5.038855in}{2.377260in}}%
\pgfpathlineto{\pgfqpoint{5.047188in}{2.400231in}}%
\pgfpathlineto{\pgfqpoint{5.055517in}{2.423077in}}%
\pgfpathlineto{\pgfqpoint{5.063841in}{2.445793in}}%
\pgfpathlineto{\pgfqpoint{5.048576in}{2.428160in}}%
\pgfpathlineto{\pgfqpoint{5.033334in}{2.410717in}}%
\pgfpathlineto{\pgfqpoint{5.018115in}{2.393464in}}%
\pgfpathlineto{\pgfqpoint{5.009807in}{2.371138in}}%
\pgfpathlineto{\pgfqpoint{5.001495in}{2.348689in}}%
\pgfpathlineto{\pgfqpoint{4.993179in}{2.326121in}}%
\pgfpathlineto{\pgfqpoint{4.984859in}{2.303439in}}%
\pgfpathclose%
\pgfusepath{fill}%
\end{pgfscope}%
\begin{pgfscope}%
\pgfpathrectangle{\pgfqpoint{1.150000in}{0.150000in}}{\pgfqpoint{5.700000in}{5.700000in}}%
\pgfusepath{clip}%
\pgfsetbuttcap%
\pgfsetroundjoin%
\definecolor{currentfill}{rgb}{0.159194,0.482237,0.558073}%
\pgfsetfillcolor{currentfill}%
\pgfsetfillopacity{0.800000}%
\pgfsetlinewidth{0.000000pt}%
\definecolor{currentstroke}{rgb}{0.000000,0.000000,0.000000}%
\pgfsetstrokecolor{currentstroke}%
\pgfsetdash{}{0pt}%
\pgfpathmoveto{\pgfqpoint{4.757688in}{1.781534in}}%
\pgfpathlineto{\pgfqpoint{4.772691in}{1.794368in}}%
\pgfpathlineto{\pgfqpoint{4.787713in}{1.807383in}}%
\pgfpathlineto{\pgfqpoint{4.802755in}{1.820579in}}%
\pgfpathlineto{\pgfqpoint{4.817817in}{1.833957in}}%
\pgfpathlineto{\pgfqpoint{4.826191in}{1.857789in}}%
\pgfpathlineto{\pgfqpoint{4.834563in}{1.881619in}}%
\pgfpathlineto{\pgfqpoint{4.842934in}{1.905441in}}%
\pgfpathlineto{\pgfqpoint{4.851302in}{1.929250in}}%
\pgfpathlineto{\pgfqpoint{4.836218in}{1.915176in}}%
\pgfpathlineto{\pgfqpoint{4.821154in}{1.901286in}}%
\pgfpathlineto{\pgfqpoint{4.806110in}{1.887578in}}%
\pgfpathlineto{\pgfqpoint{4.791086in}{1.874053in}}%
\pgfpathlineto{\pgfqpoint{4.782739in}{1.850926in}}%
\pgfpathlineto{\pgfqpoint{4.774391in}{1.827793in}}%
\pgfpathlineto{\pgfqpoint{4.766040in}{1.804661in}}%
\pgfpathlineto{\pgfqpoint{4.757688in}{1.781534in}}%
\pgfpathclose%
\pgfusepath{fill}%
\end{pgfscope}%
\begin{pgfscope}%
\pgfpathrectangle{\pgfqpoint{1.150000in}{0.150000in}}{\pgfqpoint{5.700000in}{5.700000in}}%
\pgfusepath{clip}%
\pgfsetbuttcap%
\pgfsetroundjoin%
\definecolor{currentfill}{rgb}{0.119699,0.618490,0.536347}%
\pgfsetfillcolor{currentfill}%
\pgfsetfillopacity{0.800000}%
\pgfsetlinewidth{0.000000pt}%
\definecolor{currentstroke}{rgb}{0.000000,0.000000,0.000000}%
\pgfsetstrokecolor{currentstroke}%
\pgfsetdash{}{0pt}%
\pgfpathmoveto{\pgfqpoint{4.951542in}{2.211670in}}%
\pgfpathlineto{\pgfqpoint{4.966717in}{2.227825in}}%
\pgfpathlineto{\pgfqpoint{4.981913in}{2.244167in}}%
\pgfpathlineto{\pgfqpoint{4.997131in}{2.260697in}}%
\pgfpathlineto{\pgfqpoint{5.005483in}{2.284220in}}%
\pgfpathlineto{\pgfqpoint{5.013832in}{2.307644in}}%
\pgfpathlineto{\pgfqpoint{5.022177in}{2.330961in}}%
\pgfpathlineto{\pgfqpoint{5.030518in}{2.354168in}}%
\pgfpathlineto{\pgfqpoint{5.015276in}{2.337069in}}%
\pgfpathlineto{\pgfqpoint{5.000056in}{2.320160in}}%
\pgfpathlineto{\pgfqpoint{4.984859in}{2.303439in}}%
\pgfpathlineto{\pgfqpoint{4.976536in}{2.280648in}}%
\pgfpathlineto{\pgfqpoint{4.968208in}{2.257753in}}%
\pgfpathlineto{\pgfqpoint{4.959877in}{2.234759in}}%
\pgfpathlineto{\pgfqpoint{4.951542in}{2.211670in}}%
\pgfpathclose%
\pgfusepath{fill}%
\end{pgfscope}%
\begin{pgfscope}%
\pgfpathrectangle{\pgfqpoint{1.150000in}{0.150000in}}{\pgfqpoint{5.700000in}{5.700000in}}%
\pgfusepath{clip}%
\pgfsetbuttcap%
\pgfsetroundjoin%
\definecolor{currentfill}{rgb}{0.229739,0.322361,0.545706}%
\pgfsetfillcolor{currentfill}%
\pgfsetfillopacity{0.800000}%
\pgfsetlinewidth{0.000000pt}%
\definecolor{currentstroke}{rgb}{0.000000,0.000000,0.000000}%
\pgfsetstrokecolor{currentstroke}%
\pgfsetdash{}{0pt}%
\pgfpathmoveto{\pgfqpoint{4.530887in}{1.296903in}}%
\pgfpathlineto{\pgfqpoint{4.545731in}{1.305121in}}%
\pgfpathlineto{\pgfqpoint{4.560590in}{1.313513in}}%
\pgfpathlineto{\pgfqpoint{4.575465in}{1.322079in}}%
\pgfpathlineto{\pgfqpoint{4.590357in}{1.330819in}}%
\pgfpathlineto{\pgfqpoint{4.598734in}{1.352366in}}%
\pgfpathlineto{\pgfqpoint{4.607109in}{1.374062in}}%
\pgfpathlineto{\pgfqpoint{4.615484in}{1.395900in}}%
\pgfpathlineto{\pgfqpoint{4.623858in}{1.417871in}}%
\pgfpathlineto{\pgfqpoint{4.608950in}{1.408288in}}%
\pgfpathlineto{\pgfqpoint{4.594059in}{1.398881in}}%
\pgfpathlineto{\pgfqpoint{4.579184in}{1.389648in}}%
\pgfpathlineto{\pgfqpoint{4.564326in}{1.380591in}}%
\pgfpathlineto{\pgfqpoint{4.555969in}{1.359450in}}%
\pgfpathlineto{\pgfqpoint{4.547609in}{1.338449in}}%
\pgfpathlineto{\pgfqpoint{4.539249in}{1.317598in}}%
\pgfpathlineto{\pgfqpoint{4.530887in}{1.296903in}}%
\pgfpathclose%
\pgfusepath{fill}%
\end{pgfscope}%
\begin{pgfscope}%
\pgfpathrectangle{\pgfqpoint{1.150000in}{0.150000in}}{\pgfqpoint{5.700000in}{5.700000in}}%
\pgfusepath{clip}%
\pgfsetbuttcap%
\pgfsetroundjoin%
\definecolor{currentfill}{rgb}{0.169646,0.456262,0.558030}%
\pgfsetfillcolor{currentfill}%
\pgfsetfillopacity{0.800000}%
\pgfsetlinewidth{0.000000pt}%
\definecolor{currentstroke}{rgb}{0.000000,0.000000,0.000000}%
\pgfsetstrokecolor{currentstroke}%
\pgfsetdash{}{0pt}%
\pgfpathmoveto{\pgfqpoint{4.724262in}{1.689224in}}%
\pgfpathlineto{\pgfqpoint{4.739244in}{1.701335in}}%
\pgfpathlineto{\pgfqpoint{4.754245in}{1.713626in}}%
\pgfpathlineto{\pgfqpoint{4.769265in}{1.726096in}}%
\pgfpathlineto{\pgfqpoint{4.784305in}{1.738748in}}%
\pgfpathlineto{\pgfqpoint{4.792685in}{1.762519in}}%
\pgfpathlineto{\pgfqpoint{4.801064in}{1.786316in}}%
\pgfpathlineto{\pgfqpoint{4.809441in}{1.810131in}}%
\pgfpathlineto{\pgfqpoint{4.817817in}{1.833957in}}%
\pgfpathlineto{\pgfqpoint{4.802755in}{1.820579in}}%
\pgfpathlineto{\pgfqpoint{4.787713in}{1.807383in}}%
\pgfpathlineto{\pgfqpoint{4.772691in}{1.794368in}}%
\pgfpathlineto{\pgfqpoint{4.757688in}{1.781534in}}%
\pgfpathlineto{\pgfqpoint{4.749334in}{1.758421in}}%
\pgfpathlineto{\pgfqpoint{4.740978in}{1.735327in}}%
\pgfpathlineto{\pgfqpoint{4.732620in}{1.712259in}}%
\pgfpathlineto{\pgfqpoint{4.724262in}{1.689224in}}%
\pgfpathclose%
\pgfusepath{fill}%
\end{pgfscope}%
\begin{pgfscope}%
\pgfpathrectangle{\pgfqpoint{1.150000in}{0.150000in}}{\pgfqpoint{5.700000in}{5.700000in}}%
\pgfusepath{clip}%
\pgfsetbuttcap%
\pgfsetroundjoin%
\definecolor{currentfill}{rgb}{0.121831,0.589055,0.545623}%
\pgfsetfillcolor{currentfill}%
\pgfsetfillopacity{0.800000}%
\pgfsetlinewidth{0.000000pt}%
\definecolor{currentstroke}{rgb}{0.000000,0.000000,0.000000}%
\pgfsetstrokecolor{currentstroke}%
\pgfsetdash{}{0pt}%
\pgfpathmoveto{\pgfqpoint{4.918172in}{2.118482in}}%
\pgfpathlineto{\pgfqpoint{4.933323in}{2.134037in}}%
\pgfpathlineto{\pgfqpoint{4.948496in}{2.149778in}}%
\pgfpathlineto{\pgfqpoint{4.963690in}{2.165706in}}%
\pgfpathlineto{\pgfqpoint{4.972055in}{2.189578in}}%
\pgfpathlineto{\pgfqpoint{4.980417in}{2.213370in}}%
\pgfpathlineto{\pgfqpoint{4.988776in}{2.237079in}}%
\pgfpathlineto{\pgfqpoint{4.997131in}{2.260697in}}%
\pgfpathlineto{\pgfqpoint{4.981913in}{2.244167in}}%
\pgfpathlineto{\pgfqpoint{4.966717in}{2.227825in}}%
\pgfpathlineto{\pgfqpoint{4.951542in}{2.211670in}}%
\pgfpathlineto{\pgfqpoint{4.943205in}{2.188493in}}%
\pgfpathlineto{\pgfqpoint{4.934863in}{2.165232in}}%
\pgfpathlineto{\pgfqpoint{4.926519in}{2.141894in}}%
\pgfpathlineto{\pgfqpoint{4.918172in}{2.118482in}}%
\pgfpathclose%
\pgfusepath{fill}%
\end{pgfscope}%
\begin{pgfscope}%
\pgfpathrectangle{\pgfqpoint{1.150000in}{0.150000in}}{\pgfqpoint{5.700000in}{5.700000in}}%
\pgfusepath{clip}%
\pgfsetbuttcap%
\pgfsetroundjoin%
\definecolor{currentfill}{rgb}{0.180629,0.429975,0.557282}%
\pgfsetfillcolor{currentfill}%
\pgfsetfillopacity{0.800000}%
\pgfsetlinewidth{0.000000pt}%
\definecolor{currentstroke}{rgb}{0.000000,0.000000,0.000000}%
\pgfsetstrokecolor{currentstroke}%
\pgfsetdash{}{0pt}%
\pgfpathmoveto{\pgfqpoint{4.690812in}{1.597545in}}%
\pgfpathlineto{\pgfqpoint{4.705775in}{1.608902in}}%
\pgfpathlineto{\pgfqpoint{4.720755in}{1.620438in}}%
\pgfpathlineto{\pgfqpoint{4.735755in}{1.632152in}}%
\pgfpathlineto{\pgfqpoint{4.750773in}{1.644046in}}%
\pgfpathlineto{\pgfqpoint{4.759157in}{1.667650in}}%
\pgfpathlineto{\pgfqpoint{4.767541in}{1.691306in}}%
\pgfpathlineto{\pgfqpoint{4.775924in}{1.715007in}}%
\pgfpathlineto{\pgfqpoint{4.784305in}{1.738748in}}%
\pgfpathlineto{\pgfqpoint{4.769265in}{1.726096in}}%
\pgfpathlineto{\pgfqpoint{4.754245in}{1.713626in}}%
\pgfpathlineto{\pgfqpoint{4.739244in}{1.701335in}}%
\pgfpathlineto{\pgfqpoint{4.724262in}{1.689224in}}%
\pgfpathlineto{\pgfqpoint{4.715901in}{1.666228in}}%
\pgfpathlineto{\pgfqpoint{4.707540in}{1.643279in}}%
\pgfpathlineto{\pgfqpoint{4.699177in}{1.620382in}}%
\pgfpathlineto{\pgfqpoint{4.690812in}{1.597545in}}%
\pgfpathclose%
\pgfusepath{fill}%
\end{pgfscope}%
\begin{pgfscope}%
\pgfpathrectangle{\pgfqpoint{1.150000in}{0.150000in}}{\pgfqpoint{5.700000in}{5.700000in}}%
\pgfusepath{clip}%
\pgfsetbuttcap%
\pgfsetroundjoin%
\definecolor{currentfill}{rgb}{0.128729,0.563265,0.551229}%
\pgfsetfillcolor{currentfill}%
\pgfsetfillopacity{0.800000}%
\pgfsetlinewidth{0.000000pt}%
\definecolor{currentstroke}{rgb}{0.000000,0.000000,0.000000}%
\pgfsetstrokecolor{currentstroke}%
\pgfsetdash{}{0pt}%
\pgfpathmoveto{\pgfqpoint{4.884756in}{2.024220in}}%
\pgfpathlineto{\pgfqpoint{4.899884in}{2.039142in}}%
\pgfpathlineto{\pgfqpoint{4.915033in}{2.054249in}}%
\pgfpathlineto{\pgfqpoint{4.930204in}{2.069542in}}%
\pgfpathlineto{\pgfqpoint{4.938579in}{2.093673in}}%
\pgfpathlineto{\pgfqpoint{4.946952in}{2.117748in}}%
\pgfpathlineto{\pgfqpoint{4.955323in}{2.141761in}}%
\pgfpathlineto{\pgfqpoint{4.963690in}{2.165706in}}%
\pgfpathlineto{\pgfqpoint{4.948496in}{2.149778in}}%
\pgfpathlineto{\pgfqpoint{4.933323in}{2.134037in}}%
\pgfpathlineto{\pgfqpoint{4.918172in}{2.118482in}}%
\pgfpathlineto{\pgfqpoint{4.909822in}{2.095003in}}%
\pgfpathlineto{\pgfqpoint{4.901470in}{2.071463in}}%
\pgfpathlineto{\pgfqpoint{4.893114in}{2.047867in}}%
\pgfpathlineto{\pgfqpoint{4.884756in}{2.024220in}}%
\pgfpathclose%
\pgfusepath{fill}%
\end{pgfscope}%
\begin{pgfscope}%
\pgfpathrectangle{\pgfqpoint{1.150000in}{0.150000in}}{\pgfqpoint{5.700000in}{5.700000in}}%
\pgfusepath{clip}%
\pgfsetbuttcap%
\pgfsetroundjoin%
\definecolor{currentfill}{rgb}{0.192357,0.403199,0.555836}%
\pgfsetfillcolor{currentfill}%
\pgfsetfillopacity{0.800000}%
\pgfsetlinewidth{0.000000pt}%
\definecolor{currentstroke}{rgb}{0.000000,0.000000,0.000000}%
\pgfsetstrokecolor{currentstroke}%
\pgfsetdash{}{0pt}%
\pgfpathmoveto{\pgfqpoint{4.657344in}{1.506940in}}%
\pgfpathlineto{\pgfqpoint{4.672287in}{1.517513in}}%
\pgfpathlineto{\pgfqpoint{4.687248in}{1.528263in}}%
\pgfpathlineto{\pgfqpoint{4.702227in}{1.539190in}}%
\pgfpathlineto{\pgfqpoint{4.717224in}{1.550295in}}%
\pgfpathlineto{\pgfqpoint{4.725613in}{1.573619in}}%
\pgfpathlineto{\pgfqpoint{4.734000in}{1.597023in}}%
\pgfpathlineto{\pgfqpoint{4.742387in}{1.620501in}}%
\pgfpathlineto{\pgfqpoint{4.750773in}{1.644046in}}%
\pgfpathlineto{\pgfqpoint{4.735755in}{1.632152in}}%
\pgfpathlineto{\pgfqpoint{4.720755in}{1.620438in}}%
\pgfpathlineto{\pgfqpoint{4.705775in}{1.608902in}}%
\pgfpathlineto{\pgfqpoint{4.690812in}{1.597545in}}%
\pgfpathlineto{\pgfqpoint{4.682447in}{1.574776in}}%
\pgfpathlineto{\pgfqpoint{4.674080in}{1.552080in}}%
\pgfpathlineto{\pgfqpoint{4.665712in}{1.529466in}}%
\pgfpathlineto{\pgfqpoint{4.657344in}{1.506940in}}%
\pgfpathclose%
\pgfusepath{fill}%
\end{pgfscope}%
\begin{pgfscope}%
\pgfpathrectangle{\pgfqpoint{1.150000in}{0.150000in}}{\pgfqpoint{5.700000in}{5.700000in}}%
\pgfusepath{clip}%
\pgfsetbuttcap%
\pgfsetroundjoin%
\definecolor{currentfill}{rgb}{0.137770,0.537492,0.554906}%
\pgfsetfillcolor{currentfill}%
\pgfsetfillopacity{0.800000}%
\pgfsetlinewidth{0.000000pt}%
\definecolor{currentstroke}{rgb}{0.000000,0.000000,0.000000}%
\pgfsetstrokecolor{currentstroke}%
\pgfsetdash{}{0pt}%
\pgfpathmoveto{\pgfqpoint{4.851302in}{1.929250in}}%
\pgfpathlineto{\pgfqpoint{4.866407in}{1.943507in}}%
\pgfpathlineto{\pgfqpoint{4.881532in}{1.957948in}}%
\pgfpathlineto{\pgfqpoint{4.896678in}{1.972573in}}%
\pgfpathlineto{\pgfqpoint{4.905063in}{1.996870in}}%
\pgfpathlineto{\pgfqpoint{4.913445in}{2.021134in}}%
\pgfpathlineto{\pgfqpoint{4.921826in}{2.045360in}}%
\pgfpathlineto{\pgfqpoint{4.930204in}{2.069542in}}%
\pgfpathlineto{\pgfqpoint{4.915033in}{2.054249in}}%
\pgfpathlineto{\pgfqpoint{4.899884in}{2.039142in}}%
\pgfpathlineto{\pgfqpoint{4.884756in}{2.024220in}}%
\pgfpathlineto{\pgfqpoint{4.876396in}{2.000529in}}%
\pgfpathlineto{\pgfqpoint{4.868034in}{1.976800in}}%
\pgfpathlineto{\pgfqpoint{4.859669in}{1.953038in}}%
\pgfpathlineto{\pgfqpoint{4.851302in}{1.929250in}}%
\pgfpathclose%
\pgfusepath{fill}%
\end{pgfscope}%
\begin{pgfscope}%
\pgfpathrectangle{\pgfqpoint{1.150000in}{0.150000in}}{\pgfqpoint{5.700000in}{5.700000in}}%
\pgfusepath{clip}%
\pgfsetbuttcap%
\pgfsetroundjoin%
\definecolor{currentfill}{rgb}{0.204903,0.375746,0.553533}%
\pgfsetfillcolor{currentfill}%
\pgfsetfillopacity{0.800000}%
\pgfsetlinewidth{0.000000pt}%
\definecolor{currentstroke}{rgb}{0.000000,0.000000,0.000000}%
\pgfsetstrokecolor{currentstroke}%
\pgfsetdash{}{0pt}%
\pgfpathmoveto{\pgfqpoint{4.623858in}{1.417871in}}%
\pgfpathlineto{\pgfqpoint{4.638783in}{1.427629in}}%
\pgfpathlineto{\pgfqpoint{4.653726in}{1.437564in}}%
\pgfpathlineto{\pgfqpoint{4.668685in}{1.447674in}}%
\pgfpathlineto{\pgfqpoint{4.683663in}{1.457961in}}%
\pgfpathlineto{\pgfqpoint{4.692054in}{1.480886in}}%
\pgfpathlineto{\pgfqpoint{4.700445in}{1.503921in}}%
\pgfpathlineto{\pgfqpoint{4.708835in}{1.527060in}}%
\pgfpathlineto{\pgfqpoint{4.717224in}{1.550295in}}%
\pgfpathlineto{\pgfqpoint{4.702227in}{1.539190in}}%
\pgfpathlineto{\pgfqpoint{4.687248in}{1.528263in}}%
\pgfpathlineto{\pgfqpoint{4.672287in}{1.517513in}}%
\pgfpathlineto{\pgfqpoint{4.657344in}{1.506940in}}%
\pgfpathlineto{\pgfqpoint{4.648974in}{1.484510in}}%
\pgfpathlineto{\pgfqpoint{4.640603in}{1.462184in}}%
\pgfpathlineto{\pgfqpoint{4.632231in}{1.439968in}}%
\pgfpathlineto{\pgfqpoint{4.623858in}{1.417871in}}%
\pgfpathclose%
\pgfusepath{fill}%
\end{pgfscope}%
\begin{pgfscope}%
\pgfpathrectangle{\pgfqpoint{1.150000in}{0.150000in}}{\pgfqpoint{5.700000in}{5.700000in}}%
\pgfusepath{clip}%
\pgfsetbuttcap%
\pgfsetroundjoin%
\definecolor{currentfill}{rgb}{0.149039,0.508051,0.557250}%
\pgfsetfillcolor{currentfill}%
\pgfsetfillopacity{0.800000}%
\pgfsetlinewidth{0.000000pt}%
\definecolor{currentstroke}{rgb}{0.000000,0.000000,0.000000}%
\pgfsetstrokecolor{currentstroke}%
\pgfsetdash{}{0pt}%
\pgfpathmoveto{\pgfqpoint{4.817817in}{1.833957in}}%
\pgfpathlineto{\pgfqpoint{4.832898in}{1.847517in}}%
\pgfpathlineto{\pgfqpoint{4.848000in}{1.861260in}}%
\pgfpathlineto{\pgfqpoint{4.863122in}{1.875185in}}%
\pgfpathlineto{\pgfqpoint{4.871514in}{1.899549in}}%
\pgfpathlineto{\pgfqpoint{4.879904in}{1.923906in}}%
\pgfpathlineto{\pgfqpoint{4.888292in}{1.948250in}}%
\pgfpathlineto{\pgfqpoint{4.896678in}{1.972573in}}%
\pgfpathlineto{\pgfqpoint{4.881532in}{1.957948in}}%
\pgfpathlineto{\pgfqpoint{4.866407in}{1.943507in}}%
\pgfpathlineto{\pgfqpoint{4.851302in}{1.929250in}}%
\pgfpathlineto{\pgfqpoint{4.842934in}{1.905441in}}%
\pgfpathlineto{\pgfqpoint{4.834563in}{1.881619in}}%
\pgfpathlineto{\pgfqpoint{4.826191in}{1.857789in}}%
\pgfpathlineto{\pgfqpoint{4.817817in}{1.833957in}}%
\pgfpathclose%
\pgfusepath{fill}%
\end{pgfscope}%
\begin{pgfscope}%
\pgfpathrectangle{\pgfqpoint{1.150000in}{0.150000in}}{\pgfqpoint{5.700000in}{5.700000in}}%
\pgfusepath{clip}%
\pgfsetbuttcap%
\pgfsetroundjoin%
\definecolor{currentfill}{rgb}{0.218130,0.347432,0.550038}%
\pgfsetfillcolor{currentfill}%
\pgfsetfillopacity{0.800000}%
\pgfsetlinewidth{0.000000pt}%
\definecolor{currentstroke}{rgb}{0.000000,0.000000,0.000000}%
\pgfsetstrokecolor{currentstroke}%
\pgfsetdash{}{0pt}%
\pgfpathmoveto{\pgfqpoint{4.590357in}{1.330819in}}%
\pgfpathlineto{\pgfqpoint{4.605266in}{1.339734in}}%
\pgfpathlineto{\pgfqpoint{4.620191in}{1.348822in}}%
\pgfpathlineto{\pgfqpoint{4.635132in}{1.358086in}}%
\pgfpathlineto{\pgfqpoint{4.650091in}{1.367524in}}%
\pgfpathlineto{\pgfqpoint{4.658485in}{1.389928in}}%
\pgfpathlineto{\pgfqpoint{4.666878in}{1.412474in}}%
\pgfpathlineto{\pgfqpoint{4.675271in}{1.435154in}}%
\pgfpathlineto{\pgfqpoint{4.683663in}{1.457961in}}%
\pgfpathlineto{\pgfqpoint{4.668685in}{1.447674in}}%
\pgfpathlineto{\pgfqpoint{4.653726in}{1.437564in}}%
\pgfpathlineto{\pgfqpoint{4.638783in}{1.427629in}}%
\pgfpathlineto{\pgfqpoint{4.623858in}{1.417871in}}%
\pgfpathlineto{\pgfqpoint{4.615484in}{1.395900in}}%
\pgfpathlineto{\pgfqpoint{4.607109in}{1.374062in}}%
\pgfpathlineto{\pgfqpoint{4.598734in}{1.352366in}}%
\pgfpathlineto{\pgfqpoint{4.590357in}{1.330819in}}%
\pgfpathclose%
\pgfusepath{fill}%
\end{pgfscope}%
\begin{pgfscope}%
\pgfpathrectangle{\pgfqpoint{1.150000in}{0.150000in}}{\pgfqpoint{5.700000in}{5.700000in}}%
\pgfusepath{clip}%
\pgfsetbuttcap%
\pgfsetroundjoin%
\definecolor{currentfill}{rgb}{0.159194,0.482237,0.558073}%
\pgfsetfillcolor{currentfill}%
\pgfsetfillopacity{0.800000}%
\pgfsetlinewidth{0.000000pt}%
\definecolor{currentstroke}{rgb}{0.000000,0.000000,0.000000}%
\pgfsetstrokecolor{currentstroke}%
\pgfsetdash{}{0pt}%
\pgfpathmoveto{\pgfqpoint{4.784305in}{1.738748in}}%
\pgfpathlineto{\pgfqpoint{4.799364in}{1.751579in}}%
\pgfpathlineto{\pgfqpoint{4.814442in}{1.764592in}}%
\pgfpathlineto{\pgfqpoint{4.829541in}{1.777786in}}%
\pgfpathlineto{\pgfqpoint{4.837938in}{1.802114in}}%
\pgfpathlineto{\pgfqpoint{4.846334in}{1.826461in}}%
\pgfpathlineto{\pgfqpoint{4.854729in}{1.850820in}}%
\pgfpathlineto{\pgfqpoint{4.863122in}{1.875185in}}%
\pgfpathlineto{\pgfqpoint{4.848000in}{1.861260in}}%
\pgfpathlineto{\pgfqpoint{4.832898in}{1.847517in}}%
\pgfpathlineto{\pgfqpoint{4.817817in}{1.833957in}}%
\pgfpathlineto{\pgfqpoint{4.809441in}{1.810131in}}%
\pgfpathlineto{\pgfqpoint{4.801064in}{1.786316in}}%
\pgfpathlineto{\pgfqpoint{4.792685in}{1.762519in}}%
\pgfpathlineto{\pgfqpoint{4.784305in}{1.738748in}}%
\pgfpathclose%
\pgfusepath{fill}%
\end{pgfscope}%
\begin{pgfscope}%
\pgfpathrectangle{\pgfqpoint{1.150000in}{0.150000in}}{\pgfqpoint{5.700000in}{5.700000in}}%
\pgfusepath{clip}%
\pgfsetbuttcap%
\pgfsetroundjoin%
\definecolor{currentfill}{rgb}{0.171176,0.452530,0.557965}%
\pgfsetfillcolor{currentfill}%
\pgfsetfillopacity{0.800000}%
\pgfsetlinewidth{0.000000pt}%
\definecolor{currentstroke}{rgb}{0.000000,0.000000,0.000000}%
\pgfsetstrokecolor{currentstroke}%
\pgfsetdash{}{0pt}%
\pgfpathmoveto{\pgfqpoint{4.750773in}{1.644046in}}%
\pgfpathlineto{\pgfqpoint{4.765810in}{1.656118in}}%
\pgfpathlineto{\pgfqpoint{4.780865in}{1.668370in}}%
\pgfpathlineto{\pgfqpoint{4.795941in}{1.680802in}}%
\pgfpathlineto{\pgfqpoint{4.804342in}{1.704985in}}%
\pgfpathlineto{\pgfqpoint{4.812743in}{1.729215in}}%
\pgfpathlineto{\pgfqpoint{4.821142in}{1.753484in}}%
\pgfpathlineto{\pgfqpoint{4.829541in}{1.777786in}}%
\pgfpathlineto{\pgfqpoint{4.814442in}{1.764592in}}%
\pgfpathlineto{\pgfqpoint{4.799364in}{1.751579in}}%
\pgfpathlineto{\pgfqpoint{4.784305in}{1.738748in}}%
\pgfpathlineto{\pgfqpoint{4.775924in}{1.715007in}}%
\pgfpathlineto{\pgfqpoint{4.767541in}{1.691306in}}%
\pgfpathlineto{\pgfqpoint{4.759157in}{1.667650in}}%
\pgfpathlineto{\pgfqpoint{4.750773in}{1.644046in}}%
\pgfpathclose%
\pgfusepath{fill}%
\end{pgfscope}%
\begin{pgfscope}%
\pgfpathrectangle{\pgfqpoint{1.150000in}{0.150000in}}{\pgfqpoint{5.700000in}{5.700000in}}%
\pgfusepath{clip}%
\pgfsetbuttcap%
\pgfsetroundjoin%
\definecolor{currentfill}{rgb}{0.182256,0.426184,0.557120}%
\pgfsetfillcolor{currentfill}%
\pgfsetfillopacity{0.800000}%
\pgfsetlinewidth{0.000000pt}%
\definecolor{currentstroke}{rgb}{0.000000,0.000000,0.000000}%
\pgfsetstrokecolor{currentstroke}%
\pgfsetdash{}{0pt}%
\pgfpathmoveto{\pgfqpoint{4.717224in}{1.550295in}}%
\pgfpathlineto{\pgfqpoint{4.732240in}{1.561578in}}%
\pgfpathlineto{\pgfqpoint{4.747274in}{1.573039in}}%
\pgfpathlineto{\pgfqpoint{4.762327in}{1.584678in}}%
\pgfpathlineto{\pgfqpoint{4.770731in}{1.608604in}}%
\pgfpathlineto{\pgfqpoint{4.779135in}{1.632604in}}%
\pgfpathlineto{\pgfqpoint{4.787538in}{1.656673in}}%
\pgfpathlineto{\pgfqpoint{4.795941in}{1.680802in}}%
\pgfpathlineto{\pgfqpoint{4.780865in}{1.668370in}}%
\pgfpathlineto{\pgfqpoint{4.765810in}{1.656118in}}%
\pgfpathlineto{\pgfqpoint{4.750773in}{1.644046in}}%
\pgfpathlineto{\pgfqpoint{4.742387in}{1.620501in}}%
\pgfpathlineto{\pgfqpoint{4.734000in}{1.597023in}}%
\pgfpathlineto{\pgfqpoint{4.725613in}{1.573619in}}%
\pgfpathlineto{\pgfqpoint{4.717224in}{1.550295in}}%
\pgfpathclose%
\pgfusepath{fill}%
\end{pgfscope}%
\begin{pgfscope}%
\pgfpathrectangle{\pgfqpoint{1.150000in}{0.150000in}}{\pgfqpoint{5.700000in}{5.700000in}}%
\pgfusepath{clip}%
\pgfsetbuttcap%
\pgfsetroundjoin%
\definecolor{currentfill}{rgb}{0.194100,0.399323,0.555565}%
\pgfsetfillcolor{currentfill}%
\pgfsetfillopacity{0.800000}%
\pgfsetlinewidth{0.000000pt}%
\definecolor{currentstroke}{rgb}{0.000000,0.000000,0.000000}%
\pgfsetstrokecolor{currentstroke}%
\pgfsetdash{}{0pt}%
\pgfpathmoveto{\pgfqpoint{4.683663in}{1.457961in}}%
\pgfpathlineto{\pgfqpoint{4.698658in}{1.468424in}}%
\pgfpathlineto{\pgfqpoint{4.713671in}{1.479063in}}%
\pgfpathlineto{\pgfqpoint{4.728702in}{1.489880in}}%
\pgfpathlineto{\pgfqpoint{4.737109in}{1.513429in}}%
\pgfpathlineto{\pgfqpoint{4.745516in}{1.537084in}}%
\pgfpathlineto{\pgfqpoint{4.753921in}{1.560836in}}%
\pgfpathlineto{\pgfqpoint{4.762327in}{1.584678in}}%
\pgfpathlineto{\pgfqpoint{4.747274in}{1.573039in}}%
\pgfpathlineto{\pgfqpoint{4.732240in}{1.561578in}}%
\pgfpathlineto{\pgfqpoint{4.717224in}{1.550295in}}%
\pgfpathlineto{\pgfqpoint{4.708835in}{1.527060in}}%
\pgfpathlineto{\pgfqpoint{4.700445in}{1.503921in}}%
\pgfpathlineto{\pgfqpoint{4.692054in}{1.480886in}}%
\pgfpathlineto{\pgfqpoint{4.683663in}{1.457961in}}%
\pgfpathclose%
\pgfusepath{fill}%
\end{pgfscope}%
\begin{pgfscope}%
\pgfpathrectangle{\pgfqpoint{1.150000in}{0.150000in}}{\pgfqpoint{5.700000in}{5.700000in}}%
\pgfusepath{clip}%
\pgfsetbuttcap%
\pgfsetroundjoin%
\definecolor{currentfill}{rgb}{0.206756,0.371758,0.553117}%
\pgfsetfillcolor{currentfill}%
\pgfsetfillopacity{0.800000}%
\pgfsetlinewidth{0.000000pt}%
\definecolor{currentstroke}{rgb}{0.000000,0.000000,0.000000}%
\pgfsetstrokecolor{currentstroke}%
\pgfsetdash{}{0pt}%
\pgfpathmoveto{\pgfqpoint{4.650091in}{1.367524in}}%
\pgfpathlineto{\pgfqpoint{4.665067in}{1.377137in}}%
\pgfpathlineto{\pgfqpoint{4.680061in}{1.386926in}}%
\pgfpathlineto{\pgfqpoint{4.695072in}{1.396890in}}%
\pgfpathlineto{\pgfqpoint{4.703480in}{1.419941in}}%
\pgfpathlineto{\pgfqpoint{4.711888in}{1.443128in}}%
\pgfpathlineto{\pgfqpoint{4.720295in}{1.466443in}}%
\pgfpathlineto{\pgfqpoint{4.728702in}{1.489880in}}%
\pgfpathlineto{\pgfqpoint{4.713671in}{1.479063in}}%
\pgfpathlineto{\pgfqpoint{4.698658in}{1.468424in}}%
\pgfpathlineto{\pgfqpoint{4.683663in}{1.457961in}}%
\pgfpathlineto{\pgfqpoint{4.675271in}{1.435154in}}%
\pgfpathlineto{\pgfqpoint{4.666878in}{1.412474in}}%
\pgfpathlineto{\pgfqpoint{4.658485in}{1.389928in}}%
\pgfpathlineto{\pgfqpoint{4.650091in}{1.367524in}}%
\pgfpathclose%
\pgfusepath{fill}%
\end{pgfscope}%
\end{pgfpicture}%
\makeatother%
\endgroup%
}
    \caption{Pohľad na graf funkcie v $\mathbb{R}^3$.}
    \label{fig:graph_surface}
\end{figure}
\vspace*{\fill}
\newpage

%%%%%%%%%%%%%%%%%%%%%%%%%%%%%%%%%%%%%%%%%%%%%%%%%%%%%%%%%%%%%%%%%%%%%

\subsection{Newtonova metóda}

Táto metóda patrí medzi metódy druhého rádu, pretože pri hľadaní smeru poklesu využíva nielen gradient (prvé derivácie), ale aj Hessovu maticu (druhé derivácie). Hlavná myšlienka spočíva v tom, že v každom kroku funkciu $f$ aproximujeme Taylorovým polynómom druhého rádu a hľadáme jeho minimum.

Členy minimalizujúcej postupnosti $\{x^{[k]}\}$ počítame iteračným vzťahom:
$$ x^{[k+1]} = x^{[k]} - [\nabla^2 f(x^{[k]})]^{-1} \cdot \nabla f(x^{[k]}), $$
kde $\nabla f(x^{[k]})$ je gradient funkcie a $\nabla^2 f(x^{[k]})$ je Hessova matica funkcie v bode $x^{[k]}$.
Vďaka využitiu informácií o zakrivení funkcie táto metóda konverguje spravidla veľmi rýchlo (kvadraticky) v blízkosti lokálneho minima.

\subsubsection{Analýza metódy pri rôznych hodnotách počiatočnej aproximácie}

Na analýzu metódy pri rôznych počiatočných bodoch $x^{[0]}$ opäť potrebujeme vhodné ukončovacie kritérium. V súlade s teóriou volíme podmienku pre zmenu funkčnej hodnoty:
$$ |f(x^{[k]}) - f(x^{[k-1]})| < 0{,}001. $$
Budeme teda sledovať, po koľkých krokoch sa Newtonova metóda pre rôzne $x^{[0]}$ „ukončí“.

\vspace{0.5cm}
\noindent \textbf{Počiatočný bod} $x^{[0]} = [0; 0]$ \\
Ako prvý volíme počiatok súradnicovej sústavy. Pre Newtonovu metódu očakávame veľmi rýchlu konvergenciu.

\begin{table}[h!]
    \centering
    \begin{tabular}{ccccc}
        \toprule
        \textbf{Iterácia} & \textbf{Bod } $x^{[k]}$ & \textbf{Hodnota } $f(x^{[k]})$ & \textbf{Rozdiel } $|f_k - f_{k-1}|$ \\
        \midrule
        0 & $[0.000000; 0.000000]$ & $2.000000$ & $0.367879$ \\
        1 & $[0.000000; -1.000000]$ & $2.367879$ & $0.777743$ \\
        2 & $[0.483674; -0.862755]$ & $1.590136$ & $0.020643$ \\
        3 & $[0.400337; -0.759623]$ & $1.569493$ & $0.000496$ \\
        \bottomrule
    \end{tabular}
    \caption{Priebeh NM pre $x^{[0]}=[0;0]$. Minimum nájdený v $(0.400337, -0.759623)$.}
\end{table}

\noindent Vidíme, že v tomto prípade metóda dosiahla požadovanú presnosť už po 3. iterácii. To potvrdzuje vysokú efektivitu Newtonovej metódy pre hladké konvexné funkcie, kedy „skočí“ takmer priamo do minima.

\vspace{0.5cm}
\noindent \textbf{Počiatočný bod} $x^{[0]} = [1.5; 0.5]$ \\
Skúsme bod, ktorý je ďalej od minima a kde funkcia rastie vplyvom členu $e^y$.

\begin{table}[h!]
    \centering
    \begin{tabular}{cccc}
        \toprule
        \textbf{Iterácia} & \textbf{Bod } $x^{[k]}$ & \textbf{Hodnota } $f(x^{[k]})$ & \textbf{Rozdiel } $|f_k - f_{k-1}|$ \\
        \midrule
        0 & $[1{,}5;\;0{,}5]$ & $4{,}934351$ & $0{,}028395$ \\
        1 & $[0{,}624720;\;1{,}165504]$ & $4{,}905957$ & $1{,}941525$ \\
        2 & $[-0{,}022409;\;0{,}598473]$ & $2{,}964432$ & $0{,}810767$ \\
        3 & $[-0{,}150650;\;0{,}109123]$ & $2{,}153665$ & $0{,}432282$ \\
        4 & $[-0{,}082210;\;-0{,}512321]$ & $1{,}721383$ & $0{,}143755$ \\
        5 & $[0{,}268582;\;-0{,}662316]$ & $1{,}577628$ & $0{,}008577$ \\
        6 & $[0{,}379564;\;-0{,}738632]$ & $1{,}569051$ & $0{,}000055$ \\
        \bottomrule
    \end{tabular}
    \caption{Priebeh NM pre $x^{[0]} = [1{,}5;\;0{,}5]$.}
\end{table}



\noindent Aj pre tento vzdialenejší bod metóda konverguje veľmi svižne. Ukončovacie kritérium bolo splnené už v 6. kroku. Oproti metódam prvého rádu (ako metóda najväčšieho spádu) je to výrazný rozdiel. Následuje graf průběhu.

\begin{figure}[H]
    \centering
    \resizebox{0.8\textwidth}{!}{
    %% Creator: Matplotlib, PGF backend
%%
%% To include the figure in your LaTeX document, write
%%   \input{<filename>.pgf}
%%
%% Make sure the required packages are loaded in your preamble
%%   \usepackage{pgf}
%%
%% Also ensure that all the required font packages are loaded; for instance,
%% the lmodern package is sometimes necessary when using math font.
%%   \usepackage{lmodern}
%%
%% Figures using additional raster images can only be included by \input if
%% they are in the same directory as the main LaTeX file. For loading figures
%% from other directories you can use the `import` package
%%   \usepackage{import}
%%
%% and then include the figures with
%%   \import{<path to file>}{<filename>.pgf}
%%
%% Matplotlib used the following preamble
%%   
%%   \usepackage{fontspec}
%%   \setmainfont{DejaVuSerif.ttf}[Path=\detokenize{/home/radimek/Documents/projekt_mat_prog/mat_prog_kernel/lib/python3.12/site-packages/matplotlib/mpl-data/fonts/ttf/}]
%%   \setsansfont{DejaVuSans.ttf}[Path=\detokenize{/home/radimek/Documents/projekt_mat_prog/mat_prog_kernel/lib/python3.12/site-packages/matplotlib/mpl-data/fonts/ttf/}]
%%   \setmonofont{DejaVuSansMono.ttf}[Path=\detokenize{/home/radimek/Documents/projekt_mat_prog/mat_prog_kernel/lib/python3.12/site-packages/matplotlib/mpl-data/fonts/ttf/}]
%%   \makeatletter\@ifpackageloaded{underscore}{}{\usepackage[strings]{underscore}}\makeatother
%%
\begingroup%
\makeatletter%
\begin{pgfpicture}%
\pgfpathrectangle{\pgfpointorigin}{\pgfqpoint{8.000000in}{6.000000in}}%
\pgfusepath{use as bounding box, clip}%
\begin{pgfscope}%
\pgfsetbuttcap%
\pgfsetmiterjoin%
\definecolor{currentfill}{rgb}{1.000000,1.000000,1.000000}%
\pgfsetfillcolor{currentfill}%
\pgfsetlinewidth{0.000000pt}%
\definecolor{currentstroke}{rgb}{1.000000,1.000000,1.000000}%
\pgfsetstrokecolor{currentstroke}%
\pgfsetdash{}{0pt}%
\pgfpathmoveto{\pgfqpoint{0.000000in}{0.000000in}}%
\pgfpathlineto{\pgfqpoint{8.000000in}{0.000000in}}%
\pgfpathlineto{\pgfqpoint{8.000000in}{6.000000in}}%
\pgfpathlineto{\pgfqpoint{0.000000in}{6.000000in}}%
\pgfpathlineto{\pgfqpoint{0.000000in}{0.000000in}}%
\pgfpathclose%
\pgfusepath{fill}%
\end{pgfscope}%
\begin{pgfscope}%
\pgfsetbuttcap%
\pgfsetmiterjoin%
\definecolor{currentfill}{rgb}{1.000000,1.000000,1.000000}%
\pgfsetfillcolor{currentfill}%
\pgfsetlinewidth{0.000000pt}%
\definecolor{currentstroke}{rgb}{0.000000,0.000000,0.000000}%
\pgfsetstrokecolor{currentstroke}%
\pgfsetstrokeopacity{0.000000}%
\pgfsetdash{}{0pt}%
\pgfpathmoveto{\pgfqpoint{0.766095in}{0.571603in}}%
\pgfpathlineto{\pgfqpoint{7.739560in}{0.571603in}}%
\pgfpathlineto{\pgfqpoint{7.739560in}{5.797238in}}%
\pgfpathlineto{\pgfqpoint{0.766095in}{5.797238in}}%
\pgfpathlineto{\pgfqpoint{0.766095in}{0.571603in}}%
\pgfpathclose%
\pgfusepath{fill}%
\end{pgfscope}%
\begin{pgfscope}%
\pgfpathrectangle{\pgfqpoint{0.766095in}{0.571603in}}{\pgfqpoint{6.973465in}{5.225635in}}%
\pgfusepath{clip}%
\pgfsetbuttcap%
\pgfsetroundjoin%
\definecolor{currentfill}{rgb}{1.000000,0.000000,0.000000}%
\pgfsetfillcolor{currentfill}%
\pgfsetlinewidth{1.003750pt}%
\definecolor{currentstroke}{rgb}{1.000000,0.000000,0.000000}%
\pgfsetstrokecolor{currentstroke}%
\pgfsetdash{}{0pt}%
\pgfsys@defobject{currentmarker}{\pgfqpoint{-0.041667in}{-0.041667in}}{\pgfqpoint{0.041667in}{0.041667in}}{%
\pgfpathmoveto{\pgfqpoint{0.000000in}{-0.041667in}}%
\pgfpathcurveto{\pgfqpoint{0.011050in}{-0.041667in}}{\pgfqpoint{0.021649in}{-0.037276in}}{\pgfqpoint{0.029463in}{-0.029463in}}%
\pgfpathcurveto{\pgfqpoint{0.037276in}{-0.021649in}}{\pgfqpoint{0.041667in}{-0.011050in}}{\pgfqpoint{0.041667in}{0.000000in}}%
\pgfpathcurveto{\pgfqpoint{0.041667in}{0.011050in}}{\pgfqpoint{0.037276in}{0.021649in}}{\pgfqpoint{0.029463in}{0.029463in}}%
\pgfpathcurveto{\pgfqpoint{0.021649in}{0.037276in}}{\pgfqpoint{0.011050in}{0.041667in}}{\pgfqpoint{0.000000in}{0.041667in}}%
\pgfpathcurveto{\pgfqpoint{-0.011050in}{0.041667in}}{\pgfqpoint{-0.021649in}{0.037276in}}{\pgfqpoint{-0.029463in}{0.029463in}}%
\pgfpathcurveto{\pgfqpoint{-0.037276in}{0.021649in}}{\pgfqpoint{-0.041667in}{0.011050in}}{\pgfqpoint{-0.041667in}{0.000000in}}%
\pgfpathcurveto{\pgfqpoint{-0.041667in}{-0.011050in}}{\pgfqpoint{-0.037276in}{-0.021649in}}{\pgfqpoint{-0.029463in}{-0.029463in}}%
\pgfpathcurveto{\pgfqpoint{-0.021649in}{-0.037276in}}{\pgfqpoint{-0.011050in}{-0.041667in}}{\pgfqpoint{0.000000in}{-0.041667in}}%
\pgfpathlineto{\pgfqpoint{0.000000in}{-0.041667in}}%
\pgfpathclose%
\pgfusepath{stroke,fill}%
}%
\begin{pgfscope}%
\pgfsys@transformshift{6.577316in}{3.706984in}%
\pgfsys@useobject{currentmarker}{}%
\end{pgfscope}%
\begin{pgfscope}%
\pgfsys@transformshift{4.542738in}{5.098057in}%
\pgfsys@useobject{currentmarker}{}%
\end{pgfscope}%
\begin{pgfscope}%
\pgfsys@transformshift{3.038495in}{3.912818in}%
\pgfsys@useobject{currentmarker}{}%
\end{pgfscope}%
\begin{pgfscope}%
\pgfsys@transformshift{2.740398in}{2.889953in}%
\pgfsys@useobject{currentmarker}{}%
\end{pgfscope}%
\begin{pgfscope}%
\pgfsys@transformshift{2.899488in}{1.590976in}%
\pgfsys@useobject{currentmarker}{}%
\end{pgfscope}%
\begin{pgfscope}%
\pgfsys@transformshift{3.714898in}{1.277449in}%
\pgfsys@useobject{currentmarker}{}%
\end{pgfscope}%
\begin{pgfscope}%
\pgfsys@transformshift{3.972874in}{1.117929in}%
\pgfsys@useobject{currentmarker}{}%
\end{pgfscope}%
\end{pgfscope}%
\begin{pgfscope}%
\pgfsetbuttcap%
\pgfsetroundjoin%
\definecolor{currentfill}{rgb}{0.000000,0.000000,0.000000}%
\pgfsetfillcolor{currentfill}%
\pgfsetlinewidth{0.803000pt}%
\definecolor{currentstroke}{rgb}{0.000000,0.000000,0.000000}%
\pgfsetstrokecolor{currentstroke}%
\pgfsetdash{}{0pt}%
\pgfsys@defobject{currentmarker}{\pgfqpoint{0.000000in}{-0.048611in}}{\pgfqpoint{0.000000in}{0.000000in}}{%
\pgfpathmoveto{\pgfqpoint{0.000000in}{0.000000in}}%
\pgfpathlineto{\pgfqpoint{0.000000in}{-0.048611in}}%
\pgfusepath{stroke,fill}%
}%
\begin{pgfscope}%
\pgfsys@transformshift{0.766095in}{0.571603in}%
\pgfsys@useobject{currentmarker}{}%
\end{pgfscope}%
\end{pgfscope}%
\begin{pgfscope}%
\definecolor{textcolor}{rgb}{0.000000,0.000000,0.000000}%
\pgfsetstrokecolor{textcolor}%
\pgfsetfillcolor{textcolor}%
\pgftext[x=0.766095in,y=0.474381in,,top]{\color{textcolor}\sffamily\fontsize{10.000000}{12.000000}\selectfont \ensuremath{-}1.0}%
\end{pgfscope}%
\begin{pgfscope}%
\pgfsetbuttcap%
\pgfsetroundjoin%
\definecolor{currentfill}{rgb}{0.000000,0.000000,0.000000}%
\pgfsetfillcolor{currentfill}%
\pgfsetlinewidth{0.803000pt}%
\definecolor{currentstroke}{rgb}{0.000000,0.000000,0.000000}%
\pgfsetstrokecolor{currentstroke}%
\pgfsetdash{}{0pt}%
\pgfsys@defobject{currentmarker}{\pgfqpoint{0.000000in}{-0.048611in}}{\pgfqpoint{0.000000in}{0.000000in}}{%
\pgfpathmoveto{\pgfqpoint{0.000000in}{0.000000in}}%
\pgfpathlineto{\pgfqpoint{0.000000in}{-0.048611in}}%
\pgfusepath{stroke,fill}%
}%
\begin{pgfscope}%
\pgfsys@transformshift{1.928339in}{0.571603in}%
\pgfsys@useobject{currentmarker}{}%
\end{pgfscope}%
\end{pgfscope}%
\begin{pgfscope}%
\definecolor{textcolor}{rgb}{0.000000,0.000000,0.000000}%
\pgfsetstrokecolor{textcolor}%
\pgfsetfillcolor{textcolor}%
\pgftext[x=1.928339in,y=0.474381in,,top]{\color{textcolor}\sffamily\fontsize{10.000000}{12.000000}\selectfont \ensuremath{-}0.5}%
\end{pgfscope}%
\begin{pgfscope}%
\pgfsetbuttcap%
\pgfsetroundjoin%
\definecolor{currentfill}{rgb}{0.000000,0.000000,0.000000}%
\pgfsetfillcolor{currentfill}%
\pgfsetlinewidth{0.803000pt}%
\definecolor{currentstroke}{rgb}{0.000000,0.000000,0.000000}%
\pgfsetstrokecolor{currentstroke}%
\pgfsetdash{}{0pt}%
\pgfsys@defobject{currentmarker}{\pgfqpoint{0.000000in}{-0.048611in}}{\pgfqpoint{0.000000in}{0.000000in}}{%
\pgfpathmoveto{\pgfqpoint{0.000000in}{0.000000in}}%
\pgfpathlineto{\pgfqpoint{0.000000in}{-0.048611in}}%
\pgfusepath{stroke,fill}%
}%
\begin{pgfscope}%
\pgfsys@transformshift{3.090583in}{0.571603in}%
\pgfsys@useobject{currentmarker}{}%
\end{pgfscope}%
\end{pgfscope}%
\begin{pgfscope}%
\definecolor{textcolor}{rgb}{0.000000,0.000000,0.000000}%
\pgfsetstrokecolor{textcolor}%
\pgfsetfillcolor{textcolor}%
\pgftext[x=3.090583in,y=0.474381in,,top]{\color{textcolor}\sffamily\fontsize{10.000000}{12.000000}\selectfont 0.0}%
\end{pgfscope}%
\begin{pgfscope}%
\pgfsetbuttcap%
\pgfsetroundjoin%
\definecolor{currentfill}{rgb}{0.000000,0.000000,0.000000}%
\pgfsetfillcolor{currentfill}%
\pgfsetlinewidth{0.803000pt}%
\definecolor{currentstroke}{rgb}{0.000000,0.000000,0.000000}%
\pgfsetstrokecolor{currentstroke}%
\pgfsetdash{}{0pt}%
\pgfsys@defobject{currentmarker}{\pgfqpoint{0.000000in}{-0.048611in}}{\pgfqpoint{0.000000in}{0.000000in}}{%
\pgfpathmoveto{\pgfqpoint{0.000000in}{0.000000in}}%
\pgfpathlineto{\pgfqpoint{0.000000in}{-0.048611in}}%
\pgfusepath{stroke,fill}%
}%
\begin{pgfscope}%
\pgfsys@transformshift{4.252828in}{0.571603in}%
\pgfsys@useobject{currentmarker}{}%
\end{pgfscope}%
\end{pgfscope}%
\begin{pgfscope}%
\definecolor{textcolor}{rgb}{0.000000,0.000000,0.000000}%
\pgfsetstrokecolor{textcolor}%
\pgfsetfillcolor{textcolor}%
\pgftext[x=4.252828in,y=0.474381in,,top]{\color{textcolor}\sffamily\fontsize{10.000000}{12.000000}\selectfont 0.5}%
\end{pgfscope}%
\begin{pgfscope}%
\pgfsetbuttcap%
\pgfsetroundjoin%
\definecolor{currentfill}{rgb}{0.000000,0.000000,0.000000}%
\pgfsetfillcolor{currentfill}%
\pgfsetlinewidth{0.803000pt}%
\definecolor{currentstroke}{rgb}{0.000000,0.000000,0.000000}%
\pgfsetstrokecolor{currentstroke}%
\pgfsetdash{}{0pt}%
\pgfsys@defobject{currentmarker}{\pgfqpoint{0.000000in}{-0.048611in}}{\pgfqpoint{0.000000in}{0.000000in}}{%
\pgfpathmoveto{\pgfqpoint{0.000000in}{0.000000in}}%
\pgfpathlineto{\pgfqpoint{0.000000in}{-0.048611in}}%
\pgfusepath{stroke,fill}%
}%
\begin{pgfscope}%
\pgfsys@transformshift{5.415072in}{0.571603in}%
\pgfsys@useobject{currentmarker}{}%
\end{pgfscope}%
\end{pgfscope}%
\begin{pgfscope}%
\definecolor{textcolor}{rgb}{0.000000,0.000000,0.000000}%
\pgfsetstrokecolor{textcolor}%
\pgfsetfillcolor{textcolor}%
\pgftext[x=5.415072in,y=0.474381in,,top]{\color{textcolor}\sffamily\fontsize{10.000000}{12.000000}\selectfont 1.0}%
\end{pgfscope}%
\begin{pgfscope}%
\pgfsetbuttcap%
\pgfsetroundjoin%
\definecolor{currentfill}{rgb}{0.000000,0.000000,0.000000}%
\pgfsetfillcolor{currentfill}%
\pgfsetlinewidth{0.803000pt}%
\definecolor{currentstroke}{rgb}{0.000000,0.000000,0.000000}%
\pgfsetstrokecolor{currentstroke}%
\pgfsetdash{}{0pt}%
\pgfsys@defobject{currentmarker}{\pgfqpoint{0.000000in}{-0.048611in}}{\pgfqpoint{0.000000in}{0.000000in}}{%
\pgfpathmoveto{\pgfqpoint{0.000000in}{0.000000in}}%
\pgfpathlineto{\pgfqpoint{0.000000in}{-0.048611in}}%
\pgfusepath{stroke,fill}%
}%
\begin{pgfscope}%
\pgfsys@transformshift{6.577316in}{0.571603in}%
\pgfsys@useobject{currentmarker}{}%
\end{pgfscope}%
\end{pgfscope}%
\begin{pgfscope}%
\definecolor{textcolor}{rgb}{0.000000,0.000000,0.000000}%
\pgfsetstrokecolor{textcolor}%
\pgfsetfillcolor{textcolor}%
\pgftext[x=6.577316in,y=0.474381in,,top]{\color{textcolor}\sffamily\fontsize{10.000000}{12.000000}\selectfont 1.5}%
\end{pgfscope}%
\begin{pgfscope}%
\pgfsetbuttcap%
\pgfsetroundjoin%
\definecolor{currentfill}{rgb}{0.000000,0.000000,0.000000}%
\pgfsetfillcolor{currentfill}%
\pgfsetlinewidth{0.803000pt}%
\definecolor{currentstroke}{rgb}{0.000000,0.000000,0.000000}%
\pgfsetstrokecolor{currentstroke}%
\pgfsetdash{}{0pt}%
\pgfsys@defobject{currentmarker}{\pgfqpoint{0.000000in}{-0.048611in}}{\pgfqpoint{0.000000in}{0.000000in}}{%
\pgfpathmoveto{\pgfqpoint{0.000000in}{0.000000in}}%
\pgfpathlineto{\pgfqpoint{0.000000in}{-0.048611in}}%
\pgfusepath{stroke,fill}%
}%
\begin{pgfscope}%
\pgfsys@transformshift{7.739560in}{0.571603in}%
\pgfsys@useobject{currentmarker}{}%
\end{pgfscope}%
\end{pgfscope}%
\begin{pgfscope}%
\definecolor{textcolor}{rgb}{0.000000,0.000000,0.000000}%
\pgfsetstrokecolor{textcolor}%
\pgfsetfillcolor{textcolor}%
\pgftext[x=7.739560in,y=0.474381in,,top]{\color{textcolor}\sffamily\fontsize{10.000000}{12.000000}\selectfont 2.0}%
\end{pgfscope}%
\begin{pgfscope}%
\definecolor{textcolor}{rgb}{0.000000,0.000000,0.000000}%
\pgfsetstrokecolor{textcolor}%
\pgfsetfillcolor{textcolor}%
\pgftext[x=4.252828in,y=0.284413in,,top]{\color{textcolor}\sffamily\fontsize{10.000000}{12.000000}\selectfont x}%
\end{pgfscope}%
\begin{pgfscope}%
\pgfsetbuttcap%
\pgfsetroundjoin%
\definecolor{currentfill}{rgb}{0.000000,0.000000,0.000000}%
\pgfsetfillcolor{currentfill}%
\pgfsetlinewidth{0.803000pt}%
\definecolor{currentstroke}{rgb}{0.000000,0.000000,0.000000}%
\pgfsetstrokecolor{currentstroke}%
\pgfsetdash{}{0pt}%
\pgfsys@defobject{currentmarker}{\pgfqpoint{-0.048611in}{0.000000in}}{\pgfqpoint{-0.000000in}{0.000000in}}{%
\pgfpathmoveto{\pgfqpoint{-0.000000in}{0.000000in}}%
\pgfpathlineto{\pgfqpoint{-0.048611in}{0.000000in}}%
\pgfusepath{stroke,fill}%
}%
\begin{pgfscope}%
\pgfsys@transformshift{0.766095in}{0.571603in}%
\pgfsys@useobject{currentmarker}{}%
\end{pgfscope}%
\end{pgfscope}%
\begin{pgfscope}%
\definecolor{textcolor}{rgb}{0.000000,0.000000,0.000000}%
\pgfsetstrokecolor{textcolor}%
\pgfsetfillcolor{textcolor}%
\pgftext[x=0.339968in, y=0.518842in, left, base]{\color{textcolor}\sffamily\fontsize{10.000000}{12.000000}\selectfont \ensuremath{-}1.0}%
\end{pgfscope}%
\begin{pgfscope}%
\pgfsetbuttcap%
\pgfsetroundjoin%
\definecolor{currentfill}{rgb}{0.000000,0.000000,0.000000}%
\pgfsetfillcolor{currentfill}%
\pgfsetlinewidth{0.803000pt}%
\definecolor{currentstroke}{rgb}{0.000000,0.000000,0.000000}%
\pgfsetstrokecolor{currentstroke}%
\pgfsetdash{}{0pt}%
\pgfsys@defobject{currentmarker}{\pgfqpoint{-0.048611in}{0.000000in}}{\pgfqpoint{-0.000000in}{0.000000in}}{%
\pgfpathmoveto{\pgfqpoint{-0.000000in}{0.000000in}}%
\pgfpathlineto{\pgfqpoint{-0.048611in}{0.000000in}}%
\pgfusepath{stroke,fill}%
}%
\begin{pgfscope}%
\pgfsys@transformshift{0.766095in}{1.616730in}%
\pgfsys@useobject{currentmarker}{}%
\end{pgfscope}%
\end{pgfscope}%
\begin{pgfscope}%
\definecolor{textcolor}{rgb}{0.000000,0.000000,0.000000}%
\pgfsetstrokecolor{textcolor}%
\pgfsetfillcolor{textcolor}%
\pgftext[x=0.339968in, y=1.563969in, left, base]{\color{textcolor}\sffamily\fontsize{10.000000}{12.000000}\selectfont \ensuremath{-}0.5}%
\end{pgfscope}%
\begin{pgfscope}%
\pgfsetbuttcap%
\pgfsetroundjoin%
\definecolor{currentfill}{rgb}{0.000000,0.000000,0.000000}%
\pgfsetfillcolor{currentfill}%
\pgfsetlinewidth{0.803000pt}%
\definecolor{currentstroke}{rgb}{0.000000,0.000000,0.000000}%
\pgfsetstrokecolor{currentstroke}%
\pgfsetdash{}{0pt}%
\pgfsys@defobject{currentmarker}{\pgfqpoint{-0.048611in}{0.000000in}}{\pgfqpoint{-0.000000in}{0.000000in}}{%
\pgfpathmoveto{\pgfqpoint{-0.000000in}{0.000000in}}%
\pgfpathlineto{\pgfqpoint{-0.048611in}{0.000000in}}%
\pgfusepath{stroke,fill}%
}%
\begin{pgfscope}%
\pgfsys@transformshift{0.766095in}{2.661857in}%
\pgfsys@useobject{currentmarker}{}%
\end{pgfscope}%
\end{pgfscope}%
\begin{pgfscope}%
\definecolor{textcolor}{rgb}{0.000000,0.000000,0.000000}%
\pgfsetstrokecolor{textcolor}%
\pgfsetfillcolor{textcolor}%
\pgftext[x=0.447993in, y=2.609096in, left, base]{\color{textcolor}\sffamily\fontsize{10.000000}{12.000000}\selectfont 0.0}%
\end{pgfscope}%
\begin{pgfscope}%
\pgfsetbuttcap%
\pgfsetroundjoin%
\definecolor{currentfill}{rgb}{0.000000,0.000000,0.000000}%
\pgfsetfillcolor{currentfill}%
\pgfsetlinewidth{0.803000pt}%
\definecolor{currentstroke}{rgb}{0.000000,0.000000,0.000000}%
\pgfsetstrokecolor{currentstroke}%
\pgfsetdash{}{0pt}%
\pgfsys@defobject{currentmarker}{\pgfqpoint{-0.048611in}{0.000000in}}{\pgfqpoint{-0.000000in}{0.000000in}}{%
\pgfpathmoveto{\pgfqpoint{-0.000000in}{0.000000in}}%
\pgfpathlineto{\pgfqpoint{-0.048611in}{0.000000in}}%
\pgfusepath{stroke,fill}%
}%
\begin{pgfscope}%
\pgfsys@transformshift{0.766095in}{3.706984in}%
\pgfsys@useobject{currentmarker}{}%
\end{pgfscope}%
\end{pgfscope}%
\begin{pgfscope}%
\definecolor{textcolor}{rgb}{0.000000,0.000000,0.000000}%
\pgfsetstrokecolor{textcolor}%
\pgfsetfillcolor{textcolor}%
\pgftext[x=0.447993in, y=3.654223in, left, base]{\color{textcolor}\sffamily\fontsize{10.000000}{12.000000}\selectfont 0.5}%
\end{pgfscope}%
\begin{pgfscope}%
\pgfsetbuttcap%
\pgfsetroundjoin%
\definecolor{currentfill}{rgb}{0.000000,0.000000,0.000000}%
\pgfsetfillcolor{currentfill}%
\pgfsetlinewidth{0.803000pt}%
\definecolor{currentstroke}{rgb}{0.000000,0.000000,0.000000}%
\pgfsetstrokecolor{currentstroke}%
\pgfsetdash{}{0pt}%
\pgfsys@defobject{currentmarker}{\pgfqpoint{-0.048611in}{0.000000in}}{\pgfqpoint{-0.000000in}{0.000000in}}{%
\pgfpathmoveto{\pgfqpoint{-0.000000in}{0.000000in}}%
\pgfpathlineto{\pgfqpoint{-0.048611in}{0.000000in}}%
\pgfusepath{stroke,fill}%
}%
\begin{pgfscope}%
\pgfsys@transformshift{0.766095in}{4.752111in}%
\pgfsys@useobject{currentmarker}{}%
\end{pgfscope}%
\end{pgfscope}%
\begin{pgfscope}%
\definecolor{textcolor}{rgb}{0.000000,0.000000,0.000000}%
\pgfsetstrokecolor{textcolor}%
\pgfsetfillcolor{textcolor}%
\pgftext[x=0.447993in, y=4.699350in, left, base]{\color{textcolor}\sffamily\fontsize{10.000000}{12.000000}\selectfont 1.0}%
\end{pgfscope}%
\begin{pgfscope}%
\pgfsetbuttcap%
\pgfsetroundjoin%
\definecolor{currentfill}{rgb}{0.000000,0.000000,0.000000}%
\pgfsetfillcolor{currentfill}%
\pgfsetlinewidth{0.803000pt}%
\definecolor{currentstroke}{rgb}{0.000000,0.000000,0.000000}%
\pgfsetstrokecolor{currentstroke}%
\pgfsetdash{}{0pt}%
\pgfsys@defobject{currentmarker}{\pgfqpoint{-0.048611in}{0.000000in}}{\pgfqpoint{-0.000000in}{0.000000in}}{%
\pgfpathmoveto{\pgfqpoint{-0.000000in}{0.000000in}}%
\pgfpathlineto{\pgfqpoint{-0.048611in}{0.000000in}}%
\pgfusepath{stroke,fill}%
}%
\begin{pgfscope}%
\pgfsys@transformshift{0.766095in}{5.797238in}%
\pgfsys@useobject{currentmarker}{}%
\end{pgfscope}%
\end{pgfscope}%
\begin{pgfscope}%
\definecolor{textcolor}{rgb}{0.000000,0.000000,0.000000}%
\pgfsetstrokecolor{textcolor}%
\pgfsetfillcolor{textcolor}%
\pgftext[x=0.447993in, y=5.744477in, left, base]{\color{textcolor}\sffamily\fontsize{10.000000}{12.000000}\selectfont 1.5}%
\end{pgfscope}%
\begin{pgfscope}%
\definecolor{textcolor}{rgb}{0.000000,0.000000,0.000000}%
\pgfsetstrokecolor{textcolor}%
\pgfsetfillcolor{textcolor}%
\pgftext[x=0.284413in,y=3.184421in,,bottom,rotate=90.000000]{\color{textcolor}\sffamily\fontsize{10.000000}{12.000000}\selectfont y}%
\end{pgfscope}%
\begin{pgfscope}%
\pgfpathrectangle{\pgfqpoint{0.766095in}{0.571603in}}{\pgfqpoint{6.973465in}{5.225635in}}%
\pgfusepath{clip}%
\pgfsetbuttcap%
\pgfsetroundjoin%
\pgfsetlinewidth{1.505625pt}%
\definecolor{currentstroke}{rgb}{0.273809,0.031497,0.358853}%
\pgfsetstrokecolor{currentstroke}%
\pgfsetdash{}{0pt}%
\pgfpathmoveto{\pgfqpoint{4.015742in}{0.571603in}}%
\pgfpathlineto{\pgfqpoint{3.884843in}{0.624122in}}%
\pgfpathlineto{\pgfqpoint{3.764089in}{0.676641in}}%
\pgfpathlineto{\pgfqpoint{3.651938in}{0.729160in}}%
\pgfpathlineto{\pgfqpoint{3.547844in}{0.781679in}}%
\pgfpathlineto{\pgfqpoint{3.451176in}{0.834198in}}%
\pgfpathlineto{\pgfqpoint{3.359243in}{0.888000in}}%
\pgfpathlineto{\pgfqpoint{3.278241in}{0.939236in}}%
\pgfpathlineto{\pgfqpoint{3.201262in}{0.991755in}}%
\pgfpathlineto{\pgfqpoint{3.130161in}{1.044274in}}%
\pgfpathlineto{\pgfqpoint{3.064711in}{1.096793in}}%
\pgfpathlineto{\pgfqpoint{3.004622in}{1.149312in}}%
\pgfpathlineto{\pgfqpoint{2.949975in}{1.201831in}}%
\pgfpathlineto{\pgfqpoint{2.900283in}{1.254350in}}%
\pgfpathlineto{\pgfqpoint{2.855790in}{1.306869in}}%
\pgfpathlineto{\pgfqpoint{2.833605in}{1.335302in}}%
\pgfpathlineto{\pgfqpoint{2.797935in}{1.385647in}}%
\pgfpathlineto{\pgfqpoint{2.763520in}{1.441828in}}%
\pgfpathlineto{\pgfqpoint{2.738039in}{1.490685in}}%
\pgfpathlineto{\pgfqpoint{2.725941in}{1.516944in}}%
\pgfpathlineto{\pgfqpoint{2.705717in}{1.569463in}}%
\pgfpathlineto{\pgfqpoint{2.693435in}{1.609808in}}%
\pgfpathlineto{\pgfqpoint{2.690183in}{1.621982in}}%
\pgfpathlineto{\pgfqpoint{2.684450in}{1.648242in}}%
\pgfpathlineto{\pgfqpoint{2.679957in}{1.674501in}}%
\pgfpathlineto{\pgfqpoint{2.676754in}{1.700761in}}%
\pgfpathlineto{\pgfqpoint{2.674891in}{1.727020in}}%
\pgfpathlineto{\pgfqpoint{2.674419in}{1.753280in}}%
\pgfpathlineto{\pgfqpoint{2.675391in}{1.779539in}}%
\pgfpathlineto{\pgfqpoint{2.677859in}{1.805799in}}%
\pgfpathlineto{\pgfqpoint{2.681878in}{1.832058in}}%
\pgfpathlineto{\pgfqpoint{2.687499in}{1.858318in}}%
\pgfpathlineto{\pgfqpoint{2.694859in}{1.884577in}}%
\pgfpathlineto{\pgfqpoint{2.704413in}{1.910836in}}%
\pgfpathlineto{\pgfqpoint{2.715864in}{1.937096in}}%
\pgfpathlineto{\pgfqpoint{2.729330in}{1.963355in}}%
\pgfpathlineto{\pgfqpoint{2.745868in}{1.989615in}}%
\pgfpathlineto{\pgfqpoint{2.764774in}{2.015874in}}%
\pgfpathlineto{\pgfqpoint{2.798562in}{2.053503in}}%
\pgfpathlineto{\pgfqpoint{2.833605in}{2.084711in}}%
\pgfpathlineto{\pgfqpoint{2.868647in}{2.110029in}}%
\pgfpathlineto{\pgfqpoint{2.903690in}{2.130773in}}%
\pgfpathlineto{\pgfqpoint{2.938732in}{2.147836in}}%
\pgfpathlineto{\pgfqpoint{2.973775in}{2.161491in}}%
\pgfpathlineto{\pgfqpoint{3.012563in}{2.173431in}}%
\pgfpathlineto{\pgfqpoint{3.043860in}{2.180840in}}%
\pgfpathlineto{\pgfqpoint{3.078903in}{2.186921in}}%
\pgfpathlineto{\pgfqpoint{3.113945in}{2.190900in}}%
\pgfpathlineto{\pgfqpoint{3.148988in}{2.192899in}}%
\pgfpathlineto{\pgfqpoint{3.184030in}{2.193030in}}%
\pgfpathlineto{\pgfqpoint{3.219073in}{2.191399in}}%
\pgfpathlineto{\pgfqpoint{3.254115in}{2.188102in}}%
\pgfpathlineto{\pgfqpoint{3.289158in}{2.183231in}}%
\pgfpathlineto{\pgfqpoint{3.339566in}{2.173431in}}%
\pgfpathlineto{\pgfqpoint{3.359243in}{2.169035in}}%
\pgfpathlineto{\pgfqpoint{3.394285in}{2.159809in}}%
\pgfpathlineto{\pgfqpoint{3.435689in}{2.147172in}}%
\pgfpathlineto{\pgfqpoint{3.499413in}{2.124574in}}%
\pgfpathlineto{\pgfqpoint{3.570917in}{2.094653in}}%
\pgfpathlineto{\pgfqpoint{3.639583in}{2.061954in}}%
\pgfpathlineto{\pgfqpoint{3.709668in}{2.025015in}}%
\pgfpathlineto{\pgfqpoint{3.779753in}{1.984948in}}%
\pgfpathlineto{\pgfqpoint{3.857741in}{1.937096in}}%
\pgfpathlineto{\pgfqpoint{3.938635in}{1.884577in}}%
\pgfpathlineto{\pgfqpoint{4.016139in}{1.832058in}}%
\pgfpathlineto{\pgfqpoint{4.091160in}{1.779539in}}%
\pgfpathlineto{\pgfqpoint{4.165221in}{1.726399in}}%
\pgfpathlineto{\pgfqpoint{4.306988in}{1.621982in}}%
\pgfpathlineto{\pgfqpoint{4.481437in}{1.490685in}}%
\pgfpathlineto{\pgfqpoint{4.667105in}{1.349211in}}%
\pgfpathlineto{\pgfqpoint{4.667105in}{1.349211in}}%
\pgfusepath{stroke}%
\end{pgfscope}%
\begin{pgfscope}%
\pgfpathrectangle{\pgfqpoint{0.766095in}{0.571603in}}{\pgfqpoint{6.973465in}{5.225635in}}%
\pgfusepath{clip}%
\pgfsetbuttcap%
\pgfsetroundjoin%
\pgfsetlinewidth{1.505625pt}%
\definecolor{currentstroke}{rgb}{0.273809,0.031497,0.358853}%
\pgfsetstrokecolor{currentstroke}%
\pgfsetdash{}{0pt}%
\pgfpathmoveto{\pgfqpoint{4.914423in}{1.159220in}}%
\pgfpathlineto{\pgfqpoint{4.927255in}{1.149312in}}%
\pgfpathlineto{\pgfqpoint{4.936157in}{1.142428in}}%
\pgfpathlineto{\pgfqpoint{4.961208in}{1.123052in}}%
\pgfpathlineto{\pgfqpoint{4.971200in}{1.115305in}}%
\pgfpathlineto{\pgfqpoint{4.995086in}{1.096793in}}%
\pgfpathlineto{\pgfqpoint{5.006242in}{1.088116in}}%
\pgfpathlineto{\pgfqpoint{5.028875in}{1.070533in}}%
\pgfpathlineto{\pgfqpoint{5.041285in}{1.060849in}}%
\pgfpathlineto{\pgfqpoint{5.062563in}{1.044274in}}%
\pgfpathlineto{\pgfqpoint{5.076327in}{1.033492in}}%
\pgfpathlineto{\pgfqpoint{5.096135in}{1.018014in}}%
\pgfpathlineto{\pgfqpoint{5.111370in}{1.006029in}}%
\pgfpathlineto{\pgfqpoint{5.129572in}{0.991755in}}%
\pgfpathlineto{\pgfqpoint{5.146412in}{0.978443in}}%
\pgfpathlineto{\pgfqpoint{5.162855in}{0.965495in}}%
\pgfpathlineto{\pgfqpoint{5.181455in}{0.950713in}}%
\pgfpathlineto{\pgfqpoint{5.195963in}{0.939236in}}%
\pgfpathlineto{\pgfqpoint{5.216497in}{0.922817in}}%
\pgfpathlineto{\pgfqpoint{5.228870in}{0.912976in}}%
\pgfpathlineto{\pgfqpoint{5.251540in}{0.894727in}}%
\pgfpathlineto{\pgfqpoint{5.261550in}{0.886717in}}%
\pgfpathlineto{\pgfqpoint{5.286583in}{0.866411in}}%
\pgfpathlineto{\pgfqpoint{5.293972in}{0.860458in}}%
\pgfpathlineto{\pgfqpoint{5.321625in}{0.837834in}}%
\pgfpathlineto{\pgfqpoint{5.326103in}{0.834198in}}%
\pgfpathlineto{\pgfqpoint{5.356668in}{0.808954in}}%
\pgfpathlineto{\pgfqpoint{5.357907in}{0.807939in}}%
\pgfpathlineto{\pgfqpoint{5.389254in}{0.781679in}}%
\pgfpathlineto{\pgfqpoint{5.391710in}{0.779570in}}%
\pgfpathlineto{\pgfqpoint{5.420124in}{0.755420in}}%
\pgfpathlineto{\pgfqpoint{5.426753in}{0.749642in}}%
\pgfpathlineto{\pgfqpoint{5.450509in}{0.729160in}}%
\pgfpathlineto{\pgfqpoint{5.461795in}{0.719156in}}%
\pgfpathlineto{\pgfqpoint{5.480350in}{0.702901in}}%
\pgfpathlineto{\pgfqpoint{5.496838in}{0.688007in}}%
\pgfpathlineto{\pgfqpoint{5.509581in}{0.676641in}}%
\pgfpathlineto{\pgfqpoint{5.531880in}{0.656067in}}%
\pgfpathlineto{\pgfqpoint{5.538128in}{0.650382in}}%
\pgfpathlineto{\pgfqpoint{5.565866in}{0.624122in}}%
\pgfpathlineto{\pgfqpoint{5.566923in}{0.623062in}}%
\pgfpathlineto{\pgfqpoint{5.592451in}{0.597863in}}%
\pgfpathlineto{\pgfqpoint{5.601965in}{0.587985in}}%
\pgfpathlineto{\pgfqpoint{5.618021in}{0.571603in}}%
\pgfusepath{stroke}%
\end{pgfscope}%
\begin{pgfscope}%
\pgfpathrectangle{\pgfqpoint{0.766095in}{0.571603in}}{\pgfqpoint{6.973465in}{5.225635in}}%
\pgfusepath{clip}%
\pgfsetbuttcap%
\pgfsetroundjoin%
\pgfsetlinewidth{1.505625pt}%
\definecolor{currentstroke}{rgb}{0.278791,0.062145,0.386592}%
\pgfsetstrokecolor{currentstroke}%
\pgfsetdash{}{0pt}%
\pgfpathmoveto{\pgfqpoint{3.440985in}{0.571603in}}%
\pgfpathlineto{\pgfqpoint{3.324200in}{0.627344in}}%
\pgfpathlineto{\pgfqpoint{3.219073in}{0.680666in}}%
\pgfpathlineto{\pgfqpoint{3.128742in}{0.729160in}}%
\pgfpathlineto{\pgfqpoint{3.036406in}{0.781679in}}%
\pgfpathlineto{\pgfqpoint{2.938732in}{0.841063in}}%
\pgfpathlineto{\pgfqpoint{2.867903in}{0.886717in}}%
\pgfpathlineto{\pgfqpoint{2.791407in}{0.939236in}}%
\pgfpathlineto{\pgfqpoint{2.719704in}{0.991755in}}%
\pgfpathlineto{\pgfqpoint{2.652639in}{1.044274in}}%
\pgfpathlineto{\pgfqpoint{2.588307in}{1.098348in}}%
\pgfpathlineto{\pgfqpoint{2.531999in}{1.149312in}}%
\pgfpathlineto{\pgfqpoint{2.478014in}{1.201831in}}%
\pgfpathlineto{\pgfqpoint{2.428248in}{1.254350in}}%
\pgfpathlineto{\pgfqpoint{2.382348in}{1.306869in}}%
\pgfpathlineto{\pgfqpoint{2.340383in}{1.359388in}}%
\pgfpathlineto{\pgfqpoint{2.302262in}{1.411906in}}%
\pgfpathlineto{\pgfqpoint{2.267809in}{1.464425in}}%
\pgfpathlineto{\pgfqpoint{2.236943in}{1.516944in}}%
\pgfpathlineto{\pgfqpoint{2.209761in}{1.569463in}}%
\pgfpathlineto{\pgfqpoint{2.186070in}{1.621982in}}%
\pgfpathlineto{\pgfqpoint{2.165694in}{1.674501in}}%
\pgfpathlineto{\pgfqpoint{2.148953in}{1.727020in}}%
\pgfpathlineto{\pgfqpoint{2.135418in}{1.779539in}}%
\pgfpathlineto{\pgfqpoint{2.125428in}{1.832058in}}%
\pgfpathlineto{\pgfqpoint{2.118825in}{1.884577in}}%
\pgfpathlineto{\pgfqpoint{2.115601in}{1.937096in}}%
\pgfpathlineto{\pgfqpoint{2.115817in}{1.989615in}}%
\pgfpathlineto{\pgfqpoint{2.119525in}{2.042134in}}%
\pgfpathlineto{\pgfqpoint{2.126756in}{2.094653in}}%
\pgfpathlineto{\pgfqpoint{2.137704in}{2.147172in}}%
\pgfpathlineto{\pgfqpoint{2.152552in}{2.199691in}}%
\pgfpathlineto{\pgfqpoint{2.171190in}{2.252210in}}%
\pgfpathlineto{\pgfqpoint{2.194200in}{2.304729in}}%
\pgfpathlineto{\pgfqpoint{2.207277in}{2.330988in}}%
\pgfpathlineto{\pgfqpoint{2.237882in}{2.384919in}}%
\pgfpathlineto{\pgfqpoint{2.272924in}{2.437462in}}%
\pgfpathlineto{\pgfqpoint{2.312508in}{2.488545in}}%
\pgfpathlineto{\pgfqpoint{2.343009in}{2.523509in}}%
\pgfpathlineto{\pgfqpoint{2.385815in}{2.567323in}}%
\pgfpathlineto{\pgfqpoint{2.414010in}{2.593583in}}%
\pgfpathlineto{\pgfqpoint{2.448137in}{2.622684in}}%
\pgfpathlineto{\pgfqpoint{2.483179in}{2.650040in}}%
\pgfpathlineto{\pgfqpoint{2.518222in}{2.675133in}}%
\pgfpathlineto{\pgfqpoint{2.553995in}{2.698621in}}%
\pgfpathlineto{\pgfqpoint{2.598365in}{2.724880in}}%
\pgfpathlineto{\pgfqpoint{2.647996in}{2.751140in}}%
\pgfpathlineto{\pgfqpoint{2.658392in}{2.756334in}}%
\pgfpathlineto{\pgfqpoint{2.705012in}{2.777399in}}%
\pgfpathlineto{\pgfqpoint{2.728477in}{2.787161in}}%
\pgfpathlineto{\pgfqpoint{2.772900in}{2.803659in}}%
\pgfpathlineto{\pgfqpoint{2.798562in}{2.812303in}}%
\pgfpathlineto{\pgfqpoint{2.860281in}{2.829918in}}%
\pgfpathlineto{\pgfqpoint{2.868647in}{2.832105in}}%
\pgfpathlineto{\pgfqpoint{2.903690in}{2.840007in}}%
\pgfpathlineto{\pgfqpoint{2.938732in}{2.846670in}}%
\pgfpathlineto{\pgfqpoint{2.973775in}{2.852083in}}%
\pgfpathlineto{\pgfqpoint{3.008818in}{2.856235in}}%
\pgfpathlineto{\pgfqpoint{3.043860in}{2.859075in}}%
\pgfpathlineto{\pgfqpoint{3.078903in}{2.860653in}}%
\pgfpathlineto{\pgfqpoint{3.113945in}{2.860956in}}%
\pgfpathlineto{\pgfqpoint{3.148988in}{2.859969in}}%
\pgfpathlineto{\pgfqpoint{3.198625in}{2.856177in}}%
\pgfpathlineto{\pgfqpoint{3.219073in}{2.854041in}}%
\pgfpathlineto{\pgfqpoint{3.254115in}{2.849041in}}%
\pgfpathlineto{\pgfqpoint{3.289158in}{2.842683in}}%
\pgfpathlineto{\pgfqpoint{3.343591in}{2.829918in}}%
\pgfpathlineto{\pgfqpoint{3.359243in}{2.825800in}}%
\pgfpathlineto{\pgfqpoint{3.394285in}{2.815202in}}%
\pgfpathlineto{\pgfqpoint{3.429328in}{2.803190in}}%
\pgfpathlineto{\pgfqpoint{3.493192in}{2.777399in}}%
\pgfpathlineto{\pgfqpoint{3.499413in}{2.774727in}}%
\pgfpathlineto{\pgfqpoint{3.548508in}{2.751140in}}%
\pgfpathlineto{\pgfqpoint{3.597638in}{2.724880in}}%
\pgfpathlineto{\pgfqpoint{3.604541in}{2.721059in}}%
\pgfpathlineto{\pgfqpoint{3.642236in}{2.698621in}}%
\pgfpathlineto{\pgfqpoint{3.683249in}{2.672361in}}%
\pgfpathlineto{\pgfqpoint{3.744711in}{2.629876in}}%
\pgfpathlineto{\pgfqpoint{3.814796in}{2.576755in}}%
\pgfpathlineto{\pgfqpoint{3.890232in}{2.514804in}}%
\pgfpathlineto{\pgfqpoint{3.954966in}{2.458639in}}%
\pgfpathlineto{\pgfqpoint{4.060094in}{2.362727in}}%
\pgfpathlineto{\pgfqpoint{4.122432in}{2.303966in}}%
\pgfpathlineto{\pgfqpoint{4.122432in}{2.303966in}}%
\pgfusepath{stroke}%
\end{pgfscope}%
\begin{pgfscope}%
\pgfpathrectangle{\pgfqpoint{0.766095in}{0.571603in}}{\pgfqpoint{6.973465in}{5.225635in}}%
\pgfusepath{clip}%
\pgfsetbuttcap%
\pgfsetroundjoin%
\pgfsetlinewidth{1.505625pt}%
\definecolor{currentstroke}{rgb}{0.278791,0.062145,0.386592}%
\pgfsetstrokecolor{currentstroke}%
\pgfsetdash{}{0pt}%
\pgfpathmoveto{\pgfqpoint{4.347969in}{2.088304in}}%
\pgfpathlineto{\pgfqpoint{4.368798in}{2.068393in}}%
\pgfpathlineto{\pgfqpoint{4.375477in}{2.062120in}}%
\pgfpathlineto{\pgfqpoint{4.396379in}{2.042134in}}%
\pgfpathlineto{\pgfqpoint{4.410519in}{2.028855in}}%
\pgfpathlineto{\pgfqpoint{4.424109in}{2.015874in}}%
\pgfpathlineto{\pgfqpoint{4.445562in}{1.995749in}}%
\pgfpathlineto{\pgfqpoint{4.451996in}{1.989615in}}%
\pgfpathlineto{\pgfqpoint{4.480034in}{1.963355in}}%
\pgfpathlineto{\pgfqpoint{4.480604in}{1.962829in}}%
\pgfpathlineto{\pgfqpoint{4.508078in}{1.937096in}}%
\pgfpathlineto{\pgfqpoint{4.515647in}{1.930131in}}%
\pgfpathlineto{\pgfqpoint{4.536314in}{1.910836in}}%
\pgfpathlineto{\pgfqpoint{4.550689in}{1.897649in}}%
\pgfpathlineto{\pgfqpoint{4.564747in}{1.884577in}}%
\pgfpathlineto{\pgfqpoint{4.585732in}{1.865399in}}%
\pgfpathlineto{\pgfqpoint{4.593380in}{1.858318in}}%
\pgfpathlineto{\pgfqpoint{4.620774in}{1.833390in}}%
\pgfpathlineto{\pgfqpoint{4.622220in}{1.832058in}}%
\pgfpathlineto{\pgfqpoint{4.651151in}{1.805799in}}%
\pgfpathlineto{\pgfqpoint{4.655817in}{1.801631in}}%
\pgfpathlineto{\pgfqpoint{4.680267in}{1.779539in}}%
\pgfpathlineto{\pgfqpoint{4.690859in}{1.770124in}}%
\pgfpathlineto{\pgfqpoint{4.709606in}{1.753280in}}%
\pgfpathlineto{\pgfqpoint{4.725902in}{1.738871in}}%
\pgfpathlineto{\pgfqpoint{4.739170in}{1.727020in}}%
\pgfpathlineto{\pgfqpoint{4.760944in}{1.707875in}}%
\pgfpathlineto{\pgfqpoint{4.768959in}{1.700761in}}%
\pgfpathlineto{\pgfqpoint{4.795987in}{1.677137in}}%
\pgfpathlineto{\pgfqpoint{4.798976in}{1.674501in}}%
\pgfpathlineto{\pgfqpoint{4.829176in}{1.648242in}}%
\pgfpathlineto{\pgfqpoint{4.831030in}{1.646651in}}%
\pgfpathlineto{\pgfqpoint{4.859539in}{1.621982in}}%
\pgfpathlineto{\pgfqpoint{4.866072in}{1.616409in}}%
\pgfpathlineto{\pgfqpoint{4.890137in}{1.595723in}}%
\pgfpathlineto{\pgfqpoint{4.901115in}{1.586416in}}%
\pgfpathlineto{\pgfqpoint{4.920969in}{1.569463in}}%
\pgfpathlineto{\pgfqpoint{4.936157in}{1.556669in}}%
\pgfpathlineto{\pgfqpoint{4.952036in}{1.543204in}}%
\pgfpathlineto{\pgfqpoint{4.971200in}{1.527163in}}%
\pgfpathlineto{\pgfqpoint{4.983335in}{1.516944in}}%
\pgfpathlineto{\pgfqpoint{5.006242in}{1.497895in}}%
\pgfpathlineto{\pgfqpoint{5.014865in}{1.490685in}}%
\pgfpathlineto{\pgfqpoint{5.041285in}{1.468859in}}%
\pgfpathlineto{\pgfqpoint{5.046624in}{1.464425in}}%
\pgfpathlineto{\pgfqpoint{5.076327in}{1.440049in}}%
\pgfpathlineto{\pgfqpoint{5.078612in}{1.438166in}}%
\pgfpathlineto{\pgfqpoint{5.110812in}{1.411906in}}%
\pgfpathlineto{\pgfqpoint{5.111370in}{1.411455in}}%
\pgfpathlineto{\pgfqpoint{5.143188in}{1.385647in}}%
\pgfpathlineto{\pgfqpoint{5.146412in}{1.383058in}}%
\pgfpathlineto{\pgfqpoint{5.175790in}{1.359388in}}%
\pgfpathlineto{\pgfqpoint{5.181455in}{1.354867in}}%
\pgfpathlineto{\pgfqpoint{5.208616in}{1.333128in}}%
\pgfpathlineto{\pgfqpoint{5.216497in}{1.326876in}}%
\pgfpathlineto{\pgfqpoint{5.241661in}{1.306869in}}%
\pgfpathlineto{\pgfqpoint{5.251540in}{1.299080in}}%
\pgfpathlineto{\pgfqpoint{5.274922in}{1.280609in}}%
\pgfpathlineto{\pgfqpoint{5.286583in}{1.271470in}}%
\pgfpathlineto{\pgfqpoint{5.308394in}{1.254350in}}%
\pgfpathlineto{\pgfqpoint{5.321625in}{1.244041in}}%
\pgfpathlineto{\pgfqpoint{5.342075in}{1.228090in}}%
\pgfpathlineto{\pgfqpoint{5.356668in}{1.216785in}}%
\pgfpathlineto{\pgfqpoint{5.375958in}{1.201831in}}%
\pgfpathlineto{\pgfqpoint{5.391710in}{1.189695in}}%
\pgfpathlineto{\pgfqpoint{5.410039in}{1.175571in}}%
\pgfpathlineto{\pgfqpoint{5.426753in}{1.162764in}}%
\pgfpathlineto{\pgfqpoint{5.444313in}{1.149312in}}%
\pgfpathlineto{\pgfqpoint{5.461795in}{1.135985in}}%
\pgfpathlineto{\pgfqpoint{5.478774in}{1.123052in}}%
\pgfpathlineto{\pgfqpoint{5.496838in}{1.109352in}}%
\pgfpathlineto{\pgfqpoint{5.513415in}{1.096793in}}%
\pgfpathlineto{\pgfqpoint{5.531880in}{1.082855in}}%
\pgfpathlineto{\pgfqpoint{5.548231in}{1.070533in}}%
\pgfpathlineto{\pgfqpoint{5.566923in}{1.056489in}}%
\pgfpathlineto{\pgfqpoint{5.583213in}{1.044274in}}%
\pgfpathlineto{\pgfqpoint{5.601965in}{1.030246in}}%
\pgfpathlineto{\pgfqpoint{5.618356in}{1.018014in}}%
\pgfpathlineto{\pgfqpoint{5.637008in}{1.004117in}}%
\pgfpathlineto{\pgfqpoint{5.653649in}{0.991755in}}%
\pgfpathlineto{\pgfqpoint{5.672050in}{0.978097in}}%
\pgfpathlineto{\pgfqpoint{5.689085in}{0.965495in}}%
\pgfpathlineto{\pgfqpoint{5.707093in}{0.952176in}}%
\pgfpathlineto{\pgfqpoint{5.724655in}{0.939236in}}%
\pgfpathlineto{\pgfqpoint{5.742136in}{0.926347in}}%
\pgfpathlineto{\pgfqpoint{5.760347in}{0.912976in}}%
\pgfpathlineto{\pgfqpoint{5.777178in}{0.900602in}}%
\pgfpathlineto{\pgfqpoint{5.796152in}{0.886717in}}%
\pgfpathlineto{\pgfqpoint{5.812221in}{0.874931in}}%
\pgfpathlineto{\pgfqpoint{5.832057in}{0.860458in}}%
\pgfpathlineto{\pgfqpoint{5.847263in}{0.849327in}}%
\pgfpathlineto{\pgfqpoint{5.868051in}{0.834198in}}%
\pgfpathlineto{\pgfqpoint{5.882306in}{0.823781in}}%
\pgfpathlineto{\pgfqpoint{5.904118in}{0.807939in}}%
\pgfpathlineto{\pgfqpoint{5.917348in}{0.798282in}}%
\pgfpathlineto{\pgfqpoint{5.940246in}{0.781679in}}%
\pgfpathlineto{\pgfqpoint{5.952391in}{0.772820in}}%
\pgfpathlineto{\pgfqpoint{5.976417in}{0.755420in}}%
\pgfpathlineto{\pgfqpoint{5.987433in}{0.747386in}}%
\pgfpathlineto{\pgfqpoint{6.012616in}{0.729160in}}%
\pgfpathlineto{\pgfqpoint{6.022476in}{0.721966in}}%
\pgfpathlineto{\pgfqpoint{6.048823in}{0.702901in}}%
\pgfpathlineto{\pgfqpoint{6.057518in}{0.696550in}}%
\pgfpathlineto{\pgfqpoint{6.085018in}{0.676641in}}%
\pgfpathlineto{\pgfqpoint{6.092561in}{0.671124in}}%
\pgfpathlineto{\pgfqpoint{6.121180in}{0.650382in}}%
\pgfpathlineto{\pgfqpoint{6.127603in}{0.645672in}}%
\pgfpathlineto{\pgfqpoint{6.157285in}{0.624122in}}%
\pgfpathlineto{\pgfqpoint{6.162646in}{0.620179in}}%
\pgfpathlineto{\pgfqpoint{6.193307in}{0.597863in}}%
\pgfpathlineto{\pgfqpoint{6.197689in}{0.594628in}}%
\pgfpathlineto{\pgfqpoint{6.229219in}{0.571603in}}%
\pgfusepath{stroke}%
\end{pgfscope}%
\begin{pgfscope}%
\pgfpathrectangle{\pgfqpoint{0.766095in}{0.571603in}}{\pgfqpoint{6.973465in}{5.225635in}}%
\pgfusepath{clip}%
\pgfsetbuttcap%
\pgfsetroundjoin%
\pgfsetlinewidth{1.505625pt}%
\definecolor{currentstroke}{rgb}{0.281924,0.089666,0.412415}%
\pgfsetstrokecolor{currentstroke}%
\pgfsetdash{}{0pt}%
\pgfpathmoveto{\pgfqpoint{3.053770in}{0.571603in}}%
\pgfpathlineto{\pgfqpoint{2.938732in}{0.630724in}}%
\pgfpathlineto{\pgfqpoint{2.853891in}{0.676641in}}%
\pgfpathlineto{\pgfqpoint{2.761741in}{0.729160in}}%
\pgfpathlineto{\pgfqpoint{2.658392in}{0.791948in}}%
\pgfpathlineto{\pgfqpoint{2.588307in}{0.836918in}}%
\pgfpathlineto{\pgfqpoint{2.514756in}{0.886717in}}%
\pgfpathlineto{\pgfqpoint{2.441619in}{0.939236in}}%
\pgfpathlineto{\pgfqpoint{2.372775in}{0.991755in}}%
\pgfpathlineto{\pgfqpoint{2.307967in}{1.044366in}}%
\pgfpathlineto{\pgfqpoint{2.237882in}{1.105570in}}%
\pgfpathlineto{\pgfqpoint{2.190879in}{1.149312in}}%
\pgfpathlineto{\pgfqpoint{2.132754in}{1.207264in}}%
\pgfpathlineto{\pgfqpoint{2.088794in}{1.254350in}}%
\pgfpathlineto{\pgfqpoint{2.043209in}{1.306869in}}%
\pgfpathlineto{\pgfqpoint{2.001060in}{1.359388in}}%
\pgfpathlineto{\pgfqpoint{1.962301in}{1.411906in}}%
\pgfpathlineto{\pgfqpoint{1.926874in}{1.464425in}}%
\pgfpathlineto{\pgfqpoint{1.894700in}{1.516944in}}%
\pgfpathlineto{\pgfqpoint{1.865689in}{1.569463in}}%
\pgfpathlineto{\pgfqpoint{1.839737in}{1.621982in}}%
\pgfpathlineto{\pgfqpoint{1.816723in}{1.674501in}}%
\pgfpathlineto{\pgfqpoint{1.796877in}{1.727020in}}%
\pgfpathlineto{\pgfqpoint{1.779786in}{1.779539in}}%
\pgfpathlineto{\pgfqpoint{1.765746in}{1.832058in}}%
\pgfpathlineto{\pgfqpoint{1.754444in}{1.884577in}}%
\pgfpathlineto{\pgfqpoint{1.745959in}{1.937096in}}%
\pgfpathlineto{\pgfqpoint{1.740403in}{1.989615in}}%
\pgfpathlineto{\pgfqpoint{1.737619in}{2.042134in}}%
\pgfpathlineto{\pgfqpoint{1.737622in}{2.094653in}}%
\pgfpathlineto{\pgfqpoint{1.740419in}{2.147172in}}%
\pgfpathlineto{\pgfqpoint{1.746008in}{2.199691in}}%
\pgfpathlineto{\pgfqpoint{1.754582in}{2.252210in}}%
\pgfpathlineto{\pgfqpoint{1.766036in}{2.304729in}}%
\pgfpathlineto{\pgfqpoint{1.782329in}{2.363769in}}%
\pgfpathlineto{\pgfqpoint{1.797793in}{2.409766in}}%
\pgfpathlineto{\pgfqpoint{1.818172in}{2.462285in}}%
\pgfpathlineto{\pgfqpoint{1.841983in}{2.514804in}}%
\pgfpathlineto{\pgfqpoint{1.869057in}{2.567323in}}%
\pgfpathlineto{\pgfqpoint{1.899591in}{2.619842in}}%
\pgfpathlineto{\pgfqpoint{1.933816in}{2.672361in}}%
\pgfpathlineto{\pgfqpoint{1.971971in}{2.724880in}}%
\pgfpathlineto{\pgfqpoint{2.014299in}{2.777399in}}%
\pgfpathlineto{\pgfqpoint{2.037117in}{2.803659in}}%
\pgfpathlineto{\pgfqpoint{2.086424in}{2.856177in}}%
\pgfpathlineto{\pgfqpoint{2.113015in}{2.882437in}}%
\pgfpathlineto{\pgfqpoint{2.140920in}{2.908696in}}%
\pgfpathlineto{\pgfqpoint{2.170254in}{2.934956in}}%
\pgfpathlineto{\pgfqpoint{2.202839in}{2.962569in}}%
\pgfpathlineto{\pgfqpoint{2.268347in}{3.013734in}}%
\pgfpathlineto{\pgfqpoint{2.307967in}{3.042280in}}%
\pgfpathlineto{\pgfqpoint{2.378052in}{3.088674in}}%
\pgfpathlineto{\pgfqpoint{2.427926in}{3.118772in}}%
\pgfpathlineto{\pgfqpoint{2.483179in}{3.149778in}}%
\pgfpathlineto{\pgfqpoint{2.553264in}{3.185458in}}%
\pgfpathlineto{\pgfqpoint{2.623350in}{3.217528in}}%
\pgfpathlineto{\pgfqpoint{2.693435in}{3.246149in}}%
\pgfpathlineto{\pgfqpoint{2.763520in}{3.271457in}}%
\pgfpathlineto{\pgfqpoint{2.833605in}{3.293558in}}%
\pgfpathlineto{\pgfqpoint{2.903690in}{3.312539in}}%
\pgfpathlineto{\pgfqpoint{2.975822in}{3.328848in}}%
\pgfpathlineto{\pgfqpoint{3.043860in}{3.341062in}}%
\pgfpathlineto{\pgfqpoint{3.113945in}{3.350515in}}%
\pgfpathlineto{\pgfqpoint{3.184030in}{3.356575in}}%
\pgfpathlineto{\pgfqpoint{3.254115in}{3.359018in}}%
\pgfpathlineto{\pgfqpoint{3.324200in}{3.357679in}}%
\pgfpathlineto{\pgfqpoint{3.363548in}{3.355107in}}%
\pgfpathlineto{\pgfqpoint{3.394285in}{3.352204in}}%
\pgfpathlineto{\pgfqpoint{3.429328in}{3.347776in}}%
\pgfpathlineto{\pgfqpoint{3.464371in}{3.342180in}}%
\pgfpathlineto{\pgfqpoint{3.527765in}{3.328848in}}%
\pgfpathlineto{\pgfqpoint{3.534456in}{3.327257in}}%
\pgfpathlineto{\pgfqpoint{3.569498in}{3.317683in}}%
\pgfpathlineto{\pgfqpoint{3.616273in}{3.302589in}}%
\pgfpathlineto{\pgfqpoint{3.639583in}{3.294136in}}%
\pgfpathlineto{\pgfqpoint{3.682766in}{3.276329in}}%
\pgfpathlineto{\pgfqpoint{3.709668in}{3.263981in}}%
\pgfpathlineto{\pgfqpoint{3.744711in}{3.246211in}}%
\pgfpathlineto{\pgfqpoint{3.784141in}{3.223810in}}%
\pgfpathlineto{\pgfqpoint{3.825419in}{3.197551in}}%
\pgfpathlineto{\pgfqpoint{3.862719in}{3.171291in}}%
\pgfpathlineto{\pgfqpoint{3.896886in}{3.145032in}}%
\pgfpathlineto{\pgfqpoint{3.928559in}{3.118772in}}%
\pgfpathlineto{\pgfqpoint{3.958226in}{3.092513in}}%
\pgfpathlineto{\pgfqpoint{4.012525in}{3.039994in}}%
\pgfpathlineto{\pgfqpoint{4.037843in}{3.013734in}}%
\pgfpathlineto{\pgfqpoint{4.085669in}{2.961215in}}%
\pgfpathlineto{\pgfqpoint{4.108416in}{2.934956in}}%
\pgfpathlineto{\pgfqpoint{4.165221in}{2.866628in}}%
\pgfpathlineto{\pgfqpoint{4.240719in}{2.770454in}}%
\pgfpathlineto{\pgfqpoint{4.240719in}{2.770454in}}%
\pgfusepath{stroke}%
\end{pgfscope}%
\begin{pgfscope}%
\pgfpathrectangle{\pgfqpoint{0.766095in}{0.571603in}}{\pgfqpoint{6.973465in}{5.225635in}}%
\pgfusepath{clip}%
\pgfsetbuttcap%
\pgfsetroundjoin%
\pgfsetlinewidth{1.505625pt}%
\definecolor{currentstroke}{rgb}{0.281924,0.089666,0.412415}%
\pgfsetstrokecolor{currentstroke}%
\pgfsetdash{}{0pt}%
\pgfpathmoveto{\pgfqpoint{4.429569in}{2.521687in}}%
\pgfpathlineto{\pgfqpoint{4.495801in}{2.436026in}}%
\pgfpathlineto{\pgfqpoint{4.600873in}{2.304729in}}%
\pgfpathlineto{\pgfqpoint{4.690859in}{2.197469in}}%
\pgfpathlineto{\pgfqpoint{4.760944in}{2.117148in}}%
\pgfpathlineto{\pgfqpoint{4.831030in}{2.039621in}}%
\pgfpathlineto{\pgfqpoint{4.902431in}{1.963355in}}%
\pgfpathlineto{\pgfqpoint{4.978840in}{1.884577in}}%
\pgfpathlineto{\pgfqpoint{5.076327in}{1.788132in}}%
\pgfpathlineto{\pgfqpoint{5.146412in}{1.721227in}}%
\pgfpathlineto{\pgfqpoint{5.225181in}{1.648242in}}%
\pgfpathlineto{\pgfqpoint{5.321625in}{1.561980in}}%
\pgfpathlineto{\pgfqpoint{5.403838in}{1.490685in}}%
\pgfpathlineto{\pgfqpoint{5.497557in}{1.411906in}}%
\pgfpathlineto{\pgfqpoint{5.601965in}{1.326752in}}%
\pgfpathlineto{\pgfqpoint{5.707093in}{1.243564in}}%
\pgfpathlineto{\pgfqpoint{5.829803in}{1.149312in}}%
\pgfpathlineto{\pgfqpoint{5.970946in}{1.044274in}}%
\pgfpathlineto{\pgfqpoint{6.116390in}{0.939236in}}%
\pgfpathlineto{\pgfqpoint{6.267774in}{0.832821in}}%
\pgfpathlineto{\pgfqpoint{6.457214in}{0.702901in}}%
\pgfpathlineto{\pgfqpoint{6.618199in}{0.594720in}}%
\pgfpathlineto{\pgfqpoint{6.652946in}{0.571603in}}%
\pgfpathlineto{\pgfqpoint{6.652946in}{0.571603in}}%
\pgfusepath{stroke}%
\end{pgfscope}%
\begin{pgfscope}%
\pgfpathrectangle{\pgfqpoint{0.766095in}{0.571603in}}{\pgfqpoint{6.973465in}{5.225635in}}%
\pgfusepath{clip}%
\pgfsetbuttcap%
\pgfsetroundjoin%
\pgfsetlinewidth{1.505625pt}%
\definecolor{currentstroke}{rgb}{0.283197,0.115680,0.436115}%
\pgfsetstrokecolor{currentstroke}%
\pgfsetdash{}{0pt}%
\pgfpathmoveto{\pgfqpoint{2.744085in}{0.571603in}}%
\pgfpathlineto{\pgfqpoint{2.646168in}{0.624122in}}%
\pgfpathlineto{\pgfqpoint{2.553111in}{0.676641in}}%
\pgfpathlineto{\pgfqpoint{2.448137in}{0.739578in}}%
\pgfpathlineto{\pgfqpoint{2.378052in}{0.783791in}}%
\pgfpathlineto{\pgfqpoint{2.302129in}{0.834198in}}%
\pgfpathlineto{\pgfqpoint{2.227268in}{0.886717in}}%
\pgfpathlineto{\pgfqpoint{2.156539in}{0.939236in}}%
\pgfpathlineto{\pgfqpoint{2.089813in}{0.991755in}}%
\pgfpathlineto{\pgfqpoint{2.026938in}{1.044274in}}%
\pgfpathlineto{\pgfqpoint{1.957541in}{1.106471in}}%
\pgfpathlineto{\pgfqpoint{1.912551in}{1.149312in}}%
\pgfpathlineto{\pgfqpoint{1.852414in}{1.210611in}}%
\pgfpathlineto{\pgfqpoint{1.812244in}{1.254350in}}%
\pgfpathlineto{\pgfqpoint{1.767204in}{1.306869in}}%
\pgfpathlineto{\pgfqpoint{1.725360in}{1.359388in}}%
\pgfpathlineto{\pgfqpoint{1.686662in}{1.411906in}}%
\pgfpathlineto{\pgfqpoint{1.651048in}{1.464425in}}%
\pgfpathlineto{\pgfqpoint{1.618444in}{1.516944in}}%
\pgfpathlineto{\pgfqpoint{1.588764in}{1.569463in}}%
\pgfpathlineto{\pgfqpoint{1.561913in}{1.621982in}}%
\pgfpathlineto{\pgfqpoint{1.537031in}{1.676338in}}%
\pgfpathlineto{\pgfqpoint{1.516590in}{1.727020in}}%
\pgfpathlineto{\pgfqpoint{1.497927in}{1.779539in}}%
\pgfpathlineto{\pgfqpoint{1.482018in}{1.832058in}}%
\pgfpathlineto{\pgfqpoint{1.466946in}{1.892029in}}%
\pgfpathlineto{\pgfqpoint{1.457848in}{1.937096in}}%
\pgfpathlineto{\pgfqpoint{1.449625in}{1.989615in}}%
\pgfpathlineto{\pgfqpoint{1.443887in}{2.042134in}}%
\pgfpathlineto{\pgfqpoint{1.440648in}{2.094653in}}%
\pgfpathlineto{\pgfqpoint{1.439911in}{2.147172in}}%
\pgfpathlineto{\pgfqpoint{1.441678in}{2.199691in}}%
\pgfpathlineto{\pgfqpoint{1.445943in}{2.252210in}}%
\pgfpathlineto{\pgfqpoint{1.452694in}{2.304729in}}%
\pgfpathlineto{\pgfqpoint{1.461913in}{2.357248in}}%
\pgfpathlineto{\pgfqpoint{1.473741in}{2.409766in}}%
\pgfpathlineto{\pgfqpoint{1.488170in}{2.462285in}}%
\pgfpathlineto{\pgfqpoint{1.505116in}{2.514804in}}%
\pgfpathlineto{\pgfqpoint{1.524874in}{2.567323in}}%
\pgfpathlineto{\pgfqpoint{1.547300in}{2.619842in}}%
\pgfpathlineto{\pgfqpoint{1.572464in}{2.672361in}}%
\pgfpathlineto{\pgfqpoint{1.607116in}{2.736122in}}%
\pgfpathlineto{\pgfqpoint{1.631909in}{2.777399in}}%
\pgfpathlineto{\pgfqpoint{1.666245in}{2.829918in}}%
\pgfpathlineto{\pgfqpoint{1.703854in}{2.882437in}}%
\pgfpathlineto{\pgfqpoint{1.744885in}{2.934956in}}%
\pgfpathlineto{\pgfqpoint{1.782329in}{2.979143in}}%
\pgfpathlineto{\pgfqpoint{1.817371in}{3.017921in}}%
\pgfpathlineto{\pgfqpoint{1.864262in}{3.066253in}}%
\pgfpathlineto{\pgfqpoint{1.919244in}{3.118772in}}%
\pgfpathlineto{\pgfqpoint{1.957541in}{3.152883in}}%
\pgfpathlineto{\pgfqpoint{2.010897in}{3.197551in}}%
\pgfpathlineto{\pgfqpoint{2.062669in}{3.238121in}}%
\pgfpathlineto{\pgfqpoint{2.114699in}{3.276329in}}%
\pgfpathlineto{\pgfqpoint{2.167797in}{3.313023in}}%
\pgfpathlineto{\pgfqpoint{2.237882in}{3.358194in}}%
\pgfpathlineto{\pgfqpoint{2.307967in}{3.399915in}}%
\pgfpathlineto{\pgfqpoint{2.378052in}{3.438632in}}%
\pgfpathlineto{\pgfqpoint{2.448137in}{3.474488in}}%
\pgfpathlineto{\pgfqpoint{2.528978in}{3.512664in}}%
\pgfpathlineto{\pgfqpoint{2.588882in}{3.538924in}}%
\pgfpathlineto{\pgfqpoint{2.658392in}{3.567175in}}%
\pgfpathlineto{\pgfqpoint{2.728477in}{3.593468in}}%
\pgfpathlineto{\pgfqpoint{2.798753in}{3.617702in}}%
\pgfpathlineto{\pgfqpoint{2.883278in}{3.643962in}}%
\pgfpathlineto{\pgfqpoint{2.938732in}{3.659615in}}%
\pgfpathlineto{\pgfqpoint{3.008818in}{3.677583in}}%
\pgfpathlineto{\pgfqpoint{3.093271in}{3.696481in}}%
\pgfpathlineto{\pgfqpoint{3.148988in}{3.707293in}}%
\pgfpathlineto{\pgfqpoint{3.219073in}{3.719029in}}%
\pgfpathlineto{\pgfqpoint{3.289158in}{3.728497in}}%
\pgfpathlineto{\pgfqpoint{3.359243in}{3.735647in}}%
\pgfpathlineto{\pgfqpoint{3.429328in}{3.740391in}}%
\pgfpathlineto{\pgfqpoint{3.499413in}{3.742514in}}%
\pgfpathlineto{\pgfqpoint{3.569498in}{3.741772in}}%
\pgfpathlineto{\pgfqpoint{3.639583in}{3.737885in}}%
\pgfpathlineto{\pgfqpoint{3.709668in}{3.730528in}}%
\pgfpathlineto{\pgfqpoint{3.779753in}{3.719231in}}%
\pgfpathlineto{\pgfqpoint{3.814796in}{3.711850in}}%
\pgfpathlineto{\pgfqpoint{3.874458in}{3.696481in}}%
\pgfpathlineto{\pgfqpoint{3.884881in}{3.693448in}}%
\pgfpathlineto{\pgfqpoint{3.919924in}{3.682054in}}%
\pgfpathlineto{\pgfqpoint{3.954966in}{3.669207in}}%
\pgfpathlineto{\pgfqpoint{4.012661in}{3.643962in}}%
\pgfpathlineto{\pgfqpoint{4.025051in}{3.637925in}}%
\pgfpathlineto{\pgfqpoint{4.062832in}{3.617702in}}%
\pgfpathlineto{\pgfqpoint{4.105492in}{3.591443in}}%
\pgfpathlineto{\pgfqpoint{4.142922in}{3.565183in}}%
\pgfpathlineto{\pgfqpoint{4.176268in}{3.538924in}}%
\pgfpathlineto{\pgfqpoint{4.206380in}{3.512664in}}%
\pgfpathlineto{\pgfqpoint{4.235306in}{3.484957in}}%
\pgfpathlineto{\pgfqpoint{4.270349in}{3.447661in}}%
\pgfpathlineto{\pgfqpoint{4.305391in}{3.406202in}}%
\pgfpathlineto{\pgfqpoint{4.344069in}{3.355107in}}%
\pgfpathlineto{\pgfqpoint{4.380041in}{3.302589in}}%
\pgfpathlineto{\pgfqpoint{4.413187in}{3.250070in}}%
\pgfpathlineto{\pgfqpoint{4.459068in}{3.171291in}}%
\pgfpathlineto{\pgfqpoint{4.502193in}{3.092513in}}%
\pgfpathlineto{\pgfqpoint{4.557615in}{2.987475in}}%
\pgfpathlineto{\pgfqpoint{4.640572in}{2.829918in}}%
\pgfpathlineto{\pgfqpoint{4.712878in}{2.698621in}}%
\pgfpathlineto{\pgfqpoint{4.774257in}{2.593583in}}%
\pgfpathlineto{\pgfqpoint{4.831030in}{2.502124in}}%
\pgfpathlineto{\pgfqpoint{4.873961in}{2.436026in}}%
\pgfpathlineto{\pgfqpoint{4.927727in}{2.357248in}}%
\pgfpathlineto{\pgfqpoint{4.984333in}{2.278469in}}%
\pgfpathlineto{\pgfqpoint{5.043995in}{2.199691in}}%
\pgfpathlineto{\pgfqpoint{5.106698in}{2.120912in}}%
\pgfpathlineto{\pgfqpoint{5.181455in}{2.031923in}}%
\pgfpathlineto{\pgfqpoint{5.241753in}{1.963355in}}%
\pgfpathlineto{\pgfqpoint{5.321625in}{1.876834in}}%
\pgfpathlineto{\pgfqpoint{5.391710in}{1.804202in}}%
\pgfpathlineto{\pgfqpoint{5.469299in}{1.727020in}}%
\pgfpathlineto{\pgfqpoint{5.566923in}{1.634304in}}%
\pgfpathlineto{\pgfqpoint{5.637916in}{1.569463in}}%
\pgfpathlineto{\pgfqpoint{5.727208in}{1.490685in}}%
\pgfpathlineto{\pgfqpoint{5.819922in}{1.411906in}}%
\pgfpathlineto{\pgfqpoint{5.917348in}{1.332015in}}%
\pgfpathlineto{\pgfqpoint{6.022476in}{1.248691in}}%
\pgfpathlineto{\pgfqpoint{6.127603in}{1.168067in}}%
\pgfpathlineto{\pgfqpoint{6.232731in}{1.089845in}}%
\pgfpathlineto{\pgfqpoint{6.337859in}{1.013768in}}%
\pgfpathlineto{\pgfqpoint{6.481344in}{0.912976in}}%
\pgfpathlineto{\pgfqpoint{6.489374in}{0.907442in}}%
\pgfpathlineto{\pgfqpoint{6.489374in}{0.907442in}}%
\pgfusepath{stroke}%
\end{pgfscope}%
\begin{pgfscope}%
\pgfpathrectangle{\pgfqpoint{0.766095in}{0.571603in}}{\pgfqpoint{6.973465in}{5.225635in}}%
\pgfusepath{clip}%
\pgfsetbuttcap%
\pgfsetroundjoin%
\pgfsetlinewidth{1.505625pt}%
\definecolor{currentstroke}{rgb}{0.283197,0.115680,0.436115}%
\pgfsetstrokecolor{currentstroke}%
\pgfsetdash{}{0pt}%
\pgfpathmoveto{\pgfqpoint{6.748050in}{0.733430in}}%
\pgfpathlineto{\pgfqpoint{6.754548in}{0.729160in}}%
\pgfpathlineto{\pgfqpoint{6.758369in}{0.726650in}}%
\pgfpathlineto{\pgfqpoint{6.793412in}{0.703753in}}%
\pgfpathlineto{\pgfqpoint{6.794722in}{0.702901in}}%
\pgfpathlineto{\pgfqpoint{6.828454in}{0.680946in}}%
\pgfpathlineto{\pgfqpoint{6.835111in}{0.676641in}}%
\pgfpathlineto{\pgfqpoint{6.863497in}{0.658269in}}%
\pgfpathlineto{\pgfqpoint{6.875766in}{0.650382in}}%
\pgfpathlineto{\pgfqpoint{6.898539in}{0.635720in}}%
\pgfpathlineto{\pgfqpoint{6.916682in}{0.624122in}}%
\pgfpathlineto{\pgfqpoint{6.933582in}{0.613296in}}%
\pgfpathlineto{\pgfqpoint{6.957853in}{0.597863in}}%
\pgfpathlineto{\pgfqpoint{6.968624in}{0.590994in}}%
\pgfpathlineto{\pgfqpoint{6.999271in}{0.571603in}}%
\pgfusepath{stroke}%
\end{pgfscope}%
\begin{pgfscope}%
\pgfpathrectangle{\pgfqpoint{0.766095in}{0.571603in}}{\pgfqpoint{6.973465in}{5.225635in}}%
\pgfusepath{clip}%
\pgfsetbuttcap%
\pgfsetroundjoin%
\pgfsetlinewidth{1.505625pt}%
\definecolor{currentstroke}{rgb}{0.282623,0.140926,0.457517}%
\pgfsetstrokecolor{currentstroke}%
\pgfsetdash{}{0pt}%
\pgfpathmoveto{\pgfqpoint{2.479988in}{0.571603in}}%
\pgfpathlineto{\pgfqpoint{2.378052in}{0.628471in}}%
\pgfpathlineto{\pgfqpoint{2.295832in}{0.676641in}}%
\pgfpathlineto{\pgfqpoint{2.202839in}{0.734061in}}%
\pgfpathlineto{\pgfqpoint{2.129524in}{0.781679in}}%
\pgfpathlineto{\pgfqpoint{2.052795in}{0.834198in}}%
\pgfpathlineto{\pgfqpoint{1.980098in}{0.886717in}}%
\pgfpathlineto{\pgfqpoint{1.911320in}{0.939236in}}%
\pgfpathlineto{\pgfqpoint{1.846328in}{0.991755in}}%
\pgfpathlineto{\pgfqpoint{1.782329in}{1.046679in}}%
\pgfpathlineto{\pgfqpoint{1.727368in}{1.096793in}}%
\pgfpathlineto{\pgfqpoint{1.673074in}{1.149312in}}%
\pgfpathlineto{\pgfqpoint{1.622229in}{1.201831in}}%
\pgfpathlineto{\pgfqpoint{1.572073in}{1.257195in}}%
\pgfpathlineto{\pgfqpoint{1.530098in}{1.306869in}}%
\pgfpathlineto{\pgfqpoint{1.488729in}{1.359388in}}%
\pgfpathlineto{\pgfqpoint{1.450327in}{1.411906in}}%
\pgfpathlineto{\pgfqpoint{1.414831in}{1.464425in}}%
\pgfpathlineto{\pgfqpoint{1.382167in}{1.516944in}}%
\pgfpathlineto{\pgfqpoint{1.352255in}{1.569463in}}%
\pgfpathlineto{\pgfqpoint{1.325004in}{1.621982in}}%
\pgfpathlineto{\pgfqpoint{1.300492in}{1.674501in}}%
\pgfpathlineto{\pgfqpoint{1.278531in}{1.727020in}}%
\pgfpathlineto{\pgfqpoint{1.256691in}{1.786463in}}%
\pgfpathlineto{\pgfqpoint{1.242125in}{1.832058in}}%
\pgfpathlineto{\pgfqpoint{1.227570in}{1.884577in}}%
\pgfpathlineto{\pgfqpoint{1.215460in}{1.937096in}}%
\pgfpathlineto{\pgfqpoint{1.205759in}{1.989615in}}%
\pgfpathlineto{\pgfqpoint{1.198359in}{2.042134in}}%
\pgfpathlineto{\pgfqpoint{1.193269in}{2.094653in}}%
\pgfpathlineto{\pgfqpoint{1.190492in}{2.147172in}}%
\pgfpathlineto{\pgfqpoint{1.190030in}{2.199691in}}%
\pgfpathlineto{\pgfqpoint{1.191877in}{2.252210in}}%
\pgfpathlineto{\pgfqpoint{1.196026in}{2.304729in}}%
\pgfpathlineto{\pgfqpoint{1.202465in}{2.357248in}}%
\pgfpathlineto{\pgfqpoint{1.211177in}{2.409766in}}%
\pgfpathlineto{\pgfqpoint{1.222153in}{2.462285in}}%
\pgfpathlineto{\pgfqpoint{1.235644in}{2.514804in}}%
\pgfpathlineto{\pgfqpoint{1.251384in}{2.567323in}}%
\pgfpathlineto{\pgfqpoint{1.269631in}{2.619842in}}%
\pgfpathlineto{\pgfqpoint{1.291733in}{2.675916in}}%
\pgfpathlineto{\pgfqpoint{1.313505in}{2.724880in}}%
\pgfpathlineto{\pgfqpoint{1.339292in}{2.777399in}}%
\pgfpathlineto{\pgfqpoint{1.367710in}{2.829918in}}%
\pgfpathlineto{\pgfqpoint{1.398884in}{2.882437in}}%
\pgfpathlineto{\pgfqpoint{1.432936in}{2.934956in}}%
\pgfpathlineto{\pgfqpoint{1.469989in}{2.987475in}}%
\pgfpathlineto{\pgfqpoint{1.510162in}{3.039994in}}%
\pgfpathlineto{\pgfqpoint{1.553571in}{3.092513in}}%
\pgfpathlineto{\pgfqpoint{1.600322in}{3.145032in}}%
\pgfpathlineto{\pgfqpoint{1.642158in}{3.188882in}}%
\pgfpathlineto{\pgfqpoint{1.677277in}{3.223810in}}%
\pgfpathlineto{\pgfqpoint{1.733631in}{3.276329in}}%
\pgfpathlineto{\pgfqpoint{1.782329in}{3.318925in}}%
\pgfpathlineto{\pgfqpoint{1.826007in}{3.355107in}}%
\pgfpathlineto{\pgfqpoint{1.893433in}{3.407626in}}%
\pgfpathlineto{\pgfqpoint{1.966055in}{3.460145in}}%
\pgfpathlineto{\pgfqpoint{2.044343in}{3.512664in}}%
\pgfpathlineto{\pgfqpoint{2.097711in}{3.546313in}}%
\pgfpathlineto{\pgfqpoint{2.173643in}{3.591443in}}%
\pgfpathlineto{\pgfqpoint{2.237882in}{3.627244in}}%
\pgfpathlineto{\pgfqpoint{2.319963in}{3.670221in}}%
\pgfpathlineto{\pgfqpoint{2.378052in}{3.698882in}}%
\pgfpathlineto{\pgfqpoint{2.448137in}{3.731533in}}%
\pgfpathlineto{\pgfqpoint{2.518222in}{3.762330in}}%
\pgfpathlineto{\pgfqpoint{2.588307in}{3.791351in}}%
\pgfpathlineto{\pgfqpoint{2.682700in}{3.827778in}}%
\pgfpathlineto{\pgfqpoint{2.763520in}{3.856637in}}%
\pgfpathlineto{\pgfqpoint{2.868647in}{3.891009in}}%
\pgfpathlineto{\pgfqpoint{2.938732in}{3.912114in}}%
\pgfpathlineto{\pgfqpoint{3.043860in}{3.940965in}}%
\pgfpathlineto{\pgfqpoint{3.148988in}{3.966572in}}%
\pgfpathlineto{\pgfqpoint{3.254115in}{3.988935in}}%
\pgfpathlineto{\pgfqpoint{3.359243in}{4.007929in}}%
\pgfpathlineto{\pgfqpoint{3.429328in}{4.018646in}}%
\pgfpathlineto{\pgfqpoint{3.499413in}{4.027785in}}%
\pgfpathlineto{\pgfqpoint{3.597714in}{4.037854in}}%
\pgfpathlineto{\pgfqpoint{3.639583in}{4.041058in}}%
\pgfpathlineto{\pgfqpoint{3.709668in}{4.044879in}}%
\pgfpathlineto{\pgfqpoint{3.779753in}{4.046707in}}%
\pgfpathlineto{\pgfqpoint{3.849838in}{4.046348in}}%
\pgfpathlineto{\pgfqpoint{3.919924in}{4.043578in}}%
\pgfpathlineto{\pgfqpoint{3.992689in}{4.037854in}}%
\pgfpathlineto{\pgfqpoint{4.060094in}{4.029447in}}%
\pgfpathlineto{\pgfqpoint{4.130179in}{4.017296in}}%
\pgfpathlineto{\pgfqpoint{4.165221in}{4.009728in}}%
\pgfpathlineto{\pgfqpoint{4.235306in}{3.990890in}}%
\pgfpathlineto{\pgfqpoint{4.270349in}{3.979465in}}%
\pgfpathlineto{\pgfqpoint{4.323524in}{3.959075in}}%
\pgfpathlineto{\pgfqpoint{4.340434in}{3.951742in}}%
\pgfpathlineto{\pgfqpoint{4.379904in}{3.932816in}}%
\pgfpathlineto{\pgfqpoint{4.426495in}{3.906556in}}%
\pgfpathlineto{\pgfqpoint{4.466310in}{3.880297in}}%
\pgfpathlineto{\pgfqpoint{4.500869in}{3.854037in}}%
\pgfpathlineto{\pgfqpoint{4.531286in}{3.827778in}}%
\pgfpathlineto{\pgfqpoint{4.558396in}{3.801519in}}%
\pgfpathlineto{\pgfqpoint{4.585732in}{3.771839in}}%
\pgfpathlineto{\pgfqpoint{4.620774in}{3.728126in}}%
\pgfpathlineto{\pgfqpoint{4.643019in}{3.696481in}}%
\pgfpathlineto{\pgfqpoint{4.659995in}{3.670221in}}%
\pgfpathlineto{\pgfqpoint{4.690859in}{3.616679in}}%
\pgfpathlineto{\pgfqpoint{4.716760in}{3.565183in}}%
\pgfpathlineto{\pgfqpoint{4.740451in}{3.512664in}}%
\pgfpathlineto{\pgfqpoint{4.762138in}{3.460145in}}%
\pgfpathlineto{\pgfqpoint{4.795987in}{3.369708in}}%
\pgfpathlineto{\pgfqpoint{4.837197in}{3.250070in}}%
\pgfpathlineto{\pgfqpoint{4.901115in}{3.062944in}}%
\pgfpathlineto{\pgfqpoint{4.936157in}{2.966488in}}%
\pgfpathlineto{\pgfqpoint{4.958238in}{2.908696in}}%
\pgfpathlineto{\pgfqpoint{4.990208in}{2.829918in}}%
\pgfpathlineto{\pgfqpoint{5.024477in}{2.751140in}}%
\pgfpathlineto{\pgfqpoint{5.061301in}{2.672361in}}%
\pgfpathlineto{\pgfqpoint{5.100909in}{2.593583in}}%
\pgfpathlineto{\pgfqpoint{5.146412in}{2.509711in}}%
\pgfpathlineto{\pgfqpoint{5.189152in}{2.436026in}}%
\pgfpathlineto{\pgfqpoint{5.238057in}{2.357248in}}%
\pgfpathlineto{\pgfqpoint{5.290374in}{2.278469in}}%
\pgfpathlineto{\pgfqpoint{5.346073in}{2.199691in}}%
\pgfpathlineto{\pgfqpoint{5.405317in}{2.120912in}}%
\pgfpathlineto{\pgfqpoint{5.468172in}{2.042134in}}%
\pgfpathlineto{\pgfqpoint{5.534650in}{1.963355in}}%
\pgfpathlineto{\pgfqpoint{5.604771in}{1.884577in}}%
\pgfpathlineto{\pgfqpoint{5.678557in}{1.805799in}}%
\pgfpathlineto{\pgfqpoint{5.756036in}{1.727020in}}%
\pgfpathlineto{\pgfqpoint{5.847263in}{1.638829in}}%
\pgfpathlineto{\pgfqpoint{5.922159in}{1.569463in}}%
\pgfpathlineto{\pgfqpoint{6.022476in}{1.480510in}}%
\pgfpathlineto{\pgfqpoint{6.102939in}{1.411906in}}%
\pgfpathlineto{\pgfqpoint{6.198832in}{1.333128in}}%
\pgfpathlineto{\pgfqpoint{6.302816in}{1.250816in}}%
\pgfpathlineto{\pgfqpoint{6.387805in}{1.185715in}}%
\pgfpathlineto{\pgfqpoint{6.387805in}{1.185715in}}%
\pgfusepath{stroke}%
\end{pgfscope}%
\begin{pgfscope}%
\pgfpathrectangle{\pgfqpoint{0.766095in}{0.571603in}}{\pgfqpoint{6.973465in}{5.225635in}}%
\pgfusepath{clip}%
\pgfsetbuttcap%
\pgfsetroundjoin%
\pgfsetlinewidth{1.505625pt}%
\definecolor{currentstroke}{rgb}{0.282623,0.140926,0.457517}%
\pgfsetstrokecolor{currentstroke}%
\pgfsetdash{}{0pt}%
\pgfpathmoveto{\pgfqpoint{6.639949in}{1.002280in}}%
\pgfpathlineto{\pgfqpoint{6.653242in}{0.992937in}}%
\pgfpathlineto{\pgfqpoint{6.654926in}{0.991755in}}%
\pgfpathlineto{\pgfqpoint{6.688284in}{0.968525in}}%
\pgfpathlineto{\pgfqpoint{6.692646in}{0.965495in}}%
\pgfpathlineto{\pgfqpoint{6.723327in}{0.944329in}}%
\pgfpathlineto{\pgfqpoint{6.730732in}{0.939236in}}%
\pgfpathlineto{\pgfqpoint{6.758369in}{0.920343in}}%
\pgfpathlineto{\pgfqpoint{6.769181in}{0.912976in}}%
\pgfpathlineto{\pgfqpoint{6.793412in}{0.896559in}}%
\pgfpathlineto{\pgfqpoint{6.807990in}{0.886717in}}%
\pgfpathlineto{\pgfqpoint{6.828454in}{0.872972in}}%
\pgfpathlineto{\pgfqpoint{6.847159in}{0.860458in}}%
\pgfpathlineto{\pgfqpoint{6.863497in}{0.849577in}}%
\pgfpathlineto{\pgfqpoint{6.886684in}{0.834198in}}%
\pgfpathlineto{\pgfqpoint{6.898539in}{0.826367in}}%
\pgfpathlineto{\pgfqpoint{6.926563in}{0.807939in}}%
\pgfpathlineto{\pgfqpoint{6.933582in}{0.803340in}}%
\pgfpathlineto{\pgfqpoint{6.966794in}{0.781679in}}%
\pgfpathlineto{\pgfqpoint{6.968624in}{0.780489in}}%
\pgfpathlineto{\pgfqpoint{7.003667in}{0.757781in}}%
\pgfpathlineto{\pgfqpoint{7.007327in}{0.755420in}}%
\pgfpathlineto{\pgfqpoint{7.038709in}{0.735222in}}%
\pgfpathlineto{\pgfqpoint{7.048177in}{0.729160in}}%
\pgfpathlineto{\pgfqpoint{7.073752in}{0.712821in}}%
\pgfpathlineto{\pgfqpoint{7.089367in}{0.702901in}}%
\pgfpathlineto{\pgfqpoint{7.108795in}{0.690577in}}%
\pgfpathlineto{\pgfqpoint{7.130893in}{0.676641in}}%
\pgfpathlineto{\pgfqpoint{7.143837in}{0.668487in}}%
\pgfpathlineto{\pgfqpoint{7.172753in}{0.650382in}}%
\pgfpathlineto{\pgfqpoint{7.178880in}{0.646547in}}%
\pgfpathlineto{\pgfqpoint{7.213922in}{0.624747in}}%
\pgfpathlineto{\pgfqpoint{7.214930in}{0.624122in}}%
\pgfpathlineto{\pgfqpoint{7.248965in}{0.603026in}}%
\pgfpathlineto{\pgfqpoint{7.257350in}{0.597863in}}%
\pgfpathlineto{\pgfqpoint{7.284007in}{0.581441in}}%
\pgfpathlineto{\pgfqpoint{7.300088in}{0.571603in}}%
\pgfusepath{stroke}%
\end{pgfscope}%
\begin{pgfscope}%
\pgfpathrectangle{\pgfqpoint{0.766095in}{0.571603in}}{\pgfqpoint{6.973465in}{5.225635in}}%
\pgfusepath{clip}%
\pgfsetbuttcap%
\pgfsetroundjoin%
\pgfsetlinewidth{1.505625pt}%
\definecolor{currentstroke}{rgb}{0.280255,0.165693,0.476498}%
\pgfsetstrokecolor{currentstroke}%
\pgfsetdash{}{0pt}%
\pgfpathmoveto{\pgfqpoint{2.247058in}{0.571603in}}%
\pgfpathlineto{\pgfqpoint{2.155466in}{0.624122in}}%
\pgfpathlineto{\pgfqpoint{2.062669in}{0.680127in}}%
\pgfpathlineto{\pgfqpoint{1.985364in}{0.729160in}}%
\pgfpathlineto{\pgfqpoint{1.906630in}{0.781679in}}%
\pgfpathlineto{\pgfqpoint{1.817371in}{0.844786in}}%
\pgfpathlineto{\pgfqpoint{1.747286in}{0.897292in}}%
\pgfpathlineto{\pgfqpoint{1.677201in}{0.952849in}}%
\pgfpathlineto{\pgfqpoint{1.630456in}{0.991755in}}%
\pgfpathlineto{\pgfqpoint{1.570473in}{1.044274in}}%
\pgfpathlineto{\pgfqpoint{1.501988in}{1.108452in}}%
\pgfpathlineto{\pgfqpoint{1.460802in}{1.149312in}}%
\pgfpathlineto{\pgfqpoint{1.410853in}{1.201831in}}%
\pgfpathlineto{\pgfqpoint{1.361818in}{1.256835in}}%
\pgfpathlineto{\pgfqpoint{1.320139in}{1.306869in}}%
\pgfpathlineto{\pgfqpoint{1.279277in}{1.359388in}}%
\pgfpathlineto{\pgfqpoint{1.241257in}{1.411906in}}%
\pgfpathlineto{\pgfqpoint{1.206017in}{1.464425in}}%
\pgfpathlineto{\pgfqpoint{1.173483in}{1.516944in}}%
\pgfpathlineto{\pgfqpoint{1.143578in}{1.569463in}}%
\pgfpathlineto{\pgfqpoint{1.116216in}{1.621982in}}%
\pgfpathlineto{\pgfqpoint{1.091491in}{1.674501in}}%
\pgfpathlineto{\pgfqpoint{1.069175in}{1.727020in}}%
\pgfpathlineto{\pgfqpoint{1.046435in}{1.787540in}}%
\pgfpathlineto{\pgfqpoint{1.031729in}{1.832058in}}%
\pgfpathlineto{\pgfqpoint{1.016472in}{1.884577in}}%
\pgfpathlineto{\pgfqpoint{1.003560in}{1.937096in}}%
\pgfpathlineto{\pgfqpoint{0.992911in}{1.989615in}}%
\pgfpathlineto{\pgfqpoint{0.984439in}{2.042134in}}%
\pgfpathlineto{\pgfqpoint{0.978149in}{2.094653in}}%
\pgfpathlineto{\pgfqpoint{0.974090in}{2.147172in}}%
\pgfpathlineto{\pgfqpoint{0.972209in}{2.199691in}}%
\pgfpathlineto{\pgfqpoint{0.972469in}{2.252210in}}%
\pgfpathlineto{\pgfqpoint{0.974862in}{2.304729in}}%
\pgfpathlineto{\pgfqpoint{0.979442in}{2.357248in}}%
\pgfpathlineto{\pgfqpoint{0.986208in}{2.409766in}}%
\pgfpathlineto{\pgfqpoint{0.995115in}{2.462285in}}%
\pgfpathlineto{\pgfqpoint{1.006145in}{2.514804in}}%
\pgfpathlineto{\pgfqpoint{1.019440in}{2.567323in}}%
\pgfpathlineto{\pgfqpoint{1.034966in}{2.619842in}}%
\pgfpathlineto{\pgfqpoint{1.052716in}{2.672361in}}%
\pgfpathlineto{\pgfqpoint{1.072814in}{2.724880in}}%
\pgfpathlineto{\pgfqpoint{1.095272in}{2.777399in}}%
\pgfpathlineto{\pgfqpoint{1.120066in}{2.829918in}}%
\pgfpathlineto{\pgfqpoint{1.151563in}{2.890014in}}%
\pgfpathlineto{\pgfqpoint{1.186605in}{2.950450in}}%
\pgfpathlineto{\pgfqpoint{1.221648in}{3.005628in}}%
\pgfpathlineto{\pgfqpoint{1.256691in}{3.056538in}}%
\pgfpathlineto{\pgfqpoint{1.291733in}{3.103943in}}%
\pgfpathlineto{\pgfqpoint{1.326776in}{3.148437in}}%
\pgfpathlineto{\pgfqpoint{1.368095in}{3.197551in}}%
\pgfpathlineto{\pgfqpoint{1.415415in}{3.250070in}}%
\pgfpathlineto{\pgfqpoint{1.466946in}{3.303546in}}%
\pgfpathlineto{\pgfqpoint{1.520309in}{3.355107in}}%
\pgfpathlineto{\pgfqpoint{1.578274in}{3.407626in}}%
\pgfpathlineto{\pgfqpoint{1.642158in}{3.461701in}}%
\pgfpathlineto{\pgfqpoint{1.712244in}{3.516977in}}%
\pgfpathlineto{\pgfqpoint{1.782329in}{3.568644in}}%
\pgfpathlineto{\pgfqpoint{1.853286in}{3.617702in}}%
\pgfpathlineto{\pgfqpoint{1.934591in}{3.670221in}}%
\pgfpathlineto{\pgfqpoint{2.021629in}{3.722740in}}%
\pgfpathlineto{\pgfqpoint{2.097711in}{3.765703in}}%
\pgfpathlineto{\pgfqpoint{2.167797in}{3.803213in}}%
\pgfpathlineto{\pgfqpoint{2.272924in}{3.855839in}}%
\pgfpathlineto{\pgfqpoint{2.382485in}{3.906556in}}%
\pgfpathlineto{\pgfqpoint{2.483179in}{3.949736in}}%
\pgfpathlineto{\pgfqpoint{2.571798in}{3.985335in}}%
\pgfpathlineto{\pgfqpoint{2.658392in}{4.017980in}}%
\pgfpathlineto{\pgfqpoint{2.763520in}{4.054972in}}%
\pgfpathlineto{\pgfqpoint{2.868647in}{4.089295in}}%
\pgfpathlineto{\pgfqpoint{2.959137in}{4.116632in}}%
\pgfpathlineto{\pgfqpoint{3.052463in}{4.142892in}}%
\pgfpathlineto{\pgfqpoint{3.153705in}{4.169151in}}%
\pgfpathlineto{\pgfqpoint{3.265303in}{4.195411in}}%
\pgfpathlineto{\pgfqpoint{3.359243in}{4.215354in}}%
\pgfpathlineto{\pgfqpoint{3.464371in}{4.235326in}}%
\pgfpathlineto{\pgfqpoint{3.569498in}{4.252827in}}%
\pgfpathlineto{\pgfqpoint{3.674626in}{4.267644in}}%
\pgfpathlineto{\pgfqpoint{3.779753in}{4.279711in}}%
\pgfpathlineto{\pgfqpoint{3.884881in}{4.288767in}}%
\pgfpathlineto{\pgfqpoint{3.954966in}{4.293070in}}%
\pgfpathlineto{\pgfqpoint{4.025051in}{4.295840in}}%
\pgfpathlineto{\pgfqpoint{4.095136in}{4.296933in}}%
\pgfpathlineto{\pgfqpoint{4.165221in}{4.296181in}}%
\pgfpathlineto{\pgfqpoint{4.235306in}{4.293396in}}%
\pgfpathlineto{\pgfqpoint{4.305391in}{4.288355in}}%
\pgfpathlineto{\pgfqpoint{4.375477in}{4.280802in}}%
\pgfpathlineto{\pgfqpoint{4.445562in}{4.270285in}}%
\pgfpathlineto{\pgfqpoint{4.515647in}{4.256188in}}%
\pgfpathlineto{\pgfqpoint{4.550689in}{4.247751in}}%
\pgfpathlineto{\pgfqpoint{4.620774in}{4.226936in}}%
\pgfpathlineto{\pgfqpoint{4.655817in}{4.214435in}}%
\pgfpathlineto{\pgfqpoint{4.702075in}{4.195411in}}%
\pgfpathlineto{\pgfqpoint{4.725902in}{4.184208in}}%
\pgfpathlineto{\pgfqpoint{4.760944in}{4.166012in}}%
\pgfpathlineto{\pgfqpoint{4.799474in}{4.142892in}}%
\pgfpathlineto{\pgfqpoint{4.836955in}{4.116632in}}%
\pgfpathlineto{\pgfqpoint{4.869309in}{4.090373in}}%
\pgfpathlineto{\pgfqpoint{4.901115in}{4.060408in}}%
\pgfpathlineto{\pgfqpoint{4.936157in}{4.021363in}}%
\pgfpathlineto{\pgfqpoint{4.944035in}{4.011594in}}%
\pgfpathlineto{\pgfqpoint{4.971200in}{3.974044in}}%
\pgfpathlineto{\pgfqpoint{4.980907in}{3.959075in}}%
\pgfpathlineto{\pgfqpoint{5.006242in}{3.915072in}}%
\pgfpathlineto{\pgfqpoint{5.023402in}{3.880297in}}%
\pgfpathlineto{\pgfqpoint{5.045651in}{3.827778in}}%
\pgfpathlineto{\pgfqpoint{5.064213in}{3.775259in}}%
\pgfpathlineto{\pgfqpoint{5.080123in}{3.722740in}}%
\pgfpathlineto{\pgfqpoint{5.093844in}{3.670221in}}%
\pgfpathlineto{\pgfqpoint{5.111808in}{3.591443in}}%
\pgfpathlineto{\pgfqpoint{5.117173in}{3.565304in}}%
\pgfpathlineto{\pgfqpoint{5.117173in}{3.565304in}}%
\pgfusepath{stroke}%
\end{pgfscope}%
\begin{pgfscope}%
\pgfpathrectangle{\pgfqpoint{0.766095in}{0.571603in}}{\pgfqpoint{6.973465in}{5.225635in}}%
\pgfusepath{clip}%
\pgfsetbuttcap%
\pgfsetroundjoin%
\pgfsetlinewidth{1.505625pt}%
\definecolor{currentstroke}{rgb}{0.280255,0.165693,0.476498}%
\pgfsetstrokecolor{currentstroke}%
\pgfsetdash{}{0pt}%
\pgfpathmoveto{\pgfqpoint{5.175520in}{3.257880in}}%
\pgfpathlineto{\pgfqpoint{5.193604in}{3.171291in}}%
\pgfpathlineto{\pgfqpoint{5.211847in}{3.092513in}}%
\pgfpathlineto{\pgfqpoint{5.232048in}{3.013734in}}%
\pgfpathlineto{\pgfqpoint{5.254662in}{2.934956in}}%
\pgfpathlineto{\pgfqpoint{5.279811in}{2.856177in}}%
\pgfpathlineto{\pgfqpoint{5.307785in}{2.777399in}}%
\pgfpathlineto{\pgfqpoint{5.338797in}{2.698621in}}%
\pgfpathlineto{\pgfqpoint{5.372998in}{2.619842in}}%
\pgfpathlineto{\pgfqpoint{5.410528in}{2.541064in}}%
\pgfpathlineto{\pgfqpoint{5.451518in}{2.462285in}}%
\pgfpathlineto{\pgfqpoint{5.496838in}{2.382271in}}%
\pgfpathlineto{\pgfqpoint{5.531880in}{2.324495in}}%
\pgfpathlineto{\pgfqpoint{5.566923in}{2.269714in}}%
\pgfpathlineto{\pgfqpoint{5.601965in}{2.217549in}}%
\pgfpathlineto{\pgfqpoint{5.637008in}{2.167680in}}%
\pgfpathlineto{\pgfqpoint{5.672050in}{2.119831in}}%
\pgfpathlineto{\pgfqpoint{5.711297in}{2.068393in}}%
\pgfpathlineto{\pgfqpoint{5.774656in}{1.989615in}}%
\pgfpathlineto{\pgfqpoint{5.847263in}{1.904787in}}%
\pgfpathlineto{\pgfqpoint{5.913035in}{1.832058in}}%
\pgfpathlineto{\pgfqpoint{5.988123in}{1.753280in}}%
\pgfpathlineto{\pgfqpoint{6.067076in}{1.674501in}}%
\pgfpathlineto{\pgfqpoint{6.149980in}{1.595723in}}%
\pgfpathlineto{\pgfqpoint{6.236826in}{1.516944in}}%
\pgfpathlineto{\pgfqpoint{6.337859in}{1.429436in}}%
\pgfpathlineto{\pgfqpoint{6.422082in}{1.359388in}}%
\pgfpathlineto{\pgfqpoint{6.520507in}{1.280609in}}%
\pgfpathlineto{\pgfqpoint{6.622726in}{1.201831in}}%
\pgfpathlineto{\pgfqpoint{6.728681in}{1.123052in}}%
\pgfpathlineto{\pgfqpoint{6.838326in}{1.044274in}}%
\pgfpathlineto{\pgfqpoint{6.968624in}{0.953896in}}%
\pgfpathlineto{\pgfqpoint{7.073752in}{0.883247in}}%
\pgfpathlineto{\pgfqpoint{7.213922in}{0.791814in}}%
\pgfpathlineto{\pgfqpoint{7.354679in}{0.702901in}}%
\pgfpathlineto{\pgfqpoint{7.494263in}{0.617165in}}%
\pgfpathlineto{\pgfqpoint{7.570022in}{0.571603in}}%
\pgfpathlineto{\pgfqpoint{7.570022in}{0.571603in}}%
\pgfusepath{stroke}%
\end{pgfscope}%
\begin{pgfscope}%
\pgfpathrectangle{\pgfqpoint{0.766095in}{0.571603in}}{\pgfqpoint{6.973465in}{5.225635in}}%
\pgfusepath{clip}%
\pgfsetbuttcap%
\pgfsetroundjoin%
\pgfsetlinewidth{1.505625pt}%
\definecolor{currentstroke}{rgb}{0.276194,0.190074,0.493001}%
\pgfsetstrokecolor{currentstroke}%
\pgfsetdash{}{0pt}%
\pgfpathmoveto{\pgfqpoint{2.037051in}{0.571603in}}%
\pgfpathlineto{\pgfqpoint{1.947838in}{0.624122in}}%
\pgfpathlineto{\pgfqpoint{1.852414in}{0.683341in}}%
\pgfpathlineto{\pgfqpoint{1.781985in}{0.729160in}}%
\pgfpathlineto{\pgfqpoint{1.705177in}{0.781679in}}%
\pgfpathlineto{\pgfqpoint{1.632204in}{0.834198in}}%
\pgfpathlineto{\pgfqpoint{1.562958in}{0.886717in}}%
\pgfpathlineto{\pgfqpoint{1.497318in}{0.939236in}}%
\pgfpathlineto{\pgfqpoint{1.431903in}{0.994653in}}%
\pgfpathlineto{\pgfqpoint{1.361818in}{1.057987in}}%
\pgfpathlineto{\pgfqpoint{1.321102in}{1.096793in}}%
\pgfpathlineto{\pgfqpoint{1.256691in}{1.162138in}}%
\pgfpathlineto{\pgfqpoint{1.219750in}{1.201831in}}%
\pgfpathlineto{\pgfqpoint{1.173697in}{1.254350in}}%
\pgfpathlineto{\pgfqpoint{1.130512in}{1.306869in}}%
\pgfpathlineto{\pgfqpoint{1.090144in}{1.359388in}}%
\pgfpathlineto{\pgfqpoint{1.046435in}{1.420870in}}%
\pgfpathlineto{\pgfqpoint{1.011393in}{1.474310in}}%
\pgfpathlineto{\pgfqpoint{0.976350in}{1.532397in}}%
\pgfpathlineto{\pgfqpoint{0.955595in}{1.569463in}}%
\pgfpathlineto{\pgfqpoint{0.928318in}{1.621982in}}%
\pgfpathlineto{\pgfqpoint{0.903418in}{1.674501in}}%
\pgfpathlineto{\pgfqpoint{0.880976in}{1.727020in}}%
\pgfpathlineto{\pgfqpoint{0.860818in}{1.779539in}}%
\pgfpathlineto{\pgfqpoint{0.842915in}{1.832058in}}%
\pgfpathlineto{\pgfqpoint{0.827273in}{1.884577in}}%
\pgfpathlineto{\pgfqpoint{0.813827in}{1.937096in}}%
\pgfpathlineto{\pgfqpoint{0.801138in}{1.996779in}}%
\pgfpathlineto{\pgfqpoint{0.793349in}{2.042134in}}%
\pgfpathlineto{\pgfqpoint{0.786315in}{2.094653in}}%
\pgfpathlineto{\pgfqpoint{0.781346in}{2.147172in}}%
\pgfpathlineto{\pgfqpoint{0.778441in}{2.199691in}}%
\pgfpathlineto{\pgfqpoint{0.777596in}{2.252210in}}%
\pgfpathlineto{\pgfqpoint{0.778808in}{2.304729in}}%
\pgfpathlineto{\pgfqpoint{0.782068in}{2.357248in}}%
\pgfpathlineto{\pgfqpoint{0.787367in}{2.409766in}}%
\pgfpathlineto{\pgfqpoint{0.794694in}{2.462285in}}%
\pgfpathlineto{\pgfqpoint{0.804091in}{2.514804in}}%
\pgfpathlineto{\pgfqpoint{0.815649in}{2.567323in}}%
\pgfpathlineto{\pgfqpoint{0.829225in}{2.619842in}}%
\pgfpathlineto{\pgfqpoint{0.844968in}{2.672361in}}%
\pgfpathlineto{\pgfqpoint{0.862860in}{2.724880in}}%
\pgfpathlineto{\pgfqpoint{0.882967in}{2.777399in}}%
\pgfpathlineto{\pgfqpoint{0.906265in}{2.832189in}}%
\pgfpathlineto{\pgfqpoint{0.929936in}{2.882437in}}%
\pgfpathlineto{\pgfqpoint{0.956932in}{2.934956in}}%
\pgfpathlineto{\pgfqpoint{0.986322in}{2.987475in}}%
\pgfpathlineto{\pgfqpoint{1.018202in}{3.039994in}}%
\pgfpathlineto{\pgfqpoint{1.052671in}{3.092513in}}%
\pgfpathlineto{\pgfqpoint{1.089823in}{3.145032in}}%
\pgfpathlineto{\pgfqpoint{1.129751in}{3.197551in}}%
\pgfpathlineto{\pgfqpoint{1.172542in}{3.250070in}}%
\pgfpathlineto{\pgfqpoint{1.221648in}{3.306304in}}%
\pgfpathlineto{\pgfqpoint{1.267267in}{3.355107in}}%
\pgfpathlineto{\pgfqpoint{1.326776in}{3.414679in}}%
\pgfpathlineto{\pgfqpoint{1.375222in}{3.460145in}}%
\pgfpathlineto{\pgfqpoint{1.434524in}{3.512664in}}%
\pgfpathlineto{\pgfqpoint{1.501988in}{3.568561in}}%
\pgfpathlineto{\pgfqpoint{1.572073in}{3.622917in}}%
\pgfpathlineto{\pgfqpoint{1.642158in}{3.673974in}}%
\pgfpathlineto{\pgfqpoint{1.713205in}{3.722740in}}%
\pgfpathlineto{\pgfqpoint{1.794710in}{3.775259in}}%
\pgfpathlineto{\pgfqpoint{1.881519in}{3.827778in}}%
\pgfpathlineto{\pgfqpoint{1.957541in}{3.871041in}}%
\pgfpathlineto{\pgfqpoint{2.027626in}{3.909011in}}%
\pgfpathlineto{\pgfqpoint{2.132754in}{3.962591in}}%
\pgfpathlineto{\pgfqpoint{2.237882in}{4.012621in}}%
\pgfpathlineto{\pgfqpoint{2.354149in}{4.064113in}}%
\pgfpathlineto{\pgfqpoint{2.448137in}{4.103060in}}%
\pgfpathlineto{\pgfqpoint{2.553264in}{4.144149in}}%
\pgfpathlineto{\pgfqpoint{2.658392in}{4.182528in}}%
\pgfpathlineto{\pgfqpoint{2.772699in}{4.221670in}}%
\pgfpathlineto{\pgfqpoint{2.868647in}{4.252386in}}%
\pgfpathlineto{\pgfqpoint{2.973775in}{4.283951in}}%
\pgfpathlineto{\pgfqpoint{3.078903in}{4.313458in}}%
\pgfpathlineto{\pgfqpoint{3.184030in}{4.340948in}}%
\pgfpathlineto{\pgfqpoint{3.289158in}{4.366460in}}%
\pgfpathlineto{\pgfqpoint{3.394285in}{4.390026in}}%
\pgfpathlineto{\pgfqpoint{3.499413in}{4.411673in}}%
\pgfpathlineto{\pgfqpoint{3.606420in}{4.431746in}}%
\pgfpathlineto{\pgfqpoint{3.744711in}{4.454515in}}%
\pgfpathlineto{\pgfqpoint{3.849838in}{4.469424in}}%
\pgfpathlineto{\pgfqpoint{3.972823in}{4.484265in}}%
\pgfpathlineto{\pgfqpoint{4.060094in}{4.492804in}}%
\pgfpathlineto{\pgfqpoint{4.165221in}{4.500914in}}%
\pgfpathlineto{\pgfqpoint{4.270349in}{4.506423in}}%
\pgfpathlineto{\pgfqpoint{4.375477in}{4.508978in}}%
\pgfpathlineto{\pgfqpoint{4.445562in}{4.508832in}}%
\pgfpathlineto{\pgfqpoint{4.515647in}{4.507034in}}%
\pgfpathlineto{\pgfqpoint{4.585732in}{4.503412in}}%
\pgfpathlineto{\pgfqpoint{4.655817in}{4.497766in}}%
\pgfpathlineto{\pgfqpoint{4.725902in}{4.489863in}}%
\pgfpathlineto{\pgfqpoint{4.795987in}{4.479219in}}%
\pgfpathlineto{\pgfqpoint{4.866072in}{4.465331in}}%
\pgfpathlineto{\pgfqpoint{4.901115in}{4.457094in}}%
\pgfpathlineto{\pgfqpoint{4.971200in}{4.436973in}}%
\pgfpathlineto{\pgfqpoint{5.006242in}{4.424956in}}%
\pgfpathlineto{\pgfqpoint{5.055375in}{4.405486in}}%
\pgfpathlineto{\pgfqpoint{5.076327in}{4.395976in}}%
\pgfpathlineto{\pgfqpoint{5.111370in}{4.378596in}}%
\pgfpathlineto{\pgfqpoint{5.155194in}{4.352967in}}%
\pgfpathlineto{\pgfqpoint{5.193097in}{4.326708in}}%
\pgfpathlineto{\pgfqpoint{5.225422in}{4.300449in}}%
\pgfpathlineto{\pgfqpoint{5.253276in}{4.274189in}}%
\pgfpathlineto{\pgfqpoint{5.286583in}{4.236583in}}%
\pgfpathlineto{\pgfqpoint{5.298190in}{4.221670in}}%
\pgfpathlineto{\pgfqpoint{5.321625in}{4.187315in}}%
\pgfpathlineto{\pgfqpoint{5.332516in}{4.169151in}}%
\pgfpathlineto{\pgfqpoint{5.356668in}{4.121771in}}%
\pgfpathlineto{\pgfqpoint{5.369825in}{4.090373in}}%
\pgfpathlineto{\pgfqpoint{5.387880in}{4.037854in}}%
\pgfpathlineto{\pgfqpoint{5.395300in}{4.011594in}}%
\pgfpathlineto{\pgfqpoint{5.407517in}{3.959075in}}%
\pgfpathlineto{\pgfqpoint{5.417029in}{3.906556in}}%
\pgfpathlineto{\pgfqpoint{5.426753in}{3.834214in}}%
\pgfpathlineto{\pgfqpoint{5.432535in}{3.775259in}}%
\pgfpathlineto{\pgfqpoint{5.439991in}{3.670221in}}%
\pgfpathlineto{\pgfqpoint{5.457525in}{3.381367in}}%
\pgfpathlineto{\pgfqpoint{5.467168in}{3.276329in}}%
\pgfpathlineto{\pgfqpoint{5.476566in}{3.197551in}}%
\pgfpathlineto{\pgfqpoint{5.488253in}{3.118772in}}%
\pgfpathlineto{\pgfqpoint{5.502462in}{3.039994in}}%
\pgfpathlineto{\pgfqpoint{5.519406in}{2.961215in}}%
\pgfpathlineto{\pgfqpoint{5.539365in}{2.882437in}}%
\pgfpathlineto{\pgfqpoint{5.562482in}{2.803659in}}%
\pgfpathlineto{\pgfqpoint{5.588877in}{2.724880in}}%
\pgfpathlineto{\pgfqpoint{5.618753in}{2.646102in}}%
\pgfpathlineto{\pgfqpoint{5.652208in}{2.567323in}}%
\pgfpathlineto{\pgfqpoint{5.689339in}{2.488545in}}%
\pgfpathlineto{\pgfqpoint{5.730242in}{2.409766in}}%
\pgfpathlineto{\pgfqpoint{5.775015in}{2.330988in}}%
\pgfpathlineto{\pgfqpoint{5.823624in}{2.252210in}}%
\pgfpathlineto{\pgfqpoint{5.876239in}{2.173431in}}%
\pgfpathlineto{\pgfqpoint{5.932817in}{2.094653in}}%
\pgfpathlineto{\pgfqpoint{5.993471in}{2.015874in}}%
\pgfpathlineto{\pgfqpoint{6.058188in}{1.937096in}}%
\pgfpathlineto{\pgfqpoint{6.103542in}{1.884577in}}%
\pgfpathlineto{\pgfqpoint{6.175035in}{1.805799in}}%
\pgfpathlineto{\pgfqpoint{6.250620in}{1.727020in}}%
\pgfpathlineto{\pgfqpoint{6.330317in}{1.648242in}}%
\pgfpathlineto{\pgfqpoint{6.414091in}{1.569463in}}%
\pgfpathlineto{\pgfqpoint{6.501922in}{1.490685in}}%
\pgfpathlineto{\pgfqpoint{6.593828in}{1.411906in}}%
\pgfpathlineto{\pgfqpoint{6.689801in}{1.333128in}}%
\pgfpathlineto{\pgfqpoint{6.793412in}{1.251530in}}%
\pgfpathlineto{\pgfqpoint{6.898539in}{1.171961in}}%
\pgfpathlineto{\pgfqpoint{7.003667in}{1.095268in}}%
\pgfpathlineto{\pgfqpoint{7.113272in}{1.018014in}}%
\pgfpathlineto{\pgfqpoint{7.228836in}{0.939236in}}%
\pgfpathlineto{\pgfqpoint{7.340065in}{0.865780in}}%
\pgfpathlineto{\pgfqpoint{7.340065in}{0.865780in}}%
\pgfusepath{stroke}%
\end{pgfscope}%
\begin{pgfscope}%
\pgfpathrectangle{\pgfqpoint{0.766095in}{0.571603in}}{\pgfqpoint{6.973465in}{5.225635in}}%
\pgfusepath{clip}%
\pgfsetbuttcap%
\pgfsetroundjoin%
\pgfsetlinewidth{1.505625pt}%
\definecolor{currentstroke}{rgb}{0.276194,0.190074,0.493001}%
\pgfsetstrokecolor{currentstroke}%
\pgfsetdash{}{0pt}%
\pgfpathmoveto{\pgfqpoint{7.603691in}{0.699455in}}%
\pgfpathlineto{\pgfqpoint{7.634433in}{0.680676in}}%
\pgfpathlineto{\pgfqpoint{7.641068in}{0.676641in}}%
\pgfpathlineto{\pgfqpoint{7.669475in}{0.659426in}}%
\pgfpathlineto{\pgfqpoint{7.684479in}{0.650382in}}%
\pgfpathlineto{\pgfqpoint{7.704518in}{0.638338in}}%
\pgfpathlineto{\pgfqpoint{7.728302in}{0.624122in}}%
\pgfpathlineto{\pgfqpoint{7.739560in}{0.617410in}}%
\pgfusepath{stroke}%
\end{pgfscope}%
\begin{pgfscope}%
\pgfpathrectangle{\pgfqpoint{0.766095in}{0.571603in}}{\pgfqpoint{6.973465in}{5.225635in}}%
\pgfusepath{clip}%
\pgfsetbuttcap%
\pgfsetroundjoin%
\pgfsetlinewidth{1.505625pt}%
\definecolor{currentstroke}{rgb}{0.270595,0.214069,0.507052}%
\pgfsetstrokecolor{currentstroke}%
\pgfsetdash{}{0pt}%
\pgfpathmoveto{\pgfqpoint{1.844905in}{0.571603in}}%
\pgfpathlineto{\pgfqpoint{1.817371in}{0.588003in}}%
\pgfpathlineto{\pgfqpoint{1.809501in}{0.592698in}}%
\pgfusepath{stroke}%
\end{pgfscope}%
\begin{pgfscope}%
\pgfpathrectangle{\pgfqpoint{0.766095in}{0.571603in}}{\pgfqpoint{6.973465in}{5.225635in}}%
\pgfusepath{clip}%
\pgfsetbuttcap%
\pgfsetroundjoin%
\pgfsetlinewidth{1.505625pt}%
\definecolor{currentstroke}{rgb}{0.270595,0.214069,0.507052}%
\pgfsetstrokecolor{currentstroke}%
\pgfsetdash{}{0pt}%
\pgfpathmoveto{\pgfqpoint{1.548015in}{0.762294in}}%
\pgfpathlineto{\pgfqpoint{1.537031in}{0.770056in}}%
\pgfpathlineto{\pgfqpoint{1.520643in}{0.781679in}}%
\pgfpathlineto{\pgfqpoint{1.501988in}{0.795200in}}%
\pgfpathlineto{\pgfqpoint{1.484479in}{0.807939in}}%
\pgfpathlineto{\pgfqpoint{1.466946in}{0.820975in}}%
\pgfpathlineto{\pgfqpoint{1.449235in}{0.834198in}}%
\pgfpathlineto{\pgfqpoint{1.431903in}{0.847424in}}%
\pgfpathlineto{\pgfqpoint{1.414897in}{0.860458in}}%
\pgfpathlineto{\pgfqpoint{1.396861in}{0.874588in}}%
\pgfpathlineto{\pgfqpoint{1.381450in}{0.886717in}}%
\pgfpathlineto{\pgfqpoint{1.361818in}{0.902513in}}%
\pgfpathlineto{\pgfqpoint{1.348877in}{0.912976in}}%
\pgfpathlineto{\pgfqpoint{1.326776in}{0.931248in}}%
\pgfpathlineto{\pgfqpoint{1.317162in}{0.939236in}}%
\pgfpathlineto{\pgfqpoint{1.291733in}{0.960844in}}%
\pgfpathlineto{\pgfqpoint{1.286288in}{0.965495in}}%
\pgfpathlineto{\pgfqpoint{1.256691in}{0.991356in}}%
\pgfpathlineto{\pgfqpoint{1.256236in}{0.991755in}}%
\pgfpathlineto{\pgfqpoint{1.227073in}{1.018014in}}%
\pgfpathlineto{\pgfqpoint{1.221648in}{1.023013in}}%
\pgfpathlineto{\pgfqpoint{1.198714in}{1.044274in}}%
\pgfpathlineto{\pgfqpoint{1.186605in}{1.055761in}}%
\pgfpathlineto{\pgfqpoint{1.171133in}{1.070533in}}%
\pgfpathlineto{\pgfqpoint{1.151563in}{1.089657in}}%
\pgfpathlineto{\pgfqpoint{1.144309in}{1.096793in}}%
\pgfpathlineto{\pgfqpoint{1.118248in}{1.123052in}}%
\pgfpathlineto{\pgfqpoint{1.116520in}{1.124840in}}%
\pgfpathlineto{\pgfqpoint{1.093031in}{1.149312in}}%
\pgfpathlineto{\pgfqpoint{1.081478in}{1.161641in}}%
\pgfpathlineto{\pgfqpoint{1.068518in}{1.175571in}}%
\pgfpathlineto{\pgfqpoint{1.046435in}{1.199893in}}%
\pgfpathlineto{\pgfqpoint{1.044689in}{1.201831in}}%
\pgfpathlineto{\pgfqpoint{1.021685in}{1.228090in}}%
\pgfpathlineto{\pgfqpoint{1.011393in}{1.240144in}}%
\pgfpathlineto{\pgfqpoint{0.999358in}{1.254350in}}%
\pgfpathlineto{\pgfqpoint{0.977679in}{1.280609in}}%
\pgfpathlineto{\pgfqpoint{0.976350in}{1.282268in}}%
\pgfpathlineto{\pgfqpoint{0.956810in}{1.306869in}}%
\pgfpathlineto{\pgfqpoint{0.941308in}{1.326901in}}%
\pgfpathlineto{\pgfqpoint{0.936530in}{1.333128in}}%
\pgfpathlineto{\pgfqpoint{0.916990in}{1.359388in}}%
\pgfpathlineto{\pgfqpoint{0.906265in}{1.374224in}}%
\pgfpathlineto{\pgfqpoint{0.898081in}{1.385647in}}%
\pgfpathlineto{\pgfqpoint{0.879841in}{1.411906in}}%
\pgfpathlineto{\pgfqpoint{0.871223in}{1.424710in}}%
\pgfpathlineto{\pgfqpoint{0.862250in}{1.438166in}}%
\pgfpathlineto{\pgfqpoint{0.845297in}{1.464425in}}%
\pgfpathlineto{\pgfqpoint{0.836180in}{1.479023in}}%
\pgfpathlineto{\pgfqpoint{0.828968in}{1.490685in}}%
\pgfpathlineto{\pgfqpoint{0.813288in}{1.516944in}}%
\pgfpathlineto{\pgfqpoint{0.801138in}{1.537995in}}%
\pgfpathlineto{\pgfqpoint{0.798162in}{1.543204in}}%
\pgfpathlineto{\pgfqpoint{0.783737in}{1.569463in}}%
\pgfpathlineto{\pgfqpoint{0.769816in}{1.595723in}}%
\pgfpathlineto{\pgfqpoint{0.766095in}{1.603039in}}%
\pgfusepath{stroke}%
\end{pgfscope}%
\begin{pgfscope}%
\pgfpathrectangle{\pgfqpoint{0.766095in}{0.571603in}}{\pgfqpoint{6.973465in}{5.225635in}}%
\pgfusepath{clip}%
\pgfsetbuttcap%
\pgfsetroundjoin%
\pgfsetlinewidth{1.505625pt}%
\definecolor{currentstroke}{rgb}{0.270595,0.214069,0.507052}%
\pgfsetstrokecolor{currentstroke}%
\pgfsetdash{}{0pt}%
\pgfpathmoveto{\pgfqpoint{0.766095in}{2.944414in}}%
\pgfpathlineto{\pgfqpoint{0.788505in}{2.987475in}}%
\pgfpathlineto{\pgfqpoint{0.817950in}{3.039994in}}%
\pgfpathlineto{\pgfqpoint{0.849788in}{3.092513in}}%
\pgfpathlineto{\pgfqpoint{0.884109in}{3.145032in}}%
\pgfpathlineto{\pgfqpoint{0.921001in}{3.197551in}}%
\pgfpathlineto{\pgfqpoint{0.960550in}{3.250070in}}%
\pgfpathlineto{\pgfqpoint{1.002840in}{3.302589in}}%
\pgfpathlineto{\pgfqpoint{1.047978in}{3.355107in}}%
\pgfpathlineto{\pgfqpoint{1.096237in}{3.407626in}}%
\pgfpathlineto{\pgfqpoint{1.151563in}{3.464138in}}%
\pgfpathlineto{\pgfqpoint{1.202185in}{3.512664in}}%
\pgfpathlineto{\pgfqpoint{1.260203in}{3.565183in}}%
\pgfpathlineto{\pgfqpoint{1.326776in}{3.621688in}}%
\pgfpathlineto{\pgfqpoint{1.396861in}{3.677449in}}%
\pgfpathlineto{\pgfqpoint{1.466946in}{3.729884in}}%
\pgfpathlineto{\pgfqpoint{1.537031in}{3.779371in}}%
\pgfpathlineto{\pgfqpoint{1.609536in}{3.827778in}}%
\pgfpathlineto{\pgfqpoint{1.693038in}{3.880297in}}%
\pgfpathlineto{\pgfqpoint{1.782329in}{3.933205in}}%
\pgfpathlineto{\pgfqpoint{1.887456in}{3.991426in}}%
\pgfpathlineto{\pgfqpoint{1.976609in}{4.037854in}}%
\pgfpathlineto{\pgfqpoint{2.062669in}{4.080254in}}%
\pgfpathlineto{\pgfqpoint{2.140063in}{4.116632in}}%
\pgfpathlineto{\pgfqpoint{2.258563in}{4.169151in}}%
\pgfpathlineto{\pgfqpoint{2.343009in}{4.204441in}}%
\pgfpathlineto{\pgfqpoint{2.452591in}{4.247930in}}%
\pgfpathlineto{\pgfqpoint{2.588307in}{4.298206in}}%
\pgfpathlineto{\pgfqpoint{2.693435in}{4.334639in}}%
\pgfpathlineto{\pgfqpoint{2.798562in}{4.369093in}}%
\pgfpathlineto{\pgfqpoint{2.916339in}{4.405486in}}%
\pgfpathlineto{\pgfqpoint{3.043860in}{4.442211in}}%
\pgfpathlineto{\pgfqpoint{3.148988in}{4.470607in}}%
\pgfpathlineto{\pgfqpoint{3.254115in}{4.497328in}}%
\pgfpathlineto{\pgfqpoint{3.359243in}{4.522414in}}%
\pgfpathlineto{\pgfqpoint{3.464371in}{4.545901in}}%
\pgfpathlineto{\pgfqpoint{3.604541in}{4.574729in}}%
\pgfpathlineto{\pgfqpoint{3.744711in}{4.600745in}}%
\pgfpathlineto{\pgfqpoint{3.884881in}{4.623921in}}%
\pgfpathlineto{\pgfqpoint{4.025051in}{4.644209in}}%
\pgfpathlineto{\pgfqpoint{4.165221in}{4.661323in}}%
\pgfpathlineto{\pgfqpoint{4.270349in}{4.672086in}}%
\pgfpathlineto{\pgfqpoint{4.375477in}{4.680807in}}%
\pgfpathlineto{\pgfqpoint{4.480604in}{4.687499in}}%
\pgfpathlineto{\pgfqpoint{4.585732in}{4.691906in}}%
\pgfpathlineto{\pgfqpoint{4.690859in}{4.693724in}}%
\pgfpathlineto{\pgfqpoint{4.795987in}{4.692581in}}%
\pgfpathlineto{\pgfqpoint{4.866072in}{4.689955in}}%
\pgfpathlineto{\pgfqpoint{4.936157in}{4.685655in}}%
\pgfpathlineto{\pgfqpoint{5.006242in}{4.679500in}}%
\pgfpathlineto{\pgfqpoint{5.076327in}{4.671275in}}%
\pgfpathlineto{\pgfqpoint{5.146412in}{4.660387in}}%
\pgfpathlineto{\pgfqpoint{5.216497in}{4.646585in}}%
\pgfpathlineto{\pgfqpoint{5.251540in}{4.638364in}}%
\pgfpathlineto{\pgfqpoint{5.321625in}{4.618696in}}%
\pgfpathlineto{\pgfqpoint{5.356668in}{4.606899in}}%
\pgfpathlineto{\pgfqpoint{5.402653in}{4.589303in}}%
\pgfpathlineto{\pgfqpoint{5.426753in}{4.578689in}}%
\pgfpathlineto{\pgfqpoint{5.461795in}{4.561788in}}%
\pgfpathlineto{\pgfqpoint{5.505708in}{4.536784in}}%
\pgfpathlineto{\pgfqpoint{5.544361in}{4.510524in}}%
\pgfpathlineto{\pgfqpoint{5.577003in}{4.484265in}}%
\pgfpathlineto{\pgfqpoint{5.604812in}{4.458005in}}%
\pgfpathlineto{\pgfqpoint{5.637008in}{4.421078in}}%
\pgfpathlineto{\pgfqpoint{5.648814in}{4.405486in}}%
\pgfpathlineto{\pgfqpoint{5.672050in}{4.369491in}}%
\pgfpathlineto{\pgfqpoint{5.681308in}{4.352967in}}%
\pgfpathlineto{\pgfqpoint{5.694166in}{4.326708in}}%
\pgfpathlineto{\pgfqpoint{5.707093in}{4.295421in}}%
\pgfpathlineto{\pgfqpoint{5.714578in}{4.274189in}}%
\pgfpathlineto{\pgfqpoint{5.722494in}{4.247930in}}%
\pgfpathlineto{\pgfqpoint{5.734754in}{4.195411in}}%
\pgfpathlineto{\pgfqpoint{5.743045in}{4.142892in}}%
\pgfpathlineto{\pgfqpoint{5.748127in}{4.090373in}}%
\pgfpathlineto{\pgfqpoint{5.750785in}{4.037854in}}%
\pgfpathlineto{\pgfqpoint{5.751559in}{3.985335in}}%
\pgfpathlineto{\pgfqpoint{5.750160in}{3.906556in}}%
\pgfpathlineto{\pgfqpoint{5.749142in}{3.879002in}}%
\pgfpathlineto{\pgfqpoint{5.749142in}{3.879002in}}%
\pgfusepath{stroke}%
\end{pgfscope}%
\begin{pgfscope}%
\pgfpathrectangle{\pgfqpoint{0.766095in}{0.571603in}}{\pgfqpoint{6.973465in}{5.225635in}}%
\pgfusepath{clip}%
\pgfsetbuttcap%
\pgfsetroundjoin%
\pgfsetlinewidth{1.505625pt}%
\definecolor{currentstroke}{rgb}{0.270595,0.214069,0.507052}%
\pgfsetstrokecolor{currentstroke}%
\pgfsetdash{}{0pt}%
\pgfpathmoveto{\pgfqpoint{5.732355in}{3.566485in}}%
\pgfpathlineto{\pgfqpoint{5.729502in}{3.486405in}}%
\pgfpathlineto{\pgfqpoint{5.728289in}{3.407626in}}%
\pgfpathlineto{\pgfqpoint{5.729027in}{3.328848in}}%
\pgfpathlineto{\pgfqpoint{5.732038in}{3.250070in}}%
\pgfpathlineto{\pgfqpoint{5.737609in}{3.171291in}}%
\pgfpathlineto{\pgfqpoint{5.745949in}{3.092513in}}%
\pgfpathlineto{\pgfqpoint{5.757252in}{3.013734in}}%
\pgfpathlineto{\pgfqpoint{5.771795in}{2.934956in}}%
\pgfpathlineto{\pgfqpoint{5.789638in}{2.856177in}}%
\pgfpathlineto{\pgfqpoint{5.803485in}{2.803659in}}%
\pgfpathlineto{\pgfqpoint{5.827220in}{2.724880in}}%
\pgfpathlineto{\pgfqpoint{5.847263in}{2.666539in}}%
\pgfpathlineto{\pgfqpoint{5.864633in}{2.619842in}}%
\pgfpathlineto{\pgfqpoint{5.885902in}{2.567323in}}%
\pgfpathlineto{\pgfqpoint{5.917348in}{2.496476in}}%
\pgfpathlineto{\pgfqpoint{5.933534in}{2.462285in}}%
\pgfpathlineto{\pgfqpoint{5.960017in}{2.409766in}}%
\pgfpathlineto{\pgfqpoint{5.988295in}{2.357248in}}%
\pgfpathlineto{\pgfqpoint{6.022476in}{2.297827in}}%
\pgfpathlineto{\pgfqpoint{6.057518in}{2.240662in}}%
\pgfpathlineto{\pgfqpoint{6.092561in}{2.186722in}}%
\pgfpathlineto{\pgfqpoint{6.127603in}{2.135543in}}%
\pgfpathlineto{\pgfqpoint{6.162646in}{2.086743in}}%
\pgfpathlineto{\pgfqpoint{6.197689in}{2.040009in}}%
\pgfpathlineto{\pgfqpoint{6.237126in}{1.989615in}}%
\pgfpathlineto{\pgfqpoint{6.280072in}{1.937096in}}%
\pgfpathlineto{\pgfqpoint{6.348043in}{1.858318in}}%
\pgfpathlineto{\pgfqpoint{6.420231in}{1.779539in}}%
\pgfpathlineto{\pgfqpoint{6.496649in}{1.700761in}}%
\pgfpathlineto{\pgfqpoint{6.577321in}{1.621982in}}%
\pgfpathlineto{\pgfqpoint{6.662185in}{1.543204in}}%
\pgfpathlineto{\pgfqpoint{6.758369in}{1.458348in}}%
\pgfpathlineto{\pgfqpoint{6.844542in}{1.385647in}}%
\pgfpathlineto{\pgfqpoint{6.942010in}{1.306869in}}%
\pgfpathlineto{\pgfqpoint{7.043628in}{1.228090in}}%
\pgfpathlineto{\pgfqpoint{7.149352in}{1.149312in}}%
\pgfpathlineto{\pgfqpoint{7.259150in}{1.070533in}}%
\pgfpathlineto{\pgfqpoint{7.372998in}{0.991755in}}%
\pgfpathlineto{\pgfqpoint{7.494263in}{0.910762in}}%
\pgfpathlineto{\pgfqpoint{7.634433in}{0.820387in}}%
\pgfpathlineto{\pgfqpoint{7.739560in}{0.754702in}}%
\pgfpathlineto{\pgfqpoint{7.739560in}{0.754702in}}%
\pgfusepath{stroke}%
\end{pgfscope}%
\begin{pgfscope}%
\pgfpathrectangle{\pgfqpoint{0.766095in}{0.571603in}}{\pgfqpoint{6.973465in}{5.225635in}}%
\pgfusepath{clip}%
\pgfsetbuttcap%
\pgfsetroundjoin%
\pgfsetlinewidth{1.505625pt}%
\definecolor{currentstroke}{rgb}{0.263663,0.237631,0.518762}%
\pgfsetstrokecolor{currentstroke}%
\pgfsetdash{}{0pt}%
\pgfpathmoveto{\pgfqpoint{1.667213in}{0.571603in}}%
\pgfpathlineto{\pgfqpoint{1.642158in}{0.586850in}}%
\pgfpathlineto{\pgfqpoint{1.624098in}{0.597863in}}%
\pgfpathlineto{\pgfqpoint{1.607116in}{0.608435in}}%
\pgfpathlineto{\pgfqpoint{1.581975in}{0.624122in}}%
\pgfpathlineto{\pgfqpoint{1.572073in}{0.630431in}}%
\pgfpathlineto{\pgfqpoint{1.540837in}{0.650382in}}%
\pgfpathlineto{\pgfqpoint{1.537031in}{0.652864in}}%
\pgfpathlineto{\pgfqpoint{1.501988in}{0.675792in}}%
\pgfpathlineto{\pgfqpoint{1.500696in}{0.676641in}}%
\pgfpathlineto{\pgfqpoint{1.466946in}{0.699278in}}%
\pgfpathlineto{\pgfqpoint{1.461562in}{0.702901in}}%
\pgfpathlineto{\pgfqpoint{1.431903in}{0.723279in}}%
\pgfpathlineto{\pgfqpoint{1.423372in}{0.729160in}}%
\pgfpathlineto{\pgfqpoint{1.396861in}{0.747826in}}%
\pgfpathlineto{\pgfqpoint{1.386114in}{0.755420in}}%
\pgfpathlineto{\pgfqpoint{1.361818in}{0.772954in}}%
\pgfpathlineto{\pgfqpoint{1.349776in}{0.781679in}}%
\pgfpathlineto{\pgfqpoint{1.336705in}{0.791352in}}%
\pgfusepath{stroke}%
\end{pgfscope}%
\begin{pgfscope}%
\pgfpathrectangle{\pgfqpoint{0.766095in}{0.571603in}}{\pgfqpoint{6.973465in}{5.225635in}}%
\pgfusepath{clip}%
\pgfsetbuttcap%
\pgfsetroundjoin%
\pgfsetlinewidth{1.505625pt}%
\definecolor{currentstroke}{rgb}{0.263663,0.237631,0.518762}%
\pgfsetstrokecolor{currentstroke}%
\pgfsetdash{}{0pt}%
\pgfpathmoveto{\pgfqpoint{1.094759in}{0.988018in}}%
\pgfpathlineto{\pgfqpoint{1.090574in}{0.991755in}}%
\pgfpathlineto{\pgfqpoint{1.081478in}{1.000059in}}%
\pgfpathlineto{\pgfqpoint{1.061927in}{1.018014in}}%
\pgfpathlineto{\pgfqpoint{1.046435in}{1.032567in}}%
\pgfpathlineto{\pgfqpoint{1.034050in}{1.044274in}}%
\pgfpathlineto{\pgfqpoint{1.011393in}{1.066181in}}%
\pgfpathlineto{\pgfqpoint{1.006921in}{1.070533in}}%
\pgfpathlineto{\pgfqpoint{0.980585in}{1.096793in}}%
\pgfpathlineto{\pgfqpoint{0.976350in}{1.101120in}}%
\pgfpathlineto{\pgfqpoint{0.955037in}{1.123052in}}%
\pgfpathlineto{\pgfqpoint{0.941308in}{1.137515in}}%
\pgfpathlineto{\pgfqpoint{0.930189in}{1.149312in}}%
\pgfpathlineto{\pgfqpoint{0.906265in}{1.175301in}}%
\pgfpathlineto{\pgfqpoint{0.906018in}{1.175571in}}%
\pgfpathlineto{\pgfqpoint{0.882680in}{1.201831in}}%
\pgfpathlineto{\pgfqpoint{0.871223in}{1.215040in}}%
\pgfpathlineto{\pgfqpoint{0.859990in}{1.228090in}}%
\pgfpathlineto{\pgfqpoint{0.837951in}{1.254350in}}%
\pgfpathlineto{\pgfqpoint{0.836180in}{1.256522in}}%
\pgfpathlineto{\pgfqpoint{0.816699in}{1.280609in}}%
\pgfpathlineto{\pgfqpoint{0.801138in}{1.300344in}}%
\pgfpathlineto{\pgfqpoint{0.796036in}{1.306869in}}%
\pgfpathlineto{\pgfqpoint{0.776094in}{1.333128in}}%
\pgfpathlineto{\pgfqpoint{0.766095in}{1.346669in}}%
\pgfusepath{stroke}%
\end{pgfscope}%
\begin{pgfscope}%
\pgfpathrectangle{\pgfqpoint{0.766095in}{0.571603in}}{\pgfqpoint{6.973465in}{5.225635in}}%
\pgfusepath{clip}%
\pgfsetbuttcap%
\pgfsetroundjoin%
\pgfsetlinewidth{1.505625pt}%
\definecolor{currentstroke}{rgb}{0.263663,0.237631,0.518762}%
\pgfsetstrokecolor{currentstroke}%
\pgfsetdash{}{0pt}%
\pgfpathmoveto{\pgfqpoint{0.766095in}{3.243445in}}%
\pgfpathlineto{\pgfqpoint{0.801138in}{3.290411in}}%
\pgfpathlineto{\pgfqpoint{0.836180in}{3.334808in}}%
\pgfpathlineto{\pgfqpoint{0.875039in}{3.381367in}}%
\pgfpathlineto{\pgfqpoint{0.921642in}{3.433886in}}%
\pgfpathlineto{\pgfqpoint{0.976350in}{3.491740in}}%
\pgfpathlineto{\pgfqpoint{1.023799in}{3.538924in}}%
\pgfpathlineto{\pgfqpoint{1.081478in}{3.593122in}}%
\pgfpathlineto{\pgfqpoint{1.151563in}{3.654693in}}%
\pgfpathlineto{\pgfqpoint{1.221648in}{3.712359in}}%
\pgfpathlineto{\pgfqpoint{1.268536in}{3.749000in}}%
\pgfpathlineto{\pgfqpoint{1.339189in}{3.801519in}}%
\pgfpathlineto{\pgfqpoint{1.414066in}{3.854037in}}%
\pgfpathlineto{\pgfqpoint{1.493388in}{3.906556in}}%
\pgfpathlineto{\pgfqpoint{1.577472in}{3.959075in}}%
\pgfpathlineto{\pgfqpoint{1.677201in}{4.017574in}}%
\pgfpathlineto{\pgfqpoint{1.761302in}{4.064113in}}%
\pgfpathlineto{\pgfqpoint{1.861736in}{4.116632in}}%
\pgfpathlineto{\pgfqpoint{1.968492in}{4.169151in}}%
\pgfpathlineto{\pgfqpoint{2.082087in}{4.221670in}}%
\pgfpathlineto{\pgfqpoint{2.167797in}{4.259162in}}%
\pgfpathlineto{\pgfqpoint{2.272924in}{4.302993in}}%
\pgfpathlineto{\pgfqpoint{2.400336in}{4.352967in}}%
\pgfpathlineto{\pgfqpoint{2.518222in}{4.396397in}}%
\pgfpathlineto{\pgfqpoint{2.623350in}{4.433148in}}%
\pgfpathlineto{\pgfqpoint{2.763520in}{4.479190in}}%
\pgfpathlineto{\pgfqpoint{2.868647in}{4.511788in}}%
\pgfpathlineto{\pgfqpoint{3.008818in}{4.552616in}}%
\pgfpathlineto{\pgfqpoint{3.143060in}{4.589303in}}%
\pgfpathlineto{\pgfqpoint{3.254115in}{4.617779in}}%
\pgfpathlineto{\pgfqpoint{3.394285in}{4.651427in}}%
\pgfpathlineto{\pgfqpoint{3.534456in}{4.682694in}}%
\pgfpathlineto{\pgfqpoint{3.674626in}{4.711590in}}%
\pgfpathlineto{\pgfqpoint{3.814796in}{4.738119in}}%
\pgfpathlineto{\pgfqpoint{3.954966in}{4.762275in}}%
\pgfpathlineto{\pgfqpoint{4.095136in}{4.784042in}}%
\pgfpathlineto{\pgfqpoint{4.235306in}{4.803395in}}%
\pgfpathlineto{\pgfqpoint{4.375477in}{4.820112in}}%
\pgfpathlineto{\pgfqpoint{4.480604in}{4.830883in}}%
\pgfpathlineto{\pgfqpoint{4.620774in}{4.842668in}}%
\pgfpathlineto{\pgfqpoint{4.725902in}{4.849543in}}%
\pgfpathlineto{\pgfqpoint{4.831030in}{4.854412in}}%
\pgfpathlineto{\pgfqpoint{4.936157in}{4.857116in}}%
\pgfpathlineto{\pgfqpoint{5.041285in}{4.857466in}}%
\pgfpathlineto{\pgfqpoint{5.146412in}{4.855132in}}%
\pgfpathlineto{\pgfqpoint{5.216497in}{4.851890in}}%
\pgfpathlineto{\pgfqpoint{5.286583in}{4.846926in}}%
\pgfpathlineto{\pgfqpoint{5.356668in}{4.840203in}}%
\pgfpathlineto{\pgfqpoint{5.426753in}{4.831520in}}%
\pgfpathlineto{\pgfqpoint{5.496838in}{4.820390in}}%
\pgfpathlineto{\pgfqpoint{5.566923in}{4.806337in}}%
\pgfpathlineto{\pgfqpoint{5.601965in}{4.798148in}}%
\pgfpathlineto{\pgfqpoint{5.672050in}{4.778386in}}%
\pgfpathlineto{\pgfqpoint{5.707093in}{4.766694in}}%
\pgfpathlineto{\pgfqpoint{5.758599in}{4.746860in}}%
\pgfpathlineto{\pgfqpoint{5.777178in}{4.738626in}}%
\pgfpathlineto{\pgfqpoint{5.814573in}{4.720600in}}%
\pgfpathlineto{\pgfqpoint{5.860152in}{4.694341in}}%
\pgfpathlineto{\pgfqpoint{5.898061in}{4.668081in}}%
\pgfpathlineto{\pgfqpoint{5.929852in}{4.641822in}}%
\pgfpathlineto{\pgfqpoint{5.956702in}{4.615562in}}%
\pgfpathlineto{\pgfqpoint{5.987433in}{4.578541in}}%
\pgfpathlineto{\pgfqpoint{5.998496in}{4.563043in}}%
\pgfpathlineto{\pgfqpoint{6.022476in}{4.522159in}}%
\pgfpathlineto{\pgfqpoint{6.028362in}{4.510524in}}%
\pgfpathlineto{\pgfqpoint{6.039730in}{4.484265in}}%
\pgfpathlineto{\pgfqpoint{6.049211in}{4.458005in}}%
\pgfpathlineto{\pgfqpoint{6.057518in}{4.429665in}}%
\pgfpathlineto{\pgfqpoint{6.063199in}{4.405486in}}%
\pgfpathlineto{\pgfqpoint{6.068066in}{4.379227in}}%
\pgfpathlineto{\pgfqpoint{6.074400in}{4.326708in}}%
\pgfpathlineto{\pgfqpoint{6.076971in}{4.274189in}}%
\pgfpathlineto{\pgfqpoint{6.076551in}{4.221670in}}%
\pgfpathlineto{\pgfqpoint{6.073776in}{4.169151in}}%
\pgfpathlineto{\pgfqpoint{6.069176in}{4.116632in}}%
\pgfpathlineto{\pgfqpoint{6.059797in}{4.037854in}}%
\pgfpathlineto{\pgfqpoint{6.044309in}{3.932816in}}%
\pgfpathlineto{\pgfqpoint{5.999835in}{3.643962in}}%
\pgfpathlineto{\pgfqpoint{5.987082in}{3.538924in}}%
\pgfpathlineto{\pgfqpoint{5.979627in}{3.460145in}}%
\pgfpathlineto{\pgfqpoint{5.974425in}{3.381367in}}%
\pgfpathlineto{\pgfqpoint{5.971744in}{3.302589in}}%
\pgfpathlineto{\pgfqpoint{5.971822in}{3.223810in}}%
\pgfpathlineto{\pgfqpoint{5.974878in}{3.145032in}}%
\pgfpathlineto{\pgfqpoint{5.981109in}{3.066253in}}%
\pgfpathlineto{\pgfqpoint{5.990669in}{2.987475in}}%
\pgfpathlineto{\pgfqpoint{5.998934in}{2.934956in}}%
\pgfpathlineto{\pgfqpoint{6.008809in}{2.882437in}}%
\pgfpathlineto{\pgfqpoint{6.022476in}{2.821192in}}%
\pgfpathlineto{\pgfqpoint{6.033476in}{2.777399in}}%
\pgfpathlineto{\pgfqpoint{6.048328in}{2.724880in}}%
\pgfpathlineto{\pgfqpoint{6.064899in}{2.672361in}}%
\pgfpathlineto{\pgfqpoint{6.083203in}{2.619842in}}%
\pgfpathlineto{\pgfqpoint{6.103276in}{2.567323in}}%
\pgfpathlineto{\pgfqpoint{6.127603in}{2.509322in}}%
\pgfpathlineto{\pgfqpoint{6.148809in}{2.462285in}}%
\pgfpathlineto{\pgfqpoint{6.174313in}{2.409766in}}%
\pgfpathlineto{\pgfqpoint{6.201677in}{2.357248in}}%
\pgfpathlineto{\pgfqpoint{6.232731in}{2.301578in}}%
\pgfpathlineto{\pgfqpoint{6.267774in}{2.242816in}}%
\pgfpathlineto{\pgfqpoint{6.302816in}{2.187619in}}%
\pgfpathlineto{\pgfqpoint{6.337859in}{2.135445in}}%
\pgfpathlineto{\pgfqpoint{6.372901in}{2.085854in}}%
\pgfpathlineto{\pgfqpoint{6.407944in}{2.038490in}}%
\pgfpathlineto{\pgfqpoint{6.445724in}{1.989615in}}%
\pgfpathlineto{\pgfqpoint{6.488191in}{1.937096in}}%
\pgfpathlineto{\pgfqpoint{6.555541in}{1.858318in}}%
\pgfpathlineto{\pgfqpoint{6.627219in}{1.779539in}}%
\pgfpathlineto{\pgfqpoint{6.703235in}{1.700761in}}%
\pgfpathlineto{\pgfqpoint{6.783610in}{1.621982in}}%
\pgfpathlineto{\pgfqpoint{6.868330in}{1.543204in}}%
\pgfpathlineto{\pgfqpoint{6.957347in}{1.464425in}}%
\pgfpathlineto{\pgfqpoint{7.050696in}{1.385647in}}%
\pgfpathlineto{\pgfqpoint{7.148362in}{1.306869in}}%
\pgfpathlineto{\pgfqpoint{7.250298in}{1.228090in}}%
\pgfpathlineto{\pgfqpoint{7.263564in}{1.218114in}}%
\pgfpathlineto{\pgfqpoint{7.263564in}{1.218114in}}%
\pgfusepath{stroke}%
\end{pgfscope}%
\begin{pgfscope}%
\pgfpathrectangle{\pgfqpoint{0.766095in}{0.571603in}}{\pgfqpoint{6.973465in}{5.225635in}}%
\pgfusepath{clip}%
\pgfsetbuttcap%
\pgfsetroundjoin%
\pgfsetlinewidth{1.505625pt}%
\definecolor{currentstroke}{rgb}{0.263663,0.237631,0.518762}%
\pgfsetstrokecolor{currentstroke}%
\pgfsetdash{}{0pt}%
\pgfpathmoveto{\pgfqpoint{7.516629in}{1.035976in}}%
\pgfpathlineto{\pgfqpoint{7.529305in}{1.027254in}}%
\pgfpathlineto{\pgfqpoint{7.542760in}{1.018014in}}%
\pgfpathlineto{\pgfqpoint{7.564348in}{1.003335in}}%
\pgfpathlineto{\pgfqpoint{7.581415in}{0.991755in}}%
\pgfpathlineto{\pgfqpoint{7.599390in}{0.979673in}}%
\pgfpathlineto{\pgfqpoint{7.620535in}{0.965495in}}%
\pgfpathlineto{\pgfqpoint{7.634433in}{0.956260in}}%
\pgfpathlineto{\pgfqpoint{7.660119in}{0.939236in}}%
\pgfpathlineto{\pgfqpoint{7.669475in}{0.933089in}}%
\pgfpathlineto{\pgfqpoint{7.700169in}{0.912976in}}%
\pgfpathlineto{\pgfqpoint{7.704518in}{0.910151in}}%
\pgfpathlineto{\pgfqpoint{7.739560in}{0.887438in}}%
\pgfusepath{stroke}%
\end{pgfscope}%
\begin{pgfscope}%
\pgfpathrectangle{\pgfqpoint{0.766095in}{0.571603in}}{\pgfqpoint{6.973465in}{5.225635in}}%
\pgfusepath{clip}%
\pgfsetbuttcap%
\pgfsetroundjoin%
\pgfsetlinewidth{1.505625pt}%
\definecolor{currentstroke}{rgb}{0.255645,0.260703,0.528312}%
\pgfsetstrokecolor{currentstroke}%
\pgfsetdash{}{0pt}%
\pgfpathmoveto{\pgfqpoint{1.501452in}{0.571603in}}%
\pgfpathlineto{\pgfqpoint{1.469066in}{0.591738in}}%
\pgfusepath{stroke}%
\end{pgfscope}%
\begin{pgfscope}%
\pgfpathrectangle{\pgfqpoint{0.766095in}{0.571603in}}{\pgfqpoint{6.973465in}{5.225635in}}%
\pgfusepath{clip}%
\pgfsetbuttcap%
\pgfsetroundjoin%
\pgfsetlinewidth{1.505625pt}%
\definecolor{currentstroke}{rgb}{0.255645,0.260703,0.528312}%
\pgfsetstrokecolor{currentstroke}%
\pgfsetdash{}{0pt}%
\pgfpathmoveto{\pgfqpoint{1.210944in}{0.766461in}}%
\pgfpathlineto{\pgfqpoint{1.190296in}{0.781679in}}%
\pgfpathlineto{\pgfqpoint{1.186605in}{0.784456in}}%
\pgfpathlineto{\pgfqpoint{1.155540in}{0.807939in}}%
\pgfpathlineto{\pgfqpoint{1.151563in}{0.811009in}}%
\pgfpathlineto{\pgfqpoint{1.121660in}{0.834198in}}%
\pgfpathlineto{\pgfqpoint{1.116520in}{0.838269in}}%
\pgfpathlineto{\pgfqpoint{1.088642in}{0.860458in}}%
\pgfpathlineto{\pgfqpoint{1.081478in}{0.866282in}}%
\pgfpathlineto{\pgfqpoint{1.056469in}{0.886717in}}%
\pgfpathlineto{\pgfqpoint{1.046435in}{0.895093in}}%
\pgfpathlineto{\pgfqpoint{1.025125in}{0.912976in}}%
\pgfpathlineto{\pgfqpoint{1.011393in}{0.924751in}}%
\pgfpathlineto{\pgfqpoint{0.994592in}{0.939236in}}%
\pgfpathlineto{\pgfqpoint{0.976350in}{0.955309in}}%
\pgfpathlineto{\pgfqpoint{0.964854in}{0.965495in}}%
\pgfpathlineto{\pgfqpoint{0.941308in}{0.986821in}}%
\pgfpathlineto{\pgfqpoint{0.935893in}{0.991755in}}%
\pgfpathlineto{\pgfqpoint{0.907710in}{1.018014in}}%
\pgfpathlineto{\pgfqpoint{0.906265in}{1.019393in}}%
\pgfpathlineto{\pgfqpoint{0.880357in}{1.044274in}}%
\pgfpathlineto{\pgfqpoint{0.871223in}{1.053245in}}%
\pgfpathlineto{\pgfqpoint{0.853736in}{1.070533in}}%
\pgfpathlineto{\pgfqpoint{0.836180in}{1.088286in}}%
\pgfpathlineto{\pgfqpoint{0.827825in}{1.096793in}}%
\pgfpathlineto{\pgfqpoint{0.802627in}{1.123052in}}%
\pgfpathlineto{\pgfqpoint{0.801138in}{1.124645in}}%
\pgfpathlineto{\pgfqpoint{0.778235in}{1.149312in}}%
\pgfpathlineto{\pgfqpoint{0.766095in}{1.162696in}}%
\pgfusepath{stroke}%
\end{pgfscope}%
\begin{pgfscope}%
\pgfpathrectangle{\pgfqpoint{0.766095in}{0.571603in}}{\pgfqpoint{6.973465in}{5.225635in}}%
\pgfusepath{clip}%
\pgfsetbuttcap%
\pgfsetroundjoin%
\pgfsetlinewidth{1.505625pt}%
\definecolor{currentstroke}{rgb}{0.255645,0.260703,0.528312}%
\pgfsetstrokecolor{currentstroke}%
\pgfsetdash{}{0pt}%
\pgfpathmoveto{\pgfqpoint{0.766095in}{3.462488in}}%
\pgfpathlineto{\pgfqpoint{0.812294in}{3.512664in}}%
\pgfpathlineto{\pgfqpoint{0.871223in}{3.572813in}}%
\pgfpathlineto{\pgfqpoint{0.917853in}{3.617702in}}%
\pgfpathlineto{\pgfqpoint{0.976350in}{3.671080in}}%
\pgfpathlineto{\pgfqpoint{1.046435in}{3.731059in}}%
\pgfpathlineto{\pgfqpoint{1.116520in}{3.787418in}}%
\pgfpathlineto{\pgfqpoint{1.186605in}{3.840558in}}%
\pgfpathlineto{\pgfqpoint{1.256691in}{3.890832in}}%
\pgfpathlineto{\pgfqpoint{1.326776in}{3.938554in}}%
\pgfpathlineto{\pgfqpoint{1.399011in}{3.985335in}}%
\pgfpathlineto{\pgfqpoint{1.484699in}{4.037854in}}%
\pgfpathlineto{\pgfqpoint{1.575267in}{4.090373in}}%
\pgfpathlineto{\pgfqpoint{1.677201in}{4.146049in}}%
\pgfpathlineto{\pgfqpoint{1.782329in}{4.200116in}}%
\pgfpathlineto{\pgfqpoint{1.887456in}{4.251140in}}%
\pgfpathlineto{\pgfqpoint{1.994941in}{4.300449in}}%
\pgfpathlineto{\pgfqpoint{2.116467in}{4.352967in}}%
\pgfpathlineto{\pgfqpoint{2.237882in}{4.402406in}}%
\pgfpathlineto{\pgfqpoint{2.343009in}{4.442900in}}%
\pgfpathlineto{\pgfqpoint{2.455698in}{4.484265in}}%
\pgfpathlineto{\pgfqpoint{2.588307in}{4.530176in}}%
\pgfpathlineto{\pgfqpoint{2.693435in}{4.564754in}}%
\pgfpathlineto{\pgfqpoint{2.857629in}{4.615562in}}%
\pgfpathlineto{\pgfqpoint{2.973775in}{4.649299in}}%
\pgfpathlineto{\pgfqpoint{3.113945in}{4.687838in}}%
\pgfpathlineto{\pgfqpoint{3.240240in}{4.720600in}}%
\pgfpathlineto{\pgfqpoint{3.394285in}{4.757998in}}%
\pgfpathlineto{\pgfqpoint{3.534456in}{4.789840in}}%
\pgfpathlineto{\pgfqpoint{3.674626in}{4.819619in}}%
\pgfpathlineto{\pgfqpoint{3.814796in}{4.847351in}}%
\pgfpathlineto{\pgfqpoint{3.954966in}{4.873049in}}%
\pgfpathlineto{\pgfqpoint{4.095136in}{4.896716in}}%
\pgfpathlineto{\pgfqpoint{4.235306in}{4.918351in}}%
\pgfpathlineto{\pgfqpoint{4.375477in}{4.937944in}}%
\pgfpathlineto{\pgfqpoint{4.528665in}{4.956935in}}%
\pgfpathlineto{\pgfqpoint{4.655817in}{4.970510in}}%
\pgfpathlineto{\pgfqpoint{4.795987in}{4.983387in}}%
\pgfpathlineto{\pgfqpoint{4.936157in}{4.993444in}}%
\pgfpathlineto{\pgfqpoint{5.076327in}{5.000754in}}%
\pgfpathlineto{\pgfqpoint{5.181455in}{5.004169in}}%
\pgfpathlineto{\pgfqpoint{5.286583in}{5.005563in}}%
\pgfpathlineto{\pgfqpoint{5.391710in}{5.004661in}}%
\pgfpathlineto{\pgfqpoint{5.496838in}{5.001136in}}%
\pgfpathlineto{\pgfqpoint{5.601965in}{4.994587in}}%
\pgfpathlineto{\pgfqpoint{5.672050in}{4.988303in}}%
\pgfpathlineto{\pgfqpoint{5.742136in}{4.980143in}}%
\pgfpathlineto{\pgfqpoint{5.812221in}{4.969680in}}%
\pgfpathlineto{\pgfqpoint{5.882306in}{4.956899in}}%
\pgfpathlineto{\pgfqpoint{5.952391in}{4.940633in}}%
\pgfpathlineto{\pgfqpoint{5.989697in}{4.930676in}}%
\pgfpathlineto{\pgfqpoint{6.057518in}{4.908767in}}%
\pgfpathlineto{\pgfqpoint{6.092561in}{4.895321in}}%
\pgfpathlineto{\pgfqpoint{6.132286in}{4.878157in}}%
\pgfpathlineto{\pgfqpoint{6.182785in}{4.851897in}}%
\pgfpathlineto{\pgfqpoint{6.224335in}{4.825638in}}%
\pgfpathlineto{\pgfqpoint{6.258782in}{4.799378in}}%
\pgfpathlineto{\pgfqpoint{6.287515in}{4.773119in}}%
\pgfpathlineto{\pgfqpoint{6.311599in}{4.746860in}}%
\pgfpathlineto{\pgfqpoint{6.337859in}{4.711304in}}%
\pgfpathlineto{\pgfqpoint{6.348478in}{4.694341in}}%
\pgfpathlineto{\pgfqpoint{6.362363in}{4.668081in}}%
\pgfpathlineto{\pgfqpoint{6.373807in}{4.641822in}}%
\pgfpathlineto{\pgfqpoint{6.382944in}{4.615562in}}%
\pgfpathlineto{\pgfqpoint{6.390203in}{4.589303in}}%
\pgfpathlineto{\pgfqpoint{6.395787in}{4.563043in}}%
\pgfpathlineto{\pgfqpoint{6.399874in}{4.536784in}}%
\pgfpathlineto{\pgfqpoint{6.402622in}{4.510524in}}%
\pgfpathlineto{\pgfqpoint{6.404656in}{4.458005in}}%
\pgfpathlineto{\pgfqpoint{6.402857in}{4.405486in}}%
\pgfpathlineto{\pgfqpoint{6.398013in}{4.352967in}}%
\pgfpathlineto{\pgfqpoint{6.390769in}{4.300449in}}%
\pgfpathlineto{\pgfqpoint{6.381660in}{4.247930in}}%
\pgfpathlineto{\pgfqpoint{6.365357in}{4.169151in}}%
\pgfpathlineto{\pgfqpoint{6.339084in}{4.058263in}}%
\pgfpathlineto{\pgfqpoint{6.339084in}{4.058263in}}%
\pgfusepath{stroke}%
\end{pgfscope}%
\begin{pgfscope}%
\pgfpathrectangle{\pgfqpoint{0.766095in}{0.571603in}}{\pgfqpoint{6.973465in}{5.225635in}}%
\pgfusepath{clip}%
\pgfsetbuttcap%
\pgfsetroundjoin%
\pgfsetlinewidth{1.505625pt}%
\definecolor{currentstroke}{rgb}{0.255645,0.260703,0.528312}%
\pgfsetstrokecolor{currentstroke}%
\pgfsetdash{}{0pt}%
\pgfpathmoveto{\pgfqpoint{6.264342in}{3.754455in}}%
\pgfpathlineto{\pgfqpoint{6.246101in}{3.670221in}}%
\pgfpathlineto{\pgfqpoint{6.230990in}{3.591443in}}%
\pgfpathlineto{\pgfqpoint{6.217950in}{3.512664in}}%
\pgfpathlineto{\pgfqpoint{6.207394in}{3.433886in}}%
\pgfpathlineto{\pgfqpoint{6.199548in}{3.355107in}}%
\pgfpathlineto{\pgfqpoint{6.194581in}{3.276329in}}%
\pgfpathlineto{\pgfqpoint{6.192726in}{3.197551in}}%
\pgfpathlineto{\pgfqpoint{6.193314in}{3.145032in}}%
\pgfpathlineto{\pgfqpoint{6.195417in}{3.092513in}}%
\pgfpathlineto{\pgfqpoint{6.199069in}{3.039994in}}%
\pgfpathlineto{\pgfqpoint{6.204290in}{2.987475in}}%
\pgfpathlineto{\pgfqpoint{6.211148in}{2.934956in}}%
\pgfpathlineto{\pgfqpoint{6.219687in}{2.882437in}}%
\pgfpathlineto{\pgfqpoint{6.232731in}{2.817327in}}%
\pgfpathlineto{\pgfqpoint{6.241895in}{2.777399in}}%
\pgfpathlineto{\pgfqpoint{6.255608in}{2.724880in}}%
\pgfpathlineto{\pgfqpoint{6.271131in}{2.672361in}}%
\pgfpathlineto{\pgfqpoint{6.288406in}{2.619842in}}%
\pgfpathlineto{\pgfqpoint{6.307554in}{2.567323in}}%
\pgfpathlineto{\pgfqpoint{6.328510in}{2.514804in}}%
\pgfpathlineto{\pgfqpoint{6.351335in}{2.462285in}}%
\pgfpathlineto{\pgfqpoint{6.376070in}{2.409766in}}%
\pgfpathlineto{\pgfqpoint{6.407944in}{2.347392in}}%
\pgfpathlineto{\pgfqpoint{6.431155in}{2.304729in}}%
\pgfpathlineto{\pgfqpoint{6.461575in}{2.252210in}}%
\pgfpathlineto{\pgfqpoint{6.493927in}{2.199691in}}%
\pgfpathlineto{\pgfqpoint{6.528216in}{2.147172in}}%
\pgfpathlineto{\pgfqpoint{6.564450in}{2.094653in}}%
\pgfpathlineto{\pgfqpoint{6.622486in}{2.015874in}}%
\pgfpathlineto{\pgfqpoint{6.663594in}{1.963355in}}%
\pgfpathlineto{\pgfqpoint{6.728979in}{1.884577in}}%
\pgfpathlineto{\pgfqpoint{6.793412in}{1.811747in}}%
\pgfpathlineto{\pgfqpoint{6.828454in}{1.773801in}}%
\pgfpathlineto{\pgfqpoint{6.873014in}{1.727020in}}%
\pgfpathlineto{\pgfqpoint{6.951687in}{1.648242in}}%
\pgfpathlineto{\pgfqpoint{7.034829in}{1.569463in}}%
\pgfpathlineto{\pgfqpoint{7.122357in}{1.490685in}}%
\pgfpathlineto{\pgfqpoint{7.214363in}{1.411906in}}%
\pgfpathlineto{\pgfqpoint{7.310715in}{1.333128in}}%
\pgfpathlineto{\pgfqpoint{7.411476in}{1.254350in}}%
\pgfpathlineto{\pgfqpoint{7.516615in}{1.175571in}}%
\pgfpathlineto{\pgfqpoint{7.634433in}{1.090950in}}%
\pgfpathlineto{\pgfqpoint{7.739560in}{1.018270in}}%
\pgfpathlineto{\pgfqpoint{7.739560in}{1.018270in}}%
\pgfusepath{stroke}%
\end{pgfscope}%
\begin{pgfscope}%
\pgfpathrectangle{\pgfqpoint{0.766095in}{0.571603in}}{\pgfqpoint{6.973465in}{5.225635in}}%
\pgfusepath{clip}%
\pgfsetbuttcap%
\pgfsetroundjoin%
\pgfsetlinewidth{1.505625pt}%
\definecolor{currentstroke}{rgb}{0.246811,0.283237,0.535941}%
\pgfsetstrokecolor{currentstroke}%
\pgfsetdash{}{0pt}%
\pgfpathmoveto{\pgfqpoint{1.345956in}{0.571603in}}%
\pgfpathlineto{\pgfqpoint{1.326776in}{0.583736in}}%
\pgfpathlineto{\pgfqpoint{1.304501in}{0.597863in}}%
\pgfpathlineto{\pgfqpoint{1.291733in}{0.606125in}}%
\pgfpathlineto{\pgfqpoint{1.264001in}{0.624122in}}%
\pgfpathlineto{\pgfqpoint{1.256691in}{0.628963in}}%
\pgfpathlineto{\pgfqpoint{1.224447in}{0.650382in}}%
\pgfpathlineto{\pgfqpoint{1.221648in}{0.652279in}}%
\pgfpathlineto{\pgfqpoint{1.186605in}{0.676118in}}%
\pgfpathlineto{\pgfqpoint{1.185840in}{0.676641in}}%
\pgfpathlineto{\pgfqpoint{1.151563in}{0.700536in}}%
\pgfpathlineto{\pgfqpoint{1.148183in}{0.702901in}}%
\pgfpathlineto{\pgfqpoint{1.116520in}{0.725509in}}%
\pgfpathlineto{\pgfqpoint{1.111427in}{0.729160in}}%
\pgfpathlineto{\pgfqpoint{1.081478in}{0.751072in}}%
\pgfpathlineto{\pgfqpoint{1.075559in}{0.755420in}}%
\pgfpathlineto{\pgfqpoint{1.046435in}{0.777259in}}%
\pgfpathlineto{\pgfqpoint{1.040566in}{0.781679in}}%
\pgfpathlineto{\pgfqpoint{1.027841in}{0.791463in}}%
\pgfusepath{stroke}%
\end{pgfscope}%
\begin{pgfscope}%
\pgfpathrectangle{\pgfqpoint{0.766095in}{0.571603in}}{\pgfqpoint{6.973465in}{5.225635in}}%
\pgfusepath{clip}%
\pgfsetbuttcap%
\pgfsetroundjoin%
\pgfsetlinewidth{1.505625pt}%
\definecolor{currentstroke}{rgb}{0.246811,0.283237,0.535941}%
\pgfsetstrokecolor{currentstroke}%
\pgfsetdash{}{0pt}%
\pgfpathmoveto{\pgfqpoint{0.789648in}{0.992704in}}%
\pgfpathlineto{\pgfqpoint{0.766095in}{1.015002in}}%
\pgfusepath{stroke}%
\end{pgfscope}%
\begin{pgfscope}%
\pgfpathrectangle{\pgfqpoint{0.766095in}{0.571603in}}{\pgfqpoint{6.973465in}{5.225635in}}%
\pgfusepath{clip}%
\pgfsetbuttcap%
\pgfsetroundjoin%
\pgfsetlinewidth{1.505625pt}%
\definecolor{currentstroke}{rgb}{0.246811,0.283237,0.535941}%
\pgfsetstrokecolor{currentstroke}%
\pgfsetdash{}{0pt}%
\pgfpathmoveto{\pgfqpoint{0.766095in}{3.639454in}}%
\pgfpathlineto{\pgfqpoint{0.770765in}{3.643962in}}%
\pgfpathlineto{\pgfqpoint{0.798518in}{3.670221in}}%
\pgfpathlineto{\pgfqpoint{0.801138in}{3.672651in}}%
\pgfpathlineto{\pgfqpoint{0.827122in}{3.696481in}}%
\pgfpathlineto{\pgfqpoint{0.836180in}{3.704632in}}%
\pgfpathlineto{\pgfqpoint{0.856524in}{3.722740in}}%
\pgfpathlineto{\pgfqpoint{0.871223in}{3.735581in}}%
\pgfpathlineto{\pgfqpoint{0.886750in}{3.749000in}}%
\pgfpathlineto{\pgfqpoint{0.906265in}{3.765556in}}%
\pgfpathlineto{\pgfqpoint{0.917825in}{3.775259in}}%
\pgfpathlineto{\pgfqpoint{0.941308in}{3.794612in}}%
\pgfpathlineto{\pgfqpoint{0.949776in}{3.801519in}}%
\pgfpathlineto{\pgfqpoint{0.976350in}{3.822801in}}%
\pgfpathlineto{\pgfqpoint{0.982628in}{3.827778in}}%
\pgfpathlineto{\pgfqpoint{1.011393in}{3.850173in}}%
\pgfpathlineto{\pgfqpoint{1.016407in}{3.854037in}}%
\pgfpathlineto{\pgfqpoint{1.046435in}{3.876772in}}%
\pgfpathlineto{\pgfqpoint{1.051137in}{3.880297in}}%
\pgfpathlineto{\pgfqpoint{1.081478in}{3.902641in}}%
\pgfpathlineto{\pgfqpoint{1.086845in}{3.906556in}}%
\pgfpathlineto{\pgfqpoint{1.116520in}{3.927822in}}%
\pgfpathlineto{\pgfqpoint{1.123554in}{3.932816in}}%
\pgfpathlineto{\pgfqpoint{1.151563in}{3.952352in}}%
\pgfpathlineto{\pgfqpoint{1.161290in}{3.959075in}}%
\pgfpathlineto{\pgfqpoint{1.186605in}{3.976268in}}%
\pgfpathlineto{\pgfqpoint{1.200074in}{3.985335in}}%
\pgfpathlineto{\pgfqpoint{1.221648in}{3.999605in}}%
\pgfpathlineto{\pgfqpoint{1.239932in}{4.011594in}}%
\pgfpathlineto{\pgfqpoint{1.256691in}{4.022394in}}%
\pgfpathlineto{\pgfqpoint{1.280884in}{4.037854in}}%
\pgfpathlineto{\pgfqpoint{1.291733in}{4.044667in}}%
\pgfpathlineto{\pgfqpoint{1.322953in}{4.064113in}}%
\pgfpathlineto{\pgfqpoint{1.326776in}{4.066453in}}%
\pgfpathlineto{\pgfqpoint{1.361818in}{4.087716in}}%
\pgfpathlineto{\pgfqpoint{1.366236in}{4.090373in}}%
\pgfpathlineto{\pgfqpoint{1.396861in}{4.108476in}}%
\pgfpathlineto{\pgfqpoint{1.410765in}{4.116632in}}%
\pgfpathlineto{\pgfqpoint{1.431903in}{4.128820in}}%
\pgfpathlineto{\pgfqpoint{1.456490in}{4.142892in}}%
\pgfpathlineto{\pgfqpoint{1.466946in}{4.148774in}}%
\pgfpathlineto{\pgfqpoint{1.501988in}{4.168340in}}%
\pgfpathlineto{\pgfqpoint{1.503455in}{4.169151in}}%
\pgfpathlineto{\pgfqpoint{1.537031in}{4.187404in}}%
\pgfpathlineto{\pgfqpoint{1.551861in}{4.195411in}}%
\pgfpathlineto{\pgfqpoint{1.572073in}{4.206137in}}%
\pgfpathlineto{\pgfqpoint{1.601536in}{4.221670in}}%
\pgfpathlineto{\pgfqpoint{1.607116in}{4.224562in}}%
\pgfpathlineto{\pgfqpoint{1.642158in}{4.242565in}}%
\pgfpathlineto{\pgfqpoint{1.652679in}{4.247930in}}%
\pgfpathlineto{\pgfqpoint{1.677201in}{4.260220in}}%
\pgfpathlineto{\pgfqpoint{1.705241in}{4.274189in}}%
\pgfpathlineto{\pgfqpoint{1.712244in}{4.277618in}}%
\pgfpathlineto{\pgfqpoint{1.747286in}{4.294634in}}%
\pgfpathlineto{\pgfqpoint{1.759348in}{4.300449in}}%
\pgfpathlineto{\pgfqpoint{1.782329in}{4.311337in}}%
\pgfpathlineto{\pgfqpoint{1.814944in}{4.326708in}}%
\pgfpathlineto{\pgfqpoint{1.817371in}{4.327832in}}%
\pgfpathlineto{\pgfqpoint{1.852414in}{4.343910in}}%
\pgfpathlineto{\pgfqpoint{1.872265in}{4.352967in}}%
\pgfpathlineto{\pgfqpoint{1.887456in}{4.359780in}}%
\pgfpathlineto{\pgfqpoint{1.922499in}{4.375389in}}%
\pgfpathlineto{\pgfqpoint{1.931187in}{4.379227in}}%
\pgfpathlineto{\pgfqpoint{1.957541in}{4.390670in}}%
\pgfpathlineto{\pgfqpoint{1.991814in}{4.405486in}}%
\pgfpathlineto{\pgfqpoint{1.992584in}{4.405813in}}%
\pgfpathlineto{\pgfqpoint{2.027626in}{4.420552in}}%
\pgfpathlineto{\pgfqpoint{2.054355in}{4.431746in}}%
\pgfpathlineto{\pgfqpoint{2.062669in}{4.435168in}}%
\pgfpathlineto{\pgfqpoint{2.097711in}{4.449469in}}%
\pgfpathlineto{\pgfqpoint{2.118746in}{4.458005in}}%
\pgfpathlineto{\pgfqpoint{2.132754in}{4.463592in}}%
\pgfpathlineto{\pgfqpoint{2.167797in}{4.477467in}}%
\pgfpathlineto{\pgfqpoint{2.185081in}{4.484265in}}%
\pgfpathlineto{\pgfqpoint{2.202839in}{4.491129in}}%
\pgfpathlineto{\pgfqpoint{2.237882in}{4.504586in}}%
\pgfpathlineto{\pgfqpoint{2.253457in}{4.510524in}}%
\pgfpathlineto{\pgfqpoint{2.272924in}{4.517818in}}%
\pgfpathlineto{\pgfqpoint{2.307967in}{4.530866in}}%
\pgfpathlineto{\pgfqpoint{2.323973in}{4.536784in}}%
\pgfpathlineto{\pgfqpoint{2.343009in}{4.543699in}}%
\pgfpathlineto{\pgfqpoint{2.378052in}{4.556347in}}%
\pgfpathlineto{\pgfqpoint{2.396727in}{4.563043in}}%
\pgfpathlineto{\pgfqpoint{2.413094in}{4.568809in}}%
\pgfpathlineto{\pgfqpoint{2.448137in}{4.581066in}}%
\pgfpathlineto{\pgfqpoint{2.471819in}{4.589303in}}%
\pgfpathlineto{\pgfqpoint{2.483179in}{4.593185in}}%
\pgfpathlineto{\pgfqpoint{2.518222in}{4.605057in}}%
\pgfpathlineto{\pgfqpoint{2.549349in}{4.615562in}}%
\pgfpathlineto{\pgfqpoint{2.553264in}{4.616860in}}%
\pgfpathlineto{\pgfqpoint{2.588307in}{4.628357in}}%
\pgfpathlineto{\pgfqpoint{2.623350in}{4.639818in}}%
\pgfpathlineto{\pgfqpoint{2.629539in}{4.641822in}}%
\pgfpathlineto{\pgfqpoint{2.658392in}{4.650997in}}%
\pgfpathlineto{\pgfqpoint{2.693435in}{4.662089in}}%
\pgfpathlineto{\pgfqpoint{2.712503in}{4.668081in}}%
\pgfpathlineto{\pgfqpoint{2.728477in}{4.673012in}}%
\pgfpathlineto{\pgfqpoint{2.763520in}{4.683741in}}%
\pgfpathlineto{\pgfqpoint{2.798262in}{4.694341in}}%
\pgfpathlineto{\pgfqpoint{2.798562in}{4.694431in}}%
\pgfpathlineto{\pgfqpoint{2.833605in}{4.704804in}}%
\pgfpathlineto{\pgfqpoint{2.868647in}{4.715140in}}%
\pgfpathlineto{\pgfqpoint{2.887304in}{4.720600in}}%
\pgfpathlineto{\pgfqpoint{2.903690in}{4.725309in}}%
\pgfpathlineto{\pgfqpoint{2.938732in}{4.735295in}}%
\pgfpathlineto{\pgfqpoint{2.973775in}{4.745242in}}%
\pgfpathlineto{\pgfqpoint{2.979540in}{4.746860in}}%
\pgfpathlineto{\pgfqpoint{3.008818in}{4.754927in}}%
\pgfpathlineto{\pgfqpoint{3.043860in}{4.764529in}}%
\pgfpathlineto{\pgfqpoint{3.075369in}{4.773119in}}%
\pgfpathlineto{\pgfqpoint{3.078903in}{4.774065in}}%
\pgfpathlineto{\pgfqpoint{3.113945in}{4.783328in}}%
\pgfpathlineto{\pgfqpoint{3.148988in}{4.792550in}}%
\pgfpathlineto{\pgfqpoint{3.175114in}{4.799378in}}%
\pgfpathlineto{\pgfqpoint{3.184030in}{4.801666in}}%
\pgfpathlineto{\pgfqpoint{3.219073in}{4.810553in}}%
\pgfpathlineto{\pgfqpoint{3.254115in}{4.819398in}}%
\pgfpathlineto{\pgfqpoint{3.279026in}{4.825638in}}%
\pgfpathlineto{\pgfqpoint{3.289158in}{4.828129in}}%
\pgfpathlineto{\pgfqpoint{3.324200in}{4.836643in}}%
\pgfpathlineto{\pgfqpoint{3.359243in}{4.845113in}}%
\pgfpathlineto{\pgfqpoint{3.382745in}{4.850753in}}%
\pgfusepath{stroke}%
\end{pgfscope}%
\begin{pgfscope}%
\pgfpathrectangle{\pgfqpoint{0.766095in}{0.571603in}}{\pgfqpoint{6.973465in}{5.225635in}}%
\pgfusepath{clip}%
\pgfsetbuttcap%
\pgfsetroundjoin%
\pgfsetlinewidth{1.505625pt}%
\definecolor{currentstroke}{rgb}{0.246811,0.283237,0.535941}%
\pgfsetstrokecolor{currentstroke}%
\pgfsetdash{}{0pt}%
\pgfpathmoveto{\pgfqpoint{3.686618in}{4.918322in}}%
\pgfpathlineto{\pgfqpoint{3.814796in}{4.944291in}}%
\pgfpathlineto{\pgfqpoint{3.954966in}{4.970991in}}%
\pgfpathlineto{\pgfqpoint{4.095136in}{4.995939in}}%
\pgfpathlineto{\pgfqpoint{4.235306in}{5.019147in}}%
\pgfpathlineto{\pgfqpoint{4.410519in}{5.045672in}}%
\pgfpathlineto{\pgfqpoint{4.585732in}{5.069400in}}%
\pgfpathlineto{\pgfqpoint{4.760944in}{5.090246in}}%
\pgfpathlineto{\pgfqpoint{4.936157in}{5.107879in}}%
\pgfpathlineto{\pgfqpoint{5.076327in}{5.119638in}}%
\pgfpathlineto{\pgfqpoint{5.216497in}{5.129013in}}%
\pgfpathlineto{\pgfqpoint{5.356668in}{5.135913in}}%
\pgfpathlineto{\pgfqpoint{5.461795in}{5.139233in}}%
\pgfpathlineto{\pgfqpoint{5.572415in}{5.140752in}}%
\pgfpathlineto{\pgfqpoint{5.672050in}{5.140203in}}%
\pgfpathlineto{\pgfqpoint{5.777178in}{5.137331in}}%
\pgfpathlineto{\pgfqpoint{5.882306in}{5.131780in}}%
\pgfpathlineto{\pgfqpoint{5.987433in}{5.123130in}}%
\pgfpathlineto{\pgfqpoint{6.064581in}{5.114492in}}%
\pgfpathlineto{\pgfqpoint{6.127603in}{5.105411in}}%
\pgfpathlineto{\pgfqpoint{6.197689in}{5.093243in}}%
\pgfpathlineto{\pgfqpoint{6.232731in}{5.086159in}}%
\pgfpathlineto{\pgfqpoint{6.302816in}{5.069296in}}%
\pgfpathlineto{\pgfqpoint{6.337859in}{5.059553in}}%
\pgfpathlineto{\pgfqpoint{6.409521in}{5.035714in}}%
\pgfpathlineto{\pgfqpoint{6.472030in}{5.009454in}}%
\pgfpathlineto{\pgfqpoint{6.513071in}{4.988376in}}%
\pgfpathlineto{\pgfqpoint{6.522405in}{4.983195in}}%
\pgfpathlineto{\pgfqpoint{6.563620in}{4.956935in}}%
\pgfpathlineto{\pgfqpoint{6.597661in}{4.930676in}}%
\pgfpathlineto{\pgfqpoint{6.625900in}{4.904416in}}%
\pgfpathlineto{\pgfqpoint{6.653242in}{4.873034in}}%
\pgfpathlineto{\pgfqpoint{6.668598in}{4.851897in}}%
\pgfpathlineto{\pgfqpoint{6.688284in}{4.818218in}}%
\pgfpathlineto{\pgfqpoint{6.697383in}{4.799378in}}%
\pgfpathlineto{\pgfqpoint{6.707663in}{4.773119in}}%
\pgfpathlineto{\pgfqpoint{6.715713in}{4.746860in}}%
\pgfpathlineto{\pgfqpoint{6.723327in}{4.711254in}}%
\pgfpathlineto{\pgfqpoint{6.726020in}{4.694341in}}%
\pgfpathlineto{\pgfqpoint{6.728685in}{4.668081in}}%
\pgfpathlineto{\pgfqpoint{6.729958in}{4.641822in}}%
\pgfpathlineto{\pgfqpoint{6.729987in}{4.615562in}}%
\pgfpathlineto{\pgfqpoint{6.726823in}{4.563043in}}%
\pgfpathlineto{\pgfqpoint{6.720043in}{4.510524in}}%
\pgfpathlineto{\pgfqpoint{6.710342in}{4.458005in}}%
\pgfpathlineto{\pgfqpoint{6.698387in}{4.405486in}}%
\pgfpathlineto{\pgfqpoint{6.684640in}{4.352967in}}%
\pgfpathlineto{\pgfqpoint{6.661450in}{4.274189in}}%
\pgfpathlineto{\pgfqpoint{6.627580in}{4.169151in}}%
\pgfpathlineto{\pgfqpoint{6.523344in}{3.854037in}}%
\pgfpathlineto{\pgfqpoint{6.492162in}{3.749000in}}%
\pgfpathlineto{\pgfqpoint{6.470988in}{3.670221in}}%
\pgfpathlineto{\pgfqpoint{6.452026in}{3.591443in}}%
\pgfpathlineto{\pgfqpoint{6.435583in}{3.512664in}}%
\pgfpathlineto{\pgfqpoint{6.421874in}{3.433886in}}%
\pgfpathlineto{\pgfqpoint{6.411173in}{3.355107in}}%
\pgfpathlineto{\pgfqpoint{6.403601in}{3.276329in}}%
\pgfpathlineto{\pgfqpoint{6.400388in}{3.223810in}}%
\pgfpathlineto{\pgfqpoint{6.398709in}{3.171291in}}%
\pgfpathlineto{\pgfqpoint{6.398604in}{3.118772in}}%
\pgfpathlineto{\pgfqpoint{6.400115in}{3.066253in}}%
\pgfpathlineto{\pgfqpoint{6.403280in}{3.013734in}}%
\pgfpathlineto{\pgfqpoint{6.408136in}{2.961215in}}%
\pgfpathlineto{\pgfqpoint{6.414663in}{2.908696in}}%
\pgfpathlineto{\pgfqpoint{6.422945in}{2.856177in}}%
\pgfpathlineto{\pgfqpoint{6.433021in}{2.803659in}}%
\pgfpathlineto{\pgfqpoint{6.444911in}{2.751140in}}%
\pgfpathlineto{\pgfqpoint{6.458567in}{2.698621in}}%
\pgfpathlineto{\pgfqpoint{6.478029in}{2.634027in}}%
\pgfpathlineto{\pgfqpoint{6.491490in}{2.593583in}}%
\pgfpathlineto{\pgfqpoint{6.513071in}{2.535306in}}%
\pgfpathlineto{\pgfqpoint{6.531934in}{2.488545in}}%
\pgfpathlineto{\pgfqpoint{6.555033in}{2.436026in}}%
\pgfpathlineto{\pgfqpoint{6.583156in}{2.377382in}}%
\pgfpathlineto{\pgfqpoint{6.606989in}{2.330988in}}%
\pgfpathlineto{\pgfqpoint{6.635889in}{2.278469in}}%
\pgfpathlineto{\pgfqpoint{6.666757in}{2.225950in}}%
\pgfpathlineto{\pgfqpoint{6.699598in}{2.173431in}}%
\pgfpathlineto{\pgfqpoint{6.734415in}{2.120912in}}%
\pgfpathlineto{\pgfqpoint{6.771216in}{2.068393in}}%
\pgfpathlineto{\pgfqpoint{6.828454in}{1.991879in}}%
\pgfpathlineto{\pgfqpoint{6.863497in}{1.947627in}}%
\pgfpathlineto{\pgfqpoint{6.898539in}{1.905012in}}%
\pgfpathlineto{\pgfqpoint{6.938416in}{1.858318in}}%
\pgfpathlineto{\pgfqpoint{7.003667in}{1.785757in}}%
\pgfpathlineto{\pgfqpoint{7.038709in}{1.748414in}}%
\pgfpathlineto{\pgfqpoint{7.108795in}{1.676759in}}%
\pgfpathlineto{\pgfqpoint{7.164855in}{1.621982in}}%
\pgfpathlineto{\pgfqpoint{7.249418in}{1.543204in}}%
\pgfpathlineto{\pgfqpoint{7.338426in}{1.464425in}}%
\pgfpathlineto{\pgfqpoint{7.431986in}{1.385647in}}%
\pgfpathlineto{\pgfqpoint{7.530050in}{1.306869in}}%
\pgfpathlineto{\pgfqpoint{7.634433in}{1.226694in}}%
\pgfpathlineto{\pgfqpoint{7.739560in}{1.149304in}}%
\pgfpathlineto{\pgfqpoint{7.739560in}{1.149304in}}%
\pgfusepath{stroke}%
\end{pgfscope}%
\begin{pgfscope}%
\pgfpathrectangle{\pgfqpoint{0.766095in}{0.571603in}}{\pgfqpoint{6.973465in}{5.225635in}}%
\pgfusepath{clip}%
\pgfsetbuttcap%
\pgfsetroundjoin%
\pgfsetlinewidth{1.505625pt}%
\definecolor{currentstroke}{rgb}{0.237441,0.305202,0.541921}%
\pgfsetstrokecolor{currentstroke}%
\pgfsetdash{}{0pt}%
\pgfpathmoveto{\pgfqpoint{1.199166in}{0.571603in}}%
\pgfpathlineto{\pgfqpoint{1.186605in}{0.579687in}}%
\pgfpathlineto{\pgfqpoint{1.175844in}{0.586633in}}%
\pgfusepath{stroke}%
\end{pgfscope}%
\begin{pgfscope}%
\pgfpathrectangle{\pgfqpoint{0.766095in}{0.571603in}}{\pgfqpoint{6.973465in}{5.225635in}}%
\pgfusepath{clip}%
\pgfsetbuttcap%
\pgfsetroundjoin%
\pgfsetlinewidth{1.505625pt}%
\definecolor{currentstroke}{rgb}{0.237441,0.305202,0.541921}%
\pgfsetstrokecolor{currentstroke}%
\pgfsetdash{}{0pt}%
\pgfpathmoveto{\pgfqpoint{0.920496in}{0.765420in}}%
\pgfpathlineto{\pgfqpoint{0.906265in}{0.776278in}}%
\pgfpathlineto{\pgfqpoint{0.899219in}{0.781679in}}%
\pgfpathlineto{\pgfqpoint{0.871223in}{0.803579in}}%
\pgfpathlineto{\pgfqpoint{0.865676in}{0.807939in}}%
\pgfpathlineto{\pgfqpoint{0.836180in}{0.831602in}}%
\pgfpathlineto{\pgfqpoint{0.832961in}{0.834198in}}%
\pgfpathlineto{\pgfqpoint{0.801138in}{0.860391in}}%
\pgfpathlineto{\pgfqpoint{0.801057in}{0.860458in}}%
\pgfpathlineto{\pgfqpoint{0.770002in}{0.886717in}}%
\pgfpathlineto{\pgfqpoint{0.766095in}{0.890091in}}%
\pgfusepath{stroke}%
\end{pgfscope}%
\begin{pgfscope}%
\pgfpathrectangle{\pgfqpoint{0.766095in}{0.571603in}}{\pgfqpoint{6.973465in}{5.225635in}}%
\pgfusepath{clip}%
\pgfsetbuttcap%
\pgfsetroundjoin%
\pgfsetlinewidth{1.505625pt}%
\definecolor{currentstroke}{rgb}{0.237441,0.305202,0.541921}%
\pgfsetstrokecolor{currentstroke}%
\pgfsetdash{}{0pt}%
\pgfpathmoveto{\pgfqpoint{0.766095in}{3.789341in}}%
\pgfpathlineto{\pgfqpoint{0.811702in}{3.827778in}}%
\pgfpathlineto{\pgfqpoint{0.877075in}{3.880297in}}%
\pgfpathlineto{\pgfqpoint{0.946115in}{3.932816in}}%
\pgfpathlineto{\pgfqpoint{1.019013in}{3.985335in}}%
\pgfpathlineto{\pgfqpoint{1.095953in}{4.037854in}}%
\pgfpathlineto{\pgfqpoint{1.186605in}{4.096315in}}%
\pgfpathlineto{\pgfqpoint{1.262750in}{4.142892in}}%
\pgfpathlineto{\pgfqpoint{1.361818in}{4.200276in}}%
\pgfpathlineto{\pgfqpoint{1.466946in}{4.257673in}}%
\pgfpathlineto{\pgfqpoint{1.549334in}{4.300449in}}%
\pgfpathlineto{\pgfqpoint{1.655742in}{4.352967in}}%
\pgfpathlineto{\pgfqpoint{1.768182in}{4.405486in}}%
\pgfpathlineto{\pgfqpoint{1.887456in}{4.458201in}}%
\pgfpathlineto{\pgfqpoint{2.027626in}{4.516464in}}%
\pgfpathlineto{\pgfqpoint{2.146111in}{4.563043in}}%
\pgfpathlineto{\pgfqpoint{2.287400in}{4.615562in}}%
\pgfpathlineto{\pgfqpoint{2.437423in}{4.668081in}}%
\pgfpathlineto{\pgfqpoint{2.588307in}{4.717806in}}%
\pgfpathlineto{\pgfqpoint{2.693435in}{4.750755in}}%
\pgfpathlineto{\pgfqpoint{2.856599in}{4.799378in}}%
\pgfpathlineto{\pgfqpoint{3.008818in}{4.842041in}}%
\pgfpathlineto{\pgfqpoint{3.148988in}{4.879357in}}%
\pgfpathlineto{\pgfqpoint{3.324200in}{4.923209in}}%
\pgfpathlineto{\pgfqpoint{3.466741in}{4.956935in}}%
\pgfpathlineto{\pgfqpoint{3.639583in}{4.995297in}}%
\pgfpathlineto{\pgfqpoint{3.814796in}{5.031774in}}%
\pgfpathlineto{\pgfqpoint{3.969848in}{5.061973in}}%
\pgfpathlineto{\pgfqpoint{4.130179in}{5.091154in}}%
\pgfpathlineto{\pgfqpoint{4.305391in}{5.120729in}}%
\pgfpathlineto{\pgfqpoint{4.480604in}{5.147928in}}%
\pgfpathlineto{\pgfqpoint{4.655817in}{5.172714in}}%
\pgfpathlineto{\pgfqpoint{4.831030in}{5.195038in}}%
\pgfpathlineto{\pgfqpoint{5.006242in}{5.214672in}}%
\pgfpathlineto{\pgfqpoint{5.146412in}{5.228394in}}%
\pgfpathlineto{\pgfqpoint{5.321625in}{5.242990in}}%
\pgfpathlineto{\pgfqpoint{5.461795in}{5.252264in}}%
\pgfpathlineto{\pgfqpoint{5.601965in}{5.259282in}}%
\pgfpathlineto{\pgfqpoint{5.742136in}{5.263817in}}%
\pgfpathlineto{\pgfqpoint{5.847263in}{5.265349in}}%
\pgfpathlineto{\pgfqpoint{5.952391in}{5.265055in}}%
\pgfpathlineto{\pgfqpoint{6.057518in}{5.262688in}}%
\pgfpathlineto{\pgfqpoint{6.162646in}{5.257955in}}%
\pgfpathlineto{\pgfqpoint{6.267774in}{5.250502in}}%
\pgfpathlineto{\pgfqpoint{6.337859in}{5.243710in}}%
\pgfpathlineto{\pgfqpoint{6.407944in}{5.235030in}}%
\pgfpathlineto{\pgfqpoint{6.478029in}{5.224502in}}%
\pgfpathlineto{\pgfqpoint{6.548114in}{5.211454in}}%
\pgfpathlineto{\pgfqpoint{6.627381in}{5.193271in}}%
\pgfpathlineto{\pgfqpoint{6.688284in}{5.175756in}}%
\pgfpathlineto{\pgfqpoint{6.723327in}{5.164251in}}%
\pgfpathlineto{\pgfqpoint{6.783713in}{5.140752in}}%
\pgfpathlineto{\pgfqpoint{6.828454in}{5.119497in}}%
\pgfpathlineto{\pgfqpoint{6.838169in}{5.114492in}}%
\pgfpathlineto{\pgfqpoint{6.882356in}{5.088233in}}%
\pgfpathlineto{\pgfqpoint{6.918586in}{5.061973in}}%
\pgfpathlineto{\pgfqpoint{6.948397in}{5.035714in}}%
\pgfpathlineto{\pgfqpoint{6.972971in}{5.009454in}}%
\pgfpathlineto{\pgfqpoint{6.993027in}{4.983195in}}%
\pgfpathlineto{\pgfqpoint{7.009372in}{4.956935in}}%
\pgfpathlineto{\pgfqpoint{7.022411in}{4.930676in}}%
\pgfpathlineto{\pgfqpoint{7.032700in}{4.904416in}}%
\pgfpathlineto{\pgfqpoint{7.040516in}{4.878157in}}%
\pgfpathlineto{\pgfqpoint{7.046098in}{4.851897in}}%
\pgfpathlineto{\pgfqpoint{7.049778in}{4.825638in}}%
\pgfpathlineto{\pgfqpoint{7.051759in}{4.799378in}}%
\pgfpathlineto{\pgfqpoint{7.052222in}{4.773119in}}%
\pgfpathlineto{\pgfqpoint{7.051325in}{4.746860in}}%
\pgfpathlineto{\pgfqpoint{7.046012in}{4.694341in}}%
\pgfpathlineto{\pgfqpoint{7.036771in}{4.641822in}}%
\pgfpathlineto{\pgfqpoint{7.024285in}{4.589303in}}%
\pgfpathlineto{\pgfqpoint{7.009320in}{4.536784in}}%
\pgfpathlineto{\pgfqpoint{6.992282in}{4.484265in}}%
\pgfpathlineto{\pgfqpoint{6.963905in}{4.405486in}}%
\pgfpathlineto{\pgfqpoint{6.948866in}{4.366529in}}%
\pgfpathlineto{\pgfqpoint{6.948866in}{4.366529in}}%
\pgfusepath{stroke}%
\end{pgfscope}%
\begin{pgfscope}%
\pgfpathrectangle{\pgfqpoint{0.766095in}{0.571603in}}{\pgfqpoint{6.973465in}{5.225635in}}%
\pgfusepath{clip}%
\pgfsetbuttcap%
\pgfsetroundjoin%
\pgfsetlinewidth{1.505625pt}%
\definecolor{currentstroke}{rgb}{0.237441,0.305202,0.541921}%
\pgfsetstrokecolor{currentstroke}%
\pgfsetdash{}{0pt}%
\pgfpathmoveto{\pgfqpoint{6.829985in}{4.077292in}}%
\pgfpathlineto{\pgfqpoint{6.782387in}{3.959075in}}%
\pgfpathlineto{\pgfqpoint{6.742670in}{3.854037in}}%
\pgfpathlineto{\pgfqpoint{6.715040in}{3.775259in}}%
\pgfpathlineto{\pgfqpoint{6.688284in}{3.692095in}}%
\pgfpathlineto{\pgfqpoint{6.666594in}{3.617702in}}%
\pgfpathlineto{\pgfqpoint{6.646337in}{3.538924in}}%
\pgfpathlineto{\pgfqpoint{6.628992in}{3.460145in}}%
\pgfpathlineto{\pgfqpoint{6.614807in}{3.381367in}}%
\pgfpathlineto{\pgfqpoint{6.607141in}{3.328848in}}%
\pgfpathlineto{\pgfqpoint{6.601009in}{3.276329in}}%
\pgfpathlineto{\pgfqpoint{6.596445in}{3.223810in}}%
\pgfpathlineto{\pgfqpoint{6.593486in}{3.171291in}}%
\pgfpathlineto{\pgfqpoint{6.592165in}{3.118772in}}%
\pgfpathlineto{\pgfqpoint{6.592518in}{3.066253in}}%
\pgfpathlineto{\pgfqpoint{6.594578in}{3.013734in}}%
\pgfpathlineto{\pgfqpoint{6.598378in}{2.961215in}}%
\pgfpathlineto{\pgfqpoint{6.603952in}{2.908696in}}%
\pgfpathlineto{\pgfqpoint{6.611335in}{2.856177in}}%
\pgfpathlineto{\pgfqpoint{6.620541in}{2.803659in}}%
\pgfpathlineto{\pgfqpoint{6.631551in}{2.751140in}}%
\pgfpathlineto{\pgfqpoint{6.644461in}{2.698621in}}%
\pgfpathlineto{\pgfqpoint{6.659256in}{2.646102in}}%
\pgfpathlineto{\pgfqpoint{6.675936in}{2.593583in}}%
\pgfpathlineto{\pgfqpoint{6.694563in}{2.541064in}}%
\pgfpathlineto{\pgfqpoint{6.715109in}{2.488545in}}%
\pgfpathlineto{\pgfqpoint{6.737602in}{2.436026in}}%
\pgfpathlineto{\pgfqpoint{6.762091in}{2.383507in}}%
\pgfpathlineto{\pgfqpoint{6.793412in}{2.321838in}}%
\pgfpathlineto{\pgfqpoint{6.816943in}{2.278469in}}%
\pgfpathlineto{\pgfqpoint{6.847364in}{2.225950in}}%
\pgfpathlineto{\pgfqpoint{6.879794in}{2.173431in}}%
\pgfpathlineto{\pgfqpoint{6.914237in}{2.120912in}}%
\pgfpathlineto{\pgfqpoint{6.950698in}{2.068393in}}%
\pgfpathlineto{\pgfqpoint{7.003667in}{1.996902in}}%
\pgfpathlineto{\pgfqpoint{7.038709in}{1.952135in}}%
\pgfpathlineto{\pgfqpoint{7.073752in}{1.909087in}}%
\pgfpathlineto{\pgfqpoint{7.116848in}{1.858318in}}%
\pgfpathlineto{\pgfqpoint{7.178880in}{1.789011in}}%
\pgfpathlineto{\pgfqpoint{7.213922in}{1.751440in}}%
\pgfpathlineto{\pgfqpoint{7.284007in}{1.679508in}}%
\pgfpathlineto{\pgfqpoint{7.319050in}{1.644926in}}%
\pgfpathlineto{\pgfqpoint{7.398563in}{1.569463in}}%
\pgfpathlineto{\pgfqpoint{7.486118in}{1.490685in}}%
\pgfpathlineto{\pgfqpoint{7.578247in}{1.411906in}}%
\pgfpathlineto{\pgfqpoint{7.674976in}{1.333128in}}%
\pgfpathlineto{\pgfqpoint{7.739560in}{1.282505in}}%
\pgfpathlineto{\pgfqpoint{7.739560in}{1.282505in}}%
\pgfusepath{stroke}%
\end{pgfscope}%
\begin{pgfscope}%
\pgfpathrectangle{\pgfqpoint{0.766095in}{0.571603in}}{\pgfqpoint{6.973465in}{5.225635in}}%
\pgfusepath{clip}%
\pgfsetbuttcap%
\pgfsetroundjoin%
\pgfsetlinewidth{1.505625pt}%
\definecolor{currentstroke}{rgb}{0.227802,0.326594,0.546532}%
\pgfsetstrokecolor{currentstroke}%
\pgfsetdash{}{0pt}%
\pgfpathmoveto{\pgfqpoint{1.060043in}{0.571603in}}%
\pgfpathlineto{\pgfqpoint{1.046435in}{0.580516in}}%
\pgfpathlineto{\pgfqpoint{1.034915in}{0.588085in}}%
\pgfusepath{stroke}%
\end{pgfscope}%
\begin{pgfscope}%
\pgfpathrectangle{\pgfqpoint{0.766095in}{0.571603in}}{\pgfqpoint{6.973465in}{5.225635in}}%
\pgfusepath{clip}%
\pgfsetbuttcap%
\pgfsetroundjoin%
\pgfsetlinewidth{1.505625pt}%
\definecolor{currentstroke}{rgb}{0.227802,0.326594,0.546532}%
\pgfsetstrokecolor{currentstroke}%
\pgfsetdash{}{0pt}%
\pgfpathmoveto{\pgfqpoint{0.781215in}{0.769224in}}%
\pgfpathlineto{\pgfqpoint{0.766095in}{0.780964in}}%
\pgfusepath{stroke}%
\end{pgfscope}%
\begin{pgfscope}%
\pgfpathrectangle{\pgfqpoint{0.766095in}{0.571603in}}{\pgfqpoint{6.973465in}{5.225635in}}%
\pgfusepath{clip}%
\pgfsetbuttcap%
\pgfsetroundjoin%
\pgfsetlinewidth{1.505625pt}%
\definecolor{currentstroke}{rgb}{0.227802,0.326594,0.546532}%
\pgfsetstrokecolor{currentstroke}%
\pgfsetdash{}{0pt}%
\pgfpathmoveto{\pgfqpoint{0.766095in}{3.920029in}}%
\pgfpathlineto{\pgfqpoint{0.782507in}{3.932816in}}%
\pgfpathlineto{\pgfqpoint{0.801138in}{3.947074in}}%
\pgfpathlineto{\pgfqpoint{0.816961in}{3.959075in}}%
\pgfpathlineto{\pgfqpoint{0.836180in}{3.973395in}}%
\pgfpathlineto{\pgfqpoint{0.852346in}{3.985335in}}%
\pgfpathlineto{\pgfqpoint{0.871223in}{3.999032in}}%
\pgfpathlineto{\pgfqpoint{0.888684in}{4.011594in}}%
\pgfpathlineto{\pgfqpoint{0.906265in}{4.024022in}}%
\pgfpathlineto{\pgfqpoint{0.925998in}{4.037854in}}%
\pgfpathlineto{\pgfqpoint{0.941308in}{4.048400in}}%
\pgfpathlineto{\pgfqpoint{0.964307in}{4.064113in}}%
\pgfpathlineto{\pgfqpoint{0.976350in}{4.072199in}}%
\pgfpathlineto{\pgfqpoint{1.003635in}{4.090373in}}%
\pgfpathlineto{\pgfqpoint{1.011393in}{4.095451in}}%
\pgfpathlineto{\pgfqpoint{1.044001in}{4.116632in}}%
\pgfpathlineto{\pgfqpoint{1.046435in}{4.118186in}}%
\pgfpathlineto{\pgfqpoint{1.081478in}{4.140373in}}%
\pgfpathlineto{\pgfqpoint{1.085488in}{4.142892in}}%
\pgfpathlineto{\pgfqpoint{1.116520in}{4.162049in}}%
\pgfpathlineto{\pgfqpoint{1.128109in}{4.169151in}}%
\pgfpathlineto{\pgfqpoint{1.151563in}{4.183280in}}%
\pgfpathlineto{\pgfqpoint{1.171842in}{4.195411in}}%
\pgfpathlineto{\pgfqpoint{1.186605in}{4.204091in}}%
\pgfpathlineto{\pgfqpoint{1.216705in}{4.221670in}}%
\pgfpathlineto{\pgfqpoint{1.221648in}{4.224508in}}%
\pgfpathlineto{\pgfqpoint{1.256691in}{4.244469in}}%
\pgfpathlineto{\pgfqpoint{1.262812in}{4.247930in}}%
\pgfpathlineto{\pgfqpoint{1.291733in}{4.264002in}}%
\pgfpathlineto{\pgfqpoint{1.310180in}{4.274189in}}%
\pgfpathlineto{\pgfqpoint{1.326776in}{4.283198in}}%
\pgfpathlineto{\pgfqpoint{1.358742in}{4.300449in}}%
\pgfpathlineto{\pgfqpoint{1.361818in}{4.302080in}}%
\pgfpathlineto{\pgfqpoint{1.396861in}{4.320517in}}%
\pgfpathlineto{\pgfqpoint{1.408702in}{4.326708in}}%
\pgfpathlineto{\pgfqpoint{1.431903in}{4.338632in}}%
\pgfpathlineto{\pgfqpoint{1.459951in}{4.352967in}}%
\pgfpathlineto{\pgfqpoint{1.466946in}{4.356482in}}%
\pgfpathlineto{\pgfqpoint{1.501988in}{4.373959in}}%
\pgfpathlineto{\pgfqpoint{1.512621in}{4.379227in}}%
\pgfpathlineto{\pgfqpoint{1.537031in}{4.391116in}}%
\pgfpathlineto{\pgfqpoint{1.566683in}{4.405486in}}%
\pgfpathlineto{\pgfqpoint{1.572073in}{4.408055in}}%
\pgfpathlineto{\pgfqpoint{1.607116in}{4.424621in}}%
\pgfpathlineto{\pgfqpoint{1.622275in}{4.431746in}}%
\pgfpathlineto{\pgfqpoint{1.642158in}{4.440933in}}%
\pgfpathlineto{\pgfqpoint{1.677201in}{4.457048in}}%
\pgfpathlineto{\pgfqpoint{1.679301in}{4.458005in}}%
\pgfpathlineto{\pgfqpoint{1.712244in}{4.472771in}}%
\pgfpathlineto{\pgfqpoint{1.737990in}{4.484265in}}%
\pgfpathlineto{\pgfqpoint{1.747286in}{4.488344in}}%
\pgfpathlineto{\pgfqpoint{1.782329in}{4.503614in}}%
\pgfpathlineto{\pgfqpoint{1.798276in}{4.510524in}}%
\pgfpathlineto{\pgfqpoint{1.817371in}{4.518657in}}%
\pgfpathlineto{\pgfqpoint{1.852414in}{4.533508in}}%
\pgfpathlineto{\pgfqpoint{1.860201in}{4.536784in}}%
\pgfpathlineto{\pgfqpoint{1.887456in}{4.548053in}}%
\pgfpathlineto{\pgfqpoint{1.922499in}{4.562495in}}%
\pgfpathlineto{\pgfqpoint{1.923841in}{4.563043in}}%
\pgfpathlineto{\pgfqpoint{1.957541in}{4.576574in}}%
\pgfpathlineto{\pgfqpoint{1.989327in}{4.589303in}}%
\pgfpathlineto{\pgfqpoint{1.992584in}{4.590584in}}%
\pgfpathlineto{\pgfqpoint{2.027626in}{4.604260in}}%
\pgfpathlineto{\pgfqpoint{2.056675in}{4.615562in}}%
\pgfpathlineto{\pgfqpoint{2.062669in}{4.617854in}}%
\pgfpathlineto{\pgfqpoint{2.097711in}{4.631148in}}%
\pgfpathlineto{\pgfqpoint{2.125938in}{4.641822in}}%
\pgfpathlineto{\pgfqpoint{2.132754in}{4.644355in}}%
\pgfpathlineto{\pgfqpoint{2.167797in}{4.657275in}}%
\pgfpathlineto{\pgfqpoint{2.197190in}{4.668081in}}%
\pgfpathlineto{\pgfqpoint{2.202839in}{4.670122in}}%
\pgfpathlineto{\pgfqpoint{2.237882in}{4.682678in}}%
\pgfpathlineto{\pgfqpoint{2.270507in}{4.694341in}}%
\pgfpathlineto{\pgfqpoint{2.272924in}{4.695190in}}%
\pgfpathlineto{\pgfqpoint{2.307967in}{4.707389in}}%
\pgfpathlineto{\pgfqpoint{2.343009in}{4.719566in}}%
\pgfpathlineto{\pgfqpoint{2.346013in}{4.720600in}}%
\pgfpathlineto{\pgfqpoint{2.378052in}{4.731442in}}%
\pgfpathlineto{\pgfqpoint{2.413094in}{4.743269in}}%
\pgfpathlineto{\pgfqpoint{2.423809in}{4.746860in}}%
\pgfpathlineto{\pgfqpoint{2.448137in}{4.754868in}}%
\pgfpathlineto{\pgfqpoint{2.483179in}{4.766354in}}%
\pgfpathlineto{\pgfqpoint{2.503927in}{4.773119in}}%
\pgfpathlineto{\pgfqpoint{2.518222in}{4.777698in}}%
\pgfpathlineto{\pgfqpoint{2.553264in}{4.788849in}}%
\pgfpathlineto{\pgfqpoint{2.586438in}{4.799378in}}%
\pgfpathlineto{\pgfqpoint{2.588307in}{4.799961in}}%
\pgfpathlineto{\pgfqpoint{2.623350in}{4.810784in}}%
\pgfpathlineto{\pgfqpoint{2.658392in}{4.821584in}}%
\pgfpathlineto{\pgfqpoint{2.671641in}{4.825638in}}%
\pgfpathlineto{\pgfqpoint{2.693435in}{4.832188in}}%
\pgfpathlineto{\pgfqpoint{2.728477in}{4.842666in}}%
\pgfpathlineto{\pgfqpoint{2.759452in}{4.851897in}}%
\pgfpathlineto{\pgfqpoint{2.763520in}{4.853088in}}%
\pgfpathlineto{\pgfqpoint{2.798562in}{4.863250in}}%
\pgfpathlineto{\pgfqpoint{2.833605in}{4.873388in}}%
\pgfpathlineto{\pgfqpoint{2.850197in}{4.878157in}}%
\pgfpathlineto{\pgfqpoint{2.868647in}{4.883364in}}%
\pgfpathlineto{\pgfqpoint{2.903690in}{4.893192in}}%
\pgfpathlineto{\pgfqpoint{2.938732in}{4.902995in}}%
\pgfpathlineto{\pgfqpoint{2.943861in}{4.904416in}}%
\pgfpathlineto{\pgfqpoint{2.973775in}{4.912556in}}%
\pgfpathlineto{\pgfqpoint{3.008818in}{4.922055in}}%
\pgfpathlineto{\pgfqpoint{3.040732in}{4.930676in}}%
\pgfpathlineto{\pgfqpoint{3.043860in}{4.931506in}}%
\pgfpathlineto{\pgfqpoint{3.078903in}{4.940706in}}%
\pgfpathlineto{\pgfqpoint{3.113945in}{4.949879in}}%
\pgfpathlineto{\pgfqpoint{3.141028in}{4.956935in}}%
\pgfpathlineto{\pgfqpoint{3.148988in}{4.958971in}}%
\pgfpathlineto{\pgfqpoint{3.184030in}{4.967851in}}%
\pgfpathlineto{\pgfqpoint{3.219073in}{4.976703in}}%
\pgfpathlineto{\pgfqpoint{3.244906in}{4.983195in}}%
\pgfpathlineto{\pgfqpoint{3.254115in}{4.985466in}}%
\pgfpathlineto{\pgfqpoint{3.289158in}{4.994029in}}%
\pgfpathlineto{\pgfqpoint{3.324200in}{5.002564in}}%
\pgfpathlineto{\pgfqpoint{3.352630in}{5.009454in}}%
\pgfpathlineto{\pgfqpoint{3.359243in}{5.011027in}}%
\pgfpathlineto{\pgfqpoint{3.394285in}{5.019277in}}%
\pgfpathlineto{\pgfqpoint{3.429328in}{5.027497in}}%
\pgfpathlineto{\pgfqpoint{3.464371in}{5.035688in}}%
\pgfpathlineto{\pgfqpoint{3.464484in}{5.035714in}}%
\pgfpathlineto{\pgfqpoint{3.499413in}{5.043628in}}%
\pgfpathlineto{\pgfqpoint{3.522605in}{5.048862in}}%
\pgfusepath{stroke}%
\end{pgfscope}%
\begin{pgfscope}%
\pgfpathrectangle{\pgfqpoint{0.766095in}{0.571603in}}{\pgfqpoint{6.973465in}{5.225635in}}%
\pgfusepath{clip}%
\pgfsetbuttcap%
\pgfsetroundjoin%
\pgfsetlinewidth{1.505625pt}%
\definecolor{currentstroke}{rgb}{0.227802,0.326594,0.546532}%
\pgfsetstrokecolor{currentstroke}%
\pgfsetdash{}{0pt}%
\pgfpathmoveto{\pgfqpoint{3.826999in}{5.114036in}}%
\pgfpathlineto{\pgfqpoint{3.962152in}{5.140752in}}%
\pgfpathlineto{\pgfqpoint{4.165221in}{5.178343in}}%
\pgfpathlineto{\pgfqpoint{4.340434in}{5.208525in}}%
\pgfpathlineto{\pgfqpoint{4.515647in}{5.236583in}}%
\pgfpathlineto{\pgfqpoint{4.690859in}{5.262496in}}%
\pgfpathlineto{\pgfqpoint{4.866072in}{5.286235in}}%
\pgfpathlineto{\pgfqpoint{5.041285in}{5.307761in}}%
\pgfpathlineto{\pgfqpoint{5.216497in}{5.327022in}}%
\pgfpathlineto{\pgfqpoint{5.391710in}{5.343710in}}%
\pgfpathlineto{\pgfqpoint{5.531880in}{5.355232in}}%
\pgfpathlineto{\pgfqpoint{5.707093in}{5.366977in}}%
\pgfpathlineto{\pgfqpoint{5.847263in}{5.374211in}}%
\pgfpathlineto{\pgfqpoint{5.987433in}{5.379134in}}%
\pgfpathlineto{\pgfqpoint{6.127603in}{5.381457in}}%
\pgfpathlineto{\pgfqpoint{6.232731in}{5.381322in}}%
\pgfpathlineto{\pgfqpoint{6.337859in}{5.379350in}}%
\pgfpathlineto{\pgfqpoint{6.442986in}{5.375208in}}%
\pgfpathlineto{\pgfqpoint{6.548114in}{5.368449in}}%
\pgfpathlineto{\pgfqpoint{6.653242in}{5.358781in}}%
\pgfpathlineto{\pgfqpoint{6.723327in}{5.350466in}}%
\pgfpathlineto{\pgfqpoint{6.793412in}{5.339933in}}%
\pgfpathlineto{\pgfqpoint{6.877511in}{5.324568in}}%
\pgfpathlineto{\pgfqpoint{6.933582in}{5.311788in}}%
\pgfpathlineto{\pgfqpoint{7.003667in}{5.292846in}}%
\pgfpathlineto{\pgfqpoint{7.073752in}{5.269233in}}%
\pgfpathlineto{\pgfqpoint{7.129773in}{5.245790in}}%
\pgfpathlineto{\pgfqpoint{7.143837in}{5.238925in}}%
\pgfpathlineto{\pgfqpoint{7.180713in}{5.219530in}}%
\pgfpathlineto{\pgfqpoint{7.221955in}{5.193271in}}%
\pgfpathlineto{\pgfqpoint{7.255701in}{5.167011in}}%
\pgfpathlineto{\pgfqpoint{7.284007in}{5.140011in}}%
\pgfpathlineto{\pgfqpoint{7.305751in}{5.114492in}}%
\pgfpathlineto{\pgfqpoint{7.323963in}{5.088233in}}%
\pgfpathlineto{\pgfqpoint{7.338407in}{5.061973in}}%
\pgfpathlineto{\pgfqpoint{7.349743in}{5.035714in}}%
\pgfpathlineto{\pgfqpoint{7.358256in}{5.009454in}}%
\pgfpathlineto{\pgfqpoint{7.364295in}{4.983195in}}%
\pgfpathlineto{\pgfqpoint{7.368197in}{4.956935in}}%
\pgfpathlineto{\pgfqpoint{7.370191in}{4.930676in}}%
\pgfpathlineto{\pgfqpoint{7.370480in}{4.904416in}}%
\pgfpathlineto{\pgfqpoint{7.369245in}{4.878157in}}%
\pgfpathlineto{\pgfqpoint{7.366647in}{4.851897in}}%
\pgfpathlineto{\pgfqpoint{7.357918in}{4.799378in}}%
\pgfpathlineto{\pgfqpoint{7.345159in}{4.746860in}}%
\pgfpathlineto{\pgfqpoint{7.329200in}{4.694341in}}%
\pgfpathlineto{\pgfqpoint{7.310652in}{4.641822in}}%
\pgfpathlineto{\pgfqpoint{7.284007in}{4.574898in}}%
\pgfpathlineto{\pgfqpoint{7.256114in}{4.510524in}}%
\pgfpathlineto{\pgfqpoint{7.213922in}{4.419617in}}%
\pgfpathlineto{\pgfqpoint{7.156064in}{4.300449in}}%
\pgfpathlineto{\pgfqpoint{7.053861in}{4.090373in}}%
\pgfpathlineto{\pgfqpoint{7.003667in}{3.981614in}}%
\pgfpathlineto{\pgfqpoint{6.968624in}{3.901348in}}%
\pgfpathlineto{\pgfqpoint{6.933582in}{3.815693in}}%
\pgfpathlineto{\pgfqpoint{6.908154in}{3.749000in}}%
\pgfpathlineto{\pgfqpoint{6.880561in}{3.670221in}}%
\pgfpathlineto{\pgfqpoint{6.855800in}{3.591443in}}%
\pgfpathlineto{\pgfqpoint{6.834052in}{3.512664in}}%
\pgfpathlineto{\pgfqpoint{6.815489in}{3.433886in}}%
\pgfpathlineto{\pgfqpoint{6.804991in}{3.381367in}}%
\pgfpathlineto{\pgfqpoint{6.793412in}{3.310897in}}%
\pgfpathlineto{\pgfqpoint{6.788685in}{3.276329in}}%
\pgfpathlineto{\pgfqpoint{6.782932in}{3.223810in}}%
\pgfpathlineto{\pgfqpoint{6.778852in}{3.171291in}}%
\pgfpathlineto{\pgfqpoint{6.776476in}{3.118772in}}%
\pgfpathlineto{\pgfqpoint{6.775833in}{3.066253in}}%
\pgfpathlineto{\pgfqpoint{6.776952in}{3.013734in}}%
\pgfpathlineto{\pgfqpoint{6.779863in}{2.961215in}}%
\pgfpathlineto{\pgfqpoint{6.784596in}{2.908696in}}%
\pgfpathlineto{\pgfqpoint{6.793412in}{2.841541in}}%
\pgfpathlineto{\pgfqpoint{6.799601in}{2.803659in}}%
\pgfpathlineto{\pgfqpoint{6.809903in}{2.751140in}}%
\pgfpathlineto{\pgfqpoint{6.822140in}{2.698621in}}%
\pgfpathlineto{\pgfqpoint{6.836282in}{2.646102in}}%
\pgfpathlineto{\pgfqpoint{6.852362in}{2.593583in}}%
\pgfpathlineto{\pgfqpoint{6.870414in}{2.541064in}}%
\pgfpathlineto{\pgfqpoint{6.890425in}{2.488545in}}%
\pgfpathlineto{\pgfqpoint{6.912414in}{2.436026in}}%
\pgfpathlineto{\pgfqpoint{6.936431in}{2.383507in}}%
\pgfpathlineto{\pgfqpoint{6.968624in}{2.319186in}}%
\pgfpathlineto{\pgfqpoint{6.990429in}{2.278469in}}%
\pgfpathlineto{\pgfqpoint{7.020472in}{2.225950in}}%
\pgfpathlineto{\pgfqpoint{7.052553in}{2.173431in}}%
\pgfpathlineto{\pgfqpoint{7.086676in}{2.120912in}}%
\pgfpathlineto{\pgfqpoint{7.122844in}{2.068393in}}%
\pgfpathlineto{\pgfqpoint{7.161066in}{2.015874in}}%
\pgfpathlineto{\pgfqpoint{7.213922in}{1.947599in}}%
\pgfpathlineto{\pgfqpoint{7.248965in}{1.904559in}}%
\pgfpathlineto{\pgfqpoint{7.288121in}{1.858318in}}%
\pgfpathlineto{\pgfqpoint{7.354092in}{1.784490in}}%
\pgfpathlineto{\pgfqpoint{7.389135in}{1.747006in}}%
\pgfpathlineto{\pgfqpoint{7.433753in}{1.700761in}}%
\pgfpathlineto{\pgfqpoint{7.513561in}{1.621982in}}%
\pgfpathlineto{\pgfqpoint{7.599390in}{1.542012in}}%
\pgfpathlineto{\pgfqpoint{7.687169in}{1.464425in}}%
\pgfpathlineto{\pgfqpoint{7.739560in}{1.419992in}}%
\pgfpathlineto{\pgfqpoint{7.739560in}{1.419992in}}%
\pgfusepath{stroke}%
\end{pgfscope}%
\begin{pgfscope}%
\pgfpathrectangle{\pgfqpoint{0.766095in}{0.571603in}}{\pgfqpoint{6.973465in}{5.225635in}}%
\pgfusepath{clip}%
\pgfsetbuttcap%
\pgfsetroundjoin%
\pgfsetlinewidth{1.505625pt}%
\definecolor{currentstroke}{rgb}{0.218130,0.347432,0.550038}%
\pgfsetstrokecolor{currentstroke}%
\pgfsetdash{}{0pt}%
\pgfpathmoveto{\pgfqpoint{0.927694in}{0.571603in}}%
\pgfpathlineto{\pgfqpoint{0.906265in}{0.585888in}}%
\pgfpathlineto{\pgfqpoint{0.888364in}{0.597863in}}%
\pgfpathlineto{\pgfqpoint{0.871223in}{0.609553in}}%
\pgfpathlineto{\pgfqpoint{0.849939in}{0.624122in}}%
\pgfpathlineto{\pgfqpoint{0.836180in}{0.633725in}}%
\pgfpathlineto{\pgfqpoint{0.812408in}{0.650382in}}%
\pgfpathlineto{\pgfqpoint{0.801138in}{0.658434in}}%
\pgfpathlineto{\pgfqpoint{0.775758in}{0.676641in}}%
\pgfpathlineto{\pgfqpoint{0.766095in}{0.683711in}}%
\pgfusepath{stroke}%
\end{pgfscope}%
\begin{pgfscope}%
\pgfpathrectangle{\pgfqpoint{0.766095in}{0.571603in}}{\pgfqpoint{6.973465in}{5.225635in}}%
\pgfusepath{clip}%
\pgfsetbuttcap%
\pgfsetroundjoin%
\pgfsetlinewidth{1.505625pt}%
\definecolor{currentstroke}{rgb}{0.218130,0.347432,0.550038}%
\pgfsetstrokecolor{currentstroke}%
\pgfsetdash{}{0pt}%
\pgfpathmoveto{\pgfqpoint{0.766095in}{4.036275in}}%
\pgfpathlineto{\pgfqpoint{0.805121in}{4.064113in}}%
\pgfpathlineto{\pgfqpoint{0.881778in}{4.116632in}}%
\pgfpathlineto{\pgfqpoint{0.976350in}{4.177857in}}%
\pgfpathlineto{\pgfqpoint{1.047487in}{4.221670in}}%
\pgfpathlineto{\pgfqpoint{1.151563in}{4.282379in}}%
\pgfpathlineto{\pgfqpoint{1.231459in}{4.326708in}}%
\pgfpathlineto{\pgfqpoint{1.330836in}{4.379227in}}%
\pgfpathlineto{\pgfqpoint{1.435574in}{4.431746in}}%
\pgfpathlineto{\pgfqpoint{1.545993in}{4.484265in}}%
\pgfpathlineto{\pgfqpoint{1.677201in}{4.543240in}}%
\pgfpathlineto{\pgfqpoint{1.785103in}{4.589303in}}%
\pgfpathlineto{\pgfqpoint{1.922499in}{4.644891in}}%
\pgfpathlineto{\pgfqpoint{2.062669in}{4.698494in}}%
\pgfpathlineto{\pgfqpoint{2.202839in}{4.749260in}}%
\pgfpathlineto{\pgfqpoint{2.348895in}{4.799378in}}%
\pgfpathlineto{\pgfqpoint{2.518222in}{4.854218in}}%
\pgfpathlineto{\pgfqpoint{2.693435in}{4.907632in}}%
\pgfpathlineto{\pgfqpoint{2.868647in}{4.957979in}}%
\pgfpathlineto{\pgfqpoint{3.059143in}{5.009454in}}%
\pgfpathlineto{\pgfqpoint{3.254115in}{5.058915in}}%
\pgfpathlineto{\pgfqpoint{3.394285in}{5.092593in}}%
\pgfpathlineto{\pgfqpoint{3.606117in}{5.140752in}}%
\pgfpathlineto{\pgfqpoint{3.814796in}{5.184952in}}%
\pgfpathlineto{\pgfqpoint{3.990009in}{5.219923in}}%
\pgfpathlineto{\pgfqpoint{4.200264in}{5.259070in}}%
\pgfpathlineto{\pgfqpoint{4.375477in}{5.289632in}}%
\pgfpathlineto{\pgfqpoint{4.550689in}{5.318280in}}%
\pgfpathlineto{\pgfqpoint{4.725902in}{5.345008in}}%
\pgfpathlineto{\pgfqpoint{4.901115in}{5.369801in}}%
\pgfpathlineto{\pgfqpoint{5.076327in}{5.392638in}}%
\pgfpathlineto{\pgfqpoint{5.251540in}{5.413487in}}%
\pgfpathlineto{\pgfqpoint{5.426753in}{5.432308in}}%
\pgfpathlineto{\pgfqpoint{5.601965in}{5.448812in}}%
\pgfpathlineto{\pgfqpoint{5.742136in}{5.460380in}}%
\pgfpathlineto{\pgfqpoint{5.917348in}{5.472470in}}%
\pgfpathlineto{\pgfqpoint{6.057518in}{5.480251in}}%
\pgfpathlineto{\pgfqpoint{6.197689in}{5.485940in}}%
\pgfpathlineto{\pgfqpoint{6.337859in}{5.489402in}}%
\pgfpathlineto{\pgfqpoint{6.478029in}{5.490369in}}%
\pgfpathlineto{\pgfqpoint{6.583156in}{5.489210in}}%
\pgfpathlineto{\pgfqpoint{6.688284in}{5.486203in}}%
\pgfpathlineto{\pgfqpoint{6.793412in}{5.481045in}}%
\pgfpathlineto{\pgfqpoint{6.898539in}{5.473148in}}%
\pgfpathlineto{\pgfqpoint{7.003667in}{5.462298in}}%
\pgfpathlineto{\pgfqpoint{7.073752in}{5.453029in}}%
\pgfpathlineto{\pgfqpoint{7.143837in}{5.441565in}}%
\pgfpathlineto{\pgfqpoint{7.213922in}{5.427925in}}%
\pgfpathlineto{\pgfqpoint{7.284007in}{5.410991in}}%
\pgfpathlineto{\pgfqpoint{7.319050in}{5.401313in}}%
\pgfpathlineto{\pgfqpoint{7.392258in}{5.377087in}}%
\pgfpathlineto{\pgfqpoint{7.455002in}{5.350827in}}%
\pgfpathlineto{\pgfqpoint{7.494263in}{5.330630in}}%
\pgfpathlineto{\pgfqpoint{7.505208in}{5.324568in}}%
\pgfpathlineto{\pgfqpoint{7.545880in}{5.298308in}}%
\pgfpathlineto{\pgfqpoint{7.578969in}{5.272049in}}%
\pgfpathlineto{\pgfqpoint{7.605882in}{5.245790in}}%
\pgfpathlineto{\pgfqpoint{7.627592in}{5.219530in}}%
\pgfpathlineto{\pgfqpoint{7.644912in}{5.193271in}}%
\pgfpathlineto{\pgfqpoint{7.658496in}{5.167011in}}%
\pgfpathlineto{\pgfqpoint{7.669475in}{5.138767in}}%
\pgfpathlineto{\pgfqpoint{7.676343in}{5.114492in}}%
\pgfpathlineto{\pgfqpoint{7.681323in}{5.088233in}}%
\pgfpathlineto{\pgfqpoint{7.684107in}{5.061973in}}%
\pgfpathlineto{\pgfqpoint{7.684932in}{5.035714in}}%
\pgfpathlineto{\pgfqpoint{7.684007in}{5.009454in}}%
\pgfpathlineto{\pgfqpoint{7.681517in}{4.983195in}}%
\pgfpathlineto{\pgfqpoint{7.677627in}{4.956935in}}%
\pgfpathlineto{\pgfqpoint{7.669475in}{4.918206in}}%
\pgfpathlineto{\pgfqpoint{7.658818in}{4.878157in}}%
\pgfpathlineto{\pgfqpoint{7.641485in}{4.825638in}}%
\pgfpathlineto{\pgfqpoint{7.621115in}{4.773119in}}%
\pgfpathlineto{\pgfqpoint{7.598368in}{4.720600in}}%
\pgfpathlineto{\pgfqpoint{7.560644in}{4.641822in}}%
\pgfpathlineto{\pgfqpoint{7.519850in}{4.563043in}}%
\pgfpathlineto{\pgfqpoint{7.459220in}{4.452029in}}%
\pgfpathlineto{\pgfqpoint{7.391934in}{4.331982in}}%
\pgfpathlineto{\pgfqpoint{7.391934in}{4.331982in}}%
\pgfusepath{stroke}%
\end{pgfscope}%
\begin{pgfscope}%
\pgfpathrectangle{\pgfqpoint{0.766095in}{0.571603in}}{\pgfqpoint{6.973465in}{5.225635in}}%
\pgfusepath{clip}%
\pgfsetbuttcap%
\pgfsetroundjoin%
\pgfsetlinewidth{1.505625pt}%
\definecolor{currentstroke}{rgb}{0.218130,0.347432,0.550038}%
\pgfsetstrokecolor{currentstroke}%
\pgfsetdash{}{0pt}%
\pgfpathmoveto{\pgfqpoint{7.242463in}{4.057484in}}%
\pgfpathlineto{\pgfqpoint{7.232351in}{4.037854in}}%
\pgfpathlineto{\pgfqpoint{7.219074in}{4.011594in}}%
\pgfpathlineto{\pgfqpoint{7.213922in}{4.001258in}}%
\pgfpathlineto{\pgfqpoint{7.205985in}{3.985335in}}%
\pgfpathlineto{\pgfqpoint{7.193120in}{3.959075in}}%
\pgfpathlineto{\pgfqpoint{7.180535in}{3.932816in}}%
\pgfpathlineto{\pgfqpoint{7.178880in}{3.929307in}}%
\pgfpathlineto{\pgfqpoint{7.168141in}{3.906556in}}%
\pgfpathlineto{\pgfqpoint{7.156031in}{3.880297in}}%
\pgfpathlineto{\pgfqpoint{7.144226in}{3.854037in}}%
\pgfpathlineto{\pgfqpoint{7.143837in}{3.853156in}}%
\pgfpathlineto{\pgfqpoint{7.132633in}{3.827778in}}%
\pgfpathlineto{\pgfqpoint{7.121359in}{3.801519in}}%
\pgfpathlineto{\pgfqpoint{7.110412in}{3.775259in}}%
\pgfpathlineto{\pgfqpoint{7.108795in}{3.771281in}}%
\pgfpathlineto{\pgfqpoint{7.099719in}{3.749000in}}%
\pgfpathlineto{\pgfqpoint{7.089353in}{3.722740in}}%
\pgfpathlineto{\pgfqpoint{7.079335in}{3.696481in}}%
\pgfpathlineto{\pgfqpoint{7.073752in}{3.681357in}}%
\pgfpathlineto{\pgfqpoint{7.069632in}{3.670221in}}%
\pgfpathlineto{\pgfqpoint{7.060241in}{3.643962in}}%
\pgfpathlineto{\pgfqpoint{7.051215in}{3.617702in}}%
\pgfpathlineto{\pgfqpoint{7.042558in}{3.591443in}}%
\pgfpathlineto{\pgfqpoint{7.038709in}{3.579282in}}%
\pgfpathlineto{\pgfqpoint{7.034235in}{3.565183in}}%
\pgfpathlineto{\pgfqpoint{7.026261in}{3.538924in}}%
\pgfpathlineto{\pgfqpoint{7.018671in}{3.512664in}}%
\pgfpathlineto{\pgfqpoint{7.011469in}{3.486405in}}%
\pgfpathlineto{\pgfqpoint{7.004659in}{3.460145in}}%
\pgfpathlineto{\pgfqpoint{7.003667in}{3.456099in}}%
\pgfpathlineto{\pgfqpoint{6.998200in}{3.433886in}}%
\pgfpathlineto{\pgfqpoint{6.992134in}{3.407626in}}%
\pgfpathlineto{\pgfqpoint{6.986475in}{3.381367in}}%
\pgfpathlineto{\pgfqpoint{6.981226in}{3.355107in}}%
\pgfpathlineto{\pgfqpoint{6.976390in}{3.328848in}}%
\pgfpathlineto{\pgfqpoint{6.971969in}{3.302589in}}%
\pgfpathlineto{\pgfqpoint{6.968624in}{3.280648in}}%
\pgfpathlineto{\pgfqpoint{6.967963in}{3.276329in}}%
\pgfpathlineto{\pgfqpoint{6.964355in}{3.250070in}}%
\pgfpathlineto{\pgfqpoint{6.961177in}{3.223810in}}%
\pgfpathlineto{\pgfqpoint{6.958431in}{3.197551in}}%
\pgfpathlineto{\pgfqpoint{6.956122in}{3.171291in}}%
\pgfpathlineto{\pgfqpoint{6.954251in}{3.145032in}}%
\pgfpathlineto{\pgfqpoint{6.952823in}{3.118772in}}%
\pgfpathlineto{\pgfqpoint{6.951840in}{3.092513in}}%
\pgfpathlineto{\pgfqpoint{6.951306in}{3.066253in}}%
\pgfpathlineto{\pgfqpoint{6.951223in}{3.039994in}}%
\pgfpathlineto{\pgfqpoint{6.951596in}{3.013734in}}%
\pgfpathlineto{\pgfqpoint{6.952428in}{2.987475in}}%
\pgfpathlineto{\pgfqpoint{6.953721in}{2.961215in}}%
\pgfpathlineto{\pgfqpoint{6.955480in}{2.934956in}}%
\pgfpathlineto{\pgfqpoint{6.957707in}{2.908696in}}%
\pgfpathlineto{\pgfqpoint{6.960406in}{2.882437in}}%
\pgfpathlineto{\pgfqpoint{6.963582in}{2.856177in}}%
\pgfpathlineto{\pgfqpoint{6.967236in}{2.829918in}}%
\pgfpathlineto{\pgfqpoint{6.968624in}{2.821113in}}%
\pgfpathlineto{\pgfqpoint{6.971352in}{2.803659in}}%
\pgfpathlineto{\pgfqpoint{6.975940in}{2.777399in}}%
\pgfpathlineto{\pgfqpoint{6.981015in}{2.751140in}}%
\pgfpathlineto{\pgfqpoint{6.986581in}{2.724880in}}%
\pgfpathlineto{\pgfqpoint{6.992642in}{2.698621in}}%
\pgfpathlineto{\pgfqpoint{6.999201in}{2.672361in}}%
\pgfpathlineto{\pgfqpoint{7.003667in}{2.655757in}}%
\pgfpathlineto{\pgfqpoint{7.006243in}{2.646102in}}%
\pgfpathlineto{\pgfqpoint{7.013754in}{2.619842in}}%
\pgfpathlineto{\pgfqpoint{7.021772in}{2.593583in}}%
\pgfpathlineto{\pgfqpoint{7.030303in}{2.567323in}}%
\pgfpathlineto{\pgfqpoint{7.038709in}{2.542924in}}%
\pgfpathlineto{\pgfqpoint{7.039346in}{2.541064in}}%
\pgfpathlineto{\pgfqpoint{7.048840in}{2.514804in}}%
\pgfpathlineto{\pgfqpoint{7.058857in}{2.488545in}}%
\pgfpathlineto{\pgfqpoint{7.069401in}{2.462285in}}%
\pgfpathlineto{\pgfqpoint{7.073752in}{2.451962in}}%
\pgfpathlineto{\pgfqpoint{7.080425in}{2.436026in}}%
\pgfpathlineto{\pgfqpoint{7.091949in}{2.409766in}}%
\pgfpathlineto{\pgfqpoint{7.104010in}{2.383507in}}%
\pgfpathlineto{\pgfqpoint{7.108795in}{2.373531in}}%
\pgfpathlineto{\pgfqpoint{7.116556in}{2.357248in}}%
\pgfpathlineto{\pgfqpoint{7.129610in}{2.330988in}}%
\pgfpathlineto{\pgfqpoint{7.143214in}{2.304729in}}%
\pgfpathlineto{\pgfqpoint{7.143837in}{2.303571in}}%
\pgfpathlineto{\pgfqpoint{7.157271in}{2.278469in}}%
\pgfpathlineto{\pgfqpoint{7.171881in}{2.252210in}}%
\pgfpathlineto{\pgfqpoint{7.178880in}{2.240083in}}%
\pgfpathlineto{\pgfqpoint{7.186993in}{2.225950in}}%
\pgfpathlineto{\pgfqpoint{7.202617in}{2.199691in}}%
\pgfpathlineto{\pgfqpoint{7.213922in}{2.181346in}}%
\pgfpathlineto{\pgfqpoint{7.218776in}{2.173431in}}%
\pgfpathlineto{\pgfqpoint{7.235422in}{2.147172in}}%
\pgfpathlineto{\pgfqpoint{7.248965in}{2.126517in}}%
\pgfpathlineto{\pgfqpoint{7.252622in}{2.120912in}}%
\pgfpathlineto{\pgfqpoint{7.270302in}{2.094653in}}%
\pgfpathlineto{\pgfqpoint{7.284007in}{2.074937in}}%
\pgfpathlineto{\pgfqpoint{7.288537in}{2.068393in}}%
\pgfpathlineto{\pgfqpoint{7.307260in}{2.042134in}}%
\pgfpathlineto{\pgfqpoint{7.319050in}{2.026090in}}%
\pgfpathlineto{\pgfqpoint{7.326526in}{2.015874in}}%
\pgfpathlineto{\pgfqpoint{7.346305in}{1.989615in}}%
\pgfpathlineto{\pgfqpoint{7.354092in}{1.979561in}}%
\pgfpathlineto{\pgfqpoint{7.366599in}{1.963355in}}%
\pgfpathlineto{\pgfqpoint{7.387446in}{1.937096in}}%
\pgfpathlineto{\pgfqpoint{7.389135in}{1.935022in}}%
\pgfpathlineto{\pgfqpoint{7.408765in}{1.910836in}}%
\pgfpathlineto{\pgfqpoint{7.424177in}{1.892358in}}%
\pgfpathlineto{\pgfqpoint{7.430646in}{1.884577in}}%
\pgfpathlineto{\pgfqpoint{7.453039in}{1.858318in}}%
\pgfpathlineto{\pgfqpoint{7.459220in}{1.851245in}}%
\pgfpathlineto{\pgfqpoint{7.475940in}{1.832058in}}%
\pgfpathlineto{\pgfqpoint{7.494263in}{1.811558in}}%
\pgfpathlineto{\pgfqpoint{7.499396in}{1.805799in}}%
\pgfpathlineto{\pgfqpoint{7.523356in}{1.779539in}}%
\pgfpathlineto{\pgfqpoint{7.529305in}{1.773168in}}%
\pgfpathlineto{\pgfqpoint{7.547829in}{1.753280in}}%
\pgfpathlineto{\pgfqpoint{7.564348in}{1.735957in}}%
\pgfpathlineto{\pgfqpoint{7.572849in}{1.727020in}}%
\pgfpathlineto{\pgfqpoint{7.598406in}{1.700761in}}%
\pgfpathlineto{\pgfqpoint{7.599390in}{1.699770in}}%
\pgfpathlineto{\pgfqpoint{7.624443in}{1.674501in}}%
\pgfpathlineto{\pgfqpoint{7.634433in}{1.664643in}}%
\pgfpathlineto{\pgfqpoint{7.651022in}{1.648242in}}%
\pgfpathlineto{\pgfqpoint{7.669475in}{1.630383in}}%
\pgfpathlineto{\pgfqpoint{7.678140in}{1.621982in}}%
\pgfpathlineto{\pgfqpoint{7.704518in}{1.596936in}}%
\pgfpathlineto{\pgfqpoint{7.705794in}{1.595723in}}%
\pgfpathlineto{\pgfqpoint{7.733939in}{1.569463in}}%
\pgfpathlineto{\pgfqpoint{7.739560in}{1.564321in}}%
\pgfusepath{stroke}%
\end{pgfscope}%
\begin{pgfscope}%
\pgfpathrectangle{\pgfqpoint{0.766095in}{0.571603in}}{\pgfqpoint{6.973465in}{5.225635in}}%
\pgfusepath{clip}%
\pgfsetbuttcap%
\pgfsetroundjoin%
\pgfsetlinewidth{1.505625pt}%
\definecolor{currentstroke}{rgb}{0.208623,0.367752,0.552675}%
\pgfsetstrokecolor{currentstroke}%
\pgfsetdash{}{0pt}%
\pgfpathmoveto{\pgfqpoint{0.801341in}{0.571603in}}%
\pgfpathlineto{\pgfqpoint{0.801138in}{0.571741in}}%
\pgfpathlineto{\pgfqpoint{0.766095in}{0.595521in}}%
\pgfusepath{stroke}%
\end{pgfscope}%
\begin{pgfscope}%
\pgfpathrectangle{\pgfqpoint{0.766095in}{0.571603in}}{\pgfqpoint{6.973465in}{5.225635in}}%
\pgfusepath{clip}%
\pgfsetbuttcap%
\pgfsetroundjoin%
\pgfsetlinewidth{1.505625pt}%
\definecolor{currentstroke}{rgb}{0.208623,0.367752,0.552675}%
\pgfsetstrokecolor{currentstroke}%
\pgfsetdash{}{0pt}%
\pgfpathmoveto{\pgfqpoint{0.766095in}{4.141070in}}%
\pgfpathlineto{\pgfqpoint{0.836180in}{4.187342in}}%
\pgfpathlineto{\pgfqpoint{0.906265in}{4.231645in}}%
\pgfpathlineto{\pgfqpoint{0.976350in}{4.274193in}}%
\pgfpathlineto{\pgfqpoint{1.081478in}{4.334707in}}%
\pgfpathlineto{\pgfqpoint{1.162729in}{4.379227in}}%
\pgfpathlineto{\pgfqpoint{1.263299in}{4.431746in}}%
\pgfpathlineto{\pgfqpoint{1.369204in}{4.484265in}}%
\pgfpathlineto{\pgfqpoint{1.480740in}{4.536784in}}%
\pgfpathlineto{\pgfqpoint{1.607116in}{4.593167in}}%
\pgfpathlineto{\pgfqpoint{1.721954in}{4.641822in}}%
\pgfpathlineto{\pgfqpoint{1.852414in}{4.694362in}}%
\pgfpathlineto{\pgfqpoint{1.992584in}{4.747831in}}%
\pgfpathlineto{\pgfqpoint{2.135141in}{4.799378in}}%
\pgfpathlineto{\pgfqpoint{2.307967in}{4.858321in}}%
\pgfpathlineto{\pgfqpoint{2.450528in}{4.904416in}}%
\pgfpathlineto{\pgfqpoint{2.623350in}{4.957330in}}%
\pgfpathlineto{\pgfqpoint{2.803773in}{5.009454in}}%
\pgfpathlineto{\pgfqpoint{3.008818in}{5.065146in}}%
\pgfpathlineto{\pgfqpoint{3.201944in}{5.114492in}}%
\pgfpathlineto{\pgfqpoint{3.394285in}{5.160768in}}%
\pgfpathlineto{\pgfqpoint{3.535995in}{5.193271in}}%
\pgfpathlineto{\pgfqpoint{3.744711in}{5.238530in}}%
\pgfpathlineto{\pgfqpoint{3.919924in}{5.274469in}}%
\pgfpathlineto{\pgfqpoint{4.130179in}{5.315016in}}%
\pgfpathlineto{\pgfqpoint{4.305391in}{5.346895in}}%
\pgfpathlineto{\pgfqpoint{4.481018in}{5.377087in}}%
\pgfpathlineto{\pgfqpoint{4.725902in}{5.416113in}}%
\pgfpathlineto{\pgfqpoint{4.936157in}{5.447023in}}%
\pgfpathlineto{\pgfqpoint{5.044093in}{5.461932in}}%
\pgfpathlineto{\pgfqpoint{5.044093in}{5.461932in}}%
\pgfusepath{stroke}%
\end{pgfscope}%
\begin{pgfscope}%
\pgfpathrectangle{\pgfqpoint{0.766095in}{0.571603in}}{\pgfqpoint{6.973465in}{5.225635in}}%
\pgfusepath{clip}%
\pgfsetbuttcap%
\pgfsetroundjoin%
\pgfsetlinewidth{1.505625pt}%
\definecolor{currentstroke}{rgb}{0.208623,0.367752,0.552675}%
\pgfsetstrokecolor{currentstroke}%
\pgfsetdash{}{0pt}%
\pgfpathmoveto{\pgfqpoint{5.352900in}{5.500818in}}%
\pgfpathlineto{\pgfqpoint{5.356668in}{5.501264in}}%
\pgfpathlineto{\pgfqpoint{5.391710in}{5.505355in}}%
\pgfpathlineto{\pgfqpoint{5.418074in}{5.508384in}}%
\pgfpathlineto{\pgfqpoint{5.426753in}{5.509359in}}%
\pgfpathlineto{\pgfqpoint{5.461795in}{5.513211in}}%
\pgfpathlineto{\pgfqpoint{5.496838in}{5.517005in}}%
\pgfpathlineto{\pgfqpoint{5.531880in}{5.520741in}}%
\pgfpathlineto{\pgfqpoint{5.566923in}{5.524418in}}%
\pgfpathlineto{\pgfqpoint{5.601965in}{5.528032in}}%
\pgfpathlineto{\pgfqpoint{5.637008in}{5.531584in}}%
\pgfpathlineto{\pgfqpoint{5.667791in}{5.534644in}}%
\pgfpathlineto{\pgfqpoint{5.672050in}{5.535057in}}%
\pgfpathlineto{\pgfqpoint{5.707093in}{5.538363in}}%
\pgfpathlineto{\pgfqpoint{5.742136in}{5.541602in}}%
\pgfpathlineto{\pgfqpoint{5.777178in}{5.544771in}}%
\pgfpathlineto{\pgfqpoint{5.812221in}{5.547869in}}%
\pgfpathlineto{\pgfqpoint{5.847263in}{5.550894in}}%
\pgfpathlineto{\pgfqpoint{5.882306in}{5.553844in}}%
\pgfpathlineto{\pgfqpoint{5.917348in}{5.556717in}}%
\pgfpathlineto{\pgfqpoint{5.952391in}{5.559510in}}%
\pgfpathlineto{\pgfqpoint{5.970488in}{5.560903in}}%
\pgfpathlineto{\pgfqpoint{5.987433in}{5.562175in}}%
\pgfpathlineto{\pgfqpoint{6.022476in}{5.564709in}}%
\pgfpathlineto{\pgfqpoint{6.057518in}{5.567158in}}%
\pgfpathlineto{\pgfqpoint{6.092561in}{5.569518in}}%
\pgfpathlineto{\pgfqpoint{6.127603in}{5.571788in}}%
\pgfpathlineto{\pgfqpoint{6.162646in}{5.573966in}}%
\pgfpathlineto{\pgfqpoint{6.197689in}{5.576046in}}%
\pgfpathlineto{\pgfqpoint{6.232731in}{5.578028in}}%
\pgfpathlineto{\pgfqpoint{6.267774in}{5.579907in}}%
\pgfpathlineto{\pgfqpoint{6.302816in}{5.581681in}}%
\pgfpathlineto{\pgfqpoint{6.337859in}{5.583345in}}%
\pgfpathlineto{\pgfqpoint{6.372901in}{5.584897in}}%
\pgfpathlineto{\pgfqpoint{6.407944in}{5.586331in}}%
\pgfpathlineto{\pgfqpoint{6.430218in}{5.587163in}}%
\pgfpathlineto{\pgfqpoint{6.442986in}{5.587626in}}%
\pgfpathlineto{\pgfqpoint{6.478029in}{5.588768in}}%
\pgfpathlineto{\pgfqpoint{6.513071in}{5.589785in}}%
\pgfpathlineto{\pgfqpoint{6.548114in}{5.590672in}}%
\pgfpathlineto{\pgfqpoint{6.583156in}{5.591424in}}%
\pgfpathlineto{\pgfqpoint{6.618199in}{5.592036in}}%
\pgfpathlineto{\pgfqpoint{6.653242in}{5.592504in}}%
\pgfpathlineto{\pgfqpoint{6.688284in}{5.592823in}}%
\pgfpathlineto{\pgfqpoint{6.723327in}{5.592985in}}%
\pgfpathlineto{\pgfqpoint{6.758369in}{5.592986in}}%
\pgfpathlineto{\pgfqpoint{6.793412in}{5.592818in}}%
\pgfpathlineto{\pgfqpoint{6.828454in}{5.592476in}}%
\pgfpathlineto{\pgfqpoint{6.863497in}{5.591951in}}%
\pgfpathlineto{\pgfqpoint{6.898539in}{5.591238in}}%
\pgfpathlineto{\pgfqpoint{6.933582in}{5.590327in}}%
\pgfpathlineto{\pgfqpoint{6.968624in}{5.589210in}}%
\pgfpathlineto{\pgfqpoint{7.003667in}{5.587878in}}%
\pgfpathlineto{\pgfqpoint{7.019917in}{5.587163in}}%
\pgfpathlineto{\pgfqpoint{7.038709in}{5.586282in}}%
\pgfpathlineto{\pgfqpoint{7.073752in}{5.584404in}}%
\pgfpathlineto{\pgfqpoint{7.108795in}{5.582266in}}%
\pgfpathlineto{\pgfqpoint{7.143837in}{5.579857in}}%
\pgfpathlineto{\pgfqpoint{7.178880in}{5.577163in}}%
\pgfpathlineto{\pgfqpoint{7.213922in}{5.574169in}}%
\pgfpathlineto{\pgfqpoint{7.248965in}{5.570861in}}%
\pgfpathlineto{\pgfqpoint{7.284007in}{5.567222in}}%
\pgfpathlineto{\pgfqpoint{7.319050in}{5.563235in}}%
\pgfpathlineto{\pgfqpoint{7.337971in}{5.560903in}}%
\pgfpathlineto{\pgfqpoint{7.354092in}{5.558766in}}%
\pgfpathlineto{\pgfqpoint{7.389135in}{5.553749in}}%
\pgfpathlineto{\pgfqpoint{7.424177in}{5.548291in}}%
\pgfpathlineto{\pgfqpoint{7.459220in}{5.542364in}}%
\pgfpathlineto{\pgfqpoint{7.494263in}{5.535939in}}%
\pgfpathlineto{\pgfqpoint{7.500904in}{5.534644in}}%
\pgfpathlineto{\pgfqpoint{7.529305in}{5.528625in}}%
\pgfpathlineto{\pgfqpoint{7.564348in}{5.520612in}}%
\pgfpathlineto{\pgfqpoint{7.599390in}{5.511942in}}%
\pgfpathlineto{\pgfqpoint{7.612880in}{5.508384in}}%
\pgfpathlineto{\pgfqpoint{7.634433in}{5.502152in}}%
\pgfpathlineto{\pgfqpoint{7.669475in}{5.491275in}}%
\pgfpathlineto{\pgfqpoint{7.696842in}{5.482125in}}%
\pgfpathlineto{\pgfqpoint{7.704518in}{5.479284in}}%
\pgfpathlineto{\pgfqpoint{7.739560in}{5.465490in}}%
\pgfusepath{stroke}%
\end{pgfscope}%
\begin{pgfscope}%
\pgfpathrectangle{\pgfqpoint{0.766095in}{0.571603in}}{\pgfqpoint{6.973465in}{5.225635in}}%
\pgfusepath{clip}%
\pgfsetbuttcap%
\pgfsetroundjoin%
\pgfsetlinewidth{1.505625pt}%
\definecolor{currentstroke}{rgb}{0.208623,0.367752,0.552675}%
\pgfsetstrokecolor{currentstroke}%
\pgfsetdash{}{0pt}%
\pgfpathmoveto{\pgfqpoint{7.739560in}{4.555901in}}%
\pgfpathlineto{\pgfqpoint{7.728323in}{4.537821in}}%
\pgfusepath{stroke}%
\end{pgfscope}%
\begin{pgfscope}%
\pgfpathrectangle{\pgfqpoint{0.766095in}{0.571603in}}{\pgfqpoint{6.973465in}{5.225635in}}%
\pgfusepath{clip}%
\pgfsetbuttcap%
\pgfsetroundjoin%
\pgfsetlinewidth{1.505625pt}%
\definecolor{currentstroke}{rgb}{0.208623,0.367752,0.552675}%
\pgfsetstrokecolor{currentstroke}%
\pgfsetdash{}{0pt}%
\pgfpathmoveto{\pgfqpoint{7.562732in}{4.272825in}}%
\pgfpathlineto{\pgfqpoint{7.494263in}{4.159059in}}%
\pgfpathlineto{\pgfqpoint{7.439670in}{4.064113in}}%
\pgfpathlineto{\pgfqpoint{7.389135in}{3.971096in}}%
\pgfpathlineto{\pgfqpoint{7.354092in}{3.902827in}}%
\pgfpathlineto{\pgfqpoint{7.317815in}{3.827778in}}%
\pgfpathlineto{\pgfqpoint{7.282498in}{3.749000in}}%
\pgfpathlineto{\pgfqpoint{7.250203in}{3.670221in}}%
\pgfpathlineto{\pgfqpoint{7.221093in}{3.591443in}}%
\pgfpathlineto{\pgfqpoint{7.195361in}{3.512664in}}%
\pgfpathlineto{\pgfqpoint{7.178880in}{3.455209in}}%
\pgfpathlineto{\pgfqpoint{7.166578in}{3.407626in}}%
\pgfpathlineto{\pgfqpoint{7.154637in}{3.355107in}}%
\pgfpathlineto{\pgfqpoint{7.143837in}{3.299265in}}%
\pgfpathlineto{\pgfqpoint{7.135840in}{3.250070in}}%
\pgfpathlineto{\pgfqpoint{7.129037in}{3.197551in}}%
\pgfpathlineto{\pgfqpoint{7.124024in}{3.145032in}}%
\pgfpathlineto{\pgfqpoint{7.120821in}{3.092513in}}%
\pgfpathlineto{\pgfqpoint{7.119452in}{3.039994in}}%
\pgfpathlineto{\pgfqpoint{7.119940in}{2.987475in}}%
\pgfpathlineto{\pgfqpoint{7.122308in}{2.934956in}}%
\pgfpathlineto{\pgfqpoint{7.126582in}{2.882437in}}%
\pgfpathlineto{\pgfqpoint{7.132787in}{2.829918in}}%
\pgfpathlineto{\pgfqpoint{7.143837in}{2.761690in}}%
\pgfpathlineto{\pgfqpoint{7.151045in}{2.724880in}}%
\pgfpathlineto{\pgfqpoint{7.163119in}{2.672361in}}%
\pgfpathlineto{\pgfqpoint{7.178880in}{2.614287in}}%
\pgfpathlineto{\pgfqpoint{7.193286in}{2.567323in}}%
\pgfpathlineto{\pgfqpoint{7.213922in}{2.508130in}}%
\pgfpathlineto{\pgfqpoint{7.231537in}{2.462285in}}%
\pgfpathlineto{\pgfqpoint{7.253742in}{2.409766in}}%
\pgfpathlineto{\pgfqpoint{7.284007in}{2.345015in}}%
\pgfpathlineto{\pgfqpoint{7.304278in}{2.304729in}}%
\pgfpathlineto{\pgfqpoint{7.332661in}{2.252210in}}%
\pgfpathlineto{\pgfqpoint{7.363131in}{2.199691in}}%
\pgfpathlineto{\pgfqpoint{7.395689in}{2.147172in}}%
\pgfpathlineto{\pgfqpoint{7.430337in}{2.094653in}}%
\pgfpathlineto{\pgfqpoint{7.467081in}{2.042134in}}%
\pgfpathlineto{\pgfqpoint{7.505928in}{1.989615in}}%
\pgfpathlineto{\pgfqpoint{7.564348in}{1.915575in}}%
\pgfpathlineto{\pgfqpoint{7.599390in}{1.873515in}}%
\pgfpathlineto{\pgfqpoint{7.635182in}{1.832058in}}%
\pgfpathlineto{\pgfqpoint{7.682465in}{1.779539in}}%
\pgfpathlineto{\pgfqpoint{7.739560in}{1.719146in}}%
\pgfpathlineto{\pgfqpoint{7.739560in}{1.719146in}}%
\pgfusepath{stroke}%
\end{pgfscope}%
\begin{pgfscope}%
\pgfpathrectangle{\pgfqpoint{0.766095in}{0.571603in}}{\pgfqpoint{6.973465in}{5.225635in}}%
\pgfusepath{clip}%
\pgfsetbuttcap%
\pgfsetroundjoin%
\pgfsetlinewidth{1.505625pt}%
\definecolor{currentstroke}{rgb}{0.199430,0.387607,0.554642}%
\pgfsetstrokecolor{currentstroke}%
\pgfsetdash{}{0pt}%
\pgfpathmoveto{\pgfqpoint{0.766095in}{4.236610in}}%
\pgfpathlineto{\pgfqpoint{0.783898in}{4.247930in}}%
\pgfpathlineto{\pgfqpoint{0.801138in}{4.258705in}}%
\pgfpathlineto{\pgfqpoint{0.826074in}{4.274189in}}%
\pgfpathlineto{\pgfqpoint{0.836180in}{4.280358in}}%
\pgfpathlineto{\pgfqpoint{0.869305in}{4.300449in}}%
\pgfpathlineto{\pgfqpoint{0.871223in}{4.301592in}}%
\pgfpathlineto{\pgfqpoint{0.906265in}{4.322329in}}%
\pgfpathlineto{\pgfqpoint{0.913713in}{4.326708in}}%
\pgfpathlineto{\pgfqpoint{0.941308in}{4.342658in}}%
\pgfpathlineto{\pgfqpoint{0.959251in}{4.352967in}}%
\pgfpathlineto{\pgfqpoint{0.976350in}{4.362627in}}%
\pgfpathlineto{\pgfqpoint{1.005904in}{4.379227in}}%
\pgfpathlineto{\pgfqpoint{1.011393in}{4.382258in}}%
\pgfpathlineto{\pgfqpoint{1.046435in}{4.401477in}}%
\pgfpathlineto{\pgfqpoint{1.053793in}{4.405486in}}%
\pgfpathlineto{\pgfqpoint{1.081478in}{4.420319in}}%
\pgfpathlineto{\pgfqpoint{1.102919in}{4.431746in}}%
\pgfpathlineto{\pgfqpoint{1.116520in}{4.438874in}}%
\pgfpathlineto{\pgfqpoint{1.151563in}{4.457139in}}%
\pgfpathlineto{\pgfqpoint{1.153237in}{4.458005in}}%
\pgfpathlineto{\pgfqpoint{1.186605in}{4.474978in}}%
\pgfpathlineto{\pgfqpoint{1.204951in}{4.484265in}}%
\pgfpathlineto{\pgfqpoint{1.221648in}{4.492576in}}%
\pgfpathlineto{\pgfqpoint{1.256691in}{4.509937in}}%
\pgfpathlineto{\pgfqpoint{1.257885in}{4.510524in}}%
\pgfpathlineto{\pgfqpoint{1.291733in}{4.526887in}}%
\pgfpathlineto{\pgfqpoint{1.312293in}{4.536784in}}%
\pgfpathlineto{\pgfqpoint{1.326776in}{4.543639in}}%
\pgfpathlineto{\pgfqpoint{1.361818in}{4.560141in}}%
\pgfpathlineto{\pgfqpoint{1.368024in}{4.563043in}}%
\pgfpathlineto{\pgfqpoint{1.396861in}{4.576306in}}%
\pgfpathlineto{\pgfqpoint{1.425222in}{4.589303in}}%
\pgfpathlineto{\pgfqpoint{1.431903in}{4.592313in}}%
\pgfpathlineto{\pgfqpoint{1.466946in}{4.607998in}}%
\pgfpathlineto{\pgfqpoint{1.483924in}{4.615562in}}%
\pgfpathlineto{\pgfqpoint{1.501988in}{4.623475in}}%
\pgfpathlineto{\pgfqpoint{1.537031in}{4.638758in}}%
\pgfpathlineto{\pgfqpoint{1.544102in}{4.641822in}}%
\pgfpathlineto{\pgfqpoint{1.572073in}{4.653737in}}%
\pgfpathlineto{\pgfqpoint{1.605838in}{4.668081in}}%
\pgfpathlineto{\pgfqpoint{1.607116in}{4.668615in}}%
\pgfpathlineto{\pgfqpoint{1.642158in}{4.683140in}}%
\pgfpathlineto{\pgfqpoint{1.669254in}{4.694341in}}%
\pgfpathlineto{\pgfqpoint{1.677201in}{4.697570in}}%
\pgfpathlineto{\pgfqpoint{1.712244in}{4.711721in}}%
\pgfpathlineto{\pgfqpoint{1.734311in}{4.720600in}}%
\pgfpathlineto{\pgfqpoint{1.747286in}{4.725733in}}%
\pgfpathlineto{\pgfqpoint{1.782329in}{4.739519in}}%
\pgfpathlineto{\pgfqpoint{1.801067in}{4.746860in}}%
\pgfpathlineto{\pgfqpoint{1.817371in}{4.753139in}}%
\pgfpathlineto{\pgfqpoint{1.852414in}{4.766569in}}%
\pgfpathlineto{\pgfqpoint{1.869582in}{4.773119in}}%
\pgfpathlineto{\pgfqpoint{1.887456in}{4.779823in}}%
\pgfpathlineto{\pgfqpoint{1.922499in}{4.792906in}}%
\pgfpathlineto{\pgfqpoint{1.939914in}{4.799378in}}%
\pgfpathlineto{\pgfqpoint{1.957541in}{4.805818in}}%
\pgfpathlineto{\pgfqpoint{1.992584in}{4.818562in}}%
\pgfpathlineto{\pgfqpoint{2.012122in}{4.825638in}}%
\pgfpathlineto{\pgfqpoint{2.027626in}{4.831157in}}%
\pgfpathlineto{\pgfqpoint{2.062669in}{4.843570in}}%
\pgfpathlineto{\pgfqpoint{2.086259in}{4.851897in}}%
\pgfpathlineto{\pgfqpoint{2.097711in}{4.855871in}}%
\pgfpathlineto{\pgfqpoint{2.132754in}{4.867960in}}%
\pgfpathlineto{\pgfqpoint{2.162380in}{4.878157in}}%
\pgfpathlineto{\pgfqpoint{2.167797in}{4.879989in}}%
\pgfpathlineto{\pgfqpoint{2.202839in}{4.891761in}}%
\pgfpathlineto{\pgfqpoint{2.237882in}{4.903519in}}%
\pgfpathlineto{\pgfqpoint{2.240576in}{4.904416in}}%
\pgfpathlineto{\pgfqpoint{2.272924in}{4.915003in}}%
\pgfpathlineto{\pgfqpoint{2.307967in}{4.926451in}}%
\pgfpathlineto{\pgfqpoint{2.320969in}{4.930676in}}%
\pgfpathlineto{\pgfqpoint{2.343009in}{4.937713in}}%
\pgfpathlineto{\pgfqpoint{2.378052in}{4.948858in}}%
\pgfpathlineto{\pgfqpoint{2.403526in}{4.956935in}}%
\pgfpathlineto{\pgfqpoint{2.413094in}{4.959916in}}%
\pgfpathlineto{\pgfqpoint{2.448137in}{4.970766in}}%
\pgfpathlineto{\pgfqpoint{2.483179in}{4.981601in}}%
\pgfpathlineto{\pgfqpoint{2.488370in}{4.983195in}}%
\pgfpathlineto{\pgfqpoint{2.518222in}{4.992200in}}%
\pgfpathlineto{\pgfqpoint{2.553264in}{5.002745in}}%
\pgfpathlineto{\pgfqpoint{2.575643in}{5.009454in}}%
\pgfpathlineto{\pgfqpoint{2.588307in}{5.013184in}}%
\pgfpathlineto{\pgfqpoint{2.623350in}{5.023446in}}%
\pgfpathlineto{\pgfqpoint{2.658392in}{5.033694in}}%
\pgfpathlineto{\pgfqpoint{2.665346in}{5.035714in}}%
\pgfpathlineto{\pgfqpoint{2.693435in}{5.043728in}}%
\pgfpathlineto{\pgfqpoint{2.728477in}{5.053697in}}%
\pgfpathlineto{\pgfqpoint{2.757639in}{5.061973in}}%
\pgfpathlineto{\pgfqpoint{2.763520in}{5.063612in}}%
\pgfpathlineto{\pgfqpoint{2.798562in}{5.073310in}}%
\pgfpathlineto{\pgfqpoint{2.833605in}{5.082994in}}%
\pgfpathlineto{\pgfqpoint{2.852652in}{5.088233in}}%
\pgfpathlineto{\pgfqpoint{2.868647in}{5.092554in}}%
\pgfpathlineto{\pgfqpoint{2.903690in}{5.101970in}}%
\pgfpathlineto{\pgfqpoint{2.938732in}{5.111373in}}%
\pgfpathlineto{\pgfqpoint{2.950431in}{5.114492in}}%
\pgfpathlineto{\pgfqpoint{2.973775in}{5.120605in}}%
\pgfpathlineto{\pgfqpoint{3.008818in}{5.129745in}}%
\pgfpathlineto{\pgfqpoint{3.043860in}{5.138871in}}%
\pgfpathlineto{\pgfqpoint{3.051133in}{5.140752in}}%
\pgfpathlineto{\pgfqpoint{3.078903in}{5.147802in}}%
\pgfpathlineto{\pgfqpoint{3.113945in}{5.156670in}}%
\pgfpathlineto{\pgfqpoint{3.148988in}{5.165524in}}%
\pgfpathlineto{\pgfqpoint{3.154917in}{5.167011in}}%
\pgfpathlineto{\pgfqpoint{3.184030in}{5.174178in}}%
\pgfpathlineto{\pgfqpoint{3.219073in}{5.182780in}}%
\pgfpathlineto{\pgfqpoint{3.254115in}{5.191366in}}%
\pgfpathlineto{\pgfqpoint{3.261947in}{5.193271in}}%
\pgfpathlineto{\pgfqpoint{3.289158in}{5.199768in}}%
\pgfpathlineto{\pgfqpoint{3.324200in}{5.208106in}}%
\pgfpathlineto{\pgfqpoint{3.359243in}{5.216428in}}%
\pgfpathlineto{\pgfqpoint{3.372389in}{5.219530in}}%
\pgfpathlineto{\pgfqpoint{3.394285in}{5.224602in}}%
\pgfpathlineto{\pgfqpoint{3.429328in}{5.232680in}}%
\pgfpathlineto{\pgfqpoint{3.464371in}{5.240742in}}%
\pgfpathlineto{\pgfqpoint{3.486414in}{5.245790in}}%
\pgfpathlineto{\pgfqpoint{3.499413in}{5.248711in}}%
\pgfpathlineto{\pgfqpoint{3.534456in}{5.256533in}}%
\pgfpathlineto{\pgfqpoint{3.569498in}{5.264338in}}%
\pgfpathlineto{\pgfqpoint{3.604193in}{5.272049in}}%
\pgfpathlineto{\pgfqpoint{3.604541in}{5.272125in}}%
\pgfpathlineto{\pgfqpoint{3.627515in}{5.277088in}}%
\pgfusepath{stroke}%
\end{pgfscope}%
\begin{pgfscope}%
\pgfpathrectangle{\pgfqpoint{0.766095in}{0.571603in}}{\pgfqpoint{6.973465in}{5.225635in}}%
\pgfusepath{clip}%
\pgfsetbuttcap%
\pgfsetroundjoin%
\pgfsetlinewidth{1.505625pt}%
\definecolor{currentstroke}{rgb}{0.199430,0.387607,0.554642}%
\pgfsetstrokecolor{currentstroke}%
\pgfsetdash{}{0pt}%
\pgfpathmoveto{\pgfqpoint{3.932257in}{5.340609in}}%
\pgfpathlineto{\pgfqpoint{3.954966in}{5.345154in}}%
\pgfpathlineto{\pgfqpoint{3.983430in}{5.350827in}}%
\pgfpathlineto{\pgfqpoint{3.990009in}{5.352113in}}%
\pgfpathlineto{\pgfqpoint{4.025051in}{5.358900in}}%
\pgfpathlineto{\pgfqpoint{4.060094in}{5.365667in}}%
\pgfpathlineto{\pgfqpoint{4.095136in}{5.372413in}}%
\pgfpathlineto{\pgfqpoint{4.119550in}{5.377087in}}%
\pgfpathlineto{\pgfqpoint{4.130179in}{5.379082in}}%
\pgfpathlineto{\pgfqpoint{4.165221in}{5.385604in}}%
\pgfpathlineto{\pgfqpoint{4.200264in}{5.392103in}}%
\pgfpathlineto{\pgfqpoint{4.235306in}{5.398580in}}%
\pgfpathlineto{\pgfqpoint{4.261237in}{5.403346in}}%
\pgfpathlineto{\pgfqpoint{4.270349in}{5.404988in}}%
\pgfpathlineto{\pgfqpoint{4.305391in}{5.411243in}}%
\pgfpathlineto{\pgfqpoint{4.340434in}{5.417474in}}%
\pgfpathlineto{\pgfqpoint{4.375477in}{5.423682in}}%
\pgfpathlineto{\pgfqpoint{4.409061in}{5.429606in}}%
\pgfpathlineto{\pgfqpoint{4.410519in}{5.429858in}}%
\pgfpathlineto{\pgfqpoint{4.445562in}{5.435845in}}%
\pgfpathlineto{\pgfqpoint{4.480604in}{5.441807in}}%
\pgfpathlineto{\pgfqpoint{4.515647in}{5.447743in}}%
\pgfpathlineto{\pgfqpoint{4.550689in}{5.453653in}}%
\pgfpathlineto{\pgfqpoint{4.563930in}{5.455865in}}%
\pgfpathlineto{\pgfqpoint{4.585732in}{5.459433in}}%
\pgfpathlineto{\pgfqpoint{4.620774in}{5.465124in}}%
\pgfpathlineto{\pgfqpoint{4.655817in}{5.470787in}}%
\pgfpathlineto{\pgfqpoint{4.690859in}{5.476422in}}%
\pgfpathlineto{\pgfqpoint{4.725902in}{5.482028in}}%
\pgfpathlineto{\pgfqpoint{4.726514in}{5.482125in}}%
\pgfpathlineto{\pgfqpoint{4.760944in}{5.487448in}}%
\pgfpathlineto{\pgfqpoint{4.795987in}{5.492835in}}%
\pgfpathlineto{\pgfqpoint{4.831030in}{5.498191in}}%
\pgfpathlineto{\pgfqpoint{4.866072in}{5.503517in}}%
\pgfpathlineto{\pgfqpoint{4.898309in}{5.508384in}}%
\pgfpathlineto{\pgfqpoint{4.901115in}{5.508799in}}%
\pgfpathlineto{\pgfqpoint{4.936157in}{5.513907in}}%
\pgfpathlineto{\pgfqpoint{4.971200in}{5.518982in}}%
\pgfpathlineto{\pgfqpoint{5.006242in}{5.524023in}}%
\pgfpathlineto{\pgfqpoint{5.041285in}{5.529031in}}%
\pgfpathlineto{\pgfqpoint{5.076327in}{5.534003in}}%
\pgfpathlineto{\pgfqpoint{5.080908in}{5.534644in}}%
\pgfpathlineto{\pgfqpoint{5.111370in}{5.538812in}}%
\pgfpathlineto{\pgfqpoint{5.146412in}{5.543565in}}%
\pgfpathlineto{\pgfqpoint{5.181455in}{5.548281in}}%
\pgfpathlineto{\pgfqpoint{5.216497in}{5.552958in}}%
\pgfpathlineto{\pgfqpoint{5.251540in}{5.557598in}}%
\pgfpathlineto{\pgfqpoint{5.276784in}{5.560903in}}%
\pgfpathlineto{\pgfqpoint{5.286583in}{5.562158in}}%
\pgfpathlineto{\pgfqpoint{5.321625in}{5.566578in}}%
\pgfpathlineto{\pgfqpoint{5.356668in}{5.570956in}}%
\pgfpathlineto{\pgfqpoint{5.391710in}{5.575292in}}%
\pgfpathlineto{\pgfqpoint{5.426753in}{5.579584in}}%
\pgfpathlineto{\pgfqpoint{5.461795in}{5.583833in}}%
\pgfpathlineto{\pgfqpoint{5.489600in}{5.587163in}}%
\pgfpathlineto{\pgfqpoint{5.496838in}{5.588010in}}%
\pgfpathlineto{\pgfqpoint{5.531880in}{5.592036in}}%
\pgfpathlineto{\pgfqpoint{5.566923in}{5.596016in}}%
\pgfpathlineto{\pgfqpoint{5.601965in}{5.599947in}}%
\pgfpathlineto{\pgfqpoint{5.637008in}{5.603829in}}%
\pgfpathlineto{\pgfqpoint{5.672050in}{5.607661in}}%
\pgfpathlineto{\pgfqpoint{5.707093in}{5.611441in}}%
\pgfpathlineto{\pgfqpoint{5.725806in}{5.613422in}}%
\pgfpathlineto{\pgfqpoint{5.742136in}{5.615111in}}%
\pgfpathlineto{\pgfqpoint{5.777178in}{5.618664in}}%
\pgfpathlineto{\pgfqpoint{5.812221in}{5.622161in}}%
\pgfpathlineto{\pgfqpoint{5.847263in}{5.625602in}}%
\pgfpathlineto{\pgfqpoint{5.882306in}{5.628984in}}%
\pgfpathlineto{\pgfqpoint{5.917348in}{5.632308in}}%
\pgfpathlineto{\pgfqpoint{5.952391in}{5.635570in}}%
\pgfpathlineto{\pgfqpoint{5.987433in}{5.638769in}}%
\pgfpathlineto{\pgfqpoint{5.997706in}{5.639682in}}%
\pgfpathlineto{\pgfqpoint{6.022476in}{5.641829in}}%
\pgfpathlineto{\pgfqpoint{6.057518in}{5.644792in}}%
\pgfpathlineto{\pgfqpoint{6.092561in}{5.647689in}}%
\pgfpathlineto{\pgfqpoint{6.127603in}{5.650516in}}%
\pgfpathlineto{\pgfqpoint{6.162646in}{5.653273in}}%
\pgfpathlineto{\pgfqpoint{6.197689in}{5.655957in}}%
\pgfpathlineto{\pgfqpoint{6.232731in}{5.658566in}}%
\pgfpathlineto{\pgfqpoint{6.267774in}{5.661097in}}%
\pgfpathlineto{\pgfqpoint{6.302816in}{5.663550in}}%
\pgfpathlineto{\pgfqpoint{6.337859in}{5.665920in}}%
\pgfpathlineto{\pgfqpoint{6.338181in}{5.665941in}}%
\pgfpathlineto{\pgfqpoint{6.372901in}{5.668125in}}%
\pgfpathlineto{\pgfqpoint{6.407944in}{5.670244in}}%
\pgfpathlineto{\pgfqpoint{6.442986in}{5.672275in}}%
\pgfpathlineto{\pgfqpoint{6.478029in}{5.674216in}}%
\pgfpathlineto{\pgfqpoint{6.513071in}{5.676064in}}%
\pgfpathlineto{\pgfqpoint{6.548114in}{5.677815in}}%
\pgfpathlineto{\pgfqpoint{6.583156in}{5.679468in}}%
\pgfpathlineto{\pgfqpoint{6.618199in}{5.681019in}}%
\pgfpathlineto{\pgfqpoint{6.653242in}{5.682463in}}%
\pgfpathlineto{\pgfqpoint{6.688284in}{5.683798in}}%
\pgfpathlineto{\pgfqpoint{6.723327in}{5.685021in}}%
\pgfpathlineto{\pgfqpoint{6.758369in}{5.686126in}}%
\pgfpathlineto{\pgfqpoint{6.793412in}{5.687110in}}%
\pgfpathlineto{\pgfqpoint{6.828454in}{5.687969in}}%
\pgfpathlineto{\pgfqpoint{6.863497in}{5.688698in}}%
\pgfpathlineto{\pgfqpoint{6.898539in}{5.689292in}}%
\pgfpathlineto{\pgfqpoint{6.933582in}{5.689746in}}%
\pgfpathlineto{\pgfqpoint{6.968624in}{5.690056in}}%
\pgfpathlineto{\pgfqpoint{7.003667in}{5.690215in}}%
\pgfpathlineto{\pgfqpoint{7.038709in}{5.690217in}}%
\pgfpathlineto{\pgfqpoint{7.073752in}{5.690057in}}%
\pgfpathlineto{\pgfqpoint{7.108795in}{5.689727in}}%
\pgfpathlineto{\pgfqpoint{7.143837in}{5.689222in}}%
\pgfpathlineto{\pgfqpoint{7.178880in}{5.688533in}}%
\pgfpathlineto{\pgfqpoint{7.213922in}{5.687652in}}%
\pgfpathlineto{\pgfqpoint{7.248965in}{5.686573in}}%
\pgfpathlineto{\pgfqpoint{7.284007in}{5.685285in}}%
\pgfpathlineto{\pgfqpoint{7.319050in}{5.683779in}}%
\pgfpathlineto{\pgfqpoint{7.354092in}{5.682047in}}%
\pgfpathlineto{\pgfqpoint{7.389135in}{5.680077in}}%
\pgfpathlineto{\pgfqpoint{7.424177in}{5.677859in}}%
\pgfpathlineto{\pgfqpoint{7.459220in}{5.675380in}}%
\pgfpathlineto{\pgfqpoint{7.494263in}{5.672628in}}%
\pgfpathlineto{\pgfqpoint{7.529305in}{5.669590in}}%
\pgfpathlineto{\pgfqpoint{7.564348in}{5.666251in}}%
\pgfpathlineto{\pgfqpoint{7.567364in}{5.665941in}}%
\pgfpathlineto{\pgfqpoint{7.599390in}{5.662410in}}%
\pgfpathlineto{\pgfqpoint{7.634433in}{5.658195in}}%
\pgfpathlineto{\pgfqpoint{7.669475in}{5.653603in}}%
\pgfpathlineto{\pgfqpoint{7.704518in}{5.648613in}}%
\pgfpathlineto{\pgfqpoint{7.739560in}{5.643201in}}%
\pgfusepath{stroke}%
\end{pgfscope}%
\begin{pgfscope}%
\pgfpathrectangle{\pgfqpoint{0.766095in}{0.571603in}}{\pgfqpoint{6.973465in}{5.225635in}}%
\pgfusepath{clip}%
\pgfsetbuttcap%
\pgfsetroundjoin%
\pgfsetlinewidth{1.505625pt}%
\definecolor{currentstroke}{rgb}{0.199430,0.387607,0.554642}%
\pgfsetstrokecolor{currentstroke}%
\pgfsetdash{}{0pt}%
\pgfpathmoveto{\pgfqpoint{7.739560in}{4.247425in}}%
\pgfpathlineto{\pgfqpoint{7.723394in}{4.222669in}}%
\pgfusepath{stroke}%
\end{pgfscope}%
\begin{pgfscope}%
\pgfpathrectangle{\pgfqpoint{0.766095in}{0.571603in}}{\pgfqpoint{6.973465in}{5.225635in}}%
\pgfusepath{clip}%
\pgfsetbuttcap%
\pgfsetroundjoin%
\pgfsetlinewidth{1.505625pt}%
\definecolor{currentstroke}{rgb}{0.199430,0.387607,0.554642}%
\pgfsetstrokecolor{currentstroke}%
\pgfsetdash{}{0pt}%
\pgfpathmoveto{\pgfqpoint{7.561876in}{3.955217in}}%
\pgfpathlineto{\pgfqpoint{7.549552in}{3.932816in}}%
\pgfpathlineto{\pgfqpoint{7.535455in}{3.906556in}}%
\pgfpathlineto{\pgfqpoint{7.529305in}{3.894852in}}%
\pgfpathlineto{\pgfqpoint{7.521652in}{3.880297in}}%
\pgfpathlineto{\pgfqpoint{7.508162in}{3.854037in}}%
\pgfpathlineto{\pgfqpoint{7.495039in}{3.827778in}}%
\pgfpathlineto{\pgfqpoint{7.494263in}{3.826190in}}%
\pgfpathlineto{\pgfqpoint{7.482190in}{3.801519in}}%
\pgfpathlineto{\pgfqpoint{7.469714in}{3.775259in}}%
\pgfpathlineto{\pgfqpoint{7.459220in}{3.752495in}}%
\pgfpathlineto{\pgfqpoint{7.457606in}{3.749000in}}%
\pgfpathlineto{\pgfqpoint{7.445804in}{3.722740in}}%
\pgfpathlineto{\pgfqpoint{7.434393in}{3.696481in}}%
\pgfpathlineto{\pgfqpoint{7.424177in}{3.672139in}}%
\pgfpathlineto{\pgfqpoint{7.423371in}{3.670221in}}%
\pgfpathlineto{\pgfqpoint{7.412670in}{3.643962in}}%
\pgfpathlineto{\pgfqpoint{7.402373in}{3.617702in}}%
\pgfpathlineto{\pgfqpoint{7.392483in}{3.591443in}}%
\pgfpathlineto{\pgfqpoint{7.389135in}{3.582199in}}%
\pgfpathlineto{\pgfqpoint{7.382956in}{3.565183in}}%
\pgfpathlineto{\pgfqpoint{7.373818in}{3.538924in}}%
\pgfpathlineto{\pgfqpoint{7.365099in}{3.512664in}}%
\pgfpathlineto{\pgfqpoint{7.356801in}{3.486405in}}%
\pgfpathlineto{\pgfqpoint{7.354092in}{3.477399in}}%
\pgfpathlineto{\pgfqpoint{7.348887in}{3.460145in}}%
\pgfpathlineto{\pgfqpoint{7.341381in}{3.433886in}}%
\pgfpathlineto{\pgfqpoint{7.334307in}{3.407626in}}%
\pgfpathlineto{\pgfqpoint{7.327668in}{3.381367in}}%
\pgfpathlineto{\pgfqpoint{7.321465in}{3.355107in}}%
\pgfpathlineto{\pgfqpoint{7.319050in}{3.344130in}}%
\pgfpathlineto{\pgfqpoint{7.315675in}{3.328848in}}%
\pgfpathlineto{\pgfqpoint{7.310312in}{3.302589in}}%
\pgfpathlineto{\pgfqpoint{7.305396in}{3.276329in}}%
\pgfpathlineto{\pgfqpoint{7.300929in}{3.250070in}}%
\pgfpathlineto{\pgfqpoint{7.296914in}{3.223810in}}%
\pgfpathlineto{\pgfqpoint{7.293353in}{3.197551in}}%
\pgfpathlineto{\pgfqpoint{7.290249in}{3.171291in}}%
\pgfpathlineto{\pgfqpoint{7.287603in}{3.145032in}}%
\pgfpathlineto{\pgfqpoint{7.285419in}{3.118772in}}%
\pgfpathlineto{\pgfqpoint{7.284007in}{3.097233in}}%
\pgfpathlineto{\pgfqpoint{7.283696in}{3.092513in}}%
\pgfpathlineto{\pgfqpoint{7.282433in}{3.066253in}}%
\pgfpathlineto{\pgfqpoint{7.281642in}{3.039994in}}%
\pgfpathlineto{\pgfqpoint{7.281326in}{3.013734in}}%
\pgfpathlineto{\pgfqpoint{7.281489in}{2.987475in}}%
\pgfpathlineto{\pgfqpoint{7.282131in}{2.961215in}}%
\pgfpathlineto{\pgfqpoint{7.283257in}{2.934956in}}%
\pgfpathlineto{\pgfqpoint{7.284007in}{2.922742in}}%
\pgfpathlineto{\pgfqpoint{7.284863in}{2.908696in}}%
\pgfpathlineto{\pgfqpoint{7.286948in}{2.882437in}}%
\pgfpathlineto{\pgfqpoint{7.289522in}{2.856177in}}%
\pgfpathlineto{\pgfqpoint{7.292588in}{2.829918in}}%
\pgfpathlineto{\pgfqpoint{7.296149in}{2.803659in}}%
\pgfpathlineto{\pgfqpoint{7.300207in}{2.777399in}}%
\pgfpathlineto{\pgfqpoint{7.304767in}{2.751140in}}%
\pgfpathlineto{\pgfqpoint{7.309831in}{2.724880in}}%
\pgfpathlineto{\pgfqpoint{7.315404in}{2.698621in}}%
\pgfpathlineto{\pgfqpoint{7.319050in}{2.682885in}}%
\pgfpathlineto{\pgfqpoint{7.321470in}{2.672361in}}%
\pgfpathlineto{\pgfqpoint{7.328022in}{2.646102in}}%
\pgfpathlineto{\pgfqpoint{7.335090in}{2.619842in}}%
\pgfpathlineto{\pgfqpoint{7.342677in}{2.593583in}}%
\pgfpathlineto{\pgfqpoint{7.350787in}{2.567323in}}%
\pgfpathlineto{\pgfqpoint{7.354092in}{2.557272in}}%
\pgfpathlineto{\pgfqpoint{7.359387in}{2.541064in}}%
\pgfpathlineto{\pgfqpoint{7.368490in}{2.514804in}}%
\pgfpathlineto{\pgfqpoint{7.378126in}{2.488545in}}%
\pgfpathlineto{\pgfqpoint{7.388299in}{2.462285in}}%
\pgfpathlineto{\pgfqpoint{7.389135in}{2.460235in}}%
\pgfpathlineto{\pgfqpoint{7.398942in}{2.436026in}}%
\pgfpathlineto{\pgfqpoint{7.410122in}{2.409766in}}%
\pgfpathlineto{\pgfqpoint{7.421848in}{2.383507in}}%
\pgfpathlineto{\pgfqpoint{7.424177in}{2.378518in}}%
\pgfpathlineto{\pgfqpoint{7.434055in}{2.357248in}}%
\pgfpathlineto{\pgfqpoint{7.446798in}{2.330988in}}%
\pgfpathlineto{\pgfqpoint{7.459220in}{2.306462in}}%
\pgfpathlineto{\pgfqpoint{7.460093in}{2.304729in}}%
\pgfpathlineto{\pgfqpoint{7.473860in}{2.278469in}}%
\pgfpathlineto{\pgfqpoint{7.488191in}{2.252210in}}%
\pgfpathlineto{\pgfqpoint{7.494263in}{2.241496in}}%
\pgfpathlineto{\pgfqpoint{7.503030in}{2.225950in}}%
\pgfpathlineto{\pgfqpoint{7.518398in}{2.199691in}}%
\pgfpathlineto{\pgfqpoint{7.529305in}{2.181714in}}%
\pgfpathlineto{\pgfqpoint{7.534308in}{2.173431in}}%
\pgfpathlineto{\pgfqpoint{7.550721in}{2.147172in}}%
\pgfpathlineto{\pgfqpoint{7.564348in}{2.126110in}}%
\pgfpathlineto{\pgfqpoint{7.567696in}{2.120912in}}%
\pgfpathlineto{\pgfqpoint{7.585163in}{2.094653in}}%
\pgfpathlineto{\pgfqpoint{7.599390in}{2.073954in}}%
\pgfpathlineto{\pgfqpoint{7.603197in}{2.068393in}}%
\pgfpathlineto{\pgfqpoint{7.621728in}{2.042134in}}%
\pgfpathlineto{\pgfqpoint{7.634433in}{2.024679in}}%
\pgfpathlineto{\pgfqpoint{7.640817in}{2.015874in}}%
\pgfpathlineto{\pgfqpoint{7.660424in}{1.989615in}}%
\pgfpathlineto{\pgfqpoint{7.669475in}{1.977836in}}%
\pgfpathlineto{\pgfqpoint{7.680565in}{1.963355in}}%
\pgfpathlineto{\pgfqpoint{7.701258in}{1.937096in}}%
\pgfpathlineto{\pgfqpoint{7.704518in}{1.933067in}}%
\pgfpathlineto{\pgfqpoint{7.722449in}{1.910836in}}%
\pgfpathlineto{\pgfqpoint{7.739560in}{1.890201in}}%
\pgfusepath{stroke}%
\end{pgfscope}%
\begin{pgfscope}%
\pgfpathrectangle{\pgfqpoint{0.766095in}{0.571603in}}{\pgfqpoint{6.973465in}{5.225635in}}%
\pgfusepath{clip}%
\pgfsetbuttcap%
\pgfsetroundjoin%
\pgfsetlinewidth{1.505625pt}%
\definecolor{currentstroke}{rgb}{0.190631,0.407061,0.556089}%
\pgfsetstrokecolor{currentstroke}%
\pgfsetdash{}{0pt}%
\pgfpathmoveto{\pgfqpoint{7.597422in}{3.686987in}}%
\pgfpathlineto{\pgfqpoint{7.589995in}{3.670221in}}%
\pgfpathlineto{\pgfqpoint{7.578767in}{3.643962in}}%
\pgfpathlineto{\pgfqpoint{7.567959in}{3.617702in}}%
\pgfpathlineto{\pgfqpoint{7.564348in}{3.608603in}}%
\pgfpathlineto{\pgfqpoint{7.557521in}{3.591443in}}%
\pgfpathlineto{\pgfqpoint{7.547484in}{3.565183in}}%
\pgfpathlineto{\pgfqpoint{7.537876in}{3.538924in}}%
\pgfpathlineto{\pgfqpoint{7.529305in}{3.514405in}}%
\pgfpathlineto{\pgfqpoint{7.528695in}{3.512664in}}%
\pgfpathlineto{\pgfqpoint{7.519887in}{3.486405in}}%
\pgfpathlineto{\pgfqpoint{7.511517in}{3.460145in}}%
\pgfpathlineto{\pgfqpoint{7.503589in}{3.433886in}}%
\pgfpathlineto{\pgfqpoint{7.496104in}{3.407626in}}%
\pgfpathlineto{\pgfqpoint{7.494263in}{3.400778in}}%
\pgfpathlineto{\pgfqpoint{7.489025in}{3.381367in}}%
\pgfpathlineto{\pgfqpoint{7.482383in}{3.355107in}}%
\pgfpathlineto{\pgfqpoint{7.476193in}{3.328848in}}%
\pgfpathlineto{\pgfqpoint{7.470458in}{3.302589in}}%
\pgfpathlineto{\pgfqpoint{7.465178in}{3.276329in}}%
\pgfpathlineto{\pgfqpoint{7.460356in}{3.250070in}}%
\pgfpathlineto{\pgfqpoint{7.459220in}{3.243242in}}%
\pgfpathlineto{\pgfqpoint{7.455972in}{3.223810in}}%
\pgfpathlineto{\pgfqpoint{7.452044in}{3.197551in}}%
\pgfpathlineto{\pgfqpoint{7.448584in}{3.171291in}}%
\pgfpathlineto{\pgfqpoint{7.445594in}{3.145032in}}%
\pgfpathlineto{\pgfqpoint{7.443076in}{3.118772in}}%
\pgfpathlineto{\pgfqpoint{7.441032in}{3.092513in}}%
\pgfpathlineto{\pgfqpoint{7.439464in}{3.066253in}}%
\pgfpathlineto{\pgfqpoint{7.438376in}{3.039994in}}%
\pgfpathlineto{\pgfqpoint{7.437768in}{3.013734in}}%
\pgfpathlineto{\pgfqpoint{7.437643in}{2.987475in}}%
\pgfpathlineto{\pgfqpoint{7.438005in}{2.961215in}}%
\pgfpathlineto{\pgfqpoint{7.438855in}{2.934956in}}%
\pgfpathlineto{\pgfqpoint{7.440197in}{2.908696in}}%
\pgfpathlineto{\pgfqpoint{7.442032in}{2.882437in}}%
\pgfpathlineto{\pgfqpoint{7.444363in}{2.856177in}}%
\pgfpathlineto{\pgfqpoint{7.447195in}{2.829918in}}%
\pgfpathlineto{\pgfqpoint{7.450528in}{2.803659in}}%
\pgfpathlineto{\pgfqpoint{7.454367in}{2.777399in}}%
\pgfpathlineto{\pgfqpoint{7.458715in}{2.751140in}}%
\pgfpathlineto{\pgfqpoint{7.459220in}{2.748411in}}%
\pgfpathlineto{\pgfqpoint{7.463543in}{2.724880in}}%
\pgfpathlineto{\pgfqpoint{7.468880in}{2.698621in}}%
\pgfpathlineto{\pgfqpoint{7.474732in}{2.672361in}}%
\pgfpathlineto{\pgfqpoint{7.481101in}{2.646102in}}%
\pgfpathlineto{\pgfqpoint{7.487993in}{2.619842in}}%
\pgfpathlineto{\pgfqpoint{7.494263in}{2.597645in}}%
\pgfpathlineto{\pgfqpoint{7.495402in}{2.593583in}}%
\pgfpathlineto{\pgfqpoint{7.503294in}{2.567323in}}%
\pgfpathlineto{\pgfqpoint{7.511715in}{2.541064in}}%
\pgfpathlineto{\pgfqpoint{7.520669in}{2.514804in}}%
\pgfpathlineto{\pgfqpoint{7.529305in}{2.490914in}}%
\pgfpathlineto{\pgfqpoint{7.530156in}{2.488545in}}%
\pgfpathlineto{\pgfqpoint{7.540121in}{2.462285in}}%
\pgfpathlineto{\pgfqpoint{7.550629in}{2.436026in}}%
\pgfpathlineto{\pgfqpoint{7.561684in}{2.409766in}}%
\pgfpathlineto{\pgfqpoint{7.564348in}{2.403734in}}%
\pgfpathlineto{\pgfqpoint{7.573229in}{2.383507in}}%
\pgfpathlineto{\pgfqpoint{7.585308in}{2.357248in}}%
\pgfpathlineto{\pgfqpoint{7.597946in}{2.330988in}}%
\pgfpathlineto{\pgfqpoint{7.599390in}{2.328110in}}%
\pgfpathlineto{\pgfqpoint{7.611064in}{2.304729in}}%
\pgfpathlineto{\pgfqpoint{7.624737in}{2.278469in}}%
\pgfpathlineto{\pgfqpoint{7.634433in}{2.260580in}}%
\pgfpathlineto{\pgfqpoint{7.638948in}{2.252210in}}%
\pgfpathlineto{\pgfqpoint{7.653662in}{2.225950in}}%
\pgfpathlineto{\pgfqpoint{7.668953in}{2.199691in}}%
\pgfpathlineto{\pgfqpoint{7.669475in}{2.198825in}}%
\pgfpathlineto{\pgfqpoint{7.684721in}{2.173431in}}%
\pgfpathlineto{\pgfqpoint{7.701068in}{2.147172in}}%
\pgfpathlineto{\pgfqpoint{7.704518in}{2.141812in}}%
\pgfpathlineto{\pgfqpoint{7.717913in}{2.120912in}}%
\pgfpathlineto{\pgfqpoint{7.735326in}{2.094653in}}%
\pgfpathlineto{\pgfqpoint{7.739560in}{2.088465in}}%
\pgfusepath{stroke}%
\end{pgfscope}%
\begin{pgfscope}%
\pgfpathrectangle{\pgfqpoint{0.766095in}{0.571603in}}{\pgfqpoint{6.973465in}{5.225635in}}%
\pgfusepath{clip}%
\pgfsetbuttcap%
\pgfsetroundjoin%
\pgfsetlinewidth{1.505625pt}%
\definecolor{currentstroke}{rgb}{0.190631,0.407061,0.556089}%
\pgfsetstrokecolor{currentstroke}%
\pgfsetdash{}{0pt}%
\pgfpathmoveto{\pgfqpoint{0.766095in}{4.324652in}}%
\pgfpathlineto{\pgfqpoint{0.836180in}{4.366144in}}%
\pgfpathlineto{\pgfqpoint{0.906265in}{4.406177in}}%
\pgfpathlineto{\pgfqpoint{1.011393in}{4.463365in}}%
\pgfpathlineto{\pgfqpoint{1.116520in}{4.517623in}}%
\pgfpathlineto{\pgfqpoint{1.221648in}{4.569224in}}%
\pgfpathlineto{\pgfqpoint{1.326776in}{4.618418in}}%
\pgfpathlineto{\pgfqpoint{1.438081in}{4.668081in}}%
\pgfpathlineto{\pgfqpoint{1.572073in}{4.724837in}}%
\pgfpathlineto{\pgfqpoint{1.712244in}{4.781053in}}%
\pgfpathlineto{\pgfqpoint{1.852414in}{4.834368in}}%
\pgfpathlineto{\pgfqpoint{1.992584in}{4.885045in}}%
\pgfpathlineto{\pgfqpoint{2.132754in}{4.933326in}}%
\pgfpathlineto{\pgfqpoint{2.307967in}{4.990502in}}%
\pgfpathlineto{\pgfqpoint{2.453614in}{5.035714in}}%
\pgfpathlineto{\pgfqpoint{2.631735in}{5.088233in}}%
\pgfpathlineto{\pgfqpoint{2.833605in}{5.144438in}}%
\pgfpathlineto{\pgfqpoint{3.018948in}{5.193271in}}%
\pgfpathlineto{\pgfqpoint{3.229828in}{5.245790in}}%
\pgfpathlineto{\pgfqpoint{3.453698in}{5.298308in}}%
\pgfpathlineto{\pgfqpoint{3.674626in}{5.347084in}}%
\pgfpathlineto{\pgfqpoint{3.849838in}{5.383764in}}%
\pgfpathlineto{\pgfqpoint{4.080595in}{5.429606in}}%
\pgfpathlineto{\pgfqpoint{4.117347in}{5.436583in}}%
\pgfpathlineto{\pgfqpoint{4.117347in}{5.436583in}}%
\pgfusepath{stroke}%
\end{pgfscope}%
\begin{pgfscope}%
\pgfpathrectangle{\pgfqpoint{0.766095in}{0.571603in}}{\pgfqpoint{6.973465in}{5.225635in}}%
\pgfusepath{clip}%
\pgfsetbuttcap%
\pgfsetroundjoin%
\pgfsetlinewidth{1.505625pt}%
\definecolor{currentstroke}{rgb}{0.190631,0.407061,0.556089}%
\pgfsetstrokecolor{currentstroke}%
\pgfsetdash{}{0pt}%
\pgfpathmoveto{\pgfqpoint{4.423555in}{5.492523in}}%
\pgfpathlineto{\pgfqpoint{4.445562in}{5.496364in}}%
\pgfpathlineto{\pgfqpoint{4.480604in}{5.502462in}}%
\pgfpathlineto{\pgfqpoint{4.514760in}{5.508384in}}%
\pgfpathlineto{\pgfqpoint{4.515647in}{5.508535in}}%
\pgfpathlineto{\pgfqpoint{4.550689in}{5.514424in}}%
\pgfpathlineto{\pgfqpoint{4.585732in}{5.520291in}}%
\pgfpathlineto{\pgfqpoint{4.620774in}{5.526136in}}%
\pgfpathlineto{\pgfqpoint{4.655817in}{5.531960in}}%
\pgfpathlineto{\pgfqpoint{4.672100in}{5.534644in}}%
\pgfpathlineto{\pgfqpoint{4.690859in}{5.537673in}}%
\pgfpathlineto{\pgfqpoint{4.725902in}{5.543289in}}%
\pgfpathlineto{\pgfqpoint{4.760944in}{5.548880in}}%
\pgfpathlineto{\pgfqpoint{4.795987in}{5.554448in}}%
\pgfpathlineto{\pgfqpoint{4.831030in}{5.559991in}}%
\pgfpathlineto{\pgfqpoint{4.836860in}{5.560903in}}%
\pgfpathlineto{\pgfqpoint{4.866072in}{5.565379in}}%
\pgfpathlineto{\pgfqpoint{4.901115in}{5.570715in}}%
\pgfpathlineto{\pgfqpoint{4.936157in}{5.576025in}}%
\pgfpathlineto{\pgfqpoint{4.971200in}{5.581308in}}%
\pgfpathlineto{\pgfqpoint{5.006242in}{5.586565in}}%
\pgfpathlineto{\pgfqpoint{5.010277in}{5.587163in}}%
\pgfpathlineto{\pgfqpoint{5.041285in}{5.591660in}}%
\pgfpathlineto{\pgfqpoint{5.076327in}{5.596710in}}%
\pgfpathlineto{\pgfqpoint{5.111370in}{5.601731in}}%
\pgfpathlineto{\pgfqpoint{5.146412in}{5.606723in}}%
\pgfpathlineto{\pgfqpoint{5.181455in}{5.611685in}}%
\pgfpathlineto{\pgfqpoint{5.193866in}{5.613422in}}%
\pgfpathlineto{\pgfqpoint{5.216497in}{5.616522in}}%
\pgfpathlineto{\pgfqpoint{5.251540in}{5.621278in}}%
\pgfpathlineto{\pgfqpoint{5.286583in}{5.626001in}}%
\pgfpathlineto{\pgfqpoint{5.321625in}{5.630692in}}%
\pgfpathlineto{\pgfqpoint{5.356668in}{5.635350in}}%
\pgfpathlineto{\pgfqpoint{5.389506in}{5.639682in}}%
\pgfpathlineto{\pgfqpoint{5.391710in}{5.639966in}}%
\pgfpathlineto{\pgfqpoint{5.426753in}{5.644417in}}%
\pgfpathlineto{\pgfqpoint{5.461795in}{5.648833in}}%
\pgfpathlineto{\pgfqpoint{5.496838in}{5.653213in}}%
\pgfpathlineto{\pgfqpoint{5.531880in}{5.657557in}}%
\pgfpathlineto{\pgfqpoint{5.566923in}{5.661862in}}%
\pgfpathlineto{\pgfqpoint{5.600426in}{5.665941in}}%
\pgfpathlineto{\pgfqpoint{5.601965in}{5.666124in}}%
\pgfpathlineto{\pgfqpoint{5.637008in}{5.670222in}}%
\pgfpathlineto{\pgfqpoint{5.672050in}{5.674279in}}%
\pgfpathlineto{\pgfqpoint{5.707093in}{5.678295in}}%
\pgfpathlineto{\pgfqpoint{5.742136in}{5.682269in}}%
\pgfpathlineto{\pgfqpoint{5.777178in}{5.686200in}}%
\pgfpathlineto{\pgfqpoint{5.812221in}{5.690088in}}%
\pgfpathlineto{\pgfqpoint{5.831562in}{5.692201in}}%
\pgfpathlineto{\pgfqpoint{5.847263in}{5.693876in}}%
\pgfpathlineto{\pgfqpoint{5.882306in}{5.697551in}}%
\pgfpathlineto{\pgfqpoint{5.917348in}{5.701179in}}%
\pgfpathlineto{\pgfqpoint{5.952391in}{5.704759in}}%
\pgfpathlineto{\pgfqpoint{5.987433in}{5.708290in}}%
\pgfpathlineto{\pgfqpoint{6.022476in}{5.711771in}}%
\pgfpathlineto{\pgfqpoint{6.057518in}{5.715201in}}%
\pgfpathlineto{\pgfqpoint{6.091353in}{5.718460in}}%
\pgfpathlineto{\pgfqpoint{6.092561in}{5.718574in}}%
\pgfpathlineto{\pgfqpoint{6.127603in}{5.721786in}}%
\pgfpathlineto{\pgfqpoint{6.162646in}{5.724943in}}%
\pgfpathlineto{\pgfqpoint{6.197689in}{5.728043in}}%
\pgfpathlineto{\pgfqpoint{6.232731in}{5.731086in}}%
\pgfpathlineto{\pgfqpoint{6.267774in}{5.734069in}}%
\pgfpathlineto{\pgfqpoint{6.302816in}{5.736991in}}%
\pgfpathlineto{\pgfqpoint{6.337859in}{5.739850in}}%
\pgfpathlineto{\pgfqpoint{6.372901in}{5.742645in}}%
\pgfpathlineto{\pgfqpoint{6.399608in}{5.744720in}}%
\pgfpathlineto{\pgfqpoint{6.407944in}{5.745351in}}%
\pgfpathlineto{\pgfqpoint{6.442986in}{5.747918in}}%
\pgfpathlineto{\pgfqpoint{6.478029in}{5.750417in}}%
\pgfpathlineto{\pgfqpoint{6.513071in}{5.752844in}}%
\pgfpathlineto{\pgfqpoint{6.548114in}{5.755199in}}%
\pgfpathlineto{\pgfqpoint{6.583156in}{5.757479in}}%
\pgfpathlineto{\pgfqpoint{6.618199in}{5.759681in}}%
\pgfpathlineto{\pgfqpoint{6.653242in}{5.761804in}}%
\pgfpathlineto{\pgfqpoint{6.688284in}{5.763846in}}%
\pgfpathlineto{\pgfqpoint{6.723327in}{5.765803in}}%
\pgfpathlineto{\pgfqpoint{6.758369in}{5.767672in}}%
\pgfpathlineto{\pgfqpoint{6.793412in}{5.769453in}}%
\pgfpathlineto{\pgfqpoint{6.825141in}{5.770979in}}%
\pgfpathlineto{\pgfqpoint{6.828454in}{5.771134in}}%
\pgfpathlineto{\pgfqpoint{6.863497in}{5.772665in}}%
\pgfpathlineto{\pgfqpoint{6.898539in}{5.774101in}}%
\pgfpathlineto{\pgfqpoint{6.933582in}{5.775437in}}%
\pgfpathlineto{\pgfqpoint{6.968624in}{5.776671in}}%
\pgfpathlineto{\pgfqpoint{7.003667in}{5.777799in}}%
\pgfpathlineto{\pgfqpoint{7.038709in}{5.778817in}}%
\pgfpathlineto{\pgfqpoint{7.073752in}{5.779723in}}%
\pgfpathlineto{\pgfqpoint{7.108795in}{5.780511in}}%
\pgfpathlineto{\pgfqpoint{7.143837in}{5.781179in}}%
\pgfpathlineto{\pgfqpoint{7.178880in}{5.781721in}}%
\pgfpathlineto{\pgfqpoint{7.213922in}{5.782133in}}%
\pgfpathlineto{\pgfqpoint{7.248965in}{5.782410in}}%
\pgfpathlineto{\pgfqpoint{7.284007in}{5.782547in}}%
\pgfpathlineto{\pgfqpoint{7.319050in}{5.782539in}}%
\pgfpathlineto{\pgfqpoint{7.354092in}{5.782381in}}%
\pgfpathlineto{\pgfqpoint{7.389135in}{5.782066in}}%
\pgfpathlineto{\pgfqpoint{7.424177in}{5.781588in}}%
\pgfpathlineto{\pgfqpoint{7.459220in}{5.780941in}}%
\pgfpathlineto{\pgfqpoint{7.494263in}{5.780118in}}%
\pgfpathlineto{\pgfqpoint{7.529305in}{5.779111in}}%
\pgfpathlineto{\pgfqpoint{7.564348in}{5.777913in}}%
\pgfpathlineto{\pgfqpoint{7.599390in}{5.776516in}}%
\pgfpathlineto{\pgfqpoint{7.634433in}{5.774911in}}%
\pgfpathlineto{\pgfqpoint{7.669475in}{5.773088in}}%
\pgfpathlineto{\pgfqpoint{7.704518in}{5.771039in}}%
\pgfpathlineto{\pgfqpoint{7.705446in}{5.770979in}}%
\pgfpathlineto{\pgfqpoint{7.739560in}{5.768636in}}%
\pgfusepath{stroke}%
\end{pgfscope}%
\begin{pgfscope}%
\pgfpathrectangle{\pgfqpoint{0.766095in}{0.571603in}}{\pgfqpoint{6.973465in}{5.225635in}}%
\pgfusepath{clip}%
\pgfsetbuttcap%
\pgfsetroundjoin%
\pgfsetlinewidth{1.505625pt}%
\definecolor{currentstroke}{rgb}{0.182256,0.426184,0.557120}%
\pgfsetstrokecolor{currentstroke}%
\pgfsetdash{}{0pt}%
\pgfpathmoveto{\pgfqpoint{7.739560in}{3.644895in}}%
\pgfpathlineto{\pgfqpoint{7.739144in}{3.643962in}}%
\pgfpathlineto{\pgfqpoint{7.734049in}{3.632161in}}%
\pgfusepath{stroke}%
\end{pgfscope}%
\begin{pgfscope}%
\pgfpathrectangle{\pgfqpoint{0.766095in}{0.571603in}}{\pgfqpoint{6.973465in}{5.225635in}}%
\pgfusepath{clip}%
\pgfsetbuttcap%
\pgfsetroundjoin%
\pgfsetlinewidth{1.505625pt}%
\definecolor{currentstroke}{rgb}{0.182256,0.426184,0.557120}%
\pgfsetstrokecolor{currentstroke}%
\pgfsetdash{}{0pt}%
\pgfpathmoveto{\pgfqpoint{7.633412in}{3.336391in}}%
\pgfpathlineto{\pgfqpoint{7.631528in}{3.328848in}}%
\pgfpathlineto{\pgfqpoint{7.625420in}{3.302589in}}%
\pgfpathlineto{\pgfqpoint{7.619779in}{3.276329in}}%
\pgfpathlineto{\pgfqpoint{7.614607in}{3.250070in}}%
\pgfpathlineto{\pgfqpoint{7.609905in}{3.223810in}}%
\pgfpathlineto{\pgfqpoint{7.605675in}{3.197551in}}%
\pgfpathlineto{\pgfqpoint{7.601920in}{3.171291in}}%
\pgfpathlineto{\pgfqpoint{7.599390in}{3.151049in}}%
\pgfpathlineto{\pgfqpoint{7.598635in}{3.145032in}}%
\pgfpathlineto{\pgfqpoint{7.595813in}{3.118772in}}%
\pgfpathlineto{\pgfqpoint{7.593475in}{3.092513in}}%
\pgfpathlineto{\pgfqpoint{7.591622in}{3.066253in}}%
\pgfpathlineto{\pgfqpoint{7.590256in}{3.039994in}}%
\pgfpathlineto{\pgfqpoint{7.589379in}{3.013734in}}%
\pgfpathlineto{\pgfqpoint{7.588994in}{2.987475in}}%
\pgfpathlineto{\pgfqpoint{7.589102in}{2.961215in}}%
\pgfpathlineto{\pgfqpoint{7.589707in}{2.934956in}}%
\pgfpathlineto{\pgfqpoint{7.590810in}{2.908696in}}%
\pgfpathlineto{\pgfqpoint{7.592415in}{2.882437in}}%
\pgfpathlineto{\pgfqpoint{7.594523in}{2.856177in}}%
\pgfpathlineto{\pgfqpoint{7.597137in}{2.829918in}}%
\pgfpathlineto{\pgfqpoint{7.599390in}{2.810978in}}%
\pgfpathlineto{\pgfqpoint{7.600254in}{2.803659in}}%
\pgfpathlineto{\pgfqpoint{7.603865in}{2.777399in}}%
\pgfpathlineto{\pgfqpoint{7.607986in}{2.751140in}}%
\pgfpathlineto{\pgfqpoint{7.612623in}{2.724880in}}%
\pgfpathlineto{\pgfqpoint{7.617776in}{2.698621in}}%
\pgfpathlineto{\pgfqpoint{7.623450in}{2.672361in}}%
\pgfpathlineto{\pgfqpoint{7.629648in}{2.646102in}}%
\pgfpathlineto{\pgfqpoint{7.634433in}{2.627421in}}%
\pgfpathlineto{\pgfqpoint{7.636361in}{2.619842in}}%
\pgfpathlineto{\pgfqpoint{7.643569in}{2.593583in}}%
\pgfpathlineto{\pgfqpoint{7.651307in}{2.567323in}}%
\pgfpathlineto{\pgfqpoint{7.659580in}{2.541064in}}%
\pgfpathlineto{\pgfqpoint{7.668392in}{2.514804in}}%
\pgfpathlineto{\pgfqpoint{7.669475in}{2.511762in}}%
\pgfpathlineto{\pgfqpoint{7.677691in}{2.488545in}}%
\pgfpathlineto{\pgfqpoint{7.687526in}{2.462285in}}%
\pgfpathlineto{\pgfqpoint{7.697907in}{2.436026in}}%
\pgfpathlineto{\pgfqpoint{7.704518in}{2.420142in}}%
\pgfpathlineto{\pgfqpoint{7.708812in}{2.409766in}}%
\pgfpathlineto{\pgfqpoint{7.720224in}{2.383507in}}%
\pgfpathlineto{\pgfqpoint{7.732194in}{2.357248in}}%
\pgfpathlineto{\pgfqpoint{7.739560in}{2.341800in}}%
\pgfusepath{stroke}%
\end{pgfscope}%
\begin{pgfscope}%
\pgfpathrectangle{\pgfqpoint{0.766095in}{0.571603in}}{\pgfqpoint{6.973465in}{5.225635in}}%
\pgfusepath{clip}%
\pgfsetbuttcap%
\pgfsetroundjoin%
\pgfsetlinewidth{1.505625pt}%
\definecolor{currentstroke}{rgb}{0.182256,0.426184,0.557120}%
\pgfsetstrokecolor{currentstroke}%
\pgfsetdash{}{0pt}%
\pgfpathmoveto{\pgfqpoint{0.766095in}{4.406249in}}%
\pgfpathlineto{\pgfqpoint{0.871223in}{4.465175in}}%
\pgfpathlineto{\pgfqpoint{0.976350in}{4.521031in}}%
\pgfpathlineto{\pgfqpoint{1.081478in}{4.574100in}}%
\pgfpathlineto{\pgfqpoint{1.186605in}{4.624642in}}%
\pgfpathlineto{\pgfqpoint{1.291733in}{4.672896in}}%
\pgfpathlineto{\pgfqpoint{1.400465in}{4.720600in}}%
\pgfpathlineto{\pgfqpoint{1.537031in}{4.777530in}}%
\pgfpathlineto{\pgfqpoint{1.677201in}{4.832924in}}%
\pgfpathlineto{\pgfqpoint{1.817371in}{4.885538in}}%
\pgfpathlineto{\pgfqpoint{1.957541in}{4.935623in}}%
\pgfpathlineto{\pgfqpoint{2.097711in}{4.983413in}}%
\pgfpathlineto{\pgfqpoint{2.272924in}{5.040073in}}%
\pgfpathlineto{\pgfqpoint{2.448137in}{5.093746in}}%
\pgfpathlineto{\pgfqpoint{2.623350in}{5.144693in}}%
\pgfpathlineto{\pgfqpoint{2.799002in}{5.193271in}}%
\pgfpathlineto{\pgfqpoint{3.008818in}{5.248186in}}%
\pgfpathlineto{\pgfqpoint{3.219073in}{5.300203in}}%
\pgfpathlineto{\pgfqpoint{3.435495in}{5.350827in}}%
\pgfpathlineto{\pgfqpoint{3.674626in}{5.403567in}}%
\pgfpathlineto{\pgfqpoint{3.927091in}{5.455865in}}%
\pgfpathlineto{\pgfqpoint{4.165221in}{5.502181in}}%
\pgfpathlineto{\pgfqpoint{4.340583in}{5.534644in}}%
\pgfpathlineto{\pgfqpoint{4.620774in}{5.583427in}}%
\pgfpathlineto{\pgfqpoint{4.803535in}{5.613422in}}%
\pgfpathlineto{\pgfqpoint{5.076327in}{5.655353in}}%
\pgfpathlineto{\pgfqpoint{5.167521in}{5.668734in}}%
\pgfpathlineto{\pgfqpoint{5.167521in}{5.668734in}}%
\pgfusepath{stroke}%
\end{pgfscope}%
\begin{pgfscope}%
\pgfpathrectangle{\pgfqpoint{0.766095in}{0.571603in}}{\pgfqpoint{6.973465in}{5.225635in}}%
\pgfusepath{clip}%
\pgfsetbuttcap%
\pgfsetroundjoin%
\pgfsetlinewidth{1.505625pt}%
\definecolor{currentstroke}{rgb}{0.182256,0.426184,0.557120}%
\pgfsetstrokecolor{currentstroke}%
\pgfsetdash{}{0pt}%
\pgfpathmoveto{\pgfqpoint{5.475876in}{5.711114in}}%
\pgfpathlineto{\pgfqpoint{5.496838in}{5.713875in}}%
\pgfpathlineto{\pgfqpoint{5.531874in}{5.718460in}}%
\pgfpathlineto{\pgfqpoint{5.531880in}{5.718461in}}%
\pgfpathlineto{\pgfqpoint{5.566923in}{5.722880in}}%
\pgfpathlineto{\pgfqpoint{5.601965in}{5.727269in}}%
\pgfpathlineto{\pgfqpoint{5.637008in}{5.731626in}}%
\pgfpathlineto{\pgfqpoint{5.672050in}{5.735950in}}%
\pgfpathlineto{\pgfqpoint{5.707093in}{5.740242in}}%
\pgfpathlineto{\pgfqpoint{5.742136in}{5.744501in}}%
\pgfpathlineto{\pgfqpoint{5.743966in}{5.744720in}}%
\pgfpathlineto{\pgfqpoint{5.777178in}{5.748602in}}%
\pgfpathlineto{\pgfqpoint{5.812221in}{5.752661in}}%
\pgfpathlineto{\pgfqpoint{5.847263in}{5.756685in}}%
\pgfpathlineto{\pgfqpoint{5.882306in}{5.760671in}}%
\pgfpathlineto{\pgfqpoint{5.917348in}{5.764621in}}%
\pgfpathlineto{\pgfqpoint{5.952391in}{5.768532in}}%
\pgfpathlineto{\pgfqpoint{5.974607in}{5.770979in}}%
\pgfpathlineto{\pgfqpoint{5.987433in}{5.772359in}}%
\pgfpathlineto{\pgfqpoint{6.022476in}{5.776070in}}%
\pgfpathlineto{\pgfqpoint{6.057518in}{5.779739in}}%
\pgfpathlineto{\pgfqpoint{6.092561in}{5.783367in}}%
\pgfpathlineto{\pgfqpoint{6.127603in}{5.786951in}}%
\pgfpathlineto{\pgfqpoint{6.162646in}{5.790492in}}%
\pgfpathlineto{\pgfqpoint{6.197689in}{5.793988in}}%
\pgfpathlineto{\pgfqpoint{6.230715in}{5.797238in}}%
\pgfusepath{stroke}%
\end{pgfscope}%
\begin{pgfscope}%
\pgfpathrectangle{\pgfqpoint{0.766095in}{0.571603in}}{\pgfqpoint{6.973465in}{5.225635in}}%
\pgfusepath{clip}%
\pgfsetbuttcap%
\pgfsetroundjoin%
\pgfsetlinewidth{1.505625pt}%
\definecolor{currentstroke}{rgb}{0.174274,0.445044,0.557792}%
\pgfsetstrokecolor{currentstroke}%
\pgfsetdash{}{0pt}%
\pgfpathmoveto{\pgfqpoint{7.739560in}{3.069086in}}%
\pgfpathlineto{\pgfqpoint{7.739333in}{3.066253in}}%
\pgfpathlineto{\pgfqpoint{7.737713in}{3.039994in}}%
\pgfpathlineto{\pgfqpoint{7.736590in}{3.013734in}}%
\pgfpathlineto{\pgfqpoint{7.735965in}{2.987475in}}%
\pgfpathlineto{\pgfqpoint{7.735841in}{2.961215in}}%
\pgfpathlineto{\pgfqpoint{7.736220in}{2.934956in}}%
\pgfpathlineto{\pgfqpoint{7.737104in}{2.908696in}}%
\pgfpathlineto{\pgfqpoint{7.738495in}{2.882437in}}%
\pgfpathlineto{\pgfqpoint{7.739560in}{2.867728in}}%
\pgfusepath{stroke}%
\end{pgfscope}%
\begin{pgfscope}%
\pgfpathrectangle{\pgfqpoint{0.766095in}{0.571603in}}{\pgfqpoint{6.973465in}{5.225635in}}%
\pgfusepath{clip}%
\pgfsetbuttcap%
\pgfsetroundjoin%
\pgfsetlinewidth{1.505625pt}%
\definecolor{currentstroke}{rgb}{0.174274,0.445044,0.557792}%
\pgfsetstrokecolor{currentstroke}%
\pgfsetdash{}{0pt}%
\pgfpathmoveto{\pgfqpoint{0.766095in}{4.482351in}}%
\pgfpathlineto{\pgfqpoint{0.836180in}{4.520361in}}%
\pgfpathlineto{\pgfqpoint{0.941308in}{4.575025in}}%
\pgfpathlineto{\pgfqpoint{1.046435in}{4.627034in}}%
\pgfpathlineto{\pgfqpoint{1.151563in}{4.676638in}}%
\pgfpathlineto{\pgfqpoint{1.256691in}{4.724064in}}%
\pgfpathlineto{\pgfqpoint{1.370497in}{4.773119in}}%
\pgfpathlineto{\pgfqpoint{1.501988in}{4.827100in}}%
\pgfpathlineto{\pgfqpoint{1.642158in}{4.881748in}}%
\pgfpathlineto{\pgfqpoint{1.782329in}{4.933728in}}%
\pgfpathlineto{\pgfqpoint{1.922499in}{4.983285in}}%
\pgfpathlineto{\pgfqpoint{2.097711in}{5.042007in}}%
\pgfpathlineto{\pgfqpoint{2.242504in}{5.088233in}}%
\pgfpathlineto{\pgfqpoint{2.415267in}{5.140752in}}%
\pgfpathlineto{\pgfqpoint{2.623350in}{5.200542in}}%
\pgfpathlineto{\pgfqpoint{2.798562in}{5.248343in}}%
\pgfpathlineto{\pgfqpoint{3.008818in}{5.302750in}}%
\pgfpathlineto{\pgfqpoint{3.219073in}{5.354282in}}%
\pgfpathlineto{\pgfqpoint{3.430091in}{5.403346in}}%
\pgfpathlineto{\pgfqpoint{3.674626in}{5.457045in}}%
\pgfpathlineto{\pgfqpoint{3.922687in}{5.508384in}}%
\pgfpathlineto{\pgfqpoint{4.200264in}{5.562302in}}%
\pgfpathlineto{\pgfqpoint{4.481519in}{5.613422in}}%
\pgfpathlineto{\pgfqpoint{4.642448in}{5.641109in}}%
\pgfpathlineto{\pgfqpoint{4.642448in}{5.641109in}}%
\pgfusepath{stroke}%
\end{pgfscope}%
\begin{pgfscope}%
\pgfpathrectangle{\pgfqpoint{0.766095in}{0.571603in}}{\pgfqpoint{6.973465in}{5.225635in}}%
\pgfusepath{clip}%
\pgfsetbuttcap%
\pgfsetroundjoin%
\pgfsetlinewidth{1.505625pt}%
\definecolor{currentstroke}{rgb}{0.174274,0.445044,0.557792}%
\pgfsetstrokecolor{currentstroke}%
\pgfsetdash{}{0pt}%
\pgfpathmoveto{\pgfqpoint{4.949683in}{5.691041in}}%
\pgfpathlineto{\pgfqpoint{4.957035in}{5.692201in}}%
\pgfpathlineto{\pgfqpoint{4.971200in}{5.694389in}}%
\pgfpathlineto{\pgfqpoint{5.006242in}{5.699763in}}%
\pgfpathlineto{\pgfqpoint{5.041285in}{5.705118in}}%
\pgfpathlineto{\pgfqpoint{5.076327in}{5.710455in}}%
\pgfpathlineto{\pgfqpoint{5.111370in}{5.715773in}}%
\pgfpathlineto{\pgfqpoint{5.129206in}{5.718460in}}%
\pgfpathlineto{\pgfqpoint{5.146412in}{5.720999in}}%
\pgfpathlineto{\pgfqpoint{5.181455in}{5.726130in}}%
\pgfpathlineto{\pgfqpoint{5.216497in}{5.731240in}}%
\pgfpathlineto{\pgfqpoint{5.251540in}{5.736331in}}%
\pgfpathlineto{\pgfqpoint{5.286583in}{5.741400in}}%
\pgfpathlineto{\pgfqpoint{5.309681in}{5.744720in}}%
\pgfpathlineto{\pgfqpoint{5.321625in}{5.746400in}}%
\pgfpathlineto{\pgfqpoint{5.356668in}{5.751283in}}%
\pgfpathlineto{\pgfqpoint{5.391710in}{5.756144in}}%
\pgfpathlineto{\pgfqpoint{5.426753in}{5.760982in}}%
\pgfpathlineto{\pgfqpoint{5.461795in}{5.765797in}}%
\pgfpathlineto{\pgfqpoint{5.496838in}{5.770590in}}%
\pgfpathlineto{\pgfqpoint{5.499721in}{5.770979in}}%
\pgfpathlineto{\pgfqpoint{5.531880in}{5.775230in}}%
\pgfpathlineto{\pgfqpoint{5.566923in}{5.779835in}}%
\pgfpathlineto{\pgfqpoint{5.601965in}{5.784414in}}%
\pgfpathlineto{\pgfqpoint{5.637008in}{5.788969in}}%
\pgfpathlineto{\pgfqpoint{5.672050in}{5.793497in}}%
\pgfpathlineto{\pgfqpoint{5.701216in}{5.797238in}}%
\pgfusepath{stroke}%
\end{pgfscope}%
\begin{pgfscope}%
\pgfpathrectangle{\pgfqpoint{0.766095in}{0.571603in}}{\pgfqpoint{6.973465in}{5.225635in}}%
\pgfusepath{clip}%
\pgfsetbuttcap%
\pgfsetroundjoin%
\pgfsetlinewidth{1.505625pt}%
\definecolor{currentstroke}{rgb}{0.166617,0.463708,0.558119}%
\pgfsetstrokecolor{currentstroke}%
\pgfsetdash{}{0pt}%
\pgfpathmoveto{\pgfqpoint{0.766095in}{4.553685in}}%
\pgfpathlineto{\pgfqpoint{0.783700in}{4.563043in}}%
\pgfpathlineto{\pgfqpoint{0.801138in}{4.572161in}}%
\pgfpathlineto{\pgfqpoint{0.834075in}{4.589303in}}%
\pgfpathlineto{\pgfqpoint{0.836180in}{4.590380in}}%
\pgfpathlineto{\pgfqpoint{0.871223in}{4.608196in}}%
\pgfpathlineto{\pgfqpoint{0.885781in}{4.615562in}}%
\pgfpathlineto{\pgfqpoint{0.906265in}{4.625757in}}%
\pgfpathlineto{\pgfqpoint{0.938675in}{4.641822in}}%
\pgfpathlineto{\pgfqpoint{0.941308in}{4.643105in}}%
\pgfpathlineto{\pgfqpoint{0.976350in}{4.660074in}}%
\pgfpathlineto{\pgfqpoint{0.992956in}{4.668081in}}%
\pgfpathlineto{\pgfqpoint{1.011393in}{4.676825in}}%
\pgfpathlineto{\pgfqpoint{1.046435in}{4.693378in}}%
\pgfpathlineto{\pgfqpoint{1.048486in}{4.694341in}}%
\pgfpathlineto{\pgfqpoint{1.081478in}{4.709566in}}%
\pgfpathlineto{\pgfqpoint{1.105469in}{4.720600in}}%
\pgfpathlineto{\pgfqpoint{1.116520in}{4.725600in}}%
\pgfpathlineto{\pgfqpoint{1.151563in}{4.741371in}}%
\pgfpathlineto{\pgfqpoint{1.163820in}{4.746860in}}%
\pgfpathlineto{\pgfqpoint{1.186605in}{4.756895in}}%
\pgfpathlineto{\pgfqpoint{1.221648in}{4.772281in}}%
\pgfpathlineto{\pgfqpoint{1.223569in}{4.773119in}}%
\pgfpathlineto{\pgfqpoint{1.256691in}{4.787326in}}%
\pgfpathlineto{\pgfqpoint{1.284859in}{4.799378in}}%
\pgfpathlineto{\pgfqpoint{1.291733in}{4.802271in}}%
\pgfpathlineto{\pgfqpoint{1.326776in}{4.816932in}}%
\pgfpathlineto{\pgfqpoint{1.347657in}{4.825638in}}%
\pgfpathlineto{\pgfqpoint{1.361818in}{4.831445in}}%
\pgfpathlineto{\pgfqpoint{1.396861in}{4.845749in}}%
\pgfpathlineto{\pgfqpoint{1.411991in}{4.851897in}}%
\pgfpathlineto{\pgfqpoint{1.431903in}{4.859856in}}%
\pgfpathlineto{\pgfqpoint{1.466946in}{4.873813in}}%
\pgfpathlineto{\pgfqpoint{1.477909in}{4.878157in}}%
\pgfpathlineto{\pgfqpoint{1.501988in}{4.887540in}}%
\pgfpathlineto{\pgfqpoint{1.537031in}{4.901158in}}%
\pgfpathlineto{\pgfqpoint{1.545463in}{4.904416in}}%
\pgfpathlineto{\pgfqpoint{1.572073in}{4.914530in}}%
\pgfpathlineto{\pgfqpoint{1.607116in}{4.927817in}}%
\pgfpathlineto{\pgfqpoint{1.614699in}{4.930676in}}%
\pgfpathlineto{\pgfqpoint{1.642158in}{4.940857in}}%
\pgfpathlineto{\pgfqpoint{1.677201in}{4.953822in}}%
\pgfpathlineto{\pgfqpoint{1.685663in}{4.956935in}}%
\pgfpathlineto{\pgfqpoint{1.712244in}{4.966553in}}%
\pgfpathlineto{\pgfqpoint{1.747286in}{4.979202in}}%
\pgfpathlineto{\pgfqpoint{1.758401in}{4.983195in}}%
\pgfpathlineto{\pgfqpoint{1.782329in}{4.991646in}}%
\pgfpathlineto{\pgfqpoint{1.817371in}{5.003988in}}%
\pgfpathlineto{\pgfqpoint{1.832955in}{5.009454in}}%
\pgfpathlineto{\pgfqpoint{1.852414in}{5.016165in}}%
\pgfpathlineto{\pgfqpoint{1.887456in}{5.028208in}}%
\pgfpathlineto{\pgfqpoint{1.909365in}{5.035714in}}%
\pgfpathlineto{\pgfqpoint{1.922499in}{5.040138in}}%
\pgfpathlineto{\pgfqpoint{1.957541in}{5.051888in}}%
\pgfpathlineto{\pgfqpoint{1.987670in}{5.061973in}}%
\pgfpathlineto{\pgfqpoint{1.992584in}{5.063591in}}%
\pgfpathlineto{\pgfqpoint{2.027626in}{5.075054in}}%
\pgfpathlineto{\pgfqpoint{2.062669in}{5.086509in}}%
\pgfpathlineto{\pgfqpoint{2.067973in}{5.088233in}}%
\pgfpathlineto{\pgfqpoint{2.097711in}{5.097733in}}%
\pgfpathlineto{\pgfqpoint{2.132754in}{5.108908in}}%
\pgfpathlineto{\pgfqpoint{2.150330in}{5.114492in}}%
\pgfpathlineto{\pgfqpoint{2.167797in}{5.119947in}}%
\pgfpathlineto{\pgfqpoint{2.202839in}{5.130850in}}%
\pgfpathlineto{\pgfqpoint{2.234707in}{5.140752in}}%
\pgfpathlineto{\pgfqpoint{2.237882in}{5.141721in}}%
\pgfpathlineto{\pgfqpoint{2.272924in}{5.152357in}}%
\pgfpathlineto{\pgfqpoint{2.307967in}{5.162985in}}%
\pgfpathlineto{\pgfqpoint{2.321304in}{5.167011in}}%
\pgfpathlineto{\pgfqpoint{2.343009in}{5.173452in}}%
\pgfpathlineto{\pgfqpoint{2.378052in}{5.183819in}}%
\pgfpathlineto{\pgfqpoint{2.410041in}{5.193271in}}%
\pgfpathlineto{\pgfqpoint{2.413094in}{5.194157in}}%
\pgfpathlineto{\pgfqpoint{2.448137in}{5.204269in}}%
\pgfpathlineto{\pgfqpoint{2.483179in}{5.214374in}}%
\pgfpathlineto{\pgfqpoint{2.501130in}{5.219530in}}%
\pgfpathlineto{\pgfqpoint{2.518222in}{5.224356in}}%
\pgfpathlineto{\pgfqpoint{2.553264in}{5.234211in}}%
\pgfpathlineto{\pgfqpoint{2.588307in}{5.244058in}}%
\pgfpathlineto{\pgfqpoint{2.594505in}{5.245790in}}%
\pgfpathlineto{\pgfqpoint{2.623350in}{5.253709in}}%
\pgfpathlineto{\pgfqpoint{2.658392in}{5.263313in}}%
\pgfpathlineto{\pgfqpoint{2.690312in}{5.272049in}}%
\pgfpathlineto{\pgfqpoint{2.693435in}{5.272889in}}%
\pgfpathlineto{\pgfqpoint{2.728477in}{5.282253in}}%
\pgfpathlineto{\pgfqpoint{2.763520in}{5.291611in}}%
\pgfpathlineto{\pgfqpoint{2.788665in}{5.298308in}}%
\pgfpathlineto{\pgfqpoint{2.798562in}{5.300899in}}%
\pgfpathlineto{\pgfqpoint{2.833605in}{5.310022in}}%
\pgfpathlineto{\pgfqpoint{2.868647in}{5.319139in}}%
\pgfpathlineto{\pgfqpoint{2.889585in}{5.324568in}}%
\pgfpathlineto{\pgfqpoint{2.903690in}{5.328161in}}%
\pgfpathlineto{\pgfqpoint{2.938732in}{5.337049in}}%
\pgfpathlineto{\pgfqpoint{2.973775in}{5.345929in}}%
\pgfpathlineto{\pgfqpoint{2.993174in}{5.350827in}}%
\pgfpathlineto{\pgfqpoint{3.008818in}{5.354708in}}%
\pgfpathlineto{\pgfqpoint{3.043860in}{5.363364in}}%
\pgfpathlineto{\pgfqpoint{3.078903in}{5.372012in}}%
\pgfpathlineto{\pgfqpoint{3.099533in}{5.377087in}}%
\pgfpathlineto{\pgfqpoint{3.113945in}{5.380569in}}%
\pgfpathlineto{\pgfqpoint{3.148988in}{5.388997in}}%
\pgfpathlineto{\pgfqpoint{3.184030in}{5.397419in}}%
\pgfpathlineto{\pgfqpoint{3.208761in}{5.403346in}}%
\pgfpathlineto{\pgfqpoint{3.219073in}{5.405774in}}%
\pgfpathlineto{\pgfqpoint{3.254115in}{5.413979in}}%
\pgfpathlineto{\pgfqpoint{3.289158in}{5.422177in}}%
\pgfpathlineto{\pgfqpoint{3.320955in}{5.429606in}}%
\pgfpathlineto{\pgfqpoint{3.324200in}{5.430350in}}%
\pgfpathlineto{\pgfqpoint{3.359243in}{5.438337in}}%
\pgfpathlineto{\pgfqpoint{3.394285in}{5.446316in}}%
\pgfpathlineto{\pgfqpoint{3.429328in}{5.454287in}}%
\pgfpathlineto{\pgfqpoint{3.436306in}{5.455865in}}%
\pgfpathlineto{\pgfqpoint{3.464371in}{5.462097in}}%
\pgfpathlineto{\pgfqpoint{3.499413in}{5.469860in}}%
\pgfpathlineto{\pgfqpoint{3.534456in}{5.477616in}}%
\pgfpathlineto{\pgfqpoint{3.554901in}{5.482125in}}%
\pgfpathlineto{\pgfqpoint{3.569498in}{5.485285in}}%
\pgfpathlineto{\pgfqpoint{3.604541in}{5.492837in}}%
\pgfpathlineto{\pgfqpoint{3.639583in}{5.500380in}}%
\pgfpathlineto{\pgfqpoint{3.674626in}{5.507916in}}%
\pgfpathlineto{\pgfqpoint{3.676818in}{5.508384in}}%
\pgfpathlineto{\pgfqpoint{3.709668in}{5.515270in}}%
\pgfpathlineto{\pgfqpoint{3.732434in}{5.520035in}}%
\pgfusepath{stroke}%
\end{pgfscope}%
\begin{pgfscope}%
\pgfpathrectangle{\pgfqpoint{0.766095in}{0.571603in}}{\pgfqpoint{6.973465in}{5.225635in}}%
\pgfusepath{clip}%
\pgfsetbuttcap%
\pgfsetroundjoin%
\pgfsetlinewidth{1.505625pt}%
\definecolor{currentstroke}{rgb}{0.166617,0.463708,0.558119}%
\pgfsetstrokecolor{currentstroke}%
\pgfsetdash{}{0pt}%
\pgfpathmoveto{\pgfqpoint{4.037532in}{5.581810in}}%
\pgfpathlineto{\pgfqpoint{4.060094in}{5.586251in}}%
\pgfpathlineto{\pgfqpoint{4.064756in}{5.587163in}}%
\pgfpathlineto{\pgfqpoint{4.095136in}{5.592990in}}%
\pgfpathlineto{\pgfqpoint{4.130179in}{5.599696in}}%
\pgfpathlineto{\pgfqpoint{4.165221in}{5.606392in}}%
\pgfpathlineto{\pgfqpoint{4.200264in}{5.613080in}}%
\pgfpathlineto{\pgfqpoint{4.202071in}{5.613422in}}%
\pgfpathlineto{\pgfqpoint{4.235306in}{5.619596in}}%
\pgfpathlineto{\pgfqpoint{4.270349in}{5.626093in}}%
\pgfpathlineto{\pgfqpoint{4.305391in}{5.632580in}}%
\pgfpathlineto{\pgfqpoint{4.340434in}{5.639057in}}%
\pgfpathlineto{\pgfqpoint{4.343837in}{5.639682in}}%
\pgfpathlineto{\pgfqpoint{4.375477in}{5.645374in}}%
\pgfpathlineto{\pgfqpoint{4.410519in}{5.651663in}}%
\pgfpathlineto{\pgfqpoint{4.445562in}{5.657942in}}%
\pgfpathlineto{\pgfqpoint{4.480604in}{5.664210in}}%
\pgfpathlineto{\pgfqpoint{4.490349in}{5.665941in}}%
\pgfpathlineto{\pgfqpoint{4.515647in}{5.670349in}}%
\pgfpathlineto{\pgfqpoint{4.550689in}{5.676431in}}%
\pgfpathlineto{\pgfqpoint{4.585732in}{5.682502in}}%
\pgfpathlineto{\pgfqpoint{4.620774in}{5.688561in}}%
\pgfpathlineto{\pgfqpoint{4.641922in}{5.692201in}}%
\pgfpathlineto{\pgfqpoint{4.655817in}{5.694545in}}%
\pgfpathlineto{\pgfqpoint{4.690859in}{5.700421in}}%
\pgfpathlineto{\pgfqpoint{4.725902in}{5.706284in}}%
\pgfpathlineto{\pgfqpoint{4.760944in}{5.712135in}}%
\pgfpathlineto{\pgfqpoint{4.795987in}{5.717973in}}%
\pgfpathlineto{\pgfqpoint{4.798937in}{5.718460in}}%
\pgfpathlineto{\pgfqpoint{4.831030in}{5.723655in}}%
\pgfpathlineto{\pgfqpoint{4.866072in}{5.729311in}}%
\pgfpathlineto{\pgfqpoint{4.901115in}{5.734953in}}%
\pgfpathlineto{\pgfqpoint{4.936157in}{5.740581in}}%
\pgfpathlineto{\pgfqpoint{4.962036in}{5.744720in}}%
\pgfpathlineto{\pgfqpoint{4.971200in}{5.746155in}}%
\pgfpathlineto{\pgfqpoint{5.006242in}{5.751603in}}%
\pgfpathlineto{\pgfqpoint{5.041285in}{5.757036in}}%
\pgfpathlineto{\pgfqpoint{5.076327in}{5.762454in}}%
\pgfpathlineto{\pgfqpoint{5.111370in}{5.767857in}}%
\pgfpathlineto{\pgfqpoint{5.131738in}{5.770979in}}%
\pgfpathlineto{\pgfqpoint{5.146412in}{5.773182in}}%
\pgfpathlineto{\pgfqpoint{5.181455in}{5.778406in}}%
\pgfpathlineto{\pgfqpoint{5.216497in}{5.783612in}}%
\pgfpathlineto{\pgfqpoint{5.251540in}{5.788803in}}%
\pgfpathlineto{\pgfqpoint{5.286583in}{5.793976in}}%
\pgfpathlineto{\pgfqpoint{5.308810in}{5.797238in}}%
\pgfusepath{stroke}%
\end{pgfscope}%
\begin{pgfscope}%
\pgfpathrectangle{\pgfqpoint{0.766095in}{0.571603in}}{\pgfqpoint{6.973465in}{5.225635in}}%
\pgfusepath{clip}%
\pgfsetbuttcap%
\pgfsetroundjoin%
\pgfsetlinewidth{1.505625pt}%
\definecolor{currentstroke}{rgb}{0.159194,0.482237,0.558073}%
\pgfsetstrokecolor{currentstroke}%
\pgfsetdash{}{0pt}%
\pgfpathmoveto{\pgfqpoint{0.766095in}{4.620952in}}%
\pgfpathlineto{\pgfqpoint{0.871223in}{4.673597in}}%
\pgfpathlineto{\pgfqpoint{0.976350in}{4.723829in}}%
\pgfpathlineto{\pgfqpoint{1.084317in}{4.773119in}}%
\pgfpathlineto{\pgfqpoint{1.221648in}{4.832670in}}%
\pgfpathlineto{\pgfqpoint{1.331529in}{4.878157in}}%
\pgfpathlineto{\pgfqpoint{1.466946in}{4.931657in}}%
\pgfpathlineto{\pgfqpoint{1.607116in}{4.984382in}}%
\pgfpathlineto{\pgfqpoint{1.750291in}{5.035714in}}%
\pgfpathlineto{\pgfqpoint{1.922499in}{5.094323in}}%
\pgfpathlineto{\pgfqpoint{2.097711in}{5.150864in}}%
\pgfpathlineto{\pgfqpoint{2.237882in}{5.194109in}}%
\pgfpathlineto{\pgfqpoint{2.413275in}{5.245790in}}%
\pgfpathlineto{\pgfqpoint{2.623350in}{5.304462in}}%
\pgfpathlineto{\pgfqpoint{2.798562in}{5.351088in}}%
\pgfpathlineto{\pgfqpoint{3.008818in}{5.404327in}}%
\pgfpathlineto{\pgfqpoint{3.254115in}{5.463064in}}%
\pgfpathlineto{\pgfqpoint{3.464371in}{5.510870in}}%
\pgfpathlineto{\pgfqpoint{3.709668in}{5.563785in}}%
\pgfpathlineto{\pgfqpoint{3.954966in}{5.613924in}}%
\pgfpathlineto{\pgfqpoint{4.222420in}{5.665581in}}%
\pgfpathlineto{\pgfqpoint{4.222420in}{5.665581in}}%
\pgfusepath{stroke}%
\end{pgfscope}%
\begin{pgfscope}%
\pgfpathrectangle{\pgfqpoint{0.766095in}{0.571603in}}{\pgfqpoint{6.973465in}{5.225635in}}%
\pgfusepath{clip}%
\pgfsetbuttcap%
\pgfsetroundjoin%
\pgfsetlinewidth{1.505625pt}%
\definecolor{currentstroke}{rgb}{0.159194,0.482237,0.558073}%
\pgfsetstrokecolor{currentstroke}%
\pgfsetdash{}{0pt}%
\pgfpathmoveto{\pgfqpoint{4.528694in}{5.721166in}}%
\pgfpathlineto{\pgfqpoint{4.550689in}{5.724987in}}%
\pgfpathlineto{\pgfqpoint{4.585732in}{5.731067in}}%
\pgfpathlineto{\pgfqpoint{4.620774in}{5.737137in}}%
\pgfpathlineto{\pgfqpoint{4.655817in}{5.743197in}}%
\pgfpathlineto{\pgfqpoint{4.664676in}{5.744720in}}%
\pgfpathlineto{\pgfqpoint{4.690859in}{5.749131in}}%
\pgfpathlineto{\pgfqpoint{4.725902in}{5.755014in}}%
\pgfpathlineto{\pgfqpoint{4.760944in}{5.760887in}}%
\pgfpathlineto{\pgfqpoint{4.795987in}{5.766750in}}%
\pgfpathlineto{\pgfqpoint{4.821359in}{5.770979in}}%
\pgfpathlineto{\pgfqpoint{4.831030in}{5.772559in}}%
\pgfpathlineto{\pgfqpoint{4.866072in}{5.778247in}}%
\pgfpathlineto{\pgfqpoint{4.901115in}{5.783923in}}%
\pgfpathlineto{\pgfqpoint{4.936157in}{5.789588in}}%
\pgfpathlineto{\pgfqpoint{4.971200in}{5.795242in}}%
\pgfpathlineto{\pgfqpoint{4.983652in}{5.797238in}}%
\pgfusepath{stroke}%
\end{pgfscope}%
\begin{pgfscope}%
\pgfpathrectangle{\pgfqpoint{0.766095in}{0.571603in}}{\pgfqpoint{6.973465in}{5.225635in}}%
\pgfusepath{clip}%
\pgfsetbuttcap%
\pgfsetroundjoin%
\pgfsetlinewidth{1.505625pt}%
\definecolor{currentstroke}{rgb}{0.151918,0.500685,0.557587}%
\pgfsetstrokecolor{currentstroke}%
\pgfsetdash{}{0pt}%
\pgfpathmoveto{\pgfqpoint{0.766095in}{4.684533in}}%
\pgfpathlineto{\pgfqpoint{0.785857in}{4.694341in}}%
\pgfpathlineto{\pgfqpoint{0.801138in}{4.701801in}}%
\pgfpathlineto{\pgfqpoint{0.836180in}{4.718835in}}%
\pgfpathlineto{\pgfqpoint{0.839835in}{4.720600in}}%
\pgfpathlineto{\pgfqpoint{0.871223in}{4.735514in}}%
\pgfpathlineto{\pgfqpoint{0.895184in}{4.746860in}}%
\pgfpathlineto{\pgfqpoint{0.906265in}{4.752021in}}%
\pgfpathlineto{\pgfqpoint{0.941308in}{4.768260in}}%
\pgfpathlineto{\pgfqpoint{0.951847in}{4.773119in}}%
\pgfpathlineto{\pgfqpoint{0.976350in}{4.784233in}}%
\pgfpathlineto{\pgfqpoint{1.009848in}{4.799378in}}%
\pgfpathlineto{\pgfqpoint{1.011393in}{4.800066in}}%
\pgfpathlineto{\pgfqpoint{1.046435in}{4.815550in}}%
\pgfpathlineto{\pgfqpoint{1.069334in}{4.825638in}}%
\pgfpathlineto{\pgfqpoint{1.081478in}{4.830900in}}%
\pgfpathlineto{\pgfqpoint{1.116520in}{4.846014in}}%
\pgfpathlineto{\pgfqpoint{1.130225in}{4.851897in}}%
\pgfpathlineto{\pgfqpoint{1.151563in}{4.860909in}}%
\pgfpathlineto{\pgfqpoint{1.186605in}{4.875661in}}%
\pgfpathlineto{\pgfqpoint{1.192569in}{4.878157in}}%
\pgfpathlineto{\pgfqpoint{1.221648in}{4.890129in}}%
\pgfpathlineto{\pgfqpoint{1.256419in}{4.904416in}}%
\pgfpathlineto{\pgfqpoint{1.256691in}{4.904526in}}%
\pgfpathlineto{\pgfqpoint{1.291733in}{4.918596in}}%
\pgfpathlineto{\pgfqpoint{1.321873in}{4.930676in}}%
\pgfpathlineto{\pgfqpoint{1.326776in}{4.932609in}}%
\pgfpathlineto{\pgfqpoint{1.361818in}{4.946343in}}%
\pgfpathlineto{\pgfqpoint{1.388905in}{4.956935in}}%
\pgfpathlineto{\pgfqpoint{1.396861in}{4.959996in}}%
\pgfpathlineto{\pgfqpoint{1.431903in}{4.973403in}}%
\pgfpathlineto{\pgfqpoint{1.457558in}{4.983195in}}%
\pgfpathlineto{\pgfqpoint{1.466946in}{4.986719in}}%
\pgfpathlineto{\pgfqpoint{1.501988in}{4.999808in}}%
\pgfpathlineto{\pgfqpoint{1.527875in}{5.009454in}}%
\pgfpathlineto{\pgfqpoint{1.537031in}{5.012810in}}%
\pgfpathlineto{\pgfqpoint{1.572073in}{5.025588in}}%
\pgfpathlineto{\pgfqpoint{1.599898in}{5.035714in}}%
\pgfpathlineto{\pgfqpoint{1.607116in}{5.038297in}}%
\pgfpathlineto{\pgfqpoint{1.642158in}{5.050772in}}%
\pgfpathlineto{\pgfqpoint{1.673667in}{5.061973in}}%
\pgfpathlineto{\pgfqpoint{1.677201in}{5.063209in}}%
\pgfpathlineto{\pgfqpoint{1.712244in}{5.075388in}}%
\pgfpathlineto{\pgfqpoint{1.747286in}{5.087557in}}%
\pgfpathlineto{\pgfqpoint{1.749243in}{5.088233in}}%
\pgfpathlineto{\pgfqpoint{1.782329in}{5.099463in}}%
\pgfpathlineto{\pgfqpoint{1.817371in}{5.111344in}}%
\pgfpathlineto{\pgfqpoint{1.826703in}{5.114492in}}%
\pgfpathlineto{\pgfqpoint{1.852414in}{5.123022in}}%
\pgfpathlineto{\pgfqpoint{1.887456in}{5.134622in}}%
\pgfpathlineto{\pgfqpoint{1.906037in}{5.140752in}}%
\pgfpathlineto{\pgfqpoint{1.922499in}{5.146092in}}%
\pgfpathlineto{\pgfqpoint{1.957541in}{5.157417in}}%
\pgfpathlineto{\pgfqpoint{1.987276in}{5.167011in}}%
\pgfpathlineto{\pgfqpoint{1.992584in}{5.168695in}}%
\pgfpathlineto{\pgfqpoint{2.027626in}{5.179752in}}%
\pgfpathlineto{\pgfqpoint{2.062669in}{5.190800in}}%
\pgfpathlineto{\pgfqpoint{2.070544in}{5.193271in}}%
\pgfpathlineto{\pgfqpoint{2.097711in}{5.201650in}}%
\pgfpathlineto{\pgfqpoint{2.132754in}{5.212436in}}%
\pgfpathlineto{\pgfqpoint{2.155859in}{5.219530in}}%
\pgfpathlineto{\pgfqpoint{2.167797in}{5.223134in}}%
\pgfpathlineto{\pgfqpoint{2.202839in}{5.233664in}}%
\pgfpathlineto{\pgfqpoint{2.237882in}{5.244188in}}%
\pgfpathlineto{\pgfqpoint{2.243244in}{5.245790in}}%
\pgfpathlineto{\pgfqpoint{2.272924in}{5.254505in}}%
\pgfpathlineto{\pgfqpoint{2.307967in}{5.264778in}}%
\pgfpathlineto{\pgfqpoint{2.332823in}{5.272049in}}%
\pgfpathlineto{\pgfqpoint{2.343009in}{5.274978in}}%
\pgfpathlineto{\pgfqpoint{2.378052in}{5.285007in}}%
\pgfpathlineto{\pgfqpoint{2.413094in}{5.295030in}}%
\pgfpathlineto{\pgfqpoint{2.424607in}{5.298308in}}%
\pgfpathlineto{\pgfqpoint{2.448137in}{5.304895in}}%
\pgfpathlineto{\pgfqpoint{2.483179in}{5.314679in}}%
\pgfpathlineto{\pgfqpoint{2.518222in}{5.324457in}}%
\pgfpathlineto{\pgfqpoint{2.518622in}{5.324568in}}%
\pgfpathlineto{\pgfqpoint{2.553264in}{5.334010in}}%
\pgfpathlineto{\pgfqpoint{2.588307in}{5.343554in}}%
\pgfpathlineto{\pgfqpoint{2.615062in}{5.350827in}}%
\pgfpathlineto{\pgfqpoint{2.623350in}{5.353042in}}%
\pgfpathlineto{\pgfqpoint{2.658392in}{5.362357in}}%
\pgfpathlineto{\pgfqpoint{2.693435in}{5.371668in}}%
\pgfpathlineto{\pgfqpoint{2.713893in}{5.377087in}}%
\pgfpathlineto{\pgfqpoint{2.728477in}{5.380884in}}%
\pgfpathlineto{\pgfqpoint{2.763520in}{5.389970in}}%
\pgfpathlineto{\pgfqpoint{2.798562in}{5.399051in}}%
\pgfpathlineto{\pgfqpoint{2.815196in}{5.403346in}}%
\pgfpathlineto{\pgfqpoint{2.833605in}{5.408018in}}%
\pgfpathlineto{\pgfqpoint{2.868647in}{5.416880in}}%
\pgfpathlineto{\pgfqpoint{2.903690in}{5.425737in}}%
\pgfpathlineto{\pgfqpoint{2.919057in}{5.429606in}}%
\pgfpathlineto{\pgfqpoint{2.938732in}{5.434474in}}%
\pgfpathlineto{\pgfqpoint{2.973775in}{5.443116in}}%
\pgfpathlineto{\pgfqpoint{3.008818in}{5.451753in}}%
\pgfpathlineto{\pgfqpoint{3.025560in}{5.455865in}}%
\pgfpathlineto{\pgfqpoint{3.043860in}{5.460281in}}%
\pgfpathlineto{\pgfqpoint{3.078903in}{5.468708in}}%
\pgfpathlineto{\pgfqpoint{3.113945in}{5.477131in}}%
\pgfpathlineto{\pgfqpoint{3.134786in}{5.482125in}}%
\pgfpathlineto{\pgfqpoint{3.148988in}{5.485468in}}%
\pgfpathlineto{\pgfqpoint{3.184030in}{5.493684in}}%
\pgfpathlineto{\pgfqpoint{3.219073in}{5.501895in}}%
\pgfpathlineto{\pgfqpoint{3.246814in}{5.508384in}}%
\pgfpathlineto{\pgfqpoint{3.254115in}{5.510062in}}%
\pgfpathlineto{\pgfqpoint{3.289158in}{5.518071in}}%
\pgfpathlineto{\pgfqpoint{3.312496in}{5.523402in}}%
\pgfusepath{stroke}%
\end{pgfscope}%
\begin{pgfscope}%
\pgfpathrectangle{\pgfqpoint{0.766095in}{0.571603in}}{\pgfqpoint{6.973465in}{5.225635in}}%
\pgfusepath{clip}%
\pgfsetbuttcap%
\pgfsetroundjoin%
\pgfsetlinewidth{1.505625pt}%
\definecolor{currentstroke}{rgb}{0.151918,0.500685,0.557587}%
\pgfsetstrokecolor{currentstroke}%
\pgfsetdash{}{0pt}%
\pgfpathmoveto{\pgfqpoint{3.616486in}{5.590472in}}%
\pgfpathlineto{\pgfqpoint{3.639583in}{5.595352in}}%
\pgfpathlineto{\pgfqpoint{3.674626in}{5.602751in}}%
\pgfpathlineto{\pgfqpoint{3.709668in}{5.610146in}}%
\pgfpathlineto{\pgfqpoint{3.725255in}{5.613422in}}%
\pgfpathlineto{\pgfqpoint{3.744711in}{5.617437in}}%
\pgfpathlineto{\pgfqpoint{3.779753in}{5.624643in}}%
\pgfpathlineto{\pgfqpoint{3.814796in}{5.631844in}}%
\pgfpathlineto{\pgfqpoint{3.849838in}{5.639040in}}%
\pgfpathlineto{\pgfqpoint{3.852982in}{5.639682in}}%
\pgfpathlineto{\pgfqpoint{3.884881in}{5.646072in}}%
\pgfpathlineto{\pgfqpoint{3.919924in}{5.653082in}}%
\pgfpathlineto{\pgfqpoint{3.954966in}{5.660087in}}%
\pgfpathlineto{\pgfqpoint{3.984303in}{5.665941in}}%
\pgfpathlineto{\pgfqpoint{3.990009in}{5.667059in}}%
\pgfpathlineto{\pgfqpoint{4.025051in}{5.673881in}}%
\pgfpathlineto{\pgfqpoint{4.060094in}{5.680698in}}%
\pgfpathlineto{\pgfqpoint{4.095136in}{5.687509in}}%
\pgfpathlineto{\pgfqpoint{4.119340in}{5.692201in}}%
\pgfpathlineto{\pgfqpoint{4.130179in}{5.694262in}}%
\pgfpathlineto{\pgfqpoint{4.165221in}{5.700894in}}%
\pgfpathlineto{\pgfqpoint{4.200264in}{5.707519in}}%
\pgfpathlineto{\pgfqpoint{4.235306in}{5.714138in}}%
\pgfpathlineto{\pgfqpoint{4.258259in}{5.718460in}}%
\pgfpathlineto{\pgfqpoint{4.270349in}{5.720694in}}%
\pgfpathlineto{\pgfqpoint{4.305391in}{5.727136in}}%
\pgfpathlineto{\pgfqpoint{4.340434in}{5.733571in}}%
\pgfpathlineto{\pgfqpoint{4.375477in}{5.740000in}}%
\pgfpathlineto{\pgfqpoint{4.401273in}{5.744720in}}%
\pgfpathlineto{\pgfqpoint{4.410519in}{5.746379in}}%
\pgfpathlineto{\pgfqpoint{4.445562in}{5.752633in}}%
\pgfpathlineto{\pgfqpoint{4.480604in}{5.758880in}}%
\pgfpathlineto{\pgfqpoint{4.515647in}{5.765119in}}%
\pgfpathlineto{\pgfqpoint{4.548605in}{5.770979in}}%
\pgfpathlineto{\pgfqpoint{4.550689in}{5.771343in}}%
\pgfpathlineto{\pgfqpoint{4.585732in}{5.777409in}}%
\pgfpathlineto{\pgfqpoint{4.620774in}{5.783468in}}%
\pgfpathlineto{\pgfqpoint{4.655817in}{5.789520in}}%
\pgfpathlineto{\pgfqpoint{4.690859in}{5.795564in}}%
\pgfpathlineto{\pgfqpoint{4.700626in}{5.797238in}}%
\pgfusepath{stroke}%
\end{pgfscope}%
\begin{pgfscope}%
\pgfpathrectangle{\pgfqpoint{0.766095in}{0.571603in}}{\pgfqpoint{6.973465in}{5.225635in}}%
\pgfusepath{clip}%
\pgfsetbuttcap%
\pgfsetroundjoin%
\pgfsetlinewidth{1.505625pt}%
\definecolor{currentstroke}{rgb}{0.144759,0.519093,0.556572}%
\pgfsetstrokecolor{currentstroke}%
\pgfsetdash{}{0pt}%
\pgfpathmoveto{\pgfqpoint{0.766095in}{4.744940in}}%
\pgfpathlineto{\pgfqpoint{0.770067in}{4.746860in}}%
\pgfpathlineto{\pgfqpoint{0.801138in}{4.761633in}}%
\pgfpathlineto{\pgfqpoint{0.825381in}{4.773119in}}%
\pgfpathlineto{\pgfqpoint{0.836180in}{4.778153in}}%
\pgfpathlineto{\pgfqpoint{0.871223in}{4.794404in}}%
\pgfpathlineto{\pgfqpoint{0.882002in}{4.799378in}}%
\pgfpathlineto{\pgfqpoint{0.906265in}{4.810393in}}%
\pgfpathlineto{\pgfqpoint{0.939950in}{4.825638in}}%
\pgfpathlineto{\pgfqpoint{0.941308in}{4.826243in}}%
\pgfpathlineto{\pgfqpoint{0.976350in}{4.841744in}}%
\pgfpathlineto{\pgfqpoint{0.999372in}{4.851897in}}%
\pgfpathlineto{\pgfqpoint{1.011393in}{4.857113in}}%
\pgfpathlineto{\pgfqpoint{1.046435in}{4.872245in}}%
\pgfpathlineto{\pgfqpoint{1.060187in}{4.878157in}}%
\pgfpathlineto{\pgfqpoint{1.081478in}{4.887162in}}%
\pgfpathlineto{\pgfqpoint{1.116520in}{4.901934in}}%
\pgfpathlineto{\pgfqpoint{1.122442in}{4.904416in}}%
\pgfpathlineto{\pgfqpoint{1.151563in}{4.916425in}}%
\pgfpathlineto{\pgfqpoint{1.186190in}{4.930676in}}%
\pgfpathlineto{\pgfqpoint{1.186605in}{4.930844in}}%
\pgfpathlineto{\pgfqpoint{1.221648in}{4.944939in}}%
\pgfpathlineto{\pgfqpoint{1.251526in}{4.956935in}}%
\pgfpathlineto{\pgfqpoint{1.256691in}{4.958975in}}%
\pgfpathlineto{\pgfqpoint{1.291733in}{4.972737in}}%
\pgfpathlineto{\pgfqpoint{1.318422in}{4.983195in}}%
\pgfpathlineto{\pgfqpoint{1.326776in}{4.986415in}}%
\pgfpathlineto{\pgfqpoint{1.361818in}{4.999852in}}%
\pgfpathlineto{\pgfqpoint{1.386923in}{5.009454in}}%
\pgfpathlineto{\pgfqpoint{1.396861in}{5.013194in}}%
\pgfpathlineto{\pgfqpoint{1.431903in}{5.026314in}}%
\pgfpathlineto{\pgfqpoint{1.457069in}{5.035714in}}%
\pgfpathlineto{\pgfqpoint{1.466946in}{5.039342in}}%
\pgfpathlineto{\pgfqpoint{1.501988in}{5.052154in}}%
\pgfpathlineto{\pgfqpoint{1.528902in}{5.061973in}}%
\pgfpathlineto{\pgfqpoint{1.537031in}{5.064890in}}%
\pgfpathlineto{\pgfqpoint{1.572073in}{5.077401in}}%
\pgfpathlineto{\pgfqpoint{1.602459in}{5.088233in}}%
\pgfpathlineto{\pgfqpoint{1.607116in}{5.089866in}}%
\pgfpathlineto{\pgfqpoint{1.642158in}{5.102083in}}%
\pgfpathlineto{\pgfqpoint{1.677201in}{5.114291in}}%
\pgfpathlineto{\pgfqpoint{1.677783in}{5.114492in}}%
\pgfpathlineto{\pgfqpoint{1.712244in}{5.126226in}}%
\pgfpathlineto{\pgfqpoint{1.747286in}{5.138147in}}%
\pgfpathlineto{\pgfqpoint{1.754981in}{5.140752in}}%
\pgfpathlineto{\pgfqpoint{1.782329in}{5.149855in}}%
\pgfpathlineto{\pgfqpoint{1.817371in}{5.161498in}}%
\pgfpathlineto{\pgfqpoint{1.834027in}{5.167011in}}%
\pgfpathlineto{\pgfqpoint{1.852414in}{5.172997in}}%
\pgfpathlineto{\pgfqpoint{1.887456in}{5.184367in}}%
\pgfpathlineto{\pgfqpoint{1.914951in}{5.193271in}}%
\pgfpathlineto{\pgfqpoint{1.922499in}{5.195674in}}%
\pgfpathlineto{\pgfqpoint{1.957541in}{5.206778in}}%
\pgfpathlineto{\pgfqpoint{1.992584in}{5.217874in}}%
\pgfpathlineto{\pgfqpoint{1.997843in}{5.219530in}}%
\pgfpathlineto{\pgfqpoint{2.027626in}{5.228754in}}%
\pgfpathlineto{\pgfqpoint{2.062669in}{5.239590in}}%
\pgfpathlineto{\pgfqpoint{2.082778in}{5.245790in}}%
\pgfpathlineto{\pgfqpoint{2.097711in}{5.250317in}}%
\pgfpathlineto{\pgfqpoint{2.132754in}{5.260899in}}%
\pgfpathlineto{\pgfqpoint{2.167797in}{5.271475in}}%
\pgfpathlineto{\pgfqpoint{2.169709in}{5.272049in}}%
\pgfpathlineto{\pgfqpoint{2.202839in}{5.281822in}}%
\pgfpathlineto{\pgfqpoint{2.237882in}{5.292150in}}%
\pgfpathlineto{\pgfqpoint{2.258835in}{5.298308in}}%
\pgfpathlineto{\pgfqpoint{2.272924in}{5.302380in}}%
\pgfpathlineto{\pgfqpoint{2.307967in}{5.312466in}}%
\pgfpathlineto{\pgfqpoint{2.343009in}{5.322546in}}%
\pgfpathlineto{\pgfqpoint{2.350074in}{5.324568in}}%
\pgfpathlineto{\pgfqpoint{2.378052in}{5.332440in}}%
\pgfpathlineto{\pgfqpoint{2.413094in}{5.342284in}}%
\pgfpathlineto{\pgfqpoint{2.443548in}{5.350827in}}%
\pgfpathlineto{\pgfqpoint{2.448137in}{5.352093in}}%
\pgfpathlineto{\pgfqpoint{2.483179in}{5.361705in}}%
\pgfpathlineto{\pgfqpoint{2.518222in}{5.371312in}}%
\pgfpathlineto{\pgfqpoint{2.539346in}{5.377087in}}%
\pgfpathlineto{\pgfqpoint{2.553264in}{5.380827in}}%
\pgfpathlineto{\pgfqpoint{2.588307in}{5.390208in}}%
\pgfpathlineto{\pgfqpoint{2.623350in}{5.399583in}}%
\pgfpathlineto{\pgfqpoint{2.637469in}{5.403346in}}%
\pgfpathlineto{\pgfqpoint{2.658392in}{5.408828in}}%
\pgfpathlineto{\pgfqpoint{2.693435in}{5.417982in}}%
\pgfpathlineto{\pgfqpoint{2.728477in}{5.427131in}}%
\pgfpathlineto{\pgfqpoint{2.737998in}{5.429606in}}%
\pgfpathlineto{\pgfqpoint{2.763520in}{5.436125in}}%
\pgfpathlineto{\pgfqpoint{2.798562in}{5.445058in}}%
\pgfpathlineto{\pgfqpoint{2.833605in}{5.453987in}}%
\pgfpathlineto{\pgfqpoint{2.841015in}{5.455865in}}%
\pgfpathlineto{\pgfqpoint{2.857672in}{5.460016in}}%
\pgfusepath{stroke}%
\end{pgfscope}%
\begin{pgfscope}%
\pgfpathrectangle{\pgfqpoint{0.766095in}{0.571603in}}{\pgfqpoint{6.973465in}{5.225635in}}%
\pgfusepath{clip}%
\pgfsetbuttcap%
\pgfsetroundjoin%
\pgfsetlinewidth{1.505625pt}%
\definecolor{currentstroke}{rgb}{0.144759,0.519093,0.556572}%
\pgfsetstrokecolor{currentstroke}%
\pgfsetdash{}{0pt}%
\pgfpathmoveto{\pgfqpoint{3.160231in}{5.533340in}}%
\pgfpathlineto{\pgfqpoint{3.165759in}{5.534644in}}%
\pgfpathlineto{\pgfqpoint{3.184030in}{5.538877in}}%
\pgfpathlineto{\pgfqpoint{3.219073in}{5.546970in}}%
\pgfpathlineto{\pgfqpoint{3.254115in}{5.555060in}}%
\pgfpathlineto{\pgfqpoint{3.279480in}{5.560903in}}%
\pgfpathlineto{\pgfqpoint{3.289158in}{5.563093in}}%
\pgfpathlineto{\pgfqpoint{3.324200in}{5.570987in}}%
\pgfpathlineto{\pgfqpoint{3.359243in}{5.578876in}}%
\pgfpathlineto{\pgfqpoint{3.394285in}{5.586763in}}%
\pgfpathlineto{\pgfqpoint{3.396074in}{5.587163in}}%
\pgfpathlineto{\pgfqpoint{3.429328in}{5.594469in}}%
\pgfpathlineto{\pgfqpoint{3.464371in}{5.602163in}}%
\pgfpathlineto{\pgfqpoint{3.499413in}{5.609853in}}%
\pgfpathlineto{\pgfqpoint{3.515737in}{5.613422in}}%
\pgfpathlineto{\pgfqpoint{3.534456in}{5.617442in}}%
\pgfpathlineto{\pgfqpoint{3.569498in}{5.624943in}}%
\pgfpathlineto{\pgfqpoint{3.604541in}{5.632440in}}%
\pgfpathlineto{\pgfqpoint{3.638411in}{5.639682in}}%
\pgfpathlineto{\pgfqpoint{3.639583in}{5.639928in}}%
\pgfpathlineto{\pgfqpoint{3.674626in}{5.647240in}}%
\pgfpathlineto{\pgfqpoint{3.709668in}{5.654547in}}%
\pgfpathlineto{\pgfqpoint{3.744711in}{5.661851in}}%
\pgfpathlineto{\pgfqpoint{3.764395in}{5.665941in}}%
\pgfpathlineto{\pgfqpoint{3.779753in}{5.669075in}}%
\pgfpathlineto{\pgfqpoint{3.814796in}{5.676196in}}%
\pgfpathlineto{\pgfqpoint{3.849838in}{5.683313in}}%
\pgfpathlineto{\pgfqpoint{3.884881in}{5.690427in}}%
\pgfpathlineto{\pgfqpoint{3.893663in}{5.692201in}}%
\pgfpathlineto{\pgfqpoint{3.919924in}{5.697408in}}%
\pgfpathlineto{\pgfqpoint{3.954966in}{5.704341in}}%
\pgfpathlineto{\pgfqpoint{3.990009in}{5.711271in}}%
\pgfpathlineto{\pgfqpoint{4.025051in}{5.718196in}}%
\pgfpathlineto{\pgfqpoint{4.026394in}{5.718460in}}%
\pgfpathlineto{\pgfqpoint{4.060094in}{5.724956in}}%
\pgfpathlineto{\pgfqpoint{4.095136in}{5.731704in}}%
\pgfpathlineto{\pgfqpoint{4.130179in}{5.738448in}}%
\pgfpathlineto{\pgfqpoint{4.162801in}{5.744720in}}%
\pgfpathlineto{\pgfqpoint{4.165221in}{5.745176in}}%
\pgfpathlineto{\pgfqpoint{4.200264in}{5.751746in}}%
\pgfpathlineto{\pgfqpoint{4.235306in}{5.758311in}}%
\pgfpathlineto{\pgfqpoint{4.270349in}{5.764871in}}%
\pgfpathlineto{\pgfqpoint{4.303011in}{5.770979in}}%
\pgfpathlineto{\pgfqpoint{4.305391in}{5.771416in}}%
\pgfpathlineto{\pgfqpoint{4.340434in}{5.777805in}}%
\pgfpathlineto{\pgfqpoint{4.375477in}{5.784188in}}%
\pgfpathlineto{\pgfqpoint{4.410519in}{5.790566in}}%
\pgfpathlineto{\pgfqpoint{4.445562in}{5.796939in}}%
\pgfpathlineto{\pgfqpoint{4.447217in}{5.797238in}}%
\pgfusepath{stroke}%
\end{pgfscope}%
\begin{pgfscope}%
\pgfpathrectangle{\pgfqpoint{0.766095in}{0.571603in}}{\pgfqpoint{6.973465in}{5.225635in}}%
\pgfusepath{clip}%
\pgfsetbuttcap%
\pgfsetroundjoin%
\pgfsetlinewidth{1.505625pt}%
\definecolor{currentstroke}{rgb}{0.137770,0.537492,0.554906}%
\pgfsetstrokecolor{currentstroke}%
\pgfsetdash{}{0pt}%
\pgfpathmoveto{\pgfqpoint{0.766095in}{4.802388in}}%
\pgfpathlineto{\pgfqpoint{0.801138in}{4.818638in}}%
\pgfpathlineto{\pgfqpoint{0.816297in}{4.825638in}}%
\pgfpathlineto{\pgfqpoint{0.836180in}{4.834671in}}%
\pgfpathlineto{\pgfqpoint{0.871223in}{4.850535in}}%
\pgfpathlineto{\pgfqpoint{0.874251in}{4.851897in}}%
\pgfpathlineto{\pgfqpoint{0.906265in}{4.866069in}}%
\pgfpathlineto{\pgfqpoint{0.933648in}{4.878157in}}%
\pgfpathlineto{\pgfqpoint{0.941308in}{4.881484in}}%
\pgfpathlineto{\pgfqpoint{0.976350in}{4.896621in}}%
\pgfpathlineto{\pgfqpoint{0.994463in}{4.904416in}}%
\pgfpathlineto{\pgfqpoint{1.011393in}{4.911585in}}%
\pgfpathlineto{\pgfqpoint{1.046435in}{4.926365in}}%
\pgfpathlineto{\pgfqpoint{1.056705in}{4.930676in}}%
\pgfpathlineto{\pgfqpoint{1.081478in}{4.940905in}}%
\pgfpathlineto{\pgfqpoint{1.116520in}{4.955338in}}%
\pgfpathlineto{\pgfqpoint{1.120423in}{4.956935in}}%
\pgfpathlineto{\pgfqpoint{1.151563in}{4.969479in}}%
\pgfpathlineto{\pgfqpoint{1.185671in}{4.983195in}}%
\pgfpathlineto{\pgfqpoint{1.186605in}{4.983564in}}%
\pgfpathlineto{\pgfqpoint{1.221648in}{4.997340in}}%
\pgfpathlineto{\pgfqpoint{1.252512in}{5.009454in}}%
\pgfpathlineto{\pgfqpoint{1.256691in}{5.011068in}}%
\pgfpathlineto{\pgfqpoint{1.291733in}{5.024521in}}%
\pgfpathlineto{\pgfqpoint{1.320940in}{5.035714in}}%
\pgfpathlineto{\pgfqpoint{1.326776in}{5.037914in}}%
\pgfpathlineto{\pgfqpoint{1.361818in}{5.051052in}}%
\pgfpathlineto{\pgfqpoint{1.390997in}{5.061973in}}%
\pgfpathlineto{\pgfqpoint{1.396861in}{5.064132in}}%
\pgfpathlineto{\pgfqpoint{1.431903in}{5.076964in}}%
\pgfpathlineto{\pgfqpoint{1.462721in}{5.088233in}}%
\pgfpathlineto{\pgfqpoint{1.466946in}{5.089752in}}%
\pgfpathlineto{\pgfqpoint{1.501988in}{5.102285in}}%
\pgfpathlineto{\pgfqpoint{1.536150in}{5.114492in}}%
\pgfpathlineto{\pgfqpoint{1.537031in}{5.114802in}}%
\pgfpathlineto{\pgfqpoint{1.572073in}{5.127043in}}%
\pgfpathlineto{\pgfqpoint{1.607116in}{5.139275in}}%
\pgfpathlineto{\pgfqpoint{1.611369in}{5.140752in}}%
\pgfpathlineto{\pgfqpoint{1.642158in}{5.151265in}}%
\pgfpathlineto{\pgfqpoint{1.677201in}{5.163213in}}%
\pgfpathlineto{\pgfqpoint{1.688392in}{5.167011in}}%
\pgfpathlineto{\pgfqpoint{1.712244in}{5.174975in}}%
\pgfpathlineto{\pgfqpoint{1.747286in}{5.186646in}}%
\pgfpathlineto{\pgfqpoint{1.767239in}{5.193271in}}%
\pgfpathlineto{\pgfqpoint{1.782329in}{5.198199in}}%
\pgfpathlineto{\pgfqpoint{1.817371in}{5.209599in}}%
\pgfpathlineto{\pgfqpoint{1.847939in}{5.219530in}}%
\pgfpathlineto{\pgfqpoint{1.852414in}{5.220960in}}%
\pgfpathlineto{\pgfqpoint{1.887456in}{5.232096in}}%
\pgfpathlineto{\pgfqpoint{1.922499in}{5.243225in}}%
\pgfpathlineto{\pgfqpoint{1.930613in}{5.245790in}}%
\pgfpathlineto{\pgfqpoint{1.957541in}{5.254159in}}%
\pgfpathlineto{\pgfqpoint{1.992584in}{5.265031in}}%
\pgfpathlineto{\pgfqpoint{2.015264in}{5.272049in}}%
\pgfpathlineto{\pgfqpoint{2.027626in}{5.275811in}}%
\pgfpathlineto{\pgfqpoint{2.062669in}{5.286431in}}%
\pgfpathlineto{\pgfqpoint{2.097711in}{5.297044in}}%
\pgfpathlineto{\pgfqpoint{2.101908in}{5.298308in}}%
\pgfpathlineto{\pgfqpoint{2.132754in}{5.307446in}}%
\pgfpathlineto{\pgfqpoint{2.167797in}{5.317814in}}%
\pgfpathlineto{\pgfqpoint{2.190681in}{5.324568in}}%
\pgfpathlineto{\pgfqpoint{2.202839in}{5.328097in}}%
\pgfpathlineto{\pgfqpoint{2.237882in}{5.338224in}}%
\pgfpathlineto{\pgfqpoint{2.272924in}{5.348347in}}%
\pgfpathlineto{\pgfqpoint{2.281552in}{5.350827in}}%
\pgfpathlineto{\pgfqpoint{2.307967in}{5.358295in}}%
\pgfpathlineto{\pgfqpoint{2.343009in}{5.368183in}}%
\pgfpathlineto{\pgfqpoint{2.374598in}{5.377087in}}%
\pgfpathlineto{\pgfqpoint{2.378052in}{5.378044in}}%
\pgfpathlineto{\pgfqpoint{2.413094in}{5.387703in}}%
\pgfpathlineto{\pgfqpoint{2.438001in}{5.394564in}}%
\pgfusepath{stroke}%
\end{pgfscope}%
\begin{pgfscope}%
\pgfpathrectangle{\pgfqpoint{0.766095in}{0.571603in}}{\pgfqpoint{6.973465in}{5.225635in}}%
\pgfusepath{clip}%
\pgfsetbuttcap%
\pgfsetroundjoin%
\pgfsetlinewidth{1.505625pt}%
\definecolor{currentstroke}{rgb}{0.137770,0.537492,0.554906}%
\pgfsetstrokecolor{currentstroke}%
\pgfsetdash{}{0pt}%
\pgfpathmoveto{\pgfqpoint{2.738974in}{5.474209in}}%
\pgfpathlineto{\pgfqpoint{2.763520in}{5.480502in}}%
\pgfpathlineto{\pgfqpoint{2.769881in}{5.482125in}}%
\pgfpathlineto{\pgfqpoint{2.798562in}{5.489317in}}%
\pgfpathlineto{\pgfqpoint{2.833605in}{5.498091in}}%
\pgfpathlineto{\pgfqpoint{2.868647in}{5.506862in}}%
\pgfpathlineto{\pgfqpoint{2.874761in}{5.508384in}}%
\pgfpathlineto{\pgfqpoint{2.903690in}{5.515465in}}%
\pgfpathlineto{\pgfqpoint{2.938732in}{5.524030in}}%
\pgfpathlineto{\pgfqpoint{2.973775in}{5.532592in}}%
\pgfpathlineto{\pgfqpoint{2.982210in}{5.534644in}}%
\pgfpathlineto{\pgfqpoint{3.008818in}{5.541003in}}%
\pgfpathlineto{\pgfqpoint{3.043860in}{5.549364in}}%
\pgfpathlineto{\pgfqpoint{3.078903in}{5.557721in}}%
\pgfpathlineto{\pgfqpoint{3.092295in}{5.560903in}}%
\pgfpathlineto{\pgfqpoint{3.113945in}{5.565958in}}%
\pgfpathlineto{\pgfqpoint{3.148988in}{5.574118in}}%
\pgfpathlineto{\pgfqpoint{3.184030in}{5.582275in}}%
\pgfpathlineto{\pgfqpoint{3.205082in}{5.587163in}}%
\pgfpathlineto{\pgfqpoint{3.219073in}{5.590354in}}%
\pgfpathlineto{\pgfqpoint{3.254115in}{5.598318in}}%
\pgfpathlineto{\pgfqpoint{3.289158in}{5.606279in}}%
\pgfpathlineto{\pgfqpoint{3.320630in}{5.613422in}}%
\pgfpathlineto{\pgfqpoint{3.324200in}{5.614218in}}%
\pgfpathlineto{\pgfqpoint{3.359243in}{5.621989in}}%
\pgfpathlineto{\pgfqpoint{3.394285in}{5.629757in}}%
\pgfpathlineto{\pgfqpoint{3.429328in}{5.637523in}}%
\pgfpathlineto{\pgfqpoint{3.439113in}{5.639682in}}%
\pgfpathlineto{\pgfqpoint{3.464371in}{5.645155in}}%
\pgfpathlineto{\pgfqpoint{3.499413in}{5.652734in}}%
\pgfpathlineto{\pgfqpoint{3.534456in}{5.660310in}}%
\pgfpathlineto{\pgfqpoint{3.560547in}{5.665941in}}%
\pgfpathlineto{\pgfqpoint{3.569498in}{5.667838in}}%
\pgfpathlineto{\pgfqpoint{3.604541in}{5.675232in}}%
\pgfpathlineto{\pgfqpoint{3.639583in}{5.682622in}}%
\pgfpathlineto{\pgfqpoint{3.674626in}{5.690010in}}%
\pgfpathlineto{\pgfqpoint{3.685065in}{5.692201in}}%
\pgfpathlineto{\pgfqpoint{3.709668in}{5.697272in}}%
\pgfpathlineto{\pgfqpoint{3.744711in}{5.704480in}}%
\pgfpathlineto{\pgfqpoint{3.779753in}{5.711684in}}%
\pgfpathlineto{\pgfqpoint{3.812732in}{5.718460in}}%
\pgfpathlineto{\pgfqpoint{3.814796in}{5.718876in}}%
\pgfpathlineto{\pgfqpoint{3.849838in}{5.725904in}}%
\pgfpathlineto{\pgfqpoint{3.884881in}{5.732929in}}%
\pgfpathlineto{\pgfqpoint{3.919924in}{5.739951in}}%
\pgfpathlineto{\pgfqpoint{3.943774in}{5.744720in}}%
\pgfpathlineto{\pgfqpoint{3.954966in}{5.746917in}}%
\pgfpathlineto{\pgfqpoint{3.990009in}{5.753764in}}%
\pgfpathlineto{\pgfqpoint{4.025051in}{5.760609in}}%
\pgfpathlineto{\pgfqpoint{4.060094in}{5.767451in}}%
\pgfpathlineto{\pgfqpoint{4.078225in}{5.770979in}}%
\pgfpathlineto{\pgfqpoint{4.095136in}{5.774210in}}%
\pgfpathlineto{\pgfqpoint{4.130179in}{5.780880in}}%
\pgfpathlineto{\pgfqpoint{4.165221in}{5.787546in}}%
\pgfpathlineto{\pgfqpoint{4.200264in}{5.794210in}}%
\pgfpathlineto{\pgfqpoint{4.216250in}{5.797238in}}%
\pgfusepath{stroke}%
\end{pgfscope}%
\begin{pgfscope}%
\pgfpathrectangle{\pgfqpoint{0.766095in}{0.571603in}}{\pgfqpoint{6.973465in}{5.225635in}}%
\pgfusepath{clip}%
\pgfsetbuttcap%
\pgfsetroundjoin%
\pgfsetlinewidth{1.505625pt}%
\definecolor{currentstroke}{rgb}{0.131172,0.555899,0.552459}%
\pgfsetstrokecolor{currentstroke}%
\pgfsetdash{}{0pt}%
\pgfpathmoveto{\pgfqpoint{0.766095in}{4.857215in}}%
\pgfpathlineto{\pgfqpoint{0.801138in}{4.873068in}}%
\pgfpathlineto{\pgfqpoint{0.812441in}{4.878157in}}%
\pgfpathlineto{\pgfqpoint{0.836180in}{4.888673in}}%
\pgfpathlineto{\pgfqpoint{0.871223in}{4.904152in}}%
\pgfpathlineto{\pgfqpoint{0.871824in}{4.904416in}}%
\pgfpathlineto{\pgfqpoint{0.906265in}{4.919290in}}%
\pgfpathlineto{\pgfqpoint{0.932698in}{4.930676in}}%
\pgfpathlineto{\pgfqpoint{0.941308in}{4.934325in}}%
\pgfpathlineto{\pgfqpoint{0.976350in}{4.949101in}}%
\pgfpathlineto{\pgfqpoint{0.994995in}{4.956935in}}%
\pgfpathlineto{\pgfqpoint{1.011393in}{4.963714in}}%
\pgfpathlineto{\pgfqpoint{1.046435in}{4.978144in}}%
\pgfpathlineto{\pgfqpoint{1.058755in}{4.983195in}}%
\pgfpathlineto{\pgfqpoint{1.081478in}{4.992360in}}%
\pgfpathlineto{\pgfqpoint{1.116520in}{5.006453in}}%
\pgfpathlineto{\pgfqpoint{1.124022in}{5.009454in}}%
\pgfpathlineto{\pgfqpoint{1.151563in}{5.020296in}}%
\pgfpathlineto{\pgfqpoint{1.186605in}{5.034061in}}%
\pgfpathlineto{\pgfqpoint{1.190838in}{5.035714in}}%
\pgfpathlineto{\pgfqpoint{1.221648in}{5.047554in}}%
\pgfpathlineto{\pgfqpoint{1.256691in}{5.060999in}}%
\pgfpathlineto{\pgfqpoint{1.259245in}{5.061973in}}%
\pgfpathlineto{\pgfqpoint{1.291733in}{5.074165in}}%
\pgfpathlineto{\pgfqpoint{1.326776in}{5.087299in}}%
\pgfpathlineto{\pgfqpoint{1.329283in}{5.088233in}}%
\pgfpathlineto{\pgfqpoint{1.361818in}{5.100159in}}%
\pgfpathlineto{\pgfqpoint{1.396861in}{5.112989in}}%
\pgfpathlineto{\pgfqpoint{1.400990in}{5.114492in}}%
\pgfpathlineto{\pgfqpoint{1.431903in}{5.125565in}}%
\pgfpathlineto{\pgfqpoint{1.466946in}{5.138098in}}%
\pgfpathlineto{\pgfqpoint{1.474403in}{5.140752in}}%
\pgfpathlineto{\pgfqpoint{1.501988in}{5.150409in}}%
\pgfpathlineto{\pgfqpoint{1.537031in}{5.162653in}}%
\pgfpathlineto{\pgfqpoint{1.549555in}{5.167011in}}%
\pgfpathlineto{\pgfqpoint{1.572073in}{5.174719in}}%
\pgfpathlineto{\pgfqpoint{1.607116in}{5.186681in}}%
\pgfpathlineto{\pgfqpoint{1.626480in}{5.193271in}}%
\pgfpathlineto{\pgfqpoint{1.642158in}{5.198519in}}%
\pgfpathlineto{\pgfqpoint{1.677201in}{5.210206in}}%
\pgfpathlineto{\pgfqpoint{1.705206in}{5.219530in}}%
\pgfpathlineto{\pgfqpoint{1.712244in}{5.221835in}}%
\pgfpathlineto{\pgfqpoint{1.747286in}{5.233254in}}%
\pgfpathlineto{\pgfqpoint{1.782329in}{5.244665in}}%
\pgfpathlineto{\pgfqpoint{1.785802in}{5.245790in}}%
\pgfpathlineto{\pgfqpoint{1.817371in}{5.255846in}}%
\pgfpathlineto{\pgfqpoint{1.852414in}{5.266995in}}%
\pgfpathlineto{\pgfqpoint{1.868354in}{5.272049in}}%
\pgfpathlineto{\pgfqpoint{1.887456in}{5.278006in}}%
\pgfpathlineto{\pgfqpoint{1.922499in}{5.288900in}}%
\pgfpathlineto{\pgfqpoint{1.952804in}{5.298308in}}%
\pgfpathlineto{\pgfqpoint{1.957541in}{5.299755in}}%
\pgfpathlineto{\pgfqpoint{1.992584in}{5.310399in}}%
\pgfpathlineto{\pgfqpoint{2.018509in}{5.318270in}}%
\pgfusepath{stroke}%
\end{pgfscope}%
\begin{pgfscope}%
\pgfpathrectangle{\pgfqpoint{0.766095in}{0.571603in}}{\pgfqpoint{6.973465in}{5.225635in}}%
\pgfusepath{clip}%
\pgfsetbuttcap%
\pgfsetroundjoin%
\pgfsetlinewidth{1.505625pt}%
\definecolor{currentstroke}{rgb}{0.131172,0.555899,0.552459}%
\pgfsetstrokecolor{currentstroke}%
\pgfsetdash{}{0pt}%
\pgfpathmoveto{\pgfqpoint{2.317512in}{5.405090in}}%
\pgfpathlineto{\pgfqpoint{2.343009in}{5.412145in}}%
\pgfpathlineto{\pgfqpoint{2.378052in}{5.421833in}}%
\pgfpathlineto{\pgfqpoint{2.406210in}{5.429606in}}%
\pgfpathlineto{\pgfqpoint{2.413094in}{5.431475in}}%
\pgfpathlineto{\pgfqpoint{2.448137in}{5.440940in}}%
\pgfpathlineto{\pgfqpoint{2.483179in}{5.450402in}}%
\pgfpathlineto{\pgfqpoint{2.503468in}{5.455865in}}%
\pgfpathlineto{\pgfqpoint{2.518222in}{5.459772in}}%
\pgfpathlineto{\pgfqpoint{2.553264in}{5.469016in}}%
\pgfpathlineto{\pgfqpoint{2.588307in}{5.478258in}}%
\pgfpathlineto{\pgfqpoint{2.603023in}{5.482125in}}%
\pgfpathlineto{\pgfqpoint{2.623350in}{5.487376in}}%
\pgfpathlineto{\pgfqpoint{2.658392in}{5.496405in}}%
\pgfpathlineto{\pgfqpoint{2.693435in}{5.505431in}}%
\pgfpathlineto{\pgfqpoint{2.704948in}{5.508384in}}%
\pgfpathlineto{\pgfqpoint{2.728477in}{5.514318in}}%
\pgfpathlineto{\pgfqpoint{2.763520in}{5.523136in}}%
\pgfpathlineto{\pgfqpoint{2.798562in}{5.531951in}}%
\pgfpathlineto{\pgfqpoint{2.809311in}{5.534644in}}%
\pgfpathlineto{\pgfqpoint{2.833605in}{5.540626in}}%
\pgfpathlineto{\pgfqpoint{2.868647in}{5.549238in}}%
\pgfpathlineto{\pgfqpoint{2.903690in}{5.557847in}}%
\pgfpathlineto{\pgfqpoint{2.916178in}{5.560903in}}%
\pgfpathlineto{\pgfqpoint{2.938732in}{5.566329in}}%
\pgfpathlineto{\pgfqpoint{2.973775in}{5.574738in}}%
\pgfpathlineto{\pgfqpoint{3.008818in}{5.583146in}}%
\pgfpathlineto{\pgfqpoint{3.025612in}{5.587163in}}%
\pgfpathlineto{\pgfqpoint{3.043860in}{5.591452in}}%
\pgfpathlineto{\pgfqpoint{3.078903in}{5.599664in}}%
\pgfpathlineto{\pgfqpoint{3.113945in}{5.607874in}}%
\pgfpathlineto{\pgfqpoint{3.137673in}{5.613422in}}%
\pgfpathlineto{\pgfqpoint{3.148988in}{5.616022in}}%
\pgfpathlineto{\pgfqpoint{3.184030in}{5.624041in}}%
\pgfpathlineto{\pgfqpoint{3.219073in}{5.632057in}}%
\pgfpathlineto{\pgfqpoint{3.252417in}{5.639682in}}%
\pgfpathlineto{\pgfqpoint{3.254115in}{5.640063in}}%
\pgfpathlineto{\pgfqpoint{3.289158in}{5.647892in}}%
\pgfpathlineto{\pgfqpoint{3.324200in}{5.655719in}}%
\pgfpathlineto{\pgfqpoint{3.359243in}{5.663545in}}%
\pgfpathlineto{\pgfqpoint{3.370020in}{5.665941in}}%
\pgfpathlineto{\pgfqpoint{3.394285in}{5.671243in}}%
\pgfpathlineto{\pgfqpoint{3.429328in}{5.678884in}}%
\pgfpathlineto{\pgfqpoint{3.464371in}{5.686523in}}%
\pgfpathlineto{\pgfqpoint{3.490459in}{5.692201in}}%
\pgfpathlineto{\pgfqpoint{3.499413in}{5.694115in}}%
\pgfpathlineto{\pgfqpoint{3.534456in}{5.701573in}}%
\pgfpathlineto{\pgfqpoint{3.569498in}{5.709030in}}%
\pgfpathlineto{\pgfqpoint{3.604541in}{5.716484in}}%
\pgfpathlineto{\pgfqpoint{3.613869in}{5.718460in}}%
\pgfpathlineto{\pgfqpoint{3.639583in}{5.723810in}}%
\pgfpathlineto{\pgfqpoint{3.674626in}{5.731087in}}%
\pgfpathlineto{\pgfqpoint{3.709668in}{5.738362in}}%
\pgfpathlineto{\pgfqpoint{3.740322in}{5.744720in}}%
\pgfpathlineto{\pgfqpoint{3.744711in}{5.745614in}}%
\pgfpathlineto{\pgfqpoint{3.779753in}{5.752714in}}%
\pgfpathlineto{\pgfqpoint{3.814796in}{5.759813in}}%
\pgfpathlineto{\pgfqpoint{3.849838in}{5.766910in}}%
\pgfpathlineto{\pgfqpoint{3.869986in}{5.770979in}}%
\pgfpathlineto{\pgfqpoint{3.884881in}{5.773933in}}%
\pgfpathlineto{\pgfqpoint{3.919924in}{5.780859in}}%
\pgfpathlineto{\pgfqpoint{3.954966in}{5.787782in}}%
\pgfpathlineto{\pgfqpoint{3.990009in}{5.794703in}}%
\pgfpathlineto{\pgfqpoint{4.002899in}{5.797238in}}%
\pgfusepath{stroke}%
\end{pgfscope}%
\begin{pgfscope}%
\pgfpathrectangle{\pgfqpoint{0.766095in}{0.571603in}}{\pgfqpoint{6.973465in}{5.225635in}}%
\pgfusepath{clip}%
\pgfsetbuttcap%
\pgfsetroundjoin%
\pgfsetlinewidth{1.505625pt}%
\definecolor{currentstroke}{rgb}{0.125394,0.574318,0.549086}%
\pgfsetstrokecolor{currentstroke}%
\pgfsetdash{}{0pt}%
\pgfpathmoveto{\pgfqpoint{0.766095in}{4.909693in}}%
\pgfpathlineto{\pgfqpoint{0.801138in}{4.925153in}}%
\pgfpathlineto{\pgfqpoint{0.813713in}{4.930676in}}%
\pgfpathlineto{\pgfqpoint{0.836180in}{4.940385in}}%
\pgfpathlineto{\pgfqpoint{0.871223in}{4.955484in}}%
\pgfpathlineto{\pgfqpoint{0.874612in}{4.956935in}}%
\pgfpathlineto{\pgfqpoint{0.906265in}{4.970276in}}%
\pgfpathlineto{\pgfqpoint{0.936985in}{4.983195in}}%
\pgfpathlineto{\pgfqpoint{0.941308in}{4.984984in}}%
\pgfpathlineto{\pgfqpoint{0.976350in}{4.999402in}}%
\pgfpathlineto{\pgfqpoint{1.000847in}{5.009454in}}%
\pgfpathlineto{\pgfqpoint{1.011393in}{5.013712in}}%
\pgfpathlineto{\pgfqpoint{1.046435in}{5.027795in}}%
\pgfpathlineto{\pgfqpoint{1.066204in}{5.035714in}}%
\pgfpathlineto{\pgfqpoint{1.081478in}{5.041734in}}%
\pgfpathlineto{\pgfqpoint{1.116520in}{5.055491in}}%
\pgfpathlineto{\pgfqpoint{1.133096in}{5.061973in}}%
\pgfpathlineto{\pgfqpoint{1.151563in}{5.069080in}}%
\pgfpathlineto{\pgfqpoint{1.186605in}{5.082519in}}%
\pgfpathlineto{\pgfqpoint{1.201563in}{5.088233in}}%
\pgfpathlineto{\pgfqpoint{1.221648in}{5.095782in}}%
\pgfpathlineto{\pgfqpoint{1.256691in}{5.108911in}}%
\pgfpathlineto{\pgfqpoint{1.271646in}{5.114492in}}%
\pgfpathlineto{\pgfqpoint{1.291733in}{5.121868in}}%
\pgfpathlineto{\pgfqpoint{1.326776in}{5.134695in}}%
\pgfpathlineto{\pgfqpoint{1.343380in}{5.140752in}}%
\pgfpathlineto{\pgfqpoint{1.361818in}{5.147368in}}%
\pgfpathlineto{\pgfqpoint{1.396861in}{5.159901in}}%
\pgfpathlineto{\pgfqpoint{1.416802in}{5.167011in}}%
\pgfpathlineto{\pgfqpoint{1.431903in}{5.172309in}}%
\pgfpathlineto{\pgfqpoint{1.466946in}{5.184555in}}%
\pgfpathlineto{\pgfqpoint{1.491944in}{5.193271in}}%
\pgfpathlineto{\pgfqpoint{1.501988in}{5.196716in}}%
\pgfpathlineto{\pgfqpoint{1.537031in}{5.208682in}}%
\pgfpathlineto{\pgfqpoint{1.568838in}{5.219530in}}%
\pgfpathlineto{\pgfqpoint{1.572073in}{5.220616in}}%
\pgfpathlineto{\pgfqpoint{1.599321in}{5.229708in}}%
\pgfusepath{stroke}%
\end{pgfscope}%
\begin{pgfscope}%
\pgfpathrectangle{\pgfqpoint{0.766095in}{0.571603in}}{\pgfqpoint{6.973465in}{5.225635in}}%
\pgfusepath{clip}%
\pgfsetbuttcap%
\pgfsetroundjoin%
\pgfsetlinewidth{1.505625pt}%
\definecolor{currentstroke}{rgb}{0.125394,0.574318,0.549086}%
\pgfsetstrokecolor{currentstroke}%
\pgfsetdash{}{0pt}%
\pgfpathmoveto{\pgfqpoint{1.895820in}{5.324809in}}%
\pgfpathlineto{\pgfqpoint{1.922499in}{5.332933in}}%
\pgfpathlineto{\pgfqpoint{1.957541in}{5.343585in}}%
\pgfpathlineto{\pgfqpoint{1.981422in}{5.350827in}}%
\pgfpathlineto{\pgfqpoint{1.992584in}{5.354157in}}%
\pgfpathlineto{\pgfqpoint{2.027626in}{5.364568in}}%
\pgfpathlineto{\pgfqpoint{2.062669in}{5.374973in}}%
\pgfpathlineto{\pgfqpoint{2.069822in}{5.377087in}}%
\pgfpathlineto{\pgfqpoint{2.097711in}{5.385193in}}%
\pgfpathlineto{\pgfqpoint{2.132754in}{5.395363in}}%
\pgfpathlineto{\pgfqpoint{2.160308in}{5.403346in}}%
\pgfpathlineto{\pgfqpoint{2.167797in}{5.405480in}}%
\pgfpathlineto{\pgfqpoint{2.202839in}{5.415420in}}%
\pgfpathlineto{\pgfqpoint{2.237882in}{5.425354in}}%
\pgfpathlineto{\pgfqpoint{2.252930in}{5.429606in}}%
\pgfpathlineto{\pgfqpoint{2.272924in}{5.435161in}}%
\pgfpathlineto{\pgfqpoint{2.307967in}{5.444871in}}%
\pgfpathlineto{\pgfqpoint{2.343009in}{5.454577in}}%
\pgfpathlineto{\pgfqpoint{2.347684in}{5.455865in}}%
\pgfpathlineto{\pgfqpoint{2.378052in}{5.464095in}}%
\pgfpathlineto{\pgfqpoint{2.413094in}{5.473581in}}%
\pgfpathlineto{\pgfqpoint{2.444682in}{5.482125in}}%
\pgfpathlineto{\pgfqpoint{2.448137in}{5.483044in}}%
\pgfpathlineto{\pgfqpoint{2.483179in}{5.492315in}}%
\pgfpathlineto{\pgfqpoint{2.518222in}{5.501583in}}%
\pgfpathlineto{\pgfqpoint{2.543983in}{5.508384in}}%
\pgfpathlineto{\pgfqpoint{2.553264in}{5.510794in}}%
\pgfpathlineto{\pgfqpoint{2.588307in}{5.519852in}}%
\pgfpathlineto{\pgfqpoint{2.623350in}{5.528907in}}%
\pgfpathlineto{\pgfqpoint{2.645603in}{5.534644in}}%
\pgfpathlineto{\pgfqpoint{2.658392in}{5.537886in}}%
\pgfpathlineto{\pgfqpoint{2.693435in}{5.546735in}}%
\pgfpathlineto{\pgfqpoint{2.728477in}{5.555582in}}%
\pgfpathlineto{\pgfqpoint{2.749606in}{5.560903in}}%
\pgfpathlineto{\pgfqpoint{2.763520in}{5.564348in}}%
\pgfpathlineto{\pgfqpoint{2.798562in}{5.572994in}}%
\pgfpathlineto{\pgfqpoint{2.833605in}{5.581637in}}%
\pgfpathlineto{\pgfqpoint{2.856055in}{5.587163in}}%
\pgfpathlineto{\pgfqpoint{2.868647in}{5.590209in}}%
\pgfpathlineto{\pgfqpoint{2.903690in}{5.598656in}}%
\pgfpathlineto{\pgfqpoint{2.938732in}{5.607100in}}%
\pgfpathlineto{\pgfqpoint{2.965008in}{5.613422in}}%
\pgfpathlineto{\pgfqpoint{2.973775in}{5.615495in}}%
\pgfpathlineto{\pgfqpoint{3.008818in}{5.623747in}}%
\pgfpathlineto{\pgfqpoint{3.043860in}{5.631997in}}%
\pgfpathlineto{\pgfqpoint{3.076522in}{5.639682in}}%
\pgfpathlineto{\pgfqpoint{3.078903in}{5.640232in}}%
\pgfpathlineto{\pgfqpoint{3.113945in}{5.648293in}}%
\pgfpathlineto{\pgfqpoint{3.148988in}{5.656352in}}%
\pgfpathlineto{\pgfqpoint{3.184030in}{5.664410in}}%
\pgfpathlineto{\pgfqpoint{3.190721in}{5.665941in}}%
\pgfpathlineto{\pgfqpoint{3.219073in}{5.672318in}}%
\pgfpathlineto{\pgfqpoint{3.254115in}{5.680190in}}%
\pgfpathlineto{\pgfqpoint{3.289158in}{5.688061in}}%
\pgfpathlineto{\pgfqpoint{3.307638in}{5.692201in}}%
\pgfpathlineto{\pgfqpoint{3.324200in}{5.695846in}}%
\pgfpathlineto{\pgfqpoint{3.359243in}{5.703535in}}%
\pgfpathlineto{\pgfqpoint{3.394285in}{5.711223in}}%
\pgfpathlineto{\pgfqpoint{3.427290in}{5.718460in}}%
\pgfpathlineto{\pgfqpoint{3.429328in}{5.718899in}}%
\pgfpathlineto{\pgfqpoint{3.464371in}{5.726409in}}%
\pgfpathlineto{\pgfqpoint{3.499413in}{5.733917in}}%
\pgfpathlineto{\pgfqpoint{3.534456in}{5.741424in}}%
\pgfpathlineto{\pgfqpoint{3.549890in}{5.744720in}}%
\pgfpathlineto{\pgfqpoint{3.569498in}{5.748833in}}%
\pgfpathlineto{\pgfqpoint{3.604541in}{5.756165in}}%
\pgfpathlineto{\pgfqpoint{3.639583in}{5.763495in}}%
\pgfpathlineto{\pgfqpoint{3.674626in}{5.770825in}}%
\pgfpathlineto{\pgfqpoint{3.675368in}{5.770979in}}%
\pgfpathlineto{\pgfqpoint{3.709668in}{5.777987in}}%
\pgfpathlineto{\pgfqpoint{3.744711in}{5.785144in}}%
\pgfpathlineto{\pgfqpoint{3.779753in}{5.792300in}}%
\pgfpathlineto{\pgfqpoint{3.803980in}{5.797238in}}%
\pgfusepath{stroke}%
\end{pgfscope}%
\begin{pgfscope}%
\pgfpathrectangle{\pgfqpoint{0.766095in}{0.571603in}}{\pgfqpoint{6.973465in}{5.225635in}}%
\pgfusepath{clip}%
\pgfsetbuttcap%
\pgfsetroundjoin%
\pgfsetlinewidth{1.505625pt}%
\definecolor{currentstroke}{rgb}{0.121148,0.592739,0.544641}%
\pgfsetstrokecolor{currentstroke}%
\pgfsetdash{}{0pt}%
\pgfpathmoveto{\pgfqpoint{0.766095in}{4.960032in}}%
\pgfpathlineto{\pgfqpoint{0.801138in}{4.975105in}}%
\pgfpathlineto{\pgfqpoint{0.820013in}{4.983195in}}%
\pgfpathlineto{\pgfqpoint{0.836180in}{4.990014in}}%
\pgfpathlineto{\pgfqpoint{0.871223in}{5.004736in}}%
\pgfpathlineto{\pgfqpoint{0.882505in}{5.009454in}}%
\pgfpathlineto{\pgfqpoint{0.906265in}{5.019232in}}%
\pgfpathlineto{\pgfqpoint{0.941308in}{5.033613in}}%
\pgfpathlineto{\pgfqpoint{0.946454in}{5.035714in}}%
\pgfpathlineto{\pgfqpoint{0.976350in}{5.047722in}}%
\pgfpathlineto{\pgfqpoint{1.011393in}{5.061771in}}%
\pgfpathlineto{\pgfqpoint{1.011901in}{5.061973in}}%
\pgfpathlineto{\pgfqpoint{1.046435in}{5.075515in}}%
\pgfpathlineto{\pgfqpoint{1.078917in}{5.088233in}}%
\pgfpathlineto{\pgfqpoint{1.081478in}{5.089219in}}%
\pgfpathlineto{\pgfqpoint{1.116520in}{5.102643in}}%
\pgfpathlineto{\pgfqpoint{1.147501in}{5.114492in}}%
\pgfpathlineto{\pgfqpoint{1.151563in}{5.116021in}}%
\pgfpathlineto{\pgfqpoint{1.186605in}{5.129137in}}%
\pgfpathlineto{\pgfqpoint{1.215389in}{5.139894in}}%
\pgfusepath{stroke}%
\end{pgfscope}%
\begin{pgfscope}%
\pgfpathrectangle{\pgfqpoint{0.766095in}{0.571603in}}{\pgfqpoint{6.973465in}{5.225635in}}%
\pgfusepath{clip}%
\pgfsetbuttcap%
\pgfsetroundjoin%
\pgfsetlinewidth{1.505625pt}%
\definecolor{currentstroke}{rgb}{0.121148,0.592739,0.544641}%
\pgfsetstrokecolor{currentstroke}%
\pgfsetdash{}{0pt}%
\pgfpathmoveto{\pgfqpoint{1.508996in}{5.243665in}}%
\pgfpathlineto{\pgfqpoint{1.515249in}{5.245790in}}%
\pgfpathlineto{\pgfqpoint{1.537031in}{5.253070in}}%
\pgfpathlineto{\pgfqpoint{1.572073in}{5.264750in}}%
\pgfpathlineto{\pgfqpoint{1.594027in}{5.272049in}}%
\pgfpathlineto{\pgfqpoint{1.607116in}{5.276330in}}%
\pgfpathlineto{\pgfqpoint{1.642158in}{5.287747in}}%
\pgfpathlineto{\pgfqpoint{1.674609in}{5.298308in}}%
\pgfpathlineto{\pgfqpoint{1.677201in}{5.299138in}}%
\pgfpathlineto{\pgfqpoint{1.712244in}{5.310298in}}%
\pgfpathlineto{\pgfqpoint{1.747286in}{5.321450in}}%
\pgfpathlineto{\pgfqpoint{1.757127in}{5.324568in}}%
\pgfpathlineto{\pgfqpoint{1.782329in}{5.332424in}}%
\pgfpathlineto{\pgfqpoint{1.817371in}{5.343325in}}%
\pgfpathlineto{\pgfqpoint{1.841542in}{5.350827in}}%
\pgfpathlineto{\pgfqpoint{1.852414in}{5.354147in}}%
\pgfpathlineto{\pgfqpoint{1.887456in}{5.364803in}}%
\pgfpathlineto{\pgfqpoint{1.922499in}{5.375453in}}%
\pgfpathlineto{\pgfqpoint{1.927901in}{5.377087in}}%
\pgfpathlineto{\pgfqpoint{1.957541in}{5.385905in}}%
\pgfpathlineto{\pgfqpoint{1.992584in}{5.396316in}}%
\pgfpathlineto{\pgfqpoint{2.016300in}{5.403346in}}%
\pgfpathlineto{\pgfqpoint{2.027626in}{5.406649in}}%
\pgfpathlineto{\pgfqpoint{2.062669in}{5.416827in}}%
\pgfpathlineto{\pgfqpoint{2.097711in}{5.426999in}}%
\pgfpathlineto{\pgfqpoint{2.106730in}{5.429606in}}%
\pgfpathlineto{\pgfqpoint{2.132754in}{5.437005in}}%
\pgfpathlineto{\pgfqpoint{2.167797in}{5.446949in}}%
\pgfpathlineto{\pgfqpoint{2.199245in}{5.455865in}}%
\pgfpathlineto{\pgfqpoint{2.202839in}{5.456868in}}%
\pgfpathlineto{\pgfqpoint{2.237882in}{5.466590in}}%
\pgfpathlineto{\pgfqpoint{2.272924in}{5.476307in}}%
\pgfpathlineto{\pgfqpoint{2.293953in}{5.482125in}}%
\pgfpathlineto{\pgfqpoint{2.307967in}{5.485937in}}%
\pgfpathlineto{\pgfqpoint{2.343009in}{5.495438in}}%
\pgfpathlineto{\pgfqpoint{2.378052in}{5.504935in}}%
\pgfpathlineto{\pgfqpoint{2.390827in}{5.508384in}}%
\pgfpathlineto{\pgfqpoint{2.413094in}{5.514297in}}%
\pgfpathlineto{\pgfqpoint{2.448137in}{5.523581in}}%
\pgfpathlineto{\pgfqpoint{2.483179in}{5.532863in}}%
\pgfpathlineto{\pgfqpoint{2.489934in}{5.534644in}}%
\pgfpathlineto{\pgfqpoint{2.518222in}{5.541978in}}%
\pgfpathlineto{\pgfqpoint{2.553264in}{5.551051in}}%
\pgfpathlineto{\pgfqpoint{2.588307in}{5.560122in}}%
\pgfpathlineto{\pgfqpoint{2.591339in}{5.560903in}}%
\pgfpathlineto{\pgfqpoint{2.623350in}{5.569010in}}%
\pgfpathlineto{\pgfqpoint{2.658392in}{5.577877in}}%
\pgfpathlineto{\pgfqpoint{2.693435in}{5.586743in}}%
\pgfpathlineto{\pgfqpoint{2.695103in}{5.587163in}}%
\pgfpathlineto{\pgfqpoint{2.728477in}{5.595421in}}%
\pgfpathlineto{\pgfqpoint{2.763520in}{5.604087in}}%
\pgfpathlineto{\pgfqpoint{2.798562in}{5.612752in}}%
\pgfpathlineto{\pgfqpoint{2.801285in}{5.613422in}}%
\pgfpathlineto{\pgfqpoint{2.833605in}{5.621239in}}%
\pgfpathlineto{\pgfqpoint{2.868647in}{5.629709in}}%
\pgfpathlineto{\pgfqpoint{2.903690in}{5.638177in}}%
\pgfpathlineto{\pgfqpoint{2.909942in}{5.639682in}}%
\pgfpathlineto{\pgfqpoint{2.938732in}{5.646490in}}%
\pgfpathlineto{\pgfqpoint{2.973775in}{5.654768in}}%
\pgfpathlineto{\pgfqpoint{3.008818in}{5.663045in}}%
\pgfpathlineto{\pgfqpoint{3.021126in}{5.665941in}}%
\pgfpathlineto{\pgfqpoint{3.043860in}{5.671200in}}%
\pgfpathlineto{\pgfqpoint{3.078903in}{5.679290in}}%
\pgfpathlineto{\pgfqpoint{3.113945in}{5.687378in}}%
\pgfpathlineto{\pgfqpoint{3.134884in}{5.692201in}}%
\pgfpathlineto{\pgfqpoint{3.148988in}{5.695393in}}%
\pgfpathlineto{\pgfqpoint{3.184030in}{5.703299in}}%
\pgfpathlineto{\pgfqpoint{3.219073in}{5.711203in}}%
\pgfpathlineto{\pgfqpoint{3.251262in}{5.718460in}}%
\pgfpathlineto{\pgfqpoint{3.254115in}{5.719092in}}%
\pgfpathlineto{\pgfqpoint{3.289158in}{5.726817in}}%
\pgfpathlineto{\pgfqpoint{3.324200in}{5.734541in}}%
\pgfpathlineto{\pgfqpoint{3.359243in}{5.742265in}}%
\pgfpathlineto{\pgfqpoint{3.370423in}{5.744720in}}%
\pgfpathlineto{\pgfqpoint{3.394285in}{5.749868in}}%
\pgfpathlineto{\pgfqpoint{3.429328in}{5.757416in}}%
\pgfpathlineto{\pgfqpoint{3.464371in}{5.764962in}}%
\pgfpathlineto{\pgfqpoint{3.492343in}{5.770979in}}%
\pgfpathlineto{\pgfqpoint{3.499413in}{5.772473in}}%
\pgfpathlineto{\pgfqpoint{3.534456in}{5.779847in}}%
\pgfpathlineto{\pgfqpoint{3.569498in}{5.787220in}}%
\pgfpathlineto{\pgfqpoint{3.604541in}{5.794593in}}%
\pgfpathlineto{\pgfqpoint{3.617161in}{5.797238in}}%
\pgfusepath{stroke}%
\end{pgfscope}%
\begin{pgfscope}%
\pgfpathrectangle{\pgfqpoint{0.766095in}{0.571603in}}{\pgfqpoint{6.973465in}{5.225635in}}%
\pgfusepath{clip}%
\pgfsetbuttcap%
\pgfsetroundjoin%
\pgfsetlinewidth{1.505625pt}%
\definecolor{currentstroke}{rgb}{0.119423,0.611141,0.538982}%
\pgfsetstrokecolor{currentstroke}%
\pgfsetdash{}{0pt}%
\pgfpathmoveto{\pgfqpoint{0.766095in}{5.008404in}}%
\pgfpathlineto{\pgfqpoint{0.768559in}{5.009454in}}%
\pgfpathlineto{\pgfqpoint{0.797150in}{5.021445in}}%
\pgfusepath{stroke}%
\end{pgfscope}%
\begin{pgfscope}%
\pgfpathrectangle{\pgfqpoint{0.766095in}{0.571603in}}{\pgfqpoint{6.973465in}{5.225635in}}%
\pgfusepath{clip}%
\pgfsetbuttcap%
\pgfsetroundjoin%
\pgfsetlinewidth{1.505625pt}%
\definecolor{currentstroke}{rgb}{0.119423,0.611141,0.538982}%
\pgfsetstrokecolor{currentstroke}%
\pgfsetdash{}{0pt}%
\pgfpathmoveto{\pgfqpoint{1.086459in}{5.136765in}}%
\pgfpathlineto{\pgfqpoint{1.096934in}{5.140752in}}%
\pgfpathlineto{\pgfqpoint{1.116520in}{5.148087in}}%
\pgfpathlineto{\pgfqpoint{1.151563in}{5.161169in}}%
\pgfpathlineto{\pgfqpoint{1.167269in}{5.167011in}}%
\pgfpathlineto{\pgfqpoint{1.186605in}{5.174088in}}%
\pgfpathlineto{\pgfqpoint{1.221648in}{5.186873in}}%
\pgfpathlineto{\pgfqpoint{1.239242in}{5.193271in}}%
\pgfpathlineto{\pgfqpoint{1.256691in}{5.199514in}}%
\pgfpathlineto{\pgfqpoint{1.291733in}{5.212009in}}%
\pgfpathlineto{\pgfqpoint{1.312886in}{5.219530in}}%
\pgfpathlineto{\pgfqpoint{1.326776in}{5.224390in}}%
\pgfpathlineto{\pgfqpoint{1.361818in}{5.236603in}}%
\pgfpathlineto{\pgfqpoint{1.388232in}{5.245790in}}%
\pgfpathlineto{\pgfqpoint{1.396861in}{5.248743in}}%
\pgfpathlineto{\pgfqpoint{1.431903in}{5.260680in}}%
\pgfpathlineto{\pgfqpoint{1.465310in}{5.272049in}}%
\pgfpathlineto{\pgfqpoint{1.466946in}{5.272597in}}%
\pgfpathlineto{\pgfqpoint{1.501988in}{5.284266in}}%
\pgfpathlineto{\pgfqpoint{1.537031in}{5.295926in}}%
\pgfpathlineto{\pgfqpoint{1.544225in}{5.298308in}}%
\pgfpathlineto{\pgfqpoint{1.572073in}{5.307383in}}%
\pgfpathlineto{\pgfqpoint{1.607116in}{5.318782in}}%
\pgfpathlineto{\pgfqpoint{1.624960in}{5.324568in}}%
\pgfpathlineto{\pgfqpoint{1.642158in}{5.330055in}}%
\pgfpathlineto{\pgfqpoint{1.677201in}{5.341199in}}%
\pgfpathlineto{\pgfqpoint{1.707519in}{5.350827in}}%
\pgfpathlineto{\pgfqpoint{1.712244in}{5.352304in}}%
\pgfpathlineto{\pgfqpoint{1.747286in}{5.363198in}}%
\pgfpathlineto{\pgfqpoint{1.782329in}{5.374086in}}%
\pgfpathlineto{\pgfqpoint{1.792027in}{5.377087in}}%
\pgfpathlineto{\pgfqpoint{1.817371in}{5.384801in}}%
\pgfpathlineto{\pgfqpoint{1.852414in}{5.395447in}}%
\pgfpathlineto{\pgfqpoint{1.878466in}{5.403346in}}%
\pgfpathlineto{\pgfqpoint{1.887456in}{5.406028in}}%
\pgfpathlineto{\pgfqpoint{1.922499in}{5.416436in}}%
\pgfpathlineto{\pgfqpoint{1.957541in}{5.426839in}}%
\pgfpathlineto{\pgfqpoint{1.966900in}{5.429606in}}%
\pgfpathlineto{\pgfqpoint{1.992584in}{5.437075in}}%
\pgfpathlineto{\pgfqpoint{2.027626in}{5.447247in}}%
\pgfpathlineto{\pgfqpoint{2.057355in}{5.455865in}}%
\pgfpathlineto{\pgfqpoint{2.062669in}{5.457381in}}%
\pgfpathlineto{\pgfqpoint{2.097711in}{5.467327in}}%
\pgfpathlineto{\pgfqpoint{2.132754in}{5.477268in}}%
\pgfpathlineto{\pgfqpoint{2.149925in}{5.482125in}}%
\pgfpathlineto{\pgfqpoint{2.167797in}{5.487097in}}%
\pgfpathlineto{\pgfqpoint{2.202839in}{5.496818in}}%
\pgfpathlineto{\pgfqpoint{2.237882in}{5.506535in}}%
\pgfpathlineto{\pgfqpoint{2.244580in}{5.508384in}}%
\pgfpathlineto{\pgfqpoint{2.272924in}{5.516080in}}%
\pgfpathlineto{\pgfqpoint{2.307967in}{5.525582in}}%
\pgfpathlineto{\pgfqpoint{2.341403in}{5.534644in}}%
\pgfpathlineto{\pgfqpoint{2.343009in}{5.535072in}}%
\pgfpathlineto{\pgfqpoint{2.378052in}{5.544363in}}%
\pgfpathlineto{\pgfqpoint{2.413094in}{5.553652in}}%
\pgfpathlineto{\pgfqpoint{2.440491in}{5.560903in}}%
\pgfpathlineto{\pgfqpoint{2.448137in}{5.562894in}}%
\pgfpathlineto{\pgfqpoint{2.483179in}{5.571976in}}%
\pgfpathlineto{\pgfqpoint{2.518222in}{5.581056in}}%
\pgfpathlineto{\pgfqpoint{2.541833in}{5.587163in}}%
\pgfpathlineto{\pgfqpoint{2.553264in}{5.590070in}}%
\pgfpathlineto{\pgfqpoint{2.588307in}{5.598949in}}%
\pgfpathlineto{\pgfqpoint{2.623350in}{5.607826in}}%
\pgfpathlineto{\pgfqpoint{2.645489in}{5.613422in}}%
\pgfpathlineto{\pgfqpoint{2.658392in}{5.616630in}}%
\pgfpathlineto{\pgfqpoint{2.693435in}{5.625310in}}%
\pgfpathlineto{\pgfqpoint{2.728477in}{5.633988in}}%
\pgfpathlineto{\pgfqpoint{2.751513in}{5.639682in}}%
\pgfpathlineto{\pgfqpoint{2.763520in}{5.642600in}}%
\pgfpathlineto{\pgfqpoint{2.798562in}{5.651086in}}%
\pgfpathlineto{\pgfqpoint{2.833605in}{5.659570in}}%
\pgfpathlineto{\pgfqpoint{2.859957in}{5.665941in}}%
\pgfpathlineto{\pgfqpoint{2.868647in}{5.668007in}}%
\pgfpathlineto{\pgfqpoint{2.903690in}{5.676303in}}%
\pgfpathlineto{\pgfqpoint{2.938732in}{5.684597in}}%
\pgfpathlineto{\pgfqpoint{2.970871in}{5.692201in}}%
\pgfpathlineto{\pgfqpoint{2.973775in}{5.692876in}}%
\pgfpathlineto{\pgfqpoint{3.008818in}{5.700986in}}%
\pgfpathlineto{\pgfqpoint{3.043860in}{5.709095in}}%
\pgfpathlineto{\pgfqpoint{3.078903in}{5.717204in}}%
\pgfpathlineto{\pgfqpoint{3.084356in}{5.718460in}}%
\pgfpathlineto{\pgfqpoint{3.113945in}{5.725160in}}%
\pgfpathlineto{\pgfqpoint{3.148988in}{5.733088in}}%
\pgfpathlineto{\pgfqpoint{3.184030in}{5.741015in}}%
\pgfpathlineto{\pgfqpoint{3.200455in}{5.744720in}}%
\pgfpathlineto{\pgfqpoint{3.219073in}{5.748847in}}%
\pgfpathlineto{\pgfqpoint{3.254115in}{5.756597in}}%
\pgfpathlineto{\pgfqpoint{3.289158in}{5.764346in}}%
\pgfpathlineto{\pgfqpoint{3.319175in}{5.770979in}}%
\pgfpathlineto{\pgfqpoint{3.324200in}{5.772070in}}%
\pgfpathlineto{\pgfqpoint{3.359243in}{5.779646in}}%
\pgfpathlineto{\pgfqpoint{3.394285in}{5.787221in}}%
\pgfpathlineto{\pgfqpoint{3.429328in}{5.794796in}}%
\pgfpathlineto{\pgfqpoint{3.440670in}{5.797238in}}%
\pgfusepath{stroke}%
\end{pgfscope}%
\begin{pgfscope}%
\pgfpathrectangle{\pgfqpoint{0.766095in}{0.571603in}}{\pgfqpoint{6.973465in}{5.225635in}}%
\pgfusepath{clip}%
\pgfsetbuttcap%
\pgfsetroundjoin%
\pgfsetlinewidth{1.505625pt}%
\definecolor{currentstroke}{rgb}{0.121380,0.629492,0.531973}%
\pgfsetstrokecolor{currentstroke}%
\pgfsetdash{}{0pt}%
\pgfpathmoveto{\pgfqpoint{0.766095in}{5.054907in}}%
\pgfpathlineto{\pgfqpoint{0.783078in}{5.061973in}}%
\pgfpathlineto{\pgfqpoint{0.801138in}{5.069369in}}%
\pgfpathlineto{\pgfqpoint{0.836180in}{5.083669in}}%
\pgfpathlineto{\pgfqpoint{0.847415in}{5.088233in}}%
\pgfpathlineto{\pgfqpoint{0.871223in}{5.097751in}}%
\pgfpathlineto{\pgfqpoint{0.906265in}{5.111724in}}%
\pgfpathlineto{\pgfqpoint{0.913244in}{5.114492in}}%
\pgfpathlineto{\pgfqpoint{0.941308in}{5.125449in}}%
\pgfpathlineto{\pgfqpoint{0.976350in}{5.139103in}}%
\pgfpathlineto{\pgfqpoint{0.980604in}{5.140752in}}%
\pgfpathlineto{\pgfqpoint{1.011393in}{5.152494in}}%
\pgfpathlineto{\pgfqpoint{1.046435in}{5.165838in}}%
\pgfpathlineto{\pgfqpoint{1.049533in}{5.167011in}}%
\pgfpathlineto{\pgfqpoint{1.081478in}{5.178916in}}%
\pgfpathlineto{\pgfqpoint{1.116520in}{5.191957in}}%
\pgfpathlineto{\pgfqpoint{1.120068in}{5.193271in}}%
\pgfpathlineto{\pgfqpoint{1.151563in}{5.204743in}}%
\pgfpathlineto{\pgfqpoint{1.186605in}{5.217490in}}%
\pgfpathlineto{\pgfqpoint{1.192242in}{5.219530in}}%
\pgfpathlineto{\pgfqpoint{1.221648in}{5.230003in}}%
\pgfpathlineto{\pgfqpoint{1.256691in}{5.242463in}}%
\pgfpathlineto{\pgfqpoint{1.266089in}{5.245790in}}%
\pgfpathlineto{\pgfqpoint{1.291733in}{5.254722in}}%
\pgfpathlineto{\pgfqpoint{1.326776in}{5.266902in}}%
\pgfpathlineto{\pgfqpoint{1.341639in}{5.272049in}}%
\pgfpathlineto{\pgfqpoint{1.361818in}{5.278926in}}%
\pgfpathlineto{\pgfqpoint{1.396861in}{5.290833in}}%
\pgfpathlineto{\pgfqpoint{1.418919in}{5.298308in}}%
\pgfpathlineto{\pgfqpoint{1.431903in}{5.302639in}}%
\pgfpathlineto{\pgfqpoint{1.466946in}{5.314279in}}%
\pgfpathlineto{\pgfqpoint{1.497958in}{5.324568in}}%
\pgfpathlineto{\pgfqpoint{1.501988in}{5.325884in}}%
\pgfpathlineto{\pgfqpoint{1.537031in}{5.337265in}}%
\pgfpathlineto{\pgfqpoint{1.572073in}{5.348639in}}%
\pgfpathlineto{\pgfqpoint{1.578849in}{5.350827in}}%
\pgfpathlineto{\pgfqpoint{1.607116in}{5.359812in}}%
\pgfpathlineto{\pgfqpoint{1.642158in}{5.370933in}}%
\pgfpathlineto{\pgfqpoint{1.661605in}{5.377087in}}%
\pgfpathlineto{\pgfqpoint{1.677201in}{5.381943in}}%
\pgfpathlineto{\pgfqpoint{1.712244in}{5.392817in}}%
\pgfpathlineto{\pgfqpoint{1.746201in}{5.403346in}}%
\pgfpathlineto{\pgfqpoint{1.747286in}{5.403677in}}%
\pgfpathlineto{\pgfqpoint{1.782329in}{5.414311in}}%
\pgfpathlineto{\pgfqpoint{1.817371in}{5.424938in}}%
\pgfpathlineto{\pgfqpoint{1.832815in}{5.429606in}}%
\pgfpathlineto{\pgfqpoint{1.852414in}{5.435434in}}%
\pgfpathlineto{\pgfqpoint{1.887456in}{5.445827in}}%
\pgfpathlineto{\pgfqpoint{1.921328in}{5.455865in}}%
\pgfpathlineto{\pgfqpoint{1.922499in}{5.456207in}}%
\pgfpathlineto{\pgfqpoint{1.957541in}{5.466370in}}%
\pgfpathlineto{\pgfqpoint{1.992584in}{5.476528in}}%
\pgfpathlineto{\pgfqpoint{2.011942in}{5.482125in}}%
\pgfpathlineto{\pgfqpoint{2.027626in}{5.486586in}}%
\pgfpathlineto{\pgfqpoint{2.062669in}{5.496520in}}%
\pgfpathlineto{\pgfqpoint{2.097711in}{5.506451in}}%
\pgfpathlineto{\pgfqpoint{2.104563in}{5.508384in}}%
\pgfpathlineto{\pgfqpoint{2.132754in}{5.516208in}}%
\pgfpathlineto{\pgfqpoint{2.167797in}{5.525921in}}%
\pgfpathlineto{\pgfqpoint{2.199297in}{5.534644in}}%
\pgfpathlineto{\pgfqpoint{2.202839in}{5.535609in}}%
\pgfpathlineto{\pgfqpoint{2.237882in}{5.545108in}}%
\pgfpathlineto{\pgfqpoint{2.272924in}{5.554603in}}%
\pgfpathlineto{\pgfqpoint{2.296219in}{5.560903in}}%
\pgfpathlineto{\pgfqpoint{2.307967in}{5.564028in}}%
\pgfpathlineto{\pgfqpoint{2.343009in}{5.573316in}}%
\pgfpathlineto{\pgfqpoint{2.378052in}{5.582600in}}%
\pgfpathlineto{\pgfqpoint{2.395320in}{5.587163in}}%
\pgfpathlineto{\pgfqpoint{2.413094in}{5.591782in}}%
\pgfpathlineto{\pgfqpoint{2.448137in}{5.600863in}}%
\pgfpathlineto{\pgfqpoint{2.483179in}{5.609942in}}%
\pgfpathlineto{\pgfqpoint{2.496659in}{5.613422in}}%
\pgfpathlineto{\pgfqpoint{2.518222in}{5.618898in}}%
\pgfpathlineto{\pgfqpoint{2.553264in}{5.627778in}}%
\pgfpathlineto{\pgfqpoint{2.588307in}{5.636656in}}%
\pgfpathlineto{\pgfqpoint{2.600291in}{5.639682in}}%
\pgfpathlineto{\pgfqpoint{2.623350in}{5.645406in}}%
\pgfpathlineto{\pgfqpoint{2.643729in}{5.650456in}}%
\pgfusepath{stroke}%
\end{pgfscope}%
\begin{pgfscope}%
\pgfpathrectangle{\pgfqpoint{0.766095in}{0.571603in}}{\pgfqpoint{6.973465in}{5.225635in}}%
\pgfusepath{clip}%
\pgfsetbuttcap%
\pgfsetroundjoin%
\pgfsetlinewidth{1.505625pt}%
\definecolor{currentstroke}{rgb}{0.121380,0.629492,0.531973}%
\pgfsetstrokecolor{currentstroke}%
\pgfsetdash{}{0pt}%
\pgfpathmoveto{\pgfqpoint{3.023878in}{5.741274in}}%
\pgfpathlineto{\pgfqpoint{3.038761in}{5.744720in}}%
\pgfpathlineto{\pgfqpoint{3.043860in}{5.745880in}}%
\pgfpathlineto{\pgfqpoint{3.078903in}{5.753822in}}%
\pgfpathlineto{\pgfqpoint{3.113945in}{5.761763in}}%
\pgfpathlineto{\pgfqpoint{3.148988in}{5.769703in}}%
\pgfpathlineto{\pgfqpoint{3.154641in}{5.770979in}}%
\pgfpathlineto{\pgfqpoint{3.184030in}{5.777498in}}%
\pgfpathlineto{\pgfqpoint{3.219073in}{5.785263in}}%
\pgfpathlineto{\pgfqpoint{3.254115in}{5.793029in}}%
\pgfpathlineto{\pgfqpoint{3.273157in}{5.797238in}}%
\pgfusepath{stroke}%
\end{pgfscope}%
\begin{pgfscope}%
\pgfpathrectangle{\pgfqpoint{0.766095in}{0.571603in}}{\pgfqpoint{6.973465in}{5.225635in}}%
\pgfusepath{clip}%
\pgfsetbuttcap%
\pgfsetroundjoin%
\pgfsetlinewidth{1.505625pt}%
\definecolor{currentstroke}{rgb}{0.128087,0.647749,0.523491}%
\pgfsetstrokecolor{currentstroke}%
\pgfsetdash{}{0pt}%
\pgfpathmoveto{\pgfqpoint{0.766095in}{5.099774in}}%
\pgfpathlineto{\pgfqpoint{0.801138in}{5.114014in}}%
\pgfpathlineto{\pgfqpoint{0.802320in}{5.114492in}}%
\pgfpathlineto{\pgfqpoint{0.836180in}{5.127956in}}%
\pgfpathlineto{\pgfqpoint{0.868413in}{5.140752in}}%
\pgfpathlineto{\pgfqpoint{0.871223in}{5.141850in}}%
\pgfpathlineto{\pgfqpoint{0.906265in}{5.155465in}}%
\pgfpathlineto{\pgfqpoint{0.936037in}{5.167011in}}%
\pgfpathlineto{\pgfqpoint{0.941308in}{5.169023in}}%
\pgfpathlineto{\pgfqpoint{0.976350in}{5.182330in}}%
\pgfpathlineto{\pgfqpoint{1.005216in}{5.193271in}}%
\pgfpathlineto{\pgfqpoint{1.011393in}{5.195575in}}%
\pgfpathlineto{\pgfqpoint{1.046435in}{5.208581in}}%
\pgfpathlineto{\pgfqpoint{1.075986in}{5.219530in}}%
\pgfpathlineto{\pgfqpoint{1.081478in}{5.221533in}}%
\pgfpathlineto{\pgfqpoint{1.116520in}{5.234246in}}%
\pgfpathlineto{\pgfqpoint{1.148380in}{5.245790in}}%
\pgfpathlineto{\pgfqpoint{1.151563in}{5.246925in}}%
\pgfpathlineto{\pgfqpoint{1.186605in}{5.259353in}}%
\pgfpathlineto{\pgfqpoint{1.221648in}{5.271771in}}%
\pgfpathlineto{\pgfqpoint{1.222438in}{5.272049in}}%
\pgfpathlineto{\pgfqpoint{1.256691in}{5.283927in}}%
\pgfpathlineto{\pgfqpoint{1.291733in}{5.296067in}}%
\pgfpathlineto{\pgfqpoint{1.298234in}{5.298308in}}%
\pgfpathlineto{\pgfqpoint{1.326776in}{5.307993in}}%
\pgfpathlineto{\pgfqpoint{1.361818in}{5.319863in}}%
\pgfpathlineto{\pgfqpoint{1.375761in}{5.324568in}}%
\pgfpathlineto{\pgfqpoint{1.396861in}{5.331575in}}%
\pgfpathlineto{\pgfqpoint{1.431903in}{5.343182in}}%
\pgfpathlineto{\pgfqpoint{1.455043in}{5.350827in}}%
\pgfpathlineto{\pgfqpoint{1.466946in}{5.354698in}}%
\pgfpathlineto{\pgfqpoint{1.501988in}{5.366047in}}%
\pgfpathlineto{\pgfqpoint{1.536104in}{5.377087in}}%
\pgfpathlineto{\pgfqpoint{1.537031in}{5.377382in}}%
\pgfpathlineto{\pgfqpoint{1.572073in}{5.388480in}}%
\pgfpathlineto{\pgfqpoint{1.607116in}{5.399571in}}%
\pgfpathlineto{\pgfqpoint{1.619089in}{5.403346in}}%
\pgfpathlineto{\pgfqpoint{1.642158in}{5.410503in}}%
\pgfpathlineto{\pgfqpoint{1.677201in}{5.421350in}}%
\pgfpathlineto{\pgfqpoint{1.703920in}{5.429606in}}%
\pgfpathlineto{\pgfqpoint{1.712244in}{5.432137in}}%
\pgfpathlineto{\pgfqpoint{1.747286in}{5.442745in}}%
\pgfpathlineto{\pgfqpoint{1.782329in}{5.453347in}}%
\pgfpathlineto{\pgfqpoint{1.790687in}{5.455865in}}%
\pgfpathlineto{\pgfqpoint{1.817371in}{5.463775in}}%
\pgfpathlineto{\pgfqpoint{1.852414in}{5.474145in}}%
\pgfpathlineto{\pgfqpoint{1.879423in}{5.482125in}}%
\pgfpathlineto{\pgfqpoint{1.887456in}{5.484460in}}%
\pgfpathlineto{\pgfqpoint{1.922499in}{5.494603in}}%
\pgfpathlineto{\pgfqpoint{1.957541in}{5.504741in}}%
\pgfpathlineto{\pgfqpoint{1.970178in}{5.508384in}}%
\pgfpathlineto{\pgfqpoint{1.992584in}{5.514739in}}%
\pgfpathlineto{\pgfqpoint{2.027626in}{5.524656in}}%
\pgfpathlineto{\pgfqpoint{2.049415in}{5.530819in}}%
\pgfusepath{stroke}%
\end{pgfscope}%
\begin{pgfscope}%
\pgfpathrectangle{\pgfqpoint{0.766095in}{0.571603in}}{\pgfqpoint{6.973465in}{5.225635in}}%
\pgfusepath{clip}%
\pgfsetbuttcap%
\pgfsetroundjoin%
\pgfsetlinewidth{1.505625pt}%
\definecolor{currentstroke}{rgb}{0.128087,0.647749,0.523491}%
\pgfsetstrokecolor{currentstroke}%
\pgfsetdash{}{0pt}%
\pgfpathmoveto{\pgfqpoint{2.426892in}{5.632278in}}%
\pgfpathlineto{\pgfqpoint{2.448137in}{5.637778in}}%
\pgfpathlineto{\pgfqpoint{2.455523in}{5.639682in}}%
\pgfpathlineto{\pgfqpoint{2.483179in}{5.646694in}}%
\pgfpathlineto{\pgfqpoint{2.518222in}{5.655567in}}%
\pgfpathlineto{\pgfqpoint{2.553264in}{5.664439in}}%
\pgfpathlineto{\pgfqpoint{2.559223in}{5.665941in}}%
\pgfpathlineto{\pgfqpoint{2.588307in}{5.673152in}}%
\pgfpathlineto{\pgfqpoint{2.623350in}{5.681831in}}%
\pgfpathlineto{\pgfqpoint{2.658392in}{5.690510in}}%
\pgfpathlineto{\pgfqpoint{2.665249in}{5.692201in}}%
\pgfpathlineto{\pgfqpoint{2.693435in}{5.699037in}}%
\pgfpathlineto{\pgfqpoint{2.728477in}{5.707527in}}%
\pgfpathlineto{\pgfqpoint{2.763520in}{5.716016in}}%
\pgfpathlineto{\pgfqpoint{2.773646in}{5.718460in}}%
\pgfpathlineto{\pgfqpoint{2.798562in}{5.724374in}}%
\pgfpathlineto{\pgfqpoint{2.833605in}{5.732679in}}%
\pgfpathlineto{\pgfqpoint{2.868647in}{5.740983in}}%
\pgfpathlineto{\pgfqpoint{2.884460in}{5.744720in}}%
\pgfpathlineto{\pgfqpoint{2.903690in}{5.749188in}}%
\pgfpathlineto{\pgfqpoint{2.938732in}{5.757312in}}%
\pgfpathlineto{\pgfqpoint{2.973775in}{5.765435in}}%
\pgfpathlineto{\pgfqpoint{2.997728in}{5.770979in}}%
\pgfpathlineto{\pgfqpoint{3.008818in}{5.773503in}}%
\pgfpathlineto{\pgfqpoint{3.043860in}{5.781449in}}%
\pgfpathlineto{\pgfqpoint{3.078903in}{5.789396in}}%
\pgfpathlineto{\pgfqpoint{3.113487in}{5.797238in}}%
\pgfusepath{stroke}%
\end{pgfscope}%
\begin{pgfscope}%
\pgfpathrectangle{\pgfqpoint{0.766095in}{0.571603in}}{\pgfqpoint{6.973465in}{5.225635in}}%
\pgfusepath{clip}%
\pgfsetbuttcap%
\pgfsetroundjoin%
\pgfsetlinewidth{1.505625pt}%
\definecolor{currentstroke}{rgb}{0.140210,0.665859,0.513427}%
\pgfsetstrokecolor{currentstroke}%
\pgfsetdash{}{0pt}%
\pgfpathmoveto{\pgfqpoint{0.766095in}{5.143158in}}%
\pgfpathlineto{\pgfqpoint{0.801138in}{5.157029in}}%
\pgfpathlineto{\pgfqpoint{0.826421in}{5.167011in}}%
\pgfpathlineto{\pgfqpoint{0.836180in}{5.170804in}}%
\pgfpathlineto{\pgfqpoint{0.871223in}{5.184360in}}%
\pgfpathlineto{\pgfqpoint{0.894318in}{5.193271in}}%
\pgfpathlineto{\pgfqpoint{0.906265in}{5.197808in}}%
\pgfpathlineto{\pgfqpoint{0.941308in}{5.211058in}}%
\pgfpathlineto{\pgfqpoint{0.963774in}{5.219530in}}%
\pgfpathlineto{\pgfqpoint{0.976350in}{5.224198in}}%
\pgfpathlineto{\pgfqpoint{1.011393in}{5.237151in}}%
\pgfpathlineto{\pgfqpoint{1.034824in}{5.245790in}}%
\pgfpathlineto{\pgfqpoint{1.046435in}{5.250003in}}%
\pgfpathlineto{\pgfqpoint{1.081478in}{5.262666in}}%
\pgfpathlineto{\pgfqpoint{1.107500in}{5.272049in}}%
\pgfpathlineto{\pgfqpoint{1.116520in}{5.275250in}}%
\pgfpathlineto{\pgfqpoint{1.151563in}{5.287631in}}%
\pgfpathlineto{\pgfqpoint{1.181833in}{5.298308in}}%
\pgfpathlineto{\pgfqpoint{1.186605in}{5.299966in}}%
\pgfpathlineto{\pgfqpoint{1.221648in}{5.312071in}}%
\pgfpathlineto{\pgfqpoint{1.256691in}{5.324165in}}%
\pgfpathlineto{\pgfqpoint{1.257863in}{5.324568in}}%
\pgfpathlineto{\pgfqpoint{1.291733in}{5.336010in}}%
\pgfpathlineto{\pgfqpoint{1.326776in}{5.347837in}}%
\pgfpathlineto{\pgfqpoint{1.335674in}{5.350827in}}%
\pgfpathlineto{\pgfqpoint{1.361818in}{5.359474in}}%
\pgfpathlineto{\pgfqpoint{1.396861in}{5.371039in}}%
\pgfpathlineto{\pgfqpoint{1.415240in}{5.377087in}}%
\pgfpathlineto{\pgfqpoint{1.431903in}{5.382483in}}%
\pgfpathlineto{\pgfqpoint{1.466946in}{5.393794in}}%
\pgfpathlineto{\pgfqpoint{1.496581in}{5.403346in}}%
\pgfpathlineto{\pgfqpoint{1.501988in}{5.405062in}}%
\pgfpathlineto{\pgfqpoint{1.537031in}{5.416124in}}%
\pgfpathlineto{\pgfqpoint{1.572073in}{5.427180in}}%
\pgfpathlineto{\pgfqpoint{1.579798in}{5.429606in}}%
\pgfpathlineto{\pgfqpoint{1.607116in}{5.438050in}}%
\pgfpathlineto{\pgfqpoint{1.642158in}{5.448863in}}%
\pgfpathlineto{\pgfqpoint{1.664901in}{5.455865in}}%
\pgfpathlineto{\pgfqpoint{1.677201in}{5.459592in}}%
\pgfpathlineto{\pgfqpoint{1.712244in}{5.470169in}}%
\pgfpathlineto{\pgfqpoint{1.747286in}{5.480741in}}%
\pgfpathlineto{\pgfqpoint{1.751896in}{5.482125in}}%
\pgfpathlineto{\pgfqpoint{1.782329in}{5.491116in}}%
\pgfpathlineto{\pgfqpoint{1.817371in}{5.501457in}}%
\pgfpathlineto{\pgfqpoint{1.840892in}{5.508384in}}%
\pgfpathlineto{\pgfqpoint{1.852414in}{5.511723in}}%
\pgfpathlineto{\pgfqpoint{1.887456in}{5.521839in}}%
\pgfpathlineto{\pgfqpoint{1.922499in}{5.531951in}}%
\pgfpathlineto{\pgfqpoint{1.931866in}{5.534644in}}%
\pgfpathlineto{\pgfqpoint{1.957541in}{5.541905in}}%
\pgfpathlineto{\pgfqpoint{1.992584in}{5.551797in}}%
\pgfpathlineto{\pgfqpoint{2.024864in}{5.560903in}}%
\pgfpathlineto{\pgfqpoint{2.027626in}{5.561670in}}%
\pgfpathlineto{\pgfqpoint{2.049278in}{5.567650in}}%
\pgfusepath{stroke}%
\end{pgfscope}%
\begin{pgfscope}%
\pgfpathrectangle{\pgfqpoint{0.766095in}{0.571603in}}{\pgfqpoint{6.973465in}{5.225635in}}%
\pgfusepath{clip}%
\pgfsetbuttcap%
\pgfsetroundjoin%
\pgfsetlinewidth{1.505625pt}%
\definecolor{currentstroke}{rgb}{0.140210,0.665859,0.513427}%
\pgfsetstrokecolor{currentstroke}%
\pgfsetdash{}{0pt}%
\pgfpathmoveto{\pgfqpoint{2.427003in}{5.668172in}}%
\pgfpathlineto{\pgfqpoint{2.448137in}{5.673521in}}%
\pgfpathlineto{\pgfqpoint{2.483179in}{5.682381in}}%
\pgfpathlineto{\pgfqpoint{2.518222in}{5.691241in}}%
\pgfpathlineto{\pgfqpoint{2.522036in}{5.692201in}}%
\pgfpathlineto{\pgfqpoint{2.553264in}{5.699931in}}%
\pgfpathlineto{\pgfqpoint{2.588307in}{5.708600in}}%
\pgfpathlineto{\pgfqpoint{2.623350in}{5.717268in}}%
\pgfpathlineto{\pgfqpoint{2.628190in}{5.718460in}}%
\pgfpathlineto{\pgfqpoint{2.658392in}{5.725776in}}%
\pgfpathlineto{\pgfqpoint{2.693435in}{5.734257in}}%
\pgfpathlineto{\pgfqpoint{2.728477in}{5.742738in}}%
\pgfpathlineto{\pgfqpoint{2.736694in}{5.744720in}}%
\pgfpathlineto{\pgfqpoint{2.763520in}{5.751080in}}%
\pgfpathlineto{\pgfqpoint{2.798562in}{5.759379in}}%
\pgfpathlineto{\pgfqpoint{2.833605in}{5.767677in}}%
\pgfpathlineto{\pgfqpoint{2.847589in}{5.770979in}}%
\pgfpathlineto{\pgfqpoint{2.868647in}{5.775868in}}%
\pgfpathlineto{\pgfqpoint{2.903690in}{5.783988in}}%
\pgfpathlineto{\pgfqpoint{2.938732in}{5.792108in}}%
\pgfpathlineto{\pgfqpoint{2.960912in}{5.797238in}}%
\pgfusepath{stroke}%
\end{pgfscope}%
\begin{pgfscope}%
\pgfpathrectangle{\pgfqpoint{0.766095in}{0.571603in}}{\pgfqpoint{6.973465in}{5.225635in}}%
\pgfusepath{clip}%
\pgfsetbuttcap%
\pgfsetroundjoin%
\pgfsetlinewidth{1.505625pt}%
\definecolor{currentstroke}{rgb}{0.157851,0.683765,0.501686}%
\pgfsetstrokecolor{currentstroke}%
\pgfsetdash{}{0pt}%
\pgfpathmoveto{\pgfqpoint{0.766095in}{5.185035in}}%
\pgfpathlineto{\pgfqpoint{0.787068in}{5.193271in}}%
\pgfpathlineto{\pgfqpoint{0.801138in}{5.198710in}}%
\pgfpathlineto{\pgfqpoint{0.836180in}{5.212202in}}%
\pgfpathlineto{\pgfqpoint{0.855275in}{5.219530in}}%
\pgfpathlineto{\pgfqpoint{0.871223in}{5.225555in}}%
\pgfpathlineto{\pgfqpoint{0.906265in}{5.238744in}}%
\pgfpathlineto{\pgfqpoint{0.925045in}{5.245790in}}%
\pgfpathlineto{\pgfqpoint{0.941308in}{5.251795in}}%
\pgfpathlineto{\pgfqpoint{0.976350in}{5.264690in}}%
\pgfpathlineto{\pgfqpoint{0.996410in}{5.272049in}}%
\pgfpathlineto{\pgfqpoint{1.011393in}{5.277459in}}%
\pgfpathlineto{\pgfqpoint{1.046435in}{5.290066in}}%
\pgfpathlineto{\pgfqpoint{1.069404in}{5.298308in}}%
\pgfpathlineto{\pgfqpoint{1.081478in}{5.302573in}}%
\pgfpathlineto{\pgfqpoint{1.116520in}{5.314900in}}%
\pgfpathlineto{\pgfqpoint{1.144055in}{5.324568in}}%
\pgfpathlineto{\pgfqpoint{1.151563in}{5.327162in}}%
\pgfpathlineto{\pgfqpoint{1.186605in}{5.339217in}}%
\pgfpathlineto{\pgfqpoint{1.220393in}{5.350827in}}%
\pgfpathlineto{\pgfqpoint{1.221648in}{5.351252in}}%
\pgfpathlineto{\pgfqpoint{1.256691in}{5.363040in}}%
\pgfpathlineto{\pgfqpoint{1.291733in}{5.374819in}}%
\pgfpathlineto{\pgfqpoint{1.298511in}{5.377087in}}%
\pgfpathlineto{\pgfqpoint{1.326776in}{5.386395in}}%
\pgfpathlineto{\pgfqpoint{1.361818in}{5.397915in}}%
\pgfpathlineto{\pgfqpoint{1.378395in}{5.403346in}}%
\pgfpathlineto{\pgfqpoint{1.396861in}{5.409302in}}%
\pgfpathlineto{\pgfqpoint{1.431903in}{5.420570in}}%
\pgfpathlineto{\pgfqpoint{1.460051in}{5.429606in}}%
\pgfpathlineto{\pgfqpoint{1.466946in}{5.431784in}}%
\pgfpathlineto{\pgfqpoint{1.501988in}{5.442806in}}%
\pgfpathlineto{\pgfqpoint{1.537031in}{5.453820in}}%
\pgfpathlineto{\pgfqpoint{1.543567in}{5.455865in}}%
\pgfpathlineto{\pgfqpoint{1.572073in}{5.464644in}}%
\pgfpathlineto{\pgfqpoint{1.607116in}{5.475419in}}%
\pgfpathlineto{\pgfqpoint{1.628978in}{5.482125in}}%
\pgfpathlineto{\pgfqpoint{1.642158in}{5.486104in}}%
\pgfpathlineto{\pgfqpoint{1.677201in}{5.496645in}}%
\pgfpathlineto{\pgfqpoint{1.712244in}{5.507180in}}%
\pgfpathlineto{\pgfqpoint{1.716269in}{5.508384in}}%
\pgfpathlineto{\pgfqpoint{1.747286in}{5.517517in}}%
\pgfpathlineto{\pgfqpoint{1.782329in}{5.527825in}}%
\pgfpathlineto{\pgfqpoint{1.805561in}{5.534644in}}%
\pgfpathlineto{\pgfqpoint{1.817371in}{5.538055in}}%
\pgfpathlineto{\pgfqpoint{1.852414in}{5.548140in}}%
\pgfpathlineto{\pgfqpoint{1.887456in}{5.558220in}}%
\pgfpathlineto{\pgfqpoint{1.896820in}{5.560903in}}%
\pgfpathlineto{\pgfqpoint{1.922499in}{5.568143in}}%
\pgfpathlineto{\pgfqpoint{1.957541in}{5.578006in}}%
\pgfpathlineto{\pgfqpoint{1.990096in}{5.587163in}}%
\pgfpathlineto{\pgfqpoint{1.992584in}{5.587851in}}%
\pgfpathlineto{\pgfqpoint{2.027626in}{5.597503in}}%
\pgfpathlineto{\pgfqpoint{2.049143in}{5.603426in}}%
\pgfusepath{stroke}%
\end{pgfscope}%
\begin{pgfscope}%
\pgfpathrectangle{\pgfqpoint{0.766095in}{0.571603in}}{\pgfqpoint{6.973465in}{5.225635in}}%
\pgfusepath{clip}%
\pgfsetbuttcap%
\pgfsetroundjoin%
\pgfsetlinewidth{1.505625pt}%
\definecolor{currentstroke}{rgb}{0.157851,0.683765,0.501686}%
\pgfsetstrokecolor{currentstroke}%
\pgfsetdash{}{0pt}%
\pgfpathmoveto{\pgfqpoint{2.427121in}{5.702979in}}%
\pgfpathlineto{\pgfqpoint{2.448137in}{5.708282in}}%
\pgfpathlineto{\pgfqpoint{2.483179in}{5.717122in}}%
\pgfpathlineto{\pgfqpoint{2.488503in}{5.718460in}}%
\pgfpathlineto{\pgfqpoint{2.518222in}{5.725804in}}%
\pgfpathlineto{\pgfqpoint{2.553264in}{5.734457in}}%
\pgfpathlineto{\pgfqpoint{2.588307in}{5.743109in}}%
\pgfpathlineto{\pgfqpoint{2.594858in}{5.744720in}}%
\pgfpathlineto{\pgfqpoint{2.623350in}{5.751611in}}%
\pgfpathlineto{\pgfqpoint{2.658392in}{5.760078in}}%
\pgfpathlineto{\pgfqpoint{2.693435in}{5.768545in}}%
\pgfpathlineto{\pgfqpoint{2.703543in}{5.770979in}}%
\pgfpathlineto{\pgfqpoint{2.728477in}{5.776884in}}%
\pgfpathlineto{\pgfqpoint{2.763520in}{5.785171in}}%
\pgfpathlineto{\pgfqpoint{2.798562in}{5.793457in}}%
\pgfpathlineto{\pgfqpoint{2.814594in}{5.797238in}}%
\pgfusepath{stroke}%
\end{pgfscope}%
\begin{pgfscope}%
\pgfpathrectangle{\pgfqpoint{0.766095in}{0.571603in}}{\pgfqpoint{6.973465in}{5.225635in}}%
\pgfusepath{clip}%
\pgfsetbuttcap%
\pgfsetroundjoin%
\pgfsetlinewidth{1.505625pt}%
\definecolor{currentstroke}{rgb}{0.180653,0.701402,0.488189}%
\pgfsetstrokecolor{currentstroke}%
\pgfsetdash{}{0pt}%
\pgfpathmoveto{\pgfqpoint{0.766095in}{5.225628in}}%
\pgfpathlineto{\pgfqpoint{0.801138in}{5.239051in}}%
\pgfpathlineto{\pgfqpoint{0.818788in}{5.245790in}}%
\pgfpathlineto{\pgfqpoint{0.836180in}{5.252326in}}%
\pgfpathlineto{\pgfqpoint{0.871223in}{5.265449in}}%
\pgfpathlineto{\pgfqpoint{0.888905in}{5.272049in}}%
\pgfpathlineto{\pgfqpoint{0.906265in}{5.278428in}}%
\pgfpathlineto{\pgfqpoint{0.941308in}{5.291259in}}%
\pgfpathlineto{\pgfqpoint{0.960620in}{5.298308in}}%
\pgfpathlineto{\pgfqpoint{0.976350in}{5.303961in}}%
\pgfpathlineto{\pgfqpoint{1.011393in}{5.316508in}}%
\pgfpathlineto{\pgfqpoint{1.033965in}{5.324568in}}%
\pgfpathlineto{\pgfqpoint{1.046435in}{5.328952in}}%
\pgfpathlineto{\pgfqpoint{1.081478in}{5.341221in}}%
\pgfpathlineto{\pgfqpoint{1.108968in}{5.350827in}}%
\pgfpathlineto{\pgfqpoint{1.116520in}{5.353426in}}%
\pgfpathlineto{\pgfqpoint{1.151563in}{5.365425in}}%
\pgfpathlineto{\pgfqpoint{1.185656in}{5.377087in}}%
\pgfpathlineto{\pgfqpoint{1.186605in}{5.377407in}}%
\pgfpathlineto{\pgfqpoint{1.221648in}{5.389142in}}%
\pgfpathlineto{\pgfqpoint{1.256691in}{5.400868in}}%
\pgfpathlineto{\pgfqpoint{1.264131in}{5.403346in}}%
\pgfpathlineto{\pgfqpoint{1.291733in}{5.412397in}}%
\pgfpathlineto{\pgfqpoint{1.326776in}{5.423867in}}%
\pgfpathlineto{\pgfqpoint{1.344365in}{5.429606in}}%
\pgfpathlineto{\pgfqpoint{1.361818in}{5.435211in}}%
\pgfpathlineto{\pgfqpoint{1.396861in}{5.446431in}}%
\pgfpathlineto{\pgfqpoint{1.426369in}{5.455865in}}%
\pgfpathlineto{\pgfqpoint{1.431903in}{5.457607in}}%
\pgfpathlineto{\pgfqpoint{1.466946in}{5.468583in}}%
\pgfpathlineto{\pgfqpoint{1.501988in}{5.479551in}}%
\pgfpathlineto{\pgfqpoint{1.510245in}{5.482125in}}%
\pgfpathlineto{\pgfqpoint{1.537031in}{5.490342in}}%
\pgfpathlineto{\pgfqpoint{1.572073in}{5.501074in}}%
\pgfpathlineto{\pgfqpoint{1.595996in}{5.508384in}}%
\pgfpathlineto{\pgfqpoint{1.607116in}{5.511729in}}%
\pgfpathlineto{\pgfqpoint{1.642158in}{5.522229in}}%
\pgfpathlineto{\pgfqpoint{1.677201in}{5.532723in}}%
\pgfpathlineto{\pgfqpoint{1.683644in}{5.534644in}}%
\pgfpathlineto{\pgfqpoint{1.712244in}{5.543036in}}%
\pgfpathlineto{\pgfqpoint{1.747286in}{5.553304in}}%
\pgfpathlineto{\pgfqpoint{1.773263in}{5.560903in}}%
\pgfpathlineto{\pgfqpoint{1.782329in}{5.563513in}}%
\pgfpathlineto{\pgfqpoint{1.817371in}{5.573561in}}%
\pgfpathlineto{\pgfqpoint{1.852414in}{5.583605in}}%
\pgfpathlineto{\pgfqpoint{1.864871in}{5.587163in}}%
\pgfpathlineto{\pgfqpoint{1.887456in}{5.593511in}}%
\pgfpathlineto{\pgfqpoint{1.922499in}{5.603340in}}%
\pgfpathlineto{\pgfqpoint{1.957541in}{5.613165in}}%
\pgfpathlineto{\pgfqpoint{1.958464in}{5.613422in}}%
\pgfpathlineto{\pgfqpoint{1.992584in}{5.622789in}}%
\pgfpathlineto{\pgfqpoint{2.027626in}{5.632404in}}%
\pgfpathlineto{\pgfqpoint{2.054186in}{5.639682in}}%
\pgfpathlineto{\pgfqpoint{2.062669in}{5.641969in}}%
\pgfpathlineto{\pgfqpoint{2.063040in}{5.642068in}}%
\pgfusepath{stroke}%
\end{pgfscope}%
\begin{pgfscope}%
\pgfpathrectangle{\pgfqpoint{0.766095in}{0.571603in}}{\pgfqpoint{6.973465in}{5.225635in}}%
\pgfusepath{clip}%
\pgfsetbuttcap%
\pgfsetroundjoin%
\pgfsetlinewidth{1.505625pt}%
\definecolor{currentstroke}{rgb}{0.180653,0.701402,0.488189}%
\pgfsetstrokecolor{currentstroke}%
\pgfsetdash{}{0pt}%
\pgfpathmoveto{\pgfqpoint{2.441328in}{5.740429in}}%
\pgfpathlineto{\pgfqpoint{2.448137in}{5.742142in}}%
\pgfpathlineto{\pgfqpoint{2.458414in}{5.744720in}}%
\pgfpathlineto{\pgfqpoint{2.483179in}{5.750828in}}%
\pgfpathlineto{\pgfqpoint{2.518222in}{5.759459in}}%
\pgfpathlineto{\pgfqpoint{2.553264in}{5.768089in}}%
\pgfpathlineto{\pgfqpoint{2.565036in}{5.770979in}}%
\pgfpathlineto{\pgfqpoint{2.588307in}{5.776599in}}%
\pgfpathlineto{\pgfqpoint{2.623350in}{5.785047in}}%
\pgfpathlineto{\pgfqpoint{2.658392in}{5.793494in}}%
\pgfpathlineto{\pgfqpoint{2.673967in}{5.797238in}}%
\pgfusepath{stroke}%
\end{pgfscope}%
\begin{pgfscope}%
\pgfpathrectangle{\pgfqpoint{0.766095in}{0.571603in}}{\pgfqpoint{6.973465in}{5.225635in}}%
\pgfusepath{clip}%
\pgfsetbuttcap%
\pgfsetroundjoin%
\pgfsetlinewidth{1.505625pt}%
\definecolor{currentstroke}{rgb}{0.208030,0.718701,0.472873}%
\pgfsetstrokecolor{currentstroke}%
\pgfsetdash{}{0pt}%
\pgfpathmoveto{\pgfqpoint{0.766095in}{5.264967in}}%
\pgfpathlineto{\pgfqpoint{0.784745in}{5.272049in}}%
\pgfpathlineto{\pgfqpoint{0.801138in}{5.278177in}}%
\pgfpathlineto{\pgfqpoint{0.836180in}{5.291231in}}%
\pgfpathlineto{\pgfqpoint{0.855240in}{5.298308in}}%
\pgfpathlineto{\pgfqpoint{0.871223in}{5.304151in}}%
\pgfpathlineto{\pgfqpoint{0.906265in}{5.316915in}}%
\pgfpathlineto{\pgfqpoint{0.927335in}{5.324568in}}%
\pgfpathlineto{\pgfqpoint{0.941308in}{5.329564in}}%
\pgfpathlineto{\pgfqpoint{0.976350in}{5.342046in}}%
\pgfpathlineto{\pgfqpoint{1.001061in}{5.350827in}}%
\pgfpathlineto{\pgfqpoint{1.011393in}{5.354442in}}%
\pgfpathlineto{\pgfqpoint{1.046435in}{5.366649in}}%
\pgfpathlineto{\pgfqpoint{1.076445in}{5.377087in}}%
\pgfpathlineto{\pgfqpoint{1.081478in}{5.378810in}}%
\pgfpathlineto{\pgfqpoint{1.116520in}{5.390750in}}%
\pgfpathlineto{\pgfqpoint{1.151563in}{5.402679in}}%
\pgfpathlineto{\pgfqpoint{1.153535in}{5.403346in}}%
\pgfpathlineto{\pgfqpoint{1.186605in}{5.414371in}}%
\pgfpathlineto{\pgfqpoint{1.221648in}{5.426039in}}%
\pgfpathlineto{\pgfqpoint{1.232402in}{5.429606in}}%
\pgfpathlineto{\pgfqpoint{1.256691in}{5.437535in}}%
\pgfpathlineto{\pgfqpoint{1.291733in}{5.448950in}}%
\pgfpathlineto{\pgfqpoint{1.313016in}{5.455865in}}%
\pgfpathlineto{\pgfqpoint{1.326776in}{5.460266in}}%
\pgfpathlineto{\pgfqpoint{1.361818in}{5.471433in}}%
\pgfpathlineto{\pgfqpoint{1.395398in}{5.482125in}}%
\pgfpathlineto{\pgfqpoint{1.396861in}{5.482583in}}%
\pgfpathlineto{\pgfqpoint{1.431903in}{5.493509in}}%
\pgfpathlineto{\pgfqpoint{1.466946in}{5.504428in}}%
\pgfpathlineto{\pgfqpoint{1.479689in}{5.508384in}}%
\pgfpathlineto{\pgfqpoint{1.501988in}{5.515199in}}%
\pgfpathlineto{\pgfqpoint{1.537031in}{5.525882in}}%
\pgfpathlineto{\pgfqpoint{1.565809in}{5.534644in}}%
\pgfpathlineto{\pgfqpoint{1.572073in}{5.536521in}}%
\pgfpathlineto{\pgfqpoint{1.607116in}{5.546975in}}%
\pgfpathlineto{\pgfqpoint{1.642158in}{5.557424in}}%
\pgfpathlineto{\pgfqpoint{1.653870in}{5.560903in}}%
\pgfpathlineto{\pgfqpoint{1.677201in}{5.567725in}}%
\pgfpathlineto{\pgfqpoint{1.712244in}{5.577950in}}%
\pgfpathlineto{\pgfqpoint{1.743845in}{5.587163in}}%
\pgfpathlineto{\pgfqpoint{1.747286in}{5.588150in}}%
\pgfpathlineto{\pgfqpoint{1.782329in}{5.598157in}}%
\pgfpathlineto{\pgfqpoint{1.817371in}{5.608159in}}%
\pgfpathlineto{\pgfqpoint{1.835857in}{5.613422in}}%
\pgfpathlineto{\pgfqpoint{1.852414in}{5.618061in}}%
\pgfpathlineto{\pgfqpoint{1.887456in}{5.627851in}}%
\pgfpathlineto{\pgfqpoint{1.922499in}{5.637637in}}%
\pgfpathlineto{\pgfqpoint{1.929850in}{5.639682in}}%
\pgfpathlineto{\pgfqpoint{1.957541in}{5.647261in}}%
\pgfpathlineto{\pgfqpoint{1.992584in}{5.656840in}}%
\pgfpathlineto{\pgfqpoint{2.025899in}{5.665941in}}%
\pgfpathlineto{\pgfqpoint{2.027626in}{5.666406in}}%
\pgfpathlineto{\pgfqpoint{2.048903in}{5.672098in}}%
\pgfusepath{stroke}%
\end{pgfscope}%
\begin{pgfscope}%
\pgfpathrectangle{\pgfqpoint{0.766095in}{0.571603in}}{\pgfqpoint{6.973465in}{5.225635in}}%
\pgfusepath{clip}%
\pgfsetbuttcap%
\pgfsetroundjoin%
\pgfsetlinewidth{1.505625pt}%
\definecolor{currentstroke}{rgb}{0.208030,0.718701,0.472873}%
\pgfsetstrokecolor{currentstroke}%
\pgfsetdash{}{0pt}%
\pgfpathmoveto{\pgfqpoint{2.427330in}{5.769917in}}%
\pgfpathlineto{\pgfqpoint{2.431576in}{5.770979in}}%
\pgfpathlineto{\pgfqpoint{2.448137in}{5.775056in}}%
\pgfpathlineto{\pgfqpoint{2.483179in}{5.783661in}}%
\pgfpathlineto{\pgfqpoint{2.518222in}{5.792264in}}%
\pgfpathlineto{\pgfqpoint{2.538522in}{5.797238in}}%
\pgfusepath{stroke}%
\end{pgfscope}%
\begin{pgfscope}%
\pgfpathrectangle{\pgfqpoint{0.766095in}{0.571603in}}{\pgfqpoint{6.973465in}{5.225635in}}%
\pgfusepath{clip}%
\pgfsetbuttcap%
\pgfsetroundjoin%
\pgfsetlinewidth{1.505625pt}%
\definecolor{currentstroke}{rgb}{0.239374,0.735588,0.455688}%
\pgfsetstrokecolor{currentstroke}%
\pgfsetdash{}{0pt}%
\pgfpathmoveto{\pgfqpoint{0.766095in}{5.303164in}}%
\pgfpathlineto{\pgfqpoint{0.801138in}{5.316144in}}%
\pgfpathlineto{\pgfqpoint{0.823940in}{5.324568in}}%
\pgfpathlineto{\pgfqpoint{0.836180in}{5.329020in}}%
\pgfpathlineto{\pgfqpoint{0.862063in}{5.338395in}}%
\pgfusepath{stroke}%
\end{pgfscope}%
\begin{pgfscope}%
\pgfpathrectangle{\pgfqpoint{0.766095in}{0.571603in}}{\pgfqpoint{6.973465in}{5.225635in}}%
\pgfusepath{clip}%
\pgfsetbuttcap%
\pgfsetroundjoin%
\pgfsetlinewidth{1.505625pt}%
\definecolor{currentstroke}{rgb}{0.239374,0.735588,0.455688}%
\pgfsetstrokecolor{currentstroke}%
\pgfsetdash{}{0pt}%
\pgfpathmoveto{\pgfqpoint{1.231880in}{5.465177in}}%
\pgfpathlineto{\pgfqpoint{1.256691in}{5.473218in}}%
\pgfpathlineto{\pgfqpoint{1.284224in}{5.482125in}}%
\pgfpathlineto{\pgfqpoint{1.291733in}{5.484516in}}%
\pgfpathlineto{\pgfqpoint{1.326776in}{5.495628in}}%
\pgfpathlineto{\pgfqpoint{1.361818in}{5.506731in}}%
\pgfpathlineto{\pgfqpoint{1.367060in}{5.508384in}}%
\pgfpathlineto{\pgfqpoint{1.396861in}{5.517637in}}%
\pgfpathlineto{\pgfqpoint{1.431903in}{5.528501in}}%
\pgfpathlineto{\pgfqpoint{1.451766in}{5.534644in}}%
\pgfpathlineto{\pgfqpoint{1.466946in}{5.539264in}}%
\pgfpathlineto{\pgfqpoint{1.501988in}{5.549896in}}%
\pgfpathlineto{\pgfqpoint{1.537031in}{5.560522in}}%
\pgfpathlineto{\pgfqpoint{1.538294in}{5.560903in}}%
\pgfpathlineto{\pgfqpoint{1.572073in}{5.570935in}}%
\pgfpathlineto{\pgfqpoint{1.607116in}{5.581334in}}%
\pgfpathlineto{\pgfqpoint{1.626807in}{5.587163in}}%
\pgfpathlineto{\pgfqpoint{1.642158in}{5.591635in}}%
\pgfpathlineto{\pgfqpoint{1.677201in}{5.601813in}}%
\pgfpathlineto{\pgfqpoint{1.712244in}{5.611986in}}%
\pgfpathlineto{\pgfqpoint{1.717212in}{5.613422in}}%
\pgfpathlineto{\pgfqpoint{1.747286in}{5.621978in}}%
\pgfpathlineto{\pgfqpoint{1.782329in}{5.631935in}}%
\pgfpathlineto{\pgfqpoint{1.809631in}{5.639682in}}%
\pgfpathlineto{\pgfqpoint{1.817371in}{5.641843in}}%
\pgfpathlineto{\pgfqpoint{1.852414in}{5.651590in}}%
\pgfpathlineto{\pgfqpoint{1.887456in}{5.661333in}}%
\pgfpathlineto{\pgfqpoint{1.904075in}{5.665941in}}%
\pgfpathlineto{\pgfqpoint{1.922499in}{5.670969in}}%
\pgfpathlineto{\pgfqpoint{1.957541in}{5.680507in}}%
\pgfpathlineto{\pgfqpoint{1.992584in}{5.690041in}}%
\pgfpathlineto{\pgfqpoint{2.000551in}{5.692201in}}%
\pgfpathlineto{\pgfqpoint{2.027626in}{5.699423in}}%
\pgfpathlineto{\pgfqpoint{2.062669in}{5.708757in}}%
\pgfpathlineto{\pgfqpoint{2.097711in}{5.718089in}}%
\pgfpathlineto{\pgfqpoint{2.099110in}{5.718460in}}%
\pgfpathlineto{\pgfqpoint{2.132754in}{5.727236in}}%
\pgfpathlineto{\pgfqpoint{2.167797in}{5.736373in}}%
\pgfpathlineto{\pgfqpoint{2.199830in}{5.744720in}}%
\pgfpathlineto{\pgfqpoint{2.202839in}{5.745491in}}%
\pgfpathlineto{\pgfqpoint{2.237882in}{5.754437in}}%
\pgfpathlineto{\pgfqpoint{2.272924in}{5.763381in}}%
\pgfpathlineto{\pgfqpoint{2.302720in}{5.770979in}}%
\pgfpathlineto{\pgfqpoint{2.307967in}{5.772296in}}%
\pgfpathlineto{\pgfqpoint{2.343009in}{5.781053in}}%
\pgfpathlineto{\pgfqpoint{2.378052in}{5.789809in}}%
\pgfpathlineto{\pgfqpoint{2.407808in}{5.797238in}}%
\pgfusepath{stroke}%
\end{pgfscope}%
\begin{pgfscope}%
\pgfpathrectangle{\pgfqpoint{0.766095in}{0.571603in}}{\pgfqpoint{6.973465in}{5.225635in}}%
\pgfusepath{clip}%
\pgfsetbuttcap%
\pgfsetroundjoin%
\pgfsetlinewidth{1.505625pt}%
\definecolor{currentstroke}{rgb}{0.274149,0.751988,0.436601}%
\pgfsetstrokecolor{currentstroke}%
\pgfsetdash{}{0pt}%
\pgfpathmoveto{\pgfqpoint{0.766095in}{5.340240in}}%
\pgfpathlineto{\pgfqpoint{0.794904in}{5.350827in}}%
\pgfpathlineto{\pgfqpoint{0.801138in}{5.353083in}}%
\pgfpathlineto{\pgfqpoint{0.836180in}{5.365702in}}%
\pgfpathlineto{\pgfqpoint{0.867842in}{5.377087in}}%
\pgfpathlineto{\pgfqpoint{0.871223in}{5.378284in}}%
\pgfpathlineto{\pgfqpoint{0.906265in}{5.390626in}}%
\pgfpathlineto{\pgfqpoint{0.941308in}{5.402956in}}%
\pgfpathlineto{\pgfqpoint{0.942423in}{5.403346in}}%
\pgfpathlineto{\pgfqpoint{0.976350in}{5.415037in}}%
\pgfpathlineto{\pgfqpoint{1.011393in}{5.427098in}}%
\pgfpathlineto{\pgfqpoint{1.018713in}{5.429606in}}%
\pgfpathlineto{\pgfqpoint{1.046435in}{5.438958in}}%
\pgfpathlineto{\pgfqpoint{1.081478in}{5.450757in}}%
\pgfpathlineto{\pgfqpoint{1.096703in}{5.455865in}}%
\pgfpathlineto{\pgfqpoint{1.116520in}{5.462412in}}%
\pgfpathlineto{\pgfqpoint{1.151563in}{5.473956in}}%
\pgfpathlineto{\pgfqpoint{1.176414in}{5.482125in}}%
\pgfpathlineto{\pgfqpoint{1.186605in}{5.485423in}}%
\pgfpathlineto{\pgfqpoint{1.221648in}{5.496718in}}%
\pgfpathlineto{\pgfqpoint{1.256691in}{5.508003in}}%
\pgfpathlineto{\pgfqpoint{1.257879in}{5.508384in}}%
\pgfpathlineto{\pgfqpoint{1.291733in}{5.519063in}}%
\pgfpathlineto{\pgfqpoint{1.326776in}{5.530107in}}%
\pgfpathlineto{\pgfqpoint{1.341221in}{5.534644in}}%
\pgfpathlineto{\pgfqpoint{1.361818in}{5.541013in}}%
\pgfpathlineto{\pgfqpoint{1.396861in}{5.551820in}}%
\pgfpathlineto{\pgfqpoint{1.426352in}{5.560903in}}%
\pgfpathlineto{\pgfqpoint{1.431903in}{5.562586in}}%
\pgfpathlineto{\pgfqpoint{1.454739in}{5.569479in}}%
\pgfusepath{stroke}%
\end{pgfscope}%
\begin{pgfscope}%
\pgfpathrectangle{\pgfqpoint{0.766095in}{0.571603in}}{\pgfqpoint{6.973465in}{5.225635in}}%
\pgfusepath{clip}%
\pgfsetbuttcap%
\pgfsetroundjoin%
\pgfsetlinewidth{1.505625pt}%
\definecolor{currentstroke}{rgb}{0.274149,0.751988,0.436601}%
\pgfsetstrokecolor{currentstroke}%
\pgfsetdash{}{0pt}%
\pgfpathmoveto{\pgfqpoint{1.830220in}{5.678161in}}%
\pgfpathlineto{\pgfqpoint{1.852414in}{5.684302in}}%
\pgfpathlineto{\pgfqpoint{1.880994in}{5.692201in}}%
\pgfpathlineto{\pgfqpoint{1.887456in}{5.693958in}}%
\pgfpathlineto{\pgfqpoint{1.922499in}{5.703452in}}%
\pgfpathlineto{\pgfqpoint{1.957541in}{5.712942in}}%
\pgfpathlineto{\pgfqpoint{1.977961in}{5.718460in}}%
\pgfpathlineto{\pgfqpoint{1.992584in}{5.722349in}}%
\pgfpathlineto{\pgfqpoint{2.027626in}{5.731641in}}%
\pgfpathlineto{\pgfqpoint{2.062669in}{5.740931in}}%
\pgfpathlineto{\pgfqpoint{2.077003in}{5.744720in}}%
\pgfpathlineto{\pgfqpoint{2.097711in}{5.750106in}}%
\pgfpathlineto{\pgfqpoint{2.132754in}{5.759203in}}%
\pgfpathlineto{\pgfqpoint{2.167797in}{5.768297in}}%
\pgfpathlineto{\pgfqpoint{2.178166in}{5.770979in}}%
\pgfpathlineto{\pgfqpoint{2.202839in}{5.777259in}}%
\pgfpathlineto{\pgfqpoint{2.237882in}{5.786165in}}%
\pgfpathlineto{\pgfqpoint{2.272924in}{5.795069in}}%
\pgfpathlineto{\pgfqpoint{2.281495in}{5.797238in}}%
\pgfusepath{stroke}%
\end{pgfscope}%
\begin{pgfscope}%
\pgfpathrectangle{\pgfqpoint{0.766095in}{0.571603in}}{\pgfqpoint{6.973465in}{5.225635in}}%
\pgfusepath{clip}%
\pgfsetbuttcap%
\pgfsetroundjoin%
\pgfsetlinewidth{1.505625pt}%
\definecolor{currentstroke}{rgb}{0.311925,0.767822,0.415586}%
\pgfsetstrokecolor{currentstroke}%
\pgfsetdash{}{0pt}%
\pgfpathmoveto{\pgfqpoint{0.766095in}{5.376375in}}%
\pgfpathlineto{\pgfqpoint{0.768054in}{5.377087in}}%
\pgfpathlineto{\pgfqpoint{0.801138in}{5.388931in}}%
\pgfpathlineto{\pgfqpoint{0.836180in}{5.401460in}}%
\pgfpathlineto{\pgfqpoint{0.841482in}{5.403346in}}%
\pgfpathlineto{\pgfqpoint{0.871223in}{5.413765in}}%
\pgfpathlineto{\pgfqpoint{0.906265in}{5.426021in}}%
\pgfpathlineto{\pgfqpoint{0.916559in}{5.429606in}}%
\pgfpathlineto{\pgfqpoint{0.941308in}{5.438093in}}%
\pgfpathlineto{\pgfqpoint{0.976350in}{5.450082in}}%
\pgfpathlineto{\pgfqpoint{0.993311in}{5.455865in}}%
\pgfpathlineto{\pgfqpoint{1.011393in}{5.461937in}}%
\pgfpathlineto{\pgfqpoint{1.046435in}{5.473666in}}%
\pgfpathlineto{\pgfqpoint{1.071761in}{5.482125in}}%
\pgfpathlineto{\pgfqpoint{1.081478in}{5.485320in}}%
\pgfpathlineto{\pgfqpoint{1.116520in}{5.496797in}}%
\pgfpathlineto{\pgfqpoint{1.151563in}{5.508264in}}%
\pgfpathlineto{\pgfqpoint{1.151934in}{5.508384in}}%
\pgfpathlineto{\pgfqpoint{1.186605in}{5.519496in}}%
\pgfpathlineto{\pgfqpoint{1.221648in}{5.530717in}}%
\pgfpathlineto{\pgfqpoint{1.233956in}{5.534644in}}%
\pgfpathlineto{\pgfqpoint{1.256691in}{5.541785in}}%
\pgfpathlineto{\pgfqpoint{1.291733in}{5.552766in}}%
\pgfpathlineto{\pgfqpoint{1.317750in}{5.560903in}}%
\pgfpathlineto{\pgfqpoint{1.326776in}{5.563682in}}%
\pgfpathlineto{\pgfqpoint{1.361818in}{5.574430in}}%
\pgfpathlineto{\pgfqpoint{1.396861in}{5.585170in}}%
\pgfpathlineto{\pgfqpoint{1.403391in}{5.587163in}}%
\pgfpathlineto{\pgfqpoint{1.431903in}{5.595729in}}%
\pgfpathlineto{\pgfqpoint{1.454580in}{5.602531in}}%
\pgfusepath{stroke}%
\end{pgfscope}%
\begin{pgfscope}%
\pgfpathrectangle{\pgfqpoint{0.766095in}{0.571603in}}{\pgfqpoint{6.973465in}{5.225635in}}%
\pgfusepath{clip}%
\pgfsetbuttcap%
\pgfsetroundjoin%
\pgfsetlinewidth{1.505625pt}%
\definecolor{currentstroke}{rgb}{0.311925,0.767822,0.415586}%
\pgfsetstrokecolor{currentstroke}%
\pgfsetdash{}{0pt}%
\pgfpathmoveto{\pgfqpoint{1.830361in}{5.710161in}}%
\pgfpathlineto{\pgfqpoint{1.852414in}{5.716229in}}%
\pgfpathlineto{\pgfqpoint{1.860550in}{5.718460in}}%
\pgfpathlineto{\pgfqpoint{1.887456in}{5.725720in}}%
\pgfpathlineto{\pgfqpoint{1.922499in}{5.735163in}}%
\pgfpathlineto{\pgfqpoint{1.957541in}{5.744602in}}%
\pgfpathlineto{\pgfqpoint{1.957978in}{5.744720in}}%
\pgfpathlineto{\pgfqpoint{1.992584in}{5.753851in}}%
\pgfpathlineto{\pgfqpoint{2.027626in}{5.763095in}}%
\pgfpathlineto{\pgfqpoint{2.057541in}{5.770979in}}%
\pgfpathlineto{\pgfqpoint{2.062669in}{5.772309in}}%
\pgfpathlineto{\pgfqpoint{2.097711in}{5.781362in}}%
\pgfpathlineto{\pgfqpoint{2.132754in}{5.790412in}}%
\pgfpathlineto{\pgfqpoint{2.159218in}{5.797238in}}%
\pgfusepath{stroke}%
\end{pgfscope}%
\begin{pgfscope}%
\pgfpathrectangle{\pgfqpoint{0.766095in}{0.571603in}}{\pgfqpoint{6.973465in}{5.225635in}}%
\pgfusepath{clip}%
\pgfsetbuttcap%
\pgfsetroundjoin%
\pgfsetlinewidth{1.505625pt}%
\definecolor{currentstroke}{rgb}{0.352360,0.783011,0.392636}%
\pgfsetstrokecolor{currentstroke}%
\pgfsetdash{}{0pt}%
\pgfpathmoveto{\pgfqpoint{0.766095in}{5.411445in}}%
\pgfpathlineto{\pgfqpoint{0.801138in}{5.423894in}}%
\pgfpathlineto{\pgfqpoint{0.817272in}{5.429606in}}%
\pgfpathlineto{\pgfqpoint{0.836180in}{5.436197in}}%
\pgfpathlineto{\pgfqpoint{0.871223in}{5.448376in}}%
\pgfpathlineto{\pgfqpoint{0.892833in}{5.455865in}}%
\pgfpathlineto{\pgfqpoint{0.906265in}{5.460449in}}%
\pgfpathlineto{\pgfqpoint{0.941308in}{5.472364in}}%
\pgfpathlineto{\pgfqpoint{0.970068in}{5.482125in}}%
\pgfpathlineto{\pgfqpoint{0.976350in}{5.484224in}}%
\pgfpathlineto{\pgfqpoint{1.011393in}{5.495882in}}%
\pgfpathlineto{\pgfqpoint{1.046435in}{5.507528in}}%
\pgfpathlineto{\pgfqpoint{1.049024in}{5.508384in}}%
\pgfpathlineto{\pgfqpoint{1.081478in}{5.518952in}}%
\pgfpathlineto{\pgfqpoint{1.116520in}{5.530349in}}%
\pgfpathlineto{\pgfqpoint{1.129773in}{5.534644in}}%
\pgfpathlineto{\pgfqpoint{1.151563in}{5.541596in}}%
\pgfpathlineto{\pgfqpoint{1.186605in}{5.552750in}}%
\pgfpathlineto{\pgfqpoint{1.212274in}{5.560903in}}%
\pgfpathlineto{\pgfqpoint{1.221648in}{5.563835in}}%
\pgfpathlineto{\pgfqpoint{1.256691in}{5.574751in}}%
\pgfpathlineto{\pgfqpoint{1.291733in}{5.585659in}}%
\pgfpathlineto{\pgfqpoint{1.296587in}{5.587163in}}%
\pgfpathlineto{\pgfqpoint{1.326776in}{5.596373in}}%
\pgfpathlineto{\pgfqpoint{1.361818in}{5.607050in}}%
\pgfpathlineto{\pgfqpoint{1.382784in}{5.613422in}}%
\pgfpathlineto{\pgfqpoint{1.396861in}{5.617634in}}%
\pgfpathlineto{\pgfqpoint{1.431903in}{5.628086in}}%
\pgfpathlineto{\pgfqpoint{1.464990in}{5.637948in}}%
\pgfusepath{stroke}%
\end{pgfscope}%
\begin{pgfscope}%
\pgfpathrectangle{\pgfqpoint{0.766095in}{0.571603in}}{\pgfqpoint{6.973465in}{5.225635in}}%
\pgfusepath{clip}%
\pgfsetbuttcap%
\pgfsetroundjoin%
\pgfsetlinewidth{1.505625pt}%
\definecolor{currentstroke}{rgb}{0.352360,0.783011,0.392636}%
\pgfsetstrokecolor{currentstroke}%
\pgfsetdash{}{0pt}%
\pgfpathmoveto{\pgfqpoint{1.841129in}{5.744311in}}%
\pgfpathlineto{\pgfqpoint{1.842622in}{5.744720in}}%
\pgfpathlineto{\pgfqpoint{1.852414in}{5.747353in}}%
\pgfpathlineto{\pgfqpoint{1.887456in}{5.756745in}}%
\pgfpathlineto{\pgfqpoint{1.922499in}{5.766134in}}%
\pgfpathlineto{\pgfqpoint{1.940626in}{5.770979in}}%
\pgfpathlineto{\pgfqpoint{1.957541in}{5.775429in}}%
\pgfpathlineto{\pgfqpoint{1.992584in}{5.784624in}}%
\pgfpathlineto{\pgfqpoint{2.027626in}{5.793817in}}%
\pgfpathlineto{\pgfqpoint{2.040708in}{5.797238in}}%
\pgfusepath{stroke}%
\end{pgfscope}%
\begin{pgfscope}%
\pgfpathrectangle{\pgfqpoint{0.766095in}{0.571603in}}{\pgfqpoint{6.973465in}{5.225635in}}%
\pgfusepath{clip}%
\pgfsetbuttcap%
\pgfsetroundjoin%
\pgfsetlinewidth{1.505625pt}%
\definecolor{currentstroke}{rgb}{0.395174,0.797475,0.367757}%
\pgfsetstrokecolor{currentstroke}%
\pgfsetdash{}{0pt}%
\pgfpathmoveto{\pgfqpoint{0.766095in}{5.445653in}}%
\pgfpathlineto{\pgfqpoint{0.795087in}{5.455865in}}%
\pgfpathlineto{\pgfqpoint{0.801138in}{5.457964in}}%
\pgfpathlineto{\pgfqpoint{0.836180in}{5.470063in}}%
\pgfpathlineto{\pgfqpoint{0.871152in}{5.482125in}}%
\pgfpathlineto{\pgfqpoint{0.871223in}{5.482149in}}%
\pgfpathlineto{\pgfqpoint{0.906265in}{5.493987in}}%
\pgfpathlineto{\pgfqpoint{0.941308in}{5.505812in}}%
\pgfpathlineto{\pgfqpoint{0.948963in}{5.508384in}}%
\pgfpathlineto{\pgfqpoint{0.976350in}{5.517446in}}%
\pgfpathlineto{\pgfqpoint{1.011393in}{5.529018in}}%
\pgfpathlineto{\pgfqpoint{1.028484in}{5.534644in}}%
\pgfpathlineto{\pgfqpoint{1.046435in}{5.540463in}}%
\pgfpathlineto{\pgfqpoint{1.081478in}{5.551787in}}%
\pgfpathlineto{\pgfqpoint{1.109734in}{5.560903in}}%
\pgfpathlineto{\pgfqpoint{1.116520in}{5.563059in}}%
\pgfpathlineto{\pgfqpoint{1.151563in}{5.574143in}}%
\pgfpathlineto{\pgfqpoint{1.186605in}{5.585217in}}%
\pgfpathlineto{\pgfqpoint{1.192790in}{5.587163in}}%
\pgfpathlineto{\pgfqpoint{1.221648in}{5.596104in}}%
\pgfpathlineto{\pgfqpoint{1.256691in}{5.606944in}}%
\pgfpathlineto{\pgfqpoint{1.277686in}{5.613422in}}%
\pgfpathlineto{\pgfqpoint{1.291733in}{5.617690in}}%
\pgfpathlineto{\pgfqpoint{1.326776in}{5.628301in}}%
\pgfpathlineto{\pgfqpoint{1.361818in}{5.638905in}}%
\pgfpathlineto{\pgfqpoint{1.364396in}{5.639682in}}%
\pgfpathlineto{\pgfqpoint{1.396861in}{5.649309in}}%
\pgfpathlineto{\pgfqpoint{1.431903in}{5.659690in}}%
\pgfpathlineto{\pgfqpoint{1.453052in}{5.665941in}}%
\pgfpathlineto{\pgfqpoint{1.454276in}{5.666297in}}%
\pgfusepath{stroke}%
\end{pgfscope}%
\begin{pgfscope}%
\pgfpathrectangle{\pgfqpoint{0.766095in}{0.571603in}}{\pgfqpoint{6.973465in}{5.225635in}}%
\pgfusepath{clip}%
\pgfsetbuttcap%
\pgfsetroundjoin%
\pgfsetlinewidth{1.505625pt}%
\definecolor{currentstroke}{rgb}{0.395174,0.797475,0.367757}%
\pgfsetstrokecolor{currentstroke}%
\pgfsetdash{}{0pt}%
\pgfpathmoveto{\pgfqpoint{1.830623in}{5.771914in}}%
\pgfpathlineto{\pgfqpoint{1.852414in}{5.777729in}}%
\pgfpathlineto{\pgfqpoint{1.887456in}{5.787064in}}%
\pgfpathlineto{\pgfqpoint{1.922499in}{5.796397in}}%
\pgfpathlineto{\pgfqpoint{1.925673in}{5.797238in}}%
\pgfusepath{stroke}%
\end{pgfscope}%
\begin{pgfscope}%
\pgfpathrectangle{\pgfqpoint{0.766095in}{0.571603in}}{\pgfqpoint{6.973465in}{5.225635in}}%
\pgfusepath{clip}%
\pgfsetbuttcap%
\pgfsetroundjoin%
\pgfsetlinewidth{1.505625pt}%
\definecolor{currentstroke}{rgb}{0.440137,0.811138,0.340967}%
\pgfsetstrokecolor{currentstroke}%
\pgfsetdash{}{0pt}%
\pgfpathmoveto{\pgfqpoint{0.766095in}{5.479046in}}%
\pgfpathlineto{\pgfqpoint{0.774927in}{5.482125in}}%
\pgfpathlineto{\pgfqpoint{0.801138in}{5.491124in}}%
\pgfpathlineto{\pgfqpoint{0.836180in}{5.503128in}}%
\pgfpathlineto{\pgfqpoint{0.851576in}{5.508384in}}%
\pgfpathlineto{\pgfqpoint{0.860772in}{5.511476in}}%
\pgfusepath{stroke}%
\end{pgfscope}%
\begin{pgfscope}%
\pgfpathrectangle{\pgfqpoint{0.766095in}{0.571603in}}{\pgfqpoint{6.973465in}{5.225635in}}%
\pgfusepath{clip}%
\pgfsetbuttcap%
\pgfsetroundjoin%
\pgfsetlinewidth{1.505625pt}%
\definecolor{currentstroke}{rgb}{0.440137,0.811138,0.340967}%
\pgfsetstrokecolor{currentstroke}%
\pgfsetdash{}{0pt}%
\pgfpathmoveto{\pgfqpoint{1.232944in}{5.631101in}}%
\pgfpathlineto{\pgfqpoint{1.256691in}{5.638394in}}%
\pgfpathlineto{\pgfqpoint{1.260904in}{5.639682in}}%
\pgfpathlineto{\pgfqpoint{1.291733in}{5.648963in}}%
\pgfpathlineto{\pgfqpoint{1.326776in}{5.659499in}}%
\pgfpathlineto{\pgfqpoint{1.348253in}{5.665941in}}%
\pgfpathlineto{\pgfqpoint{1.361818in}{5.669948in}}%
\pgfpathlineto{\pgfqpoint{1.396861in}{5.680264in}}%
\pgfpathlineto{\pgfqpoint{1.431903in}{5.690573in}}%
\pgfpathlineto{\pgfqpoint{1.437458in}{5.692201in}}%
\pgfpathlineto{\pgfqpoint{1.466946in}{5.700706in}}%
\pgfpathlineto{\pgfqpoint{1.501988in}{5.710801in}}%
\pgfpathlineto{\pgfqpoint{1.528616in}{5.718460in}}%
\pgfpathlineto{\pgfqpoint{1.537031in}{5.720843in}}%
\pgfpathlineto{\pgfqpoint{1.572073in}{5.730729in}}%
\pgfpathlineto{\pgfqpoint{1.607116in}{5.740610in}}%
\pgfpathlineto{\pgfqpoint{1.621736in}{5.744720in}}%
\pgfpathlineto{\pgfqpoint{1.642158in}{5.750372in}}%
\pgfpathlineto{\pgfqpoint{1.677201in}{5.760049in}}%
\pgfpathlineto{\pgfqpoint{1.712244in}{5.769721in}}%
\pgfpathlineto{\pgfqpoint{1.716819in}{5.770979in}}%
\pgfpathlineto{\pgfqpoint{1.747286in}{5.779224in}}%
\pgfpathlineto{\pgfqpoint{1.782329in}{5.788698in}}%
\pgfpathlineto{\pgfqpoint{1.813943in}{5.797238in}}%
\pgfusepath{stroke}%
\end{pgfscope}%
\begin{pgfscope}%
\pgfpathrectangle{\pgfqpoint{0.766095in}{0.571603in}}{\pgfqpoint{6.973465in}{5.225635in}}%
\pgfusepath{clip}%
\pgfsetbuttcap%
\pgfsetroundjoin%
\pgfsetlinewidth{1.505625pt}%
\definecolor{currentstroke}{rgb}{0.487026,0.823929,0.312321}%
\pgfsetstrokecolor{currentstroke}%
\pgfsetdash{}{0pt}%
\pgfpathmoveto{\pgfqpoint{0.766095in}{5.511594in}}%
\pgfpathlineto{\pgfqpoint{0.801138in}{5.523516in}}%
\pgfpathlineto{\pgfqpoint{0.833892in}{5.534644in}}%
\pgfpathlineto{\pgfqpoint{0.836180in}{5.535409in}}%
\pgfpathlineto{\pgfqpoint{0.871223in}{5.547075in}}%
\pgfpathlineto{\pgfqpoint{0.889934in}{5.553298in}}%
\pgfusepath{stroke}%
\end{pgfscope}%
\begin{pgfscope}%
\pgfpathrectangle{\pgfqpoint{0.766095in}{0.571603in}}{\pgfqpoint{6.973465in}{5.225635in}}%
\pgfusepath{clip}%
\pgfsetbuttcap%
\pgfsetroundjoin%
\pgfsetlinewidth{1.505625pt}%
\definecolor{currentstroke}{rgb}{0.487026,0.823929,0.312321}%
\pgfsetstrokecolor{currentstroke}%
\pgfsetdash{}{0pt}%
\pgfpathmoveto{\pgfqpoint{1.262751in}{5.670880in}}%
\pgfpathlineto{\pgfqpoint{1.291733in}{5.679536in}}%
\pgfpathlineto{\pgfqpoint{1.326776in}{5.689995in}}%
\pgfpathlineto{\pgfqpoint{1.334194in}{5.692201in}}%
\pgfpathlineto{\pgfqpoint{1.361818in}{5.700287in}}%
\pgfpathlineto{\pgfqpoint{1.396861in}{5.710529in}}%
\pgfpathlineto{\pgfqpoint{1.424041in}{5.718460in}}%
\pgfpathlineto{\pgfqpoint{1.431903in}{5.720719in}}%
\pgfpathlineto{\pgfqpoint{1.466946in}{5.730749in}}%
\pgfpathlineto{\pgfqpoint{1.501988in}{5.740773in}}%
\pgfpathlineto{\pgfqpoint{1.515830in}{5.744720in}}%
\pgfpathlineto{\pgfqpoint{1.537031in}{5.750672in}}%
\pgfpathlineto{\pgfqpoint{1.572073in}{5.760489in}}%
\pgfpathlineto{\pgfqpoint{1.607116in}{5.770301in}}%
\pgfpathlineto{\pgfqpoint{1.609547in}{5.770979in}}%
\pgfpathlineto{\pgfqpoint{1.642158in}{5.779930in}}%
\pgfpathlineto{\pgfqpoint{1.677201in}{5.789541in}}%
\pgfpathlineto{\pgfqpoint{1.705302in}{5.797238in}}%
\pgfusepath{stroke}%
\end{pgfscope}%
\begin{pgfscope}%
\pgfpathrectangle{\pgfqpoint{0.766095in}{0.571603in}}{\pgfqpoint{6.973465in}{5.225635in}}%
\pgfusepath{clip}%
\pgfsetbuttcap%
\pgfsetroundjoin%
\pgfsetlinewidth{1.505625pt}%
\definecolor{currentstroke}{rgb}{0.535621,0.835785,0.281908}%
\pgfsetstrokecolor{currentstroke}%
\pgfsetdash{}{0pt}%
\pgfpathmoveto{\pgfqpoint{0.766095in}{5.543347in}}%
\pgfpathlineto{\pgfqpoint{0.801138in}{5.555171in}}%
\pgfpathlineto{\pgfqpoint{0.818180in}{5.560903in}}%
\pgfpathlineto{\pgfqpoint{0.836180in}{5.566866in}}%
\pgfpathlineto{\pgfqpoint{0.871223in}{5.578437in}}%
\pgfpathlineto{\pgfqpoint{0.875919in}{5.579985in}}%
\pgfusepath{stroke}%
\end{pgfscope}%
\begin{pgfscope}%
\pgfpathrectangle{\pgfqpoint{0.766095in}{0.571603in}}{\pgfqpoint{6.973465in}{5.225635in}}%
\pgfusepath{clip}%
\pgfsetbuttcap%
\pgfsetroundjoin%
\pgfsetlinewidth{1.505625pt}%
\definecolor{currentstroke}{rgb}{0.535621,0.835785,0.281908}%
\pgfsetstrokecolor{currentstroke}%
\pgfsetdash{}{0pt}%
\pgfpathmoveto{\pgfqpoint{1.248986in}{5.696763in}}%
\pgfpathlineto{\pgfqpoint{1.256691in}{5.699052in}}%
\pgfpathlineto{\pgfqpoint{1.291733in}{5.709439in}}%
\pgfpathlineto{\pgfqpoint{1.322202in}{5.718460in}}%
\pgfpathlineto{\pgfqpoint{1.326776in}{5.719794in}}%
\pgfpathlineto{\pgfqpoint{1.361818in}{5.729966in}}%
\pgfpathlineto{\pgfqpoint{1.396861in}{5.740132in}}%
\pgfpathlineto{\pgfqpoint{1.412720in}{5.744720in}}%
\pgfpathlineto{\pgfqpoint{1.431903in}{5.750184in}}%
\pgfpathlineto{\pgfqpoint{1.466946in}{5.760140in}}%
\pgfpathlineto{\pgfqpoint{1.501988in}{5.770091in}}%
\pgfpathlineto{\pgfqpoint{1.505130in}{5.770979in}}%
\pgfpathlineto{\pgfqpoint{1.537031in}{5.779860in}}%
\pgfpathlineto{\pgfqpoint{1.572073in}{5.789607in}}%
\pgfpathlineto{\pgfqpoint{1.599551in}{5.797238in}}%
\pgfusepath{stroke}%
\end{pgfscope}%
\begin{pgfscope}%
\pgfpathrectangle{\pgfqpoint{0.766095in}{0.571603in}}{\pgfqpoint{6.973465in}{5.225635in}}%
\pgfusepath{clip}%
\pgfsetbuttcap%
\pgfsetroundjoin%
\pgfsetlinewidth{1.505625pt}%
\definecolor{currentstroke}{rgb}{0.585678,0.846661,0.249897}%
\pgfsetstrokecolor{currentstroke}%
\pgfsetdash{}{0pt}%
\pgfpathmoveto{\pgfqpoint{0.766095in}{5.574395in}}%
\pgfpathlineto{\pgfqpoint{0.801138in}{5.586120in}}%
\pgfpathlineto{\pgfqpoint{0.804270in}{5.587163in}}%
\pgfpathlineto{\pgfqpoint{0.836180in}{5.597628in}}%
\pgfpathlineto{\pgfqpoint{0.863794in}{5.606671in}}%
\pgfusepath{stroke}%
\end{pgfscope}%
\begin{pgfscope}%
\pgfpathrectangle{\pgfqpoint{0.766095in}{0.571603in}}{\pgfqpoint{6.973465in}{5.225635in}}%
\pgfusepath{clip}%
\pgfsetbuttcap%
\pgfsetroundjoin%
\pgfsetlinewidth{1.505625pt}%
\definecolor{currentstroke}{rgb}{0.585678,0.846661,0.249897}%
\pgfsetstrokecolor{currentstroke}%
\pgfsetdash{}{0pt}%
\pgfpathmoveto{\pgfqpoint{1.237114in}{5.722629in}}%
\pgfpathlineto{\pgfqpoint{1.256691in}{5.728392in}}%
\pgfpathlineto{\pgfqpoint{1.291733in}{5.738700in}}%
\pgfpathlineto{\pgfqpoint{1.312248in}{5.744720in}}%
\pgfpathlineto{\pgfqpoint{1.326776in}{5.748918in}}%
\pgfpathlineto{\pgfqpoint{1.361818in}{5.759013in}}%
\pgfpathlineto{\pgfqpoint{1.396861in}{5.769101in}}%
\pgfpathlineto{\pgfqpoint{1.403409in}{5.770979in}}%
\pgfpathlineto{\pgfqpoint{1.431903in}{5.779025in}}%
\pgfpathlineto{\pgfqpoint{1.466946in}{5.788907in}}%
\pgfpathlineto{\pgfqpoint{1.496527in}{5.797238in}}%
\pgfusepath{stroke}%
\end{pgfscope}%
\begin{pgfscope}%
\pgfpathrectangle{\pgfqpoint{0.766095in}{0.571603in}}{\pgfqpoint{6.973465in}{5.225635in}}%
\pgfusepath{clip}%
\pgfsetbuttcap%
\pgfsetroundjoin%
\pgfsetlinewidth{1.505625pt}%
\definecolor{currentstroke}{rgb}{0.636902,0.856542,0.216620}%
\pgfsetstrokecolor{currentstroke}%
\pgfsetdash{}{0pt}%
\pgfpathmoveto{\pgfqpoint{0.766095in}{5.604765in}}%
\pgfpathlineto{\pgfqpoint{0.792219in}{5.613422in}}%
\pgfpathlineto{\pgfqpoint{0.801138in}{5.616333in}}%
\pgfpathlineto{\pgfqpoint{0.836180in}{5.627725in}}%
\pgfpathlineto{\pgfqpoint{0.859998in}{5.635459in}}%
\pgfusepath{stroke}%
\end{pgfscope}%
\begin{pgfscope}%
\pgfpathrectangle{\pgfqpoint{0.766095in}{0.571603in}}{\pgfqpoint{6.973465in}{5.225635in}}%
\pgfusepath{clip}%
\pgfsetbuttcap%
\pgfsetroundjoin%
\pgfsetlinewidth{1.505625pt}%
\definecolor{currentstroke}{rgb}{0.636902,0.856542,0.216620}%
\pgfsetstrokecolor{currentstroke}%
\pgfsetdash{}{0pt}%
\pgfpathmoveto{\pgfqpoint{1.233634in}{5.750385in}}%
\pgfpathlineto{\pgfqpoint{1.256691in}{5.757118in}}%
\pgfpathlineto{\pgfqpoint{1.291733in}{5.767344in}}%
\pgfpathlineto{\pgfqpoint{1.304232in}{5.770979in}}%
\pgfpathlineto{\pgfqpoint{1.326776in}{5.777437in}}%
\pgfpathlineto{\pgfqpoint{1.361818in}{5.787453in}}%
\pgfpathlineto{\pgfqpoint{1.396080in}{5.797238in}}%
\pgfusepath{stroke}%
\end{pgfscope}%
\begin{pgfscope}%
\pgfpathrectangle{\pgfqpoint{0.766095in}{0.571603in}}{\pgfqpoint{6.973465in}{5.225635in}}%
\pgfusepath{clip}%
\pgfsetbuttcap%
\pgfsetroundjoin%
\pgfsetlinewidth{1.505625pt}%
\definecolor{currentstroke}{rgb}{0.688944,0.865448,0.182725}%
\pgfsetstrokecolor{currentstroke}%
\pgfsetdash{}{0pt}%
\pgfpathmoveto{\pgfqpoint{0.766095in}{5.634485in}}%
\pgfpathlineto{\pgfqpoint{0.781934in}{5.639682in}}%
\pgfpathlineto{\pgfqpoint{0.801138in}{5.645889in}}%
\pgfpathlineto{\pgfqpoint{0.836180in}{5.657182in}}%
\pgfpathlineto{\pgfqpoint{0.845220in}{5.660090in}}%
\pgfusepath{stroke}%
\end{pgfscope}%
\begin{pgfscope}%
\pgfpathrectangle{\pgfqpoint{0.766095in}{0.571603in}}{\pgfqpoint{6.973465in}{5.225635in}}%
\pgfusepath{clip}%
\pgfsetbuttcap%
\pgfsetroundjoin%
\pgfsetlinewidth{1.505625pt}%
\definecolor{currentstroke}{rgb}{0.688944,0.865448,0.182725}%
\pgfsetstrokecolor{currentstroke}%
\pgfsetdash{}{0pt}%
\pgfpathmoveto{\pgfqpoint{1.219058in}{5.774350in}}%
\pgfpathlineto{\pgfqpoint{1.221648in}{5.775103in}}%
\pgfpathlineto{\pgfqpoint{1.256691in}{5.785254in}}%
\pgfpathlineto{\pgfqpoint{1.291733in}{5.795397in}}%
\pgfpathlineto{\pgfqpoint{1.298121in}{5.797238in}}%
\pgfusepath{stroke}%
\end{pgfscope}%
\begin{pgfscope}%
\pgfpathrectangle{\pgfqpoint{0.766095in}{0.571603in}}{\pgfqpoint{6.973465in}{5.225635in}}%
\pgfusepath{clip}%
\pgfsetbuttcap%
\pgfsetroundjoin%
\pgfsetlinewidth{1.505625pt}%
\definecolor{currentstroke}{rgb}{0.741388,0.873449,0.149561}%
\pgfsetstrokecolor{currentstroke}%
\pgfsetdash{}{0pt}%
\pgfpathmoveto{\pgfqpoint{0.766095in}{5.663580in}}%
\pgfpathlineto{\pgfqpoint{0.773362in}{5.665941in}}%
\pgfpathlineto{\pgfqpoint{0.789855in}{5.671220in}}%
\pgfusepath{stroke}%
\end{pgfscope}%
\begin{pgfscope}%
\pgfpathrectangle{\pgfqpoint{0.766095in}{0.571603in}}{\pgfqpoint{6.973465in}{5.225635in}}%
\pgfusepath{clip}%
\pgfsetbuttcap%
\pgfsetroundjoin%
\pgfsetlinewidth{1.505625pt}%
\definecolor{currentstroke}{rgb}{0.741388,0.873449,0.149561}%
\pgfsetstrokecolor{currentstroke}%
\pgfsetdash{}{0pt}%
\pgfpathmoveto{\pgfqpoint{1.163583in}{5.785843in}}%
\pgfpathlineto{\pgfqpoint{1.186605in}{5.792594in}}%
\pgfpathlineto{\pgfqpoint{1.202488in}{5.797238in}}%
\pgfusepath{stroke}%
\end{pgfscope}%
\begin{pgfscope}%
\pgfpathrectangle{\pgfqpoint{0.766095in}{0.571603in}}{\pgfqpoint{6.973465in}{5.225635in}}%
\pgfusepath{clip}%
\pgfsetbuttcap%
\pgfsetroundjoin%
\pgfsetlinewidth{1.505625pt}%
\definecolor{currentstroke}{rgb}{0.845561,0.887322,0.099702}%
\pgfsetstrokecolor{currentstroke}%
\pgfsetdash{}{0pt}%
\pgfpathmoveto{\pgfqpoint{0.766095in}{5.719969in}}%
\pgfpathlineto{\pgfqpoint{0.801138in}{5.730976in}}%
\pgfpathlineto{\pgfqpoint{0.836180in}{5.741971in}}%
\pgfpathlineto{\pgfqpoint{0.844974in}{5.744720in}}%
\pgfpathlineto{\pgfqpoint{0.871223in}{5.752801in}}%
\pgfpathlineto{\pgfqpoint{0.906265in}{5.763568in}}%
\pgfpathlineto{\pgfqpoint{0.930436in}{5.770979in}}%
\pgfpathlineto{\pgfqpoint{0.941308in}{5.774263in}}%
\pgfpathlineto{\pgfqpoint{0.976350in}{5.784808in}}%
\pgfpathlineto{\pgfqpoint{1.011393in}{5.795344in}}%
\pgfpathlineto{\pgfqpoint{1.017721in}{5.797238in}}%
\pgfusepath{stroke}%
\end{pgfscope}%
\begin{pgfscope}%
\pgfpathrectangle{\pgfqpoint{0.766095in}{0.571603in}}{\pgfqpoint{6.973465in}{5.225635in}}%
\pgfusepath{clip}%
\pgfsetbuttcap%
\pgfsetroundjoin%
\pgfsetlinewidth{1.505625pt}%
\definecolor{currentstroke}{rgb}{0.896320,0.893616,0.096335}%
\pgfsetstrokecolor{currentstroke}%
\pgfsetdash{}{0pt}%
\pgfpathmoveto{\pgfqpoint{0.766095in}{5.747318in}}%
\pgfpathlineto{\pgfqpoint{0.801138in}{5.758225in}}%
\pgfpathlineto{\pgfqpoint{0.836180in}{5.769119in}}%
\pgfpathlineto{\pgfqpoint{0.842187in}{5.770979in}}%
\pgfpathlineto{\pgfqpoint{0.871223in}{5.779835in}}%
\pgfpathlineto{\pgfqpoint{0.906265in}{5.790505in}}%
\pgfpathlineto{\pgfqpoint{0.928432in}{5.797238in}}%
\pgfusepath{stroke}%
\end{pgfscope}%
\begin{pgfscope}%
\pgfpathrectangle{\pgfqpoint{0.766095in}{0.571603in}}{\pgfqpoint{6.973465in}{5.225635in}}%
\pgfusepath{clip}%
\pgfsetbuttcap%
\pgfsetroundjoin%
\pgfsetlinewidth{1.505625pt}%
\definecolor{currentstroke}{rgb}{0.945636,0.899815,0.112838}%
\pgfsetstrokecolor{currentstroke}%
\pgfsetdash{}{0pt}%
\pgfpathmoveto{\pgfqpoint{0.766095in}{5.774148in}}%
\pgfpathlineto{\pgfqpoint{0.801138in}{5.784954in}}%
\pgfpathlineto{\pgfqpoint{0.836180in}{5.795748in}}%
\pgfpathlineto{\pgfqpoint{0.841040in}{5.797238in}}%
\pgfusepath{stroke}%
\end{pgfscope}%
\begin{pgfscope}%
\pgfpathrectangle{\pgfqpoint{0.766095in}{0.571603in}}{\pgfqpoint{6.973465in}{5.225635in}}%
\pgfusepath{clip}%
\pgfsetrectcap%
\pgfsetroundjoin%
\pgfsetlinewidth{1.505625pt}%
\definecolor{currentstroke}{rgb}{0.000000,0.000000,0.000000}%
\pgfsetstrokecolor{currentstroke}%
\pgfsetdash{}{0pt}%
\pgfpathmoveto{\pgfqpoint{6.577316in}{3.706984in}}%
\pgfpathlineto{\pgfqpoint{4.542738in}{5.098057in}}%
\pgfpathlineto{\pgfqpoint{3.038495in}{3.912818in}}%
\pgfpathlineto{\pgfqpoint{2.740398in}{2.889953in}}%
\pgfpathlineto{\pgfqpoint{2.899488in}{1.590976in}}%
\pgfpathlineto{\pgfqpoint{3.714898in}{1.277449in}}%
\pgfpathlineto{\pgfqpoint{3.972874in}{1.117929in}}%
\pgfusepath{stroke}%
\end{pgfscope}%
\begin{pgfscope}%
\pgfsetrectcap%
\pgfsetmiterjoin%
\pgfsetlinewidth{0.803000pt}%
\definecolor{currentstroke}{rgb}{0.000000,0.000000,0.000000}%
\pgfsetstrokecolor{currentstroke}%
\pgfsetdash{}{0pt}%
\pgfpathmoveto{\pgfqpoint{0.766095in}{0.571603in}}%
\pgfpathlineto{\pgfqpoint{0.766095in}{5.797238in}}%
\pgfusepath{stroke}%
\end{pgfscope}%
\begin{pgfscope}%
\pgfsetrectcap%
\pgfsetmiterjoin%
\pgfsetlinewidth{0.803000pt}%
\definecolor{currentstroke}{rgb}{0.000000,0.000000,0.000000}%
\pgfsetstrokecolor{currentstroke}%
\pgfsetdash{}{0pt}%
\pgfpathmoveto{\pgfqpoint{7.739560in}{0.571603in}}%
\pgfpathlineto{\pgfqpoint{7.739560in}{5.797238in}}%
\pgfusepath{stroke}%
\end{pgfscope}%
\begin{pgfscope}%
\pgfsetrectcap%
\pgfsetmiterjoin%
\pgfsetlinewidth{0.803000pt}%
\definecolor{currentstroke}{rgb}{0.000000,0.000000,0.000000}%
\pgfsetstrokecolor{currentstroke}%
\pgfsetdash{}{0pt}%
\pgfpathmoveto{\pgfqpoint{0.766095in}{0.571603in}}%
\pgfpathlineto{\pgfqpoint{7.739560in}{0.571603in}}%
\pgfusepath{stroke}%
\end{pgfscope}%
\begin{pgfscope}%
\pgfsetrectcap%
\pgfsetmiterjoin%
\pgfsetlinewidth{0.803000pt}%
\definecolor{currentstroke}{rgb}{0.000000,0.000000,0.000000}%
\pgfsetstrokecolor{currentstroke}%
\pgfsetdash{}{0pt}%
\pgfpathmoveto{\pgfqpoint{0.766095in}{5.797238in}}%
\pgfpathlineto{\pgfqpoint{7.739560in}{5.797238in}}%
\pgfusepath{stroke}%
\end{pgfscope}%
\begin{pgfscope}%
\definecolor{textcolor}{rgb}{0.273809,0.031497,0.358853}%
\pgfsetstrokecolor{textcolor}%
\pgfsetfillcolor{textcolor}%
\pgftext[x=4.702045in, y=1.283617in, left, base,rotate=322.628138]{\color{textcolor}\sffamily\fontsize{8.000000}{9.600000}\selectfont 1.8}%
\end{pgfscope}%
\begin{pgfscope}%
\definecolor{textcolor}{rgb}{0.278791,0.062145,0.386592}%
\pgfsetstrokecolor{textcolor}%
\pgfsetfillcolor{textcolor}%
\pgftext[x=4.150196in, y=2.235001in, left, base,rotate=316.384801]{\color{textcolor}\sffamily\fontsize{8.000000}{9.600000}\selectfont 2.1}%
\end{pgfscope}%
\begin{pgfscope}%
\definecolor{textcolor}{rgb}{0.281924,0.089666,0.412415}%
\pgfsetstrokecolor{textcolor}%
\pgfsetfillcolor{textcolor}%
\pgftext[x=4.257295in, y=2.697860in, left, base,rotate=307.261522]{\color{textcolor}\sffamily\fontsize{8.000000}{9.600000}\selectfont 2.4}%
\end{pgfscope}%
\begin{pgfscope}%
\definecolor{textcolor}{rgb}{0.283197,0.115680,0.436115}%
\pgfsetstrokecolor{textcolor}%
\pgfsetfillcolor{textcolor}%
\pgftext[x=6.527717in, y=0.843302in, left, base,rotate=326.225659]{\color{textcolor}\sffamily\fontsize{8.000000}{9.600000}\selectfont 2.7}%
\end{pgfscope}%
\begin{pgfscope}%
\definecolor{textcolor}{rgb}{0.282623,0.140926,0.457517}%
\pgfsetstrokecolor{textcolor}%
\pgfsetfillcolor{textcolor}%
\pgftext[x=6.423513in, y=1.119805in, left, base,rotate=324.131978]{\color{textcolor}\sffamily\fontsize{8.000000}{9.600000}\selectfont 3.0}%
\end{pgfscope}%
\begin{pgfscope}%
\definecolor{textcolor}{rgb}{0.280255,0.165693,0.476498}%
\pgfsetstrokecolor{textcolor}%
\pgfsetfillcolor{textcolor}%
\pgftext[x=5.100005in, y=3.492796in, left, base,rotate=280.617184]{\color{textcolor}\sffamily\fontsize{8.000000}{9.600000}\selectfont 3.3}%
\end{pgfscope}%
\begin{pgfscope}%
\definecolor{textcolor}{rgb}{0.276194,0.190074,0.493001}%
\pgfsetstrokecolor{textcolor}%
\pgfsetfillcolor{textcolor}%
\pgftext[x=7.380160in, y=0.802667in, left, base,rotate=327.896753]{\color{textcolor}\sffamily\fontsize{8.000000}{9.600000}\selectfont 3.6}%
\end{pgfscope}%
\begin{pgfscope}%
\definecolor{textcolor}{rgb}{0.270595,0.214069,0.507052}%
\pgfsetstrokecolor{textcolor}%
\pgfsetfillcolor{textcolor}%
\pgftext[x=1.586327in, y=0.697153in, left, base,rotate=327.208004]{\color{textcolor}\sffamily\fontsize{8.000000}{9.600000}\selectfont 3.9}%
\end{pgfscope}%
\begin{pgfscope}%
\definecolor{textcolor}{rgb}{0.270595,0.214069,0.507052}%
\pgfsetstrokecolor{textcolor}%
\pgfsetfillcolor{textcolor}%
\pgftext[x=5.766362in, y=3.632778in, left, base,rotate=86.716516]{\color{textcolor}\sffamily\fontsize{8.000000}{9.600000}\selectfont 3.9}%
\end{pgfscope}%
\begin{pgfscope}%
\definecolor{textcolor}{rgb}{0.263663,0.237631,0.518762}%
\pgfsetstrokecolor{textcolor}%
\pgfsetfillcolor{textcolor}%
\pgftext[x=1.125347in, y=0.918374in, left, base,rotate=321.079533]{\color{textcolor}\sffamily\fontsize{8.000000}{9.600000}\selectfont 4.2}%
\end{pgfscope}%
\begin{pgfscope}%
\definecolor{textcolor}{rgb}{0.263663,0.237631,0.518762}%
\pgfsetstrokecolor{textcolor}%
\pgfsetfillcolor{textcolor}%
\pgftext[x=7.299445in, y=1.152177in, left, base,rotate=324.415799]{\color{textcolor}\sffamily\fontsize{8.000000}{9.600000}\selectfont 4.2}%
\end{pgfscope}%
\begin{pgfscope}%
\definecolor{textcolor}{rgb}{0.255645,0.260703,0.528312}%
\pgfsetstrokecolor{textcolor}%
\pgfsetfillcolor{textcolor}%
\pgftext[x=1.247935in, y=0.700502in, left, base,rotate=326.083584]{\color{textcolor}\sffamily\fontsize{8.000000}{9.600000}\selectfont 4.5}%
\end{pgfscope}%
\begin{pgfscope}%
\definecolor{textcolor}{rgb}{0.255645,0.260703,0.528312}%
\pgfsetstrokecolor{textcolor}%
\pgfsetfillcolor{textcolor}%
\pgftext[x=6.309223in, y=3.813408in, left, base,rotate=75.964698]{\color{textcolor}\sffamily\fontsize{8.000000}{9.600000}\selectfont 4.5}%
\end{pgfscope}%
\begin{pgfscope}%
\definecolor{textcolor}{rgb}{0.246811,0.283237,0.535941}%
\pgfsetstrokecolor{textcolor}%
\pgfsetfillcolor{textcolor}%
\pgftext[x=0.818889in, y=0.922450in, left, base,rotate=319.985225]{\color{textcolor}\sffamily\fontsize{8.000000}{9.600000}\selectfont 4.8}%
\end{pgfscope}%
\begin{pgfscope}%
\definecolor{textcolor}{rgb}{0.246811,0.283237,0.535941}%
\pgfsetstrokecolor{textcolor}%
\pgfsetfillcolor{textcolor}%
\pgftext[x=3.454827in, y=4.836525in, left, base,rotate=12.493939]{\color{textcolor}\sffamily\fontsize{8.000000}{9.600000}\selectfont 4.8}%
\end{pgfscope}%
\begin{pgfscope}%
\definecolor{textcolor}{rgb}{0.237441,0.305202,0.541921}%
\pgfsetstrokecolor{textcolor}%
\pgfsetfillcolor{textcolor}%
\pgftext[x=0.956400in, y=0.698807in, left, base,rotate=325.180238]{\color{textcolor}\sffamily\fontsize{8.000000}{9.600000}\selectfont 5.1}%
\end{pgfscope}%
\begin{pgfscope}%
\definecolor{textcolor}{rgb}{0.237441,0.305202,0.541921}%
\pgfsetstrokecolor{textcolor}%
\pgfsetfillcolor{textcolor}%
\pgftext[x=6.884341in, y=4.128324in, left, base,rotate=67.391368]{\color{textcolor}\sffamily\fontsize{8.000000}{9.600000}\selectfont 5.1}%
\end{pgfscope}%
\begin{pgfscope}%
\definecolor{textcolor}{rgb}{0.227802,0.326594,0.546532}%
\pgfsetstrokecolor{textcolor}%
\pgfsetfillcolor{textcolor}%
\pgftext[x=0.816468in, y=0.702225in, left, base,rotate=324.648301]{\color{textcolor}\sffamily\fontsize{8.000000}{9.600000}\selectfont 5.4}%
\end{pgfscope}%
\begin{pgfscope}%
\definecolor{textcolor}{rgb}{0.227802,0.326594,0.546532}%
\pgfsetstrokecolor{textcolor}%
\pgfsetfillcolor{textcolor}%
\pgftext[x=3.594565in, y=5.033950in, left, base,rotate=11.985085]{\color{textcolor}\sffamily\fontsize{8.000000}{9.600000}\selectfont 5.4}%
\end{pgfscope}%
\begin{pgfscope}%
\definecolor{textcolor}{rgb}{0.218130,0.347432,0.550038}%
\pgfsetstrokecolor{textcolor}%
\pgfsetfillcolor{textcolor}%
\pgftext[x=7.300292in, y=4.103215in, left, base,rotate=61.218624]{\color{textcolor}\sffamily\fontsize{8.000000}{9.600000}\selectfont 5.7}%
\end{pgfscope}%
\begin{pgfscope}%
\definecolor{textcolor}{rgb}{0.208623,0.367752,0.552675}%
\pgfsetstrokecolor{textcolor}%
\pgfsetfillcolor{textcolor}%
\pgftext[x=5.114547in, y=5.440726in, left, base,rotate=7.140731]{\color{textcolor}\sffamily\fontsize{8.000000}{9.600000}\selectfont 6.0}%
\end{pgfscope}%
\begin{pgfscope}%
\definecolor{textcolor}{rgb}{0.208623,0.367752,0.552675}%
\pgfsetstrokecolor{textcolor}%
\pgfsetfillcolor{textcolor}%
\pgftext[x=7.624086in, y=4.314399in, left, base,rotate=57.774471]{\color{textcolor}\sffamily\fontsize{8.000000}{9.600000}\selectfont 6.0}%
\end{pgfscope}%
\begin{pgfscope}%
\definecolor{textcolor}{rgb}{0.199430,0.387607,0.554642}%
\pgfsetstrokecolor{textcolor}%
\pgfsetfillcolor{textcolor}%
\pgftext[x=3.699471in, y=5.261531in, left, base,rotate=11.720729]{\color{textcolor}\sffamily\fontsize{8.000000}{9.600000}\selectfont 6.3}%
\end{pgfscope}%
\begin{pgfscope}%
\definecolor{textcolor}{rgb}{0.199430,0.387607,0.554642}%
\pgfsetstrokecolor{textcolor}%
\pgfsetfillcolor{textcolor}%
\pgftext[x=7.620533in, y=3.998965in, left, base,rotate=58.670370]{\color{textcolor}\sffamily\fontsize{8.000000}{9.600000}\selectfont 6.3}%
\end{pgfscope}%
\begin{pgfscope}%
\definecolor{textcolor}{rgb}{0.190631,0.407061,0.556089}%
\pgfsetstrokecolor{textcolor}%
\pgfsetfillcolor{textcolor}%
\pgftext[x=7.651615in, y=3.735265in, left, base,rotate=62.455786]{\color{textcolor}\sffamily\fontsize{8.000000}{9.600000}\selectfont 6.6}%
\end{pgfscope}%
\begin{pgfscope}%
\definecolor{textcolor}{rgb}{0.190631,0.407061,0.556089}%
\pgfsetstrokecolor{textcolor}%
\pgfsetfillcolor{textcolor}%
\pgftext[x=4.188915in, y=5.419138in, left, base,rotate=10.316214]{\color{textcolor}\sffamily\fontsize{8.000000}{9.600000}\selectfont 6.6}%
\end{pgfscope}%
\begin{pgfscope}%
\definecolor{textcolor}{rgb}{0.182256,0.426184,0.557120}%
\pgfsetstrokecolor{textcolor}%
\pgfsetfillcolor{textcolor}%
\pgftext[x=7.677813in, y=3.392887in, left, base,rotate=71.042093]{\color{textcolor}\sffamily\fontsize{8.000000}{9.600000}\selectfont 6.9}%
\end{pgfscope}%
\begin{pgfscope}%
\definecolor{textcolor}{rgb}{0.182256,0.426184,0.557120}%
\pgfsetstrokecolor{textcolor}%
\pgfsetfillcolor{textcolor}%
\pgftext[x=5.238243in, y=5.648118in, left, base,rotate=7.789317]{\color{textcolor}\sffamily\fontsize{8.000000}{9.600000}\selectfont 6.9}%
\end{pgfscope}%
\begin{pgfscope}%
\definecolor{textcolor}{rgb}{0.174274,0.445044,0.557792}%
\pgfsetstrokecolor{textcolor}%
\pgfsetfillcolor{textcolor}%
\pgftext[x=4.713657in, y=5.622206in, left, base,rotate=9.179416]{\color{textcolor}\sffamily\fontsize{8.000000}{9.600000}\selectfont 7.2}%
\end{pgfscope}%
\begin{pgfscope}%
\definecolor{textcolor}{rgb}{0.166617,0.463708,0.558119}%
\pgfsetstrokecolor{textcolor}%
\pgfsetfillcolor{textcolor}%
\pgftext[x=3.804303in, y=5.503969in, left, base,rotate=11.365855]{\color{textcolor}\sffamily\fontsize{8.000000}{9.600000}\selectfont 7.5}%
\end{pgfscope}%
\begin{pgfscope}%
\definecolor{textcolor}{rgb}{0.159194,0.482237,0.558073}%
\pgfsetstrokecolor{textcolor}%
\pgfsetfillcolor{textcolor}%
\pgftext[x=4.293978in, y=5.647957in, left, base,rotate=10.235895]{\color{textcolor}\sffamily\fontsize{8.000000}{9.600000}\selectfont 7.8}%
\end{pgfscope}%
\begin{pgfscope}%
\definecolor{textcolor}{rgb}{0.151918,0.500685,0.557587}%
\pgfsetstrokecolor{textcolor}%
\pgfsetfillcolor{textcolor}%
\pgftext[x=3.384620in, y=5.508649in, left, base,rotate=12.350386]{\color{textcolor}\sffamily\fontsize{8.000000}{9.600000}\selectfont 8.1}%
\end{pgfscope}%
\begin{pgfscope}%
\definecolor{textcolor}{rgb}{0.144759,0.519093,0.556572}%
\pgfsetstrokecolor{textcolor}%
\pgfsetfillcolor{textcolor}%
\pgftext[x=2.930098in, y=5.446749in, left, base,rotate=13.540323]{\color{textcolor}\sffamily\fontsize{8.000000}{9.600000}\selectfont 8.4}%
\end{pgfscope}%
\begin{pgfscope}%
\definecolor{textcolor}{rgb}{0.137770,0.537492,0.554906}%
\pgfsetstrokecolor{textcolor}%
\pgfsetfillcolor{textcolor}%
\pgftext[x=2.510663in, y=5.382950in, left, base,rotate=14.741173]{\color{textcolor}\sffamily\fontsize{8.000000}{9.600000}\selectfont 8.7}%
\end{pgfscope}%
\begin{pgfscope}%
\definecolor{textcolor}{rgb}{0.131172,0.555899,0.552459}%
\pgfsetstrokecolor{textcolor}%
\pgfsetfillcolor{textcolor}%
\pgftext[x=2.091406in, y=5.308479in, left, base,rotate=16.094687]{\color{textcolor}\sffamily\fontsize{8.000000}{9.600000}\selectfont 9.0}%
\end{pgfscope}%
\begin{pgfscope}%
\definecolor{textcolor}{rgb}{0.125394,0.574318,0.549086}%
\pgfsetstrokecolor{textcolor}%
\pgfsetfillcolor{textcolor}%
\pgftext[x=1.672401in, y=5.222138in, left, base,rotate=17.663615]{\color{textcolor}\sffamily\fontsize{8.000000}{9.600000}\selectfont 9.3}%
\end{pgfscope}%
\begin{pgfscope}%
\definecolor{textcolor}{rgb}{0.121148,0.592739,0.544641}%
\pgfsetstrokecolor{textcolor}%
\pgfsetfillcolor{textcolor}%
\pgftext[x=1.288626in, y=5.134638in, left, base,rotate=19.362015]{\color{textcolor}\sffamily\fontsize{8.000000}{9.600000}\selectfont 9.6}%
\end{pgfscope}%
\begin{pgfscope}%
\definecolor{textcolor}{rgb}{0.119423,0.611141,0.538982}%
\pgfsetstrokecolor{textcolor}%
\pgfsetfillcolor{textcolor}%
\pgftext[x=0.870461in, y=5.019321in, left, base,rotate=21.615220]{\color{textcolor}\sffamily\fontsize{8.000000}{9.600000}\selectfont 9.9}%
\end{pgfscope}%
\begin{pgfscope}%
\definecolor{textcolor}{rgb}{0.121380,0.629492,0.531973}%
\pgfsetstrokecolor{textcolor}%
\pgfsetfillcolor{textcolor}%
\pgftext[x=2.720324in, y=5.638324in, left, base,rotate=13.346775]{\color{textcolor}\sffamily\fontsize{8.000000}{9.600000}\selectfont 10.2}%
\end{pgfscope}%
\begin{pgfscope}%
\definecolor{textcolor}{rgb}{0.128087,0.647749,0.523491}%
\pgfsetstrokecolor{textcolor}%
\pgfsetfillcolor{textcolor}%
\pgftext[x=2.126298in, y=5.521010in, left, base,rotate=14.967894]{\color{textcolor}\sffamily\fontsize{8.000000}{9.600000}\selectfont 10.5}%
\end{pgfscope}%
\begin{pgfscope}%
\definecolor{textcolor}{rgb}{0.140210,0.665859,0.513427}%
\pgfsetstrokecolor{textcolor}%
\pgfsetfillcolor{textcolor}%
\pgftext[x=2.126128in, y=5.557641in, left, base,rotate=14.809432]{\color{textcolor}\sffamily\fontsize{8.000000}{9.600000}\selectfont 10.8}%
\end{pgfscope}%
\begin{pgfscope}%
\definecolor{textcolor}{rgb}{0.157851,0.683765,0.501686}%
\pgfsetstrokecolor{textcolor}%
\pgfsetfillcolor{textcolor}%
\pgftext[x=2.125985in, y=5.593172in, left, base,rotate=14.675619]{\color{textcolor}\sffamily\fontsize{8.000000}{9.600000}\selectfont 11.1}%
\end{pgfscope}%
\begin{pgfscope}%
\definecolor{textcolor}{rgb}{0.180653,0.701402,0.488189}%
\pgfsetstrokecolor{textcolor}%
\pgfsetfillcolor{textcolor}%
\pgftext[x=2.139852in, y=5.631560in, left, base,rotate=14.495923]{\color{textcolor}\sffamily\fontsize{8.000000}{9.600000}\selectfont 11.4}%
\end{pgfscope}%
\begin{pgfscope}%
\definecolor{textcolor}{rgb}{0.208030,0.718701,0.472873}%
\pgfsetstrokecolor{textcolor}%
\pgfsetfillcolor{textcolor}%
\pgftext[x=2.125696in, y=5.661467in, left, base,rotate=14.403108]{\color{textcolor}\sffamily\fontsize{8.000000}{9.600000}\selectfont 11.7}%
\end{pgfscope}%
\begin{pgfscope}%
\definecolor{textcolor}{rgb}{0.239374,0.735588,0.455688}%
\pgfsetstrokecolor{textcolor}%
\pgfsetfillcolor{textcolor}%
\pgftext[x=0.939233in, y=5.334458in, left, base,rotate=18.814929]{\color{textcolor}\sffamily\fontsize{8.000000}{9.600000}\selectfont 12.0}%
\end{pgfscope}%
\begin{pgfscope}%
\definecolor{textcolor}{rgb}{0.274149,0.751988,0.436601}%
\pgfsetstrokecolor{textcolor}%
\pgfsetfillcolor{textcolor}%
\pgftext[x=1.531760in, y=5.561279in, left, base,rotate=16.052212]{\color{textcolor}\sffamily\fontsize{8.000000}{9.600000}\selectfont 12.3}%
\end{pgfscope}%
\begin{pgfscope}%
\definecolor{textcolor}{rgb}{0.311925,0.767822,0.415586}%
\pgfsetstrokecolor{textcolor}%
\pgfsetfillcolor{textcolor}%
\pgftext[x=1.531584in, y=5.594089in, left, base,rotate=15.894007]{\color{textcolor}\sffamily\fontsize{8.000000}{9.600000}\selectfont 12.6}%
\end{pgfscope}%
\begin{pgfscope}%
\definecolor{textcolor}{rgb}{0.352360,0.783011,0.392636}%
\pgfsetstrokecolor{textcolor}%
\pgfsetfillcolor{textcolor}%
\pgftext[x=1.541972in, y=5.629217in, left, base,rotate=15.700752]{\color{textcolor}\sffamily\fontsize{8.000000}{9.600000}\selectfont 12.9}%
\end{pgfscope}%
\begin{pgfscope}%
\definecolor{textcolor}{rgb}{0.395174,0.797475,0.367757}%
\pgfsetstrokecolor{textcolor}%
\pgfsetfillcolor{textcolor}%
\pgftext[x=1.531247in, y=5.657388in, left, base,rotate=15.587761]{\color{textcolor}\sffamily\fontsize{8.000000}{9.600000}\selectfont 13.2}%
\end{pgfscope}%
\begin{pgfscope}%
\definecolor{textcolor}{rgb}{0.440137,0.811138,0.340967}%
\pgfsetstrokecolor{textcolor}%
\pgfsetfillcolor{textcolor}%
\pgftext[x=0.937945in, y=5.505756in, left, base,rotate=17.727303]{\color{textcolor}\sffamily\fontsize{8.000000}{9.600000}\selectfont 13.5}%
\end{pgfscope}%
\begin{pgfscope}%
\definecolor{textcolor}{rgb}{0.487026,0.823929,0.312321}%
\pgfsetstrokecolor{textcolor}%
\pgfsetfillcolor{textcolor}%
\pgftext[x=0.967060in, y=5.547174in, left, base,rotate=17.403797]{\color{textcolor}\sffamily\fontsize{8.000000}{9.600000}\selectfont 13.8}%
\end{pgfscope}%
\begin{pgfscope}%
\definecolor{textcolor}{rgb}{0.535621,0.835785,0.281908}%
\pgfsetstrokecolor{textcolor}%
\pgfsetfillcolor{textcolor}%
\pgftext[x=0.953040in, y=5.573665in, left, base,rotate=17.281728]{\color{textcolor}\sffamily\fontsize{8.000000}{9.600000}\selectfont 14.1}%
\end{pgfscope}%
\begin{pgfscope}%
\definecolor{textcolor}{rgb}{0.585678,0.846661,0.249897}%
\pgfsetstrokecolor{textcolor}%
\pgfsetfillcolor{textcolor}%
\pgftext[x=0.940906in, y=5.600163in, left, base,rotate=17.156542]{\color{textcolor}\sffamily\fontsize{8.000000}{9.600000}\selectfont 14.4}%
\end{pgfscope}%
\begin{pgfscope}%
\definecolor{textcolor}{rgb}{0.636902,0.856542,0.216620}%
\pgfsetstrokecolor{textcolor}%
\pgfsetfillcolor{textcolor}%
\pgftext[x=0.937108in, y=5.628688in, left, base,rotate=17.003150]{\color{textcolor}\sffamily\fontsize{8.000000}{9.600000}\selectfont 14.7}%
\end{pgfscope}%
\begin{pgfscope}%
\definecolor{textcolor}{rgb}{0.688944,0.865448,0.182725}%
\pgfsetstrokecolor{textcolor}%
\pgfsetfillcolor{textcolor}%
\pgftext[x=0.922317in, y=5.653176in, left, base,rotate=16.897443]{\color{textcolor}\sffamily\fontsize{8.000000}{9.600000}\selectfont 15.0}%
\end{pgfscope}%
\begin{pgfscope}%
\definecolor{textcolor}{rgb}{0.741388,0.873449,0.149561}%
\pgfsetstrokecolor{textcolor}%
\pgfsetfillcolor{textcolor}%
\pgftext[x=0.866976in, y=5.664339in, left, base,rotate=16.962219]{\color{textcolor}\sffamily\fontsize{8.000000}{9.600000}\selectfont 15.3}%
\end{pgfscope}%
\begin{pgfscope}%
\definecolor{textcolor}{rgb}{0.793760,0.880678,0.120005}%
\pgfsetstrokecolor{textcolor}%
\pgfsetfillcolor{textcolor}%
\pgftext[x=0.824752in, y=5.679303in, left, base,rotate=16.966786]{\color{textcolor}\sffamily\fontsize{8.000000}{9.600000}\selectfont 15.6}%
\end{pgfscope}%
\end{pgfpicture}%
\makeatother%
\endgroup%
}
    \caption{Your figure caption}
    \label{fig:newton_contour}
\end{figure}

\begin{figure}[H]
    \centering
    \resizebox{1\textwidth}{!}{
    %% Creator: Matplotlib, PGF backend
%%
%% To include the figure in your LaTeX document, write
%%   \input{<filename>.pgf}
%%
%% Make sure the required packages are loaded in your preamble
%%   \usepackage{pgf}
%%
%% Also ensure that all the required font packages are loaded; for instance,
%% the lmodern package is sometimes necessary when using math font.
%%   \usepackage{lmodern}
%%
%% Figures using additional raster images can only be included by \input if
%% they are in the same directory as the main LaTeX file. For loading figures
%% from other directories you can use the `import` package
%%   \usepackage{import}
%%
%% and then include the figures with
%%   \import{<path to file>}{<filename>.pgf}
%%
%% Matplotlib used the following preamble
%%   
%%   \usepackage{fontspec}
%%   \setmainfont{DejaVuSerif.ttf}[Path=\detokenize{/home/radimek/Documents/projekt_mat_prog/mat_prog_kernel/lib/python3.12/site-packages/matplotlib/mpl-data/fonts/ttf/}]
%%   \setsansfont{DejaVuSans.ttf}[Path=\detokenize{/home/radimek/Documents/projekt_mat_prog/mat_prog_kernel/lib/python3.12/site-packages/matplotlib/mpl-data/fonts/ttf/}]
%%   \setmonofont{DejaVuSansMono.ttf}[Path=\detokenize{/home/radimek/Documents/projekt_mat_prog/mat_prog_kernel/lib/python3.12/site-packages/matplotlib/mpl-data/fonts/ttf/}]
%%   \makeatletter\@ifpackageloaded{underscore}{}{\usepackage[strings]{underscore}}\makeatother
%%
\begingroup%
\makeatletter%
\begin{pgfpicture}%
\pgfpathrectangle{\pgfpointorigin}{\pgfqpoint{8.000000in}{6.000000in}}%
\pgfusepath{use as bounding box, clip}%
\begin{pgfscope}%
\pgfsetbuttcap%
\pgfsetmiterjoin%
\definecolor{currentfill}{rgb}{1.000000,1.000000,1.000000}%
\pgfsetfillcolor{currentfill}%
\pgfsetlinewidth{0.000000pt}%
\definecolor{currentstroke}{rgb}{1.000000,1.000000,1.000000}%
\pgfsetstrokecolor{currentstroke}%
\pgfsetdash{}{0pt}%
\pgfpathmoveto{\pgfqpoint{0.000000in}{0.000000in}}%
\pgfpathlineto{\pgfqpoint{8.000000in}{0.000000in}}%
\pgfpathlineto{\pgfqpoint{8.000000in}{6.000000in}}%
\pgfpathlineto{\pgfqpoint{0.000000in}{6.000000in}}%
\pgfpathlineto{\pgfqpoint{0.000000in}{0.000000in}}%
\pgfpathclose%
\pgfusepath{fill}%
\end{pgfscope}%
\begin{pgfscope}%
\pgfsetbuttcap%
\pgfsetmiterjoin%
\definecolor{currentfill}{rgb}{1.000000,1.000000,1.000000}%
\pgfsetfillcolor{currentfill}%
\pgfsetlinewidth{0.000000pt}%
\definecolor{currentstroke}{rgb}{0.000000,0.000000,0.000000}%
\pgfsetstrokecolor{currentstroke}%
\pgfsetstrokeopacity{0.000000}%
\pgfsetdash{}{0pt}%
\pgfpathmoveto{\pgfqpoint{1.150000in}{0.150000in}}%
\pgfpathlineto{\pgfqpoint{6.850000in}{0.150000in}}%
\pgfpathlineto{\pgfqpoint{6.850000in}{5.850000in}}%
\pgfpathlineto{\pgfqpoint{1.150000in}{5.850000in}}%
\pgfpathlineto{\pgfqpoint{1.150000in}{0.150000in}}%
\pgfpathclose%
\pgfusepath{fill}%
\end{pgfscope}%
\begin{pgfscope}%
\pgfsetbuttcap%
\pgfsetmiterjoin%
\definecolor{currentfill}{rgb}{0.950000,0.950000,0.950000}%
\pgfsetfillcolor{currentfill}%
\pgfsetfillopacity{0.500000}%
\pgfsetlinewidth{1.003750pt}%
\definecolor{currentstroke}{rgb}{0.950000,0.950000,0.950000}%
\pgfsetstrokecolor{currentstroke}%
\pgfsetstrokeopacity{0.500000}%
\pgfsetdash{}{0pt}%
\pgfpathmoveto{\pgfqpoint{1.580389in}{1.555437in}}%
\pgfpathlineto{\pgfqpoint{3.462715in}{3.133240in}}%
\pgfpathlineto{\pgfqpoint{3.436549in}{5.408715in}}%
\pgfpathlineto{\pgfqpoint{1.464144in}{3.969343in}}%
\pgfusepath{stroke,fill}%
\end{pgfscope}%
\begin{pgfscope}%
\pgfsetbuttcap%
\pgfsetmiterjoin%
\definecolor{currentfill}{rgb}{0.900000,0.900000,0.900000}%
\pgfsetfillcolor{currentfill}%
\pgfsetfillopacity{0.500000}%
\pgfsetlinewidth{1.003750pt}%
\definecolor{currentstroke}{rgb}{0.900000,0.900000,0.900000}%
\pgfsetstrokecolor{currentstroke}%
\pgfsetstrokeopacity{0.500000}%
\pgfsetdash{}{0pt}%
\pgfpathmoveto{\pgfqpoint{3.462715in}{3.133240in}}%
\pgfpathlineto{\pgfqpoint{6.483177in}{2.255311in}}%
\pgfpathlineto{\pgfqpoint{6.590967in}{4.609162in}}%
\pgfpathlineto{\pgfqpoint{3.436549in}{5.408715in}}%
\pgfusepath{stroke,fill}%
\end{pgfscope}%
\begin{pgfscope}%
\pgfsetbuttcap%
\pgfsetmiterjoin%
\definecolor{currentfill}{rgb}{0.925000,0.925000,0.925000}%
\pgfsetfillcolor{currentfill}%
\pgfsetfillopacity{0.500000}%
\pgfsetlinewidth{1.003750pt}%
\definecolor{currentstroke}{rgb}{0.925000,0.925000,0.925000}%
\pgfsetstrokecolor{currentstroke}%
\pgfsetstrokeopacity{0.500000}%
\pgfsetdash{}{0pt}%
\pgfpathmoveto{\pgfqpoint{1.580389in}{1.555437in}}%
\pgfpathlineto{\pgfqpoint{4.782226in}{0.509717in}}%
\pgfpathlineto{\pgfqpoint{6.483177in}{2.255311in}}%
\pgfpathlineto{\pgfqpoint{3.462715in}{3.133240in}}%
\pgfusepath{stroke,fill}%
\end{pgfscope}%
\begin{pgfscope}%
\pgfsetrectcap%
\pgfsetroundjoin%
\pgfsetlinewidth{0.803000pt}%
\definecolor{currentstroke}{rgb}{0.000000,0.000000,0.000000}%
\pgfsetstrokecolor{currentstroke}%
\pgfsetdash{}{0pt}%
\pgfpathmoveto{\pgfqpoint{1.580389in}{1.555437in}}%
\pgfpathlineto{\pgfqpoint{4.782226in}{0.509717in}}%
\pgfusepath{stroke}%
\end{pgfscope}%
\begin{pgfscope}%
\definecolor{textcolor}{rgb}{0.000000,0.000000,0.000000}%
\pgfsetstrokecolor{textcolor}%
\pgfsetfillcolor{textcolor}%
\pgftext[x=2.913491in,y=0.557898in,,]{\color{textcolor}\sffamily\fontsize{10.000000}{12.000000}\selectfont x}%
\end{pgfscope}%
\begin{pgfscope}%
\pgfsetbuttcap%
\pgfsetroundjoin%
\pgfsetlinewidth{0.803000pt}%
\definecolor{currentstroke}{rgb}{0.690196,0.690196,0.690196}%
\pgfsetstrokecolor{currentstroke}%
\pgfsetdash{}{0pt}%
\pgfpathmoveto{\pgfqpoint{1.774309in}{1.492103in}}%
\pgfpathlineto{\pgfqpoint{3.646411in}{3.079847in}}%
\pgfpathlineto{\pgfqpoint{3.628011in}{5.360185in}}%
\pgfusepath{stroke}%
\end{pgfscope}%
\begin{pgfscope}%
\pgfsetbuttcap%
\pgfsetroundjoin%
\pgfsetlinewidth{0.803000pt}%
\definecolor{currentstroke}{rgb}{0.690196,0.690196,0.690196}%
\pgfsetstrokecolor{currentstroke}%
\pgfsetdash{}{0pt}%
\pgfpathmoveto{\pgfqpoint{2.222368in}{1.345767in}}%
\pgfpathlineto{\pgfqpoint{4.070468in}{2.956591in}}%
\pgfpathlineto{\pgfqpoint{4.070186in}{5.248106in}}%
\pgfusepath{stroke}%
\end{pgfscope}%
\begin{pgfscope}%
\pgfsetbuttcap%
\pgfsetroundjoin%
\pgfsetlinewidth{0.803000pt}%
\definecolor{currentstroke}{rgb}{0.690196,0.690196,0.690196}%
\pgfsetstrokecolor{currentstroke}%
\pgfsetdash{}{0pt}%
\pgfpathmoveto{\pgfqpoint{2.677247in}{1.197204in}}%
\pgfpathlineto{\pgfqpoint{4.500444in}{2.831614in}}%
\pgfpathlineto{\pgfqpoint{4.518800in}{5.134396in}}%
\pgfusepath{stroke}%
\end{pgfscope}%
\begin{pgfscope}%
\pgfsetbuttcap%
\pgfsetroundjoin%
\pgfsetlinewidth{0.803000pt}%
\definecolor{currentstroke}{rgb}{0.690196,0.690196,0.690196}%
\pgfsetstrokecolor{currentstroke}%
\pgfsetdash{}{0pt}%
\pgfpathmoveto{\pgfqpoint{3.139103in}{1.046362in}}%
\pgfpathlineto{\pgfqpoint{4.936464in}{2.704880in}}%
\pgfpathlineto{\pgfqpoint{4.973994in}{5.019017in}}%
\pgfusepath{stroke}%
\end{pgfscope}%
\begin{pgfscope}%
\pgfsetbuttcap%
\pgfsetroundjoin%
\pgfsetlinewidth{0.803000pt}%
\definecolor{currentstroke}{rgb}{0.690196,0.690196,0.690196}%
\pgfsetstrokecolor{currentstroke}%
\pgfsetdash{}{0pt}%
\pgfpathmoveto{\pgfqpoint{3.608098in}{0.893188in}}%
\pgfpathlineto{\pgfqpoint{5.378655in}{2.576352in}}%
\pgfpathlineto{\pgfqpoint{5.435914in}{4.901934in}}%
\pgfusepath{stroke}%
\end{pgfscope}%
\begin{pgfscope}%
\pgfsetbuttcap%
\pgfsetroundjoin%
\pgfsetlinewidth{0.803000pt}%
\definecolor{currentstroke}{rgb}{0.690196,0.690196,0.690196}%
\pgfsetstrokecolor{currentstroke}%
\pgfsetdash{}{0pt}%
\pgfpathmoveto{\pgfqpoint{4.084398in}{0.737628in}}%
\pgfpathlineto{\pgfqpoint{5.827149in}{2.445993in}}%
\pgfpathlineto{\pgfqpoint{5.904712in}{4.783107in}}%
\pgfusepath{stroke}%
\end{pgfscope}%
\begin{pgfscope}%
\pgfsetbuttcap%
\pgfsetroundjoin%
\pgfsetlinewidth{0.803000pt}%
\definecolor{currentstroke}{rgb}{0.690196,0.690196,0.690196}%
\pgfsetstrokecolor{currentstroke}%
\pgfsetdash{}{0pt}%
\pgfpathmoveto{\pgfqpoint{4.568177in}{0.579626in}}%
\pgfpathlineto{\pgfqpoint{6.282083in}{2.313762in}}%
\pgfpathlineto{\pgfqpoint{6.380540in}{4.662499in}}%
\pgfusepath{stroke}%
\end{pgfscope}%
\begin{pgfscope}%
\pgfsetrectcap%
\pgfsetroundjoin%
\pgfsetlinewidth{0.803000pt}%
\definecolor{currentstroke}{rgb}{0.000000,0.000000,0.000000}%
\pgfsetstrokecolor{currentstroke}%
\pgfsetdash{}{0pt}%
\pgfpathmoveto{\pgfqpoint{1.790612in}{1.505929in}}%
\pgfpathlineto{\pgfqpoint{1.741635in}{1.464392in}}%
\pgfusepath{stroke}%
\end{pgfscope}%
\begin{pgfscope}%
\definecolor{textcolor}{rgb}{0.000000,0.000000,0.000000}%
\pgfsetstrokecolor{textcolor}%
\pgfsetfillcolor{textcolor}%
\pgftext[x=1.669876in,y=1.274184in,,top]{\color{textcolor}\sffamily\fontsize{10.000000}{12.000000}\selectfont \ensuremath{-}1.0}%
\end{pgfscope}%
\begin{pgfscope}%
\pgfsetrectcap%
\pgfsetroundjoin%
\pgfsetlinewidth{0.803000pt}%
\definecolor{currentstroke}{rgb}{0.000000,0.000000,0.000000}%
\pgfsetstrokecolor{currentstroke}%
\pgfsetdash{}{0pt}%
\pgfpathmoveto{\pgfqpoint{2.238471in}{1.359803in}}%
\pgfpathlineto{\pgfqpoint{2.190092in}{1.317635in}}%
\pgfusepath{stroke}%
\end{pgfscope}%
\begin{pgfscope}%
\definecolor{textcolor}{rgb}{0.000000,0.000000,0.000000}%
\pgfsetstrokecolor{textcolor}%
\pgfsetfillcolor{textcolor}%
\pgftext[x=2.118230in,y=1.125843in,,top]{\color{textcolor}\sffamily\fontsize{10.000000}{12.000000}\selectfont \ensuremath{-}0.5}%
\end{pgfscope}%
\begin{pgfscope}%
\pgfsetrectcap%
\pgfsetroundjoin%
\pgfsetlinewidth{0.803000pt}%
\definecolor{currentstroke}{rgb}{0.000000,0.000000,0.000000}%
\pgfsetstrokecolor{currentstroke}%
\pgfsetdash{}{0pt}%
\pgfpathmoveto{\pgfqpoint{2.693143in}{1.211454in}}%
\pgfpathlineto{\pgfqpoint{2.645386in}{1.168642in}}%
\pgfusepath{stroke}%
\end{pgfscope}%
\begin{pgfscope}%
\definecolor{textcolor}{rgb}{0.000000,0.000000,0.000000}%
\pgfsetstrokecolor{textcolor}%
\pgfsetfillcolor{textcolor}%
\pgftext[x=2.573426in,y=0.975238in,,top]{\color{textcolor}\sffamily\fontsize{10.000000}{12.000000}\selectfont 0.0}%
\end{pgfscope}%
\begin{pgfscope}%
\pgfsetrectcap%
\pgfsetroundjoin%
\pgfsetlinewidth{0.803000pt}%
\definecolor{currentstroke}{rgb}{0.000000,0.000000,0.000000}%
\pgfsetstrokecolor{currentstroke}%
\pgfsetdash{}{0pt}%
\pgfpathmoveto{\pgfqpoint{3.154783in}{1.060831in}}%
\pgfpathlineto{\pgfqpoint{3.107673in}{1.017359in}}%
\pgfusepath{stroke}%
\end{pgfscope}%
\begin{pgfscope}%
\definecolor{textcolor}{rgb}{0.000000,0.000000,0.000000}%
\pgfsetstrokecolor{textcolor}%
\pgfsetfillcolor{textcolor}%
\pgftext[x=3.035622in,y=0.822318in,,top]{\color{textcolor}\sffamily\fontsize{10.000000}{12.000000}\selectfont 0.5}%
\end{pgfscope}%
\begin{pgfscope}%
\pgfsetrectcap%
\pgfsetroundjoin%
\pgfsetlinewidth{0.803000pt}%
\definecolor{currentstroke}{rgb}{0.000000,0.000000,0.000000}%
\pgfsetstrokecolor{currentstroke}%
\pgfsetdash{}{0pt}%
\pgfpathmoveto{\pgfqpoint{3.623554in}{0.907881in}}%
\pgfpathlineto{\pgfqpoint{3.577116in}{0.863735in}}%
\pgfusepath{stroke}%
\end{pgfscope}%
\begin{pgfscope}%
\definecolor{textcolor}{rgb}{0.000000,0.000000,0.000000}%
\pgfsetstrokecolor{textcolor}%
\pgfsetfillcolor{textcolor}%
\pgftext[x=3.504979in,y=0.667028in,,top]{\color{textcolor}\sffamily\fontsize{10.000000}{12.000000}\selectfont 1.0}%
\end{pgfscope}%
\begin{pgfscope}%
\pgfsetrectcap%
\pgfsetroundjoin%
\pgfsetlinewidth{0.803000pt}%
\definecolor{currentstroke}{rgb}{0.000000,0.000000,0.000000}%
\pgfsetstrokecolor{currentstroke}%
\pgfsetdash{}{0pt}%
\pgfpathmoveto{\pgfqpoint{4.099622in}{0.752551in}}%
\pgfpathlineto{\pgfqpoint{4.053883in}{0.707715in}}%
\pgfusepath{stroke}%
\end{pgfscope}%
\begin{pgfscope}%
\definecolor{textcolor}{rgb}{0.000000,0.000000,0.000000}%
\pgfsetstrokecolor{textcolor}%
\pgfsetfillcolor{textcolor}%
\pgftext[x=3.981665in,y=0.509314in,,top]{\color{textcolor}\sffamily\fontsize{10.000000}{12.000000}\selectfont 1.5}%
\end{pgfscope}%
\begin{pgfscope}%
\pgfsetrectcap%
\pgfsetroundjoin%
\pgfsetlinewidth{0.803000pt}%
\definecolor{currentstroke}{rgb}{0.000000,0.000000,0.000000}%
\pgfsetstrokecolor{currentstroke}%
\pgfsetdash{}{0pt}%
\pgfpathmoveto{\pgfqpoint{4.583158in}{0.594784in}}%
\pgfpathlineto{\pgfqpoint{4.538146in}{0.549241in}}%
\pgfusepath{stroke}%
\end{pgfscope}%
\begin{pgfscope}%
\definecolor{textcolor}{rgb}{0.000000,0.000000,0.000000}%
\pgfsetstrokecolor{textcolor}%
\pgfsetfillcolor{textcolor}%
\pgftext[x=4.465855in,y=0.349116in,,top]{\color{textcolor}\sffamily\fontsize{10.000000}{12.000000}\selectfont 2.0}%
\end{pgfscope}%
\begin{pgfscope}%
\pgfsetrectcap%
\pgfsetroundjoin%
\pgfsetlinewidth{0.803000pt}%
\definecolor{currentstroke}{rgb}{0.000000,0.000000,0.000000}%
\pgfsetstrokecolor{currentstroke}%
\pgfsetdash{}{0pt}%
\pgfpathmoveto{\pgfqpoint{6.483177in}{2.255311in}}%
\pgfpathlineto{\pgfqpoint{4.782226in}{0.509717in}}%
\pgfusepath{stroke}%
\end{pgfscope}%
\begin{pgfscope}%
\definecolor{textcolor}{rgb}{0.000000,0.000000,0.000000}%
\pgfsetstrokecolor{textcolor}%
\pgfsetfillcolor{textcolor}%
\pgftext[x=6.045209in,y=1.032725in,,]{\color{textcolor}\sffamily\fontsize{10.000000}{12.000000}\selectfont y}%
\end{pgfscope}%
\begin{pgfscope}%
\pgfsetbuttcap%
\pgfsetroundjoin%
\pgfsetlinewidth{0.803000pt}%
\definecolor{currentstroke}{rgb}{0.690196,0.690196,0.690196}%
\pgfsetstrokecolor{currentstroke}%
\pgfsetdash{}{0pt}%
\pgfpathmoveto{\pgfqpoint{1.600541in}{4.068879in}}%
\pgfpathlineto{\pgfqpoint{1.710097in}{1.664161in}}%
\pgfpathlineto{\pgfqpoint{4.899919in}{0.630499in}}%
\pgfusepath{stroke}%
\end{pgfscope}%
\begin{pgfscope}%
\pgfsetbuttcap%
\pgfsetroundjoin%
\pgfsetlinewidth{0.803000pt}%
\definecolor{currentstroke}{rgb}{0.690196,0.690196,0.690196}%
\pgfsetstrokecolor{currentstroke}%
\pgfsetdash{}{0pt}%
\pgfpathmoveto{\pgfqpoint{1.966540in}{4.335969in}}%
\pgfpathlineto{\pgfqpoint{2.058485in}{1.956187in}}%
\pgfpathlineto{\pgfqpoint{5.215679in}{0.954546in}}%
\pgfusepath{stroke}%
\end{pgfscope}%
\begin{pgfscope}%
\pgfsetbuttcap%
\pgfsetroundjoin%
\pgfsetlinewidth{0.803000pt}%
\definecolor{currentstroke}{rgb}{0.690196,0.690196,0.690196}%
\pgfsetstrokecolor{currentstroke}%
\pgfsetdash{}{0pt}%
\pgfpathmoveto{\pgfqpoint{2.321081in}{4.594697in}}%
\pgfpathlineto{\pgfqpoint{2.396435in}{2.239463in}}%
\pgfpathlineto{\pgfqpoint{5.521485in}{1.268378in}}%
\pgfusepath{stroke}%
\end{pgfscope}%
\begin{pgfscope}%
\pgfsetbuttcap%
\pgfsetroundjoin%
\pgfsetlinewidth{0.803000pt}%
\definecolor{currentstroke}{rgb}{0.690196,0.690196,0.690196}%
\pgfsetstrokecolor{currentstroke}%
\pgfsetdash{}{0pt}%
\pgfpathmoveto{\pgfqpoint{2.664695in}{4.845450in}}%
\pgfpathlineto{\pgfqpoint{2.724408in}{2.514377in}}%
\pgfpathlineto{\pgfqpoint{5.817800in}{1.572471in}}%
\pgfusepath{stroke}%
\end{pgfscope}%
\begin{pgfscope}%
\pgfsetbuttcap%
\pgfsetroundjoin%
\pgfsetlinewidth{0.803000pt}%
\definecolor{currentstroke}{rgb}{0.690196,0.690196,0.690196}%
\pgfsetstrokecolor{currentstroke}%
\pgfsetdash{}{0pt}%
\pgfpathmoveto{\pgfqpoint{2.997878in}{5.088592in}}%
\pgfpathlineto{\pgfqpoint{3.042841in}{2.781293in}}%
\pgfpathlineto{\pgfqpoint{6.105061in}{1.867271in}}%
\pgfusepath{stroke}%
\end{pgfscope}%
\begin{pgfscope}%
\pgfsetbuttcap%
\pgfsetroundjoin%
\pgfsetlinewidth{0.803000pt}%
\definecolor{currentstroke}{rgb}{0.690196,0.690196,0.690196}%
\pgfsetstrokecolor{currentstroke}%
\pgfsetdash{}{0pt}%
\pgfpathmoveto{\pgfqpoint{3.321099in}{5.324464in}}%
\pgfpathlineto{\pgfqpoint{3.352144in}{3.040557in}}%
\pgfpathlineto{\pgfqpoint{6.383674in}{2.153197in}}%
\pgfusepath{stroke}%
\end{pgfscope}%
\begin{pgfscope}%
\pgfsetrectcap%
\pgfsetroundjoin%
\pgfsetlinewidth{0.803000pt}%
\definecolor{currentstroke}{rgb}{0.000000,0.000000,0.000000}%
\pgfsetstrokecolor{currentstroke}%
\pgfsetdash{}{0pt}%
\pgfpathmoveto{\pgfqpoint{4.873038in}{0.639210in}}%
\pgfpathlineto{\pgfqpoint{4.953750in}{0.613055in}}%
\pgfusepath{stroke}%
\end{pgfscope}%
\begin{pgfscope}%
\definecolor{textcolor}{rgb}{0.000000,0.000000,0.000000}%
\pgfsetstrokecolor{textcolor}%
\pgfsetfillcolor{textcolor}%
\pgftext[x=5.078779in,y=0.444104in,,top]{\color{textcolor}\sffamily\fontsize{10.000000}{12.000000}\selectfont \ensuremath{-}1.0}%
\end{pgfscope}%
\begin{pgfscope}%
\pgfsetrectcap%
\pgfsetroundjoin%
\pgfsetlinewidth{0.803000pt}%
\definecolor{currentstroke}{rgb}{0.000000,0.000000,0.000000}%
\pgfsetstrokecolor{currentstroke}%
\pgfsetdash{}{0pt}%
\pgfpathmoveto{\pgfqpoint{5.189095in}{0.962980in}}%
\pgfpathlineto{\pgfqpoint{5.268915in}{0.937657in}}%
\pgfusepath{stroke}%
\end{pgfscope}%
\begin{pgfscope}%
\definecolor{textcolor}{rgb}{0.000000,0.000000,0.000000}%
\pgfsetstrokecolor{textcolor}%
\pgfsetfillcolor{textcolor}%
\pgftext[x=5.391794in,y=0.771261in,,top]{\color{textcolor}\sffamily\fontsize{10.000000}{12.000000}\selectfont \ensuremath{-}0.5}%
\end{pgfscope}%
\begin{pgfscope}%
\pgfsetrectcap%
\pgfsetroundjoin%
\pgfsetlinewidth{0.803000pt}%
\definecolor{currentstroke}{rgb}{0.000000,0.000000,0.000000}%
\pgfsetstrokecolor{currentstroke}%
\pgfsetdash{}{0pt}%
\pgfpathmoveto{\pgfqpoint{5.495192in}{1.276549in}}%
\pgfpathlineto{\pgfqpoint{5.574136in}{1.252017in}}%
\pgfusepath{stroke}%
\end{pgfscope}%
\begin{pgfscope}%
\definecolor{textcolor}{rgb}{0.000000,0.000000,0.000000}%
\pgfsetstrokecolor{textcolor}%
\pgfsetfillcolor{textcolor}%
\pgftext[x=5.694938in,y=1.088101in,,top]{\color{textcolor}\sffamily\fontsize{10.000000}{12.000000}\selectfont 0.0}%
\end{pgfscope}%
\begin{pgfscope}%
\pgfsetrectcap%
\pgfsetroundjoin%
\pgfsetlinewidth{0.803000pt}%
\definecolor{currentstroke}{rgb}{0.000000,0.000000,0.000000}%
\pgfsetstrokecolor{currentstroke}%
\pgfsetdash{}{0pt}%
\pgfpathmoveto{\pgfqpoint{5.791794in}{1.580390in}}%
\pgfpathlineto{\pgfqpoint{5.869878in}{1.556614in}}%
\pgfusepath{stroke}%
\end{pgfscope}%
\begin{pgfscope}%
\definecolor{textcolor}{rgb}{0.000000,0.000000,0.000000}%
\pgfsetstrokecolor{textcolor}%
\pgfsetfillcolor{textcolor}%
\pgftext[x=5.988671in,y=1.395104in,,top]{\color{textcolor}\sffamily\fontsize{10.000000}{12.000000}\selectfont 0.5}%
\end{pgfscope}%
\begin{pgfscope}%
\pgfsetrectcap%
\pgfsetroundjoin%
\pgfsetlinewidth{0.803000pt}%
\definecolor{currentstroke}{rgb}{0.000000,0.000000,0.000000}%
\pgfsetstrokecolor{currentstroke}%
\pgfsetdash{}{0pt}%
\pgfpathmoveto{\pgfqpoint{6.079335in}{1.874949in}}%
\pgfpathlineto{\pgfqpoint{6.156574in}{1.851895in}}%
\pgfusepath{stroke}%
\end{pgfscope}%
\begin{pgfscope}%
\definecolor{textcolor}{rgb}{0.000000,0.000000,0.000000}%
\pgfsetstrokecolor{textcolor}%
\pgfsetfillcolor{textcolor}%
\pgftext[x=6.273424in,y=1.692723in,,top]{\color{textcolor}\sffamily\fontsize{10.000000}{12.000000}\selectfont 1.0}%
\end{pgfscope}%
\begin{pgfscope}%
\pgfsetrectcap%
\pgfsetroundjoin%
\pgfsetlinewidth{0.803000pt}%
\definecolor{currentstroke}{rgb}{0.000000,0.000000,0.000000}%
\pgfsetstrokecolor{currentstroke}%
\pgfsetdash{}{0pt}%
\pgfpathmoveto{\pgfqpoint{6.358225in}{2.160646in}}%
\pgfpathlineto{\pgfqpoint{6.434634in}{2.138280in}}%
\pgfusepath{stroke}%
\end{pgfscope}%
\begin{pgfscope}%
\definecolor{textcolor}{rgb}{0.000000,0.000000,0.000000}%
\pgfsetstrokecolor{textcolor}%
\pgfsetfillcolor{textcolor}%
\pgftext[x=6.549603in,y=1.981379in,,top]{\color{textcolor}\sffamily\fontsize{10.000000}{12.000000}\selectfont 1.5}%
\end{pgfscope}%
\begin{pgfscope}%
\pgfsetrectcap%
\pgfsetroundjoin%
\pgfsetlinewidth{0.803000pt}%
\definecolor{currentstroke}{rgb}{0.000000,0.000000,0.000000}%
\pgfsetstrokecolor{currentstroke}%
\pgfsetdash{}{0pt}%
\pgfpathmoveto{\pgfqpoint{6.483177in}{2.255311in}}%
\pgfpathlineto{\pgfqpoint{6.590967in}{4.609162in}}%
\pgfusepath{stroke}%
\end{pgfscope}%
\begin{pgfscope}%
\definecolor{textcolor}{rgb}{0.000000,0.000000,0.000000}%
\pgfsetstrokecolor{textcolor}%
\pgfsetfillcolor{textcolor}%
\pgftext[x=7.097978in,y=3.481758in,,,rotate=87.378092]{\color{textcolor}\sffamily\fontsize{10.000000}{12.000000}\selectfont f(x,y)}%
\end{pgfscope}%
\begin{pgfscope}%
\pgfsetbuttcap%
\pgfsetroundjoin%
\pgfsetlinewidth{0.803000pt}%
\definecolor{currentstroke}{rgb}{0.690196,0.690196,0.690196}%
\pgfsetstrokecolor{currentstroke}%
\pgfsetdash{}{0pt}%
\pgfpathmoveto{\pgfqpoint{6.492331in}{2.455207in}}%
\pgfpathlineto{\pgfqpoint{3.460489in}{3.326854in}}%
\pgfpathlineto{\pgfqpoint{1.570533in}{1.760117in}}%
\pgfusepath{stroke}%
\end{pgfscope}%
\begin{pgfscope}%
\pgfsetbuttcap%
\pgfsetroundjoin%
\pgfsetlinewidth{0.803000pt}%
\definecolor{currentstroke}{rgb}{0.690196,0.690196,0.690196}%
\pgfsetstrokecolor{currentstroke}%
\pgfsetdash{}{0pt}%
\pgfpathmoveto{\pgfqpoint{6.504333in}{2.717297in}}%
\pgfpathlineto{\pgfqpoint{3.457571in}{3.580602in}}%
\pgfpathlineto{\pgfqpoint{1.557605in}{2.028568in}}%
\pgfusepath{stroke}%
\end{pgfscope}%
\begin{pgfscope}%
\pgfsetbuttcap%
\pgfsetroundjoin%
\pgfsetlinewidth{0.803000pt}%
\definecolor{currentstroke}{rgb}{0.690196,0.690196,0.690196}%
\pgfsetstrokecolor{currentstroke}%
\pgfsetdash{}{0pt}%
\pgfpathmoveto{\pgfqpoint{6.516455in}{2.982005in}}%
\pgfpathlineto{\pgfqpoint{3.454626in}{3.836763in}}%
\pgfpathlineto{\pgfqpoint{1.544543in}{2.299803in}}%
\pgfusepath{stroke}%
\end{pgfscope}%
\begin{pgfscope}%
\pgfsetbuttcap%
\pgfsetroundjoin%
\pgfsetlinewidth{0.803000pt}%
\definecolor{currentstroke}{rgb}{0.690196,0.690196,0.690196}%
\pgfsetstrokecolor{currentstroke}%
\pgfsetdash{}{0pt}%
\pgfpathmoveto{\pgfqpoint{6.528698in}{3.249370in}}%
\pgfpathlineto{\pgfqpoint{3.451652in}{4.095372in}}%
\pgfpathlineto{\pgfqpoint{1.531345in}{2.573865in}}%
\pgfusepath{stroke}%
\end{pgfscope}%
\begin{pgfscope}%
\pgfsetbuttcap%
\pgfsetroundjoin%
\pgfsetlinewidth{0.803000pt}%
\definecolor{currentstroke}{rgb}{0.690196,0.690196,0.690196}%
\pgfsetstrokecolor{currentstroke}%
\pgfsetdash{}{0pt}%
\pgfpathmoveto{\pgfqpoint{6.541065in}{3.519432in}}%
\pgfpathlineto{\pgfqpoint{3.448649in}{4.356463in}}%
\pgfpathlineto{\pgfqpoint{1.518009in}{2.850798in}}%
\pgfusepath{stroke}%
\end{pgfscope}%
\begin{pgfscope}%
\pgfsetbuttcap%
\pgfsetroundjoin%
\pgfsetlinewidth{0.803000pt}%
\definecolor{currentstroke}{rgb}{0.690196,0.690196,0.690196}%
\pgfsetstrokecolor{currentstroke}%
\pgfsetdash{}{0pt}%
\pgfpathmoveto{\pgfqpoint{6.553557in}{3.792232in}}%
\pgfpathlineto{\pgfqpoint{3.445618in}{4.620074in}}%
\pgfpathlineto{\pgfqpoint{1.504533in}{3.130648in}}%
\pgfusepath{stroke}%
\end{pgfscope}%
\begin{pgfscope}%
\pgfsetbuttcap%
\pgfsetroundjoin%
\pgfsetlinewidth{0.803000pt}%
\definecolor{currentstroke}{rgb}{0.690196,0.690196,0.690196}%
\pgfsetstrokecolor{currentstroke}%
\pgfsetdash{}{0pt}%
\pgfpathmoveto{\pgfqpoint{6.566177in}{4.067813in}}%
\pgfpathlineto{\pgfqpoint{3.442557in}{4.886241in}}%
\pgfpathlineto{\pgfqpoint{1.490913in}{3.413461in}}%
\pgfusepath{stroke}%
\end{pgfscope}%
\begin{pgfscope}%
\pgfsetbuttcap%
\pgfsetroundjoin%
\pgfsetlinewidth{0.803000pt}%
\definecolor{currentstroke}{rgb}{0.690196,0.690196,0.690196}%
\pgfsetstrokecolor{currentstroke}%
\pgfsetdash{}{0pt}%
\pgfpathmoveto{\pgfqpoint{6.578926in}{4.346216in}}%
\pgfpathlineto{\pgfqpoint{3.439467in}{5.155001in}}%
\pgfpathlineto{\pgfqpoint{1.477149in}{3.699285in}}%
\pgfusepath{stroke}%
\end{pgfscope}%
\begin{pgfscope}%
\pgfsetrectcap%
\pgfsetroundjoin%
\pgfsetlinewidth{0.803000pt}%
\definecolor{currentstroke}{rgb}{0.000000,0.000000,0.000000}%
\pgfsetstrokecolor{currentstroke}%
\pgfsetdash{}{0pt}%
\pgfpathmoveto{\pgfqpoint{6.466881in}{2.462524in}}%
\pgfpathlineto{\pgfqpoint{6.543292in}{2.440556in}}%
\pgfusepath{stroke}%
\end{pgfscope}%
\begin{pgfscope}%
\definecolor{textcolor}{rgb}{0.000000,0.000000,0.000000}%
\pgfsetstrokecolor{textcolor}%
\pgfsetfillcolor{textcolor}%
\pgftext[x=6.746997in,y=2.491173in,,top]{\color{textcolor}\sffamily\fontsize{10.000000}{12.000000}\selectfont 2}%
\end{pgfscope}%
\begin{pgfscope}%
\pgfsetrectcap%
\pgfsetroundjoin%
\pgfsetlinewidth{0.803000pt}%
\definecolor{currentstroke}{rgb}{0.000000,0.000000,0.000000}%
\pgfsetstrokecolor{currentstroke}%
\pgfsetdash{}{0pt}%
\pgfpathmoveto{\pgfqpoint{6.478752in}{2.724546in}}%
\pgfpathlineto{\pgfqpoint{6.555557in}{2.702783in}}%
\pgfusepath{stroke}%
\end{pgfscope}%
\begin{pgfscope}%
\definecolor{textcolor}{rgb}{0.000000,0.000000,0.000000}%
\pgfsetstrokecolor{textcolor}%
\pgfsetfillcolor{textcolor}%
\pgftext[x=6.760239in,y=2.752927in,,top]{\color{textcolor}\sffamily\fontsize{10.000000}{12.000000}\selectfont 4}%
\end{pgfscope}%
\begin{pgfscope}%
\pgfsetrectcap%
\pgfsetroundjoin%
\pgfsetlinewidth{0.803000pt}%
\definecolor{currentstroke}{rgb}{0.000000,0.000000,0.000000}%
\pgfsetstrokecolor{currentstroke}%
\pgfsetdash{}{0pt}%
\pgfpathmoveto{\pgfqpoint{6.490741in}{2.989183in}}%
\pgfpathlineto{\pgfqpoint{6.567944in}{2.967631in}}%
\pgfusepath{stroke}%
\end{pgfscope}%
\begin{pgfscope}%
\definecolor{textcolor}{rgb}{0.000000,0.000000,0.000000}%
\pgfsetstrokecolor{textcolor}%
\pgfsetfillcolor{textcolor}%
\pgftext[x=6.773612in,y=3.017290in,,top]{\color{textcolor}\sffamily\fontsize{10.000000}{12.000000}\selectfont 6}%
\end{pgfscope}%
\begin{pgfscope}%
\pgfsetrectcap%
\pgfsetroundjoin%
\pgfsetlinewidth{0.803000pt}%
\definecolor{currentstroke}{rgb}{0.000000,0.000000,0.000000}%
\pgfsetstrokecolor{currentstroke}%
\pgfsetdash{}{0pt}%
\pgfpathmoveto{\pgfqpoint{6.502850in}{3.256476in}}%
\pgfpathlineto{\pgfqpoint{6.580456in}{3.235139in}}%
\pgfusepath{stroke}%
\end{pgfscope}%
\begin{pgfscope}%
\definecolor{textcolor}{rgb}{0.000000,0.000000,0.000000}%
\pgfsetstrokecolor{textcolor}%
\pgfsetfillcolor{textcolor}%
\pgftext[x=6.787119in,y=3.284301in,,top]{\color{textcolor}\sffamily\fontsize{10.000000}{12.000000}\selectfont 8}%
\end{pgfscope}%
\begin{pgfscope}%
\pgfsetrectcap%
\pgfsetroundjoin%
\pgfsetlinewidth{0.803000pt}%
\definecolor{currentstroke}{rgb}{0.000000,0.000000,0.000000}%
\pgfsetstrokecolor{currentstroke}%
\pgfsetdash{}{0pt}%
\pgfpathmoveto{\pgfqpoint{6.515082in}{3.526464in}}%
\pgfpathlineto{\pgfqpoint{6.593095in}{3.505349in}}%
\pgfusepath{stroke}%
\end{pgfscope}%
\begin{pgfscope}%
\definecolor{textcolor}{rgb}{0.000000,0.000000,0.000000}%
\pgfsetstrokecolor{textcolor}%
\pgfsetfillcolor{textcolor}%
\pgftext[x=6.800762in,y=3.554001in,,top]{\color{textcolor}\sffamily\fontsize{10.000000}{12.000000}\selectfont 10}%
\end{pgfscope}%
\begin{pgfscope}%
\pgfsetrectcap%
\pgfsetroundjoin%
\pgfsetlinewidth{0.803000pt}%
\definecolor{currentstroke}{rgb}{0.000000,0.000000,0.000000}%
\pgfsetstrokecolor{currentstroke}%
\pgfsetdash{}{0pt}%
\pgfpathmoveto{\pgfqpoint{6.527438in}{3.799189in}}%
\pgfpathlineto{\pgfqpoint{6.605861in}{3.778300in}}%
\pgfusepath{stroke}%
\end{pgfscope}%
\begin{pgfscope}%
\definecolor{textcolor}{rgb}{0.000000,0.000000,0.000000}%
\pgfsetstrokecolor{textcolor}%
\pgfsetfillcolor{textcolor}%
\pgftext[x=6.814544in,y=3.826430in,,top]{\color{textcolor}\sffamily\fontsize{10.000000}{12.000000}\selectfont 12}%
\end{pgfscope}%
\begin{pgfscope}%
\pgfsetrectcap%
\pgfsetroundjoin%
\pgfsetlinewidth{0.803000pt}%
\definecolor{currentstroke}{rgb}{0.000000,0.000000,0.000000}%
\pgfsetstrokecolor{currentstroke}%
\pgfsetdash{}{0pt}%
\pgfpathmoveto{\pgfqpoint{6.539919in}{4.074693in}}%
\pgfpathlineto{\pgfqpoint{6.618758in}{4.054036in}}%
\pgfusepath{stroke}%
\end{pgfscope}%
\begin{pgfscope}%
\definecolor{textcolor}{rgb}{0.000000,0.000000,0.000000}%
\pgfsetstrokecolor{textcolor}%
\pgfsetfillcolor{textcolor}%
\pgftext[x=6.828465in,y=4.101629in,,top]{\color{textcolor}\sffamily\fontsize{10.000000}{12.000000}\selectfont 14}%
\end{pgfscope}%
\begin{pgfscope}%
\pgfsetrectcap%
\pgfsetroundjoin%
\pgfsetlinewidth{0.803000pt}%
\definecolor{currentstroke}{rgb}{0.000000,0.000000,0.000000}%
\pgfsetstrokecolor{currentstroke}%
\pgfsetdash{}{0pt}%
\pgfpathmoveto{\pgfqpoint{6.552528in}{4.353017in}}%
\pgfpathlineto{\pgfqpoint{6.631787in}{4.332598in}}%
\pgfusepath{stroke}%
\end{pgfscope}%
\begin{pgfscope}%
\definecolor{textcolor}{rgb}{0.000000,0.000000,0.000000}%
\pgfsetstrokecolor{textcolor}%
\pgfsetfillcolor{textcolor}%
\pgftext[x=6.842529in,y=4.379642in,,top]{\color{textcolor}\sffamily\fontsize{10.000000}{12.000000}\selectfont 16}%
\end{pgfscope}%
\begin{pgfscope}%
\pgfpathrectangle{\pgfqpoint{1.150000in}{0.150000in}}{\pgfqpoint{5.700000in}{5.700000in}}%
\pgfusepath{clip}%
\pgfsetrectcap%
\pgfsetroundjoin%
\pgfsetlinewidth{2.007500pt}%
\definecolor{currentstroke}{rgb}{0.000000,0.000000,0.000000}%
\pgfsetstrokecolor{currentstroke}%
\pgfsetdash{}{0pt}%
\pgfpathmoveto{\pgfqpoint{5.157229in}{2.372390in}}%
\pgfpathlineto{\pgfqpoint{4.752998in}{2.972348in}}%
\pgfpathlineto{\pgfqpoint{3.826061in}{2.578215in}}%
\pgfpathlineto{\pgfqpoint{3.401354in}{2.232610in}}%
\pgfpathlineto{\pgfqpoint{3.055108in}{1.796121in}}%
\pgfpathlineto{\pgfqpoint{3.274671in}{1.586296in}}%
\pgfpathlineto{\pgfqpoint{3.325628in}{1.506449in}}%
\pgfusepath{stroke}%
\end{pgfscope}%
\begin{pgfscope}%
\pgfpathrectangle{\pgfqpoint{1.150000in}{0.150000in}}{\pgfqpoint{5.700000in}{5.700000in}}%
\pgfusepath{clip}%
\pgfsetbuttcap%
\pgfsetroundjoin%
\definecolor{currentfill}{rgb}{1.000000,0.000000,0.000000}%
\pgfsetfillcolor{currentfill}%
\pgfsetfillopacity{0.300000}%
\pgfsetlinewidth{1.003750pt}%
\definecolor{currentstroke}{rgb}{1.000000,0.000000,0.000000}%
\pgfsetstrokecolor{currentstroke}%
\pgfsetstrokeopacity{0.300000}%
\pgfsetdash{}{0pt}%
\pgfpathmoveto{\pgfqpoint{4.752998in}{2.923244in}}%
\pgfpathcurveto{\pgfqpoint{4.766021in}{2.923244in}}{\pgfqpoint{4.778512in}{2.928418in}}{\pgfqpoint{4.787721in}{2.937626in}}%
\pgfpathcurveto{\pgfqpoint{4.796929in}{2.946834in}}{\pgfqpoint{4.802103in}{2.959325in}}{\pgfqpoint{4.802103in}{2.972348in}}%
\pgfpathcurveto{\pgfqpoint{4.802103in}{2.985371in}}{\pgfqpoint{4.796929in}{2.997862in}}{\pgfqpoint{4.787721in}{3.007070in}}%
\pgfpathcurveto{\pgfqpoint{4.778512in}{3.016279in}}{\pgfqpoint{4.766021in}{3.021453in}}{\pgfqpoint{4.752998in}{3.021453in}}%
\pgfpathcurveto{\pgfqpoint{4.739976in}{3.021453in}}{\pgfqpoint{4.727485in}{3.016279in}}{\pgfqpoint{4.718276in}{3.007070in}}%
\pgfpathcurveto{\pgfqpoint{4.709068in}{2.997862in}}{\pgfqpoint{4.703894in}{2.985371in}}{\pgfqpoint{4.703894in}{2.972348in}}%
\pgfpathcurveto{\pgfqpoint{4.703894in}{2.959325in}}{\pgfqpoint{4.709068in}{2.946834in}}{\pgfqpoint{4.718276in}{2.937626in}}%
\pgfpathcurveto{\pgfqpoint{4.727485in}{2.928418in}}{\pgfqpoint{4.739976in}{2.923244in}}{\pgfqpoint{4.752998in}{2.923244in}}%
\pgfpathlineto{\pgfqpoint{4.752998in}{2.923244in}}%
\pgfpathclose%
\pgfusepath{stroke,fill}%
\end{pgfscope}%
\begin{pgfscope}%
\pgfpathrectangle{\pgfqpoint{1.150000in}{0.150000in}}{\pgfqpoint{5.700000in}{5.700000in}}%
\pgfusepath{clip}%
\pgfsetbuttcap%
\pgfsetroundjoin%
\definecolor{currentfill}{rgb}{1.000000,0.000000,0.000000}%
\pgfsetfillcolor{currentfill}%
\pgfsetfillopacity{0.342535}%
\pgfsetlinewidth{1.003750pt}%
\definecolor{currentstroke}{rgb}{1.000000,0.000000,0.000000}%
\pgfsetstrokecolor{currentstroke}%
\pgfsetstrokeopacity{0.342535}%
\pgfsetdash{}{0pt}%
\pgfpathmoveto{\pgfqpoint{3.826061in}{2.529110in}}%
\pgfpathcurveto{\pgfqpoint{3.839084in}{2.529110in}}{\pgfqpoint{3.851575in}{2.534284in}}{\pgfqpoint{3.860783in}{2.543493in}}%
\pgfpathcurveto{\pgfqpoint{3.869992in}{2.552701in}}{\pgfqpoint{3.875166in}{2.565192in}}{\pgfqpoint{3.875166in}{2.578215in}}%
\pgfpathcurveto{\pgfqpoint{3.875166in}{2.591237in}}{\pgfqpoint{3.869992in}{2.603729in}}{\pgfqpoint{3.860783in}{2.612937in}}%
\pgfpathcurveto{\pgfqpoint{3.851575in}{2.622145in}}{\pgfqpoint{3.839084in}{2.627319in}}{\pgfqpoint{3.826061in}{2.627319in}}%
\pgfpathcurveto{\pgfqpoint{3.813038in}{2.627319in}}{\pgfqpoint{3.800547in}{2.622145in}}{\pgfqpoint{3.791339in}{2.612937in}}%
\pgfpathcurveto{\pgfqpoint{3.782131in}{2.603729in}}{\pgfqpoint{3.776957in}{2.591237in}}{\pgfqpoint{3.776957in}{2.578215in}}%
\pgfpathcurveto{\pgfqpoint{3.776957in}{2.565192in}}{\pgfqpoint{3.782131in}{2.552701in}}{\pgfqpoint{3.791339in}{2.543493in}}%
\pgfpathcurveto{\pgfqpoint{3.800547in}{2.534284in}}{\pgfqpoint{3.813038in}{2.529110in}}{\pgfqpoint{3.826061in}{2.529110in}}%
\pgfpathlineto{\pgfqpoint{3.826061in}{2.529110in}}%
\pgfpathclose%
\pgfusepath{stroke,fill}%
\end{pgfscope}%
\begin{pgfscope}%
\pgfpathrectangle{\pgfqpoint{1.150000in}{0.150000in}}{\pgfqpoint{5.700000in}{5.700000in}}%
\pgfusepath{clip}%
\pgfsetbuttcap%
\pgfsetroundjoin%
\definecolor{currentfill}{rgb}{1.000000,0.000000,0.000000}%
\pgfsetfillcolor{currentfill}%
\pgfsetfillopacity{0.504777}%
\pgfsetlinewidth{1.003750pt}%
\definecolor{currentstroke}{rgb}{1.000000,0.000000,0.000000}%
\pgfsetstrokecolor{currentstroke}%
\pgfsetstrokeopacity{0.504777}%
\pgfsetdash{}{0pt}%
\pgfpathmoveto{\pgfqpoint{3.401354in}{2.183505in}}%
\pgfpathcurveto{\pgfqpoint{3.414376in}{2.183505in}}{\pgfqpoint{3.426867in}{2.188679in}}{\pgfqpoint{3.436076in}{2.197887in}}%
\pgfpathcurveto{\pgfqpoint{3.445284in}{2.207096in}}{\pgfqpoint{3.450458in}{2.219587in}}{\pgfqpoint{3.450458in}{2.232610in}}%
\pgfpathcurveto{\pgfqpoint{3.450458in}{2.245632in}}{\pgfqpoint{3.445284in}{2.258123in}}{\pgfqpoint{3.436076in}{2.267332in}}%
\pgfpathcurveto{\pgfqpoint{3.426867in}{2.276540in}}{\pgfqpoint{3.414376in}{2.281714in}}{\pgfqpoint{3.401354in}{2.281714in}}%
\pgfpathcurveto{\pgfqpoint{3.388331in}{2.281714in}}{\pgfqpoint{3.375840in}{2.276540in}}{\pgfqpoint{3.366631in}{2.267332in}}%
\pgfpathcurveto{\pgfqpoint{3.357423in}{2.258123in}}{\pgfqpoint{3.352249in}{2.245632in}}{\pgfqpoint{3.352249in}{2.232610in}}%
\pgfpathcurveto{\pgfqpoint{3.352249in}{2.219587in}}{\pgfqpoint{3.357423in}{2.207096in}}{\pgfqpoint{3.366631in}{2.197887in}}%
\pgfpathcurveto{\pgfqpoint{3.375840in}{2.188679in}}{\pgfqpoint{3.388331in}{2.183505in}}{\pgfqpoint{3.401354in}{2.183505in}}%
\pgfpathlineto{\pgfqpoint{3.401354in}{2.183505in}}%
\pgfpathclose%
\pgfusepath{stroke,fill}%
\end{pgfscope}%
\begin{pgfscope}%
\pgfpathrectangle{\pgfqpoint{1.150000in}{0.150000in}}{\pgfqpoint{5.700000in}{5.700000in}}%
\pgfusepath{clip}%
\pgfsetbuttcap%
\pgfsetroundjoin%
\definecolor{currentfill}{rgb}{1.000000,0.000000,0.000000}%
\pgfsetfillcolor{currentfill}%
\pgfsetfillopacity{0.791384}%
\pgfsetlinewidth{1.003750pt}%
\definecolor{currentstroke}{rgb}{1.000000,0.000000,0.000000}%
\pgfsetstrokecolor{currentstroke}%
\pgfsetstrokeopacity{0.791384}%
\pgfsetdash{}{0pt}%
\pgfpathmoveto{\pgfqpoint{3.055108in}{1.747016in}}%
\pgfpathcurveto{\pgfqpoint{3.068131in}{1.747016in}}{\pgfqpoint{3.080622in}{1.752190in}}{\pgfqpoint{3.089831in}{1.761398in}}%
\pgfpathcurveto{\pgfqpoint{3.099039in}{1.770607in}}{\pgfqpoint{3.104213in}{1.783098in}}{\pgfqpoint{3.104213in}{1.796121in}}%
\pgfpathcurveto{\pgfqpoint{3.104213in}{1.809143in}}{\pgfqpoint{3.099039in}{1.821634in}}{\pgfqpoint{3.089831in}{1.830843in}}%
\pgfpathcurveto{\pgfqpoint{3.080622in}{1.840051in}}{\pgfqpoint{3.068131in}{1.845225in}}{\pgfqpoint{3.055108in}{1.845225in}}%
\pgfpathcurveto{\pgfqpoint{3.042086in}{1.845225in}}{\pgfqpoint{3.029595in}{1.840051in}}{\pgfqpoint{3.020386in}{1.830843in}}%
\pgfpathcurveto{\pgfqpoint{3.011178in}{1.821634in}}{\pgfqpoint{3.006004in}{1.809143in}}{\pgfqpoint{3.006004in}{1.796121in}}%
\pgfpathcurveto{\pgfqpoint{3.006004in}{1.783098in}}{\pgfqpoint{3.011178in}{1.770607in}}{\pgfqpoint{3.020386in}{1.761398in}}%
\pgfpathcurveto{\pgfqpoint{3.029595in}{1.752190in}}{\pgfqpoint{3.042086in}{1.747016in}}{\pgfqpoint{3.055108in}{1.747016in}}%
\pgfpathlineto{\pgfqpoint{3.055108in}{1.747016in}}%
\pgfpathclose%
\pgfusepath{stroke,fill}%
\end{pgfscope}%
\begin{pgfscope}%
\pgfpathrectangle{\pgfqpoint{1.150000in}{0.150000in}}{\pgfqpoint{5.700000in}{5.700000in}}%
\pgfusepath{clip}%
\pgfsetbuttcap%
\pgfsetroundjoin%
\definecolor{currentfill}{rgb}{1.000000,0.000000,0.000000}%
\pgfsetfillcolor{currentfill}%
\pgfsetfillopacity{0.797645}%
\pgfsetlinewidth{1.003750pt}%
\definecolor{currentstroke}{rgb}{1.000000,0.000000,0.000000}%
\pgfsetstrokecolor{currentstroke}%
\pgfsetstrokeopacity{0.797645}%
\pgfsetdash{}{0pt}%
\pgfpathmoveto{\pgfqpoint{5.157229in}{2.323285in}}%
\pgfpathcurveto{\pgfqpoint{5.170251in}{2.323285in}}{\pgfqpoint{5.182742in}{2.328459in}}{\pgfqpoint{5.191951in}{2.337667in}}%
\pgfpathcurveto{\pgfqpoint{5.201159in}{2.346876in}}{\pgfqpoint{5.206333in}{2.359367in}}{\pgfqpoint{5.206333in}{2.372390in}}%
\pgfpathcurveto{\pgfqpoint{5.206333in}{2.385412in}}{\pgfqpoint{5.201159in}{2.397903in}}{\pgfqpoint{5.191951in}{2.407112in}}%
\pgfpathcurveto{\pgfqpoint{5.182742in}{2.416320in}}{\pgfqpoint{5.170251in}{2.421494in}}{\pgfqpoint{5.157229in}{2.421494in}}%
\pgfpathcurveto{\pgfqpoint{5.144206in}{2.421494in}}{\pgfqpoint{5.131715in}{2.416320in}}{\pgfqpoint{5.122506in}{2.407112in}}%
\pgfpathcurveto{\pgfqpoint{5.113298in}{2.397903in}}{\pgfqpoint{5.108124in}{2.385412in}}{\pgfqpoint{5.108124in}{2.372390in}}%
\pgfpathcurveto{\pgfqpoint{5.108124in}{2.359367in}}{\pgfqpoint{5.113298in}{2.346876in}}{\pgfqpoint{5.122506in}{2.337667in}}%
\pgfpathcurveto{\pgfqpoint{5.131715in}{2.328459in}}{\pgfqpoint{5.144206in}{2.323285in}}{\pgfqpoint{5.157229in}{2.323285in}}%
\pgfpathlineto{\pgfqpoint{5.157229in}{2.323285in}}%
\pgfpathclose%
\pgfusepath{stroke,fill}%
\end{pgfscope}%
\begin{pgfscope}%
\pgfpathrectangle{\pgfqpoint{1.150000in}{0.150000in}}{\pgfqpoint{5.700000in}{5.700000in}}%
\pgfusepath{clip}%
\pgfsetbuttcap%
\pgfsetroundjoin%
\definecolor{currentfill}{rgb}{1.000000,0.000000,0.000000}%
\pgfsetfillcolor{currentfill}%
\pgfsetfillopacity{0.937672}%
\pgfsetlinewidth{1.003750pt}%
\definecolor{currentstroke}{rgb}{1.000000,0.000000,0.000000}%
\pgfsetstrokecolor{currentstroke}%
\pgfsetstrokeopacity{0.937672}%
\pgfsetdash{}{0pt}%
\pgfpathmoveto{\pgfqpoint{3.274671in}{1.537191in}}%
\pgfpathcurveto{\pgfqpoint{3.287694in}{1.537191in}}{\pgfqpoint{3.300185in}{1.542365in}}{\pgfqpoint{3.309393in}{1.551574in}}%
\pgfpathcurveto{\pgfqpoint{3.318602in}{1.560782in}}{\pgfqpoint{3.323776in}{1.573273in}}{\pgfqpoint{3.323776in}{1.586296in}}%
\pgfpathcurveto{\pgfqpoint{3.323776in}{1.599319in}}{\pgfqpoint{3.318602in}{1.611810in}}{\pgfqpoint{3.309393in}{1.621018in}}%
\pgfpathcurveto{\pgfqpoint{3.300185in}{1.630227in}}{\pgfqpoint{3.287694in}{1.635401in}}{\pgfqpoint{3.274671in}{1.635401in}}%
\pgfpathcurveto{\pgfqpoint{3.261648in}{1.635401in}}{\pgfqpoint{3.249157in}{1.630227in}}{\pgfqpoint{3.239949in}{1.621018in}}%
\pgfpathcurveto{\pgfqpoint{3.230740in}{1.611810in}}{\pgfqpoint{3.225566in}{1.599319in}}{\pgfqpoint{3.225566in}{1.586296in}}%
\pgfpathcurveto{\pgfqpoint{3.225566in}{1.573273in}}{\pgfqpoint{3.230740in}{1.560782in}}{\pgfqpoint{3.239949in}{1.551574in}}%
\pgfpathcurveto{\pgfqpoint{3.249157in}{1.542365in}}{\pgfqpoint{3.261648in}{1.537191in}}{\pgfqpoint{3.274671in}{1.537191in}}%
\pgfpathlineto{\pgfqpoint{3.274671in}{1.537191in}}%
\pgfpathclose%
\pgfusepath{stroke,fill}%
\end{pgfscope}%
\begin{pgfscope}%
\pgfpathrectangle{\pgfqpoint{1.150000in}{0.150000in}}{\pgfqpoint{5.700000in}{5.700000in}}%
\pgfusepath{clip}%
\pgfsetbuttcap%
\pgfsetroundjoin%
\definecolor{currentfill}{rgb}{1.000000,0.000000,0.000000}%
\pgfsetfillcolor{currentfill}%
\pgfsetlinewidth{1.003750pt}%
\definecolor{currentstroke}{rgb}{1.000000,0.000000,0.000000}%
\pgfsetstrokecolor{currentstroke}%
\pgfsetdash{}{0pt}%
\pgfpathmoveto{\pgfqpoint{3.325628in}{1.457344in}}%
\pgfpathcurveto{\pgfqpoint{3.338651in}{1.457344in}}{\pgfqpoint{3.351142in}{1.462518in}}{\pgfqpoint{3.360350in}{1.471726in}}%
\pgfpathcurveto{\pgfqpoint{3.369559in}{1.480935in}}{\pgfqpoint{3.374732in}{1.493426in}}{\pgfqpoint{3.374732in}{1.506449in}}%
\pgfpathcurveto{\pgfqpoint{3.374732in}{1.519471in}}{\pgfqpoint{3.369559in}{1.531962in}}{\pgfqpoint{3.360350in}{1.541171in}}%
\pgfpathcurveto{\pgfqpoint{3.351142in}{1.550379in}}{\pgfqpoint{3.338651in}{1.555553in}}{\pgfqpoint{3.325628in}{1.555553in}}%
\pgfpathcurveto{\pgfqpoint{3.312605in}{1.555553in}}{\pgfqpoint{3.300114in}{1.550379in}}{\pgfqpoint{3.290906in}{1.541171in}}%
\pgfpathcurveto{\pgfqpoint{3.281697in}{1.531962in}}{\pgfqpoint{3.276523in}{1.519471in}}{\pgfqpoint{3.276523in}{1.506449in}}%
\pgfpathcurveto{\pgfqpoint{3.276523in}{1.493426in}}{\pgfqpoint{3.281697in}{1.480935in}}{\pgfqpoint{3.290906in}{1.471726in}}%
\pgfpathcurveto{\pgfqpoint{3.300114in}{1.462518in}}{\pgfqpoint{3.312605in}{1.457344in}}{\pgfqpoint{3.325628in}{1.457344in}}%
\pgfpathlineto{\pgfqpoint{3.325628in}{1.457344in}}%
\pgfpathclose%
\pgfusepath{stroke,fill}%
\end{pgfscope}%
\begin{pgfscope}%
\pgfpathrectangle{\pgfqpoint{1.150000in}{0.150000in}}{\pgfqpoint{5.700000in}{5.700000in}}%
\pgfusepath{clip}%
\pgfsetbuttcap%
\pgfsetroundjoin%
\definecolor{currentfill}{rgb}{0.121148,0.592739,0.544641}%
\pgfsetfillcolor{currentfill}%
\pgfsetfillopacity{0.700000}%
\pgfsetlinewidth{0.000000pt}%
\definecolor{currentstroke}{rgb}{0.000000,0.000000,0.000000}%
\pgfsetstrokecolor{currentstroke}%
\pgfsetdash{}{0pt}%
\pgfpathmoveto{\pgfqpoint{3.249301in}{3.881386in}}%
\pgfpathlineto{\pgfqpoint{3.262598in}{3.865868in}}%
\pgfpathlineto{\pgfqpoint{3.275893in}{3.850491in}}%
\pgfpathlineto{\pgfqpoint{3.289185in}{3.835254in}}%
\pgfpathlineto{\pgfqpoint{3.302476in}{3.820155in}}%
\pgfpathlineto{\pgfqpoint{3.310161in}{3.846101in}}%
\pgfpathlineto{\pgfqpoint{3.317839in}{3.872475in}}%
\pgfpathlineto{\pgfqpoint{3.325512in}{3.899284in}}%
\pgfpathlineto{\pgfqpoint{3.333178in}{3.926539in}}%
\pgfpathlineto{\pgfqpoint{3.319881in}{3.942074in}}%
\pgfpathlineto{\pgfqpoint{3.306582in}{3.957747in}}%
\pgfpathlineto{\pgfqpoint{3.293281in}{3.973561in}}%
\pgfpathlineto{\pgfqpoint{3.279977in}{3.989516in}}%
\pgfpathlineto{\pgfqpoint{3.272318in}{3.961816in}}%
\pgfpathlineto{\pgfqpoint{3.264652in}{3.934566in}}%
\pgfpathlineto{\pgfqpoint{3.256980in}{3.907759in}}%
\pgfpathlineto{\pgfqpoint{3.249301in}{3.881386in}}%
\pgfpathclose%
\pgfusepath{fill}%
\end{pgfscope}%
\begin{pgfscope}%
\pgfpathrectangle{\pgfqpoint{1.150000in}{0.150000in}}{\pgfqpoint{5.700000in}{5.700000in}}%
\pgfusepath{clip}%
\pgfsetbuttcap%
\pgfsetroundjoin%
\definecolor{currentfill}{rgb}{0.126453,0.570633,0.549841}%
\pgfsetfillcolor{currentfill}%
\pgfsetfillopacity{0.700000}%
\pgfsetlinewidth{0.000000pt}%
\definecolor{currentstroke}{rgb}{0.000000,0.000000,0.000000}%
\pgfsetstrokecolor{currentstroke}%
\pgfsetdash{}{0pt}%
\pgfpathmoveto{\pgfqpoint{3.302476in}{3.820155in}}%
\pgfpathlineto{\pgfqpoint{3.315765in}{3.805194in}}%
\pgfpathlineto{\pgfqpoint{3.329053in}{3.790368in}}%
\pgfpathlineto{\pgfqpoint{3.342338in}{3.775676in}}%
\pgfpathlineto{\pgfqpoint{3.355623in}{3.761118in}}%
\pgfpathlineto{\pgfqpoint{3.363313in}{3.786638in}}%
\pgfpathlineto{\pgfqpoint{3.370997in}{3.812579in}}%
\pgfpathlineto{\pgfqpoint{3.378675in}{3.838952in}}%
\pgfpathlineto{\pgfqpoint{3.386348in}{3.865763in}}%
\pgfpathlineto{\pgfqpoint{3.373058in}{3.880755in}}%
\pgfpathlineto{\pgfqpoint{3.359767in}{3.895881in}}%
\pgfpathlineto{\pgfqpoint{3.346473in}{3.911142in}}%
\pgfpathlineto{\pgfqpoint{3.333178in}{3.926539in}}%
\pgfpathlineto{\pgfqpoint{3.325512in}{3.899284in}}%
\pgfpathlineto{\pgfqpoint{3.317839in}{3.872475in}}%
\pgfpathlineto{\pgfqpoint{3.310161in}{3.846101in}}%
\pgfpathlineto{\pgfqpoint{3.302476in}{3.820155in}}%
\pgfpathclose%
\pgfusepath{fill}%
\end{pgfscope}%
\begin{pgfscope}%
\pgfpathrectangle{\pgfqpoint{1.150000in}{0.150000in}}{\pgfqpoint{5.700000in}{5.700000in}}%
\pgfusepath{clip}%
\pgfsetbuttcap%
\pgfsetroundjoin%
\definecolor{currentfill}{rgb}{0.131172,0.555899,0.552459}%
\pgfsetfillcolor{currentfill}%
\pgfsetfillopacity{0.700000}%
\pgfsetlinewidth{0.000000pt}%
\definecolor{currentstroke}{rgb}{0.000000,0.000000,0.000000}%
\pgfsetstrokecolor{currentstroke}%
\pgfsetdash{}{0pt}%
\pgfpathmoveto{\pgfqpoint{3.218518in}{3.780072in}}%
\pgfpathlineto{\pgfqpoint{3.231810in}{3.764969in}}%
\pgfpathlineto{\pgfqpoint{3.245100in}{3.750005in}}%
\pgfpathlineto{\pgfqpoint{3.258388in}{3.735181in}}%
\pgfpathlineto{\pgfqpoint{3.271674in}{3.720495in}}%
\pgfpathlineto{\pgfqpoint{3.279384in}{3.744807in}}%
\pgfpathlineto{\pgfqpoint{3.287088in}{3.769517in}}%
\pgfpathlineto{\pgfqpoint{3.294785in}{3.794630in}}%
\pgfpathlineto{\pgfqpoint{3.302476in}{3.820155in}}%
\pgfpathlineto{\pgfqpoint{3.289185in}{3.835254in}}%
\pgfpathlineto{\pgfqpoint{3.275893in}{3.850491in}}%
\pgfpathlineto{\pgfqpoint{3.262598in}{3.865868in}}%
\pgfpathlineto{\pgfqpoint{3.249301in}{3.881386in}}%
\pgfpathlineto{\pgfqpoint{3.241615in}{3.855438in}}%
\pgfpathlineto{\pgfqpoint{3.233923in}{3.829909in}}%
\pgfpathlineto{\pgfqpoint{3.226224in}{3.804790in}}%
\pgfpathlineto{\pgfqpoint{3.218518in}{3.780072in}}%
\pgfpathclose%
\pgfusepath{fill}%
\end{pgfscope}%
\begin{pgfscope}%
\pgfpathrectangle{\pgfqpoint{1.150000in}{0.150000in}}{\pgfqpoint{5.700000in}{5.700000in}}%
\pgfusepath{clip}%
\pgfsetbuttcap%
\pgfsetroundjoin%
\definecolor{currentfill}{rgb}{0.133743,0.548535,0.553541}%
\pgfsetfillcolor{currentfill}%
\pgfsetfillopacity{0.700000}%
\pgfsetlinewidth{0.000000pt}%
\definecolor{currentstroke}{rgb}{0.000000,0.000000,0.000000}%
\pgfsetstrokecolor{currentstroke}%
\pgfsetdash{}{0pt}%
\pgfpathmoveto{\pgfqpoint{3.355623in}{3.761118in}}%
\pgfpathlineto{\pgfqpoint{3.368906in}{3.746692in}}%
\pgfpathlineto{\pgfqpoint{3.382187in}{3.732397in}}%
\pgfpathlineto{\pgfqpoint{3.395468in}{3.718231in}}%
\pgfpathlineto{\pgfqpoint{3.408748in}{3.704195in}}%
\pgfpathlineto{\pgfqpoint{3.416442in}{3.729290in}}%
\pgfpathlineto{\pgfqpoint{3.424132in}{3.754802in}}%
\pgfpathlineto{\pgfqpoint{3.431816in}{3.780739in}}%
\pgfpathlineto{\pgfqpoint{3.439495in}{3.807108in}}%
\pgfpathlineto{\pgfqpoint{3.426210in}{3.821577in}}%
\pgfpathlineto{\pgfqpoint{3.412924in}{3.836175in}}%
\pgfpathlineto{\pgfqpoint{3.399637in}{3.850903in}}%
\pgfpathlineto{\pgfqpoint{3.386348in}{3.865763in}}%
\pgfpathlineto{\pgfqpoint{3.378675in}{3.838952in}}%
\pgfpathlineto{\pgfqpoint{3.370997in}{3.812579in}}%
\pgfpathlineto{\pgfqpoint{3.363313in}{3.786638in}}%
\pgfpathlineto{\pgfqpoint{3.355623in}{3.761118in}}%
\pgfpathclose%
\pgfusepath{fill}%
\end{pgfscope}%
\begin{pgfscope}%
\pgfpathrectangle{\pgfqpoint{1.150000in}{0.150000in}}{\pgfqpoint{5.700000in}{5.700000in}}%
\pgfusepath{clip}%
\pgfsetbuttcap%
\pgfsetroundjoin%
\definecolor{currentfill}{rgb}{0.119423,0.611141,0.538982}%
\pgfsetfillcolor{currentfill}%
\pgfsetfillopacity{0.700000}%
\pgfsetlinewidth{0.000000pt}%
\definecolor{currentstroke}{rgb}{0.000000,0.000000,0.000000}%
\pgfsetstrokecolor{currentstroke}%
\pgfsetdash{}{0pt}%
\pgfpathmoveto{\pgfqpoint{3.333178in}{3.926539in}}%
\pgfpathlineto{\pgfqpoint{3.346473in}{3.911142in}}%
\pgfpathlineto{\pgfqpoint{3.359767in}{3.895881in}}%
\pgfpathlineto{\pgfqpoint{3.373058in}{3.880755in}}%
\pgfpathlineto{\pgfqpoint{3.386348in}{3.865763in}}%
\pgfpathlineto{\pgfqpoint{3.394016in}{3.893020in}}%
\pgfpathlineto{\pgfqpoint{3.401677in}{3.920734in}}%
\pgfpathlineto{\pgfqpoint{3.409334in}{3.948912in}}%
\pgfpathlineto{\pgfqpoint{3.416985in}{3.977563in}}%
\pgfpathlineto{\pgfqpoint{3.403688in}{3.993013in}}%
\pgfpathlineto{\pgfqpoint{3.390389in}{4.008598in}}%
\pgfpathlineto{\pgfqpoint{3.377088in}{4.024318in}}%
\pgfpathlineto{\pgfqpoint{3.363785in}{4.040175in}}%
\pgfpathlineto{\pgfqpoint{3.356142in}{4.011056in}}%
\pgfpathlineto{\pgfqpoint{3.348493in}{3.982416in}}%
\pgfpathlineto{\pgfqpoint{3.340839in}{3.954247in}}%
\pgfpathlineto{\pgfqpoint{3.333178in}{3.926539in}}%
\pgfpathclose%
\pgfusepath{fill}%
\end{pgfscope}%
\begin{pgfscope}%
\pgfpathrectangle{\pgfqpoint{1.150000in}{0.150000in}}{\pgfqpoint{5.700000in}{5.700000in}}%
\pgfusepath{clip}%
\pgfsetbuttcap%
\pgfsetroundjoin%
\definecolor{currentfill}{rgb}{0.121831,0.589055,0.545623}%
\pgfsetfillcolor{currentfill}%
\pgfsetfillopacity{0.700000}%
\pgfsetlinewidth{0.000000pt}%
\definecolor{currentstroke}{rgb}{0.000000,0.000000,0.000000}%
\pgfsetstrokecolor{currentstroke}%
\pgfsetdash{}{0pt}%
\pgfpathmoveto{\pgfqpoint{3.386348in}{3.865763in}}%
\pgfpathlineto{\pgfqpoint{3.399637in}{3.850903in}}%
\pgfpathlineto{\pgfqpoint{3.412924in}{3.836175in}}%
\pgfpathlineto{\pgfqpoint{3.426210in}{3.821577in}}%
\pgfpathlineto{\pgfqpoint{3.439495in}{3.807108in}}%
\pgfpathlineto{\pgfqpoint{3.447168in}{3.833918in}}%
\pgfpathlineto{\pgfqpoint{3.454837in}{3.861178in}}%
\pgfpathlineto{\pgfqpoint{3.462500in}{3.888896in}}%
\pgfpathlineto{\pgfqpoint{3.470158in}{3.917082in}}%
\pgfpathlineto{\pgfqpoint{3.456867in}{3.932006in}}%
\pgfpathlineto{\pgfqpoint{3.443574in}{3.947061in}}%
\pgfpathlineto{\pgfqpoint{3.430280in}{3.962246in}}%
\pgfpathlineto{\pgfqpoint{3.416985in}{3.977563in}}%
\pgfpathlineto{\pgfqpoint{3.409334in}{3.948912in}}%
\pgfpathlineto{\pgfqpoint{3.401677in}{3.920734in}}%
\pgfpathlineto{\pgfqpoint{3.394016in}{3.893020in}}%
\pgfpathlineto{\pgfqpoint{3.386348in}{3.865763in}}%
\pgfpathclose%
\pgfusepath{fill}%
\end{pgfscope}%
\begin{pgfscope}%
\pgfpathrectangle{\pgfqpoint{1.150000in}{0.150000in}}{\pgfqpoint{5.700000in}{5.700000in}}%
\pgfusepath{clip}%
\pgfsetbuttcap%
\pgfsetroundjoin%
\definecolor{currentfill}{rgb}{0.122312,0.633153,0.530398}%
\pgfsetfillcolor{currentfill}%
\pgfsetfillopacity{0.700000}%
\pgfsetlinewidth{0.000000pt}%
\definecolor{currentstroke}{rgb}{0.000000,0.000000,0.000000}%
\pgfsetstrokecolor{currentstroke}%
\pgfsetdash{}{0pt}%
\pgfpathmoveto{\pgfqpoint{3.279977in}{3.989516in}}%
\pgfpathlineto{\pgfqpoint{3.293281in}{3.973561in}}%
\pgfpathlineto{\pgfqpoint{3.306582in}{3.957747in}}%
\pgfpathlineto{\pgfqpoint{3.319881in}{3.942074in}}%
\pgfpathlineto{\pgfqpoint{3.333178in}{3.926539in}}%
\pgfpathlineto{\pgfqpoint{3.340839in}{3.954247in}}%
\pgfpathlineto{\pgfqpoint{3.348493in}{3.982416in}}%
\pgfpathlineto{\pgfqpoint{3.356142in}{4.011056in}}%
\pgfpathlineto{\pgfqpoint{3.363785in}{4.040175in}}%
\pgfpathlineto{\pgfqpoint{3.350480in}{4.056170in}}%
\pgfpathlineto{\pgfqpoint{3.337173in}{4.072304in}}%
\pgfpathlineto{\pgfqpoint{3.323864in}{4.088579in}}%
\pgfpathlineto{\pgfqpoint{3.310552in}{4.104996in}}%
\pgfpathlineto{\pgfqpoint{3.302917in}{4.075407in}}%
\pgfpathlineto{\pgfqpoint{3.295277in}{4.046303in}}%
\pgfpathlineto{\pgfqpoint{3.287630in}{4.017676in}}%
\pgfpathlineto{\pgfqpoint{3.279977in}{3.989516in}}%
\pgfpathclose%
\pgfusepath{fill}%
\end{pgfscope}%
\begin{pgfscope}%
\pgfpathrectangle{\pgfqpoint{1.150000in}{0.150000in}}{\pgfqpoint{5.700000in}{5.700000in}}%
\pgfusepath{clip}%
\pgfsetbuttcap%
\pgfsetroundjoin%
\definecolor{currentfill}{rgb}{0.139147,0.533812,0.555298}%
\pgfsetfillcolor{currentfill}%
\pgfsetfillopacity{0.700000}%
\pgfsetlinewidth{0.000000pt}%
\definecolor{currentstroke}{rgb}{0.000000,0.000000,0.000000}%
\pgfsetstrokecolor{currentstroke}%
\pgfsetdash{}{0pt}%
\pgfpathmoveto{\pgfqpoint{3.271674in}{3.720495in}}%
\pgfpathlineto{\pgfqpoint{3.284958in}{3.705945in}}%
\pgfpathlineto{\pgfqpoint{3.298241in}{3.691530in}}%
\pgfpathlineto{\pgfqpoint{3.311523in}{3.677250in}}%
\pgfpathlineto{\pgfqpoint{3.324803in}{3.663103in}}%
\pgfpathlineto{\pgfqpoint{3.332517in}{3.687013in}}%
\pgfpathlineto{\pgfqpoint{3.340225in}{3.711313in}}%
\pgfpathlineto{\pgfqpoint{3.347927in}{3.736013in}}%
\pgfpathlineto{\pgfqpoint{3.355623in}{3.761118in}}%
\pgfpathlineto{\pgfqpoint{3.342338in}{3.775676in}}%
\pgfpathlineto{\pgfqpoint{3.329053in}{3.790368in}}%
\pgfpathlineto{\pgfqpoint{3.315765in}{3.805194in}}%
\pgfpathlineto{\pgfqpoint{3.302476in}{3.820155in}}%
\pgfpathlineto{\pgfqpoint{3.294785in}{3.794630in}}%
\pgfpathlineto{\pgfqpoint{3.287088in}{3.769517in}}%
\pgfpathlineto{\pgfqpoint{3.279384in}{3.744807in}}%
\pgfpathlineto{\pgfqpoint{3.271674in}{3.720495in}}%
\pgfpathclose%
\pgfusepath{fill}%
\end{pgfscope}%
\begin{pgfscope}%
\pgfpathrectangle{\pgfqpoint{1.150000in}{0.150000in}}{\pgfqpoint{5.700000in}{5.700000in}}%
\pgfusepath{clip}%
\pgfsetbuttcap%
\pgfsetroundjoin%
\definecolor{currentfill}{rgb}{0.127568,0.566949,0.550556}%
\pgfsetfillcolor{currentfill}%
\pgfsetfillopacity{0.700000}%
\pgfsetlinewidth{0.000000pt}%
\definecolor{currentstroke}{rgb}{0.000000,0.000000,0.000000}%
\pgfsetstrokecolor{currentstroke}%
\pgfsetdash{}{0pt}%
\pgfpathmoveto{\pgfqpoint{3.439495in}{3.807108in}}%
\pgfpathlineto{\pgfqpoint{3.452778in}{3.792767in}}%
\pgfpathlineto{\pgfqpoint{3.466061in}{3.778552in}}%
\pgfpathlineto{\pgfqpoint{3.479343in}{3.764464in}}%
\pgfpathlineto{\pgfqpoint{3.492624in}{3.750500in}}%
\pgfpathlineto{\pgfqpoint{3.500303in}{3.776865in}}%
\pgfpathlineto{\pgfqpoint{3.507978in}{3.803673in}}%
\pgfpathlineto{\pgfqpoint{3.515648in}{3.830934in}}%
\pgfpathlineto{\pgfqpoint{3.523313in}{3.858656in}}%
\pgfpathlineto{\pgfqpoint{3.510026in}{3.873073in}}%
\pgfpathlineto{\pgfqpoint{3.496738in}{3.887616in}}%
\pgfpathlineto{\pgfqpoint{3.483449in}{3.902285in}}%
\pgfpathlineto{\pgfqpoint{3.470158in}{3.917082in}}%
\pgfpathlineto{\pgfqpoint{3.462500in}{3.888896in}}%
\pgfpathlineto{\pgfqpoint{3.454837in}{3.861178in}}%
\pgfpathlineto{\pgfqpoint{3.447168in}{3.833918in}}%
\pgfpathlineto{\pgfqpoint{3.439495in}{3.807108in}}%
\pgfpathclose%
\pgfusepath{fill}%
\end{pgfscope}%
\begin{pgfscope}%
\pgfpathrectangle{\pgfqpoint{1.150000in}{0.150000in}}{\pgfqpoint{5.700000in}{5.700000in}}%
\pgfusepath{clip}%
\pgfsetbuttcap%
\pgfsetroundjoin%
\definecolor{currentfill}{rgb}{0.141935,0.526453,0.555991}%
\pgfsetfillcolor{currentfill}%
\pgfsetfillopacity{0.700000}%
\pgfsetlinewidth{0.000000pt}%
\definecolor{currentstroke}{rgb}{0.000000,0.000000,0.000000}%
\pgfsetstrokecolor{currentstroke}%
\pgfsetdash{}{0pt}%
\pgfpathmoveto{\pgfqpoint{3.408748in}{3.704195in}}%
\pgfpathlineto{\pgfqpoint{3.422026in}{3.690285in}}%
\pgfpathlineto{\pgfqpoint{3.435304in}{3.676502in}}%
\pgfpathlineto{\pgfqpoint{3.448581in}{3.662845in}}%
\pgfpathlineto{\pgfqpoint{3.461858in}{3.649311in}}%
\pgfpathlineto{\pgfqpoint{3.469557in}{3.673984in}}%
\pgfpathlineto{\pgfqpoint{3.477251in}{3.699068in}}%
\pgfpathlineto{\pgfqpoint{3.484940in}{3.724571in}}%
\pgfpathlineto{\pgfqpoint{3.492624in}{3.750500in}}%
\pgfpathlineto{\pgfqpoint{3.479343in}{3.764464in}}%
\pgfpathlineto{\pgfqpoint{3.466061in}{3.778552in}}%
\pgfpathlineto{\pgfqpoint{3.452778in}{3.792767in}}%
\pgfpathlineto{\pgfqpoint{3.439495in}{3.807108in}}%
\pgfpathlineto{\pgfqpoint{3.431816in}{3.780739in}}%
\pgfpathlineto{\pgfqpoint{3.424132in}{3.754802in}}%
\pgfpathlineto{\pgfqpoint{3.416442in}{3.729290in}}%
\pgfpathlineto{\pgfqpoint{3.408748in}{3.704195in}}%
\pgfpathclose%
\pgfusepath{fill}%
\end{pgfscope}%
\begin{pgfscope}%
\pgfpathrectangle{\pgfqpoint{1.150000in}{0.150000in}}{\pgfqpoint{5.700000in}{5.700000in}}%
\pgfusepath{clip}%
\pgfsetbuttcap%
\pgfsetroundjoin%
\definecolor{currentfill}{rgb}{0.147607,0.511733,0.557049}%
\pgfsetfillcolor{currentfill}%
\pgfsetfillopacity{0.700000}%
\pgfsetlinewidth{0.000000pt}%
\definecolor{currentstroke}{rgb}{0.000000,0.000000,0.000000}%
\pgfsetstrokecolor{currentstroke}%
\pgfsetdash{}{0pt}%
\pgfpathmoveto{\pgfqpoint{3.324803in}{3.663103in}}%
\pgfpathlineto{\pgfqpoint{3.338081in}{3.649087in}}%
\pgfpathlineto{\pgfqpoint{3.351359in}{3.635201in}}%
\pgfpathlineto{\pgfqpoint{3.364636in}{3.621445in}}%
\pgfpathlineto{\pgfqpoint{3.377912in}{3.607817in}}%
\pgfpathlineto{\pgfqpoint{3.385629in}{3.631327in}}%
\pgfpathlineto{\pgfqpoint{3.393341in}{3.655221in}}%
\pgfpathlineto{\pgfqpoint{3.401047in}{3.679507in}}%
\pgfpathlineto{\pgfqpoint{3.408748in}{3.704195in}}%
\pgfpathlineto{\pgfqpoint{3.395468in}{3.718231in}}%
\pgfpathlineto{\pgfqpoint{3.382187in}{3.732397in}}%
\pgfpathlineto{\pgfqpoint{3.368906in}{3.746692in}}%
\pgfpathlineto{\pgfqpoint{3.355623in}{3.761118in}}%
\pgfpathlineto{\pgfqpoint{3.347927in}{3.736013in}}%
\pgfpathlineto{\pgfqpoint{3.340225in}{3.711313in}}%
\pgfpathlineto{\pgfqpoint{3.332517in}{3.687013in}}%
\pgfpathlineto{\pgfqpoint{3.324803in}{3.663103in}}%
\pgfpathclose%
\pgfusepath{fill}%
\end{pgfscope}%
\begin{pgfscope}%
\pgfpathrectangle{\pgfqpoint{1.150000in}{0.150000in}}{\pgfqpoint{5.700000in}{5.700000in}}%
\pgfusepath{clip}%
\pgfsetbuttcap%
\pgfsetroundjoin%
\definecolor{currentfill}{rgb}{0.144759,0.519093,0.556572}%
\pgfsetfillcolor{currentfill}%
\pgfsetfillopacity{0.700000}%
\pgfsetlinewidth{0.000000pt}%
\definecolor{currentstroke}{rgb}{0.000000,0.000000,0.000000}%
\pgfsetstrokecolor{currentstroke}%
\pgfsetdash{}{0pt}%
\pgfpathmoveto{\pgfqpoint{3.187626in}{3.685077in}}%
\pgfpathlineto{\pgfqpoint{3.200914in}{3.670363in}}%
\pgfpathlineto{\pgfqpoint{3.214200in}{3.655790in}}%
\pgfpathlineto{\pgfqpoint{3.227484in}{3.641356in}}%
\pgfpathlineto{\pgfqpoint{3.240767in}{3.627059in}}%
\pgfpathlineto{\pgfqpoint{3.248504in}{3.649860in}}%
\pgfpathlineto{\pgfqpoint{3.256234in}{3.673029in}}%
\pgfpathlineto{\pgfqpoint{3.263957in}{3.696571in}}%
\pgfpathlineto{\pgfqpoint{3.271674in}{3.720495in}}%
\pgfpathlineto{\pgfqpoint{3.258388in}{3.735181in}}%
\pgfpathlineto{\pgfqpoint{3.245100in}{3.750005in}}%
\pgfpathlineto{\pgfqpoint{3.231810in}{3.764969in}}%
\pgfpathlineto{\pgfqpoint{3.218518in}{3.780072in}}%
\pgfpathlineto{\pgfqpoint{3.210806in}{3.755750in}}%
\pgfpathlineto{\pgfqpoint{3.203086in}{3.731815in}}%
\pgfpathlineto{\pgfqpoint{3.195360in}{3.708260in}}%
\pgfpathlineto{\pgfqpoint{3.187626in}{3.685077in}}%
\pgfpathclose%
\pgfusepath{fill}%
\end{pgfscope}%
\begin{pgfscope}%
\pgfpathrectangle{\pgfqpoint{1.150000in}{0.150000in}}{\pgfqpoint{5.700000in}{5.700000in}}%
\pgfusepath{clip}%
\pgfsetbuttcap%
\pgfsetroundjoin%
\definecolor{currentfill}{rgb}{0.135066,0.544853,0.554029}%
\pgfsetfillcolor{currentfill}%
\pgfsetfillopacity{0.700000}%
\pgfsetlinewidth{0.000000pt}%
\definecolor{currentstroke}{rgb}{0.000000,0.000000,0.000000}%
\pgfsetstrokecolor{currentstroke}%
\pgfsetdash{}{0pt}%
\pgfpathmoveto{\pgfqpoint{3.492624in}{3.750500in}}%
\pgfpathlineto{\pgfqpoint{3.505905in}{3.736660in}}%
\pgfpathlineto{\pgfqpoint{3.519185in}{3.722942in}}%
\pgfpathlineto{\pgfqpoint{3.532464in}{3.709346in}}%
\pgfpathlineto{\pgfqpoint{3.545743in}{3.695870in}}%
\pgfpathlineto{\pgfqpoint{3.553428in}{3.721791in}}%
\pgfpathlineto{\pgfqpoint{3.561108in}{3.748149in}}%
\pgfpathlineto{\pgfqpoint{3.568783in}{3.774954in}}%
\pgfpathlineto{\pgfqpoint{3.576455in}{3.802215in}}%
\pgfpathlineto{\pgfqpoint{3.563170in}{3.816142in}}%
\pgfpathlineto{\pgfqpoint{3.549885in}{3.830191in}}%
\pgfpathlineto{\pgfqpoint{3.536599in}{3.844362in}}%
\pgfpathlineto{\pgfqpoint{3.523313in}{3.858656in}}%
\pgfpathlineto{\pgfqpoint{3.515648in}{3.830934in}}%
\pgfpathlineto{\pgfqpoint{3.507978in}{3.803673in}}%
\pgfpathlineto{\pgfqpoint{3.500303in}{3.776865in}}%
\pgfpathlineto{\pgfqpoint{3.492624in}{3.750500in}}%
\pgfpathclose%
\pgfusepath{fill}%
\end{pgfscope}%
\begin{pgfscope}%
\pgfpathrectangle{\pgfqpoint{1.150000in}{0.150000in}}{\pgfqpoint{5.700000in}{5.700000in}}%
\pgfusepath{clip}%
\pgfsetbuttcap%
\pgfsetroundjoin%
\definecolor{currentfill}{rgb}{0.122312,0.633153,0.530398}%
\pgfsetfillcolor{currentfill}%
\pgfsetfillopacity{0.700000}%
\pgfsetlinewidth{0.000000pt}%
\definecolor{currentstroke}{rgb}{0.000000,0.000000,0.000000}%
\pgfsetstrokecolor{currentstroke}%
\pgfsetdash{}{0pt}%
\pgfpathmoveto{\pgfqpoint{3.416985in}{3.977563in}}%
\pgfpathlineto{\pgfqpoint{3.430280in}{3.962246in}}%
\pgfpathlineto{\pgfqpoint{3.443574in}{3.947061in}}%
\pgfpathlineto{\pgfqpoint{3.456867in}{3.932006in}}%
\pgfpathlineto{\pgfqpoint{3.470158in}{3.917082in}}%
\pgfpathlineto{\pgfqpoint{3.477812in}{3.945743in}}%
\pgfpathlineto{\pgfqpoint{3.485461in}{3.974889in}}%
\pgfpathlineto{\pgfqpoint{3.493106in}{4.004530in}}%
\pgfpathlineto{\pgfqpoint{3.500746in}{4.034673in}}%
\pgfpathlineto{\pgfqpoint{3.487446in}{4.050078in}}%
\pgfpathlineto{\pgfqpoint{3.474145in}{4.065614in}}%
\pgfpathlineto{\pgfqpoint{3.460843in}{4.081280in}}%
\pgfpathlineto{\pgfqpoint{3.447538in}{4.097079in}}%
\pgfpathlineto{\pgfqpoint{3.439907in}{4.066445in}}%
\pgfpathlineto{\pgfqpoint{3.432272in}{4.036320in}}%
\pgfpathlineto{\pgfqpoint{3.424631in}{4.006696in}}%
\pgfpathlineto{\pgfqpoint{3.416985in}{3.977563in}}%
\pgfpathclose%
\pgfusepath{fill}%
\end{pgfscope}%
\begin{pgfscope}%
\pgfpathrectangle{\pgfqpoint{1.150000in}{0.150000in}}{\pgfqpoint{5.700000in}{5.700000in}}%
\pgfusepath{clip}%
\pgfsetbuttcap%
\pgfsetroundjoin%
\definecolor{currentfill}{rgb}{0.119423,0.611141,0.538982}%
\pgfsetfillcolor{currentfill}%
\pgfsetfillopacity{0.700000}%
\pgfsetlinewidth{0.000000pt}%
\definecolor{currentstroke}{rgb}{0.000000,0.000000,0.000000}%
\pgfsetstrokecolor{currentstroke}%
\pgfsetdash{}{0pt}%
\pgfpathmoveto{\pgfqpoint{3.470158in}{3.917082in}}%
\pgfpathlineto{\pgfqpoint{3.483449in}{3.902285in}}%
\pgfpathlineto{\pgfqpoint{3.496738in}{3.887616in}}%
\pgfpathlineto{\pgfqpoint{3.510026in}{3.873073in}}%
\pgfpathlineto{\pgfqpoint{3.523313in}{3.858656in}}%
\pgfpathlineto{\pgfqpoint{3.530974in}{3.886847in}}%
\pgfpathlineto{\pgfqpoint{3.538630in}{3.915518in}}%
\pgfpathlineto{\pgfqpoint{3.546283in}{3.944676in}}%
\pgfpathlineto{\pgfqpoint{3.553931in}{3.974332in}}%
\pgfpathlineto{\pgfqpoint{3.540637in}{3.989228in}}%
\pgfpathlineto{\pgfqpoint{3.527341in}{4.004249in}}%
\pgfpathlineto{\pgfqpoint{3.514044in}{4.019397in}}%
\pgfpathlineto{\pgfqpoint{3.500746in}{4.034673in}}%
\pgfpathlineto{\pgfqpoint{3.493106in}{4.004530in}}%
\pgfpathlineto{\pgfqpoint{3.485461in}{3.974889in}}%
\pgfpathlineto{\pgfqpoint{3.477812in}{3.945743in}}%
\pgfpathlineto{\pgfqpoint{3.470158in}{3.917082in}}%
\pgfpathclose%
\pgfusepath{fill}%
\end{pgfscope}%
\begin{pgfscope}%
\pgfpathrectangle{\pgfqpoint{1.150000in}{0.150000in}}{\pgfqpoint{5.700000in}{5.700000in}}%
\pgfusepath{clip}%
\pgfsetbuttcap%
\pgfsetroundjoin%
\definecolor{currentfill}{rgb}{0.132268,0.655014,0.519661}%
\pgfsetfillcolor{currentfill}%
\pgfsetfillopacity{0.700000}%
\pgfsetlinewidth{0.000000pt}%
\definecolor{currentstroke}{rgb}{0.000000,0.000000,0.000000}%
\pgfsetstrokecolor{currentstroke}%
\pgfsetdash{}{0pt}%
\pgfpathmoveto{\pgfqpoint{3.363785in}{4.040175in}}%
\pgfpathlineto{\pgfqpoint{3.377088in}{4.024318in}}%
\pgfpathlineto{\pgfqpoint{3.390389in}{4.008598in}}%
\pgfpathlineto{\pgfqpoint{3.403688in}{3.993013in}}%
\pgfpathlineto{\pgfqpoint{3.416985in}{3.977563in}}%
\pgfpathlineto{\pgfqpoint{3.424631in}{4.006696in}}%
\pgfpathlineto{\pgfqpoint{3.432272in}{4.036320in}}%
\pgfpathlineto{\pgfqpoint{3.439907in}{4.066445in}}%
\pgfpathlineto{\pgfqpoint{3.447538in}{4.097079in}}%
\pgfpathlineto{\pgfqpoint{3.434232in}{4.113012in}}%
\pgfpathlineto{\pgfqpoint{3.420924in}{4.129079in}}%
\pgfpathlineto{\pgfqpoint{3.407615in}{4.145283in}}%
\pgfpathlineto{\pgfqpoint{3.394302in}{4.161624in}}%
\pgfpathlineto{\pgfqpoint{3.386681in}{4.130497in}}%
\pgfpathlineto{\pgfqpoint{3.379055in}{4.099886in}}%
\pgfpathlineto{\pgfqpoint{3.371423in}{4.069782in}}%
\pgfpathlineto{\pgfqpoint{3.363785in}{4.040175in}}%
\pgfpathclose%
\pgfusepath{fill}%
\end{pgfscope}%
\begin{pgfscope}%
\pgfpathrectangle{\pgfqpoint{1.150000in}{0.150000in}}{\pgfqpoint{5.700000in}{5.700000in}}%
\pgfusepath{clip}%
\pgfsetbuttcap%
\pgfsetroundjoin%
\definecolor{currentfill}{rgb}{0.149039,0.508051,0.557250}%
\pgfsetfillcolor{currentfill}%
\pgfsetfillopacity{0.700000}%
\pgfsetlinewidth{0.000000pt}%
\definecolor{currentstroke}{rgb}{0.000000,0.000000,0.000000}%
\pgfsetstrokecolor{currentstroke}%
\pgfsetdash{}{0pt}%
\pgfpathmoveto{\pgfqpoint{3.461858in}{3.649311in}}%
\pgfpathlineto{\pgfqpoint{3.475134in}{3.635901in}}%
\pgfpathlineto{\pgfqpoint{3.488410in}{3.622613in}}%
\pgfpathlineto{\pgfqpoint{3.501685in}{3.609446in}}%
\pgfpathlineto{\pgfqpoint{3.514960in}{3.596399in}}%
\pgfpathlineto{\pgfqpoint{3.522663in}{3.620651in}}%
\pgfpathlineto{\pgfqpoint{3.530361in}{3.645308in}}%
\pgfpathlineto{\pgfqpoint{3.538055in}{3.670379in}}%
\pgfpathlineto{\pgfqpoint{3.545743in}{3.695870in}}%
\pgfpathlineto{\pgfqpoint{3.532464in}{3.709346in}}%
\pgfpathlineto{\pgfqpoint{3.519185in}{3.722942in}}%
\pgfpathlineto{\pgfqpoint{3.505905in}{3.736660in}}%
\pgfpathlineto{\pgfqpoint{3.492624in}{3.750500in}}%
\pgfpathlineto{\pgfqpoint{3.484940in}{3.724571in}}%
\pgfpathlineto{\pgfqpoint{3.477251in}{3.699068in}}%
\pgfpathlineto{\pgfqpoint{3.469557in}{3.673984in}}%
\pgfpathlineto{\pgfqpoint{3.461858in}{3.649311in}}%
\pgfpathclose%
\pgfusepath{fill}%
\end{pgfscope}%
\begin{pgfscope}%
\pgfpathrectangle{\pgfqpoint{1.150000in}{0.150000in}}{\pgfqpoint{5.700000in}{5.700000in}}%
\pgfusepath{clip}%
\pgfsetbuttcap%
\pgfsetroundjoin%
\definecolor{currentfill}{rgb}{0.154815,0.493313,0.557840}%
\pgfsetfillcolor{currentfill}%
\pgfsetfillopacity{0.700000}%
\pgfsetlinewidth{0.000000pt}%
\definecolor{currentstroke}{rgb}{0.000000,0.000000,0.000000}%
\pgfsetstrokecolor{currentstroke}%
\pgfsetdash{}{0pt}%
\pgfpathmoveto{\pgfqpoint{3.377912in}{3.607817in}}%
\pgfpathlineto{\pgfqpoint{3.391187in}{3.594317in}}%
\pgfpathlineto{\pgfqpoint{3.404461in}{3.580942in}}%
\pgfpathlineto{\pgfqpoint{3.417735in}{3.567692in}}%
\pgfpathlineto{\pgfqpoint{3.431008in}{3.554566in}}%
\pgfpathlineto{\pgfqpoint{3.438729in}{3.577676in}}%
\pgfpathlineto{\pgfqpoint{3.446444in}{3.601165in}}%
\pgfpathlineto{\pgfqpoint{3.454154in}{3.625041in}}%
\pgfpathlineto{\pgfqpoint{3.461858in}{3.649311in}}%
\pgfpathlineto{\pgfqpoint{3.448581in}{3.662845in}}%
\pgfpathlineto{\pgfqpoint{3.435304in}{3.676502in}}%
\pgfpathlineto{\pgfqpoint{3.422026in}{3.690285in}}%
\pgfpathlineto{\pgfqpoint{3.408748in}{3.704195in}}%
\pgfpathlineto{\pgfqpoint{3.401047in}{3.679507in}}%
\pgfpathlineto{\pgfqpoint{3.393341in}{3.655221in}}%
\pgfpathlineto{\pgfqpoint{3.385629in}{3.631327in}}%
\pgfpathlineto{\pgfqpoint{3.377912in}{3.607817in}}%
\pgfpathclose%
\pgfusepath{fill}%
\end{pgfscope}%
\begin{pgfscope}%
\pgfpathrectangle{\pgfqpoint{1.150000in}{0.150000in}}{\pgfqpoint{5.700000in}{5.700000in}}%
\pgfusepath{clip}%
\pgfsetbuttcap%
\pgfsetroundjoin%
\definecolor{currentfill}{rgb}{0.153364,0.497000,0.557724}%
\pgfsetfillcolor{currentfill}%
\pgfsetfillopacity{0.700000}%
\pgfsetlinewidth{0.000000pt}%
\definecolor{currentstroke}{rgb}{0.000000,0.000000,0.000000}%
\pgfsetstrokecolor{currentstroke}%
\pgfsetdash{}{0pt}%
\pgfpathmoveto{\pgfqpoint{3.240767in}{3.627059in}}%
\pgfpathlineto{\pgfqpoint{3.254048in}{3.612898in}}%
\pgfpathlineto{\pgfqpoint{3.267328in}{3.598872in}}%
\pgfpathlineto{\pgfqpoint{3.280606in}{3.584980in}}%
\pgfpathlineto{\pgfqpoint{3.293883in}{3.571220in}}%
\pgfpathlineto{\pgfqpoint{3.301623in}{3.593642in}}%
\pgfpathlineto{\pgfqpoint{3.309356in}{3.616425in}}%
\pgfpathlineto{\pgfqpoint{3.317082in}{3.639576in}}%
\pgfpathlineto{\pgfqpoint{3.324803in}{3.663103in}}%
\pgfpathlineto{\pgfqpoint{3.311523in}{3.677250in}}%
\pgfpathlineto{\pgfqpoint{3.298241in}{3.691530in}}%
\pgfpathlineto{\pgfqpoint{3.284958in}{3.705945in}}%
\pgfpathlineto{\pgfqpoint{3.271674in}{3.720495in}}%
\pgfpathlineto{\pgfqpoint{3.263957in}{3.696571in}}%
\pgfpathlineto{\pgfqpoint{3.256234in}{3.673029in}}%
\pgfpathlineto{\pgfqpoint{3.248504in}{3.649860in}}%
\pgfpathlineto{\pgfqpoint{3.240767in}{3.627059in}}%
\pgfpathclose%
\pgfusepath{fill}%
\end{pgfscope}%
\begin{pgfscope}%
\pgfpathrectangle{\pgfqpoint{1.150000in}{0.150000in}}{\pgfqpoint{5.700000in}{5.700000in}}%
\pgfusepath{clip}%
\pgfsetbuttcap%
\pgfsetroundjoin%
\definecolor{currentfill}{rgb}{0.121831,0.589055,0.545623}%
\pgfsetfillcolor{currentfill}%
\pgfsetfillopacity{0.700000}%
\pgfsetlinewidth{0.000000pt}%
\definecolor{currentstroke}{rgb}{0.000000,0.000000,0.000000}%
\pgfsetstrokecolor{currentstroke}%
\pgfsetdash{}{0pt}%
\pgfpathmoveto{\pgfqpoint{3.523313in}{3.858656in}}%
\pgfpathlineto{\pgfqpoint{3.536599in}{3.844362in}}%
\pgfpathlineto{\pgfqpoint{3.549885in}{3.830191in}}%
\pgfpathlineto{\pgfqpoint{3.563170in}{3.816142in}}%
\pgfpathlineto{\pgfqpoint{3.576455in}{3.802215in}}%
\pgfpathlineto{\pgfqpoint{3.584122in}{3.829939in}}%
\pgfpathlineto{\pgfqpoint{3.591786in}{3.858136in}}%
\pgfpathlineto{\pgfqpoint{3.599446in}{3.886814in}}%
\pgfpathlineto{\pgfqpoint{3.607102in}{3.915984in}}%
\pgfpathlineto{\pgfqpoint{3.593810in}{3.930388in}}%
\pgfpathlineto{\pgfqpoint{3.580518in}{3.944913in}}%
\pgfpathlineto{\pgfqpoint{3.567225in}{3.959561in}}%
\pgfpathlineto{\pgfqpoint{3.553931in}{3.974332in}}%
\pgfpathlineto{\pgfqpoint{3.546283in}{3.944676in}}%
\pgfpathlineto{\pgfqpoint{3.538630in}{3.915518in}}%
\pgfpathlineto{\pgfqpoint{3.530974in}{3.886847in}}%
\pgfpathlineto{\pgfqpoint{3.523313in}{3.858656in}}%
\pgfpathclose%
\pgfusepath{fill}%
\end{pgfscope}%
\begin{pgfscope}%
\pgfpathrectangle{\pgfqpoint{1.150000in}{0.150000in}}{\pgfqpoint{5.700000in}{5.700000in}}%
\pgfusepath{clip}%
\pgfsetbuttcap%
\pgfsetroundjoin%
\definecolor{currentfill}{rgb}{0.153894,0.680203,0.504172}%
\pgfsetfillcolor{currentfill}%
\pgfsetfillopacity{0.700000}%
\pgfsetlinewidth{0.000000pt}%
\definecolor{currentstroke}{rgb}{0.000000,0.000000,0.000000}%
\pgfsetstrokecolor{currentstroke}%
\pgfsetdash{}{0pt}%
\pgfpathmoveto{\pgfqpoint{3.310552in}{4.104996in}}%
\pgfpathlineto{\pgfqpoint{3.323864in}{4.088579in}}%
\pgfpathlineto{\pgfqpoint{3.337173in}{4.072304in}}%
\pgfpathlineto{\pgfqpoint{3.350480in}{4.056170in}}%
\pgfpathlineto{\pgfqpoint{3.363785in}{4.040175in}}%
\pgfpathlineto{\pgfqpoint{3.371423in}{4.069782in}}%
\pgfpathlineto{\pgfqpoint{3.379055in}{4.099886in}}%
\pgfpathlineto{\pgfqpoint{3.386681in}{4.130497in}}%
\pgfpathlineto{\pgfqpoint{3.394302in}{4.161624in}}%
\pgfpathlineto{\pgfqpoint{3.380988in}{4.178105in}}%
\pgfpathlineto{\pgfqpoint{3.367671in}{4.194725in}}%
\pgfpathlineto{\pgfqpoint{3.354352in}{4.211486in}}%
\pgfpathlineto{\pgfqpoint{3.341030in}{4.228391in}}%
\pgfpathlineto{\pgfqpoint{3.333419in}{4.196768in}}%
\pgfpathlineto{\pgfqpoint{3.325803in}{4.165668in}}%
\pgfpathlineto{\pgfqpoint{3.318180in}{4.135080in}}%
\pgfpathlineto{\pgfqpoint{3.310552in}{4.104996in}}%
\pgfpathclose%
\pgfusepath{fill}%
\end{pgfscope}%
\begin{pgfscope}%
\pgfpathrectangle{\pgfqpoint{1.150000in}{0.150000in}}{\pgfqpoint{5.700000in}{5.700000in}}%
\pgfusepath{clip}%
\pgfsetbuttcap%
\pgfsetroundjoin%
\definecolor{currentfill}{rgb}{0.141935,0.526453,0.555991}%
\pgfsetfillcolor{currentfill}%
\pgfsetfillopacity{0.700000}%
\pgfsetlinewidth{0.000000pt}%
\definecolor{currentstroke}{rgb}{0.000000,0.000000,0.000000}%
\pgfsetstrokecolor{currentstroke}%
\pgfsetdash{}{0pt}%
\pgfpathmoveto{\pgfqpoint{3.545743in}{3.695870in}}%
\pgfpathlineto{\pgfqpoint{3.559023in}{3.682514in}}%
\pgfpathlineto{\pgfqpoint{3.572301in}{3.669276in}}%
\pgfpathlineto{\pgfqpoint{3.585580in}{3.656156in}}%
\pgfpathlineto{\pgfqpoint{3.598859in}{3.643152in}}%
\pgfpathlineto{\pgfqpoint{3.606548in}{3.668631in}}%
\pgfpathlineto{\pgfqpoint{3.614233in}{3.694541in}}%
\pgfpathlineto{\pgfqpoint{3.621913in}{3.720892in}}%
\pgfpathlineto{\pgfqpoint{3.629590in}{3.747693in}}%
\pgfpathlineto{\pgfqpoint{3.616307in}{3.761147in}}%
\pgfpathlineto{\pgfqpoint{3.603023in}{3.774718in}}%
\pgfpathlineto{\pgfqpoint{3.589739in}{3.788407in}}%
\pgfpathlineto{\pgfqpoint{3.576455in}{3.802215in}}%
\pgfpathlineto{\pgfqpoint{3.568783in}{3.774954in}}%
\pgfpathlineto{\pgfqpoint{3.561108in}{3.748149in}}%
\pgfpathlineto{\pgfqpoint{3.553428in}{3.721791in}}%
\pgfpathlineto{\pgfqpoint{3.545743in}{3.695870in}}%
\pgfpathclose%
\pgfusepath{fill}%
\end{pgfscope}%
\begin{pgfscope}%
\pgfpathrectangle{\pgfqpoint{1.150000in}{0.150000in}}{\pgfqpoint{5.700000in}{5.700000in}}%
\pgfusepath{clip}%
\pgfsetbuttcap%
\pgfsetroundjoin%
\definecolor{currentfill}{rgb}{0.160665,0.478540,0.558115}%
\pgfsetfillcolor{currentfill}%
\pgfsetfillopacity{0.700000}%
\pgfsetlinewidth{0.000000pt}%
\definecolor{currentstroke}{rgb}{0.000000,0.000000,0.000000}%
\pgfsetstrokecolor{currentstroke}%
\pgfsetdash{}{0pt}%
\pgfpathmoveto{\pgfqpoint{3.293883in}{3.571220in}}%
\pgfpathlineto{\pgfqpoint{3.307159in}{3.557592in}}%
\pgfpathlineto{\pgfqpoint{3.320434in}{3.544093in}}%
\pgfpathlineto{\pgfqpoint{3.333709in}{3.530724in}}%
\pgfpathlineto{\pgfqpoint{3.346982in}{3.517483in}}%
\pgfpathlineto{\pgfqpoint{3.354723in}{3.539526in}}%
\pgfpathlineto{\pgfqpoint{3.362459in}{3.561924in}}%
\pgfpathlineto{\pgfqpoint{3.370188in}{3.584686in}}%
\pgfpathlineto{\pgfqpoint{3.377912in}{3.607817in}}%
\pgfpathlineto{\pgfqpoint{3.364636in}{3.621445in}}%
\pgfpathlineto{\pgfqpoint{3.351359in}{3.635201in}}%
\pgfpathlineto{\pgfqpoint{3.338081in}{3.649087in}}%
\pgfpathlineto{\pgfqpoint{3.324803in}{3.663103in}}%
\pgfpathlineto{\pgfqpoint{3.317082in}{3.639576in}}%
\pgfpathlineto{\pgfqpoint{3.309356in}{3.616425in}}%
\pgfpathlineto{\pgfqpoint{3.301623in}{3.593642in}}%
\pgfpathlineto{\pgfqpoint{3.293883in}{3.571220in}}%
\pgfpathclose%
\pgfusepath{fill}%
\end{pgfscope}%
\begin{pgfscope}%
\pgfpathrectangle{\pgfqpoint{1.150000in}{0.150000in}}{\pgfqpoint{5.700000in}{5.700000in}}%
\pgfusepath{clip}%
\pgfsetbuttcap%
\pgfsetroundjoin%
\definecolor{currentfill}{rgb}{0.127568,0.566949,0.550556}%
\pgfsetfillcolor{currentfill}%
\pgfsetfillopacity{0.700000}%
\pgfsetlinewidth{0.000000pt}%
\definecolor{currentstroke}{rgb}{0.000000,0.000000,0.000000}%
\pgfsetstrokecolor{currentstroke}%
\pgfsetdash{}{0pt}%
\pgfpathmoveto{\pgfqpoint{3.576455in}{3.802215in}}%
\pgfpathlineto{\pgfqpoint{3.589739in}{3.788407in}}%
\pgfpathlineto{\pgfqpoint{3.603023in}{3.774718in}}%
\pgfpathlineto{\pgfqpoint{3.616307in}{3.761147in}}%
\pgfpathlineto{\pgfqpoint{3.629590in}{3.747693in}}%
\pgfpathlineto{\pgfqpoint{3.637264in}{3.774951in}}%
\pgfpathlineto{\pgfqpoint{3.644934in}{3.802676in}}%
\pgfpathlineto{\pgfqpoint{3.652600in}{3.830877in}}%
\pgfpathlineto{\pgfqpoint{3.660263in}{3.859564in}}%
\pgfpathlineto{\pgfqpoint{3.646973in}{3.873492in}}%
\pgfpathlineto{\pgfqpoint{3.633683in}{3.887537in}}%
\pgfpathlineto{\pgfqpoint{3.620393in}{3.901701in}}%
\pgfpathlineto{\pgfqpoint{3.607102in}{3.915984in}}%
\pgfpathlineto{\pgfqpoint{3.599446in}{3.886814in}}%
\pgfpathlineto{\pgfqpoint{3.591786in}{3.858136in}}%
\pgfpathlineto{\pgfqpoint{3.584122in}{3.829939in}}%
\pgfpathlineto{\pgfqpoint{3.576455in}{3.802215in}}%
\pgfpathclose%
\pgfusepath{fill}%
\end{pgfscope}%
\begin{pgfscope}%
\pgfpathrectangle{\pgfqpoint{1.150000in}{0.150000in}}{\pgfqpoint{5.700000in}{5.700000in}}%
\pgfusepath{clip}%
\pgfsetbuttcap%
\pgfsetroundjoin%
\definecolor{currentfill}{rgb}{0.156270,0.489624,0.557936}%
\pgfsetfillcolor{currentfill}%
\pgfsetfillopacity{0.700000}%
\pgfsetlinewidth{0.000000pt}%
\definecolor{currentstroke}{rgb}{0.000000,0.000000,0.000000}%
\pgfsetstrokecolor{currentstroke}%
\pgfsetdash{}{0pt}%
\pgfpathmoveto{\pgfqpoint{3.514960in}{3.596399in}}%
\pgfpathlineto{\pgfqpoint{3.528236in}{3.583471in}}%
\pgfpathlineto{\pgfqpoint{3.541511in}{3.570661in}}%
\pgfpathlineto{\pgfqpoint{3.554786in}{3.557968in}}%
\pgfpathlineto{\pgfqpoint{3.568062in}{3.545392in}}%
\pgfpathlineto{\pgfqpoint{3.575768in}{3.569225in}}%
\pgfpathlineto{\pgfqpoint{3.583469in}{3.593457in}}%
\pgfpathlineto{\pgfqpoint{3.591166in}{3.618097in}}%
\pgfpathlineto{\pgfqpoint{3.598859in}{3.643152in}}%
\pgfpathlineto{\pgfqpoint{3.585580in}{3.656156in}}%
\pgfpathlineto{\pgfqpoint{3.572301in}{3.669276in}}%
\pgfpathlineto{\pgfqpoint{3.559023in}{3.682514in}}%
\pgfpathlineto{\pgfqpoint{3.545743in}{3.695870in}}%
\pgfpathlineto{\pgfqpoint{3.538055in}{3.670379in}}%
\pgfpathlineto{\pgfqpoint{3.530361in}{3.645308in}}%
\pgfpathlineto{\pgfqpoint{3.522663in}{3.620651in}}%
\pgfpathlineto{\pgfqpoint{3.514960in}{3.596399in}}%
\pgfpathclose%
\pgfusepath{fill}%
\end{pgfscope}%
\begin{pgfscope}%
\pgfpathrectangle{\pgfqpoint{1.150000in}{0.150000in}}{\pgfqpoint{5.700000in}{5.700000in}}%
\pgfusepath{clip}%
\pgfsetbuttcap%
\pgfsetroundjoin%
\definecolor{currentfill}{rgb}{0.163625,0.471133,0.558148}%
\pgfsetfillcolor{currentfill}%
\pgfsetfillopacity{0.700000}%
\pgfsetlinewidth{0.000000pt}%
\definecolor{currentstroke}{rgb}{0.000000,0.000000,0.000000}%
\pgfsetstrokecolor{currentstroke}%
\pgfsetdash{}{0pt}%
\pgfpathmoveto{\pgfqpoint{3.431008in}{3.554566in}}%
\pgfpathlineto{\pgfqpoint{3.444281in}{3.541563in}}%
\pgfpathlineto{\pgfqpoint{3.457554in}{3.528682in}}%
\pgfpathlineto{\pgfqpoint{3.470827in}{3.515921in}}%
\pgfpathlineto{\pgfqpoint{3.484099in}{3.503280in}}%
\pgfpathlineto{\pgfqpoint{3.491822in}{3.525991in}}%
\pgfpathlineto{\pgfqpoint{3.499540in}{3.549077in}}%
\pgfpathlineto{\pgfqpoint{3.507253in}{3.572543in}}%
\pgfpathlineto{\pgfqpoint{3.514960in}{3.596399in}}%
\pgfpathlineto{\pgfqpoint{3.501685in}{3.609446in}}%
\pgfpathlineto{\pgfqpoint{3.488410in}{3.622613in}}%
\pgfpathlineto{\pgfqpoint{3.475134in}{3.635901in}}%
\pgfpathlineto{\pgfqpoint{3.461858in}{3.649311in}}%
\pgfpathlineto{\pgfqpoint{3.454154in}{3.625041in}}%
\pgfpathlineto{\pgfqpoint{3.446444in}{3.601165in}}%
\pgfpathlineto{\pgfqpoint{3.438729in}{3.577676in}}%
\pgfpathlineto{\pgfqpoint{3.431008in}{3.554566in}}%
\pgfpathclose%
\pgfusepath{fill}%
\end{pgfscope}%
\begin{pgfscope}%
\pgfpathrectangle{\pgfqpoint{1.150000in}{0.150000in}}{\pgfqpoint{5.700000in}{5.700000in}}%
\pgfusepath{clip}%
\pgfsetbuttcap%
\pgfsetroundjoin%
\definecolor{currentfill}{rgb}{0.157729,0.485932,0.558013}%
\pgfsetfillcolor{currentfill}%
\pgfsetfillopacity{0.700000}%
\pgfsetlinewidth{0.000000pt}%
\definecolor{currentstroke}{rgb}{0.000000,0.000000,0.000000}%
\pgfsetstrokecolor{currentstroke}%
\pgfsetdash{}{0pt}%
\pgfpathmoveto{\pgfqpoint{3.156620in}{3.595928in}}%
\pgfpathlineto{\pgfqpoint{3.169905in}{3.581583in}}%
\pgfpathlineto{\pgfqpoint{3.183189in}{3.567377in}}%
\pgfpathlineto{\pgfqpoint{3.196471in}{3.553310in}}%
\pgfpathlineto{\pgfqpoint{3.209752in}{3.539380in}}%
\pgfpathlineto{\pgfqpoint{3.217516in}{3.560784in}}%
\pgfpathlineto{\pgfqpoint{3.225273in}{3.582528in}}%
\pgfpathlineto{\pgfqpoint{3.233024in}{3.604617in}}%
\pgfpathlineto{\pgfqpoint{3.240767in}{3.627059in}}%
\pgfpathlineto{\pgfqpoint{3.227484in}{3.641356in}}%
\pgfpathlineto{\pgfqpoint{3.214200in}{3.655790in}}%
\pgfpathlineto{\pgfqpoint{3.200914in}{3.670363in}}%
\pgfpathlineto{\pgfqpoint{3.187626in}{3.685077in}}%
\pgfpathlineto{\pgfqpoint{3.179885in}{3.662259in}}%
\pgfpathlineto{\pgfqpoint{3.172137in}{3.639800in}}%
\pgfpathlineto{\pgfqpoint{3.164382in}{3.617692in}}%
\pgfpathlineto{\pgfqpoint{3.156620in}{3.595928in}}%
\pgfpathclose%
\pgfusepath{fill}%
\end{pgfscope}%
\begin{pgfscope}%
\pgfpathrectangle{\pgfqpoint{1.150000in}{0.150000in}}{\pgfqpoint{5.700000in}{5.700000in}}%
\pgfusepath{clip}%
\pgfsetbuttcap%
\pgfsetroundjoin%
\definecolor{currentfill}{rgb}{0.149039,0.508051,0.557250}%
\pgfsetfillcolor{currentfill}%
\pgfsetfillopacity{0.700000}%
\pgfsetlinewidth{0.000000pt}%
\definecolor{currentstroke}{rgb}{0.000000,0.000000,0.000000}%
\pgfsetstrokecolor{currentstroke}%
\pgfsetdash{}{0pt}%
\pgfpathmoveto{\pgfqpoint{3.598859in}{3.643152in}}%
\pgfpathlineto{\pgfqpoint{3.612138in}{3.630264in}}%
\pgfpathlineto{\pgfqpoint{3.625418in}{3.617491in}}%
\pgfpathlineto{\pgfqpoint{3.638697in}{3.604831in}}%
\pgfpathlineto{\pgfqpoint{3.651977in}{3.592285in}}%
\pgfpathlineto{\pgfqpoint{3.659670in}{3.617323in}}%
\pgfpathlineto{\pgfqpoint{3.667359in}{3.642787in}}%
\pgfpathlineto{\pgfqpoint{3.675044in}{3.668686in}}%
\pgfpathlineto{\pgfqpoint{3.682726in}{3.695028in}}%
\pgfpathlineto{\pgfqpoint{3.669442in}{3.708023in}}%
\pgfpathlineto{\pgfqpoint{3.656158in}{3.721132in}}%
\pgfpathlineto{\pgfqpoint{3.642874in}{3.734355in}}%
\pgfpathlineto{\pgfqpoint{3.629590in}{3.747693in}}%
\pgfpathlineto{\pgfqpoint{3.621913in}{3.720892in}}%
\pgfpathlineto{\pgfqpoint{3.614233in}{3.694541in}}%
\pgfpathlineto{\pgfqpoint{3.606548in}{3.668631in}}%
\pgfpathlineto{\pgfqpoint{3.598859in}{3.643152in}}%
\pgfpathclose%
\pgfusepath{fill}%
\end{pgfscope}%
\begin{pgfscope}%
\pgfpathrectangle{\pgfqpoint{1.150000in}{0.150000in}}{\pgfqpoint{5.700000in}{5.700000in}}%
\pgfusepath{clip}%
\pgfsetbuttcap%
\pgfsetroundjoin%
\definecolor{currentfill}{rgb}{0.132268,0.655014,0.519661}%
\pgfsetfillcolor{currentfill}%
\pgfsetfillopacity{0.700000}%
\pgfsetlinewidth{0.000000pt}%
\definecolor{currentstroke}{rgb}{0.000000,0.000000,0.000000}%
\pgfsetstrokecolor{currentstroke}%
\pgfsetdash{}{0pt}%
\pgfpathmoveto{\pgfqpoint{3.500746in}{4.034673in}}%
\pgfpathlineto{\pgfqpoint{3.514044in}{4.019397in}}%
\pgfpathlineto{\pgfqpoint{3.527341in}{4.004249in}}%
\pgfpathlineto{\pgfqpoint{3.540637in}{3.989228in}}%
\pgfpathlineto{\pgfqpoint{3.553931in}{3.974332in}}%
\pgfpathlineto{\pgfqpoint{3.561576in}{4.004495in}}%
\pgfpathlineto{\pgfqpoint{3.569216in}{4.035173in}}%
\pgfpathlineto{\pgfqpoint{3.576854in}{4.066378in}}%
\pgfpathlineto{\pgfqpoint{3.584487in}{4.098118in}}%
\pgfpathlineto{\pgfqpoint{3.571183in}{4.113517in}}%
\pgfpathlineto{\pgfqpoint{3.557878in}{4.129041in}}%
\pgfpathlineto{\pgfqpoint{3.544572in}{4.144693in}}%
\pgfpathlineto{\pgfqpoint{3.531264in}{4.160474in}}%
\pgfpathlineto{\pgfqpoint{3.523640in}{4.128221in}}%
\pgfpathlineto{\pgfqpoint{3.516013in}{4.096509in}}%
\pgfpathlineto{\pgfqpoint{3.508381in}{4.065330in}}%
\pgfpathlineto{\pgfqpoint{3.500746in}{4.034673in}}%
\pgfpathclose%
\pgfusepath{fill}%
\end{pgfscope}%
\begin{pgfscope}%
\pgfpathrectangle{\pgfqpoint{1.150000in}{0.150000in}}{\pgfqpoint{5.700000in}{5.700000in}}%
\pgfusepath{clip}%
\pgfsetbuttcap%
\pgfsetroundjoin%
\definecolor{currentfill}{rgb}{0.133743,0.548535,0.553541}%
\pgfsetfillcolor{currentfill}%
\pgfsetfillopacity{0.700000}%
\pgfsetlinewidth{0.000000pt}%
\definecolor{currentstroke}{rgb}{0.000000,0.000000,0.000000}%
\pgfsetstrokecolor{currentstroke}%
\pgfsetdash{}{0pt}%
\pgfpathmoveto{\pgfqpoint{3.629590in}{3.747693in}}%
\pgfpathlineto{\pgfqpoint{3.642874in}{3.734355in}}%
\pgfpathlineto{\pgfqpoint{3.656158in}{3.721132in}}%
\pgfpathlineto{\pgfqpoint{3.669442in}{3.708023in}}%
\pgfpathlineto{\pgfqpoint{3.682726in}{3.695028in}}%
\pgfpathlineto{\pgfqpoint{3.690405in}{3.721823in}}%
\pgfpathlineto{\pgfqpoint{3.698080in}{3.749078in}}%
\pgfpathlineto{\pgfqpoint{3.705752in}{3.776804in}}%
\pgfpathlineto{\pgfqpoint{3.713422in}{3.805008in}}%
\pgfpathlineto{\pgfqpoint{3.700132in}{3.818475in}}%
\pgfpathlineto{\pgfqpoint{3.686842in}{3.832057in}}%
\pgfpathlineto{\pgfqpoint{3.673553in}{3.845752in}}%
\pgfpathlineto{\pgfqpoint{3.660263in}{3.859564in}}%
\pgfpathlineto{\pgfqpoint{3.652600in}{3.830877in}}%
\pgfpathlineto{\pgfqpoint{3.644934in}{3.802676in}}%
\pgfpathlineto{\pgfqpoint{3.637264in}{3.774951in}}%
\pgfpathlineto{\pgfqpoint{3.629590in}{3.747693in}}%
\pgfpathclose%
\pgfusepath{fill}%
\end{pgfscope}%
\begin{pgfscope}%
\pgfpathrectangle{\pgfqpoint{1.150000in}{0.150000in}}{\pgfqpoint{5.700000in}{5.700000in}}%
\pgfusepath{clip}%
\pgfsetbuttcap%
\pgfsetroundjoin%
\definecolor{currentfill}{rgb}{0.153894,0.680203,0.504172}%
\pgfsetfillcolor{currentfill}%
\pgfsetfillopacity{0.700000}%
\pgfsetlinewidth{0.000000pt}%
\definecolor{currentstroke}{rgb}{0.000000,0.000000,0.000000}%
\pgfsetstrokecolor{currentstroke}%
\pgfsetdash{}{0pt}%
\pgfpathmoveto{\pgfqpoint{3.447538in}{4.097079in}}%
\pgfpathlineto{\pgfqpoint{3.460843in}{4.081280in}}%
\pgfpathlineto{\pgfqpoint{3.474145in}{4.065614in}}%
\pgfpathlineto{\pgfqpoint{3.487446in}{4.050078in}}%
\pgfpathlineto{\pgfqpoint{3.500746in}{4.034673in}}%
\pgfpathlineto{\pgfqpoint{3.508381in}{4.065330in}}%
\pgfpathlineto{\pgfqpoint{3.516013in}{4.096509in}}%
\pgfpathlineto{\pgfqpoint{3.523640in}{4.128221in}}%
\pgfpathlineto{\pgfqpoint{3.531264in}{4.160474in}}%
\pgfpathlineto{\pgfqpoint{3.517955in}{4.176384in}}%
\pgfpathlineto{\pgfqpoint{3.504644in}{4.192425in}}%
\pgfpathlineto{\pgfqpoint{3.491331in}{4.208598in}}%
\pgfpathlineto{\pgfqpoint{3.478016in}{4.224904in}}%
\pgfpathlineto{\pgfqpoint{3.470403in}{4.192135in}}%
\pgfpathlineto{\pgfqpoint{3.462786in}{4.159914in}}%
\pgfpathlineto{\pgfqpoint{3.455165in}{4.128232in}}%
\pgfpathlineto{\pgfqpoint{3.447538in}{4.097079in}}%
\pgfpathclose%
\pgfusepath{fill}%
\end{pgfscope}%
\begin{pgfscope}%
\pgfpathrectangle{\pgfqpoint{1.150000in}{0.150000in}}{\pgfqpoint{5.700000in}{5.700000in}}%
\pgfusepath{clip}%
\pgfsetbuttcap%
\pgfsetroundjoin%
\definecolor{currentfill}{rgb}{0.122312,0.633153,0.530398}%
\pgfsetfillcolor{currentfill}%
\pgfsetfillopacity{0.700000}%
\pgfsetlinewidth{0.000000pt}%
\definecolor{currentstroke}{rgb}{0.000000,0.000000,0.000000}%
\pgfsetstrokecolor{currentstroke}%
\pgfsetdash{}{0pt}%
\pgfpathmoveto{\pgfqpoint{3.553931in}{3.974332in}}%
\pgfpathlineto{\pgfqpoint{3.567225in}{3.959561in}}%
\pgfpathlineto{\pgfqpoint{3.580518in}{3.944913in}}%
\pgfpathlineto{\pgfqpoint{3.593810in}{3.930388in}}%
\pgfpathlineto{\pgfqpoint{3.607102in}{3.915984in}}%
\pgfpathlineto{\pgfqpoint{3.614754in}{3.945655in}}%
\pgfpathlineto{\pgfqpoint{3.622403in}{3.975836in}}%
\pgfpathlineto{\pgfqpoint{3.630049in}{4.006536in}}%
\pgfpathlineto{\pgfqpoint{3.637692in}{4.037765in}}%
\pgfpathlineto{\pgfqpoint{3.624392in}{4.052669in}}%
\pgfpathlineto{\pgfqpoint{3.611092in}{4.067696in}}%
\pgfpathlineto{\pgfqpoint{3.597790in}{4.082845in}}%
\pgfpathlineto{\pgfqpoint{3.584487in}{4.098118in}}%
\pgfpathlineto{\pgfqpoint{3.576854in}{4.066378in}}%
\pgfpathlineto{\pgfqpoint{3.569216in}{4.035173in}}%
\pgfpathlineto{\pgfqpoint{3.561576in}{4.004495in}}%
\pgfpathlineto{\pgfqpoint{3.553931in}{3.974332in}}%
\pgfpathclose%
\pgfusepath{fill}%
\end{pgfscope}%
\begin{pgfscope}%
\pgfpathrectangle{\pgfqpoint{1.150000in}{0.150000in}}{\pgfqpoint{5.700000in}{5.700000in}}%
\pgfusepath{clip}%
\pgfsetbuttcap%
\pgfsetroundjoin%
\definecolor{currentfill}{rgb}{0.168126,0.459988,0.558082}%
\pgfsetfillcolor{currentfill}%
\pgfsetfillopacity{0.700000}%
\pgfsetlinewidth{0.000000pt}%
\definecolor{currentstroke}{rgb}{0.000000,0.000000,0.000000}%
\pgfsetstrokecolor{currentstroke}%
\pgfsetdash{}{0pt}%
\pgfpathmoveto{\pgfqpoint{3.346982in}{3.517483in}}%
\pgfpathlineto{\pgfqpoint{3.360255in}{3.504368in}}%
\pgfpathlineto{\pgfqpoint{3.373527in}{3.491379in}}%
\pgfpathlineto{\pgfqpoint{3.386798in}{3.478514in}}%
\pgfpathlineto{\pgfqpoint{3.400070in}{3.465773in}}%
\pgfpathlineto{\pgfqpoint{3.407813in}{3.487439in}}%
\pgfpathlineto{\pgfqpoint{3.415550in}{3.509455in}}%
\pgfpathlineto{\pgfqpoint{3.423282in}{3.531829in}}%
\pgfpathlineto{\pgfqpoint{3.431008in}{3.554566in}}%
\pgfpathlineto{\pgfqpoint{3.417735in}{3.567692in}}%
\pgfpathlineto{\pgfqpoint{3.404461in}{3.580942in}}%
\pgfpathlineto{\pgfqpoint{3.391187in}{3.594317in}}%
\pgfpathlineto{\pgfqpoint{3.377912in}{3.607817in}}%
\pgfpathlineto{\pgfqpoint{3.370188in}{3.584686in}}%
\pgfpathlineto{\pgfqpoint{3.362459in}{3.561924in}}%
\pgfpathlineto{\pgfqpoint{3.354723in}{3.539526in}}%
\pgfpathlineto{\pgfqpoint{3.346982in}{3.517483in}}%
\pgfpathclose%
\pgfusepath{fill}%
\end{pgfscope}%
\begin{pgfscope}%
\pgfpathrectangle{\pgfqpoint{1.150000in}{0.150000in}}{\pgfqpoint{5.700000in}{5.700000in}}%
\pgfusepath{clip}%
\pgfsetbuttcap%
\pgfsetroundjoin%
\definecolor{currentfill}{rgb}{0.180653,0.701402,0.488189}%
\pgfsetfillcolor{currentfill}%
\pgfsetfillopacity{0.700000}%
\pgfsetlinewidth{0.000000pt}%
\definecolor{currentstroke}{rgb}{0.000000,0.000000,0.000000}%
\pgfsetstrokecolor{currentstroke}%
\pgfsetdash{}{0pt}%
\pgfpathmoveto{\pgfqpoint{3.394302in}{4.161624in}}%
\pgfpathlineto{\pgfqpoint{3.407615in}{4.145283in}}%
\pgfpathlineto{\pgfqpoint{3.420924in}{4.129079in}}%
\pgfpathlineto{\pgfqpoint{3.434232in}{4.113012in}}%
\pgfpathlineto{\pgfqpoint{3.447538in}{4.097079in}}%
\pgfpathlineto{\pgfqpoint{3.455165in}{4.128232in}}%
\pgfpathlineto{\pgfqpoint{3.462786in}{4.159914in}}%
\pgfpathlineto{\pgfqpoint{3.470403in}{4.192135in}}%
\pgfpathlineto{\pgfqpoint{3.478016in}{4.224904in}}%
\pgfpathlineto{\pgfqpoint{3.464700in}{4.241345in}}%
\pgfpathlineto{\pgfqpoint{3.451381in}{4.257921in}}%
\pgfpathlineto{\pgfqpoint{3.438060in}{4.274634in}}%
\pgfpathlineto{\pgfqpoint{3.424737in}{4.291486in}}%
\pgfpathlineto{\pgfqpoint{3.417136in}{4.258198in}}%
\pgfpathlineto{\pgfqpoint{3.409530in}{4.225465in}}%
\pgfpathlineto{\pgfqpoint{3.401919in}{4.193277in}}%
\pgfpathlineto{\pgfqpoint{3.394302in}{4.161624in}}%
\pgfpathclose%
\pgfusepath{fill}%
\end{pgfscope}%
\begin{pgfscope}%
\pgfpathrectangle{\pgfqpoint{1.150000in}{0.150000in}}{\pgfqpoint{5.700000in}{5.700000in}}%
\pgfusepath{clip}%
\pgfsetbuttcap%
\pgfsetroundjoin%
\definecolor{currentfill}{rgb}{0.119423,0.611141,0.538982}%
\pgfsetfillcolor{currentfill}%
\pgfsetfillopacity{0.700000}%
\pgfsetlinewidth{0.000000pt}%
\definecolor{currentstroke}{rgb}{0.000000,0.000000,0.000000}%
\pgfsetstrokecolor{currentstroke}%
\pgfsetdash{}{0pt}%
\pgfpathmoveto{\pgfqpoint{3.607102in}{3.915984in}}%
\pgfpathlineto{\pgfqpoint{3.620393in}{3.901701in}}%
\pgfpathlineto{\pgfqpoint{3.633683in}{3.887537in}}%
\pgfpathlineto{\pgfqpoint{3.646973in}{3.873492in}}%
\pgfpathlineto{\pgfqpoint{3.660263in}{3.859564in}}%
\pgfpathlineto{\pgfqpoint{3.667923in}{3.888745in}}%
\pgfpathlineto{\pgfqpoint{3.675580in}{3.918429in}}%
\pgfpathlineto{\pgfqpoint{3.683234in}{3.948627in}}%
\pgfpathlineto{\pgfqpoint{3.690886in}{3.979348in}}%
\pgfpathlineto{\pgfqpoint{3.677588in}{3.993775in}}%
\pgfpathlineto{\pgfqpoint{3.664290in}{4.008319in}}%
\pgfpathlineto{\pgfqpoint{3.650992in}{4.022982in}}%
\pgfpathlineto{\pgfqpoint{3.637692in}{4.037765in}}%
\pgfpathlineto{\pgfqpoint{3.630049in}{4.006536in}}%
\pgfpathlineto{\pgfqpoint{3.622403in}{3.975836in}}%
\pgfpathlineto{\pgfqpoint{3.614754in}{3.945655in}}%
\pgfpathlineto{\pgfqpoint{3.607102in}{3.915984in}}%
\pgfpathclose%
\pgfusepath{fill}%
\end{pgfscope}%
\begin{pgfscope}%
\pgfpathrectangle{\pgfqpoint{1.150000in}{0.150000in}}{\pgfqpoint{5.700000in}{5.700000in}}%
\pgfusepath{clip}%
\pgfsetbuttcap%
\pgfsetroundjoin%
\definecolor{currentfill}{rgb}{0.165117,0.467423,0.558141}%
\pgfsetfillcolor{currentfill}%
\pgfsetfillopacity{0.700000}%
\pgfsetlinewidth{0.000000pt}%
\definecolor{currentstroke}{rgb}{0.000000,0.000000,0.000000}%
\pgfsetstrokecolor{currentstroke}%
\pgfsetdash{}{0pt}%
\pgfpathmoveto{\pgfqpoint{3.209752in}{3.539380in}}%
\pgfpathlineto{\pgfqpoint{3.223031in}{3.525585in}}%
\pgfpathlineto{\pgfqpoint{3.236309in}{3.511925in}}%
\pgfpathlineto{\pgfqpoint{3.249585in}{3.498399in}}%
\pgfpathlineto{\pgfqpoint{3.262861in}{3.485004in}}%
\pgfpathlineto{\pgfqpoint{3.270626in}{3.506051in}}%
\pgfpathlineto{\pgfqpoint{3.278385in}{3.527432in}}%
\pgfpathlineto{\pgfqpoint{3.286138in}{3.549153in}}%
\pgfpathlineto{\pgfqpoint{3.293883in}{3.571220in}}%
\pgfpathlineto{\pgfqpoint{3.280606in}{3.584980in}}%
\pgfpathlineto{\pgfqpoint{3.267328in}{3.598872in}}%
\pgfpathlineto{\pgfqpoint{3.254048in}{3.612898in}}%
\pgfpathlineto{\pgfqpoint{3.240767in}{3.627059in}}%
\pgfpathlineto{\pgfqpoint{3.233024in}{3.604617in}}%
\pgfpathlineto{\pgfqpoint{3.225273in}{3.582528in}}%
\pgfpathlineto{\pgfqpoint{3.217516in}{3.560784in}}%
\pgfpathlineto{\pgfqpoint{3.209752in}{3.539380in}}%
\pgfpathclose%
\pgfusepath{fill}%
\end{pgfscope}%
\begin{pgfscope}%
\pgfpathrectangle{\pgfqpoint{1.150000in}{0.150000in}}{\pgfqpoint{5.700000in}{5.700000in}}%
\pgfusepath{clip}%
\pgfsetbuttcap%
\pgfsetroundjoin%
\definecolor{currentfill}{rgb}{0.163625,0.471133,0.558148}%
\pgfsetfillcolor{currentfill}%
\pgfsetfillopacity{0.700000}%
\pgfsetlinewidth{0.000000pt}%
\definecolor{currentstroke}{rgb}{0.000000,0.000000,0.000000}%
\pgfsetstrokecolor{currentstroke}%
\pgfsetdash{}{0pt}%
\pgfpathmoveto{\pgfqpoint{3.568062in}{3.545392in}}%
\pgfpathlineto{\pgfqpoint{3.581337in}{3.532931in}}%
\pgfpathlineto{\pgfqpoint{3.594614in}{3.520584in}}%
\pgfpathlineto{\pgfqpoint{3.607890in}{3.508351in}}%
\pgfpathlineto{\pgfqpoint{3.621167in}{3.496231in}}%
\pgfpathlineto{\pgfqpoint{3.628876in}{3.519646in}}%
\pgfpathlineto{\pgfqpoint{3.636580in}{3.543455in}}%
\pgfpathlineto{\pgfqpoint{3.644281in}{3.567665in}}%
\pgfpathlineto{\pgfqpoint{3.651977in}{3.592285in}}%
\pgfpathlineto{\pgfqpoint{3.638697in}{3.604831in}}%
\pgfpathlineto{\pgfqpoint{3.625418in}{3.617491in}}%
\pgfpathlineto{\pgfqpoint{3.612138in}{3.630264in}}%
\pgfpathlineto{\pgfqpoint{3.598859in}{3.643152in}}%
\pgfpathlineto{\pgfqpoint{3.591166in}{3.618097in}}%
\pgfpathlineto{\pgfqpoint{3.583469in}{3.593457in}}%
\pgfpathlineto{\pgfqpoint{3.575768in}{3.569225in}}%
\pgfpathlineto{\pgfqpoint{3.568062in}{3.545392in}}%
\pgfpathclose%
\pgfusepath{fill}%
\end{pgfscope}%
\begin{pgfscope}%
\pgfpathrectangle{\pgfqpoint{1.150000in}{0.150000in}}{\pgfqpoint{5.700000in}{5.700000in}}%
\pgfusepath{clip}%
\pgfsetbuttcap%
\pgfsetroundjoin%
\definecolor{currentfill}{rgb}{0.169646,0.456262,0.558030}%
\pgfsetfillcolor{currentfill}%
\pgfsetfillopacity{0.700000}%
\pgfsetlinewidth{0.000000pt}%
\definecolor{currentstroke}{rgb}{0.000000,0.000000,0.000000}%
\pgfsetstrokecolor{currentstroke}%
\pgfsetdash{}{0pt}%
\pgfpathmoveto{\pgfqpoint{3.484099in}{3.503280in}}%
\pgfpathlineto{\pgfqpoint{3.497372in}{3.490758in}}%
\pgfpathlineto{\pgfqpoint{3.510645in}{3.478353in}}%
\pgfpathlineto{\pgfqpoint{3.523918in}{3.466066in}}%
\pgfpathlineto{\pgfqpoint{3.537191in}{3.453894in}}%
\pgfpathlineto{\pgfqpoint{3.544916in}{3.476209in}}%
\pgfpathlineto{\pgfqpoint{3.552636in}{3.498892in}}%
\pgfpathlineto{\pgfqpoint{3.560351in}{3.521950in}}%
\pgfpathlineto{\pgfqpoint{3.568062in}{3.545392in}}%
\pgfpathlineto{\pgfqpoint{3.554786in}{3.557968in}}%
\pgfpathlineto{\pgfqpoint{3.541511in}{3.570661in}}%
\pgfpathlineto{\pgfqpoint{3.528236in}{3.583471in}}%
\pgfpathlineto{\pgfqpoint{3.514960in}{3.596399in}}%
\pgfpathlineto{\pgfqpoint{3.507253in}{3.572543in}}%
\pgfpathlineto{\pgfqpoint{3.499540in}{3.549077in}}%
\pgfpathlineto{\pgfqpoint{3.491822in}{3.525991in}}%
\pgfpathlineto{\pgfqpoint{3.484099in}{3.503280in}}%
\pgfpathclose%
\pgfusepath{fill}%
\end{pgfscope}%
\begin{pgfscope}%
\pgfpathrectangle{\pgfqpoint{1.150000in}{0.150000in}}{\pgfqpoint{5.700000in}{5.700000in}}%
\pgfusepath{clip}%
\pgfsetbuttcap%
\pgfsetroundjoin%
\definecolor{currentfill}{rgb}{0.121148,0.592739,0.544641}%
\pgfsetfillcolor{currentfill}%
\pgfsetfillopacity{0.700000}%
\pgfsetlinewidth{0.000000pt}%
\definecolor{currentstroke}{rgb}{0.000000,0.000000,0.000000}%
\pgfsetstrokecolor{currentstroke}%
\pgfsetdash{}{0pt}%
\pgfpathmoveto{\pgfqpoint{3.660263in}{3.859564in}}%
\pgfpathlineto{\pgfqpoint{3.673553in}{3.845752in}}%
\pgfpathlineto{\pgfqpoint{3.686842in}{3.832057in}}%
\pgfpathlineto{\pgfqpoint{3.700132in}{3.818475in}}%
\pgfpathlineto{\pgfqpoint{3.713422in}{3.805008in}}%
\pgfpathlineto{\pgfqpoint{3.721089in}{3.833701in}}%
\pgfpathlineto{\pgfqpoint{3.728753in}{3.862892in}}%
\pgfpathlineto{\pgfqpoint{3.736414in}{3.892590in}}%
\pgfpathlineto{\pgfqpoint{3.744074in}{3.922805in}}%
\pgfpathlineto{\pgfqpoint{3.730777in}{3.936769in}}%
\pgfpathlineto{\pgfqpoint{3.717480in}{3.950846in}}%
\pgfpathlineto{\pgfqpoint{3.704183in}{3.965039in}}%
\pgfpathlineto{\pgfqpoint{3.690886in}{3.979348in}}%
\pgfpathlineto{\pgfqpoint{3.683234in}{3.948627in}}%
\pgfpathlineto{\pgfqpoint{3.675580in}{3.918429in}}%
\pgfpathlineto{\pgfqpoint{3.667923in}{3.888745in}}%
\pgfpathlineto{\pgfqpoint{3.660263in}{3.859564in}}%
\pgfpathclose%
\pgfusepath{fill}%
\end{pgfscope}%
\begin{pgfscope}%
\pgfpathrectangle{\pgfqpoint{1.150000in}{0.150000in}}{\pgfqpoint{5.700000in}{5.700000in}}%
\pgfusepath{clip}%
\pgfsetbuttcap%
\pgfsetroundjoin%
\definecolor{currentfill}{rgb}{0.220124,0.725509,0.466226}%
\pgfsetfillcolor{currentfill}%
\pgfsetfillopacity{0.700000}%
\pgfsetlinewidth{0.000000pt}%
\definecolor{currentstroke}{rgb}{0.000000,0.000000,0.000000}%
\pgfsetstrokecolor{currentstroke}%
\pgfsetdash{}{0pt}%
\pgfpathmoveto{\pgfqpoint{3.341030in}{4.228391in}}%
\pgfpathlineto{\pgfqpoint{3.354352in}{4.211486in}}%
\pgfpathlineto{\pgfqpoint{3.367671in}{4.194725in}}%
\pgfpathlineto{\pgfqpoint{3.380988in}{4.178105in}}%
\pgfpathlineto{\pgfqpoint{3.394302in}{4.161624in}}%
\pgfpathlineto{\pgfqpoint{3.401919in}{4.193277in}}%
\pgfpathlineto{\pgfqpoint{3.409530in}{4.225465in}}%
\pgfpathlineto{\pgfqpoint{3.417136in}{4.258198in}}%
\pgfpathlineto{\pgfqpoint{3.424737in}{4.291486in}}%
\pgfpathlineto{\pgfqpoint{3.411412in}{4.308476in}}%
\pgfpathlineto{\pgfqpoint{3.398083in}{4.325608in}}%
\pgfpathlineto{\pgfqpoint{3.384753in}{4.342882in}}%
\pgfpathlineto{\pgfqpoint{3.371419in}{4.360299in}}%
\pgfpathlineto{\pgfqpoint{3.363830in}{4.326489in}}%
\pgfpathlineto{\pgfqpoint{3.356236in}{4.293242in}}%
\pgfpathlineto{\pgfqpoint{3.348636in}{4.260545in}}%
\pgfpathlineto{\pgfqpoint{3.341030in}{4.228391in}}%
\pgfpathclose%
\pgfusepath{fill}%
\end{pgfscope}%
\begin{pgfscope}%
\pgfpathrectangle{\pgfqpoint{1.150000in}{0.150000in}}{\pgfqpoint{5.700000in}{5.700000in}}%
\pgfusepath{clip}%
\pgfsetbuttcap%
\pgfsetroundjoin%
\definecolor{currentfill}{rgb}{0.140536,0.530132,0.555659}%
\pgfsetfillcolor{currentfill}%
\pgfsetfillopacity{0.700000}%
\pgfsetlinewidth{0.000000pt}%
\definecolor{currentstroke}{rgb}{0.000000,0.000000,0.000000}%
\pgfsetstrokecolor{currentstroke}%
\pgfsetdash{}{0pt}%
\pgfpathmoveto{\pgfqpoint{3.682726in}{3.695028in}}%
\pgfpathlineto{\pgfqpoint{3.696011in}{3.682146in}}%
\pgfpathlineto{\pgfqpoint{3.709296in}{3.669374in}}%
\pgfpathlineto{\pgfqpoint{3.722581in}{3.656714in}}%
\pgfpathlineto{\pgfqpoint{3.735867in}{3.644164in}}%
\pgfpathlineto{\pgfqpoint{3.743551in}{3.670496in}}%
\pgfpathlineto{\pgfqpoint{3.751231in}{3.697283in}}%
\pgfpathlineto{\pgfqpoint{3.758909in}{3.724534in}}%
\pgfpathlineto{\pgfqpoint{3.766584in}{3.752259in}}%
\pgfpathlineto{\pgfqpoint{3.753293in}{3.765280in}}%
\pgfpathlineto{\pgfqpoint{3.740002in}{3.778411in}}%
\pgfpathlineto{\pgfqpoint{3.726712in}{3.791654in}}%
\pgfpathlineto{\pgfqpoint{3.713422in}{3.805008in}}%
\pgfpathlineto{\pgfqpoint{3.705752in}{3.776804in}}%
\pgfpathlineto{\pgfqpoint{3.698080in}{3.749078in}}%
\pgfpathlineto{\pgfqpoint{3.690405in}{3.721823in}}%
\pgfpathlineto{\pgfqpoint{3.682726in}{3.695028in}}%
\pgfpathclose%
\pgfusepath{fill}%
\end{pgfscope}%
\begin{pgfscope}%
\pgfpathrectangle{\pgfqpoint{1.150000in}{0.150000in}}{\pgfqpoint{5.700000in}{5.700000in}}%
\pgfusepath{clip}%
\pgfsetbuttcap%
\pgfsetroundjoin%
\definecolor{currentfill}{rgb}{0.156270,0.489624,0.557936}%
\pgfsetfillcolor{currentfill}%
\pgfsetfillopacity{0.700000}%
\pgfsetlinewidth{0.000000pt}%
\definecolor{currentstroke}{rgb}{0.000000,0.000000,0.000000}%
\pgfsetstrokecolor{currentstroke}%
\pgfsetdash{}{0pt}%
\pgfpathmoveto{\pgfqpoint{3.651977in}{3.592285in}}%
\pgfpathlineto{\pgfqpoint{3.665258in}{3.579851in}}%
\pgfpathlineto{\pgfqpoint{3.678539in}{3.567527in}}%
\pgfpathlineto{\pgfqpoint{3.691821in}{3.555315in}}%
\pgfpathlineto{\pgfqpoint{3.705104in}{3.543212in}}%
\pgfpathlineto{\pgfqpoint{3.712799in}{3.567810in}}%
\pgfpathlineto{\pgfqpoint{3.720492in}{3.592829in}}%
\pgfpathlineto{\pgfqpoint{3.728181in}{3.618278in}}%
\pgfpathlineto{\pgfqpoint{3.735867in}{3.644164in}}%
\pgfpathlineto{\pgfqpoint{3.722581in}{3.656714in}}%
\pgfpathlineto{\pgfqpoint{3.709296in}{3.669374in}}%
\pgfpathlineto{\pgfqpoint{3.696011in}{3.682146in}}%
\pgfpathlineto{\pgfqpoint{3.682726in}{3.695028in}}%
\pgfpathlineto{\pgfqpoint{3.675044in}{3.668686in}}%
\pgfpathlineto{\pgfqpoint{3.667359in}{3.642787in}}%
\pgfpathlineto{\pgfqpoint{3.659670in}{3.617323in}}%
\pgfpathlineto{\pgfqpoint{3.651977in}{3.592285in}}%
\pgfpathclose%
\pgfusepath{fill}%
\end{pgfscope}%
\begin{pgfscope}%
\pgfpathrectangle{\pgfqpoint{1.150000in}{0.150000in}}{\pgfqpoint{5.700000in}{5.700000in}}%
\pgfusepath{clip}%
\pgfsetbuttcap%
\pgfsetroundjoin%
\definecolor{currentfill}{rgb}{0.175841,0.441290,0.557685}%
\pgfsetfillcolor{currentfill}%
\pgfsetfillopacity{0.700000}%
\pgfsetlinewidth{0.000000pt}%
\definecolor{currentstroke}{rgb}{0.000000,0.000000,0.000000}%
\pgfsetstrokecolor{currentstroke}%
\pgfsetdash{}{0pt}%
\pgfpathmoveto{\pgfqpoint{3.400070in}{3.465773in}}%
\pgfpathlineto{\pgfqpoint{3.413341in}{3.453155in}}%
\pgfpathlineto{\pgfqpoint{3.426612in}{3.440658in}}%
\pgfpathlineto{\pgfqpoint{3.439883in}{3.428281in}}%
\pgfpathlineto{\pgfqpoint{3.453154in}{3.416024in}}%
\pgfpathlineto{\pgfqpoint{3.460899in}{3.437314in}}%
\pgfpathlineto{\pgfqpoint{3.468637in}{3.458949in}}%
\pgfpathlineto{\pgfqpoint{3.476371in}{3.480935in}}%
\pgfpathlineto{\pgfqpoint{3.484099in}{3.503280in}}%
\pgfpathlineto{\pgfqpoint{3.470827in}{3.515921in}}%
\pgfpathlineto{\pgfqpoint{3.457554in}{3.528682in}}%
\pgfpathlineto{\pgfqpoint{3.444281in}{3.541563in}}%
\pgfpathlineto{\pgfqpoint{3.431008in}{3.554566in}}%
\pgfpathlineto{\pgfqpoint{3.423282in}{3.531829in}}%
\pgfpathlineto{\pgfqpoint{3.415550in}{3.509455in}}%
\pgfpathlineto{\pgfqpoint{3.407813in}{3.487439in}}%
\pgfpathlineto{\pgfqpoint{3.400070in}{3.465773in}}%
\pgfpathclose%
\pgfusepath{fill}%
\end{pgfscope}%
\begin{pgfscope}%
\pgfpathrectangle{\pgfqpoint{1.150000in}{0.150000in}}{\pgfqpoint{5.700000in}{5.700000in}}%
\pgfusepath{clip}%
\pgfsetbuttcap%
\pgfsetroundjoin%
\definecolor{currentfill}{rgb}{0.174274,0.445044,0.557792}%
\pgfsetfillcolor{currentfill}%
\pgfsetfillopacity{0.700000}%
\pgfsetlinewidth{0.000000pt}%
\definecolor{currentstroke}{rgb}{0.000000,0.000000,0.000000}%
\pgfsetstrokecolor{currentstroke}%
\pgfsetdash{}{0pt}%
\pgfpathmoveto{\pgfqpoint{3.262861in}{3.485004in}}%
\pgfpathlineto{\pgfqpoint{3.276135in}{3.471741in}}%
\pgfpathlineto{\pgfqpoint{3.289409in}{3.458607in}}%
\pgfpathlineto{\pgfqpoint{3.302682in}{3.445602in}}%
\pgfpathlineto{\pgfqpoint{3.315954in}{3.432725in}}%
\pgfpathlineto{\pgfqpoint{3.323720in}{3.453416in}}%
\pgfpathlineto{\pgfqpoint{3.331480in}{3.474435in}}%
\pgfpathlineto{\pgfqpoint{3.339234in}{3.495788in}}%
\pgfpathlineto{\pgfqpoint{3.346982in}{3.517483in}}%
\pgfpathlineto{\pgfqpoint{3.333709in}{3.530724in}}%
\pgfpathlineto{\pgfqpoint{3.320434in}{3.544093in}}%
\pgfpathlineto{\pgfqpoint{3.307159in}{3.557592in}}%
\pgfpathlineto{\pgfqpoint{3.293883in}{3.571220in}}%
\pgfpathlineto{\pgfqpoint{3.286138in}{3.549153in}}%
\pgfpathlineto{\pgfqpoint{3.278385in}{3.527432in}}%
\pgfpathlineto{\pgfqpoint{3.270626in}{3.506051in}}%
\pgfpathlineto{\pgfqpoint{3.262861in}{3.485004in}}%
\pgfpathclose%
\pgfusepath{fill}%
\end{pgfscope}%
\begin{pgfscope}%
\pgfpathrectangle{\pgfqpoint{1.150000in}{0.150000in}}{\pgfqpoint{5.700000in}{5.700000in}}%
\pgfusepath{clip}%
\pgfsetbuttcap%
\pgfsetroundjoin%
\definecolor{currentfill}{rgb}{0.125394,0.574318,0.549086}%
\pgfsetfillcolor{currentfill}%
\pgfsetfillopacity{0.700000}%
\pgfsetlinewidth{0.000000pt}%
\definecolor{currentstroke}{rgb}{0.000000,0.000000,0.000000}%
\pgfsetstrokecolor{currentstroke}%
\pgfsetdash{}{0pt}%
\pgfpathmoveto{\pgfqpoint{3.713422in}{3.805008in}}%
\pgfpathlineto{\pgfqpoint{3.726712in}{3.791654in}}%
\pgfpathlineto{\pgfqpoint{3.740002in}{3.778411in}}%
\pgfpathlineto{\pgfqpoint{3.753293in}{3.765280in}}%
\pgfpathlineto{\pgfqpoint{3.766584in}{3.752259in}}%
\pgfpathlineto{\pgfqpoint{3.774256in}{3.780466in}}%
\pgfpathlineto{\pgfqpoint{3.781927in}{3.809165in}}%
\pgfpathlineto{\pgfqpoint{3.789595in}{3.838365in}}%
\pgfpathlineto{\pgfqpoint{3.797262in}{3.868076in}}%
\pgfpathlineto{\pgfqpoint{3.783965in}{3.881591in}}%
\pgfpathlineto{\pgfqpoint{3.770668in}{3.895217in}}%
\pgfpathlineto{\pgfqpoint{3.757371in}{3.908955in}}%
\pgfpathlineto{\pgfqpoint{3.744074in}{3.922805in}}%
\pgfpathlineto{\pgfqpoint{3.736414in}{3.892590in}}%
\pgfpathlineto{\pgfqpoint{3.728753in}{3.862892in}}%
\pgfpathlineto{\pgfqpoint{3.721089in}{3.833701in}}%
\pgfpathlineto{\pgfqpoint{3.713422in}{3.805008in}}%
\pgfpathclose%
\pgfusepath{fill}%
\end{pgfscope}%
\begin{pgfscope}%
\pgfpathrectangle{\pgfqpoint{1.150000in}{0.150000in}}{\pgfqpoint{5.700000in}{5.700000in}}%
\pgfusepath{clip}%
\pgfsetbuttcap%
\pgfsetroundjoin%
\definecolor{currentfill}{rgb}{0.169646,0.456262,0.558030}%
\pgfsetfillcolor{currentfill}%
\pgfsetfillopacity{0.700000}%
\pgfsetlinewidth{0.000000pt}%
\definecolor{currentstroke}{rgb}{0.000000,0.000000,0.000000}%
\pgfsetstrokecolor{currentstroke}%
\pgfsetdash{}{0pt}%
\pgfpathmoveto{\pgfqpoint{3.621167in}{3.496231in}}%
\pgfpathlineto{\pgfqpoint{3.634445in}{3.484222in}}%
\pgfpathlineto{\pgfqpoint{3.647724in}{3.472324in}}%
\pgfpathlineto{\pgfqpoint{3.661004in}{3.460536in}}%
\pgfpathlineto{\pgfqpoint{3.674284in}{3.448858in}}%
\pgfpathlineto{\pgfqpoint{3.681995in}{3.471856in}}%
\pgfpathlineto{\pgfqpoint{3.689701in}{3.495243in}}%
\pgfpathlineto{\pgfqpoint{3.697404in}{3.519025in}}%
\pgfpathlineto{\pgfqpoint{3.705104in}{3.543212in}}%
\pgfpathlineto{\pgfqpoint{3.691821in}{3.555315in}}%
\pgfpathlineto{\pgfqpoint{3.678539in}{3.567527in}}%
\pgfpathlineto{\pgfqpoint{3.665258in}{3.579851in}}%
\pgfpathlineto{\pgfqpoint{3.651977in}{3.592285in}}%
\pgfpathlineto{\pgfqpoint{3.644281in}{3.567665in}}%
\pgfpathlineto{\pgfqpoint{3.636580in}{3.543455in}}%
\pgfpathlineto{\pgfqpoint{3.628876in}{3.519646in}}%
\pgfpathlineto{\pgfqpoint{3.621167in}{3.496231in}}%
\pgfpathclose%
\pgfusepath{fill}%
\end{pgfscope}%
\begin{pgfscope}%
\pgfpathrectangle{\pgfqpoint{1.150000in}{0.150000in}}{\pgfqpoint{5.700000in}{5.700000in}}%
\pgfusepath{clip}%
\pgfsetbuttcap%
\pgfsetroundjoin%
\definecolor{currentfill}{rgb}{0.177423,0.437527,0.557565}%
\pgfsetfillcolor{currentfill}%
\pgfsetfillopacity{0.700000}%
\pgfsetlinewidth{0.000000pt}%
\definecolor{currentstroke}{rgb}{0.000000,0.000000,0.000000}%
\pgfsetstrokecolor{currentstroke}%
\pgfsetdash{}{0pt}%
\pgfpathmoveto{\pgfqpoint{3.537191in}{3.453894in}}%
\pgfpathlineto{\pgfqpoint{3.550465in}{3.441838in}}%
\pgfpathlineto{\pgfqpoint{3.563739in}{3.429895in}}%
\pgfpathlineto{\pgfqpoint{3.577014in}{3.418066in}}%
\pgfpathlineto{\pgfqpoint{3.590290in}{3.406349in}}%
\pgfpathlineto{\pgfqpoint{3.598016in}{3.428268in}}%
\pgfpathlineto{\pgfqpoint{3.605737in}{3.450549in}}%
\pgfpathlineto{\pgfqpoint{3.613455in}{3.473201in}}%
\pgfpathlineto{\pgfqpoint{3.621167in}{3.496231in}}%
\pgfpathlineto{\pgfqpoint{3.607890in}{3.508351in}}%
\pgfpathlineto{\pgfqpoint{3.594614in}{3.520584in}}%
\pgfpathlineto{\pgfqpoint{3.581337in}{3.532931in}}%
\pgfpathlineto{\pgfqpoint{3.568062in}{3.545392in}}%
\pgfpathlineto{\pgfqpoint{3.560351in}{3.521950in}}%
\pgfpathlineto{\pgfqpoint{3.552636in}{3.498892in}}%
\pgfpathlineto{\pgfqpoint{3.544916in}{3.476209in}}%
\pgfpathlineto{\pgfqpoint{3.537191in}{3.453894in}}%
\pgfpathclose%
\pgfusepath{fill}%
\end{pgfscope}%
\begin{pgfscope}%
\pgfpathrectangle{\pgfqpoint{1.150000in}{0.150000in}}{\pgfqpoint{5.700000in}{5.700000in}}%
\pgfusepath{clip}%
\pgfsetbuttcap%
\pgfsetroundjoin%
\definecolor{currentfill}{rgb}{0.147607,0.511733,0.557049}%
\pgfsetfillcolor{currentfill}%
\pgfsetfillopacity{0.700000}%
\pgfsetlinewidth{0.000000pt}%
\definecolor{currentstroke}{rgb}{0.000000,0.000000,0.000000}%
\pgfsetstrokecolor{currentstroke}%
\pgfsetdash{}{0pt}%
\pgfpathmoveto{\pgfqpoint{3.735867in}{3.644164in}}%
\pgfpathlineto{\pgfqpoint{3.749154in}{3.631722in}}%
\pgfpathlineto{\pgfqpoint{3.762442in}{3.619389in}}%
\pgfpathlineto{\pgfqpoint{3.775731in}{3.607163in}}%
\pgfpathlineto{\pgfqpoint{3.789020in}{3.595044in}}%
\pgfpathlineto{\pgfqpoint{3.796707in}{3.620916in}}%
\pgfpathlineto{\pgfqpoint{3.804392in}{3.647237in}}%
\pgfpathlineto{\pgfqpoint{3.812074in}{3.674016in}}%
\pgfpathlineto{\pgfqpoint{3.819754in}{3.701262in}}%
\pgfpathlineto{\pgfqpoint{3.806460in}{3.713850in}}%
\pgfpathlineto{\pgfqpoint{3.793167in}{3.726545in}}%
\pgfpathlineto{\pgfqpoint{3.779875in}{3.739348in}}%
\pgfpathlineto{\pgfqpoint{3.766584in}{3.752259in}}%
\pgfpathlineto{\pgfqpoint{3.758909in}{3.724534in}}%
\pgfpathlineto{\pgfqpoint{3.751231in}{3.697283in}}%
\pgfpathlineto{\pgfqpoint{3.743551in}{3.670496in}}%
\pgfpathlineto{\pgfqpoint{3.735867in}{3.644164in}}%
\pgfpathclose%
\pgfusepath{fill}%
\end{pgfscope}%
\begin{pgfscope}%
\pgfpathrectangle{\pgfqpoint{1.150000in}{0.150000in}}{\pgfqpoint{5.700000in}{5.700000in}}%
\pgfusepath{clip}%
\pgfsetbuttcap%
\pgfsetroundjoin%
\definecolor{currentfill}{rgb}{0.169646,0.456262,0.558030}%
\pgfsetfillcolor{currentfill}%
\pgfsetfillopacity{0.700000}%
\pgfsetlinewidth{0.000000pt}%
\definecolor{currentstroke}{rgb}{0.000000,0.000000,0.000000}%
\pgfsetstrokecolor{currentstroke}%
\pgfsetdash{}{0pt}%
\pgfpathmoveto{\pgfqpoint{3.125497in}{3.512186in}}%
\pgfpathlineto{\pgfqpoint{3.138782in}{3.498186in}}%
\pgfpathlineto{\pgfqpoint{3.152065in}{3.484325in}}%
\pgfpathlineto{\pgfqpoint{3.165346in}{3.470603in}}%
\pgfpathlineto{\pgfqpoint{3.178625in}{3.457017in}}%
\pgfpathlineto{\pgfqpoint{3.186418in}{3.477132in}}%
\pgfpathlineto{\pgfqpoint{3.194203in}{3.497560in}}%
\pgfpathlineto{\pgfqpoint{3.201981in}{3.518307in}}%
\pgfpathlineto{\pgfqpoint{3.209752in}{3.539380in}}%
\pgfpathlineto{\pgfqpoint{3.196471in}{3.553310in}}%
\pgfpathlineto{\pgfqpoint{3.183189in}{3.567377in}}%
\pgfpathlineto{\pgfqpoint{3.169905in}{3.581583in}}%
\pgfpathlineto{\pgfqpoint{3.156620in}{3.595928in}}%
\pgfpathlineto{\pgfqpoint{3.148850in}{3.574502in}}%
\pgfpathlineto{\pgfqpoint{3.141073in}{3.553408in}}%
\pgfpathlineto{\pgfqpoint{3.133289in}{3.532638in}}%
\pgfpathlineto{\pgfqpoint{3.125497in}{3.512186in}}%
\pgfpathclose%
\pgfusepath{fill}%
\end{pgfscope}%
\begin{pgfscope}%
\pgfpathrectangle{\pgfqpoint{1.150000in}{0.150000in}}{\pgfqpoint{5.700000in}{5.700000in}}%
\pgfusepath{clip}%
\pgfsetbuttcap%
\pgfsetroundjoin%
\definecolor{currentfill}{rgb}{0.182256,0.426184,0.557120}%
\pgfsetfillcolor{currentfill}%
\pgfsetfillopacity{0.700000}%
\pgfsetlinewidth{0.000000pt}%
\definecolor{currentstroke}{rgb}{0.000000,0.000000,0.000000}%
\pgfsetstrokecolor{currentstroke}%
\pgfsetdash{}{0pt}%
\pgfpathmoveto{\pgfqpoint{3.315954in}{3.432725in}}%
\pgfpathlineto{\pgfqpoint{3.329225in}{3.419974in}}%
\pgfpathlineto{\pgfqpoint{3.342497in}{3.407349in}}%
\pgfpathlineto{\pgfqpoint{3.355767in}{3.394847in}}%
\pgfpathlineto{\pgfqpoint{3.369038in}{3.382469in}}%
\pgfpathlineto{\pgfqpoint{3.376805in}{3.402805in}}%
\pgfpathlineto{\pgfqpoint{3.384566in}{3.423463in}}%
\pgfpathlineto{\pgfqpoint{3.392321in}{3.444450in}}%
\pgfpathlineto{\pgfqpoint{3.400070in}{3.465773in}}%
\pgfpathlineto{\pgfqpoint{3.386798in}{3.478514in}}%
\pgfpathlineto{\pgfqpoint{3.373527in}{3.491379in}}%
\pgfpathlineto{\pgfqpoint{3.360255in}{3.504368in}}%
\pgfpathlineto{\pgfqpoint{3.346982in}{3.517483in}}%
\pgfpathlineto{\pgfqpoint{3.339234in}{3.495788in}}%
\pgfpathlineto{\pgfqpoint{3.331480in}{3.474435in}}%
\pgfpathlineto{\pgfqpoint{3.323720in}{3.453416in}}%
\pgfpathlineto{\pgfqpoint{3.315954in}{3.432725in}}%
\pgfpathclose%
\pgfusepath{fill}%
\end{pgfscope}%
\begin{pgfscope}%
\pgfpathrectangle{\pgfqpoint{1.150000in}{0.150000in}}{\pgfqpoint{5.700000in}{5.700000in}}%
\pgfusepath{clip}%
\pgfsetbuttcap%
\pgfsetroundjoin%
\definecolor{currentfill}{rgb}{0.157851,0.683765,0.501686}%
\pgfsetfillcolor{currentfill}%
\pgfsetfillopacity{0.700000}%
\pgfsetlinewidth{0.000000pt}%
\definecolor{currentstroke}{rgb}{0.000000,0.000000,0.000000}%
\pgfsetstrokecolor{currentstroke}%
\pgfsetdash{}{0pt}%
\pgfpathmoveto{\pgfqpoint{3.584487in}{4.098118in}}%
\pgfpathlineto{\pgfqpoint{3.597790in}{4.082845in}}%
\pgfpathlineto{\pgfqpoint{3.611092in}{4.067696in}}%
\pgfpathlineto{\pgfqpoint{3.624392in}{4.052669in}}%
\pgfpathlineto{\pgfqpoint{3.637692in}{4.037765in}}%
\pgfpathlineto{\pgfqpoint{3.645332in}{4.069534in}}%
\pgfpathlineto{\pgfqpoint{3.652969in}{4.101852in}}%
\pgfpathlineto{\pgfqpoint{3.660604in}{4.134730in}}%
\pgfpathlineto{\pgfqpoint{3.668236in}{4.168178in}}%
\pgfpathlineto{\pgfqpoint{3.654926in}{4.183608in}}%
\pgfpathlineto{\pgfqpoint{3.641615in}{4.199160in}}%
\pgfpathlineto{\pgfqpoint{3.628303in}{4.214837in}}%
\pgfpathlineto{\pgfqpoint{3.614989in}{4.230638in}}%
\pgfpathlineto{\pgfqpoint{3.607368in}{4.196653in}}%
\pgfpathlineto{\pgfqpoint{3.599744in}{4.163246in}}%
\pgfpathlineto{\pgfqpoint{3.592117in}{4.130404in}}%
\pgfpathlineto{\pgfqpoint{3.584487in}{4.098118in}}%
\pgfpathclose%
\pgfusepath{fill}%
\end{pgfscope}%
\begin{pgfscope}%
\pgfpathrectangle{\pgfqpoint{1.150000in}{0.150000in}}{\pgfqpoint{5.700000in}{5.700000in}}%
\pgfusepath{clip}%
\pgfsetbuttcap%
\pgfsetroundjoin%
\definecolor{currentfill}{rgb}{0.185783,0.704891,0.485273}%
\pgfsetfillcolor{currentfill}%
\pgfsetfillopacity{0.700000}%
\pgfsetlinewidth{0.000000pt}%
\definecolor{currentstroke}{rgb}{0.000000,0.000000,0.000000}%
\pgfsetstrokecolor{currentstroke}%
\pgfsetdash{}{0pt}%
\pgfpathmoveto{\pgfqpoint{3.531264in}{4.160474in}}%
\pgfpathlineto{\pgfqpoint{3.544572in}{4.144693in}}%
\pgfpathlineto{\pgfqpoint{3.557878in}{4.129041in}}%
\pgfpathlineto{\pgfqpoint{3.571183in}{4.113517in}}%
\pgfpathlineto{\pgfqpoint{3.584487in}{4.098118in}}%
\pgfpathlineto{\pgfqpoint{3.592117in}{4.130404in}}%
\pgfpathlineto{\pgfqpoint{3.599744in}{4.163246in}}%
\pgfpathlineto{\pgfqpoint{3.607368in}{4.196653in}}%
\pgfpathlineto{\pgfqpoint{3.614989in}{4.230638in}}%
\pgfpathlineto{\pgfqpoint{3.601674in}{4.246564in}}%
\pgfpathlineto{\pgfqpoint{3.588358in}{4.262618in}}%
\pgfpathlineto{\pgfqpoint{3.575041in}{4.278800in}}%
\pgfpathlineto{\pgfqpoint{3.561721in}{4.295111in}}%
\pgfpathlineto{\pgfqpoint{3.554112in}{4.260587in}}%
\pgfpathlineto{\pgfqpoint{3.546500in}{4.226647in}}%
\pgfpathlineto{\pgfqpoint{3.538884in}{4.193279in}}%
\pgfpathlineto{\pgfqpoint{3.531264in}{4.160474in}}%
\pgfpathclose%
\pgfusepath{fill}%
\end{pgfscope}%
\begin{pgfscope}%
\pgfpathrectangle{\pgfqpoint{1.150000in}{0.150000in}}{\pgfqpoint{5.700000in}{5.700000in}}%
\pgfusepath{clip}%
\pgfsetbuttcap%
\pgfsetroundjoin%
\definecolor{currentfill}{rgb}{0.137339,0.662252,0.515571}%
\pgfsetfillcolor{currentfill}%
\pgfsetfillopacity{0.700000}%
\pgfsetlinewidth{0.000000pt}%
\definecolor{currentstroke}{rgb}{0.000000,0.000000,0.000000}%
\pgfsetstrokecolor{currentstroke}%
\pgfsetdash{}{0pt}%
\pgfpathmoveto{\pgfqpoint{3.637692in}{4.037765in}}%
\pgfpathlineto{\pgfqpoint{3.650992in}{4.022982in}}%
\pgfpathlineto{\pgfqpoint{3.664290in}{4.008319in}}%
\pgfpathlineto{\pgfqpoint{3.677588in}{3.993775in}}%
\pgfpathlineto{\pgfqpoint{3.690886in}{3.979348in}}%
\pgfpathlineto{\pgfqpoint{3.698535in}{4.010603in}}%
\pgfpathlineto{\pgfqpoint{3.706181in}{4.042400in}}%
\pgfpathlineto{\pgfqpoint{3.713826in}{4.074751in}}%
\pgfpathlineto{\pgfqpoint{3.721468in}{4.107665in}}%
\pgfpathlineto{\pgfqpoint{3.708161in}{4.122614in}}%
\pgfpathlineto{\pgfqpoint{3.694854in}{4.137682in}}%
\pgfpathlineto{\pgfqpoint{3.681545in}{4.152870in}}%
\pgfpathlineto{\pgfqpoint{3.668236in}{4.168178in}}%
\pgfpathlineto{\pgfqpoint{3.660604in}{4.134730in}}%
\pgfpathlineto{\pgfqpoint{3.652969in}{4.101852in}}%
\pgfpathlineto{\pgfqpoint{3.645332in}{4.069534in}}%
\pgfpathlineto{\pgfqpoint{3.637692in}{4.037765in}}%
\pgfpathclose%
\pgfusepath{fill}%
\end{pgfscope}%
\begin{pgfscope}%
\pgfpathrectangle{\pgfqpoint{1.150000in}{0.150000in}}{\pgfqpoint{5.700000in}{5.700000in}}%
\pgfusepath{clip}%
\pgfsetbuttcap%
\pgfsetroundjoin%
\definecolor{currentfill}{rgb}{0.162142,0.474838,0.558140}%
\pgfsetfillcolor{currentfill}%
\pgfsetfillopacity{0.700000}%
\pgfsetlinewidth{0.000000pt}%
\definecolor{currentstroke}{rgb}{0.000000,0.000000,0.000000}%
\pgfsetstrokecolor{currentstroke}%
\pgfsetdash{}{0pt}%
\pgfpathmoveto{\pgfqpoint{3.705104in}{3.543212in}}%
\pgfpathlineto{\pgfqpoint{3.718387in}{3.531217in}}%
\pgfpathlineto{\pgfqpoint{3.731672in}{3.519331in}}%
\pgfpathlineto{\pgfqpoint{3.744957in}{3.507551in}}%
\pgfpathlineto{\pgfqpoint{3.758244in}{3.495878in}}%
\pgfpathlineto{\pgfqpoint{3.765942in}{3.520039in}}%
\pgfpathlineto{\pgfqpoint{3.773638in}{3.544615in}}%
\pgfpathlineto{\pgfqpoint{3.781330in}{3.569614in}}%
\pgfpathlineto{\pgfqpoint{3.789020in}{3.595044in}}%
\pgfpathlineto{\pgfqpoint{3.775731in}{3.607163in}}%
\pgfpathlineto{\pgfqpoint{3.762442in}{3.619389in}}%
\pgfpathlineto{\pgfqpoint{3.749154in}{3.631722in}}%
\pgfpathlineto{\pgfqpoint{3.735867in}{3.644164in}}%
\pgfpathlineto{\pgfqpoint{3.728181in}{3.618278in}}%
\pgfpathlineto{\pgfqpoint{3.720492in}{3.592829in}}%
\pgfpathlineto{\pgfqpoint{3.712799in}{3.567810in}}%
\pgfpathlineto{\pgfqpoint{3.705104in}{3.543212in}}%
\pgfpathclose%
\pgfusepath{fill}%
\end{pgfscope}%
\begin{pgfscope}%
\pgfpathrectangle{\pgfqpoint{1.150000in}{0.150000in}}{\pgfqpoint{5.700000in}{5.700000in}}%
\pgfusepath{clip}%
\pgfsetbuttcap%
\pgfsetroundjoin%
\definecolor{currentfill}{rgb}{0.183898,0.422383,0.556944}%
\pgfsetfillcolor{currentfill}%
\pgfsetfillopacity{0.700000}%
\pgfsetlinewidth{0.000000pt}%
\definecolor{currentstroke}{rgb}{0.000000,0.000000,0.000000}%
\pgfsetstrokecolor{currentstroke}%
\pgfsetdash{}{0pt}%
\pgfpathmoveto{\pgfqpoint{3.453154in}{3.416024in}}%
\pgfpathlineto{\pgfqpoint{3.466425in}{3.403885in}}%
\pgfpathlineto{\pgfqpoint{3.479697in}{3.391864in}}%
\pgfpathlineto{\pgfqpoint{3.492969in}{3.379960in}}%
\pgfpathlineto{\pgfqpoint{3.506241in}{3.368171in}}%
\pgfpathlineto{\pgfqpoint{3.513986in}{3.389086in}}%
\pgfpathlineto{\pgfqpoint{3.521726in}{3.410340in}}%
\pgfpathlineto{\pgfqpoint{3.529461in}{3.431941in}}%
\pgfpathlineto{\pgfqpoint{3.537191in}{3.453894in}}%
\pgfpathlineto{\pgfqpoint{3.523918in}{3.466066in}}%
\pgfpathlineto{\pgfqpoint{3.510645in}{3.478353in}}%
\pgfpathlineto{\pgfqpoint{3.497372in}{3.490758in}}%
\pgfpathlineto{\pgfqpoint{3.484099in}{3.503280in}}%
\pgfpathlineto{\pgfqpoint{3.476371in}{3.480935in}}%
\pgfpathlineto{\pgfqpoint{3.468637in}{3.458949in}}%
\pgfpathlineto{\pgfqpoint{3.460899in}{3.437314in}}%
\pgfpathlineto{\pgfqpoint{3.453154in}{3.416024in}}%
\pgfpathclose%
\pgfusepath{fill}%
\end{pgfscope}%
\begin{pgfscope}%
\pgfpathrectangle{\pgfqpoint{1.150000in}{0.150000in}}{\pgfqpoint{5.700000in}{5.700000in}}%
\pgfusepath{clip}%
\pgfsetbuttcap%
\pgfsetroundjoin%
\definecolor{currentfill}{rgb}{0.132444,0.552216,0.553018}%
\pgfsetfillcolor{currentfill}%
\pgfsetfillopacity{0.700000}%
\pgfsetlinewidth{0.000000pt}%
\definecolor{currentstroke}{rgb}{0.000000,0.000000,0.000000}%
\pgfsetstrokecolor{currentstroke}%
\pgfsetdash{}{0pt}%
\pgfpathmoveto{\pgfqpoint{3.766584in}{3.752259in}}%
\pgfpathlineto{\pgfqpoint{3.779875in}{3.739348in}}%
\pgfpathlineto{\pgfqpoint{3.793167in}{3.726545in}}%
\pgfpathlineto{\pgfqpoint{3.806460in}{3.713850in}}%
\pgfpathlineto{\pgfqpoint{3.819754in}{3.701262in}}%
\pgfpathlineto{\pgfqpoint{3.827432in}{3.728985in}}%
\pgfpathlineto{\pgfqpoint{3.835108in}{3.757194in}}%
\pgfpathlineto{\pgfqpoint{3.842783in}{3.785898in}}%
\pgfpathlineto{\pgfqpoint{3.850456in}{3.815106in}}%
\pgfpathlineto{\pgfqpoint{3.837157in}{3.828187in}}%
\pgfpathlineto{\pgfqpoint{3.823858in}{3.841375in}}%
\pgfpathlineto{\pgfqpoint{3.810560in}{3.854671in}}%
\pgfpathlineto{\pgfqpoint{3.797262in}{3.868076in}}%
\pgfpathlineto{\pgfqpoint{3.789595in}{3.838365in}}%
\pgfpathlineto{\pgfqpoint{3.781927in}{3.809165in}}%
\pgfpathlineto{\pgfqpoint{3.774256in}{3.780466in}}%
\pgfpathlineto{\pgfqpoint{3.766584in}{3.752259in}}%
\pgfpathclose%
\pgfusepath{fill}%
\end{pgfscope}%
\begin{pgfscope}%
\pgfpathrectangle{\pgfqpoint{1.150000in}{0.150000in}}{\pgfqpoint{5.700000in}{5.700000in}}%
\pgfusepath{clip}%
\pgfsetbuttcap%
\pgfsetroundjoin%
\definecolor{currentfill}{rgb}{0.226397,0.728888,0.462789}%
\pgfsetfillcolor{currentfill}%
\pgfsetfillopacity{0.700000}%
\pgfsetlinewidth{0.000000pt}%
\definecolor{currentstroke}{rgb}{0.000000,0.000000,0.000000}%
\pgfsetstrokecolor{currentstroke}%
\pgfsetdash{}{0pt}%
\pgfpathmoveto{\pgfqpoint{3.478016in}{4.224904in}}%
\pgfpathlineto{\pgfqpoint{3.491331in}{4.208598in}}%
\pgfpathlineto{\pgfqpoint{3.504644in}{4.192425in}}%
\pgfpathlineto{\pgfqpoint{3.517955in}{4.176384in}}%
\pgfpathlineto{\pgfqpoint{3.531264in}{4.160474in}}%
\pgfpathlineto{\pgfqpoint{3.538884in}{4.193279in}}%
\pgfpathlineto{\pgfqpoint{3.546500in}{4.226647in}}%
\pgfpathlineto{\pgfqpoint{3.554112in}{4.260587in}}%
\pgfpathlineto{\pgfqpoint{3.561721in}{4.295111in}}%
\pgfpathlineto{\pgfqpoint{3.548400in}{4.311552in}}%
\pgfpathlineto{\pgfqpoint{3.535078in}{4.328125in}}%
\pgfpathlineto{\pgfqpoint{3.521753in}{4.344830in}}%
\pgfpathlineto{\pgfqpoint{3.508426in}{4.361670in}}%
\pgfpathlineto{\pgfqpoint{3.500829in}{4.326604in}}%
\pgfpathlineto{\pgfqpoint{3.493229in}{4.292129in}}%
\pgfpathlineto{\pgfqpoint{3.485625in}{4.258232in}}%
\pgfpathlineto{\pgfqpoint{3.478016in}{4.224904in}}%
\pgfpathclose%
\pgfusepath{fill}%
\end{pgfscope}%
\begin{pgfscope}%
\pgfpathrectangle{\pgfqpoint{1.150000in}{0.150000in}}{\pgfqpoint{5.700000in}{5.700000in}}%
\pgfusepath{clip}%
\pgfsetbuttcap%
\pgfsetroundjoin%
\definecolor{currentfill}{rgb}{0.124780,0.640461,0.527068}%
\pgfsetfillcolor{currentfill}%
\pgfsetfillopacity{0.700000}%
\pgfsetlinewidth{0.000000pt}%
\definecolor{currentstroke}{rgb}{0.000000,0.000000,0.000000}%
\pgfsetstrokecolor{currentstroke}%
\pgfsetdash{}{0pt}%
\pgfpathmoveto{\pgfqpoint{3.690886in}{3.979348in}}%
\pgfpathlineto{\pgfqpoint{3.704183in}{3.965039in}}%
\pgfpathlineto{\pgfqpoint{3.717480in}{3.950846in}}%
\pgfpathlineto{\pgfqpoint{3.730777in}{3.936769in}}%
\pgfpathlineto{\pgfqpoint{3.744074in}{3.922805in}}%
\pgfpathlineto{\pgfqpoint{3.751731in}{3.953547in}}%
\pgfpathlineto{\pgfqpoint{3.759386in}{3.984826in}}%
\pgfpathlineto{\pgfqpoint{3.767040in}{4.016651in}}%
\pgfpathlineto{\pgfqpoint{3.774692in}{4.049034in}}%
\pgfpathlineto{\pgfqpoint{3.761387in}{4.063518in}}%
\pgfpathlineto{\pgfqpoint{3.748081in}{4.078118in}}%
\pgfpathlineto{\pgfqpoint{3.734775in}{4.092833in}}%
\pgfpathlineto{\pgfqpoint{3.721468in}{4.107665in}}%
\pgfpathlineto{\pgfqpoint{3.713826in}{4.074751in}}%
\pgfpathlineto{\pgfqpoint{3.706181in}{4.042400in}}%
\pgfpathlineto{\pgfqpoint{3.698535in}{4.010603in}}%
\pgfpathlineto{\pgfqpoint{3.690886in}{3.979348in}}%
\pgfpathclose%
\pgfusepath{fill}%
\end{pgfscope}%
\begin{pgfscope}%
\pgfpathrectangle{\pgfqpoint{1.150000in}{0.150000in}}{\pgfqpoint{5.700000in}{5.700000in}}%
\pgfusepath{clip}%
\pgfsetbuttcap%
\pgfsetroundjoin%
\definecolor{currentfill}{rgb}{0.177423,0.437527,0.557565}%
\pgfsetfillcolor{currentfill}%
\pgfsetfillopacity{0.700000}%
\pgfsetlinewidth{0.000000pt}%
\definecolor{currentstroke}{rgb}{0.000000,0.000000,0.000000}%
\pgfsetstrokecolor{currentstroke}%
\pgfsetdash{}{0pt}%
\pgfpathmoveto{\pgfqpoint{3.178625in}{3.457017in}}%
\pgfpathlineto{\pgfqpoint{3.191904in}{3.443567in}}%
\pgfpathlineto{\pgfqpoint{3.205181in}{3.430251in}}%
\pgfpathlineto{\pgfqpoint{3.218457in}{3.417068in}}%
\pgfpathlineto{\pgfqpoint{3.231732in}{3.404017in}}%
\pgfpathlineto{\pgfqpoint{3.239524in}{3.423797in}}%
\pgfpathlineto{\pgfqpoint{3.247310in}{3.443883in}}%
\pgfpathlineto{\pgfqpoint{3.255089in}{3.464284in}}%
\pgfpathlineto{\pgfqpoint{3.262861in}{3.485004in}}%
\pgfpathlineto{\pgfqpoint{3.249585in}{3.498399in}}%
\pgfpathlineto{\pgfqpoint{3.236309in}{3.511925in}}%
\pgfpathlineto{\pgfqpoint{3.223031in}{3.525585in}}%
\pgfpathlineto{\pgfqpoint{3.209752in}{3.539380in}}%
\pgfpathlineto{\pgfqpoint{3.201981in}{3.518307in}}%
\pgfpathlineto{\pgfqpoint{3.194203in}{3.497560in}}%
\pgfpathlineto{\pgfqpoint{3.186418in}{3.477132in}}%
\pgfpathlineto{\pgfqpoint{3.178625in}{3.457017in}}%
\pgfpathclose%
\pgfusepath{fill}%
\end{pgfscope}%
\begin{pgfscope}%
\pgfpathrectangle{\pgfqpoint{1.150000in}{0.150000in}}{\pgfqpoint{5.700000in}{5.700000in}}%
\pgfusepath{clip}%
\pgfsetbuttcap%
\pgfsetroundjoin%
\definecolor{currentfill}{rgb}{0.119699,0.618490,0.536347}%
\pgfsetfillcolor{currentfill}%
\pgfsetfillopacity{0.700000}%
\pgfsetlinewidth{0.000000pt}%
\definecolor{currentstroke}{rgb}{0.000000,0.000000,0.000000}%
\pgfsetstrokecolor{currentstroke}%
\pgfsetdash{}{0pt}%
\pgfpathmoveto{\pgfqpoint{3.744074in}{3.922805in}}%
\pgfpathlineto{\pgfqpoint{3.757371in}{3.908955in}}%
\pgfpathlineto{\pgfqpoint{3.770668in}{3.895217in}}%
\pgfpathlineto{\pgfqpoint{3.783965in}{3.881591in}}%
\pgfpathlineto{\pgfqpoint{3.797262in}{3.868076in}}%
\pgfpathlineto{\pgfqpoint{3.804927in}{3.898308in}}%
\pgfpathlineto{\pgfqpoint{3.812590in}{3.929070in}}%
\pgfpathlineto{\pgfqpoint{3.820252in}{3.960373in}}%
\pgfpathlineto{\pgfqpoint{3.827913in}{3.992227in}}%
\pgfpathlineto{\pgfqpoint{3.814608in}{4.006261in}}%
\pgfpathlineto{\pgfqpoint{3.801302in}{4.020406in}}%
\pgfpathlineto{\pgfqpoint{3.787997in}{4.034663in}}%
\pgfpathlineto{\pgfqpoint{3.774692in}{4.049034in}}%
\pgfpathlineto{\pgfqpoint{3.767040in}{4.016651in}}%
\pgfpathlineto{\pgfqpoint{3.759386in}{3.984826in}}%
\pgfpathlineto{\pgfqpoint{3.751731in}{3.953547in}}%
\pgfpathlineto{\pgfqpoint{3.744074in}{3.922805in}}%
\pgfpathclose%
\pgfusepath{fill}%
\end{pgfscope}%
\begin{pgfscope}%
\pgfpathrectangle{\pgfqpoint{1.150000in}{0.150000in}}{\pgfqpoint{5.700000in}{5.700000in}}%
\pgfusepath{clip}%
\pgfsetbuttcap%
\pgfsetroundjoin%
\definecolor{currentfill}{rgb}{0.274149,0.751988,0.436601}%
\pgfsetfillcolor{currentfill}%
\pgfsetfillopacity{0.700000}%
\pgfsetlinewidth{0.000000pt}%
\definecolor{currentstroke}{rgb}{0.000000,0.000000,0.000000}%
\pgfsetstrokecolor{currentstroke}%
\pgfsetdash{}{0pt}%
\pgfpathmoveto{\pgfqpoint{3.424737in}{4.291486in}}%
\pgfpathlineto{\pgfqpoint{3.438060in}{4.274634in}}%
\pgfpathlineto{\pgfqpoint{3.451381in}{4.257921in}}%
\pgfpathlineto{\pgfqpoint{3.464700in}{4.241345in}}%
\pgfpathlineto{\pgfqpoint{3.478016in}{4.224904in}}%
\pgfpathlineto{\pgfqpoint{3.485625in}{4.258232in}}%
\pgfpathlineto{\pgfqpoint{3.493229in}{4.292129in}}%
\pgfpathlineto{\pgfqpoint{3.500829in}{4.326604in}}%
\pgfpathlineto{\pgfqpoint{3.508426in}{4.361670in}}%
\pgfpathlineto{\pgfqpoint{3.495097in}{4.378644in}}%
\pgfpathlineto{\pgfqpoint{3.481766in}{4.395756in}}%
\pgfpathlineto{\pgfqpoint{3.468432in}{4.413004in}}%
\pgfpathlineto{\pgfqpoint{3.455096in}{4.430392in}}%
\pgfpathlineto{\pgfqpoint{3.447513in}{4.394781in}}%
\pgfpathlineto{\pgfqpoint{3.439926in}{4.359767in}}%
\pgfpathlineto{\pgfqpoint{3.432334in}{4.325339in}}%
\pgfpathlineto{\pgfqpoint{3.424737in}{4.291486in}}%
\pgfpathclose%
\pgfusepath{fill}%
\end{pgfscope}%
\begin{pgfscope}%
\pgfpathrectangle{\pgfqpoint{1.150000in}{0.150000in}}{\pgfqpoint{5.700000in}{5.700000in}}%
\pgfusepath{clip}%
\pgfsetbuttcap%
\pgfsetroundjoin%
\definecolor{currentfill}{rgb}{0.154815,0.493313,0.557840}%
\pgfsetfillcolor{currentfill}%
\pgfsetfillopacity{0.700000}%
\pgfsetlinewidth{0.000000pt}%
\definecolor{currentstroke}{rgb}{0.000000,0.000000,0.000000}%
\pgfsetstrokecolor{currentstroke}%
\pgfsetdash{}{0pt}%
\pgfpathmoveto{\pgfqpoint{3.789020in}{3.595044in}}%
\pgfpathlineto{\pgfqpoint{3.802311in}{3.583031in}}%
\pgfpathlineto{\pgfqpoint{3.815602in}{3.571123in}}%
\pgfpathlineto{\pgfqpoint{3.828895in}{3.559320in}}%
\pgfpathlineto{\pgfqpoint{3.842190in}{3.547620in}}%
\pgfpathlineto{\pgfqpoint{3.849880in}{3.573032in}}%
\pgfpathlineto{\pgfqpoint{3.857568in}{3.598888in}}%
\pgfpathlineto{\pgfqpoint{3.865254in}{3.625196in}}%
\pgfpathlineto{\pgfqpoint{3.872939in}{3.651966in}}%
\pgfpathlineto{\pgfqpoint{3.859641in}{3.664133in}}%
\pgfpathlineto{\pgfqpoint{3.846344in}{3.676404in}}%
\pgfpathlineto{\pgfqpoint{3.833049in}{3.688780in}}%
\pgfpathlineto{\pgfqpoint{3.819754in}{3.701262in}}%
\pgfpathlineto{\pgfqpoint{3.812074in}{3.674016in}}%
\pgfpathlineto{\pgfqpoint{3.804392in}{3.647237in}}%
\pgfpathlineto{\pgfqpoint{3.796707in}{3.620916in}}%
\pgfpathlineto{\pgfqpoint{3.789020in}{3.595044in}}%
\pgfpathclose%
\pgfusepath{fill}%
\end{pgfscope}%
\begin{pgfscope}%
\pgfpathrectangle{\pgfqpoint{1.150000in}{0.150000in}}{\pgfqpoint{5.700000in}{5.700000in}}%
\pgfusepath{clip}%
\pgfsetbuttcap%
\pgfsetroundjoin%
\definecolor{currentfill}{rgb}{0.177423,0.437527,0.557565}%
\pgfsetfillcolor{currentfill}%
\pgfsetfillopacity{0.700000}%
\pgfsetlinewidth{0.000000pt}%
\definecolor{currentstroke}{rgb}{0.000000,0.000000,0.000000}%
\pgfsetstrokecolor{currentstroke}%
\pgfsetdash{}{0pt}%
\pgfpathmoveto{\pgfqpoint{3.674284in}{3.448858in}}%
\pgfpathlineto{\pgfqpoint{3.687566in}{3.437288in}}%
\pgfpathlineto{\pgfqpoint{3.700848in}{3.425825in}}%
\pgfpathlineto{\pgfqpoint{3.714132in}{3.414469in}}%
\pgfpathlineto{\pgfqpoint{3.727417in}{3.403220in}}%
\pgfpathlineto{\pgfqpoint{3.735129in}{3.425803in}}%
\pgfpathlineto{\pgfqpoint{3.742837in}{3.448768in}}%
\pgfpathlineto{\pgfqpoint{3.750542in}{3.472124in}}%
\pgfpathlineto{\pgfqpoint{3.758244in}{3.495878in}}%
\pgfpathlineto{\pgfqpoint{3.744957in}{3.507551in}}%
\pgfpathlineto{\pgfqpoint{3.731672in}{3.519331in}}%
\pgfpathlineto{\pgfqpoint{3.718387in}{3.531217in}}%
\pgfpathlineto{\pgfqpoint{3.705104in}{3.543212in}}%
\pgfpathlineto{\pgfqpoint{3.697404in}{3.519025in}}%
\pgfpathlineto{\pgfqpoint{3.689701in}{3.495243in}}%
\pgfpathlineto{\pgfqpoint{3.681995in}{3.471856in}}%
\pgfpathlineto{\pgfqpoint{3.674284in}{3.448858in}}%
\pgfpathclose%
\pgfusepath{fill}%
\end{pgfscope}%
\begin{pgfscope}%
\pgfpathrectangle{\pgfqpoint{1.150000in}{0.150000in}}{\pgfqpoint{5.700000in}{5.700000in}}%
\pgfusepath{clip}%
\pgfsetbuttcap%
\pgfsetroundjoin%
\definecolor{currentfill}{rgb}{0.183898,0.422383,0.556944}%
\pgfsetfillcolor{currentfill}%
\pgfsetfillopacity{0.700000}%
\pgfsetlinewidth{0.000000pt}%
\definecolor{currentstroke}{rgb}{0.000000,0.000000,0.000000}%
\pgfsetstrokecolor{currentstroke}%
\pgfsetdash{}{0pt}%
\pgfpathmoveto{\pgfqpoint{3.590290in}{3.406349in}}%
\pgfpathlineto{\pgfqpoint{3.603566in}{3.394744in}}%
\pgfpathlineto{\pgfqpoint{3.616843in}{3.383249in}}%
\pgfpathlineto{\pgfqpoint{3.630122in}{3.371864in}}%
\pgfpathlineto{\pgfqpoint{3.643401in}{3.360588in}}%
\pgfpathlineto{\pgfqpoint{3.651128in}{3.382112in}}%
\pgfpathlineto{\pgfqpoint{3.658851in}{3.403993in}}%
\pgfpathlineto{\pgfqpoint{3.666570in}{3.426239in}}%
\pgfpathlineto{\pgfqpoint{3.674284in}{3.448858in}}%
\pgfpathlineto{\pgfqpoint{3.661004in}{3.460536in}}%
\pgfpathlineto{\pgfqpoint{3.647724in}{3.472324in}}%
\pgfpathlineto{\pgfqpoint{3.634445in}{3.484222in}}%
\pgfpathlineto{\pgfqpoint{3.621167in}{3.496231in}}%
\pgfpathlineto{\pgfqpoint{3.613455in}{3.473201in}}%
\pgfpathlineto{\pgfqpoint{3.605737in}{3.450549in}}%
\pgfpathlineto{\pgfqpoint{3.598016in}{3.428268in}}%
\pgfpathlineto{\pgfqpoint{3.590290in}{3.406349in}}%
\pgfpathclose%
\pgfusepath{fill}%
\end{pgfscope}%
\begin{pgfscope}%
\pgfpathrectangle{\pgfqpoint{1.150000in}{0.150000in}}{\pgfqpoint{5.700000in}{5.700000in}}%
\pgfusepath{clip}%
\pgfsetbuttcap%
\pgfsetroundjoin%
\definecolor{currentfill}{rgb}{0.188923,0.410910,0.556326}%
\pgfsetfillcolor{currentfill}%
\pgfsetfillopacity{0.700000}%
\pgfsetlinewidth{0.000000pt}%
\definecolor{currentstroke}{rgb}{0.000000,0.000000,0.000000}%
\pgfsetstrokecolor{currentstroke}%
\pgfsetdash{}{0pt}%
\pgfpathmoveto{\pgfqpoint{3.369038in}{3.382469in}}%
\pgfpathlineto{\pgfqpoint{3.382308in}{3.370214in}}%
\pgfpathlineto{\pgfqpoint{3.395579in}{3.358079in}}%
\pgfpathlineto{\pgfqpoint{3.408849in}{3.346065in}}%
\pgfpathlineto{\pgfqpoint{3.422120in}{3.334169in}}%
\pgfpathlineto{\pgfqpoint{3.429887in}{3.354151in}}%
\pgfpathlineto{\pgfqpoint{3.437649in}{3.374449in}}%
\pgfpathlineto{\pgfqpoint{3.445404in}{3.395071in}}%
\pgfpathlineto{\pgfqpoint{3.453154in}{3.416024in}}%
\pgfpathlineto{\pgfqpoint{3.439883in}{3.428281in}}%
\pgfpathlineto{\pgfqpoint{3.426612in}{3.440658in}}%
\pgfpathlineto{\pgfqpoint{3.413341in}{3.453155in}}%
\pgfpathlineto{\pgfqpoint{3.400070in}{3.465773in}}%
\pgfpathlineto{\pgfqpoint{3.392321in}{3.444450in}}%
\pgfpathlineto{\pgfqpoint{3.384566in}{3.423463in}}%
\pgfpathlineto{\pgfqpoint{3.376805in}{3.402805in}}%
\pgfpathlineto{\pgfqpoint{3.369038in}{3.382469in}}%
\pgfpathclose%
\pgfusepath{fill}%
\end{pgfscope}%
\begin{pgfscope}%
\pgfpathrectangle{\pgfqpoint{1.150000in}{0.150000in}}{\pgfqpoint{5.700000in}{5.700000in}}%
\pgfusepath{clip}%
\pgfsetbuttcap%
\pgfsetroundjoin%
\definecolor{currentfill}{rgb}{0.137770,0.537492,0.554906}%
\pgfsetfillcolor{currentfill}%
\pgfsetfillopacity{0.700000}%
\pgfsetlinewidth{0.000000pt}%
\definecolor{currentstroke}{rgb}{0.000000,0.000000,0.000000}%
\pgfsetstrokecolor{currentstroke}%
\pgfsetdash{}{0pt}%
\pgfpathmoveto{\pgfqpoint{3.819754in}{3.701262in}}%
\pgfpathlineto{\pgfqpoint{3.833049in}{3.688780in}}%
\pgfpathlineto{\pgfqpoint{3.846344in}{3.676404in}}%
\pgfpathlineto{\pgfqpoint{3.859641in}{3.664133in}}%
\pgfpathlineto{\pgfqpoint{3.872939in}{3.651966in}}%
\pgfpathlineto{\pgfqpoint{3.880621in}{3.679206in}}%
\pgfpathlineto{\pgfqpoint{3.888302in}{3.706926in}}%
\pgfpathlineto{\pgfqpoint{3.895982in}{3.735136in}}%
\pgfpathlineto{\pgfqpoint{3.903661in}{3.763844in}}%
\pgfpathlineto{\pgfqpoint{3.890358in}{3.776503in}}%
\pgfpathlineto{\pgfqpoint{3.877057in}{3.789265in}}%
\pgfpathlineto{\pgfqpoint{3.863756in}{3.802133in}}%
\pgfpathlineto{\pgfqpoint{3.850456in}{3.815106in}}%
\pgfpathlineto{\pgfqpoint{3.842783in}{3.785898in}}%
\pgfpathlineto{\pgfqpoint{3.835108in}{3.757194in}}%
\pgfpathlineto{\pgfqpoint{3.827432in}{3.728985in}}%
\pgfpathlineto{\pgfqpoint{3.819754in}{3.701262in}}%
\pgfpathclose%
\pgfusepath{fill}%
\end{pgfscope}%
\begin{pgfscope}%
\pgfpathrectangle{\pgfqpoint{1.150000in}{0.150000in}}{\pgfqpoint{5.700000in}{5.700000in}}%
\pgfusepath{clip}%
\pgfsetbuttcap%
\pgfsetroundjoin%
\definecolor{currentfill}{rgb}{0.185556,0.418570,0.556753}%
\pgfsetfillcolor{currentfill}%
\pgfsetfillopacity{0.700000}%
\pgfsetlinewidth{0.000000pt}%
\definecolor{currentstroke}{rgb}{0.000000,0.000000,0.000000}%
\pgfsetstrokecolor{currentstroke}%
\pgfsetdash{}{0pt}%
\pgfpathmoveto{\pgfqpoint{3.231732in}{3.404017in}}%
\pgfpathlineto{\pgfqpoint{3.245006in}{3.391097in}}%
\pgfpathlineto{\pgfqpoint{3.258279in}{3.378306in}}%
\pgfpathlineto{\pgfqpoint{3.271552in}{3.365644in}}%
\pgfpathlineto{\pgfqpoint{3.284824in}{3.353109in}}%
\pgfpathlineto{\pgfqpoint{3.292616in}{3.372554in}}%
\pgfpathlineto{\pgfqpoint{3.300402in}{3.392300in}}%
\pgfpathlineto{\pgfqpoint{3.308181in}{3.412356in}}%
\pgfpathlineto{\pgfqpoint{3.315954in}{3.432725in}}%
\pgfpathlineto{\pgfqpoint{3.302682in}{3.445602in}}%
\pgfpathlineto{\pgfqpoint{3.289409in}{3.458607in}}%
\pgfpathlineto{\pgfqpoint{3.276135in}{3.471741in}}%
\pgfpathlineto{\pgfqpoint{3.262861in}{3.485004in}}%
\pgfpathlineto{\pgfqpoint{3.255089in}{3.464284in}}%
\pgfpathlineto{\pgfqpoint{3.247310in}{3.443883in}}%
\pgfpathlineto{\pgfqpoint{3.239524in}{3.423797in}}%
\pgfpathlineto{\pgfqpoint{3.231732in}{3.404017in}}%
\pgfpathclose%
\pgfusepath{fill}%
\end{pgfscope}%
\begin{pgfscope}%
\pgfpathrectangle{\pgfqpoint{1.150000in}{0.150000in}}{\pgfqpoint{5.700000in}{5.700000in}}%
\pgfusepath{clip}%
\pgfsetbuttcap%
\pgfsetroundjoin%
\definecolor{currentfill}{rgb}{0.120092,0.600104,0.542530}%
\pgfsetfillcolor{currentfill}%
\pgfsetfillopacity{0.700000}%
\pgfsetlinewidth{0.000000pt}%
\definecolor{currentstroke}{rgb}{0.000000,0.000000,0.000000}%
\pgfsetstrokecolor{currentstroke}%
\pgfsetdash{}{0pt}%
\pgfpathmoveto{\pgfqpoint{3.797262in}{3.868076in}}%
\pgfpathlineto{\pgfqpoint{3.810560in}{3.854671in}}%
\pgfpathlineto{\pgfqpoint{3.823858in}{3.841375in}}%
\pgfpathlineto{\pgfqpoint{3.837157in}{3.828187in}}%
\pgfpathlineto{\pgfqpoint{3.850456in}{3.815106in}}%
\pgfpathlineto{\pgfqpoint{3.858128in}{3.844830in}}%
\pgfpathlineto{\pgfqpoint{3.865798in}{3.875078in}}%
\pgfpathlineto{\pgfqpoint{3.873468in}{3.905860in}}%
\pgfpathlineto{\pgfqpoint{3.881136in}{3.937187in}}%
\pgfpathlineto{\pgfqpoint{3.867830in}{3.950784in}}%
\pgfpathlineto{\pgfqpoint{3.854524in}{3.964489in}}%
\pgfpathlineto{\pgfqpoint{3.841218in}{3.978303in}}%
\pgfpathlineto{\pgfqpoint{3.827913in}{3.992227in}}%
\pgfpathlineto{\pgfqpoint{3.820252in}{3.960373in}}%
\pgfpathlineto{\pgfqpoint{3.812590in}{3.929070in}}%
\pgfpathlineto{\pgfqpoint{3.804927in}{3.898308in}}%
\pgfpathlineto{\pgfqpoint{3.797262in}{3.868076in}}%
\pgfpathclose%
\pgfusepath{fill}%
\end{pgfscope}%
\begin{pgfscope}%
\pgfpathrectangle{\pgfqpoint{1.150000in}{0.150000in}}{\pgfqpoint{5.700000in}{5.700000in}}%
\pgfusepath{clip}%
\pgfsetbuttcap%
\pgfsetroundjoin%
\definecolor{currentfill}{rgb}{0.169646,0.456262,0.558030}%
\pgfsetfillcolor{currentfill}%
\pgfsetfillopacity{0.700000}%
\pgfsetlinewidth{0.000000pt}%
\definecolor{currentstroke}{rgb}{0.000000,0.000000,0.000000}%
\pgfsetstrokecolor{currentstroke}%
\pgfsetdash{}{0pt}%
\pgfpathmoveto{\pgfqpoint{3.758244in}{3.495878in}}%
\pgfpathlineto{\pgfqpoint{3.771532in}{3.484311in}}%
\pgfpathlineto{\pgfqpoint{3.784821in}{3.472848in}}%
\pgfpathlineto{\pgfqpoint{3.798112in}{3.461490in}}%
\pgfpathlineto{\pgfqpoint{3.811404in}{3.450234in}}%
\pgfpathlineto{\pgfqpoint{3.819104in}{3.473958in}}%
\pgfpathlineto{\pgfqpoint{3.826802in}{3.498092in}}%
\pgfpathlineto{\pgfqpoint{3.834497in}{3.522643in}}%
\pgfpathlineto{\pgfqpoint{3.842190in}{3.547620in}}%
\pgfpathlineto{\pgfqpoint{3.828895in}{3.559320in}}%
\pgfpathlineto{\pgfqpoint{3.815602in}{3.571123in}}%
\pgfpathlineto{\pgfqpoint{3.802311in}{3.583031in}}%
\pgfpathlineto{\pgfqpoint{3.789020in}{3.595044in}}%
\pgfpathlineto{\pgfqpoint{3.781330in}{3.569614in}}%
\pgfpathlineto{\pgfqpoint{3.773638in}{3.544615in}}%
\pgfpathlineto{\pgfqpoint{3.765942in}{3.520039in}}%
\pgfpathlineto{\pgfqpoint{3.758244in}{3.495878in}}%
\pgfpathclose%
\pgfusepath{fill}%
\end{pgfscope}%
\begin{pgfscope}%
\pgfpathrectangle{\pgfqpoint{1.150000in}{0.150000in}}{\pgfqpoint{5.700000in}{5.700000in}}%
\pgfusepath{clip}%
\pgfsetbuttcap%
\pgfsetroundjoin%
\definecolor{currentfill}{rgb}{0.190631,0.407061,0.556089}%
\pgfsetfillcolor{currentfill}%
\pgfsetfillopacity{0.700000}%
\pgfsetlinewidth{0.000000pt}%
\definecolor{currentstroke}{rgb}{0.000000,0.000000,0.000000}%
\pgfsetstrokecolor{currentstroke}%
\pgfsetdash{}{0pt}%
\pgfpathmoveto{\pgfqpoint{3.506241in}{3.368171in}}%
\pgfpathlineto{\pgfqpoint{3.519514in}{3.356497in}}%
\pgfpathlineto{\pgfqpoint{3.532788in}{3.344936in}}%
\pgfpathlineto{\pgfqpoint{3.546062in}{3.333489in}}%
\pgfpathlineto{\pgfqpoint{3.559337in}{3.322154in}}%
\pgfpathlineto{\pgfqpoint{3.567082in}{3.342695in}}%
\pgfpathlineto{\pgfqpoint{3.574823in}{3.363570in}}%
\pgfpathlineto{\pgfqpoint{3.582559in}{3.384786in}}%
\pgfpathlineto{\pgfqpoint{3.590290in}{3.406349in}}%
\pgfpathlineto{\pgfqpoint{3.577014in}{3.418066in}}%
\pgfpathlineto{\pgfqpoint{3.563739in}{3.429895in}}%
\pgfpathlineto{\pgfqpoint{3.550465in}{3.441838in}}%
\pgfpathlineto{\pgfqpoint{3.537191in}{3.453894in}}%
\pgfpathlineto{\pgfqpoint{3.529461in}{3.431941in}}%
\pgfpathlineto{\pgfqpoint{3.521726in}{3.410340in}}%
\pgfpathlineto{\pgfqpoint{3.513986in}{3.389086in}}%
\pgfpathlineto{\pgfqpoint{3.506241in}{3.368171in}}%
\pgfpathclose%
\pgfusepath{fill}%
\end{pgfscope}%
\begin{pgfscope}%
\pgfpathrectangle{\pgfqpoint{1.150000in}{0.150000in}}{\pgfqpoint{5.700000in}{5.700000in}}%
\pgfusepath{clip}%
\pgfsetbuttcap%
\pgfsetroundjoin%
\definecolor{currentfill}{rgb}{0.327796,0.773980,0.406640}%
\pgfsetfillcolor{currentfill}%
\pgfsetfillopacity{0.700000}%
\pgfsetlinewidth{0.000000pt}%
\definecolor{currentstroke}{rgb}{0.000000,0.000000,0.000000}%
\pgfsetstrokecolor{currentstroke}%
\pgfsetdash{}{0pt}%
\pgfpathmoveto{\pgfqpoint{3.371419in}{4.360299in}}%
\pgfpathlineto{\pgfqpoint{3.384753in}{4.342882in}}%
\pgfpathlineto{\pgfqpoint{3.398083in}{4.325608in}}%
\pgfpathlineto{\pgfqpoint{3.411412in}{4.308476in}}%
\pgfpathlineto{\pgfqpoint{3.424737in}{4.291486in}}%
\pgfpathlineto{\pgfqpoint{3.432334in}{4.325339in}}%
\pgfpathlineto{\pgfqpoint{3.439926in}{4.359767in}}%
\pgfpathlineto{\pgfqpoint{3.447513in}{4.394781in}}%
\pgfpathlineto{\pgfqpoint{3.455096in}{4.430392in}}%
\pgfpathlineto{\pgfqpoint{3.441758in}{4.447920in}}%
\pgfpathlineto{\pgfqpoint{3.428416in}{4.465589in}}%
\pgfpathlineto{\pgfqpoint{3.415072in}{4.483402in}}%
\pgfpathlineto{\pgfqpoint{3.401725in}{4.501359in}}%
\pgfpathlineto{\pgfqpoint{3.394156in}{4.465199in}}%
\pgfpathlineto{\pgfqpoint{3.386582in}{4.429643in}}%
\pgfpathlineto{\pgfqpoint{3.379003in}{4.394680in}}%
\pgfpathlineto{\pgfqpoint{3.371419in}{4.360299in}}%
\pgfpathclose%
\pgfusepath{fill}%
\end{pgfscope}%
\begin{pgfscope}%
\pgfpathrectangle{\pgfqpoint{1.150000in}{0.150000in}}{\pgfqpoint{5.700000in}{5.700000in}}%
\pgfusepath{clip}%
\pgfsetbuttcap%
\pgfsetroundjoin%
\definecolor{currentfill}{rgb}{0.123463,0.581687,0.547445}%
\pgfsetfillcolor{currentfill}%
\pgfsetfillopacity{0.700000}%
\pgfsetlinewidth{0.000000pt}%
\definecolor{currentstroke}{rgb}{0.000000,0.000000,0.000000}%
\pgfsetstrokecolor{currentstroke}%
\pgfsetdash{}{0pt}%
\pgfpathmoveto{\pgfqpoint{3.850456in}{3.815106in}}%
\pgfpathlineto{\pgfqpoint{3.863756in}{3.802133in}}%
\pgfpathlineto{\pgfqpoint{3.877057in}{3.789265in}}%
\pgfpathlineto{\pgfqpoint{3.890358in}{3.776503in}}%
\pgfpathlineto{\pgfqpoint{3.903661in}{3.763844in}}%
\pgfpathlineto{\pgfqpoint{3.911339in}{3.793061in}}%
\pgfpathlineto{\pgfqpoint{3.919016in}{3.822797in}}%
\pgfpathlineto{\pgfqpoint{3.926692in}{3.853061in}}%
\pgfpathlineto{\pgfqpoint{3.934368in}{3.883863in}}%
\pgfpathlineto{\pgfqpoint{3.921059in}{3.897036in}}%
\pgfpathlineto{\pgfqpoint{3.907751in}{3.910314in}}%
\pgfpathlineto{\pgfqpoint{3.894443in}{3.923697in}}%
\pgfpathlineto{\pgfqpoint{3.881136in}{3.937187in}}%
\pgfpathlineto{\pgfqpoint{3.873468in}{3.905860in}}%
\pgfpathlineto{\pgfqpoint{3.865798in}{3.875078in}}%
\pgfpathlineto{\pgfqpoint{3.858128in}{3.844830in}}%
\pgfpathlineto{\pgfqpoint{3.850456in}{3.815106in}}%
\pgfpathclose%
\pgfusepath{fill}%
\end{pgfscope}%
\begin{pgfscope}%
\pgfpathrectangle{\pgfqpoint{1.150000in}{0.150000in}}{\pgfqpoint{5.700000in}{5.700000in}}%
\pgfusepath{clip}%
\pgfsetbuttcap%
\pgfsetroundjoin%
\definecolor{currentfill}{rgb}{0.180629,0.429975,0.557282}%
\pgfsetfillcolor{currentfill}%
\pgfsetfillopacity{0.700000}%
\pgfsetlinewidth{0.000000pt}%
\definecolor{currentstroke}{rgb}{0.000000,0.000000,0.000000}%
\pgfsetstrokecolor{currentstroke}%
\pgfsetdash{}{0pt}%
\pgfpathmoveto{\pgfqpoint{3.094256in}{3.433435in}}%
\pgfpathlineto{\pgfqpoint{3.107540in}{3.419759in}}%
\pgfpathlineto{\pgfqpoint{3.120823in}{3.406222in}}%
\pgfpathlineto{\pgfqpoint{3.134105in}{3.392822in}}%
\pgfpathlineto{\pgfqpoint{3.147385in}{3.379559in}}%
\pgfpathlineto{\pgfqpoint{3.155206in}{3.398485in}}%
\pgfpathlineto{\pgfqpoint{3.163020in}{3.417700in}}%
\pgfpathlineto{\pgfqpoint{3.170826in}{3.437208in}}%
\pgfpathlineto{\pgfqpoint{3.178625in}{3.457017in}}%
\pgfpathlineto{\pgfqpoint{3.165346in}{3.470603in}}%
\pgfpathlineto{\pgfqpoint{3.152065in}{3.484325in}}%
\pgfpathlineto{\pgfqpoint{3.138782in}{3.498186in}}%
\pgfpathlineto{\pgfqpoint{3.125497in}{3.512186in}}%
\pgfpathlineto{\pgfqpoint{3.117698in}{3.492046in}}%
\pgfpathlineto{\pgfqpoint{3.109892in}{3.472211in}}%
\pgfpathlineto{\pgfqpoint{3.102077in}{3.452676in}}%
\pgfpathlineto{\pgfqpoint{3.094256in}{3.433435in}}%
\pgfpathclose%
\pgfusepath{fill}%
\end{pgfscope}%
\begin{pgfscope}%
\pgfpathrectangle{\pgfqpoint{1.150000in}{0.150000in}}{\pgfqpoint{5.700000in}{5.700000in}}%
\pgfusepath{clip}%
\pgfsetbuttcap%
\pgfsetroundjoin%
\definecolor{currentfill}{rgb}{0.160665,0.478540,0.558115}%
\pgfsetfillcolor{currentfill}%
\pgfsetfillopacity{0.700000}%
\pgfsetlinewidth{0.000000pt}%
\definecolor{currentstroke}{rgb}{0.000000,0.000000,0.000000}%
\pgfsetstrokecolor{currentstroke}%
\pgfsetdash{}{0pt}%
\pgfpathmoveto{\pgfqpoint{3.842190in}{3.547620in}}%
\pgfpathlineto{\pgfqpoint{3.855485in}{3.536023in}}%
\pgfpathlineto{\pgfqpoint{3.868782in}{3.524529in}}%
\pgfpathlineto{\pgfqpoint{3.882081in}{3.513135in}}%
\pgfpathlineto{\pgfqpoint{3.895381in}{3.501843in}}%
\pgfpathlineto{\pgfqpoint{3.903074in}{3.526797in}}%
\pgfpathlineto{\pgfqpoint{3.910765in}{3.552189in}}%
\pgfpathlineto{\pgfqpoint{3.918455in}{3.578028in}}%
\pgfpathlineto{\pgfqpoint{3.926143in}{3.604322in}}%
\pgfpathlineto{\pgfqpoint{3.912840in}{3.616081in}}%
\pgfpathlineto{\pgfqpoint{3.899538in}{3.627940in}}%
\pgfpathlineto{\pgfqpoint{3.886238in}{3.639902in}}%
\pgfpathlineto{\pgfqpoint{3.872939in}{3.651966in}}%
\pgfpathlineto{\pgfqpoint{3.865254in}{3.625196in}}%
\pgfpathlineto{\pgfqpoint{3.857568in}{3.598888in}}%
\pgfpathlineto{\pgfqpoint{3.849880in}{3.573032in}}%
\pgfpathlineto{\pgfqpoint{3.842190in}{3.547620in}}%
\pgfpathclose%
\pgfusepath{fill}%
\end{pgfscope}%
\begin{pgfscope}%
\pgfpathrectangle{\pgfqpoint{1.150000in}{0.150000in}}{\pgfqpoint{5.700000in}{5.700000in}}%
\pgfusepath{clip}%
\pgfsetbuttcap%
\pgfsetroundjoin%
\definecolor{currentfill}{rgb}{0.197636,0.391528,0.554969}%
\pgfsetfillcolor{currentfill}%
\pgfsetfillopacity{0.700000}%
\pgfsetlinewidth{0.000000pt}%
\definecolor{currentstroke}{rgb}{0.000000,0.000000,0.000000}%
\pgfsetstrokecolor{currentstroke}%
\pgfsetdash{}{0pt}%
\pgfpathmoveto{\pgfqpoint{3.422120in}{3.334169in}}%
\pgfpathlineto{\pgfqpoint{3.435391in}{3.322392in}}%
\pgfpathlineto{\pgfqpoint{3.448663in}{3.310733in}}%
\pgfpathlineto{\pgfqpoint{3.461934in}{3.299189in}}%
\pgfpathlineto{\pgfqpoint{3.475207in}{3.287761in}}%
\pgfpathlineto{\pgfqpoint{3.482974in}{3.307390in}}%
\pgfpathlineto{\pgfqpoint{3.490735in}{3.327329in}}%
\pgfpathlineto{\pgfqpoint{3.498491in}{3.347588in}}%
\pgfpathlineto{\pgfqpoint{3.506241in}{3.368171in}}%
\pgfpathlineto{\pgfqpoint{3.492969in}{3.379960in}}%
\pgfpathlineto{\pgfqpoint{3.479697in}{3.391864in}}%
\pgfpathlineto{\pgfqpoint{3.466425in}{3.403885in}}%
\pgfpathlineto{\pgfqpoint{3.453154in}{3.416024in}}%
\pgfpathlineto{\pgfqpoint{3.445404in}{3.395071in}}%
\pgfpathlineto{\pgfqpoint{3.437649in}{3.374449in}}%
\pgfpathlineto{\pgfqpoint{3.429887in}{3.354151in}}%
\pgfpathlineto{\pgfqpoint{3.422120in}{3.334169in}}%
\pgfpathclose%
\pgfusepath{fill}%
\end{pgfscope}%
\begin{pgfscope}%
\pgfpathrectangle{\pgfqpoint{1.150000in}{0.150000in}}{\pgfqpoint{5.700000in}{5.700000in}}%
\pgfusepath{clip}%
\pgfsetbuttcap%
\pgfsetroundjoin%
\definecolor{currentfill}{rgb}{0.194100,0.399323,0.555565}%
\pgfsetfillcolor{currentfill}%
\pgfsetfillopacity{0.700000}%
\pgfsetlinewidth{0.000000pt}%
\definecolor{currentstroke}{rgb}{0.000000,0.000000,0.000000}%
\pgfsetstrokecolor{currentstroke}%
\pgfsetdash{}{0pt}%
\pgfpathmoveto{\pgfqpoint{3.284824in}{3.353109in}}%
\pgfpathlineto{\pgfqpoint{3.298095in}{3.340700in}}%
\pgfpathlineto{\pgfqpoint{3.311367in}{3.328416in}}%
\pgfpathlineto{\pgfqpoint{3.324638in}{3.316256in}}%
\pgfpathlineto{\pgfqpoint{3.337909in}{3.304219in}}%
\pgfpathlineto{\pgfqpoint{3.345700in}{3.323331in}}%
\pgfpathlineto{\pgfqpoint{3.353486in}{3.342739in}}%
\pgfpathlineto{\pgfqpoint{3.361265in}{3.362450in}}%
\pgfpathlineto{\pgfqpoint{3.369038in}{3.382469in}}%
\pgfpathlineto{\pgfqpoint{3.355767in}{3.394847in}}%
\pgfpathlineto{\pgfqpoint{3.342497in}{3.407349in}}%
\pgfpathlineto{\pgfqpoint{3.329225in}{3.419974in}}%
\pgfpathlineto{\pgfqpoint{3.315954in}{3.432725in}}%
\pgfpathlineto{\pgfqpoint{3.308181in}{3.412356in}}%
\pgfpathlineto{\pgfqpoint{3.300402in}{3.392300in}}%
\pgfpathlineto{\pgfqpoint{3.292616in}{3.372554in}}%
\pgfpathlineto{\pgfqpoint{3.284824in}{3.353109in}}%
\pgfpathclose%
\pgfusepath{fill}%
\end{pgfscope}%
\begin{pgfscope}%
\pgfpathrectangle{\pgfqpoint{1.150000in}{0.150000in}}{\pgfqpoint{5.700000in}{5.700000in}}%
\pgfusepath{clip}%
\pgfsetbuttcap%
\pgfsetroundjoin%
\definecolor{currentfill}{rgb}{0.144759,0.519093,0.556572}%
\pgfsetfillcolor{currentfill}%
\pgfsetfillopacity{0.700000}%
\pgfsetlinewidth{0.000000pt}%
\definecolor{currentstroke}{rgb}{0.000000,0.000000,0.000000}%
\pgfsetstrokecolor{currentstroke}%
\pgfsetdash{}{0pt}%
\pgfpathmoveto{\pgfqpoint{3.872939in}{3.651966in}}%
\pgfpathlineto{\pgfqpoint{3.886238in}{3.639902in}}%
\pgfpathlineto{\pgfqpoint{3.899538in}{3.627940in}}%
\pgfpathlineto{\pgfqpoint{3.912840in}{3.616081in}}%
\pgfpathlineto{\pgfqpoint{3.926143in}{3.604322in}}%
\pgfpathlineto{\pgfqpoint{3.933829in}{3.631082in}}%
\pgfpathlineto{\pgfqpoint{3.941514in}{3.658315in}}%
\pgfpathlineto{\pgfqpoint{3.949199in}{3.686032in}}%
\pgfpathlineto{\pgfqpoint{3.956883in}{3.714242in}}%
\pgfpathlineto{\pgfqpoint{3.943575in}{3.726489in}}%
\pgfpathlineto{\pgfqpoint{3.930269in}{3.738839in}}%
\pgfpathlineto{\pgfqpoint{3.916965in}{3.751290in}}%
\pgfpathlineto{\pgfqpoint{3.903661in}{3.763844in}}%
\pgfpathlineto{\pgfqpoint{3.895982in}{3.735136in}}%
\pgfpathlineto{\pgfqpoint{3.888302in}{3.706926in}}%
\pgfpathlineto{\pgfqpoint{3.880621in}{3.679206in}}%
\pgfpathlineto{\pgfqpoint{3.872939in}{3.651966in}}%
\pgfpathclose%
\pgfusepath{fill}%
\end{pgfscope}%
\begin{pgfscope}%
\pgfpathrectangle{\pgfqpoint{1.150000in}{0.150000in}}{\pgfqpoint{5.700000in}{5.700000in}}%
\pgfusepath{clip}%
\pgfsetbuttcap%
\pgfsetroundjoin%
\definecolor{currentfill}{rgb}{0.183898,0.422383,0.556944}%
\pgfsetfillcolor{currentfill}%
\pgfsetfillopacity{0.700000}%
\pgfsetlinewidth{0.000000pt}%
\definecolor{currentstroke}{rgb}{0.000000,0.000000,0.000000}%
\pgfsetstrokecolor{currentstroke}%
\pgfsetdash{}{0pt}%
\pgfpathmoveto{\pgfqpoint{3.727417in}{3.403220in}}%
\pgfpathlineto{\pgfqpoint{3.740703in}{3.392076in}}%
\pgfpathlineto{\pgfqpoint{3.753991in}{3.381036in}}%
\pgfpathlineto{\pgfqpoint{3.767281in}{3.370100in}}%
\pgfpathlineto{\pgfqpoint{3.780572in}{3.359267in}}%
\pgfpathlineto{\pgfqpoint{3.788285in}{3.381436in}}%
\pgfpathlineto{\pgfqpoint{3.795994in}{3.403981in}}%
\pgfpathlineto{\pgfqpoint{3.803701in}{3.426911in}}%
\pgfpathlineto{\pgfqpoint{3.811404in}{3.450234in}}%
\pgfpathlineto{\pgfqpoint{3.798112in}{3.461490in}}%
\pgfpathlineto{\pgfqpoint{3.784821in}{3.472848in}}%
\pgfpathlineto{\pgfqpoint{3.771532in}{3.484311in}}%
\pgfpathlineto{\pgfqpoint{3.758244in}{3.495878in}}%
\pgfpathlineto{\pgfqpoint{3.750542in}{3.472124in}}%
\pgfpathlineto{\pgfqpoint{3.742837in}{3.448768in}}%
\pgfpathlineto{\pgfqpoint{3.735129in}{3.425803in}}%
\pgfpathlineto{\pgfqpoint{3.727417in}{3.403220in}}%
\pgfpathclose%
\pgfusepath{fill}%
\end{pgfscope}%
\begin{pgfscope}%
\pgfpathrectangle{\pgfqpoint{1.150000in}{0.150000in}}{\pgfqpoint{5.700000in}{5.700000in}}%
\pgfusepath{clip}%
\pgfsetbuttcap%
\pgfsetroundjoin%
\definecolor{currentfill}{rgb}{0.190631,0.407061,0.556089}%
\pgfsetfillcolor{currentfill}%
\pgfsetfillopacity{0.700000}%
\pgfsetlinewidth{0.000000pt}%
\definecolor{currentstroke}{rgb}{0.000000,0.000000,0.000000}%
\pgfsetstrokecolor{currentstroke}%
\pgfsetdash{}{0pt}%
\pgfpathmoveto{\pgfqpoint{3.643401in}{3.360588in}}%
\pgfpathlineto{\pgfqpoint{3.656682in}{3.349420in}}%
\pgfpathlineto{\pgfqpoint{3.669963in}{3.338359in}}%
\pgfpathlineto{\pgfqpoint{3.683247in}{3.327405in}}%
\pgfpathlineto{\pgfqpoint{3.696531in}{3.316557in}}%
\pgfpathlineto{\pgfqpoint{3.704259in}{3.337688in}}%
\pgfpathlineto{\pgfqpoint{3.711982in}{3.359170in}}%
\pgfpathlineto{\pgfqpoint{3.719701in}{3.381011in}}%
\pgfpathlineto{\pgfqpoint{3.727417in}{3.403220in}}%
\pgfpathlineto{\pgfqpoint{3.714132in}{3.414469in}}%
\pgfpathlineto{\pgfqpoint{3.700848in}{3.425825in}}%
\pgfpathlineto{\pgfqpoint{3.687566in}{3.437288in}}%
\pgfpathlineto{\pgfqpoint{3.674284in}{3.448858in}}%
\pgfpathlineto{\pgfqpoint{3.666570in}{3.426239in}}%
\pgfpathlineto{\pgfqpoint{3.658851in}{3.403993in}}%
\pgfpathlineto{\pgfqpoint{3.651128in}{3.382112in}}%
\pgfpathlineto{\pgfqpoint{3.643401in}{3.360588in}}%
\pgfpathclose%
\pgfusepath{fill}%
\end{pgfscope}%
\begin{pgfscope}%
\pgfpathrectangle{\pgfqpoint{1.150000in}{0.150000in}}{\pgfqpoint{5.700000in}{5.700000in}}%
\pgfusepath{clip}%
\pgfsetbuttcap%
\pgfsetroundjoin%
\definecolor{currentfill}{rgb}{0.196571,0.711827,0.479221}%
\pgfsetfillcolor{currentfill}%
\pgfsetfillopacity{0.700000}%
\pgfsetlinewidth{0.000000pt}%
\definecolor{currentstroke}{rgb}{0.000000,0.000000,0.000000}%
\pgfsetstrokecolor{currentstroke}%
\pgfsetdash{}{0pt}%
\pgfpathmoveto{\pgfqpoint{3.668236in}{4.168178in}}%
\pgfpathlineto{\pgfqpoint{3.681545in}{4.152870in}}%
\pgfpathlineto{\pgfqpoint{3.694854in}{4.137682in}}%
\pgfpathlineto{\pgfqpoint{3.708161in}{4.122614in}}%
\pgfpathlineto{\pgfqpoint{3.721468in}{4.107665in}}%
\pgfpathlineto{\pgfqpoint{3.729109in}{4.141153in}}%
\pgfpathlineto{\pgfqpoint{3.736748in}{4.175225in}}%
\pgfpathlineto{\pgfqpoint{3.744385in}{4.209893in}}%
\pgfpathlineto{\pgfqpoint{3.752021in}{4.245168in}}%
\pgfpathlineto{\pgfqpoint{3.738703in}{4.260666in}}%
\pgfpathlineto{\pgfqpoint{3.725384in}{4.276284in}}%
\pgfpathlineto{\pgfqpoint{3.712064in}{4.292022in}}%
\pgfpathlineto{\pgfqpoint{3.698743in}{4.307882in}}%
\pgfpathlineto{\pgfqpoint{3.691119in}{4.272047in}}%
\pgfpathlineto{\pgfqpoint{3.683493in}{4.236826in}}%
\pgfpathlineto{\pgfqpoint{3.675866in}{4.202206in}}%
\pgfpathlineto{\pgfqpoint{3.668236in}{4.168178in}}%
\pgfpathclose%
\pgfusepath{fill}%
\end{pgfscope}%
\begin{pgfscope}%
\pgfpathrectangle{\pgfqpoint{1.150000in}{0.150000in}}{\pgfqpoint{5.700000in}{5.700000in}}%
\pgfusepath{clip}%
\pgfsetbuttcap%
\pgfsetroundjoin%
\definecolor{currentfill}{rgb}{0.166383,0.690856,0.496502}%
\pgfsetfillcolor{currentfill}%
\pgfsetfillopacity{0.700000}%
\pgfsetlinewidth{0.000000pt}%
\definecolor{currentstroke}{rgb}{0.000000,0.000000,0.000000}%
\pgfsetstrokecolor{currentstroke}%
\pgfsetdash{}{0pt}%
\pgfpathmoveto{\pgfqpoint{3.721468in}{4.107665in}}%
\pgfpathlineto{\pgfqpoint{3.734775in}{4.092833in}}%
\pgfpathlineto{\pgfqpoint{3.748081in}{4.078118in}}%
\pgfpathlineto{\pgfqpoint{3.761387in}{4.063518in}}%
\pgfpathlineto{\pgfqpoint{3.774692in}{4.049034in}}%
\pgfpathlineto{\pgfqpoint{3.782343in}{4.081984in}}%
\pgfpathlineto{\pgfqpoint{3.789992in}{4.115513in}}%
\pgfpathlineto{\pgfqpoint{3.797640in}{4.149630in}}%
\pgfpathlineto{\pgfqpoint{3.805288in}{4.184347in}}%
\pgfpathlineto{\pgfqpoint{3.791972in}{4.199378in}}%
\pgfpathlineto{\pgfqpoint{3.778656in}{4.214525in}}%
\pgfpathlineto{\pgfqpoint{3.765339in}{4.229788in}}%
\pgfpathlineto{\pgfqpoint{3.752021in}{4.245168in}}%
\pgfpathlineto{\pgfqpoint{3.744385in}{4.209893in}}%
\pgfpathlineto{\pgfqpoint{3.736748in}{4.175225in}}%
\pgfpathlineto{\pgfqpoint{3.729109in}{4.141153in}}%
\pgfpathlineto{\pgfqpoint{3.721468in}{4.107665in}}%
\pgfpathclose%
\pgfusepath{fill}%
\end{pgfscope}%
\begin{pgfscope}%
\pgfpathrectangle{\pgfqpoint{1.150000in}{0.150000in}}{\pgfqpoint{5.700000in}{5.700000in}}%
\pgfusepath{clip}%
\pgfsetbuttcap%
\pgfsetroundjoin%
\definecolor{currentfill}{rgb}{0.232815,0.732247,0.459277}%
\pgfsetfillcolor{currentfill}%
\pgfsetfillopacity{0.700000}%
\pgfsetlinewidth{0.000000pt}%
\definecolor{currentstroke}{rgb}{0.000000,0.000000,0.000000}%
\pgfsetstrokecolor{currentstroke}%
\pgfsetdash{}{0pt}%
\pgfpathmoveto{\pgfqpoint{3.614989in}{4.230638in}}%
\pgfpathlineto{\pgfqpoint{3.628303in}{4.214837in}}%
\pgfpathlineto{\pgfqpoint{3.641615in}{4.199160in}}%
\pgfpathlineto{\pgfqpoint{3.654926in}{4.183608in}}%
\pgfpathlineto{\pgfqpoint{3.668236in}{4.168178in}}%
\pgfpathlineto{\pgfqpoint{3.675866in}{4.202206in}}%
\pgfpathlineto{\pgfqpoint{3.683493in}{4.236826in}}%
\pgfpathlineto{\pgfqpoint{3.691119in}{4.272047in}}%
\pgfpathlineto{\pgfqpoint{3.698743in}{4.307882in}}%
\pgfpathlineto{\pgfqpoint{3.685421in}{4.323863in}}%
\pgfpathlineto{\pgfqpoint{3.672098in}{4.339968in}}%
\pgfpathlineto{\pgfqpoint{3.658773in}{4.356198in}}%
\pgfpathlineto{\pgfqpoint{3.645447in}{4.372553in}}%
\pgfpathlineto{\pgfqpoint{3.637836in}{4.336155in}}%
\pgfpathlineto{\pgfqpoint{3.630223in}{4.300378in}}%
\pgfpathlineto{\pgfqpoint{3.622607in}{4.265209in}}%
\pgfpathlineto{\pgfqpoint{3.614989in}{4.230638in}}%
\pgfpathclose%
\pgfusepath{fill}%
\end{pgfscope}%
\begin{pgfscope}%
\pgfpathrectangle{\pgfqpoint{1.150000in}{0.150000in}}{\pgfqpoint{5.700000in}{5.700000in}}%
\pgfusepath{clip}%
\pgfsetbuttcap%
\pgfsetroundjoin%
\definecolor{currentfill}{rgb}{0.175841,0.441290,0.557685}%
\pgfsetfillcolor{currentfill}%
\pgfsetfillopacity{0.700000}%
\pgfsetlinewidth{0.000000pt}%
\definecolor{currentstroke}{rgb}{0.000000,0.000000,0.000000}%
\pgfsetstrokecolor{currentstroke}%
\pgfsetdash{}{0pt}%
\pgfpathmoveto{\pgfqpoint{3.811404in}{3.450234in}}%
\pgfpathlineto{\pgfqpoint{3.824697in}{3.439082in}}%
\pgfpathlineto{\pgfqpoint{3.837993in}{3.428031in}}%
\pgfpathlineto{\pgfqpoint{3.851290in}{3.417082in}}%
\pgfpathlineto{\pgfqpoint{3.864588in}{3.406233in}}%
\pgfpathlineto{\pgfqpoint{3.872290in}{3.429522in}}%
\pgfpathlineto{\pgfqpoint{3.879990in}{3.453214in}}%
\pgfpathlineto{\pgfqpoint{3.887687in}{3.477318in}}%
\pgfpathlineto{\pgfqpoint{3.895381in}{3.501843in}}%
\pgfpathlineto{\pgfqpoint{3.882081in}{3.513135in}}%
\pgfpathlineto{\pgfqpoint{3.868782in}{3.524529in}}%
\pgfpathlineto{\pgfqpoint{3.855485in}{3.536023in}}%
\pgfpathlineto{\pgfqpoint{3.842190in}{3.547620in}}%
\pgfpathlineto{\pgfqpoint{3.834497in}{3.522643in}}%
\pgfpathlineto{\pgfqpoint{3.826802in}{3.498092in}}%
\pgfpathlineto{\pgfqpoint{3.819104in}{3.473958in}}%
\pgfpathlineto{\pgfqpoint{3.811404in}{3.450234in}}%
\pgfpathclose%
\pgfusepath{fill}%
\end{pgfscope}%
\begin{pgfscope}%
\pgfpathrectangle{\pgfqpoint{1.150000in}{0.150000in}}{\pgfqpoint{5.700000in}{5.700000in}}%
\pgfusepath{clip}%
\pgfsetbuttcap%
\pgfsetroundjoin%
\definecolor{currentfill}{rgb}{0.197636,0.391528,0.554969}%
\pgfsetfillcolor{currentfill}%
\pgfsetfillopacity{0.700000}%
\pgfsetlinewidth{0.000000pt}%
\definecolor{currentstroke}{rgb}{0.000000,0.000000,0.000000}%
\pgfsetstrokecolor{currentstroke}%
\pgfsetdash{}{0pt}%
\pgfpathmoveto{\pgfqpoint{3.559337in}{3.322154in}}%
\pgfpathlineto{\pgfqpoint{3.572613in}{3.310929in}}%
\pgfpathlineto{\pgfqpoint{3.585890in}{3.299816in}}%
\pgfpathlineto{\pgfqpoint{3.599169in}{3.288812in}}%
\pgfpathlineto{\pgfqpoint{3.612448in}{3.277916in}}%
\pgfpathlineto{\pgfqpoint{3.620193in}{3.298085in}}%
\pgfpathlineto{\pgfqpoint{3.627934in}{3.318581in}}%
\pgfpathlineto{\pgfqpoint{3.635670in}{3.339413in}}%
\pgfpathlineto{\pgfqpoint{3.643401in}{3.360588in}}%
\pgfpathlineto{\pgfqpoint{3.630122in}{3.371864in}}%
\pgfpathlineto{\pgfqpoint{3.616843in}{3.383249in}}%
\pgfpathlineto{\pgfqpoint{3.603566in}{3.394744in}}%
\pgfpathlineto{\pgfqpoint{3.590290in}{3.406349in}}%
\pgfpathlineto{\pgfqpoint{3.582559in}{3.384786in}}%
\pgfpathlineto{\pgfqpoint{3.574823in}{3.363570in}}%
\pgfpathlineto{\pgfqpoint{3.567082in}{3.342695in}}%
\pgfpathlineto{\pgfqpoint{3.559337in}{3.322154in}}%
\pgfpathclose%
\pgfusepath{fill}%
\end{pgfscope}%
\begin{pgfscope}%
\pgfpathrectangle{\pgfqpoint{1.150000in}{0.150000in}}{\pgfqpoint{5.700000in}{5.700000in}}%
\pgfusepath{clip}%
\pgfsetbuttcap%
\pgfsetroundjoin%
\definecolor{currentfill}{rgb}{0.143303,0.669459,0.511215}%
\pgfsetfillcolor{currentfill}%
\pgfsetfillopacity{0.700000}%
\pgfsetlinewidth{0.000000pt}%
\definecolor{currentstroke}{rgb}{0.000000,0.000000,0.000000}%
\pgfsetstrokecolor{currentstroke}%
\pgfsetdash{}{0pt}%
\pgfpathmoveto{\pgfqpoint{3.774692in}{4.049034in}}%
\pgfpathlineto{\pgfqpoint{3.787997in}{4.034663in}}%
\pgfpathlineto{\pgfqpoint{3.801302in}{4.020406in}}%
\pgfpathlineto{\pgfqpoint{3.814608in}{4.006261in}}%
\pgfpathlineto{\pgfqpoint{3.827913in}{3.992227in}}%
\pgfpathlineto{\pgfqpoint{3.835573in}{4.024642in}}%
\pgfpathlineto{\pgfqpoint{3.843232in}{4.057629in}}%
\pgfpathlineto{\pgfqpoint{3.850890in}{4.091198in}}%
\pgfpathlineto{\pgfqpoint{3.858548in}{4.125360in}}%
\pgfpathlineto{\pgfqpoint{3.845233in}{4.139938in}}%
\pgfpathlineto{\pgfqpoint{3.831918in}{4.154628in}}%
\pgfpathlineto{\pgfqpoint{3.818603in}{4.169431in}}%
\pgfpathlineto{\pgfqpoint{3.805288in}{4.184347in}}%
\pgfpathlineto{\pgfqpoint{3.797640in}{4.149630in}}%
\pgfpathlineto{\pgfqpoint{3.789992in}{4.115513in}}%
\pgfpathlineto{\pgfqpoint{3.782343in}{4.081984in}}%
\pgfpathlineto{\pgfqpoint{3.774692in}{4.049034in}}%
\pgfpathclose%
\pgfusepath{fill}%
\end{pgfscope}%
\begin{pgfscope}%
\pgfpathrectangle{\pgfqpoint{1.150000in}{0.150000in}}{\pgfqpoint{5.700000in}{5.700000in}}%
\pgfusepath{clip}%
\pgfsetbuttcap%
\pgfsetroundjoin%
\definecolor{currentfill}{rgb}{0.281477,0.755203,0.432552}%
\pgfsetfillcolor{currentfill}%
\pgfsetfillopacity{0.700000}%
\pgfsetlinewidth{0.000000pt}%
\definecolor{currentstroke}{rgb}{0.000000,0.000000,0.000000}%
\pgfsetstrokecolor{currentstroke}%
\pgfsetdash{}{0pt}%
\pgfpathmoveto{\pgfqpoint{3.561721in}{4.295111in}}%
\pgfpathlineto{\pgfqpoint{3.575041in}{4.278800in}}%
\pgfpathlineto{\pgfqpoint{3.588358in}{4.262618in}}%
\pgfpathlineto{\pgfqpoint{3.601674in}{4.246564in}}%
\pgfpathlineto{\pgfqpoint{3.614989in}{4.230638in}}%
\pgfpathlineto{\pgfqpoint{3.622607in}{4.265209in}}%
\pgfpathlineto{\pgfqpoint{3.630223in}{4.300378in}}%
\pgfpathlineto{\pgfqpoint{3.637836in}{4.336155in}}%
\pgfpathlineto{\pgfqpoint{3.645447in}{4.372553in}}%
\pgfpathlineto{\pgfqpoint{3.632120in}{4.389034in}}%
\pgfpathlineto{\pgfqpoint{3.618791in}{4.405644in}}%
\pgfpathlineto{\pgfqpoint{3.605460in}{4.422382in}}%
\pgfpathlineto{\pgfqpoint{3.592127in}{4.439250in}}%
\pgfpathlineto{\pgfqpoint{3.584530in}{4.402286in}}%
\pgfpathlineto{\pgfqpoint{3.576930in}{4.365949in}}%
\pgfpathlineto{\pgfqpoint{3.569327in}{4.330228in}}%
\pgfpathlineto{\pgfqpoint{3.561721in}{4.295111in}}%
\pgfpathclose%
\pgfusepath{fill}%
\end{pgfscope}%
\begin{pgfscope}%
\pgfpathrectangle{\pgfqpoint{1.150000in}{0.150000in}}{\pgfqpoint{5.700000in}{5.700000in}}%
\pgfusepath{clip}%
\pgfsetbuttcap%
\pgfsetroundjoin%
\definecolor{currentfill}{rgb}{0.188923,0.410910,0.556326}%
\pgfsetfillcolor{currentfill}%
\pgfsetfillopacity{0.700000}%
\pgfsetlinewidth{0.000000pt}%
\definecolor{currentstroke}{rgb}{0.000000,0.000000,0.000000}%
\pgfsetstrokecolor{currentstroke}%
\pgfsetdash{}{0pt}%
\pgfpathmoveto{\pgfqpoint{3.147385in}{3.379559in}}%
\pgfpathlineto{\pgfqpoint{3.160663in}{3.366431in}}%
\pgfpathlineto{\pgfqpoint{3.173941in}{3.353437in}}%
\pgfpathlineto{\pgfqpoint{3.187218in}{3.340576in}}%
\pgfpathlineto{\pgfqpoint{3.200493in}{3.327847in}}%
\pgfpathlineto{\pgfqpoint{3.208313in}{3.346459in}}%
\pgfpathlineto{\pgfqpoint{3.216126in}{3.365355in}}%
\pgfpathlineto{\pgfqpoint{3.223933in}{3.384538in}}%
\pgfpathlineto{\pgfqpoint{3.231732in}{3.404017in}}%
\pgfpathlineto{\pgfqpoint{3.218457in}{3.417068in}}%
\pgfpathlineto{\pgfqpoint{3.205181in}{3.430251in}}%
\pgfpathlineto{\pgfqpoint{3.191904in}{3.443567in}}%
\pgfpathlineto{\pgfqpoint{3.178625in}{3.457017in}}%
\pgfpathlineto{\pgfqpoint{3.170826in}{3.437208in}}%
\pgfpathlineto{\pgfqpoint{3.163020in}{3.417700in}}%
\pgfpathlineto{\pgfqpoint{3.155206in}{3.398485in}}%
\pgfpathlineto{\pgfqpoint{3.147385in}{3.379559in}}%
\pgfpathclose%
\pgfusepath{fill}%
\end{pgfscope}%
\begin{pgfscope}%
\pgfpathrectangle{\pgfqpoint{1.150000in}{0.150000in}}{\pgfqpoint{5.700000in}{5.700000in}}%
\pgfusepath{clip}%
\pgfsetbuttcap%
\pgfsetroundjoin%
\definecolor{currentfill}{rgb}{0.128729,0.563265,0.551229}%
\pgfsetfillcolor{currentfill}%
\pgfsetfillopacity{0.700000}%
\pgfsetlinewidth{0.000000pt}%
\definecolor{currentstroke}{rgb}{0.000000,0.000000,0.000000}%
\pgfsetstrokecolor{currentstroke}%
\pgfsetdash{}{0pt}%
\pgfpathmoveto{\pgfqpoint{3.903661in}{3.763844in}}%
\pgfpathlineto{\pgfqpoint{3.916965in}{3.751290in}}%
\pgfpathlineto{\pgfqpoint{3.930269in}{3.738839in}}%
\pgfpathlineto{\pgfqpoint{3.943575in}{3.726489in}}%
\pgfpathlineto{\pgfqpoint{3.956883in}{3.714242in}}%
\pgfpathlineto{\pgfqpoint{3.964566in}{3.742954in}}%
\pgfpathlineto{\pgfqpoint{3.972248in}{3.772179in}}%
\pgfpathlineto{\pgfqpoint{3.979931in}{3.801926in}}%
\pgfpathlineto{\pgfqpoint{3.987613in}{3.832206in}}%
\pgfpathlineto{\pgfqpoint{3.974300in}{3.844966in}}%
\pgfpathlineto{\pgfqpoint{3.960989in}{3.857829in}}%
\pgfpathlineto{\pgfqpoint{3.947678in}{3.870794in}}%
\pgfpathlineto{\pgfqpoint{3.934368in}{3.883863in}}%
\pgfpathlineto{\pgfqpoint{3.926692in}{3.853061in}}%
\pgfpathlineto{\pgfqpoint{3.919016in}{3.822797in}}%
\pgfpathlineto{\pgfqpoint{3.911339in}{3.793061in}}%
\pgfpathlineto{\pgfqpoint{3.903661in}{3.763844in}}%
\pgfpathclose%
\pgfusepath{fill}%
\end{pgfscope}%
\begin{pgfscope}%
\pgfpathrectangle{\pgfqpoint{1.150000in}{0.150000in}}{\pgfqpoint{5.700000in}{5.700000in}}%
\pgfusepath{clip}%
\pgfsetbuttcap%
\pgfsetroundjoin%
\definecolor{currentfill}{rgb}{0.128087,0.647749,0.523491}%
\pgfsetfillcolor{currentfill}%
\pgfsetfillopacity{0.700000}%
\pgfsetlinewidth{0.000000pt}%
\definecolor{currentstroke}{rgb}{0.000000,0.000000,0.000000}%
\pgfsetstrokecolor{currentstroke}%
\pgfsetdash{}{0pt}%
\pgfpathmoveto{\pgfqpoint{3.827913in}{3.992227in}}%
\pgfpathlineto{\pgfqpoint{3.841218in}{3.978303in}}%
\pgfpathlineto{\pgfqpoint{3.854524in}{3.964489in}}%
\pgfpathlineto{\pgfqpoint{3.867830in}{3.950784in}}%
\pgfpathlineto{\pgfqpoint{3.881136in}{3.937187in}}%
\pgfpathlineto{\pgfqpoint{3.888805in}{3.969069in}}%
\pgfpathlineto{\pgfqpoint{3.896472in}{4.001517in}}%
\pgfpathlineto{\pgfqpoint{3.904140in}{4.034540in}}%
\pgfpathlineto{\pgfqpoint{3.911808in}{4.068150in}}%
\pgfpathlineto{\pgfqpoint{3.898493in}{4.082289in}}%
\pgfpathlineto{\pgfqpoint{3.885178in}{4.096537in}}%
\pgfpathlineto{\pgfqpoint{3.871863in}{4.110893in}}%
\pgfpathlineto{\pgfqpoint{3.858548in}{4.125360in}}%
\pgfpathlineto{\pgfqpoint{3.850890in}{4.091198in}}%
\pgfpathlineto{\pgfqpoint{3.843232in}{4.057629in}}%
\pgfpathlineto{\pgfqpoint{3.835573in}{4.024642in}}%
\pgfpathlineto{\pgfqpoint{3.827913in}{3.992227in}}%
\pgfpathclose%
\pgfusepath{fill}%
\end{pgfscope}%
\begin{pgfscope}%
\pgfpathrectangle{\pgfqpoint{1.150000in}{0.150000in}}{\pgfqpoint{5.700000in}{5.700000in}}%
\pgfusepath{clip}%
\pgfsetbuttcap%
\pgfsetroundjoin%
\definecolor{currentfill}{rgb}{0.335885,0.777018,0.402049}%
\pgfsetfillcolor{currentfill}%
\pgfsetfillopacity{0.700000}%
\pgfsetlinewidth{0.000000pt}%
\definecolor{currentstroke}{rgb}{0.000000,0.000000,0.000000}%
\pgfsetstrokecolor{currentstroke}%
\pgfsetdash{}{0pt}%
\pgfpathmoveto{\pgfqpoint{3.508426in}{4.361670in}}%
\pgfpathlineto{\pgfqpoint{3.521753in}{4.344830in}}%
\pgfpathlineto{\pgfqpoint{3.535078in}{4.328125in}}%
\pgfpathlineto{\pgfqpoint{3.548400in}{4.311552in}}%
\pgfpathlineto{\pgfqpoint{3.561721in}{4.295111in}}%
\pgfpathlineto{\pgfqpoint{3.569327in}{4.330228in}}%
\pgfpathlineto{\pgfqpoint{3.576930in}{4.365949in}}%
\pgfpathlineto{\pgfqpoint{3.584530in}{4.402286in}}%
\pgfpathlineto{\pgfqpoint{3.592127in}{4.439250in}}%
\pgfpathlineto{\pgfqpoint{3.578793in}{4.456249in}}%
\pgfpathlineto{\pgfqpoint{3.565456in}{4.473380in}}%
\pgfpathlineto{\pgfqpoint{3.552118in}{4.490645in}}%
\pgfpathlineto{\pgfqpoint{3.538777in}{4.508045in}}%
\pgfpathlineto{\pgfqpoint{3.531194in}{4.470512in}}%
\pgfpathlineto{\pgfqpoint{3.523608in}{4.433613in}}%
\pgfpathlineto{\pgfqpoint{3.516019in}{4.397336in}}%
\pgfpathlineto{\pgfqpoint{3.508426in}{4.361670in}}%
\pgfpathclose%
\pgfusepath{fill}%
\end{pgfscope}%
\begin{pgfscope}%
\pgfpathrectangle{\pgfqpoint{1.150000in}{0.150000in}}{\pgfqpoint{5.700000in}{5.700000in}}%
\pgfusepath{clip}%
\pgfsetbuttcap%
\pgfsetroundjoin%
\definecolor{currentfill}{rgb}{0.201239,0.383670,0.554294}%
\pgfsetfillcolor{currentfill}%
\pgfsetfillopacity{0.700000}%
\pgfsetlinewidth{0.000000pt}%
\definecolor{currentstroke}{rgb}{0.000000,0.000000,0.000000}%
\pgfsetstrokecolor{currentstroke}%
\pgfsetdash{}{0pt}%
\pgfpathmoveto{\pgfqpoint{3.337909in}{3.304219in}}%
\pgfpathlineto{\pgfqpoint{3.351179in}{3.292305in}}%
\pgfpathlineto{\pgfqpoint{3.364450in}{3.280511in}}%
\pgfpathlineto{\pgfqpoint{3.377722in}{3.268837in}}%
\pgfpathlineto{\pgfqpoint{3.390993in}{3.257282in}}%
\pgfpathlineto{\pgfqpoint{3.398784in}{3.276061in}}%
\pgfpathlineto{\pgfqpoint{3.406569in}{3.295131in}}%
\pgfpathlineto{\pgfqpoint{3.414347in}{3.314498in}}%
\pgfpathlineto{\pgfqpoint{3.422120in}{3.334169in}}%
\pgfpathlineto{\pgfqpoint{3.408849in}{3.346065in}}%
\pgfpathlineto{\pgfqpoint{3.395579in}{3.358079in}}%
\pgfpathlineto{\pgfqpoint{3.382308in}{3.370214in}}%
\pgfpathlineto{\pgfqpoint{3.369038in}{3.382469in}}%
\pgfpathlineto{\pgfqpoint{3.361265in}{3.362450in}}%
\pgfpathlineto{\pgfqpoint{3.353486in}{3.342739in}}%
\pgfpathlineto{\pgfqpoint{3.345700in}{3.323331in}}%
\pgfpathlineto{\pgfqpoint{3.337909in}{3.304219in}}%
\pgfpathclose%
\pgfusepath{fill}%
\end{pgfscope}%
\begin{pgfscope}%
\pgfpathrectangle{\pgfqpoint{1.150000in}{0.150000in}}{\pgfqpoint{5.700000in}{5.700000in}}%
\pgfusepath{clip}%
\pgfsetbuttcap%
\pgfsetroundjoin%
\definecolor{currentfill}{rgb}{0.150476,0.504369,0.557430}%
\pgfsetfillcolor{currentfill}%
\pgfsetfillopacity{0.700000}%
\pgfsetlinewidth{0.000000pt}%
\definecolor{currentstroke}{rgb}{0.000000,0.000000,0.000000}%
\pgfsetstrokecolor{currentstroke}%
\pgfsetdash{}{0pt}%
\pgfpathmoveto{\pgfqpoint{3.926143in}{3.604322in}}%
\pgfpathlineto{\pgfqpoint{3.939447in}{3.592664in}}%
\pgfpathlineto{\pgfqpoint{3.952753in}{3.581106in}}%
\pgfpathlineto{\pgfqpoint{3.966061in}{3.569647in}}%
\pgfpathlineto{\pgfqpoint{3.979371in}{3.558287in}}%
\pgfpathlineto{\pgfqpoint{3.987061in}{3.584566in}}%
\pgfpathlineto{\pgfqpoint{3.994749in}{3.611314in}}%
\pgfpathlineto{\pgfqpoint{4.002438in}{3.638540in}}%
\pgfpathlineto{\pgfqpoint{4.010125in}{3.666253in}}%
\pgfpathlineto{\pgfqpoint{3.996812in}{3.678101in}}%
\pgfpathlineto{\pgfqpoint{3.983501in}{3.690048in}}%
\pgfpathlineto{\pgfqpoint{3.970191in}{3.702095in}}%
\pgfpathlineto{\pgfqpoint{3.956883in}{3.714242in}}%
\pgfpathlineto{\pgfqpoint{3.949199in}{3.686032in}}%
\pgfpathlineto{\pgfqpoint{3.941514in}{3.658315in}}%
\pgfpathlineto{\pgfqpoint{3.933829in}{3.631082in}}%
\pgfpathlineto{\pgfqpoint{3.926143in}{3.604322in}}%
\pgfpathclose%
\pgfusepath{fill}%
\end{pgfscope}%
\begin{pgfscope}%
\pgfpathrectangle{\pgfqpoint{1.150000in}{0.150000in}}{\pgfqpoint{5.700000in}{5.700000in}}%
\pgfusepath{clip}%
\pgfsetbuttcap%
\pgfsetroundjoin%
\definecolor{currentfill}{rgb}{0.204903,0.375746,0.553533}%
\pgfsetfillcolor{currentfill}%
\pgfsetfillopacity{0.700000}%
\pgfsetlinewidth{0.000000pt}%
\definecolor{currentstroke}{rgb}{0.000000,0.000000,0.000000}%
\pgfsetstrokecolor{currentstroke}%
\pgfsetdash{}{0pt}%
\pgfpathmoveto{\pgfqpoint{3.475207in}{3.287761in}}%
\pgfpathlineto{\pgfqpoint{3.488480in}{3.276448in}}%
\pgfpathlineto{\pgfqpoint{3.501754in}{3.265248in}}%
\pgfpathlineto{\pgfqpoint{3.515029in}{3.254161in}}%
\pgfpathlineto{\pgfqpoint{3.528305in}{3.243186in}}%
\pgfpathlineto{\pgfqpoint{3.536070in}{3.262461in}}%
\pgfpathlineto{\pgfqpoint{3.543831in}{3.282044in}}%
\pgfpathlineto{\pgfqpoint{3.551587in}{3.301939in}}%
\pgfpathlineto{\pgfqpoint{3.559337in}{3.322154in}}%
\pgfpathlineto{\pgfqpoint{3.546062in}{3.333489in}}%
\pgfpathlineto{\pgfqpoint{3.532788in}{3.344936in}}%
\pgfpathlineto{\pgfqpoint{3.519514in}{3.356497in}}%
\pgfpathlineto{\pgfqpoint{3.506241in}{3.368171in}}%
\pgfpathlineto{\pgfqpoint{3.498491in}{3.347588in}}%
\pgfpathlineto{\pgfqpoint{3.490735in}{3.327329in}}%
\pgfpathlineto{\pgfqpoint{3.482974in}{3.307390in}}%
\pgfpathlineto{\pgfqpoint{3.475207in}{3.287761in}}%
\pgfpathclose%
\pgfusepath{fill}%
\end{pgfscope}%
\begin{pgfscope}%
\pgfpathrectangle{\pgfqpoint{1.150000in}{0.150000in}}{\pgfqpoint{5.700000in}{5.700000in}}%
\pgfusepath{clip}%
\pgfsetbuttcap%
\pgfsetroundjoin%
\definecolor{currentfill}{rgb}{0.166617,0.463708,0.558119}%
\pgfsetfillcolor{currentfill}%
\pgfsetfillopacity{0.700000}%
\pgfsetlinewidth{0.000000pt}%
\definecolor{currentstroke}{rgb}{0.000000,0.000000,0.000000}%
\pgfsetstrokecolor{currentstroke}%
\pgfsetdash{}{0pt}%
\pgfpathmoveto{\pgfqpoint{3.895381in}{3.501843in}}%
\pgfpathlineto{\pgfqpoint{3.908683in}{3.490651in}}%
\pgfpathlineto{\pgfqpoint{3.921987in}{3.479558in}}%
\pgfpathlineto{\pgfqpoint{3.935293in}{3.468564in}}%
\pgfpathlineto{\pgfqpoint{3.948600in}{3.457669in}}%
\pgfpathlineto{\pgfqpoint{3.956295in}{3.482166in}}%
\pgfpathlineto{\pgfqpoint{3.963988in}{3.507096in}}%
\pgfpathlineto{\pgfqpoint{3.971680in}{3.532466in}}%
\pgfpathlineto{\pgfqpoint{3.979371in}{3.558287in}}%
\pgfpathlineto{\pgfqpoint{3.966061in}{3.569647in}}%
\pgfpathlineto{\pgfqpoint{3.952753in}{3.581106in}}%
\pgfpathlineto{\pgfqpoint{3.939447in}{3.592664in}}%
\pgfpathlineto{\pgfqpoint{3.926143in}{3.604322in}}%
\pgfpathlineto{\pgfqpoint{3.918455in}{3.578028in}}%
\pgfpathlineto{\pgfqpoint{3.910765in}{3.552189in}}%
\pgfpathlineto{\pgfqpoint{3.903074in}{3.526797in}}%
\pgfpathlineto{\pgfqpoint{3.895381in}{3.501843in}}%
\pgfpathclose%
\pgfusepath{fill}%
\end{pgfscope}%
\begin{pgfscope}%
\pgfpathrectangle{\pgfqpoint{1.150000in}{0.150000in}}{\pgfqpoint{5.700000in}{5.700000in}}%
\pgfusepath{clip}%
\pgfsetbuttcap%
\pgfsetroundjoin%
\definecolor{currentfill}{rgb}{0.121380,0.629492,0.531973}%
\pgfsetfillcolor{currentfill}%
\pgfsetfillopacity{0.700000}%
\pgfsetlinewidth{0.000000pt}%
\definecolor{currentstroke}{rgb}{0.000000,0.000000,0.000000}%
\pgfsetstrokecolor{currentstroke}%
\pgfsetdash{}{0pt}%
\pgfpathmoveto{\pgfqpoint{3.881136in}{3.937187in}}%
\pgfpathlineto{\pgfqpoint{3.894443in}{3.923697in}}%
\pgfpathlineto{\pgfqpoint{3.907751in}{3.910314in}}%
\pgfpathlineto{\pgfqpoint{3.921059in}{3.897036in}}%
\pgfpathlineto{\pgfqpoint{3.934368in}{3.883863in}}%
\pgfpathlineto{\pgfqpoint{3.942044in}{3.915214in}}%
\pgfpathlineto{\pgfqpoint{3.949720in}{3.947124in}}%
\pgfpathlineto{\pgfqpoint{3.957396in}{3.979604in}}%
\pgfpathlineto{\pgfqpoint{3.965073in}{4.012665in}}%
\pgfpathlineto{\pgfqpoint{3.951756in}{4.026377in}}%
\pgfpathlineto{\pgfqpoint{3.938439in}{4.040195in}}%
\pgfpathlineto{\pgfqpoint{3.925123in}{4.054119in}}%
\pgfpathlineto{\pgfqpoint{3.911808in}{4.068150in}}%
\pgfpathlineto{\pgfqpoint{3.904140in}{4.034540in}}%
\pgfpathlineto{\pgfqpoint{3.896472in}{4.001517in}}%
\pgfpathlineto{\pgfqpoint{3.888805in}{3.969069in}}%
\pgfpathlineto{\pgfqpoint{3.881136in}{3.937187in}}%
\pgfpathclose%
\pgfusepath{fill}%
\end{pgfscope}%
\begin{pgfscope}%
\pgfpathrectangle{\pgfqpoint{1.150000in}{0.150000in}}{\pgfqpoint{5.700000in}{5.700000in}}%
\pgfusepath{clip}%
\pgfsetbuttcap%
\pgfsetroundjoin%
\definecolor{currentfill}{rgb}{0.197636,0.391528,0.554969}%
\pgfsetfillcolor{currentfill}%
\pgfsetfillopacity{0.700000}%
\pgfsetlinewidth{0.000000pt}%
\definecolor{currentstroke}{rgb}{0.000000,0.000000,0.000000}%
\pgfsetstrokecolor{currentstroke}%
\pgfsetdash{}{0pt}%
\pgfpathmoveto{\pgfqpoint{3.200493in}{3.327847in}}%
\pgfpathlineto{\pgfqpoint{3.213768in}{3.315248in}}%
\pgfpathlineto{\pgfqpoint{3.227042in}{3.302779in}}%
\pgfpathlineto{\pgfqpoint{3.240316in}{3.290437in}}%
\pgfpathlineto{\pgfqpoint{3.253589in}{3.278223in}}%
\pgfpathlineto{\pgfqpoint{3.261407in}{3.296522in}}%
\pgfpathlineto{\pgfqpoint{3.269220in}{3.315099in}}%
\pgfpathlineto{\pgfqpoint{3.277025in}{3.333959in}}%
\pgfpathlineto{\pgfqpoint{3.284824in}{3.353109in}}%
\pgfpathlineto{\pgfqpoint{3.271552in}{3.365644in}}%
\pgfpathlineto{\pgfqpoint{3.258279in}{3.378306in}}%
\pgfpathlineto{\pgfqpoint{3.245006in}{3.391097in}}%
\pgfpathlineto{\pgfqpoint{3.231732in}{3.404017in}}%
\pgfpathlineto{\pgfqpoint{3.223933in}{3.384538in}}%
\pgfpathlineto{\pgfqpoint{3.216126in}{3.365355in}}%
\pgfpathlineto{\pgfqpoint{3.208313in}{3.346459in}}%
\pgfpathlineto{\pgfqpoint{3.200493in}{3.327847in}}%
\pgfpathclose%
\pgfusepath{fill}%
\end{pgfscope}%
\begin{pgfscope}%
\pgfpathrectangle{\pgfqpoint{1.150000in}{0.150000in}}{\pgfqpoint{5.700000in}{5.700000in}}%
\pgfusepath{clip}%
\pgfsetbuttcap%
\pgfsetroundjoin%
\definecolor{currentfill}{rgb}{0.190631,0.407061,0.556089}%
\pgfsetfillcolor{currentfill}%
\pgfsetfillopacity{0.700000}%
\pgfsetlinewidth{0.000000pt}%
\definecolor{currentstroke}{rgb}{0.000000,0.000000,0.000000}%
\pgfsetstrokecolor{currentstroke}%
\pgfsetdash{}{0pt}%
\pgfpathmoveto{\pgfqpoint{3.780572in}{3.359267in}}%
\pgfpathlineto{\pgfqpoint{3.793865in}{3.348537in}}%
\pgfpathlineto{\pgfqpoint{3.807159in}{3.337908in}}%
\pgfpathlineto{\pgfqpoint{3.820456in}{3.327380in}}%
\pgfpathlineto{\pgfqpoint{3.833754in}{3.316953in}}%
\pgfpathlineto{\pgfqpoint{3.841467in}{3.338708in}}%
\pgfpathlineto{\pgfqpoint{3.849177in}{3.360834in}}%
\pgfpathlineto{\pgfqpoint{3.856884in}{3.383340in}}%
\pgfpathlineto{\pgfqpoint{3.864588in}{3.406233in}}%
\pgfpathlineto{\pgfqpoint{3.851290in}{3.417082in}}%
\pgfpathlineto{\pgfqpoint{3.837993in}{3.428031in}}%
\pgfpathlineto{\pgfqpoint{3.824697in}{3.439082in}}%
\pgfpathlineto{\pgfqpoint{3.811404in}{3.450234in}}%
\pgfpathlineto{\pgfqpoint{3.803701in}{3.426911in}}%
\pgfpathlineto{\pgfqpoint{3.795994in}{3.403981in}}%
\pgfpathlineto{\pgfqpoint{3.788285in}{3.381436in}}%
\pgfpathlineto{\pgfqpoint{3.780572in}{3.359267in}}%
\pgfpathclose%
\pgfusepath{fill}%
\end{pgfscope}%
\begin{pgfscope}%
\pgfpathrectangle{\pgfqpoint{1.150000in}{0.150000in}}{\pgfqpoint{5.700000in}{5.700000in}}%
\pgfusepath{clip}%
\pgfsetbuttcap%
\pgfsetroundjoin%
\definecolor{currentfill}{rgb}{0.395174,0.797475,0.367757}%
\pgfsetfillcolor{currentfill}%
\pgfsetfillopacity{0.700000}%
\pgfsetlinewidth{0.000000pt}%
\definecolor{currentstroke}{rgb}{0.000000,0.000000,0.000000}%
\pgfsetstrokecolor{currentstroke}%
\pgfsetdash{}{0pt}%
\pgfpathmoveto{\pgfqpoint{3.455096in}{4.430392in}}%
\pgfpathlineto{\pgfqpoint{3.468432in}{4.413004in}}%
\pgfpathlineto{\pgfqpoint{3.481766in}{4.395756in}}%
\pgfpathlineto{\pgfqpoint{3.495097in}{4.378644in}}%
\pgfpathlineto{\pgfqpoint{3.508426in}{4.361670in}}%
\pgfpathlineto{\pgfqpoint{3.516019in}{4.397336in}}%
\pgfpathlineto{\pgfqpoint{3.523608in}{4.433613in}}%
\pgfpathlineto{\pgfqpoint{3.531194in}{4.470512in}}%
\pgfpathlineto{\pgfqpoint{3.538777in}{4.508045in}}%
\pgfpathlineto{\pgfqpoint{3.525433in}{4.525581in}}%
\pgfpathlineto{\pgfqpoint{3.512088in}{4.543254in}}%
\pgfpathlineto{\pgfqpoint{3.498740in}{4.561065in}}%
\pgfpathlineto{\pgfqpoint{3.485389in}{4.579016in}}%
\pgfpathlineto{\pgfqpoint{3.477821in}{4.540911in}}%
\pgfpathlineto{\pgfqpoint{3.470250in}{4.503446in}}%
\pgfpathlineto{\pgfqpoint{3.462675in}{4.466610in}}%
\pgfpathlineto{\pgfqpoint{3.455096in}{4.430392in}}%
\pgfpathclose%
\pgfusepath{fill}%
\end{pgfscope}%
\begin{pgfscope}%
\pgfpathrectangle{\pgfqpoint{1.150000in}{0.150000in}}{\pgfqpoint{5.700000in}{5.700000in}}%
\pgfusepath{clip}%
\pgfsetbuttcap%
\pgfsetroundjoin%
\definecolor{currentfill}{rgb}{0.197636,0.391528,0.554969}%
\pgfsetfillcolor{currentfill}%
\pgfsetfillopacity{0.700000}%
\pgfsetlinewidth{0.000000pt}%
\definecolor{currentstroke}{rgb}{0.000000,0.000000,0.000000}%
\pgfsetstrokecolor{currentstroke}%
\pgfsetdash{}{0pt}%
\pgfpathmoveto{\pgfqpoint{3.696531in}{3.316557in}}%
\pgfpathlineto{\pgfqpoint{3.709817in}{3.305814in}}%
\pgfpathlineto{\pgfqpoint{3.723105in}{3.295175in}}%
\pgfpathlineto{\pgfqpoint{3.736394in}{3.284640in}}%
\pgfpathlineto{\pgfqpoint{3.749686in}{3.274208in}}%
\pgfpathlineto{\pgfqpoint{3.757413in}{3.294946in}}%
\pgfpathlineto{\pgfqpoint{3.765136in}{3.316030in}}%
\pgfpathlineto{\pgfqpoint{3.772856in}{3.337468in}}%
\pgfpathlineto{\pgfqpoint{3.780572in}{3.359267in}}%
\pgfpathlineto{\pgfqpoint{3.767281in}{3.370100in}}%
\pgfpathlineto{\pgfqpoint{3.753991in}{3.381036in}}%
\pgfpathlineto{\pgfqpoint{3.740703in}{3.392076in}}%
\pgfpathlineto{\pgfqpoint{3.727417in}{3.403220in}}%
\pgfpathlineto{\pgfqpoint{3.719701in}{3.381011in}}%
\pgfpathlineto{\pgfqpoint{3.711982in}{3.359170in}}%
\pgfpathlineto{\pgfqpoint{3.704259in}{3.337688in}}%
\pgfpathlineto{\pgfqpoint{3.696531in}{3.316557in}}%
\pgfpathclose%
\pgfusepath{fill}%
\end{pgfscope}%
\begin{pgfscope}%
\pgfpathrectangle{\pgfqpoint{1.150000in}{0.150000in}}{\pgfqpoint{5.700000in}{5.700000in}}%
\pgfusepath{clip}%
\pgfsetbuttcap%
\pgfsetroundjoin%
\definecolor{currentfill}{rgb}{0.135066,0.544853,0.554029}%
\pgfsetfillcolor{currentfill}%
\pgfsetfillopacity{0.700000}%
\pgfsetlinewidth{0.000000pt}%
\definecolor{currentstroke}{rgb}{0.000000,0.000000,0.000000}%
\pgfsetstrokecolor{currentstroke}%
\pgfsetdash{}{0pt}%
\pgfpathmoveto{\pgfqpoint{3.956883in}{3.714242in}}%
\pgfpathlineto{\pgfqpoint{3.970191in}{3.702095in}}%
\pgfpathlineto{\pgfqpoint{3.983501in}{3.690048in}}%
\pgfpathlineto{\pgfqpoint{3.996812in}{3.678101in}}%
\pgfpathlineto{\pgfqpoint{4.010125in}{3.666253in}}%
\pgfpathlineto{\pgfqpoint{4.017813in}{3.694462in}}%
\pgfpathlineto{\pgfqpoint{4.025501in}{3.723178in}}%
\pgfpathlineto{\pgfqpoint{4.033188in}{3.752410in}}%
\pgfpathlineto{\pgfqpoint{4.040876in}{3.782169in}}%
\pgfpathlineto{\pgfqpoint{4.027559in}{3.794528in}}%
\pgfpathlineto{\pgfqpoint{4.014242in}{3.806987in}}%
\pgfpathlineto{\pgfqpoint{4.000927in}{3.819546in}}%
\pgfpathlineto{\pgfqpoint{3.987613in}{3.832206in}}%
\pgfpathlineto{\pgfqpoint{3.979931in}{3.801926in}}%
\pgfpathlineto{\pgfqpoint{3.972248in}{3.772179in}}%
\pgfpathlineto{\pgfqpoint{3.964566in}{3.742954in}}%
\pgfpathlineto{\pgfqpoint{3.956883in}{3.714242in}}%
\pgfpathclose%
\pgfusepath{fill}%
\end{pgfscope}%
\begin{pgfscope}%
\pgfpathrectangle{\pgfqpoint{1.150000in}{0.150000in}}{\pgfqpoint{5.700000in}{5.700000in}}%
\pgfusepath{clip}%
\pgfsetbuttcap%
\pgfsetroundjoin%
\definecolor{currentfill}{rgb}{0.182256,0.426184,0.557120}%
\pgfsetfillcolor{currentfill}%
\pgfsetfillopacity{0.700000}%
\pgfsetlinewidth{0.000000pt}%
\definecolor{currentstroke}{rgb}{0.000000,0.000000,0.000000}%
\pgfsetstrokecolor{currentstroke}%
\pgfsetdash{}{0pt}%
\pgfpathmoveto{\pgfqpoint{3.864588in}{3.406233in}}%
\pgfpathlineto{\pgfqpoint{3.877889in}{3.395484in}}%
\pgfpathlineto{\pgfqpoint{3.891192in}{3.384834in}}%
\pgfpathlineto{\pgfqpoint{3.904496in}{3.374283in}}%
\pgfpathlineto{\pgfqpoint{3.917803in}{3.363830in}}%
\pgfpathlineto{\pgfqpoint{3.925505in}{3.386684in}}%
\pgfpathlineto{\pgfqpoint{3.933206in}{3.409936in}}%
\pgfpathlineto{\pgfqpoint{3.940904in}{3.433595in}}%
\pgfpathlineto{\pgfqpoint{3.948600in}{3.457669in}}%
\pgfpathlineto{\pgfqpoint{3.935293in}{3.468564in}}%
\pgfpathlineto{\pgfqpoint{3.921987in}{3.479558in}}%
\pgfpathlineto{\pgfqpoint{3.908683in}{3.490651in}}%
\pgfpathlineto{\pgfqpoint{3.895381in}{3.501843in}}%
\pgfpathlineto{\pgfqpoint{3.887687in}{3.477318in}}%
\pgfpathlineto{\pgfqpoint{3.879990in}{3.453214in}}%
\pgfpathlineto{\pgfqpoint{3.872290in}{3.429522in}}%
\pgfpathlineto{\pgfqpoint{3.864588in}{3.406233in}}%
\pgfpathclose%
\pgfusepath{fill}%
\end{pgfscope}%
\begin{pgfscope}%
\pgfpathrectangle{\pgfqpoint{1.150000in}{0.150000in}}{\pgfqpoint{5.700000in}{5.700000in}}%
\pgfusepath{clip}%
\pgfsetbuttcap%
\pgfsetroundjoin%
\definecolor{currentfill}{rgb}{0.204903,0.375746,0.553533}%
\pgfsetfillcolor{currentfill}%
\pgfsetfillopacity{0.700000}%
\pgfsetlinewidth{0.000000pt}%
\definecolor{currentstroke}{rgb}{0.000000,0.000000,0.000000}%
\pgfsetstrokecolor{currentstroke}%
\pgfsetdash{}{0pt}%
\pgfpathmoveto{\pgfqpoint{3.612448in}{3.277916in}}%
\pgfpathlineto{\pgfqpoint{3.625729in}{3.267129in}}%
\pgfpathlineto{\pgfqpoint{3.639011in}{3.256448in}}%
\pgfpathlineto{\pgfqpoint{3.652295in}{3.245874in}}%
\pgfpathlineto{\pgfqpoint{3.665580in}{3.235406in}}%
\pgfpathlineto{\pgfqpoint{3.673324in}{3.255203in}}%
\pgfpathlineto{\pgfqpoint{3.681064in}{3.275322in}}%
\pgfpathlineto{\pgfqpoint{3.688800in}{3.295771in}}%
\pgfpathlineto{\pgfqpoint{3.696531in}{3.316557in}}%
\pgfpathlineto{\pgfqpoint{3.683247in}{3.327405in}}%
\pgfpathlineto{\pgfqpoint{3.669963in}{3.338359in}}%
\pgfpathlineto{\pgfqpoint{3.656682in}{3.349420in}}%
\pgfpathlineto{\pgfqpoint{3.643401in}{3.360588in}}%
\pgfpathlineto{\pgfqpoint{3.635670in}{3.339413in}}%
\pgfpathlineto{\pgfqpoint{3.627934in}{3.318581in}}%
\pgfpathlineto{\pgfqpoint{3.620193in}{3.298085in}}%
\pgfpathlineto{\pgfqpoint{3.612448in}{3.277916in}}%
\pgfpathclose%
\pgfusepath{fill}%
\end{pgfscope}%
\begin{pgfscope}%
\pgfpathrectangle{\pgfqpoint{1.150000in}{0.150000in}}{\pgfqpoint{5.700000in}{5.700000in}}%
\pgfusepath{clip}%
\pgfsetbuttcap%
\pgfsetroundjoin%
\definecolor{currentfill}{rgb}{0.119423,0.611141,0.538982}%
\pgfsetfillcolor{currentfill}%
\pgfsetfillopacity{0.700000}%
\pgfsetlinewidth{0.000000pt}%
\definecolor{currentstroke}{rgb}{0.000000,0.000000,0.000000}%
\pgfsetstrokecolor{currentstroke}%
\pgfsetdash{}{0pt}%
\pgfpathmoveto{\pgfqpoint{3.934368in}{3.883863in}}%
\pgfpathlineto{\pgfqpoint{3.947678in}{3.870794in}}%
\pgfpathlineto{\pgfqpoint{3.960989in}{3.857829in}}%
\pgfpathlineto{\pgfqpoint{3.974300in}{3.844966in}}%
\pgfpathlineto{\pgfqpoint{3.987613in}{3.832206in}}%
\pgfpathlineto{\pgfqpoint{3.995296in}{3.863028in}}%
\pgfpathlineto{\pgfqpoint{4.002979in}{3.894403in}}%
\pgfpathlineto{\pgfqpoint{4.010663in}{3.926341in}}%
\pgfpathlineto{\pgfqpoint{4.018347in}{3.958854in}}%
\pgfpathlineto{\pgfqpoint{4.005027in}{3.972152in}}%
\pgfpathlineto{\pgfqpoint{3.991708in}{3.985553in}}%
\pgfpathlineto{\pgfqpoint{3.978390in}{3.999057in}}%
\pgfpathlineto{\pgfqpoint{3.965073in}{4.012665in}}%
\pgfpathlineto{\pgfqpoint{3.957396in}{3.979604in}}%
\pgfpathlineto{\pgfqpoint{3.949720in}{3.947124in}}%
\pgfpathlineto{\pgfqpoint{3.942044in}{3.915214in}}%
\pgfpathlineto{\pgfqpoint{3.934368in}{3.883863in}}%
\pgfpathclose%
\pgfusepath{fill}%
\end{pgfscope}%
\begin{pgfscope}%
\pgfpathrectangle{\pgfqpoint{1.150000in}{0.150000in}}{\pgfqpoint{5.700000in}{5.700000in}}%
\pgfusepath{clip}%
\pgfsetbuttcap%
\pgfsetroundjoin%
\definecolor{currentfill}{rgb}{0.210503,0.363727,0.552206}%
\pgfsetfillcolor{currentfill}%
\pgfsetfillopacity{0.700000}%
\pgfsetlinewidth{0.000000pt}%
\definecolor{currentstroke}{rgb}{0.000000,0.000000,0.000000}%
\pgfsetstrokecolor{currentstroke}%
\pgfsetdash{}{0pt}%
\pgfpathmoveto{\pgfqpoint{3.390993in}{3.257282in}}%
\pgfpathlineto{\pgfqpoint{3.404265in}{3.245845in}}%
\pgfpathlineto{\pgfqpoint{3.417537in}{3.234526in}}%
\pgfpathlineto{\pgfqpoint{3.430810in}{3.223322in}}%
\pgfpathlineto{\pgfqpoint{3.444084in}{3.212233in}}%
\pgfpathlineto{\pgfqpoint{3.451873in}{3.230680in}}%
\pgfpathlineto{\pgfqpoint{3.459657in}{3.249413in}}%
\pgfpathlineto{\pgfqpoint{3.467435in}{3.268438in}}%
\pgfpathlineto{\pgfqpoint{3.475207in}{3.287761in}}%
\pgfpathlineto{\pgfqpoint{3.461934in}{3.299189in}}%
\pgfpathlineto{\pgfqpoint{3.448663in}{3.310733in}}%
\pgfpathlineto{\pgfqpoint{3.435391in}{3.322392in}}%
\pgfpathlineto{\pgfqpoint{3.422120in}{3.334169in}}%
\pgfpathlineto{\pgfqpoint{3.414347in}{3.314498in}}%
\pgfpathlineto{\pgfqpoint{3.406569in}{3.295131in}}%
\pgfpathlineto{\pgfqpoint{3.398784in}{3.276061in}}%
\pgfpathlineto{\pgfqpoint{3.390993in}{3.257282in}}%
\pgfpathclose%
\pgfusepath{fill}%
\end{pgfscope}%
\begin{pgfscope}%
\pgfpathrectangle{\pgfqpoint{1.150000in}{0.150000in}}{\pgfqpoint{5.700000in}{5.700000in}}%
\pgfusepath{clip}%
\pgfsetbuttcap%
\pgfsetroundjoin%
\definecolor{currentfill}{rgb}{0.468053,0.818921,0.323998}%
\pgfsetfillcolor{currentfill}%
\pgfsetfillopacity{0.700000}%
\pgfsetlinewidth{0.000000pt}%
\definecolor{currentstroke}{rgb}{0.000000,0.000000,0.000000}%
\pgfsetstrokecolor{currentstroke}%
\pgfsetdash{}{0pt}%
\pgfpathmoveto{\pgfqpoint{3.401725in}{4.501359in}}%
\pgfpathlineto{\pgfqpoint{3.415072in}{4.483402in}}%
\pgfpathlineto{\pgfqpoint{3.428416in}{4.465589in}}%
\pgfpathlineto{\pgfqpoint{3.441758in}{4.447920in}}%
\pgfpathlineto{\pgfqpoint{3.455096in}{4.430392in}}%
\pgfpathlineto{\pgfqpoint{3.462675in}{4.466610in}}%
\pgfpathlineto{\pgfqpoint{3.470250in}{4.503446in}}%
\pgfpathlineto{\pgfqpoint{3.477821in}{4.540911in}}%
\pgfpathlineto{\pgfqpoint{3.485389in}{4.579016in}}%
\pgfpathlineto{\pgfqpoint{3.472035in}{4.597109in}}%
\pgfpathlineto{\pgfqpoint{3.458678in}{4.615344in}}%
\pgfpathlineto{\pgfqpoint{3.445319in}{4.633723in}}%
\pgfpathlineto{\pgfqpoint{3.431956in}{4.652247in}}%
\pgfpathlineto{\pgfqpoint{3.424405in}{4.613565in}}%
\pgfpathlineto{\pgfqpoint{3.416849in}{4.575530in}}%
\pgfpathlineto{\pgfqpoint{3.409290in}{4.538132in}}%
\pgfpathlineto{\pgfqpoint{3.401725in}{4.501359in}}%
\pgfpathclose%
\pgfusepath{fill}%
\end{pgfscope}%
\begin{pgfscope}%
\pgfpathrectangle{\pgfqpoint{1.150000in}{0.150000in}}{\pgfqpoint{5.700000in}{5.700000in}}%
\pgfusepath{clip}%
\pgfsetbuttcap%
\pgfsetroundjoin%
\definecolor{currentfill}{rgb}{0.157729,0.485932,0.558013}%
\pgfsetfillcolor{currentfill}%
\pgfsetfillopacity{0.700000}%
\pgfsetlinewidth{0.000000pt}%
\definecolor{currentstroke}{rgb}{0.000000,0.000000,0.000000}%
\pgfsetstrokecolor{currentstroke}%
\pgfsetdash{}{0pt}%
\pgfpathmoveto{\pgfqpoint{3.979371in}{3.558287in}}%
\pgfpathlineto{\pgfqpoint{3.992682in}{3.547024in}}%
\pgfpathlineto{\pgfqpoint{4.005996in}{3.535859in}}%
\pgfpathlineto{\pgfqpoint{4.019311in}{3.524790in}}%
\pgfpathlineto{\pgfqpoint{4.032628in}{3.513817in}}%
\pgfpathlineto{\pgfqpoint{4.040320in}{3.539618in}}%
\pgfpathlineto{\pgfqpoint{4.048012in}{3.565882in}}%
\pgfpathlineto{\pgfqpoint{4.055703in}{3.592618in}}%
\pgfpathlineto{\pgfqpoint{4.063395in}{3.619835in}}%
\pgfpathlineto{\pgfqpoint{4.050075in}{3.631294in}}%
\pgfpathlineto{\pgfqpoint{4.036756in}{3.642850in}}%
\pgfpathlineto{\pgfqpoint{4.023440in}{3.654503in}}%
\pgfpathlineto{\pgfqpoint{4.010125in}{3.666253in}}%
\pgfpathlineto{\pgfqpoint{4.002438in}{3.638540in}}%
\pgfpathlineto{\pgfqpoint{3.994749in}{3.611314in}}%
\pgfpathlineto{\pgfqpoint{3.987061in}{3.584566in}}%
\pgfpathlineto{\pgfqpoint{3.979371in}{3.558287in}}%
\pgfpathclose%
\pgfusepath{fill}%
\end{pgfscope}%
\begin{pgfscope}%
\pgfpathrectangle{\pgfqpoint{1.150000in}{0.150000in}}{\pgfqpoint{5.700000in}{5.700000in}}%
\pgfusepath{clip}%
\pgfsetbuttcap%
\pgfsetroundjoin%
\definecolor{currentfill}{rgb}{0.206756,0.371758,0.553117}%
\pgfsetfillcolor{currentfill}%
\pgfsetfillopacity{0.700000}%
\pgfsetlinewidth{0.000000pt}%
\definecolor{currentstroke}{rgb}{0.000000,0.000000,0.000000}%
\pgfsetstrokecolor{currentstroke}%
\pgfsetdash{}{0pt}%
\pgfpathmoveto{\pgfqpoint{3.253589in}{3.278223in}}%
\pgfpathlineto{\pgfqpoint{3.266861in}{3.266135in}}%
\pgfpathlineto{\pgfqpoint{3.280134in}{3.254171in}}%
\pgfpathlineto{\pgfqpoint{3.293406in}{3.242332in}}%
\pgfpathlineto{\pgfqpoint{3.306678in}{3.230615in}}%
\pgfpathlineto{\pgfqpoint{3.314496in}{3.248602in}}%
\pgfpathlineto{\pgfqpoint{3.322306in}{3.266861in}}%
\pgfpathlineto{\pgfqpoint{3.330111in}{3.285398in}}%
\pgfpathlineto{\pgfqpoint{3.337909in}{3.304219in}}%
\pgfpathlineto{\pgfqpoint{3.324638in}{3.316256in}}%
\pgfpathlineto{\pgfqpoint{3.311367in}{3.328416in}}%
\pgfpathlineto{\pgfqpoint{3.298095in}{3.340700in}}%
\pgfpathlineto{\pgfqpoint{3.284824in}{3.353109in}}%
\pgfpathlineto{\pgfqpoint{3.277025in}{3.333959in}}%
\pgfpathlineto{\pgfqpoint{3.269220in}{3.315099in}}%
\pgfpathlineto{\pgfqpoint{3.261407in}{3.296522in}}%
\pgfpathlineto{\pgfqpoint{3.253589in}{3.278223in}}%
\pgfpathclose%
\pgfusepath{fill}%
\end{pgfscope}%
\begin{pgfscope}%
\pgfpathrectangle{\pgfqpoint{1.150000in}{0.150000in}}{\pgfqpoint{5.700000in}{5.700000in}}%
\pgfusepath{clip}%
\pgfsetbuttcap%
\pgfsetroundjoin%
\definecolor{currentfill}{rgb}{0.192357,0.403199,0.555836}%
\pgfsetfillcolor{currentfill}%
\pgfsetfillopacity{0.700000}%
\pgfsetlinewidth{0.000000pt}%
\definecolor{currentstroke}{rgb}{0.000000,0.000000,0.000000}%
\pgfsetstrokecolor{currentstroke}%
\pgfsetdash{}{0pt}%
\pgfpathmoveto{\pgfqpoint{3.062894in}{3.359289in}}%
\pgfpathlineto{\pgfqpoint{3.076179in}{3.345915in}}%
\pgfpathlineto{\pgfqpoint{3.089464in}{3.332679in}}%
\pgfpathlineto{\pgfqpoint{3.102746in}{3.319581in}}%
\pgfpathlineto{\pgfqpoint{3.116028in}{3.306619in}}%
\pgfpathlineto{\pgfqpoint{3.123878in}{3.324451in}}%
\pgfpathlineto{\pgfqpoint{3.131721in}{3.342547in}}%
\pgfpathlineto{\pgfqpoint{3.139556in}{3.360915in}}%
\pgfpathlineto{\pgfqpoint{3.147385in}{3.379559in}}%
\pgfpathlineto{\pgfqpoint{3.134105in}{3.392822in}}%
\pgfpathlineto{\pgfqpoint{3.120823in}{3.406222in}}%
\pgfpathlineto{\pgfqpoint{3.107540in}{3.419759in}}%
\pgfpathlineto{\pgfqpoint{3.094256in}{3.433435in}}%
\pgfpathlineto{\pgfqpoint{3.086427in}{3.414482in}}%
\pgfpathlineto{\pgfqpoint{3.078590in}{3.395810in}}%
\pgfpathlineto{\pgfqpoint{3.070746in}{3.377414in}}%
\pgfpathlineto{\pgfqpoint{3.062894in}{3.359289in}}%
\pgfpathclose%
\pgfusepath{fill}%
\end{pgfscope}%
\begin{pgfscope}%
\pgfpathrectangle{\pgfqpoint{1.150000in}{0.150000in}}{\pgfqpoint{5.700000in}{5.700000in}}%
\pgfusepath{clip}%
\pgfsetbuttcap%
\pgfsetroundjoin%
\definecolor{currentfill}{rgb}{0.212395,0.359683,0.551710}%
\pgfsetfillcolor{currentfill}%
\pgfsetfillopacity{0.700000}%
\pgfsetlinewidth{0.000000pt}%
\definecolor{currentstroke}{rgb}{0.000000,0.000000,0.000000}%
\pgfsetstrokecolor{currentstroke}%
\pgfsetdash{}{0pt}%
\pgfpathmoveto{\pgfqpoint{3.528305in}{3.243186in}}%
\pgfpathlineto{\pgfqpoint{3.541581in}{3.232321in}}%
\pgfpathlineto{\pgfqpoint{3.554859in}{3.221567in}}%
\pgfpathlineto{\pgfqpoint{3.568139in}{3.210923in}}%
\pgfpathlineto{\pgfqpoint{3.581419in}{3.200386in}}%
\pgfpathlineto{\pgfqpoint{3.589184in}{3.219311in}}%
\pgfpathlineto{\pgfqpoint{3.596943in}{3.238536in}}%
\pgfpathlineto{\pgfqpoint{3.604698in}{3.258069in}}%
\pgfpathlineto{\pgfqpoint{3.612448in}{3.277916in}}%
\pgfpathlineto{\pgfqpoint{3.599169in}{3.288812in}}%
\pgfpathlineto{\pgfqpoint{3.585890in}{3.299816in}}%
\pgfpathlineto{\pgfqpoint{3.572613in}{3.310929in}}%
\pgfpathlineto{\pgfqpoint{3.559337in}{3.322154in}}%
\pgfpathlineto{\pgfqpoint{3.551587in}{3.301939in}}%
\pgfpathlineto{\pgfqpoint{3.543831in}{3.282044in}}%
\pgfpathlineto{\pgfqpoint{3.536070in}{3.262461in}}%
\pgfpathlineto{\pgfqpoint{3.528305in}{3.243186in}}%
\pgfpathclose%
\pgfusepath{fill}%
\end{pgfscope}%
\begin{pgfscope}%
\pgfpathrectangle{\pgfqpoint{1.150000in}{0.150000in}}{\pgfqpoint{5.700000in}{5.700000in}}%
\pgfusepath{clip}%
\pgfsetbuttcap%
\pgfsetroundjoin%
\definecolor{currentfill}{rgb}{0.172719,0.448791,0.557885}%
\pgfsetfillcolor{currentfill}%
\pgfsetfillopacity{0.700000}%
\pgfsetlinewidth{0.000000pt}%
\definecolor{currentstroke}{rgb}{0.000000,0.000000,0.000000}%
\pgfsetstrokecolor{currentstroke}%
\pgfsetdash{}{0pt}%
\pgfpathmoveto{\pgfqpoint{3.948600in}{3.457669in}}%
\pgfpathlineto{\pgfqpoint{3.961910in}{3.446871in}}%
\pgfpathlineto{\pgfqpoint{3.975221in}{3.436169in}}%
\pgfpathlineto{\pgfqpoint{3.988535in}{3.425565in}}%
\pgfpathlineto{\pgfqpoint{4.001851in}{3.415055in}}%
\pgfpathlineto{\pgfqpoint{4.009547in}{3.439097in}}%
\pgfpathlineto{\pgfqpoint{4.017242in}{3.463565in}}%
\pgfpathlineto{\pgfqpoint{4.024936in}{3.488469in}}%
\pgfpathlineto{\pgfqpoint{4.032628in}{3.513817in}}%
\pgfpathlineto{\pgfqpoint{4.019311in}{3.524790in}}%
\pgfpathlineto{\pgfqpoint{4.005996in}{3.535859in}}%
\pgfpathlineto{\pgfqpoint{3.992682in}{3.547024in}}%
\pgfpathlineto{\pgfqpoint{3.979371in}{3.558287in}}%
\pgfpathlineto{\pgfqpoint{3.971680in}{3.532466in}}%
\pgfpathlineto{\pgfqpoint{3.963988in}{3.507096in}}%
\pgfpathlineto{\pgfqpoint{3.956295in}{3.482166in}}%
\pgfpathlineto{\pgfqpoint{3.948600in}{3.457669in}}%
\pgfpathclose%
\pgfusepath{fill}%
\end{pgfscope}%
\begin{pgfscope}%
\pgfpathrectangle{\pgfqpoint{1.150000in}{0.150000in}}{\pgfqpoint{5.700000in}{5.700000in}}%
\pgfusepath{clip}%
\pgfsetbuttcap%
\pgfsetroundjoin%
\definecolor{currentfill}{rgb}{0.140536,0.530132,0.555659}%
\pgfsetfillcolor{currentfill}%
\pgfsetfillopacity{0.700000}%
\pgfsetlinewidth{0.000000pt}%
\definecolor{currentstroke}{rgb}{0.000000,0.000000,0.000000}%
\pgfsetstrokecolor{currentstroke}%
\pgfsetdash{}{0pt}%
\pgfpathmoveto{\pgfqpoint{4.010125in}{3.666253in}}%
\pgfpathlineto{\pgfqpoint{4.023440in}{3.654503in}}%
\pgfpathlineto{\pgfqpoint{4.036756in}{3.642850in}}%
\pgfpathlineto{\pgfqpoint{4.050075in}{3.631294in}}%
\pgfpathlineto{\pgfqpoint{4.063395in}{3.619835in}}%
\pgfpathlineto{\pgfqpoint{4.071086in}{3.647543in}}%
\pgfpathlineto{\pgfqpoint{4.078778in}{3.675751in}}%
\pgfpathlineto{\pgfqpoint{4.086470in}{3.704470in}}%
\pgfpathlineto{\pgfqpoint{4.094163in}{3.733709in}}%
\pgfpathlineto{\pgfqpoint{4.080839in}{3.745678in}}%
\pgfpathlineto{\pgfqpoint{4.067516in}{3.757744in}}%
\pgfpathlineto{\pgfqpoint{4.054196in}{3.769907in}}%
\pgfpathlineto{\pgfqpoint{4.040876in}{3.782169in}}%
\pgfpathlineto{\pgfqpoint{4.033188in}{3.752410in}}%
\pgfpathlineto{\pgfqpoint{4.025501in}{3.723178in}}%
\pgfpathlineto{\pgfqpoint{4.017813in}{3.694462in}}%
\pgfpathlineto{\pgfqpoint{4.010125in}{3.666253in}}%
\pgfpathclose%
\pgfusepath{fill}%
\end{pgfscope}%
\begin{pgfscope}%
\pgfpathrectangle{\pgfqpoint{1.150000in}{0.150000in}}{\pgfqpoint{5.700000in}{5.700000in}}%
\pgfusepath{clip}%
\pgfsetbuttcap%
\pgfsetroundjoin%
\definecolor{currentfill}{rgb}{0.121148,0.592739,0.544641}%
\pgfsetfillcolor{currentfill}%
\pgfsetfillopacity{0.700000}%
\pgfsetlinewidth{0.000000pt}%
\definecolor{currentstroke}{rgb}{0.000000,0.000000,0.000000}%
\pgfsetstrokecolor{currentstroke}%
\pgfsetdash{}{0pt}%
\pgfpathmoveto{\pgfqpoint{3.987613in}{3.832206in}}%
\pgfpathlineto{\pgfqpoint{4.000927in}{3.819546in}}%
\pgfpathlineto{\pgfqpoint{4.014242in}{3.806987in}}%
\pgfpathlineto{\pgfqpoint{4.027559in}{3.794528in}}%
\pgfpathlineto{\pgfqpoint{4.040876in}{3.782169in}}%
\pgfpathlineto{\pgfqpoint{4.048565in}{3.812464in}}%
\pgfpathlineto{\pgfqpoint{4.056255in}{3.843305in}}%
\pgfpathlineto{\pgfqpoint{4.063945in}{3.874704in}}%
\pgfpathlineto{\pgfqpoint{4.071637in}{3.906671in}}%
\pgfpathlineto{\pgfqpoint{4.058313in}{3.919567in}}%
\pgfpathlineto{\pgfqpoint{4.044990in}{3.932562in}}%
\pgfpathlineto{\pgfqpoint{4.031668in}{3.945657in}}%
\pgfpathlineto{\pgfqpoint{4.018347in}{3.958854in}}%
\pgfpathlineto{\pgfqpoint{4.010663in}{3.926341in}}%
\pgfpathlineto{\pgfqpoint{4.002979in}{3.894403in}}%
\pgfpathlineto{\pgfqpoint{3.995296in}{3.863028in}}%
\pgfpathlineto{\pgfqpoint{3.987613in}{3.832206in}}%
\pgfpathclose%
\pgfusepath{fill}%
\end{pgfscope}%
\begin{pgfscope}%
\pgfpathrectangle{\pgfqpoint{1.150000in}{0.150000in}}{\pgfqpoint{5.700000in}{5.700000in}}%
\pgfusepath{clip}%
\pgfsetbuttcap%
\pgfsetroundjoin%
\definecolor{currentfill}{rgb}{0.252899,0.742211,0.448284}%
\pgfsetfillcolor{currentfill}%
\pgfsetfillopacity{0.700000}%
\pgfsetlinewidth{0.000000pt}%
\definecolor{currentstroke}{rgb}{0.000000,0.000000,0.000000}%
\pgfsetstrokecolor{currentstroke}%
\pgfsetdash{}{0pt}%
\pgfpathmoveto{\pgfqpoint{3.752021in}{4.245168in}}%
\pgfpathlineto{\pgfqpoint{3.765339in}{4.229788in}}%
\pgfpathlineto{\pgfqpoint{3.778656in}{4.214525in}}%
\pgfpathlineto{\pgfqpoint{3.791972in}{4.199378in}}%
\pgfpathlineto{\pgfqpoint{3.805288in}{4.184347in}}%
\pgfpathlineto{\pgfqpoint{3.812934in}{4.219675in}}%
\pgfpathlineto{\pgfqpoint{3.820581in}{4.255625in}}%
\pgfpathlineto{\pgfqpoint{3.828227in}{4.292209in}}%
\pgfpathlineto{\pgfqpoint{3.835873in}{4.329437in}}%
\pgfpathlineto{\pgfqpoint{3.822545in}{4.345041in}}%
\pgfpathlineto{\pgfqpoint{3.809216in}{4.360761in}}%
\pgfpathlineto{\pgfqpoint{3.795887in}{4.376598in}}%
\pgfpathlineto{\pgfqpoint{3.782557in}{4.392553in}}%
\pgfpathlineto{\pgfqpoint{3.774924in}{4.354741in}}%
\pgfpathlineto{\pgfqpoint{3.767290in}{4.317580in}}%
\pgfpathlineto{\pgfqpoint{3.759656in}{4.281060in}}%
\pgfpathlineto{\pgfqpoint{3.752021in}{4.245168in}}%
\pgfpathclose%
\pgfusepath{fill}%
\end{pgfscope}%
\begin{pgfscope}%
\pgfpathrectangle{\pgfqpoint{1.150000in}{0.150000in}}{\pgfqpoint{5.700000in}{5.700000in}}%
\pgfusepath{clip}%
\pgfsetbuttcap%
\pgfsetroundjoin%
\definecolor{currentfill}{rgb}{0.195860,0.395433,0.555276}%
\pgfsetfillcolor{currentfill}%
\pgfsetfillopacity{0.700000}%
\pgfsetlinewidth{0.000000pt}%
\definecolor{currentstroke}{rgb}{0.000000,0.000000,0.000000}%
\pgfsetstrokecolor{currentstroke}%
\pgfsetdash{}{0pt}%
\pgfpathmoveto{\pgfqpoint{3.833754in}{3.316953in}}%
\pgfpathlineto{\pgfqpoint{3.847054in}{3.306625in}}%
\pgfpathlineto{\pgfqpoint{3.860357in}{3.296396in}}%
\pgfpathlineto{\pgfqpoint{3.873661in}{3.286266in}}%
\pgfpathlineto{\pgfqpoint{3.886968in}{3.276233in}}%
\pgfpathlineto{\pgfqpoint{3.894681in}{3.297575in}}%
\pgfpathlineto{\pgfqpoint{3.902391in}{3.319284in}}%
\pgfpathlineto{\pgfqpoint{3.910098in}{3.341366in}}%
\pgfpathlineto{\pgfqpoint{3.917803in}{3.363830in}}%
\pgfpathlineto{\pgfqpoint{3.904496in}{3.374283in}}%
\pgfpathlineto{\pgfqpoint{3.891192in}{3.384834in}}%
\pgfpathlineto{\pgfqpoint{3.877889in}{3.395484in}}%
\pgfpathlineto{\pgfqpoint{3.864588in}{3.406233in}}%
\pgfpathlineto{\pgfqpoint{3.856884in}{3.383340in}}%
\pgfpathlineto{\pgfqpoint{3.849177in}{3.360834in}}%
\pgfpathlineto{\pgfqpoint{3.841467in}{3.338708in}}%
\pgfpathlineto{\pgfqpoint{3.833754in}{3.316953in}}%
\pgfpathclose%
\pgfusepath{fill}%
\end{pgfscope}%
\begin{pgfscope}%
\pgfpathrectangle{\pgfqpoint{1.150000in}{0.150000in}}{\pgfqpoint{5.700000in}{5.700000in}}%
\pgfusepath{clip}%
\pgfsetbuttcap%
\pgfsetroundjoin%
\definecolor{currentfill}{rgb}{0.214000,0.722114,0.469588}%
\pgfsetfillcolor{currentfill}%
\pgfsetfillopacity{0.700000}%
\pgfsetlinewidth{0.000000pt}%
\definecolor{currentstroke}{rgb}{0.000000,0.000000,0.000000}%
\pgfsetstrokecolor{currentstroke}%
\pgfsetdash{}{0pt}%
\pgfpathmoveto{\pgfqpoint{3.805288in}{4.184347in}}%
\pgfpathlineto{\pgfqpoint{3.818603in}{4.169431in}}%
\pgfpathlineto{\pgfqpoint{3.831918in}{4.154628in}}%
\pgfpathlineto{\pgfqpoint{3.845233in}{4.139938in}}%
\pgfpathlineto{\pgfqpoint{3.858548in}{4.125360in}}%
\pgfpathlineto{\pgfqpoint{3.866206in}{4.160127in}}%
\pgfpathlineto{\pgfqpoint{3.873864in}{4.195509in}}%
\pgfpathlineto{\pgfqpoint{3.881522in}{4.231518in}}%
\pgfpathlineto{\pgfqpoint{3.889180in}{4.268165in}}%
\pgfpathlineto{\pgfqpoint{3.875854in}{4.283313in}}%
\pgfpathlineto{\pgfqpoint{3.862528in}{4.298574in}}%
\pgfpathlineto{\pgfqpoint{3.849201in}{4.313948in}}%
\pgfpathlineto{\pgfqpoint{3.835873in}{4.329437in}}%
\pgfpathlineto{\pgfqpoint{3.828227in}{4.292209in}}%
\pgfpathlineto{\pgfqpoint{3.820581in}{4.255625in}}%
\pgfpathlineto{\pgfqpoint{3.812934in}{4.219675in}}%
\pgfpathlineto{\pgfqpoint{3.805288in}{4.184347in}}%
\pgfpathclose%
\pgfusepath{fill}%
\end{pgfscope}%
\begin{pgfscope}%
\pgfpathrectangle{\pgfqpoint{1.150000in}{0.150000in}}{\pgfqpoint{5.700000in}{5.700000in}}%
\pgfusepath{clip}%
\pgfsetbuttcap%
\pgfsetroundjoin%
\definecolor{currentfill}{rgb}{0.204903,0.375746,0.553533}%
\pgfsetfillcolor{currentfill}%
\pgfsetfillopacity{0.700000}%
\pgfsetlinewidth{0.000000pt}%
\definecolor{currentstroke}{rgb}{0.000000,0.000000,0.000000}%
\pgfsetstrokecolor{currentstroke}%
\pgfsetdash{}{0pt}%
\pgfpathmoveto{\pgfqpoint{3.749686in}{3.274208in}}%
\pgfpathlineto{\pgfqpoint{3.762979in}{3.263878in}}%
\pgfpathlineto{\pgfqpoint{3.776274in}{3.253649in}}%
\pgfpathlineto{\pgfqpoint{3.789571in}{3.243521in}}%
\pgfpathlineto{\pgfqpoint{3.802870in}{3.233493in}}%
\pgfpathlineto{\pgfqpoint{3.810596in}{3.253839in}}%
\pgfpathlineto{\pgfqpoint{3.818319in}{3.274526in}}%
\pgfpathlineto{\pgfqpoint{3.826038in}{3.295562in}}%
\pgfpathlineto{\pgfqpoint{3.833754in}{3.316953in}}%
\pgfpathlineto{\pgfqpoint{3.820456in}{3.327380in}}%
\pgfpathlineto{\pgfqpoint{3.807159in}{3.337908in}}%
\pgfpathlineto{\pgfqpoint{3.793865in}{3.348537in}}%
\pgfpathlineto{\pgfqpoint{3.780572in}{3.359267in}}%
\pgfpathlineto{\pgfqpoint{3.772856in}{3.337468in}}%
\pgfpathlineto{\pgfqpoint{3.765136in}{3.316030in}}%
\pgfpathlineto{\pgfqpoint{3.757413in}{3.294946in}}%
\pgfpathlineto{\pgfqpoint{3.749686in}{3.274208in}}%
\pgfpathclose%
\pgfusepath{fill}%
\end{pgfscope}%
\begin{pgfscope}%
\pgfpathrectangle{\pgfqpoint{1.150000in}{0.150000in}}{\pgfqpoint{5.700000in}{5.700000in}}%
\pgfusepath{clip}%
\pgfsetbuttcap%
\pgfsetroundjoin%
\definecolor{currentfill}{rgb}{0.201239,0.383670,0.554294}%
\pgfsetfillcolor{currentfill}%
\pgfsetfillopacity{0.700000}%
\pgfsetlinewidth{0.000000pt}%
\definecolor{currentstroke}{rgb}{0.000000,0.000000,0.000000}%
\pgfsetstrokecolor{currentstroke}%
\pgfsetdash{}{0pt}%
\pgfpathmoveto{\pgfqpoint{3.116028in}{3.306619in}}%
\pgfpathlineto{\pgfqpoint{3.129308in}{3.293793in}}%
\pgfpathlineto{\pgfqpoint{3.142587in}{3.281099in}}%
\pgfpathlineto{\pgfqpoint{3.155865in}{3.268539in}}%
\pgfpathlineto{\pgfqpoint{3.169143in}{3.256110in}}%
\pgfpathlineto{\pgfqpoint{3.176991in}{3.273649in}}%
\pgfpathlineto{\pgfqpoint{3.184832in}{3.291448in}}%
\pgfpathlineto{\pgfqpoint{3.192666in}{3.309512in}}%
\pgfpathlineto{\pgfqpoint{3.200493in}{3.327847in}}%
\pgfpathlineto{\pgfqpoint{3.187218in}{3.340576in}}%
\pgfpathlineto{\pgfqpoint{3.173941in}{3.353437in}}%
\pgfpathlineto{\pgfqpoint{3.160663in}{3.366431in}}%
\pgfpathlineto{\pgfqpoint{3.147385in}{3.379559in}}%
\pgfpathlineto{\pgfqpoint{3.139556in}{3.360915in}}%
\pgfpathlineto{\pgfqpoint{3.131721in}{3.342547in}}%
\pgfpathlineto{\pgfqpoint{3.123878in}{3.324451in}}%
\pgfpathlineto{\pgfqpoint{3.116028in}{3.306619in}}%
\pgfpathclose%
\pgfusepath{fill}%
\end{pgfscope}%
\begin{pgfscope}%
\pgfpathrectangle{\pgfqpoint{1.150000in}{0.150000in}}{\pgfqpoint{5.700000in}{5.700000in}}%
\pgfusepath{clip}%
\pgfsetbuttcap%
\pgfsetroundjoin%
\definecolor{currentfill}{rgb}{0.304148,0.764704,0.419943}%
\pgfsetfillcolor{currentfill}%
\pgfsetfillopacity{0.700000}%
\pgfsetlinewidth{0.000000pt}%
\definecolor{currentstroke}{rgb}{0.000000,0.000000,0.000000}%
\pgfsetstrokecolor{currentstroke}%
\pgfsetdash{}{0pt}%
\pgfpathmoveto{\pgfqpoint{3.698743in}{4.307882in}}%
\pgfpathlineto{\pgfqpoint{3.712064in}{4.292022in}}%
\pgfpathlineto{\pgfqpoint{3.725384in}{4.276284in}}%
\pgfpathlineto{\pgfqpoint{3.738703in}{4.260666in}}%
\pgfpathlineto{\pgfqpoint{3.752021in}{4.245168in}}%
\pgfpathlineto{\pgfqpoint{3.759656in}{4.281060in}}%
\pgfpathlineto{\pgfqpoint{3.767290in}{4.317580in}}%
\pgfpathlineto{\pgfqpoint{3.774924in}{4.354741in}}%
\pgfpathlineto{\pgfqpoint{3.782557in}{4.392553in}}%
\pgfpathlineto{\pgfqpoint{3.769225in}{4.408628in}}%
\pgfpathlineto{\pgfqpoint{3.755893in}{4.424822in}}%
\pgfpathlineto{\pgfqpoint{3.742560in}{4.441138in}}%
\pgfpathlineto{\pgfqpoint{3.729225in}{4.457575in}}%
\pgfpathlineto{\pgfqpoint{3.721606in}{4.419175in}}%
\pgfpathlineto{\pgfqpoint{3.713986in}{4.381434in}}%
\pgfpathlineto{\pgfqpoint{3.706365in}{4.344340in}}%
\pgfpathlineto{\pgfqpoint{3.698743in}{4.307882in}}%
\pgfpathclose%
\pgfusepath{fill}%
\end{pgfscope}%
\begin{pgfscope}%
\pgfpathrectangle{\pgfqpoint{1.150000in}{0.150000in}}{\pgfqpoint{5.700000in}{5.700000in}}%
\pgfusepath{clip}%
\pgfsetbuttcap%
\pgfsetroundjoin%
\definecolor{currentfill}{rgb}{0.180653,0.701402,0.488189}%
\pgfsetfillcolor{currentfill}%
\pgfsetfillopacity{0.700000}%
\pgfsetlinewidth{0.000000pt}%
\definecolor{currentstroke}{rgb}{0.000000,0.000000,0.000000}%
\pgfsetstrokecolor{currentstroke}%
\pgfsetdash{}{0pt}%
\pgfpathmoveto{\pgfqpoint{3.858548in}{4.125360in}}%
\pgfpathlineto{\pgfqpoint{3.871863in}{4.110893in}}%
\pgfpathlineto{\pgfqpoint{3.885178in}{4.096537in}}%
\pgfpathlineto{\pgfqpoint{3.898493in}{4.082289in}}%
\pgfpathlineto{\pgfqpoint{3.911808in}{4.068150in}}%
\pgfpathlineto{\pgfqpoint{3.919476in}{4.102358in}}%
\pgfpathlineto{\pgfqpoint{3.927144in}{4.137175in}}%
\pgfpathlineto{\pgfqpoint{3.934813in}{4.172612in}}%
\pgfpathlineto{\pgfqpoint{3.942484in}{4.208680in}}%
\pgfpathlineto{\pgfqpoint{3.929158in}{4.223387in}}%
\pgfpathlineto{\pgfqpoint{3.915832in}{4.238203in}}%
\pgfpathlineto{\pgfqpoint{3.902506in}{4.253128in}}%
\pgfpathlineto{\pgfqpoint{3.889180in}{4.268165in}}%
\pgfpathlineto{\pgfqpoint{3.881522in}{4.231518in}}%
\pgfpathlineto{\pgfqpoint{3.873864in}{4.195509in}}%
\pgfpathlineto{\pgfqpoint{3.866206in}{4.160127in}}%
\pgfpathlineto{\pgfqpoint{3.858548in}{4.125360in}}%
\pgfpathclose%
\pgfusepath{fill}%
\end{pgfscope}%
\begin{pgfscope}%
\pgfpathrectangle{\pgfqpoint{1.150000in}{0.150000in}}{\pgfqpoint{5.700000in}{5.700000in}}%
\pgfusepath{clip}%
\pgfsetbuttcap%
\pgfsetroundjoin%
\definecolor{currentfill}{rgb}{0.187231,0.414746,0.556547}%
\pgfsetfillcolor{currentfill}%
\pgfsetfillopacity{0.700000}%
\pgfsetlinewidth{0.000000pt}%
\definecolor{currentstroke}{rgb}{0.000000,0.000000,0.000000}%
\pgfsetstrokecolor{currentstroke}%
\pgfsetdash{}{0pt}%
\pgfpathmoveto{\pgfqpoint{3.917803in}{3.363830in}}%
\pgfpathlineto{\pgfqpoint{3.931112in}{3.353474in}}%
\pgfpathlineto{\pgfqpoint{3.944423in}{3.343215in}}%
\pgfpathlineto{\pgfqpoint{3.957736in}{3.333052in}}%
\pgfpathlineto{\pgfqpoint{3.971052in}{3.322984in}}%
\pgfpathlineto{\pgfqpoint{3.978754in}{3.345404in}}%
\pgfpathlineto{\pgfqpoint{3.986455in}{3.368217in}}%
\pgfpathlineto{\pgfqpoint{3.994154in}{3.391432in}}%
\pgfpathlineto{\pgfqpoint{4.001851in}{3.415055in}}%
\pgfpathlineto{\pgfqpoint{3.988535in}{3.425565in}}%
\pgfpathlineto{\pgfqpoint{3.975221in}{3.436169in}}%
\pgfpathlineto{\pgfqpoint{3.961910in}{3.446871in}}%
\pgfpathlineto{\pgfqpoint{3.948600in}{3.457669in}}%
\pgfpathlineto{\pgfqpoint{3.940904in}{3.433595in}}%
\pgfpathlineto{\pgfqpoint{3.933206in}{3.409936in}}%
\pgfpathlineto{\pgfqpoint{3.925505in}{3.386684in}}%
\pgfpathlineto{\pgfqpoint{3.917803in}{3.363830in}}%
\pgfpathclose%
\pgfusepath{fill}%
\end{pgfscope}%
\begin{pgfscope}%
\pgfpathrectangle{\pgfqpoint{1.150000in}{0.150000in}}{\pgfqpoint{5.700000in}{5.700000in}}%
\pgfusepath{clip}%
\pgfsetbuttcap%
\pgfsetroundjoin%
\definecolor{currentfill}{rgb}{0.352360,0.783011,0.392636}%
\pgfsetfillcolor{currentfill}%
\pgfsetfillopacity{0.700000}%
\pgfsetlinewidth{0.000000pt}%
\definecolor{currentstroke}{rgb}{0.000000,0.000000,0.000000}%
\pgfsetstrokecolor{currentstroke}%
\pgfsetdash{}{0pt}%
\pgfpathmoveto{\pgfqpoint{3.645447in}{4.372553in}}%
\pgfpathlineto{\pgfqpoint{3.658773in}{4.356198in}}%
\pgfpathlineto{\pgfqpoint{3.672098in}{4.339968in}}%
\pgfpathlineto{\pgfqpoint{3.685421in}{4.323863in}}%
\pgfpathlineto{\pgfqpoint{3.698743in}{4.307882in}}%
\pgfpathlineto{\pgfqpoint{3.706365in}{4.344340in}}%
\pgfpathlineto{\pgfqpoint{3.713986in}{4.381434in}}%
\pgfpathlineto{\pgfqpoint{3.721606in}{4.419175in}}%
\pgfpathlineto{\pgfqpoint{3.729225in}{4.457575in}}%
\pgfpathlineto{\pgfqpoint{3.715889in}{4.474136in}}%
\pgfpathlineto{\pgfqpoint{3.702552in}{4.490821in}}%
\pgfpathlineto{\pgfqpoint{3.689213in}{4.507631in}}%
\pgfpathlineto{\pgfqpoint{3.675872in}{4.524567in}}%
\pgfpathlineto{\pgfqpoint{3.668268in}{4.485577in}}%
\pgfpathlineto{\pgfqpoint{3.660663in}{4.447252in}}%
\pgfpathlineto{\pgfqpoint{3.653056in}{4.409581in}}%
\pgfpathlineto{\pgfqpoint{3.645447in}{4.372553in}}%
\pgfpathclose%
\pgfusepath{fill}%
\end{pgfscope}%
\begin{pgfscope}%
\pgfpathrectangle{\pgfqpoint{1.150000in}{0.150000in}}{\pgfqpoint{5.700000in}{5.700000in}}%
\pgfusepath{clip}%
\pgfsetbuttcap%
\pgfsetroundjoin%
\definecolor{currentfill}{rgb}{0.212395,0.359683,0.551710}%
\pgfsetfillcolor{currentfill}%
\pgfsetfillopacity{0.700000}%
\pgfsetlinewidth{0.000000pt}%
\definecolor{currentstroke}{rgb}{0.000000,0.000000,0.000000}%
\pgfsetstrokecolor{currentstroke}%
\pgfsetdash{}{0pt}%
\pgfpathmoveto{\pgfqpoint{3.665580in}{3.235406in}}%
\pgfpathlineto{\pgfqpoint{3.678867in}{3.225043in}}%
\pgfpathlineto{\pgfqpoint{3.692155in}{3.214783in}}%
\pgfpathlineto{\pgfqpoint{3.705446in}{3.204628in}}%
\pgfpathlineto{\pgfqpoint{3.718738in}{3.194574in}}%
\pgfpathlineto{\pgfqpoint{3.726481in}{3.213999in}}%
\pgfpathlineto{\pgfqpoint{3.734220in}{3.233742in}}%
\pgfpathlineto{\pgfqpoint{3.741955in}{3.253809in}}%
\pgfpathlineto{\pgfqpoint{3.749686in}{3.274208in}}%
\pgfpathlineto{\pgfqpoint{3.736394in}{3.284640in}}%
\pgfpathlineto{\pgfqpoint{3.723105in}{3.295175in}}%
\pgfpathlineto{\pgfqpoint{3.709817in}{3.305814in}}%
\pgfpathlineto{\pgfqpoint{3.696531in}{3.316557in}}%
\pgfpathlineto{\pgfqpoint{3.688800in}{3.295771in}}%
\pgfpathlineto{\pgfqpoint{3.681064in}{3.275322in}}%
\pgfpathlineto{\pgfqpoint{3.673324in}{3.255203in}}%
\pgfpathlineto{\pgfqpoint{3.665580in}{3.235406in}}%
\pgfpathclose%
\pgfusepath{fill}%
\end{pgfscope}%
\begin{pgfscope}%
\pgfpathrectangle{\pgfqpoint{1.150000in}{0.150000in}}{\pgfqpoint{5.700000in}{5.700000in}}%
\pgfusepath{clip}%
\pgfsetbuttcap%
\pgfsetroundjoin%
\definecolor{currentfill}{rgb}{0.214298,0.355619,0.551184}%
\pgfsetfillcolor{currentfill}%
\pgfsetfillopacity{0.700000}%
\pgfsetlinewidth{0.000000pt}%
\definecolor{currentstroke}{rgb}{0.000000,0.000000,0.000000}%
\pgfsetstrokecolor{currentstroke}%
\pgfsetdash{}{0pt}%
\pgfpathmoveto{\pgfqpoint{3.306678in}{3.230615in}}%
\pgfpathlineto{\pgfqpoint{3.319951in}{3.219020in}}%
\pgfpathlineto{\pgfqpoint{3.333223in}{3.207545in}}%
\pgfpathlineto{\pgfqpoint{3.346496in}{3.196191in}}%
\pgfpathlineto{\pgfqpoint{3.359769in}{3.184955in}}%
\pgfpathlineto{\pgfqpoint{3.367584in}{3.202631in}}%
\pgfpathlineto{\pgfqpoint{3.375393in}{3.220573in}}%
\pgfpathlineto{\pgfqpoint{3.383196in}{3.238788in}}%
\pgfpathlineto{\pgfqpoint{3.390993in}{3.257282in}}%
\pgfpathlineto{\pgfqpoint{3.377722in}{3.268837in}}%
\pgfpathlineto{\pgfqpoint{3.364450in}{3.280511in}}%
\pgfpathlineto{\pgfqpoint{3.351179in}{3.292305in}}%
\pgfpathlineto{\pgfqpoint{3.337909in}{3.304219in}}%
\pgfpathlineto{\pgfqpoint{3.330111in}{3.285398in}}%
\pgfpathlineto{\pgfqpoint{3.322306in}{3.266861in}}%
\pgfpathlineto{\pgfqpoint{3.314496in}{3.248602in}}%
\pgfpathlineto{\pgfqpoint{3.306678in}{3.230615in}}%
\pgfpathclose%
\pgfusepath{fill}%
\end{pgfscope}%
\begin{pgfscope}%
\pgfpathrectangle{\pgfqpoint{1.150000in}{0.150000in}}{\pgfqpoint{5.700000in}{5.700000in}}%
\pgfusepath{clip}%
\pgfsetbuttcap%
\pgfsetroundjoin%
\definecolor{currentfill}{rgb}{0.218130,0.347432,0.550038}%
\pgfsetfillcolor{currentfill}%
\pgfsetfillopacity{0.700000}%
\pgfsetlinewidth{0.000000pt}%
\definecolor{currentstroke}{rgb}{0.000000,0.000000,0.000000}%
\pgfsetstrokecolor{currentstroke}%
\pgfsetdash{}{0pt}%
\pgfpathmoveto{\pgfqpoint{3.444084in}{3.212233in}}%
\pgfpathlineto{\pgfqpoint{3.457358in}{3.201259in}}%
\pgfpathlineto{\pgfqpoint{3.470634in}{3.190399in}}%
\pgfpathlineto{\pgfqpoint{3.483910in}{3.179651in}}%
\pgfpathlineto{\pgfqpoint{3.497187in}{3.169014in}}%
\pgfpathlineto{\pgfqpoint{3.504975in}{3.187130in}}%
\pgfpathlineto{\pgfqpoint{3.512757in}{3.205526in}}%
\pgfpathlineto{\pgfqpoint{3.520533in}{3.224209in}}%
\pgfpathlineto{\pgfqpoint{3.528305in}{3.243186in}}%
\pgfpathlineto{\pgfqpoint{3.515029in}{3.254161in}}%
\pgfpathlineto{\pgfqpoint{3.501754in}{3.265248in}}%
\pgfpathlineto{\pgfqpoint{3.488480in}{3.276448in}}%
\pgfpathlineto{\pgfqpoint{3.475207in}{3.287761in}}%
\pgfpathlineto{\pgfqpoint{3.467435in}{3.268438in}}%
\pgfpathlineto{\pgfqpoint{3.459657in}{3.249413in}}%
\pgfpathlineto{\pgfqpoint{3.451873in}{3.230680in}}%
\pgfpathlineto{\pgfqpoint{3.444084in}{3.212233in}}%
\pgfpathclose%
\pgfusepath{fill}%
\end{pgfscope}%
\begin{pgfscope}%
\pgfpathrectangle{\pgfqpoint{1.150000in}{0.150000in}}{\pgfqpoint{5.700000in}{5.700000in}}%
\pgfusepath{clip}%
\pgfsetbuttcap%
\pgfsetroundjoin%
\definecolor{currentfill}{rgb}{0.157851,0.683765,0.501686}%
\pgfsetfillcolor{currentfill}%
\pgfsetfillopacity{0.700000}%
\pgfsetlinewidth{0.000000pt}%
\definecolor{currentstroke}{rgb}{0.000000,0.000000,0.000000}%
\pgfsetstrokecolor{currentstroke}%
\pgfsetdash{}{0pt}%
\pgfpathmoveto{\pgfqpoint{3.911808in}{4.068150in}}%
\pgfpathlineto{\pgfqpoint{3.925123in}{4.054119in}}%
\pgfpathlineto{\pgfqpoint{3.938439in}{4.040195in}}%
\pgfpathlineto{\pgfqpoint{3.951756in}{4.026377in}}%
\pgfpathlineto{\pgfqpoint{3.965073in}{4.012665in}}%
\pgfpathlineto{\pgfqpoint{3.972750in}{4.046316in}}%
\pgfpathlineto{\pgfqpoint{3.980428in}{4.080570in}}%
\pgfpathlineto{\pgfqpoint{3.988107in}{4.115438in}}%
\pgfpathlineto{\pgfqpoint{3.995788in}{4.150930in}}%
\pgfpathlineto{\pgfqpoint{3.982462in}{4.165208in}}%
\pgfpathlineto{\pgfqpoint{3.969135in}{4.179592in}}%
\pgfpathlineto{\pgfqpoint{3.955809in}{4.194082in}}%
\pgfpathlineto{\pgfqpoint{3.942484in}{4.208680in}}%
\pgfpathlineto{\pgfqpoint{3.934813in}{4.172612in}}%
\pgfpathlineto{\pgfqpoint{3.927144in}{4.137175in}}%
\pgfpathlineto{\pgfqpoint{3.919476in}{4.102358in}}%
\pgfpathlineto{\pgfqpoint{3.911808in}{4.068150in}}%
\pgfpathclose%
\pgfusepath{fill}%
\end{pgfscope}%
\begin{pgfscope}%
\pgfpathrectangle{\pgfqpoint{1.150000in}{0.150000in}}{\pgfqpoint{5.700000in}{5.700000in}}%
\pgfusepath{clip}%
\pgfsetbuttcap%
\pgfsetroundjoin%
\definecolor{currentfill}{rgb}{0.125394,0.574318,0.549086}%
\pgfsetfillcolor{currentfill}%
\pgfsetfillopacity{0.700000}%
\pgfsetlinewidth{0.000000pt}%
\definecolor{currentstroke}{rgb}{0.000000,0.000000,0.000000}%
\pgfsetstrokecolor{currentstroke}%
\pgfsetdash{}{0pt}%
\pgfpathmoveto{\pgfqpoint{4.040876in}{3.782169in}}%
\pgfpathlineto{\pgfqpoint{4.054196in}{3.769907in}}%
\pgfpathlineto{\pgfqpoint{4.067516in}{3.757744in}}%
\pgfpathlineto{\pgfqpoint{4.080839in}{3.745678in}}%
\pgfpathlineto{\pgfqpoint{4.094163in}{3.733709in}}%
\pgfpathlineto{\pgfqpoint{4.101857in}{3.763479in}}%
\pgfpathlineto{\pgfqpoint{4.109552in}{3.793789in}}%
\pgfpathlineto{\pgfqpoint{4.117248in}{3.824651in}}%
\pgfpathlineto{\pgfqpoint{4.124947in}{3.856074in}}%
\pgfpathlineto{\pgfqpoint{4.111617in}{3.868577in}}%
\pgfpathlineto{\pgfqpoint{4.098289in}{3.881177in}}%
\pgfpathlineto{\pgfqpoint{4.084962in}{3.893875in}}%
\pgfpathlineto{\pgfqpoint{4.071637in}{3.906671in}}%
\pgfpathlineto{\pgfqpoint{4.063945in}{3.874704in}}%
\pgfpathlineto{\pgfqpoint{4.056255in}{3.843305in}}%
\pgfpathlineto{\pgfqpoint{4.048565in}{3.812464in}}%
\pgfpathlineto{\pgfqpoint{4.040876in}{3.782169in}}%
\pgfpathclose%
\pgfusepath{fill}%
\end{pgfscope}%
\begin{pgfscope}%
\pgfpathrectangle{\pgfqpoint{1.150000in}{0.150000in}}{\pgfqpoint{5.700000in}{5.700000in}}%
\pgfusepath{clip}%
\pgfsetbuttcap%
\pgfsetroundjoin%
\definecolor{currentfill}{rgb}{0.163625,0.471133,0.558148}%
\pgfsetfillcolor{currentfill}%
\pgfsetfillopacity{0.700000}%
\pgfsetlinewidth{0.000000pt}%
\definecolor{currentstroke}{rgb}{0.000000,0.000000,0.000000}%
\pgfsetstrokecolor{currentstroke}%
\pgfsetdash{}{0pt}%
\pgfpathmoveto{\pgfqpoint{4.032628in}{3.513817in}}%
\pgfpathlineto{\pgfqpoint{4.045948in}{3.502939in}}%
\pgfpathlineto{\pgfqpoint{4.059270in}{3.492157in}}%
\pgfpathlineto{\pgfqpoint{4.072594in}{3.481468in}}%
\pgfpathlineto{\pgfqpoint{4.085920in}{3.470873in}}%
\pgfpathlineto{\pgfqpoint{4.093614in}{3.496198in}}%
\pgfpathlineto{\pgfqpoint{4.101307in}{3.521979in}}%
\pgfpathlineto{\pgfqpoint{4.109001in}{3.548226in}}%
\pgfpathlineto{\pgfqpoint{4.116695in}{3.574949in}}%
\pgfpathlineto{\pgfqpoint{4.103367in}{3.586029in}}%
\pgfpathlineto{\pgfqpoint{4.090041in}{3.597203in}}%
\pgfpathlineto{\pgfqpoint{4.076717in}{3.608471in}}%
\pgfpathlineto{\pgfqpoint{4.063395in}{3.619835in}}%
\pgfpathlineto{\pgfqpoint{4.055703in}{3.592618in}}%
\pgfpathlineto{\pgfqpoint{4.048012in}{3.565882in}}%
\pgfpathlineto{\pgfqpoint{4.040320in}{3.539618in}}%
\pgfpathlineto{\pgfqpoint{4.032628in}{3.513817in}}%
\pgfpathclose%
\pgfusepath{fill}%
\end{pgfscope}%
\begin{pgfscope}%
\pgfpathrectangle{\pgfqpoint{1.150000in}{0.150000in}}{\pgfqpoint{5.700000in}{5.700000in}}%
\pgfusepath{clip}%
\pgfsetbuttcap%
\pgfsetroundjoin%
\definecolor{currentfill}{rgb}{0.412913,0.803041,0.357269}%
\pgfsetfillcolor{currentfill}%
\pgfsetfillopacity{0.700000}%
\pgfsetlinewidth{0.000000pt}%
\definecolor{currentstroke}{rgb}{0.000000,0.000000,0.000000}%
\pgfsetstrokecolor{currentstroke}%
\pgfsetdash{}{0pt}%
\pgfpathmoveto{\pgfqpoint{3.592127in}{4.439250in}}%
\pgfpathlineto{\pgfqpoint{3.605460in}{4.422382in}}%
\pgfpathlineto{\pgfqpoint{3.618791in}{4.405644in}}%
\pgfpathlineto{\pgfqpoint{3.632120in}{4.389034in}}%
\pgfpathlineto{\pgfqpoint{3.645447in}{4.372553in}}%
\pgfpathlineto{\pgfqpoint{3.653056in}{4.409581in}}%
\pgfpathlineto{\pgfqpoint{3.660663in}{4.447252in}}%
\pgfpathlineto{\pgfqpoint{3.668268in}{4.485577in}}%
\pgfpathlineto{\pgfqpoint{3.675872in}{4.524567in}}%
\pgfpathlineto{\pgfqpoint{3.662530in}{4.541631in}}%
\pgfpathlineto{\pgfqpoint{3.649186in}{4.558823in}}%
\pgfpathlineto{\pgfqpoint{3.635840in}{4.576145in}}%
\pgfpathlineto{\pgfqpoint{3.622492in}{4.593597in}}%
\pgfpathlineto{\pgfqpoint{3.614904in}{4.554013in}}%
\pgfpathlineto{\pgfqpoint{3.607314in}{4.515102in}}%
\pgfpathlineto{\pgfqpoint{3.599722in}{4.476851in}}%
\pgfpathlineto{\pgfqpoint{3.592127in}{4.439250in}}%
\pgfpathclose%
\pgfusepath{fill}%
\end{pgfscope}%
\begin{pgfscope}%
\pgfpathrectangle{\pgfqpoint{1.150000in}{0.150000in}}{\pgfqpoint{5.700000in}{5.700000in}}%
\pgfusepath{clip}%
\pgfsetbuttcap%
\pgfsetroundjoin%
\definecolor{currentfill}{rgb}{0.146180,0.515413,0.556823}%
\pgfsetfillcolor{currentfill}%
\pgfsetfillopacity{0.700000}%
\pgfsetlinewidth{0.000000pt}%
\definecolor{currentstroke}{rgb}{0.000000,0.000000,0.000000}%
\pgfsetstrokecolor{currentstroke}%
\pgfsetdash{}{0pt}%
\pgfpathmoveto{\pgfqpoint{4.063395in}{3.619835in}}%
\pgfpathlineto{\pgfqpoint{4.076717in}{3.608471in}}%
\pgfpathlineto{\pgfqpoint{4.090041in}{3.597203in}}%
\pgfpathlineto{\pgfqpoint{4.103367in}{3.586029in}}%
\pgfpathlineto{\pgfqpoint{4.116695in}{3.574949in}}%
\pgfpathlineto{\pgfqpoint{4.124389in}{3.602156in}}%
\pgfpathlineto{\pgfqpoint{4.132084in}{3.629859in}}%
\pgfpathlineto{\pgfqpoint{4.139780in}{3.658066in}}%
\pgfpathlineto{\pgfqpoint{4.147477in}{3.686787in}}%
\pgfpathlineto{\pgfqpoint{4.134145in}{3.698376in}}%
\pgfpathlineto{\pgfqpoint{4.120816in}{3.710059in}}%
\pgfpathlineto{\pgfqpoint{4.107488in}{3.721836in}}%
\pgfpathlineto{\pgfqpoint{4.094163in}{3.733709in}}%
\pgfpathlineto{\pgfqpoint{4.086470in}{3.704470in}}%
\pgfpathlineto{\pgfqpoint{4.078778in}{3.675751in}}%
\pgfpathlineto{\pgfqpoint{4.071086in}{3.647543in}}%
\pgfpathlineto{\pgfqpoint{4.063395in}{3.619835in}}%
\pgfpathclose%
\pgfusepath{fill}%
\end{pgfscope}%
\begin{pgfscope}%
\pgfpathrectangle{\pgfqpoint{1.150000in}{0.150000in}}{\pgfqpoint{5.700000in}{5.700000in}}%
\pgfusepath{clip}%
\pgfsetbuttcap%
\pgfsetroundjoin%
\definecolor{currentfill}{rgb}{0.137339,0.662252,0.515571}%
\pgfsetfillcolor{currentfill}%
\pgfsetfillopacity{0.700000}%
\pgfsetlinewidth{0.000000pt}%
\definecolor{currentstroke}{rgb}{0.000000,0.000000,0.000000}%
\pgfsetstrokecolor{currentstroke}%
\pgfsetdash{}{0pt}%
\pgfpathmoveto{\pgfqpoint{3.965073in}{4.012665in}}%
\pgfpathlineto{\pgfqpoint{3.978390in}{3.999057in}}%
\pgfpathlineto{\pgfqpoint{3.991708in}{3.985553in}}%
\pgfpathlineto{\pgfqpoint{4.005027in}{3.972152in}}%
\pgfpathlineto{\pgfqpoint{4.018347in}{3.958854in}}%
\pgfpathlineto{\pgfqpoint{4.026033in}{3.991951in}}%
\pgfpathlineto{\pgfqpoint{4.033720in}{4.025645in}}%
\pgfpathlineto{\pgfqpoint{4.041409in}{4.059945in}}%
\pgfpathlineto{\pgfqpoint{4.049100in}{4.094864in}}%
\pgfpathlineto{\pgfqpoint{4.035771in}{4.108726in}}%
\pgfpathlineto{\pgfqpoint{4.022443in}{4.122690in}}%
\pgfpathlineto{\pgfqpoint{4.009115in}{4.136758in}}%
\pgfpathlineto{\pgfqpoint{3.995788in}{4.150930in}}%
\pgfpathlineto{\pgfqpoint{3.988107in}{4.115438in}}%
\pgfpathlineto{\pgfqpoint{3.980428in}{4.080570in}}%
\pgfpathlineto{\pgfqpoint{3.972750in}{4.046316in}}%
\pgfpathlineto{\pgfqpoint{3.965073in}{4.012665in}}%
\pgfpathclose%
\pgfusepath{fill}%
\end{pgfscope}%
\begin{pgfscope}%
\pgfpathrectangle{\pgfqpoint{1.150000in}{0.150000in}}{\pgfqpoint{5.700000in}{5.700000in}}%
\pgfusepath{clip}%
\pgfsetbuttcap%
\pgfsetroundjoin%
\definecolor{currentfill}{rgb}{0.208623,0.367752,0.552675}%
\pgfsetfillcolor{currentfill}%
\pgfsetfillopacity{0.700000}%
\pgfsetlinewidth{0.000000pt}%
\definecolor{currentstroke}{rgb}{0.000000,0.000000,0.000000}%
\pgfsetstrokecolor{currentstroke}%
\pgfsetdash{}{0pt}%
\pgfpathmoveto{\pgfqpoint{3.169143in}{3.256110in}}%
\pgfpathlineto{\pgfqpoint{3.182419in}{3.243811in}}%
\pgfpathlineto{\pgfqpoint{3.195695in}{3.231642in}}%
\pgfpathlineto{\pgfqpoint{3.208971in}{3.219600in}}%
\pgfpathlineto{\pgfqpoint{3.222246in}{3.207685in}}%
\pgfpathlineto{\pgfqpoint{3.230092in}{3.224932in}}%
\pgfpathlineto{\pgfqpoint{3.237931in}{3.242434in}}%
\pgfpathlineto{\pgfqpoint{3.245763in}{3.260195in}}%
\pgfpathlineto{\pgfqpoint{3.253589in}{3.278223in}}%
\pgfpathlineto{\pgfqpoint{3.240316in}{3.290437in}}%
\pgfpathlineto{\pgfqpoint{3.227042in}{3.302779in}}%
\pgfpathlineto{\pgfqpoint{3.213768in}{3.315248in}}%
\pgfpathlineto{\pgfqpoint{3.200493in}{3.327847in}}%
\pgfpathlineto{\pgfqpoint{3.192666in}{3.309512in}}%
\pgfpathlineto{\pgfqpoint{3.184832in}{3.291448in}}%
\pgfpathlineto{\pgfqpoint{3.176991in}{3.273649in}}%
\pgfpathlineto{\pgfqpoint{3.169143in}{3.256110in}}%
\pgfpathclose%
\pgfusepath{fill}%
\end{pgfscope}%
\begin{pgfscope}%
\pgfpathrectangle{\pgfqpoint{1.150000in}{0.150000in}}{\pgfqpoint{5.700000in}{5.700000in}}%
\pgfusepath{clip}%
\pgfsetbuttcap%
\pgfsetroundjoin%
\definecolor{currentfill}{rgb}{0.179019,0.433756,0.557430}%
\pgfsetfillcolor{currentfill}%
\pgfsetfillopacity{0.700000}%
\pgfsetlinewidth{0.000000pt}%
\definecolor{currentstroke}{rgb}{0.000000,0.000000,0.000000}%
\pgfsetstrokecolor{currentstroke}%
\pgfsetdash{}{0pt}%
\pgfpathmoveto{\pgfqpoint{4.001851in}{3.415055in}}%
\pgfpathlineto{\pgfqpoint{4.015170in}{3.404641in}}%
\pgfpathlineto{\pgfqpoint{4.028490in}{3.394322in}}%
\pgfpathlineto{\pgfqpoint{4.041813in}{3.384096in}}%
\pgfpathlineto{\pgfqpoint{4.055139in}{3.373964in}}%
\pgfpathlineto{\pgfqpoint{4.062836in}{3.397551in}}%
\pgfpathlineto{\pgfqpoint{4.070531in}{3.421559in}}%
\pgfpathlineto{\pgfqpoint{4.078226in}{3.445997in}}%
\pgfpathlineto{\pgfqpoint{4.085920in}{3.470873in}}%
\pgfpathlineto{\pgfqpoint{4.072594in}{3.481468in}}%
\pgfpathlineto{\pgfqpoint{4.059270in}{3.492157in}}%
\pgfpathlineto{\pgfqpoint{4.045948in}{3.502939in}}%
\pgfpathlineto{\pgfqpoint{4.032628in}{3.513817in}}%
\pgfpathlineto{\pgfqpoint{4.024936in}{3.488469in}}%
\pgfpathlineto{\pgfqpoint{4.017242in}{3.463565in}}%
\pgfpathlineto{\pgfqpoint{4.009547in}{3.439097in}}%
\pgfpathlineto{\pgfqpoint{4.001851in}{3.415055in}}%
\pgfpathclose%
\pgfusepath{fill}%
\end{pgfscope}%
\begin{pgfscope}%
\pgfpathrectangle{\pgfqpoint{1.150000in}{0.150000in}}{\pgfqpoint{5.700000in}{5.700000in}}%
\pgfusepath{clip}%
\pgfsetbuttcap%
\pgfsetroundjoin%
\definecolor{currentfill}{rgb}{0.218130,0.347432,0.550038}%
\pgfsetfillcolor{currentfill}%
\pgfsetfillopacity{0.700000}%
\pgfsetlinewidth{0.000000pt}%
\definecolor{currentstroke}{rgb}{0.000000,0.000000,0.000000}%
\pgfsetstrokecolor{currentstroke}%
\pgfsetdash{}{0pt}%
\pgfpathmoveto{\pgfqpoint{3.581419in}{3.200386in}}%
\pgfpathlineto{\pgfqpoint{3.594701in}{3.189958in}}%
\pgfpathlineto{\pgfqpoint{3.607985in}{3.179637in}}%
\pgfpathlineto{\pgfqpoint{3.621270in}{3.169421in}}%
\pgfpathlineto{\pgfqpoint{3.634556in}{3.159312in}}%
\pgfpathlineto{\pgfqpoint{3.642319in}{3.177885in}}%
\pgfpathlineto{\pgfqpoint{3.650077in}{3.196754in}}%
\pgfpathlineto{\pgfqpoint{3.657831in}{3.215926in}}%
\pgfpathlineto{\pgfqpoint{3.665580in}{3.235406in}}%
\pgfpathlineto{\pgfqpoint{3.652295in}{3.245874in}}%
\pgfpathlineto{\pgfqpoint{3.639011in}{3.256448in}}%
\pgfpathlineto{\pgfqpoint{3.625729in}{3.267129in}}%
\pgfpathlineto{\pgfqpoint{3.612448in}{3.277916in}}%
\pgfpathlineto{\pgfqpoint{3.604698in}{3.258069in}}%
\pgfpathlineto{\pgfqpoint{3.596943in}{3.238536in}}%
\pgfpathlineto{\pgfqpoint{3.589184in}{3.219311in}}%
\pgfpathlineto{\pgfqpoint{3.581419in}{3.200386in}}%
\pgfpathclose%
\pgfusepath{fill}%
\end{pgfscope}%
\begin{pgfscope}%
\pgfpathrectangle{\pgfqpoint{1.150000in}{0.150000in}}{\pgfqpoint{5.700000in}{5.700000in}}%
\pgfusepath{clip}%
\pgfsetbuttcap%
\pgfsetroundjoin%
\definecolor{currentfill}{rgb}{0.477504,0.821444,0.318195}%
\pgfsetfillcolor{currentfill}%
\pgfsetfillopacity{0.700000}%
\pgfsetlinewidth{0.000000pt}%
\definecolor{currentstroke}{rgb}{0.000000,0.000000,0.000000}%
\pgfsetstrokecolor{currentstroke}%
\pgfsetdash{}{0pt}%
\pgfpathmoveto{\pgfqpoint{3.538777in}{4.508045in}}%
\pgfpathlineto{\pgfqpoint{3.552118in}{4.490645in}}%
\pgfpathlineto{\pgfqpoint{3.565456in}{4.473380in}}%
\pgfpathlineto{\pgfqpoint{3.578793in}{4.456249in}}%
\pgfpathlineto{\pgfqpoint{3.592127in}{4.439250in}}%
\pgfpathlineto{\pgfqpoint{3.599722in}{4.476851in}}%
\pgfpathlineto{\pgfqpoint{3.607314in}{4.515102in}}%
\pgfpathlineto{\pgfqpoint{3.614904in}{4.554013in}}%
\pgfpathlineto{\pgfqpoint{3.622492in}{4.593597in}}%
\pgfpathlineto{\pgfqpoint{3.609142in}{4.611182in}}%
\pgfpathlineto{\pgfqpoint{3.595790in}{4.628900in}}%
\pgfpathlineto{\pgfqpoint{3.582435in}{4.646753in}}%
\pgfpathlineto{\pgfqpoint{3.569078in}{4.664741in}}%
\pgfpathlineto{\pgfqpoint{3.561507in}{4.624559in}}%
\pgfpathlineto{\pgfqpoint{3.553933in}{4.585057in}}%
\pgfpathlineto{\pgfqpoint{3.546356in}{4.546223in}}%
\pgfpathlineto{\pgfqpoint{3.538777in}{4.508045in}}%
\pgfpathclose%
\pgfusepath{fill}%
\end{pgfscope}%
\begin{pgfscope}%
\pgfpathrectangle{\pgfqpoint{1.150000in}{0.150000in}}{\pgfqpoint{5.700000in}{5.700000in}}%
\pgfusepath{clip}%
\pgfsetbuttcap%
\pgfsetroundjoin%
\definecolor{currentfill}{rgb}{0.126326,0.644107,0.525311}%
\pgfsetfillcolor{currentfill}%
\pgfsetfillopacity{0.700000}%
\pgfsetlinewidth{0.000000pt}%
\definecolor{currentstroke}{rgb}{0.000000,0.000000,0.000000}%
\pgfsetstrokecolor{currentstroke}%
\pgfsetdash{}{0pt}%
\pgfpathmoveto{\pgfqpoint{4.018347in}{3.958854in}}%
\pgfpathlineto{\pgfqpoint{4.031668in}{3.945657in}}%
\pgfpathlineto{\pgfqpoint{4.044990in}{3.932562in}}%
\pgfpathlineto{\pgfqpoint{4.058313in}{3.919567in}}%
\pgfpathlineto{\pgfqpoint{4.071637in}{3.906671in}}%
\pgfpathlineto{\pgfqpoint{4.079330in}{3.939217in}}%
\pgfpathlineto{\pgfqpoint{4.087026in}{3.972352in}}%
\pgfpathlineto{\pgfqpoint{4.094723in}{4.006088in}}%
\pgfpathlineto{\pgfqpoint{4.102422in}{4.040436in}}%
\pgfpathlineto{\pgfqpoint{4.089090in}{4.053892in}}%
\pgfpathlineto{\pgfqpoint{4.075759in}{4.067448in}}%
\pgfpathlineto{\pgfqpoint{4.062429in}{4.081105in}}%
\pgfpathlineto{\pgfqpoint{4.049100in}{4.094864in}}%
\pgfpathlineto{\pgfqpoint{4.041409in}{4.059945in}}%
\pgfpathlineto{\pgfqpoint{4.033720in}{4.025645in}}%
\pgfpathlineto{\pgfqpoint{4.026033in}{3.991951in}}%
\pgfpathlineto{\pgfqpoint{4.018347in}{3.958854in}}%
\pgfpathclose%
\pgfusepath{fill}%
\end{pgfscope}%
\begin{pgfscope}%
\pgfpathrectangle{\pgfqpoint{1.150000in}{0.150000in}}{\pgfqpoint{5.700000in}{5.700000in}}%
\pgfusepath{clip}%
\pgfsetbuttcap%
\pgfsetroundjoin%
\definecolor{currentfill}{rgb}{0.129933,0.559582,0.551864}%
\pgfsetfillcolor{currentfill}%
\pgfsetfillopacity{0.700000}%
\pgfsetlinewidth{0.000000pt}%
\definecolor{currentstroke}{rgb}{0.000000,0.000000,0.000000}%
\pgfsetstrokecolor{currentstroke}%
\pgfsetdash{}{0pt}%
\pgfpathmoveto{\pgfqpoint{4.094163in}{3.733709in}}%
\pgfpathlineto{\pgfqpoint{4.107488in}{3.721836in}}%
\pgfpathlineto{\pgfqpoint{4.120816in}{3.710059in}}%
\pgfpathlineto{\pgfqpoint{4.134145in}{3.698376in}}%
\pgfpathlineto{\pgfqpoint{4.147477in}{3.686787in}}%
\pgfpathlineto{\pgfqpoint{4.155175in}{3.716033in}}%
\pgfpathlineto{\pgfqpoint{4.162875in}{3.745814in}}%
\pgfpathlineto{\pgfqpoint{4.170577in}{3.776139in}}%
\pgfpathlineto{\pgfqpoint{4.178281in}{3.807021in}}%
\pgfpathlineto{\pgfqpoint{4.164945in}{3.819141in}}%
\pgfpathlineto{\pgfqpoint{4.151610in}{3.831357in}}%
\pgfpathlineto{\pgfqpoint{4.138278in}{3.843667in}}%
\pgfpathlineto{\pgfqpoint{4.124947in}{3.856074in}}%
\pgfpathlineto{\pgfqpoint{4.117248in}{3.824651in}}%
\pgfpathlineto{\pgfqpoint{4.109552in}{3.793789in}}%
\pgfpathlineto{\pgfqpoint{4.101857in}{3.763479in}}%
\pgfpathlineto{\pgfqpoint{4.094163in}{3.733709in}}%
\pgfpathclose%
\pgfusepath{fill}%
\end{pgfscope}%
\begin{pgfscope}%
\pgfpathrectangle{\pgfqpoint{1.150000in}{0.150000in}}{\pgfqpoint{5.700000in}{5.700000in}}%
\pgfusepath{clip}%
\pgfsetbuttcap%
\pgfsetroundjoin%
\definecolor{currentfill}{rgb}{0.203063,0.379716,0.553925}%
\pgfsetfillcolor{currentfill}%
\pgfsetfillopacity{0.700000}%
\pgfsetlinewidth{0.000000pt}%
\definecolor{currentstroke}{rgb}{0.000000,0.000000,0.000000}%
\pgfsetstrokecolor{currentstroke}%
\pgfsetdash{}{0pt}%
\pgfpathmoveto{\pgfqpoint{3.886968in}{3.276233in}}%
\pgfpathlineto{\pgfqpoint{3.900278in}{3.266297in}}%
\pgfpathlineto{\pgfqpoint{3.913589in}{3.256458in}}%
\pgfpathlineto{\pgfqpoint{3.926903in}{3.246715in}}%
\pgfpathlineto{\pgfqpoint{3.940220in}{3.237067in}}%
\pgfpathlineto{\pgfqpoint{3.947932in}{3.257997in}}%
\pgfpathlineto{\pgfqpoint{3.955641in}{3.279288in}}%
\pgfpathlineto{\pgfqpoint{3.963347in}{3.300948in}}%
\pgfpathlineto{\pgfqpoint{3.971052in}{3.322984in}}%
\pgfpathlineto{\pgfqpoint{3.957736in}{3.333052in}}%
\pgfpathlineto{\pgfqpoint{3.944423in}{3.343215in}}%
\pgfpathlineto{\pgfqpoint{3.931112in}{3.353474in}}%
\pgfpathlineto{\pgfqpoint{3.917803in}{3.363830in}}%
\pgfpathlineto{\pgfqpoint{3.910098in}{3.341366in}}%
\pgfpathlineto{\pgfqpoint{3.902391in}{3.319284in}}%
\pgfpathlineto{\pgfqpoint{3.894681in}{3.297575in}}%
\pgfpathlineto{\pgfqpoint{3.886968in}{3.276233in}}%
\pgfpathclose%
\pgfusepath{fill}%
\end{pgfscope}%
\begin{pgfscope}%
\pgfpathrectangle{\pgfqpoint{1.150000in}{0.150000in}}{\pgfqpoint{5.700000in}{5.700000in}}%
\pgfusepath{clip}%
\pgfsetbuttcap%
\pgfsetroundjoin%
\definecolor{currentfill}{rgb}{0.210503,0.363727,0.552206}%
\pgfsetfillcolor{currentfill}%
\pgfsetfillopacity{0.700000}%
\pgfsetlinewidth{0.000000pt}%
\definecolor{currentstroke}{rgb}{0.000000,0.000000,0.000000}%
\pgfsetstrokecolor{currentstroke}%
\pgfsetdash{}{0pt}%
\pgfpathmoveto{\pgfqpoint{3.802870in}{3.233493in}}%
\pgfpathlineto{\pgfqpoint{3.816171in}{3.223565in}}%
\pgfpathlineto{\pgfqpoint{3.829474in}{3.213736in}}%
\pgfpathlineto{\pgfqpoint{3.842780in}{3.204004in}}%
\pgfpathlineto{\pgfqpoint{3.856088in}{3.194371in}}%
\pgfpathlineto{\pgfqpoint{3.863813in}{3.214325in}}%
\pgfpathlineto{\pgfqpoint{3.871535in}{3.234616in}}%
\pgfpathlineto{\pgfqpoint{3.879253in}{3.255249in}}%
\pgfpathlineto{\pgfqpoint{3.886968in}{3.276233in}}%
\pgfpathlineto{\pgfqpoint{3.873661in}{3.286266in}}%
\pgfpathlineto{\pgfqpoint{3.860357in}{3.296396in}}%
\pgfpathlineto{\pgfqpoint{3.847054in}{3.306625in}}%
\pgfpathlineto{\pgfqpoint{3.833754in}{3.316953in}}%
\pgfpathlineto{\pgfqpoint{3.826038in}{3.295562in}}%
\pgfpathlineto{\pgfqpoint{3.818319in}{3.274526in}}%
\pgfpathlineto{\pgfqpoint{3.810596in}{3.253839in}}%
\pgfpathlineto{\pgfqpoint{3.802870in}{3.233493in}}%
\pgfpathclose%
\pgfusepath{fill}%
\end{pgfscope}%
\begin{pgfscope}%
\pgfpathrectangle{\pgfqpoint{1.150000in}{0.150000in}}{\pgfqpoint{5.700000in}{5.700000in}}%
\pgfusepath{clip}%
\pgfsetbuttcap%
\pgfsetroundjoin%
\definecolor{currentfill}{rgb}{0.221989,0.339161,0.548752}%
\pgfsetfillcolor{currentfill}%
\pgfsetfillopacity{0.700000}%
\pgfsetlinewidth{0.000000pt}%
\definecolor{currentstroke}{rgb}{0.000000,0.000000,0.000000}%
\pgfsetstrokecolor{currentstroke}%
\pgfsetdash{}{0pt}%
\pgfpathmoveto{\pgfqpoint{3.359769in}{3.184955in}}%
\pgfpathlineto{\pgfqpoint{3.373043in}{3.173837in}}%
\pgfpathlineto{\pgfqpoint{3.386317in}{3.162836in}}%
\pgfpathlineto{\pgfqpoint{3.399592in}{3.151951in}}%
\pgfpathlineto{\pgfqpoint{3.412868in}{3.141181in}}%
\pgfpathlineto{\pgfqpoint{3.420681in}{3.158546in}}%
\pgfpathlineto{\pgfqpoint{3.428488in}{3.176172in}}%
\pgfpathlineto{\pgfqpoint{3.436289in}{3.194066in}}%
\pgfpathlineto{\pgfqpoint{3.444084in}{3.212233in}}%
\pgfpathlineto{\pgfqpoint{3.430810in}{3.223322in}}%
\pgfpathlineto{\pgfqpoint{3.417537in}{3.234526in}}%
\pgfpathlineto{\pgfqpoint{3.404265in}{3.245845in}}%
\pgfpathlineto{\pgfqpoint{3.390993in}{3.257282in}}%
\pgfpathlineto{\pgfqpoint{3.383196in}{3.238788in}}%
\pgfpathlineto{\pgfqpoint{3.375393in}{3.220573in}}%
\pgfpathlineto{\pgfqpoint{3.367584in}{3.202631in}}%
\pgfpathlineto{\pgfqpoint{3.359769in}{3.184955in}}%
\pgfpathclose%
\pgfusepath{fill}%
\end{pgfscope}%
\begin{pgfscope}%
\pgfpathrectangle{\pgfqpoint{1.150000in}{0.150000in}}{\pgfqpoint{5.700000in}{5.700000in}}%
\pgfusepath{clip}%
\pgfsetbuttcap%
\pgfsetroundjoin%
\definecolor{currentfill}{rgb}{0.223925,0.334994,0.548053}%
\pgfsetfillcolor{currentfill}%
\pgfsetfillopacity{0.700000}%
\pgfsetlinewidth{0.000000pt}%
\definecolor{currentstroke}{rgb}{0.000000,0.000000,0.000000}%
\pgfsetstrokecolor{currentstroke}%
\pgfsetdash{}{0pt}%
\pgfpathmoveto{\pgfqpoint{3.497187in}{3.169014in}}%
\pgfpathlineto{\pgfqpoint{3.510466in}{3.158488in}}%
\pgfpathlineto{\pgfqpoint{3.523745in}{3.148073in}}%
\pgfpathlineto{\pgfqpoint{3.537027in}{3.137767in}}%
\pgfpathlineto{\pgfqpoint{3.550309in}{3.127569in}}%
\pgfpathlineto{\pgfqpoint{3.558095in}{3.145354in}}%
\pgfpathlineto{\pgfqpoint{3.565875in}{3.163414in}}%
\pgfpathlineto{\pgfqpoint{3.573649in}{3.181756in}}%
\pgfpathlineto{\pgfqpoint{3.581419in}{3.200386in}}%
\pgfpathlineto{\pgfqpoint{3.568139in}{3.210923in}}%
\pgfpathlineto{\pgfqpoint{3.554859in}{3.221567in}}%
\pgfpathlineto{\pgfqpoint{3.541581in}{3.232321in}}%
\pgfpathlineto{\pgfqpoint{3.528305in}{3.243186in}}%
\pgfpathlineto{\pgfqpoint{3.520533in}{3.224209in}}%
\pgfpathlineto{\pgfqpoint{3.512757in}{3.205526in}}%
\pgfpathlineto{\pgfqpoint{3.504975in}{3.187130in}}%
\pgfpathlineto{\pgfqpoint{3.497187in}{3.169014in}}%
\pgfpathclose%
\pgfusepath{fill}%
\end{pgfscope}%
\begin{pgfscope}%
\pgfpathrectangle{\pgfqpoint{1.150000in}{0.150000in}}{\pgfqpoint{5.700000in}{5.700000in}}%
\pgfusepath{clip}%
\pgfsetbuttcap%
\pgfsetroundjoin%
\definecolor{currentfill}{rgb}{0.194100,0.399323,0.555565}%
\pgfsetfillcolor{currentfill}%
\pgfsetfillopacity{0.700000}%
\pgfsetlinewidth{0.000000pt}%
\definecolor{currentstroke}{rgb}{0.000000,0.000000,0.000000}%
\pgfsetstrokecolor{currentstroke}%
\pgfsetdash{}{0pt}%
\pgfpathmoveto{\pgfqpoint{3.971052in}{3.322984in}}%
\pgfpathlineto{\pgfqpoint{3.984370in}{3.313011in}}%
\pgfpathlineto{\pgfqpoint{3.997691in}{3.303132in}}%
\pgfpathlineto{\pgfqpoint{4.011015in}{3.293348in}}%
\pgfpathlineto{\pgfqpoint{4.024341in}{3.283656in}}%
\pgfpathlineto{\pgfqpoint{4.032043in}{3.305644in}}%
\pgfpathlineto{\pgfqpoint{4.039743in}{3.328019in}}%
\pgfpathlineto{\pgfqpoint{4.047442in}{3.350789in}}%
\pgfpathlineto{\pgfqpoint{4.055139in}{3.373964in}}%
\pgfpathlineto{\pgfqpoint{4.041813in}{3.384096in}}%
\pgfpathlineto{\pgfqpoint{4.028490in}{3.394322in}}%
\pgfpathlineto{\pgfqpoint{4.015170in}{3.404641in}}%
\pgfpathlineto{\pgfqpoint{4.001851in}{3.415055in}}%
\pgfpathlineto{\pgfqpoint{3.994154in}{3.391432in}}%
\pgfpathlineto{\pgfqpoint{3.986455in}{3.368217in}}%
\pgfpathlineto{\pgfqpoint{3.978754in}{3.345404in}}%
\pgfpathlineto{\pgfqpoint{3.971052in}{3.322984in}}%
\pgfpathclose%
\pgfusepath{fill}%
\end{pgfscope}%
\begin{pgfscope}%
\pgfpathrectangle{\pgfqpoint{1.150000in}{0.150000in}}{\pgfqpoint{5.700000in}{5.700000in}}%
\pgfusepath{clip}%
\pgfsetbuttcap%
\pgfsetroundjoin%
\definecolor{currentfill}{rgb}{0.218130,0.347432,0.550038}%
\pgfsetfillcolor{currentfill}%
\pgfsetfillopacity{0.700000}%
\pgfsetlinewidth{0.000000pt}%
\definecolor{currentstroke}{rgb}{0.000000,0.000000,0.000000}%
\pgfsetstrokecolor{currentstroke}%
\pgfsetdash{}{0pt}%
\pgfpathmoveto{\pgfqpoint{3.718738in}{3.194574in}}%
\pgfpathlineto{\pgfqpoint{3.732032in}{3.184623in}}%
\pgfpathlineto{\pgfqpoint{3.745328in}{3.174773in}}%
\pgfpathlineto{\pgfqpoint{3.758627in}{3.165024in}}%
\pgfpathlineto{\pgfqpoint{3.771928in}{3.155375in}}%
\pgfpathlineto{\pgfqpoint{3.779669in}{3.174429in}}%
\pgfpathlineto{\pgfqpoint{3.787406in}{3.193795in}}%
\pgfpathlineto{\pgfqpoint{3.795140in}{3.213481in}}%
\pgfpathlineto{\pgfqpoint{3.802870in}{3.233493in}}%
\pgfpathlineto{\pgfqpoint{3.789571in}{3.243521in}}%
\pgfpathlineto{\pgfqpoint{3.776274in}{3.253649in}}%
\pgfpathlineto{\pgfqpoint{3.762979in}{3.263878in}}%
\pgfpathlineto{\pgfqpoint{3.749686in}{3.274208in}}%
\pgfpathlineto{\pgfqpoint{3.741955in}{3.253809in}}%
\pgfpathlineto{\pgfqpoint{3.734220in}{3.233742in}}%
\pgfpathlineto{\pgfqpoint{3.726481in}{3.213999in}}%
\pgfpathlineto{\pgfqpoint{3.718738in}{3.194574in}}%
\pgfpathclose%
\pgfusepath{fill}%
\end{pgfscope}%
\begin{pgfscope}%
\pgfpathrectangle{\pgfqpoint{1.150000in}{0.150000in}}{\pgfqpoint{5.700000in}{5.700000in}}%
\pgfusepath{clip}%
\pgfsetbuttcap%
\pgfsetroundjoin%
\definecolor{currentfill}{rgb}{0.168126,0.459988,0.558082}%
\pgfsetfillcolor{currentfill}%
\pgfsetfillopacity{0.700000}%
\pgfsetlinewidth{0.000000pt}%
\definecolor{currentstroke}{rgb}{0.000000,0.000000,0.000000}%
\pgfsetstrokecolor{currentstroke}%
\pgfsetdash{}{0pt}%
\pgfpathmoveto{\pgfqpoint{4.085920in}{3.470873in}}%
\pgfpathlineto{\pgfqpoint{4.099249in}{3.460371in}}%
\pgfpathlineto{\pgfqpoint{4.112580in}{3.449962in}}%
\pgfpathlineto{\pgfqpoint{4.125914in}{3.439645in}}%
\pgfpathlineto{\pgfqpoint{4.139250in}{3.429419in}}%
\pgfpathlineto{\pgfqpoint{4.146945in}{3.454267in}}%
\pgfpathlineto{\pgfqpoint{4.154640in}{3.479567in}}%
\pgfpathlineto{\pgfqpoint{4.162335in}{3.505327in}}%
\pgfpathlineto{\pgfqpoint{4.170031in}{3.531556in}}%
\pgfpathlineto{\pgfqpoint{4.156693in}{3.542266in}}%
\pgfpathlineto{\pgfqpoint{4.143358in}{3.553068in}}%
\pgfpathlineto{\pgfqpoint{4.130025in}{3.563962in}}%
\pgfpathlineto{\pgfqpoint{4.116695in}{3.574949in}}%
\pgfpathlineto{\pgfqpoint{4.109001in}{3.548226in}}%
\pgfpathlineto{\pgfqpoint{4.101307in}{3.521979in}}%
\pgfpathlineto{\pgfqpoint{4.093614in}{3.496198in}}%
\pgfpathlineto{\pgfqpoint{4.085920in}{3.470873in}}%
\pgfpathclose%
\pgfusepath{fill}%
\end{pgfscope}%
\begin{pgfscope}%
\pgfpathrectangle{\pgfqpoint{1.150000in}{0.150000in}}{\pgfqpoint{5.700000in}{5.700000in}}%
\pgfusepath{clip}%
\pgfsetbuttcap%
\pgfsetroundjoin%
\definecolor{currentfill}{rgb}{0.153364,0.497000,0.557724}%
\pgfsetfillcolor{currentfill}%
\pgfsetfillopacity{0.700000}%
\pgfsetlinewidth{0.000000pt}%
\definecolor{currentstroke}{rgb}{0.000000,0.000000,0.000000}%
\pgfsetstrokecolor{currentstroke}%
\pgfsetdash{}{0pt}%
\pgfpathmoveto{\pgfqpoint{4.116695in}{3.574949in}}%
\pgfpathlineto{\pgfqpoint{4.130025in}{3.563962in}}%
\pgfpathlineto{\pgfqpoint{4.143358in}{3.553068in}}%
\pgfpathlineto{\pgfqpoint{4.156693in}{3.542266in}}%
\pgfpathlineto{\pgfqpoint{4.170031in}{3.531556in}}%
\pgfpathlineto{\pgfqpoint{4.177727in}{3.558265in}}%
\pgfpathlineto{\pgfqpoint{4.185425in}{3.585463in}}%
\pgfpathlineto{\pgfqpoint{4.193123in}{3.613160in}}%
\pgfpathlineto{\pgfqpoint{4.200823in}{3.641365in}}%
\pgfpathlineto{\pgfqpoint{4.187484in}{3.652582in}}%
\pgfpathlineto{\pgfqpoint{4.174146in}{3.663891in}}%
\pgfpathlineto{\pgfqpoint{4.160810in}{3.675293in}}%
\pgfpathlineto{\pgfqpoint{4.147477in}{3.686787in}}%
\pgfpathlineto{\pgfqpoint{4.139780in}{3.658066in}}%
\pgfpathlineto{\pgfqpoint{4.132084in}{3.629859in}}%
\pgfpathlineto{\pgfqpoint{4.124389in}{3.602156in}}%
\pgfpathlineto{\pgfqpoint{4.116695in}{3.574949in}}%
\pgfpathclose%
\pgfusepath{fill}%
\end{pgfscope}%
\begin{pgfscope}%
\pgfpathrectangle{\pgfqpoint{1.150000in}{0.150000in}}{\pgfqpoint{5.700000in}{5.700000in}}%
\pgfusepath{clip}%
\pgfsetbuttcap%
\pgfsetroundjoin%
\definecolor{currentfill}{rgb}{0.218130,0.347432,0.550038}%
\pgfsetfillcolor{currentfill}%
\pgfsetfillopacity{0.700000}%
\pgfsetlinewidth{0.000000pt}%
\definecolor{currentstroke}{rgb}{0.000000,0.000000,0.000000}%
\pgfsetstrokecolor{currentstroke}%
\pgfsetdash{}{0pt}%
\pgfpathmoveto{\pgfqpoint{3.222246in}{3.207685in}}%
\pgfpathlineto{\pgfqpoint{3.235520in}{3.195896in}}%
\pgfpathlineto{\pgfqpoint{3.248795in}{3.184232in}}%
\pgfpathlineto{\pgfqpoint{3.262070in}{3.172691in}}%
\pgfpathlineto{\pgfqpoint{3.275344in}{3.161273in}}%
\pgfpathlineto{\pgfqpoint{3.283188in}{3.178229in}}%
\pgfpathlineto{\pgfqpoint{3.291024in}{3.195434in}}%
\pgfpathlineto{\pgfqpoint{3.298855in}{3.212894in}}%
\pgfpathlineto{\pgfqpoint{3.306678in}{3.230615in}}%
\pgfpathlineto{\pgfqpoint{3.293406in}{3.242332in}}%
\pgfpathlineto{\pgfqpoint{3.280134in}{3.254171in}}%
\pgfpathlineto{\pgfqpoint{3.266861in}{3.266135in}}%
\pgfpathlineto{\pgfqpoint{3.253589in}{3.278223in}}%
\pgfpathlineto{\pgfqpoint{3.245763in}{3.260195in}}%
\pgfpathlineto{\pgfqpoint{3.237931in}{3.242434in}}%
\pgfpathlineto{\pgfqpoint{3.230092in}{3.224932in}}%
\pgfpathlineto{\pgfqpoint{3.222246in}{3.207685in}}%
\pgfpathclose%
\pgfusepath{fill}%
\end{pgfscope}%
\begin{pgfscope}%
\pgfpathrectangle{\pgfqpoint{1.150000in}{0.150000in}}{\pgfqpoint{5.700000in}{5.700000in}}%
\pgfusepath{clip}%
\pgfsetbuttcap%
\pgfsetroundjoin%
\definecolor{currentfill}{rgb}{0.555484,0.840254,0.269281}%
\pgfsetfillcolor{currentfill}%
\pgfsetfillopacity{0.700000}%
\pgfsetlinewidth{0.000000pt}%
\definecolor{currentstroke}{rgb}{0.000000,0.000000,0.000000}%
\pgfsetstrokecolor{currentstroke}%
\pgfsetdash{}{0pt}%
\pgfpathmoveto{\pgfqpoint{3.485389in}{4.579016in}}%
\pgfpathlineto{\pgfqpoint{3.498740in}{4.561065in}}%
\pgfpathlineto{\pgfqpoint{3.512088in}{4.543254in}}%
\pgfpathlineto{\pgfqpoint{3.525433in}{4.525581in}}%
\pgfpathlineto{\pgfqpoint{3.538777in}{4.508045in}}%
\pgfpathlineto{\pgfqpoint{3.546356in}{4.546223in}}%
\pgfpathlineto{\pgfqpoint{3.553933in}{4.585057in}}%
\pgfpathlineto{\pgfqpoint{3.561507in}{4.624559in}}%
\pgfpathlineto{\pgfqpoint{3.569078in}{4.664741in}}%
\pgfpathlineto{\pgfqpoint{3.555719in}{4.682866in}}%
\pgfpathlineto{\pgfqpoint{3.542356in}{4.701129in}}%
\pgfpathlineto{\pgfqpoint{3.528991in}{4.719531in}}%
\pgfpathlineto{\pgfqpoint{3.515623in}{4.738075in}}%
\pgfpathlineto{\pgfqpoint{3.508070in}{4.697291in}}%
\pgfpathlineto{\pgfqpoint{3.500513in}{4.657195in}}%
\pgfpathlineto{\pgfqpoint{3.492952in}{4.617774in}}%
\pgfpathlineto{\pgfqpoint{3.485389in}{4.579016in}}%
\pgfpathclose%
\pgfusepath{fill}%
\end{pgfscope}%
\begin{pgfscope}%
\pgfpathrectangle{\pgfqpoint{1.150000in}{0.150000in}}{\pgfqpoint{5.700000in}{5.700000in}}%
\pgfusepath{clip}%
\pgfsetbuttcap%
\pgfsetroundjoin%
\definecolor{currentfill}{rgb}{0.203063,0.379716,0.553925}%
\pgfsetfillcolor{currentfill}%
\pgfsetfillopacity{0.700000}%
\pgfsetlinewidth{0.000000pt}%
\definecolor{currentstroke}{rgb}{0.000000,0.000000,0.000000}%
\pgfsetstrokecolor{currentstroke}%
\pgfsetdash{}{0pt}%
\pgfpathmoveto{\pgfqpoint{3.031409in}{3.289387in}}%
\pgfpathlineto{\pgfqpoint{3.044697in}{3.276293in}}%
\pgfpathlineto{\pgfqpoint{3.057983in}{3.263338in}}%
\pgfpathlineto{\pgfqpoint{3.071269in}{3.250520in}}%
\pgfpathlineto{\pgfqpoint{3.084552in}{3.237838in}}%
\pgfpathlineto{\pgfqpoint{3.092432in}{3.254662in}}%
\pgfpathlineto{\pgfqpoint{3.100305in}{3.271730in}}%
\pgfpathlineto{\pgfqpoint{3.108170in}{3.289048in}}%
\pgfpathlineto{\pgfqpoint{3.116028in}{3.306619in}}%
\pgfpathlineto{\pgfqpoint{3.102746in}{3.319581in}}%
\pgfpathlineto{\pgfqpoint{3.089464in}{3.332679in}}%
\pgfpathlineto{\pgfqpoint{3.076179in}{3.345915in}}%
\pgfpathlineto{\pgfqpoint{3.062894in}{3.359289in}}%
\pgfpathlineto{\pgfqpoint{3.055034in}{3.341430in}}%
\pgfpathlineto{\pgfqpoint{3.047167in}{3.323829in}}%
\pgfpathlineto{\pgfqpoint{3.039292in}{3.306484in}}%
\pgfpathlineto{\pgfqpoint{3.031409in}{3.289387in}}%
\pgfpathclose%
\pgfusepath{fill}%
\end{pgfscope}%
\begin{pgfscope}%
\pgfpathrectangle{\pgfqpoint{1.150000in}{0.150000in}}{\pgfqpoint{5.700000in}{5.700000in}}%
\pgfusepath{clip}%
\pgfsetbuttcap%
\pgfsetroundjoin%
\definecolor{currentfill}{rgb}{0.120638,0.625828,0.533488}%
\pgfsetfillcolor{currentfill}%
\pgfsetfillopacity{0.700000}%
\pgfsetlinewidth{0.000000pt}%
\definecolor{currentstroke}{rgb}{0.000000,0.000000,0.000000}%
\pgfsetstrokecolor{currentstroke}%
\pgfsetdash{}{0pt}%
\pgfpathmoveto{\pgfqpoint{4.071637in}{3.906671in}}%
\pgfpathlineto{\pgfqpoint{4.084962in}{3.893875in}}%
\pgfpathlineto{\pgfqpoint{4.098289in}{3.881177in}}%
\pgfpathlineto{\pgfqpoint{4.111617in}{3.868577in}}%
\pgfpathlineto{\pgfqpoint{4.124947in}{3.856074in}}%
\pgfpathlineto{\pgfqpoint{4.132647in}{3.888070in}}%
\pgfpathlineto{\pgfqpoint{4.140349in}{3.920648in}}%
\pgfpathlineto{\pgfqpoint{4.148054in}{3.953822in}}%
\pgfpathlineto{\pgfqpoint{4.155762in}{3.987600in}}%
\pgfpathlineto{\pgfqpoint{4.142425in}{4.000662in}}%
\pgfpathlineto{\pgfqpoint{4.129090in}{4.013822in}}%
\pgfpathlineto{\pgfqpoint{4.115756in}{4.027079in}}%
\pgfpathlineto{\pgfqpoint{4.102422in}{4.040436in}}%
\pgfpathlineto{\pgfqpoint{4.094723in}{4.006088in}}%
\pgfpathlineto{\pgfqpoint{4.087026in}{3.972352in}}%
\pgfpathlineto{\pgfqpoint{4.079330in}{3.939217in}}%
\pgfpathlineto{\pgfqpoint{4.071637in}{3.906671in}}%
\pgfpathclose%
\pgfusepath{fill}%
\end{pgfscope}%
\begin{pgfscope}%
\pgfpathrectangle{\pgfqpoint{1.150000in}{0.150000in}}{\pgfqpoint{5.700000in}{5.700000in}}%
\pgfusepath{clip}%
\pgfsetbuttcap%
\pgfsetroundjoin%
\definecolor{currentfill}{rgb}{0.183898,0.422383,0.556944}%
\pgfsetfillcolor{currentfill}%
\pgfsetfillopacity{0.700000}%
\pgfsetlinewidth{0.000000pt}%
\definecolor{currentstroke}{rgb}{0.000000,0.000000,0.000000}%
\pgfsetstrokecolor{currentstroke}%
\pgfsetdash{}{0pt}%
\pgfpathmoveto{\pgfqpoint{4.055139in}{3.373964in}}%
\pgfpathlineto{\pgfqpoint{4.068467in}{3.363924in}}%
\pgfpathlineto{\pgfqpoint{4.081798in}{3.353977in}}%
\pgfpathlineto{\pgfqpoint{4.095132in}{3.344122in}}%
\pgfpathlineto{\pgfqpoint{4.108469in}{3.334358in}}%
\pgfpathlineto{\pgfqpoint{4.116165in}{3.357491in}}%
\pgfpathlineto{\pgfqpoint{4.123860in}{3.381040in}}%
\pgfpathlineto{\pgfqpoint{4.131555in}{3.405013in}}%
\pgfpathlineto{\pgfqpoint{4.139250in}{3.429419in}}%
\pgfpathlineto{\pgfqpoint{4.125914in}{3.439645in}}%
\pgfpathlineto{\pgfqpoint{4.112580in}{3.449962in}}%
\pgfpathlineto{\pgfqpoint{4.099249in}{3.460371in}}%
\pgfpathlineto{\pgfqpoint{4.085920in}{3.470873in}}%
\pgfpathlineto{\pgfqpoint{4.078226in}{3.445997in}}%
\pgfpathlineto{\pgfqpoint{4.070531in}{3.421559in}}%
\pgfpathlineto{\pgfqpoint{4.062836in}{3.397551in}}%
\pgfpathlineto{\pgfqpoint{4.055139in}{3.373964in}}%
\pgfpathclose%
\pgfusepath{fill}%
\end{pgfscope}%
\begin{pgfscope}%
\pgfpathrectangle{\pgfqpoint{1.150000in}{0.150000in}}{\pgfqpoint{5.700000in}{5.700000in}}%
\pgfusepath{clip}%
\pgfsetbuttcap%
\pgfsetroundjoin%
\definecolor{currentfill}{rgb}{0.225863,0.330805,0.547314}%
\pgfsetfillcolor{currentfill}%
\pgfsetfillopacity{0.700000}%
\pgfsetlinewidth{0.000000pt}%
\definecolor{currentstroke}{rgb}{0.000000,0.000000,0.000000}%
\pgfsetstrokecolor{currentstroke}%
\pgfsetdash{}{0pt}%
\pgfpathmoveto{\pgfqpoint{3.634556in}{3.159312in}}%
\pgfpathlineto{\pgfqpoint{3.647845in}{3.149307in}}%
\pgfpathlineto{\pgfqpoint{3.661135in}{3.139406in}}%
\pgfpathlineto{\pgfqpoint{3.674428in}{3.129608in}}%
\pgfpathlineto{\pgfqpoint{3.687722in}{3.119913in}}%
\pgfpathlineto{\pgfqpoint{3.695483in}{3.138136in}}%
\pgfpathlineto{\pgfqpoint{3.703239in}{3.156649in}}%
\pgfpathlineto{\pgfqpoint{3.710990in}{3.175460in}}%
\pgfpathlineto{\pgfqpoint{3.718738in}{3.194574in}}%
\pgfpathlineto{\pgfqpoint{3.705446in}{3.204628in}}%
\pgfpathlineto{\pgfqpoint{3.692155in}{3.214783in}}%
\pgfpathlineto{\pgfqpoint{3.678867in}{3.225043in}}%
\pgfpathlineto{\pgfqpoint{3.665580in}{3.235406in}}%
\pgfpathlineto{\pgfqpoint{3.657831in}{3.215926in}}%
\pgfpathlineto{\pgfqpoint{3.650077in}{3.196754in}}%
\pgfpathlineto{\pgfqpoint{3.642319in}{3.177885in}}%
\pgfpathlineto{\pgfqpoint{3.634556in}{3.159312in}}%
\pgfpathclose%
\pgfusepath{fill}%
\end{pgfscope}%
\begin{pgfscope}%
\pgfpathrectangle{\pgfqpoint{1.150000in}{0.150000in}}{\pgfqpoint{5.700000in}{5.700000in}}%
\pgfusepath{clip}%
\pgfsetbuttcap%
\pgfsetroundjoin%
\definecolor{currentfill}{rgb}{0.136408,0.541173,0.554483}%
\pgfsetfillcolor{currentfill}%
\pgfsetfillopacity{0.700000}%
\pgfsetlinewidth{0.000000pt}%
\definecolor{currentstroke}{rgb}{0.000000,0.000000,0.000000}%
\pgfsetstrokecolor{currentstroke}%
\pgfsetdash{}{0pt}%
\pgfpathmoveto{\pgfqpoint{4.147477in}{3.686787in}}%
\pgfpathlineto{\pgfqpoint{4.160810in}{3.675293in}}%
\pgfpathlineto{\pgfqpoint{4.174146in}{3.663891in}}%
\pgfpathlineto{\pgfqpoint{4.187484in}{3.652582in}}%
\pgfpathlineto{\pgfqpoint{4.200823in}{3.641365in}}%
\pgfpathlineto{\pgfqpoint{4.208525in}{3.670088in}}%
\pgfpathlineto{\pgfqpoint{4.216229in}{3.699341in}}%
\pgfpathlineto{\pgfqpoint{4.223935in}{3.729132in}}%
\pgfpathlineto{\pgfqpoint{4.231644in}{3.759473in}}%
\pgfpathlineto{\pgfqpoint{4.218300in}{3.771221in}}%
\pgfpathlineto{\pgfqpoint{4.204958in}{3.783061in}}%
\pgfpathlineto{\pgfqpoint{4.191619in}{3.794994in}}%
\pgfpathlineto{\pgfqpoint{4.178281in}{3.807021in}}%
\pgfpathlineto{\pgfqpoint{4.170577in}{3.776139in}}%
\pgfpathlineto{\pgfqpoint{4.162875in}{3.745814in}}%
\pgfpathlineto{\pgfqpoint{4.155175in}{3.716033in}}%
\pgfpathlineto{\pgfqpoint{4.147477in}{3.686787in}}%
\pgfpathclose%
\pgfusepath{fill}%
\end{pgfscope}%
\begin{pgfscope}%
\pgfpathrectangle{\pgfqpoint{1.150000in}{0.150000in}}{\pgfqpoint{5.700000in}{5.700000in}}%
\pgfusepath{clip}%
\pgfsetbuttcap%
\pgfsetroundjoin%
\definecolor{currentfill}{rgb}{0.212395,0.359683,0.551710}%
\pgfsetfillcolor{currentfill}%
\pgfsetfillopacity{0.700000}%
\pgfsetlinewidth{0.000000pt}%
\definecolor{currentstroke}{rgb}{0.000000,0.000000,0.000000}%
\pgfsetstrokecolor{currentstroke}%
\pgfsetdash{}{0pt}%
\pgfpathmoveto{\pgfqpoint{3.084552in}{3.237838in}}%
\pgfpathlineto{\pgfqpoint{3.097835in}{3.225291in}}%
\pgfpathlineto{\pgfqpoint{3.111117in}{3.212877in}}%
\pgfpathlineto{\pgfqpoint{3.124398in}{3.200596in}}%
\pgfpathlineto{\pgfqpoint{3.137678in}{3.188446in}}%
\pgfpathlineto{\pgfqpoint{3.145555in}{3.204999in}}%
\pgfpathlineto{\pgfqpoint{3.153425in}{3.221790in}}%
\pgfpathlineto{\pgfqpoint{3.161287in}{3.238826in}}%
\pgfpathlineto{\pgfqpoint{3.169143in}{3.256110in}}%
\pgfpathlineto{\pgfqpoint{3.155865in}{3.268539in}}%
\pgfpathlineto{\pgfqpoint{3.142587in}{3.281099in}}%
\pgfpathlineto{\pgfqpoint{3.129308in}{3.293793in}}%
\pgfpathlineto{\pgfqpoint{3.116028in}{3.306619in}}%
\pgfpathlineto{\pgfqpoint{3.108170in}{3.289048in}}%
\pgfpathlineto{\pgfqpoint{3.100305in}{3.271730in}}%
\pgfpathlineto{\pgfqpoint{3.092432in}{3.254662in}}%
\pgfpathlineto{\pgfqpoint{3.084552in}{3.237838in}}%
\pgfpathclose%
\pgfusepath{fill}%
\end{pgfscope}%
\begin{pgfscope}%
\pgfpathrectangle{\pgfqpoint{1.150000in}{0.150000in}}{\pgfqpoint{5.700000in}{5.700000in}}%
\pgfusepath{clip}%
\pgfsetbuttcap%
\pgfsetroundjoin%
\definecolor{currentfill}{rgb}{0.636902,0.856542,0.216620}%
\pgfsetfillcolor{currentfill}%
\pgfsetfillopacity{0.700000}%
\pgfsetlinewidth{0.000000pt}%
\definecolor{currentstroke}{rgb}{0.000000,0.000000,0.000000}%
\pgfsetstrokecolor{currentstroke}%
\pgfsetdash{}{0pt}%
\pgfpathmoveto{\pgfqpoint{3.431956in}{4.652247in}}%
\pgfpathlineto{\pgfqpoint{3.445319in}{4.633723in}}%
\pgfpathlineto{\pgfqpoint{3.458678in}{4.615344in}}%
\pgfpathlineto{\pgfqpoint{3.472035in}{4.597109in}}%
\pgfpathlineto{\pgfqpoint{3.485389in}{4.579016in}}%
\pgfpathlineto{\pgfqpoint{3.492952in}{4.617774in}}%
\pgfpathlineto{\pgfqpoint{3.500513in}{4.657195in}}%
\pgfpathlineto{\pgfqpoint{3.508070in}{4.697291in}}%
\pgfpathlineto{\pgfqpoint{3.515623in}{4.738075in}}%
\pgfpathlineto{\pgfqpoint{3.502253in}{4.756760in}}%
\pgfpathlineto{\pgfqpoint{3.488879in}{4.775589in}}%
\pgfpathlineto{\pgfqpoint{3.475501in}{4.794563in}}%
\pgfpathlineto{\pgfqpoint{3.462121in}{4.813684in}}%
\pgfpathlineto{\pgfqpoint{3.454585in}{4.772295in}}%
\pgfpathlineto{\pgfqpoint{3.447046in}{4.731600in}}%
\pgfpathlineto{\pgfqpoint{3.439503in}{4.691588in}}%
\pgfpathlineto{\pgfqpoint{3.431956in}{4.652247in}}%
\pgfpathclose%
\pgfusepath{fill}%
\end{pgfscope}%
\begin{pgfscope}%
\pgfpathrectangle{\pgfqpoint{1.150000in}{0.150000in}}{\pgfqpoint{5.700000in}{5.700000in}}%
\pgfusepath{clip}%
\pgfsetbuttcap%
\pgfsetroundjoin%
\definecolor{currentfill}{rgb}{0.119512,0.607464,0.540218}%
\pgfsetfillcolor{currentfill}%
\pgfsetfillopacity{0.700000}%
\pgfsetlinewidth{0.000000pt}%
\definecolor{currentstroke}{rgb}{0.000000,0.000000,0.000000}%
\pgfsetstrokecolor{currentstroke}%
\pgfsetdash{}{0pt}%
\pgfpathmoveto{\pgfqpoint{4.124947in}{3.856074in}}%
\pgfpathlineto{\pgfqpoint{4.138278in}{3.843667in}}%
\pgfpathlineto{\pgfqpoint{4.151610in}{3.831357in}}%
\pgfpathlineto{\pgfqpoint{4.164945in}{3.819141in}}%
\pgfpathlineto{\pgfqpoint{4.178281in}{3.807021in}}%
\pgfpathlineto{\pgfqpoint{4.185987in}{3.838468in}}%
\pgfpathlineto{\pgfqpoint{4.193696in}{3.870493in}}%
\pgfpathlineto{\pgfqpoint{4.201407in}{3.903105in}}%
\pgfpathlineto{\pgfqpoint{4.209122in}{3.936317in}}%
\pgfpathlineto{\pgfqpoint{4.195780in}{3.948995in}}%
\pgfpathlineto{\pgfqpoint{4.182439in}{3.961767in}}%
\pgfpathlineto{\pgfqpoint{4.169100in}{3.974636in}}%
\pgfpathlineto{\pgfqpoint{4.155762in}{3.987600in}}%
\pgfpathlineto{\pgfqpoint{4.148054in}{3.953822in}}%
\pgfpathlineto{\pgfqpoint{4.140349in}{3.920648in}}%
\pgfpathlineto{\pgfqpoint{4.132647in}{3.888070in}}%
\pgfpathlineto{\pgfqpoint{4.124947in}{3.856074in}}%
\pgfpathclose%
\pgfusepath{fill}%
\end{pgfscope}%
\begin{pgfscope}%
\pgfpathrectangle{\pgfqpoint{1.150000in}{0.150000in}}{\pgfqpoint{5.700000in}{5.700000in}}%
\pgfusepath{clip}%
\pgfsetbuttcap%
\pgfsetroundjoin%
\definecolor{currentfill}{rgb}{0.229739,0.322361,0.545706}%
\pgfsetfillcolor{currentfill}%
\pgfsetfillopacity{0.700000}%
\pgfsetlinewidth{0.000000pt}%
\definecolor{currentstroke}{rgb}{0.000000,0.000000,0.000000}%
\pgfsetstrokecolor{currentstroke}%
\pgfsetdash{}{0pt}%
\pgfpathmoveto{\pgfqpoint{3.412868in}{3.141181in}}%
\pgfpathlineto{\pgfqpoint{3.426145in}{3.130525in}}%
\pgfpathlineto{\pgfqpoint{3.439422in}{3.119983in}}%
\pgfpathlineto{\pgfqpoint{3.452701in}{3.109553in}}%
\pgfpathlineto{\pgfqpoint{3.465981in}{3.099234in}}%
\pgfpathlineto{\pgfqpoint{3.473791in}{3.116289in}}%
\pgfpathlineto{\pgfqpoint{3.481595in}{3.133599in}}%
\pgfpathlineto{\pgfqpoint{3.489394in}{3.151173in}}%
\pgfpathlineto{\pgfqpoint{3.497187in}{3.169014in}}%
\pgfpathlineto{\pgfqpoint{3.483910in}{3.179651in}}%
\pgfpathlineto{\pgfqpoint{3.470634in}{3.190399in}}%
\pgfpathlineto{\pgfqpoint{3.457358in}{3.201259in}}%
\pgfpathlineto{\pgfqpoint{3.444084in}{3.212233in}}%
\pgfpathlineto{\pgfqpoint{3.436289in}{3.194066in}}%
\pgfpathlineto{\pgfqpoint{3.428488in}{3.176172in}}%
\pgfpathlineto{\pgfqpoint{3.420681in}{3.158546in}}%
\pgfpathlineto{\pgfqpoint{3.412868in}{3.141181in}}%
\pgfpathclose%
\pgfusepath{fill}%
\end{pgfscope}%
\begin{pgfscope}%
\pgfpathrectangle{\pgfqpoint{1.150000in}{0.150000in}}{\pgfqpoint{5.700000in}{5.700000in}}%
\pgfusepath{clip}%
\pgfsetbuttcap%
\pgfsetroundjoin%
\definecolor{currentfill}{rgb}{0.225863,0.330805,0.547314}%
\pgfsetfillcolor{currentfill}%
\pgfsetfillopacity{0.700000}%
\pgfsetlinewidth{0.000000pt}%
\definecolor{currentstroke}{rgb}{0.000000,0.000000,0.000000}%
\pgfsetstrokecolor{currentstroke}%
\pgfsetdash{}{0pt}%
\pgfpathmoveto{\pgfqpoint{3.275344in}{3.161273in}}%
\pgfpathlineto{\pgfqpoint{3.288619in}{3.149976in}}%
\pgfpathlineto{\pgfqpoint{3.301894in}{3.138801in}}%
\pgfpathlineto{\pgfqpoint{3.315170in}{3.127744in}}%
\pgfpathlineto{\pgfqpoint{3.328445in}{3.116807in}}%
\pgfpathlineto{\pgfqpoint{3.336286in}{3.133472in}}%
\pgfpathlineto{\pgfqpoint{3.344120in}{3.150381in}}%
\pgfpathlineto{\pgfqpoint{3.351948in}{3.167541in}}%
\pgfpathlineto{\pgfqpoint{3.359769in}{3.184955in}}%
\pgfpathlineto{\pgfqpoint{3.346496in}{3.196191in}}%
\pgfpathlineto{\pgfqpoint{3.333223in}{3.207545in}}%
\pgfpathlineto{\pgfqpoint{3.319951in}{3.219020in}}%
\pgfpathlineto{\pgfqpoint{3.306678in}{3.230615in}}%
\pgfpathlineto{\pgfqpoint{3.298855in}{3.212894in}}%
\pgfpathlineto{\pgfqpoint{3.291024in}{3.195434in}}%
\pgfpathlineto{\pgfqpoint{3.283188in}{3.178229in}}%
\pgfpathlineto{\pgfqpoint{3.275344in}{3.161273in}}%
\pgfpathclose%
\pgfusepath{fill}%
\end{pgfscope}%
\begin{pgfscope}%
\pgfpathrectangle{\pgfqpoint{1.150000in}{0.150000in}}{\pgfqpoint{5.700000in}{5.700000in}}%
\pgfusepath{clip}%
\pgfsetbuttcap%
\pgfsetroundjoin%
\definecolor{currentfill}{rgb}{0.335885,0.777018,0.402049}%
\pgfsetfillcolor{currentfill}%
\pgfsetfillopacity{0.700000}%
\pgfsetlinewidth{0.000000pt}%
\definecolor{currentstroke}{rgb}{0.000000,0.000000,0.000000}%
\pgfsetstrokecolor{currentstroke}%
\pgfsetdash{}{0pt}%
\pgfpathmoveto{\pgfqpoint{3.835873in}{4.329437in}}%
\pgfpathlineto{\pgfqpoint{3.849201in}{4.313948in}}%
\pgfpathlineto{\pgfqpoint{3.862528in}{4.298574in}}%
\pgfpathlineto{\pgfqpoint{3.875854in}{4.283313in}}%
\pgfpathlineto{\pgfqpoint{3.889180in}{4.268165in}}%
\pgfpathlineto{\pgfqpoint{3.896840in}{4.305462in}}%
\pgfpathlineto{\pgfqpoint{3.904500in}{4.343420in}}%
\pgfpathlineto{\pgfqpoint{3.912161in}{4.382052in}}%
\pgfpathlineto{\pgfqpoint{3.919824in}{4.421370in}}%
\pgfpathlineto{\pgfqpoint{3.906485in}{4.437116in}}%
\pgfpathlineto{\pgfqpoint{3.893144in}{4.452975in}}%
\pgfpathlineto{\pgfqpoint{3.879804in}{4.468948in}}%
\pgfpathlineto{\pgfqpoint{3.866462in}{4.485036in}}%
\pgfpathlineto{\pgfqpoint{3.858814in}{4.445109in}}%
\pgfpathlineto{\pgfqpoint{3.851166in}{4.405875in}}%
\pgfpathlineto{\pgfqpoint{3.843520in}{4.367322in}}%
\pgfpathlineto{\pgfqpoint{3.835873in}{4.329437in}}%
\pgfpathclose%
\pgfusepath{fill}%
\end{pgfscope}%
\begin{pgfscope}%
\pgfpathrectangle{\pgfqpoint{1.150000in}{0.150000in}}{\pgfqpoint{5.700000in}{5.700000in}}%
\pgfusepath{clip}%
\pgfsetbuttcap%
\pgfsetroundjoin%
\definecolor{currentfill}{rgb}{0.281477,0.755203,0.432552}%
\pgfsetfillcolor{currentfill}%
\pgfsetfillopacity{0.700000}%
\pgfsetlinewidth{0.000000pt}%
\definecolor{currentstroke}{rgb}{0.000000,0.000000,0.000000}%
\pgfsetstrokecolor{currentstroke}%
\pgfsetdash{}{0pt}%
\pgfpathmoveto{\pgfqpoint{3.889180in}{4.268165in}}%
\pgfpathlineto{\pgfqpoint{3.902506in}{4.253128in}}%
\pgfpathlineto{\pgfqpoint{3.915832in}{4.238203in}}%
\pgfpathlineto{\pgfqpoint{3.929158in}{4.223387in}}%
\pgfpathlineto{\pgfqpoint{3.942484in}{4.208680in}}%
\pgfpathlineto{\pgfqpoint{3.950155in}{4.245392in}}%
\pgfpathlineto{\pgfqpoint{3.957828in}{4.282758in}}%
\pgfpathlineto{\pgfqpoint{3.965502in}{4.320791in}}%
\pgfpathlineto{\pgfqpoint{3.973178in}{4.359503in}}%
\pgfpathlineto{\pgfqpoint{3.959840in}{4.374804in}}%
\pgfpathlineto{\pgfqpoint{3.946502in}{4.390215in}}%
\pgfpathlineto{\pgfqpoint{3.933163in}{4.405737in}}%
\pgfpathlineto{\pgfqpoint{3.919824in}{4.421370in}}%
\pgfpathlineto{\pgfqpoint{3.912161in}{4.382052in}}%
\pgfpathlineto{\pgfqpoint{3.904500in}{4.343420in}}%
\pgfpathlineto{\pgfqpoint{3.896840in}{4.305462in}}%
\pgfpathlineto{\pgfqpoint{3.889180in}{4.268165in}}%
\pgfpathclose%
\pgfusepath{fill}%
\end{pgfscope}%
\begin{pgfscope}%
\pgfpathrectangle{\pgfqpoint{1.150000in}{0.150000in}}{\pgfqpoint{5.700000in}{5.700000in}}%
\pgfusepath{clip}%
\pgfsetbuttcap%
\pgfsetroundjoin%
\definecolor{currentfill}{rgb}{0.208623,0.367752,0.552675}%
\pgfsetfillcolor{currentfill}%
\pgfsetfillopacity{0.700000}%
\pgfsetlinewidth{0.000000pt}%
\definecolor{currentstroke}{rgb}{0.000000,0.000000,0.000000}%
\pgfsetstrokecolor{currentstroke}%
\pgfsetdash{}{0pt}%
\pgfpathmoveto{\pgfqpoint{3.940220in}{3.237067in}}%
\pgfpathlineto{\pgfqpoint{3.953540in}{3.227514in}}%
\pgfpathlineto{\pgfqpoint{3.966862in}{3.218054in}}%
\pgfpathlineto{\pgfqpoint{3.980186in}{3.208689in}}%
\pgfpathlineto{\pgfqpoint{3.993514in}{3.199416in}}%
\pgfpathlineto{\pgfqpoint{4.001224in}{3.219935in}}%
\pgfpathlineto{\pgfqpoint{4.008932in}{3.240810in}}%
\pgfpathlineto{\pgfqpoint{4.016637in}{3.262047in}}%
\pgfpathlineto{\pgfqpoint{4.024341in}{3.283656in}}%
\pgfpathlineto{\pgfqpoint{4.011015in}{3.293348in}}%
\pgfpathlineto{\pgfqpoint{3.997691in}{3.303132in}}%
\pgfpathlineto{\pgfqpoint{3.984370in}{3.313011in}}%
\pgfpathlineto{\pgfqpoint{3.971052in}{3.322984in}}%
\pgfpathlineto{\pgfqpoint{3.963347in}{3.300948in}}%
\pgfpathlineto{\pgfqpoint{3.955641in}{3.279288in}}%
\pgfpathlineto{\pgfqpoint{3.947932in}{3.257997in}}%
\pgfpathlineto{\pgfqpoint{3.940220in}{3.237067in}}%
\pgfpathclose%
\pgfusepath{fill}%
\end{pgfscope}%
\begin{pgfscope}%
\pgfpathrectangle{\pgfqpoint{1.150000in}{0.150000in}}{\pgfqpoint{5.700000in}{5.700000in}}%
\pgfusepath{clip}%
\pgfsetbuttcap%
\pgfsetroundjoin%
\definecolor{currentfill}{rgb}{0.218130,0.347432,0.550038}%
\pgfsetfillcolor{currentfill}%
\pgfsetfillopacity{0.700000}%
\pgfsetlinewidth{0.000000pt}%
\definecolor{currentstroke}{rgb}{0.000000,0.000000,0.000000}%
\pgfsetstrokecolor{currentstroke}%
\pgfsetdash{}{0pt}%
\pgfpathmoveto{\pgfqpoint{3.856088in}{3.194371in}}%
\pgfpathlineto{\pgfqpoint{3.869399in}{3.184834in}}%
\pgfpathlineto{\pgfqpoint{3.882712in}{3.175393in}}%
\pgfpathlineto{\pgfqpoint{3.896028in}{3.166048in}}%
\pgfpathlineto{\pgfqpoint{3.909347in}{3.156799in}}%
\pgfpathlineto{\pgfqpoint{3.917069in}{3.176363in}}%
\pgfpathlineto{\pgfqpoint{3.924789in}{3.196257in}}%
\pgfpathlineto{\pgfqpoint{3.932506in}{3.216489in}}%
\pgfpathlineto{\pgfqpoint{3.940220in}{3.237067in}}%
\pgfpathlineto{\pgfqpoint{3.926903in}{3.246715in}}%
\pgfpathlineto{\pgfqpoint{3.913589in}{3.256458in}}%
\pgfpathlineto{\pgfqpoint{3.900278in}{3.266297in}}%
\pgfpathlineto{\pgfqpoint{3.886968in}{3.276233in}}%
\pgfpathlineto{\pgfqpoint{3.879253in}{3.255249in}}%
\pgfpathlineto{\pgfqpoint{3.871535in}{3.234616in}}%
\pgfpathlineto{\pgfqpoint{3.863813in}{3.214325in}}%
\pgfpathlineto{\pgfqpoint{3.856088in}{3.194371in}}%
\pgfpathclose%
\pgfusepath{fill}%
\end{pgfscope}%
\begin{pgfscope}%
\pgfpathrectangle{\pgfqpoint{1.150000in}{0.150000in}}{\pgfqpoint{5.700000in}{5.700000in}}%
\pgfusepath{clip}%
\pgfsetbuttcap%
\pgfsetroundjoin%
\definecolor{currentfill}{rgb}{0.157729,0.485932,0.558013}%
\pgfsetfillcolor{currentfill}%
\pgfsetfillopacity{0.700000}%
\pgfsetlinewidth{0.000000pt}%
\definecolor{currentstroke}{rgb}{0.000000,0.000000,0.000000}%
\pgfsetstrokecolor{currentstroke}%
\pgfsetdash{}{0pt}%
\pgfpathmoveto{\pgfqpoint{4.170031in}{3.531556in}}%
\pgfpathlineto{\pgfqpoint{4.183371in}{3.520938in}}%
\pgfpathlineto{\pgfqpoint{4.196713in}{3.510409in}}%
\pgfpathlineto{\pgfqpoint{4.210058in}{3.499971in}}%
\pgfpathlineto{\pgfqpoint{4.223406in}{3.489623in}}%
\pgfpathlineto{\pgfqpoint{4.231104in}{3.515834in}}%
\pgfpathlineto{\pgfqpoint{4.238804in}{3.542529in}}%
\pgfpathlineto{\pgfqpoint{4.246504in}{3.569716in}}%
\pgfpathlineto{\pgfqpoint{4.254207in}{3.597407in}}%
\pgfpathlineto{\pgfqpoint{4.240857in}{3.608261in}}%
\pgfpathlineto{\pgfqpoint{4.227510in}{3.619205in}}%
\pgfpathlineto{\pgfqpoint{4.214166in}{3.630239in}}%
\pgfpathlineto{\pgfqpoint{4.200823in}{3.641365in}}%
\pgfpathlineto{\pgfqpoint{4.193123in}{3.613160in}}%
\pgfpathlineto{\pgfqpoint{4.185425in}{3.585463in}}%
\pgfpathlineto{\pgfqpoint{4.177727in}{3.558265in}}%
\pgfpathlineto{\pgfqpoint{4.170031in}{3.531556in}}%
\pgfpathclose%
\pgfusepath{fill}%
\end{pgfscope}%
\begin{pgfscope}%
\pgfpathrectangle{\pgfqpoint{1.150000in}{0.150000in}}{\pgfqpoint{5.700000in}{5.700000in}}%
\pgfusepath{clip}%
\pgfsetbuttcap%
\pgfsetroundjoin%
\definecolor{currentfill}{rgb}{0.386433,0.794644,0.372886}%
\pgfsetfillcolor{currentfill}%
\pgfsetfillopacity{0.700000}%
\pgfsetlinewidth{0.000000pt}%
\definecolor{currentstroke}{rgb}{0.000000,0.000000,0.000000}%
\pgfsetstrokecolor{currentstroke}%
\pgfsetdash{}{0pt}%
\pgfpathmoveto{\pgfqpoint{3.782557in}{4.392553in}}%
\pgfpathlineto{\pgfqpoint{3.795887in}{4.376598in}}%
\pgfpathlineto{\pgfqpoint{3.809216in}{4.360761in}}%
\pgfpathlineto{\pgfqpoint{3.822545in}{4.345041in}}%
\pgfpathlineto{\pgfqpoint{3.835873in}{4.329437in}}%
\pgfpathlineto{\pgfqpoint{3.843520in}{4.367322in}}%
\pgfpathlineto{\pgfqpoint{3.851166in}{4.405875in}}%
\pgfpathlineto{\pgfqpoint{3.858814in}{4.445109in}}%
\pgfpathlineto{\pgfqpoint{3.866462in}{4.485036in}}%
\pgfpathlineto{\pgfqpoint{3.853120in}{4.501241in}}%
\pgfpathlineto{\pgfqpoint{3.839777in}{4.517562in}}%
\pgfpathlineto{\pgfqpoint{3.826433in}{4.534001in}}%
\pgfpathlineto{\pgfqpoint{3.813088in}{4.550559in}}%
\pgfpathlineto{\pgfqpoint{3.805455in}{4.510020in}}%
\pgfpathlineto{\pgfqpoint{3.797822in}{4.470181in}}%
\pgfpathlineto{\pgfqpoint{3.790189in}{4.431029in}}%
\pgfpathlineto{\pgfqpoint{3.782557in}{4.392553in}}%
\pgfpathclose%
\pgfusepath{fill}%
\end{pgfscope}%
\begin{pgfscope}%
\pgfpathrectangle{\pgfqpoint{1.150000in}{0.150000in}}{\pgfqpoint{5.700000in}{5.700000in}}%
\pgfusepath{clip}%
\pgfsetbuttcap%
\pgfsetroundjoin%
\definecolor{currentfill}{rgb}{0.239374,0.735588,0.455688}%
\pgfsetfillcolor{currentfill}%
\pgfsetfillopacity{0.700000}%
\pgfsetlinewidth{0.000000pt}%
\definecolor{currentstroke}{rgb}{0.000000,0.000000,0.000000}%
\pgfsetstrokecolor{currentstroke}%
\pgfsetdash{}{0pt}%
\pgfpathmoveto{\pgfqpoint{3.942484in}{4.208680in}}%
\pgfpathlineto{\pgfqpoint{3.955809in}{4.194082in}}%
\pgfpathlineto{\pgfqpoint{3.969135in}{4.179592in}}%
\pgfpathlineto{\pgfqpoint{3.982462in}{4.165208in}}%
\pgfpathlineto{\pgfqpoint{3.995788in}{4.150930in}}%
\pgfpathlineto{\pgfqpoint{4.003471in}{4.187059in}}%
\pgfpathlineto{\pgfqpoint{4.011155in}{4.223836in}}%
\pgfpathlineto{\pgfqpoint{4.018842in}{4.261273in}}%
\pgfpathlineto{\pgfqpoint{4.026531in}{4.299382in}}%
\pgfpathlineto{\pgfqpoint{4.013192in}{4.314252in}}%
\pgfpathlineto{\pgfqpoint{3.999854in}{4.329228in}}%
\pgfpathlineto{\pgfqpoint{3.986516in}{4.344311in}}%
\pgfpathlineto{\pgfqpoint{3.973178in}{4.359503in}}%
\pgfpathlineto{\pgfqpoint{3.965502in}{4.320791in}}%
\pgfpathlineto{\pgfqpoint{3.957828in}{4.282758in}}%
\pgfpathlineto{\pgfqpoint{3.950155in}{4.245392in}}%
\pgfpathlineto{\pgfqpoint{3.942484in}{4.208680in}}%
\pgfpathclose%
\pgfusepath{fill}%
\end{pgfscope}%
\begin{pgfscope}%
\pgfpathrectangle{\pgfqpoint{1.150000in}{0.150000in}}{\pgfqpoint{5.700000in}{5.700000in}}%
\pgfusepath{clip}%
\pgfsetbuttcap%
\pgfsetroundjoin%
\definecolor{currentfill}{rgb}{0.231674,0.318106,0.544834}%
\pgfsetfillcolor{currentfill}%
\pgfsetfillopacity{0.700000}%
\pgfsetlinewidth{0.000000pt}%
\definecolor{currentstroke}{rgb}{0.000000,0.000000,0.000000}%
\pgfsetstrokecolor{currentstroke}%
\pgfsetdash{}{0pt}%
\pgfpathmoveto{\pgfqpoint{3.550309in}{3.127569in}}%
\pgfpathlineto{\pgfqpoint{3.563593in}{3.117478in}}%
\pgfpathlineto{\pgfqpoint{3.576879in}{3.107495in}}%
\pgfpathlineto{\pgfqpoint{3.590166in}{3.097617in}}%
\pgfpathlineto{\pgfqpoint{3.603456in}{3.087845in}}%
\pgfpathlineto{\pgfqpoint{3.611238in}{3.105300in}}%
\pgfpathlineto{\pgfqpoint{3.619016in}{3.123026in}}%
\pgfpathlineto{\pgfqpoint{3.626789in}{3.141027in}}%
\pgfpathlineto{\pgfqpoint{3.634556in}{3.159312in}}%
\pgfpathlineto{\pgfqpoint{3.621270in}{3.169421in}}%
\pgfpathlineto{\pgfqpoint{3.607985in}{3.179637in}}%
\pgfpathlineto{\pgfqpoint{3.594701in}{3.189958in}}%
\pgfpathlineto{\pgfqpoint{3.581419in}{3.200386in}}%
\pgfpathlineto{\pgfqpoint{3.573649in}{3.181756in}}%
\pgfpathlineto{\pgfqpoint{3.565875in}{3.163414in}}%
\pgfpathlineto{\pgfqpoint{3.558095in}{3.145354in}}%
\pgfpathlineto{\pgfqpoint{3.550309in}{3.127569in}}%
\pgfpathclose%
\pgfusepath{fill}%
\end{pgfscope}%
\begin{pgfscope}%
\pgfpathrectangle{\pgfqpoint{1.150000in}{0.150000in}}{\pgfqpoint{5.700000in}{5.700000in}}%
\pgfusepath{clip}%
\pgfsetbuttcap%
\pgfsetroundjoin%
\definecolor{currentfill}{rgb}{0.174274,0.445044,0.557792}%
\pgfsetfillcolor{currentfill}%
\pgfsetfillopacity{0.700000}%
\pgfsetlinewidth{0.000000pt}%
\definecolor{currentstroke}{rgb}{0.000000,0.000000,0.000000}%
\pgfsetstrokecolor{currentstroke}%
\pgfsetdash{}{0pt}%
\pgfpathmoveto{\pgfqpoint{4.139250in}{3.429419in}}%
\pgfpathlineto{\pgfqpoint{4.152589in}{3.419284in}}%
\pgfpathlineto{\pgfqpoint{4.165931in}{3.409239in}}%
\pgfpathlineto{\pgfqpoint{4.179276in}{3.399285in}}%
\pgfpathlineto{\pgfqpoint{4.192623in}{3.389419in}}%
\pgfpathlineto{\pgfqpoint{4.200318in}{3.413793in}}%
\pgfpathlineto{\pgfqpoint{4.208014in}{3.438612in}}%
\pgfpathlineto{\pgfqpoint{4.215710in}{3.463885in}}%
\pgfpathlineto{\pgfqpoint{4.223406in}{3.489623in}}%
\pgfpathlineto{\pgfqpoint{4.210058in}{3.499971in}}%
\pgfpathlineto{\pgfqpoint{4.196713in}{3.510409in}}%
\pgfpathlineto{\pgfqpoint{4.183371in}{3.520938in}}%
\pgfpathlineto{\pgfqpoint{4.170031in}{3.531556in}}%
\pgfpathlineto{\pgfqpoint{4.162335in}{3.505327in}}%
\pgfpathlineto{\pgfqpoint{4.154640in}{3.479567in}}%
\pgfpathlineto{\pgfqpoint{4.146945in}{3.454267in}}%
\pgfpathlineto{\pgfqpoint{4.139250in}{3.429419in}}%
\pgfpathclose%
\pgfusepath{fill}%
\end{pgfscope}%
\begin{pgfscope}%
\pgfpathrectangle{\pgfqpoint{1.150000in}{0.150000in}}{\pgfqpoint{5.700000in}{5.700000in}}%
\pgfusepath{clip}%
\pgfsetbuttcap%
\pgfsetroundjoin%
\definecolor{currentfill}{rgb}{0.199430,0.387607,0.554642}%
\pgfsetfillcolor{currentfill}%
\pgfsetfillopacity{0.700000}%
\pgfsetlinewidth{0.000000pt}%
\definecolor{currentstroke}{rgb}{0.000000,0.000000,0.000000}%
\pgfsetstrokecolor{currentstroke}%
\pgfsetdash{}{0pt}%
\pgfpathmoveto{\pgfqpoint{4.024341in}{3.283656in}}%
\pgfpathlineto{\pgfqpoint{4.037670in}{3.274057in}}%
\pgfpathlineto{\pgfqpoint{4.051002in}{3.264550in}}%
\pgfpathlineto{\pgfqpoint{4.064336in}{3.255135in}}%
\pgfpathlineto{\pgfqpoint{4.077674in}{3.245810in}}%
\pgfpathlineto{\pgfqpoint{4.085374in}{3.267366in}}%
\pgfpathlineto{\pgfqpoint{4.093074in}{3.289304in}}%
\pgfpathlineto{\pgfqpoint{4.100772in}{3.311631in}}%
\pgfpathlineto{\pgfqpoint{4.108469in}{3.334358in}}%
\pgfpathlineto{\pgfqpoint{4.095132in}{3.344122in}}%
\pgfpathlineto{\pgfqpoint{4.081798in}{3.353977in}}%
\pgfpathlineto{\pgfqpoint{4.068467in}{3.363924in}}%
\pgfpathlineto{\pgfqpoint{4.055139in}{3.373964in}}%
\pgfpathlineto{\pgfqpoint{4.047442in}{3.350789in}}%
\pgfpathlineto{\pgfqpoint{4.039743in}{3.328019in}}%
\pgfpathlineto{\pgfqpoint{4.032043in}{3.305644in}}%
\pgfpathlineto{\pgfqpoint{4.024341in}{3.283656in}}%
\pgfpathclose%
\pgfusepath{fill}%
\end{pgfscope}%
\begin{pgfscope}%
\pgfpathrectangle{\pgfqpoint{1.150000in}{0.150000in}}{\pgfqpoint{5.700000in}{5.700000in}}%
\pgfusepath{clip}%
\pgfsetbuttcap%
\pgfsetroundjoin%
\definecolor{currentfill}{rgb}{0.208030,0.718701,0.472873}%
\pgfsetfillcolor{currentfill}%
\pgfsetfillopacity{0.700000}%
\pgfsetlinewidth{0.000000pt}%
\definecolor{currentstroke}{rgb}{0.000000,0.000000,0.000000}%
\pgfsetstrokecolor{currentstroke}%
\pgfsetdash{}{0pt}%
\pgfpathmoveto{\pgfqpoint{3.995788in}{4.150930in}}%
\pgfpathlineto{\pgfqpoint{4.009115in}{4.136758in}}%
\pgfpathlineto{\pgfqpoint{4.022443in}{4.122690in}}%
\pgfpathlineto{\pgfqpoint{4.035771in}{4.108726in}}%
\pgfpathlineto{\pgfqpoint{4.049100in}{4.094864in}}%
\pgfpathlineto{\pgfqpoint{4.056792in}{4.130413in}}%
\pgfpathlineto{\pgfqpoint{4.064487in}{4.166603in}}%
\pgfpathlineto{\pgfqpoint{4.072185in}{4.203447in}}%
\pgfpathlineto{\pgfqpoint{4.079886in}{4.240956in}}%
\pgfpathlineto{\pgfqpoint{4.066546in}{4.255406in}}%
\pgfpathlineto{\pgfqpoint{4.053207in}{4.269960in}}%
\pgfpathlineto{\pgfqpoint{4.039869in}{4.284618in}}%
\pgfpathlineto{\pgfqpoint{4.026531in}{4.299382in}}%
\pgfpathlineto{\pgfqpoint{4.018842in}{4.261273in}}%
\pgfpathlineto{\pgfqpoint{4.011155in}{4.223836in}}%
\pgfpathlineto{\pgfqpoint{4.003471in}{4.187059in}}%
\pgfpathlineto{\pgfqpoint{3.995788in}{4.150930in}}%
\pgfpathclose%
\pgfusepath{fill}%
\end{pgfscope}%
\begin{pgfscope}%
\pgfpathrectangle{\pgfqpoint{1.150000in}{0.150000in}}{\pgfqpoint{5.700000in}{5.700000in}}%
\pgfusepath{clip}%
\pgfsetbuttcap%
\pgfsetroundjoin%
\definecolor{currentfill}{rgb}{0.449368,0.813768,0.335384}%
\pgfsetfillcolor{currentfill}%
\pgfsetfillopacity{0.700000}%
\pgfsetlinewidth{0.000000pt}%
\definecolor{currentstroke}{rgb}{0.000000,0.000000,0.000000}%
\pgfsetstrokecolor{currentstroke}%
\pgfsetdash{}{0pt}%
\pgfpathmoveto{\pgfqpoint{3.729225in}{4.457575in}}%
\pgfpathlineto{\pgfqpoint{3.742560in}{4.441138in}}%
\pgfpathlineto{\pgfqpoint{3.755893in}{4.424822in}}%
\pgfpathlineto{\pgfqpoint{3.769225in}{4.408628in}}%
\pgfpathlineto{\pgfqpoint{3.782557in}{4.392553in}}%
\pgfpathlineto{\pgfqpoint{3.790189in}{4.431029in}}%
\pgfpathlineto{\pgfqpoint{3.797822in}{4.470181in}}%
\pgfpathlineto{\pgfqpoint{3.805455in}{4.510020in}}%
\pgfpathlineto{\pgfqpoint{3.813088in}{4.550559in}}%
\pgfpathlineto{\pgfqpoint{3.799741in}{4.567238in}}%
\pgfpathlineto{\pgfqpoint{3.786394in}{4.584036in}}%
\pgfpathlineto{\pgfqpoint{3.773045in}{4.600957in}}%
\pgfpathlineto{\pgfqpoint{3.759694in}{4.618001in}}%
\pgfpathlineto{\pgfqpoint{3.752077in}{4.576846in}}%
\pgfpathlineto{\pgfqpoint{3.744460in}{4.536398in}}%
\pgfpathlineto{\pgfqpoint{3.736843in}{4.496645in}}%
\pgfpathlineto{\pgfqpoint{3.729225in}{4.457575in}}%
\pgfpathclose%
\pgfusepath{fill}%
\end{pgfscope}%
\begin{pgfscope}%
\pgfpathrectangle{\pgfqpoint{1.150000in}{0.150000in}}{\pgfqpoint{5.700000in}{5.700000in}}%
\pgfusepath{clip}%
\pgfsetbuttcap%
\pgfsetroundjoin%
\definecolor{currentfill}{rgb}{0.225863,0.330805,0.547314}%
\pgfsetfillcolor{currentfill}%
\pgfsetfillopacity{0.700000}%
\pgfsetlinewidth{0.000000pt}%
\definecolor{currentstroke}{rgb}{0.000000,0.000000,0.000000}%
\pgfsetstrokecolor{currentstroke}%
\pgfsetdash{}{0pt}%
\pgfpathmoveto{\pgfqpoint{3.771928in}{3.155375in}}%
\pgfpathlineto{\pgfqpoint{3.785231in}{3.145824in}}%
\pgfpathlineto{\pgfqpoint{3.798536in}{3.136373in}}%
\pgfpathlineto{\pgfqpoint{3.811844in}{3.127020in}}%
\pgfpathlineto{\pgfqpoint{3.825154in}{3.117764in}}%
\pgfpathlineto{\pgfqpoint{3.832893in}{3.136448in}}%
\pgfpathlineto{\pgfqpoint{3.840628in}{3.155439in}}%
\pgfpathlineto{\pgfqpoint{3.848360in}{3.174744in}}%
\pgfpathlineto{\pgfqpoint{3.856088in}{3.194371in}}%
\pgfpathlineto{\pgfqpoint{3.842780in}{3.204004in}}%
\pgfpathlineto{\pgfqpoint{3.829474in}{3.213736in}}%
\pgfpathlineto{\pgfqpoint{3.816171in}{3.223565in}}%
\pgfpathlineto{\pgfqpoint{3.802870in}{3.233493in}}%
\pgfpathlineto{\pgfqpoint{3.795140in}{3.213481in}}%
\pgfpathlineto{\pgfqpoint{3.787406in}{3.193795in}}%
\pgfpathlineto{\pgfqpoint{3.779669in}{3.174429in}}%
\pgfpathlineto{\pgfqpoint{3.771928in}{3.155375in}}%
\pgfpathclose%
\pgfusepath{fill}%
\end{pgfscope}%
\begin{pgfscope}%
\pgfpathrectangle{\pgfqpoint{1.150000in}{0.150000in}}{\pgfqpoint{5.700000in}{5.700000in}}%
\pgfusepath{clip}%
\pgfsetbuttcap%
\pgfsetroundjoin%
\definecolor{currentfill}{rgb}{0.141935,0.526453,0.555991}%
\pgfsetfillcolor{currentfill}%
\pgfsetfillopacity{0.700000}%
\pgfsetlinewidth{0.000000pt}%
\definecolor{currentstroke}{rgb}{0.000000,0.000000,0.000000}%
\pgfsetstrokecolor{currentstroke}%
\pgfsetdash{}{0pt}%
\pgfpathmoveto{\pgfqpoint{4.200823in}{3.641365in}}%
\pgfpathlineto{\pgfqpoint{4.214166in}{3.630239in}}%
\pgfpathlineto{\pgfqpoint{4.227510in}{3.619205in}}%
\pgfpathlineto{\pgfqpoint{4.240857in}{3.608261in}}%
\pgfpathlineto{\pgfqpoint{4.254207in}{3.597407in}}%
\pgfpathlineto{\pgfqpoint{4.261912in}{3.625609in}}%
\pgfpathlineto{\pgfqpoint{4.269619in}{3.654335in}}%
\pgfpathlineto{\pgfqpoint{4.277328in}{3.683594in}}%
\pgfpathlineto{\pgfqpoint{4.285040in}{3.713396in}}%
\pgfpathlineto{\pgfqpoint{4.271688in}{3.724779in}}%
\pgfpathlineto{\pgfqpoint{4.258338in}{3.736253in}}%
\pgfpathlineto{\pgfqpoint{4.244990in}{3.747818in}}%
\pgfpathlineto{\pgfqpoint{4.231644in}{3.759473in}}%
\pgfpathlineto{\pgfqpoint{4.223935in}{3.729132in}}%
\pgfpathlineto{\pgfqpoint{4.216229in}{3.699341in}}%
\pgfpathlineto{\pgfqpoint{4.208525in}{3.670088in}}%
\pgfpathlineto{\pgfqpoint{4.200823in}{3.641365in}}%
\pgfpathclose%
\pgfusepath{fill}%
\end{pgfscope}%
\begin{pgfscope}%
\pgfpathrectangle{\pgfqpoint{1.150000in}{0.150000in}}{\pgfqpoint{5.700000in}{5.700000in}}%
\pgfusepath{clip}%
\pgfsetbuttcap%
\pgfsetroundjoin%
\definecolor{currentfill}{rgb}{0.220057,0.343307,0.549413}%
\pgfsetfillcolor{currentfill}%
\pgfsetfillopacity{0.700000}%
\pgfsetlinewidth{0.000000pt}%
\definecolor{currentstroke}{rgb}{0.000000,0.000000,0.000000}%
\pgfsetstrokecolor{currentstroke}%
\pgfsetdash{}{0pt}%
\pgfpathmoveto{\pgfqpoint{3.137678in}{3.188446in}}%
\pgfpathlineto{\pgfqpoint{3.150957in}{3.176426in}}%
\pgfpathlineto{\pgfqpoint{3.164236in}{3.164535in}}%
\pgfpathlineto{\pgfqpoint{3.177515in}{3.152772in}}%
\pgfpathlineto{\pgfqpoint{3.190793in}{3.141136in}}%
\pgfpathlineto{\pgfqpoint{3.198666in}{3.157418in}}%
\pgfpathlineto{\pgfqpoint{3.206533in}{3.173933in}}%
\pgfpathlineto{\pgfqpoint{3.214393in}{3.190687in}}%
\pgfpathlineto{\pgfqpoint{3.222246in}{3.207685in}}%
\pgfpathlineto{\pgfqpoint{3.208971in}{3.219600in}}%
\pgfpathlineto{\pgfqpoint{3.195695in}{3.231642in}}%
\pgfpathlineto{\pgfqpoint{3.182419in}{3.243811in}}%
\pgfpathlineto{\pgfqpoint{3.169143in}{3.256110in}}%
\pgfpathlineto{\pgfqpoint{3.161287in}{3.238826in}}%
\pgfpathlineto{\pgfqpoint{3.153425in}{3.221790in}}%
\pgfpathlineto{\pgfqpoint{3.145555in}{3.204999in}}%
\pgfpathlineto{\pgfqpoint{3.137678in}{3.188446in}}%
\pgfpathclose%
\pgfusepath{fill}%
\end{pgfscope}%
\begin{pgfscope}%
\pgfpathrectangle{\pgfqpoint{1.150000in}{0.150000in}}{\pgfqpoint{5.700000in}{5.700000in}}%
\pgfusepath{clip}%
\pgfsetbuttcap%
\pgfsetroundjoin%
\definecolor{currentfill}{rgb}{0.121831,0.589055,0.545623}%
\pgfsetfillcolor{currentfill}%
\pgfsetfillopacity{0.700000}%
\pgfsetlinewidth{0.000000pt}%
\definecolor{currentstroke}{rgb}{0.000000,0.000000,0.000000}%
\pgfsetstrokecolor{currentstroke}%
\pgfsetdash{}{0pt}%
\pgfpathmoveto{\pgfqpoint{4.178281in}{3.807021in}}%
\pgfpathlineto{\pgfqpoint{4.191619in}{3.794994in}}%
\pgfpathlineto{\pgfqpoint{4.204958in}{3.783061in}}%
\pgfpathlineto{\pgfqpoint{4.218300in}{3.771221in}}%
\pgfpathlineto{\pgfqpoint{4.231644in}{3.759473in}}%
\pgfpathlineto{\pgfqpoint{4.239355in}{3.790374in}}%
\pgfpathlineto{\pgfqpoint{4.247069in}{3.821847in}}%
\pgfpathlineto{\pgfqpoint{4.254787in}{3.853900in}}%
\pgfpathlineto{\pgfqpoint{4.262508in}{3.886547in}}%
\pgfpathlineto{\pgfqpoint{4.249159in}{3.898850in}}%
\pgfpathlineto{\pgfqpoint{4.235811in}{3.911245in}}%
\pgfpathlineto{\pgfqpoint{4.222466in}{3.923734in}}%
\pgfpathlineto{\pgfqpoint{4.209122in}{3.936317in}}%
\pgfpathlineto{\pgfqpoint{4.201407in}{3.903105in}}%
\pgfpathlineto{\pgfqpoint{4.193696in}{3.870493in}}%
\pgfpathlineto{\pgfqpoint{4.185987in}{3.838468in}}%
\pgfpathlineto{\pgfqpoint{4.178281in}{3.807021in}}%
\pgfpathclose%
\pgfusepath{fill}%
\end{pgfscope}%
\begin{pgfscope}%
\pgfpathrectangle{\pgfqpoint{1.150000in}{0.150000in}}{\pgfqpoint{5.700000in}{5.700000in}}%
\pgfusepath{clip}%
\pgfsetbuttcap%
\pgfsetroundjoin%
\definecolor{currentfill}{rgb}{0.175707,0.697900,0.491033}%
\pgfsetfillcolor{currentfill}%
\pgfsetfillopacity{0.700000}%
\pgfsetlinewidth{0.000000pt}%
\definecolor{currentstroke}{rgb}{0.000000,0.000000,0.000000}%
\pgfsetstrokecolor{currentstroke}%
\pgfsetdash{}{0pt}%
\pgfpathmoveto{\pgfqpoint{4.049100in}{4.094864in}}%
\pgfpathlineto{\pgfqpoint{4.062429in}{4.081105in}}%
\pgfpathlineto{\pgfqpoint{4.075759in}{4.067448in}}%
\pgfpathlineto{\pgfqpoint{4.089090in}{4.053892in}}%
\pgfpathlineto{\pgfqpoint{4.102422in}{4.040436in}}%
\pgfpathlineto{\pgfqpoint{4.110125in}{4.075407in}}%
\pgfpathlineto{\pgfqpoint{4.117829in}{4.111013in}}%
\pgfpathlineto{\pgfqpoint{4.125537in}{4.147265in}}%
\pgfpathlineto{\pgfqpoint{4.133249in}{4.184176in}}%
\pgfpathlineto{\pgfqpoint{4.119907in}{4.198219in}}%
\pgfpathlineto{\pgfqpoint{4.106566in}{4.212363in}}%
\pgfpathlineto{\pgfqpoint{4.093226in}{4.226608in}}%
\pgfpathlineto{\pgfqpoint{4.079886in}{4.240956in}}%
\pgfpathlineto{\pgfqpoint{4.072185in}{4.203447in}}%
\pgfpathlineto{\pgfqpoint{4.064487in}{4.166603in}}%
\pgfpathlineto{\pgfqpoint{4.056792in}{4.130413in}}%
\pgfpathlineto{\pgfqpoint{4.049100in}{4.094864in}}%
\pgfpathclose%
\pgfusepath{fill}%
\end{pgfscope}%
\begin{pgfscope}%
\pgfpathrectangle{\pgfqpoint{1.150000in}{0.150000in}}{\pgfqpoint{5.700000in}{5.700000in}}%
\pgfusepath{clip}%
\pgfsetbuttcap%
\pgfsetroundjoin%
\definecolor{currentfill}{rgb}{0.515992,0.831158,0.294279}%
\pgfsetfillcolor{currentfill}%
\pgfsetfillopacity{0.700000}%
\pgfsetlinewidth{0.000000pt}%
\definecolor{currentstroke}{rgb}{0.000000,0.000000,0.000000}%
\pgfsetstrokecolor{currentstroke}%
\pgfsetdash{}{0pt}%
\pgfpathmoveto{\pgfqpoint{3.675872in}{4.524567in}}%
\pgfpathlineto{\pgfqpoint{3.689213in}{4.507631in}}%
\pgfpathlineto{\pgfqpoint{3.702552in}{4.490821in}}%
\pgfpathlineto{\pgfqpoint{3.715889in}{4.474136in}}%
\pgfpathlineto{\pgfqpoint{3.729225in}{4.457575in}}%
\pgfpathlineto{\pgfqpoint{3.736843in}{4.496645in}}%
\pgfpathlineto{\pgfqpoint{3.744460in}{4.536398in}}%
\pgfpathlineto{\pgfqpoint{3.752077in}{4.576846in}}%
\pgfpathlineto{\pgfqpoint{3.759694in}{4.618001in}}%
\pgfpathlineto{\pgfqpoint{3.746342in}{4.635169in}}%
\pgfpathlineto{\pgfqpoint{3.732989in}{4.652462in}}%
\pgfpathlineto{\pgfqpoint{3.719634in}{4.669880in}}%
\pgfpathlineto{\pgfqpoint{3.706276in}{4.687426in}}%
\pgfpathlineto{\pgfqpoint{3.698677in}{4.645652in}}%
\pgfpathlineto{\pgfqpoint{3.691076in}{4.604593in}}%
\pgfpathlineto{\pgfqpoint{3.683475in}{4.564235in}}%
\pgfpathlineto{\pgfqpoint{3.675872in}{4.524567in}}%
\pgfpathclose%
\pgfusepath{fill}%
\end{pgfscope}%
\begin{pgfscope}%
\pgfpathrectangle{\pgfqpoint{1.150000in}{0.150000in}}{\pgfqpoint{5.700000in}{5.700000in}}%
\pgfusepath{clip}%
\pgfsetbuttcap%
\pgfsetroundjoin%
\definecolor{currentfill}{rgb}{0.188923,0.410910,0.556326}%
\pgfsetfillcolor{currentfill}%
\pgfsetfillopacity{0.700000}%
\pgfsetlinewidth{0.000000pt}%
\definecolor{currentstroke}{rgb}{0.000000,0.000000,0.000000}%
\pgfsetstrokecolor{currentstroke}%
\pgfsetdash{}{0pt}%
\pgfpathmoveto{\pgfqpoint{4.108469in}{3.334358in}}%
\pgfpathlineto{\pgfqpoint{4.121808in}{3.324684in}}%
\pgfpathlineto{\pgfqpoint{4.135150in}{3.315101in}}%
\pgfpathlineto{\pgfqpoint{4.148496in}{3.305607in}}%
\pgfpathlineto{\pgfqpoint{4.161844in}{3.296203in}}%
\pgfpathlineto{\pgfqpoint{4.169539in}{3.318883in}}%
\pgfpathlineto{\pgfqpoint{4.177234in}{3.341973in}}%
\pgfpathlineto{\pgfqpoint{4.184929in}{3.365482in}}%
\pgfpathlineto{\pgfqpoint{4.192623in}{3.389419in}}%
\pgfpathlineto{\pgfqpoint{4.179276in}{3.399285in}}%
\pgfpathlineto{\pgfqpoint{4.165931in}{3.409239in}}%
\pgfpathlineto{\pgfqpoint{4.152589in}{3.419284in}}%
\pgfpathlineto{\pgfqpoint{4.139250in}{3.429419in}}%
\pgfpathlineto{\pgfqpoint{4.131555in}{3.405013in}}%
\pgfpathlineto{\pgfqpoint{4.123860in}{3.381040in}}%
\pgfpathlineto{\pgfqpoint{4.116165in}{3.357491in}}%
\pgfpathlineto{\pgfqpoint{4.108469in}{3.334358in}}%
\pgfpathclose%
\pgfusepath{fill}%
\end{pgfscope}%
\begin{pgfscope}%
\pgfpathrectangle{\pgfqpoint{1.150000in}{0.150000in}}{\pgfqpoint{5.700000in}{5.700000in}}%
\pgfusepath{clip}%
\pgfsetbuttcap%
\pgfsetroundjoin%
\definecolor{currentfill}{rgb}{0.231674,0.318106,0.544834}%
\pgfsetfillcolor{currentfill}%
\pgfsetfillopacity{0.700000}%
\pgfsetlinewidth{0.000000pt}%
\definecolor{currentstroke}{rgb}{0.000000,0.000000,0.000000}%
\pgfsetstrokecolor{currentstroke}%
\pgfsetdash{}{0pt}%
\pgfpathmoveto{\pgfqpoint{3.687722in}{3.119913in}}%
\pgfpathlineto{\pgfqpoint{3.701018in}{3.110319in}}%
\pgfpathlineto{\pgfqpoint{3.714317in}{3.100827in}}%
\pgfpathlineto{\pgfqpoint{3.727618in}{3.091435in}}%
\pgfpathlineto{\pgfqpoint{3.740921in}{3.082143in}}%
\pgfpathlineto{\pgfqpoint{3.748679in}{3.100017in}}%
\pgfpathlineto{\pgfqpoint{3.756433in}{3.118175in}}%
\pgfpathlineto{\pgfqpoint{3.764182in}{3.136626in}}%
\pgfpathlineto{\pgfqpoint{3.771928in}{3.155375in}}%
\pgfpathlineto{\pgfqpoint{3.758627in}{3.165024in}}%
\pgfpathlineto{\pgfqpoint{3.745328in}{3.174773in}}%
\pgfpathlineto{\pgfqpoint{3.732032in}{3.184623in}}%
\pgfpathlineto{\pgfqpoint{3.718738in}{3.194574in}}%
\pgfpathlineto{\pgfqpoint{3.710990in}{3.175460in}}%
\pgfpathlineto{\pgfqpoint{3.703239in}{3.156649in}}%
\pgfpathlineto{\pgfqpoint{3.695483in}{3.138136in}}%
\pgfpathlineto{\pgfqpoint{3.687722in}{3.119913in}}%
\pgfpathclose%
\pgfusepath{fill}%
\end{pgfscope}%
\begin{pgfscope}%
\pgfpathrectangle{\pgfqpoint{1.150000in}{0.150000in}}{\pgfqpoint{5.700000in}{5.700000in}}%
\pgfusepath{clip}%
\pgfsetbuttcap%
\pgfsetroundjoin%
\definecolor{currentfill}{rgb}{0.153894,0.680203,0.504172}%
\pgfsetfillcolor{currentfill}%
\pgfsetfillopacity{0.700000}%
\pgfsetlinewidth{0.000000pt}%
\definecolor{currentstroke}{rgb}{0.000000,0.000000,0.000000}%
\pgfsetstrokecolor{currentstroke}%
\pgfsetdash{}{0pt}%
\pgfpathmoveto{\pgfqpoint{4.102422in}{4.040436in}}%
\pgfpathlineto{\pgfqpoint{4.115756in}{4.027079in}}%
\pgfpathlineto{\pgfqpoint{4.129090in}{4.013822in}}%
\pgfpathlineto{\pgfqpoint{4.142425in}{4.000662in}}%
\pgfpathlineto{\pgfqpoint{4.155762in}{3.987600in}}%
\pgfpathlineto{\pgfqpoint{4.163472in}{4.021996in}}%
\pgfpathlineto{\pgfqpoint{4.171186in}{4.057020in}}%
\pgfpathlineto{\pgfqpoint{4.178903in}{4.092684in}}%
\pgfpathlineto{\pgfqpoint{4.186625in}{4.129000in}}%
\pgfpathlineto{\pgfqpoint{4.173279in}{4.142646in}}%
\pgfpathlineto{\pgfqpoint{4.159935in}{4.156391in}}%
\pgfpathlineto{\pgfqpoint{4.146591in}{4.170234in}}%
\pgfpathlineto{\pgfqpoint{4.133249in}{4.184176in}}%
\pgfpathlineto{\pgfqpoint{4.125537in}{4.147265in}}%
\pgfpathlineto{\pgfqpoint{4.117829in}{4.111013in}}%
\pgfpathlineto{\pgfqpoint{4.110125in}{4.075407in}}%
\pgfpathlineto{\pgfqpoint{4.102422in}{4.040436in}}%
\pgfpathclose%
\pgfusepath{fill}%
\end{pgfscope}%
\begin{pgfscope}%
\pgfpathrectangle{\pgfqpoint{1.150000in}{0.150000in}}{\pgfqpoint{5.700000in}{5.700000in}}%
\pgfusepath{clip}%
\pgfsetbuttcap%
\pgfsetroundjoin%
\definecolor{currentfill}{rgb}{0.233603,0.313828,0.543914}%
\pgfsetfillcolor{currentfill}%
\pgfsetfillopacity{0.700000}%
\pgfsetlinewidth{0.000000pt}%
\definecolor{currentstroke}{rgb}{0.000000,0.000000,0.000000}%
\pgfsetstrokecolor{currentstroke}%
\pgfsetdash{}{0pt}%
\pgfpathmoveto{\pgfqpoint{3.328445in}{3.116807in}}%
\pgfpathlineto{\pgfqpoint{3.341722in}{3.105987in}}%
\pgfpathlineto{\pgfqpoint{3.354999in}{3.095284in}}%
\pgfpathlineto{\pgfqpoint{3.368277in}{3.084696in}}%
\pgfpathlineto{\pgfqpoint{3.381556in}{3.074224in}}%
\pgfpathlineto{\pgfqpoint{3.389393in}{3.090599in}}%
\pgfpathlineto{\pgfqpoint{3.397224in}{3.107214in}}%
\pgfpathlineto{\pgfqpoint{3.405049in}{3.124072in}}%
\pgfpathlineto{\pgfqpoint{3.412868in}{3.141181in}}%
\pgfpathlineto{\pgfqpoint{3.399592in}{3.151951in}}%
\pgfpathlineto{\pgfqpoint{3.386317in}{3.162836in}}%
\pgfpathlineto{\pgfqpoint{3.373043in}{3.173837in}}%
\pgfpathlineto{\pgfqpoint{3.359769in}{3.184955in}}%
\pgfpathlineto{\pgfqpoint{3.351948in}{3.167541in}}%
\pgfpathlineto{\pgfqpoint{3.344120in}{3.150381in}}%
\pgfpathlineto{\pgfqpoint{3.336286in}{3.133472in}}%
\pgfpathlineto{\pgfqpoint{3.328445in}{3.116807in}}%
\pgfpathclose%
\pgfusepath{fill}%
\end{pgfscope}%
\begin{pgfscope}%
\pgfpathrectangle{\pgfqpoint{1.150000in}{0.150000in}}{\pgfqpoint{5.700000in}{5.700000in}}%
\pgfusepath{clip}%
\pgfsetbuttcap%
\pgfsetroundjoin%
\definecolor{currentfill}{rgb}{0.235526,0.309527,0.542944}%
\pgfsetfillcolor{currentfill}%
\pgfsetfillopacity{0.700000}%
\pgfsetlinewidth{0.000000pt}%
\definecolor{currentstroke}{rgb}{0.000000,0.000000,0.000000}%
\pgfsetstrokecolor{currentstroke}%
\pgfsetdash{}{0pt}%
\pgfpathmoveto{\pgfqpoint{3.465981in}{3.099234in}}%
\pgfpathlineto{\pgfqpoint{3.479262in}{3.089026in}}%
\pgfpathlineto{\pgfqpoint{3.492544in}{3.078928in}}%
\pgfpathlineto{\pgfqpoint{3.505828in}{3.068939in}}%
\pgfpathlineto{\pgfqpoint{3.519114in}{3.059059in}}%
\pgfpathlineto{\pgfqpoint{3.526921in}{3.075803in}}%
\pgfpathlineto{\pgfqpoint{3.534722in}{3.092799in}}%
\pgfpathlineto{\pgfqpoint{3.542519in}{3.110052in}}%
\pgfpathlineto{\pgfqpoint{3.550309in}{3.127569in}}%
\pgfpathlineto{\pgfqpoint{3.537027in}{3.137767in}}%
\pgfpathlineto{\pgfqpoint{3.523745in}{3.148073in}}%
\pgfpathlineto{\pgfqpoint{3.510466in}{3.158488in}}%
\pgfpathlineto{\pgfqpoint{3.497187in}{3.169014in}}%
\pgfpathlineto{\pgfqpoint{3.489394in}{3.151173in}}%
\pgfpathlineto{\pgfqpoint{3.481595in}{3.133599in}}%
\pgfpathlineto{\pgfqpoint{3.473791in}{3.116289in}}%
\pgfpathlineto{\pgfqpoint{3.465981in}{3.099234in}}%
\pgfpathclose%
\pgfusepath{fill}%
\end{pgfscope}%
\begin{pgfscope}%
\pgfpathrectangle{\pgfqpoint{1.150000in}{0.150000in}}{\pgfqpoint{5.700000in}{5.700000in}}%
\pgfusepath{clip}%
\pgfsetbuttcap%
\pgfsetroundjoin%
\definecolor{currentfill}{rgb}{0.585678,0.846661,0.249897}%
\pgfsetfillcolor{currentfill}%
\pgfsetfillopacity{0.700000}%
\pgfsetlinewidth{0.000000pt}%
\definecolor{currentstroke}{rgb}{0.000000,0.000000,0.000000}%
\pgfsetstrokecolor{currentstroke}%
\pgfsetdash{}{0pt}%
\pgfpathmoveto{\pgfqpoint{3.622492in}{4.593597in}}%
\pgfpathlineto{\pgfqpoint{3.635840in}{4.576145in}}%
\pgfpathlineto{\pgfqpoint{3.649186in}{4.558823in}}%
\pgfpathlineto{\pgfqpoint{3.662530in}{4.541631in}}%
\pgfpathlineto{\pgfqpoint{3.675872in}{4.524567in}}%
\pgfpathlineto{\pgfqpoint{3.683475in}{4.564235in}}%
\pgfpathlineto{\pgfqpoint{3.691076in}{4.604593in}}%
\pgfpathlineto{\pgfqpoint{3.698677in}{4.645652in}}%
\pgfpathlineto{\pgfqpoint{3.706276in}{4.687426in}}%
\pgfpathlineto{\pgfqpoint{3.692917in}{4.705101in}}%
\pgfpathlineto{\pgfqpoint{3.679556in}{4.722905in}}%
\pgfpathlineto{\pgfqpoint{3.666193in}{4.740839in}}%
\pgfpathlineto{\pgfqpoint{3.652828in}{4.758906in}}%
\pgfpathlineto{\pgfqpoint{3.645246in}{4.716508in}}%
\pgfpathlineto{\pgfqpoint{3.637663in}{4.674833in}}%
\pgfpathlineto{\pgfqpoint{3.630078in}{4.633867in}}%
\pgfpathlineto{\pgfqpoint{3.622492in}{4.593597in}}%
\pgfpathclose%
\pgfusepath{fill}%
\end{pgfscope}%
\begin{pgfscope}%
\pgfpathrectangle{\pgfqpoint{1.150000in}{0.150000in}}{\pgfqpoint{5.700000in}{5.700000in}}%
\pgfusepath{clip}%
\pgfsetbuttcap%
\pgfsetroundjoin%
\definecolor{currentfill}{rgb}{0.163625,0.471133,0.558148}%
\pgfsetfillcolor{currentfill}%
\pgfsetfillopacity{0.700000}%
\pgfsetlinewidth{0.000000pt}%
\definecolor{currentstroke}{rgb}{0.000000,0.000000,0.000000}%
\pgfsetstrokecolor{currentstroke}%
\pgfsetdash{}{0pt}%
\pgfpathmoveto{\pgfqpoint{4.223406in}{3.489623in}}%
\pgfpathlineto{\pgfqpoint{4.236757in}{3.479364in}}%
\pgfpathlineto{\pgfqpoint{4.250111in}{3.469193in}}%
\pgfpathlineto{\pgfqpoint{4.263467in}{3.459111in}}%
\pgfpathlineto{\pgfqpoint{4.276827in}{3.449116in}}%
\pgfpathlineto{\pgfqpoint{4.284525in}{3.474831in}}%
\pgfpathlineto{\pgfqpoint{4.292225in}{3.501023in}}%
\pgfpathlineto{\pgfqpoint{4.299927in}{3.527703in}}%
\pgfpathlineto{\pgfqpoint{4.307632in}{3.554879in}}%
\pgfpathlineto{\pgfqpoint{4.294271in}{3.565379in}}%
\pgfpathlineto{\pgfqpoint{4.280914in}{3.575966in}}%
\pgfpathlineto{\pgfqpoint{4.267559in}{3.586642in}}%
\pgfpathlineto{\pgfqpoint{4.254207in}{3.597407in}}%
\pgfpathlineto{\pgfqpoint{4.246504in}{3.569716in}}%
\pgfpathlineto{\pgfqpoint{4.238804in}{3.542529in}}%
\pgfpathlineto{\pgfqpoint{4.231104in}{3.515834in}}%
\pgfpathlineto{\pgfqpoint{4.223406in}{3.489623in}}%
\pgfpathclose%
\pgfusepath{fill}%
\end{pgfscope}%
\begin{pgfscope}%
\pgfpathrectangle{\pgfqpoint{1.150000in}{0.150000in}}{\pgfqpoint{5.700000in}{5.700000in}}%
\pgfusepath{clip}%
\pgfsetbuttcap%
\pgfsetroundjoin%
\definecolor{currentfill}{rgb}{0.125394,0.574318,0.549086}%
\pgfsetfillcolor{currentfill}%
\pgfsetfillopacity{0.700000}%
\pgfsetlinewidth{0.000000pt}%
\definecolor{currentstroke}{rgb}{0.000000,0.000000,0.000000}%
\pgfsetstrokecolor{currentstroke}%
\pgfsetdash{}{0pt}%
\pgfpathmoveto{\pgfqpoint{4.231644in}{3.759473in}}%
\pgfpathlineto{\pgfqpoint{4.244990in}{3.747818in}}%
\pgfpathlineto{\pgfqpoint{4.258338in}{3.736253in}}%
\pgfpathlineto{\pgfqpoint{4.271688in}{3.724779in}}%
\pgfpathlineto{\pgfqpoint{4.285040in}{3.713396in}}%
\pgfpathlineto{\pgfqpoint{4.292756in}{3.743752in}}%
\pgfpathlineto{\pgfqpoint{4.300475in}{3.774674in}}%
\pgfpathlineto{\pgfqpoint{4.308197in}{3.806171in}}%
\pgfpathlineto{\pgfqpoint{4.315923in}{3.838254in}}%
\pgfpathlineto{\pgfqpoint{4.302566in}{3.850191in}}%
\pgfpathlineto{\pgfqpoint{4.289211in}{3.862218in}}%
\pgfpathlineto{\pgfqpoint{4.275859in}{3.874337in}}%
\pgfpathlineto{\pgfqpoint{4.262508in}{3.886547in}}%
\pgfpathlineto{\pgfqpoint{4.254787in}{3.853900in}}%
\pgfpathlineto{\pgfqpoint{4.247069in}{3.821847in}}%
\pgfpathlineto{\pgfqpoint{4.239355in}{3.790374in}}%
\pgfpathlineto{\pgfqpoint{4.231644in}{3.759473in}}%
\pgfpathclose%
\pgfusepath{fill}%
\end{pgfscope}%
\begin{pgfscope}%
\pgfpathrectangle{\pgfqpoint{1.150000in}{0.150000in}}{\pgfqpoint{5.700000in}{5.700000in}}%
\pgfusepath{clip}%
\pgfsetbuttcap%
\pgfsetroundjoin%
\definecolor{currentfill}{rgb}{0.227802,0.326594,0.546532}%
\pgfsetfillcolor{currentfill}%
\pgfsetfillopacity{0.700000}%
\pgfsetlinewidth{0.000000pt}%
\definecolor{currentstroke}{rgb}{0.000000,0.000000,0.000000}%
\pgfsetstrokecolor{currentstroke}%
\pgfsetdash{}{0pt}%
\pgfpathmoveto{\pgfqpoint{3.190793in}{3.141136in}}%
\pgfpathlineto{\pgfqpoint{3.204071in}{3.129625in}}%
\pgfpathlineto{\pgfqpoint{3.217348in}{3.118239in}}%
\pgfpathlineto{\pgfqpoint{3.230626in}{3.106976in}}%
\pgfpathlineto{\pgfqpoint{3.243904in}{3.095836in}}%
\pgfpathlineto{\pgfqpoint{3.251774in}{3.111848in}}%
\pgfpathlineto{\pgfqpoint{3.259638in}{3.128088in}}%
\pgfpathlineto{\pgfqpoint{3.267494in}{3.144561in}}%
\pgfpathlineto{\pgfqpoint{3.275344in}{3.161273in}}%
\pgfpathlineto{\pgfqpoint{3.262070in}{3.172691in}}%
\pgfpathlineto{\pgfqpoint{3.248795in}{3.184232in}}%
\pgfpathlineto{\pgfqpoint{3.235520in}{3.195896in}}%
\pgfpathlineto{\pgfqpoint{3.222246in}{3.207685in}}%
\pgfpathlineto{\pgfqpoint{3.214393in}{3.190687in}}%
\pgfpathlineto{\pgfqpoint{3.206533in}{3.173933in}}%
\pgfpathlineto{\pgfqpoint{3.198666in}{3.157418in}}%
\pgfpathlineto{\pgfqpoint{3.190793in}{3.141136in}}%
\pgfpathclose%
\pgfusepath{fill}%
\end{pgfscope}%
\begin{pgfscope}%
\pgfpathrectangle{\pgfqpoint{1.150000in}{0.150000in}}{\pgfqpoint{5.700000in}{5.700000in}}%
\pgfusepath{clip}%
\pgfsetbuttcap%
\pgfsetroundjoin%
\definecolor{currentfill}{rgb}{0.212395,0.359683,0.551710}%
\pgfsetfillcolor{currentfill}%
\pgfsetfillopacity{0.700000}%
\pgfsetlinewidth{0.000000pt}%
\definecolor{currentstroke}{rgb}{0.000000,0.000000,0.000000}%
\pgfsetstrokecolor{currentstroke}%
\pgfsetdash{}{0pt}%
\pgfpathmoveto{\pgfqpoint{2.999800in}{3.223391in}}%
\pgfpathlineto{\pgfqpoint{3.013092in}{3.210557in}}%
\pgfpathlineto{\pgfqpoint{3.026382in}{3.197860in}}%
\pgfpathlineto{\pgfqpoint{3.039670in}{3.185301in}}%
\pgfpathlineto{\pgfqpoint{3.052958in}{3.172878in}}%
\pgfpathlineto{\pgfqpoint{3.060868in}{3.188777in}}%
\pgfpathlineto{\pgfqpoint{3.068770in}{3.204900in}}%
\pgfpathlineto{\pgfqpoint{3.076665in}{3.221252in}}%
\pgfpathlineto{\pgfqpoint{3.084552in}{3.237838in}}%
\pgfpathlineto{\pgfqpoint{3.071269in}{3.250520in}}%
\pgfpathlineto{\pgfqpoint{3.057983in}{3.263338in}}%
\pgfpathlineto{\pgfqpoint{3.044697in}{3.276293in}}%
\pgfpathlineto{\pgfqpoint{3.031409in}{3.289387in}}%
\pgfpathlineto{\pgfqpoint{3.023518in}{3.272535in}}%
\pgfpathlineto{\pgfqpoint{3.015620in}{3.255921in}}%
\pgfpathlineto{\pgfqpoint{3.007714in}{3.239542in}}%
\pgfpathlineto{\pgfqpoint{2.999800in}{3.223391in}}%
\pgfpathclose%
\pgfusepath{fill}%
\end{pgfscope}%
\begin{pgfscope}%
\pgfpathrectangle{\pgfqpoint{1.150000in}{0.150000in}}{\pgfqpoint{5.700000in}{5.700000in}}%
\pgfusepath{clip}%
\pgfsetbuttcap%
\pgfsetroundjoin%
\definecolor{currentfill}{rgb}{0.179019,0.433756,0.557430}%
\pgfsetfillcolor{currentfill}%
\pgfsetfillopacity{0.700000}%
\pgfsetlinewidth{0.000000pt}%
\definecolor{currentstroke}{rgb}{0.000000,0.000000,0.000000}%
\pgfsetstrokecolor{currentstroke}%
\pgfsetdash{}{0pt}%
\pgfpathmoveto{\pgfqpoint{4.192623in}{3.389419in}}%
\pgfpathlineto{\pgfqpoint{4.205974in}{3.379643in}}%
\pgfpathlineto{\pgfqpoint{4.219328in}{3.369955in}}%
\pgfpathlineto{\pgfqpoint{4.232684in}{3.360355in}}%
\pgfpathlineto{\pgfqpoint{4.246044in}{3.350842in}}%
\pgfpathlineto{\pgfqpoint{4.253739in}{3.374741in}}%
\pgfpathlineto{\pgfqpoint{4.261434in}{3.399080in}}%
\pgfpathlineto{\pgfqpoint{4.269130in}{3.423869in}}%
\pgfpathlineto{\pgfqpoint{4.276827in}{3.449116in}}%
\pgfpathlineto{\pgfqpoint{4.263467in}{3.459111in}}%
\pgfpathlineto{\pgfqpoint{4.250111in}{3.469193in}}%
\pgfpathlineto{\pgfqpoint{4.236757in}{3.479364in}}%
\pgfpathlineto{\pgfqpoint{4.223406in}{3.489623in}}%
\pgfpathlineto{\pgfqpoint{4.215710in}{3.463885in}}%
\pgfpathlineto{\pgfqpoint{4.208014in}{3.438612in}}%
\pgfpathlineto{\pgfqpoint{4.200318in}{3.413793in}}%
\pgfpathlineto{\pgfqpoint{4.192623in}{3.389419in}}%
\pgfpathclose%
\pgfusepath{fill}%
\end{pgfscope}%
\begin{pgfscope}%
\pgfpathrectangle{\pgfqpoint{1.150000in}{0.150000in}}{\pgfqpoint{5.700000in}{5.700000in}}%
\pgfusepath{clip}%
\pgfsetbuttcap%
\pgfsetroundjoin%
\definecolor{currentfill}{rgb}{0.214298,0.355619,0.551184}%
\pgfsetfillcolor{currentfill}%
\pgfsetfillopacity{0.700000}%
\pgfsetlinewidth{0.000000pt}%
\definecolor{currentstroke}{rgb}{0.000000,0.000000,0.000000}%
\pgfsetstrokecolor{currentstroke}%
\pgfsetdash{}{0pt}%
\pgfpathmoveto{\pgfqpoint{3.993514in}{3.199416in}}%
\pgfpathlineto{\pgfqpoint{4.006845in}{3.190236in}}%
\pgfpathlineto{\pgfqpoint{4.020178in}{3.181147in}}%
\pgfpathlineto{\pgfqpoint{4.033515in}{3.172150in}}%
\pgfpathlineto{\pgfqpoint{4.046855in}{3.163244in}}%
\pgfpathlineto{\pgfqpoint{4.054562in}{3.183353in}}%
\pgfpathlineto{\pgfqpoint{4.062268in}{3.203812in}}%
\pgfpathlineto{\pgfqpoint{4.069972in}{3.224628in}}%
\pgfpathlineto{\pgfqpoint{4.077674in}{3.245810in}}%
\pgfpathlineto{\pgfqpoint{4.064336in}{3.255135in}}%
\pgfpathlineto{\pgfqpoint{4.051002in}{3.264550in}}%
\pgfpathlineto{\pgfqpoint{4.037670in}{3.274057in}}%
\pgfpathlineto{\pgfqpoint{4.024341in}{3.283656in}}%
\pgfpathlineto{\pgfqpoint{4.016637in}{3.262047in}}%
\pgfpathlineto{\pgfqpoint{4.008932in}{3.240810in}}%
\pgfpathlineto{\pgfqpoint{4.001224in}{3.219935in}}%
\pgfpathlineto{\pgfqpoint{3.993514in}{3.199416in}}%
\pgfpathclose%
\pgfusepath{fill}%
\end{pgfscope}%
\begin{pgfscope}%
\pgfpathrectangle{\pgfqpoint{1.150000in}{0.150000in}}{\pgfqpoint{5.700000in}{5.700000in}}%
\pgfusepath{clip}%
\pgfsetbuttcap%
\pgfsetroundjoin%
\definecolor{currentfill}{rgb}{0.237441,0.305202,0.541921}%
\pgfsetfillcolor{currentfill}%
\pgfsetfillopacity{0.700000}%
\pgfsetlinewidth{0.000000pt}%
\definecolor{currentstroke}{rgb}{0.000000,0.000000,0.000000}%
\pgfsetstrokecolor{currentstroke}%
\pgfsetdash{}{0pt}%
\pgfpathmoveto{\pgfqpoint{3.603456in}{3.087845in}}%
\pgfpathlineto{\pgfqpoint{3.616747in}{3.078177in}}%
\pgfpathlineto{\pgfqpoint{3.630040in}{3.068614in}}%
\pgfpathlineto{\pgfqpoint{3.643335in}{3.059153in}}%
\pgfpathlineto{\pgfqpoint{3.656632in}{3.049795in}}%
\pgfpathlineto{\pgfqpoint{3.664412in}{3.066921in}}%
\pgfpathlineto{\pgfqpoint{3.672187in}{3.084311in}}%
\pgfpathlineto{\pgfqpoint{3.679957in}{3.101973in}}%
\pgfpathlineto{\pgfqpoint{3.687722in}{3.119913in}}%
\pgfpathlineto{\pgfqpoint{3.674428in}{3.129608in}}%
\pgfpathlineto{\pgfqpoint{3.661135in}{3.139406in}}%
\pgfpathlineto{\pgfqpoint{3.647845in}{3.149307in}}%
\pgfpathlineto{\pgfqpoint{3.634556in}{3.159312in}}%
\pgfpathlineto{\pgfqpoint{3.626789in}{3.141027in}}%
\pgfpathlineto{\pgfqpoint{3.619016in}{3.123026in}}%
\pgfpathlineto{\pgfqpoint{3.611238in}{3.105300in}}%
\pgfpathlineto{\pgfqpoint{3.603456in}{3.087845in}}%
\pgfpathclose%
\pgfusepath{fill}%
\end{pgfscope}%
\begin{pgfscope}%
\pgfpathrectangle{\pgfqpoint{1.150000in}{0.150000in}}{\pgfqpoint{5.700000in}{5.700000in}}%
\pgfusepath{clip}%
\pgfsetbuttcap%
\pgfsetroundjoin%
\definecolor{currentfill}{rgb}{0.147607,0.511733,0.557049}%
\pgfsetfillcolor{currentfill}%
\pgfsetfillopacity{0.700000}%
\pgfsetlinewidth{0.000000pt}%
\definecolor{currentstroke}{rgb}{0.000000,0.000000,0.000000}%
\pgfsetstrokecolor{currentstroke}%
\pgfsetdash{}{0pt}%
\pgfpathmoveto{\pgfqpoint{4.254207in}{3.597407in}}%
\pgfpathlineto{\pgfqpoint{4.267559in}{3.586642in}}%
\pgfpathlineto{\pgfqpoint{4.280914in}{3.575966in}}%
\pgfpathlineto{\pgfqpoint{4.294271in}{3.565379in}}%
\pgfpathlineto{\pgfqpoint{4.307632in}{3.554879in}}%
\pgfpathlineto{\pgfqpoint{4.315338in}{3.582562in}}%
\pgfpathlineto{\pgfqpoint{4.323047in}{3.610763in}}%
\pgfpathlineto{\pgfqpoint{4.330759in}{3.639490in}}%
\pgfpathlineto{\pgfqpoint{4.338475in}{3.668755in}}%
\pgfpathlineto{\pgfqpoint{4.325112in}{3.679783in}}%
\pgfpathlineto{\pgfqpoint{4.311753in}{3.690898in}}%
\pgfpathlineto{\pgfqpoint{4.298395in}{3.702102in}}%
\pgfpathlineto{\pgfqpoint{4.285040in}{3.713396in}}%
\pgfpathlineto{\pgfqpoint{4.277328in}{3.683594in}}%
\pgfpathlineto{\pgfqpoint{4.269619in}{3.654335in}}%
\pgfpathlineto{\pgfqpoint{4.261912in}{3.625609in}}%
\pgfpathlineto{\pgfqpoint{4.254207in}{3.597407in}}%
\pgfpathclose%
\pgfusepath{fill}%
\end{pgfscope}%
\begin{pgfscope}%
\pgfpathrectangle{\pgfqpoint{1.150000in}{0.150000in}}{\pgfqpoint{5.700000in}{5.700000in}}%
\pgfusepath{clip}%
\pgfsetbuttcap%
\pgfsetroundjoin%
\definecolor{currentfill}{rgb}{0.137339,0.662252,0.515571}%
\pgfsetfillcolor{currentfill}%
\pgfsetfillopacity{0.700000}%
\pgfsetlinewidth{0.000000pt}%
\definecolor{currentstroke}{rgb}{0.000000,0.000000,0.000000}%
\pgfsetstrokecolor{currentstroke}%
\pgfsetdash{}{0pt}%
\pgfpathmoveto{\pgfqpoint{4.155762in}{3.987600in}}%
\pgfpathlineto{\pgfqpoint{4.169100in}{3.974636in}}%
\pgfpathlineto{\pgfqpoint{4.182439in}{3.961767in}}%
\pgfpathlineto{\pgfqpoint{4.195780in}{3.948995in}}%
\pgfpathlineto{\pgfqpoint{4.209122in}{3.936317in}}%
\pgfpathlineto{\pgfqpoint{4.216840in}{3.970140in}}%
\pgfpathlineto{\pgfqpoint{4.224562in}{4.004584in}}%
\pgfpathlineto{\pgfqpoint{4.232288in}{4.039662in}}%
\pgfpathlineto{\pgfqpoint{4.240018in}{4.075385in}}%
\pgfpathlineto{\pgfqpoint{4.226668in}{4.088644in}}%
\pgfpathlineto{\pgfqpoint{4.213319in}{4.102000in}}%
\pgfpathlineto{\pgfqpoint{4.199971in}{4.115451in}}%
\pgfpathlineto{\pgfqpoint{4.186625in}{4.129000in}}%
\pgfpathlineto{\pgfqpoint{4.178903in}{4.092684in}}%
\pgfpathlineto{\pgfqpoint{4.171186in}{4.057020in}}%
\pgfpathlineto{\pgfqpoint{4.163472in}{4.021996in}}%
\pgfpathlineto{\pgfqpoint{4.155762in}{3.987600in}}%
\pgfpathclose%
\pgfusepath{fill}%
\end{pgfscope}%
\begin{pgfscope}%
\pgfpathrectangle{\pgfqpoint{1.150000in}{0.150000in}}{\pgfqpoint{5.700000in}{5.700000in}}%
\pgfusepath{clip}%
\pgfsetbuttcap%
\pgfsetroundjoin%
\definecolor{currentfill}{rgb}{0.223925,0.334994,0.548053}%
\pgfsetfillcolor{currentfill}%
\pgfsetfillopacity{0.700000}%
\pgfsetlinewidth{0.000000pt}%
\definecolor{currentstroke}{rgb}{0.000000,0.000000,0.000000}%
\pgfsetstrokecolor{currentstroke}%
\pgfsetdash{}{0pt}%
\pgfpathmoveto{\pgfqpoint{3.909347in}{3.156799in}}%
\pgfpathlineto{\pgfqpoint{3.922668in}{3.147643in}}%
\pgfpathlineto{\pgfqpoint{3.935992in}{3.138582in}}%
\pgfpathlineto{\pgfqpoint{3.949319in}{3.129614in}}%
\pgfpathlineto{\pgfqpoint{3.962649in}{3.120739in}}%
\pgfpathlineto{\pgfqpoint{3.970370in}{3.139913in}}%
\pgfpathlineto{\pgfqpoint{3.978087in}{3.159413in}}%
\pgfpathlineto{\pgfqpoint{3.985802in}{3.179244in}}%
\pgfpathlineto{\pgfqpoint{3.993514in}{3.199416in}}%
\pgfpathlineto{\pgfqpoint{3.980186in}{3.208689in}}%
\pgfpathlineto{\pgfqpoint{3.966862in}{3.218054in}}%
\pgfpathlineto{\pgfqpoint{3.953540in}{3.227514in}}%
\pgfpathlineto{\pgfqpoint{3.940220in}{3.237067in}}%
\pgfpathlineto{\pgfqpoint{3.932506in}{3.216489in}}%
\pgfpathlineto{\pgfqpoint{3.924789in}{3.196257in}}%
\pgfpathlineto{\pgfqpoint{3.917069in}{3.176363in}}%
\pgfpathlineto{\pgfqpoint{3.909347in}{3.156799in}}%
\pgfpathclose%
\pgfusepath{fill}%
\end{pgfscope}%
\begin{pgfscope}%
\pgfpathrectangle{\pgfqpoint{1.150000in}{0.150000in}}{\pgfqpoint{5.700000in}{5.700000in}}%
\pgfusepath{clip}%
\pgfsetbuttcap%
\pgfsetroundjoin%
\definecolor{currentfill}{rgb}{0.657642,0.860219,0.203082}%
\pgfsetfillcolor{currentfill}%
\pgfsetfillopacity{0.700000}%
\pgfsetlinewidth{0.000000pt}%
\definecolor{currentstroke}{rgb}{0.000000,0.000000,0.000000}%
\pgfsetstrokecolor{currentstroke}%
\pgfsetdash{}{0pt}%
\pgfpathmoveto{\pgfqpoint{3.569078in}{4.664741in}}%
\pgfpathlineto{\pgfqpoint{3.582435in}{4.646753in}}%
\pgfpathlineto{\pgfqpoint{3.595790in}{4.628900in}}%
\pgfpathlineto{\pgfqpoint{3.609142in}{4.611182in}}%
\pgfpathlineto{\pgfqpoint{3.622492in}{4.593597in}}%
\pgfpathlineto{\pgfqpoint{3.630078in}{4.633867in}}%
\pgfpathlineto{\pgfqpoint{3.637663in}{4.674833in}}%
\pgfpathlineto{\pgfqpoint{3.645246in}{4.716508in}}%
\pgfpathlineto{\pgfqpoint{3.652828in}{4.758906in}}%
\pgfpathlineto{\pgfqpoint{3.639460in}{4.777105in}}%
\pgfpathlineto{\pgfqpoint{3.626090in}{4.795438in}}%
\pgfpathlineto{\pgfqpoint{3.612717in}{4.813907in}}%
\pgfpathlineto{\pgfqpoint{3.599342in}{4.832513in}}%
\pgfpathlineto{\pgfqpoint{3.591779in}{4.789488in}}%
\pgfpathlineto{\pgfqpoint{3.584214in}{4.747193in}}%
\pgfpathlineto{\pgfqpoint{3.576647in}{4.705615in}}%
\pgfpathlineto{\pgfqpoint{3.569078in}{4.664741in}}%
\pgfpathclose%
\pgfusepath{fill}%
\end{pgfscope}%
\begin{pgfscope}%
\pgfpathrectangle{\pgfqpoint{1.150000in}{0.150000in}}{\pgfqpoint{5.700000in}{5.700000in}}%
\pgfusepath{clip}%
\pgfsetbuttcap%
\pgfsetroundjoin%
\definecolor{currentfill}{rgb}{0.204903,0.375746,0.553533}%
\pgfsetfillcolor{currentfill}%
\pgfsetfillopacity{0.700000}%
\pgfsetlinewidth{0.000000pt}%
\definecolor{currentstroke}{rgb}{0.000000,0.000000,0.000000}%
\pgfsetstrokecolor{currentstroke}%
\pgfsetdash{}{0pt}%
\pgfpathmoveto{\pgfqpoint{4.077674in}{3.245810in}}%
\pgfpathlineto{\pgfqpoint{4.091015in}{3.236576in}}%
\pgfpathlineto{\pgfqpoint{4.104358in}{3.227432in}}%
\pgfpathlineto{\pgfqpoint{4.117706in}{3.218378in}}%
\pgfpathlineto{\pgfqpoint{4.131056in}{3.209412in}}%
\pgfpathlineto{\pgfqpoint{4.138754in}{3.230537in}}%
\pgfpathlineto{\pgfqpoint{4.146452in}{3.252038in}}%
\pgfpathlineto{\pgfqpoint{4.154148in}{3.273924in}}%
\pgfpathlineto{\pgfqpoint{4.161844in}{3.296203in}}%
\pgfpathlineto{\pgfqpoint{4.148496in}{3.305607in}}%
\pgfpathlineto{\pgfqpoint{4.135150in}{3.315101in}}%
\pgfpathlineto{\pgfqpoint{4.121808in}{3.324684in}}%
\pgfpathlineto{\pgfqpoint{4.108469in}{3.334358in}}%
\pgfpathlineto{\pgfqpoint{4.100772in}{3.311631in}}%
\pgfpathlineto{\pgfqpoint{4.093074in}{3.289304in}}%
\pgfpathlineto{\pgfqpoint{4.085374in}{3.267366in}}%
\pgfpathlineto{\pgfqpoint{4.077674in}{3.245810in}}%
\pgfpathclose%
\pgfusepath{fill}%
\end{pgfscope}%
\begin{pgfscope}%
\pgfpathrectangle{\pgfqpoint{1.150000in}{0.150000in}}{\pgfqpoint{5.700000in}{5.700000in}}%
\pgfusepath{clip}%
\pgfsetbuttcap%
\pgfsetroundjoin%
\definecolor{currentfill}{rgb}{0.231674,0.318106,0.544834}%
\pgfsetfillcolor{currentfill}%
\pgfsetfillopacity{0.700000}%
\pgfsetlinewidth{0.000000pt}%
\definecolor{currentstroke}{rgb}{0.000000,0.000000,0.000000}%
\pgfsetstrokecolor{currentstroke}%
\pgfsetdash{}{0pt}%
\pgfpathmoveto{\pgfqpoint{3.825154in}{3.117764in}}%
\pgfpathlineto{\pgfqpoint{3.838467in}{3.108604in}}%
\pgfpathlineto{\pgfqpoint{3.851783in}{3.099541in}}%
\pgfpathlineto{\pgfqpoint{3.865101in}{3.090574in}}%
\pgfpathlineto{\pgfqpoint{3.878423in}{3.081701in}}%
\pgfpathlineto{\pgfqpoint{3.886159in}{3.100016in}}%
\pgfpathlineto{\pgfqpoint{3.893891in}{3.118632in}}%
\pgfpathlineto{\pgfqpoint{3.901621in}{3.137558in}}%
\pgfpathlineto{\pgfqpoint{3.909347in}{3.156799in}}%
\pgfpathlineto{\pgfqpoint{3.896028in}{3.166048in}}%
\pgfpathlineto{\pgfqpoint{3.882712in}{3.175393in}}%
\pgfpathlineto{\pgfqpoint{3.869399in}{3.184834in}}%
\pgfpathlineto{\pgfqpoint{3.856088in}{3.194371in}}%
\pgfpathlineto{\pgfqpoint{3.848360in}{3.174744in}}%
\pgfpathlineto{\pgfqpoint{3.840628in}{3.155439in}}%
\pgfpathlineto{\pgfqpoint{3.832893in}{3.136448in}}%
\pgfpathlineto{\pgfqpoint{3.825154in}{3.117764in}}%
\pgfpathclose%
\pgfusepath{fill}%
\end{pgfscope}%
\begin{pgfscope}%
\pgfpathrectangle{\pgfqpoint{1.150000in}{0.150000in}}{\pgfqpoint{5.700000in}{5.700000in}}%
\pgfusepath{clip}%
\pgfsetbuttcap%
\pgfsetroundjoin%
\definecolor{currentfill}{rgb}{0.126326,0.644107,0.525311}%
\pgfsetfillcolor{currentfill}%
\pgfsetfillopacity{0.700000}%
\pgfsetlinewidth{0.000000pt}%
\definecolor{currentstroke}{rgb}{0.000000,0.000000,0.000000}%
\pgfsetstrokecolor{currentstroke}%
\pgfsetdash{}{0pt}%
\pgfpathmoveto{\pgfqpoint{4.209122in}{3.936317in}}%
\pgfpathlineto{\pgfqpoint{4.222466in}{3.923734in}}%
\pgfpathlineto{\pgfqpoint{4.235811in}{3.911245in}}%
\pgfpathlineto{\pgfqpoint{4.249159in}{3.898850in}}%
\pgfpathlineto{\pgfqpoint{4.262508in}{3.886547in}}%
\pgfpathlineto{\pgfqpoint{4.270232in}{3.919798in}}%
\pgfpathlineto{\pgfqpoint{4.277961in}{3.953665in}}%
\pgfpathlineto{\pgfqpoint{4.285695in}{3.988158in}}%
\pgfpathlineto{\pgfqpoint{4.293433in}{4.023291in}}%
\pgfpathlineto{\pgfqpoint{4.280077in}{4.036174in}}%
\pgfpathlineto{\pgfqpoint{4.266722in}{4.049150in}}%
\pgfpathlineto{\pgfqpoint{4.253369in}{4.062220in}}%
\pgfpathlineto{\pgfqpoint{4.240018in}{4.075385in}}%
\pgfpathlineto{\pgfqpoint{4.232288in}{4.039662in}}%
\pgfpathlineto{\pgfqpoint{4.224562in}{4.004584in}}%
\pgfpathlineto{\pgfqpoint{4.216840in}{3.970140in}}%
\pgfpathlineto{\pgfqpoint{4.209122in}{3.936317in}}%
\pgfpathclose%
\pgfusepath{fill}%
\end{pgfscope}%
\begin{pgfscope}%
\pgfpathrectangle{\pgfqpoint{1.150000in}{0.150000in}}{\pgfqpoint{5.700000in}{5.700000in}}%
\pgfusepath{clip}%
\pgfsetbuttcap%
\pgfsetroundjoin%
\definecolor{currentfill}{rgb}{0.195860,0.395433,0.555276}%
\pgfsetfillcolor{currentfill}%
\pgfsetfillopacity{0.700000}%
\pgfsetlinewidth{0.000000pt}%
\definecolor{currentstroke}{rgb}{0.000000,0.000000,0.000000}%
\pgfsetstrokecolor{currentstroke}%
\pgfsetdash{}{0pt}%
\pgfpathmoveto{\pgfqpoint{4.161844in}{3.296203in}}%
\pgfpathlineto{\pgfqpoint{4.175196in}{3.286887in}}%
\pgfpathlineto{\pgfqpoint{4.188550in}{3.277659in}}%
\pgfpathlineto{\pgfqpoint{4.201909in}{3.268519in}}%
\pgfpathlineto{\pgfqpoint{4.215270in}{3.259466in}}%
\pgfpathlineto{\pgfqpoint{4.222964in}{3.281695in}}%
\pgfpathlineto{\pgfqpoint{4.230657in}{3.304327in}}%
\pgfpathlineto{\pgfqpoint{4.238350in}{3.327374in}}%
\pgfpathlineto{\pgfqpoint{4.246044in}{3.350842in}}%
\pgfpathlineto{\pgfqpoint{4.232684in}{3.360355in}}%
\pgfpathlineto{\pgfqpoint{4.219328in}{3.369955in}}%
\pgfpathlineto{\pgfqpoint{4.205974in}{3.379643in}}%
\pgfpathlineto{\pgfqpoint{4.192623in}{3.389419in}}%
\pgfpathlineto{\pgfqpoint{4.184929in}{3.365482in}}%
\pgfpathlineto{\pgfqpoint{4.177234in}{3.341973in}}%
\pgfpathlineto{\pgfqpoint{4.169539in}{3.318883in}}%
\pgfpathlineto{\pgfqpoint{4.161844in}{3.296203in}}%
\pgfpathclose%
\pgfusepath{fill}%
\end{pgfscope}%
\begin{pgfscope}%
\pgfpathrectangle{\pgfqpoint{1.150000in}{0.150000in}}{\pgfqpoint{5.700000in}{5.700000in}}%
\pgfusepath{clip}%
\pgfsetbuttcap%
\pgfsetroundjoin%
\definecolor{currentfill}{rgb}{0.221989,0.339161,0.548752}%
\pgfsetfillcolor{currentfill}%
\pgfsetfillopacity{0.700000}%
\pgfsetlinewidth{0.000000pt}%
\definecolor{currentstroke}{rgb}{0.000000,0.000000,0.000000}%
\pgfsetstrokecolor{currentstroke}%
\pgfsetdash{}{0pt}%
\pgfpathmoveto{\pgfqpoint{3.052958in}{3.172878in}}%
\pgfpathlineto{\pgfqpoint{3.066244in}{3.160590in}}%
\pgfpathlineto{\pgfqpoint{3.079529in}{3.148435in}}%
\pgfpathlineto{\pgfqpoint{3.092814in}{3.136412in}}%
\pgfpathlineto{\pgfqpoint{3.106098in}{3.124520in}}%
\pgfpathlineto{\pgfqpoint{3.114004in}{3.140169in}}%
\pgfpathlineto{\pgfqpoint{3.121902in}{3.156036in}}%
\pgfpathlineto{\pgfqpoint{3.129794in}{3.172127in}}%
\pgfpathlineto{\pgfqpoint{3.137678in}{3.188446in}}%
\pgfpathlineto{\pgfqpoint{3.124398in}{3.200596in}}%
\pgfpathlineto{\pgfqpoint{3.111117in}{3.212877in}}%
\pgfpathlineto{\pgfqpoint{3.097835in}{3.225291in}}%
\pgfpathlineto{\pgfqpoint{3.084552in}{3.237838in}}%
\pgfpathlineto{\pgfqpoint{3.076665in}{3.221252in}}%
\pgfpathlineto{\pgfqpoint{3.068770in}{3.204900in}}%
\pgfpathlineto{\pgfqpoint{3.060868in}{3.188777in}}%
\pgfpathlineto{\pgfqpoint{3.052958in}{3.172878in}}%
\pgfpathclose%
\pgfusepath{fill}%
\end{pgfscope}%
\begin{pgfscope}%
\pgfpathrectangle{\pgfqpoint{1.150000in}{0.150000in}}{\pgfqpoint{5.700000in}{5.700000in}}%
\pgfusepath{clip}%
\pgfsetbuttcap%
\pgfsetroundjoin%
\definecolor{currentfill}{rgb}{0.239346,0.300855,0.540844}%
\pgfsetfillcolor{currentfill}%
\pgfsetfillopacity{0.700000}%
\pgfsetlinewidth{0.000000pt}%
\definecolor{currentstroke}{rgb}{0.000000,0.000000,0.000000}%
\pgfsetstrokecolor{currentstroke}%
\pgfsetdash{}{0pt}%
\pgfpathmoveto{\pgfqpoint{3.381556in}{3.074224in}}%
\pgfpathlineto{\pgfqpoint{3.394836in}{3.063866in}}%
\pgfpathlineto{\pgfqpoint{3.408116in}{3.053620in}}%
\pgfpathlineto{\pgfqpoint{3.421398in}{3.043487in}}%
\pgfpathlineto{\pgfqpoint{3.434682in}{3.033466in}}%
\pgfpathlineto{\pgfqpoint{3.442515in}{3.049552in}}%
\pgfpathlineto{\pgfqpoint{3.450343in}{3.065871in}}%
\pgfpathlineto{\pgfqpoint{3.458165in}{3.082430in}}%
\pgfpathlineto{\pgfqpoint{3.465981in}{3.099234in}}%
\pgfpathlineto{\pgfqpoint{3.452701in}{3.109553in}}%
\pgfpathlineto{\pgfqpoint{3.439422in}{3.119983in}}%
\pgfpathlineto{\pgfqpoint{3.426145in}{3.130525in}}%
\pgfpathlineto{\pgfqpoint{3.412868in}{3.141181in}}%
\pgfpathlineto{\pgfqpoint{3.405049in}{3.124072in}}%
\pgfpathlineto{\pgfqpoint{3.397224in}{3.107214in}}%
\pgfpathlineto{\pgfqpoint{3.389393in}{3.090599in}}%
\pgfpathlineto{\pgfqpoint{3.381556in}{3.074224in}}%
\pgfpathclose%
\pgfusepath{fill}%
\end{pgfscope}%
\begin{pgfscope}%
\pgfpathrectangle{\pgfqpoint{1.150000in}{0.150000in}}{\pgfqpoint{5.700000in}{5.700000in}}%
\pgfusepath{clip}%
\pgfsetbuttcap%
\pgfsetroundjoin%
\definecolor{currentfill}{rgb}{0.129933,0.559582,0.551864}%
\pgfsetfillcolor{currentfill}%
\pgfsetfillopacity{0.700000}%
\pgfsetlinewidth{0.000000pt}%
\definecolor{currentstroke}{rgb}{0.000000,0.000000,0.000000}%
\pgfsetstrokecolor{currentstroke}%
\pgfsetdash{}{0pt}%
\pgfpathmoveto{\pgfqpoint{4.285040in}{3.713396in}}%
\pgfpathlineto{\pgfqpoint{4.298395in}{3.702102in}}%
\pgfpathlineto{\pgfqpoint{4.311753in}{3.690898in}}%
\pgfpathlineto{\pgfqpoint{4.325112in}{3.679783in}}%
\pgfpathlineto{\pgfqpoint{4.338475in}{3.668755in}}%
\pgfpathlineto{\pgfqpoint{4.346193in}{3.698569in}}%
\pgfpathlineto{\pgfqpoint{4.353916in}{3.728941in}}%
\pgfpathlineto{\pgfqpoint{4.361642in}{3.759882in}}%
\pgfpathlineto{\pgfqpoint{4.369373in}{3.791404in}}%
\pgfpathlineto{\pgfqpoint{4.356007in}{3.802983in}}%
\pgfpathlineto{\pgfqpoint{4.342644in}{3.814651in}}%
\pgfpathlineto{\pgfqpoint{4.329282in}{3.826408in}}%
\pgfpathlineto{\pgfqpoint{4.315923in}{3.838254in}}%
\pgfpathlineto{\pgfqpoint{4.308197in}{3.806171in}}%
\pgfpathlineto{\pgfqpoint{4.300475in}{3.774674in}}%
\pgfpathlineto{\pgfqpoint{4.292756in}{3.743752in}}%
\pgfpathlineto{\pgfqpoint{4.285040in}{3.713396in}}%
\pgfpathclose%
\pgfusepath{fill}%
\end{pgfscope}%
\begin{pgfscope}%
\pgfpathrectangle{\pgfqpoint{1.150000in}{0.150000in}}{\pgfqpoint{5.700000in}{5.700000in}}%
\pgfusepath{clip}%
\pgfsetbuttcap%
\pgfsetroundjoin%
\definecolor{currentfill}{rgb}{0.237441,0.305202,0.541921}%
\pgfsetfillcolor{currentfill}%
\pgfsetfillopacity{0.700000}%
\pgfsetlinewidth{0.000000pt}%
\definecolor{currentstroke}{rgb}{0.000000,0.000000,0.000000}%
\pgfsetstrokecolor{currentstroke}%
\pgfsetdash{}{0pt}%
\pgfpathmoveto{\pgfqpoint{3.740921in}{3.082143in}}%
\pgfpathlineto{\pgfqpoint{3.754227in}{3.072950in}}%
\pgfpathlineto{\pgfqpoint{3.767535in}{3.063856in}}%
\pgfpathlineto{\pgfqpoint{3.780846in}{3.054860in}}%
\pgfpathlineto{\pgfqpoint{3.794159in}{3.045960in}}%
\pgfpathlineto{\pgfqpoint{3.801914in}{3.063485in}}%
\pgfpathlineto{\pgfqpoint{3.809665in}{3.081289in}}%
\pgfpathlineto{\pgfqpoint{3.817411in}{3.099380in}}%
\pgfpathlineto{\pgfqpoint{3.825154in}{3.117764in}}%
\pgfpathlineto{\pgfqpoint{3.811844in}{3.127020in}}%
\pgfpathlineto{\pgfqpoint{3.798536in}{3.136373in}}%
\pgfpathlineto{\pgfqpoint{3.785231in}{3.145824in}}%
\pgfpathlineto{\pgfqpoint{3.771928in}{3.155375in}}%
\pgfpathlineto{\pgfqpoint{3.764182in}{3.136626in}}%
\pgfpathlineto{\pgfqpoint{3.756433in}{3.118175in}}%
\pgfpathlineto{\pgfqpoint{3.748679in}{3.100017in}}%
\pgfpathlineto{\pgfqpoint{3.740921in}{3.082143in}}%
\pgfpathclose%
\pgfusepath{fill}%
\end{pgfscope}%
\begin{pgfscope}%
\pgfpathrectangle{\pgfqpoint{1.150000in}{0.150000in}}{\pgfqpoint{5.700000in}{5.700000in}}%
\pgfusepath{clip}%
\pgfsetbuttcap%
\pgfsetroundjoin%
\definecolor{currentfill}{rgb}{0.741388,0.873449,0.149561}%
\pgfsetfillcolor{currentfill}%
\pgfsetfillopacity{0.700000}%
\pgfsetlinewidth{0.000000pt}%
\definecolor{currentstroke}{rgb}{0.000000,0.000000,0.000000}%
\pgfsetstrokecolor{currentstroke}%
\pgfsetdash{}{0pt}%
\pgfpathmoveto{\pgfqpoint{3.515623in}{4.738075in}}%
\pgfpathlineto{\pgfqpoint{3.528991in}{4.719531in}}%
\pgfpathlineto{\pgfqpoint{3.542356in}{4.701129in}}%
\pgfpathlineto{\pgfqpoint{3.555719in}{4.682866in}}%
\pgfpathlineto{\pgfqpoint{3.569078in}{4.664741in}}%
\pgfpathlineto{\pgfqpoint{3.576647in}{4.705615in}}%
\pgfpathlineto{\pgfqpoint{3.584214in}{4.747193in}}%
\pgfpathlineto{\pgfqpoint{3.591779in}{4.789488in}}%
\pgfpathlineto{\pgfqpoint{3.599342in}{4.832513in}}%
\pgfpathlineto{\pgfqpoint{3.585963in}{4.851256in}}%
\pgfpathlineto{\pgfqpoint{3.572582in}{4.870139in}}%
\pgfpathlineto{\pgfqpoint{3.559198in}{4.889162in}}%
\pgfpathlineto{\pgfqpoint{3.545811in}{4.908327in}}%
\pgfpathlineto{\pgfqpoint{3.538268in}{4.864671in}}%
\pgfpathlineto{\pgfqpoint{3.530723in}{4.821752in}}%
\pgfpathlineto{\pgfqpoint{3.523174in}{4.779558in}}%
\pgfpathlineto{\pgfqpoint{3.515623in}{4.738075in}}%
\pgfpathclose%
\pgfusepath{fill}%
\end{pgfscope}%
\begin{pgfscope}%
\pgfpathrectangle{\pgfqpoint{1.150000in}{0.150000in}}{\pgfqpoint{5.700000in}{5.700000in}}%
\pgfusepath{clip}%
\pgfsetbuttcap%
\pgfsetroundjoin%
\definecolor{currentfill}{rgb}{0.235526,0.309527,0.542944}%
\pgfsetfillcolor{currentfill}%
\pgfsetfillopacity{0.700000}%
\pgfsetlinewidth{0.000000pt}%
\definecolor{currentstroke}{rgb}{0.000000,0.000000,0.000000}%
\pgfsetstrokecolor{currentstroke}%
\pgfsetdash{}{0pt}%
\pgfpathmoveto{\pgfqpoint{3.243904in}{3.095836in}}%
\pgfpathlineto{\pgfqpoint{3.257182in}{3.084817in}}%
\pgfpathlineto{\pgfqpoint{3.270461in}{3.073919in}}%
\pgfpathlineto{\pgfqpoint{3.283740in}{3.063141in}}%
\pgfpathlineto{\pgfqpoint{3.297019in}{3.052480in}}%
\pgfpathlineto{\pgfqpoint{3.304885in}{3.068222in}}%
\pgfpathlineto{\pgfqpoint{3.312745in}{3.084187in}}%
\pgfpathlineto{\pgfqpoint{3.320599in}{3.100380in}}%
\pgfpathlineto{\pgfqpoint{3.328445in}{3.116807in}}%
\pgfpathlineto{\pgfqpoint{3.315170in}{3.127744in}}%
\pgfpathlineto{\pgfqpoint{3.301894in}{3.138801in}}%
\pgfpathlineto{\pgfqpoint{3.288619in}{3.149976in}}%
\pgfpathlineto{\pgfqpoint{3.275344in}{3.161273in}}%
\pgfpathlineto{\pgfqpoint{3.267494in}{3.144561in}}%
\pgfpathlineto{\pgfqpoint{3.259638in}{3.128088in}}%
\pgfpathlineto{\pgfqpoint{3.251774in}{3.111848in}}%
\pgfpathlineto{\pgfqpoint{3.243904in}{3.095836in}}%
\pgfpathclose%
\pgfusepath{fill}%
\end{pgfscope}%
\begin{pgfscope}%
\pgfpathrectangle{\pgfqpoint{1.150000in}{0.150000in}}{\pgfqpoint{5.700000in}{5.700000in}}%
\pgfusepath{clip}%
\pgfsetbuttcap%
\pgfsetroundjoin%
\definecolor{currentfill}{rgb}{0.243113,0.292092,0.538516}%
\pgfsetfillcolor{currentfill}%
\pgfsetfillopacity{0.700000}%
\pgfsetlinewidth{0.000000pt}%
\definecolor{currentstroke}{rgb}{0.000000,0.000000,0.000000}%
\pgfsetstrokecolor{currentstroke}%
\pgfsetdash{}{0pt}%
\pgfpathmoveto{\pgfqpoint{3.519114in}{3.059059in}}%
\pgfpathlineto{\pgfqpoint{3.532401in}{3.049285in}}%
\pgfpathlineto{\pgfqpoint{3.545690in}{3.039619in}}%
\pgfpathlineto{\pgfqpoint{3.558980in}{3.030058in}}%
\pgfpathlineto{\pgfqpoint{3.572273in}{3.020603in}}%
\pgfpathlineto{\pgfqpoint{3.580077in}{3.037039in}}%
\pgfpathlineto{\pgfqpoint{3.587875in}{3.053720in}}%
\pgfpathlineto{\pgfqpoint{3.595668in}{3.070654in}}%
\pgfpathlineto{\pgfqpoint{3.603456in}{3.087845in}}%
\pgfpathlineto{\pgfqpoint{3.590166in}{3.097617in}}%
\pgfpathlineto{\pgfqpoint{3.576879in}{3.107495in}}%
\pgfpathlineto{\pgfqpoint{3.563593in}{3.117478in}}%
\pgfpathlineto{\pgfqpoint{3.550309in}{3.127569in}}%
\pgfpathlineto{\pgfqpoint{3.542519in}{3.110052in}}%
\pgfpathlineto{\pgfqpoint{3.534722in}{3.092799in}}%
\pgfpathlineto{\pgfqpoint{3.526921in}{3.075803in}}%
\pgfpathlineto{\pgfqpoint{3.519114in}{3.059059in}}%
\pgfpathclose%
\pgfusepath{fill}%
\end{pgfscope}%
\begin{pgfscope}%
\pgfpathrectangle{\pgfqpoint{1.150000in}{0.150000in}}{\pgfqpoint{5.700000in}{5.700000in}}%
\pgfusepath{clip}%
\pgfsetbuttcap%
\pgfsetroundjoin%
\definecolor{currentfill}{rgb}{0.168126,0.459988,0.558082}%
\pgfsetfillcolor{currentfill}%
\pgfsetfillopacity{0.700000}%
\pgfsetlinewidth{0.000000pt}%
\definecolor{currentstroke}{rgb}{0.000000,0.000000,0.000000}%
\pgfsetstrokecolor{currentstroke}%
\pgfsetdash{}{0pt}%
\pgfpathmoveto{\pgfqpoint{4.276827in}{3.449116in}}%
\pgfpathlineto{\pgfqpoint{4.290189in}{3.439208in}}%
\pgfpathlineto{\pgfqpoint{4.303555in}{3.429388in}}%
\pgfpathlineto{\pgfqpoint{4.316924in}{3.419653in}}%
\pgfpathlineto{\pgfqpoint{4.330296in}{3.410004in}}%
\pgfpathlineto{\pgfqpoint{4.337994in}{3.435224in}}%
\pgfpathlineto{\pgfqpoint{4.345694in}{3.460915in}}%
\pgfpathlineto{\pgfqpoint{4.353397in}{3.487088in}}%
\pgfpathlineto{\pgfqpoint{4.361102in}{3.513752in}}%
\pgfpathlineto{\pgfqpoint{4.347730in}{3.523904in}}%
\pgfpathlineto{\pgfqpoint{4.334361in}{3.534142in}}%
\pgfpathlineto{\pgfqpoint{4.320995in}{3.544467in}}%
\pgfpathlineto{\pgfqpoint{4.307632in}{3.554879in}}%
\pgfpathlineto{\pgfqpoint{4.299927in}{3.527703in}}%
\pgfpathlineto{\pgfqpoint{4.292225in}{3.501023in}}%
\pgfpathlineto{\pgfqpoint{4.284525in}{3.474831in}}%
\pgfpathlineto{\pgfqpoint{4.276827in}{3.449116in}}%
\pgfpathclose%
\pgfusepath{fill}%
\end{pgfscope}%
\begin{pgfscope}%
\pgfpathrectangle{\pgfqpoint{1.150000in}{0.150000in}}{\pgfqpoint{5.700000in}{5.700000in}}%
\pgfusepath{clip}%
\pgfsetbuttcap%
\pgfsetroundjoin%
\definecolor{currentfill}{rgb}{0.352360,0.783011,0.392636}%
\pgfsetfillcolor{currentfill}%
\pgfsetfillopacity{0.700000}%
\pgfsetlinewidth{0.000000pt}%
\definecolor{currentstroke}{rgb}{0.000000,0.000000,0.000000}%
\pgfsetstrokecolor{currentstroke}%
\pgfsetdash{}{0pt}%
\pgfpathmoveto{\pgfqpoint{3.973178in}{4.359503in}}%
\pgfpathlineto{\pgfqpoint{3.986516in}{4.344311in}}%
\pgfpathlineto{\pgfqpoint{3.999854in}{4.329228in}}%
\pgfpathlineto{\pgfqpoint{4.013192in}{4.314252in}}%
\pgfpathlineto{\pgfqpoint{4.026531in}{4.299382in}}%
\pgfpathlineto{\pgfqpoint{4.034222in}{4.338175in}}%
\pgfpathlineto{\pgfqpoint{4.041917in}{4.377665in}}%
\pgfpathlineto{\pgfqpoint{4.049615in}{4.417865in}}%
\pgfpathlineto{\pgfqpoint{4.036267in}{4.433196in}}%
\pgfpathlineto{\pgfqpoint{4.022918in}{4.448635in}}%
\pgfpathlineto{\pgfqpoint{4.009570in}{4.464181in}}%
\pgfpathlineto{\pgfqpoint{3.996222in}{4.479837in}}%
\pgfpathlineto{\pgfqpoint{3.988538in}{4.439014in}}%
\pgfpathlineto{\pgfqpoint{3.980857in}{4.398907in}}%
\pgfpathlineto{\pgfqpoint{3.973178in}{4.359503in}}%
\pgfpathclose%
\pgfusepath{fill}%
\end{pgfscope}%
\begin{pgfscope}%
\pgfpathrectangle{\pgfqpoint{1.150000in}{0.150000in}}{\pgfqpoint{5.700000in}{5.700000in}}%
\pgfusepath{clip}%
\pgfsetbuttcap%
\pgfsetroundjoin%
\definecolor{currentfill}{rgb}{0.404001,0.800275,0.362552}%
\pgfsetfillcolor{currentfill}%
\pgfsetfillopacity{0.700000}%
\pgfsetlinewidth{0.000000pt}%
\definecolor{currentstroke}{rgb}{0.000000,0.000000,0.000000}%
\pgfsetstrokecolor{currentstroke}%
\pgfsetdash{}{0pt}%
\pgfpathmoveto{\pgfqpoint{3.919824in}{4.421370in}}%
\pgfpathlineto{\pgfqpoint{3.933163in}{4.405737in}}%
\pgfpathlineto{\pgfqpoint{3.946502in}{4.390215in}}%
\pgfpathlineto{\pgfqpoint{3.959840in}{4.374804in}}%
\pgfpathlineto{\pgfqpoint{3.973178in}{4.359503in}}%
\pgfpathlineto{\pgfqpoint{3.980857in}{4.398907in}}%
\pgfpathlineto{\pgfqpoint{3.988538in}{4.439014in}}%
\pgfpathlineto{\pgfqpoint{3.996222in}{4.479837in}}%
\pgfpathlineto{\pgfqpoint{3.982873in}{4.495602in}}%
\pgfpathlineto{\pgfqpoint{3.969524in}{4.511478in}}%
\pgfpathlineto{\pgfqpoint{3.956174in}{4.527465in}}%
\pgfpathlineto{\pgfqpoint{3.942824in}{4.543564in}}%
\pgfpathlineto{\pgfqpoint{3.935155in}{4.502113in}}%
\pgfpathlineto{\pgfqpoint{3.927489in}{4.461386in}}%
\pgfpathlineto{\pgfqpoint{3.919824in}{4.421370in}}%
\pgfpathclose%
\pgfusepath{fill}%
\end{pgfscope}%
\begin{pgfscope}%
\pgfpathrectangle{\pgfqpoint{1.150000in}{0.150000in}}{\pgfqpoint{5.700000in}{5.700000in}}%
\pgfusepath{clip}%
\pgfsetbuttcap%
\pgfsetroundjoin%
\definecolor{currentfill}{rgb}{0.304148,0.764704,0.419943}%
\pgfsetfillcolor{currentfill}%
\pgfsetfillopacity{0.700000}%
\pgfsetlinewidth{0.000000pt}%
\definecolor{currentstroke}{rgb}{0.000000,0.000000,0.000000}%
\pgfsetstrokecolor{currentstroke}%
\pgfsetdash{}{0pt}%
\pgfpathmoveto{\pgfqpoint{4.026531in}{4.299382in}}%
\pgfpathlineto{\pgfqpoint{4.039869in}{4.284618in}}%
\pgfpathlineto{\pgfqpoint{4.053207in}{4.269960in}}%
\pgfpathlineto{\pgfqpoint{4.066546in}{4.255406in}}%
\pgfpathlineto{\pgfqpoint{4.079886in}{4.240956in}}%
\pgfpathlineto{\pgfqpoint{4.087590in}{4.279142in}}%
\pgfpathlineto{\pgfqpoint{4.095297in}{4.318018in}}%
\pgfpathlineto{\pgfqpoint{4.103008in}{4.357596in}}%
\pgfpathlineto{\pgfqpoint{4.089659in}{4.372506in}}%
\pgfpathlineto{\pgfqpoint{4.076311in}{4.387521in}}%
\pgfpathlineto{\pgfqpoint{4.062963in}{4.402640in}}%
\pgfpathlineto{\pgfqpoint{4.049615in}{4.417865in}}%
\pgfpathlineto{\pgfqpoint{4.041917in}{4.377665in}}%
\pgfpathlineto{\pgfqpoint{4.034222in}{4.338175in}}%
\pgfpathlineto{\pgfqpoint{4.026531in}{4.299382in}}%
\pgfpathclose%
\pgfusepath{fill}%
\end{pgfscope}%
\begin{pgfscope}%
\pgfpathrectangle{\pgfqpoint{1.150000in}{0.150000in}}{\pgfqpoint{5.700000in}{5.700000in}}%
\pgfusepath{clip}%
\pgfsetbuttcap%
\pgfsetroundjoin%
\definecolor{currentfill}{rgb}{0.151918,0.500685,0.557587}%
\pgfsetfillcolor{currentfill}%
\pgfsetfillopacity{0.700000}%
\pgfsetlinewidth{0.000000pt}%
\definecolor{currentstroke}{rgb}{0.000000,0.000000,0.000000}%
\pgfsetstrokecolor{currentstroke}%
\pgfsetdash{}{0pt}%
\pgfpathmoveto{\pgfqpoint{4.307632in}{3.554879in}}%
\pgfpathlineto{\pgfqpoint{4.320995in}{3.544467in}}%
\pgfpathlineto{\pgfqpoint{4.334361in}{3.534142in}}%
\pgfpathlineto{\pgfqpoint{4.347730in}{3.523904in}}%
\pgfpathlineto{\pgfqpoint{4.361102in}{3.513752in}}%
\pgfpathlineto{\pgfqpoint{4.368809in}{3.540917in}}%
\pgfpathlineto{\pgfqpoint{4.376520in}{3.568593in}}%
\pgfpathlineto{\pgfqpoint{4.384234in}{3.596790in}}%
\pgfpathlineto{\pgfqpoint{4.391951in}{3.625520in}}%
\pgfpathlineto{\pgfqpoint{4.378578in}{3.636198in}}%
\pgfpathlineto{\pgfqpoint{4.365207in}{3.646964in}}%
\pgfpathlineto{\pgfqpoint{4.351840in}{3.657816in}}%
\pgfpathlineto{\pgfqpoint{4.338475in}{3.668755in}}%
\pgfpathlineto{\pgfqpoint{4.330759in}{3.639490in}}%
\pgfpathlineto{\pgfqpoint{4.323047in}{3.610763in}}%
\pgfpathlineto{\pgfqpoint{4.315338in}{3.582562in}}%
\pgfpathlineto{\pgfqpoint{4.307632in}{3.554879in}}%
\pgfpathclose%
\pgfusepath{fill}%
\end{pgfscope}%
\begin{pgfscope}%
\pgfpathrectangle{\pgfqpoint{1.150000in}{0.150000in}}{\pgfqpoint{5.700000in}{5.700000in}}%
\pgfusepath{clip}%
\pgfsetbuttcap%
\pgfsetroundjoin%
\definecolor{currentfill}{rgb}{0.468053,0.818921,0.323998}%
\pgfsetfillcolor{currentfill}%
\pgfsetfillopacity{0.700000}%
\pgfsetlinewidth{0.000000pt}%
\definecolor{currentstroke}{rgb}{0.000000,0.000000,0.000000}%
\pgfsetstrokecolor{currentstroke}%
\pgfsetdash{}{0pt}%
\pgfpathmoveto{\pgfqpoint{3.866462in}{4.485036in}}%
\pgfpathlineto{\pgfqpoint{3.879804in}{4.468948in}}%
\pgfpathlineto{\pgfqpoint{3.893144in}{4.452975in}}%
\pgfpathlineto{\pgfqpoint{3.906485in}{4.437116in}}%
\pgfpathlineto{\pgfqpoint{3.919824in}{4.421370in}}%
\pgfpathlineto{\pgfqpoint{3.927489in}{4.461386in}}%
\pgfpathlineto{\pgfqpoint{3.935155in}{4.502113in}}%
\pgfpathlineto{\pgfqpoint{3.942824in}{4.543564in}}%
\pgfpathlineto{\pgfqpoint{3.929473in}{4.559776in}}%
\pgfpathlineto{\pgfqpoint{3.916121in}{4.576102in}}%
\pgfpathlineto{\pgfqpoint{3.902769in}{4.592543in}}%
\pgfpathlineto{\pgfqpoint{3.889416in}{4.609100in}}%
\pgfpathlineto{\pgfqpoint{3.881763in}{4.567019in}}%
\pgfpathlineto{\pgfqpoint{3.874112in}{4.525669in}}%
\pgfpathlineto{\pgfqpoint{3.866462in}{4.485036in}}%
\pgfpathclose%
\pgfusepath{fill}%
\end{pgfscope}%
\begin{pgfscope}%
\pgfpathrectangle{\pgfqpoint{1.150000in}{0.150000in}}{\pgfqpoint{5.700000in}{5.700000in}}%
\pgfusepath{clip}%
\pgfsetbuttcap%
\pgfsetroundjoin%
\definecolor{currentfill}{rgb}{0.120638,0.625828,0.533488}%
\pgfsetfillcolor{currentfill}%
\pgfsetfillopacity{0.700000}%
\pgfsetlinewidth{0.000000pt}%
\definecolor{currentstroke}{rgb}{0.000000,0.000000,0.000000}%
\pgfsetstrokecolor{currentstroke}%
\pgfsetdash{}{0pt}%
\pgfpathmoveto{\pgfqpoint{4.262508in}{3.886547in}}%
\pgfpathlineto{\pgfqpoint{4.275859in}{3.874337in}}%
\pgfpathlineto{\pgfqpoint{4.289211in}{3.862218in}}%
\pgfpathlineto{\pgfqpoint{4.302566in}{3.850191in}}%
\pgfpathlineto{\pgfqpoint{4.315923in}{3.838254in}}%
\pgfpathlineto{\pgfqpoint{4.323654in}{3.870936in}}%
\pgfpathlineto{\pgfqpoint{4.331389in}{3.904227in}}%
\pgfpathlineto{\pgfqpoint{4.339129in}{3.938138in}}%
\pgfpathlineto{\pgfqpoint{4.346874in}{3.972682in}}%
\pgfpathlineto{\pgfqpoint{4.333511in}{3.985197in}}%
\pgfpathlineto{\pgfqpoint{4.320150in}{3.997803in}}%
\pgfpathlineto{\pgfqpoint{4.306791in}{4.010501in}}%
\pgfpathlineto{\pgfqpoint{4.293433in}{4.023291in}}%
\pgfpathlineto{\pgfqpoint{4.285695in}{3.988158in}}%
\pgfpathlineto{\pgfqpoint{4.277961in}{3.953665in}}%
\pgfpathlineto{\pgfqpoint{4.270232in}{3.919798in}}%
\pgfpathlineto{\pgfqpoint{4.262508in}{3.886547in}}%
\pgfpathclose%
\pgfusepath{fill}%
\end{pgfscope}%
\begin{pgfscope}%
\pgfpathrectangle{\pgfqpoint{1.150000in}{0.150000in}}{\pgfqpoint{5.700000in}{5.700000in}}%
\pgfusepath{clip}%
\pgfsetbuttcap%
\pgfsetroundjoin%
\definecolor{currentfill}{rgb}{0.259857,0.745492,0.444467}%
\pgfsetfillcolor{currentfill}%
\pgfsetfillopacity{0.700000}%
\pgfsetlinewidth{0.000000pt}%
\definecolor{currentstroke}{rgb}{0.000000,0.000000,0.000000}%
\pgfsetstrokecolor{currentstroke}%
\pgfsetdash{}{0pt}%
\pgfpathmoveto{\pgfqpoint{4.079886in}{4.240956in}}%
\pgfpathlineto{\pgfqpoint{4.093226in}{4.226608in}}%
\pgfpathlineto{\pgfqpoint{4.106566in}{4.212363in}}%
\pgfpathlineto{\pgfqpoint{4.119907in}{4.198219in}}%
\pgfpathlineto{\pgfqpoint{4.133249in}{4.184176in}}%
\pgfpathlineto{\pgfqpoint{4.140964in}{4.221758in}}%
\pgfpathlineto{\pgfqpoint{4.148683in}{4.260023in}}%
\pgfpathlineto{\pgfqpoint{4.156406in}{4.298983in}}%
\pgfpathlineto{\pgfqpoint{4.143056in}{4.313484in}}%
\pgfpathlineto{\pgfqpoint{4.129706in}{4.328086in}}%
\pgfpathlineto{\pgfqpoint{4.116357in}{4.342789in}}%
\pgfpathlineto{\pgfqpoint{4.103008in}{4.357596in}}%
\pgfpathlineto{\pgfqpoint{4.095297in}{4.318018in}}%
\pgfpathlineto{\pgfqpoint{4.087590in}{4.279142in}}%
\pgfpathlineto{\pgfqpoint{4.079886in}{4.240956in}}%
\pgfpathclose%
\pgfusepath{fill}%
\end{pgfscope}%
\begin{pgfscope}%
\pgfpathrectangle{\pgfqpoint{1.150000in}{0.150000in}}{\pgfqpoint{5.700000in}{5.700000in}}%
\pgfusepath{clip}%
\pgfsetbuttcap%
\pgfsetroundjoin%
\definecolor{currentfill}{rgb}{0.183898,0.422383,0.556944}%
\pgfsetfillcolor{currentfill}%
\pgfsetfillopacity{0.700000}%
\pgfsetlinewidth{0.000000pt}%
\definecolor{currentstroke}{rgb}{0.000000,0.000000,0.000000}%
\pgfsetstrokecolor{currentstroke}%
\pgfsetdash{}{0pt}%
\pgfpathmoveto{\pgfqpoint{4.246044in}{3.350842in}}%
\pgfpathlineto{\pgfqpoint{4.259407in}{3.341416in}}%
\pgfpathlineto{\pgfqpoint{4.272774in}{3.332077in}}%
\pgfpathlineto{\pgfqpoint{4.286144in}{3.322824in}}%
\pgfpathlineto{\pgfqpoint{4.299517in}{3.313656in}}%
\pgfpathlineto{\pgfqpoint{4.307210in}{3.337082in}}%
\pgfpathlineto{\pgfqpoint{4.314904in}{3.360943in}}%
\pgfpathlineto{\pgfqpoint{4.322599in}{3.385247in}}%
\pgfpathlineto{\pgfqpoint{4.330296in}{3.410004in}}%
\pgfpathlineto{\pgfqpoint{4.316924in}{3.419653in}}%
\pgfpathlineto{\pgfqpoint{4.303555in}{3.429388in}}%
\pgfpathlineto{\pgfqpoint{4.290189in}{3.439208in}}%
\pgfpathlineto{\pgfqpoint{4.276827in}{3.449116in}}%
\pgfpathlineto{\pgfqpoint{4.269130in}{3.423869in}}%
\pgfpathlineto{\pgfqpoint{4.261434in}{3.399080in}}%
\pgfpathlineto{\pgfqpoint{4.253739in}{3.374741in}}%
\pgfpathlineto{\pgfqpoint{4.246044in}{3.350842in}}%
\pgfpathclose%
\pgfusepath{fill}%
\end{pgfscope}%
\begin{pgfscope}%
\pgfpathrectangle{\pgfqpoint{1.150000in}{0.150000in}}{\pgfqpoint{5.700000in}{5.700000in}}%
\pgfusepath{clip}%
\pgfsetbuttcap%
\pgfsetroundjoin%
\definecolor{currentfill}{rgb}{0.535621,0.835785,0.281908}%
\pgfsetfillcolor{currentfill}%
\pgfsetfillopacity{0.700000}%
\pgfsetlinewidth{0.000000pt}%
\definecolor{currentstroke}{rgb}{0.000000,0.000000,0.000000}%
\pgfsetstrokecolor{currentstroke}%
\pgfsetdash{}{0pt}%
\pgfpathmoveto{\pgfqpoint{3.813088in}{4.550559in}}%
\pgfpathlineto{\pgfqpoint{3.826433in}{4.534001in}}%
\pgfpathlineto{\pgfqpoint{3.839777in}{4.517562in}}%
\pgfpathlineto{\pgfqpoint{3.853120in}{4.501241in}}%
\pgfpathlineto{\pgfqpoint{3.866462in}{4.485036in}}%
\pgfpathlineto{\pgfqpoint{3.874112in}{4.525669in}}%
\pgfpathlineto{\pgfqpoint{3.881763in}{4.567019in}}%
\pgfpathlineto{\pgfqpoint{3.889416in}{4.609100in}}%
\pgfpathlineto{\pgfqpoint{3.876061in}{4.625774in}}%
\pgfpathlineto{\pgfqpoint{3.862706in}{4.642565in}}%
\pgfpathlineto{\pgfqpoint{3.849349in}{4.659474in}}%
\pgfpathlineto{\pgfqpoint{3.835992in}{4.676504in}}%
\pgfpathlineto{\pgfqpoint{3.828356in}{4.633788in}}%
\pgfpathlineto{\pgfqpoint{3.820721in}{4.591811in}}%
\pgfpathlineto{\pgfqpoint{3.813088in}{4.550559in}}%
\pgfpathclose%
\pgfusepath{fill}%
\end{pgfscope}%
\begin{pgfscope}%
\pgfpathrectangle{\pgfqpoint{1.150000in}{0.150000in}}{\pgfqpoint{5.700000in}{5.700000in}}%
\pgfusepath{clip}%
\pgfsetbuttcap%
\pgfsetroundjoin%
\definecolor{currentfill}{rgb}{0.243113,0.292092,0.538516}%
\pgfsetfillcolor{currentfill}%
\pgfsetfillopacity{0.700000}%
\pgfsetlinewidth{0.000000pt}%
\definecolor{currentstroke}{rgb}{0.000000,0.000000,0.000000}%
\pgfsetstrokecolor{currentstroke}%
\pgfsetdash{}{0pt}%
\pgfpathmoveto{\pgfqpoint{3.656632in}{3.049795in}}%
\pgfpathlineto{\pgfqpoint{3.669932in}{3.040538in}}%
\pgfpathlineto{\pgfqpoint{3.683234in}{3.031383in}}%
\pgfpathlineto{\pgfqpoint{3.696538in}{3.022328in}}%
\pgfpathlineto{\pgfqpoint{3.709845in}{3.013372in}}%
\pgfpathlineto{\pgfqpoint{3.717621in}{3.030169in}}%
\pgfpathlineto{\pgfqpoint{3.725392in}{3.047226in}}%
\pgfpathlineto{\pgfqpoint{3.733159in}{3.064548in}}%
\pgfpathlineto{\pgfqpoint{3.740921in}{3.082143in}}%
\pgfpathlineto{\pgfqpoint{3.727618in}{3.091435in}}%
\pgfpathlineto{\pgfqpoint{3.714317in}{3.100827in}}%
\pgfpathlineto{\pgfqpoint{3.701018in}{3.110319in}}%
\pgfpathlineto{\pgfqpoint{3.687722in}{3.119913in}}%
\pgfpathlineto{\pgfqpoint{3.679957in}{3.101973in}}%
\pgfpathlineto{\pgfqpoint{3.672187in}{3.084311in}}%
\pgfpathlineto{\pgfqpoint{3.664412in}{3.066921in}}%
\pgfpathlineto{\pgfqpoint{3.656632in}{3.049795in}}%
\pgfpathclose%
\pgfusepath{fill}%
\end{pgfscope}%
\begin{pgfscope}%
\pgfpathrectangle{\pgfqpoint{1.150000in}{0.150000in}}{\pgfqpoint{5.700000in}{5.700000in}}%
\pgfusepath{clip}%
\pgfsetbuttcap%
\pgfsetroundjoin%
\definecolor{currentfill}{rgb}{0.229739,0.322361,0.545706}%
\pgfsetfillcolor{currentfill}%
\pgfsetfillopacity{0.700000}%
\pgfsetlinewidth{0.000000pt}%
\definecolor{currentstroke}{rgb}{0.000000,0.000000,0.000000}%
\pgfsetstrokecolor{currentstroke}%
\pgfsetdash{}{0pt}%
\pgfpathmoveto{\pgfqpoint{3.106098in}{3.124520in}}%
\pgfpathlineto{\pgfqpoint{3.119381in}{3.112758in}}%
\pgfpathlineto{\pgfqpoint{3.132663in}{3.101126in}}%
\pgfpathlineto{\pgfqpoint{3.145946in}{3.089620in}}%
\pgfpathlineto{\pgfqpoint{3.159228in}{3.078242in}}%
\pgfpathlineto{\pgfqpoint{3.167130in}{3.093640in}}%
\pgfpathlineto{\pgfqpoint{3.175024in}{3.109252in}}%
\pgfpathlineto{\pgfqpoint{3.182912in}{3.125082in}}%
\pgfpathlineto{\pgfqpoint{3.190793in}{3.141136in}}%
\pgfpathlineto{\pgfqpoint{3.177515in}{3.152772in}}%
\pgfpathlineto{\pgfqpoint{3.164236in}{3.164535in}}%
\pgfpathlineto{\pgfqpoint{3.150957in}{3.176426in}}%
\pgfpathlineto{\pgfqpoint{3.137678in}{3.188446in}}%
\pgfpathlineto{\pgfqpoint{3.129794in}{3.172127in}}%
\pgfpathlineto{\pgfqpoint{3.121902in}{3.156036in}}%
\pgfpathlineto{\pgfqpoint{3.114004in}{3.140169in}}%
\pgfpathlineto{\pgfqpoint{3.106098in}{3.124520in}}%
\pgfpathclose%
\pgfusepath{fill}%
\end{pgfscope}%
\begin{pgfscope}%
\pgfpathrectangle{\pgfqpoint{1.150000in}{0.150000in}}{\pgfqpoint{5.700000in}{5.700000in}}%
\pgfusepath{clip}%
\pgfsetbuttcap%
\pgfsetroundjoin%
\definecolor{currentfill}{rgb}{0.220057,0.343307,0.549413}%
\pgfsetfillcolor{currentfill}%
\pgfsetfillopacity{0.700000}%
\pgfsetlinewidth{0.000000pt}%
\definecolor{currentstroke}{rgb}{0.000000,0.000000,0.000000}%
\pgfsetstrokecolor{currentstroke}%
\pgfsetdash{}{0pt}%
\pgfpathmoveto{\pgfqpoint{4.046855in}{3.163244in}}%
\pgfpathlineto{\pgfqpoint{4.060198in}{3.154428in}}%
\pgfpathlineto{\pgfqpoint{4.073544in}{3.145703in}}%
\pgfpathlineto{\pgfqpoint{4.086894in}{3.137066in}}%
\pgfpathlineto{\pgfqpoint{4.100247in}{3.128518in}}%
\pgfpathlineto{\pgfqpoint{4.107952in}{3.148217in}}%
\pgfpathlineto{\pgfqpoint{4.115655in}{3.168260in}}%
\pgfpathlineto{\pgfqpoint{4.123356in}{3.188656in}}%
\pgfpathlineto{\pgfqpoint{4.131056in}{3.209412in}}%
\pgfpathlineto{\pgfqpoint{4.117706in}{3.218378in}}%
\pgfpathlineto{\pgfqpoint{4.104358in}{3.227432in}}%
\pgfpathlineto{\pgfqpoint{4.091015in}{3.236576in}}%
\pgfpathlineto{\pgfqpoint{4.077674in}{3.245810in}}%
\pgfpathlineto{\pgfqpoint{4.069972in}{3.224628in}}%
\pgfpathlineto{\pgfqpoint{4.062268in}{3.203812in}}%
\pgfpathlineto{\pgfqpoint{4.054562in}{3.183353in}}%
\pgfpathlineto{\pgfqpoint{4.046855in}{3.163244in}}%
\pgfpathclose%
\pgfusepath{fill}%
\end{pgfscope}%
\begin{pgfscope}%
\pgfpathrectangle{\pgfqpoint{1.150000in}{0.150000in}}{\pgfqpoint{5.700000in}{5.700000in}}%
\pgfusepath{clip}%
\pgfsetbuttcap%
\pgfsetroundjoin%
\definecolor{currentfill}{rgb}{0.227802,0.326594,0.546532}%
\pgfsetfillcolor{currentfill}%
\pgfsetfillopacity{0.700000}%
\pgfsetlinewidth{0.000000pt}%
\definecolor{currentstroke}{rgb}{0.000000,0.000000,0.000000}%
\pgfsetstrokecolor{currentstroke}%
\pgfsetdash{}{0pt}%
\pgfpathmoveto{\pgfqpoint{3.962649in}{3.120739in}}%
\pgfpathlineto{\pgfqpoint{3.975983in}{3.111957in}}%
\pgfpathlineto{\pgfqpoint{3.989319in}{3.103266in}}%
\pgfpathlineto{\pgfqpoint{4.002659in}{3.094666in}}%
\pgfpathlineto{\pgfqpoint{4.016002in}{3.086157in}}%
\pgfpathlineto{\pgfqpoint{4.023719in}{3.104942in}}%
\pgfpathlineto{\pgfqpoint{4.031433in}{3.124046in}}%
\pgfpathlineto{\pgfqpoint{4.039145in}{3.143478in}}%
\pgfpathlineto{\pgfqpoint{4.046855in}{3.163244in}}%
\pgfpathlineto{\pgfqpoint{4.033515in}{3.172150in}}%
\pgfpathlineto{\pgfqpoint{4.020178in}{3.181147in}}%
\pgfpathlineto{\pgfqpoint{4.006845in}{3.190236in}}%
\pgfpathlineto{\pgfqpoint{3.993514in}{3.199416in}}%
\pgfpathlineto{\pgfqpoint{3.985802in}{3.179244in}}%
\pgfpathlineto{\pgfqpoint{3.978087in}{3.159413in}}%
\pgfpathlineto{\pgfqpoint{3.970370in}{3.139913in}}%
\pgfpathlineto{\pgfqpoint{3.962649in}{3.120739in}}%
\pgfpathclose%
\pgfusepath{fill}%
\end{pgfscope}%
\begin{pgfscope}%
\pgfpathrectangle{\pgfqpoint{1.150000in}{0.150000in}}{\pgfqpoint{5.700000in}{5.700000in}}%
\pgfusepath{clip}%
\pgfsetbuttcap%
\pgfsetroundjoin%
\definecolor{currentfill}{rgb}{0.226397,0.728888,0.462789}%
\pgfsetfillcolor{currentfill}%
\pgfsetfillopacity{0.700000}%
\pgfsetlinewidth{0.000000pt}%
\definecolor{currentstroke}{rgb}{0.000000,0.000000,0.000000}%
\pgfsetstrokecolor{currentstroke}%
\pgfsetdash{}{0pt}%
\pgfpathmoveto{\pgfqpoint{4.133249in}{4.184176in}}%
\pgfpathlineto{\pgfqpoint{4.146591in}{4.170234in}}%
\pgfpathlineto{\pgfqpoint{4.159935in}{4.156391in}}%
\pgfpathlineto{\pgfqpoint{4.173279in}{4.142646in}}%
\pgfpathlineto{\pgfqpoint{4.186625in}{4.129000in}}%
\pgfpathlineto{\pgfqpoint{4.194350in}{4.165980in}}%
\pgfpathlineto{\pgfqpoint{4.202080in}{4.203636in}}%
\pgfpathlineto{\pgfqpoint{4.209814in}{4.241981in}}%
\pgfpathlineto{\pgfqpoint{4.196461in}{4.256083in}}%
\pgfpathlineto{\pgfqpoint{4.183109in}{4.270283in}}%
\pgfpathlineto{\pgfqpoint{4.169757in}{4.284583in}}%
\pgfpathlineto{\pgfqpoint{4.156406in}{4.298983in}}%
\pgfpathlineto{\pgfqpoint{4.148683in}{4.260023in}}%
\pgfpathlineto{\pgfqpoint{4.140964in}{4.221758in}}%
\pgfpathlineto{\pgfqpoint{4.133249in}{4.184176in}}%
\pgfpathclose%
\pgfusepath{fill}%
\end{pgfscope}%
\begin{pgfscope}%
\pgfpathrectangle{\pgfqpoint{1.150000in}{0.150000in}}{\pgfqpoint{5.700000in}{5.700000in}}%
\pgfusepath{clip}%
\pgfsetbuttcap%
\pgfsetroundjoin%
\definecolor{currentfill}{rgb}{0.824940,0.884720,0.106217}%
\pgfsetfillcolor{currentfill}%
\pgfsetfillopacity{0.700000}%
\pgfsetlinewidth{0.000000pt}%
\definecolor{currentstroke}{rgb}{0.000000,0.000000,0.000000}%
\pgfsetstrokecolor{currentstroke}%
\pgfsetdash{}{0pt}%
\pgfpathmoveto{\pgfqpoint{3.462121in}{4.813684in}}%
\pgfpathlineto{\pgfqpoint{3.475501in}{4.794563in}}%
\pgfpathlineto{\pgfqpoint{3.488879in}{4.775589in}}%
\pgfpathlineto{\pgfqpoint{3.502253in}{4.756760in}}%
\pgfpathlineto{\pgfqpoint{3.515623in}{4.738075in}}%
\pgfpathlineto{\pgfqpoint{3.523174in}{4.779558in}}%
\pgfpathlineto{\pgfqpoint{3.530723in}{4.821752in}}%
\pgfpathlineto{\pgfqpoint{3.538268in}{4.864671in}}%
\pgfpathlineto{\pgfqpoint{3.545811in}{4.908327in}}%
\pgfpathlineto{\pgfqpoint{3.532421in}{4.927636in}}%
\pgfpathlineto{\pgfqpoint{3.519027in}{4.947089in}}%
\pgfpathlineto{\pgfqpoint{3.505630in}{4.966687in}}%
\pgfpathlineto{\pgfqpoint{3.492230in}{4.986434in}}%
\pgfpathlineto{\pgfqpoint{3.484707in}{4.942142in}}%
\pgfpathlineto{\pgfqpoint{3.477182in}{4.898595in}}%
\pgfpathlineto{\pgfqpoint{3.469653in}{4.855780in}}%
\pgfpathlineto{\pgfqpoint{3.462121in}{4.813684in}}%
\pgfpathclose%
\pgfusepath{fill}%
\end{pgfscope}%
\begin{pgfscope}%
\pgfpathrectangle{\pgfqpoint{1.150000in}{0.150000in}}{\pgfqpoint{5.700000in}{5.700000in}}%
\pgfusepath{clip}%
\pgfsetbuttcap%
\pgfsetroundjoin%
\definecolor{currentfill}{rgb}{0.210503,0.363727,0.552206}%
\pgfsetfillcolor{currentfill}%
\pgfsetfillopacity{0.700000}%
\pgfsetlinewidth{0.000000pt}%
\definecolor{currentstroke}{rgb}{0.000000,0.000000,0.000000}%
\pgfsetstrokecolor{currentstroke}%
\pgfsetdash{}{0pt}%
\pgfpathmoveto{\pgfqpoint{4.131056in}{3.209412in}}%
\pgfpathlineto{\pgfqpoint{4.144409in}{3.200535in}}%
\pgfpathlineto{\pgfqpoint{4.157766in}{3.191746in}}%
\pgfpathlineto{\pgfqpoint{4.171127in}{3.183045in}}%
\pgfpathlineto{\pgfqpoint{4.184491in}{3.174431in}}%
\pgfpathlineto{\pgfqpoint{4.192187in}{3.195125in}}%
\pgfpathlineto{\pgfqpoint{4.199882in}{3.216190in}}%
\pgfpathlineto{\pgfqpoint{4.207576in}{3.237634in}}%
\pgfpathlineto{\pgfqpoint{4.215270in}{3.259466in}}%
\pgfpathlineto{\pgfqpoint{4.201909in}{3.268519in}}%
\pgfpathlineto{\pgfqpoint{4.188550in}{3.277659in}}%
\pgfpathlineto{\pgfqpoint{4.175196in}{3.286887in}}%
\pgfpathlineto{\pgfqpoint{4.161844in}{3.296203in}}%
\pgfpathlineto{\pgfqpoint{4.154148in}{3.273924in}}%
\pgfpathlineto{\pgfqpoint{4.146452in}{3.252038in}}%
\pgfpathlineto{\pgfqpoint{4.138754in}{3.230537in}}%
\pgfpathlineto{\pgfqpoint{4.131056in}{3.209412in}}%
\pgfpathclose%
\pgfusepath{fill}%
\end{pgfscope}%
\begin{pgfscope}%
\pgfpathrectangle{\pgfqpoint{1.150000in}{0.150000in}}{\pgfqpoint{5.700000in}{5.700000in}}%
\pgfusepath{clip}%
\pgfsetbuttcap%
\pgfsetroundjoin%
\definecolor{currentfill}{rgb}{0.606045,0.850733,0.236712}%
\pgfsetfillcolor{currentfill}%
\pgfsetfillopacity{0.700000}%
\pgfsetlinewidth{0.000000pt}%
\definecolor{currentstroke}{rgb}{0.000000,0.000000,0.000000}%
\pgfsetstrokecolor{currentstroke}%
\pgfsetdash{}{0pt}%
\pgfpathmoveto{\pgfqpoint{3.759694in}{4.618001in}}%
\pgfpathlineto{\pgfqpoint{3.773045in}{4.600957in}}%
\pgfpathlineto{\pgfqpoint{3.786394in}{4.584036in}}%
\pgfpathlineto{\pgfqpoint{3.799741in}{4.567238in}}%
\pgfpathlineto{\pgfqpoint{3.813088in}{4.550559in}}%
\pgfpathlineto{\pgfqpoint{3.820721in}{4.591811in}}%
\pgfpathlineto{\pgfqpoint{3.828356in}{4.633788in}}%
\pgfpathlineto{\pgfqpoint{3.835992in}{4.676504in}}%
\pgfpathlineto{\pgfqpoint{3.822633in}{4.693653in}}%
\pgfpathlineto{\pgfqpoint{3.809272in}{4.710925in}}%
\pgfpathlineto{\pgfqpoint{3.795910in}{4.728319in}}%
\pgfpathlineto{\pgfqpoint{3.782546in}{4.745836in}}%
\pgfpathlineto{\pgfqpoint{3.774929in}{4.702483in}}%
\pgfpathlineto{\pgfqpoint{3.767311in}{4.659876in}}%
\pgfpathlineto{\pgfqpoint{3.759694in}{4.618001in}}%
\pgfpathclose%
\pgfusepath{fill}%
\end{pgfscope}%
\begin{pgfscope}%
\pgfpathrectangle{\pgfqpoint{1.150000in}{0.150000in}}{\pgfqpoint{5.700000in}{5.700000in}}%
\pgfusepath{clip}%
\pgfsetbuttcap%
\pgfsetroundjoin%
\definecolor{currentfill}{rgb}{0.135066,0.544853,0.554029}%
\pgfsetfillcolor{currentfill}%
\pgfsetfillopacity{0.700000}%
\pgfsetlinewidth{0.000000pt}%
\definecolor{currentstroke}{rgb}{0.000000,0.000000,0.000000}%
\pgfsetstrokecolor{currentstroke}%
\pgfsetdash{}{0pt}%
\pgfpathmoveto{\pgfqpoint{4.338475in}{3.668755in}}%
\pgfpathlineto{\pgfqpoint{4.351840in}{3.657816in}}%
\pgfpathlineto{\pgfqpoint{4.365207in}{3.646964in}}%
\pgfpathlineto{\pgfqpoint{4.378578in}{3.636198in}}%
\pgfpathlineto{\pgfqpoint{4.391951in}{3.625520in}}%
\pgfpathlineto{\pgfqpoint{4.399672in}{3.654791in}}%
\pgfpathlineto{\pgfqpoint{4.407397in}{3.684615in}}%
\pgfpathlineto{\pgfqpoint{4.415127in}{3.715003in}}%
\pgfpathlineto{\pgfqpoint{4.422862in}{3.745965in}}%
\pgfpathlineto{\pgfqpoint{4.409486in}{3.757194in}}%
\pgfpathlineto{\pgfqpoint{4.396112in}{3.768510in}}%
\pgfpathlineto{\pgfqpoint{4.382742in}{3.779913in}}%
\pgfpathlineto{\pgfqpoint{4.369373in}{3.791404in}}%
\pgfpathlineto{\pgfqpoint{4.361642in}{3.759882in}}%
\pgfpathlineto{\pgfqpoint{4.353916in}{3.728941in}}%
\pgfpathlineto{\pgfqpoint{4.346193in}{3.698569in}}%
\pgfpathlineto{\pgfqpoint{4.338475in}{3.668755in}}%
\pgfpathclose%
\pgfusepath{fill}%
\end{pgfscope}%
\begin{pgfscope}%
\pgfpathrectangle{\pgfqpoint{1.150000in}{0.150000in}}{\pgfqpoint{5.700000in}{5.700000in}}%
\pgfusepath{clip}%
\pgfsetbuttcap%
\pgfsetroundjoin%
\definecolor{currentfill}{rgb}{0.235526,0.309527,0.542944}%
\pgfsetfillcolor{currentfill}%
\pgfsetfillopacity{0.700000}%
\pgfsetlinewidth{0.000000pt}%
\definecolor{currentstroke}{rgb}{0.000000,0.000000,0.000000}%
\pgfsetstrokecolor{currentstroke}%
\pgfsetdash{}{0pt}%
\pgfpathmoveto{\pgfqpoint{3.878423in}{3.081701in}}%
\pgfpathlineto{\pgfqpoint{3.891747in}{3.072923in}}%
\pgfpathlineto{\pgfqpoint{3.905074in}{3.064239in}}%
\pgfpathlineto{\pgfqpoint{3.918405in}{3.055648in}}%
\pgfpathlineto{\pgfqpoint{3.931738in}{3.047150in}}%
\pgfpathlineto{\pgfqpoint{3.939471in}{3.065095in}}%
\pgfpathlineto{\pgfqpoint{3.947200in}{3.083337in}}%
\pgfpathlineto{\pgfqpoint{3.954926in}{3.101883in}}%
\pgfpathlineto{\pgfqpoint{3.962649in}{3.120739in}}%
\pgfpathlineto{\pgfqpoint{3.949319in}{3.129614in}}%
\pgfpathlineto{\pgfqpoint{3.935992in}{3.138582in}}%
\pgfpathlineto{\pgfqpoint{3.922668in}{3.147643in}}%
\pgfpathlineto{\pgfqpoint{3.909347in}{3.156799in}}%
\pgfpathlineto{\pgfqpoint{3.901621in}{3.137558in}}%
\pgfpathlineto{\pgfqpoint{3.893891in}{3.118632in}}%
\pgfpathlineto{\pgfqpoint{3.886159in}{3.100016in}}%
\pgfpathlineto{\pgfqpoint{3.878423in}{3.081701in}}%
\pgfpathclose%
\pgfusepath{fill}%
\end{pgfscope}%
\begin{pgfscope}%
\pgfpathrectangle{\pgfqpoint{1.150000in}{0.150000in}}{\pgfqpoint{5.700000in}{5.700000in}}%
\pgfusepath{clip}%
\pgfsetbuttcap%
\pgfsetroundjoin%
\definecolor{currentfill}{rgb}{0.191090,0.708366,0.482284}%
\pgfsetfillcolor{currentfill}%
\pgfsetfillopacity{0.700000}%
\pgfsetlinewidth{0.000000pt}%
\definecolor{currentstroke}{rgb}{0.000000,0.000000,0.000000}%
\pgfsetstrokecolor{currentstroke}%
\pgfsetdash{}{0pt}%
\pgfpathmoveto{\pgfqpoint{4.186625in}{4.129000in}}%
\pgfpathlineto{\pgfqpoint{4.199971in}{4.115451in}}%
\pgfpathlineto{\pgfqpoint{4.213319in}{4.102000in}}%
\pgfpathlineto{\pgfqpoint{4.226668in}{4.088644in}}%
\pgfpathlineto{\pgfqpoint{4.240018in}{4.075385in}}%
\pgfpathlineto{\pgfqpoint{4.247753in}{4.111765in}}%
\pgfpathlineto{\pgfqpoint{4.255492in}{4.148815in}}%
\pgfpathlineto{\pgfqpoint{4.263237in}{4.186547in}}%
\pgfpathlineto{\pgfqpoint{4.249880in}{4.200260in}}%
\pgfpathlineto{\pgfqpoint{4.236524in}{4.214070in}}%
\pgfpathlineto{\pgfqpoint{4.223169in}{4.227977in}}%
\pgfpathlineto{\pgfqpoint{4.209814in}{4.241981in}}%
\pgfpathlineto{\pgfqpoint{4.202080in}{4.203636in}}%
\pgfpathlineto{\pgfqpoint{4.194350in}{4.165980in}}%
\pgfpathlineto{\pgfqpoint{4.186625in}{4.129000in}}%
\pgfpathclose%
\pgfusepath{fill}%
\end{pgfscope}%
\begin{pgfscope}%
\pgfpathrectangle{\pgfqpoint{1.150000in}{0.150000in}}{\pgfqpoint{5.700000in}{5.700000in}}%
\pgfusepath{clip}%
\pgfsetbuttcap%
\pgfsetroundjoin%
\definecolor{currentfill}{rgb}{0.246811,0.283237,0.535941}%
\pgfsetfillcolor{currentfill}%
\pgfsetfillopacity{0.700000}%
\pgfsetlinewidth{0.000000pt}%
\definecolor{currentstroke}{rgb}{0.000000,0.000000,0.000000}%
\pgfsetstrokecolor{currentstroke}%
\pgfsetdash{}{0pt}%
\pgfpathmoveto{\pgfqpoint{3.434682in}{3.033466in}}%
\pgfpathlineto{\pgfqpoint{3.447966in}{3.023555in}}%
\pgfpathlineto{\pgfqpoint{3.461252in}{3.013754in}}%
\pgfpathlineto{\pgfqpoint{3.474540in}{3.004062in}}%
\pgfpathlineto{\pgfqpoint{3.487829in}{2.994478in}}%
\pgfpathlineto{\pgfqpoint{3.495659in}{3.010274in}}%
\pgfpathlineto{\pgfqpoint{3.503483in}{3.026300in}}%
\pgfpathlineto{\pgfqpoint{3.511301in}{3.042559in}}%
\pgfpathlineto{\pgfqpoint{3.519114in}{3.059059in}}%
\pgfpathlineto{\pgfqpoint{3.505828in}{3.068939in}}%
\pgfpathlineto{\pgfqpoint{3.492544in}{3.078928in}}%
\pgfpathlineto{\pgfqpoint{3.479262in}{3.089026in}}%
\pgfpathlineto{\pgfqpoint{3.465981in}{3.099234in}}%
\pgfpathlineto{\pgfqpoint{3.458165in}{3.082430in}}%
\pgfpathlineto{\pgfqpoint{3.450343in}{3.065871in}}%
\pgfpathlineto{\pgfqpoint{3.442515in}{3.049552in}}%
\pgfpathlineto{\pgfqpoint{3.434682in}{3.033466in}}%
\pgfpathclose%
\pgfusepath{fill}%
\end{pgfscope}%
\begin{pgfscope}%
\pgfpathrectangle{\pgfqpoint{1.150000in}{0.150000in}}{\pgfqpoint{5.700000in}{5.700000in}}%
\pgfusepath{clip}%
\pgfsetbuttcap%
\pgfsetroundjoin%
\definecolor{currentfill}{rgb}{0.243113,0.292092,0.538516}%
\pgfsetfillcolor{currentfill}%
\pgfsetfillopacity{0.700000}%
\pgfsetlinewidth{0.000000pt}%
\definecolor{currentstroke}{rgb}{0.000000,0.000000,0.000000}%
\pgfsetstrokecolor{currentstroke}%
\pgfsetdash{}{0pt}%
\pgfpathmoveto{\pgfqpoint{3.297019in}{3.052480in}}%
\pgfpathlineto{\pgfqpoint{3.310299in}{3.041938in}}%
\pgfpathlineto{\pgfqpoint{3.323580in}{3.031512in}}%
\pgfpathlineto{\pgfqpoint{3.336862in}{3.021201in}}%
\pgfpathlineto{\pgfqpoint{3.350145in}{3.011006in}}%
\pgfpathlineto{\pgfqpoint{3.358007in}{3.026478in}}%
\pgfpathlineto{\pgfqpoint{3.365863in}{3.042169in}}%
\pgfpathlineto{\pgfqpoint{3.373712in}{3.058082in}}%
\pgfpathlineto{\pgfqpoint{3.381556in}{3.074224in}}%
\pgfpathlineto{\pgfqpoint{3.368277in}{3.084696in}}%
\pgfpathlineto{\pgfqpoint{3.354999in}{3.095284in}}%
\pgfpathlineto{\pgfqpoint{3.341722in}{3.105987in}}%
\pgfpathlineto{\pgfqpoint{3.328445in}{3.116807in}}%
\pgfpathlineto{\pgfqpoint{3.320599in}{3.100380in}}%
\pgfpathlineto{\pgfqpoint{3.312745in}{3.084187in}}%
\pgfpathlineto{\pgfqpoint{3.304885in}{3.068222in}}%
\pgfpathlineto{\pgfqpoint{3.297019in}{3.052480in}}%
\pgfpathclose%
\pgfusepath{fill}%
\end{pgfscope}%
\begin{pgfscope}%
\pgfpathrectangle{\pgfqpoint{1.150000in}{0.150000in}}{\pgfqpoint{5.700000in}{5.700000in}}%
\pgfusepath{clip}%
\pgfsetbuttcap%
\pgfsetroundjoin%
\definecolor{currentfill}{rgb}{0.119512,0.607464,0.540218}%
\pgfsetfillcolor{currentfill}%
\pgfsetfillopacity{0.700000}%
\pgfsetlinewidth{0.000000pt}%
\definecolor{currentstroke}{rgb}{0.000000,0.000000,0.000000}%
\pgfsetstrokecolor{currentstroke}%
\pgfsetdash{}{0pt}%
\pgfpathmoveto{\pgfqpoint{4.315923in}{3.838254in}}%
\pgfpathlineto{\pgfqpoint{4.329282in}{3.826408in}}%
\pgfpathlineto{\pgfqpoint{4.342644in}{3.814651in}}%
\pgfpathlineto{\pgfqpoint{4.356007in}{3.802983in}}%
\pgfpathlineto{\pgfqpoint{4.369373in}{3.791404in}}%
\pgfpathlineto{\pgfqpoint{4.377109in}{3.823518in}}%
\pgfpathlineto{\pgfqpoint{4.384849in}{3.856235in}}%
\pgfpathlineto{\pgfqpoint{4.392595in}{3.889566in}}%
\pgfpathlineto{\pgfqpoint{4.400346in}{3.923524in}}%
\pgfpathlineto{\pgfqpoint{4.386975in}{3.935679in}}%
\pgfpathlineto{\pgfqpoint{4.373606in}{3.947924in}}%
\pgfpathlineto{\pgfqpoint{4.360239in}{3.960258in}}%
\pgfpathlineto{\pgfqpoint{4.346874in}{3.972682in}}%
\pgfpathlineto{\pgfqpoint{4.339129in}{3.938138in}}%
\pgfpathlineto{\pgfqpoint{4.331389in}{3.904227in}}%
\pgfpathlineto{\pgfqpoint{4.323654in}{3.870936in}}%
\pgfpathlineto{\pgfqpoint{4.315923in}{3.838254in}}%
\pgfpathclose%
\pgfusepath{fill}%
\end{pgfscope}%
\begin{pgfscope}%
\pgfpathrectangle{\pgfqpoint{1.150000in}{0.150000in}}{\pgfqpoint{5.700000in}{5.700000in}}%
\pgfusepath{clip}%
\pgfsetbuttcap%
\pgfsetroundjoin%
\definecolor{currentfill}{rgb}{0.199430,0.387607,0.554642}%
\pgfsetfillcolor{currentfill}%
\pgfsetfillopacity{0.700000}%
\pgfsetlinewidth{0.000000pt}%
\definecolor{currentstroke}{rgb}{0.000000,0.000000,0.000000}%
\pgfsetstrokecolor{currentstroke}%
\pgfsetdash{}{0pt}%
\pgfpathmoveto{\pgfqpoint{4.215270in}{3.259466in}}%
\pgfpathlineto{\pgfqpoint{4.228635in}{3.250500in}}%
\pgfpathlineto{\pgfqpoint{4.242003in}{3.241621in}}%
\pgfpathlineto{\pgfqpoint{4.255375in}{3.232827in}}%
\pgfpathlineto{\pgfqpoint{4.268751in}{3.224119in}}%
\pgfpathlineto{\pgfqpoint{4.276442in}{3.245896in}}%
\pgfpathlineto{\pgfqpoint{4.284133in}{3.268072in}}%
\pgfpathlineto{\pgfqpoint{4.291825in}{3.290656in}}%
\pgfpathlineto{\pgfqpoint{4.299517in}{3.313656in}}%
\pgfpathlineto{\pgfqpoint{4.286144in}{3.322824in}}%
\pgfpathlineto{\pgfqpoint{4.272774in}{3.332077in}}%
\pgfpathlineto{\pgfqpoint{4.259407in}{3.341416in}}%
\pgfpathlineto{\pgfqpoint{4.246044in}{3.350842in}}%
\pgfpathlineto{\pgfqpoint{4.238350in}{3.327374in}}%
\pgfpathlineto{\pgfqpoint{4.230657in}{3.304327in}}%
\pgfpathlineto{\pgfqpoint{4.222964in}{3.281695in}}%
\pgfpathlineto{\pgfqpoint{4.215270in}{3.259466in}}%
\pgfpathclose%
\pgfusepath{fill}%
\end{pgfscope}%
\begin{pgfscope}%
\pgfpathrectangle{\pgfqpoint{1.150000in}{0.150000in}}{\pgfqpoint{5.700000in}{5.700000in}}%
\pgfusepath{clip}%
\pgfsetbuttcap%
\pgfsetroundjoin%
\definecolor{currentfill}{rgb}{0.678489,0.863742,0.189503}%
\pgfsetfillcolor{currentfill}%
\pgfsetfillopacity{0.700000}%
\pgfsetlinewidth{0.000000pt}%
\definecolor{currentstroke}{rgb}{0.000000,0.000000,0.000000}%
\pgfsetstrokecolor{currentstroke}%
\pgfsetdash{}{0pt}%
\pgfpathmoveto{\pgfqpoint{3.706276in}{4.687426in}}%
\pgfpathlineto{\pgfqpoint{3.719634in}{4.669880in}}%
\pgfpathlineto{\pgfqpoint{3.732989in}{4.652462in}}%
\pgfpathlineto{\pgfqpoint{3.746342in}{4.635169in}}%
\pgfpathlineto{\pgfqpoint{3.759694in}{4.618001in}}%
\pgfpathlineto{\pgfqpoint{3.767311in}{4.659876in}}%
\pgfpathlineto{\pgfqpoint{3.774929in}{4.702483in}}%
\pgfpathlineto{\pgfqpoint{3.782546in}{4.745836in}}%
\pgfpathlineto{\pgfqpoint{3.769181in}{4.763478in}}%
\pgfpathlineto{\pgfqpoint{3.755814in}{4.781247in}}%
\pgfpathlineto{\pgfqpoint{3.742445in}{4.799141in}}%
\pgfpathlineto{\pgfqpoint{3.729074in}{4.817164in}}%
\pgfpathlineto{\pgfqpoint{3.721475in}{4.773169in}}%
\pgfpathlineto{\pgfqpoint{3.713876in}{4.729928in}}%
\pgfpathlineto{\pgfqpoint{3.706276in}{4.687426in}}%
\pgfpathclose%
\pgfusepath{fill}%
\end{pgfscope}%
\begin{pgfscope}%
\pgfpathrectangle{\pgfqpoint{1.150000in}{0.150000in}}{\pgfqpoint{5.700000in}{5.700000in}}%
\pgfusepath{clip}%
\pgfsetbuttcap%
\pgfsetroundjoin%
\definecolor{currentfill}{rgb}{0.248629,0.278775,0.534556}%
\pgfsetfillcolor{currentfill}%
\pgfsetfillopacity{0.700000}%
\pgfsetlinewidth{0.000000pt}%
\definecolor{currentstroke}{rgb}{0.000000,0.000000,0.000000}%
\pgfsetstrokecolor{currentstroke}%
\pgfsetdash{}{0pt}%
\pgfpathmoveto{\pgfqpoint{3.572273in}{3.020603in}}%
\pgfpathlineto{\pgfqpoint{3.585568in}{3.011252in}}%
\pgfpathlineto{\pgfqpoint{3.598864in}{3.002005in}}%
\pgfpathlineto{\pgfqpoint{3.612163in}{2.992861in}}%
\pgfpathlineto{\pgfqpoint{3.625464in}{2.983819in}}%
\pgfpathlineto{\pgfqpoint{3.633264in}{2.999946in}}%
\pgfpathlineto{\pgfqpoint{3.641058in}{3.016313in}}%
\pgfpathlineto{\pgfqpoint{3.648848in}{3.032928in}}%
\pgfpathlineto{\pgfqpoint{3.656632in}{3.049795in}}%
\pgfpathlineto{\pgfqpoint{3.643335in}{3.059153in}}%
\pgfpathlineto{\pgfqpoint{3.630040in}{3.068614in}}%
\pgfpathlineto{\pgfqpoint{3.616747in}{3.078177in}}%
\pgfpathlineto{\pgfqpoint{3.603456in}{3.087845in}}%
\pgfpathlineto{\pgfqpoint{3.595668in}{3.070654in}}%
\pgfpathlineto{\pgfqpoint{3.587875in}{3.053720in}}%
\pgfpathlineto{\pgfqpoint{3.580077in}{3.037039in}}%
\pgfpathlineto{\pgfqpoint{3.572273in}{3.020603in}}%
\pgfpathclose%
\pgfusepath{fill}%
\end{pgfscope}%
\begin{pgfscope}%
\pgfpathrectangle{\pgfqpoint{1.150000in}{0.150000in}}{\pgfqpoint{5.700000in}{5.700000in}}%
\pgfusepath{clip}%
\pgfsetbuttcap%
\pgfsetroundjoin%
\definecolor{currentfill}{rgb}{0.243113,0.292092,0.538516}%
\pgfsetfillcolor{currentfill}%
\pgfsetfillopacity{0.700000}%
\pgfsetlinewidth{0.000000pt}%
\definecolor{currentstroke}{rgb}{0.000000,0.000000,0.000000}%
\pgfsetstrokecolor{currentstroke}%
\pgfsetdash{}{0pt}%
\pgfpathmoveto{\pgfqpoint{3.794159in}{3.045960in}}%
\pgfpathlineto{\pgfqpoint{3.807476in}{3.037158in}}%
\pgfpathlineto{\pgfqpoint{3.820795in}{3.028452in}}%
\pgfpathlineto{\pgfqpoint{3.834117in}{3.019841in}}%
\pgfpathlineto{\pgfqpoint{3.847442in}{3.011325in}}%
\pgfpathlineto{\pgfqpoint{3.855193in}{3.028500in}}%
\pgfpathlineto{\pgfqpoint{3.862940in}{3.045950in}}%
\pgfpathlineto{\pgfqpoint{3.870683in}{3.063682in}}%
\pgfpathlineto{\pgfqpoint{3.878423in}{3.081701in}}%
\pgfpathlineto{\pgfqpoint{3.865101in}{3.090574in}}%
\pgfpathlineto{\pgfqpoint{3.851783in}{3.099541in}}%
\pgfpathlineto{\pgfqpoint{3.838467in}{3.108604in}}%
\pgfpathlineto{\pgfqpoint{3.825154in}{3.117764in}}%
\pgfpathlineto{\pgfqpoint{3.817411in}{3.099380in}}%
\pgfpathlineto{\pgfqpoint{3.809665in}{3.081289in}}%
\pgfpathlineto{\pgfqpoint{3.801914in}{3.063485in}}%
\pgfpathlineto{\pgfqpoint{3.794159in}{3.045960in}}%
\pgfpathclose%
\pgfusepath{fill}%
\end{pgfscope}%
\begin{pgfscope}%
\pgfpathrectangle{\pgfqpoint{1.150000in}{0.150000in}}{\pgfqpoint{5.700000in}{5.700000in}}%
\pgfusepath{clip}%
\pgfsetbuttcap%
\pgfsetroundjoin%
\definecolor{currentfill}{rgb}{0.157729,0.485932,0.558013}%
\pgfsetfillcolor{currentfill}%
\pgfsetfillopacity{0.700000}%
\pgfsetlinewidth{0.000000pt}%
\definecolor{currentstroke}{rgb}{0.000000,0.000000,0.000000}%
\pgfsetstrokecolor{currentstroke}%
\pgfsetdash{}{0pt}%
\pgfpathmoveto{\pgfqpoint{4.361102in}{3.513752in}}%
\pgfpathlineto{\pgfqpoint{4.374477in}{3.503685in}}%
\pgfpathlineto{\pgfqpoint{4.387855in}{3.493704in}}%
\pgfpathlineto{\pgfqpoint{4.401236in}{3.483807in}}%
\pgfpathlineto{\pgfqpoint{4.414621in}{3.473995in}}%
\pgfpathlineto{\pgfqpoint{4.422329in}{3.500643in}}%
\pgfpathlineto{\pgfqpoint{4.430040in}{3.527796in}}%
\pgfpathlineto{\pgfqpoint{4.437755in}{3.555465in}}%
\pgfpathlineto{\pgfqpoint{4.445473in}{3.583659in}}%
\pgfpathlineto{\pgfqpoint{4.432088in}{3.593997in}}%
\pgfpathlineto{\pgfqpoint{4.418706in}{3.604419in}}%
\pgfpathlineto{\pgfqpoint{4.405327in}{3.614927in}}%
\pgfpathlineto{\pgfqpoint{4.391951in}{3.625520in}}%
\pgfpathlineto{\pgfqpoint{4.384234in}{3.596790in}}%
\pgfpathlineto{\pgfqpoint{4.376520in}{3.568593in}}%
\pgfpathlineto{\pgfqpoint{4.368809in}{3.540917in}}%
\pgfpathlineto{\pgfqpoint{4.361102in}{3.513752in}}%
\pgfpathclose%
\pgfusepath{fill}%
\end{pgfscope}%
\begin{pgfscope}%
\pgfpathrectangle{\pgfqpoint{1.150000in}{0.150000in}}{\pgfqpoint{5.700000in}{5.700000in}}%
\pgfusepath{clip}%
\pgfsetbuttcap%
\pgfsetroundjoin%
\definecolor{currentfill}{rgb}{0.174274,0.445044,0.557792}%
\pgfsetfillcolor{currentfill}%
\pgfsetfillopacity{0.700000}%
\pgfsetlinewidth{0.000000pt}%
\definecolor{currentstroke}{rgb}{0.000000,0.000000,0.000000}%
\pgfsetstrokecolor{currentstroke}%
\pgfsetdash{}{0pt}%
\pgfpathmoveto{\pgfqpoint{4.330296in}{3.410004in}}%
\pgfpathlineto{\pgfqpoint{4.343671in}{3.400441in}}%
\pgfpathlineto{\pgfqpoint{4.357050in}{3.390963in}}%
\pgfpathlineto{\pgfqpoint{4.370432in}{3.381569in}}%
\pgfpathlineto{\pgfqpoint{4.383817in}{3.372259in}}%
\pgfpathlineto{\pgfqpoint{4.391515in}{3.396984in}}%
\pgfpathlineto{\pgfqpoint{4.399214in}{3.422175in}}%
\pgfpathlineto{\pgfqpoint{4.406916in}{3.447842in}}%
\pgfpathlineto{\pgfqpoint{4.414621in}{3.473995in}}%
\pgfpathlineto{\pgfqpoint{4.401236in}{3.483807in}}%
\pgfpathlineto{\pgfqpoint{4.387855in}{3.493704in}}%
\pgfpathlineto{\pgfqpoint{4.374477in}{3.503685in}}%
\pgfpathlineto{\pgfqpoint{4.361102in}{3.513752in}}%
\pgfpathlineto{\pgfqpoint{4.353397in}{3.487088in}}%
\pgfpathlineto{\pgfqpoint{4.345694in}{3.460915in}}%
\pgfpathlineto{\pgfqpoint{4.337994in}{3.435224in}}%
\pgfpathlineto{\pgfqpoint{4.330296in}{3.410004in}}%
\pgfpathclose%
\pgfusepath{fill}%
\end{pgfscope}%
\begin{pgfscope}%
\pgfpathrectangle{\pgfqpoint{1.150000in}{0.150000in}}{\pgfqpoint{5.700000in}{5.700000in}}%
\pgfusepath{clip}%
\pgfsetbuttcap%
\pgfsetroundjoin%
\definecolor{currentfill}{rgb}{0.166383,0.690856,0.496502}%
\pgfsetfillcolor{currentfill}%
\pgfsetfillopacity{0.700000}%
\pgfsetlinewidth{0.000000pt}%
\definecolor{currentstroke}{rgb}{0.000000,0.000000,0.000000}%
\pgfsetstrokecolor{currentstroke}%
\pgfsetdash{}{0pt}%
\pgfpathmoveto{\pgfqpoint{4.240018in}{4.075385in}}%
\pgfpathlineto{\pgfqpoint{4.253369in}{4.062220in}}%
\pgfpathlineto{\pgfqpoint{4.266722in}{4.049150in}}%
\pgfpathlineto{\pgfqpoint{4.280077in}{4.036174in}}%
\pgfpathlineto{\pgfqpoint{4.293433in}{4.023291in}}%
\pgfpathlineto{\pgfqpoint{4.301176in}{4.059074in}}%
\pgfpathlineto{\pgfqpoint{4.308925in}{4.095520in}}%
\pgfpathlineto{\pgfqpoint{4.316679in}{4.132642in}}%
\pgfpathlineto{\pgfqpoint{4.303316in}{4.145977in}}%
\pgfpathlineto{\pgfqpoint{4.289955in}{4.159405in}}%
\pgfpathlineto{\pgfqpoint{4.276596in}{4.172929in}}%
\pgfpathlineto{\pgfqpoint{4.263237in}{4.186547in}}%
\pgfpathlineto{\pgfqpoint{4.255492in}{4.148815in}}%
\pgfpathlineto{\pgfqpoint{4.247753in}{4.111765in}}%
\pgfpathlineto{\pgfqpoint{4.240018in}{4.075385in}}%
\pgfpathclose%
\pgfusepath{fill}%
\end{pgfscope}%
\begin{pgfscope}%
\pgfpathrectangle{\pgfqpoint{1.150000in}{0.150000in}}{\pgfqpoint{5.700000in}{5.700000in}}%
\pgfusepath{clip}%
\pgfsetbuttcap%
\pgfsetroundjoin%
\definecolor{currentfill}{rgb}{0.237441,0.305202,0.541921}%
\pgfsetfillcolor{currentfill}%
\pgfsetfillopacity{0.700000}%
\pgfsetlinewidth{0.000000pt}%
\definecolor{currentstroke}{rgb}{0.000000,0.000000,0.000000}%
\pgfsetstrokecolor{currentstroke}%
\pgfsetdash{}{0pt}%
\pgfpathmoveto{\pgfqpoint{3.159228in}{3.078242in}}%
\pgfpathlineto{\pgfqpoint{3.172510in}{3.066988in}}%
\pgfpathlineto{\pgfqpoint{3.185792in}{3.055860in}}%
\pgfpathlineto{\pgfqpoint{3.199073in}{3.044854in}}%
\pgfpathlineto{\pgfqpoint{3.212356in}{3.033971in}}%
\pgfpathlineto{\pgfqpoint{3.220253in}{3.049120in}}%
\pgfpathlineto{\pgfqpoint{3.228143in}{3.064477in}}%
\pgfpathlineto{\pgfqpoint{3.236027in}{3.080048in}}%
\pgfpathlineto{\pgfqpoint{3.243904in}{3.095836in}}%
\pgfpathlineto{\pgfqpoint{3.230626in}{3.106976in}}%
\pgfpathlineto{\pgfqpoint{3.217348in}{3.118239in}}%
\pgfpathlineto{\pgfqpoint{3.204071in}{3.129625in}}%
\pgfpathlineto{\pgfqpoint{3.190793in}{3.141136in}}%
\pgfpathlineto{\pgfqpoint{3.182912in}{3.125082in}}%
\pgfpathlineto{\pgfqpoint{3.175024in}{3.109252in}}%
\pgfpathlineto{\pgfqpoint{3.167130in}{3.093640in}}%
\pgfpathlineto{\pgfqpoint{3.159228in}{3.078242in}}%
\pgfpathclose%
\pgfusepath{fill}%
\end{pgfscope}%
\begin{pgfscope}%
\pgfpathrectangle{\pgfqpoint{1.150000in}{0.150000in}}{\pgfqpoint{5.700000in}{5.700000in}}%
\pgfusepath{clip}%
\pgfsetbuttcap%
\pgfsetroundjoin%
\definecolor{currentfill}{rgb}{0.221989,0.339161,0.548752}%
\pgfsetfillcolor{currentfill}%
\pgfsetfillopacity{0.700000}%
\pgfsetlinewidth{0.000000pt}%
\definecolor{currentstroke}{rgb}{0.000000,0.000000,0.000000}%
\pgfsetstrokecolor{currentstroke}%
\pgfsetdash{}{0pt}%
\pgfpathmoveto{\pgfqpoint{2.968067in}{3.160990in}}%
\pgfpathlineto{\pgfqpoint{2.981363in}{3.148394in}}%
\pgfpathlineto{\pgfqpoint{2.994657in}{3.135936in}}%
\pgfpathlineto{\pgfqpoint{3.007950in}{3.123615in}}%
\pgfpathlineto{\pgfqpoint{3.021242in}{3.111429in}}%
\pgfpathlineto{\pgfqpoint{3.029182in}{3.126479in}}%
\pgfpathlineto{\pgfqpoint{3.037115in}{3.141733in}}%
\pgfpathlineto{\pgfqpoint{3.045040in}{3.157199in}}%
\pgfpathlineto{\pgfqpoint{3.052958in}{3.172878in}}%
\pgfpathlineto{\pgfqpoint{3.039670in}{3.185301in}}%
\pgfpathlineto{\pgfqpoint{3.026382in}{3.197860in}}%
\pgfpathlineto{\pgfqpoint{3.013092in}{3.210557in}}%
\pgfpathlineto{\pgfqpoint{2.999800in}{3.223391in}}%
\pgfpathlineto{\pgfqpoint{2.991879in}{3.207466in}}%
\pgfpathlineto{\pgfqpoint{2.983949in}{3.191760in}}%
\pgfpathlineto{\pgfqpoint{2.976012in}{3.176269in}}%
\pgfpathlineto{\pgfqpoint{2.968067in}{3.160990in}}%
\pgfpathclose%
\pgfusepath{fill}%
\end{pgfscope}%
\begin{pgfscope}%
\pgfpathrectangle{\pgfqpoint{1.150000in}{0.150000in}}{\pgfqpoint{5.700000in}{5.700000in}}%
\pgfusepath{clip}%
\pgfsetbuttcap%
\pgfsetroundjoin%
\definecolor{currentfill}{rgb}{0.751884,0.874951,0.143228}%
\pgfsetfillcolor{currentfill}%
\pgfsetfillopacity{0.700000}%
\pgfsetlinewidth{0.000000pt}%
\definecolor{currentstroke}{rgb}{0.000000,0.000000,0.000000}%
\pgfsetstrokecolor{currentstroke}%
\pgfsetdash{}{0pt}%
\pgfpathmoveto{\pgfqpoint{3.652828in}{4.758906in}}%
\pgfpathlineto{\pgfqpoint{3.666193in}{4.740839in}}%
\pgfpathlineto{\pgfqpoint{3.679556in}{4.722905in}}%
\pgfpathlineto{\pgfqpoint{3.692917in}{4.705101in}}%
\pgfpathlineto{\pgfqpoint{3.706276in}{4.687426in}}%
\pgfpathlineto{\pgfqpoint{3.713876in}{4.729928in}}%
\pgfpathlineto{\pgfqpoint{3.721475in}{4.773169in}}%
\pgfpathlineto{\pgfqpoint{3.729074in}{4.817164in}}%
\pgfpathlineto{\pgfqpoint{3.715701in}{4.835316in}}%
\pgfpathlineto{\pgfqpoint{3.702325in}{4.853598in}}%
\pgfpathlineto{\pgfqpoint{3.688948in}{4.872012in}}%
\pgfpathlineto{\pgfqpoint{3.675568in}{4.890558in}}%
\pgfpathlineto{\pgfqpoint{3.667988in}{4.845917in}}%
\pgfpathlineto{\pgfqpoint{3.660408in}{4.802038in}}%
\pgfpathlineto{\pgfqpoint{3.652828in}{4.758906in}}%
\pgfpathclose%
\pgfusepath{fill}%
\end{pgfscope}%
\begin{pgfscope}%
\pgfpathrectangle{\pgfqpoint{1.150000in}{0.150000in}}{\pgfqpoint{5.700000in}{5.700000in}}%
\pgfusepath{clip}%
\pgfsetbuttcap%
\pgfsetroundjoin%
\definecolor{currentfill}{rgb}{0.188923,0.410910,0.556326}%
\pgfsetfillcolor{currentfill}%
\pgfsetfillopacity{0.700000}%
\pgfsetlinewidth{0.000000pt}%
\definecolor{currentstroke}{rgb}{0.000000,0.000000,0.000000}%
\pgfsetstrokecolor{currentstroke}%
\pgfsetdash{}{0pt}%
\pgfpathmoveto{\pgfqpoint{4.299517in}{3.313656in}}%
\pgfpathlineto{\pgfqpoint{4.312894in}{3.304574in}}%
\pgfpathlineto{\pgfqpoint{4.326274in}{3.295577in}}%
\pgfpathlineto{\pgfqpoint{4.339658in}{3.286663in}}%
\pgfpathlineto{\pgfqpoint{4.353046in}{3.277834in}}%
\pgfpathlineto{\pgfqpoint{4.360736in}{3.300788in}}%
\pgfpathlineto{\pgfqpoint{4.368428in}{3.324171in}}%
\pgfpathlineto{\pgfqpoint{4.376122in}{3.347991in}}%
\pgfpathlineto{\pgfqpoint{4.383817in}{3.372259in}}%
\pgfpathlineto{\pgfqpoint{4.370432in}{3.381569in}}%
\pgfpathlineto{\pgfqpoint{4.357050in}{3.390963in}}%
\pgfpathlineto{\pgfqpoint{4.343671in}{3.400441in}}%
\pgfpathlineto{\pgfqpoint{4.330296in}{3.410004in}}%
\pgfpathlineto{\pgfqpoint{4.322599in}{3.385247in}}%
\pgfpathlineto{\pgfqpoint{4.314904in}{3.360943in}}%
\pgfpathlineto{\pgfqpoint{4.307210in}{3.337082in}}%
\pgfpathlineto{\pgfqpoint{4.299517in}{3.313656in}}%
\pgfpathclose%
\pgfusepath{fill}%
\end{pgfscope}%
\begin{pgfscope}%
\pgfpathrectangle{\pgfqpoint{1.150000in}{0.150000in}}{\pgfqpoint{5.700000in}{5.700000in}}%
\pgfusepath{clip}%
\pgfsetbuttcap%
\pgfsetroundjoin%
\definecolor{currentfill}{rgb}{0.140536,0.530132,0.555659}%
\pgfsetfillcolor{currentfill}%
\pgfsetfillopacity{0.700000}%
\pgfsetlinewidth{0.000000pt}%
\definecolor{currentstroke}{rgb}{0.000000,0.000000,0.000000}%
\pgfsetstrokecolor{currentstroke}%
\pgfsetdash{}{0pt}%
\pgfpathmoveto{\pgfqpoint{4.391951in}{3.625520in}}%
\pgfpathlineto{\pgfqpoint{4.405327in}{3.614927in}}%
\pgfpathlineto{\pgfqpoint{4.418706in}{3.604419in}}%
\pgfpathlineto{\pgfqpoint{4.432088in}{3.593997in}}%
\pgfpathlineto{\pgfqpoint{4.445473in}{3.583659in}}%
\pgfpathlineto{\pgfqpoint{4.453196in}{3.612390in}}%
\pgfpathlineto{\pgfqpoint{4.460923in}{3.641668in}}%
\pgfpathlineto{\pgfqpoint{4.468655in}{3.671504in}}%
\pgfpathlineto{\pgfqpoint{4.476392in}{3.701907in}}%
\pgfpathlineto{\pgfqpoint{4.463006in}{3.712794in}}%
\pgfpathlineto{\pgfqpoint{4.449621in}{3.723766in}}%
\pgfpathlineto{\pgfqpoint{4.436240in}{3.734822in}}%
\pgfpathlineto{\pgfqpoint{4.422862in}{3.745965in}}%
\pgfpathlineto{\pgfqpoint{4.415127in}{3.715003in}}%
\pgfpathlineto{\pgfqpoint{4.407397in}{3.684615in}}%
\pgfpathlineto{\pgfqpoint{4.399672in}{3.654791in}}%
\pgfpathlineto{\pgfqpoint{4.391951in}{3.625520in}}%
\pgfpathclose%
\pgfusepath{fill}%
\end{pgfscope}%
\begin{pgfscope}%
\pgfpathrectangle{\pgfqpoint{1.150000in}{0.150000in}}{\pgfqpoint{5.700000in}{5.700000in}}%
\pgfusepath{clip}%
\pgfsetbuttcap%
\pgfsetroundjoin%
\definecolor{currentfill}{rgb}{0.121148,0.592739,0.544641}%
\pgfsetfillcolor{currentfill}%
\pgfsetfillopacity{0.700000}%
\pgfsetlinewidth{0.000000pt}%
\definecolor{currentstroke}{rgb}{0.000000,0.000000,0.000000}%
\pgfsetstrokecolor{currentstroke}%
\pgfsetdash{}{0pt}%
\pgfpathmoveto{\pgfqpoint{4.369373in}{3.791404in}}%
\pgfpathlineto{\pgfqpoint{4.382742in}{3.779913in}}%
\pgfpathlineto{\pgfqpoint{4.396112in}{3.768510in}}%
\pgfpathlineto{\pgfqpoint{4.409486in}{3.757194in}}%
\pgfpathlineto{\pgfqpoint{4.422862in}{3.745965in}}%
\pgfpathlineto{\pgfqpoint{4.430601in}{3.777513in}}%
\pgfpathlineto{\pgfqpoint{4.438346in}{3.809658in}}%
\pgfpathlineto{\pgfqpoint{4.446096in}{3.842411in}}%
\pgfpathlineto{\pgfqpoint{4.453853in}{3.875783in}}%
\pgfpathlineto{\pgfqpoint{4.440472in}{3.887587in}}%
\pgfpathlineto{\pgfqpoint{4.427095in}{3.899478in}}%
\pgfpathlineto{\pgfqpoint{4.413719in}{3.911457in}}%
\pgfpathlineto{\pgfqpoint{4.400346in}{3.923524in}}%
\pgfpathlineto{\pgfqpoint{4.392595in}{3.889566in}}%
\pgfpathlineto{\pgfqpoint{4.384849in}{3.856235in}}%
\pgfpathlineto{\pgfqpoint{4.377109in}{3.823518in}}%
\pgfpathlineto{\pgfqpoint{4.369373in}{3.791404in}}%
\pgfpathclose%
\pgfusepath{fill}%
\end{pgfscope}%
\begin{pgfscope}%
\pgfpathrectangle{\pgfqpoint{1.150000in}{0.150000in}}{\pgfqpoint{5.700000in}{5.700000in}}%
\pgfusepath{clip}%
\pgfsetbuttcap%
\pgfsetroundjoin%
\definecolor{currentfill}{rgb}{0.248629,0.278775,0.534556}%
\pgfsetfillcolor{currentfill}%
\pgfsetfillopacity{0.700000}%
\pgfsetlinewidth{0.000000pt}%
\definecolor{currentstroke}{rgb}{0.000000,0.000000,0.000000}%
\pgfsetstrokecolor{currentstroke}%
\pgfsetdash{}{0pt}%
\pgfpathmoveto{\pgfqpoint{3.709845in}{3.013372in}}%
\pgfpathlineto{\pgfqpoint{3.723154in}{3.004516in}}%
\pgfpathlineto{\pgfqpoint{3.736466in}{2.995758in}}%
\pgfpathlineto{\pgfqpoint{3.749780in}{2.987098in}}%
\pgfpathlineto{\pgfqpoint{3.763098in}{2.978535in}}%
\pgfpathlineto{\pgfqpoint{3.770870in}{2.995004in}}%
\pgfpathlineto{\pgfqpoint{3.778637in}{3.011726in}}%
\pgfpathlineto{\pgfqpoint{3.786400in}{3.028710in}}%
\pgfpathlineto{\pgfqpoint{3.794159in}{3.045960in}}%
\pgfpathlineto{\pgfqpoint{3.780846in}{3.054860in}}%
\pgfpathlineto{\pgfqpoint{3.767535in}{3.063856in}}%
\pgfpathlineto{\pgfqpoint{3.754227in}{3.072950in}}%
\pgfpathlineto{\pgfqpoint{3.740921in}{3.082143in}}%
\pgfpathlineto{\pgfqpoint{3.733159in}{3.064548in}}%
\pgfpathlineto{\pgfqpoint{3.725392in}{3.047226in}}%
\pgfpathlineto{\pgfqpoint{3.717621in}{3.030169in}}%
\pgfpathlineto{\pgfqpoint{3.709845in}{3.013372in}}%
\pgfpathclose%
\pgfusepath{fill}%
\end{pgfscope}%
\begin{pgfscope}%
\pgfpathrectangle{\pgfqpoint{1.150000in}{0.150000in}}{\pgfqpoint{5.700000in}{5.700000in}}%
\pgfusepath{clip}%
\pgfsetbuttcap%
\pgfsetroundjoin%
\definecolor{currentfill}{rgb}{0.146616,0.673050,0.508936}%
\pgfsetfillcolor{currentfill}%
\pgfsetfillopacity{0.700000}%
\pgfsetlinewidth{0.000000pt}%
\definecolor{currentstroke}{rgb}{0.000000,0.000000,0.000000}%
\pgfsetstrokecolor{currentstroke}%
\pgfsetdash{}{0pt}%
\pgfpathmoveto{\pgfqpoint{4.293433in}{4.023291in}}%
\pgfpathlineto{\pgfqpoint{4.306791in}{4.010501in}}%
\pgfpathlineto{\pgfqpoint{4.320150in}{3.997803in}}%
\pgfpathlineto{\pgfqpoint{4.333511in}{3.985197in}}%
\pgfpathlineto{\pgfqpoint{4.346874in}{3.972682in}}%
\pgfpathlineto{\pgfqpoint{4.354625in}{4.007870in}}%
\pgfpathlineto{\pgfqpoint{4.362381in}{4.043715in}}%
\pgfpathlineto{\pgfqpoint{4.370144in}{4.080228in}}%
\pgfpathlineto{\pgfqpoint{4.356775in}{4.093193in}}%
\pgfpathlineto{\pgfqpoint{4.343408in}{4.106250in}}%
\pgfpathlineto{\pgfqpoint{4.330043in}{4.119400in}}%
\pgfpathlineto{\pgfqpoint{4.316679in}{4.132642in}}%
\pgfpathlineto{\pgfqpoint{4.308925in}{4.095520in}}%
\pgfpathlineto{\pgfqpoint{4.301176in}{4.059074in}}%
\pgfpathlineto{\pgfqpoint{4.293433in}{4.023291in}}%
\pgfpathclose%
\pgfusepath{fill}%
\end{pgfscope}%
\begin{pgfscope}%
\pgfpathrectangle{\pgfqpoint{1.150000in}{0.150000in}}{\pgfqpoint{5.700000in}{5.700000in}}%
\pgfusepath{clip}%
\pgfsetbuttcap%
\pgfsetroundjoin%
\definecolor{currentfill}{rgb}{0.225863,0.330805,0.547314}%
\pgfsetfillcolor{currentfill}%
\pgfsetfillopacity{0.700000}%
\pgfsetlinewidth{0.000000pt}%
\definecolor{currentstroke}{rgb}{0.000000,0.000000,0.000000}%
\pgfsetstrokecolor{currentstroke}%
\pgfsetdash{}{0pt}%
\pgfpathmoveto{\pgfqpoint{4.100247in}{3.128518in}}%
\pgfpathlineto{\pgfqpoint{4.113604in}{3.120059in}}%
\pgfpathlineto{\pgfqpoint{4.126964in}{3.111687in}}%
\pgfpathlineto{\pgfqpoint{4.140328in}{3.103403in}}%
\pgfpathlineto{\pgfqpoint{4.153695in}{3.095206in}}%
\pgfpathlineto{\pgfqpoint{4.161396in}{3.114496in}}%
\pgfpathlineto{\pgfqpoint{4.169096in}{3.134124in}}%
\pgfpathlineto{\pgfqpoint{4.176794in}{3.154100in}}%
\pgfpathlineto{\pgfqpoint{4.184491in}{3.174431in}}%
\pgfpathlineto{\pgfqpoint{4.171127in}{3.183045in}}%
\pgfpathlineto{\pgfqpoint{4.157766in}{3.191746in}}%
\pgfpathlineto{\pgfqpoint{4.144409in}{3.200535in}}%
\pgfpathlineto{\pgfqpoint{4.131056in}{3.209412in}}%
\pgfpathlineto{\pgfqpoint{4.123356in}{3.188656in}}%
\pgfpathlineto{\pgfqpoint{4.115655in}{3.168260in}}%
\pgfpathlineto{\pgfqpoint{4.107952in}{3.148217in}}%
\pgfpathlineto{\pgfqpoint{4.100247in}{3.128518in}}%
\pgfpathclose%
\pgfusepath{fill}%
\end{pgfscope}%
\begin{pgfscope}%
\pgfpathrectangle{\pgfqpoint{1.150000in}{0.150000in}}{\pgfqpoint{5.700000in}{5.700000in}}%
\pgfusepath{clip}%
\pgfsetbuttcap%
\pgfsetroundjoin%
\definecolor{currentfill}{rgb}{0.233603,0.313828,0.543914}%
\pgfsetfillcolor{currentfill}%
\pgfsetfillopacity{0.700000}%
\pgfsetlinewidth{0.000000pt}%
\definecolor{currentstroke}{rgb}{0.000000,0.000000,0.000000}%
\pgfsetstrokecolor{currentstroke}%
\pgfsetdash{}{0pt}%
\pgfpathmoveto{\pgfqpoint{4.016002in}{3.086157in}}%
\pgfpathlineto{\pgfqpoint{4.029348in}{3.077739in}}%
\pgfpathlineto{\pgfqpoint{4.042698in}{3.069410in}}%
\pgfpathlineto{\pgfqpoint{4.056051in}{3.061170in}}%
\pgfpathlineto{\pgfqpoint{4.069408in}{3.053019in}}%
\pgfpathlineto{\pgfqpoint{4.077121in}{3.071414in}}%
\pgfpathlineto{\pgfqpoint{4.084832in}{3.090124in}}%
\pgfpathlineto{\pgfqpoint{4.092541in}{3.109157in}}%
\pgfpathlineto{\pgfqpoint{4.100247in}{3.128518in}}%
\pgfpathlineto{\pgfqpoint{4.086894in}{3.137066in}}%
\pgfpathlineto{\pgfqpoint{4.073544in}{3.145703in}}%
\pgfpathlineto{\pgfqpoint{4.060198in}{3.154428in}}%
\pgfpathlineto{\pgfqpoint{4.046855in}{3.163244in}}%
\pgfpathlineto{\pgfqpoint{4.039145in}{3.143478in}}%
\pgfpathlineto{\pgfqpoint{4.031433in}{3.124046in}}%
\pgfpathlineto{\pgfqpoint{4.023719in}{3.104942in}}%
\pgfpathlineto{\pgfqpoint{4.016002in}{3.086157in}}%
\pgfpathclose%
\pgfusepath{fill}%
\end{pgfscope}%
\begin{pgfscope}%
\pgfpathrectangle{\pgfqpoint{1.150000in}{0.150000in}}{\pgfqpoint{5.700000in}{5.700000in}}%
\pgfusepath{clip}%
\pgfsetbuttcap%
\pgfsetroundjoin%
\definecolor{currentfill}{rgb}{0.250425,0.274290,0.533103}%
\pgfsetfillcolor{currentfill}%
\pgfsetfillopacity{0.700000}%
\pgfsetlinewidth{0.000000pt}%
\definecolor{currentstroke}{rgb}{0.000000,0.000000,0.000000}%
\pgfsetstrokecolor{currentstroke}%
\pgfsetdash{}{0pt}%
\pgfpathmoveto{\pgfqpoint{3.350145in}{3.011006in}}%
\pgfpathlineto{\pgfqpoint{3.363428in}{3.000924in}}%
\pgfpathlineto{\pgfqpoint{3.376713in}{2.990956in}}%
\pgfpathlineto{\pgfqpoint{3.389999in}{2.981099in}}%
\pgfpathlineto{\pgfqpoint{3.403287in}{2.971355in}}%
\pgfpathlineto{\pgfqpoint{3.411145in}{2.986558in}}%
\pgfpathlineto{\pgfqpoint{3.418996in}{3.001974in}}%
\pgfpathlineto{\pgfqpoint{3.426842in}{3.017609in}}%
\pgfpathlineto{\pgfqpoint{3.434682in}{3.033466in}}%
\pgfpathlineto{\pgfqpoint{3.421398in}{3.043487in}}%
\pgfpathlineto{\pgfqpoint{3.408116in}{3.053620in}}%
\pgfpathlineto{\pgfqpoint{3.394836in}{3.063866in}}%
\pgfpathlineto{\pgfqpoint{3.381556in}{3.074224in}}%
\pgfpathlineto{\pgfqpoint{3.373712in}{3.058082in}}%
\pgfpathlineto{\pgfqpoint{3.365863in}{3.042169in}}%
\pgfpathlineto{\pgfqpoint{3.358007in}{3.026478in}}%
\pgfpathlineto{\pgfqpoint{3.350145in}{3.011006in}}%
\pgfpathclose%
\pgfusepath{fill}%
\end{pgfscope}%
\begin{pgfscope}%
\pgfpathrectangle{\pgfqpoint{1.150000in}{0.150000in}}{\pgfqpoint{5.700000in}{5.700000in}}%
\pgfusepath{clip}%
\pgfsetbuttcap%
\pgfsetroundjoin%
\definecolor{currentfill}{rgb}{0.216210,0.351535,0.550627}%
\pgfsetfillcolor{currentfill}%
\pgfsetfillopacity{0.700000}%
\pgfsetlinewidth{0.000000pt}%
\definecolor{currentstroke}{rgb}{0.000000,0.000000,0.000000}%
\pgfsetstrokecolor{currentstroke}%
\pgfsetdash{}{0pt}%
\pgfpathmoveto{\pgfqpoint{4.184491in}{3.174431in}}%
\pgfpathlineto{\pgfqpoint{4.197859in}{3.165903in}}%
\pgfpathlineto{\pgfqpoint{4.211230in}{3.157462in}}%
\pgfpathlineto{\pgfqpoint{4.224605in}{3.149106in}}%
\pgfpathlineto{\pgfqpoint{4.237984in}{3.140836in}}%
\pgfpathlineto{\pgfqpoint{4.245676in}{3.161100in}}%
\pgfpathlineto{\pgfqpoint{4.253368in}{3.181729in}}%
\pgfpathlineto{\pgfqpoint{4.261060in}{3.202733in}}%
\pgfpathlineto{\pgfqpoint{4.268751in}{3.224119in}}%
\pgfpathlineto{\pgfqpoint{4.255375in}{3.232827in}}%
\pgfpathlineto{\pgfqpoint{4.242003in}{3.241621in}}%
\pgfpathlineto{\pgfqpoint{4.228635in}{3.250500in}}%
\pgfpathlineto{\pgfqpoint{4.215270in}{3.259466in}}%
\pgfpathlineto{\pgfqpoint{4.207576in}{3.237634in}}%
\pgfpathlineto{\pgfqpoint{4.199882in}{3.216190in}}%
\pgfpathlineto{\pgfqpoint{4.192187in}{3.195125in}}%
\pgfpathlineto{\pgfqpoint{4.184491in}{3.174431in}}%
\pgfpathclose%
\pgfusepath{fill}%
\end{pgfscope}%
\begin{pgfscope}%
\pgfpathrectangle{\pgfqpoint{1.150000in}{0.150000in}}{\pgfqpoint{5.700000in}{5.700000in}}%
\pgfusepath{clip}%
\pgfsetbuttcap%
\pgfsetroundjoin%
\definecolor{currentfill}{rgb}{0.231674,0.318106,0.544834}%
\pgfsetfillcolor{currentfill}%
\pgfsetfillopacity{0.700000}%
\pgfsetlinewidth{0.000000pt}%
\definecolor{currentstroke}{rgb}{0.000000,0.000000,0.000000}%
\pgfsetstrokecolor{currentstroke}%
\pgfsetdash{}{0pt}%
\pgfpathmoveto{\pgfqpoint{3.021242in}{3.111429in}}%
\pgfpathlineto{\pgfqpoint{3.034533in}{3.099379in}}%
\pgfpathlineto{\pgfqpoint{3.047823in}{3.087461in}}%
\pgfpathlineto{\pgfqpoint{3.061112in}{3.075676in}}%
\pgfpathlineto{\pgfqpoint{3.074400in}{3.064022in}}%
\pgfpathlineto{\pgfqpoint{3.082336in}{3.078841in}}%
\pgfpathlineto{\pgfqpoint{3.090264in}{3.093861in}}%
\pgfpathlineto{\pgfqpoint{3.098184in}{3.109086in}}%
\pgfpathlineto{\pgfqpoint{3.106098in}{3.124520in}}%
\pgfpathlineto{\pgfqpoint{3.092814in}{3.136412in}}%
\pgfpathlineto{\pgfqpoint{3.079529in}{3.148435in}}%
\pgfpathlineto{\pgfqpoint{3.066244in}{3.160590in}}%
\pgfpathlineto{\pgfqpoint{3.052958in}{3.172878in}}%
\pgfpathlineto{\pgfqpoint{3.045040in}{3.157199in}}%
\pgfpathlineto{\pgfqpoint{3.037115in}{3.141733in}}%
\pgfpathlineto{\pgfqpoint{3.029182in}{3.126479in}}%
\pgfpathlineto{\pgfqpoint{3.021242in}{3.111429in}}%
\pgfpathclose%
\pgfusepath{fill}%
\end{pgfscope}%
\begin{pgfscope}%
\pgfpathrectangle{\pgfqpoint{1.150000in}{0.150000in}}{\pgfqpoint{5.700000in}{5.700000in}}%
\pgfusepath{clip}%
\pgfsetbuttcap%
\pgfsetroundjoin%
\definecolor{currentfill}{rgb}{0.252194,0.269783,0.531579}%
\pgfsetfillcolor{currentfill}%
\pgfsetfillopacity{0.700000}%
\pgfsetlinewidth{0.000000pt}%
\definecolor{currentstroke}{rgb}{0.000000,0.000000,0.000000}%
\pgfsetstrokecolor{currentstroke}%
\pgfsetdash{}{0pt}%
\pgfpathmoveto{\pgfqpoint{3.487829in}{2.994478in}}%
\pgfpathlineto{\pgfqpoint{3.501120in}{2.985001in}}%
\pgfpathlineto{\pgfqpoint{3.514413in}{2.975631in}}%
\pgfpathlineto{\pgfqpoint{3.527708in}{2.966367in}}%
\pgfpathlineto{\pgfqpoint{3.541004in}{2.957208in}}%
\pgfpathlineto{\pgfqpoint{3.548830in}{2.972716in}}%
\pgfpathlineto{\pgfqpoint{3.556650in}{2.988447in}}%
\pgfpathlineto{\pgfqpoint{3.564464in}{3.004408in}}%
\pgfpathlineto{\pgfqpoint{3.572273in}{3.020603in}}%
\pgfpathlineto{\pgfqpoint{3.558980in}{3.030058in}}%
\pgfpathlineto{\pgfqpoint{3.545690in}{3.039619in}}%
\pgfpathlineto{\pgfqpoint{3.532401in}{3.049285in}}%
\pgfpathlineto{\pgfqpoint{3.519114in}{3.059059in}}%
\pgfpathlineto{\pgfqpoint{3.511301in}{3.042559in}}%
\pgfpathlineto{\pgfqpoint{3.503483in}{3.026300in}}%
\pgfpathlineto{\pgfqpoint{3.495659in}{3.010274in}}%
\pgfpathlineto{\pgfqpoint{3.487829in}{2.994478in}}%
\pgfpathclose%
\pgfusepath{fill}%
\end{pgfscope}%
\begin{pgfscope}%
\pgfpathrectangle{\pgfqpoint{1.150000in}{0.150000in}}{\pgfqpoint{5.700000in}{5.700000in}}%
\pgfusepath{clip}%
\pgfsetbuttcap%
\pgfsetroundjoin%
\definecolor{currentfill}{rgb}{0.835270,0.886029,0.102646}%
\pgfsetfillcolor{currentfill}%
\pgfsetfillopacity{0.700000}%
\pgfsetlinewidth{0.000000pt}%
\definecolor{currentstroke}{rgb}{0.000000,0.000000,0.000000}%
\pgfsetstrokecolor{currentstroke}%
\pgfsetdash{}{0pt}%
\pgfpathmoveto{\pgfqpoint{3.599342in}{4.832513in}}%
\pgfpathlineto{\pgfqpoint{3.612717in}{4.813907in}}%
\pgfpathlineto{\pgfqpoint{3.626090in}{4.795438in}}%
\pgfpathlineto{\pgfqpoint{3.639460in}{4.777105in}}%
\pgfpathlineto{\pgfqpoint{3.652828in}{4.758906in}}%
\pgfpathlineto{\pgfqpoint{3.660408in}{4.802038in}}%
\pgfpathlineto{\pgfqpoint{3.667988in}{4.845917in}}%
\pgfpathlineto{\pgfqpoint{3.675568in}{4.890558in}}%
\pgfpathlineto{\pgfqpoint{3.662185in}{4.909238in}}%
\pgfpathlineto{\pgfqpoint{3.648800in}{4.928053in}}%
\pgfpathlineto{\pgfqpoint{3.635412in}{4.947004in}}%
\pgfpathlineto{\pgfqpoint{3.622021in}{4.966092in}}%
\pgfpathlineto{\pgfqpoint{3.614463in}{4.920802in}}%
\pgfpathlineto{\pgfqpoint{3.606903in}{4.876279in}}%
\pgfpathlineto{\pgfqpoint{3.599342in}{4.832513in}}%
\pgfpathclose%
\pgfusepath{fill}%
\end{pgfscope}%
\begin{pgfscope}%
\pgfpathrectangle{\pgfqpoint{1.150000in}{0.150000in}}{\pgfqpoint{5.700000in}{5.700000in}}%
\pgfusepath{clip}%
\pgfsetbuttcap%
\pgfsetroundjoin%
\definecolor{currentfill}{rgb}{0.241237,0.296485,0.539709}%
\pgfsetfillcolor{currentfill}%
\pgfsetfillopacity{0.700000}%
\pgfsetlinewidth{0.000000pt}%
\definecolor{currentstroke}{rgb}{0.000000,0.000000,0.000000}%
\pgfsetstrokecolor{currentstroke}%
\pgfsetdash{}{0pt}%
\pgfpathmoveto{\pgfqpoint{3.931738in}{3.047150in}}%
\pgfpathlineto{\pgfqpoint{3.945075in}{3.038744in}}%
\pgfpathlineto{\pgfqpoint{3.958415in}{3.030430in}}%
\pgfpathlineto{\pgfqpoint{3.971758in}{3.022206in}}%
\pgfpathlineto{\pgfqpoint{3.985105in}{3.014074in}}%
\pgfpathlineto{\pgfqpoint{3.992834in}{3.031651in}}%
\pgfpathlineto{\pgfqpoint{4.000559in}{3.049519in}}%
\pgfpathlineto{\pgfqpoint{4.008282in}{3.067685in}}%
\pgfpathlineto{\pgfqpoint{4.016002in}{3.086157in}}%
\pgfpathlineto{\pgfqpoint{4.002659in}{3.094666in}}%
\pgfpathlineto{\pgfqpoint{3.989319in}{3.103266in}}%
\pgfpathlineto{\pgfqpoint{3.975983in}{3.111957in}}%
\pgfpathlineto{\pgfqpoint{3.962649in}{3.120739in}}%
\pgfpathlineto{\pgfqpoint{3.954926in}{3.101883in}}%
\pgfpathlineto{\pgfqpoint{3.947200in}{3.083337in}}%
\pgfpathlineto{\pgfqpoint{3.939471in}{3.065095in}}%
\pgfpathlineto{\pgfqpoint{3.931738in}{3.047150in}}%
\pgfpathclose%
\pgfusepath{fill}%
\end{pgfscope}%
\begin{pgfscope}%
\pgfpathrectangle{\pgfqpoint{1.150000in}{0.150000in}}{\pgfqpoint{5.700000in}{5.700000in}}%
\pgfusepath{clip}%
\pgfsetbuttcap%
\pgfsetroundjoin%
\definecolor{currentfill}{rgb}{0.244972,0.287675,0.537260}%
\pgfsetfillcolor{currentfill}%
\pgfsetfillopacity{0.700000}%
\pgfsetlinewidth{0.000000pt}%
\definecolor{currentstroke}{rgb}{0.000000,0.000000,0.000000}%
\pgfsetstrokecolor{currentstroke}%
\pgfsetdash{}{0pt}%
\pgfpathmoveto{\pgfqpoint{3.212356in}{3.033971in}}%
\pgfpathlineto{\pgfqpoint{3.225638in}{3.023210in}}%
\pgfpathlineto{\pgfqpoint{3.238921in}{3.012569in}}%
\pgfpathlineto{\pgfqpoint{3.252204in}{3.002047in}}%
\pgfpathlineto{\pgfqpoint{3.265488in}{2.991643in}}%
\pgfpathlineto{\pgfqpoint{3.273381in}{3.006543in}}%
\pgfpathlineto{\pgfqpoint{3.281267in}{3.021645in}}%
\pgfpathlineto{\pgfqpoint{3.289146in}{3.036956in}}%
\pgfpathlineto{\pgfqpoint{3.297019in}{3.052480in}}%
\pgfpathlineto{\pgfqpoint{3.283740in}{3.063141in}}%
\pgfpathlineto{\pgfqpoint{3.270461in}{3.073919in}}%
\pgfpathlineto{\pgfqpoint{3.257182in}{3.084817in}}%
\pgfpathlineto{\pgfqpoint{3.243904in}{3.095836in}}%
\pgfpathlineto{\pgfqpoint{3.236027in}{3.080048in}}%
\pgfpathlineto{\pgfqpoint{3.228143in}{3.064477in}}%
\pgfpathlineto{\pgfqpoint{3.220253in}{3.049120in}}%
\pgfpathlineto{\pgfqpoint{3.212356in}{3.033971in}}%
\pgfpathclose%
\pgfusepath{fill}%
\end{pgfscope}%
\begin{pgfscope}%
\pgfpathrectangle{\pgfqpoint{1.150000in}{0.150000in}}{\pgfqpoint{5.700000in}{5.700000in}}%
\pgfusepath{clip}%
\pgfsetbuttcap%
\pgfsetroundjoin%
\definecolor{currentfill}{rgb}{0.162142,0.474838,0.558140}%
\pgfsetfillcolor{currentfill}%
\pgfsetfillopacity{0.700000}%
\pgfsetlinewidth{0.000000pt}%
\definecolor{currentstroke}{rgb}{0.000000,0.000000,0.000000}%
\pgfsetstrokecolor{currentstroke}%
\pgfsetdash{}{0pt}%
\pgfpathmoveto{\pgfqpoint{4.414621in}{3.473995in}}%
\pgfpathlineto{\pgfqpoint{4.428009in}{3.464267in}}%
\pgfpathlineto{\pgfqpoint{4.441401in}{3.454622in}}%
\pgfpathlineto{\pgfqpoint{4.454796in}{3.445060in}}%
\pgfpathlineto{\pgfqpoint{4.468194in}{3.435582in}}%
\pgfpathlineto{\pgfqpoint{4.475902in}{3.461713in}}%
\pgfpathlineto{\pgfqpoint{4.483612in}{3.488345in}}%
\pgfpathlineto{\pgfqpoint{4.491327in}{3.515486in}}%
\pgfpathlineto{\pgfqpoint{4.499046in}{3.543147in}}%
\pgfpathlineto{\pgfqpoint{4.485648in}{3.553150in}}%
\pgfpathlineto{\pgfqpoint{4.472253in}{3.563236in}}%
\pgfpathlineto{\pgfqpoint{4.458862in}{3.573406in}}%
\pgfpathlineto{\pgfqpoint{4.445473in}{3.583659in}}%
\pgfpathlineto{\pgfqpoint{4.437755in}{3.555465in}}%
\pgfpathlineto{\pgfqpoint{4.430040in}{3.527796in}}%
\pgfpathlineto{\pgfqpoint{4.422329in}{3.500643in}}%
\pgfpathlineto{\pgfqpoint{4.414621in}{3.473995in}}%
\pgfpathclose%
\pgfusepath{fill}%
\end{pgfscope}%
\begin{pgfscope}%
\pgfpathrectangle{\pgfqpoint{1.150000in}{0.150000in}}{\pgfqpoint{5.700000in}{5.700000in}}%
\pgfusepath{clip}%
\pgfsetbuttcap%
\pgfsetroundjoin%
\definecolor{currentfill}{rgb}{0.132268,0.655014,0.519661}%
\pgfsetfillcolor{currentfill}%
\pgfsetfillopacity{0.700000}%
\pgfsetlinewidth{0.000000pt}%
\definecolor{currentstroke}{rgb}{0.000000,0.000000,0.000000}%
\pgfsetstrokecolor{currentstroke}%
\pgfsetdash{}{0pt}%
\pgfpathmoveto{\pgfqpoint{4.346874in}{3.972682in}}%
\pgfpathlineto{\pgfqpoint{4.360239in}{3.960258in}}%
\pgfpathlineto{\pgfqpoint{4.373606in}{3.947924in}}%
\pgfpathlineto{\pgfqpoint{4.386975in}{3.935679in}}%
\pgfpathlineto{\pgfqpoint{4.400346in}{3.923524in}}%
\pgfpathlineto{\pgfqpoint{4.408103in}{3.958119in}}%
\pgfpathlineto{\pgfqpoint{4.415867in}{3.993364in}}%
\pgfpathlineto{\pgfqpoint{4.423637in}{4.029271in}}%
\pgfpathlineto{\pgfqpoint{4.410261in}{4.041875in}}%
\pgfpathlineto{\pgfqpoint{4.396887in}{4.054569in}}%
\pgfpathlineto{\pgfqpoint{4.383514in}{4.067353in}}%
\pgfpathlineto{\pgfqpoint{4.370144in}{4.080228in}}%
\pgfpathlineto{\pgfqpoint{4.362381in}{4.043715in}}%
\pgfpathlineto{\pgfqpoint{4.354625in}{4.007870in}}%
\pgfpathlineto{\pgfqpoint{4.346874in}{3.972682in}}%
\pgfpathclose%
\pgfusepath{fill}%
\end{pgfscope}%
\begin{pgfscope}%
\pgfpathrectangle{\pgfqpoint{1.150000in}{0.150000in}}{\pgfqpoint{5.700000in}{5.700000in}}%
\pgfusepath{clip}%
\pgfsetbuttcap%
\pgfsetroundjoin%
\definecolor{currentfill}{rgb}{0.204903,0.375746,0.553533}%
\pgfsetfillcolor{currentfill}%
\pgfsetfillopacity{0.700000}%
\pgfsetlinewidth{0.000000pt}%
\definecolor{currentstroke}{rgb}{0.000000,0.000000,0.000000}%
\pgfsetstrokecolor{currentstroke}%
\pgfsetdash{}{0pt}%
\pgfpathmoveto{\pgfqpoint{4.268751in}{3.224119in}}%
\pgfpathlineto{\pgfqpoint{4.282130in}{3.215496in}}%
\pgfpathlineto{\pgfqpoint{4.295513in}{3.206958in}}%
\pgfpathlineto{\pgfqpoint{4.308900in}{3.198503in}}%
\pgfpathlineto{\pgfqpoint{4.322291in}{3.190133in}}%
\pgfpathlineto{\pgfqpoint{4.329979in}{3.211459in}}%
\pgfpathlineto{\pgfqpoint{4.337667in}{3.233178in}}%
\pgfpathlineto{\pgfqpoint{4.345356in}{3.255301in}}%
\pgfpathlineto{\pgfqpoint{4.353046in}{3.277834in}}%
\pgfpathlineto{\pgfqpoint{4.339658in}{3.286663in}}%
\pgfpathlineto{\pgfqpoint{4.326274in}{3.295577in}}%
\pgfpathlineto{\pgfqpoint{4.312894in}{3.304574in}}%
\pgfpathlineto{\pgfqpoint{4.299517in}{3.313656in}}%
\pgfpathlineto{\pgfqpoint{4.291825in}{3.290656in}}%
\pgfpathlineto{\pgfqpoint{4.284133in}{3.268072in}}%
\pgfpathlineto{\pgfqpoint{4.276442in}{3.245896in}}%
\pgfpathlineto{\pgfqpoint{4.268751in}{3.224119in}}%
\pgfpathclose%
\pgfusepath{fill}%
\end{pgfscope}%
\begin{pgfscope}%
\pgfpathrectangle{\pgfqpoint{1.150000in}{0.150000in}}{\pgfqpoint{5.700000in}{5.700000in}}%
\pgfusepath{clip}%
\pgfsetbuttcap%
\pgfsetroundjoin%
\definecolor{currentfill}{rgb}{0.179019,0.433756,0.557430}%
\pgfsetfillcolor{currentfill}%
\pgfsetfillopacity{0.700000}%
\pgfsetlinewidth{0.000000pt}%
\definecolor{currentstroke}{rgb}{0.000000,0.000000,0.000000}%
\pgfsetstrokecolor{currentstroke}%
\pgfsetdash{}{0pt}%
\pgfpathmoveto{\pgfqpoint{4.383817in}{3.372259in}}%
\pgfpathlineto{\pgfqpoint{4.397207in}{3.363034in}}%
\pgfpathlineto{\pgfqpoint{4.410599in}{3.353891in}}%
\pgfpathlineto{\pgfqpoint{4.423996in}{3.344832in}}%
\pgfpathlineto{\pgfqpoint{4.437396in}{3.335855in}}%
\pgfpathlineto{\pgfqpoint{4.445092in}{3.360086in}}%
\pgfpathlineto{\pgfqpoint{4.452790in}{3.384778in}}%
\pgfpathlineto{\pgfqpoint{4.460490in}{3.409940in}}%
\pgfpathlineto{\pgfqpoint{4.468194in}{3.435582in}}%
\pgfpathlineto{\pgfqpoint{4.454796in}{3.445060in}}%
\pgfpathlineto{\pgfqpoint{4.441401in}{3.454622in}}%
\pgfpathlineto{\pgfqpoint{4.428009in}{3.464267in}}%
\pgfpathlineto{\pgfqpoint{4.414621in}{3.473995in}}%
\pgfpathlineto{\pgfqpoint{4.406916in}{3.447842in}}%
\pgfpathlineto{\pgfqpoint{4.399214in}{3.422175in}}%
\pgfpathlineto{\pgfqpoint{4.391515in}{3.396984in}}%
\pgfpathlineto{\pgfqpoint{4.383817in}{3.372259in}}%
\pgfpathclose%
\pgfusepath{fill}%
\end{pgfscope}%
\begin{pgfscope}%
\pgfpathrectangle{\pgfqpoint{1.150000in}{0.150000in}}{\pgfqpoint{5.700000in}{5.700000in}}%
\pgfusepath{clip}%
\pgfsetbuttcap%
\pgfsetroundjoin%
\definecolor{currentfill}{rgb}{0.253935,0.265254,0.529983}%
\pgfsetfillcolor{currentfill}%
\pgfsetfillopacity{0.700000}%
\pgfsetlinewidth{0.000000pt}%
\definecolor{currentstroke}{rgb}{0.000000,0.000000,0.000000}%
\pgfsetstrokecolor{currentstroke}%
\pgfsetdash{}{0pt}%
\pgfpathmoveto{\pgfqpoint{3.625464in}{2.983819in}}%
\pgfpathlineto{\pgfqpoint{3.638767in}{2.974879in}}%
\pgfpathlineto{\pgfqpoint{3.652073in}{2.966040in}}%
\pgfpathlineto{\pgfqpoint{3.665382in}{2.957301in}}%
\pgfpathlineto{\pgfqpoint{3.678692in}{2.948662in}}%
\pgfpathlineto{\pgfqpoint{3.686488in}{2.964480in}}%
\pgfpathlineto{\pgfqpoint{3.694278in}{2.980534in}}%
\pgfpathlineto{\pgfqpoint{3.702064in}{2.996829in}}%
\pgfpathlineto{\pgfqpoint{3.709845in}{3.013372in}}%
\pgfpathlineto{\pgfqpoint{3.696538in}{3.022328in}}%
\pgfpathlineto{\pgfqpoint{3.683234in}{3.031383in}}%
\pgfpathlineto{\pgfqpoint{3.669932in}{3.040538in}}%
\pgfpathlineto{\pgfqpoint{3.656632in}{3.049795in}}%
\pgfpathlineto{\pgfqpoint{3.648848in}{3.032928in}}%
\pgfpathlineto{\pgfqpoint{3.641058in}{3.016313in}}%
\pgfpathlineto{\pgfqpoint{3.633264in}{2.999946in}}%
\pgfpathlineto{\pgfqpoint{3.625464in}{2.983819in}}%
\pgfpathclose%
\pgfusepath{fill}%
\end{pgfscope}%
\begin{pgfscope}%
\pgfpathrectangle{\pgfqpoint{1.150000in}{0.150000in}}{\pgfqpoint{5.700000in}{5.700000in}}%
\pgfusepath{clip}%
\pgfsetbuttcap%
\pgfsetroundjoin%
\definecolor{currentfill}{rgb}{0.124395,0.578002,0.548287}%
\pgfsetfillcolor{currentfill}%
\pgfsetfillopacity{0.700000}%
\pgfsetlinewidth{0.000000pt}%
\definecolor{currentstroke}{rgb}{0.000000,0.000000,0.000000}%
\pgfsetstrokecolor{currentstroke}%
\pgfsetdash{}{0pt}%
\pgfpathmoveto{\pgfqpoint{4.422862in}{3.745965in}}%
\pgfpathlineto{\pgfqpoint{4.436240in}{3.734822in}}%
\pgfpathlineto{\pgfqpoint{4.449621in}{3.723766in}}%
\pgfpathlineto{\pgfqpoint{4.463006in}{3.712794in}}%
\pgfpathlineto{\pgfqpoint{4.476392in}{3.701907in}}%
\pgfpathlineto{\pgfqpoint{4.484135in}{3.732891in}}%
\pgfpathlineto{\pgfqpoint{4.491883in}{3.764465in}}%
\pgfpathlineto{\pgfqpoint{4.499637in}{3.796641in}}%
\pgfpathlineto{\pgfqpoint{4.507398in}{3.829431in}}%
\pgfpathlineto{\pgfqpoint{4.494008in}{3.840890in}}%
\pgfpathlineto{\pgfqpoint{4.480620in}{3.852435in}}%
\pgfpathlineto{\pgfqpoint{4.467235in}{3.864066in}}%
\pgfpathlineto{\pgfqpoint{4.453853in}{3.875783in}}%
\pgfpathlineto{\pgfqpoint{4.446096in}{3.842411in}}%
\pgfpathlineto{\pgfqpoint{4.438346in}{3.809658in}}%
\pgfpathlineto{\pgfqpoint{4.430601in}{3.777513in}}%
\pgfpathlineto{\pgfqpoint{4.422862in}{3.745965in}}%
\pgfpathclose%
\pgfusepath{fill}%
\end{pgfscope}%
\begin{pgfscope}%
\pgfpathrectangle{\pgfqpoint{1.150000in}{0.150000in}}{\pgfqpoint{5.700000in}{5.700000in}}%
\pgfusepath{clip}%
\pgfsetbuttcap%
\pgfsetroundjoin%
\definecolor{currentfill}{rgb}{0.246811,0.283237,0.535941}%
\pgfsetfillcolor{currentfill}%
\pgfsetfillopacity{0.700000}%
\pgfsetlinewidth{0.000000pt}%
\definecolor{currentstroke}{rgb}{0.000000,0.000000,0.000000}%
\pgfsetstrokecolor{currentstroke}%
\pgfsetdash{}{0pt}%
\pgfpathmoveto{\pgfqpoint{3.847442in}{3.011325in}}%
\pgfpathlineto{\pgfqpoint{3.860770in}{3.002903in}}%
\pgfpathlineto{\pgfqpoint{3.874101in}{2.994575in}}%
\pgfpathlineto{\pgfqpoint{3.887435in}{2.986340in}}%
\pgfpathlineto{\pgfqpoint{3.900773in}{2.978198in}}%
\pgfpathlineto{\pgfqpoint{3.908520in}{2.995025in}}%
\pgfpathlineto{\pgfqpoint{3.916263in}{3.012121in}}%
\pgfpathlineto{\pgfqpoint{3.924002in}{3.029494in}}%
\pgfpathlineto{\pgfqpoint{3.931738in}{3.047150in}}%
\pgfpathlineto{\pgfqpoint{3.918405in}{3.055648in}}%
\pgfpathlineto{\pgfqpoint{3.905074in}{3.064239in}}%
\pgfpathlineto{\pgfqpoint{3.891747in}{3.072923in}}%
\pgfpathlineto{\pgfqpoint{3.878423in}{3.081701in}}%
\pgfpathlineto{\pgfqpoint{3.870683in}{3.063682in}}%
\pgfpathlineto{\pgfqpoint{3.862940in}{3.045950in}}%
\pgfpathlineto{\pgfqpoint{3.855193in}{3.028500in}}%
\pgfpathlineto{\pgfqpoint{3.847442in}{3.011325in}}%
\pgfpathclose%
\pgfusepath{fill}%
\end{pgfscope}%
\begin{pgfscope}%
\pgfpathrectangle{\pgfqpoint{1.150000in}{0.150000in}}{\pgfqpoint{5.700000in}{5.700000in}}%
\pgfusepath{clip}%
\pgfsetbuttcap%
\pgfsetroundjoin%
\definecolor{currentfill}{rgb}{0.146180,0.515413,0.556823}%
\pgfsetfillcolor{currentfill}%
\pgfsetfillopacity{0.700000}%
\pgfsetlinewidth{0.000000pt}%
\definecolor{currentstroke}{rgb}{0.000000,0.000000,0.000000}%
\pgfsetstrokecolor{currentstroke}%
\pgfsetdash{}{0pt}%
\pgfpathmoveto{\pgfqpoint{4.445473in}{3.583659in}}%
\pgfpathlineto{\pgfqpoint{4.458862in}{3.573406in}}%
\pgfpathlineto{\pgfqpoint{4.472253in}{3.563236in}}%
\pgfpathlineto{\pgfqpoint{4.485648in}{3.553150in}}%
\pgfpathlineto{\pgfqpoint{4.499046in}{3.543147in}}%
\pgfpathlineto{\pgfqpoint{4.506769in}{3.571339in}}%
\pgfpathlineto{\pgfqpoint{4.514498in}{3.600071in}}%
\pgfpathlineto{\pgfqpoint{4.522231in}{3.629356in}}%
\pgfpathlineto{\pgfqpoint{4.529970in}{3.659203in}}%
\pgfpathlineto{\pgfqpoint{4.516571in}{3.669754in}}%
\pgfpathlineto{\pgfqpoint{4.503175in}{3.680388in}}%
\pgfpathlineto{\pgfqpoint{4.489782in}{3.691105in}}%
\pgfpathlineto{\pgfqpoint{4.476392in}{3.701907in}}%
\pgfpathlineto{\pgfqpoint{4.468655in}{3.671504in}}%
\pgfpathlineto{\pgfqpoint{4.460923in}{3.641668in}}%
\pgfpathlineto{\pgfqpoint{4.453196in}{3.612390in}}%
\pgfpathlineto{\pgfqpoint{4.445473in}{3.583659in}}%
\pgfpathclose%
\pgfusepath{fill}%
\end{pgfscope}%
\begin{pgfscope}%
\pgfpathrectangle{\pgfqpoint{1.150000in}{0.150000in}}{\pgfqpoint{5.700000in}{5.700000in}}%
\pgfusepath{clip}%
\pgfsetbuttcap%
\pgfsetroundjoin%
\definecolor{currentfill}{rgb}{0.239346,0.300855,0.540844}%
\pgfsetfillcolor{currentfill}%
\pgfsetfillopacity{0.700000}%
\pgfsetlinewidth{0.000000pt}%
\definecolor{currentstroke}{rgb}{0.000000,0.000000,0.000000}%
\pgfsetstrokecolor{currentstroke}%
\pgfsetdash{}{0pt}%
\pgfpathmoveto{\pgfqpoint{3.074400in}{3.064022in}}%
\pgfpathlineto{\pgfqpoint{3.087688in}{3.052498in}}%
\pgfpathlineto{\pgfqpoint{3.100976in}{3.041102in}}%
\pgfpathlineto{\pgfqpoint{3.114263in}{3.029834in}}%
\pgfpathlineto{\pgfqpoint{3.127550in}{3.018692in}}%
\pgfpathlineto{\pgfqpoint{3.135480in}{3.033282in}}%
\pgfpathlineto{\pgfqpoint{3.143403in}{3.048067in}}%
\pgfpathlineto{\pgfqpoint{3.151319in}{3.063052in}}%
\pgfpathlineto{\pgfqpoint{3.159228in}{3.078242in}}%
\pgfpathlineto{\pgfqpoint{3.145946in}{3.089620in}}%
\pgfpathlineto{\pgfqpoint{3.132663in}{3.101126in}}%
\pgfpathlineto{\pgfqpoint{3.119381in}{3.112758in}}%
\pgfpathlineto{\pgfqpoint{3.106098in}{3.124520in}}%
\pgfpathlineto{\pgfqpoint{3.098184in}{3.109086in}}%
\pgfpathlineto{\pgfqpoint{3.090264in}{3.093861in}}%
\pgfpathlineto{\pgfqpoint{3.082336in}{3.078841in}}%
\pgfpathlineto{\pgfqpoint{3.074400in}{3.064022in}}%
\pgfpathclose%
\pgfusepath{fill}%
\end{pgfscope}%
\begin{pgfscope}%
\pgfpathrectangle{\pgfqpoint{1.150000in}{0.150000in}}{\pgfqpoint{5.700000in}{5.700000in}}%
\pgfusepath{clip}%
\pgfsetbuttcap%
\pgfsetroundjoin%
\definecolor{currentfill}{rgb}{0.916242,0.896091,0.100717}%
\pgfsetfillcolor{currentfill}%
\pgfsetfillopacity{0.700000}%
\pgfsetlinewidth{0.000000pt}%
\definecolor{currentstroke}{rgb}{0.000000,0.000000,0.000000}%
\pgfsetstrokecolor{currentstroke}%
\pgfsetdash{}{0pt}%
\pgfpathmoveto{\pgfqpoint{3.545811in}{4.908327in}}%
\pgfpathlineto{\pgfqpoint{3.559198in}{4.889162in}}%
\pgfpathlineto{\pgfqpoint{3.572582in}{4.870139in}}%
\pgfpathlineto{\pgfqpoint{3.585963in}{4.851256in}}%
\pgfpathlineto{\pgfqpoint{3.599342in}{4.832513in}}%
\pgfpathlineto{\pgfqpoint{3.606903in}{4.876279in}}%
\pgfpathlineto{\pgfqpoint{3.614463in}{4.920802in}}%
\pgfpathlineto{\pgfqpoint{3.622021in}{4.966092in}}%
\pgfpathlineto{\pgfqpoint{3.608628in}{4.985320in}}%
\pgfpathlineto{\pgfqpoint{3.595231in}{5.004687in}}%
\pgfpathlineto{\pgfqpoint{3.581831in}{5.024196in}}%
\pgfpathlineto{\pgfqpoint{3.568428in}{5.043847in}}%
\pgfpathlineto{\pgfqpoint{3.560891in}{4.997902in}}%
\pgfpathlineto{\pgfqpoint{3.553352in}{4.952733in}}%
\pgfpathlineto{\pgfqpoint{3.545811in}{4.908327in}}%
\pgfpathclose%
\pgfusepath{fill}%
\end{pgfscope}%
\begin{pgfscope}%
\pgfpathrectangle{\pgfqpoint{1.150000in}{0.150000in}}{\pgfqpoint{5.700000in}{5.700000in}}%
\pgfusepath{clip}%
\pgfsetbuttcap%
\pgfsetroundjoin%
\definecolor{currentfill}{rgb}{0.194100,0.399323,0.555565}%
\pgfsetfillcolor{currentfill}%
\pgfsetfillopacity{0.700000}%
\pgfsetlinewidth{0.000000pt}%
\definecolor{currentstroke}{rgb}{0.000000,0.000000,0.000000}%
\pgfsetstrokecolor{currentstroke}%
\pgfsetdash{}{0pt}%
\pgfpathmoveto{\pgfqpoint{4.353046in}{3.277834in}}%
\pgfpathlineto{\pgfqpoint{4.366437in}{3.269089in}}%
\pgfpathlineto{\pgfqpoint{4.379832in}{3.260426in}}%
\pgfpathlineto{\pgfqpoint{4.393231in}{3.251846in}}%
\pgfpathlineto{\pgfqpoint{4.406634in}{3.243349in}}%
\pgfpathlineto{\pgfqpoint{4.414322in}{3.265831in}}%
\pgfpathlineto{\pgfqpoint{4.422012in}{3.288736in}}%
\pgfpathlineto{\pgfqpoint{4.429703in}{3.312075in}}%
\pgfpathlineto{\pgfqpoint{4.437396in}{3.335855in}}%
\pgfpathlineto{\pgfqpoint{4.423996in}{3.344832in}}%
\pgfpathlineto{\pgfqpoint{4.410599in}{3.353891in}}%
\pgfpathlineto{\pgfqpoint{4.397207in}{3.363034in}}%
\pgfpathlineto{\pgfqpoint{4.383817in}{3.372259in}}%
\pgfpathlineto{\pgfqpoint{4.376122in}{3.347991in}}%
\pgfpathlineto{\pgfqpoint{4.368428in}{3.324171in}}%
\pgfpathlineto{\pgfqpoint{4.360736in}{3.300788in}}%
\pgfpathlineto{\pgfqpoint{4.353046in}{3.277834in}}%
\pgfpathclose%
\pgfusepath{fill}%
\end{pgfscope}%
\begin{pgfscope}%
\pgfpathrectangle{\pgfqpoint{1.150000in}{0.150000in}}{\pgfqpoint{5.700000in}{5.700000in}}%
\pgfusepath{clip}%
\pgfsetbuttcap%
\pgfsetroundjoin%
\definecolor{currentfill}{rgb}{0.124780,0.640461,0.527068}%
\pgfsetfillcolor{currentfill}%
\pgfsetfillopacity{0.700000}%
\pgfsetlinewidth{0.000000pt}%
\definecolor{currentstroke}{rgb}{0.000000,0.000000,0.000000}%
\pgfsetstrokecolor{currentstroke}%
\pgfsetdash{}{0pt}%
\pgfpathmoveto{\pgfqpoint{4.400346in}{3.923524in}}%
\pgfpathlineto{\pgfqpoint{4.413719in}{3.911457in}}%
\pgfpathlineto{\pgfqpoint{4.427095in}{3.899478in}}%
\pgfpathlineto{\pgfqpoint{4.440472in}{3.887587in}}%
\pgfpathlineto{\pgfqpoint{4.453853in}{3.875783in}}%
\pgfpathlineto{\pgfqpoint{4.461615in}{3.909787in}}%
\pgfpathlineto{\pgfqpoint{4.469385in}{3.944435in}}%
\pgfpathlineto{\pgfqpoint{4.477161in}{3.979738in}}%
\pgfpathlineto{\pgfqpoint{4.463777in}{3.991989in}}%
\pgfpathlineto{\pgfqpoint{4.450395in}{4.004328in}}%
\pgfpathlineto{\pgfqpoint{4.437015in}{4.016755in}}%
\pgfpathlineto{\pgfqpoint{4.423637in}{4.029271in}}%
\pgfpathlineto{\pgfqpoint{4.415867in}{3.993364in}}%
\pgfpathlineto{\pgfqpoint{4.408103in}{3.958119in}}%
\pgfpathlineto{\pgfqpoint{4.400346in}{3.923524in}}%
\pgfpathclose%
\pgfusepath{fill}%
\end{pgfscope}%
\begin{pgfscope}%
\pgfpathrectangle{\pgfqpoint{1.150000in}{0.150000in}}{\pgfqpoint{5.700000in}{5.700000in}}%
\pgfusepath{clip}%
\pgfsetbuttcap%
\pgfsetroundjoin%
\definecolor{currentfill}{rgb}{0.253935,0.265254,0.529983}%
\pgfsetfillcolor{currentfill}%
\pgfsetfillopacity{0.700000}%
\pgfsetlinewidth{0.000000pt}%
\definecolor{currentstroke}{rgb}{0.000000,0.000000,0.000000}%
\pgfsetstrokecolor{currentstroke}%
\pgfsetdash{}{0pt}%
\pgfpathmoveto{\pgfqpoint{3.763098in}{2.978535in}}%
\pgfpathlineto{\pgfqpoint{3.776418in}{2.970069in}}%
\pgfpathlineto{\pgfqpoint{3.789741in}{2.961699in}}%
\pgfpathlineto{\pgfqpoint{3.803067in}{2.953424in}}%
\pgfpathlineto{\pgfqpoint{3.816397in}{2.945244in}}%
\pgfpathlineto{\pgfqpoint{3.824164in}{2.961384in}}%
\pgfpathlineto{\pgfqpoint{3.831928in}{2.977773in}}%
\pgfpathlineto{\pgfqpoint{3.839687in}{2.994418in}}%
\pgfpathlineto{\pgfqpoint{3.847442in}{3.011325in}}%
\pgfpathlineto{\pgfqpoint{3.834117in}{3.019841in}}%
\pgfpathlineto{\pgfqpoint{3.820795in}{3.028452in}}%
\pgfpathlineto{\pgfqpoint{3.807476in}{3.037158in}}%
\pgfpathlineto{\pgfqpoint{3.794159in}{3.045960in}}%
\pgfpathlineto{\pgfqpoint{3.786400in}{3.028710in}}%
\pgfpathlineto{\pgfqpoint{3.778637in}{3.011726in}}%
\pgfpathlineto{\pgfqpoint{3.770870in}{2.995004in}}%
\pgfpathlineto{\pgfqpoint{3.763098in}{2.978535in}}%
\pgfpathclose%
\pgfusepath{fill}%
\end{pgfscope}%
\begin{pgfscope}%
\pgfpathrectangle{\pgfqpoint{1.150000in}{0.150000in}}{\pgfqpoint{5.700000in}{5.700000in}}%
\pgfusepath{clip}%
\pgfsetbuttcap%
\pgfsetroundjoin%
\definecolor{currentfill}{rgb}{0.255645,0.260703,0.528312}%
\pgfsetfillcolor{currentfill}%
\pgfsetfillopacity{0.700000}%
\pgfsetlinewidth{0.000000pt}%
\definecolor{currentstroke}{rgb}{0.000000,0.000000,0.000000}%
\pgfsetstrokecolor{currentstroke}%
\pgfsetdash{}{0pt}%
\pgfpathmoveto{\pgfqpoint{3.403287in}{2.971355in}}%
\pgfpathlineto{\pgfqpoint{3.416576in}{2.961720in}}%
\pgfpathlineto{\pgfqpoint{3.429866in}{2.952195in}}%
\pgfpathlineto{\pgfqpoint{3.443158in}{2.942779in}}%
\pgfpathlineto{\pgfqpoint{3.456452in}{2.933472in}}%
\pgfpathlineto{\pgfqpoint{3.464305in}{2.948406in}}%
\pgfpathlineto{\pgfqpoint{3.472152in}{2.963549in}}%
\pgfpathlineto{\pgfqpoint{3.479994in}{2.978904in}}%
\pgfpathlineto{\pgfqpoint{3.487829in}{2.994478in}}%
\pgfpathlineto{\pgfqpoint{3.474540in}{3.004062in}}%
\pgfpathlineto{\pgfqpoint{3.461252in}{3.013754in}}%
\pgfpathlineto{\pgfqpoint{3.447966in}{3.023555in}}%
\pgfpathlineto{\pgfqpoint{3.434682in}{3.033466in}}%
\pgfpathlineto{\pgfqpoint{3.426842in}{3.017609in}}%
\pgfpathlineto{\pgfqpoint{3.418996in}{3.001974in}}%
\pgfpathlineto{\pgfqpoint{3.411145in}{2.986558in}}%
\pgfpathlineto{\pgfqpoint{3.403287in}{2.971355in}}%
\pgfpathclose%
\pgfusepath{fill}%
\end{pgfscope}%
\begin{pgfscope}%
\pgfpathrectangle{\pgfqpoint{1.150000in}{0.150000in}}{\pgfqpoint{5.700000in}{5.700000in}}%
\pgfusepath{clip}%
\pgfsetbuttcap%
\pgfsetroundjoin%
\definecolor{currentfill}{rgb}{0.252194,0.269783,0.531579}%
\pgfsetfillcolor{currentfill}%
\pgfsetfillopacity{0.700000}%
\pgfsetlinewidth{0.000000pt}%
\definecolor{currentstroke}{rgb}{0.000000,0.000000,0.000000}%
\pgfsetstrokecolor{currentstroke}%
\pgfsetdash{}{0pt}%
\pgfpathmoveto{\pgfqpoint{3.265488in}{2.991643in}}%
\pgfpathlineto{\pgfqpoint{3.278773in}{2.981357in}}%
\pgfpathlineto{\pgfqpoint{3.292058in}{2.971188in}}%
\pgfpathlineto{\pgfqpoint{3.305345in}{2.961134in}}%
\pgfpathlineto{\pgfqpoint{3.318632in}{2.951195in}}%
\pgfpathlineto{\pgfqpoint{3.326520in}{2.965846in}}%
\pgfpathlineto{\pgfqpoint{3.334401in}{2.980694in}}%
\pgfpathlineto{\pgfqpoint{3.342276in}{2.995746in}}%
\pgfpathlineto{\pgfqpoint{3.350145in}{3.011006in}}%
\pgfpathlineto{\pgfqpoint{3.336862in}{3.021201in}}%
\pgfpathlineto{\pgfqpoint{3.323580in}{3.031512in}}%
\pgfpathlineto{\pgfqpoint{3.310299in}{3.041938in}}%
\pgfpathlineto{\pgfqpoint{3.297019in}{3.052480in}}%
\pgfpathlineto{\pgfqpoint{3.289146in}{3.036956in}}%
\pgfpathlineto{\pgfqpoint{3.281267in}{3.021645in}}%
\pgfpathlineto{\pgfqpoint{3.273381in}{3.006543in}}%
\pgfpathlineto{\pgfqpoint{3.265488in}{2.991643in}}%
\pgfpathclose%
\pgfusepath{fill}%
\end{pgfscope}%
\begin{pgfscope}%
\pgfpathrectangle{\pgfqpoint{1.150000in}{0.150000in}}{\pgfqpoint{5.700000in}{5.700000in}}%
\pgfusepath{clip}%
\pgfsetbuttcap%
\pgfsetroundjoin%
\definecolor{currentfill}{rgb}{0.229739,0.322361,0.545706}%
\pgfsetfillcolor{currentfill}%
\pgfsetfillopacity{0.700000}%
\pgfsetlinewidth{0.000000pt}%
\definecolor{currentstroke}{rgb}{0.000000,0.000000,0.000000}%
\pgfsetstrokecolor{currentstroke}%
\pgfsetdash{}{0pt}%
\pgfpathmoveto{\pgfqpoint{4.153695in}{3.095206in}}%
\pgfpathlineto{\pgfqpoint{4.167067in}{3.087096in}}%
\pgfpathlineto{\pgfqpoint{4.180442in}{3.079072in}}%
\pgfpathlineto{\pgfqpoint{4.193821in}{3.071133in}}%
\pgfpathlineto{\pgfqpoint{4.207204in}{3.063279in}}%
\pgfpathlineto{\pgfqpoint{4.214901in}{3.082159in}}%
\pgfpathlineto{\pgfqpoint{4.222596in}{3.101374in}}%
\pgfpathlineto{\pgfqpoint{4.230290in}{3.120930in}}%
\pgfpathlineto{\pgfqpoint{4.237984in}{3.140836in}}%
\pgfpathlineto{\pgfqpoint{4.224605in}{3.149106in}}%
\pgfpathlineto{\pgfqpoint{4.211230in}{3.157462in}}%
\pgfpathlineto{\pgfqpoint{4.197859in}{3.165903in}}%
\pgfpathlineto{\pgfqpoint{4.184491in}{3.174431in}}%
\pgfpathlineto{\pgfqpoint{4.176794in}{3.154100in}}%
\pgfpathlineto{\pgfqpoint{4.169096in}{3.134124in}}%
\pgfpathlineto{\pgfqpoint{4.161396in}{3.114496in}}%
\pgfpathlineto{\pgfqpoint{4.153695in}{3.095206in}}%
\pgfpathclose%
\pgfusepath{fill}%
\end{pgfscope}%
\begin{pgfscope}%
\pgfpathrectangle{\pgfqpoint{1.150000in}{0.150000in}}{\pgfqpoint{5.700000in}{5.700000in}}%
\pgfusepath{clip}%
\pgfsetbuttcap%
\pgfsetroundjoin%
\definecolor{currentfill}{rgb}{0.257322,0.256130,0.526563}%
\pgfsetfillcolor{currentfill}%
\pgfsetfillopacity{0.700000}%
\pgfsetlinewidth{0.000000pt}%
\definecolor{currentstroke}{rgb}{0.000000,0.000000,0.000000}%
\pgfsetstrokecolor{currentstroke}%
\pgfsetdash{}{0pt}%
\pgfpathmoveto{\pgfqpoint{3.541004in}{2.957208in}}%
\pgfpathlineto{\pgfqpoint{3.554303in}{2.948153in}}%
\pgfpathlineto{\pgfqpoint{3.567604in}{2.939202in}}%
\pgfpathlineto{\pgfqpoint{3.580907in}{2.930354in}}%
\pgfpathlineto{\pgfqpoint{3.594213in}{2.921608in}}%
\pgfpathlineto{\pgfqpoint{3.602034in}{2.936828in}}%
\pgfpathlineto{\pgfqpoint{3.609849in}{2.952266in}}%
\pgfpathlineto{\pgfqpoint{3.617659in}{2.967928in}}%
\pgfpathlineto{\pgfqpoint{3.625464in}{2.983819in}}%
\pgfpathlineto{\pgfqpoint{3.612163in}{2.992861in}}%
\pgfpathlineto{\pgfqpoint{3.598864in}{3.002005in}}%
\pgfpathlineto{\pgfqpoint{3.585568in}{3.011252in}}%
\pgfpathlineto{\pgfqpoint{3.572273in}{3.020603in}}%
\pgfpathlineto{\pgfqpoint{3.564464in}{3.004408in}}%
\pgfpathlineto{\pgfqpoint{3.556650in}{2.988447in}}%
\pgfpathlineto{\pgfqpoint{3.548830in}{2.972716in}}%
\pgfpathlineto{\pgfqpoint{3.541004in}{2.957208in}}%
\pgfpathclose%
\pgfusepath{fill}%
\end{pgfscope}%
\begin{pgfscope}%
\pgfpathrectangle{\pgfqpoint{1.150000in}{0.150000in}}{\pgfqpoint{5.700000in}{5.700000in}}%
\pgfusepath{clip}%
\pgfsetbuttcap%
\pgfsetroundjoin%
\definecolor{currentfill}{rgb}{0.128729,0.563265,0.551229}%
\pgfsetfillcolor{currentfill}%
\pgfsetfillopacity{0.700000}%
\pgfsetlinewidth{0.000000pt}%
\definecolor{currentstroke}{rgb}{0.000000,0.000000,0.000000}%
\pgfsetstrokecolor{currentstroke}%
\pgfsetdash{}{0pt}%
\pgfpathmoveto{\pgfqpoint{4.476392in}{3.701907in}}%
\pgfpathlineto{\pgfqpoint{4.489782in}{3.691105in}}%
\pgfpathlineto{\pgfqpoint{4.503175in}{3.680388in}}%
\pgfpathlineto{\pgfqpoint{4.516571in}{3.669754in}}%
\pgfpathlineto{\pgfqpoint{4.529970in}{3.659203in}}%
\pgfpathlineto{\pgfqpoint{4.537714in}{3.689623in}}%
\pgfpathlineto{\pgfqpoint{4.545465in}{3.720628in}}%
\pgfpathlineto{\pgfqpoint{4.553222in}{3.752229in}}%
\pgfpathlineto{\pgfqpoint{4.560986in}{3.784437in}}%
\pgfpathlineto{\pgfqpoint{4.547585in}{3.795560in}}%
\pgfpathlineto{\pgfqpoint{4.534186in}{3.806766in}}%
\pgfpathlineto{\pgfqpoint{4.520791in}{3.818056in}}%
\pgfpathlineto{\pgfqpoint{4.507398in}{3.829431in}}%
\pgfpathlineto{\pgfqpoint{4.499637in}{3.796641in}}%
\pgfpathlineto{\pgfqpoint{4.491883in}{3.764465in}}%
\pgfpathlineto{\pgfqpoint{4.484135in}{3.732891in}}%
\pgfpathlineto{\pgfqpoint{4.476392in}{3.701907in}}%
\pgfpathclose%
\pgfusepath{fill}%
\end{pgfscope}%
\begin{pgfscope}%
\pgfpathrectangle{\pgfqpoint{1.150000in}{0.150000in}}{\pgfqpoint{5.700000in}{5.700000in}}%
\pgfusepath{clip}%
\pgfsetbuttcap%
\pgfsetroundjoin%
\definecolor{currentfill}{rgb}{0.237441,0.305202,0.541921}%
\pgfsetfillcolor{currentfill}%
\pgfsetfillopacity{0.700000}%
\pgfsetlinewidth{0.000000pt}%
\definecolor{currentstroke}{rgb}{0.000000,0.000000,0.000000}%
\pgfsetstrokecolor{currentstroke}%
\pgfsetdash{}{0pt}%
\pgfpathmoveto{\pgfqpoint{4.069408in}{3.053019in}}%
\pgfpathlineto{\pgfqpoint{4.082768in}{3.044956in}}%
\pgfpathlineto{\pgfqpoint{4.096133in}{3.036981in}}%
\pgfpathlineto{\pgfqpoint{4.109501in}{3.029094in}}%
\pgfpathlineto{\pgfqpoint{4.122872in}{3.021293in}}%
\pgfpathlineto{\pgfqpoint{4.130581in}{3.039300in}}%
\pgfpathlineto{\pgfqpoint{4.138288in}{3.057616in}}%
\pgfpathlineto{\pgfqpoint{4.145993in}{3.076249in}}%
\pgfpathlineto{\pgfqpoint{4.153695in}{3.095206in}}%
\pgfpathlineto{\pgfqpoint{4.140328in}{3.103403in}}%
\pgfpathlineto{\pgfqpoint{4.126964in}{3.111687in}}%
\pgfpathlineto{\pgfqpoint{4.113604in}{3.120059in}}%
\pgfpathlineto{\pgfqpoint{4.100247in}{3.128518in}}%
\pgfpathlineto{\pgfqpoint{4.092541in}{3.109157in}}%
\pgfpathlineto{\pgfqpoint{4.084832in}{3.090124in}}%
\pgfpathlineto{\pgfqpoint{4.077121in}{3.071414in}}%
\pgfpathlineto{\pgfqpoint{4.069408in}{3.053019in}}%
\pgfpathclose%
\pgfusepath{fill}%
\end{pgfscope}%
\begin{pgfscope}%
\pgfpathrectangle{\pgfqpoint{1.150000in}{0.150000in}}{\pgfqpoint{5.700000in}{5.700000in}}%
\pgfusepath{clip}%
\pgfsetbuttcap%
\pgfsetroundjoin%
\definecolor{currentfill}{rgb}{0.220057,0.343307,0.549413}%
\pgfsetfillcolor{currentfill}%
\pgfsetfillopacity{0.700000}%
\pgfsetlinewidth{0.000000pt}%
\definecolor{currentstroke}{rgb}{0.000000,0.000000,0.000000}%
\pgfsetstrokecolor{currentstroke}%
\pgfsetdash{}{0pt}%
\pgfpathmoveto{\pgfqpoint{4.237984in}{3.140836in}}%
\pgfpathlineto{\pgfqpoint{4.251367in}{3.132651in}}%
\pgfpathlineto{\pgfqpoint{4.264753in}{3.124550in}}%
\pgfpathlineto{\pgfqpoint{4.278144in}{3.116533in}}%
\pgfpathlineto{\pgfqpoint{4.291539in}{3.108600in}}%
\pgfpathlineto{\pgfqpoint{4.299227in}{3.128434in}}%
\pgfpathlineto{\pgfqpoint{4.306915in}{3.148629in}}%
\pgfpathlineto{\pgfqpoint{4.314603in}{3.169192in}}%
\pgfpathlineto{\pgfqpoint{4.322291in}{3.190133in}}%
\pgfpathlineto{\pgfqpoint{4.308900in}{3.198503in}}%
\pgfpathlineto{\pgfqpoint{4.295513in}{3.206958in}}%
\pgfpathlineto{\pgfqpoint{4.282130in}{3.215496in}}%
\pgfpathlineto{\pgfqpoint{4.268751in}{3.224119in}}%
\pgfpathlineto{\pgfqpoint{4.261060in}{3.202733in}}%
\pgfpathlineto{\pgfqpoint{4.253368in}{3.181729in}}%
\pgfpathlineto{\pgfqpoint{4.245676in}{3.161100in}}%
\pgfpathlineto{\pgfqpoint{4.237984in}{3.140836in}}%
\pgfpathclose%
\pgfusepath{fill}%
\end{pgfscope}%
\begin{pgfscope}%
\pgfpathrectangle{\pgfqpoint{1.150000in}{0.150000in}}{\pgfqpoint{5.700000in}{5.700000in}}%
\pgfusepath{clip}%
\pgfsetbuttcap%
\pgfsetroundjoin%
\definecolor{currentfill}{rgb}{0.166617,0.463708,0.558119}%
\pgfsetfillcolor{currentfill}%
\pgfsetfillopacity{0.700000}%
\pgfsetlinewidth{0.000000pt}%
\definecolor{currentstroke}{rgb}{0.000000,0.000000,0.000000}%
\pgfsetstrokecolor{currentstroke}%
\pgfsetdash{}{0pt}%
\pgfpathmoveto{\pgfqpoint{4.468194in}{3.435582in}}%
\pgfpathlineto{\pgfqpoint{4.481597in}{3.426185in}}%
\pgfpathlineto{\pgfqpoint{4.495002in}{3.416871in}}%
\pgfpathlineto{\pgfqpoint{4.508412in}{3.407638in}}%
\pgfpathlineto{\pgfqpoint{4.521825in}{3.398486in}}%
\pgfpathlineto{\pgfqpoint{4.529531in}{3.424103in}}%
\pgfpathlineto{\pgfqpoint{4.537241in}{3.450213in}}%
\pgfpathlineto{\pgfqpoint{4.544954in}{3.476828in}}%
\pgfpathlineto{\pgfqpoint{4.552673in}{3.503957in}}%
\pgfpathlineto{\pgfqpoint{4.539261in}{3.513632in}}%
\pgfpathlineto{\pgfqpoint{4.525852in}{3.523388in}}%
\pgfpathlineto{\pgfqpoint{4.512448in}{3.533227in}}%
\pgfpathlineto{\pgfqpoint{4.499046in}{3.543147in}}%
\pgfpathlineto{\pgfqpoint{4.491327in}{3.515486in}}%
\pgfpathlineto{\pgfqpoint{4.483612in}{3.488345in}}%
\pgfpathlineto{\pgfqpoint{4.475902in}{3.461713in}}%
\pgfpathlineto{\pgfqpoint{4.468194in}{3.435582in}}%
\pgfpathclose%
\pgfusepath{fill}%
\end{pgfscope}%
\begin{pgfscope}%
\pgfpathrectangle{\pgfqpoint{1.150000in}{0.150000in}}{\pgfqpoint{5.700000in}{5.700000in}}%
\pgfusepath{clip}%
\pgfsetbuttcap%
\pgfsetroundjoin%
\definecolor{currentfill}{rgb}{0.120081,0.622161,0.534946}%
\pgfsetfillcolor{currentfill}%
\pgfsetfillopacity{0.700000}%
\pgfsetlinewidth{0.000000pt}%
\definecolor{currentstroke}{rgb}{0.000000,0.000000,0.000000}%
\pgfsetstrokecolor{currentstroke}%
\pgfsetdash{}{0pt}%
\pgfpathmoveto{\pgfqpoint{4.453853in}{3.875783in}}%
\pgfpathlineto{\pgfqpoint{4.467235in}{3.864066in}}%
\pgfpathlineto{\pgfqpoint{4.480620in}{3.852435in}}%
\pgfpathlineto{\pgfqpoint{4.494008in}{3.840890in}}%
\pgfpathlineto{\pgfqpoint{4.507398in}{3.829431in}}%
\pgfpathlineto{\pgfqpoint{4.515165in}{3.862845in}}%
\pgfpathlineto{\pgfqpoint{4.522940in}{3.896897in}}%
\pgfpathlineto{\pgfqpoint{4.530722in}{3.931598in}}%
\pgfpathlineto{\pgfqpoint{4.517328in}{3.943504in}}%
\pgfpathlineto{\pgfqpoint{4.503937in}{3.955496in}}%
\pgfpathlineto{\pgfqpoint{4.490548in}{3.967574in}}%
\pgfpathlineto{\pgfqpoint{4.477161in}{3.979738in}}%
\pgfpathlineto{\pgfqpoint{4.469385in}{3.944435in}}%
\pgfpathlineto{\pgfqpoint{4.461615in}{3.909787in}}%
\pgfpathlineto{\pgfqpoint{4.453853in}{3.875783in}}%
\pgfpathclose%
\pgfusepath{fill}%
\end{pgfscope}%
\begin{pgfscope}%
\pgfpathrectangle{\pgfqpoint{1.150000in}{0.150000in}}{\pgfqpoint{5.700000in}{5.700000in}}%
\pgfusepath{clip}%
\pgfsetbuttcap%
\pgfsetroundjoin%
\definecolor{currentfill}{rgb}{0.244972,0.287675,0.537260}%
\pgfsetfillcolor{currentfill}%
\pgfsetfillopacity{0.700000}%
\pgfsetlinewidth{0.000000pt}%
\definecolor{currentstroke}{rgb}{0.000000,0.000000,0.000000}%
\pgfsetstrokecolor{currentstroke}%
\pgfsetdash{}{0pt}%
\pgfpathmoveto{\pgfqpoint{3.985105in}{3.014074in}}%
\pgfpathlineto{\pgfqpoint{3.998456in}{3.006032in}}%
\pgfpathlineto{\pgfqpoint{4.011810in}{2.998079in}}%
\pgfpathlineto{\pgfqpoint{4.025167in}{2.990215in}}%
\pgfpathlineto{\pgfqpoint{4.038529in}{2.982440in}}%
\pgfpathlineto{\pgfqpoint{4.046253in}{2.999649in}}%
\pgfpathlineto{\pgfqpoint{4.053974in}{3.017143in}}%
\pgfpathlineto{\pgfqpoint{4.061692in}{3.034931in}}%
\pgfpathlineto{\pgfqpoint{4.069408in}{3.053019in}}%
\pgfpathlineto{\pgfqpoint{4.056051in}{3.061170in}}%
\pgfpathlineto{\pgfqpoint{4.042698in}{3.069410in}}%
\pgfpathlineto{\pgfqpoint{4.029348in}{3.077739in}}%
\pgfpathlineto{\pgfqpoint{4.016002in}{3.086157in}}%
\pgfpathlineto{\pgfqpoint{4.008282in}{3.067685in}}%
\pgfpathlineto{\pgfqpoint{4.000559in}{3.049519in}}%
\pgfpathlineto{\pgfqpoint{3.992834in}{3.031651in}}%
\pgfpathlineto{\pgfqpoint{3.985105in}{3.014074in}}%
\pgfpathclose%
\pgfusepath{fill}%
\end{pgfscope}%
\begin{pgfscope}%
\pgfpathrectangle{\pgfqpoint{1.150000in}{0.150000in}}{\pgfqpoint{5.700000in}{5.700000in}}%
\pgfusepath{clip}%
\pgfsetbuttcap%
\pgfsetroundjoin%
\definecolor{currentfill}{rgb}{0.246811,0.283237,0.535941}%
\pgfsetfillcolor{currentfill}%
\pgfsetfillopacity{0.700000}%
\pgfsetlinewidth{0.000000pt}%
\definecolor{currentstroke}{rgb}{0.000000,0.000000,0.000000}%
\pgfsetstrokecolor{currentstroke}%
\pgfsetdash{}{0pt}%
\pgfpathmoveto{\pgfqpoint{3.127550in}{3.018692in}}%
\pgfpathlineto{\pgfqpoint{3.140836in}{3.007676in}}%
\pgfpathlineto{\pgfqpoint{3.154123in}{2.996784in}}%
\pgfpathlineto{\pgfqpoint{3.167410in}{2.986015in}}%
\pgfpathlineto{\pgfqpoint{3.180697in}{2.975369in}}%
\pgfpathlineto{\pgfqpoint{3.188622in}{2.989730in}}%
\pgfpathlineto{\pgfqpoint{3.196540in}{3.004281in}}%
\pgfpathlineto{\pgfqpoint{3.204451in}{3.019027in}}%
\pgfpathlineto{\pgfqpoint{3.212356in}{3.033971in}}%
\pgfpathlineto{\pgfqpoint{3.199073in}{3.044854in}}%
\pgfpathlineto{\pgfqpoint{3.185792in}{3.055860in}}%
\pgfpathlineto{\pgfqpoint{3.172510in}{3.066988in}}%
\pgfpathlineto{\pgfqpoint{3.159228in}{3.078242in}}%
\pgfpathlineto{\pgfqpoint{3.151319in}{3.063052in}}%
\pgfpathlineto{\pgfqpoint{3.143403in}{3.048067in}}%
\pgfpathlineto{\pgfqpoint{3.135480in}{3.033282in}}%
\pgfpathlineto{\pgfqpoint{3.127550in}{3.018692in}}%
\pgfpathclose%
\pgfusepath{fill}%
\end{pgfscope}%
\begin{pgfscope}%
\pgfpathrectangle{\pgfqpoint{1.150000in}{0.150000in}}{\pgfqpoint{5.700000in}{5.700000in}}%
\pgfusepath{clip}%
\pgfsetbuttcap%
\pgfsetroundjoin%
\definecolor{currentfill}{rgb}{0.150476,0.504369,0.557430}%
\pgfsetfillcolor{currentfill}%
\pgfsetfillopacity{0.700000}%
\pgfsetlinewidth{0.000000pt}%
\definecolor{currentstroke}{rgb}{0.000000,0.000000,0.000000}%
\pgfsetstrokecolor{currentstroke}%
\pgfsetdash{}{0pt}%
\pgfpathmoveto{\pgfqpoint{4.499046in}{3.543147in}}%
\pgfpathlineto{\pgfqpoint{4.512448in}{3.533227in}}%
\pgfpathlineto{\pgfqpoint{4.525852in}{3.523388in}}%
\pgfpathlineto{\pgfqpoint{4.539261in}{3.513632in}}%
\pgfpathlineto{\pgfqpoint{4.552673in}{3.503957in}}%
\pgfpathlineto{\pgfqpoint{4.560396in}{3.531610in}}%
\pgfpathlineto{\pgfqpoint{4.568124in}{3.559799in}}%
\pgfpathlineto{\pgfqpoint{4.575858in}{3.588534in}}%
\pgfpathlineto{\pgfqpoint{4.583598in}{3.617825in}}%
\pgfpathlineto{\pgfqpoint{4.570186in}{3.628046in}}%
\pgfpathlineto{\pgfqpoint{4.556777in}{3.638350in}}%
\pgfpathlineto{\pgfqpoint{4.543372in}{3.648735in}}%
\pgfpathlineto{\pgfqpoint{4.529970in}{3.659203in}}%
\pgfpathlineto{\pgfqpoint{4.522231in}{3.629356in}}%
\pgfpathlineto{\pgfqpoint{4.514498in}{3.600071in}}%
\pgfpathlineto{\pgfqpoint{4.506769in}{3.571339in}}%
\pgfpathlineto{\pgfqpoint{4.499046in}{3.543147in}}%
\pgfpathclose%
\pgfusepath{fill}%
\end{pgfscope}%
\begin{pgfscope}%
\pgfpathrectangle{\pgfqpoint{1.150000in}{0.150000in}}{\pgfqpoint{5.700000in}{5.700000in}}%
\pgfusepath{clip}%
\pgfsetbuttcap%
\pgfsetroundjoin%
\definecolor{currentfill}{rgb}{0.993248,0.906157,0.143936}%
\pgfsetfillcolor{currentfill}%
\pgfsetfillopacity{0.700000}%
\pgfsetlinewidth{0.000000pt}%
\definecolor{currentstroke}{rgb}{0.000000,0.000000,0.000000}%
\pgfsetstrokecolor{currentstroke}%
\pgfsetdash{}{0pt}%
\pgfpathmoveto{\pgfqpoint{3.492230in}{4.986434in}}%
\pgfpathlineto{\pgfqpoint{3.505630in}{4.966687in}}%
\pgfpathlineto{\pgfqpoint{3.519027in}{4.947089in}}%
\pgfpathlineto{\pgfqpoint{3.532421in}{4.927636in}}%
\pgfpathlineto{\pgfqpoint{3.545811in}{4.908327in}}%
\pgfpathlineto{\pgfqpoint{3.553352in}{4.952733in}}%
\pgfpathlineto{\pgfqpoint{3.560891in}{4.997902in}}%
\pgfpathlineto{\pgfqpoint{3.568428in}{5.043847in}}%
\pgfpathlineto{\pgfqpoint{3.555022in}{5.063643in}}%
\pgfpathlineto{\pgfqpoint{3.541612in}{5.083584in}}%
\pgfpathlineto{\pgfqpoint{3.528198in}{5.103672in}}%
\pgfpathlineto{\pgfqpoint{3.514781in}{5.123909in}}%
\pgfpathlineto{\pgfqpoint{3.507266in}{5.077304in}}%
\pgfpathlineto{\pgfqpoint{3.499749in}{5.031483in}}%
\pgfpathlineto{\pgfqpoint{3.492230in}{4.986434in}}%
\pgfpathclose%
\pgfusepath{fill}%
\end{pgfscope}%
\begin{pgfscope}%
\pgfpathrectangle{\pgfqpoint{1.150000in}{0.150000in}}{\pgfqpoint{5.700000in}{5.700000in}}%
\pgfusepath{clip}%
\pgfsetbuttcap%
\pgfsetroundjoin%
\definecolor{currentfill}{rgb}{0.183898,0.422383,0.556944}%
\pgfsetfillcolor{currentfill}%
\pgfsetfillopacity{0.700000}%
\pgfsetlinewidth{0.000000pt}%
\definecolor{currentstroke}{rgb}{0.000000,0.000000,0.000000}%
\pgfsetstrokecolor{currentstroke}%
\pgfsetdash{}{0pt}%
\pgfpathmoveto{\pgfqpoint{4.437396in}{3.335855in}}%
\pgfpathlineto{\pgfqpoint{4.450800in}{3.326960in}}%
\pgfpathlineto{\pgfqpoint{4.464208in}{3.318147in}}%
\pgfpathlineto{\pgfqpoint{4.477620in}{3.309415in}}%
\pgfpathlineto{\pgfqpoint{4.491036in}{3.300764in}}%
\pgfpathlineto{\pgfqpoint{4.498729in}{3.324503in}}%
\pgfpathlineto{\pgfqpoint{4.506424in}{3.348696in}}%
\pgfpathlineto{\pgfqpoint{4.514123in}{3.373354in}}%
\pgfpathlineto{\pgfqpoint{4.521825in}{3.398486in}}%
\pgfpathlineto{\pgfqpoint{4.508412in}{3.407638in}}%
\pgfpathlineto{\pgfqpoint{4.495002in}{3.416871in}}%
\pgfpathlineto{\pgfqpoint{4.481597in}{3.426185in}}%
\pgfpathlineto{\pgfqpoint{4.468194in}{3.435582in}}%
\pgfpathlineto{\pgfqpoint{4.460490in}{3.409940in}}%
\pgfpathlineto{\pgfqpoint{4.452790in}{3.384778in}}%
\pgfpathlineto{\pgfqpoint{4.445092in}{3.360086in}}%
\pgfpathlineto{\pgfqpoint{4.437396in}{3.335855in}}%
\pgfpathclose%
\pgfusepath{fill}%
\end{pgfscope}%
\begin{pgfscope}%
\pgfpathrectangle{\pgfqpoint{1.150000in}{0.150000in}}{\pgfqpoint{5.700000in}{5.700000in}}%
\pgfusepath{clip}%
\pgfsetbuttcap%
\pgfsetroundjoin%
\definecolor{currentfill}{rgb}{0.231674,0.318106,0.544834}%
\pgfsetfillcolor{currentfill}%
\pgfsetfillopacity{0.700000}%
\pgfsetlinewidth{0.000000pt}%
\definecolor{currentstroke}{rgb}{0.000000,0.000000,0.000000}%
\pgfsetstrokecolor{currentstroke}%
\pgfsetdash{}{0pt}%
\pgfpathmoveto{\pgfqpoint{2.936208in}{3.101894in}}%
\pgfpathlineto{\pgfqpoint{2.949509in}{3.089515in}}%
\pgfpathlineto{\pgfqpoint{2.962809in}{3.077275in}}%
\pgfpathlineto{\pgfqpoint{2.976107in}{3.065171in}}%
\pgfpathlineto{\pgfqpoint{2.989405in}{3.053203in}}%
\pgfpathlineto{\pgfqpoint{2.997375in}{3.067472in}}%
\pgfpathlineto{\pgfqpoint{3.005339in}{3.081930in}}%
\pgfpathlineto{\pgfqpoint{3.013294in}{3.096581in}}%
\pgfpathlineto{\pgfqpoint{3.021242in}{3.111429in}}%
\pgfpathlineto{\pgfqpoint{3.007950in}{3.123615in}}%
\pgfpathlineto{\pgfqpoint{2.994657in}{3.135936in}}%
\pgfpathlineto{\pgfqpoint{2.981363in}{3.148394in}}%
\pgfpathlineto{\pgfqpoint{2.968067in}{3.160990in}}%
\pgfpathlineto{\pgfqpoint{2.960114in}{3.145917in}}%
\pgfpathlineto{\pgfqpoint{2.952153in}{3.131046in}}%
\pgfpathlineto{\pgfqpoint{2.944185in}{3.116373in}}%
\pgfpathlineto{\pgfqpoint{2.936208in}{3.101894in}}%
\pgfpathclose%
\pgfusepath{fill}%
\end{pgfscope}%
\begin{pgfscope}%
\pgfpathrectangle{\pgfqpoint{1.150000in}{0.150000in}}{\pgfqpoint{5.700000in}{5.700000in}}%
\pgfusepath{clip}%
\pgfsetbuttcap%
\pgfsetroundjoin%
\definecolor{currentfill}{rgb}{0.210503,0.363727,0.552206}%
\pgfsetfillcolor{currentfill}%
\pgfsetfillopacity{0.700000}%
\pgfsetlinewidth{0.000000pt}%
\definecolor{currentstroke}{rgb}{0.000000,0.000000,0.000000}%
\pgfsetstrokecolor{currentstroke}%
\pgfsetdash{}{0pt}%
\pgfpathmoveto{\pgfqpoint{4.322291in}{3.190133in}}%
\pgfpathlineto{\pgfqpoint{4.335685in}{3.181846in}}%
\pgfpathlineto{\pgfqpoint{4.349084in}{3.173642in}}%
\pgfpathlineto{\pgfqpoint{4.362487in}{3.165520in}}%
\pgfpathlineto{\pgfqpoint{4.375893in}{3.157481in}}%
\pgfpathlineto{\pgfqpoint{4.383578in}{3.178357in}}%
\pgfpathlineto{\pgfqpoint{4.391262in}{3.199621in}}%
\pgfpathlineto{\pgfqpoint{4.398948in}{3.221282in}}%
\pgfpathlineto{\pgfqpoint{4.406634in}{3.243349in}}%
\pgfpathlineto{\pgfqpoint{4.393231in}{3.251846in}}%
\pgfpathlineto{\pgfqpoint{4.379832in}{3.260426in}}%
\pgfpathlineto{\pgfqpoint{4.366437in}{3.269089in}}%
\pgfpathlineto{\pgfqpoint{4.353046in}{3.277834in}}%
\pgfpathlineto{\pgfqpoint{4.345356in}{3.255301in}}%
\pgfpathlineto{\pgfqpoint{4.337667in}{3.233178in}}%
\pgfpathlineto{\pgfqpoint{4.329979in}{3.211459in}}%
\pgfpathlineto{\pgfqpoint{4.322291in}{3.190133in}}%
\pgfpathclose%
\pgfusepath{fill}%
\end{pgfscope}%
\begin{pgfscope}%
\pgfpathrectangle{\pgfqpoint{1.150000in}{0.150000in}}{\pgfqpoint{5.700000in}{5.700000in}}%
\pgfusepath{clip}%
\pgfsetbuttcap%
\pgfsetroundjoin%
\definecolor{currentfill}{rgb}{0.257322,0.256130,0.526563}%
\pgfsetfillcolor{currentfill}%
\pgfsetfillopacity{0.700000}%
\pgfsetlinewidth{0.000000pt}%
\definecolor{currentstroke}{rgb}{0.000000,0.000000,0.000000}%
\pgfsetstrokecolor{currentstroke}%
\pgfsetdash{}{0pt}%
\pgfpathmoveto{\pgfqpoint{3.678692in}{2.948662in}}%
\pgfpathlineto{\pgfqpoint{3.692006in}{2.940121in}}%
\pgfpathlineto{\pgfqpoint{3.705322in}{2.931680in}}%
\pgfpathlineto{\pgfqpoint{3.718641in}{2.923335in}}%
\pgfpathlineto{\pgfqpoint{3.731963in}{2.915088in}}%
\pgfpathlineto{\pgfqpoint{3.739754in}{2.930598in}}%
\pgfpathlineto{\pgfqpoint{3.747540in}{2.946339in}}%
\pgfpathlineto{\pgfqpoint{3.755321in}{2.962316in}}%
\pgfpathlineto{\pgfqpoint{3.763098in}{2.978535in}}%
\pgfpathlineto{\pgfqpoint{3.749780in}{2.987098in}}%
\pgfpathlineto{\pgfqpoint{3.736466in}{2.995758in}}%
\pgfpathlineto{\pgfqpoint{3.723154in}{3.004516in}}%
\pgfpathlineto{\pgfqpoint{3.709845in}{3.013372in}}%
\pgfpathlineto{\pgfqpoint{3.702064in}{2.996829in}}%
\pgfpathlineto{\pgfqpoint{3.694278in}{2.980534in}}%
\pgfpathlineto{\pgfqpoint{3.686488in}{2.964480in}}%
\pgfpathlineto{\pgfqpoint{3.678692in}{2.948662in}}%
\pgfpathclose%
\pgfusepath{fill}%
\end{pgfscope}%
\begin{pgfscope}%
\pgfpathrectangle{\pgfqpoint{1.150000in}{0.150000in}}{\pgfqpoint{5.700000in}{5.700000in}}%
\pgfusepath{clip}%
\pgfsetbuttcap%
\pgfsetroundjoin%
\definecolor{currentfill}{rgb}{0.252194,0.269783,0.531579}%
\pgfsetfillcolor{currentfill}%
\pgfsetfillopacity{0.700000}%
\pgfsetlinewidth{0.000000pt}%
\definecolor{currentstroke}{rgb}{0.000000,0.000000,0.000000}%
\pgfsetstrokecolor{currentstroke}%
\pgfsetdash{}{0pt}%
\pgfpathmoveto{\pgfqpoint{3.900773in}{2.978198in}}%
\pgfpathlineto{\pgfqpoint{3.914114in}{2.970148in}}%
\pgfpathlineto{\pgfqpoint{3.927458in}{2.962190in}}%
\pgfpathlineto{\pgfqpoint{3.940806in}{2.954323in}}%
\pgfpathlineto{\pgfqpoint{3.954158in}{2.946546in}}%
\pgfpathlineto{\pgfqpoint{3.961900in}{2.963024in}}%
\pgfpathlineto{\pgfqpoint{3.969638in}{2.979768in}}%
\pgfpathlineto{\pgfqpoint{3.977374in}{2.996782in}}%
\pgfpathlineto{\pgfqpoint{3.985105in}{3.014074in}}%
\pgfpathlineto{\pgfqpoint{3.971758in}{3.022206in}}%
\pgfpathlineto{\pgfqpoint{3.958415in}{3.030430in}}%
\pgfpathlineto{\pgfqpoint{3.945075in}{3.038744in}}%
\pgfpathlineto{\pgfqpoint{3.931738in}{3.047150in}}%
\pgfpathlineto{\pgfqpoint{3.924002in}{3.029494in}}%
\pgfpathlineto{\pgfqpoint{3.916263in}{3.012121in}}%
\pgfpathlineto{\pgfqpoint{3.908520in}{2.995025in}}%
\pgfpathlineto{\pgfqpoint{3.900773in}{2.978198in}}%
\pgfpathclose%
\pgfusepath{fill}%
\end{pgfscope}%
\begin{pgfscope}%
\pgfpathrectangle{\pgfqpoint{1.150000in}{0.150000in}}{\pgfqpoint{5.700000in}{5.700000in}}%
\pgfusepath{clip}%
\pgfsetbuttcap%
\pgfsetroundjoin%
\definecolor{currentfill}{rgb}{0.199430,0.387607,0.554642}%
\pgfsetfillcolor{currentfill}%
\pgfsetfillopacity{0.700000}%
\pgfsetlinewidth{0.000000pt}%
\definecolor{currentstroke}{rgb}{0.000000,0.000000,0.000000}%
\pgfsetstrokecolor{currentstroke}%
\pgfsetdash{}{0pt}%
\pgfpathmoveto{\pgfqpoint{4.406634in}{3.243349in}}%
\pgfpathlineto{\pgfqpoint{4.420041in}{3.234934in}}%
\pgfpathlineto{\pgfqpoint{4.433453in}{3.226600in}}%
\pgfpathlineto{\pgfqpoint{4.446868in}{3.218348in}}%
\pgfpathlineto{\pgfqpoint{4.460287in}{3.210176in}}%
\pgfpathlineto{\pgfqpoint{4.467972in}{3.232187in}}%
\pgfpathlineto{\pgfqpoint{4.475658in}{3.254616in}}%
\pgfpathlineto{\pgfqpoint{4.483346in}{3.277472in}}%
\pgfpathlineto{\pgfqpoint{4.491036in}{3.300764in}}%
\pgfpathlineto{\pgfqpoint{4.477620in}{3.309415in}}%
\pgfpathlineto{\pgfqpoint{4.464208in}{3.318147in}}%
\pgfpathlineto{\pgfqpoint{4.450800in}{3.326960in}}%
\pgfpathlineto{\pgfqpoint{4.437396in}{3.335855in}}%
\pgfpathlineto{\pgfqpoint{4.429703in}{3.312075in}}%
\pgfpathlineto{\pgfqpoint{4.422012in}{3.288736in}}%
\pgfpathlineto{\pgfqpoint{4.414322in}{3.265831in}}%
\pgfpathlineto{\pgfqpoint{4.406634in}{3.243349in}}%
\pgfpathclose%
\pgfusepath{fill}%
\end{pgfscope}%
\begin{pgfscope}%
\pgfpathrectangle{\pgfqpoint{1.150000in}{0.150000in}}{\pgfqpoint{5.700000in}{5.700000in}}%
\pgfusepath{clip}%
\pgfsetbuttcap%
\pgfsetroundjoin%
\definecolor{currentfill}{rgb}{0.133743,0.548535,0.553541}%
\pgfsetfillcolor{currentfill}%
\pgfsetfillopacity{0.700000}%
\pgfsetlinewidth{0.000000pt}%
\definecolor{currentstroke}{rgb}{0.000000,0.000000,0.000000}%
\pgfsetstrokecolor{currentstroke}%
\pgfsetdash{}{0pt}%
\pgfpathmoveto{\pgfqpoint{4.529970in}{3.659203in}}%
\pgfpathlineto{\pgfqpoint{4.543372in}{3.648735in}}%
\pgfpathlineto{\pgfqpoint{4.556777in}{3.638350in}}%
\pgfpathlineto{\pgfqpoint{4.570186in}{3.628046in}}%
\pgfpathlineto{\pgfqpoint{4.583598in}{3.617825in}}%
\pgfpathlineto{\pgfqpoint{4.591343in}{3.647684in}}%
\pgfpathlineto{\pgfqpoint{4.599096in}{3.678121in}}%
\pgfpathlineto{\pgfqpoint{4.606855in}{3.709148in}}%
\pgfpathlineto{\pgfqpoint{4.614621in}{3.740776in}}%
\pgfpathlineto{\pgfqpoint{4.601207in}{3.751568in}}%
\pgfpathlineto{\pgfqpoint{4.587797in}{3.762442in}}%
\pgfpathlineto{\pgfqpoint{4.574390in}{3.773398in}}%
\pgfpathlineto{\pgfqpoint{4.560986in}{3.784437in}}%
\pgfpathlineto{\pgfqpoint{4.553222in}{3.752229in}}%
\pgfpathlineto{\pgfqpoint{4.545465in}{3.720628in}}%
\pgfpathlineto{\pgfqpoint{4.537714in}{3.689623in}}%
\pgfpathlineto{\pgfqpoint{4.529970in}{3.659203in}}%
\pgfpathclose%
\pgfusepath{fill}%
\end{pgfscope}%
\begin{pgfscope}%
\pgfpathrectangle{\pgfqpoint{1.150000in}{0.150000in}}{\pgfqpoint{5.700000in}{5.700000in}}%
\pgfusepath{clip}%
\pgfsetbuttcap%
\pgfsetroundjoin%
\definecolor{currentfill}{rgb}{0.257322,0.256130,0.526563}%
\pgfsetfillcolor{currentfill}%
\pgfsetfillopacity{0.700000}%
\pgfsetlinewidth{0.000000pt}%
\definecolor{currentstroke}{rgb}{0.000000,0.000000,0.000000}%
\pgfsetstrokecolor{currentstroke}%
\pgfsetdash{}{0pt}%
\pgfpathmoveto{\pgfqpoint{3.318632in}{2.951195in}}%
\pgfpathlineto{\pgfqpoint{3.331921in}{2.941370in}}%
\pgfpathlineto{\pgfqpoint{3.345210in}{2.931658in}}%
\pgfpathlineto{\pgfqpoint{3.358501in}{2.922058in}}%
\pgfpathlineto{\pgfqpoint{3.371794in}{2.912569in}}%
\pgfpathlineto{\pgfqpoint{3.379676in}{2.926971in}}%
\pgfpathlineto{\pgfqpoint{3.387553in}{2.941566in}}%
\pgfpathlineto{\pgfqpoint{3.395423in}{2.956359in}}%
\pgfpathlineto{\pgfqpoint{3.403287in}{2.971355in}}%
\pgfpathlineto{\pgfqpoint{3.389999in}{2.981099in}}%
\pgfpathlineto{\pgfqpoint{3.376713in}{2.990956in}}%
\pgfpathlineto{\pgfqpoint{3.363428in}{3.000924in}}%
\pgfpathlineto{\pgfqpoint{3.350145in}{3.011006in}}%
\pgfpathlineto{\pgfqpoint{3.342276in}{2.995746in}}%
\pgfpathlineto{\pgfqpoint{3.334401in}{2.980694in}}%
\pgfpathlineto{\pgfqpoint{3.326520in}{2.965846in}}%
\pgfpathlineto{\pgfqpoint{3.318632in}{2.951195in}}%
\pgfpathclose%
\pgfusepath{fill}%
\end{pgfscope}%
\begin{pgfscope}%
\pgfpathrectangle{\pgfqpoint{1.150000in}{0.150000in}}{\pgfqpoint{5.700000in}{5.700000in}}%
\pgfusepath{clip}%
\pgfsetbuttcap%
\pgfsetroundjoin%
\definecolor{currentfill}{rgb}{0.260571,0.246922,0.522828}%
\pgfsetfillcolor{currentfill}%
\pgfsetfillopacity{0.700000}%
\pgfsetlinewidth{0.000000pt}%
\definecolor{currentstroke}{rgb}{0.000000,0.000000,0.000000}%
\pgfsetstrokecolor{currentstroke}%
\pgfsetdash{}{0pt}%
\pgfpathmoveto{\pgfqpoint{3.456452in}{2.933472in}}%
\pgfpathlineto{\pgfqpoint{3.469748in}{2.924271in}}%
\pgfpathlineto{\pgfqpoint{3.483045in}{2.915177in}}%
\pgfpathlineto{\pgfqpoint{3.496345in}{2.906189in}}%
\pgfpathlineto{\pgfqpoint{3.509646in}{2.897306in}}%
\pgfpathlineto{\pgfqpoint{3.517494in}{2.911972in}}%
\pgfpathlineto{\pgfqpoint{3.525337in}{2.926841in}}%
\pgfpathlineto{\pgfqpoint{3.533173in}{2.941918in}}%
\pgfpathlineto{\pgfqpoint{3.541004in}{2.957208in}}%
\pgfpathlineto{\pgfqpoint{3.527708in}{2.966367in}}%
\pgfpathlineto{\pgfqpoint{3.514413in}{2.975631in}}%
\pgfpathlineto{\pgfqpoint{3.501120in}{2.985001in}}%
\pgfpathlineto{\pgfqpoint{3.487829in}{2.994478in}}%
\pgfpathlineto{\pgfqpoint{3.479994in}{2.978904in}}%
\pgfpathlineto{\pgfqpoint{3.472152in}{2.963549in}}%
\pgfpathlineto{\pgfqpoint{3.464305in}{2.948406in}}%
\pgfpathlineto{\pgfqpoint{3.456452in}{2.933472in}}%
\pgfpathclose%
\pgfusepath{fill}%
\end{pgfscope}%
\begin{pgfscope}%
\pgfpathrectangle{\pgfqpoint{1.150000in}{0.150000in}}{\pgfqpoint{5.700000in}{5.700000in}}%
\pgfusepath{clip}%
\pgfsetbuttcap%
\pgfsetroundjoin%
\definecolor{currentfill}{rgb}{0.119512,0.607464,0.540218}%
\pgfsetfillcolor{currentfill}%
\pgfsetfillopacity{0.700000}%
\pgfsetlinewidth{0.000000pt}%
\definecolor{currentstroke}{rgb}{0.000000,0.000000,0.000000}%
\pgfsetstrokecolor{currentstroke}%
\pgfsetdash{}{0pt}%
\pgfpathmoveto{\pgfqpoint{4.507398in}{3.829431in}}%
\pgfpathlineto{\pgfqpoint{4.520791in}{3.818056in}}%
\pgfpathlineto{\pgfqpoint{4.534186in}{3.806766in}}%
\pgfpathlineto{\pgfqpoint{4.547585in}{3.795560in}}%
\pgfpathlineto{\pgfqpoint{4.560986in}{3.784437in}}%
\pgfpathlineto{\pgfqpoint{4.568757in}{3.817264in}}%
\pgfpathlineto{\pgfqpoint{4.576535in}{3.850722in}}%
\pgfpathlineto{\pgfqpoint{4.584322in}{3.884823in}}%
\pgfpathlineto{\pgfqpoint{4.570918in}{3.896390in}}%
\pgfpathlineto{\pgfqpoint{4.557516in}{3.908042in}}%
\pgfpathlineto{\pgfqpoint{4.544118in}{3.919778in}}%
\pgfpathlineto{\pgfqpoint{4.530722in}{3.931598in}}%
\pgfpathlineto{\pgfqpoint{4.522940in}{3.896897in}}%
\pgfpathlineto{\pgfqpoint{4.515165in}{3.862845in}}%
\pgfpathlineto{\pgfqpoint{4.507398in}{3.829431in}}%
\pgfpathclose%
\pgfusepath{fill}%
\end{pgfscope}%
\begin{pgfscope}%
\pgfpathrectangle{\pgfqpoint{1.150000in}{0.150000in}}{\pgfqpoint{5.700000in}{5.700000in}}%
\pgfusepath{clip}%
\pgfsetbuttcap%
\pgfsetroundjoin%
\definecolor{currentfill}{rgb}{0.239346,0.300855,0.540844}%
\pgfsetfillcolor{currentfill}%
\pgfsetfillopacity{0.700000}%
\pgfsetlinewidth{0.000000pt}%
\definecolor{currentstroke}{rgb}{0.000000,0.000000,0.000000}%
\pgfsetstrokecolor{currentstroke}%
\pgfsetdash{}{0pt}%
\pgfpathmoveto{\pgfqpoint{2.989405in}{3.053203in}}%
\pgfpathlineto{\pgfqpoint{3.002701in}{3.041370in}}%
\pgfpathlineto{\pgfqpoint{3.015996in}{3.029669in}}%
\pgfpathlineto{\pgfqpoint{3.029291in}{3.018101in}}%
\pgfpathlineto{\pgfqpoint{3.042585in}{3.006664in}}%
\pgfpathlineto{\pgfqpoint{3.050550in}{3.020724in}}%
\pgfpathlineto{\pgfqpoint{3.058507in}{3.034967in}}%
\pgfpathlineto{\pgfqpoint{3.066458in}{3.049399in}}%
\pgfpathlineto{\pgfqpoint{3.074400in}{3.064022in}}%
\pgfpathlineto{\pgfqpoint{3.061112in}{3.075676in}}%
\pgfpathlineto{\pgfqpoint{3.047823in}{3.087461in}}%
\pgfpathlineto{\pgfqpoint{3.034533in}{3.099379in}}%
\pgfpathlineto{\pgfqpoint{3.021242in}{3.111429in}}%
\pgfpathlineto{\pgfqpoint{3.013294in}{3.096581in}}%
\pgfpathlineto{\pgfqpoint{3.005339in}{3.081930in}}%
\pgfpathlineto{\pgfqpoint{2.997375in}{3.067472in}}%
\pgfpathlineto{\pgfqpoint{2.989405in}{3.053203in}}%
\pgfpathclose%
\pgfusepath{fill}%
\end{pgfscope}%
\begin{pgfscope}%
\pgfpathrectangle{\pgfqpoint{1.150000in}{0.150000in}}{\pgfqpoint{5.700000in}{5.700000in}}%
\pgfusepath{clip}%
\pgfsetbuttcap%
\pgfsetroundjoin%
\definecolor{currentfill}{rgb}{0.257322,0.256130,0.526563}%
\pgfsetfillcolor{currentfill}%
\pgfsetfillopacity{0.700000}%
\pgfsetlinewidth{0.000000pt}%
\definecolor{currentstroke}{rgb}{0.000000,0.000000,0.000000}%
\pgfsetstrokecolor{currentstroke}%
\pgfsetdash{}{0pt}%
\pgfpathmoveto{\pgfqpoint{3.816397in}{2.945244in}}%
\pgfpathlineto{\pgfqpoint{3.829729in}{2.937158in}}%
\pgfpathlineto{\pgfqpoint{3.843065in}{2.929166in}}%
\pgfpathlineto{\pgfqpoint{3.856404in}{2.921267in}}%
\pgfpathlineto{\pgfqpoint{3.869747in}{2.913461in}}%
\pgfpathlineto{\pgfqpoint{3.877509in}{2.929272in}}%
\pgfpathlineto{\pgfqpoint{3.885268in}{2.945328in}}%
\pgfpathlineto{\pgfqpoint{3.893022in}{2.961635in}}%
\pgfpathlineto{\pgfqpoint{3.900773in}{2.978198in}}%
\pgfpathlineto{\pgfqpoint{3.887435in}{2.986340in}}%
\pgfpathlineto{\pgfqpoint{3.874101in}{2.994575in}}%
\pgfpathlineto{\pgfqpoint{3.860770in}{3.002903in}}%
\pgfpathlineto{\pgfqpoint{3.847442in}{3.011325in}}%
\pgfpathlineto{\pgfqpoint{3.839687in}{2.994418in}}%
\pgfpathlineto{\pgfqpoint{3.831928in}{2.977773in}}%
\pgfpathlineto{\pgfqpoint{3.824164in}{2.961384in}}%
\pgfpathlineto{\pgfqpoint{3.816397in}{2.945244in}}%
\pgfpathclose%
\pgfusepath{fill}%
\end{pgfscope}%
\begin{pgfscope}%
\pgfpathrectangle{\pgfqpoint{1.150000in}{0.150000in}}{\pgfqpoint{5.700000in}{5.700000in}}%
\pgfusepath{clip}%
\pgfsetbuttcap%
\pgfsetroundjoin%
\definecolor{currentfill}{rgb}{0.253935,0.265254,0.529983}%
\pgfsetfillcolor{currentfill}%
\pgfsetfillopacity{0.700000}%
\pgfsetlinewidth{0.000000pt}%
\definecolor{currentstroke}{rgb}{0.000000,0.000000,0.000000}%
\pgfsetstrokecolor{currentstroke}%
\pgfsetdash{}{0pt}%
\pgfpathmoveto{\pgfqpoint{3.180697in}{2.975369in}}%
\pgfpathlineto{\pgfqpoint{3.193985in}{2.964844in}}%
\pgfpathlineto{\pgfqpoint{3.207273in}{2.954439in}}%
\pgfpathlineto{\pgfqpoint{3.220561in}{2.944154in}}%
\pgfpathlineto{\pgfqpoint{3.233850in}{2.933987in}}%
\pgfpathlineto{\pgfqpoint{3.241770in}{2.948119in}}%
\pgfpathlineto{\pgfqpoint{3.249683in}{2.962436in}}%
\pgfpathlineto{\pgfqpoint{3.257589in}{2.976943in}}%
\pgfpathlineto{\pgfqpoint{3.265488in}{2.991643in}}%
\pgfpathlineto{\pgfqpoint{3.252204in}{3.002047in}}%
\pgfpathlineto{\pgfqpoint{3.238921in}{3.012569in}}%
\pgfpathlineto{\pgfqpoint{3.225638in}{3.023210in}}%
\pgfpathlineto{\pgfqpoint{3.212356in}{3.033971in}}%
\pgfpathlineto{\pgfqpoint{3.204451in}{3.019027in}}%
\pgfpathlineto{\pgfqpoint{3.196540in}{3.004281in}}%
\pgfpathlineto{\pgfqpoint{3.188622in}{2.989730in}}%
\pgfpathlineto{\pgfqpoint{3.180697in}{2.975369in}}%
\pgfpathclose%
\pgfusepath{fill}%
\end{pgfscope}%
\begin{pgfscope}%
\pgfpathrectangle{\pgfqpoint{1.150000in}{0.150000in}}{\pgfqpoint{5.700000in}{5.700000in}}%
\pgfusepath{clip}%
\pgfsetbuttcap%
\pgfsetroundjoin%
\definecolor{currentfill}{rgb}{0.262138,0.242286,0.520837}%
\pgfsetfillcolor{currentfill}%
\pgfsetfillopacity{0.700000}%
\pgfsetlinewidth{0.000000pt}%
\definecolor{currentstroke}{rgb}{0.000000,0.000000,0.000000}%
\pgfsetstrokecolor{currentstroke}%
\pgfsetdash{}{0pt}%
\pgfpathmoveto{\pgfqpoint{3.594213in}{2.921608in}}%
\pgfpathlineto{\pgfqpoint{3.607521in}{2.912964in}}%
\pgfpathlineto{\pgfqpoint{3.620832in}{2.904421in}}%
\pgfpathlineto{\pgfqpoint{3.634145in}{2.895978in}}%
\pgfpathlineto{\pgfqpoint{3.647460in}{2.887634in}}%
\pgfpathlineto{\pgfqpoint{3.655276in}{2.902565in}}%
\pgfpathlineto{\pgfqpoint{3.663087in}{2.917710in}}%
\pgfpathlineto{\pgfqpoint{3.670892in}{2.933074in}}%
\pgfpathlineto{\pgfqpoint{3.678692in}{2.948662in}}%
\pgfpathlineto{\pgfqpoint{3.665382in}{2.957301in}}%
\pgfpathlineto{\pgfqpoint{3.652073in}{2.966040in}}%
\pgfpathlineto{\pgfqpoint{3.638767in}{2.974879in}}%
\pgfpathlineto{\pgfqpoint{3.625464in}{2.983819in}}%
\pgfpathlineto{\pgfqpoint{3.617659in}{2.967928in}}%
\pgfpathlineto{\pgfqpoint{3.609849in}{2.952266in}}%
\pgfpathlineto{\pgfqpoint{3.602034in}{2.936828in}}%
\pgfpathlineto{\pgfqpoint{3.594213in}{2.921608in}}%
\pgfpathclose%
\pgfusepath{fill}%
\end{pgfscope}%
\begin{pgfscope}%
\pgfpathrectangle{\pgfqpoint{1.150000in}{0.150000in}}{\pgfqpoint{5.700000in}{5.700000in}}%
\pgfusepath{clip}%
\pgfsetbuttcap%
\pgfsetroundjoin%
\definecolor{currentfill}{rgb}{0.171176,0.452530,0.557965}%
\pgfsetfillcolor{currentfill}%
\pgfsetfillopacity{0.700000}%
\pgfsetlinewidth{0.000000pt}%
\definecolor{currentstroke}{rgb}{0.000000,0.000000,0.000000}%
\pgfsetstrokecolor{currentstroke}%
\pgfsetdash{}{0pt}%
\pgfpathmoveto{\pgfqpoint{4.521825in}{3.398486in}}%
\pgfpathlineto{\pgfqpoint{4.535242in}{3.389416in}}%
\pgfpathlineto{\pgfqpoint{4.548663in}{3.380425in}}%
\pgfpathlineto{\pgfqpoint{4.562088in}{3.371515in}}%
\pgfpathlineto{\pgfqpoint{4.575518in}{3.362685in}}%
\pgfpathlineto{\pgfqpoint{4.583221in}{3.387787in}}%
\pgfpathlineto{\pgfqpoint{4.590928in}{3.413378in}}%
\pgfpathlineto{\pgfqpoint{4.598641in}{3.439467in}}%
\pgfpathlineto{\pgfqpoint{4.606357in}{3.466064in}}%
\pgfpathlineto{\pgfqpoint{4.592931in}{3.475417in}}%
\pgfpathlineto{\pgfqpoint{4.579508in}{3.484850in}}%
\pgfpathlineto{\pgfqpoint{4.566088in}{3.494363in}}%
\pgfpathlineto{\pgfqpoint{4.552673in}{3.503957in}}%
\pgfpathlineto{\pgfqpoint{4.544954in}{3.476828in}}%
\pgfpathlineto{\pgfqpoint{4.537241in}{3.450213in}}%
\pgfpathlineto{\pgfqpoint{4.529531in}{3.424103in}}%
\pgfpathlineto{\pgfqpoint{4.521825in}{3.398486in}}%
\pgfpathclose%
\pgfusepath{fill}%
\end{pgfscope}%
\begin{pgfscope}%
\pgfpathrectangle{\pgfqpoint{1.150000in}{0.150000in}}{\pgfqpoint{5.700000in}{5.700000in}}%
\pgfusepath{clip}%
\pgfsetbuttcap%
\pgfsetroundjoin%
\definecolor{currentfill}{rgb}{0.154815,0.493313,0.557840}%
\pgfsetfillcolor{currentfill}%
\pgfsetfillopacity{0.700000}%
\pgfsetlinewidth{0.000000pt}%
\definecolor{currentstroke}{rgb}{0.000000,0.000000,0.000000}%
\pgfsetstrokecolor{currentstroke}%
\pgfsetdash{}{0pt}%
\pgfpathmoveto{\pgfqpoint{4.552673in}{3.503957in}}%
\pgfpathlineto{\pgfqpoint{4.566088in}{3.494363in}}%
\pgfpathlineto{\pgfqpoint{4.579508in}{3.484850in}}%
\pgfpathlineto{\pgfqpoint{4.592931in}{3.475417in}}%
\pgfpathlineto{\pgfqpoint{4.606357in}{3.466064in}}%
\pgfpathlineto{\pgfqpoint{4.614080in}{3.493181in}}%
\pgfpathlineto{\pgfqpoint{4.621807in}{3.520827in}}%
\pgfpathlineto{\pgfqpoint{4.629540in}{3.549013in}}%
\pgfpathlineto{\pgfqpoint{4.637280in}{3.577749in}}%
\pgfpathlineto{\pgfqpoint{4.623854in}{3.587648in}}%
\pgfpathlineto{\pgfqpoint{4.610432in}{3.597626in}}%
\pgfpathlineto{\pgfqpoint{4.597013in}{3.607685in}}%
\pgfpathlineto{\pgfqpoint{4.583598in}{3.617825in}}%
\pgfpathlineto{\pgfqpoint{4.575858in}{3.588534in}}%
\pgfpathlineto{\pgfqpoint{4.568124in}{3.559799in}}%
\pgfpathlineto{\pgfqpoint{4.560396in}{3.531610in}}%
\pgfpathlineto{\pgfqpoint{4.552673in}{3.503957in}}%
\pgfpathclose%
\pgfusepath{fill}%
\end{pgfscope}%
\begin{pgfscope}%
\pgfpathrectangle{\pgfqpoint{1.150000in}{0.150000in}}{\pgfqpoint{5.700000in}{5.700000in}}%
\pgfusepath{clip}%
\pgfsetbuttcap%
\pgfsetroundjoin%
\definecolor{currentfill}{rgb}{0.233603,0.313828,0.543914}%
\pgfsetfillcolor{currentfill}%
\pgfsetfillopacity{0.700000}%
\pgfsetlinewidth{0.000000pt}%
\definecolor{currentstroke}{rgb}{0.000000,0.000000,0.000000}%
\pgfsetstrokecolor{currentstroke}%
\pgfsetdash{}{0pt}%
\pgfpathmoveto{\pgfqpoint{4.207204in}{3.063279in}}%
\pgfpathlineto{\pgfqpoint{4.220591in}{3.055511in}}%
\pgfpathlineto{\pgfqpoint{4.233982in}{3.047827in}}%
\pgfpathlineto{\pgfqpoint{4.247377in}{3.040226in}}%
\pgfpathlineto{\pgfqpoint{4.260777in}{3.032710in}}%
\pgfpathlineto{\pgfqpoint{4.268469in}{3.051181in}}%
\pgfpathlineto{\pgfqpoint{4.276160in}{3.069981in}}%
\pgfpathlineto{\pgfqpoint{4.283849in}{3.089118in}}%
\pgfpathlineto{\pgfqpoint{4.291539in}{3.108600in}}%
\pgfpathlineto{\pgfqpoint{4.278144in}{3.116533in}}%
\pgfpathlineto{\pgfqpoint{4.264753in}{3.124550in}}%
\pgfpathlineto{\pgfqpoint{4.251367in}{3.132651in}}%
\pgfpathlineto{\pgfqpoint{4.237984in}{3.140836in}}%
\pgfpathlineto{\pgfqpoint{4.230290in}{3.120930in}}%
\pgfpathlineto{\pgfqpoint{4.222596in}{3.101374in}}%
\pgfpathlineto{\pgfqpoint{4.214901in}{3.082159in}}%
\pgfpathlineto{\pgfqpoint{4.207204in}{3.063279in}}%
\pgfpathclose%
\pgfusepath{fill}%
\end{pgfscope}%
\begin{pgfscope}%
\pgfpathrectangle{\pgfqpoint{1.150000in}{0.150000in}}{\pgfqpoint{5.700000in}{5.700000in}}%
\pgfusepath{clip}%
\pgfsetbuttcap%
\pgfsetroundjoin%
\definecolor{currentfill}{rgb}{0.243113,0.292092,0.538516}%
\pgfsetfillcolor{currentfill}%
\pgfsetfillopacity{0.700000}%
\pgfsetlinewidth{0.000000pt}%
\definecolor{currentstroke}{rgb}{0.000000,0.000000,0.000000}%
\pgfsetstrokecolor{currentstroke}%
\pgfsetdash{}{0pt}%
\pgfpathmoveto{\pgfqpoint{4.122872in}{3.021293in}}%
\pgfpathlineto{\pgfqpoint{4.136248in}{3.013579in}}%
\pgfpathlineto{\pgfqpoint{4.149628in}{3.005951in}}%
\pgfpathlineto{\pgfqpoint{4.163012in}{2.998408in}}%
\pgfpathlineto{\pgfqpoint{4.176400in}{2.990951in}}%
\pgfpathlineto{\pgfqpoint{4.184104in}{3.008569in}}%
\pgfpathlineto{\pgfqpoint{4.191806in}{3.026492in}}%
\pgfpathlineto{\pgfqpoint{4.199506in}{3.044726in}}%
\pgfpathlineto{\pgfqpoint{4.207204in}{3.063279in}}%
\pgfpathlineto{\pgfqpoint{4.193821in}{3.071133in}}%
\pgfpathlineto{\pgfqpoint{4.180442in}{3.079072in}}%
\pgfpathlineto{\pgfqpoint{4.167067in}{3.087096in}}%
\pgfpathlineto{\pgfqpoint{4.153695in}{3.095206in}}%
\pgfpathlineto{\pgfqpoint{4.145993in}{3.076249in}}%
\pgfpathlineto{\pgfqpoint{4.138288in}{3.057616in}}%
\pgfpathlineto{\pgfqpoint{4.130581in}{3.039300in}}%
\pgfpathlineto{\pgfqpoint{4.122872in}{3.021293in}}%
\pgfpathclose%
\pgfusepath{fill}%
\end{pgfscope}%
\begin{pgfscope}%
\pgfpathrectangle{\pgfqpoint{1.150000in}{0.150000in}}{\pgfqpoint{5.700000in}{5.700000in}}%
\pgfusepath{clip}%
\pgfsetbuttcap%
\pgfsetroundjoin%
\definecolor{currentfill}{rgb}{0.223925,0.334994,0.548053}%
\pgfsetfillcolor{currentfill}%
\pgfsetfillopacity{0.700000}%
\pgfsetlinewidth{0.000000pt}%
\definecolor{currentstroke}{rgb}{0.000000,0.000000,0.000000}%
\pgfsetstrokecolor{currentstroke}%
\pgfsetdash{}{0pt}%
\pgfpathmoveto{\pgfqpoint{4.291539in}{3.108600in}}%
\pgfpathlineto{\pgfqpoint{4.304937in}{3.100750in}}%
\pgfpathlineto{\pgfqpoint{4.318340in}{3.092983in}}%
\pgfpathlineto{\pgfqpoint{4.331748in}{3.085299in}}%
\pgfpathlineto{\pgfqpoint{4.345159in}{3.077697in}}%
\pgfpathlineto{\pgfqpoint{4.352843in}{3.097102in}}%
\pgfpathlineto{\pgfqpoint{4.360526in}{3.116862in}}%
\pgfpathlineto{\pgfqpoint{4.368210in}{3.136986in}}%
\pgfpathlineto{\pgfqpoint{4.375893in}{3.157481in}}%
\pgfpathlineto{\pgfqpoint{4.362487in}{3.165520in}}%
\pgfpathlineto{\pgfqpoint{4.349084in}{3.173642in}}%
\pgfpathlineto{\pgfqpoint{4.335685in}{3.181846in}}%
\pgfpathlineto{\pgfqpoint{4.322291in}{3.190133in}}%
\pgfpathlineto{\pgfqpoint{4.314603in}{3.169192in}}%
\pgfpathlineto{\pgfqpoint{4.306915in}{3.148629in}}%
\pgfpathlineto{\pgfqpoint{4.299227in}{3.128434in}}%
\pgfpathlineto{\pgfqpoint{4.291539in}{3.108600in}}%
\pgfpathclose%
\pgfusepath{fill}%
\end{pgfscope}%
\begin{pgfscope}%
\pgfpathrectangle{\pgfqpoint{1.150000in}{0.150000in}}{\pgfqpoint{5.700000in}{5.700000in}}%
\pgfusepath{clip}%
\pgfsetbuttcap%
\pgfsetroundjoin%
\definecolor{currentfill}{rgb}{0.187231,0.414746,0.556547}%
\pgfsetfillcolor{currentfill}%
\pgfsetfillopacity{0.700000}%
\pgfsetlinewidth{0.000000pt}%
\definecolor{currentstroke}{rgb}{0.000000,0.000000,0.000000}%
\pgfsetstrokecolor{currentstroke}%
\pgfsetdash{}{0pt}%
\pgfpathmoveto{\pgfqpoint{4.491036in}{3.300764in}}%
\pgfpathlineto{\pgfqpoint{4.504456in}{3.292194in}}%
\pgfpathlineto{\pgfqpoint{4.517880in}{3.283705in}}%
\pgfpathlineto{\pgfqpoint{4.531308in}{3.275295in}}%
\pgfpathlineto{\pgfqpoint{4.544740in}{3.266965in}}%
\pgfpathlineto{\pgfqpoint{4.552430in}{3.290211in}}%
\pgfpathlineto{\pgfqpoint{4.560122in}{3.313907in}}%
\pgfpathlineto{\pgfqpoint{4.567818in}{3.338062in}}%
\pgfpathlineto{\pgfqpoint{4.575518in}{3.362685in}}%
\pgfpathlineto{\pgfqpoint{4.562088in}{3.371515in}}%
\pgfpathlineto{\pgfqpoint{4.548663in}{3.380425in}}%
\pgfpathlineto{\pgfqpoint{4.535242in}{3.389416in}}%
\pgfpathlineto{\pgfqpoint{4.521825in}{3.398486in}}%
\pgfpathlineto{\pgfqpoint{4.514123in}{3.373354in}}%
\pgfpathlineto{\pgfqpoint{4.506424in}{3.348696in}}%
\pgfpathlineto{\pgfqpoint{4.498729in}{3.324503in}}%
\pgfpathlineto{\pgfqpoint{4.491036in}{3.300764in}}%
\pgfpathclose%
\pgfusepath{fill}%
\end{pgfscope}%
\begin{pgfscope}%
\pgfpathrectangle{\pgfqpoint{1.150000in}{0.150000in}}{\pgfqpoint{5.700000in}{5.700000in}}%
\pgfusepath{clip}%
\pgfsetbuttcap%
\pgfsetroundjoin%
\definecolor{currentfill}{rgb}{0.248629,0.278775,0.534556}%
\pgfsetfillcolor{currentfill}%
\pgfsetfillopacity{0.700000}%
\pgfsetlinewidth{0.000000pt}%
\definecolor{currentstroke}{rgb}{0.000000,0.000000,0.000000}%
\pgfsetstrokecolor{currentstroke}%
\pgfsetdash{}{0pt}%
\pgfpathmoveto{\pgfqpoint{4.038529in}{2.982440in}}%
\pgfpathlineto{\pgfqpoint{4.051894in}{2.974754in}}%
\pgfpathlineto{\pgfqpoint{4.065263in}{2.967155in}}%
\pgfpathlineto{\pgfqpoint{4.078636in}{2.959643in}}%
\pgfpathlineto{\pgfqpoint{4.092013in}{2.952218in}}%
\pgfpathlineto{\pgfqpoint{4.099732in}{2.969058in}}%
\pgfpathlineto{\pgfqpoint{4.107448in}{2.986179in}}%
\pgfpathlineto{\pgfqpoint{4.115161in}{3.003589in}}%
\pgfpathlineto{\pgfqpoint{4.122872in}{3.021293in}}%
\pgfpathlineto{\pgfqpoint{4.109501in}{3.029094in}}%
\pgfpathlineto{\pgfqpoint{4.096133in}{3.036981in}}%
\pgfpathlineto{\pgfqpoint{4.082768in}{3.044956in}}%
\pgfpathlineto{\pgfqpoint{4.069408in}{3.053019in}}%
\pgfpathlineto{\pgfqpoint{4.061692in}{3.034931in}}%
\pgfpathlineto{\pgfqpoint{4.053974in}{3.017143in}}%
\pgfpathlineto{\pgfqpoint{4.046253in}{2.999649in}}%
\pgfpathlineto{\pgfqpoint{4.038529in}{2.982440in}}%
\pgfpathclose%
\pgfusepath{fill}%
\end{pgfscope}%
\begin{pgfscope}%
\pgfpathrectangle{\pgfqpoint{1.150000in}{0.150000in}}{\pgfqpoint{5.700000in}{5.700000in}}%
\pgfusepath{clip}%
\pgfsetbuttcap%
\pgfsetroundjoin%
\definecolor{currentfill}{rgb}{0.121148,0.592739,0.544641}%
\pgfsetfillcolor{currentfill}%
\pgfsetfillopacity{0.700000}%
\pgfsetlinewidth{0.000000pt}%
\definecolor{currentstroke}{rgb}{0.000000,0.000000,0.000000}%
\pgfsetstrokecolor{currentstroke}%
\pgfsetdash{}{0pt}%
\pgfpathmoveto{\pgfqpoint{4.560986in}{3.784437in}}%
\pgfpathlineto{\pgfqpoint{4.574390in}{3.773398in}}%
\pgfpathlineto{\pgfqpoint{4.587797in}{3.762442in}}%
\pgfpathlineto{\pgfqpoint{4.601207in}{3.751568in}}%
\pgfpathlineto{\pgfqpoint{4.614621in}{3.740776in}}%
\pgfpathlineto{\pgfqpoint{4.622394in}{3.773018in}}%
\pgfpathlineto{\pgfqpoint{4.630176in}{3.805883in}}%
\pgfpathlineto{\pgfqpoint{4.637966in}{3.839385in}}%
\pgfpathlineto{\pgfqpoint{4.624550in}{3.850621in}}%
\pgfpathlineto{\pgfqpoint{4.611138in}{3.861939in}}%
\pgfpathlineto{\pgfqpoint{4.597729in}{3.873339in}}%
\pgfpathlineto{\pgfqpoint{4.584322in}{3.884823in}}%
\pgfpathlineto{\pgfqpoint{4.576535in}{3.850722in}}%
\pgfpathlineto{\pgfqpoint{4.568757in}{3.817264in}}%
\pgfpathlineto{\pgfqpoint{4.560986in}{3.784437in}}%
\pgfpathclose%
\pgfusepath{fill}%
\end{pgfscope}%
\begin{pgfscope}%
\pgfpathrectangle{\pgfqpoint{1.150000in}{0.150000in}}{\pgfqpoint{5.700000in}{5.700000in}}%
\pgfusepath{clip}%
\pgfsetbuttcap%
\pgfsetroundjoin%
\definecolor{currentfill}{rgb}{0.214298,0.355619,0.551184}%
\pgfsetfillcolor{currentfill}%
\pgfsetfillopacity{0.700000}%
\pgfsetlinewidth{0.000000pt}%
\definecolor{currentstroke}{rgb}{0.000000,0.000000,0.000000}%
\pgfsetstrokecolor{currentstroke}%
\pgfsetdash{}{0pt}%
\pgfpathmoveto{\pgfqpoint{4.375893in}{3.157481in}}%
\pgfpathlineto{\pgfqpoint{4.389305in}{3.149524in}}%
\pgfpathlineto{\pgfqpoint{4.402720in}{3.141648in}}%
\pgfpathlineto{\pgfqpoint{4.416140in}{3.133853in}}%
\pgfpathlineto{\pgfqpoint{4.429564in}{3.126139in}}%
\pgfpathlineto{\pgfqpoint{4.437243in}{3.146565in}}%
\pgfpathlineto{\pgfqpoint{4.444923in}{3.167374in}}%
\pgfpathlineto{\pgfqpoint{4.452605in}{3.188575in}}%
\pgfpathlineto{\pgfqpoint{4.460287in}{3.210176in}}%
\pgfpathlineto{\pgfqpoint{4.446868in}{3.218348in}}%
\pgfpathlineto{\pgfqpoint{4.433453in}{3.226600in}}%
\pgfpathlineto{\pgfqpoint{4.420041in}{3.234934in}}%
\pgfpathlineto{\pgfqpoint{4.406634in}{3.243349in}}%
\pgfpathlineto{\pgfqpoint{4.398948in}{3.221282in}}%
\pgfpathlineto{\pgfqpoint{4.391262in}{3.199621in}}%
\pgfpathlineto{\pgfqpoint{4.383578in}{3.178357in}}%
\pgfpathlineto{\pgfqpoint{4.375893in}{3.157481in}}%
\pgfpathclose%
\pgfusepath{fill}%
\end{pgfscope}%
\begin{pgfscope}%
\pgfpathrectangle{\pgfqpoint{1.150000in}{0.150000in}}{\pgfqpoint{5.700000in}{5.700000in}}%
\pgfusepath{clip}%
\pgfsetbuttcap%
\pgfsetroundjoin%
\definecolor{currentfill}{rgb}{0.246811,0.283237,0.535941}%
\pgfsetfillcolor{currentfill}%
\pgfsetfillopacity{0.700000}%
\pgfsetlinewidth{0.000000pt}%
\definecolor{currentstroke}{rgb}{0.000000,0.000000,0.000000}%
\pgfsetstrokecolor{currentstroke}%
\pgfsetdash{}{0pt}%
\pgfpathmoveto{\pgfqpoint{3.042585in}{3.006664in}}%
\pgfpathlineto{\pgfqpoint{3.055878in}{2.995356in}}%
\pgfpathlineto{\pgfqpoint{3.069171in}{2.984177in}}%
\pgfpathlineto{\pgfqpoint{3.082464in}{2.973125in}}%
\pgfpathlineto{\pgfqpoint{3.095757in}{2.962200in}}%
\pgfpathlineto{\pgfqpoint{3.103716in}{2.976051in}}%
\pgfpathlineto{\pgfqpoint{3.111668in}{2.990081in}}%
\pgfpathlineto{\pgfqpoint{3.119612in}{3.004293in}}%
\pgfpathlineto{\pgfqpoint{3.127550in}{3.018692in}}%
\pgfpathlineto{\pgfqpoint{3.114263in}{3.029834in}}%
\pgfpathlineto{\pgfqpoint{3.100976in}{3.041102in}}%
\pgfpathlineto{\pgfqpoint{3.087688in}{3.052498in}}%
\pgfpathlineto{\pgfqpoint{3.074400in}{3.064022in}}%
\pgfpathlineto{\pgfqpoint{3.066458in}{3.049399in}}%
\pgfpathlineto{\pgfqpoint{3.058507in}{3.034967in}}%
\pgfpathlineto{\pgfqpoint{3.050550in}{3.020724in}}%
\pgfpathlineto{\pgfqpoint{3.042585in}{3.006664in}}%
\pgfpathclose%
\pgfusepath{fill}%
\end{pgfscope}%
\begin{pgfscope}%
\pgfpathrectangle{\pgfqpoint{1.150000in}{0.150000in}}{\pgfqpoint{5.700000in}{5.700000in}}%
\pgfusepath{clip}%
\pgfsetbuttcap%
\pgfsetroundjoin%
\definecolor{currentfill}{rgb}{0.262138,0.242286,0.520837}%
\pgfsetfillcolor{currentfill}%
\pgfsetfillopacity{0.700000}%
\pgfsetlinewidth{0.000000pt}%
\definecolor{currentstroke}{rgb}{0.000000,0.000000,0.000000}%
\pgfsetstrokecolor{currentstroke}%
\pgfsetdash{}{0pt}%
\pgfpathmoveto{\pgfqpoint{3.731963in}{2.915088in}}%
\pgfpathlineto{\pgfqpoint{3.745288in}{2.906938in}}%
\pgfpathlineto{\pgfqpoint{3.758617in}{2.898883in}}%
\pgfpathlineto{\pgfqpoint{3.771948in}{2.890924in}}%
\pgfpathlineto{\pgfqpoint{3.785282in}{2.883059in}}%
\pgfpathlineto{\pgfqpoint{3.793068in}{2.898261in}}%
\pgfpathlineto{\pgfqpoint{3.800848in}{2.913689in}}%
\pgfpathlineto{\pgfqpoint{3.808625in}{2.929348in}}%
\pgfpathlineto{\pgfqpoint{3.816397in}{2.945244in}}%
\pgfpathlineto{\pgfqpoint{3.803067in}{2.953424in}}%
\pgfpathlineto{\pgfqpoint{3.789741in}{2.961699in}}%
\pgfpathlineto{\pgfqpoint{3.776418in}{2.970069in}}%
\pgfpathlineto{\pgfqpoint{3.763098in}{2.978535in}}%
\pgfpathlineto{\pgfqpoint{3.755321in}{2.962316in}}%
\pgfpathlineto{\pgfqpoint{3.747540in}{2.946339in}}%
\pgfpathlineto{\pgfqpoint{3.739754in}{2.930598in}}%
\pgfpathlineto{\pgfqpoint{3.731963in}{2.915088in}}%
\pgfpathclose%
\pgfusepath{fill}%
\end{pgfscope}%
\begin{pgfscope}%
\pgfpathrectangle{\pgfqpoint{1.150000in}{0.150000in}}{\pgfqpoint{5.700000in}{5.700000in}}%
\pgfusepath{clip}%
\pgfsetbuttcap%
\pgfsetroundjoin%
\definecolor{currentfill}{rgb}{0.137770,0.537492,0.554906}%
\pgfsetfillcolor{currentfill}%
\pgfsetfillopacity{0.700000}%
\pgfsetlinewidth{0.000000pt}%
\definecolor{currentstroke}{rgb}{0.000000,0.000000,0.000000}%
\pgfsetstrokecolor{currentstroke}%
\pgfsetdash{}{0pt}%
\pgfpathmoveto{\pgfqpoint{4.583598in}{3.617825in}}%
\pgfpathlineto{\pgfqpoint{4.597013in}{3.607685in}}%
\pgfpathlineto{\pgfqpoint{4.610432in}{3.597626in}}%
\pgfpathlineto{\pgfqpoint{4.623854in}{3.587648in}}%
\pgfpathlineto{\pgfqpoint{4.637280in}{3.577749in}}%
\pgfpathlineto{\pgfqpoint{4.645026in}{3.607048in}}%
\pgfpathlineto{\pgfqpoint{4.652779in}{3.636919in}}%
\pgfpathlineto{\pgfqpoint{4.660538in}{3.667373in}}%
\pgfpathlineto{\pgfqpoint{4.668306in}{3.698423in}}%
\pgfpathlineto{\pgfqpoint{4.654880in}{3.708890in}}%
\pgfpathlineto{\pgfqpoint{4.641457in}{3.719438in}}%
\pgfpathlineto{\pgfqpoint{4.628037in}{3.730067in}}%
\pgfpathlineto{\pgfqpoint{4.614621in}{3.740776in}}%
\pgfpathlineto{\pgfqpoint{4.606855in}{3.709148in}}%
\pgfpathlineto{\pgfqpoint{4.599096in}{3.678121in}}%
\pgfpathlineto{\pgfqpoint{4.591343in}{3.647684in}}%
\pgfpathlineto{\pgfqpoint{4.583598in}{3.617825in}}%
\pgfpathclose%
\pgfusepath{fill}%
\end{pgfscope}%
\begin{pgfscope}%
\pgfpathrectangle{\pgfqpoint{1.150000in}{0.150000in}}{\pgfqpoint{5.700000in}{5.700000in}}%
\pgfusepath{clip}%
\pgfsetbuttcap%
\pgfsetroundjoin%
\definecolor{currentfill}{rgb}{0.262138,0.242286,0.520837}%
\pgfsetfillcolor{currentfill}%
\pgfsetfillopacity{0.700000}%
\pgfsetlinewidth{0.000000pt}%
\definecolor{currentstroke}{rgb}{0.000000,0.000000,0.000000}%
\pgfsetstrokecolor{currentstroke}%
\pgfsetdash{}{0pt}%
\pgfpathmoveto{\pgfqpoint{3.371794in}{2.912569in}}%
\pgfpathlineto{\pgfqpoint{3.385088in}{2.903190in}}%
\pgfpathlineto{\pgfqpoint{3.398383in}{2.893922in}}%
\pgfpathlineto{\pgfqpoint{3.411681in}{2.884762in}}%
\pgfpathlineto{\pgfqpoint{3.424980in}{2.875710in}}%
\pgfpathlineto{\pgfqpoint{3.432857in}{2.889863in}}%
\pgfpathlineto{\pgfqpoint{3.440728in}{2.904205in}}%
\pgfpathlineto{\pgfqpoint{3.448593in}{2.918739in}}%
\pgfpathlineto{\pgfqpoint{3.456452in}{2.933472in}}%
\pgfpathlineto{\pgfqpoint{3.443158in}{2.942779in}}%
\pgfpathlineto{\pgfqpoint{3.429866in}{2.952195in}}%
\pgfpathlineto{\pgfqpoint{3.416576in}{2.961720in}}%
\pgfpathlineto{\pgfqpoint{3.403287in}{2.971355in}}%
\pgfpathlineto{\pgfqpoint{3.395423in}{2.956359in}}%
\pgfpathlineto{\pgfqpoint{3.387553in}{2.941566in}}%
\pgfpathlineto{\pgfqpoint{3.379676in}{2.926971in}}%
\pgfpathlineto{\pgfqpoint{3.371794in}{2.912569in}}%
\pgfpathclose%
\pgfusepath{fill}%
\end{pgfscope}%
\begin{pgfscope}%
\pgfpathrectangle{\pgfqpoint{1.150000in}{0.150000in}}{\pgfqpoint{5.700000in}{5.700000in}}%
\pgfusepath{clip}%
\pgfsetbuttcap%
\pgfsetroundjoin%
\definecolor{currentfill}{rgb}{0.255645,0.260703,0.528312}%
\pgfsetfillcolor{currentfill}%
\pgfsetfillopacity{0.700000}%
\pgfsetlinewidth{0.000000pt}%
\definecolor{currentstroke}{rgb}{0.000000,0.000000,0.000000}%
\pgfsetstrokecolor{currentstroke}%
\pgfsetdash{}{0pt}%
\pgfpathmoveto{\pgfqpoint{3.954158in}{2.946546in}}%
\pgfpathlineto{\pgfqpoint{3.967513in}{2.938859in}}%
\pgfpathlineto{\pgfqpoint{3.980872in}{2.931262in}}%
\pgfpathlineto{\pgfqpoint{3.994235in}{2.923754in}}%
\pgfpathlineto{\pgfqpoint{4.007601in}{2.916335in}}%
\pgfpathlineto{\pgfqpoint{4.015338in}{2.932466in}}%
\pgfpathlineto{\pgfqpoint{4.023072in}{2.948856in}}%
\pgfpathlineto{\pgfqpoint{4.030802in}{2.965512in}}%
\pgfpathlineto{\pgfqpoint{4.038529in}{2.982440in}}%
\pgfpathlineto{\pgfqpoint{4.025167in}{2.990215in}}%
\pgfpathlineto{\pgfqpoint{4.011810in}{2.998079in}}%
\pgfpathlineto{\pgfqpoint{3.998456in}{3.006032in}}%
\pgfpathlineto{\pgfqpoint{3.985105in}{3.014074in}}%
\pgfpathlineto{\pgfqpoint{3.977374in}{2.996782in}}%
\pgfpathlineto{\pgfqpoint{3.969638in}{2.979768in}}%
\pgfpathlineto{\pgfqpoint{3.961900in}{2.963024in}}%
\pgfpathlineto{\pgfqpoint{3.954158in}{2.946546in}}%
\pgfpathclose%
\pgfusepath{fill}%
\end{pgfscope}%
\begin{pgfscope}%
\pgfpathrectangle{\pgfqpoint{1.150000in}{0.150000in}}{\pgfqpoint{5.700000in}{5.700000in}}%
\pgfusepath{clip}%
\pgfsetbuttcap%
\pgfsetroundjoin%
\definecolor{currentfill}{rgb}{0.263663,0.237631,0.518762}%
\pgfsetfillcolor{currentfill}%
\pgfsetfillopacity{0.700000}%
\pgfsetlinewidth{0.000000pt}%
\definecolor{currentstroke}{rgb}{0.000000,0.000000,0.000000}%
\pgfsetstrokecolor{currentstroke}%
\pgfsetdash{}{0pt}%
\pgfpathmoveto{\pgfqpoint{3.509646in}{2.897306in}}%
\pgfpathlineto{\pgfqpoint{3.522950in}{2.888527in}}%
\pgfpathlineto{\pgfqpoint{3.536256in}{2.879852in}}%
\pgfpathlineto{\pgfqpoint{3.549565in}{2.871280in}}%
\pgfpathlineto{\pgfqpoint{3.562875in}{2.862810in}}%
\pgfpathlineto{\pgfqpoint{3.570718in}{2.877208in}}%
\pgfpathlineto{\pgfqpoint{3.578555in}{2.891804in}}%
\pgfpathlineto{\pgfqpoint{3.586387in}{2.906602in}}%
\pgfpathlineto{\pgfqpoint{3.594213in}{2.921608in}}%
\pgfpathlineto{\pgfqpoint{3.580907in}{2.930354in}}%
\pgfpathlineto{\pgfqpoint{3.567604in}{2.939202in}}%
\pgfpathlineto{\pgfqpoint{3.554303in}{2.948153in}}%
\pgfpathlineto{\pgfqpoint{3.541004in}{2.957208in}}%
\pgfpathlineto{\pgfqpoint{3.533173in}{2.941918in}}%
\pgfpathlineto{\pgfqpoint{3.525337in}{2.926841in}}%
\pgfpathlineto{\pgfqpoint{3.517494in}{2.911972in}}%
\pgfpathlineto{\pgfqpoint{3.509646in}{2.897306in}}%
\pgfpathclose%
\pgfusepath{fill}%
\end{pgfscope}%
\begin{pgfscope}%
\pgfpathrectangle{\pgfqpoint{1.150000in}{0.150000in}}{\pgfqpoint{5.700000in}{5.700000in}}%
\pgfusepath{clip}%
\pgfsetbuttcap%
\pgfsetroundjoin%
\definecolor{currentfill}{rgb}{0.258965,0.251537,0.524736}%
\pgfsetfillcolor{currentfill}%
\pgfsetfillopacity{0.700000}%
\pgfsetlinewidth{0.000000pt}%
\definecolor{currentstroke}{rgb}{0.000000,0.000000,0.000000}%
\pgfsetstrokecolor{currentstroke}%
\pgfsetdash{}{0pt}%
\pgfpathmoveto{\pgfqpoint{3.233850in}{2.933987in}}%
\pgfpathlineto{\pgfqpoint{3.247140in}{2.923937in}}%
\pgfpathlineto{\pgfqpoint{3.260431in}{2.914004in}}%
\pgfpathlineto{\pgfqpoint{3.273723in}{2.904186in}}%
\pgfpathlineto{\pgfqpoint{3.287016in}{2.894483in}}%
\pgfpathlineto{\pgfqpoint{3.294930in}{2.908387in}}%
\pgfpathlineto{\pgfqpoint{3.302837in}{2.922470in}}%
\pgfpathlineto{\pgfqpoint{3.310738in}{2.936738in}}%
\pgfpathlineto{\pgfqpoint{3.318632in}{2.951195in}}%
\pgfpathlineto{\pgfqpoint{3.305345in}{2.961134in}}%
\pgfpathlineto{\pgfqpoint{3.292058in}{2.971188in}}%
\pgfpathlineto{\pgfqpoint{3.278773in}{2.981357in}}%
\pgfpathlineto{\pgfqpoint{3.265488in}{2.991643in}}%
\pgfpathlineto{\pgfqpoint{3.257589in}{2.976943in}}%
\pgfpathlineto{\pgfqpoint{3.249683in}{2.962436in}}%
\pgfpathlineto{\pgfqpoint{3.241770in}{2.948119in}}%
\pgfpathlineto{\pgfqpoint{3.233850in}{2.933987in}}%
\pgfpathclose%
\pgfusepath{fill}%
\end{pgfscope}%
\begin{pgfscope}%
\pgfpathrectangle{\pgfqpoint{1.150000in}{0.150000in}}{\pgfqpoint{5.700000in}{5.700000in}}%
\pgfusepath{clip}%
\pgfsetbuttcap%
\pgfsetroundjoin%
\definecolor{currentfill}{rgb}{0.203063,0.379716,0.553925}%
\pgfsetfillcolor{currentfill}%
\pgfsetfillopacity{0.700000}%
\pgfsetlinewidth{0.000000pt}%
\definecolor{currentstroke}{rgb}{0.000000,0.000000,0.000000}%
\pgfsetstrokecolor{currentstroke}%
\pgfsetdash{}{0pt}%
\pgfpathmoveto{\pgfqpoint{4.460287in}{3.210176in}}%
\pgfpathlineto{\pgfqpoint{4.473711in}{3.202085in}}%
\pgfpathlineto{\pgfqpoint{4.487139in}{3.194074in}}%
\pgfpathlineto{\pgfqpoint{4.500572in}{3.186143in}}%
\pgfpathlineto{\pgfqpoint{4.514008in}{3.178292in}}%
\pgfpathlineto{\pgfqpoint{4.521688in}{3.199832in}}%
\pgfpathlineto{\pgfqpoint{4.529370in}{3.221785in}}%
\pgfpathlineto{\pgfqpoint{4.537054in}{3.244160in}}%
\pgfpathlineto{\pgfqpoint{4.544740in}{3.266965in}}%
\pgfpathlineto{\pgfqpoint{4.531308in}{3.275295in}}%
\pgfpathlineto{\pgfqpoint{4.517880in}{3.283705in}}%
\pgfpathlineto{\pgfqpoint{4.504456in}{3.292194in}}%
\pgfpathlineto{\pgfqpoint{4.491036in}{3.300764in}}%
\pgfpathlineto{\pgfqpoint{4.483346in}{3.277472in}}%
\pgfpathlineto{\pgfqpoint{4.475658in}{3.254616in}}%
\pgfpathlineto{\pgfqpoint{4.467972in}{3.232187in}}%
\pgfpathlineto{\pgfqpoint{4.460287in}{3.210176in}}%
\pgfpathclose%
\pgfusepath{fill}%
\end{pgfscope}%
\begin{pgfscope}%
\pgfpathrectangle{\pgfqpoint{1.150000in}{0.150000in}}{\pgfqpoint{5.700000in}{5.700000in}}%
\pgfusepath{clip}%
\pgfsetbuttcap%
\pgfsetroundjoin%
\definecolor{currentfill}{rgb}{0.160665,0.478540,0.558115}%
\pgfsetfillcolor{currentfill}%
\pgfsetfillopacity{0.700000}%
\pgfsetlinewidth{0.000000pt}%
\definecolor{currentstroke}{rgb}{0.000000,0.000000,0.000000}%
\pgfsetstrokecolor{currentstroke}%
\pgfsetdash{}{0pt}%
\pgfpathmoveto{\pgfqpoint{4.606357in}{3.466064in}}%
\pgfpathlineto{\pgfqpoint{4.619788in}{3.456791in}}%
\pgfpathlineto{\pgfqpoint{4.633223in}{3.447598in}}%
\pgfpathlineto{\pgfqpoint{4.646661in}{3.438483in}}%
\pgfpathlineto{\pgfqpoint{4.660104in}{3.429447in}}%
\pgfpathlineto{\pgfqpoint{4.667824in}{3.456027in}}%
\pgfpathlineto{\pgfqpoint{4.675550in}{3.483132in}}%
\pgfpathlineto{\pgfqpoint{4.683282in}{3.510770in}}%
\pgfpathlineto{\pgfqpoint{4.691020in}{3.538953in}}%
\pgfpathlineto{\pgfqpoint{4.677579in}{3.548534in}}%
\pgfpathlineto{\pgfqpoint{4.664143in}{3.558193in}}%
\pgfpathlineto{\pgfqpoint{4.650709in}{3.567931in}}%
\pgfpathlineto{\pgfqpoint{4.637280in}{3.577749in}}%
\pgfpathlineto{\pgfqpoint{4.629540in}{3.549013in}}%
\pgfpathlineto{\pgfqpoint{4.621807in}{3.520827in}}%
\pgfpathlineto{\pgfqpoint{4.614080in}{3.493181in}}%
\pgfpathlineto{\pgfqpoint{4.606357in}{3.466064in}}%
\pgfpathclose%
\pgfusepath{fill}%
\end{pgfscope}%
\begin{pgfscope}%
\pgfpathrectangle{\pgfqpoint{1.150000in}{0.150000in}}{\pgfqpoint{5.700000in}{5.700000in}}%
\pgfusepath{clip}%
\pgfsetbuttcap%
\pgfsetroundjoin%
\definecolor{currentfill}{rgb}{0.175841,0.441290,0.557685}%
\pgfsetfillcolor{currentfill}%
\pgfsetfillopacity{0.700000}%
\pgfsetlinewidth{0.000000pt}%
\definecolor{currentstroke}{rgb}{0.000000,0.000000,0.000000}%
\pgfsetstrokecolor{currentstroke}%
\pgfsetdash{}{0pt}%
\pgfpathmoveto{\pgfqpoint{4.575518in}{3.362685in}}%
\pgfpathlineto{\pgfqpoint{4.588951in}{3.353934in}}%
\pgfpathlineto{\pgfqpoint{4.602388in}{3.345263in}}%
\pgfpathlineto{\pgfqpoint{4.615829in}{3.336670in}}%
\pgfpathlineto{\pgfqpoint{4.629275in}{3.328156in}}%
\pgfpathlineto{\pgfqpoint{4.636975in}{3.352744in}}%
\pgfpathlineto{\pgfqpoint{4.644680in}{3.377816in}}%
\pgfpathlineto{\pgfqpoint{4.652389in}{3.403380in}}%
\pgfpathlineto{\pgfqpoint{4.660104in}{3.429447in}}%
\pgfpathlineto{\pgfqpoint{4.646661in}{3.438483in}}%
\pgfpathlineto{\pgfqpoint{4.633223in}{3.447598in}}%
\pgfpathlineto{\pgfqpoint{4.619788in}{3.456791in}}%
\pgfpathlineto{\pgfqpoint{4.606357in}{3.466064in}}%
\pgfpathlineto{\pgfqpoint{4.598641in}{3.439467in}}%
\pgfpathlineto{\pgfqpoint{4.590928in}{3.413378in}}%
\pgfpathlineto{\pgfqpoint{4.583221in}{3.387787in}}%
\pgfpathlineto{\pgfqpoint{4.575518in}{3.362685in}}%
\pgfpathclose%
\pgfusepath{fill}%
\end{pgfscope}%
\begin{pgfscope}%
\pgfpathrectangle{\pgfqpoint{1.150000in}{0.150000in}}{\pgfqpoint{5.700000in}{5.700000in}}%
\pgfusepath{clip}%
\pgfsetbuttcap%
\pgfsetroundjoin%
\definecolor{currentfill}{rgb}{0.260571,0.246922,0.522828}%
\pgfsetfillcolor{currentfill}%
\pgfsetfillopacity{0.700000}%
\pgfsetlinewidth{0.000000pt}%
\definecolor{currentstroke}{rgb}{0.000000,0.000000,0.000000}%
\pgfsetstrokecolor{currentstroke}%
\pgfsetdash{}{0pt}%
\pgfpathmoveto{\pgfqpoint{3.869747in}{2.913461in}}%
\pgfpathlineto{\pgfqpoint{3.883093in}{2.905746in}}%
\pgfpathlineto{\pgfqpoint{3.896442in}{2.898123in}}%
\pgfpathlineto{\pgfqpoint{3.909796in}{2.890592in}}%
\pgfpathlineto{\pgfqpoint{3.923152in}{2.883151in}}%
\pgfpathlineto{\pgfqpoint{3.930910in}{2.898634in}}%
\pgfpathlineto{\pgfqpoint{3.938663in}{2.914357in}}%
\pgfpathlineto{\pgfqpoint{3.946412in}{2.930326in}}%
\pgfpathlineto{\pgfqpoint{3.954158in}{2.946546in}}%
\pgfpathlineto{\pgfqpoint{3.940806in}{2.954323in}}%
\pgfpathlineto{\pgfqpoint{3.927458in}{2.962190in}}%
\pgfpathlineto{\pgfqpoint{3.914114in}{2.970148in}}%
\pgfpathlineto{\pgfqpoint{3.900773in}{2.978198in}}%
\pgfpathlineto{\pgfqpoint{3.893022in}{2.961635in}}%
\pgfpathlineto{\pgfqpoint{3.885268in}{2.945328in}}%
\pgfpathlineto{\pgfqpoint{3.877509in}{2.929272in}}%
\pgfpathlineto{\pgfqpoint{3.869747in}{2.913461in}}%
\pgfpathclose%
\pgfusepath{fill}%
\end{pgfscope}%
\begin{pgfscope}%
\pgfpathrectangle{\pgfqpoint{1.150000in}{0.150000in}}{\pgfqpoint{5.700000in}{5.700000in}}%
\pgfusepath{clip}%
\pgfsetbuttcap%
\pgfsetroundjoin%
\definecolor{currentfill}{rgb}{0.265145,0.232956,0.516599}%
\pgfsetfillcolor{currentfill}%
\pgfsetfillopacity{0.700000}%
\pgfsetlinewidth{0.000000pt}%
\definecolor{currentstroke}{rgb}{0.000000,0.000000,0.000000}%
\pgfsetstrokecolor{currentstroke}%
\pgfsetdash{}{0pt}%
\pgfpathmoveto{\pgfqpoint{3.647460in}{2.887634in}}%
\pgfpathlineto{\pgfqpoint{3.660779in}{2.879390in}}%
\pgfpathlineto{\pgfqpoint{3.674100in}{2.871243in}}%
\pgfpathlineto{\pgfqpoint{3.687425in}{2.863195in}}%
\pgfpathlineto{\pgfqpoint{3.700752in}{2.855243in}}%
\pgfpathlineto{\pgfqpoint{3.708562in}{2.869887in}}%
\pgfpathlineto{\pgfqpoint{3.716368in}{2.884738in}}%
\pgfpathlineto{\pgfqpoint{3.724168in}{2.899804in}}%
\pgfpathlineto{\pgfqpoint{3.731963in}{2.915088in}}%
\pgfpathlineto{\pgfqpoint{3.718641in}{2.923335in}}%
\pgfpathlineto{\pgfqpoint{3.705322in}{2.931680in}}%
\pgfpathlineto{\pgfqpoint{3.692006in}{2.940121in}}%
\pgfpathlineto{\pgfqpoint{3.678692in}{2.948662in}}%
\pgfpathlineto{\pgfqpoint{3.670892in}{2.933074in}}%
\pgfpathlineto{\pgfqpoint{3.663087in}{2.917710in}}%
\pgfpathlineto{\pgfqpoint{3.655276in}{2.902565in}}%
\pgfpathlineto{\pgfqpoint{3.647460in}{2.887634in}}%
\pgfpathclose%
\pgfusepath{fill}%
\end{pgfscope}%
\begin{pgfscope}%
\pgfpathrectangle{\pgfqpoint{1.150000in}{0.150000in}}{\pgfqpoint{5.700000in}{5.700000in}}%
\pgfusepath{clip}%
\pgfsetbuttcap%
\pgfsetroundjoin%
\definecolor{currentfill}{rgb}{0.124395,0.578002,0.548287}%
\pgfsetfillcolor{currentfill}%
\pgfsetfillopacity{0.700000}%
\pgfsetlinewidth{0.000000pt}%
\definecolor{currentstroke}{rgb}{0.000000,0.000000,0.000000}%
\pgfsetstrokecolor{currentstroke}%
\pgfsetdash{}{0pt}%
\pgfpathmoveto{\pgfqpoint{4.614621in}{3.740776in}}%
\pgfpathlineto{\pgfqpoint{4.628037in}{3.730067in}}%
\pgfpathlineto{\pgfqpoint{4.641457in}{3.719438in}}%
\pgfpathlineto{\pgfqpoint{4.654880in}{3.708890in}}%
\pgfpathlineto{\pgfqpoint{4.668306in}{3.698423in}}%
\pgfpathlineto{\pgfqpoint{4.676081in}{3.730080in}}%
\pgfpathlineto{\pgfqpoint{4.683865in}{3.762355in}}%
\pgfpathlineto{\pgfqpoint{4.691657in}{3.795260in}}%
\pgfpathlineto{\pgfqpoint{4.678230in}{3.806170in}}%
\pgfpathlineto{\pgfqpoint{4.664805in}{3.817160in}}%
\pgfpathlineto{\pgfqpoint{4.651384in}{3.828232in}}%
\pgfpathlineto{\pgfqpoint{4.637966in}{3.839385in}}%
\pgfpathlineto{\pgfqpoint{4.630176in}{3.805883in}}%
\pgfpathlineto{\pgfqpoint{4.622394in}{3.773018in}}%
\pgfpathlineto{\pgfqpoint{4.614621in}{3.740776in}}%
\pgfpathclose%
\pgfusepath{fill}%
\end{pgfscope}%
\begin{pgfscope}%
\pgfpathrectangle{\pgfqpoint{1.150000in}{0.150000in}}{\pgfqpoint{5.700000in}{5.700000in}}%
\pgfusepath{clip}%
\pgfsetbuttcap%
\pgfsetroundjoin%
\definecolor{currentfill}{rgb}{0.253935,0.265254,0.529983}%
\pgfsetfillcolor{currentfill}%
\pgfsetfillopacity{0.700000}%
\pgfsetlinewidth{0.000000pt}%
\definecolor{currentstroke}{rgb}{0.000000,0.000000,0.000000}%
\pgfsetstrokecolor{currentstroke}%
\pgfsetdash{}{0pt}%
\pgfpathmoveto{\pgfqpoint{3.095757in}{2.962200in}}%
\pgfpathlineto{\pgfqpoint{3.109049in}{2.951401in}}%
\pgfpathlineto{\pgfqpoint{3.122342in}{2.940725in}}%
\pgfpathlineto{\pgfqpoint{3.135635in}{2.930173in}}%
\pgfpathlineto{\pgfqpoint{3.148928in}{2.919743in}}%
\pgfpathlineto{\pgfqpoint{3.156881in}{2.933385in}}%
\pgfpathlineto{\pgfqpoint{3.164826in}{2.947201in}}%
\pgfpathlineto{\pgfqpoint{3.172765in}{2.961194in}}%
\pgfpathlineto{\pgfqpoint{3.180697in}{2.975369in}}%
\pgfpathlineto{\pgfqpoint{3.167410in}{2.986015in}}%
\pgfpathlineto{\pgfqpoint{3.154123in}{2.996784in}}%
\pgfpathlineto{\pgfqpoint{3.140836in}{3.007676in}}%
\pgfpathlineto{\pgfqpoint{3.127550in}{3.018692in}}%
\pgfpathlineto{\pgfqpoint{3.119612in}{3.004293in}}%
\pgfpathlineto{\pgfqpoint{3.111668in}{2.990081in}}%
\pgfpathlineto{\pgfqpoint{3.103716in}{2.976051in}}%
\pgfpathlineto{\pgfqpoint{3.095757in}{2.962200in}}%
\pgfpathclose%
\pgfusepath{fill}%
\end{pgfscope}%
\begin{pgfscope}%
\pgfpathrectangle{\pgfqpoint{1.150000in}{0.150000in}}{\pgfqpoint{5.700000in}{5.700000in}}%
\pgfusepath{clip}%
\pgfsetbuttcap%
\pgfsetroundjoin%
\definecolor{currentfill}{rgb}{0.239346,0.300855,0.540844}%
\pgfsetfillcolor{currentfill}%
\pgfsetfillopacity{0.700000}%
\pgfsetlinewidth{0.000000pt}%
\definecolor{currentstroke}{rgb}{0.000000,0.000000,0.000000}%
\pgfsetstrokecolor{currentstroke}%
\pgfsetdash{}{0pt}%
\pgfpathmoveto{\pgfqpoint{2.904223in}{3.045838in}}%
\pgfpathlineto{\pgfqpoint{2.917530in}{3.033656in}}%
\pgfpathlineto{\pgfqpoint{2.930836in}{3.021612in}}%
\pgfpathlineto{\pgfqpoint{2.944141in}{3.009706in}}%
\pgfpathlineto{\pgfqpoint{2.957445in}{2.997934in}}%
\pgfpathlineto{\pgfqpoint{2.965446in}{3.011488in}}%
\pgfpathlineto{\pgfqpoint{2.973440in}{3.025215in}}%
\pgfpathlineto{\pgfqpoint{2.981426in}{3.039118in}}%
\pgfpathlineto{\pgfqpoint{2.989405in}{3.053203in}}%
\pgfpathlineto{\pgfqpoint{2.976107in}{3.065171in}}%
\pgfpathlineto{\pgfqpoint{2.962809in}{3.077275in}}%
\pgfpathlineto{\pgfqpoint{2.949509in}{3.089515in}}%
\pgfpathlineto{\pgfqpoint{2.936208in}{3.101894in}}%
\pgfpathlineto{\pgfqpoint{2.928224in}{3.087605in}}%
\pgfpathlineto{\pgfqpoint{2.920231in}{3.073502in}}%
\pgfpathlineto{\pgfqpoint{2.912231in}{3.059581in}}%
\pgfpathlineto{\pgfqpoint{2.904223in}{3.045838in}}%
\pgfpathclose%
\pgfusepath{fill}%
\end{pgfscope}%
\begin{pgfscope}%
\pgfpathrectangle{\pgfqpoint{1.150000in}{0.150000in}}{\pgfqpoint{5.700000in}{5.700000in}}%
\pgfusepath{clip}%
\pgfsetbuttcap%
\pgfsetroundjoin%
\definecolor{currentfill}{rgb}{0.143343,0.522773,0.556295}%
\pgfsetfillcolor{currentfill}%
\pgfsetfillopacity{0.700000}%
\pgfsetlinewidth{0.000000pt}%
\definecolor{currentstroke}{rgb}{0.000000,0.000000,0.000000}%
\pgfsetstrokecolor{currentstroke}%
\pgfsetdash{}{0pt}%
\pgfpathmoveto{\pgfqpoint{4.637280in}{3.577749in}}%
\pgfpathlineto{\pgfqpoint{4.650709in}{3.567931in}}%
\pgfpathlineto{\pgfqpoint{4.664143in}{3.558193in}}%
\pgfpathlineto{\pgfqpoint{4.677579in}{3.548534in}}%
\pgfpathlineto{\pgfqpoint{4.691020in}{3.538953in}}%
\pgfpathlineto{\pgfqpoint{4.698765in}{3.567692in}}%
\pgfpathlineto{\pgfqpoint{4.706518in}{3.596998in}}%
\pgfpathlineto{\pgfqpoint{4.714278in}{3.626882in}}%
\pgfpathlineto{\pgfqpoint{4.722046in}{3.657354in}}%
\pgfpathlineto{\pgfqpoint{4.708605in}{3.667502in}}%
\pgfpathlineto{\pgfqpoint{4.695169in}{3.677730in}}%
\pgfpathlineto{\pgfqpoint{4.681736in}{3.688037in}}%
\pgfpathlineto{\pgfqpoint{4.668306in}{3.698423in}}%
\pgfpathlineto{\pgfqpoint{4.660538in}{3.667373in}}%
\pgfpathlineto{\pgfqpoint{4.652779in}{3.636919in}}%
\pgfpathlineto{\pgfqpoint{4.645026in}{3.607048in}}%
\pgfpathlineto{\pgfqpoint{4.637280in}{3.577749in}}%
\pgfpathclose%
\pgfusepath{fill}%
\end{pgfscope}%
\begin{pgfscope}%
\pgfpathrectangle{\pgfqpoint{1.150000in}{0.150000in}}{\pgfqpoint{5.700000in}{5.700000in}}%
\pgfusepath{clip}%
\pgfsetbuttcap%
\pgfsetroundjoin%
\definecolor{currentfill}{rgb}{0.192357,0.403199,0.555836}%
\pgfsetfillcolor{currentfill}%
\pgfsetfillopacity{0.700000}%
\pgfsetlinewidth{0.000000pt}%
\definecolor{currentstroke}{rgb}{0.000000,0.000000,0.000000}%
\pgfsetstrokecolor{currentstroke}%
\pgfsetdash{}{0pt}%
\pgfpathmoveto{\pgfqpoint{4.544740in}{3.266965in}}%
\pgfpathlineto{\pgfqpoint{4.558177in}{3.258715in}}%
\pgfpathlineto{\pgfqpoint{4.571618in}{3.250543in}}%
\pgfpathlineto{\pgfqpoint{4.585064in}{3.242450in}}%
\pgfpathlineto{\pgfqpoint{4.598514in}{3.234435in}}%
\pgfpathlineto{\pgfqpoint{4.606199in}{3.257190in}}%
\pgfpathlineto{\pgfqpoint{4.613887in}{3.280388in}}%
\pgfpathlineto{\pgfqpoint{4.621579in}{3.304040in}}%
\pgfpathlineto{\pgfqpoint{4.629275in}{3.328156in}}%
\pgfpathlineto{\pgfqpoint{4.615829in}{3.336670in}}%
\pgfpathlineto{\pgfqpoint{4.602388in}{3.345263in}}%
\pgfpathlineto{\pgfqpoint{4.588951in}{3.353934in}}%
\pgfpathlineto{\pgfqpoint{4.575518in}{3.362685in}}%
\pgfpathlineto{\pgfqpoint{4.567818in}{3.338062in}}%
\pgfpathlineto{\pgfqpoint{4.560122in}{3.313907in}}%
\pgfpathlineto{\pgfqpoint{4.552430in}{3.290211in}}%
\pgfpathlineto{\pgfqpoint{4.544740in}{3.266965in}}%
\pgfpathclose%
\pgfusepath{fill}%
\end{pgfscope}%
\begin{pgfscope}%
\pgfpathrectangle{\pgfqpoint{1.150000in}{0.150000in}}{\pgfqpoint{5.700000in}{5.700000in}}%
\pgfusepath{clip}%
\pgfsetbuttcap%
\pgfsetroundjoin%
\definecolor{currentfill}{rgb}{0.237441,0.305202,0.541921}%
\pgfsetfillcolor{currentfill}%
\pgfsetfillopacity{0.700000}%
\pgfsetlinewidth{0.000000pt}%
\definecolor{currentstroke}{rgb}{0.000000,0.000000,0.000000}%
\pgfsetstrokecolor{currentstroke}%
\pgfsetdash{}{0pt}%
\pgfpathmoveto{\pgfqpoint{4.260777in}{3.032710in}}%
\pgfpathlineto{\pgfqpoint{4.274181in}{3.025276in}}%
\pgfpathlineto{\pgfqpoint{4.287589in}{3.017926in}}%
\pgfpathlineto{\pgfqpoint{4.301002in}{3.010658in}}%
\pgfpathlineto{\pgfqpoint{4.314419in}{3.003472in}}%
\pgfpathlineto{\pgfqpoint{4.322105in}{3.021534in}}%
\pgfpathlineto{\pgfqpoint{4.329791in}{3.039921in}}%
\pgfpathlineto{\pgfqpoint{4.337475in}{3.058639in}}%
\pgfpathlineto{\pgfqpoint{4.345159in}{3.077697in}}%
\pgfpathlineto{\pgfqpoint{4.331748in}{3.085299in}}%
\pgfpathlineto{\pgfqpoint{4.318340in}{3.092983in}}%
\pgfpathlineto{\pgfqpoint{4.304937in}{3.100750in}}%
\pgfpathlineto{\pgfqpoint{4.291539in}{3.108600in}}%
\pgfpathlineto{\pgfqpoint{4.283849in}{3.089118in}}%
\pgfpathlineto{\pgfqpoint{4.276160in}{3.069981in}}%
\pgfpathlineto{\pgfqpoint{4.268469in}{3.051181in}}%
\pgfpathlineto{\pgfqpoint{4.260777in}{3.032710in}}%
\pgfpathclose%
\pgfusepath{fill}%
\end{pgfscope}%
\begin{pgfscope}%
\pgfpathrectangle{\pgfqpoint{1.150000in}{0.150000in}}{\pgfqpoint{5.700000in}{5.700000in}}%
\pgfusepath{clip}%
\pgfsetbuttcap%
\pgfsetroundjoin%
\definecolor{currentfill}{rgb}{0.229739,0.322361,0.545706}%
\pgfsetfillcolor{currentfill}%
\pgfsetfillopacity{0.700000}%
\pgfsetlinewidth{0.000000pt}%
\definecolor{currentstroke}{rgb}{0.000000,0.000000,0.000000}%
\pgfsetstrokecolor{currentstroke}%
\pgfsetdash{}{0pt}%
\pgfpathmoveto{\pgfqpoint{4.345159in}{3.077697in}}%
\pgfpathlineto{\pgfqpoint{4.358575in}{3.070176in}}%
\pgfpathlineto{\pgfqpoint{4.371996in}{3.062737in}}%
\pgfpathlineto{\pgfqpoint{4.385421in}{3.055379in}}%
\pgfpathlineto{\pgfqpoint{4.398850in}{3.048102in}}%
\pgfpathlineto{\pgfqpoint{4.406528in}{3.067078in}}%
\pgfpathlineto{\pgfqpoint{4.414206in}{3.086404in}}%
\pgfpathlineto{\pgfqpoint{4.421885in}{3.106089in}}%
\pgfpathlineto{\pgfqpoint{4.429564in}{3.126139in}}%
\pgfpathlineto{\pgfqpoint{4.416140in}{3.133853in}}%
\pgfpathlineto{\pgfqpoint{4.402720in}{3.141648in}}%
\pgfpathlineto{\pgfqpoint{4.389305in}{3.149524in}}%
\pgfpathlineto{\pgfqpoint{4.375893in}{3.157481in}}%
\pgfpathlineto{\pgfqpoint{4.368210in}{3.136986in}}%
\pgfpathlineto{\pgfqpoint{4.360526in}{3.116862in}}%
\pgfpathlineto{\pgfqpoint{4.352843in}{3.097102in}}%
\pgfpathlineto{\pgfqpoint{4.345159in}{3.077697in}}%
\pgfpathclose%
\pgfusepath{fill}%
\end{pgfscope}%
\begin{pgfscope}%
\pgfpathrectangle{\pgfqpoint{1.150000in}{0.150000in}}{\pgfqpoint{5.700000in}{5.700000in}}%
\pgfusepath{clip}%
\pgfsetbuttcap%
\pgfsetroundjoin%
\definecolor{currentfill}{rgb}{0.246811,0.283237,0.535941}%
\pgfsetfillcolor{currentfill}%
\pgfsetfillopacity{0.700000}%
\pgfsetlinewidth{0.000000pt}%
\definecolor{currentstroke}{rgb}{0.000000,0.000000,0.000000}%
\pgfsetstrokecolor{currentstroke}%
\pgfsetdash{}{0pt}%
\pgfpathmoveto{\pgfqpoint{4.176400in}{2.990951in}}%
\pgfpathlineto{\pgfqpoint{4.189792in}{2.983578in}}%
\pgfpathlineto{\pgfqpoint{4.203189in}{2.976290in}}%
\pgfpathlineto{\pgfqpoint{4.216589in}{2.969086in}}%
\pgfpathlineto{\pgfqpoint{4.229995in}{2.961965in}}%
\pgfpathlineto{\pgfqpoint{4.237693in}{2.979195in}}%
\pgfpathlineto{\pgfqpoint{4.245389in}{2.996724in}}%
\pgfpathlineto{\pgfqpoint{4.253084in}{3.014560in}}%
\pgfpathlineto{\pgfqpoint{4.260777in}{3.032710in}}%
\pgfpathlineto{\pgfqpoint{4.247377in}{3.040226in}}%
\pgfpathlineto{\pgfqpoint{4.233982in}{3.047827in}}%
\pgfpathlineto{\pgfqpoint{4.220591in}{3.055511in}}%
\pgfpathlineto{\pgfqpoint{4.207204in}{3.063279in}}%
\pgfpathlineto{\pgfqpoint{4.199506in}{3.044726in}}%
\pgfpathlineto{\pgfqpoint{4.191806in}{3.026492in}}%
\pgfpathlineto{\pgfqpoint{4.184104in}{3.008569in}}%
\pgfpathlineto{\pgfqpoint{4.176400in}{2.990951in}}%
\pgfpathclose%
\pgfusepath{fill}%
\end{pgfscope}%
\begin{pgfscope}%
\pgfpathrectangle{\pgfqpoint{1.150000in}{0.150000in}}{\pgfqpoint{5.700000in}{5.700000in}}%
\pgfusepath{clip}%
\pgfsetbuttcap%
\pgfsetroundjoin%
\definecolor{currentfill}{rgb}{0.218130,0.347432,0.550038}%
\pgfsetfillcolor{currentfill}%
\pgfsetfillopacity{0.700000}%
\pgfsetlinewidth{0.000000pt}%
\definecolor{currentstroke}{rgb}{0.000000,0.000000,0.000000}%
\pgfsetstrokecolor{currentstroke}%
\pgfsetdash{}{0pt}%
\pgfpathmoveto{\pgfqpoint{4.429564in}{3.126139in}}%
\pgfpathlineto{\pgfqpoint{4.442992in}{3.118506in}}%
\pgfpathlineto{\pgfqpoint{4.456425in}{3.110953in}}%
\pgfpathlineto{\pgfqpoint{4.469863in}{3.103479in}}%
\pgfpathlineto{\pgfqpoint{4.483305in}{3.096085in}}%
\pgfpathlineto{\pgfqpoint{4.490979in}{3.116061in}}%
\pgfpathlineto{\pgfqpoint{4.498654in}{3.136415in}}%
\pgfpathlineto{\pgfqpoint{4.506330in}{3.157156in}}%
\pgfpathlineto{\pgfqpoint{4.514008in}{3.178292in}}%
\pgfpathlineto{\pgfqpoint{4.500572in}{3.186143in}}%
\pgfpathlineto{\pgfqpoint{4.487139in}{3.194074in}}%
\pgfpathlineto{\pgfqpoint{4.473711in}{3.202085in}}%
\pgfpathlineto{\pgfqpoint{4.460287in}{3.210176in}}%
\pgfpathlineto{\pgfqpoint{4.452605in}{3.188575in}}%
\pgfpathlineto{\pgfqpoint{4.444923in}{3.167374in}}%
\pgfpathlineto{\pgfqpoint{4.437243in}{3.146565in}}%
\pgfpathlineto{\pgfqpoint{4.429564in}{3.126139in}}%
\pgfpathclose%
\pgfusepath{fill}%
\end{pgfscope}%
\begin{pgfscope}%
\pgfpathrectangle{\pgfqpoint{1.150000in}{0.150000in}}{\pgfqpoint{5.700000in}{5.700000in}}%
\pgfusepath{clip}%
\pgfsetbuttcap%
\pgfsetroundjoin%
\definecolor{currentfill}{rgb}{0.252194,0.269783,0.531579}%
\pgfsetfillcolor{currentfill}%
\pgfsetfillopacity{0.700000}%
\pgfsetlinewidth{0.000000pt}%
\definecolor{currentstroke}{rgb}{0.000000,0.000000,0.000000}%
\pgfsetstrokecolor{currentstroke}%
\pgfsetdash{}{0pt}%
\pgfpathmoveto{\pgfqpoint{4.092013in}{2.952218in}}%
\pgfpathlineto{\pgfqpoint{4.105394in}{2.944880in}}%
\pgfpathlineto{\pgfqpoint{4.118779in}{2.937627in}}%
\pgfpathlineto{\pgfqpoint{4.132169in}{2.930461in}}%
\pgfpathlineto{\pgfqpoint{4.145562in}{2.923379in}}%
\pgfpathlineto{\pgfqpoint{4.153275in}{2.939851in}}%
\pgfpathlineto{\pgfqpoint{4.160986in}{2.956599in}}%
\pgfpathlineto{\pgfqpoint{4.168694in}{2.973630in}}%
\pgfpathlineto{\pgfqpoint{4.176400in}{2.990951in}}%
\pgfpathlineto{\pgfqpoint{4.163012in}{2.998408in}}%
\pgfpathlineto{\pgfqpoint{4.149628in}{3.005951in}}%
\pgfpathlineto{\pgfqpoint{4.136248in}{3.013579in}}%
\pgfpathlineto{\pgfqpoint{4.122872in}{3.021293in}}%
\pgfpathlineto{\pgfqpoint{4.115161in}{3.003589in}}%
\pgfpathlineto{\pgfqpoint{4.107448in}{2.986179in}}%
\pgfpathlineto{\pgfqpoint{4.099732in}{2.969058in}}%
\pgfpathlineto{\pgfqpoint{4.092013in}{2.952218in}}%
\pgfpathclose%
\pgfusepath{fill}%
\end{pgfscope}%
\begin{pgfscope}%
\pgfpathrectangle{\pgfqpoint{1.150000in}{0.150000in}}{\pgfqpoint{5.700000in}{5.700000in}}%
\pgfusepath{clip}%
\pgfsetbuttcap%
\pgfsetroundjoin%
\definecolor{currentfill}{rgb}{0.265145,0.232956,0.516599}%
\pgfsetfillcolor{currentfill}%
\pgfsetfillopacity{0.700000}%
\pgfsetlinewidth{0.000000pt}%
\definecolor{currentstroke}{rgb}{0.000000,0.000000,0.000000}%
\pgfsetstrokecolor{currentstroke}%
\pgfsetdash{}{0pt}%
\pgfpathmoveto{\pgfqpoint{3.785282in}{2.883059in}}%
\pgfpathlineto{\pgfqpoint{3.798620in}{2.875289in}}%
\pgfpathlineto{\pgfqpoint{3.811961in}{2.867613in}}%
\pgfpathlineto{\pgfqpoint{3.825306in}{2.860029in}}%
\pgfpathlineto{\pgfqpoint{3.838654in}{2.852538in}}%
\pgfpathlineto{\pgfqpoint{3.846433in}{2.867432in}}%
\pgfpathlineto{\pgfqpoint{3.854209in}{2.882546in}}%
\pgfpathlineto{\pgfqpoint{3.861980in}{2.897887in}}%
\pgfpathlineto{\pgfqpoint{3.869747in}{2.913461in}}%
\pgfpathlineto{\pgfqpoint{3.856404in}{2.921267in}}%
\pgfpathlineto{\pgfqpoint{3.843065in}{2.929166in}}%
\pgfpathlineto{\pgfqpoint{3.829729in}{2.937158in}}%
\pgfpathlineto{\pgfqpoint{3.816397in}{2.945244in}}%
\pgfpathlineto{\pgfqpoint{3.808625in}{2.929348in}}%
\pgfpathlineto{\pgfqpoint{3.800848in}{2.913689in}}%
\pgfpathlineto{\pgfqpoint{3.793068in}{2.898261in}}%
\pgfpathlineto{\pgfqpoint{3.785282in}{2.883059in}}%
\pgfpathclose%
\pgfusepath{fill}%
\end{pgfscope}%
\begin{pgfscope}%
\pgfpathrectangle{\pgfqpoint{1.150000in}{0.150000in}}{\pgfqpoint{5.700000in}{5.700000in}}%
\pgfusepath{clip}%
\pgfsetbuttcap%
\pgfsetroundjoin%
\definecolor{currentfill}{rgb}{0.266580,0.228262,0.514349}%
\pgfsetfillcolor{currentfill}%
\pgfsetfillopacity{0.700000}%
\pgfsetlinewidth{0.000000pt}%
\definecolor{currentstroke}{rgb}{0.000000,0.000000,0.000000}%
\pgfsetstrokecolor{currentstroke}%
\pgfsetdash{}{0pt}%
\pgfpathmoveto{\pgfqpoint{3.424980in}{2.875710in}}%
\pgfpathlineto{\pgfqpoint{3.438281in}{2.866765in}}%
\pgfpathlineto{\pgfqpoint{3.451584in}{2.857927in}}%
\pgfpathlineto{\pgfqpoint{3.464889in}{2.849194in}}%
\pgfpathlineto{\pgfqpoint{3.478196in}{2.840567in}}%
\pgfpathlineto{\pgfqpoint{3.486067in}{2.854472in}}%
\pgfpathlineto{\pgfqpoint{3.493933in}{2.868561in}}%
\pgfpathlineto{\pgfqpoint{3.501793in}{2.882837in}}%
\pgfpathlineto{\pgfqpoint{3.509646in}{2.897306in}}%
\pgfpathlineto{\pgfqpoint{3.496345in}{2.906189in}}%
\pgfpathlineto{\pgfqpoint{3.483045in}{2.915177in}}%
\pgfpathlineto{\pgfqpoint{3.469748in}{2.924271in}}%
\pgfpathlineto{\pgfqpoint{3.456452in}{2.933472in}}%
\pgfpathlineto{\pgfqpoint{3.448593in}{2.918739in}}%
\pgfpathlineto{\pgfqpoint{3.440728in}{2.904205in}}%
\pgfpathlineto{\pgfqpoint{3.432857in}{2.889863in}}%
\pgfpathlineto{\pgfqpoint{3.424980in}{2.875710in}}%
\pgfpathclose%
\pgfusepath{fill}%
\end{pgfscope}%
\begin{pgfscope}%
\pgfpathrectangle{\pgfqpoint{1.150000in}{0.150000in}}{\pgfqpoint{5.700000in}{5.700000in}}%
\pgfusepath{clip}%
\pgfsetbuttcap%
\pgfsetroundjoin%
\definecolor{currentfill}{rgb}{0.263663,0.237631,0.518762}%
\pgfsetfillcolor{currentfill}%
\pgfsetfillopacity{0.700000}%
\pgfsetlinewidth{0.000000pt}%
\definecolor{currentstroke}{rgb}{0.000000,0.000000,0.000000}%
\pgfsetstrokecolor{currentstroke}%
\pgfsetdash{}{0pt}%
\pgfpathmoveto{\pgfqpoint{3.287016in}{2.894483in}}%
\pgfpathlineto{\pgfqpoint{3.300310in}{2.884894in}}%
\pgfpathlineto{\pgfqpoint{3.313606in}{2.875418in}}%
\pgfpathlineto{\pgfqpoint{3.326902in}{2.866054in}}%
\pgfpathlineto{\pgfqpoint{3.340200in}{2.856801in}}%
\pgfpathlineto{\pgfqpoint{3.348108in}{2.870476in}}%
\pgfpathlineto{\pgfqpoint{3.356010in}{2.884326in}}%
\pgfpathlineto{\pgfqpoint{3.363905in}{2.898355in}}%
\pgfpathlineto{\pgfqpoint{3.371794in}{2.912569in}}%
\pgfpathlineto{\pgfqpoint{3.358501in}{2.922058in}}%
\pgfpathlineto{\pgfqpoint{3.345210in}{2.931658in}}%
\pgfpathlineto{\pgfqpoint{3.331921in}{2.941370in}}%
\pgfpathlineto{\pgfqpoint{3.318632in}{2.951195in}}%
\pgfpathlineto{\pgfqpoint{3.310738in}{2.936738in}}%
\pgfpathlineto{\pgfqpoint{3.302837in}{2.922470in}}%
\pgfpathlineto{\pgfqpoint{3.294930in}{2.908387in}}%
\pgfpathlineto{\pgfqpoint{3.287016in}{2.894483in}}%
\pgfpathclose%
\pgfusepath{fill}%
\end{pgfscope}%
\begin{pgfscope}%
\pgfpathrectangle{\pgfqpoint{1.150000in}{0.150000in}}{\pgfqpoint{5.700000in}{5.700000in}}%
\pgfusepath{clip}%
\pgfsetbuttcap%
\pgfsetroundjoin%
\definecolor{currentfill}{rgb}{0.246811,0.283237,0.535941}%
\pgfsetfillcolor{currentfill}%
\pgfsetfillopacity{0.700000}%
\pgfsetlinewidth{0.000000pt}%
\definecolor{currentstroke}{rgb}{0.000000,0.000000,0.000000}%
\pgfsetstrokecolor{currentstroke}%
\pgfsetdash{}{0pt}%
\pgfpathmoveto{\pgfqpoint{2.957445in}{2.997934in}}%
\pgfpathlineto{\pgfqpoint{2.970747in}{2.986297in}}%
\pgfpathlineto{\pgfqpoint{2.984049in}{2.974794in}}%
\pgfpathlineto{\pgfqpoint{2.997350in}{2.963422in}}%
\pgfpathlineto{\pgfqpoint{3.010650in}{2.952181in}}%
\pgfpathlineto{\pgfqpoint{3.018645in}{2.965546in}}%
\pgfpathlineto{\pgfqpoint{3.026633in}{2.979078in}}%
\pgfpathlineto{\pgfqpoint{3.034612in}{2.992783in}}%
\pgfpathlineto{\pgfqpoint{3.042585in}{3.006664in}}%
\pgfpathlineto{\pgfqpoint{3.029291in}{3.018101in}}%
\pgfpathlineto{\pgfqpoint{3.015996in}{3.029669in}}%
\pgfpathlineto{\pgfqpoint{3.002701in}{3.041370in}}%
\pgfpathlineto{\pgfqpoint{2.989405in}{3.053203in}}%
\pgfpathlineto{\pgfqpoint{2.981426in}{3.039118in}}%
\pgfpathlineto{\pgfqpoint{2.973440in}{3.025215in}}%
\pgfpathlineto{\pgfqpoint{2.965446in}{3.011488in}}%
\pgfpathlineto{\pgfqpoint{2.957445in}{2.997934in}}%
\pgfpathclose%
\pgfusepath{fill}%
\end{pgfscope}%
\begin{pgfscope}%
\pgfpathrectangle{\pgfqpoint{1.150000in}{0.150000in}}{\pgfqpoint{5.700000in}{5.700000in}}%
\pgfusepath{clip}%
\pgfsetbuttcap%
\pgfsetroundjoin%
\definecolor{currentfill}{rgb}{0.127568,0.566949,0.550556}%
\pgfsetfillcolor{currentfill}%
\pgfsetfillopacity{0.700000}%
\pgfsetlinewidth{0.000000pt}%
\definecolor{currentstroke}{rgb}{0.000000,0.000000,0.000000}%
\pgfsetstrokecolor{currentstroke}%
\pgfsetdash{}{0pt}%
\pgfpathmoveto{\pgfqpoint{4.668306in}{3.698423in}}%
\pgfpathlineto{\pgfqpoint{4.681736in}{3.688037in}}%
\pgfpathlineto{\pgfqpoint{4.695169in}{3.677730in}}%
\pgfpathlineto{\pgfqpoint{4.708605in}{3.667502in}}%
\pgfpathlineto{\pgfqpoint{4.722046in}{3.657354in}}%
\pgfpathlineto{\pgfqpoint{4.729822in}{3.688428in}}%
\pgfpathlineto{\pgfqpoint{4.737606in}{3.720114in}}%
\pgfpathlineto{\pgfqpoint{4.745400in}{3.752423in}}%
\pgfpathlineto{\pgfqpoint{4.731959in}{3.763013in}}%
\pgfpathlineto{\pgfqpoint{4.718522in}{3.773682in}}%
\pgfpathlineto{\pgfqpoint{4.705088in}{3.784431in}}%
\pgfpathlineto{\pgfqpoint{4.691657in}{3.795260in}}%
\pgfpathlineto{\pgfqpoint{4.683865in}{3.762355in}}%
\pgfpathlineto{\pgfqpoint{4.676081in}{3.730080in}}%
\pgfpathlineto{\pgfqpoint{4.668306in}{3.698423in}}%
\pgfpathclose%
\pgfusepath{fill}%
\end{pgfscope}%
\begin{pgfscope}%
\pgfpathrectangle{\pgfqpoint{1.150000in}{0.150000in}}{\pgfqpoint{5.700000in}{5.700000in}}%
\pgfusepath{clip}%
\pgfsetbuttcap%
\pgfsetroundjoin%
\definecolor{currentfill}{rgb}{0.267968,0.223549,0.512008}%
\pgfsetfillcolor{currentfill}%
\pgfsetfillopacity{0.700000}%
\pgfsetlinewidth{0.000000pt}%
\definecolor{currentstroke}{rgb}{0.000000,0.000000,0.000000}%
\pgfsetstrokecolor{currentstroke}%
\pgfsetdash{}{0pt}%
\pgfpathmoveto{\pgfqpoint{3.562875in}{2.862810in}}%
\pgfpathlineto{\pgfqpoint{3.576189in}{2.854441in}}%
\pgfpathlineto{\pgfqpoint{3.589505in}{2.846173in}}%
\pgfpathlineto{\pgfqpoint{3.602823in}{2.838006in}}%
\pgfpathlineto{\pgfqpoint{3.616145in}{2.829938in}}%
\pgfpathlineto{\pgfqpoint{3.623982in}{2.844068in}}%
\pgfpathlineto{\pgfqpoint{3.631813in}{2.858391in}}%
\pgfpathlineto{\pgfqpoint{3.639639in}{2.872911in}}%
\pgfpathlineto{\pgfqpoint{3.647460in}{2.887634in}}%
\pgfpathlineto{\pgfqpoint{3.634145in}{2.895978in}}%
\pgfpathlineto{\pgfqpoint{3.620832in}{2.904421in}}%
\pgfpathlineto{\pgfqpoint{3.607521in}{2.912964in}}%
\pgfpathlineto{\pgfqpoint{3.594213in}{2.921608in}}%
\pgfpathlineto{\pgfqpoint{3.586387in}{2.906602in}}%
\pgfpathlineto{\pgfqpoint{3.578555in}{2.891804in}}%
\pgfpathlineto{\pgfqpoint{3.570718in}{2.877208in}}%
\pgfpathlineto{\pgfqpoint{3.562875in}{2.862810in}}%
\pgfpathclose%
\pgfusepath{fill}%
\end{pgfscope}%
\begin{pgfscope}%
\pgfpathrectangle{\pgfqpoint{1.150000in}{0.150000in}}{\pgfqpoint{5.700000in}{5.700000in}}%
\pgfusepath{clip}%
\pgfsetbuttcap%
\pgfsetroundjoin%
\definecolor{currentfill}{rgb}{0.258965,0.251537,0.524736}%
\pgfsetfillcolor{currentfill}%
\pgfsetfillopacity{0.700000}%
\pgfsetlinewidth{0.000000pt}%
\definecolor{currentstroke}{rgb}{0.000000,0.000000,0.000000}%
\pgfsetstrokecolor{currentstroke}%
\pgfsetdash{}{0pt}%
\pgfpathmoveto{\pgfqpoint{4.007601in}{2.916335in}}%
\pgfpathlineto{\pgfqpoint{4.020972in}{2.909004in}}%
\pgfpathlineto{\pgfqpoint{4.034347in}{2.901761in}}%
\pgfpathlineto{\pgfqpoint{4.047725in}{2.894605in}}%
\pgfpathlineto{\pgfqpoint{4.061108in}{2.887535in}}%
\pgfpathlineto{\pgfqpoint{4.068839in}{2.903318in}}%
\pgfpathlineto{\pgfqpoint{4.076567in}{2.919355in}}%
\pgfpathlineto{\pgfqpoint{4.084291in}{2.935653in}}%
\pgfpathlineto{\pgfqpoint{4.092013in}{2.952218in}}%
\pgfpathlineto{\pgfqpoint{4.078636in}{2.959643in}}%
\pgfpathlineto{\pgfqpoint{4.065263in}{2.967155in}}%
\pgfpathlineto{\pgfqpoint{4.051894in}{2.974754in}}%
\pgfpathlineto{\pgfqpoint{4.038529in}{2.982440in}}%
\pgfpathlineto{\pgfqpoint{4.030802in}{2.965512in}}%
\pgfpathlineto{\pgfqpoint{4.023072in}{2.948856in}}%
\pgfpathlineto{\pgfqpoint{4.015338in}{2.932466in}}%
\pgfpathlineto{\pgfqpoint{4.007601in}{2.916335in}}%
\pgfpathclose%
\pgfusepath{fill}%
\end{pgfscope}%
\begin{pgfscope}%
\pgfpathrectangle{\pgfqpoint{1.150000in}{0.150000in}}{\pgfqpoint{5.700000in}{5.700000in}}%
\pgfusepath{clip}%
\pgfsetbuttcap%
\pgfsetroundjoin%
\definecolor{currentfill}{rgb}{0.260571,0.246922,0.522828}%
\pgfsetfillcolor{currentfill}%
\pgfsetfillopacity{0.700000}%
\pgfsetlinewidth{0.000000pt}%
\definecolor{currentstroke}{rgb}{0.000000,0.000000,0.000000}%
\pgfsetstrokecolor{currentstroke}%
\pgfsetdash{}{0pt}%
\pgfpathmoveto{\pgfqpoint{3.148928in}{2.919743in}}%
\pgfpathlineto{\pgfqpoint{3.162221in}{2.909434in}}%
\pgfpathlineto{\pgfqpoint{3.175515in}{2.899245in}}%
\pgfpathlineto{\pgfqpoint{3.188810in}{2.889176in}}%
\pgfpathlineto{\pgfqpoint{3.202105in}{2.879225in}}%
\pgfpathlineto{\pgfqpoint{3.210051in}{2.892659in}}%
\pgfpathlineto{\pgfqpoint{3.217991in}{2.906261in}}%
\pgfpathlineto{\pgfqpoint{3.225924in}{2.920036in}}%
\pgfpathlineto{\pgfqpoint{3.233850in}{2.933987in}}%
\pgfpathlineto{\pgfqpoint{3.220561in}{2.944154in}}%
\pgfpathlineto{\pgfqpoint{3.207273in}{2.954439in}}%
\pgfpathlineto{\pgfqpoint{3.193985in}{2.964844in}}%
\pgfpathlineto{\pgfqpoint{3.180697in}{2.975369in}}%
\pgfpathlineto{\pgfqpoint{3.172765in}{2.961194in}}%
\pgfpathlineto{\pgfqpoint{3.164826in}{2.947201in}}%
\pgfpathlineto{\pgfqpoint{3.156881in}{2.933385in}}%
\pgfpathlineto{\pgfqpoint{3.148928in}{2.919743in}}%
\pgfpathclose%
\pgfusepath{fill}%
\end{pgfscope}%
\begin{pgfscope}%
\pgfpathrectangle{\pgfqpoint{1.150000in}{0.150000in}}{\pgfqpoint{5.700000in}{5.700000in}}%
\pgfusepath{clip}%
\pgfsetbuttcap%
\pgfsetroundjoin%
\definecolor{currentfill}{rgb}{0.163625,0.471133,0.558148}%
\pgfsetfillcolor{currentfill}%
\pgfsetfillopacity{0.700000}%
\pgfsetlinewidth{0.000000pt}%
\definecolor{currentstroke}{rgb}{0.000000,0.000000,0.000000}%
\pgfsetstrokecolor{currentstroke}%
\pgfsetdash{}{0pt}%
\pgfpathmoveto{\pgfqpoint{4.660104in}{3.429447in}}%
\pgfpathlineto{\pgfqpoint{4.673551in}{3.420489in}}%
\pgfpathlineto{\pgfqpoint{4.687002in}{3.411610in}}%
\pgfpathlineto{\pgfqpoint{4.700457in}{3.402808in}}%
\pgfpathlineto{\pgfqpoint{4.713916in}{3.394084in}}%
\pgfpathlineto{\pgfqpoint{4.721633in}{3.420129in}}%
\pgfpathlineto{\pgfqpoint{4.729356in}{3.446692in}}%
\pgfpathlineto{\pgfqpoint{4.737086in}{3.473784in}}%
\pgfpathlineto{\pgfqpoint{4.744822in}{3.501415in}}%
\pgfpathlineto{\pgfqpoint{4.731366in}{3.510682in}}%
\pgfpathlineto{\pgfqpoint{4.717913in}{3.520028in}}%
\pgfpathlineto{\pgfqpoint{4.704465in}{3.529451in}}%
\pgfpathlineto{\pgfqpoint{4.691020in}{3.538953in}}%
\pgfpathlineto{\pgfqpoint{4.683282in}{3.510770in}}%
\pgfpathlineto{\pgfqpoint{4.675550in}{3.483132in}}%
\pgfpathlineto{\pgfqpoint{4.667824in}{3.456027in}}%
\pgfpathlineto{\pgfqpoint{4.660104in}{3.429447in}}%
\pgfpathclose%
\pgfusepath{fill}%
\end{pgfscope}%
\begin{pgfscope}%
\pgfpathrectangle{\pgfqpoint{1.150000in}{0.150000in}}{\pgfqpoint{5.700000in}{5.700000in}}%
\pgfusepath{clip}%
\pgfsetbuttcap%
\pgfsetroundjoin%
\definecolor{currentfill}{rgb}{0.208623,0.367752,0.552675}%
\pgfsetfillcolor{currentfill}%
\pgfsetfillopacity{0.700000}%
\pgfsetlinewidth{0.000000pt}%
\definecolor{currentstroke}{rgb}{0.000000,0.000000,0.000000}%
\pgfsetstrokecolor{currentstroke}%
\pgfsetdash{}{0pt}%
\pgfpathmoveto{\pgfqpoint{4.514008in}{3.178292in}}%
\pgfpathlineto{\pgfqpoint{4.527450in}{3.170520in}}%
\pgfpathlineto{\pgfqpoint{4.540896in}{3.162827in}}%
\pgfpathlineto{\pgfqpoint{4.554346in}{3.155212in}}%
\pgfpathlineto{\pgfqpoint{4.567802in}{3.147675in}}%
\pgfpathlineto{\pgfqpoint{4.575476in}{3.168745in}}%
\pgfpathlineto{\pgfqpoint{4.583153in}{3.190222in}}%
\pgfpathlineto{\pgfqpoint{4.590832in}{3.212116in}}%
\pgfpathlineto{\pgfqpoint{4.598514in}{3.234435in}}%
\pgfpathlineto{\pgfqpoint{4.585064in}{3.242450in}}%
\pgfpathlineto{\pgfqpoint{4.571618in}{3.250543in}}%
\pgfpathlineto{\pgfqpoint{4.558177in}{3.258715in}}%
\pgfpathlineto{\pgfqpoint{4.544740in}{3.266965in}}%
\pgfpathlineto{\pgfqpoint{4.537054in}{3.244160in}}%
\pgfpathlineto{\pgfqpoint{4.529370in}{3.221785in}}%
\pgfpathlineto{\pgfqpoint{4.521688in}{3.199832in}}%
\pgfpathlineto{\pgfqpoint{4.514008in}{3.178292in}}%
\pgfpathclose%
\pgfusepath{fill}%
\end{pgfscope}%
\begin{pgfscope}%
\pgfpathrectangle{\pgfqpoint{1.150000in}{0.150000in}}{\pgfqpoint{5.700000in}{5.700000in}}%
\pgfusepath{clip}%
\pgfsetbuttcap%
\pgfsetroundjoin%
\definecolor{currentfill}{rgb}{0.180629,0.429975,0.557282}%
\pgfsetfillcolor{currentfill}%
\pgfsetfillopacity{0.700000}%
\pgfsetlinewidth{0.000000pt}%
\definecolor{currentstroke}{rgb}{0.000000,0.000000,0.000000}%
\pgfsetstrokecolor{currentstroke}%
\pgfsetdash{}{0pt}%
\pgfpathmoveto{\pgfqpoint{4.629275in}{3.328156in}}%
\pgfpathlineto{\pgfqpoint{4.642725in}{3.319720in}}%
\pgfpathlineto{\pgfqpoint{4.656180in}{3.311361in}}%
\pgfpathlineto{\pgfqpoint{4.669639in}{3.303081in}}%
\pgfpathlineto{\pgfqpoint{4.683102in}{3.294877in}}%
\pgfpathlineto{\pgfqpoint{4.690798in}{3.318953in}}%
\pgfpathlineto{\pgfqpoint{4.698499in}{3.343506in}}%
\pgfpathlineto{\pgfqpoint{4.706205in}{3.368546in}}%
\pgfpathlineto{\pgfqpoint{4.713916in}{3.394084in}}%
\pgfpathlineto{\pgfqpoint{4.700457in}{3.402808in}}%
\pgfpathlineto{\pgfqpoint{4.687002in}{3.411610in}}%
\pgfpathlineto{\pgfqpoint{4.673551in}{3.420489in}}%
\pgfpathlineto{\pgfqpoint{4.660104in}{3.429447in}}%
\pgfpathlineto{\pgfqpoint{4.652389in}{3.403380in}}%
\pgfpathlineto{\pgfqpoint{4.644680in}{3.377816in}}%
\pgfpathlineto{\pgfqpoint{4.636975in}{3.352744in}}%
\pgfpathlineto{\pgfqpoint{4.629275in}{3.328156in}}%
\pgfpathclose%
\pgfusepath{fill}%
\end{pgfscope}%
\begin{pgfscope}%
\pgfpathrectangle{\pgfqpoint{1.150000in}{0.150000in}}{\pgfqpoint{5.700000in}{5.700000in}}%
\pgfusepath{clip}%
\pgfsetbuttcap%
\pgfsetroundjoin%
\definecolor{currentfill}{rgb}{0.147607,0.511733,0.557049}%
\pgfsetfillcolor{currentfill}%
\pgfsetfillopacity{0.700000}%
\pgfsetlinewidth{0.000000pt}%
\definecolor{currentstroke}{rgb}{0.000000,0.000000,0.000000}%
\pgfsetstrokecolor{currentstroke}%
\pgfsetdash{}{0pt}%
\pgfpathmoveto{\pgfqpoint{4.691020in}{3.538953in}}%
\pgfpathlineto{\pgfqpoint{4.704465in}{3.529451in}}%
\pgfpathlineto{\pgfqpoint{4.717913in}{3.520028in}}%
\pgfpathlineto{\pgfqpoint{4.731366in}{3.510682in}}%
\pgfpathlineto{\pgfqpoint{4.744822in}{3.501415in}}%
\pgfpathlineto{\pgfqpoint{4.752566in}{3.529595in}}%
\pgfpathlineto{\pgfqpoint{4.760317in}{3.558337in}}%
\pgfpathlineto{\pgfqpoint{4.768076in}{3.587651in}}%
\pgfpathlineto{\pgfqpoint{4.775843in}{3.617548in}}%
\pgfpathlineto{\pgfqpoint{4.762388in}{3.627382in}}%
\pgfpathlineto{\pgfqpoint{4.748937in}{3.637295in}}%
\pgfpathlineto{\pgfqpoint{4.735489in}{3.647285in}}%
\pgfpathlineto{\pgfqpoint{4.722046in}{3.657354in}}%
\pgfpathlineto{\pgfqpoint{4.714278in}{3.626882in}}%
\pgfpathlineto{\pgfqpoint{4.706518in}{3.596998in}}%
\pgfpathlineto{\pgfqpoint{4.698765in}{3.567692in}}%
\pgfpathlineto{\pgfqpoint{4.691020in}{3.538953in}}%
\pgfpathclose%
\pgfusepath{fill}%
\end{pgfscope}%
\begin{pgfscope}%
\pgfpathrectangle{\pgfqpoint{1.150000in}{0.150000in}}{\pgfqpoint{5.700000in}{5.700000in}}%
\pgfusepath{clip}%
\pgfsetbuttcap%
\pgfsetroundjoin%
\definecolor{currentfill}{rgb}{0.267968,0.223549,0.512008}%
\pgfsetfillcolor{currentfill}%
\pgfsetfillopacity{0.700000}%
\pgfsetlinewidth{0.000000pt}%
\definecolor{currentstroke}{rgb}{0.000000,0.000000,0.000000}%
\pgfsetstrokecolor{currentstroke}%
\pgfsetdash{}{0pt}%
\pgfpathmoveto{\pgfqpoint{3.700752in}{2.855243in}}%
\pgfpathlineto{\pgfqpoint{3.714082in}{2.847388in}}%
\pgfpathlineto{\pgfqpoint{3.727416in}{2.839629in}}%
\pgfpathlineto{\pgfqpoint{3.740753in}{2.831966in}}%
\pgfpathlineto{\pgfqpoint{3.754093in}{2.824397in}}%
\pgfpathlineto{\pgfqpoint{3.761898in}{2.838752in}}%
\pgfpathlineto{\pgfqpoint{3.769697in}{2.853310in}}%
\pgfpathlineto{\pgfqpoint{3.777492in}{2.868078in}}%
\pgfpathlineto{\pgfqpoint{3.785282in}{2.883059in}}%
\pgfpathlineto{\pgfqpoint{3.771948in}{2.890924in}}%
\pgfpathlineto{\pgfqpoint{3.758617in}{2.898883in}}%
\pgfpathlineto{\pgfqpoint{3.745288in}{2.906938in}}%
\pgfpathlineto{\pgfqpoint{3.731963in}{2.915088in}}%
\pgfpathlineto{\pgfqpoint{3.724168in}{2.899804in}}%
\pgfpathlineto{\pgfqpoint{3.716368in}{2.884738in}}%
\pgfpathlineto{\pgfqpoint{3.708562in}{2.869887in}}%
\pgfpathlineto{\pgfqpoint{3.700752in}{2.855243in}}%
\pgfpathclose%
\pgfusepath{fill}%
\end{pgfscope}%
\begin{pgfscope}%
\pgfpathrectangle{\pgfqpoint{1.150000in}{0.150000in}}{\pgfqpoint{5.700000in}{5.700000in}}%
\pgfusepath{clip}%
\pgfsetbuttcap%
\pgfsetroundjoin%
\definecolor{currentfill}{rgb}{0.263663,0.237631,0.518762}%
\pgfsetfillcolor{currentfill}%
\pgfsetfillopacity{0.700000}%
\pgfsetlinewidth{0.000000pt}%
\definecolor{currentstroke}{rgb}{0.000000,0.000000,0.000000}%
\pgfsetstrokecolor{currentstroke}%
\pgfsetdash{}{0pt}%
\pgfpathmoveto{\pgfqpoint{3.923152in}{2.883151in}}%
\pgfpathlineto{\pgfqpoint{3.936513in}{2.875799in}}%
\pgfpathlineto{\pgfqpoint{3.949878in}{2.868538in}}%
\pgfpathlineto{\pgfqpoint{3.963246in}{2.861365in}}%
\pgfpathlineto{\pgfqpoint{3.976619in}{2.854282in}}%
\pgfpathlineto{\pgfqpoint{3.984370in}{2.869437in}}%
\pgfpathlineto{\pgfqpoint{3.992117in}{2.884827in}}%
\pgfpathlineto{\pgfqpoint{3.999861in}{2.900458in}}%
\pgfpathlineto{\pgfqpoint{4.007601in}{2.916335in}}%
\pgfpathlineto{\pgfqpoint{3.994235in}{2.923754in}}%
\pgfpathlineto{\pgfqpoint{3.980872in}{2.931262in}}%
\pgfpathlineto{\pgfqpoint{3.967513in}{2.938859in}}%
\pgfpathlineto{\pgfqpoint{3.954158in}{2.946546in}}%
\pgfpathlineto{\pgfqpoint{3.946412in}{2.930326in}}%
\pgfpathlineto{\pgfqpoint{3.938663in}{2.914357in}}%
\pgfpathlineto{\pgfqpoint{3.930910in}{2.898634in}}%
\pgfpathlineto{\pgfqpoint{3.923152in}{2.883151in}}%
\pgfpathclose%
\pgfusepath{fill}%
\end{pgfscope}%
\begin{pgfscope}%
\pgfpathrectangle{\pgfqpoint{1.150000in}{0.150000in}}{\pgfqpoint{5.700000in}{5.700000in}}%
\pgfusepath{clip}%
\pgfsetbuttcap%
\pgfsetroundjoin%
\definecolor{currentfill}{rgb}{0.195860,0.395433,0.555276}%
\pgfsetfillcolor{currentfill}%
\pgfsetfillopacity{0.700000}%
\pgfsetlinewidth{0.000000pt}%
\definecolor{currentstroke}{rgb}{0.000000,0.000000,0.000000}%
\pgfsetstrokecolor{currentstroke}%
\pgfsetdash{}{0pt}%
\pgfpathmoveto{\pgfqpoint{4.598514in}{3.234435in}}%
\pgfpathlineto{\pgfqpoint{4.611968in}{3.226499in}}%
\pgfpathlineto{\pgfqpoint{4.625428in}{3.218640in}}%
\pgfpathlineto{\pgfqpoint{4.638891in}{3.210858in}}%
\pgfpathlineto{\pgfqpoint{4.652360in}{3.203154in}}%
\pgfpathlineto{\pgfqpoint{4.660040in}{3.225417in}}%
\pgfpathlineto{\pgfqpoint{4.667723in}{3.248119in}}%
\pgfpathlineto{\pgfqpoint{4.675411in}{3.271269in}}%
\pgfpathlineto{\pgfqpoint{4.683102in}{3.294877in}}%
\pgfpathlineto{\pgfqpoint{4.669639in}{3.303081in}}%
\pgfpathlineto{\pgfqpoint{4.656180in}{3.311361in}}%
\pgfpathlineto{\pgfqpoint{4.642725in}{3.319720in}}%
\pgfpathlineto{\pgfqpoint{4.629275in}{3.328156in}}%
\pgfpathlineto{\pgfqpoint{4.621579in}{3.304040in}}%
\pgfpathlineto{\pgfqpoint{4.613887in}{3.280388in}}%
\pgfpathlineto{\pgfqpoint{4.606199in}{3.257190in}}%
\pgfpathlineto{\pgfqpoint{4.598514in}{3.234435in}}%
\pgfpathclose%
\pgfusepath{fill}%
\end{pgfscope}%
\begin{pgfscope}%
\pgfpathrectangle{\pgfqpoint{1.150000in}{0.150000in}}{\pgfqpoint{5.700000in}{5.700000in}}%
\pgfusepath{clip}%
\pgfsetbuttcap%
\pgfsetroundjoin%
\definecolor{currentfill}{rgb}{0.253935,0.265254,0.529983}%
\pgfsetfillcolor{currentfill}%
\pgfsetfillopacity{0.700000}%
\pgfsetlinewidth{0.000000pt}%
\definecolor{currentstroke}{rgb}{0.000000,0.000000,0.000000}%
\pgfsetstrokecolor{currentstroke}%
\pgfsetdash{}{0pt}%
\pgfpathmoveto{\pgfqpoint{3.010650in}{2.952181in}}%
\pgfpathlineto{\pgfqpoint{3.023950in}{2.941070in}}%
\pgfpathlineto{\pgfqpoint{3.037250in}{2.930087in}}%
\pgfpathlineto{\pgfqpoint{3.050549in}{2.919232in}}%
\pgfpathlineto{\pgfqpoint{3.063848in}{2.908503in}}%
\pgfpathlineto{\pgfqpoint{3.071836in}{2.921679in}}%
\pgfpathlineto{\pgfqpoint{3.079817in}{2.935018in}}%
\pgfpathlineto{\pgfqpoint{3.087790in}{2.948524in}}%
\pgfpathlineto{\pgfqpoint{3.095757in}{2.962200in}}%
\pgfpathlineto{\pgfqpoint{3.082464in}{2.973125in}}%
\pgfpathlineto{\pgfqpoint{3.069171in}{2.984177in}}%
\pgfpathlineto{\pgfqpoint{3.055878in}{2.995356in}}%
\pgfpathlineto{\pgfqpoint{3.042585in}{3.006664in}}%
\pgfpathlineto{\pgfqpoint{3.034612in}{2.992783in}}%
\pgfpathlineto{\pgfqpoint{3.026633in}{2.979078in}}%
\pgfpathlineto{\pgfqpoint{3.018645in}{2.965546in}}%
\pgfpathlineto{\pgfqpoint{3.010650in}{2.952181in}}%
\pgfpathclose%
\pgfusepath{fill}%
\end{pgfscope}%
\begin{pgfscope}%
\pgfpathrectangle{\pgfqpoint{1.150000in}{0.150000in}}{\pgfqpoint{5.700000in}{5.700000in}}%
\pgfusepath{clip}%
\pgfsetbuttcap%
\pgfsetroundjoin%
\definecolor{currentfill}{rgb}{0.241237,0.296485,0.539709}%
\pgfsetfillcolor{currentfill}%
\pgfsetfillopacity{0.700000}%
\pgfsetlinewidth{0.000000pt}%
\definecolor{currentstroke}{rgb}{0.000000,0.000000,0.000000}%
\pgfsetstrokecolor{currentstroke}%
\pgfsetdash{}{0pt}%
\pgfpathmoveto{\pgfqpoint{4.314419in}{3.003472in}}%
\pgfpathlineto{\pgfqpoint{4.327841in}{2.996368in}}%
\pgfpathlineto{\pgfqpoint{4.341267in}{2.989345in}}%
\pgfpathlineto{\pgfqpoint{4.354698in}{2.982403in}}%
\pgfpathlineto{\pgfqpoint{4.368134in}{2.975542in}}%
\pgfpathlineto{\pgfqpoint{4.375814in}{2.993196in}}%
\pgfpathlineto{\pgfqpoint{4.383493in}{3.011169in}}%
\pgfpathlineto{\pgfqpoint{4.391172in}{3.029468in}}%
\pgfpathlineto{\pgfqpoint{4.398850in}{3.048102in}}%
\pgfpathlineto{\pgfqpoint{4.385421in}{3.055379in}}%
\pgfpathlineto{\pgfqpoint{4.371996in}{3.062737in}}%
\pgfpathlineto{\pgfqpoint{4.358575in}{3.070176in}}%
\pgfpathlineto{\pgfqpoint{4.345159in}{3.077697in}}%
\pgfpathlineto{\pgfqpoint{4.337475in}{3.058639in}}%
\pgfpathlineto{\pgfqpoint{4.329791in}{3.039921in}}%
\pgfpathlineto{\pgfqpoint{4.322105in}{3.021534in}}%
\pgfpathlineto{\pgfqpoint{4.314419in}{3.003472in}}%
\pgfpathclose%
\pgfusepath{fill}%
\end{pgfscope}%
\begin{pgfscope}%
\pgfpathrectangle{\pgfqpoint{1.150000in}{0.150000in}}{\pgfqpoint{5.700000in}{5.700000in}}%
\pgfusepath{clip}%
\pgfsetbuttcap%
\pgfsetroundjoin%
\definecolor{currentfill}{rgb}{0.267968,0.223549,0.512008}%
\pgfsetfillcolor{currentfill}%
\pgfsetfillopacity{0.700000}%
\pgfsetlinewidth{0.000000pt}%
\definecolor{currentstroke}{rgb}{0.000000,0.000000,0.000000}%
\pgfsetstrokecolor{currentstroke}%
\pgfsetdash{}{0pt}%
\pgfpathmoveto{\pgfqpoint{3.340200in}{2.856801in}}%
\pgfpathlineto{\pgfqpoint{3.353500in}{2.847658in}}%
\pgfpathlineto{\pgfqpoint{3.366802in}{2.838625in}}%
\pgfpathlineto{\pgfqpoint{3.380105in}{2.829701in}}%
\pgfpathlineto{\pgfqpoint{3.393410in}{2.820885in}}%
\pgfpathlineto{\pgfqpoint{3.401312in}{2.834331in}}%
\pgfpathlineto{\pgfqpoint{3.409207in}{2.847948in}}%
\pgfpathlineto{\pgfqpoint{3.417097in}{2.861740in}}%
\pgfpathlineto{\pgfqpoint{3.424980in}{2.875710in}}%
\pgfpathlineto{\pgfqpoint{3.411681in}{2.884762in}}%
\pgfpathlineto{\pgfqpoint{3.398383in}{2.893922in}}%
\pgfpathlineto{\pgfqpoint{3.385088in}{2.903190in}}%
\pgfpathlineto{\pgfqpoint{3.371794in}{2.912569in}}%
\pgfpathlineto{\pgfqpoint{3.363905in}{2.898355in}}%
\pgfpathlineto{\pgfqpoint{3.356010in}{2.884326in}}%
\pgfpathlineto{\pgfqpoint{3.348108in}{2.870476in}}%
\pgfpathlineto{\pgfqpoint{3.340200in}{2.856801in}}%
\pgfpathclose%
\pgfusepath{fill}%
\end{pgfscope}%
\begin{pgfscope}%
\pgfpathrectangle{\pgfqpoint{1.150000in}{0.150000in}}{\pgfqpoint{5.700000in}{5.700000in}}%
\pgfusepath{clip}%
\pgfsetbuttcap%
\pgfsetroundjoin%
\definecolor{currentfill}{rgb}{0.132444,0.552216,0.553018}%
\pgfsetfillcolor{currentfill}%
\pgfsetfillopacity{0.700000}%
\pgfsetlinewidth{0.000000pt}%
\definecolor{currentstroke}{rgb}{0.000000,0.000000,0.000000}%
\pgfsetstrokecolor{currentstroke}%
\pgfsetdash{}{0pt}%
\pgfpathmoveto{\pgfqpoint{4.722046in}{3.657354in}}%
\pgfpathlineto{\pgfqpoint{4.735489in}{3.647285in}}%
\pgfpathlineto{\pgfqpoint{4.748937in}{3.637295in}}%
\pgfpathlineto{\pgfqpoint{4.762388in}{3.627382in}}%
\pgfpathlineto{\pgfqpoint{4.775843in}{3.617548in}}%
\pgfpathlineto{\pgfqpoint{4.783619in}{3.648039in}}%
\pgfpathlineto{\pgfqpoint{4.791404in}{3.679137in}}%
\pgfpathlineto{\pgfqpoint{4.799198in}{3.710853in}}%
\pgfpathlineto{\pgfqpoint{4.785743in}{3.721128in}}%
\pgfpathlineto{\pgfqpoint{4.772292in}{3.731481in}}%
\pgfpathlineto{\pgfqpoint{4.758844in}{3.741913in}}%
\pgfpathlineto{\pgfqpoint{4.745400in}{3.752423in}}%
\pgfpathlineto{\pgfqpoint{4.737606in}{3.720114in}}%
\pgfpathlineto{\pgfqpoint{4.729822in}{3.688428in}}%
\pgfpathlineto{\pgfqpoint{4.722046in}{3.657354in}}%
\pgfpathclose%
\pgfusepath{fill}%
\end{pgfscope}%
\begin{pgfscope}%
\pgfpathrectangle{\pgfqpoint{1.150000in}{0.150000in}}{\pgfqpoint{5.700000in}{5.700000in}}%
\pgfusepath{clip}%
\pgfsetbuttcap%
\pgfsetroundjoin%
\definecolor{currentfill}{rgb}{0.231674,0.318106,0.544834}%
\pgfsetfillcolor{currentfill}%
\pgfsetfillopacity{0.700000}%
\pgfsetlinewidth{0.000000pt}%
\definecolor{currentstroke}{rgb}{0.000000,0.000000,0.000000}%
\pgfsetstrokecolor{currentstroke}%
\pgfsetdash{}{0pt}%
\pgfpathmoveto{\pgfqpoint{4.398850in}{3.048102in}}%
\pgfpathlineto{\pgfqpoint{4.412284in}{3.040905in}}%
\pgfpathlineto{\pgfqpoint{4.425723in}{3.033789in}}%
\pgfpathlineto{\pgfqpoint{4.439167in}{3.026751in}}%
\pgfpathlineto{\pgfqpoint{4.452615in}{3.019794in}}%
\pgfpathlineto{\pgfqpoint{4.460287in}{3.038341in}}%
\pgfpathlineto{\pgfqpoint{4.467959in}{3.057233in}}%
\pgfpathlineto{\pgfqpoint{4.475632in}{3.076478in}}%
\pgfpathlineto{\pgfqpoint{4.483305in}{3.096085in}}%
\pgfpathlineto{\pgfqpoint{4.469863in}{3.103479in}}%
\pgfpathlineto{\pgfqpoint{4.456425in}{3.110953in}}%
\pgfpathlineto{\pgfqpoint{4.442992in}{3.118506in}}%
\pgfpathlineto{\pgfqpoint{4.429564in}{3.126139in}}%
\pgfpathlineto{\pgfqpoint{4.421885in}{3.106089in}}%
\pgfpathlineto{\pgfqpoint{4.414206in}{3.086404in}}%
\pgfpathlineto{\pgfqpoint{4.406528in}{3.067078in}}%
\pgfpathlineto{\pgfqpoint{4.398850in}{3.048102in}}%
\pgfpathclose%
\pgfusepath{fill}%
\end{pgfscope}%
\begin{pgfscope}%
\pgfpathrectangle{\pgfqpoint{1.150000in}{0.150000in}}{\pgfqpoint{5.700000in}{5.700000in}}%
\pgfusepath{clip}%
\pgfsetbuttcap%
\pgfsetroundjoin%
\definecolor{currentfill}{rgb}{0.248629,0.278775,0.534556}%
\pgfsetfillcolor{currentfill}%
\pgfsetfillopacity{0.700000}%
\pgfsetlinewidth{0.000000pt}%
\definecolor{currentstroke}{rgb}{0.000000,0.000000,0.000000}%
\pgfsetstrokecolor{currentstroke}%
\pgfsetdash{}{0pt}%
\pgfpathmoveto{\pgfqpoint{4.229995in}{2.961965in}}%
\pgfpathlineto{\pgfqpoint{4.243404in}{2.954927in}}%
\pgfpathlineto{\pgfqpoint{4.256818in}{2.947973in}}%
\pgfpathlineto{\pgfqpoint{4.270237in}{2.941100in}}%
\pgfpathlineto{\pgfqpoint{4.283661in}{2.934310in}}%
\pgfpathlineto{\pgfqpoint{4.291352in}{2.951152in}}%
\pgfpathlineto{\pgfqpoint{4.299043in}{2.968288in}}%
\pgfpathlineto{\pgfqpoint{4.306731in}{2.985725in}}%
\pgfpathlineto{\pgfqpoint{4.314419in}{3.003472in}}%
\pgfpathlineto{\pgfqpoint{4.301002in}{3.010658in}}%
\pgfpathlineto{\pgfqpoint{4.287589in}{3.017926in}}%
\pgfpathlineto{\pgfqpoint{4.274181in}{3.025276in}}%
\pgfpathlineto{\pgfqpoint{4.260777in}{3.032710in}}%
\pgfpathlineto{\pgfqpoint{4.253084in}{3.014560in}}%
\pgfpathlineto{\pgfqpoint{4.245389in}{2.996724in}}%
\pgfpathlineto{\pgfqpoint{4.237693in}{2.979195in}}%
\pgfpathlineto{\pgfqpoint{4.229995in}{2.961965in}}%
\pgfpathclose%
\pgfusepath{fill}%
\end{pgfscope}%
\begin{pgfscope}%
\pgfpathrectangle{\pgfqpoint{1.150000in}{0.150000in}}{\pgfqpoint{5.700000in}{5.700000in}}%
\pgfusepath{clip}%
\pgfsetbuttcap%
\pgfsetroundjoin%
\definecolor{currentfill}{rgb}{0.270595,0.214069,0.507052}%
\pgfsetfillcolor{currentfill}%
\pgfsetfillopacity{0.700000}%
\pgfsetlinewidth{0.000000pt}%
\definecolor{currentstroke}{rgb}{0.000000,0.000000,0.000000}%
\pgfsetstrokecolor{currentstroke}%
\pgfsetdash{}{0pt}%
\pgfpathmoveto{\pgfqpoint{3.478196in}{2.840567in}}%
\pgfpathlineto{\pgfqpoint{3.491505in}{2.832044in}}%
\pgfpathlineto{\pgfqpoint{3.504817in}{2.823624in}}%
\pgfpathlineto{\pgfqpoint{3.518131in}{2.815308in}}%
\pgfpathlineto{\pgfqpoint{3.531448in}{2.807093in}}%
\pgfpathlineto{\pgfqpoint{3.539313in}{2.820750in}}%
\pgfpathlineto{\pgfqpoint{3.547173in}{2.834586in}}%
\pgfpathlineto{\pgfqpoint{3.555027in}{2.848604in}}%
\pgfpathlineto{\pgfqpoint{3.562875in}{2.862810in}}%
\pgfpathlineto{\pgfqpoint{3.549565in}{2.871280in}}%
\pgfpathlineto{\pgfqpoint{3.536256in}{2.879852in}}%
\pgfpathlineto{\pgfqpoint{3.522950in}{2.888527in}}%
\pgfpathlineto{\pgfqpoint{3.509646in}{2.897306in}}%
\pgfpathlineto{\pgfqpoint{3.501793in}{2.882837in}}%
\pgfpathlineto{\pgfqpoint{3.493933in}{2.868561in}}%
\pgfpathlineto{\pgfqpoint{3.486067in}{2.854472in}}%
\pgfpathlineto{\pgfqpoint{3.478196in}{2.840567in}}%
\pgfpathclose%
\pgfusepath{fill}%
\end{pgfscope}%
\begin{pgfscope}%
\pgfpathrectangle{\pgfqpoint{1.150000in}{0.150000in}}{\pgfqpoint{5.700000in}{5.700000in}}%
\pgfusepath{clip}%
\pgfsetbuttcap%
\pgfsetroundjoin%
\definecolor{currentfill}{rgb}{0.265145,0.232956,0.516599}%
\pgfsetfillcolor{currentfill}%
\pgfsetfillopacity{0.700000}%
\pgfsetlinewidth{0.000000pt}%
\definecolor{currentstroke}{rgb}{0.000000,0.000000,0.000000}%
\pgfsetstrokecolor{currentstroke}%
\pgfsetdash{}{0pt}%
\pgfpathmoveto{\pgfqpoint{3.202105in}{2.879225in}}%
\pgfpathlineto{\pgfqpoint{3.215401in}{2.869391in}}%
\pgfpathlineto{\pgfqpoint{3.228698in}{2.859674in}}%
\pgfpathlineto{\pgfqpoint{3.241996in}{2.850072in}}%
\pgfpathlineto{\pgfqpoint{3.255295in}{2.840585in}}%
\pgfpathlineto{\pgfqpoint{3.263235in}{2.853811in}}%
\pgfpathlineto{\pgfqpoint{3.271169in}{2.867199in}}%
\pgfpathlineto{\pgfqpoint{3.279096in}{2.880756in}}%
\pgfpathlineto{\pgfqpoint{3.287016in}{2.894483in}}%
\pgfpathlineto{\pgfqpoint{3.273723in}{2.904186in}}%
\pgfpathlineto{\pgfqpoint{3.260431in}{2.914004in}}%
\pgfpathlineto{\pgfqpoint{3.247140in}{2.923937in}}%
\pgfpathlineto{\pgfqpoint{3.233850in}{2.933987in}}%
\pgfpathlineto{\pgfqpoint{3.225924in}{2.920036in}}%
\pgfpathlineto{\pgfqpoint{3.217991in}{2.906261in}}%
\pgfpathlineto{\pgfqpoint{3.210051in}{2.892659in}}%
\pgfpathlineto{\pgfqpoint{3.202105in}{2.879225in}}%
\pgfpathclose%
\pgfusepath{fill}%
\end{pgfscope}%
\begin{pgfscope}%
\pgfpathrectangle{\pgfqpoint{1.150000in}{0.150000in}}{\pgfqpoint{5.700000in}{5.700000in}}%
\pgfusepath{clip}%
\pgfsetbuttcap%
\pgfsetroundjoin%
\definecolor{currentfill}{rgb}{0.221989,0.339161,0.548752}%
\pgfsetfillcolor{currentfill}%
\pgfsetfillopacity{0.700000}%
\pgfsetlinewidth{0.000000pt}%
\definecolor{currentstroke}{rgb}{0.000000,0.000000,0.000000}%
\pgfsetstrokecolor{currentstroke}%
\pgfsetdash{}{0pt}%
\pgfpathmoveto{\pgfqpoint{4.483305in}{3.096085in}}%
\pgfpathlineto{\pgfqpoint{4.496752in}{3.088770in}}%
\pgfpathlineto{\pgfqpoint{4.510204in}{3.081534in}}%
\pgfpathlineto{\pgfqpoint{4.523660in}{3.074376in}}%
\pgfpathlineto{\pgfqpoint{4.537122in}{3.067296in}}%
\pgfpathlineto{\pgfqpoint{4.544789in}{3.086823in}}%
\pgfpathlineto{\pgfqpoint{4.552459in}{3.106723in}}%
\pgfpathlineto{\pgfqpoint{4.560129in}{3.127004in}}%
\pgfpathlineto{\pgfqpoint{4.567802in}{3.147675in}}%
\pgfpathlineto{\pgfqpoint{4.554346in}{3.155212in}}%
\pgfpathlineto{\pgfqpoint{4.540896in}{3.162827in}}%
\pgfpathlineto{\pgfqpoint{4.527450in}{3.170520in}}%
\pgfpathlineto{\pgfqpoint{4.514008in}{3.178292in}}%
\pgfpathlineto{\pgfqpoint{4.506330in}{3.157156in}}%
\pgfpathlineto{\pgfqpoint{4.498654in}{3.136415in}}%
\pgfpathlineto{\pgfqpoint{4.490979in}{3.116061in}}%
\pgfpathlineto{\pgfqpoint{4.483305in}{3.096085in}}%
\pgfpathclose%
\pgfusepath{fill}%
\end{pgfscope}%
\begin{pgfscope}%
\pgfpathrectangle{\pgfqpoint{1.150000in}{0.150000in}}{\pgfqpoint{5.700000in}{5.700000in}}%
\pgfusepath{clip}%
\pgfsetbuttcap%
\pgfsetroundjoin%
\definecolor{currentfill}{rgb}{0.267968,0.223549,0.512008}%
\pgfsetfillcolor{currentfill}%
\pgfsetfillopacity{0.700000}%
\pgfsetlinewidth{0.000000pt}%
\definecolor{currentstroke}{rgb}{0.000000,0.000000,0.000000}%
\pgfsetstrokecolor{currentstroke}%
\pgfsetdash{}{0pt}%
\pgfpathmoveto{\pgfqpoint{3.838654in}{2.852538in}}%
\pgfpathlineto{\pgfqpoint{3.852005in}{2.845139in}}%
\pgfpathlineto{\pgfqpoint{3.865361in}{2.837832in}}%
\pgfpathlineto{\pgfqpoint{3.878720in}{2.830616in}}%
\pgfpathlineto{\pgfqpoint{3.892083in}{2.823490in}}%
\pgfpathlineto{\pgfqpoint{3.899857in}{2.838076in}}%
\pgfpathlineto{\pgfqpoint{3.907626in}{2.852877in}}%
\pgfpathlineto{\pgfqpoint{3.915391in}{2.867900in}}%
\pgfpathlineto{\pgfqpoint{3.923152in}{2.883151in}}%
\pgfpathlineto{\pgfqpoint{3.909796in}{2.890592in}}%
\pgfpathlineto{\pgfqpoint{3.896442in}{2.898123in}}%
\pgfpathlineto{\pgfqpoint{3.883093in}{2.905746in}}%
\pgfpathlineto{\pgfqpoint{3.869747in}{2.913461in}}%
\pgfpathlineto{\pgfqpoint{3.861980in}{2.897887in}}%
\pgfpathlineto{\pgfqpoint{3.854209in}{2.882546in}}%
\pgfpathlineto{\pgfqpoint{3.846433in}{2.867432in}}%
\pgfpathlineto{\pgfqpoint{3.838654in}{2.852538in}}%
\pgfpathclose%
\pgfusepath{fill}%
\end{pgfscope}%
\begin{pgfscope}%
\pgfpathrectangle{\pgfqpoint{1.150000in}{0.150000in}}{\pgfqpoint{5.700000in}{5.700000in}}%
\pgfusepath{clip}%
\pgfsetbuttcap%
\pgfsetroundjoin%
\definecolor{currentfill}{rgb}{0.255645,0.260703,0.528312}%
\pgfsetfillcolor{currentfill}%
\pgfsetfillopacity{0.700000}%
\pgfsetlinewidth{0.000000pt}%
\definecolor{currentstroke}{rgb}{0.000000,0.000000,0.000000}%
\pgfsetstrokecolor{currentstroke}%
\pgfsetdash{}{0pt}%
\pgfpathmoveto{\pgfqpoint{4.145562in}{2.923379in}}%
\pgfpathlineto{\pgfqpoint{4.158960in}{2.916382in}}%
\pgfpathlineto{\pgfqpoint{4.172363in}{2.909469in}}%
\pgfpathlineto{\pgfqpoint{4.185770in}{2.902640in}}%
\pgfpathlineto{\pgfqpoint{4.199181in}{2.895895in}}%
\pgfpathlineto{\pgfqpoint{4.206888in}{2.911999in}}%
\pgfpathlineto{\pgfqpoint{4.214592in}{2.928374in}}%
\pgfpathlineto{\pgfqpoint{4.222295in}{2.945027in}}%
\pgfpathlineto{\pgfqpoint{4.229995in}{2.961965in}}%
\pgfpathlineto{\pgfqpoint{4.216589in}{2.969086in}}%
\pgfpathlineto{\pgfqpoint{4.203189in}{2.976290in}}%
\pgfpathlineto{\pgfqpoint{4.189792in}{2.983578in}}%
\pgfpathlineto{\pgfqpoint{4.176400in}{2.990951in}}%
\pgfpathlineto{\pgfqpoint{4.168694in}{2.973630in}}%
\pgfpathlineto{\pgfqpoint{4.160986in}{2.956599in}}%
\pgfpathlineto{\pgfqpoint{4.153275in}{2.939851in}}%
\pgfpathlineto{\pgfqpoint{4.145562in}{2.923379in}}%
\pgfpathclose%
\pgfusepath{fill}%
\end{pgfscope}%
\begin{pgfscope}%
\pgfpathrectangle{\pgfqpoint{1.150000in}{0.150000in}}{\pgfqpoint{5.700000in}{5.700000in}}%
\pgfusepath{clip}%
\pgfsetbuttcap%
\pgfsetroundjoin%
\definecolor{currentfill}{rgb}{0.168126,0.459988,0.558082}%
\pgfsetfillcolor{currentfill}%
\pgfsetfillopacity{0.700000}%
\pgfsetlinewidth{0.000000pt}%
\definecolor{currentstroke}{rgb}{0.000000,0.000000,0.000000}%
\pgfsetstrokecolor{currentstroke}%
\pgfsetdash{}{0pt}%
\pgfpathmoveto{\pgfqpoint{4.713916in}{3.394084in}}%
\pgfpathlineto{\pgfqpoint{4.727380in}{3.385436in}}%
\pgfpathlineto{\pgfqpoint{4.740848in}{3.376866in}}%
\pgfpathlineto{\pgfqpoint{4.754321in}{3.368372in}}%
\pgfpathlineto{\pgfqpoint{4.767798in}{3.359954in}}%
\pgfpathlineto{\pgfqpoint{4.775511in}{3.385465in}}%
\pgfpathlineto{\pgfqpoint{4.783231in}{3.411488in}}%
\pgfpathlineto{\pgfqpoint{4.790957in}{3.438034in}}%
\pgfpathlineto{\pgfqpoint{4.798690in}{3.465114in}}%
\pgfpathlineto{\pgfqpoint{4.785217in}{3.474074in}}%
\pgfpathlineto{\pgfqpoint{4.771748in}{3.483111in}}%
\pgfpathlineto{\pgfqpoint{4.758283in}{3.492224in}}%
\pgfpathlineto{\pgfqpoint{4.744822in}{3.501415in}}%
\pgfpathlineto{\pgfqpoint{4.737086in}{3.473784in}}%
\pgfpathlineto{\pgfqpoint{4.729356in}{3.446692in}}%
\pgfpathlineto{\pgfqpoint{4.721633in}{3.420129in}}%
\pgfpathlineto{\pgfqpoint{4.713916in}{3.394084in}}%
\pgfpathclose%
\pgfusepath{fill}%
\end{pgfscope}%
\begin{pgfscope}%
\pgfpathrectangle{\pgfqpoint{1.150000in}{0.150000in}}{\pgfqpoint{5.700000in}{5.700000in}}%
\pgfusepath{clip}%
\pgfsetbuttcap%
\pgfsetroundjoin%
\definecolor{currentfill}{rgb}{0.270595,0.214069,0.507052}%
\pgfsetfillcolor{currentfill}%
\pgfsetfillopacity{0.700000}%
\pgfsetlinewidth{0.000000pt}%
\definecolor{currentstroke}{rgb}{0.000000,0.000000,0.000000}%
\pgfsetstrokecolor{currentstroke}%
\pgfsetdash{}{0pt}%
\pgfpathmoveto{\pgfqpoint{3.616145in}{2.829938in}}%
\pgfpathlineto{\pgfqpoint{3.629469in}{2.821969in}}%
\pgfpathlineto{\pgfqpoint{3.642796in}{2.814098in}}%
\pgfpathlineto{\pgfqpoint{3.656126in}{2.806325in}}%
\pgfpathlineto{\pgfqpoint{3.669459in}{2.798649in}}%
\pgfpathlineto{\pgfqpoint{3.677290in}{2.812511in}}%
\pgfpathlineto{\pgfqpoint{3.685116in}{2.826561in}}%
\pgfpathlineto{\pgfqpoint{3.692937in}{2.840803in}}%
\pgfpathlineto{\pgfqpoint{3.700752in}{2.855243in}}%
\pgfpathlineto{\pgfqpoint{3.687425in}{2.863195in}}%
\pgfpathlineto{\pgfqpoint{3.674100in}{2.871243in}}%
\pgfpathlineto{\pgfqpoint{3.660779in}{2.879390in}}%
\pgfpathlineto{\pgfqpoint{3.647460in}{2.887634in}}%
\pgfpathlineto{\pgfqpoint{3.639639in}{2.872911in}}%
\pgfpathlineto{\pgfqpoint{3.631813in}{2.858391in}}%
\pgfpathlineto{\pgfqpoint{3.623982in}{2.844068in}}%
\pgfpathlineto{\pgfqpoint{3.616145in}{2.829938in}}%
\pgfpathclose%
\pgfusepath{fill}%
\end{pgfscope}%
\begin{pgfscope}%
\pgfpathrectangle{\pgfqpoint{1.150000in}{0.150000in}}{\pgfqpoint{5.700000in}{5.700000in}}%
\pgfusepath{clip}%
\pgfsetbuttcap%
\pgfsetroundjoin%
\definecolor{currentfill}{rgb}{0.151918,0.500685,0.557587}%
\pgfsetfillcolor{currentfill}%
\pgfsetfillopacity{0.700000}%
\pgfsetlinewidth{0.000000pt}%
\definecolor{currentstroke}{rgb}{0.000000,0.000000,0.000000}%
\pgfsetstrokecolor{currentstroke}%
\pgfsetdash{}{0pt}%
\pgfpathmoveto{\pgfqpoint{4.744822in}{3.501415in}}%
\pgfpathlineto{\pgfqpoint{4.758283in}{3.492224in}}%
\pgfpathlineto{\pgfqpoint{4.771748in}{3.483111in}}%
\pgfpathlineto{\pgfqpoint{4.785217in}{3.474074in}}%
\pgfpathlineto{\pgfqpoint{4.798690in}{3.465114in}}%
\pgfpathlineto{\pgfqpoint{4.806431in}{3.492737in}}%
\pgfpathlineto{\pgfqpoint{4.814180in}{3.520916in}}%
\pgfpathlineto{\pgfqpoint{4.821937in}{3.549661in}}%
\pgfpathlineto{\pgfqpoint{4.829702in}{3.578983in}}%
\pgfpathlineto{\pgfqpoint{4.816232in}{3.588509in}}%
\pgfpathlineto{\pgfqpoint{4.802765in}{3.598111in}}%
\pgfpathlineto{\pgfqpoint{4.789302in}{3.607791in}}%
\pgfpathlineto{\pgfqpoint{4.775843in}{3.617548in}}%
\pgfpathlineto{\pgfqpoint{4.768076in}{3.587651in}}%
\pgfpathlineto{\pgfqpoint{4.760317in}{3.558337in}}%
\pgfpathlineto{\pgfqpoint{4.752566in}{3.529595in}}%
\pgfpathlineto{\pgfqpoint{4.744822in}{3.501415in}}%
\pgfpathclose%
\pgfusepath{fill}%
\end{pgfscope}%
\begin{pgfscope}%
\pgfpathrectangle{\pgfqpoint{1.150000in}{0.150000in}}{\pgfqpoint{5.700000in}{5.700000in}}%
\pgfusepath{clip}%
\pgfsetbuttcap%
\pgfsetroundjoin%
\definecolor{currentfill}{rgb}{0.183898,0.422383,0.556944}%
\pgfsetfillcolor{currentfill}%
\pgfsetfillopacity{0.700000}%
\pgfsetlinewidth{0.000000pt}%
\definecolor{currentstroke}{rgb}{0.000000,0.000000,0.000000}%
\pgfsetstrokecolor{currentstroke}%
\pgfsetdash{}{0pt}%
\pgfpathmoveto{\pgfqpoint{4.683102in}{3.294877in}}%
\pgfpathlineto{\pgfqpoint{4.696570in}{3.286751in}}%
\pgfpathlineto{\pgfqpoint{4.710043in}{3.278701in}}%
\pgfpathlineto{\pgfqpoint{4.723520in}{3.270728in}}%
\pgfpathlineto{\pgfqpoint{4.737002in}{3.262831in}}%
\pgfpathlineto{\pgfqpoint{4.744693in}{3.286394in}}%
\pgfpathlineto{\pgfqpoint{4.752389in}{3.310429in}}%
\pgfpathlineto{\pgfqpoint{4.760091in}{3.334946in}}%
\pgfpathlineto{\pgfqpoint{4.767798in}{3.359954in}}%
\pgfpathlineto{\pgfqpoint{4.754321in}{3.368372in}}%
\pgfpathlineto{\pgfqpoint{4.740848in}{3.376866in}}%
\pgfpathlineto{\pgfqpoint{4.727380in}{3.385436in}}%
\pgfpathlineto{\pgfqpoint{4.713916in}{3.394084in}}%
\pgfpathlineto{\pgfqpoint{4.706205in}{3.368546in}}%
\pgfpathlineto{\pgfqpoint{4.698499in}{3.343506in}}%
\pgfpathlineto{\pgfqpoint{4.690798in}{3.318953in}}%
\pgfpathlineto{\pgfqpoint{4.683102in}{3.294877in}}%
\pgfpathclose%
\pgfusepath{fill}%
\end{pgfscope}%
\begin{pgfscope}%
\pgfpathrectangle{\pgfqpoint{1.150000in}{0.150000in}}{\pgfqpoint{5.700000in}{5.700000in}}%
\pgfusepath{clip}%
\pgfsetbuttcap%
\pgfsetroundjoin%
\definecolor{currentfill}{rgb}{0.212395,0.359683,0.551710}%
\pgfsetfillcolor{currentfill}%
\pgfsetfillopacity{0.700000}%
\pgfsetlinewidth{0.000000pt}%
\definecolor{currentstroke}{rgb}{0.000000,0.000000,0.000000}%
\pgfsetstrokecolor{currentstroke}%
\pgfsetdash{}{0pt}%
\pgfpathmoveto{\pgfqpoint{4.567802in}{3.147675in}}%
\pgfpathlineto{\pgfqpoint{4.581262in}{3.140217in}}%
\pgfpathlineto{\pgfqpoint{4.594726in}{3.132836in}}%
\pgfpathlineto{\pgfqpoint{4.608196in}{3.125532in}}%
\pgfpathlineto{\pgfqpoint{4.621671in}{3.118305in}}%
\pgfpathlineto{\pgfqpoint{4.629339in}{3.138905in}}%
\pgfpathlineto{\pgfqpoint{4.637010in}{3.159907in}}%
\pgfpathlineto{\pgfqpoint{4.644683in}{3.181321in}}%
\pgfpathlineto{\pgfqpoint{4.652360in}{3.203154in}}%
\pgfpathlineto{\pgfqpoint{4.638891in}{3.210858in}}%
\pgfpathlineto{\pgfqpoint{4.625428in}{3.218640in}}%
\pgfpathlineto{\pgfqpoint{4.611968in}{3.226499in}}%
\pgfpathlineto{\pgfqpoint{4.598514in}{3.234435in}}%
\pgfpathlineto{\pgfqpoint{4.590832in}{3.212116in}}%
\pgfpathlineto{\pgfqpoint{4.583153in}{3.190222in}}%
\pgfpathlineto{\pgfqpoint{4.575476in}{3.168745in}}%
\pgfpathlineto{\pgfqpoint{4.567802in}{3.147675in}}%
\pgfpathclose%
\pgfusepath{fill}%
\end{pgfscope}%
\begin{pgfscope}%
\pgfpathrectangle{\pgfqpoint{1.150000in}{0.150000in}}{\pgfqpoint{5.700000in}{5.700000in}}%
\pgfusepath{clip}%
\pgfsetbuttcap%
\pgfsetroundjoin%
\definecolor{currentfill}{rgb}{0.260571,0.246922,0.522828}%
\pgfsetfillcolor{currentfill}%
\pgfsetfillopacity{0.700000}%
\pgfsetlinewidth{0.000000pt}%
\definecolor{currentstroke}{rgb}{0.000000,0.000000,0.000000}%
\pgfsetstrokecolor{currentstroke}%
\pgfsetdash{}{0pt}%
\pgfpathmoveto{\pgfqpoint{3.063848in}{2.908503in}}%
\pgfpathlineto{\pgfqpoint{3.077147in}{2.897899in}}%
\pgfpathlineto{\pgfqpoint{3.090447in}{2.887419in}}%
\pgfpathlineto{\pgfqpoint{3.103746in}{2.877063in}}%
\pgfpathlineto{\pgfqpoint{3.117046in}{2.866829in}}%
\pgfpathlineto{\pgfqpoint{3.125027in}{2.879817in}}%
\pgfpathlineto{\pgfqpoint{3.133001in}{2.892962in}}%
\pgfpathlineto{\pgfqpoint{3.140968in}{2.906270in}}%
\pgfpathlineto{\pgfqpoint{3.148928in}{2.919743in}}%
\pgfpathlineto{\pgfqpoint{3.135635in}{2.930173in}}%
\pgfpathlineto{\pgfqpoint{3.122342in}{2.940725in}}%
\pgfpathlineto{\pgfqpoint{3.109049in}{2.951401in}}%
\pgfpathlineto{\pgfqpoint{3.095757in}{2.962200in}}%
\pgfpathlineto{\pgfqpoint{3.087790in}{2.948524in}}%
\pgfpathlineto{\pgfqpoint{3.079817in}{2.935018in}}%
\pgfpathlineto{\pgfqpoint{3.071836in}{2.921679in}}%
\pgfpathlineto{\pgfqpoint{3.063848in}{2.908503in}}%
\pgfpathclose%
\pgfusepath{fill}%
\end{pgfscope}%
\begin{pgfscope}%
\pgfpathrectangle{\pgfqpoint{1.150000in}{0.150000in}}{\pgfqpoint{5.700000in}{5.700000in}}%
\pgfusepath{clip}%
\pgfsetbuttcap%
\pgfsetroundjoin%
\definecolor{currentfill}{rgb}{0.262138,0.242286,0.520837}%
\pgfsetfillcolor{currentfill}%
\pgfsetfillopacity{0.700000}%
\pgfsetlinewidth{0.000000pt}%
\definecolor{currentstroke}{rgb}{0.000000,0.000000,0.000000}%
\pgfsetstrokecolor{currentstroke}%
\pgfsetdash{}{0pt}%
\pgfpathmoveto{\pgfqpoint{4.061108in}{2.887535in}}%
\pgfpathlineto{\pgfqpoint{4.074495in}{2.880552in}}%
\pgfpathlineto{\pgfqpoint{4.087886in}{2.873655in}}%
\pgfpathlineto{\pgfqpoint{4.101282in}{2.866844in}}%
\pgfpathlineto{\pgfqpoint{4.114682in}{2.860117in}}%
\pgfpathlineto{\pgfqpoint{4.122406in}{2.875552in}}%
\pgfpathlineto{\pgfqpoint{4.130128in}{2.891236in}}%
\pgfpathlineto{\pgfqpoint{4.137846in}{2.907176in}}%
\pgfpathlineto{\pgfqpoint{4.145562in}{2.923379in}}%
\pgfpathlineto{\pgfqpoint{4.132169in}{2.930461in}}%
\pgfpathlineto{\pgfqpoint{4.118779in}{2.937627in}}%
\pgfpathlineto{\pgfqpoint{4.105394in}{2.944880in}}%
\pgfpathlineto{\pgfqpoint{4.092013in}{2.952218in}}%
\pgfpathlineto{\pgfqpoint{4.084291in}{2.935653in}}%
\pgfpathlineto{\pgfqpoint{4.076567in}{2.919355in}}%
\pgfpathlineto{\pgfqpoint{4.068839in}{2.903318in}}%
\pgfpathlineto{\pgfqpoint{4.061108in}{2.887535in}}%
\pgfpathclose%
\pgfusepath{fill}%
\end{pgfscope}%
\begin{pgfscope}%
\pgfpathrectangle{\pgfqpoint{1.150000in}{0.150000in}}{\pgfqpoint{5.700000in}{5.700000in}}%
\pgfusepath{clip}%
\pgfsetbuttcap%
\pgfsetroundjoin%
\definecolor{currentfill}{rgb}{0.246811,0.283237,0.535941}%
\pgfsetfillcolor{currentfill}%
\pgfsetfillopacity{0.700000}%
\pgfsetlinewidth{0.000000pt}%
\definecolor{currentstroke}{rgb}{0.000000,0.000000,0.000000}%
\pgfsetstrokecolor{currentstroke}%
\pgfsetdash{}{0pt}%
\pgfpathmoveto{\pgfqpoint{2.872111in}{2.992579in}}%
\pgfpathlineto{\pgfqpoint{2.885426in}{2.980574in}}%
\pgfpathlineto{\pgfqpoint{2.898739in}{2.968707in}}%
\pgfpathlineto{\pgfqpoint{2.912051in}{2.956976in}}%
\pgfpathlineto{\pgfqpoint{2.925362in}{2.945381in}}%
\pgfpathlineto{\pgfqpoint{2.933394in}{2.958277in}}%
\pgfpathlineto{\pgfqpoint{2.941419in}{2.971333in}}%
\pgfpathlineto{\pgfqpoint{2.949435in}{2.984550in}}%
\pgfpathlineto{\pgfqpoint{2.957445in}{2.997934in}}%
\pgfpathlineto{\pgfqpoint{2.944141in}{3.009706in}}%
\pgfpathlineto{\pgfqpoint{2.930836in}{3.021612in}}%
\pgfpathlineto{\pgfqpoint{2.917530in}{3.033656in}}%
\pgfpathlineto{\pgfqpoint{2.904223in}{3.045838in}}%
\pgfpathlineto{\pgfqpoint{2.896207in}{3.032270in}}%
\pgfpathlineto{\pgfqpoint{2.888183in}{3.018873in}}%
\pgfpathlineto{\pgfqpoint{2.880151in}{3.005644in}}%
\pgfpathlineto{\pgfqpoint{2.872111in}{2.992579in}}%
\pgfpathclose%
\pgfusepath{fill}%
\end{pgfscope}%
\begin{pgfscope}%
\pgfpathrectangle{\pgfqpoint{1.150000in}{0.150000in}}{\pgfqpoint{5.700000in}{5.700000in}}%
\pgfusepath{clip}%
\pgfsetbuttcap%
\pgfsetroundjoin%
\definecolor{currentfill}{rgb}{0.136408,0.541173,0.554483}%
\pgfsetfillcolor{currentfill}%
\pgfsetfillopacity{0.700000}%
\pgfsetlinewidth{0.000000pt}%
\definecolor{currentstroke}{rgb}{0.000000,0.000000,0.000000}%
\pgfsetstrokecolor{currentstroke}%
\pgfsetdash{}{0pt}%
\pgfpathmoveto{\pgfqpoint{4.775843in}{3.617548in}}%
\pgfpathlineto{\pgfqpoint{4.789302in}{3.607791in}}%
\pgfpathlineto{\pgfqpoint{4.802765in}{3.598111in}}%
\pgfpathlineto{\pgfqpoint{4.816232in}{3.588509in}}%
\pgfpathlineto{\pgfqpoint{4.829702in}{3.578983in}}%
\pgfpathlineto{\pgfqpoint{4.837477in}{3.608894in}}%
\pgfpathlineto{\pgfqpoint{4.845261in}{3.639405in}}%
\pgfpathlineto{\pgfqpoint{4.853055in}{3.670528in}}%
\pgfpathlineto{\pgfqpoint{4.839585in}{3.680493in}}%
\pgfpathlineto{\pgfqpoint{4.826119in}{3.690536in}}%
\pgfpathlineto{\pgfqpoint{4.812657in}{3.700656in}}%
\pgfpathlineto{\pgfqpoint{4.799198in}{3.710853in}}%
\pgfpathlineto{\pgfqpoint{4.791404in}{3.679137in}}%
\pgfpathlineto{\pgfqpoint{4.783619in}{3.648039in}}%
\pgfpathlineto{\pgfqpoint{4.775843in}{3.617548in}}%
\pgfpathclose%
\pgfusepath{fill}%
\end{pgfscope}%
\begin{pgfscope}%
\pgfpathrectangle{\pgfqpoint{1.150000in}{0.150000in}}{\pgfqpoint{5.700000in}{5.700000in}}%
\pgfusepath{clip}%
\pgfsetbuttcap%
\pgfsetroundjoin%
\definecolor{currentfill}{rgb}{0.270595,0.214069,0.507052}%
\pgfsetfillcolor{currentfill}%
\pgfsetfillopacity{0.700000}%
\pgfsetlinewidth{0.000000pt}%
\definecolor{currentstroke}{rgb}{0.000000,0.000000,0.000000}%
\pgfsetstrokecolor{currentstroke}%
\pgfsetdash{}{0pt}%
\pgfpathmoveto{\pgfqpoint{3.754093in}{2.824397in}}%
\pgfpathlineto{\pgfqpoint{3.767437in}{2.816922in}}%
\pgfpathlineto{\pgfqpoint{3.780784in}{2.809541in}}%
\pgfpathlineto{\pgfqpoint{3.794134in}{2.802252in}}%
\pgfpathlineto{\pgfqpoint{3.807489in}{2.795057in}}%
\pgfpathlineto{\pgfqpoint{3.815287in}{2.809124in}}%
\pgfpathlineto{\pgfqpoint{3.823080in}{2.823390in}}%
\pgfpathlineto{\pgfqpoint{3.830869in}{2.837859in}}%
\pgfpathlineto{\pgfqpoint{3.838654in}{2.852538in}}%
\pgfpathlineto{\pgfqpoint{3.825306in}{2.860029in}}%
\pgfpathlineto{\pgfqpoint{3.811961in}{2.867613in}}%
\pgfpathlineto{\pgfqpoint{3.798620in}{2.875289in}}%
\pgfpathlineto{\pgfqpoint{3.785282in}{2.883059in}}%
\pgfpathlineto{\pgfqpoint{3.777492in}{2.868078in}}%
\pgfpathlineto{\pgfqpoint{3.769697in}{2.853310in}}%
\pgfpathlineto{\pgfqpoint{3.761898in}{2.838752in}}%
\pgfpathlineto{\pgfqpoint{3.754093in}{2.824397in}}%
\pgfpathclose%
\pgfusepath{fill}%
\end{pgfscope}%
\begin{pgfscope}%
\pgfpathrectangle{\pgfqpoint{1.150000in}{0.150000in}}{\pgfqpoint{5.700000in}{5.700000in}}%
\pgfusepath{clip}%
\pgfsetbuttcap%
\pgfsetroundjoin%
\definecolor{currentfill}{rgb}{0.199430,0.387607,0.554642}%
\pgfsetfillcolor{currentfill}%
\pgfsetfillopacity{0.700000}%
\pgfsetlinewidth{0.000000pt}%
\definecolor{currentstroke}{rgb}{0.000000,0.000000,0.000000}%
\pgfsetstrokecolor{currentstroke}%
\pgfsetdash{}{0pt}%
\pgfpathmoveto{\pgfqpoint{4.652360in}{3.203154in}}%
\pgfpathlineto{\pgfqpoint{4.665833in}{3.195527in}}%
\pgfpathlineto{\pgfqpoint{4.679311in}{3.187976in}}%
\pgfpathlineto{\pgfqpoint{4.692794in}{3.180501in}}%
\pgfpathlineto{\pgfqpoint{4.706282in}{3.173102in}}%
\pgfpathlineto{\pgfqpoint{4.713956in}{3.194875in}}%
\pgfpathlineto{\pgfqpoint{4.721634in}{3.217081in}}%
\pgfpathlineto{\pgfqpoint{4.729316in}{3.239729in}}%
\pgfpathlineto{\pgfqpoint{4.737002in}{3.262831in}}%
\pgfpathlineto{\pgfqpoint{4.723520in}{3.270728in}}%
\pgfpathlineto{\pgfqpoint{4.710043in}{3.278701in}}%
\pgfpathlineto{\pgfqpoint{4.696570in}{3.286751in}}%
\pgfpathlineto{\pgfqpoint{4.683102in}{3.294877in}}%
\pgfpathlineto{\pgfqpoint{4.675411in}{3.271269in}}%
\pgfpathlineto{\pgfqpoint{4.667723in}{3.248119in}}%
\pgfpathlineto{\pgfqpoint{4.660040in}{3.225417in}}%
\pgfpathlineto{\pgfqpoint{4.652360in}{3.203154in}}%
\pgfpathclose%
\pgfusepath{fill}%
\end{pgfscope}%
\begin{pgfscope}%
\pgfpathrectangle{\pgfqpoint{1.150000in}{0.150000in}}{\pgfqpoint{5.700000in}{5.700000in}}%
\pgfusepath{clip}%
\pgfsetbuttcap%
\pgfsetroundjoin%
\definecolor{currentfill}{rgb}{0.266580,0.228262,0.514349}%
\pgfsetfillcolor{currentfill}%
\pgfsetfillopacity{0.700000}%
\pgfsetlinewidth{0.000000pt}%
\definecolor{currentstroke}{rgb}{0.000000,0.000000,0.000000}%
\pgfsetstrokecolor{currentstroke}%
\pgfsetdash{}{0pt}%
\pgfpathmoveto{\pgfqpoint{3.976619in}{2.854282in}}%
\pgfpathlineto{\pgfqpoint{3.989995in}{2.847286in}}%
\pgfpathlineto{\pgfqpoint{4.003376in}{2.840378in}}%
\pgfpathlineto{\pgfqpoint{4.016761in}{2.833557in}}%
\pgfpathlineto{\pgfqpoint{4.030150in}{2.826823in}}%
\pgfpathlineto{\pgfqpoint{4.037895in}{2.841651in}}%
\pgfpathlineto{\pgfqpoint{4.045636in}{2.856708in}}%
\pgfpathlineto{\pgfqpoint{4.053374in}{2.872001in}}%
\pgfpathlineto{\pgfqpoint{4.061108in}{2.887535in}}%
\pgfpathlineto{\pgfqpoint{4.047725in}{2.894605in}}%
\pgfpathlineto{\pgfqpoint{4.034347in}{2.901761in}}%
\pgfpathlineto{\pgfqpoint{4.020972in}{2.909004in}}%
\pgfpathlineto{\pgfqpoint{4.007601in}{2.916335in}}%
\pgfpathlineto{\pgfqpoint{3.999861in}{2.900458in}}%
\pgfpathlineto{\pgfqpoint{3.992117in}{2.884827in}}%
\pgfpathlineto{\pgfqpoint{3.984370in}{2.869437in}}%
\pgfpathlineto{\pgfqpoint{3.976619in}{2.854282in}}%
\pgfpathclose%
\pgfusepath{fill}%
\end{pgfscope}%
\begin{pgfscope}%
\pgfpathrectangle{\pgfqpoint{1.150000in}{0.150000in}}{\pgfqpoint{5.700000in}{5.700000in}}%
\pgfusepath{clip}%
\pgfsetbuttcap%
\pgfsetroundjoin%
\definecolor{currentfill}{rgb}{0.271828,0.209303,0.504434}%
\pgfsetfillcolor{currentfill}%
\pgfsetfillopacity{0.700000}%
\pgfsetlinewidth{0.000000pt}%
\definecolor{currentstroke}{rgb}{0.000000,0.000000,0.000000}%
\pgfsetstrokecolor{currentstroke}%
\pgfsetdash{}{0pt}%
\pgfpathmoveto{\pgfqpoint{3.393410in}{2.820885in}}%
\pgfpathlineto{\pgfqpoint{3.406717in}{2.812176in}}%
\pgfpathlineto{\pgfqpoint{3.420026in}{2.803573in}}%
\pgfpathlineto{\pgfqpoint{3.433337in}{2.795076in}}%
\pgfpathlineto{\pgfqpoint{3.446651in}{2.786685in}}%
\pgfpathlineto{\pgfqpoint{3.454546in}{2.799903in}}%
\pgfpathlineto{\pgfqpoint{3.462435in}{2.813287in}}%
\pgfpathlineto{\pgfqpoint{3.470319in}{2.826840in}}%
\pgfpathlineto{\pgfqpoint{3.478196in}{2.840567in}}%
\pgfpathlineto{\pgfqpoint{3.464889in}{2.849194in}}%
\pgfpathlineto{\pgfqpoint{3.451584in}{2.857927in}}%
\pgfpathlineto{\pgfqpoint{3.438281in}{2.866765in}}%
\pgfpathlineto{\pgfqpoint{3.424980in}{2.875710in}}%
\pgfpathlineto{\pgfqpoint{3.417097in}{2.861740in}}%
\pgfpathlineto{\pgfqpoint{3.409207in}{2.847948in}}%
\pgfpathlineto{\pgfqpoint{3.401312in}{2.834331in}}%
\pgfpathlineto{\pgfqpoint{3.393410in}{2.820885in}}%
\pgfpathclose%
\pgfusepath{fill}%
\end{pgfscope}%
\begin{pgfscope}%
\pgfpathrectangle{\pgfqpoint{1.150000in}{0.150000in}}{\pgfqpoint{5.700000in}{5.700000in}}%
\pgfusepath{clip}%
\pgfsetbuttcap%
\pgfsetroundjoin%
\definecolor{currentfill}{rgb}{0.269308,0.218818,0.509577}%
\pgfsetfillcolor{currentfill}%
\pgfsetfillopacity{0.700000}%
\pgfsetlinewidth{0.000000pt}%
\definecolor{currentstroke}{rgb}{0.000000,0.000000,0.000000}%
\pgfsetstrokecolor{currentstroke}%
\pgfsetdash{}{0pt}%
\pgfpathmoveto{\pgfqpoint{3.255295in}{2.840585in}}%
\pgfpathlineto{\pgfqpoint{3.268595in}{2.831212in}}%
\pgfpathlineto{\pgfqpoint{3.281897in}{2.821951in}}%
\pgfpathlineto{\pgfqpoint{3.295200in}{2.812803in}}%
\pgfpathlineto{\pgfqpoint{3.308505in}{2.803765in}}%
\pgfpathlineto{\pgfqpoint{3.316439in}{2.816783in}}%
\pgfpathlineto{\pgfqpoint{3.324366in}{2.829958in}}%
\pgfpathlineto{\pgfqpoint{3.332286in}{2.843296in}}%
\pgfpathlineto{\pgfqpoint{3.340200in}{2.856801in}}%
\pgfpathlineto{\pgfqpoint{3.326902in}{2.866054in}}%
\pgfpathlineto{\pgfqpoint{3.313606in}{2.875418in}}%
\pgfpathlineto{\pgfqpoint{3.300310in}{2.884894in}}%
\pgfpathlineto{\pgfqpoint{3.287016in}{2.894483in}}%
\pgfpathlineto{\pgfqpoint{3.279096in}{2.880756in}}%
\pgfpathlineto{\pgfqpoint{3.271169in}{2.867199in}}%
\pgfpathlineto{\pgfqpoint{3.263235in}{2.853811in}}%
\pgfpathlineto{\pgfqpoint{3.255295in}{2.840585in}}%
\pgfpathclose%
\pgfusepath{fill}%
\end{pgfscope}%
\begin{pgfscope}%
\pgfpathrectangle{\pgfqpoint{1.150000in}{0.150000in}}{\pgfqpoint{5.700000in}{5.700000in}}%
\pgfusepath{clip}%
\pgfsetbuttcap%
\pgfsetroundjoin%
\definecolor{currentfill}{rgb}{0.273006,0.204520,0.501721}%
\pgfsetfillcolor{currentfill}%
\pgfsetfillopacity{0.700000}%
\pgfsetlinewidth{0.000000pt}%
\definecolor{currentstroke}{rgb}{0.000000,0.000000,0.000000}%
\pgfsetstrokecolor{currentstroke}%
\pgfsetdash{}{0pt}%
\pgfpathmoveto{\pgfqpoint{3.531448in}{2.807093in}}%
\pgfpathlineto{\pgfqpoint{3.544767in}{2.798980in}}%
\pgfpathlineto{\pgfqpoint{3.558089in}{2.790968in}}%
\pgfpathlineto{\pgfqpoint{3.571414in}{2.783056in}}%
\pgfpathlineto{\pgfqpoint{3.584742in}{2.775244in}}%
\pgfpathlineto{\pgfqpoint{3.592601in}{2.788653in}}%
\pgfpathlineto{\pgfqpoint{3.600454in}{2.802235in}}%
\pgfpathlineto{\pgfqpoint{3.608302in}{2.815995in}}%
\pgfpathlineto{\pgfqpoint{3.616145in}{2.829938in}}%
\pgfpathlineto{\pgfqpoint{3.602823in}{2.838006in}}%
\pgfpathlineto{\pgfqpoint{3.589505in}{2.846173in}}%
\pgfpathlineto{\pgfqpoint{3.576189in}{2.854441in}}%
\pgfpathlineto{\pgfqpoint{3.562875in}{2.862810in}}%
\pgfpathlineto{\pgfqpoint{3.555027in}{2.848604in}}%
\pgfpathlineto{\pgfqpoint{3.547173in}{2.834586in}}%
\pgfpathlineto{\pgfqpoint{3.539313in}{2.820750in}}%
\pgfpathlineto{\pgfqpoint{3.531448in}{2.807093in}}%
\pgfpathclose%
\pgfusepath{fill}%
\end{pgfscope}%
\begin{pgfscope}%
\pgfpathrectangle{\pgfqpoint{1.150000in}{0.150000in}}{\pgfqpoint{5.700000in}{5.700000in}}%
\pgfusepath{clip}%
\pgfsetbuttcap%
\pgfsetroundjoin%
\definecolor{currentfill}{rgb}{0.253935,0.265254,0.529983}%
\pgfsetfillcolor{currentfill}%
\pgfsetfillopacity{0.700000}%
\pgfsetlinewidth{0.000000pt}%
\definecolor{currentstroke}{rgb}{0.000000,0.000000,0.000000}%
\pgfsetstrokecolor{currentstroke}%
\pgfsetdash{}{0pt}%
\pgfpathmoveto{\pgfqpoint{2.925362in}{2.945381in}}%
\pgfpathlineto{\pgfqpoint{2.938671in}{2.933920in}}%
\pgfpathlineto{\pgfqpoint{2.951980in}{2.922593in}}%
\pgfpathlineto{\pgfqpoint{2.965289in}{2.911397in}}%
\pgfpathlineto{\pgfqpoint{2.978596in}{2.900332in}}%
\pgfpathlineto{\pgfqpoint{2.986621in}{2.913060in}}%
\pgfpathlineto{\pgfqpoint{2.994638in}{2.925941in}}%
\pgfpathlineto{\pgfqpoint{3.002648in}{2.938981in}}%
\pgfpathlineto{\pgfqpoint{3.010650in}{2.952181in}}%
\pgfpathlineto{\pgfqpoint{2.997350in}{2.963422in}}%
\pgfpathlineto{\pgfqpoint{2.984049in}{2.974794in}}%
\pgfpathlineto{\pgfqpoint{2.970747in}{2.986297in}}%
\pgfpathlineto{\pgfqpoint{2.957445in}{2.997934in}}%
\pgfpathlineto{\pgfqpoint{2.949435in}{2.984550in}}%
\pgfpathlineto{\pgfqpoint{2.941419in}{2.971333in}}%
\pgfpathlineto{\pgfqpoint{2.933394in}{2.958277in}}%
\pgfpathlineto{\pgfqpoint{2.925362in}{2.945381in}}%
\pgfpathclose%
\pgfusepath{fill}%
\end{pgfscope}%
\begin{pgfscope}%
\pgfpathrectangle{\pgfqpoint{1.150000in}{0.150000in}}{\pgfqpoint{5.700000in}{5.700000in}}%
\pgfusepath{clip}%
\pgfsetbuttcap%
\pgfsetroundjoin%
\definecolor{currentfill}{rgb}{0.244972,0.287675,0.537260}%
\pgfsetfillcolor{currentfill}%
\pgfsetfillopacity{0.700000}%
\pgfsetlinewidth{0.000000pt}%
\definecolor{currentstroke}{rgb}{0.000000,0.000000,0.000000}%
\pgfsetstrokecolor{currentstroke}%
\pgfsetdash{}{0pt}%
\pgfpathmoveto{\pgfqpoint{4.368134in}{2.975542in}}%
\pgfpathlineto{\pgfqpoint{4.381574in}{2.968761in}}%
\pgfpathlineto{\pgfqpoint{4.395020in}{2.962060in}}%
\pgfpathlineto{\pgfqpoint{4.408470in}{2.955438in}}%
\pgfpathlineto{\pgfqpoint{4.421925in}{2.948897in}}%
\pgfpathlineto{\pgfqpoint{4.429599in}{2.966143in}}%
\pgfpathlineto{\pgfqpoint{4.437271in}{2.983702in}}%
\pgfpathlineto{\pgfqpoint{4.444943in}{3.001583in}}%
\pgfpathlineto{\pgfqpoint{4.452615in}{3.019794in}}%
\pgfpathlineto{\pgfqpoint{4.439167in}{3.026751in}}%
\pgfpathlineto{\pgfqpoint{4.425723in}{3.033789in}}%
\pgfpathlineto{\pgfqpoint{4.412284in}{3.040905in}}%
\pgfpathlineto{\pgfqpoint{4.398850in}{3.048102in}}%
\pgfpathlineto{\pgfqpoint{4.391172in}{3.029468in}}%
\pgfpathlineto{\pgfqpoint{4.383493in}{3.011169in}}%
\pgfpathlineto{\pgfqpoint{4.375814in}{2.993196in}}%
\pgfpathlineto{\pgfqpoint{4.368134in}{2.975542in}}%
\pgfpathclose%
\pgfusepath{fill}%
\end{pgfscope}%
\begin{pgfscope}%
\pgfpathrectangle{\pgfqpoint{1.150000in}{0.150000in}}{\pgfqpoint{5.700000in}{5.700000in}}%
\pgfusepath{clip}%
\pgfsetbuttcap%
\pgfsetroundjoin%
\definecolor{currentfill}{rgb}{0.235526,0.309527,0.542944}%
\pgfsetfillcolor{currentfill}%
\pgfsetfillopacity{0.700000}%
\pgfsetlinewidth{0.000000pt}%
\definecolor{currentstroke}{rgb}{0.000000,0.000000,0.000000}%
\pgfsetstrokecolor{currentstroke}%
\pgfsetdash{}{0pt}%
\pgfpathmoveto{\pgfqpoint{4.452615in}{3.019794in}}%
\pgfpathlineto{\pgfqpoint{4.466069in}{3.012915in}}%
\pgfpathlineto{\pgfqpoint{4.479527in}{3.006115in}}%
\pgfpathlineto{\pgfqpoint{4.492990in}{2.999394in}}%
\pgfpathlineto{\pgfqpoint{4.506458in}{2.992750in}}%
\pgfpathlineto{\pgfqpoint{4.514123in}{3.010869in}}%
\pgfpathlineto{\pgfqpoint{4.521789in}{3.029327in}}%
\pgfpathlineto{\pgfqpoint{4.529455in}{3.048134in}}%
\pgfpathlineto{\pgfqpoint{4.537122in}{3.067296in}}%
\pgfpathlineto{\pgfqpoint{4.523660in}{3.074376in}}%
\pgfpathlineto{\pgfqpoint{4.510204in}{3.081534in}}%
\pgfpathlineto{\pgfqpoint{4.496752in}{3.088770in}}%
\pgfpathlineto{\pgfqpoint{4.483305in}{3.096085in}}%
\pgfpathlineto{\pgfqpoint{4.475632in}{3.076478in}}%
\pgfpathlineto{\pgfqpoint{4.467959in}{3.057233in}}%
\pgfpathlineto{\pgfqpoint{4.460287in}{3.038341in}}%
\pgfpathlineto{\pgfqpoint{4.452615in}{3.019794in}}%
\pgfpathclose%
\pgfusepath{fill}%
\end{pgfscope}%
\begin{pgfscope}%
\pgfpathrectangle{\pgfqpoint{1.150000in}{0.150000in}}{\pgfqpoint{5.700000in}{5.700000in}}%
\pgfusepath{clip}%
\pgfsetbuttcap%
\pgfsetroundjoin%
\definecolor{currentfill}{rgb}{0.172719,0.448791,0.557885}%
\pgfsetfillcolor{currentfill}%
\pgfsetfillopacity{0.700000}%
\pgfsetlinewidth{0.000000pt}%
\definecolor{currentstroke}{rgb}{0.000000,0.000000,0.000000}%
\pgfsetstrokecolor{currentstroke}%
\pgfsetdash{}{0pt}%
\pgfpathmoveto{\pgfqpoint{4.767798in}{3.359954in}}%
\pgfpathlineto{\pgfqpoint{4.781279in}{3.351613in}}%
\pgfpathlineto{\pgfqpoint{4.794766in}{3.343347in}}%
\pgfpathlineto{\pgfqpoint{4.808257in}{3.335156in}}%
\pgfpathlineto{\pgfqpoint{4.821752in}{3.327041in}}%
\pgfpathlineto{\pgfqpoint{4.829461in}{3.352018in}}%
\pgfpathlineto{\pgfqpoint{4.837176in}{3.377502in}}%
\pgfpathlineto{\pgfqpoint{4.844898in}{3.403503in}}%
\pgfpathlineto{\pgfqpoint{4.852627in}{3.430031in}}%
\pgfpathlineto{\pgfqpoint{4.839136in}{3.438689in}}%
\pgfpathlineto{\pgfqpoint{4.825650in}{3.447421in}}%
\pgfpathlineto{\pgfqpoint{4.812168in}{3.456230in}}%
\pgfpathlineto{\pgfqpoint{4.798690in}{3.465114in}}%
\pgfpathlineto{\pgfqpoint{4.790957in}{3.438034in}}%
\pgfpathlineto{\pgfqpoint{4.783231in}{3.411488in}}%
\pgfpathlineto{\pgfqpoint{4.775511in}{3.385465in}}%
\pgfpathlineto{\pgfqpoint{4.767798in}{3.359954in}}%
\pgfpathclose%
\pgfusepath{fill}%
\end{pgfscope}%
\begin{pgfscope}%
\pgfpathrectangle{\pgfqpoint{1.150000in}{0.150000in}}{\pgfqpoint{5.700000in}{5.700000in}}%
\pgfusepath{clip}%
\pgfsetbuttcap%
\pgfsetroundjoin%
\definecolor{currentfill}{rgb}{0.265145,0.232956,0.516599}%
\pgfsetfillcolor{currentfill}%
\pgfsetfillopacity{0.700000}%
\pgfsetlinewidth{0.000000pt}%
\definecolor{currentstroke}{rgb}{0.000000,0.000000,0.000000}%
\pgfsetstrokecolor{currentstroke}%
\pgfsetdash{}{0pt}%
\pgfpathmoveto{\pgfqpoint{3.117046in}{2.866829in}}%
\pgfpathlineto{\pgfqpoint{3.130346in}{2.856716in}}%
\pgfpathlineto{\pgfqpoint{3.143647in}{2.846723in}}%
\pgfpathlineto{\pgfqpoint{3.156948in}{2.836850in}}%
\pgfpathlineto{\pgfqpoint{3.170250in}{2.827095in}}%
\pgfpathlineto{\pgfqpoint{3.178224in}{2.839894in}}%
\pgfpathlineto{\pgfqpoint{3.186191in}{2.852846in}}%
\pgfpathlineto{\pgfqpoint{3.194151in}{2.865955in}}%
\pgfpathlineto{\pgfqpoint{3.202105in}{2.879225in}}%
\pgfpathlineto{\pgfqpoint{3.188810in}{2.889176in}}%
\pgfpathlineto{\pgfqpoint{3.175515in}{2.899245in}}%
\pgfpathlineto{\pgfqpoint{3.162221in}{2.909434in}}%
\pgfpathlineto{\pgfqpoint{3.148928in}{2.919743in}}%
\pgfpathlineto{\pgfqpoint{3.140968in}{2.906270in}}%
\pgfpathlineto{\pgfqpoint{3.133001in}{2.892962in}}%
\pgfpathlineto{\pgfqpoint{3.125027in}{2.879817in}}%
\pgfpathlineto{\pgfqpoint{3.117046in}{2.866829in}}%
\pgfpathclose%
\pgfusepath{fill}%
\end{pgfscope}%
\begin{pgfscope}%
\pgfpathrectangle{\pgfqpoint{1.150000in}{0.150000in}}{\pgfqpoint{5.700000in}{5.700000in}}%
\pgfusepath{clip}%
\pgfsetbuttcap%
\pgfsetroundjoin%
\definecolor{currentfill}{rgb}{0.252194,0.269783,0.531579}%
\pgfsetfillcolor{currentfill}%
\pgfsetfillopacity{0.700000}%
\pgfsetlinewidth{0.000000pt}%
\definecolor{currentstroke}{rgb}{0.000000,0.000000,0.000000}%
\pgfsetstrokecolor{currentstroke}%
\pgfsetdash{}{0pt}%
\pgfpathmoveto{\pgfqpoint{4.283661in}{2.934310in}}%
\pgfpathlineto{\pgfqpoint{4.297089in}{2.927601in}}%
\pgfpathlineto{\pgfqpoint{4.310522in}{2.920974in}}%
\pgfpathlineto{\pgfqpoint{4.323960in}{2.914428in}}%
\pgfpathlineto{\pgfqpoint{4.337402in}{2.907962in}}%
\pgfpathlineto{\pgfqpoint{4.345087in}{2.924416in}}%
\pgfpathlineto{\pgfqpoint{4.352771in}{2.941159in}}%
\pgfpathlineto{\pgfqpoint{4.360453in}{2.958198in}}%
\pgfpathlineto{\pgfqpoint{4.368134in}{2.975542in}}%
\pgfpathlineto{\pgfqpoint{4.354698in}{2.982403in}}%
\pgfpathlineto{\pgfqpoint{4.341267in}{2.989345in}}%
\pgfpathlineto{\pgfqpoint{4.327841in}{2.996368in}}%
\pgfpathlineto{\pgfqpoint{4.314419in}{3.003472in}}%
\pgfpathlineto{\pgfqpoint{4.306731in}{2.985725in}}%
\pgfpathlineto{\pgfqpoint{4.299043in}{2.968288in}}%
\pgfpathlineto{\pgfqpoint{4.291352in}{2.951152in}}%
\pgfpathlineto{\pgfqpoint{4.283661in}{2.934310in}}%
\pgfpathclose%
\pgfusepath{fill}%
\end{pgfscope}%
\begin{pgfscope}%
\pgfpathrectangle{\pgfqpoint{1.150000in}{0.150000in}}{\pgfqpoint{5.700000in}{5.700000in}}%
\pgfusepath{clip}%
\pgfsetbuttcap%
\pgfsetroundjoin%
\definecolor{currentfill}{rgb}{0.156270,0.489624,0.557936}%
\pgfsetfillcolor{currentfill}%
\pgfsetfillopacity{0.700000}%
\pgfsetlinewidth{0.000000pt}%
\definecolor{currentstroke}{rgb}{0.000000,0.000000,0.000000}%
\pgfsetstrokecolor{currentstroke}%
\pgfsetdash{}{0pt}%
\pgfpathmoveto{\pgfqpoint{4.798690in}{3.465114in}}%
\pgfpathlineto{\pgfqpoint{4.812168in}{3.456230in}}%
\pgfpathlineto{\pgfqpoint{4.825650in}{3.447421in}}%
\pgfpathlineto{\pgfqpoint{4.839136in}{3.438689in}}%
\pgfpathlineto{\pgfqpoint{4.852627in}{3.430031in}}%
\pgfpathlineto{\pgfqpoint{4.860365in}{3.457099in}}%
\pgfpathlineto{\pgfqpoint{4.868110in}{3.484715in}}%
\pgfpathlineto{\pgfqpoint{4.875864in}{3.512892in}}%
\pgfpathlineto{\pgfqpoint{4.883627in}{3.541641in}}%
\pgfpathlineto{\pgfqpoint{4.870140in}{3.550863in}}%
\pgfpathlineto{\pgfqpoint{4.856656in}{3.560160in}}%
\pgfpathlineto{\pgfqpoint{4.843177in}{3.569534in}}%
\pgfpathlineto{\pgfqpoint{4.829702in}{3.578983in}}%
\pgfpathlineto{\pgfqpoint{4.821937in}{3.549661in}}%
\pgfpathlineto{\pgfqpoint{4.814180in}{3.520916in}}%
\pgfpathlineto{\pgfqpoint{4.806431in}{3.492737in}}%
\pgfpathlineto{\pgfqpoint{4.798690in}{3.465114in}}%
\pgfpathclose%
\pgfusepath{fill}%
\end{pgfscope}%
\begin{pgfscope}%
\pgfpathrectangle{\pgfqpoint{1.150000in}{0.150000in}}{\pgfqpoint{5.700000in}{5.700000in}}%
\pgfusepath{clip}%
\pgfsetbuttcap%
\pgfsetroundjoin%
\definecolor{currentfill}{rgb}{0.225863,0.330805,0.547314}%
\pgfsetfillcolor{currentfill}%
\pgfsetfillopacity{0.700000}%
\pgfsetlinewidth{0.000000pt}%
\definecolor{currentstroke}{rgb}{0.000000,0.000000,0.000000}%
\pgfsetstrokecolor{currentstroke}%
\pgfsetdash{}{0pt}%
\pgfpathmoveto{\pgfqpoint{4.537122in}{3.067296in}}%
\pgfpathlineto{\pgfqpoint{4.550588in}{3.060295in}}%
\pgfpathlineto{\pgfqpoint{4.564059in}{3.053370in}}%
\pgfpathlineto{\pgfqpoint{4.577536in}{3.046524in}}%
\pgfpathlineto{\pgfqpoint{4.591017in}{3.039754in}}%
\pgfpathlineto{\pgfqpoint{4.598678in}{3.058832in}}%
\pgfpathlineto{\pgfqpoint{4.606340in}{3.078277in}}%
\pgfpathlineto{\pgfqpoint{4.614005in}{3.098099in}}%
\pgfpathlineto{\pgfqpoint{4.621671in}{3.118305in}}%
\pgfpathlineto{\pgfqpoint{4.608196in}{3.125532in}}%
\pgfpathlineto{\pgfqpoint{4.594726in}{3.132836in}}%
\pgfpathlineto{\pgfqpoint{4.581262in}{3.140217in}}%
\pgfpathlineto{\pgfqpoint{4.567802in}{3.147675in}}%
\pgfpathlineto{\pgfqpoint{4.560129in}{3.127004in}}%
\pgfpathlineto{\pgfqpoint{4.552459in}{3.106723in}}%
\pgfpathlineto{\pgfqpoint{4.544789in}{3.086823in}}%
\pgfpathlineto{\pgfqpoint{4.537122in}{3.067296in}}%
\pgfpathclose%
\pgfusepath{fill}%
\end{pgfscope}%
\begin{pgfscope}%
\pgfpathrectangle{\pgfqpoint{1.150000in}{0.150000in}}{\pgfqpoint{5.700000in}{5.700000in}}%
\pgfusepath{clip}%
\pgfsetbuttcap%
\pgfsetroundjoin%
\definecolor{currentfill}{rgb}{0.270595,0.214069,0.507052}%
\pgfsetfillcolor{currentfill}%
\pgfsetfillopacity{0.700000}%
\pgfsetlinewidth{0.000000pt}%
\definecolor{currentstroke}{rgb}{0.000000,0.000000,0.000000}%
\pgfsetstrokecolor{currentstroke}%
\pgfsetdash{}{0pt}%
\pgfpathmoveto{\pgfqpoint{3.892083in}{2.823490in}}%
\pgfpathlineto{\pgfqpoint{3.905450in}{2.816454in}}%
\pgfpathlineto{\pgfqpoint{3.918820in}{2.809508in}}%
\pgfpathlineto{\pgfqpoint{3.932195in}{2.802651in}}%
\pgfpathlineto{\pgfqpoint{3.945574in}{2.795882in}}%
\pgfpathlineto{\pgfqpoint{3.953341in}{2.810160in}}%
\pgfpathlineto{\pgfqpoint{3.961105in}{2.824649in}}%
\pgfpathlineto{\pgfqpoint{3.968864in}{2.839354in}}%
\pgfpathlineto{\pgfqpoint{3.976619in}{2.854282in}}%
\pgfpathlineto{\pgfqpoint{3.963246in}{2.861365in}}%
\pgfpathlineto{\pgfqpoint{3.949878in}{2.868538in}}%
\pgfpathlineto{\pgfqpoint{3.936513in}{2.875799in}}%
\pgfpathlineto{\pgfqpoint{3.923152in}{2.883151in}}%
\pgfpathlineto{\pgfqpoint{3.915391in}{2.867900in}}%
\pgfpathlineto{\pgfqpoint{3.907626in}{2.852877in}}%
\pgfpathlineto{\pgfqpoint{3.899857in}{2.838076in}}%
\pgfpathlineto{\pgfqpoint{3.892083in}{2.823490in}}%
\pgfpathclose%
\pgfusepath{fill}%
\end{pgfscope}%
\begin{pgfscope}%
\pgfpathrectangle{\pgfqpoint{1.150000in}{0.150000in}}{\pgfqpoint{5.700000in}{5.700000in}}%
\pgfusepath{clip}%
\pgfsetbuttcap%
\pgfsetroundjoin%
\definecolor{currentfill}{rgb}{0.273006,0.204520,0.501721}%
\pgfsetfillcolor{currentfill}%
\pgfsetfillopacity{0.700000}%
\pgfsetlinewidth{0.000000pt}%
\definecolor{currentstroke}{rgb}{0.000000,0.000000,0.000000}%
\pgfsetstrokecolor{currentstroke}%
\pgfsetdash{}{0pt}%
\pgfpathmoveto{\pgfqpoint{3.669459in}{2.798649in}}%
\pgfpathlineto{\pgfqpoint{3.682796in}{2.791070in}}%
\pgfpathlineto{\pgfqpoint{3.696136in}{2.783586in}}%
\pgfpathlineto{\pgfqpoint{3.709479in}{2.776197in}}%
\pgfpathlineto{\pgfqpoint{3.722825in}{2.768904in}}%
\pgfpathlineto{\pgfqpoint{3.730650in}{2.782498in}}%
\pgfpathlineto{\pgfqpoint{3.738469in}{2.796275in}}%
\pgfpathlineto{\pgfqpoint{3.746284in}{2.810239in}}%
\pgfpathlineto{\pgfqpoint{3.754093in}{2.824397in}}%
\pgfpathlineto{\pgfqpoint{3.740753in}{2.831966in}}%
\pgfpathlineto{\pgfqpoint{3.727416in}{2.839629in}}%
\pgfpathlineto{\pgfqpoint{3.714082in}{2.847388in}}%
\pgfpathlineto{\pgfqpoint{3.700752in}{2.855243in}}%
\pgfpathlineto{\pgfqpoint{3.692937in}{2.840803in}}%
\pgfpathlineto{\pgfqpoint{3.685116in}{2.826561in}}%
\pgfpathlineto{\pgfqpoint{3.677290in}{2.812511in}}%
\pgfpathlineto{\pgfqpoint{3.669459in}{2.798649in}}%
\pgfpathclose%
\pgfusepath{fill}%
\end{pgfscope}%
\begin{pgfscope}%
\pgfpathrectangle{\pgfqpoint{1.150000in}{0.150000in}}{\pgfqpoint{5.700000in}{5.700000in}}%
\pgfusepath{clip}%
\pgfsetbuttcap%
\pgfsetroundjoin%
\definecolor{currentfill}{rgb}{0.258965,0.251537,0.524736}%
\pgfsetfillcolor{currentfill}%
\pgfsetfillopacity{0.700000}%
\pgfsetlinewidth{0.000000pt}%
\definecolor{currentstroke}{rgb}{0.000000,0.000000,0.000000}%
\pgfsetstrokecolor{currentstroke}%
\pgfsetdash{}{0pt}%
\pgfpathmoveto{\pgfqpoint{4.199181in}{2.895895in}}%
\pgfpathlineto{\pgfqpoint{4.212597in}{2.889233in}}%
\pgfpathlineto{\pgfqpoint{4.226018in}{2.882654in}}%
\pgfpathlineto{\pgfqpoint{4.239444in}{2.876157in}}%
\pgfpathlineto{\pgfqpoint{4.252874in}{2.869742in}}%
\pgfpathlineto{\pgfqpoint{4.260574in}{2.885478in}}%
\pgfpathlineto{\pgfqpoint{4.268272in}{2.901480in}}%
\pgfpathlineto{\pgfqpoint{4.275967in}{2.917755in}}%
\pgfpathlineto{\pgfqpoint{4.283661in}{2.934310in}}%
\pgfpathlineto{\pgfqpoint{4.270237in}{2.941100in}}%
\pgfpathlineto{\pgfqpoint{4.256818in}{2.947973in}}%
\pgfpathlineto{\pgfqpoint{4.243404in}{2.954927in}}%
\pgfpathlineto{\pgfqpoint{4.229995in}{2.961965in}}%
\pgfpathlineto{\pgfqpoint{4.222295in}{2.945027in}}%
\pgfpathlineto{\pgfqpoint{4.214592in}{2.928374in}}%
\pgfpathlineto{\pgfqpoint{4.206888in}{2.911999in}}%
\pgfpathlineto{\pgfqpoint{4.199181in}{2.895895in}}%
\pgfpathclose%
\pgfusepath{fill}%
\end{pgfscope}%
\begin{pgfscope}%
\pgfpathrectangle{\pgfqpoint{1.150000in}{0.150000in}}{\pgfqpoint{5.700000in}{5.700000in}}%
\pgfusepath{clip}%
\pgfsetbuttcap%
\pgfsetroundjoin%
\definecolor{currentfill}{rgb}{0.188923,0.410910,0.556326}%
\pgfsetfillcolor{currentfill}%
\pgfsetfillopacity{0.700000}%
\pgfsetlinewidth{0.000000pt}%
\definecolor{currentstroke}{rgb}{0.000000,0.000000,0.000000}%
\pgfsetstrokecolor{currentstroke}%
\pgfsetdash{}{0pt}%
\pgfpathmoveto{\pgfqpoint{4.737002in}{3.262831in}}%
\pgfpathlineto{\pgfqpoint{4.750489in}{3.255009in}}%
\pgfpathlineto{\pgfqpoint{4.763980in}{3.247263in}}%
\pgfpathlineto{\pgfqpoint{4.777477in}{3.239593in}}%
\pgfpathlineto{\pgfqpoint{4.790979in}{3.231997in}}%
\pgfpathlineto{\pgfqpoint{4.798664in}{3.255048in}}%
\pgfpathlineto{\pgfqpoint{4.806354in}{3.278566in}}%
\pgfpathlineto{\pgfqpoint{4.814050in}{3.302560in}}%
\pgfpathlineto{\pgfqpoint{4.821752in}{3.327041in}}%
\pgfpathlineto{\pgfqpoint{4.808257in}{3.335156in}}%
\pgfpathlineto{\pgfqpoint{4.794766in}{3.343347in}}%
\pgfpathlineto{\pgfqpoint{4.781279in}{3.351613in}}%
\pgfpathlineto{\pgfqpoint{4.767798in}{3.359954in}}%
\pgfpathlineto{\pgfqpoint{4.760091in}{3.334946in}}%
\pgfpathlineto{\pgfqpoint{4.752389in}{3.310429in}}%
\pgfpathlineto{\pgfqpoint{4.744693in}{3.286394in}}%
\pgfpathlineto{\pgfqpoint{4.737002in}{3.262831in}}%
\pgfpathclose%
\pgfusepath{fill}%
\end{pgfscope}%
\begin{pgfscope}%
\pgfpathrectangle{\pgfqpoint{1.150000in}{0.150000in}}{\pgfqpoint{5.700000in}{5.700000in}}%
\pgfusepath{clip}%
\pgfsetbuttcap%
\pgfsetroundjoin%
\definecolor{currentfill}{rgb}{0.140536,0.530132,0.555659}%
\pgfsetfillcolor{currentfill}%
\pgfsetfillopacity{0.700000}%
\pgfsetlinewidth{0.000000pt}%
\definecolor{currentstroke}{rgb}{0.000000,0.000000,0.000000}%
\pgfsetstrokecolor{currentstroke}%
\pgfsetdash{}{0pt}%
\pgfpathmoveto{\pgfqpoint{4.829702in}{3.578983in}}%
\pgfpathlineto{\pgfqpoint{4.843177in}{3.569534in}}%
\pgfpathlineto{\pgfqpoint{4.856656in}{3.560160in}}%
\pgfpathlineto{\pgfqpoint{4.870140in}{3.550863in}}%
\pgfpathlineto{\pgfqpoint{4.883627in}{3.541641in}}%
\pgfpathlineto{\pgfqpoint{4.891399in}{3.570972in}}%
\pgfpathlineto{\pgfqpoint{4.899182in}{3.600898in}}%
\pgfpathlineto{\pgfqpoint{4.906974in}{3.631429in}}%
\pgfpathlineto{\pgfqpoint{4.893488in}{3.641090in}}%
\pgfpathlineto{\pgfqpoint{4.880006in}{3.650826in}}%
\pgfpathlineto{\pgfqpoint{4.866529in}{3.660639in}}%
\pgfpathlineto{\pgfqpoint{4.853055in}{3.670528in}}%
\pgfpathlineto{\pgfqpoint{4.845261in}{3.639405in}}%
\pgfpathlineto{\pgfqpoint{4.837477in}{3.608894in}}%
\pgfpathlineto{\pgfqpoint{4.829702in}{3.578983in}}%
\pgfpathclose%
\pgfusepath{fill}%
\end{pgfscope}%
\begin{pgfscope}%
\pgfpathrectangle{\pgfqpoint{1.150000in}{0.150000in}}{\pgfqpoint{5.700000in}{5.700000in}}%
\pgfusepath{clip}%
\pgfsetbuttcap%
\pgfsetroundjoin%
\definecolor{currentfill}{rgb}{0.216210,0.351535,0.550627}%
\pgfsetfillcolor{currentfill}%
\pgfsetfillopacity{0.700000}%
\pgfsetlinewidth{0.000000pt}%
\definecolor{currentstroke}{rgb}{0.000000,0.000000,0.000000}%
\pgfsetstrokecolor{currentstroke}%
\pgfsetdash{}{0pt}%
\pgfpathmoveto{\pgfqpoint{4.621671in}{3.118305in}}%
\pgfpathlineto{\pgfqpoint{4.635150in}{3.111155in}}%
\pgfpathlineto{\pgfqpoint{4.648635in}{3.104082in}}%
\pgfpathlineto{\pgfqpoint{4.662125in}{3.097085in}}%
\pgfpathlineto{\pgfqpoint{4.675619in}{3.090163in}}%
\pgfpathlineto{\pgfqpoint{4.683281in}{3.110294in}}%
\pgfpathlineto{\pgfqpoint{4.690945in}{3.130821in}}%
\pgfpathlineto{\pgfqpoint{4.698612in}{3.151754in}}%
\pgfpathlineto{\pgfqpoint{4.706282in}{3.173102in}}%
\pgfpathlineto{\pgfqpoint{4.692794in}{3.180501in}}%
\pgfpathlineto{\pgfqpoint{4.679311in}{3.187976in}}%
\pgfpathlineto{\pgfqpoint{4.665833in}{3.195527in}}%
\pgfpathlineto{\pgfqpoint{4.652360in}{3.203154in}}%
\pgfpathlineto{\pgfqpoint{4.644683in}{3.181321in}}%
\pgfpathlineto{\pgfqpoint{4.637010in}{3.159907in}}%
\pgfpathlineto{\pgfqpoint{4.629339in}{3.138905in}}%
\pgfpathlineto{\pgfqpoint{4.621671in}{3.118305in}}%
\pgfpathclose%
\pgfusepath{fill}%
\end{pgfscope}%
\begin{pgfscope}%
\pgfpathrectangle{\pgfqpoint{1.150000in}{0.150000in}}{\pgfqpoint{5.700000in}{5.700000in}}%
\pgfusepath{clip}%
\pgfsetbuttcap%
\pgfsetroundjoin%
\definecolor{currentfill}{rgb}{0.263663,0.237631,0.518762}%
\pgfsetfillcolor{currentfill}%
\pgfsetfillopacity{0.700000}%
\pgfsetlinewidth{0.000000pt}%
\definecolor{currentstroke}{rgb}{0.000000,0.000000,0.000000}%
\pgfsetstrokecolor{currentstroke}%
\pgfsetdash{}{0pt}%
\pgfpathmoveto{\pgfqpoint{4.114682in}{2.860117in}}%
\pgfpathlineto{\pgfqpoint{4.128086in}{2.853476in}}%
\pgfpathlineto{\pgfqpoint{4.141496in}{2.846918in}}%
\pgfpathlineto{\pgfqpoint{4.154909in}{2.840445in}}%
\pgfpathlineto{\pgfqpoint{4.168328in}{2.834055in}}%
\pgfpathlineto{\pgfqpoint{4.176045in}{2.849142in}}%
\pgfpathlineto{\pgfqpoint{4.183760in}{2.864473in}}%
\pgfpathlineto{\pgfqpoint{4.191472in}{2.880055in}}%
\pgfpathlineto{\pgfqpoint{4.199181in}{2.895895in}}%
\pgfpathlineto{\pgfqpoint{4.185770in}{2.902640in}}%
\pgfpathlineto{\pgfqpoint{4.172363in}{2.909469in}}%
\pgfpathlineto{\pgfqpoint{4.158960in}{2.916382in}}%
\pgfpathlineto{\pgfqpoint{4.145562in}{2.923379in}}%
\pgfpathlineto{\pgfqpoint{4.137846in}{2.907176in}}%
\pgfpathlineto{\pgfqpoint{4.130128in}{2.891236in}}%
\pgfpathlineto{\pgfqpoint{4.122406in}{2.875552in}}%
\pgfpathlineto{\pgfqpoint{4.114682in}{2.860117in}}%
\pgfpathclose%
\pgfusepath{fill}%
\end{pgfscope}%
\begin{pgfscope}%
\pgfpathrectangle{\pgfqpoint{1.150000in}{0.150000in}}{\pgfqpoint{5.700000in}{5.700000in}}%
\pgfusepath{clip}%
\pgfsetbuttcap%
\pgfsetroundjoin%
\definecolor{currentfill}{rgb}{0.260571,0.246922,0.522828}%
\pgfsetfillcolor{currentfill}%
\pgfsetfillopacity{0.700000}%
\pgfsetlinewidth{0.000000pt}%
\definecolor{currentstroke}{rgb}{0.000000,0.000000,0.000000}%
\pgfsetstrokecolor{currentstroke}%
\pgfsetdash{}{0pt}%
\pgfpathmoveto{\pgfqpoint{2.978596in}{2.900332in}}%
\pgfpathlineto{\pgfqpoint{2.991904in}{2.889397in}}%
\pgfpathlineto{\pgfqpoint{3.005210in}{2.878590in}}%
\pgfpathlineto{\pgfqpoint{3.018517in}{2.867910in}}%
\pgfpathlineto{\pgfqpoint{3.031824in}{2.857357in}}%
\pgfpathlineto{\pgfqpoint{3.039841in}{2.869917in}}%
\pgfpathlineto{\pgfqpoint{3.047851in}{2.882625in}}%
\pgfpathlineto{\pgfqpoint{3.055853in}{2.895486in}}%
\pgfpathlineto{\pgfqpoint{3.063848in}{2.908503in}}%
\pgfpathlineto{\pgfqpoint{3.050549in}{2.919232in}}%
\pgfpathlineto{\pgfqpoint{3.037250in}{2.930087in}}%
\pgfpathlineto{\pgfqpoint{3.023950in}{2.941070in}}%
\pgfpathlineto{\pgfqpoint{3.010650in}{2.952181in}}%
\pgfpathlineto{\pgfqpoint{3.002648in}{2.938981in}}%
\pgfpathlineto{\pgfqpoint{2.994638in}{2.925941in}}%
\pgfpathlineto{\pgfqpoint{2.986621in}{2.913060in}}%
\pgfpathlineto{\pgfqpoint{2.978596in}{2.900332in}}%
\pgfpathclose%
\pgfusepath{fill}%
\end{pgfscope}%
\begin{pgfscope}%
\pgfpathrectangle{\pgfqpoint{1.150000in}{0.150000in}}{\pgfqpoint{5.700000in}{5.700000in}}%
\pgfusepath{clip}%
\pgfsetbuttcap%
\pgfsetroundjoin%
\definecolor{currentfill}{rgb}{0.273006,0.204520,0.501721}%
\pgfsetfillcolor{currentfill}%
\pgfsetfillopacity{0.700000}%
\pgfsetlinewidth{0.000000pt}%
\definecolor{currentstroke}{rgb}{0.000000,0.000000,0.000000}%
\pgfsetstrokecolor{currentstroke}%
\pgfsetdash{}{0pt}%
\pgfpathmoveto{\pgfqpoint{3.308505in}{2.803765in}}%
\pgfpathlineto{\pgfqpoint{3.321811in}{2.794838in}}%
\pgfpathlineto{\pgfqpoint{3.335119in}{2.786021in}}%
\pgfpathlineto{\pgfqpoint{3.348429in}{2.777313in}}%
\pgfpathlineto{\pgfqpoint{3.361741in}{2.768712in}}%
\pgfpathlineto{\pgfqpoint{3.369668in}{2.781521in}}%
\pgfpathlineto{\pgfqpoint{3.377588in}{2.794484in}}%
\pgfpathlineto{\pgfqpoint{3.385502in}{2.807603in}}%
\pgfpathlineto{\pgfqpoint{3.393410in}{2.820885in}}%
\pgfpathlineto{\pgfqpoint{3.380105in}{2.829701in}}%
\pgfpathlineto{\pgfqpoint{3.366802in}{2.838625in}}%
\pgfpathlineto{\pgfqpoint{3.353500in}{2.847658in}}%
\pgfpathlineto{\pgfqpoint{3.340200in}{2.856801in}}%
\pgfpathlineto{\pgfqpoint{3.332286in}{2.843296in}}%
\pgfpathlineto{\pgfqpoint{3.324366in}{2.829958in}}%
\pgfpathlineto{\pgfqpoint{3.316439in}{2.816783in}}%
\pgfpathlineto{\pgfqpoint{3.308505in}{2.803765in}}%
\pgfpathclose%
\pgfusepath{fill}%
\end{pgfscope}%
\begin{pgfscope}%
\pgfpathrectangle{\pgfqpoint{1.150000in}{0.150000in}}{\pgfqpoint{5.700000in}{5.700000in}}%
\pgfusepath{clip}%
\pgfsetbuttcap%
\pgfsetroundjoin%
\definecolor{currentfill}{rgb}{0.274128,0.199721,0.498911}%
\pgfsetfillcolor{currentfill}%
\pgfsetfillopacity{0.700000}%
\pgfsetlinewidth{0.000000pt}%
\definecolor{currentstroke}{rgb}{0.000000,0.000000,0.000000}%
\pgfsetstrokecolor{currentstroke}%
\pgfsetdash{}{0pt}%
\pgfpathmoveto{\pgfqpoint{3.446651in}{2.786685in}}%
\pgfpathlineto{\pgfqpoint{3.459967in}{2.778397in}}%
\pgfpathlineto{\pgfqpoint{3.473285in}{2.770213in}}%
\pgfpathlineto{\pgfqpoint{3.486605in}{2.762132in}}%
\pgfpathlineto{\pgfqpoint{3.499929in}{2.754153in}}%
\pgfpathlineto{\pgfqpoint{3.507817in}{2.767144in}}%
\pgfpathlineto{\pgfqpoint{3.515700in}{2.780294in}}%
\pgfpathlineto{\pgfqpoint{3.523577in}{2.793609in}}%
\pgfpathlineto{\pgfqpoint{3.531448in}{2.807093in}}%
\pgfpathlineto{\pgfqpoint{3.518131in}{2.815308in}}%
\pgfpathlineto{\pgfqpoint{3.504817in}{2.823624in}}%
\pgfpathlineto{\pgfqpoint{3.491505in}{2.832044in}}%
\pgfpathlineto{\pgfqpoint{3.478196in}{2.840567in}}%
\pgfpathlineto{\pgfqpoint{3.470319in}{2.826840in}}%
\pgfpathlineto{\pgfqpoint{3.462435in}{2.813287in}}%
\pgfpathlineto{\pgfqpoint{3.454546in}{2.799903in}}%
\pgfpathlineto{\pgfqpoint{3.446651in}{2.786685in}}%
\pgfpathclose%
\pgfusepath{fill}%
\end{pgfscope}%
\begin{pgfscope}%
\pgfpathrectangle{\pgfqpoint{1.150000in}{0.150000in}}{\pgfqpoint{5.700000in}{5.700000in}}%
\pgfusepath{clip}%
\pgfsetbuttcap%
\pgfsetroundjoin%
\definecolor{currentfill}{rgb}{0.273006,0.204520,0.501721}%
\pgfsetfillcolor{currentfill}%
\pgfsetfillopacity{0.700000}%
\pgfsetlinewidth{0.000000pt}%
\definecolor{currentstroke}{rgb}{0.000000,0.000000,0.000000}%
\pgfsetstrokecolor{currentstroke}%
\pgfsetdash{}{0pt}%
\pgfpathmoveto{\pgfqpoint{3.807489in}{2.795057in}}%
\pgfpathlineto{\pgfqpoint{3.820847in}{2.787953in}}%
\pgfpathlineto{\pgfqpoint{3.834208in}{2.780941in}}%
\pgfpathlineto{\pgfqpoint{3.847574in}{2.774020in}}%
\pgfpathlineto{\pgfqpoint{3.860943in}{2.767190in}}%
\pgfpathlineto{\pgfqpoint{3.868735in}{2.780969in}}%
\pgfpathlineto{\pgfqpoint{3.876522in}{2.794942in}}%
\pgfpathlineto{\pgfqpoint{3.884305in}{2.809114in}}%
\pgfpathlineto{\pgfqpoint{3.892083in}{2.823490in}}%
\pgfpathlineto{\pgfqpoint{3.878720in}{2.830616in}}%
\pgfpathlineto{\pgfqpoint{3.865361in}{2.837832in}}%
\pgfpathlineto{\pgfqpoint{3.852005in}{2.845139in}}%
\pgfpathlineto{\pgfqpoint{3.838654in}{2.852538in}}%
\pgfpathlineto{\pgfqpoint{3.830869in}{2.837859in}}%
\pgfpathlineto{\pgfqpoint{3.823080in}{2.823390in}}%
\pgfpathlineto{\pgfqpoint{3.815287in}{2.809124in}}%
\pgfpathlineto{\pgfqpoint{3.807489in}{2.795057in}}%
\pgfpathclose%
\pgfusepath{fill}%
\end{pgfscope}%
\begin{pgfscope}%
\pgfpathrectangle{\pgfqpoint{1.150000in}{0.150000in}}{\pgfqpoint{5.700000in}{5.700000in}}%
\pgfusepath{clip}%
\pgfsetbuttcap%
\pgfsetroundjoin%
\definecolor{currentfill}{rgb}{0.203063,0.379716,0.553925}%
\pgfsetfillcolor{currentfill}%
\pgfsetfillopacity{0.700000}%
\pgfsetlinewidth{0.000000pt}%
\definecolor{currentstroke}{rgb}{0.000000,0.000000,0.000000}%
\pgfsetstrokecolor{currentstroke}%
\pgfsetdash{}{0pt}%
\pgfpathmoveto{\pgfqpoint{4.706282in}{3.173102in}}%
\pgfpathlineto{\pgfqpoint{4.719775in}{3.165779in}}%
\pgfpathlineto{\pgfqpoint{4.733273in}{3.158532in}}%
\pgfpathlineto{\pgfqpoint{4.746777in}{3.151360in}}%
\pgfpathlineto{\pgfqpoint{4.760285in}{3.144262in}}%
\pgfpathlineto{\pgfqpoint{4.767952in}{3.165544in}}%
\pgfpathlineto{\pgfqpoint{4.775623in}{3.187255in}}%
\pgfpathlineto{\pgfqpoint{4.783298in}{3.209402in}}%
\pgfpathlineto{\pgfqpoint{4.790979in}{3.231997in}}%
\pgfpathlineto{\pgfqpoint{4.777477in}{3.239593in}}%
\pgfpathlineto{\pgfqpoint{4.763980in}{3.247263in}}%
\pgfpathlineto{\pgfqpoint{4.750489in}{3.255009in}}%
\pgfpathlineto{\pgfqpoint{4.737002in}{3.262831in}}%
\pgfpathlineto{\pgfqpoint{4.729316in}{3.239729in}}%
\pgfpathlineto{\pgfqpoint{4.721634in}{3.217081in}}%
\pgfpathlineto{\pgfqpoint{4.713956in}{3.194875in}}%
\pgfpathlineto{\pgfqpoint{4.706282in}{3.173102in}}%
\pgfpathclose%
\pgfusepath{fill}%
\end{pgfscope}%
\begin{pgfscope}%
\pgfpathrectangle{\pgfqpoint{1.150000in}{0.150000in}}{\pgfqpoint{5.700000in}{5.700000in}}%
\pgfusepath{clip}%
\pgfsetbuttcap%
\pgfsetroundjoin%
\definecolor{currentfill}{rgb}{0.269308,0.218818,0.509577}%
\pgfsetfillcolor{currentfill}%
\pgfsetfillopacity{0.700000}%
\pgfsetlinewidth{0.000000pt}%
\definecolor{currentstroke}{rgb}{0.000000,0.000000,0.000000}%
\pgfsetstrokecolor{currentstroke}%
\pgfsetdash{}{0pt}%
\pgfpathmoveto{\pgfqpoint{3.170250in}{2.827095in}}%
\pgfpathlineto{\pgfqpoint{3.183553in}{2.817457in}}%
\pgfpathlineto{\pgfqpoint{3.196857in}{2.807935in}}%
\pgfpathlineto{\pgfqpoint{3.210162in}{2.798529in}}%
\pgfpathlineto{\pgfqpoint{3.223468in}{2.789238in}}%
\pgfpathlineto{\pgfqpoint{3.231435in}{2.801849in}}%
\pgfpathlineto{\pgfqpoint{3.239395in}{2.814608in}}%
\pgfpathlineto{\pgfqpoint{3.247348in}{2.827519in}}%
\pgfpathlineto{\pgfqpoint{3.255295in}{2.840585in}}%
\pgfpathlineto{\pgfqpoint{3.241996in}{2.850072in}}%
\pgfpathlineto{\pgfqpoint{3.228698in}{2.859674in}}%
\pgfpathlineto{\pgfqpoint{3.215401in}{2.869391in}}%
\pgfpathlineto{\pgfqpoint{3.202105in}{2.879225in}}%
\pgfpathlineto{\pgfqpoint{3.194151in}{2.865955in}}%
\pgfpathlineto{\pgfqpoint{3.186191in}{2.852846in}}%
\pgfpathlineto{\pgfqpoint{3.178224in}{2.839894in}}%
\pgfpathlineto{\pgfqpoint{3.170250in}{2.827095in}}%
\pgfpathclose%
\pgfusepath{fill}%
\end{pgfscope}%
\begin{pgfscope}%
\pgfpathrectangle{\pgfqpoint{1.150000in}{0.150000in}}{\pgfqpoint{5.700000in}{5.700000in}}%
\pgfusepath{clip}%
\pgfsetbuttcap%
\pgfsetroundjoin%
\definecolor{currentfill}{rgb}{0.267968,0.223549,0.512008}%
\pgfsetfillcolor{currentfill}%
\pgfsetfillopacity{0.700000}%
\pgfsetlinewidth{0.000000pt}%
\definecolor{currentstroke}{rgb}{0.000000,0.000000,0.000000}%
\pgfsetstrokecolor{currentstroke}%
\pgfsetdash{}{0pt}%
\pgfpathmoveto{\pgfqpoint{4.030150in}{2.826823in}}%
\pgfpathlineto{\pgfqpoint{4.043544in}{2.820175in}}%
\pgfpathlineto{\pgfqpoint{4.056942in}{2.813614in}}%
\pgfpathlineto{\pgfqpoint{4.070344in}{2.807137in}}%
\pgfpathlineto{\pgfqpoint{4.083751in}{2.800746in}}%
\pgfpathlineto{\pgfqpoint{4.091489in}{2.815247in}}%
\pgfpathlineto{\pgfqpoint{4.099223in}{2.829971in}}%
\pgfpathlineto{\pgfqpoint{4.106954in}{2.844926in}}%
\pgfpathlineto{\pgfqpoint{4.114682in}{2.860117in}}%
\pgfpathlineto{\pgfqpoint{4.101282in}{2.866844in}}%
\pgfpathlineto{\pgfqpoint{4.087886in}{2.873655in}}%
\pgfpathlineto{\pgfqpoint{4.074495in}{2.880552in}}%
\pgfpathlineto{\pgfqpoint{4.061108in}{2.887535in}}%
\pgfpathlineto{\pgfqpoint{4.053374in}{2.872001in}}%
\pgfpathlineto{\pgfqpoint{4.045636in}{2.856708in}}%
\pgfpathlineto{\pgfqpoint{4.037895in}{2.841651in}}%
\pgfpathlineto{\pgfqpoint{4.030150in}{2.826823in}}%
\pgfpathclose%
\pgfusepath{fill}%
\end{pgfscope}%
\begin{pgfscope}%
\pgfpathrectangle{\pgfqpoint{1.150000in}{0.150000in}}{\pgfqpoint{5.700000in}{5.700000in}}%
\pgfusepath{clip}%
\pgfsetbuttcap%
\pgfsetroundjoin%
\definecolor{currentfill}{rgb}{0.275191,0.194905,0.496005}%
\pgfsetfillcolor{currentfill}%
\pgfsetfillopacity{0.700000}%
\pgfsetlinewidth{0.000000pt}%
\definecolor{currentstroke}{rgb}{0.000000,0.000000,0.000000}%
\pgfsetstrokecolor{currentstroke}%
\pgfsetdash{}{0pt}%
\pgfpathmoveto{\pgfqpoint{3.584742in}{2.775244in}}%
\pgfpathlineto{\pgfqpoint{3.598072in}{2.767530in}}%
\pgfpathlineto{\pgfqpoint{3.611406in}{2.759915in}}%
\pgfpathlineto{\pgfqpoint{3.624742in}{2.752397in}}%
\pgfpathlineto{\pgfqpoint{3.638082in}{2.744977in}}%
\pgfpathlineto{\pgfqpoint{3.645935in}{2.758138in}}%
\pgfpathlineto{\pgfqpoint{3.653782in}{2.771467in}}%
\pgfpathlineto{\pgfqpoint{3.661623in}{2.784969in}}%
\pgfpathlineto{\pgfqpoint{3.669459in}{2.798649in}}%
\pgfpathlineto{\pgfqpoint{3.656126in}{2.806325in}}%
\pgfpathlineto{\pgfqpoint{3.642796in}{2.814098in}}%
\pgfpathlineto{\pgfqpoint{3.629469in}{2.821969in}}%
\pgfpathlineto{\pgfqpoint{3.616145in}{2.829938in}}%
\pgfpathlineto{\pgfqpoint{3.608302in}{2.815995in}}%
\pgfpathlineto{\pgfqpoint{3.600454in}{2.802235in}}%
\pgfpathlineto{\pgfqpoint{3.592601in}{2.788653in}}%
\pgfpathlineto{\pgfqpoint{3.584742in}{2.775244in}}%
\pgfpathclose%
\pgfusepath{fill}%
\end{pgfscope}%
\begin{pgfscope}%
\pgfpathrectangle{\pgfqpoint{1.150000in}{0.150000in}}{\pgfqpoint{5.700000in}{5.700000in}}%
\pgfusepath{clip}%
\pgfsetbuttcap%
\pgfsetroundjoin%
\definecolor{currentfill}{rgb}{0.160665,0.478540,0.558115}%
\pgfsetfillcolor{currentfill}%
\pgfsetfillopacity{0.700000}%
\pgfsetlinewidth{0.000000pt}%
\definecolor{currentstroke}{rgb}{0.000000,0.000000,0.000000}%
\pgfsetstrokecolor{currentstroke}%
\pgfsetdash{}{0pt}%
\pgfpathmoveto{\pgfqpoint{4.852627in}{3.430031in}}%
\pgfpathlineto{\pgfqpoint{4.866123in}{3.421449in}}%
\pgfpathlineto{\pgfqpoint{4.879623in}{3.412942in}}%
\pgfpathlineto{\pgfqpoint{4.893128in}{3.404509in}}%
\pgfpathlineto{\pgfqpoint{4.906637in}{3.396150in}}%
\pgfpathlineto{\pgfqpoint{4.914370in}{3.422662in}}%
\pgfpathlineto{\pgfqpoint{4.922111in}{3.449717in}}%
\pgfpathlineto{\pgfqpoint{4.929861in}{3.477327in}}%
\pgfpathlineto{\pgfqpoint{4.937621in}{3.505503in}}%
\pgfpathlineto{\pgfqpoint{4.924116in}{3.514426in}}%
\pgfpathlineto{\pgfqpoint{4.910615in}{3.523422in}}%
\pgfpathlineto{\pgfqpoint{4.897119in}{3.532494in}}%
\pgfpathlineto{\pgfqpoint{4.883627in}{3.541641in}}%
\pgfpathlineto{\pgfqpoint{4.875864in}{3.512892in}}%
\pgfpathlineto{\pgfqpoint{4.868110in}{3.484715in}}%
\pgfpathlineto{\pgfqpoint{4.860365in}{3.457099in}}%
\pgfpathlineto{\pgfqpoint{4.852627in}{3.430031in}}%
\pgfpathclose%
\pgfusepath{fill}%
\end{pgfscope}%
\begin{pgfscope}%
\pgfpathrectangle{\pgfqpoint{1.150000in}{0.150000in}}{\pgfqpoint{5.700000in}{5.700000in}}%
\pgfusepath{clip}%
\pgfsetbuttcap%
\pgfsetroundjoin%
\definecolor{currentfill}{rgb}{0.175841,0.441290,0.557685}%
\pgfsetfillcolor{currentfill}%
\pgfsetfillopacity{0.700000}%
\pgfsetlinewidth{0.000000pt}%
\definecolor{currentstroke}{rgb}{0.000000,0.000000,0.000000}%
\pgfsetstrokecolor{currentstroke}%
\pgfsetdash{}{0pt}%
\pgfpathmoveto{\pgfqpoint{4.821752in}{3.327041in}}%
\pgfpathlineto{\pgfqpoint{4.835253in}{3.319000in}}%
\pgfpathlineto{\pgfqpoint{4.848758in}{3.311035in}}%
\pgfpathlineto{\pgfqpoint{4.862268in}{3.303143in}}%
\pgfpathlineto{\pgfqpoint{4.875783in}{3.295326in}}%
\pgfpathlineto{\pgfqpoint{4.883486in}{3.319770in}}%
\pgfpathlineto{\pgfqpoint{4.891196in}{3.344715in}}%
\pgfpathlineto{\pgfqpoint{4.898913in}{3.370171in}}%
\pgfpathlineto{\pgfqpoint{4.906637in}{3.396150in}}%
\pgfpathlineto{\pgfqpoint{4.893128in}{3.404509in}}%
\pgfpathlineto{\pgfqpoint{4.879623in}{3.412942in}}%
\pgfpathlineto{\pgfqpoint{4.866123in}{3.421449in}}%
\pgfpathlineto{\pgfqpoint{4.852627in}{3.430031in}}%
\pgfpathlineto{\pgfqpoint{4.844898in}{3.403503in}}%
\pgfpathlineto{\pgfqpoint{4.837176in}{3.377502in}}%
\pgfpathlineto{\pgfqpoint{4.829461in}{3.352018in}}%
\pgfpathlineto{\pgfqpoint{4.821752in}{3.327041in}}%
\pgfpathclose%
\pgfusepath{fill}%
\end{pgfscope}%
\begin{pgfscope}%
\pgfpathrectangle{\pgfqpoint{1.150000in}{0.150000in}}{\pgfqpoint{5.700000in}{5.700000in}}%
\pgfusepath{clip}%
\pgfsetbuttcap%
\pgfsetroundjoin%
\definecolor{currentfill}{rgb}{0.246811,0.283237,0.535941}%
\pgfsetfillcolor{currentfill}%
\pgfsetfillopacity{0.700000}%
\pgfsetlinewidth{0.000000pt}%
\definecolor{currentstroke}{rgb}{0.000000,0.000000,0.000000}%
\pgfsetstrokecolor{currentstroke}%
\pgfsetdash{}{0pt}%
\pgfpathmoveto{\pgfqpoint{4.421925in}{2.948897in}}%
\pgfpathlineto{\pgfqpoint{4.435386in}{2.942434in}}%
\pgfpathlineto{\pgfqpoint{4.448852in}{2.936050in}}%
\pgfpathlineto{\pgfqpoint{4.462322in}{2.929744in}}%
\pgfpathlineto{\pgfqpoint{4.475798in}{2.923516in}}%
\pgfpathlineto{\pgfqpoint{4.483464in}{2.940354in}}%
\pgfpathlineto{\pgfqpoint{4.491129in}{2.957501in}}%
\pgfpathlineto{\pgfqpoint{4.498794in}{2.974963in}}%
\pgfpathlineto{\pgfqpoint{4.506458in}{2.992750in}}%
\pgfpathlineto{\pgfqpoint{4.492990in}{2.999394in}}%
\pgfpathlineto{\pgfqpoint{4.479527in}{3.006115in}}%
\pgfpathlineto{\pgfqpoint{4.466069in}{3.012915in}}%
\pgfpathlineto{\pgfqpoint{4.452615in}{3.019794in}}%
\pgfpathlineto{\pgfqpoint{4.444943in}{3.001583in}}%
\pgfpathlineto{\pgfqpoint{4.437271in}{2.983702in}}%
\pgfpathlineto{\pgfqpoint{4.429599in}{2.966143in}}%
\pgfpathlineto{\pgfqpoint{4.421925in}{2.948897in}}%
\pgfpathclose%
\pgfusepath{fill}%
\end{pgfscope}%
\begin{pgfscope}%
\pgfpathrectangle{\pgfqpoint{1.150000in}{0.150000in}}{\pgfqpoint{5.700000in}{5.700000in}}%
\pgfusepath{clip}%
\pgfsetbuttcap%
\pgfsetroundjoin%
\definecolor{currentfill}{rgb}{0.239346,0.300855,0.540844}%
\pgfsetfillcolor{currentfill}%
\pgfsetfillopacity{0.700000}%
\pgfsetlinewidth{0.000000pt}%
\definecolor{currentstroke}{rgb}{0.000000,0.000000,0.000000}%
\pgfsetstrokecolor{currentstroke}%
\pgfsetdash{}{0pt}%
\pgfpathmoveto{\pgfqpoint{4.506458in}{2.992750in}}%
\pgfpathlineto{\pgfqpoint{4.519932in}{2.986185in}}%
\pgfpathlineto{\pgfqpoint{4.533411in}{2.979696in}}%
\pgfpathlineto{\pgfqpoint{4.546894in}{2.973286in}}%
\pgfpathlineto{\pgfqpoint{4.560384in}{2.966952in}}%
\pgfpathlineto{\pgfqpoint{4.568041in}{2.984642in}}%
\pgfpathlineto{\pgfqpoint{4.575699in}{3.002667in}}%
\pgfpathlineto{\pgfqpoint{4.583357in}{3.021035in}}%
\pgfpathlineto{\pgfqpoint{4.591017in}{3.039754in}}%
\pgfpathlineto{\pgfqpoint{4.577536in}{3.046524in}}%
\pgfpathlineto{\pgfqpoint{4.564059in}{3.053370in}}%
\pgfpathlineto{\pgfqpoint{4.550588in}{3.060295in}}%
\pgfpathlineto{\pgfqpoint{4.537122in}{3.067296in}}%
\pgfpathlineto{\pgfqpoint{4.529455in}{3.048134in}}%
\pgfpathlineto{\pgfqpoint{4.521789in}{3.029327in}}%
\pgfpathlineto{\pgfqpoint{4.514123in}{3.010869in}}%
\pgfpathlineto{\pgfqpoint{4.506458in}{2.992750in}}%
\pgfpathclose%
\pgfusepath{fill}%
\end{pgfscope}%
\begin{pgfscope}%
\pgfpathrectangle{\pgfqpoint{1.150000in}{0.150000in}}{\pgfqpoint{5.700000in}{5.700000in}}%
\pgfusepath{clip}%
\pgfsetbuttcap%
\pgfsetroundjoin%
\definecolor{currentfill}{rgb}{0.144759,0.519093,0.556572}%
\pgfsetfillcolor{currentfill}%
\pgfsetfillopacity{0.700000}%
\pgfsetlinewidth{0.000000pt}%
\definecolor{currentstroke}{rgb}{0.000000,0.000000,0.000000}%
\pgfsetstrokecolor{currentstroke}%
\pgfsetdash{}{0pt}%
\pgfpathmoveto{\pgfqpoint{4.883627in}{3.541641in}}%
\pgfpathlineto{\pgfqpoint{4.897119in}{3.532494in}}%
\pgfpathlineto{\pgfqpoint{4.910615in}{3.523422in}}%
\pgfpathlineto{\pgfqpoint{4.924116in}{3.514426in}}%
\pgfpathlineto{\pgfqpoint{4.937621in}{3.505503in}}%
\pgfpathlineto{\pgfqpoint{4.945390in}{3.534256in}}%
\pgfpathlineto{\pgfqpoint{4.953169in}{3.563597in}}%
\pgfpathlineto{\pgfqpoint{4.960959in}{3.593538in}}%
\pgfpathlineto{\pgfqpoint{4.947456in}{3.602898in}}%
\pgfpathlineto{\pgfqpoint{4.933958in}{3.612334in}}%
\pgfpathlineto{\pgfqpoint{4.920464in}{3.621844in}}%
\pgfpathlineto{\pgfqpoint{4.906974in}{3.631429in}}%
\pgfpathlineto{\pgfqpoint{4.899182in}{3.600898in}}%
\pgfpathlineto{\pgfqpoint{4.891399in}{3.570972in}}%
\pgfpathlineto{\pgfqpoint{4.883627in}{3.541641in}}%
\pgfpathclose%
\pgfusepath{fill}%
\end{pgfscope}%
\begin{pgfscope}%
\pgfpathrectangle{\pgfqpoint{1.150000in}{0.150000in}}{\pgfqpoint{5.700000in}{5.700000in}}%
\pgfusepath{clip}%
\pgfsetbuttcap%
\pgfsetroundjoin%
\definecolor{currentfill}{rgb}{0.265145,0.232956,0.516599}%
\pgfsetfillcolor{currentfill}%
\pgfsetfillopacity{0.700000}%
\pgfsetlinewidth{0.000000pt}%
\definecolor{currentstroke}{rgb}{0.000000,0.000000,0.000000}%
\pgfsetstrokecolor{currentstroke}%
\pgfsetdash{}{0pt}%
\pgfpathmoveto{\pgfqpoint{3.031824in}{2.857357in}}%
\pgfpathlineto{\pgfqpoint{3.045130in}{2.846929in}}%
\pgfpathlineto{\pgfqpoint{3.058437in}{2.836626in}}%
\pgfpathlineto{\pgfqpoint{3.071743in}{2.826445in}}%
\pgfpathlineto{\pgfqpoint{3.085051in}{2.816387in}}%
\pgfpathlineto{\pgfqpoint{3.093060in}{2.828778in}}%
\pgfpathlineto{\pgfqpoint{3.101062in}{2.841313in}}%
\pgfpathlineto{\pgfqpoint{3.109058in}{2.853996in}}%
\pgfpathlineto{\pgfqpoint{3.117046in}{2.866829in}}%
\pgfpathlineto{\pgfqpoint{3.103746in}{2.877063in}}%
\pgfpathlineto{\pgfqpoint{3.090447in}{2.887419in}}%
\pgfpathlineto{\pgfqpoint{3.077147in}{2.897899in}}%
\pgfpathlineto{\pgfqpoint{3.063848in}{2.908503in}}%
\pgfpathlineto{\pgfqpoint{3.055853in}{2.895486in}}%
\pgfpathlineto{\pgfqpoint{3.047851in}{2.882625in}}%
\pgfpathlineto{\pgfqpoint{3.039841in}{2.869917in}}%
\pgfpathlineto{\pgfqpoint{3.031824in}{2.857357in}}%
\pgfpathclose%
\pgfusepath{fill}%
\end{pgfscope}%
\begin{pgfscope}%
\pgfpathrectangle{\pgfqpoint{1.150000in}{0.150000in}}{\pgfqpoint{5.700000in}{5.700000in}}%
\pgfusepath{clip}%
\pgfsetbuttcap%
\pgfsetroundjoin%
\definecolor{currentfill}{rgb}{0.255645,0.260703,0.528312}%
\pgfsetfillcolor{currentfill}%
\pgfsetfillopacity{0.700000}%
\pgfsetlinewidth{0.000000pt}%
\definecolor{currentstroke}{rgb}{0.000000,0.000000,0.000000}%
\pgfsetstrokecolor{currentstroke}%
\pgfsetdash{}{0pt}%
\pgfpathmoveto{\pgfqpoint{4.337402in}{2.907962in}}%
\pgfpathlineto{\pgfqpoint{4.350850in}{2.901577in}}%
\pgfpathlineto{\pgfqpoint{4.364303in}{2.895272in}}%
\pgfpathlineto{\pgfqpoint{4.377760in}{2.889046in}}%
\pgfpathlineto{\pgfqpoint{4.391223in}{2.882900in}}%
\pgfpathlineto{\pgfqpoint{4.398901in}{2.898965in}}%
\pgfpathlineto{\pgfqpoint{4.406577in}{2.915315in}}%
\pgfpathlineto{\pgfqpoint{4.414252in}{2.931957in}}%
\pgfpathlineto{\pgfqpoint{4.421925in}{2.948897in}}%
\pgfpathlineto{\pgfqpoint{4.408470in}{2.955438in}}%
\pgfpathlineto{\pgfqpoint{4.395020in}{2.962060in}}%
\pgfpathlineto{\pgfqpoint{4.381574in}{2.968761in}}%
\pgfpathlineto{\pgfqpoint{4.368134in}{2.975542in}}%
\pgfpathlineto{\pgfqpoint{4.360453in}{2.958198in}}%
\pgfpathlineto{\pgfqpoint{4.352771in}{2.941159in}}%
\pgfpathlineto{\pgfqpoint{4.345087in}{2.924416in}}%
\pgfpathlineto{\pgfqpoint{4.337402in}{2.907962in}}%
\pgfpathclose%
\pgfusepath{fill}%
\end{pgfscope}%
\begin{pgfscope}%
\pgfpathrectangle{\pgfqpoint{1.150000in}{0.150000in}}{\pgfqpoint{5.700000in}{5.700000in}}%
\pgfusepath{clip}%
\pgfsetbuttcap%
\pgfsetroundjoin%
\definecolor{currentfill}{rgb}{0.229739,0.322361,0.545706}%
\pgfsetfillcolor{currentfill}%
\pgfsetfillopacity{0.700000}%
\pgfsetlinewidth{0.000000pt}%
\definecolor{currentstroke}{rgb}{0.000000,0.000000,0.000000}%
\pgfsetstrokecolor{currentstroke}%
\pgfsetdash{}{0pt}%
\pgfpathmoveto{\pgfqpoint{4.591017in}{3.039754in}}%
\pgfpathlineto{\pgfqpoint{4.604504in}{3.033060in}}%
\pgfpathlineto{\pgfqpoint{4.617996in}{3.026444in}}%
\pgfpathlineto{\pgfqpoint{4.631493in}{3.019903in}}%
\pgfpathlineto{\pgfqpoint{4.644996in}{3.013438in}}%
\pgfpathlineto{\pgfqpoint{4.652649in}{3.032067in}}%
\pgfpathlineto{\pgfqpoint{4.660304in}{3.051059in}}%
\pgfpathlineto{\pgfqpoint{4.667960in}{3.070421in}}%
\pgfpathlineto{\pgfqpoint{4.675619in}{3.090163in}}%
\pgfpathlineto{\pgfqpoint{4.662125in}{3.097085in}}%
\pgfpathlineto{\pgfqpoint{4.648635in}{3.104082in}}%
\pgfpathlineto{\pgfqpoint{4.635150in}{3.111155in}}%
\pgfpathlineto{\pgfqpoint{4.621671in}{3.118305in}}%
\pgfpathlineto{\pgfqpoint{4.614005in}{3.098099in}}%
\pgfpathlineto{\pgfqpoint{4.606340in}{3.078277in}}%
\pgfpathlineto{\pgfqpoint{4.598678in}{3.058832in}}%
\pgfpathlineto{\pgfqpoint{4.591017in}{3.039754in}}%
\pgfpathclose%
\pgfusepath{fill}%
\end{pgfscope}%
\begin{pgfscope}%
\pgfpathrectangle{\pgfqpoint{1.150000in}{0.150000in}}{\pgfqpoint{5.700000in}{5.700000in}}%
\pgfusepath{clip}%
\pgfsetbuttcap%
\pgfsetroundjoin%
\definecolor{currentfill}{rgb}{0.192357,0.403199,0.555836}%
\pgfsetfillcolor{currentfill}%
\pgfsetfillopacity{0.700000}%
\pgfsetlinewidth{0.000000pt}%
\definecolor{currentstroke}{rgb}{0.000000,0.000000,0.000000}%
\pgfsetstrokecolor{currentstroke}%
\pgfsetdash{}{0pt}%
\pgfpathmoveto{\pgfqpoint{4.790979in}{3.231997in}}%
\pgfpathlineto{\pgfqpoint{4.804485in}{3.224476in}}%
\pgfpathlineto{\pgfqpoint{4.817997in}{3.217030in}}%
\pgfpathlineto{\pgfqpoint{4.831513in}{3.209658in}}%
\pgfpathlineto{\pgfqpoint{4.845035in}{3.202360in}}%
\pgfpathlineto{\pgfqpoint{4.852714in}{3.224900in}}%
\pgfpathlineto{\pgfqpoint{4.860398in}{3.247901in}}%
\pgfpathlineto{\pgfqpoint{4.868087in}{3.271373in}}%
\pgfpathlineto{\pgfqpoint{4.875783in}{3.295326in}}%
\pgfpathlineto{\pgfqpoint{4.862268in}{3.303143in}}%
\pgfpathlineto{\pgfqpoint{4.848758in}{3.311035in}}%
\pgfpathlineto{\pgfqpoint{4.835253in}{3.319000in}}%
\pgfpathlineto{\pgfqpoint{4.821752in}{3.327041in}}%
\pgfpathlineto{\pgfqpoint{4.814050in}{3.302560in}}%
\pgfpathlineto{\pgfqpoint{4.806354in}{3.278566in}}%
\pgfpathlineto{\pgfqpoint{4.798664in}{3.255048in}}%
\pgfpathlineto{\pgfqpoint{4.790979in}{3.231997in}}%
\pgfpathclose%
\pgfusepath{fill}%
\end{pgfscope}%
\begin{pgfscope}%
\pgfpathrectangle{\pgfqpoint{1.150000in}{0.150000in}}{\pgfqpoint{5.700000in}{5.700000in}}%
\pgfusepath{clip}%
\pgfsetbuttcap%
\pgfsetroundjoin%
\definecolor{currentfill}{rgb}{0.275191,0.194905,0.496005}%
\pgfsetfillcolor{currentfill}%
\pgfsetfillopacity{0.700000}%
\pgfsetlinewidth{0.000000pt}%
\definecolor{currentstroke}{rgb}{0.000000,0.000000,0.000000}%
\pgfsetstrokecolor{currentstroke}%
\pgfsetdash{}{0pt}%
\pgfpathmoveto{\pgfqpoint{3.722825in}{2.768904in}}%
\pgfpathlineto{\pgfqpoint{3.736175in}{2.761704in}}%
\pgfpathlineto{\pgfqpoint{3.749529in}{2.754599in}}%
\pgfpathlineto{\pgfqpoint{3.762886in}{2.747586in}}%
\pgfpathlineto{\pgfqpoint{3.776247in}{2.740666in}}%
\pgfpathlineto{\pgfqpoint{3.784065in}{2.753992in}}%
\pgfpathlineto{\pgfqpoint{3.791878in}{2.767496in}}%
\pgfpathlineto{\pgfqpoint{3.799685in}{2.781183in}}%
\pgfpathlineto{\pgfqpoint{3.807489in}{2.795057in}}%
\pgfpathlineto{\pgfqpoint{3.794134in}{2.802252in}}%
\pgfpathlineto{\pgfqpoint{3.780784in}{2.809541in}}%
\pgfpathlineto{\pgfqpoint{3.767437in}{2.816922in}}%
\pgfpathlineto{\pgfqpoint{3.754093in}{2.824397in}}%
\pgfpathlineto{\pgfqpoint{3.746284in}{2.810239in}}%
\pgfpathlineto{\pgfqpoint{3.738469in}{2.796275in}}%
\pgfpathlineto{\pgfqpoint{3.730650in}{2.782498in}}%
\pgfpathlineto{\pgfqpoint{3.722825in}{2.768904in}}%
\pgfpathclose%
\pgfusepath{fill}%
\end{pgfscope}%
\begin{pgfscope}%
\pgfpathrectangle{\pgfqpoint{1.150000in}{0.150000in}}{\pgfqpoint{5.700000in}{5.700000in}}%
\pgfusepath{clip}%
\pgfsetbuttcap%
\pgfsetroundjoin%
\definecolor{currentfill}{rgb}{0.252194,0.269783,0.531579}%
\pgfsetfillcolor{currentfill}%
\pgfsetfillopacity{0.700000}%
\pgfsetlinewidth{0.000000pt}%
\definecolor{currentstroke}{rgb}{0.000000,0.000000,0.000000}%
\pgfsetstrokecolor{currentstroke}%
\pgfsetdash{}{0pt}%
\pgfpathmoveto{\pgfqpoint{2.839872in}{2.941898in}}%
\pgfpathlineto{\pgfqpoint{2.853195in}{2.930049in}}%
\pgfpathlineto{\pgfqpoint{2.866516in}{2.918337in}}%
\pgfpathlineto{\pgfqpoint{2.879836in}{2.906762in}}%
\pgfpathlineto{\pgfqpoint{2.893155in}{2.895323in}}%
\pgfpathlineto{\pgfqpoint{2.901218in}{2.907615in}}%
\pgfpathlineto{\pgfqpoint{2.909274in}{2.920053in}}%
\pgfpathlineto{\pgfqpoint{2.917321in}{2.932641in}}%
\pgfpathlineto{\pgfqpoint{2.925362in}{2.945381in}}%
\pgfpathlineto{\pgfqpoint{2.912051in}{2.956976in}}%
\pgfpathlineto{\pgfqpoint{2.898739in}{2.968707in}}%
\pgfpathlineto{\pgfqpoint{2.885426in}{2.980574in}}%
\pgfpathlineto{\pgfqpoint{2.872111in}{2.992579in}}%
\pgfpathlineto{\pgfqpoint{2.864063in}{2.979676in}}%
\pgfpathlineto{\pgfqpoint{2.856008in}{2.966929in}}%
\pgfpathlineto{\pgfqpoint{2.847944in}{2.954338in}}%
\pgfpathlineto{\pgfqpoint{2.839872in}{2.941898in}}%
\pgfpathclose%
\pgfusepath{fill}%
\end{pgfscope}%
\begin{pgfscope}%
\pgfpathrectangle{\pgfqpoint{1.150000in}{0.150000in}}{\pgfqpoint{5.700000in}{5.700000in}}%
\pgfusepath{clip}%
\pgfsetbuttcap%
\pgfsetroundjoin%
\definecolor{currentfill}{rgb}{0.271828,0.209303,0.504434}%
\pgfsetfillcolor{currentfill}%
\pgfsetfillopacity{0.700000}%
\pgfsetlinewidth{0.000000pt}%
\definecolor{currentstroke}{rgb}{0.000000,0.000000,0.000000}%
\pgfsetstrokecolor{currentstroke}%
\pgfsetdash{}{0pt}%
\pgfpathmoveto{\pgfqpoint{3.945574in}{2.795882in}}%
\pgfpathlineto{\pgfqpoint{3.958957in}{2.789202in}}%
\pgfpathlineto{\pgfqpoint{3.972345in}{2.782609in}}%
\pgfpathlineto{\pgfqpoint{3.985737in}{2.776103in}}%
\pgfpathlineto{\pgfqpoint{3.999133in}{2.769685in}}%
\pgfpathlineto{\pgfqpoint{4.006893in}{2.783655in}}%
\pgfpathlineto{\pgfqpoint{4.014649in}{2.797831in}}%
\pgfpathlineto{\pgfqpoint{4.022402in}{2.812218in}}%
\pgfpathlineto{\pgfqpoint{4.030150in}{2.826823in}}%
\pgfpathlineto{\pgfqpoint{4.016761in}{2.833557in}}%
\pgfpathlineto{\pgfqpoint{4.003376in}{2.840378in}}%
\pgfpathlineto{\pgfqpoint{3.989995in}{2.847286in}}%
\pgfpathlineto{\pgfqpoint{3.976619in}{2.854282in}}%
\pgfpathlineto{\pgfqpoint{3.968864in}{2.839354in}}%
\pgfpathlineto{\pgfqpoint{3.961105in}{2.824649in}}%
\pgfpathlineto{\pgfqpoint{3.953341in}{2.810160in}}%
\pgfpathlineto{\pgfqpoint{3.945574in}{2.795882in}}%
\pgfpathclose%
\pgfusepath{fill}%
\end{pgfscope}%
\begin{pgfscope}%
\pgfpathrectangle{\pgfqpoint{1.150000in}{0.150000in}}{\pgfqpoint{5.700000in}{5.700000in}}%
\pgfusepath{clip}%
\pgfsetbuttcap%
\pgfsetroundjoin%
\definecolor{currentfill}{rgb}{0.260571,0.246922,0.522828}%
\pgfsetfillcolor{currentfill}%
\pgfsetfillopacity{0.700000}%
\pgfsetlinewidth{0.000000pt}%
\definecolor{currentstroke}{rgb}{0.000000,0.000000,0.000000}%
\pgfsetstrokecolor{currentstroke}%
\pgfsetdash{}{0pt}%
\pgfpathmoveto{\pgfqpoint{4.252874in}{2.869742in}}%
\pgfpathlineto{\pgfqpoint{4.266309in}{2.863409in}}%
\pgfpathlineto{\pgfqpoint{4.279750in}{2.857157in}}%
\pgfpathlineto{\pgfqpoint{4.293195in}{2.850986in}}%
\pgfpathlineto{\pgfqpoint{4.306645in}{2.844896in}}%
\pgfpathlineto{\pgfqpoint{4.314337in}{2.860264in}}%
\pgfpathlineto{\pgfqpoint{4.322027in}{2.875893in}}%
\pgfpathlineto{\pgfqpoint{4.329716in}{2.891790in}}%
\pgfpathlineto{\pgfqpoint{4.337402in}{2.907962in}}%
\pgfpathlineto{\pgfqpoint{4.323960in}{2.914428in}}%
\pgfpathlineto{\pgfqpoint{4.310522in}{2.920974in}}%
\pgfpathlineto{\pgfqpoint{4.297089in}{2.927601in}}%
\pgfpathlineto{\pgfqpoint{4.283661in}{2.934310in}}%
\pgfpathlineto{\pgfqpoint{4.275967in}{2.917755in}}%
\pgfpathlineto{\pgfqpoint{4.268272in}{2.901480in}}%
\pgfpathlineto{\pgfqpoint{4.260574in}{2.885478in}}%
\pgfpathlineto{\pgfqpoint{4.252874in}{2.869742in}}%
\pgfpathclose%
\pgfusepath{fill}%
\end{pgfscope}%
\begin{pgfscope}%
\pgfpathrectangle{\pgfqpoint{1.150000in}{0.150000in}}{\pgfqpoint{5.700000in}{5.700000in}}%
\pgfusepath{clip}%
\pgfsetbuttcap%
\pgfsetroundjoin%
\definecolor{currentfill}{rgb}{0.218130,0.347432,0.550038}%
\pgfsetfillcolor{currentfill}%
\pgfsetfillopacity{0.700000}%
\pgfsetlinewidth{0.000000pt}%
\definecolor{currentstroke}{rgb}{0.000000,0.000000,0.000000}%
\pgfsetstrokecolor{currentstroke}%
\pgfsetdash{}{0pt}%
\pgfpathmoveto{\pgfqpoint{4.675619in}{3.090163in}}%
\pgfpathlineto{\pgfqpoint{4.689119in}{3.083318in}}%
\pgfpathlineto{\pgfqpoint{4.702625in}{3.076547in}}%
\pgfpathlineto{\pgfqpoint{4.716136in}{3.069852in}}%
\pgfpathlineto{\pgfqpoint{4.729652in}{3.063232in}}%
\pgfpathlineto{\pgfqpoint{4.737305in}{3.082893in}}%
\pgfpathlineto{\pgfqpoint{4.744962in}{3.102946in}}%
\pgfpathlineto{\pgfqpoint{4.752622in}{3.123399in}}%
\pgfpathlineto{\pgfqpoint{4.760285in}{3.144262in}}%
\pgfpathlineto{\pgfqpoint{4.746777in}{3.151360in}}%
\pgfpathlineto{\pgfqpoint{4.733273in}{3.158532in}}%
\pgfpathlineto{\pgfqpoint{4.719775in}{3.165779in}}%
\pgfpathlineto{\pgfqpoint{4.706282in}{3.173102in}}%
\pgfpathlineto{\pgfqpoint{4.698612in}{3.151754in}}%
\pgfpathlineto{\pgfqpoint{4.690945in}{3.130821in}}%
\pgfpathlineto{\pgfqpoint{4.683281in}{3.110294in}}%
\pgfpathlineto{\pgfqpoint{4.675619in}{3.090163in}}%
\pgfpathclose%
\pgfusepath{fill}%
\end{pgfscope}%
\begin{pgfscope}%
\pgfpathrectangle{\pgfqpoint{1.150000in}{0.150000in}}{\pgfqpoint{5.700000in}{5.700000in}}%
\pgfusepath{clip}%
\pgfsetbuttcap%
\pgfsetroundjoin%
\definecolor{currentfill}{rgb}{0.275191,0.194905,0.496005}%
\pgfsetfillcolor{currentfill}%
\pgfsetfillopacity{0.700000}%
\pgfsetlinewidth{0.000000pt}%
\definecolor{currentstroke}{rgb}{0.000000,0.000000,0.000000}%
\pgfsetstrokecolor{currentstroke}%
\pgfsetdash{}{0pt}%
\pgfpathmoveto{\pgfqpoint{3.361741in}{2.768712in}}%
\pgfpathlineto{\pgfqpoint{3.375055in}{2.760219in}}%
\pgfpathlineto{\pgfqpoint{3.388370in}{2.751832in}}%
\pgfpathlineto{\pgfqpoint{3.401688in}{2.743550in}}%
\pgfpathlineto{\pgfqpoint{3.415009in}{2.735374in}}%
\pgfpathlineto{\pgfqpoint{3.422928in}{2.747975in}}%
\pgfpathlineto{\pgfqpoint{3.430842in}{2.760725in}}%
\pgfpathlineto{\pgfqpoint{3.438749in}{2.773626in}}%
\pgfpathlineto{\pgfqpoint{3.446651in}{2.786685in}}%
\pgfpathlineto{\pgfqpoint{3.433337in}{2.795076in}}%
\pgfpathlineto{\pgfqpoint{3.420026in}{2.803573in}}%
\pgfpathlineto{\pgfqpoint{3.406717in}{2.812176in}}%
\pgfpathlineto{\pgfqpoint{3.393410in}{2.820885in}}%
\pgfpathlineto{\pgfqpoint{3.385502in}{2.807603in}}%
\pgfpathlineto{\pgfqpoint{3.377588in}{2.794484in}}%
\pgfpathlineto{\pgfqpoint{3.369668in}{2.781521in}}%
\pgfpathlineto{\pgfqpoint{3.361741in}{2.768712in}}%
\pgfpathclose%
\pgfusepath{fill}%
\end{pgfscope}%
\begin{pgfscope}%
\pgfpathrectangle{\pgfqpoint{1.150000in}{0.150000in}}{\pgfqpoint{5.700000in}{5.700000in}}%
\pgfusepath{clip}%
\pgfsetbuttcap%
\pgfsetroundjoin%
\definecolor{currentfill}{rgb}{0.273006,0.204520,0.501721}%
\pgfsetfillcolor{currentfill}%
\pgfsetfillopacity{0.700000}%
\pgfsetlinewidth{0.000000pt}%
\definecolor{currentstroke}{rgb}{0.000000,0.000000,0.000000}%
\pgfsetstrokecolor{currentstroke}%
\pgfsetdash{}{0pt}%
\pgfpathmoveto{\pgfqpoint{3.223468in}{2.789238in}}%
\pgfpathlineto{\pgfqpoint{3.236775in}{2.780060in}}%
\pgfpathlineto{\pgfqpoint{3.250084in}{2.770995in}}%
\pgfpathlineto{\pgfqpoint{3.263394in}{2.762042in}}%
\pgfpathlineto{\pgfqpoint{3.276706in}{2.753201in}}%
\pgfpathlineto{\pgfqpoint{3.284665in}{2.765624in}}%
\pgfpathlineto{\pgfqpoint{3.292618in}{2.778190in}}%
\pgfpathlineto{\pgfqpoint{3.300565in}{2.790902in}}%
\pgfpathlineto{\pgfqpoint{3.308505in}{2.803765in}}%
\pgfpathlineto{\pgfqpoint{3.295200in}{2.812803in}}%
\pgfpathlineto{\pgfqpoint{3.281897in}{2.821951in}}%
\pgfpathlineto{\pgfqpoint{3.268595in}{2.831212in}}%
\pgfpathlineto{\pgfqpoint{3.255295in}{2.840585in}}%
\pgfpathlineto{\pgfqpoint{3.247348in}{2.827519in}}%
\pgfpathlineto{\pgfqpoint{3.239395in}{2.814608in}}%
\pgfpathlineto{\pgfqpoint{3.231435in}{2.801849in}}%
\pgfpathlineto{\pgfqpoint{3.223468in}{2.789238in}}%
\pgfpathclose%
\pgfusepath{fill}%
\end{pgfscope}%
\begin{pgfscope}%
\pgfpathrectangle{\pgfqpoint{1.150000in}{0.150000in}}{\pgfqpoint{5.700000in}{5.700000in}}%
\pgfusepath{clip}%
\pgfsetbuttcap%
\pgfsetroundjoin%
\definecolor{currentfill}{rgb}{0.277134,0.185228,0.489898}%
\pgfsetfillcolor{currentfill}%
\pgfsetfillopacity{0.700000}%
\pgfsetlinewidth{0.000000pt}%
\definecolor{currentstroke}{rgb}{0.000000,0.000000,0.000000}%
\pgfsetstrokecolor{currentstroke}%
\pgfsetdash{}{0pt}%
\pgfpathmoveto{\pgfqpoint{3.499929in}{2.754153in}}%
\pgfpathlineto{\pgfqpoint{3.513254in}{2.746275in}}%
\pgfpathlineto{\pgfqpoint{3.526583in}{2.738499in}}%
\pgfpathlineto{\pgfqpoint{3.539915in}{2.730822in}}%
\pgfpathlineto{\pgfqpoint{3.553249in}{2.723245in}}%
\pgfpathlineto{\pgfqpoint{3.561131in}{2.736008in}}%
\pgfpathlineto{\pgfqpoint{3.569007in}{2.748926in}}%
\pgfpathlineto{\pgfqpoint{3.576877in}{2.762003in}}%
\pgfpathlineto{\pgfqpoint{3.584742in}{2.775244in}}%
\pgfpathlineto{\pgfqpoint{3.571414in}{2.783056in}}%
\pgfpathlineto{\pgfqpoint{3.558089in}{2.790968in}}%
\pgfpathlineto{\pgfqpoint{3.544767in}{2.798980in}}%
\pgfpathlineto{\pgfqpoint{3.531448in}{2.807093in}}%
\pgfpathlineto{\pgfqpoint{3.523577in}{2.793609in}}%
\pgfpathlineto{\pgfqpoint{3.515700in}{2.780294in}}%
\pgfpathlineto{\pgfqpoint{3.507817in}{2.767144in}}%
\pgfpathlineto{\pgfqpoint{3.499929in}{2.754153in}}%
\pgfpathclose%
\pgfusepath{fill}%
\end{pgfscope}%
\begin{pgfscope}%
\pgfpathrectangle{\pgfqpoint{1.150000in}{0.150000in}}{\pgfqpoint{5.700000in}{5.700000in}}%
\pgfusepath{clip}%
\pgfsetbuttcap%
\pgfsetroundjoin%
\definecolor{currentfill}{rgb}{0.266580,0.228262,0.514349}%
\pgfsetfillcolor{currentfill}%
\pgfsetfillopacity{0.700000}%
\pgfsetlinewidth{0.000000pt}%
\definecolor{currentstroke}{rgb}{0.000000,0.000000,0.000000}%
\pgfsetstrokecolor{currentstroke}%
\pgfsetdash{}{0pt}%
\pgfpathmoveto{\pgfqpoint{4.168328in}{2.834055in}}%
\pgfpathlineto{\pgfqpoint{4.181751in}{2.827748in}}%
\pgfpathlineto{\pgfqpoint{4.195179in}{2.821524in}}%
\pgfpathlineto{\pgfqpoint{4.208612in}{2.815383in}}%
\pgfpathlineto{\pgfqpoint{4.222050in}{2.809323in}}%
\pgfpathlineto{\pgfqpoint{4.229760in}{2.824062in}}%
\pgfpathlineto{\pgfqpoint{4.237467in}{2.839041in}}%
\pgfpathlineto{\pgfqpoint{4.245172in}{2.854265in}}%
\pgfpathlineto{\pgfqpoint{4.252874in}{2.869742in}}%
\pgfpathlineto{\pgfqpoint{4.239444in}{2.876157in}}%
\pgfpathlineto{\pgfqpoint{4.226018in}{2.882654in}}%
\pgfpathlineto{\pgfqpoint{4.212597in}{2.889233in}}%
\pgfpathlineto{\pgfqpoint{4.199181in}{2.895895in}}%
\pgfpathlineto{\pgfqpoint{4.191472in}{2.880055in}}%
\pgfpathlineto{\pgfqpoint{4.183760in}{2.864473in}}%
\pgfpathlineto{\pgfqpoint{4.176045in}{2.849142in}}%
\pgfpathlineto{\pgfqpoint{4.168328in}{2.834055in}}%
\pgfpathclose%
\pgfusepath{fill}%
\end{pgfscope}%
\begin{pgfscope}%
\pgfpathrectangle{\pgfqpoint{1.150000in}{0.150000in}}{\pgfqpoint{5.700000in}{5.700000in}}%
\pgfusepath{clip}%
\pgfsetbuttcap%
\pgfsetroundjoin%
\definecolor{currentfill}{rgb}{0.275191,0.194905,0.496005}%
\pgfsetfillcolor{currentfill}%
\pgfsetfillopacity{0.700000}%
\pgfsetlinewidth{0.000000pt}%
\definecolor{currentstroke}{rgb}{0.000000,0.000000,0.000000}%
\pgfsetstrokecolor{currentstroke}%
\pgfsetdash{}{0pt}%
\pgfpathmoveto{\pgfqpoint{3.860943in}{2.767190in}}%
\pgfpathlineto{\pgfqpoint{3.874317in}{2.760449in}}%
\pgfpathlineto{\pgfqpoint{3.887694in}{2.753799in}}%
\pgfpathlineto{\pgfqpoint{3.901076in}{2.747237in}}%
\pgfpathlineto{\pgfqpoint{3.914462in}{2.740764in}}%
\pgfpathlineto{\pgfqpoint{3.922247in}{2.754255in}}%
\pgfpathlineto{\pgfqpoint{3.930027in}{2.767935in}}%
\pgfpathlineto{\pgfqpoint{3.937803in}{2.781809in}}%
\pgfpathlineto{\pgfqpoint{3.945574in}{2.795882in}}%
\pgfpathlineto{\pgfqpoint{3.932195in}{2.802651in}}%
\pgfpathlineto{\pgfqpoint{3.918820in}{2.809508in}}%
\pgfpathlineto{\pgfqpoint{3.905450in}{2.816454in}}%
\pgfpathlineto{\pgfqpoint{3.892083in}{2.823490in}}%
\pgfpathlineto{\pgfqpoint{3.884305in}{2.809114in}}%
\pgfpathlineto{\pgfqpoint{3.876522in}{2.794942in}}%
\pgfpathlineto{\pgfqpoint{3.868735in}{2.780969in}}%
\pgfpathlineto{\pgfqpoint{3.860943in}{2.767190in}}%
\pgfpathclose%
\pgfusepath{fill}%
\end{pgfscope}%
\begin{pgfscope}%
\pgfpathrectangle{\pgfqpoint{1.150000in}{0.150000in}}{\pgfqpoint{5.700000in}{5.700000in}}%
\pgfusepath{clip}%
\pgfsetbuttcap%
\pgfsetroundjoin%
\definecolor{currentfill}{rgb}{0.258965,0.251537,0.524736}%
\pgfsetfillcolor{currentfill}%
\pgfsetfillopacity{0.700000}%
\pgfsetlinewidth{0.000000pt}%
\definecolor{currentstroke}{rgb}{0.000000,0.000000,0.000000}%
\pgfsetstrokecolor{currentstroke}%
\pgfsetdash{}{0pt}%
\pgfpathmoveto{\pgfqpoint{2.893155in}{2.895323in}}%
\pgfpathlineto{\pgfqpoint{2.906473in}{2.884018in}}%
\pgfpathlineto{\pgfqpoint{2.919790in}{2.872846in}}%
\pgfpathlineto{\pgfqpoint{2.933106in}{2.861806in}}%
\pgfpathlineto{\pgfqpoint{2.946422in}{2.850897in}}%
\pgfpathlineto{\pgfqpoint{2.954477in}{2.863041in}}%
\pgfpathlineto{\pgfqpoint{2.962524in}{2.875326in}}%
\pgfpathlineto{\pgfqpoint{2.970564in}{2.887755in}}%
\pgfpathlineto{\pgfqpoint{2.978596in}{2.900332in}}%
\pgfpathlineto{\pgfqpoint{2.965289in}{2.911397in}}%
\pgfpathlineto{\pgfqpoint{2.951980in}{2.922593in}}%
\pgfpathlineto{\pgfqpoint{2.938671in}{2.933920in}}%
\pgfpathlineto{\pgfqpoint{2.925362in}{2.945381in}}%
\pgfpathlineto{\pgfqpoint{2.917321in}{2.932641in}}%
\pgfpathlineto{\pgfqpoint{2.909274in}{2.920053in}}%
\pgfpathlineto{\pgfqpoint{2.901218in}{2.907615in}}%
\pgfpathlineto{\pgfqpoint{2.893155in}{2.895323in}}%
\pgfpathclose%
\pgfusepath{fill}%
\end{pgfscope}%
\begin{pgfscope}%
\pgfpathrectangle{\pgfqpoint{1.150000in}{0.150000in}}{\pgfqpoint{5.700000in}{5.700000in}}%
\pgfusepath{clip}%
\pgfsetbuttcap%
\pgfsetroundjoin%
\definecolor{currentfill}{rgb}{0.206756,0.371758,0.553117}%
\pgfsetfillcolor{currentfill}%
\pgfsetfillopacity{0.700000}%
\pgfsetlinewidth{0.000000pt}%
\definecolor{currentstroke}{rgb}{0.000000,0.000000,0.000000}%
\pgfsetstrokecolor{currentstroke}%
\pgfsetdash{}{0pt}%
\pgfpathmoveto{\pgfqpoint{4.760285in}{3.144262in}}%
\pgfpathlineto{\pgfqpoint{4.773798in}{3.137240in}}%
\pgfpathlineto{\pgfqpoint{4.787317in}{3.130291in}}%
\pgfpathlineto{\pgfqpoint{4.800842in}{3.123417in}}%
\pgfpathlineto{\pgfqpoint{4.814371in}{3.116617in}}%
\pgfpathlineto{\pgfqpoint{4.822031in}{3.137409in}}%
\pgfpathlineto{\pgfqpoint{4.829694in}{3.158624in}}%
\pgfpathlineto{\pgfqpoint{4.837362in}{3.180271in}}%
\pgfpathlineto{\pgfqpoint{4.845035in}{3.202360in}}%
\pgfpathlineto{\pgfqpoint{4.831513in}{3.209658in}}%
\pgfpathlineto{\pgfqpoint{4.817997in}{3.217030in}}%
\pgfpathlineto{\pgfqpoint{4.804485in}{3.224476in}}%
\pgfpathlineto{\pgfqpoint{4.790979in}{3.231997in}}%
\pgfpathlineto{\pgfqpoint{4.783298in}{3.209402in}}%
\pgfpathlineto{\pgfqpoint{4.775623in}{3.187255in}}%
\pgfpathlineto{\pgfqpoint{4.767952in}{3.165544in}}%
\pgfpathlineto{\pgfqpoint{4.760285in}{3.144262in}}%
\pgfpathclose%
\pgfusepath{fill}%
\end{pgfscope}%
\begin{pgfscope}%
\pgfpathrectangle{\pgfqpoint{1.150000in}{0.150000in}}{\pgfqpoint{5.700000in}{5.700000in}}%
\pgfusepath{clip}%
\pgfsetbuttcap%
\pgfsetroundjoin%
\definecolor{currentfill}{rgb}{0.163625,0.471133,0.558148}%
\pgfsetfillcolor{currentfill}%
\pgfsetfillopacity{0.700000}%
\pgfsetlinewidth{0.000000pt}%
\definecolor{currentstroke}{rgb}{0.000000,0.000000,0.000000}%
\pgfsetstrokecolor{currentstroke}%
\pgfsetdash{}{0pt}%
\pgfpathmoveto{\pgfqpoint{4.906637in}{3.396150in}}%
\pgfpathlineto{\pgfqpoint{4.920152in}{3.387866in}}%
\pgfpathlineto{\pgfqpoint{4.933671in}{3.379655in}}%
\pgfpathlineto{\pgfqpoint{4.947195in}{3.371518in}}%
\pgfpathlineto{\pgfqpoint{4.960724in}{3.363454in}}%
\pgfpathlineto{\pgfqpoint{4.968451in}{3.389411in}}%
\pgfpathlineto{\pgfqpoint{4.976187in}{3.415906in}}%
\pgfpathlineto{\pgfqpoint{4.983932in}{3.442950in}}%
\pgfpathlineto{\pgfqpoint{4.991687in}{3.470553in}}%
\pgfpathlineto{\pgfqpoint{4.978163in}{3.479180in}}%
\pgfpathlineto{\pgfqpoint{4.964645in}{3.487881in}}%
\pgfpathlineto{\pgfqpoint{4.951130in}{3.496655in}}%
\pgfpathlineto{\pgfqpoint{4.937621in}{3.505503in}}%
\pgfpathlineto{\pgfqpoint{4.929861in}{3.477327in}}%
\pgfpathlineto{\pgfqpoint{4.922111in}{3.449717in}}%
\pgfpathlineto{\pgfqpoint{4.914370in}{3.422662in}}%
\pgfpathlineto{\pgfqpoint{4.906637in}{3.396150in}}%
\pgfpathclose%
\pgfusepath{fill}%
\end{pgfscope}%
\begin{pgfscope}%
\pgfpathrectangle{\pgfqpoint{1.150000in}{0.150000in}}{\pgfqpoint{5.700000in}{5.700000in}}%
\pgfusepath{clip}%
\pgfsetbuttcap%
\pgfsetroundjoin%
\definecolor{currentfill}{rgb}{0.270595,0.214069,0.507052}%
\pgfsetfillcolor{currentfill}%
\pgfsetfillopacity{0.700000}%
\pgfsetlinewidth{0.000000pt}%
\definecolor{currentstroke}{rgb}{0.000000,0.000000,0.000000}%
\pgfsetstrokecolor{currentstroke}%
\pgfsetdash{}{0pt}%
\pgfpathmoveto{\pgfqpoint{3.085051in}{2.816387in}}%
\pgfpathlineto{\pgfqpoint{3.098358in}{2.806450in}}%
\pgfpathlineto{\pgfqpoint{3.111666in}{2.796633in}}%
\pgfpathlineto{\pgfqpoint{3.124975in}{2.786935in}}%
\pgfpathlineto{\pgfqpoint{3.138285in}{2.777355in}}%
\pgfpathlineto{\pgfqpoint{3.146286in}{2.789578in}}%
\pgfpathlineto{\pgfqpoint{3.154281in}{2.801940in}}%
\pgfpathlineto{\pgfqpoint{3.162269in}{2.814445in}}%
\pgfpathlineto{\pgfqpoint{3.170250in}{2.827095in}}%
\pgfpathlineto{\pgfqpoint{3.156948in}{2.836850in}}%
\pgfpathlineto{\pgfqpoint{3.143647in}{2.846723in}}%
\pgfpathlineto{\pgfqpoint{3.130346in}{2.856716in}}%
\pgfpathlineto{\pgfqpoint{3.117046in}{2.866829in}}%
\pgfpathlineto{\pgfqpoint{3.109058in}{2.853996in}}%
\pgfpathlineto{\pgfqpoint{3.101062in}{2.841313in}}%
\pgfpathlineto{\pgfqpoint{3.093060in}{2.828778in}}%
\pgfpathlineto{\pgfqpoint{3.085051in}{2.816387in}}%
\pgfpathclose%
\pgfusepath{fill}%
\end{pgfscope}%
\begin{pgfscope}%
\pgfpathrectangle{\pgfqpoint{1.150000in}{0.150000in}}{\pgfqpoint{5.700000in}{5.700000in}}%
\pgfusepath{clip}%
\pgfsetbuttcap%
\pgfsetroundjoin%
\definecolor{currentfill}{rgb}{0.180629,0.429975,0.557282}%
\pgfsetfillcolor{currentfill}%
\pgfsetfillopacity{0.700000}%
\pgfsetlinewidth{0.000000pt}%
\definecolor{currentstroke}{rgb}{0.000000,0.000000,0.000000}%
\pgfsetstrokecolor{currentstroke}%
\pgfsetdash{}{0pt}%
\pgfpathmoveto{\pgfqpoint{4.875783in}{3.295326in}}%
\pgfpathlineto{\pgfqpoint{4.889304in}{3.287582in}}%
\pgfpathlineto{\pgfqpoint{4.902829in}{3.279913in}}%
\pgfpathlineto{\pgfqpoint{4.916359in}{3.272316in}}%
\pgfpathlineto{\pgfqpoint{4.929895in}{3.264793in}}%
\pgfpathlineto{\pgfqpoint{4.937591in}{3.288704in}}%
\pgfpathlineto{\pgfqpoint{4.945294in}{3.313111in}}%
\pgfpathlineto{\pgfqpoint{4.953005in}{3.338024in}}%
\pgfpathlineto{\pgfqpoint{4.960724in}{3.363454in}}%
\pgfpathlineto{\pgfqpoint{4.947195in}{3.371518in}}%
\pgfpathlineto{\pgfqpoint{4.933671in}{3.379655in}}%
\pgfpathlineto{\pgfqpoint{4.920152in}{3.387866in}}%
\pgfpathlineto{\pgfqpoint{4.906637in}{3.396150in}}%
\pgfpathlineto{\pgfqpoint{4.898913in}{3.370171in}}%
\pgfpathlineto{\pgfqpoint{4.891196in}{3.344715in}}%
\pgfpathlineto{\pgfqpoint{4.883486in}{3.319770in}}%
\pgfpathlineto{\pgfqpoint{4.875783in}{3.295326in}}%
\pgfpathclose%
\pgfusepath{fill}%
\end{pgfscope}%
\begin{pgfscope}%
\pgfpathrectangle{\pgfqpoint{1.150000in}{0.150000in}}{\pgfqpoint{5.700000in}{5.700000in}}%
\pgfusepath{clip}%
\pgfsetbuttcap%
\pgfsetroundjoin%
\definecolor{currentfill}{rgb}{0.277134,0.185228,0.489898}%
\pgfsetfillcolor{currentfill}%
\pgfsetfillopacity{0.700000}%
\pgfsetlinewidth{0.000000pt}%
\definecolor{currentstroke}{rgb}{0.000000,0.000000,0.000000}%
\pgfsetstrokecolor{currentstroke}%
\pgfsetdash{}{0pt}%
\pgfpathmoveto{\pgfqpoint{3.638082in}{2.744977in}}%
\pgfpathlineto{\pgfqpoint{3.651425in}{2.737652in}}%
\pgfpathlineto{\pgfqpoint{3.664772in}{2.730424in}}%
\pgfpathlineto{\pgfqpoint{3.678122in}{2.723291in}}%
\pgfpathlineto{\pgfqpoint{3.691475in}{2.716253in}}%
\pgfpathlineto{\pgfqpoint{3.699320in}{2.729166in}}%
\pgfpathlineto{\pgfqpoint{3.707161in}{2.742243in}}%
\pgfpathlineto{\pgfqpoint{3.714996in}{2.755487in}}%
\pgfpathlineto{\pgfqpoint{3.722825in}{2.768904in}}%
\pgfpathlineto{\pgfqpoint{3.709479in}{2.776197in}}%
\pgfpathlineto{\pgfqpoint{3.696136in}{2.783586in}}%
\pgfpathlineto{\pgfqpoint{3.682796in}{2.791070in}}%
\pgfpathlineto{\pgfqpoint{3.669459in}{2.798649in}}%
\pgfpathlineto{\pgfqpoint{3.661623in}{2.784969in}}%
\pgfpathlineto{\pgfqpoint{3.653782in}{2.771467in}}%
\pgfpathlineto{\pgfqpoint{3.645935in}{2.758138in}}%
\pgfpathlineto{\pgfqpoint{3.638082in}{2.744977in}}%
\pgfpathclose%
\pgfusepath{fill}%
\end{pgfscope}%
\begin{pgfscope}%
\pgfpathrectangle{\pgfqpoint{1.150000in}{0.150000in}}{\pgfqpoint{5.700000in}{5.700000in}}%
\pgfusepath{clip}%
\pgfsetbuttcap%
\pgfsetroundjoin%
\definecolor{currentfill}{rgb}{0.149039,0.508051,0.557250}%
\pgfsetfillcolor{currentfill}%
\pgfsetfillopacity{0.700000}%
\pgfsetlinewidth{0.000000pt}%
\definecolor{currentstroke}{rgb}{0.000000,0.000000,0.000000}%
\pgfsetstrokecolor{currentstroke}%
\pgfsetdash{}{0pt}%
\pgfpathmoveto{\pgfqpoint{4.937621in}{3.505503in}}%
\pgfpathlineto{\pgfqpoint{4.951130in}{3.496655in}}%
\pgfpathlineto{\pgfqpoint{4.964645in}{3.487881in}}%
\pgfpathlineto{\pgfqpoint{4.978163in}{3.479180in}}%
\pgfpathlineto{\pgfqpoint{4.991687in}{3.470553in}}%
\pgfpathlineto{\pgfqpoint{4.999452in}{3.498728in}}%
\pgfpathlineto{\pgfqpoint{5.007227in}{3.527486in}}%
\pgfpathlineto{\pgfqpoint{5.015013in}{3.556838in}}%
\pgfpathlineto{\pgfqpoint{5.001493in}{3.565902in}}%
\pgfpathlineto{\pgfqpoint{4.987977in}{3.575040in}}%
\pgfpathlineto{\pgfqpoint{4.974466in}{3.584252in}}%
\pgfpathlineto{\pgfqpoint{4.960959in}{3.593538in}}%
\pgfpathlineto{\pgfqpoint{4.953169in}{3.563597in}}%
\pgfpathlineto{\pgfqpoint{4.945390in}{3.534256in}}%
\pgfpathlineto{\pgfqpoint{4.937621in}{3.505503in}}%
\pgfpathclose%
\pgfusepath{fill}%
\end{pgfscope}%
\begin{pgfscope}%
\pgfpathrectangle{\pgfqpoint{1.150000in}{0.150000in}}{\pgfqpoint{5.700000in}{5.700000in}}%
\pgfusepath{clip}%
\pgfsetbuttcap%
\pgfsetroundjoin%
\definecolor{currentfill}{rgb}{0.270595,0.214069,0.507052}%
\pgfsetfillcolor{currentfill}%
\pgfsetfillopacity{0.700000}%
\pgfsetlinewidth{0.000000pt}%
\definecolor{currentstroke}{rgb}{0.000000,0.000000,0.000000}%
\pgfsetstrokecolor{currentstroke}%
\pgfsetdash{}{0pt}%
\pgfpathmoveto{\pgfqpoint{4.083751in}{2.800746in}}%
\pgfpathlineto{\pgfqpoint{4.097163in}{2.794440in}}%
\pgfpathlineto{\pgfqpoint{4.110579in}{2.788218in}}%
\pgfpathlineto{\pgfqpoint{4.124000in}{2.782080in}}%
\pgfpathlineto{\pgfqpoint{4.137426in}{2.776025in}}%
\pgfpathlineto{\pgfqpoint{4.145156in}{2.790198in}}%
\pgfpathlineto{\pgfqpoint{4.152883in}{2.804589in}}%
\pgfpathlineto{\pgfqpoint{4.160607in}{2.819206in}}%
\pgfpathlineto{\pgfqpoint{4.168328in}{2.834055in}}%
\pgfpathlineto{\pgfqpoint{4.154909in}{2.840445in}}%
\pgfpathlineto{\pgfqpoint{4.141496in}{2.846918in}}%
\pgfpathlineto{\pgfqpoint{4.128086in}{2.853476in}}%
\pgfpathlineto{\pgfqpoint{4.114682in}{2.860117in}}%
\pgfpathlineto{\pgfqpoint{4.106954in}{2.844926in}}%
\pgfpathlineto{\pgfqpoint{4.099223in}{2.829971in}}%
\pgfpathlineto{\pgfqpoint{4.091489in}{2.815247in}}%
\pgfpathlineto{\pgfqpoint{4.083751in}{2.800746in}}%
\pgfpathclose%
\pgfusepath{fill}%
\end{pgfscope}%
\begin{pgfscope}%
\pgfpathrectangle{\pgfqpoint{1.150000in}{0.150000in}}{\pgfqpoint{5.700000in}{5.700000in}}%
\pgfusepath{clip}%
\pgfsetbuttcap%
\pgfsetroundjoin%
\definecolor{currentfill}{rgb}{0.241237,0.296485,0.539709}%
\pgfsetfillcolor{currentfill}%
\pgfsetfillopacity{0.700000}%
\pgfsetlinewidth{0.000000pt}%
\definecolor{currentstroke}{rgb}{0.000000,0.000000,0.000000}%
\pgfsetstrokecolor{currentstroke}%
\pgfsetdash{}{0pt}%
\pgfpathmoveto{\pgfqpoint{4.560384in}{2.966952in}}%
\pgfpathlineto{\pgfqpoint{4.573878in}{2.960695in}}%
\pgfpathlineto{\pgfqpoint{4.587378in}{2.954514in}}%
\pgfpathlineto{\pgfqpoint{4.600884in}{2.948409in}}%
\pgfpathlineto{\pgfqpoint{4.614395in}{2.942380in}}%
\pgfpathlineto{\pgfqpoint{4.622044in}{2.959642in}}%
\pgfpathlineto{\pgfqpoint{4.629693in}{2.977234in}}%
\pgfpathlineto{\pgfqpoint{4.637344in}{2.995163in}}%
\pgfpathlineto{\pgfqpoint{4.644996in}{3.013438in}}%
\pgfpathlineto{\pgfqpoint{4.631493in}{3.019903in}}%
\pgfpathlineto{\pgfqpoint{4.617996in}{3.026444in}}%
\pgfpathlineto{\pgfqpoint{4.604504in}{3.033060in}}%
\pgfpathlineto{\pgfqpoint{4.591017in}{3.039754in}}%
\pgfpathlineto{\pgfqpoint{4.583357in}{3.021035in}}%
\pgfpathlineto{\pgfqpoint{4.575699in}{3.002667in}}%
\pgfpathlineto{\pgfqpoint{4.568041in}{2.984642in}}%
\pgfpathlineto{\pgfqpoint{4.560384in}{2.966952in}}%
\pgfpathclose%
\pgfusepath{fill}%
\end{pgfscope}%
\begin{pgfscope}%
\pgfpathrectangle{\pgfqpoint{1.150000in}{0.150000in}}{\pgfqpoint{5.700000in}{5.700000in}}%
\pgfusepath{clip}%
\pgfsetbuttcap%
\pgfsetroundjoin%
\definecolor{currentfill}{rgb}{0.195860,0.395433,0.555276}%
\pgfsetfillcolor{currentfill}%
\pgfsetfillopacity{0.700000}%
\pgfsetlinewidth{0.000000pt}%
\definecolor{currentstroke}{rgb}{0.000000,0.000000,0.000000}%
\pgfsetstrokecolor{currentstroke}%
\pgfsetdash{}{0pt}%
\pgfpathmoveto{\pgfqpoint{4.845035in}{3.202360in}}%
\pgfpathlineto{\pgfqpoint{4.858563in}{3.195136in}}%
\pgfpathlineto{\pgfqpoint{4.872095in}{3.187985in}}%
\pgfpathlineto{\pgfqpoint{4.885633in}{3.180907in}}%
\pgfpathlineto{\pgfqpoint{4.899176in}{3.173903in}}%
\pgfpathlineto{\pgfqpoint{4.906847in}{3.195932in}}%
\pgfpathlineto{\pgfqpoint{4.914523in}{3.218417in}}%
\pgfpathlineto{\pgfqpoint{4.922206in}{3.241367in}}%
\pgfpathlineto{\pgfqpoint{4.929895in}{3.264793in}}%
\pgfpathlineto{\pgfqpoint{4.916359in}{3.272316in}}%
\pgfpathlineto{\pgfqpoint{4.902829in}{3.279913in}}%
\pgfpathlineto{\pgfqpoint{4.889304in}{3.287582in}}%
\pgfpathlineto{\pgfqpoint{4.875783in}{3.295326in}}%
\pgfpathlineto{\pgfqpoint{4.868087in}{3.271373in}}%
\pgfpathlineto{\pgfqpoint{4.860398in}{3.247901in}}%
\pgfpathlineto{\pgfqpoint{4.852714in}{3.224900in}}%
\pgfpathlineto{\pgfqpoint{4.845035in}{3.202360in}}%
\pgfpathclose%
\pgfusepath{fill}%
\end{pgfscope}%
\begin{pgfscope}%
\pgfpathrectangle{\pgfqpoint{1.150000in}{0.150000in}}{\pgfqpoint{5.700000in}{5.700000in}}%
\pgfusepath{clip}%
\pgfsetbuttcap%
\pgfsetroundjoin%
\definecolor{currentfill}{rgb}{0.250425,0.274290,0.533103}%
\pgfsetfillcolor{currentfill}%
\pgfsetfillopacity{0.700000}%
\pgfsetlinewidth{0.000000pt}%
\definecolor{currentstroke}{rgb}{0.000000,0.000000,0.000000}%
\pgfsetstrokecolor{currentstroke}%
\pgfsetdash{}{0pt}%
\pgfpathmoveto{\pgfqpoint{4.475798in}{2.923516in}}%
\pgfpathlineto{\pgfqpoint{4.489280in}{2.917366in}}%
\pgfpathlineto{\pgfqpoint{4.502766in}{2.911294in}}%
\pgfpathlineto{\pgfqpoint{4.516258in}{2.905299in}}%
\pgfpathlineto{\pgfqpoint{4.529756in}{2.899381in}}%
\pgfpathlineto{\pgfqpoint{4.537413in}{2.915811in}}%
\pgfpathlineto{\pgfqpoint{4.545070in}{2.932544in}}%
\pgfpathlineto{\pgfqpoint{4.552727in}{2.949588in}}%
\pgfpathlineto{\pgfqpoint{4.560384in}{2.966952in}}%
\pgfpathlineto{\pgfqpoint{4.546894in}{2.973286in}}%
\pgfpathlineto{\pgfqpoint{4.533411in}{2.979696in}}%
\pgfpathlineto{\pgfqpoint{4.519932in}{2.986185in}}%
\pgfpathlineto{\pgfqpoint{4.506458in}{2.992750in}}%
\pgfpathlineto{\pgfqpoint{4.498794in}{2.974963in}}%
\pgfpathlineto{\pgfqpoint{4.491129in}{2.957501in}}%
\pgfpathlineto{\pgfqpoint{4.483464in}{2.940354in}}%
\pgfpathlineto{\pgfqpoint{4.475798in}{2.923516in}}%
\pgfpathclose%
\pgfusepath{fill}%
\end{pgfscope}%
\begin{pgfscope}%
\pgfpathrectangle{\pgfqpoint{1.150000in}{0.150000in}}{\pgfqpoint{5.700000in}{5.700000in}}%
\pgfusepath{clip}%
\pgfsetbuttcap%
\pgfsetroundjoin%
\definecolor{currentfill}{rgb}{0.277134,0.185228,0.489898}%
\pgfsetfillcolor{currentfill}%
\pgfsetfillopacity{0.700000}%
\pgfsetlinewidth{0.000000pt}%
\definecolor{currentstroke}{rgb}{0.000000,0.000000,0.000000}%
\pgfsetstrokecolor{currentstroke}%
\pgfsetdash{}{0pt}%
\pgfpathmoveto{\pgfqpoint{3.776247in}{2.740666in}}%
\pgfpathlineto{\pgfqpoint{3.789612in}{2.733838in}}%
\pgfpathlineto{\pgfqpoint{3.802980in}{2.727101in}}%
\pgfpathlineto{\pgfqpoint{3.816353in}{2.720455in}}%
\pgfpathlineto{\pgfqpoint{3.829730in}{2.713900in}}%
\pgfpathlineto{\pgfqpoint{3.837540in}{2.726959in}}%
\pgfpathlineto{\pgfqpoint{3.845346in}{2.740190in}}%
\pgfpathlineto{\pgfqpoint{3.853147in}{2.753598in}}%
\pgfpathlineto{\pgfqpoint{3.860943in}{2.767190in}}%
\pgfpathlineto{\pgfqpoint{3.847574in}{2.774020in}}%
\pgfpathlineto{\pgfqpoint{3.834208in}{2.780941in}}%
\pgfpathlineto{\pgfqpoint{3.820847in}{2.787953in}}%
\pgfpathlineto{\pgfqpoint{3.807489in}{2.795057in}}%
\pgfpathlineto{\pgfqpoint{3.799685in}{2.781183in}}%
\pgfpathlineto{\pgfqpoint{3.791878in}{2.767496in}}%
\pgfpathlineto{\pgfqpoint{3.784065in}{2.753992in}}%
\pgfpathlineto{\pgfqpoint{3.776247in}{2.740666in}}%
\pgfpathclose%
\pgfusepath{fill}%
\end{pgfscope}%
\begin{pgfscope}%
\pgfpathrectangle{\pgfqpoint{1.150000in}{0.150000in}}{\pgfqpoint{5.700000in}{5.700000in}}%
\pgfusepath{clip}%
\pgfsetbuttcap%
\pgfsetroundjoin%
\definecolor{currentfill}{rgb}{0.231674,0.318106,0.544834}%
\pgfsetfillcolor{currentfill}%
\pgfsetfillopacity{0.700000}%
\pgfsetlinewidth{0.000000pt}%
\definecolor{currentstroke}{rgb}{0.000000,0.000000,0.000000}%
\pgfsetstrokecolor{currentstroke}%
\pgfsetdash{}{0pt}%
\pgfpathmoveto{\pgfqpoint{4.644996in}{3.013438in}}%
\pgfpathlineto{\pgfqpoint{4.658504in}{3.007049in}}%
\pgfpathlineto{\pgfqpoint{4.672017in}{3.000735in}}%
\pgfpathlineto{\pgfqpoint{4.685536in}{2.994496in}}%
\pgfpathlineto{\pgfqpoint{4.699061in}{2.988332in}}%
\pgfpathlineto{\pgfqpoint{4.706705in}{3.006513in}}%
\pgfpathlineto{\pgfqpoint{4.714352in}{3.025051in}}%
\pgfpathlineto{\pgfqpoint{4.722001in}{3.043954in}}%
\pgfpathlineto{\pgfqpoint{4.729652in}{3.063232in}}%
\pgfpathlineto{\pgfqpoint{4.716136in}{3.069852in}}%
\pgfpathlineto{\pgfqpoint{4.702625in}{3.076547in}}%
\pgfpathlineto{\pgfqpoint{4.689119in}{3.083318in}}%
\pgfpathlineto{\pgfqpoint{4.675619in}{3.090163in}}%
\pgfpathlineto{\pgfqpoint{4.667960in}{3.070421in}}%
\pgfpathlineto{\pgfqpoint{4.660304in}{3.051059in}}%
\pgfpathlineto{\pgfqpoint{4.652649in}{3.032067in}}%
\pgfpathlineto{\pgfqpoint{4.644996in}{3.013438in}}%
\pgfpathclose%
\pgfusepath{fill}%
\end{pgfscope}%
\begin{pgfscope}%
\pgfpathrectangle{\pgfqpoint{1.150000in}{0.150000in}}{\pgfqpoint{5.700000in}{5.700000in}}%
\pgfusepath{clip}%
\pgfsetbuttcap%
\pgfsetroundjoin%
\definecolor{currentfill}{rgb}{0.265145,0.232956,0.516599}%
\pgfsetfillcolor{currentfill}%
\pgfsetfillopacity{0.700000}%
\pgfsetlinewidth{0.000000pt}%
\definecolor{currentstroke}{rgb}{0.000000,0.000000,0.000000}%
\pgfsetstrokecolor{currentstroke}%
\pgfsetdash{}{0pt}%
\pgfpathmoveto{\pgfqpoint{2.946422in}{2.850897in}}%
\pgfpathlineto{\pgfqpoint{2.959737in}{2.840118in}}%
\pgfpathlineto{\pgfqpoint{2.973052in}{2.829466in}}%
\pgfpathlineto{\pgfqpoint{2.986367in}{2.818943in}}%
\pgfpathlineto{\pgfqpoint{2.999682in}{2.808545in}}%
\pgfpathlineto{\pgfqpoint{3.007728in}{2.820541in}}%
\pgfpathlineto{\pgfqpoint{3.015767in}{2.832672in}}%
\pgfpathlineto{\pgfqpoint{3.023799in}{2.844944in}}%
\pgfpathlineto{\pgfqpoint{3.031824in}{2.857357in}}%
\pgfpathlineto{\pgfqpoint{3.018517in}{2.867910in}}%
\pgfpathlineto{\pgfqpoint{3.005210in}{2.878590in}}%
\pgfpathlineto{\pgfqpoint{2.991904in}{2.889397in}}%
\pgfpathlineto{\pgfqpoint{2.978596in}{2.900332in}}%
\pgfpathlineto{\pgfqpoint{2.970564in}{2.887755in}}%
\pgfpathlineto{\pgfqpoint{2.962524in}{2.875326in}}%
\pgfpathlineto{\pgfqpoint{2.954477in}{2.863041in}}%
\pgfpathlineto{\pgfqpoint{2.946422in}{2.850897in}}%
\pgfpathclose%
\pgfusepath{fill}%
\end{pgfscope}%
\begin{pgfscope}%
\pgfpathrectangle{\pgfqpoint{1.150000in}{0.150000in}}{\pgfqpoint{5.700000in}{5.700000in}}%
\pgfusepath{clip}%
\pgfsetbuttcap%
\pgfsetroundjoin%
\definecolor{currentfill}{rgb}{0.257322,0.256130,0.526563}%
\pgfsetfillcolor{currentfill}%
\pgfsetfillopacity{0.700000}%
\pgfsetlinewidth{0.000000pt}%
\definecolor{currentstroke}{rgb}{0.000000,0.000000,0.000000}%
\pgfsetstrokecolor{currentstroke}%
\pgfsetdash{}{0pt}%
\pgfpathmoveto{\pgfqpoint{4.391223in}{2.882900in}}%
\pgfpathlineto{\pgfqpoint{4.404692in}{2.876832in}}%
\pgfpathlineto{\pgfqpoint{4.418165in}{2.870844in}}%
\pgfpathlineto{\pgfqpoint{4.431644in}{2.864933in}}%
\pgfpathlineto{\pgfqpoint{4.445128in}{2.859101in}}%
\pgfpathlineto{\pgfqpoint{4.452797in}{2.874779in}}%
\pgfpathlineto{\pgfqpoint{4.460465in}{2.890736in}}%
\pgfpathlineto{\pgfqpoint{4.468132in}{2.906979in}}%
\pgfpathlineto{\pgfqpoint{4.475798in}{2.923516in}}%
\pgfpathlineto{\pgfqpoint{4.462322in}{2.929744in}}%
\pgfpathlineto{\pgfqpoint{4.448852in}{2.936050in}}%
\pgfpathlineto{\pgfqpoint{4.435386in}{2.942434in}}%
\pgfpathlineto{\pgfqpoint{4.421925in}{2.948897in}}%
\pgfpathlineto{\pgfqpoint{4.414252in}{2.931957in}}%
\pgfpathlineto{\pgfqpoint{4.406577in}{2.915315in}}%
\pgfpathlineto{\pgfqpoint{4.398901in}{2.898965in}}%
\pgfpathlineto{\pgfqpoint{4.391223in}{2.882900in}}%
\pgfpathclose%
\pgfusepath{fill}%
\end{pgfscope}%
\begin{pgfscope}%
\pgfpathrectangle{\pgfqpoint{1.150000in}{0.150000in}}{\pgfqpoint{5.700000in}{5.700000in}}%
\pgfusepath{clip}%
\pgfsetbuttcap%
\pgfsetroundjoin%
\definecolor{currentfill}{rgb}{0.276194,0.190074,0.493001}%
\pgfsetfillcolor{currentfill}%
\pgfsetfillopacity{0.700000}%
\pgfsetlinewidth{0.000000pt}%
\definecolor{currentstroke}{rgb}{0.000000,0.000000,0.000000}%
\pgfsetstrokecolor{currentstroke}%
\pgfsetdash{}{0pt}%
\pgfpathmoveto{\pgfqpoint{3.276706in}{2.753201in}}%
\pgfpathlineto{\pgfqpoint{3.290019in}{2.744469in}}%
\pgfpathlineto{\pgfqpoint{3.303335in}{2.735847in}}%
\pgfpathlineto{\pgfqpoint{3.316652in}{2.727334in}}%
\pgfpathlineto{\pgfqpoint{3.329971in}{2.718929in}}%
\pgfpathlineto{\pgfqpoint{3.337923in}{2.731164in}}%
\pgfpathlineto{\pgfqpoint{3.345868in}{2.743537in}}%
\pgfpathlineto{\pgfqpoint{3.353808in}{2.756052in}}%
\pgfpathlineto{\pgfqpoint{3.361741in}{2.768712in}}%
\pgfpathlineto{\pgfqpoint{3.348429in}{2.777313in}}%
\pgfpathlineto{\pgfqpoint{3.335119in}{2.786021in}}%
\pgfpathlineto{\pgfqpoint{3.321811in}{2.794838in}}%
\pgfpathlineto{\pgfqpoint{3.308505in}{2.803765in}}%
\pgfpathlineto{\pgfqpoint{3.300565in}{2.790902in}}%
\pgfpathlineto{\pgfqpoint{3.292618in}{2.778190in}}%
\pgfpathlineto{\pgfqpoint{3.284665in}{2.765624in}}%
\pgfpathlineto{\pgfqpoint{3.276706in}{2.753201in}}%
\pgfpathclose%
\pgfusepath{fill}%
\end{pgfscope}%
\begin{pgfscope}%
\pgfpathrectangle{\pgfqpoint{1.150000in}{0.150000in}}{\pgfqpoint{5.700000in}{5.700000in}}%
\pgfusepath{clip}%
\pgfsetbuttcap%
\pgfsetroundjoin%
\definecolor{currentfill}{rgb}{0.274128,0.199721,0.498911}%
\pgfsetfillcolor{currentfill}%
\pgfsetfillopacity{0.700000}%
\pgfsetlinewidth{0.000000pt}%
\definecolor{currentstroke}{rgb}{0.000000,0.000000,0.000000}%
\pgfsetstrokecolor{currentstroke}%
\pgfsetdash{}{0pt}%
\pgfpathmoveto{\pgfqpoint{3.999133in}{2.769685in}}%
\pgfpathlineto{\pgfqpoint{4.012533in}{2.763353in}}%
\pgfpathlineto{\pgfqpoint{4.025939in}{2.757106in}}%
\pgfpathlineto{\pgfqpoint{4.039348in}{2.750945in}}%
\pgfpathlineto{\pgfqpoint{4.052763in}{2.744870in}}%
\pgfpathlineto{\pgfqpoint{4.060516in}{2.758532in}}%
\pgfpathlineto{\pgfqpoint{4.068264in}{2.772395in}}%
\pgfpathlineto{\pgfqpoint{4.076010in}{2.786465in}}%
\pgfpathlineto{\pgfqpoint{4.083751in}{2.800746in}}%
\pgfpathlineto{\pgfqpoint{4.070344in}{2.807137in}}%
\pgfpathlineto{\pgfqpoint{4.056942in}{2.813614in}}%
\pgfpathlineto{\pgfqpoint{4.043544in}{2.820175in}}%
\pgfpathlineto{\pgfqpoint{4.030150in}{2.826823in}}%
\pgfpathlineto{\pgfqpoint{4.022402in}{2.812218in}}%
\pgfpathlineto{\pgfqpoint{4.014649in}{2.797831in}}%
\pgfpathlineto{\pgfqpoint{4.006893in}{2.783655in}}%
\pgfpathlineto{\pgfqpoint{3.999133in}{2.769685in}}%
\pgfpathclose%
\pgfusepath{fill}%
\end{pgfscope}%
\begin{pgfscope}%
\pgfpathrectangle{\pgfqpoint{1.150000in}{0.150000in}}{\pgfqpoint{5.700000in}{5.700000in}}%
\pgfusepath{clip}%
\pgfsetbuttcap%
\pgfsetroundjoin%
\definecolor{currentfill}{rgb}{0.278012,0.180367,0.486697}%
\pgfsetfillcolor{currentfill}%
\pgfsetfillopacity{0.700000}%
\pgfsetlinewidth{0.000000pt}%
\definecolor{currentstroke}{rgb}{0.000000,0.000000,0.000000}%
\pgfsetstrokecolor{currentstroke}%
\pgfsetdash{}{0pt}%
\pgfpathmoveto{\pgfqpoint{3.415009in}{2.735374in}}%
\pgfpathlineto{\pgfqpoint{3.428331in}{2.727302in}}%
\pgfpathlineto{\pgfqpoint{3.441657in}{2.719334in}}%
\pgfpathlineto{\pgfqpoint{3.454984in}{2.711468in}}%
\pgfpathlineto{\pgfqpoint{3.468315in}{2.703705in}}%
\pgfpathlineto{\pgfqpoint{3.476227in}{2.716098in}}%
\pgfpathlineto{\pgfqpoint{3.484133in}{2.728634in}}%
\pgfpathlineto{\pgfqpoint{3.492034in}{2.741318in}}%
\pgfpathlineto{\pgfqpoint{3.499929in}{2.754153in}}%
\pgfpathlineto{\pgfqpoint{3.486605in}{2.762132in}}%
\pgfpathlineto{\pgfqpoint{3.473285in}{2.770213in}}%
\pgfpathlineto{\pgfqpoint{3.459967in}{2.778397in}}%
\pgfpathlineto{\pgfqpoint{3.446651in}{2.786685in}}%
\pgfpathlineto{\pgfqpoint{3.438749in}{2.773626in}}%
\pgfpathlineto{\pgfqpoint{3.430842in}{2.760725in}}%
\pgfpathlineto{\pgfqpoint{3.422928in}{2.747975in}}%
\pgfpathlineto{\pgfqpoint{3.415009in}{2.735374in}}%
\pgfpathclose%
\pgfusepath{fill}%
\end{pgfscope}%
\begin{pgfscope}%
\pgfpathrectangle{\pgfqpoint{1.150000in}{0.150000in}}{\pgfqpoint{5.700000in}{5.700000in}}%
\pgfusepath{clip}%
\pgfsetbuttcap%
\pgfsetroundjoin%
\definecolor{currentfill}{rgb}{0.221989,0.339161,0.548752}%
\pgfsetfillcolor{currentfill}%
\pgfsetfillopacity{0.700000}%
\pgfsetlinewidth{0.000000pt}%
\definecolor{currentstroke}{rgb}{0.000000,0.000000,0.000000}%
\pgfsetstrokecolor{currentstroke}%
\pgfsetdash{}{0pt}%
\pgfpathmoveto{\pgfqpoint{4.729652in}{3.063232in}}%
\pgfpathlineto{\pgfqpoint{4.743173in}{3.056686in}}%
\pgfpathlineto{\pgfqpoint{4.756700in}{3.050215in}}%
\pgfpathlineto{\pgfqpoint{4.770233in}{3.043818in}}%
\pgfpathlineto{\pgfqpoint{4.783771in}{3.037495in}}%
\pgfpathlineto{\pgfqpoint{4.791416in}{3.056687in}}%
\pgfpathlineto{\pgfqpoint{4.799064in}{3.076265in}}%
\pgfpathlineto{\pgfqpoint{4.806716in}{3.096239in}}%
\pgfpathlineto{\pgfqpoint{4.814371in}{3.116617in}}%
\pgfpathlineto{\pgfqpoint{4.800842in}{3.123417in}}%
\pgfpathlineto{\pgfqpoint{4.787317in}{3.130291in}}%
\pgfpathlineto{\pgfqpoint{4.773798in}{3.137240in}}%
\pgfpathlineto{\pgfqpoint{4.760285in}{3.144262in}}%
\pgfpathlineto{\pgfqpoint{4.752622in}{3.123399in}}%
\pgfpathlineto{\pgfqpoint{4.744962in}{3.102946in}}%
\pgfpathlineto{\pgfqpoint{4.737305in}{3.082893in}}%
\pgfpathlineto{\pgfqpoint{4.729652in}{3.063232in}}%
\pgfpathclose%
\pgfusepath{fill}%
\end{pgfscope}%
\begin{pgfscope}%
\pgfpathrectangle{\pgfqpoint{1.150000in}{0.150000in}}{\pgfqpoint{5.700000in}{5.700000in}}%
\pgfusepath{clip}%
\pgfsetbuttcap%
\pgfsetroundjoin%
\definecolor{currentfill}{rgb}{0.263663,0.237631,0.518762}%
\pgfsetfillcolor{currentfill}%
\pgfsetfillopacity{0.700000}%
\pgfsetlinewidth{0.000000pt}%
\definecolor{currentstroke}{rgb}{0.000000,0.000000,0.000000}%
\pgfsetstrokecolor{currentstroke}%
\pgfsetdash{}{0pt}%
\pgfpathmoveto{\pgfqpoint{4.306645in}{2.844896in}}%
\pgfpathlineto{\pgfqpoint{4.320100in}{2.838886in}}%
\pgfpathlineto{\pgfqpoint{4.333561in}{2.832956in}}%
\pgfpathlineto{\pgfqpoint{4.347027in}{2.827106in}}%
\pgfpathlineto{\pgfqpoint{4.360498in}{2.821335in}}%
\pgfpathlineto{\pgfqpoint{4.368182in}{2.836335in}}%
\pgfpathlineto{\pgfqpoint{4.375864in}{2.851591in}}%
\pgfpathlineto{\pgfqpoint{4.383545in}{2.867111in}}%
\pgfpathlineto{\pgfqpoint{4.391223in}{2.882900in}}%
\pgfpathlineto{\pgfqpoint{4.377760in}{2.889046in}}%
\pgfpathlineto{\pgfqpoint{4.364303in}{2.895272in}}%
\pgfpathlineto{\pgfqpoint{4.350850in}{2.901577in}}%
\pgfpathlineto{\pgfqpoint{4.337402in}{2.907962in}}%
\pgfpathlineto{\pgfqpoint{4.329716in}{2.891790in}}%
\pgfpathlineto{\pgfqpoint{4.322027in}{2.875893in}}%
\pgfpathlineto{\pgfqpoint{4.314337in}{2.860264in}}%
\pgfpathlineto{\pgfqpoint{4.306645in}{2.844896in}}%
\pgfpathclose%
\pgfusepath{fill}%
\end{pgfscope}%
\begin{pgfscope}%
\pgfpathrectangle{\pgfqpoint{1.150000in}{0.150000in}}{\pgfqpoint{5.700000in}{5.700000in}}%
\pgfusepath{clip}%
\pgfsetbuttcap%
\pgfsetroundjoin%
\definecolor{currentfill}{rgb}{0.274128,0.199721,0.498911}%
\pgfsetfillcolor{currentfill}%
\pgfsetfillopacity{0.700000}%
\pgfsetlinewidth{0.000000pt}%
\definecolor{currentstroke}{rgb}{0.000000,0.000000,0.000000}%
\pgfsetstrokecolor{currentstroke}%
\pgfsetdash{}{0pt}%
\pgfpathmoveto{\pgfqpoint{3.138285in}{2.777355in}}%
\pgfpathlineto{\pgfqpoint{3.151595in}{2.767893in}}%
\pgfpathlineto{\pgfqpoint{3.164907in}{2.758547in}}%
\pgfpathlineto{\pgfqpoint{3.178219in}{2.749316in}}%
\pgfpathlineto{\pgfqpoint{3.191533in}{2.740201in}}%
\pgfpathlineto{\pgfqpoint{3.199527in}{2.752256in}}%
\pgfpathlineto{\pgfqpoint{3.207514in}{2.764444in}}%
\pgfpathlineto{\pgfqpoint{3.215494in}{2.776771in}}%
\pgfpathlineto{\pgfqpoint{3.223468in}{2.789238in}}%
\pgfpathlineto{\pgfqpoint{3.210162in}{2.798529in}}%
\pgfpathlineto{\pgfqpoint{3.196857in}{2.807935in}}%
\pgfpathlineto{\pgfqpoint{3.183553in}{2.817457in}}%
\pgfpathlineto{\pgfqpoint{3.170250in}{2.827095in}}%
\pgfpathlineto{\pgfqpoint{3.162269in}{2.814445in}}%
\pgfpathlineto{\pgfqpoint{3.154281in}{2.801940in}}%
\pgfpathlineto{\pgfqpoint{3.146286in}{2.789578in}}%
\pgfpathlineto{\pgfqpoint{3.138285in}{2.777355in}}%
\pgfpathclose%
\pgfusepath{fill}%
\end{pgfscope}%
\begin{pgfscope}%
\pgfpathrectangle{\pgfqpoint{1.150000in}{0.150000in}}{\pgfqpoint{5.700000in}{5.700000in}}%
\pgfusepath{clip}%
\pgfsetbuttcap%
\pgfsetroundjoin%
\definecolor{currentfill}{rgb}{0.278826,0.175490,0.483397}%
\pgfsetfillcolor{currentfill}%
\pgfsetfillopacity{0.700000}%
\pgfsetlinewidth{0.000000pt}%
\definecolor{currentstroke}{rgb}{0.000000,0.000000,0.000000}%
\pgfsetstrokecolor{currentstroke}%
\pgfsetdash{}{0pt}%
\pgfpathmoveto{\pgfqpoint{3.553249in}{2.723245in}}%
\pgfpathlineto{\pgfqpoint{3.566586in}{2.715767in}}%
\pgfpathlineto{\pgfqpoint{3.579927in}{2.708387in}}%
\pgfpathlineto{\pgfqpoint{3.593270in}{2.701105in}}%
\pgfpathlineto{\pgfqpoint{3.606617in}{2.693920in}}%
\pgfpathlineto{\pgfqpoint{3.614492in}{2.706455in}}%
\pgfpathlineto{\pgfqpoint{3.622361in}{2.719140in}}%
\pgfpathlineto{\pgfqpoint{3.630224in}{2.731979in}}%
\pgfpathlineto{\pgfqpoint{3.638082in}{2.744977in}}%
\pgfpathlineto{\pgfqpoint{3.624742in}{2.752397in}}%
\pgfpathlineto{\pgfqpoint{3.611406in}{2.759915in}}%
\pgfpathlineto{\pgfqpoint{3.598072in}{2.767530in}}%
\pgfpathlineto{\pgfqpoint{3.584742in}{2.775244in}}%
\pgfpathlineto{\pgfqpoint{3.576877in}{2.762003in}}%
\pgfpathlineto{\pgfqpoint{3.569007in}{2.748926in}}%
\pgfpathlineto{\pgfqpoint{3.561131in}{2.736008in}}%
\pgfpathlineto{\pgfqpoint{3.553249in}{2.723245in}}%
\pgfpathclose%
\pgfusepath{fill}%
\end{pgfscope}%
\begin{pgfscope}%
\pgfpathrectangle{\pgfqpoint{1.150000in}{0.150000in}}{\pgfqpoint{5.700000in}{5.700000in}}%
\pgfusepath{clip}%
\pgfsetbuttcap%
\pgfsetroundjoin%
\definecolor{currentfill}{rgb}{0.168126,0.459988,0.558082}%
\pgfsetfillcolor{currentfill}%
\pgfsetfillopacity{0.700000}%
\pgfsetlinewidth{0.000000pt}%
\definecolor{currentstroke}{rgb}{0.000000,0.000000,0.000000}%
\pgfsetstrokecolor{currentstroke}%
\pgfsetdash{}{0pt}%
\pgfpathmoveto{\pgfqpoint{4.960724in}{3.363454in}}%
\pgfpathlineto{\pgfqpoint{4.974258in}{3.355463in}}%
\pgfpathlineto{\pgfqpoint{4.987797in}{3.347545in}}%
\pgfpathlineto{\pgfqpoint{5.001341in}{3.339700in}}%
\pgfpathlineto{\pgfqpoint{5.014890in}{3.331927in}}%
\pgfpathlineto{\pgfqpoint{5.022611in}{3.357330in}}%
\pgfpathlineto{\pgfqpoint{5.030341in}{3.383265in}}%
\pgfpathlineto{\pgfqpoint{5.038080in}{3.409743in}}%
\pgfpathlineto{\pgfqpoint{5.045829in}{3.436776in}}%
\pgfpathlineto{\pgfqpoint{5.032286in}{3.445111in}}%
\pgfpathlineto{\pgfqpoint{5.018748in}{3.453519in}}%
\pgfpathlineto{\pgfqpoint{5.005215in}{3.462000in}}%
\pgfpathlineto{\pgfqpoint{4.991687in}{3.470553in}}%
\pgfpathlineto{\pgfqpoint{4.983932in}{3.442950in}}%
\pgfpathlineto{\pgfqpoint{4.976187in}{3.415906in}}%
\pgfpathlineto{\pgfqpoint{4.968451in}{3.389411in}}%
\pgfpathlineto{\pgfqpoint{4.960724in}{3.363454in}}%
\pgfpathclose%
\pgfusepath{fill}%
\end{pgfscope}%
\begin{pgfscope}%
\pgfpathrectangle{\pgfqpoint{1.150000in}{0.150000in}}{\pgfqpoint{5.700000in}{5.700000in}}%
\pgfusepath{clip}%
\pgfsetbuttcap%
\pgfsetroundjoin%
\definecolor{currentfill}{rgb}{0.153364,0.497000,0.557724}%
\pgfsetfillcolor{currentfill}%
\pgfsetfillopacity{0.700000}%
\pgfsetlinewidth{0.000000pt}%
\definecolor{currentstroke}{rgb}{0.000000,0.000000,0.000000}%
\pgfsetstrokecolor{currentstroke}%
\pgfsetdash{}{0pt}%
\pgfpathmoveto{\pgfqpoint{4.991687in}{3.470553in}}%
\pgfpathlineto{\pgfqpoint{5.005215in}{3.462000in}}%
\pgfpathlineto{\pgfqpoint{5.018748in}{3.453519in}}%
\pgfpathlineto{\pgfqpoint{5.032286in}{3.445111in}}%
\pgfpathlineto{\pgfqpoint{5.045829in}{3.436776in}}%
\pgfpathlineto{\pgfqpoint{5.053589in}{3.464374in}}%
\pgfpathlineto{\pgfqpoint{5.061359in}{3.492549in}}%
\pgfpathlineto{\pgfqpoint{5.069140in}{3.521312in}}%
\pgfpathlineto{\pgfqpoint{5.055601in}{3.530084in}}%
\pgfpathlineto{\pgfqpoint{5.042067in}{3.538929in}}%
\pgfpathlineto{\pgfqpoint{5.028538in}{3.547847in}}%
\pgfpathlineto{\pgfqpoint{5.015013in}{3.556838in}}%
\pgfpathlineto{\pgfqpoint{5.007227in}{3.527486in}}%
\pgfpathlineto{\pgfqpoint{4.999452in}{3.498728in}}%
\pgfpathlineto{\pgfqpoint{4.991687in}{3.470553in}}%
\pgfpathclose%
\pgfusepath{fill}%
\end{pgfscope}%
\begin{pgfscope}%
\pgfpathrectangle{\pgfqpoint{1.150000in}{0.150000in}}{\pgfqpoint{5.700000in}{5.700000in}}%
\pgfusepath{clip}%
\pgfsetbuttcap%
\pgfsetroundjoin%
\definecolor{currentfill}{rgb}{0.267968,0.223549,0.512008}%
\pgfsetfillcolor{currentfill}%
\pgfsetfillopacity{0.700000}%
\pgfsetlinewidth{0.000000pt}%
\definecolor{currentstroke}{rgb}{0.000000,0.000000,0.000000}%
\pgfsetstrokecolor{currentstroke}%
\pgfsetdash{}{0pt}%
\pgfpathmoveto{\pgfqpoint{4.222050in}{2.809323in}}%
\pgfpathlineto{\pgfqpoint{4.235493in}{2.803346in}}%
\pgfpathlineto{\pgfqpoint{4.248941in}{2.797449in}}%
\pgfpathlineto{\pgfqpoint{4.262394in}{2.791634in}}%
\pgfpathlineto{\pgfqpoint{4.275852in}{2.785899in}}%
\pgfpathlineto{\pgfqpoint{4.283554in}{2.800290in}}%
\pgfpathlineto{\pgfqpoint{4.291254in}{2.814915in}}%
\pgfpathlineto{\pgfqpoint{4.298950in}{2.829782in}}%
\pgfpathlineto{\pgfqpoint{4.306645in}{2.844896in}}%
\pgfpathlineto{\pgfqpoint{4.293195in}{2.850986in}}%
\pgfpathlineto{\pgfqpoint{4.279750in}{2.857157in}}%
\pgfpathlineto{\pgfqpoint{4.266309in}{2.863409in}}%
\pgfpathlineto{\pgfqpoint{4.252874in}{2.869742in}}%
\pgfpathlineto{\pgfqpoint{4.245172in}{2.854265in}}%
\pgfpathlineto{\pgfqpoint{4.237467in}{2.839041in}}%
\pgfpathlineto{\pgfqpoint{4.229760in}{2.824062in}}%
\pgfpathlineto{\pgfqpoint{4.222050in}{2.809323in}}%
\pgfpathclose%
\pgfusepath{fill}%
\end{pgfscope}%
\begin{pgfscope}%
\pgfpathrectangle{\pgfqpoint{1.150000in}{0.150000in}}{\pgfqpoint{5.700000in}{5.700000in}}%
\pgfusepath{clip}%
\pgfsetbuttcap%
\pgfsetroundjoin%
\definecolor{currentfill}{rgb}{0.210503,0.363727,0.552206}%
\pgfsetfillcolor{currentfill}%
\pgfsetfillopacity{0.700000}%
\pgfsetlinewidth{0.000000pt}%
\definecolor{currentstroke}{rgb}{0.000000,0.000000,0.000000}%
\pgfsetstrokecolor{currentstroke}%
\pgfsetdash{}{0pt}%
\pgfpathmoveto{\pgfqpoint{4.814371in}{3.116617in}}%
\pgfpathlineto{\pgfqpoint{4.827906in}{3.109891in}}%
\pgfpathlineto{\pgfqpoint{4.841447in}{3.103238in}}%
\pgfpathlineto{\pgfqpoint{4.854993in}{3.096658in}}%
\pgfpathlineto{\pgfqpoint{4.868545in}{3.090151in}}%
\pgfpathlineto{\pgfqpoint{4.876196in}{3.110454in}}%
\pgfpathlineto{\pgfqpoint{4.883851in}{3.131173in}}%
\pgfpathlineto{\pgfqpoint{4.891511in}{3.152320in}}%
\pgfpathlineto{\pgfqpoint{4.899176in}{3.173903in}}%
\pgfpathlineto{\pgfqpoint{4.885633in}{3.180907in}}%
\pgfpathlineto{\pgfqpoint{4.872095in}{3.187985in}}%
\pgfpathlineto{\pgfqpoint{4.858563in}{3.195136in}}%
\pgfpathlineto{\pgfqpoint{4.845035in}{3.202360in}}%
\pgfpathlineto{\pgfqpoint{4.837362in}{3.180271in}}%
\pgfpathlineto{\pgfqpoint{4.829694in}{3.158624in}}%
\pgfpathlineto{\pgfqpoint{4.822031in}{3.137409in}}%
\pgfpathlineto{\pgfqpoint{4.814371in}{3.116617in}}%
\pgfpathclose%
\pgfusepath{fill}%
\end{pgfscope}%
\begin{pgfscope}%
\pgfpathrectangle{\pgfqpoint{1.150000in}{0.150000in}}{\pgfqpoint{5.700000in}{5.700000in}}%
\pgfusepath{clip}%
\pgfsetbuttcap%
\pgfsetroundjoin%
\definecolor{currentfill}{rgb}{0.183898,0.422383,0.556944}%
\pgfsetfillcolor{currentfill}%
\pgfsetfillopacity{0.700000}%
\pgfsetlinewidth{0.000000pt}%
\definecolor{currentstroke}{rgb}{0.000000,0.000000,0.000000}%
\pgfsetstrokecolor{currentstroke}%
\pgfsetdash{}{0pt}%
\pgfpathmoveto{\pgfqpoint{4.929895in}{3.264793in}}%
\pgfpathlineto{\pgfqpoint{4.943436in}{3.257343in}}%
\pgfpathlineto{\pgfqpoint{4.956982in}{3.249965in}}%
\pgfpathlineto{\pgfqpoint{4.970533in}{3.242660in}}%
\pgfpathlineto{\pgfqpoint{4.984090in}{3.235427in}}%
\pgfpathlineto{\pgfqpoint{4.991779in}{3.258806in}}%
\pgfpathlineto{\pgfqpoint{4.999475in}{3.282676in}}%
\pgfpathlineto{\pgfqpoint{5.007178in}{3.307046in}}%
\pgfpathlineto{\pgfqpoint{5.014890in}{3.331927in}}%
\pgfpathlineto{\pgfqpoint{5.001341in}{3.339700in}}%
\pgfpathlineto{\pgfqpoint{4.987797in}{3.347545in}}%
\pgfpathlineto{\pgfqpoint{4.974258in}{3.355463in}}%
\pgfpathlineto{\pgfqpoint{4.960724in}{3.363454in}}%
\pgfpathlineto{\pgfqpoint{4.953005in}{3.338024in}}%
\pgfpathlineto{\pgfqpoint{4.945294in}{3.313111in}}%
\pgfpathlineto{\pgfqpoint{4.937591in}{3.288704in}}%
\pgfpathlineto{\pgfqpoint{4.929895in}{3.264793in}}%
\pgfpathclose%
\pgfusepath{fill}%
\end{pgfscope}%
\begin{pgfscope}%
\pgfpathrectangle{\pgfqpoint{1.150000in}{0.150000in}}{\pgfqpoint{5.700000in}{5.700000in}}%
\pgfusepath{clip}%
\pgfsetbuttcap%
\pgfsetroundjoin%
\definecolor{currentfill}{rgb}{0.276194,0.190074,0.493001}%
\pgfsetfillcolor{currentfill}%
\pgfsetfillopacity{0.700000}%
\pgfsetlinewidth{0.000000pt}%
\definecolor{currentstroke}{rgb}{0.000000,0.000000,0.000000}%
\pgfsetstrokecolor{currentstroke}%
\pgfsetdash{}{0pt}%
\pgfpathmoveto{\pgfqpoint{3.914462in}{2.740764in}}%
\pgfpathlineto{\pgfqpoint{3.927853in}{2.734379in}}%
\pgfpathlineto{\pgfqpoint{3.941247in}{2.728081in}}%
\pgfpathlineto{\pgfqpoint{3.954646in}{2.721871in}}%
\pgfpathlineto{\pgfqpoint{3.968050in}{2.715748in}}%
\pgfpathlineto{\pgfqpoint{3.975827in}{2.728952in}}%
\pgfpathlineto{\pgfqpoint{3.983600in}{2.742339in}}%
\pgfpathlineto{\pgfqpoint{3.991368in}{2.755915in}}%
\pgfpathlineto{\pgfqpoint{3.999133in}{2.769685in}}%
\pgfpathlineto{\pgfqpoint{3.985737in}{2.776103in}}%
\pgfpathlineto{\pgfqpoint{3.972345in}{2.782609in}}%
\pgfpathlineto{\pgfqpoint{3.958957in}{2.789202in}}%
\pgfpathlineto{\pgfqpoint{3.945574in}{2.795882in}}%
\pgfpathlineto{\pgfqpoint{3.937803in}{2.781809in}}%
\pgfpathlineto{\pgfqpoint{3.930027in}{2.767935in}}%
\pgfpathlineto{\pgfqpoint{3.922247in}{2.754255in}}%
\pgfpathlineto{\pgfqpoint{3.914462in}{2.740764in}}%
\pgfpathclose%
\pgfusepath{fill}%
\end{pgfscope}%
\begin{pgfscope}%
\pgfpathrectangle{\pgfqpoint{1.150000in}{0.150000in}}{\pgfqpoint{5.700000in}{5.700000in}}%
\pgfusepath{clip}%
\pgfsetbuttcap%
\pgfsetroundjoin%
\definecolor{currentfill}{rgb}{0.278826,0.175490,0.483397}%
\pgfsetfillcolor{currentfill}%
\pgfsetfillopacity{0.700000}%
\pgfsetlinewidth{0.000000pt}%
\definecolor{currentstroke}{rgb}{0.000000,0.000000,0.000000}%
\pgfsetstrokecolor{currentstroke}%
\pgfsetdash{}{0pt}%
\pgfpathmoveto{\pgfqpoint{3.691475in}{2.716253in}}%
\pgfpathlineto{\pgfqpoint{3.704832in}{2.709309in}}%
\pgfpathlineto{\pgfqpoint{3.718193in}{2.702459in}}%
\pgfpathlineto{\pgfqpoint{3.731557in}{2.695702in}}%
\pgfpathlineto{\pgfqpoint{3.744925in}{2.689037in}}%
\pgfpathlineto{\pgfqpoint{3.752763in}{2.701702in}}%
\pgfpathlineto{\pgfqpoint{3.760596in}{2.714526in}}%
\pgfpathlineto{\pgfqpoint{3.768424in}{2.727512in}}%
\pgfpathlineto{\pgfqpoint{3.776247in}{2.740666in}}%
\pgfpathlineto{\pgfqpoint{3.762886in}{2.747586in}}%
\pgfpathlineto{\pgfqpoint{3.749529in}{2.754599in}}%
\pgfpathlineto{\pgfqpoint{3.736175in}{2.761704in}}%
\pgfpathlineto{\pgfqpoint{3.722825in}{2.768904in}}%
\pgfpathlineto{\pgfqpoint{3.714996in}{2.755487in}}%
\pgfpathlineto{\pgfqpoint{3.707161in}{2.742243in}}%
\pgfpathlineto{\pgfqpoint{3.699320in}{2.729166in}}%
\pgfpathlineto{\pgfqpoint{3.691475in}{2.716253in}}%
\pgfpathclose%
\pgfusepath{fill}%
\end{pgfscope}%
\begin{pgfscope}%
\pgfpathrectangle{\pgfqpoint{1.150000in}{0.150000in}}{\pgfqpoint{5.700000in}{5.700000in}}%
\pgfusepath{clip}%
\pgfsetbuttcap%
\pgfsetroundjoin%
\definecolor{currentfill}{rgb}{0.269308,0.218818,0.509577}%
\pgfsetfillcolor{currentfill}%
\pgfsetfillopacity{0.700000}%
\pgfsetlinewidth{0.000000pt}%
\definecolor{currentstroke}{rgb}{0.000000,0.000000,0.000000}%
\pgfsetstrokecolor{currentstroke}%
\pgfsetdash{}{0pt}%
\pgfpathmoveto{\pgfqpoint{2.999682in}{2.808545in}}%
\pgfpathlineto{\pgfqpoint{3.012996in}{2.798273in}}%
\pgfpathlineto{\pgfqpoint{3.026311in}{2.788125in}}%
\pgfpathlineto{\pgfqpoint{3.039626in}{2.778100in}}%
\pgfpathlineto{\pgfqpoint{3.052941in}{2.768197in}}%
\pgfpathlineto{\pgfqpoint{3.060979in}{2.780044in}}%
\pgfpathlineto{\pgfqpoint{3.069010in}{2.792023in}}%
\pgfpathlineto{\pgfqpoint{3.077034in}{2.804136in}}%
\pgfpathlineto{\pgfqpoint{3.085051in}{2.816387in}}%
\pgfpathlineto{\pgfqpoint{3.071743in}{2.826445in}}%
\pgfpathlineto{\pgfqpoint{3.058437in}{2.836626in}}%
\pgfpathlineto{\pgfqpoint{3.045130in}{2.846929in}}%
\pgfpathlineto{\pgfqpoint{3.031824in}{2.857357in}}%
\pgfpathlineto{\pgfqpoint{3.023799in}{2.844944in}}%
\pgfpathlineto{\pgfqpoint{3.015767in}{2.832672in}}%
\pgfpathlineto{\pgfqpoint{3.007728in}{2.820541in}}%
\pgfpathlineto{\pgfqpoint{2.999682in}{2.808545in}}%
\pgfpathclose%
\pgfusepath{fill}%
\end{pgfscope}%
\begin{pgfscope}%
\pgfpathrectangle{\pgfqpoint{1.150000in}{0.150000in}}{\pgfqpoint{5.700000in}{5.700000in}}%
\pgfusepath{clip}%
\pgfsetbuttcap%
\pgfsetroundjoin%
\definecolor{currentfill}{rgb}{0.257322,0.256130,0.526563}%
\pgfsetfillcolor{currentfill}%
\pgfsetfillopacity{0.700000}%
\pgfsetlinewidth{0.000000pt}%
\definecolor{currentstroke}{rgb}{0.000000,0.000000,0.000000}%
\pgfsetstrokecolor{currentstroke}%
\pgfsetdash{}{0pt}%
\pgfpathmoveto{\pgfqpoint{2.807505in}{2.893596in}}%
\pgfpathlineto{\pgfqpoint{2.820836in}{2.881882in}}%
\pgfpathlineto{\pgfqpoint{2.834167in}{2.870306in}}%
\pgfpathlineto{\pgfqpoint{2.847496in}{2.858867in}}%
\pgfpathlineto{\pgfqpoint{2.860823in}{2.847563in}}%
\pgfpathlineto{\pgfqpoint{2.868918in}{2.859298in}}%
\pgfpathlineto{\pgfqpoint{2.877005in}{2.871167in}}%
\pgfpathlineto{\pgfqpoint{2.885084in}{2.883175in}}%
\pgfpathlineto{\pgfqpoint{2.893155in}{2.895323in}}%
\pgfpathlineto{\pgfqpoint{2.879836in}{2.906762in}}%
\pgfpathlineto{\pgfqpoint{2.866516in}{2.918337in}}%
\pgfpathlineto{\pgfqpoint{2.853195in}{2.930049in}}%
\pgfpathlineto{\pgfqpoint{2.839872in}{2.941898in}}%
\pgfpathlineto{\pgfqpoint{2.831792in}{2.929607in}}%
\pgfpathlineto{\pgfqpoint{2.823704in}{2.917461in}}%
\pgfpathlineto{\pgfqpoint{2.815609in}{2.905458in}}%
\pgfpathlineto{\pgfqpoint{2.807505in}{2.893596in}}%
\pgfpathclose%
\pgfusepath{fill}%
\end{pgfscope}%
\begin{pgfscope}%
\pgfpathrectangle{\pgfqpoint{1.150000in}{0.150000in}}{\pgfqpoint{5.700000in}{5.700000in}}%
\pgfusepath{clip}%
\pgfsetbuttcap%
\pgfsetroundjoin%
\definecolor{currentfill}{rgb}{0.271828,0.209303,0.504434}%
\pgfsetfillcolor{currentfill}%
\pgfsetfillopacity{0.700000}%
\pgfsetlinewidth{0.000000pt}%
\definecolor{currentstroke}{rgb}{0.000000,0.000000,0.000000}%
\pgfsetstrokecolor{currentstroke}%
\pgfsetdash{}{0pt}%
\pgfpathmoveto{\pgfqpoint{4.137426in}{2.776025in}}%
\pgfpathlineto{\pgfqpoint{4.150857in}{2.770054in}}%
\pgfpathlineto{\pgfqpoint{4.164293in}{2.764166in}}%
\pgfpathlineto{\pgfqpoint{4.177733in}{2.758359in}}%
\pgfpathlineto{\pgfqpoint{4.191179in}{2.752635in}}%
\pgfpathlineto{\pgfqpoint{4.198902in}{2.766480in}}%
\pgfpathlineto{\pgfqpoint{4.206621in}{2.780538in}}%
\pgfpathlineto{\pgfqpoint{4.214337in}{2.794817in}}%
\pgfpathlineto{\pgfqpoint{4.222050in}{2.809323in}}%
\pgfpathlineto{\pgfqpoint{4.208612in}{2.815383in}}%
\pgfpathlineto{\pgfqpoint{4.195179in}{2.821524in}}%
\pgfpathlineto{\pgfqpoint{4.181751in}{2.827748in}}%
\pgfpathlineto{\pgfqpoint{4.168328in}{2.834055in}}%
\pgfpathlineto{\pgfqpoint{4.160607in}{2.819206in}}%
\pgfpathlineto{\pgfqpoint{4.152883in}{2.804589in}}%
\pgfpathlineto{\pgfqpoint{4.145156in}{2.790198in}}%
\pgfpathlineto{\pgfqpoint{4.137426in}{2.776025in}}%
\pgfpathclose%
\pgfusepath{fill}%
\end{pgfscope}%
\begin{pgfscope}%
\pgfpathrectangle{\pgfqpoint{1.150000in}{0.150000in}}{\pgfqpoint{5.700000in}{5.700000in}}%
\pgfusepath{clip}%
\pgfsetbuttcap%
\pgfsetroundjoin%
\definecolor{currentfill}{rgb}{0.199430,0.387607,0.554642}%
\pgfsetfillcolor{currentfill}%
\pgfsetfillopacity{0.700000}%
\pgfsetlinewidth{0.000000pt}%
\definecolor{currentstroke}{rgb}{0.000000,0.000000,0.000000}%
\pgfsetstrokecolor{currentstroke}%
\pgfsetdash{}{0pt}%
\pgfpathmoveto{\pgfqpoint{4.899176in}{3.173903in}}%
\pgfpathlineto{\pgfqpoint{4.912725in}{3.166972in}}%
\pgfpathlineto{\pgfqpoint{4.926279in}{3.160113in}}%
\pgfpathlineto{\pgfqpoint{4.939839in}{3.153326in}}%
\pgfpathlineto{\pgfqpoint{4.953405in}{3.146612in}}%
\pgfpathlineto{\pgfqpoint{4.961067in}{3.168131in}}%
\pgfpathlineto{\pgfqpoint{4.968735in}{3.190099in}}%
\pgfpathlineto{\pgfqpoint{4.976409in}{3.212528in}}%
\pgfpathlineto{\pgfqpoint{4.984090in}{3.235427in}}%
\pgfpathlineto{\pgfqpoint{4.970533in}{3.242660in}}%
\pgfpathlineto{\pgfqpoint{4.956982in}{3.249965in}}%
\pgfpathlineto{\pgfqpoint{4.943436in}{3.257343in}}%
\pgfpathlineto{\pgfqpoint{4.929895in}{3.264793in}}%
\pgfpathlineto{\pgfqpoint{4.922206in}{3.241367in}}%
\pgfpathlineto{\pgfqpoint{4.914523in}{3.218417in}}%
\pgfpathlineto{\pgfqpoint{4.906847in}{3.195932in}}%
\pgfpathlineto{\pgfqpoint{4.899176in}{3.173903in}}%
\pgfpathclose%
\pgfusepath{fill}%
\end{pgfscope}%
\begin{pgfscope}%
\pgfpathrectangle{\pgfqpoint{1.150000in}{0.150000in}}{\pgfqpoint{5.700000in}{5.700000in}}%
\pgfusepath{clip}%
\pgfsetbuttcap%
\pgfsetroundjoin%
\definecolor{currentfill}{rgb}{0.278826,0.175490,0.483397}%
\pgfsetfillcolor{currentfill}%
\pgfsetfillopacity{0.700000}%
\pgfsetlinewidth{0.000000pt}%
\definecolor{currentstroke}{rgb}{0.000000,0.000000,0.000000}%
\pgfsetstrokecolor{currentstroke}%
\pgfsetdash{}{0pt}%
\pgfpathmoveto{\pgfqpoint{3.329971in}{2.718929in}}%
\pgfpathlineto{\pgfqpoint{3.343292in}{2.710632in}}%
\pgfpathlineto{\pgfqpoint{3.356615in}{2.702440in}}%
\pgfpathlineto{\pgfqpoint{3.369940in}{2.694355in}}%
\pgfpathlineto{\pgfqpoint{3.383268in}{2.686374in}}%
\pgfpathlineto{\pgfqpoint{3.391213in}{2.698421in}}%
\pgfpathlineto{\pgfqpoint{3.399151in}{2.710601in}}%
\pgfpathlineto{\pgfqpoint{3.407083in}{2.722917in}}%
\pgfpathlineto{\pgfqpoint{3.415009in}{2.735374in}}%
\pgfpathlineto{\pgfqpoint{3.401688in}{2.743550in}}%
\pgfpathlineto{\pgfqpoint{3.388370in}{2.751832in}}%
\pgfpathlineto{\pgfqpoint{3.375055in}{2.760219in}}%
\pgfpathlineto{\pgfqpoint{3.361741in}{2.768712in}}%
\pgfpathlineto{\pgfqpoint{3.353808in}{2.756052in}}%
\pgfpathlineto{\pgfqpoint{3.345868in}{2.743537in}}%
\pgfpathlineto{\pgfqpoint{3.337923in}{2.731164in}}%
\pgfpathlineto{\pgfqpoint{3.329971in}{2.718929in}}%
\pgfpathclose%
\pgfusepath{fill}%
\end{pgfscope}%
\begin{pgfscope}%
\pgfpathrectangle{\pgfqpoint{1.150000in}{0.150000in}}{\pgfqpoint{5.700000in}{5.700000in}}%
\pgfusepath{clip}%
\pgfsetbuttcap%
\pgfsetroundjoin%
\definecolor{currentfill}{rgb}{0.276194,0.190074,0.493001}%
\pgfsetfillcolor{currentfill}%
\pgfsetfillopacity{0.700000}%
\pgfsetlinewidth{0.000000pt}%
\definecolor{currentstroke}{rgb}{0.000000,0.000000,0.000000}%
\pgfsetstrokecolor{currentstroke}%
\pgfsetdash{}{0pt}%
\pgfpathmoveto{\pgfqpoint{3.191533in}{2.740201in}}%
\pgfpathlineto{\pgfqpoint{3.204848in}{2.731198in}}%
\pgfpathlineto{\pgfqpoint{3.218165in}{2.722309in}}%
\pgfpathlineto{\pgfqpoint{3.231483in}{2.713532in}}%
\pgfpathlineto{\pgfqpoint{3.244802in}{2.704866in}}%
\pgfpathlineto{\pgfqpoint{3.252788in}{2.716753in}}%
\pgfpathlineto{\pgfqpoint{3.260767in}{2.728769in}}%
\pgfpathlineto{\pgfqpoint{3.268740in}{2.740917in}}%
\pgfpathlineto{\pgfqpoint{3.276706in}{2.753201in}}%
\pgfpathlineto{\pgfqpoint{3.263394in}{2.762042in}}%
\pgfpathlineto{\pgfqpoint{3.250084in}{2.770995in}}%
\pgfpathlineto{\pgfqpoint{3.236775in}{2.780060in}}%
\pgfpathlineto{\pgfqpoint{3.223468in}{2.789238in}}%
\pgfpathlineto{\pgfqpoint{3.215494in}{2.776771in}}%
\pgfpathlineto{\pgfqpoint{3.207514in}{2.764444in}}%
\pgfpathlineto{\pgfqpoint{3.199527in}{2.752256in}}%
\pgfpathlineto{\pgfqpoint{3.191533in}{2.740201in}}%
\pgfpathclose%
\pgfusepath{fill}%
\end{pgfscope}%
\begin{pgfscope}%
\pgfpathrectangle{\pgfqpoint{1.150000in}{0.150000in}}{\pgfqpoint{5.700000in}{5.700000in}}%
\pgfusepath{clip}%
\pgfsetbuttcap%
\pgfsetroundjoin%
\definecolor{currentfill}{rgb}{0.279574,0.170599,0.479997}%
\pgfsetfillcolor{currentfill}%
\pgfsetfillopacity{0.700000}%
\pgfsetlinewidth{0.000000pt}%
\definecolor{currentstroke}{rgb}{0.000000,0.000000,0.000000}%
\pgfsetstrokecolor{currentstroke}%
\pgfsetdash{}{0pt}%
\pgfpathmoveto{\pgfqpoint{3.468315in}{2.703705in}}%
\pgfpathlineto{\pgfqpoint{3.481648in}{2.696043in}}%
\pgfpathlineto{\pgfqpoint{3.494983in}{2.688482in}}%
\pgfpathlineto{\pgfqpoint{3.508322in}{2.681021in}}%
\pgfpathlineto{\pgfqpoint{3.521664in}{2.673659in}}%
\pgfpathlineto{\pgfqpoint{3.529569in}{2.685844in}}%
\pgfpathlineto{\pgfqpoint{3.537468in}{2.698168in}}%
\pgfpathlineto{\pgfqpoint{3.545361in}{2.710633in}}%
\pgfpathlineto{\pgfqpoint{3.553249in}{2.723245in}}%
\pgfpathlineto{\pgfqpoint{3.539915in}{2.730822in}}%
\pgfpathlineto{\pgfqpoint{3.526583in}{2.738499in}}%
\pgfpathlineto{\pgfqpoint{3.513254in}{2.746275in}}%
\pgfpathlineto{\pgfqpoint{3.499929in}{2.754153in}}%
\pgfpathlineto{\pgfqpoint{3.492034in}{2.741318in}}%
\pgfpathlineto{\pgfqpoint{3.484133in}{2.728634in}}%
\pgfpathlineto{\pgfqpoint{3.476227in}{2.716098in}}%
\pgfpathlineto{\pgfqpoint{3.468315in}{2.703705in}}%
\pgfpathclose%
\pgfusepath{fill}%
\end{pgfscope}%
\begin{pgfscope}%
\pgfpathrectangle{\pgfqpoint{1.150000in}{0.150000in}}{\pgfqpoint{5.700000in}{5.700000in}}%
\pgfusepath{clip}%
\pgfsetbuttcap%
\pgfsetroundjoin%
\definecolor{currentfill}{rgb}{0.243113,0.292092,0.538516}%
\pgfsetfillcolor{currentfill}%
\pgfsetfillopacity{0.700000}%
\pgfsetlinewidth{0.000000pt}%
\definecolor{currentstroke}{rgb}{0.000000,0.000000,0.000000}%
\pgfsetstrokecolor{currentstroke}%
\pgfsetdash{}{0pt}%
\pgfpathmoveto{\pgfqpoint{4.614395in}{2.942380in}}%
\pgfpathlineto{\pgfqpoint{4.627912in}{2.936427in}}%
\pgfpathlineto{\pgfqpoint{4.641434in}{2.930549in}}%
\pgfpathlineto{\pgfqpoint{4.654962in}{2.924747in}}%
\pgfpathlineto{\pgfqpoint{4.668496in}{2.919019in}}%
\pgfpathlineto{\pgfqpoint{4.676136in}{2.935852in}}%
\pgfpathlineto{\pgfqpoint{4.683776in}{2.953010in}}%
\pgfpathlineto{\pgfqpoint{4.691418in}{2.970501in}}%
\pgfpathlineto{\pgfqpoint{4.699061in}{2.988332in}}%
\pgfpathlineto{\pgfqpoint{4.685536in}{2.994496in}}%
\pgfpathlineto{\pgfqpoint{4.672017in}{3.000735in}}%
\pgfpathlineto{\pgfqpoint{4.658504in}{3.007049in}}%
\pgfpathlineto{\pgfqpoint{4.644996in}{3.013438in}}%
\pgfpathlineto{\pgfqpoint{4.637344in}{2.995163in}}%
\pgfpathlineto{\pgfqpoint{4.629693in}{2.977234in}}%
\pgfpathlineto{\pgfqpoint{4.622044in}{2.959642in}}%
\pgfpathlineto{\pgfqpoint{4.614395in}{2.942380in}}%
\pgfpathclose%
\pgfusepath{fill}%
\end{pgfscope}%
\begin{pgfscope}%
\pgfpathrectangle{\pgfqpoint{1.150000in}{0.150000in}}{\pgfqpoint{5.700000in}{5.700000in}}%
\pgfusepath{clip}%
\pgfsetbuttcap%
\pgfsetroundjoin%
\definecolor{currentfill}{rgb}{0.252194,0.269783,0.531579}%
\pgfsetfillcolor{currentfill}%
\pgfsetfillopacity{0.700000}%
\pgfsetlinewidth{0.000000pt}%
\definecolor{currentstroke}{rgb}{0.000000,0.000000,0.000000}%
\pgfsetstrokecolor{currentstroke}%
\pgfsetdash{}{0pt}%
\pgfpathmoveto{\pgfqpoint{4.529756in}{2.899381in}}%
\pgfpathlineto{\pgfqpoint{4.543259in}{2.893539in}}%
\pgfpathlineto{\pgfqpoint{4.556768in}{2.887774in}}%
\pgfpathlineto{\pgfqpoint{4.570282in}{2.882085in}}%
\pgfpathlineto{\pgfqpoint{4.583802in}{2.876472in}}%
\pgfpathlineto{\pgfqpoint{4.591451in}{2.892494in}}%
\pgfpathlineto{\pgfqpoint{4.599099in}{2.908814in}}%
\pgfpathlineto{\pgfqpoint{4.606747in}{2.925440in}}%
\pgfpathlineto{\pgfqpoint{4.614395in}{2.942380in}}%
\pgfpathlineto{\pgfqpoint{4.600884in}{2.948409in}}%
\pgfpathlineto{\pgfqpoint{4.587378in}{2.954514in}}%
\pgfpathlineto{\pgfqpoint{4.573878in}{2.960695in}}%
\pgfpathlineto{\pgfqpoint{4.560384in}{2.966952in}}%
\pgfpathlineto{\pgfqpoint{4.552727in}{2.949588in}}%
\pgfpathlineto{\pgfqpoint{4.545070in}{2.932544in}}%
\pgfpathlineto{\pgfqpoint{4.537413in}{2.915811in}}%
\pgfpathlineto{\pgfqpoint{4.529756in}{2.899381in}}%
\pgfpathclose%
\pgfusepath{fill}%
\end{pgfscope}%
\begin{pgfscope}%
\pgfpathrectangle{\pgfqpoint{1.150000in}{0.150000in}}{\pgfqpoint{5.700000in}{5.700000in}}%
\pgfusepath{clip}%
\pgfsetbuttcap%
\pgfsetroundjoin%
\definecolor{currentfill}{rgb}{0.278012,0.180367,0.486697}%
\pgfsetfillcolor{currentfill}%
\pgfsetfillopacity{0.700000}%
\pgfsetlinewidth{0.000000pt}%
\definecolor{currentstroke}{rgb}{0.000000,0.000000,0.000000}%
\pgfsetstrokecolor{currentstroke}%
\pgfsetdash{}{0pt}%
\pgfpathmoveto{\pgfqpoint{3.829730in}{2.713900in}}%
\pgfpathlineto{\pgfqpoint{3.843110in}{2.707435in}}%
\pgfpathlineto{\pgfqpoint{3.856495in}{2.701060in}}%
\pgfpathlineto{\pgfqpoint{3.869884in}{2.694774in}}%
\pgfpathlineto{\pgfqpoint{3.883278in}{2.688576in}}%
\pgfpathlineto{\pgfqpoint{3.891081in}{2.701366in}}%
\pgfpathlineto{\pgfqpoint{3.898879in}{2.714324in}}%
\pgfpathlineto{\pgfqpoint{3.906673in}{2.727455in}}%
\pgfpathlineto{\pgfqpoint{3.914462in}{2.740764in}}%
\pgfpathlineto{\pgfqpoint{3.901076in}{2.747237in}}%
\pgfpathlineto{\pgfqpoint{3.887694in}{2.753799in}}%
\pgfpathlineto{\pgfqpoint{3.874317in}{2.760449in}}%
\pgfpathlineto{\pgfqpoint{3.860943in}{2.767190in}}%
\pgfpathlineto{\pgfqpoint{3.853147in}{2.753598in}}%
\pgfpathlineto{\pgfqpoint{3.845346in}{2.740190in}}%
\pgfpathlineto{\pgfqpoint{3.837540in}{2.726959in}}%
\pgfpathlineto{\pgfqpoint{3.829730in}{2.713900in}}%
\pgfpathclose%
\pgfusepath{fill}%
\end{pgfscope}%
\begin{pgfscope}%
\pgfpathrectangle{\pgfqpoint{1.150000in}{0.150000in}}{\pgfqpoint{5.700000in}{5.700000in}}%
\pgfusepath{clip}%
\pgfsetbuttcap%
\pgfsetroundjoin%
\definecolor{currentfill}{rgb}{0.235526,0.309527,0.542944}%
\pgfsetfillcolor{currentfill}%
\pgfsetfillopacity{0.700000}%
\pgfsetlinewidth{0.000000pt}%
\definecolor{currentstroke}{rgb}{0.000000,0.000000,0.000000}%
\pgfsetstrokecolor{currentstroke}%
\pgfsetdash{}{0pt}%
\pgfpathmoveto{\pgfqpoint{4.699061in}{2.988332in}}%
\pgfpathlineto{\pgfqpoint{4.712591in}{2.982243in}}%
\pgfpathlineto{\pgfqpoint{4.726127in}{2.976228in}}%
\pgfpathlineto{\pgfqpoint{4.739669in}{2.970288in}}%
\pgfpathlineto{\pgfqpoint{4.753216in}{2.964421in}}%
\pgfpathlineto{\pgfqpoint{4.760852in}{2.982152in}}%
\pgfpathlineto{\pgfqpoint{4.768489in}{3.000236in}}%
\pgfpathlineto{\pgfqpoint{4.776129in}{3.018681in}}%
\pgfpathlineto{\pgfqpoint{4.783771in}{3.037495in}}%
\pgfpathlineto{\pgfqpoint{4.770233in}{3.043818in}}%
\pgfpathlineto{\pgfqpoint{4.756700in}{3.050215in}}%
\pgfpathlineto{\pgfqpoint{4.743173in}{3.056686in}}%
\pgfpathlineto{\pgfqpoint{4.729652in}{3.063232in}}%
\pgfpathlineto{\pgfqpoint{4.722001in}{3.043954in}}%
\pgfpathlineto{\pgfqpoint{4.714352in}{3.025051in}}%
\pgfpathlineto{\pgfqpoint{4.706705in}{3.006513in}}%
\pgfpathlineto{\pgfqpoint{4.699061in}{2.988332in}}%
\pgfpathclose%
\pgfusepath{fill}%
\end{pgfscope}%
\begin{pgfscope}%
\pgfpathrectangle{\pgfqpoint{1.150000in}{0.150000in}}{\pgfqpoint{5.700000in}{5.700000in}}%
\pgfusepath{clip}%
\pgfsetbuttcap%
\pgfsetroundjoin%
\definecolor{currentfill}{rgb}{0.258965,0.251537,0.524736}%
\pgfsetfillcolor{currentfill}%
\pgfsetfillopacity{0.700000}%
\pgfsetlinewidth{0.000000pt}%
\definecolor{currentstroke}{rgb}{0.000000,0.000000,0.000000}%
\pgfsetstrokecolor{currentstroke}%
\pgfsetdash{}{0pt}%
\pgfpathmoveto{\pgfqpoint{4.445128in}{2.859101in}}%
\pgfpathlineto{\pgfqpoint{4.458618in}{2.853347in}}%
\pgfpathlineto{\pgfqpoint{4.472113in}{2.847670in}}%
\pgfpathlineto{\pgfqpoint{4.485614in}{2.842071in}}%
\pgfpathlineto{\pgfqpoint{4.499121in}{2.836548in}}%
\pgfpathlineto{\pgfqpoint{4.506781in}{2.851838in}}%
\pgfpathlineto{\pgfqpoint{4.514440in}{2.867402in}}%
\pgfpathlineto{\pgfqpoint{4.522098in}{2.883247in}}%
\pgfpathlineto{\pgfqpoint{4.529756in}{2.899381in}}%
\pgfpathlineto{\pgfqpoint{4.516258in}{2.905299in}}%
\pgfpathlineto{\pgfqpoint{4.502766in}{2.911294in}}%
\pgfpathlineto{\pgfqpoint{4.489280in}{2.917366in}}%
\pgfpathlineto{\pgfqpoint{4.475798in}{2.923516in}}%
\pgfpathlineto{\pgfqpoint{4.468132in}{2.906979in}}%
\pgfpathlineto{\pgfqpoint{4.460465in}{2.890736in}}%
\pgfpathlineto{\pgfqpoint{4.452797in}{2.874779in}}%
\pgfpathlineto{\pgfqpoint{4.445128in}{2.859101in}}%
\pgfpathclose%
\pgfusepath{fill}%
\end{pgfscope}%
\begin{pgfscope}%
\pgfpathrectangle{\pgfqpoint{1.150000in}{0.150000in}}{\pgfqpoint{5.700000in}{5.700000in}}%
\pgfusepath{clip}%
\pgfsetbuttcap%
\pgfsetroundjoin%
\definecolor{currentfill}{rgb}{0.275191,0.194905,0.496005}%
\pgfsetfillcolor{currentfill}%
\pgfsetfillopacity{0.700000}%
\pgfsetlinewidth{0.000000pt}%
\definecolor{currentstroke}{rgb}{0.000000,0.000000,0.000000}%
\pgfsetstrokecolor{currentstroke}%
\pgfsetdash{}{0pt}%
\pgfpathmoveto{\pgfqpoint{4.052763in}{2.744870in}}%
\pgfpathlineto{\pgfqpoint{4.066182in}{2.738879in}}%
\pgfpathlineto{\pgfqpoint{4.079606in}{2.732972in}}%
\pgfpathlineto{\pgfqpoint{4.093035in}{2.727150in}}%
\pgfpathlineto{\pgfqpoint{4.106469in}{2.721411in}}%
\pgfpathlineto{\pgfqpoint{4.114214in}{2.734765in}}%
\pgfpathlineto{\pgfqpoint{4.121955in}{2.748315in}}%
\pgfpathlineto{\pgfqpoint{4.129692in}{2.762067in}}%
\pgfpathlineto{\pgfqpoint{4.137426in}{2.776025in}}%
\pgfpathlineto{\pgfqpoint{4.124000in}{2.782080in}}%
\pgfpathlineto{\pgfqpoint{4.110579in}{2.788218in}}%
\pgfpathlineto{\pgfqpoint{4.097163in}{2.794440in}}%
\pgfpathlineto{\pgfqpoint{4.083751in}{2.800746in}}%
\pgfpathlineto{\pgfqpoint{4.076010in}{2.786465in}}%
\pgfpathlineto{\pgfqpoint{4.068264in}{2.772395in}}%
\pgfpathlineto{\pgfqpoint{4.060516in}{2.758532in}}%
\pgfpathlineto{\pgfqpoint{4.052763in}{2.744870in}}%
\pgfpathclose%
\pgfusepath{fill}%
\end{pgfscope}%
\begin{pgfscope}%
\pgfpathrectangle{\pgfqpoint{1.150000in}{0.150000in}}{\pgfqpoint{5.700000in}{5.700000in}}%
\pgfusepath{clip}%
\pgfsetbuttcap%
\pgfsetroundjoin%
\definecolor{currentfill}{rgb}{0.156270,0.489624,0.557936}%
\pgfsetfillcolor{currentfill}%
\pgfsetfillopacity{0.700000}%
\pgfsetlinewidth{0.000000pt}%
\definecolor{currentstroke}{rgb}{0.000000,0.000000,0.000000}%
\pgfsetstrokecolor{currentstroke}%
\pgfsetdash{}{0pt}%
\pgfpathmoveto{\pgfqpoint{5.045829in}{3.436776in}}%
\pgfpathlineto{\pgfqpoint{5.059377in}{3.428513in}}%
\pgfpathlineto{\pgfqpoint{5.072930in}{3.420322in}}%
\pgfpathlineto{\pgfqpoint{5.086488in}{3.412203in}}%
\pgfpathlineto{\pgfqpoint{5.100051in}{3.404156in}}%
\pgfpathlineto{\pgfqpoint{5.107804in}{3.431178in}}%
\pgfpathlineto{\pgfqpoint{5.115568in}{3.458771in}}%
\pgfpathlineto{\pgfqpoint{5.123344in}{3.486947in}}%
\pgfpathlineto{\pgfqpoint{5.109786in}{3.495430in}}%
\pgfpathlineto{\pgfqpoint{5.096232in}{3.503986in}}%
\pgfpathlineto{\pgfqpoint{5.082684in}{3.512613in}}%
\pgfpathlineto{\pgfqpoint{5.069140in}{3.521312in}}%
\pgfpathlineto{\pgfqpoint{5.061359in}{3.492549in}}%
\pgfpathlineto{\pgfqpoint{5.053589in}{3.464374in}}%
\pgfpathlineto{\pgfqpoint{5.045829in}{3.436776in}}%
\pgfpathclose%
\pgfusepath{fill}%
\end{pgfscope}%
\begin{pgfscope}%
\pgfpathrectangle{\pgfqpoint{1.150000in}{0.150000in}}{\pgfqpoint{5.700000in}{5.700000in}}%
\pgfusepath{clip}%
\pgfsetbuttcap%
\pgfsetroundjoin%
\definecolor{currentfill}{rgb}{0.263663,0.237631,0.518762}%
\pgfsetfillcolor{currentfill}%
\pgfsetfillopacity{0.700000}%
\pgfsetlinewidth{0.000000pt}%
\definecolor{currentstroke}{rgb}{0.000000,0.000000,0.000000}%
\pgfsetstrokecolor{currentstroke}%
\pgfsetdash{}{0pt}%
\pgfpathmoveto{\pgfqpoint{2.860823in}{2.847563in}}%
\pgfpathlineto{\pgfqpoint{2.874150in}{2.836394in}}%
\pgfpathlineto{\pgfqpoint{2.887476in}{2.825357in}}%
\pgfpathlineto{\pgfqpoint{2.900802in}{2.814453in}}%
\pgfpathlineto{\pgfqpoint{2.914126in}{2.803679in}}%
\pgfpathlineto{\pgfqpoint{2.922212in}{2.815286in}}%
\pgfpathlineto{\pgfqpoint{2.930289in}{2.827023in}}%
\pgfpathlineto{\pgfqpoint{2.938359in}{2.838892in}}%
\pgfpathlineto{\pgfqpoint{2.946422in}{2.850897in}}%
\pgfpathlineto{\pgfqpoint{2.933106in}{2.861806in}}%
\pgfpathlineto{\pgfqpoint{2.919790in}{2.872846in}}%
\pgfpathlineto{\pgfqpoint{2.906473in}{2.884018in}}%
\pgfpathlineto{\pgfqpoint{2.893155in}{2.895323in}}%
\pgfpathlineto{\pgfqpoint{2.885084in}{2.883175in}}%
\pgfpathlineto{\pgfqpoint{2.877005in}{2.871167in}}%
\pgfpathlineto{\pgfqpoint{2.868918in}{2.859298in}}%
\pgfpathlineto{\pgfqpoint{2.860823in}{2.847563in}}%
\pgfpathclose%
\pgfusepath{fill}%
\end{pgfscope}%
\begin{pgfscope}%
\pgfpathrectangle{\pgfqpoint{1.150000in}{0.150000in}}{\pgfqpoint{5.700000in}{5.700000in}}%
\pgfusepath{clip}%
\pgfsetbuttcap%
\pgfsetroundjoin%
\definecolor{currentfill}{rgb}{0.171176,0.452530,0.557965}%
\pgfsetfillcolor{currentfill}%
\pgfsetfillopacity{0.700000}%
\pgfsetlinewidth{0.000000pt}%
\definecolor{currentstroke}{rgb}{0.000000,0.000000,0.000000}%
\pgfsetstrokecolor{currentstroke}%
\pgfsetdash{}{0pt}%
\pgfpathmoveto{\pgfqpoint{5.014890in}{3.331927in}}%
\pgfpathlineto{\pgfqpoint{5.028445in}{3.324226in}}%
\pgfpathlineto{\pgfqpoint{5.042005in}{3.316597in}}%
\pgfpathlineto{\pgfqpoint{5.055570in}{3.309040in}}%
\pgfpathlineto{\pgfqpoint{5.069140in}{3.301555in}}%
\pgfpathlineto{\pgfqpoint{5.076854in}{3.326404in}}%
\pgfpathlineto{\pgfqpoint{5.084576in}{3.351780in}}%
\pgfpathlineto{\pgfqpoint{5.092308in}{3.377694in}}%
\pgfpathlineto{\pgfqpoint{5.100051in}{3.404156in}}%
\pgfpathlineto{\pgfqpoint{5.086488in}{3.412203in}}%
\pgfpathlineto{\pgfqpoint{5.072930in}{3.420322in}}%
\pgfpathlineto{\pgfqpoint{5.059377in}{3.428513in}}%
\pgfpathlineto{\pgfqpoint{5.045829in}{3.436776in}}%
\pgfpathlineto{\pgfqpoint{5.038080in}{3.409743in}}%
\pgfpathlineto{\pgfqpoint{5.030341in}{3.383265in}}%
\pgfpathlineto{\pgfqpoint{5.022611in}{3.357330in}}%
\pgfpathlineto{\pgfqpoint{5.014890in}{3.331927in}}%
\pgfpathclose%
\pgfusepath{fill}%
\end{pgfscope}%
\begin{pgfscope}%
\pgfpathrectangle{\pgfqpoint{1.150000in}{0.150000in}}{\pgfqpoint{5.700000in}{5.700000in}}%
\pgfusepath{clip}%
\pgfsetbuttcap%
\pgfsetroundjoin%
\definecolor{currentfill}{rgb}{0.279574,0.170599,0.479997}%
\pgfsetfillcolor{currentfill}%
\pgfsetfillopacity{0.700000}%
\pgfsetlinewidth{0.000000pt}%
\definecolor{currentstroke}{rgb}{0.000000,0.000000,0.000000}%
\pgfsetstrokecolor{currentstroke}%
\pgfsetdash{}{0pt}%
\pgfpathmoveto{\pgfqpoint{3.606617in}{2.693920in}}%
\pgfpathlineto{\pgfqpoint{3.619968in}{2.686832in}}%
\pgfpathlineto{\pgfqpoint{3.633321in}{2.679839in}}%
\pgfpathlineto{\pgfqpoint{3.646678in}{2.672942in}}%
\pgfpathlineto{\pgfqpoint{3.660039in}{2.666139in}}%
\pgfpathlineto{\pgfqpoint{3.667906in}{2.678446in}}%
\pgfpathlineto{\pgfqpoint{3.675768in}{2.690897in}}%
\pgfpathlineto{\pgfqpoint{3.683624in}{2.703498in}}%
\pgfpathlineto{\pgfqpoint{3.691475in}{2.716253in}}%
\pgfpathlineto{\pgfqpoint{3.678122in}{2.723291in}}%
\pgfpathlineto{\pgfqpoint{3.664772in}{2.730424in}}%
\pgfpathlineto{\pgfqpoint{3.651425in}{2.737652in}}%
\pgfpathlineto{\pgfqpoint{3.638082in}{2.744977in}}%
\pgfpathlineto{\pgfqpoint{3.630224in}{2.731979in}}%
\pgfpathlineto{\pgfqpoint{3.622361in}{2.719140in}}%
\pgfpathlineto{\pgfqpoint{3.614492in}{2.706455in}}%
\pgfpathlineto{\pgfqpoint{3.606617in}{2.693920in}}%
\pgfpathclose%
\pgfusepath{fill}%
\end{pgfscope}%
\begin{pgfscope}%
\pgfpathrectangle{\pgfqpoint{1.150000in}{0.150000in}}{\pgfqpoint{5.700000in}{5.700000in}}%
\pgfusepath{clip}%
\pgfsetbuttcap%
\pgfsetroundjoin%
\definecolor{currentfill}{rgb}{0.223925,0.334994,0.548053}%
\pgfsetfillcolor{currentfill}%
\pgfsetfillopacity{0.700000}%
\pgfsetlinewidth{0.000000pt}%
\definecolor{currentstroke}{rgb}{0.000000,0.000000,0.000000}%
\pgfsetstrokecolor{currentstroke}%
\pgfsetdash{}{0pt}%
\pgfpathmoveto{\pgfqpoint{4.783771in}{3.037495in}}%
\pgfpathlineto{\pgfqpoint{4.797315in}{3.031245in}}%
\pgfpathlineto{\pgfqpoint{4.810865in}{3.025069in}}%
\pgfpathlineto{\pgfqpoint{4.824420in}{3.018966in}}%
\pgfpathlineto{\pgfqpoint{4.837981in}{3.012937in}}%
\pgfpathlineto{\pgfqpoint{4.845617in}{3.031659in}}%
\pgfpathlineto{\pgfqpoint{4.853256in}{3.050763in}}%
\pgfpathlineto{\pgfqpoint{4.860899in}{3.070258in}}%
\pgfpathlineto{\pgfqpoint{4.868545in}{3.090151in}}%
\pgfpathlineto{\pgfqpoint{4.854993in}{3.096658in}}%
\pgfpathlineto{\pgfqpoint{4.841447in}{3.103238in}}%
\pgfpathlineto{\pgfqpoint{4.827906in}{3.109891in}}%
\pgfpathlineto{\pgfqpoint{4.814371in}{3.116617in}}%
\pgfpathlineto{\pgfqpoint{4.806716in}{3.096239in}}%
\pgfpathlineto{\pgfqpoint{4.799064in}{3.076265in}}%
\pgfpathlineto{\pgfqpoint{4.791416in}{3.056687in}}%
\pgfpathlineto{\pgfqpoint{4.783771in}{3.037495in}}%
\pgfpathclose%
\pgfusepath{fill}%
\end{pgfscope}%
\begin{pgfscope}%
\pgfpathrectangle{\pgfqpoint{1.150000in}{0.150000in}}{\pgfqpoint{5.700000in}{5.700000in}}%
\pgfusepath{clip}%
\pgfsetbuttcap%
\pgfsetroundjoin%
\definecolor{currentfill}{rgb}{0.274128,0.199721,0.498911}%
\pgfsetfillcolor{currentfill}%
\pgfsetfillopacity{0.700000}%
\pgfsetlinewidth{0.000000pt}%
\definecolor{currentstroke}{rgb}{0.000000,0.000000,0.000000}%
\pgfsetstrokecolor{currentstroke}%
\pgfsetdash{}{0pt}%
\pgfpathmoveto{\pgfqpoint{3.052941in}{2.768197in}}%
\pgfpathlineto{\pgfqpoint{3.066257in}{2.758415in}}%
\pgfpathlineto{\pgfqpoint{3.079574in}{2.748754in}}%
\pgfpathlineto{\pgfqpoint{3.092891in}{2.739212in}}%
\pgfpathlineto{\pgfqpoint{3.106208in}{2.729787in}}%
\pgfpathlineto{\pgfqpoint{3.114238in}{2.741487in}}%
\pgfpathlineto{\pgfqpoint{3.122261in}{2.753313in}}%
\pgfpathlineto{\pgfqpoint{3.130276in}{2.765268in}}%
\pgfpathlineto{\pgfqpoint{3.138285in}{2.777355in}}%
\pgfpathlineto{\pgfqpoint{3.124975in}{2.786935in}}%
\pgfpathlineto{\pgfqpoint{3.111666in}{2.796633in}}%
\pgfpathlineto{\pgfqpoint{3.098358in}{2.806450in}}%
\pgfpathlineto{\pgfqpoint{3.085051in}{2.816387in}}%
\pgfpathlineto{\pgfqpoint{3.077034in}{2.804136in}}%
\pgfpathlineto{\pgfqpoint{3.069010in}{2.792023in}}%
\pgfpathlineto{\pgfqpoint{3.060979in}{2.780044in}}%
\pgfpathlineto{\pgfqpoint{3.052941in}{2.768197in}}%
\pgfpathclose%
\pgfusepath{fill}%
\end{pgfscope}%
\begin{pgfscope}%
\pgfpathrectangle{\pgfqpoint{1.150000in}{0.150000in}}{\pgfqpoint{5.700000in}{5.700000in}}%
\pgfusepath{clip}%
\pgfsetbuttcap%
\pgfsetroundjoin%
\definecolor{currentfill}{rgb}{0.265145,0.232956,0.516599}%
\pgfsetfillcolor{currentfill}%
\pgfsetfillopacity{0.700000}%
\pgfsetlinewidth{0.000000pt}%
\definecolor{currentstroke}{rgb}{0.000000,0.000000,0.000000}%
\pgfsetstrokecolor{currentstroke}%
\pgfsetdash{}{0pt}%
\pgfpathmoveto{\pgfqpoint{4.360498in}{2.821335in}}%
\pgfpathlineto{\pgfqpoint{4.373975in}{2.815643in}}%
\pgfpathlineto{\pgfqpoint{4.387457in}{2.810030in}}%
\pgfpathlineto{\pgfqpoint{4.400944in}{2.804495in}}%
\pgfpathlineto{\pgfqpoint{4.414437in}{2.799039in}}%
\pgfpathlineto{\pgfqpoint{4.422113in}{2.813671in}}%
\pgfpathlineto{\pgfqpoint{4.429786in}{2.828554in}}%
\pgfpathlineto{\pgfqpoint{4.437458in}{2.843695in}}%
\pgfpathlineto{\pgfqpoint{4.445128in}{2.859101in}}%
\pgfpathlineto{\pgfqpoint{4.431644in}{2.864933in}}%
\pgfpathlineto{\pgfqpoint{4.418165in}{2.870844in}}%
\pgfpathlineto{\pgfqpoint{4.404692in}{2.876832in}}%
\pgfpathlineto{\pgfqpoint{4.391223in}{2.882900in}}%
\pgfpathlineto{\pgfqpoint{4.383545in}{2.867111in}}%
\pgfpathlineto{\pgfqpoint{4.375864in}{2.851591in}}%
\pgfpathlineto{\pgfqpoint{4.368182in}{2.836335in}}%
\pgfpathlineto{\pgfqpoint{4.360498in}{2.821335in}}%
\pgfpathclose%
\pgfusepath{fill}%
\end{pgfscope}%
\begin{pgfscope}%
\pgfpathrectangle{\pgfqpoint{1.150000in}{0.150000in}}{\pgfqpoint{5.700000in}{5.700000in}}%
\pgfusepath{clip}%
\pgfsetbuttcap%
\pgfsetroundjoin%
\definecolor{currentfill}{rgb}{0.187231,0.414746,0.556547}%
\pgfsetfillcolor{currentfill}%
\pgfsetfillopacity{0.700000}%
\pgfsetlinewidth{0.000000pt}%
\definecolor{currentstroke}{rgb}{0.000000,0.000000,0.000000}%
\pgfsetstrokecolor{currentstroke}%
\pgfsetdash{}{0pt}%
\pgfpathmoveto{\pgfqpoint{4.984090in}{3.235427in}}%
\pgfpathlineto{\pgfqpoint{4.997653in}{3.228267in}}%
\pgfpathlineto{\pgfqpoint{5.011221in}{3.221178in}}%
\pgfpathlineto{\pgfqpoint{5.024794in}{3.214161in}}%
\pgfpathlineto{\pgfqpoint{5.038373in}{3.207215in}}%
\pgfpathlineto{\pgfqpoint{5.046053in}{3.230062in}}%
\pgfpathlineto{\pgfqpoint{5.053740in}{3.253394in}}%
\pgfpathlineto{\pgfqpoint{5.061436in}{3.277222in}}%
\pgfpathlineto{\pgfqpoint{5.069140in}{3.301555in}}%
\pgfpathlineto{\pgfqpoint{5.055570in}{3.309040in}}%
\pgfpathlineto{\pgfqpoint{5.042005in}{3.316597in}}%
\pgfpathlineto{\pgfqpoint{5.028445in}{3.324226in}}%
\pgfpathlineto{\pgfqpoint{5.014890in}{3.331927in}}%
\pgfpathlineto{\pgfqpoint{5.007178in}{3.307046in}}%
\pgfpathlineto{\pgfqpoint{4.999475in}{3.282676in}}%
\pgfpathlineto{\pgfqpoint{4.991779in}{3.258806in}}%
\pgfpathlineto{\pgfqpoint{4.984090in}{3.235427in}}%
\pgfpathclose%
\pgfusepath{fill}%
\end{pgfscope}%
\begin{pgfscope}%
\pgfpathrectangle{\pgfqpoint{1.150000in}{0.150000in}}{\pgfqpoint{5.700000in}{5.700000in}}%
\pgfusepath{clip}%
\pgfsetbuttcap%
\pgfsetroundjoin%
\definecolor{currentfill}{rgb}{0.214298,0.355619,0.551184}%
\pgfsetfillcolor{currentfill}%
\pgfsetfillopacity{0.700000}%
\pgfsetlinewidth{0.000000pt}%
\definecolor{currentstroke}{rgb}{0.000000,0.000000,0.000000}%
\pgfsetstrokecolor{currentstroke}%
\pgfsetdash{}{0pt}%
\pgfpathmoveto{\pgfqpoint{4.868545in}{3.090151in}}%
\pgfpathlineto{\pgfqpoint{4.882103in}{3.083718in}}%
\pgfpathlineto{\pgfqpoint{4.895666in}{3.077356in}}%
\pgfpathlineto{\pgfqpoint{4.909235in}{3.071067in}}%
\pgfpathlineto{\pgfqpoint{4.922810in}{3.064851in}}%
\pgfpathlineto{\pgfqpoint{4.930451in}{3.084663in}}%
\pgfpathlineto{\pgfqpoint{4.938097in}{3.104888in}}%
\pgfpathlineto{\pgfqpoint{4.945748in}{3.125535in}}%
\pgfpathlineto{\pgfqpoint{4.953405in}{3.146612in}}%
\pgfpathlineto{\pgfqpoint{4.939839in}{3.153326in}}%
\pgfpathlineto{\pgfqpoint{4.926279in}{3.160113in}}%
\pgfpathlineto{\pgfqpoint{4.912725in}{3.166972in}}%
\pgfpathlineto{\pgfqpoint{4.899176in}{3.173903in}}%
\pgfpathlineto{\pgfqpoint{4.891511in}{3.152320in}}%
\pgfpathlineto{\pgfqpoint{4.883851in}{3.131173in}}%
\pgfpathlineto{\pgfqpoint{4.876196in}{3.110454in}}%
\pgfpathlineto{\pgfqpoint{4.868545in}{3.090151in}}%
\pgfpathclose%
\pgfusepath{fill}%
\end{pgfscope}%
\begin{pgfscope}%
\pgfpathrectangle{\pgfqpoint{1.150000in}{0.150000in}}{\pgfqpoint{5.700000in}{5.700000in}}%
\pgfusepath{clip}%
\pgfsetbuttcap%
\pgfsetroundjoin%
\definecolor{currentfill}{rgb}{0.269308,0.218818,0.509577}%
\pgfsetfillcolor{currentfill}%
\pgfsetfillopacity{0.700000}%
\pgfsetlinewidth{0.000000pt}%
\definecolor{currentstroke}{rgb}{0.000000,0.000000,0.000000}%
\pgfsetstrokecolor{currentstroke}%
\pgfsetdash{}{0pt}%
\pgfpathmoveto{\pgfqpoint{4.275852in}{2.785899in}}%
\pgfpathlineto{\pgfqpoint{4.289316in}{2.780245in}}%
\pgfpathlineto{\pgfqpoint{4.302785in}{2.774670in}}%
\pgfpathlineto{\pgfqpoint{4.316259in}{2.769175in}}%
\pgfpathlineto{\pgfqpoint{4.329739in}{2.763760in}}%
\pgfpathlineto{\pgfqpoint{4.337432in}{2.777803in}}%
\pgfpathlineto{\pgfqpoint{4.345123in}{2.792075in}}%
\pgfpathlineto{\pgfqpoint{4.352812in}{2.806584in}}%
\pgfpathlineto{\pgfqpoint{4.360498in}{2.821335in}}%
\pgfpathlineto{\pgfqpoint{4.347027in}{2.827106in}}%
\pgfpathlineto{\pgfqpoint{4.333561in}{2.832956in}}%
\pgfpathlineto{\pgfqpoint{4.320100in}{2.838886in}}%
\pgfpathlineto{\pgfqpoint{4.306645in}{2.844896in}}%
\pgfpathlineto{\pgfqpoint{4.298950in}{2.829782in}}%
\pgfpathlineto{\pgfqpoint{4.291254in}{2.814915in}}%
\pgfpathlineto{\pgfqpoint{4.283554in}{2.800290in}}%
\pgfpathlineto{\pgfqpoint{4.275852in}{2.785899in}}%
\pgfpathclose%
\pgfusepath{fill}%
\end{pgfscope}%
\begin{pgfscope}%
\pgfpathrectangle{\pgfqpoint{1.150000in}{0.150000in}}{\pgfqpoint{5.700000in}{5.700000in}}%
\pgfusepath{clip}%
\pgfsetbuttcap%
\pgfsetroundjoin%
\definecolor{currentfill}{rgb}{0.279574,0.170599,0.479997}%
\pgfsetfillcolor{currentfill}%
\pgfsetfillopacity{0.700000}%
\pgfsetlinewidth{0.000000pt}%
\definecolor{currentstroke}{rgb}{0.000000,0.000000,0.000000}%
\pgfsetstrokecolor{currentstroke}%
\pgfsetdash{}{0pt}%
\pgfpathmoveto{\pgfqpoint{3.744925in}{2.689037in}}%
\pgfpathlineto{\pgfqpoint{3.758297in}{2.682464in}}%
\pgfpathlineto{\pgfqpoint{3.771673in}{2.675983in}}%
\pgfpathlineto{\pgfqpoint{3.785053in}{2.669593in}}%
\pgfpathlineto{\pgfqpoint{3.798437in}{2.663294in}}%
\pgfpathlineto{\pgfqpoint{3.806268in}{2.675711in}}%
\pgfpathlineto{\pgfqpoint{3.814094in}{2.688281in}}%
\pgfpathlineto{\pgfqpoint{3.821914in}{2.701009in}}%
\pgfpathlineto{\pgfqpoint{3.829730in}{2.713900in}}%
\pgfpathlineto{\pgfqpoint{3.816353in}{2.720455in}}%
\pgfpathlineto{\pgfqpoint{3.802980in}{2.727101in}}%
\pgfpathlineto{\pgfqpoint{3.789612in}{2.733838in}}%
\pgfpathlineto{\pgfqpoint{3.776247in}{2.740666in}}%
\pgfpathlineto{\pgfqpoint{3.768424in}{2.727512in}}%
\pgfpathlineto{\pgfqpoint{3.760596in}{2.714526in}}%
\pgfpathlineto{\pgfqpoint{3.752763in}{2.701702in}}%
\pgfpathlineto{\pgfqpoint{3.744925in}{2.689037in}}%
\pgfpathclose%
\pgfusepath{fill}%
\end{pgfscope}%
\begin{pgfscope}%
\pgfpathrectangle{\pgfqpoint{1.150000in}{0.150000in}}{\pgfqpoint{5.700000in}{5.700000in}}%
\pgfusepath{clip}%
\pgfsetbuttcap%
\pgfsetroundjoin%
\definecolor{currentfill}{rgb}{0.277134,0.185228,0.489898}%
\pgfsetfillcolor{currentfill}%
\pgfsetfillopacity{0.700000}%
\pgfsetlinewidth{0.000000pt}%
\definecolor{currentstroke}{rgb}{0.000000,0.000000,0.000000}%
\pgfsetstrokecolor{currentstroke}%
\pgfsetdash{}{0pt}%
\pgfpathmoveto{\pgfqpoint{3.968050in}{2.715748in}}%
\pgfpathlineto{\pgfqpoint{3.981458in}{2.709711in}}%
\pgfpathlineto{\pgfqpoint{3.994871in}{2.703760in}}%
\pgfpathlineto{\pgfqpoint{4.008289in}{2.697895in}}%
\pgfpathlineto{\pgfqpoint{4.021711in}{2.692115in}}%
\pgfpathlineto{\pgfqpoint{4.029480in}{2.705030in}}%
\pgfpathlineto{\pgfqpoint{4.037245in}{2.718124in}}%
\pgfpathlineto{\pgfqpoint{4.045006in}{2.731402in}}%
\pgfpathlineto{\pgfqpoint{4.052763in}{2.744870in}}%
\pgfpathlineto{\pgfqpoint{4.039348in}{2.750945in}}%
\pgfpathlineto{\pgfqpoint{4.025939in}{2.757106in}}%
\pgfpathlineto{\pgfqpoint{4.012533in}{2.763353in}}%
\pgfpathlineto{\pgfqpoint{3.999133in}{2.769685in}}%
\pgfpathlineto{\pgfqpoint{3.991368in}{2.755915in}}%
\pgfpathlineto{\pgfqpoint{3.983600in}{2.742339in}}%
\pgfpathlineto{\pgfqpoint{3.975827in}{2.728952in}}%
\pgfpathlineto{\pgfqpoint{3.968050in}{2.715748in}}%
\pgfpathclose%
\pgfusepath{fill}%
\end{pgfscope}%
\begin{pgfscope}%
\pgfpathrectangle{\pgfqpoint{1.150000in}{0.150000in}}{\pgfqpoint{5.700000in}{5.700000in}}%
\pgfusepath{clip}%
\pgfsetbuttcap%
\pgfsetroundjoin%
\definecolor{currentfill}{rgb}{0.269308,0.218818,0.509577}%
\pgfsetfillcolor{currentfill}%
\pgfsetfillopacity{0.700000}%
\pgfsetlinewidth{0.000000pt}%
\definecolor{currentstroke}{rgb}{0.000000,0.000000,0.000000}%
\pgfsetstrokecolor{currentstroke}%
\pgfsetdash{}{0pt}%
\pgfpathmoveto{\pgfqpoint{2.914126in}{2.803679in}}%
\pgfpathlineto{\pgfqpoint{2.927451in}{2.793035in}}%
\pgfpathlineto{\pgfqpoint{2.940775in}{2.782519in}}%
\pgfpathlineto{\pgfqpoint{2.954098in}{2.772131in}}%
\pgfpathlineto{\pgfqpoint{2.967422in}{2.761869in}}%
\pgfpathlineto{\pgfqpoint{2.975498in}{2.773348in}}%
\pgfpathlineto{\pgfqpoint{2.983566in}{2.784951in}}%
\pgfpathlineto{\pgfqpoint{2.991628in}{2.796683in}}%
\pgfpathlineto{\pgfqpoint{2.999682in}{2.808545in}}%
\pgfpathlineto{\pgfqpoint{2.986367in}{2.818943in}}%
\pgfpathlineto{\pgfqpoint{2.973052in}{2.829466in}}%
\pgfpathlineto{\pgfqpoint{2.959737in}{2.840118in}}%
\pgfpathlineto{\pgfqpoint{2.946422in}{2.850897in}}%
\pgfpathlineto{\pgfqpoint{2.938359in}{2.838892in}}%
\pgfpathlineto{\pgfqpoint{2.930289in}{2.827023in}}%
\pgfpathlineto{\pgfqpoint{2.922212in}{2.815286in}}%
\pgfpathlineto{\pgfqpoint{2.914126in}{2.803679in}}%
\pgfpathclose%
\pgfusepath{fill}%
\end{pgfscope}%
\begin{pgfscope}%
\pgfpathrectangle{\pgfqpoint{1.150000in}{0.150000in}}{\pgfqpoint{5.700000in}{5.700000in}}%
\pgfusepath{clip}%
\pgfsetbuttcap%
\pgfsetroundjoin%
\definecolor{currentfill}{rgb}{0.280255,0.165693,0.476498}%
\pgfsetfillcolor{currentfill}%
\pgfsetfillopacity{0.700000}%
\pgfsetlinewidth{0.000000pt}%
\definecolor{currentstroke}{rgb}{0.000000,0.000000,0.000000}%
\pgfsetstrokecolor{currentstroke}%
\pgfsetdash{}{0pt}%
\pgfpathmoveto{\pgfqpoint{3.383268in}{2.686374in}}%
\pgfpathlineto{\pgfqpoint{3.396598in}{2.678498in}}%
\pgfpathlineto{\pgfqpoint{3.409931in}{2.670725in}}%
\pgfpathlineto{\pgfqpoint{3.423266in}{2.663055in}}%
\pgfpathlineto{\pgfqpoint{3.436604in}{2.655487in}}%
\pgfpathlineto{\pgfqpoint{3.444541in}{2.667346in}}%
\pgfpathlineto{\pgfqpoint{3.452472in}{2.679332in}}%
\pgfpathlineto{\pgfqpoint{3.460396in}{2.691451in}}%
\pgfpathlineto{\pgfqpoint{3.468315in}{2.703705in}}%
\pgfpathlineto{\pgfqpoint{3.454984in}{2.711468in}}%
\pgfpathlineto{\pgfqpoint{3.441657in}{2.719334in}}%
\pgfpathlineto{\pgfqpoint{3.428331in}{2.727302in}}%
\pgfpathlineto{\pgfqpoint{3.415009in}{2.735374in}}%
\pgfpathlineto{\pgfqpoint{3.407083in}{2.722917in}}%
\pgfpathlineto{\pgfqpoint{3.399151in}{2.710601in}}%
\pgfpathlineto{\pgfqpoint{3.391213in}{2.698421in}}%
\pgfpathlineto{\pgfqpoint{3.383268in}{2.686374in}}%
\pgfpathclose%
\pgfusepath{fill}%
\end{pgfscope}%
\begin{pgfscope}%
\pgfpathrectangle{\pgfqpoint{1.150000in}{0.150000in}}{\pgfqpoint{5.700000in}{5.700000in}}%
\pgfusepath{clip}%
\pgfsetbuttcap%
\pgfsetroundjoin%
\definecolor{currentfill}{rgb}{0.278826,0.175490,0.483397}%
\pgfsetfillcolor{currentfill}%
\pgfsetfillopacity{0.700000}%
\pgfsetlinewidth{0.000000pt}%
\definecolor{currentstroke}{rgb}{0.000000,0.000000,0.000000}%
\pgfsetstrokecolor{currentstroke}%
\pgfsetdash{}{0pt}%
\pgfpathmoveto{\pgfqpoint{3.244802in}{2.704866in}}%
\pgfpathlineto{\pgfqpoint{3.258123in}{2.696310in}}%
\pgfpathlineto{\pgfqpoint{3.271446in}{2.687864in}}%
\pgfpathlineto{\pgfqpoint{3.284771in}{2.679526in}}%
\pgfpathlineto{\pgfqpoint{3.298098in}{2.671297in}}%
\pgfpathlineto{\pgfqpoint{3.306076in}{2.683016in}}%
\pgfpathlineto{\pgfqpoint{3.314047in}{2.694859in}}%
\pgfpathlineto{\pgfqpoint{3.322012in}{2.706829in}}%
\pgfpathlineto{\pgfqpoint{3.329971in}{2.718929in}}%
\pgfpathlineto{\pgfqpoint{3.316652in}{2.727334in}}%
\pgfpathlineto{\pgfqpoint{3.303335in}{2.735847in}}%
\pgfpathlineto{\pgfqpoint{3.290019in}{2.744469in}}%
\pgfpathlineto{\pgfqpoint{3.276706in}{2.753201in}}%
\pgfpathlineto{\pgfqpoint{3.268740in}{2.740917in}}%
\pgfpathlineto{\pgfqpoint{3.260767in}{2.728769in}}%
\pgfpathlineto{\pgfqpoint{3.252788in}{2.716753in}}%
\pgfpathlineto{\pgfqpoint{3.244802in}{2.704866in}}%
\pgfpathclose%
\pgfusepath{fill}%
\end{pgfscope}%
\begin{pgfscope}%
\pgfpathrectangle{\pgfqpoint{1.150000in}{0.150000in}}{\pgfqpoint{5.700000in}{5.700000in}}%
\pgfusepath{clip}%
\pgfsetbuttcap%
\pgfsetroundjoin%
\definecolor{currentfill}{rgb}{0.201239,0.383670,0.554294}%
\pgfsetfillcolor{currentfill}%
\pgfsetfillopacity{0.700000}%
\pgfsetlinewidth{0.000000pt}%
\definecolor{currentstroke}{rgb}{0.000000,0.000000,0.000000}%
\pgfsetstrokecolor{currentstroke}%
\pgfsetdash{}{0pt}%
\pgfpathmoveto{\pgfqpoint{4.953405in}{3.146612in}}%
\pgfpathlineto{\pgfqpoint{4.966976in}{3.139970in}}%
\pgfpathlineto{\pgfqpoint{4.980553in}{3.133400in}}%
\pgfpathlineto{\pgfqpoint{4.994135in}{3.126901in}}%
\pgfpathlineto{\pgfqpoint{5.007724in}{3.120474in}}%
\pgfpathlineto{\pgfqpoint{5.015377in}{3.141481in}}%
\pgfpathlineto{\pgfqpoint{5.023035in}{3.162934in}}%
\pgfpathlineto{\pgfqpoint{5.030701in}{3.184842in}}%
\pgfpathlineto{\pgfqpoint{5.038373in}{3.207215in}}%
\pgfpathlineto{\pgfqpoint{5.024794in}{3.214161in}}%
\pgfpathlineto{\pgfqpoint{5.011221in}{3.221178in}}%
\pgfpathlineto{\pgfqpoint{4.997653in}{3.228267in}}%
\pgfpathlineto{\pgfqpoint{4.984090in}{3.235427in}}%
\pgfpathlineto{\pgfqpoint{4.976409in}{3.212528in}}%
\pgfpathlineto{\pgfqpoint{4.968735in}{3.190099in}}%
\pgfpathlineto{\pgfqpoint{4.961067in}{3.168131in}}%
\pgfpathlineto{\pgfqpoint{4.953405in}{3.146612in}}%
\pgfpathclose%
\pgfusepath{fill}%
\end{pgfscope}%
\begin{pgfscope}%
\pgfpathrectangle{\pgfqpoint{1.150000in}{0.150000in}}{\pgfqpoint{5.700000in}{5.700000in}}%
\pgfusepath{clip}%
\pgfsetbuttcap%
\pgfsetroundjoin%
\definecolor{currentfill}{rgb}{0.273006,0.204520,0.501721}%
\pgfsetfillcolor{currentfill}%
\pgfsetfillopacity{0.700000}%
\pgfsetlinewidth{0.000000pt}%
\definecolor{currentstroke}{rgb}{0.000000,0.000000,0.000000}%
\pgfsetstrokecolor{currentstroke}%
\pgfsetdash{}{0pt}%
\pgfpathmoveto{\pgfqpoint{4.191179in}{2.752635in}}%
\pgfpathlineto{\pgfqpoint{4.204630in}{2.746993in}}%
\pgfpathlineto{\pgfqpoint{4.218087in}{2.741432in}}%
\pgfpathlineto{\pgfqpoint{4.231548in}{2.735952in}}%
\pgfpathlineto{\pgfqpoint{4.245015in}{2.730553in}}%
\pgfpathlineto{\pgfqpoint{4.252729in}{2.744069in}}%
\pgfpathlineto{\pgfqpoint{4.260440in}{2.757795in}}%
\pgfpathlineto{\pgfqpoint{4.268147in}{2.771736in}}%
\pgfpathlineto{\pgfqpoint{4.275852in}{2.785899in}}%
\pgfpathlineto{\pgfqpoint{4.262394in}{2.791634in}}%
\pgfpathlineto{\pgfqpoint{4.248941in}{2.797449in}}%
\pgfpathlineto{\pgfqpoint{4.235493in}{2.803346in}}%
\pgfpathlineto{\pgfqpoint{4.222050in}{2.809323in}}%
\pgfpathlineto{\pgfqpoint{4.214337in}{2.794817in}}%
\pgfpathlineto{\pgfqpoint{4.206621in}{2.780538in}}%
\pgfpathlineto{\pgfqpoint{4.198902in}{2.766480in}}%
\pgfpathlineto{\pgfqpoint{4.191179in}{2.752635in}}%
\pgfpathclose%
\pgfusepath{fill}%
\end{pgfscope}%
\begin{pgfscope}%
\pgfpathrectangle{\pgfqpoint{1.150000in}{0.150000in}}{\pgfqpoint{5.700000in}{5.700000in}}%
\pgfusepath{clip}%
\pgfsetbuttcap%
\pgfsetroundjoin%
\definecolor{currentfill}{rgb}{0.280868,0.160771,0.472899}%
\pgfsetfillcolor{currentfill}%
\pgfsetfillopacity{0.700000}%
\pgfsetlinewidth{0.000000pt}%
\definecolor{currentstroke}{rgb}{0.000000,0.000000,0.000000}%
\pgfsetstrokecolor{currentstroke}%
\pgfsetdash{}{0pt}%
\pgfpathmoveto{\pgfqpoint{3.521664in}{2.673659in}}%
\pgfpathlineto{\pgfqpoint{3.535009in}{2.666397in}}%
\pgfpathlineto{\pgfqpoint{3.548357in}{2.659233in}}%
\pgfpathlineto{\pgfqpoint{3.561708in}{2.652166in}}%
\pgfpathlineto{\pgfqpoint{3.575062in}{2.645197in}}%
\pgfpathlineto{\pgfqpoint{3.582960in}{2.657174in}}%
\pgfpathlineto{\pgfqpoint{3.590851in}{2.669284in}}%
\pgfpathlineto{\pgfqpoint{3.598737in}{2.681531in}}%
\pgfpathlineto{\pgfqpoint{3.606617in}{2.693920in}}%
\pgfpathlineto{\pgfqpoint{3.593270in}{2.701105in}}%
\pgfpathlineto{\pgfqpoint{3.579927in}{2.708387in}}%
\pgfpathlineto{\pgfqpoint{3.566586in}{2.715767in}}%
\pgfpathlineto{\pgfqpoint{3.553249in}{2.723245in}}%
\pgfpathlineto{\pgfqpoint{3.545361in}{2.710633in}}%
\pgfpathlineto{\pgfqpoint{3.537468in}{2.698168in}}%
\pgfpathlineto{\pgfqpoint{3.529569in}{2.685844in}}%
\pgfpathlineto{\pgfqpoint{3.521664in}{2.673659in}}%
\pgfpathclose%
\pgfusepath{fill}%
\end{pgfscope}%
\begin{pgfscope}%
\pgfpathrectangle{\pgfqpoint{1.150000in}{0.150000in}}{\pgfqpoint{5.700000in}{5.700000in}}%
\pgfusepath{clip}%
\pgfsetbuttcap%
\pgfsetroundjoin%
\definecolor{currentfill}{rgb}{0.276194,0.190074,0.493001}%
\pgfsetfillcolor{currentfill}%
\pgfsetfillopacity{0.700000}%
\pgfsetlinewidth{0.000000pt}%
\definecolor{currentstroke}{rgb}{0.000000,0.000000,0.000000}%
\pgfsetstrokecolor{currentstroke}%
\pgfsetdash{}{0pt}%
\pgfpathmoveto{\pgfqpoint{3.106208in}{2.729787in}}%
\pgfpathlineto{\pgfqpoint{3.119527in}{2.720481in}}%
\pgfpathlineto{\pgfqpoint{3.132847in}{2.711290in}}%
\pgfpathlineto{\pgfqpoint{3.146168in}{2.702215in}}%
\pgfpathlineto{\pgfqpoint{3.159490in}{2.693255in}}%
\pgfpathlineto{\pgfqpoint{3.167511in}{2.704807in}}%
\pgfpathlineto{\pgfqpoint{3.175525in}{2.716479in}}%
\pgfpathlineto{\pgfqpoint{3.183532in}{2.728276in}}%
\pgfpathlineto{\pgfqpoint{3.191533in}{2.740201in}}%
\pgfpathlineto{\pgfqpoint{3.178219in}{2.749316in}}%
\pgfpathlineto{\pgfqpoint{3.164907in}{2.758547in}}%
\pgfpathlineto{\pgfqpoint{3.151595in}{2.767893in}}%
\pgfpathlineto{\pgfqpoint{3.138285in}{2.777355in}}%
\pgfpathlineto{\pgfqpoint{3.130276in}{2.765268in}}%
\pgfpathlineto{\pgfqpoint{3.122261in}{2.753313in}}%
\pgfpathlineto{\pgfqpoint{3.114238in}{2.741487in}}%
\pgfpathlineto{\pgfqpoint{3.106208in}{2.729787in}}%
\pgfpathclose%
\pgfusepath{fill}%
\end{pgfscope}%
\begin{pgfscope}%
\pgfpathrectangle{\pgfqpoint{1.150000in}{0.150000in}}{\pgfqpoint{5.700000in}{5.700000in}}%
\pgfusepath{clip}%
\pgfsetbuttcap%
\pgfsetroundjoin%
\definecolor{currentfill}{rgb}{0.160665,0.478540,0.558115}%
\pgfsetfillcolor{currentfill}%
\pgfsetfillopacity{0.700000}%
\pgfsetlinewidth{0.000000pt}%
\definecolor{currentstroke}{rgb}{0.000000,0.000000,0.000000}%
\pgfsetstrokecolor{currentstroke}%
\pgfsetdash{}{0pt}%
\pgfpathmoveto{\pgfqpoint{5.100051in}{3.404156in}}%
\pgfpathlineto{\pgfqpoint{5.113619in}{3.396180in}}%
\pgfpathlineto{\pgfqpoint{5.127193in}{3.388276in}}%
\pgfpathlineto{\pgfqpoint{5.140771in}{3.380443in}}%
\pgfpathlineto{\pgfqpoint{5.154356in}{3.372680in}}%
\pgfpathlineto{\pgfqpoint{5.162101in}{3.399127in}}%
\pgfpathlineto{\pgfqpoint{5.169859in}{3.426139in}}%
\pgfpathlineto{\pgfqpoint{5.177628in}{3.453728in}}%
\pgfpathlineto{\pgfqpoint{5.164049in}{3.461926in}}%
\pgfpathlineto{\pgfqpoint{5.150476in}{3.470195in}}%
\pgfpathlineto{\pgfqpoint{5.136907in}{3.478535in}}%
\pgfpathlineto{\pgfqpoint{5.123344in}{3.486947in}}%
\pgfpathlineto{\pgfqpoint{5.115568in}{3.458771in}}%
\pgfpathlineto{\pgfqpoint{5.107804in}{3.431178in}}%
\pgfpathlineto{\pgfqpoint{5.100051in}{3.404156in}}%
\pgfpathclose%
\pgfusepath{fill}%
\end{pgfscope}%
\begin{pgfscope}%
\pgfpathrectangle{\pgfqpoint{1.150000in}{0.150000in}}{\pgfqpoint{5.700000in}{5.700000in}}%
\pgfusepath{clip}%
\pgfsetbuttcap%
\pgfsetroundjoin%
\definecolor{currentfill}{rgb}{0.174274,0.445044,0.557792}%
\pgfsetfillcolor{currentfill}%
\pgfsetfillopacity{0.700000}%
\pgfsetlinewidth{0.000000pt}%
\definecolor{currentstroke}{rgb}{0.000000,0.000000,0.000000}%
\pgfsetstrokecolor{currentstroke}%
\pgfsetdash{}{0pt}%
\pgfpathmoveto{\pgfqpoint{5.069140in}{3.301555in}}%
\pgfpathlineto{\pgfqpoint{5.082716in}{3.294141in}}%
\pgfpathlineto{\pgfqpoint{5.096298in}{3.286798in}}%
\pgfpathlineto{\pgfqpoint{5.109885in}{3.279526in}}%
\pgfpathlineto{\pgfqpoint{5.123477in}{3.272325in}}%
\pgfpathlineto{\pgfqpoint{5.131182in}{3.296621in}}%
\pgfpathlineto{\pgfqpoint{5.138896in}{3.321438in}}%
\pgfpathlineto{\pgfqpoint{5.146621in}{3.346788in}}%
\pgfpathlineto{\pgfqpoint{5.154356in}{3.372680in}}%
\pgfpathlineto{\pgfqpoint{5.140771in}{3.380443in}}%
\pgfpathlineto{\pgfqpoint{5.127193in}{3.388276in}}%
\pgfpathlineto{\pgfqpoint{5.113619in}{3.396180in}}%
\pgfpathlineto{\pgfqpoint{5.100051in}{3.404156in}}%
\pgfpathlineto{\pgfqpoint{5.092308in}{3.377694in}}%
\pgfpathlineto{\pgfqpoint{5.084576in}{3.351780in}}%
\pgfpathlineto{\pgfqpoint{5.076854in}{3.326404in}}%
\pgfpathlineto{\pgfqpoint{5.069140in}{3.301555in}}%
\pgfpathclose%
\pgfusepath{fill}%
\end{pgfscope}%
\begin{pgfscope}%
\pgfpathrectangle{\pgfqpoint{1.150000in}{0.150000in}}{\pgfqpoint{5.700000in}{5.700000in}}%
\pgfusepath{clip}%
\pgfsetbuttcap%
\pgfsetroundjoin%
\definecolor{currentfill}{rgb}{0.279574,0.170599,0.479997}%
\pgfsetfillcolor{currentfill}%
\pgfsetfillopacity{0.700000}%
\pgfsetlinewidth{0.000000pt}%
\definecolor{currentstroke}{rgb}{0.000000,0.000000,0.000000}%
\pgfsetstrokecolor{currentstroke}%
\pgfsetdash{}{0pt}%
\pgfpathmoveto{\pgfqpoint{3.883278in}{2.688576in}}%
\pgfpathlineto{\pgfqpoint{3.896676in}{2.682466in}}%
\pgfpathlineto{\pgfqpoint{3.910078in}{2.676445in}}%
\pgfpathlineto{\pgfqpoint{3.923485in}{2.670510in}}%
\pgfpathlineto{\pgfqpoint{3.936897in}{2.664663in}}%
\pgfpathlineto{\pgfqpoint{3.944692in}{2.677185in}}%
\pgfpathlineto{\pgfqpoint{3.952482in}{2.689870in}}%
\pgfpathlineto{\pgfqpoint{3.960268in}{2.702722in}}%
\pgfpathlineto{\pgfqpoint{3.968050in}{2.715748in}}%
\pgfpathlineto{\pgfqpoint{3.954646in}{2.721871in}}%
\pgfpathlineto{\pgfqpoint{3.941247in}{2.728081in}}%
\pgfpathlineto{\pgfqpoint{3.927853in}{2.734379in}}%
\pgfpathlineto{\pgfqpoint{3.914462in}{2.740764in}}%
\pgfpathlineto{\pgfqpoint{3.906673in}{2.727455in}}%
\pgfpathlineto{\pgfqpoint{3.898879in}{2.714324in}}%
\pgfpathlineto{\pgfqpoint{3.891081in}{2.701366in}}%
\pgfpathlineto{\pgfqpoint{3.883278in}{2.688576in}}%
\pgfpathclose%
\pgfusepath{fill}%
\end{pgfscope}%
\begin{pgfscope}%
\pgfpathrectangle{\pgfqpoint{1.150000in}{0.150000in}}{\pgfqpoint{5.700000in}{5.700000in}}%
\pgfusepath{clip}%
\pgfsetbuttcap%
\pgfsetroundjoin%
\definecolor{currentfill}{rgb}{0.246811,0.283237,0.535941}%
\pgfsetfillcolor{currentfill}%
\pgfsetfillopacity{0.700000}%
\pgfsetlinewidth{0.000000pt}%
\definecolor{currentstroke}{rgb}{0.000000,0.000000,0.000000}%
\pgfsetstrokecolor{currentstroke}%
\pgfsetdash{}{0pt}%
\pgfpathmoveto{\pgfqpoint{4.668496in}{2.919019in}}%
\pgfpathlineto{\pgfqpoint{4.682036in}{2.913366in}}%
\pgfpathlineto{\pgfqpoint{4.695581in}{2.907787in}}%
\pgfpathlineto{\pgfqpoint{4.709133in}{2.902282in}}%
\pgfpathlineto{\pgfqpoint{4.722690in}{2.896851in}}%
\pgfpathlineto{\pgfqpoint{4.730320in}{2.913256in}}%
\pgfpathlineto{\pgfqpoint{4.737951in}{2.929981in}}%
\pgfpathlineto{\pgfqpoint{4.745583in}{2.947033in}}%
\pgfpathlineto{\pgfqpoint{4.753216in}{2.964421in}}%
\pgfpathlineto{\pgfqpoint{4.739669in}{2.970288in}}%
\pgfpathlineto{\pgfqpoint{4.726127in}{2.976228in}}%
\pgfpathlineto{\pgfqpoint{4.712591in}{2.982243in}}%
\pgfpathlineto{\pgfqpoint{4.699061in}{2.988332in}}%
\pgfpathlineto{\pgfqpoint{4.691418in}{2.970501in}}%
\pgfpathlineto{\pgfqpoint{4.683776in}{2.953010in}}%
\pgfpathlineto{\pgfqpoint{4.676136in}{2.935852in}}%
\pgfpathlineto{\pgfqpoint{4.668496in}{2.919019in}}%
\pgfpathclose%
\pgfusepath{fill}%
\end{pgfscope}%
\begin{pgfscope}%
\pgfpathrectangle{\pgfqpoint{1.150000in}{0.150000in}}{\pgfqpoint{5.700000in}{5.700000in}}%
\pgfusepath{clip}%
\pgfsetbuttcap%
\pgfsetroundjoin%
\definecolor{currentfill}{rgb}{0.253935,0.265254,0.529983}%
\pgfsetfillcolor{currentfill}%
\pgfsetfillopacity{0.700000}%
\pgfsetlinewidth{0.000000pt}%
\definecolor{currentstroke}{rgb}{0.000000,0.000000,0.000000}%
\pgfsetstrokecolor{currentstroke}%
\pgfsetdash{}{0pt}%
\pgfpathmoveto{\pgfqpoint{4.583802in}{2.876472in}}%
\pgfpathlineto{\pgfqpoint{4.597328in}{2.870935in}}%
\pgfpathlineto{\pgfqpoint{4.610860in}{2.865473in}}%
\pgfpathlineto{\pgfqpoint{4.624398in}{2.860086in}}%
\pgfpathlineto{\pgfqpoint{4.637942in}{2.854774in}}%
\pgfpathlineto{\pgfqpoint{4.645580in}{2.870387in}}%
\pgfpathlineto{\pgfqpoint{4.653218in}{2.886294in}}%
\pgfpathlineto{\pgfqpoint{4.660857in}{2.902502in}}%
\pgfpathlineto{\pgfqpoint{4.668496in}{2.919019in}}%
\pgfpathlineto{\pgfqpoint{4.654962in}{2.924747in}}%
\pgfpathlineto{\pgfqpoint{4.641434in}{2.930549in}}%
\pgfpathlineto{\pgfqpoint{4.627912in}{2.936427in}}%
\pgfpathlineto{\pgfqpoint{4.614395in}{2.942380in}}%
\pgfpathlineto{\pgfqpoint{4.606747in}{2.925440in}}%
\pgfpathlineto{\pgfqpoint{4.599099in}{2.908814in}}%
\pgfpathlineto{\pgfqpoint{4.591451in}{2.892494in}}%
\pgfpathlineto{\pgfqpoint{4.583802in}{2.876472in}}%
\pgfpathclose%
\pgfusepath{fill}%
\end{pgfscope}%
\begin{pgfscope}%
\pgfpathrectangle{\pgfqpoint{1.150000in}{0.150000in}}{\pgfqpoint{5.700000in}{5.700000in}}%
\pgfusepath{clip}%
\pgfsetbuttcap%
\pgfsetroundjoin%
\definecolor{currentfill}{rgb}{0.237441,0.305202,0.541921}%
\pgfsetfillcolor{currentfill}%
\pgfsetfillopacity{0.700000}%
\pgfsetlinewidth{0.000000pt}%
\definecolor{currentstroke}{rgb}{0.000000,0.000000,0.000000}%
\pgfsetstrokecolor{currentstroke}%
\pgfsetdash{}{0pt}%
\pgfpathmoveto{\pgfqpoint{4.753216in}{2.964421in}}%
\pgfpathlineto{\pgfqpoint{4.766770in}{2.958627in}}%
\pgfpathlineto{\pgfqpoint{4.780329in}{2.952908in}}%
\pgfpathlineto{\pgfqpoint{4.793895in}{2.947261in}}%
\pgfpathlineto{\pgfqpoint{4.807466in}{2.941688in}}%
\pgfpathlineto{\pgfqpoint{4.815091in}{2.958971in}}%
\pgfpathlineto{\pgfqpoint{4.822719in}{2.976601in}}%
\pgfpathlineto{\pgfqpoint{4.830349in}{2.994587in}}%
\pgfpathlineto{\pgfqpoint{4.837981in}{3.012937in}}%
\pgfpathlineto{\pgfqpoint{4.824420in}{3.018966in}}%
\pgfpathlineto{\pgfqpoint{4.810865in}{3.025069in}}%
\pgfpathlineto{\pgfqpoint{4.797315in}{3.031245in}}%
\pgfpathlineto{\pgfqpoint{4.783771in}{3.037495in}}%
\pgfpathlineto{\pgfqpoint{4.776129in}{3.018681in}}%
\pgfpathlineto{\pgfqpoint{4.768489in}{3.000236in}}%
\pgfpathlineto{\pgfqpoint{4.760852in}{2.982152in}}%
\pgfpathlineto{\pgfqpoint{4.753216in}{2.964421in}}%
\pgfpathclose%
\pgfusepath{fill}%
\end{pgfscope}%
\begin{pgfscope}%
\pgfpathrectangle{\pgfqpoint{1.150000in}{0.150000in}}{\pgfqpoint{5.700000in}{5.700000in}}%
\pgfusepath{clip}%
\pgfsetbuttcap%
\pgfsetroundjoin%
\definecolor{currentfill}{rgb}{0.280868,0.160771,0.472899}%
\pgfsetfillcolor{currentfill}%
\pgfsetfillopacity{0.700000}%
\pgfsetlinewidth{0.000000pt}%
\definecolor{currentstroke}{rgb}{0.000000,0.000000,0.000000}%
\pgfsetstrokecolor{currentstroke}%
\pgfsetdash{}{0pt}%
\pgfpathmoveto{\pgfqpoint{3.660039in}{2.666139in}}%
\pgfpathlineto{\pgfqpoint{3.673404in}{2.659431in}}%
\pgfpathlineto{\pgfqpoint{3.686772in}{2.652816in}}%
\pgfpathlineto{\pgfqpoint{3.700144in}{2.646294in}}%
\pgfpathlineto{\pgfqpoint{3.713520in}{2.639865in}}%
\pgfpathlineto{\pgfqpoint{3.721379in}{2.651944in}}%
\pgfpathlineto{\pgfqpoint{3.729233in}{2.664162in}}%
\pgfpathlineto{\pgfqpoint{3.737082in}{2.676525in}}%
\pgfpathlineto{\pgfqpoint{3.744925in}{2.689037in}}%
\pgfpathlineto{\pgfqpoint{3.731557in}{2.695702in}}%
\pgfpathlineto{\pgfqpoint{3.718193in}{2.702459in}}%
\pgfpathlineto{\pgfqpoint{3.704832in}{2.709309in}}%
\pgfpathlineto{\pgfqpoint{3.691475in}{2.716253in}}%
\pgfpathlineto{\pgfqpoint{3.683624in}{2.703498in}}%
\pgfpathlineto{\pgfqpoint{3.675768in}{2.690897in}}%
\pgfpathlineto{\pgfqpoint{3.667906in}{2.678446in}}%
\pgfpathlineto{\pgfqpoint{3.660039in}{2.666139in}}%
\pgfpathclose%
\pgfusepath{fill}%
\end{pgfscope}%
\begin{pgfscope}%
\pgfpathrectangle{\pgfqpoint{1.150000in}{0.150000in}}{\pgfqpoint{5.700000in}{5.700000in}}%
\pgfusepath{clip}%
\pgfsetbuttcap%
\pgfsetroundjoin%
\definecolor{currentfill}{rgb}{0.260571,0.246922,0.522828}%
\pgfsetfillcolor{currentfill}%
\pgfsetfillopacity{0.700000}%
\pgfsetlinewidth{0.000000pt}%
\definecolor{currentstroke}{rgb}{0.000000,0.000000,0.000000}%
\pgfsetstrokecolor{currentstroke}%
\pgfsetdash{}{0pt}%
\pgfpathmoveto{\pgfqpoint{4.499121in}{2.836548in}}%
\pgfpathlineto{\pgfqpoint{4.512633in}{2.831103in}}%
\pgfpathlineto{\pgfqpoint{4.526151in}{2.825733in}}%
\pgfpathlineto{\pgfqpoint{4.539675in}{2.820440in}}%
\pgfpathlineto{\pgfqpoint{4.553205in}{2.815223in}}%
\pgfpathlineto{\pgfqpoint{4.560856in}{2.830124in}}%
\pgfpathlineto{\pgfqpoint{4.568505in}{2.845295in}}%
\pgfpathlineto{\pgfqpoint{4.576154in}{2.860742in}}%
\pgfpathlineto{\pgfqpoint{4.583802in}{2.876472in}}%
\pgfpathlineto{\pgfqpoint{4.570282in}{2.882085in}}%
\pgfpathlineto{\pgfqpoint{4.556768in}{2.887774in}}%
\pgfpathlineto{\pgfqpoint{4.543259in}{2.893539in}}%
\pgfpathlineto{\pgfqpoint{4.529756in}{2.899381in}}%
\pgfpathlineto{\pgfqpoint{4.522098in}{2.883247in}}%
\pgfpathlineto{\pgfqpoint{4.514440in}{2.867402in}}%
\pgfpathlineto{\pgfqpoint{4.506781in}{2.851838in}}%
\pgfpathlineto{\pgfqpoint{4.499121in}{2.836548in}}%
\pgfpathclose%
\pgfusepath{fill}%
\end{pgfscope}%
\begin{pgfscope}%
\pgfpathrectangle{\pgfqpoint{1.150000in}{0.150000in}}{\pgfqpoint{5.700000in}{5.700000in}}%
\pgfusepath{clip}%
\pgfsetbuttcap%
\pgfsetroundjoin%
\definecolor{currentfill}{rgb}{0.276194,0.190074,0.493001}%
\pgfsetfillcolor{currentfill}%
\pgfsetfillopacity{0.700000}%
\pgfsetlinewidth{0.000000pt}%
\definecolor{currentstroke}{rgb}{0.000000,0.000000,0.000000}%
\pgfsetstrokecolor{currentstroke}%
\pgfsetdash{}{0pt}%
\pgfpathmoveto{\pgfqpoint{4.106469in}{2.721411in}}%
\pgfpathlineto{\pgfqpoint{4.119908in}{2.715755in}}%
\pgfpathlineto{\pgfqpoint{4.133352in}{2.710182in}}%
\pgfpathlineto{\pgfqpoint{4.146801in}{2.704691in}}%
\pgfpathlineto{\pgfqpoint{4.160255in}{2.699283in}}%
\pgfpathlineto{\pgfqpoint{4.167992in}{2.712329in}}%
\pgfpathlineto{\pgfqpoint{4.175724in}{2.725566in}}%
\pgfpathlineto{\pgfqpoint{4.183453in}{2.738999in}}%
\pgfpathlineto{\pgfqpoint{4.191179in}{2.752635in}}%
\pgfpathlineto{\pgfqpoint{4.177733in}{2.758359in}}%
\pgfpathlineto{\pgfqpoint{4.164293in}{2.764166in}}%
\pgfpathlineto{\pgfqpoint{4.150857in}{2.770054in}}%
\pgfpathlineto{\pgfqpoint{4.137426in}{2.776025in}}%
\pgfpathlineto{\pgfqpoint{4.129692in}{2.762067in}}%
\pgfpathlineto{\pgfqpoint{4.121955in}{2.748315in}}%
\pgfpathlineto{\pgfqpoint{4.114214in}{2.734765in}}%
\pgfpathlineto{\pgfqpoint{4.106469in}{2.721411in}}%
\pgfpathclose%
\pgfusepath{fill}%
\end{pgfscope}%
\begin{pgfscope}%
\pgfpathrectangle{\pgfqpoint{1.150000in}{0.150000in}}{\pgfqpoint{5.700000in}{5.700000in}}%
\pgfusepath{clip}%
\pgfsetbuttcap%
\pgfsetroundjoin%
\definecolor{currentfill}{rgb}{0.227802,0.326594,0.546532}%
\pgfsetfillcolor{currentfill}%
\pgfsetfillopacity{0.700000}%
\pgfsetlinewidth{0.000000pt}%
\definecolor{currentstroke}{rgb}{0.000000,0.000000,0.000000}%
\pgfsetstrokecolor{currentstroke}%
\pgfsetdash{}{0pt}%
\pgfpathmoveto{\pgfqpoint{4.837981in}{3.012937in}}%
\pgfpathlineto{\pgfqpoint{4.851549in}{3.006980in}}%
\pgfpathlineto{\pgfqpoint{4.865122in}{3.001095in}}%
\pgfpathlineto{\pgfqpoint{4.878701in}{2.995283in}}%
\pgfpathlineto{\pgfqpoint{4.892286in}{2.989543in}}%
\pgfpathlineto{\pgfqpoint{4.899912in}{3.007796in}}%
\pgfpathlineto{\pgfqpoint{4.907541in}{3.026426in}}%
\pgfpathlineto{\pgfqpoint{4.915173in}{3.045441in}}%
\pgfpathlineto{\pgfqpoint{4.922810in}{3.064851in}}%
\pgfpathlineto{\pgfqpoint{4.909235in}{3.071067in}}%
\pgfpathlineto{\pgfqpoint{4.895666in}{3.077356in}}%
\pgfpathlineto{\pgfqpoint{4.882103in}{3.083718in}}%
\pgfpathlineto{\pgfqpoint{4.868545in}{3.090151in}}%
\pgfpathlineto{\pgfqpoint{4.860899in}{3.070258in}}%
\pgfpathlineto{\pgfqpoint{4.853256in}{3.050763in}}%
\pgfpathlineto{\pgfqpoint{4.845617in}{3.031659in}}%
\pgfpathlineto{\pgfqpoint{4.837981in}{3.012937in}}%
\pgfpathclose%
\pgfusepath{fill}%
\end{pgfscope}%
\begin{pgfscope}%
\pgfpathrectangle{\pgfqpoint{1.150000in}{0.150000in}}{\pgfqpoint{5.700000in}{5.700000in}}%
\pgfusepath{clip}%
\pgfsetbuttcap%
\pgfsetroundjoin%
\definecolor{currentfill}{rgb}{0.273006,0.204520,0.501721}%
\pgfsetfillcolor{currentfill}%
\pgfsetfillopacity{0.700000}%
\pgfsetlinewidth{0.000000pt}%
\definecolor{currentstroke}{rgb}{0.000000,0.000000,0.000000}%
\pgfsetstrokecolor{currentstroke}%
\pgfsetdash{}{0pt}%
\pgfpathmoveto{\pgfqpoint{2.967422in}{2.761869in}}%
\pgfpathlineto{\pgfqpoint{2.980745in}{2.751732in}}%
\pgfpathlineto{\pgfqpoint{2.994069in}{2.741719in}}%
\pgfpathlineto{\pgfqpoint{3.007393in}{2.731830in}}%
\pgfpathlineto{\pgfqpoint{3.020717in}{2.722063in}}%
\pgfpathlineto{\pgfqpoint{3.028784in}{2.733413in}}%
\pgfpathlineto{\pgfqpoint{3.036844in}{2.744884in}}%
\pgfpathlineto{\pgfqpoint{3.044896in}{2.756478in}}%
\pgfpathlineto{\pgfqpoint{3.052941in}{2.768197in}}%
\pgfpathlineto{\pgfqpoint{3.039626in}{2.778100in}}%
\pgfpathlineto{\pgfqpoint{3.026311in}{2.788125in}}%
\pgfpathlineto{\pgfqpoint{3.012996in}{2.798273in}}%
\pgfpathlineto{\pgfqpoint{2.999682in}{2.808545in}}%
\pgfpathlineto{\pgfqpoint{2.991628in}{2.796683in}}%
\pgfpathlineto{\pgfqpoint{2.983566in}{2.784951in}}%
\pgfpathlineto{\pgfqpoint{2.975498in}{2.773348in}}%
\pgfpathlineto{\pgfqpoint{2.967422in}{2.761869in}}%
\pgfpathclose%
\pgfusepath{fill}%
\end{pgfscope}%
\begin{pgfscope}%
\pgfpathrectangle{\pgfqpoint{1.150000in}{0.150000in}}{\pgfqpoint{5.700000in}{5.700000in}}%
\pgfusepath{clip}%
\pgfsetbuttcap%
\pgfsetroundjoin%
\definecolor{currentfill}{rgb}{0.188923,0.410910,0.556326}%
\pgfsetfillcolor{currentfill}%
\pgfsetfillopacity{0.700000}%
\pgfsetlinewidth{0.000000pt}%
\definecolor{currentstroke}{rgb}{0.000000,0.000000,0.000000}%
\pgfsetstrokecolor{currentstroke}%
\pgfsetdash{}{0pt}%
\pgfpathmoveto{\pgfqpoint{5.038373in}{3.207215in}}%
\pgfpathlineto{\pgfqpoint{5.051958in}{3.200340in}}%
\pgfpathlineto{\pgfqpoint{5.065548in}{3.193537in}}%
\pgfpathlineto{\pgfqpoint{5.079145in}{3.186805in}}%
\pgfpathlineto{\pgfqpoint{5.092747in}{3.180143in}}%
\pgfpathlineto{\pgfqpoint{5.100417in}{3.202459in}}%
\pgfpathlineto{\pgfqpoint{5.108095in}{3.225254in}}%
\pgfpathlineto{\pgfqpoint{5.115782in}{3.248539in}}%
\pgfpathlineto{\pgfqpoint{5.123477in}{3.272325in}}%
\pgfpathlineto{\pgfqpoint{5.109885in}{3.279526in}}%
\pgfpathlineto{\pgfqpoint{5.096298in}{3.286798in}}%
\pgfpathlineto{\pgfqpoint{5.082716in}{3.294141in}}%
\pgfpathlineto{\pgfqpoint{5.069140in}{3.301555in}}%
\pgfpathlineto{\pgfqpoint{5.061436in}{3.277222in}}%
\pgfpathlineto{\pgfqpoint{5.053740in}{3.253394in}}%
\pgfpathlineto{\pgfqpoint{5.046053in}{3.230062in}}%
\pgfpathlineto{\pgfqpoint{5.038373in}{3.207215in}}%
\pgfpathclose%
\pgfusepath{fill}%
\end{pgfscope}%
\begin{pgfscope}%
\pgfpathrectangle{\pgfqpoint{1.150000in}{0.150000in}}{\pgfqpoint{5.700000in}{5.700000in}}%
\pgfusepath{clip}%
\pgfsetbuttcap%
\pgfsetroundjoin%
\definecolor{currentfill}{rgb}{0.262138,0.242286,0.520837}%
\pgfsetfillcolor{currentfill}%
\pgfsetfillopacity{0.700000}%
\pgfsetlinewidth{0.000000pt}%
\definecolor{currentstroke}{rgb}{0.000000,0.000000,0.000000}%
\pgfsetstrokecolor{currentstroke}%
\pgfsetdash{}{0pt}%
\pgfpathmoveto{\pgfqpoint{2.775009in}{2.847496in}}%
\pgfpathlineto{\pgfqpoint{2.788351in}{2.835898in}}%
\pgfpathlineto{\pgfqpoint{2.801691in}{2.824437in}}%
\pgfpathlineto{\pgfqpoint{2.815029in}{2.813113in}}%
\pgfpathlineto{\pgfqpoint{2.828367in}{2.801925in}}%
\pgfpathlineto{\pgfqpoint{2.836493in}{2.813144in}}%
\pgfpathlineto{\pgfqpoint{2.844611in}{2.824489in}}%
\pgfpathlineto{\pgfqpoint{2.852721in}{2.835961in}}%
\pgfpathlineto{\pgfqpoint{2.860823in}{2.847563in}}%
\pgfpathlineto{\pgfqpoint{2.847496in}{2.858867in}}%
\pgfpathlineto{\pgfqpoint{2.834167in}{2.870306in}}%
\pgfpathlineto{\pgfqpoint{2.820836in}{2.881882in}}%
\pgfpathlineto{\pgfqpoint{2.807505in}{2.893596in}}%
\pgfpathlineto{\pgfqpoint{2.799393in}{2.881871in}}%
\pgfpathlineto{\pgfqpoint{2.791273in}{2.870281in}}%
\pgfpathlineto{\pgfqpoint{2.783145in}{2.858823in}}%
\pgfpathlineto{\pgfqpoint{2.775009in}{2.847496in}}%
\pgfpathclose%
\pgfusepath{fill}%
\end{pgfscope}%
\begin{pgfscope}%
\pgfpathrectangle{\pgfqpoint{1.150000in}{0.150000in}}{\pgfqpoint{5.700000in}{5.700000in}}%
\pgfusepath{clip}%
\pgfsetbuttcap%
\pgfsetroundjoin%
\definecolor{currentfill}{rgb}{0.266580,0.228262,0.514349}%
\pgfsetfillcolor{currentfill}%
\pgfsetfillopacity{0.700000}%
\pgfsetlinewidth{0.000000pt}%
\definecolor{currentstroke}{rgb}{0.000000,0.000000,0.000000}%
\pgfsetstrokecolor{currentstroke}%
\pgfsetdash{}{0pt}%
\pgfpathmoveto{\pgfqpoint{4.414437in}{2.799039in}}%
\pgfpathlineto{\pgfqpoint{4.427936in}{2.793660in}}%
\pgfpathlineto{\pgfqpoint{4.441440in}{2.788359in}}%
\pgfpathlineto{\pgfqpoint{4.454951in}{2.783135in}}%
\pgfpathlineto{\pgfqpoint{4.468467in}{2.777989in}}%
\pgfpathlineto{\pgfqpoint{4.476133in}{2.792253in}}%
\pgfpathlineto{\pgfqpoint{4.483797in}{2.806763in}}%
\pgfpathlineto{\pgfqpoint{4.491460in}{2.821526in}}%
\pgfpathlineto{\pgfqpoint{4.499121in}{2.836548in}}%
\pgfpathlineto{\pgfqpoint{4.485614in}{2.842071in}}%
\pgfpathlineto{\pgfqpoint{4.472113in}{2.847670in}}%
\pgfpathlineto{\pgfqpoint{4.458618in}{2.853347in}}%
\pgfpathlineto{\pgfqpoint{4.445128in}{2.859101in}}%
\pgfpathlineto{\pgfqpoint{4.437458in}{2.843695in}}%
\pgfpathlineto{\pgfqpoint{4.429786in}{2.828554in}}%
\pgfpathlineto{\pgfqpoint{4.422113in}{2.813671in}}%
\pgfpathlineto{\pgfqpoint{4.414437in}{2.799039in}}%
\pgfpathclose%
\pgfusepath{fill}%
\end{pgfscope}%
\begin{pgfscope}%
\pgfpathrectangle{\pgfqpoint{1.150000in}{0.150000in}}{\pgfqpoint{5.700000in}{5.700000in}}%
\pgfusepath{clip}%
\pgfsetbuttcap%
\pgfsetroundjoin%
\definecolor{currentfill}{rgb}{0.216210,0.351535,0.550627}%
\pgfsetfillcolor{currentfill}%
\pgfsetfillopacity{0.700000}%
\pgfsetlinewidth{0.000000pt}%
\definecolor{currentstroke}{rgb}{0.000000,0.000000,0.000000}%
\pgfsetstrokecolor{currentstroke}%
\pgfsetdash{}{0pt}%
\pgfpathmoveto{\pgfqpoint{4.922810in}{3.064851in}}%
\pgfpathlineto{\pgfqpoint{4.936391in}{3.058706in}}%
\pgfpathlineto{\pgfqpoint{4.949978in}{3.052633in}}%
\pgfpathlineto{\pgfqpoint{4.963571in}{3.046631in}}%
\pgfpathlineto{\pgfqpoint{4.977170in}{3.040701in}}%
\pgfpathlineto{\pgfqpoint{4.984801in}{3.060024in}}%
\pgfpathlineto{\pgfqpoint{4.992437in}{3.079754in}}%
\pgfpathlineto{\pgfqpoint{5.000077in}{3.099901in}}%
\pgfpathlineto{\pgfqpoint{5.007724in}{3.120474in}}%
\pgfpathlineto{\pgfqpoint{4.994135in}{3.126901in}}%
\pgfpathlineto{\pgfqpoint{4.980553in}{3.133400in}}%
\pgfpathlineto{\pgfqpoint{4.966976in}{3.139970in}}%
\pgfpathlineto{\pgfqpoint{4.953405in}{3.146612in}}%
\pgfpathlineto{\pgfqpoint{4.945748in}{3.125535in}}%
\pgfpathlineto{\pgfqpoint{4.938097in}{3.104888in}}%
\pgfpathlineto{\pgfqpoint{4.930451in}{3.084663in}}%
\pgfpathlineto{\pgfqpoint{4.922810in}{3.064851in}}%
\pgfpathclose%
\pgfusepath{fill}%
\end{pgfscope}%
\begin{pgfscope}%
\pgfpathrectangle{\pgfqpoint{1.150000in}{0.150000in}}{\pgfqpoint{5.700000in}{5.700000in}}%
\pgfusepath{clip}%
\pgfsetbuttcap%
\pgfsetroundjoin%
\definecolor{currentfill}{rgb}{0.280255,0.165693,0.476498}%
\pgfsetfillcolor{currentfill}%
\pgfsetfillopacity{0.700000}%
\pgfsetlinewidth{0.000000pt}%
\definecolor{currentstroke}{rgb}{0.000000,0.000000,0.000000}%
\pgfsetstrokecolor{currentstroke}%
\pgfsetdash{}{0pt}%
\pgfpathmoveto{\pgfqpoint{3.298098in}{2.671297in}}%
\pgfpathlineto{\pgfqpoint{3.311427in}{2.663175in}}%
\pgfpathlineto{\pgfqpoint{3.324758in}{2.655159in}}%
\pgfpathlineto{\pgfqpoint{3.338092in}{2.647249in}}%
\pgfpathlineto{\pgfqpoint{3.351428in}{2.639444in}}%
\pgfpathlineto{\pgfqpoint{3.359397in}{2.650995in}}%
\pgfpathlineto{\pgfqpoint{3.367361in}{2.662664in}}%
\pgfpathlineto{\pgfqpoint{3.375318in}{2.674456in}}%
\pgfpathlineto{\pgfqpoint{3.383268in}{2.686374in}}%
\pgfpathlineto{\pgfqpoint{3.369940in}{2.694355in}}%
\pgfpathlineto{\pgfqpoint{3.356615in}{2.702440in}}%
\pgfpathlineto{\pgfqpoint{3.343292in}{2.710632in}}%
\pgfpathlineto{\pgfqpoint{3.329971in}{2.718929in}}%
\pgfpathlineto{\pgfqpoint{3.322012in}{2.706829in}}%
\pgfpathlineto{\pgfqpoint{3.314047in}{2.694859in}}%
\pgfpathlineto{\pgfqpoint{3.306076in}{2.683016in}}%
\pgfpathlineto{\pgfqpoint{3.298098in}{2.671297in}}%
\pgfpathclose%
\pgfusepath{fill}%
\end{pgfscope}%
\begin{pgfscope}%
\pgfpathrectangle{\pgfqpoint{1.150000in}{0.150000in}}{\pgfqpoint{5.700000in}{5.700000in}}%
\pgfusepath{clip}%
\pgfsetbuttcap%
\pgfsetroundjoin%
\definecolor{currentfill}{rgb}{0.280868,0.160771,0.472899}%
\pgfsetfillcolor{currentfill}%
\pgfsetfillopacity{0.700000}%
\pgfsetlinewidth{0.000000pt}%
\definecolor{currentstroke}{rgb}{0.000000,0.000000,0.000000}%
\pgfsetstrokecolor{currentstroke}%
\pgfsetdash{}{0pt}%
\pgfpathmoveto{\pgfqpoint{3.798437in}{2.663294in}}%
\pgfpathlineto{\pgfqpoint{3.811826in}{2.657084in}}%
\pgfpathlineto{\pgfqpoint{3.825218in}{2.650965in}}%
\pgfpathlineto{\pgfqpoint{3.838615in}{2.644934in}}%
\pgfpathlineto{\pgfqpoint{3.852017in}{2.638992in}}%
\pgfpathlineto{\pgfqpoint{3.859840in}{2.651161in}}%
\pgfpathlineto{\pgfqpoint{3.867657in}{2.663478in}}%
\pgfpathlineto{\pgfqpoint{3.875470in}{2.675948in}}%
\pgfpathlineto{\pgfqpoint{3.883278in}{2.688576in}}%
\pgfpathlineto{\pgfqpoint{3.869884in}{2.694774in}}%
\pgfpathlineto{\pgfqpoint{3.856495in}{2.701060in}}%
\pgfpathlineto{\pgfqpoint{3.843110in}{2.707435in}}%
\pgfpathlineto{\pgfqpoint{3.829730in}{2.713900in}}%
\pgfpathlineto{\pgfqpoint{3.821914in}{2.701009in}}%
\pgfpathlineto{\pgfqpoint{3.814094in}{2.688281in}}%
\pgfpathlineto{\pgfqpoint{3.806268in}{2.675711in}}%
\pgfpathlineto{\pgfqpoint{3.798437in}{2.663294in}}%
\pgfpathclose%
\pgfusepath{fill}%
\end{pgfscope}%
\begin{pgfscope}%
\pgfpathrectangle{\pgfqpoint{1.150000in}{0.150000in}}{\pgfqpoint{5.700000in}{5.700000in}}%
\pgfusepath{clip}%
\pgfsetbuttcap%
\pgfsetroundjoin%
\definecolor{currentfill}{rgb}{0.281412,0.155834,0.469201}%
\pgfsetfillcolor{currentfill}%
\pgfsetfillopacity{0.700000}%
\pgfsetlinewidth{0.000000pt}%
\definecolor{currentstroke}{rgb}{0.000000,0.000000,0.000000}%
\pgfsetstrokecolor{currentstroke}%
\pgfsetdash{}{0pt}%
\pgfpathmoveto{\pgfqpoint{3.436604in}{2.655487in}}%
\pgfpathlineto{\pgfqpoint{3.449945in}{2.648021in}}%
\pgfpathlineto{\pgfqpoint{3.463289in}{2.640655in}}%
\pgfpathlineto{\pgfqpoint{3.476635in}{2.633390in}}%
\pgfpathlineto{\pgfqpoint{3.489985in}{2.626224in}}%
\pgfpathlineto{\pgfqpoint{3.497914in}{2.637895in}}%
\pgfpathlineto{\pgfqpoint{3.505836in}{2.649688in}}%
\pgfpathlineto{\pgfqpoint{3.513753in}{2.661609in}}%
\pgfpathlineto{\pgfqpoint{3.521664in}{2.673659in}}%
\pgfpathlineto{\pgfqpoint{3.508322in}{2.681021in}}%
\pgfpathlineto{\pgfqpoint{3.494983in}{2.688482in}}%
\pgfpathlineto{\pgfqpoint{3.481648in}{2.696043in}}%
\pgfpathlineto{\pgfqpoint{3.468315in}{2.703705in}}%
\pgfpathlineto{\pgfqpoint{3.460396in}{2.691451in}}%
\pgfpathlineto{\pgfqpoint{3.452472in}{2.679332in}}%
\pgfpathlineto{\pgfqpoint{3.444541in}{2.667346in}}%
\pgfpathlineto{\pgfqpoint{3.436604in}{2.655487in}}%
\pgfpathclose%
\pgfusepath{fill}%
\end{pgfscope}%
\begin{pgfscope}%
\pgfpathrectangle{\pgfqpoint{1.150000in}{0.150000in}}{\pgfqpoint{5.700000in}{5.700000in}}%
\pgfusepath{clip}%
\pgfsetbuttcap%
\pgfsetroundjoin%
\definecolor{currentfill}{rgb}{0.278826,0.175490,0.483397}%
\pgfsetfillcolor{currentfill}%
\pgfsetfillopacity{0.700000}%
\pgfsetlinewidth{0.000000pt}%
\definecolor{currentstroke}{rgb}{0.000000,0.000000,0.000000}%
\pgfsetstrokecolor{currentstroke}%
\pgfsetdash{}{0pt}%
\pgfpathmoveto{\pgfqpoint{3.159490in}{2.693255in}}%
\pgfpathlineto{\pgfqpoint{3.172813in}{2.684409in}}%
\pgfpathlineto{\pgfqpoint{3.186138in}{2.675675in}}%
\pgfpathlineto{\pgfqpoint{3.199465in}{2.667053in}}%
\pgfpathlineto{\pgfqpoint{3.212792in}{2.658543in}}%
\pgfpathlineto{\pgfqpoint{3.220805in}{2.669946in}}%
\pgfpathlineto{\pgfqpoint{3.228811in}{2.681466in}}%
\pgfpathlineto{\pgfqpoint{3.236810in}{2.693104in}}%
\pgfpathlineto{\pgfqpoint{3.244802in}{2.704866in}}%
\pgfpathlineto{\pgfqpoint{3.231483in}{2.713532in}}%
\pgfpathlineto{\pgfqpoint{3.218165in}{2.722309in}}%
\pgfpathlineto{\pgfqpoint{3.204848in}{2.731198in}}%
\pgfpathlineto{\pgfqpoint{3.191533in}{2.740201in}}%
\pgfpathlineto{\pgfqpoint{3.183532in}{2.728276in}}%
\pgfpathlineto{\pgfqpoint{3.175525in}{2.716479in}}%
\pgfpathlineto{\pgfqpoint{3.167511in}{2.704807in}}%
\pgfpathlineto{\pgfqpoint{3.159490in}{2.693255in}}%
\pgfpathclose%
\pgfusepath{fill}%
\end{pgfscope}%
\begin{pgfscope}%
\pgfpathrectangle{\pgfqpoint{1.150000in}{0.150000in}}{\pgfqpoint{5.700000in}{5.700000in}}%
\pgfusepath{clip}%
\pgfsetbuttcap%
\pgfsetroundjoin%
\definecolor{currentfill}{rgb}{0.278012,0.180367,0.486697}%
\pgfsetfillcolor{currentfill}%
\pgfsetfillopacity{0.700000}%
\pgfsetlinewidth{0.000000pt}%
\definecolor{currentstroke}{rgb}{0.000000,0.000000,0.000000}%
\pgfsetstrokecolor{currentstroke}%
\pgfsetdash{}{0pt}%
\pgfpathmoveto{\pgfqpoint{4.021711in}{2.692115in}}%
\pgfpathlineto{\pgfqpoint{4.035138in}{2.686420in}}%
\pgfpathlineto{\pgfqpoint{4.048570in}{2.680809in}}%
\pgfpathlineto{\pgfqpoint{4.062007in}{2.675282in}}%
\pgfpathlineto{\pgfqpoint{4.075450in}{2.669839in}}%
\pgfpathlineto{\pgfqpoint{4.083211in}{2.682466in}}%
\pgfpathlineto{\pgfqpoint{4.090967in}{2.695267in}}%
\pgfpathlineto{\pgfqpoint{4.098720in}{2.708246in}}%
\pgfpathlineto{\pgfqpoint{4.106469in}{2.721411in}}%
\pgfpathlineto{\pgfqpoint{4.093035in}{2.727150in}}%
\pgfpathlineto{\pgfqpoint{4.079606in}{2.732972in}}%
\pgfpathlineto{\pgfqpoint{4.066182in}{2.738879in}}%
\pgfpathlineto{\pgfqpoint{4.052763in}{2.744870in}}%
\pgfpathlineto{\pgfqpoint{4.045006in}{2.731402in}}%
\pgfpathlineto{\pgfqpoint{4.037245in}{2.718124in}}%
\pgfpathlineto{\pgfqpoint{4.029480in}{2.705030in}}%
\pgfpathlineto{\pgfqpoint{4.021711in}{2.692115in}}%
\pgfpathclose%
\pgfusepath{fill}%
\end{pgfscope}%
\begin{pgfscope}%
\pgfpathrectangle{\pgfqpoint{1.150000in}{0.150000in}}{\pgfqpoint{5.700000in}{5.700000in}}%
\pgfusepath{clip}%
\pgfsetbuttcap%
\pgfsetroundjoin%
\definecolor{currentfill}{rgb}{0.270595,0.214069,0.507052}%
\pgfsetfillcolor{currentfill}%
\pgfsetfillopacity{0.700000}%
\pgfsetlinewidth{0.000000pt}%
\definecolor{currentstroke}{rgb}{0.000000,0.000000,0.000000}%
\pgfsetstrokecolor{currentstroke}%
\pgfsetdash{}{0pt}%
\pgfpathmoveto{\pgfqpoint{4.329739in}{2.763760in}}%
\pgfpathlineto{\pgfqpoint{4.343224in}{2.758424in}}%
\pgfpathlineto{\pgfqpoint{4.356715in}{2.753167in}}%
\pgfpathlineto{\pgfqpoint{4.370212in}{2.747988in}}%
\pgfpathlineto{\pgfqpoint{4.383715in}{2.742887in}}%
\pgfpathlineto{\pgfqpoint{4.391399in}{2.756581in}}%
\pgfpathlineto{\pgfqpoint{4.399080in}{2.770501in}}%
\pgfpathlineto{\pgfqpoint{4.406760in}{2.784651in}}%
\pgfpathlineto{\pgfqpoint{4.414437in}{2.799039in}}%
\pgfpathlineto{\pgfqpoint{4.400944in}{2.804495in}}%
\pgfpathlineto{\pgfqpoint{4.387457in}{2.810030in}}%
\pgfpathlineto{\pgfqpoint{4.373975in}{2.815643in}}%
\pgfpathlineto{\pgfqpoint{4.360498in}{2.821335in}}%
\pgfpathlineto{\pgfqpoint{4.352812in}{2.806584in}}%
\pgfpathlineto{\pgfqpoint{4.345123in}{2.792075in}}%
\pgfpathlineto{\pgfqpoint{4.337432in}{2.777803in}}%
\pgfpathlineto{\pgfqpoint{4.329739in}{2.763760in}}%
\pgfpathclose%
\pgfusepath{fill}%
\end{pgfscope}%
\begin{pgfscope}%
\pgfpathrectangle{\pgfqpoint{1.150000in}{0.150000in}}{\pgfqpoint{5.700000in}{5.700000in}}%
\pgfusepath{clip}%
\pgfsetbuttcap%
\pgfsetroundjoin%
\definecolor{currentfill}{rgb}{0.163625,0.471133,0.558148}%
\pgfsetfillcolor{currentfill}%
\pgfsetfillopacity{0.700000}%
\pgfsetlinewidth{0.000000pt}%
\definecolor{currentstroke}{rgb}{0.000000,0.000000,0.000000}%
\pgfsetstrokecolor{currentstroke}%
\pgfsetdash{}{0pt}%
\pgfpathmoveto{\pgfqpoint{5.154356in}{3.372680in}}%
\pgfpathlineto{\pgfqpoint{5.167945in}{3.364988in}}%
\pgfpathlineto{\pgfqpoint{5.181540in}{3.357367in}}%
\pgfpathlineto{\pgfqpoint{5.195141in}{3.349816in}}%
\pgfpathlineto{\pgfqpoint{5.208747in}{3.342335in}}%
\pgfpathlineto{\pgfqpoint{5.216484in}{3.368207in}}%
\pgfpathlineto{\pgfqpoint{5.224233in}{3.394639in}}%
\pgfpathlineto{\pgfqpoint{5.231995in}{3.421642in}}%
\pgfpathlineto{\pgfqpoint{5.218395in}{3.429558in}}%
\pgfpathlineto{\pgfqpoint{5.204801in}{3.437544in}}%
\pgfpathlineto{\pgfqpoint{5.191211in}{3.445601in}}%
\pgfpathlineto{\pgfqpoint{5.177628in}{3.453728in}}%
\pgfpathlineto{\pgfqpoint{5.169859in}{3.426139in}}%
\pgfpathlineto{\pgfqpoint{5.162101in}{3.399127in}}%
\pgfpathlineto{\pgfqpoint{5.154356in}{3.372680in}}%
\pgfpathclose%
\pgfusepath{fill}%
\end{pgfscope}%
\begin{pgfscope}%
\pgfpathrectangle{\pgfqpoint{1.150000in}{0.150000in}}{\pgfqpoint{5.700000in}{5.700000in}}%
\pgfusepath{clip}%
\pgfsetbuttcap%
\pgfsetroundjoin%
\definecolor{currentfill}{rgb}{0.204903,0.375746,0.553533}%
\pgfsetfillcolor{currentfill}%
\pgfsetfillopacity{0.700000}%
\pgfsetlinewidth{0.000000pt}%
\definecolor{currentstroke}{rgb}{0.000000,0.000000,0.000000}%
\pgfsetstrokecolor{currentstroke}%
\pgfsetdash{}{0pt}%
\pgfpathmoveto{\pgfqpoint{5.007724in}{3.120474in}}%
\pgfpathlineto{\pgfqpoint{5.021319in}{3.114117in}}%
\pgfpathlineto{\pgfqpoint{5.034919in}{3.107832in}}%
\pgfpathlineto{\pgfqpoint{5.048526in}{3.101618in}}%
\pgfpathlineto{\pgfqpoint{5.062138in}{3.095474in}}%
\pgfpathlineto{\pgfqpoint{5.069780in}{3.115972in}}%
\pgfpathlineto{\pgfqpoint{5.077429in}{3.136909in}}%
\pgfpathlineto{\pgfqpoint{5.085084in}{3.158296in}}%
\pgfpathlineto{\pgfqpoint{5.092747in}{3.180143in}}%
\pgfpathlineto{\pgfqpoint{5.079145in}{3.186805in}}%
\pgfpathlineto{\pgfqpoint{5.065548in}{3.193537in}}%
\pgfpathlineto{\pgfqpoint{5.051958in}{3.200340in}}%
\pgfpathlineto{\pgfqpoint{5.038373in}{3.207215in}}%
\pgfpathlineto{\pgfqpoint{5.030701in}{3.184842in}}%
\pgfpathlineto{\pgfqpoint{5.023035in}{3.162934in}}%
\pgfpathlineto{\pgfqpoint{5.015377in}{3.141481in}}%
\pgfpathlineto{\pgfqpoint{5.007724in}{3.120474in}}%
\pgfpathclose%
\pgfusepath{fill}%
\end{pgfscope}%
\begin{pgfscope}%
\pgfpathrectangle{\pgfqpoint{1.150000in}{0.150000in}}{\pgfqpoint{5.700000in}{5.700000in}}%
\pgfusepath{clip}%
\pgfsetbuttcap%
\pgfsetroundjoin%
\definecolor{currentfill}{rgb}{0.281412,0.155834,0.469201}%
\pgfsetfillcolor{currentfill}%
\pgfsetfillopacity{0.700000}%
\pgfsetlinewidth{0.000000pt}%
\definecolor{currentstroke}{rgb}{0.000000,0.000000,0.000000}%
\pgfsetstrokecolor{currentstroke}%
\pgfsetdash{}{0pt}%
\pgfpathmoveto{\pgfqpoint{3.575062in}{2.645197in}}%
\pgfpathlineto{\pgfqpoint{3.588420in}{2.638324in}}%
\pgfpathlineto{\pgfqpoint{3.601782in}{2.631547in}}%
\pgfpathlineto{\pgfqpoint{3.615147in}{2.624865in}}%
\pgfpathlineto{\pgfqpoint{3.628515in}{2.618278in}}%
\pgfpathlineto{\pgfqpoint{3.636405in}{2.630047in}}%
\pgfpathlineto{\pgfqpoint{3.644288in}{2.641944in}}%
\pgfpathlineto{\pgfqpoint{3.652167in}{2.653973in}}%
\pgfpathlineto{\pgfqpoint{3.660039in}{2.666139in}}%
\pgfpathlineto{\pgfqpoint{3.646678in}{2.672942in}}%
\pgfpathlineto{\pgfqpoint{3.633321in}{2.679839in}}%
\pgfpathlineto{\pgfqpoint{3.619968in}{2.686832in}}%
\pgfpathlineto{\pgfqpoint{3.606617in}{2.693920in}}%
\pgfpathlineto{\pgfqpoint{3.598737in}{2.681531in}}%
\pgfpathlineto{\pgfqpoint{3.590851in}{2.669284in}}%
\pgfpathlineto{\pgfqpoint{3.582960in}{2.657174in}}%
\pgfpathlineto{\pgfqpoint{3.575062in}{2.645197in}}%
\pgfpathclose%
\pgfusepath{fill}%
\end{pgfscope}%
\begin{pgfscope}%
\pgfpathrectangle{\pgfqpoint{1.150000in}{0.150000in}}{\pgfqpoint{5.700000in}{5.700000in}}%
\pgfusepath{clip}%
\pgfsetbuttcap%
\pgfsetroundjoin%
\definecolor{currentfill}{rgb}{0.266580,0.228262,0.514349}%
\pgfsetfillcolor{currentfill}%
\pgfsetfillopacity{0.700000}%
\pgfsetlinewidth{0.000000pt}%
\definecolor{currentstroke}{rgb}{0.000000,0.000000,0.000000}%
\pgfsetstrokecolor{currentstroke}%
\pgfsetdash{}{0pt}%
\pgfpathmoveto{\pgfqpoint{2.828367in}{2.801925in}}%
\pgfpathlineto{\pgfqpoint{2.841704in}{2.790871in}}%
\pgfpathlineto{\pgfqpoint{2.855040in}{2.779949in}}%
\pgfpathlineto{\pgfqpoint{2.868375in}{2.769160in}}%
\pgfpathlineto{\pgfqpoint{2.881709in}{2.758502in}}%
\pgfpathlineto{\pgfqpoint{2.889825in}{2.769613in}}%
\pgfpathlineto{\pgfqpoint{2.897933in}{2.780845in}}%
\pgfpathlineto{\pgfqpoint{2.906034in}{2.792200in}}%
\pgfpathlineto{\pgfqpoint{2.914126in}{2.803679in}}%
\pgfpathlineto{\pgfqpoint{2.900802in}{2.814453in}}%
\pgfpathlineto{\pgfqpoint{2.887476in}{2.825357in}}%
\pgfpathlineto{\pgfqpoint{2.874150in}{2.836394in}}%
\pgfpathlineto{\pgfqpoint{2.860823in}{2.847563in}}%
\pgfpathlineto{\pgfqpoint{2.852721in}{2.835961in}}%
\pgfpathlineto{\pgfqpoint{2.844611in}{2.824489in}}%
\pgfpathlineto{\pgfqpoint{2.836493in}{2.813144in}}%
\pgfpathlineto{\pgfqpoint{2.828367in}{2.801925in}}%
\pgfpathclose%
\pgfusepath{fill}%
\end{pgfscope}%
\begin{pgfscope}%
\pgfpathrectangle{\pgfqpoint{1.150000in}{0.150000in}}{\pgfqpoint{5.700000in}{5.700000in}}%
\pgfusepath{clip}%
\pgfsetbuttcap%
\pgfsetroundjoin%
\definecolor{currentfill}{rgb}{0.177423,0.437527,0.557565}%
\pgfsetfillcolor{currentfill}%
\pgfsetfillopacity{0.700000}%
\pgfsetlinewidth{0.000000pt}%
\definecolor{currentstroke}{rgb}{0.000000,0.000000,0.000000}%
\pgfsetstrokecolor{currentstroke}%
\pgfsetdash{}{0pt}%
\pgfpathmoveto{\pgfqpoint{5.123477in}{3.272325in}}%
\pgfpathlineto{\pgfqpoint{5.137076in}{3.265194in}}%
\pgfpathlineto{\pgfqpoint{5.150680in}{3.258134in}}%
\pgfpathlineto{\pgfqpoint{5.164289in}{3.251144in}}%
\pgfpathlineto{\pgfqpoint{5.177905in}{3.244224in}}%
\pgfpathlineto{\pgfqpoint{5.185600in}{3.267967in}}%
\pgfpathlineto{\pgfqpoint{5.193305in}{3.292226in}}%
\pgfpathlineto{\pgfqpoint{5.201021in}{3.317012in}}%
\pgfpathlineto{\pgfqpoint{5.208747in}{3.342335in}}%
\pgfpathlineto{\pgfqpoint{5.195141in}{3.349816in}}%
\pgfpathlineto{\pgfqpoint{5.181540in}{3.357367in}}%
\pgfpathlineto{\pgfqpoint{5.167945in}{3.364988in}}%
\pgfpathlineto{\pgfqpoint{5.154356in}{3.372680in}}%
\pgfpathlineto{\pgfqpoint{5.146621in}{3.346788in}}%
\pgfpathlineto{\pgfqpoint{5.138896in}{3.321438in}}%
\pgfpathlineto{\pgfqpoint{5.131182in}{3.296621in}}%
\pgfpathlineto{\pgfqpoint{5.123477in}{3.272325in}}%
\pgfpathclose%
\pgfusepath{fill}%
\end{pgfscope}%
\begin{pgfscope}%
\pgfpathrectangle{\pgfqpoint{1.150000in}{0.150000in}}{\pgfqpoint{5.700000in}{5.700000in}}%
\pgfusepath{clip}%
\pgfsetbuttcap%
\pgfsetroundjoin%
\definecolor{currentfill}{rgb}{0.276194,0.190074,0.493001}%
\pgfsetfillcolor{currentfill}%
\pgfsetfillopacity{0.700000}%
\pgfsetlinewidth{0.000000pt}%
\definecolor{currentstroke}{rgb}{0.000000,0.000000,0.000000}%
\pgfsetstrokecolor{currentstroke}%
\pgfsetdash{}{0pt}%
\pgfpathmoveto{\pgfqpoint{3.020717in}{2.722063in}}%
\pgfpathlineto{\pgfqpoint{3.034042in}{2.712416in}}%
\pgfpathlineto{\pgfqpoint{3.047367in}{2.702890in}}%
\pgfpathlineto{\pgfqpoint{3.060693in}{2.693483in}}%
\pgfpathlineto{\pgfqpoint{3.074020in}{2.684195in}}%
\pgfpathlineto{\pgfqpoint{3.082078in}{2.695418in}}%
\pgfpathlineto{\pgfqpoint{3.090128in}{2.706756in}}%
\pgfpathlineto{\pgfqpoint{3.098172in}{2.718211in}}%
\pgfpathlineto{\pgfqpoint{3.106208in}{2.729787in}}%
\pgfpathlineto{\pgfqpoint{3.092891in}{2.739212in}}%
\pgfpathlineto{\pgfqpoint{3.079574in}{2.748754in}}%
\pgfpathlineto{\pgfqpoint{3.066257in}{2.758415in}}%
\pgfpathlineto{\pgfqpoint{3.052941in}{2.768197in}}%
\pgfpathlineto{\pgfqpoint{3.044896in}{2.756478in}}%
\pgfpathlineto{\pgfqpoint{3.036844in}{2.744884in}}%
\pgfpathlineto{\pgfqpoint{3.028784in}{2.733413in}}%
\pgfpathlineto{\pgfqpoint{3.020717in}{2.722063in}}%
\pgfpathclose%
\pgfusepath{fill}%
\end{pgfscope}%
\begin{pgfscope}%
\pgfpathrectangle{\pgfqpoint{1.150000in}{0.150000in}}{\pgfqpoint{5.700000in}{5.700000in}}%
\pgfusepath{clip}%
\pgfsetbuttcap%
\pgfsetroundjoin%
\definecolor{currentfill}{rgb}{0.274128,0.199721,0.498911}%
\pgfsetfillcolor{currentfill}%
\pgfsetfillopacity{0.700000}%
\pgfsetlinewidth{0.000000pt}%
\definecolor{currentstroke}{rgb}{0.000000,0.000000,0.000000}%
\pgfsetstrokecolor{currentstroke}%
\pgfsetdash{}{0pt}%
\pgfpathmoveto{\pgfqpoint{4.245015in}{2.730553in}}%
\pgfpathlineto{\pgfqpoint{4.258487in}{2.725234in}}%
\pgfpathlineto{\pgfqpoint{4.271965in}{2.719996in}}%
\pgfpathlineto{\pgfqpoint{4.285449in}{2.714837in}}%
\pgfpathlineto{\pgfqpoint{4.298938in}{2.709757in}}%
\pgfpathlineto{\pgfqpoint{4.306642in}{2.722945in}}%
\pgfpathlineto{\pgfqpoint{4.314344in}{2.736337in}}%
\pgfpathlineto{\pgfqpoint{4.322043in}{2.749940in}}%
\pgfpathlineto{\pgfqpoint{4.329739in}{2.763760in}}%
\pgfpathlineto{\pgfqpoint{4.316259in}{2.769175in}}%
\pgfpathlineto{\pgfqpoint{4.302785in}{2.774670in}}%
\pgfpathlineto{\pgfqpoint{4.289316in}{2.780245in}}%
\pgfpathlineto{\pgfqpoint{4.275852in}{2.785899in}}%
\pgfpathlineto{\pgfqpoint{4.268147in}{2.771736in}}%
\pgfpathlineto{\pgfqpoint{4.260440in}{2.757795in}}%
\pgfpathlineto{\pgfqpoint{4.252729in}{2.744069in}}%
\pgfpathlineto{\pgfqpoint{4.245015in}{2.730553in}}%
\pgfpathclose%
\pgfusepath{fill}%
\end{pgfscope}%
\begin{pgfscope}%
\pgfpathrectangle{\pgfqpoint{1.150000in}{0.150000in}}{\pgfqpoint{5.700000in}{5.700000in}}%
\pgfusepath{clip}%
\pgfsetbuttcap%
\pgfsetroundjoin%
\definecolor{currentfill}{rgb}{0.280255,0.165693,0.476498}%
\pgfsetfillcolor{currentfill}%
\pgfsetfillopacity{0.700000}%
\pgfsetlinewidth{0.000000pt}%
\definecolor{currentstroke}{rgb}{0.000000,0.000000,0.000000}%
\pgfsetstrokecolor{currentstroke}%
\pgfsetdash{}{0pt}%
\pgfpathmoveto{\pgfqpoint{3.936897in}{2.664663in}}%
\pgfpathlineto{\pgfqpoint{3.950313in}{2.658902in}}%
\pgfpathlineto{\pgfqpoint{3.963734in}{2.653226in}}%
\pgfpathlineto{\pgfqpoint{3.977160in}{2.647637in}}%
\pgfpathlineto{\pgfqpoint{3.990590in}{2.642133in}}%
\pgfpathlineto{\pgfqpoint{3.998377in}{2.654387in}}%
\pgfpathlineto{\pgfqpoint{4.006159in}{2.666798in}}%
\pgfpathlineto{\pgfqpoint{4.013937in}{2.679373in}}%
\pgfpathlineto{\pgfqpoint{4.021711in}{2.692115in}}%
\pgfpathlineto{\pgfqpoint{4.008289in}{2.697895in}}%
\pgfpathlineto{\pgfqpoint{3.994871in}{2.703760in}}%
\pgfpathlineto{\pgfqpoint{3.981458in}{2.709711in}}%
\pgfpathlineto{\pgfqpoint{3.968050in}{2.715748in}}%
\pgfpathlineto{\pgfqpoint{3.960268in}{2.702722in}}%
\pgfpathlineto{\pgfqpoint{3.952482in}{2.689870in}}%
\pgfpathlineto{\pgfqpoint{3.944692in}{2.677185in}}%
\pgfpathlineto{\pgfqpoint{3.936897in}{2.664663in}}%
\pgfpathclose%
\pgfusepath{fill}%
\end{pgfscope}%
\begin{pgfscope}%
\pgfpathrectangle{\pgfqpoint{1.150000in}{0.150000in}}{\pgfqpoint{5.700000in}{5.700000in}}%
\pgfusepath{clip}%
\pgfsetbuttcap%
\pgfsetroundjoin%
\definecolor{currentfill}{rgb}{0.246811,0.283237,0.535941}%
\pgfsetfillcolor{currentfill}%
\pgfsetfillopacity{0.700000}%
\pgfsetlinewidth{0.000000pt}%
\definecolor{currentstroke}{rgb}{0.000000,0.000000,0.000000}%
\pgfsetstrokecolor{currentstroke}%
\pgfsetdash{}{0pt}%
\pgfpathmoveto{\pgfqpoint{4.722690in}{2.896851in}}%
\pgfpathlineto{\pgfqpoint{4.736254in}{2.891494in}}%
\pgfpathlineto{\pgfqpoint{4.749824in}{2.886210in}}%
\pgfpathlineto{\pgfqpoint{4.763400in}{2.881000in}}%
\pgfpathlineto{\pgfqpoint{4.776982in}{2.875863in}}%
\pgfpathlineto{\pgfqpoint{4.784601in}{2.891839in}}%
\pgfpathlineto{\pgfqpoint{4.792221in}{2.908130in}}%
\pgfpathlineto{\pgfqpoint{4.799843in}{2.924744in}}%
\pgfpathlineto{\pgfqpoint{4.807466in}{2.941688in}}%
\pgfpathlineto{\pgfqpoint{4.793895in}{2.947261in}}%
\pgfpathlineto{\pgfqpoint{4.780329in}{2.952908in}}%
\pgfpathlineto{\pgfqpoint{4.766770in}{2.958627in}}%
\pgfpathlineto{\pgfqpoint{4.753216in}{2.964421in}}%
\pgfpathlineto{\pgfqpoint{4.745583in}{2.947033in}}%
\pgfpathlineto{\pgfqpoint{4.737951in}{2.929981in}}%
\pgfpathlineto{\pgfqpoint{4.730320in}{2.913256in}}%
\pgfpathlineto{\pgfqpoint{4.722690in}{2.896851in}}%
\pgfpathclose%
\pgfusepath{fill}%
\end{pgfscope}%
\begin{pgfscope}%
\pgfpathrectangle{\pgfqpoint{1.150000in}{0.150000in}}{\pgfqpoint{5.700000in}{5.700000in}}%
\pgfusepath{clip}%
\pgfsetbuttcap%
\pgfsetroundjoin%
\definecolor{currentfill}{rgb}{0.281412,0.155834,0.469201}%
\pgfsetfillcolor{currentfill}%
\pgfsetfillopacity{0.700000}%
\pgfsetlinewidth{0.000000pt}%
\definecolor{currentstroke}{rgb}{0.000000,0.000000,0.000000}%
\pgfsetstrokecolor{currentstroke}%
\pgfsetdash{}{0pt}%
\pgfpathmoveto{\pgfqpoint{3.713520in}{2.639865in}}%
\pgfpathlineto{\pgfqpoint{3.726899in}{2.633528in}}%
\pgfpathlineto{\pgfqpoint{3.740283in}{2.627283in}}%
\pgfpathlineto{\pgfqpoint{3.753671in}{2.621129in}}%
\pgfpathlineto{\pgfqpoint{3.767063in}{2.615065in}}%
\pgfpathlineto{\pgfqpoint{3.774915in}{2.626915in}}%
\pgfpathlineto{\pgfqpoint{3.782761in}{2.638900in}}%
\pgfpathlineto{\pgfqpoint{3.790602in}{2.651025in}}%
\pgfpathlineto{\pgfqpoint{3.798437in}{2.663294in}}%
\pgfpathlineto{\pgfqpoint{3.785053in}{2.669593in}}%
\pgfpathlineto{\pgfqpoint{3.771673in}{2.675983in}}%
\pgfpathlineto{\pgfqpoint{3.758297in}{2.682464in}}%
\pgfpathlineto{\pgfqpoint{3.744925in}{2.689037in}}%
\pgfpathlineto{\pgfqpoint{3.737082in}{2.676525in}}%
\pgfpathlineto{\pgfqpoint{3.729233in}{2.664162in}}%
\pgfpathlineto{\pgfqpoint{3.721379in}{2.651944in}}%
\pgfpathlineto{\pgfqpoint{3.713520in}{2.639865in}}%
\pgfpathclose%
\pgfusepath{fill}%
\end{pgfscope}%
\begin{pgfscope}%
\pgfpathrectangle{\pgfqpoint{1.150000in}{0.150000in}}{\pgfqpoint{5.700000in}{5.700000in}}%
\pgfusepath{clip}%
\pgfsetbuttcap%
\pgfsetroundjoin%
\definecolor{currentfill}{rgb}{0.239346,0.300855,0.540844}%
\pgfsetfillcolor{currentfill}%
\pgfsetfillopacity{0.700000}%
\pgfsetlinewidth{0.000000pt}%
\definecolor{currentstroke}{rgb}{0.000000,0.000000,0.000000}%
\pgfsetstrokecolor{currentstroke}%
\pgfsetdash{}{0pt}%
\pgfpathmoveto{\pgfqpoint{4.807466in}{2.941688in}}%
\pgfpathlineto{\pgfqpoint{4.821044in}{2.936187in}}%
\pgfpathlineto{\pgfqpoint{4.834628in}{2.930759in}}%
\pgfpathlineto{\pgfqpoint{4.848218in}{2.925403in}}%
\pgfpathlineto{\pgfqpoint{4.861814in}{2.920120in}}%
\pgfpathlineto{\pgfqpoint{4.869428in}{2.936954in}}%
\pgfpathlineto{\pgfqpoint{4.877045in}{2.954130in}}%
\pgfpathlineto{\pgfqpoint{4.884664in}{2.971657in}}%
\pgfpathlineto{\pgfqpoint{4.892286in}{2.989543in}}%
\pgfpathlineto{\pgfqpoint{4.878701in}{2.995283in}}%
\pgfpathlineto{\pgfqpoint{4.865122in}{3.001095in}}%
\pgfpathlineto{\pgfqpoint{4.851549in}{3.006980in}}%
\pgfpathlineto{\pgfqpoint{4.837981in}{3.012937in}}%
\pgfpathlineto{\pgfqpoint{4.830349in}{2.994587in}}%
\pgfpathlineto{\pgfqpoint{4.822719in}{2.976601in}}%
\pgfpathlineto{\pgfqpoint{4.815091in}{2.958971in}}%
\pgfpathlineto{\pgfqpoint{4.807466in}{2.941688in}}%
\pgfpathclose%
\pgfusepath{fill}%
\end{pgfscope}%
\begin{pgfscope}%
\pgfpathrectangle{\pgfqpoint{1.150000in}{0.150000in}}{\pgfqpoint{5.700000in}{5.700000in}}%
\pgfusepath{clip}%
\pgfsetbuttcap%
\pgfsetroundjoin%
\definecolor{currentfill}{rgb}{0.255645,0.260703,0.528312}%
\pgfsetfillcolor{currentfill}%
\pgfsetfillopacity{0.700000}%
\pgfsetlinewidth{0.000000pt}%
\definecolor{currentstroke}{rgb}{0.000000,0.000000,0.000000}%
\pgfsetstrokecolor{currentstroke}%
\pgfsetdash{}{0pt}%
\pgfpathmoveto{\pgfqpoint{4.637942in}{2.854774in}}%
\pgfpathlineto{\pgfqpoint{4.651491in}{2.849537in}}%
\pgfpathlineto{\pgfqpoint{4.665047in}{2.844374in}}%
\pgfpathlineto{\pgfqpoint{4.678609in}{2.839285in}}%
\pgfpathlineto{\pgfqpoint{4.692177in}{2.834270in}}%
\pgfpathlineto{\pgfqpoint{4.699805in}{2.849475in}}%
\pgfpathlineto{\pgfqpoint{4.707433in}{2.864968in}}%
\pgfpathlineto{\pgfqpoint{4.715061in}{2.880758in}}%
\pgfpathlineto{\pgfqpoint{4.722690in}{2.896851in}}%
\pgfpathlineto{\pgfqpoint{4.709133in}{2.902282in}}%
\pgfpathlineto{\pgfqpoint{4.695581in}{2.907787in}}%
\pgfpathlineto{\pgfqpoint{4.682036in}{2.913366in}}%
\pgfpathlineto{\pgfqpoint{4.668496in}{2.919019in}}%
\pgfpathlineto{\pgfqpoint{4.660857in}{2.902502in}}%
\pgfpathlineto{\pgfqpoint{4.653218in}{2.886294in}}%
\pgfpathlineto{\pgfqpoint{4.645580in}{2.870387in}}%
\pgfpathlineto{\pgfqpoint{4.637942in}{2.854774in}}%
\pgfpathclose%
\pgfusepath{fill}%
\end{pgfscope}%
\begin{pgfscope}%
\pgfpathrectangle{\pgfqpoint{1.150000in}{0.150000in}}{\pgfqpoint{5.700000in}{5.700000in}}%
\pgfusepath{clip}%
\pgfsetbuttcap%
\pgfsetroundjoin%
\definecolor{currentfill}{rgb}{0.192357,0.403199,0.555836}%
\pgfsetfillcolor{currentfill}%
\pgfsetfillopacity{0.700000}%
\pgfsetlinewidth{0.000000pt}%
\definecolor{currentstroke}{rgb}{0.000000,0.000000,0.000000}%
\pgfsetstrokecolor{currentstroke}%
\pgfsetdash{}{0pt}%
\pgfpathmoveto{\pgfqpoint{5.092747in}{3.180143in}}%
\pgfpathlineto{\pgfqpoint{5.106355in}{3.173551in}}%
\pgfpathlineto{\pgfqpoint{5.119969in}{3.167030in}}%
\pgfpathlineto{\pgfqpoint{5.133589in}{3.160580in}}%
\pgfpathlineto{\pgfqpoint{5.147215in}{3.154199in}}%
\pgfpathlineto{\pgfqpoint{5.154875in}{3.175983in}}%
\pgfpathlineto{\pgfqpoint{5.162543in}{3.198242in}}%
\pgfpathlineto{\pgfqpoint{5.170219in}{3.220986in}}%
\pgfpathlineto{\pgfqpoint{5.177905in}{3.244224in}}%
\pgfpathlineto{\pgfqpoint{5.164289in}{3.251144in}}%
\pgfpathlineto{\pgfqpoint{5.150680in}{3.258134in}}%
\pgfpathlineto{\pgfqpoint{5.137076in}{3.265194in}}%
\pgfpathlineto{\pgfqpoint{5.123477in}{3.272325in}}%
\pgfpathlineto{\pgfqpoint{5.115782in}{3.248539in}}%
\pgfpathlineto{\pgfqpoint{5.108095in}{3.225254in}}%
\pgfpathlineto{\pgfqpoint{5.100417in}{3.202459in}}%
\pgfpathlineto{\pgfqpoint{5.092747in}{3.180143in}}%
\pgfpathclose%
\pgfusepath{fill}%
\end{pgfscope}%
\begin{pgfscope}%
\pgfpathrectangle{\pgfqpoint{1.150000in}{0.150000in}}{\pgfqpoint{5.700000in}{5.700000in}}%
\pgfusepath{clip}%
\pgfsetbuttcap%
\pgfsetroundjoin%
\definecolor{currentfill}{rgb}{0.229739,0.322361,0.545706}%
\pgfsetfillcolor{currentfill}%
\pgfsetfillopacity{0.700000}%
\pgfsetlinewidth{0.000000pt}%
\definecolor{currentstroke}{rgb}{0.000000,0.000000,0.000000}%
\pgfsetstrokecolor{currentstroke}%
\pgfsetdash{}{0pt}%
\pgfpathmoveto{\pgfqpoint{4.892286in}{2.989543in}}%
\pgfpathlineto{\pgfqpoint{4.905878in}{2.983875in}}%
\pgfpathlineto{\pgfqpoint{4.919475in}{2.978279in}}%
\pgfpathlineto{\pgfqpoint{4.933079in}{2.972754in}}%
\pgfpathlineto{\pgfqpoint{4.946690in}{2.967301in}}%
\pgfpathlineto{\pgfqpoint{4.954304in}{2.985085in}}%
\pgfpathlineto{\pgfqpoint{4.961922in}{3.003241in}}%
\pgfpathlineto{\pgfqpoint{4.969544in}{3.021777in}}%
\pgfpathlineto{\pgfqpoint{4.977170in}{3.040701in}}%
\pgfpathlineto{\pgfqpoint{4.963571in}{3.046631in}}%
\pgfpathlineto{\pgfqpoint{4.949978in}{3.052633in}}%
\pgfpathlineto{\pgfqpoint{4.936391in}{3.058706in}}%
\pgfpathlineto{\pgfqpoint{4.922810in}{3.064851in}}%
\pgfpathlineto{\pgfqpoint{4.915173in}{3.045441in}}%
\pgfpathlineto{\pgfqpoint{4.907541in}{3.026426in}}%
\pgfpathlineto{\pgfqpoint{4.899912in}{3.007796in}}%
\pgfpathlineto{\pgfqpoint{4.892286in}{2.989543in}}%
\pgfpathclose%
\pgfusepath{fill}%
\end{pgfscope}%
\begin{pgfscope}%
\pgfpathrectangle{\pgfqpoint{1.150000in}{0.150000in}}{\pgfqpoint{5.700000in}{5.700000in}}%
\pgfusepath{clip}%
\pgfsetbuttcap%
\pgfsetroundjoin%
\definecolor{currentfill}{rgb}{0.277134,0.185228,0.489898}%
\pgfsetfillcolor{currentfill}%
\pgfsetfillopacity{0.700000}%
\pgfsetlinewidth{0.000000pt}%
\definecolor{currentstroke}{rgb}{0.000000,0.000000,0.000000}%
\pgfsetstrokecolor{currentstroke}%
\pgfsetdash{}{0pt}%
\pgfpathmoveto{\pgfqpoint{4.160255in}{2.699283in}}%
\pgfpathlineto{\pgfqpoint{4.173715in}{2.693956in}}%
\pgfpathlineto{\pgfqpoint{4.187180in}{2.688711in}}%
\pgfpathlineto{\pgfqpoint{4.200650in}{2.683547in}}%
\pgfpathlineto{\pgfqpoint{4.214126in}{2.678464in}}%
\pgfpathlineto{\pgfqpoint{4.221854in}{2.691202in}}%
\pgfpathlineto{\pgfqpoint{4.229577in}{2.704125in}}%
\pgfpathlineto{\pgfqpoint{4.237298in}{2.717240in}}%
\pgfpathlineto{\pgfqpoint{4.245015in}{2.730553in}}%
\pgfpathlineto{\pgfqpoint{4.231548in}{2.735952in}}%
\pgfpathlineto{\pgfqpoint{4.218087in}{2.741432in}}%
\pgfpathlineto{\pgfqpoint{4.204630in}{2.746993in}}%
\pgfpathlineto{\pgfqpoint{4.191179in}{2.752635in}}%
\pgfpathlineto{\pgfqpoint{4.183453in}{2.738999in}}%
\pgfpathlineto{\pgfqpoint{4.175724in}{2.725566in}}%
\pgfpathlineto{\pgfqpoint{4.167992in}{2.712329in}}%
\pgfpathlineto{\pgfqpoint{4.160255in}{2.699283in}}%
\pgfpathclose%
\pgfusepath{fill}%
\end{pgfscope}%
\begin{pgfscope}%
\pgfpathrectangle{\pgfqpoint{1.150000in}{0.150000in}}{\pgfqpoint{5.700000in}{5.700000in}}%
\pgfusepath{clip}%
\pgfsetbuttcap%
\pgfsetroundjoin%
\definecolor{currentfill}{rgb}{0.262138,0.242286,0.520837}%
\pgfsetfillcolor{currentfill}%
\pgfsetfillopacity{0.700000}%
\pgfsetlinewidth{0.000000pt}%
\definecolor{currentstroke}{rgb}{0.000000,0.000000,0.000000}%
\pgfsetstrokecolor{currentstroke}%
\pgfsetdash{}{0pt}%
\pgfpathmoveto{\pgfqpoint{4.553205in}{2.815223in}}%
\pgfpathlineto{\pgfqpoint{4.566741in}{2.810081in}}%
\pgfpathlineto{\pgfqpoint{4.580283in}{2.805015in}}%
\pgfpathlineto{\pgfqpoint{4.593831in}{2.800024in}}%
\pgfpathlineto{\pgfqpoint{4.607385in}{2.795108in}}%
\pgfpathlineto{\pgfqpoint{4.615025in}{2.809621in}}%
\pgfpathlineto{\pgfqpoint{4.622664in}{2.824399in}}%
\pgfpathlineto{\pgfqpoint{4.630303in}{2.839447in}}%
\pgfpathlineto{\pgfqpoint{4.637942in}{2.854774in}}%
\pgfpathlineto{\pgfqpoint{4.624398in}{2.860086in}}%
\pgfpathlineto{\pgfqpoint{4.610860in}{2.865473in}}%
\pgfpathlineto{\pgfqpoint{4.597328in}{2.870935in}}%
\pgfpathlineto{\pgfqpoint{4.583802in}{2.876472in}}%
\pgfpathlineto{\pgfqpoint{4.576154in}{2.860742in}}%
\pgfpathlineto{\pgfqpoint{4.568505in}{2.845295in}}%
\pgfpathlineto{\pgfqpoint{4.560856in}{2.830124in}}%
\pgfpathlineto{\pgfqpoint{4.553205in}{2.815223in}}%
\pgfpathclose%
\pgfusepath{fill}%
\end{pgfscope}%
\begin{pgfscope}%
\pgfpathrectangle{\pgfqpoint{1.150000in}{0.150000in}}{\pgfqpoint{5.700000in}{5.700000in}}%
\pgfusepath{clip}%
\pgfsetbuttcap%
\pgfsetroundjoin%
\definecolor{currentfill}{rgb}{0.281412,0.155834,0.469201}%
\pgfsetfillcolor{currentfill}%
\pgfsetfillopacity{0.700000}%
\pgfsetlinewidth{0.000000pt}%
\definecolor{currentstroke}{rgb}{0.000000,0.000000,0.000000}%
\pgfsetstrokecolor{currentstroke}%
\pgfsetdash{}{0pt}%
\pgfpathmoveto{\pgfqpoint{3.351428in}{2.639444in}}%
\pgfpathlineto{\pgfqpoint{3.364766in}{2.631743in}}%
\pgfpathlineto{\pgfqpoint{3.378107in}{2.624146in}}%
\pgfpathlineto{\pgfqpoint{3.391450in}{2.616652in}}%
\pgfpathlineto{\pgfqpoint{3.404797in}{2.609260in}}%
\pgfpathlineto{\pgfqpoint{3.412758in}{2.620642in}}%
\pgfpathlineto{\pgfqpoint{3.420713in}{2.632139in}}%
\pgfpathlineto{\pgfqpoint{3.428662in}{2.643752in}}%
\pgfpathlineto{\pgfqpoint{3.436604in}{2.655487in}}%
\pgfpathlineto{\pgfqpoint{3.423266in}{2.663055in}}%
\pgfpathlineto{\pgfqpoint{3.409931in}{2.670725in}}%
\pgfpathlineto{\pgfqpoint{3.396598in}{2.678498in}}%
\pgfpathlineto{\pgfqpoint{3.383268in}{2.686374in}}%
\pgfpathlineto{\pgfqpoint{3.375318in}{2.674456in}}%
\pgfpathlineto{\pgfqpoint{3.367361in}{2.662664in}}%
\pgfpathlineto{\pgfqpoint{3.359397in}{2.650995in}}%
\pgfpathlineto{\pgfqpoint{3.351428in}{2.639444in}}%
\pgfpathclose%
\pgfusepath{fill}%
\end{pgfscope}%
\begin{pgfscope}%
\pgfpathrectangle{\pgfqpoint{1.150000in}{0.150000in}}{\pgfqpoint{5.700000in}{5.700000in}}%
\pgfusepath{clip}%
\pgfsetbuttcap%
\pgfsetroundjoin%
\definecolor{currentfill}{rgb}{0.271828,0.209303,0.504434}%
\pgfsetfillcolor{currentfill}%
\pgfsetfillopacity{0.700000}%
\pgfsetlinewidth{0.000000pt}%
\definecolor{currentstroke}{rgb}{0.000000,0.000000,0.000000}%
\pgfsetstrokecolor{currentstroke}%
\pgfsetdash{}{0pt}%
\pgfpathmoveto{\pgfqpoint{2.881709in}{2.758502in}}%
\pgfpathlineto{\pgfqpoint{2.895043in}{2.747973in}}%
\pgfpathlineto{\pgfqpoint{2.908377in}{2.737572in}}%
\pgfpathlineto{\pgfqpoint{2.921710in}{2.727299in}}%
\pgfpathlineto{\pgfqpoint{2.935044in}{2.717153in}}%
\pgfpathlineto{\pgfqpoint{2.943149in}{2.728157in}}%
\pgfpathlineto{\pgfqpoint{2.951248in}{2.739276in}}%
\pgfpathlineto{\pgfqpoint{2.959338in}{2.750512in}}%
\pgfpathlineto{\pgfqpoint{2.967422in}{2.761869in}}%
\pgfpathlineto{\pgfqpoint{2.954098in}{2.772131in}}%
\pgfpathlineto{\pgfqpoint{2.940775in}{2.782519in}}%
\pgfpathlineto{\pgfqpoint{2.927451in}{2.793035in}}%
\pgfpathlineto{\pgfqpoint{2.914126in}{2.803679in}}%
\pgfpathlineto{\pgfqpoint{2.906034in}{2.792200in}}%
\pgfpathlineto{\pgfqpoint{2.897933in}{2.780845in}}%
\pgfpathlineto{\pgfqpoint{2.889825in}{2.769613in}}%
\pgfpathlineto{\pgfqpoint{2.881709in}{2.758502in}}%
\pgfpathclose%
\pgfusepath{fill}%
\end{pgfscope}%
\begin{pgfscope}%
\pgfpathrectangle{\pgfqpoint{1.150000in}{0.150000in}}{\pgfqpoint{5.700000in}{5.700000in}}%
\pgfusepath{clip}%
\pgfsetbuttcap%
\pgfsetroundjoin%
\definecolor{currentfill}{rgb}{0.280868,0.160771,0.472899}%
\pgfsetfillcolor{currentfill}%
\pgfsetfillopacity{0.700000}%
\pgfsetlinewidth{0.000000pt}%
\definecolor{currentstroke}{rgb}{0.000000,0.000000,0.000000}%
\pgfsetstrokecolor{currentstroke}%
\pgfsetdash{}{0pt}%
\pgfpathmoveto{\pgfqpoint{3.212792in}{2.658543in}}%
\pgfpathlineto{\pgfqpoint{3.226122in}{2.650143in}}%
\pgfpathlineto{\pgfqpoint{3.239454in}{2.641852in}}%
\pgfpathlineto{\pgfqpoint{3.252787in}{2.633670in}}%
\pgfpathlineto{\pgfqpoint{3.266123in}{2.625597in}}%
\pgfpathlineto{\pgfqpoint{3.274126in}{2.636852in}}%
\pgfpathlineto{\pgfqpoint{3.282123in}{2.648218in}}%
\pgfpathlineto{\pgfqpoint{3.290114in}{2.659699in}}%
\pgfpathlineto{\pgfqpoint{3.298098in}{2.671297in}}%
\pgfpathlineto{\pgfqpoint{3.284771in}{2.679526in}}%
\pgfpathlineto{\pgfqpoint{3.271446in}{2.687864in}}%
\pgfpathlineto{\pgfqpoint{3.258123in}{2.696310in}}%
\pgfpathlineto{\pgfqpoint{3.244802in}{2.704866in}}%
\pgfpathlineto{\pgfqpoint{3.236810in}{2.693104in}}%
\pgfpathlineto{\pgfqpoint{3.228811in}{2.681466in}}%
\pgfpathlineto{\pgfqpoint{3.220805in}{2.669946in}}%
\pgfpathlineto{\pgfqpoint{3.212792in}{2.658543in}}%
\pgfpathclose%
\pgfusepath{fill}%
\end{pgfscope}%
\begin{pgfscope}%
\pgfpathrectangle{\pgfqpoint{1.150000in}{0.150000in}}{\pgfqpoint{5.700000in}{5.700000in}}%
\pgfusepath{clip}%
\pgfsetbuttcap%
\pgfsetroundjoin%
\definecolor{currentfill}{rgb}{0.218130,0.347432,0.550038}%
\pgfsetfillcolor{currentfill}%
\pgfsetfillopacity{0.700000}%
\pgfsetlinewidth{0.000000pt}%
\definecolor{currentstroke}{rgb}{0.000000,0.000000,0.000000}%
\pgfsetstrokecolor{currentstroke}%
\pgfsetdash{}{0pt}%
\pgfpathmoveto{\pgfqpoint{4.977170in}{3.040701in}}%
\pgfpathlineto{\pgfqpoint{4.990775in}{3.034843in}}%
\pgfpathlineto{\pgfqpoint{5.004387in}{3.029055in}}%
\pgfpathlineto{\pgfqpoint{5.018004in}{3.023338in}}%
\pgfpathlineto{\pgfqpoint{5.031628in}{3.017691in}}%
\pgfpathlineto{\pgfqpoint{5.039248in}{3.036525in}}%
\pgfpathlineto{\pgfqpoint{5.046872in}{3.055760in}}%
\pgfpathlineto{\pgfqpoint{5.054502in}{3.075407in}}%
\pgfpathlineto{\pgfqpoint{5.062138in}{3.095474in}}%
\pgfpathlineto{\pgfqpoint{5.048526in}{3.101618in}}%
\pgfpathlineto{\pgfqpoint{5.034919in}{3.107832in}}%
\pgfpathlineto{\pgfqpoint{5.021319in}{3.114117in}}%
\pgfpathlineto{\pgfqpoint{5.007724in}{3.120474in}}%
\pgfpathlineto{\pgfqpoint{5.000077in}{3.099901in}}%
\pgfpathlineto{\pgfqpoint{4.992437in}{3.079754in}}%
\pgfpathlineto{\pgfqpoint{4.984801in}{3.060024in}}%
\pgfpathlineto{\pgfqpoint{4.977170in}{3.040701in}}%
\pgfpathclose%
\pgfusepath{fill}%
\end{pgfscope}%
\begin{pgfscope}%
\pgfpathrectangle{\pgfqpoint{1.150000in}{0.150000in}}{\pgfqpoint{5.700000in}{5.700000in}}%
\pgfusepath{clip}%
\pgfsetbuttcap%
\pgfsetroundjoin%
\definecolor{currentfill}{rgb}{0.281887,0.150881,0.465405}%
\pgfsetfillcolor{currentfill}%
\pgfsetfillopacity{0.700000}%
\pgfsetlinewidth{0.000000pt}%
\definecolor{currentstroke}{rgb}{0.000000,0.000000,0.000000}%
\pgfsetstrokecolor{currentstroke}%
\pgfsetdash{}{0pt}%
\pgfpathmoveto{\pgfqpoint{3.489985in}{2.626224in}}%
\pgfpathlineto{\pgfqpoint{3.503338in}{2.619158in}}%
\pgfpathlineto{\pgfqpoint{3.516694in}{2.612189in}}%
\pgfpathlineto{\pgfqpoint{3.530053in}{2.605318in}}%
\pgfpathlineto{\pgfqpoint{3.543415in}{2.598545in}}%
\pgfpathlineto{\pgfqpoint{3.551336in}{2.610027in}}%
\pgfpathlineto{\pgfqpoint{3.559251in}{2.621627in}}%
\pgfpathlineto{\pgfqpoint{3.567159in}{2.633349in}}%
\pgfpathlineto{\pgfqpoint{3.575062in}{2.645197in}}%
\pgfpathlineto{\pgfqpoint{3.561708in}{2.652166in}}%
\pgfpathlineto{\pgfqpoint{3.548357in}{2.659233in}}%
\pgfpathlineto{\pgfqpoint{3.535009in}{2.666397in}}%
\pgfpathlineto{\pgfqpoint{3.521664in}{2.673659in}}%
\pgfpathlineto{\pgfqpoint{3.513753in}{2.661609in}}%
\pgfpathlineto{\pgfqpoint{3.505836in}{2.649688in}}%
\pgfpathlineto{\pgfqpoint{3.497914in}{2.637895in}}%
\pgfpathlineto{\pgfqpoint{3.489985in}{2.626224in}}%
\pgfpathclose%
\pgfusepath{fill}%
\end{pgfscope}%
\begin{pgfscope}%
\pgfpathrectangle{\pgfqpoint{1.150000in}{0.150000in}}{\pgfqpoint{5.700000in}{5.700000in}}%
\pgfusepath{clip}%
\pgfsetbuttcap%
\pgfsetroundjoin%
\definecolor{currentfill}{rgb}{0.266580,0.228262,0.514349}%
\pgfsetfillcolor{currentfill}%
\pgfsetfillopacity{0.700000}%
\pgfsetlinewidth{0.000000pt}%
\definecolor{currentstroke}{rgb}{0.000000,0.000000,0.000000}%
\pgfsetstrokecolor{currentstroke}%
\pgfsetdash{}{0pt}%
\pgfpathmoveto{\pgfqpoint{4.468467in}{2.777989in}}%
\pgfpathlineto{\pgfqpoint{4.481989in}{2.772919in}}%
\pgfpathlineto{\pgfqpoint{4.495516in}{2.767925in}}%
\pgfpathlineto{\pgfqpoint{4.509050in}{2.763008in}}%
\pgfpathlineto{\pgfqpoint{4.522590in}{2.758167in}}%
\pgfpathlineto{\pgfqpoint{4.530246in}{2.772062in}}%
\pgfpathlineto{\pgfqpoint{4.537901in}{2.786199in}}%
\pgfpathlineto{\pgfqpoint{4.545553in}{2.800583in}}%
\pgfpathlineto{\pgfqpoint{4.553205in}{2.815223in}}%
\pgfpathlineto{\pgfqpoint{4.539675in}{2.820440in}}%
\pgfpathlineto{\pgfqpoint{4.526151in}{2.825733in}}%
\pgfpathlineto{\pgfqpoint{4.512633in}{2.831103in}}%
\pgfpathlineto{\pgfqpoint{4.499121in}{2.836548in}}%
\pgfpathlineto{\pgfqpoint{4.491460in}{2.821526in}}%
\pgfpathlineto{\pgfqpoint{4.483797in}{2.806763in}}%
\pgfpathlineto{\pgfqpoint{4.476133in}{2.792253in}}%
\pgfpathlineto{\pgfqpoint{4.468467in}{2.777989in}}%
\pgfpathclose%
\pgfusepath{fill}%
\end{pgfscope}%
\begin{pgfscope}%
\pgfpathrectangle{\pgfqpoint{1.150000in}{0.150000in}}{\pgfqpoint{5.700000in}{5.700000in}}%
\pgfusepath{clip}%
\pgfsetbuttcap%
\pgfsetroundjoin%
\definecolor{currentfill}{rgb}{0.278826,0.175490,0.483397}%
\pgfsetfillcolor{currentfill}%
\pgfsetfillopacity{0.700000}%
\pgfsetlinewidth{0.000000pt}%
\definecolor{currentstroke}{rgb}{0.000000,0.000000,0.000000}%
\pgfsetstrokecolor{currentstroke}%
\pgfsetdash{}{0pt}%
\pgfpathmoveto{\pgfqpoint{3.074020in}{2.684195in}}%
\pgfpathlineto{\pgfqpoint{3.087348in}{2.675024in}}%
\pgfpathlineto{\pgfqpoint{3.100677in}{2.665969in}}%
\pgfpathlineto{\pgfqpoint{3.114007in}{2.657029in}}%
\pgfpathlineto{\pgfqpoint{3.127338in}{2.648205in}}%
\pgfpathlineto{\pgfqpoint{3.135386in}{2.659300in}}%
\pgfpathlineto{\pgfqpoint{3.143427in}{2.670505in}}%
\pgfpathlineto{\pgfqpoint{3.151462in}{2.681822in}}%
\pgfpathlineto{\pgfqpoint{3.159490in}{2.693255in}}%
\pgfpathlineto{\pgfqpoint{3.146168in}{2.702215in}}%
\pgfpathlineto{\pgfqpoint{3.132847in}{2.711290in}}%
\pgfpathlineto{\pgfqpoint{3.119527in}{2.720481in}}%
\pgfpathlineto{\pgfqpoint{3.106208in}{2.729787in}}%
\pgfpathlineto{\pgfqpoint{3.098172in}{2.718211in}}%
\pgfpathlineto{\pgfqpoint{3.090128in}{2.706756in}}%
\pgfpathlineto{\pgfqpoint{3.082078in}{2.695418in}}%
\pgfpathlineto{\pgfqpoint{3.074020in}{2.684195in}}%
\pgfpathclose%
\pgfusepath{fill}%
\end{pgfscope}%
\begin{pgfscope}%
\pgfpathrectangle{\pgfqpoint{1.150000in}{0.150000in}}{\pgfqpoint{5.700000in}{5.700000in}}%
\pgfusepath{clip}%
\pgfsetbuttcap%
\pgfsetroundjoin%
\definecolor{currentfill}{rgb}{0.281412,0.155834,0.469201}%
\pgfsetfillcolor{currentfill}%
\pgfsetfillopacity{0.700000}%
\pgfsetlinewidth{0.000000pt}%
\definecolor{currentstroke}{rgb}{0.000000,0.000000,0.000000}%
\pgfsetstrokecolor{currentstroke}%
\pgfsetdash{}{0pt}%
\pgfpathmoveto{\pgfqpoint{3.852017in}{2.638992in}}%
\pgfpathlineto{\pgfqpoint{3.865423in}{2.633138in}}%
\pgfpathlineto{\pgfqpoint{3.878833in}{2.627373in}}%
\pgfpathlineto{\pgfqpoint{3.892248in}{2.621694in}}%
\pgfpathlineto{\pgfqpoint{3.905668in}{2.616102in}}%
\pgfpathlineto{\pgfqpoint{3.913483in}{2.628023in}}%
\pgfpathlineto{\pgfqpoint{3.921292in}{2.640086in}}%
\pgfpathlineto{\pgfqpoint{3.929097in}{2.652298in}}%
\pgfpathlineto{\pgfqpoint{3.936897in}{2.664663in}}%
\pgfpathlineto{\pgfqpoint{3.923485in}{2.670510in}}%
\pgfpathlineto{\pgfqpoint{3.910078in}{2.676445in}}%
\pgfpathlineto{\pgfqpoint{3.896676in}{2.682466in}}%
\pgfpathlineto{\pgfqpoint{3.883278in}{2.688576in}}%
\pgfpathlineto{\pgfqpoint{3.875470in}{2.675948in}}%
\pgfpathlineto{\pgfqpoint{3.867657in}{2.663478in}}%
\pgfpathlineto{\pgfqpoint{3.859840in}{2.651161in}}%
\pgfpathlineto{\pgfqpoint{3.852017in}{2.638992in}}%
\pgfpathclose%
\pgfusepath{fill}%
\end{pgfscope}%
\begin{pgfscope}%
\pgfpathrectangle{\pgfqpoint{1.150000in}{0.150000in}}{\pgfqpoint{5.700000in}{5.700000in}}%
\pgfusepath{clip}%
\pgfsetbuttcap%
\pgfsetroundjoin%
\definecolor{currentfill}{rgb}{0.166617,0.463708,0.558119}%
\pgfsetfillcolor{currentfill}%
\pgfsetfillopacity{0.700000}%
\pgfsetlinewidth{0.000000pt}%
\definecolor{currentstroke}{rgb}{0.000000,0.000000,0.000000}%
\pgfsetstrokecolor{currentstroke}%
\pgfsetdash{}{0pt}%
\pgfpathmoveto{\pgfqpoint{5.208747in}{3.342335in}}%
\pgfpathlineto{\pgfqpoint{5.222359in}{3.334925in}}%
\pgfpathlineto{\pgfqpoint{5.235976in}{3.327584in}}%
\pgfpathlineto{\pgfqpoint{5.249599in}{3.320312in}}%
\pgfpathlineto{\pgfqpoint{5.263228in}{3.313110in}}%
\pgfpathlineto{\pgfqpoint{5.270956in}{3.338408in}}%
\pgfpathlineto{\pgfqpoint{5.278696in}{3.364260in}}%
\pgfpathlineto{\pgfqpoint{5.286448in}{3.390677in}}%
\pgfpathlineto{\pgfqpoint{5.272827in}{3.398314in}}%
\pgfpathlineto{\pgfqpoint{5.259210in}{3.406020in}}%
\pgfpathlineto{\pgfqpoint{5.245600in}{3.413796in}}%
\pgfpathlineto{\pgfqpoint{5.231995in}{3.421642in}}%
\pgfpathlineto{\pgfqpoint{5.224233in}{3.394639in}}%
\pgfpathlineto{\pgfqpoint{5.216484in}{3.368207in}}%
\pgfpathlineto{\pgfqpoint{5.208747in}{3.342335in}}%
\pgfpathclose%
\pgfusepath{fill}%
\end{pgfscope}%
\begin{pgfscope}%
\pgfpathrectangle{\pgfqpoint{1.150000in}{0.150000in}}{\pgfqpoint{5.700000in}{5.700000in}}%
\pgfusepath{clip}%
\pgfsetbuttcap%
\pgfsetroundjoin%
\definecolor{currentfill}{rgb}{0.278826,0.175490,0.483397}%
\pgfsetfillcolor{currentfill}%
\pgfsetfillopacity{0.700000}%
\pgfsetlinewidth{0.000000pt}%
\definecolor{currentstroke}{rgb}{0.000000,0.000000,0.000000}%
\pgfsetstrokecolor{currentstroke}%
\pgfsetdash{}{0pt}%
\pgfpathmoveto{\pgfqpoint{4.075450in}{2.669839in}}%
\pgfpathlineto{\pgfqpoint{4.088897in}{2.664479in}}%
\pgfpathlineto{\pgfqpoint{4.102350in}{2.659201in}}%
\pgfpathlineto{\pgfqpoint{4.115808in}{2.654007in}}%
\pgfpathlineto{\pgfqpoint{4.129271in}{2.648894in}}%
\pgfpathlineto{\pgfqpoint{4.137023in}{2.661233in}}%
\pgfpathlineto{\pgfqpoint{4.144771in}{2.673741in}}%
\pgfpathlineto{\pgfqpoint{4.152515in}{2.686422in}}%
\pgfpathlineto{\pgfqpoint{4.160255in}{2.699283in}}%
\pgfpathlineto{\pgfqpoint{4.146801in}{2.704691in}}%
\pgfpathlineto{\pgfqpoint{4.133352in}{2.710182in}}%
\pgfpathlineto{\pgfqpoint{4.119908in}{2.715755in}}%
\pgfpathlineto{\pgfqpoint{4.106469in}{2.721411in}}%
\pgfpathlineto{\pgfqpoint{4.098720in}{2.708246in}}%
\pgfpathlineto{\pgfqpoint{4.090967in}{2.695267in}}%
\pgfpathlineto{\pgfqpoint{4.083211in}{2.682466in}}%
\pgfpathlineto{\pgfqpoint{4.075450in}{2.669839in}}%
\pgfpathclose%
\pgfusepath{fill}%
\end{pgfscope}%
\begin{pgfscope}%
\pgfpathrectangle{\pgfqpoint{1.150000in}{0.150000in}}{\pgfqpoint{5.700000in}{5.700000in}}%
\pgfusepath{clip}%
\pgfsetbuttcap%
\pgfsetroundjoin%
\definecolor{currentfill}{rgb}{0.271828,0.209303,0.504434}%
\pgfsetfillcolor{currentfill}%
\pgfsetfillopacity{0.700000}%
\pgfsetlinewidth{0.000000pt}%
\definecolor{currentstroke}{rgb}{0.000000,0.000000,0.000000}%
\pgfsetstrokecolor{currentstroke}%
\pgfsetdash{}{0pt}%
\pgfpathmoveto{\pgfqpoint{4.383715in}{2.742887in}}%
\pgfpathlineto{\pgfqpoint{4.397223in}{2.737864in}}%
\pgfpathlineto{\pgfqpoint{4.410737in}{2.732919in}}%
\pgfpathlineto{\pgfqpoint{4.424257in}{2.728051in}}%
\pgfpathlineto{\pgfqpoint{4.437783in}{2.723260in}}%
\pgfpathlineto{\pgfqpoint{4.445457in}{2.736606in}}%
\pgfpathlineto{\pgfqpoint{4.453129in}{2.750172in}}%
\pgfpathlineto{\pgfqpoint{4.460799in}{2.763964in}}%
\pgfpathlineto{\pgfqpoint{4.468467in}{2.777989in}}%
\pgfpathlineto{\pgfqpoint{4.454951in}{2.783135in}}%
\pgfpathlineto{\pgfqpoint{4.441440in}{2.788359in}}%
\pgfpathlineto{\pgfqpoint{4.427936in}{2.793660in}}%
\pgfpathlineto{\pgfqpoint{4.414437in}{2.799039in}}%
\pgfpathlineto{\pgfqpoint{4.406760in}{2.784651in}}%
\pgfpathlineto{\pgfqpoint{4.399080in}{2.770501in}}%
\pgfpathlineto{\pgfqpoint{4.391399in}{2.756581in}}%
\pgfpathlineto{\pgfqpoint{4.383715in}{2.742887in}}%
\pgfpathclose%
\pgfusepath{fill}%
\end{pgfscope}%
\begin{pgfscope}%
\pgfpathrectangle{\pgfqpoint{1.150000in}{0.150000in}}{\pgfqpoint{5.700000in}{5.700000in}}%
\pgfusepath{clip}%
\pgfsetbuttcap%
\pgfsetroundjoin%
\definecolor{currentfill}{rgb}{0.282290,0.145912,0.461510}%
\pgfsetfillcolor{currentfill}%
\pgfsetfillopacity{0.700000}%
\pgfsetlinewidth{0.000000pt}%
\definecolor{currentstroke}{rgb}{0.000000,0.000000,0.000000}%
\pgfsetstrokecolor{currentstroke}%
\pgfsetdash{}{0pt}%
\pgfpathmoveto{\pgfqpoint{3.628515in}{2.618278in}}%
\pgfpathlineto{\pgfqpoint{3.641888in}{2.611786in}}%
\pgfpathlineto{\pgfqpoint{3.655264in}{2.605387in}}%
\pgfpathlineto{\pgfqpoint{3.668644in}{2.599081in}}%
\pgfpathlineto{\pgfqpoint{3.682028in}{2.592868in}}%
\pgfpathlineto{\pgfqpoint{3.689909in}{2.604428in}}%
\pgfpathlineto{\pgfqpoint{3.697785in}{2.616112in}}%
\pgfpathlineto{\pgfqpoint{3.705655in}{2.627923in}}%
\pgfpathlineto{\pgfqpoint{3.713520in}{2.639865in}}%
\pgfpathlineto{\pgfqpoint{3.700144in}{2.646294in}}%
\pgfpathlineto{\pgfqpoint{3.686772in}{2.652816in}}%
\pgfpathlineto{\pgfqpoint{3.673404in}{2.659431in}}%
\pgfpathlineto{\pgfqpoint{3.660039in}{2.666139in}}%
\pgfpathlineto{\pgfqpoint{3.652167in}{2.653973in}}%
\pgfpathlineto{\pgfqpoint{3.644288in}{2.641944in}}%
\pgfpathlineto{\pgfqpoint{3.636405in}{2.630047in}}%
\pgfpathlineto{\pgfqpoint{3.628515in}{2.618278in}}%
\pgfpathclose%
\pgfusepath{fill}%
\end{pgfscope}%
\begin{pgfscope}%
\pgfpathrectangle{\pgfqpoint{1.150000in}{0.150000in}}{\pgfqpoint{5.700000in}{5.700000in}}%
\pgfusepath{clip}%
\pgfsetbuttcap%
\pgfsetroundjoin%
\definecolor{currentfill}{rgb}{0.206756,0.371758,0.553117}%
\pgfsetfillcolor{currentfill}%
\pgfsetfillopacity{0.700000}%
\pgfsetlinewidth{0.000000pt}%
\definecolor{currentstroke}{rgb}{0.000000,0.000000,0.000000}%
\pgfsetstrokecolor{currentstroke}%
\pgfsetdash{}{0pt}%
\pgfpathmoveto{\pgfqpoint{5.062138in}{3.095474in}}%
\pgfpathlineto{\pgfqpoint{5.075757in}{3.089401in}}%
\pgfpathlineto{\pgfqpoint{5.089382in}{3.083398in}}%
\pgfpathlineto{\pgfqpoint{5.103013in}{3.077465in}}%
\pgfpathlineto{\pgfqpoint{5.116651in}{3.071602in}}%
\pgfpathlineto{\pgfqpoint{5.124281in}{3.091590in}}%
\pgfpathlineto{\pgfqpoint{5.131919in}{3.112012in}}%
\pgfpathlineto{\pgfqpoint{5.139563in}{3.132878in}}%
\pgfpathlineto{\pgfqpoint{5.147215in}{3.154199in}}%
\pgfpathlineto{\pgfqpoint{5.133589in}{3.160580in}}%
\pgfpathlineto{\pgfqpoint{5.119969in}{3.167030in}}%
\pgfpathlineto{\pgfqpoint{5.106355in}{3.173551in}}%
\pgfpathlineto{\pgfqpoint{5.092747in}{3.180143in}}%
\pgfpathlineto{\pgfqpoint{5.085084in}{3.158296in}}%
\pgfpathlineto{\pgfqpoint{5.077429in}{3.136909in}}%
\pgfpathlineto{\pgfqpoint{5.069780in}{3.115972in}}%
\pgfpathlineto{\pgfqpoint{5.062138in}{3.095474in}}%
\pgfpathclose%
\pgfusepath{fill}%
\end{pgfscope}%
\begin{pgfscope}%
\pgfpathrectangle{\pgfqpoint{1.150000in}{0.150000in}}{\pgfqpoint{5.700000in}{5.700000in}}%
\pgfusepath{clip}%
\pgfsetbuttcap%
\pgfsetroundjoin%
\definecolor{currentfill}{rgb}{0.180629,0.429975,0.557282}%
\pgfsetfillcolor{currentfill}%
\pgfsetfillopacity{0.700000}%
\pgfsetlinewidth{0.000000pt}%
\definecolor{currentstroke}{rgb}{0.000000,0.000000,0.000000}%
\pgfsetstrokecolor{currentstroke}%
\pgfsetdash{}{0pt}%
\pgfpathmoveto{\pgfqpoint{5.177905in}{3.244224in}}%
\pgfpathlineto{\pgfqpoint{5.191526in}{3.237374in}}%
\pgfpathlineto{\pgfqpoint{5.205154in}{3.230593in}}%
\pgfpathlineto{\pgfqpoint{5.218787in}{3.223882in}}%
\pgfpathlineto{\pgfqpoint{5.232427in}{3.217241in}}%
\pgfpathlineto{\pgfqpoint{5.240111in}{3.240432in}}%
\pgfpathlineto{\pgfqpoint{5.247806in}{3.264133in}}%
\pgfpathlineto{\pgfqpoint{5.255512in}{3.288356in}}%
\pgfpathlineto{\pgfqpoint{5.263228in}{3.313110in}}%
\pgfpathlineto{\pgfqpoint{5.249599in}{3.320312in}}%
\pgfpathlineto{\pgfqpoint{5.235976in}{3.327584in}}%
\pgfpathlineto{\pgfqpoint{5.222359in}{3.334925in}}%
\pgfpathlineto{\pgfqpoint{5.208747in}{3.342335in}}%
\pgfpathlineto{\pgfqpoint{5.201021in}{3.317012in}}%
\pgfpathlineto{\pgfqpoint{5.193305in}{3.292226in}}%
\pgfpathlineto{\pgfqpoint{5.185600in}{3.267967in}}%
\pgfpathlineto{\pgfqpoint{5.177905in}{3.244224in}}%
\pgfpathclose%
\pgfusepath{fill}%
\end{pgfscope}%
\begin{pgfscope}%
\pgfpathrectangle{\pgfqpoint{1.150000in}{0.150000in}}{\pgfqpoint{5.700000in}{5.700000in}}%
\pgfusepath{clip}%
\pgfsetbuttcap%
\pgfsetroundjoin%
\definecolor{currentfill}{rgb}{0.275191,0.194905,0.496005}%
\pgfsetfillcolor{currentfill}%
\pgfsetfillopacity{0.700000}%
\pgfsetlinewidth{0.000000pt}%
\definecolor{currentstroke}{rgb}{0.000000,0.000000,0.000000}%
\pgfsetstrokecolor{currentstroke}%
\pgfsetdash{}{0pt}%
\pgfpathmoveto{\pgfqpoint{2.935044in}{2.717153in}}%
\pgfpathlineto{\pgfqpoint{2.948377in}{2.707131in}}%
\pgfpathlineto{\pgfqpoint{2.961710in}{2.697234in}}%
\pgfpathlineto{\pgfqpoint{2.975044in}{2.687460in}}%
\pgfpathlineto{\pgfqpoint{2.988378in}{2.677808in}}%
\pgfpathlineto{\pgfqpoint{2.996474in}{2.688704in}}%
\pgfpathlineto{\pgfqpoint{3.004562in}{2.699710in}}%
\pgfpathlineto{\pgfqpoint{3.012643in}{2.710829in}}%
\pgfpathlineto{\pgfqpoint{3.020717in}{2.722063in}}%
\pgfpathlineto{\pgfqpoint{3.007393in}{2.731830in}}%
\pgfpathlineto{\pgfqpoint{2.994069in}{2.741719in}}%
\pgfpathlineto{\pgfqpoint{2.980745in}{2.751732in}}%
\pgfpathlineto{\pgfqpoint{2.967422in}{2.761869in}}%
\pgfpathlineto{\pgfqpoint{2.959338in}{2.750512in}}%
\pgfpathlineto{\pgfqpoint{2.951248in}{2.739276in}}%
\pgfpathlineto{\pgfqpoint{2.943149in}{2.728157in}}%
\pgfpathlineto{\pgfqpoint{2.935044in}{2.717153in}}%
\pgfpathclose%
\pgfusepath{fill}%
\end{pgfscope}%
\begin{pgfscope}%
\pgfpathrectangle{\pgfqpoint{1.150000in}{0.150000in}}{\pgfqpoint{5.700000in}{5.700000in}}%
\pgfusepath{clip}%
\pgfsetbuttcap%
\pgfsetroundjoin%
\definecolor{currentfill}{rgb}{0.275191,0.194905,0.496005}%
\pgfsetfillcolor{currentfill}%
\pgfsetfillopacity{0.700000}%
\pgfsetlinewidth{0.000000pt}%
\definecolor{currentstroke}{rgb}{0.000000,0.000000,0.000000}%
\pgfsetstrokecolor{currentstroke}%
\pgfsetdash{}{0pt}%
\pgfpathmoveto{\pgfqpoint{4.298938in}{2.709757in}}%
\pgfpathlineto{\pgfqpoint{4.312432in}{2.704757in}}%
\pgfpathlineto{\pgfqpoint{4.325933in}{2.699835in}}%
\pgfpathlineto{\pgfqpoint{4.339439in}{2.694993in}}%
\pgfpathlineto{\pgfqpoint{4.352951in}{2.690228in}}%
\pgfpathlineto{\pgfqpoint{4.360646in}{2.703087in}}%
\pgfpathlineto{\pgfqpoint{4.368338in}{2.716146in}}%
\pgfpathlineto{\pgfqpoint{4.376028in}{2.729410in}}%
\pgfpathlineto{\pgfqpoint{4.383715in}{2.742887in}}%
\pgfpathlineto{\pgfqpoint{4.370212in}{2.747988in}}%
\pgfpathlineto{\pgfqpoint{4.356715in}{2.753167in}}%
\pgfpathlineto{\pgfqpoint{4.343224in}{2.758424in}}%
\pgfpathlineto{\pgfqpoint{4.329739in}{2.763760in}}%
\pgfpathlineto{\pgfqpoint{4.322043in}{2.749940in}}%
\pgfpathlineto{\pgfqpoint{4.314344in}{2.736337in}}%
\pgfpathlineto{\pgfqpoint{4.306642in}{2.722945in}}%
\pgfpathlineto{\pgfqpoint{4.298938in}{2.709757in}}%
\pgfpathclose%
\pgfusepath{fill}%
\end{pgfscope}%
\begin{pgfscope}%
\pgfpathrectangle{\pgfqpoint{1.150000in}{0.150000in}}{\pgfqpoint{5.700000in}{5.700000in}}%
\pgfusepath{clip}%
\pgfsetbuttcap%
\pgfsetroundjoin%
\definecolor{currentfill}{rgb}{0.265145,0.232956,0.516599}%
\pgfsetfillcolor{currentfill}%
\pgfsetfillopacity{0.700000}%
\pgfsetlinewidth{0.000000pt}%
\definecolor{currentstroke}{rgb}{0.000000,0.000000,0.000000}%
\pgfsetstrokecolor{currentstroke}%
\pgfsetdash{}{0pt}%
\pgfpathmoveto{\pgfqpoint{2.742384in}{2.803445in}}%
\pgfpathlineto{\pgfqpoint{2.755736in}{2.791942in}}%
\pgfpathlineto{\pgfqpoint{2.769087in}{2.780576in}}%
\pgfpathlineto{\pgfqpoint{2.782437in}{2.769347in}}%
\pgfpathlineto{\pgfqpoint{2.795785in}{2.758254in}}%
\pgfpathlineto{\pgfqpoint{2.803942in}{2.768995in}}%
\pgfpathlineto{\pgfqpoint{2.812092in}{2.779852in}}%
\pgfpathlineto{\pgfqpoint{2.820233in}{2.790828in}}%
\pgfpathlineto{\pgfqpoint{2.828367in}{2.801925in}}%
\pgfpathlineto{\pgfqpoint{2.815029in}{2.813113in}}%
\pgfpathlineto{\pgfqpoint{2.801691in}{2.824437in}}%
\pgfpathlineto{\pgfqpoint{2.788351in}{2.835898in}}%
\pgfpathlineto{\pgfqpoint{2.775009in}{2.847496in}}%
\pgfpathlineto{\pgfqpoint{2.766865in}{2.836297in}}%
\pgfpathlineto{\pgfqpoint{2.758713in}{2.825224in}}%
\pgfpathlineto{\pgfqpoint{2.750553in}{2.814274in}}%
\pgfpathlineto{\pgfqpoint{2.742384in}{2.803445in}}%
\pgfpathclose%
\pgfusepath{fill}%
\end{pgfscope}%
\begin{pgfscope}%
\pgfpathrectangle{\pgfqpoint{1.150000in}{0.150000in}}{\pgfqpoint{5.700000in}{5.700000in}}%
\pgfusepath{clip}%
\pgfsetbuttcap%
\pgfsetroundjoin%
\definecolor{currentfill}{rgb}{0.281887,0.150881,0.465405}%
\pgfsetfillcolor{currentfill}%
\pgfsetfillopacity{0.700000}%
\pgfsetlinewidth{0.000000pt}%
\definecolor{currentstroke}{rgb}{0.000000,0.000000,0.000000}%
\pgfsetstrokecolor{currentstroke}%
\pgfsetdash{}{0pt}%
\pgfpathmoveto{\pgfqpoint{3.767063in}{2.615065in}}%
\pgfpathlineto{\pgfqpoint{3.780460in}{2.609092in}}%
\pgfpathlineto{\pgfqpoint{3.793861in}{2.603208in}}%
\pgfpathlineto{\pgfqpoint{3.807266in}{2.597413in}}%
\pgfpathlineto{\pgfqpoint{3.820676in}{2.591707in}}%
\pgfpathlineto{\pgfqpoint{3.828519in}{2.603329in}}%
\pgfpathlineto{\pgfqpoint{3.836357in}{2.615081in}}%
\pgfpathlineto{\pgfqpoint{3.844189in}{2.626967in}}%
\pgfpathlineto{\pgfqpoint{3.852017in}{2.638992in}}%
\pgfpathlineto{\pgfqpoint{3.838615in}{2.644934in}}%
\pgfpathlineto{\pgfqpoint{3.825218in}{2.650965in}}%
\pgfpathlineto{\pgfqpoint{3.811826in}{2.657084in}}%
\pgfpathlineto{\pgfqpoint{3.798437in}{2.663294in}}%
\pgfpathlineto{\pgfqpoint{3.790602in}{2.651025in}}%
\pgfpathlineto{\pgfqpoint{3.782761in}{2.638900in}}%
\pgfpathlineto{\pgfqpoint{3.774915in}{2.626915in}}%
\pgfpathlineto{\pgfqpoint{3.767063in}{2.615065in}}%
\pgfpathclose%
\pgfusepath{fill}%
\end{pgfscope}%
\begin{pgfscope}%
\pgfpathrectangle{\pgfqpoint{1.150000in}{0.150000in}}{\pgfqpoint{5.700000in}{5.700000in}}%
\pgfusepath{clip}%
\pgfsetbuttcap%
\pgfsetroundjoin%
\definecolor{currentfill}{rgb}{0.280868,0.160771,0.472899}%
\pgfsetfillcolor{currentfill}%
\pgfsetfillopacity{0.700000}%
\pgfsetlinewidth{0.000000pt}%
\definecolor{currentstroke}{rgb}{0.000000,0.000000,0.000000}%
\pgfsetstrokecolor{currentstroke}%
\pgfsetdash{}{0pt}%
\pgfpathmoveto{\pgfqpoint{3.990590in}{2.642133in}}%
\pgfpathlineto{\pgfqpoint{4.004026in}{2.636713in}}%
\pgfpathlineto{\pgfqpoint{4.017467in}{2.631378in}}%
\pgfpathlineto{\pgfqpoint{4.030912in}{2.626127in}}%
\pgfpathlineto{\pgfqpoint{4.044363in}{2.620960in}}%
\pgfpathlineto{\pgfqpoint{4.052142in}{2.632946in}}%
\pgfpathlineto{\pgfqpoint{4.059915in}{2.645084in}}%
\pgfpathlineto{\pgfqpoint{4.067685in}{2.657380in}}%
\pgfpathlineto{\pgfqpoint{4.075450in}{2.669839in}}%
\pgfpathlineto{\pgfqpoint{4.062007in}{2.675282in}}%
\pgfpathlineto{\pgfqpoint{4.048570in}{2.680809in}}%
\pgfpathlineto{\pgfqpoint{4.035138in}{2.686420in}}%
\pgfpathlineto{\pgfqpoint{4.021711in}{2.692115in}}%
\pgfpathlineto{\pgfqpoint{4.013937in}{2.679373in}}%
\pgfpathlineto{\pgfqpoint{4.006159in}{2.666798in}}%
\pgfpathlineto{\pgfqpoint{3.998377in}{2.654387in}}%
\pgfpathlineto{\pgfqpoint{3.990590in}{2.642133in}}%
\pgfpathclose%
\pgfusepath{fill}%
\end{pgfscope}%
\begin{pgfscope}%
\pgfpathrectangle{\pgfqpoint{1.150000in}{0.150000in}}{\pgfqpoint{5.700000in}{5.700000in}}%
\pgfusepath{clip}%
\pgfsetbuttcap%
\pgfsetroundjoin%
\definecolor{currentfill}{rgb}{0.281887,0.150881,0.465405}%
\pgfsetfillcolor{currentfill}%
\pgfsetfillopacity{0.700000}%
\pgfsetlinewidth{0.000000pt}%
\definecolor{currentstroke}{rgb}{0.000000,0.000000,0.000000}%
\pgfsetstrokecolor{currentstroke}%
\pgfsetdash{}{0pt}%
\pgfpathmoveto{\pgfqpoint{3.266123in}{2.625597in}}%
\pgfpathlineto{\pgfqpoint{3.279460in}{2.617630in}}%
\pgfpathlineto{\pgfqpoint{3.292800in}{2.609770in}}%
\pgfpathlineto{\pgfqpoint{3.306142in}{2.602016in}}%
\pgfpathlineto{\pgfqpoint{3.319486in}{2.594367in}}%
\pgfpathlineto{\pgfqpoint{3.327481in}{2.605473in}}%
\pgfpathlineto{\pgfqpoint{3.335470in}{2.616687in}}%
\pgfpathlineto{\pgfqpoint{3.343452in}{2.628009in}}%
\pgfpathlineto{\pgfqpoint{3.351428in}{2.639444in}}%
\pgfpathlineto{\pgfqpoint{3.338092in}{2.647249in}}%
\pgfpathlineto{\pgfqpoint{3.324758in}{2.655159in}}%
\pgfpathlineto{\pgfqpoint{3.311427in}{2.663175in}}%
\pgfpathlineto{\pgfqpoint{3.298098in}{2.671297in}}%
\pgfpathlineto{\pgfqpoint{3.290114in}{2.659699in}}%
\pgfpathlineto{\pgfqpoint{3.282123in}{2.648218in}}%
\pgfpathlineto{\pgfqpoint{3.274126in}{2.636852in}}%
\pgfpathlineto{\pgfqpoint{3.266123in}{2.625597in}}%
\pgfpathclose%
\pgfusepath{fill}%
\end{pgfscope}%
\begin{pgfscope}%
\pgfpathrectangle{\pgfqpoint{1.150000in}{0.150000in}}{\pgfqpoint{5.700000in}{5.700000in}}%
\pgfusepath{clip}%
\pgfsetbuttcap%
\pgfsetroundjoin%
\definecolor{currentfill}{rgb}{0.194100,0.399323,0.555565}%
\pgfsetfillcolor{currentfill}%
\pgfsetfillopacity{0.700000}%
\pgfsetlinewidth{0.000000pt}%
\definecolor{currentstroke}{rgb}{0.000000,0.000000,0.000000}%
\pgfsetstrokecolor{currentstroke}%
\pgfsetdash{}{0pt}%
\pgfpathmoveto{\pgfqpoint{5.147215in}{3.154199in}}%
\pgfpathlineto{\pgfqpoint{5.160847in}{3.147888in}}%
\pgfpathlineto{\pgfqpoint{5.174486in}{3.141646in}}%
\pgfpathlineto{\pgfqpoint{5.188130in}{3.135474in}}%
\pgfpathlineto{\pgfqpoint{5.201781in}{3.129372in}}%
\pgfpathlineto{\pgfqpoint{5.209429in}{3.150625in}}%
\pgfpathlineto{\pgfqpoint{5.217086in}{3.172348in}}%
\pgfpathlineto{\pgfqpoint{5.224752in}{3.194550in}}%
\pgfpathlineto{\pgfqpoint{5.232427in}{3.217241in}}%
\pgfpathlineto{\pgfqpoint{5.218787in}{3.223882in}}%
\pgfpathlineto{\pgfqpoint{5.205154in}{3.230593in}}%
\pgfpathlineto{\pgfqpoint{5.191526in}{3.237374in}}%
\pgfpathlineto{\pgfqpoint{5.177905in}{3.244224in}}%
\pgfpathlineto{\pgfqpoint{5.170219in}{3.220986in}}%
\pgfpathlineto{\pgfqpoint{5.162543in}{3.198242in}}%
\pgfpathlineto{\pgfqpoint{5.154875in}{3.175983in}}%
\pgfpathlineto{\pgfqpoint{5.147215in}{3.154199in}}%
\pgfpathclose%
\pgfusepath{fill}%
\end{pgfscope}%
\begin{pgfscope}%
\pgfpathrectangle{\pgfqpoint{1.150000in}{0.150000in}}{\pgfqpoint{5.700000in}{5.700000in}}%
\pgfusepath{clip}%
\pgfsetbuttcap%
\pgfsetroundjoin%
\definecolor{currentfill}{rgb}{0.248629,0.278775,0.534556}%
\pgfsetfillcolor{currentfill}%
\pgfsetfillopacity{0.700000}%
\pgfsetlinewidth{0.000000pt}%
\definecolor{currentstroke}{rgb}{0.000000,0.000000,0.000000}%
\pgfsetstrokecolor{currentstroke}%
\pgfsetdash{}{0pt}%
\pgfpathmoveto{\pgfqpoint{4.776982in}{2.875863in}}%
\pgfpathlineto{\pgfqpoint{4.790571in}{2.870798in}}%
\pgfpathlineto{\pgfqpoint{4.804166in}{2.865807in}}%
\pgfpathlineto{\pgfqpoint{4.817767in}{2.860887in}}%
\pgfpathlineto{\pgfqpoint{4.831375in}{2.856040in}}%
\pgfpathlineto{\pgfqpoint{4.838982in}{2.871588in}}%
\pgfpathlineto{\pgfqpoint{4.846591in}{2.887445in}}%
\pgfpathlineto{\pgfqpoint{4.854202in}{2.903619in}}%
\pgfpathlineto{\pgfqpoint{4.861814in}{2.920120in}}%
\pgfpathlineto{\pgfqpoint{4.848218in}{2.925403in}}%
\pgfpathlineto{\pgfqpoint{4.834628in}{2.930759in}}%
\pgfpathlineto{\pgfqpoint{4.821044in}{2.936187in}}%
\pgfpathlineto{\pgfqpoint{4.807466in}{2.941688in}}%
\pgfpathlineto{\pgfqpoint{4.799843in}{2.924744in}}%
\pgfpathlineto{\pgfqpoint{4.792221in}{2.908130in}}%
\pgfpathlineto{\pgfqpoint{4.784601in}{2.891839in}}%
\pgfpathlineto{\pgfqpoint{4.776982in}{2.875863in}}%
\pgfpathclose%
\pgfusepath{fill}%
\end{pgfscope}%
\begin{pgfscope}%
\pgfpathrectangle{\pgfqpoint{1.150000in}{0.150000in}}{\pgfqpoint{5.700000in}{5.700000in}}%
\pgfusepath{clip}%
\pgfsetbuttcap%
\pgfsetroundjoin%
\definecolor{currentfill}{rgb}{0.282290,0.145912,0.461510}%
\pgfsetfillcolor{currentfill}%
\pgfsetfillopacity{0.700000}%
\pgfsetlinewidth{0.000000pt}%
\definecolor{currentstroke}{rgb}{0.000000,0.000000,0.000000}%
\pgfsetstrokecolor{currentstroke}%
\pgfsetdash{}{0pt}%
\pgfpathmoveto{\pgfqpoint{3.404797in}{2.609260in}}%
\pgfpathlineto{\pgfqpoint{3.418146in}{2.601969in}}%
\pgfpathlineto{\pgfqpoint{3.431497in}{2.594780in}}%
\pgfpathlineto{\pgfqpoint{3.444852in}{2.587690in}}%
\pgfpathlineto{\pgfqpoint{3.458210in}{2.580701in}}%
\pgfpathlineto{\pgfqpoint{3.466163in}{2.591915in}}%
\pgfpathlineto{\pgfqpoint{3.474110in}{2.603238in}}%
\pgfpathlineto{\pgfqpoint{3.482050in}{2.614673in}}%
\pgfpathlineto{\pgfqpoint{3.489985in}{2.626224in}}%
\pgfpathlineto{\pgfqpoint{3.476635in}{2.633390in}}%
\pgfpathlineto{\pgfqpoint{3.463289in}{2.640655in}}%
\pgfpathlineto{\pgfqpoint{3.449945in}{2.648021in}}%
\pgfpathlineto{\pgfqpoint{3.436604in}{2.655487in}}%
\pgfpathlineto{\pgfqpoint{3.428662in}{2.643752in}}%
\pgfpathlineto{\pgfqpoint{3.420713in}{2.632139in}}%
\pgfpathlineto{\pgfqpoint{3.412758in}{2.620642in}}%
\pgfpathlineto{\pgfqpoint{3.404797in}{2.609260in}}%
\pgfpathclose%
\pgfusepath{fill}%
\end{pgfscope}%
\begin{pgfscope}%
\pgfpathrectangle{\pgfqpoint{1.150000in}{0.150000in}}{\pgfqpoint{5.700000in}{5.700000in}}%
\pgfusepath{clip}%
\pgfsetbuttcap%
\pgfsetroundjoin%
\definecolor{currentfill}{rgb}{0.241237,0.296485,0.539709}%
\pgfsetfillcolor{currentfill}%
\pgfsetfillopacity{0.700000}%
\pgfsetlinewidth{0.000000pt}%
\definecolor{currentstroke}{rgb}{0.000000,0.000000,0.000000}%
\pgfsetstrokecolor{currentstroke}%
\pgfsetdash{}{0pt}%
\pgfpathmoveto{\pgfqpoint{4.861814in}{2.920120in}}%
\pgfpathlineto{\pgfqpoint{4.875417in}{2.914908in}}%
\pgfpathlineto{\pgfqpoint{4.889026in}{2.909769in}}%
\pgfpathlineto{\pgfqpoint{4.902641in}{2.904700in}}%
\pgfpathlineto{\pgfqpoint{4.916263in}{2.899704in}}%
\pgfpathlineto{\pgfqpoint{4.923866in}{2.916089in}}%
\pgfpathlineto{\pgfqpoint{4.931471in}{2.932811in}}%
\pgfpathlineto{\pgfqpoint{4.939079in}{2.949879in}}%
\pgfpathlineto{\pgfqpoint{4.946690in}{2.967301in}}%
\pgfpathlineto{\pgfqpoint{4.933079in}{2.972754in}}%
\pgfpathlineto{\pgfqpoint{4.919475in}{2.978279in}}%
\pgfpathlineto{\pgfqpoint{4.905878in}{2.983875in}}%
\pgfpathlineto{\pgfqpoint{4.892286in}{2.989543in}}%
\pgfpathlineto{\pgfqpoint{4.884664in}{2.971657in}}%
\pgfpathlineto{\pgfqpoint{4.877045in}{2.954130in}}%
\pgfpathlineto{\pgfqpoint{4.869428in}{2.936954in}}%
\pgfpathlineto{\pgfqpoint{4.861814in}{2.920120in}}%
\pgfpathclose%
\pgfusepath{fill}%
\end{pgfscope}%
\begin{pgfscope}%
\pgfpathrectangle{\pgfqpoint{1.150000in}{0.150000in}}{\pgfqpoint{5.700000in}{5.700000in}}%
\pgfusepath{clip}%
\pgfsetbuttcap%
\pgfsetroundjoin%
\definecolor{currentfill}{rgb}{0.280868,0.160771,0.472899}%
\pgfsetfillcolor{currentfill}%
\pgfsetfillopacity{0.700000}%
\pgfsetlinewidth{0.000000pt}%
\definecolor{currentstroke}{rgb}{0.000000,0.000000,0.000000}%
\pgfsetstrokecolor{currentstroke}%
\pgfsetdash{}{0pt}%
\pgfpathmoveto{\pgfqpoint{3.127338in}{2.648205in}}%
\pgfpathlineto{\pgfqpoint{3.140670in}{2.639494in}}%
\pgfpathlineto{\pgfqpoint{3.154004in}{2.630896in}}%
\pgfpathlineto{\pgfqpoint{3.167339in}{2.622410in}}%
\pgfpathlineto{\pgfqpoint{3.180676in}{2.614035in}}%
\pgfpathlineto{\pgfqpoint{3.188715in}{2.625001in}}%
\pgfpathlineto{\pgfqpoint{3.196748in}{2.636073in}}%
\pgfpathlineto{\pgfqpoint{3.204773in}{2.647253in}}%
\pgfpathlineto{\pgfqpoint{3.212792in}{2.658543in}}%
\pgfpathlineto{\pgfqpoint{3.199465in}{2.667053in}}%
\pgfpathlineto{\pgfqpoint{3.186138in}{2.675675in}}%
\pgfpathlineto{\pgfqpoint{3.172813in}{2.684409in}}%
\pgfpathlineto{\pgfqpoint{3.159490in}{2.693255in}}%
\pgfpathlineto{\pgfqpoint{3.151462in}{2.681822in}}%
\pgfpathlineto{\pgfqpoint{3.143427in}{2.670505in}}%
\pgfpathlineto{\pgfqpoint{3.135386in}{2.659300in}}%
\pgfpathlineto{\pgfqpoint{3.127338in}{2.648205in}}%
\pgfpathclose%
\pgfusepath{fill}%
\end{pgfscope}%
\begin{pgfscope}%
\pgfpathrectangle{\pgfqpoint{1.150000in}{0.150000in}}{\pgfqpoint{5.700000in}{5.700000in}}%
\pgfusepath{clip}%
\pgfsetbuttcap%
\pgfsetroundjoin%
\definecolor{currentfill}{rgb}{0.255645,0.260703,0.528312}%
\pgfsetfillcolor{currentfill}%
\pgfsetfillopacity{0.700000}%
\pgfsetlinewidth{0.000000pt}%
\definecolor{currentstroke}{rgb}{0.000000,0.000000,0.000000}%
\pgfsetstrokecolor{currentstroke}%
\pgfsetdash{}{0pt}%
\pgfpathmoveto{\pgfqpoint{4.692177in}{2.834270in}}%
\pgfpathlineto{\pgfqpoint{4.705752in}{2.829329in}}%
\pgfpathlineto{\pgfqpoint{4.719333in}{2.824462in}}%
\pgfpathlineto{\pgfqpoint{4.732920in}{2.819668in}}%
\pgfpathlineto{\pgfqpoint{4.746513in}{2.814947in}}%
\pgfpathlineto{\pgfqpoint{4.754130in}{2.829743in}}%
\pgfpathlineto{\pgfqpoint{4.761747in}{2.844822in}}%
\pgfpathlineto{\pgfqpoint{4.769364in}{2.860193in}}%
\pgfpathlineto{\pgfqpoint{4.776982in}{2.875863in}}%
\pgfpathlineto{\pgfqpoint{4.763400in}{2.881000in}}%
\pgfpathlineto{\pgfqpoint{4.749824in}{2.886210in}}%
\pgfpathlineto{\pgfqpoint{4.736254in}{2.891494in}}%
\pgfpathlineto{\pgfqpoint{4.722690in}{2.896851in}}%
\pgfpathlineto{\pgfqpoint{4.715061in}{2.880758in}}%
\pgfpathlineto{\pgfqpoint{4.707433in}{2.864968in}}%
\pgfpathlineto{\pgfqpoint{4.699805in}{2.849475in}}%
\pgfpathlineto{\pgfqpoint{4.692177in}{2.834270in}}%
\pgfpathclose%
\pgfusepath{fill}%
\end{pgfscope}%
\begin{pgfscope}%
\pgfpathrectangle{\pgfqpoint{1.150000in}{0.150000in}}{\pgfqpoint{5.700000in}{5.700000in}}%
\pgfusepath{clip}%
\pgfsetbuttcap%
\pgfsetroundjoin%
\definecolor{currentfill}{rgb}{0.231674,0.318106,0.544834}%
\pgfsetfillcolor{currentfill}%
\pgfsetfillopacity{0.700000}%
\pgfsetlinewidth{0.000000pt}%
\definecolor{currentstroke}{rgb}{0.000000,0.000000,0.000000}%
\pgfsetstrokecolor{currentstroke}%
\pgfsetdash{}{0pt}%
\pgfpathmoveto{\pgfqpoint{4.946690in}{2.967301in}}%
\pgfpathlineto{\pgfqpoint{4.960306in}{2.961919in}}%
\pgfpathlineto{\pgfqpoint{4.973929in}{2.956608in}}%
\pgfpathlineto{\pgfqpoint{4.987559in}{2.951368in}}%
\pgfpathlineto{\pgfqpoint{5.001195in}{2.946198in}}%
\pgfpathlineto{\pgfqpoint{5.008797in}{2.963513in}}%
\pgfpathlineto{\pgfqpoint{5.016403in}{2.981195in}}%
\pgfpathlineto{\pgfqpoint{5.024013in}{2.999251in}}%
\pgfpathlineto{\pgfqpoint{5.031628in}{3.017691in}}%
\pgfpathlineto{\pgfqpoint{5.018004in}{3.023338in}}%
\pgfpathlineto{\pgfqpoint{5.004387in}{3.029055in}}%
\pgfpathlineto{\pgfqpoint{4.990775in}{3.034843in}}%
\pgfpathlineto{\pgfqpoint{4.977170in}{3.040701in}}%
\pgfpathlineto{\pgfqpoint{4.969544in}{3.021777in}}%
\pgfpathlineto{\pgfqpoint{4.961922in}{3.003241in}}%
\pgfpathlineto{\pgfqpoint{4.954304in}{2.985085in}}%
\pgfpathlineto{\pgfqpoint{4.946690in}{2.967301in}}%
\pgfpathclose%
\pgfusepath{fill}%
\end{pgfscope}%
\begin{pgfscope}%
\pgfpathrectangle{\pgfqpoint{1.150000in}{0.150000in}}{\pgfqpoint{5.700000in}{5.700000in}}%
\pgfusepath{clip}%
\pgfsetbuttcap%
\pgfsetroundjoin%
\definecolor{currentfill}{rgb}{0.277134,0.185228,0.489898}%
\pgfsetfillcolor{currentfill}%
\pgfsetfillopacity{0.700000}%
\pgfsetlinewidth{0.000000pt}%
\definecolor{currentstroke}{rgb}{0.000000,0.000000,0.000000}%
\pgfsetstrokecolor{currentstroke}%
\pgfsetdash{}{0pt}%
\pgfpathmoveto{\pgfqpoint{4.214126in}{2.678464in}}%
\pgfpathlineto{\pgfqpoint{4.227607in}{2.673461in}}%
\pgfpathlineto{\pgfqpoint{4.241095in}{2.668538in}}%
\pgfpathlineto{\pgfqpoint{4.254587in}{2.663696in}}%
\pgfpathlineto{\pgfqpoint{4.268086in}{2.658932in}}%
\pgfpathlineto{\pgfqpoint{4.275804in}{2.671361in}}%
\pgfpathlineto{\pgfqpoint{4.283518in}{2.683971in}}%
\pgfpathlineto{\pgfqpoint{4.291230in}{2.696768in}}%
\pgfpathlineto{\pgfqpoint{4.298938in}{2.709757in}}%
\pgfpathlineto{\pgfqpoint{4.285449in}{2.714837in}}%
\pgfpathlineto{\pgfqpoint{4.271965in}{2.719996in}}%
\pgfpathlineto{\pgfqpoint{4.258487in}{2.725234in}}%
\pgfpathlineto{\pgfqpoint{4.245015in}{2.730553in}}%
\pgfpathlineto{\pgfqpoint{4.237298in}{2.717240in}}%
\pgfpathlineto{\pgfqpoint{4.229577in}{2.704125in}}%
\pgfpathlineto{\pgfqpoint{4.221854in}{2.691202in}}%
\pgfpathlineto{\pgfqpoint{4.214126in}{2.678464in}}%
\pgfpathclose%
\pgfusepath{fill}%
\end{pgfscope}%
\begin{pgfscope}%
\pgfpathrectangle{\pgfqpoint{1.150000in}{0.150000in}}{\pgfqpoint{5.700000in}{5.700000in}}%
\pgfusepath{clip}%
\pgfsetbuttcap%
\pgfsetroundjoin%
\definecolor{currentfill}{rgb}{0.282623,0.140926,0.457517}%
\pgfsetfillcolor{currentfill}%
\pgfsetfillopacity{0.700000}%
\pgfsetlinewidth{0.000000pt}%
\definecolor{currentstroke}{rgb}{0.000000,0.000000,0.000000}%
\pgfsetstrokecolor{currentstroke}%
\pgfsetdash{}{0pt}%
\pgfpathmoveto{\pgfqpoint{3.543415in}{2.598545in}}%
\pgfpathlineto{\pgfqpoint{3.556781in}{2.591868in}}%
\pgfpathlineto{\pgfqpoint{3.570151in}{2.585287in}}%
\pgfpathlineto{\pgfqpoint{3.583524in}{2.578801in}}%
\pgfpathlineto{\pgfqpoint{3.596901in}{2.572410in}}%
\pgfpathlineto{\pgfqpoint{3.604813in}{2.583704in}}%
\pgfpathlineto{\pgfqpoint{3.612720in}{2.595111in}}%
\pgfpathlineto{\pgfqpoint{3.620620in}{2.606634in}}%
\pgfpathlineto{\pgfqpoint{3.628515in}{2.618278in}}%
\pgfpathlineto{\pgfqpoint{3.615147in}{2.624865in}}%
\pgfpathlineto{\pgfqpoint{3.601782in}{2.631547in}}%
\pgfpathlineto{\pgfqpoint{3.588420in}{2.638324in}}%
\pgfpathlineto{\pgfqpoint{3.575062in}{2.645197in}}%
\pgfpathlineto{\pgfqpoint{3.567159in}{2.633349in}}%
\pgfpathlineto{\pgfqpoint{3.559251in}{2.621627in}}%
\pgfpathlineto{\pgfqpoint{3.551336in}{2.610027in}}%
\pgfpathlineto{\pgfqpoint{3.543415in}{2.598545in}}%
\pgfpathclose%
\pgfusepath{fill}%
\end{pgfscope}%
\begin{pgfscope}%
\pgfpathrectangle{\pgfqpoint{1.150000in}{0.150000in}}{\pgfqpoint{5.700000in}{5.700000in}}%
\pgfusepath{clip}%
\pgfsetbuttcap%
\pgfsetroundjoin%
\definecolor{currentfill}{rgb}{0.262138,0.242286,0.520837}%
\pgfsetfillcolor{currentfill}%
\pgfsetfillopacity{0.700000}%
\pgfsetlinewidth{0.000000pt}%
\definecolor{currentstroke}{rgb}{0.000000,0.000000,0.000000}%
\pgfsetstrokecolor{currentstroke}%
\pgfsetdash{}{0pt}%
\pgfpathmoveto{\pgfqpoint{4.607385in}{2.795108in}}%
\pgfpathlineto{\pgfqpoint{4.620945in}{2.790267in}}%
\pgfpathlineto{\pgfqpoint{4.634511in}{2.785500in}}%
\pgfpathlineto{\pgfqpoint{4.648084in}{2.780808in}}%
\pgfpathlineto{\pgfqpoint{4.661663in}{2.776189in}}%
\pgfpathlineto{\pgfqpoint{4.669293in}{2.790314in}}%
\pgfpathlineto{\pgfqpoint{4.676921in}{2.804697in}}%
\pgfpathlineto{\pgfqpoint{4.684550in}{2.819347in}}%
\pgfpathlineto{\pgfqpoint{4.692177in}{2.834270in}}%
\pgfpathlineto{\pgfqpoint{4.678609in}{2.839285in}}%
\pgfpathlineto{\pgfqpoint{4.665047in}{2.844374in}}%
\pgfpathlineto{\pgfqpoint{4.651491in}{2.849537in}}%
\pgfpathlineto{\pgfqpoint{4.637942in}{2.854774in}}%
\pgfpathlineto{\pgfqpoint{4.630303in}{2.839447in}}%
\pgfpathlineto{\pgfqpoint{4.622664in}{2.824399in}}%
\pgfpathlineto{\pgfqpoint{4.615025in}{2.809621in}}%
\pgfpathlineto{\pgfqpoint{4.607385in}{2.795108in}}%
\pgfpathclose%
\pgfusepath{fill}%
\end{pgfscope}%
\begin{pgfscope}%
\pgfpathrectangle{\pgfqpoint{1.150000in}{0.150000in}}{\pgfqpoint{5.700000in}{5.700000in}}%
\pgfusepath{clip}%
\pgfsetbuttcap%
\pgfsetroundjoin%
\definecolor{currentfill}{rgb}{0.169646,0.456262,0.558030}%
\pgfsetfillcolor{currentfill}%
\pgfsetfillopacity{0.700000}%
\pgfsetlinewidth{0.000000pt}%
\definecolor{currentstroke}{rgb}{0.000000,0.000000,0.000000}%
\pgfsetstrokecolor{currentstroke}%
\pgfsetdash{}{0pt}%
\pgfpathmoveto{\pgfqpoint{5.263228in}{3.313110in}}%
\pgfpathlineto{\pgfqpoint{5.276863in}{3.305978in}}%
\pgfpathlineto{\pgfqpoint{5.290503in}{3.298914in}}%
\pgfpathlineto{\pgfqpoint{5.304150in}{3.291919in}}%
\pgfpathlineto{\pgfqpoint{5.317802in}{3.284993in}}%
\pgfpathlineto{\pgfqpoint{5.325520in}{3.309717in}}%
\pgfpathlineto{\pgfqpoint{5.333250in}{3.334990in}}%
\pgfpathlineto{\pgfqpoint{5.340992in}{3.360822in}}%
\pgfpathlineto{\pgfqpoint{5.327348in}{3.368182in}}%
\pgfpathlineto{\pgfqpoint{5.313709in}{3.375611in}}%
\pgfpathlineto{\pgfqpoint{5.300076in}{3.383110in}}%
\pgfpathlineto{\pgfqpoint{5.286448in}{3.390677in}}%
\pgfpathlineto{\pgfqpoint{5.278696in}{3.364260in}}%
\pgfpathlineto{\pgfqpoint{5.270956in}{3.338408in}}%
\pgfpathlineto{\pgfqpoint{5.263228in}{3.313110in}}%
\pgfpathclose%
\pgfusepath{fill}%
\end{pgfscope}%
\begin{pgfscope}%
\pgfpathrectangle{\pgfqpoint{1.150000in}{0.150000in}}{\pgfqpoint{5.700000in}{5.700000in}}%
\pgfusepath{clip}%
\pgfsetbuttcap%
\pgfsetroundjoin%
\definecolor{currentfill}{rgb}{0.220057,0.343307,0.549413}%
\pgfsetfillcolor{currentfill}%
\pgfsetfillopacity{0.700000}%
\pgfsetlinewidth{0.000000pt}%
\definecolor{currentstroke}{rgb}{0.000000,0.000000,0.000000}%
\pgfsetstrokecolor{currentstroke}%
\pgfsetdash{}{0pt}%
\pgfpathmoveto{\pgfqpoint{5.031628in}{3.017691in}}%
\pgfpathlineto{\pgfqpoint{5.045258in}{3.012116in}}%
\pgfpathlineto{\pgfqpoint{5.058895in}{3.006610in}}%
\pgfpathlineto{\pgfqpoint{5.072538in}{3.001174in}}%
\pgfpathlineto{\pgfqpoint{5.086188in}{2.995809in}}%
\pgfpathlineto{\pgfqpoint{5.093796in}{3.014153in}}%
\pgfpathlineto{\pgfqpoint{5.101408in}{3.032893in}}%
\pgfpathlineto{\pgfqpoint{5.109026in}{3.052040in}}%
\pgfpathlineto{\pgfqpoint{5.116651in}{3.071602in}}%
\pgfpathlineto{\pgfqpoint{5.103013in}{3.077465in}}%
\pgfpathlineto{\pgfqpoint{5.089382in}{3.083398in}}%
\pgfpathlineto{\pgfqpoint{5.075757in}{3.089401in}}%
\pgfpathlineto{\pgfqpoint{5.062138in}{3.095474in}}%
\pgfpathlineto{\pgfqpoint{5.054502in}{3.075407in}}%
\pgfpathlineto{\pgfqpoint{5.046872in}{3.055760in}}%
\pgfpathlineto{\pgfqpoint{5.039248in}{3.036525in}}%
\pgfpathlineto{\pgfqpoint{5.031628in}{3.017691in}}%
\pgfpathclose%
\pgfusepath{fill}%
\end{pgfscope}%
\begin{pgfscope}%
\pgfpathrectangle{\pgfqpoint{1.150000in}{0.150000in}}{\pgfqpoint{5.700000in}{5.700000in}}%
\pgfusepath{clip}%
\pgfsetbuttcap%
\pgfsetroundjoin%
\definecolor{currentfill}{rgb}{0.270595,0.214069,0.507052}%
\pgfsetfillcolor{currentfill}%
\pgfsetfillopacity{0.700000}%
\pgfsetlinewidth{0.000000pt}%
\definecolor{currentstroke}{rgb}{0.000000,0.000000,0.000000}%
\pgfsetstrokecolor{currentstroke}%
\pgfsetdash{}{0pt}%
\pgfpathmoveto{\pgfqpoint{2.795785in}{2.758254in}}%
\pgfpathlineto{\pgfqpoint{2.809132in}{2.747295in}}%
\pgfpathlineto{\pgfqpoint{2.822479in}{2.736469in}}%
\pgfpathlineto{\pgfqpoint{2.835825in}{2.725774in}}%
\pgfpathlineto{\pgfqpoint{2.849170in}{2.715211in}}%
\pgfpathlineto{\pgfqpoint{2.857316in}{2.725864in}}%
\pgfpathlineto{\pgfqpoint{2.865455in}{2.736629in}}%
\pgfpathlineto{\pgfqpoint{2.873586in}{2.747508in}}%
\pgfpathlineto{\pgfqpoint{2.881709in}{2.758502in}}%
\pgfpathlineto{\pgfqpoint{2.868375in}{2.769160in}}%
\pgfpathlineto{\pgfqpoint{2.855040in}{2.779949in}}%
\pgfpathlineto{\pgfqpoint{2.841704in}{2.790871in}}%
\pgfpathlineto{\pgfqpoint{2.828367in}{2.801925in}}%
\pgfpathlineto{\pgfqpoint{2.820233in}{2.790828in}}%
\pgfpathlineto{\pgfqpoint{2.812092in}{2.779852in}}%
\pgfpathlineto{\pgfqpoint{2.803942in}{2.768995in}}%
\pgfpathlineto{\pgfqpoint{2.795785in}{2.758254in}}%
\pgfpathclose%
\pgfusepath{fill}%
\end{pgfscope}%
\begin{pgfscope}%
\pgfpathrectangle{\pgfqpoint{1.150000in}{0.150000in}}{\pgfqpoint{5.700000in}{5.700000in}}%
\pgfusepath{clip}%
\pgfsetbuttcap%
\pgfsetroundjoin%
\definecolor{currentfill}{rgb}{0.278012,0.180367,0.486697}%
\pgfsetfillcolor{currentfill}%
\pgfsetfillopacity{0.700000}%
\pgfsetlinewidth{0.000000pt}%
\definecolor{currentstroke}{rgb}{0.000000,0.000000,0.000000}%
\pgfsetstrokecolor{currentstroke}%
\pgfsetdash{}{0pt}%
\pgfpathmoveto{\pgfqpoint{2.988378in}{2.677808in}}%
\pgfpathlineto{\pgfqpoint{3.001712in}{2.668277in}}%
\pgfpathlineto{\pgfqpoint{3.015047in}{2.658866in}}%
\pgfpathlineto{\pgfqpoint{3.028383in}{2.649575in}}%
\pgfpathlineto{\pgfqpoint{3.041719in}{2.640402in}}%
\pgfpathlineto{\pgfqpoint{3.049805in}{2.651190in}}%
\pgfpathlineto{\pgfqpoint{3.057884in}{2.662084in}}%
\pgfpathlineto{\pgfqpoint{3.065956in}{2.673084in}}%
\pgfpathlineto{\pgfqpoint{3.074020in}{2.684195in}}%
\pgfpathlineto{\pgfqpoint{3.060693in}{2.693483in}}%
\pgfpathlineto{\pgfqpoint{3.047367in}{2.702890in}}%
\pgfpathlineto{\pgfqpoint{3.034042in}{2.712416in}}%
\pgfpathlineto{\pgfqpoint{3.020717in}{2.722063in}}%
\pgfpathlineto{\pgfqpoint{3.012643in}{2.710829in}}%
\pgfpathlineto{\pgfqpoint{3.004562in}{2.699710in}}%
\pgfpathlineto{\pgfqpoint{2.996474in}{2.688704in}}%
\pgfpathlineto{\pgfqpoint{2.988378in}{2.677808in}}%
\pgfpathclose%
\pgfusepath{fill}%
\end{pgfscope}%
\begin{pgfscope}%
\pgfpathrectangle{\pgfqpoint{1.150000in}{0.150000in}}{\pgfqpoint{5.700000in}{5.700000in}}%
\pgfusepath{clip}%
\pgfsetbuttcap%
\pgfsetroundjoin%
\definecolor{currentfill}{rgb}{0.267968,0.223549,0.512008}%
\pgfsetfillcolor{currentfill}%
\pgfsetfillopacity{0.700000}%
\pgfsetlinewidth{0.000000pt}%
\definecolor{currentstroke}{rgb}{0.000000,0.000000,0.000000}%
\pgfsetstrokecolor{currentstroke}%
\pgfsetdash{}{0pt}%
\pgfpathmoveto{\pgfqpoint{4.522590in}{2.758167in}}%
\pgfpathlineto{\pgfqpoint{4.536136in}{2.753401in}}%
\pgfpathlineto{\pgfqpoint{4.549689in}{2.748711in}}%
\pgfpathlineto{\pgfqpoint{4.563247in}{2.744096in}}%
\pgfpathlineto{\pgfqpoint{4.576812in}{2.739557in}}%
\pgfpathlineto{\pgfqpoint{4.584457in}{2.753083in}}%
\pgfpathlineto{\pgfqpoint{4.592101in}{2.766846in}}%
\pgfpathlineto{\pgfqpoint{4.599743in}{2.780852in}}%
\pgfpathlineto{\pgfqpoint{4.607385in}{2.795108in}}%
\pgfpathlineto{\pgfqpoint{4.593831in}{2.800024in}}%
\pgfpathlineto{\pgfqpoint{4.580283in}{2.805015in}}%
\pgfpathlineto{\pgfqpoint{4.566741in}{2.810081in}}%
\pgfpathlineto{\pgfqpoint{4.553205in}{2.815223in}}%
\pgfpathlineto{\pgfqpoint{4.545553in}{2.800583in}}%
\pgfpathlineto{\pgfqpoint{4.537901in}{2.786199in}}%
\pgfpathlineto{\pgfqpoint{4.530246in}{2.772062in}}%
\pgfpathlineto{\pgfqpoint{4.522590in}{2.758167in}}%
\pgfpathclose%
\pgfusepath{fill}%
\end{pgfscope}%
\begin{pgfscope}%
\pgfpathrectangle{\pgfqpoint{1.150000in}{0.150000in}}{\pgfqpoint{5.700000in}{5.700000in}}%
\pgfusepath{clip}%
\pgfsetbuttcap%
\pgfsetroundjoin%
\definecolor{currentfill}{rgb}{0.281887,0.150881,0.465405}%
\pgfsetfillcolor{currentfill}%
\pgfsetfillopacity{0.700000}%
\pgfsetlinewidth{0.000000pt}%
\definecolor{currentstroke}{rgb}{0.000000,0.000000,0.000000}%
\pgfsetstrokecolor{currentstroke}%
\pgfsetdash{}{0pt}%
\pgfpathmoveto{\pgfqpoint{3.905668in}{2.616102in}}%
\pgfpathlineto{\pgfqpoint{3.919093in}{2.610597in}}%
\pgfpathlineto{\pgfqpoint{3.932522in}{2.605178in}}%
\pgfpathlineto{\pgfqpoint{3.945957in}{2.599845in}}%
\pgfpathlineto{\pgfqpoint{3.959396in}{2.594597in}}%
\pgfpathlineto{\pgfqpoint{3.967202in}{2.606268in}}%
\pgfpathlineto{\pgfqpoint{3.975003in}{2.618079in}}%
\pgfpathlineto{\pgfqpoint{3.982799in}{2.630032in}}%
\pgfpathlineto{\pgfqpoint{3.990590in}{2.642133in}}%
\pgfpathlineto{\pgfqpoint{3.977160in}{2.647637in}}%
\pgfpathlineto{\pgfqpoint{3.963734in}{2.653226in}}%
\pgfpathlineto{\pgfqpoint{3.950313in}{2.658902in}}%
\pgfpathlineto{\pgfqpoint{3.936897in}{2.664663in}}%
\pgfpathlineto{\pgfqpoint{3.929097in}{2.652298in}}%
\pgfpathlineto{\pgfqpoint{3.921292in}{2.640086in}}%
\pgfpathlineto{\pgfqpoint{3.913483in}{2.628023in}}%
\pgfpathlineto{\pgfqpoint{3.905668in}{2.616102in}}%
\pgfpathclose%
\pgfusepath{fill}%
\end{pgfscope}%
\begin{pgfscope}%
\pgfpathrectangle{\pgfqpoint{1.150000in}{0.150000in}}{\pgfqpoint{5.700000in}{5.700000in}}%
\pgfusepath{clip}%
\pgfsetbuttcap%
\pgfsetroundjoin%
\definecolor{currentfill}{rgb}{0.182256,0.426184,0.557120}%
\pgfsetfillcolor{currentfill}%
\pgfsetfillopacity{0.700000}%
\pgfsetlinewidth{0.000000pt}%
\definecolor{currentstroke}{rgb}{0.000000,0.000000,0.000000}%
\pgfsetstrokecolor{currentstroke}%
\pgfsetdash{}{0pt}%
\pgfpathmoveto{\pgfqpoint{5.232427in}{3.217241in}}%
\pgfpathlineto{\pgfqpoint{5.246072in}{3.210668in}}%
\pgfpathlineto{\pgfqpoint{5.259724in}{3.204165in}}%
\pgfpathlineto{\pgfqpoint{5.273382in}{3.197731in}}%
\pgfpathlineto{\pgfqpoint{5.287046in}{3.191365in}}%
\pgfpathlineto{\pgfqpoint{5.294719in}{3.214004in}}%
\pgfpathlineto{\pgfqpoint{5.302402in}{3.237148in}}%
\pgfpathlineto{\pgfqpoint{5.310097in}{3.260807in}}%
\pgfpathlineto{\pgfqpoint{5.317802in}{3.284993in}}%
\pgfpathlineto{\pgfqpoint{5.304150in}{3.291919in}}%
\pgfpathlineto{\pgfqpoint{5.290503in}{3.298914in}}%
\pgfpathlineto{\pgfqpoint{5.276863in}{3.305978in}}%
\pgfpathlineto{\pgfqpoint{5.263228in}{3.313110in}}%
\pgfpathlineto{\pgfqpoint{5.255512in}{3.288356in}}%
\pgfpathlineto{\pgfqpoint{5.247806in}{3.264133in}}%
\pgfpathlineto{\pgfqpoint{5.240111in}{3.240432in}}%
\pgfpathlineto{\pgfqpoint{5.232427in}{3.217241in}}%
\pgfpathclose%
\pgfusepath{fill}%
\end{pgfscope}%
\begin{pgfscope}%
\pgfpathrectangle{\pgfqpoint{1.150000in}{0.150000in}}{\pgfqpoint{5.700000in}{5.700000in}}%
\pgfusepath{clip}%
\pgfsetbuttcap%
\pgfsetroundjoin%
\definecolor{currentfill}{rgb}{0.282623,0.140926,0.457517}%
\pgfsetfillcolor{currentfill}%
\pgfsetfillopacity{0.700000}%
\pgfsetlinewidth{0.000000pt}%
\definecolor{currentstroke}{rgb}{0.000000,0.000000,0.000000}%
\pgfsetstrokecolor{currentstroke}%
\pgfsetdash{}{0pt}%
\pgfpathmoveto{\pgfqpoint{3.682028in}{2.592868in}}%
\pgfpathlineto{\pgfqpoint{3.695416in}{2.586747in}}%
\pgfpathlineto{\pgfqpoint{3.708808in}{2.580718in}}%
\pgfpathlineto{\pgfqpoint{3.722204in}{2.574780in}}%
\pgfpathlineto{\pgfqpoint{3.735605in}{2.568932in}}%
\pgfpathlineto{\pgfqpoint{3.743478in}{2.580284in}}%
\pgfpathlineto{\pgfqpoint{3.751345in}{2.591754in}}%
\pgfpathlineto{\pgfqpoint{3.759207in}{2.603346in}}%
\pgfpathlineto{\pgfqpoint{3.767063in}{2.615065in}}%
\pgfpathlineto{\pgfqpoint{3.753671in}{2.621129in}}%
\pgfpathlineto{\pgfqpoint{3.740283in}{2.627283in}}%
\pgfpathlineto{\pgfqpoint{3.726899in}{2.633528in}}%
\pgfpathlineto{\pgfqpoint{3.713520in}{2.639865in}}%
\pgfpathlineto{\pgfqpoint{3.705655in}{2.627923in}}%
\pgfpathlineto{\pgfqpoint{3.697785in}{2.616112in}}%
\pgfpathlineto{\pgfqpoint{3.689909in}{2.604428in}}%
\pgfpathlineto{\pgfqpoint{3.682028in}{2.592868in}}%
\pgfpathclose%
\pgfusepath{fill}%
\end{pgfscope}%
\begin{pgfscope}%
\pgfpathrectangle{\pgfqpoint{1.150000in}{0.150000in}}{\pgfqpoint{5.700000in}{5.700000in}}%
\pgfusepath{clip}%
\pgfsetbuttcap%
\pgfsetroundjoin%
\definecolor{currentfill}{rgb}{0.208623,0.367752,0.552675}%
\pgfsetfillcolor{currentfill}%
\pgfsetfillopacity{0.700000}%
\pgfsetlinewidth{0.000000pt}%
\definecolor{currentstroke}{rgb}{0.000000,0.000000,0.000000}%
\pgfsetstrokecolor{currentstroke}%
\pgfsetdash{}{0pt}%
\pgfpathmoveto{\pgfqpoint{5.116651in}{3.071602in}}%
\pgfpathlineto{\pgfqpoint{5.130295in}{3.065809in}}%
\pgfpathlineto{\pgfqpoint{5.143945in}{3.060086in}}%
\pgfpathlineto{\pgfqpoint{5.157602in}{3.054432in}}%
\pgfpathlineto{\pgfqpoint{5.171265in}{3.048847in}}%
\pgfpathlineto{\pgfqpoint{5.178883in}{3.068325in}}%
\pgfpathlineto{\pgfqpoint{5.186509in}{3.088231in}}%
\pgfpathlineto{\pgfqpoint{5.194141in}{3.108577in}}%
\pgfpathlineto{\pgfqpoint{5.201781in}{3.129372in}}%
\pgfpathlineto{\pgfqpoint{5.188130in}{3.135474in}}%
\pgfpathlineto{\pgfqpoint{5.174486in}{3.141646in}}%
\pgfpathlineto{\pgfqpoint{5.160847in}{3.147888in}}%
\pgfpathlineto{\pgfqpoint{5.147215in}{3.154199in}}%
\pgfpathlineto{\pgfqpoint{5.139563in}{3.132878in}}%
\pgfpathlineto{\pgfqpoint{5.131919in}{3.112012in}}%
\pgfpathlineto{\pgfqpoint{5.124281in}{3.091590in}}%
\pgfpathlineto{\pgfqpoint{5.116651in}{3.071602in}}%
\pgfpathclose%
\pgfusepath{fill}%
\end{pgfscope}%
\begin{pgfscope}%
\pgfpathrectangle{\pgfqpoint{1.150000in}{0.150000in}}{\pgfqpoint{5.700000in}{5.700000in}}%
\pgfusepath{clip}%
\pgfsetbuttcap%
\pgfsetroundjoin%
\definecolor{currentfill}{rgb}{0.279574,0.170599,0.479997}%
\pgfsetfillcolor{currentfill}%
\pgfsetfillopacity{0.700000}%
\pgfsetlinewidth{0.000000pt}%
\definecolor{currentstroke}{rgb}{0.000000,0.000000,0.000000}%
\pgfsetstrokecolor{currentstroke}%
\pgfsetdash{}{0pt}%
\pgfpathmoveto{\pgfqpoint{4.129271in}{2.648894in}}%
\pgfpathlineto{\pgfqpoint{4.142739in}{2.643864in}}%
\pgfpathlineto{\pgfqpoint{4.156213in}{2.638915in}}%
\pgfpathlineto{\pgfqpoint{4.169693in}{2.634047in}}%
\pgfpathlineto{\pgfqpoint{4.183178in}{2.629260in}}%
\pgfpathlineto{\pgfqpoint{4.190921in}{2.641310in}}%
\pgfpathlineto{\pgfqpoint{4.198660in}{2.653524in}}%
\pgfpathlineto{\pgfqpoint{4.206395in}{2.665906in}}%
\pgfpathlineto{\pgfqpoint{4.214126in}{2.678464in}}%
\pgfpathlineto{\pgfqpoint{4.200650in}{2.683547in}}%
\pgfpathlineto{\pgfqpoint{4.187180in}{2.688711in}}%
\pgfpathlineto{\pgfqpoint{4.173715in}{2.693956in}}%
\pgfpathlineto{\pgfqpoint{4.160255in}{2.699283in}}%
\pgfpathlineto{\pgfqpoint{4.152515in}{2.686422in}}%
\pgfpathlineto{\pgfqpoint{4.144771in}{2.673741in}}%
\pgfpathlineto{\pgfqpoint{4.137023in}{2.661233in}}%
\pgfpathlineto{\pgfqpoint{4.129271in}{2.648894in}}%
\pgfpathclose%
\pgfusepath{fill}%
\end{pgfscope}%
\begin{pgfscope}%
\pgfpathrectangle{\pgfqpoint{1.150000in}{0.150000in}}{\pgfqpoint{5.700000in}{5.700000in}}%
\pgfusepath{clip}%
\pgfsetbuttcap%
\pgfsetroundjoin%
\definecolor{currentfill}{rgb}{0.271828,0.209303,0.504434}%
\pgfsetfillcolor{currentfill}%
\pgfsetfillopacity{0.700000}%
\pgfsetlinewidth{0.000000pt}%
\definecolor{currentstroke}{rgb}{0.000000,0.000000,0.000000}%
\pgfsetstrokecolor{currentstroke}%
\pgfsetdash{}{0pt}%
\pgfpathmoveto{\pgfqpoint{4.437783in}{2.723260in}}%
\pgfpathlineto{\pgfqpoint{4.451315in}{2.718546in}}%
\pgfpathlineto{\pgfqpoint{4.464853in}{2.713909in}}%
\pgfpathlineto{\pgfqpoint{4.478397in}{2.709348in}}%
\pgfpathlineto{\pgfqpoint{4.491948in}{2.704863in}}%
\pgfpathlineto{\pgfqpoint{4.499611in}{2.717860in}}%
\pgfpathlineto{\pgfqpoint{4.507273in}{2.731072in}}%
\pgfpathlineto{\pgfqpoint{4.514933in}{2.744506in}}%
\pgfpathlineto{\pgfqpoint{4.522590in}{2.758167in}}%
\pgfpathlineto{\pgfqpoint{4.509050in}{2.763008in}}%
\pgfpathlineto{\pgfqpoint{4.495516in}{2.767925in}}%
\pgfpathlineto{\pgfqpoint{4.481989in}{2.772919in}}%
\pgfpathlineto{\pgfqpoint{4.468467in}{2.777989in}}%
\pgfpathlineto{\pgfqpoint{4.460799in}{2.763964in}}%
\pgfpathlineto{\pgfqpoint{4.453129in}{2.750172in}}%
\pgfpathlineto{\pgfqpoint{4.445457in}{2.736606in}}%
\pgfpathlineto{\pgfqpoint{4.437783in}{2.723260in}}%
\pgfpathclose%
\pgfusepath{fill}%
\end{pgfscope}%
\begin{pgfscope}%
\pgfpathrectangle{\pgfqpoint{1.150000in}{0.150000in}}{\pgfqpoint{5.700000in}{5.700000in}}%
\pgfusepath{clip}%
\pgfsetbuttcap%
\pgfsetroundjoin%
\definecolor{currentfill}{rgb}{0.282623,0.140926,0.457517}%
\pgfsetfillcolor{currentfill}%
\pgfsetfillopacity{0.700000}%
\pgfsetlinewidth{0.000000pt}%
\definecolor{currentstroke}{rgb}{0.000000,0.000000,0.000000}%
\pgfsetstrokecolor{currentstroke}%
\pgfsetdash{}{0pt}%
\pgfpathmoveto{\pgfqpoint{3.319486in}{2.594367in}}%
\pgfpathlineto{\pgfqpoint{3.332833in}{2.586822in}}%
\pgfpathlineto{\pgfqpoint{3.346183in}{2.579380in}}%
\pgfpathlineto{\pgfqpoint{3.359535in}{2.572042in}}%
\pgfpathlineto{\pgfqpoint{3.372890in}{2.564806in}}%
\pgfpathlineto{\pgfqpoint{3.380876in}{2.575764in}}%
\pgfpathlineto{\pgfqpoint{3.388856in}{2.586824in}}%
\pgfpathlineto{\pgfqpoint{3.396829in}{2.597988in}}%
\pgfpathlineto{\pgfqpoint{3.404797in}{2.609260in}}%
\pgfpathlineto{\pgfqpoint{3.391450in}{2.616652in}}%
\pgfpathlineto{\pgfqpoint{3.378107in}{2.624146in}}%
\pgfpathlineto{\pgfqpoint{3.364766in}{2.631743in}}%
\pgfpathlineto{\pgfqpoint{3.351428in}{2.639444in}}%
\pgfpathlineto{\pgfqpoint{3.343452in}{2.628009in}}%
\pgfpathlineto{\pgfqpoint{3.335470in}{2.616687in}}%
\pgfpathlineto{\pgfqpoint{3.327481in}{2.605473in}}%
\pgfpathlineto{\pgfqpoint{3.319486in}{2.594367in}}%
\pgfpathclose%
\pgfusepath{fill}%
\end{pgfscope}%
\begin{pgfscope}%
\pgfpathrectangle{\pgfqpoint{1.150000in}{0.150000in}}{\pgfqpoint{5.700000in}{5.700000in}}%
\pgfusepath{clip}%
\pgfsetbuttcap%
\pgfsetroundjoin%
\definecolor{currentfill}{rgb}{0.281887,0.150881,0.465405}%
\pgfsetfillcolor{currentfill}%
\pgfsetfillopacity{0.700000}%
\pgfsetlinewidth{0.000000pt}%
\definecolor{currentstroke}{rgb}{0.000000,0.000000,0.000000}%
\pgfsetstrokecolor{currentstroke}%
\pgfsetdash{}{0pt}%
\pgfpathmoveto{\pgfqpoint{3.180676in}{2.614035in}}%
\pgfpathlineto{\pgfqpoint{3.194015in}{2.605770in}}%
\pgfpathlineto{\pgfqpoint{3.207356in}{2.597615in}}%
\pgfpathlineto{\pgfqpoint{3.220698in}{2.589569in}}%
\pgfpathlineto{\pgfqpoint{3.234043in}{2.581631in}}%
\pgfpathlineto{\pgfqpoint{3.242072in}{2.592470in}}%
\pgfpathlineto{\pgfqpoint{3.250096in}{2.603409in}}%
\pgfpathlineto{\pgfqpoint{3.258112in}{2.614450in}}%
\pgfpathlineto{\pgfqpoint{3.266123in}{2.625597in}}%
\pgfpathlineto{\pgfqpoint{3.252787in}{2.633670in}}%
\pgfpathlineto{\pgfqpoint{3.239454in}{2.641852in}}%
\pgfpathlineto{\pgfqpoint{3.226122in}{2.650143in}}%
\pgfpathlineto{\pgfqpoint{3.212792in}{2.658543in}}%
\pgfpathlineto{\pgfqpoint{3.204773in}{2.647253in}}%
\pgfpathlineto{\pgfqpoint{3.196748in}{2.636073in}}%
\pgfpathlineto{\pgfqpoint{3.188715in}{2.625001in}}%
\pgfpathlineto{\pgfqpoint{3.180676in}{2.614035in}}%
\pgfpathclose%
\pgfusepath{fill}%
\end{pgfscope}%
\begin{pgfscope}%
\pgfpathrectangle{\pgfqpoint{1.150000in}{0.150000in}}{\pgfqpoint{5.700000in}{5.700000in}}%
\pgfusepath{clip}%
\pgfsetbuttcap%
\pgfsetroundjoin%
\definecolor{currentfill}{rgb}{0.274128,0.199721,0.498911}%
\pgfsetfillcolor{currentfill}%
\pgfsetfillopacity{0.700000}%
\pgfsetlinewidth{0.000000pt}%
\definecolor{currentstroke}{rgb}{0.000000,0.000000,0.000000}%
\pgfsetstrokecolor{currentstroke}%
\pgfsetdash{}{0pt}%
\pgfpathmoveto{\pgfqpoint{2.849170in}{2.715211in}}%
\pgfpathlineto{\pgfqpoint{2.862514in}{2.704777in}}%
\pgfpathlineto{\pgfqpoint{2.875858in}{2.694472in}}%
\pgfpathlineto{\pgfqpoint{2.889202in}{2.684294in}}%
\pgfpathlineto{\pgfqpoint{2.902546in}{2.674243in}}%
\pgfpathlineto{\pgfqpoint{2.910682in}{2.684808in}}%
\pgfpathlineto{\pgfqpoint{2.918810in}{2.695481in}}%
\pgfpathlineto{\pgfqpoint{2.926930in}{2.706261in}}%
\pgfpathlineto{\pgfqpoint{2.935044in}{2.717153in}}%
\pgfpathlineto{\pgfqpoint{2.921710in}{2.727299in}}%
\pgfpathlineto{\pgfqpoint{2.908377in}{2.737572in}}%
\pgfpathlineto{\pgfqpoint{2.895043in}{2.747973in}}%
\pgfpathlineto{\pgfqpoint{2.881709in}{2.758502in}}%
\pgfpathlineto{\pgfqpoint{2.873586in}{2.747508in}}%
\pgfpathlineto{\pgfqpoint{2.865455in}{2.736629in}}%
\pgfpathlineto{\pgfqpoint{2.857316in}{2.725864in}}%
\pgfpathlineto{\pgfqpoint{2.849170in}{2.715211in}}%
\pgfpathclose%
\pgfusepath{fill}%
\end{pgfscope}%
\begin{pgfscope}%
\pgfpathrectangle{\pgfqpoint{1.150000in}{0.150000in}}{\pgfqpoint{5.700000in}{5.700000in}}%
\pgfusepath{clip}%
\pgfsetbuttcap%
\pgfsetroundjoin%
\definecolor{currentfill}{rgb}{0.282884,0.135920,0.453427}%
\pgfsetfillcolor{currentfill}%
\pgfsetfillopacity{0.700000}%
\pgfsetlinewidth{0.000000pt}%
\definecolor{currentstroke}{rgb}{0.000000,0.000000,0.000000}%
\pgfsetstrokecolor{currentstroke}%
\pgfsetdash{}{0pt}%
\pgfpathmoveto{\pgfqpoint{3.458210in}{2.580701in}}%
\pgfpathlineto{\pgfqpoint{3.471571in}{2.573810in}}%
\pgfpathlineto{\pgfqpoint{3.484936in}{2.567017in}}%
\pgfpathlineto{\pgfqpoint{3.498304in}{2.560323in}}%
\pgfpathlineto{\pgfqpoint{3.511675in}{2.553725in}}%
\pgfpathlineto{\pgfqpoint{3.519619in}{2.564771in}}%
\pgfpathlineto{\pgfqpoint{3.527557in}{2.575920in}}%
\pgfpathlineto{\pgfqpoint{3.535489in}{2.587177in}}%
\pgfpathlineto{\pgfqpoint{3.543415in}{2.598545in}}%
\pgfpathlineto{\pgfqpoint{3.530053in}{2.605318in}}%
\pgfpathlineto{\pgfqpoint{3.516694in}{2.612189in}}%
\pgfpathlineto{\pgfqpoint{3.503338in}{2.619158in}}%
\pgfpathlineto{\pgfqpoint{3.489985in}{2.626224in}}%
\pgfpathlineto{\pgfqpoint{3.482050in}{2.614673in}}%
\pgfpathlineto{\pgfqpoint{3.474110in}{2.603238in}}%
\pgfpathlineto{\pgfqpoint{3.466163in}{2.591915in}}%
\pgfpathlineto{\pgfqpoint{3.458210in}{2.580701in}}%
\pgfpathclose%
\pgfusepath{fill}%
\end{pgfscope}%
\begin{pgfscope}%
\pgfpathrectangle{\pgfqpoint{1.150000in}{0.150000in}}{\pgfqpoint{5.700000in}{5.700000in}}%
\pgfusepath{clip}%
\pgfsetbuttcap%
\pgfsetroundjoin%
\definecolor{currentfill}{rgb}{0.197636,0.391528,0.554969}%
\pgfsetfillcolor{currentfill}%
\pgfsetfillopacity{0.700000}%
\pgfsetlinewidth{0.000000pt}%
\definecolor{currentstroke}{rgb}{0.000000,0.000000,0.000000}%
\pgfsetstrokecolor{currentstroke}%
\pgfsetdash{}{0pt}%
\pgfpathmoveto{\pgfqpoint{5.201781in}{3.129372in}}%
\pgfpathlineto{\pgfqpoint{5.215438in}{3.123338in}}%
\pgfpathlineto{\pgfqpoint{5.229102in}{3.117374in}}%
\pgfpathlineto{\pgfqpoint{5.242772in}{3.111478in}}%
\pgfpathlineto{\pgfqpoint{5.256449in}{3.105651in}}%
\pgfpathlineto{\pgfqpoint{5.264085in}{3.126374in}}%
\pgfpathlineto{\pgfqpoint{5.271729in}{3.147560in}}%
\pgfpathlineto{\pgfqpoint{5.279383in}{3.169221in}}%
\pgfpathlineto{\pgfqpoint{5.287046in}{3.191365in}}%
\pgfpathlineto{\pgfqpoint{5.273382in}{3.197731in}}%
\pgfpathlineto{\pgfqpoint{5.259724in}{3.204165in}}%
\pgfpathlineto{\pgfqpoint{5.246072in}{3.210668in}}%
\pgfpathlineto{\pgfqpoint{5.232427in}{3.217241in}}%
\pgfpathlineto{\pgfqpoint{5.224752in}{3.194550in}}%
\pgfpathlineto{\pgfqpoint{5.217086in}{3.172348in}}%
\pgfpathlineto{\pgfqpoint{5.209429in}{3.150625in}}%
\pgfpathlineto{\pgfqpoint{5.201781in}{3.129372in}}%
\pgfpathclose%
\pgfusepath{fill}%
\end{pgfscope}%
\begin{pgfscope}%
\pgfpathrectangle{\pgfqpoint{1.150000in}{0.150000in}}{\pgfqpoint{5.700000in}{5.700000in}}%
\pgfusepath{clip}%
\pgfsetbuttcap%
\pgfsetroundjoin%
\definecolor{currentfill}{rgb}{0.275191,0.194905,0.496005}%
\pgfsetfillcolor{currentfill}%
\pgfsetfillopacity{0.700000}%
\pgfsetlinewidth{0.000000pt}%
\definecolor{currentstroke}{rgb}{0.000000,0.000000,0.000000}%
\pgfsetstrokecolor{currentstroke}%
\pgfsetdash{}{0pt}%
\pgfpathmoveto{\pgfqpoint{4.352951in}{2.690228in}}%
\pgfpathlineto{\pgfqpoint{4.366469in}{2.685541in}}%
\pgfpathlineto{\pgfqpoint{4.379993in}{2.680932in}}%
\pgfpathlineto{\pgfqpoint{4.393523in}{2.676401in}}%
\pgfpathlineto{\pgfqpoint{4.407059in}{2.671946in}}%
\pgfpathlineto{\pgfqpoint{4.414744in}{2.684477in}}%
\pgfpathlineto{\pgfqpoint{4.422427in}{2.697202in}}%
\pgfpathlineto{\pgfqpoint{4.430106in}{2.710127in}}%
\pgfpathlineto{\pgfqpoint{4.437783in}{2.723260in}}%
\pgfpathlineto{\pgfqpoint{4.424257in}{2.728051in}}%
\pgfpathlineto{\pgfqpoint{4.410737in}{2.732919in}}%
\pgfpathlineto{\pgfqpoint{4.397223in}{2.737864in}}%
\pgfpathlineto{\pgfqpoint{4.383715in}{2.742887in}}%
\pgfpathlineto{\pgfqpoint{4.376028in}{2.729410in}}%
\pgfpathlineto{\pgfqpoint{4.368338in}{2.716146in}}%
\pgfpathlineto{\pgfqpoint{4.360646in}{2.703087in}}%
\pgfpathlineto{\pgfqpoint{4.352951in}{2.690228in}}%
\pgfpathclose%
\pgfusepath{fill}%
\end{pgfscope}%
\begin{pgfscope}%
\pgfpathrectangle{\pgfqpoint{1.150000in}{0.150000in}}{\pgfqpoint{5.700000in}{5.700000in}}%
\pgfusepath{clip}%
\pgfsetbuttcap%
\pgfsetroundjoin%
\definecolor{currentfill}{rgb}{0.282290,0.145912,0.461510}%
\pgfsetfillcolor{currentfill}%
\pgfsetfillopacity{0.700000}%
\pgfsetlinewidth{0.000000pt}%
\definecolor{currentstroke}{rgb}{0.000000,0.000000,0.000000}%
\pgfsetstrokecolor{currentstroke}%
\pgfsetdash{}{0pt}%
\pgfpathmoveto{\pgfqpoint{3.820676in}{2.591707in}}%
\pgfpathlineto{\pgfqpoint{3.834090in}{2.586090in}}%
\pgfpathlineto{\pgfqpoint{3.847509in}{2.580560in}}%
\pgfpathlineto{\pgfqpoint{3.860933in}{2.575118in}}%
\pgfpathlineto{\pgfqpoint{3.874361in}{2.569762in}}%
\pgfpathlineto{\pgfqpoint{3.882196in}{2.581155in}}%
\pgfpathlineto{\pgfqpoint{3.890025in}{2.592673in}}%
\pgfpathlineto{\pgfqpoint{3.897849in}{2.604321in}}%
\pgfpathlineto{\pgfqpoint{3.905668in}{2.616102in}}%
\pgfpathlineto{\pgfqpoint{3.892248in}{2.621694in}}%
\pgfpathlineto{\pgfqpoint{3.878833in}{2.627373in}}%
\pgfpathlineto{\pgfqpoint{3.865423in}{2.633138in}}%
\pgfpathlineto{\pgfqpoint{3.852017in}{2.638992in}}%
\pgfpathlineto{\pgfqpoint{3.844189in}{2.626967in}}%
\pgfpathlineto{\pgfqpoint{3.836357in}{2.615081in}}%
\pgfpathlineto{\pgfqpoint{3.828519in}{2.603329in}}%
\pgfpathlineto{\pgfqpoint{3.820676in}{2.591707in}}%
\pgfpathclose%
\pgfusepath{fill}%
\end{pgfscope}%
\begin{pgfscope}%
\pgfpathrectangle{\pgfqpoint{1.150000in}{0.150000in}}{\pgfqpoint{5.700000in}{5.700000in}}%
\pgfusepath{clip}%
\pgfsetbuttcap%
\pgfsetroundjoin%
\definecolor{currentfill}{rgb}{0.280868,0.160771,0.472899}%
\pgfsetfillcolor{currentfill}%
\pgfsetfillopacity{0.700000}%
\pgfsetlinewidth{0.000000pt}%
\definecolor{currentstroke}{rgb}{0.000000,0.000000,0.000000}%
\pgfsetstrokecolor{currentstroke}%
\pgfsetdash{}{0pt}%
\pgfpathmoveto{\pgfqpoint{4.044363in}{2.620960in}}%
\pgfpathlineto{\pgfqpoint{4.057820in}{2.615876in}}%
\pgfpathlineto{\pgfqpoint{4.071281in}{2.610875in}}%
\pgfpathlineto{\pgfqpoint{4.084748in}{2.605957in}}%
\pgfpathlineto{\pgfqpoint{4.098220in}{2.601121in}}%
\pgfpathlineto{\pgfqpoint{4.105989in}{2.612837in}}%
\pgfpathlineto{\pgfqpoint{4.113754in}{2.624702in}}%
\pgfpathlineto{\pgfqpoint{4.121515in}{2.636719in}}%
\pgfpathlineto{\pgfqpoint{4.129271in}{2.648894in}}%
\pgfpathlineto{\pgfqpoint{4.115808in}{2.654007in}}%
\pgfpathlineto{\pgfqpoint{4.102350in}{2.659201in}}%
\pgfpathlineto{\pgfqpoint{4.088897in}{2.664479in}}%
\pgfpathlineto{\pgfqpoint{4.075450in}{2.669839in}}%
\pgfpathlineto{\pgfqpoint{4.067685in}{2.657380in}}%
\pgfpathlineto{\pgfqpoint{4.059915in}{2.645084in}}%
\pgfpathlineto{\pgfqpoint{4.052142in}{2.632946in}}%
\pgfpathlineto{\pgfqpoint{4.044363in}{2.620960in}}%
\pgfpathclose%
\pgfusepath{fill}%
\end{pgfscope}%
\begin{pgfscope}%
\pgfpathrectangle{\pgfqpoint{1.150000in}{0.150000in}}{\pgfqpoint{5.700000in}{5.700000in}}%
\pgfusepath{clip}%
\pgfsetbuttcap%
\pgfsetroundjoin%
\definecolor{currentfill}{rgb}{0.280255,0.165693,0.476498}%
\pgfsetfillcolor{currentfill}%
\pgfsetfillopacity{0.700000}%
\pgfsetlinewidth{0.000000pt}%
\definecolor{currentstroke}{rgb}{0.000000,0.000000,0.000000}%
\pgfsetstrokecolor{currentstroke}%
\pgfsetdash{}{0pt}%
\pgfpathmoveto{\pgfqpoint{3.041719in}{2.640402in}}%
\pgfpathlineto{\pgfqpoint{3.055057in}{2.631346in}}%
\pgfpathlineto{\pgfqpoint{3.068395in}{2.622407in}}%
\pgfpathlineto{\pgfqpoint{3.081735in}{2.613583in}}%
\pgfpathlineto{\pgfqpoint{3.095075in}{2.604874in}}%
\pgfpathlineto{\pgfqpoint{3.103151in}{2.615554in}}%
\pgfpathlineto{\pgfqpoint{3.111220in}{2.626334in}}%
\pgfpathlineto{\pgfqpoint{3.119282in}{2.637217in}}%
\pgfpathlineto{\pgfqpoint{3.127338in}{2.648205in}}%
\pgfpathlineto{\pgfqpoint{3.114007in}{2.657029in}}%
\pgfpathlineto{\pgfqpoint{3.100677in}{2.665969in}}%
\pgfpathlineto{\pgfqpoint{3.087348in}{2.675024in}}%
\pgfpathlineto{\pgfqpoint{3.074020in}{2.684195in}}%
\pgfpathlineto{\pgfqpoint{3.065956in}{2.673084in}}%
\pgfpathlineto{\pgfqpoint{3.057884in}{2.662084in}}%
\pgfpathlineto{\pgfqpoint{3.049805in}{2.651190in}}%
\pgfpathlineto{\pgfqpoint{3.041719in}{2.640402in}}%
\pgfpathclose%
\pgfusepath{fill}%
\end{pgfscope}%
\begin{pgfscope}%
\pgfpathrectangle{\pgfqpoint{1.150000in}{0.150000in}}{\pgfqpoint{5.700000in}{5.700000in}}%
\pgfusepath{clip}%
\pgfsetbuttcap%
\pgfsetroundjoin%
\definecolor{currentfill}{rgb}{0.241237,0.296485,0.539709}%
\pgfsetfillcolor{currentfill}%
\pgfsetfillopacity{0.700000}%
\pgfsetlinewidth{0.000000pt}%
\definecolor{currentstroke}{rgb}{0.000000,0.000000,0.000000}%
\pgfsetstrokecolor{currentstroke}%
\pgfsetdash{}{0pt}%
\pgfpathmoveto{\pgfqpoint{4.916263in}{2.899704in}}%
\pgfpathlineto{\pgfqpoint{4.929892in}{2.894778in}}%
\pgfpathlineto{\pgfqpoint{4.943527in}{2.889924in}}%
\pgfpathlineto{\pgfqpoint{4.957169in}{2.885141in}}%
\pgfpathlineto{\pgfqpoint{4.970818in}{2.880428in}}%
\pgfpathlineto{\pgfqpoint{4.978408in}{2.896364in}}%
\pgfpathlineto{\pgfqpoint{4.986000in}{2.912632in}}%
\pgfpathlineto{\pgfqpoint{4.993596in}{2.929241in}}%
\pgfpathlineto{\pgfqpoint{5.001195in}{2.946198in}}%
\pgfpathlineto{\pgfqpoint{4.987559in}{2.951368in}}%
\pgfpathlineto{\pgfqpoint{4.973929in}{2.956608in}}%
\pgfpathlineto{\pgfqpoint{4.960306in}{2.961919in}}%
\pgfpathlineto{\pgfqpoint{4.946690in}{2.967301in}}%
\pgfpathlineto{\pgfqpoint{4.939079in}{2.949879in}}%
\pgfpathlineto{\pgfqpoint{4.931471in}{2.932811in}}%
\pgfpathlineto{\pgfqpoint{4.923866in}{2.916089in}}%
\pgfpathlineto{\pgfqpoint{4.916263in}{2.899704in}}%
\pgfpathclose%
\pgfusepath{fill}%
\end{pgfscope}%
\begin{pgfscope}%
\pgfpathrectangle{\pgfqpoint{1.150000in}{0.150000in}}{\pgfqpoint{5.700000in}{5.700000in}}%
\pgfusepath{clip}%
\pgfsetbuttcap%
\pgfsetroundjoin%
\definecolor{currentfill}{rgb}{0.250425,0.274290,0.533103}%
\pgfsetfillcolor{currentfill}%
\pgfsetfillopacity{0.700000}%
\pgfsetlinewidth{0.000000pt}%
\definecolor{currentstroke}{rgb}{0.000000,0.000000,0.000000}%
\pgfsetstrokecolor{currentstroke}%
\pgfsetdash{}{0pt}%
\pgfpathmoveto{\pgfqpoint{4.831375in}{2.856040in}}%
\pgfpathlineto{\pgfqpoint{4.844989in}{2.851265in}}%
\pgfpathlineto{\pgfqpoint{4.858610in}{2.846562in}}%
\pgfpathlineto{\pgfqpoint{4.872238in}{2.841930in}}%
\pgfpathlineto{\pgfqpoint{4.885873in}{2.837370in}}%
\pgfpathlineto{\pgfqpoint{4.893468in}{2.852489in}}%
\pgfpathlineto{\pgfqpoint{4.901064in}{2.867912in}}%
\pgfpathlineto{\pgfqpoint{4.908663in}{2.883647in}}%
\pgfpathlineto{\pgfqpoint{4.916263in}{2.899704in}}%
\pgfpathlineto{\pgfqpoint{4.902641in}{2.904700in}}%
\pgfpathlineto{\pgfqpoint{4.889026in}{2.909769in}}%
\pgfpathlineto{\pgfqpoint{4.875417in}{2.914908in}}%
\pgfpathlineto{\pgfqpoint{4.861814in}{2.920120in}}%
\pgfpathlineto{\pgfqpoint{4.854202in}{2.903619in}}%
\pgfpathlineto{\pgfqpoint{4.846591in}{2.887445in}}%
\pgfpathlineto{\pgfqpoint{4.838982in}{2.871588in}}%
\pgfpathlineto{\pgfqpoint{4.831375in}{2.856040in}}%
\pgfpathclose%
\pgfusepath{fill}%
\end{pgfscope}%
\begin{pgfscope}%
\pgfpathrectangle{\pgfqpoint{1.150000in}{0.150000in}}{\pgfqpoint{5.700000in}{5.700000in}}%
\pgfusepath{clip}%
\pgfsetbuttcap%
\pgfsetroundjoin%
\definecolor{currentfill}{rgb}{0.172719,0.448791,0.557885}%
\pgfsetfillcolor{currentfill}%
\pgfsetfillopacity{0.700000}%
\pgfsetlinewidth{0.000000pt}%
\definecolor{currentstroke}{rgb}{0.000000,0.000000,0.000000}%
\pgfsetstrokecolor{currentstroke}%
\pgfsetdash{}{0pt}%
\pgfpathmoveto{\pgfqpoint{5.317802in}{3.284993in}}%
\pgfpathlineto{\pgfqpoint{5.331461in}{3.278136in}}%
\pgfpathlineto{\pgfqpoint{5.345126in}{3.271348in}}%
\pgfpathlineto{\pgfqpoint{5.358797in}{3.264627in}}%
\pgfpathlineto{\pgfqpoint{5.372474in}{3.257975in}}%
\pgfpathlineto{\pgfqpoint{5.380180in}{3.282125in}}%
\pgfpathlineto{\pgfqpoint{5.387898in}{3.306819in}}%
\pgfpathlineto{\pgfqpoint{5.395630in}{3.332067in}}%
\pgfpathlineto{\pgfqpoint{5.381961in}{3.339153in}}%
\pgfpathlineto{\pgfqpoint{5.368299in}{3.346308in}}%
\pgfpathlineto{\pgfqpoint{5.354643in}{3.353531in}}%
\pgfpathlineto{\pgfqpoint{5.340992in}{3.360822in}}%
\pgfpathlineto{\pgfqpoint{5.333250in}{3.334990in}}%
\pgfpathlineto{\pgfqpoint{5.325520in}{3.309717in}}%
\pgfpathlineto{\pgfqpoint{5.317802in}{3.284993in}}%
\pgfpathclose%
\pgfusepath{fill}%
\end{pgfscope}%
\begin{pgfscope}%
\pgfpathrectangle{\pgfqpoint{1.150000in}{0.150000in}}{\pgfqpoint{5.700000in}{5.700000in}}%
\pgfusepath{clip}%
\pgfsetbuttcap%
\pgfsetroundjoin%
\definecolor{currentfill}{rgb}{0.282884,0.135920,0.453427}%
\pgfsetfillcolor{currentfill}%
\pgfsetfillopacity{0.700000}%
\pgfsetlinewidth{0.000000pt}%
\definecolor{currentstroke}{rgb}{0.000000,0.000000,0.000000}%
\pgfsetstrokecolor{currentstroke}%
\pgfsetdash{}{0pt}%
\pgfpathmoveto{\pgfqpoint{3.596901in}{2.572410in}}%
\pgfpathlineto{\pgfqpoint{3.610282in}{2.566114in}}%
\pgfpathlineto{\pgfqpoint{3.623666in}{2.559911in}}%
\pgfpathlineto{\pgfqpoint{3.637055in}{2.553801in}}%
\pgfpathlineto{\pgfqpoint{3.650447in}{2.547784in}}%
\pgfpathlineto{\pgfqpoint{3.658351in}{2.558889in}}%
\pgfpathlineto{\pgfqpoint{3.666249in}{2.570102in}}%
\pgfpathlineto{\pgfqpoint{3.674141in}{2.581427in}}%
\pgfpathlineto{\pgfqpoint{3.682028in}{2.592868in}}%
\pgfpathlineto{\pgfqpoint{3.668644in}{2.599081in}}%
\pgfpathlineto{\pgfqpoint{3.655264in}{2.605387in}}%
\pgfpathlineto{\pgfqpoint{3.641888in}{2.611786in}}%
\pgfpathlineto{\pgfqpoint{3.628515in}{2.618278in}}%
\pgfpathlineto{\pgfqpoint{3.620620in}{2.606634in}}%
\pgfpathlineto{\pgfqpoint{3.612720in}{2.595111in}}%
\pgfpathlineto{\pgfqpoint{3.604813in}{2.583704in}}%
\pgfpathlineto{\pgfqpoint{3.596901in}{2.572410in}}%
\pgfpathclose%
\pgfusepath{fill}%
\end{pgfscope}%
\begin{pgfscope}%
\pgfpathrectangle{\pgfqpoint{1.150000in}{0.150000in}}{\pgfqpoint{5.700000in}{5.700000in}}%
\pgfusepath{clip}%
\pgfsetbuttcap%
\pgfsetroundjoin%
\definecolor{currentfill}{rgb}{0.231674,0.318106,0.544834}%
\pgfsetfillcolor{currentfill}%
\pgfsetfillopacity{0.700000}%
\pgfsetlinewidth{0.000000pt}%
\definecolor{currentstroke}{rgb}{0.000000,0.000000,0.000000}%
\pgfsetstrokecolor{currentstroke}%
\pgfsetdash{}{0pt}%
\pgfpathmoveto{\pgfqpoint{5.001195in}{2.946198in}}%
\pgfpathlineto{\pgfqpoint{5.014837in}{2.941099in}}%
\pgfpathlineto{\pgfqpoint{5.028486in}{2.936071in}}%
\pgfpathlineto{\pgfqpoint{5.042142in}{2.931112in}}%
\pgfpathlineto{\pgfqpoint{5.055805in}{2.926224in}}%
\pgfpathlineto{\pgfqpoint{5.063395in}{2.943069in}}%
\pgfpathlineto{\pgfqpoint{5.070988in}{2.960276in}}%
\pgfpathlineto{\pgfqpoint{5.078586in}{2.977853in}}%
\pgfpathlineto{\pgfqpoint{5.086188in}{2.995809in}}%
\pgfpathlineto{\pgfqpoint{5.072538in}{3.001174in}}%
\pgfpathlineto{\pgfqpoint{5.058895in}{3.006610in}}%
\pgfpathlineto{\pgfqpoint{5.045258in}{3.012116in}}%
\pgfpathlineto{\pgfqpoint{5.031628in}{3.017691in}}%
\pgfpathlineto{\pgfqpoint{5.024013in}{2.999251in}}%
\pgfpathlineto{\pgfqpoint{5.016403in}{2.981195in}}%
\pgfpathlineto{\pgfqpoint{5.008797in}{2.963513in}}%
\pgfpathlineto{\pgfqpoint{5.001195in}{2.946198in}}%
\pgfpathclose%
\pgfusepath{fill}%
\end{pgfscope}%
\begin{pgfscope}%
\pgfpathrectangle{\pgfqpoint{1.150000in}{0.150000in}}{\pgfqpoint{5.700000in}{5.700000in}}%
\pgfusepath{clip}%
\pgfsetbuttcap%
\pgfsetroundjoin%
\definecolor{currentfill}{rgb}{0.257322,0.256130,0.526563}%
\pgfsetfillcolor{currentfill}%
\pgfsetfillopacity{0.700000}%
\pgfsetlinewidth{0.000000pt}%
\definecolor{currentstroke}{rgb}{0.000000,0.000000,0.000000}%
\pgfsetstrokecolor{currentstroke}%
\pgfsetdash{}{0pt}%
\pgfpathmoveto{\pgfqpoint{4.746513in}{2.814947in}}%
\pgfpathlineto{\pgfqpoint{4.760113in}{2.810299in}}%
\pgfpathlineto{\pgfqpoint{4.773720in}{2.805723in}}%
\pgfpathlineto{\pgfqpoint{4.787333in}{2.801220in}}%
\pgfpathlineto{\pgfqpoint{4.800953in}{2.796790in}}%
\pgfpathlineto{\pgfqpoint{4.808558in}{2.811177in}}%
\pgfpathlineto{\pgfqpoint{4.816163in}{2.825842in}}%
\pgfpathlineto{\pgfqpoint{4.823769in}{2.840794in}}%
\pgfpathlineto{\pgfqpoint{4.831375in}{2.856040in}}%
\pgfpathlineto{\pgfqpoint{4.817767in}{2.860887in}}%
\pgfpathlineto{\pgfqpoint{4.804166in}{2.865807in}}%
\pgfpathlineto{\pgfqpoint{4.790571in}{2.870798in}}%
\pgfpathlineto{\pgfqpoint{4.776982in}{2.875863in}}%
\pgfpathlineto{\pgfqpoint{4.769364in}{2.860193in}}%
\pgfpathlineto{\pgfqpoint{4.761747in}{2.844822in}}%
\pgfpathlineto{\pgfqpoint{4.754130in}{2.829743in}}%
\pgfpathlineto{\pgfqpoint{4.746513in}{2.814947in}}%
\pgfpathclose%
\pgfusepath{fill}%
\end{pgfscope}%
\begin{pgfscope}%
\pgfpathrectangle{\pgfqpoint{1.150000in}{0.150000in}}{\pgfqpoint{5.700000in}{5.700000in}}%
\pgfusepath{clip}%
\pgfsetbuttcap%
\pgfsetroundjoin%
\definecolor{currentfill}{rgb}{0.278012,0.180367,0.486697}%
\pgfsetfillcolor{currentfill}%
\pgfsetfillopacity{0.700000}%
\pgfsetlinewidth{0.000000pt}%
\definecolor{currentstroke}{rgb}{0.000000,0.000000,0.000000}%
\pgfsetstrokecolor{currentstroke}%
\pgfsetdash{}{0pt}%
\pgfpathmoveto{\pgfqpoint{4.268086in}{2.658932in}}%
\pgfpathlineto{\pgfqpoint{4.281590in}{2.654248in}}%
\pgfpathlineto{\pgfqpoint{4.295100in}{2.649643in}}%
\pgfpathlineto{\pgfqpoint{4.308616in}{2.645116in}}%
\pgfpathlineto{\pgfqpoint{4.322139in}{2.640668in}}%
\pgfpathlineto{\pgfqpoint{4.329847in}{2.652788in}}%
\pgfpathlineto{\pgfqpoint{4.337551in}{2.665084in}}%
\pgfpathlineto{\pgfqpoint{4.345253in}{2.677562in}}%
\pgfpathlineto{\pgfqpoint{4.352951in}{2.690228in}}%
\pgfpathlineto{\pgfqpoint{4.339439in}{2.694993in}}%
\pgfpathlineto{\pgfqpoint{4.325933in}{2.699835in}}%
\pgfpathlineto{\pgfqpoint{4.312432in}{2.704757in}}%
\pgfpathlineto{\pgfqpoint{4.298938in}{2.709757in}}%
\pgfpathlineto{\pgfqpoint{4.291230in}{2.696768in}}%
\pgfpathlineto{\pgfqpoint{4.283518in}{2.683971in}}%
\pgfpathlineto{\pgfqpoint{4.275804in}{2.671361in}}%
\pgfpathlineto{\pgfqpoint{4.268086in}{2.658932in}}%
\pgfpathclose%
\pgfusepath{fill}%
\end{pgfscope}%
\begin{pgfscope}%
\pgfpathrectangle{\pgfqpoint{1.150000in}{0.150000in}}{\pgfqpoint{5.700000in}{5.700000in}}%
\pgfusepath{clip}%
\pgfsetbuttcap%
\pgfsetroundjoin%
\definecolor{currentfill}{rgb}{0.221989,0.339161,0.548752}%
\pgfsetfillcolor{currentfill}%
\pgfsetfillopacity{0.700000}%
\pgfsetlinewidth{0.000000pt}%
\definecolor{currentstroke}{rgb}{0.000000,0.000000,0.000000}%
\pgfsetstrokecolor{currentstroke}%
\pgfsetdash{}{0pt}%
\pgfpathmoveto{\pgfqpoint{5.086188in}{2.995809in}}%
\pgfpathlineto{\pgfqpoint{5.099845in}{2.990513in}}%
\pgfpathlineto{\pgfqpoint{5.113508in}{2.985287in}}%
\pgfpathlineto{\pgfqpoint{5.127177in}{2.980131in}}%
\pgfpathlineto{\pgfqpoint{5.140854in}{2.975044in}}%
\pgfpathlineto{\pgfqpoint{5.148448in}{2.992897in}}%
\pgfpathlineto{\pgfqpoint{5.156048in}{3.011143in}}%
\pgfpathlineto{\pgfqpoint{5.163653in}{3.029790in}}%
\pgfpathlineto{\pgfqpoint{5.171265in}{3.048847in}}%
\pgfpathlineto{\pgfqpoint{5.157602in}{3.054432in}}%
\pgfpathlineto{\pgfqpoint{5.143945in}{3.060086in}}%
\pgfpathlineto{\pgfqpoint{5.130295in}{3.065809in}}%
\pgfpathlineto{\pgfqpoint{5.116651in}{3.071602in}}%
\pgfpathlineto{\pgfqpoint{5.109026in}{3.052040in}}%
\pgfpathlineto{\pgfqpoint{5.101408in}{3.032893in}}%
\pgfpathlineto{\pgfqpoint{5.093796in}{3.014153in}}%
\pgfpathlineto{\pgfqpoint{5.086188in}{2.995809in}}%
\pgfpathclose%
\pgfusepath{fill}%
\end{pgfscope}%
\begin{pgfscope}%
\pgfpathrectangle{\pgfqpoint{1.150000in}{0.150000in}}{\pgfqpoint{5.700000in}{5.700000in}}%
\pgfusepath{clip}%
\pgfsetbuttcap%
\pgfsetroundjoin%
\definecolor{currentfill}{rgb}{0.263663,0.237631,0.518762}%
\pgfsetfillcolor{currentfill}%
\pgfsetfillopacity{0.700000}%
\pgfsetlinewidth{0.000000pt}%
\definecolor{currentstroke}{rgb}{0.000000,0.000000,0.000000}%
\pgfsetstrokecolor{currentstroke}%
\pgfsetdash{}{0pt}%
\pgfpathmoveto{\pgfqpoint{4.661663in}{2.776189in}}%
\pgfpathlineto{\pgfqpoint{4.675249in}{2.771645in}}%
\pgfpathlineto{\pgfqpoint{4.688841in}{2.767173in}}%
\pgfpathlineto{\pgfqpoint{4.702440in}{2.762776in}}%
\pgfpathlineto{\pgfqpoint{4.716045in}{2.758451in}}%
\pgfpathlineto{\pgfqpoint{4.723663in}{2.772187in}}%
\pgfpathlineto{\pgfqpoint{4.731280in}{2.786176in}}%
\pgfpathlineto{\pgfqpoint{4.738897in}{2.800427in}}%
\pgfpathlineto{\pgfqpoint{4.746513in}{2.814947in}}%
\pgfpathlineto{\pgfqpoint{4.732920in}{2.819668in}}%
\pgfpathlineto{\pgfqpoint{4.719333in}{2.824462in}}%
\pgfpathlineto{\pgfqpoint{4.705752in}{2.829329in}}%
\pgfpathlineto{\pgfqpoint{4.692177in}{2.834270in}}%
\pgfpathlineto{\pgfqpoint{4.684550in}{2.819347in}}%
\pgfpathlineto{\pgfqpoint{4.676921in}{2.804697in}}%
\pgfpathlineto{\pgfqpoint{4.669293in}{2.790314in}}%
\pgfpathlineto{\pgfqpoint{4.661663in}{2.776189in}}%
\pgfpathclose%
\pgfusepath{fill}%
\end{pgfscope}%
\begin{pgfscope}%
\pgfpathrectangle{\pgfqpoint{1.150000in}{0.150000in}}{\pgfqpoint{5.700000in}{5.700000in}}%
\pgfusepath{clip}%
\pgfsetbuttcap%
\pgfsetroundjoin%
\definecolor{currentfill}{rgb}{0.185556,0.418570,0.556753}%
\pgfsetfillcolor{currentfill}%
\pgfsetfillopacity{0.700000}%
\pgfsetlinewidth{0.000000pt}%
\definecolor{currentstroke}{rgb}{0.000000,0.000000,0.000000}%
\pgfsetstrokecolor{currentstroke}%
\pgfsetdash{}{0pt}%
\pgfpathmoveto{\pgfqpoint{5.287046in}{3.191365in}}%
\pgfpathlineto{\pgfqpoint{5.300716in}{3.185068in}}%
\pgfpathlineto{\pgfqpoint{5.314393in}{3.178839in}}%
\pgfpathlineto{\pgfqpoint{5.328076in}{3.172679in}}%
\pgfpathlineto{\pgfqpoint{5.341766in}{3.166586in}}%
\pgfpathlineto{\pgfqpoint{5.349426in}{3.188673in}}%
\pgfpathlineto{\pgfqpoint{5.357097in}{3.211260in}}%
\pgfpathlineto{\pgfqpoint{5.364780in}{3.234357in}}%
\pgfpathlineto{\pgfqpoint{5.372474in}{3.257975in}}%
\pgfpathlineto{\pgfqpoint{5.358797in}{3.264627in}}%
\pgfpathlineto{\pgfqpoint{5.345126in}{3.271348in}}%
\pgfpathlineto{\pgfqpoint{5.331461in}{3.278136in}}%
\pgfpathlineto{\pgfqpoint{5.317802in}{3.284993in}}%
\pgfpathlineto{\pgfqpoint{5.310097in}{3.260807in}}%
\pgfpathlineto{\pgfqpoint{5.302402in}{3.237148in}}%
\pgfpathlineto{\pgfqpoint{5.294719in}{3.214004in}}%
\pgfpathlineto{\pgfqpoint{5.287046in}{3.191365in}}%
\pgfpathclose%
\pgfusepath{fill}%
\end{pgfscope}%
\begin{pgfscope}%
\pgfpathrectangle{\pgfqpoint{1.150000in}{0.150000in}}{\pgfqpoint{5.700000in}{5.700000in}}%
\pgfusepath{clip}%
\pgfsetbuttcap%
\pgfsetroundjoin%
\definecolor{currentfill}{rgb}{0.278012,0.180367,0.486697}%
\pgfsetfillcolor{currentfill}%
\pgfsetfillopacity{0.700000}%
\pgfsetlinewidth{0.000000pt}%
\definecolor{currentstroke}{rgb}{0.000000,0.000000,0.000000}%
\pgfsetstrokecolor{currentstroke}%
\pgfsetdash{}{0pt}%
\pgfpathmoveto{\pgfqpoint{2.902546in}{2.674243in}}%
\pgfpathlineto{\pgfqpoint{2.915890in}{2.664316in}}%
\pgfpathlineto{\pgfqpoint{2.929234in}{2.654514in}}%
\pgfpathlineto{\pgfqpoint{2.942578in}{2.644835in}}%
\pgfpathlineto{\pgfqpoint{2.955922in}{2.635279in}}%
\pgfpathlineto{\pgfqpoint{2.964047in}{2.645757in}}%
\pgfpathlineto{\pgfqpoint{2.972165in}{2.656336in}}%
\pgfpathlineto{\pgfqpoint{2.980275in}{2.667019in}}%
\pgfpathlineto{\pgfqpoint{2.988378in}{2.677808in}}%
\pgfpathlineto{\pgfqpoint{2.975044in}{2.687460in}}%
\pgfpathlineto{\pgfqpoint{2.961710in}{2.697234in}}%
\pgfpathlineto{\pgfqpoint{2.948377in}{2.707131in}}%
\pgfpathlineto{\pgfqpoint{2.935044in}{2.717153in}}%
\pgfpathlineto{\pgfqpoint{2.926930in}{2.706261in}}%
\pgfpathlineto{\pgfqpoint{2.918810in}{2.695481in}}%
\pgfpathlineto{\pgfqpoint{2.910682in}{2.684808in}}%
\pgfpathlineto{\pgfqpoint{2.902546in}{2.674243in}}%
\pgfpathclose%
\pgfusepath{fill}%
\end{pgfscope}%
\begin{pgfscope}%
\pgfpathrectangle{\pgfqpoint{1.150000in}{0.150000in}}{\pgfqpoint{5.700000in}{5.700000in}}%
\pgfusepath{clip}%
\pgfsetbuttcap%
\pgfsetroundjoin%
\definecolor{currentfill}{rgb}{0.281887,0.150881,0.465405}%
\pgfsetfillcolor{currentfill}%
\pgfsetfillopacity{0.700000}%
\pgfsetlinewidth{0.000000pt}%
\definecolor{currentstroke}{rgb}{0.000000,0.000000,0.000000}%
\pgfsetstrokecolor{currentstroke}%
\pgfsetdash{}{0pt}%
\pgfpathmoveto{\pgfqpoint{3.959396in}{2.594597in}}%
\pgfpathlineto{\pgfqpoint{3.972841in}{2.589433in}}%
\pgfpathlineto{\pgfqpoint{3.986290in}{2.584355in}}%
\pgfpathlineto{\pgfqpoint{3.999745in}{2.579360in}}%
\pgfpathlineto{\pgfqpoint{4.013205in}{2.574449in}}%
\pgfpathlineto{\pgfqpoint{4.021002in}{2.585872in}}%
\pgfpathlineto{\pgfqpoint{4.028793in}{2.597428in}}%
\pgfpathlineto{\pgfqpoint{4.036581in}{2.609123in}}%
\pgfpathlineto{\pgfqpoint{4.044363in}{2.620960in}}%
\pgfpathlineto{\pgfqpoint{4.030912in}{2.626127in}}%
\pgfpathlineto{\pgfqpoint{4.017467in}{2.631378in}}%
\pgfpathlineto{\pgfqpoint{4.004026in}{2.636713in}}%
\pgfpathlineto{\pgfqpoint{3.990590in}{2.642133in}}%
\pgfpathlineto{\pgfqpoint{3.982799in}{2.630032in}}%
\pgfpathlineto{\pgfqpoint{3.975003in}{2.618079in}}%
\pgfpathlineto{\pgfqpoint{3.967202in}{2.606268in}}%
\pgfpathlineto{\pgfqpoint{3.959396in}{2.594597in}}%
\pgfpathclose%
\pgfusepath{fill}%
\end{pgfscope}%
\begin{pgfscope}%
\pgfpathrectangle{\pgfqpoint{1.150000in}{0.150000in}}{\pgfqpoint{5.700000in}{5.700000in}}%
\pgfusepath{clip}%
\pgfsetbuttcap%
\pgfsetroundjoin%
\definecolor{currentfill}{rgb}{0.282884,0.135920,0.453427}%
\pgfsetfillcolor{currentfill}%
\pgfsetfillopacity{0.700000}%
\pgfsetlinewidth{0.000000pt}%
\definecolor{currentstroke}{rgb}{0.000000,0.000000,0.000000}%
\pgfsetstrokecolor{currentstroke}%
\pgfsetdash{}{0pt}%
\pgfpathmoveto{\pgfqpoint{3.735605in}{2.568932in}}%
\pgfpathlineto{\pgfqpoint{3.749010in}{2.563175in}}%
\pgfpathlineto{\pgfqpoint{3.762419in}{2.557507in}}%
\pgfpathlineto{\pgfqpoint{3.775833in}{2.551929in}}%
\pgfpathlineto{\pgfqpoint{3.789251in}{2.546439in}}%
\pgfpathlineto{\pgfqpoint{3.797116in}{2.557582in}}%
\pgfpathlineto{\pgfqpoint{3.804974in}{2.568838in}}%
\pgfpathlineto{\pgfqpoint{3.812828in}{2.580212in}}%
\pgfpathlineto{\pgfqpoint{3.820676in}{2.591707in}}%
\pgfpathlineto{\pgfqpoint{3.807266in}{2.597413in}}%
\pgfpathlineto{\pgfqpoint{3.793861in}{2.603208in}}%
\pgfpathlineto{\pgfqpoint{3.780460in}{2.609092in}}%
\pgfpathlineto{\pgfqpoint{3.767063in}{2.615065in}}%
\pgfpathlineto{\pgfqpoint{3.759207in}{2.603346in}}%
\pgfpathlineto{\pgfqpoint{3.751345in}{2.591754in}}%
\pgfpathlineto{\pgfqpoint{3.743478in}{2.580284in}}%
\pgfpathlineto{\pgfqpoint{3.735605in}{2.568932in}}%
\pgfpathclose%
\pgfusepath{fill}%
\end{pgfscope}%
\begin{pgfscope}%
\pgfpathrectangle{\pgfqpoint{1.150000in}{0.150000in}}{\pgfqpoint{5.700000in}{5.700000in}}%
\pgfusepath{clip}%
\pgfsetbuttcap%
\pgfsetroundjoin%
\definecolor{currentfill}{rgb}{0.267968,0.223549,0.512008}%
\pgfsetfillcolor{currentfill}%
\pgfsetfillopacity{0.700000}%
\pgfsetlinewidth{0.000000pt}%
\definecolor{currentstroke}{rgb}{0.000000,0.000000,0.000000}%
\pgfsetstrokecolor{currentstroke}%
\pgfsetdash{}{0pt}%
\pgfpathmoveto{\pgfqpoint{4.576812in}{2.739557in}}%
\pgfpathlineto{\pgfqpoint{4.590383in}{2.735092in}}%
\pgfpathlineto{\pgfqpoint{4.603961in}{2.730701in}}%
\pgfpathlineto{\pgfqpoint{4.617545in}{2.726385in}}%
\pgfpathlineto{\pgfqpoint{4.631136in}{2.722143in}}%
\pgfpathlineto{\pgfqpoint{4.638770in}{2.735301in}}%
\pgfpathlineto{\pgfqpoint{4.646402in}{2.748690in}}%
\pgfpathlineto{\pgfqpoint{4.654033in}{2.762317in}}%
\pgfpathlineto{\pgfqpoint{4.661663in}{2.776189in}}%
\pgfpathlineto{\pgfqpoint{4.648084in}{2.780808in}}%
\pgfpathlineto{\pgfqpoint{4.634511in}{2.785500in}}%
\pgfpathlineto{\pgfqpoint{4.620945in}{2.790267in}}%
\pgfpathlineto{\pgfqpoint{4.607385in}{2.795108in}}%
\pgfpathlineto{\pgfqpoint{4.599743in}{2.780852in}}%
\pgfpathlineto{\pgfqpoint{4.592101in}{2.766846in}}%
\pgfpathlineto{\pgfqpoint{4.584457in}{2.753083in}}%
\pgfpathlineto{\pgfqpoint{4.576812in}{2.739557in}}%
\pgfpathclose%
\pgfusepath{fill}%
\end{pgfscope}%
\begin{pgfscope}%
\pgfpathrectangle{\pgfqpoint{1.150000in}{0.150000in}}{\pgfqpoint{5.700000in}{5.700000in}}%
\pgfusepath{clip}%
\pgfsetbuttcap%
\pgfsetroundjoin%
\definecolor{currentfill}{rgb}{0.267968,0.223549,0.512008}%
\pgfsetfillcolor{currentfill}%
\pgfsetfillopacity{0.700000}%
\pgfsetlinewidth{0.000000pt}%
\definecolor{currentstroke}{rgb}{0.000000,0.000000,0.000000}%
\pgfsetstrokecolor{currentstroke}%
\pgfsetdash{}{0pt}%
\pgfpathmoveto{\pgfqpoint{2.709629in}{2.761310in}}%
\pgfpathlineto{\pgfqpoint{2.722993in}{2.749881in}}%
\pgfpathlineto{\pgfqpoint{2.736355in}{2.738590in}}%
\pgfpathlineto{\pgfqpoint{2.749716in}{2.727436in}}%
\pgfpathlineto{\pgfqpoint{2.763076in}{2.716417in}}%
\pgfpathlineto{\pgfqpoint{2.771265in}{2.726711in}}%
\pgfpathlineto{\pgfqpoint{2.779447in}{2.737114in}}%
\pgfpathlineto{\pgfqpoint{2.787620in}{2.747627in}}%
\pgfpathlineto{\pgfqpoint{2.795785in}{2.758254in}}%
\pgfpathlineto{\pgfqpoint{2.782437in}{2.769347in}}%
\pgfpathlineto{\pgfqpoint{2.769087in}{2.780576in}}%
\pgfpathlineto{\pgfqpoint{2.755736in}{2.791942in}}%
\pgfpathlineto{\pgfqpoint{2.742384in}{2.803445in}}%
\pgfpathlineto{\pgfqpoint{2.734208in}{2.792737in}}%
\pgfpathlineto{\pgfqpoint{2.726023in}{2.782146in}}%
\pgfpathlineto{\pgfqpoint{2.717830in}{2.771671in}}%
\pgfpathlineto{\pgfqpoint{2.709629in}{2.761310in}}%
\pgfpathclose%
\pgfusepath{fill}%
\end{pgfscope}%
\begin{pgfscope}%
\pgfpathrectangle{\pgfqpoint{1.150000in}{0.150000in}}{\pgfqpoint{5.700000in}{5.700000in}}%
\pgfusepath{clip}%
\pgfsetbuttcap%
\pgfsetroundjoin%
\definecolor{currentfill}{rgb}{0.210503,0.363727,0.552206}%
\pgfsetfillcolor{currentfill}%
\pgfsetfillopacity{0.700000}%
\pgfsetlinewidth{0.000000pt}%
\definecolor{currentstroke}{rgb}{0.000000,0.000000,0.000000}%
\pgfsetstrokecolor{currentstroke}%
\pgfsetdash{}{0pt}%
\pgfpathmoveto{\pgfqpoint{5.171265in}{3.048847in}}%
\pgfpathlineto{\pgfqpoint{5.184935in}{3.043332in}}%
\pgfpathlineto{\pgfqpoint{5.198612in}{3.037885in}}%
\pgfpathlineto{\pgfqpoint{5.212295in}{3.032508in}}%
\pgfpathlineto{\pgfqpoint{5.225985in}{3.027199in}}%
\pgfpathlineto{\pgfqpoint{5.233590in}{3.046166in}}%
\pgfpathlineto{\pgfqpoint{5.241202in}{3.065557in}}%
\pgfpathlineto{\pgfqpoint{5.248821in}{3.085382in}}%
\pgfpathlineto{\pgfqpoint{5.256449in}{3.105651in}}%
\pgfpathlineto{\pgfqpoint{5.242772in}{3.111478in}}%
\pgfpathlineto{\pgfqpoint{5.229102in}{3.117374in}}%
\pgfpathlineto{\pgfqpoint{5.215438in}{3.123338in}}%
\pgfpathlineto{\pgfqpoint{5.201781in}{3.129372in}}%
\pgfpathlineto{\pgfqpoint{5.194141in}{3.108577in}}%
\pgfpathlineto{\pgfqpoint{5.186509in}{3.088231in}}%
\pgfpathlineto{\pgfqpoint{5.178883in}{3.068325in}}%
\pgfpathlineto{\pgfqpoint{5.171265in}{3.048847in}}%
\pgfpathclose%
\pgfusepath{fill}%
\end{pgfscope}%
\begin{pgfscope}%
\pgfpathrectangle{\pgfqpoint{1.150000in}{0.150000in}}{\pgfqpoint{5.700000in}{5.700000in}}%
\pgfusepath{clip}%
\pgfsetbuttcap%
\pgfsetroundjoin%
\definecolor{currentfill}{rgb}{0.282623,0.140926,0.457517}%
\pgfsetfillcolor{currentfill}%
\pgfsetfillopacity{0.700000}%
\pgfsetlinewidth{0.000000pt}%
\definecolor{currentstroke}{rgb}{0.000000,0.000000,0.000000}%
\pgfsetstrokecolor{currentstroke}%
\pgfsetdash{}{0pt}%
\pgfpathmoveto{\pgfqpoint{3.234043in}{2.581631in}}%
\pgfpathlineto{\pgfqpoint{3.247389in}{2.573801in}}%
\pgfpathlineto{\pgfqpoint{3.260738in}{2.566076in}}%
\pgfpathlineto{\pgfqpoint{3.274089in}{2.558458in}}%
\pgfpathlineto{\pgfqpoint{3.287443in}{2.550945in}}%
\pgfpathlineto{\pgfqpoint{3.295463in}{2.561655in}}%
\pgfpathlineto{\pgfqpoint{3.303477in}{2.572460in}}%
\pgfpathlineto{\pgfqpoint{3.311485in}{2.583363in}}%
\pgfpathlineto{\pgfqpoint{3.319486in}{2.594367in}}%
\pgfpathlineto{\pgfqpoint{3.306142in}{2.602016in}}%
\pgfpathlineto{\pgfqpoint{3.292800in}{2.609770in}}%
\pgfpathlineto{\pgfqpoint{3.279460in}{2.617630in}}%
\pgfpathlineto{\pgfqpoint{3.266123in}{2.625597in}}%
\pgfpathlineto{\pgfqpoint{3.258112in}{2.614450in}}%
\pgfpathlineto{\pgfqpoint{3.250096in}{2.603409in}}%
\pgfpathlineto{\pgfqpoint{3.242072in}{2.592470in}}%
\pgfpathlineto{\pgfqpoint{3.234043in}{2.581631in}}%
\pgfpathclose%
\pgfusepath{fill}%
\end{pgfscope}%
\begin{pgfscope}%
\pgfpathrectangle{\pgfqpoint{1.150000in}{0.150000in}}{\pgfqpoint{5.700000in}{5.700000in}}%
\pgfusepath{clip}%
\pgfsetbuttcap%
\pgfsetroundjoin%
\definecolor{currentfill}{rgb}{0.283072,0.130895,0.449241}%
\pgfsetfillcolor{currentfill}%
\pgfsetfillopacity{0.700000}%
\pgfsetlinewidth{0.000000pt}%
\definecolor{currentstroke}{rgb}{0.000000,0.000000,0.000000}%
\pgfsetstrokecolor{currentstroke}%
\pgfsetdash{}{0pt}%
\pgfpathmoveto{\pgfqpoint{3.372890in}{2.564806in}}%
\pgfpathlineto{\pgfqpoint{3.386248in}{2.557671in}}%
\pgfpathlineto{\pgfqpoint{3.399608in}{2.550638in}}%
\pgfpathlineto{\pgfqpoint{3.412972in}{2.543704in}}%
\pgfpathlineto{\pgfqpoint{3.426339in}{2.536870in}}%
\pgfpathlineto{\pgfqpoint{3.434316in}{2.547680in}}%
\pgfpathlineto{\pgfqpoint{3.442287in}{2.558587in}}%
\pgfpathlineto{\pgfqpoint{3.450252in}{2.569592in}}%
\pgfpathlineto{\pgfqpoint{3.458210in}{2.580701in}}%
\pgfpathlineto{\pgfqpoint{3.444852in}{2.587690in}}%
\pgfpathlineto{\pgfqpoint{3.431497in}{2.594780in}}%
\pgfpathlineto{\pgfqpoint{3.418146in}{2.601969in}}%
\pgfpathlineto{\pgfqpoint{3.404797in}{2.609260in}}%
\pgfpathlineto{\pgfqpoint{3.396829in}{2.597988in}}%
\pgfpathlineto{\pgfqpoint{3.388856in}{2.586824in}}%
\pgfpathlineto{\pgfqpoint{3.380876in}{2.575764in}}%
\pgfpathlineto{\pgfqpoint{3.372890in}{2.564806in}}%
\pgfpathclose%
\pgfusepath{fill}%
\end{pgfscope}%
\begin{pgfscope}%
\pgfpathrectangle{\pgfqpoint{1.150000in}{0.150000in}}{\pgfqpoint{5.700000in}{5.700000in}}%
\pgfusepath{clip}%
\pgfsetbuttcap%
\pgfsetroundjoin%
\definecolor{currentfill}{rgb}{0.279574,0.170599,0.479997}%
\pgfsetfillcolor{currentfill}%
\pgfsetfillopacity{0.700000}%
\pgfsetlinewidth{0.000000pt}%
\definecolor{currentstroke}{rgb}{0.000000,0.000000,0.000000}%
\pgfsetstrokecolor{currentstroke}%
\pgfsetdash{}{0pt}%
\pgfpathmoveto{\pgfqpoint{4.183178in}{2.629260in}}%
\pgfpathlineto{\pgfqpoint{4.196669in}{2.624553in}}%
\pgfpathlineto{\pgfqpoint{4.210166in}{2.619927in}}%
\pgfpathlineto{\pgfqpoint{4.223668in}{2.615381in}}%
\pgfpathlineto{\pgfqpoint{4.237177in}{2.610914in}}%
\pgfpathlineto{\pgfqpoint{4.244910in}{2.622675in}}%
\pgfpathlineto{\pgfqpoint{4.252639in}{2.634594in}}%
\pgfpathlineto{\pgfqpoint{4.260364in}{2.646678in}}%
\pgfpathlineto{\pgfqpoint{4.268086in}{2.658932in}}%
\pgfpathlineto{\pgfqpoint{4.254587in}{2.663696in}}%
\pgfpathlineto{\pgfqpoint{4.241095in}{2.668538in}}%
\pgfpathlineto{\pgfqpoint{4.227607in}{2.673461in}}%
\pgfpathlineto{\pgfqpoint{4.214126in}{2.678464in}}%
\pgfpathlineto{\pgfqpoint{4.206395in}{2.665906in}}%
\pgfpathlineto{\pgfqpoint{4.198660in}{2.653524in}}%
\pgfpathlineto{\pgfqpoint{4.190921in}{2.641310in}}%
\pgfpathlineto{\pgfqpoint{4.183178in}{2.629260in}}%
\pgfpathclose%
\pgfusepath{fill}%
\end{pgfscope}%
\begin{pgfscope}%
\pgfpathrectangle{\pgfqpoint{1.150000in}{0.150000in}}{\pgfqpoint{5.700000in}{5.700000in}}%
\pgfusepath{clip}%
\pgfsetbuttcap%
\pgfsetroundjoin%
\definecolor{currentfill}{rgb}{0.281887,0.150881,0.465405}%
\pgfsetfillcolor{currentfill}%
\pgfsetfillopacity{0.700000}%
\pgfsetlinewidth{0.000000pt}%
\definecolor{currentstroke}{rgb}{0.000000,0.000000,0.000000}%
\pgfsetstrokecolor{currentstroke}%
\pgfsetdash{}{0pt}%
\pgfpathmoveto{\pgfqpoint{3.095075in}{2.604874in}}%
\pgfpathlineto{\pgfqpoint{3.108417in}{2.596278in}}%
\pgfpathlineto{\pgfqpoint{3.121761in}{2.587796in}}%
\pgfpathlineto{\pgfqpoint{3.135106in}{2.579425in}}%
\pgfpathlineto{\pgfqpoint{3.148453in}{2.571166in}}%
\pgfpathlineto{\pgfqpoint{3.156519in}{2.581738in}}%
\pgfpathlineto{\pgfqpoint{3.164578in}{2.592406in}}%
\pgfpathlineto{\pgfqpoint{3.172630in}{2.603170in}}%
\pgfpathlineto{\pgfqpoint{3.180676in}{2.614035in}}%
\pgfpathlineto{\pgfqpoint{3.167339in}{2.622410in}}%
\pgfpathlineto{\pgfqpoint{3.154004in}{2.630896in}}%
\pgfpathlineto{\pgfqpoint{3.140670in}{2.639494in}}%
\pgfpathlineto{\pgfqpoint{3.127338in}{2.648205in}}%
\pgfpathlineto{\pgfqpoint{3.119282in}{2.637217in}}%
\pgfpathlineto{\pgfqpoint{3.111220in}{2.626334in}}%
\pgfpathlineto{\pgfqpoint{3.103151in}{2.615554in}}%
\pgfpathlineto{\pgfqpoint{3.095075in}{2.604874in}}%
\pgfpathclose%
\pgfusepath{fill}%
\end{pgfscope}%
\begin{pgfscope}%
\pgfpathrectangle{\pgfqpoint{1.150000in}{0.150000in}}{\pgfqpoint{5.700000in}{5.700000in}}%
\pgfusepath{clip}%
\pgfsetbuttcap%
\pgfsetroundjoin%
\definecolor{currentfill}{rgb}{0.271828,0.209303,0.504434}%
\pgfsetfillcolor{currentfill}%
\pgfsetfillopacity{0.700000}%
\pgfsetlinewidth{0.000000pt}%
\definecolor{currentstroke}{rgb}{0.000000,0.000000,0.000000}%
\pgfsetstrokecolor{currentstroke}%
\pgfsetdash{}{0pt}%
\pgfpathmoveto{\pgfqpoint{4.491948in}{2.704863in}}%
\pgfpathlineto{\pgfqpoint{4.505504in}{2.700454in}}%
\pgfpathlineto{\pgfqpoint{4.519067in}{2.696120in}}%
\pgfpathlineto{\pgfqpoint{4.532637in}{2.691862in}}%
\pgfpathlineto{\pgfqpoint{4.546213in}{2.687679in}}%
\pgfpathlineto{\pgfqpoint{4.553866in}{2.700327in}}%
\pgfpathlineto{\pgfqpoint{4.561516in}{2.713185in}}%
\pgfpathlineto{\pgfqpoint{4.569165in}{2.726260in}}%
\pgfpathlineto{\pgfqpoint{4.576812in}{2.739557in}}%
\pgfpathlineto{\pgfqpoint{4.563247in}{2.744096in}}%
\pgfpathlineto{\pgfqpoint{4.549689in}{2.748711in}}%
\pgfpathlineto{\pgfqpoint{4.536136in}{2.753401in}}%
\pgfpathlineto{\pgfqpoint{4.522590in}{2.758167in}}%
\pgfpathlineto{\pgfqpoint{4.514933in}{2.744506in}}%
\pgfpathlineto{\pgfqpoint{4.507273in}{2.731072in}}%
\pgfpathlineto{\pgfqpoint{4.499611in}{2.717860in}}%
\pgfpathlineto{\pgfqpoint{4.491948in}{2.704863in}}%
\pgfpathclose%
\pgfusepath{fill}%
\end{pgfscope}%
\begin{pgfscope}%
\pgfpathrectangle{\pgfqpoint{1.150000in}{0.150000in}}{\pgfqpoint{5.700000in}{5.700000in}}%
\pgfusepath{clip}%
\pgfsetbuttcap%
\pgfsetroundjoin%
\definecolor{currentfill}{rgb}{0.283072,0.130895,0.449241}%
\pgfsetfillcolor{currentfill}%
\pgfsetfillopacity{0.700000}%
\pgfsetlinewidth{0.000000pt}%
\definecolor{currentstroke}{rgb}{0.000000,0.000000,0.000000}%
\pgfsetstrokecolor{currentstroke}%
\pgfsetdash{}{0pt}%
\pgfpathmoveto{\pgfqpoint{3.511675in}{2.553725in}}%
\pgfpathlineto{\pgfqpoint{3.525049in}{2.547224in}}%
\pgfpathlineto{\pgfqpoint{3.538428in}{2.540819in}}%
\pgfpathlineto{\pgfqpoint{3.551809in}{2.534510in}}%
\pgfpathlineto{\pgfqpoint{3.565195in}{2.528295in}}%
\pgfpathlineto{\pgfqpoint{3.573130in}{2.539172in}}%
\pgfpathlineto{\pgfqpoint{3.581060in}{2.550148in}}%
\pgfpathlineto{\pgfqpoint{3.588983in}{2.561226in}}%
\pgfpathlineto{\pgfqpoint{3.596901in}{2.572410in}}%
\pgfpathlineto{\pgfqpoint{3.583524in}{2.578801in}}%
\pgfpathlineto{\pgfqpoint{3.570151in}{2.585287in}}%
\pgfpathlineto{\pgfqpoint{3.556781in}{2.591868in}}%
\pgfpathlineto{\pgfqpoint{3.543415in}{2.598545in}}%
\pgfpathlineto{\pgfqpoint{3.535489in}{2.587177in}}%
\pgfpathlineto{\pgfqpoint{3.527557in}{2.575920in}}%
\pgfpathlineto{\pgfqpoint{3.519619in}{2.564771in}}%
\pgfpathlineto{\pgfqpoint{3.511675in}{2.553725in}}%
\pgfpathclose%
\pgfusepath{fill}%
\end{pgfscope}%
\begin{pgfscope}%
\pgfpathrectangle{\pgfqpoint{1.150000in}{0.150000in}}{\pgfqpoint{5.700000in}{5.700000in}}%
\pgfusepath{clip}%
\pgfsetbuttcap%
\pgfsetroundjoin%
\definecolor{currentfill}{rgb}{0.199430,0.387607,0.554642}%
\pgfsetfillcolor{currentfill}%
\pgfsetfillopacity{0.700000}%
\pgfsetlinewidth{0.000000pt}%
\definecolor{currentstroke}{rgb}{0.000000,0.000000,0.000000}%
\pgfsetstrokecolor{currentstroke}%
\pgfsetdash{}{0pt}%
\pgfpathmoveto{\pgfqpoint{5.256449in}{3.105651in}}%
\pgfpathlineto{\pgfqpoint{5.270132in}{3.099893in}}%
\pgfpathlineto{\pgfqpoint{5.283822in}{3.094203in}}%
\pgfpathlineto{\pgfqpoint{5.297518in}{3.088581in}}%
\pgfpathlineto{\pgfqpoint{5.311221in}{3.083028in}}%
\pgfpathlineto{\pgfqpoint{5.318844in}{3.103220in}}%
\pgfpathlineto{\pgfqpoint{5.326475in}{3.123870in}}%
\pgfpathlineto{\pgfqpoint{5.334115in}{3.144989in}}%
\pgfpathlineto{\pgfqpoint{5.341766in}{3.166586in}}%
\pgfpathlineto{\pgfqpoint{5.328076in}{3.172679in}}%
\pgfpathlineto{\pgfqpoint{5.314393in}{3.178839in}}%
\pgfpathlineto{\pgfqpoint{5.300716in}{3.185068in}}%
\pgfpathlineto{\pgfqpoint{5.287046in}{3.191365in}}%
\pgfpathlineto{\pgfqpoint{5.279383in}{3.169221in}}%
\pgfpathlineto{\pgfqpoint{5.271729in}{3.147560in}}%
\pgfpathlineto{\pgfqpoint{5.264085in}{3.126374in}}%
\pgfpathlineto{\pgfqpoint{5.256449in}{3.105651in}}%
\pgfpathclose%
\pgfusepath{fill}%
\end{pgfscope}%
\begin{pgfscope}%
\pgfpathrectangle{\pgfqpoint{1.150000in}{0.150000in}}{\pgfqpoint{5.700000in}{5.700000in}}%
\pgfusepath{clip}%
\pgfsetbuttcap%
\pgfsetroundjoin%
\definecolor{currentfill}{rgb}{0.282623,0.140926,0.457517}%
\pgfsetfillcolor{currentfill}%
\pgfsetfillopacity{0.700000}%
\pgfsetlinewidth{0.000000pt}%
\definecolor{currentstroke}{rgb}{0.000000,0.000000,0.000000}%
\pgfsetstrokecolor{currentstroke}%
\pgfsetdash{}{0pt}%
\pgfpathmoveto{\pgfqpoint{3.874361in}{2.569762in}}%
\pgfpathlineto{\pgfqpoint{3.887795in}{2.564493in}}%
\pgfpathlineto{\pgfqpoint{3.901233in}{2.559311in}}%
\pgfpathlineto{\pgfqpoint{3.914676in}{2.554214in}}%
\pgfpathlineto{\pgfqpoint{3.928125in}{2.549202in}}%
\pgfpathlineto{\pgfqpoint{3.935950in}{2.560366in}}%
\pgfpathlineto{\pgfqpoint{3.943770in}{2.571650in}}%
\pgfpathlineto{\pgfqpoint{3.951586in}{2.583059in}}%
\pgfpathlineto{\pgfqpoint{3.959396in}{2.594597in}}%
\pgfpathlineto{\pgfqpoint{3.945957in}{2.599845in}}%
\pgfpathlineto{\pgfqpoint{3.932522in}{2.605178in}}%
\pgfpathlineto{\pgfqpoint{3.919093in}{2.610597in}}%
\pgfpathlineto{\pgfqpoint{3.905668in}{2.616102in}}%
\pgfpathlineto{\pgfqpoint{3.897849in}{2.604321in}}%
\pgfpathlineto{\pgfqpoint{3.890025in}{2.592673in}}%
\pgfpathlineto{\pgfqpoint{3.882196in}{2.581155in}}%
\pgfpathlineto{\pgfqpoint{3.874361in}{2.569762in}}%
\pgfpathclose%
\pgfusepath{fill}%
\end{pgfscope}%
\begin{pgfscope}%
\pgfpathrectangle{\pgfqpoint{1.150000in}{0.150000in}}{\pgfqpoint{5.700000in}{5.700000in}}%
\pgfusepath{clip}%
\pgfsetbuttcap%
\pgfsetroundjoin%
\definecolor{currentfill}{rgb}{0.273006,0.204520,0.501721}%
\pgfsetfillcolor{currentfill}%
\pgfsetfillopacity{0.700000}%
\pgfsetlinewidth{0.000000pt}%
\definecolor{currentstroke}{rgb}{0.000000,0.000000,0.000000}%
\pgfsetstrokecolor{currentstroke}%
\pgfsetdash{}{0pt}%
\pgfpathmoveto{\pgfqpoint{2.763076in}{2.716417in}}%
\pgfpathlineto{\pgfqpoint{2.776435in}{2.705533in}}%
\pgfpathlineto{\pgfqpoint{2.789793in}{2.694782in}}%
\pgfpathlineto{\pgfqpoint{2.803150in}{2.684163in}}%
\pgfpathlineto{\pgfqpoint{2.816507in}{2.673674in}}%
\pgfpathlineto{\pgfqpoint{2.824684in}{2.683900in}}%
\pgfpathlineto{\pgfqpoint{2.832854in}{2.694231in}}%
\pgfpathlineto{\pgfqpoint{2.841016in}{2.704667in}}%
\pgfpathlineto{\pgfqpoint{2.849170in}{2.715211in}}%
\pgfpathlineto{\pgfqpoint{2.835825in}{2.725774in}}%
\pgfpathlineto{\pgfqpoint{2.822479in}{2.736469in}}%
\pgfpathlineto{\pgfqpoint{2.809132in}{2.747295in}}%
\pgfpathlineto{\pgfqpoint{2.795785in}{2.758254in}}%
\pgfpathlineto{\pgfqpoint{2.787620in}{2.747627in}}%
\pgfpathlineto{\pgfqpoint{2.779447in}{2.737114in}}%
\pgfpathlineto{\pgfqpoint{2.771265in}{2.726711in}}%
\pgfpathlineto{\pgfqpoint{2.763076in}{2.716417in}}%
\pgfpathclose%
\pgfusepath{fill}%
\end{pgfscope}%
\begin{pgfscope}%
\pgfpathrectangle{\pgfqpoint{1.150000in}{0.150000in}}{\pgfqpoint{5.700000in}{5.700000in}}%
\pgfusepath{clip}%
\pgfsetbuttcap%
\pgfsetroundjoin%
\definecolor{currentfill}{rgb}{0.174274,0.445044,0.557792}%
\pgfsetfillcolor{currentfill}%
\pgfsetfillopacity{0.700000}%
\pgfsetlinewidth{0.000000pt}%
\definecolor{currentstroke}{rgb}{0.000000,0.000000,0.000000}%
\pgfsetstrokecolor{currentstroke}%
\pgfsetdash{}{0pt}%
\pgfpathmoveto{\pgfqpoint{5.372474in}{3.257975in}}%
\pgfpathlineto{\pgfqpoint{5.386157in}{3.251391in}}%
\pgfpathlineto{\pgfqpoint{5.399847in}{3.244875in}}%
\pgfpathlineto{\pgfqpoint{5.413543in}{3.238426in}}%
\pgfpathlineto{\pgfqpoint{5.427245in}{3.232045in}}%
\pgfpathlineto{\pgfqpoint{5.434939in}{3.255622in}}%
\pgfpathlineto{\pgfqpoint{5.442645in}{3.279737in}}%
\pgfpathlineto{\pgfqpoint{5.450364in}{3.304401in}}%
\pgfpathlineto{\pgfqpoint{5.436671in}{3.311216in}}%
\pgfpathlineto{\pgfqpoint{5.422985in}{3.318098in}}%
\pgfpathlineto{\pgfqpoint{5.409304in}{3.325048in}}%
\pgfpathlineto{\pgfqpoint{5.395630in}{3.332067in}}%
\pgfpathlineto{\pgfqpoint{5.387898in}{3.306819in}}%
\pgfpathlineto{\pgfqpoint{5.380180in}{3.282125in}}%
\pgfpathlineto{\pgfqpoint{5.372474in}{3.257975in}}%
\pgfpathclose%
\pgfusepath{fill}%
\end{pgfscope}%
\begin{pgfscope}%
\pgfpathrectangle{\pgfqpoint{1.150000in}{0.150000in}}{\pgfqpoint{5.700000in}{5.700000in}}%
\pgfusepath{clip}%
\pgfsetbuttcap%
\pgfsetroundjoin%
\definecolor{currentfill}{rgb}{0.280255,0.165693,0.476498}%
\pgfsetfillcolor{currentfill}%
\pgfsetfillopacity{0.700000}%
\pgfsetlinewidth{0.000000pt}%
\definecolor{currentstroke}{rgb}{0.000000,0.000000,0.000000}%
\pgfsetstrokecolor{currentstroke}%
\pgfsetdash{}{0pt}%
\pgfpathmoveto{\pgfqpoint{2.955922in}{2.635279in}}%
\pgfpathlineto{\pgfqpoint{2.969267in}{2.625843in}}%
\pgfpathlineto{\pgfqpoint{2.982612in}{2.616528in}}%
\pgfpathlineto{\pgfqpoint{2.995958in}{2.607332in}}%
\pgfpathlineto{\pgfqpoint{3.009305in}{2.598254in}}%
\pgfpathlineto{\pgfqpoint{3.017419in}{2.608645in}}%
\pgfpathlineto{\pgfqpoint{3.025527in}{2.619131in}}%
\pgfpathlineto{\pgfqpoint{3.033627in}{2.629716in}}%
\pgfpathlineto{\pgfqpoint{3.041719in}{2.640402in}}%
\pgfpathlineto{\pgfqpoint{3.028383in}{2.649575in}}%
\pgfpathlineto{\pgfqpoint{3.015047in}{2.658866in}}%
\pgfpathlineto{\pgfqpoint{3.001712in}{2.668277in}}%
\pgfpathlineto{\pgfqpoint{2.988378in}{2.677808in}}%
\pgfpathlineto{\pgfqpoint{2.980275in}{2.667019in}}%
\pgfpathlineto{\pgfqpoint{2.972165in}{2.656336in}}%
\pgfpathlineto{\pgfqpoint{2.964047in}{2.645757in}}%
\pgfpathlineto{\pgfqpoint{2.955922in}{2.635279in}}%
\pgfpathclose%
\pgfusepath{fill}%
\end{pgfscope}%
\begin{pgfscope}%
\pgfpathrectangle{\pgfqpoint{1.150000in}{0.150000in}}{\pgfqpoint{5.700000in}{5.700000in}}%
\pgfusepath{clip}%
\pgfsetbuttcap%
\pgfsetroundjoin%
\definecolor{currentfill}{rgb}{0.275191,0.194905,0.496005}%
\pgfsetfillcolor{currentfill}%
\pgfsetfillopacity{0.700000}%
\pgfsetlinewidth{0.000000pt}%
\definecolor{currentstroke}{rgb}{0.000000,0.000000,0.000000}%
\pgfsetstrokecolor{currentstroke}%
\pgfsetdash{}{0pt}%
\pgfpathmoveto{\pgfqpoint{4.407059in}{2.671946in}}%
\pgfpathlineto{\pgfqpoint{4.420602in}{2.667569in}}%
\pgfpathlineto{\pgfqpoint{4.434150in}{2.663268in}}%
\pgfpathlineto{\pgfqpoint{4.447705in}{2.659044in}}%
\pgfpathlineto{\pgfqpoint{4.461267in}{2.654896in}}%
\pgfpathlineto{\pgfqpoint{4.468941in}{2.667097in}}%
\pgfpathlineto{\pgfqpoint{4.476612in}{2.679487in}}%
\pgfpathlineto{\pgfqpoint{4.484281in}{2.692074in}}%
\pgfpathlineto{\pgfqpoint{4.491948in}{2.704863in}}%
\pgfpathlineto{\pgfqpoint{4.478397in}{2.709348in}}%
\pgfpathlineto{\pgfqpoint{4.464853in}{2.713909in}}%
\pgfpathlineto{\pgfqpoint{4.451315in}{2.718546in}}%
\pgfpathlineto{\pgfqpoint{4.437783in}{2.723260in}}%
\pgfpathlineto{\pgfqpoint{4.430106in}{2.710127in}}%
\pgfpathlineto{\pgfqpoint{4.422427in}{2.697202in}}%
\pgfpathlineto{\pgfqpoint{4.414744in}{2.684477in}}%
\pgfpathlineto{\pgfqpoint{4.407059in}{2.671946in}}%
\pgfpathclose%
\pgfusepath{fill}%
\end{pgfscope}%
\begin{pgfscope}%
\pgfpathrectangle{\pgfqpoint{1.150000in}{0.150000in}}{\pgfqpoint{5.700000in}{5.700000in}}%
\pgfusepath{clip}%
\pgfsetbuttcap%
\pgfsetroundjoin%
\definecolor{currentfill}{rgb}{0.281412,0.155834,0.469201}%
\pgfsetfillcolor{currentfill}%
\pgfsetfillopacity{0.700000}%
\pgfsetlinewidth{0.000000pt}%
\definecolor{currentstroke}{rgb}{0.000000,0.000000,0.000000}%
\pgfsetstrokecolor{currentstroke}%
\pgfsetdash{}{0pt}%
\pgfpathmoveto{\pgfqpoint{4.098220in}{2.601121in}}%
\pgfpathlineto{\pgfqpoint{4.111698in}{2.596367in}}%
\pgfpathlineto{\pgfqpoint{4.125182in}{2.591694in}}%
\pgfpathlineto{\pgfqpoint{4.138671in}{2.587103in}}%
\pgfpathlineto{\pgfqpoint{4.152166in}{2.582593in}}%
\pgfpathlineto{\pgfqpoint{4.159925in}{2.594040in}}%
\pgfpathlineto{\pgfqpoint{4.167680in}{2.605630in}}%
\pgfpathlineto{\pgfqpoint{4.175431in}{2.617368in}}%
\pgfpathlineto{\pgfqpoint{4.183178in}{2.629260in}}%
\pgfpathlineto{\pgfqpoint{4.169693in}{2.634047in}}%
\pgfpathlineto{\pgfqpoint{4.156213in}{2.638915in}}%
\pgfpathlineto{\pgfqpoint{4.142739in}{2.643864in}}%
\pgfpathlineto{\pgfqpoint{4.129271in}{2.648894in}}%
\pgfpathlineto{\pgfqpoint{4.121515in}{2.636719in}}%
\pgfpathlineto{\pgfqpoint{4.113754in}{2.624702in}}%
\pgfpathlineto{\pgfqpoint{4.105989in}{2.612837in}}%
\pgfpathlineto{\pgfqpoint{4.098220in}{2.601121in}}%
\pgfpathclose%
\pgfusepath{fill}%
\end{pgfscope}%
\begin{pgfscope}%
\pgfpathrectangle{\pgfqpoint{1.150000in}{0.150000in}}{\pgfqpoint{5.700000in}{5.700000in}}%
\pgfusepath{clip}%
\pgfsetbuttcap%
\pgfsetroundjoin%
\definecolor{currentfill}{rgb}{0.283072,0.130895,0.449241}%
\pgfsetfillcolor{currentfill}%
\pgfsetfillopacity{0.700000}%
\pgfsetlinewidth{0.000000pt}%
\definecolor{currentstroke}{rgb}{0.000000,0.000000,0.000000}%
\pgfsetstrokecolor{currentstroke}%
\pgfsetdash{}{0pt}%
\pgfpathmoveto{\pgfqpoint{3.650447in}{2.547784in}}%
\pgfpathlineto{\pgfqpoint{3.663844in}{2.541859in}}%
\pgfpathlineto{\pgfqpoint{3.677245in}{2.536026in}}%
\pgfpathlineto{\pgfqpoint{3.690650in}{2.530285in}}%
\pgfpathlineto{\pgfqpoint{3.704059in}{2.524633in}}%
\pgfpathlineto{\pgfqpoint{3.711954in}{2.535550in}}%
\pgfpathlineto{\pgfqpoint{3.719843in}{2.546569in}}%
\pgfpathlineto{\pgfqpoint{3.727727in}{2.557695in}}%
\pgfpathlineto{\pgfqpoint{3.735605in}{2.568932in}}%
\pgfpathlineto{\pgfqpoint{3.722204in}{2.574780in}}%
\pgfpathlineto{\pgfqpoint{3.708808in}{2.580718in}}%
\pgfpathlineto{\pgfqpoint{3.695416in}{2.586747in}}%
\pgfpathlineto{\pgfqpoint{3.682028in}{2.592868in}}%
\pgfpathlineto{\pgfqpoint{3.674141in}{2.581427in}}%
\pgfpathlineto{\pgfqpoint{3.666249in}{2.570102in}}%
\pgfpathlineto{\pgfqpoint{3.658351in}{2.558889in}}%
\pgfpathlineto{\pgfqpoint{3.650447in}{2.547784in}}%
\pgfpathclose%
\pgfusepath{fill}%
\end{pgfscope}%
\begin{pgfscope}%
\pgfpathrectangle{\pgfqpoint{1.150000in}{0.150000in}}{\pgfqpoint{5.700000in}{5.700000in}}%
\pgfusepath{clip}%
\pgfsetbuttcap%
\pgfsetroundjoin%
\definecolor{currentfill}{rgb}{0.243113,0.292092,0.538516}%
\pgfsetfillcolor{currentfill}%
\pgfsetfillopacity{0.700000}%
\pgfsetlinewidth{0.000000pt}%
\definecolor{currentstroke}{rgb}{0.000000,0.000000,0.000000}%
\pgfsetstrokecolor{currentstroke}%
\pgfsetdash{}{0pt}%
\pgfpathmoveto{\pgfqpoint{4.970818in}{2.880428in}}%
\pgfpathlineto{\pgfqpoint{4.984473in}{2.875786in}}%
\pgfpathlineto{\pgfqpoint{4.998135in}{2.871214in}}%
\pgfpathlineto{\pgfqpoint{5.011805in}{2.866713in}}%
\pgfpathlineto{\pgfqpoint{5.025481in}{2.862281in}}%
\pgfpathlineto{\pgfqpoint{5.033057in}{2.877768in}}%
\pgfpathlineto{\pgfqpoint{5.040637in}{2.893581in}}%
\pgfpathlineto{\pgfqpoint{5.048219in}{2.909731in}}%
\pgfpathlineto{\pgfqpoint{5.055805in}{2.926224in}}%
\pgfpathlineto{\pgfqpoint{5.042142in}{2.931112in}}%
\pgfpathlineto{\pgfqpoint{5.028486in}{2.936071in}}%
\pgfpathlineto{\pgfqpoint{5.014837in}{2.941099in}}%
\pgfpathlineto{\pgfqpoint{5.001195in}{2.946198in}}%
\pgfpathlineto{\pgfqpoint{4.993596in}{2.929241in}}%
\pgfpathlineto{\pgfqpoint{4.986000in}{2.912632in}}%
\pgfpathlineto{\pgfqpoint{4.978408in}{2.896364in}}%
\pgfpathlineto{\pgfqpoint{4.970818in}{2.880428in}}%
\pgfpathclose%
\pgfusepath{fill}%
\end{pgfscope}%
\begin{pgfscope}%
\pgfpathrectangle{\pgfqpoint{1.150000in}{0.150000in}}{\pgfqpoint{5.700000in}{5.700000in}}%
\pgfusepath{clip}%
\pgfsetbuttcap%
\pgfsetroundjoin%
\definecolor{currentfill}{rgb}{0.250425,0.274290,0.533103}%
\pgfsetfillcolor{currentfill}%
\pgfsetfillopacity{0.700000}%
\pgfsetlinewidth{0.000000pt}%
\definecolor{currentstroke}{rgb}{0.000000,0.000000,0.000000}%
\pgfsetstrokecolor{currentstroke}%
\pgfsetdash{}{0pt}%
\pgfpathmoveto{\pgfqpoint{4.885873in}{2.837370in}}%
\pgfpathlineto{\pgfqpoint{4.899514in}{2.832882in}}%
\pgfpathlineto{\pgfqpoint{4.913162in}{2.828464in}}%
\pgfpathlineto{\pgfqpoint{4.926817in}{2.824117in}}%
\pgfpathlineto{\pgfqpoint{4.940479in}{2.819842in}}%
\pgfpathlineto{\pgfqpoint{4.948061in}{2.834531in}}%
\pgfpathlineto{\pgfqpoint{4.955644in}{2.849519in}}%
\pgfpathlineto{\pgfqpoint{4.963230in}{2.864816in}}%
\pgfpathlineto{\pgfqpoint{4.970818in}{2.880428in}}%
\pgfpathlineto{\pgfqpoint{4.957169in}{2.885141in}}%
\pgfpathlineto{\pgfqpoint{4.943527in}{2.889924in}}%
\pgfpathlineto{\pgfqpoint{4.929892in}{2.894778in}}%
\pgfpathlineto{\pgfqpoint{4.916263in}{2.899704in}}%
\pgfpathlineto{\pgfqpoint{4.908663in}{2.883647in}}%
\pgfpathlineto{\pgfqpoint{4.901064in}{2.867912in}}%
\pgfpathlineto{\pgfqpoint{4.893468in}{2.852489in}}%
\pgfpathlineto{\pgfqpoint{4.885873in}{2.837370in}}%
\pgfpathclose%
\pgfusepath{fill}%
\end{pgfscope}%
\begin{pgfscope}%
\pgfpathrectangle{\pgfqpoint{1.150000in}{0.150000in}}{\pgfqpoint{5.700000in}{5.700000in}}%
\pgfusepath{clip}%
\pgfsetbuttcap%
\pgfsetroundjoin%
\definecolor{currentfill}{rgb}{0.233603,0.313828,0.543914}%
\pgfsetfillcolor{currentfill}%
\pgfsetfillopacity{0.700000}%
\pgfsetlinewidth{0.000000pt}%
\definecolor{currentstroke}{rgb}{0.000000,0.000000,0.000000}%
\pgfsetstrokecolor{currentstroke}%
\pgfsetdash{}{0pt}%
\pgfpathmoveto{\pgfqpoint{5.055805in}{2.926224in}}%
\pgfpathlineto{\pgfqpoint{5.069475in}{2.921405in}}%
\pgfpathlineto{\pgfqpoint{5.083151in}{2.916656in}}%
\pgfpathlineto{\pgfqpoint{5.096834in}{2.911977in}}%
\pgfpathlineto{\pgfqpoint{5.110525in}{2.907367in}}%
\pgfpathlineto{\pgfqpoint{5.118100in}{2.923743in}}%
\pgfpathlineto{\pgfqpoint{5.125680in}{2.940475in}}%
\pgfpathlineto{\pgfqpoint{5.133265in}{2.957573in}}%
\pgfpathlineto{\pgfqpoint{5.140854in}{2.975044in}}%
\pgfpathlineto{\pgfqpoint{5.127177in}{2.980131in}}%
\pgfpathlineto{\pgfqpoint{5.113508in}{2.985287in}}%
\pgfpathlineto{\pgfqpoint{5.099845in}{2.990513in}}%
\pgfpathlineto{\pgfqpoint{5.086188in}{2.995809in}}%
\pgfpathlineto{\pgfqpoint{5.078586in}{2.977853in}}%
\pgfpathlineto{\pgfqpoint{5.070988in}{2.960276in}}%
\pgfpathlineto{\pgfqpoint{5.063395in}{2.943069in}}%
\pgfpathlineto{\pgfqpoint{5.055805in}{2.926224in}}%
\pgfpathclose%
\pgfusepath{fill}%
\end{pgfscope}%
\begin{pgfscope}%
\pgfpathrectangle{\pgfqpoint{1.150000in}{0.150000in}}{\pgfqpoint{5.700000in}{5.700000in}}%
\pgfusepath{clip}%
\pgfsetbuttcap%
\pgfsetroundjoin%
\definecolor{currentfill}{rgb}{0.257322,0.256130,0.526563}%
\pgfsetfillcolor{currentfill}%
\pgfsetfillopacity{0.700000}%
\pgfsetlinewidth{0.000000pt}%
\definecolor{currentstroke}{rgb}{0.000000,0.000000,0.000000}%
\pgfsetstrokecolor{currentstroke}%
\pgfsetdash{}{0pt}%
\pgfpathmoveto{\pgfqpoint{4.800953in}{2.796790in}}%
\pgfpathlineto{\pgfqpoint{4.814580in}{2.792431in}}%
\pgfpathlineto{\pgfqpoint{4.828214in}{2.788145in}}%
\pgfpathlineto{\pgfqpoint{4.841854in}{2.783930in}}%
\pgfpathlineto{\pgfqpoint{4.855501in}{2.779787in}}%
\pgfpathlineto{\pgfqpoint{4.863093in}{2.793765in}}%
\pgfpathlineto{\pgfqpoint{4.870685in}{2.808016in}}%
\pgfpathlineto{\pgfqpoint{4.878279in}{2.822549in}}%
\pgfpathlineto{\pgfqpoint{4.885873in}{2.837370in}}%
\pgfpathlineto{\pgfqpoint{4.872238in}{2.841930in}}%
\pgfpathlineto{\pgfqpoint{4.858610in}{2.846562in}}%
\pgfpathlineto{\pgfqpoint{4.844989in}{2.851265in}}%
\pgfpathlineto{\pgfqpoint{4.831375in}{2.856040in}}%
\pgfpathlineto{\pgfqpoint{4.823769in}{2.840794in}}%
\pgfpathlineto{\pgfqpoint{4.816163in}{2.825842in}}%
\pgfpathlineto{\pgfqpoint{4.808558in}{2.811177in}}%
\pgfpathlineto{\pgfqpoint{4.800953in}{2.796790in}}%
\pgfpathclose%
\pgfusepath{fill}%
\end{pgfscope}%
\begin{pgfscope}%
\pgfpathrectangle{\pgfqpoint{1.150000in}{0.150000in}}{\pgfqpoint{5.700000in}{5.700000in}}%
\pgfusepath{clip}%
\pgfsetbuttcap%
\pgfsetroundjoin%
\definecolor{currentfill}{rgb}{0.187231,0.414746,0.556547}%
\pgfsetfillcolor{currentfill}%
\pgfsetfillopacity{0.700000}%
\pgfsetlinewidth{0.000000pt}%
\definecolor{currentstroke}{rgb}{0.000000,0.000000,0.000000}%
\pgfsetstrokecolor{currentstroke}%
\pgfsetdash{}{0pt}%
\pgfpathmoveto{\pgfqpoint{5.341766in}{3.166586in}}%
\pgfpathlineto{\pgfqpoint{5.355462in}{3.160562in}}%
\pgfpathlineto{\pgfqpoint{5.369165in}{3.154606in}}%
\pgfpathlineto{\pgfqpoint{5.382874in}{3.148717in}}%
\pgfpathlineto{\pgfqpoint{5.396590in}{3.142896in}}%
\pgfpathlineto{\pgfqpoint{5.404237in}{3.164431in}}%
\pgfpathlineto{\pgfqpoint{5.411895in}{3.186460in}}%
\pgfpathlineto{\pgfqpoint{5.419564in}{3.208995in}}%
\pgfpathlineto{\pgfqpoint{5.427245in}{3.232045in}}%
\pgfpathlineto{\pgfqpoint{5.413543in}{3.238426in}}%
\pgfpathlineto{\pgfqpoint{5.399847in}{3.244875in}}%
\pgfpathlineto{\pgfqpoint{5.386157in}{3.251391in}}%
\pgfpathlineto{\pgfqpoint{5.372474in}{3.257975in}}%
\pgfpathlineto{\pgfqpoint{5.364780in}{3.234357in}}%
\pgfpathlineto{\pgfqpoint{5.357097in}{3.211260in}}%
\pgfpathlineto{\pgfqpoint{5.349426in}{3.188673in}}%
\pgfpathlineto{\pgfqpoint{5.341766in}{3.166586in}}%
\pgfpathclose%
\pgfusepath{fill}%
\end{pgfscope}%
\begin{pgfscope}%
\pgfpathrectangle{\pgfqpoint{1.150000in}{0.150000in}}{\pgfqpoint{5.700000in}{5.700000in}}%
\pgfusepath{clip}%
\pgfsetbuttcap%
\pgfsetroundjoin%
\definecolor{currentfill}{rgb}{0.223925,0.334994,0.548053}%
\pgfsetfillcolor{currentfill}%
\pgfsetfillopacity{0.700000}%
\pgfsetlinewidth{0.000000pt}%
\definecolor{currentstroke}{rgb}{0.000000,0.000000,0.000000}%
\pgfsetstrokecolor{currentstroke}%
\pgfsetdash{}{0pt}%
\pgfpathmoveto{\pgfqpoint{5.140854in}{2.975044in}}%
\pgfpathlineto{\pgfqpoint{5.154537in}{2.970026in}}%
\pgfpathlineto{\pgfqpoint{5.168227in}{2.965077in}}%
\pgfpathlineto{\pgfqpoint{5.181924in}{2.960196in}}%
\pgfpathlineto{\pgfqpoint{5.195628in}{2.955385in}}%
\pgfpathlineto{\pgfqpoint{5.203209in}{2.972749in}}%
\pgfpathlineto{\pgfqpoint{5.210795in}{2.990500in}}%
\pgfpathlineto{\pgfqpoint{5.218386in}{3.008647in}}%
\pgfpathlineto{\pgfqpoint{5.225985in}{3.027199in}}%
\pgfpathlineto{\pgfqpoint{5.212295in}{3.032508in}}%
\pgfpathlineto{\pgfqpoint{5.198612in}{3.037885in}}%
\pgfpathlineto{\pgfqpoint{5.184935in}{3.043332in}}%
\pgfpathlineto{\pgfqpoint{5.171265in}{3.048847in}}%
\pgfpathlineto{\pgfqpoint{5.163653in}{3.029790in}}%
\pgfpathlineto{\pgfqpoint{5.156048in}{3.011143in}}%
\pgfpathlineto{\pgfqpoint{5.148448in}{2.992897in}}%
\pgfpathlineto{\pgfqpoint{5.140854in}{2.975044in}}%
\pgfpathclose%
\pgfusepath{fill}%
\end{pgfscope}%
\begin{pgfscope}%
\pgfpathrectangle{\pgfqpoint{1.150000in}{0.150000in}}{\pgfqpoint{5.700000in}{5.700000in}}%
\pgfusepath{clip}%
\pgfsetbuttcap%
\pgfsetroundjoin%
\definecolor{currentfill}{rgb}{0.278012,0.180367,0.486697}%
\pgfsetfillcolor{currentfill}%
\pgfsetfillopacity{0.700000}%
\pgfsetlinewidth{0.000000pt}%
\definecolor{currentstroke}{rgb}{0.000000,0.000000,0.000000}%
\pgfsetstrokecolor{currentstroke}%
\pgfsetdash{}{0pt}%
\pgfpathmoveto{\pgfqpoint{4.322139in}{2.640668in}}%
\pgfpathlineto{\pgfqpoint{4.335667in}{2.636298in}}%
\pgfpathlineto{\pgfqpoint{4.349201in}{2.632006in}}%
\pgfpathlineto{\pgfqpoint{4.362742in}{2.627791in}}%
\pgfpathlineto{\pgfqpoint{4.376288in}{2.623653in}}%
\pgfpathlineto{\pgfqpoint{4.383986in}{2.635464in}}%
\pgfpathlineto{\pgfqpoint{4.391680in}{2.647446in}}%
\pgfpathlineto{\pgfqpoint{4.399371in}{2.659605in}}%
\pgfpathlineto{\pgfqpoint{4.407059in}{2.671946in}}%
\pgfpathlineto{\pgfqpoint{4.393523in}{2.676401in}}%
\pgfpathlineto{\pgfqpoint{4.379993in}{2.680932in}}%
\pgfpathlineto{\pgfqpoint{4.366469in}{2.685541in}}%
\pgfpathlineto{\pgfqpoint{4.352951in}{2.690228in}}%
\pgfpathlineto{\pgfqpoint{4.345253in}{2.677562in}}%
\pgfpathlineto{\pgfqpoint{4.337551in}{2.665084in}}%
\pgfpathlineto{\pgfqpoint{4.329847in}{2.652788in}}%
\pgfpathlineto{\pgfqpoint{4.322139in}{2.640668in}}%
\pgfpathclose%
\pgfusepath{fill}%
\end{pgfscope}%
\begin{pgfscope}%
\pgfpathrectangle{\pgfqpoint{1.150000in}{0.150000in}}{\pgfqpoint{5.700000in}{5.700000in}}%
\pgfusepath{clip}%
\pgfsetbuttcap%
\pgfsetroundjoin%
\definecolor{currentfill}{rgb}{0.283072,0.130895,0.449241}%
\pgfsetfillcolor{currentfill}%
\pgfsetfillopacity{0.700000}%
\pgfsetlinewidth{0.000000pt}%
\definecolor{currentstroke}{rgb}{0.000000,0.000000,0.000000}%
\pgfsetstrokecolor{currentstroke}%
\pgfsetdash{}{0pt}%
\pgfpathmoveto{\pgfqpoint{3.287443in}{2.550945in}}%
\pgfpathlineto{\pgfqpoint{3.300799in}{2.543536in}}%
\pgfpathlineto{\pgfqpoint{3.314158in}{2.536230in}}%
\pgfpathlineto{\pgfqpoint{3.327519in}{2.529028in}}%
\pgfpathlineto{\pgfqpoint{3.340883in}{2.521928in}}%
\pgfpathlineto{\pgfqpoint{3.348894in}{2.532510in}}%
\pgfpathlineto{\pgfqpoint{3.356899in}{2.543181in}}%
\pgfpathlineto{\pgfqpoint{3.364897in}{2.553946in}}%
\pgfpathlineto{\pgfqpoint{3.372890in}{2.564806in}}%
\pgfpathlineto{\pgfqpoint{3.359535in}{2.572042in}}%
\pgfpathlineto{\pgfqpoint{3.346183in}{2.579380in}}%
\pgfpathlineto{\pgfqpoint{3.332833in}{2.586822in}}%
\pgfpathlineto{\pgfqpoint{3.319486in}{2.594367in}}%
\pgfpathlineto{\pgfqpoint{3.311485in}{2.583363in}}%
\pgfpathlineto{\pgfqpoint{3.303477in}{2.572460in}}%
\pgfpathlineto{\pgfqpoint{3.295463in}{2.561655in}}%
\pgfpathlineto{\pgfqpoint{3.287443in}{2.550945in}}%
\pgfpathclose%
\pgfusepath{fill}%
\end{pgfscope}%
\begin{pgfscope}%
\pgfpathrectangle{\pgfqpoint{1.150000in}{0.150000in}}{\pgfqpoint{5.700000in}{5.700000in}}%
\pgfusepath{clip}%
\pgfsetbuttcap%
\pgfsetroundjoin%
\definecolor{currentfill}{rgb}{0.263663,0.237631,0.518762}%
\pgfsetfillcolor{currentfill}%
\pgfsetfillopacity{0.700000}%
\pgfsetlinewidth{0.000000pt}%
\definecolor{currentstroke}{rgb}{0.000000,0.000000,0.000000}%
\pgfsetstrokecolor{currentstroke}%
\pgfsetdash{}{0pt}%
\pgfpathmoveto{\pgfqpoint{4.716045in}{2.758451in}}%
\pgfpathlineto{\pgfqpoint{4.729658in}{2.754200in}}%
\pgfpathlineto{\pgfqpoint{4.743276in}{2.750021in}}%
\pgfpathlineto{\pgfqpoint{4.756902in}{2.745915in}}%
\pgfpathlineto{\pgfqpoint{4.770534in}{2.741881in}}%
\pgfpathlineto{\pgfqpoint{4.778140in}{2.755227in}}%
\pgfpathlineto{\pgfqpoint{4.785744in}{2.768822in}}%
\pgfpathlineto{\pgfqpoint{4.793349in}{2.782674in}}%
\pgfpathlineto{\pgfqpoint{4.800953in}{2.796790in}}%
\pgfpathlineto{\pgfqpoint{4.787333in}{2.801220in}}%
\pgfpathlineto{\pgfqpoint{4.773720in}{2.805723in}}%
\pgfpathlineto{\pgfqpoint{4.760113in}{2.810299in}}%
\pgfpathlineto{\pgfqpoint{4.746513in}{2.814947in}}%
\pgfpathlineto{\pgfqpoint{4.738897in}{2.800427in}}%
\pgfpathlineto{\pgfqpoint{4.731280in}{2.786176in}}%
\pgfpathlineto{\pgfqpoint{4.723663in}{2.772187in}}%
\pgfpathlineto{\pgfqpoint{4.716045in}{2.758451in}}%
\pgfpathclose%
\pgfusepath{fill}%
\end{pgfscope}%
\begin{pgfscope}%
\pgfpathrectangle{\pgfqpoint{1.150000in}{0.150000in}}{\pgfqpoint{5.700000in}{5.700000in}}%
\pgfusepath{clip}%
\pgfsetbuttcap%
\pgfsetroundjoin%
\definecolor{currentfill}{rgb}{0.282623,0.140926,0.457517}%
\pgfsetfillcolor{currentfill}%
\pgfsetfillopacity{0.700000}%
\pgfsetlinewidth{0.000000pt}%
\definecolor{currentstroke}{rgb}{0.000000,0.000000,0.000000}%
\pgfsetstrokecolor{currentstroke}%
\pgfsetdash{}{0pt}%
\pgfpathmoveto{\pgfqpoint{3.148453in}{2.571166in}}%
\pgfpathlineto{\pgfqpoint{3.161801in}{2.563018in}}%
\pgfpathlineto{\pgfqpoint{3.175151in}{2.554979in}}%
\pgfpathlineto{\pgfqpoint{3.188503in}{2.547048in}}%
\pgfpathlineto{\pgfqpoint{3.201858in}{2.539226in}}%
\pgfpathlineto{\pgfqpoint{3.209914in}{2.549690in}}%
\pgfpathlineto{\pgfqpoint{3.217963in}{2.560244in}}%
\pgfpathlineto{\pgfqpoint{3.226006in}{2.570890in}}%
\pgfpathlineto{\pgfqpoint{3.234043in}{2.581631in}}%
\pgfpathlineto{\pgfqpoint{3.220698in}{2.589569in}}%
\pgfpathlineto{\pgfqpoint{3.207356in}{2.597615in}}%
\pgfpathlineto{\pgfqpoint{3.194015in}{2.605770in}}%
\pgfpathlineto{\pgfqpoint{3.180676in}{2.614035in}}%
\pgfpathlineto{\pgfqpoint{3.172630in}{2.603170in}}%
\pgfpathlineto{\pgfqpoint{3.164578in}{2.592406in}}%
\pgfpathlineto{\pgfqpoint{3.156519in}{2.581738in}}%
\pgfpathlineto{\pgfqpoint{3.148453in}{2.571166in}}%
\pgfpathclose%
\pgfusepath{fill}%
\end{pgfscope}%
\begin{pgfscope}%
\pgfpathrectangle{\pgfqpoint{1.150000in}{0.150000in}}{\pgfqpoint{5.700000in}{5.700000in}}%
\pgfusepath{clip}%
\pgfsetbuttcap%
\pgfsetroundjoin%
\definecolor{currentfill}{rgb}{0.276194,0.190074,0.493001}%
\pgfsetfillcolor{currentfill}%
\pgfsetfillopacity{0.700000}%
\pgfsetlinewidth{0.000000pt}%
\definecolor{currentstroke}{rgb}{0.000000,0.000000,0.000000}%
\pgfsetstrokecolor{currentstroke}%
\pgfsetdash{}{0pt}%
\pgfpathmoveto{\pgfqpoint{2.816507in}{2.673674in}}%
\pgfpathlineto{\pgfqpoint{2.829863in}{2.663315in}}%
\pgfpathlineto{\pgfqpoint{2.843218in}{2.653085in}}%
\pgfpathlineto{\pgfqpoint{2.856573in}{2.642982in}}%
\pgfpathlineto{\pgfqpoint{2.869928in}{2.633006in}}%
\pgfpathlineto{\pgfqpoint{2.878094in}{2.643165in}}%
\pgfpathlineto{\pgfqpoint{2.886252in}{2.653423in}}%
\pgfpathlineto{\pgfqpoint{2.894403in}{2.663781in}}%
\pgfpathlineto{\pgfqpoint{2.902546in}{2.674243in}}%
\pgfpathlineto{\pgfqpoint{2.889202in}{2.684294in}}%
\pgfpathlineto{\pgfqpoint{2.875858in}{2.694472in}}%
\pgfpathlineto{\pgfqpoint{2.862514in}{2.704777in}}%
\pgfpathlineto{\pgfqpoint{2.849170in}{2.715211in}}%
\pgfpathlineto{\pgfqpoint{2.841016in}{2.704667in}}%
\pgfpathlineto{\pgfqpoint{2.832854in}{2.694231in}}%
\pgfpathlineto{\pgfqpoint{2.824684in}{2.683900in}}%
\pgfpathlineto{\pgfqpoint{2.816507in}{2.673674in}}%
\pgfpathclose%
\pgfusepath{fill}%
\end{pgfscope}%
\begin{pgfscope}%
\pgfpathrectangle{\pgfqpoint{1.150000in}{0.150000in}}{\pgfqpoint{5.700000in}{5.700000in}}%
\pgfusepath{clip}%
\pgfsetbuttcap%
\pgfsetroundjoin%
\definecolor{currentfill}{rgb}{0.283187,0.125848,0.444960}%
\pgfsetfillcolor{currentfill}%
\pgfsetfillopacity{0.700000}%
\pgfsetlinewidth{0.000000pt}%
\definecolor{currentstroke}{rgb}{0.000000,0.000000,0.000000}%
\pgfsetstrokecolor{currentstroke}%
\pgfsetdash{}{0pt}%
\pgfpathmoveto{\pgfqpoint{3.426339in}{2.536870in}}%
\pgfpathlineto{\pgfqpoint{3.439709in}{2.530136in}}%
\pgfpathlineto{\pgfqpoint{3.453082in}{2.523499in}}%
\pgfpathlineto{\pgfqpoint{3.466459in}{2.516961in}}%
\pgfpathlineto{\pgfqpoint{3.479839in}{2.510520in}}%
\pgfpathlineto{\pgfqpoint{3.487807in}{2.521181in}}%
\pgfpathlineto{\pgfqpoint{3.495769in}{2.531933in}}%
\pgfpathlineto{\pgfqpoint{3.503725in}{2.542780in}}%
\pgfpathlineto{\pgfqpoint{3.511675in}{2.553725in}}%
\pgfpathlineto{\pgfqpoint{3.498304in}{2.560323in}}%
\pgfpathlineto{\pgfqpoint{3.484936in}{2.567017in}}%
\pgfpathlineto{\pgfqpoint{3.471571in}{2.573810in}}%
\pgfpathlineto{\pgfqpoint{3.458210in}{2.580701in}}%
\pgfpathlineto{\pgfqpoint{3.450252in}{2.569592in}}%
\pgfpathlineto{\pgfqpoint{3.442287in}{2.558587in}}%
\pgfpathlineto{\pgfqpoint{3.434316in}{2.547680in}}%
\pgfpathlineto{\pgfqpoint{3.426339in}{2.536870in}}%
\pgfpathclose%
\pgfusepath{fill}%
\end{pgfscope}%
\begin{pgfscope}%
\pgfpathrectangle{\pgfqpoint{1.150000in}{0.150000in}}{\pgfqpoint{5.700000in}{5.700000in}}%
\pgfusepath{clip}%
\pgfsetbuttcap%
\pgfsetroundjoin%
\definecolor{currentfill}{rgb}{0.283072,0.130895,0.449241}%
\pgfsetfillcolor{currentfill}%
\pgfsetfillopacity{0.700000}%
\pgfsetlinewidth{0.000000pt}%
\definecolor{currentstroke}{rgb}{0.000000,0.000000,0.000000}%
\pgfsetstrokecolor{currentstroke}%
\pgfsetdash{}{0pt}%
\pgfpathmoveto{\pgfqpoint{3.789251in}{2.546439in}}%
\pgfpathlineto{\pgfqpoint{3.802675in}{2.541038in}}%
\pgfpathlineto{\pgfqpoint{3.816102in}{2.535725in}}%
\pgfpathlineto{\pgfqpoint{3.829535in}{2.530499in}}%
\pgfpathlineto{\pgfqpoint{3.842972in}{2.525360in}}%
\pgfpathlineto{\pgfqpoint{3.850827in}{2.536293in}}%
\pgfpathlineto{\pgfqpoint{3.858677in}{2.547336in}}%
\pgfpathlineto{\pgfqpoint{3.866522in}{2.558491in}}%
\pgfpathlineto{\pgfqpoint{3.874361in}{2.569762in}}%
\pgfpathlineto{\pgfqpoint{3.860933in}{2.575118in}}%
\pgfpathlineto{\pgfqpoint{3.847509in}{2.580560in}}%
\pgfpathlineto{\pgfqpoint{3.834090in}{2.586090in}}%
\pgfpathlineto{\pgfqpoint{3.820676in}{2.591707in}}%
\pgfpathlineto{\pgfqpoint{3.812828in}{2.580212in}}%
\pgfpathlineto{\pgfqpoint{3.804974in}{2.568838in}}%
\pgfpathlineto{\pgfqpoint{3.797116in}{2.557582in}}%
\pgfpathlineto{\pgfqpoint{3.789251in}{2.546439in}}%
\pgfpathclose%
\pgfusepath{fill}%
\end{pgfscope}%
\begin{pgfscope}%
\pgfpathrectangle{\pgfqpoint{1.150000in}{0.150000in}}{\pgfqpoint{5.700000in}{5.700000in}}%
\pgfusepath{clip}%
\pgfsetbuttcap%
\pgfsetroundjoin%
\definecolor{currentfill}{rgb}{0.212395,0.359683,0.551710}%
\pgfsetfillcolor{currentfill}%
\pgfsetfillopacity{0.700000}%
\pgfsetlinewidth{0.000000pt}%
\definecolor{currentstroke}{rgb}{0.000000,0.000000,0.000000}%
\pgfsetstrokecolor{currentstroke}%
\pgfsetdash{}{0pt}%
\pgfpathmoveto{\pgfqpoint{5.225985in}{3.027199in}}%
\pgfpathlineto{\pgfqpoint{5.239681in}{3.021959in}}%
\pgfpathlineto{\pgfqpoint{5.253385in}{3.016787in}}%
\pgfpathlineto{\pgfqpoint{5.267096in}{3.011683in}}%
\pgfpathlineto{\pgfqpoint{5.280813in}{3.006648in}}%
\pgfpathlineto{\pgfqpoint{5.288404in}{3.025104in}}%
\pgfpathlineto{\pgfqpoint{5.296002in}{3.043980in}}%
\pgfpathlineto{\pgfqpoint{5.303608in}{3.063285in}}%
\pgfpathlineto{\pgfqpoint{5.311221in}{3.083028in}}%
\pgfpathlineto{\pgfqpoint{5.297518in}{3.088581in}}%
\pgfpathlineto{\pgfqpoint{5.283822in}{3.094203in}}%
\pgfpathlineto{\pgfqpoint{5.270132in}{3.099893in}}%
\pgfpathlineto{\pgfqpoint{5.256449in}{3.105651in}}%
\pgfpathlineto{\pgfqpoint{5.248821in}{3.085382in}}%
\pgfpathlineto{\pgfqpoint{5.241202in}{3.065557in}}%
\pgfpathlineto{\pgfqpoint{5.233590in}{3.046166in}}%
\pgfpathlineto{\pgfqpoint{5.225985in}{3.027199in}}%
\pgfpathclose%
\pgfusepath{fill}%
\end{pgfscope}%
\begin{pgfscope}%
\pgfpathrectangle{\pgfqpoint{1.150000in}{0.150000in}}{\pgfqpoint{5.700000in}{5.700000in}}%
\pgfusepath{clip}%
\pgfsetbuttcap%
\pgfsetroundjoin%
\definecolor{currentfill}{rgb}{0.282290,0.145912,0.461510}%
\pgfsetfillcolor{currentfill}%
\pgfsetfillopacity{0.700000}%
\pgfsetlinewidth{0.000000pt}%
\definecolor{currentstroke}{rgb}{0.000000,0.000000,0.000000}%
\pgfsetstrokecolor{currentstroke}%
\pgfsetdash{}{0pt}%
\pgfpathmoveto{\pgfqpoint{4.013205in}{2.574449in}}%
\pgfpathlineto{\pgfqpoint{4.026670in}{2.569622in}}%
\pgfpathlineto{\pgfqpoint{4.040141in}{2.564878in}}%
\pgfpathlineto{\pgfqpoint{4.053617in}{2.560216in}}%
\pgfpathlineto{\pgfqpoint{4.067099in}{2.555637in}}%
\pgfpathlineto{\pgfqpoint{4.074886in}{2.566810in}}%
\pgfpathlineto{\pgfqpoint{4.082669in}{2.578112in}}%
\pgfpathlineto{\pgfqpoint{4.090447in}{2.589548in}}%
\pgfpathlineto{\pgfqpoint{4.098220in}{2.601121in}}%
\pgfpathlineto{\pgfqpoint{4.084748in}{2.605957in}}%
\pgfpathlineto{\pgfqpoint{4.071281in}{2.610875in}}%
\pgfpathlineto{\pgfqpoint{4.057820in}{2.615876in}}%
\pgfpathlineto{\pgfqpoint{4.044363in}{2.620960in}}%
\pgfpathlineto{\pgfqpoint{4.036581in}{2.609123in}}%
\pgfpathlineto{\pgfqpoint{4.028793in}{2.597428in}}%
\pgfpathlineto{\pgfqpoint{4.021002in}{2.585872in}}%
\pgfpathlineto{\pgfqpoint{4.013205in}{2.574449in}}%
\pgfpathclose%
\pgfusepath{fill}%
\end{pgfscope}%
\begin{pgfscope}%
\pgfpathrectangle{\pgfqpoint{1.150000in}{0.150000in}}{\pgfqpoint{5.700000in}{5.700000in}}%
\pgfusepath{clip}%
\pgfsetbuttcap%
\pgfsetroundjoin%
\definecolor{currentfill}{rgb}{0.267968,0.223549,0.512008}%
\pgfsetfillcolor{currentfill}%
\pgfsetfillopacity{0.700000}%
\pgfsetlinewidth{0.000000pt}%
\definecolor{currentstroke}{rgb}{0.000000,0.000000,0.000000}%
\pgfsetstrokecolor{currentstroke}%
\pgfsetdash{}{0pt}%
\pgfpathmoveto{\pgfqpoint{4.631136in}{2.722143in}}%
\pgfpathlineto{\pgfqpoint{4.644733in}{2.717975in}}%
\pgfpathlineto{\pgfqpoint{4.658337in}{2.713881in}}%
\pgfpathlineto{\pgfqpoint{4.671948in}{2.709860in}}%
\pgfpathlineto{\pgfqpoint{4.685565in}{2.705913in}}%
\pgfpathlineto{\pgfqpoint{4.693187in}{2.718701in}}%
\pgfpathlineto{\pgfqpoint{4.700808in}{2.731715in}}%
\pgfpathlineto{\pgfqpoint{4.708427in}{2.744963in}}%
\pgfpathlineto{\pgfqpoint{4.716045in}{2.758451in}}%
\pgfpathlineto{\pgfqpoint{4.702440in}{2.762776in}}%
\pgfpathlineto{\pgfqpoint{4.688841in}{2.767173in}}%
\pgfpathlineto{\pgfqpoint{4.675249in}{2.771645in}}%
\pgfpathlineto{\pgfqpoint{4.661663in}{2.776189in}}%
\pgfpathlineto{\pgfqpoint{4.654033in}{2.762317in}}%
\pgfpathlineto{\pgfqpoint{4.646402in}{2.748690in}}%
\pgfpathlineto{\pgfqpoint{4.638770in}{2.735301in}}%
\pgfpathlineto{\pgfqpoint{4.631136in}{2.722143in}}%
\pgfpathclose%
\pgfusepath{fill}%
\end{pgfscope}%
\begin{pgfscope}%
\pgfpathrectangle{\pgfqpoint{1.150000in}{0.150000in}}{\pgfqpoint{5.700000in}{5.700000in}}%
\pgfusepath{clip}%
\pgfsetbuttcap%
\pgfsetroundjoin%
\definecolor{currentfill}{rgb}{0.281412,0.155834,0.469201}%
\pgfsetfillcolor{currentfill}%
\pgfsetfillopacity{0.700000}%
\pgfsetlinewidth{0.000000pt}%
\definecolor{currentstroke}{rgb}{0.000000,0.000000,0.000000}%
\pgfsetstrokecolor{currentstroke}%
\pgfsetdash{}{0pt}%
\pgfpathmoveto{\pgfqpoint{3.009305in}{2.598254in}}%
\pgfpathlineto{\pgfqpoint{3.022653in}{2.589294in}}%
\pgfpathlineto{\pgfqpoint{3.036002in}{2.580450in}}%
\pgfpathlineto{\pgfqpoint{3.049351in}{2.571722in}}%
\pgfpathlineto{\pgfqpoint{3.062702in}{2.563108in}}%
\pgfpathlineto{\pgfqpoint{3.070806in}{2.573411in}}%
\pgfpathlineto{\pgfqpoint{3.078903in}{2.583804in}}%
\pgfpathlineto{\pgfqpoint{3.086993in}{2.594291in}}%
\pgfpathlineto{\pgfqpoint{3.095075in}{2.604874in}}%
\pgfpathlineto{\pgfqpoint{3.081735in}{2.613583in}}%
\pgfpathlineto{\pgfqpoint{3.068395in}{2.622407in}}%
\pgfpathlineto{\pgfqpoint{3.055057in}{2.631346in}}%
\pgfpathlineto{\pgfqpoint{3.041719in}{2.640402in}}%
\pgfpathlineto{\pgfqpoint{3.033627in}{2.629716in}}%
\pgfpathlineto{\pgfqpoint{3.025527in}{2.619131in}}%
\pgfpathlineto{\pgfqpoint{3.017419in}{2.608645in}}%
\pgfpathlineto{\pgfqpoint{3.009305in}{2.598254in}}%
\pgfpathclose%
\pgfusepath{fill}%
\end{pgfscope}%
\begin{pgfscope}%
\pgfpathrectangle{\pgfqpoint{1.150000in}{0.150000in}}{\pgfqpoint{5.700000in}{5.700000in}}%
\pgfusepath{clip}%
\pgfsetbuttcap%
\pgfsetroundjoin%
\definecolor{currentfill}{rgb}{0.283229,0.120777,0.440584}%
\pgfsetfillcolor{currentfill}%
\pgfsetfillopacity{0.700000}%
\pgfsetlinewidth{0.000000pt}%
\definecolor{currentstroke}{rgb}{0.000000,0.000000,0.000000}%
\pgfsetstrokecolor{currentstroke}%
\pgfsetdash{}{0pt}%
\pgfpathmoveto{\pgfqpoint{3.565195in}{2.528295in}}%
\pgfpathlineto{\pgfqpoint{3.578584in}{2.522175in}}%
\pgfpathlineto{\pgfqpoint{3.591978in}{2.516148in}}%
\pgfpathlineto{\pgfqpoint{3.605375in}{2.510215in}}%
\pgfpathlineto{\pgfqpoint{3.618777in}{2.504374in}}%
\pgfpathlineto{\pgfqpoint{3.626703in}{2.515082in}}%
\pgfpathlineto{\pgfqpoint{3.634623in}{2.525884in}}%
\pgfpathlineto{\pgfqpoint{3.642538in}{2.536784in}}%
\pgfpathlineto{\pgfqpoint{3.650447in}{2.547784in}}%
\pgfpathlineto{\pgfqpoint{3.637055in}{2.553801in}}%
\pgfpathlineto{\pgfqpoint{3.623666in}{2.559911in}}%
\pgfpathlineto{\pgfqpoint{3.610282in}{2.566114in}}%
\pgfpathlineto{\pgfqpoint{3.596901in}{2.572410in}}%
\pgfpathlineto{\pgfqpoint{3.588983in}{2.561226in}}%
\pgfpathlineto{\pgfqpoint{3.581060in}{2.550148in}}%
\pgfpathlineto{\pgfqpoint{3.573130in}{2.539172in}}%
\pgfpathlineto{\pgfqpoint{3.565195in}{2.528295in}}%
\pgfpathclose%
\pgfusepath{fill}%
\end{pgfscope}%
\begin{pgfscope}%
\pgfpathrectangle{\pgfqpoint{1.150000in}{0.150000in}}{\pgfqpoint{5.700000in}{5.700000in}}%
\pgfusepath{clip}%
\pgfsetbuttcap%
\pgfsetroundjoin%
\definecolor{currentfill}{rgb}{0.280255,0.165693,0.476498}%
\pgfsetfillcolor{currentfill}%
\pgfsetfillopacity{0.700000}%
\pgfsetlinewidth{0.000000pt}%
\definecolor{currentstroke}{rgb}{0.000000,0.000000,0.000000}%
\pgfsetstrokecolor{currentstroke}%
\pgfsetdash{}{0pt}%
\pgfpathmoveto{\pgfqpoint{4.237177in}{2.610914in}}%
\pgfpathlineto{\pgfqpoint{4.250691in}{2.606526in}}%
\pgfpathlineto{\pgfqpoint{4.264211in}{2.602218in}}%
\pgfpathlineto{\pgfqpoint{4.277737in}{2.597988in}}%
\pgfpathlineto{\pgfqpoint{4.291270in}{2.593837in}}%
\pgfpathlineto{\pgfqpoint{4.298993in}{2.605308in}}%
\pgfpathlineto{\pgfqpoint{4.306712in}{2.616934in}}%
\pgfpathlineto{\pgfqpoint{4.314427in}{2.628719in}}%
\pgfpathlineto{\pgfqpoint{4.322139in}{2.640668in}}%
\pgfpathlineto{\pgfqpoint{4.308616in}{2.645116in}}%
\pgfpathlineto{\pgfqpoint{4.295100in}{2.649643in}}%
\pgfpathlineto{\pgfqpoint{4.281590in}{2.654248in}}%
\pgfpathlineto{\pgfqpoint{4.268086in}{2.658932in}}%
\pgfpathlineto{\pgfqpoint{4.260364in}{2.646678in}}%
\pgfpathlineto{\pgfqpoint{4.252639in}{2.634594in}}%
\pgfpathlineto{\pgfqpoint{4.244910in}{2.622675in}}%
\pgfpathlineto{\pgfqpoint{4.237177in}{2.610914in}}%
\pgfpathclose%
\pgfusepath{fill}%
\end{pgfscope}%
\begin{pgfscope}%
\pgfpathrectangle{\pgfqpoint{1.150000in}{0.150000in}}{\pgfqpoint{5.700000in}{5.700000in}}%
\pgfusepath{clip}%
\pgfsetbuttcap%
\pgfsetroundjoin%
\definecolor{currentfill}{rgb}{0.201239,0.383670,0.554294}%
\pgfsetfillcolor{currentfill}%
\pgfsetfillopacity{0.700000}%
\pgfsetlinewidth{0.000000pt}%
\definecolor{currentstroke}{rgb}{0.000000,0.000000,0.000000}%
\pgfsetstrokecolor{currentstroke}%
\pgfsetdash{}{0pt}%
\pgfpathmoveto{\pgfqpoint{5.311221in}{3.083028in}}%
\pgfpathlineto{\pgfqpoint{5.324931in}{3.077542in}}%
\pgfpathlineto{\pgfqpoint{5.338648in}{3.072125in}}%
\pgfpathlineto{\pgfqpoint{5.352372in}{3.066775in}}%
\pgfpathlineto{\pgfqpoint{5.366102in}{3.061493in}}%
\pgfpathlineto{\pgfqpoint{5.373710in}{3.081153in}}%
\pgfpathlineto{\pgfqpoint{5.381327in}{3.101267in}}%
\pgfpathlineto{\pgfqpoint{5.388954in}{3.121845in}}%
\pgfpathlineto{\pgfqpoint{5.396590in}{3.142896in}}%
\pgfpathlineto{\pgfqpoint{5.382874in}{3.148717in}}%
\pgfpathlineto{\pgfqpoint{5.369165in}{3.154606in}}%
\pgfpathlineto{\pgfqpoint{5.355462in}{3.160562in}}%
\pgfpathlineto{\pgfqpoint{5.341766in}{3.166586in}}%
\pgfpathlineto{\pgfqpoint{5.334115in}{3.144989in}}%
\pgfpathlineto{\pgfqpoint{5.326475in}{3.123870in}}%
\pgfpathlineto{\pgfqpoint{5.318844in}{3.103220in}}%
\pgfpathlineto{\pgfqpoint{5.311221in}{3.083028in}}%
\pgfpathclose%
\pgfusepath{fill}%
\end{pgfscope}%
\begin{pgfscope}%
\pgfpathrectangle{\pgfqpoint{1.150000in}{0.150000in}}{\pgfqpoint{5.700000in}{5.700000in}}%
\pgfusepath{clip}%
\pgfsetbuttcap%
\pgfsetroundjoin%
\definecolor{currentfill}{rgb}{0.273006,0.204520,0.501721}%
\pgfsetfillcolor{currentfill}%
\pgfsetfillopacity{0.700000}%
\pgfsetlinewidth{0.000000pt}%
\definecolor{currentstroke}{rgb}{0.000000,0.000000,0.000000}%
\pgfsetstrokecolor{currentstroke}%
\pgfsetdash{}{0pt}%
\pgfpathmoveto{\pgfqpoint{4.546213in}{2.687679in}}%
\pgfpathlineto{\pgfqpoint{4.559795in}{2.683571in}}%
\pgfpathlineto{\pgfqpoint{4.573385in}{2.679537in}}%
\pgfpathlineto{\pgfqpoint{4.586980in}{2.675578in}}%
\pgfpathlineto{\pgfqpoint{4.600583in}{2.671693in}}%
\pgfpathlineto{\pgfqpoint{4.608224in}{2.683992in}}%
\pgfpathlineto{\pgfqpoint{4.615863in}{2.696495in}}%
\pgfpathlineto{\pgfqpoint{4.623500in}{2.709210in}}%
\pgfpathlineto{\pgfqpoint{4.631136in}{2.722143in}}%
\pgfpathlineto{\pgfqpoint{4.617545in}{2.726385in}}%
\pgfpathlineto{\pgfqpoint{4.603961in}{2.730701in}}%
\pgfpathlineto{\pgfqpoint{4.590383in}{2.735092in}}%
\pgfpathlineto{\pgfqpoint{4.576812in}{2.739557in}}%
\pgfpathlineto{\pgfqpoint{4.569165in}{2.726260in}}%
\pgfpathlineto{\pgfqpoint{4.561516in}{2.713185in}}%
\pgfpathlineto{\pgfqpoint{4.553866in}{2.700327in}}%
\pgfpathlineto{\pgfqpoint{4.546213in}{2.687679in}}%
\pgfpathclose%
\pgfusepath{fill}%
\end{pgfscope}%
\begin{pgfscope}%
\pgfpathrectangle{\pgfqpoint{1.150000in}{0.150000in}}{\pgfqpoint{5.700000in}{5.700000in}}%
\pgfusepath{clip}%
\pgfsetbuttcap%
\pgfsetroundjoin%
\definecolor{currentfill}{rgb}{0.177423,0.437527,0.557565}%
\pgfsetfillcolor{currentfill}%
\pgfsetfillopacity{0.700000}%
\pgfsetlinewidth{0.000000pt}%
\definecolor{currentstroke}{rgb}{0.000000,0.000000,0.000000}%
\pgfsetstrokecolor{currentstroke}%
\pgfsetdash{}{0pt}%
\pgfpathmoveto{\pgfqpoint{5.427245in}{3.232045in}}%
\pgfpathlineto{\pgfqpoint{5.440954in}{3.225732in}}%
\pgfpathlineto{\pgfqpoint{5.454670in}{3.219486in}}%
\pgfpathlineto{\pgfqpoint{5.468392in}{3.213307in}}%
\pgfpathlineto{\pgfqpoint{5.482120in}{3.207196in}}%
\pgfpathlineto{\pgfqpoint{5.489800in}{3.230200in}}%
\pgfpathlineto{\pgfqpoint{5.497493in}{3.253736in}}%
\pgfpathlineto{\pgfqpoint{5.505199in}{3.277816in}}%
\pgfpathlineto{\pgfqpoint{5.491481in}{3.284362in}}%
\pgfpathlineto{\pgfqpoint{5.477769in}{3.290974in}}%
\pgfpathlineto{\pgfqpoint{5.464063in}{3.297654in}}%
\pgfpathlineto{\pgfqpoint{5.450364in}{3.304401in}}%
\pgfpathlineto{\pgfqpoint{5.442645in}{3.279737in}}%
\pgfpathlineto{\pgfqpoint{5.434939in}{3.255622in}}%
\pgfpathlineto{\pgfqpoint{5.427245in}{3.232045in}}%
\pgfpathclose%
\pgfusepath{fill}%
\end{pgfscope}%
\begin{pgfscope}%
\pgfpathrectangle{\pgfqpoint{1.150000in}{0.150000in}}{\pgfqpoint{5.700000in}{5.700000in}}%
\pgfusepath{clip}%
\pgfsetbuttcap%
\pgfsetroundjoin%
\definecolor{currentfill}{rgb}{0.282884,0.135920,0.453427}%
\pgfsetfillcolor{currentfill}%
\pgfsetfillopacity{0.700000}%
\pgfsetlinewidth{0.000000pt}%
\definecolor{currentstroke}{rgb}{0.000000,0.000000,0.000000}%
\pgfsetstrokecolor{currentstroke}%
\pgfsetdash{}{0pt}%
\pgfpathmoveto{\pgfqpoint{3.928125in}{2.549202in}}%
\pgfpathlineto{\pgfqpoint{3.941578in}{2.544275in}}%
\pgfpathlineto{\pgfqpoint{3.955037in}{2.539433in}}%
\pgfpathlineto{\pgfqpoint{3.968501in}{2.534676in}}%
\pgfpathlineto{\pgfqpoint{3.981970in}{2.530002in}}%
\pgfpathlineto{\pgfqpoint{3.989786in}{2.540936in}}%
\pgfpathlineto{\pgfqpoint{3.997597in}{2.551986in}}%
\pgfpathlineto{\pgfqpoint{4.005404in}{2.563155in}}%
\pgfpathlineto{\pgfqpoint{4.013205in}{2.574449in}}%
\pgfpathlineto{\pgfqpoint{3.999745in}{2.579360in}}%
\pgfpathlineto{\pgfqpoint{3.986290in}{2.584355in}}%
\pgfpathlineto{\pgfqpoint{3.972841in}{2.589433in}}%
\pgfpathlineto{\pgfqpoint{3.959396in}{2.594597in}}%
\pgfpathlineto{\pgfqpoint{3.951586in}{2.583059in}}%
\pgfpathlineto{\pgfqpoint{3.943770in}{2.571650in}}%
\pgfpathlineto{\pgfqpoint{3.935950in}{2.560366in}}%
\pgfpathlineto{\pgfqpoint{3.928125in}{2.549202in}}%
\pgfpathclose%
\pgfusepath{fill}%
\end{pgfscope}%
\begin{pgfscope}%
\pgfpathrectangle{\pgfqpoint{1.150000in}{0.150000in}}{\pgfqpoint{5.700000in}{5.700000in}}%
\pgfusepath{clip}%
\pgfsetbuttcap%
\pgfsetroundjoin%
\definecolor{currentfill}{rgb}{0.279574,0.170599,0.479997}%
\pgfsetfillcolor{currentfill}%
\pgfsetfillopacity{0.700000}%
\pgfsetlinewidth{0.000000pt}%
\definecolor{currentstroke}{rgb}{0.000000,0.000000,0.000000}%
\pgfsetstrokecolor{currentstroke}%
\pgfsetdash{}{0pt}%
\pgfpathmoveto{\pgfqpoint{2.869928in}{2.633006in}}%
\pgfpathlineto{\pgfqpoint{2.883283in}{2.623155in}}%
\pgfpathlineto{\pgfqpoint{2.896638in}{2.613428in}}%
\pgfpathlineto{\pgfqpoint{2.909994in}{2.603824in}}%
\pgfpathlineto{\pgfqpoint{2.923349in}{2.594343in}}%
\pgfpathlineto{\pgfqpoint{2.931503in}{2.604434in}}%
\pgfpathlineto{\pgfqpoint{2.939650in}{2.614619in}}%
\pgfpathlineto{\pgfqpoint{2.947790in}{2.624900in}}%
\pgfpathlineto{\pgfqpoint{2.955922in}{2.635279in}}%
\pgfpathlineto{\pgfqpoint{2.942578in}{2.644835in}}%
\pgfpathlineto{\pgfqpoint{2.929234in}{2.654514in}}%
\pgfpathlineto{\pgfqpoint{2.915890in}{2.664316in}}%
\pgfpathlineto{\pgfqpoint{2.902546in}{2.674243in}}%
\pgfpathlineto{\pgfqpoint{2.894403in}{2.663781in}}%
\pgfpathlineto{\pgfqpoint{2.886252in}{2.653423in}}%
\pgfpathlineto{\pgfqpoint{2.878094in}{2.643165in}}%
\pgfpathlineto{\pgfqpoint{2.869928in}{2.633006in}}%
\pgfpathclose%
\pgfusepath{fill}%
\end{pgfscope}%
\begin{pgfscope}%
\pgfpathrectangle{\pgfqpoint{1.150000in}{0.150000in}}{\pgfqpoint{5.700000in}{5.700000in}}%
\pgfusepath{clip}%
\pgfsetbuttcap%
\pgfsetroundjoin%
\definecolor{currentfill}{rgb}{0.283187,0.125848,0.444960}%
\pgfsetfillcolor{currentfill}%
\pgfsetfillopacity{0.700000}%
\pgfsetlinewidth{0.000000pt}%
\definecolor{currentstroke}{rgb}{0.000000,0.000000,0.000000}%
\pgfsetstrokecolor{currentstroke}%
\pgfsetdash{}{0pt}%
\pgfpathmoveto{\pgfqpoint{3.704059in}{2.524633in}}%
\pgfpathlineto{\pgfqpoint{3.717473in}{2.519073in}}%
\pgfpathlineto{\pgfqpoint{3.730891in}{2.513601in}}%
\pgfpathlineto{\pgfqpoint{3.744314in}{2.508220in}}%
\pgfpathlineto{\pgfqpoint{3.757742in}{2.502927in}}%
\pgfpathlineto{\pgfqpoint{3.765627in}{2.513654in}}%
\pgfpathlineto{\pgfqpoint{3.773507in}{2.524479in}}%
\pgfpathlineto{\pgfqpoint{3.781382in}{2.535406in}}%
\pgfpathlineto{\pgfqpoint{3.789251in}{2.546439in}}%
\pgfpathlineto{\pgfqpoint{3.775833in}{2.551929in}}%
\pgfpathlineto{\pgfqpoint{3.762419in}{2.557507in}}%
\pgfpathlineto{\pgfqpoint{3.749010in}{2.563175in}}%
\pgfpathlineto{\pgfqpoint{3.735605in}{2.568932in}}%
\pgfpathlineto{\pgfqpoint{3.727727in}{2.557695in}}%
\pgfpathlineto{\pgfqpoint{3.719843in}{2.546569in}}%
\pgfpathlineto{\pgfqpoint{3.711954in}{2.535550in}}%
\pgfpathlineto{\pgfqpoint{3.704059in}{2.524633in}}%
\pgfpathclose%
\pgfusepath{fill}%
\end{pgfscope}%
\begin{pgfscope}%
\pgfpathrectangle{\pgfqpoint{1.150000in}{0.150000in}}{\pgfqpoint{5.700000in}{5.700000in}}%
\pgfusepath{clip}%
\pgfsetbuttcap%
\pgfsetroundjoin%
\definecolor{currentfill}{rgb}{0.281412,0.155834,0.469201}%
\pgfsetfillcolor{currentfill}%
\pgfsetfillopacity{0.700000}%
\pgfsetlinewidth{0.000000pt}%
\definecolor{currentstroke}{rgb}{0.000000,0.000000,0.000000}%
\pgfsetstrokecolor{currentstroke}%
\pgfsetdash{}{0pt}%
\pgfpathmoveto{\pgfqpoint{4.152166in}{2.582593in}}%
\pgfpathlineto{\pgfqpoint{4.165666in}{2.578163in}}%
\pgfpathlineto{\pgfqpoint{4.179173in}{2.573813in}}%
\pgfpathlineto{\pgfqpoint{4.192685in}{2.569544in}}%
\pgfpathlineto{\pgfqpoint{4.206204in}{2.565354in}}%
\pgfpathlineto{\pgfqpoint{4.213953in}{2.576532in}}%
\pgfpathlineto{\pgfqpoint{4.221698in}{2.587848in}}%
\pgfpathlineto{\pgfqpoint{4.229439in}{2.599307in}}%
\pgfpathlineto{\pgfqpoint{4.237177in}{2.610914in}}%
\pgfpathlineto{\pgfqpoint{4.223668in}{2.615381in}}%
\pgfpathlineto{\pgfqpoint{4.210166in}{2.619927in}}%
\pgfpathlineto{\pgfqpoint{4.196669in}{2.624553in}}%
\pgfpathlineto{\pgfqpoint{4.183178in}{2.629260in}}%
\pgfpathlineto{\pgfqpoint{4.175431in}{2.617368in}}%
\pgfpathlineto{\pgfqpoint{4.167680in}{2.605630in}}%
\pgfpathlineto{\pgfqpoint{4.159925in}{2.594040in}}%
\pgfpathlineto{\pgfqpoint{4.152166in}{2.582593in}}%
\pgfpathclose%
\pgfusepath{fill}%
\end{pgfscope}%
\begin{pgfscope}%
\pgfpathrectangle{\pgfqpoint{1.150000in}{0.150000in}}{\pgfqpoint{5.700000in}{5.700000in}}%
\pgfusepath{clip}%
\pgfsetbuttcap%
\pgfsetroundjoin%
\definecolor{currentfill}{rgb}{0.276194,0.190074,0.493001}%
\pgfsetfillcolor{currentfill}%
\pgfsetfillopacity{0.700000}%
\pgfsetlinewidth{0.000000pt}%
\definecolor{currentstroke}{rgb}{0.000000,0.000000,0.000000}%
\pgfsetstrokecolor{currentstroke}%
\pgfsetdash{}{0pt}%
\pgfpathmoveto{\pgfqpoint{4.461267in}{2.654896in}}%
\pgfpathlineto{\pgfqpoint{4.474835in}{2.650823in}}%
\pgfpathlineto{\pgfqpoint{4.488409in}{2.646827in}}%
\pgfpathlineto{\pgfqpoint{4.501990in}{2.642905in}}%
\pgfpathlineto{\pgfqpoint{4.515577in}{2.639059in}}%
\pgfpathlineto{\pgfqpoint{4.523240in}{2.650931in}}%
\pgfpathlineto{\pgfqpoint{4.530900in}{2.662987in}}%
\pgfpathlineto{\pgfqpoint{4.538558in}{2.675235in}}%
\pgfpathlineto{\pgfqpoint{4.546213in}{2.687679in}}%
\pgfpathlineto{\pgfqpoint{4.532637in}{2.691862in}}%
\pgfpathlineto{\pgfqpoint{4.519067in}{2.696120in}}%
\pgfpathlineto{\pgfqpoint{4.505504in}{2.700454in}}%
\pgfpathlineto{\pgfqpoint{4.491948in}{2.704863in}}%
\pgfpathlineto{\pgfqpoint{4.484281in}{2.692074in}}%
\pgfpathlineto{\pgfqpoint{4.476612in}{2.679487in}}%
\pgfpathlineto{\pgfqpoint{4.468941in}{2.667097in}}%
\pgfpathlineto{\pgfqpoint{4.461267in}{2.654896in}}%
\pgfpathclose%
\pgfusepath{fill}%
\end{pgfscope}%
\begin{pgfscope}%
\pgfpathrectangle{\pgfqpoint{1.150000in}{0.150000in}}{\pgfqpoint{5.700000in}{5.700000in}}%
\pgfusepath{clip}%
\pgfsetbuttcap%
\pgfsetroundjoin%
\definecolor{currentfill}{rgb}{0.270595,0.214069,0.507052}%
\pgfsetfillcolor{currentfill}%
\pgfsetfillopacity{0.700000}%
\pgfsetlinewidth{0.000000pt}%
\definecolor{currentstroke}{rgb}{0.000000,0.000000,0.000000}%
\pgfsetstrokecolor{currentstroke}%
\pgfsetdash{}{0pt}%
\pgfpathmoveto{\pgfqpoint{2.676742in}{2.720979in}}%
\pgfpathlineto{\pgfqpoint{2.690119in}{2.709605in}}%
\pgfpathlineto{\pgfqpoint{2.703494in}{2.698368in}}%
\pgfpathlineto{\pgfqpoint{2.716867in}{2.687269in}}%
\pgfpathlineto{\pgfqpoint{2.730240in}{2.676304in}}%
\pgfpathlineto{\pgfqpoint{2.738461in}{2.686176in}}%
\pgfpathlineto{\pgfqpoint{2.746674in}{2.696152in}}%
\pgfpathlineto{\pgfqpoint{2.754879in}{2.706232in}}%
\pgfpathlineto{\pgfqpoint{2.763076in}{2.716417in}}%
\pgfpathlineto{\pgfqpoint{2.749716in}{2.727436in}}%
\pgfpathlineto{\pgfqpoint{2.736355in}{2.738590in}}%
\pgfpathlineto{\pgfqpoint{2.722993in}{2.749881in}}%
\pgfpathlineto{\pgfqpoint{2.709629in}{2.761310in}}%
\pgfpathlineto{\pgfqpoint{2.701420in}{2.751062in}}%
\pgfpathlineto{\pgfqpoint{2.693202in}{2.740925in}}%
\pgfpathlineto{\pgfqpoint{2.684976in}{2.730898in}}%
\pgfpathlineto{\pgfqpoint{2.676742in}{2.720979in}}%
\pgfpathclose%
\pgfusepath{fill}%
\end{pgfscope}%
\begin{pgfscope}%
\pgfpathrectangle{\pgfqpoint{1.150000in}{0.150000in}}{\pgfqpoint{5.700000in}{5.700000in}}%
\pgfusepath{clip}%
\pgfsetbuttcap%
\pgfsetroundjoin%
\definecolor{currentfill}{rgb}{0.283072,0.130895,0.449241}%
\pgfsetfillcolor{currentfill}%
\pgfsetfillopacity{0.700000}%
\pgfsetlinewidth{0.000000pt}%
\definecolor{currentstroke}{rgb}{0.000000,0.000000,0.000000}%
\pgfsetstrokecolor{currentstroke}%
\pgfsetdash{}{0pt}%
\pgfpathmoveto{\pgfqpoint{3.201858in}{2.539226in}}%
\pgfpathlineto{\pgfqpoint{3.215214in}{2.531511in}}%
\pgfpathlineto{\pgfqpoint{3.228572in}{2.523903in}}%
\pgfpathlineto{\pgfqpoint{3.241933in}{2.516401in}}%
\pgfpathlineto{\pgfqpoint{3.255297in}{2.509003in}}%
\pgfpathlineto{\pgfqpoint{3.263343in}{2.519359in}}%
\pgfpathlineto{\pgfqpoint{3.271383in}{2.529799in}}%
\pgfpathlineto{\pgfqpoint{3.279416in}{2.540327in}}%
\pgfpathlineto{\pgfqpoint{3.287443in}{2.550945in}}%
\pgfpathlineto{\pgfqpoint{3.274089in}{2.558458in}}%
\pgfpathlineto{\pgfqpoint{3.260738in}{2.566076in}}%
\pgfpathlineto{\pgfqpoint{3.247389in}{2.573801in}}%
\pgfpathlineto{\pgfqpoint{3.234043in}{2.581631in}}%
\pgfpathlineto{\pgfqpoint{3.226006in}{2.570890in}}%
\pgfpathlineto{\pgfqpoint{3.217963in}{2.560244in}}%
\pgfpathlineto{\pgfqpoint{3.209914in}{2.549690in}}%
\pgfpathlineto{\pgfqpoint{3.201858in}{2.539226in}}%
\pgfpathclose%
\pgfusepath{fill}%
\end{pgfscope}%
\begin{pgfscope}%
\pgfpathrectangle{\pgfqpoint{1.150000in}{0.150000in}}{\pgfqpoint{5.700000in}{5.700000in}}%
\pgfusepath{clip}%
\pgfsetbuttcap%
\pgfsetroundjoin%
\definecolor{currentfill}{rgb}{0.243113,0.292092,0.538516}%
\pgfsetfillcolor{currentfill}%
\pgfsetfillopacity{0.700000}%
\pgfsetlinewidth{0.000000pt}%
\definecolor{currentstroke}{rgb}{0.000000,0.000000,0.000000}%
\pgfsetstrokecolor{currentstroke}%
\pgfsetdash{}{0pt}%
\pgfpathmoveto{\pgfqpoint{5.025481in}{2.862281in}}%
\pgfpathlineto{\pgfqpoint{5.039164in}{2.857920in}}%
\pgfpathlineto{\pgfqpoint{5.052854in}{2.853628in}}%
\pgfpathlineto{\pgfqpoint{5.066552in}{2.849406in}}%
\pgfpathlineto{\pgfqpoint{5.080257in}{2.845254in}}%
\pgfpathlineto{\pgfqpoint{5.087819in}{2.860290in}}%
\pgfpathlineto{\pgfqpoint{5.095384in}{2.875649in}}%
\pgfpathlineto{\pgfqpoint{5.102953in}{2.891339in}}%
\pgfpathlineto{\pgfqpoint{5.110525in}{2.907367in}}%
\pgfpathlineto{\pgfqpoint{5.096834in}{2.911977in}}%
\pgfpathlineto{\pgfqpoint{5.083151in}{2.916656in}}%
\pgfpathlineto{\pgfqpoint{5.069475in}{2.921405in}}%
\pgfpathlineto{\pgfqpoint{5.055805in}{2.926224in}}%
\pgfpathlineto{\pgfqpoint{5.048219in}{2.909731in}}%
\pgfpathlineto{\pgfqpoint{5.040637in}{2.893581in}}%
\pgfpathlineto{\pgfqpoint{5.033057in}{2.877768in}}%
\pgfpathlineto{\pgfqpoint{5.025481in}{2.862281in}}%
\pgfpathclose%
\pgfusepath{fill}%
\end{pgfscope}%
\begin{pgfscope}%
\pgfpathrectangle{\pgfqpoint{1.150000in}{0.150000in}}{\pgfqpoint{5.700000in}{5.700000in}}%
\pgfusepath{clip}%
\pgfsetbuttcap%
\pgfsetroundjoin%
\definecolor{currentfill}{rgb}{0.188923,0.410910,0.556326}%
\pgfsetfillcolor{currentfill}%
\pgfsetfillopacity{0.700000}%
\pgfsetlinewidth{0.000000pt}%
\definecolor{currentstroke}{rgb}{0.000000,0.000000,0.000000}%
\pgfsetstrokecolor{currentstroke}%
\pgfsetdash{}{0pt}%
\pgfpathmoveto{\pgfqpoint{5.396590in}{3.142896in}}%
\pgfpathlineto{\pgfqpoint{5.410313in}{3.137142in}}%
\pgfpathlineto{\pgfqpoint{5.424043in}{3.131456in}}%
\pgfpathlineto{\pgfqpoint{5.437779in}{3.125837in}}%
\pgfpathlineto{\pgfqpoint{5.451522in}{3.120285in}}%
\pgfpathlineto{\pgfqpoint{5.459155in}{3.141268in}}%
\pgfpathlineto{\pgfqpoint{5.466798in}{3.162741in}}%
\pgfpathlineto{\pgfqpoint{5.474453in}{3.184713in}}%
\pgfpathlineto{\pgfqpoint{5.482120in}{3.207196in}}%
\pgfpathlineto{\pgfqpoint{5.468392in}{3.213307in}}%
\pgfpathlineto{\pgfqpoint{5.454670in}{3.219486in}}%
\pgfpathlineto{\pgfqpoint{5.440954in}{3.225732in}}%
\pgfpathlineto{\pgfqpoint{5.427245in}{3.232045in}}%
\pgfpathlineto{\pgfqpoint{5.419564in}{3.208995in}}%
\pgfpathlineto{\pgfqpoint{5.411895in}{3.186460in}}%
\pgfpathlineto{\pgfqpoint{5.404237in}{3.164431in}}%
\pgfpathlineto{\pgfqpoint{5.396590in}{3.142896in}}%
\pgfpathclose%
\pgfusepath{fill}%
\end{pgfscope}%
\begin{pgfscope}%
\pgfpathrectangle{\pgfqpoint{1.150000in}{0.150000in}}{\pgfqpoint{5.700000in}{5.700000in}}%
\pgfusepath{clip}%
\pgfsetbuttcap%
\pgfsetroundjoin%
\definecolor{currentfill}{rgb}{0.283229,0.120777,0.440584}%
\pgfsetfillcolor{currentfill}%
\pgfsetfillopacity{0.700000}%
\pgfsetlinewidth{0.000000pt}%
\definecolor{currentstroke}{rgb}{0.000000,0.000000,0.000000}%
\pgfsetstrokecolor{currentstroke}%
\pgfsetdash{}{0pt}%
\pgfpathmoveto{\pgfqpoint{3.340883in}{2.521928in}}%
\pgfpathlineto{\pgfqpoint{3.354250in}{2.514929in}}%
\pgfpathlineto{\pgfqpoint{3.367620in}{2.508032in}}%
\pgfpathlineto{\pgfqpoint{3.380993in}{2.501235in}}%
\pgfpathlineto{\pgfqpoint{3.394369in}{2.494537in}}%
\pgfpathlineto{\pgfqpoint{3.402371in}{2.504990in}}%
\pgfpathlineto{\pgfqpoint{3.410366in}{2.515528in}}%
\pgfpathlineto{\pgfqpoint{3.418356in}{2.526154in}}%
\pgfpathlineto{\pgfqpoint{3.426339in}{2.536870in}}%
\pgfpathlineto{\pgfqpoint{3.412972in}{2.543704in}}%
\pgfpathlineto{\pgfqpoint{3.399608in}{2.550638in}}%
\pgfpathlineto{\pgfqpoint{3.386248in}{2.557671in}}%
\pgfpathlineto{\pgfqpoint{3.372890in}{2.564806in}}%
\pgfpathlineto{\pgfqpoint{3.364897in}{2.553946in}}%
\pgfpathlineto{\pgfqpoint{3.356899in}{2.543181in}}%
\pgfpathlineto{\pgfqpoint{3.348894in}{2.532510in}}%
\pgfpathlineto{\pgfqpoint{3.340883in}{2.521928in}}%
\pgfpathclose%
\pgfusepath{fill}%
\end{pgfscope}%
\begin{pgfscope}%
\pgfpathrectangle{\pgfqpoint{1.150000in}{0.150000in}}{\pgfqpoint{5.700000in}{5.700000in}}%
\pgfusepath{clip}%
\pgfsetbuttcap%
\pgfsetroundjoin%
\definecolor{currentfill}{rgb}{0.233603,0.313828,0.543914}%
\pgfsetfillcolor{currentfill}%
\pgfsetfillopacity{0.700000}%
\pgfsetlinewidth{0.000000pt}%
\definecolor{currentstroke}{rgb}{0.000000,0.000000,0.000000}%
\pgfsetstrokecolor{currentstroke}%
\pgfsetdash{}{0pt}%
\pgfpathmoveto{\pgfqpoint{5.110525in}{2.907367in}}%
\pgfpathlineto{\pgfqpoint{5.124222in}{2.902826in}}%
\pgfpathlineto{\pgfqpoint{5.137926in}{2.898355in}}%
\pgfpathlineto{\pgfqpoint{5.151638in}{2.893952in}}%
\pgfpathlineto{\pgfqpoint{5.165357in}{2.889618in}}%
\pgfpathlineto{\pgfqpoint{5.172918in}{2.905525in}}%
\pgfpathlineto{\pgfqpoint{5.180483in}{2.921782in}}%
\pgfpathlineto{\pgfqpoint{5.188053in}{2.938399in}}%
\pgfpathlineto{\pgfqpoint{5.195628in}{2.955385in}}%
\pgfpathlineto{\pgfqpoint{5.181924in}{2.960196in}}%
\pgfpathlineto{\pgfqpoint{5.168227in}{2.965077in}}%
\pgfpathlineto{\pgfqpoint{5.154537in}{2.970026in}}%
\pgfpathlineto{\pgfqpoint{5.140854in}{2.975044in}}%
\pgfpathlineto{\pgfqpoint{5.133265in}{2.957573in}}%
\pgfpathlineto{\pgfqpoint{5.125680in}{2.940475in}}%
\pgfpathlineto{\pgfqpoint{5.118100in}{2.923743in}}%
\pgfpathlineto{\pgfqpoint{5.110525in}{2.907367in}}%
\pgfpathclose%
\pgfusepath{fill}%
\end{pgfscope}%
\begin{pgfscope}%
\pgfpathrectangle{\pgfqpoint{1.150000in}{0.150000in}}{\pgfqpoint{5.700000in}{5.700000in}}%
\pgfusepath{clip}%
\pgfsetbuttcap%
\pgfsetroundjoin%
\definecolor{currentfill}{rgb}{0.250425,0.274290,0.533103}%
\pgfsetfillcolor{currentfill}%
\pgfsetfillopacity{0.700000}%
\pgfsetlinewidth{0.000000pt}%
\definecolor{currentstroke}{rgb}{0.000000,0.000000,0.000000}%
\pgfsetstrokecolor{currentstroke}%
\pgfsetdash{}{0pt}%
\pgfpathmoveto{\pgfqpoint{4.940479in}{2.819842in}}%
\pgfpathlineto{\pgfqpoint{4.954147in}{2.815637in}}%
\pgfpathlineto{\pgfqpoint{4.967823in}{2.811502in}}%
\pgfpathlineto{\pgfqpoint{4.981506in}{2.807438in}}%
\pgfpathlineto{\pgfqpoint{4.995197in}{2.803444in}}%
\pgfpathlineto{\pgfqpoint{5.002765in}{2.817703in}}%
\pgfpathlineto{\pgfqpoint{5.010335in}{2.832257in}}%
\pgfpathlineto{\pgfqpoint{5.017907in}{2.847114in}}%
\pgfpathlineto{\pgfqpoint{5.025481in}{2.862281in}}%
\pgfpathlineto{\pgfqpoint{5.011805in}{2.866713in}}%
\pgfpathlineto{\pgfqpoint{4.998135in}{2.871214in}}%
\pgfpathlineto{\pgfqpoint{4.984473in}{2.875786in}}%
\pgfpathlineto{\pgfqpoint{4.970818in}{2.880428in}}%
\pgfpathlineto{\pgfqpoint{4.963230in}{2.864816in}}%
\pgfpathlineto{\pgfqpoint{4.955644in}{2.849519in}}%
\pgfpathlineto{\pgfqpoint{4.948061in}{2.834531in}}%
\pgfpathlineto{\pgfqpoint{4.940479in}{2.819842in}}%
\pgfpathclose%
\pgfusepath{fill}%
\end{pgfscope}%
\begin{pgfscope}%
\pgfpathrectangle{\pgfqpoint{1.150000in}{0.150000in}}{\pgfqpoint{5.700000in}{5.700000in}}%
\pgfusepath{clip}%
\pgfsetbuttcap%
\pgfsetroundjoin%
\definecolor{currentfill}{rgb}{0.282623,0.140926,0.457517}%
\pgfsetfillcolor{currentfill}%
\pgfsetfillopacity{0.700000}%
\pgfsetlinewidth{0.000000pt}%
\definecolor{currentstroke}{rgb}{0.000000,0.000000,0.000000}%
\pgfsetstrokecolor{currentstroke}%
\pgfsetdash{}{0pt}%
\pgfpathmoveto{\pgfqpoint{3.062702in}{2.563108in}}%
\pgfpathlineto{\pgfqpoint{3.076055in}{2.554609in}}%
\pgfpathlineto{\pgfqpoint{3.089409in}{2.546222in}}%
\pgfpathlineto{\pgfqpoint{3.102764in}{2.537947in}}%
\pgfpathlineto{\pgfqpoint{3.116121in}{2.529784in}}%
\pgfpathlineto{\pgfqpoint{3.124214in}{2.539998in}}%
\pgfpathlineto{\pgfqpoint{3.132300in}{2.550298in}}%
\pgfpathlineto{\pgfqpoint{3.140380in}{2.560687in}}%
\pgfpathlineto{\pgfqpoint{3.148453in}{2.571166in}}%
\pgfpathlineto{\pgfqpoint{3.135106in}{2.579425in}}%
\pgfpathlineto{\pgfqpoint{3.121761in}{2.587796in}}%
\pgfpathlineto{\pgfqpoint{3.108417in}{2.596278in}}%
\pgfpathlineto{\pgfqpoint{3.095075in}{2.604874in}}%
\pgfpathlineto{\pgfqpoint{3.086993in}{2.594291in}}%
\pgfpathlineto{\pgfqpoint{3.078903in}{2.583804in}}%
\pgfpathlineto{\pgfqpoint{3.070806in}{2.573411in}}%
\pgfpathlineto{\pgfqpoint{3.062702in}{2.563108in}}%
\pgfpathclose%
\pgfusepath{fill}%
\end{pgfscope}%
\begin{pgfscope}%
\pgfpathrectangle{\pgfqpoint{1.150000in}{0.150000in}}{\pgfqpoint{5.700000in}{5.700000in}}%
\pgfusepath{clip}%
\pgfsetbuttcap%
\pgfsetroundjoin%
\definecolor{currentfill}{rgb}{0.223925,0.334994,0.548053}%
\pgfsetfillcolor{currentfill}%
\pgfsetfillopacity{0.700000}%
\pgfsetlinewidth{0.000000pt}%
\definecolor{currentstroke}{rgb}{0.000000,0.000000,0.000000}%
\pgfsetstrokecolor{currentstroke}%
\pgfsetdash{}{0pt}%
\pgfpathmoveto{\pgfqpoint{5.195628in}{2.955385in}}%
\pgfpathlineto{\pgfqpoint{5.209340in}{2.950643in}}%
\pgfpathlineto{\pgfqpoint{5.223058in}{2.945968in}}%
\pgfpathlineto{\pgfqpoint{5.236783in}{2.941362in}}%
\pgfpathlineto{\pgfqpoint{5.250516in}{2.936825in}}%
\pgfpathlineto{\pgfqpoint{5.258081in}{2.953699in}}%
\pgfpathlineto{\pgfqpoint{5.265652in}{2.970954in}}%
\pgfpathlineto{\pgfqpoint{5.273229in}{2.988601in}}%
\pgfpathlineto{\pgfqpoint{5.280813in}{3.006648in}}%
\pgfpathlineto{\pgfqpoint{5.267096in}{3.011683in}}%
\pgfpathlineto{\pgfqpoint{5.253385in}{3.016787in}}%
\pgfpathlineto{\pgfqpoint{5.239681in}{3.021959in}}%
\pgfpathlineto{\pgfqpoint{5.225985in}{3.027199in}}%
\pgfpathlineto{\pgfqpoint{5.218386in}{3.008647in}}%
\pgfpathlineto{\pgfqpoint{5.210795in}{2.990500in}}%
\pgfpathlineto{\pgfqpoint{5.203209in}{2.972749in}}%
\pgfpathlineto{\pgfqpoint{5.195628in}{2.955385in}}%
\pgfpathclose%
\pgfusepath{fill}%
\end{pgfscope}%
\begin{pgfscope}%
\pgfpathrectangle{\pgfqpoint{1.150000in}{0.150000in}}{\pgfqpoint{5.700000in}{5.700000in}}%
\pgfusepath{clip}%
\pgfsetbuttcap%
\pgfsetroundjoin%
\definecolor{currentfill}{rgb}{0.257322,0.256130,0.526563}%
\pgfsetfillcolor{currentfill}%
\pgfsetfillopacity{0.700000}%
\pgfsetlinewidth{0.000000pt}%
\definecolor{currentstroke}{rgb}{0.000000,0.000000,0.000000}%
\pgfsetstrokecolor{currentstroke}%
\pgfsetdash{}{0pt}%
\pgfpathmoveto{\pgfqpoint{4.855501in}{2.779787in}}%
\pgfpathlineto{\pgfqpoint{4.869155in}{2.775715in}}%
\pgfpathlineto{\pgfqpoint{4.882817in}{2.771715in}}%
\pgfpathlineto{\pgfqpoint{4.896485in}{2.767786in}}%
\pgfpathlineto{\pgfqpoint{4.910161in}{2.763927in}}%
\pgfpathlineto{\pgfqpoint{4.917739in}{2.777495in}}%
\pgfpathlineto{\pgfqpoint{4.925318in}{2.791331in}}%
\pgfpathlineto{\pgfqpoint{4.932898in}{2.805444in}}%
\pgfpathlineto{\pgfqpoint{4.940479in}{2.819842in}}%
\pgfpathlineto{\pgfqpoint{4.926817in}{2.824117in}}%
\pgfpathlineto{\pgfqpoint{4.913162in}{2.828464in}}%
\pgfpathlineto{\pgfqpoint{4.899514in}{2.832882in}}%
\pgfpathlineto{\pgfqpoint{4.885873in}{2.837370in}}%
\pgfpathlineto{\pgfqpoint{4.878279in}{2.822549in}}%
\pgfpathlineto{\pgfqpoint{4.870685in}{2.808016in}}%
\pgfpathlineto{\pgfqpoint{4.863093in}{2.793765in}}%
\pgfpathlineto{\pgfqpoint{4.855501in}{2.779787in}}%
\pgfpathclose%
\pgfusepath{fill}%
\end{pgfscope}%
\begin{pgfscope}%
\pgfpathrectangle{\pgfqpoint{1.150000in}{0.150000in}}{\pgfqpoint{5.700000in}{5.700000in}}%
\pgfusepath{clip}%
\pgfsetbuttcap%
\pgfsetroundjoin%
\definecolor{currentfill}{rgb}{0.283229,0.120777,0.440584}%
\pgfsetfillcolor{currentfill}%
\pgfsetfillopacity{0.700000}%
\pgfsetlinewidth{0.000000pt}%
\definecolor{currentstroke}{rgb}{0.000000,0.000000,0.000000}%
\pgfsetstrokecolor{currentstroke}%
\pgfsetdash{}{0pt}%
\pgfpathmoveto{\pgfqpoint{3.479839in}{2.510520in}}%
\pgfpathlineto{\pgfqpoint{3.493222in}{2.504175in}}%
\pgfpathlineto{\pgfqpoint{3.506610in}{2.497926in}}%
\pgfpathlineto{\pgfqpoint{3.520001in}{2.491773in}}%
\pgfpathlineto{\pgfqpoint{3.533395in}{2.485715in}}%
\pgfpathlineto{\pgfqpoint{3.541354in}{2.496227in}}%
\pgfpathlineto{\pgfqpoint{3.549307in}{2.506826in}}%
\pgfpathlineto{\pgfqpoint{3.557254in}{2.517514in}}%
\pgfpathlineto{\pgfqpoint{3.565195in}{2.528295in}}%
\pgfpathlineto{\pgfqpoint{3.551809in}{2.534510in}}%
\pgfpathlineto{\pgfqpoint{3.538428in}{2.540819in}}%
\pgfpathlineto{\pgfqpoint{3.525049in}{2.547224in}}%
\pgfpathlineto{\pgfqpoint{3.511675in}{2.553725in}}%
\pgfpathlineto{\pgfqpoint{3.503725in}{2.542780in}}%
\pgfpathlineto{\pgfqpoint{3.495769in}{2.531933in}}%
\pgfpathlineto{\pgfqpoint{3.487807in}{2.521181in}}%
\pgfpathlineto{\pgfqpoint{3.479839in}{2.510520in}}%
\pgfpathclose%
\pgfusepath{fill}%
\end{pgfscope}%
\begin{pgfscope}%
\pgfpathrectangle{\pgfqpoint{1.150000in}{0.150000in}}{\pgfqpoint{5.700000in}{5.700000in}}%
\pgfusepath{clip}%
\pgfsetbuttcap%
\pgfsetroundjoin%
\definecolor{currentfill}{rgb}{0.278012,0.180367,0.486697}%
\pgfsetfillcolor{currentfill}%
\pgfsetfillopacity{0.700000}%
\pgfsetlinewidth{0.000000pt}%
\definecolor{currentstroke}{rgb}{0.000000,0.000000,0.000000}%
\pgfsetstrokecolor{currentstroke}%
\pgfsetdash{}{0pt}%
\pgfpathmoveto{\pgfqpoint{4.376288in}{2.623653in}}%
\pgfpathlineto{\pgfqpoint{4.389841in}{2.619593in}}%
\pgfpathlineto{\pgfqpoint{4.403401in}{2.615609in}}%
\pgfpathlineto{\pgfqpoint{4.416967in}{2.611702in}}%
\pgfpathlineto{\pgfqpoint{4.430539in}{2.607871in}}%
\pgfpathlineto{\pgfqpoint{4.438226in}{2.619372in}}%
\pgfpathlineto{\pgfqpoint{4.445909in}{2.631039in}}%
\pgfpathlineto{\pgfqpoint{4.453590in}{2.642878in}}%
\pgfpathlineto{\pgfqpoint{4.461267in}{2.654896in}}%
\pgfpathlineto{\pgfqpoint{4.447705in}{2.659044in}}%
\pgfpathlineto{\pgfqpoint{4.434150in}{2.663268in}}%
\pgfpathlineto{\pgfqpoint{4.420602in}{2.667569in}}%
\pgfpathlineto{\pgfqpoint{4.407059in}{2.671946in}}%
\pgfpathlineto{\pgfqpoint{4.399371in}{2.659605in}}%
\pgfpathlineto{\pgfqpoint{4.391680in}{2.647446in}}%
\pgfpathlineto{\pgfqpoint{4.383986in}{2.635464in}}%
\pgfpathlineto{\pgfqpoint{4.376288in}{2.623653in}}%
\pgfpathclose%
\pgfusepath{fill}%
\end{pgfscope}%
\begin{pgfscope}%
\pgfpathrectangle{\pgfqpoint{1.150000in}{0.150000in}}{\pgfqpoint{5.700000in}{5.700000in}}%
\pgfusepath{clip}%
\pgfsetbuttcap%
\pgfsetroundjoin%
\definecolor{currentfill}{rgb}{0.214298,0.355619,0.551184}%
\pgfsetfillcolor{currentfill}%
\pgfsetfillopacity{0.700000}%
\pgfsetlinewidth{0.000000pt}%
\definecolor{currentstroke}{rgb}{0.000000,0.000000,0.000000}%
\pgfsetstrokecolor{currentstroke}%
\pgfsetdash{}{0pt}%
\pgfpathmoveto{\pgfqpoint{5.280813in}{3.006648in}}%
\pgfpathlineto{\pgfqpoint{5.294538in}{3.001680in}}%
\pgfpathlineto{\pgfqpoint{5.308269in}{2.996781in}}%
\pgfpathlineto{\pgfqpoint{5.322008in}{2.991949in}}%
\pgfpathlineto{\pgfqpoint{5.335754in}{2.987185in}}%
\pgfpathlineto{\pgfqpoint{5.343330in}{3.005131in}}%
\pgfpathlineto{\pgfqpoint{5.350912in}{3.023491in}}%
\pgfpathlineto{\pgfqpoint{5.358503in}{3.042275in}}%
\pgfpathlineto{\pgfqpoint{5.366102in}{3.061493in}}%
\pgfpathlineto{\pgfqpoint{5.352372in}{3.066775in}}%
\pgfpathlineto{\pgfqpoint{5.338648in}{3.072125in}}%
\pgfpathlineto{\pgfqpoint{5.324931in}{3.077542in}}%
\pgfpathlineto{\pgfqpoint{5.311221in}{3.083028in}}%
\pgfpathlineto{\pgfqpoint{5.303608in}{3.063285in}}%
\pgfpathlineto{\pgfqpoint{5.296002in}{3.043980in}}%
\pgfpathlineto{\pgfqpoint{5.288404in}{3.025104in}}%
\pgfpathlineto{\pgfqpoint{5.280813in}{3.006648in}}%
\pgfpathclose%
\pgfusepath{fill}%
\end{pgfscope}%
\begin{pgfscope}%
\pgfpathrectangle{\pgfqpoint{1.150000in}{0.150000in}}{\pgfqpoint{5.700000in}{5.700000in}}%
\pgfusepath{clip}%
\pgfsetbuttcap%
\pgfsetroundjoin%
\definecolor{currentfill}{rgb}{0.283072,0.130895,0.449241}%
\pgfsetfillcolor{currentfill}%
\pgfsetfillopacity{0.700000}%
\pgfsetlinewidth{0.000000pt}%
\definecolor{currentstroke}{rgb}{0.000000,0.000000,0.000000}%
\pgfsetstrokecolor{currentstroke}%
\pgfsetdash{}{0pt}%
\pgfpathmoveto{\pgfqpoint{3.842972in}{2.525360in}}%
\pgfpathlineto{\pgfqpoint{3.856415in}{2.520308in}}%
\pgfpathlineto{\pgfqpoint{3.869862in}{2.515342in}}%
\pgfpathlineto{\pgfqpoint{3.883315in}{2.510461in}}%
\pgfpathlineto{\pgfqpoint{3.896772in}{2.505667in}}%
\pgfpathlineto{\pgfqpoint{3.904618in}{2.516391in}}%
\pgfpathlineto{\pgfqpoint{3.912459in}{2.527219in}}%
\pgfpathlineto{\pgfqpoint{3.920294in}{2.538155in}}%
\pgfpathlineto{\pgfqpoint{3.928125in}{2.549202in}}%
\pgfpathlineto{\pgfqpoint{3.914676in}{2.554214in}}%
\pgfpathlineto{\pgfqpoint{3.901233in}{2.559311in}}%
\pgfpathlineto{\pgfqpoint{3.887795in}{2.564493in}}%
\pgfpathlineto{\pgfqpoint{3.874361in}{2.569762in}}%
\pgfpathlineto{\pgfqpoint{3.866522in}{2.558491in}}%
\pgfpathlineto{\pgfqpoint{3.858677in}{2.547336in}}%
\pgfpathlineto{\pgfqpoint{3.850827in}{2.536293in}}%
\pgfpathlineto{\pgfqpoint{3.842972in}{2.525360in}}%
\pgfpathclose%
\pgfusepath{fill}%
\end{pgfscope}%
\begin{pgfscope}%
\pgfpathrectangle{\pgfqpoint{1.150000in}{0.150000in}}{\pgfqpoint{5.700000in}{5.700000in}}%
\pgfusepath{clip}%
\pgfsetbuttcap%
\pgfsetroundjoin%
\definecolor{currentfill}{rgb}{0.263663,0.237631,0.518762}%
\pgfsetfillcolor{currentfill}%
\pgfsetfillopacity{0.700000}%
\pgfsetlinewidth{0.000000pt}%
\definecolor{currentstroke}{rgb}{0.000000,0.000000,0.000000}%
\pgfsetstrokecolor{currentstroke}%
\pgfsetdash{}{0pt}%
\pgfpathmoveto{\pgfqpoint{4.770534in}{2.741881in}}%
\pgfpathlineto{\pgfqpoint{4.784174in}{2.737920in}}%
\pgfpathlineto{\pgfqpoint{4.797820in}{2.734031in}}%
\pgfpathlineto{\pgfqpoint{4.811474in}{2.730213in}}%
\pgfpathlineto{\pgfqpoint{4.825134in}{2.726467in}}%
\pgfpathlineto{\pgfqpoint{4.832726in}{2.739423in}}%
\pgfpathlineto{\pgfqpoint{4.840318in}{2.752624in}}%
\pgfpathlineto{\pgfqpoint{4.847910in}{2.766076in}}%
\pgfpathlineto{\pgfqpoint{4.855501in}{2.779787in}}%
\pgfpathlineto{\pgfqpoint{4.841854in}{2.783930in}}%
\pgfpathlineto{\pgfqpoint{4.828214in}{2.788145in}}%
\pgfpathlineto{\pgfqpoint{4.814580in}{2.792431in}}%
\pgfpathlineto{\pgfqpoint{4.800953in}{2.796790in}}%
\pgfpathlineto{\pgfqpoint{4.793349in}{2.782674in}}%
\pgfpathlineto{\pgfqpoint{4.785744in}{2.768822in}}%
\pgfpathlineto{\pgfqpoint{4.778140in}{2.755227in}}%
\pgfpathlineto{\pgfqpoint{4.770534in}{2.741881in}}%
\pgfpathclose%
\pgfusepath{fill}%
\end{pgfscope}%
\begin{pgfscope}%
\pgfpathrectangle{\pgfqpoint{1.150000in}{0.150000in}}{\pgfqpoint{5.700000in}{5.700000in}}%
\pgfusepath{clip}%
\pgfsetbuttcap%
\pgfsetroundjoin%
\definecolor{currentfill}{rgb}{0.282290,0.145912,0.461510}%
\pgfsetfillcolor{currentfill}%
\pgfsetfillopacity{0.700000}%
\pgfsetlinewidth{0.000000pt}%
\definecolor{currentstroke}{rgb}{0.000000,0.000000,0.000000}%
\pgfsetstrokecolor{currentstroke}%
\pgfsetdash{}{0pt}%
\pgfpathmoveto{\pgfqpoint{4.067099in}{2.555637in}}%
\pgfpathlineto{\pgfqpoint{4.080587in}{2.551139in}}%
\pgfpathlineto{\pgfqpoint{4.094080in}{2.546723in}}%
\pgfpathlineto{\pgfqpoint{4.107579in}{2.542389in}}%
\pgfpathlineto{\pgfqpoint{4.121083in}{2.538136in}}%
\pgfpathlineto{\pgfqpoint{4.128861in}{2.549060in}}%
\pgfpathlineto{\pgfqpoint{4.136634in}{2.560107in}}%
\pgfpathlineto{\pgfqpoint{4.144402in}{2.571283in}}%
\pgfpathlineto{\pgfqpoint{4.152166in}{2.582593in}}%
\pgfpathlineto{\pgfqpoint{4.138671in}{2.587103in}}%
\pgfpathlineto{\pgfqpoint{4.125182in}{2.591694in}}%
\pgfpathlineto{\pgfqpoint{4.111698in}{2.596367in}}%
\pgfpathlineto{\pgfqpoint{4.098220in}{2.601121in}}%
\pgfpathlineto{\pgfqpoint{4.090447in}{2.589548in}}%
\pgfpathlineto{\pgfqpoint{4.082669in}{2.578112in}}%
\pgfpathlineto{\pgfqpoint{4.074886in}{2.566810in}}%
\pgfpathlineto{\pgfqpoint{4.067099in}{2.555637in}}%
\pgfpathclose%
\pgfusepath{fill}%
\end{pgfscope}%
\begin{pgfscope}%
\pgfpathrectangle{\pgfqpoint{1.150000in}{0.150000in}}{\pgfqpoint{5.700000in}{5.700000in}}%
\pgfusepath{clip}%
\pgfsetbuttcap%
\pgfsetroundjoin%
\definecolor{currentfill}{rgb}{0.275191,0.194905,0.496005}%
\pgfsetfillcolor{currentfill}%
\pgfsetfillopacity{0.700000}%
\pgfsetlinewidth{0.000000pt}%
\definecolor{currentstroke}{rgb}{0.000000,0.000000,0.000000}%
\pgfsetstrokecolor{currentstroke}%
\pgfsetdash{}{0pt}%
\pgfpathmoveto{\pgfqpoint{2.730240in}{2.676304in}}%
\pgfpathlineto{\pgfqpoint{2.743611in}{2.665475in}}%
\pgfpathlineto{\pgfqpoint{2.756981in}{2.654778in}}%
\pgfpathlineto{\pgfqpoint{2.770351in}{2.644214in}}%
\pgfpathlineto{\pgfqpoint{2.783719in}{2.633780in}}%
\pgfpathlineto{\pgfqpoint{2.791928in}{2.643605in}}%
\pgfpathlineto{\pgfqpoint{2.800129in}{2.653528in}}%
\pgfpathlineto{\pgfqpoint{2.808322in}{2.663551in}}%
\pgfpathlineto{\pgfqpoint{2.816507in}{2.673674in}}%
\pgfpathlineto{\pgfqpoint{2.803150in}{2.684163in}}%
\pgfpathlineto{\pgfqpoint{2.789793in}{2.694782in}}%
\pgfpathlineto{\pgfqpoint{2.776435in}{2.705533in}}%
\pgfpathlineto{\pgfqpoint{2.763076in}{2.716417in}}%
\pgfpathlineto{\pgfqpoint{2.754879in}{2.706232in}}%
\pgfpathlineto{\pgfqpoint{2.746674in}{2.696152in}}%
\pgfpathlineto{\pgfqpoint{2.738461in}{2.686176in}}%
\pgfpathlineto{\pgfqpoint{2.730240in}{2.676304in}}%
\pgfpathclose%
\pgfusepath{fill}%
\end{pgfscope}%
\begin{pgfscope}%
\pgfpathrectangle{\pgfqpoint{1.150000in}{0.150000in}}{\pgfqpoint{5.700000in}{5.700000in}}%
\pgfusepath{clip}%
\pgfsetbuttcap%
\pgfsetroundjoin%
\definecolor{currentfill}{rgb}{0.280868,0.160771,0.472899}%
\pgfsetfillcolor{currentfill}%
\pgfsetfillopacity{0.700000}%
\pgfsetlinewidth{0.000000pt}%
\definecolor{currentstroke}{rgb}{0.000000,0.000000,0.000000}%
\pgfsetstrokecolor{currentstroke}%
\pgfsetdash{}{0pt}%
\pgfpathmoveto{\pgfqpoint{2.923349in}{2.594343in}}%
\pgfpathlineto{\pgfqpoint{2.936705in}{2.584983in}}%
\pgfpathlineto{\pgfqpoint{2.950062in}{2.575743in}}%
\pgfpathlineto{\pgfqpoint{2.963419in}{2.566622in}}%
\pgfpathlineto{\pgfqpoint{2.976777in}{2.557620in}}%
\pgfpathlineto{\pgfqpoint{2.984920in}{2.567643in}}%
\pgfpathlineto{\pgfqpoint{2.993055in}{2.577756in}}%
\pgfpathlineto{\pgfqpoint{3.001184in}{2.587959in}}%
\pgfpathlineto{\pgfqpoint{3.009305in}{2.598254in}}%
\pgfpathlineto{\pgfqpoint{2.995958in}{2.607332in}}%
\pgfpathlineto{\pgfqpoint{2.982612in}{2.616528in}}%
\pgfpathlineto{\pgfqpoint{2.969267in}{2.625843in}}%
\pgfpathlineto{\pgfqpoint{2.955922in}{2.635279in}}%
\pgfpathlineto{\pgfqpoint{2.947790in}{2.624900in}}%
\pgfpathlineto{\pgfqpoint{2.939650in}{2.614619in}}%
\pgfpathlineto{\pgfqpoint{2.931503in}{2.604434in}}%
\pgfpathlineto{\pgfqpoint{2.923349in}{2.594343in}}%
\pgfpathclose%
\pgfusepath{fill}%
\end{pgfscope}%
\begin{pgfscope}%
\pgfpathrectangle{\pgfqpoint{1.150000in}{0.150000in}}{\pgfqpoint{5.700000in}{5.700000in}}%
\pgfusepath{clip}%
\pgfsetbuttcap%
\pgfsetroundjoin%
\definecolor{currentfill}{rgb}{0.283197,0.115680,0.436115}%
\pgfsetfillcolor{currentfill}%
\pgfsetfillopacity{0.700000}%
\pgfsetlinewidth{0.000000pt}%
\definecolor{currentstroke}{rgb}{0.000000,0.000000,0.000000}%
\pgfsetstrokecolor{currentstroke}%
\pgfsetdash{}{0pt}%
\pgfpathmoveto{\pgfqpoint{3.618777in}{2.504374in}}%
\pgfpathlineto{\pgfqpoint{3.632182in}{2.498626in}}%
\pgfpathlineto{\pgfqpoint{3.645592in}{2.492969in}}%
\pgfpathlineto{\pgfqpoint{3.659006in}{2.487404in}}%
\pgfpathlineto{\pgfqpoint{3.672425in}{2.481930in}}%
\pgfpathlineto{\pgfqpoint{3.680342in}{2.492468in}}%
\pgfpathlineto{\pgfqpoint{3.688253in}{2.503096in}}%
\pgfpathlineto{\pgfqpoint{3.696159in}{2.513817in}}%
\pgfpathlineto{\pgfqpoint{3.704059in}{2.524633in}}%
\pgfpathlineto{\pgfqpoint{3.690650in}{2.530285in}}%
\pgfpathlineto{\pgfqpoint{3.677245in}{2.536026in}}%
\pgfpathlineto{\pgfqpoint{3.663844in}{2.541859in}}%
\pgfpathlineto{\pgfqpoint{3.650447in}{2.547784in}}%
\pgfpathlineto{\pgfqpoint{3.642538in}{2.536784in}}%
\pgfpathlineto{\pgfqpoint{3.634623in}{2.525884in}}%
\pgfpathlineto{\pgfqpoint{3.626703in}{2.515082in}}%
\pgfpathlineto{\pgfqpoint{3.618777in}{2.504374in}}%
\pgfpathclose%
\pgfusepath{fill}%
\end{pgfscope}%
\begin{pgfscope}%
\pgfpathrectangle{\pgfqpoint{1.150000in}{0.150000in}}{\pgfqpoint{5.700000in}{5.700000in}}%
\pgfusepath{clip}%
\pgfsetbuttcap%
\pgfsetroundjoin%
\definecolor{currentfill}{rgb}{0.269308,0.218818,0.509577}%
\pgfsetfillcolor{currentfill}%
\pgfsetfillopacity{0.700000}%
\pgfsetlinewidth{0.000000pt}%
\definecolor{currentstroke}{rgb}{0.000000,0.000000,0.000000}%
\pgfsetstrokecolor{currentstroke}%
\pgfsetdash{}{0pt}%
\pgfpathmoveto{\pgfqpoint{4.685565in}{2.705913in}}%
\pgfpathlineto{\pgfqpoint{4.699190in}{2.702038in}}%
\pgfpathlineto{\pgfqpoint{4.712821in}{2.698237in}}%
\pgfpathlineto{\pgfqpoint{4.726459in}{2.694508in}}%
\pgfpathlineto{\pgfqpoint{4.740105in}{2.690852in}}%
\pgfpathlineto{\pgfqpoint{4.747714in}{2.703270in}}%
\pgfpathlineto{\pgfqpoint{4.755322in}{2.715910in}}%
\pgfpathlineto{\pgfqpoint{4.762928in}{2.728778in}}%
\pgfpathlineto{\pgfqpoint{4.770534in}{2.741881in}}%
\pgfpathlineto{\pgfqpoint{4.756902in}{2.745915in}}%
\pgfpathlineto{\pgfqpoint{4.743276in}{2.750021in}}%
\pgfpathlineto{\pgfqpoint{4.729658in}{2.754200in}}%
\pgfpathlineto{\pgfqpoint{4.716045in}{2.758451in}}%
\pgfpathlineto{\pgfqpoint{4.708427in}{2.744963in}}%
\pgfpathlineto{\pgfqpoint{4.700808in}{2.731715in}}%
\pgfpathlineto{\pgfqpoint{4.693187in}{2.718701in}}%
\pgfpathlineto{\pgfqpoint{4.685565in}{2.705913in}}%
\pgfpathclose%
\pgfusepath{fill}%
\end{pgfscope}%
\begin{pgfscope}%
\pgfpathrectangle{\pgfqpoint{1.150000in}{0.150000in}}{\pgfqpoint{5.700000in}{5.700000in}}%
\pgfusepath{clip}%
\pgfsetbuttcap%
\pgfsetroundjoin%
\definecolor{currentfill}{rgb}{0.179019,0.433756,0.557430}%
\pgfsetfillcolor{currentfill}%
\pgfsetfillopacity{0.700000}%
\pgfsetlinewidth{0.000000pt}%
\definecolor{currentstroke}{rgb}{0.000000,0.000000,0.000000}%
\pgfsetstrokecolor{currentstroke}%
\pgfsetdash{}{0pt}%
\pgfpathmoveto{\pgfqpoint{5.482120in}{3.207196in}}%
\pgfpathlineto{\pgfqpoint{5.495856in}{3.201151in}}%
\pgfpathlineto{\pgfqpoint{5.509598in}{3.195174in}}%
\pgfpathlineto{\pgfqpoint{5.523347in}{3.189263in}}%
\pgfpathlineto{\pgfqpoint{5.537102in}{3.183418in}}%
\pgfpathlineto{\pgfqpoint{5.544767in}{3.205850in}}%
\pgfpathlineto{\pgfqpoint{5.552445in}{3.228808in}}%
\pgfpathlineto{\pgfqpoint{5.560138in}{3.252304in}}%
\pgfpathlineto{\pgfqpoint{5.546393in}{3.258582in}}%
\pgfpathlineto{\pgfqpoint{5.532655in}{3.264927in}}%
\pgfpathlineto{\pgfqpoint{5.518924in}{3.271338in}}%
\pgfpathlineto{\pgfqpoint{5.505199in}{3.277816in}}%
\pgfpathlineto{\pgfqpoint{5.497493in}{3.253736in}}%
\pgfpathlineto{\pgfqpoint{5.489800in}{3.230200in}}%
\pgfpathlineto{\pgfqpoint{5.482120in}{3.207196in}}%
\pgfpathclose%
\pgfusepath{fill}%
\end{pgfscope}%
\begin{pgfscope}%
\pgfpathrectangle{\pgfqpoint{1.150000in}{0.150000in}}{\pgfqpoint{5.700000in}{5.700000in}}%
\pgfusepath{clip}%
\pgfsetbuttcap%
\pgfsetroundjoin%
\definecolor{currentfill}{rgb}{0.203063,0.379716,0.553925}%
\pgfsetfillcolor{currentfill}%
\pgfsetfillopacity{0.700000}%
\pgfsetlinewidth{0.000000pt}%
\definecolor{currentstroke}{rgb}{0.000000,0.000000,0.000000}%
\pgfsetstrokecolor{currentstroke}%
\pgfsetdash{}{0pt}%
\pgfpathmoveto{\pgfqpoint{5.366102in}{3.061493in}}%
\pgfpathlineto{\pgfqpoint{5.379840in}{3.056278in}}%
\pgfpathlineto{\pgfqpoint{5.393585in}{3.051130in}}%
\pgfpathlineto{\pgfqpoint{5.407336in}{3.046050in}}%
\pgfpathlineto{\pgfqpoint{5.421095in}{3.041037in}}%
\pgfpathlineto{\pgfqpoint{5.428688in}{3.060167in}}%
\pgfpathlineto{\pgfqpoint{5.436289in}{3.079745in}}%
\pgfpathlineto{\pgfqpoint{5.443901in}{3.099781in}}%
\pgfpathlineto{\pgfqpoint{5.451522in}{3.120285in}}%
\pgfpathlineto{\pgfqpoint{5.437779in}{3.125837in}}%
\pgfpathlineto{\pgfqpoint{5.424043in}{3.131456in}}%
\pgfpathlineto{\pgfqpoint{5.410313in}{3.137142in}}%
\pgfpathlineto{\pgfqpoint{5.396590in}{3.142896in}}%
\pgfpathlineto{\pgfqpoint{5.388954in}{3.121845in}}%
\pgfpathlineto{\pgfqpoint{5.381327in}{3.101267in}}%
\pgfpathlineto{\pgfqpoint{5.373710in}{3.081153in}}%
\pgfpathlineto{\pgfqpoint{5.366102in}{3.061493in}}%
\pgfpathclose%
\pgfusepath{fill}%
\end{pgfscope}%
\begin{pgfscope}%
\pgfpathrectangle{\pgfqpoint{1.150000in}{0.150000in}}{\pgfqpoint{5.700000in}{5.700000in}}%
\pgfusepath{clip}%
\pgfsetbuttcap%
\pgfsetroundjoin%
\definecolor{currentfill}{rgb}{0.280255,0.165693,0.476498}%
\pgfsetfillcolor{currentfill}%
\pgfsetfillopacity{0.700000}%
\pgfsetlinewidth{0.000000pt}%
\definecolor{currentstroke}{rgb}{0.000000,0.000000,0.000000}%
\pgfsetstrokecolor{currentstroke}%
\pgfsetdash{}{0pt}%
\pgfpathmoveto{\pgfqpoint{4.291270in}{2.593837in}}%
\pgfpathlineto{\pgfqpoint{4.304808in}{2.589764in}}%
\pgfpathlineto{\pgfqpoint{4.318353in}{2.585769in}}%
\pgfpathlineto{\pgfqpoint{4.331904in}{2.581851in}}%
\pgfpathlineto{\pgfqpoint{4.345462in}{2.578011in}}%
\pgfpathlineto{\pgfqpoint{4.353174in}{2.589192in}}%
\pgfpathlineto{\pgfqpoint{4.360883in}{2.600523in}}%
\pgfpathlineto{\pgfqpoint{4.368587in}{2.612008in}}%
\pgfpathlineto{\pgfqpoint{4.376288in}{2.623653in}}%
\pgfpathlineto{\pgfqpoint{4.362742in}{2.627791in}}%
\pgfpathlineto{\pgfqpoint{4.349201in}{2.632006in}}%
\pgfpathlineto{\pgfqpoint{4.335667in}{2.636298in}}%
\pgfpathlineto{\pgfqpoint{4.322139in}{2.640668in}}%
\pgfpathlineto{\pgfqpoint{4.314427in}{2.628719in}}%
\pgfpathlineto{\pgfqpoint{4.306712in}{2.616934in}}%
\pgfpathlineto{\pgfqpoint{4.298993in}{2.605308in}}%
\pgfpathlineto{\pgfqpoint{4.291270in}{2.593837in}}%
\pgfpathclose%
\pgfusepath{fill}%
\end{pgfscope}%
\begin{pgfscope}%
\pgfpathrectangle{\pgfqpoint{1.150000in}{0.150000in}}{\pgfqpoint{5.700000in}{5.700000in}}%
\pgfusepath{clip}%
\pgfsetbuttcap%
\pgfsetroundjoin%
\definecolor{currentfill}{rgb}{0.283229,0.120777,0.440584}%
\pgfsetfillcolor{currentfill}%
\pgfsetfillopacity{0.700000}%
\pgfsetlinewidth{0.000000pt}%
\definecolor{currentstroke}{rgb}{0.000000,0.000000,0.000000}%
\pgfsetstrokecolor{currentstroke}%
\pgfsetdash{}{0pt}%
\pgfpathmoveto{\pgfqpoint{3.255297in}{2.509003in}}%
\pgfpathlineto{\pgfqpoint{3.268662in}{2.501710in}}%
\pgfpathlineto{\pgfqpoint{3.282031in}{2.494521in}}%
\pgfpathlineto{\pgfqpoint{3.295402in}{2.487435in}}%
\pgfpathlineto{\pgfqpoint{3.308776in}{2.480451in}}%
\pgfpathlineto{\pgfqpoint{3.316812in}{2.490698in}}%
\pgfpathlineto{\pgfqpoint{3.324842in}{2.501024in}}%
\pgfpathlineto{\pgfqpoint{3.332866in}{2.511434in}}%
\pgfpathlineto{\pgfqpoint{3.340883in}{2.521928in}}%
\pgfpathlineto{\pgfqpoint{3.327519in}{2.529028in}}%
\pgfpathlineto{\pgfqpoint{3.314158in}{2.536230in}}%
\pgfpathlineto{\pgfqpoint{3.300799in}{2.543536in}}%
\pgfpathlineto{\pgfqpoint{3.287443in}{2.550945in}}%
\pgfpathlineto{\pgfqpoint{3.279416in}{2.540327in}}%
\pgfpathlineto{\pgfqpoint{3.271383in}{2.529799in}}%
\pgfpathlineto{\pgfqpoint{3.263343in}{2.519359in}}%
\pgfpathlineto{\pgfqpoint{3.255297in}{2.509003in}}%
\pgfpathclose%
\pgfusepath{fill}%
\end{pgfscope}%
\begin{pgfscope}%
\pgfpathrectangle{\pgfqpoint{1.150000in}{0.150000in}}{\pgfqpoint{5.700000in}{5.700000in}}%
\pgfusepath{clip}%
\pgfsetbuttcap%
\pgfsetroundjoin%
\definecolor{currentfill}{rgb}{0.273006,0.204520,0.501721}%
\pgfsetfillcolor{currentfill}%
\pgfsetfillopacity{0.700000}%
\pgfsetlinewidth{0.000000pt}%
\definecolor{currentstroke}{rgb}{0.000000,0.000000,0.000000}%
\pgfsetstrokecolor{currentstroke}%
\pgfsetdash{}{0pt}%
\pgfpathmoveto{\pgfqpoint{4.600583in}{2.671693in}}%
\pgfpathlineto{\pgfqpoint{4.614192in}{2.667882in}}%
\pgfpathlineto{\pgfqpoint{4.627808in}{2.664145in}}%
\pgfpathlineto{\pgfqpoint{4.641431in}{2.660482in}}%
\pgfpathlineto{\pgfqpoint{4.655061in}{2.656892in}}%
\pgfpathlineto{\pgfqpoint{4.662690in}{2.668840in}}%
\pgfpathlineto{\pgfqpoint{4.670317in}{2.680989in}}%
\pgfpathlineto{\pgfqpoint{4.677942in}{2.693344in}}%
\pgfpathlineto{\pgfqpoint{4.685565in}{2.705913in}}%
\pgfpathlineto{\pgfqpoint{4.671948in}{2.709860in}}%
\pgfpathlineto{\pgfqpoint{4.658337in}{2.713881in}}%
\pgfpathlineto{\pgfqpoint{4.644733in}{2.717975in}}%
\pgfpathlineto{\pgfqpoint{4.631136in}{2.722143in}}%
\pgfpathlineto{\pgfqpoint{4.623500in}{2.709210in}}%
\pgfpathlineto{\pgfqpoint{4.615863in}{2.696495in}}%
\pgfpathlineto{\pgfqpoint{4.608224in}{2.683992in}}%
\pgfpathlineto{\pgfqpoint{4.600583in}{2.671693in}}%
\pgfpathclose%
\pgfusepath{fill}%
\end{pgfscope}%
\begin{pgfscope}%
\pgfpathrectangle{\pgfqpoint{1.150000in}{0.150000in}}{\pgfqpoint{5.700000in}{5.700000in}}%
\pgfusepath{clip}%
\pgfsetbuttcap%
\pgfsetroundjoin%
\definecolor{currentfill}{rgb}{0.283072,0.130895,0.449241}%
\pgfsetfillcolor{currentfill}%
\pgfsetfillopacity{0.700000}%
\pgfsetlinewidth{0.000000pt}%
\definecolor{currentstroke}{rgb}{0.000000,0.000000,0.000000}%
\pgfsetstrokecolor{currentstroke}%
\pgfsetdash{}{0pt}%
\pgfpathmoveto{\pgfqpoint{3.116121in}{2.529784in}}%
\pgfpathlineto{\pgfqpoint{3.129479in}{2.521731in}}%
\pgfpathlineto{\pgfqpoint{3.142840in}{2.513788in}}%
\pgfpathlineto{\pgfqpoint{3.156202in}{2.505953in}}%
\pgfpathlineto{\pgfqpoint{3.169567in}{2.498227in}}%
\pgfpathlineto{\pgfqpoint{3.177649in}{2.508353in}}%
\pgfpathlineto{\pgfqpoint{3.185725in}{2.518560in}}%
\pgfpathlineto{\pgfqpoint{3.193795in}{2.528850in}}%
\pgfpathlineto{\pgfqpoint{3.201858in}{2.539226in}}%
\pgfpathlineto{\pgfqpoint{3.188503in}{2.547048in}}%
\pgfpathlineto{\pgfqpoint{3.175151in}{2.554979in}}%
\pgfpathlineto{\pgfqpoint{3.161801in}{2.563018in}}%
\pgfpathlineto{\pgfqpoint{3.148453in}{2.571166in}}%
\pgfpathlineto{\pgfqpoint{3.140380in}{2.560687in}}%
\pgfpathlineto{\pgfqpoint{3.132300in}{2.550298in}}%
\pgfpathlineto{\pgfqpoint{3.124214in}{2.539998in}}%
\pgfpathlineto{\pgfqpoint{3.116121in}{2.529784in}}%
\pgfpathclose%
\pgfusepath{fill}%
\end{pgfscope}%
\begin{pgfscope}%
\pgfpathrectangle{\pgfqpoint{1.150000in}{0.150000in}}{\pgfqpoint{5.700000in}{5.700000in}}%
\pgfusepath{clip}%
\pgfsetbuttcap%
\pgfsetroundjoin%
\definecolor{currentfill}{rgb}{0.282884,0.135920,0.453427}%
\pgfsetfillcolor{currentfill}%
\pgfsetfillopacity{0.700000}%
\pgfsetlinewidth{0.000000pt}%
\definecolor{currentstroke}{rgb}{0.000000,0.000000,0.000000}%
\pgfsetstrokecolor{currentstroke}%
\pgfsetdash{}{0pt}%
\pgfpathmoveto{\pgfqpoint{3.981970in}{2.530002in}}%
\pgfpathlineto{\pgfqpoint{3.995445in}{2.525411in}}%
\pgfpathlineto{\pgfqpoint{4.008925in}{2.520904in}}%
\pgfpathlineto{\pgfqpoint{4.022411in}{2.516479in}}%
\pgfpathlineto{\pgfqpoint{4.035903in}{2.512137in}}%
\pgfpathlineto{\pgfqpoint{4.043709in}{2.522841in}}%
\pgfpathlineto{\pgfqpoint{4.051511in}{2.533657in}}%
\pgfpathlineto{\pgfqpoint{4.059307in}{2.544587in}}%
\pgfpathlineto{\pgfqpoint{4.067099in}{2.555637in}}%
\pgfpathlineto{\pgfqpoint{4.053617in}{2.560216in}}%
\pgfpathlineto{\pgfqpoint{4.040141in}{2.564878in}}%
\pgfpathlineto{\pgfqpoint{4.026670in}{2.569622in}}%
\pgfpathlineto{\pgfqpoint{4.013205in}{2.574449in}}%
\pgfpathlineto{\pgfqpoint{4.005404in}{2.563155in}}%
\pgfpathlineto{\pgfqpoint{3.997597in}{2.551986in}}%
\pgfpathlineto{\pgfqpoint{3.989786in}{2.540936in}}%
\pgfpathlineto{\pgfqpoint{3.981970in}{2.530002in}}%
\pgfpathclose%
\pgfusepath{fill}%
\end{pgfscope}%
\begin{pgfscope}%
\pgfpathrectangle{\pgfqpoint{1.150000in}{0.150000in}}{\pgfqpoint{5.700000in}{5.700000in}}%
\pgfusepath{clip}%
\pgfsetbuttcap%
\pgfsetroundjoin%
\definecolor{currentfill}{rgb}{0.283229,0.120777,0.440584}%
\pgfsetfillcolor{currentfill}%
\pgfsetfillopacity{0.700000}%
\pgfsetlinewidth{0.000000pt}%
\definecolor{currentstroke}{rgb}{0.000000,0.000000,0.000000}%
\pgfsetstrokecolor{currentstroke}%
\pgfsetdash{}{0pt}%
\pgfpathmoveto{\pgfqpoint{3.757742in}{2.502927in}}%
\pgfpathlineto{\pgfqpoint{3.771174in}{2.497722in}}%
\pgfpathlineto{\pgfqpoint{3.784611in}{2.492605in}}%
\pgfpathlineto{\pgfqpoint{3.798052in}{2.487576in}}%
\pgfpathlineto{\pgfqpoint{3.811499in}{2.482634in}}%
\pgfpathlineto{\pgfqpoint{3.819376in}{2.493172in}}%
\pgfpathlineto{\pgfqpoint{3.827247in}{2.503803in}}%
\pgfpathlineto{\pgfqpoint{3.835112in}{2.514531in}}%
\pgfpathlineto{\pgfqpoint{3.842972in}{2.525360in}}%
\pgfpathlineto{\pgfqpoint{3.829535in}{2.530499in}}%
\pgfpathlineto{\pgfqpoint{3.816102in}{2.535725in}}%
\pgfpathlineto{\pgfqpoint{3.802675in}{2.541038in}}%
\pgfpathlineto{\pgfqpoint{3.789251in}{2.546439in}}%
\pgfpathlineto{\pgfqpoint{3.781382in}{2.535406in}}%
\pgfpathlineto{\pgfqpoint{3.773507in}{2.524479in}}%
\pgfpathlineto{\pgfqpoint{3.765627in}{2.513654in}}%
\pgfpathlineto{\pgfqpoint{3.757742in}{2.502927in}}%
\pgfpathclose%
\pgfusepath{fill}%
\end{pgfscope}%
\begin{pgfscope}%
\pgfpathrectangle{\pgfqpoint{1.150000in}{0.150000in}}{\pgfqpoint{5.700000in}{5.700000in}}%
\pgfusepath{clip}%
\pgfsetbuttcap%
\pgfsetroundjoin%
\definecolor{currentfill}{rgb}{0.283197,0.115680,0.436115}%
\pgfsetfillcolor{currentfill}%
\pgfsetfillopacity{0.700000}%
\pgfsetlinewidth{0.000000pt}%
\definecolor{currentstroke}{rgb}{0.000000,0.000000,0.000000}%
\pgfsetstrokecolor{currentstroke}%
\pgfsetdash{}{0pt}%
\pgfpathmoveto{\pgfqpoint{3.394369in}{2.494537in}}%
\pgfpathlineto{\pgfqpoint{3.407749in}{2.487939in}}%
\pgfpathlineto{\pgfqpoint{3.421131in}{2.481439in}}%
\pgfpathlineto{\pgfqpoint{3.434517in}{2.475037in}}%
\pgfpathlineto{\pgfqpoint{3.447907in}{2.468732in}}%
\pgfpathlineto{\pgfqpoint{3.455899in}{2.479056in}}%
\pgfpathlineto{\pgfqpoint{3.463885in}{2.489460in}}%
\pgfpathlineto{\pgfqpoint{3.471865in}{2.499947in}}%
\pgfpathlineto{\pgfqpoint{3.479839in}{2.510520in}}%
\pgfpathlineto{\pgfqpoint{3.466459in}{2.516961in}}%
\pgfpathlineto{\pgfqpoint{3.453082in}{2.523499in}}%
\pgfpathlineto{\pgfqpoint{3.439709in}{2.530136in}}%
\pgfpathlineto{\pgfqpoint{3.426339in}{2.536870in}}%
\pgfpathlineto{\pgfqpoint{3.418356in}{2.526154in}}%
\pgfpathlineto{\pgfqpoint{3.410366in}{2.515528in}}%
\pgfpathlineto{\pgfqpoint{3.402371in}{2.504990in}}%
\pgfpathlineto{\pgfqpoint{3.394369in}{2.494537in}}%
\pgfpathclose%
\pgfusepath{fill}%
\end{pgfscope}%
\begin{pgfscope}%
\pgfpathrectangle{\pgfqpoint{1.150000in}{0.150000in}}{\pgfqpoint{5.700000in}{5.700000in}}%
\pgfusepath{clip}%
\pgfsetbuttcap%
\pgfsetroundjoin%
\definecolor{currentfill}{rgb}{0.278012,0.180367,0.486697}%
\pgfsetfillcolor{currentfill}%
\pgfsetfillopacity{0.700000}%
\pgfsetlinewidth{0.000000pt}%
\definecolor{currentstroke}{rgb}{0.000000,0.000000,0.000000}%
\pgfsetstrokecolor{currentstroke}%
\pgfsetdash{}{0pt}%
\pgfpathmoveto{\pgfqpoint{2.783719in}{2.633780in}}%
\pgfpathlineto{\pgfqpoint{2.797087in}{2.623476in}}%
\pgfpathlineto{\pgfqpoint{2.810455in}{2.613301in}}%
\pgfpathlineto{\pgfqpoint{2.823822in}{2.603253in}}%
\pgfpathlineto{\pgfqpoint{2.837190in}{2.593331in}}%
\pgfpathlineto{\pgfqpoint{2.845386in}{2.603109in}}%
\pgfpathlineto{\pgfqpoint{2.853574in}{2.612980in}}%
\pgfpathlineto{\pgfqpoint{2.861755in}{2.622945in}}%
\pgfpathlineto{\pgfqpoint{2.869928in}{2.633006in}}%
\pgfpathlineto{\pgfqpoint{2.856573in}{2.642982in}}%
\pgfpathlineto{\pgfqpoint{2.843218in}{2.653085in}}%
\pgfpathlineto{\pgfqpoint{2.829863in}{2.663315in}}%
\pgfpathlineto{\pgfqpoint{2.816507in}{2.673674in}}%
\pgfpathlineto{\pgfqpoint{2.808322in}{2.663551in}}%
\pgfpathlineto{\pgfqpoint{2.800129in}{2.653528in}}%
\pgfpathlineto{\pgfqpoint{2.791928in}{2.643605in}}%
\pgfpathlineto{\pgfqpoint{2.783719in}{2.633780in}}%
\pgfpathclose%
\pgfusepath{fill}%
\end{pgfscope}%
\begin{pgfscope}%
\pgfpathrectangle{\pgfqpoint{1.150000in}{0.150000in}}{\pgfqpoint{5.700000in}{5.700000in}}%
\pgfusepath{clip}%
\pgfsetbuttcap%
\pgfsetroundjoin%
\definecolor{currentfill}{rgb}{0.190631,0.407061,0.556089}%
\pgfsetfillcolor{currentfill}%
\pgfsetfillopacity{0.700000}%
\pgfsetlinewidth{0.000000pt}%
\definecolor{currentstroke}{rgb}{0.000000,0.000000,0.000000}%
\pgfsetstrokecolor{currentstroke}%
\pgfsetdash{}{0pt}%
\pgfpathmoveto{\pgfqpoint{5.451522in}{3.120285in}}%
\pgfpathlineto{\pgfqpoint{5.465273in}{3.114800in}}%
\pgfpathlineto{\pgfqpoint{5.479030in}{3.109382in}}%
\pgfpathlineto{\pgfqpoint{5.492794in}{3.104031in}}%
\pgfpathlineto{\pgfqpoint{5.506566in}{3.098746in}}%
\pgfpathlineto{\pgfqpoint{5.514182in}{3.119178in}}%
\pgfpathlineto{\pgfqpoint{5.521810in}{3.140093in}}%
\pgfpathlineto{\pgfqpoint{5.529450in}{3.161503in}}%
\pgfpathlineto{\pgfqpoint{5.537102in}{3.183418in}}%
\pgfpathlineto{\pgfqpoint{5.523347in}{3.189263in}}%
\pgfpathlineto{\pgfqpoint{5.509598in}{3.195174in}}%
\pgfpathlineto{\pgfqpoint{5.495856in}{3.201151in}}%
\pgfpathlineto{\pgfqpoint{5.482120in}{3.207196in}}%
\pgfpathlineto{\pgfqpoint{5.474453in}{3.184713in}}%
\pgfpathlineto{\pgfqpoint{5.466798in}{3.162741in}}%
\pgfpathlineto{\pgfqpoint{5.459155in}{3.141268in}}%
\pgfpathlineto{\pgfqpoint{5.451522in}{3.120285in}}%
\pgfpathclose%
\pgfusepath{fill}%
\end{pgfscope}%
\begin{pgfscope}%
\pgfpathrectangle{\pgfqpoint{1.150000in}{0.150000in}}{\pgfqpoint{5.700000in}{5.700000in}}%
\pgfusepath{clip}%
\pgfsetbuttcap%
\pgfsetroundjoin%
\definecolor{currentfill}{rgb}{0.281412,0.155834,0.469201}%
\pgfsetfillcolor{currentfill}%
\pgfsetfillopacity{0.700000}%
\pgfsetlinewidth{0.000000pt}%
\definecolor{currentstroke}{rgb}{0.000000,0.000000,0.000000}%
\pgfsetstrokecolor{currentstroke}%
\pgfsetdash{}{0pt}%
\pgfpathmoveto{\pgfqpoint{4.206204in}{2.565354in}}%
\pgfpathlineto{\pgfqpoint{4.219728in}{2.561244in}}%
\pgfpathlineto{\pgfqpoint{4.233259in}{2.557212in}}%
\pgfpathlineto{\pgfqpoint{4.246795in}{2.553260in}}%
\pgfpathlineto{\pgfqpoint{4.260338in}{2.549386in}}%
\pgfpathlineto{\pgfqpoint{4.268077in}{2.560294in}}%
\pgfpathlineto{\pgfqpoint{4.275812in}{2.571335in}}%
\pgfpathlineto{\pgfqpoint{4.283543in}{2.582514in}}%
\pgfpathlineto{\pgfqpoint{4.291270in}{2.593837in}}%
\pgfpathlineto{\pgfqpoint{4.277737in}{2.597988in}}%
\pgfpathlineto{\pgfqpoint{4.264211in}{2.602218in}}%
\pgfpathlineto{\pgfqpoint{4.250691in}{2.606526in}}%
\pgfpathlineto{\pgfqpoint{4.237177in}{2.610914in}}%
\pgfpathlineto{\pgfqpoint{4.229439in}{2.599307in}}%
\pgfpathlineto{\pgfqpoint{4.221698in}{2.587848in}}%
\pgfpathlineto{\pgfqpoint{4.213953in}{2.576532in}}%
\pgfpathlineto{\pgfqpoint{4.206204in}{2.565354in}}%
\pgfpathclose%
\pgfusepath{fill}%
\end{pgfscope}%
\begin{pgfscope}%
\pgfpathrectangle{\pgfqpoint{1.150000in}{0.150000in}}{\pgfqpoint{5.700000in}{5.700000in}}%
\pgfusepath{clip}%
\pgfsetbuttcap%
\pgfsetroundjoin%
\definecolor{currentfill}{rgb}{0.282290,0.145912,0.461510}%
\pgfsetfillcolor{currentfill}%
\pgfsetfillopacity{0.700000}%
\pgfsetlinewidth{0.000000pt}%
\definecolor{currentstroke}{rgb}{0.000000,0.000000,0.000000}%
\pgfsetstrokecolor{currentstroke}%
\pgfsetdash{}{0pt}%
\pgfpathmoveto{\pgfqpoint{2.976777in}{2.557620in}}%
\pgfpathlineto{\pgfqpoint{2.990135in}{2.548735in}}%
\pgfpathlineto{\pgfqpoint{3.003495in}{2.539967in}}%
\pgfpathlineto{\pgfqpoint{3.016856in}{2.531314in}}%
\pgfpathlineto{\pgfqpoint{3.030218in}{2.522776in}}%
\pgfpathlineto{\pgfqpoint{3.038350in}{2.532731in}}%
\pgfpathlineto{\pgfqpoint{3.046474in}{2.542771in}}%
\pgfpathlineto{\pgfqpoint{3.054592in}{2.552896in}}%
\pgfpathlineto{\pgfqpoint{3.062702in}{2.563108in}}%
\pgfpathlineto{\pgfqpoint{3.049351in}{2.571722in}}%
\pgfpathlineto{\pgfqpoint{3.036002in}{2.580450in}}%
\pgfpathlineto{\pgfqpoint{3.022653in}{2.589294in}}%
\pgfpathlineto{\pgfqpoint{3.009305in}{2.598254in}}%
\pgfpathlineto{\pgfqpoint{3.001184in}{2.587959in}}%
\pgfpathlineto{\pgfqpoint{2.993055in}{2.577756in}}%
\pgfpathlineto{\pgfqpoint{2.984920in}{2.567643in}}%
\pgfpathlineto{\pgfqpoint{2.976777in}{2.557620in}}%
\pgfpathclose%
\pgfusepath{fill}%
\end{pgfscope}%
\begin{pgfscope}%
\pgfpathrectangle{\pgfqpoint{1.150000in}{0.150000in}}{\pgfqpoint{5.700000in}{5.700000in}}%
\pgfusepath{clip}%
\pgfsetbuttcap%
\pgfsetroundjoin%
\definecolor{currentfill}{rgb}{0.276194,0.190074,0.493001}%
\pgfsetfillcolor{currentfill}%
\pgfsetfillopacity{0.700000}%
\pgfsetlinewidth{0.000000pt}%
\definecolor{currentstroke}{rgb}{0.000000,0.000000,0.000000}%
\pgfsetstrokecolor{currentstroke}%
\pgfsetdash{}{0pt}%
\pgfpathmoveto{\pgfqpoint{4.515577in}{2.639059in}}%
\pgfpathlineto{\pgfqpoint{4.529171in}{2.635288in}}%
\pgfpathlineto{\pgfqpoint{4.542772in}{2.631592in}}%
\pgfpathlineto{\pgfqpoint{4.556380in}{2.627971in}}%
\pgfpathlineto{\pgfqpoint{4.569994in}{2.624423in}}%
\pgfpathlineto{\pgfqpoint{4.577645in}{2.635965in}}%
\pgfpathlineto{\pgfqpoint{4.585294in}{2.647686in}}%
\pgfpathlineto{\pgfqpoint{4.592939in}{2.659593in}}%
\pgfpathlineto{\pgfqpoint{4.600583in}{2.671693in}}%
\pgfpathlineto{\pgfqpoint{4.586980in}{2.675578in}}%
\pgfpathlineto{\pgfqpoint{4.573385in}{2.679537in}}%
\pgfpathlineto{\pgfqpoint{4.559795in}{2.683571in}}%
\pgfpathlineto{\pgfqpoint{4.546213in}{2.687679in}}%
\pgfpathlineto{\pgfqpoint{4.538558in}{2.675235in}}%
\pgfpathlineto{\pgfqpoint{4.530900in}{2.662987in}}%
\pgfpathlineto{\pgfqpoint{4.523240in}{2.650931in}}%
\pgfpathlineto{\pgfqpoint{4.515577in}{2.639059in}}%
\pgfpathclose%
\pgfusepath{fill}%
\end{pgfscope}%
\begin{pgfscope}%
\pgfpathrectangle{\pgfqpoint{1.150000in}{0.150000in}}{\pgfqpoint{5.700000in}{5.700000in}}%
\pgfusepath{clip}%
\pgfsetbuttcap%
\pgfsetroundjoin%
\definecolor{currentfill}{rgb}{0.243113,0.292092,0.538516}%
\pgfsetfillcolor{currentfill}%
\pgfsetfillopacity{0.700000}%
\pgfsetlinewidth{0.000000pt}%
\definecolor{currentstroke}{rgb}{0.000000,0.000000,0.000000}%
\pgfsetstrokecolor{currentstroke}%
\pgfsetdash{}{0pt}%
\pgfpathmoveto{\pgfqpoint{5.080257in}{2.845254in}}%
\pgfpathlineto{\pgfqpoint{5.093969in}{2.841170in}}%
\pgfpathlineto{\pgfqpoint{5.107688in}{2.837156in}}%
\pgfpathlineto{\pgfqpoint{5.121414in}{2.833211in}}%
\pgfpathlineto{\pgfqpoint{5.135148in}{2.829335in}}%
\pgfpathlineto{\pgfqpoint{5.142696in}{2.843922in}}%
\pgfpathlineto{\pgfqpoint{5.150246in}{2.858826in}}%
\pgfpathlineto{\pgfqpoint{5.157800in}{2.874055in}}%
\pgfpathlineto{\pgfqpoint{5.165357in}{2.889618in}}%
\pgfpathlineto{\pgfqpoint{5.151638in}{2.893952in}}%
\pgfpathlineto{\pgfqpoint{5.137926in}{2.898355in}}%
\pgfpathlineto{\pgfqpoint{5.124222in}{2.902826in}}%
\pgfpathlineto{\pgfqpoint{5.110525in}{2.907367in}}%
\pgfpathlineto{\pgfqpoint{5.102953in}{2.891339in}}%
\pgfpathlineto{\pgfqpoint{5.095384in}{2.875649in}}%
\pgfpathlineto{\pgfqpoint{5.087819in}{2.860290in}}%
\pgfpathlineto{\pgfqpoint{5.080257in}{2.845254in}}%
\pgfpathclose%
\pgfusepath{fill}%
\end{pgfscope}%
\begin{pgfscope}%
\pgfpathrectangle{\pgfqpoint{1.150000in}{0.150000in}}{\pgfqpoint{5.700000in}{5.700000in}}%
\pgfusepath{clip}%
\pgfsetbuttcap%
\pgfsetroundjoin%
\definecolor{currentfill}{rgb}{0.283091,0.110553,0.431554}%
\pgfsetfillcolor{currentfill}%
\pgfsetfillopacity{0.700000}%
\pgfsetlinewidth{0.000000pt}%
\definecolor{currentstroke}{rgb}{0.000000,0.000000,0.000000}%
\pgfsetstrokecolor{currentstroke}%
\pgfsetdash{}{0pt}%
\pgfpathmoveto{\pgfqpoint{3.533395in}{2.485715in}}%
\pgfpathlineto{\pgfqpoint{3.546794in}{2.479751in}}%
\pgfpathlineto{\pgfqpoint{3.560197in}{2.473881in}}%
\pgfpathlineto{\pgfqpoint{3.573603in}{2.468104in}}%
\pgfpathlineto{\pgfqpoint{3.587014in}{2.462420in}}%
\pgfpathlineto{\pgfqpoint{3.594963in}{2.472783in}}%
\pgfpathlineto{\pgfqpoint{3.602907in}{2.483228in}}%
\pgfpathlineto{\pgfqpoint{3.610845in}{2.493757in}}%
\pgfpathlineto{\pgfqpoint{3.618777in}{2.504374in}}%
\pgfpathlineto{\pgfqpoint{3.605375in}{2.510215in}}%
\pgfpathlineto{\pgfqpoint{3.591978in}{2.516148in}}%
\pgfpathlineto{\pgfqpoint{3.578584in}{2.522175in}}%
\pgfpathlineto{\pgfqpoint{3.565195in}{2.528295in}}%
\pgfpathlineto{\pgfqpoint{3.557254in}{2.517514in}}%
\pgfpathlineto{\pgfqpoint{3.549307in}{2.506826in}}%
\pgfpathlineto{\pgfqpoint{3.541354in}{2.496227in}}%
\pgfpathlineto{\pgfqpoint{3.533395in}{2.485715in}}%
\pgfpathclose%
\pgfusepath{fill}%
\end{pgfscope}%
\begin{pgfscope}%
\pgfpathrectangle{\pgfqpoint{1.150000in}{0.150000in}}{\pgfqpoint{5.700000in}{5.700000in}}%
\pgfusepath{clip}%
\pgfsetbuttcap%
\pgfsetroundjoin%
\definecolor{currentfill}{rgb}{0.235526,0.309527,0.542944}%
\pgfsetfillcolor{currentfill}%
\pgfsetfillopacity{0.700000}%
\pgfsetlinewidth{0.000000pt}%
\definecolor{currentstroke}{rgb}{0.000000,0.000000,0.000000}%
\pgfsetstrokecolor{currentstroke}%
\pgfsetdash{}{0pt}%
\pgfpathmoveto{\pgfqpoint{5.165357in}{2.889618in}}%
\pgfpathlineto{\pgfqpoint{5.179083in}{2.885353in}}%
\pgfpathlineto{\pgfqpoint{5.192816in}{2.881157in}}%
\pgfpathlineto{\pgfqpoint{5.206557in}{2.877029in}}%
\pgfpathlineto{\pgfqpoint{5.220305in}{2.872969in}}%
\pgfpathlineto{\pgfqpoint{5.227851in}{2.888405in}}%
\pgfpathlineto{\pgfqpoint{5.235401in}{2.904187in}}%
\pgfpathlineto{\pgfqpoint{5.242956in}{2.920324in}}%
\pgfpathlineto{\pgfqpoint{5.250516in}{2.936825in}}%
\pgfpathlineto{\pgfqpoint{5.236783in}{2.941362in}}%
\pgfpathlineto{\pgfqpoint{5.223058in}{2.945968in}}%
\pgfpathlineto{\pgfqpoint{5.209340in}{2.950643in}}%
\pgfpathlineto{\pgfqpoint{5.195628in}{2.955385in}}%
\pgfpathlineto{\pgfqpoint{5.188053in}{2.938399in}}%
\pgfpathlineto{\pgfqpoint{5.180483in}{2.921782in}}%
\pgfpathlineto{\pgfqpoint{5.172918in}{2.905525in}}%
\pgfpathlineto{\pgfqpoint{5.165357in}{2.889618in}}%
\pgfpathclose%
\pgfusepath{fill}%
\end{pgfscope}%
\begin{pgfscope}%
\pgfpathrectangle{\pgfqpoint{1.150000in}{0.150000in}}{\pgfqpoint{5.700000in}{5.700000in}}%
\pgfusepath{clip}%
\pgfsetbuttcap%
\pgfsetroundjoin%
\definecolor{currentfill}{rgb}{0.252194,0.269783,0.531579}%
\pgfsetfillcolor{currentfill}%
\pgfsetfillopacity{0.700000}%
\pgfsetlinewidth{0.000000pt}%
\definecolor{currentstroke}{rgb}{0.000000,0.000000,0.000000}%
\pgfsetstrokecolor{currentstroke}%
\pgfsetdash{}{0pt}%
\pgfpathmoveto{\pgfqpoint{4.995197in}{2.803444in}}%
\pgfpathlineto{\pgfqpoint{5.008894in}{2.799520in}}%
\pgfpathlineto{\pgfqpoint{5.022599in}{2.795665in}}%
\pgfpathlineto{\pgfqpoint{5.036311in}{2.791881in}}%
\pgfpathlineto{\pgfqpoint{5.050031in}{2.788166in}}%
\pgfpathlineto{\pgfqpoint{5.057584in}{2.801995in}}%
\pgfpathlineto{\pgfqpoint{5.065139in}{2.816114in}}%
\pgfpathlineto{\pgfqpoint{5.072697in}{2.830531in}}%
\pgfpathlineto{\pgfqpoint{5.080257in}{2.845254in}}%
\pgfpathlineto{\pgfqpoint{5.066552in}{2.849406in}}%
\pgfpathlineto{\pgfqpoint{5.052854in}{2.853628in}}%
\pgfpathlineto{\pgfqpoint{5.039164in}{2.857920in}}%
\pgfpathlineto{\pgfqpoint{5.025481in}{2.862281in}}%
\pgfpathlineto{\pgfqpoint{5.017907in}{2.847114in}}%
\pgfpathlineto{\pgfqpoint{5.010335in}{2.832257in}}%
\pgfpathlineto{\pgfqpoint{5.002765in}{2.817703in}}%
\pgfpathlineto{\pgfqpoint{4.995197in}{2.803444in}}%
\pgfpathclose%
\pgfusepath{fill}%
\end{pgfscope}%
\begin{pgfscope}%
\pgfpathrectangle{\pgfqpoint{1.150000in}{0.150000in}}{\pgfqpoint{5.700000in}{5.700000in}}%
\pgfusepath{clip}%
\pgfsetbuttcap%
\pgfsetroundjoin%
\definecolor{currentfill}{rgb}{0.225863,0.330805,0.547314}%
\pgfsetfillcolor{currentfill}%
\pgfsetfillopacity{0.700000}%
\pgfsetlinewidth{0.000000pt}%
\definecolor{currentstroke}{rgb}{0.000000,0.000000,0.000000}%
\pgfsetstrokecolor{currentstroke}%
\pgfsetdash{}{0pt}%
\pgfpathmoveto{\pgfqpoint{5.250516in}{2.936825in}}%
\pgfpathlineto{\pgfqpoint{5.264255in}{2.932355in}}%
\pgfpathlineto{\pgfqpoint{5.278003in}{2.927954in}}%
\pgfpathlineto{\pgfqpoint{5.291757in}{2.923620in}}%
\pgfpathlineto{\pgfqpoint{5.305519in}{2.919354in}}%
\pgfpathlineto{\pgfqpoint{5.313068in}{2.935738in}}%
\pgfpathlineto{\pgfqpoint{5.320624in}{2.952498in}}%
\pgfpathlineto{\pgfqpoint{5.328186in}{2.969644in}}%
\pgfpathlineto{\pgfqpoint{5.335754in}{2.987185in}}%
\pgfpathlineto{\pgfqpoint{5.322008in}{2.991949in}}%
\pgfpathlineto{\pgfqpoint{5.308269in}{2.996781in}}%
\pgfpathlineto{\pgfqpoint{5.294538in}{3.001680in}}%
\pgfpathlineto{\pgfqpoint{5.280813in}{3.006648in}}%
\pgfpathlineto{\pgfqpoint{5.273229in}{2.988601in}}%
\pgfpathlineto{\pgfqpoint{5.265652in}{2.970954in}}%
\pgfpathlineto{\pgfqpoint{5.258081in}{2.953699in}}%
\pgfpathlineto{\pgfqpoint{5.250516in}{2.936825in}}%
\pgfpathclose%
\pgfusepath{fill}%
\end{pgfscope}%
\begin{pgfscope}%
\pgfpathrectangle{\pgfqpoint{1.150000in}{0.150000in}}{\pgfqpoint{5.700000in}{5.700000in}}%
\pgfusepath{clip}%
\pgfsetbuttcap%
\pgfsetroundjoin%
\definecolor{currentfill}{rgb}{0.258965,0.251537,0.524736}%
\pgfsetfillcolor{currentfill}%
\pgfsetfillopacity{0.700000}%
\pgfsetlinewidth{0.000000pt}%
\definecolor{currentstroke}{rgb}{0.000000,0.000000,0.000000}%
\pgfsetstrokecolor{currentstroke}%
\pgfsetdash{}{0pt}%
\pgfpathmoveto{\pgfqpoint{4.910161in}{2.763927in}}%
\pgfpathlineto{\pgfqpoint{4.923843in}{2.760140in}}%
\pgfpathlineto{\pgfqpoint{4.937533in}{2.756422in}}%
\pgfpathlineto{\pgfqpoint{4.951231in}{2.752776in}}%
\pgfpathlineto{\pgfqpoint{4.964935in}{2.749199in}}%
\pgfpathlineto{\pgfqpoint{4.972499in}{2.762357in}}%
\pgfpathlineto{\pgfqpoint{4.980064in}{2.775778in}}%
\pgfpathlineto{\pgfqpoint{4.987630in}{2.789471in}}%
\pgfpathlineto{\pgfqpoint{4.995197in}{2.803444in}}%
\pgfpathlineto{\pgfqpoint{4.981506in}{2.807438in}}%
\pgfpathlineto{\pgfqpoint{4.967823in}{2.811502in}}%
\pgfpathlineto{\pgfqpoint{4.954147in}{2.815637in}}%
\pgfpathlineto{\pgfqpoint{4.940479in}{2.819842in}}%
\pgfpathlineto{\pgfqpoint{4.932898in}{2.805444in}}%
\pgfpathlineto{\pgfqpoint{4.925318in}{2.791331in}}%
\pgfpathlineto{\pgfqpoint{4.917739in}{2.777495in}}%
\pgfpathlineto{\pgfqpoint{4.910161in}{2.763927in}}%
\pgfpathclose%
\pgfusepath{fill}%
\end{pgfscope}%
\begin{pgfscope}%
\pgfpathrectangle{\pgfqpoint{1.150000in}{0.150000in}}{\pgfqpoint{5.700000in}{5.700000in}}%
\pgfusepath{clip}%
\pgfsetbuttcap%
\pgfsetroundjoin%
\definecolor{currentfill}{rgb}{0.214298,0.355619,0.551184}%
\pgfsetfillcolor{currentfill}%
\pgfsetfillopacity{0.700000}%
\pgfsetlinewidth{0.000000pt}%
\definecolor{currentstroke}{rgb}{0.000000,0.000000,0.000000}%
\pgfsetstrokecolor{currentstroke}%
\pgfsetdash{}{0pt}%
\pgfpathmoveto{\pgfqpoint{5.335754in}{2.987185in}}%
\pgfpathlineto{\pgfqpoint{5.349507in}{2.982489in}}%
\pgfpathlineto{\pgfqpoint{5.363268in}{2.977859in}}%
\pgfpathlineto{\pgfqpoint{5.377035in}{2.973298in}}%
\pgfpathlineto{\pgfqpoint{5.390811in}{2.968803in}}%
\pgfpathlineto{\pgfqpoint{5.398370in}{2.986238in}}%
\pgfpathlineto{\pgfqpoint{5.405937in}{3.004083in}}%
\pgfpathlineto{\pgfqpoint{5.413512in}{3.022346in}}%
\pgfpathlineto{\pgfqpoint{5.421095in}{3.041037in}}%
\pgfpathlineto{\pgfqpoint{5.407336in}{3.046050in}}%
\pgfpathlineto{\pgfqpoint{5.393585in}{3.051130in}}%
\pgfpathlineto{\pgfqpoint{5.379840in}{3.056278in}}%
\pgfpathlineto{\pgfqpoint{5.366102in}{3.061493in}}%
\pgfpathlineto{\pgfqpoint{5.358503in}{3.042275in}}%
\pgfpathlineto{\pgfqpoint{5.350912in}{3.023491in}}%
\pgfpathlineto{\pgfqpoint{5.343330in}{3.005131in}}%
\pgfpathlineto{\pgfqpoint{5.335754in}{2.987185in}}%
\pgfpathclose%
\pgfusepath{fill}%
\end{pgfscope}%
\begin{pgfscope}%
\pgfpathrectangle{\pgfqpoint{1.150000in}{0.150000in}}{\pgfqpoint{5.700000in}{5.700000in}}%
\pgfusepath{clip}%
\pgfsetbuttcap%
\pgfsetroundjoin%
\definecolor{currentfill}{rgb}{0.283187,0.125848,0.444960}%
\pgfsetfillcolor{currentfill}%
\pgfsetfillopacity{0.700000}%
\pgfsetlinewidth{0.000000pt}%
\definecolor{currentstroke}{rgb}{0.000000,0.000000,0.000000}%
\pgfsetstrokecolor{currentstroke}%
\pgfsetdash{}{0pt}%
\pgfpathmoveto{\pgfqpoint{3.896772in}{2.505667in}}%
\pgfpathlineto{\pgfqpoint{3.910235in}{2.500957in}}%
\pgfpathlineto{\pgfqpoint{3.923703in}{2.496332in}}%
\pgfpathlineto{\pgfqpoint{3.937177in}{2.491791in}}%
\pgfpathlineto{\pgfqpoint{3.950656in}{2.487334in}}%
\pgfpathlineto{\pgfqpoint{3.958492in}{2.497849in}}%
\pgfpathlineto{\pgfqpoint{3.966323in}{2.508462in}}%
\pgfpathlineto{\pgfqpoint{3.974149in}{2.519178in}}%
\pgfpathlineto{\pgfqpoint{3.981970in}{2.530002in}}%
\pgfpathlineto{\pgfqpoint{3.968501in}{2.534676in}}%
\pgfpathlineto{\pgfqpoint{3.955037in}{2.539433in}}%
\pgfpathlineto{\pgfqpoint{3.941578in}{2.544275in}}%
\pgfpathlineto{\pgfqpoint{3.928125in}{2.549202in}}%
\pgfpathlineto{\pgfqpoint{3.920294in}{2.538155in}}%
\pgfpathlineto{\pgfqpoint{3.912459in}{2.527219in}}%
\pgfpathlineto{\pgfqpoint{3.904618in}{2.516391in}}%
\pgfpathlineto{\pgfqpoint{3.896772in}{2.505667in}}%
\pgfpathclose%
\pgfusepath{fill}%
\end{pgfscope}%
\begin{pgfscope}%
\pgfpathrectangle{\pgfqpoint{1.150000in}{0.150000in}}{\pgfqpoint{5.700000in}{5.700000in}}%
\pgfusepath{clip}%
\pgfsetbuttcap%
\pgfsetroundjoin%
\definecolor{currentfill}{rgb}{0.278012,0.180367,0.486697}%
\pgfsetfillcolor{currentfill}%
\pgfsetfillopacity{0.700000}%
\pgfsetlinewidth{0.000000pt}%
\definecolor{currentstroke}{rgb}{0.000000,0.000000,0.000000}%
\pgfsetstrokecolor{currentstroke}%
\pgfsetdash{}{0pt}%
\pgfpathmoveto{\pgfqpoint{4.430539in}{2.607871in}}%
\pgfpathlineto{\pgfqpoint{4.444118in}{2.604116in}}%
\pgfpathlineto{\pgfqpoint{4.457704in}{2.600437in}}%
\pgfpathlineto{\pgfqpoint{4.471296in}{2.596833in}}%
\pgfpathlineto{\pgfqpoint{4.484895in}{2.593305in}}%
\pgfpathlineto{\pgfqpoint{4.492571in}{2.604495in}}%
\pgfpathlineto{\pgfqpoint{4.500243in}{2.615848in}}%
\pgfpathlineto{\pgfqpoint{4.507911in}{2.627367in}}%
\pgfpathlineto{\pgfqpoint{4.515577in}{2.639059in}}%
\pgfpathlineto{\pgfqpoint{4.501990in}{2.642905in}}%
\pgfpathlineto{\pgfqpoint{4.488409in}{2.646827in}}%
\pgfpathlineto{\pgfqpoint{4.474835in}{2.650823in}}%
\pgfpathlineto{\pgfqpoint{4.461267in}{2.654896in}}%
\pgfpathlineto{\pgfqpoint{4.453590in}{2.642878in}}%
\pgfpathlineto{\pgfqpoint{4.445909in}{2.631039in}}%
\pgfpathlineto{\pgfqpoint{4.438226in}{2.619372in}}%
\pgfpathlineto{\pgfqpoint{4.430539in}{2.607871in}}%
\pgfpathclose%
\pgfusepath{fill}%
\end{pgfscope}%
\begin{pgfscope}%
\pgfpathrectangle{\pgfqpoint{1.150000in}{0.150000in}}{\pgfqpoint{5.700000in}{5.700000in}}%
\pgfusepath{clip}%
\pgfsetbuttcap%
\pgfsetroundjoin%
\definecolor{currentfill}{rgb}{0.283197,0.115680,0.436115}%
\pgfsetfillcolor{currentfill}%
\pgfsetfillopacity{0.700000}%
\pgfsetlinewidth{0.000000pt}%
\definecolor{currentstroke}{rgb}{0.000000,0.000000,0.000000}%
\pgfsetstrokecolor{currentstroke}%
\pgfsetdash{}{0pt}%
\pgfpathmoveto{\pgfqpoint{3.672425in}{2.481930in}}%
\pgfpathlineto{\pgfqpoint{3.685847in}{2.476546in}}%
\pgfpathlineto{\pgfqpoint{3.699275in}{2.471251in}}%
\pgfpathlineto{\pgfqpoint{3.712707in}{2.466046in}}%
\pgfpathlineto{\pgfqpoint{3.726144in}{2.460930in}}%
\pgfpathlineto{\pgfqpoint{3.734052in}{2.471299in}}%
\pgfpathlineto{\pgfqpoint{3.741954in}{2.481753in}}%
\pgfpathlineto{\pgfqpoint{3.749850in}{2.492294in}}%
\pgfpathlineto{\pgfqpoint{3.757742in}{2.502927in}}%
\pgfpathlineto{\pgfqpoint{3.744314in}{2.508220in}}%
\pgfpathlineto{\pgfqpoint{3.730891in}{2.513601in}}%
\pgfpathlineto{\pgfqpoint{3.717473in}{2.519073in}}%
\pgfpathlineto{\pgfqpoint{3.704059in}{2.524633in}}%
\pgfpathlineto{\pgfqpoint{3.696159in}{2.513817in}}%
\pgfpathlineto{\pgfqpoint{3.688253in}{2.503096in}}%
\pgfpathlineto{\pgfqpoint{3.680342in}{2.492468in}}%
\pgfpathlineto{\pgfqpoint{3.672425in}{2.481930in}}%
\pgfpathclose%
\pgfusepath{fill}%
\end{pgfscope}%
\begin{pgfscope}%
\pgfpathrectangle{\pgfqpoint{1.150000in}{0.150000in}}{\pgfqpoint{5.700000in}{5.700000in}}%
\pgfusepath{clip}%
\pgfsetbuttcap%
\pgfsetroundjoin%
\definecolor{currentfill}{rgb}{0.263663,0.237631,0.518762}%
\pgfsetfillcolor{currentfill}%
\pgfsetfillopacity{0.700000}%
\pgfsetlinewidth{0.000000pt}%
\definecolor{currentstroke}{rgb}{0.000000,0.000000,0.000000}%
\pgfsetstrokecolor{currentstroke}%
\pgfsetdash{}{0pt}%
\pgfpathmoveto{\pgfqpoint{4.825134in}{2.726467in}}%
\pgfpathlineto{\pgfqpoint{4.838802in}{2.722793in}}%
\pgfpathlineto{\pgfqpoint{4.852477in}{2.719190in}}%
\pgfpathlineto{\pgfqpoint{4.866159in}{2.715659in}}%
\pgfpathlineto{\pgfqpoint{4.879848in}{2.712198in}}%
\pgfpathlineto{\pgfqpoint{4.887427in}{2.724764in}}%
\pgfpathlineto{\pgfqpoint{4.895005in}{2.737569in}}%
\pgfpathlineto{\pgfqpoint{4.902583in}{2.750621in}}%
\pgfpathlineto{\pgfqpoint{4.910161in}{2.763927in}}%
\pgfpathlineto{\pgfqpoint{4.896485in}{2.767786in}}%
\pgfpathlineto{\pgfqpoint{4.882817in}{2.771715in}}%
\pgfpathlineto{\pgfqpoint{4.869155in}{2.775715in}}%
\pgfpathlineto{\pgfqpoint{4.855501in}{2.779787in}}%
\pgfpathlineto{\pgfqpoint{4.847910in}{2.766076in}}%
\pgfpathlineto{\pgfqpoint{4.840318in}{2.752624in}}%
\pgfpathlineto{\pgfqpoint{4.832726in}{2.739423in}}%
\pgfpathlineto{\pgfqpoint{4.825134in}{2.726467in}}%
\pgfpathclose%
\pgfusepath{fill}%
\end{pgfscope}%
\begin{pgfscope}%
\pgfpathrectangle{\pgfqpoint{1.150000in}{0.150000in}}{\pgfqpoint{5.700000in}{5.700000in}}%
\pgfusepath{clip}%
\pgfsetbuttcap%
\pgfsetroundjoin%
\definecolor{currentfill}{rgb}{0.280255,0.165693,0.476498}%
\pgfsetfillcolor{currentfill}%
\pgfsetfillopacity{0.700000}%
\pgfsetlinewidth{0.000000pt}%
\definecolor{currentstroke}{rgb}{0.000000,0.000000,0.000000}%
\pgfsetstrokecolor{currentstroke}%
\pgfsetdash{}{0pt}%
\pgfpathmoveto{\pgfqpoint{2.837190in}{2.593331in}}%
\pgfpathlineto{\pgfqpoint{2.850557in}{2.583535in}}%
\pgfpathlineto{\pgfqpoint{2.863924in}{2.573863in}}%
\pgfpathlineto{\pgfqpoint{2.877291in}{2.564315in}}%
\pgfpathlineto{\pgfqpoint{2.890658in}{2.554889in}}%
\pgfpathlineto{\pgfqpoint{2.898842in}{2.564619in}}%
\pgfpathlineto{\pgfqpoint{2.907019in}{2.574437in}}%
\pgfpathlineto{\pgfqpoint{2.915188in}{2.584344in}}%
\pgfpathlineto{\pgfqpoint{2.923349in}{2.594343in}}%
\pgfpathlineto{\pgfqpoint{2.909994in}{2.603824in}}%
\pgfpathlineto{\pgfqpoint{2.896638in}{2.613428in}}%
\pgfpathlineto{\pgfqpoint{2.883283in}{2.623155in}}%
\pgfpathlineto{\pgfqpoint{2.869928in}{2.633006in}}%
\pgfpathlineto{\pgfqpoint{2.861755in}{2.622945in}}%
\pgfpathlineto{\pgfqpoint{2.853574in}{2.612980in}}%
\pgfpathlineto{\pgfqpoint{2.845386in}{2.603109in}}%
\pgfpathlineto{\pgfqpoint{2.837190in}{2.593331in}}%
\pgfpathclose%
\pgfusepath{fill}%
\end{pgfscope}%
\begin{pgfscope}%
\pgfpathrectangle{\pgfqpoint{1.150000in}{0.150000in}}{\pgfqpoint{5.700000in}{5.700000in}}%
\pgfusepath{clip}%
\pgfsetbuttcap%
\pgfsetroundjoin%
\definecolor{currentfill}{rgb}{0.180629,0.429975,0.557282}%
\pgfsetfillcolor{currentfill}%
\pgfsetfillopacity{0.700000}%
\pgfsetlinewidth{0.000000pt}%
\definecolor{currentstroke}{rgb}{0.000000,0.000000,0.000000}%
\pgfsetstrokecolor{currentstroke}%
\pgfsetdash{}{0pt}%
\pgfpathmoveto{\pgfqpoint{5.537102in}{3.183418in}}%
\pgfpathlineto{\pgfqpoint{5.550865in}{3.177641in}}%
\pgfpathlineto{\pgfqpoint{5.564635in}{3.171929in}}%
\pgfpathlineto{\pgfqpoint{5.578411in}{3.166284in}}%
\pgfpathlineto{\pgfqpoint{5.592195in}{3.160706in}}%
\pgfpathlineto{\pgfqpoint{5.599844in}{3.182564in}}%
\pgfpathlineto{\pgfqpoint{5.607507in}{3.204944in}}%
\pgfpathlineto{\pgfqpoint{5.615183in}{3.227857in}}%
\pgfpathlineto{\pgfqpoint{5.601412in}{3.233869in}}%
\pgfpathlineto{\pgfqpoint{5.587647in}{3.239948in}}%
\pgfpathlineto{\pgfqpoint{5.573889in}{3.246093in}}%
\pgfpathlineto{\pgfqpoint{5.560138in}{3.252304in}}%
\pgfpathlineto{\pgfqpoint{5.552445in}{3.228808in}}%
\pgfpathlineto{\pgfqpoint{5.544767in}{3.205850in}}%
\pgfpathlineto{\pgfqpoint{5.537102in}{3.183418in}}%
\pgfpathclose%
\pgfusepath{fill}%
\end{pgfscope}%
\begin{pgfscope}%
\pgfpathrectangle{\pgfqpoint{1.150000in}{0.150000in}}{\pgfqpoint{5.700000in}{5.700000in}}%
\pgfusepath{clip}%
\pgfsetbuttcap%
\pgfsetroundjoin%
\definecolor{currentfill}{rgb}{0.282290,0.145912,0.461510}%
\pgfsetfillcolor{currentfill}%
\pgfsetfillopacity{0.700000}%
\pgfsetlinewidth{0.000000pt}%
\definecolor{currentstroke}{rgb}{0.000000,0.000000,0.000000}%
\pgfsetstrokecolor{currentstroke}%
\pgfsetdash{}{0pt}%
\pgfpathmoveto{\pgfqpoint{4.121083in}{2.538136in}}%
\pgfpathlineto{\pgfqpoint{4.134594in}{2.533963in}}%
\pgfpathlineto{\pgfqpoint{4.148111in}{2.529871in}}%
\pgfpathlineto{\pgfqpoint{4.161633in}{2.525859in}}%
\pgfpathlineto{\pgfqpoint{4.175162in}{2.521926in}}%
\pgfpathlineto{\pgfqpoint{4.182929in}{2.532600in}}%
\pgfpathlineto{\pgfqpoint{4.190692in}{2.543393in}}%
\pgfpathlineto{\pgfqpoint{4.198450in}{2.554309in}}%
\pgfpathlineto{\pgfqpoint{4.206204in}{2.565354in}}%
\pgfpathlineto{\pgfqpoint{4.192685in}{2.569544in}}%
\pgfpathlineto{\pgfqpoint{4.179173in}{2.573813in}}%
\pgfpathlineto{\pgfqpoint{4.165666in}{2.578163in}}%
\pgfpathlineto{\pgfqpoint{4.152166in}{2.582593in}}%
\pgfpathlineto{\pgfqpoint{4.144402in}{2.571283in}}%
\pgfpathlineto{\pgfqpoint{4.136634in}{2.560107in}}%
\pgfpathlineto{\pgfqpoint{4.128861in}{2.549060in}}%
\pgfpathlineto{\pgfqpoint{4.121083in}{2.538136in}}%
\pgfpathclose%
\pgfusepath{fill}%
\end{pgfscope}%
\begin{pgfscope}%
\pgfpathrectangle{\pgfqpoint{1.150000in}{0.150000in}}{\pgfqpoint{5.700000in}{5.700000in}}%
\pgfusepath{clip}%
\pgfsetbuttcap%
\pgfsetroundjoin%
\definecolor{currentfill}{rgb}{0.204903,0.375746,0.553533}%
\pgfsetfillcolor{currentfill}%
\pgfsetfillopacity{0.700000}%
\pgfsetlinewidth{0.000000pt}%
\definecolor{currentstroke}{rgb}{0.000000,0.000000,0.000000}%
\pgfsetstrokecolor{currentstroke}%
\pgfsetdash{}{0pt}%
\pgfpathmoveto{\pgfqpoint{5.421095in}{3.041037in}}%
\pgfpathlineto{\pgfqpoint{5.434861in}{3.036091in}}%
\pgfpathlineto{\pgfqpoint{5.448635in}{3.031212in}}%
\pgfpathlineto{\pgfqpoint{5.462416in}{3.026400in}}%
\pgfpathlineto{\pgfqpoint{5.476204in}{3.021654in}}%
\pgfpathlineto{\pgfqpoint{5.483780in}{3.040253in}}%
\pgfpathlineto{\pgfqpoint{5.491365in}{3.059294in}}%
\pgfpathlineto{\pgfqpoint{5.498960in}{3.078789in}}%
\pgfpathlineto{\pgfqpoint{5.506566in}{3.098746in}}%
\pgfpathlineto{\pgfqpoint{5.492794in}{3.104031in}}%
\pgfpathlineto{\pgfqpoint{5.479030in}{3.109382in}}%
\pgfpathlineto{\pgfqpoint{5.465273in}{3.114800in}}%
\pgfpathlineto{\pgfqpoint{5.451522in}{3.120285in}}%
\pgfpathlineto{\pgfqpoint{5.443901in}{3.099781in}}%
\pgfpathlineto{\pgfqpoint{5.436289in}{3.079745in}}%
\pgfpathlineto{\pgfqpoint{5.428688in}{3.060167in}}%
\pgfpathlineto{\pgfqpoint{5.421095in}{3.041037in}}%
\pgfpathclose%
\pgfusepath{fill}%
\end{pgfscope}%
\begin{pgfscope}%
\pgfpathrectangle{\pgfqpoint{1.150000in}{0.150000in}}{\pgfqpoint{5.700000in}{5.700000in}}%
\pgfusepath{clip}%
\pgfsetbuttcap%
\pgfsetroundjoin%
\definecolor{currentfill}{rgb}{0.283229,0.120777,0.440584}%
\pgfsetfillcolor{currentfill}%
\pgfsetfillopacity{0.700000}%
\pgfsetlinewidth{0.000000pt}%
\definecolor{currentstroke}{rgb}{0.000000,0.000000,0.000000}%
\pgfsetstrokecolor{currentstroke}%
\pgfsetdash{}{0pt}%
\pgfpathmoveto{\pgfqpoint{3.169567in}{2.498227in}}%
\pgfpathlineto{\pgfqpoint{3.182933in}{2.490608in}}%
\pgfpathlineto{\pgfqpoint{3.196302in}{2.483096in}}%
\pgfpathlineto{\pgfqpoint{3.209673in}{2.475690in}}%
\pgfpathlineto{\pgfqpoint{3.223047in}{2.468388in}}%
\pgfpathlineto{\pgfqpoint{3.231119in}{2.478425in}}%
\pgfpathlineto{\pgfqpoint{3.239185in}{2.488539in}}%
\pgfpathlineto{\pgfqpoint{3.247244in}{2.498731in}}%
\pgfpathlineto{\pgfqpoint{3.255297in}{2.509003in}}%
\pgfpathlineto{\pgfqpoint{3.241933in}{2.516401in}}%
\pgfpathlineto{\pgfqpoint{3.228572in}{2.523903in}}%
\pgfpathlineto{\pgfqpoint{3.215214in}{2.531511in}}%
\pgfpathlineto{\pgfqpoint{3.201858in}{2.539226in}}%
\pgfpathlineto{\pgfqpoint{3.193795in}{2.528850in}}%
\pgfpathlineto{\pgfqpoint{3.185725in}{2.518560in}}%
\pgfpathlineto{\pgfqpoint{3.177649in}{2.508353in}}%
\pgfpathlineto{\pgfqpoint{3.169567in}{2.498227in}}%
\pgfpathclose%
\pgfusepath{fill}%
\end{pgfscope}%
\begin{pgfscope}%
\pgfpathrectangle{\pgfqpoint{1.150000in}{0.150000in}}{\pgfqpoint{5.700000in}{5.700000in}}%
\pgfusepath{clip}%
\pgfsetbuttcap%
\pgfsetroundjoin%
\definecolor{currentfill}{rgb}{0.283091,0.110553,0.431554}%
\pgfsetfillcolor{currentfill}%
\pgfsetfillopacity{0.700000}%
\pgfsetlinewidth{0.000000pt}%
\definecolor{currentstroke}{rgb}{0.000000,0.000000,0.000000}%
\pgfsetstrokecolor{currentstroke}%
\pgfsetdash{}{0pt}%
\pgfpathmoveto{\pgfqpoint{3.308776in}{2.480451in}}%
\pgfpathlineto{\pgfqpoint{3.322152in}{2.473569in}}%
\pgfpathlineto{\pgfqpoint{3.335532in}{2.466788in}}%
\pgfpathlineto{\pgfqpoint{3.348915in}{2.460107in}}%
\pgfpathlineto{\pgfqpoint{3.362301in}{2.453526in}}%
\pgfpathlineto{\pgfqpoint{3.370327in}{2.463663in}}%
\pgfpathlineto{\pgfqpoint{3.378347in}{2.473876in}}%
\pgfpathlineto{\pgfqpoint{3.386361in}{2.484167in}}%
\pgfpathlineto{\pgfqpoint{3.394369in}{2.494537in}}%
\pgfpathlineto{\pgfqpoint{3.380993in}{2.501235in}}%
\pgfpathlineto{\pgfqpoint{3.367620in}{2.508032in}}%
\pgfpathlineto{\pgfqpoint{3.354250in}{2.514929in}}%
\pgfpathlineto{\pgfqpoint{3.340883in}{2.521928in}}%
\pgfpathlineto{\pgfqpoint{3.332866in}{2.511434in}}%
\pgfpathlineto{\pgfqpoint{3.324842in}{2.501024in}}%
\pgfpathlineto{\pgfqpoint{3.316812in}{2.490698in}}%
\pgfpathlineto{\pgfqpoint{3.308776in}{2.480451in}}%
\pgfpathclose%
\pgfusepath{fill}%
\end{pgfscope}%
\begin{pgfscope}%
\pgfpathrectangle{\pgfqpoint{1.150000in}{0.150000in}}{\pgfqpoint{5.700000in}{5.700000in}}%
\pgfusepath{clip}%
\pgfsetbuttcap%
\pgfsetroundjoin%
\definecolor{currentfill}{rgb}{0.271828,0.209303,0.504434}%
\pgfsetfillcolor{currentfill}%
\pgfsetfillopacity{0.700000}%
\pgfsetlinewidth{0.000000pt}%
\definecolor{currentstroke}{rgb}{0.000000,0.000000,0.000000}%
\pgfsetstrokecolor{currentstroke}%
\pgfsetdash{}{0pt}%
\pgfpathmoveto{\pgfqpoint{2.643723in}{2.682362in}}%
\pgfpathlineto{\pgfqpoint{2.657113in}{2.671022in}}%
\pgfpathlineto{\pgfqpoint{2.670501in}{2.659820in}}%
\pgfpathlineto{\pgfqpoint{2.683888in}{2.648754in}}%
\pgfpathlineto{\pgfqpoint{2.697274in}{2.637825in}}%
\pgfpathlineto{\pgfqpoint{2.705527in}{2.647296in}}%
\pgfpathlineto{\pgfqpoint{2.713773in}{2.656865in}}%
\pgfpathlineto{\pgfqpoint{2.722010in}{2.666534in}}%
\pgfpathlineto{\pgfqpoint{2.730240in}{2.676304in}}%
\pgfpathlineto{\pgfqpoint{2.716867in}{2.687269in}}%
\pgfpathlineto{\pgfqpoint{2.703494in}{2.698368in}}%
\pgfpathlineto{\pgfqpoint{2.690119in}{2.709605in}}%
\pgfpathlineto{\pgfqpoint{2.676742in}{2.720979in}}%
\pgfpathlineto{\pgfqpoint{2.668500in}{2.711167in}}%
\pgfpathlineto{\pgfqpoint{2.660249in}{2.701461in}}%
\pgfpathlineto{\pgfqpoint{2.651990in}{2.691860in}}%
\pgfpathlineto{\pgfqpoint{2.643723in}{2.682362in}}%
\pgfpathclose%
\pgfusepath{fill}%
\end{pgfscope}%
\begin{pgfscope}%
\pgfpathrectangle{\pgfqpoint{1.150000in}{0.150000in}}{\pgfqpoint{5.700000in}{5.700000in}}%
\pgfusepath{clip}%
\pgfsetbuttcap%
\pgfsetroundjoin%
\definecolor{currentfill}{rgb}{0.267968,0.223549,0.512008}%
\pgfsetfillcolor{currentfill}%
\pgfsetfillopacity{0.700000}%
\pgfsetlinewidth{0.000000pt}%
\definecolor{currentstroke}{rgb}{0.000000,0.000000,0.000000}%
\pgfsetstrokecolor{currentstroke}%
\pgfsetdash{}{0pt}%
\pgfpathmoveto{\pgfqpoint{4.740105in}{2.690852in}}%
\pgfpathlineto{\pgfqpoint{4.753757in}{2.687268in}}%
\pgfpathlineto{\pgfqpoint{4.767417in}{2.683756in}}%
\pgfpathlineto{\pgfqpoint{4.781083in}{2.680316in}}%
\pgfpathlineto{\pgfqpoint{4.794757in}{2.676949in}}%
\pgfpathlineto{\pgfqpoint{4.802353in}{2.688996in}}%
\pgfpathlineto{\pgfqpoint{4.809948in}{2.701261in}}%
\pgfpathlineto{\pgfqpoint{4.817541in}{2.713749in}}%
\pgfpathlineto{\pgfqpoint{4.825134in}{2.726467in}}%
\pgfpathlineto{\pgfqpoint{4.811474in}{2.730213in}}%
\pgfpathlineto{\pgfqpoint{4.797820in}{2.734031in}}%
\pgfpathlineto{\pgfqpoint{4.784174in}{2.737920in}}%
\pgfpathlineto{\pgfqpoint{4.770534in}{2.741881in}}%
\pgfpathlineto{\pgfqpoint{4.762928in}{2.728778in}}%
\pgfpathlineto{\pgfqpoint{4.755322in}{2.715910in}}%
\pgfpathlineto{\pgfqpoint{4.747714in}{2.703270in}}%
\pgfpathlineto{\pgfqpoint{4.740105in}{2.690852in}}%
\pgfpathclose%
\pgfusepath{fill}%
\end{pgfscope}%
\begin{pgfscope}%
\pgfpathrectangle{\pgfqpoint{1.150000in}{0.150000in}}{\pgfqpoint{5.700000in}{5.700000in}}%
\pgfusepath{clip}%
\pgfsetbuttcap%
\pgfsetroundjoin%
\definecolor{currentfill}{rgb}{0.283072,0.130895,0.449241}%
\pgfsetfillcolor{currentfill}%
\pgfsetfillopacity{0.700000}%
\pgfsetlinewidth{0.000000pt}%
\definecolor{currentstroke}{rgb}{0.000000,0.000000,0.000000}%
\pgfsetstrokecolor{currentstroke}%
\pgfsetdash{}{0pt}%
\pgfpathmoveto{\pgfqpoint{3.030218in}{2.522776in}}%
\pgfpathlineto{\pgfqpoint{3.043581in}{2.514352in}}%
\pgfpathlineto{\pgfqpoint{3.056946in}{2.506041in}}%
\pgfpathlineto{\pgfqpoint{3.070312in}{2.497842in}}%
\pgfpathlineto{\pgfqpoint{3.083680in}{2.489754in}}%
\pgfpathlineto{\pgfqpoint{3.091801in}{2.499641in}}%
\pgfpathlineto{\pgfqpoint{3.099914in}{2.509607in}}%
\pgfpathlineto{\pgfqpoint{3.108021in}{2.519654in}}%
\pgfpathlineto{\pgfqpoint{3.116121in}{2.529784in}}%
\pgfpathlineto{\pgfqpoint{3.102764in}{2.537947in}}%
\pgfpathlineto{\pgfqpoint{3.089409in}{2.546222in}}%
\pgfpathlineto{\pgfqpoint{3.076055in}{2.554609in}}%
\pgfpathlineto{\pgfqpoint{3.062702in}{2.563108in}}%
\pgfpathlineto{\pgfqpoint{3.054592in}{2.552896in}}%
\pgfpathlineto{\pgfqpoint{3.046474in}{2.542771in}}%
\pgfpathlineto{\pgfqpoint{3.038350in}{2.532731in}}%
\pgfpathlineto{\pgfqpoint{3.030218in}{2.522776in}}%
\pgfpathclose%
\pgfusepath{fill}%
\end{pgfscope}%
\begin{pgfscope}%
\pgfpathrectangle{\pgfqpoint{1.150000in}{0.150000in}}{\pgfqpoint{5.700000in}{5.700000in}}%
\pgfusepath{clip}%
\pgfsetbuttcap%
\pgfsetroundjoin%
\definecolor{currentfill}{rgb}{0.283091,0.110553,0.431554}%
\pgfsetfillcolor{currentfill}%
\pgfsetfillopacity{0.700000}%
\pgfsetlinewidth{0.000000pt}%
\definecolor{currentstroke}{rgb}{0.000000,0.000000,0.000000}%
\pgfsetstrokecolor{currentstroke}%
\pgfsetdash{}{0pt}%
\pgfpathmoveto{\pgfqpoint{3.447907in}{2.468732in}}%
\pgfpathlineto{\pgfqpoint{3.461300in}{2.462523in}}%
\pgfpathlineto{\pgfqpoint{3.474697in}{2.456411in}}%
\pgfpathlineto{\pgfqpoint{3.488097in}{2.450395in}}%
\pgfpathlineto{\pgfqpoint{3.501501in}{2.444473in}}%
\pgfpathlineto{\pgfqpoint{3.509484in}{2.454668in}}%
\pgfpathlineto{\pgfqpoint{3.517460in}{2.464938in}}%
\pgfpathlineto{\pgfqpoint{3.525431in}{2.475286in}}%
\pgfpathlineto{\pgfqpoint{3.533395in}{2.485715in}}%
\pgfpathlineto{\pgfqpoint{3.520001in}{2.491773in}}%
\pgfpathlineto{\pgfqpoint{3.506610in}{2.497926in}}%
\pgfpathlineto{\pgfqpoint{3.493222in}{2.504175in}}%
\pgfpathlineto{\pgfqpoint{3.479839in}{2.510520in}}%
\pgfpathlineto{\pgfqpoint{3.471865in}{2.499947in}}%
\pgfpathlineto{\pgfqpoint{3.463885in}{2.489460in}}%
\pgfpathlineto{\pgfqpoint{3.455899in}{2.479056in}}%
\pgfpathlineto{\pgfqpoint{3.447907in}{2.468732in}}%
\pgfpathclose%
\pgfusepath{fill}%
\end{pgfscope}%
\begin{pgfscope}%
\pgfpathrectangle{\pgfqpoint{1.150000in}{0.150000in}}{\pgfqpoint{5.700000in}{5.700000in}}%
\pgfusepath{clip}%
\pgfsetbuttcap%
\pgfsetroundjoin%
\definecolor{currentfill}{rgb}{0.280255,0.165693,0.476498}%
\pgfsetfillcolor{currentfill}%
\pgfsetfillopacity{0.700000}%
\pgfsetlinewidth{0.000000pt}%
\definecolor{currentstroke}{rgb}{0.000000,0.000000,0.000000}%
\pgfsetstrokecolor{currentstroke}%
\pgfsetdash{}{0pt}%
\pgfpathmoveto{\pgfqpoint{4.345462in}{2.578011in}}%
\pgfpathlineto{\pgfqpoint{4.359026in}{2.574248in}}%
\pgfpathlineto{\pgfqpoint{4.372597in}{2.570561in}}%
\pgfpathlineto{\pgfqpoint{4.386174in}{2.566952in}}%
\pgfpathlineto{\pgfqpoint{4.399757in}{2.563418in}}%
\pgfpathlineto{\pgfqpoint{4.407458in}{2.574310in}}%
\pgfpathlineto{\pgfqpoint{4.415156in}{2.585345in}}%
\pgfpathlineto{\pgfqpoint{4.422849in}{2.596530in}}%
\pgfpathlineto{\pgfqpoint{4.430539in}{2.607871in}}%
\pgfpathlineto{\pgfqpoint{4.416967in}{2.611702in}}%
\pgfpathlineto{\pgfqpoint{4.403401in}{2.615609in}}%
\pgfpathlineto{\pgfqpoint{4.389841in}{2.619593in}}%
\pgfpathlineto{\pgfqpoint{4.376288in}{2.623653in}}%
\pgfpathlineto{\pgfqpoint{4.368587in}{2.612008in}}%
\pgfpathlineto{\pgfqpoint{4.360883in}{2.600523in}}%
\pgfpathlineto{\pgfqpoint{4.353174in}{2.589192in}}%
\pgfpathlineto{\pgfqpoint{4.345462in}{2.578011in}}%
\pgfpathclose%
\pgfusepath{fill}%
\end{pgfscope}%
\begin{pgfscope}%
\pgfpathrectangle{\pgfqpoint{1.150000in}{0.150000in}}{\pgfqpoint{5.700000in}{5.700000in}}%
\pgfusepath{clip}%
\pgfsetbuttcap%
\pgfsetroundjoin%
\definecolor{currentfill}{rgb}{0.283197,0.115680,0.436115}%
\pgfsetfillcolor{currentfill}%
\pgfsetfillopacity{0.700000}%
\pgfsetlinewidth{0.000000pt}%
\definecolor{currentstroke}{rgb}{0.000000,0.000000,0.000000}%
\pgfsetstrokecolor{currentstroke}%
\pgfsetdash{}{0pt}%
\pgfpathmoveto{\pgfqpoint{3.811499in}{2.482634in}}%
\pgfpathlineto{\pgfqpoint{3.824951in}{2.477779in}}%
\pgfpathlineto{\pgfqpoint{3.838408in}{2.473010in}}%
\pgfpathlineto{\pgfqpoint{3.851870in}{2.468327in}}%
\pgfpathlineto{\pgfqpoint{3.865337in}{2.463729in}}%
\pgfpathlineto{\pgfqpoint{3.873204in}{2.474077in}}%
\pgfpathlineto{\pgfqpoint{3.881065in}{2.484514in}}%
\pgfpathlineto{\pgfqpoint{3.888921in}{2.495042in}}%
\pgfpathlineto{\pgfqpoint{3.896772in}{2.505667in}}%
\pgfpathlineto{\pgfqpoint{3.883315in}{2.510461in}}%
\pgfpathlineto{\pgfqpoint{3.869862in}{2.515342in}}%
\pgfpathlineto{\pgfqpoint{3.856415in}{2.520308in}}%
\pgfpathlineto{\pgfqpoint{3.842972in}{2.525360in}}%
\pgfpathlineto{\pgfqpoint{3.835112in}{2.514531in}}%
\pgfpathlineto{\pgfqpoint{3.827247in}{2.503803in}}%
\pgfpathlineto{\pgfqpoint{3.819376in}{2.493172in}}%
\pgfpathlineto{\pgfqpoint{3.811499in}{2.482634in}}%
\pgfpathclose%
\pgfusepath{fill}%
\end{pgfscope}%
\begin{pgfscope}%
\pgfpathrectangle{\pgfqpoint{1.150000in}{0.150000in}}{\pgfqpoint{5.700000in}{5.700000in}}%
\pgfusepath{clip}%
\pgfsetbuttcap%
\pgfsetroundjoin%
\definecolor{currentfill}{rgb}{0.192357,0.403199,0.555836}%
\pgfsetfillcolor{currentfill}%
\pgfsetfillopacity{0.700000}%
\pgfsetlinewidth{0.000000pt}%
\definecolor{currentstroke}{rgb}{0.000000,0.000000,0.000000}%
\pgfsetstrokecolor{currentstroke}%
\pgfsetdash{}{0pt}%
\pgfpathmoveto{\pgfqpoint{5.506566in}{3.098746in}}%
\pgfpathlineto{\pgfqpoint{5.520345in}{3.093528in}}%
\pgfpathlineto{\pgfqpoint{5.534130in}{3.088377in}}%
\pgfpathlineto{\pgfqpoint{5.547924in}{3.083292in}}%
\pgfpathlineto{\pgfqpoint{5.561724in}{3.078273in}}%
\pgfpathlineto{\pgfqpoint{5.569324in}{3.098152in}}%
\pgfpathlineto{\pgfqpoint{5.576935in}{3.118510in}}%
\pgfpathlineto{\pgfqpoint{5.584559in}{3.139358in}}%
\pgfpathlineto{\pgfqpoint{5.592195in}{3.160706in}}%
\pgfpathlineto{\pgfqpoint{5.578411in}{3.166284in}}%
\pgfpathlineto{\pgfqpoint{5.564635in}{3.171929in}}%
\pgfpathlineto{\pgfqpoint{5.550865in}{3.177641in}}%
\pgfpathlineto{\pgfqpoint{5.537102in}{3.183418in}}%
\pgfpathlineto{\pgfqpoint{5.529450in}{3.161503in}}%
\pgfpathlineto{\pgfqpoint{5.521810in}{3.140093in}}%
\pgfpathlineto{\pgfqpoint{5.514182in}{3.119178in}}%
\pgfpathlineto{\pgfqpoint{5.506566in}{3.098746in}}%
\pgfpathclose%
\pgfusepath{fill}%
\end{pgfscope}%
\begin{pgfscope}%
\pgfpathrectangle{\pgfqpoint{1.150000in}{0.150000in}}{\pgfqpoint{5.700000in}{5.700000in}}%
\pgfusepath{clip}%
\pgfsetbuttcap%
\pgfsetroundjoin%
\definecolor{currentfill}{rgb}{0.273006,0.204520,0.501721}%
\pgfsetfillcolor{currentfill}%
\pgfsetfillopacity{0.700000}%
\pgfsetlinewidth{0.000000pt}%
\definecolor{currentstroke}{rgb}{0.000000,0.000000,0.000000}%
\pgfsetstrokecolor{currentstroke}%
\pgfsetdash{}{0pt}%
\pgfpathmoveto{\pgfqpoint{4.655061in}{2.656892in}}%
\pgfpathlineto{\pgfqpoint{4.668698in}{2.653375in}}%
\pgfpathlineto{\pgfqpoint{4.682342in}{2.649931in}}%
\pgfpathlineto{\pgfqpoint{4.695993in}{2.646560in}}%
\pgfpathlineto{\pgfqpoint{4.709652in}{2.643262in}}%
\pgfpathlineto{\pgfqpoint{4.717268in}{2.654860in}}%
\pgfpathlineto{\pgfqpoint{4.724882in}{2.666653in}}%
\pgfpathlineto{\pgfqpoint{4.732494in}{2.678649in}}%
\pgfpathlineto{\pgfqpoint{4.740105in}{2.690852in}}%
\pgfpathlineto{\pgfqpoint{4.726459in}{2.694508in}}%
\pgfpathlineto{\pgfqpoint{4.712821in}{2.698237in}}%
\pgfpathlineto{\pgfqpoint{4.699190in}{2.702038in}}%
\pgfpathlineto{\pgfqpoint{4.685565in}{2.705913in}}%
\pgfpathlineto{\pgfqpoint{4.677942in}{2.693344in}}%
\pgfpathlineto{\pgfqpoint{4.670317in}{2.680989in}}%
\pgfpathlineto{\pgfqpoint{4.662690in}{2.668840in}}%
\pgfpathlineto{\pgfqpoint{4.655061in}{2.656892in}}%
\pgfpathclose%
\pgfusepath{fill}%
\end{pgfscope}%
\begin{pgfscope}%
\pgfpathrectangle{\pgfqpoint{1.150000in}{0.150000in}}{\pgfqpoint{5.700000in}{5.700000in}}%
\pgfusepath{clip}%
\pgfsetbuttcap%
\pgfsetroundjoin%
\definecolor{currentfill}{rgb}{0.282884,0.135920,0.453427}%
\pgfsetfillcolor{currentfill}%
\pgfsetfillopacity{0.700000}%
\pgfsetlinewidth{0.000000pt}%
\definecolor{currentstroke}{rgb}{0.000000,0.000000,0.000000}%
\pgfsetstrokecolor{currentstroke}%
\pgfsetdash{}{0pt}%
\pgfpathmoveto{\pgfqpoint{4.035903in}{2.512137in}}%
\pgfpathlineto{\pgfqpoint{4.049400in}{2.507877in}}%
\pgfpathlineto{\pgfqpoint{4.062903in}{2.503698in}}%
\pgfpathlineto{\pgfqpoint{4.076412in}{2.499601in}}%
\pgfpathlineto{\pgfqpoint{4.089927in}{2.495585in}}%
\pgfpathlineto{\pgfqpoint{4.097723in}{2.506060in}}%
\pgfpathlineto{\pgfqpoint{4.105515in}{2.516640in}}%
\pgfpathlineto{\pgfqpoint{4.113301in}{2.527331in}}%
\pgfpathlineto{\pgfqpoint{4.121083in}{2.538136in}}%
\pgfpathlineto{\pgfqpoint{4.107579in}{2.542389in}}%
\pgfpathlineto{\pgfqpoint{4.094080in}{2.546723in}}%
\pgfpathlineto{\pgfqpoint{4.080587in}{2.551139in}}%
\pgfpathlineto{\pgfqpoint{4.067099in}{2.555637in}}%
\pgfpathlineto{\pgfqpoint{4.059307in}{2.544587in}}%
\pgfpathlineto{\pgfqpoint{4.051511in}{2.533657in}}%
\pgfpathlineto{\pgfqpoint{4.043709in}{2.522841in}}%
\pgfpathlineto{\pgfqpoint{4.035903in}{2.512137in}}%
\pgfpathclose%
\pgfusepath{fill}%
\end{pgfscope}%
\begin{pgfscope}%
\pgfpathrectangle{\pgfqpoint{1.150000in}{0.150000in}}{\pgfqpoint{5.700000in}{5.700000in}}%
\pgfusepath{clip}%
\pgfsetbuttcap%
\pgfsetroundjoin%
\definecolor{currentfill}{rgb}{0.282910,0.105393,0.426902}%
\pgfsetfillcolor{currentfill}%
\pgfsetfillopacity{0.700000}%
\pgfsetlinewidth{0.000000pt}%
\definecolor{currentstroke}{rgb}{0.000000,0.000000,0.000000}%
\pgfsetstrokecolor{currentstroke}%
\pgfsetdash{}{0pt}%
\pgfpathmoveto{\pgfqpoint{3.587014in}{2.462420in}}%
\pgfpathlineto{\pgfqpoint{3.600429in}{2.456829in}}%
\pgfpathlineto{\pgfqpoint{3.613848in}{2.451329in}}%
\pgfpathlineto{\pgfqpoint{3.627271in}{2.445921in}}%
\pgfpathlineto{\pgfqpoint{3.640699in}{2.440603in}}%
\pgfpathlineto{\pgfqpoint{3.648639in}{2.450817in}}%
\pgfpathlineto{\pgfqpoint{3.656573in}{2.461107in}}%
\pgfpathlineto{\pgfqpoint{3.664502in}{2.471477in}}%
\pgfpathlineto{\pgfqpoint{3.672425in}{2.481930in}}%
\pgfpathlineto{\pgfqpoint{3.659006in}{2.487404in}}%
\pgfpathlineto{\pgfqpoint{3.645592in}{2.492969in}}%
\pgfpathlineto{\pgfqpoint{3.632182in}{2.498626in}}%
\pgfpathlineto{\pgfqpoint{3.618777in}{2.504374in}}%
\pgfpathlineto{\pgfqpoint{3.610845in}{2.493757in}}%
\pgfpathlineto{\pgfqpoint{3.602907in}{2.483228in}}%
\pgfpathlineto{\pgfqpoint{3.594963in}{2.472783in}}%
\pgfpathlineto{\pgfqpoint{3.587014in}{2.462420in}}%
\pgfpathclose%
\pgfusepath{fill}%
\end{pgfscope}%
\begin{pgfscope}%
\pgfpathrectangle{\pgfqpoint{1.150000in}{0.150000in}}{\pgfqpoint{5.700000in}{5.700000in}}%
\pgfusepath{clip}%
\pgfsetbuttcap%
\pgfsetroundjoin%
\definecolor{currentfill}{rgb}{0.281887,0.150881,0.465405}%
\pgfsetfillcolor{currentfill}%
\pgfsetfillopacity{0.700000}%
\pgfsetlinewidth{0.000000pt}%
\definecolor{currentstroke}{rgb}{0.000000,0.000000,0.000000}%
\pgfsetstrokecolor{currentstroke}%
\pgfsetdash{}{0pt}%
\pgfpathmoveto{\pgfqpoint{2.890658in}{2.554889in}}%
\pgfpathlineto{\pgfqpoint{2.904026in}{2.545584in}}%
\pgfpathlineto{\pgfqpoint{2.917394in}{2.536399in}}%
\pgfpathlineto{\pgfqpoint{2.930763in}{2.527334in}}%
\pgfpathlineto{\pgfqpoint{2.944133in}{2.518387in}}%
\pgfpathlineto{\pgfqpoint{2.952305in}{2.528069in}}%
\pgfpathlineto{\pgfqpoint{2.960469in}{2.537834in}}%
\pgfpathlineto{\pgfqpoint{2.968627in}{2.547684in}}%
\pgfpathlineto{\pgfqpoint{2.976777in}{2.557620in}}%
\pgfpathlineto{\pgfqpoint{2.963419in}{2.566622in}}%
\pgfpathlineto{\pgfqpoint{2.950062in}{2.575743in}}%
\pgfpathlineto{\pgfqpoint{2.936705in}{2.584983in}}%
\pgfpathlineto{\pgfqpoint{2.923349in}{2.594343in}}%
\pgfpathlineto{\pgfqpoint{2.915188in}{2.584344in}}%
\pgfpathlineto{\pgfqpoint{2.907019in}{2.574437in}}%
\pgfpathlineto{\pgfqpoint{2.898842in}{2.564619in}}%
\pgfpathlineto{\pgfqpoint{2.890658in}{2.554889in}}%
\pgfpathclose%
\pgfusepath{fill}%
\end{pgfscope}%
\begin{pgfscope}%
\pgfpathrectangle{\pgfqpoint{1.150000in}{0.150000in}}{\pgfqpoint{5.700000in}{5.700000in}}%
\pgfusepath{clip}%
\pgfsetbuttcap%
\pgfsetroundjoin%
\definecolor{currentfill}{rgb}{0.276194,0.190074,0.493001}%
\pgfsetfillcolor{currentfill}%
\pgfsetfillopacity{0.700000}%
\pgfsetlinewidth{0.000000pt}%
\definecolor{currentstroke}{rgb}{0.000000,0.000000,0.000000}%
\pgfsetstrokecolor{currentstroke}%
\pgfsetdash{}{0pt}%
\pgfpathmoveto{\pgfqpoint{2.697274in}{2.637825in}}%
\pgfpathlineto{\pgfqpoint{2.710658in}{2.627029in}}%
\pgfpathlineto{\pgfqpoint{2.724042in}{2.616367in}}%
\pgfpathlineto{\pgfqpoint{2.737424in}{2.605837in}}%
\pgfpathlineto{\pgfqpoint{2.750806in}{2.595438in}}%
\pgfpathlineto{\pgfqpoint{2.759046in}{2.604882in}}%
\pgfpathlineto{\pgfqpoint{2.767278in}{2.614420in}}%
\pgfpathlineto{\pgfqpoint{2.775503in}{2.624052in}}%
\pgfpathlineto{\pgfqpoint{2.783719in}{2.633780in}}%
\pgfpathlineto{\pgfqpoint{2.770351in}{2.644214in}}%
\pgfpathlineto{\pgfqpoint{2.756981in}{2.654778in}}%
\pgfpathlineto{\pgfqpoint{2.743611in}{2.665475in}}%
\pgfpathlineto{\pgfqpoint{2.730240in}{2.676304in}}%
\pgfpathlineto{\pgfqpoint{2.722010in}{2.666534in}}%
\pgfpathlineto{\pgfqpoint{2.713773in}{2.656865in}}%
\pgfpathlineto{\pgfqpoint{2.705527in}{2.647296in}}%
\pgfpathlineto{\pgfqpoint{2.697274in}{2.637825in}}%
\pgfpathclose%
\pgfusepath{fill}%
\end{pgfscope}%
\begin{pgfscope}%
\pgfpathrectangle{\pgfqpoint{1.150000in}{0.150000in}}{\pgfqpoint{5.700000in}{5.700000in}}%
\pgfusepath{clip}%
\pgfsetbuttcap%
\pgfsetroundjoin%
\definecolor{currentfill}{rgb}{0.281412,0.155834,0.469201}%
\pgfsetfillcolor{currentfill}%
\pgfsetfillopacity{0.700000}%
\pgfsetlinewidth{0.000000pt}%
\definecolor{currentstroke}{rgb}{0.000000,0.000000,0.000000}%
\pgfsetstrokecolor{currentstroke}%
\pgfsetdash{}{0pt}%
\pgfpathmoveto{\pgfqpoint{4.260338in}{2.549386in}}%
\pgfpathlineto{\pgfqpoint{4.273887in}{2.545590in}}%
\pgfpathlineto{\pgfqpoint{4.287443in}{2.541873in}}%
\pgfpathlineto{\pgfqpoint{4.301005in}{2.538233in}}%
\pgfpathlineto{\pgfqpoint{4.314574in}{2.534670in}}%
\pgfpathlineto{\pgfqpoint{4.322302in}{2.545308in}}%
\pgfpathlineto{\pgfqpoint{4.330026in}{2.556074in}}%
\pgfpathlineto{\pgfqpoint{4.337746in}{2.566973in}}%
\pgfpathlineto{\pgfqpoint{4.345462in}{2.578011in}}%
\pgfpathlineto{\pgfqpoint{4.331904in}{2.581851in}}%
\pgfpathlineto{\pgfqpoint{4.318353in}{2.585769in}}%
\pgfpathlineto{\pgfqpoint{4.304808in}{2.589764in}}%
\pgfpathlineto{\pgfqpoint{4.291270in}{2.593837in}}%
\pgfpathlineto{\pgfqpoint{4.283543in}{2.582514in}}%
\pgfpathlineto{\pgfqpoint{4.275812in}{2.571335in}}%
\pgfpathlineto{\pgfqpoint{4.268077in}{2.560294in}}%
\pgfpathlineto{\pgfqpoint{4.260338in}{2.549386in}}%
\pgfpathclose%
\pgfusepath{fill}%
\end{pgfscope}%
\begin{pgfscope}%
\pgfpathrectangle{\pgfqpoint{1.150000in}{0.150000in}}{\pgfqpoint{5.700000in}{5.700000in}}%
\pgfusepath{clip}%
\pgfsetbuttcap%
\pgfsetroundjoin%
\definecolor{currentfill}{rgb}{0.235526,0.309527,0.542944}%
\pgfsetfillcolor{currentfill}%
\pgfsetfillopacity{0.700000}%
\pgfsetlinewidth{0.000000pt}%
\definecolor{currentstroke}{rgb}{0.000000,0.000000,0.000000}%
\pgfsetstrokecolor{currentstroke}%
\pgfsetdash{}{0pt}%
\pgfpathmoveto{\pgfqpoint{5.220305in}{2.872969in}}%
\pgfpathlineto{\pgfqpoint{5.234061in}{2.868978in}}%
\pgfpathlineto{\pgfqpoint{5.247824in}{2.865054in}}%
\pgfpathlineto{\pgfqpoint{5.261595in}{2.861199in}}%
\pgfpathlineto{\pgfqpoint{5.275373in}{2.857411in}}%
\pgfpathlineto{\pgfqpoint{5.282903in}{2.872376in}}%
\pgfpathlineto{\pgfqpoint{5.290436in}{2.887683in}}%
\pgfpathlineto{\pgfqpoint{5.297975in}{2.903339in}}%
\pgfpathlineto{\pgfqpoint{5.305519in}{2.919354in}}%
\pgfpathlineto{\pgfqpoint{5.291757in}{2.923620in}}%
\pgfpathlineto{\pgfqpoint{5.278003in}{2.927954in}}%
\pgfpathlineto{\pgfqpoint{5.264255in}{2.932355in}}%
\pgfpathlineto{\pgfqpoint{5.250516in}{2.936825in}}%
\pgfpathlineto{\pgfqpoint{5.242956in}{2.920324in}}%
\pgfpathlineto{\pgfqpoint{5.235401in}{2.904187in}}%
\pgfpathlineto{\pgfqpoint{5.227851in}{2.888405in}}%
\pgfpathlineto{\pgfqpoint{5.220305in}{2.872969in}}%
\pgfpathclose%
\pgfusepath{fill}%
\end{pgfscope}%
\begin{pgfscope}%
\pgfpathrectangle{\pgfqpoint{1.150000in}{0.150000in}}{\pgfqpoint{5.700000in}{5.700000in}}%
\pgfusepath{clip}%
\pgfsetbuttcap%
\pgfsetroundjoin%
\definecolor{currentfill}{rgb}{0.243113,0.292092,0.538516}%
\pgfsetfillcolor{currentfill}%
\pgfsetfillopacity{0.700000}%
\pgfsetlinewidth{0.000000pt}%
\definecolor{currentstroke}{rgb}{0.000000,0.000000,0.000000}%
\pgfsetstrokecolor{currentstroke}%
\pgfsetdash{}{0pt}%
\pgfpathmoveto{\pgfqpoint{5.135148in}{2.829335in}}%
\pgfpathlineto{\pgfqpoint{5.148890in}{2.825528in}}%
\pgfpathlineto{\pgfqpoint{5.162639in}{2.821790in}}%
\pgfpathlineto{\pgfqpoint{5.176396in}{2.818120in}}%
\pgfpathlineto{\pgfqpoint{5.190160in}{2.814518in}}%
\pgfpathlineto{\pgfqpoint{5.197691in}{2.828654in}}%
\pgfpathlineto{\pgfqpoint{5.205226in}{2.843102in}}%
\pgfpathlineto{\pgfqpoint{5.212764in}{2.857871in}}%
\pgfpathlineto{\pgfqpoint{5.220305in}{2.872969in}}%
\pgfpathlineto{\pgfqpoint{5.206557in}{2.877029in}}%
\pgfpathlineto{\pgfqpoint{5.192816in}{2.881157in}}%
\pgfpathlineto{\pgfqpoint{5.179083in}{2.885353in}}%
\pgfpathlineto{\pgfqpoint{5.165357in}{2.889618in}}%
\pgfpathlineto{\pgfqpoint{5.157800in}{2.874055in}}%
\pgfpathlineto{\pgfqpoint{5.150246in}{2.858826in}}%
\pgfpathlineto{\pgfqpoint{5.142696in}{2.843922in}}%
\pgfpathlineto{\pgfqpoint{5.135148in}{2.829335in}}%
\pgfpathclose%
\pgfusepath{fill}%
\end{pgfscope}%
\begin{pgfscope}%
\pgfpathrectangle{\pgfqpoint{1.150000in}{0.150000in}}{\pgfqpoint{5.700000in}{5.700000in}}%
\pgfusepath{clip}%
\pgfsetbuttcap%
\pgfsetroundjoin%
\definecolor{currentfill}{rgb}{0.275191,0.194905,0.496005}%
\pgfsetfillcolor{currentfill}%
\pgfsetfillopacity{0.700000}%
\pgfsetlinewidth{0.000000pt}%
\definecolor{currentstroke}{rgb}{0.000000,0.000000,0.000000}%
\pgfsetstrokecolor{currentstroke}%
\pgfsetdash{}{0pt}%
\pgfpathmoveto{\pgfqpoint{4.569994in}{2.624423in}}%
\pgfpathlineto{\pgfqpoint{4.583616in}{2.620950in}}%
\pgfpathlineto{\pgfqpoint{4.597244in}{2.617551in}}%
\pgfpathlineto{\pgfqpoint{4.610880in}{2.614225in}}%
\pgfpathlineto{\pgfqpoint{4.624522in}{2.610973in}}%
\pgfpathlineto{\pgfqpoint{4.632161in}{2.622184in}}%
\pgfpathlineto{\pgfqpoint{4.639797in}{2.633570in}}%
\pgfpathlineto{\pgfqpoint{4.647430in}{2.645137in}}%
\pgfpathlineto{\pgfqpoint{4.655061in}{2.656892in}}%
\pgfpathlineto{\pgfqpoint{4.641431in}{2.660482in}}%
\pgfpathlineto{\pgfqpoint{4.627808in}{2.664145in}}%
\pgfpathlineto{\pgfqpoint{4.614192in}{2.667882in}}%
\pgfpathlineto{\pgfqpoint{4.600583in}{2.671693in}}%
\pgfpathlineto{\pgfqpoint{4.592939in}{2.659593in}}%
\pgfpathlineto{\pgfqpoint{4.585294in}{2.647686in}}%
\pgfpathlineto{\pgfqpoint{4.577645in}{2.635965in}}%
\pgfpathlineto{\pgfqpoint{4.569994in}{2.624423in}}%
\pgfpathclose%
\pgfusepath{fill}%
\end{pgfscope}%
\begin{pgfscope}%
\pgfpathrectangle{\pgfqpoint{1.150000in}{0.150000in}}{\pgfqpoint{5.700000in}{5.700000in}}%
\pgfusepath{clip}%
\pgfsetbuttcap%
\pgfsetroundjoin%
\definecolor{currentfill}{rgb}{0.225863,0.330805,0.547314}%
\pgfsetfillcolor{currentfill}%
\pgfsetfillopacity{0.700000}%
\pgfsetlinewidth{0.000000pt}%
\definecolor{currentstroke}{rgb}{0.000000,0.000000,0.000000}%
\pgfsetstrokecolor{currentstroke}%
\pgfsetdash{}{0pt}%
\pgfpathmoveto{\pgfqpoint{5.305519in}{2.919354in}}%
\pgfpathlineto{\pgfqpoint{5.319288in}{2.915156in}}%
\pgfpathlineto{\pgfqpoint{5.333065in}{2.911025in}}%
\pgfpathlineto{\pgfqpoint{5.346850in}{2.906962in}}%
\pgfpathlineto{\pgfqpoint{5.360642in}{2.902966in}}%
\pgfpathlineto{\pgfqpoint{5.368175in}{2.918858in}}%
\pgfpathlineto{\pgfqpoint{5.375713in}{2.935122in}}%
\pgfpathlineto{\pgfqpoint{5.383259in}{2.951768in}}%
\pgfpathlineto{\pgfqpoint{5.390811in}{2.968803in}}%
\pgfpathlineto{\pgfqpoint{5.377035in}{2.973298in}}%
\pgfpathlineto{\pgfqpoint{5.363268in}{2.977859in}}%
\pgfpathlineto{\pgfqpoint{5.349507in}{2.982489in}}%
\pgfpathlineto{\pgfqpoint{5.335754in}{2.987185in}}%
\pgfpathlineto{\pgfqpoint{5.328186in}{2.969644in}}%
\pgfpathlineto{\pgfqpoint{5.320624in}{2.952498in}}%
\pgfpathlineto{\pgfqpoint{5.313068in}{2.935738in}}%
\pgfpathlineto{\pgfqpoint{5.305519in}{2.919354in}}%
\pgfpathclose%
\pgfusepath{fill}%
\end{pgfscope}%
\begin{pgfscope}%
\pgfpathrectangle{\pgfqpoint{1.150000in}{0.150000in}}{\pgfqpoint{5.700000in}{5.700000in}}%
\pgfusepath{clip}%
\pgfsetbuttcap%
\pgfsetroundjoin%
\definecolor{currentfill}{rgb}{0.252194,0.269783,0.531579}%
\pgfsetfillcolor{currentfill}%
\pgfsetfillopacity{0.700000}%
\pgfsetlinewidth{0.000000pt}%
\definecolor{currentstroke}{rgb}{0.000000,0.000000,0.000000}%
\pgfsetstrokecolor{currentstroke}%
\pgfsetdash{}{0pt}%
\pgfpathmoveto{\pgfqpoint{5.050031in}{2.788166in}}%
\pgfpathlineto{\pgfqpoint{5.063758in}{2.784521in}}%
\pgfpathlineto{\pgfqpoint{5.077492in}{2.780945in}}%
\pgfpathlineto{\pgfqpoint{5.091234in}{2.777438in}}%
\pgfpathlineto{\pgfqpoint{5.104984in}{2.774000in}}%
\pgfpathlineto{\pgfqpoint{5.112522in}{2.787399in}}%
\pgfpathlineto{\pgfqpoint{5.120062in}{2.801082in}}%
\pgfpathlineto{\pgfqpoint{5.127604in}{2.815058in}}%
\pgfpathlineto{\pgfqpoint{5.135148in}{2.829335in}}%
\pgfpathlineto{\pgfqpoint{5.121414in}{2.833211in}}%
\pgfpathlineto{\pgfqpoint{5.107688in}{2.837156in}}%
\pgfpathlineto{\pgfqpoint{5.093969in}{2.841170in}}%
\pgfpathlineto{\pgfqpoint{5.080257in}{2.845254in}}%
\pgfpathlineto{\pgfqpoint{5.072697in}{2.830531in}}%
\pgfpathlineto{\pgfqpoint{5.065139in}{2.816114in}}%
\pgfpathlineto{\pgfqpoint{5.057584in}{2.801995in}}%
\pgfpathlineto{\pgfqpoint{5.050031in}{2.788166in}}%
\pgfpathclose%
\pgfusepath{fill}%
\end{pgfscope}%
\begin{pgfscope}%
\pgfpathrectangle{\pgfqpoint{1.150000in}{0.150000in}}{\pgfqpoint{5.700000in}{5.700000in}}%
\pgfusepath{clip}%
\pgfsetbuttcap%
\pgfsetroundjoin%
\definecolor{currentfill}{rgb}{0.216210,0.351535,0.550627}%
\pgfsetfillcolor{currentfill}%
\pgfsetfillopacity{0.700000}%
\pgfsetlinewidth{0.000000pt}%
\definecolor{currentstroke}{rgb}{0.000000,0.000000,0.000000}%
\pgfsetstrokecolor{currentstroke}%
\pgfsetdash{}{0pt}%
\pgfpathmoveto{\pgfqpoint{5.390811in}{2.968803in}}%
\pgfpathlineto{\pgfqpoint{5.404593in}{2.964376in}}%
\pgfpathlineto{\pgfqpoint{5.418384in}{2.960015in}}%
\pgfpathlineto{\pgfqpoint{5.432181in}{2.955722in}}%
\pgfpathlineto{\pgfqpoint{5.445987in}{2.951495in}}%
\pgfpathlineto{\pgfqpoint{5.453529in}{2.968419in}}%
\pgfpathlineto{\pgfqpoint{5.461079in}{2.985747in}}%
\pgfpathlineto{\pgfqpoint{5.468637in}{3.003489in}}%
\pgfpathlineto{\pgfqpoint{5.476204in}{3.021654in}}%
\pgfpathlineto{\pgfqpoint{5.462416in}{3.026400in}}%
\pgfpathlineto{\pgfqpoint{5.448635in}{3.031212in}}%
\pgfpathlineto{\pgfqpoint{5.434861in}{3.036091in}}%
\pgfpathlineto{\pgfqpoint{5.421095in}{3.041037in}}%
\pgfpathlineto{\pgfqpoint{5.413512in}{3.022346in}}%
\pgfpathlineto{\pgfqpoint{5.405937in}{3.004083in}}%
\pgfpathlineto{\pgfqpoint{5.398370in}{2.986238in}}%
\pgfpathlineto{\pgfqpoint{5.390811in}{2.968803in}}%
\pgfpathclose%
\pgfusepath{fill}%
\end{pgfscope}%
\begin{pgfscope}%
\pgfpathrectangle{\pgfqpoint{1.150000in}{0.150000in}}{\pgfqpoint{5.700000in}{5.700000in}}%
\pgfusepath{clip}%
\pgfsetbuttcap%
\pgfsetroundjoin%
\definecolor{currentfill}{rgb}{0.283091,0.110553,0.431554}%
\pgfsetfillcolor{currentfill}%
\pgfsetfillopacity{0.700000}%
\pgfsetlinewidth{0.000000pt}%
\definecolor{currentstroke}{rgb}{0.000000,0.000000,0.000000}%
\pgfsetstrokecolor{currentstroke}%
\pgfsetdash{}{0pt}%
\pgfpathmoveto{\pgfqpoint{3.223047in}{2.468388in}}%
\pgfpathlineto{\pgfqpoint{3.236423in}{2.461192in}}%
\pgfpathlineto{\pgfqpoint{3.249801in}{2.454099in}}%
\pgfpathlineto{\pgfqpoint{3.263183in}{2.447109in}}%
\pgfpathlineto{\pgfqpoint{3.276567in}{2.440221in}}%
\pgfpathlineto{\pgfqpoint{3.284629in}{2.450169in}}%
\pgfpathlineto{\pgfqpoint{3.292684in}{2.460189in}}%
\pgfpathlineto{\pgfqpoint{3.300733in}{2.470282in}}%
\pgfpathlineto{\pgfqpoint{3.308776in}{2.480451in}}%
\pgfpathlineto{\pgfqpoint{3.295402in}{2.487435in}}%
\pgfpathlineto{\pgfqpoint{3.282031in}{2.494521in}}%
\pgfpathlineto{\pgfqpoint{3.268662in}{2.501710in}}%
\pgfpathlineto{\pgfqpoint{3.255297in}{2.509003in}}%
\pgfpathlineto{\pgfqpoint{3.247244in}{2.498731in}}%
\pgfpathlineto{\pgfqpoint{3.239185in}{2.488539in}}%
\pgfpathlineto{\pgfqpoint{3.231119in}{2.478425in}}%
\pgfpathlineto{\pgfqpoint{3.223047in}{2.468388in}}%
\pgfpathclose%
\pgfusepath{fill}%
\end{pgfscope}%
\begin{pgfscope}%
\pgfpathrectangle{\pgfqpoint{1.150000in}{0.150000in}}{\pgfqpoint{5.700000in}{5.700000in}}%
\pgfusepath{clip}%
\pgfsetbuttcap%
\pgfsetroundjoin%
\definecolor{currentfill}{rgb}{0.182256,0.426184,0.557120}%
\pgfsetfillcolor{currentfill}%
\pgfsetfillopacity{0.700000}%
\pgfsetlinewidth{0.000000pt}%
\definecolor{currentstroke}{rgb}{0.000000,0.000000,0.000000}%
\pgfsetstrokecolor{currentstroke}%
\pgfsetdash{}{0pt}%
\pgfpathmoveto{\pgfqpoint{5.592195in}{3.160706in}}%
\pgfpathlineto{\pgfqpoint{5.605986in}{3.155193in}}%
\pgfpathlineto{\pgfqpoint{5.619784in}{3.149746in}}%
\pgfpathlineto{\pgfqpoint{5.633589in}{3.144365in}}%
\pgfpathlineto{\pgfqpoint{5.647401in}{3.139051in}}%
\pgfpathlineto{\pgfqpoint{5.655033in}{3.160336in}}%
\pgfpathlineto{\pgfqpoint{5.662679in}{3.182138in}}%
\pgfpathlineto{\pgfqpoint{5.670340in}{3.204468in}}%
\pgfpathlineto{\pgfqpoint{5.656540in}{3.210216in}}%
\pgfpathlineto{\pgfqpoint{5.642747in}{3.216030in}}%
\pgfpathlineto{\pgfqpoint{5.628962in}{3.221911in}}%
\pgfpathlineto{\pgfqpoint{5.615183in}{3.227857in}}%
\pgfpathlineto{\pgfqpoint{5.607507in}{3.204944in}}%
\pgfpathlineto{\pgfqpoint{5.599844in}{3.182564in}}%
\pgfpathlineto{\pgfqpoint{5.592195in}{3.160706in}}%
\pgfpathclose%
\pgfusepath{fill}%
\end{pgfscope}%
\begin{pgfscope}%
\pgfpathrectangle{\pgfqpoint{1.150000in}{0.150000in}}{\pgfqpoint{5.700000in}{5.700000in}}%
\pgfusepath{clip}%
\pgfsetbuttcap%
\pgfsetroundjoin%
\definecolor{currentfill}{rgb}{0.257322,0.256130,0.526563}%
\pgfsetfillcolor{currentfill}%
\pgfsetfillopacity{0.700000}%
\pgfsetlinewidth{0.000000pt}%
\definecolor{currentstroke}{rgb}{0.000000,0.000000,0.000000}%
\pgfsetstrokecolor{currentstroke}%
\pgfsetdash{}{0pt}%
\pgfpathmoveto{\pgfqpoint{4.964935in}{2.749199in}}%
\pgfpathlineto{\pgfqpoint{4.978647in}{2.745693in}}%
\pgfpathlineto{\pgfqpoint{4.992367in}{2.742257in}}%
\pgfpathlineto{\pgfqpoint{5.006094in}{2.738891in}}%
\pgfpathlineto{\pgfqpoint{5.019829in}{2.735594in}}%
\pgfpathlineto{\pgfqpoint{5.027378in}{2.748341in}}%
\pgfpathlineto{\pgfqpoint{5.034928in}{2.761347in}}%
\pgfpathlineto{\pgfqpoint{5.042478in}{2.774619in}}%
\pgfpathlineto{\pgfqpoint{5.050031in}{2.788166in}}%
\pgfpathlineto{\pgfqpoint{5.036311in}{2.791881in}}%
\pgfpathlineto{\pgfqpoint{5.022599in}{2.795665in}}%
\pgfpathlineto{\pgfqpoint{5.008894in}{2.799520in}}%
\pgfpathlineto{\pgfqpoint{4.995197in}{2.803444in}}%
\pgfpathlineto{\pgfqpoint{4.987630in}{2.789471in}}%
\pgfpathlineto{\pgfqpoint{4.980064in}{2.775778in}}%
\pgfpathlineto{\pgfqpoint{4.972499in}{2.762357in}}%
\pgfpathlineto{\pgfqpoint{4.964935in}{2.749199in}}%
\pgfpathclose%
\pgfusepath{fill}%
\end{pgfscope}%
\begin{pgfscope}%
\pgfpathrectangle{\pgfqpoint{1.150000in}{0.150000in}}{\pgfqpoint{5.700000in}{5.700000in}}%
\pgfusepath{clip}%
\pgfsetbuttcap%
\pgfsetroundjoin%
\definecolor{currentfill}{rgb}{0.283187,0.125848,0.444960}%
\pgfsetfillcolor{currentfill}%
\pgfsetfillopacity{0.700000}%
\pgfsetlinewidth{0.000000pt}%
\definecolor{currentstroke}{rgb}{0.000000,0.000000,0.000000}%
\pgfsetstrokecolor{currentstroke}%
\pgfsetdash{}{0pt}%
\pgfpathmoveto{\pgfqpoint{3.950656in}{2.487334in}}%
\pgfpathlineto{\pgfqpoint{3.964141in}{2.482961in}}%
\pgfpathlineto{\pgfqpoint{3.977631in}{2.478671in}}%
\pgfpathlineto{\pgfqpoint{3.991126in}{2.474464in}}%
\pgfpathlineto{\pgfqpoint{4.004628in}{2.470339in}}%
\pgfpathlineto{\pgfqpoint{4.012454in}{2.480643in}}%
\pgfpathlineto{\pgfqpoint{4.020275in}{2.491042in}}%
\pgfpathlineto{\pgfqpoint{4.028092in}{2.501538in}}%
\pgfpathlineto{\pgfqpoint{4.035903in}{2.512137in}}%
\pgfpathlineto{\pgfqpoint{4.022411in}{2.516479in}}%
\pgfpathlineto{\pgfqpoint{4.008925in}{2.520904in}}%
\pgfpathlineto{\pgfqpoint{3.995445in}{2.525411in}}%
\pgfpathlineto{\pgfqpoint{3.981970in}{2.530002in}}%
\pgfpathlineto{\pgfqpoint{3.974149in}{2.519178in}}%
\pgfpathlineto{\pgfqpoint{3.966323in}{2.508462in}}%
\pgfpathlineto{\pgfqpoint{3.958492in}{2.497849in}}%
\pgfpathlineto{\pgfqpoint{3.950656in}{2.487334in}}%
\pgfpathclose%
\pgfusepath{fill}%
\end{pgfscope}%
\begin{pgfscope}%
\pgfpathrectangle{\pgfqpoint{1.150000in}{0.150000in}}{\pgfqpoint{5.700000in}{5.700000in}}%
\pgfusepath{clip}%
\pgfsetbuttcap%
\pgfsetroundjoin%
\definecolor{currentfill}{rgb}{0.283091,0.110553,0.431554}%
\pgfsetfillcolor{currentfill}%
\pgfsetfillopacity{0.700000}%
\pgfsetlinewidth{0.000000pt}%
\definecolor{currentstroke}{rgb}{0.000000,0.000000,0.000000}%
\pgfsetstrokecolor{currentstroke}%
\pgfsetdash{}{0pt}%
\pgfpathmoveto{\pgfqpoint{3.726144in}{2.460930in}}%
\pgfpathlineto{\pgfqpoint{3.739585in}{2.455902in}}%
\pgfpathlineto{\pgfqpoint{3.753032in}{2.450963in}}%
\pgfpathlineto{\pgfqpoint{3.766483in}{2.446111in}}%
\pgfpathlineto{\pgfqpoint{3.779939in}{2.441346in}}%
\pgfpathlineto{\pgfqpoint{3.787837in}{2.451546in}}%
\pgfpathlineto{\pgfqpoint{3.795730in}{2.461825in}}%
\pgfpathlineto{\pgfqpoint{3.803617in}{2.472186in}}%
\pgfpathlineto{\pgfqpoint{3.811499in}{2.482634in}}%
\pgfpathlineto{\pgfqpoint{3.798052in}{2.487576in}}%
\pgfpathlineto{\pgfqpoint{3.784611in}{2.492605in}}%
\pgfpathlineto{\pgfqpoint{3.771174in}{2.497722in}}%
\pgfpathlineto{\pgfqpoint{3.757742in}{2.502927in}}%
\pgfpathlineto{\pgfqpoint{3.749850in}{2.492294in}}%
\pgfpathlineto{\pgfqpoint{3.741954in}{2.481753in}}%
\pgfpathlineto{\pgfqpoint{3.734052in}{2.471299in}}%
\pgfpathlineto{\pgfqpoint{3.726144in}{2.460930in}}%
\pgfpathclose%
\pgfusepath{fill}%
\end{pgfscope}%
\begin{pgfscope}%
\pgfpathrectangle{\pgfqpoint{1.150000in}{0.150000in}}{\pgfqpoint{5.700000in}{5.700000in}}%
\pgfusepath{clip}%
\pgfsetbuttcap%
\pgfsetroundjoin%
\definecolor{currentfill}{rgb}{0.283229,0.120777,0.440584}%
\pgfsetfillcolor{currentfill}%
\pgfsetfillopacity{0.700000}%
\pgfsetlinewidth{0.000000pt}%
\definecolor{currentstroke}{rgb}{0.000000,0.000000,0.000000}%
\pgfsetstrokecolor{currentstroke}%
\pgfsetdash{}{0pt}%
\pgfpathmoveto{\pgfqpoint{3.083680in}{2.489754in}}%
\pgfpathlineto{\pgfqpoint{3.097050in}{2.481777in}}%
\pgfpathlineto{\pgfqpoint{3.110421in}{2.473910in}}%
\pgfpathlineto{\pgfqpoint{3.123794in}{2.466152in}}%
\pgfpathlineto{\pgfqpoint{3.137169in}{2.458501in}}%
\pgfpathlineto{\pgfqpoint{3.145279in}{2.468320in}}%
\pgfpathlineto{\pgfqpoint{3.153381in}{2.478213in}}%
\pgfpathlineto{\pgfqpoint{3.161477in}{2.488181in}}%
\pgfpathlineto{\pgfqpoint{3.169567in}{2.498227in}}%
\pgfpathlineto{\pgfqpoint{3.156202in}{2.505953in}}%
\pgfpathlineto{\pgfqpoint{3.142840in}{2.513788in}}%
\pgfpathlineto{\pgfqpoint{3.129479in}{2.521731in}}%
\pgfpathlineto{\pgfqpoint{3.116121in}{2.529784in}}%
\pgfpathlineto{\pgfqpoint{3.108021in}{2.519654in}}%
\pgfpathlineto{\pgfqpoint{3.099914in}{2.509607in}}%
\pgfpathlineto{\pgfqpoint{3.091801in}{2.499641in}}%
\pgfpathlineto{\pgfqpoint{3.083680in}{2.489754in}}%
\pgfpathclose%
\pgfusepath{fill}%
\end{pgfscope}%
\begin{pgfscope}%
\pgfpathrectangle{\pgfqpoint{1.150000in}{0.150000in}}{\pgfqpoint{5.700000in}{5.700000in}}%
\pgfusepath{clip}%
\pgfsetbuttcap%
\pgfsetroundjoin%
\definecolor{currentfill}{rgb}{0.282910,0.105393,0.426902}%
\pgfsetfillcolor{currentfill}%
\pgfsetfillopacity{0.700000}%
\pgfsetlinewidth{0.000000pt}%
\definecolor{currentstroke}{rgb}{0.000000,0.000000,0.000000}%
\pgfsetstrokecolor{currentstroke}%
\pgfsetdash{}{0pt}%
\pgfpathmoveto{\pgfqpoint{3.362301in}{2.453526in}}%
\pgfpathlineto{\pgfqpoint{3.375690in}{2.447044in}}%
\pgfpathlineto{\pgfqpoint{3.389082in}{2.440660in}}%
\pgfpathlineto{\pgfqpoint{3.402478in}{2.434374in}}%
\pgfpathlineto{\pgfqpoint{3.415878in}{2.428186in}}%
\pgfpathlineto{\pgfqpoint{3.423894in}{2.438215in}}%
\pgfpathlineto{\pgfqpoint{3.431904in}{2.448314in}}%
\pgfpathlineto{\pgfqpoint{3.439909in}{2.458485in}}%
\pgfpathlineto{\pgfqpoint{3.447907in}{2.468732in}}%
\pgfpathlineto{\pgfqpoint{3.434517in}{2.475037in}}%
\pgfpathlineto{\pgfqpoint{3.421131in}{2.481439in}}%
\pgfpathlineto{\pgfqpoint{3.407749in}{2.487939in}}%
\pgfpathlineto{\pgfqpoint{3.394369in}{2.494537in}}%
\pgfpathlineto{\pgfqpoint{3.386361in}{2.484167in}}%
\pgfpathlineto{\pgfqpoint{3.378347in}{2.473876in}}%
\pgfpathlineto{\pgfqpoint{3.370327in}{2.463663in}}%
\pgfpathlineto{\pgfqpoint{3.362301in}{2.453526in}}%
\pgfpathclose%
\pgfusepath{fill}%
\end{pgfscope}%
\begin{pgfscope}%
\pgfpathrectangle{\pgfqpoint{1.150000in}{0.150000in}}{\pgfqpoint{5.700000in}{5.700000in}}%
\pgfusepath{clip}%
\pgfsetbuttcap%
\pgfsetroundjoin%
\definecolor{currentfill}{rgb}{0.278012,0.180367,0.486697}%
\pgfsetfillcolor{currentfill}%
\pgfsetfillopacity{0.700000}%
\pgfsetlinewidth{0.000000pt}%
\definecolor{currentstroke}{rgb}{0.000000,0.000000,0.000000}%
\pgfsetstrokecolor{currentstroke}%
\pgfsetdash{}{0pt}%
\pgfpathmoveto{\pgfqpoint{4.484895in}{2.593305in}}%
\pgfpathlineto{\pgfqpoint{4.498501in}{2.589851in}}%
\pgfpathlineto{\pgfqpoint{4.512114in}{2.586473in}}%
\pgfpathlineto{\pgfqpoint{4.525734in}{2.583170in}}%
\pgfpathlineto{\pgfqpoint{4.539361in}{2.579940in}}%
\pgfpathlineto{\pgfqpoint{4.547024in}{2.590820in}}%
\pgfpathlineto{\pgfqpoint{4.554684in}{2.601857in}}%
\pgfpathlineto{\pgfqpoint{4.562340in}{2.613056in}}%
\pgfpathlineto{\pgfqpoint{4.569994in}{2.624423in}}%
\pgfpathlineto{\pgfqpoint{4.556380in}{2.627971in}}%
\pgfpathlineto{\pgfqpoint{4.542772in}{2.631592in}}%
\pgfpathlineto{\pgfqpoint{4.529171in}{2.635288in}}%
\pgfpathlineto{\pgfqpoint{4.515577in}{2.639059in}}%
\pgfpathlineto{\pgfqpoint{4.507911in}{2.627367in}}%
\pgfpathlineto{\pgfqpoint{4.500243in}{2.615848in}}%
\pgfpathlineto{\pgfqpoint{4.492571in}{2.604495in}}%
\pgfpathlineto{\pgfqpoint{4.484895in}{2.593305in}}%
\pgfpathclose%
\pgfusepath{fill}%
\end{pgfscope}%
\begin{pgfscope}%
\pgfpathrectangle{\pgfqpoint{1.150000in}{0.150000in}}{\pgfqpoint{5.700000in}{5.700000in}}%
\pgfusepath{clip}%
\pgfsetbuttcap%
\pgfsetroundjoin%
\definecolor{currentfill}{rgb}{0.279574,0.170599,0.479997}%
\pgfsetfillcolor{currentfill}%
\pgfsetfillopacity{0.700000}%
\pgfsetlinewidth{0.000000pt}%
\definecolor{currentstroke}{rgb}{0.000000,0.000000,0.000000}%
\pgfsetstrokecolor{currentstroke}%
\pgfsetdash{}{0pt}%
\pgfpathmoveto{\pgfqpoint{2.750806in}{2.595438in}}%
\pgfpathlineto{\pgfqpoint{2.764187in}{2.585169in}}%
\pgfpathlineto{\pgfqpoint{2.777568in}{2.575028in}}%
\pgfpathlineto{\pgfqpoint{2.790948in}{2.565015in}}%
\pgfpathlineto{\pgfqpoint{2.804328in}{2.555129in}}%
\pgfpathlineto{\pgfqpoint{2.812555in}{2.564545in}}%
\pgfpathlineto{\pgfqpoint{2.820774in}{2.574051in}}%
\pgfpathlineto{\pgfqpoint{2.828986in}{2.583646in}}%
\pgfpathlineto{\pgfqpoint{2.837190in}{2.593331in}}%
\pgfpathlineto{\pgfqpoint{2.823822in}{2.603253in}}%
\pgfpathlineto{\pgfqpoint{2.810455in}{2.613301in}}%
\pgfpathlineto{\pgfqpoint{2.797087in}{2.623476in}}%
\pgfpathlineto{\pgfqpoint{2.783719in}{2.633780in}}%
\pgfpathlineto{\pgfqpoint{2.775503in}{2.624052in}}%
\pgfpathlineto{\pgfqpoint{2.767278in}{2.614420in}}%
\pgfpathlineto{\pgfqpoint{2.759046in}{2.604882in}}%
\pgfpathlineto{\pgfqpoint{2.750806in}{2.595438in}}%
\pgfpathclose%
\pgfusepath{fill}%
\end{pgfscope}%
\begin{pgfscope}%
\pgfpathrectangle{\pgfqpoint{1.150000in}{0.150000in}}{\pgfqpoint{5.700000in}{5.700000in}}%
\pgfusepath{clip}%
\pgfsetbuttcap%
\pgfsetroundjoin%
\definecolor{currentfill}{rgb}{0.204903,0.375746,0.553533}%
\pgfsetfillcolor{currentfill}%
\pgfsetfillopacity{0.700000}%
\pgfsetlinewidth{0.000000pt}%
\definecolor{currentstroke}{rgb}{0.000000,0.000000,0.000000}%
\pgfsetstrokecolor{currentstroke}%
\pgfsetdash{}{0pt}%
\pgfpathmoveto{\pgfqpoint{5.476204in}{3.021654in}}%
\pgfpathlineto{\pgfqpoint{5.489999in}{3.016975in}}%
\pgfpathlineto{\pgfqpoint{5.503802in}{3.012363in}}%
\pgfpathlineto{\pgfqpoint{5.517613in}{3.007817in}}%
\pgfpathlineto{\pgfqpoint{5.531431in}{3.003337in}}%
\pgfpathlineto{\pgfqpoint{5.538989in}{3.021404in}}%
\pgfpathlineto{\pgfqpoint{5.546557in}{3.039909in}}%
\pgfpathlineto{\pgfqpoint{5.554135in}{3.058862in}}%
\pgfpathlineto{\pgfqpoint{5.561724in}{3.078273in}}%
\pgfpathlineto{\pgfqpoint{5.547924in}{3.083292in}}%
\pgfpathlineto{\pgfqpoint{5.534130in}{3.088377in}}%
\pgfpathlineto{\pgfqpoint{5.520345in}{3.093528in}}%
\pgfpathlineto{\pgfqpoint{5.506566in}{3.098746in}}%
\pgfpathlineto{\pgfqpoint{5.498960in}{3.078789in}}%
\pgfpathlineto{\pgfqpoint{5.491365in}{3.059294in}}%
\pgfpathlineto{\pgfqpoint{5.483780in}{3.040253in}}%
\pgfpathlineto{\pgfqpoint{5.476204in}{3.021654in}}%
\pgfpathclose%
\pgfusepath{fill}%
\end{pgfscope}%
\begin{pgfscope}%
\pgfpathrectangle{\pgfqpoint{1.150000in}{0.150000in}}{\pgfqpoint{5.700000in}{5.700000in}}%
\pgfusepath{clip}%
\pgfsetbuttcap%
\pgfsetroundjoin%
\definecolor{currentfill}{rgb}{0.282290,0.145912,0.461510}%
\pgfsetfillcolor{currentfill}%
\pgfsetfillopacity{0.700000}%
\pgfsetlinewidth{0.000000pt}%
\definecolor{currentstroke}{rgb}{0.000000,0.000000,0.000000}%
\pgfsetstrokecolor{currentstroke}%
\pgfsetdash{}{0pt}%
\pgfpathmoveto{\pgfqpoint{4.175162in}{2.521926in}}%
\pgfpathlineto{\pgfqpoint{4.188697in}{2.518074in}}%
\pgfpathlineto{\pgfqpoint{4.202238in}{2.514300in}}%
\pgfpathlineto{\pgfqpoint{4.215785in}{2.510605in}}%
\pgfpathlineto{\pgfqpoint{4.229338in}{2.506989in}}%
\pgfpathlineto{\pgfqpoint{4.237095in}{2.517413in}}%
\pgfpathlineto{\pgfqpoint{4.244847in}{2.527950in}}%
\pgfpathlineto{\pgfqpoint{4.252595in}{2.538606in}}%
\pgfpathlineto{\pgfqpoint{4.260338in}{2.549386in}}%
\pgfpathlineto{\pgfqpoint{4.246795in}{2.553260in}}%
\pgfpathlineto{\pgfqpoint{4.233259in}{2.557212in}}%
\pgfpathlineto{\pgfqpoint{4.219728in}{2.561244in}}%
\pgfpathlineto{\pgfqpoint{4.206204in}{2.565354in}}%
\pgfpathlineto{\pgfqpoint{4.198450in}{2.554309in}}%
\pgfpathlineto{\pgfqpoint{4.190692in}{2.543393in}}%
\pgfpathlineto{\pgfqpoint{4.182929in}{2.532600in}}%
\pgfpathlineto{\pgfqpoint{4.175162in}{2.521926in}}%
\pgfpathclose%
\pgfusepath{fill}%
\end{pgfscope}%
\begin{pgfscope}%
\pgfpathrectangle{\pgfqpoint{1.150000in}{0.150000in}}{\pgfqpoint{5.700000in}{5.700000in}}%
\pgfusepath{clip}%
\pgfsetbuttcap%
\pgfsetroundjoin%
\definecolor{currentfill}{rgb}{0.263663,0.237631,0.518762}%
\pgfsetfillcolor{currentfill}%
\pgfsetfillopacity{0.700000}%
\pgfsetlinewidth{0.000000pt}%
\definecolor{currentstroke}{rgb}{0.000000,0.000000,0.000000}%
\pgfsetstrokecolor{currentstroke}%
\pgfsetdash{}{0pt}%
\pgfpathmoveto{\pgfqpoint{4.879848in}{2.712198in}}%
\pgfpathlineto{\pgfqpoint{4.893545in}{2.708808in}}%
\pgfpathlineto{\pgfqpoint{4.907249in}{2.705489in}}%
\pgfpathlineto{\pgfqpoint{4.920961in}{2.702241in}}%
\pgfpathlineto{\pgfqpoint{4.934681in}{2.699063in}}%
\pgfpathlineto{\pgfqpoint{4.942244in}{2.711238in}}%
\pgfpathlineto{\pgfqpoint{4.949808in}{2.723647in}}%
\pgfpathlineto{\pgfqpoint{4.957371in}{2.736299in}}%
\pgfpathlineto{\pgfqpoint{4.964935in}{2.749199in}}%
\pgfpathlineto{\pgfqpoint{4.951231in}{2.752776in}}%
\pgfpathlineto{\pgfqpoint{4.937533in}{2.756422in}}%
\pgfpathlineto{\pgfqpoint{4.923843in}{2.760140in}}%
\pgfpathlineto{\pgfqpoint{4.910161in}{2.763927in}}%
\pgfpathlineto{\pgfqpoint{4.902583in}{2.750621in}}%
\pgfpathlineto{\pgfqpoint{4.895005in}{2.737569in}}%
\pgfpathlineto{\pgfqpoint{4.887427in}{2.724764in}}%
\pgfpathlineto{\pgfqpoint{4.879848in}{2.712198in}}%
\pgfpathclose%
\pgfusepath{fill}%
\end{pgfscope}%
\begin{pgfscope}%
\pgfpathrectangle{\pgfqpoint{1.150000in}{0.150000in}}{\pgfqpoint{5.700000in}{5.700000in}}%
\pgfusepath{clip}%
\pgfsetbuttcap%
\pgfsetroundjoin%
\definecolor{currentfill}{rgb}{0.282910,0.105393,0.426902}%
\pgfsetfillcolor{currentfill}%
\pgfsetfillopacity{0.700000}%
\pgfsetlinewidth{0.000000pt}%
\definecolor{currentstroke}{rgb}{0.000000,0.000000,0.000000}%
\pgfsetstrokecolor{currentstroke}%
\pgfsetdash{}{0pt}%
\pgfpathmoveto{\pgfqpoint{3.501501in}{2.444473in}}%
\pgfpathlineto{\pgfqpoint{3.514909in}{2.438646in}}%
\pgfpathlineto{\pgfqpoint{3.528322in}{2.432913in}}%
\pgfpathlineto{\pgfqpoint{3.541738in}{2.427273in}}%
\pgfpathlineto{\pgfqpoint{3.555158in}{2.421726in}}%
\pgfpathlineto{\pgfqpoint{3.563131in}{2.431792in}}%
\pgfpathlineto{\pgfqpoint{3.571098in}{2.441927in}}%
\pgfpathlineto{\pgfqpoint{3.579059in}{2.452136in}}%
\pgfpathlineto{\pgfqpoint{3.587014in}{2.462420in}}%
\pgfpathlineto{\pgfqpoint{3.573603in}{2.468104in}}%
\pgfpathlineto{\pgfqpoint{3.560197in}{2.473881in}}%
\pgfpathlineto{\pgfqpoint{3.546794in}{2.479751in}}%
\pgfpathlineto{\pgfqpoint{3.533395in}{2.485715in}}%
\pgfpathlineto{\pgfqpoint{3.525431in}{2.475286in}}%
\pgfpathlineto{\pgfqpoint{3.517460in}{2.464938in}}%
\pgfpathlineto{\pgfqpoint{3.509484in}{2.454668in}}%
\pgfpathlineto{\pgfqpoint{3.501501in}{2.444473in}}%
\pgfpathclose%
\pgfusepath{fill}%
\end{pgfscope}%
\begin{pgfscope}%
\pgfpathrectangle{\pgfqpoint{1.150000in}{0.150000in}}{\pgfqpoint{5.700000in}{5.700000in}}%
\pgfusepath{clip}%
\pgfsetbuttcap%
\pgfsetroundjoin%
\definecolor{currentfill}{rgb}{0.282884,0.135920,0.453427}%
\pgfsetfillcolor{currentfill}%
\pgfsetfillopacity{0.700000}%
\pgfsetlinewidth{0.000000pt}%
\definecolor{currentstroke}{rgb}{0.000000,0.000000,0.000000}%
\pgfsetstrokecolor{currentstroke}%
\pgfsetdash{}{0pt}%
\pgfpathmoveto{\pgfqpoint{2.944133in}{2.518387in}}%
\pgfpathlineto{\pgfqpoint{2.957504in}{2.509558in}}%
\pgfpathlineto{\pgfqpoint{2.970875in}{2.500845in}}%
\pgfpathlineto{\pgfqpoint{2.984247in}{2.492248in}}%
\pgfpathlineto{\pgfqpoint{2.997621in}{2.483765in}}%
\pgfpathlineto{\pgfqpoint{3.005781in}{2.493399in}}%
\pgfpathlineto{\pgfqpoint{3.013934in}{2.503112in}}%
\pgfpathlineto{\pgfqpoint{3.022079in}{2.512903in}}%
\pgfpathlineto{\pgfqpoint{3.030218in}{2.522776in}}%
\pgfpathlineto{\pgfqpoint{3.016856in}{2.531314in}}%
\pgfpathlineto{\pgfqpoint{3.003495in}{2.539967in}}%
\pgfpathlineto{\pgfqpoint{2.990135in}{2.548735in}}%
\pgfpathlineto{\pgfqpoint{2.976777in}{2.557620in}}%
\pgfpathlineto{\pgfqpoint{2.968627in}{2.547684in}}%
\pgfpathlineto{\pgfqpoint{2.960469in}{2.537834in}}%
\pgfpathlineto{\pgfqpoint{2.952305in}{2.528069in}}%
\pgfpathlineto{\pgfqpoint{2.944133in}{2.518387in}}%
\pgfpathclose%
\pgfusepath{fill}%
\end{pgfscope}%
\begin{pgfscope}%
\pgfpathrectangle{\pgfqpoint{1.150000in}{0.150000in}}{\pgfqpoint{5.700000in}{5.700000in}}%
\pgfusepath{clip}%
\pgfsetbuttcap%
\pgfsetroundjoin%
\definecolor{currentfill}{rgb}{0.267968,0.223549,0.512008}%
\pgfsetfillcolor{currentfill}%
\pgfsetfillopacity{0.700000}%
\pgfsetlinewidth{0.000000pt}%
\definecolor{currentstroke}{rgb}{0.000000,0.000000,0.000000}%
\pgfsetstrokecolor{currentstroke}%
\pgfsetdash{}{0pt}%
\pgfpathmoveto{\pgfqpoint{4.794757in}{2.676949in}}%
\pgfpathlineto{\pgfqpoint{4.808439in}{2.673652in}}%
\pgfpathlineto{\pgfqpoint{4.822127in}{2.670428in}}%
\pgfpathlineto{\pgfqpoint{4.835824in}{2.667274in}}%
\pgfpathlineto{\pgfqpoint{4.849527in}{2.664192in}}%
\pgfpathlineto{\pgfqpoint{4.857109in}{2.675869in}}%
\pgfpathlineto{\pgfqpoint{4.864690in}{2.687758in}}%
\pgfpathlineto{\pgfqpoint{4.872269in}{2.699865in}}%
\pgfpathlineto{\pgfqpoint{4.879848in}{2.712198in}}%
\pgfpathlineto{\pgfqpoint{4.866159in}{2.715659in}}%
\pgfpathlineto{\pgfqpoint{4.852477in}{2.719190in}}%
\pgfpathlineto{\pgfqpoint{4.838802in}{2.722793in}}%
\pgfpathlineto{\pgfqpoint{4.825134in}{2.726467in}}%
\pgfpathlineto{\pgfqpoint{4.817541in}{2.713749in}}%
\pgfpathlineto{\pgfqpoint{4.809948in}{2.701261in}}%
\pgfpathlineto{\pgfqpoint{4.802353in}{2.688996in}}%
\pgfpathlineto{\pgfqpoint{4.794757in}{2.676949in}}%
\pgfpathclose%
\pgfusepath{fill}%
\end{pgfscope}%
\begin{pgfscope}%
\pgfpathrectangle{\pgfqpoint{1.150000in}{0.150000in}}{\pgfqpoint{5.700000in}{5.700000in}}%
\pgfusepath{clip}%
\pgfsetbuttcap%
\pgfsetroundjoin%
\definecolor{currentfill}{rgb}{0.194100,0.399323,0.555565}%
\pgfsetfillcolor{currentfill}%
\pgfsetfillopacity{0.700000}%
\pgfsetlinewidth{0.000000pt}%
\definecolor{currentstroke}{rgb}{0.000000,0.000000,0.000000}%
\pgfsetstrokecolor{currentstroke}%
\pgfsetdash{}{0pt}%
\pgfpathmoveto{\pgfqpoint{5.561724in}{3.078273in}}%
\pgfpathlineto{\pgfqpoint{5.575532in}{3.073320in}}%
\pgfpathlineto{\pgfqpoint{5.589347in}{3.068433in}}%
\pgfpathlineto{\pgfqpoint{5.603170in}{3.063612in}}%
\pgfpathlineto{\pgfqpoint{5.617000in}{3.058857in}}%
\pgfpathlineto{\pgfqpoint{5.624582in}{3.078184in}}%
\pgfpathlineto{\pgfqpoint{5.632176in}{3.097985in}}%
\pgfpathlineto{\pgfqpoint{5.639782in}{3.118271in}}%
\pgfpathlineto{\pgfqpoint{5.647401in}{3.139051in}}%
\pgfpathlineto{\pgfqpoint{5.633589in}{3.144365in}}%
\pgfpathlineto{\pgfqpoint{5.619784in}{3.149746in}}%
\pgfpathlineto{\pgfqpoint{5.605986in}{3.155193in}}%
\pgfpathlineto{\pgfqpoint{5.592195in}{3.160706in}}%
\pgfpathlineto{\pgfqpoint{5.584559in}{3.139358in}}%
\pgfpathlineto{\pgfqpoint{5.576935in}{3.118510in}}%
\pgfpathlineto{\pgfqpoint{5.569324in}{3.098152in}}%
\pgfpathlineto{\pgfqpoint{5.561724in}{3.078273in}}%
\pgfpathclose%
\pgfusepath{fill}%
\end{pgfscope}%
\begin{pgfscope}%
\pgfpathrectangle{\pgfqpoint{1.150000in}{0.150000in}}{\pgfqpoint{5.700000in}{5.700000in}}%
\pgfusepath{clip}%
\pgfsetbuttcap%
\pgfsetroundjoin%
\definecolor{currentfill}{rgb}{0.280255,0.165693,0.476498}%
\pgfsetfillcolor{currentfill}%
\pgfsetfillopacity{0.700000}%
\pgfsetlinewidth{0.000000pt}%
\definecolor{currentstroke}{rgb}{0.000000,0.000000,0.000000}%
\pgfsetstrokecolor{currentstroke}%
\pgfsetdash{}{0pt}%
\pgfpathmoveto{\pgfqpoint{4.399757in}{2.563418in}}%
\pgfpathlineto{\pgfqpoint{4.413348in}{2.559961in}}%
\pgfpathlineto{\pgfqpoint{4.426945in}{2.556580in}}%
\pgfpathlineto{\pgfqpoint{4.440549in}{2.553275in}}%
\pgfpathlineto{\pgfqpoint{4.454160in}{2.550044in}}%
\pgfpathlineto{\pgfqpoint{4.461849in}{2.560645in}}%
\pgfpathlineto{\pgfqpoint{4.469535in}{2.571385in}}%
\pgfpathlineto{\pgfqpoint{4.477217in}{2.582270in}}%
\pgfpathlineto{\pgfqpoint{4.484895in}{2.593305in}}%
\pgfpathlineto{\pgfqpoint{4.471296in}{2.596833in}}%
\pgfpathlineto{\pgfqpoint{4.457704in}{2.600437in}}%
\pgfpathlineto{\pgfqpoint{4.444118in}{2.604116in}}%
\pgfpathlineto{\pgfqpoint{4.430539in}{2.607871in}}%
\pgfpathlineto{\pgfqpoint{4.422849in}{2.596530in}}%
\pgfpathlineto{\pgfqpoint{4.415156in}{2.585345in}}%
\pgfpathlineto{\pgfqpoint{4.407458in}{2.574310in}}%
\pgfpathlineto{\pgfqpoint{4.399757in}{2.563418in}}%
\pgfpathclose%
\pgfusepath{fill}%
\end{pgfscope}%
\begin{pgfscope}%
\pgfpathrectangle{\pgfqpoint{1.150000in}{0.150000in}}{\pgfqpoint{5.700000in}{5.700000in}}%
\pgfusepath{clip}%
\pgfsetbuttcap%
\pgfsetroundjoin%
\definecolor{currentfill}{rgb}{0.283197,0.115680,0.436115}%
\pgfsetfillcolor{currentfill}%
\pgfsetfillopacity{0.700000}%
\pgfsetlinewidth{0.000000pt}%
\definecolor{currentstroke}{rgb}{0.000000,0.000000,0.000000}%
\pgfsetstrokecolor{currentstroke}%
\pgfsetdash{}{0pt}%
\pgfpathmoveto{\pgfqpoint{3.865337in}{2.463729in}}%
\pgfpathlineto{\pgfqpoint{3.878809in}{2.459217in}}%
\pgfpathlineto{\pgfqpoint{3.892287in}{2.454789in}}%
\pgfpathlineto{\pgfqpoint{3.905770in}{2.450446in}}%
\pgfpathlineto{\pgfqpoint{3.919259in}{2.446187in}}%
\pgfpathlineto{\pgfqpoint{3.927116in}{2.456345in}}%
\pgfpathlineto{\pgfqpoint{3.934968in}{2.466586in}}%
\pgfpathlineto{\pgfqpoint{3.942815in}{2.476915in}}%
\pgfpathlineto{\pgfqpoint{3.950656in}{2.487334in}}%
\pgfpathlineto{\pgfqpoint{3.937177in}{2.491791in}}%
\pgfpathlineto{\pgfqpoint{3.923703in}{2.496332in}}%
\pgfpathlineto{\pgfqpoint{3.910235in}{2.500957in}}%
\pgfpathlineto{\pgfqpoint{3.896772in}{2.505667in}}%
\pgfpathlineto{\pgfqpoint{3.888921in}{2.495042in}}%
\pgfpathlineto{\pgfqpoint{3.881065in}{2.484514in}}%
\pgfpathlineto{\pgfqpoint{3.873204in}{2.474077in}}%
\pgfpathlineto{\pgfqpoint{3.865337in}{2.463729in}}%
\pgfpathclose%
\pgfusepath{fill}%
\end{pgfscope}%
\begin{pgfscope}%
\pgfpathrectangle{\pgfqpoint{1.150000in}{0.150000in}}{\pgfqpoint{5.700000in}{5.700000in}}%
\pgfusepath{clip}%
\pgfsetbuttcap%
\pgfsetroundjoin%
\definecolor{currentfill}{rgb}{0.282910,0.105393,0.426902}%
\pgfsetfillcolor{currentfill}%
\pgfsetfillopacity{0.700000}%
\pgfsetlinewidth{0.000000pt}%
\definecolor{currentstroke}{rgb}{0.000000,0.000000,0.000000}%
\pgfsetstrokecolor{currentstroke}%
\pgfsetdash{}{0pt}%
\pgfpathmoveto{\pgfqpoint{3.640699in}{2.440603in}}%
\pgfpathlineto{\pgfqpoint{3.654132in}{2.435376in}}%
\pgfpathlineto{\pgfqpoint{3.667569in}{2.430239in}}%
\pgfpathlineto{\pgfqpoint{3.681010in}{2.425191in}}%
\pgfpathlineto{\pgfqpoint{3.694457in}{2.420232in}}%
\pgfpathlineto{\pgfqpoint{3.702387in}{2.430296in}}%
\pgfpathlineto{\pgfqpoint{3.710312in}{2.440432in}}%
\pgfpathlineto{\pgfqpoint{3.718230in}{2.450642in}}%
\pgfpathlineto{\pgfqpoint{3.726144in}{2.460930in}}%
\pgfpathlineto{\pgfqpoint{3.712707in}{2.466046in}}%
\pgfpathlineto{\pgfqpoint{3.699275in}{2.471251in}}%
\pgfpathlineto{\pgfqpoint{3.685847in}{2.476546in}}%
\pgfpathlineto{\pgfqpoint{3.672425in}{2.481930in}}%
\pgfpathlineto{\pgfqpoint{3.664502in}{2.471477in}}%
\pgfpathlineto{\pgfqpoint{3.656573in}{2.461107in}}%
\pgfpathlineto{\pgfqpoint{3.648639in}{2.450817in}}%
\pgfpathlineto{\pgfqpoint{3.640699in}{2.440603in}}%
\pgfpathclose%
\pgfusepath{fill}%
\end{pgfscope}%
\begin{pgfscope}%
\pgfpathrectangle{\pgfqpoint{1.150000in}{0.150000in}}{\pgfqpoint{5.700000in}{5.700000in}}%
\pgfusepath{clip}%
\pgfsetbuttcap%
\pgfsetroundjoin%
\definecolor{currentfill}{rgb}{0.282884,0.135920,0.453427}%
\pgfsetfillcolor{currentfill}%
\pgfsetfillopacity{0.700000}%
\pgfsetlinewidth{0.000000pt}%
\definecolor{currentstroke}{rgb}{0.000000,0.000000,0.000000}%
\pgfsetstrokecolor{currentstroke}%
\pgfsetdash{}{0pt}%
\pgfpathmoveto{\pgfqpoint{4.089927in}{2.495585in}}%
\pgfpathlineto{\pgfqpoint{4.103448in}{2.491650in}}%
\pgfpathlineto{\pgfqpoint{4.116974in}{2.487796in}}%
\pgfpathlineto{\pgfqpoint{4.130507in}{2.484021in}}%
\pgfpathlineto{\pgfqpoint{4.144046in}{2.480327in}}%
\pgfpathlineto{\pgfqpoint{4.151832in}{2.490571in}}%
\pgfpathlineto{\pgfqpoint{4.159613in}{2.500916in}}%
\pgfpathlineto{\pgfqpoint{4.167390in}{2.511367in}}%
\pgfpathlineto{\pgfqpoint{4.175162in}{2.521926in}}%
\pgfpathlineto{\pgfqpoint{4.161633in}{2.525859in}}%
\pgfpathlineto{\pgfqpoint{4.148111in}{2.529871in}}%
\pgfpathlineto{\pgfqpoint{4.134594in}{2.533963in}}%
\pgfpathlineto{\pgfqpoint{4.121083in}{2.538136in}}%
\pgfpathlineto{\pgfqpoint{4.113301in}{2.527331in}}%
\pgfpathlineto{\pgfqpoint{4.105515in}{2.516640in}}%
\pgfpathlineto{\pgfqpoint{4.097723in}{2.506060in}}%
\pgfpathlineto{\pgfqpoint{4.089927in}{2.495585in}}%
\pgfpathclose%
\pgfusepath{fill}%
\end{pgfscope}%
\begin{pgfscope}%
\pgfpathrectangle{\pgfqpoint{1.150000in}{0.150000in}}{\pgfqpoint{5.700000in}{5.700000in}}%
\pgfusepath{clip}%
\pgfsetbuttcap%
\pgfsetroundjoin%
\definecolor{currentfill}{rgb}{0.271828,0.209303,0.504434}%
\pgfsetfillcolor{currentfill}%
\pgfsetfillopacity{0.700000}%
\pgfsetlinewidth{0.000000pt}%
\definecolor{currentstroke}{rgb}{0.000000,0.000000,0.000000}%
\pgfsetstrokecolor{currentstroke}%
\pgfsetdash{}{0pt}%
\pgfpathmoveto{\pgfqpoint{4.709652in}{2.643262in}}%
\pgfpathlineto{\pgfqpoint{4.723317in}{2.640036in}}%
\pgfpathlineto{\pgfqpoint{4.736990in}{2.636883in}}%
\pgfpathlineto{\pgfqpoint{4.750671in}{2.633802in}}%
\pgfpathlineto{\pgfqpoint{4.764358in}{2.630792in}}%
\pgfpathlineto{\pgfqpoint{4.771961in}{2.642039in}}%
\pgfpathlineto{\pgfqpoint{4.779561in}{2.653477in}}%
\pgfpathlineto{\pgfqpoint{4.787160in}{2.665111in}}%
\pgfpathlineto{\pgfqpoint{4.794757in}{2.676949in}}%
\pgfpathlineto{\pgfqpoint{4.781083in}{2.680316in}}%
\pgfpathlineto{\pgfqpoint{4.767417in}{2.683756in}}%
\pgfpathlineto{\pgfqpoint{4.753757in}{2.687268in}}%
\pgfpathlineto{\pgfqpoint{4.740105in}{2.690852in}}%
\pgfpathlineto{\pgfqpoint{4.732494in}{2.678649in}}%
\pgfpathlineto{\pgfqpoint{4.724882in}{2.666653in}}%
\pgfpathlineto{\pgfqpoint{4.717268in}{2.654860in}}%
\pgfpathlineto{\pgfqpoint{4.709652in}{2.643262in}}%
\pgfpathclose%
\pgfusepath{fill}%
\end{pgfscope}%
\begin{pgfscope}%
\pgfpathrectangle{\pgfqpoint{1.150000in}{0.150000in}}{\pgfqpoint{5.700000in}{5.700000in}}%
\pgfusepath{clip}%
\pgfsetbuttcap%
\pgfsetroundjoin%
\definecolor{currentfill}{rgb}{0.281412,0.155834,0.469201}%
\pgfsetfillcolor{currentfill}%
\pgfsetfillopacity{0.700000}%
\pgfsetlinewidth{0.000000pt}%
\definecolor{currentstroke}{rgb}{0.000000,0.000000,0.000000}%
\pgfsetstrokecolor{currentstroke}%
\pgfsetdash{}{0pt}%
\pgfpathmoveto{\pgfqpoint{2.804328in}{2.555129in}}%
\pgfpathlineto{\pgfqpoint{2.817708in}{2.545367in}}%
\pgfpathlineto{\pgfqpoint{2.831088in}{2.535731in}}%
\pgfpathlineto{\pgfqpoint{2.844468in}{2.526217in}}%
\pgfpathlineto{\pgfqpoint{2.857848in}{2.516826in}}%
\pgfpathlineto{\pgfqpoint{2.866062in}{2.526215in}}%
\pgfpathlineto{\pgfqpoint{2.874268in}{2.535688in}}%
\pgfpathlineto{\pgfqpoint{2.882467in}{2.545246in}}%
\pgfpathlineto{\pgfqpoint{2.890658in}{2.554889in}}%
\pgfpathlineto{\pgfqpoint{2.877291in}{2.564315in}}%
\pgfpathlineto{\pgfqpoint{2.863924in}{2.573863in}}%
\pgfpathlineto{\pgfqpoint{2.850557in}{2.583535in}}%
\pgfpathlineto{\pgfqpoint{2.837190in}{2.593331in}}%
\pgfpathlineto{\pgfqpoint{2.828986in}{2.583646in}}%
\pgfpathlineto{\pgfqpoint{2.820774in}{2.574051in}}%
\pgfpathlineto{\pgfqpoint{2.812555in}{2.564545in}}%
\pgfpathlineto{\pgfqpoint{2.804328in}{2.555129in}}%
\pgfpathclose%
\pgfusepath{fill}%
\end{pgfscope}%
\begin{pgfscope}%
\pgfpathrectangle{\pgfqpoint{1.150000in}{0.150000in}}{\pgfqpoint{5.700000in}{5.700000in}}%
\pgfusepath{clip}%
\pgfsetbuttcap%
\pgfsetroundjoin%
\definecolor{currentfill}{rgb}{0.283091,0.110553,0.431554}%
\pgfsetfillcolor{currentfill}%
\pgfsetfillopacity{0.700000}%
\pgfsetlinewidth{0.000000pt}%
\definecolor{currentstroke}{rgb}{0.000000,0.000000,0.000000}%
\pgfsetstrokecolor{currentstroke}%
\pgfsetdash{}{0pt}%
\pgfpathmoveto{\pgfqpoint{3.137169in}{2.458501in}}%
\pgfpathlineto{\pgfqpoint{3.150547in}{2.450959in}}%
\pgfpathlineto{\pgfqpoint{3.163926in}{2.443522in}}%
\pgfpathlineto{\pgfqpoint{3.177308in}{2.436192in}}%
\pgfpathlineto{\pgfqpoint{3.190693in}{2.428967in}}%
\pgfpathlineto{\pgfqpoint{3.198791in}{2.438717in}}%
\pgfpathlineto{\pgfqpoint{3.206883in}{2.448536in}}%
\pgfpathlineto{\pgfqpoint{3.214968in}{2.458426in}}%
\pgfpathlineto{\pgfqpoint{3.223047in}{2.468388in}}%
\pgfpathlineto{\pgfqpoint{3.209673in}{2.475690in}}%
\pgfpathlineto{\pgfqpoint{3.196302in}{2.483096in}}%
\pgfpathlineto{\pgfqpoint{3.182933in}{2.490608in}}%
\pgfpathlineto{\pgfqpoint{3.169567in}{2.498227in}}%
\pgfpathlineto{\pgfqpoint{3.161477in}{2.488181in}}%
\pgfpathlineto{\pgfqpoint{3.153381in}{2.478213in}}%
\pgfpathlineto{\pgfqpoint{3.145279in}{2.468320in}}%
\pgfpathlineto{\pgfqpoint{3.137169in}{2.458501in}}%
\pgfpathclose%
\pgfusepath{fill}%
\end{pgfscope}%
\begin{pgfscope}%
\pgfpathrectangle{\pgfqpoint{1.150000in}{0.150000in}}{\pgfqpoint{5.700000in}{5.700000in}}%
\pgfusepath{clip}%
\pgfsetbuttcap%
\pgfsetroundjoin%
\definecolor{currentfill}{rgb}{0.282910,0.105393,0.426902}%
\pgfsetfillcolor{currentfill}%
\pgfsetfillopacity{0.700000}%
\pgfsetlinewidth{0.000000pt}%
\definecolor{currentstroke}{rgb}{0.000000,0.000000,0.000000}%
\pgfsetstrokecolor{currentstroke}%
\pgfsetdash{}{0pt}%
\pgfpathmoveto{\pgfqpoint{3.276567in}{2.440221in}}%
\pgfpathlineto{\pgfqpoint{3.289954in}{2.433435in}}%
\pgfpathlineto{\pgfqpoint{3.303344in}{2.426750in}}%
\pgfpathlineto{\pgfqpoint{3.316737in}{2.420166in}}%
\pgfpathlineto{\pgfqpoint{3.330133in}{2.413682in}}%
\pgfpathlineto{\pgfqpoint{3.338184in}{2.423541in}}%
\pgfpathlineto{\pgfqpoint{3.346229in}{2.433466in}}%
\pgfpathlineto{\pgfqpoint{3.354268in}{2.443461in}}%
\pgfpathlineto{\pgfqpoint{3.362301in}{2.453526in}}%
\pgfpathlineto{\pgfqpoint{3.348915in}{2.460107in}}%
\pgfpathlineto{\pgfqpoint{3.335532in}{2.466788in}}%
\pgfpathlineto{\pgfqpoint{3.322152in}{2.473569in}}%
\pgfpathlineto{\pgfqpoint{3.308776in}{2.480451in}}%
\pgfpathlineto{\pgfqpoint{3.300733in}{2.470282in}}%
\pgfpathlineto{\pgfqpoint{3.292684in}{2.460189in}}%
\pgfpathlineto{\pgfqpoint{3.284629in}{2.450169in}}%
\pgfpathlineto{\pgfqpoint{3.276567in}{2.440221in}}%
\pgfpathclose%
\pgfusepath{fill}%
\end{pgfscope}%
\begin{pgfscope}%
\pgfpathrectangle{\pgfqpoint{1.150000in}{0.150000in}}{\pgfqpoint{5.700000in}{5.700000in}}%
\pgfusepath{clip}%
\pgfsetbuttcap%
\pgfsetroundjoin%
\definecolor{currentfill}{rgb}{0.235526,0.309527,0.542944}%
\pgfsetfillcolor{currentfill}%
\pgfsetfillopacity{0.700000}%
\pgfsetlinewidth{0.000000pt}%
\definecolor{currentstroke}{rgb}{0.000000,0.000000,0.000000}%
\pgfsetstrokecolor{currentstroke}%
\pgfsetdash{}{0pt}%
\pgfpathmoveto{\pgfqpoint{5.275373in}{2.857411in}}%
\pgfpathlineto{\pgfqpoint{5.289160in}{2.853691in}}%
\pgfpathlineto{\pgfqpoint{5.302954in}{2.850039in}}%
\pgfpathlineto{\pgfqpoint{5.316755in}{2.846454in}}%
\pgfpathlineto{\pgfqpoint{5.330565in}{2.842937in}}%
\pgfpathlineto{\pgfqpoint{5.338077in}{2.857431in}}%
\pgfpathlineto{\pgfqpoint{5.345594in}{2.872261in}}%
\pgfpathlineto{\pgfqpoint{5.353115in}{2.887436in}}%
\pgfpathlineto{\pgfqpoint{5.360642in}{2.902966in}}%
\pgfpathlineto{\pgfqpoint{5.346850in}{2.906962in}}%
\pgfpathlineto{\pgfqpoint{5.333065in}{2.911025in}}%
\pgfpathlineto{\pgfqpoint{5.319288in}{2.915156in}}%
\pgfpathlineto{\pgfqpoint{5.305519in}{2.919354in}}%
\pgfpathlineto{\pgfqpoint{5.297975in}{2.903339in}}%
\pgfpathlineto{\pgfqpoint{5.290436in}{2.887683in}}%
\pgfpathlineto{\pgfqpoint{5.282903in}{2.872376in}}%
\pgfpathlineto{\pgfqpoint{5.275373in}{2.857411in}}%
\pgfpathclose%
\pgfusepath{fill}%
\end{pgfscope}%
\begin{pgfscope}%
\pgfpathrectangle{\pgfqpoint{1.150000in}{0.150000in}}{\pgfqpoint{5.700000in}{5.700000in}}%
\pgfusepath{clip}%
\pgfsetbuttcap%
\pgfsetroundjoin%
\definecolor{currentfill}{rgb}{0.274128,0.199721,0.498911}%
\pgfsetfillcolor{currentfill}%
\pgfsetfillopacity{0.700000}%
\pgfsetlinewidth{0.000000pt}%
\definecolor{currentstroke}{rgb}{0.000000,0.000000,0.000000}%
\pgfsetstrokecolor{currentstroke}%
\pgfsetdash{}{0pt}%
\pgfpathmoveto{\pgfqpoint{2.610568in}{2.645391in}}%
\pgfpathlineto{\pgfqpoint{2.623972in}{2.634065in}}%
\pgfpathlineto{\pgfqpoint{2.637375in}{2.622876in}}%
\pgfpathlineto{\pgfqpoint{2.650776in}{2.611825in}}%
\pgfpathlineto{\pgfqpoint{2.664176in}{2.600909in}}%
\pgfpathlineto{\pgfqpoint{2.672463in}{2.609995in}}%
\pgfpathlineto{\pgfqpoint{2.680741in}{2.619175in}}%
\pgfpathlineto{\pgfqpoint{2.689012in}{2.628451in}}%
\pgfpathlineto{\pgfqpoint{2.697274in}{2.637825in}}%
\pgfpathlineto{\pgfqpoint{2.683888in}{2.648754in}}%
\pgfpathlineto{\pgfqpoint{2.670501in}{2.659820in}}%
\pgfpathlineto{\pgfqpoint{2.657113in}{2.671022in}}%
\pgfpathlineto{\pgfqpoint{2.643723in}{2.682362in}}%
\pgfpathlineto{\pgfqpoint{2.635447in}{2.672967in}}%
\pgfpathlineto{\pgfqpoint{2.627162in}{2.663674in}}%
\pgfpathlineto{\pgfqpoint{2.618869in}{2.654483in}}%
\pgfpathlineto{\pgfqpoint{2.610568in}{2.645391in}}%
\pgfpathclose%
\pgfusepath{fill}%
\end{pgfscope}%
\begin{pgfscope}%
\pgfpathrectangle{\pgfqpoint{1.150000in}{0.150000in}}{\pgfqpoint{5.700000in}{5.700000in}}%
\pgfusepath{clip}%
\pgfsetbuttcap%
\pgfsetroundjoin%
\definecolor{currentfill}{rgb}{0.281412,0.155834,0.469201}%
\pgfsetfillcolor{currentfill}%
\pgfsetfillopacity{0.700000}%
\pgfsetlinewidth{0.000000pt}%
\definecolor{currentstroke}{rgb}{0.000000,0.000000,0.000000}%
\pgfsetstrokecolor{currentstroke}%
\pgfsetdash{}{0pt}%
\pgfpathmoveto{\pgfqpoint{4.314574in}{2.534670in}}%
\pgfpathlineto{\pgfqpoint{4.328149in}{2.531185in}}%
\pgfpathlineto{\pgfqpoint{4.341731in}{2.527777in}}%
\pgfpathlineto{\pgfqpoint{4.355319in}{2.524445in}}%
\pgfpathlineto{\pgfqpoint{4.368914in}{2.521190in}}%
\pgfpathlineto{\pgfqpoint{4.376631in}{2.531557in}}%
\pgfpathlineto{\pgfqpoint{4.384344in}{2.542047in}}%
\pgfpathlineto{\pgfqpoint{4.392053in}{2.552666in}}%
\pgfpathlineto{\pgfqpoint{4.399757in}{2.563418in}}%
\pgfpathlineto{\pgfqpoint{4.386174in}{2.566952in}}%
\pgfpathlineto{\pgfqpoint{4.372597in}{2.570561in}}%
\pgfpathlineto{\pgfqpoint{4.359026in}{2.574248in}}%
\pgfpathlineto{\pgfqpoint{4.345462in}{2.578011in}}%
\pgfpathlineto{\pgfqpoint{4.337746in}{2.566973in}}%
\pgfpathlineto{\pgfqpoint{4.330026in}{2.556074in}}%
\pgfpathlineto{\pgfqpoint{4.322302in}{2.545308in}}%
\pgfpathlineto{\pgfqpoint{4.314574in}{2.534670in}}%
\pgfpathclose%
\pgfusepath{fill}%
\end{pgfscope}%
\begin{pgfscope}%
\pgfpathrectangle{\pgfqpoint{1.150000in}{0.150000in}}{\pgfqpoint{5.700000in}{5.700000in}}%
\pgfusepath{clip}%
\pgfsetbuttcap%
\pgfsetroundjoin%
\definecolor{currentfill}{rgb}{0.225863,0.330805,0.547314}%
\pgfsetfillcolor{currentfill}%
\pgfsetfillopacity{0.700000}%
\pgfsetlinewidth{0.000000pt}%
\definecolor{currentstroke}{rgb}{0.000000,0.000000,0.000000}%
\pgfsetstrokecolor{currentstroke}%
\pgfsetdash{}{0pt}%
\pgfpathmoveto{\pgfqpoint{5.360642in}{2.902966in}}%
\pgfpathlineto{\pgfqpoint{5.374442in}{2.899037in}}%
\pgfpathlineto{\pgfqpoint{5.388250in}{2.895175in}}%
\pgfpathlineto{\pgfqpoint{5.402065in}{2.891380in}}%
\pgfpathlineto{\pgfqpoint{5.415888in}{2.887653in}}%
\pgfpathlineto{\pgfqpoint{5.423403in}{2.903054in}}%
\pgfpathlineto{\pgfqpoint{5.430924in}{2.918821in}}%
\pgfpathlineto{\pgfqpoint{5.438452in}{2.934966in}}%
\pgfpathlineto{\pgfqpoint{5.445987in}{2.951495in}}%
\pgfpathlineto{\pgfqpoint{5.432181in}{2.955722in}}%
\pgfpathlineto{\pgfqpoint{5.418384in}{2.960015in}}%
\pgfpathlineto{\pgfqpoint{5.404593in}{2.964376in}}%
\pgfpathlineto{\pgfqpoint{5.390811in}{2.968803in}}%
\pgfpathlineto{\pgfqpoint{5.383259in}{2.951768in}}%
\pgfpathlineto{\pgfqpoint{5.375713in}{2.935122in}}%
\pgfpathlineto{\pgfqpoint{5.368175in}{2.918858in}}%
\pgfpathlineto{\pgfqpoint{5.360642in}{2.902966in}}%
\pgfpathclose%
\pgfusepath{fill}%
\end{pgfscope}%
\begin{pgfscope}%
\pgfpathrectangle{\pgfqpoint{1.150000in}{0.150000in}}{\pgfqpoint{5.700000in}{5.700000in}}%
\pgfusepath{clip}%
\pgfsetbuttcap%
\pgfsetroundjoin%
\definecolor{currentfill}{rgb}{0.243113,0.292092,0.538516}%
\pgfsetfillcolor{currentfill}%
\pgfsetfillopacity{0.700000}%
\pgfsetlinewidth{0.000000pt}%
\definecolor{currentstroke}{rgb}{0.000000,0.000000,0.000000}%
\pgfsetstrokecolor{currentstroke}%
\pgfsetdash{}{0pt}%
\pgfpathmoveto{\pgfqpoint{5.190160in}{2.814518in}}%
\pgfpathlineto{\pgfqpoint{5.203932in}{2.810985in}}%
\pgfpathlineto{\pgfqpoint{5.217712in}{2.807520in}}%
\pgfpathlineto{\pgfqpoint{5.231500in}{2.804124in}}%
\pgfpathlineto{\pgfqpoint{5.245295in}{2.800795in}}%
\pgfpathlineto{\pgfqpoint{5.252809in}{2.814479in}}%
\pgfpathlineto{\pgfqpoint{5.260327in}{2.828471in}}%
\pgfpathlineto{\pgfqpoint{5.267848in}{2.842779in}}%
\pgfpathlineto{\pgfqpoint{5.275373in}{2.857411in}}%
\pgfpathlineto{\pgfqpoint{5.261595in}{2.861199in}}%
\pgfpathlineto{\pgfqpoint{5.247824in}{2.865054in}}%
\pgfpathlineto{\pgfqpoint{5.234061in}{2.868978in}}%
\pgfpathlineto{\pgfqpoint{5.220305in}{2.872969in}}%
\pgfpathlineto{\pgfqpoint{5.212764in}{2.857871in}}%
\pgfpathlineto{\pgfqpoint{5.205226in}{2.843102in}}%
\pgfpathlineto{\pgfqpoint{5.197691in}{2.828654in}}%
\pgfpathlineto{\pgfqpoint{5.190160in}{2.814518in}}%
\pgfpathclose%
\pgfusepath{fill}%
\end{pgfscope}%
\begin{pgfscope}%
\pgfpathrectangle{\pgfqpoint{1.150000in}{0.150000in}}{\pgfqpoint{5.700000in}{5.700000in}}%
\pgfusepath{clip}%
\pgfsetbuttcap%
\pgfsetroundjoin%
\definecolor{currentfill}{rgb}{0.183898,0.422383,0.556944}%
\pgfsetfillcolor{currentfill}%
\pgfsetfillopacity{0.700000}%
\pgfsetlinewidth{0.000000pt}%
\definecolor{currentstroke}{rgb}{0.000000,0.000000,0.000000}%
\pgfsetstrokecolor{currentstroke}%
\pgfsetdash{}{0pt}%
\pgfpathmoveto{\pgfqpoint{5.647401in}{3.139051in}}%
\pgfpathlineto{\pgfqpoint{5.661221in}{3.133801in}}%
\pgfpathlineto{\pgfqpoint{5.675048in}{3.128618in}}%
\pgfpathlineto{\pgfqpoint{5.688883in}{3.123500in}}%
\pgfpathlineto{\pgfqpoint{5.702725in}{3.118447in}}%
\pgfpathlineto{\pgfqpoint{5.710339in}{3.139160in}}%
\pgfpathlineto{\pgfqpoint{5.717967in}{3.160384in}}%
\pgfpathlineto{\pgfqpoint{5.725610in}{3.182130in}}%
\pgfpathlineto{\pgfqpoint{5.711781in}{3.187616in}}%
\pgfpathlineto{\pgfqpoint{5.697960in}{3.193168in}}%
\pgfpathlineto{\pgfqpoint{5.684146in}{3.198785in}}%
\pgfpathlineto{\pgfqpoint{5.670340in}{3.204468in}}%
\pgfpathlineto{\pgfqpoint{5.662679in}{3.182138in}}%
\pgfpathlineto{\pgfqpoint{5.655033in}{3.160336in}}%
\pgfpathlineto{\pgfqpoint{5.647401in}{3.139051in}}%
\pgfpathclose%
\pgfusepath{fill}%
\end{pgfscope}%
\begin{pgfscope}%
\pgfpathrectangle{\pgfqpoint{1.150000in}{0.150000in}}{\pgfqpoint{5.700000in}{5.700000in}}%
\pgfusepath{clip}%
\pgfsetbuttcap%
\pgfsetroundjoin%
\definecolor{currentfill}{rgb}{0.282656,0.100196,0.422160}%
\pgfsetfillcolor{currentfill}%
\pgfsetfillopacity{0.700000}%
\pgfsetlinewidth{0.000000pt}%
\definecolor{currentstroke}{rgb}{0.000000,0.000000,0.000000}%
\pgfsetstrokecolor{currentstroke}%
\pgfsetdash{}{0pt}%
\pgfpathmoveto{\pgfqpoint{3.415878in}{2.428186in}}%
\pgfpathlineto{\pgfqpoint{3.429281in}{2.422095in}}%
\pgfpathlineto{\pgfqpoint{3.442687in}{2.416099in}}%
\pgfpathlineto{\pgfqpoint{3.456097in}{2.410199in}}%
\pgfpathlineto{\pgfqpoint{3.469512in}{2.404395in}}%
\pgfpathlineto{\pgfqpoint{3.477518in}{2.414314in}}%
\pgfpathlineto{\pgfqpoint{3.485518in}{2.424299in}}%
\pgfpathlineto{\pgfqpoint{3.493513in}{2.434351in}}%
\pgfpathlineto{\pgfqpoint{3.501501in}{2.444473in}}%
\pgfpathlineto{\pgfqpoint{3.488097in}{2.450395in}}%
\pgfpathlineto{\pgfqpoint{3.474697in}{2.456411in}}%
\pgfpathlineto{\pgfqpoint{3.461300in}{2.462523in}}%
\pgfpathlineto{\pgfqpoint{3.447907in}{2.468732in}}%
\pgfpathlineto{\pgfqpoint{3.439909in}{2.458485in}}%
\pgfpathlineto{\pgfqpoint{3.431904in}{2.448314in}}%
\pgfpathlineto{\pgfqpoint{3.423894in}{2.438215in}}%
\pgfpathlineto{\pgfqpoint{3.415878in}{2.428186in}}%
\pgfpathclose%
\pgfusepath{fill}%
\end{pgfscope}%
\begin{pgfscope}%
\pgfpathrectangle{\pgfqpoint{1.150000in}{0.150000in}}{\pgfqpoint{5.700000in}{5.700000in}}%
\pgfusepath{clip}%
\pgfsetbuttcap%
\pgfsetroundjoin%
\definecolor{currentfill}{rgb}{0.283187,0.125848,0.444960}%
\pgfsetfillcolor{currentfill}%
\pgfsetfillopacity{0.700000}%
\pgfsetlinewidth{0.000000pt}%
\definecolor{currentstroke}{rgb}{0.000000,0.000000,0.000000}%
\pgfsetstrokecolor{currentstroke}%
\pgfsetdash{}{0pt}%
\pgfpathmoveto{\pgfqpoint{2.997621in}{2.483765in}}%
\pgfpathlineto{\pgfqpoint{3.010996in}{2.475397in}}%
\pgfpathlineto{\pgfqpoint{3.024372in}{2.467141in}}%
\pgfpathlineto{\pgfqpoint{3.037750in}{2.458998in}}%
\pgfpathlineto{\pgfqpoint{3.051129in}{2.450967in}}%
\pgfpathlineto{\pgfqpoint{3.059277in}{2.460552in}}%
\pgfpathlineto{\pgfqpoint{3.067419in}{2.470211in}}%
\pgfpathlineto{\pgfqpoint{3.075553in}{2.479945in}}%
\pgfpathlineto{\pgfqpoint{3.083680in}{2.489754in}}%
\pgfpathlineto{\pgfqpoint{3.070312in}{2.497842in}}%
\pgfpathlineto{\pgfqpoint{3.056946in}{2.506041in}}%
\pgfpathlineto{\pgfqpoint{3.043581in}{2.514352in}}%
\pgfpathlineto{\pgfqpoint{3.030218in}{2.522776in}}%
\pgfpathlineto{\pgfqpoint{3.022079in}{2.512903in}}%
\pgfpathlineto{\pgfqpoint{3.013934in}{2.503112in}}%
\pgfpathlineto{\pgfqpoint{3.005781in}{2.493399in}}%
\pgfpathlineto{\pgfqpoint{2.997621in}{2.483765in}}%
\pgfpathclose%
\pgfusepath{fill}%
\end{pgfscope}%
\begin{pgfscope}%
\pgfpathrectangle{\pgfqpoint{1.150000in}{0.150000in}}{\pgfqpoint{5.700000in}{5.700000in}}%
\pgfusepath{clip}%
\pgfsetbuttcap%
\pgfsetroundjoin%
\definecolor{currentfill}{rgb}{0.275191,0.194905,0.496005}%
\pgfsetfillcolor{currentfill}%
\pgfsetfillopacity{0.700000}%
\pgfsetlinewidth{0.000000pt}%
\definecolor{currentstroke}{rgb}{0.000000,0.000000,0.000000}%
\pgfsetstrokecolor{currentstroke}%
\pgfsetdash{}{0pt}%
\pgfpathmoveto{\pgfqpoint{4.624522in}{2.610973in}}%
\pgfpathlineto{\pgfqpoint{4.638172in}{2.607795in}}%
\pgfpathlineto{\pgfqpoint{4.651829in}{2.604689in}}%
\pgfpathlineto{\pgfqpoint{4.665493in}{2.601657in}}%
\pgfpathlineto{\pgfqpoint{4.679165in}{2.598697in}}%
\pgfpathlineto{\pgfqpoint{4.686790in}{2.609577in}}%
\pgfpathlineto{\pgfqpoint{4.694413in}{2.620627in}}%
\pgfpathlineto{\pgfqpoint{4.702034in}{2.631853in}}%
\pgfpathlineto{\pgfqpoint{4.709652in}{2.643262in}}%
\pgfpathlineto{\pgfqpoint{4.695993in}{2.646560in}}%
\pgfpathlineto{\pgfqpoint{4.682342in}{2.649931in}}%
\pgfpathlineto{\pgfqpoint{4.668698in}{2.653375in}}%
\pgfpathlineto{\pgfqpoint{4.655061in}{2.656892in}}%
\pgfpathlineto{\pgfqpoint{4.647430in}{2.645137in}}%
\pgfpathlineto{\pgfqpoint{4.639797in}{2.633570in}}%
\pgfpathlineto{\pgfqpoint{4.632161in}{2.622184in}}%
\pgfpathlineto{\pgfqpoint{4.624522in}{2.610973in}}%
\pgfpathclose%
\pgfusepath{fill}%
\end{pgfscope}%
\begin{pgfscope}%
\pgfpathrectangle{\pgfqpoint{1.150000in}{0.150000in}}{\pgfqpoint{5.700000in}{5.700000in}}%
\pgfusepath{clip}%
\pgfsetbuttcap%
\pgfsetroundjoin%
\definecolor{currentfill}{rgb}{0.250425,0.274290,0.533103}%
\pgfsetfillcolor{currentfill}%
\pgfsetfillopacity{0.700000}%
\pgfsetlinewidth{0.000000pt}%
\definecolor{currentstroke}{rgb}{0.000000,0.000000,0.000000}%
\pgfsetstrokecolor{currentstroke}%
\pgfsetdash{}{0pt}%
\pgfpathmoveto{\pgfqpoint{5.104984in}{2.774000in}}%
\pgfpathlineto{\pgfqpoint{5.118741in}{2.770631in}}%
\pgfpathlineto{\pgfqpoint{5.132506in}{2.767331in}}%
\pgfpathlineto{\pgfqpoint{5.146279in}{2.764100in}}%
\pgfpathlineto{\pgfqpoint{5.160060in}{2.760937in}}%
\pgfpathlineto{\pgfqpoint{5.167582in}{2.773904in}}%
\pgfpathlineto{\pgfqpoint{5.175106in}{2.787152in}}%
\pgfpathlineto{\pgfqpoint{5.182632in}{2.800687in}}%
\pgfpathlineto{\pgfqpoint{5.190160in}{2.814518in}}%
\pgfpathlineto{\pgfqpoint{5.176396in}{2.818120in}}%
\pgfpathlineto{\pgfqpoint{5.162639in}{2.821790in}}%
\pgfpathlineto{\pgfqpoint{5.148890in}{2.825528in}}%
\pgfpathlineto{\pgfqpoint{5.135148in}{2.829335in}}%
\pgfpathlineto{\pgfqpoint{5.127604in}{2.815058in}}%
\pgfpathlineto{\pgfqpoint{5.120062in}{2.801082in}}%
\pgfpathlineto{\pgfqpoint{5.112522in}{2.787399in}}%
\pgfpathlineto{\pgfqpoint{5.104984in}{2.774000in}}%
\pgfpathclose%
\pgfusepath{fill}%
\end{pgfscope}%
\begin{pgfscope}%
\pgfpathrectangle{\pgfqpoint{1.150000in}{0.150000in}}{\pgfqpoint{5.700000in}{5.700000in}}%
\pgfusepath{clip}%
\pgfsetbuttcap%
\pgfsetroundjoin%
\definecolor{currentfill}{rgb}{0.216210,0.351535,0.550627}%
\pgfsetfillcolor{currentfill}%
\pgfsetfillopacity{0.700000}%
\pgfsetlinewidth{0.000000pt}%
\definecolor{currentstroke}{rgb}{0.000000,0.000000,0.000000}%
\pgfsetstrokecolor{currentstroke}%
\pgfsetdash{}{0pt}%
\pgfpathmoveto{\pgfqpoint{5.445987in}{2.951495in}}%
\pgfpathlineto{\pgfqpoint{5.459800in}{2.947335in}}%
\pgfpathlineto{\pgfqpoint{5.473621in}{2.943241in}}%
\pgfpathlineto{\pgfqpoint{5.487449in}{2.939214in}}%
\pgfpathlineto{\pgfqpoint{5.501286in}{2.935254in}}%
\pgfpathlineto{\pgfqpoint{5.508810in}{2.951666in}}%
\pgfpathlineto{\pgfqpoint{5.516342in}{2.968478in}}%
\pgfpathlineto{\pgfqpoint{5.523882in}{2.985698in}}%
\pgfpathlineto{\pgfqpoint{5.531431in}{3.003337in}}%
\pgfpathlineto{\pgfqpoint{5.517613in}{3.007817in}}%
\pgfpathlineto{\pgfqpoint{5.503802in}{3.012363in}}%
\pgfpathlineto{\pgfqpoint{5.489999in}{3.016975in}}%
\pgfpathlineto{\pgfqpoint{5.476204in}{3.021654in}}%
\pgfpathlineto{\pgfqpoint{5.468637in}{3.003489in}}%
\pgfpathlineto{\pgfqpoint{5.461079in}{2.985747in}}%
\pgfpathlineto{\pgfqpoint{5.453529in}{2.968419in}}%
\pgfpathlineto{\pgfqpoint{5.445987in}{2.951495in}}%
\pgfpathclose%
\pgfusepath{fill}%
\end{pgfscope}%
\begin{pgfscope}%
\pgfpathrectangle{\pgfqpoint{1.150000in}{0.150000in}}{\pgfqpoint{5.700000in}{5.700000in}}%
\pgfusepath{clip}%
\pgfsetbuttcap%
\pgfsetroundjoin%
\definecolor{currentfill}{rgb}{0.283091,0.110553,0.431554}%
\pgfsetfillcolor{currentfill}%
\pgfsetfillopacity{0.700000}%
\pgfsetlinewidth{0.000000pt}%
\definecolor{currentstroke}{rgb}{0.000000,0.000000,0.000000}%
\pgfsetstrokecolor{currentstroke}%
\pgfsetdash{}{0pt}%
\pgfpathmoveto{\pgfqpoint{3.779939in}{2.441346in}}%
\pgfpathlineto{\pgfqpoint{3.793401in}{2.436668in}}%
\pgfpathlineto{\pgfqpoint{3.806867in}{2.432077in}}%
\pgfpathlineto{\pgfqpoint{3.820339in}{2.427571in}}%
\pgfpathlineto{\pgfqpoint{3.833816in}{2.423151in}}%
\pgfpathlineto{\pgfqpoint{3.841704in}{2.433181in}}%
\pgfpathlineto{\pgfqpoint{3.849587in}{2.443285in}}%
\pgfpathlineto{\pgfqpoint{3.857465in}{2.453466in}}%
\pgfpathlineto{\pgfqpoint{3.865337in}{2.463729in}}%
\pgfpathlineto{\pgfqpoint{3.851870in}{2.468327in}}%
\pgfpathlineto{\pgfqpoint{3.838408in}{2.473010in}}%
\pgfpathlineto{\pgfqpoint{3.824951in}{2.477779in}}%
\pgfpathlineto{\pgfqpoint{3.811499in}{2.482634in}}%
\pgfpathlineto{\pgfqpoint{3.803617in}{2.472186in}}%
\pgfpathlineto{\pgfqpoint{3.795730in}{2.461825in}}%
\pgfpathlineto{\pgfqpoint{3.787837in}{2.451546in}}%
\pgfpathlineto{\pgfqpoint{3.779939in}{2.441346in}}%
\pgfpathclose%
\pgfusepath{fill}%
\end{pgfscope}%
\begin{pgfscope}%
\pgfpathrectangle{\pgfqpoint{1.150000in}{0.150000in}}{\pgfqpoint{5.700000in}{5.700000in}}%
\pgfusepath{clip}%
\pgfsetbuttcap%
\pgfsetroundjoin%
\definecolor{currentfill}{rgb}{0.283187,0.125848,0.444960}%
\pgfsetfillcolor{currentfill}%
\pgfsetfillopacity{0.700000}%
\pgfsetlinewidth{0.000000pt}%
\definecolor{currentstroke}{rgb}{0.000000,0.000000,0.000000}%
\pgfsetstrokecolor{currentstroke}%
\pgfsetdash{}{0pt}%
\pgfpathmoveto{\pgfqpoint{4.004628in}{2.470339in}}%
\pgfpathlineto{\pgfqpoint{4.018135in}{2.466296in}}%
\pgfpathlineto{\pgfqpoint{4.031648in}{2.462336in}}%
\pgfpathlineto{\pgfqpoint{4.045167in}{2.458457in}}%
\pgfpathlineto{\pgfqpoint{4.058692in}{2.454659in}}%
\pgfpathlineto{\pgfqpoint{4.066508in}{2.464753in}}%
\pgfpathlineto{\pgfqpoint{4.074319in}{2.474936in}}%
\pgfpathlineto{\pgfqpoint{4.082126in}{2.485212in}}%
\pgfpathlineto{\pgfqpoint{4.089927in}{2.495585in}}%
\pgfpathlineto{\pgfqpoint{4.076412in}{2.499601in}}%
\pgfpathlineto{\pgfqpoint{4.062903in}{2.503698in}}%
\pgfpathlineto{\pgfqpoint{4.049400in}{2.507877in}}%
\pgfpathlineto{\pgfqpoint{4.035903in}{2.512137in}}%
\pgfpathlineto{\pgfqpoint{4.028092in}{2.501538in}}%
\pgfpathlineto{\pgfqpoint{4.020275in}{2.491042in}}%
\pgfpathlineto{\pgfqpoint{4.012454in}{2.480643in}}%
\pgfpathlineto{\pgfqpoint{4.004628in}{2.470339in}}%
\pgfpathclose%
\pgfusepath{fill}%
\end{pgfscope}%
\begin{pgfscope}%
\pgfpathrectangle{\pgfqpoint{1.150000in}{0.150000in}}{\pgfqpoint{5.700000in}{5.700000in}}%
\pgfusepath{clip}%
\pgfsetbuttcap%
\pgfsetroundjoin%
\definecolor{currentfill}{rgb}{0.257322,0.256130,0.526563}%
\pgfsetfillcolor{currentfill}%
\pgfsetfillopacity{0.700000}%
\pgfsetlinewidth{0.000000pt}%
\definecolor{currentstroke}{rgb}{0.000000,0.000000,0.000000}%
\pgfsetstrokecolor{currentstroke}%
\pgfsetdash{}{0pt}%
\pgfpathmoveto{\pgfqpoint{5.019829in}{2.735594in}}%
\pgfpathlineto{\pgfqpoint{5.033571in}{2.732367in}}%
\pgfpathlineto{\pgfqpoint{5.047321in}{2.729210in}}%
\pgfpathlineto{\pgfqpoint{5.061079in}{2.726122in}}%
\pgfpathlineto{\pgfqpoint{5.074845in}{2.723103in}}%
\pgfpathlineto{\pgfqpoint{5.082378in}{2.735438in}}%
\pgfpathlineto{\pgfqpoint{5.089912in}{2.748028in}}%
\pgfpathlineto{\pgfqpoint{5.097447in}{2.760879in}}%
\pgfpathlineto{\pgfqpoint{5.104984in}{2.774000in}}%
\pgfpathlineto{\pgfqpoint{5.091234in}{2.777438in}}%
\pgfpathlineto{\pgfqpoint{5.077492in}{2.780945in}}%
\pgfpathlineto{\pgfqpoint{5.063758in}{2.784521in}}%
\pgfpathlineto{\pgfqpoint{5.050031in}{2.788166in}}%
\pgfpathlineto{\pgfqpoint{5.042478in}{2.774619in}}%
\pgfpathlineto{\pgfqpoint{5.034928in}{2.761347in}}%
\pgfpathlineto{\pgfqpoint{5.027378in}{2.748341in}}%
\pgfpathlineto{\pgfqpoint{5.019829in}{2.735594in}}%
\pgfpathclose%
\pgfusepath{fill}%
\end{pgfscope}%
\begin{pgfscope}%
\pgfpathrectangle{\pgfqpoint{1.150000in}{0.150000in}}{\pgfqpoint{5.700000in}{5.700000in}}%
\pgfusepath{clip}%
\pgfsetbuttcap%
\pgfsetroundjoin%
\definecolor{currentfill}{rgb}{0.206756,0.371758,0.553117}%
\pgfsetfillcolor{currentfill}%
\pgfsetfillopacity{0.700000}%
\pgfsetlinewidth{0.000000pt}%
\definecolor{currentstroke}{rgb}{0.000000,0.000000,0.000000}%
\pgfsetstrokecolor{currentstroke}%
\pgfsetdash{}{0pt}%
\pgfpathmoveto{\pgfqpoint{5.531431in}{3.003337in}}%
\pgfpathlineto{\pgfqpoint{5.545257in}{2.998923in}}%
\pgfpathlineto{\pgfqpoint{5.559090in}{2.994576in}}%
\pgfpathlineto{\pgfqpoint{5.572931in}{2.990295in}}%
\pgfpathlineto{\pgfqpoint{5.586780in}{2.986079in}}%
\pgfpathlineto{\pgfqpoint{5.594320in}{3.003614in}}%
\pgfpathlineto{\pgfqpoint{5.601870in}{3.021582in}}%
\pgfpathlineto{\pgfqpoint{5.609430in}{3.039993in}}%
\pgfpathlineto{\pgfqpoint{5.617000in}{3.058857in}}%
\pgfpathlineto{\pgfqpoint{5.603170in}{3.063612in}}%
\pgfpathlineto{\pgfqpoint{5.589347in}{3.068433in}}%
\pgfpathlineto{\pgfqpoint{5.575532in}{3.073320in}}%
\pgfpathlineto{\pgfqpoint{5.561724in}{3.078273in}}%
\pgfpathlineto{\pgfqpoint{5.554135in}{3.058862in}}%
\pgfpathlineto{\pgfqpoint{5.546557in}{3.039909in}}%
\pgfpathlineto{\pgfqpoint{5.538989in}{3.021404in}}%
\pgfpathlineto{\pgfqpoint{5.531431in}{3.003337in}}%
\pgfpathclose%
\pgfusepath{fill}%
\end{pgfscope}%
\begin{pgfscope}%
\pgfpathrectangle{\pgfqpoint{1.150000in}{0.150000in}}{\pgfqpoint{5.700000in}{5.700000in}}%
\pgfusepath{clip}%
\pgfsetbuttcap%
\pgfsetroundjoin%
\definecolor{currentfill}{rgb}{0.282656,0.100196,0.422160}%
\pgfsetfillcolor{currentfill}%
\pgfsetfillopacity{0.700000}%
\pgfsetlinewidth{0.000000pt}%
\definecolor{currentstroke}{rgb}{0.000000,0.000000,0.000000}%
\pgfsetstrokecolor{currentstroke}%
\pgfsetdash{}{0pt}%
\pgfpathmoveto{\pgfqpoint{3.555158in}{2.421726in}}%
\pgfpathlineto{\pgfqpoint{3.568583in}{2.416272in}}%
\pgfpathlineto{\pgfqpoint{3.582011in}{2.410909in}}%
\pgfpathlineto{\pgfqpoint{3.595445in}{2.405638in}}%
\pgfpathlineto{\pgfqpoint{3.608882in}{2.400457in}}%
\pgfpathlineto{\pgfqpoint{3.616845in}{2.410393in}}%
\pgfpathlineto{\pgfqpoint{3.624802in}{2.420394in}}%
\pgfpathlineto{\pgfqpoint{3.632754in}{2.430463in}}%
\pgfpathlineto{\pgfqpoint{3.640699in}{2.440603in}}%
\pgfpathlineto{\pgfqpoint{3.627271in}{2.445921in}}%
\pgfpathlineto{\pgfqpoint{3.613848in}{2.451329in}}%
\pgfpathlineto{\pgfqpoint{3.600429in}{2.456829in}}%
\pgfpathlineto{\pgfqpoint{3.587014in}{2.462420in}}%
\pgfpathlineto{\pgfqpoint{3.579059in}{2.452136in}}%
\pgfpathlineto{\pgfqpoint{3.571098in}{2.441927in}}%
\pgfpathlineto{\pgfqpoint{3.563131in}{2.431792in}}%
\pgfpathlineto{\pgfqpoint{3.555158in}{2.421726in}}%
\pgfpathclose%
\pgfusepath{fill}%
\end{pgfscope}%
\begin{pgfscope}%
\pgfpathrectangle{\pgfqpoint{1.150000in}{0.150000in}}{\pgfqpoint{5.700000in}{5.700000in}}%
\pgfusepath{clip}%
\pgfsetbuttcap%
\pgfsetroundjoin%
\definecolor{currentfill}{rgb}{0.282623,0.140926,0.457517}%
\pgfsetfillcolor{currentfill}%
\pgfsetfillopacity{0.700000}%
\pgfsetlinewidth{0.000000pt}%
\definecolor{currentstroke}{rgb}{0.000000,0.000000,0.000000}%
\pgfsetstrokecolor{currentstroke}%
\pgfsetdash{}{0pt}%
\pgfpathmoveto{\pgfqpoint{2.857848in}{2.516826in}}%
\pgfpathlineto{\pgfqpoint{2.871228in}{2.507556in}}%
\pgfpathlineto{\pgfqpoint{2.884609in}{2.498407in}}%
\pgfpathlineto{\pgfqpoint{2.897991in}{2.489377in}}%
\pgfpathlineto{\pgfqpoint{2.911373in}{2.480465in}}%
\pgfpathlineto{\pgfqpoint{2.919574in}{2.489827in}}%
\pgfpathlineto{\pgfqpoint{2.927768in}{2.499267in}}%
\pgfpathlineto{\pgfqpoint{2.935954in}{2.508787in}}%
\pgfpathlineto{\pgfqpoint{2.944133in}{2.518387in}}%
\pgfpathlineto{\pgfqpoint{2.930763in}{2.527334in}}%
\pgfpathlineto{\pgfqpoint{2.917394in}{2.536399in}}%
\pgfpathlineto{\pgfqpoint{2.904026in}{2.545584in}}%
\pgfpathlineto{\pgfqpoint{2.890658in}{2.554889in}}%
\pgfpathlineto{\pgfqpoint{2.882467in}{2.545246in}}%
\pgfpathlineto{\pgfqpoint{2.874268in}{2.535688in}}%
\pgfpathlineto{\pgfqpoint{2.866062in}{2.526215in}}%
\pgfpathlineto{\pgfqpoint{2.857848in}{2.516826in}}%
\pgfpathclose%
\pgfusepath{fill}%
\end{pgfscope}%
\begin{pgfscope}%
\pgfpathrectangle{\pgfqpoint{1.150000in}{0.150000in}}{\pgfqpoint{5.700000in}{5.700000in}}%
\pgfusepath{clip}%
\pgfsetbuttcap%
\pgfsetroundjoin%
\definecolor{currentfill}{rgb}{0.278012,0.180367,0.486697}%
\pgfsetfillcolor{currentfill}%
\pgfsetfillopacity{0.700000}%
\pgfsetlinewidth{0.000000pt}%
\definecolor{currentstroke}{rgb}{0.000000,0.000000,0.000000}%
\pgfsetstrokecolor{currentstroke}%
\pgfsetdash{}{0pt}%
\pgfpathmoveto{\pgfqpoint{4.539361in}{2.579940in}}%
\pgfpathlineto{\pgfqpoint{4.552995in}{2.576785in}}%
\pgfpathlineto{\pgfqpoint{4.566636in}{2.573705in}}%
\pgfpathlineto{\pgfqpoint{4.580284in}{2.570698in}}%
\pgfpathlineto{\pgfqpoint{4.593939in}{2.567765in}}%
\pgfpathlineto{\pgfqpoint{4.601589in}{2.578333in}}%
\pgfpathlineto{\pgfqpoint{4.609237in}{2.589054in}}%
\pgfpathlineto{\pgfqpoint{4.616881in}{2.599932in}}%
\pgfpathlineto{\pgfqpoint{4.624522in}{2.610973in}}%
\pgfpathlineto{\pgfqpoint{4.610880in}{2.614225in}}%
\pgfpathlineto{\pgfqpoint{4.597244in}{2.617551in}}%
\pgfpathlineto{\pgfqpoint{4.583616in}{2.620950in}}%
\pgfpathlineto{\pgfqpoint{4.569994in}{2.624423in}}%
\pgfpathlineto{\pgfqpoint{4.562340in}{2.613056in}}%
\pgfpathlineto{\pgfqpoint{4.554684in}{2.601857in}}%
\pgfpathlineto{\pgfqpoint{4.547024in}{2.590820in}}%
\pgfpathlineto{\pgfqpoint{4.539361in}{2.579940in}}%
\pgfpathclose%
\pgfusepath{fill}%
\end{pgfscope}%
\begin{pgfscope}%
\pgfpathrectangle{\pgfqpoint{1.150000in}{0.150000in}}{\pgfqpoint{5.700000in}{5.700000in}}%
\pgfusepath{clip}%
\pgfsetbuttcap%
\pgfsetroundjoin%
\definecolor{currentfill}{rgb}{0.282290,0.145912,0.461510}%
\pgfsetfillcolor{currentfill}%
\pgfsetfillopacity{0.700000}%
\pgfsetlinewidth{0.000000pt}%
\definecolor{currentstroke}{rgb}{0.000000,0.000000,0.000000}%
\pgfsetstrokecolor{currentstroke}%
\pgfsetdash{}{0pt}%
\pgfpathmoveto{\pgfqpoint{4.229338in}{2.506989in}}%
\pgfpathlineto{\pgfqpoint{4.242899in}{2.503452in}}%
\pgfpathlineto{\pgfqpoint{4.256465in}{2.499992in}}%
\pgfpathlineto{\pgfqpoint{4.270038in}{2.496610in}}%
\pgfpathlineto{\pgfqpoint{4.283618in}{2.493306in}}%
\pgfpathlineto{\pgfqpoint{4.291363in}{2.503479in}}%
\pgfpathlineto{\pgfqpoint{4.299105in}{2.513761in}}%
\pgfpathlineto{\pgfqpoint{4.306841in}{2.524156in}}%
\pgfpathlineto{\pgfqpoint{4.314574in}{2.534670in}}%
\pgfpathlineto{\pgfqpoint{4.301005in}{2.538233in}}%
\pgfpathlineto{\pgfqpoint{4.287443in}{2.541873in}}%
\pgfpathlineto{\pgfqpoint{4.273887in}{2.545590in}}%
\pgfpathlineto{\pgfqpoint{4.260338in}{2.549386in}}%
\pgfpathlineto{\pgfqpoint{4.252595in}{2.538606in}}%
\pgfpathlineto{\pgfqpoint{4.244847in}{2.527950in}}%
\pgfpathlineto{\pgfqpoint{4.237095in}{2.517413in}}%
\pgfpathlineto{\pgfqpoint{4.229338in}{2.506989in}}%
\pgfpathclose%
\pgfusepath{fill}%
\end{pgfscope}%
\begin{pgfscope}%
\pgfpathrectangle{\pgfqpoint{1.150000in}{0.150000in}}{\pgfqpoint{5.700000in}{5.700000in}}%
\pgfusepath{clip}%
\pgfsetbuttcap%
\pgfsetroundjoin%
\definecolor{currentfill}{rgb}{0.277134,0.185228,0.489898}%
\pgfsetfillcolor{currentfill}%
\pgfsetfillopacity{0.700000}%
\pgfsetlinewidth{0.000000pt}%
\definecolor{currentstroke}{rgb}{0.000000,0.000000,0.000000}%
\pgfsetstrokecolor{currentstroke}%
\pgfsetdash{}{0pt}%
\pgfpathmoveto{\pgfqpoint{2.664176in}{2.600909in}}%
\pgfpathlineto{\pgfqpoint{2.677575in}{2.590128in}}%
\pgfpathlineto{\pgfqpoint{2.690973in}{2.579480in}}%
\pgfpathlineto{\pgfqpoint{2.704369in}{2.568965in}}%
\pgfpathlineto{\pgfqpoint{2.717765in}{2.558580in}}%
\pgfpathlineto{\pgfqpoint{2.726037in}{2.567659in}}%
\pgfpathlineto{\pgfqpoint{2.734302in}{2.576827in}}%
\pgfpathlineto{\pgfqpoint{2.742558in}{2.586087in}}%
\pgfpathlineto{\pgfqpoint{2.750806in}{2.595438in}}%
\pgfpathlineto{\pgfqpoint{2.737424in}{2.605837in}}%
\pgfpathlineto{\pgfqpoint{2.724042in}{2.616367in}}%
\pgfpathlineto{\pgfqpoint{2.710658in}{2.627029in}}%
\pgfpathlineto{\pgfqpoint{2.697274in}{2.637825in}}%
\pgfpathlineto{\pgfqpoint{2.689012in}{2.628451in}}%
\pgfpathlineto{\pgfqpoint{2.680741in}{2.619175in}}%
\pgfpathlineto{\pgfqpoint{2.672463in}{2.609995in}}%
\pgfpathlineto{\pgfqpoint{2.664176in}{2.600909in}}%
\pgfpathclose%
\pgfusepath{fill}%
\end{pgfscope}%
\begin{pgfscope}%
\pgfpathrectangle{\pgfqpoint{1.150000in}{0.150000in}}{\pgfqpoint{5.700000in}{5.700000in}}%
\pgfusepath{clip}%
\pgfsetbuttcap%
\pgfsetroundjoin%
\definecolor{currentfill}{rgb}{0.263663,0.237631,0.518762}%
\pgfsetfillcolor{currentfill}%
\pgfsetfillopacity{0.700000}%
\pgfsetlinewidth{0.000000pt}%
\definecolor{currentstroke}{rgb}{0.000000,0.000000,0.000000}%
\pgfsetstrokecolor{currentstroke}%
\pgfsetdash{}{0pt}%
\pgfpathmoveto{\pgfqpoint{4.934681in}{2.699063in}}%
\pgfpathlineto{\pgfqpoint{4.948408in}{2.695955in}}%
\pgfpathlineto{\pgfqpoint{4.962142in}{2.692918in}}%
\pgfpathlineto{\pgfqpoint{4.975885in}{2.689950in}}%
\pgfpathlineto{\pgfqpoint{4.989635in}{2.687052in}}%
\pgfpathlineto{\pgfqpoint{4.997183in}{2.698836in}}%
\pgfpathlineto{\pgfqpoint{5.004732in}{2.710849in}}%
\pgfpathlineto{\pgfqpoint{5.012280in}{2.723099in}}%
\pgfpathlineto{\pgfqpoint{5.019829in}{2.735594in}}%
\pgfpathlineto{\pgfqpoint{5.006094in}{2.738891in}}%
\pgfpathlineto{\pgfqpoint{4.992367in}{2.742257in}}%
\pgfpathlineto{\pgfqpoint{4.978647in}{2.745693in}}%
\pgfpathlineto{\pgfqpoint{4.964935in}{2.749199in}}%
\pgfpathlineto{\pgfqpoint{4.957371in}{2.736299in}}%
\pgfpathlineto{\pgfqpoint{4.949808in}{2.723647in}}%
\pgfpathlineto{\pgfqpoint{4.942244in}{2.711238in}}%
\pgfpathlineto{\pgfqpoint{4.934681in}{2.699063in}}%
\pgfpathclose%
\pgfusepath{fill}%
\end{pgfscope}%
\begin{pgfscope}%
\pgfpathrectangle{\pgfqpoint{1.150000in}{0.150000in}}{\pgfqpoint{5.700000in}{5.700000in}}%
\pgfusepath{clip}%
\pgfsetbuttcap%
\pgfsetroundjoin%
\definecolor{currentfill}{rgb}{0.194100,0.399323,0.555565}%
\pgfsetfillcolor{currentfill}%
\pgfsetfillopacity{0.700000}%
\pgfsetlinewidth{0.000000pt}%
\definecolor{currentstroke}{rgb}{0.000000,0.000000,0.000000}%
\pgfsetstrokecolor{currentstroke}%
\pgfsetdash{}{0pt}%
\pgfpathmoveto{\pgfqpoint{5.617000in}{3.058857in}}%
\pgfpathlineto{\pgfqpoint{5.630838in}{3.054168in}}%
\pgfpathlineto{\pgfqpoint{5.644684in}{3.049544in}}%
\pgfpathlineto{\pgfqpoint{5.658537in}{3.044986in}}%
\pgfpathlineto{\pgfqpoint{5.672398in}{3.040494in}}%
\pgfpathlineto{\pgfqpoint{5.679962in}{3.059269in}}%
\pgfpathlineto{\pgfqpoint{5.687537in}{3.078512in}}%
\pgfpathlineto{\pgfqpoint{5.695124in}{3.098235in}}%
\pgfpathlineto{\pgfqpoint{5.702725in}{3.118447in}}%
\pgfpathlineto{\pgfqpoint{5.688883in}{3.123500in}}%
\pgfpathlineto{\pgfqpoint{5.675048in}{3.128618in}}%
\pgfpathlineto{\pgfqpoint{5.661221in}{3.133801in}}%
\pgfpathlineto{\pgfqpoint{5.647401in}{3.139051in}}%
\pgfpathlineto{\pgfqpoint{5.639782in}{3.118271in}}%
\pgfpathlineto{\pgfqpoint{5.632176in}{3.097985in}}%
\pgfpathlineto{\pgfqpoint{5.624582in}{3.078184in}}%
\pgfpathlineto{\pgfqpoint{5.617000in}{3.058857in}}%
\pgfpathclose%
\pgfusepath{fill}%
\end{pgfscope}%
\begin{pgfscope}%
\pgfpathrectangle{\pgfqpoint{1.150000in}{0.150000in}}{\pgfqpoint{5.700000in}{5.700000in}}%
\pgfusepath{clip}%
\pgfsetbuttcap%
\pgfsetroundjoin%
\definecolor{currentfill}{rgb}{0.267968,0.223549,0.512008}%
\pgfsetfillcolor{currentfill}%
\pgfsetfillopacity{0.700000}%
\pgfsetlinewidth{0.000000pt}%
\definecolor{currentstroke}{rgb}{0.000000,0.000000,0.000000}%
\pgfsetstrokecolor{currentstroke}%
\pgfsetdash{}{0pt}%
\pgfpathmoveto{\pgfqpoint{4.849527in}{2.664192in}}%
\pgfpathlineto{\pgfqpoint{4.863239in}{2.661181in}}%
\pgfpathlineto{\pgfqpoint{4.876957in}{2.658241in}}%
\pgfpathlineto{\pgfqpoint{4.890684in}{2.655371in}}%
\pgfpathlineto{\pgfqpoint{4.904418in}{2.652572in}}%
\pgfpathlineto{\pgfqpoint{4.911985in}{2.663877in}}%
\pgfpathlineto{\pgfqpoint{4.919551in}{2.675390in}}%
\pgfpathlineto{\pgfqpoint{4.927116in}{2.687116in}}%
\pgfpathlineto{\pgfqpoint{4.934681in}{2.699063in}}%
\pgfpathlineto{\pgfqpoint{4.920961in}{2.702241in}}%
\pgfpathlineto{\pgfqpoint{4.907249in}{2.705489in}}%
\pgfpathlineto{\pgfqpoint{4.893545in}{2.708808in}}%
\pgfpathlineto{\pgfqpoint{4.879848in}{2.712198in}}%
\pgfpathlineto{\pgfqpoint{4.872269in}{2.699865in}}%
\pgfpathlineto{\pgfqpoint{4.864690in}{2.687758in}}%
\pgfpathlineto{\pgfqpoint{4.857109in}{2.675869in}}%
\pgfpathlineto{\pgfqpoint{4.849527in}{2.664192in}}%
\pgfpathclose%
\pgfusepath{fill}%
\end{pgfscope}%
\begin{pgfscope}%
\pgfpathrectangle{\pgfqpoint{1.150000in}{0.150000in}}{\pgfqpoint{5.700000in}{5.700000in}}%
\pgfusepath{clip}%
\pgfsetbuttcap%
\pgfsetroundjoin%
\definecolor{currentfill}{rgb}{0.283197,0.115680,0.436115}%
\pgfsetfillcolor{currentfill}%
\pgfsetfillopacity{0.700000}%
\pgfsetlinewidth{0.000000pt}%
\definecolor{currentstroke}{rgb}{0.000000,0.000000,0.000000}%
\pgfsetstrokecolor{currentstroke}%
\pgfsetdash{}{0pt}%
\pgfpathmoveto{\pgfqpoint{3.919259in}{2.446187in}}%
\pgfpathlineto{\pgfqpoint{3.932754in}{2.442011in}}%
\pgfpathlineto{\pgfqpoint{3.946254in}{2.437919in}}%
\pgfpathlineto{\pgfqpoint{3.959759in}{2.433909in}}%
\pgfpathlineto{\pgfqpoint{3.973271in}{2.429983in}}%
\pgfpathlineto{\pgfqpoint{3.981118in}{2.439950in}}%
\pgfpathlineto{\pgfqpoint{3.988960in}{2.449996in}}%
\pgfpathlineto{\pgfqpoint{3.996796in}{2.460125in}}%
\pgfpathlineto{\pgfqpoint{4.004628in}{2.470339in}}%
\pgfpathlineto{\pgfqpoint{3.991126in}{2.474464in}}%
\pgfpathlineto{\pgfqpoint{3.977631in}{2.478671in}}%
\pgfpathlineto{\pgfqpoint{3.964141in}{2.482961in}}%
\pgfpathlineto{\pgfqpoint{3.950656in}{2.487334in}}%
\pgfpathlineto{\pgfqpoint{3.942815in}{2.476915in}}%
\pgfpathlineto{\pgfqpoint{3.934968in}{2.466586in}}%
\pgfpathlineto{\pgfqpoint{3.927116in}{2.456345in}}%
\pgfpathlineto{\pgfqpoint{3.919259in}{2.446187in}}%
\pgfpathclose%
\pgfusepath{fill}%
\end{pgfscope}%
\begin{pgfscope}%
\pgfpathrectangle{\pgfqpoint{1.150000in}{0.150000in}}{\pgfqpoint{5.700000in}{5.700000in}}%
\pgfusepath{clip}%
\pgfsetbuttcap%
\pgfsetroundjoin%
\definecolor{currentfill}{rgb}{0.282910,0.105393,0.426902}%
\pgfsetfillcolor{currentfill}%
\pgfsetfillopacity{0.700000}%
\pgfsetlinewidth{0.000000pt}%
\definecolor{currentstroke}{rgb}{0.000000,0.000000,0.000000}%
\pgfsetstrokecolor{currentstroke}%
\pgfsetdash{}{0pt}%
\pgfpathmoveto{\pgfqpoint{3.190693in}{2.428967in}}%
\pgfpathlineto{\pgfqpoint{3.204079in}{2.421847in}}%
\pgfpathlineto{\pgfqpoint{3.217469in}{2.414830in}}%
\pgfpathlineto{\pgfqpoint{3.230861in}{2.407917in}}%
\pgfpathlineto{\pgfqpoint{3.244256in}{2.401105in}}%
\pgfpathlineto{\pgfqpoint{3.252343in}{2.410786in}}%
\pgfpathlineto{\pgfqpoint{3.260424in}{2.420531in}}%
\pgfpathlineto{\pgfqpoint{3.268499in}{2.430342in}}%
\pgfpathlineto{\pgfqpoint{3.276567in}{2.440221in}}%
\pgfpathlineto{\pgfqpoint{3.263183in}{2.447109in}}%
\pgfpathlineto{\pgfqpoint{3.249801in}{2.454099in}}%
\pgfpathlineto{\pgfqpoint{3.236423in}{2.461192in}}%
\pgfpathlineto{\pgfqpoint{3.223047in}{2.468388in}}%
\pgfpathlineto{\pgfqpoint{3.214968in}{2.458426in}}%
\pgfpathlineto{\pgfqpoint{3.206883in}{2.448536in}}%
\pgfpathlineto{\pgfqpoint{3.198791in}{2.438717in}}%
\pgfpathlineto{\pgfqpoint{3.190693in}{2.428967in}}%
\pgfpathclose%
\pgfusepath{fill}%
\end{pgfscope}%
\begin{pgfscope}%
\pgfpathrectangle{\pgfqpoint{1.150000in}{0.150000in}}{\pgfqpoint{5.700000in}{5.700000in}}%
\pgfusepath{clip}%
\pgfsetbuttcap%
\pgfsetroundjoin%
\definecolor{currentfill}{rgb}{0.282656,0.100196,0.422160}%
\pgfsetfillcolor{currentfill}%
\pgfsetfillopacity{0.700000}%
\pgfsetlinewidth{0.000000pt}%
\definecolor{currentstroke}{rgb}{0.000000,0.000000,0.000000}%
\pgfsetstrokecolor{currentstroke}%
\pgfsetdash{}{0pt}%
\pgfpathmoveto{\pgfqpoint{3.694457in}{2.420232in}}%
\pgfpathlineto{\pgfqpoint{3.707908in}{2.415362in}}%
\pgfpathlineto{\pgfqpoint{3.721364in}{2.410579in}}%
\pgfpathlineto{\pgfqpoint{3.734825in}{2.405885in}}%
\pgfpathlineto{\pgfqpoint{3.748291in}{2.401278in}}%
\pgfpathlineto{\pgfqpoint{3.756211in}{2.411192in}}%
\pgfpathlineto{\pgfqpoint{3.764126in}{2.421172in}}%
\pgfpathlineto{\pgfqpoint{3.772036in}{2.431223in}}%
\pgfpathlineto{\pgfqpoint{3.779939in}{2.441346in}}%
\pgfpathlineto{\pgfqpoint{3.766483in}{2.446111in}}%
\pgfpathlineto{\pgfqpoint{3.753032in}{2.450963in}}%
\pgfpathlineto{\pgfqpoint{3.739585in}{2.455902in}}%
\pgfpathlineto{\pgfqpoint{3.726144in}{2.460930in}}%
\pgfpathlineto{\pgfqpoint{3.718230in}{2.450642in}}%
\pgfpathlineto{\pgfqpoint{3.710312in}{2.440432in}}%
\pgfpathlineto{\pgfqpoint{3.702387in}{2.430296in}}%
\pgfpathlineto{\pgfqpoint{3.694457in}{2.420232in}}%
\pgfpathclose%
\pgfusepath{fill}%
\end{pgfscope}%
\begin{pgfscope}%
\pgfpathrectangle{\pgfqpoint{1.150000in}{0.150000in}}{\pgfqpoint{5.700000in}{5.700000in}}%
\pgfusepath{clip}%
\pgfsetbuttcap%
\pgfsetroundjoin%
\definecolor{currentfill}{rgb}{0.279574,0.170599,0.479997}%
\pgfsetfillcolor{currentfill}%
\pgfsetfillopacity{0.700000}%
\pgfsetlinewidth{0.000000pt}%
\definecolor{currentstroke}{rgb}{0.000000,0.000000,0.000000}%
\pgfsetstrokecolor{currentstroke}%
\pgfsetdash{}{0pt}%
\pgfpathmoveto{\pgfqpoint{4.454160in}{2.550044in}}%
\pgfpathlineto{\pgfqpoint{4.467778in}{2.546890in}}%
\pgfpathlineto{\pgfqpoint{4.481403in}{2.543810in}}%
\pgfpathlineto{\pgfqpoint{4.495035in}{2.540805in}}%
\pgfpathlineto{\pgfqpoint{4.508674in}{2.537874in}}%
\pgfpathlineto{\pgfqpoint{4.516351in}{2.548184in}}%
\pgfpathlineto{\pgfqpoint{4.524024in}{2.558628in}}%
\pgfpathlineto{\pgfqpoint{4.531694in}{2.569211in}}%
\pgfpathlineto{\pgfqpoint{4.539361in}{2.579940in}}%
\pgfpathlineto{\pgfqpoint{4.525734in}{2.583170in}}%
\pgfpathlineto{\pgfqpoint{4.512114in}{2.586473in}}%
\pgfpathlineto{\pgfqpoint{4.498501in}{2.589851in}}%
\pgfpathlineto{\pgfqpoint{4.484895in}{2.593305in}}%
\pgfpathlineto{\pgfqpoint{4.477217in}{2.582270in}}%
\pgfpathlineto{\pgfqpoint{4.469535in}{2.571385in}}%
\pgfpathlineto{\pgfqpoint{4.461849in}{2.560645in}}%
\pgfpathlineto{\pgfqpoint{4.454160in}{2.550044in}}%
\pgfpathclose%
\pgfusepath{fill}%
\end{pgfscope}%
\begin{pgfscope}%
\pgfpathrectangle{\pgfqpoint{1.150000in}{0.150000in}}{\pgfqpoint{5.700000in}{5.700000in}}%
\pgfusepath{clip}%
\pgfsetbuttcap%
\pgfsetroundjoin%
\definecolor{currentfill}{rgb}{0.282327,0.094955,0.417331}%
\pgfsetfillcolor{currentfill}%
\pgfsetfillopacity{0.700000}%
\pgfsetlinewidth{0.000000pt}%
\definecolor{currentstroke}{rgb}{0.000000,0.000000,0.000000}%
\pgfsetstrokecolor{currentstroke}%
\pgfsetdash{}{0pt}%
\pgfpathmoveto{\pgfqpoint{3.330133in}{2.413682in}}%
\pgfpathlineto{\pgfqpoint{3.343532in}{2.407296in}}%
\pgfpathlineto{\pgfqpoint{3.356935in}{2.401009in}}%
\pgfpathlineto{\pgfqpoint{3.370341in}{2.394820in}}%
\pgfpathlineto{\pgfqpoint{3.383751in}{2.388729in}}%
\pgfpathlineto{\pgfqpoint{3.391792in}{2.398499in}}%
\pgfpathlineto{\pgfqpoint{3.399826in}{2.408330in}}%
\pgfpathlineto{\pgfqpoint{3.407855in}{2.418225in}}%
\pgfpathlineto{\pgfqpoint{3.415878in}{2.428186in}}%
\pgfpathlineto{\pgfqpoint{3.402478in}{2.434374in}}%
\pgfpathlineto{\pgfqpoint{3.389082in}{2.440660in}}%
\pgfpathlineto{\pgfqpoint{3.375690in}{2.447044in}}%
\pgfpathlineto{\pgfqpoint{3.362301in}{2.453526in}}%
\pgfpathlineto{\pgfqpoint{3.354268in}{2.443461in}}%
\pgfpathlineto{\pgfqpoint{3.346229in}{2.433466in}}%
\pgfpathlineto{\pgfqpoint{3.338184in}{2.423541in}}%
\pgfpathlineto{\pgfqpoint{3.330133in}{2.413682in}}%
\pgfpathclose%
\pgfusepath{fill}%
\end{pgfscope}%
\begin{pgfscope}%
\pgfpathrectangle{\pgfqpoint{1.150000in}{0.150000in}}{\pgfqpoint{5.700000in}{5.700000in}}%
\pgfusepath{clip}%
\pgfsetbuttcap%
\pgfsetroundjoin%
\definecolor{currentfill}{rgb}{0.283197,0.115680,0.436115}%
\pgfsetfillcolor{currentfill}%
\pgfsetfillopacity{0.700000}%
\pgfsetlinewidth{0.000000pt}%
\definecolor{currentstroke}{rgb}{0.000000,0.000000,0.000000}%
\pgfsetstrokecolor{currentstroke}%
\pgfsetdash{}{0pt}%
\pgfpathmoveto{\pgfqpoint{3.051129in}{2.450967in}}%
\pgfpathlineto{\pgfqpoint{3.064510in}{2.443045in}}%
\pgfpathlineto{\pgfqpoint{3.077893in}{2.435234in}}%
\pgfpathlineto{\pgfqpoint{3.091278in}{2.427532in}}%
\pgfpathlineto{\pgfqpoint{3.104665in}{2.419937in}}%
\pgfpathlineto{\pgfqpoint{3.112801in}{2.429475in}}%
\pgfpathlineto{\pgfqpoint{3.120931in}{2.439080in}}%
\pgfpathlineto{\pgfqpoint{3.129053in}{2.448755in}}%
\pgfpathlineto{\pgfqpoint{3.137169in}{2.458501in}}%
\pgfpathlineto{\pgfqpoint{3.123794in}{2.466152in}}%
\pgfpathlineto{\pgfqpoint{3.110421in}{2.473910in}}%
\pgfpathlineto{\pgfqpoint{3.097050in}{2.481777in}}%
\pgfpathlineto{\pgfqpoint{3.083680in}{2.489754in}}%
\pgfpathlineto{\pgfqpoint{3.075553in}{2.479945in}}%
\pgfpathlineto{\pgfqpoint{3.067419in}{2.470211in}}%
\pgfpathlineto{\pgfqpoint{3.059277in}{2.460552in}}%
\pgfpathlineto{\pgfqpoint{3.051129in}{2.450967in}}%
\pgfpathclose%
\pgfusepath{fill}%
\end{pgfscope}%
\begin{pgfscope}%
\pgfpathrectangle{\pgfqpoint{1.150000in}{0.150000in}}{\pgfqpoint{5.700000in}{5.700000in}}%
\pgfusepath{clip}%
\pgfsetbuttcap%
\pgfsetroundjoin%
\definecolor{currentfill}{rgb}{0.282884,0.135920,0.453427}%
\pgfsetfillcolor{currentfill}%
\pgfsetfillopacity{0.700000}%
\pgfsetlinewidth{0.000000pt}%
\definecolor{currentstroke}{rgb}{0.000000,0.000000,0.000000}%
\pgfsetstrokecolor{currentstroke}%
\pgfsetdash{}{0pt}%
\pgfpathmoveto{\pgfqpoint{4.144046in}{2.480327in}}%
\pgfpathlineto{\pgfqpoint{4.157592in}{2.476712in}}%
\pgfpathlineto{\pgfqpoint{4.171143in}{2.473177in}}%
\pgfpathlineto{\pgfqpoint{4.184701in}{2.469720in}}%
\pgfpathlineto{\pgfqpoint{4.198266in}{2.466342in}}%
\pgfpathlineto{\pgfqpoint{4.206041in}{2.476356in}}%
\pgfpathlineto{\pgfqpoint{4.213811in}{2.486465in}}%
\pgfpathlineto{\pgfqpoint{4.221577in}{2.496675in}}%
\pgfpathlineto{\pgfqpoint{4.229338in}{2.506989in}}%
\pgfpathlineto{\pgfqpoint{4.215785in}{2.510605in}}%
\pgfpathlineto{\pgfqpoint{4.202238in}{2.514300in}}%
\pgfpathlineto{\pgfqpoint{4.188697in}{2.518074in}}%
\pgfpathlineto{\pgfqpoint{4.175162in}{2.521926in}}%
\pgfpathlineto{\pgfqpoint{4.167390in}{2.511367in}}%
\pgfpathlineto{\pgfqpoint{4.159613in}{2.500916in}}%
\pgfpathlineto{\pgfqpoint{4.151832in}{2.490571in}}%
\pgfpathlineto{\pgfqpoint{4.144046in}{2.480327in}}%
\pgfpathclose%
\pgfusepath{fill}%
\end{pgfscope}%
\begin{pgfscope}%
\pgfpathrectangle{\pgfqpoint{1.150000in}{0.150000in}}{\pgfqpoint{5.700000in}{5.700000in}}%
\pgfusepath{clip}%
\pgfsetbuttcap%
\pgfsetroundjoin%
\definecolor{currentfill}{rgb}{0.280255,0.165693,0.476498}%
\pgfsetfillcolor{currentfill}%
\pgfsetfillopacity{0.700000}%
\pgfsetlinewidth{0.000000pt}%
\definecolor{currentstroke}{rgb}{0.000000,0.000000,0.000000}%
\pgfsetstrokecolor{currentstroke}%
\pgfsetdash{}{0pt}%
\pgfpathmoveto{\pgfqpoint{2.717765in}{2.558580in}}%
\pgfpathlineto{\pgfqpoint{2.731160in}{2.548325in}}%
\pgfpathlineto{\pgfqpoint{2.744555in}{2.538199in}}%
\pgfpathlineto{\pgfqpoint{2.757949in}{2.528201in}}%
\pgfpathlineto{\pgfqpoint{2.771342in}{2.518329in}}%
\pgfpathlineto{\pgfqpoint{2.779600in}{2.527400in}}%
\pgfpathlineto{\pgfqpoint{2.787851in}{2.536557in}}%
\pgfpathlineto{\pgfqpoint{2.796093in}{2.545799in}}%
\pgfpathlineto{\pgfqpoint{2.804328in}{2.555129in}}%
\pgfpathlineto{\pgfqpoint{2.790948in}{2.565015in}}%
\pgfpathlineto{\pgfqpoint{2.777568in}{2.575028in}}%
\pgfpathlineto{\pgfqpoint{2.764187in}{2.585169in}}%
\pgfpathlineto{\pgfqpoint{2.750806in}{2.595438in}}%
\pgfpathlineto{\pgfqpoint{2.742558in}{2.586087in}}%
\pgfpathlineto{\pgfqpoint{2.734302in}{2.576827in}}%
\pgfpathlineto{\pgfqpoint{2.726037in}{2.567659in}}%
\pgfpathlineto{\pgfqpoint{2.717765in}{2.558580in}}%
\pgfpathclose%
\pgfusepath{fill}%
\end{pgfscope}%
\begin{pgfscope}%
\pgfpathrectangle{\pgfqpoint{1.150000in}{0.150000in}}{\pgfqpoint{5.700000in}{5.700000in}}%
\pgfusepath{clip}%
\pgfsetbuttcap%
\pgfsetroundjoin%
\definecolor{currentfill}{rgb}{0.271828,0.209303,0.504434}%
\pgfsetfillcolor{currentfill}%
\pgfsetfillopacity{0.700000}%
\pgfsetlinewidth{0.000000pt}%
\definecolor{currentstroke}{rgb}{0.000000,0.000000,0.000000}%
\pgfsetstrokecolor{currentstroke}%
\pgfsetdash{}{0pt}%
\pgfpathmoveto{\pgfqpoint{4.764358in}{2.630792in}}%
\pgfpathlineto{\pgfqpoint{4.778054in}{2.627855in}}%
\pgfpathlineto{\pgfqpoint{4.791757in}{2.624989in}}%
\pgfpathlineto{\pgfqpoint{4.805467in}{2.622194in}}%
\pgfpathlineto{\pgfqpoint{4.819185in}{2.619471in}}%
\pgfpathlineto{\pgfqpoint{4.826773in}{2.630367in}}%
\pgfpathlineto{\pgfqpoint{4.834360in}{2.641448in}}%
\pgfpathlineto{\pgfqpoint{4.841944in}{2.652720in}}%
\pgfpathlineto{\pgfqpoint{4.849527in}{2.664192in}}%
\pgfpathlineto{\pgfqpoint{4.835824in}{2.667274in}}%
\pgfpathlineto{\pgfqpoint{4.822127in}{2.670428in}}%
\pgfpathlineto{\pgfqpoint{4.808439in}{2.673652in}}%
\pgfpathlineto{\pgfqpoint{4.794757in}{2.676949in}}%
\pgfpathlineto{\pgfqpoint{4.787160in}{2.665111in}}%
\pgfpathlineto{\pgfqpoint{4.779561in}{2.653477in}}%
\pgfpathlineto{\pgfqpoint{4.771961in}{2.642039in}}%
\pgfpathlineto{\pgfqpoint{4.764358in}{2.630792in}}%
\pgfpathclose%
\pgfusepath{fill}%
\end{pgfscope}%
\begin{pgfscope}%
\pgfpathrectangle{\pgfqpoint{1.150000in}{0.150000in}}{\pgfqpoint{5.700000in}{5.700000in}}%
\pgfusepath{clip}%
\pgfsetbuttcap%
\pgfsetroundjoin%
\definecolor{currentfill}{rgb}{0.282327,0.094955,0.417331}%
\pgfsetfillcolor{currentfill}%
\pgfsetfillopacity{0.700000}%
\pgfsetlinewidth{0.000000pt}%
\definecolor{currentstroke}{rgb}{0.000000,0.000000,0.000000}%
\pgfsetstrokecolor{currentstroke}%
\pgfsetdash{}{0pt}%
\pgfpathmoveto{\pgfqpoint{3.469512in}{2.404395in}}%
\pgfpathlineto{\pgfqpoint{3.482930in}{2.398685in}}%
\pgfpathlineto{\pgfqpoint{3.496352in}{2.393068in}}%
\pgfpathlineto{\pgfqpoint{3.509778in}{2.387546in}}%
\pgfpathlineto{\pgfqpoint{3.523208in}{2.382116in}}%
\pgfpathlineto{\pgfqpoint{3.531204in}{2.391926in}}%
\pgfpathlineto{\pgfqpoint{3.539195in}{2.401796in}}%
\pgfpathlineto{\pgfqpoint{3.547179in}{2.411728in}}%
\pgfpathlineto{\pgfqpoint{3.555158in}{2.421726in}}%
\pgfpathlineto{\pgfqpoint{3.541738in}{2.427273in}}%
\pgfpathlineto{\pgfqpoint{3.528322in}{2.432913in}}%
\pgfpathlineto{\pgfqpoint{3.514909in}{2.438646in}}%
\pgfpathlineto{\pgfqpoint{3.501501in}{2.444473in}}%
\pgfpathlineto{\pgfqpoint{3.493513in}{2.434351in}}%
\pgfpathlineto{\pgfqpoint{3.485518in}{2.424299in}}%
\pgfpathlineto{\pgfqpoint{3.477518in}{2.414314in}}%
\pgfpathlineto{\pgfqpoint{3.469512in}{2.404395in}}%
\pgfpathclose%
\pgfusepath{fill}%
\end{pgfscope}%
\begin{pgfscope}%
\pgfpathrectangle{\pgfqpoint{1.150000in}{0.150000in}}{\pgfqpoint{5.700000in}{5.700000in}}%
\pgfusepath{clip}%
\pgfsetbuttcap%
\pgfsetroundjoin%
\definecolor{currentfill}{rgb}{0.185556,0.418570,0.556753}%
\pgfsetfillcolor{currentfill}%
\pgfsetfillopacity{0.700000}%
\pgfsetlinewidth{0.000000pt}%
\definecolor{currentstroke}{rgb}{0.000000,0.000000,0.000000}%
\pgfsetstrokecolor{currentstroke}%
\pgfsetdash{}{0pt}%
\pgfpathmoveto{\pgfqpoint{5.702725in}{3.118447in}}%
\pgfpathlineto{\pgfqpoint{5.716574in}{3.113460in}}%
\pgfpathlineto{\pgfqpoint{5.730432in}{3.108538in}}%
\pgfpathlineto{\pgfqpoint{5.744297in}{3.103681in}}%
\pgfpathlineto{\pgfqpoint{5.758169in}{3.098890in}}%
\pgfpathlineto{\pgfqpoint{5.765764in}{3.119029in}}%
\pgfpathlineto{\pgfqpoint{5.773373in}{3.139675in}}%
\pgfpathlineto{\pgfqpoint{5.780997in}{3.160838in}}%
\pgfpathlineto{\pgfqpoint{5.767139in}{3.166063in}}%
\pgfpathlineto{\pgfqpoint{5.753289in}{3.171353in}}%
\pgfpathlineto{\pgfqpoint{5.739445in}{3.176709in}}%
\pgfpathlineto{\pgfqpoint{5.725610in}{3.182130in}}%
\pgfpathlineto{\pgfqpoint{5.717967in}{3.160384in}}%
\pgfpathlineto{\pgfqpoint{5.710339in}{3.139160in}}%
\pgfpathlineto{\pgfqpoint{5.702725in}{3.118447in}}%
\pgfpathclose%
\pgfusepath{fill}%
\end{pgfscope}%
\begin{pgfscope}%
\pgfpathrectangle{\pgfqpoint{1.150000in}{0.150000in}}{\pgfqpoint{5.700000in}{5.700000in}}%
\pgfusepath{clip}%
\pgfsetbuttcap%
\pgfsetroundjoin%
\definecolor{currentfill}{rgb}{0.283072,0.130895,0.449241}%
\pgfsetfillcolor{currentfill}%
\pgfsetfillopacity{0.700000}%
\pgfsetlinewidth{0.000000pt}%
\definecolor{currentstroke}{rgb}{0.000000,0.000000,0.000000}%
\pgfsetstrokecolor{currentstroke}%
\pgfsetdash{}{0pt}%
\pgfpathmoveto{\pgfqpoint{2.911373in}{2.480465in}}%
\pgfpathlineto{\pgfqpoint{2.924756in}{2.471671in}}%
\pgfpathlineto{\pgfqpoint{2.938140in}{2.462994in}}%
\pgfpathlineto{\pgfqpoint{2.951525in}{2.454432in}}%
\pgfpathlineto{\pgfqpoint{2.964911in}{2.445986in}}%
\pgfpathlineto{\pgfqpoint{2.973099in}{2.455319in}}%
\pgfpathlineto{\pgfqpoint{2.981280in}{2.464726in}}%
\pgfpathlineto{\pgfqpoint{2.989454in}{2.474208in}}%
\pgfpathlineto{\pgfqpoint{2.997621in}{2.483765in}}%
\pgfpathlineto{\pgfqpoint{2.984247in}{2.492248in}}%
\pgfpathlineto{\pgfqpoint{2.970875in}{2.500845in}}%
\pgfpathlineto{\pgfqpoint{2.957504in}{2.509558in}}%
\pgfpathlineto{\pgfqpoint{2.944133in}{2.518387in}}%
\pgfpathlineto{\pgfqpoint{2.935954in}{2.508787in}}%
\pgfpathlineto{\pgfqpoint{2.927768in}{2.499267in}}%
\pgfpathlineto{\pgfqpoint{2.919574in}{2.489827in}}%
\pgfpathlineto{\pgfqpoint{2.911373in}{2.480465in}}%
\pgfpathclose%
\pgfusepath{fill}%
\end{pgfscope}%
\begin{pgfscope}%
\pgfpathrectangle{\pgfqpoint{1.150000in}{0.150000in}}{\pgfqpoint{5.700000in}{5.700000in}}%
\pgfusepath{clip}%
\pgfsetbuttcap%
\pgfsetroundjoin%
\definecolor{currentfill}{rgb}{0.235526,0.309527,0.542944}%
\pgfsetfillcolor{currentfill}%
\pgfsetfillopacity{0.700000}%
\pgfsetlinewidth{0.000000pt}%
\definecolor{currentstroke}{rgb}{0.000000,0.000000,0.000000}%
\pgfsetstrokecolor{currentstroke}%
\pgfsetdash{}{0pt}%
\pgfpathmoveto{\pgfqpoint{5.330565in}{2.842937in}}%
\pgfpathlineto{\pgfqpoint{5.344383in}{2.839487in}}%
\pgfpathlineto{\pgfqpoint{5.358208in}{2.836104in}}%
\pgfpathlineto{\pgfqpoint{5.372042in}{2.832789in}}%
\pgfpathlineto{\pgfqpoint{5.385884in}{2.829540in}}%
\pgfpathlineto{\pgfqpoint{5.393377in}{2.843562in}}%
\pgfpathlineto{\pgfqpoint{5.400876in}{2.857916in}}%
\pgfpathlineto{\pgfqpoint{5.408379in}{2.872610in}}%
\pgfpathlineto{\pgfqpoint{5.415888in}{2.887653in}}%
\pgfpathlineto{\pgfqpoint{5.402065in}{2.891380in}}%
\pgfpathlineto{\pgfqpoint{5.388250in}{2.895175in}}%
\pgfpathlineto{\pgfqpoint{5.374442in}{2.899037in}}%
\pgfpathlineto{\pgfqpoint{5.360642in}{2.902966in}}%
\pgfpathlineto{\pgfqpoint{5.353115in}{2.887436in}}%
\pgfpathlineto{\pgfqpoint{5.345594in}{2.872261in}}%
\pgfpathlineto{\pgfqpoint{5.338077in}{2.857431in}}%
\pgfpathlineto{\pgfqpoint{5.330565in}{2.842937in}}%
\pgfpathclose%
\pgfusepath{fill}%
\end{pgfscope}%
\begin{pgfscope}%
\pgfpathrectangle{\pgfqpoint{1.150000in}{0.150000in}}{\pgfqpoint{5.700000in}{5.700000in}}%
\pgfusepath{clip}%
\pgfsetbuttcap%
\pgfsetroundjoin%
\definecolor{currentfill}{rgb}{0.225863,0.330805,0.547314}%
\pgfsetfillcolor{currentfill}%
\pgfsetfillopacity{0.700000}%
\pgfsetlinewidth{0.000000pt}%
\definecolor{currentstroke}{rgb}{0.000000,0.000000,0.000000}%
\pgfsetstrokecolor{currentstroke}%
\pgfsetdash{}{0pt}%
\pgfpathmoveto{\pgfqpoint{5.415888in}{2.887653in}}%
\pgfpathlineto{\pgfqpoint{5.429720in}{2.883992in}}%
\pgfpathlineto{\pgfqpoint{5.443559in}{2.880397in}}%
\pgfpathlineto{\pgfqpoint{5.457406in}{2.876869in}}%
\pgfpathlineto{\pgfqpoint{5.471262in}{2.873408in}}%
\pgfpathlineto{\pgfqpoint{5.478758in}{2.888317in}}%
\pgfpathlineto{\pgfqpoint{5.486260in}{2.903588in}}%
\pgfpathlineto{\pgfqpoint{5.493769in}{2.919231in}}%
\pgfpathlineto{\pgfqpoint{5.501286in}{2.935254in}}%
\pgfpathlineto{\pgfqpoint{5.487449in}{2.939214in}}%
\pgfpathlineto{\pgfqpoint{5.473621in}{2.943241in}}%
\pgfpathlineto{\pgfqpoint{5.459800in}{2.947335in}}%
\pgfpathlineto{\pgfqpoint{5.445987in}{2.951495in}}%
\pgfpathlineto{\pgfqpoint{5.438452in}{2.934966in}}%
\pgfpathlineto{\pgfqpoint{5.430924in}{2.918821in}}%
\pgfpathlineto{\pgfqpoint{5.423403in}{2.903054in}}%
\pgfpathlineto{\pgfqpoint{5.415888in}{2.887653in}}%
\pgfpathclose%
\pgfusepath{fill}%
\end{pgfscope}%
\begin{pgfscope}%
\pgfpathrectangle{\pgfqpoint{1.150000in}{0.150000in}}{\pgfqpoint{5.700000in}{5.700000in}}%
\pgfusepath{clip}%
\pgfsetbuttcap%
\pgfsetroundjoin%
\definecolor{currentfill}{rgb}{0.243113,0.292092,0.538516}%
\pgfsetfillcolor{currentfill}%
\pgfsetfillopacity{0.700000}%
\pgfsetlinewidth{0.000000pt}%
\definecolor{currentstroke}{rgb}{0.000000,0.000000,0.000000}%
\pgfsetstrokecolor{currentstroke}%
\pgfsetdash{}{0pt}%
\pgfpathmoveto{\pgfqpoint{5.245295in}{2.800795in}}%
\pgfpathlineto{\pgfqpoint{5.259098in}{2.797534in}}%
\pgfpathlineto{\pgfqpoint{5.272910in}{2.794340in}}%
\pgfpathlineto{\pgfqpoint{5.286729in}{2.791215in}}%
\pgfpathlineto{\pgfqpoint{5.300557in}{2.788157in}}%
\pgfpathlineto{\pgfqpoint{5.308054in}{2.801390in}}%
\pgfpathlineto{\pgfqpoint{5.315554in}{2.814925in}}%
\pgfpathlineto{\pgfqpoint{5.323057in}{2.828771in}}%
\pgfpathlineto{\pgfqpoint{5.330565in}{2.842937in}}%
\pgfpathlineto{\pgfqpoint{5.316755in}{2.846454in}}%
\pgfpathlineto{\pgfqpoint{5.302954in}{2.850039in}}%
\pgfpathlineto{\pgfqpoint{5.289160in}{2.853691in}}%
\pgfpathlineto{\pgfqpoint{5.275373in}{2.857411in}}%
\pgfpathlineto{\pgfqpoint{5.267848in}{2.842779in}}%
\pgfpathlineto{\pgfqpoint{5.260327in}{2.828471in}}%
\pgfpathlineto{\pgfqpoint{5.252809in}{2.814479in}}%
\pgfpathlineto{\pgfqpoint{5.245295in}{2.800795in}}%
\pgfpathclose%
\pgfusepath{fill}%
\end{pgfscope}%
\begin{pgfscope}%
\pgfpathrectangle{\pgfqpoint{1.150000in}{0.150000in}}{\pgfqpoint{5.700000in}{5.700000in}}%
\pgfusepath{clip}%
\pgfsetbuttcap%
\pgfsetroundjoin%
\definecolor{currentfill}{rgb}{0.281412,0.155834,0.469201}%
\pgfsetfillcolor{currentfill}%
\pgfsetfillopacity{0.700000}%
\pgfsetlinewidth{0.000000pt}%
\definecolor{currentstroke}{rgb}{0.000000,0.000000,0.000000}%
\pgfsetstrokecolor{currentstroke}%
\pgfsetdash{}{0pt}%
\pgfpathmoveto{\pgfqpoint{4.368914in}{2.521190in}}%
\pgfpathlineto{\pgfqpoint{4.382516in}{2.518012in}}%
\pgfpathlineto{\pgfqpoint{4.396125in}{2.514909in}}%
\pgfpathlineto{\pgfqpoint{4.409741in}{2.511882in}}%
\pgfpathlineto{\pgfqpoint{4.423364in}{2.508931in}}%
\pgfpathlineto{\pgfqpoint{4.431069in}{2.519026in}}%
\pgfpathlineto{\pgfqpoint{4.438770in}{2.529240in}}%
\pgfpathlineto{\pgfqpoint{4.446467in}{2.539578in}}%
\pgfpathlineto{\pgfqpoint{4.454160in}{2.550044in}}%
\pgfpathlineto{\pgfqpoint{4.440549in}{2.553275in}}%
\pgfpathlineto{\pgfqpoint{4.426945in}{2.556580in}}%
\pgfpathlineto{\pgfqpoint{4.413348in}{2.559961in}}%
\pgfpathlineto{\pgfqpoint{4.399757in}{2.563418in}}%
\pgfpathlineto{\pgfqpoint{4.392053in}{2.552666in}}%
\pgfpathlineto{\pgfqpoint{4.384344in}{2.542047in}}%
\pgfpathlineto{\pgfqpoint{4.376631in}{2.531557in}}%
\pgfpathlineto{\pgfqpoint{4.368914in}{2.521190in}}%
\pgfpathclose%
\pgfusepath{fill}%
\end{pgfscope}%
\begin{pgfscope}%
\pgfpathrectangle{\pgfqpoint{1.150000in}{0.150000in}}{\pgfqpoint{5.700000in}{5.700000in}}%
\pgfusepath{clip}%
\pgfsetbuttcap%
\pgfsetroundjoin%
\definecolor{currentfill}{rgb}{0.216210,0.351535,0.550627}%
\pgfsetfillcolor{currentfill}%
\pgfsetfillopacity{0.700000}%
\pgfsetlinewidth{0.000000pt}%
\definecolor{currentstroke}{rgb}{0.000000,0.000000,0.000000}%
\pgfsetstrokecolor{currentstroke}%
\pgfsetdash{}{0pt}%
\pgfpathmoveto{\pgfqpoint{5.501286in}{2.935254in}}%
\pgfpathlineto{\pgfqpoint{5.515130in}{2.931359in}}%
\pgfpathlineto{\pgfqpoint{5.528982in}{2.927531in}}%
\pgfpathlineto{\pgfqpoint{5.542843in}{2.923769in}}%
\pgfpathlineto{\pgfqpoint{5.556711in}{2.920073in}}%
\pgfpathlineto{\pgfqpoint{5.564216in}{2.935974in}}%
\pgfpathlineto{\pgfqpoint{5.571729in}{2.952269in}}%
\pgfpathlineto{\pgfqpoint{5.579250in}{2.968967in}}%
\pgfpathlineto{\pgfqpoint{5.586780in}{2.986079in}}%
\pgfpathlineto{\pgfqpoint{5.572931in}{2.990295in}}%
\pgfpathlineto{\pgfqpoint{5.559090in}{2.994576in}}%
\pgfpathlineto{\pgfqpoint{5.545257in}{2.998923in}}%
\pgfpathlineto{\pgfqpoint{5.531431in}{3.003337in}}%
\pgfpathlineto{\pgfqpoint{5.523882in}{2.985698in}}%
\pgfpathlineto{\pgfqpoint{5.516342in}{2.968478in}}%
\pgfpathlineto{\pgfqpoint{5.508810in}{2.951666in}}%
\pgfpathlineto{\pgfqpoint{5.501286in}{2.935254in}}%
\pgfpathclose%
\pgfusepath{fill}%
\end{pgfscope}%
\begin{pgfscope}%
\pgfpathrectangle{\pgfqpoint{1.150000in}{0.150000in}}{\pgfqpoint{5.700000in}{5.700000in}}%
\pgfusepath{clip}%
\pgfsetbuttcap%
\pgfsetroundjoin%
\definecolor{currentfill}{rgb}{0.282910,0.105393,0.426902}%
\pgfsetfillcolor{currentfill}%
\pgfsetfillopacity{0.700000}%
\pgfsetlinewidth{0.000000pt}%
\definecolor{currentstroke}{rgb}{0.000000,0.000000,0.000000}%
\pgfsetstrokecolor{currentstroke}%
\pgfsetdash{}{0pt}%
\pgfpathmoveto{\pgfqpoint{3.833816in}{2.423151in}}%
\pgfpathlineto{\pgfqpoint{3.847298in}{2.418816in}}%
\pgfpathlineto{\pgfqpoint{3.860786in}{2.414566in}}%
\pgfpathlineto{\pgfqpoint{3.874279in}{2.410401in}}%
\pgfpathlineto{\pgfqpoint{3.887778in}{2.406319in}}%
\pgfpathlineto{\pgfqpoint{3.895656in}{2.416178in}}%
\pgfpathlineto{\pgfqpoint{3.903529in}{2.426107in}}%
\pgfpathlineto{\pgfqpoint{3.911397in}{2.436109in}}%
\pgfpathlineto{\pgfqpoint{3.919259in}{2.446187in}}%
\pgfpathlineto{\pgfqpoint{3.905770in}{2.450446in}}%
\pgfpathlineto{\pgfqpoint{3.892287in}{2.454789in}}%
\pgfpathlineto{\pgfqpoint{3.878809in}{2.459217in}}%
\pgfpathlineto{\pgfqpoint{3.865337in}{2.463729in}}%
\pgfpathlineto{\pgfqpoint{3.857465in}{2.453466in}}%
\pgfpathlineto{\pgfqpoint{3.849587in}{2.443285in}}%
\pgfpathlineto{\pgfqpoint{3.841704in}{2.433181in}}%
\pgfpathlineto{\pgfqpoint{3.833816in}{2.423151in}}%
\pgfpathclose%
\pgfusepath{fill}%
\end{pgfscope}%
\begin{pgfscope}%
\pgfpathrectangle{\pgfqpoint{1.150000in}{0.150000in}}{\pgfqpoint{5.700000in}{5.700000in}}%
\pgfusepath{clip}%
\pgfsetbuttcap%
\pgfsetroundjoin%
\definecolor{currentfill}{rgb}{0.250425,0.274290,0.533103}%
\pgfsetfillcolor{currentfill}%
\pgfsetfillopacity{0.700000}%
\pgfsetlinewidth{0.000000pt}%
\definecolor{currentstroke}{rgb}{0.000000,0.000000,0.000000}%
\pgfsetstrokecolor{currentstroke}%
\pgfsetdash{}{0pt}%
\pgfpathmoveto{\pgfqpoint{5.160060in}{2.760937in}}%
\pgfpathlineto{\pgfqpoint{5.173849in}{2.757843in}}%
\pgfpathlineto{\pgfqpoint{5.187646in}{2.754817in}}%
\pgfpathlineto{\pgfqpoint{5.201451in}{2.751860in}}%
\pgfpathlineto{\pgfqpoint{5.215263in}{2.748970in}}%
\pgfpathlineto{\pgfqpoint{5.222768in}{2.761505in}}%
\pgfpathlineto{\pgfqpoint{5.230275in}{2.774316in}}%
\pgfpathlineto{\pgfqpoint{5.237784in}{2.787410in}}%
\pgfpathlineto{\pgfqpoint{5.245295in}{2.800795in}}%
\pgfpathlineto{\pgfqpoint{5.231500in}{2.804124in}}%
\pgfpathlineto{\pgfqpoint{5.217712in}{2.807520in}}%
\pgfpathlineto{\pgfqpoint{5.203932in}{2.810985in}}%
\pgfpathlineto{\pgfqpoint{5.190160in}{2.814518in}}%
\pgfpathlineto{\pgfqpoint{5.182632in}{2.800687in}}%
\pgfpathlineto{\pgfqpoint{5.175106in}{2.787152in}}%
\pgfpathlineto{\pgfqpoint{5.167582in}{2.773904in}}%
\pgfpathlineto{\pgfqpoint{5.160060in}{2.760937in}}%
\pgfpathclose%
\pgfusepath{fill}%
\end{pgfscope}%
\begin{pgfscope}%
\pgfpathrectangle{\pgfqpoint{1.150000in}{0.150000in}}{\pgfqpoint{5.700000in}{5.700000in}}%
\pgfusepath{clip}%
\pgfsetbuttcap%
\pgfsetroundjoin%
\definecolor{currentfill}{rgb}{0.275191,0.194905,0.496005}%
\pgfsetfillcolor{currentfill}%
\pgfsetfillopacity{0.700000}%
\pgfsetlinewidth{0.000000pt}%
\definecolor{currentstroke}{rgb}{0.000000,0.000000,0.000000}%
\pgfsetstrokecolor{currentstroke}%
\pgfsetdash{}{0pt}%
\pgfpathmoveto{\pgfqpoint{4.679165in}{2.598697in}}%
\pgfpathlineto{\pgfqpoint{4.692844in}{2.595810in}}%
\pgfpathlineto{\pgfqpoint{4.706531in}{2.592996in}}%
\pgfpathlineto{\pgfqpoint{4.720225in}{2.590253in}}%
\pgfpathlineto{\pgfqpoint{4.733926in}{2.587583in}}%
\pgfpathlineto{\pgfqpoint{4.741538in}{2.598131in}}%
\pgfpathlineto{\pgfqpoint{4.749147in}{2.608844in}}%
\pgfpathlineto{\pgfqpoint{4.756754in}{2.619729in}}%
\pgfpathlineto{\pgfqpoint{4.764358in}{2.630792in}}%
\pgfpathlineto{\pgfqpoint{4.750671in}{2.633802in}}%
\pgfpathlineto{\pgfqpoint{4.736990in}{2.636883in}}%
\pgfpathlineto{\pgfqpoint{4.723317in}{2.640036in}}%
\pgfpathlineto{\pgfqpoint{4.709652in}{2.643262in}}%
\pgfpathlineto{\pgfqpoint{4.702034in}{2.631853in}}%
\pgfpathlineto{\pgfqpoint{4.694413in}{2.620627in}}%
\pgfpathlineto{\pgfqpoint{4.686790in}{2.609577in}}%
\pgfpathlineto{\pgfqpoint{4.679165in}{2.598697in}}%
\pgfpathclose%
\pgfusepath{fill}%
\end{pgfscope}%
\begin{pgfscope}%
\pgfpathrectangle{\pgfqpoint{1.150000in}{0.150000in}}{\pgfqpoint{5.700000in}{5.700000in}}%
\pgfusepath{clip}%
\pgfsetbuttcap%
\pgfsetroundjoin%
\definecolor{currentfill}{rgb}{0.283187,0.125848,0.444960}%
\pgfsetfillcolor{currentfill}%
\pgfsetfillopacity{0.700000}%
\pgfsetlinewidth{0.000000pt}%
\definecolor{currentstroke}{rgb}{0.000000,0.000000,0.000000}%
\pgfsetstrokecolor{currentstroke}%
\pgfsetdash{}{0pt}%
\pgfpathmoveto{\pgfqpoint{4.058692in}{2.454659in}}%
\pgfpathlineto{\pgfqpoint{4.072223in}{2.450942in}}%
\pgfpathlineto{\pgfqpoint{4.085760in}{2.447305in}}%
\pgfpathlineto{\pgfqpoint{4.099304in}{2.443749in}}%
\pgfpathlineto{\pgfqpoint{4.112853in}{2.440273in}}%
\pgfpathlineto{\pgfqpoint{4.120659in}{2.450156in}}%
\pgfpathlineto{\pgfqpoint{4.128460in}{2.460124in}}%
\pgfpathlineto{\pgfqpoint{4.136255in}{2.470179in}}%
\pgfpathlineto{\pgfqpoint{4.144046in}{2.480327in}}%
\pgfpathlineto{\pgfqpoint{4.130507in}{2.484021in}}%
\pgfpathlineto{\pgfqpoint{4.116974in}{2.487796in}}%
\pgfpathlineto{\pgfqpoint{4.103448in}{2.491650in}}%
\pgfpathlineto{\pgfqpoint{4.089927in}{2.495585in}}%
\pgfpathlineto{\pgfqpoint{4.082126in}{2.485212in}}%
\pgfpathlineto{\pgfqpoint{4.074319in}{2.474936in}}%
\pgfpathlineto{\pgfqpoint{4.066508in}{2.464753in}}%
\pgfpathlineto{\pgfqpoint{4.058692in}{2.454659in}}%
\pgfpathclose%
\pgfusepath{fill}%
\end{pgfscope}%
\begin{pgfscope}%
\pgfpathrectangle{\pgfqpoint{1.150000in}{0.150000in}}{\pgfqpoint{5.700000in}{5.700000in}}%
\pgfusepath{clip}%
\pgfsetbuttcap%
\pgfsetroundjoin%
\definecolor{currentfill}{rgb}{0.282327,0.094955,0.417331}%
\pgfsetfillcolor{currentfill}%
\pgfsetfillopacity{0.700000}%
\pgfsetlinewidth{0.000000pt}%
\definecolor{currentstroke}{rgb}{0.000000,0.000000,0.000000}%
\pgfsetstrokecolor{currentstroke}%
\pgfsetdash{}{0pt}%
\pgfpathmoveto{\pgfqpoint{3.608882in}{2.400457in}}%
\pgfpathlineto{\pgfqpoint{3.622324in}{2.395367in}}%
\pgfpathlineto{\pgfqpoint{3.635771in}{2.390367in}}%
\pgfpathlineto{\pgfqpoint{3.649223in}{2.385457in}}%
\pgfpathlineto{\pgfqpoint{3.662679in}{2.380635in}}%
\pgfpathlineto{\pgfqpoint{3.670632in}{2.390441in}}%
\pgfpathlineto{\pgfqpoint{3.678579in}{2.400307in}}%
\pgfpathlineto{\pgfqpoint{3.686521in}{2.410237in}}%
\pgfpathlineto{\pgfqpoint{3.694457in}{2.420232in}}%
\pgfpathlineto{\pgfqpoint{3.681010in}{2.425191in}}%
\pgfpathlineto{\pgfqpoint{3.667569in}{2.430239in}}%
\pgfpathlineto{\pgfqpoint{3.654132in}{2.435376in}}%
\pgfpathlineto{\pgfqpoint{3.640699in}{2.440603in}}%
\pgfpathlineto{\pgfqpoint{3.632754in}{2.430463in}}%
\pgfpathlineto{\pgfqpoint{3.624802in}{2.420394in}}%
\pgfpathlineto{\pgfqpoint{3.616845in}{2.410393in}}%
\pgfpathlineto{\pgfqpoint{3.608882in}{2.400457in}}%
\pgfpathclose%
\pgfusepath{fill}%
\end{pgfscope}%
\begin{pgfscope}%
\pgfpathrectangle{\pgfqpoint{1.150000in}{0.150000in}}{\pgfqpoint{5.700000in}{5.700000in}}%
\pgfusepath{clip}%
\pgfsetbuttcap%
\pgfsetroundjoin%
\definecolor{currentfill}{rgb}{0.206756,0.371758,0.553117}%
\pgfsetfillcolor{currentfill}%
\pgfsetfillopacity{0.700000}%
\pgfsetlinewidth{0.000000pt}%
\definecolor{currentstroke}{rgb}{0.000000,0.000000,0.000000}%
\pgfsetstrokecolor{currentstroke}%
\pgfsetdash{}{0pt}%
\pgfpathmoveto{\pgfqpoint{5.586780in}{2.986079in}}%
\pgfpathlineto{\pgfqpoint{5.600637in}{2.981929in}}%
\pgfpathlineto{\pgfqpoint{5.614502in}{2.977846in}}%
\pgfpathlineto{\pgfqpoint{5.628375in}{2.973827in}}%
\pgfpathlineto{\pgfqpoint{5.642256in}{2.969875in}}%
\pgfpathlineto{\pgfqpoint{5.649776in}{2.986878in}}%
\pgfpathlineto{\pgfqpoint{5.657306in}{3.004308in}}%
\pgfpathlineto{\pgfqpoint{5.664847in}{3.022177in}}%
\pgfpathlineto{\pgfqpoint{5.672398in}{3.040494in}}%
\pgfpathlineto{\pgfqpoint{5.658537in}{3.044986in}}%
\pgfpathlineto{\pgfqpoint{5.644684in}{3.049544in}}%
\pgfpathlineto{\pgfqpoint{5.630838in}{3.054168in}}%
\pgfpathlineto{\pgfqpoint{5.617000in}{3.058857in}}%
\pgfpathlineto{\pgfqpoint{5.609430in}{3.039993in}}%
\pgfpathlineto{\pgfqpoint{5.601870in}{3.021582in}}%
\pgfpathlineto{\pgfqpoint{5.594320in}{3.003614in}}%
\pgfpathlineto{\pgfqpoint{5.586780in}{2.986079in}}%
\pgfpathclose%
\pgfusepath{fill}%
\end{pgfscope}%
\begin{pgfscope}%
\pgfpathrectangle{\pgfqpoint{1.150000in}{0.150000in}}{\pgfqpoint{5.700000in}{5.700000in}}%
\pgfusepath{clip}%
\pgfsetbuttcap%
\pgfsetroundjoin%
\definecolor{currentfill}{rgb}{0.257322,0.256130,0.526563}%
\pgfsetfillcolor{currentfill}%
\pgfsetfillopacity{0.700000}%
\pgfsetlinewidth{0.000000pt}%
\definecolor{currentstroke}{rgb}{0.000000,0.000000,0.000000}%
\pgfsetstrokecolor{currentstroke}%
\pgfsetdash{}{0pt}%
\pgfpathmoveto{\pgfqpoint{5.074845in}{2.723103in}}%
\pgfpathlineto{\pgfqpoint{5.088618in}{2.720153in}}%
\pgfpathlineto{\pgfqpoint{5.102400in}{2.717272in}}%
\pgfpathlineto{\pgfqpoint{5.116190in}{2.714460in}}%
\pgfpathlineto{\pgfqpoint{5.129987in}{2.711717in}}%
\pgfpathlineto{\pgfqpoint{5.137504in}{2.723640in}}%
\pgfpathlineto{\pgfqpoint{5.145021in}{2.735813in}}%
\pgfpathlineto{\pgfqpoint{5.152540in}{2.748243in}}%
\pgfpathlineto{\pgfqpoint{5.160060in}{2.760937in}}%
\pgfpathlineto{\pgfqpoint{5.146279in}{2.764100in}}%
\pgfpathlineto{\pgfqpoint{5.132506in}{2.767331in}}%
\pgfpathlineto{\pgfqpoint{5.118741in}{2.770631in}}%
\pgfpathlineto{\pgfqpoint{5.104984in}{2.774000in}}%
\pgfpathlineto{\pgfqpoint{5.097447in}{2.760879in}}%
\pgfpathlineto{\pgfqpoint{5.089912in}{2.748028in}}%
\pgfpathlineto{\pgfqpoint{5.082378in}{2.735438in}}%
\pgfpathlineto{\pgfqpoint{5.074845in}{2.723103in}}%
\pgfpathclose%
\pgfusepath{fill}%
\end{pgfscope}%
\begin{pgfscope}%
\pgfpathrectangle{\pgfqpoint{1.150000in}{0.150000in}}{\pgfqpoint{5.700000in}{5.700000in}}%
\pgfusepath{clip}%
\pgfsetbuttcap%
\pgfsetroundjoin%
\definecolor{currentfill}{rgb}{0.281887,0.150881,0.465405}%
\pgfsetfillcolor{currentfill}%
\pgfsetfillopacity{0.700000}%
\pgfsetlinewidth{0.000000pt}%
\definecolor{currentstroke}{rgb}{0.000000,0.000000,0.000000}%
\pgfsetstrokecolor{currentstroke}%
\pgfsetdash{}{0pt}%
\pgfpathmoveto{\pgfqpoint{2.771342in}{2.518329in}}%
\pgfpathlineto{\pgfqpoint{2.784736in}{2.508582in}}%
\pgfpathlineto{\pgfqpoint{2.798129in}{2.498960in}}%
\pgfpathlineto{\pgfqpoint{2.811523in}{2.489462in}}%
\pgfpathlineto{\pgfqpoint{2.824916in}{2.480086in}}%
\pgfpathlineto{\pgfqpoint{2.833161in}{2.489150in}}%
\pgfpathlineto{\pgfqpoint{2.841397in}{2.498294in}}%
\pgfpathlineto{\pgfqpoint{2.849626in}{2.507519in}}%
\pgfpathlineto{\pgfqpoint{2.857848in}{2.516826in}}%
\pgfpathlineto{\pgfqpoint{2.844468in}{2.526217in}}%
\pgfpathlineto{\pgfqpoint{2.831088in}{2.535731in}}%
\pgfpathlineto{\pgfqpoint{2.817708in}{2.545367in}}%
\pgfpathlineto{\pgfqpoint{2.804328in}{2.555129in}}%
\pgfpathlineto{\pgfqpoint{2.796093in}{2.545799in}}%
\pgfpathlineto{\pgfqpoint{2.787851in}{2.536557in}}%
\pgfpathlineto{\pgfqpoint{2.779600in}{2.527400in}}%
\pgfpathlineto{\pgfqpoint{2.771342in}{2.518329in}}%
\pgfpathclose%
\pgfusepath{fill}%
\end{pgfscope}%
\begin{pgfscope}%
\pgfpathrectangle{\pgfqpoint{1.150000in}{0.150000in}}{\pgfqpoint{5.700000in}{5.700000in}}%
\pgfusepath{clip}%
\pgfsetbuttcap%
\pgfsetroundjoin%
\definecolor{currentfill}{rgb}{0.282327,0.094955,0.417331}%
\pgfsetfillcolor{currentfill}%
\pgfsetfillopacity{0.700000}%
\pgfsetlinewidth{0.000000pt}%
\definecolor{currentstroke}{rgb}{0.000000,0.000000,0.000000}%
\pgfsetstrokecolor{currentstroke}%
\pgfsetdash{}{0pt}%
\pgfpathmoveto{\pgfqpoint{3.244256in}{2.401105in}}%
\pgfpathlineto{\pgfqpoint{3.257653in}{2.394396in}}%
\pgfpathlineto{\pgfqpoint{3.271054in}{2.387788in}}%
\pgfpathlineto{\pgfqpoint{3.284458in}{2.381280in}}%
\pgfpathlineto{\pgfqpoint{3.297864in}{2.374872in}}%
\pgfpathlineto{\pgfqpoint{3.305941in}{2.384484in}}%
\pgfpathlineto{\pgfqpoint{3.314011in}{2.394155in}}%
\pgfpathlineto{\pgfqpoint{3.322075in}{2.403887in}}%
\pgfpathlineto{\pgfqpoint{3.330133in}{2.413682in}}%
\pgfpathlineto{\pgfqpoint{3.316737in}{2.420166in}}%
\pgfpathlineto{\pgfqpoint{3.303344in}{2.426750in}}%
\pgfpathlineto{\pgfqpoint{3.289954in}{2.433435in}}%
\pgfpathlineto{\pgfqpoint{3.276567in}{2.440221in}}%
\pgfpathlineto{\pgfqpoint{3.268499in}{2.430342in}}%
\pgfpathlineto{\pgfqpoint{3.260424in}{2.420531in}}%
\pgfpathlineto{\pgfqpoint{3.252343in}{2.410786in}}%
\pgfpathlineto{\pgfqpoint{3.244256in}{2.401105in}}%
\pgfpathclose%
\pgfusepath{fill}%
\end{pgfscope}%
\begin{pgfscope}%
\pgfpathrectangle{\pgfqpoint{1.150000in}{0.150000in}}{\pgfqpoint{5.700000in}{5.700000in}}%
\pgfusepath{clip}%
\pgfsetbuttcap%
\pgfsetroundjoin%
\definecolor{currentfill}{rgb}{0.282910,0.105393,0.426902}%
\pgfsetfillcolor{currentfill}%
\pgfsetfillopacity{0.700000}%
\pgfsetlinewidth{0.000000pt}%
\definecolor{currentstroke}{rgb}{0.000000,0.000000,0.000000}%
\pgfsetstrokecolor{currentstroke}%
\pgfsetdash{}{0pt}%
\pgfpathmoveto{\pgfqpoint{3.104665in}{2.419937in}}%
\pgfpathlineto{\pgfqpoint{3.118053in}{2.412451in}}%
\pgfpathlineto{\pgfqpoint{3.131444in}{2.405071in}}%
\pgfpathlineto{\pgfqpoint{3.144838in}{2.397797in}}%
\pgfpathlineto{\pgfqpoint{3.158233in}{2.390628in}}%
\pgfpathlineto{\pgfqpoint{3.166358in}{2.400117in}}%
\pgfpathlineto{\pgfqpoint{3.174476in}{2.409669in}}%
\pgfpathlineto{\pgfqpoint{3.182588in}{2.419285in}}%
\pgfpathlineto{\pgfqpoint{3.190693in}{2.428967in}}%
\pgfpathlineto{\pgfqpoint{3.177308in}{2.436192in}}%
\pgfpathlineto{\pgfqpoint{3.163926in}{2.443522in}}%
\pgfpathlineto{\pgfqpoint{3.150547in}{2.450959in}}%
\pgfpathlineto{\pgfqpoint{3.137169in}{2.458501in}}%
\pgfpathlineto{\pgfqpoint{3.129053in}{2.448755in}}%
\pgfpathlineto{\pgfqpoint{3.120931in}{2.439080in}}%
\pgfpathlineto{\pgfqpoint{3.112801in}{2.429475in}}%
\pgfpathlineto{\pgfqpoint{3.104665in}{2.419937in}}%
\pgfpathclose%
\pgfusepath{fill}%
\end{pgfscope}%
\begin{pgfscope}%
\pgfpathrectangle{\pgfqpoint{1.150000in}{0.150000in}}{\pgfqpoint{5.700000in}{5.700000in}}%
\pgfusepath{clip}%
\pgfsetbuttcap%
\pgfsetroundjoin%
\definecolor{currentfill}{rgb}{0.282290,0.145912,0.461510}%
\pgfsetfillcolor{currentfill}%
\pgfsetfillopacity{0.700000}%
\pgfsetlinewidth{0.000000pt}%
\definecolor{currentstroke}{rgb}{0.000000,0.000000,0.000000}%
\pgfsetstrokecolor{currentstroke}%
\pgfsetdash{}{0pt}%
\pgfpathmoveto{\pgfqpoint{4.283618in}{2.493306in}}%
\pgfpathlineto{\pgfqpoint{4.297204in}{2.490079in}}%
\pgfpathlineto{\pgfqpoint{4.310797in}{2.486930in}}%
\pgfpathlineto{\pgfqpoint{4.324397in}{2.483857in}}%
\pgfpathlineto{\pgfqpoint{4.338004in}{2.480861in}}%
\pgfpathlineto{\pgfqpoint{4.345738in}{2.490782in}}%
\pgfpathlineto{\pgfqpoint{4.353468in}{2.500808in}}%
\pgfpathlineto{\pgfqpoint{4.361193in}{2.510942in}}%
\pgfpathlineto{\pgfqpoint{4.368914in}{2.521190in}}%
\pgfpathlineto{\pgfqpoint{4.355319in}{2.524445in}}%
\pgfpathlineto{\pgfqpoint{4.341731in}{2.527777in}}%
\pgfpathlineto{\pgfqpoint{4.328149in}{2.531185in}}%
\pgfpathlineto{\pgfqpoint{4.314574in}{2.534670in}}%
\pgfpathlineto{\pgfqpoint{4.306841in}{2.524156in}}%
\pgfpathlineto{\pgfqpoint{4.299105in}{2.513761in}}%
\pgfpathlineto{\pgfqpoint{4.291363in}{2.503479in}}%
\pgfpathlineto{\pgfqpoint{4.283618in}{2.493306in}}%
\pgfpathclose%
\pgfusepath{fill}%
\end{pgfscope}%
\begin{pgfscope}%
\pgfpathrectangle{\pgfqpoint{1.150000in}{0.150000in}}{\pgfqpoint{5.700000in}{5.700000in}}%
\pgfusepath{clip}%
\pgfsetbuttcap%
\pgfsetroundjoin%
\definecolor{currentfill}{rgb}{0.195860,0.395433,0.555276}%
\pgfsetfillcolor{currentfill}%
\pgfsetfillopacity{0.700000}%
\pgfsetlinewidth{0.000000pt}%
\definecolor{currentstroke}{rgb}{0.000000,0.000000,0.000000}%
\pgfsetstrokecolor{currentstroke}%
\pgfsetdash{}{0pt}%
\pgfpathmoveto{\pgfqpoint{5.672398in}{3.040494in}}%
\pgfpathlineto{\pgfqpoint{5.686267in}{3.036067in}}%
\pgfpathlineto{\pgfqpoint{5.700144in}{3.031705in}}%
\pgfpathlineto{\pgfqpoint{5.714029in}{3.027409in}}%
\pgfpathlineto{\pgfqpoint{5.727922in}{3.023177in}}%
\pgfpathlineto{\pgfqpoint{5.735465in}{3.041400in}}%
\pgfpathlineto{\pgfqpoint{5.743020in}{3.060086in}}%
\pgfpathlineto{\pgfqpoint{5.750588in}{3.079245in}}%
\pgfpathlineto{\pgfqpoint{5.758169in}{3.098890in}}%
\pgfpathlineto{\pgfqpoint{5.744297in}{3.103681in}}%
\pgfpathlineto{\pgfqpoint{5.730432in}{3.108538in}}%
\pgfpathlineto{\pgfqpoint{5.716574in}{3.113460in}}%
\pgfpathlineto{\pgfqpoint{5.702725in}{3.118447in}}%
\pgfpathlineto{\pgfqpoint{5.695124in}{3.098235in}}%
\pgfpathlineto{\pgfqpoint{5.687537in}{3.078512in}}%
\pgfpathlineto{\pgfqpoint{5.679962in}{3.059269in}}%
\pgfpathlineto{\pgfqpoint{5.672398in}{3.040494in}}%
\pgfpathclose%
\pgfusepath{fill}%
\end{pgfscope}%
\begin{pgfscope}%
\pgfpathrectangle{\pgfqpoint{1.150000in}{0.150000in}}{\pgfqpoint{5.700000in}{5.700000in}}%
\pgfusepath{clip}%
\pgfsetbuttcap%
\pgfsetroundjoin%
\definecolor{currentfill}{rgb}{0.277134,0.185228,0.489898}%
\pgfsetfillcolor{currentfill}%
\pgfsetfillopacity{0.700000}%
\pgfsetlinewidth{0.000000pt}%
\definecolor{currentstroke}{rgb}{0.000000,0.000000,0.000000}%
\pgfsetstrokecolor{currentstroke}%
\pgfsetdash{}{0pt}%
\pgfpathmoveto{\pgfqpoint{4.593939in}{2.567765in}}%
\pgfpathlineto{\pgfqpoint{4.607602in}{2.564905in}}%
\pgfpathlineto{\pgfqpoint{4.621272in}{2.562118in}}%
\pgfpathlineto{\pgfqpoint{4.634949in}{2.559405in}}%
\pgfpathlineto{\pgfqpoint{4.648634in}{2.556765in}}%
\pgfpathlineto{\pgfqpoint{4.656272in}{2.567022in}}%
\pgfpathlineto{\pgfqpoint{4.663906in}{2.577426in}}%
\pgfpathlineto{\pgfqpoint{4.671537in}{2.587982in}}%
\pgfpathlineto{\pgfqpoint{4.679165in}{2.598697in}}%
\pgfpathlineto{\pgfqpoint{4.665493in}{2.601657in}}%
\pgfpathlineto{\pgfqpoint{4.651829in}{2.604689in}}%
\pgfpathlineto{\pgfqpoint{4.638172in}{2.607795in}}%
\pgfpathlineto{\pgfqpoint{4.624522in}{2.610973in}}%
\pgfpathlineto{\pgfqpoint{4.616881in}{2.599932in}}%
\pgfpathlineto{\pgfqpoint{4.609237in}{2.589054in}}%
\pgfpathlineto{\pgfqpoint{4.601589in}{2.578333in}}%
\pgfpathlineto{\pgfqpoint{4.593939in}{2.567765in}}%
\pgfpathclose%
\pgfusepath{fill}%
\end{pgfscope}%
\begin{pgfscope}%
\pgfpathrectangle{\pgfqpoint{1.150000in}{0.150000in}}{\pgfqpoint{5.700000in}{5.700000in}}%
\pgfusepath{clip}%
\pgfsetbuttcap%
\pgfsetroundjoin%
\definecolor{currentfill}{rgb}{0.262138,0.242286,0.520837}%
\pgfsetfillcolor{currentfill}%
\pgfsetfillopacity{0.700000}%
\pgfsetlinewidth{0.000000pt}%
\definecolor{currentstroke}{rgb}{0.000000,0.000000,0.000000}%
\pgfsetstrokecolor{currentstroke}%
\pgfsetdash{}{0pt}%
\pgfpathmoveto{\pgfqpoint{4.989635in}{2.687052in}}%
\pgfpathlineto{\pgfqpoint{5.003393in}{2.684225in}}%
\pgfpathlineto{\pgfqpoint{5.017159in}{2.681466in}}%
\pgfpathlineto{\pgfqpoint{5.030933in}{2.678778in}}%
\pgfpathlineto{\pgfqpoint{5.044715in}{2.676158in}}%
\pgfpathlineto{\pgfqpoint{5.052247in}{2.687549in}}%
\pgfpathlineto{\pgfqpoint{5.059780in}{2.699166in}}%
\pgfpathlineto{\pgfqpoint{5.067312in}{2.711015in}}%
\pgfpathlineto{\pgfqpoint{5.074845in}{2.723103in}}%
\pgfpathlineto{\pgfqpoint{5.061079in}{2.726122in}}%
\pgfpathlineto{\pgfqpoint{5.047321in}{2.729210in}}%
\pgfpathlineto{\pgfqpoint{5.033571in}{2.732367in}}%
\pgfpathlineto{\pgfqpoint{5.019829in}{2.735594in}}%
\pgfpathlineto{\pgfqpoint{5.012280in}{2.723099in}}%
\pgfpathlineto{\pgfqpoint{5.004732in}{2.710849in}}%
\pgfpathlineto{\pgfqpoint{4.997183in}{2.698836in}}%
\pgfpathlineto{\pgfqpoint{4.989635in}{2.687052in}}%
\pgfpathclose%
\pgfusepath{fill}%
\end{pgfscope}%
\begin{pgfscope}%
\pgfpathrectangle{\pgfqpoint{1.150000in}{0.150000in}}{\pgfqpoint{5.700000in}{5.700000in}}%
\pgfusepath{clip}%
\pgfsetbuttcap%
\pgfsetroundjoin%
\definecolor{currentfill}{rgb}{0.275191,0.194905,0.496005}%
\pgfsetfillcolor{currentfill}%
\pgfsetfillopacity{0.700000}%
\pgfsetlinewidth{0.000000pt}%
\definecolor{currentstroke}{rgb}{0.000000,0.000000,0.000000}%
\pgfsetstrokecolor{currentstroke}%
\pgfsetdash{}{0pt}%
\pgfpathmoveto{\pgfqpoint{2.577276in}{2.610018in}}%
\pgfpathlineto{\pgfqpoint{2.590696in}{2.598685in}}%
\pgfpathlineto{\pgfqpoint{2.604114in}{2.587490in}}%
\pgfpathlineto{\pgfqpoint{2.617530in}{2.576433in}}%
\pgfpathlineto{\pgfqpoint{2.630945in}{2.565511in}}%
\pgfpathlineto{\pgfqpoint{2.639266in}{2.574220in}}%
\pgfpathlineto{\pgfqpoint{2.647578in}{2.583023in}}%
\pgfpathlineto{\pgfqpoint{2.655881in}{2.591919in}}%
\pgfpathlineto{\pgfqpoint{2.664176in}{2.600909in}}%
\pgfpathlineto{\pgfqpoint{2.650776in}{2.611825in}}%
\pgfpathlineto{\pgfqpoint{2.637375in}{2.622876in}}%
\pgfpathlineto{\pgfqpoint{2.623972in}{2.634065in}}%
\pgfpathlineto{\pgfqpoint{2.610568in}{2.645391in}}%
\pgfpathlineto{\pgfqpoint{2.602258in}{2.636399in}}%
\pgfpathlineto{\pgfqpoint{2.593939in}{2.627507in}}%
\pgfpathlineto{\pgfqpoint{2.585612in}{2.618713in}}%
\pgfpathlineto{\pgfqpoint{2.577276in}{2.610018in}}%
\pgfpathclose%
\pgfusepath{fill}%
\end{pgfscope}%
\begin{pgfscope}%
\pgfpathrectangle{\pgfqpoint{1.150000in}{0.150000in}}{\pgfqpoint{5.700000in}{5.700000in}}%
\pgfusepath{clip}%
\pgfsetbuttcap%
\pgfsetroundjoin%
\definecolor{currentfill}{rgb}{0.281924,0.089666,0.412415}%
\pgfsetfillcolor{currentfill}%
\pgfsetfillopacity{0.700000}%
\pgfsetlinewidth{0.000000pt}%
\definecolor{currentstroke}{rgb}{0.000000,0.000000,0.000000}%
\pgfsetstrokecolor{currentstroke}%
\pgfsetdash{}{0pt}%
\pgfpathmoveto{\pgfqpoint{3.383751in}{2.388729in}}%
\pgfpathlineto{\pgfqpoint{3.397164in}{2.382734in}}%
\pgfpathlineto{\pgfqpoint{3.410581in}{2.376836in}}%
\pgfpathlineto{\pgfqpoint{3.424001in}{2.371033in}}%
\pgfpathlineto{\pgfqpoint{3.437426in}{2.365325in}}%
\pgfpathlineto{\pgfqpoint{3.445456in}{2.375006in}}%
\pgfpathlineto{\pgfqpoint{3.453481in}{2.384743in}}%
\pgfpathlineto{\pgfqpoint{3.461499in}{2.394538in}}%
\pgfpathlineto{\pgfqpoint{3.469512in}{2.404395in}}%
\pgfpathlineto{\pgfqpoint{3.456097in}{2.410199in}}%
\pgfpathlineto{\pgfqpoint{3.442687in}{2.416099in}}%
\pgfpathlineto{\pgfqpoint{3.429281in}{2.422095in}}%
\pgfpathlineto{\pgfqpoint{3.415878in}{2.428186in}}%
\pgfpathlineto{\pgfqpoint{3.407855in}{2.418225in}}%
\pgfpathlineto{\pgfqpoint{3.399826in}{2.408330in}}%
\pgfpathlineto{\pgfqpoint{3.391792in}{2.398499in}}%
\pgfpathlineto{\pgfqpoint{3.383751in}{2.388729in}}%
\pgfpathclose%
\pgfusepath{fill}%
\end{pgfscope}%
\begin{pgfscope}%
\pgfpathrectangle{\pgfqpoint{1.150000in}{0.150000in}}{\pgfqpoint{5.700000in}{5.700000in}}%
\pgfusepath{clip}%
\pgfsetbuttcap%
\pgfsetroundjoin%
\definecolor{currentfill}{rgb}{0.283197,0.115680,0.436115}%
\pgfsetfillcolor{currentfill}%
\pgfsetfillopacity{0.700000}%
\pgfsetlinewidth{0.000000pt}%
\definecolor{currentstroke}{rgb}{0.000000,0.000000,0.000000}%
\pgfsetstrokecolor{currentstroke}%
\pgfsetdash{}{0pt}%
\pgfpathmoveto{\pgfqpoint{2.964911in}{2.445986in}}%
\pgfpathlineto{\pgfqpoint{2.978298in}{2.437653in}}%
\pgfpathlineto{\pgfqpoint{2.991686in}{2.429433in}}%
\pgfpathlineto{\pgfqpoint{3.005076in}{2.421326in}}%
\pgfpathlineto{\pgfqpoint{3.018468in}{2.413330in}}%
\pgfpathlineto{\pgfqpoint{3.026644in}{2.422635in}}%
\pgfpathlineto{\pgfqpoint{3.034813in}{2.432009in}}%
\pgfpathlineto{\pgfqpoint{3.042974in}{2.441453in}}%
\pgfpathlineto{\pgfqpoint{3.051129in}{2.450967in}}%
\pgfpathlineto{\pgfqpoint{3.037750in}{2.458998in}}%
\pgfpathlineto{\pgfqpoint{3.024372in}{2.467141in}}%
\pgfpathlineto{\pgfqpoint{3.010996in}{2.475397in}}%
\pgfpathlineto{\pgfqpoint{2.997621in}{2.483765in}}%
\pgfpathlineto{\pgfqpoint{2.989454in}{2.474208in}}%
\pgfpathlineto{\pgfqpoint{2.981280in}{2.464726in}}%
\pgfpathlineto{\pgfqpoint{2.973099in}{2.455319in}}%
\pgfpathlineto{\pgfqpoint{2.964911in}{2.445986in}}%
\pgfpathclose%
\pgfusepath{fill}%
\end{pgfscope}%
\begin{pgfscope}%
\pgfpathrectangle{\pgfqpoint{1.150000in}{0.150000in}}{\pgfqpoint{5.700000in}{5.700000in}}%
\pgfusepath{clip}%
\pgfsetbuttcap%
\pgfsetroundjoin%
\definecolor{currentfill}{rgb}{0.282656,0.100196,0.422160}%
\pgfsetfillcolor{currentfill}%
\pgfsetfillopacity{0.700000}%
\pgfsetlinewidth{0.000000pt}%
\definecolor{currentstroke}{rgb}{0.000000,0.000000,0.000000}%
\pgfsetstrokecolor{currentstroke}%
\pgfsetdash{}{0pt}%
\pgfpathmoveto{\pgfqpoint{3.748291in}{2.401278in}}%
\pgfpathlineto{\pgfqpoint{3.761762in}{2.396757in}}%
\pgfpathlineto{\pgfqpoint{3.775238in}{2.392323in}}%
\pgfpathlineto{\pgfqpoint{3.788720in}{2.387976in}}%
\pgfpathlineto{\pgfqpoint{3.802207in}{2.383713in}}%
\pgfpathlineto{\pgfqpoint{3.810117in}{2.393477in}}%
\pgfpathlineto{\pgfqpoint{3.818022in}{2.403302in}}%
\pgfpathlineto{\pgfqpoint{3.825922in}{2.413193in}}%
\pgfpathlineto{\pgfqpoint{3.833816in}{2.423151in}}%
\pgfpathlineto{\pgfqpoint{3.820339in}{2.427571in}}%
\pgfpathlineto{\pgfqpoint{3.806867in}{2.432077in}}%
\pgfpathlineto{\pgfqpoint{3.793401in}{2.436668in}}%
\pgfpathlineto{\pgfqpoint{3.779939in}{2.441346in}}%
\pgfpathlineto{\pgfqpoint{3.772036in}{2.431223in}}%
\pgfpathlineto{\pgfqpoint{3.764126in}{2.421172in}}%
\pgfpathlineto{\pgfqpoint{3.756211in}{2.411192in}}%
\pgfpathlineto{\pgfqpoint{3.748291in}{2.401278in}}%
\pgfpathclose%
\pgfusepath{fill}%
\end{pgfscope}%
\begin{pgfscope}%
\pgfpathrectangle{\pgfqpoint{1.150000in}{0.150000in}}{\pgfqpoint{5.700000in}{5.700000in}}%
\pgfusepath{clip}%
\pgfsetbuttcap%
\pgfsetroundjoin%
\definecolor{currentfill}{rgb}{0.283197,0.115680,0.436115}%
\pgfsetfillcolor{currentfill}%
\pgfsetfillopacity{0.700000}%
\pgfsetlinewidth{0.000000pt}%
\definecolor{currentstroke}{rgb}{0.000000,0.000000,0.000000}%
\pgfsetstrokecolor{currentstroke}%
\pgfsetdash{}{0pt}%
\pgfpathmoveto{\pgfqpoint{3.973271in}{2.429983in}}%
\pgfpathlineto{\pgfqpoint{3.986788in}{2.426138in}}%
\pgfpathlineto{\pgfqpoint{4.000312in}{2.422375in}}%
\pgfpathlineto{\pgfqpoint{4.013841in}{2.418695in}}%
\pgfpathlineto{\pgfqpoint{4.027376in}{2.415095in}}%
\pgfpathlineto{\pgfqpoint{4.035213in}{2.424872in}}%
\pgfpathlineto{\pgfqpoint{4.043044in}{2.434722in}}%
\pgfpathlineto{\pgfqpoint{4.050871in}{2.444650in}}%
\pgfpathlineto{\pgfqpoint{4.058692in}{2.454659in}}%
\pgfpathlineto{\pgfqpoint{4.045167in}{2.458457in}}%
\pgfpathlineto{\pgfqpoint{4.031648in}{2.462336in}}%
\pgfpathlineto{\pgfqpoint{4.018135in}{2.466296in}}%
\pgfpathlineto{\pgfqpoint{4.004628in}{2.470339in}}%
\pgfpathlineto{\pgfqpoint{3.996796in}{2.460125in}}%
\pgfpathlineto{\pgfqpoint{3.988960in}{2.449996in}}%
\pgfpathlineto{\pgfqpoint{3.981118in}{2.439950in}}%
\pgfpathlineto{\pgfqpoint{3.973271in}{2.429983in}}%
\pgfpathclose%
\pgfusepath{fill}%
\end{pgfscope}%
\begin{pgfscope}%
\pgfpathrectangle{\pgfqpoint{1.150000in}{0.150000in}}{\pgfqpoint{5.700000in}{5.700000in}}%
\pgfusepath{clip}%
\pgfsetbuttcap%
\pgfsetroundjoin%
\definecolor{currentfill}{rgb}{0.266580,0.228262,0.514349}%
\pgfsetfillcolor{currentfill}%
\pgfsetfillopacity{0.700000}%
\pgfsetlinewidth{0.000000pt}%
\definecolor{currentstroke}{rgb}{0.000000,0.000000,0.000000}%
\pgfsetstrokecolor{currentstroke}%
\pgfsetdash{}{0pt}%
\pgfpathmoveto{\pgfqpoint{4.904418in}{2.652572in}}%
\pgfpathlineto{\pgfqpoint{4.918160in}{2.649844in}}%
\pgfpathlineto{\pgfqpoint{4.931910in}{2.647185in}}%
\pgfpathlineto{\pgfqpoint{4.945668in}{2.644597in}}%
\pgfpathlineto{\pgfqpoint{4.959434in}{2.642079in}}%
\pgfpathlineto{\pgfqpoint{4.966986in}{2.653012in}}%
\pgfpathlineto{\pgfqpoint{4.974536in}{2.664148in}}%
\pgfpathlineto{\pgfqpoint{4.982086in}{2.675492in}}%
\pgfpathlineto{\pgfqpoint{4.989635in}{2.687052in}}%
\pgfpathlineto{\pgfqpoint{4.975885in}{2.689950in}}%
\pgfpathlineto{\pgfqpoint{4.962142in}{2.692918in}}%
\pgfpathlineto{\pgfqpoint{4.948408in}{2.695955in}}%
\pgfpathlineto{\pgfqpoint{4.934681in}{2.699063in}}%
\pgfpathlineto{\pgfqpoint{4.927116in}{2.687116in}}%
\pgfpathlineto{\pgfqpoint{4.919551in}{2.675390in}}%
\pgfpathlineto{\pgfqpoint{4.911985in}{2.663877in}}%
\pgfpathlineto{\pgfqpoint{4.904418in}{2.652572in}}%
\pgfpathclose%
\pgfusepath{fill}%
\end{pgfscope}%
\begin{pgfscope}%
\pgfpathrectangle{\pgfqpoint{1.150000in}{0.150000in}}{\pgfqpoint{5.700000in}{5.700000in}}%
\pgfusepath{clip}%
\pgfsetbuttcap%
\pgfsetroundjoin%
\definecolor{currentfill}{rgb}{0.279574,0.170599,0.479997}%
\pgfsetfillcolor{currentfill}%
\pgfsetfillopacity{0.700000}%
\pgfsetlinewidth{0.000000pt}%
\definecolor{currentstroke}{rgb}{0.000000,0.000000,0.000000}%
\pgfsetstrokecolor{currentstroke}%
\pgfsetdash{}{0pt}%
\pgfpathmoveto{\pgfqpoint{4.508674in}{2.537874in}}%
\pgfpathlineto{\pgfqpoint{4.522320in}{2.535019in}}%
\pgfpathlineto{\pgfqpoint{4.535974in}{2.532237in}}%
\pgfpathlineto{\pgfqpoint{4.549635in}{2.529529in}}%
\pgfpathlineto{\pgfqpoint{4.563303in}{2.526895in}}%
\pgfpathlineto{\pgfqpoint{4.570967in}{2.536913in}}%
\pgfpathlineto{\pgfqpoint{4.578628in}{2.547060in}}%
\pgfpathlineto{\pgfqpoint{4.586285in}{2.557342in}}%
\pgfpathlineto{\pgfqpoint{4.593939in}{2.567765in}}%
\pgfpathlineto{\pgfqpoint{4.580284in}{2.570698in}}%
\pgfpathlineto{\pgfqpoint{4.566636in}{2.573705in}}%
\pgfpathlineto{\pgfqpoint{4.552995in}{2.576785in}}%
\pgfpathlineto{\pgfqpoint{4.539361in}{2.579940in}}%
\pgfpathlineto{\pgfqpoint{4.531694in}{2.569211in}}%
\pgfpathlineto{\pgfqpoint{4.524024in}{2.558628in}}%
\pgfpathlineto{\pgfqpoint{4.516351in}{2.548184in}}%
\pgfpathlineto{\pgfqpoint{4.508674in}{2.537874in}}%
\pgfpathclose%
\pgfusepath{fill}%
\end{pgfscope}%
\begin{pgfscope}%
\pgfpathrectangle{\pgfqpoint{1.150000in}{0.150000in}}{\pgfqpoint{5.700000in}{5.700000in}}%
\pgfusepath{clip}%
\pgfsetbuttcap%
\pgfsetroundjoin%
\definecolor{currentfill}{rgb}{0.281924,0.089666,0.412415}%
\pgfsetfillcolor{currentfill}%
\pgfsetfillopacity{0.700000}%
\pgfsetlinewidth{0.000000pt}%
\definecolor{currentstroke}{rgb}{0.000000,0.000000,0.000000}%
\pgfsetstrokecolor{currentstroke}%
\pgfsetdash{}{0pt}%
\pgfpathmoveto{\pgfqpoint{3.523208in}{2.382116in}}%
\pgfpathlineto{\pgfqpoint{3.536643in}{2.376778in}}%
\pgfpathlineto{\pgfqpoint{3.550081in}{2.371533in}}%
\pgfpathlineto{\pgfqpoint{3.563525in}{2.366379in}}%
\pgfpathlineto{\pgfqpoint{3.576972in}{2.361316in}}%
\pgfpathlineto{\pgfqpoint{3.584959in}{2.371016in}}%
\pgfpathlineto{\pgfqpoint{3.592939in}{2.380771in}}%
\pgfpathlineto{\pgfqpoint{3.600913in}{2.390584in}}%
\pgfpathlineto{\pgfqpoint{3.608882in}{2.400457in}}%
\pgfpathlineto{\pgfqpoint{3.595445in}{2.405638in}}%
\pgfpathlineto{\pgfqpoint{3.582011in}{2.410909in}}%
\pgfpathlineto{\pgfqpoint{3.568583in}{2.416272in}}%
\pgfpathlineto{\pgfqpoint{3.555158in}{2.421726in}}%
\pgfpathlineto{\pgfqpoint{3.547179in}{2.411728in}}%
\pgfpathlineto{\pgfqpoint{3.539195in}{2.401796in}}%
\pgfpathlineto{\pgfqpoint{3.531204in}{2.391926in}}%
\pgfpathlineto{\pgfqpoint{3.523208in}{2.382116in}}%
\pgfpathclose%
\pgfusepath{fill}%
\end{pgfscope}%
\begin{pgfscope}%
\pgfpathrectangle{\pgfqpoint{1.150000in}{0.150000in}}{\pgfqpoint{5.700000in}{5.700000in}}%
\pgfusepath{clip}%
\pgfsetbuttcap%
\pgfsetroundjoin%
\definecolor{currentfill}{rgb}{0.282884,0.135920,0.453427}%
\pgfsetfillcolor{currentfill}%
\pgfsetfillopacity{0.700000}%
\pgfsetlinewidth{0.000000pt}%
\definecolor{currentstroke}{rgb}{0.000000,0.000000,0.000000}%
\pgfsetstrokecolor{currentstroke}%
\pgfsetdash{}{0pt}%
\pgfpathmoveto{\pgfqpoint{4.198266in}{2.466342in}}%
\pgfpathlineto{\pgfqpoint{4.211837in}{2.463043in}}%
\pgfpathlineto{\pgfqpoint{4.225414in}{2.459822in}}%
\pgfpathlineto{\pgfqpoint{4.238998in}{2.456679in}}%
\pgfpathlineto{\pgfqpoint{4.252589in}{2.453614in}}%
\pgfpathlineto{\pgfqpoint{4.260353in}{2.463396in}}%
\pgfpathlineto{\pgfqpoint{4.268113in}{2.473269in}}%
\pgfpathlineto{\pgfqpoint{4.275868in}{2.483238in}}%
\pgfpathlineto{\pgfqpoint{4.283618in}{2.493306in}}%
\pgfpathlineto{\pgfqpoint{4.270038in}{2.496610in}}%
\pgfpathlineto{\pgfqpoint{4.256465in}{2.499992in}}%
\pgfpathlineto{\pgfqpoint{4.242899in}{2.503452in}}%
\pgfpathlineto{\pgfqpoint{4.229338in}{2.506989in}}%
\pgfpathlineto{\pgfqpoint{4.221577in}{2.496675in}}%
\pgfpathlineto{\pgfqpoint{4.213811in}{2.486465in}}%
\pgfpathlineto{\pgfqpoint{4.206041in}{2.476356in}}%
\pgfpathlineto{\pgfqpoint{4.198266in}{2.466342in}}%
\pgfpathclose%
\pgfusepath{fill}%
\end{pgfscope}%
\begin{pgfscope}%
\pgfpathrectangle{\pgfqpoint{1.150000in}{0.150000in}}{\pgfqpoint{5.700000in}{5.700000in}}%
\pgfusepath{clip}%
\pgfsetbuttcap%
\pgfsetroundjoin%
\definecolor{currentfill}{rgb}{0.185556,0.418570,0.556753}%
\pgfsetfillcolor{currentfill}%
\pgfsetfillopacity{0.700000}%
\pgfsetlinewidth{0.000000pt}%
\definecolor{currentstroke}{rgb}{0.000000,0.000000,0.000000}%
\pgfsetstrokecolor{currentstroke}%
\pgfsetdash{}{0pt}%
\pgfpathmoveto{\pgfqpoint{5.758169in}{3.098890in}}%
\pgfpathlineto{\pgfqpoint{5.772050in}{3.094163in}}%
\pgfpathlineto{\pgfqpoint{5.785938in}{3.089502in}}%
\pgfpathlineto{\pgfqpoint{5.799834in}{3.084905in}}%
\pgfpathlineto{\pgfqpoint{5.813738in}{3.080373in}}%
\pgfpathlineto{\pgfqpoint{5.821312in}{3.099939in}}%
\pgfpathlineto{\pgfqpoint{5.828901in}{3.120007in}}%
\pgfpathlineto{\pgfqpoint{5.836505in}{3.140586in}}%
\pgfpathlineto{\pgfqpoint{5.822617in}{3.145551in}}%
\pgfpathlineto{\pgfqpoint{5.808736in}{3.150582in}}%
\pgfpathlineto{\pgfqpoint{5.794863in}{3.155677in}}%
\pgfpathlineto{\pgfqpoint{5.780997in}{3.160838in}}%
\pgfpathlineto{\pgfqpoint{5.773373in}{3.139675in}}%
\pgfpathlineto{\pgfqpoint{5.765764in}{3.119029in}}%
\pgfpathlineto{\pgfqpoint{5.758169in}{3.098890in}}%
\pgfpathclose%
\pgfusepath{fill}%
\end{pgfscope}%
\begin{pgfscope}%
\pgfpathrectangle{\pgfqpoint{1.150000in}{0.150000in}}{\pgfqpoint{5.700000in}{5.700000in}}%
\pgfusepath{clip}%
\pgfsetbuttcap%
\pgfsetroundjoin%
\definecolor{currentfill}{rgb}{0.282884,0.135920,0.453427}%
\pgfsetfillcolor{currentfill}%
\pgfsetfillopacity{0.700000}%
\pgfsetlinewidth{0.000000pt}%
\definecolor{currentstroke}{rgb}{0.000000,0.000000,0.000000}%
\pgfsetstrokecolor{currentstroke}%
\pgfsetdash{}{0pt}%
\pgfpathmoveto{\pgfqpoint{2.824916in}{2.480086in}}%
\pgfpathlineto{\pgfqpoint{2.838310in}{2.470831in}}%
\pgfpathlineto{\pgfqpoint{2.851705in}{2.461697in}}%
\pgfpathlineto{\pgfqpoint{2.865099in}{2.452682in}}%
\pgfpathlineto{\pgfqpoint{2.878495in}{2.443786in}}%
\pgfpathlineto{\pgfqpoint{2.886726in}{2.452842in}}%
\pgfpathlineto{\pgfqpoint{2.894949in}{2.461974in}}%
\pgfpathlineto{\pgfqpoint{2.903165in}{2.471181in}}%
\pgfpathlineto{\pgfqpoint{2.911373in}{2.480465in}}%
\pgfpathlineto{\pgfqpoint{2.897991in}{2.489377in}}%
\pgfpathlineto{\pgfqpoint{2.884609in}{2.498407in}}%
\pgfpathlineto{\pgfqpoint{2.871228in}{2.507556in}}%
\pgfpathlineto{\pgfqpoint{2.857848in}{2.516826in}}%
\pgfpathlineto{\pgfqpoint{2.849626in}{2.507519in}}%
\pgfpathlineto{\pgfqpoint{2.841397in}{2.498294in}}%
\pgfpathlineto{\pgfqpoint{2.833161in}{2.489150in}}%
\pgfpathlineto{\pgfqpoint{2.824916in}{2.480086in}}%
\pgfpathclose%
\pgfusepath{fill}%
\end{pgfscope}%
\begin{pgfscope}%
\pgfpathrectangle{\pgfqpoint{1.150000in}{0.150000in}}{\pgfqpoint{5.700000in}{5.700000in}}%
\pgfusepath{clip}%
\pgfsetbuttcap%
\pgfsetroundjoin%
\definecolor{currentfill}{rgb}{0.278826,0.175490,0.483397}%
\pgfsetfillcolor{currentfill}%
\pgfsetfillopacity{0.700000}%
\pgfsetlinewidth{0.000000pt}%
\definecolor{currentstroke}{rgb}{0.000000,0.000000,0.000000}%
\pgfsetstrokecolor{currentstroke}%
\pgfsetdash{}{0pt}%
\pgfpathmoveto{\pgfqpoint{2.630945in}{2.565511in}}%
\pgfpathlineto{\pgfqpoint{2.644359in}{2.554724in}}%
\pgfpathlineto{\pgfqpoint{2.657772in}{2.544070in}}%
\pgfpathlineto{\pgfqpoint{2.671183in}{2.533548in}}%
\pgfpathlineto{\pgfqpoint{2.684594in}{2.523158in}}%
\pgfpathlineto{\pgfqpoint{2.692899in}{2.531881in}}%
\pgfpathlineto{\pgfqpoint{2.701196in}{2.540692in}}%
\pgfpathlineto{\pgfqpoint{2.709485in}{2.549591in}}%
\pgfpathlineto{\pgfqpoint{2.717765in}{2.558580in}}%
\pgfpathlineto{\pgfqpoint{2.704369in}{2.568965in}}%
\pgfpathlineto{\pgfqpoint{2.690973in}{2.579480in}}%
\pgfpathlineto{\pgfqpoint{2.677575in}{2.590128in}}%
\pgfpathlineto{\pgfqpoint{2.664176in}{2.600909in}}%
\pgfpathlineto{\pgfqpoint{2.655881in}{2.591919in}}%
\pgfpathlineto{\pgfqpoint{2.647578in}{2.583023in}}%
\pgfpathlineto{\pgfqpoint{2.639266in}{2.574220in}}%
\pgfpathlineto{\pgfqpoint{2.630945in}{2.565511in}}%
\pgfpathclose%
\pgfusepath{fill}%
\end{pgfscope}%
\begin{pgfscope}%
\pgfpathrectangle{\pgfqpoint{1.150000in}{0.150000in}}{\pgfqpoint{5.700000in}{5.700000in}}%
\pgfusepath{clip}%
\pgfsetbuttcap%
\pgfsetroundjoin%
\definecolor{currentfill}{rgb}{0.270595,0.214069,0.507052}%
\pgfsetfillcolor{currentfill}%
\pgfsetfillopacity{0.700000}%
\pgfsetlinewidth{0.000000pt}%
\definecolor{currentstroke}{rgb}{0.000000,0.000000,0.000000}%
\pgfsetstrokecolor{currentstroke}%
\pgfsetdash{}{0pt}%
\pgfpathmoveto{\pgfqpoint{4.819185in}{2.619471in}}%
\pgfpathlineto{\pgfqpoint{4.832911in}{2.616819in}}%
\pgfpathlineto{\pgfqpoint{4.846645in}{2.614239in}}%
\pgfpathlineto{\pgfqpoint{4.860386in}{2.611729in}}%
\pgfpathlineto{\pgfqpoint{4.874135in}{2.609290in}}%
\pgfpathlineto{\pgfqpoint{4.881709in}{2.619832in}}%
\pgfpathlineto{\pgfqpoint{4.889280in}{2.630556in}}%
\pgfpathlineto{\pgfqpoint{4.896850in}{2.641467in}}%
\pgfpathlineto{\pgfqpoint{4.904418in}{2.652572in}}%
\pgfpathlineto{\pgfqpoint{4.890684in}{2.655371in}}%
\pgfpathlineto{\pgfqpoint{4.876957in}{2.658241in}}%
\pgfpathlineto{\pgfqpoint{4.863239in}{2.661181in}}%
\pgfpathlineto{\pgfqpoint{4.849527in}{2.664192in}}%
\pgfpathlineto{\pgfqpoint{4.841944in}{2.652720in}}%
\pgfpathlineto{\pgfqpoint{4.834360in}{2.641448in}}%
\pgfpathlineto{\pgfqpoint{4.826773in}{2.630367in}}%
\pgfpathlineto{\pgfqpoint{4.819185in}{2.619471in}}%
\pgfpathclose%
\pgfusepath{fill}%
\end{pgfscope}%
\begin{pgfscope}%
\pgfpathrectangle{\pgfqpoint{1.150000in}{0.150000in}}{\pgfqpoint{5.700000in}{5.700000in}}%
\pgfusepath{clip}%
\pgfsetbuttcap%
\pgfsetroundjoin%
\definecolor{currentfill}{rgb}{0.225863,0.330805,0.547314}%
\pgfsetfillcolor{currentfill}%
\pgfsetfillopacity{0.700000}%
\pgfsetlinewidth{0.000000pt}%
\definecolor{currentstroke}{rgb}{0.000000,0.000000,0.000000}%
\pgfsetstrokecolor{currentstroke}%
\pgfsetdash{}{0pt}%
\pgfpathmoveto{\pgfqpoint{5.471262in}{2.873408in}}%
\pgfpathlineto{\pgfqpoint{5.485125in}{2.870013in}}%
\pgfpathlineto{\pgfqpoint{5.498997in}{2.866685in}}%
\pgfpathlineto{\pgfqpoint{5.512877in}{2.863423in}}%
\pgfpathlineto{\pgfqpoint{5.526765in}{2.860227in}}%
\pgfpathlineto{\pgfqpoint{5.534242in}{2.874643in}}%
\pgfpathlineto{\pgfqpoint{5.541724in}{2.889417in}}%
\pgfpathlineto{\pgfqpoint{5.549214in}{2.904558in}}%
\pgfpathlineto{\pgfqpoint{5.556711in}{2.920073in}}%
\pgfpathlineto{\pgfqpoint{5.542843in}{2.923769in}}%
\pgfpathlineto{\pgfqpoint{5.528982in}{2.927531in}}%
\pgfpathlineto{\pgfqpoint{5.515130in}{2.931359in}}%
\pgfpathlineto{\pgfqpoint{5.501286in}{2.935254in}}%
\pgfpathlineto{\pgfqpoint{5.493769in}{2.919231in}}%
\pgfpathlineto{\pgfqpoint{5.486260in}{2.903588in}}%
\pgfpathlineto{\pgfqpoint{5.478758in}{2.888317in}}%
\pgfpathlineto{\pgfqpoint{5.471262in}{2.873408in}}%
\pgfpathclose%
\pgfusepath{fill}%
\end{pgfscope}%
\begin{pgfscope}%
\pgfpathrectangle{\pgfqpoint{1.150000in}{0.150000in}}{\pgfqpoint{5.700000in}{5.700000in}}%
\pgfusepath{clip}%
\pgfsetbuttcap%
\pgfsetroundjoin%
\definecolor{currentfill}{rgb}{0.235526,0.309527,0.542944}%
\pgfsetfillcolor{currentfill}%
\pgfsetfillopacity{0.700000}%
\pgfsetlinewidth{0.000000pt}%
\definecolor{currentstroke}{rgb}{0.000000,0.000000,0.000000}%
\pgfsetstrokecolor{currentstroke}%
\pgfsetdash{}{0pt}%
\pgfpathmoveto{\pgfqpoint{5.385884in}{2.829540in}}%
\pgfpathlineto{\pgfqpoint{5.399734in}{2.826359in}}%
\pgfpathlineto{\pgfqpoint{5.413592in}{2.823244in}}%
\pgfpathlineto{\pgfqpoint{5.427458in}{2.820196in}}%
\pgfpathlineto{\pgfqpoint{5.441333in}{2.817215in}}%
\pgfpathlineto{\pgfqpoint{5.448807in}{2.830764in}}%
\pgfpathlineto{\pgfqpoint{5.456287in}{2.844640in}}%
\pgfpathlineto{\pgfqpoint{5.463771in}{2.858852in}}%
\pgfpathlineto{\pgfqpoint{5.471262in}{2.873408in}}%
\pgfpathlineto{\pgfqpoint{5.457406in}{2.876869in}}%
\pgfpathlineto{\pgfqpoint{5.443559in}{2.880397in}}%
\pgfpathlineto{\pgfqpoint{5.429720in}{2.883992in}}%
\pgfpathlineto{\pgfqpoint{5.415888in}{2.887653in}}%
\pgfpathlineto{\pgfqpoint{5.408379in}{2.872610in}}%
\pgfpathlineto{\pgfqpoint{5.400876in}{2.857916in}}%
\pgfpathlineto{\pgfqpoint{5.393377in}{2.843562in}}%
\pgfpathlineto{\pgfqpoint{5.385884in}{2.829540in}}%
\pgfpathclose%
\pgfusepath{fill}%
\end{pgfscope}%
\begin{pgfscope}%
\pgfpathrectangle{\pgfqpoint{1.150000in}{0.150000in}}{\pgfqpoint{5.700000in}{5.700000in}}%
\pgfusepath{clip}%
\pgfsetbuttcap%
\pgfsetroundjoin%
\definecolor{currentfill}{rgb}{0.216210,0.351535,0.550627}%
\pgfsetfillcolor{currentfill}%
\pgfsetfillopacity{0.700000}%
\pgfsetlinewidth{0.000000pt}%
\definecolor{currentstroke}{rgb}{0.000000,0.000000,0.000000}%
\pgfsetstrokecolor{currentstroke}%
\pgfsetdash{}{0pt}%
\pgfpathmoveto{\pgfqpoint{5.556711in}{2.920073in}}%
\pgfpathlineto{\pgfqpoint{5.570588in}{2.916443in}}%
\pgfpathlineto{\pgfqpoint{5.584472in}{2.912879in}}%
\pgfpathlineto{\pgfqpoint{5.598365in}{2.909381in}}%
\pgfpathlineto{\pgfqpoint{5.612267in}{2.905949in}}%
\pgfpathlineto{\pgfqpoint{5.619751in}{2.921337in}}%
\pgfpathlineto{\pgfqpoint{5.627244in}{2.937114in}}%
\pgfpathlineto{\pgfqpoint{5.634745in}{2.953290in}}%
\pgfpathlineto{\pgfqpoint{5.642256in}{2.969875in}}%
\pgfpathlineto{\pgfqpoint{5.628375in}{2.973827in}}%
\pgfpathlineto{\pgfqpoint{5.614502in}{2.977846in}}%
\pgfpathlineto{\pgfqpoint{5.600637in}{2.981929in}}%
\pgfpathlineto{\pgfqpoint{5.586780in}{2.986079in}}%
\pgfpathlineto{\pgfqpoint{5.579250in}{2.968967in}}%
\pgfpathlineto{\pgfqpoint{5.571729in}{2.952269in}}%
\pgfpathlineto{\pgfqpoint{5.564216in}{2.935974in}}%
\pgfpathlineto{\pgfqpoint{5.556711in}{2.920073in}}%
\pgfpathclose%
\pgfusepath{fill}%
\end{pgfscope}%
\begin{pgfscope}%
\pgfpathrectangle{\pgfqpoint{1.150000in}{0.150000in}}{\pgfqpoint{5.700000in}{5.700000in}}%
\pgfusepath{clip}%
\pgfsetbuttcap%
\pgfsetroundjoin%
\definecolor{currentfill}{rgb}{0.282910,0.105393,0.426902}%
\pgfsetfillcolor{currentfill}%
\pgfsetfillopacity{0.700000}%
\pgfsetlinewidth{0.000000pt}%
\definecolor{currentstroke}{rgb}{0.000000,0.000000,0.000000}%
\pgfsetstrokecolor{currentstroke}%
\pgfsetdash{}{0pt}%
\pgfpathmoveto{\pgfqpoint{3.887778in}{2.406319in}}%
\pgfpathlineto{\pgfqpoint{3.901282in}{2.402322in}}%
\pgfpathlineto{\pgfqpoint{3.914792in}{2.398407in}}%
\pgfpathlineto{\pgfqpoint{3.928308in}{2.394576in}}%
\pgfpathlineto{\pgfqpoint{3.941830in}{2.390828in}}%
\pgfpathlineto{\pgfqpoint{3.949698in}{2.400516in}}%
\pgfpathlineto{\pgfqpoint{3.957561in}{2.410269in}}%
\pgfpathlineto{\pgfqpoint{3.965419in}{2.420090in}}%
\pgfpathlineto{\pgfqpoint{3.973271in}{2.429983in}}%
\pgfpathlineto{\pgfqpoint{3.959759in}{2.433909in}}%
\pgfpathlineto{\pgfqpoint{3.946254in}{2.437919in}}%
\pgfpathlineto{\pgfqpoint{3.932754in}{2.442011in}}%
\pgfpathlineto{\pgfqpoint{3.919259in}{2.446187in}}%
\pgfpathlineto{\pgfqpoint{3.911397in}{2.436109in}}%
\pgfpathlineto{\pgfqpoint{3.903529in}{2.426107in}}%
\pgfpathlineto{\pgfqpoint{3.895656in}{2.416178in}}%
\pgfpathlineto{\pgfqpoint{3.887778in}{2.406319in}}%
\pgfpathclose%
\pgfusepath{fill}%
\end{pgfscope}%
\begin{pgfscope}%
\pgfpathrectangle{\pgfqpoint{1.150000in}{0.150000in}}{\pgfqpoint{5.700000in}{5.700000in}}%
\pgfusepath{clip}%
\pgfsetbuttcap%
\pgfsetroundjoin%
\definecolor{currentfill}{rgb}{0.243113,0.292092,0.538516}%
\pgfsetfillcolor{currentfill}%
\pgfsetfillopacity{0.700000}%
\pgfsetlinewidth{0.000000pt}%
\definecolor{currentstroke}{rgb}{0.000000,0.000000,0.000000}%
\pgfsetstrokecolor{currentstroke}%
\pgfsetdash{}{0pt}%
\pgfpathmoveto{\pgfqpoint{5.300557in}{2.788157in}}%
\pgfpathlineto{\pgfqpoint{5.314393in}{2.785166in}}%
\pgfpathlineto{\pgfqpoint{5.328237in}{2.782243in}}%
\pgfpathlineto{\pgfqpoint{5.342089in}{2.779388in}}%
\pgfpathlineto{\pgfqpoint{5.355949in}{2.776599in}}%
\pgfpathlineto{\pgfqpoint{5.363428in}{2.789379in}}%
\pgfpathlineto{\pgfqpoint{5.370909in}{2.802457in}}%
\pgfpathlineto{\pgfqpoint{5.378394in}{2.815841in}}%
\pgfpathlineto{\pgfqpoint{5.385884in}{2.829540in}}%
\pgfpathlineto{\pgfqpoint{5.372042in}{2.832789in}}%
\pgfpathlineto{\pgfqpoint{5.358208in}{2.836104in}}%
\pgfpathlineto{\pgfqpoint{5.344383in}{2.839487in}}%
\pgfpathlineto{\pgfqpoint{5.330565in}{2.842937in}}%
\pgfpathlineto{\pgfqpoint{5.323057in}{2.828771in}}%
\pgfpathlineto{\pgfqpoint{5.315554in}{2.814925in}}%
\pgfpathlineto{\pgfqpoint{5.308054in}{2.801390in}}%
\pgfpathlineto{\pgfqpoint{5.300557in}{2.788157in}}%
\pgfpathclose%
\pgfusepath{fill}%
\end{pgfscope}%
\begin{pgfscope}%
\pgfpathrectangle{\pgfqpoint{1.150000in}{0.150000in}}{\pgfqpoint{5.700000in}{5.700000in}}%
\pgfusepath{clip}%
\pgfsetbuttcap%
\pgfsetroundjoin%
\definecolor{currentfill}{rgb}{0.280868,0.160771,0.472899}%
\pgfsetfillcolor{currentfill}%
\pgfsetfillopacity{0.700000}%
\pgfsetlinewidth{0.000000pt}%
\definecolor{currentstroke}{rgb}{0.000000,0.000000,0.000000}%
\pgfsetstrokecolor{currentstroke}%
\pgfsetdash{}{0pt}%
\pgfpathmoveto{\pgfqpoint{4.423364in}{2.508931in}}%
\pgfpathlineto{\pgfqpoint{4.436994in}{2.506055in}}%
\pgfpathlineto{\pgfqpoint{4.450631in}{2.503254in}}%
\pgfpathlineto{\pgfqpoint{4.464275in}{2.500529in}}%
\pgfpathlineto{\pgfqpoint{4.477927in}{2.497878in}}%
\pgfpathlineto{\pgfqpoint{4.485620in}{2.507701in}}%
\pgfpathlineto{\pgfqpoint{4.493308in}{2.517638in}}%
\pgfpathlineto{\pgfqpoint{4.500993in}{2.527694in}}%
\pgfpathlineto{\pgfqpoint{4.508674in}{2.537874in}}%
\pgfpathlineto{\pgfqpoint{4.495035in}{2.540805in}}%
\pgfpathlineto{\pgfqpoint{4.481403in}{2.543810in}}%
\pgfpathlineto{\pgfqpoint{4.467778in}{2.546890in}}%
\pgfpathlineto{\pgfqpoint{4.454160in}{2.550044in}}%
\pgfpathlineto{\pgfqpoint{4.446467in}{2.539578in}}%
\pgfpathlineto{\pgfqpoint{4.438770in}{2.529240in}}%
\pgfpathlineto{\pgfqpoint{4.431069in}{2.519026in}}%
\pgfpathlineto{\pgfqpoint{4.423364in}{2.508931in}}%
\pgfpathclose%
\pgfusepath{fill}%
\end{pgfscope}%
\begin{pgfscope}%
\pgfpathrectangle{\pgfqpoint{1.150000in}{0.150000in}}{\pgfqpoint{5.700000in}{5.700000in}}%
\pgfusepath{clip}%
\pgfsetbuttcap%
\pgfsetroundjoin%
\definecolor{currentfill}{rgb}{0.282327,0.094955,0.417331}%
\pgfsetfillcolor{currentfill}%
\pgfsetfillopacity{0.700000}%
\pgfsetlinewidth{0.000000pt}%
\definecolor{currentstroke}{rgb}{0.000000,0.000000,0.000000}%
\pgfsetstrokecolor{currentstroke}%
\pgfsetdash{}{0pt}%
\pgfpathmoveto{\pgfqpoint{3.662679in}{2.380635in}}%
\pgfpathlineto{\pgfqpoint{3.676140in}{2.375902in}}%
\pgfpathlineto{\pgfqpoint{3.689606in}{2.371258in}}%
\pgfpathlineto{\pgfqpoint{3.703077in}{2.366701in}}%
\pgfpathlineto{\pgfqpoint{3.716553in}{2.362232in}}%
\pgfpathlineto{\pgfqpoint{3.724496in}{2.371907in}}%
\pgfpathlineto{\pgfqpoint{3.732433in}{2.381638in}}%
\pgfpathlineto{\pgfqpoint{3.740365in}{2.391428in}}%
\pgfpathlineto{\pgfqpoint{3.748291in}{2.401278in}}%
\pgfpathlineto{\pgfqpoint{3.734825in}{2.405885in}}%
\pgfpathlineto{\pgfqpoint{3.721364in}{2.410579in}}%
\pgfpathlineto{\pgfqpoint{3.707908in}{2.415362in}}%
\pgfpathlineto{\pgfqpoint{3.694457in}{2.420232in}}%
\pgfpathlineto{\pgfqpoint{3.686521in}{2.410237in}}%
\pgfpathlineto{\pgfqpoint{3.678579in}{2.400307in}}%
\pgfpathlineto{\pgfqpoint{3.670632in}{2.390441in}}%
\pgfpathlineto{\pgfqpoint{3.662679in}{2.380635in}}%
\pgfpathclose%
\pgfusepath{fill}%
\end{pgfscope}%
\begin{pgfscope}%
\pgfpathrectangle{\pgfqpoint{1.150000in}{0.150000in}}{\pgfqpoint{5.700000in}{5.700000in}}%
\pgfusepath{clip}%
\pgfsetbuttcap%
\pgfsetroundjoin%
\definecolor{currentfill}{rgb}{0.282327,0.094955,0.417331}%
\pgfsetfillcolor{currentfill}%
\pgfsetfillopacity{0.700000}%
\pgfsetlinewidth{0.000000pt}%
\definecolor{currentstroke}{rgb}{0.000000,0.000000,0.000000}%
\pgfsetstrokecolor{currentstroke}%
\pgfsetdash{}{0pt}%
\pgfpathmoveto{\pgfqpoint{3.158233in}{2.390628in}}%
\pgfpathlineto{\pgfqpoint{3.171631in}{2.383564in}}%
\pgfpathlineto{\pgfqpoint{3.185032in}{2.376604in}}%
\pgfpathlineto{\pgfqpoint{3.198435in}{2.369747in}}%
\pgfpathlineto{\pgfqpoint{3.211841in}{2.362992in}}%
\pgfpathlineto{\pgfqpoint{3.219955in}{2.372432in}}%
\pgfpathlineto{\pgfqpoint{3.228061in}{2.381930in}}%
\pgfpathlineto{\pgfqpoint{3.236162in}{2.391487in}}%
\pgfpathlineto{\pgfqpoint{3.244256in}{2.401105in}}%
\pgfpathlineto{\pgfqpoint{3.230861in}{2.407917in}}%
\pgfpathlineto{\pgfqpoint{3.217469in}{2.414830in}}%
\pgfpathlineto{\pgfqpoint{3.204079in}{2.421847in}}%
\pgfpathlineto{\pgfqpoint{3.190693in}{2.428967in}}%
\pgfpathlineto{\pgfqpoint{3.182588in}{2.419285in}}%
\pgfpathlineto{\pgfqpoint{3.174476in}{2.409669in}}%
\pgfpathlineto{\pgfqpoint{3.166358in}{2.400117in}}%
\pgfpathlineto{\pgfqpoint{3.158233in}{2.390628in}}%
\pgfpathclose%
\pgfusepath{fill}%
\end{pgfscope}%
\begin{pgfscope}%
\pgfpathrectangle{\pgfqpoint{1.150000in}{0.150000in}}{\pgfqpoint{5.700000in}{5.700000in}}%
\pgfusepath{clip}%
\pgfsetbuttcap%
\pgfsetroundjoin%
\definecolor{currentfill}{rgb}{0.206756,0.371758,0.553117}%
\pgfsetfillcolor{currentfill}%
\pgfsetfillopacity{0.700000}%
\pgfsetlinewidth{0.000000pt}%
\definecolor{currentstroke}{rgb}{0.000000,0.000000,0.000000}%
\pgfsetstrokecolor{currentstroke}%
\pgfsetdash{}{0pt}%
\pgfpathmoveto{\pgfqpoint{5.642256in}{2.969875in}}%
\pgfpathlineto{\pgfqpoint{5.656145in}{2.965988in}}%
\pgfpathlineto{\pgfqpoint{5.670042in}{2.962166in}}%
\pgfpathlineto{\pgfqpoint{5.683947in}{2.958410in}}%
\pgfpathlineto{\pgfqpoint{5.697861in}{2.954719in}}%
\pgfpathlineto{\pgfqpoint{5.705360in}{2.971189in}}%
\pgfpathlineto{\pgfqpoint{5.712870in}{2.988082in}}%
\pgfpathlineto{\pgfqpoint{5.720390in}{3.005408in}}%
\pgfpathlineto{\pgfqpoint{5.727922in}{3.023177in}}%
\pgfpathlineto{\pgfqpoint{5.714029in}{3.027409in}}%
\pgfpathlineto{\pgfqpoint{5.700144in}{3.031705in}}%
\pgfpathlineto{\pgfqpoint{5.686267in}{3.036067in}}%
\pgfpathlineto{\pgfqpoint{5.672398in}{3.040494in}}%
\pgfpathlineto{\pgfqpoint{5.664847in}{3.022177in}}%
\pgfpathlineto{\pgfqpoint{5.657306in}{3.004308in}}%
\pgfpathlineto{\pgfqpoint{5.649776in}{2.986878in}}%
\pgfpathlineto{\pgfqpoint{5.642256in}{2.969875in}}%
\pgfpathclose%
\pgfusepath{fill}%
\end{pgfscope}%
\begin{pgfscope}%
\pgfpathrectangle{\pgfqpoint{1.150000in}{0.150000in}}{\pgfqpoint{5.700000in}{5.700000in}}%
\pgfusepath{clip}%
\pgfsetbuttcap%
\pgfsetroundjoin%
\definecolor{currentfill}{rgb}{0.281924,0.089666,0.412415}%
\pgfsetfillcolor{currentfill}%
\pgfsetfillopacity{0.700000}%
\pgfsetlinewidth{0.000000pt}%
\definecolor{currentstroke}{rgb}{0.000000,0.000000,0.000000}%
\pgfsetstrokecolor{currentstroke}%
\pgfsetdash{}{0pt}%
\pgfpathmoveto{\pgfqpoint{3.297864in}{2.374872in}}%
\pgfpathlineto{\pgfqpoint{3.311274in}{2.368564in}}%
\pgfpathlineto{\pgfqpoint{3.324688in}{2.362354in}}%
\pgfpathlineto{\pgfqpoint{3.338105in}{2.356242in}}%
\pgfpathlineto{\pgfqpoint{3.351525in}{2.350227in}}%
\pgfpathlineto{\pgfqpoint{3.359591in}{2.359769in}}%
\pgfpathlineto{\pgfqpoint{3.367650in}{2.369366in}}%
\pgfpathlineto{\pgfqpoint{3.375703in}{2.379019in}}%
\pgfpathlineto{\pgfqpoint{3.383751in}{2.388729in}}%
\pgfpathlineto{\pgfqpoint{3.370341in}{2.394820in}}%
\pgfpathlineto{\pgfqpoint{3.356935in}{2.401009in}}%
\pgfpathlineto{\pgfqpoint{3.343532in}{2.407296in}}%
\pgfpathlineto{\pgfqpoint{3.330133in}{2.413682in}}%
\pgfpathlineto{\pgfqpoint{3.322075in}{2.403887in}}%
\pgfpathlineto{\pgfqpoint{3.314011in}{2.394155in}}%
\pgfpathlineto{\pgfqpoint{3.305941in}{2.384484in}}%
\pgfpathlineto{\pgfqpoint{3.297864in}{2.374872in}}%
\pgfpathclose%
\pgfusepath{fill}%
\end{pgfscope}%
\begin{pgfscope}%
\pgfpathrectangle{\pgfqpoint{1.150000in}{0.150000in}}{\pgfqpoint{5.700000in}{5.700000in}}%
\pgfusepath{clip}%
\pgfsetbuttcap%
\pgfsetroundjoin%
\definecolor{currentfill}{rgb}{0.250425,0.274290,0.533103}%
\pgfsetfillcolor{currentfill}%
\pgfsetfillopacity{0.700000}%
\pgfsetlinewidth{0.000000pt}%
\definecolor{currentstroke}{rgb}{0.000000,0.000000,0.000000}%
\pgfsetstrokecolor{currentstroke}%
\pgfsetdash{}{0pt}%
\pgfpathmoveto{\pgfqpoint{5.215263in}{2.748970in}}%
\pgfpathlineto{\pgfqpoint{5.229084in}{2.746149in}}%
\pgfpathlineto{\pgfqpoint{5.242914in}{2.743395in}}%
\pgfpathlineto{\pgfqpoint{5.256751in}{2.740710in}}%
\pgfpathlineto{\pgfqpoint{5.270597in}{2.738092in}}%
\pgfpathlineto{\pgfqpoint{5.278083in}{2.750194in}}%
\pgfpathlineto{\pgfqpoint{5.285572in}{2.762567in}}%
\pgfpathlineto{\pgfqpoint{5.293063in}{2.775219in}}%
\pgfpathlineto{\pgfqpoint{5.300557in}{2.788157in}}%
\pgfpathlineto{\pgfqpoint{5.286729in}{2.791215in}}%
\pgfpathlineto{\pgfqpoint{5.272910in}{2.794340in}}%
\pgfpathlineto{\pgfqpoint{5.259098in}{2.797534in}}%
\pgfpathlineto{\pgfqpoint{5.245295in}{2.800795in}}%
\pgfpathlineto{\pgfqpoint{5.237784in}{2.787410in}}%
\pgfpathlineto{\pgfqpoint{5.230275in}{2.774316in}}%
\pgfpathlineto{\pgfqpoint{5.222768in}{2.761505in}}%
\pgfpathlineto{\pgfqpoint{5.215263in}{2.748970in}}%
\pgfpathclose%
\pgfusepath{fill}%
\end{pgfscope}%
\begin{pgfscope}%
\pgfpathrectangle{\pgfqpoint{1.150000in}{0.150000in}}{\pgfqpoint{5.700000in}{5.700000in}}%
\pgfusepath{clip}%
\pgfsetbuttcap%
\pgfsetroundjoin%
\definecolor{currentfill}{rgb}{0.282910,0.105393,0.426902}%
\pgfsetfillcolor{currentfill}%
\pgfsetfillopacity{0.700000}%
\pgfsetlinewidth{0.000000pt}%
\definecolor{currentstroke}{rgb}{0.000000,0.000000,0.000000}%
\pgfsetstrokecolor{currentstroke}%
\pgfsetdash{}{0pt}%
\pgfpathmoveto{\pgfqpoint{3.018468in}{2.413330in}}%
\pgfpathlineto{\pgfqpoint{3.031861in}{2.405445in}}%
\pgfpathlineto{\pgfqpoint{3.045256in}{2.397669in}}%
\pgfpathlineto{\pgfqpoint{3.058653in}{2.390003in}}%
\pgfpathlineto{\pgfqpoint{3.072051in}{2.382445in}}%
\pgfpathlineto{\pgfqpoint{3.080215in}{2.391722in}}%
\pgfpathlineto{\pgfqpoint{3.088372in}{2.401062in}}%
\pgfpathlineto{\pgfqpoint{3.096522in}{2.410467in}}%
\pgfpathlineto{\pgfqpoint{3.104665in}{2.419937in}}%
\pgfpathlineto{\pgfqpoint{3.091278in}{2.427532in}}%
\pgfpathlineto{\pgfqpoint{3.077893in}{2.435234in}}%
\pgfpathlineto{\pgfqpoint{3.064510in}{2.443045in}}%
\pgfpathlineto{\pgfqpoint{3.051129in}{2.450967in}}%
\pgfpathlineto{\pgfqpoint{3.042974in}{2.441453in}}%
\pgfpathlineto{\pgfqpoint{3.034813in}{2.432009in}}%
\pgfpathlineto{\pgfqpoint{3.026644in}{2.422635in}}%
\pgfpathlineto{\pgfqpoint{3.018468in}{2.413330in}}%
\pgfpathclose%
\pgfusepath{fill}%
\end{pgfscope}%
\begin{pgfscope}%
\pgfpathrectangle{\pgfqpoint{1.150000in}{0.150000in}}{\pgfqpoint{5.700000in}{5.700000in}}%
\pgfusepath{clip}%
\pgfsetbuttcap%
\pgfsetroundjoin%
\definecolor{currentfill}{rgb}{0.283187,0.125848,0.444960}%
\pgfsetfillcolor{currentfill}%
\pgfsetfillopacity{0.700000}%
\pgfsetlinewidth{0.000000pt}%
\definecolor{currentstroke}{rgb}{0.000000,0.000000,0.000000}%
\pgfsetstrokecolor{currentstroke}%
\pgfsetdash{}{0pt}%
\pgfpathmoveto{\pgfqpoint{4.112853in}{2.440273in}}%
\pgfpathlineto{\pgfqpoint{4.126409in}{2.436877in}}%
\pgfpathlineto{\pgfqpoint{4.139972in}{2.433560in}}%
\pgfpathlineto{\pgfqpoint{4.153540in}{2.430322in}}%
\pgfpathlineto{\pgfqpoint{4.167116in}{2.427163in}}%
\pgfpathlineto{\pgfqpoint{4.174911in}{2.436835in}}%
\pgfpathlineto{\pgfqpoint{4.182701in}{2.446586in}}%
\pgfpathlineto{\pgfqpoint{4.190486in}{2.456421in}}%
\pgfpathlineto{\pgfqpoint{4.198266in}{2.466342in}}%
\pgfpathlineto{\pgfqpoint{4.184701in}{2.469720in}}%
\pgfpathlineto{\pgfqpoint{4.171143in}{2.473177in}}%
\pgfpathlineto{\pgfqpoint{4.157592in}{2.476712in}}%
\pgfpathlineto{\pgfqpoint{4.144046in}{2.480327in}}%
\pgfpathlineto{\pgfqpoint{4.136255in}{2.470179in}}%
\pgfpathlineto{\pgfqpoint{4.128460in}{2.460124in}}%
\pgfpathlineto{\pgfqpoint{4.120659in}{2.450156in}}%
\pgfpathlineto{\pgfqpoint{4.112853in}{2.440273in}}%
\pgfpathclose%
\pgfusepath{fill}%
\end{pgfscope}%
\begin{pgfscope}%
\pgfpathrectangle{\pgfqpoint{1.150000in}{0.150000in}}{\pgfqpoint{5.700000in}{5.700000in}}%
\pgfusepath{clip}%
\pgfsetbuttcap%
\pgfsetroundjoin%
\definecolor{currentfill}{rgb}{0.274128,0.199721,0.498911}%
\pgfsetfillcolor{currentfill}%
\pgfsetfillopacity{0.700000}%
\pgfsetlinewidth{0.000000pt}%
\definecolor{currentstroke}{rgb}{0.000000,0.000000,0.000000}%
\pgfsetstrokecolor{currentstroke}%
\pgfsetdash{}{0pt}%
\pgfpathmoveto{\pgfqpoint{4.733926in}{2.587583in}}%
\pgfpathlineto{\pgfqpoint{4.747636in}{2.584985in}}%
\pgfpathlineto{\pgfqpoint{4.761353in}{2.582459in}}%
\pgfpathlineto{\pgfqpoint{4.775077in}{2.580004in}}%
\pgfpathlineto{\pgfqpoint{4.788810in}{2.577621in}}%
\pgfpathlineto{\pgfqpoint{4.796407in}{2.587836in}}%
\pgfpathlineto{\pgfqpoint{4.804002in}{2.598212in}}%
\pgfpathlineto{\pgfqpoint{4.811595in}{2.608755in}}%
\pgfpathlineto{\pgfqpoint{4.819185in}{2.619471in}}%
\pgfpathlineto{\pgfqpoint{4.805467in}{2.622194in}}%
\pgfpathlineto{\pgfqpoint{4.791757in}{2.624989in}}%
\pgfpathlineto{\pgfqpoint{4.778054in}{2.627855in}}%
\pgfpathlineto{\pgfqpoint{4.764358in}{2.630792in}}%
\pgfpathlineto{\pgfqpoint{4.756754in}{2.619729in}}%
\pgfpathlineto{\pgfqpoint{4.749147in}{2.608844in}}%
\pgfpathlineto{\pgfqpoint{4.741538in}{2.598131in}}%
\pgfpathlineto{\pgfqpoint{4.733926in}{2.587583in}}%
\pgfpathclose%
\pgfusepath{fill}%
\end{pgfscope}%
\begin{pgfscope}%
\pgfpathrectangle{\pgfqpoint{1.150000in}{0.150000in}}{\pgfqpoint{5.700000in}{5.700000in}}%
\pgfusepath{clip}%
\pgfsetbuttcap%
\pgfsetroundjoin%
\definecolor{currentfill}{rgb}{0.255645,0.260703,0.528312}%
\pgfsetfillcolor{currentfill}%
\pgfsetfillopacity{0.700000}%
\pgfsetlinewidth{0.000000pt}%
\definecolor{currentstroke}{rgb}{0.000000,0.000000,0.000000}%
\pgfsetstrokecolor{currentstroke}%
\pgfsetdash{}{0pt}%
\pgfpathmoveto{\pgfqpoint{5.129987in}{2.711717in}}%
\pgfpathlineto{\pgfqpoint{5.143793in}{2.709042in}}%
\pgfpathlineto{\pgfqpoint{5.157607in}{2.706436in}}%
\pgfpathlineto{\pgfqpoint{5.171429in}{2.703899in}}%
\pgfpathlineto{\pgfqpoint{5.185260in}{2.701429in}}%
\pgfpathlineto{\pgfqpoint{5.192759in}{2.712940in}}%
\pgfpathlineto{\pgfqpoint{5.200259in}{2.724695in}}%
\pgfpathlineto{\pgfqpoint{5.207761in}{2.736703in}}%
\pgfpathlineto{\pgfqpoint{5.215263in}{2.748970in}}%
\pgfpathlineto{\pgfqpoint{5.201451in}{2.751860in}}%
\pgfpathlineto{\pgfqpoint{5.187646in}{2.754817in}}%
\pgfpathlineto{\pgfqpoint{5.173849in}{2.757843in}}%
\pgfpathlineto{\pgfqpoint{5.160060in}{2.760937in}}%
\pgfpathlineto{\pgfqpoint{5.152540in}{2.748243in}}%
\pgfpathlineto{\pgfqpoint{5.145021in}{2.735813in}}%
\pgfpathlineto{\pgfqpoint{5.137504in}{2.723640in}}%
\pgfpathlineto{\pgfqpoint{5.129987in}{2.711717in}}%
\pgfpathclose%
\pgfusepath{fill}%
\end{pgfscope}%
\begin{pgfscope}%
\pgfpathrectangle{\pgfqpoint{1.150000in}{0.150000in}}{\pgfqpoint{5.700000in}{5.700000in}}%
\pgfusepath{clip}%
\pgfsetbuttcap%
\pgfsetroundjoin%
\definecolor{currentfill}{rgb}{0.281446,0.084320,0.407414}%
\pgfsetfillcolor{currentfill}%
\pgfsetfillopacity{0.700000}%
\pgfsetlinewidth{0.000000pt}%
\definecolor{currentstroke}{rgb}{0.000000,0.000000,0.000000}%
\pgfsetstrokecolor{currentstroke}%
\pgfsetdash{}{0pt}%
\pgfpathmoveto{\pgfqpoint{3.437426in}{2.365325in}}%
\pgfpathlineto{\pgfqpoint{3.450854in}{2.359712in}}%
\pgfpathlineto{\pgfqpoint{3.464286in}{2.354193in}}%
\pgfpathlineto{\pgfqpoint{3.477723in}{2.348768in}}%
\pgfpathlineto{\pgfqpoint{3.491163in}{2.343435in}}%
\pgfpathlineto{\pgfqpoint{3.499183in}{2.353026in}}%
\pgfpathlineto{\pgfqpoint{3.507198in}{2.362668in}}%
\pgfpathlineto{\pgfqpoint{3.515206in}{2.372364in}}%
\pgfpathlineto{\pgfqpoint{3.523208in}{2.382116in}}%
\pgfpathlineto{\pgfqpoint{3.509778in}{2.387546in}}%
\pgfpathlineto{\pgfqpoint{3.496352in}{2.393068in}}%
\pgfpathlineto{\pgfqpoint{3.482930in}{2.398685in}}%
\pgfpathlineto{\pgfqpoint{3.469512in}{2.404395in}}%
\pgfpathlineto{\pgfqpoint{3.461499in}{2.394538in}}%
\pgfpathlineto{\pgfqpoint{3.453481in}{2.384743in}}%
\pgfpathlineto{\pgfqpoint{3.445456in}{2.375006in}}%
\pgfpathlineto{\pgfqpoint{3.437426in}{2.365325in}}%
\pgfpathclose%
\pgfusepath{fill}%
\end{pgfscope}%
\begin{pgfscope}%
\pgfpathrectangle{\pgfqpoint{1.150000in}{0.150000in}}{\pgfqpoint{5.700000in}{5.700000in}}%
\pgfusepath{clip}%
\pgfsetbuttcap%
\pgfsetroundjoin%
\definecolor{currentfill}{rgb}{0.195860,0.395433,0.555276}%
\pgfsetfillcolor{currentfill}%
\pgfsetfillopacity{0.700000}%
\pgfsetlinewidth{0.000000pt}%
\definecolor{currentstroke}{rgb}{0.000000,0.000000,0.000000}%
\pgfsetstrokecolor{currentstroke}%
\pgfsetdash{}{0pt}%
\pgfpathmoveto{\pgfqpoint{5.727922in}{3.023177in}}%
\pgfpathlineto{\pgfqpoint{5.741822in}{3.019011in}}%
\pgfpathlineto{\pgfqpoint{5.755731in}{3.014910in}}%
\pgfpathlineto{\pgfqpoint{5.769648in}{3.010874in}}%
\pgfpathlineto{\pgfqpoint{5.783573in}{3.006903in}}%
\pgfpathlineto{\pgfqpoint{5.791095in}{3.024572in}}%
\pgfpathlineto{\pgfqpoint{5.798630in}{3.042700in}}%
\pgfpathlineto{\pgfqpoint{5.806177in}{3.061297in}}%
\pgfpathlineto{\pgfqpoint{5.813738in}{3.080373in}}%
\pgfpathlineto{\pgfqpoint{5.799834in}{3.084905in}}%
\pgfpathlineto{\pgfqpoint{5.785938in}{3.089502in}}%
\pgfpathlineto{\pgfqpoint{5.772050in}{3.094163in}}%
\pgfpathlineto{\pgfqpoint{5.758169in}{3.098890in}}%
\pgfpathlineto{\pgfqpoint{5.750588in}{3.079245in}}%
\pgfpathlineto{\pgfqpoint{5.743020in}{3.060086in}}%
\pgfpathlineto{\pgfqpoint{5.735465in}{3.041400in}}%
\pgfpathlineto{\pgfqpoint{5.727922in}{3.023177in}}%
\pgfpathclose%
\pgfusepath{fill}%
\end{pgfscope}%
\begin{pgfscope}%
\pgfpathrectangle{\pgfqpoint{1.150000in}{0.150000in}}{\pgfqpoint{5.700000in}{5.700000in}}%
\pgfusepath{clip}%
\pgfsetbuttcap%
\pgfsetroundjoin%
\definecolor{currentfill}{rgb}{0.280868,0.160771,0.472899}%
\pgfsetfillcolor{currentfill}%
\pgfsetfillopacity{0.700000}%
\pgfsetlinewidth{0.000000pt}%
\definecolor{currentstroke}{rgb}{0.000000,0.000000,0.000000}%
\pgfsetstrokecolor{currentstroke}%
\pgfsetdash{}{0pt}%
\pgfpathmoveto{\pgfqpoint{2.684594in}{2.523158in}}%
\pgfpathlineto{\pgfqpoint{2.698004in}{2.512897in}}%
\pgfpathlineto{\pgfqpoint{2.711413in}{2.502766in}}%
\pgfpathlineto{\pgfqpoint{2.724822in}{2.492762in}}%
\pgfpathlineto{\pgfqpoint{2.738230in}{2.482884in}}%
\pgfpathlineto{\pgfqpoint{2.746520in}{2.491620in}}%
\pgfpathlineto{\pgfqpoint{2.754802in}{2.500439in}}%
\pgfpathlineto{\pgfqpoint{2.763076in}{2.509342in}}%
\pgfpathlineto{\pgfqpoint{2.771342in}{2.518329in}}%
\pgfpathlineto{\pgfqpoint{2.757949in}{2.528201in}}%
\pgfpathlineto{\pgfqpoint{2.744555in}{2.538199in}}%
\pgfpathlineto{\pgfqpoint{2.731160in}{2.548325in}}%
\pgfpathlineto{\pgfqpoint{2.717765in}{2.558580in}}%
\pgfpathlineto{\pgfqpoint{2.709485in}{2.549591in}}%
\pgfpathlineto{\pgfqpoint{2.701196in}{2.540692in}}%
\pgfpathlineto{\pgfqpoint{2.692899in}{2.531881in}}%
\pgfpathlineto{\pgfqpoint{2.684594in}{2.523158in}}%
\pgfpathclose%
\pgfusepath{fill}%
\end{pgfscope}%
\begin{pgfscope}%
\pgfpathrectangle{\pgfqpoint{1.150000in}{0.150000in}}{\pgfqpoint{5.700000in}{5.700000in}}%
\pgfusepath{clip}%
\pgfsetbuttcap%
\pgfsetroundjoin%
\definecolor{currentfill}{rgb}{0.283187,0.125848,0.444960}%
\pgfsetfillcolor{currentfill}%
\pgfsetfillopacity{0.700000}%
\pgfsetlinewidth{0.000000pt}%
\definecolor{currentstroke}{rgb}{0.000000,0.000000,0.000000}%
\pgfsetstrokecolor{currentstroke}%
\pgfsetdash{}{0pt}%
\pgfpathmoveto{\pgfqpoint{2.878495in}{2.443786in}}%
\pgfpathlineto{\pgfqpoint{2.891891in}{2.435007in}}%
\pgfpathlineto{\pgfqpoint{2.905288in}{2.426345in}}%
\pgfpathlineto{\pgfqpoint{2.918686in}{2.417799in}}%
\pgfpathlineto{\pgfqpoint{2.932085in}{2.409368in}}%
\pgfpathlineto{\pgfqpoint{2.940302in}{2.418416in}}%
\pgfpathlineto{\pgfqpoint{2.948512in}{2.427535in}}%
\pgfpathlineto{\pgfqpoint{2.956715in}{2.436724in}}%
\pgfpathlineto{\pgfqpoint{2.964911in}{2.445986in}}%
\pgfpathlineto{\pgfqpoint{2.951525in}{2.454432in}}%
\pgfpathlineto{\pgfqpoint{2.938140in}{2.462994in}}%
\pgfpathlineto{\pgfqpoint{2.924756in}{2.471671in}}%
\pgfpathlineto{\pgfqpoint{2.911373in}{2.480465in}}%
\pgfpathlineto{\pgfqpoint{2.903165in}{2.471181in}}%
\pgfpathlineto{\pgfqpoint{2.894949in}{2.461974in}}%
\pgfpathlineto{\pgfqpoint{2.886726in}{2.452842in}}%
\pgfpathlineto{\pgfqpoint{2.878495in}{2.443786in}}%
\pgfpathclose%
\pgfusepath{fill}%
\end{pgfscope}%
\begin{pgfscope}%
\pgfpathrectangle{\pgfqpoint{1.150000in}{0.150000in}}{\pgfqpoint{5.700000in}{5.700000in}}%
\pgfusepath{clip}%
\pgfsetbuttcap%
\pgfsetroundjoin%
\definecolor{currentfill}{rgb}{0.281887,0.150881,0.465405}%
\pgfsetfillcolor{currentfill}%
\pgfsetfillopacity{0.700000}%
\pgfsetlinewidth{0.000000pt}%
\definecolor{currentstroke}{rgb}{0.000000,0.000000,0.000000}%
\pgfsetstrokecolor{currentstroke}%
\pgfsetdash{}{0pt}%
\pgfpathmoveto{\pgfqpoint{4.338004in}{2.480861in}}%
\pgfpathlineto{\pgfqpoint{4.351618in}{2.477941in}}%
\pgfpathlineto{\pgfqpoint{4.365238in}{2.475098in}}%
\pgfpathlineto{\pgfqpoint{4.378866in}{2.472330in}}%
\pgfpathlineto{\pgfqpoint{4.392501in}{2.469638in}}%
\pgfpathlineto{\pgfqpoint{4.400223in}{2.479308in}}%
\pgfpathlineto{\pgfqpoint{4.407941in}{2.489076in}}%
\pgfpathlineto{\pgfqpoint{4.415655in}{2.498949in}}%
\pgfpathlineto{\pgfqpoint{4.423364in}{2.508931in}}%
\pgfpathlineto{\pgfqpoint{4.409741in}{2.511882in}}%
\pgfpathlineto{\pgfqpoint{4.396125in}{2.514909in}}%
\pgfpathlineto{\pgfqpoint{4.382516in}{2.518012in}}%
\pgfpathlineto{\pgfqpoint{4.368914in}{2.521190in}}%
\pgfpathlineto{\pgfqpoint{4.361193in}{2.510942in}}%
\pgfpathlineto{\pgfqpoint{4.353468in}{2.500808in}}%
\pgfpathlineto{\pgfqpoint{4.345738in}{2.490782in}}%
\pgfpathlineto{\pgfqpoint{4.338004in}{2.480861in}}%
\pgfpathclose%
\pgfusepath{fill}%
\end{pgfscope}%
\begin{pgfscope}%
\pgfpathrectangle{\pgfqpoint{1.150000in}{0.150000in}}{\pgfqpoint{5.700000in}{5.700000in}}%
\pgfusepath{clip}%
\pgfsetbuttcap%
\pgfsetroundjoin%
\definecolor{currentfill}{rgb}{0.277134,0.185228,0.489898}%
\pgfsetfillcolor{currentfill}%
\pgfsetfillopacity{0.700000}%
\pgfsetlinewidth{0.000000pt}%
\definecolor{currentstroke}{rgb}{0.000000,0.000000,0.000000}%
\pgfsetstrokecolor{currentstroke}%
\pgfsetdash{}{0pt}%
\pgfpathmoveto{\pgfqpoint{4.648634in}{2.556765in}}%
\pgfpathlineto{\pgfqpoint{4.662327in}{2.554197in}}%
\pgfpathlineto{\pgfqpoint{4.676027in}{2.551702in}}%
\pgfpathlineto{\pgfqpoint{4.689735in}{2.549280in}}%
\pgfpathlineto{\pgfqpoint{4.703451in}{2.546929in}}%
\pgfpathlineto{\pgfqpoint{4.711074in}{2.556874in}}%
\pgfpathlineto{\pgfqpoint{4.718694in}{2.566961in}}%
\pgfpathlineto{\pgfqpoint{4.726312in}{2.577195in}}%
\pgfpathlineto{\pgfqpoint{4.733926in}{2.587583in}}%
\pgfpathlineto{\pgfqpoint{4.720225in}{2.590253in}}%
\pgfpathlineto{\pgfqpoint{4.706531in}{2.592996in}}%
\pgfpathlineto{\pgfqpoint{4.692844in}{2.595810in}}%
\pgfpathlineto{\pgfqpoint{4.679165in}{2.598697in}}%
\pgfpathlineto{\pgfqpoint{4.671537in}{2.587982in}}%
\pgfpathlineto{\pgfqpoint{4.663906in}{2.577426in}}%
\pgfpathlineto{\pgfqpoint{4.656272in}{2.567022in}}%
\pgfpathlineto{\pgfqpoint{4.648634in}{2.556765in}}%
\pgfpathclose%
\pgfusepath{fill}%
\end{pgfscope}%
\begin{pgfscope}%
\pgfpathrectangle{\pgfqpoint{1.150000in}{0.150000in}}{\pgfqpoint{5.700000in}{5.700000in}}%
\pgfusepath{clip}%
\pgfsetbuttcap%
\pgfsetroundjoin%
\definecolor{currentfill}{rgb}{0.260571,0.246922,0.522828}%
\pgfsetfillcolor{currentfill}%
\pgfsetfillopacity{0.700000}%
\pgfsetlinewidth{0.000000pt}%
\definecolor{currentstroke}{rgb}{0.000000,0.000000,0.000000}%
\pgfsetstrokecolor{currentstroke}%
\pgfsetdash{}{0pt}%
\pgfpathmoveto{\pgfqpoint{5.044715in}{2.676158in}}%
\pgfpathlineto{\pgfqpoint{5.058505in}{2.673608in}}%
\pgfpathlineto{\pgfqpoint{5.072303in}{2.671127in}}%
\pgfpathlineto{\pgfqpoint{5.086109in}{2.668715in}}%
\pgfpathlineto{\pgfqpoint{5.099924in}{2.666372in}}%
\pgfpathlineto{\pgfqpoint{5.107440in}{2.677371in}}%
\pgfpathlineto{\pgfqpoint{5.114955in}{2.688589in}}%
\pgfpathlineto{\pgfqpoint{5.122471in}{2.700036in}}%
\pgfpathlineto{\pgfqpoint{5.129987in}{2.711717in}}%
\pgfpathlineto{\pgfqpoint{5.116190in}{2.714460in}}%
\pgfpathlineto{\pgfqpoint{5.102400in}{2.717272in}}%
\pgfpathlineto{\pgfqpoint{5.088618in}{2.720153in}}%
\pgfpathlineto{\pgfqpoint{5.074845in}{2.723103in}}%
\pgfpathlineto{\pgfqpoint{5.067312in}{2.711015in}}%
\pgfpathlineto{\pgfqpoint{5.059780in}{2.699166in}}%
\pgfpathlineto{\pgfqpoint{5.052247in}{2.687549in}}%
\pgfpathlineto{\pgfqpoint{5.044715in}{2.676158in}}%
\pgfpathclose%
\pgfusepath{fill}%
\end{pgfscope}%
\begin{pgfscope}%
\pgfpathrectangle{\pgfqpoint{1.150000in}{0.150000in}}{\pgfqpoint{5.700000in}{5.700000in}}%
\pgfusepath{clip}%
\pgfsetbuttcap%
\pgfsetroundjoin%
\definecolor{currentfill}{rgb}{0.282656,0.100196,0.422160}%
\pgfsetfillcolor{currentfill}%
\pgfsetfillopacity{0.700000}%
\pgfsetlinewidth{0.000000pt}%
\definecolor{currentstroke}{rgb}{0.000000,0.000000,0.000000}%
\pgfsetstrokecolor{currentstroke}%
\pgfsetdash{}{0pt}%
\pgfpathmoveto{\pgfqpoint{3.802207in}{2.383713in}}%
\pgfpathlineto{\pgfqpoint{3.815699in}{2.379536in}}%
\pgfpathlineto{\pgfqpoint{3.829197in}{2.375444in}}%
\pgfpathlineto{\pgfqpoint{3.842700in}{2.371437in}}%
\pgfpathlineto{\pgfqpoint{3.856209in}{2.367514in}}%
\pgfpathlineto{\pgfqpoint{3.864110in}{2.377127in}}%
\pgfpathlineto{\pgfqpoint{3.872004in}{2.386797in}}%
\pgfpathlineto{\pgfqpoint{3.879894in}{2.396526in}}%
\pgfpathlineto{\pgfqpoint{3.887778in}{2.406319in}}%
\pgfpathlineto{\pgfqpoint{3.874279in}{2.410401in}}%
\pgfpathlineto{\pgfqpoint{3.860786in}{2.414566in}}%
\pgfpathlineto{\pgfqpoint{3.847298in}{2.418816in}}%
\pgfpathlineto{\pgfqpoint{3.833816in}{2.423151in}}%
\pgfpathlineto{\pgfqpoint{3.825922in}{2.413193in}}%
\pgfpathlineto{\pgfqpoint{3.818022in}{2.403302in}}%
\pgfpathlineto{\pgfqpoint{3.810117in}{2.393477in}}%
\pgfpathlineto{\pgfqpoint{3.802207in}{2.383713in}}%
\pgfpathclose%
\pgfusepath{fill}%
\end{pgfscope}%
\begin{pgfscope}%
\pgfpathrectangle{\pgfqpoint{1.150000in}{0.150000in}}{\pgfqpoint{5.700000in}{5.700000in}}%
\pgfusepath{clip}%
\pgfsetbuttcap%
\pgfsetroundjoin%
\definecolor{currentfill}{rgb}{0.283197,0.115680,0.436115}%
\pgfsetfillcolor{currentfill}%
\pgfsetfillopacity{0.700000}%
\pgfsetlinewidth{0.000000pt}%
\definecolor{currentstroke}{rgb}{0.000000,0.000000,0.000000}%
\pgfsetstrokecolor{currentstroke}%
\pgfsetdash{}{0pt}%
\pgfpathmoveto{\pgfqpoint{4.027376in}{2.415095in}}%
\pgfpathlineto{\pgfqpoint{4.040918in}{2.411576in}}%
\pgfpathlineto{\pgfqpoint{4.054465in}{2.408138in}}%
\pgfpathlineto{\pgfqpoint{4.068019in}{2.404781in}}%
\pgfpathlineto{\pgfqpoint{4.081580in}{2.401504in}}%
\pgfpathlineto{\pgfqpoint{4.089406in}{2.411089in}}%
\pgfpathlineto{\pgfqpoint{4.097227in}{2.420743in}}%
\pgfpathlineto{\pgfqpoint{4.105043in}{2.430470in}}%
\pgfpathlineto{\pgfqpoint{4.112853in}{2.440273in}}%
\pgfpathlineto{\pgfqpoint{4.099304in}{2.443749in}}%
\pgfpathlineto{\pgfqpoint{4.085760in}{2.447305in}}%
\pgfpathlineto{\pgfqpoint{4.072223in}{2.450942in}}%
\pgfpathlineto{\pgfqpoint{4.058692in}{2.454659in}}%
\pgfpathlineto{\pgfqpoint{4.050871in}{2.444650in}}%
\pgfpathlineto{\pgfqpoint{4.043044in}{2.434722in}}%
\pgfpathlineto{\pgfqpoint{4.035213in}{2.424872in}}%
\pgfpathlineto{\pgfqpoint{4.027376in}{2.415095in}}%
\pgfpathclose%
\pgfusepath{fill}%
\end{pgfscope}%
\begin{pgfscope}%
\pgfpathrectangle{\pgfqpoint{1.150000in}{0.150000in}}{\pgfqpoint{5.700000in}{5.700000in}}%
\pgfusepath{clip}%
\pgfsetbuttcap%
\pgfsetroundjoin%
\definecolor{currentfill}{rgb}{0.281924,0.089666,0.412415}%
\pgfsetfillcolor{currentfill}%
\pgfsetfillopacity{0.700000}%
\pgfsetlinewidth{0.000000pt}%
\definecolor{currentstroke}{rgb}{0.000000,0.000000,0.000000}%
\pgfsetstrokecolor{currentstroke}%
\pgfsetdash{}{0pt}%
\pgfpathmoveto{\pgfqpoint{3.576972in}{2.361316in}}%
\pgfpathlineto{\pgfqpoint{3.590425in}{2.356343in}}%
\pgfpathlineto{\pgfqpoint{3.603881in}{2.351461in}}%
\pgfpathlineto{\pgfqpoint{3.617343in}{2.346668in}}%
\pgfpathlineto{\pgfqpoint{3.630809in}{2.341964in}}%
\pgfpathlineto{\pgfqpoint{3.638785in}{2.351554in}}%
\pgfpathlineto{\pgfqpoint{3.646756in}{2.361194in}}%
\pgfpathlineto{\pgfqpoint{3.654720in}{2.370887in}}%
\pgfpathlineto{\pgfqpoint{3.662679in}{2.380635in}}%
\pgfpathlineto{\pgfqpoint{3.649223in}{2.385457in}}%
\pgfpathlineto{\pgfqpoint{3.635771in}{2.390367in}}%
\pgfpathlineto{\pgfqpoint{3.622324in}{2.395367in}}%
\pgfpathlineto{\pgfqpoint{3.608882in}{2.400457in}}%
\pgfpathlineto{\pgfqpoint{3.600913in}{2.390584in}}%
\pgfpathlineto{\pgfqpoint{3.592939in}{2.380771in}}%
\pgfpathlineto{\pgfqpoint{3.584959in}{2.371016in}}%
\pgfpathlineto{\pgfqpoint{3.576972in}{2.361316in}}%
\pgfpathclose%
\pgfusepath{fill}%
\end{pgfscope}%
\begin{pgfscope}%
\pgfpathrectangle{\pgfqpoint{1.150000in}{0.150000in}}{\pgfqpoint{5.700000in}{5.700000in}}%
\pgfusepath{clip}%
\pgfsetbuttcap%
\pgfsetroundjoin%
\definecolor{currentfill}{rgb}{0.187231,0.414746,0.556547}%
\pgfsetfillcolor{currentfill}%
\pgfsetfillopacity{0.700000}%
\pgfsetlinewidth{0.000000pt}%
\definecolor{currentstroke}{rgb}{0.000000,0.000000,0.000000}%
\pgfsetstrokecolor{currentstroke}%
\pgfsetdash{}{0pt}%
\pgfpathmoveto{\pgfqpoint{5.813738in}{3.080373in}}%
\pgfpathlineto{\pgfqpoint{5.827650in}{3.075906in}}%
\pgfpathlineto{\pgfqpoint{5.841570in}{3.071504in}}%
\pgfpathlineto{\pgfqpoint{5.855498in}{3.067166in}}%
\pgfpathlineto{\pgfqpoint{5.869434in}{3.062893in}}%
\pgfpathlineto{\pgfqpoint{5.876987in}{3.081885in}}%
\pgfpathlineto{\pgfqpoint{5.884555in}{3.101374in}}%
\pgfpathlineto{\pgfqpoint{5.892138in}{3.121369in}}%
\pgfpathlineto{\pgfqpoint{5.878218in}{3.126076in}}%
\pgfpathlineto{\pgfqpoint{5.864306in}{3.130848in}}%
\pgfpathlineto{\pgfqpoint{5.850402in}{3.135685in}}%
\pgfpathlineto{\pgfqpoint{5.836505in}{3.140586in}}%
\pgfpathlineto{\pgfqpoint{5.828901in}{3.120007in}}%
\pgfpathlineto{\pgfqpoint{5.821312in}{3.099939in}}%
\pgfpathlineto{\pgfqpoint{5.813738in}{3.080373in}}%
\pgfpathclose%
\pgfusepath{fill}%
\end{pgfscope}%
\begin{pgfscope}%
\pgfpathrectangle{\pgfqpoint{1.150000in}{0.150000in}}{\pgfqpoint{5.700000in}{5.700000in}}%
\pgfusepath{clip}%
\pgfsetbuttcap%
\pgfsetroundjoin%
\definecolor{currentfill}{rgb}{0.266580,0.228262,0.514349}%
\pgfsetfillcolor{currentfill}%
\pgfsetfillopacity{0.700000}%
\pgfsetlinewidth{0.000000pt}%
\definecolor{currentstroke}{rgb}{0.000000,0.000000,0.000000}%
\pgfsetstrokecolor{currentstroke}%
\pgfsetdash{}{0pt}%
\pgfpathmoveto{\pgfqpoint{4.959434in}{2.642079in}}%
\pgfpathlineto{\pgfqpoint{4.973208in}{2.639631in}}%
\pgfpathlineto{\pgfqpoint{4.986990in}{2.637253in}}%
\pgfpathlineto{\pgfqpoint{5.000780in}{2.634945in}}%
\pgfpathlineto{\pgfqpoint{5.014579in}{2.632706in}}%
\pgfpathlineto{\pgfqpoint{5.022114in}{2.643266in}}%
\pgfpathlineto{\pgfqpoint{5.029648in}{2.654023in}}%
\pgfpathlineto{\pgfqpoint{5.037182in}{2.664985in}}%
\pgfpathlineto{\pgfqpoint{5.044715in}{2.676158in}}%
\pgfpathlineto{\pgfqpoint{5.030933in}{2.678778in}}%
\pgfpathlineto{\pgfqpoint{5.017159in}{2.681466in}}%
\pgfpathlineto{\pgfqpoint{5.003393in}{2.684225in}}%
\pgfpathlineto{\pgfqpoint{4.989635in}{2.687052in}}%
\pgfpathlineto{\pgfqpoint{4.982086in}{2.675492in}}%
\pgfpathlineto{\pgfqpoint{4.974536in}{2.664148in}}%
\pgfpathlineto{\pgfqpoint{4.966986in}{2.653012in}}%
\pgfpathlineto{\pgfqpoint{4.959434in}{2.642079in}}%
\pgfpathclose%
\pgfusepath{fill}%
\end{pgfscope}%
\begin{pgfscope}%
\pgfpathrectangle{\pgfqpoint{1.150000in}{0.150000in}}{\pgfqpoint{5.700000in}{5.700000in}}%
\pgfusepath{clip}%
\pgfsetbuttcap%
\pgfsetroundjoin%
\definecolor{currentfill}{rgb}{0.278826,0.175490,0.483397}%
\pgfsetfillcolor{currentfill}%
\pgfsetfillopacity{0.700000}%
\pgfsetlinewidth{0.000000pt}%
\definecolor{currentstroke}{rgb}{0.000000,0.000000,0.000000}%
\pgfsetstrokecolor{currentstroke}%
\pgfsetdash{}{0pt}%
\pgfpathmoveto{\pgfqpoint{4.563303in}{2.526895in}}%
\pgfpathlineto{\pgfqpoint{4.576979in}{2.524335in}}%
\pgfpathlineto{\pgfqpoint{4.590662in}{2.521848in}}%
\pgfpathlineto{\pgfqpoint{4.604353in}{2.519435in}}%
\pgfpathlineto{\pgfqpoint{4.618051in}{2.517094in}}%
\pgfpathlineto{\pgfqpoint{4.625702in}{2.526819in}}%
\pgfpathlineto{\pgfqpoint{4.633350in}{2.536669in}}%
\pgfpathlineto{\pgfqpoint{4.640994in}{2.546649in}}%
\pgfpathlineto{\pgfqpoint{4.648634in}{2.556765in}}%
\pgfpathlineto{\pgfqpoint{4.634949in}{2.559405in}}%
\pgfpathlineto{\pgfqpoint{4.621272in}{2.562118in}}%
\pgfpathlineto{\pgfqpoint{4.607602in}{2.564905in}}%
\pgfpathlineto{\pgfqpoint{4.593939in}{2.567765in}}%
\pgfpathlineto{\pgfqpoint{4.586285in}{2.557342in}}%
\pgfpathlineto{\pgfqpoint{4.578628in}{2.547060in}}%
\pgfpathlineto{\pgfqpoint{4.570967in}{2.536913in}}%
\pgfpathlineto{\pgfqpoint{4.563303in}{2.526895in}}%
\pgfpathclose%
\pgfusepath{fill}%
\end{pgfscope}%
\begin{pgfscope}%
\pgfpathrectangle{\pgfqpoint{1.150000in}{0.150000in}}{\pgfqpoint{5.700000in}{5.700000in}}%
\pgfusepath{clip}%
\pgfsetbuttcap%
\pgfsetroundjoin%
\definecolor{currentfill}{rgb}{0.282623,0.140926,0.457517}%
\pgfsetfillcolor{currentfill}%
\pgfsetfillopacity{0.700000}%
\pgfsetlinewidth{0.000000pt}%
\definecolor{currentstroke}{rgb}{0.000000,0.000000,0.000000}%
\pgfsetstrokecolor{currentstroke}%
\pgfsetdash{}{0pt}%
\pgfpathmoveto{\pgfqpoint{4.252589in}{2.453614in}}%
\pgfpathlineto{\pgfqpoint{4.266187in}{2.450626in}}%
\pgfpathlineto{\pgfqpoint{4.279791in}{2.447716in}}%
\pgfpathlineto{\pgfqpoint{4.293403in}{2.444882in}}%
\pgfpathlineto{\pgfqpoint{4.307021in}{2.442125in}}%
\pgfpathlineto{\pgfqpoint{4.314774in}{2.451676in}}%
\pgfpathlineto{\pgfqpoint{4.322522in}{2.461312in}}%
\pgfpathlineto{\pgfqpoint{4.330265in}{2.471039in}}%
\pgfpathlineto{\pgfqpoint{4.338004in}{2.480861in}}%
\pgfpathlineto{\pgfqpoint{4.324397in}{2.483857in}}%
\pgfpathlineto{\pgfqpoint{4.310797in}{2.486930in}}%
\pgfpathlineto{\pgfqpoint{4.297204in}{2.490079in}}%
\pgfpathlineto{\pgfqpoint{4.283618in}{2.493306in}}%
\pgfpathlineto{\pgfqpoint{4.275868in}{2.483238in}}%
\pgfpathlineto{\pgfqpoint{4.268113in}{2.473269in}}%
\pgfpathlineto{\pgfqpoint{4.260353in}{2.463396in}}%
\pgfpathlineto{\pgfqpoint{4.252589in}{2.453614in}}%
\pgfpathclose%
\pgfusepath{fill}%
\end{pgfscope}%
\begin{pgfscope}%
\pgfpathrectangle{\pgfqpoint{1.150000in}{0.150000in}}{\pgfqpoint{5.700000in}{5.700000in}}%
\pgfusepath{clip}%
\pgfsetbuttcap%
\pgfsetroundjoin%
\definecolor{currentfill}{rgb}{0.282290,0.145912,0.461510}%
\pgfsetfillcolor{currentfill}%
\pgfsetfillopacity{0.700000}%
\pgfsetlinewidth{0.000000pt}%
\definecolor{currentstroke}{rgb}{0.000000,0.000000,0.000000}%
\pgfsetstrokecolor{currentstroke}%
\pgfsetdash{}{0pt}%
\pgfpathmoveto{\pgfqpoint{2.738230in}{2.482884in}}%
\pgfpathlineto{\pgfqpoint{2.751638in}{2.473132in}}%
\pgfpathlineto{\pgfqpoint{2.765046in}{2.463505in}}%
\pgfpathlineto{\pgfqpoint{2.778454in}{2.454001in}}%
\pgfpathlineto{\pgfqpoint{2.791862in}{2.444620in}}%
\pgfpathlineto{\pgfqpoint{2.800137in}{2.453369in}}%
\pgfpathlineto{\pgfqpoint{2.808405in}{2.462196in}}%
\pgfpathlineto{\pgfqpoint{2.816664in}{2.471101in}}%
\pgfpathlineto{\pgfqpoint{2.824916in}{2.480086in}}%
\pgfpathlineto{\pgfqpoint{2.811523in}{2.489462in}}%
\pgfpathlineto{\pgfqpoint{2.798129in}{2.498960in}}%
\pgfpathlineto{\pgfqpoint{2.784736in}{2.508582in}}%
\pgfpathlineto{\pgfqpoint{2.771342in}{2.518329in}}%
\pgfpathlineto{\pgfqpoint{2.763076in}{2.509342in}}%
\pgfpathlineto{\pgfqpoint{2.754802in}{2.500439in}}%
\pgfpathlineto{\pgfqpoint{2.746520in}{2.491620in}}%
\pgfpathlineto{\pgfqpoint{2.738230in}{2.482884in}}%
\pgfpathclose%
\pgfusepath{fill}%
\end{pgfscope}%
\begin{pgfscope}%
\pgfpathrectangle{\pgfqpoint{1.150000in}{0.150000in}}{\pgfqpoint{5.700000in}{5.700000in}}%
\pgfusepath{clip}%
\pgfsetbuttcap%
\pgfsetroundjoin%
\definecolor{currentfill}{rgb}{0.281924,0.089666,0.412415}%
\pgfsetfillcolor{currentfill}%
\pgfsetfillopacity{0.700000}%
\pgfsetlinewidth{0.000000pt}%
\definecolor{currentstroke}{rgb}{0.000000,0.000000,0.000000}%
\pgfsetstrokecolor{currentstroke}%
\pgfsetdash{}{0pt}%
\pgfpathmoveto{\pgfqpoint{3.211841in}{2.362992in}}%
\pgfpathlineto{\pgfqpoint{3.225250in}{2.356339in}}%
\pgfpathlineto{\pgfqpoint{3.238662in}{2.349788in}}%
\pgfpathlineto{\pgfqpoint{3.252077in}{2.343337in}}%
\pgfpathlineto{\pgfqpoint{3.265495in}{2.336986in}}%
\pgfpathlineto{\pgfqpoint{3.273597in}{2.346376in}}%
\pgfpathlineto{\pgfqpoint{3.281692in}{2.355820in}}%
\pgfpathlineto{\pgfqpoint{3.289782in}{2.365318in}}%
\pgfpathlineto{\pgfqpoint{3.297864in}{2.374872in}}%
\pgfpathlineto{\pgfqpoint{3.284458in}{2.381280in}}%
\pgfpathlineto{\pgfqpoint{3.271054in}{2.387788in}}%
\pgfpathlineto{\pgfqpoint{3.257653in}{2.394396in}}%
\pgfpathlineto{\pgfqpoint{3.244256in}{2.401105in}}%
\pgfpathlineto{\pgfqpoint{3.236162in}{2.391487in}}%
\pgfpathlineto{\pgfqpoint{3.228061in}{2.381930in}}%
\pgfpathlineto{\pgfqpoint{3.219955in}{2.372432in}}%
\pgfpathlineto{\pgfqpoint{3.211841in}{2.362992in}}%
\pgfpathclose%
\pgfusepath{fill}%
\end{pgfscope}%
\begin{pgfscope}%
\pgfpathrectangle{\pgfqpoint{1.150000in}{0.150000in}}{\pgfqpoint{5.700000in}{5.700000in}}%
\pgfusepath{clip}%
\pgfsetbuttcap%
\pgfsetroundjoin%
\definecolor{currentfill}{rgb}{0.282656,0.100196,0.422160}%
\pgfsetfillcolor{currentfill}%
\pgfsetfillopacity{0.700000}%
\pgfsetlinewidth{0.000000pt}%
\definecolor{currentstroke}{rgb}{0.000000,0.000000,0.000000}%
\pgfsetstrokecolor{currentstroke}%
\pgfsetdash{}{0pt}%
\pgfpathmoveto{\pgfqpoint{3.072051in}{2.382445in}}%
\pgfpathlineto{\pgfqpoint{3.085452in}{2.374994in}}%
\pgfpathlineto{\pgfqpoint{3.098855in}{2.367650in}}%
\pgfpathlineto{\pgfqpoint{3.112260in}{2.360413in}}%
\pgfpathlineto{\pgfqpoint{3.125668in}{2.353280in}}%
\pgfpathlineto{\pgfqpoint{3.133819in}{2.362529in}}%
\pgfpathlineto{\pgfqpoint{3.141964in}{2.371835in}}%
\pgfpathlineto{\pgfqpoint{3.150102in}{2.381201in}}%
\pgfpathlineto{\pgfqpoint{3.158233in}{2.390628in}}%
\pgfpathlineto{\pgfqpoint{3.144838in}{2.397797in}}%
\pgfpathlineto{\pgfqpoint{3.131444in}{2.405071in}}%
\pgfpathlineto{\pgfqpoint{3.118053in}{2.412451in}}%
\pgfpathlineto{\pgfqpoint{3.104665in}{2.419937in}}%
\pgfpathlineto{\pgfqpoint{3.096522in}{2.410467in}}%
\pgfpathlineto{\pgfqpoint{3.088372in}{2.401062in}}%
\pgfpathlineto{\pgfqpoint{3.080215in}{2.391722in}}%
\pgfpathlineto{\pgfqpoint{3.072051in}{2.382445in}}%
\pgfpathclose%
\pgfusepath{fill}%
\end{pgfscope}%
\begin{pgfscope}%
\pgfpathrectangle{\pgfqpoint{1.150000in}{0.150000in}}{\pgfqpoint{5.700000in}{5.700000in}}%
\pgfusepath{clip}%
\pgfsetbuttcap%
\pgfsetroundjoin%
\definecolor{currentfill}{rgb}{0.281446,0.084320,0.407414}%
\pgfsetfillcolor{currentfill}%
\pgfsetfillopacity{0.700000}%
\pgfsetlinewidth{0.000000pt}%
\definecolor{currentstroke}{rgb}{0.000000,0.000000,0.000000}%
\pgfsetstrokecolor{currentstroke}%
\pgfsetdash{}{0pt}%
\pgfpathmoveto{\pgfqpoint{3.351525in}{2.350227in}}%
\pgfpathlineto{\pgfqpoint{3.364949in}{2.344309in}}%
\pgfpathlineto{\pgfqpoint{3.378376in}{2.338488in}}%
\pgfpathlineto{\pgfqpoint{3.391807in}{2.332762in}}%
\pgfpathlineto{\pgfqpoint{3.405243in}{2.327132in}}%
\pgfpathlineto{\pgfqpoint{3.413297in}{2.336605in}}%
\pgfpathlineto{\pgfqpoint{3.421346in}{2.346127in}}%
\pgfpathlineto{\pgfqpoint{3.429389in}{2.355700in}}%
\pgfpathlineto{\pgfqpoint{3.437426in}{2.365325in}}%
\pgfpathlineto{\pgfqpoint{3.424001in}{2.371033in}}%
\pgfpathlineto{\pgfqpoint{3.410581in}{2.376836in}}%
\pgfpathlineto{\pgfqpoint{3.397164in}{2.382734in}}%
\pgfpathlineto{\pgfqpoint{3.383751in}{2.388729in}}%
\pgfpathlineto{\pgfqpoint{3.375703in}{2.379019in}}%
\pgfpathlineto{\pgfqpoint{3.367650in}{2.369366in}}%
\pgfpathlineto{\pgfqpoint{3.359591in}{2.359769in}}%
\pgfpathlineto{\pgfqpoint{3.351525in}{2.350227in}}%
\pgfpathclose%
\pgfusepath{fill}%
\end{pgfscope}%
\begin{pgfscope}%
\pgfpathrectangle{\pgfqpoint{1.150000in}{0.150000in}}{\pgfqpoint{5.700000in}{5.700000in}}%
\pgfusepath{clip}%
\pgfsetbuttcap%
\pgfsetroundjoin%
\definecolor{currentfill}{rgb}{0.225863,0.330805,0.547314}%
\pgfsetfillcolor{currentfill}%
\pgfsetfillopacity{0.700000}%
\pgfsetlinewidth{0.000000pt}%
\definecolor{currentstroke}{rgb}{0.000000,0.000000,0.000000}%
\pgfsetstrokecolor{currentstroke}%
\pgfsetdash{}{0pt}%
\pgfpathmoveto{\pgfqpoint{5.526765in}{2.860227in}}%
\pgfpathlineto{\pgfqpoint{5.540662in}{2.857097in}}%
\pgfpathlineto{\pgfqpoint{5.554567in}{2.854033in}}%
\pgfpathlineto{\pgfqpoint{5.568480in}{2.851035in}}%
\pgfpathlineto{\pgfqpoint{5.582403in}{2.848103in}}%
\pgfpathlineto{\pgfqpoint{5.589858in}{2.862027in}}%
\pgfpathlineto{\pgfqpoint{5.597321in}{2.876303in}}%
\pgfpathlineto{\pgfqpoint{5.604790in}{2.890941in}}%
\pgfpathlineto{\pgfqpoint{5.612267in}{2.905949in}}%
\pgfpathlineto{\pgfqpoint{5.598365in}{2.909381in}}%
\pgfpathlineto{\pgfqpoint{5.584472in}{2.912879in}}%
\pgfpathlineto{\pgfqpoint{5.570588in}{2.916443in}}%
\pgfpathlineto{\pgfqpoint{5.556711in}{2.920073in}}%
\pgfpathlineto{\pgfqpoint{5.549214in}{2.904558in}}%
\pgfpathlineto{\pgfqpoint{5.541724in}{2.889417in}}%
\pgfpathlineto{\pgfqpoint{5.534242in}{2.874643in}}%
\pgfpathlineto{\pgfqpoint{5.526765in}{2.860227in}}%
\pgfpathclose%
\pgfusepath{fill}%
\end{pgfscope}%
\begin{pgfscope}%
\pgfpathrectangle{\pgfqpoint{1.150000in}{0.150000in}}{\pgfqpoint{5.700000in}{5.700000in}}%
\pgfusepath{clip}%
\pgfsetbuttcap%
\pgfsetroundjoin%
\definecolor{currentfill}{rgb}{0.269308,0.218818,0.509577}%
\pgfsetfillcolor{currentfill}%
\pgfsetfillopacity{0.700000}%
\pgfsetlinewidth{0.000000pt}%
\definecolor{currentstroke}{rgb}{0.000000,0.000000,0.000000}%
\pgfsetstrokecolor{currentstroke}%
\pgfsetdash{}{0pt}%
\pgfpathmoveto{\pgfqpoint{4.874135in}{2.609290in}}%
\pgfpathlineto{\pgfqpoint{4.887893in}{2.606921in}}%
\pgfpathlineto{\pgfqpoint{4.901658in}{2.604623in}}%
\pgfpathlineto{\pgfqpoint{4.915432in}{2.602395in}}%
\pgfpathlineto{\pgfqpoint{4.929213in}{2.600238in}}%
\pgfpathlineto{\pgfqpoint{4.936771in}{2.610428in}}%
\pgfpathlineto{\pgfqpoint{4.944327in}{2.620794in}}%
\pgfpathlineto{\pgfqpoint{4.951881in}{2.631342in}}%
\pgfpathlineto{\pgfqpoint{4.959434in}{2.642079in}}%
\pgfpathlineto{\pgfqpoint{4.945668in}{2.644597in}}%
\pgfpathlineto{\pgfqpoint{4.931910in}{2.647185in}}%
\pgfpathlineto{\pgfqpoint{4.918160in}{2.649844in}}%
\pgfpathlineto{\pgfqpoint{4.904418in}{2.652572in}}%
\pgfpathlineto{\pgfqpoint{4.896850in}{2.641467in}}%
\pgfpathlineto{\pgfqpoint{4.889280in}{2.630556in}}%
\pgfpathlineto{\pgfqpoint{4.881709in}{2.619832in}}%
\pgfpathlineto{\pgfqpoint{4.874135in}{2.609290in}}%
\pgfpathclose%
\pgfusepath{fill}%
\end{pgfscope}%
\begin{pgfscope}%
\pgfpathrectangle{\pgfqpoint{1.150000in}{0.150000in}}{\pgfqpoint{5.700000in}{5.700000in}}%
\pgfusepath{clip}%
\pgfsetbuttcap%
\pgfsetroundjoin%
\definecolor{currentfill}{rgb}{0.216210,0.351535,0.550627}%
\pgfsetfillcolor{currentfill}%
\pgfsetfillopacity{0.700000}%
\pgfsetlinewidth{0.000000pt}%
\definecolor{currentstroke}{rgb}{0.000000,0.000000,0.000000}%
\pgfsetstrokecolor{currentstroke}%
\pgfsetdash{}{0pt}%
\pgfpathmoveto{\pgfqpoint{5.612267in}{2.905949in}}%
\pgfpathlineto{\pgfqpoint{5.626176in}{2.902582in}}%
\pgfpathlineto{\pgfqpoint{5.640094in}{2.899281in}}%
\pgfpathlineto{\pgfqpoint{5.654021in}{2.896045in}}%
\pgfpathlineto{\pgfqpoint{5.667955in}{2.892875in}}%
\pgfpathlineto{\pgfqpoint{5.675419in}{2.907750in}}%
\pgfpathlineto{\pgfqpoint{5.682890in}{2.923009in}}%
\pgfpathlineto{\pgfqpoint{5.690371in}{2.938662in}}%
\pgfpathlineto{\pgfqpoint{5.697861in}{2.954719in}}%
\pgfpathlineto{\pgfqpoint{5.683947in}{2.958410in}}%
\pgfpathlineto{\pgfqpoint{5.670042in}{2.962166in}}%
\pgfpathlineto{\pgfqpoint{5.656145in}{2.965988in}}%
\pgfpathlineto{\pgfqpoint{5.642256in}{2.969875in}}%
\pgfpathlineto{\pgfqpoint{5.634745in}{2.953290in}}%
\pgfpathlineto{\pgfqpoint{5.627244in}{2.937114in}}%
\pgfpathlineto{\pgfqpoint{5.619751in}{2.921337in}}%
\pgfpathlineto{\pgfqpoint{5.612267in}{2.905949in}}%
\pgfpathclose%
\pgfusepath{fill}%
\end{pgfscope}%
\begin{pgfscope}%
\pgfpathrectangle{\pgfqpoint{1.150000in}{0.150000in}}{\pgfqpoint{5.700000in}{5.700000in}}%
\pgfusepath{clip}%
\pgfsetbuttcap%
\pgfsetroundjoin%
\definecolor{currentfill}{rgb}{0.233603,0.313828,0.543914}%
\pgfsetfillcolor{currentfill}%
\pgfsetfillopacity{0.700000}%
\pgfsetlinewidth{0.000000pt}%
\definecolor{currentstroke}{rgb}{0.000000,0.000000,0.000000}%
\pgfsetstrokecolor{currentstroke}%
\pgfsetdash{}{0pt}%
\pgfpathmoveto{\pgfqpoint{5.441333in}{2.817215in}}%
\pgfpathlineto{\pgfqpoint{5.455216in}{2.814300in}}%
\pgfpathlineto{\pgfqpoint{5.469108in}{2.811452in}}%
\pgfpathlineto{\pgfqpoint{5.483008in}{2.808670in}}%
\pgfpathlineto{\pgfqpoint{5.496916in}{2.805955in}}%
\pgfpathlineto{\pgfqpoint{5.504371in}{2.819031in}}%
\pgfpathlineto{\pgfqpoint{5.511830in}{2.832429in}}%
\pgfpathlineto{\pgfqpoint{5.519295in}{2.846158in}}%
\pgfpathlineto{\pgfqpoint{5.526765in}{2.860227in}}%
\pgfpathlineto{\pgfqpoint{5.512877in}{2.863423in}}%
\pgfpathlineto{\pgfqpoint{5.498997in}{2.866685in}}%
\pgfpathlineto{\pgfqpoint{5.485125in}{2.870013in}}%
\pgfpathlineto{\pgfqpoint{5.471262in}{2.873408in}}%
\pgfpathlineto{\pgfqpoint{5.463771in}{2.858852in}}%
\pgfpathlineto{\pgfqpoint{5.456287in}{2.844640in}}%
\pgfpathlineto{\pgfqpoint{5.448807in}{2.830764in}}%
\pgfpathlineto{\pgfqpoint{5.441333in}{2.817215in}}%
\pgfpathclose%
\pgfusepath{fill}%
\end{pgfscope}%
\begin{pgfscope}%
\pgfpathrectangle{\pgfqpoint{1.150000in}{0.150000in}}{\pgfqpoint{5.700000in}{5.700000in}}%
\pgfusepath{clip}%
\pgfsetbuttcap%
\pgfsetroundjoin%
\definecolor{currentfill}{rgb}{0.281924,0.089666,0.412415}%
\pgfsetfillcolor{currentfill}%
\pgfsetfillopacity{0.700000}%
\pgfsetlinewidth{0.000000pt}%
\definecolor{currentstroke}{rgb}{0.000000,0.000000,0.000000}%
\pgfsetstrokecolor{currentstroke}%
\pgfsetdash{}{0pt}%
\pgfpathmoveto{\pgfqpoint{3.716553in}{2.362232in}}%
\pgfpathlineto{\pgfqpoint{3.730034in}{2.357849in}}%
\pgfpathlineto{\pgfqpoint{3.743521in}{2.353553in}}%
\pgfpathlineto{\pgfqpoint{3.757012in}{2.349343in}}%
\pgfpathlineto{\pgfqpoint{3.770509in}{2.345219in}}%
\pgfpathlineto{\pgfqpoint{3.778442in}{2.354764in}}%
\pgfpathlineto{\pgfqpoint{3.786369in}{2.364360in}}%
\pgfpathlineto{\pgfqpoint{3.794291in}{2.374009in}}%
\pgfpathlineto{\pgfqpoint{3.802207in}{2.383713in}}%
\pgfpathlineto{\pgfqpoint{3.788720in}{2.387976in}}%
\pgfpathlineto{\pgfqpoint{3.775238in}{2.392323in}}%
\pgfpathlineto{\pgfqpoint{3.761762in}{2.396757in}}%
\pgfpathlineto{\pgfqpoint{3.748291in}{2.401278in}}%
\pgfpathlineto{\pgfqpoint{3.740365in}{2.391428in}}%
\pgfpathlineto{\pgfqpoint{3.732433in}{2.381638in}}%
\pgfpathlineto{\pgfqpoint{3.724496in}{2.371907in}}%
\pgfpathlineto{\pgfqpoint{3.716553in}{2.362232in}}%
\pgfpathclose%
\pgfusepath{fill}%
\end{pgfscope}%
\begin{pgfscope}%
\pgfpathrectangle{\pgfqpoint{1.150000in}{0.150000in}}{\pgfqpoint{5.700000in}{5.700000in}}%
\pgfusepath{clip}%
\pgfsetbuttcap%
\pgfsetroundjoin%
\definecolor{currentfill}{rgb}{0.283091,0.110553,0.431554}%
\pgfsetfillcolor{currentfill}%
\pgfsetfillopacity{0.700000}%
\pgfsetlinewidth{0.000000pt}%
\definecolor{currentstroke}{rgb}{0.000000,0.000000,0.000000}%
\pgfsetstrokecolor{currentstroke}%
\pgfsetdash{}{0pt}%
\pgfpathmoveto{\pgfqpoint{2.932085in}{2.409368in}}%
\pgfpathlineto{\pgfqpoint{2.945485in}{2.401050in}}%
\pgfpathlineto{\pgfqpoint{2.958887in}{2.392846in}}%
\pgfpathlineto{\pgfqpoint{2.972290in}{2.384755in}}%
\pgfpathlineto{\pgfqpoint{2.985694in}{2.376775in}}%
\pgfpathlineto{\pgfqpoint{2.993898in}{2.385815in}}%
\pgfpathlineto{\pgfqpoint{3.002095in}{2.394921in}}%
\pgfpathlineto{\pgfqpoint{3.010285in}{2.404092in}}%
\pgfpathlineto{\pgfqpoint{3.018468in}{2.413330in}}%
\pgfpathlineto{\pgfqpoint{3.005076in}{2.421326in}}%
\pgfpathlineto{\pgfqpoint{2.991686in}{2.429433in}}%
\pgfpathlineto{\pgfqpoint{2.978298in}{2.437653in}}%
\pgfpathlineto{\pgfqpoint{2.964911in}{2.445986in}}%
\pgfpathlineto{\pgfqpoint{2.956715in}{2.436724in}}%
\pgfpathlineto{\pgfqpoint{2.948512in}{2.427535in}}%
\pgfpathlineto{\pgfqpoint{2.940302in}{2.418416in}}%
\pgfpathlineto{\pgfqpoint{2.932085in}{2.409368in}}%
\pgfpathclose%
\pgfusepath{fill}%
\end{pgfscope}%
\begin{pgfscope}%
\pgfpathrectangle{\pgfqpoint{1.150000in}{0.150000in}}{\pgfqpoint{5.700000in}{5.700000in}}%
\pgfusepath{clip}%
\pgfsetbuttcap%
\pgfsetroundjoin%
\definecolor{currentfill}{rgb}{0.276194,0.190074,0.493001}%
\pgfsetfillcolor{currentfill}%
\pgfsetfillopacity{0.700000}%
\pgfsetlinewidth{0.000000pt}%
\definecolor{currentstroke}{rgb}{0.000000,0.000000,0.000000}%
\pgfsetstrokecolor{currentstroke}%
\pgfsetdash{}{0pt}%
\pgfpathmoveto{\pgfqpoint{2.543844in}{2.576216in}}%
\pgfpathlineto{\pgfqpoint{2.557280in}{2.564857in}}%
\pgfpathlineto{\pgfqpoint{2.570714in}{2.553635in}}%
\pgfpathlineto{\pgfqpoint{2.584147in}{2.542551in}}%
\pgfpathlineto{\pgfqpoint{2.597578in}{2.531603in}}%
\pgfpathlineto{\pgfqpoint{2.605933in}{2.539941in}}%
\pgfpathlineto{\pgfqpoint{2.614279in}{2.548371in}}%
\pgfpathlineto{\pgfqpoint{2.622617in}{2.556895in}}%
\pgfpathlineto{\pgfqpoint{2.630945in}{2.565511in}}%
\pgfpathlineto{\pgfqpoint{2.617530in}{2.576433in}}%
\pgfpathlineto{\pgfqpoint{2.604114in}{2.587490in}}%
\pgfpathlineto{\pgfqpoint{2.590696in}{2.598685in}}%
\pgfpathlineto{\pgfqpoint{2.577276in}{2.610018in}}%
\pgfpathlineto{\pgfqpoint{2.568931in}{2.601421in}}%
\pgfpathlineto{\pgfqpoint{2.560577in}{2.592921in}}%
\pgfpathlineto{\pgfqpoint{2.552215in}{2.584520in}}%
\pgfpathlineto{\pgfqpoint{2.543844in}{2.576216in}}%
\pgfpathclose%
\pgfusepath{fill}%
\end{pgfscope}%
\begin{pgfscope}%
\pgfpathrectangle{\pgfqpoint{1.150000in}{0.150000in}}{\pgfqpoint{5.700000in}{5.700000in}}%
\pgfusepath{clip}%
\pgfsetbuttcap%
\pgfsetroundjoin%
\definecolor{currentfill}{rgb}{0.282910,0.105393,0.426902}%
\pgfsetfillcolor{currentfill}%
\pgfsetfillopacity{0.700000}%
\pgfsetlinewidth{0.000000pt}%
\definecolor{currentstroke}{rgb}{0.000000,0.000000,0.000000}%
\pgfsetstrokecolor{currentstroke}%
\pgfsetdash{}{0pt}%
\pgfpathmoveto{\pgfqpoint{3.941830in}{2.390828in}}%
\pgfpathlineto{\pgfqpoint{3.955357in}{2.387161in}}%
\pgfpathlineto{\pgfqpoint{3.968891in}{2.383577in}}%
\pgfpathlineto{\pgfqpoint{3.982431in}{2.380075in}}%
\pgfpathlineto{\pgfqpoint{3.995977in}{2.376654in}}%
\pgfpathlineto{\pgfqpoint{4.003835in}{2.386171in}}%
\pgfpathlineto{\pgfqpoint{4.011687in}{2.395748in}}%
\pgfpathlineto{\pgfqpoint{4.019534in}{2.405388in}}%
\pgfpathlineto{\pgfqpoint{4.027376in}{2.415095in}}%
\pgfpathlineto{\pgfqpoint{4.013841in}{2.418695in}}%
\pgfpathlineto{\pgfqpoint{4.000312in}{2.422375in}}%
\pgfpathlineto{\pgfqpoint{3.986788in}{2.426138in}}%
\pgfpathlineto{\pgfqpoint{3.973271in}{2.429983in}}%
\pgfpathlineto{\pgfqpoint{3.965419in}{2.420090in}}%
\pgfpathlineto{\pgfqpoint{3.957561in}{2.410269in}}%
\pgfpathlineto{\pgfqpoint{3.949698in}{2.400516in}}%
\pgfpathlineto{\pgfqpoint{3.941830in}{2.390828in}}%
\pgfpathclose%
\pgfusepath{fill}%
\end{pgfscope}%
\begin{pgfscope}%
\pgfpathrectangle{\pgfqpoint{1.150000in}{0.150000in}}{\pgfqpoint{5.700000in}{5.700000in}}%
\pgfusepath{clip}%
\pgfsetbuttcap%
\pgfsetroundjoin%
\definecolor{currentfill}{rgb}{0.241237,0.296485,0.539709}%
\pgfsetfillcolor{currentfill}%
\pgfsetfillopacity{0.700000}%
\pgfsetlinewidth{0.000000pt}%
\definecolor{currentstroke}{rgb}{0.000000,0.000000,0.000000}%
\pgfsetstrokecolor{currentstroke}%
\pgfsetdash{}{0pt}%
\pgfpathmoveto{\pgfqpoint{5.355949in}{2.776599in}}%
\pgfpathlineto{\pgfqpoint{5.369818in}{2.773877in}}%
\pgfpathlineto{\pgfqpoint{5.383696in}{2.771223in}}%
\pgfpathlineto{\pgfqpoint{5.397582in}{2.768635in}}%
\pgfpathlineto{\pgfqpoint{5.411476in}{2.766115in}}%
\pgfpathlineto{\pgfqpoint{5.418935in}{2.778442in}}%
\pgfpathlineto{\pgfqpoint{5.426397in}{2.791062in}}%
\pgfpathlineto{\pgfqpoint{5.433863in}{2.803983in}}%
\pgfpathlineto{\pgfqpoint{5.441333in}{2.817215in}}%
\pgfpathlineto{\pgfqpoint{5.427458in}{2.820196in}}%
\pgfpathlineto{\pgfqpoint{5.413592in}{2.823244in}}%
\pgfpathlineto{\pgfqpoint{5.399734in}{2.826359in}}%
\pgfpathlineto{\pgfqpoint{5.385884in}{2.829540in}}%
\pgfpathlineto{\pgfqpoint{5.378394in}{2.815841in}}%
\pgfpathlineto{\pgfqpoint{5.370909in}{2.802457in}}%
\pgfpathlineto{\pgfqpoint{5.363428in}{2.789379in}}%
\pgfpathlineto{\pgfqpoint{5.355949in}{2.776599in}}%
\pgfpathclose%
\pgfusepath{fill}%
\end{pgfscope}%
\begin{pgfscope}%
\pgfpathrectangle{\pgfqpoint{1.150000in}{0.150000in}}{\pgfqpoint{5.700000in}{5.700000in}}%
\pgfusepath{clip}%
\pgfsetbuttcap%
\pgfsetroundjoin%
\definecolor{currentfill}{rgb}{0.206756,0.371758,0.553117}%
\pgfsetfillcolor{currentfill}%
\pgfsetfillopacity{0.700000}%
\pgfsetlinewidth{0.000000pt}%
\definecolor{currentstroke}{rgb}{0.000000,0.000000,0.000000}%
\pgfsetstrokecolor{currentstroke}%
\pgfsetdash{}{0pt}%
\pgfpathmoveto{\pgfqpoint{5.697861in}{2.954719in}}%
\pgfpathlineto{\pgfqpoint{5.711782in}{2.951094in}}%
\pgfpathlineto{\pgfqpoint{5.725713in}{2.947533in}}%
\pgfpathlineto{\pgfqpoint{5.739651in}{2.944038in}}%
\pgfpathlineto{\pgfqpoint{5.753598in}{2.940608in}}%
\pgfpathlineto{\pgfqpoint{5.761076in}{2.956544in}}%
\pgfpathlineto{\pgfqpoint{5.768564in}{2.972899in}}%
\pgfpathlineto{\pgfqpoint{5.776063in}{2.989682in}}%
\pgfpathlineto{\pgfqpoint{5.783573in}{3.006903in}}%
\pgfpathlineto{\pgfqpoint{5.769648in}{3.010874in}}%
\pgfpathlineto{\pgfqpoint{5.755731in}{3.014910in}}%
\pgfpathlineto{\pgfqpoint{5.741822in}{3.019011in}}%
\pgfpathlineto{\pgfqpoint{5.727922in}{3.023177in}}%
\pgfpathlineto{\pgfqpoint{5.720390in}{3.005408in}}%
\pgfpathlineto{\pgfqpoint{5.712870in}{2.988082in}}%
\pgfpathlineto{\pgfqpoint{5.705360in}{2.971189in}}%
\pgfpathlineto{\pgfqpoint{5.697861in}{2.954719in}}%
\pgfpathclose%
\pgfusepath{fill}%
\end{pgfscope}%
\begin{pgfscope}%
\pgfpathrectangle{\pgfqpoint{1.150000in}{0.150000in}}{\pgfqpoint{5.700000in}{5.700000in}}%
\pgfusepath{clip}%
\pgfsetbuttcap%
\pgfsetroundjoin%
\definecolor{currentfill}{rgb}{0.280255,0.165693,0.476498}%
\pgfsetfillcolor{currentfill}%
\pgfsetfillopacity{0.700000}%
\pgfsetlinewidth{0.000000pt}%
\definecolor{currentstroke}{rgb}{0.000000,0.000000,0.000000}%
\pgfsetstrokecolor{currentstroke}%
\pgfsetdash{}{0pt}%
\pgfpathmoveto{\pgfqpoint{4.477927in}{2.497878in}}%
\pgfpathlineto{\pgfqpoint{4.491586in}{2.495301in}}%
\pgfpathlineto{\pgfqpoint{4.505252in}{2.492799in}}%
\pgfpathlineto{\pgfqpoint{4.518925in}{2.490371in}}%
\pgfpathlineto{\pgfqpoint{4.532607in}{2.488017in}}%
\pgfpathlineto{\pgfqpoint{4.540287in}{2.497568in}}%
\pgfpathlineto{\pgfqpoint{4.547963in}{2.507228in}}%
\pgfpathlineto{\pgfqpoint{4.555635in}{2.517002in}}%
\pgfpathlineto{\pgfqpoint{4.563303in}{2.526895in}}%
\pgfpathlineto{\pgfqpoint{4.549635in}{2.529529in}}%
\pgfpathlineto{\pgfqpoint{4.535974in}{2.532237in}}%
\pgfpathlineto{\pgfqpoint{4.522320in}{2.535019in}}%
\pgfpathlineto{\pgfqpoint{4.508674in}{2.537874in}}%
\pgfpathlineto{\pgfqpoint{4.500993in}{2.527694in}}%
\pgfpathlineto{\pgfqpoint{4.493308in}{2.517638in}}%
\pgfpathlineto{\pgfqpoint{4.485620in}{2.507701in}}%
\pgfpathlineto{\pgfqpoint{4.477927in}{2.497878in}}%
\pgfpathclose%
\pgfusepath{fill}%
\end{pgfscope}%
\begin{pgfscope}%
\pgfpathrectangle{\pgfqpoint{1.150000in}{0.150000in}}{\pgfqpoint{5.700000in}{5.700000in}}%
\pgfusepath{clip}%
\pgfsetbuttcap%
\pgfsetroundjoin%
\definecolor{currentfill}{rgb}{0.281446,0.084320,0.407414}%
\pgfsetfillcolor{currentfill}%
\pgfsetfillopacity{0.700000}%
\pgfsetlinewidth{0.000000pt}%
\definecolor{currentstroke}{rgb}{0.000000,0.000000,0.000000}%
\pgfsetstrokecolor{currentstroke}%
\pgfsetdash{}{0pt}%
\pgfpathmoveto{\pgfqpoint{3.491163in}{2.343435in}}%
\pgfpathlineto{\pgfqpoint{3.504608in}{2.338195in}}%
\pgfpathlineto{\pgfqpoint{3.518057in}{2.333047in}}%
\pgfpathlineto{\pgfqpoint{3.531511in}{2.327991in}}%
\pgfpathlineto{\pgfqpoint{3.544969in}{2.323025in}}%
\pgfpathlineto{\pgfqpoint{3.552978in}{2.332526in}}%
\pgfpathlineto{\pgfqpoint{3.560982in}{2.342073in}}%
\pgfpathlineto{\pgfqpoint{3.568980in}{2.351669in}}%
\pgfpathlineto{\pgfqpoint{3.576972in}{2.361316in}}%
\pgfpathlineto{\pgfqpoint{3.563525in}{2.366379in}}%
\pgfpathlineto{\pgfqpoint{3.550081in}{2.371533in}}%
\pgfpathlineto{\pgfqpoint{3.536643in}{2.376778in}}%
\pgfpathlineto{\pgfqpoint{3.523208in}{2.382116in}}%
\pgfpathlineto{\pgfqpoint{3.515206in}{2.372364in}}%
\pgfpathlineto{\pgfqpoint{3.507198in}{2.362668in}}%
\pgfpathlineto{\pgfqpoint{3.499183in}{2.353026in}}%
\pgfpathlineto{\pgfqpoint{3.491163in}{2.343435in}}%
\pgfpathclose%
\pgfusepath{fill}%
\end{pgfscope}%
\begin{pgfscope}%
\pgfpathrectangle{\pgfqpoint{1.150000in}{0.150000in}}{\pgfqpoint{5.700000in}{5.700000in}}%
\pgfusepath{clip}%
\pgfsetbuttcap%
\pgfsetroundjoin%
\definecolor{currentfill}{rgb}{0.283187,0.125848,0.444960}%
\pgfsetfillcolor{currentfill}%
\pgfsetfillopacity{0.700000}%
\pgfsetlinewidth{0.000000pt}%
\definecolor{currentstroke}{rgb}{0.000000,0.000000,0.000000}%
\pgfsetstrokecolor{currentstroke}%
\pgfsetdash{}{0pt}%
\pgfpathmoveto{\pgfqpoint{4.167116in}{2.427163in}}%
\pgfpathlineto{\pgfqpoint{4.180698in}{2.424083in}}%
\pgfpathlineto{\pgfqpoint{4.194286in}{2.421081in}}%
\pgfpathlineto{\pgfqpoint{4.207882in}{2.418157in}}%
\pgfpathlineto{\pgfqpoint{4.221484in}{2.415311in}}%
\pgfpathlineto{\pgfqpoint{4.229268in}{2.424771in}}%
\pgfpathlineto{\pgfqpoint{4.237046in}{2.434306in}}%
\pgfpathlineto{\pgfqpoint{4.244820in}{2.443919in}}%
\pgfpathlineto{\pgfqpoint{4.252589in}{2.453614in}}%
\pgfpathlineto{\pgfqpoint{4.238998in}{2.456679in}}%
\pgfpathlineto{\pgfqpoint{4.225414in}{2.459822in}}%
\pgfpathlineto{\pgfqpoint{4.211837in}{2.463043in}}%
\pgfpathlineto{\pgfqpoint{4.198266in}{2.466342in}}%
\pgfpathlineto{\pgfqpoint{4.190486in}{2.456421in}}%
\pgfpathlineto{\pgfqpoint{4.182701in}{2.446586in}}%
\pgfpathlineto{\pgfqpoint{4.174911in}{2.436835in}}%
\pgfpathlineto{\pgfqpoint{4.167116in}{2.427163in}}%
\pgfpathclose%
\pgfusepath{fill}%
\end{pgfscope}%
\begin{pgfscope}%
\pgfpathrectangle{\pgfqpoint{1.150000in}{0.150000in}}{\pgfqpoint{5.700000in}{5.700000in}}%
\pgfusepath{clip}%
\pgfsetbuttcap%
\pgfsetroundjoin%
\definecolor{currentfill}{rgb}{0.248629,0.278775,0.534556}%
\pgfsetfillcolor{currentfill}%
\pgfsetfillopacity{0.700000}%
\pgfsetlinewidth{0.000000pt}%
\definecolor{currentstroke}{rgb}{0.000000,0.000000,0.000000}%
\pgfsetstrokecolor{currentstroke}%
\pgfsetdash{}{0pt}%
\pgfpathmoveto{\pgfqpoint{5.270597in}{2.738092in}}%
\pgfpathlineto{\pgfqpoint{5.284451in}{2.735541in}}%
\pgfpathlineto{\pgfqpoint{5.298314in}{2.733059in}}%
\pgfpathlineto{\pgfqpoint{5.312185in}{2.730643in}}%
\pgfpathlineto{\pgfqpoint{5.326064in}{2.728296in}}%
\pgfpathlineto{\pgfqpoint{5.333532in}{2.739965in}}%
\pgfpathlineto{\pgfqpoint{5.341002in}{2.751900in}}%
\pgfpathlineto{\pgfqpoint{5.348474in}{2.764109in}}%
\pgfpathlineto{\pgfqpoint{5.355949in}{2.776599in}}%
\pgfpathlineto{\pgfqpoint{5.342089in}{2.779388in}}%
\pgfpathlineto{\pgfqpoint{5.328237in}{2.782243in}}%
\pgfpathlineto{\pgfqpoint{5.314393in}{2.785166in}}%
\pgfpathlineto{\pgfqpoint{5.300557in}{2.788157in}}%
\pgfpathlineto{\pgfqpoint{5.293063in}{2.775219in}}%
\pgfpathlineto{\pgfqpoint{5.285572in}{2.762567in}}%
\pgfpathlineto{\pgfqpoint{5.278083in}{2.750194in}}%
\pgfpathlineto{\pgfqpoint{5.270597in}{2.738092in}}%
\pgfpathclose%
\pgfusepath{fill}%
\end{pgfscope}%
\begin{pgfscope}%
\pgfpathrectangle{\pgfqpoint{1.150000in}{0.150000in}}{\pgfqpoint{5.700000in}{5.700000in}}%
\pgfusepath{clip}%
\pgfsetbuttcap%
\pgfsetroundjoin%
\definecolor{currentfill}{rgb}{0.195860,0.395433,0.555276}%
\pgfsetfillcolor{currentfill}%
\pgfsetfillopacity{0.700000}%
\pgfsetlinewidth{0.000000pt}%
\definecolor{currentstroke}{rgb}{0.000000,0.000000,0.000000}%
\pgfsetstrokecolor{currentstroke}%
\pgfsetdash{}{0pt}%
\pgfpathmoveto{\pgfqpoint{5.783573in}{3.006903in}}%
\pgfpathlineto{\pgfqpoint{5.797507in}{3.002997in}}%
\pgfpathlineto{\pgfqpoint{5.811448in}{2.999156in}}%
\pgfpathlineto{\pgfqpoint{5.825398in}{2.995379in}}%
\pgfpathlineto{\pgfqpoint{5.839357in}{2.991667in}}%
\pgfpathlineto{\pgfqpoint{5.846857in}{3.008782in}}%
\pgfpathlineto{\pgfqpoint{5.854369in}{3.026352in}}%
\pgfpathlineto{\pgfqpoint{5.861895in}{3.044385in}}%
\pgfpathlineto{\pgfqpoint{5.869434in}{3.062893in}}%
\pgfpathlineto{\pgfqpoint{5.855498in}{3.067166in}}%
\pgfpathlineto{\pgfqpoint{5.841570in}{3.071504in}}%
\pgfpathlineto{\pgfqpoint{5.827650in}{3.075906in}}%
\pgfpathlineto{\pgfqpoint{5.813738in}{3.080373in}}%
\pgfpathlineto{\pgfqpoint{5.806177in}{3.061297in}}%
\pgfpathlineto{\pgfqpoint{5.798630in}{3.042700in}}%
\pgfpathlineto{\pgfqpoint{5.791095in}{3.024572in}}%
\pgfpathlineto{\pgfqpoint{5.783573in}{3.006903in}}%
\pgfpathclose%
\pgfusepath{fill}%
\end{pgfscope}%
\begin{pgfscope}%
\pgfpathrectangle{\pgfqpoint{1.150000in}{0.150000in}}{\pgfqpoint{5.700000in}{5.700000in}}%
\pgfusepath{clip}%
\pgfsetbuttcap%
\pgfsetroundjoin%
\definecolor{currentfill}{rgb}{0.273006,0.204520,0.501721}%
\pgfsetfillcolor{currentfill}%
\pgfsetfillopacity{0.700000}%
\pgfsetlinewidth{0.000000pt}%
\definecolor{currentstroke}{rgb}{0.000000,0.000000,0.000000}%
\pgfsetstrokecolor{currentstroke}%
\pgfsetdash{}{0pt}%
\pgfpathmoveto{\pgfqpoint{4.788810in}{2.577621in}}%
\pgfpathlineto{\pgfqpoint{4.802550in}{2.575309in}}%
\pgfpathlineto{\pgfqpoint{4.816299in}{2.573068in}}%
\pgfpathlineto{\pgfqpoint{4.830055in}{2.570899in}}%
\pgfpathlineto{\pgfqpoint{4.843820in}{2.568800in}}%
\pgfpathlineto{\pgfqpoint{4.851402in}{2.578682in}}%
\pgfpathlineto{\pgfqpoint{4.858982in}{2.588721in}}%
\pgfpathlineto{\pgfqpoint{4.866560in}{2.598921in}}%
\pgfpathlineto{\pgfqpoint{4.874135in}{2.609290in}}%
\pgfpathlineto{\pgfqpoint{4.860386in}{2.611729in}}%
\pgfpathlineto{\pgfqpoint{4.846645in}{2.614239in}}%
\pgfpathlineto{\pgfqpoint{4.832911in}{2.616819in}}%
\pgfpathlineto{\pgfqpoint{4.819185in}{2.619471in}}%
\pgfpathlineto{\pgfqpoint{4.811595in}{2.608755in}}%
\pgfpathlineto{\pgfqpoint{4.804002in}{2.598212in}}%
\pgfpathlineto{\pgfqpoint{4.796407in}{2.587836in}}%
\pgfpathlineto{\pgfqpoint{4.788810in}{2.577621in}}%
\pgfpathclose%
\pgfusepath{fill}%
\end{pgfscope}%
\begin{pgfscope}%
\pgfpathrectangle{\pgfqpoint{1.150000in}{0.150000in}}{\pgfqpoint{5.700000in}{5.700000in}}%
\pgfusepath{clip}%
\pgfsetbuttcap%
\pgfsetroundjoin%
\definecolor{currentfill}{rgb}{0.283072,0.130895,0.449241}%
\pgfsetfillcolor{currentfill}%
\pgfsetfillopacity{0.700000}%
\pgfsetlinewidth{0.000000pt}%
\definecolor{currentstroke}{rgb}{0.000000,0.000000,0.000000}%
\pgfsetstrokecolor{currentstroke}%
\pgfsetdash{}{0pt}%
\pgfpathmoveto{\pgfqpoint{2.791862in}{2.444620in}}%
\pgfpathlineto{\pgfqpoint{2.805270in}{2.435360in}}%
\pgfpathlineto{\pgfqpoint{2.818679in}{2.426221in}}%
\pgfpathlineto{\pgfqpoint{2.832088in}{2.417201in}}%
\pgfpathlineto{\pgfqpoint{2.845497in}{2.408300in}}%
\pgfpathlineto{\pgfqpoint{2.853758in}{2.417061in}}%
\pgfpathlineto{\pgfqpoint{2.862011in}{2.425896in}}%
\pgfpathlineto{\pgfqpoint{2.870257in}{2.434804in}}%
\pgfpathlineto{\pgfqpoint{2.878495in}{2.443786in}}%
\pgfpathlineto{\pgfqpoint{2.865099in}{2.452682in}}%
\pgfpathlineto{\pgfqpoint{2.851705in}{2.461697in}}%
\pgfpathlineto{\pgfqpoint{2.838310in}{2.470831in}}%
\pgfpathlineto{\pgfqpoint{2.824916in}{2.480086in}}%
\pgfpathlineto{\pgfqpoint{2.816664in}{2.471101in}}%
\pgfpathlineto{\pgfqpoint{2.808405in}{2.462196in}}%
\pgfpathlineto{\pgfqpoint{2.800137in}{2.453369in}}%
\pgfpathlineto{\pgfqpoint{2.791862in}{2.444620in}}%
\pgfpathclose%
\pgfusepath{fill}%
\end{pgfscope}%
\begin{pgfscope}%
\pgfpathrectangle{\pgfqpoint{1.150000in}{0.150000in}}{\pgfqpoint{5.700000in}{5.700000in}}%
\pgfusepath{clip}%
\pgfsetbuttcap%
\pgfsetroundjoin%
\definecolor{currentfill}{rgb}{0.253935,0.265254,0.529983}%
\pgfsetfillcolor{currentfill}%
\pgfsetfillopacity{0.700000}%
\pgfsetlinewidth{0.000000pt}%
\definecolor{currentstroke}{rgb}{0.000000,0.000000,0.000000}%
\pgfsetstrokecolor{currentstroke}%
\pgfsetdash{}{0pt}%
\pgfpathmoveto{\pgfqpoint{5.185260in}{2.701429in}}%
\pgfpathlineto{\pgfqpoint{5.199098in}{2.699028in}}%
\pgfpathlineto{\pgfqpoint{5.212946in}{2.696695in}}%
\pgfpathlineto{\pgfqpoint{5.226801in}{2.694430in}}%
\pgfpathlineto{\pgfqpoint{5.240666in}{2.692233in}}%
\pgfpathlineto{\pgfqpoint{5.248147in}{2.703330in}}%
\pgfpathlineto{\pgfqpoint{5.255629in}{2.714667in}}%
\pgfpathlineto{\pgfqpoint{5.263112in}{2.726252in}}%
\pgfpathlineto{\pgfqpoint{5.270597in}{2.738092in}}%
\pgfpathlineto{\pgfqpoint{5.256751in}{2.740710in}}%
\pgfpathlineto{\pgfqpoint{5.242914in}{2.743395in}}%
\pgfpathlineto{\pgfqpoint{5.229084in}{2.746149in}}%
\pgfpathlineto{\pgfqpoint{5.215263in}{2.748970in}}%
\pgfpathlineto{\pgfqpoint{5.207761in}{2.736703in}}%
\pgfpathlineto{\pgfqpoint{5.200259in}{2.724695in}}%
\pgfpathlineto{\pgfqpoint{5.192759in}{2.712940in}}%
\pgfpathlineto{\pgfqpoint{5.185260in}{2.701429in}}%
\pgfpathclose%
\pgfusepath{fill}%
\end{pgfscope}%
\begin{pgfscope}%
\pgfpathrectangle{\pgfqpoint{1.150000in}{0.150000in}}{\pgfqpoint{5.700000in}{5.700000in}}%
\pgfusepath{clip}%
\pgfsetbuttcap%
\pgfsetroundjoin%
\definecolor{currentfill}{rgb}{0.279574,0.170599,0.479997}%
\pgfsetfillcolor{currentfill}%
\pgfsetfillopacity{0.700000}%
\pgfsetlinewidth{0.000000pt}%
\definecolor{currentstroke}{rgb}{0.000000,0.000000,0.000000}%
\pgfsetstrokecolor{currentstroke}%
\pgfsetdash{}{0pt}%
\pgfpathmoveto{\pgfqpoint{2.597578in}{2.531603in}}%
\pgfpathlineto{\pgfqpoint{2.611008in}{2.520789in}}%
\pgfpathlineto{\pgfqpoint{2.624437in}{2.510109in}}%
\pgfpathlineto{\pgfqpoint{2.637864in}{2.499562in}}%
\pgfpathlineto{\pgfqpoint{2.651291in}{2.489145in}}%
\pgfpathlineto{\pgfqpoint{2.659629in}{2.497517in}}%
\pgfpathlineto{\pgfqpoint{2.667959in}{2.505976in}}%
\pgfpathlineto{\pgfqpoint{2.676281in}{2.514523in}}%
\pgfpathlineto{\pgfqpoint{2.684594in}{2.523158in}}%
\pgfpathlineto{\pgfqpoint{2.671183in}{2.533548in}}%
\pgfpathlineto{\pgfqpoint{2.657772in}{2.544070in}}%
\pgfpathlineto{\pgfqpoint{2.644359in}{2.554724in}}%
\pgfpathlineto{\pgfqpoint{2.630945in}{2.565511in}}%
\pgfpathlineto{\pgfqpoint{2.622617in}{2.556895in}}%
\pgfpathlineto{\pgfqpoint{2.614279in}{2.548371in}}%
\pgfpathlineto{\pgfqpoint{2.605933in}{2.539941in}}%
\pgfpathlineto{\pgfqpoint{2.597578in}{2.531603in}}%
\pgfpathclose%
\pgfusepath{fill}%
\end{pgfscope}%
\begin{pgfscope}%
\pgfpathrectangle{\pgfqpoint{1.150000in}{0.150000in}}{\pgfqpoint{5.700000in}{5.700000in}}%
\pgfusepath{clip}%
\pgfsetbuttcap%
\pgfsetroundjoin%
\definecolor{currentfill}{rgb}{0.282656,0.100196,0.422160}%
\pgfsetfillcolor{currentfill}%
\pgfsetfillopacity{0.700000}%
\pgfsetlinewidth{0.000000pt}%
\definecolor{currentstroke}{rgb}{0.000000,0.000000,0.000000}%
\pgfsetstrokecolor{currentstroke}%
\pgfsetdash{}{0pt}%
\pgfpathmoveto{\pgfqpoint{3.856209in}{2.367514in}}%
\pgfpathlineto{\pgfqpoint{3.869724in}{2.363674in}}%
\pgfpathlineto{\pgfqpoint{3.883244in}{2.359919in}}%
\pgfpathlineto{\pgfqpoint{3.896770in}{2.356246in}}%
\pgfpathlineto{\pgfqpoint{3.910302in}{2.352656in}}%
\pgfpathlineto{\pgfqpoint{3.918192in}{2.362118in}}%
\pgfpathlineto{\pgfqpoint{3.926077in}{2.371632in}}%
\pgfpathlineto{\pgfqpoint{3.933956in}{2.381201in}}%
\pgfpathlineto{\pgfqpoint{3.941830in}{2.390828in}}%
\pgfpathlineto{\pgfqpoint{3.928308in}{2.394576in}}%
\pgfpathlineto{\pgfqpoint{3.914792in}{2.398407in}}%
\pgfpathlineto{\pgfqpoint{3.901282in}{2.402322in}}%
\pgfpathlineto{\pgfqpoint{3.887778in}{2.406319in}}%
\pgfpathlineto{\pgfqpoint{3.879894in}{2.396526in}}%
\pgfpathlineto{\pgfqpoint{3.872004in}{2.386797in}}%
\pgfpathlineto{\pgfqpoint{3.864110in}{2.377127in}}%
\pgfpathlineto{\pgfqpoint{3.856209in}{2.367514in}}%
\pgfpathclose%
\pgfusepath{fill}%
\end{pgfscope}%
\begin{pgfscope}%
\pgfpathrectangle{\pgfqpoint{1.150000in}{0.150000in}}{\pgfqpoint{5.700000in}{5.700000in}}%
\pgfusepath{clip}%
\pgfsetbuttcap%
\pgfsetroundjoin%
\definecolor{currentfill}{rgb}{0.281887,0.150881,0.465405}%
\pgfsetfillcolor{currentfill}%
\pgfsetfillopacity{0.700000}%
\pgfsetlinewidth{0.000000pt}%
\definecolor{currentstroke}{rgb}{0.000000,0.000000,0.000000}%
\pgfsetstrokecolor{currentstroke}%
\pgfsetdash{}{0pt}%
\pgfpathmoveto{\pgfqpoint{4.392501in}{2.469638in}}%
\pgfpathlineto{\pgfqpoint{4.406143in}{2.467022in}}%
\pgfpathlineto{\pgfqpoint{4.419792in}{2.464481in}}%
\pgfpathlineto{\pgfqpoint{4.433449in}{2.462015in}}%
\pgfpathlineto{\pgfqpoint{4.447113in}{2.459623in}}%
\pgfpathlineto{\pgfqpoint{4.454823in}{2.469041in}}%
\pgfpathlineto{\pgfqpoint{4.462528in}{2.478552in}}%
\pgfpathlineto{\pgfqpoint{4.470230in}{2.488163in}}%
\pgfpathlineto{\pgfqpoint{4.477927in}{2.497878in}}%
\pgfpathlineto{\pgfqpoint{4.464275in}{2.500529in}}%
\pgfpathlineto{\pgfqpoint{4.450631in}{2.503254in}}%
\pgfpathlineto{\pgfqpoint{4.436994in}{2.506055in}}%
\pgfpathlineto{\pgfqpoint{4.423364in}{2.508931in}}%
\pgfpathlineto{\pgfqpoint{4.415655in}{2.498949in}}%
\pgfpathlineto{\pgfqpoint{4.407941in}{2.489076in}}%
\pgfpathlineto{\pgfqpoint{4.400223in}{2.479308in}}%
\pgfpathlineto{\pgfqpoint{4.392501in}{2.469638in}}%
\pgfpathclose%
\pgfusepath{fill}%
\end{pgfscope}%
\begin{pgfscope}%
\pgfpathrectangle{\pgfqpoint{1.150000in}{0.150000in}}{\pgfqpoint{5.700000in}{5.700000in}}%
\pgfusepath{clip}%
\pgfsetbuttcap%
\pgfsetroundjoin%
\definecolor{currentfill}{rgb}{0.281446,0.084320,0.407414}%
\pgfsetfillcolor{currentfill}%
\pgfsetfillopacity{0.700000}%
\pgfsetlinewidth{0.000000pt}%
\definecolor{currentstroke}{rgb}{0.000000,0.000000,0.000000}%
\pgfsetstrokecolor{currentstroke}%
\pgfsetdash{}{0pt}%
\pgfpathmoveto{\pgfqpoint{3.630809in}{2.341964in}}%
\pgfpathlineto{\pgfqpoint{3.644281in}{2.337349in}}%
\pgfpathlineto{\pgfqpoint{3.657757in}{2.332823in}}%
\pgfpathlineto{\pgfqpoint{3.671238in}{2.328384in}}%
\pgfpathlineto{\pgfqpoint{3.684724in}{2.324032in}}%
\pgfpathlineto{\pgfqpoint{3.692690in}{2.333512in}}%
\pgfpathlineto{\pgfqpoint{3.700650in}{2.343036in}}%
\pgfpathlineto{\pgfqpoint{3.708604in}{2.352609in}}%
\pgfpathlineto{\pgfqpoint{3.716553in}{2.362232in}}%
\pgfpathlineto{\pgfqpoint{3.703077in}{2.366701in}}%
\pgfpathlineto{\pgfqpoint{3.689606in}{2.371258in}}%
\pgfpathlineto{\pgfqpoint{3.676140in}{2.375902in}}%
\pgfpathlineto{\pgfqpoint{3.662679in}{2.380635in}}%
\pgfpathlineto{\pgfqpoint{3.654720in}{2.370887in}}%
\pgfpathlineto{\pgfqpoint{3.646756in}{2.361194in}}%
\pgfpathlineto{\pgfqpoint{3.638785in}{2.351554in}}%
\pgfpathlineto{\pgfqpoint{3.630809in}{2.341964in}}%
\pgfpathclose%
\pgfusepath{fill}%
\end{pgfscope}%
\begin{pgfscope}%
\pgfpathrectangle{\pgfqpoint{1.150000in}{0.150000in}}{\pgfqpoint{5.700000in}{5.700000in}}%
\pgfusepath{clip}%
\pgfsetbuttcap%
\pgfsetroundjoin%
\definecolor{currentfill}{rgb}{0.281924,0.089666,0.412415}%
\pgfsetfillcolor{currentfill}%
\pgfsetfillopacity{0.700000}%
\pgfsetlinewidth{0.000000pt}%
\definecolor{currentstroke}{rgb}{0.000000,0.000000,0.000000}%
\pgfsetstrokecolor{currentstroke}%
\pgfsetdash{}{0pt}%
\pgfpathmoveto{\pgfqpoint{3.125668in}{2.353280in}}%
\pgfpathlineto{\pgfqpoint{3.139078in}{2.346253in}}%
\pgfpathlineto{\pgfqpoint{3.152490in}{2.339329in}}%
\pgfpathlineto{\pgfqpoint{3.165905in}{2.332508in}}%
\pgfpathlineto{\pgfqpoint{3.179323in}{2.325790in}}%
\pgfpathlineto{\pgfqpoint{3.187463in}{2.335009in}}%
\pgfpathlineto{\pgfqpoint{3.195595in}{2.344282in}}%
\pgfpathlineto{\pgfqpoint{3.203722in}{2.353609in}}%
\pgfpathlineto{\pgfqpoint{3.211841in}{2.362992in}}%
\pgfpathlineto{\pgfqpoint{3.198435in}{2.369747in}}%
\pgfpathlineto{\pgfqpoint{3.185032in}{2.376604in}}%
\pgfpathlineto{\pgfqpoint{3.171631in}{2.383564in}}%
\pgfpathlineto{\pgfqpoint{3.158233in}{2.390628in}}%
\pgfpathlineto{\pgfqpoint{3.150102in}{2.381201in}}%
\pgfpathlineto{\pgfqpoint{3.141964in}{2.371835in}}%
\pgfpathlineto{\pgfqpoint{3.133819in}{2.362529in}}%
\pgfpathlineto{\pgfqpoint{3.125668in}{2.353280in}}%
\pgfpathclose%
\pgfusepath{fill}%
\end{pgfscope}%
\begin{pgfscope}%
\pgfpathrectangle{\pgfqpoint{1.150000in}{0.150000in}}{\pgfqpoint{5.700000in}{5.700000in}}%
\pgfusepath{clip}%
\pgfsetbuttcap%
\pgfsetroundjoin%
\definecolor{currentfill}{rgb}{0.276194,0.190074,0.493001}%
\pgfsetfillcolor{currentfill}%
\pgfsetfillopacity{0.700000}%
\pgfsetlinewidth{0.000000pt}%
\definecolor{currentstroke}{rgb}{0.000000,0.000000,0.000000}%
\pgfsetstrokecolor{currentstroke}%
\pgfsetdash{}{0pt}%
\pgfpathmoveto{\pgfqpoint{4.703451in}{2.546929in}}%
\pgfpathlineto{\pgfqpoint{4.717174in}{2.544651in}}%
\pgfpathlineto{\pgfqpoint{4.730905in}{2.542445in}}%
\pgfpathlineto{\pgfqpoint{4.744644in}{2.540311in}}%
\pgfpathlineto{\pgfqpoint{4.758392in}{2.538248in}}%
\pgfpathlineto{\pgfqpoint{4.766001in}{2.547880in}}%
\pgfpathlineto{\pgfqpoint{4.773607in}{2.557649in}}%
\pgfpathlineto{\pgfqpoint{4.781210in}{2.567560in}}%
\pgfpathlineto{\pgfqpoint{4.788810in}{2.577621in}}%
\pgfpathlineto{\pgfqpoint{4.775077in}{2.580004in}}%
\pgfpathlineto{\pgfqpoint{4.761353in}{2.582459in}}%
\pgfpathlineto{\pgfqpoint{4.747636in}{2.584985in}}%
\pgfpathlineto{\pgfqpoint{4.733926in}{2.587583in}}%
\pgfpathlineto{\pgfqpoint{4.726312in}{2.577195in}}%
\pgfpathlineto{\pgfqpoint{4.718694in}{2.566961in}}%
\pgfpathlineto{\pgfqpoint{4.711074in}{2.556874in}}%
\pgfpathlineto{\pgfqpoint{4.703451in}{2.546929in}}%
\pgfpathclose%
\pgfusepath{fill}%
\end{pgfscope}%
\begin{pgfscope}%
\pgfpathrectangle{\pgfqpoint{1.150000in}{0.150000in}}{\pgfqpoint{5.700000in}{5.700000in}}%
\pgfusepath{clip}%
\pgfsetbuttcap%
\pgfsetroundjoin%
\definecolor{currentfill}{rgb}{0.260571,0.246922,0.522828}%
\pgfsetfillcolor{currentfill}%
\pgfsetfillopacity{0.700000}%
\pgfsetlinewidth{0.000000pt}%
\definecolor{currentstroke}{rgb}{0.000000,0.000000,0.000000}%
\pgfsetstrokecolor{currentstroke}%
\pgfsetdash{}{0pt}%
\pgfpathmoveto{\pgfqpoint{5.099924in}{2.666372in}}%
\pgfpathlineto{\pgfqpoint{5.113747in}{2.664098in}}%
\pgfpathlineto{\pgfqpoint{5.127578in}{2.661893in}}%
\pgfpathlineto{\pgfqpoint{5.141418in}{2.659756in}}%
\pgfpathlineto{\pgfqpoint{5.155266in}{2.657687in}}%
\pgfpathlineto{\pgfqpoint{5.162765in}{2.668292in}}%
\pgfpathlineto{\pgfqpoint{5.170263in}{2.679113in}}%
\pgfpathlineto{\pgfqpoint{5.177761in}{2.690156in}}%
\pgfpathlineto{\pgfqpoint{5.185260in}{2.701429in}}%
\pgfpathlineto{\pgfqpoint{5.171429in}{2.703899in}}%
\pgfpathlineto{\pgfqpoint{5.157607in}{2.706436in}}%
\pgfpathlineto{\pgfqpoint{5.143793in}{2.709042in}}%
\pgfpathlineto{\pgfqpoint{5.129987in}{2.711717in}}%
\pgfpathlineto{\pgfqpoint{5.122471in}{2.700036in}}%
\pgfpathlineto{\pgfqpoint{5.114955in}{2.688589in}}%
\pgfpathlineto{\pgfqpoint{5.107440in}{2.677371in}}%
\pgfpathlineto{\pgfqpoint{5.099924in}{2.666372in}}%
\pgfpathclose%
\pgfusepath{fill}%
\end{pgfscope}%
\begin{pgfscope}%
\pgfpathrectangle{\pgfqpoint{1.150000in}{0.150000in}}{\pgfqpoint{5.700000in}{5.700000in}}%
\pgfusepath{clip}%
\pgfsetbuttcap%
\pgfsetroundjoin%
\definecolor{currentfill}{rgb}{0.187231,0.414746,0.556547}%
\pgfsetfillcolor{currentfill}%
\pgfsetfillopacity{0.700000}%
\pgfsetlinewidth{0.000000pt}%
\definecolor{currentstroke}{rgb}{0.000000,0.000000,0.000000}%
\pgfsetstrokecolor{currentstroke}%
\pgfsetdash{}{0pt}%
\pgfpathmoveto{\pgfqpoint{5.869434in}{3.062893in}}%
\pgfpathlineto{\pgfqpoint{5.883378in}{3.058684in}}%
\pgfpathlineto{\pgfqpoint{5.897331in}{3.054540in}}%
\pgfpathlineto{\pgfqpoint{5.911292in}{3.050460in}}%
\pgfpathlineto{\pgfqpoint{5.925261in}{3.046444in}}%
\pgfpathlineto{\pgfqpoint{5.932791in}{3.064863in}}%
\pgfpathlineto{\pgfqpoint{5.940337in}{3.083772in}}%
\pgfpathlineto{\pgfqpoint{5.947898in}{3.103183in}}%
\pgfpathlineto{\pgfqpoint{5.933946in}{3.107633in}}%
\pgfpathlineto{\pgfqpoint{5.920002in}{3.112147in}}%
\pgfpathlineto{\pgfqpoint{5.906066in}{3.116726in}}%
\pgfpathlineto{\pgfqpoint{5.892138in}{3.121369in}}%
\pgfpathlineto{\pgfqpoint{5.884555in}{3.101374in}}%
\pgfpathlineto{\pgfqpoint{5.876987in}{3.081885in}}%
\pgfpathlineto{\pgfqpoint{5.869434in}{3.062893in}}%
\pgfpathclose%
\pgfusepath{fill}%
\end{pgfscope}%
\begin{pgfscope}%
\pgfpathrectangle{\pgfqpoint{1.150000in}{0.150000in}}{\pgfqpoint{5.700000in}{5.700000in}}%
\pgfusepath{clip}%
\pgfsetbuttcap%
\pgfsetroundjoin%
\definecolor{currentfill}{rgb}{0.281446,0.084320,0.407414}%
\pgfsetfillcolor{currentfill}%
\pgfsetfillopacity{0.700000}%
\pgfsetlinewidth{0.000000pt}%
\definecolor{currentstroke}{rgb}{0.000000,0.000000,0.000000}%
\pgfsetstrokecolor{currentstroke}%
\pgfsetdash{}{0pt}%
\pgfpathmoveto{\pgfqpoint{3.265495in}{2.336986in}}%
\pgfpathlineto{\pgfqpoint{3.278916in}{2.330734in}}%
\pgfpathlineto{\pgfqpoint{3.292341in}{2.324581in}}%
\pgfpathlineto{\pgfqpoint{3.305769in}{2.318526in}}%
\pgfpathlineto{\pgfqpoint{3.319200in}{2.312568in}}%
\pgfpathlineto{\pgfqpoint{3.327291in}{2.321909in}}%
\pgfpathlineto{\pgfqpoint{3.335375in}{2.331298in}}%
\pgfpathlineto{\pgfqpoint{3.343453in}{2.340737in}}%
\pgfpathlineto{\pgfqpoint{3.351525in}{2.350227in}}%
\pgfpathlineto{\pgfqpoint{3.338105in}{2.356242in}}%
\pgfpathlineto{\pgfqpoint{3.324688in}{2.362354in}}%
\pgfpathlineto{\pgfqpoint{3.311274in}{2.368564in}}%
\pgfpathlineto{\pgfqpoint{3.297864in}{2.374872in}}%
\pgfpathlineto{\pgfqpoint{3.289782in}{2.365318in}}%
\pgfpathlineto{\pgfqpoint{3.281692in}{2.355820in}}%
\pgfpathlineto{\pgfqpoint{3.273597in}{2.346376in}}%
\pgfpathlineto{\pgfqpoint{3.265495in}{2.336986in}}%
\pgfpathclose%
\pgfusepath{fill}%
\end{pgfscope}%
\begin{pgfscope}%
\pgfpathrectangle{\pgfqpoint{1.150000in}{0.150000in}}{\pgfqpoint{5.700000in}{5.700000in}}%
\pgfusepath{clip}%
\pgfsetbuttcap%
\pgfsetroundjoin%
\definecolor{currentfill}{rgb}{0.283229,0.120777,0.440584}%
\pgfsetfillcolor{currentfill}%
\pgfsetfillopacity{0.700000}%
\pgfsetlinewidth{0.000000pt}%
\definecolor{currentstroke}{rgb}{0.000000,0.000000,0.000000}%
\pgfsetstrokecolor{currentstroke}%
\pgfsetdash{}{0pt}%
\pgfpathmoveto{\pgfqpoint{4.081580in}{2.401504in}}%
\pgfpathlineto{\pgfqpoint{4.095146in}{2.398306in}}%
\pgfpathlineto{\pgfqpoint{4.108720in}{2.395188in}}%
\pgfpathlineto{\pgfqpoint{4.122299in}{2.392149in}}%
\pgfpathlineto{\pgfqpoint{4.135886in}{2.389190in}}%
\pgfpathlineto{\pgfqpoint{4.143701in}{2.398583in}}%
\pgfpathlineto{\pgfqpoint{4.151511in}{2.408041in}}%
\pgfpathlineto{\pgfqpoint{4.159316in}{2.417566in}}%
\pgfpathlineto{\pgfqpoint{4.167116in}{2.427163in}}%
\pgfpathlineto{\pgfqpoint{4.153540in}{2.430322in}}%
\pgfpathlineto{\pgfqpoint{4.139972in}{2.433560in}}%
\pgfpathlineto{\pgfqpoint{4.126409in}{2.436877in}}%
\pgfpathlineto{\pgfqpoint{4.112853in}{2.440273in}}%
\pgfpathlineto{\pgfqpoint{4.105043in}{2.430470in}}%
\pgfpathlineto{\pgfqpoint{4.097227in}{2.420743in}}%
\pgfpathlineto{\pgfqpoint{4.089406in}{2.411089in}}%
\pgfpathlineto{\pgfqpoint{4.081580in}{2.401504in}}%
\pgfpathclose%
\pgfusepath{fill}%
\end{pgfscope}%
\begin{pgfscope}%
\pgfpathrectangle{\pgfqpoint{1.150000in}{0.150000in}}{\pgfqpoint{5.700000in}{5.700000in}}%
\pgfusepath{clip}%
\pgfsetbuttcap%
\pgfsetroundjoin%
\definecolor{currentfill}{rgb}{0.282656,0.100196,0.422160}%
\pgfsetfillcolor{currentfill}%
\pgfsetfillopacity{0.700000}%
\pgfsetlinewidth{0.000000pt}%
\definecolor{currentstroke}{rgb}{0.000000,0.000000,0.000000}%
\pgfsetstrokecolor{currentstroke}%
\pgfsetdash{}{0pt}%
\pgfpathmoveto{\pgfqpoint{2.985694in}{2.376775in}}%
\pgfpathlineto{\pgfqpoint{2.999100in}{2.368905in}}%
\pgfpathlineto{\pgfqpoint{3.012508in}{2.361146in}}%
\pgfpathlineto{\pgfqpoint{3.025917in}{2.353496in}}%
\pgfpathlineto{\pgfqpoint{3.039329in}{2.345954in}}%
\pgfpathlineto{\pgfqpoint{3.047520in}{2.354985in}}%
\pgfpathlineto{\pgfqpoint{3.055704in}{2.364077in}}%
\pgfpathlineto{\pgfqpoint{3.063881in}{2.373230in}}%
\pgfpathlineto{\pgfqpoint{3.072051in}{2.382445in}}%
\pgfpathlineto{\pgfqpoint{3.058653in}{2.390003in}}%
\pgfpathlineto{\pgfqpoint{3.045256in}{2.397669in}}%
\pgfpathlineto{\pgfqpoint{3.031861in}{2.405445in}}%
\pgfpathlineto{\pgfqpoint{3.018468in}{2.413330in}}%
\pgfpathlineto{\pgfqpoint{3.010285in}{2.404092in}}%
\pgfpathlineto{\pgfqpoint{3.002095in}{2.394921in}}%
\pgfpathlineto{\pgfqpoint{2.993898in}{2.385815in}}%
\pgfpathlineto{\pgfqpoint{2.985694in}{2.376775in}}%
\pgfpathclose%
\pgfusepath{fill}%
\end{pgfscope}%
\begin{pgfscope}%
\pgfpathrectangle{\pgfqpoint{1.150000in}{0.150000in}}{\pgfqpoint{5.700000in}{5.700000in}}%
\pgfusepath{clip}%
\pgfsetbuttcap%
\pgfsetroundjoin%
\definecolor{currentfill}{rgb}{0.280894,0.078907,0.402329}%
\pgfsetfillcolor{currentfill}%
\pgfsetfillopacity{0.700000}%
\pgfsetlinewidth{0.000000pt}%
\definecolor{currentstroke}{rgb}{0.000000,0.000000,0.000000}%
\pgfsetstrokecolor{currentstroke}%
\pgfsetdash{}{0pt}%
\pgfpathmoveto{\pgfqpoint{3.405243in}{2.327132in}}%
\pgfpathlineto{\pgfqpoint{3.418681in}{2.321596in}}%
\pgfpathlineto{\pgfqpoint{3.432124in}{2.316154in}}%
\pgfpathlineto{\pgfqpoint{3.445572in}{2.310806in}}%
\pgfpathlineto{\pgfqpoint{3.459023in}{2.305551in}}%
\pgfpathlineto{\pgfqpoint{3.467067in}{2.314954in}}%
\pgfpathlineto{\pgfqpoint{3.475105in}{2.324401in}}%
\pgfpathlineto{\pgfqpoint{3.483137in}{2.333894in}}%
\pgfpathlineto{\pgfqpoint{3.491163in}{2.343435in}}%
\pgfpathlineto{\pgfqpoint{3.477723in}{2.348768in}}%
\pgfpathlineto{\pgfqpoint{3.464286in}{2.354193in}}%
\pgfpathlineto{\pgfqpoint{3.450854in}{2.359712in}}%
\pgfpathlineto{\pgfqpoint{3.437426in}{2.365325in}}%
\pgfpathlineto{\pgfqpoint{3.429389in}{2.355700in}}%
\pgfpathlineto{\pgfqpoint{3.421346in}{2.346127in}}%
\pgfpathlineto{\pgfqpoint{3.413297in}{2.336605in}}%
\pgfpathlineto{\pgfqpoint{3.405243in}{2.327132in}}%
\pgfpathclose%
\pgfusepath{fill}%
\end{pgfscope}%
\begin{pgfscope}%
\pgfpathrectangle{\pgfqpoint{1.150000in}{0.150000in}}{\pgfqpoint{5.700000in}{5.700000in}}%
\pgfusepath{clip}%
\pgfsetbuttcap%
\pgfsetroundjoin%
\definecolor{currentfill}{rgb}{0.265145,0.232956,0.516599}%
\pgfsetfillcolor{currentfill}%
\pgfsetfillopacity{0.700000}%
\pgfsetlinewidth{0.000000pt}%
\definecolor{currentstroke}{rgb}{0.000000,0.000000,0.000000}%
\pgfsetstrokecolor{currentstroke}%
\pgfsetdash{}{0pt}%
\pgfpathmoveto{\pgfqpoint{5.014579in}{2.632706in}}%
\pgfpathlineto{\pgfqpoint{5.028385in}{2.630536in}}%
\pgfpathlineto{\pgfqpoint{5.042200in}{2.628436in}}%
\pgfpathlineto{\pgfqpoint{5.056023in}{2.626405in}}%
\pgfpathlineto{\pgfqpoint{5.069855in}{2.624443in}}%
\pgfpathlineto{\pgfqpoint{5.077374in}{2.634630in}}%
\pgfpathlineto{\pgfqpoint{5.084891in}{2.645009in}}%
\pgfpathlineto{\pgfqpoint{5.092408in}{2.655587in}}%
\pgfpathlineto{\pgfqpoint{5.099924in}{2.666372in}}%
\pgfpathlineto{\pgfqpoint{5.086109in}{2.668715in}}%
\pgfpathlineto{\pgfqpoint{5.072303in}{2.671127in}}%
\pgfpathlineto{\pgfqpoint{5.058505in}{2.673608in}}%
\pgfpathlineto{\pgfqpoint{5.044715in}{2.676158in}}%
\pgfpathlineto{\pgfqpoint{5.037182in}{2.664985in}}%
\pgfpathlineto{\pgfqpoint{5.029648in}{2.654023in}}%
\pgfpathlineto{\pgfqpoint{5.022114in}{2.643266in}}%
\pgfpathlineto{\pgfqpoint{5.014579in}{2.632706in}}%
\pgfpathclose%
\pgfusepath{fill}%
\end{pgfscope}%
\begin{pgfscope}%
\pgfpathrectangle{\pgfqpoint{1.150000in}{0.150000in}}{\pgfqpoint{5.700000in}{5.700000in}}%
\pgfusepath{clip}%
\pgfsetbuttcap%
\pgfsetroundjoin%
\definecolor{currentfill}{rgb}{0.281412,0.155834,0.469201}%
\pgfsetfillcolor{currentfill}%
\pgfsetfillopacity{0.700000}%
\pgfsetlinewidth{0.000000pt}%
\definecolor{currentstroke}{rgb}{0.000000,0.000000,0.000000}%
\pgfsetstrokecolor{currentstroke}%
\pgfsetdash{}{0pt}%
\pgfpathmoveto{\pgfqpoint{2.651291in}{2.489145in}}%
\pgfpathlineto{\pgfqpoint{2.664716in}{2.478858in}}%
\pgfpathlineto{\pgfqpoint{2.678141in}{2.468701in}}%
\pgfpathlineto{\pgfqpoint{2.691566in}{2.458671in}}%
\pgfpathlineto{\pgfqpoint{2.704989in}{2.448768in}}%
\pgfpathlineto{\pgfqpoint{2.713312in}{2.457173in}}%
\pgfpathlineto{\pgfqpoint{2.721626in}{2.465661in}}%
\pgfpathlineto{\pgfqpoint{2.729932in}{2.474231in}}%
\pgfpathlineto{\pgfqpoint{2.738230in}{2.482884in}}%
\pgfpathlineto{\pgfqpoint{2.724822in}{2.492762in}}%
\pgfpathlineto{\pgfqpoint{2.711413in}{2.502766in}}%
\pgfpathlineto{\pgfqpoint{2.698004in}{2.512897in}}%
\pgfpathlineto{\pgfqpoint{2.684594in}{2.523158in}}%
\pgfpathlineto{\pgfqpoint{2.676281in}{2.514523in}}%
\pgfpathlineto{\pgfqpoint{2.667959in}{2.505976in}}%
\pgfpathlineto{\pgfqpoint{2.659629in}{2.497517in}}%
\pgfpathlineto{\pgfqpoint{2.651291in}{2.489145in}}%
\pgfpathclose%
\pgfusepath{fill}%
\end{pgfscope}%
\begin{pgfscope}%
\pgfpathrectangle{\pgfqpoint{1.150000in}{0.150000in}}{\pgfqpoint{5.700000in}{5.700000in}}%
\pgfusepath{clip}%
\pgfsetbuttcap%
\pgfsetroundjoin%
\definecolor{currentfill}{rgb}{0.282623,0.140926,0.457517}%
\pgfsetfillcolor{currentfill}%
\pgfsetfillopacity{0.700000}%
\pgfsetlinewidth{0.000000pt}%
\definecolor{currentstroke}{rgb}{0.000000,0.000000,0.000000}%
\pgfsetstrokecolor{currentstroke}%
\pgfsetdash{}{0pt}%
\pgfpathmoveto{\pgfqpoint{4.307021in}{2.442125in}}%
\pgfpathlineto{\pgfqpoint{4.320646in}{2.439445in}}%
\pgfpathlineto{\pgfqpoint{4.334279in}{2.436841in}}%
\pgfpathlineto{\pgfqpoint{4.347919in}{2.434313in}}%
\pgfpathlineto{\pgfqpoint{4.361565in}{2.431861in}}%
\pgfpathlineto{\pgfqpoint{4.369306in}{2.441179in}}%
\pgfpathlineto{\pgfqpoint{4.377042in}{2.450578in}}%
\pgfpathlineto{\pgfqpoint{4.384774in}{2.460063in}}%
\pgfpathlineto{\pgfqpoint{4.392501in}{2.469638in}}%
\pgfpathlineto{\pgfqpoint{4.378866in}{2.472330in}}%
\pgfpathlineto{\pgfqpoint{4.365238in}{2.475098in}}%
\pgfpathlineto{\pgfqpoint{4.351618in}{2.477941in}}%
\pgfpathlineto{\pgfqpoint{4.338004in}{2.480861in}}%
\pgfpathlineto{\pgfqpoint{4.330265in}{2.471039in}}%
\pgfpathlineto{\pgfqpoint{4.322522in}{2.461312in}}%
\pgfpathlineto{\pgfqpoint{4.314774in}{2.451676in}}%
\pgfpathlineto{\pgfqpoint{4.307021in}{2.442125in}}%
\pgfpathclose%
\pgfusepath{fill}%
\end{pgfscope}%
\begin{pgfscope}%
\pgfpathrectangle{\pgfqpoint{1.150000in}{0.150000in}}{\pgfqpoint{5.700000in}{5.700000in}}%
\pgfusepath{clip}%
\pgfsetbuttcap%
\pgfsetroundjoin%
\definecolor{currentfill}{rgb}{0.278012,0.180367,0.486697}%
\pgfsetfillcolor{currentfill}%
\pgfsetfillopacity{0.700000}%
\pgfsetlinewidth{0.000000pt}%
\definecolor{currentstroke}{rgb}{0.000000,0.000000,0.000000}%
\pgfsetstrokecolor{currentstroke}%
\pgfsetdash{}{0pt}%
\pgfpathmoveto{\pgfqpoint{4.618051in}{2.517094in}}%
\pgfpathlineto{\pgfqpoint{4.631757in}{2.514827in}}%
\pgfpathlineto{\pgfqpoint{4.645471in}{2.512632in}}%
\pgfpathlineto{\pgfqpoint{4.659193in}{2.510510in}}%
\pgfpathlineto{\pgfqpoint{4.672922in}{2.508460in}}%
\pgfpathlineto{\pgfqpoint{4.680560in}{2.517892in}}%
\pgfpathlineto{\pgfqpoint{4.688194in}{2.527444in}}%
\pgfpathlineto{\pgfqpoint{4.695824in}{2.537121in}}%
\pgfpathlineto{\pgfqpoint{4.703451in}{2.546929in}}%
\pgfpathlineto{\pgfqpoint{4.689735in}{2.549280in}}%
\pgfpathlineto{\pgfqpoint{4.676027in}{2.551702in}}%
\pgfpathlineto{\pgfqpoint{4.662327in}{2.554197in}}%
\pgfpathlineto{\pgfqpoint{4.648634in}{2.556765in}}%
\pgfpathlineto{\pgfqpoint{4.640994in}{2.546649in}}%
\pgfpathlineto{\pgfqpoint{4.633350in}{2.536669in}}%
\pgfpathlineto{\pgfqpoint{4.625702in}{2.526819in}}%
\pgfpathlineto{\pgfqpoint{4.618051in}{2.517094in}}%
\pgfpathclose%
\pgfusepath{fill}%
\end{pgfscope}%
\begin{pgfscope}%
\pgfpathrectangle{\pgfqpoint{1.150000in}{0.150000in}}{\pgfqpoint{5.700000in}{5.700000in}}%
\pgfusepath{clip}%
\pgfsetbuttcap%
\pgfsetroundjoin%
\definecolor{currentfill}{rgb}{0.283229,0.120777,0.440584}%
\pgfsetfillcolor{currentfill}%
\pgfsetfillopacity{0.700000}%
\pgfsetlinewidth{0.000000pt}%
\definecolor{currentstroke}{rgb}{0.000000,0.000000,0.000000}%
\pgfsetstrokecolor{currentstroke}%
\pgfsetdash{}{0pt}%
\pgfpathmoveto{\pgfqpoint{2.845497in}{2.408300in}}%
\pgfpathlineto{\pgfqpoint{2.858907in}{2.399517in}}%
\pgfpathlineto{\pgfqpoint{2.872318in}{2.390850in}}%
\pgfpathlineto{\pgfqpoint{2.885730in}{2.382299in}}%
\pgfpathlineto{\pgfqpoint{2.899143in}{2.373863in}}%
\pgfpathlineto{\pgfqpoint{2.907389in}{2.382637in}}%
\pgfpathlineto{\pgfqpoint{2.915629in}{2.391478in}}%
\pgfpathlineto{\pgfqpoint{2.923861in}{2.400389in}}%
\pgfpathlineto{\pgfqpoint{2.932085in}{2.409368in}}%
\pgfpathlineto{\pgfqpoint{2.918686in}{2.417799in}}%
\pgfpathlineto{\pgfqpoint{2.905288in}{2.426345in}}%
\pgfpathlineto{\pgfqpoint{2.891891in}{2.435007in}}%
\pgfpathlineto{\pgfqpoint{2.878495in}{2.443786in}}%
\pgfpathlineto{\pgfqpoint{2.870257in}{2.434804in}}%
\pgfpathlineto{\pgfqpoint{2.862011in}{2.425896in}}%
\pgfpathlineto{\pgfqpoint{2.853758in}{2.417061in}}%
\pgfpathlineto{\pgfqpoint{2.845497in}{2.408300in}}%
\pgfpathclose%
\pgfusepath{fill}%
\end{pgfscope}%
\begin{pgfscope}%
\pgfpathrectangle{\pgfqpoint{1.150000in}{0.150000in}}{\pgfqpoint{5.700000in}{5.700000in}}%
\pgfusepath{clip}%
\pgfsetbuttcap%
\pgfsetroundjoin%
\definecolor{currentfill}{rgb}{0.281924,0.089666,0.412415}%
\pgfsetfillcolor{currentfill}%
\pgfsetfillopacity{0.700000}%
\pgfsetlinewidth{0.000000pt}%
\definecolor{currentstroke}{rgb}{0.000000,0.000000,0.000000}%
\pgfsetstrokecolor{currentstroke}%
\pgfsetdash{}{0pt}%
\pgfpathmoveto{\pgfqpoint{3.770509in}{2.345219in}}%
\pgfpathlineto{\pgfqpoint{3.784012in}{2.341181in}}%
\pgfpathlineto{\pgfqpoint{3.797520in}{2.337227in}}%
\pgfpathlineto{\pgfqpoint{3.811033in}{2.333358in}}%
\pgfpathlineto{\pgfqpoint{3.824553in}{2.329573in}}%
\pgfpathlineto{\pgfqpoint{3.832475in}{2.338987in}}%
\pgfpathlineto{\pgfqpoint{3.840392in}{2.348447in}}%
\pgfpathlineto{\pgfqpoint{3.848303in}{2.357955in}}%
\pgfpathlineto{\pgfqpoint{3.856209in}{2.367514in}}%
\pgfpathlineto{\pgfqpoint{3.842700in}{2.371437in}}%
\pgfpathlineto{\pgfqpoint{3.829197in}{2.375444in}}%
\pgfpathlineto{\pgfqpoint{3.815699in}{2.379536in}}%
\pgfpathlineto{\pgfqpoint{3.802207in}{2.383713in}}%
\pgfpathlineto{\pgfqpoint{3.794291in}{2.374009in}}%
\pgfpathlineto{\pgfqpoint{3.786369in}{2.364360in}}%
\pgfpathlineto{\pgfqpoint{3.778442in}{2.354764in}}%
\pgfpathlineto{\pgfqpoint{3.770509in}{2.345219in}}%
\pgfpathclose%
\pgfusepath{fill}%
\end{pgfscope}%
\begin{pgfscope}%
\pgfpathrectangle{\pgfqpoint{1.150000in}{0.150000in}}{\pgfqpoint{5.700000in}{5.700000in}}%
\pgfusepath{clip}%
\pgfsetbuttcap%
\pgfsetroundjoin%
\definecolor{currentfill}{rgb}{0.223925,0.334994,0.548053}%
\pgfsetfillcolor{currentfill}%
\pgfsetfillopacity{0.700000}%
\pgfsetlinewidth{0.000000pt}%
\definecolor{currentstroke}{rgb}{0.000000,0.000000,0.000000}%
\pgfsetstrokecolor{currentstroke}%
\pgfsetdash{}{0pt}%
\pgfpathmoveto{\pgfqpoint{5.582403in}{2.848103in}}%
\pgfpathlineto{\pgfqpoint{5.596333in}{2.845237in}}%
\pgfpathlineto{\pgfqpoint{5.610273in}{2.842437in}}%
\pgfpathlineto{\pgfqpoint{5.624221in}{2.839702in}}%
\pgfpathlineto{\pgfqpoint{5.638177in}{2.837033in}}%
\pgfpathlineto{\pgfqpoint{5.645612in}{2.850463in}}%
\pgfpathlineto{\pgfqpoint{5.653052in}{2.864241in}}%
\pgfpathlineto{\pgfqpoint{5.660500in}{2.878375in}}%
\pgfpathlineto{\pgfqpoint{5.667955in}{2.892875in}}%
\pgfpathlineto{\pgfqpoint{5.654021in}{2.896045in}}%
\pgfpathlineto{\pgfqpoint{5.640094in}{2.899281in}}%
\pgfpathlineto{\pgfqpoint{5.626176in}{2.902582in}}%
\pgfpathlineto{\pgfqpoint{5.612267in}{2.905949in}}%
\pgfpathlineto{\pgfqpoint{5.604790in}{2.890941in}}%
\pgfpathlineto{\pgfqpoint{5.597321in}{2.876303in}}%
\pgfpathlineto{\pgfqpoint{5.589858in}{2.862027in}}%
\pgfpathlineto{\pgfqpoint{5.582403in}{2.848103in}}%
\pgfpathclose%
\pgfusepath{fill}%
\end{pgfscope}%
\begin{pgfscope}%
\pgfpathrectangle{\pgfqpoint{1.150000in}{0.150000in}}{\pgfqpoint{5.700000in}{5.700000in}}%
\pgfusepath{clip}%
\pgfsetbuttcap%
\pgfsetroundjoin%
\definecolor{currentfill}{rgb}{0.216210,0.351535,0.550627}%
\pgfsetfillcolor{currentfill}%
\pgfsetfillopacity{0.700000}%
\pgfsetlinewidth{0.000000pt}%
\definecolor{currentstroke}{rgb}{0.000000,0.000000,0.000000}%
\pgfsetstrokecolor{currentstroke}%
\pgfsetdash{}{0pt}%
\pgfpathmoveto{\pgfqpoint{5.667955in}{2.892875in}}%
\pgfpathlineto{\pgfqpoint{5.681899in}{2.889770in}}%
\pgfpathlineto{\pgfqpoint{5.695851in}{2.886730in}}%
\pgfpathlineto{\pgfqpoint{5.709812in}{2.883756in}}%
\pgfpathlineto{\pgfqpoint{5.723781in}{2.880847in}}%
\pgfpathlineto{\pgfqpoint{5.731222in}{2.895208in}}%
\pgfpathlineto{\pgfqpoint{5.738672in}{2.909949in}}%
\pgfpathlineto{\pgfqpoint{5.746130in}{2.925079in}}%
\pgfpathlineto{\pgfqpoint{5.753598in}{2.940608in}}%
\pgfpathlineto{\pgfqpoint{5.739651in}{2.944038in}}%
\pgfpathlineto{\pgfqpoint{5.725713in}{2.947533in}}%
\pgfpathlineto{\pgfqpoint{5.711782in}{2.951094in}}%
\pgfpathlineto{\pgfqpoint{5.697861in}{2.954719in}}%
\pgfpathlineto{\pgfqpoint{5.690371in}{2.938662in}}%
\pgfpathlineto{\pgfqpoint{5.682890in}{2.923009in}}%
\pgfpathlineto{\pgfqpoint{5.675419in}{2.907750in}}%
\pgfpathlineto{\pgfqpoint{5.667955in}{2.892875in}}%
\pgfpathclose%
\pgfusepath{fill}%
\end{pgfscope}%
\begin{pgfscope}%
\pgfpathrectangle{\pgfqpoint{1.150000in}{0.150000in}}{\pgfqpoint{5.700000in}{5.700000in}}%
\pgfusepath{clip}%
\pgfsetbuttcap%
\pgfsetroundjoin%
\definecolor{currentfill}{rgb}{0.283091,0.110553,0.431554}%
\pgfsetfillcolor{currentfill}%
\pgfsetfillopacity{0.700000}%
\pgfsetlinewidth{0.000000pt}%
\definecolor{currentstroke}{rgb}{0.000000,0.000000,0.000000}%
\pgfsetstrokecolor{currentstroke}%
\pgfsetdash{}{0pt}%
\pgfpathmoveto{\pgfqpoint{3.995977in}{2.376654in}}%
\pgfpathlineto{\pgfqpoint{4.009529in}{2.373314in}}%
\pgfpathlineto{\pgfqpoint{4.023087in}{2.370055in}}%
\pgfpathlineto{\pgfqpoint{4.036652in}{2.366877in}}%
\pgfpathlineto{\pgfqpoint{4.050223in}{2.363778in}}%
\pgfpathlineto{\pgfqpoint{4.058070in}{2.373124in}}%
\pgfpathlineto{\pgfqpoint{4.065912in}{2.382525in}}%
\pgfpathlineto{\pgfqpoint{4.073748in}{2.391983in}}%
\pgfpathlineto{\pgfqpoint{4.081580in}{2.401504in}}%
\pgfpathlineto{\pgfqpoint{4.068019in}{2.404781in}}%
\pgfpathlineto{\pgfqpoint{4.054465in}{2.408138in}}%
\pgfpathlineto{\pgfqpoint{4.040918in}{2.411576in}}%
\pgfpathlineto{\pgfqpoint{4.027376in}{2.415095in}}%
\pgfpathlineto{\pgfqpoint{4.019534in}{2.405388in}}%
\pgfpathlineto{\pgfqpoint{4.011687in}{2.395748in}}%
\pgfpathlineto{\pgfqpoint{4.003835in}{2.386171in}}%
\pgfpathlineto{\pgfqpoint{3.995977in}{2.376654in}}%
\pgfpathclose%
\pgfusepath{fill}%
\end{pgfscope}%
\begin{pgfscope}%
\pgfpathrectangle{\pgfqpoint{1.150000in}{0.150000in}}{\pgfqpoint{5.700000in}{5.700000in}}%
\pgfusepath{clip}%
\pgfsetbuttcap%
\pgfsetroundjoin%
\definecolor{currentfill}{rgb}{0.233603,0.313828,0.543914}%
\pgfsetfillcolor{currentfill}%
\pgfsetfillopacity{0.700000}%
\pgfsetlinewidth{0.000000pt}%
\definecolor{currentstroke}{rgb}{0.000000,0.000000,0.000000}%
\pgfsetstrokecolor{currentstroke}%
\pgfsetdash{}{0pt}%
\pgfpathmoveto{\pgfqpoint{5.496916in}{2.805955in}}%
\pgfpathlineto{\pgfqpoint{5.510833in}{2.803305in}}%
\pgfpathlineto{\pgfqpoint{5.524759in}{2.800723in}}%
\pgfpathlineto{\pgfqpoint{5.538694in}{2.798206in}}%
\pgfpathlineto{\pgfqpoint{5.552637in}{2.795755in}}%
\pgfpathlineto{\pgfqpoint{5.560071in}{2.808358in}}%
\pgfpathlineto{\pgfqpoint{5.567509in}{2.821278in}}%
\pgfpathlineto{\pgfqpoint{5.574953in}{2.834523in}}%
\pgfpathlineto{\pgfqpoint{5.582403in}{2.848103in}}%
\pgfpathlineto{\pgfqpoint{5.568480in}{2.851035in}}%
\pgfpathlineto{\pgfqpoint{5.554567in}{2.854033in}}%
\pgfpathlineto{\pgfqpoint{5.540662in}{2.857097in}}%
\pgfpathlineto{\pgfqpoint{5.526765in}{2.860227in}}%
\pgfpathlineto{\pgfqpoint{5.519295in}{2.846158in}}%
\pgfpathlineto{\pgfqpoint{5.511830in}{2.832429in}}%
\pgfpathlineto{\pgfqpoint{5.504371in}{2.819031in}}%
\pgfpathlineto{\pgfqpoint{5.496916in}{2.805955in}}%
\pgfpathclose%
\pgfusepath{fill}%
\end{pgfscope}%
\begin{pgfscope}%
\pgfpathrectangle{\pgfqpoint{1.150000in}{0.150000in}}{\pgfqpoint{5.700000in}{5.700000in}}%
\pgfusepath{clip}%
\pgfsetbuttcap%
\pgfsetroundjoin%
\definecolor{currentfill}{rgb}{0.280894,0.078907,0.402329}%
\pgfsetfillcolor{currentfill}%
\pgfsetfillopacity{0.700000}%
\pgfsetlinewidth{0.000000pt}%
\definecolor{currentstroke}{rgb}{0.000000,0.000000,0.000000}%
\pgfsetstrokecolor{currentstroke}%
\pgfsetdash{}{0pt}%
\pgfpathmoveto{\pgfqpoint{3.544969in}{2.323025in}}%
\pgfpathlineto{\pgfqpoint{3.558431in}{2.318150in}}%
\pgfpathlineto{\pgfqpoint{3.571899in}{2.313366in}}%
\pgfpathlineto{\pgfqpoint{3.585371in}{2.308671in}}%
\pgfpathlineto{\pgfqpoint{3.598847in}{2.304065in}}%
\pgfpathlineto{\pgfqpoint{3.606847in}{2.313475in}}%
\pgfpathlineto{\pgfqpoint{3.614840in}{2.322927in}}%
\pgfpathlineto{\pgfqpoint{3.622828in}{2.332423in}}%
\pgfpathlineto{\pgfqpoint{3.630809in}{2.341964in}}%
\pgfpathlineto{\pgfqpoint{3.617343in}{2.346668in}}%
\pgfpathlineto{\pgfqpoint{3.603881in}{2.351461in}}%
\pgfpathlineto{\pgfqpoint{3.590425in}{2.356343in}}%
\pgfpathlineto{\pgfqpoint{3.576972in}{2.361316in}}%
\pgfpathlineto{\pgfqpoint{3.568980in}{2.351669in}}%
\pgfpathlineto{\pgfqpoint{3.560982in}{2.342073in}}%
\pgfpathlineto{\pgfqpoint{3.552978in}{2.332526in}}%
\pgfpathlineto{\pgfqpoint{3.544969in}{2.323025in}}%
\pgfpathclose%
\pgfusepath{fill}%
\end{pgfscope}%
\begin{pgfscope}%
\pgfpathrectangle{\pgfqpoint{1.150000in}{0.150000in}}{\pgfqpoint{5.700000in}{5.700000in}}%
\pgfusepath{clip}%
\pgfsetbuttcap%
\pgfsetroundjoin%
\definecolor{currentfill}{rgb}{0.206756,0.371758,0.553117}%
\pgfsetfillcolor{currentfill}%
\pgfsetfillopacity{0.700000}%
\pgfsetlinewidth{0.000000pt}%
\definecolor{currentstroke}{rgb}{0.000000,0.000000,0.000000}%
\pgfsetstrokecolor{currentstroke}%
\pgfsetdash{}{0pt}%
\pgfpathmoveto{\pgfqpoint{5.753598in}{2.940608in}}%
\pgfpathlineto{\pgfqpoint{5.767554in}{2.937242in}}%
\pgfpathlineto{\pgfqpoint{5.781518in}{2.933942in}}%
\pgfpathlineto{\pgfqpoint{5.795491in}{2.930706in}}%
\pgfpathlineto{\pgfqpoint{5.809472in}{2.927536in}}%
\pgfpathlineto{\pgfqpoint{5.816927in}{2.942939in}}%
\pgfpathlineto{\pgfqpoint{5.824393in}{2.958755in}}%
\pgfpathlineto{\pgfqpoint{5.831869in}{2.974994in}}%
\pgfpathlineto{\pgfqpoint{5.839357in}{2.991667in}}%
\pgfpathlineto{\pgfqpoint{5.825398in}{2.995379in}}%
\pgfpathlineto{\pgfqpoint{5.811448in}{2.999156in}}%
\pgfpathlineto{\pgfqpoint{5.797507in}{3.002997in}}%
\pgfpathlineto{\pgfqpoint{5.783573in}{3.006903in}}%
\pgfpathlineto{\pgfqpoint{5.776063in}{2.989682in}}%
\pgfpathlineto{\pgfqpoint{5.768564in}{2.972899in}}%
\pgfpathlineto{\pgfqpoint{5.761076in}{2.956544in}}%
\pgfpathlineto{\pgfqpoint{5.753598in}{2.940608in}}%
\pgfpathclose%
\pgfusepath{fill}%
\end{pgfscope}%
\begin{pgfscope}%
\pgfpathrectangle{\pgfqpoint{1.150000in}{0.150000in}}{\pgfqpoint{5.700000in}{5.700000in}}%
\pgfusepath{clip}%
\pgfsetbuttcap%
\pgfsetroundjoin%
\definecolor{currentfill}{rgb}{0.269308,0.218818,0.509577}%
\pgfsetfillcolor{currentfill}%
\pgfsetfillopacity{0.700000}%
\pgfsetlinewidth{0.000000pt}%
\definecolor{currentstroke}{rgb}{0.000000,0.000000,0.000000}%
\pgfsetstrokecolor{currentstroke}%
\pgfsetdash{}{0pt}%
\pgfpathmoveto{\pgfqpoint{4.929213in}{2.600238in}}%
\pgfpathlineto{\pgfqpoint{4.943003in}{2.598150in}}%
\pgfpathlineto{\pgfqpoint{4.956801in}{2.596133in}}%
\pgfpathlineto{\pgfqpoint{4.970608in}{2.594185in}}%
\pgfpathlineto{\pgfqpoint{4.984422in}{2.592308in}}%
\pgfpathlineto{\pgfqpoint{4.991964in}{2.602144in}}%
\pgfpathlineto{\pgfqpoint{4.999504in}{2.612151in}}%
\pgfpathlineto{\pgfqpoint{5.007042in}{2.622336in}}%
\pgfpathlineto{\pgfqpoint{5.014579in}{2.632706in}}%
\pgfpathlineto{\pgfqpoint{5.000780in}{2.634945in}}%
\pgfpathlineto{\pgfqpoint{4.986990in}{2.637253in}}%
\pgfpathlineto{\pgfqpoint{4.973208in}{2.639631in}}%
\pgfpathlineto{\pgfqpoint{4.959434in}{2.642079in}}%
\pgfpathlineto{\pgfqpoint{4.951881in}{2.631342in}}%
\pgfpathlineto{\pgfqpoint{4.944327in}{2.620794in}}%
\pgfpathlineto{\pgfqpoint{4.936771in}{2.610428in}}%
\pgfpathlineto{\pgfqpoint{4.929213in}{2.600238in}}%
\pgfpathclose%
\pgfusepath{fill}%
\end{pgfscope}%
\begin{pgfscope}%
\pgfpathrectangle{\pgfqpoint{1.150000in}{0.150000in}}{\pgfqpoint{5.700000in}{5.700000in}}%
\pgfusepath{clip}%
\pgfsetbuttcap%
\pgfsetroundjoin%
\definecolor{currentfill}{rgb}{0.241237,0.296485,0.539709}%
\pgfsetfillcolor{currentfill}%
\pgfsetfillopacity{0.700000}%
\pgfsetlinewidth{0.000000pt}%
\definecolor{currentstroke}{rgb}{0.000000,0.000000,0.000000}%
\pgfsetstrokecolor{currentstroke}%
\pgfsetdash{}{0pt}%
\pgfpathmoveto{\pgfqpoint{5.411476in}{2.766115in}}%
\pgfpathlineto{\pgfqpoint{5.425379in}{2.763661in}}%
\pgfpathlineto{\pgfqpoint{5.439291in}{2.761274in}}%
\pgfpathlineto{\pgfqpoint{5.453211in}{2.758953in}}%
\pgfpathlineto{\pgfqpoint{5.467141in}{2.756699in}}%
\pgfpathlineto{\pgfqpoint{5.474579in}{2.768572in}}%
\pgfpathlineto{\pgfqpoint{5.482021in}{2.780734in}}%
\pgfpathlineto{\pgfqpoint{5.489466in}{2.793192in}}%
\pgfpathlineto{\pgfqpoint{5.496916in}{2.805955in}}%
\pgfpathlineto{\pgfqpoint{5.483008in}{2.808670in}}%
\pgfpathlineto{\pgfqpoint{5.469108in}{2.811452in}}%
\pgfpathlineto{\pgfqpoint{5.455216in}{2.814300in}}%
\pgfpathlineto{\pgfqpoint{5.441333in}{2.817215in}}%
\pgfpathlineto{\pgfqpoint{5.433863in}{2.803983in}}%
\pgfpathlineto{\pgfqpoint{5.426397in}{2.791062in}}%
\pgfpathlineto{\pgfqpoint{5.418935in}{2.778442in}}%
\pgfpathlineto{\pgfqpoint{5.411476in}{2.766115in}}%
\pgfpathclose%
\pgfusepath{fill}%
\end{pgfscope}%
\begin{pgfscope}%
\pgfpathrectangle{\pgfqpoint{1.150000in}{0.150000in}}{\pgfqpoint{5.700000in}{5.700000in}}%
\pgfusepath{clip}%
\pgfsetbuttcap%
\pgfsetroundjoin%
\definecolor{currentfill}{rgb}{0.195860,0.395433,0.555276}%
\pgfsetfillcolor{currentfill}%
\pgfsetfillopacity{0.700000}%
\pgfsetlinewidth{0.000000pt}%
\definecolor{currentstroke}{rgb}{0.000000,0.000000,0.000000}%
\pgfsetstrokecolor{currentstroke}%
\pgfsetdash{}{0pt}%
\pgfpathmoveto{\pgfqpoint{5.839357in}{2.991667in}}%
\pgfpathlineto{\pgfqpoint{5.853324in}{2.988019in}}%
\pgfpathlineto{\pgfqpoint{5.867299in}{2.984436in}}%
\pgfpathlineto{\pgfqpoint{5.881284in}{2.980918in}}%
\pgfpathlineto{\pgfqpoint{5.895276in}{2.977464in}}%
\pgfpathlineto{\pgfqpoint{5.902753in}{2.994026in}}%
\pgfpathlineto{\pgfqpoint{5.910242in}{3.011036in}}%
\pgfpathlineto{\pgfqpoint{5.917745in}{3.028505in}}%
\pgfpathlineto{\pgfqpoint{5.925261in}{3.046444in}}%
\pgfpathlineto{\pgfqpoint{5.911292in}{3.050460in}}%
\pgfpathlineto{\pgfqpoint{5.897331in}{3.054540in}}%
\pgfpathlineto{\pgfqpoint{5.883378in}{3.058684in}}%
\pgfpathlineto{\pgfqpoint{5.869434in}{3.062893in}}%
\pgfpathlineto{\pgfqpoint{5.861895in}{3.044385in}}%
\pgfpathlineto{\pgfqpoint{5.854369in}{3.026352in}}%
\pgfpathlineto{\pgfqpoint{5.846857in}{3.008782in}}%
\pgfpathlineto{\pgfqpoint{5.839357in}{2.991667in}}%
\pgfpathclose%
\pgfusepath{fill}%
\end{pgfscope}%
\begin{pgfscope}%
\pgfpathrectangle{\pgfqpoint{1.150000in}{0.150000in}}{\pgfqpoint{5.700000in}{5.700000in}}%
\pgfusepath{clip}%
\pgfsetbuttcap%
\pgfsetroundjoin%
\definecolor{currentfill}{rgb}{0.280255,0.165693,0.476498}%
\pgfsetfillcolor{currentfill}%
\pgfsetfillopacity{0.700000}%
\pgfsetlinewidth{0.000000pt}%
\definecolor{currentstroke}{rgb}{0.000000,0.000000,0.000000}%
\pgfsetstrokecolor{currentstroke}%
\pgfsetdash{}{0pt}%
\pgfpathmoveto{\pgfqpoint{4.532607in}{2.488017in}}%
\pgfpathlineto{\pgfqpoint{4.546295in}{2.485737in}}%
\pgfpathlineto{\pgfqpoint{4.559992in}{2.483530in}}%
\pgfpathlineto{\pgfqpoint{4.573696in}{2.481397in}}%
\pgfpathlineto{\pgfqpoint{4.587408in}{2.479338in}}%
\pgfpathlineto{\pgfqpoint{4.595075in}{2.488616in}}%
\pgfpathlineto{\pgfqpoint{4.602737in}{2.497998in}}%
\pgfpathlineto{\pgfqpoint{4.610396in}{2.507489in}}%
\pgfpathlineto{\pgfqpoint{4.618051in}{2.517094in}}%
\pgfpathlineto{\pgfqpoint{4.604353in}{2.519435in}}%
\pgfpathlineto{\pgfqpoint{4.590662in}{2.521848in}}%
\pgfpathlineto{\pgfqpoint{4.576979in}{2.524335in}}%
\pgfpathlineto{\pgfqpoint{4.563303in}{2.526895in}}%
\pgfpathlineto{\pgfqpoint{4.555635in}{2.517002in}}%
\pgfpathlineto{\pgfqpoint{4.547963in}{2.507228in}}%
\pgfpathlineto{\pgfqpoint{4.540287in}{2.497568in}}%
\pgfpathlineto{\pgfqpoint{4.532607in}{2.488017in}}%
\pgfpathclose%
\pgfusepath{fill}%
\end{pgfscope}%
\begin{pgfscope}%
\pgfpathrectangle{\pgfqpoint{1.150000in}{0.150000in}}{\pgfqpoint{5.700000in}{5.700000in}}%
\pgfusepath{clip}%
\pgfsetbuttcap%
\pgfsetroundjoin%
\definecolor{currentfill}{rgb}{0.283072,0.130895,0.449241}%
\pgfsetfillcolor{currentfill}%
\pgfsetfillopacity{0.700000}%
\pgfsetlinewidth{0.000000pt}%
\definecolor{currentstroke}{rgb}{0.000000,0.000000,0.000000}%
\pgfsetstrokecolor{currentstroke}%
\pgfsetdash{}{0pt}%
\pgfpathmoveto{\pgfqpoint{4.221484in}{2.415311in}}%
\pgfpathlineto{\pgfqpoint{4.235093in}{2.412542in}}%
\pgfpathlineto{\pgfqpoint{4.248709in}{2.409851in}}%
\pgfpathlineto{\pgfqpoint{4.262332in}{2.407238in}}%
\pgfpathlineto{\pgfqpoint{4.275962in}{2.404701in}}%
\pgfpathlineto{\pgfqpoint{4.283734in}{2.413948in}}%
\pgfpathlineto{\pgfqpoint{4.291501in}{2.423266in}}%
\pgfpathlineto{\pgfqpoint{4.299264in}{2.432657in}}%
\pgfpathlineto{\pgfqpoint{4.307021in}{2.442125in}}%
\pgfpathlineto{\pgfqpoint{4.293403in}{2.444882in}}%
\pgfpathlineto{\pgfqpoint{4.279791in}{2.447716in}}%
\pgfpathlineto{\pgfqpoint{4.266187in}{2.450626in}}%
\pgfpathlineto{\pgfqpoint{4.252589in}{2.453614in}}%
\pgfpathlineto{\pgfqpoint{4.244820in}{2.443919in}}%
\pgfpathlineto{\pgfqpoint{4.237046in}{2.434306in}}%
\pgfpathlineto{\pgfqpoint{4.229268in}{2.424771in}}%
\pgfpathlineto{\pgfqpoint{4.221484in}{2.415311in}}%
\pgfpathclose%
\pgfusepath{fill}%
\end{pgfscope}%
\begin{pgfscope}%
\pgfpathrectangle{\pgfqpoint{1.150000in}{0.150000in}}{\pgfqpoint{5.700000in}{5.700000in}}%
\pgfusepath{clip}%
\pgfsetbuttcap%
\pgfsetroundjoin%
\definecolor{currentfill}{rgb}{0.246811,0.283237,0.535941}%
\pgfsetfillcolor{currentfill}%
\pgfsetfillopacity{0.700000}%
\pgfsetlinewidth{0.000000pt}%
\definecolor{currentstroke}{rgb}{0.000000,0.000000,0.000000}%
\pgfsetstrokecolor{currentstroke}%
\pgfsetdash{}{0pt}%
\pgfpathmoveto{\pgfqpoint{5.326064in}{2.728296in}}%
\pgfpathlineto{\pgfqpoint{5.339953in}{2.726015in}}%
\pgfpathlineto{\pgfqpoint{5.353850in}{2.723802in}}%
\pgfpathlineto{\pgfqpoint{5.367755in}{2.721655in}}%
\pgfpathlineto{\pgfqpoint{5.381670in}{2.719576in}}%
\pgfpathlineto{\pgfqpoint{5.389118in}{2.730811in}}%
\pgfpathlineto{\pgfqpoint{5.396568in}{2.742308in}}%
\pgfpathlineto{\pgfqpoint{5.404021in}{2.754073in}}%
\pgfpathlineto{\pgfqpoint{5.411476in}{2.766115in}}%
\pgfpathlineto{\pgfqpoint{5.397582in}{2.768635in}}%
\pgfpathlineto{\pgfqpoint{5.383696in}{2.771223in}}%
\pgfpathlineto{\pgfqpoint{5.369818in}{2.773877in}}%
\pgfpathlineto{\pgfqpoint{5.355949in}{2.776599in}}%
\pgfpathlineto{\pgfqpoint{5.348474in}{2.764109in}}%
\pgfpathlineto{\pgfqpoint{5.341002in}{2.751900in}}%
\pgfpathlineto{\pgfqpoint{5.333532in}{2.739965in}}%
\pgfpathlineto{\pgfqpoint{5.326064in}{2.728296in}}%
\pgfpathclose%
\pgfusepath{fill}%
\end{pgfscope}%
\begin{pgfscope}%
\pgfpathrectangle{\pgfqpoint{1.150000in}{0.150000in}}{\pgfqpoint{5.700000in}{5.700000in}}%
\pgfusepath{clip}%
\pgfsetbuttcap%
\pgfsetroundjoin%
\definecolor{currentfill}{rgb}{0.280894,0.078907,0.402329}%
\pgfsetfillcolor{currentfill}%
\pgfsetfillopacity{0.700000}%
\pgfsetlinewidth{0.000000pt}%
\definecolor{currentstroke}{rgb}{0.000000,0.000000,0.000000}%
\pgfsetstrokecolor{currentstroke}%
\pgfsetdash{}{0pt}%
\pgfpathmoveto{\pgfqpoint{3.179323in}{2.325790in}}%
\pgfpathlineto{\pgfqpoint{3.192744in}{2.319174in}}%
\pgfpathlineto{\pgfqpoint{3.206167in}{2.312659in}}%
\pgfpathlineto{\pgfqpoint{3.219594in}{2.306245in}}%
\pgfpathlineto{\pgfqpoint{3.233024in}{2.299931in}}%
\pgfpathlineto{\pgfqpoint{3.241151in}{2.309121in}}%
\pgfpathlineto{\pgfqpoint{3.249272in}{2.318359in}}%
\pgfpathlineto{\pgfqpoint{3.257387in}{2.327647in}}%
\pgfpathlineto{\pgfqpoint{3.265495in}{2.336986in}}%
\pgfpathlineto{\pgfqpoint{3.252077in}{2.343337in}}%
\pgfpathlineto{\pgfqpoint{3.238662in}{2.349788in}}%
\pgfpathlineto{\pgfqpoint{3.225250in}{2.356339in}}%
\pgfpathlineto{\pgfqpoint{3.211841in}{2.362992in}}%
\pgfpathlineto{\pgfqpoint{3.203722in}{2.353609in}}%
\pgfpathlineto{\pgfqpoint{3.195595in}{2.344282in}}%
\pgfpathlineto{\pgfqpoint{3.187463in}{2.335009in}}%
\pgfpathlineto{\pgfqpoint{3.179323in}{2.325790in}}%
\pgfpathclose%
\pgfusepath{fill}%
\end{pgfscope}%
\begin{pgfscope}%
\pgfpathrectangle{\pgfqpoint{1.150000in}{0.150000in}}{\pgfqpoint{5.700000in}{5.700000in}}%
\pgfusepath{clip}%
\pgfsetbuttcap%
\pgfsetroundjoin%
\definecolor{currentfill}{rgb}{0.281924,0.089666,0.412415}%
\pgfsetfillcolor{currentfill}%
\pgfsetfillopacity{0.700000}%
\pgfsetlinewidth{0.000000pt}%
\definecolor{currentstroke}{rgb}{0.000000,0.000000,0.000000}%
\pgfsetstrokecolor{currentstroke}%
\pgfsetdash{}{0pt}%
\pgfpathmoveto{\pgfqpoint{3.039329in}{2.345954in}}%
\pgfpathlineto{\pgfqpoint{3.052742in}{2.338519in}}%
\pgfpathlineto{\pgfqpoint{3.066157in}{2.331192in}}%
\pgfpathlineto{\pgfqpoint{3.079575in}{2.323970in}}%
\pgfpathlineto{\pgfqpoint{3.092995in}{2.316854in}}%
\pgfpathlineto{\pgfqpoint{3.101173in}{2.325877in}}%
\pgfpathlineto{\pgfqpoint{3.109345in}{2.334956in}}%
\pgfpathlineto{\pgfqpoint{3.117510in}{2.344090in}}%
\pgfpathlineto{\pgfqpoint{3.125668in}{2.353280in}}%
\pgfpathlineto{\pgfqpoint{3.112260in}{2.360413in}}%
\pgfpathlineto{\pgfqpoint{3.098855in}{2.367650in}}%
\pgfpathlineto{\pgfqpoint{3.085452in}{2.374994in}}%
\pgfpathlineto{\pgfqpoint{3.072051in}{2.382445in}}%
\pgfpathlineto{\pgfqpoint{3.063881in}{2.373230in}}%
\pgfpathlineto{\pgfqpoint{3.055704in}{2.364077in}}%
\pgfpathlineto{\pgfqpoint{3.047520in}{2.354985in}}%
\pgfpathlineto{\pgfqpoint{3.039329in}{2.345954in}}%
\pgfpathclose%
\pgfusepath{fill}%
\end{pgfscope}%
\begin{pgfscope}%
\pgfpathrectangle{\pgfqpoint{1.150000in}{0.150000in}}{\pgfqpoint{5.700000in}{5.700000in}}%
\pgfusepath{clip}%
\pgfsetbuttcap%
\pgfsetroundjoin%
\definecolor{currentfill}{rgb}{0.282623,0.140926,0.457517}%
\pgfsetfillcolor{currentfill}%
\pgfsetfillopacity{0.700000}%
\pgfsetlinewidth{0.000000pt}%
\definecolor{currentstroke}{rgb}{0.000000,0.000000,0.000000}%
\pgfsetstrokecolor{currentstroke}%
\pgfsetdash{}{0pt}%
\pgfpathmoveto{\pgfqpoint{2.704989in}{2.448768in}}%
\pgfpathlineto{\pgfqpoint{2.718413in}{2.438990in}}%
\pgfpathlineto{\pgfqpoint{2.731836in}{2.429338in}}%
\pgfpathlineto{\pgfqpoint{2.745259in}{2.419808in}}%
\pgfpathlineto{\pgfqpoint{2.758682in}{2.410402in}}%
\pgfpathlineto{\pgfqpoint{2.766989in}{2.418840in}}%
\pgfpathlineto{\pgfqpoint{2.775288in}{2.427356in}}%
\pgfpathlineto{\pgfqpoint{2.783579in}{2.435949in}}%
\pgfpathlineto{\pgfqpoint{2.791862in}{2.444620in}}%
\pgfpathlineto{\pgfqpoint{2.778454in}{2.454001in}}%
\pgfpathlineto{\pgfqpoint{2.765046in}{2.463505in}}%
\pgfpathlineto{\pgfqpoint{2.751638in}{2.473132in}}%
\pgfpathlineto{\pgfqpoint{2.738230in}{2.482884in}}%
\pgfpathlineto{\pgfqpoint{2.729932in}{2.474231in}}%
\pgfpathlineto{\pgfqpoint{2.721626in}{2.465661in}}%
\pgfpathlineto{\pgfqpoint{2.713312in}{2.457173in}}%
\pgfpathlineto{\pgfqpoint{2.704989in}{2.448768in}}%
\pgfpathclose%
\pgfusepath{fill}%
\end{pgfscope}%
\begin{pgfscope}%
\pgfpathrectangle{\pgfqpoint{1.150000in}{0.150000in}}{\pgfqpoint{5.700000in}{5.700000in}}%
\pgfusepath{clip}%
\pgfsetbuttcap%
\pgfsetroundjoin%
\definecolor{currentfill}{rgb}{0.271828,0.209303,0.504434}%
\pgfsetfillcolor{currentfill}%
\pgfsetfillopacity{0.700000}%
\pgfsetlinewidth{0.000000pt}%
\definecolor{currentstroke}{rgb}{0.000000,0.000000,0.000000}%
\pgfsetstrokecolor{currentstroke}%
\pgfsetdash{}{0pt}%
\pgfpathmoveto{\pgfqpoint{4.843820in}{2.568800in}}%
\pgfpathlineto{\pgfqpoint{4.857593in}{2.566772in}}%
\pgfpathlineto{\pgfqpoint{4.871374in}{2.564815in}}%
\pgfpathlineto{\pgfqpoint{4.885163in}{2.562929in}}%
\pgfpathlineto{\pgfqpoint{4.898960in}{2.561112in}}%
\pgfpathlineto{\pgfqpoint{4.906527in}{2.570661in}}%
\pgfpathlineto{\pgfqpoint{4.914091in}{2.580361in}}%
\pgfpathlineto{\pgfqpoint{4.921653in}{2.590217in}}%
\pgfpathlineto{\pgfqpoint{4.929213in}{2.600238in}}%
\pgfpathlineto{\pgfqpoint{4.915432in}{2.602395in}}%
\pgfpathlineto{\pgfqpoint{4.901658in}{2.604623in}}%
\pgfpathlineto{\pgfqpoint{4.887893in}{2.606921in}}%
\pgfpathlineto{\pgfqpoint{4.874135in}{2.609290in}}%
\pgfpathlineto{\pgfqpoint{4.866560in}{2.598921in}}%
\pgfpathlineto{\pgfqpoint{4.858982in}{2.588721in}}%
\pgfpathlineto{\pgfqpoint{4.851402in}{2.578682in}}%
\pgfpathlineto{\pgfqpoint{4.843820in}{2.568800in}}%
\pgfpathclose%
\pgfusepath{fill}%
\end{pgfscope}%
\begin{pgfscope}%
\pgfpathrectangle{\pgfqpoint{1.150000in}{0.150000in}}{\pgfqpoint{5.700000in}{5.700000in}}%
\pgfusepath{clip}%
\pgfsetbuttcap%
\pgfsetroundjoin%
\definecolor{currentfill}{rgb}{0.280894,0.078907,0.402329}%
\pgfsetfillcolor{currentfill}%
\pgfsetfillopacity{0.700000}%
\pgfsetlinewidth{0.000000pt}%
\definecolor{currentstroke}{rgb}{0.000000,0.000000,0.000000}%
\pgfsetstrokecolor{currentstroke}%
\pgfsetdash{}{0pt}%
\pgfpathmoveto{\pgfqpoint{3.319200in}{2.312568in}}%
\pgfpathlineto{\pgfqpoint{3.332635in}{2.306708in}}%
\pgfpathlineto{\pgfqpoint{3.346073in}{2.300943in}}%
\pgfpathlineto{\pgfqpoint{3.359516in}{2.295275in}}%
\pgfpathlineto{\pgfqpoint{3.372962in}{2.289702in}}%
\pgfpathlineto{\pgfqpoint{3.381041in}{2.298993in}}%
\pgfpathlineto{\pgfqpoint{3.389114in}{2.308327in}}%
\pgfpathlineto{\pgfqpoint{3.397182in}{2.317707in}}%
\pgfpathlineto{\pgfqpoint{3.405243in}{2.327132in}}%
\pgfpathlineto{\pgfqpoint{3.391807in}{2.332762in}}%
\pgfpathlineto{\pgfqpoint{3.378376in}{2.338488in}}%
\pgfpathlineto{\pgfqpoint{3.364949in}{2.344309in}}%
\pgfpathlineto{\pgfqpoint{3.351525in}{2.350227in}}%
\pgfpathlineto{\pgfqpoint{3.343453in}{2.340737in}}%
\pgfpathlineto{\pgfqpoint{3.335375in}{2.331298in}}%
\pgfpathlineto{\pgfqpoint{3.327291in}{2.321909in}}%
\pgfpathlineto{\pgfqpoint{3.319200in}{2.312568in}}%
\pgfpathclose%
\pgfusepath{fill}%
\end{pgfscope}%
\begin{pgfscope}%
\pgfpathrectangle{\pgfqpoint{1.150000in}{0.150000in}}{\pgfqpoint{5.700000in}{5.700000in}}%
\pgfusepath{clip}%
\pgfsetbuttcap%
\pgfsetroundjoin%
\definecolor{currentfill}{rgb}{0.281446,0.084320,0.407414}%
\pgfsetfillcolor{currentfill}%
\pgfsetfillopacity{0.700000}%
\pgfsetlinewidth{0.000000pt}%
\definecolor{currentstroke}{rgb}{0.000000,0.000000,0.000000}%
\pgfsetstrokecolor{currentstroke}%
\pgfsetdash{}{0pt}%
\pgfpathmoveto{\pgfqpoint{3.684724in}{2.324032in}}%
\pgfpathlineto{\pgfqpoint{3.698216in}{2.319768in}}%
\pgfpathlineto{\pgfqpoint{3.711712in}{2.315590in}}%
\pgfpathlineto{\pgfqpoint{3.725214in}{2.311498in}}%
\pgfpathlineto{\pgfqpoint{3.738722in}{2.307493in}}%
\pgfpathlineto{\pgfqpoint{3.746677in}{2.316861in}}%
\pgfpathlineto{\pgfqpoint{3.754627in}{2.326270in}}%
\pgfpathlineto{\pgfqpoint{3.762571in}{2.335722in}}%
\pgfpathlineto{\pgfqpoint{3.770509in}{2.345219in}}%
\pgfpathlineto{\pgfqpoint{3.757012in}{2.349343in}}%
\pgfpathlineto{\pgfqpoint{3.743521in}{2.353553in}}%
\pgfpathlineto{\pgfqpoint{3.730034in}{2.357849in}}%
\pgfpathlineto{\pgfqpoint{3.716553in}{2.362232in}}%
\pgfpathlineto{\pgfqpoint{3.708604in}{2.352609in}}%
\pgfpathlineto{\pgfqpoint{3.700650in}{2.343036in}}%
\pgfpathlineto{\pgfqpoint{3.692690in}{2.333512in}}%
\pgfpathlineto{\pgfqpoint{3.684724in}{2.324032in}}%
\pgfpathclose%
\pgfusepath{fill}%
\end{pgfscope}%
\begin{pgfscope}%
\pgfpathrectangle{\pgfqpoint{1.150000in}{0.150000in}}{\pgfqpoint{5.700000in}{5.700000in}}%
\pgfusepath{clip}%
\pgfsetbuttcap%
\pgfsetroundjoin%
\definecolor{currentfill}{rgb}{0.253935,0.265254,0.529983}%
\pgfsetfillcolor{currentfill}%
\pgfsetfillopacity{0.700000}%
\pgfsetlinewidth{0.000000pt}%
\definecolor{currentstroke}{rgb}{0.000000,0.000000,0.000000}%
\pgfsetstrokecolor{currentstroke}%
\pgfsetdash{}{0pt}%
\pgfpathmoveto{\pgfqpoint{5.240666in}{2.692233in}}%
\pgfpathlineto{\pgfqpoint{5.254539in}{2.690104in}}%
\pgfpathlineto{\pgfqpoint{5.268420in}{2.688042in}}%
\pgfpathlineto{\pgfqpoint{5.282310in}{2.686049in}}%
\pgfpathlineto{\pgfqpoint{5.296209in}{2.684122in}}%
\pgfpathlineto{\pgfqpoint{5.303671in}{2.694805in}}%
\pgfpathlineto{\pgfqpoint{5.311134in}{2.705723in}}%
\pgfpathlineto{\pgfqpoint{5.318599in}{2.716884in}}%
\pgfpathlineto{\pgfqpoint{5.326064in}{2.728296in}}%
\pgfpathlineto{\pgfqpoint{5.312185in}{2.730643in}}%
\pgfpathlineto{\pgfqpoint{5.298314in}{2.733059in}}%
\pgfpathlineto{\pgfqpoint{5.284451in}{2.735541in}}%
\pgfpathlineto{\pgfqpoint{5.270597in}{2.738092in}}%
\pgfpathlineto{\pgfqpoint{5.263112in}{2.726252in}}%
\pgfpathlineto{\pgfqpoint{5.255629in}{2.714667in}}%
\pgfpathlineto{\pgfqpoint{5.248147in}{2.703330in}}%
\pgfpathlineto{\pgfqpoint{5.240666in}{2.692233in}}%
\pgfpathclose%
\pgfusepath{fill}%
\end{pgfscope}%
\begin{pgfscope}%
\pgfpathrectangle{\pgfqpoint{1.150000in}{0.150000in}}{\pgfqpoint{5.700000in}{5.700000in}}%
\pgfusepath{clip}%
\pgfsetbuttcap%
\pgfsetroundjoin%
\definecolor{currentfill}{rgb}{0.282656,0.100196,0.422160}%
\pgfsetfillcolor{currentfill}%
\pgfsetfillopacity{0.700000}%
\pgfsetlinewidth{0.000000pt}%
\definecolor{currentstroke}{rgb}{0.000000,0.000000,0.000000}%
\pgfsetstrokecolor{currentstroke}%
\pgfsetdash{}{0pt}%
\pgfpathmoveto{\pgfqpoint{3.910302in}{2.352656in}}%
\pgfpathlineto{\pgfqpoint{3.923841in}{2.349148in}}%
\pgfpathlineto{\pgfqpoint{3.937385in}{2.345723in}}%
\pgfpathlineto{\pgfqpoint{3.950935in}{2.342379in}}%
\pgfpathlineto{\pgfqpoint{3.964491in}{2.339117in}}%
\pgfpathlineto{\pgfqpoint{3.972371in}{2.348428in}}%
\pgfpathlineto{\pgfqpoint{3.980245in}{2.357786in}}%
\pgfpathlineto{\pgfqpoint{3.988113in}{2.367193in}}%
\pgfpathlineto{\pgfqpoint{3.995977in}{2.376654in}}%
\pgfpathlineto{\pgfqpoint{3.982431in}{2.380075in}}%
\pgfpathlineto{\pgfqpoint{3.968891in}{2.383577in}}%
\pgfpathlineto{\pgfqpoint{3.955357in}{2.387161in}}%
\pgfpathlineto{\pgfqpoint{3.941830in}{2.390828in}}%
\pgfpathlineto{\pgfqpoint{3.933956in}{2.381201in}}%
\pgfpathlineto{\pgfqpoint{3.926077in}{2.371632in}}%
\pgfpathlineto{\pgfqpoint{3.918192in}{2.362118in}}%
\pgfpathlineto{\pgfqpoint{3.910302in}{2.352656in}}%
\pgfpathclose%
\pgfusepath{fill}%
\end{pgfscope}%
\begin{pgfscope}%
\pgfpathrectangle{\pgfqpoint{1.150000in}{0.150000in}}{\pgfqpoint{5.700000in}{5.700000in}}%
\pgfusepath{clip}%
\pgfsetbuttcap%
\pgfsetroundjoin%
\definecolor{currentfill}{rgb}{0.282910,0.105393,0.426902}%
\pgfsetfillcolor{currentfill}%
\pgfsetfillopacity{0.700000}%
\pgfsetlinewidth{0.000000pt}%
\definecolor{currentstroke}{rgb}{0.000000,0.000000,0.000000}%
\pgfsetstrokecolor{currentstroke}%
\pgfsetdash{}{0pt}%
\pgfpathmoveto{\pgfqpoint{2.899143in}{2.373863in}}%
\pgfpathlineto{\pgfqpoint{2.912557in}{2.365542in}}%
\pgfpathlineto{\pgfqpoint{2.925972in}{2.357333in}}%
\pgfpathlineto{\pgfqpoint{2.939388in}{2.349237in}}%
\pgfpathlineto{\pgfqpoint{2.952806in}{2.341253in}}%
\pgfpathlineto{\pgfqpoint{2.961039in}{2.350038in}}%
\pgfpathlineto{\pgfqpoint{2.969265in}{2.358887in}}%
\pgfpathlineto{\pgfqpoint{2.977483in}{2.367799in}}%
\pgfpathlineto{\pgfqpoint{2.985694in}{2.376775in}}%
\pgfpathlineto{\pgfqpoint{2.972290in}{2.384755in}}%
\pgfpathlineto{\pgfqpoint{2.958887in}{2.392846in}}%
\pgfpathlineto{\pgfqpoint{2.945485in}{2.401050in}}%
\pgfpathlineto{\pgfqpoint{2.932085in}{2.409368in}}%
\pgfpathlineto{\pgfqpoint{2.923861in}{2.400389in}}%
\pgfpathlineto{\pgfqpoint{2.915629in}{2.391478in}}%
\pgfpathlineto{\pgfqpoint{2.907389in}{2.382637in}}%
\pgfpathlineto{\pgfqpoint{2.899143in}{2.373863in}}%
\pgfpathclose%
\pgfusepath{fill}%
\end{pgfscope}%
\begin{pgfscope}%
\pgfpathrectangle{\pgfqpoint{1.150000in}{0.150000in}}{\pgfqpoint{5.700000in}{5.700000in}}%
\pgfusepath{clip}%
\pgfsetbuttcap%
\pgfsetroundjoin%
\definecolor{currentfill}{rgb}{0.187231,0.414746,0.556547}%
\pgfsetfillcolor{currentfill}%
\pgfsetfillopacity{0.700000}%
\pgfsetlinewidth{0.000000pt}%
\definecolor{currentstroke}{rgb}{0.000000,0.000000,0.000000}%
\pgfsetstrokecolor{currentstroke}%
\pgfsetdash{}{0pt}%
\pgfpathmoveto{\pgfqpoint{5.925261in}{3.046444in}}%
\pgfpathlineto{\pgfqpoint{5.939239in}{3.042493in}}%
\pgfpathlineto{\pgfqpoint{5.953225in}{3.038605in}}%
\pgfpathlineto{\pgfqpoint{5.967219in}{3.034782in}}%
\pgfpathlineto{\pgfqpoint{5.981223in}{3.031024in}}%
\pgfpathlineto{\pgfqpoint{5.988729in}{3.048868in}}%
\pgfpathlineto{\pgfqpoint{5.996251in}{3.067198in}}%
\pgfpathlineto{\pgfqpoint{6.003789in}{3.086024in}}%
\pgfpathlineto{\pgfqpoint{5.989804in}{3.090218in}}%
\pgfpathlineto{\pgfqpoint{5.975827in}{3.094476in}}%
\pgfpathlineto{\pgfqpoint{5.961858in}{3.098797in}}%
\pgfpathlineto{\pgfqpoint{5.947898in}{3.103183in}}%
\pgfpathlineto{\pgfqpoint{5.940337in}{3.083772in}}%
\pgfpathlineto{\pgfqpoint{5.932791in}{3.064863in}}%
\pgfpathlineto{\pgfqpoint{5.925261in}{3.046444in}}%
\pgfpathclose%
\pgfusepath{fill}%
\end{pgfscope}%
\begin{pgfscope}%
\pgfpathrectangle{\pgfqpoint{1.150000in}{0.150000in}}{\pgfqpoint{5.700000in}{5.700000in}}%
\pgfusepath{clip}%
\pgfsetbuttcap%
\pgfsetroundjoin%
\definecolor{currentfill}{rgb}{0.277134,0.185228,0.489898}%
\pgfsetfillcolor{currentfill}%
\pgfsetfillopacity{0.700000}%
\pgfsetlinewidth{0.000000pt}%
\definecolor{currentstroke}{rgb}{0.000000,0.000000,0.000000}%
\pgfsetstrokecolor{currentstroke}%
\pgfsetdash{}{0pt}%
\pgfpathmoveto{\pgfqpoint{2.510267in}{2.543982in}}%
\pgfpathlineto{\pgfqpoint{2.523721in}{2.532575in}}%
\pgfpathlineto{\pgfqpoint{2.537173in}{2.521307in}}%
\pgfpathlineto{\pgfqpoint{2.550623in}{2.510176in}}%
\pgfpathlineto{\pgfqpoint{2.564072in}{2.499181in}}%
\pgfpathlineto{\pgfqpoint{2.572462in}{2.507146in}}%
\pgfpathlineto{\pgfqpoint{2.580843in}{2.515206in}}%
\pgfpathlineto{\pgfqpoint{2.589215in}{2.523358in}}%
\pgfpathlineto{\pgfqpoint{2.597578in}{2.531603in}}%
\pgfpathlineto{\pgfqpoint{2.584147in}{2.542551in}}%
\pgfpathlineto{\pgfqpoint{2.570714in}{2.553635in}}%
\pgfpathlineto{\pgfqpoint{2.557280in}{2.564857in}}%
\pgfpathlineto{\pgfqpoint{2.543844in}{2.576216in}}%
\pgfpathlineto{\pgfqpoint{2.535463in}{2.568011in}}%
\pgfpathlineto{\pgfqpoint{2.527074in}{2.559903in}}%
\pgfpathlineto{\pgfqpoint{2.518675in}{2.551893in}}%
\pgfpathlineto{\pgfqpoint{2.510267in}{2.543982in}}%
\pgfpathclose%
\pgfusepath{fill}%
\end{pgfscope}%
\begin{pgfscope}%
\pgfpathrectangle{\pgfqpoint{1.150000in}{0.150000in}}{\pgfqpoint{5.700000in}{5.700000in}}%
\pgfusepath{clip}%
\pgfsetbuttcap%
\pgfsetroundjoin%
\definecolor{currentfill}{rgb}{0.281412,0.155834,0.469201}%
\pgfsetfillcolor{currentfill}%
\pgfsetfillopacity{0.700000}%
\pgfsetlinewidth{0.000000pt}%
\definecolor{currentstroke}{rgb}{0.000000,0.000000,0.000000}%
\pgfsetstrokecolor{currentstroke}%
\pgfsetdash{}{0pt}%
\pgfpathmoveto{\pgfqpoint{4.447113in}{2.459623in}}%
\pgfpathlineto{\pgfqpoint{4.460784in}{2.457307in}}%
\pgfpathlineto{\pgfqpoint{4.474463in}{2.455065in}}%
\pgfpathlineto{\pgfqpoint{4.488149in}{2.452898in}}%
\pgfpathlineto{\pgfqpoint{4.501843in}{2.450804in}}%
\pgfpathlineto{\pgfqpoint{4.509541in}{2.459968in}}%
\pgfpathlineto{\pgfqpoint{4.517234in}{2.469222in}}%
\pgfpathlineto{\pgfqpoint{4.524922in}{2.478570in}}%
\pgfpathlineto{\pgfqpoint{4.532607in}{2.488017in}}%
\pgfpathlineto{\pgfqpoint{4.518925in}{2.490371in}}%
\pgfpathlineto{\pgfqpoint{4.505252in}{2.492799in}}%
\pgfpathlineto{\pgfqpoint{4.491586in}{2.495301in}}%
\pgfpathlineto{\pgfqpoint{4.477927in}{2.497878in}}%
\pgfpathlineto{\pgfqpoint{4.470230in}{2.488163in}}%
\pgfpathlineto{\pgfqpoint{4.462528in}{2.478552in}}%
\pgfpathlineto{\pgfqpoint{4.454823in}{2.469041in}}%
\pgfpathlineto{\pgfqpoint{4.447113in}{2.459623in}}%
\pgfpathclose%
\pgfusepath{fill}%
\end{pgfscope}%
\begin{pgfscope}%
\pgfpathrectangle{\pgfqpoint{1.150000in}{0.150000in}}{\pgfqpoint{5.700000in}{5.700000in}}%
\pgfusepath{clip}%
\pgfsetbuttcap%
\pgfsetroundjoin%
\definecolor{currentfill}{rgb}{0.280267,0.073417,0.397163}%
\pgfsetfillcolor{currentfill}%
\pgfsetfillopacity{0.700000}%
\pgfsetlinewidth{0.000000pt}%
\definecolor{currentstroke}{rgb}{0.000000,0.000000,0.000000}%
\pgfsetstrokecolor{currentstroke}%
\pgfsetdash{}{0pt}%
\pgfpathmoveto{\pgfqpoint{3.459023in}{2.305551in}}%
\pgfpathlineto{\pgfqpoint{3.472478in}{2.300388in}}%
\pgfpathlineto{\pgfqpoint{3.485938in}{2.295318in}}%
\pgfpathlineto{\pgfqpoint{3.499402in}{2.290339in}}%
\pgfpathlineto{\pgfqpoint{3.512871in}{2.285452in}}%
\pgfpathlineto{\pgfqpoint{3.520904in}{2.294784in}}%
\pgfpathlineto{\pgfqpoint{3.528932in}{2.304156in}}%
\pgfpathlineto{\pgfqpoint{3.536953in}{2.313569in}}%
\pgfpathlineto{\pgfqpoint{3.544969in}{2.323025in}}%
\pgfpathlineto{\pgfqpoint{3.531511in}{2.327991in}}%
\pgfpathlineto{\pgfqpoint{3.518057in}{2.333047in}}%
\pgfpathlineto{\pgfqpoint{3.504608in}{2.338195in}}%
\pgfpathlineto{\pgfqpoint{3.491163in}{2.343435in}}%
\pgfpathlineto{\pgfqpoint{3.483137in}{2.333894in}}%
\pgfpathlineto{\pgfqpoint{3.475105in}{2.324401in}}%
\pgfpathlineto{\pgfqpoint{3.467067in}{2.314954in}}%
\pgfpathlineto{\pgfqpoint{3.459023in}{2.305551in}}%
\pgfpathclose%
\pgfusepath{fill}%
\end{pgfscope}%
\begin{pgfscope}%
\pgfpathrectangle{\pgfqpoint{1.150000in}{0.150000in}}{\pgfqpoint{5.700000in}{5.700000in}}%
\pgfusepath{clip}%
\pgfsetbuttcap%
\pgfsetroundjoin%
\definecolor{currentfill}{rgb}{0.283229,0.120777,0.440584}%
\pgfsetfillcolor{currentfill}%
\pgfsetfillopacity{0.700000}%
\pgfsetlinewidth{0.000000pt}%
\definecolor{currentstroke}{rgb}{0.000000,0.000000,0.000000}%
\pgfsetstrokecolor{currentstroke}%
\pgfsetdash{}{0pt}%
\pgfpathmoveto{\pgfqpoint{4.135886in}{2.389190in}}%
\pgfpathlineto{\pgfqpoint{4.149479in}{2.386309in}}%
\pgfpathlineto{\pgfqpoint{4.163078in}{2.383506in}}%
\pgfpathlineto{\pgfqpoint{4.176685in}{2.380782in}}%
\pgfpathlineto{\pgfqpoint{4.190298in}{2.378135in}}%
\pgfpathlineto{\pgfqpoint{4.198102in}{2.387337in}}%
\pgfpathlineto{\pgfqpoint{4.205901in}{2.396598in}}%
\pgfpathlineto{\pgfqpoint{4.213695in}{2.405921in}}%
\pgfpathlineto{\pgfqpoint{4.221484in}{2.415311in}}%
\pgfpathlineto{\pgfqpoint{4.207882in}{2.418157in}}%
\pgfpathlineto{\pgfqpoint{4.194286in}{2.421081in}}%
\pgfpathlineto{\pgfqpoint{4.180698in}{2.424083in}}%
\pgfpathlineto{\pgfqpoint{4.167116in}{2.427163in}}%
\pgfpathlineto{\pgfqpoint{4.159316in}{2.417566in}}%
\pgfpathlineto{\pgfqpoint{4.151511in}{2.408041in}}%
\pgfpathlineto{\pgfqpoint{4.143701in}{2.398583in}}%
\pgfpathlineto{\pgfqpoint{4.135886in}{2.389190in}}%
\pgfpathclose%
\pgfusepath{fill}%
\end{pgfscope}%
\begin{pgfscope}%
\pgfpathrectangle{\pgfqpoint{1.150000in}{0.150000in}}{\pgfqpoint{5.700000in}{5.700000in}}%
\pgfusepath{clip}%
\pgfsetbuttcap%
\pgfsetroundjoin%
\definecolor{currentfill}{rgb}{0.275191,0.194905,0.496005}%
\pgfsetfillcolor{currentfill}%
\pgfsetfillopacity{0.700000}%
\pgfsetlinewidth{0.000000pt}%
\definecolor{currentstroke}{rgb}{0.000000,0.000000,0.000000}%
\pgfsetstrokecolor{currentstroke}%
\pgfsetdash{}{0pt}%
\pgfpathmoveto{\pgfqpoint{4.758392in}{2.538248in}}%
\pgfpathlineto{\pgfqpoint{4.772147in}{2.536257in}}%
\pgfpathlineto{\pgfqpoint{4.785910in}{2.534338in}}%
\pgfpathlineto{\pgfqpoint{4.799682in}{2.532489in}}%
\pgfpathlineto{\pgfqpoint{4.813461in}{2.530712in}}%
\pgfpathlineto{\pgfqpoint{4.821056in}{2.540029in}}%
\pgfpathlineto{\pgfqpoint{4.828647in}{2.549480in}}%
\pgfpathlineto{\pgfqpoint{4.836235in}{2.559068in}}%
\pgfpathlineto{\pgfqpoint{4.843820in}{2.568800in}}%
\pgfpathlineto{\pgfqpoint{4.830055in}{2.570899in}}%
\pgfpathlineto{\pgfqpoint{4.816299in}{2.573068in}}%
\pgfpathlineto{\pgfqpoint{4.802550in}{2.575309in}}%
\pgfpathlineto{\pgfqpoint{4.788810in}{2.577621in}}%
\pgfpathlineto{\pgfqpoint{4.781210in}{2.567560in}}%
\pgfpathlineto{\pgfqpoint{4.773607in}{2.557649in}}%
\pgfpathlineto{\pgfqpoint{4.766001in}{2.547880in}}%
\pgfpathlineto{\pgfqpoint{4.758392in}{2.538248in}}%
\pgfpathclose%
\pgfusepath{fill}%
\end{pgfscope}%
\begin{pgfscope}%
\pgfpathrectangle{\pgfqpoint{1.150000in}{0.150000in}}{\pgfqpoint{5.700000in}{5.700000in}}%
\pgfusepath{clip}%
\pgfsetbuttcap%
\pgfsetroundjoin%
\definecolor{currentfill}{rgb}{0.258965,0.251537,0.524736}%
\pgfsetfillcolor{currentfill}%
\pgfsetfillopacity{0.700000}%
\pgfsetlinewidth{0.000000pt}%
\definecolor{currentstroke}{rgb}{0.000000,0.000000,0.000000}%
\pgfsetstrokecolor{currentstroke}%
\pgfsetdash{}{0pt}%
\pgfpathmoveto{\pgfqpoint{5.155266in}{2.657687in}}%
\pgfpathlineto{\pgfqpoint{5.169123in}{2.655688in}}%
\pgfpathlineto{\pgfqpoint{5.182989in}{2.653756in}}%
\pgfpathlineto{\pgfqpoint{5.196863in}{2.651893in}}%
\pgfpathlineto{\pgfqpoint{5.210746in}{2.650097in}}%
\pgfpathlineto{\pgfqpoint{5.218226in}{2.660308in}}%
\pgfpathlineto{\pgfqpoint{5.225705in}{2.670729in}}%
\pgfpathlineto{\pgfqpoint{5.233185in}{2.681368in}}%
\pgfpathlineto{\pgfqpoint{5.240666in}{2.692233in}}%
\pgfpathlineto{\pgfqpoint{5.226801in}{2.694430in}}%
\pgfpathlineto{\pgfqpoint{5.212946in}{2.696695in}}%
\pgfpathlineto{\pgfqpoint{5.199098in}{2.699028in}}%
\pgfpathlineto{\pgfqpoint{5.185260in}{2.701429in}}%
\pgfpathlineto{\pgfqpoint{5.177761in}{2.690156in}}%
\pgfpathlineto{\pgfqpoint{5.170263in}{2.679113in}}%
\pgfpathlineto{\pgfqpoint{5.162765in}{2.668292in}}%
\pgfpathlineto{\pgfqpoint{5.155266in}{2.657687in}}%
\pgfpathclose%
\pgfusepath{fill}%
\end{pgfscope}%
\begin{pgfscope}%
\pgfpathrectangle{\pgfqpoint{1.150000in}{0.150000in}}{\pgfqpoint{5.700000in}{5.700000in}}%
\pgfusepath{clip}%
\pgfsetbuttcap%
\pgfsetroundjoin%
\definecolor{currentfill}{rgb}{0.283187,0.125848,0.444960}%
\pgfsetfillcolor{currentfill}%
\pgfsetfillopacity{0.700000}%
\pgfsetlinewidth{0.000000pt}%
\definecolor{currentstroke}{rgb}{0.000000,0.000000,0.000000}%
\pgfsetstrokecolor{currentstroke}%
\pgfsetdash{}{0pt}%
\pgfpathmoveto{\pgfqpoint{2.758682in}{2.410402in}}%
\pgfpathlineto{\pgfqpoint{2.772106in}{2.401117in}}%
\pgfpathlineto{\pgfqpoint{2.785529in}{2.391952in}}%
\pgfpathlineto{\pgfqpoint{2.798953in}{2.382907in}}%
\pgfpathlineto{\pgfqpoint{2.812377in}{2.373981in}}%
\pgfpathlineto{\pgfqpoint{2.820669in}{2.382452in}}%
\pgfpathlineto{\pgfqpoint{2.828953in}{2.390996in}}%
\pgfpathlineto{\pgfqpoint{2.837229in}{2.399611in}}%
\pgfpathlineto{\pgfqpoint{2.845497in}{2.408300in}}%
\pgfpathlineto{\pgfqpoint{2.832088in}{2.417201in}}%
\pgfpathlineto{\pgfqpoint{2.818679in}{2.426221in}}%
\pgfpathlineto{\pgfqpoint{2.805270in}{2.435360in}}%
\pgfpathlineto{\pgfqpoint{2.791862in}{2.444620in}}%
\pgfpathlineto{\pgfqpoint{2.783579in}{2.435949in}}%
\pgfpathlineto{\pgfqpoint{2.775288in}{2.427356in}}%
\pgfpathlineto{\pgfqpoint{2.766989in}{2.418840in}}%
\pgfpathlineto{\pgfqpoint{2.758682in}{2.410402in}}%
\pgfpathclose%
\pgfusepath{fill}%
\end{pgfscope}%
\begin{pgfscope}%
\pgfpathrectangle{\pgfqpoint{1.150000in}{0.150000in}}{\pgfqpoint{5.700000in}{5.700000in}}%
\pgfusepath{clip}%
\pgfsetbuttcap%
\pgfsetroundjoin%
\definecolor{currentfill}{rgb}{0.282327,0.094955,0.417331}%
\pgfsetfillcolor{currentfill}%
\pgfsetfillopacity{0.700000}%
\pgfsetlinewidth{0.000000pt}%
\definecolor{currentstroke}{rgb}{0.000000,0.000000,0.000000}%
\pgfsetstrokecolor{currentstroke}%
\pgfsetdash{}{0pt}%
\pgfpathmoveto{\pgfqpoint{3.824553in}{2.329573in}}%
\pgfpathlineto{\pgfqpoint{3.838078in}{2.325872in}}%
\pgfpathlineto{\pgfqpoint{3.851608in}{2.322255in}}%
\pgfpathlineto{\pgfqpoint{3.865145in}{2.318721in}}%
\pgfpathlineto{\pgfqpoint{3.878687in}{2.315270in}}%
\pgfpathlineto{\pgfqpoint{3.886600in}{2.324553in}}%
\pgfpathlineto{\pgfqpoint{3.894506in}{2.333876in}}%
\pgfpathlineto{\pgfqpoint{3.902407in}{2.343243in}}%
\pgfpathlineto{\pgfqpoint{3.910302in}{2.352656in}}%
\pgfpathlineto{\pgfqpoint{3.896770in}{2.356246in}}%
\pgfpathlineto{\pgfqpoint{3.883244in}{2.359919in}}%
\pgfpathlineto{\pgfqpoint{3.869724in}{2.363674in}}%
\pgfpathlineto{\pgfqpoint{3.856209in}{2.367514in}}%
\pgfpathlineto{\pgfqpoint{3.848303in}{2.357955in}}%
\pgfpathlineto{\pgfqpoint{3.840392in}{2.348447in}}%
\pgfpathlineto{\pgfqpoint{3.832475in}{2.338987in}}%
\pgfpathlineto{\pgfqpoint{3.824553in}{2.329573in}}%
\pgfpathclose%
\pgfusepath{fill}%
\end{pgfscope}%
\begin{pgfscope}%
\pgfpathrectangle{\pgfqpoint{1.150000in}{0.150000in}}{\pgfqpoint{5.700000in}{5.700000in}}%
\pgfusepath{clip}%
\pgfsetbuttcap%
\pgfsetroundjoin%
\definecolor{currentfill}{rgb}{0.263663,0.237631,0.518762}%
\pgfsetfillcolor{currentfill}%
\pgfsetfillopacity{0.700000}%
\pgfsetlinewidth{0.000000pt}%
\definecolor{currentstroke}{rgb}{0.000000,0.000000,0.000000}%
\pgfsetstrokecolor{currentstroke}%
\pgfsetdash{}{0pt}%
\pgfpathmoveto{\pgfqpoint{5.069855in}{2.624443in}}%
\pgfpathlineto{\pgfqpoint{5.083696in}{2.622550in}}%
\pgfpathlineto{\pgfqpoint{5.097544in}{2.620726in}}%
\pgfpathlineto{\pgfqpoint{5.111402in}{2.618971in}}%
\pgfpathlineto{\pgfqpoint{5.125268in}{2.617285in}}%
\pgfpathlineto{\pgfqpoint{5.132769in}{2.627097in}}%
\pgfpathlineto{\pgfqpoint{5.140269in}{2.637097in}}%
\pgfpathlineto{\pgfqpoint{5.147768in}{2.647291in}}%
\pgfpathlineto{\pgfqpoint{5.155266in}{2.657687in}}%
\pgfpathlineto{\pgfqpoint{5.141418in}{2.659756in}}%
\pgfpathlineto{\pgfqpoint{5.127578in}{2.661893in}}%
\pgfpathlineto{\pgfqpoint{5.113747in}{2.664098in}}%
\pgfpathlineto{\pgfqpoint{5.099924in}{2.666372in}}%
\pgfpathlineto{\pgfqpoint{5.092408in}{2.655587in}}%
\pgfpathlineto{\pgfqpoint{5.084891in}{2.645009in}}%
\pgfpathlineto{\pgfqpoint{5.077374in}{2.634630in}}%
\pgfpathlineto{\pgfqpoint{5.069855in}{2.624443in}}%
\pgfpathclose%
\pgfusepath{fill}%
\end{pgfscope}%
\begin{pgfscope}%
\pgfpathrectangle{\pgfqpoint{1.150000in}{0.150000in}}{\pgfqpoint{5.700000in}{5.700000in}}%
\pgfusepath{clip}%
\pgfsetbuttcap%
\pgfsetroundjoin%
\definecolor{currentfill}{rgb}{0.282290,0.145912,0.461510}%
\pgfsetfillcolor{currentfill}%
\pgfsetfillopacity{0.700000}%
\pgfsetlinewidth{0.000000pt}%
\definecolor{currentstroke}{rgb}{0.000000,0.000000,0.000000}%
\pgfsetstrokecolor{currentstroke}%
\pgfsetdash{}{0pt}%
\pgfpathmoveto{\pgfqpoint{4.361565in}{2.431861in}}%
\pgfpathlineto{\pgfqpoint{4.375220in}{2.429485in}}%
\pgfpathlineto{\pgfqpoint{4.388881in}{2.427184in}}%
\pgfpathlineto{\pgfqpoint{4.402550in}{2.424958in}}%
\pgfpathlineto{\pgfqpoint{4.416226in}{2.422807in}}%
\pgfpathlineto{\pgfqpoint{4.423955in}{2.431892in}}%
\pgfpathlineto{\pgfqpoint{4.431679in}{2.441054in}}%
\pgfpathlineto{\pgfqpoint{4.439398in}{2.450296in}}%
\pgfpathlineto{\pgfqpoint{4.447113in}{2.459623in}}%
\pgfpathlineto{\pgfqpoint{4.433449in}{2.462015in}}%
\pgfpathlineto{\pgfqpoint{4.419792in}{2.464481in}}%
\pgfpathlineto{\pgfqpoint{4.406143in}{2.467022in}}%
\pgfpathlineto{\pgfqpoint{4.392501in}{2.469638in}}%
\pgfpathlineto{\pgfqpoint{4.384774in}{2.460063in}}%
\pgfpathlineto{\pgfqpoint{4.377042in}{2.450578in}}%
\pgfpathlineto{\pgfqpoint{4.369306in}{2.441179in}}%
\pgfpathlineto{\pgfqpoint{4.361565in}{2.431861in}}%
\pgfpathclose%
\pgfusepath{fill}%
\end{pgfscope}%
\begin{pgfscope}%
\pgfpathrectangle{\pgfqpoint{1.150000in}{0.150000in}}{\pgfqpoint{5.700000in}{5.700000in}}%
\pgfusepath{clip}%
\pgfsetbuttcap%
\pgfsetroundjoin%
\definecolor{currentfill}{rgb}{0.279574,0.170599,0.479997}%
\pgfsetfillcolor{currentfill}%
\pgfsetfillopacity{0.700000}%
\pgfsetlinewidth{0.000000pt}%
\definecolor{currentstroke}{rgb}{0.000000,0.000000,0.000000}%
\pgfsetstrokecolor{currentstroke}%
\pgfsetdash{}{0pt}%
\pgfpathmoveto{\pgfqpoint{2.564072in}{2.499181in}}%
\pgfpathlineto{\pgfqpoint{2.577519in}{2.488320in}}%
\pgfpathlineto{\pgfqpoint{2.590964in}{2.477594in}}%
\pgfpathlineto{\pgfqpoint{2.604409in}{2.466999in}}%
\pgfpathlineto{\pgfqpoint{2.617852in}{2.456536in}}%
\pgfpathlineto{\pgfqpoint{2.626225in}{2.464556in}}%
\pgfpathlineto{\pgfqpoint{2.634589in}{2.472665in}}%
\pgfpathlineto{\pgfqpoint{2.642944in}{2.480861in}}%
\pgfpathlineto{\pgfqpoint{2.651291in}{2.489145in}}%
\pgfpathlineto{\pgfqpoint{2.637864in}{2.499562in}}%
\pgfpathlineto{\pgfqpoint{2.624437in}{2.510109in}}%
\pgfpathlineto{\pgfqpoint{2.611008in}{2.520789in}}%
\pgfpathlineto{\pgfqpoint{2.597578in}{2.531603in}}%
\pgfpathlineto{\pgfqpoint{2.589215in}{2.523358in}}%
\pgfpathlineto{\pgfqpoint{2.580843in}{2.515206in}}%
\pgfpathlineto{\pgfqpoint{2.572462in}{2.507146in}}%
\pgfpathlineto{\pgfqpoint{2.564072in}{2.499181in}}%
\pgfpathclose%
\pgfusepath{fill}%
\end{pgfscope}%
\begin{pgfscope}%
\pgfpathrectangle{\pgfqpoint{1.150000in}{0.150000in}}{\pgfqpoint{5.700000in}{5.700000in}}%
\pgfusepath{clip}%
\pgfsetbuttcap%
\pgfsetroundjoin%
\definecolor{currentfill}{rgb}{0.280894,0.078907,0.402329}%
\pgfsetfillcolor{currentfill}%
\pgfsetfillopacity{0.700000}%
\pgfsetlinewidth{0.000000pt}%
\definecolor{currentstroke}{rgb}{0.000000,0.000000,0.000000}%
\pgfsetstrokecolor{currentstroke}%
\pgfsetdash{}{0pt}%
\pgfpathmoveto{\pgfqpoint{3.598847in}{2.304065in}}%
\pgfpathlineto{\pgfqpoint{3.612329in}{2.299548in}}%
\pgfpathlineto{\pgfqpoint{3.625816in}{2.295119in}}%
\pgfpathlineto{\pgfqpoint{3.639307in}{2.290778in}}%
\pgfpathlineto{\pgfqpoint{3.652804in}{2.286525in}}%
\pgfpathlineto{\pgfqpoint{3.660793in}{2.295845in}}%
\pgfpathlineto{\pgfqpoint{3.668776in}{2.305201in}}%
\pgfpathlineto{\pgfqpoint{3.676753in}{2.314596in}}%
\pgfpathlineto{\pgfqpoint{3.684724in}{2.324032in}}%
\pgfpathlineto{\pgfqpoint{3.671238in}{2.328384in}}%
\pgfpathlineto{\pgfqpoint{3.657757in}{2.332823in}}%
\pgfpathlineto{\pgfqpoint{3.644281in}{2.337349in}}%
\pgfpathlineto{\pgfqpoint{3.630809in}{2.341964in}}%
\pgfpathlineto{\pgfqpoint{3.622828in}{2.332423in}}%
\pgfpathlineto{\pgfqpoint{3.614840in}{2.322927in}}%
\pgfpathlineto{\pgfqpoint{3.606847in}{2.313475in}}%
\pgfpathlineto{\pgfqpoint{3.598847in}{2.304065in}}%
\pgfpathclose%
\pgfusepath{fill}%
\end{pgfscope}%
\begin{pgfscope}%
\pgfpathrectangle{\pgfqpoint{1.150000in}{0.150000in}}{\pgfqpoint{5.700000in}{5.700000in}}%
\pgfusepath{clip}%
\pgfsetbuttcap%
\pgfsetroundjoin%
\definecolor{currentfill}{rgb}{0.277134,0.185228,0.489898}%
\pgfsetfillcolor{currentfill}%
\pgfsetfillopacity{0.700000}%
\pgfsetlinewidth{0.000000pt}%
\definecolor{currentstroke}{rgb}{0.000000,0.000000,0.000000}%
\pgfsetstrokecolor{currentstroke}%
\pgfsetdash{}{0pt}%
\pgfpathmoveto{\pgfqpoint{4.672922in}{2.508460in}}%
\pgfpathlineto{\pgfqpoint{4.686660in}{2.506483in}}%
\pgfpathlineto{\pgfqpoint{4.700405in}{2.504578in}}%
\pgfpathlineto{\pgfqpoint{4.714159in}{2.502745in}}%
\pgfpathlineto{\pgfqpoint{4.727921in}{2.500983in}}%
\pgfpathlineto{\pgfqpoint{4.735544in}{2.510121in}}%
\pgfpathlineto{\pgfqpoint{4.743163in}{2.519375in}}%
\pgfpathlineto{\pgfqpoint{4.750779in}{2.528748in}}%
\pgfpathlineto{\pgfqpoint{4.758392in}{2.538248in}}%
\pgfpathlineto{\pgfqpoint{4.744644in}{2.540311in}}%
\pgfpathlineto{\pgfqpoint{4.730905in}{2.542445in}}%
\pgfpathlineto{\pgfqpoint{4.717174in}{2.544651in}}%
\pgfpathlineto{\pgfqpoint{4.703451in}{2.546929in}}%
\pgfpathlineto{\pgfqpoint{4.695824in}{2.537121in}}%
\pgfpathlineto{\pgfqpoint{4.688194in}{2.527444in}}%
\pgfpathlineto{\pgfqpoint{4.680560in}{2.517892in}}%
\pgfpathlineto{\pgfqpoint{4.672922in}{2.508460in}}%
\pgfpathclose%
\pgfusepath{fill}%
\end{pgfscope}%
\begin{pgfscope}%
\pgfpathrectangle{\pgfqpoint{1.150000in}{0.150000in}}{\pgfqpoint{5.700000in}{5.700000in}}%
\pgfusepath{clip}%
\pgfsetbuttcap%
\pgfsetroundjoin%
\definecolor{currentfill}{rgb}{0.214298,0.355619,0.551184}%
\pgfsetfillcolor{currentfill}%
\pgfsetfillopacity{0.700000}%
\pgfsetlinewidth{0.000000pt}%
\definecolor{currentstroke}{rgb}{0.000000,0.000000,0.000000}%
\pgfsetstrokecolor{currentstroke}%
\pgfsetdash{}{0pt}%
\pgfpathmoveto{\pgfqpoint{5.723781in}{2.880847in}}%
\pgfpathlineto{\pgfqpoint{5.737760in}{2.878003in}}%
\pgfpathlineto{\pgfqpoint{5.751747in}{2.875225in}}%
\pgfpathlineto{\pgfqpoint{5.765743in}{2.872511in}}%
\pgfpathlineto{\pgfqpoint{5.779748in}{2.869862in}}%
\pgfpathlineto{\pgfqpoint{5.787166in}{2.883709in}}%
\pgfpathlineto{\pgfqpoint{5.794592in}{2.897930in}}%
\pgfpathlineto{\pgfqpoint{5.802027in}{2.912536in}}%
\pgfpathlineto{\pgfqpoint{5.809472in}{2.927536in}}%
\pgfpathlineto{\pgfqpoint{5.795491in}{2.930706in}}%
\pgfpathlineto{\pgfqpoint{5.781518in}{2.933942in}}%
\pgfpathlineto{\pgfqpoint{5.767554in}{2.937242in}}%
\pgfpathlineto{\pgfqpoint{5.753598in}{2.940608in}}%
\pgfpathlineto{\pgfqpoint{5.746130in}{2.925079in}}%
\pgfpathlineto{\pgfqpoint{5.738672in}{2.909949in}}%
\pgfpathlineto{\pgfqpoint{5.731222in}{2.895208in}}%
\pgfpathlineto{\pgfqpoint{5.723781in}{2.880847in}}%
\pgfpathclose%
\pgfusepath{fill}%
\end{pgfscope}%
\begin{pgfscope}%
\pgfpathrectangle{\pgfqpoint{1.150000in}{0.150000in}}{\pgfqpoint{5.700000in}{5.700000in}}%
\pgfusepath{clip}%
\pgfsetbuttcap%
\pgfsetroundjoin%
\definecolor{currentfill}{rgb}{0.281446,0.084320,0.407414}%
\pgfsetfillcolor{currentfill}%
\pgfsetfillopacity{0.700000}%
\pgfsetlinewidth{0.000000pt}%
\definecolor{currentstroke}{rgb}{0.000000,0.000000,0.000000}%
\pgfsetstrokecolor{currentstroke}%
\pgfsetdash{}{0pt}%
\pgfpathmoveto{\pgfqpoint{3.092995in}{2.316854in}}%
\pgfpathlineto{\pgfqpoint{3.106418in}{2.309843in}}%
\pgfpathlineto{\pgfqpoint{3.119842in}{2.302936in}}%
\pgfpathlineto{\pgfqpoint{3.133270in}{2.296132in}}%
\pgfpathlineto{\pgfqpoint{3.146700in}{2.289430in}}%
\pgfpathlineto{\pgfqpoint{3.154866in}{2.298444in}}%
\pgfpathlineto{\pgfqpoint{3.163025in}{2.307509in}}%
\pgfpathlineto{\pgfqpoint{3.171177in}{2.316624in}}%
\pgfpathlineto{\pgfqpoint{3.179323in}{2.325790in}}%
\pgfpathlineto{\pgfqpoint{3.165905in}{2.332508in}}%
\pgfpathlineto{\pgfqpoint{3.152490in}{2.339329in}}%
\pgfpathlineto{\pgfqpoint{3.139078in}{2.346253in}}%
\pgfpathlineto{\pgfqpoint{3.125668in}{2.353280in}}%
\pgfpathlineto{\pgfqpoint{3.117510in}{2.344090in}}%
\pgfpathlineto{\pgfqpoint{3.109345in}{2.334956in}}%
\pgfpathlineto{\pgfqpoint{3.101173in}{2.325877in}}%
\pgfpathlineto{\pgfqpoint{3.092995in}{2.316854in}}%
\pgfpathclose%
\pgfusepath{fill}%
\end{pgfscope}%
\begin{pgfscope}%
\pgfpathrectangle{\pgfqpoint{1.150000in}{0.150000in}}{\pgfqpoint{5.700000in}{5.700000in}}%
\pgfusepath{clip}%
\pgfsetbuttcap%
\pgfsetroundjoin%
\definecolor{currentfill}{rgb}{0.223925,0.334994,0.548053}%
\pgfsetfillcolor{currentfill}%
\pgfsetfillopacity{0.700000}%
\pgfsetlinewidth{0.000000pt}%
\definecolor{currentstroke}{rgb}{0.000000,0.000000,0.000000}%
\pgfsetstrokecolor{currentstroke}%
\pgfsetdash{}{0pt}%
\pgfpathmoveto{\pgfqpoint{5.638177in}{2.837033in}}%
\pgfpathlineto{\pgfqpoint{5.652143in}{2.834430in}}%
\pgfpathlineto{\pgfqpoint{5.666118in}{2.831892in}}%
\pgfpathlineto{\pgfqpoint{5.680101in}{2.829420in}}%
\pgfpathlineto{\pgfqpoint{5.694094in}{2.827013in}}%
\pgfpathlineto{\pgfqpoint{5.701505in}{2.839948in}}%
\pgfpathlineto{\pgfqpoint{5.708923in}{2.853226in}}%
\pgfpathlineto{\pgfqpoint{5.716348in}{2.866856in}}%
\pgfpathlineto{\pgfqpoint{5.723781in}{2.880847in}}%
\pgfpathlineto{\pgfqpoint{5.709812in}{2.883756in}}%
\pgfpathlineto{\pgfqpoint{5.695851in}{2.886730in}}%
\pgfpathlineto{\pgfqpoint{5.681899in}{2.889770in}}%
\pgfpathlineto{\pgfqpoint{5.667955in}{2.892875in}}%
\pgfpathlineto{\pgfqpoint{5.660500in}{2.878375in}}%
\pgfpathlineto{\pgfqpoint{5.653052in}{2.864241in}}%
\pgfpathlineto{\pgfqpoint{5.645612in}{2.850463in}}%
\pgfpathlineto{\pgfqpoint{5.638177in}{2.837033in}}%
\pgfpathclose%
\pgfusepath{fill}%
\end{pgfscope}%
\begin{pgfscope}%
\pgfpathrectangle{\pgfqpoint{1.150000in}{0.150000in}}{\pgfqpoint{5.700000in}{5.700000in}}%
\pgfusepath{clip}%
\pgfsetbuttcap%
\pgfsetroundjoin%
\definecolor{currentfill}{rgb}{0.283091,0.110553,0.431554}%
\pgfsetfillcolor{currentfill}%
\pgfsetfillopacity{0.700000}%
\pgfsetlinewidth{0.000000pt}%
\definecolor{currentstroke}{rgb}{0.000000,0.000000,0.000000}%
\pgfsetstrokecolor{currentstroke}%
\pgfsetdash{}{0pt}%
\pgfpathmoveto{\pgfqpoint{4.050223in}{2.363778in}}%
\pgfpathlineto{\pgfqpoint{4.063800in}{2.360760in}}%
\pgfpathlineto{\pgfqpoint{4.077384in}{2.357821in}}%
\pgfpathlineto{\pgfqpoint{4.090975in}{2.354962in}}%
\pgfpathlineto{\pgfqpoint{4.104572in}{2.352182in}}%
\pgfpathlineto{\pgfqpoint{4.112408in}{2.361356in}}%
\pgfpathlineto{\pgfqpoint{4.120239in}{2.370579in}}%
\pgfpathlineto{\pgfqpoint{4.128065in}{2.379856in}}%
\pgfpathlineto{\pgfqpoint{4.135886in}{2.389190in}}%
\pgfpathlineto{\pgfqpoint{4.122299in}{2.392149in}}%
\pgfpathlineto{\pgfqpoint{4.108720in}{2.395188in}}%
\pgfpathlineto{\pgfqpoint{4.095146in}{2.398306in}}%
\pgfpathlineto{\pgfqpoint{4.081580in}{2.401504in}}%
\pgfpathlineto{\pgfqpoint{4.073748in}{2.391983in}}%
\pgfpathlineto{\pgfqpoint{4.065912in}{2.382525in}}%
\pgfpathlineto{\pgfqpoint{4.058070in}{2.373124in}}%
\pgfpathlineto{\pgfqpoint{4.050223in}{2.363778in}}%
\pgfpathclose%
\pgfusepath{fill}%
\end{pgfscope}%
\begin{pgfscope}%
\pgfpathrectangle{\pgfqpoint{1.150000in}{0.150000in}}{\pgfqpoint{5.700000in}{5.700000in}}%
\pgfusepath{clip}%
\pgfsetbuttcap%
\pgfsetroundjoin%
\definecolor{currentfill}{rgb}{0.280267,0.073417,0.397163}%
\pgfsetfillcolor{currentfill}%
\pgfsetfillopacity{0.700000}%
\pgfsetlinewidth{0.000000pt}%
\definecolor{currentstroke}{rgb}{0.000000,0.000000,0.000000}%
\pgfsetstrokecolor{currentstroke}%
\pgfsetdash{}{0pt}%
\pgfpathmoveto{\pgfqpoint{3.233024in}{2.299931in}}%
\pgfpathlineto{\pgfqpoint{3.246456in}{2.293716in}}%
\pgfpathlineto{\pgfqpoint{3.259893in}{2.287600in}}%
\pgfpathlineto{\pgfqpoint{3.273332in}{2.281582in}}%
\pgfpathlineto{\pgfqpoint{3.286775in}{2.275661in}}%
\pgfpathlineto{\pgfqpoint{3.294891in}{2.284822in}}%
\pgfpathlineto{\pgfqpoint{3.303000in}{2.294026in}}%
\pgfpathlineto{\pgfqpoint{3.311103in}{2.303274in}}%
\pgfpathlineto{\pgfqpoint{3.319200in}{2.312568in}}%
\pgfpathlineto{\pgfqpoint{3.305769in}{2.318526in}}%
\pgfpathlineto{\pgfqpoint{3.292341in}{2.324581in}}%
\pgfpathlineto{\pgfqpoint{3.278916in}{2.330734in}}%
\pgfpathlineto{\pgfqpoint{3.265495in}{2.336986in}}%
\pgfpathlineto{\pgfqpoint{3.257387in}{2.327647in}}%
\pgfpathlineto{\pgfqpoint{3.249272in}{2.318359in}}%
\pgfpathlineto{\pgfqpoint{3.241151in}{2.309121in}}%
\pgfpathlineto{\pgfqpoint{3.233024in}{2.299931in}}%
\pgfpathclose%
\pgfusepath{fill}%
\end{pgfscope}%
\begin{pgfscope}%
\pgfpathrectangle{\pgfqpoint{1.150000in}{0.150000in}}{\pgfqpoint{5.700000in}{5.700000in}}%
\pgfusepath{clip}%
\pgfsetbuttcap%
\pgfsetroundjoin%
\definecolor{currentfill}{rgb}{0.206756,0.371758,0.553117}%
\pgfsetfillcolor{currentfill}%
\pgfsetfillopacity{0.700000}%
\pgfsetlinewidth{0.000000pt}%
\definecolor{currentstroke}{rgb}{0.000000,0.000000,0.000000}%
\pgfsetstrokecolor{currentstroke}%
\pgfsetdash{}{0pt}%
\pgfpathmoveto{\pgfqpoint{5.809472in}{2.927536in}}%
\pgfpathlineto{\pgfqpoint{5.823463in}{2.924430in}}%
\pgfpathlineto{\pgfqpoint{5.837462in}{2.921389in}}%
\pgfpathlineto{\pgfqpoint{5.851470in}{2.918412in}}%
\pgfpathlineto{\pgfqpoint{5.865486in}{2.915500in}}%
\pgfpathlineto{\pgfqpoint{5.872917in}{2.930369in}}%
\pgfpathlineto{\pgfqpoint{5.880359in}{2.945646in}}%
\pgfpathlineto{\pgfqpoint{5.887812in}{2.961341in}}%
\pgfpathlineto{\pgfqpoint{5.895276in}{2.977464in}}%
\pgfpathlineto{\pgfqpoint{5.881284in}{2.980918in}}%
\pgfpathlineto{\pgfqpoint{5.867299in}{2.984436in}}%
\pgfpathlineto{\pgfqpoint{5.853324in}{2.988019in}}%
\pgfpathlineto{\pgfqpoint{5.839357in}{2.991667in}}%
\pgfpathlineto{\pgfqpoint{5.831869in}{2.974994in}}%
\pgfpathlineto{\pgfqpoint{5.824393in}{2.958755in}}%
\pgfpathlineto{\pgfqpoint{5.816927in}{2.942939in}}%
\pgfpathlineto{\pgfqpoint{5.809472in}{2.927536in}}%
\pgfpathclose%
\pgfusepath{fill}%
\end{pgfscope}%
\begin{pgfscope}%
\pgfpathrectangle{\pgfqpoint{1.150000in}{0.150000in}}{\pgfqpoint{5.700000in}{5.700000in}}%
\pgfusepath{clip}%
\pgfsetbuttcap%
\pgfsetroundjoin%
\definecolor{currentfill}{rgb}{0.231674,0.318106,0.544834}%
\pgfsetfillcolor{currentfill}%
\pgfsetfillopacity{0.700000}%
\pgfsetlinewidth{0.000000pt}%
\definecolor{currentstroke}{rgb}{0.000000,0.000000,0.000000}%
\pgfsetstrokecolor{currentstroke}%
\pgfsetdash{}{0pt}%
\pgfpathmoveto{\pgfqpoint{5.552637in}{2.795755in}}%
\pgfpathlineto{\pgfqpoint{5.566590in}{2.793371in}}%
\pgfpathlineto{\pgfqpoint{5.580551in}{2.791052in}}%
\pgfpathlineto{\pgfqpoint{5.594521in}{2.788799in}}%
\pgfpathlineto{\pgfqpoint{5.608500in}{2.786612in}}%
\pgfpathlineto{\pgfqpoint{5.615911in}{2.798740in}}%
\pgfpathlineto{\pgfqpoint{5.623328in}{2.811181in}}%
\pgfpathlineto{\pgfqpoint{5.630750in}{2.823942in}}%
\pgfpathlineto{\pgfqpoint{5.638177in}{2.837033in}}%
\pgfpathlineto{\pgfqpoint{5.624221in}{2.839702in}}%
\pgfpathlineto{\pgfqpoint{5.610273in}{2.842437in}}%
\pgfpathlineto{\pgfqpoint{5.596333in}{2.845237in}}%
\pgfpathlineto{\pgfqpoint{5.582403in}{2.848103in}}%
\pgfpathlineto{\pgfqpoint{5.574953in}{2.834523in}}%
\pgfpathlineto{\pgfqpoint{5.567509in}{2.821278in}}%
\pgfpathlineto{\pgfqpoint{5.560071in}{2.808358in}}%
\pgfpathlineto{\pgfqpoint{5.552637in}{2.795755in}}%
\pgfpathclose%
\pgfusepath{fill}%
\end{pgfscope}%
\begin{pgfscope}%
\pgfpathrectangle{\pgfqpoint{1.150000in}{0.150000in}}{\pgfqpoint{5.700000in}{5.700000in}}%
\pgfusepath{clip}%
\pgfsetbuttcap%
\pgfsetroundjoin%
\definecolor{currentfill}{rgb}{0.282327,0.094955,0.417331}%
\pgfsetfillcolor{currentfill}%
\pgfsetfillopacity{0.700000}%
\pgfsetlinewidth{0.000000pt}%
\definecolor{currentstroke}{rgb}{0.000000,0.000000,0.000000}%
\pgfsetstrokecolor{currentstroke}%
\pgfsetdash{}{0pt}%
\pgfpathmoveto{\pgfqpoint{2.952806in}{2.341253in}}%
\pgfpathlineto{\pgfqpoint{2.966226in}{2.333379in}}%
\pgfpathlineto{\pgfqpoint{2.979647in}{2.325616in}}%
\pgfpathlineto{\pgfqpoint{2.993070in}{2.317961in}}%
\pgfpathlineto{\pgfqpoint{3.006495in}{2.310416in}}%
\pgfpathlineto{\pgfqpoint{3.014714in}{2.319213in}}%
\pgfpathlineto{\pgfqpoint{3.022926in}{2.328068in}}%
\pgfpathlineto{\pgfqpoint{3.031131in}{2.336981in}}%
\pgfpathlineto{\pgfqpoint{3.039329in}{2.345954in}}%
\pgfpathlineto{\pgfqpoint{3.025917in}{2.353496in}}%
\pgfpathlineto{\pgfqpoint{3.012508in}{2.361146in}}%
\pgfpathlineto{\pgfqpoint{2.999100in}{2.368905in}}%
\pgfpathlineto{\pgfqpoint{2.985694in}{2.376775in}}%
\pgfpathlineto{\pgfqpoint{2.977483in}{2.367799in}}%
\pgfpathlineto{\pgfqpoint{2.969265in}{2.358887in}}%
\pgfpathlineto{\pgfqpoint{2.961039in}{2.350038in}}%
\pgfpathlineto{\pgfqpoint{2.952806in}{2.341253in}}%
\pgfpathclose%
\pgfusepath{fill}%
\end{pgfscope}%
\begin{pgfscope}%
\pgfpathrectangle{\pgfqpoint{1.150000in}{0.150000in}}{\pgfqpoint{5.700000in}{5.700000in}}%
\pgfusepath{clip}%
\pgfsetbuttcap%
\pgfsetroundjoin%
\definecolor{currentfill}{rgb}{0.195860,0.395433,0.555276}%
\pgfsetfillcolor{currentfill}%
\pgfsetfillopacity{0.700000}%
\pgfsetlinewidth{0.000000pt}%
\definecolor{currentstroke}{rgb}{0.000000,0.000000,0.000000}%
\pgfsetstrokecolor{currentstroke}%
\pgfsetdash{}{0pt}%
\pgfpathmoveto{\pgfqpoint{5.895276in}{2.977464in}}%
\pgfpathlineto{\pgfqpoint{5.909278in}{2.974074in}}%
\pgfpathlineto{\pgfqpoint{5.923288in}{2.970749in}}%
\pgfpathlineto{\pgfqpoint{5.937307in}{2.967488in}}%
\pgfpathlineto{\pgfqpoint{5.951335in}{2.964292in}}%
\pgfpathlineto{\pgfqpoint{5.958787in}{2.980299in}}%
\pgfpathlineto{\pgfqpoint{5.966252in}{2.996750in}}%
\pgfpathlineto{\pgfqpoint{5.973730in}{3.013654in}}%
\pgfpathlineto{\pgfqpoint{5.981223in}{3.031024in}}%
\pgfpathlineto{\pgfqpoint{5.967219in}{3.034782in}}%
\pgfpathlineto{\pgfqpoint{5.953225in}{3.038605in}}%
\pgfpathlineto{\pgfqpoint{5.939239in}{3.042493in}}%
\pgfpathlineto{\pgfqpoint{5.925261in}{3.046444in}}%
\pgfpathlineto{\pgfqpoint{5.917745in}{3.028505in}}%
\pgfpathlineto{\pgfqpoint{5.910242in}{3.011036in}}%
\pgfpathlineto{\pgfqpoint{5.902753in}{2.994026in}}%
\pgfpathlineto{\pgfqpoint{5.895276in}{2.977464in}}%
\pgfpathclose%
\pgfusepath{fill}%
\end{pgfscope}%
\begin{pgfscope}%
\pgfpathrectangle{\pgfqpoint{1.150000in}{0.150000in}}{\pgfqpoint{5.700000in}{5.700000in}}%
\pgfusepath{clip}%
\pgfsetbuttcap%
\pgfsetroundjoin%
\definecolor{currentfill}{rgb}{0.266580,0.228262,0.514349}%
\pgfsetfillcolor{currentfill}%
\pgfsetfillopacity{0.700000}%
\pgfsetlinewidth{0.000000pt}%
\definecolor{currentstroke}{rgb}{0.000000,0.000000,0.000000}%
\pgfsetstrokecolor{currentstroke}%
\pgfsetdash{}{0pt}%
\pgfpathmoveto{\pgfqpoint{4.984422in}{2.592308in}}%
\pgfpathlineto{\pgfqpoint{4.998246in}{2.590499in}}%
\pgfpathlineto{\pgfqpoint{5.012077in}{2.588761in}}%
\pgfpathlineto{\pgfqpoint{5.025918in}{2.587092in}}%
\pgfpathlineto{\pgfqpoint{5.039767in}{2.585492in}}%
\pgfpathlineto{\pgfqpoint{5.047291in}{2.594974in}}%
\pgfpathlineto{\pgfqpoint{5.054814in}{2.604622in}}%
\pgfpathlineto{\pgfqpoint{5.062335in}{2.614443in}}%
\pgfpathlineto{\pgfqpoint{5.069855in}{2.624443in}}%
\pgfpathlineto{\pgfqpoint{5.056023in}{2.626405in}}%
\pgfpathlineto{\pgfqpoint{5.042200in}{2.628436in}}%
\pgfpathlineto{\pgfqpoint{5.028385in}{2.630536in}}%
\pgfpathlineto{\pgfqpoint{5.014579in}{2.632706in}}%
\pgfpathlineto{\pgfqpoint{5.007042in}{2.622336in}}%
\pgfpathlineto{\pgfqpoint{4.999504in}{2.612151in}}%
\pgfpathlineto{\pgfqpoint{4.991964in}{2.602144in}}%
\pgfpathlineto{\pgfqpoint{4.984422in}{2.592308in}}%
\pgfpathclose%
\pgfusepath{fill}%
\end{pgfscope}%
\begin{pgfscope}%
\pgfpathrectangle{\pgfqpoint{1.150000in}{0.150000in}}{\pgfqpoint{5.700000in}{5.700000in}}%
\pgfusepath{clip}%
\pgfsetbuttcap%
\pgfsetroundjoin%
\definecolor{currentfill}{rgb}{0.239346,0.300855,0.540844}%
\pgfsetfillcolor{currentfill}%
\pgfsetfillopacity{0.700000}%
\pgfsetlinewidth{0.000000pt}%
\definecolor{currentstroke}{rgb}{0.000000,0.000000,0.000000}%
\pgfsetstrokecolor{currentstroke}%
\pgfsetdash{}{0pt}%
\pgfpathmoveto{\pgfqpoint{5.467141in}{2.756699in}}%
\pgfpathlineto{\pgfqpoint{5.481079in}{2.754512in}}%
\pgfpathlineto{\pgfqpoint{5.495026in}{2.752390in}}%
\pgfpathlineto{\pgfqpoint{5.508982in}{2.750336in}}%
\pgfpathlineto{\pgfqpoint{5.522947in}{2.748347in}}%
\pgfpathlineto{\pgfqpoint{5.530364in}{2.759766in}}%
\pgfpathlineto{\pgfqpoint{5.537784in}{2.771467in}}%
\pgfpathlineto{\pgfqpoint{5.545209in}{2.783461in}}%
\pgfpathlineto{\pgfqpoint{5.552637in}{2.795755in}}%
\pgfpathlineto{\pgfqpoint{5.538694in}{2.798206in}}%
\pgfpathlineto{\pgfqpoint{5.524759in}{2.800723in}}%
\pgfpathlineto{\pgfqpoint{5.510833in}{2.803305in}}%
\pgfpathlineto{\pgfqpoint{5.496916in}{2.805955in}}%
\pgfpathlineto{\pgfqpoint{5.489466in}{2.793192in}}%
\pgfpathlineto{\pgfqpoint{5.482021in}{2.780734in}}%
\pgfpathlineto{\pgfqpoint{5.474579in}{2.768572in}}%
\pgfpathlineto{\pgfqpoint{5.467141in}{2.756699in}}%
\pgfpathclose%
\pgfusepath{fill}%
\end{pgfscope}%
\begin{pgfscope}%
\pgfpathrectangle{\pgfqpoint{1.150000in}{0.150000in}}{\pgfqpoint{5.700000in}{5.700000in}}%
\pgfusepath{clip}%
\pgfsetbuttcap%
\pgfsetroundjoin%
\definecolor{currentfill}{rgb}{0.280267,0.073417,0.397163}%
\pgfsetfillcolor{currentfill}%
\pgfsetfillopacity{0.700000}%
\pgfsetlinewidth{0.000000pt}%
\definecolor{currentstroke}{rgb}{0.000000,0.000000,0.000000}%
\pgfsetstrokecolor{currentstroke}%
\pgfsetdash{}{0pt}%
\pgfpathmoveto{\pgfqpoint{3.372962in}{2.289702in}}%
\pgfpathlineto{\pgfqpoint{3.386412in}{2.284223in}}%
\pgfpathlineto{\pgfqpoint{3.399866in}{2.278839in}}%
\pgfpathlineto{\pgfqpoint{3.413324in}{2.273548in}}%
\pgfpathlineto{\pgfqpoint{3.426786in}{2.268351in}}%
\pgfpathlineto{\pgfqpoint{3.434854in}{2.277592in}}%
\pgfpathlineto{\pgfqpoint{3.442916in}{2.286872in}}%
\pgfpathlineto{\pgfqpoint{3.450973in}{2.296191in}}%
\pgfpathlineto{\pgfqpoint{3.459023in}{2.305551in}}%
\pgfpathlineto{\pgfqpoint{3.445572in}{2.310806in}}%
\pgfpathlineto{\pgfqpoint{3.432124in}{2.316154in}}%
\pgfpathlineto{\pgfqpoint{3.418681in}{2.321596in}}%
\pgfpathlineto{\pgfqpoint{3.405243in}{2.327132in}}%
\pgfpathlineto{\pgfqpoint{3.397182in}{2.317707in}}%
\pgfpathlineto{\pgfqpoint{3.389114in}{2.308327in}}%
\pgfpathlineto{\pgfqpoint{3.381041in}{2.298993in}}%
\pgfpathlineto{\pgfqpoint{3.372962in}{2.289702in}}%
\pgfpathclose%
\pgfusepath{fill}%
\end{pgfscope}%
\begin{pgfscope}%
\pgfpathrectangle{\pgfqpoint{1.150000in}{0.150000in}}{\pgfqpoint{5.700000in}{5.700000in}}%
\pgfusepath{clip}%
\pgfsetbuttcap%
\pgfsetroundjoin%
\definecolor{currentfill}{rgb}{0.282884,0.135920,0.453427}%
\pgfsetfillcolor{currentfill}%
\pgfsetfillopacity{0.700000}%
\pgfsetlinewidth{0.000000pt}%
\definecolor{currentstroke}{rgb}{0.000000,0.000000,0.000000}%
\pgfsetstrokecolor{currentstroke}%
\pgfsetdash{}{0pt}%
\pgfpathmoveto{\pgfqpoint{4.275962in}{2.404701in}}%
\pgfpathlineto{\pgfqpoint{4.289599in}{2.402240in}}%
\pgfpathlineto{\pgfqpoint{4.303243in}{2.399856in}}%
\pgfpathlineto{\pgfqpoint{4.316894in}{2.397549in}}%
\pgfpathlineto{\pgfqpoint{4.330553in}{2.395317in}}%
\pgfpathlineto{\pgfqpoint{4.338314in}{2.404352in}}%
\pgfpathlineto{\pgfqpoint{4.346069in}{2.413452in}}%
\pgfpathlineto{\pgfqpoint{4.353820in}{2.422620in}}%
\pgfpathlineto{\pgfqpoint{4.361565in}{2.431861in}}%
\pgfpathlineto{\pgfqpoint{4.347919in}{2.434313in}}%
\pgfpathlineto{\pgfqpoint{4.334279in}{2.436841in}}%
\pgfpathlineto{\pgfqpoint{4.320646in}{2.439445in}}%
\pgfpathlineto{\pgfqpoint{4.307021in}{2.442125in}}%
\pgfpathlineto{\pgfqpoint{4.299264in}{2.432657in}}%
\pgfpathlineto{\pgfqpoint{4.291501in}{2.423266in}}%
\pgfpathlineto{\pgfqpoint{4.283734in}{2.413948in}}%
\pgfpathlineto{\pgfqpoint{4.275962in}{2.404701in}}%
\pgfpathclose%
\pgfusepath{fill}%
\end{pgfscope}%
\begin{pgfscope}%
\pgfpathrectangle{\pgfqpoint{1.150000in}{0.150000in}}{\pgfqpoint{5.700000in}{5.700000in}}%
\pgfusepath{clip}%
\pgfsetbuttcap%
\pgfsetroundjoin%
\definecolor{currentfill}{rgb}{0.279574,0.170599,0.479997}%
\pgfsetfillcolor{currentfill}%
\pgfsetfillopacity{0.700000}%
\pgfsetlinewidth{0.000000pt}%
\definecolor{currentstroke}{rgb}{0.000000,0.000000,0.000000}%
\pgfsetstrokecolor{currentstroke}%
\pgfsetdash{}{0pt}%
\pgfpathmoveto{\pgfqpoint{4.587408in}{2.479338in}}%
\pgfpathlineto{\pgfqpoint{4.601127in}{2.477351in}}%
\pgfpathlineto{\pgfqpoint{4.614855in}{2.475437in}}%
\pgfpathlineto{\pgfqpoint{4.628590in}{2.473596in}}%
\pgfpathlineto{\pgfqpoint{4.642334in}{2.471828in}}%
\pgfpathlineto{\pgfqpoint{4.649987in}{2.480832in}}%
\pgfpathlineto{\pgfqpoint{4.657636in}{2.489935in}}%
\pgfpathlineto{\pgfqpoint{4.665281in}{2.499143in}}%
\pgfpathlineto{\pgfqpoint{4.672922in}{2.508460in}}%
\pgfpathlineto{\pgfqpoint{4.659193in}{2.510510in}}%
\pgfpathlineto{\pgfqpoint{4.645471in}{2.512632in}}%
\pgfpathlineto{\pgfqpoint{4.631757in}{2.514827in}}%
\pgfpathlineto{\pgfqpoint{4.618051in}{2.517094in}}%
\pgfpathlineto{\pgfqpoint{4.610396in}{2.507489in}}%
\pgfpathlineto{\pgfqpoint{4.602737in}{2.497998in}}%
\pgfpathlineto{\pgfqpoint{4.595075in}{2.488616in}}%
\pgfpathlineto{\pgfqpoint{4.587408in}{2.479338in}}%
\pgfpathclose%
\pgfusepath{fill}%
\end{pgfscope}%
\begin{pgfscope}%
\pgfpathrectangle{\pgfqpoint{1.150000in}{0.150000in}}{\pgfqpoint{5.700000in}{5.700000in}}%
\pgfusepath{clip}%
\pgfsetbuttcap%
\pgfsetroundjoin%
\definecolor{currentfill}{rgb}{0.244972,0.287675,0.537260}%
\pgfsetfillcolor{currentfill}%
\pgfsetfillopacity{0.700000}%
\pgfsetlinewidth{0.000000pt}%
\definecolor{currentstroke}{rgb}{0.000000,0.000000,0.000000}%
\pgfsetstrokecolor{currentstroke}%
\pgfsetdash{}{0pt}%
\pgfpathmoveto{\pgfqpoint{5.381670in}{2.719576in}}%
\pgfpathlineto{\pgfqpoint{5.395593in}{2.717564in}}%
\pgfpathlineto{\pgfqpoint{5.409525in}{2.715619in}}%
\pgfpathlineto{\pgfqpoint{5.423466in}{2.713740in}}%
\pgfpathlineto{\pgfqpoint{5.437416in}{2.711929in}}%
\pgfpathlineto{\pgfqpoint{5.444844in}{2.722729in}}%
\pgfpathlineto{\pgfqpoint{5.452274in}{2.733785in}}%
\pgfpathlineto{\pgfqpoint{5.459706in}{2.745106in}}%
\pgfpathlineto{\pgfqpoint{5.467141in}{2.756699in}}%
\pgfpathlineto{\pgfqpoint{5.453211in}{2.758953in}}%
\pgfpathlineto{\pgfqpoint{5.439291in}{2.761274in}}%
\pgfpathlineto{\pgfqpoint{5.425379in}{2.763661in}}%
\pgfpathlineto{\pgfqpoint{5.411476in}{2.766115in}}%
\pgfpathlineto{\pgfqpoint{5.404021in}{2.754073in}}%
\pgfpathlineto{\pgfqpoint{5.396568in}{2.742308in}}%
\pgfpathlineto{\pgfqpoint{5.389118in}{2.730811in}}%
\pgfpathlineto{\pgfqpoint{5.381670in}{2.719576in}}%
\pgfpathclose%
\pgfusepath{fill}%
\end{pgfscope}%
\begin{pgfscope}%
\pgfpathrectangle{\pgfqpoint{1.150000in}{0.150000in}}{\pgfqpoint{5.700000in}{5.700000in}}%
\pgfusepath{clip}%
\pgfsetbuttcap%
\pgfsetroundjoin%
\definecolor{currentfill}{rgb}{0.281887,0.150881,0.465405}%
\pgfsetfillcolor{currentfill}%
\pgfsetfillopacity{0.700000}%
\pgfsetlinewidth{0.000000pt}%
\definecolor{currentstroke}{rgb}{0.000000,0.000000,0.000000}%
\pgfsetstrokecolor{currentstroke}%
\pgfsetdash{}{0pt}%
\pgfpathmoveto{\pgfqpoint{2.617852in}{2.456536in}}%
\pgfpathlineto{\pgfqpoint{2.631294in}{2.446203in}}%
\pgfpathlineto{\pgfqpoint{2.644736in}{2.436000in}}%
\pgfpathlineto{\pgfqpoint{2.658177in}{2.425924in}}%
\pgfpathlineto{\pgfqpoint{2.671617in}{2.415975in}}%
\pgfpathlineto{\pgfqpoint{2.679973in}{2.424048in}}%
\pgfpathlineto{\pgfqpoint{2.688320in}{2.432205in}}%
\pgfpathlineto{\pgfqpoint{2.696659in}{2.440445in}}%
\pgfpathlineto{\pgfqpoint{2.704989in}{2.448768in}}%
\pgfpathlineto{\pgfqpoint{2.691566in}{2.458671in}}%
\pgfpathlineto{\pgfqpoint{2.678141in}{2.468701in}}%
\pgfpathlineto{\pgfqpoint{2.664716in}{2.478858in}}%
\pgfpathlineto{\pgfqpoint{2.651291in}{2.489145in}}%
\pgfpathlineto{\pgfqpoint{2.642944in}{2.480861in}}%
\pgfpathlineto{\pgfqpoint{2.634589in}{2.472665in}}%
\pgfpathlineto{\pgfqpoint{2.626225in}{2.464556in}}%
\pgfpathlineto{\pgfqpoint{2.617852in}{2.456536in}}%
\pgfpathclose%
\pgfusepath{fill}%
\end{pgfscope}%
\begin{pgfscope}%
\pgfpathrectangle{\pgfqpoint{1.150000in}{0.150000in}}{\pgfqpoint{5.700000in}{5.700000in}}%
\pgfusepath{clip}%
\pgfsetbuttcap%
\pgfsetroundjoin%
\definecolor{currentfill}{rgb}{0.281446,0.084320,0.407414}%
\pgfsetfillcolor{currentfill}%
\pgfsetfillopacity{0.700000}%
\pgfsetlinewidth{0.000000pt}%
\definecolor{currentstroke}{rgb}{0.000000,0.000000,0.000000}%
\pgfsetstrokecolor{currentstroke}%
\pgfsetdash{}{0pt}%
\pgfpathmoveto{\pgfqpoint{3.738722in}{2.307493in}}%
\pgfpathlineto{\pgfqpoint{3.752235in}{2.303572in}}%
\pgfpathlineto{\pgfqpoint{3.765753in}{2.299737in}}%
\pgfpathlineto{\pgfqpoint{3.779277in}{2.295987in}}%
\pgfpathlineto{\pgfqpoint{3.792807in}{2.292321in}}%
\pgfpathlineto{\pgfqpoint{3.800752in}{2.301578in}}%
\pgfpathlineto{\pgfqpoint{3.808691in}{2.310871in}}%
\pgfpathlineto{\pgfqpoint{3.816625in}{2.320202in}}%
\pgfpathlineto{\pgfqpoint{3.824553in}{2.329573in}}%
\pgfpathlineto{\pgfqpoint{3.811033in}{2.333358in}}%
\pgfpathlineto{\pgfqpoint{3.797520in}{2.337227in}}%
\pgfpathlineto{\pgfqpoint{3.784012in}{2.341181in}}%
\pgfpathlineto{\pgfqpoint{3.770509in}{2.345219in}}%
\pgfpathlineto{\pgfqpoint{3.762571in}{2.335722in}}%
\pgfpathlineto{\pgfqpoint{3.754627in}{2.326270in}}%
\pgfpathlineto{\pgfqpoint{3.746677in}{2.316861in}}%
\pgfpathlineto{\pgfqpoint{3.738722in}{2.307493in}}%
\pgfpathclose%
\pgfusepath{fill}%
\end{pgfscope}%
\begin{pgfscope}%
\pgfpathrectangle{\pgfqpoint{1.150000in}{0.150000in}}{\pgfqpoint{5.700000in}{5.700000in}}%
\pgfusepath{clip}%
\pgfsetbuttcap%
\pgfsetroundjoin%
\definecolor{currentfill}{rgb}{0.283197,0.115680,0.436115}%
\pgfsetfillcolor{currentfill}%
\pgfsetfillopacity{0.700000}%
\pgfsetlinewidth{0.000000pt}%
\definecolor{currentstroke}{rgb}{0.000000,0.000000,0.000000}%
\pgfsetstrokecolor{currentstroke}%
\pgfsetdash{}{0pt}%
\pgfpathmoveto{\pgfqpoint{2.812377in}{2.373981in}}%
\pgfpathlineto{\pgfqpoint{2.825802in}{2.365173in}}%
\pgfpathlineto{\pgfqpoint{2.839228in}{2.356481in}}%
\pgfpathlineto{\pgfqpoint{2.852654in}{2.347906in}}%
\pgfpathlineto{\pgfqpoint{2.866082in}{2.339446in}}%
\pgfpathlineto{\pgfqpoint{2.874358in}{2.347949in}}%
\pgfpathlineto{\pgfqpoint{2.882627in}{2.356519in}}%
\pgfpathlineto{\pgfqpoint{2.890889in}{2.365158in}}%
\pgfpathlineto{\pgfqpoint{2.899143in}{2.373863in}}%
\pgfpathlineto{\pgfqpoint{2.885730in}{2.382299in}}%
\pgfpathlineto{\pgfqpoint{2.872318in}{2.390850in}}%
\pgfpathlineto{\pgfqpoint{2.858907in}{2.399517in}}%
\pgfpathlineto{\pgfqpoint{2.845497in}{2.408300in}}%
\pgfpathlineto{\pgfqpoint{2.837229in}{2.399611in}}%
\pgfpathlineto{\pgfqpoint{2.828953in}{2.390996in}}%
\pgfpathlineto{\pgfqpoint{2.820669in}{2.382452in}}%
\pgfpathlineto{\pgfqpoint{2.812377in}{2.373981in}}%
\pgfpathclose%
\pgfusepath{fill}%
\end{pgfscope}%
\begin{pgfscope}%
\pgfpathrectangle{\pgfqpoint{1.150000in}{0.150000in}}{\pgfqpoint{5.700000in}{5.700000in}}%
\pgfusepath{clip}%
\pgfsetbuttcap%
\pgfsetroundjoin%
\definecolor{currentfill}{rgb}{0.187231,0.414746,0.556547}%
\pgfsetfillcolor{currentfill}%
\pgfsetfillopacity{0.700000}%
\pgfsetlinewidth{0.000000pt}%
\definecolor{currentstroke}{rgb}{0.000000,0.000000,0.000000}%
\pgfsetstrokecolor{currentstroke}%
\pgfsetdash{}{0pt}%
\pgfpathmoveto{\pgfqpoint{5.981223in}{3.031024in}}%
\pgfpathlineto{\pgfqpoint{5.995235in}{3.027329in}}%
\pgfpathlineto{\pgfqpoint{6.009255in}{3.023698in}}%
\pgfpathlineto{\pgfqpoint{6.023284in}{3.020131in}}%
\pgfpathlineto{\pgfqpoint{6.037322in}{3.016628in}}%
\pgfpathlineto{\pgfqpoint{6.044804in}{3.033897in}}%
\pgfpathlineto{\pgfqpoint{6.052302in}{3.051648in}}%
\pgfpathlineto{\pgfqpoint{6.059815in}{3.069889in}}%
\pgfpathlineto{\pgfqpoint{6.045795in}{3.073827in}}%
\pgfpathlineto{\pgfqpoint{6.031785in}{3.077829in}}%
\pgfpathlineto{\pgfqpoint{6.017783in}{3.081895in}}%
\pgfpathlineto{\pgfqpoint{6.003789in}{3.086024in}}%
\pgfpathlineto{\pgfqpoint{5.996251in}{3.067198in}}%
\pgfpathlineto{\pgfqpoint{5.988729in}{3.048868in}}%
\pgfpathlineto{\pgfqpoint{5.981223in}{3.031024in}}%
\pgfpathclose%
\pgfusepath{fill}%
\end{pgfscope}%
\begin{pgfscope}%
\pgfpathrectangle{\pgfqpoint{1.150000in}{0.150000in}}{\pgfqpoint{5.700000in}{5.700000in}}%
\pgfusepath{clip}%
\pgfsetbuttcap%
\pgfsetroundjoin%
\definecolor{currentfill}{rgb}{0.282910,0.105393,0.426902}%
\pgfsetfillcolor{currentfill}%
\pgfsetfillopacity{0.700000}%
\pgfsetlinewidth{0.000000pt}%
\definecolor{currentstroke}{rgb}{0.000000,0.000000,0.000000}%
\pgfsetstrokecolor{currentstroke}%
\pgfsetdash{}{0pt}%
\pgfpathmoveto{\pgfqpoint{3.964491in}{2.339117in}}%
\pgfpathlineto{\pgfqpoint{3.978054in}{2.335937in}}%
\pgfpathlineto{\pgfqpoint{3.991623in}{2.332837in}}%
\pgfpathlineto{\pgfqpoint{4.005198in}{2.329818in}}%
\pgfpathlineto{\pgfqpoint{4.018780in}{2.326879in}}%
\pgfpathlineto{\pgfqpoint{4.026649in}{2.336037in}}%
\pgfpathlineto{\pgfqpoint{4.034512in}{2.345238in}}%
\pgfpathlineto{\pgfqpoint{4.042370in}{2.354484in}}%
\pgfpathlineto{\pgfqpoint{4.050223in}{2.363778in}}%
\pgfpathlineto{\pgfqpoint{4.036652in}{2.366877in}}%
\pgfpathlineto{\pgfqpoint{4.023087in}{2.370055in}}%
\pgfpathlineto{\pgfqpoint{4.009529in}{2.373314in}}%
\pgfpathlineto{\pgfqpoint{3.995977in}{2.376654in}}%
\pgfpathlineto{\pgfqpoint{3.988113in}{2.367193in}}%
\pgfpathlineto{\pgfqpoint{3.980245in}{2.357786in}}%
\pgfpathlineto{\pgfqpoint{3.972371in}{2.348428in}}%
\pgfpathlineto{\pgfqpoint{3.964491in}{2.339117in}}%
\pgfpathclose%
\pgfusepath{fill}%
\end{pgfscope}%
\begin{pgfscope}%
\pgfpathrectangle{\pgfqpoint{1.150000in}{0.150000in}}{\pgfqpoint{5.700000in}{5.700000in}}%
\pgfusepath{clip}%
\pgfsetbuttcap%
\pgfsetroundjoin%
\definecolor{currentfill}{rgb}{0.270595,0.214069,0.507052}%
\pgfsetfillcolor{currentfill}%
\pgfsetfillopacity{0.700000}%
\pgfsetlinewidth{0.000000pt}%
\definecolor{currentstroke}{rgb}{0.000000,0.000000,0.000000}%
\pgfsetstrokecolor{currentstroke}%
\pgfsetdash{}{0pt}%
\pgfpathmoveto{\pgfqpoint{4.898960in}{2.561112in}}%
\pgfpathlineto{\pgfqpoint{4.912766in}{2.559366in}}%
\pgfpathlineto{\pgfqpoint{4.926580in}{2.557691in}}%
\pgfpathlineto{\pgfqpoint{4.940403in}{2.556085in}}%
\pgfpathlineto{\pgfqpoint{4.954234in}{2.554549in}}%
\pgfpathlineto{\pgfqpoint{4.961785in}{2.563763in}}%
\pgfpathlineto{\pgfqpoint{4.969333in}{2.573124in}}%
\pgfpathlineto{\pgfqpoint{4.976879in}{2.582636in}}%
\pgfpathlineto{\pgfqpoint{4.984422in}{2.592308in}}%
\pgfpathlineto{\pgfqpoint{4.970608in}{2.594185in}}%
\pgfpathlineto{\pgfqpoint{4.956801in}{2.596133in}}%
\pgfpathlineto{\pgfqpoint{4.943003in}{2.598150in}}%
\pgfpathlineto{\pgfqpoint{4.929213in}{2.600238in}}%
\pgfpathlineto{\pgfqpoint{4.921653in}{2.590217in}}%
\pgfpathlineto{\pgfqpoint{4.914091in}{2.580361in}}%
\pgfpathlineto{\pgfqpoint{4.906527in}{2.570661in}}%
\pgfpathlineto{\pgfqpoint{4.898960in}{2.561112in}}%
\pgfpathclose%
\pgfusepath{fill}%
\end{pgfscope}%
\begin{pgfscope}%
\pgfpathrectangle{\pgfqpoint{1.150000in}{0.150000in}}{\pgfqpoint{5.700000in}{5.700000in}}%
\pgfusepath{clip}%
\pgfsetbuttcap%
\pgfsetroundjoin%
\definecolor{currentfill}{rgb}{0.280267,0.073417,0.397163}%
\pgfsetfillcolor{currentfill}%
\pgfsetfillopacity{0.700000}%
\pgfsetlinewidth{0.000000pt}%
\definecolor{currentstroke}{rgb}{0.000000,0.000000,0.000000}%
\pgfsetstrokecolor{currentstroke}%
\pgfsetdash{}{0pt}%
\pgfpathmoveto{\pgfqpoint{3.512871in}{2.285452in}}%
\pgfpathlineto{\pgfqpoint{3.526344in}{2.280655in}}%
\pgfpathlineto{\pgfqpoint{3.539822in}{2.275948in}}%
\pgfpathlineto{\pgfqpoint{3.553305in}{2.271331in}}%
\pgfpathlineto{\pgfqpoint{3.566792in}{2.266803in}}%
\pgfpathlineto{\pgfqpoint{3.574815in}{2.276065in}}%
\pgfpathlineto{\pgfqpoint{3.582831in}{2.285362in}}%
\pgfpathlineto{\pgfqpoint{3.590842in}{2.294694in}}%
\pgfpathlineto{\pgfqpoint{3.598847in}{2.304065in}}%
\pgfpathlineto{\pgfqpoint{3.585371in}{2.308671in}}%
\pgfpathlineto{\pgfqpoint{3.571899in}{2.313366in}}%
\pgfpathlineto{\pgfqpoint{3.558431in}{2.318150in}}%
\pgfpathlineto{\pgfqpoint{3.544969in}{2.323025in}}%
\pgfpathlineto{\pgfqpoint{3.536953in}{2.313569in}}%
\pgfpathlineto{\pgfqpoint{3.528932in}{2.304156in}}%
\pgfpathlineto{\pgfqpoint{3.520904in}{2.294784in}}%
\pgfpathlineto{\pgfqpoint{3.512871in}{2.285452in}}%
\pgfpathclose%
\pgfusepath{fill}%
\end{pgfscope}%
\begin{pgfscope}%
\pgfpathrectangle{\pgfqpoint{1.150000in}{0.150000in}}{\pgfqpoint{5.700000in}{5.700000in}}%
\pgfusepath{clip}%
\pgfsetbuttcap%
\pgfsetroundjoin%
\definecolor{currentfill}{rgb}{0.252194,0.269783,0.531579}%
\pgfsetfillcolor{currentfill}%
\pgfsetfillopacity{0.700000}%
\pgfsetlinewidth{0.000000pt}%
\definecolor{currentstroke}{rgb}{0.000000,0.000000,0.000000}%
\pgfsetstrokecolor{currentstroke}%
\pgfsetdash{}{0pt}%
\pgfpathmoveto{\pgfqpoint{5.296209in}{2.684122in}}%
\pgfpathlineto{\pgfqpoint{5.310117in}{2.682264in}}%
\pgfpathlineto{\pgfqpoint{5.324034in}{2.680472in}}%
\pgfpathlineto{\pgfqpoint{5.337959in}{2.678748in}}%
\pgfpathlineto{\pgfqpoint{5.351894in}{2.677092in}}%
\pgfpathlineto{\pgfqpoint{5.359336in}{2.687360in}}%
\pgfpathlineto{\pgfqpoint{5.366779in}{2.697858in}}%
\pgfpathlineto{\pgfqpoint{5.374224in}{2.708594in}}%
\pgfpathlineto{\pgfqpoint{5.381670in}{2.719576in}}%
\pgfpathlineto{\pgfqpoint{5.367755in}{2.721655in}}%
\pgfpathlineto{\pgfqpoint{5.353850in}{2.723802in}}%
\pgfpathlineto{\pgfqpoint{5.339953in}{2.726015in}}%
\pgfpathlineto{\pgfqpoint{5.326064in}{2.728296in}}%
\pgfpathlineto{\pgfqpoint{5.318599in}{2.716884in}}%
\pgfpathlineto{\pgfqpoint{5.311134in}{2.705723in}}%
\pgfpathlineto{\pgfqpoint{5.303671in}{2.694805in}}%
\pgfpathlineto{\pgfqpoint{5.296209in}{2.684122in}}%
\pgfpathclose%
\pgfusepath{fill}%
\end{pgfscope}%
\begin{pgfscope}%
\pgfpathrectangle{\pgfqpoint{1.150000in}{0.150000in}}{\pgfqpoint{5.700000in}{5.700000in}}%
\pgfusepath{clip}%
\pgfsetbuttcap%
\pgfsetroundjoin%
\definecolor{currentfill}{rgb}{0.280868,0.160771,0.472899}%
\pgfsetfillcolor{currentfill}%
\pgfsetfillopacity{0.700000}%
\pgfsetlinewidth{0.000000pt}%
\definecolor{currentstroke}{rgb}{0.000000,0.000000,0.000000}%
\pgfsetstrokecolor{currentstroke}%
\pgfsetdash{}{0pt}%
\pgfpathmoveto{\pgfqpoint{4.501843in}{2.450804in}}%
\pgfpathlineto{\pgfqpoint{4.515545in}{2.448785in}}%
\pgfpathlineto{\pgfqpoint{4.529255in}{2.446839in}}%
\pgfpathlineto{\pgfqpoint{4.542972in}{2.444967in}}%
\pgfpathlineto{\pgfqpoint{4.556697in}{2.443168in}}%
\pgfpathlineto{\pgfqpoint{4.564381in}{2.452079in}}%
\pgfpathlineto{\pgfqpoint{4.572061in}{2.461074in}}%
\pgfpathlineto{\pgfqpoint{4.579737in}{2.470159in}}%
\pgfpathlineto{\pgfqpoint{4.587408in}{2.479338in}}%
\pgfpathlineto{\pgfqpoint{4.573696in}{2.481397in}}%
\pgfpathlineto{\pgfqpoint{4.559992in}{2.483530in}}%
\pgfpathlineto{\pgfqpoint{4.546295in}{2.485737in}}%
\pgfpathlineto{\pgfqpoint{4.532607in}{2.488017in}}%
\pgfpathlineto{\pgfqpoint{4.524922in}{2.478570in}}%
\pgfpathlineto{\pgfqpoint{4.517234in}{2.469222in}}%
\pgfpathlineto{\pgfqpoint{4.509541in}{2.459968in}}%
\pgfpathlineto{\pgfqpoint{4.501843in}{2.450804in}}%
\pgfpathclose%
\pgfusepath{fill}%
\end{pgfscope}%
\begin{pgfscope}%
\pgfpathrectangle{\pgfqpoint{1.150000in}{0.150000in}}{\pgfqpoint{5.700000in}{5.700000in}}%
\pgfusepath{clip}%
\pgfsetbuttcap%
\pgfsetroundjoin%
\definecolor{currentfill}{rgb}{0.283187,0.125848,0.444960}%
\pgfsetfillcolor{currentfill}%
\pgfsetfillopacity{0.700000}%
\pgfsetlinewidth{0.000000pt}%
\definecolor{currentstroke}{rgb}{0.000000,0.000000,0.000000}%
\pgfsetstrokecolor{currentstroke}%
\pgfsetdash{}{0pt}%
\pgfpathmoveto{\pgfqpoint{4.190298in}{2.378135in}}%
\pgfpathlineto{\pgfqpoint{4.203918in}{2.375567in}}%
\pgfpathlineto{\pgfqpoint{4.217546in}{2.373076in}}%
\pgfpathlineto{\pgfqpoint{4.231180in}{2.370662in}}%
\pgfpathlineto{\pgfqpoint{4.244822in}{2.368325in}}%
\pgfpathlineto{\pgfqpoint{4.252614in}{2.377334in}}%
\pgfpathlineto{\pgfqpoint{4.260402in}{2.386397in}}%
\pgfpathlineto{\pgfqpoint{4.268184in}{2.395518in}}%
\pgfpathlineto{\pgfqpoint{4.275962in}{2.404701in}}%
\pgfpathlineto{\pgfqpoint{4.262332in}{2.407238in}}%
\pgfpathlineto{\pgfqpoint{4.248709in}{2.409851in}}%
\pgfpathlineto{\pgfqpoint{4.235093in}{2.412542in}}%
\pgfpathlineto{\pgfqpoint{4.221484in}{2.415311in}}%
\pgfpathlineto{\pgfqpoint{4.213695in}{2.405921in}}%
\pgfpathlineto{\pgfqpoint{4.205901in}{2.396598in}}%
\pgfpathlineto{\pgfqpoint{4.198102in}{2.387337in}}%
\pgfpathlineto{\pgfqpoint{4.190298in}{2.378135in}}%
\pgfpathclose%
\pgfusepath{fill}%
\end{pgfscope}%
\begin{pgfscope}%
\pgfpathrectangle{\pgfqpoint{1.150000in}{0.150000in}}{\pgfqpoint{5.700000in}{5.700000in}}%
\pgfusepath{clip}%
\pgfsetbuttcap%
\pgfsetroundjoin%
\definecolor{currentfill}{rgb}{0.280267,0.073417,0.397163}%
\pgfsetfillcolor{currentfill}%
\pgfsetfillopacity{0.700000}%
\pgfsetlinewidth{0.000000pt}%
\definecolor{currentstroke}{rgb}{0.000000,0.000000,0.000000}%
\pgfsetstrokecolor{currentstroke}%
\pgfsetdash{}{0pt}%
\pgfpathmoveto{\pgfqpoint{3.146700in}{2.289430in}}%
\pgfpathlineto{\pgfqpoint{3.160133in}{2.282831in}}%
\pgfpathlineto{\pgfqpoint{3.173569in}{2.276333in}}%
\pgfpathlineto{\pgfqpoint{3.187008in}{2.269935in}}%
\pgfpathlineto{\pgfqpoint{3.200449in}{2.263638in}}%
\pgfpathlineto{\pgfqpoint{3.208603in}{2.272643in}}%
\pgfpathlineto{\pgfqpoint{3.216749in}{2.281693in}}%
\pgfpathlineto{\pgfqpoint{3.224890in}{2.290788in}}%
\pgfpathlineto{\pgfqpoint{3.233024in}{2.299931in}}%
\pgfpathlineto{\pgfqpoint{3.219594in}{2.306245in}}%
\pgfpathlineto{\pgfqpoint{3.206167in}{2.312659in}}%
\pgfpathlineto{\pgfqpoint{3.192744in}{2.319174in}}%
\pgfpathlineto{\pgfqpoint{3.179323in}{2.325790in}}%
\pgfpathlineto{\pgfqpoint{3.171177in}{2.316624in}}%
\pgfpathlineto{\pgfqpoint{3.163025in}{2.307509in}}%
\pgfpathlineto{\pgfqpoint{3.154866in}{2.298444in}}%
\pgfpathlineto{\pgfqpoint{3.146700in}{2.289430in}}%
\pgfpathclose%
\pgfusepath{fill}%
\end{pgfscope}%
\begin{pgfscope}%
\pgfpathrectangle{\pgfqpoint{1.150000in}{0.150000in}}{\pgfqpoint{5.700000in}{5.700000in}}%
\pgfusepath{clip}%
\pgfsetbuttcap%
\pgfsetroundjoin%
\definecolor{currentfill}{rgb}{0.274128,0.199721,0.498911}%
\pgfsetfillcolor{currentfill}%
\pgfsetfillopacity{0.700000}%
\pgfsetlinewidth{0.000000pt}%
\definecolor{currentstroke}{rgb}{0.000000,0.000000,0.000000}%
\pgfsetstrokecolor{currentstroke}%
\pgfsetdash{}{0pt}%
\pgfpathmoveto{\pgfqpoint{4.813461in}{2.530712in}}%
\pgfpathlineto{\pgfqpoint{4.827249in}{2.529005in}}%
\pgfpathlineto{\pgfqpoint{4.841046in}{2.527370in}}%
\pgfpathlineto{\pgfqpoint{4.854850in}{2.525805in}}%
\pgfpathlineto{\pgfqpoint{4.868664in}{2.524311in}}%
\pgfpathlineto{\pgfqpoint{4.876242in}{2.533314in}}%
\pgfpathlineto{\pgfqpoint{4.883818in}{2.542445in}}%
\pgfpathlineto{\pgfqpoint{4.891390in}{2.551709in}}%
\pgfpathlineto{\pgfqpoint{4.898960in}{2.561112in}}%
\pgfpathlineto{\pgfqpoint{4.885163in}{2.562929in}}%
\pgfpathlineto{\pgfqpoint{4.871374in}{2.564815in}}%
\pgfpathlineto{\pgfqpoint{4.857593in}{2.566772in}}%
\pgfpathlineto{\pgfqpoint{4.843820in}{2.568800in}}%
\pgfpathlineto{\pgfqpoint{4.836235in}{2.559068in}}%
\pgfpathlineto{\pgfqpoint{4.828647in}{2.549480in}}%
\pgfpathlineto{\pgfqpoint{4.821056in}{2.540029in}}%
\pgfpathlineto{\pgfqpoint{4.813461in}{2.530712in}}%
\pgfpathclose%
\pgfusepath{fill}%
\end{pgfscope}%
\begin{pgfscope}%
\pgfpathrectangle{\pgfqpoint{1.150000in}{0.150000in}}{\pgfqpoint{5.700000in}{5.700000in}}%
\pgfusepath{clip}%
\pgfsetbuttcap%
\pgfsetroundjoin%
\definecolor{currentfill}{rgb}{0.257322,0.256130,0.526563}%
\pgfsetfillcolor{currentfill}%
\pgfsetfillopacity{0.700000}%
\pgfsetlinewidth{0.000000pt}%
\definecolor{currentstroke}{rgb}{0.000000,0.000000,0.000000}%
\pgfsetstrokecolor{currentstroke}%
\pgfsetdash{}{0pt}%
\pgfpathmoveto{\pgfqpoint{5.210746in}{2.650097in}}%
\pgfpathlineto{\pgfqpoint{5.224637in}{2.648370in}}%
\pgfpathlineto{\pgfqpoint{5.238538in}{2.646711in}}%
\pgfpathlineto{\pgfqpoint{5.252447in}{2.645119in}}%
\pgfpathlineto{\pgfqpoint{5.266366in}{2.643596in}}%
\pgfpathlineto{\pgfqpoint{5.273826in}{2.653411in}}%
\pgfpathlineto{\pgfqpoint{5.281287in}{2.663433in}}%
\pgfpathlineto{\pgfqpoint{5.288748in}{2.673667in}}%
\pgfpathlineto{\pgfqpoint{5.296209in}{2.684122in}}%
\pgfpathlineto{\pgfqpoint{5.282310in}{2.686049in}}%
\pgfpathlineto{\pgfqpoint{5.268420in}{2.688042in}}%
\pgfpathlineto{\pgfqpoint{5.254539in}{2.690104in}}%
\pgfpathlineto{\pgfqpoint{5.240666in}{2.692233in}}%
\pgfpathlineto{\pgfqpoint{5.233185in}{2.681368in}}%
\pgfpathlineto{\pgfqpoint{5.225705in}{2.670729in}}%
\pgfpathlineto{\pgfqpoint{5.218226in}{2.660308in}}%
\pgfpathlineto{\pgfqpoint{5.210746in}{2.650097in}}%
\pgfpathclose%
\pgfusepath{fill}%
\end{pgfscope}%
\begin{pgfscope}%
\pgfpathrectangle{\pgfqpoint{1.150000in}{0.150000in}}{\pgfqpoint{5.700000in}{5.700000in}}%
\pgfusepath{clip}%
\pgfsetbuttcap%
\pgfsetroundjoin%
\definecolor{currentfill}{rgb}{0.281446,0.084320,0.407414}%
\pgfsetfillcolor{currentfill}%
\pgfsetfillopacity{0.700000}%
\pgfsetlinewidth{0.000000pt}%
\definecolor{currentstroke}{rgb}{0.000000,0.000000,0.000000}%
\pgfsetstrokecolor{currentstroke}%
\pgfsetdash{}{0pt}%
\pgfpathmoveto{\pgfqpoint{3.006495in}{2.310416in}}%
\pgfpathlineto{\pgfqpoint{3.019921in}{2.302977in}}%
\pgfpathlineto{\pgfqpoint{3.033350in}{2.295646in}}%
\pgfpathlineto{\pgfqpoint{3.046781in}{2.288421in}}%
\pgfpathlineto{\pgfqpoint{3.060214in}{2.281301in}}%
\pgfpathlineto{\pgfqpoint{3.068419in}{2.290109in}}%
\pgfpathlineto{\pgfqpoint{3.076618in}{2.298971in}}%
\pgfpathlineto{\pgfqpoint{3.084810in}{2.307886in}}%
\pgfpathlineto{\pgfqpoint{3.092995in}{2.316854in}}%
\pgfpathlineto{\pgfqpoint{3.079575in}{2.323970in}}%
\pgfpathlineto{\pgfqpoint{3.066157in}{2.331192in}}%
\pgfpathlineto{\pgfqpoint{3.052742in}{2.338519in}}%
\pgfpathlineto{\pgfqpoint{3.039329in}{2.345954in}}%
\pgfpathlineto{\pgfqpoint{3.031131in}{2.336981in}}%
\pgfpathlineto{\pgfqpoint{3.022926in}{2.328068in}}%
\pgfpathlineto{\pgfqpoint{3.014714in}{2.319213in}}%
\pgfpathlineto{\pgfqpoint{3.006495in}{2.310416in}}%
\pgfpathclose%
\pgfusepath{fill}%
\end{pgfscope}%
\begin{pgfscope}%
\pgfpathrectangle{\pgfqpoint{1.150000in}{0.150000in}}{\pgfqpoint{5.700000in}{5.700000in}}%
\pgfusepath{clip}%
\pgfsetbuttcap%
\pgfsetroundjoin%
\definecolor{currentfill}{rgb}{0.279566,0.067836,0.391917}%
\pgfsetfillcolor{currentfill}%
\pgfsetfillopacity{0.700000}%
\pgfsetlinewidth{0.000000pt}%
\definecolor{currentstroke}{rgb}{0.000000,0.000000,0.000000}%
\pgfsetstrokecolor{currentstroke}%
\pgfsetdash{}{0pt}%
\pgfpathmoveto{\pgfqpoint{3.286775in}{2.275661in}}%
\pgfpathlineto{\pgfqpoint{3.300221in}{2.269838in}}%
\pgfpathlineto{\pgfqpoint{3.313671in}{2.264111in}}%
\pgfpathlineto{\pgfqpoint{3.327125in}{2.258480in}}%
\pgfpathlineto{\pgfqpoint{3.340583in}{2.252944in}}%
\pgfpathlineto{\pgfqpoint{3.348687in}{2.262075in}}%
\pgfpathlineto{\pgfqpoint{3.356785in}{2.271244in}}%
\pgfpathlineto{\pgfqpoint{3.364876in}{2.280452in}}%
\pgfpathlineto{\pgfqpoint{3.372962in}{2.289702in}}%
\pgfpathlineto{\pgfqpoint{3.359516in}{2.295275in}}%
\pgfpathlineto{\pgfqpoint{3.346073in}{2.300943in}}%
\pgfpathlineto{\pgfqpoint{3.332635in}{2.306708in}}%
\pgfpathlineto{\pgfqpoint{3.319200in}{2.312568in}}%
\pgfpathlineto{\pgfqpoint{3.311103in}{2.303274in}}%
\pgfpathlineto{\pgfqpoint{3.303000in}{2.294026in}}%
\pgfpathlineto{\pgfqpoint{3.294891in}{2.284822in}}%
\pgfpathlineto{\pgfqpoint{3.286775in}{2.275661in}}%
\pgfpathclose%
\pgfusepath{fill}%
\end{pgfscope}%
\begin{pgfscope}%
\pgfpathrectangle{\pgfqpoint{1.150000in}{0.150000in}}{\pgfqpoint{5.700000in}{5.700000in}}%
\pgfusepath{clip}%
\pgfsetbuttcap%
\pgfsetroundjoin%
\definecolor{currentfill}{rgb}{0.282884,0.135920,0.453427}%
\pgfsetfillcolor{currentfill}%
\pgfsetfillopacity{0.700000}%
\pgfsetlinewidth{0.000000pt}%
\definecolor{currentstroke}{rgb}{0.000000,0.000000,0.000000}%
\pgfsetstrokecolor{currentstroke}%
\pgfsetdash{}{0pt}%
\pgfpathmoveto{\pgfqpoint{2.671617in}{2.415975in}}%
\pgfpathlineto{\pgfqpoint{2.685057in}{2.406151in}}%
\pgfpathlineto{\pgfqpoint{2.698496in}{2.396453in}}%
\pgfpathlineto{\pgfqpoint{2.711935in}{2.386878in}}%
\pgfpathlineto{\pgfqpoint{2.725375in}{2.377426in}}%
\pgfpathlineto{\pgfqpoint{2.733714in}{2.385553in}}%
\pgfpathlineto{\pgfqpoint{2.742045in}{2.393758in}}%
\pgfpathlineto{\pgfqpoint{2.750368in}{2.402041in}}%
\pgfpathlineto{\pgfqpoint{2.758682in}{2.410402in}}%
\pgfpathlineto{\pgfqpoint{2.745259in}{2.419808in}}%
\pgfpathlineto{\pgfqpoint{2.731836in}{2.429338in}}%
\pgfpathlineto{\pgfqpoint{2.718413in}{2.438990in}}%
\pgfpathlineto{\pgfqpoint{2.704989in}{2.448768in}}%
\pgfpathlineto{\pgfqpoint{2.696659in}{2.440445in}}%
\pgfpathlineto{\pgfqpoint{2.688320in}{2.432205in}}%
\pgfpathlineto{\pgfqpoint{2.679973in}{2.424048in}}%
\pgfpathlineto{\pgfqpoint{2.671617in}{2.415975in}}%
\pgfpathclose%
\pgfusepath{fill}%
\end{pgfscope}%
\begin{pgfscope}%
\pgfpathrectangle{\pgfqpoint{1.150000in}{0.150000in}}{\pgfqpoint{5.700000in}{5.700000in}}%
\pgfusepath{clip}%
\pgfsetbuttcap%
\pgfsetroundjoin%
\definecolor{currentfill}{rgb}{0.280894,0.078907,0.402329}%
\pgfsetfillcolor{currentfill}%
\pgfsetfillopacity{0.700000}%
\pgfsetlinewidth{0.000000pt}%
\definecolor{currentstroke}{rgb}{0.000000,0.000000,0.000000}%
\pgfsetstrokecolor{currentstroke}%
\pgfsetdash{}{0pt}%
\pgfpathmoveto{\pgfqpoint{3.652804in}{2.286525in}}%
\pgfpathlineto{\pgfqpoint{3.666306in}{2.282359in}}%
\pgfpathlineto{\pgfqpoint{3.679813in}{2.278279in}}%
\pgfpathlineto{\pgfqpoint{3.693325in}{2.274286in}}%
\pgfpathlineto{\pgfqpoint{3.706843in}{2.270379in}}%
\pgfpathlineto{\pgfqpoint{3.714821in}{2.279608in}}%
\pgfpathlineto{\pgfqpoint{3.722794in}{2.288868in}}%
\pgfpathlineto{\pgfqpoint{3.730761in}{2.298162in}}%
\pgfpathlineto{\pgfqpoint{3.738722in}{2.307493in}}%
\pgfpathlineto{\pgfqpoint{3.725214in}{2.311498in}}%
\pgfpathlineto{\pgfqpoint{3.711712in}{2.315590in}}%
\pgfpathlineto{\pgfqpoint{3.698216in}{2.319768in}}%
\pgfpathlineto{\pgfqpoint{3.684724in}{2.324032in}}%
\pgfpathlineto{\pgfqpoint{3.676753in}{2.314596in}}%
\pgfpathlineto{\pgfqpoint{3.668776in}{2.305201in}}%
\pgfpathlineto{\pgfqpoint{3.660793in}{2.295845in}}%
\pgfpathlineto{\pgfqpoint{3.652804in}{2.286525in}}%
\pgfpathclose%
\pgfusepath{fill}%
\end{pgfscope}%
\begin{pgfscope}%
\pgfpathrectangle{\pgfqpoint{1.150000in}{0.150000in}}{\pgfqpoint{5.700000in}{5.700000in}}%
\pgfusepath{clip}%
\pgfsetbuttcap%
\pgfsetroundjoin%
\definecolor{currentfill}{rgb}{0.282327,0.094955,0.417331}%
\pgfsetfillcolor{currentfill}%
\pgfsetfillopacity{0.700000}%
\pgfsetlinewidth{0.000000pt}%
\definecolor{currentstroke}{rgb}{0.000000,0.000000,0.000000}%
\pgfsetstrokecolor{currentstroke}%
\pgfsetdash{}{0pt}%
\pgfpathmoveto{\pgfqpoint{3.878687in}{2.315270in}}%
\pgfpathlineto{\pgfqpoint{3.892236in}{2.311901in}}%
\pgfpathlineto{\pgfqpoint{3.905791in}{2.308615in}}%
\pgfpathlineto{\pgfqpoint{3.919351in}{2.305410in}}%
\pgfpathlineto{\pgfqpoint{3.932918in}{2.302288in}}%
\pgfpathlineto{\pgfqpoint{3.940820in}{2.311439in}}%
\pgfpathlineto{\pgfqpoint{3.948716in}{2.320626in}}%
\pgfpathlineto{\pgfqpoint{3.956606in}{2.329851in}}%
\pgfpathlineto{\pgfqpoint{3.964491in}{2.339117in}}%
\pgfpathlineto{\pgfqpoint{3.950935in}{2.342379in}}%
\pgfpathlineto{\pgfqpoint{3.937385in}{2.345723in}}%
\pgfpathlineto{\pgfqpoint{3.923841in}{2.349148in}}%
\pgfpathlineto{\pgfqpoint{3.910302in}{2.352656in}}%
\pgfpathlineto{\pgfqpoint{3.902407in}{2.343243in}}%
\pgfpathlineto{\pgfqpoint{3.894506in}{2.333876in}}%
\pgfpathlineto{\pgfqpoint{3.886600in}{2.324553in}}%
\pgfpathlineto{\pgfqpoint{3.878687in}{2.315270in}}%
\pgfpathclose%
\pgfusepath{fill}%
\end{pgfscope}%
\begin{pgfscope}%
\pgfpathrectangle{\pgfqpoint{1.150000in}{0.150000in}}{\pgfqpoint{5.700000in}{5.700000in}}%
\pgfusepath{clip}%
\pgfsetbuttcap%
\pgfsetroundjoin%
\definecolor{currentfill}{rgb}{0.282656,0.100196,0.422160}%
\pgfsetfillcolor{currentfill}%
\pgfsetfillopacity{0.700000}%
\pgfsetlinewidth{0.000000pt}%
\definecolor{currentstroke}{rgb}{0.000000,0.000000,0.000000}%
\pgfsetstrokecolor{currentstroke}%
\pgfsetdash{}{0pt}%
\pgfpathmoveto{\pgfqpoint{2.866082in}{2.339446in}}%
\pgfpathlineto{\pgfqpoint{2.879510in}{2.331099in}}%
\pgfpathlineto{\pgfqpoint{2.892940in}{2.322866in}}%
\pgfpathlineto{\pgfqpoint{2.906371in}{2.314746in}}%
\pgfpathlineto{\pgfqpoint{2.919803in}{2.306737in}}%
\pgfpathlineto{\pgfqpoint{2.928065in}{2.315273in}}%
\pgfpathlineto{\pgfqpoint{2.936319in}{2.323870in}}%
\pgfpathlineto{\pgfqpoint{2.944566in}{2.332530in}}%
\pgfpathlineto{\pgfqpoint{2.952806in}{2.341253in}}%
\pgfpathlineto{\pgfqpoint{2.939388in}{2.349237in}}%
\pgfpathlineto{\pgfqpoint{2.925972in}{2.357333in}}%
\pgfpathlineto{\pgfqpoint{2.912557in}{2.365542in}}%
\pgfpathlineto{\pgfqpoint{2.899143in}{2.373863in}}%
\pgfpathlineto{\pgfqpoint{2.890889in}{2.365158in}}%
\pgfpathlineto{\pgfqpoint{2.882627in}{2.356519in}}%
\pgfpathlineto{\pgfqpoint{2.874358in}{2.347949in}}%
\pgfpathlineto{\pgfqpoint{2.866082in}{2.339446in}}%
\pgfpathclose%
\pgfusepath{fill}%
\end{pgfscope}%
\begin{pgfscope}%
\pgfpathrectangle{\pgfqpoint{1.150000in}{0.150000in}}{\pgfqpoint{5.700000in}{5.700000in}}%
\pgfusepath{clip}%
\pgfsetbuttcap%
\pgfsetroundjoin%
\definecolor{currentfill}{rgb}{0.281887,0.150881,0.465405}%
\pgfsetfillcolor{currentfill}%
\pgfsetfillopacity{0.700000}%
\pgfsetlinewidth{0.000000pt}%
\definecolor{currentstroke}{rgb}{0.000000,0.000000,0.000000}%
\pgfsetstrokecolor{currentstroke}%
\pgfsetdash{}{0pt}%
\pgfpathmoveto{\pgfqpoint{4.416226in}{2.422807in}}%
\pgfpathlineto{\pgfqpoint{4.429910in}{2.420732in}}%
\pgfpathlineto{\pgfqpoint{4.443602in}{2.418731in}}%
\pgfpathlineto{\pgfqpoint{4.457301in}{2.416804in}}%
\pgfpathlineto{\pgfqpoint{4.471008in}{2.414952in}}%
\pgfpathlineto{\pgfqpoint{4.478724in}{2.423803in}}%
\pgfpathlineto{\pgfqpoint{4.486435in}{2.432726in}}%
\pgfpathlineto{\pgfqpoint{4.494141in}{2.441725in}}%
\pgfpathlineto{\pgfqpoint{4.501843in}{2.450804in}}%
\pgfpathlineto{\pgfqpoint{4.488149in}{2.452898in}}%
\pgfpathlineto{\pgfqpoint{4.474463in}{2.455065in}}%
\pgfpathlineto{\pgfqpoint{4.460784in}{2.457307in}}%
\pgfpathlineto{\pgfqpoint{4.447113in}{2.459623in}}%
\pgfpathlineto{\pgfqpoint{4.439398in}{2.450296in}}%
\pgfpathlineto{\pgfqpoint{4.431679in}{2.441054in}}%
\pgfpathlineto{\pgfqpoint{4.423955in}{2.431892in}}%
\pgfpathlineto{\pgfqpoint{4.416226in}{2.422807in}}%
\pgfpathclose%
\pgfusepath{fill}%
\end{pgfscope}%
\begin{pgfscope}%
\pgfpathrectangle{\pgfqpoint{1.150000in}{0.150000in}}{\pgfqpoint{5.700000in}{5.700000in}}%
\pgfusepath{clip}%
\pgfsetbuttcap%
\pgfsetroundjoin%
\definecolor{currentfill}{rgb}{0.260571,0.246922,0.522828}%
\pgfsetfillcolor{currentfill}%
\pgfsetfillopacity{0.700000}%
\pgfsetlinewidth{0.000000pt}%
\definecolor{currentstroke}{rgb}{0.000000,0.000000,0.000000}%
\pgfsetstrokecolor{currentstroke}%
\pgfsetdash{}{0pt}%
\pgfpathmoveto{\pgfqpoint{5.125268in}{2.617285in}}%
\pgfpathlineto{\pgfqpoint{5.139143in}{2.615667in}}%
\pgfpathlineto{\pgfqpoint{5.153027in}{2.614118in}}%
\pgfpathlineto{\pgfqpoint{5.166919in}{2.612637in}}%
\pgfpathlineto{\pgfqpoint{5.180821in}{2.611224in}}%
\pgfpathlineto{\pgfqpoint{5.188303in}{2.620661in}}%
\pgfpathlineto{\pgfqpoint{5.195785in}{2.630281in}}%
\pgfpathlineto{\pgfqpoint{5.203266in}{2.640091in}}%
\pgfpathlineto{\pgfqpoint{5.210746in}{2.650097in}}%
\pgfpathlineto{\pgfqpoint{5.196863in}{2.651893in}}%
\pgfpathlineto{\pgfqpoint{5.182989in}{2.653756in}}%
\pgfpathlineto{\pgfqpoint{5.169123in}{2.655688in}}%
\pgfpathlineto{\pgfqpoint{5.155266in}{2.657687in}}%
\pgfpathlineto{\pgfqpoint{5.147768in}{2.647291in}}%
\pgfpathlineto{\pgfqpoint{5.140269in}{2.637097in}}%
\pgfpathlineto{\pgfqpoint{5.132769in}{2.627097in}}%
\pgfpathlineto{\pgfqpoint{5.125268in}{2.617285in}}%
\pgfpathclose%
\pgfusepath{fill}%
\end{pgfscope}%
\begin{pgfscope}%
\pgfpathrectangle{\pgfqpoint{1.150000in}{0.150000in}}{\pgfqpoint{5.700000in}{5.700000in}}%
\pgfusepath{clip}%
\pgfsetbuttcap%
\pgfsetroundjoin%
\definecolor{currentfill}{rgb}{0.214298,0.355619,0.551184}%
\pgfsetfillcolor{currentfill}%
\pgfsetfillopacity{0.700000}%
\pgfsetlinewidth{0.000000pt}%
\definecolor{currentstroke}{rgb}{0.000000,0.000000,0.000000}%
\pgfsetstrokecolor{currentstroke}%
\pgfsetdash{}{0pt}%
\pgfpathmoveto{\pgfqpoint{5.779748in}{2.869862in}}%
\pgfpathlineto{\pgfqpoint{5.793762in}{2.867278in}}%
\pgfpathlineto{\pgfqpoint{5.807785in}{2.864759in}}%
\pgfpathlineto{\pgfqpoint{5.821817in}{2.862305in}}%
\pgfpathlineto{\pgfqpoint{5.835859in}{2.859916in}}%
\pgfpathlineto{\pgfqpoint{5.843252in}{2.873248in}}%
\pgfpathlineto{\pgfqpoint{5.850654in}{2.886949in}}%
\pgfpathlineto{\pgfqpoint{5.858066in}{2.901030in}}%
\pgfpathlineto{\pgfqpoint{5.865486in}{2.915500in}}%
\pgfpathlineto{\pgfqpoint{5.851470in}{2.918412in}}%
\pgfpathlineto{\pgfqpoint{5.837462in}{2.921389in}}%
\pgfpathlineto{\pgfqpoint{5.823463in}{2.924430in}}%
\pgfpathlineto{\pgfqpoint{5.809472in}{2.927536in}}%
\pgfpathlineto{\pgfqpoint{5.802027in}{2.912536in}}%
\pgfpathlineto{\pgfqpoint{5.794592in}{2.897930in}}%
\pgfpathlineto{\pgfqpoint{5.787166in}{2.883709in}}%
\pgfpathlineto{\pgfqpoint{5.779748in}{2.869862in}}%
\pgfpathclose%
\pgfusepath{fill}%
\end{pgfscope}%
\begin{pgfscope}%
\pgfpathrectangle{\pgfqpoint{1.150000in}{0.150000in}}{\pgfqpoint{5.700000in}{5.700000in}}%
\pgfusepath{clip}%
\pgfsetbuttcap%
\pgfsetroundjoin%
\definecolor{currentfill}{rgb}{0.204903,0.375746,0.553533}%
\pgfsetfillcolor{currentfill}%
\pgfsetfillopacity{0.700000}%
\pgfsetlinewidth{0.000000pt}%
\definecolor{currentstroke}{rgb}{0.000000,0.000000,0.000000}%
\pgfsetstrokecolor{currentstroke}%
\pgfsetdash{}{0pt}%
\pgfpathmoveto{\pgfqpoint{5.865486in}{2.915500in}}%
\pgfpathlineto{\pgfqpoint{5.879512in}{2.912653in}}%
\pgfpathlineto{\pgfqpoint{5.893547in}{2.909870in}}%
\pgfpathlineto{\pgfqpoint{5.907591in}{2.907152in}}%
\pgfpathlineto{\pgfqpoint{5.921644in}{2.904498in}}%
\pgfpathlineto{\pgfqpoint{5.929050in}{2.918831in}}%
\pgfpathlineto{\pgfqpoint{5.936467in}{2.933568in}}%
\pgfpathlineto{\pgfqpoint{5.943895in}{2.948718in}}%
\pgfpathlineto{\pgfqpoint{5.951335in}{2.964292in}}%
\pgfpathlineto{\pgfqpoint{5.937307in}{2.967488in}}%
\pgfpathlineto{\pgfqpoint{5.923288in}{2.970749in}}%
\pgfpathlineto{\pgfqpoint{5.909278in}{2.974074in}}%
\pgfpathlineto{\pgfqpoint{5.895276in}{2.977464in}}%
\pgfpathlineto{\pgfqpoint{5.887812in}{2.961341in}}%
\pgfpathlineto{\pgfqpoint{5.880359in}{2.945646in}}%
\pgfpathlineto{\pgfqpoint{5.872917in}{2.930369in}}%
\pgfpathlineto{\pgfqpoint{5.865486in}{2.915500in}}%
\pgfpathclose%
\pgfusepath{fill}%
\end{pgfscope}%
\begin{pgfscope}%
\pgfpathrectangle{\pgfqpoint{1.150000in}{0.150000in}}{\pgfqpoint{5.700000in}{5.700000in}}%
\pgfusepath{clip}%
\pgfsetbuttcap%
\pgfsetroundjoin%
\definecolor{currentfill}{rgb}{0.279566,0.067836,0.391917}%
\pgfsetfillcolor{currentfill}%
\pgfsetfillopacity{0.700000}%
\pgfsetlinewidth{0.000000pt}%
\definecolor{currentstroke}{rgb}{0.000000,0.000000,0.000000}%
\pgfsetstrokecolor{currentstroke}%
\pgfsetdash{}{0pt}%
\pgfpathmoveto{\pgfqpoint{3.426786in}{2.268351in}}%
\pgfpathlineto{\pgfqpoint{3.440253in}{2.263246in}}%
\pgfpathlineto{\pgfqpoint{3.453723in}{2.258233in}}%
\pgfpathlineto{\pgfqpoint{3.467198in}{2.253312in}}%
\pgfpathlineto{\pgfqpoint{3.480678in}{2.248482in}}%
\pgfpathlineto{\pgfqpoint{3.488735in}{2.257673in}}%
\pgfpathlineto{\pgfqpoint{3.496786in}{2.266898in}}%
\pgfpathlineto{\pgfqpoint{3.504832in}{2.276157in}}%
\pgfpathlineto{\pgfqpoint{3.512871in}{2.285452in}}%
\pgfpathlineto{\pgfqpoint{3.499402in}{2.290339in}}%
\pgfpathlineto{\pgfqpoint{3.485938in}{2.295318in}}%
\pgfpathlineto{\pgfqpoint{3.472478in}{2.300388in}}%
\pgfpathlineto{\pgfqpoint{3.459023in}{2.305551in}}%
\pgfpathlineto{\pgfqpoint{3.450973in}{2.296191in}}%
\pgfpathlineto{\pgfqpoint{3.442916in}{2.286872in}}%
\pgfpathlineto{\pgfqpoint{3.434854in}{2.277592in}}%
\pgfpathlineto{\pgfqpoint{3.426786in}{2.268351in}}%
\pgfpathclose%
\pgfusepath{fill}%
\end{pgfscope}%
\begin{pgfscope}%
\pgfpathrectangle{\pgfqpoint{1.150000in}{0.150000in}}{\pgfqpoint{5.700000in}{5.700000in}}%
\pgfusepath{clip}%
\pgfsetbuttcap%
\pgfsetroundjoin%
\definecolor{currentfill}{rgb}{0.221989,0.339161,0.548752}%
\pgfsetfillcolor{currentfill}%
\pgfsetfillopacity{0.700000}%
\pgfsetlinewidth{0.000000pt}%
\definecolor{currentstroke}{rgb}{0.000000,0.000000,0.000000}%
\pgfsetstrokecolor{currentstroke}%
\pgfsetdash{}{0pt}%
\pgfpathmoveto{\pgfqpoint{5.694094in}{2.827013in}}%
\pgfpathlineto{\pgfqpoint{5.708095in}{2.824671in}}%
\pgfpathlineto{\pgfqpoint{5.722106in}{2.822394in}}%
\pgfpathlineto{\pgfqpoint{5.736125in}{2.820183in}}%
\pgfpathlineto{\pgfqpoint{5.750154in}{2.818037in}}%
\pgfpathlineto{\pgfqpoint{5.757542in}{2.830477in}}%
\pgfpathlineto{\pgfqpoint{5.764937in}{2.843256in}}%
\pgfpathlineto{\pgfqpoint{5.772338in}{2.856381in}}%
\pgfpathlineto{\pgfqpoint{5.779748in}{2.869862in}}%
\pgfpathlineto{\pgfqpoint{5.765743in}{2.872511in}}%
\pgfpathlineto{\pgfqpoint{5.751747in}{2.875225in}}%
\pgfpathlineto{\pgfqpoint{5.737760in}{2.878003in}}%
\pgfpathlineto{\pgfqpoint{5.723781in}{2.880847in}}%
\pgfpathlineto{\pgfqpoint{5.716348in}{2.866856in}}%
\pgfpathlineto{\pgfqpoint{5.708923in}{2.853226in}}%
\pgfpathlineto{\pgfqpoint{5.701505in}{2.839948in}}%
\pgfpathlineto{\pgfqpoint{5.694094in}{2.827013in}}%
\pgfpathclose%
\pgfusepath{fill}%
\end{pgfscope}%
\begin{pgfscope}%
\pgfpathrectangle{\pgfqpoint{1.150000in}{0.150000in}}{\pgfqpoint{5.700000in}{5.700000in}}%
\pgfusepath{clip}%
\pgfsetbuttcap%
\pgfsetroundjoin%
\definecolor{currentfill}{rgb}{0.283197,0.115680,0.436115}%
\pgfsetfillcolor{currentfill}%
\pgfsetfillopacity{0.700000}%
\pgfsetlinewidth{0.000000pt}%
\definecolor{currentstroke}{rgb}{0.000000,0.000000,0.000000}%
\pgfsetstrokecolor{currentstroke}%
\pgfsetdash{}{0pt}%
\pgfpathmoveto{\pgfqpoint{4.104572in}{2.352182in}}%
\pgfpathlineto{\pgfqpoint{4.118176in}{2.349481in}}%
\pgfpathlineto{\pgfqpoint{4.131787in}{2.346858in}}%
\pgfpathlineto{\pgfqpoint{4.145405in}{2.344313in}}%
\pgfpathlineto{\pgfqpoint{4.159029in}{2.341847in}}%
\pgfpathlineto{\pgfqpoint{4.166854in}{2.350848in}}%
\pgfpathlineto{\pgfqpoint{4.174674in}{2.359895in}}%
\pgfpathlineto{\pgfqpoint{4.182489in}{2.368989in}}%
\pgfpathlineto{\pgfqpoint{4.190298in}{2.378135in}}%
\pgfpathlineto{\pgfqpoint{4.176685in}{2.380782in}}%
\pgfpathlineto{\pgfqpoint{4.163078in}{2.383506in}}%
\pgfpathlineto{\pgfqpoint{4.149479in}{2.386309in}}%
\pgfpathlineto{\pgfqpoint{4.135886in}{2.389190in}}%
\pgfpathlineto{\pgfqpoint{4.128065in}{2.379856in}}%
\pgfpathlineto{\pgfqpoint{4.120239in}{2.370579in}}%
\pgfpathlineto{\pgfqpoint{4.112408in}{2.361356in}}%
\pgfpathlineto{\pgfqpoint{4.104572in}{2.352182in}}%
\pgfpathclose%
\pgfusepath{fill}%
\end{pgfscope}%
\begin{pgfscope}%
\pgfpathrectangle{\pgfqpoint{1.150000in}{0.150000in}}{\pgfqpoint{5.700000in}{5.700000in}}%
\pgfusepath{clip}%
\pgfsetbuttcap%
\pgfsetroundjoin%
\definecolor{currentfill}{rgb}{0.276194,0.190074,0.493001}%
\pgfsetfillcolor{currentfill}%
\pgfsetfillopacity{0.700000}%
\pgfsetlinewidth{0.000000pt}%
\definecolor{currentstroke}{rgb}{0.000000,0.000000,0.000000}%
\pgfsetstrokecolor{currentstroke}%
\pgfsetdash{}{0pt}%
\pgfpathmoveto{\pgfqpoint{4.727921in}{2.500983in}}%
\pgfpathlineto{\pgfqpoint{4.741691in}{2.499294in}}%
\pgfpathlineto{\pgfqpoint{4.755469in}{2.497676in}}%
\pgfpathlineto{\pgfqpoint{4.769255in}{2.496129in}}%
\pgfpathlineto{\pgfqpoint{4.783050in}{2.494654in}}%
\pgfpathlineto{\pgfqpoint{4.790658in}{2.503497in}}%
\pgfpathlineto{\pgfqpoint{4.798263in}{2.512451in}}%
\pgfpathlineto{\pgfqpoint{4.805864in}{2.521521in}}%
\pgfpathlineto{\pgfqpoint{4.813461in}{2.530712in}}%
\pgfpathlineto{\pgfqpoint{4.799682in}{2.532489in}}%
\pgfpathlineto{\pgfqpoint{4.785910in}{2.534338in}}%
\pgfpathlineto{\pgfqpoint{4.772147in}{2.536257in}}%
\pgfpathlineto{\pgfqpoint{4.758392in}{2.538248in}}%
\pgfpathlineto{\pgfqpoint{4.750779in}{2.528748in}}%
\pgfpathlineto{\pgfqpoint{4.743163in}{2.519375in}}%
\pgfpathlineto{\pgfqpoint{4.735544in}{2.510121in}}%
\pgfpathlineto{\pgfqpoint{4.727921in}{2.500983in}}%
\pgfpathclose%
\pgfusepath{fill}%
\end{pgfscope}%
\begin{pgfscope}%
\pgfpathrectangle{\pgfqpoint{1.150000in}{0.150000in}}{\pgfqpoint{5.700000in}{5.700000in}}%
\pgfusepath{clip}%
\pgfsetbuttcap%
\pgfsetroundjoin%
\definecolor{currentfill}{rgb}{0.277134,0.185228,0.489898}%
\pgfsetfillcolor{currentfill}%
\pgfsetfillopacity{0.700000}%
\pgfsetlinewidth{0.000000pt}%
\definecolor{currentstroke}{rgb}{0.000000,0.000000,0.000000}%
\pgfsetstrokecolor{currentstroke}%
\pgfsetdash{}{0pt}%
\pgfpathmoveto{\pgfqpoint{2.476543in}{2.513331in}}%
\pgfpathlineto{\pgfqpoint{2.490016in}{2.501856in}}%
\pgfpathlineto{\pgfqpoint{2.503486in}{2.490521in}}%
\pgfpathlineto{\pgfqpoint{2.516954in}{2.479322in}}%
\pgfpathlineto{\pgfqpoint{2.530421in}{2.468260in}}%
\pgfpathlineto{\pgfqpoint{2.538848in}{2.475847in}}%
\pgfpathlineto{\pgfqpoint{2.547265in}{2.483531in}}%
\pgfpathlineto{\pgfqpoint{2.555673in}{2.491308in}}%
\pgfpathlineto{\pgfqpoint{2.564072in}{2.499181in}}%
\pgfpathlineto{\pgfqpoint{2.550623in}{2.510176in}}%
\pgfpathlineto{\pgfqpoint{2.537173in}{2.521307in}}%
\pgfpathlineto{\pgfqpoint{2.523721in}{2.532575in}}%
\pgfpathlineto{\pgfqpoint{2.510267in}{2.543982in}}%
\pgfpathlineto{\pgfqpoint{2.501850in}{2.536170in}}%
\pgfpathlineto{\pgfqpoint{2.493424in}{2.528457in}}%
\pgfpathlineto{\pgfqpoint{2.484988in}{2.520844in}}%
\pgfpathlineto{\pgfqpoint{2.476543in}{2.513331in}}%
\pgfpathclose%
\pgfusepath{fill}%
\end{pgfscope}%
\begin{pgfscope}%
\pgfpathrectangle{\pgfqpoint{1.150000in}{0.150000in}}{\pgfqpoint{5.700000in}{5.700000in}}%
\pgfusepath{clip}%
\pgfsetbuttcap%
\pgfsetroundjoin%
\definecolor{currentfill}{rgb}{0.195860,0.395433,0.555276}%
\pgfsetfillcolor{currentfill}%
\pgfsetfillopacity{0.700000}%
\pgfsetlinewidth{0.000000pt}%
\definecolor{currentstroke}{rgb}{0.000000,0.000000,0.000000}%
\pgfsetstrokecolor{currentstroke}%
\pgfsetdash{}{0pt}%
\pgfpathmoveto{\pgfqpoint{5.951335in}{2.964292in}}%
\pgfpathlineto{\pgfqpoint{5.965372in}{2.961159in}}%
\pgfpathlineto{\pgfqpoint{5.979417in}{2.958091in}}%
\pgfpathlineto{\pgfqpoint{5.993472in}{2.955087in}}%
\pgfpathlineto{\pgfqpoint{6.007536in}{2.952147in}}%
\pgfpathlineto{\pgfqpoint{6.014962in}{2.967598in}}%
\pgfpathlineto{\pgfqpoint{6.022402in}{2.983489in}}%
\pgfpathlineto{\pgfqpoint{6.029855in}{2.999829in}}%
\pgfpathlineto{\pgfqpoint{6.037322in}{3.016628in}}%
\pgfpathlineto{\pgfqpoint{6.023284in}{3.020131in}}%
\pgfpathlineto{\pgfqpoint{6.009255in}{3.023698in}}%
\pgfpathlineto{\pgfqpoint{5.995235in}{3.027329in}}%
\pgfpathlineto{\pgfqpoint{5.981223in}{3.031024in}}%
\pgfpathlineto{\pgfqpoint{5.973730in}{3.013654in}}%
\pgfpathlineto{\pgfqpoint{5.966252in}{2.996750in}}%
\pgfpathlineto{\pgfqpoint{5.958787in}{2.980299in}}%
\pgfpathlineto{\pgfqpoint{5.951335in}{2.964292in}}%
\pgfpathclose%
\pgfusepath{fill}%
\end{pgfscope}%
\begin{pgfscope}%
\pgfpathrectangle{\pgfqpoint{1.150000in}{0.150000in}}{\pgfqpoint{5.700000in}{5.700000in}}%
\pgfusepath{clip}%
\pgfsetbuttcap%
\pgfsetroundjoin%
\definecolor{currentfill}{rgb}{0.229739,0.322361,0.545706}%
\pgfsetfillcolor{currentfill}%
\pgfsetfillopacity{0.700000}%
\pgfsetlinewidth{0.000000pt}%
\definecolor{currentstroke}{rgb}{0.000000,0.000000,0.000000}%
\pgfsetstrokecolor{currentstroke}%
\pgfsetdash{}{0pt}%
\pgfpathmoveto{\pgfqpoint{5.608500in}{2.786612in}}%
\pgfpathlineto{\pgfqpoint{5.622488in}{2.784491in}}%
\pgfpathlineto{\pgfqpoint{5.636485in}{2.782436in}}%
\pgfpathlineto{\pgfqpoint{5.650492in}{2.780446in}}%
\pgfpathlineto{\pgfqpoint{5.664508in}{2.778522in}}%
\pgfpathlineto{\pgfqpoint{5.671896in}{2.790174in}}%
\pgfpathlineto{\pgfqpoint{5.679289in}{2.802134in}}%
\pgfpathlineto{\pgfqpoint{5.686689in}{2.814411in}}%
\pgfpathlineto{\pgfqpoint{5.694094in}{2.827013in}}%
\pgfpathlineto{\pgfqpoint{5.680101in}{2.829420in}}%
\pgfpathlineto{\pgfqpoint{5.666118in}{2.831892in}}%
\pgfpathlineto{\pgfqpoint{5.652143in}{2.834430in}}%
\pgfpathlineto{\pgfqpoint{5.638177in}{2.837033in}}%
\pgfpathlineto{\pgfqpoint{5.630750in}{2.823942in}}%
\pgfpathlineto{\pgfqpoint{5.623328in}{2.811181in}}%
\pgfpathlineto{\pgfqpoint{5.615911in}{2.798740in}}%
\pgfpathlineto{\pgfqpoint{5.608500in}{2.786612in}}%
\pgfpathclose%
\pgfusepath{fill}%
\end{pgfscope}%
\begin{pgfscope}%
\pgfpathrectangle{\pgfqpoint{1.150000in}{0.150000in}}{\pgfqpoint{5.700000in}{5.700000in}}%
\pgfusepath{clip}%
\pgfsetbuttcap%
\pgfsetroundjoin%
\definecolor{currentfill}{rgb}{0.237441,0.305202,0.541921}%
\pgfsetfillcolor{currentfill}%
\pgfsetfillopacity{0.700000}%
\pgfsetlinewidth{0.000000pt}%
\definecolor{currentstroke}{rgb}{0.000000,0.000000,0.000000}%
\pgfsetstrokecolor{currentstroke}%
\pgfsetdash{}{0pt}%
\pgfpathmoveto{\pgfqpoint{5.522947in}{2.748347in}}%
\pgfpathlineto{\pgfqpoint{5.536921in}{2.746425in}}%
\pgfpathlineto{\pgfqpoint{5.550904in}{2.744569in}}%
\pgfpathlineto{\pgfqpoint{5.564896in}{2.742779in}}%
\pgfpathlineto{\pgfqpoint{5.578898in}{2.741055in}}%
\pgfpathlineto{\pgfqpoint{5.586293in}{2.752018in}}%
\pgfpathlineto{\pgfqpoint{5.593691in}{2.763259in}}%
\pgfpathlineto{\pgfqpoint{5.601093in}{2.774788in}}%
\pgfpathlineto{\pgfqpoint{5.608500in}{2.786612in}}%
\pgfpathlineto{\pgfqpoint{5.594521in}{2.788799in}}%
\pgfpathlineto{\pgfqpoint{5.580551in}{2.791052in}}%
\pgfpathlineto{\pgfqpoint{5.566590in}{2.793371in}}%
\pgfpathlineto{\pgfqpoint{5.552637in}{2.795755in}}%
\pgfpathlineto{\pgfqpoint{5.545209in}{2.783461in}}%
\pgfpathlineto{\pgfqpoint{5.537784in}{2.771467in}}%
\pgfpathlineto{\pgfqpoint{5.530364in}{2.759766in}}%
\pgfpathlineto{\pgfqpoint{5.522947in}{2.748347in}}%
\pgfpathclose%
\pgfusepath{fill}%
\end{pgfscope}%
\begin{pgfscope}%
\pgfpathrectangle{\pgfqpoint{1.150000in}{0.150000in}}{\pgfqpoint{5.700000in}{5.700000in}}%
\pgfusepath{clip}%
\pgfsetbuttcap%
\pgfsetroundjoin%
\definecolor{currentfill}{rgb}{0.265145,0.232956,0.516599}%
\pgfsetfillcolor{currentfill}%
\pgfsetfillopacity{0.700000}%
\pgfsetlinewidth{0.000000pt}%
\definecolor{currentstroke}{rgb}{0.000000,0.000000,0.000000}%
\pgfsetstrokecolor{currentstroke}%
\pgfsetdash{}{0pt}%
\pgfpathmoveto{\pgfqpoint{5.039767in}{2.585492in}}%
\pgfpathlineto{\pgfqpoint{5.053624in}{2.583961in}}%
\pgfpathlineto{\pgfqpoint{5.067491in}{2.582500in}}%
\pgfpathlineto{\pgfqpoint{5.081366in}{2.581107in}}%
\pgfpathlineto{\pgfqpoint{5.095250in}{2.579784in}}%
\pgfpathlineto{\pgfqpoint{5.102757in}{2.588910in}}%
\pgfpathlineto{\pgfqpoint{5.110262in}{2.598198in}}%
\pgfpathlineto{\pgfqpoint{5.117766in}{2.607654in}}%
\pgfpathlineto{\pgfqpoint{5.125268in}{2.617285in}}%
\pgfpathlineto{\pgfqpoint{5.111402in}{2.618971in}}%
\pgfpathlineto{\pgfqpoint{5.097544in}{2.620726in}}%
\pgfpathlineto{\pgfqpoint{5.083696in}{2.622550in}}%
\pgfpathlineto{\pgfqpoint{5.069855in}{2.624443in}}%
\pgfpathlineto{\pgfqpoint{5.062335in}{2.614443in}}%
\pgfpathlineto{\pgfqpoint{5.054814in}{2.604622in}}%
\pgfpathlineto{\pgfqpoint{5.047291in}{2.594974in}}%
\pgfpathlineto{\pgfqpoint{5.039767in}{2.585492in}}%
\pgfpathclose%
\pgfusepath{fill}%
\end{pgfscope}%
\begin{pgfscope}%
\pgfpathrectangle{\pgfqpoint{1.150000in}{0.150000in}}{\pgfqpoint{5.700000in}{5.700000in}}%
\pgfusepath{clip}%
\pgfsetbuttcap%
\pgfsetroundjoin%
\definecolor{currentfill}{rgb}{0.283229,0.120777,0.440584}%
\pgfsetfillcolor{currentfill}%
\pgfsetfillopacity{0.700000}%
\pgfsetlinewidth{0.000000pt}%
\definecolor{currentstroke}{rgb}{0.000000,0.000000,0.000000}%
\pgfsetstrokecolor{currentstroke}%
\pgfsetdash{}{0pt}%
\pgfpathmoveto{\pgfqpoint{2.725375in}{2.377426in}}%
\pgfpathlineto{\pgfqpoint{2.738814in}{2.368095in}}%
\pgfpathlineto{\pgfqpoint{2.752253in}{2.358885in}}%
\pgfpathlineto{\pgfqpoint{2.765693in}{2.349795in}}%
\pgfpathlineto{\pgfqpoint{2.779133in}{2.340824in}}%
\pgfpathlineto{\pgfqpoint{2.787456in}{2.349004in}}%
\pgfpathlineto{\pgfqpoint{2.795771in}{2.357257in}}%
\pgfpathlineto{\pgfqpoint{2.804078in}{2.365583in}}%
\pgfpathlineto{\pgfqpoint{2.812377in}{2.373981in}}%
\pgfpathlineto{\pgfqpoint{2.798953in}{2.382907in}}%
\pgfpathlineto{\pgfqpoint{2.785529in}{2.391952in}}%
\pgfpathlineto{\pgfqpoint{2.772106in}{2.401117in}}%
\pgfpathlineto{\pgfqpoint{2.758682in}{2.410402in}}%
\pgfpathlineto{\pgfqpoint{2.750368in}{2.402041in}}%
\pgfpathlineto{\pgfqpoint{2.742045in}{2.393758in}}%
\pgfpathlineto{\pgfqpoint{2.733714in}{2.385553in}}%
\pgfpathlineto{\pgfqpoint{2.725375in}{2.377426in}}%
\pgfpathclose%
\pgfusepath{fill}%
\end{pgfscope}%
\begin{pgfscope}%
\pgfpathrectangle{\pgfqpoint{1.150000in}{0.150000in}}{\pgfqpoint{5.700000in}{5.700000in}}%
\pgfusepath{clip}%
\pgfsetbuttcap%
\pgfsetroundjoin%
\definecolor{currentfill}{rgb}{0.281446,0.084320,0.407414}%
\pgfsetfillcolor{currentfill}%
\pgfsetfillopacity{0.700000}%
\pgfsetlinewidth{0.000000pt}%
\definecolor{currentstroke}{rgb}{0.000000,0.000000,0.000000}%
\pgfsetstrokecolor{currentstroke}%
\pgfsetdash{}{0pt}%
\pgfpathmoveto{\pgfqpoint{3.792807in}{2.292321in}}%
\pgfpathlineto{\pgfqpoint{3.806342in}{2.288739in}}%
\pgfpathlineto{\pgfqpoint{3.819883in}{2.285240in}}%
\pgfpathlineto{\pgfqpoint{3.833430in}{2.281825in}}%
\pgfpathlineto{\pgfqpoint{3.846984in}{2.278493in}}%
\pgfpathlineto{\pgfqpoint{3.854918in}{2.287639in}}%
\pgfpathlineto{\pgfqpoint{3.862847in}{2.296815in}}%
\pgfpathlineto{\pgfqpoint{3.870770in}{2.306025in}}%
\pgfpathlineto{\pgfqpoint{3.878687in}{2.315270in}}%
\pgfpathlineto{\pgfqpoint{3.865145in}{2.318721in}}%
\pgfpathlineto{\pgfqpoint{3.851608in}{2.322255in}}%
\pgfpathlineto{\pgfqpoint{3.838078in}{2.325872in}}%
\pgfpathlineto{\pgfqpoint{3.824553in}{2.329573in}}%
\pgfpathlineto{\pgfqpoint{3.816625in}{2.320202in}}%
\pgfpathlineto{\pgfqpoint{3.808691in}{2.310871in}}%
\pgfpathlineto{\pgfqpoint{3.800752in}{2.301578in}}%
\pgfpathlineto{\pgfqpoint{3.792807in}{2.292321in}}%
\pgfpathclose%
\pgfusepath{fill}%
\end{pgfscope}%
\begin{pgfscope}%
\pgfpathrectangle{\pgfqpoint{1.150000in}{0.150000in}}{\pgfqpoint{5.700000in}{5.700000in}}%
\pgfusepath{clip}%
\pgfsetbuttcap%
\pgfsetroundjoin%
\definecolor{currentfill}{rgb}{0.282623,0.140926,0.457517}%
\pgfsetfillcolor{currentfill}%
\pgfsetfillopacity{0.700000}%
\pgfsetlinewidth{0.000000pt}%
\definecolor{currentstroke}{rgb}{0.000000,0.000000,0.000000}%
\pgfsetstrokecolor{currentstroke}%
\pgfsetdash{}{0pt}%
\pgfpathmoveto{\pgfqpoint{4.330553in}{2.395317in}}%
\pgfpathlineto{\pgfqpoint{4.344219in}{2.393161in}}%
\pgfpathlineto{\pgfqpoint{4.357893in}{2.391081in}}%
\pgfpathlineto{\pgfqpoint{4.371574in}{2.389076in}}%
\pgfpathlineto{\pgfqpoint{4.385263in}{2.387146in}}%
\pgfpathlineto{\pgfqpoint{4.393011in}{2.395968in}}%
\pgfpathlineto{\pgfqpoint{4.400754in}{2.404849in}}%
\pgfpathlineto{\pgfqpoint{4.408493in}{2.413794in}}%
\pgfpathlineto{\pgfqpoint{4.416226in}{2.422807in}}%
\pgfpathlineto{\pgfqpoint{4.402550in}{2.424958in}}%
\pgfpathlineto{\pgfqpoint{4.388881in}{2.427184in}}%
\pgfpathlineto{\pgfqpoint{4.375220in}{2.429485in}}%
\pgfpathlineto{\pgfqpoint{4.361565in}{2.431861in}}%
\pgfpathlineto{\pgfqpoint{4.353820in}{2.422620in}}%
\pgfpathlineto{\pgfqpoint{4.346069in}{2.413452in}}%
\pgfpathlineto{\pgfqpoint{4.338314in}{2.404352in}}%
\pgfpathlineto{\pgfqpoint{4.330553in}{2.395317in}}%
\pgfpathclose%
\pgfusepath{fill}%
\end{pgfscope}%
\begin{pgfscope}%
\pgfpathrectangle{\pgfqpoint{1.150000in}{0.150000in}}{\pgfqpoint{5.700000in}{5.700000in}}%
\pgfusepath{clip}%
\pgfsetbuttcap%
\pgfsetroundjoin%
\definecolor{currentfill}{rgb}{0.280267,0.073417,0.397163}%
\pgfsetfillcolor{currentfill}%
\pgfsetfillopacity{0.700000}%
\pgfsetlinewidth{0.000000pt}%
\definecolor{currentstroke}{rgb}{0.000000,0.000000,0.000000}%
\pgfsetstrokecolor{currentstroke}%
\pgfsetdash{}{0pt}%
\pgfpathmoveto{\pgfqpoint{3.566792in}{2.266803in}}%
\pgfpathlineto{\pgfqpoint{3.580284in}{2.262364in}}%
\pgfpathlineto{\pgfqpoint{3.593782in}{2.258013in}}%
\pgfpathlineto{\pgfqpoint{3.607284in}{2.253751in}}%
\pgfpathlineto{\pgfqpoint{3.620791in}{2.249576in}}%
\pgfpathlineto{\pgfqpoint{3.628803in}{2.258767in}}%
\pgfpathlineto{\pgfqpoint{3.636809in}{2.267988in}}%
\pgfpathlineto{\pgfqpoint{3.644809in}{2.277240in}}%
\pgfpathlineto{\pgfqpoint{3.652804in}{2.286525in}}%
\pgfpathlineto{\pgfqpoint{3.639307in}{2.290778in}}%
\pgfpathlineto{\pgfqpoint{3.625816in}{2.295119in}}%
\pgfpathlineto{\pgfqpoint{3.612329in}{2.299548in}}%
\pgfpathlineto{\pgfqpoint{3.598847in}{2.304065in}}%
\pgfpathlineto{\pgfqpoint{3.590842in}{2.294694in}}%
\pgfpathlineto{\pgfqpoint{3.582831in}{2.285362in}}%
\pgfpathlineto{\pgfqpoint{3.574815in}{2.276065in}}%
\pgfpathlineto{\pgfqpoint{3.566792in}{2.266803in}}%
\pgfpathclose%
\pgfusepath{fill}%
\end{pgfscope}%
\begin{pgfscope}%
\pgfpathrectangle{\pgfqpoint{1.150000in}{0.150000in}}{\pgfqpoint{5.700000in}{5.700000in}}%
\pgfusepath{clip}%
\pgfsetbuttcap%
\pgfsetroundjoin%
\definecolor{currentfill}{rgb}{0.187231,0.414746,0.556547}%
\pgfsetfillcolor{currentfill}%
\pgfsetfillopacity{0.700000}%
\pgfsetlinewidth{0.000000pt}%
\definecolor{currentstroke}{rgb}{0.000000,0.000000,0.000000}%
\pgfsetstrokecolor{currentstroke}%
\pgfsetdash{}{0pt}%
\pgfpathmoveto{\pgfqpoint{6.037322in}{3.016628in}}%
\pgfpathlineto{\pgfqpoint{6.051369in}{3.013189in}}%
\pgfpathlineto{\pgfqpoint{6.065425in}{3.009814in}}%
\pgfpathlineto{\pgfqpoint{6.079490in}{3.006502in}}%
\pgfpathlineto{\pgfqpoint{6.093564in}{3.003255in}}%
\pgfpathlineto{\pgfqpoint{6.101020in}{3.019948in}}%
\pgfpathlineto{\pgfqpoint{6.108491in}{3.037118in}}%
\pgfpathlineto{\pgfqpoint{6.115978in}{3.054774in}}%
\pgfpathlineto{\pgfqpoint{6.101924in}{3.058458in}}%
\pgfpathlineto{\pgfqpoint{6.087879in}{3.062204in}}%
\pgfpathlineto{\pgfqpoint{6.073842in}{3.066015in}}%
\pgfpathlineto{\pgfqpoint{6.059815in}{3.069889in}}%
\pgfpathlineto{\pgfqpoint{6.052302in}{3.051648in}}%
\pgfpathlineto{\pgfqpoint{6.044804in}{3.033897in}}%
\pgfpathlineto{\pgfqpoint{6.037322in}{3.016628in}}%
\pgfpathclose%
\pgfusepath{fill}%
\end{pgfscope}%
\begin{pgfscope}%
\pgfpathrectangle{\pgfqpoint{1.150000in}{0.150000in}}{\pgfqpoint{5.700000in}{5.700000in}}%
\pgfusepath{clip}%
\pgfsetbuttcap%
\pgfsetroundjoin%
\definecolor{currentfill}{rgb}{0.278012,0.180367,0.486697}%
\pgfsetfillcolor{currentfill}%
\pgfsetfillopacity{0.700000}%
\pgfsetlinewidth{0.000000pt}%
\definecolor{currentstroke}{rgb}{0.000000,0.000000,0.000000}%
\pgfsetstrokecolor{currentstroke}%
\pgfsetdash{}{0pt}%
\pgfpathmoveto{\pgfqpoint{4.642334in}{2.471828in}}%
\pgfpathlineto{\pgfqpoint{4.656085in}{2.470132in}}%
\pgfpathlineto{\pgfqpoint{4.669845in}{2.468508in}}%
\pgfpathlineto{\pgfqpoint{4.683613in}{2.466957in}}%
\pgfpathlineto{\pgfqpoint{4.697389in}{2.465478in}}%
\pgfpathlineto{\pgfqpoint{4.705028in}{2.474207in}}%
\pgfpathlineto{\pgfqpoint{4.712663in}{2.483032in}}%
\pgfpathlineto{\pgfqpoint{4.720294in}{2.491955in}}%
\pgfpathlineto{\pgfqpoint{4.727921in}{2.500983in}}%
\pgfpathlineto{\pgfqpoint{4.714159in}{2.502745in}}%
\pgfpathlineto{\pgfqpoint{4.700405in}{2.504578in}}%
\pgfpathlineto{\pgfqpoint{4.686660in}{2.506483in}}%
\pgfpathlineto{\pgfqpoint{4.672922in}{2.508460in}}%
\pgfpathlineto{\pgfqpoint{4.665281in}{2.499143in}}%
\pgfpathlineto{\pgfqpoint{4.657636in}{2.489935in}}%
\pgfpathlineto{\pgfqpoint{4.649987in}{2.480832in}}%
\pgfpathlineto{\pgfqpoint{4.642334in}{2.471828in}}%
\pgfpathclose%
\pgfusepath{fill}%
\end{pgfscope}%
\begin{pgfscope}%
\pgfpathrectangle{\pgfqpoint{1.150000in}{0.150000in}}{\pgfqpoint{5.700000in}{5.700000in}}%
\pgfusepath{clip}%
\pgfsetbuttcap%
\pgfsetroundjoin%
\definecolor{currentfill}{rgb}{0.243113,0.292092,0.538516}%
\pgfsetfillcolor{currentfill}%
\pgfsetfillopacity{0.700000}%
\pgfsetlinewidth{0.000000pt}%
\definecolor{currentstroke}{rgb}{0.000000,0.000000,0.000000}%
\pgfsetstrokecolor{currentstroke}%
\pgfsetdash{}{0pt}%
\pgfpathmoveto{\pgfqpoint{5.437416in}{2.711929in}}%
\pgfpathlineto{\pgfqpoint{5.451376in}{2.710184in}}%
\pgfpathlineto{\pgfqpoint{5.465344in}{2.708505in}}%
\pgfpathlineto{\pgfqpoint{5.479322in}{2.706894in}}%
\pgfpathlineto{\pgfqpoint{5.493308in}{2.705349in}}%
\pgfpathlineto{\pgfqpoint{5.500714in}{2.715713in}}%
\pgfpathlineto{\pgfqpoint{5.508122in}{2.726329in}}%
\pgfpathlineto{\pgfqpoint{5.515533in}{2.737204in}}%
\pgfpathlineto{\pgfqpoint{5.522947in}{2.748347in}}%
\pgfpathlineto{\pgfqpoint{5.508982in}{2.750336in}}%
\pgfpathlineto{\pgfqpoint{5.495026in}{2.752390in}}%
\pgfpathlineto{\pgfqpoint{5.481079in}{2.754512in}}%
\pgfpathlineto{\pgfqpoint{5.467141in}{2.756699in}}%
\pgfpathlineto{\pgfqpoint{5.459706in}{2.745106in}}%
\pgfpathlineto{\pgfqpoint{5.452274in}{2.733785in}}%
\pgfpathlineto{\pgfqpoint{5.444844in}{2.722729in}}%
\pgfpathlineto{\pgfqpoint{5.437416in}{2.711929in}}%
\pgfpathclose%
\pgfusepath{fill}%
\end{pgfscope}%
\begin{pgfscope}%
\pgfpathrectangle{\pgfqpoint{1.150000in}{0.150000in}}{\pgfqpoint{5.700000in}{5.700000in}}%
\pgfusepath{clip}%
\pgfsetbuttcap%
\pgfsetroundjoin%
\definecolor{currentfill}{rgb}{0.280894,0.078907,0.402329}%
\pgfsetfillcolor{currentfill}%
\pgfsetfillopacity{0.700000}%
\pgfsetlinewidth{0.000000pt}%
\definecolor{currentstroke}{rgb}{0.000000,0.000000,0.000000}%
\pgfsetstrokecolor{currentstroke}%
\pgfsetdash{}{0pt}%
\pgfpathmoveto{\pgfqpoint{3.060214in}{2.281301in}}%
\pgfpathlineto{\pgfqpoint{3.073649in}{2.274286in}}%
\pgfpathlineto{\pgfqpoint{3.087087in}{2.267375in}}%
\pgfpathlineto{\pgfqpoint{3.100528in}{2.260568in}}%
\pgfpathlineto{\pgfqpoint{3.113971in}{2.253863in}}%
\pgfpathlineto{\pgfqpoint{3.122163in}{2.262682in}}%
\pgfpathlineto{\pgfqpoint{3.130349in}{2.271550in}}%
\pgfpathlineto{\pgfqpoint{3.138528in}{2.280465in}}%
\pgfpathlineto{\pgfqpoint{3.146700in}{2.289430in}}%
\pgfpathlineto{\pgfqpoint{3.133270in}{2.296132in}}%
\pgfpathlineto{\pgfqpoint{3.119842in}{2.302936in}}%
\pgfpathlineto{\pgfqpoint{3.106418in}{2.309843in}}%
\pgfpathlineto{\pgfqpoint{3.092995in}{2.316854in}}%
\pgfpathlineto{\pgfqpoint{3.084810in}{2.307886in}}%
\pgfpathlineto{\pgfqpoint{3.076618in}{2.298971in}}%
\pgfpathlineto{\pgfqpoint{3.068419in}{2.290109in}}%
\pgfpathlineto{\pgfqpoint{3.060214in}{2.281301in}}%
\pgfpathclose%
\pgfusepath{fill}%
\end{pgfscope}%
\begin{pgfscope}%
\pgfpathrectangle{\pgfqpoint{1.150000in}{0.150000in}}{\pgfqpoint{5.700000in}{5.700000in}}%
\pgfusepath{clip}%
\pgfsetbuttcap%
\pgfsetroundjoin%
\definecolor{currentfill}{rgb}{0.282910,0.105393,0.426902}%
\pgfsetfillcolor{currentfill}%
\pgfsetfillopacity{0.700000}%
\pgfsetlinewidth{0.000000pt}%
\definecolor{currentstroke}{rgb}{0.000000,0.000000,0.000000}%
\pgfsetstrokecolor{currentstroke}%
\pgfsetdash{}{0pt}%
\pgfpathmoveto{\pgfqpoint{4.018780in}{2.326879in}}%
\pgfpathlineto{\pgfqpoint{4.032368in}{2.324020in}}%
\pgfpathlineto{\pgfqpoint{4.045963in}{2.321241in}}%
\pgfpathlineto{\pgfqpoint{4.059565in}{2.318541in}}%
\pgfpathlineto{\pgfqpoint{4.073173in}{2.315921in}}%
\pgfpathlineto{\pgfqpoint{4.081031in}{2.324927in}}%
\pgfpathlineto{\pgfqpoint{4.088883in}{2.333971in}}%
\pgfpathlineto{\pgfqpoint{4.096730in}{2.343055in}}%
\pgfpathlineto{\pgfqpoint{4.104572in}{2.352182in}}%
\pgfpathlineto{\pgfqpoint{4.090975in}{2.354962in}}%
\pgfpathlineto{\pgfqpoint{4.077384in}{2.357821in}}%
\pgfpathlineto{\pgfqpoint{4.063800in}{2.360760in}}%
\pgfpathlineto{\pgfqpoint{4.050223in}{2.363778in}}%
\pgfpathlineto{\pgfqpoint{4.042370in}{2.354484in}}%
\pgfpathlineto{\pgfqpoint{4.034512in}{2.345238in}}%
\pgfpathlineto{\pgfqpoint{4.026649in}{2.336037in}}%
\pgfpathlineto{\pgfqpoint{4.018780in}{2.326879in}}%
\pgfpathclose%
\pgfusepath{fill}%
\end{pgfscope}%
\begin{pgfscope}%
\pgfpathrectangle{\pgfqpoint{1.150000in}{0.150000in}}{\pgfqpoint{5.700000in}{5.700000in}}%
\pgfusepath{clip}%
\pgfsetbuttcap%
\pgfsetroundjoin%
\definecolor{currentfill}{rgb}{0.280255,0.165693,0.476498}%
\pgfsetfillcolor{currentfill}%
\pgfsetfillopacity{0.700000}%
\pgfsetlinewidth{0.000000pt}%
\definecolor{currentstroke}{rgb}{0.000000,0.000000,0.000000}%
\pgfsetstrokecolor{currentstroke}%
\pgfsetdash{}{0pt}%
\pgfpathmoveto{\pgfqpoint{2.530421in}{2.468260in}}%
\pgfpathlineto{\pgfqpoint{2.543886in}{2.457332in}}%
\pgfpathlineto{\pgfqpoint{2.557350in}{2.446539in}}%
\pgfpathlineto{\pgfqpoint{2.570812in}{2.435877in}}%
\pgfpathlineto{\pgfqpoint{2.584274in}{2.425348in}}%
\pgfpathlineto{\pgfqpoint{2.592681in}{2.433010in}}%
\pgfpathlineto{\pgfqpoint{2.601080in}{2.440763in}}%
\pgfpathlineto{\pgfqpoint{2.609471in}{2.448605in}}%
\pgfpathlineto{\pgfqpoint{2.617852in}{2.456536in}}%
\pgfpathlineto{\pgfqpoint{2.604409in}{2.466999in}}%
\pgfpathlineto{\pgfqpoint{2.590964in}{2.477594in}}%
\pgfpathlineto{\pgfqpoint{2.577519in}{2.488320in}}%
\pgfpathlineto{\pgfqpoint{2.564072in}{2.499181in}}%
\pgfpathlineto{\pgfqpoint{2.555673in}{2.491308in}}%
\pgfpathlineto{\pgfqpoint{2.547265in}{2.483531in}}%
\pgfpathlineto{\pgfqpoint{2.538848in}{2.475847in}}%
\pgfpathlineto{\pgfqpoint{2.530421in}{2.468260in}}%
\pgfpathclose%
\pgfusepath{fill}%
\end{pgfscope}%
\begin{pgfscope}%
\pgfpathrectangle{\pgfqpoint{1.150000in}{0.150000in}}{\pgfqpoint{5.700000in}{5.700000in}}%
\pgfusepath{clip}%
\pgfsetbuttcap%
\pgfsetroundjoin%
\definecolor{currentfill}{rgb}{0.279566,0.067836,0.391917}%
\pgfsetfillcolor{currentfill}%
\pgfsetfillopacity{0.700000}%
\pgfsetlinewidth{0.000000pt}%
\definecolor{currentstroke}{rgb}{0.000000,0.000000,0.000000}%
\pgfsetstrokecolor{currentstroke}%
\pgfsetdash{}{0pt}%
\pgfpathmoveto{\pgfqpoint{3.200449in}{2.263638in}}%
\pgfpathlineto{\pgfqpoint{3.213894in}{2.257440in}}%
\pgfpathlineto{\pgfqpoint{3.227342in}{2.251341in}}%
\pgfpathlineto{\pgfqpoint{3.240794in}{2.245340in}}%
\pgfpathlineto{\pgfqpoint{3.254249in}{2.239437in}}%
\pgfpathlineto{\pgfqpoint{3.262390in}{2.248432in}}%
\pgfpathlineto{\pgfqpoint{3.270525in}{2.257467in}}%
\pgfpathlineto{\pgfqpoint{3.278653in}{2.266544in}}%
\pgfpathlineto{\pgfqpoint{3.286775in}{2.275661in}}%
\pgfpathlineto{\pgfqpoint{3.273332in}{2.281582in}}%
\pgfpathlineto{\pgfqpoint{3.259893in}{2.287600in}}%
\pgfpathlineto{\pgfqpoint{3.246456in}{2.293716in}}%
\pgfpathlineto{\pgfqpoint{3.233024in}{2.299931in}}%
\pgfpathlineto{\pgfqpoint{3.224890in}{2.290788in}}%
\pgfpathlineto{\pgfqpoint{3.216749in}{2.281693in}}%
\pgfpathlineto{\pgfqpoint{3.208603in}{2.272643in}}%
\pgfpathlineto{\pgfqpoint{3.200449in}{2.263638in}}%
\pgfpathclose%
\pgfusepath{fill}%
\end{pgfscope}%
\begin{pgfscope}%
\pgfpathrectangle{\pgfqpoint{1.150000in}{0.150000in}}{\pgfqpoint{5.700000in}{5.700000in}}%
\pgfusepath{clip}%
\pgfsetbuttcap%
\pgfsetroundjoin%
\definecolor{currentfill}{rgb}{0.269308,0.218818,0.509577}%
\pgfsetfillcolor{currentfill}%
\pgfsetfillopacity{0.700000}%
\pgfsetlinewidth{0.000000pt}%
\definecolor{currentstroke}{rgb}{0.000000,0.000000,0.000000}%
\pgfsetstrokecolor{currentstroke}%
\pgfsetdash{}{0pt}%
\pgfpathmoveto{\pgfqpoint{4.954234in}{2.554549in}}%
\pgfpathlineto{\pgfqpoint{4.968074in}{2.553083in}}%
\pgfpathlineto{\pgfqpoint{4.981923in}{2.551687in}}%
\pgfpathlineto{\pgfqpoint{4.995780in}{2.550361in}}%
\pgfpathlineto{\pgfqpoint{5.009646in}{2.549104in}}%
\pgfpathlineto{\pgfqpoint{5.017180in}{2.557982in}}%
\pgfpathlineto{\pgfqpoint{5.024711in}{2.567002in}}%
\pgfpathlineto{\pgfqpoint{5.032240in}{2.576170in}}%
\pgfpathlineto{\pgfqpoint{5.039767in}{2.585492in}}%
\pgfpathlineto{\pgfqpoint{5.025918in}{2.587092in}}%
\pgfpathlineto{\pgfqpoint{5.012077in}{2.588761in}}%
\pgfpathlineto{\pgfqpoint{4.998246in}{2.590499in}}%
\pgfpathlineto{\pgfqpoint{4.984422in}{2.592308in}}%
\pgfpathlineto{\pgfqpoint{4.976879in}{2.582636in}}%
\pgfpathlineto{\pgfqpoint{4.969333in}{2.573124in}}%
\pgfpathlineto{\pgfqpoint{4.961785in}{2.563763in}}%
\pgfpathlineto{\pgfqpoint{4.954234in}{2.554549in}}%
\pgfpathclose%
\pgfusepath{fill}%
\end{pgfscope}%
\begin{pgfscope}%
\pgfpathrectangle{\pgfqpoint{1.150000in}{0.150000in}}{\pgfqpoint{5.700000in}{5.700000in}}%
\pgfusepath{clip}%
\pgfsetbuttcap%
\pgfsetroundjoin%
\definecolor{currentfill}{rgb}{0.281924,0.089666,0.412415}%
\pgfsetfillcolor{currentfill}%
\pgfsetfillopacity{0.700000}%
\pgfsetlinewidth{0.000000pt}%
\definecolor{currentstroke}{rgb}{0.000000,0.000000,0.000000}%
\pgfsetstrokecolor{currentstroke}%
\pgfsetdash{}{0pt}%
\pgfpathmoveto{\pgfqpoint{2.919803in}{2.306737in}}%
\pgfpathlineto{\pgfqpoint{2.933237in}{2.298839in}}%
\pgfpathlineto{\pgfqpoint{2.946672in}{2.291052in}}%
\pgfpathlineto{\pgfqpoint{2.960109in}{2.283373in}}%
\pgfpathlineto{\pgfqpoint{2.973548in}{2.275803in}}%
\pgfpathlineto{\pgfqpoint{2.981795in}{2.284370in}}%
\pgfpathlineto{\pgfqpoint{2.990035in}{2.292995in}}%
\pgfpathlineto{\pgfqpoint{2.998268in}{2.301676in}}%
\pgfpathlineto{\pgfqpoint{3.006495in}{2.310416in}}%
\pgfpathlineto{\pgfqpoint{2.993070in}{2.317961in}}%
\pgfpathlineto{\pgfqpoint{2.979647in}{2.325616in}}%
\pgfpathlineto{\pgfqpoint{2.966226in}{2.333379in}}%
\pgfpathlineto{\pgfqpoint{2.952806in}{2.341253in}}%
\pgfpathlineto{\pgfqpoint{2.944566in}{2.332530in}}%
\pgfpathlineto{\pgfqpoint{2.936319in}{2.323870in}}%
\pgfpathlineto{\pgfqpoint{2.928065in}{2.315273in}}%
\pgfpathlineto{\pgfqpoint{2.919803in}{2.306737in}}%
\pgfpathclose%
\pgfusepath{fill}%
\end{pgfscope}%
\begin{pgfscope}%
\pgfpathrectangle{\pgfqpoint{1.150000in}{0.150000in}}{\pgfqpoint{5.700000in}{5.700000in}}%
\pgfusepath{clip}%
\pgfsetbuttcap%
\pgfsetroundjoin%
\definecolor{currentfill}{rgb}{0.279566,0.067836,0.391917}%
\pgfsetfillcolor{currentfill}%
\pgfsetfillopacity{0.700000}%
\pgfsetlinewidth{0.000000pt}%
\definecolor{currentstroke}{rgb}{0.000000,0.000000,0.000000}%
\pgfsetstrokecolor{currentstroke}%
\pgfsetdash{}{0pt}%
\pgfpathmoveto{\pgfqpoint{3.340583in}{2.252944in}}%
\pgfpathlineto{\pgfqpoint{3.354044in}{2.247503in}}%
\pgfpathlineto{\pgfqpoint{3.367510in}{2.242156in}}%
\pgfpathlineto{\pgfqpoint{3.380979in}{2.236903in}}%
\pgfpathlineto{\pgfqpoint{3.394453in}{2.231743in}}%
\pgfpathlineto{\pgfqpoint{3.402545in}{2.240844in}}%
\pgfpathlineto{\pgfqpoint{3.410631in}{2.249978in}}%
\pgfpathlineto{\pgfqpoint{3.418712in}{2.259146in}}%
\pgfpathlineto{\pgfqpoint{3.426786in}{2.268351in}}%
\pgfpathlineto{\pgfqpoint{3.413324in}{2.273548in}}%
\pgfpathlineto{\pgfqpoint{3.399866in}{2.278839in}}%
\pgfpathlineto{\pgfqpoint{3.386412in}{2.284223in}}%
\pgfpathlineto{\pgfqpoint{3.372962in}{2.289702in}}%
\pgfpathlineto{\pgfqpoint{3.364876in}{2.280452in}}%
\pgfpathlineto{\pgfqpoint{3.356785in}{2.271244in}}%
\pgfpathlineto{\pgfqpoint{3.348687in}{2.262075in}}%
\pgfpathlineto{\pgfqpoint{3.340583in}{2.252944in}}%
\pgfpathclose%
\pgfusepath{fill}%
\end{pgfscope}%
\begin{pgfscope}%
\pgfpathrectangle{\pgfqpoint{1.150000in}{0.150000in}}{\pgfqpoint{5.700000in}{5.700000in}}%
\pgfusepath{clip}%
\pgfsetbuttcap%
\pgfsetroundjoin%
\definecolor{currentfill}{rgb}{0.248629,0.278775,0.534556}%
\pgfsetfillcolor{currentfill}%
\pgfsetfillopacity{0.700000}%
\pgfsetlinewidth{0.000000pt}%
\definecolor{currentstroke}{rgb}{0.000000,0.000000,0.000000}%
\pgfsetstrokecolor{currentstroke}%
\pgfsetdash{}{0pt}%
\pgfpathmoveto{\pgfqpoint{5.351894in}{2.677092in}}%
\pgfpathlineto{\pgfqpoint{5.365837in}{2.675502in}}%
\pgfpathlineto{\pgfqpoint{5.379790in}{2.673980in}}%
\pgfpathlineto{\pgfqpoint{5.393752in}{2.672525in}}%
\pgfpathlineto{\pgfqpoint{5.407723in}{2.671137in}}%
\pgfpathlineto{\pgfqpoint{5.415145in}{2.680988in}}%
\pgfpathlineto{\pgfqpoint{5.422567in}{2.691066in}}%
\pgfpathlineto{\pgfqpoint{5.429991in}{2.701377in}}%
\pgfpathlineto{\pgfqpoint{5.437416in}{2.711929in}}%
\pgfpathlineto{\pgfqpoint{5.423466in}{2.713740in}}%
\pgfpathlineto{\pgfqpoint{5.409525in}{2.715619in}}%
\pgfpathlineto{\pgfqpoint{5.395593in}{2.717564in}}%
\pgfpathlineto{\pgfqpoint{5.381670in}{2.719576in}}%
\pgfpathlineto{\pgfqpoint{5.374224in}{2.708594in}}%
\pgfpathlineto{\pgfqpoint{5.366779in}{2.697858in}}%
\pgfpathlineto{\pgfqpoint{5.359336in}{2.687360in}}%
\pgfpathlineto{\pgfqpoint{5.351894in}{2.677092in}}%
\pgfpathclose%
\pgfusepath{fill}%
\end{pgfscope}%
\begin{pgfscope}%
\pgfpathrectangle{\pgfqpoint{1.150000in}{0.150000in}}{\pgfqpoint{5.700000in}{5.700000in}}%
\pgfusepath{clip}%
\pgfsetbuttcap%
\pgfsetroundjoin%
\definecolor{currentfill}{rgb}{0.283072,0.130895,0.449241}%
\pgfsetfillcolor{currentfill}%
\pgfsetfillopacity{0.700000}%
\pgfsetlinewidth{0.000000pt}%
\definecolor{currentstroke}{rgb}{0.000000,0.000000,0.000000}%
\pgfsetstrokecolor{currentstroke}%
\pgfsetdash{}{0pt}%
\pgfpathmoveto{\pgfqpoint{4.244822in}{2.368325in}}%
\pgfpathlineto{\pgfqpoint{4.258470in}{2.366065in}}%
\pgfpathlineto{\pgfqpoint{4.272126in}{2.363882in}}%
\pgfpathlineto{\pgfqpoint{4.285790in}{2.361774in}}%
\pgfpathlineto{\pgfqpoint{4.299460in}{2.359743in}}%
\pgfpathlineto{\pgfqpoint{4.307241in}{2.368559in}}%
\pgfpathlineto{\pgfqpoint{4.315017in}{2.377424in}}%
\pgfpathlineto{\pgfqpoint{4.322788in}{2.386342in}}%
\pgfpathlineto{\pgfqpoint{4.330553in}{2.395317in}}%
\pgfpathlineto{\pgfqpoint{4.316894in}{2.397549in}}%
\pgfpathlineto{\pgfqpoint{4.303243in}{2.399856in}}%
\pgfpathlineto{\pgfqpoint{4.289599in}{2.402240in}}%
\pgfpathlineto{\pgfqpoint{4.275962in}{2.404701in}}%
\pgfpathlineto{\pgfqpoint{4.268184in}{2.395518in}}%
\pgfpathlineto{\pgfqpoint{4.260402in}{2.386397in}}%
\pgfpathlineto{\pgfqpoint{4.252614in}{2.377334in}}%
\pgfpathlineto{\pgfqpoint{4.244822in}{2.368325in}}%
\pgfpathclose%
\pgfusepath{fill}%
\end{pgfscope}%
\begin{pgfscope}%
\pgfpathrectangle{\pgfqpoint{1.150000in}{0.150000in}}{\pgfqpoint{5.700000in}{5.700000in}}%
\pgfusepath{clip}%
\pgfsetbuttcap%
\pgfsetroundjoin%
\definecolor{currentfill}{rgb}{0.280255,0.165693,0.476498}%
\pgfsetfillcolor{currentfill}%
\pgfsetfillopacity{0.700000}%
\pgfsetlinewidth{0.000000pt}%
\definecolor{currentstroke}{rgb}{0.000000,0.000000,0.000000}%
\pgfsetstrokecolor{currentstroke}%
\pgfsetdash{}{0pt}%
\pgfpathmoveto{\pgfqpoint{4.556697in}{2.443168in}}%
\pgfpathlineto{\pgfqpoint{4.570430in}{2.441443in}}%
\pgfpathlineto{\pgfqpoint{4.584171in}{2.439791in}}%
\pgfpathlineto{\pgfqpoint{4.597921in}{2.438212in}}%
\pgfpathlineto{\pgfqpoint{4.611678in}{2.436705in}}%
\pgfpathlineto{\pgfqpoint{4.619349in}{2.445361in}}%
\pgfpathlineto{\pgfqpoint{4.627015in}{2.454097in}}%
\pgfpathlineto{\pgfqpoint{4.634677in}{2.462918in}}%
\pgfpathlineto{\pgfqpoint{4.642334in}{2.471828in}}%
\pgfpathlineto{\pgfqpoint{4.628590in}{2.473596in}}%
\pgfpathlineto{\pgfqpoint{4.614855in}{2.475437in}}%
\pgfpathlineto{\pgfqpoint{4.601127in}{2.477351in}}%
\pgfpathlineto{\pgfqpoint{4.587408in}{2.479338in}}%
\pgfpathlineto{\pgfqpoint{4.579737in}{2.470159in}}%
\pgfpathlineto{\pgfqpoint{4.572061in}{2.461074in}}%
\pgfpathlineto{\pgfqpoint{4.564381in}{2.452079in}}%
\pgfpathlineto{\pgfqpoint{4.556697in}{2.443168in}}%
\pgfpathclose%
\pgfusepath{fill}%
\end{pgfscope}%
\begin{pgfscope}%
\pgfpathrectangle{\pgfqpoint{1.150000in}{0.150000in}}{\pgfqpoint{5.700000in}{5.700000in}}%
\pgfusepath{clip}%
\pgfsetbuttcap%
\pgfsetroundjoin%
\definecolor{currentfill}{rgb}{0.280894,0.078907,0.402329}%
\pgfsetfillcolor{currentfill}%
\pgfsetfillopacity{0.700000}%
\pgfsetlinewidth{0.000000pt}%
\definecolor{currentstroke}{rgb}{0.000000,0.000000,0.000000}%
\pgfsetstrokecolor{currentstroke}%
\pgfsetdash{}{0pt}%
\pgfpathmoveto{\pgfqpoint{3.706843in}{2.270379in}}%
\pgfpathlineto{\pgfqpoint{3.720367in}{2.266557in}}%
\pgfpathlineto{\pgfqpoint{3.733896in}{2.262821in}}%
\pgfpathlineto{\pgfqpoint{3.747430in}{2.259169in}}%
\pgfpathlineto{\pgfqpoint{3.760970in}{2.255602in}}%
\pgfpathlineto{\pgfqpoint{3.768938in}{2.264739in}}%
\pgfpathlineto{\pgfqpoint{3.776900in}{2.273903in}}%
\pgfpathlineto{\pgfqpoint{3.784856in}{2.283097in}}%
\pgfpathlineto{\pgfqpoint{3.792807in}{2.292321in}}%
\pgfpathlineto{\pgfqpoint{3.779277in}{2.295987in}}%
\pgfpathlineto{\pgfqpoint{3.765753in}{2.299737in}}%
\pgfpathlineto{\pgfqpoint{3.752235in}{2.303572in}}%
\pgfpathlineto{\pgfqpoint{3.738722in}{2.307493in}}%
\pgfpathlineto{\pgfqpoint{3.730761in}{2.298162in}}%
\pgfpathlineto{\pgfqpoint{3.722794in}{2.288868in}}%
\pgfpathlineto{\pgfqpoint{3.714821in}{2.279608in}}%
\pgfpathlineto{\pgfqpoint{3.706843in}{2.270379in}}%
\pgfpathclose%
\pgfusepath{fill}%
\end{pgfscope}%
\begin{pgfscope}%
\pgfpathrectangle{\pgfqpoint{1.150000in}{0.150000in}}{\pgfqpoint{5.700000in}{5.700000in}}%
\pgfusepath{clip}%
\pgfsetbuttcap%
\pgfsetroundjoin%
\definecolor{currentfill}{rgb}{0.253935,0.265254,0.529983}%
\pgfsetfillcolor{currentfill}%
\pgfsetfillopacity{0.700000}%
\pgfsetlinewidth{0.000000pt}%
\definecolor{currentstroke}{rgb}{0.000000,0.000000,0.000000}%
\pgfsetstrokecolor{currentstroke}%
\pgfsetdash{}{0pt}%
\pgfpathmoveto{\pgfqpoint{5.266366in}{2.643596in}}%
\pgfpathlineto{\pgfqpoint{5.280293in}{2.642140in}}%
\pgfpathlineto{\pgfqpoint{5.294229in}{2.640752in}}%
\pgfpathlineto{\pgfqpoint{5.308175in}{2.639431in}}%
\pgfpathlineto{\pgfqpoint{5.322130in}{2.638178in}}%
\pgfpathlineto{\pgfqpoint{5.329571in}{2.647597in}}%
\pgfpathlineto{\pgfqpoint{5.337011in}{2.657218in}}%
\pgfpathlineto{\pgfqpoint{5.344452in}{2.667047in}}%
\pgfpathlineto{\pgfqpoint{5.351894in}{2.677092in}}%
\pgfpathlineto{\pgfqpoint{5.337959in}{2.678748in}}%
\pgfpathlineto{\pgfqpoint{5.324034in}{2.680472in}}%
\pgfpathlineto{\pgfqpoint{5.310117in}{2.682264in}}%
\pgfpathlineto{\pgfqpoint{5.296209in}{2.684122in}}%
\pgfpathlineto{\pgfqpoint{5.288748in}{2.673667in}}%
\pgfpathlineto{\pgfqpoint{5.281287in}{2.663433in}}%
\pgfpathlineto{\pgfqpoint{5.273826in}{2.653411in}}%
\pgfpathlineto{\pgfqpoint{5.266366in}{2.643596in}}%
\pgfpathclose%
\pgfusepath{fill}%
\end{pgfscope}%
\begin{pgfscope}%
\pgfpathrectangle{\pgfqpoint{1.150000in}{0.150000in}}{\pgfqpoint{5.700000in}{5.700000in}}%
\pgfusepath{clip}%
\pgfsetbuttcap%
\pgfsetroundjoin%
\definecolor{currentfill}{rgb}{0.283091,0.110553,0.431554}%
\pgfsetfillcolor{currentfill}%
\pgfsetfillopacity{0.700000}%
\pgfsetlinewidth{0.000000pt}%
\definecolor{currentstroke}{rgb}{0.000000,0.000000,0.000000}%
\pgfsetstrokecolor{currentstroke}%
\pgfsetdash{}{0pt}%
\pgfpathmoveto{\pgfqpoint{2.779133in}{2.340824in}}%
\pgfpathlineto{\pgfqpoint{2.792574in}{2.331970in}}%
\pgfpathlineto{\pgfqpoint{2.806015in}{2.323234in}}%
\pgfpathlineto{\pgfqpoint{2.819457in}{2.314613in}}%
\pgfpathlineto{\pgfqpoint{2.832899in}{2.306108in}}%
\pgfpathlineto{\pgfqpoint{2.841207in}{2.314341in}}%
\pgfpathlineto{\pgfqpoint{2.849506in}{2.322641in}}%
\pgfpathlineto{\pgfqpoint{2.857798in}{2.331010in}}%
\pgfpathlineto{\pgfqpoint{2.866082in}{2.339446in}}%
\pgfpathlineto{\pgfqpoint{2.852654in}{2.347906in}}%
\pgfpathlineto{\pgfqpoint{2.839228in}{2.356481in}}%
\pgfpathlineto{\pgfqpoint{2.825802in}{2.365173in}}%
\pgfpathlineto{\pgfqpoint{2.812377in}{2.373981in}}%
\pgfpathlineto{\pgfqpoint{2.804078in}{2.365583in}}%
\pgfpathlineto{\pgfqpoint{2.795771in}{2.357257in}}%
\pgfpathlineto{\pgfqpoint{2.787456in}{2.349004in}}%
\pgfpathlineto{\pgfqpoint{2.779133in}{2.340824in}}%
\pgfpathclose%
\pgfusepath{fill}%
\end{pgfscope}%
\begin{pgfscope}%
\pgfpathrectangle{\pgfqpoint{1.150000in}{0.150000in}}{\pgfqpoint{5.700000in}{5.700000in}}%
\pgfusepath{clip}%
\pgfsetbuttcap%
\pgfsetroundjoin%
\definecolor{currentfill}{rgb}{0.271828,0.209303,0.504434}%
\pgfsetfillcolor{currentfill}%
\pgfsetfillopacity{0.700000}%
\pgfsetlinewidth{0.000000pt}%
\definecolor{currentstroke}{rgb}{0.000000,0.000000,0.000000}%
\pgfsetstrokecolor{currentstroke}%
\pgfsetdash{}{0pt}%
\pgfpathmoveto{\pgfqpoint{4.868664in}{2.524311in}}%
\pgfpathlineto{\pgfqpoint{4.882485in}{2.522887in}}%
\pgfpathlineto{\pgfqpoint{4.896316in}{2.521534in}}%
\pgfpathlineto{\pgfqpoint{4.910154in}{2.520251in}}%
\pgfpathlineto{\pgfqpoint{4.924002in}{2.519038in}}%
\pgfpathlineto{\pgfqpoint{4.931565in}{2.527726in}}%
\pgfpathlineto{\pgfqpoint{4.939124in}{2.536537in}}%
\pgfpathlineto{\pgfqpoint{4.946681in}{2.545476in}}%
\pgfpathlineto{\pgfqpoint{4.954234in}{2.554549in}}%
\pgfpathlineto{\pgfqpoint{4.940403in}{2.556085in}}%
\pgfpathlineto{\pgfqpoint{4.926580in}{2.557691in}}%
\pgfpathlineto{\pgfqpoint{4.912766in}{2.559366in}}%
\pgfpathlineto{\pgfqpoint{4.898960in}{2.561112in}}%
\pgfpathlineto{\pgfqpoint{4.891390in}{2.551709in}}%
\pgfpathlineto{\pgfqpoint{4.883818in}{2.542445in}}%
\pgfpathlineto{\pgfqpoint{4.876242in}{2.533314in}}%
\pgfpathlineto{\pgfqpoint{4.868664in}{2.524311in}}%
\pgfpathclose%
\pgfusepath{fill}%
\end{pgfscope}%
\begin{pgfscope}%
\pgfpathrectangle{\pgfqpoint{1.150000in}{0.150000in}}{\pgfqpoint{5.700000in}{5.700000in}}%
\pgfusepath{clip}%
\pgfsetbuttcap%
\pgfsetroundjoin%
\definecolor{currentfill}{rgb}{0.282327,0.094955,0.417331}%
\pgfsetfillcolor{currentfill}%
\pgfsetfillopacity{0.700000}%
\pgfsetlinewidth{0.000000pt}%
\definecolor{currentstroke}{rgb}{0.000000,0.000000,0.000000}%
\pgfsetstrokecolor{currentstroke}%
\pgfsetdash{}{0pt}%
\pgfpathmoveto{\pgfqpoint{3.932918in}{2.302288in}}%
\pgfpathlineto{\pgfqpoint{3.946492in}{2.299246in}}%
\pgfpathlineto{\pgfqpoint{3.960071in}{2.296286in}}%
\pgfpathlineto{\pgfqpoint{3.973658in}{2.293406in}}%
\pgfpathlineto{\pgfqpoint{3.987250in}{2.290607in}}%
\pgfpathlineto{\pgfqpoint{3.995141in}{2.299626in}}%
\pgfpathlineto{\pgfqpoint{4.003026in}{2.308676in}}%
\pgfpathlineto{\pgfqpoint{4.010906in}{2.317759in}}%
\pgfpathlineto{\pgfqpoint{4.018780in}{2.326879in}}%
\pgfpathlineto{\pgfqpoint{4.005198in}{2.329818in}}%
\pgfpathlineto{\pgfqpoint{3.991623in}{2.332837in}}%
\pgfpathlineto{\pgfqpoint{3.978054in}{2.335937in}}%
\pgfpathlineto{\pgfqpoint{3.964491in}{2.339117in}}%
\pgfpathlineto{\pgfqpoint{3.956606in}{2.329851in}}%
\pgfpathlineto{\pgfqpoint{3.948716in}{2.320626in}}%
\pgfpathlineto{\pgfqpoint{3.940820in}{2.311439in}}%
\pgfpathlineto{\pgfqpoint{3.932918in}{2.302288in}}%
\pgfpathclose%
\pgfusepath{fill}%
\end{pgfscope}%
\begin{pgfscope}%
\pgfpathrectangle{\pgfqpoint{1.150000in}{0.150000in}}{\pgfqpoint{5.700000in}{5.700000in}}%
\pgfusepath{clip}%
\pgfsetbuttcap%
\pgfsetroundjoin%
\definecolor{currentfill}{rgb}{0.281887,0.150881,0.465405}%
\pgfsetfillcolor{currentfill}%
\pgfsetfillopacity{0.700000}%
\pgfsetlinewidth{0.000000pt}%
\definecolor{currentstroke}{rgb}{0.000000,0.000000,0.000000}%
\pgfsetstrokecolor{currentstroke}%
\pgfsetdash{}{0pt}%
\pgfpathmoveto{\pgfqpoint{2.584274in}{2.425348in}}%
\pgfpathlineto{\pgfqpoint{2.597734in}{2.414948in}}%
\pgfpathlineto{\pgfqpoint{2.611193in}{2.404678in}}%
\pgfpathlineto{\pgfqpoint{2.624651in}{2.394536in}}%
\pgfpathlineto{\pgfqpoint{2.638109in}{2.384520in}}%
\pgfpathlineto{\pgfqpoint{2.646499in}{2.392257in}}%
\pgfpathlineto{\pgfqpoint{2.654880in}{2.400078in}}%
\pgfpathlineto{\pgfqpoint{2.663253in}{2.407984in}}%
\pgfpathlineto{\pgfqpoint{2.671617in}{2.415975in}}%
\pgfpathlineto{\pgfqpoint{2.658177in}{2.425924in}}%
\pgfpathlineto{\pgfqpoint{2.644736in}{2.436000in}}%
\pgfpathlineto{\pgfqpoint{2.631294in}{2.446203in}}%
\pgfpathlineto{\pgfqpoint{2.617852in}{2.456536in}}%
\pgfpathlineto{\pgfqpoint{2.609471in}{2.448605in}}%
\pgfpathlineto{\pgfqpoint{2.601080in}{2.440763in}}%
\pgfpathlineto{\pgfqpoint{2.592681in}{2.433010in}}%
\pgfpathlineto{\pgfqpoint{2.584274in}{2.425348in}}%
\pgfpathclose%
\pgfusepath{fill}%
\end{pgfscope}%
\begin{pgfscope}%
\pgfpathrectangle{\pgfqpoint{1.150000in}{0.150000in}}{\pgfqpoint{5.700000in}{5.700000in}}%
\pgfusepath{clip}%
\pgfsetbuttcap%
\pgfsetroundjoin%
\definecolor{currentfill}{rgb}{0.279566,0.067836,0.391917}%
\pgfsetfillcolor{currentfill}%
\pgfsetfillopacity{0.700000}%
\pgfsetlinewidth{0.000000pt}%
\definecolor{currentstroke}{rgb}{0.000000,0.000000,0.000000}%
\pgfsetstrokecolor{currentstroke}%
\pgfsetdash{}{0pt}%
\pgfpathmoveto{\pgfqpoint{3.480678in}{2.248482in}}%
\pgfpathlineto{\pgfqpoint{3.494162in}{2.243743in}}%
\pgfpathlineto{\pgfqpoint{3.507651in}{2.239094in}}%
\pgfpathlineto{\pgfqpoint{3.521145in}{2.234535in}}%
\pgfpathlineto{\pgfqpoint{3.534643in}{2.230066in}}%
\pgfpathlineto{\pgfqpoint{3.542689in}{2.239206in}}%
\pgfpathlineto{\pgfqpoint{3.550729in}{2.248375in}}%
\pgfpathlineto{\pgfqpoint{3.558764in}{2.257573in}}%
\pgfpathlineto{\pgfqpoint{3.566792in}{2.266803in}}%
\pgfpathlineto{\pgfqpoint{3.553305in}{2.271331in}}%
\pgfpathlineto{\pgfqpoint{3.539822in}{2.275948in}}%
\pgfpathlineto{\pgfqpoint{3.526344in}{2.280655in}}%
\pgfpathlineto{\pgfqpoint{3.512871in}{2.285452in}}%
\pgfpathlineto{\pgfqpoint{3.504832in}{2.276157in}}%
\pgfpathlineto{\pgfqpoint{3.496786in}{2.266898in}}%
\pgfpathlineto{\pgfqpoint{3.488735in}{2.257673in}}%
\pgfpathlineto{\pgfqpoint{3.480678in}{2.248482in}}%
\pgfpathclose%
\pgfusepath{fill}%
\end{pgfscope}%
\begin{pgfscope}%
\pgfpathrectangle{\pgfqpoint{1.150000in}{0.150000in}}{\pgfqpoint{5.700000in}{5.700000in}}%
\pgfusepath{clip}%
\pgfsetbuttcap%
\pgfsetroundjoin%
\definecolor{currentfill}{rgb}{0.212395,0.359683,0.551710}%
\pgfsetfillcolor{currentfill}%
\pgfsetfillopacity{0.700000}%
\pgfsetlinewidth{0.000000pt}%
\definecolor{currentstroke}{rgb}{0.000000,0.000000,0.000000}%
\pgfsetstrokecolor{currentstroke}%
\pgfsetdash{}{0pt}%
\pgfpathmoveto{\pgfqpoint{5.835859in}{2.859916in}}%
\pgfpathlineto{\pgfqpoint{5.849910in}{2.857592in}}%
\pgfpathlineto{\pgfqpoint{5.863969in}{2.855332in}}%
\pgfpathlineto{\pgfqpoint{5.878039in}{2.853137in}}%
\pgfpathlineto{\pgfqpoint{5.892118in}{2.851007in}}%
\pgfpathlineto{\pgfqpoint{5.899485in}{2.863822in}}%
\pgfpathlineto{\pgfqpoint{5.906862in}{2.877003in}}%
\pgfpathlineto{\pgfqpoint{5.914248in}{2.890558in}}%
\pgfpathlineto{\pgfqpoint{5.921644in}{2.904498in}}%
\pgfpathlineto{\pgfqpoint{5.907591in}{2.907152in}}%
\pgfpathlineto{\pgfqpoint{5.893547in}{2.909870in}}%
\pgfpathlineto{\pgfqpoint{5.879512in}{2.912653in}}%
\pgfpathlineto{\pgfqpoint{5.865486in}{2.915500in}}%
\pgfpathlineto{\pgfqpoint{5.858066in}{2.901030in}}%
\pgfpathlineto{\pgfqpoint{5.850654in}{2.886949in}}%
\pgfpathlineto{\pgfqpoint{5.843252in}{2.873248in}}%
\pgfpathlineto{\pgfqpoint{5.835859in}{2.859916in}}%
\pgfpathclose%
\pgfusepath{fill}%
\end{pgfscope}%
\begin{pgfscope}%
\pgfpathrectangle{\pgfqpoint{1.150000in}{0.150000in}}{\pgfqpoint{5.700000in}{5.700000in}}%
\pgfusepath{clip}%
\pgfsetbuttcap%
\pgfsetroundjoin%
\definecolor{currentfill}{rgb}{0.204903,0.375746,0.553533}%
\pgfsetfillcolor{currentfill}%
\pgfsetfillopacity{0.700000}%
\pgfsetlinewidth{0.000000pt}%
\definecolor{currentstroke}{rgb}{0.000000,0.000000,0.000000}%
\pgfsetstrokecolor{currentstroke}%
\pgfsetdash{}{0pt}%
\pgfpathmoveto{\pgfqpoint{5.921644in}{2.904498in}}%
\pgfpathlineto{\pgfqpoint{5.935706in}{2.901908in}}%
\pgfpathlineto{\pgfqpoint{5.949778in}{2.899383in}}%
\pgfpathlineto{\pgfqpoint{5.963859in}{2.896922in}}%
\pgfpathlineto{\pgfqpoint{5.977949in}{2.894525in}}%
\pgfpathlineto{\pgfqpoint{5.985329in}{2.908323in}}%
\pgfpathlineto{\pgfqpoint{5.992719in}{2.922519in}}%
\pgfpathlineto{\pgfqpoint{6.000122in}{2.937123in}}%
\pgfpathlineto{\pgfqpoint{6.007536in}{2.952147in}}%
\pgfpathlineto{\pgfqpoint{5.993472in}{2.955087in}}%
\pgfpathlineto{\pgfqpoint{5.979417in}{2.958091in}}%
\pgfpathlineto{\pgfqpoint{5.965372in}{2.961159in}}%
\pgfpathlineto{\pgfqpoint{5.951335in}{2.964292in}}%
\pgfpathlineto{\pgfqpoint{5.943895in}{2.948718in}}%
\pgfpathlineto{\pgfqpoint{5.936467in}{2.933568in}}%
\pgfpathlineto{\pgfqpoint{5.929050in}{2.918831in}}%
\pgfpathlineto{\pgfqpoint{5.921644in}{2.904498in}}%
\pgfpathclose%
\pgfusepath{fill}%
\end{pgfscope}%
\begin{pgfscope}%
\pgfpathrectangle{\pgfqpoint{1.150000in}{0.150000in}}{\pgfqpoint{5.700000in}{5.700000in}}%
\pgfusepath{clip}%
\pgfsetbuttcap%
\pgfsetroundjoin%
\definecolor{currentfill}{rgb}{0.281412,0.155834,0.469201}%
\pgfsetfillcolor{currentfill}%
\pgfsetfillopacity{0.700000}%
\pgfsetlinewidth{0.000000pt}%
\definecolor{currentstroke}{rgb}{0.000000,0.000000,0.000000}%
\pgfsetstrokecolor{currentstroke}%
\pgfsetdash{}{0pt}%
\pgfpathmoveto{\pgfqpoint{4.471008in}{2.414952in}}%
\pgfpathlineto{\pgfqpoint{4.484722in}{2.413173in}}%
\pgfpathlineto{\pgfqpoint{4.498445in}{2.411469in}}%
\pgfpathlineto{\pgfqpoint{4.512175in}{2.409838in}}%
\pgfpathlineto{\pgfqpoint{4.525914in}{2.408282in}}%
\pgfpathlineto{\pgfqpoint{4.533617in}{2.416899in}}%
\pgfpathlineto{\pgfqpoint{4.541315in}{2.425583in}}%
\pgfpathlineto{\pgfqpoint{4.549008in}{2.434338in}}%
\pgfpathlineto{\pgfqpoint{4.556697in}{2.443168in}}%
\pgfpathlineto{\pgfqpoint{4.542972in}{2.444967in}}%
\pgfpathlineto{\pgfqpoint{4.529255in}{2.446839in}}%
\pgfpathlineto{\pgfqpoint{4.515545in}{2.448785in}}%
\pgfpathlineto{\pgfqpoint{4.501843in}{2.450804in}}%
\pgfpathlineto{\pgfqpoint{4.494141in}{2.441725in}}%
\pgfpathlineto{\pgfqpoint{4.486435in}{2.432726in}}%
\pgfpathlineto{\pgfqpoint{4.478724in}{2.423803in}}%
\pgfpathlineto{\pgfqpoint{4.471008in}{2.414952in}}%
\pgfpathclose%
\pgfusepath{fill}%
\end{pgfscope}%
\begin{pgfscope}%
\pgfpathrectangle{\pgfqpoint{1.150000in}{0.150000in}}{\pgfqpoint{5.700000in}{5.700000in}}%
\pgfusepath{clip}%
\pgfsetbuttcap%
\pgfsetroundjoin%
\definecolor{currentfill}{rgb}{0.220057,0.343307,0.549413}%
\pgfsetfillcolor{currentfill}%
\pgfsetfillopacity{0.700000}%
\pgfsetlinewidth{0.000000pt}%
\definecolor{currentstroke}{rgb}{0.000000,0.000000,0.000000}%
\pgfsetstrokecolor{currentstroke}%
\pgfsetdash{}{0pt}%
\pgfpathmoveto{\pgfqpoint{5.750154in}{2.818037in}}%
\pgfpathlineto{\pgfqpoint{5.764193in}{2.815957in}}%
\pgfpathlineto{\pgfqpoint{5.778240in}{2.813941in}}%
\pgfpathlineto{\pgfqpoint{5.792297in}{2.811990in}}%
\pgfpathlineto{\pgfqpoint{5.806364in}{2.810105in}}%
\pgfpathlineto{\pgfqpoint{5.813727in}{2.822049in}}%
\pgfpathlineto{\pgfqpoint{5.821096in}{2.834326in}}%
\pgfpathlineto{\pgfqpoint{5.828474in}{2.846945in}}%
\pgfpathlineto{\pgfqpoint{5.835859in}{2.859916in}}%
\pgfpathlineto{\pgfqpoint{5.821817in}{2.862305in}}%
\pgfpathlineto{\pgfqpoint{5.807785in}{2.864759in}}%
\pgfpathlineto{\pgfqpoint{5.793762in}{2.867278in}}%
\pgfpathlineto{\pgfqpoint{5.779748in}{2.869862in}}%
\pgfpathlineto{\pgfqpoint{5.772338in}{2.856381in}}%
\pgfpathlineto{\pgfqpoint{5.764937in}{2.843256in}}%
\pgfpathlineto{\pgfqpoint{5.757542in}{2.830477in}}%
\pgfpathlineto{\pgfqpoint{5.750154in}{2.818037in}}%
\pgfpathclose%
\pgfusepath{fill}%
\end{pgfscope}%
\begin{pgfscope}%
\pgfpathrectangle{\pgfqpoint{1.150000in}{0.150000in}}{\pgfqpoint{5.700000in}{5.700000in}}%
\pgfusepath{clip}%
\pgfsetbuttcap%
\pgfsetroundjoin%
\definecolor{currentfill}{rgb}{0.283229,0.120777,0.440584}%
\pgfsetfillcolor{currentfill}%
\pgfsetfillopacity{0.700000}%
\pgfsetlinewidth{0.000000pt}%
\definecolor{currentstroke}{rgb}{0.000000,0.000000,0.000000}%
\pgfsetstrokecolor{currentstroke}%
\pgfsetdash{}{0pt}%
\pgfpathmoveto{\pgfqpoint{4.159029in}{2.341847in}}%
\pgfpathlineto{\pgfqpoint{4.172661in}{2.339459in}}%
\pgfpathlineto{\pgfqpoint{4.186300in}{2.337148in}}%
\pgfpathlineto{\pgfqpoint{4.199945in}{2.334914in}}%
\pgfpathlineto{\pgfqpoint{4.213598in}{2.332758in}}%
\pgfpathlineto{\pgfqpoint{4.221412in}{2.341586in}}%
\pgfpathlineto{\pgfqpoint{4.229221in}{2.350455in}}%
\pgfpathlineto{\pgfqpoint{4.237024in}{2.359366in}}%
\pgfpathlineto{\pgfqpoint{4.244822in}{2.368325in}}%
\pgfpathlineto{\pgfqpoint{4.231180in}{2.370662in}}%
\pgfpathlineto{\pgfqpoint{4.217546in}{2.373076in}}%
\pgfpathlineto{\pgfqpoint{4.203918in}{2.375567in}}%
\pgfpathlineto{\pgfqpoint{4.190298in}{2.378135in}}%
\pgfpathlineto{\pgfqpoint{4.182489in}{2.368989in}}%
\pgfpathlineto{\pgfqpoint{4.174674in}{2.359895in}}%
\pgfpathlineto{\pgfqpoint{4.166854in}{2.350848in}}%
\pgfpathlineto{\pgfqpoint{4.159029in}{2.341847in}}%
\pgfpathclose%
\pgfusepath{fill}%
\end{pgfscope}%
\begin{pgfscope}%
\pgfpathrectangle{\pgfqpoint{1.150000in}{0.150000in}}{\pgfqpoint{5.700000in}{5.700000in}}%
\pgfusepath{clip}%
\pgfsetbuttcap%
\pgfsetroundjoin%
\definecolor{currentfill}{rgb}{0.195860,0.395433,0.555276}%
\pgfsetfillcolor{currentfill}%
\pgfsetfillopacity{0.700000}%
\pgfsetlinewidth{0.000000pt}%
\definecolor{currentstroke}{rgb}{0.000000,0.000000,0.000000}%
\pgfsetstrokecolor{currentstroke}%
\pgfsetdash{}{0pt}%
\pgfpathmoveto{\pgfqpoint{6.007536in}{2.952147in}}%
\pgfpathlineto{\pgfqpoint{6.021609in}{2.949271in}}%
\pgfpathlineto{\pgfqpoint{6.035691in}{2.946459in}}%
\pgfpathlineto{\pgfqpoint{6.049782in}{2.943710in}}%
\pgfpathlineto{\pgfqpoint{6.063883in}{2.941026in}}%
\pgfpathlineto{\pgfqpoint{6.071283in}{2.955922in}}%
\pgfpathlineto{\pgfqpoint{6.078696in}{2.971252in}}%
\pgfpathlineto{\pgfqpoint{6.086123in}{2.987026in}}%
\pgfpathlineto{\pgfqpoint{6.093564in}{3.003255in}}%
\pgfpathlineto{\pgfqpoint{6.079490in}{3.006502in}}%
\pgfpathlineto{\pgfqpoint{6.065425in}{3.009814in}}%
\pgfpathlineto{\pgfqpoint{6.051369in}{3.013189in}}%
\pgfpathlineto{\pgfqpoint{6.037322in}{3.016628in}}%
\pgfpathlineto{\pgfqpoint{6.029855in}{2.999829in}}%
\pgfpathlineto{\pgfqpoint{6.022402in}{2.983489in}}%
\pgfpathlineto{\pgfqpoint{6.014962in}{2.967598in}}%
\pgfpathlineto{\pgfqpoint{6.007536in}{2.952147in}}%
\pgfpathclose%
\pgfusepath{fill}%
\end{pgfscope}%
\begin{pgfscope}%
\pgfpathrectangle{\pgfqpoint{1.150000in}{0.150000in}}{\pgfqpoint{5.700000in}{5.700000in}}%
\pgfusepath{clip}%
\pgfsetbuttcap%
\pgfsetroundjoin%
\definecolor{currentfill}{rgb}{0.258965,0.251537,0.524736}%
\pgfsetfillcolor{currentfill}%
\pgfsetfillopacity{0.700000}%
\pgfsetlinewidth{0.000000pt}%
\definecolor{currentstroke}{rgb}{0.000000,0.000000,0.000000}%
\pgfsetstrokecolor{currentstroke}%
\pgfsetdash{}{0pt}%
\pgfpathmoveto{\pgfqpoint{5.180821in}{2.611224in}}%
\pgfpathlineto{\pgfqpoint{5.194731in}{2.609880in}}%
\pgfpathlineto{\pgfqpoint{5.208651in}{2.608604in}}%
\pgfpathlineto{\pgfqpoint{5.222579in}{2.607396in}}%
\pgfpathlineto{\pgfqpoint{5.236517in}{2.606256in}}%
\pgfpathlineto{\pgfqpoint{5.243980in}{2.615317in}}%
\pgfpathlineto{\pgfqpoint{5.251443in}{2.624556in}}%
\pgfpathlineto{\pgfqpoint{5.258904in}{2.633980in}}%
\pgfpathlineto{\pgfqpoint{5.266366in}{2.643596in}}%
\pgfpathlineto{\pgfqpoint{5.252447in}{2.645119in}}%
\pgfpathlineto{\pgfqpoint{5.238538in}{2.646711in}}%
\pgfpathlineto{\pgfqpoint{5.224637in}{2.648370in}}%
\pgfpathlineto{\pgfqpoint{5.210746in}{2.650097in}}%
\pgfpathlineto{\pgfqpoint{5.203266in}{2.640091in}}%
\pgfpathlineto{\pgfqpoint{5.195785in}{2.630281in}}%
\pgfpathlineto{\pgfqpoint{5.188303in}{2.620661in}}%
\pgfpathlineto{\pgfqpoint{5.180821in}{2.611224in}}%
\pgfpathclose%
\pgfusepath{fill}%
\end{pgfscope}%
\begin{pgfscope}%
\pgfpathrectangle{\pgfqpoint{1.150000in}{0.150000in}}{\pgfqpoint{5.700000in}{5.700000in}}%
\pgfusepath{clip}%
\pgfsetbuttcap%
\pgfsetroundjoin%
\definecolor{currentfill}{rgb}{0.279566,0.067836,0.391917}%
\pgfsetfillcolor{currentfill}%
\pgfsetfillopacity{0.700000}%
\pgfsetlinewidth{0.000000pt}%
\definecolor{currentstroke}{rgb}{0.000000,0.000000,0.000000}%
\pgfsetstrokecolor{currentstroke}%
\pgfsetdash{}{0pt}%
\pgfpathmoveto{\pgfqpoint{3.113971in}{2.253863in}}%
\pgfpathlineto{\pgfqpoint{3.127416in}{2.247260in}}%
\pgfpathlineto{\pgfqpoint{3.140865in}{2.240759in}}%
\pgfpathlineto{\pgfqpoint{3.154317in}{2.234358in}}%
\pgfpathlineto{\pgfqpoint{3.167771in}{2.228058in}}%
\pgfpathlineto{\pgfqpoint{3.175950in}{2.236888in}}%
\pgfpathlineto{\pgfqpoint{3.184123in}{2.245761in}}%
\pgfpathlineto{\pgfqpoint{3.192290in}{2.254677in}}%
\pgfpathlineto{\pgfqpoint{3.200449in}{2.263638in}}%
\pgfpathlineto{\pgfqpoint{3.187008in}{2.269935in}}%
\pgfpathlineto{\pgfqpoint{3.173569in}{2.276333in}}%
\pgfpathlineto{\pgfqpoint{3.160133in}{2.282831in}}%
\pgfpathlineto{\pgfqpoint{3.146700in}{2.289430in}}%
\pgfpathlineto{\pgfqpoint{3.138528in}{2.280465in}}%
\pgfpathlineto{\pgfqpoint{3.130349in}{2.271550in}}%
\pgfpathlineto{\pgfqpoint{3.122163in}{2.262682in}}%
\pgfpathlineto{\pgfqpoint{3.113971in}{2.253863in}}%
\pgfpathclose%
\pgfusepath{fill}%
\end{pgfscope}%
\begin{pgfscope}%
\pgfpathrectangle{\pgfqpoint{1.150000in}{0.150000in}}{\pgfqpoint{5.700000in}{5.700000in}}%
\pgfusepath{clip}%
\pgfsetbuttcap%
\pgfsetroundjoin%
\definecolor{currentfill}{rgb}{0.275191,0.194905,0.496005}%
\pgfsetfillcolor{currentfill}%
\pgfsetfillopacity{0.700000}%
\pgfsetlinewidth{0.000000pt}%
\definecolor{currentstroke}{rgb}{0.000000,0.000000,0.000000}%
\pgfsetstrokecolor{currentstroke}%
\pgfsetdash{}{0pt}%
\pgfpathmoveto{\pgfqpoint{4.783050in}{2.494654in}}%
\pgfpathlineto{\pgfqpoint{4.796853in}{2.493250in}}%
\pgfpathlineto{\pgfqpoint{4.810665in}{2.491917in}}%
\pgfpathlineto{\pgfqpoint{4.824485in}{2.490655in}}%
\pgfpathlineto{\pgfqpoint{4.838314in}{2.489463in}}%
\pgfpathlineto{\pgfqpoint{4.845907in}{2.498012in}}%
\pgfpathlineto{\pgfqpoint{4.853496in}{2.506665in}}%
\pgfpathlineto{\pgfqpoint{4.861081in}{2.515430in}}%
\pgfpathlineto{\pgfqpoint{4.868664in}{2.524311in}}%
\pgfpathlineto{\pgfqpoint{4.854850in}{2.525805in}}%
\pgfpathlineto{\pgfqpoint{4.841046in}{2.527370in}}%
\pgfpathlineto{\pgfqpoint{4.827249in}{2.529005in}}%
\pgfpathlineto{\pgfqpoint{4.813461in}{2.530712in}}%
\pgfpathlineto{\pgfqpoint{4.805864in}{2.521521in}}%
\pgfpathlineto{\pgfqpoint{4.798263in}{2.512451in}}%
\pgfpathlineto{\pgfqpoint{4.790658in}{2.503497in}}%
\pgfpathlineto{\pgfqpoint{4.783050in}{2.494654in}}%
\pgfpathclose%
\pgfusepath{fill}%
\end{pgfscope}%
\begin{pgfscope}%
\pgfpathrectangle{\pgfqpoint{1.150000in}{0.150000in}}{\pgfqpoint{5.700000in}{5.700000in}}%
\pgfusepath{clip}%
\pgfsetbuttcap%
\pgfsetroundjoin%
\definecolor{currentfill}{rgb}{0.227802,0.326594,0.546532}%
\pgfsetfillcolor{currentfill}%
\pgfsetfillopacity{0.700000}%
\pgfsetlinewidth{0.000000pt}%
\definecolor{currentstroke}{rgb}{0.000000,0.000000,0.000000}%
\pgfsetstrokecolor{currentstroke}%
\pgfsetdash{}{0pt}%
\pgfpathmoveto{\pgfqpoint{5.664508in}{2.778522in}}%
\pgfpathlineto{\pgfqpoint{5.678533in}{2.776663in}}%
\pgfpathlineto{\pgfqpoint{5.692567in}{2.774870in}}%
\pgfpathlineto{\pgfqpoint{5.706611in}{2.773143in}}%
\pgfpathlineto{\pgfqpoint{5.720664in}{2.771481in}}%
\pgfpathlineto{\pgfqpoint{5.728028in}{2.782657in}}%
\pgfpathlineto{\pgfqpoint{5.735398in}{2.794136in}}%
\pgfpathlineto{\pgfqpoint{5.742773in}{2.805926in}}%
\pgfpathlineto{\pgfqpoint{5.750154in}{2.818037in}}%
\pgfpathlineto{\pgfqpoint{5.736125in}{2.820183in}}%
\pgfpathlineto{\pgfqpoint{5.722106in}{2.822394in}}%
\pgfpathlineto{\pgfqpoint{5.708095in}{2.824671in}}%
\pgfpathlineto{\pgfqpoint{5.694094in}{2.827013in}}%
\pgfpathlineto{\pgfqpoint{5.686689in}{2.814411in}}%
\pgfpathlineto{\pgfqpoint{5.679289in}{2.802134in}}%
\pgfpathlineto{\pgfqpoint{5.671896in}{2.790174in}}%
\pgfpathlineto{\pgfqpoint{5.664508in}{2.778522in}}%
\pgfpathclose%
\pgfusepath{fill}%
\end{pgfscope}%
\begin{pgfscope}%
\pgfpathrectangle{\pgfqpoint{1.150000in}{0.150000in}}{\pgfqpoint{5.700000in}{5.700000in}}%
\pgfusepath{clip}%
\pgfsetbuttcap%
\pgfsetroundjoin%
\definecolor{currentfill}{rgb}{0.280894,0.078907,0.402329}%
\pgfsetfillcolor{currentfill}%
\pgfsetfillopacity{0.700000}%
\pgfsetlinewidth{0.000000pt}%
\definecolor{currentstroke}{rgb}{0.000000,0.000000,0.000000}%
\pgfsetstrokecolor{currentstroke}%
\pgfsetdash{}{0pt}%
\pgfpathmoveto{\pgfqpoint{2.973548in}{2.275803in}}%
\pgfpathlineto{\pgfqpoint{2.986988in}{2.268341in}}%
\pgfpathlineto{\pgfqpoint{3.000431in}{2.260986in}}%
\pgfpathlineto{\pgfqpoint{3.013875in}{2.253737in}}%
\pgfpathlineto{\pgfqpoint{3.027322in}{2.246593in}}%
\pgfpathlineto{\pgfqpoint{3.035556in}{2.255192in}}%
\pgfpathlineto{\pgfqpoint{3.043782in}{2.263842in}}%
\pgfpathlineto{\pgfqpoint{3.052001in}{2.272545in}}%
\pgfpathlineto{\pgfqpoint{3.060214in}{2.281301in}}%
\pgfpathlineto{\pgfqpoint{3.046781in}{2.288421in}}%
\pgfpathlineto{\pgfqpoint{3.033350in}{2.295646in}}%
\pgfpathlineto{\pgfqpoint{3.019921in}{2.302977in}}%
\pgfpathlineto{\pgfqpoint{3.006495in}{2.310416in}}%
\pgfpathlineto{\pgfqpoint{2.998268in}{2.301676in}}%
\pgfpathlineto{\pgfqpoint{2.990035in}{2.292995in}}%
\pgfpathlineto{\pgfqpoint{2.981795in}{2.284370in}}%
\pgfpathlineto{\pgfqpoint{2.973548in}{2.275803in}}%
\pgfpathclose%
\pgfusepath{fill}%
\end{pgfscope}%
\begin{pgfscope}%
\pgfpathrectangle{\pgfqpoint{1.150000in}{0.150000in}}{\pgfqpoint{5.700000in}{5.700000in}}%
\pgfusepath{clip}%
\pgfsetbuttcap%
\pgfsetroundjoin%
\definecolor{currentfill}{rgb}{0.278791,0.062145,0.386592}%
\pgfsetfillcolor{currentfill}%
\pgfsetfillopacity{0.700000}%
\pgfsetlinewidth{0.000000pt}%
\definecolor{currentstroke}{rgb}{0.000000,0.000000,0.000000}%
\pgfsetstrokecolor{currentstroke}%
\pgfsetdash{}{0pt}%
\pgfpathmoveto{\pgfqpoint{3.254249in}{2.239437in}}%
\pgfpathlineto{\pgfqpoint{3.267707in}{2.233630in}}%
\pgfpathlineto{\pgfqpoint{3.281169in}{2.227921in}}%
\pgfpathlineto{\pgfqpoint{3.294635in}{2.222307in}}%
\pgfpathlineto{\pgfqpoint{3.308104in}{2.216788in}}%
\pgfpathlineto{\pgfqpoint{3.316233in}{2.225774in}}%
\pgfpathlineto{\pgfqpoint{3.324356in}{2.234794in}}%
\pgfpathlineto{\pgfqpoint{3.332472in}{2.243851in}}%
\pgfpathlineto{\pgfqpoint{3.340583in}{2.252944in}}%
\pgfpathlineto{\pgfqpoint{3.327125in}{2.258480in}}%
\pgfpathlineto{\pgfqpoint{3.313671in}{2.264111in}}%
\pgfpathlineto{\pgfqpoint{3.300221in}{2.269838in}}%
\pgfpathlineto{\pgfqpoint{3.286775in}{2.275661in}}%
\pgfpathlineto{\pgfqpoint{3.278653in}{2.266544in}}%
\pgfpathlineto{\pgfqpoint{3.270525in}{2.257467in}}%
\pgfpathlineto{\pgfqpoint{3.262390in}{2.248432in}}%
\pgfpathlineto{\pgfqpoint{3.254249in}{2.239437in}}%
\pgfpathclose%
\pgfusepath{fill}%
\end{pgfscope}%
\begin{pgfscope}%
\pgfpathrectangle{\pgfqpoint{1.150000in}{0.150000in}}{\pgfqpoint{5.700000in}{5.700000in}}%
\pgfusepath{clip}%
\pgfsetbuttcap%
\pgfsetroundjoin%
\definecolor{currentfill}{rgb}{0.280267,0.073417,0.397163}%
\pgfsetfillcolor{currentfill}%
\pgfsetfillopacity{0.700000}%
\pgfsetlinewidth{0.000000pt}%
\definecolor{currentstroke}{rgb}{0.000000,0.000000,0.000000}%
\pgfsetstrokecolor{currentstroke}%
\pgfsetdash{}{0pt}%
\pgfpathmoveto{\pgfqpoint{3.620791in}{2.249576in}}%
\pgfpathlineto{\pgfqpoint{3.634304in}{2.245488in}}%
\pgfpathlineto{\pgfqpoint{3.647822in}{2.241487in}}%
\pgfpathlineto{\pgfqpoint{3.661345in}{2.237572in}}%
\pgfpathlineto{\pgfqpoint{3.674873in}{2.233744in}}%
\pgfpathlineto{\pgfqpoint{3.682874in}{2.242864in}}%
\pgfpathlineto{\pgfqpoint{3.690870in}{2.252009in}}%
\pgfpathlineto{\pgfqpoint{3.698859in}{2.261180in}}%
\pgfpathlineto{\pgfqpoint{3.706843in}{2.270379in}}%
\pgfpathlineto{\pgfqpoint{3.693325in}{2.274286in}}%
\pgfpathlineto{\pgfqpoint{3.679813in}{2.278279in}}%
\pgfpathlineto{\pgfqpoint{3.666306in}{2.282359in}}%
\pgfpathlineto{\pgfqpoint{3.652804in}{2.286525in}}%
\pgfpathlineto{\pgfqpoint{3.644809in}{2.277240in}}%
\pgfpathlineto{\pgfqpoint{3.636809in}{2.267988in}}%
\pgfpathlineto{\pgfqpoint{3.628803in}{2.258767in}}%
\pgfpathlineto{\pgfqpoint{3.620791in}{2.249576in}}%
\pgfpathclose%
\pgfusepath{fill}%
\end{pgfscope}%
\begin{pgfscope}%
\pgfpathrectangle{\pgfqpoint{1.150000in}{0.150000in}}{\pgfqpoint{5.700000in}{5.700000in}}%
\pgfusepath{clip}%
\pgfsetbuttcap%
\pgfsetroundjoin%
\definecolor{currentfill}{rgb}{0.281924,0.089666,0.412415}%
\pgfsetfillcolor{currentfill}%
\pgfsetfillopacity{0.700000}%
\pgfsetlinewidth{0.000000pt}%
\definecolor{currentstroke}{rgb}{0.000000,0.000000,0.000000}%
\pgfsetstrokecolor{currentstroke}%
\pgfsetdash{}{0pt}%
\pgfpathmoveto{\pgfqpoint{3.846984in}{2.278493in}}%
\pgfpathlineto{\pgfqpoint{3.860543in}{2.275244in}}%
\pgfpathlineto{\pgfqpoint{3.874108in}{2.272076in}}%
\pgfpathlineto{\pgfqpoint{3.887679in}{2.268991in}}%
\pgfpathlineto{\pgfqpoint{3.901257in}{2.265988in}}%
\pgfpathlineto{\pgfqpoint{3.909181in}{2.275022in}}%
\pgfpathlineto{\pgfqpoint{3.917099in}{2.284082in}}%
\pgfpathlineto{\pgfqpoint{3.925011in}{2.293170in}}%
\pgfpathlineto{\pgfqpoint{3.932918in}{2.302288in}}%
\pgfpathlineto{\pgfqpoint{3.919351in}{2.305410in}}%
\pgfpathlineto{\pgfqpoint{3.905791in}{2.308615in}}%
\pgfpathlineto{\pgfqpoint{3.892236in}{2.311901in}}%
\pgfpathlineto{\pgfqpoint{3.878687in}{2.315270in}}%
\pgfpathlineto{\pgfqpoint{3.870770in}{2.306025in}}%
\pgfpathlineto{\pgfqpoint{3.862847in}{2.296815in}}%
\pgfpathlineto{\pgfqpoint{3.854918in}{2.287639in}}%
\pgfpathlineto{\pgfqpoint{3.846984in}{2.278493in}}%
\pgfpathclose%
\pgfusepath{fill}%
\end{pgfscope}%
\begin{pgfscope}%
\pgfpathrectangle{\pgfqpoint{1.150000in}{0.150000in}}{\pgfqpoint{5.700000in}{5.700000in}}%
\pgfusepath{clip}%
\pgfsetbuttcap%
\pgfsetroundjoin%
\definecolor{currentfill}{rgb}{0.187231,0.414746,0.556547}%
\pgfsetfillcolor{currentfill}%
\pgfsetfillopacity{0.700000}%
\pgfsetlinewidth{0.000000pt}%
\definecolor{currentstroke}{rgb}{0.000000,0.000000,0.000000}%
\pgfsetstrokecolor{currentstroke}%
\pgfsetdash{}{0pt}%
\pgfpathmoveto{\pgfqpoint{6.093564in}{3.003255in}}%
\pgfpathlineto{\pgfqpoint{6.107647in}{3.000071in}}%
\pgfpathlineto{\pgfqpoint{6.121739in}{2.996950in}}%
\pgfpathlineto{\pgfqpoint{6.135840in}{2.993894in}}%
\pgfpathlineto{\pgfqpoint{6.149950in}{2.990901in}}%
\pgfpathlineto{\pgfqpoint{6.157379in}{3.007018in}}%
\pgfpathlineto{\pgfqpoint{6.164823in}{3.023607in}}%
\pgfpathlineto{\pgfqpoint{6.172283in}{3.040677in}}%
\pgfpathlineto{\pgfqpoint{6.158194in}{3.044106in}}%
\pgfpathlineto{\pgfqpoint{6.144113in}{3.047599in}}%
\pgfpathlineto{\pgfqpoint{6.130041in}{3.051155in}}%
\pgfpathlineto{\pgfqpoint{6.115978in}{3.054774in}}%
\pgfpathlineto{\pgfqpoint{6.108491in}{3.037118in}}%
\pgfpathlineto{\pgfqpoint{6.101020in}{3.019948in}}%
\pgfpathlineto{\pgfqpoint{6.093564in}{3.003255in}}%
\pgfpathclose%
\pgfusepath{fill}%
\end{pgfscope}%
\begin{pgfscope}%
\pgfpathrectangle{\pgfqpoint{1.150000in}{0.150000in}}{\pgfqpoint{5.700000in}{5.700000in}}%
\pgfusepath{clip}%
\pgfsetbuttcap%
\pgfsetroundjoin%
\definecolor{currentfill}{rgb}{0.282884,0.135920,0.453427}%
\pgfsetfillcolor{currentfill}%
\pgfsetfillopacity{0.700000}%
\pgfsetlinewidth{0.000000pt}%
\definecolor{currentstroke}{rgb}{0.000000,0.000000,0.000000}%
\pgfsetstrokecolor{currentstroke}%
\pgfsetdash{}{0pt}%
\pgfpathmoveto{\pgfqpoint{2.638109in}{2.384520in}}%
\pgfpathlineto{\pgfqpoint{2.651566in}{2.374631in}}%
\pgfpathlineto{\pgfqpoint{2.665023in}{2.364866in}}%
\pgfpathlineto{\pgfqpoint{2.678479in}{2.355225in}}%
\pgfpathlineto{\pgfqpoint{2.691935in}{2.345707in}}%
\pgfpathlineto{\pgfqpoint{2.700308in}{2.353517in}}%
\pgfpathlineto{\pgfqpoint{2.708672in}{2.361407in}}%
\pgfpathlineto{\pgfqpoint{2.717027in}{2.369377in}}%
\pgfpathlineto{\pgfqpoint{2.725375in}{2.377426in}}%
\pgfpathlineto{\pgfqpoint{2.711935in}{2.386878in}}%
\pgfpathlineto{\pgfqpoint{2.698496in}{2.396453in}}%
\pgfpathlineto{\pgfqpoint{2.685057in}{2.406151in}}%
\pgfpathlineto{\pgfqpoint{2.671617in}{2.415975in}}%
\pgfpathlineto{\pgfqpoint{2.663253in}{2.407984in}}%
\pgfpathlineto{\pgfqpoint{2.654880in}{2.400078in}}%
\pgfpathlineto{\pgfqpoint{2.646499in}{2.392257in}}%
\pgfpathlineto{\pgfqpoint{2.638109in}{2.384520in}}%
\pgfpathclose%
\pgfusepath{fill}%
\end{pgfscope}%
\begin{pgfscope}%
\pgfpathrectangle{\pgfqpoint{1.150000in}{0.150000in}}{\pgfqpoint{5.700000in}{5.700000in}}%
\pgfusepath{clip}%
\pgfsetbuttcap%
\pgfsetroundjoin%
\definecolor{currentfill}{rgb}{0.235526,0.309527,0.542944}%
\pgfsetfillcolor{currentfill}%
\pgfsetfillopacity{0.700000}%
\pgfsetlinewidth{0.000000pt}%
\definecolor{currentstroke}{rgb}{0.000000,0.000000,0.000000}%
\pgfsetstrokecolor{currentstroke}%
\pgfsetdash{}{0pt}%
\pgfpathmoveto{\pgfqpoint{5.578898in}{2.741055in}}%
\pgfpathlineto{\pgfqpoint{5.592909in}{2.739397in}}%
\pgfpathlineto{\pgfqpoint{5.606929in}{2.737806in}}%
\pgfpathlineto{\pgfqpoint{5.620959in}{2.736280in}}%
\pgfpathlineto{\pgfqpoint{5.634998in}{2.734820in}}%
\pgfpathlineto{\pgfqpoint{5.642370in}{2.745326in}}%
\pgfpathlineto{\pgfqpoint{5.649745in}{2.756106in}}%
\pgfpathlineto{\pgfqpoint{5.657124in}{2.767168in}}%
\pgfpathlineto{\pgfqpoint{5.664508in}{2.778522in}}%
\pgfpathlineto{\pgfqpoint{5.650492in}{2.780446in}}%
\pgfpathlineto{\pgfqpoint{5.636485in}{2.782436in}}%
\pgfpathlineto{\pgfqpoint{5.622488in}{2.784491in}}%
\pgfpathlineto{\pgfqpoint{5.608500in}{2.786612in}}%
\pgfpathlineto{\pgfqpoint{5.601093in}{2.774788in}}%
\pgfpathlineto{\pgfqpoint{5.593691in}{2.763259in}}%
\pgfpathlineto{\pgfqpoint{5.586293in}{2.752018in}}%
\pgfpathlineto{\pgfqpoint{5.578898in}{2.741055in}}%
\pgfpathclose%
\pgfusepath{fill}%
\end{pgfscope}%
\begin{pgfscope}%
\pgfpathrectangle{\pgfqpoint{1.150000in}{0.150000in}}{\pgfqpoint{5.700000in}{5.700000in}}%
\pgfusepath{clip}%
\pgfsetbuttcap%
\pgfsetroundjoin%
\definecolor{currentfill}{rgb}{0.263663,0.237631,0.518762}%
\pgfsetfillcolor{currentfill}%
\pgfsetfillopacity{0.700000}%
\pgfsetlinewidth{0.000000pt}%
\definecolor{currentstroke}{rgb}{0.000000,0.000000,0.000000}%
\pgfsetstrokecolor{currentstroke}%
\pgfsetdash{}{0pt}%
\pgfpathmoveto{\pgfqpoint{5.095250in}{2.579784in}}%
\pgfpathlineto{\pgfqpoint{5.109143in}{2.578529in}}%
\pgfpathlineto{\pgfqpoint{5.123045in}{2.577343in}}%
\pgfpathlineto{\pgfqpoint{5.136956in}{2.576225in}}%
\pgfpathlineto{\pgfqpoint{5.150876in}{2.575177in}}%
\pgfpathlineto{\pgfqpoint{5.158365in}{2.583947in}}%
\pgfpathlineto{\pgfqpoint{5.165852in}{2.592874in}}%
\pgfpathlineto{\pgfqpoint{5.173337in}{2.601964in}}%
\pgfpathlineto{\pgfqpoint{5.180821in}{2.611224in}}%
\pgfpathlineto{\pgfqpoint{5.166919in}{2.612637in}}%
\pgfpathlineto{\pgfqpoint{5.153027in}{2.614118in}}%
\pgfpathlineto{\pgfqpoint{5.139143in}{2.615667in}}%
\pgfpathlineto{\pgfqpoint{5.125268in}{2.617285in}}%
\pgfpathlineto{\pgfqpoint{5.117766in}{2.607654in}}%
\pgfpathlineto{\pgfqpoint{5.110262in}{2.598198in}}%
\pgfpathlineto{\pgfqpoint{5.102757in}{2.588910in}}%
\pgfpathlineto{\pgfqpoint{5.095250in}{2.579784in}}%
\pgfpathclose%
\pgfusepath{fill}%
\end{pgfscope}%
\begin{pgfscope}%
\pgfpathrectangle{\pgfqpoint{1.150000in}{0.150000in}}{\pgfqpoint{5.700000in}{5.700000in}}%
\pgfusepath{clip}%
\pgfsetbuttcap%
\pgfsetroundjoin%
\definecolor{currentfill}{rgb}{0.282290,0.145912,0.461510}%
\pgfsetfillcolor{currentfill}%
\pgfsetfillopacity{0.700000}%
\pgfsetlinewidth{0.000000pt}%
\definecolor{currentstroke}{rgb}{0.000000,0.000000,0.000000}%
\pgfsetstrokecolor{currentstroke}%
\pgfsetdash{}{0pt}%
\pgfpathmoveto{\pgfqpoint{4.385263in}{2.387146in}}%
\pgfpathlineto{\pgfqpoint{4.398959in}{2.385291in}}%
\pgfpathlineto{\pgfqpoint{4.412663in}{2.383511in}}%
\pgfpathlineto{\pgfqpoint{4.426375in}{2.381806in}}%
\pgfpathlineto{\pgfqpoint{4.440095in}{2.380175in}}%
\pgfpathlineto{\pgfqpoint{4.447830in}{2.388783in}}%
\pgfpathlineto{\pgfqpoint{4.455561in}{2.397446in}}%
\pgfpathlineto{\pgfqpoint{4.463287in}{2.406167in}}%
\pgfpathlineto{\pgfqpoint{4.471008in}{2.414952in}}%
\pgfpathlineto{\pgfqpoint{4.457301in}{2.416804in}}%
\pgfpathlineto{\pgfqpoint{4.443602in}{2.418731in}}%
\pgfpathlineto{\pgfqpoint{4.429910in}{2.420732in}}%
\pgfpathlineto{\pgfqpoint{4.416226in}{2.422807in}}%
\pgfpathlineto{\pgfqpoint{4.408493in}{2.413794in}}%
\pgfpathlineto{\pgfqpoint{4.400754in}{2.404849in}}%
\pgfpathlineto{\pgfqpoint{4.393011in}{2.395968in}}%
\pgfpathlineto{\pgfqpoint{4.385263in}{2.387146in}}%
\pgfpathclose%
\pgfusepath{fill}%
\end{pgfscope}%
\begin{pgfscope}%
\pgfpathrectangle{\pgfqpoint{1.150000in}{0.150000in}}{\pgfqpoint{5.700000in}{5.700000in}}%
\pgfusepath{clip}%
\pgfsetbuttcap%
\pgfsetroundjoin%
\definecolor{currentfill}{rgb}{0.282656,0.100196,0.422160}%
\pgfsetfillcolor{currentfill}%
\pgfsetfillopacity{0.700000}%
\pgfsetlinewidth{0.000000pt}%
\definecolor{currentstroke}{rgb}{0.000000,0.000000,0.000000}%
\pgfsetstrokecolor{currentstroke}%
\pgfsetdash{}{0pt}%
\pgfpathmoveto{\pgfqpoint{2.832899in}{2.306108in}}%
\pgfpathlineto{\pgfqpoint{2.846343in}{2.297717in}}%
\pgfpathlineto{\pgfqpoint{2.859788in}{2.289440in}}%
\pgfpathlineto{\pgfqpoint{2.873234in}{2.281275in}}%
\pgfpathlineto{\pgfqpoint{2.886681in}{2.273221in}}%
\pgfpathlineto{\pgfqpoint{2.894973in}{2.281506in}}%
\pgfpathlineto{\pgfqpoint{2.903257in}{2.289854in}}%
\pgfpathlineto{\pgfqpoint{2.911534in}{2.298264in}}%
\pgfpathlineto{\pgfqpoint{2.919803in}{2.306737in}}%
\pgfpathlineto{\pgfqpoint{2.906371in}{2.314746in}}%
\pgfpathlineto{\pgfqpoint{2.892940in}{2.322866in}}%
\pgfpathlineto{\pgfqpoint{2.879510in}{2.331099in}}%
\pgfpathlineto{\pgfqpoint{2.866082in}{2.339446in}}%
\pgfpathlineto{\pgfqpoint{2.857798in}{2.331010in}}%
\pgfpathlineto{\pgfqpoint{2.849506in}{2.322641in}}%
\pgfpathlineto{\pgfqpoint{2.841207in}{2.314341in}}%
\pgfpathlineto{\pgfqpoint{2.832899in}{2.306108in}}%
\pgfpathclose%
\pgfusepath{fill}%
\end{pgfscope}%
\begin{pgfscope}%
\pgfpathrectangle{\pgfqpoint{1.150000in}{0.150000in}}{\pgfqpoint{5.700000in}{5.700000in}}%
\pgfusepath{clip}%
\pgfsetbuttcap%
\pgfsetroundjoin%
\definecolor{currentfill}{rgb}{0.278791,0.062145,0.386592}%
\pgfsetfillcolor{currentfill}%
\pgfsetfillopacity{0.700000}%
\pgfsetlinewidth{0.000000pt}%
\definecolor{currentstroke}{rgb}{0.000000,0.000000,0.000000}%
\pgfsetstrokecolor{currentstroke}%
\pgfsetdash{}{0pt}%
\pgfpathmoveto{\pgfqpoint{3.394453in}{2.231743in}}%
\pgfpathlineto{\pgfqpoint{3.407930in}{2.226676in}}%
\pgfpathlineto{\pgfqpoint{3.421412in}{2.221701in}}%
\pgfpathlineto{\pgfqpoint{3.434899in}{2.216818in}}%
\pgfpathlineto{\pgfqpoint{3.448390in}{2.212026in}}%
\pgfpathlineto{\pgfqpoint{3.456471in}{2.221096in}}%
\pgfpathlineto{\pgfqpoint{3.464546in}{2.230195in}}%
\pgfpathlineto{\pgfqpoint{3.472615in}{2.239323in}}%
\pgfpathlineto{\pgfqpoint{3.480678in}{2.248482in}}%
\pgfpathlineto{\pgfqpoint{3.467198in}{2.253312in}}%
\pgfpathlineto{\pgfqpoint{3.453723in}{2.258233in}}%
\pgfpathlineto{\pgfqpoint{3.440253in}{2.263246in}}%
\pgfpathlineto{\pgfqpoint{3.426786in}{2.268351in}}%
\pgfpathlineto{\pgfqpoint{3.418712in}{2.259146in}}%
\pgfpathlineto{\pgfqpoint{3.410631in}{2.249978in}}%
\pgfpathlineto{\pgfqpoint{3.402545in}{2.240844in}}%
\pgfpathlineto{\pgfqpoint{3.394453in}{2.231743in}}%
\pgfpathclose%
\pgfusepath{fill}%
\end{pgfscope}%
\begin{pgfscope}%
\pgfpathrectangle{\pgfqpoint{1.150000in}{0.150000in}}{\pgfqpoint{5.700000in}{5.700000in}}%
\pgfusepath{clip}%
\pgfsetbuttcap%
\pgfsetroundjoin%
\definecolor{currentfill}{rgb}{0.283091,0.110553,0.431554}%
\pgfsetfillcolor{currentfill}%
\pgfsetfillopacity{0.700000}%
\pgfsetlinewidth{0.000000pt}%
\definecolor{currentstroke}{rgb}{0.000000,0.000000,0.000000}%
\pgfsetstrokecolor{currentstroke}%
\pgfsetdash{}{0pt}%
\pgfpathmoveto{\pgfqpoint{4.073173in}{2.315921in}}%
\pgfpathlineto{\pgfqpoint{4.086788in}{2.313380in}}%
\pgfpathlineto{\pgfqpoint{4.100410in}{2.310917in}}%
\pgfpathlineto{\pgfqpoint{4.114039in}{2.308533in}}%
\pgfpathlineto{\pgfqpoint{4.127675in}{2.306227in}}%
\pgfpathlineto{\pgfqpoint{4.135521in}{2.315080in}}%
\pgfpathlineto{\pgfqpoint{4.143363in}{2.323966in}}%
\pgfpathlineto{\pgfqpoint{4.151199in}{2.332887in}}%
\pgfpathlineto{\pgfqpoint{4.159029in}{2.341847in}}%
\pgfpathlineto{\pgfqpoint{4.145405in}{2.344313in}}%
\pgfpathlineto{\pgfqpoint{4.131787in}{2.346858in}}%
\pgfpathlineto{\pgfqpoint{4.118176in}{2.349481in}}%
\pgfpathlineto{\pgfqpoint{4.104572in}{2.352182in}}%
\pgfpathlineto{\pgfqpoint{4.096730in}{2.343055in}}%
\pgfpathlineto{\pgfqpoint{4.088883in}{2.333971in}}%
\pgfpathlineto{\pgfqpoint{4.081031in}{2.324927in}}%
\pgfpathlineto{\pgfqpoint{4.073173in}{2.315921in}}%
\pgfpathclose%
\pgfusepath{fill}%
\end{pgfscope}%
\begin{pgfscope}%
\pgfpathrectangle{\pgfqpoint{1.150000in}{0.150000in}}{\pgfqpoint{5.700000in}{5.700000in}}%
\pgfusepath{clip}%
\pgfsetbuttcap%
\pgfsetroundjoin%
\definecolor{currentfill}{rgb}{0.277134,0.185228,0.489898}%
\pgfsetfillcolor{currentfill}%
\pgfsetfillopacity{0.700000}%
\pgfsetlinewidth{0.000000pt}%
\definecolor{currentstroke}{rgb}{0.000000,0.000000,0.000000}%
\pgfsetstrokecolor{currentstroke}%
\pgfsetdash{}{0pt}%
\pgfpathmoveto{\pgfqpoint{4.697389in}{2.465478in}}%
\pgfpathlineto{\pgfqpoint{4.711174in}{2.464070in}}%
\pgfpathlineto{\pgfqpoint{4.724966in}{2.462735in}}%
\pgfpathlineto{\pgfqpoint{4.738768in}{2.461470in}}%
\pgfpathlineto{\pgfqpoint{4.752577in}{2.460278in}}%
\pgfpathlineto{\pgfqpoint{4.760202in}{2.468732in}}%
\pgfpathlineto{\pgfqpoint{4.767822in}{2.477277in}}%
\pgfpathlineto{\pgfqpoint{4.775438in}{2.485915in}}%
\pgfpathlineto{\pgfqpoint{4.783050in}{2.494654in}}%
\pgfpathlineto{\pgfqpoint{4.769255in}{2.496129in}}%
\pgfpathlineto{\pgfqpoint{4.755469in}{2.497676in}}%
\pgfpathlineto{\pgfqpoint{4.741691in}{2.499294in}}%
\pgfpathlineto{\pgfqpoint{4.727921in}{2.500983in}}%
\pgfpathlineto{\pgfqpoint{4.720294in}{2.491955in}}%
\pgfpathlineto{\pgfqpoint{4.712663in}{2.483032in}}%
\pgfpathlineto{\pgfqpoint{4.705028in}{2.474207in}}%
\pgfpathlineto{\pgfqpoint{4.697389in}{2.465478in}}%
\pgfpathclose%
\pgfusepath{fill}%
\end{pgfscope}%
\begin{pgfscope}%
\pgfpathrectangle{\pgfqpoint{1.150000in}{0.150000in}}{\pgfqpoint{5.700000in}{5.700000in}}%
\pgfusepath{clip}%
\pgfsetbuttcap%
\pgfsetroundjoin%
\definecolor{currentfill}{rgb}{0.241237,0.296485,0.539709}%
\pgfsetfillcolor{currentfill}%
\pgfsetfillopacity{0.700000}%
\pgfsetlinewidth{0.000000pt}%
\definecolor{currentstroke}{rgb}{0.000000,0.000000,0.000000}%
\pgfsetstrokecolor{currentstroke}%
\pgfsetdash{}{0pt}%
\pgfpathmoveto{\pgfqpoint{5.493308in}{2.705349in}}%
\pgfpathlineto{\pgfqpoint{5.507304in}{2.703870in}}%
\pgfpathlineto{\pgfqpoint{5.521310in}{2.702458in}}%
\pgfpathlineto{\pgfqpoint{5.535325in}{2.701112in}}%
\pgfpathlineto{\pgfqpoint{5.549349in}{2.699832in}}%
\pgfpathlineto{\pgfqpoint{5.556732in}{2.709760in}}%
\pgfpathlineto{\pgfqpoint{5.564118in}{2.719934in}}%
\pgfpathlineto{\pgfqpoint{5.571507in}{2.730363in}}%
\pgfpathlineto{\pgfqpoint{5.578898in}{2.741055in}}%
\pgfpathlineto{\pgfqpoint{5.564896in}{2.742779in}}%
\pgfpathlineto{\pgfqpoint{5.550904in}{2.744569in}}%
\pgfpathlineto{\pgfqpoint{5.536921in}{2.746425in}}%
\pgfpathlineto{\pgfqpoint{5.522947in}{2.748347in}}%
\pgfpathlineto{\pgfqpoint{5.515533in}{2.737204in}}%
\pgfpathlineto{\pgfqpoint{5.508122in}{2.726329in}}%
\pgfpathlineto{\pgfqpoint{5.500714in}{2.715713in}}%
\pgfpathlineto{\pgfqpoint{5.493308in}{2.705349in}}%
\pgfpathclose%
\pgfusepath{fill}%
\end{pgfscope}%
\begin{pgfscope}%
\pgfpathrectangle{\pgfqpoint{1.150000in}{0.150000in}}{\pgfqpoint{5.700000in}{5.700000in}}%
\pgfusepath{clip}%
\pgfsetbuttcap%
\pgfsetroundjoin%
\definecolor{currentfill}{rgb}{0.277134,0.185228,0.489898}%
\pgfsetfillcolor{currentfill}%
\pgfsetfillopacity{0.700000}%
\pgfsetlinewidth{0.000000pt}%
\definecolor{currentstroke}{rgb}{0.000000,0.000000,0.000000}%
\pgfsetstrokecolor{currentstroke}%
\pgfsetdash{}{0pt}%
\pgfpathmoveto{\pgfqpoint{2.442666in}{2.484301in}}%
\pgfpathlineto{\pgfqpoint{2.456158in}{2.472739in}}%
\pgfpathlineto{\pgfqpoint{2.469648in}{2.461315in}}%
\pgfpathlineto{\pgfqpoint{2.483136in}{2.450028in}}%
\pgfpathlineto{\pgfqpoint{2.496622in}{2.438878in}}%
\pgfpathlineto{\pgfqpoint{2.505086in}{2.446076in}}%
\pgfpathlineto{\pgfqpoint{2.513541in}{2.453373in}}%
\pgfpathlineto{\pgfqpoint{2.521986in}{2.460768in}}%
\pgfpathlineto{\pgfqpoint{2.530421in}{2.468260in}}%
\pgfpathlineto{\pgfqpoint{2.516954in}{2.479322in}}%
\pgfpathlineto{\pgfqpoint{2.503486in}{2.490521in}}%
\pgfpathlineto{\pgfqpoint{2.490016in}{2.501856in}}%
\pgfpathlineto{\pgfqpoint{2.476543in}{2.513331in}}%
\pgfpathlineto{\pgfqpoint{2.468088in}{2.505920in}}%
\pgfpathlineto{\pgfqpoint{2.459624in}{2.498610in}}%
\pgfpathlineto{\pgfqpoint{2.451150in}{2.491404in}}%
\pgfpathlineto{\pgfqpoint{2.442666in}{2.484301in}}%
\pgfpathclose%
\pgfusepath{fill}%
\end{pgfscope}%
\begin{pgfscope}%
\pgfpathrectangle{\pgfqpoint{1.150000in}{0.150000in}}{\pgfqpoint{5.700000in}{5.700000in}}%
\pgfusepath{clip}%
\pgfsetbuttcap%
\pgfsetroundjoin%
\definecolor{currentfill}{rgb}{0.266580,0.228262,0.514349}%
\pgfsetfillcolor{currentfill}%
\pgfsetfillopacity{0.700000}%
\pgfsetlinewidth{0.000000pt}%
\definecolor{currentstroke}{rgb}{0.000000,0.000000,0.000000}%
\pgfsetstrokecolor{currentstroke}%
\pgfsetdash{}{0pt}%
\pgfpathmoveto{\pgfqpoint{5.009646in}{2.549104in}}%
\pgfpathlineto{\pgfqpoint{5.023521in}{2.547916in}}%
\pgfpathlineto{\pgfqpoint{5.037405in}{2.546798in}}%
\pgfpathlineto{\pgfqpoint{5.051297in}{2.545749in}}%
\pgfpathlineto{\pgfqpoint{5.065199in}{2.544769in}}%
\pgfpathlineto{\pgfqpoint{5.072716in}{2.553312in}}%
\pgfpathlineto{\pgfqpoint{5.080229in}{2.561991in}}%
\pgfpathlineto{\pgfqpoint{5.087741in}{2.570813in}}%
\pgfpathlineto{\pgfqpoint{5.095250in}{2.579784in}}%
\pgfpathlineto{\pgfqpoint{5.081366in}{2.581107in}}%
\pgfpathlineto{\pgfqpoint{5.067491in}{2.582500in}}%
\pgfpathlineto{\pgfqpoint{5.053624in}{2.583961in}}%
\pgfpathlineto{\pgfqpoint{5.039767in}{2.585492in}}%
\pgfpathlineto{\pgfqpoint{5.032240in}{2.576170in}}%
\pgfpathlineto{\pgfqpoint{5.024711in}{2.567002in}}%
\pgfpathlineto{\pgfqpoint{5.017180in}{2.557982in}}%
\pgfpathlineto{\pgfqpoint{5.009646in}{2.549104in}}%
\pgfpathclose%
\pgfusepath{fill}%
\end{pgfscope}%
\begin{pgfscope}%
\pgfpathrectangle{\pgfqpoint{1.150000in}{0.150000in}}{\pgfqpoint{5.700000in}{5.700000in}}%
\pgfusepath{clip}%
\pgfsetbuttcap%
\pgfsetroundjoin%
\definecolor{currentfill}{rgb}{0.246811,0.283237,0.535941}%
\pgfsetfillcolor{currentfill}%
\pgfsetfillopacity{0.700000}%
\pgfsetlinewidth{0.000000pt}%
\definecolor{currentstroke}{rgb}{0.000000,0.000000,0.000000}%
\pgfsetstrokecolor{currentstroke}%
\pgfsetdash{}{0pt}%
\pgfpathmoveto{\pgfqpoint{5.407723in}{2.671137in}}%
\pgfpathlineto{\pgfqpoint{5.421704in}{2.669815in}}%
\pgfpathlineto{\pgfqpoint{5.435693in}{2.668561in}}%
\pgfpathlineto{\pgfqpoint{5.449692in}{2.667373in}}%
\pgfpathlineto{\pgfqpoint{5.463701in}{2.666253in}}%
\pgfpathlineto{\pgfqpoint{5.471101in}{2.675687in}}%
\pgfpathlineto{\pgfqpoint{5.478502in}{2.685343in}}%
\pgfpathlineto{\pgfqpoint{5.485904in}{2.695228in}}%
\pgfpathlineto{\pgfqpoint{5.493308in}{2.705349in}}%
\pgfpathlineto{\pgfqpoint{5.479322in}{2.706894in}}%
\pgfpathlineto{\pgfqpoint{5.465344in}{2.708505in}}%
\pgfpathlineto{\pgfqpoint{5.451376in}{2.710184in}}%
\pgfpathlineto{\pgfqpoint{5.437416in}{2.711929in}}%
\pgfpathlineto{\pgfqpoint{5.429991in}{2.701377in}}%
\pgfpathlineto{\pgfqpoint{5.422567in}{2.691066in}}%
\pgfpathlineto{\pgfqpoint{5.415145in}{2.680988in}}%
\pgfpathlineto{\pgfqpoint{5.407723in}{2.671137in}}%
\pgfpathclose%
\pgfusepath{fill}%
\end{pgfscope}%
\begin{pgfscope}%
\pgfpathrectangle{\pgfqpoint{1.150000in}{0.150000in}}{\pgfqpoint{5.700000in}{5.700000in}}%
\pgfusepath{clip}%
\pgfsetbuttcap%
\pgfsetroundjoin%
\definecolor{currentfill}{rgb}{0.280894,0.078907,0.402329}%
\pgfsetfillcolor{currentfill}%
\pgfsetfillopacity{0.700000}%
\pgfsetlinewidth{0.000000pt}%
\definecolor{currentstroke}{rgb}{0.000000,0.000000,0.000000}%
\pgfsetstrokecolor{currentstroke}%
\pgfsetdash{}{0pt}%
\pgfpathmoveto{\pgfqpoint{3.760970in}{2.255602in}}%
\pgfpathlineto{\pgfqpoint{3.774516in}{2.252119in}}%
\pgfpathlineto{\pgfqpoint{3.788068in}{2.248719in}}%
\pgfpathlineto{\pgfqpoint{3.801626in}{2.245403in}}%
\pgfpathlineto{\pgfqpoint{3.815190in}{2.242170in}}%
\pgfpathlineto{\pgfqpoint{3.823147in}{2.251216in}}%
\pgfpathlineto{\pgfqpoint{3.831098in}{2.260284in}}%
\pgfpathlineto{\pgfqpoint{3.839044in}{2.269375in}}%
\pgfpathlineto{\pgfqpoint{3.846984in}{2.278493in}}%
\pgfpathlineto{\pgfqpoint{3.833430in}{2.281825in}}%
\pgfpathlineto{\pgfqpoint{3.819883in}{2.285240in}}%
\pgfpathlineto{\pgfqpoint{3.806342in}{2.288739in}}%
\pgfpathlineto{\pgfqpoint{3.792807in}{2.292321in}}%
\pgfpathlineto{\pgfqpoint{3.784856in}{2.283097in}}%
\pgfpathlineto{\pgfqpoint{3.776900in}{2.273903in}}%
\pgfpathlineto{\pgfqpoint{3.768938in}{2.264739in}}%
\pgfpathlineto{\pgfqpoint{3.760970in}{2.255602in}}%
\pgfpathclose%
\pgfusepath{fill}%
\end{pgfscope}%
\begin{pgfscope}%
\pgfpathrectangle{\pgfqpoint{1.150000in}{0.150000in}}{\pgfqpoint{5.700000in}{5.700000in}}%
\pgfusepath{clip}%
\pgfsetbuttcap%
\pgfsetroundjoin%
\definecolor{currentfill}{rgb}{0.279566,0.067836,0.391917}%
\pgfsetfillcolor{currentfill}%
\pgfsetfillopacity{0.700000}%
\pgfsetlinewidth{0.000000pt}%
\definecolor{currentstroke}{rgb}{0.000000,0.000000,0.000000}%
\pgfsetstrokecolor{currentstroke}%
\pgfsetdash{}{0pt}%
\pgfpathmoveto{\pgfqpoint{3.534643in}{2.230066in}}%
\pgfpathlineto{\pgfqpoint{3.548146in}{2.225685in}}%
\pgfpathlineto{\pgfqpoint{3.561654in}{2.221392in}}%
\pgfpathlineto{\pgfqpoint{3.575167in}{2.217188in}}%
\pgfpathlineto{\pgfqpoint{3.588686in}{2.213072in}}%
\pgfpathlineto{\pgfqpoint{3.596721in}{2.222161in}}%
\pgfpathlineto{\pgfqpoint{3.604750in}{2.231274in}}%
\pgfpathlineto{\pgfqpoint{3.612774in}{2.240412in}}%
\pgfpathlineto{\pgfqpoint{3.620791in}{2.249576in}}%
\pgfpathlineto{\pgfqpoint{3.607284in}{2.253751in}}%
\pgfpathlineto{\pgfqpoint{3.593782in}{2.258013in}}%
\pgfpathlineto{\pgfqpoint{3.580284in}{2.262364in}}%
\pgfpathlineto{\pgfqpoint{3.566792in}{2.266803in}}%
\pgfpathlineto{\pgfqpoint{3.558764in}{2.257573in}}%
\pgfpathlineto{\pgfqpoint{3.550729in}{2.248375in}}%
\pgfpathlineto{\pgfqpoint{3.542689in}{2.239206in}}%
\pgfpathlineto{\pgfqpoint{3.534643in}{2.230066in}}%
\pgfpathclose%
\pgfusepath{fill}%
\end{pgfscope}%
\begin{pgfscope}%
\pgfpathrectangle{\pgfqpoint{1.150000in}{0.150000in}}{\pgfqpoint{5.700000in}{5.700000in}}%
\pgfusepath{clip}%
\pgfsetbuttcap%
\pgfsetroundjoin%
\definecolor{currentfill}{rgb}{0.282884,0.135920,0.453427}%
\pgfsetfillcolor{currentfill}%
\pgfsetfillopacity{0.700000}%
\pgfsetlinewidth{0.000000pt}%
\definecolor{currentstroke}{rgb}{0.000000,0.000000,0.000000}%
\pgfsetstrokecolor{currentstroke}%
\pgfsetdash{}{0pt}%
\pgfpathmoveto{\pgfqpoint{4.299460in}{2.359743in}}%
\pgfpathlineto{\pgfqpoint{4.313139in}{2.357788in}}%
\pgfpathlineto{\pgfqpoint{4.326824in}{2.355909in}}%
\pgfpathlineto{\pgfqpoint{4.340518in}{2.354105in}}%
\pgfpathlineto{\pgfqpoint{4.354218in}{2.352377in}}%
\pgfpathlineto{\pgfqpoint{4.361987in}{2.360999in}}%
\pgfpathlineto{\pgfqpoint{4.369751in}{2.369666in}}%
\pgfpathlineto{\pgfqpoint{4.377509in}{2.378380in}}%
\pgfpathlineto{\pgfqpoint{4.385263in}{2.387146in}}%
\pgfpathlineto{\pgfqpoint{4.371574in}{2.389076in}}%
\pgfpathlineto{\pgfqpoint{4.357893in}{2.391081in}}%
\pgfpathlineto{\pgfqpoint{4.344219in}{2.393161in}}%
\pgfpathlineto{\pgfqpoint{4.330553in}{2.395317in}}%
\pgfpathlineto{\pgfqpoint{4.322788in}{2.386342in}}%
\pgfpathlineto{\pgfqpoint{4.315017in}{2.377424in}}%
\pgfpathlineto{\pgfqpoint{4.307241in}{2.368559in}}%
\pgfpathlineto{\pgfqpoint{4.299460in}{2.359743in}}%
\pgfpathclose%
\pgfusepath{fill}%
\end{pgfscope}%
\begin{pgfscope}%
\pgfpathrectangle{\pgfqpoint{1.150000in}{0.150000in}}{\pgfqpoint{5.700000in}{5.700000in}}%
\pgfusepath{clip}%
\pgfsetbuttcap%
\pgfsetroundjoin%
\definecolor{currentfill}{rgb}{0.283229,0.120777,0.440584}%
\pgfsetfillcolor{currentfill}%
\pgfsetfillopacity{0.700000}%
\pgfsetlinewidth{0.000000pt}%
\definecolor{currentstroke}{rgb}{0.000000,0.000000,0.000000}%
\pgfsetstrokecolor{currentstroke}%
\pgfsetdash{}{0pt}%
\pgfpathmoveto{\pgfqpoint{2.691935in}{2.345707in}}%
\pgfpathlineto{\pgfqpoint{2.705391in}{2.336311in}}%
\pgfpathlineto{\pgfqpoint{2.718848in}{2.327035in}}%
\pgfpathlineto{\pgfqpoint{2.732304in}{2.317879in}}%
\pgfpathlineto{\pgfqpoint{2.745761in}{2.308843in}}%
\pgfpathlineto{\pgfqpoint{2.754116in}{2.316726in}}%
\pgfpathlineto{\pgfqpoint{2.762463in}{2.324684in}}%
\pgfpathlineto{\pgfqpoint{2.770802in}{2.332717in}}%
\pgfpathlineto{\pgfqpoint{2.779133in}{2.340824in}}%
\pgfpathlineto{\pgfqpoint{2.765693in}{2.349795in}}%
\pgfpathlineto{\pgfqpoint{2.752253in}{2.358885in}}%
\pgfpathlineto{\pgfqpoint{2.738814in}{2.368095in}}%
\pgfpathlineto{\pgfqpoint{2.725375in}{2.377426in}}%
\pgfpathlineto{\pgfqpoint{2.717027in}{2.369377in}}%
\pgfpathlineto{\pgfqpoint{2.708672in}{2.361407in}}%
\pgfpathlineto{\pgfqpoint{2.700308in}{2.353517in}}%
\pgfpathlineto{\pgfqpoint{2.691935in}{2.345707in}}%
\pgfpathclose%
\pgfusepath{fill}%
\end{pgfscope}%
\begin{pgfscope}%
\pgfpathrectangle{\pgfqpoint{1.150000in}{0.150000in}}{\pgfqpoint{5.700000in}{5.700000in}}%
\pgfusepath{clip}%
\pgfsetbuttcap%
\pgfsetroundjoin%
\definecolor{currentfill}{rgb}{0.278826,0.175490,0.483397}%
\pgfsetfillcolor{currentfill}%
\pgfsetfillopacity{0.700000}%
\pgfsetlinewidth{0.000000pt}%
\definecolor{currentstroke}{rgb}{0.000000,0.000000,0.000000}%
\pgfsetstrokecolor{currentstroke}%
\pgfsetdash{}{0pt}%
\pgfpathmoveto{\pgfqpoint{4.611678in}{2.436705in}}%
\pgfpathlineto{\pgfqpoint{4.625443in}{2.435271in}}%
\pgfpathlineto{\pgfqpoint{4.639217in}{2.433910in}}%
\pgfpathlineto{\pgfqpoint{4.652999in}{2.432621in}}%
\pgfpathlineto{\pgfqpoint{4.666790in}{2.431404in}}%
\pgfpathlineto{\pgfqpoint{4.674446in}{2.439805in}}%
\pgfpathlineto{\pgfqpoint{4.682098in}{2.448281in}}%
\pgfpathlineto{\pgfqpoint{4.689746in}{2.456837in}}%
\pgfpathlineto{\pgfqpoint{4.697389in}{2.465478in}}%
\pgfpathlineto{\pgfqpoint{4.683613in}{2.466957in}}%
\pgfpathlineto{\pgfqpoint{4.669845in}{2.468508in}}%
\pgfpathlineto{\pgfqpoint{4.656085in}{2.470132in}}%
\pgfpathlineto{\pgfqpoint{4.642334in}{2.471828in}}%
\pgfpathlineto{\pgfqpoint{4.634677in}{2.462918in}}%
\pgfpathlineto{\pgfqpoint{4.627015in}{2.454097in}}%
\pgfpathlineto{\pgfqpoint{4.619349in}{2.445361in}}%
\pgfpathlineto{\pgfqpoint{4.611678in}{2.436705in}}%
\pgfpathclose%
\pgfusepath{fill}%
\end{pgfscope}%
\begin{pgfscope}%
\pgfpathrectangle{\pgfqpoint{1.150000in}{0.150000in}}{\pgfqpoint{5.700000in}{5.700000in}}%
\pgfusepath{clip}%
\pgfsetbuttcap%
\pgfsetroundjoin%
\definecolor{currentfill}{rgb}{0.282656,0.100196,0.422160}%
\pgfsetfillcolor{currentfill}%
\pgfsetfillopacity{0.700000}%
\pgfsetlinewidth{0.000000pt}%
\definecolor{currentstroke}{rgb}{0.000000,0.000000,0.000000}%
\pgfsetstrokecolor{currentstroke}%
\pgfsetdash{}{0pt}%
\pgfpathmoveto{\pgfqpoint{3.987250in}{2.290607in}}%
\pgfpathlineto{\pgfqpoint{4.000849in}{2.287888in}}%
\pgfpathlineto{\pgfqpoint{4.014455in}{2.285249in}}%
\pgfpathlineto{\pgfqpoint{4.028068in}{2.282689in}}%
\pgfpathlineto{\pgfqpoint{4.041687in}{2.280209in}}%
\pgfpathlineto{\pgfqpoint{4.049567in}{2.289095in}}%
\pgfpathlineto{\pgfqpoint{4.057441in}{2.298008in}}%
\pgfpathlineto{\pgfqpoint{4.065310in}{2.306949in}}%
\pgfpathlineto{\pgfqpoint{4.073173in}{2.315921in}}%
\pgfpathlineto{\pgfqpoint{4.059565in}{2.318541in}}%
\pgfpathlineto{\pgfqpoint{4.045963in}{2.321241in}}%
\pgfpathlineto{\pgfqpoint{4.032368in}{2.324020in}}%
\pgfpathlineto{\pgfqpoint{4.018780in}{2.326879in}}%
\pgfpathlineto{\pgfqpoint{4.010906in}{2.317759in}}%
\pgfpathlineto{\pgfqpoint{4.003026in}{2.308676in}}%
\pgfpathlineto{\pgfqpoint{3.995141in}{2.299626in}}%
\pgfpathlineto{\pgfqpoint{3.987250in}{2.290607in}}%
\pgfpathclose%
\pgfusepath{fill}%
\end{pgfscope}%
\begin{pgfscope}%
\pgfpathrectangle{\pgfqpoint{1.150000in}{0.150000in}}{\pgfqpoint{5.700000in}{5.700000in}}%
\pgfusepath{clip}%
\pgfsetbuttcap%
\pgfsetroundjoin%
\definecolor{currentfill}{rgb}{0.280267,0.073417,0.397163}%
\pgfsetfillcolor{currentfill}%
\pgfsetfillopacity{0.700000}%
\pgfsetlinewidth{0.000000pt}%
\definecolor{currentstroke}{rgb}{0.000000,0.000000,0.000000}%
\pgfsetstrokecolor{currentstroke}%
\pgfsetdash{}{0pt}%
\pgfpathmoveto{\pgfqpoint{3.027322in}{2.246593in}}%
\pgfpathlineto{\pgfqpoint{3.040771in}{2.239555in}}%
\pgfpathlineto{\pgfqpoint{3.054223in}{2.232620in}}%
\pgfpathlineto{\pgfqpoint{3.067677in}{2.225789in}}%
\pgfpathlineto{\pgfqpoint{3.081133in}{2.219061in}}%
\pgfpathlineto{\pgfqpoint{3.089353in}{2.227690in}}%
\pgfpathlineto{\pgfqpoint{3.097566in}{2.236367in}}%
\pgfpathlineto{\pgfqpoint{3.105772in}{2.245091in}}%
\pgfpathlineto{\pgfqpoint{3.113971in}{2.253863in}}%
\pgfpathlineto{\pgfqpoint{3.100528in}{2.260568in}}%
\pgfpathlineto{\pgfqpoint{3.087087in}{2.267375in}}%
\pgfpathlineto{\pgfqpoint{3.073649in}{2.274286in}}%
\pgfpathlineto{\pgfqpoint{3.060214in}{2.281301in}}%
\pgfpathlineto{\pgfqpoint{3.052001in}{2.272545in}}%
\pgfpathlineto{\pgfqpoint{3.043782in}{2.263842in}}%
\pgfpathlineto{\pgfqpoint{3.035556in}{2.255192in}}%
\pgfpathlineto{\pgfqpoint{3.027322in}{2.246593in}}%
\pgfpathclose%
\pgfusepath{fill}%
\end{pgfscope}%
\begin{pgfscope}%
\pgfpathrectangle{\pgfqpoint{1.150000in}{0.150000in}}{\pgfqpoint{5.700000in}{5.700000in}}%
\pgfusepath{clip}%
\pgfsetbuttcap%
\pgfsetroundjoin%
\definecolor{currentfill}{rgb}{0.278791,0.062145,0.386592}%
\pgfsetfillcolor{currentfill}%
\pgfsetfillopacity{0.700000}%
\pgfsetlinewidth{0.000000pt}%
\definecolor{currentstroke}{rgb}{0.000000,0.000000,0.000000}%
\pgfsetstrokecolor{currentstroke}%
\pgfsetdash{}{0pt}%
\pgfpathmoveto{\pgfqpoint{3.167771in}{2.228058in}}%
\pgfpathlineto{\pgfqpoint{3.181229in}{2.221857in}}%
\pgfpathlineto{\pgfqpoint{3.194689in}{2.215755in}}%
\pgfpathlineto{\pgfqpoint{3.208153in}{2.209751in}}%
\pgfpathlineto{\pgfqpoint{3.221621in}{2.203845in}}%
\pgfpathlineto{\pgfqpoint{3.229788in}{2.212685in}}%
\pgfpathlineto{\pgfqpoint{3.237948in}{2.221564in}}%
\pgfpathlineto{\pgfqpoint{3.246101in}{2.230481in}}%
\pgfpathlineto{\pgfqpoint{3.254249in}{2.239437in}}%
\pgfpathlineto{\pgfqpoint{3.240794in}{2.245340in}}%
\pgfpathlineto{\pgfqpoint{3.227342in}{2.251341in}}%
\pgfpathlineto{\pgfqpoint{3.213894in}{2.257440in}}%
\pgfpathlineto{\pgfqpoint{3.200449in}{2.263638in}}%
\pgfpathlineto{\pgfqpoint{3.192290in}{2.254677in}}%
\pgfpathlineto{\pgfqpoint{3.184123in}{2.245761in}}%
\pgfpathlineto{\pgfqpoint{3.175950in}{2.236888in}}%
\pgfpathlineto{\pgfqpoint{3.167771in}{2.228058in}}%
\pgfpathclose%
\pgfusepath{fill}%
\end{pgfscope}%
\begin{pgfscope}%
\pgfpathrectangle{\pgfqpoint{1.150000in}{0.150000in}}{\pgfqpoint{5.700000in}{5.700000in}}%
\pgfusepath{clip}%
\pgfsetbuttcap%
\pgfsetroundjoin%
\definecolor{currentfill}{rgb}{0.252194,0.269783,0.531579}%
\pgfsetfillcolor{currentfill}%
\pgfsetfillopacity{0.700000}%
\pgfsetlinewidth{0.000000pt}%
\definecolor{currentstroke}{rgb}{0.000000,0.000000,0.000000}%
\pgfsetstrokecolor{currentstroke}%
\pgfsetdash{}{0pt}%
\pgfpathmoveto{\pgfqpoint{5.322130in}{2.638178in}}%
\pgfpathlineto{\pgfqpoint{5.336094in}{2.636993in}}%
\pgfpathlineto{\pgfqpoint{5.350067in}{2.635874in}}%
\pgfpathlineto{\pgfqpoint{5.364050in}{2.634823in}}%
\pgfpathlineto{\pgfqpoint{5.378042in}{2.633840in}}%
\pgfpathlineto{\pgfqpoint{5.385462in}{2.642862in}}%
\pgfpathlineto{\pgfqpoint{5.392882in}{2.652080in}}%
\pgfpathlineto{\pgfqpoint{5.400302in}{2.661503in}}%
\pgfpathlineto{\pgfqpoint{5.407723in}{2.671137in}}%
\pgfpathlineto{\pgfqpoint{5.393752in}{2.672525in}}%
\pgfpathlineto{\pgfqpoint{5.379790in}{2.673980in}}%
\pgfpathlineto{\pgfqpoint{5.365837in}{2.675502in}}%
\pgfpathlineto{\pgfqpoint{5.351894in}{2.677092in}}%
\pgfpathlineto{\pgfqpoint{5.344452in}{2.667047in}}%
\pgfpathlineto{\pgfqpoint{5.337011in}{2.657218in}}%
\pgfpathlineto{\pgfqpoint{5.329571in}{2.647597in}}%
\pgfpathlineto{\pgfqpoint{5.322130in}{2.638178in}}%
\pgfpathclose%
\pgfusepath{fill}%
\end{pgfscope}%
\begin{pgfscope}%
\pgfpathrectangle{\pgfqpoint{1.150000in}{0.150000in}}{\pgfqpoint{5.700000in}{5.700000in}}%
\pgfusepath{clip}%
\pgfsetbuttcap%
\pgfsetroundjoin%
\definecolor{currentfill}{rgb}{0.270595,0.214069,0.507052}%
\pgfsetfillcolor{currentfill}%
\pgfsetfillopacity{0.700000}%
\pgfsetlinewidth{0.000000pt}%
\definecolor{currentstroke}{rgb}{0.000000,0.000000,0.000000}%
\pgfsetstrokecolor{currentstroke}%
\pgfsetdash{}{0pt}%
\pgfpathmoveto{\pgfqpoint{4.924002in}{2.519038in}}%
\pgfpathlineto{\pgfqpoint{4.937859in}{2.517896in}}%
\pgfpathlineto{\pgfqpoint{4.951724in}{2.516823in}}%
\pgfpathlineto{\pgfqpoint{4.965598in}{2.515820in}}%
\pgfpathlineto{\pgfqpoint{4.979481in}{2.514886in}}%
\pgfpathlineto{\pgfqpoint{4.987027in}{2.523258in}}%
\pgfpathlineto{\pgfqpoint{4.994570in}{2.531748in}}%
\pgfpathlineto{\pgfqpoint{5.002109in}{2.540361in}}%
\pgfpathlineto{\pgfqpoint{5.009646in}{2.549104in}}%
\pgfpathlineto{\pgfqpoint{4.995780in}{2.550361in}}%
\pgfpathlineto{\pgfqpoint{4.981923in}{2.551687in}}%
\pgfpathlineto{\pgfqpoint{4.968074in}{2.553083in}}%
\pgfpathlineto{\pgfqpoint{4.954234in}{2.554549in}}%
\pgfpathlineto{\pgfqpoint{4.946681in}{2.545476in}}%
\pgfpathlineto{\pgfqpoint{4.939124in}{2.536537in}}%
\pgfpathlineto{\pgfqpoint{4.931565in}{2.527726in}}%
\pgfpathlineto{\pgfqpoint{4.924002in}{2.519038in}}%
\pgfpathclose%
\pgfusepath{fill}%
\end{pgfscope}%
\begin{pgfscope}%
\pgfpathrectangle{\pgfqpoint{1.150000in}{0.150000in}}{\pgfqpoint{5.700000in}{5.700000in}}%
\pgfusepath{clip}%
\pgfsetbuttcap%
\pgfsetroundjoin%
\definecolor{currentfill}{rgb}{0.280255,0.165693,0.476498}%
\pgfsetfillcolor{currentfill}%
\pgfsetfillopacity{0.700000}%
\pgfsetlinewidth{0.000000pt}%
\definecolor{currentstroke}{rgb}{0.000000,0.000000,0.000000}%
\pgfsetstrokecolor{currentstroke}%
\pgfsetdash{}{0pt}%
\pgfpathmoveto{\pgfqpoint{2.496622in}{2.438878in}}%
\pgfpathlineto{\pgfqpoint{2.510107in}{2.427863in}}%
\pgfpathlineto{\pgfqpoint{2.523590in}{2.416982in}}%
\pgfpathlineto{\pgfqpoint{2.537071in}{2.406233in}}%
\pgfpathlineto{\pgfqpoint{2.550551in}{2.395616in}}%
\pgfpathlineto{\pgfqpoint{2.558996in}{2.402910in}}%
\pgfpathlineto{\pgfqpoint{2.567431in}{2.410297in}}%
\pgfpathlineto{\pgfqpoint{2.575857in}{2.417776in}}%
\pgfpathlineto{\pgfqpoint{2.584274in}{2.425348in}}%
\pgfpathlineto{\pgfqpoint{2.570812in}{2.435877in}}%
\pgfpathlineto{\pgfqpoint{2.557350in}{2.446539in}}%
\pgfpathlineto{\pgfqpoint{2.543886in}{2.457332in}}%
\pgfpathlineto{\pgfqpoint{2.530421in}{2.468260in}}%
\pgfpathlineto{\pgfqpoint{2.521986in}{2.460768in}}%
\pgfpathlineto{\pgfqpoint{2.513541in}{2.453373in}}%
\pgfpathlineto{\pgfqpoint{2.505086in}{2.446076in}}%
\pgfpathlineto{\pgfqpoint{2.496622in}{2.438878in}}%
\pgfpathclose%
\pgfusepath{fill}%
\end{pgfscope}%
\begin{pgfscope}%
\pgfpathrectangle{\pgfqpoint{1.150000in}{0.150000in}}{\pgfqpoint{5.700000in}{5.700000in}}%
\pgfusepath{clip}%
\pgfsetbuttcap%
\pgfsetroundjoin%
\definecolor{currentfill}{rgb}{0.281446,0.084320,0.407414}%
\pgfsetfillcolor{currentfill}%
\pgfsetfillopacity{0.700000}%
\pgfsetlinewidth{0.000000pt}%
\definecolor{currentstroke}{rgb}{0.000000,0.000000,0.000000}%
\pgfsetstrokecolor{currentstroke}%
\pgfsetdash{}{0pt}%
\pgfpathmoveto{\pgfqpoint{2.886681in}{2.273221in}}%
\pgfpathlineto{\pgfqpoint{2.900130in}{2.265279in}}%
\pgfpathlineto{\pgfqpoint{2.913580in}{2.257447in}}%
\pgfpathlineto{\pgfqpoint{2.927032in}{2.249724in}}%
\pgfpathlineto{\pgfqpoint{2.940485in}{2.242110in}}%
\pgfpathlineto{\pgfqpoint{2.948762in}{2.250447in}}%
\pgfpathlineto{\pgfqpoint{2.957031in}{2.258841in}}%
\pgfpathlineto{\pgfqpoint{2.965293in}{2.267294in}}%
\pgfpathlineto{\pgfqpoint{2.973548in}{2.275803in}}%
\pgfpathlineto{\pgfqpoint{2.960109in}{2.283373in}}%
\pgfpathlineto{\pgfqpoint{2.946672in}{2.291052in}}%
\pgfpathlineto{\pgfqpoint{2.933237in}{2.298839in}}%
\pgfpathlineto{\pgfqpoint{2.919803in}{2.306737in}}%
\pgfpathlineto{\pgfqpoint{2.911534in}{2.298264in}}%
\pgfpathlineto{\pgfqpoint{2.903257in}{2.289854in}}%
\pgfpathlineto{\pgfqpoint{2.894973in}{2.281506in}}%
\pgfpathlineto{\pgfqpoint{2.886681in}{2.273221in}}%
\pgfpathclose%
\pgfusepath{fill}%
\end{pgfscope}%
\begin{pgfscope}%
\pgfpathrectangle{\pgfqpoint{1.150000in}{0.150000in}}{\pgfqpoint{5.700000in}{5.700000in}}%
\pgfusepath{clip}%
\pgfsetbuttcap%
\pgfsetroundjoin%
\definecolor{currentfill}{rgb}{0.203063,0.379716,0.553925}%
\pgfsetfillcolor{currentfill}%
\pgfsetfillopacity{0.700000}%
\pgfsetlinewidth{0.000000pt}%
\definecolor{currentstroke}{rgb}{0.000000,0.000000,0.000000}%
\pgfsetstrokecolor{currentstroke}%
\pgfsetdash{}{0pt}%
\pgfpathmoveto{\pgfqpoint{5.977949in}{2.894525in}}%
\pgfpathlineto{\pgfqpoint{5.992048in}{2.892193in}}%
\pgfpathlineto{\pgfqpoint{6.006157in}{2.889925in}}%
\pgfpathlineto{\pgfqpoint{6.020276in}{2.887721in}}%
\pgfpathlineto{\pgfqpoint{6.034404in}{2.885581in}}%
\pgfpathlineto{\pgfqpoint{6.041757in}{2.898842in}}%
\pgfpathlineto{\pgfqpoint{6.049121in}{2.912496in}}%
\pgfpathlineto{\pgfqpoint{6.056496in}{2.926554in}}%
\pgfpathlineto{\pgfqpoint{6.063883in}{2.941026in}}%
\pgfpathlineto{\pgfqpoint{6.049782in}{2.943710in}}%
\pgfpathlineto{\pgfqpoint{6.035691in}{2.946459in}}%
\pgfpathlineto{\pgfqpoint{6.021609in}{2.949271in}}%
\pgfpathlineto{\pgfqpoint{6.007536in}{2.952147in}}%
\pgfpathlineto{\pgfqpoint{6.000122in}{2.937123in}}%
\pgfpathlineto{\pgfqpoint{5.992719in}{2.922519in}}%
\pgfpathlineto{\pgfqpoint{5.985329in}{2.908323in}}%
\pgfpathlineto{\pgfqpoint{5.977949in}{2.894525in}}%
\pgfpathclose%
\pgfusepath{fill}%
\end{pgfscope}%
\begin{pgfscope}%
\pgfpathrectangle{\pgfqpoint{1.150000in}{0.150000in}}{\pgfqpoint{5.700000in}{5.700000in}}%
\pgfusepath{clip}%
\pgfsetbuttcap%
\pgfsetroundjoin%
\definecolor{currentfill}{rgb}{0.278791,0.062145,0.386592}%
\pgfsetfillcolor{currentfill}%
\pgfsetfillopacity{0.700000}%
\pgfsetlinewidth{0.000000pt}%
\definecolor{currentstroke}{rgb}{0.000000,0.000000,0.000000}%
\pgfsetstrokecolor{currentstroke}%
\pgfsetdash{}{0pt}%
\pgfpathmoveto{\pgfqpoint{3.308104in}{2.216788in}}%
\pgfpathlineto{\pgfqpoint{3.321578in}{2.211365in}}%
\pgfpathlineto{\pgfqpoint{3.335055in}{2.206035in}}%
\pgfpathlineto{\pgfqpoint{3.348536in}{2.200800in}}%
\pgfpathlineto{\pgfqpoint{3.362021in}{2.195658in}}%
\pgfpathlineto{\pgfqpoint{3.370138in}{2.204633in}}%
\pgfpathlineto{\pgfqpoint{3.378249in}{2.213639in}}%
\pgfpathlineto{\pgfqpoint{3.386354in}{2.222675in}}%
\pgfpathlineto{\pgfqpoint{3.394453in}{2.231743in}}%
\pgfpathlineto{\pgfqpoint{3.380979in}{2.236903in}}%
\pgfpathlineto{\pgfqpoint{3.367510in}{2.242156in}}%
\pgfpathlineto{\pgfqpoint{3.354044in}{2.247503in}}%
\pgfpathlineto{\pgfqpoint{3.340583in}{2.252944in}}%
\pgfpathlineto{\pgfqpoint{3.332472in}{2.243851in}}%
\pgfpathlineto{\pgfqpoint{3.324356in}{2.234794in}}%
\pgfpathlineto{\pgfqpoint{3.316233in}{2.225774in}}%
\pgfpathlineto{\pgfqpoint{3.308104in}{2.216788in}}%
\pgfpathclose%
\pgfusepath{fill}%
\end{pgfscope}%
\begin{pgfscope}%
\pgfpathrectangle{\pgfqpoint{1.150000in}{0.150000in}}{\pgfqpoint{5.700000in}{5.700000in}}%
\pgfusepath{clip}%
\pgfsetbuttcap%
\pgfsetroundjoin%
\definecolor{currentfill}{rgb}{0.210503,0.363727,0.552206}%
\pgfsetfillcolor{currentfill}%
\pgfsetfillopacity{0.700000}%
\pgfsetlinewidth{0.000000pt}%
\definecolor{currentstroke}{rgb}{0.000000,0.000000,0.000000}%
\pgfsetstrokecolor{currentstroke}%
\pgfsetdash{}{0pt}%
\pgfpathmoveto{\pgfqpoint{5.892118in}{2.851007in}}%
\pgfpathlineto{\pgfqpoint{5.906206in}{2.848941in}}%
\pgfpathlineto{\pgfqpoint{5.920303in}{2.846940in}}%
\pgfpathlineto{\pgfqpoint{5.934411in}{2.845003in}}%
\pgfpathlineto{\pgfqpoint{5.948528in}{2.843131in}}%
\pgfpathlineto{\pgfqpoint{5.955869in}{2.855429in}}%
\pgfpathlineto{\pgfqpoint{5.963219in}{2.868088in}}%
\pgfpathlineto{\pgfqpoint{5.970579in}{2.881117in}}%
\pgfpathlineto{\pgfqpoint{5.977949in}{2.894525in}}%
\pgfpathlineto{\pgfqpoint{5.963859in}{2.896922in}}%
\pgfpathlineto{\pgfqpoint{5.949778in}{2.899383in}}%
\pgfpathlineto{\pgfqpoint{5.935706in}{2.901908in}}%
\pgfpathlineto{\pgfqpoint{5.921644in}{2.904498in}}%
\pgfpathlineto{\pgfqpoint{5.914248in}{2.890558in}}%
\pgfpathlineto{\pgfqpoint{5.906862in}{2.877003in}}%
\pgfpathlineto{\pgfqpoint{5.899485in}{2.863822in}}%
\pgfpathlineto{\pgfqpoint{5.892118in}{2.851007in}}%
\pgfpathclose%
\pgfusepath{fill}%
\end{pgfscope}%
\begin{pgfscope}%
\pgfpathrectangle{\pgfqpoint{1.150000in}{0.150000in}}{\pgfqpoint{5.700000in}{5.700000in}}%
\pgfusepath{clip}%
\pgfsetbuttcap%
\pgfsetroundjoin%
\definecolor{currentfill}{rgb}{0.194100,0.399323,0.555565}%
\pgfsetfillcolor{currentfill}%
\pgfsetfillopacity{0.700000}%
\pgfsetlinewidth{0.000000pt}%
\definecolor{currentstroke}{rgb}{0.000000,0.000000,0.000000}%
\pgfsetstrokecolor{currentstroke}%
\pgfsetdash{}{0pt}%
\pgfpathmoveto{\pgfqpoint{6.063883in}{2.941026in}}%
\pgfpathlineto{\pgfqpoint{6.077993in}{2.938406in}}%
\pgfpathlineto{\pgfqpoint{6.092113in}{2.935850in}}%
\pgfpathlineto{\pgfqpoint{6.106242in}{2.933357in}}%
\pgfpathlineto{\pgfqpoint{6.120380in}{2.930929in}}%
\pgfpathlineto{\pgfqpoint{6.127752in}{2.945268in}}%
\pgfpathlineto{\pgfqpoint{6.135138in}{2.960036in}}%
\pgfpathlineto{\pgfqpoint{6.142537in}{2.975243in}}%
\pgfpathlineto{\pgfqpoint{6.149950in}{2.990901in}}%
\pgfpathlineto{\pgfqpoint{6.135840in}{2.993894in}}%
\pgfpathlineto{\pgfqpoint{6.121739in}{2.996950in}}%
\pgfpathlineto{\pgfqpoint{6.107647in}{3.000071in}}%
\pgfpathlineto{\pgfqpoint{6.093564in}{3.003255in}}%
\pgfpathlineto{\pgfqpoint{6.086123in}{2.987026in}}%
\pgfpathlineto{\pgfqpoint{6.078696in}{2.971252in}}%
\pgfpathlineto{\pgfqpoint{6.071283in}{2.955922in}}%
\pgfpathlineto{\pgfqpoint{6.063883in}{2.941026in}}%
\pgfpathclose%
\pgfusepath{fill}%
\end{pgfscope}%
\begin{pgfscope}%
\pgfpathrectangle{\pgfqpoint{1.150000in}{0.150000in}}{\pgfqpoint{5.700000in}{5.700000in}}%
\pgfusepath{clip}%
\pgfsetbuttcap%
\pgfsetroundjoin%
\definecolor{currentfill}{rgb}{0.218130,0.347432,0.550038}%
\pgfsetfillcolor{currentfill}%
\pgfsetfillopacity{0.700000}%
\pgfsetlinewidth{0.000000pt}%
\definecolor{currentstroke}{rgb}{0.000000,0.000000,0.000000}%
\pgfsetstrokecolor{currentstroke}%
\pgfsetdash{}{0pt}%
\pgfpathmoveto{\pgfqpoint{5.806364in}{2.810105in}}%
\pgfpathlineto{\pgfqpoint{5.820439in}{2.808284in}}%
\pgfpathlineto{\pgfqpoint{5.834525in}{2.806529in}}%
\pgfpathlineto{\pgfqpoint{5.848620in}{2.804838in}}%
\pgfpathlineto{\pgfqpoint{5.862725in}{2.803212in}}%
\pgfpathlineto{\pgfqpoint{5.870062in}{2.814659in}}%
\pgfpathlineto{\pgfqpoint{5.877406in}{2.826434in}}%
\pgfpathlineto{\pgfqpoint{5.884758in}{2.838547in}}%
\pgfpathlineto{\pgfqpoint{5.892118in}{2.851007in}}%
\pgfpathlineto{\pgfqpoint{5.878039in}{2.853137in}}%
\pgfpathlineto{\pgfqpoint{5.863969in}{2.855332in}}%
\pgfpathlineto{\pgfqpoint{5.849910in}{2.857592in}}%
\pgfpathlineto{\pgfqpoint{5.835859in}{2.859916in}}%
\pgfpathlineto{\pgfqpoint{5.828474in}{2.846945in}}%
\pgfpathlineto{\pgfqpoint{5.821096in}{2.834326in}}%
\pgfpathlineto{\pgfqpoint{5.813727in}{2.822049in}}%
\pgfpathlineto{\pgfqpoint{5.806364in}{2.810105in}}%
\pgfpathclose%
\pgfusepath{fill}%
\end{pgfscope}%
\begin{pgfscope}%
\pgfpathrectangle{\pgfqpoint{1.150000in}{0.150000in}}{\pgfqpoint{5.700000in}{5.700000in}}%
\pgfusepath{clip}%
\pgfsetbuttcap%
\pgfsetroundjoin%
\definecolor{currentfill}{rgb}{0.283187,0.125848,0.444960}%
\pgfsetfillcolor{currentfill}%
\pgfsetfillopacity{0.700000}%
\pgfsetlinewidth{0.000000pt}%
\definecolor{currentstroke}{rgb}{0.000000,0.000000,0.000000}%
\pgfsetstrokecolor{currentstroke}%
\pgfsetdash{}{0pt}%
\pgfpathmoveto{\pgfqpoint{4.213598in}{2.332758in}}%
\pgfpathlineto{\pgfqpoint{4.227259in}{2.330679in}}%
\pgfpathlineto{\pgfqpoint{4.240926in}{2.328676in}}%
\pgfpathlineto{\pgfqpoint{4.254601in}{2.326750in}}%
\pgfpathlineto{\pgfqpoint{4.268284in}{2.324900in}}%
\pgfpathlineto{\pgfqpoint{4.276086in}{2.333555in}}%
\pgfpathlineto{\pgfqpoint{4.283883in}{2.342245in}}%
\pgfpathlineto{\pgfqpoint{4.291674in}{2.350973in}}%
\pgfpathlineto{\pgfqpoint{4.299460in}{2.359743in}}%
\pgfpathlineto{\pgfqpoint{4.285790in}{2.361774in}}%
\pgfpathlineto{\pgfqpoint{4.272126in}{2.363882in}}%
\pgfpathlineto{\pgfqpoint{4.258470in}{2.366065in}}%
\pgfpathlineto{\pgfqpoint{4.244822in}{2.368325in}}%
\pgfpathlineto{\pgfqpoint{4.237024in}{2.359366in}}%
\pgfpathlineto{\pgfqpoint{4.229221in}{2.350455in}}%
\pgfpathlineto{\pgfqpoint{4.221412in}{2.341586in}}%
\pgfpathlineto{\pgfqpoint{4.213598in}{2.332758in}}%
\pgfpathclose%
\pgfusepath{fill}%
\end{pgfscope}%
\begin{pgfscope}%
\pgfpathrectangle{\pgfqpoint{1.150000in}{0.150000in}}{\pgfqpoint{5.700000in}{5.700000in}}%
\pgfusepath{clip}%
\pgfsetbuttcap%
\pgfsetroundjoin%
\definecolor{currentfill}{rgb}{0.185556,0.418570,0.556753}%
\pgfsetfillcolor{currentfill}%
\pgfsetfillopacity{0.700000}%
\pgfsetlinewidth{0.000000pt}%
\definecolor{currentstroke}{rgb}{0.000000,0.000000,0.000000}%
\pgfsetstrokecolor{currentstroke}%
\pgfsetdash{}{0pt}%
\pgfpathmoveto{\pgfqpoint{6.149950in}{2.990901in}}%
\pgfpathlineto{\pgfqpoint{6.164070in}{2.987971in}}%
\pgfpathlineto{\pgfqpoint{6.178199in}{2.985105in}}%
\pgfpathlineto{\pgfqpoint{6.192337in}{2.982303in}}%
\pgfpathlineto{\pgfqpoint{6.199745in}{2.997988in}}%
\pgfpathlineto{\pgfqpoint{6.207168in}{3.014140in}}%
\pgfpathlineto{\pgfqpoint{6.214607in}{3.030771in}}%
\pgfpathlineto{\pgfqpoint{6.200490in}{3.034009in}}%
\pgfpathlineto{\pgfqpoint{6.186382in}{3.037312in}}%
\pgfpathlineto{\pgfqpoint{6.172283in}{3.040677in}}%
\pgfpathlineto{\pgfqpoint{6.164823in}{3.023607in}}%
\pgfpathlineto{\pgfqpoint{6.157379in}{3.007018in}}%
\pgfpathlineto{\pgfqpoint{6.149950in}{2.990901in}}%
\pgfpathclose%
\pgfusepath{fill}%
\end{pgfscope}%
\begin{pgfscope}%
\pgfpathrectangle{\pgfqpoint{1.150000in}{0.150000in}}{\pgfqpoint{5.700000in}{5.700000in}}%
\pgfusepath{clip}%
\pgfsetbuttcap%
\pgfsetroundjoin%
\definecolor{currentfill}{rgb}{0.280868,0.160771,0.472899}%
\pgfsetfillcolor{currentfill}%
\pgfsetfillopacity{0.700000}%
\pgfsetlinewidth{0.000000pt}%
\definecolor{currentstroke}{rgb}{0.000000,0.000000,0.000000}%
\pgfsetstrokecolor{currentstroke}%
\pgfsetdash{}{0pt}%
\pgfpathmoveto{\pgfqpoint{4.525914in}{2.408282in}}%
\pgfpathlineto{\pgfqpoint{4.539660in}{2.406798in}}%
\pgfpathlineto{\pgfqpoint{4.553415in}{2.405388in}}%
\pgfpathlineto{\pgfqpoint{4.567178in}{2.404051in}}%
\pgfpathlineto{\pgfqpoint{4.580949in}{2.402787in}}%
\pgfpathlineto{\pgfqpoint{4.588638in}{2.411169in}}%
\pgfpathlineto{\pgfqpoint{4.596323in}{2.419614in}}%
\pgfpathlineto{\pgfqpoint{4.604003in}{2.428124in}}%
\pgfpathlineto{\pgfqpoint{4.611678in}{2.436705in}}%
\pgfpathlineto{\pgfqpoint{4.597921in}{2.438212in}}%
\pgfpathlineto{\pgfqpoint{4.584171in}{2.439791in}}%
\pgfpathlineto{\pgfqpoint{4.570430in}{2.441443in}}%
\pgfpathlineto{\pgfqpoint{4.556697in}{2.443168in}}%
\pgfpathlineto{\pgfqpoint{4.549008in}{2.434338in}}%
\pgfpathlineto{\pgfqpoint{4.541315in}{2.425583in}}%
\pgfpathlineto{\pgfqpoint{4.533617in}{2.416899in}}%
\pgfpathlineto{\pgfqpoint{4.525914in}{2.408282in}}%
\pgfpathclose%
\pgfusepath{fill}%
\end{pgfscope}%
\begin{pgfscope}%
\pgfpathrectangle{\pgfqpoint{1.150000in}{0.150000in}}{\pgfqpoint{5.700000in}{5.700000in}}%
\pgfusepath{clip}%
\pgfsetbuttcap%
\pgfsetroundjoin%
\definecolor{currentfill}{rgb}{0.280267,0.073417,0.397163}%
\pgfsetfillcolor{currentfill}%
\pgfsetfillopacity{0.700000}%
\pgfsetlinewidth{0.000000pt}%
\definecolor{currentstroke}{rgb}{0.000000,0.000000,0.000000}%
\pgfsetstrokecolor{currentstroke}%
\pgfsetdash{}{0pt}%
\pgfpathmoveto{\pgfqpoint{3.674873in}{2.233744in}}%
\pgfpathlineto{\pgfqpoint{3.688407in}{2.230001in}}%
\pgfpathlineto{\pgfqpoint{3.701947in}{2.226343in}}%
\pgfpathlineto{\pgfqpoint{3.715492in}{2.222770in}}%
\pgfpathlineto{\pgfqpoint{3.729043in}{2.219282in}}%
\pgfpathlineto{\pgfqpoint{3.737034in}{2.228331in}}%
\pgfpathlineto{\pgfqpoint{3.745018in}{2.237399in}}%
\pgfpathlineto{\pgfqpoint{3.752997in}{2.246489in}}%
\pgfpathlineto{\pgfqpoint{3.760970in}{2.255602in}}%
\pgfpathlineto{\pgfqpoint{3.747430in}{2.259169in}}%
\pgfpathlineto{\pgfqpoint{3.733896in}{2.262821in}}%
\pgfpathlineto{\pgfqpoint{3.720367in}{2.266557in}}%
\pgfpathlineto{\pgfqpoint{3.706843in}{2.270379in}}%
\pgfpathlineto{\pgfqpoint{3.698859in}{2.261180in}}%
\pgfpathlineto{\pgfqpoint{3.690870in}{2.252009in}}%
\pgfpathlineto{\pgfqpoint{3.682874in}{2.242864in}}%
\pgfpathlineto{\pgfqpoint{3.674873in}{2.233744in}}%
\pgfpathclose%
\pgfusepath{fill}%
\end{pgfscope}%
\begin{pgfscope}%
\pgfpathrectangle{\pgfqpoint{1.150000in}{0.150000in}}{\pgfqpoint{5.700000in}{5.700000in}}%
\pgfusepath{clip}%
\pgfsetbuttcap%
\pgfsetroundjoin%
\definecolor{currentfill}{rgb}{0.257322,0.256130,0.526563}%
\pgfsetfillcolor{currentfill}%
\pgfsetfillopacity{0.700000}%
\pgfsetlinewidth{0.000000pt}%
\definecolor{currentstroke}{rgb}{0.000000,0.000000,0.000000}%
\pgfsetstrokecolor{currentstroke}%
\pgfsetdash{}{0pt}%
\pgfpathmoveto{\pgfqpoint{5.236517in}{2.606256in}}%
\pgfpathlineto{\pgfqpoint{5.250464in}{2.605184in}}%
\pgfpathlineto{\pgfqpoint{5.264420in}{2.604180in}}%
\pgfpathlineto{\pgfqpoint{5.278386in}{2.603244in}}%
\pgfpathlineto{\pgfqpoint{5.292361in}{2.602376in}}%
\pgfpathlineto{\pgfqpoint{5.299804in}{2.611059in}}%
\pgfpathlineto{\pgfqpoint{5.307247in}{2.619916in}}%
\pgfpathlineto{\pgfqpoint{5.314688in}{2.628953in}}%
\pgfpathlineto{\pgfqpoint{5.322130in}{2.638178in}}%
\pgfpathlineto{\pgfqpoint{5.308175in}{2.639431in}}%
\pgfpathlineto{\pgfqpoint{5.294229in}{2.640752in}}%
\pgfpathlineto{\pgfqpoint{5.280293in}{2.642140in}}%
\pgfpathlineto{\pgfqpoint{5.266366in}{2.643596in}}%
\pgfpathlineto{\pgfqpoint{5.258904in}{2.633980in}}%
\pgfpathlineto{\pgfqpoint{5.251443in}{2.624556in}}%
\pgfpathlineto{\pgfqpoint{5.243980in}{2.615317in}}%
\pgfpathlineto{\pgfqpoint{5.236517in}{2.606256in}}%
\pgfpathclose%
\pgfusepath{fill}%
\end{pgfscope}%
\begin{pgfscope}%
\pgfpathrectangle{\pgfqpoint{1.150000in}{0.150000in}}{\pgfqpoint{5.700000in}{5.700000in}}%
\pgfusepath{clip}%
\pgfsetbuttcap%
\pgfsetroundjoin%
\definecolor{currentfill}{rgb}{0.225863,0.330805,0.547314}%
\pgfsetfillcolor{currentfill}%
\pgfsetfillopacity{0.700000}%
\pgfsetlinewidth{0.000000pt}%
\definecolor{currentstroke}{rgb}{0.000000,0.000000,0.000000}%
\pgfsetstrokecolor{currentstroke}%
\pgfsetdash{}{0pt}%
\pgfpathmoveto{\pgfqpoint{5.720664in}{2.771481in}}%
\pgfpathlineto{\pgfqpoint{5.734727in}{2.769884in}}%
\pgfpathlineto{\pgfqpoint{5.748799in}{2.768352in}}%
\pgfpathlineto{\pgfqpoint{5.762881in}{2.766886in}}%
\pgfpathlineto{\pgfqpoint{5.776973in}{2.765486in}}%
\pgfpathlineto{\pgfqpoint{5.784312in}{2.776184in}}%
\pgfpathlineto{\pgfqpoint{5.791657in}{2.787181in}}%
\pgfpathlineto{\pgfqpoint{5.799007in}{2.798485in}}%
\pgfpathlineto{\pgfqpoint{5.806364in}{2.810105in}}%
\pgfpathlineto{\pgfqpoint{5.792297in}{2.811990in}}%
\pgfpathlineto{\pgfqpoint{5.778240in}{2.813941in}}%
\pgfpathlineto{\pgfqpoint{5.764193in}{2.815957in}}%
\pgfpathlineto{\pgfqpoint{5.750154in}{2.818037in}}%
\pgfpathlineto{\pgfqpoint{5.742773in}{2.805926in}}%
\pgfpathlineto{\pgfqpoint{5.735398in}{2.794136in}}%
\pgfpathlineto{\pgfqpoint{5.728028in}{2.782657in}}%
\pgfpathlineto{\pgfqpoint{5.720664in}{2.771481in}}%
\pgfpathclose%
\pgfusepath{fill}%
\end{pgfscope}%
\begin{pgfscope}%
\pgfpathrectangle{\pgfqpoint{1.150000in}{0.150000in}}{\pgfqpoint{5.700000in}{5.700000in}}%
\pgfusepath{clip}%
\pgfsetbuttcap%
\pgfsetroundjoin%
\definecolor{currentfill}{rgb}{0.281924,0.089666,0.412415}%
\pgfsetfillcolor{currentfill}%
\pgfsetfillopacity{0.700000}%
\pgfsetlinewidth{0.000000pt}%
\definecolor{currentstroke}{rgb}{0.000000,0.000000,0.000000}%
\pgfsetstrokecolor{currentstroke}%
\pgfsetdash{}{0pt}%
\pgfpathmoveto{\pgfqpoint{3.901257in}{2.265988in}}%
\pgfpathlineto{\pgfqpoint{3.914841in}{2.263066in}}%
\pgfpathlineto{\pgfqpoint{3.928431in}{2.260225in}}%
\pgfpathlineto{\pgfqpoint{3.942028in}{2.257465in}}%
\pgfpathlineto{\pgfqpoint{3.955632in}{2.254786in}}%
\pgfpathlineto{\pgfqpoint{3.963545in}{2.263708in}}%
\pgfpathlineto{\pgfqpoint{3.971452in}{2.272650in}}%
\pgfpathlineto{\pgfqpoint{3.979354in}{2.281616in}}%
\pgfpathlineto{\pgfqpoint{3.987250in}{2.290607in}}%
\pgfpathlineto{\pgfqpoint{3.973658in}{2.293406in}}%
\pgfpathlineto{\pgfqpoint{3.960071in}{2.296286in}}%
\pgfpathlineto{\pgfqpoint{3.946492in}{2.299246in}}%
\pgfpathlineto{\pgfqpoint{3.932918in}{2.302288in}}%
\pgfpathlineto{\pgfqpoint{3.925011in}{2.293170in}}%
\pgfpathlineto{\pgfqpoint{3.917099in}{2.284082in}}%
\pgfpathlineto{\pgfqpoint{3.909181in}{2.275022in}}%
\pgfpathlineto{\pgfqpoint{3.901257in}{2.265988in}}%
\pgfpathclose%
\pgfusepath{fill}%
\end{pgfscope}%
\begin{pgfscope}%
\pgfpathrectangle{\pgfqpoint{1.150000in}{0.150000in}}{\pgfqpoint{5.700000in}{5.700000in}}%
\pgfusepath{clip}%
\pgfsetbuttcap%
\pgfsetroundjoin%
\definecolor{currentfill}{rgb}{0.273006,0.204520,0.501721}%
\pgfsetfillcolor{currentfill}%
\pgfsetfillopacity{0.700000}%
\pgfsetlinewidth{0.000000pt}%
\definecolor{currentstroke}{rgb}{0.000000,0.000000,0.000000}%
\pgfsetstrokecolor{currentstroke}%
\pgfsetdash{}{0pt}%
\pgfpathmoveto{\pgfqpoint{4.838314in}{2.489463in}}%
\pgfpathlineto{\pgfqpoint{4.852151in}{2.488343in}}%
\pgfpathlineto{\pgfqpoint{4.865998in}{2.487293in}}%
\pgfpathlineto{\pgfqpoint{4.879853in}{2.486313in}}%
\pgfpathlineto{\pgfqpoint{4.893717in}{2.485404in}}%
\pgfpathlineto{\pgfqpoint{4.901294in}{2.493656in}}%
\pgfpathlineto{\pgfqpoint{4.908867in}{2.502009in}}%
\pgfpathlineto{\pgfqpoint{4.916436in}{2.510468in}}%
\pgfpathlineto{\pgfqpoint{4.924002in}{2.519038in}}%
\pgfpathlineto{\pgfqpoint{4.910154in}{2.520251in}}%
\pgfpathlineto{\pgfqpoint{4.896316in}{2.521534in}}%
\pgfpathlineto{\pgfqpoint{4.882485in}{2.522887in}}%
\pgfpathlineto{\pgfqpoint{4.868664in}{2.524311in}}%
\pgfpathlineto{\pgfqpoint{4.861081in}{2.515430in}}%
\pgfpathlineto{\pgfqpoint{4.853496in}{2.506665in}}%
\pgfpathlineto{\pgfqpoint{4.845907in}{2.498012in}}%
\pgfpathlineto{\pgfqpoint{4.838314in}{2.489463in}}%
\pgfpathclose%
\pgfusepath{fill}%
\end{pgfscope}%
\begin{pgfscope}%
\pgfpathrectangle{\pgfqpoint{1.150000in}{0.150000in}}{\pgfqpoint{5.700000in}{5.700000in}}%
\pgfusepath{clip}%
\pgfsetbuttcap%
\pgfsetroundjoin%
\definecolor{currentfill}{rgb}{0.278791,0.062145,0.386592}%
\pgfsetfillcolor{currentfill}%
\pgfsetfillopacity{0.700000}%
\pgfsetlinewidth{0.000000pt}%
\definecolor{currentstroke}{rgb}{0.000000,0.000000,0.000000}%
\pgfsetstrokecolor{currentstroke}%
\pgfsetdash{}{0pt}%
\pgfpathmoveto{\pgfqpoint{3.448390in}{2.212026in}}%
\pgfpathlineto{\pgfqpoint{3.461885in}{2.207325in}}%
\pgfpathlineto{\pgfqpoint{3.475385in}{2.202714in}}%
\pgfpathlineto{\pgfqpoint{3.488890in}{2.198193in}}%
\pgfpathlineto{\pgfqpoint{3.502400in}{2.193761in}}%
\pgfpathlineto{\pgfqpoint{3.510469in}{2.202801in}}%
\pgfpathlineto{\pgfqpoint{3.518533in}{2.211864in}}%
\pgfpathlineto{\pgfqpoint{3.526591in}{2.220952in}}%
\pgfpathlineto{\pgfqpoint{3.534643in}{2.230066in}}%
\pgfpathlineto{\pgfqpoint{3.521145in}{2.234535in}}%
\pgfpathlineto{\pgfqpoint{3.507651in}{2.239094in}}%
\pgfpathlineto{\pgfqpoint{3.494162in}{2.243743in}}%
\pgfpathlineto{\pgfqpoint{3.480678in}{2.248482in}}%
\pgfpathlineto{\pgfqpoint{3.472615in}{2.239323in}}%
\pgfpathlineto{\pgfqpoint{3.464546in}{2.230195in}}%
\pgfpathlineto{\pgfqpoint{3.456471in}{2.221096in}}%
\pgfpathlineto{\pgfqpoint{3.448390in}{2.212026in}}%
\pgfpathclose%
\pgfusepath{fill}%
\end{pgfscope}%
\begin{pgfscope}%
\pgfpathrectangle{\pgfqpoint{1.150000in}{0.150000in}}{\pgfqpoint{5.700000in}{5.700000in}}%
\pgfusepath{clip}%
\pgfsetbuttcap%
\pgfsetroundjoin%
\definecolor{currentfill}{rgb}{0.282910,0.105393,0.426902}%
\pgfsetfillcolor{currentfill}%
\pgfsetfillopacity{0.700000}%
\pgfsetlinewidth{0.000000pt}%
\definecolor{currentstroke}{rgb}{0.000000,0.000000,0.000000}%
\pgfsetstrokecolor{currentstroke}%
\pgfsetdash{}{0pt}%
\pgfpathmoveto{\pgfqpoint{2.745761in}{2.308843in}}%
\pgfpathlineto{\pgfqpoint{2.759218in}{2.299924in}}%
\pgfpathlineto{\pgfqpoint{2.772675in}{2.291122in}}%
\pgfpathlineto{\pgfqpoint{2.786134in}{2.282437in}}%
\pgfpathlineto{\pgfqpoint{2.799593in}{2.273866in}}%
\pgfpathlineto{\pgfqpoint{2.807931in}{2.281822in}}%
\pgfpathlineto{\pgfqpoint{2.816262in}{2.289848in}}%
\pgfpathlineto{\pgfqpoint{2.824585in}{2.297944in}}%
\pgfpathlineto{\pgfqpoint{2.832899in}{2.306108in}}%
\pgfpathlineto{\pgfqpoint{2.819457in}{2.314613in}}%
\pgfpathlineto{\pgfqpoint{2.806015in}{2.323234in}}%
\pgfpathlineto{\pgfqpoint{2.792574in}{2.331970in}}%
\pgfpathlineto{\pgfqpoint{2.779133in}{2.340824in}}%
\pgfpathlineto{\pgfqpoint{2.770802in}{2.332717in}}%
\pgfpathlineto{\pgfqpoint{2.762463in}{2.324684in}}%
\pgfpathlineto{\pgfqpoint{2.754116in}{2.316726in}}%
\pgfpathlineto{\pgfqpoint{2.745761in}{2.308843in}}%
\pgfpathclose%
\pgfusepath{fill}%
\end{pgfscope}%
\begin{pgfscope}%
\pgfpathrectangle{\pgfqpoint{1.150000in}{0.150000in}}{\pgfqpoint{5.700000in}{5.700000in}}%
\pgfusepath{clip}%
\pgfsetbuttcap%
\pgfsetroundjoin%
\definecolor{currentfill}{rgb}{0.281887,0.150881,0.465405}%
\pgfsetfillcolor{currentfill}%
\pgfsetfillopacity{0.700000}%
\pgfsetlinewidth{0.000000pt}%
\definecolor{currentstroke}{rgb}{0.000000,0.000000,0.000000}%
\pgfsetstrokecolor{currentstroke}%
\pgfsetdash{}{0pt}%
\pgfpathmoveto{\pgfqpoint{2.550551in}{2.395616in}}%
\pgfpathlineto{\pgfqpoint{2.564030in}{2.385130in}}%
\pgfpathlineto{\pgfqpoint{2.577508in}{2.374773in}}%
\pgfpathlineto{\pgfqpoint{2.590985in}{2.364543in}}%
\pgfpathlineto{\pgfqpoint{2.604461in}{2.354441in}}%
\pgfpathlineto{\pgfqpoint{2.612886in}{2.361829in}}%
\pgfpathlineto{\pgfqpoint{2.621303in}{2.369306in}}%
\pgfpathlineto{\pgfqpoint{2.629710in}{2.376870in}}%
\pgfpathlineto{\pgfqpoint{2.638109in}{2.384520in}}%
\pgfpathlineto{\pgfqpoint{2.624651in}{2.394536in}}%
\pgfpathlineto{\pgfqpoint{2.611193in}{2.404678in}}%
\pgfpathlineto{\pgfqpoint{2.597734in}{2.414948in}}%
\pgfpathlineto{\pgfqpoint{2.584274in}{2.425348in}}%
\pgfpathlineto{\pgfqpoint{2.575857in}{2.417776in}}%
\pgfpathlineto{\pgfqpoint{2.567431in}{2.410297in}}%
\pgfpathlineto{\pgfqpoint{2.558996in}{2.402910in}}%
\pgfpathlineto{\pgfqpoint{2.550551in}{2.395616in}}%
\pgfpathclose%
\pgfusepath{fill}%
\end{pgfscope}%
\begin{pgfscope}%
\pgfpathrectangle{\pgfqpoint{1.150000in}{0.150000in}}{\pgfqpoint{5.700000in}{5.700000in}}%
\pgfusepath{clip}%
\pgfsetbuttcap%
\pgfsetroundjoin%
\definecolor{currentfill}{rgb}{0.231674,0.318106,0.544834}%
\pgfsetfillcolor{currentfill}%
\pgfsetfillopacity{0.700000}%
\pgfsetlinewidth{0.000000pt}%
\definecolor{currentstroke}{rgb}{0.000000,0.000000,0.000000}%
\pgfsetstrokecolor{currentstroke}%
\pgfsetdash{}{0pt}%
\pgfpathmoveto{\pgfqpoint{5.634998in}{2.734820in}}%
\pgfpathlineto{\pgfqpoint{5.649047in}{2.733425in}}%
\pgfpathlineto{\pgfqpoint{5.663105in}{2.732097in}}%
\pgfpathlineto{\pgfqpoint{5.677173in}{2.730834in}}%
\pgfpathlineto{\pgfqpoint{5.691251in}{2.729637in}}%
\pgfpathlineto{\pgfqpoint{5.698598in}{2.739685in}}%
\pgfpathlineto{\pgfqpoint{5.705949in}{2.750003in}}%
\pgfpathlineto{\pgfqpoint{5.713304in}{2.760599in}}%
\pgfpathlineto{\pgfqpoint{5.720664in}{2.771481in}}%
\pgfpathlineto{\pgfqpoint{5.706611in}{2.773143in}}%
\pgfpathlineto{\pgfqpoint{5.692567in}{2.774870in}}%
\pgfpathlineto{\pgfqpoint{5.678533in}{2.776663in}}%
\pgfpathlineto{\pgfqpoint{5.664508in}{2.778522in}}%
\pgfpathlineto{\pgfqpoint{5.657124in}{2.767168in}}%
\pgfpathlineto{\pgfqpoint{5.649745in}{2.756106in}}%
\pgfpathlineto{\pgfqpoint{5.642370in}{2.745326in}}%
\pgfpathlineto{\pgfqpoint{5.634998in}{2.734820in}}%
\pgfpathclose%
\pgfusepath{fill}%
\end{pgfscope}%
\begin{pgfscope}%
\pgfpathrectangle{\pgfqpoint{1.150000in}{0.150000in}}{\pgfqpoint{5.700000in}{5.700000in}}%
\pgfusepath{clip}%
\pgfsetbuttcap%
\pgfsetroundjoin%
\definecolor{currentfill}{rgb}{0.260571,0.246922,0.522828}%
\pgfsetfillcolor{currentfill}%
\pgfsetfillopacity{0.700000}%
\pgfsetlinewidth{0.000000pt}%
\definecolor{currentstroke}{rgb}{0.000000,0.000000,0.000000}%
\pgfsetstrokecolor{currentstroke}%
\pgfsetdash{}{0pt}%
\pgfpathmoveto{\pgfqpoint{5.150876in}{2.575177in}}%
\pgfpathlineto{\pgfqpoint{5.164805in}{2.574196in}}%
\pgfpathlineto{\pgfqpoint{5.178744in}{2.573285in}}%
\pgfpathlineto{\pgfqpoint{5.192691in}{2.572441in}}%
\pgfpathlineto{\pgfqpoint{5.206648in}{2.571666in}}%
\pgfpathlineto{\pgfqpoint{5.214118in}{2.580079in}}%
\pgfpathlineto{\pgfqpoint{5.221586in}{2.588644in}}%
\pgfpathlineto{\pgfqpoint{5.229052in}{2.597367in}}%
\pgfpathlineto{\pgfqpoint{5.236517in}{2.606256in}}%
\pgfpathlineto{\pgfqpoint{5.222579in}{2.607396in}}%
\pgfpathlineto{\pgfqpoint{5.208651in}{2.608604in}}%
\pgfpathlineto{\pgfqpoint{5.194731in}{2.609880in}}%
\pgfpathlineto{\pgfqpoint{5.180821in}{2.611224in}}%
\pgfpathlineto{\pgfqpoint{5.173337in}{2.601964in}}%
\pgfpathlineto{\pgfqpoint{5.165852in}{2.592874in}}%
\pgfpathlineto{\pgfqpoint{5.158365in}{2.583947in}}%
\pgfpathlineto{\pgfqpoint{5.150876in}{2.575177in}}%
\pgfpathclose%
\pgfusepath{fill}%
\end{pgfscope}%
\begin{pgfscope}%
\pgfpathrectangle{\pgfqpoint{1.150000in}{0.150000in}}{\pgfqpoint{5.700000in}{5.700000in}}%
\pgfusepath{clip}%
\pgfsetbuttcap%
\pgfsetroundjoin%
\definecolor{currentfill}{rgb}{0.281887,0.150881,0.465405}%
\pgfsetfillcolor{currentfill}%
\pgfsetfillopacity{0.700000}%
\pgfsetlinewidth{0.000000pt}%
\definecolor{currentstroke}{rgb}{0.000000,0.000000,0.000000}%
\pgfsetstrokecolor{currentstroke}%
\pgfsetdash{}{0pt}%
\pgfpathmoveto{\pgfqpoint{4.440095in}{2.380175in}}%
\pgfpathlineto{\pgfqpoint{4.453822in}{2.378619in}}%
\pgfpathlineto{\pgfqpoint{4.467557in}{2.377136in}}%
\pgfpathlineto{\pgfqpoint{4.481301in}{2.375728in}}%
\pgfpathlineto{\pgfqpoint{4.495052in}{2.374393in}}%
\pgfpathlineto{\pgfqpoint{4.502775in}{2.382786in}}%
\pgfpathlineto{\pgfqpoint{4.510493in}{2.391229in}}%
\pgfpathlineto{\pgfqpoint{4.518206in}{2.399726in}}%
\pgfpathlineto{\pgfqpoint{4.525914in}{2.408282in}}%
\pgfpathlineto{\pgfqpoint{4.512175in}{2.409838in}}%
\pgfpathlineto{\pgfqpoint{4.498445in}{2.411469in}}%
\pgfpathlineto{\pgfqpoint{4.484722in}{2.413173in}}%
\pgfpathlineto{\pgfqpoint{4.471008in}{2.414952in}}%
\pgfpathlineto{\pgfqpoint{4.463287in}{2.406167in}}%
\pgfpathlineto{\pgfqpoint{4.455561in}{2.397446in}}%
\pgfpathlineto{\pgfqpoint{4.447830in}{2.388783in}}%
\pgfpathlineto{\pgfqpoint{4.440095in}{2.380175in}}%
\pgfpathclose%
\pgfusepath{fill}%
\end{pgfscope}%
\begin{pgfscope}%
\pgfpathrectangle{\pgfqpoint{1.150000in}{0.150000in}}{\pgfqpoint{5.700000in}{5.700000in}}%
\pgfusepath{clip}%
\pgfsetbuttcap%
\pgfsetroundjoin%
\definecolor{currentfill}{rgb}{0.283197,0.115680,0.436115}%
\pgfsetfillcolor{currentfill}%
\pgfsetfillopacity{0.700000}%
\pgfsetlinewidth{0.000000pt}%
\definecolor{currentstroke}{rgb}{0.000000,0.000000,0.000000}%
\pgfsetstrokecolor{currentstroke}%
\pgfsetdash{}{0pt}%
\pgfpathmoveto{\pgfqpoint{4.127675in}{2.306227in}}%
\pgfpathlineto{\pgfqpoint{4.141318in}{2.303999in}}%
\pgfpathlineto{\pgfqpoint{4.154968in}{2.301848in}}%
\pgfpathlineto{\pgfqpoint{4.168625in}{2.299776in}}%
\pgfpathlineto{\pgfqpoint{4.182289in}{2.297780in}}%
\pgfpathlineto{\pgfqpoint{4.190125in}{2.306481in}}%
\pgfpathlineto{\pgfqpoint{4.197955in}{2.315208in}}%
\pgfpathlineto{\pgfqpoint{4.205779in}{2.323966in}}%
\pgfpathlineto{\pgfqpoint{4.213598in}{2.332758in}}%
\pgfpathlineto{\pgfqpoint{4.199945in}{2.334914in}}%
\pgfpathlineto{\pgfqpoint{4.186300in}{2.337148in}}%
\pgfpathlineto{\pgfqpoint{4.172661in}{2.339459in}}%
\pgfpathlineto{\pgfqpoint{4.159029in}{2.341847in}}%
\pgfpathlineto{\pgfqpoint{4.151199in}{2.332887in}}%
\pgfpathlineto{\pgfqpoint{4.143363in}{2.323966in}}%
\pgfpathlineto{\pgfqpoint{4.135521in}{2.315080in}}%
\pgfpathlineto{\pgfqpoint{4.127675in}{2.306227in}}%
\pgfpathclose%
\pgfusepath{fill}%
\end{pgfscope}%
\begin{pgfscope}%
\pgfpathrectangle{\pgfqpoint{1.150000in}{0.150000in}}{\pgfqpoint{5.700000in}{5.700000in}}%
\pgfusepath{clip}%
\pgfsetbuttcap%
\pgfsetroundjoin%
\definecolor{currentfill}{rgb}{0.278791,0.062145,0.386592}%
\pgfsetfillcolor{currentfill}%
\pgfsetfillopacity{0.700000}%
\pgfsetlinewidth{0.000000pt}%
\definecolor{currentstroke}{rgb}{0.000000,0.000000,0.000000}%
\pgfsetstrokecolor{currentstroke}%
\pgfsetdash{}{0pt}%
\pgfpathmoveto{\pgfqpoint{3.081133in}{2.219061in}}%
\pgfpathlineto{\pgfqpoint{3.094593in}{2.212435in}}%
\pgfpathlineto{\pgfqpoint{3.108055in}{2.205910in}}%
\pgfpathlineto{\pgfqpoint{3.121519in}{2.199486in}}%
\pgfpathlineto{\pgfqpoint{3.134987in}{2.193162in}}%
\pgfpathlineto{\pgfqpoint{3.143193in}{2.201823in}}%
\pgfpathlineto{\pgfqpoint{3.151392in}{2.210525in}}%
\pgfpathlineto{\pgfqpoint{3.159585in}{2.219270in}}%
\pgfpathlineto{\pgfqpoint{3.167771in}{2.228058in}}%
\pgfpathlineto{\pgfqpoint{3.154317in}{2.234358in}}%
\pgfpathlineto{\pgfqpoint{3.140865in}{2.240759in}}%
\pgfpathlineto{\pgfqpoint{3.127416in}{2.247260in}}%
\pgfpathlineto{\pgfqpoint{3.113971in}{2.253863in}}%
\pgfpathlineto{\pgfqpoint{3.105772in}{2.245091in}}%
\pgfpathlineto{\pgfqpoint{3.097566in}{2.236367in}}%
\pgfpathlineto{\pgfqpoint{3.089353in}{2.227690in}}%
\pgfpathlineto{\pgfqpoint{3.081133in}{2.219061in}}%
\pgfpathclose%
\pgfusepath{fill}%
\end{pgfscope}%
\begin{pgfscope}%
\pgfpathrectangle{\pgfqpoint{1.150000in}{0.150000in}}{\pgfqpoint{5.700000in}{5.700000in}}%
\pgfusepath{clip}%
\pgfsetbuttcap%
\pgfsetroundjoin%
\definecolor{currentfill}{rgb}{0.239346,0.300855,0.540844}%
\pgfsetfillcolor{currentfill}%
\pgfsetfillopacity{0.700000}%
\pgfsetlinewidth{0.000000pt}%
\definecolor{currentstroke}{rgb}{0.000000,0.000000,0.000000}%
\pgfsetstrokecolor{currentstroke}%
\pgfsetdash{}{0pt}%
\pgfpathmoveto{\pgfqpoint{5.549349in}{2.699832in}}%
\pgfpathlineto{\pgfqpoint{5.563383in}{2.698619in}}%
\pgfpathlineto{\pgfqpoint{5.577426in}{2.697471in}}%
\pgfpathlineto{\pgfqpoint{5.591479in}{2.696390in}}%
\pgfpathlineto{\pgfqpoint{5.605542in}{2.695375in}}%
\pgfpathlineto{\pgfqpoint{5.612902in}{2.704865in}}%
\pgfpathlineto{\pgfqpoint{5.620265in}{2.714597in}}%
\pgfpathlineto{\pgfqpoint{5.627630in}{2.724579in}}%
\pgfpathlineto{\pgfqpoint{5.634998in}{2.734820in}}%
\pgfpathlineto{\pgfqpoint{5.620959in}{2.736280in}}%
\pgfpathlineto{\pgfqpoint{5.606929in}{2.737806in}}%
\pgfpathlineto{\pgfqpoint{5.592909in}{2.739397in}}%
\pgfpathlineto{\pgfqpoint{5.578898in}{2.741055in}}%
\pgfpathlineto{\pgfqpoint{5.571507in}{2.730363in}}%
\pgfpathlineto{\pgfqpoint{5.564118in}{2.719934in}}%
\pgfpathlineto{\pgfqpoint{5.556732in}{2.709760in}}%
\pgfpathlineto{\pgfqpoint{5.549349in}{2.699832in}}%
\pgfpathclose%
\pgfusepath{fill}%
\end{pgfscope}%
\begin{pgfscope}%
\pgfpathrectangle{\pgfqpoint{1.150000in}{0.150000in}}{\pgfqpoint{5.700000in}{5.700000in}}%
\pgfusepath{clip}%
\pgfsetbuttcap%
\pgfsetroundjoin%
\definecolor{currentfill}{rgb}{0.276194,0.190074,0.493001}%
\pgfsetfillcolor{currentfill}%
\pgfsetfillopacity{0.700000}%
\pgfsetlinewidth{0.000000pt}%
\definecolor{currentstroke}{rgb}{0.000000,0.000000,0.000000}%
\pgfsetstrokecolor{currentstroke}%
\pgfsetdash{}{0pt}%
\pgfpathmoveto{\pgfqpoint{4.752577in}{2.460278in}}%
\pgfpathlineto{\pgfqpoint{4.766396in}{2.459157in}}%
\pgfpathlineto{\pgfqpoint{4.780223in}{2.458107in}}%
\pgfpathlineto{\pgfqpoint{4.794058in}{2.457128in}}%
\pgfpathlineto{\pgfqpoint{4.807903in}{2.456220in}}%
\pgfpathlineto{\pgfqpoint{4.815512in}{2.464399in}}%
\pgfpathlineto{\pgfqpoint{4.823116in}{2.472662in}}%
\pgfpathlineto{\pgfqpoint{4.830717in}{2.481015in}}%
\pgfpathlineto{\pgfqpoint{4.838314in}{2.489463in}}%
\pgfpathlineto{\pgfqpoint{4.824485in}{2.490655in}}%
\pgfpathlineto{\pgfqpoint{4.810665in}{2.491917in}}%
\pgfpathlineto{\pgfqpoint{4.796853in}{2.493250in}}%
\pgfpathlineto{\pgfqpoint{4.783050in}{2.494654in}}%
\pgfpathlineto{\pgfqpoint{4.775438in}{2.485915in}}%
\pgfpathlineto{\pgfqpoint{4.767822in}{2.477277in}}%
\pgfpathlineto{\pgfqpoint{4.760202in}{2.468732in}}%
\pgfpathlineto{\pgfqpoint{4.752577in}{2.460278in}}%
\pgfpathclose%
\pgfusepath{fill}%
\end{pgfscope}%
\begin{pgfscope}%
\pgfpathrectangle{\pgfqpoint{1.150000in}{0.150000in}}{\pgfqpoint{5.700000in}{5.700000in}}%
\pgfusepath{clip}%
\pgfsetbuttcap%
\pgfsetroundjoin%
\definecolor{currentfill}{rgb}{0.277941,0.056324,0.381191}%
\pgfsetfillcolor{currentfill}%
\pgfsetfillopacity{0.700000}%
\pgfsetlinewidth{0.000000pt}%
\definecolor{currentstroke}{rgb}{0.000000,0.000000,0.000000}%
\pgfsetstrokecolor{currentstroke}%
\pgfsetdash{}{0pt}%
\pgfpathmoveto{\pgfqpoint{3.221621in}{2.203845in}}%
\pgfpathlineto{\pgfqpoint{3.235092in}{2.198035in}}%
\pgfpathlineto{\pgfqpoint{3.248566in}{2.192323in}}%
\pgfpathlineto{\pgfqpoint{3.262044in}{2.186707in}}%
\pgfpathlineto{\pgfqpoint{3.275526in}{2.181185in}}%
\pgfpathlineto{\pgfqpoint{3.283680in}{2.190036in}}%
\pgfpathlineto{\pgfqpoint{3.291828in}{2.198920in}}%
\pgfpathlineto{\pgfqpoint{3.299969in}{2.207837in}}%
\pgfpathlineto{\pgfqpoint{3.308104in}{2.216788in}}%
\pgfpathlineto{\pgfqpoint{3.294635in}{2.222307in}}%
\pgfpathlineto{\pgfqpoint{3.281169in}{2.227921in}}%
\pgfpathlineto{\pgfqpoint{3.267707in}{2.233630in}}%
\pgfpathlineto{\pgfqpoint{3.254249in}{2.239437in}}%
\pgfpathlineto{\pgfqpoint{3.246101in}{2.230481in}}%
\pgfpathlineto{\pgfqpoint{3.237948in}{2.221564in}}%
\pgfpathlineto{\pgfqpoint{3.229788in}{2.212685in}}%
\pgfpathlineto{\pgfqpoint{3.221621in}{2.203845in}}%
\pgfpathclose%
\pgfusepath{fill}%
\end{pgfscope}%
\begin{pgfscope}%
\pgfpathrectangle{\pgfqpoint{1.150000in}{0.150000in}}{\pgfqpoint{5.700000in}{5.700000in}}%
\pgfusepath{clip}%
\pgfsetbuttcap%
\pgfsetroundjoin%
\definecolor{currentfill}{rgb}{0.280894,0.078907,0.402329}%
\pgfsetfillcolor{currentfill}%
\pgfsetfillopacity{0.700000}%
\pgfsetlinewidth{0.000000pt}%
\definecolor{currentstroke}{rgb}{0.000000,0.000000,0.000000}%
\pgfsetstrokecolor{currentstroke}%
\pgfsetdash{}{0pt}%
\pgfpathmoveto{\pgfqpoint{2.940485in}{2.242110in}}%
\pgfpathlineto{\pgfqpoint{2.953941in}{2.234604in}}%
\pgfpathlineto{\pgfqpoint{2.967398in}{2.227204in}}%
\pgfpathlineto{\pgfqpoint{2.980857in}{2.219912in}}%
\pgfpathlineto{\pgfqpoint{2.994318in}{2.212724in}}%
\pgfpathlineto{\pgfqpoint{3.002580in}{2.221112in}}%
\pgfpathlineto{\pgfqpoint{3.010834in}{2.229554in}}%
\pgfpathlineto{\pgfqpoint{3.019082in}{2.238047in}}%
\pgfpathlineto{\pgfqpoint{3.027322in}{2.246593in}}%
\pgfpathlineto{\pgfqpoint{3.013875in}{2.253737in}}%
\pgfpathlineto{\pgfqpoint{3.000431in}{2.260986in}}%
\pgfpathlineto{\pgfqpoint{2.986988in}{2.268341in}}%
\pgfpathlineto{\pgfqpoint{2.973548in}{2.275803in}}%
\pgfpathlineto{\pgfqpoint{2.965293in}{2.267294in}}%
\pgfpathlineto{\pgfqpoint{2.957031in}{2.258841in}}%
\pgfpathlineto{\pgfqpoint{2.948762in}{2.250447in}}%
\pgfpathlineto{\pgfqpoint{2.940485in}{2.242110in}}%
\pgfpathclose%
\pgfusepath{fill}%
\end{pgfscope}%
\begin{pgfscope}%
\pgfpathrectangle{\pgfqpoint{1.150000in}{0.150000in}}{\pgfqpoint{5.700000in}{5.700000in}}%
\pgfusepath{clip}%
\pgfsetbuttcap%
\pgfsetroundjoin%
\definecolor{currentfill}{rgb}{0.279566,0.067836,0.391917}%
\pgfsetfillcolor{currentfill}%
\pgfsetfillopacity{0.700000}%
\pgfsetlinewidth{0.000000pt}%
\definecolor{currentstroke}{rgb}{0.000000,0.000000,0.000000}%
\pgfsetstrokecolor{currentstroke}%
\pgfsetdash{}{0pt}%
\pgfpathmoveto{\pgfqpoint{3.588686in}{2.213072in}}%
\pgfpathlineto{\pgfqpoint{3.602209in}{2.209042in}}%
\pgfpathlineto{\pgfqpoint{3.615738in}{2.205100in}}%
\pgfpathlineto{\pgfqpoint{3.629272in}{2.201244in}}%
\pgfpathlineto{\pgfqpoint{3.642811in}{2.197474in}}%
\pgfpathlineto{\pgfqpoint{3.650835in}{2.206513in}}%
\pgfpathlineto{\pgfqpoint{3.658854in}{2.215569in}}%
\pgfpathlineto{\pgfqpoint{3.666866in}{2.224646in}}%
\pgfpathlineto{\pgfqpoint{3.674873in}{2.233744in}}%
\pgfpathlineto{\pgfqpoint{3.661345in}{2.237572in}}%
\pgfpathlineto{\pgfqpoint{3.647822in}{2.241487in}}%
\pgfpathlineto{\pgfqpoint{3.634304in}{2.245488in}}%
\pgfpathlineto{\pgfqpoint{3.620791in}{2.249576in}}%
\pgfpathlineto{\pgfqpoint{3.612774in}{2.240412in}}%
\pgfpathlineto{\pgfqpoint{3.604750in}{2.231274in}}%
\pgfpathlineto{\pgfqpoint{3.596721in}{2.222161in}}%
\pgfpathlineto{\pgfqpoint{3.588686in}{2.213072in}}%
\pgfpathclose%
\pgfusepath{fill}%
\end{pgfscope}%
\begin{pgfscope}%
\pgfpathrectangle{\pgfqpoint{1.150000in}{0.150000in}}{\pgfqpoint{5.700000in}{5.700000in}}%
\pgfusepath{clip}%
\pgfsetbuttcap%
\pgfsetroundjoin%
\definecolor{currentfill}{rgb}{0.281446,0.084320,0.407414}%
\pgfsetfillcolor{currentfill}%
\pgfsetfillopacity{0.700000}%
\pgfsetlinewidth{0.000000pt}%
\definecolor{currentstroke}{rgb}{0.000000,0.000000,0.000000}%
\pgfsetstrokecolor{currentstroke}%
\pgfsetdash{}{0pt}%
\pgfpathmoveto{\pgfqpoint{3.815190in}{2.242170in}}%
\pgfpathlineto{\pgfqpoint{3.828760in}{2.239020in}}%
\pgfpathlineto{\pgfqpoint{3.842336in}{2.235952in}}%
\pgfpathlineto{\pgfqpoint{3.855918in}{2.232967in}}%
\pgfpathlineto{\pgfqpoint{3.869506in}{2.230063in}}%
\pgfpathlineto{\pgfqpoint{3.877452in}{2.239017in}}%
\pgfpathlineto{\pgfqpoint{3.885393in}{2.247988in}}%
\pgfpathlineto{\pgfqpoint{3.893328in}{2.256977in}}%
\pgfpathlineto{\pgfqpoint{3.901257in}{2.265988in}}%
\pgfpathlineto{\pgfqpoint{3.887679in}{2.268991in}}%
\pgfpathlineto{\pgfqpoint{3.874108in}{2.272076in}}%
\pgfpathlineto{\pgfqpoint{3.860543in}{2.275244in}}%
\pgfpathlineto{\pgfqpoint{3.846984in}{2.278493in}}%
\pgfpathlineto{\pgfqpoint{3.839044in}{2.269375in}}%
\pgfpathlineto{\pgfqpoint{3.831098in}{2.260284in}}%
\pgfpathlineto{\pgfqpoint{3.823147in}{2.251216in}}%
\pgfpathlineto{\pgfqpoint{3.815190in}{2.242170in}}%
\pgfpathclose%
\pgfusepath{fill}%
\end{pgfscope}%
\begin{pgfscope}%
\pgfpathrectangle{\pgfqpoint{1.150000in}{0.150000in}}{\pgfqpoint{5.700000in}{5.700000in}}%
\pgfusepath{clip}%
\pgfsetbuttcap%
\pgfsetroundjoin%
\definecolor{currentfill}{rgb}{0.265145,0.232956,0.516599}%
\pgfsetfillcolor{currentfill}%
\pgfsetfillopacity{0.700000}%
\pgfsetlinewidth{0.000000pt}%
\definecolor{currentstroke}{rgb}{0.000000,0.000000,0.000000}%
\pgfsetstrokecolor{currentstroke}%
\pgfsetdash{}{0pt}%
\pgfpathmoveto{\pgfqpoint{5.065199in}{2.544769in}}%
\pgfpathlineto{\pgfqpoint{5.079110in}{2.543859in}}%
\pgfpathlineto{\pgfqpoint{5.093030in}{2.543017in}}%
\pgfpathlineto{\pgfqpoint{5.106959in}{2.542244in}}%
\pgfpathlineto{\pgfqpoint{5.120898in}{2.541540in}}%
\pgfpathlineto{\pgfqpoint{5.128396in}{2.549745in}}%
\pgfpathlineto{\pgfqpoint{5.135892in}{2.558082in}}%
\pgfpathlineto{\pgfqpoint{5.143385in}{2.566557in}}%
\pgfpathlineto{\pgfqpoint{5.150876in}{2.575177in}}%
\pgfpathlineto{\pgfqpoint{5.136956in}{2.576225in}}%
\pgfpathlineto{\pgfqpoint{5.123045in}{2.577343in}}%
\pgfpathlineto{\pgfqpoint{5.109143in}{2.578529in}}%
\pgfpathlineto{\pgfqpoint{5.095250in}{2.579784in}}%
\pgfpathlineto{\pgfqpoint{5.087741in}{2.570813in}}%
\pgfpathlineto{\pgfqpoint{5.080229in}{2.561991in}}%
\pgfpathlineto{\pgfqpoint{5.072716in}{2.553312in}}%
\pgfpathlineto{\pgfqpoint{5.065199in}{2.544769in}}%
\pgfpathclose%
\pgfusepath{fill}%
\end{pgfscope}%
\begin{pgfscope}%
\pgfpathrectangle{\pgfqpoint{1.150000in}{0.150000in}}{\pgfqpoint{5.700000in}{5.700000in}}%
\pgfusepath{clip}%
\pgfsetbuttcap%
\pgfsetroundjoin%
\definecolor{currentfill}{rgb}{0.244972,0.287675,0.537260}%
\pgfsetfillcolor{currentfill}%
\pgfsetfillopacity{0.700000}%
\pgfsetlinewidth{0.000000pt}%
\definecolor{currentstroke}{rgb}{0.000000,0.000000,0.000000}%
\pgfsetstrokecolor{currentstroke}%
\pgfsetdash{}{0pt}%
\pgfpathmoveto{\pgfqpoint{5.463701in}{2.666253in}}%
\pgfpathlineto{\pgfqpoint{5.477719in}{2.665198in}}%
\pgfpathlineto{\pgfqpoint{5.491747in}{2.664211in}}%
\pgfpathlineto{\pgfqpoint{5.505784in}{2.663290in}}%
\pgfpathlineto{\pgfqpoint{5.519831in}{2.662436in}}%
\pgfpathlineto{\pgfqpoint{5.527209in}{2.671453in}}%
\pgfpathlineto{\pgfqpoint{5.534587in}{2.680686in}}%
\pgfpathlineto{\pgfqpoint{5.541967in}{2.690143in}}%
\pgfpathlineto{\pgfqpoint{5.549349in}{2.699832in}}%
\pgfpathlineto{\pgfqpoint{5.535325in}{2.701112in}}%
\pgfpathlineto{\pgfqpoint{5.521310in}{2.702458in}}%
\pgfpathlineto{\pgfqpoint{5.507304in}{2.703870in}}%
\pgfpathlineto{\pgfqpoint{5.493308in}{2.705349in}}%
\pgfpathlineto{\pgfqpoint{5.485904in}{2.695228in}}%
\pgfpathlineto{\pgfqpoint{5.478502in}{2.685343in}}%
\pgfpathlineto{\pgfqpoint{5.471101in}{2.675687in}}%
\pgfpathlineto{\pgfqpoint{5.463701in}{2.666253in}}%
\pgfpathclose%
\pgfusepath{fill}%
\end{pgfscope}%
\begin{pgfscope}%
\pgfpathrectangle{\pgfqpoint{1.150000in}{0.150000in}}{\pgfqpoint{5.700000in}{5.700000in}}%
\pgfusepath{clip}%
\pgfsetbuttcap%
\pgfsetroundjoin%
\definecolor{currentfill}{rgb}{0.283072,0.130895,0.449241}%
\pgfsetfillcolor{currentfill}%
\pgfsetfillopacity{0.700000}%
\pgfsetlinewidth{0.000000pt}%
\definecolor{currentstroke}{rgb}{0.000000,0.000000,0.000000}%
\pgfsetstrokecolor{currentstroke}%
\pgfsetdash{}{0pt}%
\pgfpathmoveto{\pgfqpoint{2.604461in}{2.354441in}}%
\pgfpathlineto{\pgfqpoint{2.617936in}{2.344465in}}%
\pgfpathlineto{\pgfqpoint{2.631411in}{2.334614in}}%
\pgfpathlineto{\pgfqpoint{2.644886in}{2.324887in}}%
\pgfpathlineto{\pgfqpoint{2.658360in}{2.315283in}}%
\pgfpathlineto{\pgfqpoint{2.666767in}{2.322765in}}%
\pgfpathlineto{\pgfqpoint{2.675165in}{2.330330in}}%
\pgfpathlineto{\pgfqpoint{2.683554in}{2.337978in}}%
\pgfpathlineto{\pgfqpoint{2.691935in}{2.345707in}}%
\pgfpathlineto{\pgfqpoint{2.678479in}{2.355225in}}%
\pgfpathlineto{\pgfqpoint{2.665023in}{2.364866in}}%
\pgfpathlineto{\pgfqpoint{2.651566in}{2.374631in}}%
\pgfpathlineto{\pgfqpoint{2.638109in}{2.384520in}}%
\pgfpathlineto{\pgfqpoint{2.629710in}{2.376870in}}%
\pgfpathlineto{\pgfqpoint{2.621303in}{2.369306in}}%
\pgfpathlineto{\pgfqpoint{2.612886in}{2.361829in}}%
\pgfpathlineto{\pgfqpoint{2.604461in}{2.354441in}}%
\pgfpathclose%
\pgfusepath{fill}%
\end{pgfscope}%
\begin{pgfscope}%
\pgfpathrectangle{\pgfqpoint{1.150000in}{0.150000in}}{\pgfqpoint{5.700000in}{5.700000in}}%
\pgfusepath{clip}%
\pgfsetbuttcap%
\pgfsetroundjoin%
\definecolor{currentfill}{rgb}{0.282623,0.140926,0.457517}%
\pgfsetfillcolor{currentfill}%
\pgfsetfillopacity{0.700000}%
\pgfsetlinewidth{0.000000pt}%
\definecolor{currentstroke}{rgb}{0.000000,0.000000,0.000000}%
\pgfsetstrokecolor{currentstroke}%
\pgfsetdash{}{0pt}%
\pgfpathmoveto{\pgfqpoint{4.354218in}{2.352377in}}%
\pgfpathlineto{\pgfqpoint{4.367927in}{2.350724in}}%
\pgfpathlineto{\pgfqpoint{4.381644in}{2.349145in}}%
\pgfpathlineto{\pgfqpoint{4.395368in}{2.347642in}}%
\pgfpathlineto{\pgfqpoint{4.409100in}{2.346213in}}%
\pgfpathlineto{\pgfqpoint{4.416856in}{2.354641in}}%
\pgfpathlineto{\pgfqpoint{4.424608in}{2.363108in}}%
\pgfpathlineto{\pgfqpoint{4.432354in}{2.371618in}}%
\pgfpathlineto{\pgfqpoint{4.440095in}{2.380175in}}%
\pgfpathlineto{\pgfqpoint{4.426375in}{2.381806in}}%
\pgfpathlineto{\pgfqpoint{4.412663in}{2.383511in}}%
\pgfpathlineto{\pgfqpoint{4.398959in}{2.385291in}}%
\pgfpathlineto{\pgfqpoint{4.385263in}{2.387146in}}%
\pgfpathlineto{\pgfqpoint{4.377509in}{2.378380in}}%
\pgfpathlineto{\pgfqpoint{4.369751in}{2.369666in}}%
\pgfpathlineto{\pgfqpoint{4.361987in}{2.360999in}}%
\pgfpathlineto{\pgfqpoint{4.354218in}{2.352377in}}%
\pgfpathclose%
\pgfusepath{fill}%
\end{pgfscope}%
\begin{pgfscope}%
\pgfpathrectangle{\pgfqpoint{1.150000in}{0.150000in}}{\pgfqpoint{5.700000in}{5.700000in}}%
\pgfusepath{clip}%
\pgfsetbuttcap%
\pgfsetroundjoin%
\definecolor{currentfill}{rgb}{0.277941,0.056324,0.381191}%
\pgfsetfillcolor{currentfill}%
\pgfsetfillopacity{0.700000}%
\pgfsetlinewidth{0.000000pt}%
\definecolor{currentstroke}{rgb}{0.000000,0.000000,0.000000}%
\pgfsetstrokecolor{currentstroke}%
\pgfsetdash{}{0pt}%
\pgfpathmoveto{\pgfqpoint{3.362021in}{2.195658in}}%
\pgfpathlineto{\pgfqpoint{3.375511in}{2.190608in}}%
\pgfpathlineto{\pgfqpoint{3.389005in}{2.185651in}}%
\pgfpathlineto{\pgfqpoint{3.402503in}{2.180786in}}%
\pgfpathlineto{\pgfqpoint{3.416005in}{2.176012in}}%
\pgfpathlineto{\pgfqpoint{3.424110in}{2.184977in}}%
\pgfpathlineto{\pgfqpoint{3.432210in}{2.193967in}}%
\pgfpathlineto{\pgfqpoint{3.440303in}{2.202983in}}%
\pgfpathlineto{\pgfqpoint{3.448390in}{2.212026in}}%
\pgfpathlineto{\pgfqpoint{3.434899in}{2.216818in}}%
\pgfpathlineto{\pgfqpoint{3.421412in}{2.221701in}}%
\pgfpathlineto{\pgfqpoint{3.407930in}{2.226676in}}%
\pgfpathlineto{\pgfqpoint{3.394453in}{2.231743in}}%
\pgfpathlineto{\pgfqpoint{3.386354in}{2.222675in}}%
\pgfpathlineto{\pgfqpoint{3.378249in}{2.213639in}}%
\pgfpathlineto{\pgfqpoint{3.370138in}{2.204633in}}%
\pgfpathlineto{\pgfqpoint{3.362021in}{2.195658in}}%
\pgfpathclose%
\pgfusepath{fill}%
\end{pgfscope}%
\begin{pgfscope}%
\pgfpathrectangle{\pgfqpoint{1.150000in}{0.150000in}}{\pgfqpoint{5.700000in}{5.700000in}}%
\pgfusepath{clip}%
\pgfsetbuttcap%
\pgfsetroundjoin%
\definecolor{currentfill}{rgb}{0.282910,0.105393,0.426902}%
\pgfsetfillcolor{currentfill}%
\pgfsetfillopacity{0.700000}%
\pgfsetlinewidth{0.000000pt}%
\definecolor{currentstroke}{rgb}{0.000000,0.000000,0.000000}%
\pgfsetstrokecolor{currentstroke}%
\pgfsetdash{}{0pt}%
\pgfpathmoveto{\pgfqpoint{4.041687in}{2.280209in}}%
\pgfpathlineto{\pgfqpoint{4.055313in}{2.277807in}}%
\pgfpathlineto{\pgfqpoint{4.068946in}{2.275485in}}%
\pgfpathlineto{\pgfqpoint{4.082586in}{2.273241in}}%
\pgfpathlineto{\pgfqpoint{4.096233in}{2.271076in}}%
\pgfpathlineto{\pgfqpoint{4.104102in}{2.279830in}}%
\pgfpathlineto{\pgfqpoint{4.111965in}{2.288604in}}%
\pgfpathlineto{\pgfqpoint{4.119823in}{2.297402in}}%
\pgfpathlineto{\pgfqpoint{4.127675in}{2.306227in}}%
\pgfpathlineto{\pgfqpoint{4.114039in}{2.308533in}}%
\pgfpathlineto{\pgfqpoint{4.100410in}{2.310917in}}%
\pgfpathlineto{\pgfqpoint{4.086788in}{2.313380in}}%
\pgfpathlineto{\pgfqpoint{4.073173in}{2.315921in}}%
\pgfpathlineto{\pgfqpoint{4.065310in}{2.306949in}}%
\pgfpathlineto{\pgfqpoint{4.057441in}{2.298008in}}%
\pgfpathlineto{\pgfqpoint{4.049567in}{2.289095in}}%
\pgfpathlineto{\pgfqpoint{4.041687in}{2.280209in}}%
\pgfpathclose%
\pgfusepath{fill}%
\end{pgfscope}%
\begin{pgfscope}%
\pgfpathrectangle{\pgfqpoint{1.150000in}{0.150000in}}{\pgfqpoint{5.700000in}{5.700000in}}%
\pgfusepath{clip}%
\pgfsetbuttcap%
\pgfsetroundjoin%
\definecolor{currentfill}{rgb}{0.282327,0.094955,0.417331}%
\pgfsetfillcolor{currentfill}%
\pgfsetfillopacity{0.700000}%
\pgfsetlinewidth{0.000000pt}%
\definecolor{currentstroke}{rgb}{0.000000,0.000000,0.000000}%
\pgfsetstrokecolor{currentstroke}%
\pgfsetdash{}{0pt}%
\pgfpathmoveto{\pgfqpoint{2.799593in}{2.273866in}}%
\pgfpathlineto{\pgfqpoint{2.813053in}{2.265410in}}%
\pgfpathlineto{\pgfqpoint{2.826513in}{2.257068in}}%
\pgfpathlineto{\pgfqpoint{2.839975in}{2.248838in}}%
\pgfpathlineto{\pgfqpoint{2.853439in}{2.240720in}}%
\pgfpathlineto{\pgfqpoint{2.861761in}{2.248749in}}%
\pgfpathlineto{\pgfqpoint{2.870075in}{2.256842in}}%
\pgfpathlineto{\pgfqpoint{2.878382in}{2.265000in}}%
\pgfpathlineto{\pgfqpoint{2.886681in}{2.273221in}}%
\pgfpathlineto{\pgfqpoint{2.873234in}{2.281275in}}%
\pgfpathlineto{\pgfqpoint{2.859788in}{2.289440in}}%
\pgfpathlineto{\pgfqpoint{2.846343in}{2.297717in}}%
\pgfpathlineto{\pgfqpoint{2.832899in}{2.306108in}}%
\pgfpathlineto{\pgfqpoint{2.824585in}{2.297944in}}%
\pgfpathlineto{\pgfqpoint{2.816262in}{2.289848in}}%
\pgfpathlineto{\pgfqpoint{2.807931in}{2.281822in}}%
\pgfpathlineto{\pgfqpoint{2.799593in}{2.273866in}}%
\pgfpathclose%
\pgfusepath{fill}%
\end{pgfscope}%
\begin{pgfscope}%
\pgfpathrectangle{\pgfqpoint{1.150000in}{0.150000in}}{\pgfqpoint{5.700000in}{5.700000in}}%
\pgfusepath{clip}%
\pgfsetbuttcap%
\pgfsetroundjoin%
\definecolor{currentfill}{rgb}{0.278012,0.180367,0.486697}%
\pgfsetfillcolor{currentfill}%
\pgfsetfillopacity{0.700000}%
\pgfsetlinewidth{0.000000pt}%
\definecolor{currentstroke}{rgb}{0.000000,0.000000,0.000000}%
\pgfsetstrokecolor{currentstroke}%
\pgfsetdash{}{0pt}%
\pgfpathmoveto{\pgfqpoint{4.666790in}{2.431404in}}%
\pgfpathlineto{\pgfqpoint{4.680589in}{2.430259in}}%
\pgfpathlineto{\pgfqpoint{4.694396in}{2.429187in}}%
\pgfpathlineto{\pgfqpoint{4.708212in}{2.428186in}}%
\pgfpathlineto{\pgfqpoint{4.722036in}{2.427257in}}%
\pgfpathlineto{\pgfqpoint{4.729678in}{2.435402in}}%
\pgfpathlineto{\pgfqpoint{4.737316in}{2.443618in}}%
\pgfpathlineto{\pgfqpoint{4.744949in}{2.451908in}}%
\pgfpathlineto{\pgfqpoint{4.752577in}{2.460278in}}%
\pgfpathlineto{\pgfqpoint{4.738768in}{2.461470in}}%
\pgfpathlineto{\pgfqpoint{4.724966in}{2.462735in}}%
\pgfpathlineto{\pgfqpoint{4.711174in}{2.464070in}}%
\pgfpathlineto{\pgfqpoint{4.697389in}{2.465478in}}%
\pgfpathlineto{\pgfqpoint{4.689746in}{2.456837in}}%
\pgfpathlineto{\pgfqpoint{4.682098in}{2.448281in}}%
\pgfpathlineto{\pgfqpoint{4.674446in}{2.439805in}}%
\pgfpathlineto{\pgfqpoint{4.666790in}{2.431404in}}%
\pgfpathclose%
\pgfusepath{fill}%
\end{pgfscope}%
\begin{pgfscope}%
\pgfpathrectangle{\pgfqpoint{1.150000in}{0.150000in}}{\pgfqpoint{5.700000in}{5.700000in}}%
\pgfusepath{clip}%
\pgfsetbuttcap%
\pgfsetroundjoin%
\definecolor{currentfill}{rgb}{0.192357,0.403199,0.555836}%
\pgfsetfillcolor{currentfill}%
\pgfsetfillopacity{0.700000}%
\pgfsetlinewidth{0.000000pt}%
\definecolor{currentstroke}{rgb}{0.000000,0.000000,0.000000}%
\pgfsetstrokecolor{currentstroke}%
\pgfsetdash{}{0pt}%
\pgfpathmoveto{\pgfqpoint{6.120380in}{2.930929in}}%
\pgfpathlineto{\pgfqpoint{6.134528in}{2.928564in}}%
\pgfpathlineto{\pgfqpoint{6.148686in}{2.926263in}}%
\pgfpathlineto{\pgfqpoint{6.162853in}{2.924025in}}%
\pgfpathlineto{\pgfqpoint{6.170204in}{2.937946in}}%
\pgfpathlineto{\pgfqpoint{6.177568in}{2.952292in}}%
\pgfpathlineto{\pgfqpoint{6.184945in}{2.967075in}}%
\pgfpathlineto{\pgfqpoint{6.192337in}{2.982303in}}%
\pgfpathlineto{\pgfqpoint{6.178199in}{2.985105in}}%
\pgfpathlineto{\pgfqpoint{6.164070in}{2.987971in}}%
\pgfpathlineto{\pgfqpoint{6.149950in}{2.990901in}}%
\pgfpathlineto{\pgfqpoint{6.142537in}{2.975243in}}%
\pgfpathlineto{\pgfqpoint{6.135138in}{2.960036in}}%
\pgfpathlineto{\pgfqpoint{6.127752in}{2.945268in}}%
\pgfpathlineto{\pgfqpoint{6.120380in}{2.930929in}}%
\pgfpathclose%
\pgfusepath{fill}%
\end{pgfscope}%
\begin{pgfscope}%
\pgfpathrectangle{\pgfqpoint{1.150000in}{0.150000in}}{\pgfqpoint{5.700000in}{5.700000in}}%
\pgfusepath{clip}%
\pgfsetbuttcap%
\pgfsetroundjoin%
\definecolor{currentfill}{rgb}{0.201239,0.383670,0.554294}%
\pgfsetfillcolor{currentfill}%
\pgfsetfillopacity{0.700000}%
\pgfsetlinewidth{0.000000pt}%
\definecolor{currentstroke}{rgb}{0.000000,0.000000,0.000000}%
\pgfsetstrokecolor{currentstroke}%
\pgfsetdash{}{0pt}%
\pgfpathmoveto{\pgfqpoint{6.034404in}{2.885581in}}%
\pgfpathlineto{\pgfqpoint{6.048542in}{2.883506in}}%
\pgfpathlineto{\pgfqpoint{6.062689in}{2.881494in}}%
\pgfpathlineto{\pgfqpoint{6.076847in}{2.879546in}}%
\pgfpathlineto{\pgfqpoint{6.091014in}{2.877663in}}%
\pgfpathlineto{\pgfqpoint{6.098338in}{2.890386in}}%
\pgfpathlineto{\pgfqpoint{6.105674in}{2.903498in}}%
\pgfpathlineto{\pgfqpoint{6.113021in}{2.917009in}}%
\pgfpathlineto{\pgfqpoint{6.120380in}{2.930929in}}%
\pgfpathlineto{\pgfqpoint{6.106242in}{2.933357in}}%
\pgfpathlineto{\pgfqpoint{6.092113in}{2.935850in}}%
\pgfpathlineto{\pgfqpoint{6.077993in}{2.938406in}}%
\pgfpathlineto{\pgfqpoint{6.063883in}{2.941026in}}%
\pgfpathlineto{\pgfqpoint{6.056496in}{2.926554in}}%
\pgfpathlineto{\pgfqpoint{6.049121in}{2.912496in}}%
\pgfpathlineto{\pgfqpoint{6.041757in}{2.898842in}}%
\pgfpathlineto{\pgfqpoint{6.034404in}{2.885581in}}%
\pgfpathclose%
\pgfusepath{fill}%
\end{pgfscope}%
\begin{pgfscope}%
\pgfpathrectangle{\pgfqpoint{1.150000in}{0.150000in}}{\pgfqpoint{5.700000in}{5.700000in}}%
\pgfusepath{clip}%
\pgfsetbuttcap%
\pgfsetroundjoin%
\definecolor{currentfill}{rgb}{0.248629,0.278775,0.534556}%
\pgfsetfillcolor{currentfill}%
\pgfsetfillopacity{0.700000}%
\pgfsetlinewidth{0.000000pt}%
\definecolor{currentstroke}{rgb}{0.000000,0.000000,0.000000}%
\pgfsetstrokecolor{currentstroke}%
\pgfsetdash{}{0pt}%
\pgfpathmoveto{\pgfqpoint{5.378042in}{2.633840in}}%
\pgfpathlineto{\pgfqpoint{5.392043in}{2.632923in}}%
\pgfpathlineto{\pgfqpoint{5.406055in}{2.632074in}}%
\pgfpathlineto{\pgfqpoint{5.420075in}{2.631291in}}%
\pgfpathlineto{\pgfqpoint{5.434106in}{2.630576in}}%
\pgfpathlineto{\pgfqpoint{5.441504in}{2.639200in}}%
\pgfpathlineto{\pgfqpoint{5.448903in}{2.648016in}}%
\pgfpathlineto{\pgfqpoint{5.456302in}{2.657031in}}%
\pgfpathlineto{\pgfqpoint{5.463701in}{2.666253in}}%
\pgfpathlineto{\pgfqpoint{5.449692in}{2.667373in}}%
\pgfpathlineto{\pgfqpoint{5.435693in}{2.668561in}}%
\pgfpathlineto{\pgfqpoint{5.421704in}{2.669815in}}%
\pgfpathlineto{\pgfqpoint{5.407723in}{2.671137in}}%
\pgfpathlineto{\pgfqpoint{5.400302in}{2.661503in}}%
\pgfpathlineto{\pgfqpoint{5.392882in}{2.652080in}}%
\pgfpathlineto{\pgfqpoint{5.385462in}{2.642862in}}%
\pgfpathlineto{\pgfqpoint{5.378042in}{2.633840in}}%
\pgfpathclose%
\pgfusepath{fill}%
\end{pgfscope}%
\begin{pgfscope}%
\pgfpathrectangle{\pgfqpoint{1.150000in}{0.150000in}}{\pgfqpoint{5.700000in}{5.700000in}}%
\pgfusepath{clip}%
\pgfsetbuttcap%
\pgfsetroundjoin%
\definecolor{currentfill}{rgb}{0.208623,0.367752,0.552675}%
\pgfsetfillcolor{currentfill}%
\pgfsetfillopacity{0.700000}%
\pgfsetlinewidth{0.000000pt}%
\definecolor{currentstroke}{rgb}{0.000000,0.000000,0.000000}%
\pgfsetstrokecolor{currentstroke}%
\pgfsetdash{}{0pt}%
\pgfpathmoveto{\pgfqpoint{5.948528in}{2.843131in}}%
\pgfpathlineto{\pgfqpoint{5.962654in}{2.841323in}}%
\pgfpathlineto{\pgfqpoint{5.976791in}{2.839580in}}%
\pgfpathlineto{\pgfqpoint{5.990937in}{2.837901in}}%
\pgfpathlineto{\pgfqpoint{6.005093in}{2.836287in}}%
\pgfpathlineto{\pgfqpoint{6.012407in}{2.848067in}}%
\pgfpathlineto{\pgfqpoint{6.019730in}{2.860204in}}%
\pgfpathlineto{\pgfqpoint{6.027062in}{2.872705in}}%
\pgfpathlineto{\pgfqpoint{6.034404in}{2.885581in}}%
\pgfpathlineto{\pgfqpoint{6.020276in}{2.887721in}}%
\pgfpathlineto{\pgfqpoint{6.006157in}{2.889925in}}%
\pgfpathlineto{\pgfqpoint{5.992048in}{2.892193in}}%
\pgfpathlineto{\pgfqpoint{5.977949in}{2.894525in}}%
\pgfpathlineto{\pgfqpoint{5.970579in}{2.881117in}}%
\pgfpathlineto{\pgfqpoint{5.963219in}{2.868088in}}%
\pgfpathlineto{\pgfqpoint{5.955869in}{2.855429in}}%
\pgfpathlineto{\pgfqpoint{5.948528in}{2.843131in}}%
\pgfpathclose%
\pgfusepath{fill}%
\end{pgfscope}%
\begin{pgfscope}%
\pgfpathrectangle{\pgfqpoint{1.150000in}{0.150000in}}{\pgfqpoint{5.700000in}{5.700000in}}%
\pgfusepath{clip}%
\pgfsetbuttcap%
\pgfsetroundjoin%
\definecolor{currentfill}{rgb}{0.267968,0.223549,0.512008}%
\pgfsetfillcolor{currentfill}%
\pgfsetfillopacity{0.700000}%
\pgfsetlinewidth{0.000000pt}%
\definecolor{currentstroke}{rgb}{0.000000,0.000000,0.000000}%
\pgfsetstrokecolor{currentstroke}%
\pgfsetdash{}{0pt}%
\pgfpathmoveto{\pgfqpoint{4.979481in}{2.514886in}}%
\pgfpathlineto{\pgfqpoint{4.993373in}{2.514023in}}%
\pgfpathlineto{\pgfqpoint{5.007274in}{2.513229in}}%
\pgfpathlineto{\pgfqpoint{5.021185in}{2.512504in}}%
\pgfpathlineto{\pgfqpoint{5.035104in}{2.511849in}}%
\pgfpathlineto{\pgfqpoint{5.042633in}{2.519903in}}%
\pgfpathlineto{\pgfqpoint{5.050158in}{2.528071in}}%
\pgfpathlineto{\pgfqpoint{5.057680in}{2.536358in}}%
\pgfpathlineto{\pgfqpoint{5.065199in}{2.544769in}}%
\pgfpathlineto{\pgfqpoint{5.051297in}{2.545749in}}%
\pgfpathlineto{\pgfqpoint{5.037405in}{2.546798in}}%
\pgfpathlineto{\pgfqpoint{5.023521in}{2.547916in}}%
\pgfpathlineto{\pgfqpoint{5.009646in}{2.549104in}}%
\pgfpathlineto{\pgfqpoint{5.002109in}{2.540361in}}%
\pgfpathlineto{\pgfqpoint{4.994570in}{2.531748in}}%
\pgfpathlineto{\pgfqpoint{4.987027in}{2.523258in}}%
\pgfpathlineto{\pgfqpoint{4.979481in}{2.514886in}}%
\pgfpathclose%
\pgfusepath{fill}%
\end{pgfscope}%
\begin{pgfscope}%
\pgfpathrectangle{\pgfqpoint{1.150000in}{0.150000in}}{\pgfqpoint{5.700000in}{5.700000in}}%
\pgfusepath{clip}%
\pgfsetbuttcap%
\pgfsetroundjoin%
\definecolor{currentfill}{rgb}{0.280267,0.073417,0.397163}%
\pgfsetfillcolor{currentfill}%
\pgfsetfillopacity{0.700000}%
\pgfsetlinewidth{0.000000pt}%
\definecolor{currentstroke}{rgb}{0.000000,0.000000,0.000000}%
\pgfsetstrokecolor{currentstroke}%
\pgfsetdash{}{0pt}%
\pgfpathmoveto{\pgfqpoint{3.729043in}{2.219282in}}%
\pgfpathlineto{\pgfqpoint{3.742600in}{2.215878in}}%
\pgfpathlineto{\pgfqpoint{3.756163in}{2.212557in}}%
\pgfpathlineto{\pgfqpoint{3.769731in}{2.209321in}}%
\pgfpathlineto{\pgfqpoint{3.783306in}{2.206167in}}%
\pgfpathlineto{\pgfqpoint{3.791285in}{2.215144in}}%
\pgfpathlineto{\pgfqpoint{3.799259in}{2.224136in}}%
\pgfpathlineto{\pgfqpoint{3.807227in}{2.233144in}}%
\pgfpathlineto{\pgfqpoint{3.815190in}{2.242170in}}%
\pgfpathlineto{\pgfqpoint{3.801626in}{2.245403in}}%
\pgfpathlineto{\pgfqpoint{3.788068in}{2.248719in}}%
\pgfpathlineto{\pgfqpoint{3.774516in}{2.252119in}}%
\pgfpathlineto{\pgfqpoint{3.760970in}{2.255602in}}%
\pgfpathlineto{\pgfqpoint{3.752997in}{2.246489in}}%
\pgfpathlineto{\pgfqpoint{3.745018in}{2.237399in}}%
\pgfpathlineto{\pgfqpoint{3.737034in}{2.228331in}}%
\pgfpathlineto{\pgfqpoint{3.729043in}{2.219282in}}%
\pgfpathclose%
\pgfusepath{fill}%
\end{pgfscope}%
\begin{pgfscope}%
\pgfpathrectangle{\pgfqpoint{1.150000in}{0.150000in}}{\pgfqpoint{5.700000in}{5.700000in}}%
\pgfusepath{clip}%
\pgfsetbuttcap%
\pgfsetroundjoin%
\definecolor{currentfill}{rgb}{0.278791,0.062145,0.386592}%
\pgfsetfillcolor{currentfill}%
\pgfsetfillopacity{0.700000}%
\pgfsetlinewidth{0.000000pt}%
\definecolor{currentstroke}{rgb}{0.000000,0.000000,0.000000}%
\pgfsetstrokecolor{currentstroke}%
\pgfsetdash{}{0pt}%
\pgfpathmoveto{\pgfqpoint{3.502400in}{2.193761in}}%
\pgfpathlineto{\pgfqpoint{3.515914in}{2.189419in}}%
\pgfpathlineto{\pgfqpoint{3.529433in}{2.185165in}}%
\pgfpathlineto{\pgfqpoint{3.542958in}{2.180999in}}%
\pgfpathlineto{\pgfqpoint{3.556487in}{2.176921in}}%
\pgfpathlineto{\pgfqpoint{3.564545in}{2.185930in}}%
\pgfpathlineto{\pgfqpoint{3.572598in}{2.194957in}}%
\pgfpathlineto{\pgfqpoint{3.580645in}{2.204004in}}%
\pgfpathlineto{\pgfqpoint{3.588686in}{2.213072in}}%
\pgfpathlineto{\pgfqpoint{3.575167in}{2.217188in}}%
\pgfpathlineto{\pgfqpoint{3.561654in}{2.221392in}}%
\pgfpathlineto{\pgfqpoint{3.548146in}{2.225685in}}%
\pgfpathlineto{\pgfqpoint{3.534643in}{2.230066in}}%
\pgfpathlineto{\pgfqpoint{3.526591in}{2.220952in}}%
\pgfpathlineto{\pgfqpoint{3.518533in}{2.211864in}}%
\pgfpathlineto{\pgfqpoint{3.510469in}{2.202801in}}%
\pgfpathlineto{\pgfqpoint{3.502400in}{2.193761in}}%
\pgfpathclose%
\pgfusepath{fill}%
\end{pgfscope}%
\begin{pgfscope}%
\pgfpathrectangle{\pgfqpoint{1.150000in}{0.150000in}}{\pgfqpoint{5.700000in}{5.700000in}}%
\pgfusepath{clip}%
\pgfsetbuttcap%
\pgfsetroundjoin%
\definecolor{currentfill}{rgb}{0.277134,0.185228,0.489898}%
\pgfsetfillcolor{currentfill}%
\pgfsetfillopacity{0.700000}%
\pgfsetlinewidth{0.000000pt}%
\definecolor{currentstroke}{rgb}{0.000000,0.000000,0.000000}%
\pgfsetstrokecolor{currentstroke}%
\pgfsetdash{}{0pt}%
\pgfpathmoveto{\pgfqpoint{2.408630in}{2.456952in}}%
\pgfpathlineto{\pgfqpoint{2.422143in}{2.445281in}}%
\pgfpathlineto{\pgfqpoint{2.435654in}{2.433748in}}%
\pgfpathlineto{\pgfqpoint{2.449163in}{2.422353in}}%
\pgfpathlineto{\pgfqpoint{2.462670in}{2.411095in}}%
\pgfpathlineto{\pgfqpoint{2.471173in}{2.417887in}}%
\pgfpathlineto{\pgfqpoint{2.479666in}{2.424782in}}%
\pgfpathlineto{\pgfqpoint{2.488149in}{2.431780in}}%
\pgfpathlineto{\pgfqpoint{2.496622in}{2.438878in}}%
\pgfpathlineto{\pgfqpoint{2.483136in}{2.450028in}}%
\pgfpathlineto{\pgfqpoint{2.469648in}{2.461315in}}%
\pgfpathlineto{\pgfqpoint{2.456158in}{2.472739in}}%
\pgfpathlineto{\pgfqpoint{2.442666in}{2.484301in}}%
\pgfpathlineto{\pgfqpoint{2.434172in}{2.477304in}}%
\pgfpathlineto{\pgfqpoint{2.425668in}{2.470412in}}%
\pgfpathlineto{\pgfqpoint{2.417154in}{2.463628in}}%
\pgfpathlineto{\pgfqpoint{2.408630in}{2.456952in}}%
\pgfpathclose%
\pgfusepath{fill}%
\end{pgfscope}%
\begin{pgfscope}%
\pgfpathrectangle{\pgfqpoint{1.150000in}{0.150000in}}{\pgfqpoint{5.700000in}{5.700000in}}%
\pgfusepath{clip}%
\pgfsetbuttcap%
\pgfsetroundjoin%
\definecolor{currentfill}{rgb}{0.216210,0.351535,0.550627}%
\pgfsetfillcolor{currentfill}%
\pgfsetfillopacity{0.700000}%
\pgfsetlinewidth{0.000000pt}%
\definecolor{currentstroke}{rgb}{0.000000,0.000000,0.000000}%
\pgfsetstrokecolor{currentstroke}%
\pgfsetdash{}{0pt}%
\pgfpathmoveto{\pgfqpoint{5.862725in}{2.803212in}}%
\pgfpathlineto{\pgfqpoint{5.876839in}{2.801651in}}%
\pgfpathlineto{\pgfqpoint{5.890964in}{2.800155in}}%
\pgfpathlineto{\pgfqpoint{5.905098in}{2.798724in}}%
\pgfpathlineto{\pgfqpoint{5.919242in}{2.797357in}}%
\pgfpathlineto{\pgfqpoint{5.926552in}{2.808306in}}%
\pgfpathlineto{\pgfqpoint{5.933870in}{2.819578in}}%
\pgfpathlineto{\pgfqpoint{5.941195in}{2.831183in}}%
\pgfpathlineto{\pgfqpoint{5.948528in}{2.843131in}}%
\pgfpathlineto{\pgfqpoint{5.934411in}{2.845003in}}%
\pgfpathlineto{\pgfqpoint{5.920303in}{2.846940in}}%
\pgfpathlineto{\pgfqpoint{5.906206in}{2.848941in}}%
\pgfpathlineto{\pgfqpoint{5.892118in}{2.851007in}}%
\pgfpathlineto{\pgfqpoint{5.884758in}{2.838547in}}%
\pgfpathlineto{\pgfqpoint{5.877406in}{2.826434in}}%
\pgfpathlineto{\pgfqpoint{5.870062in}{2.814659in}}%
\pgfpathlineto{\pgfqpoint{5.862725in}{2.803212in}}%
\pgfpathclose%
\pgfusepath{fill}%
\end{pgfscope}%
\begin{pgfscope}%
\pgfpathrectangle{\pgfqpoint{1.150000in}{0.150000in}}{\pgfqpoint{5.700000in}{5.700000in}}%
\pgfusepath{clip}%
\pgfsetbuttcap%
\pgfsetroundjoin%
\definecolor{currentfill}{rgb}{0.283072,0.130895,0.449241}%
\pgfsetfillcolor{currentfill}%
\pgfsetfillopacity{0.700000}%
\pgfsetlinewidth{0.000000pt}%
\definecolor{currentstroke}{rgb}{0.000000,0.000000,0.000000}%
\pgfsetstrokecolor{currentstroke}%
\pgfsetdash{}{0pt}%
\pgfpathmoveto{\pgfqpoint{4.268284in}{2.324900in}}%
\pgfpathlineto{\pgfqpoint{4.281974in}{2.323126in}}%
\pgfpathlineto{\pgfqpoint{4.295672in}{2.321428in}}%
\pgfpathlineto{\pgfqpoint{4.309377in}{2.319806in}}%
\pgfpathlineto{\pgfqpoint{4.323090in}{2.318259in}}%
\pgfpathlineto{\pgfqpoint{4.330880in}{2.326740in}}%
\pgfpathlineto{\pgfqpoint{4.338665in}{2.335251in}}%
\pgfpathlineto{\pgfqpoint{4.346444in}{2.343795in}}%
\pgfpathlineto{\pgfqpoint{4.354218in}{2.352377in}}%
\pgfpathlineto{\pgfqpoint{4.340518in}{2.354105in}}%
\pgfpathlineto{\pgfqpoint{4.326824in}{2.355909in}}%
\pgfpathlineto{\pgfqpoint{4.313139in}{2.357788in}}%
\pgfpathlineto{\pgfqpoint{4.299460in}{2.359743in}}%
\pgfpathlineto{\pgfqpoint{4.291674in}{2.350973in}}%
\pgfpathlineto{\pgfqpoint{4.283883in}{2.342245in}}%
\pgfpathlineto{\pgfqpoint{4.276086in}{2.333555in}}%
\pgfpathlineto{\pgfqpoint{4.268284in}{2.324900in}}%
\pgfpathclose%
\pgfusepath{fill}%
\end{pgfscope}%
\begin{pgfscope}%
\pgfpathrectangle{\pgfqpoint{1.150000in}{0.150000in}}{\pgfqpoint{5.700000in}{5.700000in}}%
\pgfusepath{clip}%
\pgfsetbuttcap%
\pgfsetroundjoin%
\definecolor{currentfill}{rgb}{0.279574,0.170599,0.479997}%
\pgfsetfillcolor{currentfill}%
\pgfsetfillopacity{0.700000}%
\pgfsetlinewidth{0.000000pt}%
\definecolor{currentstroke}{rgb}{0.000000,0.000000,0.000000}%
\pgfsetstrokecolor{currentstroke}%
\pgfsetdash{}{0pt}%
\pgfpathmoveto{\pgfqpoint{4.580949in}{2.402787in}}%
\pgfpathlineto{\pgfqpoint{4.594728in}{2.401595in}}%
\pgfpathlineto{\pgfqpoint{4.608516in}{2.400477in}}%
\pgfpathlineto{\pgfqpoint{4.622312in}{2.399431in}}%
\pgfpathlineto{\pgfqpoint{4.636116in}{2.398457in}}%
\pgfpathlineto{\pgfqpoint{4.643792in}{2.406604in}}%
\pgfpathlineto{\pgfqpoint{4.651463in}{2.414808in}}%
\pgfpathlineto{\pgfqpoint{4.659128in}{2.423073in}}%
\pgfpathlineto{\pgfqpoint{4.666790in}{2.431404in}}%
\pgfpathlineto{\pgfqpoint{4.652999in}{2.432621in}}%
\pgfpathlineto{\pgfqpoint{4.639217in}{2.433910in}}%
\pgfpathlineto{\pgfqpoint{4.625443in}{2.435271in}}%
\pgfpathlineto{\pgfqpoint{4.611678in}{2.436705in}}%
\pgfpathlineto{\pgfqpoint{4.604003in}{2.428124in}}%
\pgfpathlineto{\pgfqpoint{4.596323in}{2.419614in}}%
\pgfpathlineto{\pgfqpoint{4.588638in}{2.411169in}}%
\pgfpathlineto{\pgfqpoint{4.580949in}{2.402787in}}%
\pgfpathclose%
\pgfusepath{fill}%
\end{pgfscope}%
\begin{pgfscope}%
\pgfpathrectangle{\pgfqpoint{1.150000in}{0.150000in}}{\pgfqpoint{5.700000in}{5.700000in}}%
\pgfusepath{clip}%
\pgfsetbuttcap%
\pgfsetroundjoin%
\definecolor{currentfill}{rgb}{0.282327,0.094955,0.417331}%
\pgfsetfillcolor{currentfill}%
\pgfsetfillopacity{0.700000}%
\pgfsetlinewidth{0.000000pt}%
\definecolor{currentstroke}{rgb}{0.000000,0.000000,0.000000}%
\pgfsetstrokecolor{currentstroke}%
\pgfsetdash{}{0pt}%
\pgfpathmoveto{\pgfqpoint{3.955632in}{2.254786in}}%
\pgfpathlineto{\pgfqpoint{3.969242in}{2.252187in}}%
\pgfpathlineto{\pgfqpoint{3.982858in}{2.249668in}}%
\pgfpathlineto{\pgfqpoint{3.996482in}{2.247228in}}%
\pgfpathlineto{\pgfqpoint{4.010112in}{2.244868in}}%
\pgfpathlineto{\pgfqpoint{4.018014in}{2.253677in}}%
\pgfpathlineto{\pgfqpoint{4.025911in}{2.262502in}}%
\pgfpathlineto{\pgfqpoint{4.033801in}{2.271345in}}%
\pgfpathlineto{\pgfqpoint{4.041687in}{2.280209in}}%
\pgfpathlineto{\pgfqpoint{4.028068in}{2.282689in}}%
\pgfpathlineto{\pgfqpoint{4.014455in}{2.285249in}}%
\pgfpathlineto{\pgfqpoint{4.000849in}{2.287888in}}%
\pgfpathlineto{\pgfqpoint{3.987250in}{2.290607in}}%
\pgfpathlineto{\pgfqpoint{3.979354in}{2.281616in}}%
\pgfpathlineto{\pgfqpoint{3.971452in}{2.272650in}}%
\pgfpathlineto{\pgfqpoint{3.963545in}{2.263708in}}%
\pgfpathlineto{\pgfqpoint{3.955632in}{2.254786in}}%
\pgfpathclose%
\pgfusepath{fill}%
\end{pgfscope}%
\begin{pgfscope}%
\pgfpathrectangle{\pgfqpoint{1.150000in}{0.150000in}}{\pgfqpoint{5.700000in}{5.700000in}}%
\pgfusepath{clip}%
\pgfsetbuttcap%
\pgfsetroundjoin%
\definecolor{currentfill}{rgb}{0.283229,0.120777,0.440584}%
\pgfsetfillcolor{currentfill}%
\pgfsetfillopacity{0.700000}%
\pgfsetlinewidth{0.000000pt}%
\definecolor{currentstroke}{rgb}{0.000000,0.000000,0.000000}%
\pgfsetstrokecolor{currentstroke}%
\pgfsetdash{}{0pt}%
\pgfpathmoveto{\pgfqpoint{2.658360in}{2.315283in}}%
\pgfpathlineto{\pgfqpoint{2.671834in}{2.305800in}}%
\pgfpathlineto{\pgfqpoint{2.685308in}{2.296439in}}%
\pgfpathlineto{\pgfqpoint{2.698782in}{2.287197in}}%
\pgfpathlineto{\pgfqpoint{2.712257in}{2.278075in}}%
\pgfpathlineto{\pgfqpoint{2.720645in}{2.285650in}}%
\pgfpathlineto{\pgfqpoint{2.729025in}{2.293304in}}%
\pgfpathlineto{\pgfqpoint{2.737397in}{2.301035in}}%
\pgfpathlineto{\pgfqpoint{2.745761in}{2.308843in}}%
\pgfpathlineto{\pgfqpoint{2.732304in}{2.317879in}}%
\pgfpathlineto{\pgfqpoint{2.718848in}{2.327035in}}%
\pgfpathlineto{\pgfqpoint{2.705391in}{2.336311in}}%
\pgfpathlineto{\pgfqpoint{2.691935in}{2.345707in}}%
\pgfpathlineto{\pgfqpoint{2.683554in}{2.337978in}}%
\pgfpathlineto{\pgfqpoint{2.675165in}{2.330330in}}%
\pgfpathlineto{\pgfqpoint{2.666767in}{2.322765in}}%
\pgfpathlineto{\pgfqpoint{2.658360in}{2.315283in}}%
\pgfpathclose%
\pgfusepath{fill}%
\end{pgfscope}%
\begin{pgfscope}%
\pgfpathrectangle{\pgfqpoint{1.150000in}{0.150000in}}{\pgfqpoint{5.700000in}{5.700000in}}%
\pgfusepath{clip}%
\pgfsetbuttcap%
\pgfsetroundjoin%
\definecolor{currentfill}{rgb}{0.277941,0.056324,0.381191}%
\pgfsetfillcolor{currentfill}%
\pgfsetfillopacity{0.700000}%
\pgfsetlinewidth{0.000000pt}%
\definecolor{currentstroke}{rgb}{0.000000,0.000000,0.000000}%
\pgfsetstrokecolor{currentstroke}%
\pgfsetdash{}{0pt}%
\pgfpathmoveto{\pgfqpoint{3.134987in}{2.193162in}}%
\pgfpathlineto{\pgfqpoint{3.148458in}{2.186938in}}%
\pgfpathlineto{\pgfqpoint{3.161932in}{2.180813in}}%
\pgfpathlineto{\pgfqpoint{3.175409in}{2.174786in}}%
\pgfpathlineto{\pgfqpoint{3.188890in}{2.168857in}}%
\pgfpathlineto{\pgfqpoint{3.197082in}{2.177548in}}%
\pgfpathlineto{\pgfqpoint{3.205268in}{2.186276in}}%
\pgfpathlineto{\pgfqpoint{3.213448in}{2.195042in}}%
\pgfpathlineto{\pgfqpoint{3.221621in}{2.203845in}}%
\pgfpathlineto{\pgfqpoint{3.208153in}{2.209751in}}%
\pgfpathlineto{\pgfqpoint{3.194689in}{2.215755in}}%
\pgfpathlineto{\pgfqpoint{3.181229in}{2.221857in}}%
\pgfpathlineto{\pgfqpoint{3.167771in}{2.228058in}}%
\pgfpathlineto{\pgfqpoint{3.159585in}{2.219270in}}%
\pgfpathlineto{\pgfqpoint{3.151392in}{2.210525in}}%
\pgfpathlineto{\pgfqpoint{3.143193in}{2.201823in}}%
\pgfpathlineto{\pgfqpoint{3.134987in}{2.193162in}}%
\pgfpathclose%
\pgfusepath{fill}%
\end{pgfscope}%
\begin{pgfscope}%
\pgfpathrectangle{\pgfqpoint{1.150000in}{0.150000in}}{\pgfqpoint{5.700000in}{5.700000in}}%
\pgfusepath{clip}%
\pgfsetbuttcap%
\pgfsetroundjoin%
\definecolor{currentfill}{rgb}{0.223925,0.334994,0.548053}%
\pgfsetfillcolor{currentfill}%
\pgfsetfillopacity{0.700000}%
\pgfsetlinewidth{0.000000pt}%
\definecolor{currentstroke}{rgb}{0.000000,0.000000,0.000000}%
\pgfsetstrokecolor{currentstroke}%
\pgfsetdash{}{0pt}%
\pgfpathmoveto{\pgfqpoint{5.776973in}{2.765486in}}%
\pgfpathlineto{\pgfqpoint{5.791074in}{2.764150in}}%
\pgfpathlineto{\pgfqpoint{5.805185in}{2.762880in}}%
\pgfpathlineto{\pgfqpoint{5.819306in}{2.761675in}}%
\pgfpathlineto{\pgfqpoint{5.833437in}{2.760535in}}%
\pgfpathlineto{\pgfqpoint{5.840751in}{2.770755in}}%
\pgfpathlineto{\pgfqpoint{5.848070in}{2.781269in}}%
\pgfpathlineto{\pgfqpoint{5.855394in}{2.792085in}}%
\pgfpathlineto{\pgfqpoint{5.862725in}{2.803212in}}%
\pgfpathlineto{\pgfqpoint{5.848620in}{2.804838in}}%
\pgfpathlineto{\pgfqpoint{5.834525in}{2.806529in}}%
\pgfpathlineto{\pgfqpoint{5.820439in}{2.808284in}}%
\pgfpathlineto{\pgfqpoint{5.806364in}{2.810105in}}%
\pgfpathlineto{\pgfqpoint{5.799007in}{2.798485in}}%
\pgfpathlineto{\pgfqpoint{5.791657in}{2.787181in}}%
\pgfpathlineto{\pgfqpoint{5.784312in}{2.776184in}}%
\pgfpathlineto{\pgfqpoint{5.776973in}{2.765486in}}%
\pgfpathclose%
\pgfusepath{fill}%
\end{pgfscope}%
\begin{pgfscope}%
\pgfpathrectangle{\pgfqpoint{1.150000in}{0.150000in}}{\pgfqpoint{5.700000in}{5.700000in}}%
\pgfusepath{clip}%
\pgfsetbuttcap%
\pgfsetroundjoin%
\definecolor{currentfill}{rgb}{0.279566,0.067836,0.391917}%
\pgfsetfillcolor{currentfill}%
\pgfsetfillopacity{0.700000}%
\pgfsetlinewidth{0.000000pt}%
\definecolor{currentstroke}{rgb}{0.000000,0.000000,0.000000}%
\pgfsetstrokecolor{currentstroke}%
\pgfsetdash{}{0pt}%
\pgfpathmoveto{\pgfqpoint{2.994318in}{2.212724in}}%
\pgfpathlineto{\pgfqpoint{3.007782in}{2.205642in}}%
\pgfpathlineto{\pgfqpoint{3.021248in}{2.198664in}}%
\pgfpathlineto{\pgfqpoint{3.034716in}{2.191789in}}%
\pgfpathlineto{\pgfqpoint{3.048187in}{2.185017in}}%
\pgfpathlineto{\pgfqpoint{3.056434in}{2.193457in}}%
\pgfpathlineto{\pgfqpoint{3.064674in}{2.201944in}}%
\pgfpathlineto{\pgfqpoint{3.072907in}{2.210479in}}%
\pgfpathlineto{\pgfqpoint{3.081133in}{2.219061in}}%
\pgfpathlineto{\pgfqpoint{3.067677in}{2.225789in}}%
\pgfpathlineto{\pgfqpoint{3.054223in}{2.232620in}}%
\pgfpathlineto{\pgfqpoint{3.040771in}{2.239555in}}%
\pgfpathlineto{\pgfqpoint{3.027322in}{2.246593in}}%
\pgfpathlineto{\pgfqpoint{3.019082in}{2.238047in}}%
\pgfpathlineto{\pgfqpoint{3.010834in}{2.229554in}}%
\pgfpathlineto{\pgfqpoint{3.002580in}{2.221112in}}%
\pgfpathlineto{\pgfqpoint{2.994318in}{2.212724in}}%
\pgfpathclose%
\pgfusepath{fill}%
\end{pgfscope}%
\begin{pgfscope}%
\pgfpathrectangle{\pgfqpoint{1.150000in}{0.150000in}}{\pgfqpoint{5.700000in}{5.700000in}}%
\pgfusepath{clip}%
\pgfsetbuttcap%
\pgfsetroundjoin%
\definecolor{currentfill}{rgb}{0.253935,0.265254,0.529983}%
\pgfsetfillcolor{currentfill}%
\pgfsetfillopacity{0.700000}%
\pgfsetlinewidth{0.000000pt}%
\definecolor{currentstroke}{rgb}{0.000000,0.000000,0.000000}%
\pgfsetstrokecolor{currentstroke}%
\pgfsetdash{}{0pt}%
\pgfpathmoveto{\pgfqpoint{5.292361in}{2.602376in}}%
\pgfpathlineto{\pgfqpoint{5.306345in}{2.601575in}}%
\pgfpathlineto{\pgfqpoint{5.320339in}{2.600842in}}%
\pgfpathlineto{\pgfqpoint{5.334342in}{2.600176in}}%
\pgfpathlineto{\pgfqpoint{5.348355in}{2.599578in}}%
\pgfpathlineto{\pgfqpoint{5.355778in}{2.607883in}}%
\pgfpathlineto{\pgfqpoint{5.363200in}{2.616357in}}%
\pgfpathlineto{\pgfqpoint{5.370621in}{2.625007in}}%
\pgfpathlineto{\pgfqpoint{5.378042in}{2.633840in}}%
\pgfpathlineto{\pgfqpoint{5.364050in}{2.634823in}}%
\pgfpathlineto{\pgfqpoint{5.350067in}{2.635874in}}%
\pgfpathlineto{\pgfqpoint{5.336094in}{2.636993in}}%
\pgfpathlineto{\pgfqpoint{5.322130in}{2.638178in}}%
\pgfpathlineto{\pgfqpoint{5.314688in}{2.628953in}}%
\pgfpathlineto{\pgfqpoint{5.307247in}{2.619916in}}%
\pgfpathlineto{\pgfqpoint{5.299804in}{2.611059in}}%
\pgfpathlineto{\pgfqpoint{5.292361in}{2.602376in}}%
\pgfpathclose%
\pgfusepath{fill}%
\end{pgfscope}%
\begin{pgfscope}%
\pgfpathrectangle{\pgfqpoint{1.150000in}{0.150000in}}{\pgfqpoint{5.700000in}{5.700000in}}%
\pgfusepath{clip}%
\pgfsetbuttcap%
\pgfsetroundjoin%
\definecolor{currentfill}{rgb}{0.271828,0.209303,0.504434}%
\pgfsetfillcolor{currentfill}%
\pgfsetfillopacity{0.700000}%
\pgfsetlinewidth{0.000000pt}%
\definecolor{currentstroke}{rgb}{0.000000,0.000000,0.000000}%
\pgfsetstrokecolor{currentstroke}%
\pgfsetdash{}{0pt}%
\pgfpathmoveto{\pgfqpoint{4.893717in}{2.485404in}}%
\pgfpathlineto{\pgfqpoint{4.907589in}{2.484565in}}%
\pgfpathlineto{\pgfqpoint{4.921471in}{2.483796in}}%
\pgfpathlineto{\pgfqpoint{4.935362in}{2.483098in}}%
\pgfpathlineto{\pgfqpoint{4.949262in}{2.482469in}}%
\pgfpathlineto{\pgfqpoint{4.956822in}{2.490424in}}%
\pgfpathlineto{\pgfqpoint{4.964379in}{2.498475in}}%
\pgfpathlineto{\pgfqpoint{4.971932in}{2.506627in}}%
\pgfpathlineto{\pgfqpoint{4.979481in}{2.514886in}}%
\pgfpathlineto{\pgfqpoint{4.965598in}{2.515820in}}%
\pgfpathlineto{\pgfqpoint{4.951724in}{2.516823in}}%
\pgfpathlineto{\pgfqpoint{4.937859in}{2.517896in}}%
\pgfpathlineto{\pgfqpoint{4.924002in}{2.519038in}}%
\pgfpathlineto{\pgfqpoint{4.916436in}{2.510468in}}%
\pgfpathlineto{\pgfqpoint{4.908867in}{2.502009in}}%
\pgfpathlineto{\pgfqpoint{4.901294in}{2.493656in}}%
\pgfpathlineto{\pgfqpoint{4.893717in}{2.485404in}}%
\pgfpathclose%
\pgfusepath{fill}%
\end{pgfscope}%
\begin{pgfscope}%
\pgfpathrectangle{\pgfqpoint{1.150000in}{0.150000in}}{\pgfqpoint{5.700000in}{5.700000in}}%
\pgfusepath{clip}%
\pgfsetbuttcap%
\pgfsetroundjoin%
\definecolor{currentfill}{rgb}{0.277941,0.056324,0.381191}%
\pgfsetfillcolor{currentfill}%
\pgfsetfillopacity{0.700000}%
\pgfsetlinewidth{0.000000pt}%
\definecolor{currentstroke}{rgb}{0.000000,0.000000,0.000000}%
\pgfsetstrokecolor{currentstroke}%
\pgfsetdash{}{0pt}%
\pgfpathmoveto{\pgfqpoint{3.275526in}{2.181185in}}%
\pgfpathlineto{\pgfqpoint{3.289012in}{2.175759in}}%
\pgfpathlineto{\pgfqpoint{3.302501in}{2.170428in}}%
\pgfpathlineto{\pgfqpoint{3.315994in}{2.165190in}}%
\pgfpathlineto{\pgfqpoint{3.329492in}{2.160045in}}%
\pgfpathlineto{\pgfqpoint{3.337634in}{2.168906in}}%
\pgfpathlineto{\pgfqpoint{3.345769in}{2.177795in}}%
\pgfpathlineto{\pgfqpoint{3.353898in}{2.186712in}}%
\pgfpathlineto{\pgfqpoint{3.362021in}{2.195658in}}%
\pgfpathlineto{\pgfqpoint{3.348536in}{2.200800in}}%
\pgfpathlineto{\pgfqpoint{3.335055in}{2.206035in}}%
\pgfpathlineto{\pgfqpoint{3.321578in}{2.211365in}}%
\pgfpathlineto{\pgfqpoint{3.308104in}{2.216788in}}%
\pgfpathlineto{\pgfqpoint{3.299969in}{2.207837in}}%
\pgfpathlineto{\pgfqpoint{3.291828in}{2.198920in}}%
\pgfpathlineto{\pgfqpoint{3.283680in}{2.190036in}}%
\pgfpathlineto{\pgfqpoint{3.275526in}{2.181185in}}%
\pgfpathclose%
\pgfusepath{fill}%
\end{pgfscope}%
\begin{pgfscope}%
\pgfpathrectangle{\pgfqpoint{1.150000in}{0.150000in}}{\pgfqpoint{5.700000in}{5.700000in}}%
\pgfusepath{clip}%
\pgfsetbuttcap%
\pgfsetroundjoin%
\definecolor{currentfill}{rgb}{0.281446,0.084320,0.407414}%
\pgfsetfillcolor{currentfill}%
\pgfsetfillopacity{0.700000}%
\pgfsetlinewidth{0.000000pt}%
\definecolor{currentstroke}{rgb}{0.000000,0.000000,0.000000}%
\pgfsetstrokecolor{currentstroke}%
\pgfsetdash{}{0pt}%
\pgfpathmoveto{\pgfqpoint{2.853439in}{2.240720in}}%
\pgfpathlineto{\pgfqpoint{2.866903in}{2.232713in}}%
\pgfpathlineto{\pgfqpoint{2.880369in}{2.224817in}}%
\pgfpathlineto{\pgfqpoint{2.893836in}{2.217029in}}%
\pgfpathlineto{\pgfqpoint{2.907305in}{2.209351in}}%
\pgfpathlineto{\pgfqpoint{2.915612in}{2.217452in}}%
\pgfpathlineto{\pgfqpoint{2.923910in}{2.225612in}}%
\pgfpathlineto{\pgfqpoint{2.932202in}{2.233832in}}%
\pgfpathlineto{\pgfqpoint{2.940485in}{2.242110in}}%
\pgfpathlineto{\pgfqpoint{2.927032in}{2.249724in}}%
\pgfpathlineto{\pgfqpoint{2.913580in}{2.257447in}}%
\pgfpathlineto{\pgfqpoint{2.900130in}{2.265279in}}%
\pgfpathlineto{\pgfqpoint{2.886681in}{2.273221in}}%
\pgfpathlineto{\pgfqpoint{2.878382in}{2.265000in}}%
\pgfpathlineto{\pgfqpoint{2.870075in}{2.256842in}}%
\pgfpathlineto{\pgfqpoint{2.861761in}{2.248749in}}%
\pgfpathlineto{\pgfqpoint{2.853439in}{2.240720in}}%
\pgfpathclose%
\pgfusepath{fill}%
\end{pgfscope}%
\begin{pgfscope}%
\pgfpathrectangle{\pgfqpoint{1.150000in}{0.150000in}}{\pgfqpoint{5.700000in}{5.700000in}}%
\pgfusepath{clip}%
\pgfsetbuttcap%
\pgfsetroundjoin%
\definecolor{currentfill}{rgb}{0.229739,0.322361,0.545706}%
\pgfsetfillcolor{currentfill}%
\pgfsetfillopacity{0.700000}%
\pgfsetlinewidth{0.000000pt}%
\definecolor{currentstroke}{rgb}{0.000000,0.000000,0.000000}%
\pgfsetstrokecolor{currentstroke}%
\pgfsetdash{}{0pt}%
\pgfpathmoveto{\pgfqpoint{5.691251in}{2.729637in}}%
\pgfpathlineto{\pgfqpoint{5.705338in}{2.728505in}}%
\pgfpathlineto{\pgfqpoint{5.719435in}{2.727439in}}%
\pgfpathlineto{\pgfqpoint{5.733542in}{2.726439in}}%
\pgfpathlineto{\pgfqpoint{5.747660in}{2.725504in}}%
\pgfpathlineto{\pgfqpoint{5.754982in}{2.735094in}}%
\pgfpathlineto{\pgfqpoint{5.762308in}{2.744949in}}%
\pgfpathlineto{\pgfqpoint{5.769638in}{2.755077in}}%
\pgfpathlineto{\pgfqpoint{5.776973in}{2.765486in}}%
\pgfpathlineto{\pgfqpoint{5.762881in}{2.766886in}}%
\pgfpathlineto{\pgfqpoint{5.748799in}{2.768352in}}%
\pgfpathlineto{\pgfqpoint{5.734727in}{2.769884in}}%
\pgfpathlineto{\pgfqpoint{5.720664in}{2.771481in}}%
\pgfpathlineto{\pgfqpoint{5.713304in}{2.760599in}}%
\pgfpathlineto{\pgfqpoint{5.705949in}{2.750003in}}%
\pgfpathlineto{\pgfqpoint{5.698598in}{2.739685in}}%
\pgfpathlineto{\pgfqpoint{5.691251in}{2.729637in}}%
\pgfpathclose%
\pgfusepath{fill}%
\end{pgfscope}%
\begin{pgfscope}%
\pgfpathrectangle{\pgfqpoint{1.150000in}{0.150000in}}{\pgfqpoint{5.700000in}{5.700000in}}%
\pgfusepath{clip}%
\pgfsetbuttcap%
\pgfsetroundjoin%
\definecolor{currentfill}{rgb}{0.280255,0.165693,0.476498}%
\pgfsetfillcolor{currentfill}%
\pgfsetfillopacity{0.700000}%
\pgfsetlinewidth{0.000000pt}%
\definecolor{currentstroke}{rgb}{0.000000,0.000000,0.000000}%
\pgfsetstrokecolor{currentstroke}%
\pgfsetdash{}{0pt}%
\pgfpathmoveto{\pgfqpoint{2.462670in}{2.411095in}}%
\pgfpathlineto{\pgfqpoint{2.476175in}{2.399971in}}%
\pgfpathlineto{\pgfqpoint{2.489678in}{2.388982in}}%
\pgfpathlineto{\pgfqpoint{2.503179in}{2.378126in}}%
\pgfpathlineto{\pgfqpoint{2.516680in}{2.367401in}}%
\pgfpathlineto{\pgfqpoint{2.525162in}{2.374309in}}%
\pgfpathlineto{\pgfqpoint{2.533634in}{2.381315in}}%
\pgfpathlineto{\pgfqpoint{2.542098in}{2.388418in}}%
\pgfpathlineto{\pgfqpoint{2.550551in}{2.395616in}}%
\pgfpathlineto{\pgfqpoint{2.537071in}{2.406233in}}%
\pgfpathlineto{\pgfqpoint{2.523590in}{2.416982in}}%
\pgfpathlineto{\pgfqpoint{2.510107in}{2.427863in}}%
\pgfpathlineto{\pgfqpoint{2.496622in}{2.438878in}}%
\pgfpathlineto{\pgfqpoint{2.488149in}{2.431780in}}%
\pgfpathlineto{\pgfqpoint{2.479666in}{2.424782in}}%
\pgfpathlineto{\pgfqpoint{2.471173in}{2.417887in}}%
\pgfpathlineto{\pgfqpoint{2.462670in}{2.411095in}}%
\pgfpathclose%
\pgfusepath{fill}%
\end{pgfscope}%
\begin{pgfscope}%
\pgfpathrectangle{\pgfqpoint{1.150000in}{0.150000in}}{\pgfqpoint{5.700000in}{5.700000in}}%
\pgfusepath{clip}%
\pgfsetbuttcap%
\pgfsetroundjoin%
\definecolor{currentfill}{rgb}{0.283229,0.120777,0.440584}%
\pgfsetfillcolor{currentfill}%
\pgfsetfillopacity{0.700000}%
\pgfsetlinewidth{0.000000pt}%
\definecolor{currentstroke}{rgb}{0.000000,0.000000,0.000000}%
\pgfsetstrokecolor{currentstroke}%
\pgfsetdash{}{0pt}%
\pgfpathmoveto{\pgfqpoint{4.182289in}{2.297780in}}%
\pgfpathlineto{\pgfqpoint{4.195961in}{2.295862in}}%
\pgfpathlineto{\pgfqpoint{4.209640in}{2.294020in}}%
\pgfpathlineto{\pgfqpoint{4.223327in}{2.292255in}}%
\pgfpathlineto{\pgfqpoint{4.237021in}{2.290567in}}%
\pgfpathlineto{\pgfqpoint{4.244845in}{2.299113in}}%
\pgfpathlineto{\pgfqpoint{4.252664in}{2.307682in}}%
\pgfpathlineto{\pgfqpoint{4.260477in}{2.316277in}}%
\pgfpathlineto{\pgfqpoint{4.268284in}{2.324900in}}%
\pgfpathlineto{\pgfqpoint{4.254601in}{2.326750in}}%
\pgfpathlineto{\pgfqpoint{4.240926in}{2.328676in}}%
\pgfpathlineto{\pgfqpoint{4.227259in}{2.330679in}}%
\pgfpathlineto{\pgfqpoint{4.213598in}{2.332758in}}%
\pgfpathlineto{\pgfqpoint{4.205779in}{2.323966in}}%
\pgfpathlineto{\pgfqpoint{4.197955in}{2.315208in}}%
\pgfpathlineto{\pgfqpoint{4.190125in}{2.306481in}}%
\pgfpathlineto{\pgfqpoint{4.182289in}{2.297780in}}%
\pgfpathclose%
\pgfusepath{fill}%
\end{pgfscope}%
\begin{pgfscope}%
\pgfpathrectangle{\pgfqpoint{1.150000in}{0.150000in}}{\pgfqpoint{5.700000in}{5.700000in}}%
\pgfusepath{clip}%
\pgfsetbuttcap%
\pgfsetroundjoin%
\definecolor{currentfill}{rgb}{0.279566,0.067836,0.391917}%
\pgfsetfillcolor{currentfill}%
\pgfsetfillopacity{0.700000}%
\pgfsetlinewidth{0.000000pt}%
\definecolor{currentstroke}{rgb}{0.000000,0.000000,0.000000}%
\pgfsetstrokecolor{currentstroke}%
\pgfsetdash{}{0pt}%
\pgfpathmoveto{\pgfqpoint{3.642811in}{2.197474in}}%
\pgfpathlineto{\pgfqpoint{3.656356in}{2.193790in}}%
\pgfpathlineto{\pgfqpoint{3.669907in}{2.190191in}}%
\pgfpathlineto{\pgfqpoint{3.683463in}{2.186677in}}%
\pgfpathlineto{\pgfqpoint{3.697025in}{2.183248in}}%
\pgfpathlineto{\pgfqpoint{3.705038in}{2.192235in}}%
\pgfpathlineto{\pgfqpoint{3.713045in}{2.201235in}}%
\pgfpathlineto{\pgfqpoint{3.721047in}{2.210251in}}%
\pgfpathlineto{\pgfqpoint{3.729043in}{2.219282in}}%
\pgfpathlineto{\pgfqpoint{3.715492in}{2.222770in}}%
\pgfpathlineto{\pgfqpoint{3.701947in}{2.226343in}}%
\pgfpathlineto{\pgfqpoint{3.688407in}{2.230001in}}%
\pgfpathlineto{\pgfqpoint{3.674873in}{2.233744in}}%
\pgfpathlineto{\pgfqpoint{3.666866in}{2.224646in}}%
\pgfpathlineto{\pgfqpoint{3.658854in}{2.215569in}}%
\pgfpathlineto{\pgfqpoint{3.650835in}{2.206513in}}%
\pgfpathlineto{\pgfqpoint{3.642811in}{2.197474in}}%
\pgfpathclose%
\pgfusepath{fill}%
\end{pgfscope}%
\begin{pgfscope}%
\pgfpathrectangle{\pgfqpoint{1.150000in}{0.150000in}}{\pgfqpoint{5.700000in}{5.700000in}}%
\pgfusepath{clip}%
\pgfsetbuttcap%
\pgfsetroundjoin%
\definecolor{currentfill}{rgb}{0.280868,0.160771,0.472899}%
\pgfsetfillcolor{currentfill}%
\pgfsetfillopacity{0.700000}%
\pgfsetlinewidth{0.000000pt}%
\definecolor{currentstroke}{rgb}{0.000000,0.000000,0.000000}%
\pgfsetstrokecolor{currentstroke}%
\pgfsetdash{}{0pt}%
\pgfpathmoveto{\pgfqpoint{4.495052in}{2.374393in}}%
\pgfpathlineto{\pgfqpoint{4.508812in}{2.373132in}}%
\pgfpathlineto{\pgfqpoint{4.522580in}{2.371944in}}%
\pgfpathlineto{\pgfqpoint{4.536356in}{2.370830in}}%
\pgfpathlineto{\pgfqpoint{4.550141in}{2.369789in}}%
\pgfpathlineto{\pgfqpoint{4.557850in}{2.377967in}}%
\pgfpathlineto{\pgfqpoint{4.565555in}{2.386190in}}%
\pgfpathlineto{\pgfqpoint{4.573254in}{2.394462in}}%
\pgfpathlineto{\pgfqpoint{4.580949in}{2.402787in}}%
\pgfpathlineto{\pgfqpoint{4.567178in}{2.404051in}}%
\pgfpathlineto{\pgfqpoint{4.553415in}{2.405388in}}%
\pgfpathlineto{\pgfqpoint{4.539660in}{2.406798in}}%
\pgfpathlineto{\pgfqpoint{4.525914in}{2.408282in}}%
\pgfpathlineto{\pgfqpoint{4.518206in}{2.399726in}}%
\pgfpathlineto{\pgfqpoint{4.510493in}{2.391229in}}%
\pgfpathlineto{\pgfqpoint{4.502775in}{2.382786in}}%
\pgfpathlineto{\pgfqpoint{4.495052in}{2.374393in}}%
\pgfpathclose%
\pgfusepath{fill}%
\end{pgfscope}%
\begin{pgfscope}%
\pgfpathrectangle{\pgfqpoint{1.150000in}{0.150000in}}{\pgfqpoint{5.700000in}{5.700000in}}%
\pgfusepath{clip}%
\pgfsetbuttcap%
\pgfsetroundjoin%
\definecolor{currentfill}{rgb}{0.258965,0.251537,0.524736}%
\pgfsetfillcolor{currentfill}%
\pgfsetfillopacity{0.700000}%
\pgfsetlinewidth{0.000000pt}%
\definecolor{currentstroke}{rgb}{0.000000,0.000000,0.000000}%
\pgfsetstrokecolor{currentstroke}%
\pgfsetdash{}{0pt}%
\pgfpathmoveto{\pgfqpoint{5.206648in}{2.571666in}}%
\pgfpathlineto{\pgfqpoint{5.220615in}{2.570959in}}%
\pgfpathlineto{\pgfqpoint{5.234591in}{2.570320in}}%
\pgfpathlineto{\pgfqpoint{5.248576in}{2.569749in}}%
\pgfpathlineto{\pgfqpoint{5.262571in}{2.569246in}}%
\pgfpathlineto{\pgfqpoint{5.270021in}{2.577301in}}%
\pgfpathlineto{\pgfqpoint{5.277469in}{2.585503in}}%
\pgfpathlineto{\pgfqpoint{5.284916in}{2.593859in}}%
\pgfpathlineto{\pgfqpoint{5.292361in}{2.602376in}}%
\pgfpathlineto{\pgfqpoint{5.278386in}{2.603244in}}%
\pgfpathlineto{\pgfqpoint{5.264420in}{2.604180in}}%
\pgfpathlineto{\pgfqpoint{5.250464in}{2.605184in}}%
\pgfpathlineto{\pgfqpoint{5.236517in}{2.606256in}}%
\pgfpathlineto{\pgfqpoint{5.229052in}{2.597367in}}%
\pgfpathlineto{\pgfqpoint{5.221586in}{2.588644in}}%
\pgfpathlineto{\pgfqpoint{5.214118in}{2.580079in}}%
\pgfpathlineto{\pgfqpoint{5.206648in}{2.571666in}}%
\pgfpathclose%
\pgfusepath{fill}%
\end{pgfscope}%
\begin{pgfscope}%
\pgfpathrectangle{\pgfqpoint{1.150000in}{0.150000in}}{\pgfqpoint{5.700000in}{5.700000in}}%
\pgfusepath{clip}%
\pgfsetbuttcap%
\pgfsetroundjoin%
\definecolor{currentfill}{rgb}{0.281446,0.084320,0.407414}%
\pgfsetfillcolor{currentfill}%
\pgfsetfillopacity{0.700000}%
\pgfsetlinewidth{0.000000pt}%
\definecolor{currentstroke}{rgb}{0.000000,0.000000,0.000000}%
\pgfsetstrokecolor{currentstroke}%
\pgfsetdash{}{0pt}%
\pgfpathmoveto{\pgfqpoint{3.869506in}{2.230063in}}%
\pgfpathlineto{\pgfqpoint{3.883101in}{2.227241in}}%
\pgfpathlineto{\pgfqpoint{3.896702in}{2.224500in}}%
\pgfpathlineto{\pgfqpoint{3.910310in}{2.221839in}}%
\pgfpathlineto{\pgfqpoint{3.923924in}{2.219260in}}%
\pgfpathlineto{\pgfqpoint{3.931859in}{2.228122in}}%
\pgfpathlineto{\pgfqpoint{3.939789in}{2.236995in}}%
\pgfpathlineto{\pgfqpoint{3.947713in}{2.245882in}}%
\pgfpathlineto{\pgfqpoint{3.955632in}{2.254786in}}%
\pgfpathlineto{\pgfqpoint{3.942028in}{2.257465in}}%
\pgfpathlineto{\pgfqpoint{3.928431in}{2.260225in}}%
\pgfpathlineto{\pgfqpoint{3.914841in}{2.263066in}}%
\pgfpathlineto{\pgfqpoint{3.901257in}{2.265988in}}%
\pgfpathlineto{\pgfqpoint{3.893328in}{2.256977in}}%
\pgfpathlineto{\pgfqpoint{3.885393in}{2.247988in}}%
\pgfpathlineto{\pgfqpoint{3.877452in}{2.239017in}}%
\pgfpathlineto{\pgfqpoint{3.869506in}{2.230063in}}%
\pgfpathclose%
\pgfusepath{fill}%
\end{pgfscope}%
\begin{pgfscope}%
\pgfpathrectangle{\pgfqpoint{1.150000in}{0.150000in}}{\pgfqpoint{5.700000in}{5.700000in}}%
\pgfusepath{clip}%
\pgfsetbuttcap%
\pgfsetroundjoin%
\definecolor{currentfill}{rgb}{0.277941,0.056324,0.381191}%
\pgfsetfillcolor{currentfill}%
\pgfsetfillopacity{0.700000}%
\pgfsetlinewidth{0.000000pt}%
\definecolor{currentstroke}{rgb}{0.000000,0.000000,0.000000}%
\pgfsetstrokecolor{currentstroke}%
\pgfsetdash{}{0pt}%
\pgfpathmoveto{\pgfqpoint{3.416005in}{2.176012in}}%
\pgfpathlineto{\pgfqpoint{3.429512in}{2.171329in}}%
\pgfpathlineto{\pgfqpoint{3.443024in}{2.166736in}}%
\pgfpathlineto{\pgfqpoint{3.456540in}{2.162233in}}%
\pgfpathlineto{\pgfqpoint{3.470061in}{2.157820in}}%
\pgfpathlineto{\pgfqpoint{3.478155in}{2.166774in}}%
\pgfpathlineto{\pgfqpoint{3.486242in}{2.175749in}}%
\pgfpathlineto{\pgfqpoint{3.494324in}{2.184744in}}%
\pgfpathlineto{\pgfqpoint{3.502400in}{2.193761in}}%
\pgfpathlineto{\pgfqpoint{3.488890in}{2.198193in}}%
\pgfpathlineto{\pgfqpoint{3.475385in}{2.202714in}}%
\pgfpathlineto{\pgfqpoint{3.461885in}{2.207325in}}%
\pgfpathlineto{\pgfqpoint{3.448390in}{2.212026in}}%
\pgfpathlineto{\pgfqpoint{3.440303in}{2.202983in}}%
\pgfpathlineto{\pgfqpoint{3.432210in}{2.193967in}}%
\pgfpathlineto{\pgfqpoint{3.424110in}{2.184977in}}%
\pgfpathlineto{\pgfqpoint{3.416005in}{2.176012in}}%
\pgfpathclose%
\pgfusepath{fill}%
\end{pgfscope}%
\begin{pgfscope}%
\pgfpathrectangle{\pgfqpoint{1.150000in}{0.150000in}}{\pgfqpoint{5.700000in}{5.700000in}}%
\pgfusepath{clip}%
\pgfsetbuttcap%
\pgfsetroundjoin%
\definecolor{currentfill}{rgb}{0.235526,0.309527,0.542944}%
\pgfsetfillcolor{currentfill}%
\pgfsetfillopacity{0.700000}%
\pgfsetlinewidth{0.000000pt}%
\definecolor{currentstroke}{rgb}{0.000000,0.000000,0.000000}%
\pgfsetstrokecolor{currentstroke}%
\pgfsetdash{}{0pt}%
\pgfpathmoveto{\pgfqpoint{5.605542in}{2.695375in}}%
\pgfpathlineto{\pgfqpoint{5.619614in}{2.694427in}}%
\pgfpathlineto{\pgfqpoint{5.633697in}{2.693544in}}%
\pgfpathlineto{\pgfqpoint{5.647789in}{2.692727in}}%
\pgfpathlineto{\pgfqpoint{5.661891in}{2.691976in}}%
\pgfpathlineto{\pgfqpoint{5.669227in}{2.701027in}}%
\pgfpathlineto{\pgfqpoint{5.676566in}{2.710316in}}%
\pgfpathlineto{\pgfqpoint{5.683907in}{2.719850in}}%
\pgfpathlineto{\pgfqpoint{5.691251in}{2.729637in}}%
\pgfpathlineto{\pgfqpoint{5.677173in}{2.730834in}}%
\pgfpathlineto{\pgfqpoint{5.663105in}{2.732097in}}%
\pgfpathlineto{\pgfqpoint{5.649047in}{2.733425in}}%
\pgfpathlineto{\pgfqpoint{5.634998in}{2.734820in}}%
\pgfpathlineto{\pgfqpoint{5.627630in}{2.724579in}}%
\pgfpathlineto{\pgfqpoint{5.620265in}{2.714597in}}%
\pgfpathlineto{\pgfqpoint{5.612902in}{2.704865in}}%
\pgfpathlineto{\pgfqpoint{5.605542in}{2.695375in}}%
\pgfpathclose%
\pgfusepath{fill}%
\end{pgfscope}%
\begin{pgfscope}%
\pgfpathrectangle{\pgfqpoint{1.150000in}{0.150000in}}{\pgfqpoint{5.700000in}{5.700000in}}%
\pgfusepath{clip}%
\pgfsetbuttcap%
\pgfsetroundjoin%
\definecolor{currentfill}{rgb}{0.274128,0.199721,0.498911}%
\pgfsetfillcolor{currentfill}%
\pgfsetfillopacity{0.700000}%
\pgfsetlinewidth{0.000000pt}%
\definecolor{currentstroke}{rgb}{0.000000,0.000000,0.000000}%
\pgfsetstrokecolor{currentstroke}%
\pgfsetdash{}{0pt}%
\pgfpathmoveto{\pgfqpoint{4.807903in}{2.456220in}}%
\pgfpathlineto{\pgfqpoint{4.821756in}{2.455383in}}%
\pgfpathlineto{\pgfqpoint{4.835618in}{2.454617in}}%
\pgfpathlineto{\pgfqpoint{4.849489in}{2.453922in}}%
\pgfpathlineto{\pgfqpoint{4.863369in}{2.453297in}}%
\pgfpathlineto{\pgfqpoint{4.870962in}{2.461199in}}%
\pgfpathlineto{\pgfqpoint{4.878551in}{2.469181in}}%
\pgfpathlineto{\pgfqpoint{4.886136in}{2.477247in}}%
\pgfpathlineto{\pgfqpoint{4.893717in}{2.485404in}}%
\pgfpathlineto{\pgfqpoint{4.879853in}{2.486313in}}%
\pgfpathlineto{\pgfqpoint{4.865998in}{2.487293in}}%
\pgfpathlineto{\pgfqpoint{4.852151in}{2.488343in}}%
\pgfpathlineto{\pgfqpoint{4.838314in}{2.489463in}}%
\pgfpathlineto{\pgfqpoint{4.830717in}{2.481015in}}%
\pgfpathlineto{\pgfqpoint{4.823116in}{2.472662in}}%
\pgfpathlineto{\pgfqpoint{4.815512in}{2.464399in}}%
\pgfpathlineto{\pgfqpoint{4.807903in}{2.456220in}}%
\pgfpathclose%
\pgfusepath{fill}%
\end{pgfscope}%
\begin{pgfscope}%
\pgfpathrectangle{\pgfqpoint{1.150000in}{0.150000in}}{\pgfqpoint{5.700000in}{5.700000in}}%
\pgfusepath{clip}%
\pgfsetbuttcap%
\pgfsetroundjoin%
\definecolor{currentfill}{rgb}{0.282910,0.105393,0.426902}%
\pgfsetfillcolor{currentfill}%
\pgfsetfillopacity{0.700000}%
\pgfsetlinewidth{0.000000pt}%
\definecolor{currentstroke}{rgb}{0.000000,0.000000,0.000000}%
\pgfsetstrokecolor{currentstroke}%
\pgfsetdash{}{0pt}%
\pgfpathmoveto{\pgfqpoint{2.712257in}{2.278075in}}%
\pgfpathlineto{\pgfqpoint{2.725731in}{2.269070in}}%
\pgfpathlineto{\pgfqpoint{2.739206in}{2.260183in}}%
\pgfpathlineto{\pgfqpoint{2.752682in}{2.251412in}}%
\pgfpathlineto{\pgfqpoint{2.766158in}{2.242756in}}%
\pgfpathlineto{\pgfqpoint{2.774529in}{2.250425in}}%
\pgfpathlineto{\pgfqpoint{2.782892in}{2.258167in}}%
\pgfpathlineto{\pgfqpoint{2.791246in}{2.265981in}}%
\pgfpathlineto{\pgfqpoint{2.799593in}{2.273866in}}%
\pgfpathlineto{\pgfqpoint{2.786134in}{2.282437in}}%
\pgfpathlineto{\pgfqpoint{2.772675in}{2.291122in}}%
\pgfpathlineto{\pgfqpoint{2.759218in}{2.299924in}}%
\pgfpathlineto{\pgfqpoint{2.745761in}{2.308843in}}%
\pgfpathlineto{\pgfqpoint{2.737397in}{2.301035in}}%
\pgfpathlineto{\pgfqpoint{2.729025in}{2.293304in}}%
\pgfpathlineto{\pgfqpoint{2.720645in}{2.285650in}}%
\pgfpathlineto{\pgfqpoint{2.712257in}{2.278075in}}%
\pgfpathclose%
\pgfusepath{fill}%
\end{pgfscope}%
\begin{pgfscope}%
\pgfpathrectangle{\pgfqpoint{1.150000in}{0.150000in}}{\pgfqpoint{5.700000in}{5.700000in}}%
\pgfusepath{clip}%
\pgfsetbuttcap%
\pgfsetroundjoin%
\definecolor{currentfill}{rgb}{0.282290,0.145912,0.461510}%
\pgfsetfillcolor{currentfill}%
\pgfsetfillopacity{0.700000}%
\pgfsetlinewidth{0.000000pt}%
\definecolor{currentstroke}{rgb}{0.000000,0.000000,0.000000}%
\pgfsetstrokecolor{currentstroke}%
\pgfsetdash{}{0pt}%
\pgfpathmoveto{\pgfqpoint{2.516680in}{2.367401in}}%
\pgfpathlineto{\pgfqpoint{2.530178in}{2.356807in}}%
\pgfpathlineto{\pgfqpoint{2.543676in}{2.346342in}}%
\pgfpathlineto{\pgfqpoint{2.557173in}{2.336006in}}%
\pgfpathlineto{\pgfqpoint{2.570668in}{2.325797in}}%
\pgfpathlineto{\pgfqpoint{2.579130in}{2.332819in}}%
\pgfpathlineto{\pgfqpoint{2.587583in}{2.339935in}}%
\pgfpathlineto{\pgfqpoint{2.596027in}{2.347143in}}%
\pgfpathlineto{\pgfqpoint{2.604461in}{2.354441in}}%
\pgfpathlineto{\pgfqpoint{2.590985in}{2.364543in}}%
\pgfpathlineto{\pgfqpoint{2.577508in}{2.374773in}}%
\pgfpathlineto{\pgfqpoint{2.564030in}{2.385130in}}%
\pgfpathlineto{\pgfqpoint{2.550551in}{2.395616in}}%
\pgfpathlineto{\pgfqpoint{2.542098in}{2.388418in}}%
\pgfpathlineto{\pgfqpoint{2.533634in}{2.381315in}}%
\pgfpathlineto{\pgfqpoint{2.525162in}{2.374309in}}%
\pgfpathlineto{\pgfqpoint{2.516680in}{2.367401in}}%
\pgfpathclose%
\pgfusepath{fill}%
\end{pgfscope}%
\begin{pgfscope}%
\pgfpathrectangle{\pgfqpoint{1.150000in}{0.150000in}}{\pgfqpoint{5.700000in}{5.700000in}}%
\pgfusepath{clip}%
\pgfsetbuttcap%
\pgfsetroundjoin%
\definecolor{currentfill}{rgb}{0.283091,0.110553,0.431554}%
\pgfsetfillcolor{currentfill}%
\pgfsetfillopacity{0.700000}%
\pgfsetlinewidth{0.000000pt}%
\definecolor{currentstroke}{rgb}{0.000000,0.000000,0.000000}%
\pgfsetstrokecolor{currentstroke}%
\pgfsetdash{}{0pt}%
\pgfpathmoveto{\pgfqpoint{4.096233in}{2.271076in}}%
\pgfpathlineto{\pgfqpoint{4.109887in}{2.268988in}}%
\pgfpathlineto{\pgfqpoint{4.123548in}{2.266979in}}%
\pgfpathlineto{\pgfqpoint{4.137217in}{2.265047in}}%
\pgfpathlineto{\pgfqpoint{4.150893in}{2.263193in}}%
\pgfpathlineto{\pgfqpoint{4.158750in}{2.271813in}}%
\pgfpathlineto{\pgfqpoint{4.166602in}{2.280449in}}%
\pgfpathlineto{\pgfqpoint{4.174449in}{2.289104in}}%
\pgfpathlineto{\pgfqpoint{4.182289in}{2.297780in}}%
\pgfpathlineto{\pgfqpoint{4.168625in}{2.299776in}}%
\pgfpathlineto{\pgfqpoint{4.154968in}{2.301848in}}%
\pgfpathlineto{\pgfqpoint{4.141318in}{2.303999in}}%
\pgfpathlineto{\pgfqpoint{4.127675in}{2.306227in}}%
\pgfpathlineto{\pgfqpoint{4.119823in}{2.297402in}}%
\pgfpathlineto{\pgfqpoint{4.111965in}{2.288604in}}%
\pgfpathlineto{\pgfqpoint{4.104102in}{2.279830in}}%
\pgfpathlineto{\pgfqpoint{4.096233in}{2.271076in}}%
\pgfpathclose%
\pgfusepath{fill}%
\end{pgfscope}%
\begin{pgfscope}%
\pgfpathrectangle{\pgfqpoint{1.150000in}{0.150000in}}{\pgfqpoint{5.700000in}{5.700000in}}%
\pgfusepath{clip}%
\pgfsetbuttcap%
\pgfsetroundjoin%
\definecolor{currentfill}{rgb}{0.262138,0.242286,0.520837}%
\pgfsetfillcolor{currentfill}%
\pgfsetfillopacity{0.700000}%
\pgfsetlinewidth{0.000000pt}%
\definecolor{currentstroke}{rgb}{0.000000,0.000000,0.000000}%
\pgfsetstrokecolor{currentstroke}%
\pgfsetdash{}{0pt}%
\pgfpathmoveto{\pgfqpoint{5.120898in}{2.541540in}}%
\pgfpathlineto{\pgfqpoint{5.134846in}{2.540904in}}%
\pgfpathlineto{\pgfqpoint{5.148803in}{2.540338in}}%
\pgfpathlineto{\pgfqpoint{5.162770in}{2.539840in}}%
\pgfpathlineto{\pgfqpoint{5.176746in}{2.539410in}}%
\pgfpathlineto{\pgfqpoint{5.184225in}{2.547277in}}%
\pgfpathlineto{\pgfqpoint{5.191702in}{2.555271in}}%
\pgfpathlineto{\pgfqpoint{5.199176in}{2.563399in}}%
\pgfpathlineto{\pgfqpoint{5.206648in}{2.571666in}}%
\pgfpathlineto{\pgfqpoint{5.192691in}{2.572441in}}%
\pgfpathlineto{\pgfqpoint{5.178744in}{2.573285in}}%
\pgfpathlineto{\pgfqpoint{5.164805in}{2.574196in}}%
\pgfpathlineto{\pgfqpoint{5.150876in}{2.575177in}}%
\pgfpathlineto{\pgfqpoint{5.143385in}{2.566557in}}%
\pgfpathlineto{\pgfqpoint{5.135892in}{2.558082in}}%
\pgfpathlineto{\pgfqpoint{5.128396in}{2.549745in}}%
\pgfpathlineto{\pgfqpoint{5.120898in}{2.541540in}}%
\pgfpathclose%
\pgfusepath{fill}%
\end{pgfscope}%
\begin{pgfscope}%
\pgfpathrectangle{\pgfqpoint{1.150000in}{0.150000in}}{\pgfqpoint{5.700000in}{5.700000in}}%
\pgfusepath{clip}%
\pgfsetbuttcap%
\pgfsetroundjoin%
\definecolor{currentfill}{rgb}{0.282290,0.145912,0.461510}%
\pgfsetfillcolor{currentfill}%
\pgfsetfillopacity{0.700000}%
\pgfsetlinewidth{0.000000pt}%
\definecolor{currentstroke}{rgb}{0.000000,0.000000,0.000000}%
\pgfsetstrokecolor{currentstroke}%
\pgfsetdash{}{0pt}%
\pgfpathmoveto{\pgfqpoint{4.409100in}{2.346213in}}%
\pgfpathlineto{\pgfqpoint{4.422840in}{2.344859in}}%
\pgfpathlineto{\pgfqpoint{4.436588in}{2.343578in}}%
\pgfpathlineto{\pgfqpoint{4.450345in}{2.342372in}}%
\pgfpathlineto{\pgfqpoint{4.464109in}{2.341240in}}%
\pgfpathlineto{\pgfqpoint{4.471853in}{2.349473in}}%
\pgfpathlineto{\pgfqpoint{4.479591in}{2.357740in}}%
\pgfpathlineto{\pgfqpoint{4.487324in}{2.366046in}}%
\pgfpathlineto{\pgfqpoint{4.495052in}{2.374393in}}%
\pgfpathlineto{\pgfqpoint{4.481301in}{2.375728in}}%
\pgfpathlineto{\pgfqpoint{4.467557in}{2.377136in}}%
\pgfpathlineto{\pgfqpoint{4.453822in}{2.378619in}}%
\pgfpathlineto{\pgfqpoint{4.440095in}{2.380175in}}%
\pgfpathlineto{\pgfqpoint{4.432354in}{2.371618in}}%
\pgfpathlineto{\pgfqpoint{4.424608in}{2.363108in}}%
\pgfpathlineto{\pgfqpoint{4.416856in}{2.354641in}}%
\pgfpathlineto{\pgfqpoint{4.409100in}{2.346213in}}%
\pgfpathclose%
\pgfusepath{fill}%
\end{pgfscope}%
\begin{pgfscope}%
\pgfpathrectangle{\pgfqpoint{1.150000in}{0.150000in}}{\pgfqpoint{5.700000in}{5.700000in}}%
\pgfusepath{clip}%
\pgfsetbuttcap%
\pgfsetroundjoin%
\definecolor{currentfill}{rgb}{0.241237,0.296485,0.539709}%
\pgfsetfillcolor{currentfill}%
\pgfsetfillopacity{0.700000}%
\pgfsetlinewidth{0.000000pt}%
\definecolor{currentstroke}{rgb}{0.000000,0.000000,0.000000}%
\pgfsetstrokecolor{currentstroke}%
\pgfsetdash{}{0pt}%
\pgfpathmoveto{\pgfqpoint{5.519831in}{2.662436in}}%
\pgfpathlineto{\pgfqpoint{5.533888in}{2.661648in}}%
\pgfpathlineto{\pgfqpoint{5.547955in}{2.660927in}}%
\pgfpathlineto{\pgfqpoint{5.562031in}{2.660272in}}%
\pgfpathlineto{\pgfqpoint{5.576117in}{2.659684in}}%
\pgfpathlineto{\pgfqpoint{5.583472in}{2.668282in}}%
\pgfpathlineto{\pgfqpoint{5.590827in}{2.677091in}}%
\pgfpathlineto{\pgfqpoint{5.598184in}{2.686120in}}%
\pgfpathlineto{\pgfqpoint{5.605542in}{2.695375in}}%
\pgfpathlineto{\pgfqpoint{5.591479in}{2.696390in}}%
\pgfpathlineto{\pgfqpoint{5.577426in}{2.697471in}}%
\pgfpathlineto{\pgfqpoint{5.563383in}{2.698619in}}%
\pgfpathlineto{\pgfqpoint{5.549349in}{2.699832in}}%
\pgfpathlineto{\pgfqpoint{5.541967in}{2.690143in}}%
\pgfpathlineto{\pgfqpoint{5.534587in}{2.680686in}}%
\pgfpathlineto{\pgfqpoint{5.527209in}{2.671453in}}%
\pgfpathlineto{\pgfqpoint{5.519831in}{2.662436in}}%
\pgfpathclose%
\pgfusepath{fill}%
\end{pgfscope}%
\begin{pgfscope}%
\pgfpathrectangle{\pgfqpoint{1.150000in}{0.150000in}}{\pgfqpoint{5.700000in}{5.700000in}}%
\pgfusepath{clip}%
\pgfsetbuttcap%
\pgfsetroundjoin%
\definecolor{currentfill}{rgb}{0.199430,0.387607,0.554642}%
\pgfsetfillcolor{currentfill}%
\pgfsetfillopacity{0.700000}%
\pgfsetlinewidth{0.000000pt}%
\definecolor{currentstroke}{rgb}{0.000000,0.000000,0.000000}%
\pgfsetstrokecolor{currentstroke}%
\pgfsetdash{}{0pt}%
\pgfpathmoveto{\pgfqpoint{6.091014in}{2.877663in}}%
\pgfpathlineto{\pgfqpoint{6.105191in}{2.875844in}}%
\pgfpathlineto{\pgfqpoint{6.119378in}{2.874088in}}%
\pgfpathlineto{\pgfqpoint{6.133575in}{2.872397in}}%
\pgfpathlineto{\pgfqpoint{6.140877in}{2.884715in}}%
\pgfpathlineto{\pgfqpoint{6.148191in}{2.897420in}}%
\pgfpathlineto{\pgfqpoint{6.155516in}{2.910520in}}%
\pgfpathlineto{\pgfqpoint{6.162853in}{2.924025in}}%
\pgfpathlineto{\pgfqpoint{6.148686in}{2.926263in}}%
\pgfpathlineto{\pgfqpoint{6.134528in}{2.928564in}}%
\pgfpathlineto{\pgfqpoint{6.120380in}{2.930929in}}%
\pgfpathlineto{\pgfqpoint{6.113021in}{2.917009in}}%
\pgfpathlineto{\pgfqpoint{6.105674in}{2.903498in}}%
\pgfpathlineto{\pgfqpoint{6.098338in}{2.890386in}}%
\pgfpathlineto{\pgfqpoint{6.091014in}{2.877663in}}%
\pgfpathclose%
\pgfusepath{fill}%
\end{pgfscope}%
\begin{pgfscope}%
\pgfpathrectangle{\pgfqpoint{1.150000in}{0.150000in}}{\pgfqpoint{5.700000in}{5.700000in}}%
\pgfusepath{clip}%
\pgfsetbuttcap%
\pgfsetroundjoin%
\definecolor{currentfill}{rgb}{0.278791,0.062145,0.386592}%
\pgfsetfillcolor{currentfill}%
\pgfsetfillopacity{0.700000}%
\pgfsetlinewidth{0.000000pt}%
\definecolor{currentstroke}{rgb}{0.000000,0.000000,0.000000}%
\pgfsetstrokecolor{currentstroke}%
\pgfsetdash{}{0pt}%
\pgfpathmoveto{\pgfqpoint{3.048187in}{2.185017in}}%
\pgfpathlineto{\pgfqpoint{3.061660in}{2.178347in}}%
\pgfpathlineto{\pgfqpoint{3.075136in}{2.171779in}}%
\pgfpathlineto{\pgfqpoint{3.088615in}{2.165312in}}%
\pgfpathlineto{\pgfqpoint{3.102096in}{2.158945in}}%
\pgfpathlineto{\pgfqpoint{3.110329in}{2.167435in}}%
\pgfpathlineto{\pgfqpoint{3.118555in}{2.175969in}}%
\pgfpathlineto{\pgfqpoint{3.126775in}{2.184544in}}%
\pgfpathlineto{\pgfqpoint{3.134987in}{2.193162in}}%
\pgfpathlineto{\pgfqpoint{3.121519in}{2.199486in}}%
\pgfpathlineto{\pgfqpoint{3.108055in}{2.205910in}}%
\pgfpathlineto{\pgfqpoint{3.094593in}{2.212435in}}%
\pgfpathlineto{\pgfqpoint{3.081133in}{2.219061in}}%
\pgfpathlineto{\pgfqpoint{3.072907in}{2.210479in}}%
\pgfpathlineto{\pgfqpoint{3.064674in}{2.201944in}}%
\pgfpathlineto{\pgfqpoint{3.056434in}{2.193457in}}%
\pgfpathlineto{\pgfqpoint{3.048187in}{2.185017in}}%
\pgfpathclose%
\pgfusepath{fill}%
\end{pgfscope}%
\begin{pgfscope}%
\pgfpathrectangle{\pgfqpoint{1.150000in}{0.150000in}}{\pgfqpoint{5.700000in}{5.700000in}}%
\pgfusepath{clip}%
\pgfsetbuttcap%
\pgfsetroundjoin%
\definecolor{currentfill}{rgb}{0.276194,0.190074,0.493001}%
\pgfsetfillcolor{currentfill}%
\pgfsetfillopacity{0.700000}%
\pgfsetlinewidth{0.000000pt}%
\definecolor{currentstroke}{rgb}{0.000000,0.000000,0.000000}%
\pgfsetstrokecolor{currentstroke}%
\pgfsetdash{}{0pt}%
\pgfpathmoveto{\pgfqpoint{4.722036in}{2.427257in}}%
\pgfpathlineto{\pgfqpoint{4.735870in}{2.426399in}}%
\pgfpathlineto{\pgfqpoint{4.749712in}{2.425613in}}%
\pgfpathlineto{\pgfqpoint{4.763562in}{2.424898in}}%
\pgfpathlineto{\pgfqpoint{4.777422in}{2.424254in}}%
\pgfpathlineto{\pgfqpoint{4.785049in}{2.432143in}}%
\pgfpathlineto{\pgfqpoint{4.792671in}{2.440097in}}%
\pgfpathlineto{\pgfqpoint{4.800289in}{2.448121in}}%
\pgfpathlineto{\pgfqpoint{4.807903in}{2.456220in}}%
\pgfpathlineto{\pgfqpoint{4.794058in}{2.457128in}}%
\pgfpathlineto{\pgfqpoint{4.780223in}{2.458107in}}%
\pgfpathlineto{\pgfqpoint{4.766396in}{2.459157in}}%
\pgfpathlineto{\pgfqpoint{4.752577in}{2.460278in}}%
\pgfpathlineto{\pgfqpoint{4.744949in}{2.451908in}}%
\pgfpathlineto{\pgfqpoint{4.737316in}{2.443618in}}%
\pgfpathlineto{\pgfqpoint{4.729678in}{2.435402in}}%
\pgfpathlineto{\pgfqpoint{4.722036in}{2.427257in}}%
\pgfpathclose%
\pgfusepath{fill}%
\end{pgfscope}%
\begin{pgfscope}%
\pgfpathrectangle{\pgfqpoint{1.150000in}{0.150000in}}{\pgfqpoint{5.700000in}{5.700000in}}%
\pgfusepath{clip}%
\pgfsetbuttcap%
\pgfsetroundjoin%
\definecolor{currentfill}{rgb}{0.278791,0.062145,0.386592}%
\pgfsetfillcolor{currentfill}%
\pgfsetfillopacity{0.700000}%
\pgfsetlinewidth{0.000000pt}%
\definecolor{currentstroke}{rgb}{0.000000,0.000000,0.000000}%
\pgfsetstrokecolor{currentstroke}%
\pgfsetdash{}{0pt}%
\pgfpathmoveto{\pgfqpoint{3.556487in}{2.176921in}}%
\pgfpathlineto{\pgfqpoint{3.570021in}{2.172930in}}%
\pgfpathlineto{\pgfqpoint{3.583561in}{2.169026in}}%
\pgfpathlineto{\pgfqpoint{3.597106in}{2.165209in}}%
\pgfpathlineto{\pgfqpoint{3.610657in}{2.161478in}}%
\pgfpathlineto{\pgfqpoint{3.618704in}{2.170455in}}%
\pgfpathlineto{\pgfqpoint{3.626746in}{2.179447in}}%
\pgfpathlineto{\pgfqpoint{3.634781in}{2.188452in}}%
\pgfpathlineto{\pgfqpoint{3.642811in}{2.197474in}}%
\pgfpathlineto{\pgfqpoint{3.629272in}{2.201244in}}%
\pgfpathlineto{\pgfqpoint{3.615738in}{2.205100in}}%
\pgfpathlineto{\pgfqpoint{3.602209in}{2.209042in}}%
\pgfpathlineto{\pgfqpoint{3.588686in}{2.213072in}}%
\pgfpathlineto{\pgfqpoint{3.580645in}{2.204004in}}%
\pgfpathlineto{\pgfqpoint{3.572598in}{2.194957in}}%
\pgfpathlineto{\pgfqpoint{3.564545in}{2.185930in}}%
\pgfpathlineto{\pgfqpoint{3.556487in}{2.176921in}}%
\pgfpathclose%
\pgfusepath{fill}%
\end{pgfscope}%
\begin{pgfscope}%
\pgfpathrectangle{\pgfqpoint{1.150000in}{0.150000in}}{\pgfqpoint{5.700000in}{5.700000in}}%
\pgfusepath{clip}%
\pgfsetbuttcap%
\pgfsetroundjoin%
\definecolor{currentfill}{rgb}{0.277018,0.050344,0.375715}%
\pgfsetfillcolor{currentfill}%
\pgfsetfillopacity{0.700000}%
\pgfsetlinewidth{0.000000pt}%
\definecolor{currentstroke}{rgb}{0.000000,0.000000,0.000000}%
\pgfsetstrokecolor{currentstroke}%
\pgfsetdash{}{0pt}%
\pgfpathmoveto{\pgfqpoint{3.188890in}{2.168857in}}%
\pgfpathlineto{\pgfqpoint{3.202373in}{2.163025in}}%
\pgfpathlineto{\pgfqpoint{3.215861in}{2.157290in}}%
\pgfpathlineto{\pgfqpoint{3.229352in}{2.151651in}}%
\pgfpathlineto{\pgfqpoint{3.242846in}{2.146107in}}%
\pgfpathlineto{\pgfqpoint{3.251026in}{2.154828in}}%
\pgfpathlineto{\pgfqpoint{3.259199in}{2.163582in}}%
\pgfpathlineto{\pgfqpoint{3.267366in}{2.172367in}}%
\pgfpathlineto{\pgfqpoint{3.275526in}{2.181185in}}%
\pgfpathlineto{\pgfqpoint{3.262044in}{2.186707in}}%
\pgfpathlineto{\pgfqpoint{3.248566in}{2.192323in}}%
\pgfpathlineto{\pgfqpoint{3.235092in}{2.198035in}}%
\pgfpathlineto{\pgfqpoint{3.221621in}{2.203845in}}%
\pgfpathlineto{\pgfqpoint{3.213448in}{2.195042in}}%
\pgfpathlineto{\pgfqpoint{3.205268in}{2.186276in}}%
\pgfpathlineto{\pgfqpoint{3.197082in}{2.177548in}}%
\pgfpathlineto{\pgfqpoint{3.188890in}{2.168857in}}%
\pgfpathclose%
\pgfusepath{fill}%
\end{pgfscope}%
\begin{pgfscope}%
\pgfpathrectangle{\pgfqpoint{1.150000in}{0.150000in}}{\pgfqpoint{5.700000in}{5.700000in}}%
\pgfusepath{clip}%
\pgfsetbuttcap%
\pgfsetroundjoin%
\definecolor{currentfill}{rgb}{0.280894,0.078907,0.402329}%
\pgfsetfillcolor{currentfill}%
\pgfsetfillopacity{0.700000}%
\pgfsetlinewidth{0.000000pt}%
\definecolor{currentstroke}{rgb}{0.000000,0.000000,0.000000}%
\pgfsetstrokecolor{currentstroke}%
\pgfsetdash{}{0pt}%
\pgfpathmoveto{\pgfqpoint{3.783306in}{2.206167in}}%
\pgfpathlineto{\pgfqpoint{3.796886in}{2.203096in}}%
\pgfpathlineto{\pgfqpoint{3.810473in}{2.200108in}}%
\pgfpathlineto{\pgfqpoint{3.824066in}{2.197202in}}%
\pgfpathlineto{\pgfqpoint{3.837665in}{2.194378in}}%
\pgfpathlineto{\pgfqpoint{3.845634in}{2.203283in}}%
\pgfpathlineto{\pgfqpoint{3.853597in}{2.212198in}}%
\pgfpathlineto{\pgfqpoint{3.861554in}{2.221124in}}%
\pgfpathlineto{\pgfqpoint{3.869506in}{2.230063in}}%
\pgfpathlineto{\pgfqpoint{3.855918in}{2.232967in}}%
\pgfpathlineto{\pgfqpoint{3.842336in}{2.235952in}}%
\pgfpathlineto{\pgfqpoint{3.828760in}{2.239020in}}%
\pgfpathlineto{\pgfqpoint{3.815190in}{2.242170in}}%
\pgfpathlineto{\pgfqpoint{3.807227in}{2.233144in}}%
\pgfpathlineto{\pgfqpoint{3.799259in}{2.224136in}}%
\pgfpathlineto{\pgfqpoint{3.791285in}{2.215144in}}%
\pgfpathlineto{\pgfqpoint{3.783306in}{2.206167in}}%
\pgfpathclose%
\pgfusepath{fill}%
\end{pgfscope}%
\begin{pgfscope}%
\pgfpathrectangle{\pgfqpoint{1.150000in}{0.150000in}}{\pgfqpoint{5.700000in}{5.700000in}}%
\pgfusepath{clip}%
\pgfsetbuttcap%
\pgfsetroundjoin%
\definecolor{currentfill}{rgb}{0.280267,0.073417,0.397163}%
\pgfsetfillcolor{currentfill}%
\pgfsetfillopacity{0.700000}%
\pgfsetlinewidth{0.000000pt}%
\definecolor{currentstroke}{rgb}{0.000000,0.000000,0.000000}%
\pgfsetstrokecolor{currentstroke}%
\pgfsetdash{}{0pt}%
\pgfpathmoveto{\pgfqpoint{2.907305in}{2.209351in}}%
\pgfpathlineto{\pgfqpoint{2.920776in}{2.201780in}}%
\pgfpathlineto{\pgfqpoint{2.934249in}{2.194317in}}%
\pgfpathlineto{\pgfqpoint{2.947723in}{2.186960in}}%
\pgfpathlineto{\pgfqpoint{2.961199in}{2.179708in}}%
\pgfpathlineto{\pgfqpoint{2.969490in}{2.187881in}}%
\pgfpathlineto{\pgfqpoint{2.977773in}{2.196108in}}%
\pgfpathlineto{\pgfqpoint{2.986049in}{2.204389in}}%
\pgfpathlineto{\pgfqpoint{2.994318in}{2.212724in}}%
\pgfpathlineto{\pgfqpoint{2.980857in}{2.219912in}}%
\pgfpathlineto{\pgfqpoint{2.967398in}{2.227204in}}%
\pgfpathlineto{\pgfqpoint{2.953941in}{2.234604in}}%
\pgfpathlineto{\pgfqpoint{2.940485in}{2.242110in}}%
\pgfpathlineto{\pgfqpoint{2.932202in}{2.233832in}}%
\pgfpathlineto{\pgfqpoint{2.923910in}{2.225612in}}%
\pgfpathlineto{\pgfqpoint{2.915612in}{2.217452in}}%
\pgfpathlineto{\pgfqpoint{2.907305in}{2.209351in}}%
\pgfpathclose%
\pgfusepath{fill}%
\end{pgfscope}%
\begin{pgfscope}%
\pgfpathrectangle{\pgfqpoint{1.150000in}{0.150000in}}{\pgfqpoint{5.700000in}{5.700000in}}%
\pgfusepath{clip}%
\pgfsetbuttcap%
\pgfsetroundjoin%
\definecolor{currentfill}{rgb}{0.206756,0.371758,0.553117}%
\pgfsetfillcolor{currentfill}%
\pgfsetfillopacity{0.700000}%
\pgfsetlinewidth{0.000000pt}%
\definecolor{currentstroke}{rgb}{0.000000,0.000000,0.000000}%
\pgfsetstrokecolor{currentstroke}%
\pgfsetdash{}{0pt}%
\pgfpathmoveto{\pgfqpoint{6.005093in}{2.836287in}}%
\pgfpathlineto{\pgfqpoint{6.019259in}{2.834737in}}%
\pgfpathlineto{\pgfqpoint{6.033435in}{2.833251in}}%
\pgfpathlineto{\pgfqpoint{6.047621in}{2.831830in}}%
\pgfpathlineto{\pgfqpoint{6.061817in}{2.830472in}}%
\pgfpathlineto{\pgfqpoint{6.069102in}{2.841734in}}%
\pgfpathlineto{\pgfqpoint{6.076396in}{2.853347in}}%
\pgfpathlineto{\pgfqpoint{6.083700in}{2.865320in}}%
\pgfpathlineto{\pgfqpoint{6.091014in}{2.877663in}}%
\pgfpathlineto{\pgfqpoint{6.076847in}{2.879546in}}%
\pgfpathlineto{\pgfqpoint{6.062689in}{2.881494in}}%
\pgfpathlineto{\pgfqpoint{6.048542in}{2.883506in}}%
\pgfpathlineto{\pgfqpoint{6.034404in}{2.885581in}}%
\pgfpathlineto{\pgfqpoint{6.027062in}{2.872705in}}%
\pgfpathlineto{\pgfqpoint{6.019730in}{2.860204in}}%
\pgfpathlineto{\pgfqpoint{6.012407in}{2.848067in}}%
\pgfpathlineto{\pgfqpoint{6.005093in}{2.836287in}}%
\pgfpathclose%
\pgfusepath{fill}%
\end{pgfscope}%
\begin{pgfscope}%
\pgfpathrectangle{\pgfqpoint{1.150000in}{0.150000in}}{\pgfqpoint{5.700000in}{5.700000in}}%
\pgfusepath{clip}%
\pgfsetbuttcap%
\pgfsetroundjoin%
\definecolor{currentfill}{rgb}{0.246811,0.283237,0.535941}%
\pgfsetfillcolor{currentfill}%
\pgfsetfillopacity{0.700000}%
\pgfsetlinewidth{0.000000pt}%
\definecolor{currentstroke}{rgb}{0.000000,0.000000,0.000000}%
\pgfsetstrokecolor{currentstroke}%
\pgfsetdash{}{0pt}%
\pgfpathmoveto{\pgfqpoint{5.434106in}{2.630576in}}%
\pgfpathlineto{\pgfqpoint{5.448146in}{2.629928in}}%
\pgfpathlineto{\pgfqpoint{5.462196in}{2.629346in}}%
\pgfpathlineto{\pgfqpoint{5.476256in}{2.628832in}}%
\pgfpathlineto{\pgfqpoint{5.490325in}{2.628384in}}%
\pgfpathlineto{\pgfqpoint{5.497702in}{2.636609in}}%
\pgfpathlineto{\pgfqpoint{5.505078in}{2.645021in}}%
\pgfpathlineto{\pgfqpoint{5.512454in}{2.653628in}}%
\pgfpathlineto{\pgfqpoint{5.519831in}{2.662436in}}%
\pgfpathlineto{\pgfqpoint{5.505784in}{2.663290in}}%
\pgfpathlineto{\pgfqpoint{5.491747in}{2.664211in}}%
\pgfpathlineto{\pgfqpoint{5.477719in}{2.665198in}}%
\pgfpathlineto{\pgfqpoint{5.463701in}{2.666253in}}%
\pgfpathlineto{\pgfqpoint{5.456302in}{2.657031in}}%
\pgfpathlineto{\pgfqpoint{5.448903in}{2.648016in}}%
\pgfpathlineto{\pgfqpoint{5.441504in}{2.639200in}}%
\pgfpathlineto{\pgfqpoint{5.434106in}{2.630576in}}%
\pgfpathclose%
\pgfusepath{fill}%
\end{pgfscope}%
\begin{pgfscope}%
\pgfpathrectangle{\pgfqpoint{1.150000in}{0.150000in}}{\pgfqpoint{5.700000in}{5.700000in}}%
\pgfusepath{clip}%
\pgfsetbuttcap%
\pgfsetroundjoin%
\definecolor{currentfill}{rgb}{0.266580,0.228262,0.514349}%
\pgfsetfillcolor{currentfill}%
\pgfsetfillopacity{0.700000}%
\pgfsetlinewidth{0.000000pt}%
\definecolor{currentstroke}{rgb}{0.000000,0.000000,0.000000}%
\pgfsetstrokecolor{currentstroke}%
\pgfsetdash{}{0pt}%
\pgfpathmoveto{\pgfqpoint{5.035104in}{2.511849in}}%
\pgfpathlineto{\pgfqpoint{5.049033in}{2.511263in}}%
\pgfpathlineto{\pgfqpoint{5.062971in}{2.510746in}}%
\pgfpathlineto{\pgfqpoint{5.076918in}{2.510299in}}%
\pgfpathlineto{\pgfqpoint{5.090875in}{2.509920in}}%
\pgfpathlineto{\pgfqpoint{5.098386in}{2.517657in}}%
\pgfpathlineto{\pgfqpoint{5.105893in}{2.525502in}}%
\pgfpathlineto{\pgfqpoint{5.113397in}{2.533461in}}%
\pgfpathlineto{\pgfqpoint{5.120898in}{2.541540in}}%
\pgfpathlineto{\pgfqpoint{5.106959in}{2.542244in}}%
\pgfpathlineto{\pgfqpoint{5.093030in}{2.543017in}}%
\pgfpathlineto{\pgfqpoint{5.079110in}{2.543859in}}%
\pgfpathlineto{\pgfqpoint{5.065199in}{2.544769in}}%
\pgfpathlineto{\pgfqpoint{5.057680in}{2.536358in}}%
\pgfpathlineto{\pgfqpoint{5.050158in}{2.528071in}}%
\pgfpathlineto{\pgfqpoint{5.042633in}{2.519903in}}%
\pgfpathlineto{\pgfqpoint{5.035104in}{2.511849in}}%
\pgfpathclose%
\pgfusepath{fill}%
\end{pgfscope}%
\begin{pgfscope}%
\pgfpathrectangle{\pgfqpoint{1.150000in}{0.150000in}}{\pgfqpoint{5.700000in}{5.700000in}}%
\pgfusepath{clip}%
\pgfsetbuttcap%
\pgfsetroundjoin%
\definecolor{currentfill}{rgb}{0.214298,0.355619,0.551184}%
\pgfsetfillcolor{currentfill}%
\pgfsetfillopacity{0.700000}%
\pgfsetlinewidth{0.000000pt}%
\definecolor{currentstroke}{rgb}{0.000000,0.000000,0.000000}%
\pgfsetstrokecolor{currentstroke}%
\pgfsetdash{}{0pt}%
\pgfpathmoveto{\pgfqpoint{5.919242in}{2.797357in}}%
\pgfpathlineto{\pgfqpoint{5.933396in}{2.796055in}}%
\pgfpathlineto{\pgfqpoint{5.947560in}{2.794818in}}%
\pgfpathlineto{\pgfqpoint{5.961734in}{2.793646in}}%
\pgfpathlineto{\pgfqpoint{5.975918in}{2.792538in}}%
\pgfpathlineto{\pgfqpoint{5.983201in}{2.802987in}}%
\pgfpathlineto{\pgfqpoint{5.990490in}{2.813756in}}%
\pgfpathlineto{\pgfqpoint{5.997788in}{2.824853in}}%
\pgfpathlineto{\pgfqpoint{6.005093in}{2.836287in}}%
\pgfpathlineto{\pgfqpoint{5.990937in}{2.837901in}}%
\pgfpathlineto{\pgfqpoint{5.976791in}{2.839580in}}%
\pgfpathlineto{\pgfqpoint{5.962654in}{2.841323in}}%
\pgfpathlineto{\pgfqpoint{5.948528in}{2.843131in}}%
\pgfpathlineto{\pgfqpoint{5.941195in}{2.831183in}}%
\pgfpathlineto{\pgfqpoint{5.933870in}{2.819578in}}%
\pgfpathlineto{\pgfqpoint{5.926552in}{2.808306in}}%
\pgfpathlineto{\pgfqpoint{5.919242in}{2.797357in}}%
\pgfpathclose%
\pgfusepath{fill}%
\end{pgfscope}%
\begin{pgfscope}%
\pgfpathrectangle{\pgfqpoint{1.150000in}{0.150000in}}{\pgfqpoint{5.700000in}{5.700000in}}%
\pgfusepath{clip}%
\pgfsetbuttcap%
\pgfsetroundjoin%
\definecolor{currentfill}{rgb}{0.277018,0.050344,0.375715}%
\pgfsetfillcolor{currentfill}%
\pgfsetfillopacity{0.700000}%
\pgfsetlinewidth{0.000000pt}%
\definecolor{currentstroke}{rgb}{0.000000,0.000000,0.000000}%
\pgfsetstrokecolor{currentstroke}%
\pgfsetdash{}{0pt}%
\pgfpathmoveto{\pgfqpoint{3.329492in}{2.160045in}}%
\pgfpathlineto{\pgfqpoint{3.342994in}{2.154993in}}%
\pgfpathlineto{\pgfqpoint{3.356499in}{2.150034in}}%
\pgfpathlineto{\pgfqpoint{3.370009in}{2.145166in}}%
\pgfpathlineto{\pgfqpoint{3.383524in}{2.140390in}}%
\pgfpathlineto{\pgfqpoint{3.391653in}{2.149261in}}%
\pgfpathlineto{\pgfqpoint{3.399777in}{2.158154in}}%
\pgfpathlineto{\pgfqpoint{3.407894in}{2.167071in}}%
\pgfpathlineto{\pgfqpoint{3.416005in}{2.176012in}}%
\pgfpathlineto{\pgfqpoint{3.402503in}{2.180786in}}%
\pgfpathlineto{\pgfqpoint{3.389005in}{2.185651in}}%
\pgfpathlineto{\pgfqpoint{3.375511in}{2.190608in}}%
\pgfpathlineto{\pgfqpoint{3.362021in}{2.195658in}}%
\pgfpathlineto{\pgfqpoint{3.353898in}{2.186712in}}%
\pgfpathlineto{\pgfqpoint{3.345769in}{2.177795in}}%
\pgfpathlineto{\pgfqpoint{3.337634in}{2.168906in}}%
\pgfpathlineto{\pgfqpoint{3.329492in}{2.160045in}}%
\pgfpathclose%
\pgfusepath{fill}%
\end{pgfscope}%
\begin{pgfscope}%
\pgfpathrectangle{\pgfqpoint{1.150000in}{0.150000in}}{\pgfqpoint{5.700000in}{5.700000in}}%
\pgfusepath{clip}%
\pgfsetbuttcap%
\pgfsetroundjoin%
\definecolor{currentfill}{rgb}{0.282884,0.135920,0.453427}%
\pgfsetfillcolor{currentfill}%
\pgfsetfillopacity{0.700000}%
\pgfsetlinewidth{0.000000pt}%
\definecolor{currentstroke}{rgb}{0.000000,0.000000,0.000000}%
\pgfsetstrokecolor{currentstroke}%
\pgfsetdash{}{0pt}%
\pgfpathmoveto{\pgfqpoint{4.323090in}{2.318259in}}%
\pgfpathlineto{\pgfqpoint{4.336811in}{2.316788in}}%
\pgfpathlineto{\pgfqpoint{4.350539in}{2.315391in}}%
\pgfpathlineto{\pgfqpoint{4.364276in}{2.314070in}}%
\pgfpathlineto{\pgfqpoint{4.378021in}{2.312823in}}%
\pgfpathlineto{\pgfqpoint{4.385799in}{2.321129in}}%
\pgfpathlineto{\pgfqpoint{4.393571in}{2.329461in}}%
\pgfpathlineto{\pgfqpoint{4.401338in}{2.337821in}}%
\pgfpathlineto{\pgfqpoint{4.409100in}{2.346213in}}%
\pgfpathlineto{\pgfqpoint{4.395368in}{2.347642in}}%
\pgfpathlineto{\pgfqpoint{4.381644in}{2.349145in}}%
\pgfpathlineto{\pgfqpoint{4.367927in}{2.350724in}}%
\pgfpathlineto{\pgfqpoint{4.354218in}{2.352377in}}%
\pgfpathlineto{\pgfqpoint{4.346444in}{2.343795in}}%
\pgfpathlineto{\pgfqpoint{4.338665in}{2.335251in}}%
\pgfpathlineto{\pgfqpoint{4.330880in}{2.326740in}}%
\pgfpathlineto{\pgfqpoint{4.323090in}{2.318259in}}%
\pgfpathclose%
\pgfusepath{fill}%
\end{pgfscope}%
\begin{pgfscope}%
\pgfpathrectangle{\pgfqpoint{1.150000in}{0.150000in}}{\pgfqpoint{5.700000in}{5.700000in}}%
\pgfusepath{clip}%
\pgfsetbuttcap%
\pgfsetroundjoin%
\definecolor{currentfill}{rgb}{0.282656,0.100196,0.422160}%
\pgfsetfillcolor{currentfill}%
\pgfsetfillopacity{0.700000}%
\pgfsetlinewidth{0.000000pt}%
\definecolor{currentstroke}{rgb}{0.000000,0.000000,0.000000}%
\pgfsetstrokecolor{currentstroke}%
\pgfsetdash{}{0pt}%
\pgfpathmoveto{\pgfqpoint{4.010112in}{2.244868in}}%
\pgfpathlineto{\pgfqpoint{4.023749in}{2.242587in}}%
\pgfpathlineto{\pgfqpoint{4.037393in}{2.240385in}}%
\pgfpathlineto{\pgfqpoint{4.051044in}{2.238262in}}%
\pgfpathlineto{\pgfqpoint{4.064702in}{2.236217in}}%
\pgfpathlineto{\pgfqpoint{4.072593in}{2.244913in}}%
\pgfpathlineto{\pgfqpoint{4.080479in}{2.253620in}}%
\pgfpathlineto{\pgfqpoint{4.088359in}{2.262340in}}%
\pgfpathlineto{\pgfqpoint{4.096233in}{2.271076in}}%
\pgfpathlineto{\pgfqpoint{4.082586in}{2.273241in}}%
\pgfpathlineto{\pgfqpoint{4.068946in}{2.275485in}}%
\pgfpathlineto{\pgfqpoint{4.055313in}{2.277807in}}%
\pgfpathlineto{\pgfqpoint{4.041687in}{2.280209in}}%
\pgfpathlineto{\pgfqpoint{4.033801in}{2.271345in}}%
\pgfpathlineto{\pgfqpoint{4.025911in}{2.262502in}}%
\pgfpathlineto{\pgfqpoint{4.018014in}{2.253677in}}%
\pgfpathlineto{\pgfqpoint{4.010112in}{2.244868in}}%
\pgfpathclose%
\pgfusepath{fill}%
\end{pgfscope}%
\begin{pgfscope}%
\pgfpathrectangle{\pgfqpoint{1.150000in}{0.150000in}}{\pgfqpoint{5.700000in}{5.700000in}}%
\pgfusepath{clip}%
\pgfsetbuttcap%
\pgfsetroundjoin%
\definecolor{currentfill}{rgb}{0.283072,0.130895,0.449241}%
\pgfsetfillcolor{currentfill}%
\pgfsetfillopacity{0.700000}%
\pgfsetlinewidth{0.000000pt}%
\definecolor{currentstroke}{rgb}{0.000000,0.000000,0.000000}%
\pgfsetstrokecolor{currentstroke}%
\pgfsetdash{}{0pt}%
\pgfpathmoveto{\pgfqpoint{2.570668in}{2.325797in}}%
\pgfpathlineto{\pgfqpoint{2.584163in}{2.315713in}}%
\pgfpathlineto{\pgfqpoint{2.597658in}{2.305755in}}%
\pgfpathlineto{\pgfqpoint{2.611151in}{2.295921in}}%
\pgfpathlineto{\pgfqpoint{2.624645in}{2.286211in}}%
\pgfpathlineto{\pgfqpoint{2.633087in}{2.293348in}}%
\pgfpathlineto{\pgfqpoint{2.641520in}{2.300573in}}%
\pgfpathlineto{\pgfqpoint{2.649945in}{2.307885in}}%
\pgfpathlineto{\pgfqpoint{2.658360in}{2.315283in}}%
\pgfpathlineto{\pgfqpoint{2.644886in}{2.324887in}}%
\pgfpathlineto{\pgfqpoint{2.631411in}{2.334614in}}%
\pgfpathlineto{\pgfqpoint{2.617936in}{2.344465in}}%
\pgfpathlineto{\pgfqpoint{2.604461in}{2.354441in}}%
\pgfpathlineto{\pgfqpoint{2.596027in}{2.347143in}}%
\pgfpathlineto{\pgfqpoint{2.587583in}{2.339935in}}%
\pgfpathlineto{\pgfqpoint{2.579130in}{2.332819in}}%
\pgfpathlineto{\pgfqpoint{2.570668in}{2.325797in}}%
\pgfpathclose%
\pgfusepath{fill}%
\end{pgfscope}%
\begin{pgfscope}%
\pgfpathrectangle{\pgfqpoint{1.150000in}{0.150000in}}{\pgfqpoint{5.700000in}{5.700000in}}%
\pgfusepath{clip}%
\pgfsetbuttcap%
\pgfsetroundjoin%
\definecolor{currentfill}{rgb}{0.282327,0.094955,0.417331}%
\pgfsetfillcolor{currentfill}%
\pgfsetfillopacity{0.700000}%
\pgfsetlinewidth{0.000000pt}%
\definecolor{currentstroke}{rgb}{0.000000,0.000000,0.000000}%
\pgfsetstrokecolor{currentstroke}%
\pgfsetdash{}{0pt}%
\pgfpathmoveto{\pgfqpoint{2.766158in}{2.242756in}}%
\pgfpathlineto{\pgfqpoint{2.779635in}{2.234215in}}%
\pgfpathlineto{\pgfqpoint{2.793113in}{2.225787in}}%
\pgfpathlineto{\pgfqpoint{2.806591in}{2.217472in}}%
\pgfpathlineto{\pgfqpoint{2.820071in}{2.209270in}}%
\pgfpathlineto{\pgfqpoint{2.828425in}{2.217031in}}%
\pgfpathlineto{\pgfqpoint{2.836771in}{2.224860in}}%
\pgfpathlineto{\pgfqpoint{2.845109in}{2.232757in}}%
\pgfpathlineto{\pgfqpoint{2.853439in}{2.240720in}}%
\pgfpathlineto{\pgfqpoint{2.839975in}{2.248838in}}%
\pgfpathlineto{\pgfqpoint{2.826513in}{2.257068in}}%
\pgfpathlineto{\pgfqpoint{2.813053in}{2.265410in}}%
\pgfpathlineto{\pgfqpoint{2.799593in}{2.273866in}}%
\pgfpathlineto{\pgfqpoint{2.791246in}{2.265981in}}%
\pgfpathlineto{\pgfqpoint{2.782892in}{2.258167in}}%
\pgfpathlineto{\pgfqpoint{2.774529in}{2.250425in}}%
\pgfpathlineto{\pgfqpoint{2.766158in}{2.242756in}}%
\pgfpathclose%
\pgfusepath{fill}%
\end{pgfscope}%
\begin{pgfscope}%
\pgfpathrectangle{\pgfqpoint{1.150000in}{0.150000in}}{\pgfqpoint{5.700000in}{5.700000in}}%
\pgfusepath{clip}%
\pgfsetbuttcap%
\pgfsetroundjoin%
\definecolor{currentfill}{rgb}{0.278826,0.175490,0.483397}%
\pgfsetfillcolor{currentfill}%
\pgfsetfillopacity{0.700000}%
\pgfsetlinewidth{0.000000pt}%
\definecolor{currentstroke}{rgb}{0.000000,0.000000,0.000000}%
\pgfsetstrokecolor{currentstroke}%
\pgfsetdash{}{0pt}%
\pgfpathmoveto{\pgfqpoint{4.636116in}{2.398457in}}%
\pgfpathlineto{\pgfqpoint{4.649929in}{2.397556in}}%
\pgfpathlineto{\pgfqpoint{4.663751in}{2.396727in}}%
\pgfpathlineto{\pgfqpoint{4.677581in}{2.395969in}}%
\pgfpathlineto{\pgfqpoint{4.691420in}{2.395284in}}%
\pgfpathlineto{\pgfqpoint{4.699082in}{2.403195in}}%
\pgfpathlineto{\pgfqpoint{4.706738in}{2.411158in}}%
\pgfpathlineto{\pgfqpoint{4.714390in}{2.419177in}}%
\pgfpathlineto{\pgfqpoint{4.722036in}{2.427257in}}%
\pgfpathlineto{\pgfqpoint{4.708212in}{2.428186in}}%
\pgfpathlineto{\pgfqpoint{4.694396in}{2.429187in}}%
\pgfpathlineto{\pgfqpoint{4.680589in}{2.430259in}}%
\pgfpathlineto{\pgfqpoint{4.666790in}{2.431404in}}%
\pgfpathlineto{\pgfqpoint{4.659128in}{2.423073in}}%
\pgfpathlineto{\pgfqpoint{4.651463in}{2.414808in}}%
\pgfpathlineto{\pgfqpoint{4.643792in}{2.406604in}}%
\pgfpathlineto{\pgfqpoint{4.636116in}{2.398457in}}%
\pgfpathclose%
\pgfusepath{fill}%
\end{pgfscope}%
\begin{pgfscope}%
\pgfpathrectangle{\pgfqpoint{1.150000in}{0.150000in}}{\pgfqpoint{5.700000in}{5.700000in}}%
\pgfusepath{clip}%
\pgfsetbuttcap%
\pgfsetroundjoin%
\definecolor{currentfill}{rgb}{0.220057,0.343307,0.549413}%
\pgfsetfillcolor{currentfill}%
\pgfsetfillopacity{0.700000}%
\pgfsetlinewidth{0.000000pt}%
\definecolor{currentstroke}{rgb}{0.000000,0.000000,0.000000}%
\pgfsetstrokecolor{currentstroke}%
\pgfsetdash{}{0pt}%
\pgfpathmoveto{\pgfqpoint{5.833437in}{2.760535in}}%
\pgfpathlineto{\pgfqpoint{5.847578in}{2.759460in}}%
\pgfpathlineto{\pgfqpoint{5.861729in}{2.758450in}}%
\pgfpathlineto{\pgfqpoint{5.875890in}{2.757505in}}%
\pgfpathlineto{\pgfqpoint{5.890062in}{2.756626in}}%
\pgfpathlineto{\pgfqpoint{5.897348in}{2.766366in}}%
\pgfpathlineto{\pgfqpoint{5.904640in}{2.776396in}}%
\pgfpathlineto{\pgfqpoint{5.911938in}{2.786724in}}%
\pgfpathlineto{\pgfqpoint{5.919242in}{2.797357in}}%
\pgfpathlineto{\pgfqpoint{5.905098in}{2.798724in}}%
\pgfpathlineto{\pgfqpoint{5.890964in}{2.800155in}}%
\pgfpathlineto{\pgfqpoint{5.876839in}{2.801651in}}%
\pgfpathlineto{\pgfqpoint{5.862725in}{2.803212in}}%
\pgfpathlineto{\pgfqpoint{5.855394in}{2.792085in}}%
\pgfpathlineto{\pgfqpoint{5.848070in}{2.781269in}}%
\pgfpathlineto{\pgfqpoint{5.840751in}{2.770755in}}%
\pgfpathlineto{\pgfqpoint{5.833437in}{2.760535in}}%
\pgfpathclose%
\pgfusepath{fill}%
\end{pgfscope}%
\begin{pgfscope}%
\pgfpathrectangle{\pgfqpoint{1.150000in}{0.150000in}}{\pgfqpoint{5.700000in}{5.700000in}}%
\pgfusepath{clip}%
\pgfsetbuttcap%
\pgfsetroundjoin%
\definecolor{currentfill}{rgb}{0.250425,0.274290,0.533103}%
\pgfsetfillcolor{currentfill}%
\pgfsetfillopacity{0.700000}%
\pgfsetlinewidth{0.000000pt}%
\definecolor{currentstroke}{rgb}{0.000000,0.000000,0.000000}%
\pgfsetstrokecolor{currentstroke}%
\pgfsetdash{}{0pt}%
\pgfpathmoveto{\pgfqpoint{5.348355in}{2.599578in}}%
\pgfpathlineto{\pgfqpoint{5.362378in}{2.599047in}}%
\pgfpathlineto{\pgfqpoint{5.376411in}{2.598584in}}%
\pgfpathlineto{\pgfqpoint{5.390453in}{2.598188in}}%
\pgfpathlineto{\pgfqpoint{5.404505in}{2.597860in}}%
\pgfpathlineto{\pgfqpoint{5.411907in}{2.605785in}}%
\pgfpathlineto{\pgfqpoint{5.419307in}{2.613876in}}%
\pgfpathlineto{\pgfqpoint{5.426707in}{2.622137in}}%
\pgfpathlineto{\pgfqpoint{5.434106in}{2.630576in}}%
\pgfpathlineto{\pgfqpoint{5.420075in}{2.631291in}}%
\pgfpathlineto{\pgfqpoint{5.406055in}{2.632074in}}%
\pgfpathlineto{\pgfqpoint{5.392043in}{2.632923in}}%
\pgfpathlineto{\pgfqpoint{5.378042in}{2.633840in}}%
\pgfpathlineto{\pgfqpoint{5.370621in}{2.625007in}}%
\pgfpathlineto{\pgfqpoint{5.363200in}{2.616357in}}%
\pgfpathlineto{\pgfqpoint{5.355778in}{2.607883in}}%
\pgfpathlineto{\pgfqpoint{5.348355in}{2.599578in}}%
\pgfpathclose%
\pgfusepath{fill}%
\end{pgfscope}%
\begin{pgfscope}%
\pgfpathrectangle{\pgfqpoint{1.150000in}{0.150000in}}{\pgfqpoint{5.700000in}{5.700000in}}%
\pgfusepath{clip}%
\pgfsetbuttcap%
\pgfsetroundjoin%
\definecolor{currentfill}{rgb}{0.279566,0.067836,0.391917}%
\pgfsetfillcolor{currentfill}%
\pgfsetfillopacity{0.700000}%
\pgfsetlinewidth{0.000000pt}%
\definecolor{currentstroke}{rgb}{0.000000,0.000000,0.000000}%
\pgfsetstrokecolor{currentstroke}%
\pgfsetdash{}{0pt}%
\pgfpathmoveto{\pgfqpoint{3.697025in}{2.183248in}}%
\pgfpathlineto{\pgfqpoint{3.710592in}{2.179903in}}%
\pgfpathlineto{\pgfqpoint{3.724166in}{2.176642in}}%
\pgfpathlineto{\pgfqpoint{3.737745in}{2.173464in}}%
\pgfpathlineto{\pgfqpoint{3.751330in}{2.170370in}}%
\pgfpathlineto{\pgfqpoint{3.759333in}{2.179305in}}%
\pgfpathlineto{\pgfqpoint{3.767329in}{2.188249in}}%
\pgfpathlineto{\pgfqpoint{3.775320in}{2.197202in}}%
\pgfpathlineto{\pgfqpoint{3.783306in}{2.206167in}}%
\pgfpathlineto{\pgfqpoint{3.769731in}{2.209321in}}%
\pgfpathlineto{\pgfqpoint{3.756163in}{2.212557in}}%
\pgfpathlineto{\pgfqpoint{3.742600in}{2.215878in}}%
\pgfpathlineto{\pgfqpoint{3.729043in}{2.219282in}}%
\pgfpathlineto{\pgfqpoint{3.721047in}{2.210251in}}%
\pgfpathlineto{\pgfqpoint{3.713045in}{2.201235in}}%
\pgfpathlineto{\pgfqpoint{3.705038in}{2.192235in}}%
\pgfpathlineto{\pgfqpoint{3.697025in}{2.183248in}}%
\pgfpathclose%
\pgfusepath{fill}%
\end{pgfscope}%
\begin{pgfscope}%
\pgfpathrectangle{\pgfqpoint{1.150000in}{0.150000in}}{\pgfqpoint{5.700000in}{5.700000in}}%
\pgfusepath{clip}%
\pgfsetbuttcap%
\pgfsetroundjoin%
\definecolor{currentfill}{rgb}{0.269308,0.218818,0.509577}%
\pgfsetfillcolor{currentfill}%
\pgfsetfillopacity{0.700000}%
\pgfsetlinewidth{0.000000pt}%
\definecolor{currentstroke}{rgb}{0.000000,0.000000,0.000000}%
\pgfsetstrokecolor{currentstroke}%
\pgfsetdash{}{0pt}%
\pgfpathmoveto{\pgfqpoint{4.949262in}{2.482469in}}%
\pgfpathlineto{\pgfqpoint{4.963171in}{2.481910in}}%
\pgfpathlineto{\pgfqpoint{4.977089in}{2.481421in}}%
\pgfpathlineto{\pgfqpoint{4.991017in}{2.481001in}}%
\pgfpathlineto{\pgfqpoint{5.004954in}{2.480652in}}%
\pgfpathlineto{\pgfqpoint{5.012497in}{2.488309in}}%
\pgfpathlineto{\pgfqpoint{5.020036in}{2.496057in}}%
\pgfpathlineto{\pgfqpoint{5.027572in}{2.503902in}}%
\pgfpathlineto{\pgfqpoint{5.035104in}{2.511849in}}%
\pgfpathlineto{\pgfqpoint{5.021185in}{2.512504in}}%
\pgfpathlineto{\pgfqpoint{5.007274in}{2.513229in}}%
\pgfpathlineto{\pgfqpoint{4.993373in}{2.514023in}}%
\pgfpathlineto{\pgfqpoint{4.979481in}{2.514886in}}%
\pgfpathlineto{\pgfqpoint{4.971932in}{2.506627in}}%
\pgfpathlineto{\pgfqpoint{4.964379in}{2.498475in}}%
\pgfpathlineto{\pgfqpoint{4.956822in}{2.490424in}}%
\pgfpathlineto{\pgfqpoint{4.949262in}{2.482469in}}%
\pgfpathclose%
\pgfusepath{fill}%
\end{pgfscope}%
\begin{pgfscope}%
\pgfpathrectangle{\pgfqpoint{1.150000in}{0.150000in}}{\pgfqpoint{5.700000in}{5.700000in}}%
\pgfusepath{clip}%
\pgfsetbuttcap%
\pgfsetroundjoin%
\definecolor{currentfill}{rgb}{0.277941,0.056324,0.381191}%
\pgfsetfillcolor{currentfill}%
\pgfsetfillopacity{0.700000}%
\pgfsetlinewidth{0.000000pt}%
\definecolor{currentstroke}{rgb}{0.000000,0.000000,0.000000}%
\pgfsetstrokecolor{currentstroke}%
\pgfsetdash{}{0pt}%
\pgfpathmoveto{\pgfqpoint{3.470061in}{2.157820in}}%
\pgfpathlineto{\pgfqpoint{3.483587in}{2.153495in}}%
\pgfpathlineto{\pgfqpoint{3.497118in}{2.149260in}}%
\pgfpathlineto{\pgfqpoint{3.510654in}{2.145112in}}%
\pgfpathlineto{\pgfqpoint{3.524194in}{2.141052in}}%
\pgfpathlineto{\pgfqpoint{3.532276in}{2.149996in}}%
\pgfpathlineto{\pgfqpoint{3.540352in}{2.158955in}}%
\pgfpathlineto{\pgfqpoint{3.548423in}{2.167930in}}%
\pgfpathlineto{\pgfqpoint{3.556487in}{2.176921in}}%
\pgfpathlineto{\pgfqpoint{3.542958in}{2.180999in}}%
\pgfpathlineto{\pgfqpoint{3.529433in}{2.185165in}}%
\pgfpathlineto{\pgfqpoint{3.515914in}{2.189419in}}%
\pgfpathlineto{\pgfqpoint{3.502400in}{2.193761in}}%
\pgfpathlineto{\pgfqpoint{3.494324in}{2.184744in}}%
\pgfpathlineto{\pgfqpoint{3.486242in}{2.175749in}}%
\pgfpathlineto{\pgfqpoint{3.478155in}{2.166774in}}%
\pgfpathlineto{\pgfqpoint{3.470061in}{2.157820in}}%
\pgfpathclose%
\pgfusepath{fill}%
\end{pgfscope}%
\begin{pgfscope}%
\pgfpathrectangle{\pgfqpoint{1.150000in}{0.150000in}}{\pgfqpoint{5.700000in}{5.700000in}}%
\pgfusepath{clip}%
\pgfsetbuttcap%
\pgfsetroundjoin%
\definecolor{currentfill}{rgb}{0.227802,0.326594,0.546532}%
\pgfsetfillcolor{currentfill}%
\pgfsetfillopacity{0.700000}%
\pgfsetlinewidth{0.000000pt}%
\definecolor{currentstroke}{rgb}{0.000000,0.000000,0.000000}%
\pgfsetstrokecolor{currentstroke}%
\pgfsetdash{}{0pt}%
\pgfpathmoveto{\pgfqpoint{5.747660in}{2.725504in}}%
\pgfpathlineto{\pgfqpoint{5.761787in}{2.724635in}}%
\pgfpathlineto{\pgfqpoint{5.775924in}{2.723832in}}%
\pgfpathlineto{\pgfqpoint{5.790071in}{2.723093in}}%
\pgfpathlineto{\pgfqpoint{5.804228in}{2.722420in}}%
\pgfpathlineto{\pgfqpoint{5.811525in}{2.731551in}}%
\pgfpathlineto{\pgfqpoint{5.818825in}{2.740941in}}%
\pgfpathlineto{\pgfqpoint{5.826129in}{2.750599in}}%
\pgfpathlineto{\pgfqpoint{5.833437in}{2.760535in}}%
\pgfpathlineto{\pgfqpoint{5.819306in}{2.761675in}}%
\pgfpathlineto{\pgfqpoint{5.805185in}{2.762880in}}%
\pgfpathlineto{\pgfqpoint{5.791074in}{2.764150in}}%
\pgfpathlineto{\pgfqpoint{5.776973in}{2.765486in}}%
\pgfpathlineto{\pgfqpoint{5.769638in}{2.755077in}}%
\pgfpathlineto{\pgfqpoint{5.762308in}{2.744949in}}%
\pgfpathlineto{\pgfqpoint{5.754982in}{2.735094in}}%
\pgfpathlineto{\pgfqpoint{5.747660in}{2.725504in}}%
\pgfpathclose%
\pgfusepath{fill}%
\end{pgfscope}%
\begin{pgfscope}%
\pgfpathrectangle{\pgfqpoint{1.150000in}{0.150000in}}{\pgfqpoint{5.700000in}{5.700000in}}%
\pgfusepath{clip}%
\pgfsetbuttcap%
\pgfsetroundjoin%
\definecolor{currentfill}{rgb}{0.283187,0.125848,0.444960}%
\pgfsetfillcolor{currentfill}%
\pgfsetfillopacity{0.700000}%
\pgfsetlinewidth{0.000000pt}%
\definecolor{currentstroke}{rgb}{0.000000,0.000000,0.000000}%
\pgfsetstrokecolor{currentstroke}%
\pgfsetdash{}{0pt}%
\pgfpathmoveto{\pgfqpoint{4.237021in}{2.290567in}}%
\pgfpathlineto{\pgfqpoint{4.250723in}{2.288954in}}%
\pgfpathlineto{\pgfqpoint{4.264433in}{2.287418in}}%
\pgfpathlineto{\pgfqpoint{4.278150in}{2.285957in}}%
\pgfpathlineto{\pgfqpoint{4.291875in}{2.284572in}}%
\pgfpathlineto{\pgfqpoint{4.299687in}{2.292965in}}%
\pgfpathlineto{\pgfqpoint{4.307493in}{2.301375in}}%
\pgfpathlineto{\pgfqpoint{4.315294in}{2.309805in}}%
\pgfpathlineto{\pgfqpoint{4.323090in}{2.318259in}}%
\pgfpathlineto{\pgfqpoint{4.309377in}{2.319806in}}%
\pgfpathlineto{\pgfqpoint{4.295672in}{2.321428in}}%
\pgfpathlineto{\pgfqpoint{4.281974in}{2.323126in}}%
\pgfpathlineto{\pgfqpoint{4.268284in}{2.324900in}}%
\pgfpathlineto{\pgfqpoint{4.260477in}{2.316277in}}%
\pgfpathlineto{\pgfqpoint{4.252664in}{2.307682in}}%
\pgfpathlineto{\pgfqpoint{4.244845in}{2.299113in}}%
\pgfpathlineto{\pgfqpoint{4.237021in}{2.290567in}}%
\pgfpathclose%
\pgfusepath{fill}%
\end{pgfscope}%
\begin{pgfscope}%
\pgfpathrectangle{\pgfqpoint{1.150000in}{0.150000in}}{\pgfqpoint{5.700000in}{5.700000in}}%
\pgfusepath{clip}%
\pgfsetbuttcap%
\pgfsetroundjoin%
\definecolor{currentfill}{rgb}{0.281924,0.089666,0.412415}%
\pgfsetfillcolor{currentfill}%
\pgfsetfillopacity{0.700000}%
\pgfsetlinewidth{0.000000pt}%
\definecolor{currentstroke}{rgb}{0.000000,0.000000,0.000000}%
\pgfsetstrokecolor{currentstroke}%
\pgfsetdash{}{0pt}%
\pgfpathmoveto{\pgfqpoint{3.923924in}{2.219260in}}%
\pgfpathlineto{\pgfqpoint{3.937545in}{2.216761in}}%
\pgfpathlineto{\pgfqpoint{3.951172in}{2.214342in}}%
\pgfpathlineto{\pgfqpoint{3.964807in}{2.212002in}}%
\pgfpathlineto{\pgfqpoint{3.978448in}{2.209743in}}%
\pgfpathlineto{\pgfqpoint{3.986372in}{2.218512in}}%
\pgfpathlineto{\pgfqpoint{3.994291in}{2.227287in}}%
\pgfpathlineto{\pgfqpoint{4.002204in}{2.236072in}}%
\pgfpathlineto{\pgfqpoint{4.010112in}{2.244868in}}%
\pgfpathlineto{\pgfqpoint{3.996482in}{2.247228in}}%
\pgfpathlineto{\pgfqpoint{3.982858in}{2.249668in}}%
\pgfpathlineto{\pgfqpoint{3.969242in}{2.252187in}}%
\pgfpathlineto{\pgfqpoint{3.955632in}{2.254786in}}%
\pgfpathlineto{\pgfqpoint{3.947713in}{2.245882in}}%
\pgfpathlineto{\pgfqpoint{3.939789in}{2.236995in}}%
\pgfpathlineto{\pgfqpoint{3.931859in}{2.228122in}}%
\pgfpathlineto{\pgfqpoint{3.923924in}{2.219260in}}%
\pgfpathclose%
\pgfusepath{fill}%
\end{pgfscope}%
\begin{pgfscope}%
\pgfpathrectangle{\pgfqpoint{1.150000in}{0.150000in}}{\pgfqpoint{5.700000in}{5.700000in}}%
\pgfusepath{clip}%
\pgfsetbuttcap%
\pgfsetroundjoin%
\definecolor{currentfill}{rgb}{0.280255,0.165693,0.476498}%
\pgfsetfillcolor{currentfill}%
\pgfsetfillopacity{0.700000}%
\pgfsetlinewidth{0.000000pt}%
\definecolor{currentstroke}{rgb}{0.000000,0.000000,0.000000}%
\pgfsetstrokecolor{currentstroke}%
\pgfsetdash{}{0pt}%
\pgfpathmoveto{\pgfqpoint{4.550141in}{2.369789in}}%
\pgfpathlineto{\pgfqpoint{4.563933in}{2.368820in}}%
\pgfpathlineto{\pgfqpoint{4.577735in}{2.367925in}}%
\pgfpathlineto{\pgfqpoint{4.591545in}{2.367102in}}%
\pgfpathlineto{\pgfqpoint{4.605363in}{2.366352in}}%
\pgfpathlineto{\pgfqpoint{4.613059in}{2.374314in}}%
\pgfpathlineto{\pgfqpoint{4.620750in}{2.382316in}}%
\pgfpathlineto{\pgfqpoint{4.628436in}{2.390363in}}%
\pgfpathlineto{\pgfqpoint{4.636116in}{2.398457in}}%
\pgfpathlineto{\pgfqpoint{4.622312in}{2.399431in}}%
\pgfpathlineto{\pgfqpoint{4.608516in}{2.400477in}}%
\pgfpathlineto{\pgfqpoint{4.594728in}{2.401595in}}%
\pgfpathlineto{\pgfqpoint{4.580949in}{2.402787in}}%
\pgfpathlineto{\pgfqpoint{4.573254in}{2.394462in}}%
\pgfpathlineto{\pgfqpoint{4.565555in}{2.386190in}}%
\pgfpathlineto{\pgfqpoint{4.557850in}{2.377967in}}%
\pgfpathlineto{\pgfqpoint{4.550141in}{2.369789in}}%
\pgfpathclose%
\pgfusepath{fill}%
\end{pgfscope}%
\begin{pgfscope}%
\pgfpathrectangle{\pgfqpoint{1.150000in}{0.150000in}}{\pgfqpoint{5.700000in}{5.700000in}}%
\pgfusepath{clip}%
\pgfsetbuttcap%
\pgfsetroundjoin%
\definecolor{currentfill}{rgb}{0.277941,0.056324,0.381191}%
\pgfsetfillcolor{currentfill}%
\pgfsetfillopacity{0.700000}%
\pgfsetlinewidth{0.000000pt}%
\definecolor{currentstroke}{rgb}{0.000000,0.000000,0.000000}%
\pgfsetstrokecolor{currentstroke}%
\pgfsetdash{}{0pt}%
\pgfpathmoveto{\pgfqpoint{3.102096in}{2.158945in}}%
\pgfpathlineto{\pgfqpoint{3.115581in}{2.152678in}}%
\pgfpathlineto{\pgfqpoint{3.129069in}{2.146509in}}%
\pgfpathlineto{\pgfqpoint{3.142559in}{2.140439in}}%
\pgfpathlineto{\pgfqpoint{3.156053in}{2.134467in}}%
\pgfpathlineto{\pgfqpoint{3.164272in}{2.143008in}}%
\pgfpathlineto{\pgfqpoint{3.172485in}{2.151587in}}%
\pgfpathlineto{\pgfqpoint{3.180690in}{2.160203in}}%
\pgfpathlineto{\pgfqpoint{3.188890in}{2.168857in}}%
\pgfpathlineto{\pgfqpoint{3.175409in}{2.174786in}}%
\pgfpathlineto{\pgfqpoint{3.161932in}{2.180813in}}%
\pgfpathlineto{\pgfqpoint{3.148458in}{2.186938in}}%
\pgfpathlineto{\pgfqpoint{3.134987in}{2.193162in}}%
\pgfpathlineto{\pgfqpoint{3.126775in}{2.184544in}}%
\pgfpathlineto{\pgfqpoint{3.118555in}{2.175969in}}%
\pgfpathlineto{\pgfqpoint{3.110329in}{2.167435in}}%
\pgfpathlineto{\pgfqpoint{3.102096in}{2.158945in}}%
\pgfpathclose%
\pgfusepath{fill}%
\end{pgfscope}%
\begin{pgfscope}%
\pgfpathrectangle{\pgfqpoint{1.150000in}{0.150000in}}{\pgfqpoint{5.700000in}{5.700000in}}%
\pgfusepath{clip}%
\pgfsetbuttcap%
\pgfsetroundjoin%
\definecolor{currentfill}{rgb}{0.277134,0.185228,0.489898}%
\pgfsetfillcolor{currentfill}%
\pgfsetfillopacity{0.700000}%
\pgfsetlinewidth{0.000000pt}%
\definecolor{currentstroke}{rgb}{0.000000,0.000000,0.000000}%
\pgfsetstrokecolor{currentstroke}%
\pgfsetdash{}{0pt}%
\pgfpathmoveto{\pgfqpoint{2.374429in}{2.431364in}}%
\pgfpathlineto{\pgfqpoint{2.387965in}{2.419563in}}%
\pgfpathlineto{\pgfqpoint{2.401498in}{2.407901in}}%
\pgfpathlineto{\pgfqpoint{2.415028in}{2.396377in}}%
\pgfpathlineto{\pgfqpoint{2.428557in}{2.384990in}}%
\pgfpathlineto{\pgfqpoint{2.437101in}{2.391354in}}%
\pgfpathlineto{\pgfqpoint{2.445634in}{2.397827in}}%
\pgfpathlineto{\pgfqpoint{2.454157in}{2.404407in}}%
\pgfpathlineto{\pgfqpoint{2.462670in}{2.411095in}}%
\pgfpathlineto{\pgfqpoint{2.449163in}{2.422353in}}%
\pgfpathlineto{\pgfqpoint{2.435654in}{2.433748in}}%
\pgfpathlineto{\pgfqpoint{2.422143in}{2.445281in}}%
\pgfpathlineto{\pgfqpoint{2.408630in}{2.456952in}}%
\pgfpathlineto{\pgfqpoint{2.400096in}{2.450386in}}%
\pgfpathlineto{\pgfqpoint{2.391551in}{2.443932in}}%
\pgfpathlineto{\pgfqpoint{2.382995in}{2.437591in}}%
\pgfpathlineto{\pgfqpoint{2.374429in}{2.431364in}}%
\pgfpathclose%
\pgfusepath{fill}%
\end{pgfscope}%
\begin{pgfscope}%
\pgfpathrectangle{\pgfqpoint{1.150000in}{0.150000in}}{\pgfqpoint{5.700000in}{5.700000in}}%
\pgfusepath{clip}%
\pgfsetbuttcap%
\pgfsetroundjoin%
\definecolor{currentfill}{rgb}{0.278791,0.062145,0.386592}%
\pgfsetfillcolor{currentfill}%
\pgfsetfillopacity{0.700000}%
\pgfsetlinewidth{0.000000pt}%
\definecolor{currentstroke}{rgb}{0.000000,0.000000,0.000000}%
\pgfsetstrokecolor{currentstroke}%
\pgfsetdash{}{0pt}%
\pgfpathmoveto{\pgfqpoint{2.961199in}{2.179708in}}%
\pgfpathlineto{\pgfqpoint{2.974678in}{2.172562in}}%
\pgfpathlineto{\pgfqpoint{2.988159in}{2.165520in}}%
\pgfpathlineto{\pgfqpoint{3.001642in}{2.158581in}}%
\pgfpathlineto{\pgfqpoint{3.015128in}{2.151746in}}%
\pgfpathlineto{\pgfqpoint{3.023403in}{2.159990in}}%
\pgfpathlineto{\pgfqpoint{3.031671in}{2.168283in}}%
\pgfpathlineto{\pgfqpoint{3.039932in}{2.176626in}}%
\pgfpathlineto{\pgfqpoint{3.048187in}{2.185017in}}%
\pgfpathlineto{\pgfqpoint{3.034716in}{2.191789in}}%
\pgfpathlineto{\pgfqpoint{3.021248in}{2.198664in}}%
\pgfpathlineto{\pgfqpoint{3.007782in}{2.205642in}}%
\pgfpathlineto{\pgfqpoint{2.994318in}{2.212724in}}%
\pgfpathlineto{\pgfqpoint{2.986049in}{2.204389in}}%
\pgfpathlineto{\pgfqpoint{2.977773in}{2.196108in}}%
\pgfpathlineto{\pgfqpoint{2.969490in}{2.187881in}}%
\pgfpathlineto{\pgfqpoint{2.961199in}{2.179708in}}%
\pgfpathclose%
\pgfusepath{fill}%
\end{pgfscope}%
\begin{pgfscope}%
\pgfpathrectangle{\pgfqpoint{1.150000in}{0.150000in}}{\pgfqpoint{5.700000in}{5.700000in}}%
\pgfusepath{clip}%
\pgfsetbuttcap%
\pgfsetroundjoin%
\definecolor{currentfill}{rgb}{0.255645,0.260703,0.528312}%
\pgfsetfillcolor{currentfill}%
\pgfsetfillopacity{0.700000}%
\pgfsetlinewidth{0.000000pt}%
\definecolor{currentstroke}{rgb}{0.000000,0.000000,0.000000}%
\pgfsetstrokecolor{currentstroke}%
\pgfsetdash{}{0pt}%
\pgfpathmoveto{\pgfqpoint{5.262571in}{2.569246in}}%
\pgfpathlineto{\pgfqpoint{5.276576in}{2.568812in}}%
\pgfpathlineto{\pgfqpoint{5.290590in}{2.568445in}}%
\pgfpathlineto{\pgfqpoint{5.304614in}{2.568145in}}%
\pgfpathlineto{\pgfqpoint{5.318648in}{2.567914in}}%
\pgfpathlineto{\pgfqpoint{5.326078in}{2.575610in}}%
\pgfpathlineto{\pgfqpoint{5.333505in}{2.583448in}}%
\pgfpathlineto{\pgfqpoint{5.340931in}{2.591435in}}%
\pgfpathlineto{\pgfqpoint{5.348355in}{2.599578in}}%
\pgfpathlineto{\pgfqpoint{5.334342in}{2.600176in}}%
\pgfpathlineto{\pgfqpoint{5.320339in}{2.600842in}}%
\pgfpathlineto{\pgfqpoint{5.306345in}{2.601575in}}%
\pgfpathlineto{\pgfqpoint{5.292361in}{2.602376in}}%
\pgfpathlineto{\pgfqpoint{5.284916in}{2.593859in}}%
\pgfpathlineto{\pgfqpoint{5.277469in}{2.585503in}}%
\pgfpathlineto{\pgfqpoint{5.270021in}{2.577301in}}%
\pgfpathlineto{\pgfqpoint{5.262571in}{2.569246in}}%
\pgfpathclose%
\pgfusepath{fill}%
\end{pgfscope}%
\begin{pgfscope}%
\pgfpathrectangle{\pgfqpoint{1.150000in}{0.150000in}}{\pgfqpoint{5.700000in}{5.700000in}}%
\pgfusepath{clip}%
\pgfsetbuttcap%
\pgfsetroundjoin%
\definecolor{currentfill}{rgb}{0.283197,0.115680,0.436115}%
\pgfsetfillcolor{currentfill}%
\pgfsetfillopacity{0.700000}%
\pgfsetlinewidth{0.000000pt}%
\definecolor{currentstroke}{rgb}{0.000000,0.000000,0.000000}%
\pgfsetstrokecolor{currentstroke}%
\pgfsetdash{}{0pt}%
\pgfpathmoveto{\pgfqpoint{2.624645in}{2.286211in}}%
\pgfpathlineto{\pgfqpoint{2.638138in}{2.276622in}}%
\pgfpathlineto{\pgfqpoint{2.651630in}{2.267154in}}%
\pgfpathlineto{\pgfqpoint{2.665123in}{2.257806in}}%
\pgfpathlineto{\pgfqpoint{2.678616in}{2.248577in}}%
\pgfpathlineto{\pgfqpoint{2.687039in}{2.255828in}}%
\pgfpathlineto{\pgfqpoint{2.695454in}{2.263163in}}%
\pgfpathlineto{\pgfqpoint{2.703859in}{2.270578in}}%
\pgfpathlineto{\pgfqpoint{2.712257in}{2.278075in}}%
\pgfpathlineto{\pgfqpoint{2.698782in}{2.287197in}}%
\pgfpathlineto{\pgfqpoint{2.685308in}{2.296439in}}%
\pgfpathlineto{\pgfqpoint{2.671834in}{2.305800in}}%
\pgfpathlineto{\pgfqpoint{2.658360in}{2.315283in}}%
\pgfpathlineto{\pgfqpoint{2.649945in}{2.307885in}}%
\pgfpathlineto{\pgfqpoint{2.641520in}{2.300573in}}%
\pgfpathlineto{\pgfqpoint{2.633087in}{2.293348in}}%
\pgfpathlineto{\pgfqpoint{2.624645in}{2.286211in}}%
\pgfpathclose%
\pgfusepath{fill}%
\end{pgfscope}%
\begin{pgfscope}%
\pgfpathrectangle{\pgfqpoint{1.150000in}{0.150000in}}{\pgfqpoint{5.700000in}{5.700000in}}%
\pgfusepath{clip}%
\pgfsetbuttcap%
\pgfsetroundjoin%
\definecolor{currentfill}{rgb}{0.233603,0.313828,0.543914}%
\pgfsetfillcolor{currentfill}%
\pgfsetfillopacity{0.700000}%
\pgfsetlinewidth{0.000000pt}%
\definecolor{currentstroke}{rgb}{0.000000,0.000000,0.000000}%
\pgfsetstrokecolor{currentstroke}%
\pgfsetdash{}{0pt}%
\pgfpathmoveto{\pgfqpoint{5.661891in}{2.691976in}}%
\pgfpathlineto{\pgfqpoint{5.676003in}{2.691291in}}%
\pgfpathlineto{\pgfqpoint{5.690126in}{2.690672in}}%
\pgfpathlineto{\pgfqpoint{5.704258in}{2.690119in}}%
\pgfpathlineto{\pgfqpoint{5.718401in}{2.689632in}}%
\pgfpathlineto{\pgfqpoint{5.725711in}{2.698243in}}%
\pgfpathlineto{\pgfqpoint{5.733025in}{2.707087in}}%
\pgfpathlineto{\pgfqpoint{5.740341in}{2.716171in}}%
\pgfpathlineto{\pgfqpoint{5.747660in}{2.725504in}}%
\pgfpathlineto{\pgfqpoint{5.733542in}{2.726439in}}%
\pgfpathlineto{\pgfqpoint{5.719435in}{2.727439in}}%
\pgfpathlineto{\pgfqpoint{5.705338in}{2.728505in}}%
\pgfpathlineto{\pgfqpoint{5.691251in}{2.729637in}}%
\pgfpathlineto{\pgfqpoint{5.683907in}{2.719850in}}%
\pgfpathlineto{\pgfqpoint{5.676566in}{2.710316in}}%
\pgfpathlineto{\pgfqpoint{5.669227in}{2.701027in}}%
\pgfpathlineto{\pgfqpoint{5.661891in}{2.691976in}}%
\pgfpathclose%
\pgfusepath{fill}%
\end{pgfscope}%
\begin{pgfscope}%
\pgfpathrectangle{\pgfqpoint{1.150000in}{0.150000in}}{\pgfqpoint{5.700000in}{5.700000in}}%
\pgfusepath{clip}%
\pgfsetbuttcap%
\pgfsetroundjoin%
\definecolor{currentfill}{rgb}{0.277018,0.050344,0.375715}%
\pgfsetfillcolor{currentfill}%
\pgfsetfillopacity{0.700000}%
\pgfsetlinewidth{0.000000pt}%
\definecolor{currentstroke}{rgb}{0.000000,0.000000,0.000000}%
\pgfsetstrokecolor{currentstroke}%
\pgfsetdash{}{0pt}%
\pgfpathmoveto{\pgfqpoint{3.242846in}{2.146107in}}%
\pgfpathlineto{\pgfqpoint{3.256345in}{2.140658in}}%
\pgfpathlineto{\pgfqpoint{3.269847in}{2.135304in}}%
\pgfpathlineto{\pgfqpoint{3.283353in}{2.130043in}}%
\pgfpathlineto{\pgfqpoint{3.296863in}{2.124876in}}%
\pgfpathlineto{\pgfqpoint{3.305030in}{2.133628in}}%
\pgfpathlineto{\pgfqpoint{3.313190in}{2.142406in}}%
\pgfpathlineto{\pgfqpoint{3.321344in}{2.151212in}}%
\pgfpathlineto{\pgfqpoint{3.329492in}{2.160045in}}%
\pgfpathlineto{\pgfqpoint{3.315994in}{2.165190in}}%
\pgfpathlineto{\pgfqpoint{3.302501in}{2.170428in}}%
\pgfpathlineto{\pgfqpoint{3.289012in}{2.175759in}}%
\pgfpathlineto{\pgfqpoint{3.275526in}{2.181185in}}%
\pgfpathlineto{\pgfqpoint{3.267366in}{2.172367in}}%
\pgfpathlineto{\pgfqpoint{3.259199in}{2.163582in}}%
\pgfpathlineto{\pgfqpoint{3.251026in}{2.154828in}}%
\pgfpathlineto{\pgfqpoint{3.242846in}{2.146107in}}%
\pgfpathclose%
\pgfusepath{fill}%
\end{pgfscope}%
\begin{pgfscope}%
\pgfpathrectangle{\pgfqpoint{1.150000in}{0.150000in}}{\pgfqpoint{5.700000in}{5.700000in}}%
\pgfusepath{clip}%
\pgfsetbuttcap%
\pgfsetroundjoin%
\definecolor{currentfill}{rgb}{0.271828,0.209303,0.504434}%
\pgfsetfillcolor{currentfill}%
\pgfsetfillopacity{0.700000}%
\pgfsetlinewidth{0.000000pt}%
\definecolor{currentstroke}{rgb}{0.000000,0.000000,0.000000}%
\pgfsetstrokecolor{currentstroke}%
\pgfsetdash{}{0pt}%
\pgfpathmoveto{\pgfqpoint{4.863369in}{2.453297in}}%
\pgfpathlineto{\pgfqpoint{4.877258in}{2.452743in}}%
\pgfpathlineto{\pgfqpoint{4.891156in}{2.452259in}}%
\pgfpathlineto{\pgfqpoint{4.905063in}{2.451845in}}%
\pgfpathlineto{\pgfqpoint{4.918979in}{2.451501in}}%
\pgfpathlineto{\pgfqpoint{4.926556in}{2.459126in}}%
\pgfpathlineto{\pgfqpoint{4.934129in}{2.466825in}}%
\pgfpathlineto{\pgfqpoint{4.941698in}{2.474604in}}%
\pgfpathlineto{\pgfqpoint{4.949262in}{2.482469in}}%
\pgfpathlineto{\pgfqpoint{4.935362in}{2.483098in}}%
\pgfpathlineto{\pgfqpoint{4.921471in}{2.483796in}}%
\pgfpathlineto{\pgfqpoint{4.907589in}{2.484565in}}%
\pgfpathlineto{\pgfqpoint{4.893717in}{2.485404in}}%
\pgfpathlineto{\pgfqpoint{4.886136in}{2.477247in}}%
\pgfpathlineto{\pgfqpoint{4.878551in}{2.469181in}}%
\pgfpathlineto{\pgfqpoint{4.870962in}{2.461199in}}%
\pgfpathlineto{\pgfqpoint{4.863369in}{2.453297in}}%
\pgfpathclose%
\pgfusepath{fill}%
\end{pgfscope}%
\begin{pgfscope}%
\pgfpathrectangle{\pgfqpoint{1.150000in}{0.150000in}}{\pgfqpoint{5.700000in}{5.700000in}}%
\pgfusepath{clip}%
\pgfsetbuttcap%
\pgfsetroundjoin%
\definecolor{currentfill}{rgb}{0.280894,0.078907,0.402329}%
\pgfsetfillcolor{currentfill}%
\pgfsetfillopacity{0.700000}%
\pgfsetlinewidth{0.000000pt}%
\definecolor{currentstroke}{rgb}{0.000000,0.000000,0.000000}%
\pgfsetstrokecolor{currentstroke}%
\pgfsetdash{}{0pt}%
\pgfpathmoveto{\pgfqpoint{2.820071in}{2.209270in}}%
\pgfpathlineto{\pgfqpoint{2.833552in}{2.201178in}}%
\pgfpathlineto{\pgfqpoint{2.847035in}{2.193196in}}%
\pgfpathlineto{\pgfqpoint{2.860519in}{2.185324in}}%
\pgfpathlineto{\pgfqpoint{2.874004in}{2.177561in}}%
\pgfpathlineto{\pgfqpoint{2.882341in}{2.185415in}}%
\pgfpathlineto{\pgfqpoint{2.890670in}{2.193332in}}%
\pgfpathlineto{\pgfqpoint{2.898991in}{2.201311in}}%
\pgfpathlineto{\pgfqpoint{2.907305in}{2.209351in}}%
\pgfpathlineto{\pgfqpoint{2.893836in}{2.217029in}}%
\pgfpathlineto{\pgfqpoint{2.880369in}{2.224817in}}%
\pgfpathlineto{\pgfqpoint{2.866903in}{2.232713in}}%
\pgfpathlineto{\pgfqpoint{2.853439in}{2.240720in}}%
\pgfpathlineto{\pgfqpoint{2.845109in}{2.232757in}}%
\pgfpathlineto{\pgfqpoint{2.836771in}{2.224860in}}%
\pgfpathlineto{\pgfqpoint{2.828425in}{2.217031in}}%
\pgfpathlineto{\pgfqpoint{2.820071in}{2.209270in}}%
\pgfpathclose%
\pgfusepath{fill}%
\end{pgfscope}%
\begin{pgfscope}%
\pgfpathrectangle{\pgfqpoint{1.150000in}{0.150000in}}{\pgfqpoint{5.700000in}{5.700000in}}%
\pgfusepath{clip}%
\pgfsetbuttcap%
\pgfsetroundjoin%
\definecolor{currentfill}{rgb}{0.283197,0.115680,0.436115}%
\pgfsetfillcolor{currentfill}%
\pgfsetfillopacity{0.700000}%
\pgfsetlinewidth{0.000000pt}%
\definecolor{currentstroke}{rgb}{0.000000,0.000000,0.000000}%
\pgfsetstrokecolor{currentstroke}%
\pgfsetdash{}{0pt}%
\pgfpathmoveto{\pgfqpoint{4.150893in}{2.263193in}}%
\pgfpathlineto{\pgfqpoint{4.164576in}{2.261415in}}%
\pgfpathlineto{\pgfqpoint{4.178267in}{2.259715in}}%
\pgfpathlineto{\pgfqpoint{4.191965in}{2.258091in}}%
\pgfpathlineto{\pgfqpoint{4.205671in}{2.256544in}}%
\pgfpathlineto{\pgfqpoint{4.213517in}{2.265031in}}%
\pgfpathlineto{\pgfqpoint{4.221357in}{2.273528in}}%
\pgfpathlineto{\pgfqpoint{4.229192in}{2.282039in}}%
\pgfpathlineto{\pgfqpoint{4.237021in}{2.290567in}}%
\pgfpathlineto{\pgfqpoint{4.223327in}{2.292255in}}%
\pgfpathlineto{\pgfqpoint{4.209640in}{2.294020in}}%
\pgfpathlineto{\pgfqpoint{4.195961in}{2.295862in}}%
\pgfpathlineto{\pgfqpoint{4.182289in}{2.297780in}}%
\pgfpathlineto{\pgfqpoint{4.174449in}{2.289104in}}%
\pgfpathlineto{\pgfqpoint{4.166602in}{2.280449in}}%
\pgfpathlineto{\pgfqpoint{4.158750in}{2.271813in}}%
\pgfpathlineto{\pgfqpoint{4.150893in}{2.263193in}}%
\pgfpathclose%
\pgfusepath{fill}%
\end{pgfscope}%
\begin{pgfscope}%
\pgfpathrectangle{\pgfqpoint{1.150000in}{0.150000in}}{\pgfqpoint{5.700000in}{5.700000in}}%
\pgfusepath{clip}%
\pgfsetbuttcap%
\pgfsetroundjoin%
\definecolor{currentfill}{rgb}{0.278791,0.062145,0.386592}%
\pgfsetfillcolor{currentfill}%
\pgfsetfillopacity{0.700000}%
\pgfsetlinewidth{0.000000pt}%
\definecolor{currentstroke}{rgb}{0.000000,0.000000,0.000000}%
\pgfsetstrokecolor{currentstroke}%
\pgfsetdash{}{0pt}%
\pgfpathmoveto{\pgfqpoint{3.610657in}{2.161478in}}%
\pgfpathlineto{\pgfqpoint{3.624213in}{2.157832in}}%
\pgfpathlineto{\pgfqpoint{3.637774in}{2.154272in}}%
\pgfpathlineto{\pgfqpoint{3.651342in}{2.150797in}}%
\pgfpathlineto{\pgfqpoint{3.664914in}{2.147407in}}%
\pgfpathlineto{\pgfqpoint{3.672951in}{2.156353in}}%
\pgfpathlineto{\pgfqpoint{3.680981in}{2.165308in}}%
\pgfpathlineto{\pgfqpoint{3.689006in}{2.174273in}}%
\pgfpathlineto{\pgfqpoint{3.697025in}{2.183248in}}%
\pgfpathlineto{\pgfqpoint{3.683463in}{2.186677in}}%
\pgfpathlineto{\pgfqpoint{3.669907in}{2.190191in}}%
\pgfpathlineto{\pgfqpoint{3.656356in}{2.193790in}}%
\pgfpathlineto{\pgfqpoint{3.642811in}{2.197474in}}%
\pgfpathlineto{\pgfqpoint{3.634781in}{2.188452in}}%
\pgfpathlineto{\pgfqpoint{3.626746in}{2.179447in}}%
\pgfpathlineto{\pgfqpoint{3.618704in}{2.170455in}}%
\pgfpathlineto{\pgfqpoint{3.610657in}{2.161478in}}%
\pgfpathclose%
\pgfusepath{fill}%
\end{pgfscope}%
\begin{pgfscope}%
\pgfpathrectangle{\pgfqpoint{1.150000in}{0.150000in}}{\pgfqpoint{5.700000in}{5.700000in}}%
\pgfusepath{clip}%
\pgfsetbuttcap%
\pgfsetroundjoin%
\definecolor{currentfill}{rgb}{0.281412,0.155834,0.469201}%
\pgfsetfillcolor{currentfill}%
\pgfsetfillopacity{0.700000}%
\pgfsetlinewidth{0.000000pt}%
\definecolor{currentstroke}{rgb}{0.000000,0.000000,0.000000}%
\pgfsetstrokecolor{currentstroke}%
\pgfsetdash{}{0pt}%
\pgfpathmoveto{\pgfqpoint{4.464109in}{2.341240in}}%
\pgfpathlineto{\pgfqpoint{4.477882in}{2.340182in}}%
\pgfpathlineto{\pgfqpoint{4.491663in}{2.339197in}}%
\pgfpathlineto{\pgfqpoint{4.505452in}{2.338285in}}%
\pgfpathlineto{\pgfqpoint{4.519250in}{2.337447in}}%
\pgfpathlineto{\pgfqpoint{4.526980in}{2.345485in}}%
\pgfpathlineto{\pgfqpoint{4.534706in}{2.353551in}}%
\pgfpathlineto{\pgfqpoint{4.542426in}{2.361652in}}%
\pgfpathlineto{\pgfqpoint{4.550141in}{2.369789in}}%
\pgfpathlineto{\pgfqpoint{4.536356in}{2.370830in}}%
\pgfpathlineto{\pgfqpoint{4.522580in}{2.371944in}}%
\pgfpathlineto{\pgfqpoint{4.508812in}{2.373132in}}%
\pgfpathlineto{\pgfqpoint{4.495052in}{2.374393in}}%
\pgfpathlineto{\pgfqpoint{4.487324in}{2.366046in}}%
\pgfpathlineto{\pgfqpoint{4.479591in}{2.357740in}}%
\pgfpathlineto{\pgfqpoint{4.471853in}{2.349473in}}%
\pgfpathlineto{\pgfqpoint{4.464109in}{2.341240in}}%
\pgfpathclose%
\pgfusepath{fill}%
\end{pgfscope}%
\begin{pgfscope}%
\pgfpathrectangle{\pgfqpoint{1.150000in}{0.150000in}}{\pgfqpoint{5.700000in}{5.700000in}}%
\pgfusepath{clip}%
\pgfsetbuttcap%
\pgfsetroundjoin%
\definecolor{currentfill}{rgb}{0.204903,0.375746,0.553533}%
\pgfsetfillcolor{currentfill}%
\pgfsetfillopacity{0.700000}%
\pgfsetlinewidth{0.000000pt}%
\definecolor{currentstroke}{rgb}{0.000000,0.000000,0.000000}%
\pgfsetstrokecolor{currentstroke}%
\pgfsetdash{}{0pt}%
\pgfpathmoveto{\pgfqpoint{6.061817in}{2.830472in}}%
\pgfpathlineto{\pgfqpoint{6.076023in}{2.829180in}}%
\pgfpathlineto{\pgfqpoint{6.090240in}{2.827951in}}%
\pgfpathlineto{\pgfqpoint{6.104466in}{2.826787in}}%
\pgfpathlineto{\pgfqpoint{6.111729in}{2.837659in}}%
\pgfpathlineto{\pgfqpoint{6.119001in}{2.848878in}}%
\pgfpathlineto{\pgfqpoint{6.126283in}{2.860454in}}%
\pgfpathlineto{\pgfqpoint{6.133575in}{2.872397in}}%
\pgfpathlineto{\pgfqpoint{6.119378in}{2.874088in}}%
\pgfpathlineto{\pgfqpoint{6.105191in}{2.875844in}}%
\pgfpathlineto{\pgfqpoint{6.091014in}{2.877663in}}%
\pgfpathlineto{\pgfqpoint{6.083700in}{2.865320in}}%
\pgfpathlineto{\pgfqpoint{6.076396in}{2.853347in}}%
\pgfpathlineto{\pgfqpoint{6.069102in}{2.841734in}}%
\pgfpathlineto{\pgfqpoint{6.061817in}{2.830472in}}%
\pgfpathclose%
\pgfusepath{fill}%
\end{pgfscope}%
\begin{pgfscope}%
\pgfpathrectangle{\pgfqpoint{1.150000in}{0.150000in}}{\pgfqpoint{5.700000in}{5.700000in}}%
\pgfusepath{clip}%
\pgfsetbuttcap%
\pgfsetroundjoin%
\definecolor{currentfill}{rgb}{0.237441,0.305202,0.541921}%
\pgfsetfillcolor{currentfill}%
\pgfsetfillopacity{0.700000}%
\pgfsetlinewidth{0.000000pt}%
\definecolor{currentstroke}{rgb}{0.000000,0.000000,0.000000}%
\pgfsetstrokecolor{currentstroke}%
\pgfsetdash{}{0pt}%
\pgfpathmoveto{\pgfqpoint{5.576117in}{2.659684in}}%
\pgfpathlineto{\pgfqpoint{5.590214in}{2.659161in}}%
\pgfpathlineto{\pgfqpoint{5.604320in}{2.658706in}}%
\pgfpathlineto{\pgfqpoint{5.618437in}{2.658316in}}%
\pgfpathlineto{\pgfqpoint{5.632563in}{2.657993in}}%
\pgfpathlineto{\pgfqpoint{5.639893in}{2.666171in}}%
\pgfpathlineto{\pgfqpoint{5.647225in}{2.674556in}}%
\pgfpathlineto{\pgfqpoint{5.654557in}{2.683155in}}%
\pgfpathlineto{\pgfqpoint{5.661891in}{2.691976in}}%
\pgfpathlineto{\pgfqpoint{5.647789in}{2.692727in}}%
\pgfpathlineto{\pgfqpoint{5.633697in}{2.693544in}}%
\pgfpathlineto{\pgfqpoint{5.619614in}{2.694427in}}%
\pgfpathlineto{\pgfqpoint{5.605542in}{2.695375in}}%
\pgfpathlineto{\pgfqpoint{5.598184in}{2.686120in}}%
\pgfpathlineto{\pgfqpoint{5.590827in}{2.677091in}}%
\pgfpathlineto{\pgfqpoint{5.583472in}{2.668282in}}%
\pgfpathlineto{\pgfqpoint{5.576117in}{2.659684in}}%
\pgfpathclose%
\pgfusepath{fill}%
\end{pgfscope}%
\begin{pgfscope}%
\pgfpathrectangle{\pgfqpoint{1.150000in}{0.150000in}}{\pgfqpoint{5.700000in}{5.700000in}}%
\pgfusepath{clip}%
\pgfsetbuttcap%
\pgfsetroundjoin%
\definecolor{currentfill}{rgb}{0.258965,0.251537,0.524736}%
\pgfsetfillcolor{currentfill}%
\pgfsetfillopacity{0.700000}%
\pgfsetlinewidth{0.000000pt}%
\definecolor{currentstroke}{rgb}{0.000000,0.000000,0.000000}%
\pgfsetstrokecolor{currentstroke}%
\pgfsetdash{}{0pt}%
\pgfpathmoveto{\pgfqpoint{5.176746in}{2.539410in}}%
\pgfpathlineto{\pgfqpoint{5.190732in}{2.539049in}}%
\pgfpathlineto{\pgfqpoint{5.204727in}{2.538756in}}%
\pgfpathlineto{\pgfqpoint{5.218732in}{2.538532in}}%
\pgfpathlineto{\pgfqpoint{5.232747in}{2.538376in}}%
\pgfpathlineto{\pgfqpoint{5.240207in}{2.545904in}}%
\pgfpathlineto{\pgfqpoint{5.247664in}{2.553554in}}%
\pgfpathlineto{\pgfqpoint{5.255119in}{2.561333in}}%
\pgfpathlineto{\pgfqpoint{5.262571in}{2.569246in}}%
\pgfpathlineto{\pgfqpoint{5.248576in}{2.569749in}}%
\pgfpathlineto{\pgfqpoint{5.234591in}{2.570320in}}%
\pgfpathlineto{\pgfqpoint{5.220615in}{2.570959in}}%
\pgfpathlineto{\pgfqpoint{5.206648in}{2.571666in}}%
\pgfpathlineto{\pgfqpoint{5.199176in}{2.563399in}}%
\pgfpathlineto{\pgfqpoint{5.191702in}{2.555271in}}%
\pgfpathlineto{\pgfqpoint{5.184225in}{2.547277in}}%
\pgfpathlineto{\pgfqpoint{5.176746in}{2.539410in}}%
\pgfpathclose%
\pgfusepath{fill}%
\end{pgfscope}%
\begin{pgfscope}%
\pgfpathrectangle{\pgfqpoint{1.150000in}{0.150000in}}{\pgfqpoint{5.700000in}{5.700000in}}%
\pgfusepath{clip}%
\pgfsetbuttcap%
\pgfsetroundjoin%
\definecolor{currentfill}{rgb}{0.280894,0.078907,0.402329}%
\pgfsetfillcolor{currentfill}%
\pgfsetfillopacity{0.700000}%
\pgfsetlinewidth{0.000000pt}%
\definecolor{currentstroke}{rgb}{0.000000,0.000000,0.000000}%
\pgfsetstrokecolor{currentstroke}%
\pgfsetdash{}{0pt}%
\pgfpathmoveto{\pgfqpoint{3.837665in}{2.194378in}}%
\pgfpathlineto{\pgfqpoint{3.851271in}{2.191635in}}%
\pgfpathlineto{\pgfqpoint{3.864883in}{2.188974in}}%
\pgfpathlineto{\pgfqpoint{3.878501in}{2.186394in}}%
\pgfpathlineto{\pgfqpoint{3.892126in}{2.183894in}}%
\pgfpathlineto{\pgfqpoint{3.900084in}{2.192727in}}%
\pgfpathlineto{\pgfqpoint{3.908036in}{2.201565in}}%
\pgfpathlineto{\pgfqpoint{3.915983in}{2.210408in}}%
\pgfpathlineto{\pgfqpoint{3.923924in}{2.219260in}}%
\pgfpathlineto{\pgfqpoint{3.910310in}{2.221839in}}%
\pgfpathlineto{\pgfqpoint{3.896702in}{2.224500in}}%
\pgfpathlineto{\pgfqpoint{3.883101in}{2.227241in}}%
\pgfpathlineto{\pgfqpoint{3.869506in}{2.230063in}}%
\pgfpathlineto{\pgfqpoint{3.861554in}{2.221124in}}%
\pgfpathlineto{\pgfqpoint{3.853597in}{2.212198in}}%
\pgfpathlineto{\pgfqpoint{3.845634in}{2.203283in}}%
\pgfpathlineto{\pgfqpoint{3.837665in}{2.194378in}}%
\pgfpathclose%
\pgfusepath{fill}%
\end{pgfscope}%
\begin{pgfscope}%
\pgfpathrectangle{\pgfqpoint{1.150000in}{0.150000in}}{\pgfqpoint{5.700000in}{5.700000in}}%
\pgfusepath{clip}%
\pgfsetbuttcap%
\pgfsetroundjoin%
\definecolor{currentfill}{rgb}{0.280255,0.165693,0.476498}%
\pgfsetfillcolor{currentfill}%
\pgfsetfillopacity{0.700000}%
\pgfsetlinewidth{0.000000pt}%
\definecolor{currentstroke}{rgb}{0.000000,0.000000,0.000000}%
\pgfsetstrokecolor{currentstroke}%
\pgfsetdash{}{0pt}%
\pgfpathmoveto{\pgfqpoint{2.428557in}{2.384990in}}%
\pgfpathlineto{\pgfqpoint{2.442084in}{2.373738in}}%
\pgfpathlineto{\pgfqpoint{2.455608in}{2.362620in}}%
\pgfpathlineto{\pgfqpoint{2.469131in}{2.351635in}}%
\pgfpathlineto{\pgfqpoint{2.482653in}{2.340782in}}%
\pgfpathlineto{\pgfqpoint{2.491174in}{2.347282in}}%
\pgfpathlineto{\pgfqpoint{2.499686in}{2.353886in}}%
\pgfpathlineto{\pgfqpoint{2.508188in}{2.360593in}}%
\pgfpathlineto{\pgfqpoint{2.516680in}{2.367401in}}%
\pgfpathlineto{\pgfqpoint{2.503179in}{2.378126in}}%
\pgfpathlineto{\pgfqpoint{2.489678in}{2.388982in}}%
\pgfpathlineto{\pgfqpoint{2.476175in}{2.399971in}}%
\pgfpathlineto{\pgfqpoint{2.462670in}{2.411095in}}%
\pgfpathlineto{\pgfqpoint{2.454157in}{2.404407in}}%
\pgfpathlineto{\pgfqpoint{2.445634in}{2.397827in}}%
\pgfpathlineto{\pgfqpoint{2.437101in}{2.391354in}}%
\pgfpathlineto{\pgfqpoint{2.428557in}{2.384990in}}%
\pgfpathclose%
\pgfusepath{fill}%
\end{pgfscope}%
\begin{pgfscope}%
\pgfpathrectangle{\pgfqpoint{1.150000in}{0.150000in}}{\pgfqpoint{5.700000in}{5.700000in}}%
\pgfusepath{clip}%
\pgfsetbuttcap%
\pgfsetroundjoin%
\definecolor{currentfill}{rgb}{0.277018,0.050344,0.375715}%
\pgfsetfillcolor{currentfill}%
\pgfsetfillopacity{0.700000}%
\pgfsetlinewidth{0.000000pt}%
\definecolor{currentstroke}{rgb}{0.000000,0.000000,0.000000}%
\pgfsetstrokecolor{currentstroke}%
\pgfsetdash{}{0pt}%
\pgfpathmoveto{\pgfqpoint{3.383524in}{2.140390in}}%
\pgfpathlineto{\pgfqpoint{3.397043in}{2.135705in}}%
\pgfpathlineto{\pgfqpoint{3.410566in}{2.131110in}}%
\pgfpathlineto{\pgfqpoint{3.424095in}{2.126605in}}%
\pgfpathlineto{\pgfqpoint{3.437628in}{2.122190in}}%
\pgfpathlineto{\pgfqpoint{3.445745in}{2.131070in}}%
\pgfpathlineto{\pgfqpoint{3.453856in}{2.139968in}}%
\pgfpathlineto{\pgfqpoint{3.461962in}{2.148884in}}%
\pgfpathlineto{\pgfqpoint{3.470061in}{2.157820in}}%
\pgfpathlineto{\pgfqpoint{3.456540in}{2.162233in}}%
\pgfpathlineto{\pgfqpoint{3.443024in}{2.166736in}}%
\pgfpathlineto{\pgfqpoint{3.429512in}{2.171329in}}%
\pgfpathlineto{\pgfqpoint{3.416005in}{2.176012in}}%
\pgfpathlineto{\pgfqpoint{3.407894in}{2.167071in}}%
\pgfpathlineto{\pgfqpoint{3.399777in}{2.158154in}}%
\pgfpathlineto{\pgfqpoint{3.391653in}{2.149261in}}%
\pgfpathlineto{\pgfqpoint{3.383524in}{2.140390in}}%
\pgfpathclose%
\pgfusepath{fill}%
\end{pgfscope}%
\begin{pgfscope}%
\pgfpathrectangle{\pgfqpoint{1.150000in}{0.150000in}}{\pgfqpoint{5.700000in}{5.700000in}}%
\pgfusepath{clip}%
\pgfsetbuttcap%
\pgfsetroundjoin%
\definecolor{currentfill}{rgb}{0.274128,0.199721,0.498911}%
\pgfsetfillcolor{currentfill}%
\pgfsetfillopacity{0.700000}%
\pgfsetlinewidth{0.000000pt}%
\definecolor{currentstroke}{rgb}{0.000000,0.000000,0.000000}%
\pgfsetstrokecolor{currentstroke}%
\pgfsetdash{}{0pt}%
\pgfpathmoveto{\pgfqpoint{4.777422in}{2.424254in}}%
\pgfpathlineto{\pgfqpoint{4.791290in}{2.423682in}}%
\pgfpathlineto{\pgfqpoint{4.805168in}{2.423180in}}%
\pgfpathlineto{\pgfqpoint{4.819055in}{2.422749in}}%
\pgfpathlineto{\pgfqpoint{4.832950in}{2.422390in}}%
\pgfpathlineto{\pgfqpoint{4.840562in}{2.430021in}}%
\pgfpathlineto{\pgfqpoint{4.848169in}{2.437713in}}%
\pgfpathlineto{\pgfqpoint{4.855771in}{2.445470in}}%
\pgfpathlineto{\pgfqpoint{4.863369in}{2.453297in}}%
\pgfpathlineto{\pgfqpoint{4.849489in}{2.453922in}}%
\pgfpathlineto{\pgfqpoint{4.835618in}{2.454617in}}%
\pgfpathlineto{\pgfqpoint{4.821756in}{2.455383in}}%
\pgfpathlineto{\pgfqpoint{4.807903in}{2.456220in}}%
\pgfpathlineto{\pgfqpoint{4.800289in}{2.448121in}}%
\pgfpathlineto{\pgfqpoint{4.792671in}{2.440097in}}%
\pgfpathlineto{\pgfqpoint{4.785049in}{2.432143in}}%
\pgfpathlineto{\pgfqpoint{4.777422in}{2.424254in}}%
\pgfpathclose%
\pgfusepath{fill}%
\end{pgfscope}%
\begin{pgfscope}%
\pgfpathrectangle{\pgfqpoint{1.150000in}{0.150000in}}{\pgfqpoint{5.700000in}{5.700000in}}%
\pgfusepath{clip}%
\pgfsetbuttcap%
\pgfsetroundjoin%
\definecolor{currentfill}{rgb}{0.282910,0.105393,0.426902}%
\pgfsetfillcolor{currentfill}%
\pgfsetfillopacity{0.700000}%
\pgfsetlinewidth{0.000000pt}%
\definecolor{currentstroke}{rgb}{0.000000,0.000000,0.000000}%
\pgfsetstrokecolor{currentstroke}%
\pgfsetdash{}{0pt}%
\pgfpathmoveto{\pgfqpoint{2.678616in}{2.248577in}}%
\pgfpathlineto{\pgfqpoint{2.692109in}{2.239467in}}%
\pgfpathlineto{\pgfqpoint{2.705603in}{2.230473in}}%
\pgfpathlineto{\pgfqpoint{2.719096in}{2.221597in}}%
\pgfpathlineto{\pgfqpoint{2.732591in}{2.212835in}}%
\pgfpathlineto{\pgfqpoint{2.740995in}{2.220200in}}%
\pgfpathlineto{\pgfqpoint{2.749391in}{2.227643in}}%
\pgfpathlineto{\pgfqpoint{2.757779in}{2.235162in}}%
\pgfpathlineto{\pgfqpoint{2.766158in}{2.242756in}}%
\pgfpathlineto{\pgfqpoint{2.752682in}{2.251412in}}%
\pgfpathlineto{\pgfqpoint{2.739206in}{2.260183in}}%
\pgfpathlineto{\pgfqpoint{2.725731in}{2.269070in}}%
\pgfpathlineto{\pgfqpoint{2.712257in}{2.278075in}}%
\pgfpathlineto{\pgfqpoint{2.703859in}{2.270578in}}%
\pgfpathlineto{\pgfqpoint{2.695454in}{2.263163in}}%
\pgfpathlineto{\pgfqpoint{2.687039in}{2.255828in}}%
\pgfpathlineto{\pgfqpoint{2.678616in}{2.248577in}}%
\pgfpathclose%
\pgfusepath{fill}%
\end{pgfscope}%
\begin{pgfscope}%
\pgfpathrectangle{\pgfqpoint{1.150000in}{0.150000in}}{\pgfqpoint{5.700000in}{5.700000in}}%
\pgfusepath{clip}%
\pgfsetbuttcap%
\pgfsetroundjoin%
\definecolor{currentfill}{rgb}{0.210503,0.363727,0.552206}%
\pgfsetfillcolor{currentfill}%
\pgfsetfillopacity{0.700000}%
\pgfsetlinewidth{0.000000pt}%
\definecolor{currentstroke}{rgb}{0.000000,0.000000,0.000000}%
\pgfsetstrokecolor{currentstroke}%
\pgfsetdash{}{0pt}%
\pgfpathmoveto{\pgfqpoint{5.975918in}{2.792538in}}%
\pgfpathlineto{\pgfqpoint{5.990112in}{2.791495in}}%
\pgfpathlineto{\pgfqpoint{6.004317in}{2.790516in}}%
\pgfpathlineto{\pgfqpoint{6.018532in}{2.789602in}}%
\pgfpathlineto{\pgfqpoint{6.032757in}{2.788753in}}%
\pgfpathlineto{\pgfqpoint{6.040011in}{2.798702in}}%
\pgfpathlineto{\pgfqpoint{6.047272in}{2.808966in}}%
\pgfpathlineto{\pgfqpoint{6.054540in}{2.819553in}}%
\pgfpathlineto{\pgfqpoint{6.061817in}{2.830472in}}%
\pgfpathlineto{\pgfqpoint{6.047621in}{2.831830in}}%
\pgfpathlineto{\pgfqpoint{6.033435in}{2.833251in}}%
\pgfpathlineto{\pgfqpoint{6.019259in}{2.834737in}}%
\pgfpathlineto{\pgfqpoint{6.005093in}{2.836287in}}%
\pgfpathlineto{\pgfqpoint{5.997788in}{2.824853in}}%
\pgfpathlineto{\pgfqpoint{5.990490in}{2.813756in}}%
\pgfpathlineto{\pgfqpoint{5.983201in}{2.802987in}}%
\pgfpathlineto{\pgfqpoint{5.975918in}{2.792538in}}%
\pgfpathclose%
\pgfusepath{fill}%
\end{pgfscope}%
\begin{pgfscope}%
\pgfpathrectangle{\pgfqpoint{1.150000in}{0.150000in}}{\pgfqpoint{5.700000in}{5.700000in}}%
\pgfusepath{clip}%
\pgfsetbuttcap%
\pgfsetroundjoin%
\definecolor{currentfill}{rgb}{0.282910,0.105393,0.426902}%
\pgfsetfillcolor{currentfill}%
\pgfsetfillopacity{0.700000}%
\pgfsetlinewidth{0.000000pt}%
\definecolor{currentstroke}{rgb}{0.000000,0.000000,0.000000}%
\pgfsetstrokecolor{currentstroke}%
\pgfsetdash{}{0pt}%
\pgfpathmoveto{\pgfqpoint{4.064702in}{2.236217in}}%
\pgfpathlineto{\pgfqpoint{4.078368in}{2.234251in}}%
\pgfpathlineto{\pgfqpoint{4.092040in}{2.232362in}}%
\pgfpathlineto{\pgfqpoint{4.105720in}{2.230551in}}%
\pgfpathlineto{\pgfqpoint{4.119407in}{2.228818in}}%
\pgfpathlineto{\pgfqpoint{4.127287in}{2.237400in}}%
\pgfpathlineto{\pgfqpoint{4.135161in}{2.245989in}}%
\pgfpathlineto{\pgfqpoint{4.143030in}{2.254585in}}%
\pgfpathlineto{\pgfqpoint{4.150893in}{2.263193in}}%
\pgfpathlineto{\pgfqpoint{4.137217in}{2.265047in}}%
\pgfpathlineto{\pgfqpoint{4.123548in}{2.266979in}}%
\pgfpathlineto{\pgfqpoint{4.109887in}{2.268988in}}%
\pgfpathlineto{\pgfqpoint{4.096233in}{2.271076in}}%
\pgfpathlineto{\pgfqpoint{4.088359in}{2.262340in}}%
\pgfpathlineto{\pgfqpoint{4.080479in}{2.253620in}}%
\pgfpathlineto{\pgfqpoint{4.072593in}{2.244913in}}%
\pgfpathlineto{\pgfqpoint{4.064702in}{2.236217in}}%
\pgfpathclose%
\pgfusepath{fill}%
\end{pgfscope}%
\begin{pgfscope}%
\pgfpathrectangle{\pgfqpoint{1.150000in}{0.150000in}}{\pgfqpoint{5.700000in}{5.700000in}}%
\pgfusepath{clip}%
\pgfsetbuttcap%
\pgfsetroundjoin%
\definecolor{currentfill}{rgb}{0.243113,0.292092,0.538516}%
\pgfsetfillcolor{currentfill}%
\pgfsetfillopacity{0.700000}%
\pgfsetlinewidth{0.000000pt}%
\definecolor{currentstroke}{rgb}{0.000000,0.000000,0.000000}%
\pgfsetstrokecolor{currentstroke}%
\pgfsetdash{}{0pt}%
\pgfpathmoveto{\pgfqpoint{5.490325in}{2.628384in}}%
\pgfpathlineto{\pgfqpoint{5.504405in}{2.628003in}}%
\pgfpathlineto{\pgfqpoint{5.518495in}{2.627689in}}%
\pgfpathlineto{\pgfqpoint{5.532595in}{2.627442in}}%
\pgfpathlineto{\pgfqpoint{5.546705in}{2.627261in}}%
\pgfpathlineto{\pgfqpoint{5.554058in}{2.635086in}}%
\pgfpathlineto{\pgfqpoint{5.561411in}{2.643093in}}%
\pgfpathlineto{\pgfqpoint{5.568764in}{2.651290in}}%
\pgfpathlineto{\pgfqpoint{5.576117in}{2.659684in}}%
\pgfpathlineto{\pgfqpoint{5.562031in}{2.660272in}}%
\pgfpathlineto{\pgfqpoint{5.547955in}{2.660927in}}%
\pgfpathlineto{\pgfqpoint{5.533888in}{2.661648in}}%
\pgfpathlineto{\pgfqpoint{5.519831in}{2.662436in}}%
\pgfpathlineto{\pgfqpoint{5.512454in}{2.653628in}}%
\pgfpathlineto{\pgfqpoint{5.505078in}{2.645021in}}%
\pgfpathlineto{\pgfqpoint{5.497702in}{2.636609in}}%
\pgfpathlineto{\pgfqpoint{5.490325in}{2.628384in}}%
\pgfpathclose%
\pgfusepath{fill}%
\end{pgfscope}%
\begin{pgfscope}%
\pgfpathrectangle{\pgfqpoint{1.150000in}{0.150000in}}{\pgfqpoint{5.700000in}{5.700000in}}%
\pgfusepath{clip}%
\pgfsetbuttcap%
\pgfsetroundjoin%
\definecolor{currentfill}{rgb}{0.282290,0.145912,0.461510}%
\pgfsetfillcolor{currentfill}%
\pgfsetfillopacity{0.700000}%
\pgfsetlinewidth{0.000000pt}%
\definecolor{currentstroke}{rgb}{0.000000,0.000000,0.000000}%
\pgfsetstrokecolor{currentstroke}%
\pgfsetdash{}{0pt}%
\pgfpathmoveto{\pgfqpoint{4.378021in}{2.312823in}}%
\pgfpathlineto{\pgfqpoint{4.391773in}{2.311651in}}%
\pgfpathlineto{\pgfqpoint{4.405534in}{2.310553in}}%
\pgfpathlineto{\pgfqpoint{4.419303in}{2.309530in}}%
\pgfpathlineto{\pgfqpoint{4.433080in}{2.308581in}}%
\pgfpathlineto{\pgfqpoint{4.440845in}{2.316712in}}%
\pgfpathlineto{\pgfqpoint{4.448605in}{2.324863in}}%
\pgfpathlineto{\pgfqpoint{4.456360in}{2.333038in}}%
\pgfpathlineto{\pgfqpoint{4.464109in}{2.341240in}}%
\pgfpathlineto{\pgfqpoint{4.450345in}{2.342372in}}%
\pgfpathlineto{\pgfqpoint{4.436588in}{2.343578in}}%
\pgfpathlineto{\pgfqpoint{4.422840in}{2.344859in}}%
\pgfpathlineto{\pgfqpoint{4.409100in}{2.346213in}}%
\pgfpathlineto{\pgfqpoint{4.401338in}{2.337821in}}%
\pgfpathlineto{\pgfqpoint{4.393571in}{2.329461in}}%
\pgfpathlineto{\pgfqpoint{4.385799in}{2.321129in}}%
\pgfpathlineto{\pgfqpoint{4.378021in}{2.312823in}}%
\pgfpathclose%
\pgfusepath{fill}%
\end{pgfscope}%
\begin{pgfscope}%
\pgfpathrectangle{\pgfqpoint{1.150000in}{0.150000in}}{\pgfqpoint{5.700000in}{5.700000in}}%
\pgfusepath{clip}%
\pgfsetbuttcap%
\pgfsetroundjoin%
\definecolor{currentfill}{rgb}{0.263663,0.237631,0.518762}%
\pgfsetfillcolor{currentfill}%
\pgfsetfillopacity{0.700000}%
\pgfsetlinewidth{0.000000pt}%
\definecolor{currentstroke}{rgb}{0.000000,0.000000,0.000000}%
\pgfsetstrokecolor{currentstroke}%
\pgfsetdash{}{0pt}%
\pgfpathmoveto{\pgfqpoint{5.090875in}{2.509920in}}%
\pgfpathlineto{\pgfqpoint{5.104841in}{2.509611in}}%
\pgfpathlineto{\pgfqpoint{5.118817in}{2.509370in}}%
\pgfpathlineto{\pgfqpoint{5.132803in}{2.509198in}}%
\pgfpathlineto{\pgfqpoint{5.146798in}{2.509095in}}%
\pgfpathlineto{\pgfqpoint{5.154290in}{2.516513in}}%
\pgfpathlineto{\pgfqpoint{5.161778in}{2.524034in}}%
\pgfpathlineto{\pgfqpoint{5.169264in}{2.531664in}}%
\pgfpathlineto{\pgfqpoint{5.176746in}{2.539410in}}%
\pgfpathlineto{\pgfqpoint{5.162770in}{2.539840in}}%
\pgfpathlineto{\pgfqpoint{5.148803in}{2.540338in}}%
\pgfpathlineto{\pgfqpoint{5.134846in}{2.540904in}}%
\pgfpathlineto{\pgfqpoint{5.120898in}{2.541540in}}%
\pgfpathlineto{\pgfqpoint{5.113397in}{2.533461in}}%
\pgfpathlineto{\pgfqpoint{5.105893in}{2.525502in}}%
\pgfpathlineto{\pgfqpoint{5.098386in}{2.517657in}}%
\pgfpathlineto{\pgfqpoint{5.090875in}{2.509920in}}%
\pgfpathclose%
\pgfusepath{fill}%
\end{pgfscope}%
\begin{pgfscope}%
\pgfpathrectangle{\pgfqpoint{1.150000in}{0.150000in}}{\pgfqpoint{5.700000in}{5.700000in}}%
\pgfusepath{clip}%
\pgfsetbuttcap%
\pgfsetroundjoin%
\definecolor{currentfill}{rgb}{0.277941,0.056324,0.381191}%
\pgfsetfillcolor{currentfill}%
\pgfsetfillopacity{0.700000}%
\pgfsetlinewidth{0.000000pt}%
\definecolor{currentstroke}{rgb}{0.000000,0.000000,0.000000}%
\pgfsetstrokecolor{currentstroke}%
\pgfsetdash{}{0pt}%
\pgfpathmoveto{\pgfqpoint{3.015128in}{2.151746in}}%
\pgfpathlineto{\pgfqpoint{3.028616in}{2.145012in}}%
\pgfpathlineto{\pgfqpoint{3.042106in}{2.138381in}}%
\pgfpathlineto{\pgfqpoint{3.055600in}{2.131850in}}%
\pgfpathlineto{\pgfqpoint{3.069096in}{2.125419in}}%
\pgfpathlineto{\pgfqpoint{3.077356in}{2.133734in}}%
\pgfpathlineto{\pgfqpoint{3.085610in}{2.142094in}}%
\pgfpathlineto{\pgfqpoint{3.093857in}{2.150498in}}%
\pgfpathlineto{\pgfqpoint{3.102096in}{2.158945in}}%
\pgfpathlineto{\pgfqpoint{3.088615in}{2.165312in}}%
\pgfpathlineto{\pgfqpoint{3.075136in}{2.171779in}}%
\pgfpathlineto{\pgfqpoint{3.061660in}{2.178347in}}%
\pgfpathlineto{\pgfqpoint{3.048187in}{2.185017in}}%
\pgfpathlineto{\pgfqpoint{3.039932in}{2.176626in}}%
\pgfpathlineto{\pgfqpoint{3.031671in}{2.168283in}}%
\pgfpathlineto{\pgfqpoint{3.023403in}{2.159990in}}%
\pgfpathlineto{\pgfqpoint{3.015128in}{2.151746in}}%
\pgfpathclose%
\pgfusepath{fill}%
\end{pgfscope}%
\begin{pgfscope}%
\pgfpathrectangle{\pgfqpoint{1.150000in}{0.150000in}}{\pgfqpoint{5.700000in}{5.700000in}}%
\pgfusepath{clip}%
\pgfsetbuttcap%
\pgfsetroundjoin%
\definecolor{currentfill}{rgb}{0.277941,0.056324,0.381191}%
\pgfsetfillcolor{currentfill}%
\pgfsetfillopacity{0.700000}%
\pgfsetlinewidth{0.000000pt}%
\definecolor{currentstroke}{rgb}{0.000000,0.000000,0.000000}%
\pgfsetstrokecolor{currentstroke}%
\pgfsetdash{}{0pt}%
\pgfpathmoveto{\pgfqpoint{3.524194in}{2.141052in}}%
\pgfpathlineto{\pgfqpoint{3.537740in}{2.137080in}}%
\pgfpathlineto{\pgfqpoint{3.551292in}{2.133195in}}%
\pgfpathlineto{\pgfqpoint{3.564848in}{2.129396in}}%
\pgfpathlineto{\pgfqpoint{3.578410in}{2.125684in}}%
\pgfpathlineto{\pgfqpoint{3.586480in}{2.134616in}}%
\pgfpathlineto{\pgfqpoint{3.594545in}{2.143559in}}%
\pgfpathlineto{\pgfqpoint{3.602604in}{2.152513in}}%
\pgfpathlineto{\pgfqpoint{3.610657in}{2.161478in}}%
\pgfpathlineto{\pgfqpoint{3.597106in}{2.165209in}}%
\pgfpathlineto{\pgfqpoint{3.583561in}{2.169026in}}%
\pgfpathlineto{\pgfqpoint{3.570021in}{2.172930in}}%
\pgfpathlineto{\pgfqpoint{3.556487in}{2.176921in}}%
\pgfpathlineto{\pgfqpoint{3.548423in}{2.167930in}}%
\pgfpathlineto{\pgfqpoint{3.540352in}{2.158955in}}%
\pgfpathlineto{\pgfqpoint{3.532276in}{2.149996in}}%
\pgfpathlineto{\pgfqpoint{3.524194in}{2.141052in}}%
\pgfpathclose%
\pgfusepath{fill}%
\end{pgfscope}%
\begin{pgfscope}%
\pgfpathrectangle{\pgfqpoint{1.150000in}{0.150000in}}{\pgfqpoint{5.700000in}{5.700000in}}%
\pgfusepath{clip}%
\pgfsetbuttcap%
\pgfsetroundjoin%
\definecolor{currentfill}{rgb}{0.281887,0.150881,0.465405}%
\pgfsetfillcolor{currentfill}%
\pgfsetfillopacity{0.700000}%
\pgfsetlinewidth{0.000000pt}%
\definecolor{currentstroke}{rgb}{0.000000,0.000000,0.000000}%
\pgfsetstrokecolor{currentstroke}%
\pgfsetdash{}{0pt}%
\pgfpathmoveto{\pgfqpoint{2.482653in}{2.340782in}}%
\pgfpathlineto{\pgfqpoint{2.496173in}{2.330060in}}%
\pgfpathlineto{\pgfqpoint{2.509691in}{2.319467in}}%
\pgfpathlineto{\pgfqpoint{2.523209in}{2.309003in}}%
\pgfpathlineto{\pgfqpoint{2.536725in}{2.298666in}}%
\pgfpathlineto{\pgfqpoint{2.545226in}{2.305301in}}%
\pgfpathlineto{\pgfqpoint{2.553716in}{2.312036in}}%
\pgfpathlineto{\pgfqpoint{2.562197in}{2.318868in}}%
\pgfpathlineto{\pgfqpoint{2.570668in}{2.325797in}}%
\pgfpathlineto{\pgfqpoint{2.557173in}{2.336006in}}%
\pgfpathlineto{\pgfqpoint{2.543676in}{2.346342in}}%
\pgfpathlineto{\pgfqpoint{2.530178in}{2.356807in}}%
\pgfpathlineto{\pgfqpoint{2.516680in}{2.367401in}}%
\pgfpathlineto{\pgfqpoint{2.508188in}{2.360593in}}%
\pgfpathlineto{\pgfqpoint{2.499686in}{2.353886in}}%
\pgfpathlineto{\pgfqpoint{2.491174in}{2.347282in}}%
\pgfpathlineto{\pgfqpoint{2.482653in}{2.340782in}}%
\pgfpathclose%
\pgfusepath{fill}%
\end{pgfscope}%
\begin{pgfscope}%
\pgfpathrectangle{\pgfqpoint{1.150000in}{0.150000in}}{\pgfqpoint{5.700000in}{5.700000in}}%
\pgfusepath{clip}%
\pgfsetbuttcap%
\pgfsetroundjoin%
\definecolor{currentfill}{rgb}{0.277018,0.050344,0.375715}%
\pgfsetfillcolor{currentfill}%
\pgfsetfillopacity{0.700000}%
\pgfsetlinewidth{0.000000pt}%
\definecolor{currentstroke}{rgb}{0.000000,0.000000,0.000000}%
\pgfsetstrokecolor{currentstroke}%
\pgfsetdash{}{0pt}%
\pgfpathmoveto{\pgfqpoint{3.156053in}{2.134467in}}%
\pgfpathlineto{\pgfqpoint{3.169551in}{2.128592in}}%
\pgfpathlineto{\pgfqpoint{3.183052in}{2.122814in}}%
\pgfpathlineto{\pgfqpoint{3.196556in}{2.117132in}}%
\pgfpathlineto{\pgfqpoint{3.210064in}{2.111545in}}%
\pgfpathlineto{\pgfqpoint{3.218269in}{2.120137in}}%
\pgfpathlineto{\pgfqpoint{3.226468in}{2.128761in}}%
\pgfpathlineto{\pgfqpoint{3.234660in}{2.137418in}}%
\pgfpathlineto{\pgfqpoint{3.242846in}{2.146107in}}%
\pgfpathlineto{\pgfqpoint{3.229352in}{2.151651in}}%
\pgfpathlineto{\pgfqpoint{3.215861in}{2.157290in}}%
\pgfpathlineto{\pgfqpoint{3.202373in}{2.163025in}}%
\pgfpathlineto{\pgfqpoint{3.188890in}{2.168857in}}%
\pgfpathlineto{\pgfqpoint{3.180690in}{2.160203in}}%
\pgfpathlineto{\pgfqpoint{3.172485in}{2.151587in}}%
\pgfpathlineto{\pgfqpoint{3.164272in}{2.143008in}}%
\pgfpathlineto{\pgfqpoint{3.156053in}{2.134467in}}%
\pgfpathclose%
\pgfusepath{fill}%
\end{pgfscope}%
\begin{pgfscope}%
\pgfpathrectangle{\pgfqpoint{1.150000in}{0.150000in}}{\pgfqpoint{5.700000in}{5.700000in}}%
\pgfusepath{clip}%
\pgfsetbuttcap%
\pgfsetroundjoin%
\definecolor{currentfill}{rgb}{0.280267,0.073417,0.397163}%
\pgfsetfillcolor{currentfill}%
\pgfsetfillopacity{0.700000}%
\pgfsetlinewidth{0.000000pt}%
\definecolor{currentstroke}{rgb}{0.000000,0.000000,0.000000}%
\pgfsetstrokecolor{currentstroke}%
\pgfsetdash{}{0pt}%
\pgfpathmoveto{\pgfqpoint{3.751330in}{2.170370in}}%
\pgfpathlineto{\pgfqpoint{3.764922in}{2.167358in}}%
\pgfpathlineto{\pgfqpoint{3.778520in}{2.164429in}}%
\pgfpathlineto{\pgfqpoint{3.792123in}{2.161583in}}%
\pgfpathlineto{\pgfqpoint{3.805733in}{2.158818in}}%
\pgfpathlineto{\pgfqpoint{3.813725in}{2.167702in}}%
\pgfpathlineto{\pgfqpoint{3.821711in}{2.176589in}}%
\pgfpathlineto{\pgfqpoint{3.829691in}{2.185480in}}%
\pgfpathlineto{\pgfqpoint{3.837665in}{2.194378in}}%
\pgfpathlineto{\pgfqpoint{3.824066in}{2.197202in}}%
\pgfpathlineto{\pgfqpoint{3.810473in}{2.200108in}}%
\pgfpathlineto{\pgfqpoint{3.796886in}{2.203096in}}%
\pgfpathlineto{\pgfqpoint{3.783306in}{2.206167in}}%
\pgfpathlineto{\pgfqpoint{3.775320in}{2.197202in}}%
\pgfpathlineto{\pgfqpoint{3.767329in}{2.188249in}}%
\pgfpathlineto{\pgfqpoint{3.759333in}{2.179305in}}%
\pgfpathlineto{\pgfqpoint{3.751330in}{2.170370in}}%
\pgfpathclose%
\pgfusepath{fill}%
\end{pgfscope}%
\begin{pgfscope}%
\pgfpathrectangle{\pgfqpoint{1.150000in}{0.150000in}}{\pgfqpoint{5.700000in}{5.700000in}}%
\pgfusepath{clip}%
\pgfsetbuttcap%
\pgfsetroundjoin%
\definecolor{currentfill}{rgb}{0.218130,0.347432,0.550038}%
\pgfsetfillcolor{currentfill}%
\pgfsetfillopacity{0.700000}%
\pgfsetlinewidth{0.000000pt}%
\definecolor{currentstroke}{rgb}{0.000000,0.000000,0.000000}%
\pgfsetstrokecolor{currentstroke}%
\pgfsetdash{}{0pt}%
\pgfpathmoveto{\pgfqpoint{5.890062in}{2.756626in}}%
\pgfpathlineto{\pgfqpoint{5.904243in}{2.755811in}}%
\pgfpathlineto{\pgfqpoint{5.918435in}{2.755061in}}%
\pgfpathlineto{\pgfqpoint{5.932637in}{2.754377in}}%
\pgfpathlineto{\pgfqpoint{5.946850in}{2.753757in}}%
\pgfpathlineto{\pgfqpoint{5.954108in}{2.763017in}}%
\pgfpathlineto{\pgfqpoint{5.961372in}{2.772562in}}%
\pgfpathlineto{\pgfqpoint{5.968642in}{2.782399in}}%
\pgfpathlineto{\pgfqpoint{5.975918in}{2.792538in}}%
\pgfpathlineto{\pgfqpoint{5.961734in}{2.793646in}}%
\pgfpathlineto{\pgfqpoint{5.947560in}{2.794818in}}%
\pgfpathlineto{\pgfqpoint{5.933396in}{2.796055in}}%
\pgfpathlineto{\pgfqpoint{5.919242in}{2.797357in}}%
\pgfpathlineto{\pgfqpoint{5.911938in}{2.786724in}}%
\pgfpathlineto{\pgfqpoint{5.904640in}{2.776396in}}%
\pgfpathlineto{\pgfqpoint{5.897348in}{2.766366in}}%
\pgfpathlineto{\pgfqpoint{5.890062in}{2.756626in}}%
\pgfpathclose%
\pgfusepath{fill}%
\end{pgfscope}%
\begin{pgfscope}%
\pgfpathrectangle{\pgfqpoint{1.150000in}{0.150000in}}{\pgfqpoint{5.700000in}{5.700000in}}%
\pgfusepath{clip}%
\pgfsetbuttcap%
\pgfsetroundjoin%
\definecolor{currentfill}{rgb}{0.277134,0.185228,0.489898}%
\pgfsetfillcolor{currentfill}%
\pgfsetfillopacity{0.700000}%
\pgfsetlinewidth{0.000000pt}%
\definecolor{currentstroke}{rgb}{0.000000,0.000000,0.000000}%
\pgfsetstrokecolor{currentstroke}%
\pgfsetdash{}{0pt}%
\pgfpathmoveto{\pgfqpoint{4.691420in}{2.395284in}}%
\pgfpathlineto{\pgfqpoint{4.705268in}{2.394670in}}%
\pgfpathlineto{\pgfqpoint{4.719125in}{2.394128in}}%
\pgfpathlineto{\pgfqpoint{4.732991in}{2.393658in}}%
\pgfpathlineto{\pgfqpoint{4.746865in}{2.393259in}}%
\pgfpathlineto{\pgfqpoint{4.754512in}{2.400933in}}%
\pgfpathlineto{\pgfqpoint{4.762153in}{2.408654in}}%
\pgfpathlineto{\pgfqpoint{4.769790in}{2.416426in}}%
\pgfpathlineto{\pgfqpoint{4.777422in}{2.424254in}}%
\pgfpathlineto{\pgfqpoint{4.763562in}{2.424898in}}%
\pgfpathlineto{\pgfqpoint{4.749712in}{2.425613in}}%
\pgfpathlineto{\pgfqpoint{4.735870in}{2.426399in}}%
\pgfpathlineto{\pgfqpoint{4.722036in}{2.427257in}}%
\pgfpathlineto{\pgfqpoint{4.714390in}{2.419177in}}%
\pgfpathlineto{\pgfqpoint{4.706738in}{2.411158in}}%
\pgfpathlineto{\pgfqpoint{4.699082in}{2.403195in}}%
\pgfpathlineto{\pgfqpoint{4.691420in}{2.395284in}}%
\pgfpathclose%
\pgfusepath{fill}%
\end{pgfscope}%
\begin{pgfscope}%
\pgfpathrectangle{\pgfqpoint{1.150000in}{0.150000in}}{\pgfqpoint{5.700000in}{5.700000in}}%
\pgfusepath{clip}%
\pgfsetbuttcap%
\pgfsetroundjoin%
\definecolor{currentfill}{rgb}{0.280267,0.073417,0.397163}%
\pgfsetfillcolor{currentfill}%
\pgfsetfillopacity{0.700000}%
\pgfsetlinewidth{0.000000pt}%
\definecolor{currentstroke}{rgb}{0.000000,0.000000,0.000000}%
\pgfsetstrokecolor{currentstroke}%
\pgfsetdash{}{0pt}%
\pgfpathmoveto{\pgfqpoint{2.874004in}{2.177561in}}%
\pgfpathlineto{\pgfqpoint{2.887491in}{2.169906in}}%
\pgfpathlineto{\pgfqpoint{2.900980in}{2.162358in}}%
\pgfpathlineto{\pgfqpoint{2.914470in}{2.154917in}}%
\pgfpathlineto{\pgfqpoint{2.927963in}{2.147581in}}%
\pgfpathlineto{\pgfqpoint{2.936283in}{2.155527in}}%
\pgfpathlineto{\pgfqpoint{2.944596in}{2.163531in}}%
\pgfpathlineto{\pgfqpoint{2.952901in}{2.171591in}}%
\pgfpathlineto{\pgfqpoint{2.961199in}{2.179708in}}%
\pgfpathlineto{\pgfqpoint{2.947723in}{2.186960in}}%
\pgfpathlineto{\pgfqpoint{2.934249in}{2.194317in}}%
\pgfpathlineto{\pgfqpoint{2.920776in}{2.201780in}}%
\pgfpathlineto{\pgfqpoint{2.907305in}{2.209351in}}%
\pgfpathlineto{\pgfqpoint{2.898991in}{2.201311in}}%
\pgfpathlineto{\pgfqpoint{2.890670in}{2.193332in}}%
\pgfpathlineto{\pgfqpoint{2.882341in}{2.185415in}}%
\pgfpathlineto{\pgfqpoint{2.874004in}{2.177561in}}%
\pgfpathclose%
\pgfusepath{fill}%
\end{pgfscope}%
\begin{pgfscope}%
\pgfpathrectangle{\pgfqpoint{1.150000in}{0.150000in}}{\pgfqpoint{5.700000in}{5.700000in}}%
\pgfusepath{clip}%
\pgfsetbuttcap%
\pgfsetroundjoin%
\definecolor{currentfill}{rgb}{0.248629,0.278775,0.534556}%
\pgfsetfillcolor{currentfill}%
\pgfsetfillopacity{0.700000}%
\pgfsetlinewidth{0.000000pt}%
\definecolor{currentstroke}{rgb}{0.000000,0.000000,0.000000}%
\pgfsetstrokecolor{currentstroke}%
\pgfsetdash{}{0pt}%
\pgfpathmoveto{\pgfqpoint{5.404505in}{2.597860in}}%
\pgfpathlineto{\pgfqpoint{5.418567in}{2.597598in}}%
\pgfpathlineto{\pgfqpoint{5.432639in}{2.597404in}}%
\pgfpathlineto{\pgfqpoint{5.446722in}{2.597277in}}%
\pgfpathlineto{\pgfqpoint{5.460814in}{2.597217in}}%
\pgfpathlineto{\pgfqpoint{5.468193in}{2.604763in}}%
\pgfpathlineto{\pgfqpoint{5.475571in}{2.612468in}}%
\pgfpathlineto{\pgfqpoint{5.482949in}{2.620339in}}%
\pgfpathlineto{\pgfqpoint{5.490325in}{2.628384in}}%
\pgfpathlineto{\pgfqpoint{5.476256in}{2.628832in}}%
\pgfpathlineto{\pgfqpoint{5.462196in}{2.629346in}}%
\pgfpathlineto{\pgfqpoint{5.448146in}{2.629928in}}%
\pgfpathlineto{\pgfqpoint{5.434106in}{2.630576in}}%
\pgfpathlineto{\pgfqpoint{5.426707in}{2.622137in}}%
\pgfpathlineto{\pgfqpoint{5.419307in}{2.613876in}}%
\pgfpathlineto{\pgfqpoint{5.411907in}{2.605785in}}%
\pgfpathlineto{\pgfqpoint{5.404505in}{2.597860in}}%
\pgfpathclose%
\pgfusepath{fill}%
\end{pgfscope}%
\begin{pgfscope}%
\pgfpathrectangle{\pgfqpoint{1.150000in}{0.150000in}}{\pgfqpoint{5.700000in}{5.700000in}}%
\pgfusepath{clip}%
\pgfsetbuttcap%
\pgfsetroundjoin%
\definecolor{currentfill}{rgb}{0.276022,0.044167,0.370164}%
\pgfsetfillcolor{currentfill}%
\pgfsetfillopacity{0.700000}%
\pgfsetlinewidth{0.000000pt}%
\definecolor{currentstroke}{rgb}{0.000000,0.000000,0.000000}%
\pgfsetstrokecolor{currentstroke}%
\pgfsetdash{}{0pt}%
\pgfpathmoveto{\pgfqpoint{3.296863in}{2.124876in}}%
\pgfpathlineto{\pgfqpoint{3.310377in}{2.119802in}}%
\pgfpathlineto{\pgfqpoint{3.323895in}{2.114821in}}%
\pgfpathlineto{\pgfqpoint{3.337418in}{2.109931in}}%
\pgfpathlineto{\pgfqpoint{3.350945in}{2.105132in}}%
\pgfpathlineto{\pgfqpoint{3.359099in}{2.113914in}}%
\pgfpathlineto{\pgfqpoint{3.367247in}{2.122717in}}%
\pgfpathlineto{\pgfqpoint{3.375388in}{2.131542in}}%
\pgfpathlineto{\pgfqpoint{3.383524in}{2.140390in}}%
\pgfpathlineto{\pgfqpoint{3.370009in}{2.145166in}}%
\pgfpathlineto{\pgfqpoint{3.356499in}{2.150034in}}%
\pgfpathlineto{\pgfqpoint{3.342994in}{2.154993in}}%
\pgfpathlineto{\pgfqpoint{3.329492in}{2.160045in}}%
\pgfpathlineto{\pgfqpoint{3.321344in}{2.151212in}}%
\pgfpathlineto{\pgfqpoint{3.313190in}{2.142406in}}%
\pgfpathlineto{\pgfqpoint{3.305030in}{2.133628in}}%
\pgfpathlineto{\pgfqpoint{3.296863in}{2.124876in}}%
\pgfpathclose%
\pgfusepath{fill}%
\end{pgfscope}%
\begin{pgfscope}%
\pgfpathrectangle{\pgfqpoint{1.150000in}{0.150000in}}{\pgfqpoint{5.700000in}{5.700000in}}%
\pgfusepath{clip}%
\pgfsetbuttcap%
\pgfsetroundjoin%
\definecolor{currentfill}{rgb}{0.266580,0.228262,0.514349}%
\pgfsetfillcolor{currentfill}%
\pgfsetfillopacity{0.700000}%
\pgfsetlinewidth{0.000000pt}%
\definecolor{currentstroke}{rgb}{0.000000,0.000000,0.000000}%
\pgfsetstrokecolor{currentstroke}%
\pgfsetdash{}{0pt}%
\pgfpathmoveto{\pgfqpoint{5.004954in}{2.480652in}}%
\pgfpathlineto{\pgfqpoint{5.018900in}{2.480371in}}%
\pgfpathlineto{\pgfqpoint{5.032856in}{2.480161in}}%
\pgfpathlineto{\pgfqpoint{5.046821in}{2.480019in}}%
\pgfpathlineto{\pgfqpoint{5.060796in}{2.479947in}}%
\pgfpathlineto{\pgfqpoint{5.068321in}{2.487306in}}%
\pgfpathlineto{\pgfqpoint{5.075843in}{2.494750in}}%
\pgfpathlineto{\pgfqpoint{5.083361in}{2.502287in}}%
\pgfpathlineto{\pgfqpoint{5.090875in}{2.509920in}}%
\pgfpathlineto{\pgfqpoint{5.076918in}{2.510299in}}%
\pgfpathlineto{\pgfqpoint{5.062971in}{2.510746in}}%
\pgfpathlineto{\pgfqpoint{5.049033in}{2.511263in}}%
\pgfpathlineto{\pgfqpoint{5.035104in}{2.511849in}}%
\pgfpathlineto{\pgfqpoint{5.027572in}{2.503902in}}%
\pgfpathlineto{\pgfqpoint{5.020036in}{2.496057in}}%
\pgfpathlineto{\pgfqpoint{5.012497in}{2.488309in}}%
\pgfpathlineto{\pgfqpoint{5.004954in}{2.480652in}}%
\pgfpathclose%
\pgfusepath{fill}%
\end{pgfscope}%
\begin{pgfscope}%
\pgfpathrectangle{\pgfqpoint{1.150000in}{0.150000in}}{\pgfqpoint{5.700000in}{5.700000in}}%
\pgfusepath{clip}%
\pgfsetbuttcap%
\pgfsetroundjoin%
\definecolor{currentfill}{rgb}{0.282327,0.094955,0.417331}%
\pgfsetfillcolor{currentfill}%
\pgfsetfillopacity{0.700000}%
\pgfsetlinewidth{0.000000pt}%
\definecolor{currentstroke}{rgb}{0.000000,0.000000,0.000000}%
\pgfsetstrokecolor{currentstroke}%
\pgfsetdash{}{0pt}%
\pgfpathmoveto{\pgfqpoint{3.978448in}{2.209743in}}%
\pgfpathlineto{\pgfqpoint{3.992096in}{2.207562in}}%
\pgfpathlineto{\pgfqpoint{4.005751in}{2.205461in}}%
\pgfpathlineto{\pgfqpoint{4.019413in}{2.203438in}}%
\pgfpathlineto{\pgfqpoint{4.033082in}{2.201494in}}%
\pgfpathlineto{\pgfqpoint{4.040996in}{2.210170in}}%
\pgfpathlineto{\pgfqpoint{4.048903in}{2.218848in}}%
\pgfpathlineto{\pgfqpoint{4.056806in}{2.227529in}}%
\pgfpathlineto{\pgfqpoint{4.064702in}{2.236217in}}%
\pgfpathlineto{\pgfqpoint{4.051044in}{2.238262in}}%
\pgfpathlineto{\pgfqpoint{4.037393in}{2.240385in}}%
\pgfpathlineto{\pgfqpoint{4.023749in}{2.242587in}}%
\pgfpathlineto{\pgfqpoint{4.010112in}{2.244868in}}%
\pgfpathlineto{\pgfqpoint{4.002204in}{2.236072in}}%
\pgfpathlineto{\pgfqpoint{3.994291in}{2.227287in}}%
\pgfpathlineto{\pgfqpoint{3.986372in}{2.218512in}}%
\pgfpathlineto{\pgfqpoint{3.978448in}{2.209743in}}%
\pgfpathclose%
\pgfusepath{fill}%
\end{pgfscope}%
\begin{pgfscope}%
\pgfpathrectangle{\pgfqpoint{1.150000in}{0.150000in}}{\pgfqpoint{5.700000in}{5.700000in}}%
\pgfusepath{clip}%
\pgfsetbuttcap%
\pgfsetroundjoin%
\definecolor{currentfill}{rgb}{0.282884,0.135920,0.453427}%
\pgfsetfillcolor{currentfill}%
\pgfsetfillopacity{0.700000}%
\pgfsetlinewidth{0.000000pt}%
\definecolor{currentstroke}{rgb}{0.000000,0.000000,0.000000}%
\pgfsetstrokecolor{currentstroke}%
\pgfsetdash{}{0pt}%
\pgfpathmoveto{\pgfqpoint{4.291875in}{2.284572in}}%
\pgfpathlineto{\pgfqpoint{4.305608in}{2.283263in}}%
\pgfpathlineto{\pgfqpoint{4.319348in}{2.282029in}}%
\pgfpathlineto{\pgfqpoint{4.333097in}{2.280870in}}%
\pgfpathlineto{\pgfqpoint{4.346854in}{2.279785in}}%
\pgfpathlineto{\pgfqpoint{4.354654in}{2.288023in}}%
\pgfpathlineto{\pgfqpoint{4.362448in}{2.296273in}}%
\pgfpathlineto{\pgfqpoint{4.370237in}{2.304539in}}%
\pgfpathlineto{\pgfqpoint{4.378021in}{2.312823in}}%
\pgfpathlineto{\pgfqpoint{4.364276in}{2.314070in}}%
\pgfpathlineto{\pgfqpoint{4.350539in}{2.315391in}}%
\pgfpathlineto{\pgfqpoint{4.336811in}{2.316788in}}%
\pgfpathlineto{\pgfqpoint{4.323090in}{2.318259in}}%
\pgfpathlineto{\pgfqpoint{4.315294in}{2.309805in}}%
\pgfpathlineto{\pgfqpoint{4.307493in}{2.301375in}}%
\pgfpathlineto{\pgfqpoint{4.299687in}{2.292965in}}%
\pgfpathlineto{\pgfqpoint{4.291875in}{2.284572in}}%
\pgfpathclose%
\pgfusepath{fill}%
\end{pgfscope}%
\begin{pgfscope}%
\pgfpathrectangle{\pgfqpoint{1.150000in}{0.150000in}}{\pgfqpoint{5.700000in}{5.700000in}}%
\pgfusepath{clip}%
\pgfsetbuttcap%
\pgfsetroundjoin%
\definecolor{currentfill}{rgb}{0.223925,0.334994,0.548053}%
\pgfsetfillcolor{currentfill}%
\pgfsetfillopacity{0.700000}%
\pgfsetlinewidth{0.000000pt}%
\definecolor{currentstroke}{rgb}{0.000000,0.000000,0.000000}%
\pgfsetstrokecolor{currentstroke}%
\pgfsetdash{}{0pt}%
\pgfpathmoveto{\pgfqpoint{5.804228in}{2.722420in}}%
\pgfpathlineto{\pgfqpoint{5.818396in}{2.721813in}}%
\pgfpathlineto{\pgfqpoint{5.832574in}{2.721271in}}%
\pgfpathlineto{\pgfqpoint{5.846762in}{2.720794in}}%
\pgfpathlineto{\pgfqpoint{5.860961in}{2.720383in}}%
\pgfpathlineto{\pgfqpoint{5.868230in}{2.729052in}}%
\pgfpathlineto{\pgfqpoint{5.875503in}{2.737977in}}%
\pgfpathlineto{\pgfqpoint{5.882780in}{2.747165in}}%
\pgfpathlineto{\pgfqpoint{5.890062in}{2.756626in}}%
\pgfpathlineto{\pgfqpoint{5.875890in}{2.757505in}}%
\pgfpathlineto{\pgfqpoint{5.861729in}{2.758450in}}%
\pgfpathlineto{\pgfqpoint{5.847578in}{2.759460in}}%
\pgfpathlineto{\pgfqpoint{5.833437in}{2.760535in}}%
\pgfpathlineto{\pgfqpoint{5.826129in}{2.750599in}}%
\pgfpathlineto{\pgfqpoint{5.818825in}{2.740941in}}%
\pgfpathlineto{\pgfqpoint{5.811525in}{2.731551in}}%
\pgfpathlineto{\pgfqpoint{5.804228in}{2.722420in}}%
\pgfpathclose%
\pgfusepath{fill}%
\end{pgfscope}%
\begin{pgfscope}%
\pgfpathrectangle{\pgfqpoint{1.150000in}{0.150000in}}{\pgfqpoint{5.700000in}{5.700000in}}%
\pgfusepath{clip}%
\pgfsetbuttcap%
\pgfsetroundjoin%
\definecolor{currentfill}{rgb}{0.278826,0.175490,0.483397}%
\pgfsetfillcolor{currentfill}%
\pgfsetfillopacity{0.700000}%
\pgfsetlinewidth{0.000000pt}%
\definecolor{currentstroke}{rgb}{0.000000,0.000000,0.000000}%
\pgfsetstrokecolor{currentstroke}%
\pgfsetdash{}{0pt}%
\pgfpathmoveto{\pgfqpoint{4.605363in}{2.366352in}}%
\pgfpathlineto{\pgfqpoint{4.619190in}{2.365675in}}%
\pgfpathlineto{\pgfqpoint{4.633026in}{2.365069in}}%
\pgfpathlineto{\pgfqpoint{4.646870in}{2.364536in}}%
\pgfpathlineto{\pgfqpoint{4.660724in}{2.364075in}}%
\pgfpathlineto{\pgfqpoint{4.668406in}{2.371820in}}%
\pgfpathlineto{\pgfqpoint{4.676082in}{2.379601in}}%
\pgfpathlineto{\pgfqpoint{4.683754in}{2.387421in}}%
\pgfpathlineto{\pgfqpoint{4.691420in}{2.395284in}}%
\pgfpathlineto{\pgfqpoint{4.677581in}{2.395969in}}%
\pgfpathlineto{\pgfqpoint{4.663751in}{2.396727in}}%
\pgfpathlineto{\pgfqpoint{4.649929in}{2.397556in}}%
\pgfpathlineto{\pgfqpoint{4.636116in}{2.398457in}}%
\pgfpathlineto{\pgfqpoint{4.628436in}{2.390363in}}%
\pgfpathlineto{\pgfqpoint{4.620750in}{2.382316in}}%
\pgfpathlineto{\pgfqpoint{4.613059in}{2.374314in}}%
\pgfpathlineto{\pgfqpoint{4.605363in}{2.366352in}}%
\pgfpathclose%
\pgfusepath{fill}%
\end{pgfscope}%
\begin{pgfscope}%
\pgfpathrectangle{\pgfqpoint{1.150000in}{0.150000in}}{\pgfqpoint{5.700000in}{5.700000in}}%
\pgfusepath{clip}%
\pgfsetbuttcap%
\pgfsetroundjoin%
\definecolor{currentfill}{rgb}{0.281924,0.089666,0.412415}%
\pgfsetfillcolor{currentfill}%
\pgfsetfillopacity{0.700000}%
\pgfsetlinewidth{0.000000pt}%
\definecolor{currentstroke}{rgb}{0.000000,0.000000,0.000000}%
\pgfsetstrokecolor{currentstroke}%
\pgfsetdash{}{0pt}%
\pgfpathmoveto{\pgfqpoint{2.732591in}{2.212835in}}%
\pgfpathlineto{\pgfqpoint{2.746086in}{2.204188in}}%
\pgfpathlineto{\pgfqpoint{2.759581in}{2.195655in}}%
\pgfpathlineto{\pgfqpoint{2.773078in}{2.187235in}}%
\pgfpathlineto{\pgfqpoint{2.786576in}{2.178927in}}%
\pgfpathlineto{\pgfqpoint{2.794962in}{2.186405in}}%
\pgfpathlineto{\pgfqpoint{2.803340in}{2.193955in}}%
\pgfpathlineto{\pgfqpoint{2.811710in}{2.201577in}}%
\pgfpathlineto{\pgfqpoint{2.820071in}{2.209270in}}%
\pgfpathlineto{\pgfqpoint{2.806591in}{2.217472in}}%
\pgfpathlineto{\pgfqpoint{2.793113in}{2.225787in}}%
\pgfpathlineto{\pgfqpoint{2.779635in}{2.234215in}}%
\pgfpathlineto{\pgfqpoint{2.766158in}{2.242756in}}%
\pgfpathlineto{\pgfqpoint{2.757779in}{2.235162in}}%
\pgfpathlineto{\pgfqpoint{2.749391in}{2.227643in}}%
\pgfpathlineto{\pgfqpoint{2.740995in}{2.220200in}}%
\pgfpathlineto{\pgfqpoint{2.732591in}{2.212835in}}%
\pgfpathclose%
\pgfusepath{fill}%
\end{pgfscope}%
\begin{pgfscope}%
\pgfpathrectangle{\pgfqpoint{1.150000in}{0.150000in}}{\pgfqpoint{5.700000in}{5.700000in}}%
\pgfusepath{clip}%
\pgfsetbuttcap%
\pgfsetroundjoin%
\definecolor{currentfill}{rgb}{0.283072,0.130895,0.449241}%
\pgfsetfillcolor{currentfill}%
\pgfsetfillopacity{0.700000}%
\pgfsetlinewidth{0.000000pt}%
\definecolor{currentstroke}{rgb}{0.000000,0.000000,0.000000}%
\pgfsetstrokecolor{currentstroke}%
\pgfsetdash{}{0pt}%
\pgfpathmoveto{\pgfqpoint{2.536725in}{2.298666in}}%
\pgfpathlineto{\pgfqpoint{2.550241in}{2.288455in}}%
\pgfpathlineto{\pgfqpoint{2.563756in}{2.278370in}}%
\pgfpathlineto{\pgfqpoint{2.577270in}{2.268408in}}%
\pgfpathlineto{\pgfqpoint{2.590783in}{2.258570in}}%
\pgfpathlineto{\pgfqpoint{2.599263in}{2.265341in}}%
\pgfpathlineto{\pgfqpoint{2.607732in}{2.272206in}}%
\pgfpathlineto{\pgfqpoint{2.616193in}{2.279163in}}%
\pgfpathlineto{\pgfqpoint{2.624645in}{2.286211in}}%
\pgfpathlineto{\pgfqpoint{2.611151in}{2.295921in}}%
\pgfpathlineto{\pgfqpoint{2.597658in}{2.305755in}}%
\pgfpathlineto{\pgfqpoint{2.584163in}{2.315713in}}%
\pgfpathlineto{\pgfqpoint{2.570668in}{2.325797in}}%
\pgfpathlineto{\pgfqpoint{2.562197in}{2.318868in}}%
\pgfpathlineto{\pgfqpoint{2.553716in}{2.312036in}}%
\pgfpathlineto{\pgfqpoint{2.545226in}{2.305301in}}%
\pgfpathlineto{\pgfqpoint{2.536725in}{2.298666in}}%
\pgfpathclose%
\pgfusepath{fill}%
\end{pgfscope}%
\begin{pgfscope}%
\pgfpathrectangle{\pgfqpoint{1.150000in}{0.150000in}}{\pgfqpoint{5.700000in}{5.700000in}}%
\pgfusepath{clip}%
\pgfsetbuttcap%
\pgfsetroundjoin%
\definecolor{currentfill}{rgb}{0.278791,0.062145,0.386592}%
\pgfsetfillcolor{currentfill}%
\pgfsetfillopacity{0.700000}%
\pgfsetlinewidth{0.000000pt}%
\definecolor{currentstroke}{rgb}{0.000000,0.000000,0.000000}%
\pgfsetstrokecolor{currentstroke}%
\pgfsetdash{}{0pt}%
\pgfpathmoveto{\pgfqpoint{3.664914in}{2.147407in}}%
\pgfpathlineto{\pgfqpoint{3.678493in}{2.144101in}}%
\pgfpathlineto{\pgfqpoint{3.692078in}{2.140879in}}%
\pgfpathlineto{\pgfqpoint{3.705668in}{2.137741in}}%
\pgfpathlineto{\pgfqpoint{3.719264in}{2.134686in}}%
\pgfpathlineto{\pgfqpoint{3.727289in}{2.143600in}}%
\pgfpathlineto{\pgfqpoint{3.735309in}{2.152518in}}%
\pgfpathlineto{\pgfqpoint{3.743322in}{2.161441in}}%
\pgfpathlineto{\pgfqpoint{3.751330in}{2.170370in}}%
\pgfpathlineto{\pgfqpoint{3.737745in}{2.173464in}}%
\pgfpathlineto{\pgfqpoint{3.724166in}{2.176642in}}%
\pgfpathlineto{\pgfqpoint{3.710592in}{2.179903in}}%
\pgfpathlineto{\pgfqpoint{3.697025in}{2.183248in}}%
\pgfpathlineto{\pgfqpoint{3.689006in}{2.174273in}}%
\pgfpathlineto{\pgfqpoint{3.680981in}{2.165308in}}%
\pgfpathlineto{\pgfqpoint{3.672951in}{2.156353in}}%
\pgfpathlineto{\pgfqpoint{3.664914in}{2.147407in}}%
\pgfpathclose%
\pgfusepath{fill}%
\end{pgfscope}%
\begin{pgfscope}%
\pgfpathrectangle{\pgfqpoint{1.150000in}{0.150000in}}{\pgfqpoint{5.700000in}{5.700000in}}%
\pgfusepath{clip}%
\pgfsetbuttcap%
\pgfsetroundjoin%
\definecolor{currentfill}{rgb}{0.252194,0.269783,0.531579}%
\pgfsetfillcolor{currentfill}%
\pgfsetfillopacity{0.700000}%
\pgfsetlinewidth{0.000000pt}%
\definecolor{currentstroke}{rgb}{0.000000,0.000000,0.000000}%
\pgfsetstrokecolor{currentstroke}%
\pgfsetdash{}{0pt}%
\pgfpathmoveto{\pgfqpoint{5.318648in}{2.567914in}}%
\pgfpathlineto{\pgfqpoint{5.332692in}{2.567750in}}%
\pgfpathlineto{\pgfqpoint{5.346745in}{2.567654in}}%
\pgfpathlineto{\pgfqpoint{5.360809in}{2.567626in}}%
\pgfpathlineto{\pgfqpoint{5.374883in}{2.567665in}}%
\pgfpathlineto{\pgfqpoint{5.382291in}{2.575000in}}%
\pgfpathlineto{\pgfqpoint{5.389698in}{2.582474in}}%
\pgfpathlineto{\pgfqpoint{5.397102in}{2.590091in}}%
\pgfpathlineto{\pgfqpoint{5.404505in}{2.597860in}}%
\pgfpathlineto{\pgfqpoint{5.390453in}{2.598188in}}%
\pgfpathlineto{\pgfqpoint{5.376411in}{2.598584in}}%
\pgfpathlineto{\pgfqpoint{5.362378in}{2.599047in}}%
\pgfpathlineto{\pgfqpoint{5.348355in}{2.599578in}}%
\pgfpathlineto{\pgfqpoint{5.340931in}{2.591435in}}%
\pgfpathlineto{\pgfqpoint{5.333505in}{2.583448in}}%
\pgfpathlineto{\pgfqpoint{5.326078in}{2.575610in}}%
\pgfpathlineto{\pgfqpoint{5.318648in}{2.567914in}}%
\pgfpathclose%
\pgfusepath{fill}%
\end{pgfscope}%
\begin{pgfscope}%
\pgfpathrectangle{\pgfqpoint{1.150000in}{0.150000in}}{\pgfqpoint{5.700000in}{5.700000in}}%
\pgfusepath{clip}%
\pgfsetbuttcap%
\pgfsetroundjoin%
\definecolor{currentfill}{rgb}{0.229739,0.322361,0.545706}%
\pgfsetfillcolor{currentfill}%
\pgfsetfillopacity{0.700000}%
\pgfsetlinewidth{0.000000pt}%
\definecolor{currentstroke}{rgb}{0.000000,0.000000,0.000000}%
\pgfsetstrokecolor{currentstroke}%
\pgfsetdash{}{0pt}%
\pgfpathmoveto{\pgfqpoint{5.718401in}{2.689632in}}%
\pgfpathlineto{\pgfqpoint{5.732553in}{2.689210in}}%
\pgfpathlineto{\pgfqpoint{5.746716in}{2.688854in}}%
\pgfpathlineto{\pgfqpoint{5.760890in}{2.688564in}}%
\pgfpathlineto{\pgfqpoint{5.775074in}{2.688340in}}%
\pgfpathlineto{\pgfqpoint{5.782359in}{2.696510in}}%
\pgfpathlineto{\pgfqpoint{5.789646in}{2.704908in}}%
\pgfpathlineto{\pgfqpoint{5.796936in}{2.713542in}}%
\pgfpathlineto{\pgfqpoint{5.804228in}{2.722420in}}%
\pgfpathlineto{\pgfqpoint{5.790071in}{2.723093in}}%
\pgfpathlineto{\pgfqpoint{5.775924in}{2.723832in}}%
\pgfpathlineto{\pgfqpoint{5.761787in}{2.724635in}}%
\pgfpathlineto{\pgfqpoint{5.747660in}{2.725504in}}%
\pgfpathlineto{\pgfqpoint{5.740341in}{2.716171in}}%
\pgfpathlineto{\pgfqpoint{5.733025in}{2.707087in}}%
\pgfpathlineto{\pgfqpoint{5.725711in}{2.698243in}}%
\pgfpathlineto{\pgfqpoint{5.718401in}{2.689632in}}%
\pgfpathclose%
\pgfusepath{fill}%
\end{pgfscope}%
\begin{pgfscope}%
\pgfpathrectangle{\pgfqpoint{1.150000in}{0.150000in}}{\pgfqpoint{5.700000in}{5.700000in}}%
\pgfusepath{clip}%
\pgfsetbuttcap%
\pgfsetroundjoin%
\definecolor{currentfill}{rgb}{0.277018,0.050344,0.375715}%
\pgfsetfillcolor{currentfill}%
\pgfsetfillopacity{0.700000}%
\pgfsetlinewidth{0.000000pt}%
\definecolor{currentstroke}{rgb}{0.000000,0.000000,0.000000}%
\pgfsetstrokecolor{currentstroke}%
\pgfsetdash{}{0pt}%
\pgfpathmoveto{\pgfqpoint{3.437628in}{2.122190in}}%
\pgfpathlineto{\pgfqpoint{3.451165in}{2.117864in}}%
\pgfpathlineto{\pgfqpoint{3.464708in}{2.113627in}}%
\pgfpathlineto{\pgfqpoint{3.478255in}{2.109477in}}%
\pgfpathlineto{\pgfqpoint{3.491808in}{2.105416in}}%
\pgfpathlineto{\pgfqpoint{3.499913in}{2.114305in}}%
\pgfpathlineto{\pgfqpoint{3.508013in}{2.123208in}}%
\pgfpathlineto{\pgfqpoint{3.516107in}{2.132123in}}%
\pgfpathlineto{\pgfqpoint{3.524194in}{2.141052in}}%
\pgfpathlineto{\pgfqpoint{3.510654in}{2.145112in}}%
\pgfpathlineto{\pgfqpoint{3.497118in}{2.149260in}}%
\pgfpathlineto{\pgfqpoint{3.483587in}{2.153495in}}%
\pgfpathlineto{\pgfqpoint{3.470061in}{2.157820in}}%
\pgfpathlineto{\pgfqpoint{3.461962in}{2.148884in}}%
\pgfpathlineto{\pgfqpoint{3.453856in}{2.139968in}}%
\pgfpathlineto{\pgfqpoint{3.445745in}{2.131070in}}%
\pgfpathlineto{\pgfqpoint{3.437628in}{2.122190in}}%
\pgfpathclose%
\pgfusepath{fill}%
\end{pgfscope}%
\begin{pgfscope}%
\pgfpathrectangle{\pgfqpoint{1.150000in}{0.150000in}}{\pgfqpoint{5.700000in}{5.700000in}}%
\pgfusepath{clip}%
\pgfsetbuttcap%
\pgfsetroundjoin%
\definecolor{currentfill}{rgb}{0.269308,0.218818,0.509577}%
\pgfsetfillcolor{currentfill}%
\pgfsetfillopacity{0.700000}%
\pgfsetlinewidth{0.000000pt}%
\definecolor{currentstroke}{rgb}{0.000000,0.000000,0.000000}%
\pgfsetstrokecolor{currentstroke}%
\pgfsetdash{}{0pt}%
\pgfpathmoveto{\pgfqpoint{4.918979in}{2.451501in}}%
\pgfpathlineto{\pgfqpoint{4.932905in}{2.451228in}}%
\pgfpathlineto{\pgfqpoint{4.946840in}{2.451025in}}%
\pgfpathlineto{\pgfqpoint{4.960785in}{2.450891in}}%
\pgfpathlineto{\pgfqpoint{4.974739in}{2.450827in}}%
\pgfpathlineto{\pgfqpoint{4.982299in}{2.458173in}}%
\pgfpathlineto{\pgfqpoint{4.989855in}{2.465589in}}%
\pgfpathlineto{\pgfqpoint{4.997406in}{2.473080in}}%
\pgfpathlineto{\pgfqpoint{5.004954in}{2.480652in}}%
\pgfpathlineto{\pgfqpoint{4.991017in}{2.481001in}}%
\pgfpathlineto{\pgfqpoint{4.977089in}{2.481421in}}%
\pgfpathlineto{\pgfqpoint{4.963171in}{2.481910in}}%
\pgfpathlineto{\pgfqpoint{4.949262in}{2.482469in}}%
\pgfpathlineto{\pgfqpoint{4.941698in}{2.474604in}}%
\pgfpathlineto{\pgfqpoint{4.934129in}{2.466825in}}%
\pgfpathlineto{\pgfqpoint{4.926556in}{2.459126in}}%
\pgfpathlineto{\pgfqpoint{4.918979in}{2.451501in}}%
\pgfpathclose%
\pgfusepath{fill}%
\end{pgfscope}%
\begin{pgfscope}%
\pgfpathrectangle{\pgfqpoint{1.150000in}{0.150000in}}{\pgfqpoint{5.700000in}{5.700000in}}%
\pgfusepath{clip}%
\pgfsetbuttcap%
\pgfsetroundjoin%
\definecolor{currentfill}{rgb}{0.283229,0.120777,0.440584}%
\pgfsetfillcolor{currentfill}%
\pgfsetfillopacity{0.700000}%
\pgfsetlinewidth{0.000000pt}%
\definecolor{currentstroke}{rgb}{0.000000,0.000000,0.000000}%
\pgfsetstrokecolor{currentstroke}%
\pgfsetdash{}{0pt}%
\pgfpathmoveto{\pgfqpoint{4.205671in}{2.256544in}}%
\pgfpathlineto{\pgfqpoint{4.219384in}{2.255073in}}%
\pgfpathlineto{\pgfqpoint{4.233105in}{2.253679in}}%
\pgfpathlineto{\pgfqpoint{4.246834in}{2.252360in}}%
\pgfpathlineto{\pgfqpoint{4.260571in}{2.251117in}}%
\pgfpathlineto{\pgfqpoint{4.268405in}{2.259469in}}%
\pgfpathlineto{\pgfqpoint{4.276234in}{2.267828in}}%
\pgfpathlineto{\pgfqpoint{4.284057in}{2.276194in}}%
\pgfpathlineto{\pgfqpoint{4.291875in}{2.284572in}}%
\pgfpathlineto{\pgfqpoint{4.278150in}{2.285957in}}%
\pgfpathlineto{\pgfqpoint{4.264433in}{2.287418in}}%
\pgfpathlineto{\pgfqpoint{4.250723in}{2.288954in}}%
\pgfpathlineto{\pgfqpoint{4.237021in}{2.290567in}}%
\pgfpathlineto{\pgfqpoint{4.229192in}{2.282039in}}%
\pgfpathlineto{\pgfqpoint{4.221357in}{2.273528in}}%
\pgfpathlineto{\pgfqpoint{4.213517in}{2.265031in}}%
\pgfpathlineto{\pgfqpoint{4.205671in}{2.256544in}}%
\pgfpathclose%
\pgfusepath{fill}%
\end{pgfscope}%
\begin{pgfscope}%
\pgfpathrectangle{\pgfqpoint{1.150000in}{0.150000in}}{\pgfqpoint{5.700000in}{5.700000in}}%
\pgfusepath{clip}%
\pgfsetbuttcap%
\pgfsetroundjoin%
\definecolor{currentfill}{rgb}{0.281446,0.084320,0.407414}%
\pgfsetfillcolor{currentfill}%
\pgfsetfillopacity{0.700000}%
\pgfsetlinewidth{0.000000pt}%
\definecolor{currentstroke}{rgb}{0.000000,0.000000,0.000000}%
\pgfsetstrokecolor{currentstroke}%
\pgfsetdash{}{0pt}%
\pgfpathmoveto{\pgfqpoint{3.892126in}{2.183894in}}%
\pgfpathlineto{\pgfqpoint{3.905758in}{2.181475in}}%
\pgfpathlineto{\pgfqpoint{3.919396in}{2.179136in}}%
\pgfpathlineto{\pgfqpoint{3.933042in}{2.176877in}}%
\pgfpathlineto{\pgfqpoint{3.946694in}{2.174698in}}%
\pgfpathlineto{\pgfqpoint{3.954641in}{2.183458in}}%
\pgfpathlineto{\pgfqpoint{3.962582in}{2.192218in}}%
\pgfpathlineto{\pgfqpoint{3.970518in}{2.200979in}}%
\pgfpathlineto{\pgfqpoint{3.978448in}{2.209743in}}%
\pgfpathlineto{\pgfqpoint{3.964807in}{2.212002in}}%
\pgfpathlineto{\pgfqpoint{3.951172in}{2.214342in}}%
\pgfpathlineto{\pgfqpoint{3.937545in}{2.216761in}}%
\pgfpathlineto{\pgfqpoint{3.923924in}{2.219260in}}%
\pgfpathlineto{\pgfqpoint{3.915983in}{2.210408in}}%
\pgfpathlineto{\pgfqpoint{3.908036in}{2.201565in}}%
\pgfpathlineto{\pgfqpoint{3.900084in}{2.192727in}}%
\pgfpathlineto{\pgfqpoint{3.892126in}{2.183894in}}%
\pgfpathclose%
\pgfusepath{fill}%
\end{pgfscope}%
\begin{pgfscope}%
\pgfpathrectangle{\pgfqpoint{1.150000in}{0.150000in}}{\pgfqpoint{5.700000in}{5.700000in}}%
\pgfusepath{clip}%
\pgfsetbuttcap%
\pgfsetroundjoin%
\definecolor{currentfill}{rgb}{0.277018,0.050344,0.375715}%
\pgfsetfillcolor{currentfill}%
\pgfsetfillopacity{0.700000}%
\pgfsetlinewidth{0.000000pt}%
\definecolor{currentstroke}{rgb}{0.000000,0.000000,0.000000}%
\pgfsetstrokecolor{currentstroke}%
\pgfsetdash{}{0pt}%
\pgfpathmoveto{\pgfqpoint{3.069096in}{2.125419in}}%
\pgfpathlineto{\pgfqpoint{3.082595in}{2.119089in}}%
\pgfpathlineto{\pgfqpoint{3.096097in}{2.112857in}}%
\pgfpathlineto{\pgfqpoint{3.109602in}{2.106724in}}%
\pgfpathlineto{\pgfqpoint{3.123110in}{2.100688in}}%
\pgfpathlineto{\pgfqpoint{3.131356in}{2.109074in}}%
\pgfpathlineto{\pgfqpoint{3.139595in}{2.117500in}}%
\pgfpathlineto{\pgfqpoint{3.147828in}{2.125964in}}%
\pgfpathlineto{\pgfqpoint{3.156053in}{2.134467in}}%
\pgfpathlineto{\pgfqpoint{3.142559in}{2.140439in}}%
\pgfpathlineto{\pgfqpoint{3.129069in}{2.146509in}}%
\pgfpathlineto{\pgfqpoint{3.115581in}{2.152678in}}%
\pgfpathlineto{\pgfqpoint{3.102096in}{2.158945in}}%
\pgfpathlineto{\pgfqpoint{3.093857in}{2.150498in}}%
\pgfpathlineto{\pgfqpoint{3.085610in}{2.142094in}}%
\pgfpathlineto{\pgfqpoint{3.077356in}{2.133734in}}%
\pgfpathlineto{\pgfqpoint{3.069096in}{2.125419in}}%
\pgfpathclose%
\pgfusepath{fill}%
\end{pgfscope}%
\begin{pgfscope}%
\pgfpathrectangle{\pgfqpoint{1.150000in}{0.150000in}}{\pgfqpoint{5.700000in}{5.700000in}}%
\pgfusepath{clip}%
\pgfsetbuttcap%
\pgfsetroundjoin%
\definecolor{currentfill}{rgb}{0.280255,0.165693,0.476498}%
\pgfsetfillcolor{currentfill}%
\pgfsetfillopacity{0.700000}%
\pgfsetlinewidth{0.000000pt}%
\definecolor{currentstroke}{rgb}{0.000000,0.000000,0.000000}%
\pgfsetstrokecolor{currentstroke}%
\pgfsetdash{}{0pt}%
\pgfpathmoveto{\pgfqpoint{4.519250in}{2.337447in}}%
\pgfpathlineto{\pgfqpoint{4.533056in}{2.336682in}}%
\pgfpathlineto{\pgfqpoint{4.546871in}{2.335991in}}%
\pgfpathlineto{\pgfqpoint{4.560694in}{2.335372in}}%
\pgfpathlineto{\pgfqpoint{4.574526in}{2.334825in}}%
\pgfpathlineto{\pgfqpoint{4.582243in}{2.342667in}}%
\pgfpathlineto{\pgfqpoint{4.589955in}{2.350532in}}%
\pgfpathlineto{\pgfqpoint{4.597662in}{2.358426in}}%
\pgfpathlineto{\pgfqpoint{4.605363in}{2.366352in}}%
\pgfpathlineto{\pgfqpoint{4.591545in}{2.367102in}}%
\pgfpathlineto{\pgfqpoint{4.577735in}{2.367925in}}%
\pgfpathlineto{\pgfqpoint{4.563933in}{2.368820in}}%
\pgfpathlineto{\pgfqpoint{4.550141in}{2.369789in}}%
\pgfpathlineto{\pgfqpoint{4.542426in}{2.361652in}}%
\pgfpathlineto{\pgfqpoint{4.534706in}{2.353551in}}%
\pgfpathlineto{\pgfqpoint{4.526980in}{2.345485in}}%
\pgfpathlineto{\pgfqpoint{4.519250in}{2.337447in}}%
\pgfpathclose%
\pgfusepath{fill}%
\end{pgfscope}%
\begin{pgfscope}%
\pgfpathrectangle{\pgfqpoint{1.150000in}{0.150000in}}{\pgfqpoint{5.700000in}{5.700000in}}%
\pgfusepath{clip}%
\pgfsetbuttcap%
\pgfsetroundjoin%
\definecolor{currentfill}{rgb}{0.278791,0.062145,0.386592}%
\pgfsetfillcolor{currentfill}%
\pgfsetfillopacity{0.700000}%
\pgfsetlinewidth{0.000000pt}%
\definecolor{currentstroke}{rgb}{0.000000,0.000000,0.000000}%
\pgfsetstrokecolor{currentstroke}%
\pgfsetdash{}{0pt}%
\pgfpathmoveto{\pgfqpoint{2.927963in}{2.147581in}}%
\pgfpathlineto{\pgfqpoint{2.941457in}{2.140351in}}%
\pgfpathlineto{\pgfqpoint{2.954954in}{2.133225in}}%
\pgfpathlineto{\pgfqpoint{2.968453in}{2.126202in}}%
\pgfpathlineto{\pgfqpoint{2.981954in}{2.119282in}}%
\pgfpathlineto{\pgfqpoint{2.990258in}{2.127320in}}%
\pgfpathlineto{\pgfqpoint{2.998555in}{2.135410in}}%
\pgfpathlineto{\pgfqpoint{3.006845in}{2.143552in}}%
\pgfpathlineto{\pgfqpoint{3.015128in}{2.151746in}}%
\pgfpathlineto{\pgfqpoint{3.001642in}{2.158581in}}%
\pgfpathlineto{\pgfqpoint{2.988159in}{2.165520in}}%
\pgfpathlineto{\pgfqpoint{2.974678in}{2.172562in}}%
\pgfpathlineto{\pgfqpoint{2.961199in}{2.179708in}}%
\pgfpathlineto{\pgfqpoint{2.952901in}{2.171591in}}%
\pgfpathlineto{\pgfqpoint{2.944596in}{2.163531in}}%
\pgfpathlineto{\pgfqpoint{2.936283in}{2.155527in}}%
\pgfpathlineto{\pgfqpoint{2.927963in}{2.147581in}}%
\pgfpathclose%
\pgfusepath{fill}%
\end{pgfscope}%
\begin{pgfscope}%
\pgfpathrectangle{\pgfqpoint{1.150000in}{0.150000in}}{\pgfqpoint{5.700000in}{5.700000in}}%
\pgfusepath{clip}%
\pgfsetbuttcap%
\pgfsetroundjoin%
\definecolor{currentfill}{rgb}{0.233603,0.313828,0.543914}%
\pgfsetfillcolor{currentfill}%
\pgfsetfillopacity{0.700000}%
\pgfsetlinewidth{0.000000pt}%
\definecolor{currentstroke}{rgb}{0.000000,0.000000,0.000000}%
\pgfsetstrokecolor{currentstroke}%
\pgfsetdash{}{0pt}%
\pgfpathmoveto{\pgfqpoint{5.632563in}{2.657993in}}%
\pgfpathlineto{\pgfqpoint{5.646700in}{2.657736in}}%
\pgfpathlineto{\pgfqpoint{5.660847in}{2.657545in}}%
\pgfpathlineto{\pgfqpoint{5.675005in}{2.657420in}}%
\pgfpathlineto{\pgfqpoint{5.689173in}{2.657362in}}%
\pgfpathlineto{\pgfqpoint{5.696478in}{2.665118in}}%
\pgfpathlineto{\pgfqpoint{5.703784in}{2.673077in}}%
\pgfpathlineto{\pgfqpoint{5.711092in}{2.681246in}}%
\pgfpathlineto{\pgfqpoint{5.718401in}{2.689632in}}%
\pgfpathlineto{\pgfqpoint{5.704258in}{2.690119in}}%
\pgfpathlineto{\pgfqpoint{5.690126in}{2.690672in}}%
\pgfpathlineto{\pgfqpoint{5.676003in}{2.691291in}}%
\pgfpathlineto{\pgfqpoint{5.661891in}{2.691976in}}%
\pgfpathlineto{\pgfqpoint{5.654557in}{2.683155in}}%
\pgfpathlineto{\pgfqpoint{5.647225in}{2.674556in}}%
\pgfpathlineto{\pgfqpoint{5.639893in}{2.666171in}}%
\pgfpathlineto{\pgfqpoint{5.632563in}{2.657993in}}%
\pgfpathclose%
\pgfusepath{fill}%
\end{pgfscope}%
\begin{pgfscope}%
\pgfpathrectangle{\pgfqpoint{1.150000in}{0.150000in}}{\pgfqpoint{5.700000in}{5.700000in}}%
\pgfusepath{clip}%
\pgfsetbuttcap%
\pgfsetroundjoin%
\definecolor{currentfill}{rgb}{0.255645,0.260703,0.528312}%
\pgfsetfillcolor{currentfill}%
\pgfsetfillopacity{0.700000}%
\pgfsetlinewidth{0.000000pt}%
\definecolor{currentstroke}{rgb}{0.000000,0.000000,0.000000}%
\pgfsetstrokecolor{currentstroke}%
\pgfsetdash{}{0pt}%
\pgfpathmoveto{\pgfqpoint{5.232747in}{2.538376in}}%
\pgfpathlineto{\pgfqpoint{5.246772in}{2.538288in}}%
\pgfpathlineto{\pgfqpoint{5.260806in}{2.538268in}}%
\pgfpathlineto{\pgfqpoint{5.274851in}{2.538316in}}%
\pgfpathlineto{\pgfqpoint{5.288905in}{2.538433in}}%
\pgfpathlineto{\pgfqpoint{5.296345in}{2.545620in}}%
\pgfpathlineto{\pgfqpoint{5.303782in}{2.552926in}}%
\pgfpathlineto{\pgfqpoint{5.311216in}{2.560355in}}%
\pgfpathlineto{\pgfqpoint{5.318648in}{2.567914in}}%
\pgfpathlineto{\pgfqpoint{5.304614in}{2.568145in}}%
\pgfpathlineto{\pgfqpoint{5.290590in}{2.568445in}}%
\pgfpathlineto{\pgfqpoint{5.276576in}{2.568812in}}%
\pgfpathlineto{\pgfqpoint{5.262571in}{2.569246in}}%
\pgfpathlineto{\pgfqpoint{5.255119in}{2.561333in}}%
\pgfpathlineto{\pgfqpoint{5.247664in}{2.553554in}}%
\pgfpathlineto{\pgfqpoint{5.240207in}{2.545904in}}%
\pgfpathlineto{\pgfqpoint{5.232747in}{2.538376in}}%
\pgfpathclose%
\pgfusepath{fill}%
\end{pgfscope}%
\begin{pgfscope}%
\pgfpathrectangle{\pgfqpoint{1.150000in}{0.150000in}}{\pgfqpoint{5.700000in}{5.700000in}}%
\pgfusepath{clip}%
\pgfsetbuttcap%
\pgfsetroundjoin%
\definecolor{currentfill}{rgb}{0.276022,0.044167,0.370164}%
\pgfsetfillcolor{currentfill}%
\pgfsetfillopacity{0.700000}%
\pgfsetlinewidth{0.000000pt}%
\definecolor{currentstroke}{rgb}{0.000000,0.000000,0.000000}%
\pgfsetstrokecolor{currentstroke}%
\pgfsetdash{}{0pt}%
\pgfpathmoveto{\pgfqpoint{3.210064in}{2.111545in}}%
\pgfpathlineto{\pgfqpoint{3.223575in}{2.106054in}}%
\pgfpathlineto{\pgfqpoint{3.237091in}{2.100657in}}%
\pgfpathlineto{\pgfqpoint{3.250610in}{2.095354in}}%
\pgfpathlineto{\pgfqpoint{3.264133in}{2.090144in}}%
\pgfpathlineto{\pgfqpoint{3.272325in}{2.098786in}}%
\pgfpathlineto{\pgfqpoint{3.280511in}{2.107455in}}%
\pgfpathlineto{\pgfqpoint{3.288690in}{2.116152in}}%
\pgfpathlineto{\pgfqpoint{3.296863in}{2.124876in}}%
\pgfpathlineto{\pgfqpoint{3.283353in}{2.130043in}}%
\pgfpathlineto{\pgfqpoint{3.269847in}{2.135304in}}%
\pgfpathlineto{\pgfqpoint{3.256345in}{2.140658in}}%
\pgfpathlineto{\pgfqpoint{3.242846in}{2.146107in}}%
\pgfpathlineto{\pgfqpoint{3.234660in}{2.137418in}}%
\pgfpathlineto{\pgfqpoint{3.226468in}{2.128761in}}%
\pgfpathlineto{\pgfqpoint{3.218269in}{2.120137in}}%
\pgfpathlineto{\pgfqpoint{3.210064in}{2.111545in}}%
\pgfpathclose%
\pgfusepath{fill}%
\end{pgfscope}%
\begin{pgfscope}%
\pgfpathrectangle{\pgfqpoint{1.150000in}{0.150000in}}{\pgfqpoint{5.700000in}{5.700000in}}%
\pgfusepath{clip}%
\pgfsetbuttcap%
\pgfsetroundjoin%
\definecolor{currentfill}{rgb}{0.208623,0.367752,0.552675}%
\pgfsetfillcolor{currentfill}%
\pgfsetfillopacity{0.700000}%
\pgfsetlinewidth{0.000000pt}%
\definecolor{currentstroke}{rgb}{0.000000,0.000000,0.000000}%
\pgfsetstrokecolor{currentstroke}%
\pgfsetdash{}{0pt}%
\pgfpathmoveto{\pgfqpoint{6.032757in}{2.788753in}}%
\pgfpathlineto{\pgfqpoint{6.046993in}{2.787968in}}%
\pgfpathlineto{\pgfqpoint{6.061239in}{2.787248in}}%
\pgfpathlineto{\pgfqpoint{6.075496in}{2.786593in}}%
\pgfpathlineto{\pgfqpoint{6.082727in}{2.796166in}}%
\pgfpathlineto{\pgfqpoint{6.089966in}{2.806050in}}%
\pgfpathlineto{\pgfqpoint{6.097212in}{2.816254in}}%
\pgfpathlineto{\pgfqpoint{6.104466in}{2.826787in}}%
\pgfpathlineto{\pgfqpoint{6.090240in}{2.827951in}}%
\pgfpathlineto{\pgfqpoint{6.076023in}{2.829180in}}%
\pgfpathlineto{\pgfqpoint{6.061817in}{2.830472in}}%
\pgfpathlineto{\pgfqpoint{6.054540in}{2.819553in}}%
\pgfpathlineto{\pgfqpoint{6.047272in}{2.808966in}}%
\pgfpathlineto{\pgfqpoint{6.040011in}{2.798702in}}%
\pgfpathlineto{\pgfqpoint{6.032757in}{2.788753in}}%
\pgfpathclose%
\pgfusepath{fill}%
\end{pgfscope}%
\begin{pgfscope}%
\pgfpathrectangle{\pgfqpoint{1.150000in}{0.150000in}}{\pgfqpoint{5.700000in}{5.700000in}}%
\pgfusepath{clip}%
\pgfsetbuttcap%
\pgfsetroundjoin%
\definecolor{currentfill}{rgb}{0.277134,0.185228,0.489898}%
\pgfsetfillcolor{currentfill}%
\pgfsetfillopacity{0.700000}%
\pgfsetlinewidth{0.000000pt}%
\definecolor{currentstroke}{rgb}{0.000000,0.000000,0.000000}%
\pgfsetstrokecolor{currentstroke}%
\pgfsetdash{}{0pt}%
\pgfpathmoveto{\pgfqpoint{2.340055in}{2.407641in}}%
\pgfpathlineto{\pgfqpoint{2.353614in}{2.395689in}}%
\pgfpathlineto{\pgfqpoint{2.367171in}{2.383877in}}%
\pgfpathlineto{\pgfqpoint{2.380725in}{2.372203in}}%
\pgfpathlineto{\pgfqpoint{2.394277in}{2.360666in}}%
\pgfpathlineto{\pgfqpoint{2.402863in}{2.366574in}}%
\pgfpathlineto{\pgfqpoint{2.411438in}{2.372598in}}%
\pgfpathlineto{\pgfqpoint{2.420003in}{2.378737in}}%
\pgfpathlineto{\pgfqpoint{2.428557in}{2.384990in}}%
\pgfpathlineto{\pgfqpoint{2.415028in}{2.396377in}}%
\pgfpathlineto{\pgfqpoint{2.401498in}{2.407901in}}%
\pgfpathlineto{\pgfqpoint{2.387965in}{2.419563in}}%
\pgfpathlineto{\pgfqpoint{2.374429in}{2.431364in}}%
\pgfpathlineto{\pgfqpoint{2.365852in}{2.425254in}}%
\pgfpathlineto{\pgfqpoint{2.357264in}{2.419262in}}%
\pgfpathlineto{\pgfqpoint{2.348665in}{2.413390in}}%
\pgfpathlineto{\pgfqpoint{2.340055in}{2.407641in}}%
\pgfpathclose%
\pgfusepath{fill}%
\end{pgfscope}%
\begin{pgfscope}%
\pgfpathrectangle{\pgfqpoint{1.150000in}{0.150000in}}{\pgfqpoint{5.700000in}{5.700000in}}%
\pgfusepath{clip}%
\pgfsetbuttcap%
\pgfsetroundjoin%
\definecolor{currentfill}{rgb}{0.283197,0.115680,0.436115}%
\pgfsetfillcolor{currentfill}%
\pgfsetfillopacity{0.700000}%
\pgfsetlinewidth{0.000000pt}%
\definecolor{currentstroke}{rgb}{0.000000,0.000000,0.000000}%
\pgfsetstrokecolor{currentstroke}%
\pgfsetdash{}{0pt}%
\pgfpathmoveto{\pgfqpoint{2.590783in}{2.258570in}}%
\pgfpathlineto{\pgfqpoint{2.604296in}{2.248854in}}%
\pgfpathlineto{\pgfqpoint{2.617809in}{2.239260in}}%
\pgfpathlineto{\pgfqpoint{2.631322in}{2.229785in}}%
\pgfpathlineto{\pgfqpoint{2.644834in}{2.220430in}}%
\pgfpathlineto{\pgfqpoint{2.653293in}{2.227335in}}%
\pgfpathlineto{\pgfqpoint{2.661743in}{2.234329in}}%
\pgfpathlineto{\pgfqpoint{2.670184in}{2.241410in}}%
\pgfpathlineto{\pgfqpoint{2.678616in}{2.248577in}}%
\pgfpathlineto{\pgfqpoint{2.665123in}{2.257806in}}%
\pgfpathlineto{\pgfqpoint{2.651630in}{2.267154in}}%
\pgfpathlineto{\pgfqpoint{2.638138in}{2.276622in}}%
\pgfpathlineto{\pgfqpoint{2.624645in}{2.286211in}}%
\pgfpathlineto{\pgfqpoint{2.616193in}{2.279163in}}%
\pgfpathlineto{\pgfqpoint{2.607732in}{2.272206in}}%
\pgfpathlineto{\pgfqpoint{2.599263in}{2.265341in}}%
\pgfpathlineto{\pgfqpoint{2.590783in}{2.258570in}}%
\pgfpathclose%
\pgfusepath{fill}%
\end{pgfscope}%
\begin{pgfscope}%
\pgfpathrectangle{\pgfqpoint{1.150000in}{0.150000in}}{\pgfqpoint{5.700000in}{5.700000in}}%
\pgfusepath{clip}%
\pgfsetbuttcap%
\pgfsetroundjoin%
\definecolor{currentfill}{rgb}{0.273006,0.204520,0.501721}%
\pgfsetfillcolor{currentfill}%
\pgfsetfillopacity{0.700000}%
\pgfsetlinewidth{0.000000pt}%
\definecolor{currentstroke}{rgb}{0.000000,0.000000,0.000000}%
\pgfsetstrokecolor{currentstroke}%
\pgfsetdash{}{0pt}%
\pgfpathmoveto{\pgfqpoint{4.832950in}{2.422390in}}%
\pgfpathlineto{\pgfqpoint{4.846855in}{2.422100in}}%
\pgfpathlineto{\pgfqpoint{4.860769in}{2.421882in}}%
\pgfpathlineto{\pgfqpoint{4.874693in}{2.421734in}}%
\pgfpathlineto{\pgfqpoint{4.888626in}{2.421656in}}%
\pgfpathlineto{\pgfqpoint{4.896221in}{2.429029in}}%
\pgfpathlineto{\pgfqpoint{4.903812in}{2.436458in}}%
\pgfpathlineto{\pgfqpoint{4.911398in}{2.443947in}}%
\pgfpathlineto{\pgfqpoint{4.918979in}{2.451501in}}%
\pgfpathlineto{\pgfqpoint{4.905063in}{2.451845in}}%
\pgfpathlineto{\pgfqpoint{4.891156in}{2.452259in}}%
\pgfpathlineto{\pgfqpoint{4.877258in}{2.452743in}}%
\pgfpathlineto{\pgfqpoint{4.863369in}{2.453297in}}%
\pgfpathlineto{\pgfqpoint{4.855771in}{2.445470in}}%
\pgfpathlineto{\pgfqpoint{4.848169in}{2.437713in}}%
\pgfpathlineto{\pgfqpoint{4.840562in}{2.430021in}}%
\pgfpathlineto{\pgfqpoint{4.832950in}{2.422390in}}%
\pgfpathclose%
\pgfusepath{fill}%
\end{pgfscope}%
\begin{pgfscope}%
\pgfpathrectangle{\pgfqpoint{1.150000in}{0.150000in}}{\pgfqpoint{5.700000in}{5.700000in}}%
\pgfusepath{clip}%
\pgfsetbuttcap%
\pgfsetroundjoin%
\definecolor{currentfill}{rgb}{0.277941,0.056324,0.381191}%
\pgfsetfillcolor{currentfill}%
\pgfsetfillopacity{0.700000}%
\pgfsetlinewidth{0.000000pt}%
\definecolor{currentstroke}{rgb}{0.000000,0.000000,0.000000}%
\pgfsetstrokecolor{currentstroke}%
\pgfsetdash{}{0pt}%
\pgfpathmoveto{\pgfqpoint{3.578410in}{2.125684in}}%
\pgfpathlineto{\pgfqpoint{3.591977in}{2.122057in}}%
\pgfpathlineto{\pgfqpoint{3.605550in}{2.118516in}}%
\pgfpathlineto{\pgfqpoint{3.619128in}{2.115060in}}%
\pgfpathlineto{\pgfqpoint{3.632712in}{2.111688in}}%
\pgfpathlineto{\pgfqpoint{3.640771in}{2.120610in}}%
\pgfpathlineto{\pgfqpoint{3.648825in}{2.129536in}}%
\pgfpathlineto{\pgfqpoint{3.656873in}{2.138468in}}%
\pgfpathlineto{\pgfqpoint{3.664914in}{2.147407in}}%
\pgfpathlineto{\pgfqpoint{3.651342in}{2.150797in}}%
\pgfpathlineto{\pgfqpoint{3.637774in}{2.154272in}}%
\pgfpathlineto{\pgfqpoint{3.624213in}{2.157832in}}%
\pgfpathlineto{\pgfqpoint{3.610657in}{2.161478in}}%
\pgfpathlineto{\pgfqpoint{3.602604in}{2.152513in}}%
\pgfpathlineto{\pgfqpoint{3.594545in}{2.143559in}}%
\pgfpathlineto{\pgfqpoint{3.586480in}{2.134616in}}%
\pgfpathlineto{\pgfqpoint{3.578410in}{2.125684in}}%
\pgfpathclose%
\pgfusepath{fill}%
\end{pgfscope}%
\begin{pgfscope}%
\pgfpathrectangle{\pgfqpoint{1.150000in}{0.150000in}}{\pgfqpoint{5.700000in}{5.700000in}}%
\pgfusepath{clip}%
\pgfsetbuttcap%
\pgfsetroundjoin%
\definecolor{currentfill}{rgb}{0.283091,0.110553,0.431554}%
\pgfsetfillcolor{currentfill}%
\pgfsetfillopacity{0.700000}%
\pgfsetlinewidth{0.000000pt}%
\definecolor{currentstroke}{rgb}{0.000000,0.000000,0.000000}%
\pgfsetstrokecolor{currentstroke}%
\pgfsetdash{}{0pt}%
\pgfpathmoveto{\pgfqpoint{4.119407in}{2.228818in}}%
\pgfpathlineto{\pgfqpoint{4.133101in}{2.227162in}}%
\pgfpathlineto{\pgfqpoint{4.146803in}{2.225583in}}%
\pgfpathlineto{\pgfqpoint{4.160513in}{2.224080in}}%
\pgfpathlineto{\pgfqpoint{4.174230in}{2.222655in}}%
\pgfpathlineto{\pgfqpoint{4.182099in}{2.231124in}}%
\pgfpathlineto{\pgfqpoint{4.189962in}{2.239593in}}%
\pgfpathlineto{\pgfqpoint{4.197819in}{2.248066in}}%
\pgfpathlineto{\pgfqpoint{4.205671in}{2.256544in}}%
\pgfpathlineto{\pgfqpoint{4.191965in}{2.258091in}}%
\pgfpathlineto{\pgfqpoint{4.178267in}{2.259715in}}%
\pgfpathlineto{\pgfqpoint{4.164576in}{2.261415in}}%
\pgfpathlineto{\pgfqpoint{4.150893in}{2.263193in}}%
\pgfpathlineto{\pgfqpoint{4.143030in}{2.254585in}}%
\pgfpathlineto{\pgfqpoint{4.135161in}{2.245989in}}%
\pgfpathlineto{\pgfqpoint{4.127287in}{2.237400in}}%
\pgfpathlineto{\pgfqpoint{4.119407in}{2.228818in}}%
\pgfpathclose%
\pgfusepath{fill}%
\end{pgfscope}%
\begin{pgfscope}%
\pgfpathrectangle{\pgfqpoint{1.150000in}{0.150000in}}{\pgfqpoint{5.700000in}{5.700000in}}%
\pgfusepath{clip}%
\pgfsetbuttcap%
\pgfsetroundjoin%
\definecolor{currentfill}{rgb}{0.280894,0.078907,0.402329}%
\pgfsetfillcolor{currentfill}%
\pgfsetfillopacity{0.700000}%
\pgfsetlinewidth{0.000000pt}%
\definecolor{currentstroke}{rgb}{0.000000,0.000000,0.000000}%
\pgfsetstrokecolor{currentstroke}%
\pgfsetdash{}{0pt}%
\pgfpathmoveto{\pgfqpoint{2.786576in}{2.178927in}}%
\pgfpathlineto{\pgfqpoint{2.800074in}{2.170730in}}%
\pgfpathlineto{\pgfqpoint{2.813574in}{2.162643in}}%
\pgfpathlineto{\pgfqpoint{2.827075in}{2.154667in}}%
\pgfpathlineto{\pgfqpoint{2.840578in}{2.146798in}}%
\pgfpathlineto{\pgfqpoint{2.848946in}{2.154389in}}%
\pgfpathlineto{\pgfqpoint{2.857307in}{2.162047in}}%
\pgfpathlineto{\pgfqpoint{2.865659in}{2.169772in}}%
\pgfpathlineto{\pgfqpoint{2.874004in}{2.177561in}}%
\pgfpathlineto{\pgfqpoint{2.860519in}{2.185324in}}%
\pgfpathlineto{\pgfqpoint{2.847035in}{2.193196in}}%
\pgfpathlineto{\pgfqpoint{2.833552in}{2.201178in}}%
\pgfpathlineto{\pgfqpoint{2.820071in}{2.209270in}}%
\pgfpathlineto{\pgfqpoint{2.811710in}{2.201577in}}%
\pgfpathlineto{\pgfqpoint{2.803340in}{2.193955in}}%
\pgfpathlineto{\pgfqpoint{2.794962in}{2.186405in}}%
\pgfpathlineto{\pgfqpoint{2.786576in}{2.178927in}}%
\pgfpathclose%
\pgfusepath{fill}%
\end{pgfscope}%
\begin{pgfscope}%
\pgfpathrectangle{\pgfqpoint{1.150000in}{0.150000in}}{\pgfqpoint{5.700000in}{5.700000in}}%
\pgfusepath{clip}%
\pgfsetbuttcap%
\pgfsetroundjoin%
\definecolor{currentfill}{rgb}{0.280894,0.078907,0.402329}%
\pgfsetfillcolor{currentfill}%
\pgfsetfillopacity{0.700000}%
\pgfsetlinewidth{0.000000pt}%
\definecolor{currentstroke}{rgb}{0.000000,0.000000,0.000000}%
\pgfsetstrokecolor{currentstroke}%
\pgfsetdash{}{0pt}%
\pgfpathmoveto{\pgfqpoint{3.805733in}{2.158818in}}%
\pgfpathlineto{\pgfqpoint{3.819350in}{2.156136in}}%
\pgfpathlineto{\pgfqpoint{3.832973in}{2.153534in}}%
\pgfpathlineto{\pgfqpoint{3.846602in}{2.151014in}}%
\pgfpathlineto{\pgfqpoint{3.860238in}{2.148574in}}%
\pgfpathlineto{\pgfqpoint{3.868219in}{2.157405in}}%
\pgfpathlineto{\pgfqpoint{3.876193in}{2.166235in}}%
\pgfpathlineto{\pgfqpoint{3.884163in}{2.175064in}}%
\pgfpathlineto{\pgfqpoint{3.892126in}{2.183894in}}%
\pgfpathlineto{\pgfqpoint{3.878501in}{2.186394in}}%
\pgfpathlineto{\pgfqpoint{3.864883in}{2.188974in}}%
\pgfpathlineto{\pgfqpoint{3.851271in}{2.191635in}}%
\pgfpathlineto{\pgfqpoint{3.837665in}{2.194378in}}%
\pgfpathlineto{\pgfqpoint{3.829691in}{2.185480in}}%
\pgfpathlineto{\pgfqpoint{3.821711in}{2.176589in}}%
\pgfpathlineto{\pgfqpoint{3.813725in}{2.167702in}}%
\pgfpathlineto{\pgfqpoint{3.805733in}{2.158818in}}%
\pgfpathclose%
\pgfusepath{fill}%
\end{pgfscope}%
\begin{pgfscope}%
\pgfpathrectangle{\pgfqpoint{1.150000in}{0.150000in}}{\pgfqpoint{5.700000in}{5.700000in}}%
\pgfusepath{clip}%
\pgfsetbuttcap%
\pgfsetroundjoin%
\definecolor{currentfill}{rgb}{0.281887,0.150881,0.465405}%
\pgfsetfillcolor{currentfill}%
\pgfsetfillopacity{0.700000}%
\pgfsetlinewidth{0.000000pt}%
\definecolor{currentstroke}{rgb}{0.000000,0.000000,0.000000}%
\pgfsetstrokecolor{currentstroke}%
\pgfsetdash{}{0pt}%
\pgfpathmoveto{\pgfqpoint{4.433080in}{2.308581in}}%
\pgfpathlineto{\pgfqpoint{4.446865in}{2.307705in}}%
\pgfpathlineto{\pgfqpoint{4.460659in}{2.306903in}}%
\pgfpathlineto{\pgfqpoint{4.474461in}{2.306175in}}%
\pgfpathlineto{\pgfqpoint{4.488272in}{2.305521in}}%
\pgfpathlineto{\pgfqpoint{4.496025in}{2.313476in}}%
\pgfpathlineto{\pgfqpoint{4.503772in}{2.321447in}}%
\pgfpathlineto{\pgfqpoint{4.511514in}{2.329436in}}%
\pgfpathlineto{\pgfqpoint{4.519250in}{2.337447in}}%
\pgfpathlineto{\pgfqpoint{4.505452in}{2.338285in}}%
\pgfpathlineto{\pgfqpoint{4.491663in}{2.339197in}}%
\pgfpathlineto{\pgfqpoint{4.477882in}{2.340182in}}%
\pgfpathlineto{\pgfqpoint{4.464109in}{2.341240in}}%
\pgfpathlineto{\pgfqpoint{4.456360in}{2.333038in}}%
\pgfpathlineto{\pgfqpoint{4.448605in}{2.324863in}}%
\pgfpathlineto{\pgfqpoint{4.440845in}{2.316712in}}%
\pgfpathlineto{\pgfqpoint{4.433080in}{2.308581in}}%
\pgfpathclose%
\pgfusepath{fill}%
\end{pgfscope}%
\begin{pgfscope}%
\pgfpathrectangle{\pgfqpoint{1.150000in}{0.150000in}}{\pgfqpoint{5.700000in}{5.700000in}}%
\pgfusepath{clip}%
\pgfsetbuttcap%
\pgfsetroundjoin%
\definecolor{currentfill}{rgb}{0.276022,0.044167,0.370164}%
\pgfsetfillcolor{currentfill}%
\pgfsetfillopacity{0.700000}%
\pgfsetlinewidth{0.000000pt}%
\definecolor{currentstroke}{rgb}{0.000000,0.000000,0.000000}%
\pgfsetstrokecolor{currentstroke}%
\pgfsetdash{}{0pt}%
\pgfpathmoveto{\pgfqpoint{3.350945in}{2.105132in}}%
\pgfpathlineto{\pgfqpoint{3.364476in}{2.100425in}}%
\pgfpathlineto{\pgfqpoint{3.378012in}{2.095808in}}%
\pgfpathlineto{\pgfqpoint{3.391552in}{2.091281in}}%
\pgfpathlineto{\pgfqpoint{3.405097in}{2.086844in}}%
\pgfpathlineto{\pgfqpoint{3.413239in}{2.095655in}}%
\pgfpathlineto{\pgfqpoint{3.421375in}{2.104483in}}%
\pgfpathlineto{\pgfqpoint{3.429504in}{2.113328in}}%
\pgfpathlineto{\pgfqpoint{3.437628in}{2.122190in}}%
\pgfpathlineto{\pgfqpoint{3.424095in}{2.126605in}}%
\pgfpathlineto{\pgfqpoint{3.410566in}{2.131110in}}%
\pgfpathlineto{\pgfqpoint{3.397043in}{2.135705in}}%
\pgfpathlineto{\pgfqpoint{3.383524in}{2.140390in}}%
\pgfpathlineto{\pgfqpoint{3.375388in}{2.131542in}}%
\pgfpathlineto{\pgfqpoint{3.367247in}{2.122717in}}%
\pgfpathlineto{\pgfqpoint{3.359099in}{2.113914in}}%
\pgfpathlineto{\pgfqpoint{3.350945in}{2.105132in}}%
\pgfpathclose%
\pgfusepath{fill}%
\end{pgfscope}%
\begin{pgfscope}%
\pgfpathrectangle{\pgfqpoint{1.150000in}{0.150000in}}{\pgfqpoint{5.700000in}{5.700000in}}%
\pgfusepath{clip}%
\pgfsetbuttcap%
\pgfsetroundjoin%
\definecolor{currentfill}{rgb}{0.239346,0.300855,0.540844}%
\pgfsetfillcolor{currentfill}%
\pgfsetfillopacity{0.700000}%
\pgfsetlinewidth{0.000000pt}%
\definecolor{currentstroke}{rgb}{0.000000,0.000000,0.000000}%
\pgfsetstrokecolor{currentstroke}%
\pgfsetdash{}{0pt}%
\pgfpathmoveto{\pgfqpoint{5.546705in}{2.627261in}}%
\pgfpathlineto{\pgfqpoint{5.560825in}{2.627147in}}%
\pgfpathlineto{\pgfqpoint{5.574955in}{2.627099in}}%
\pgfpathlineto{\pgfqpoint{5.589096in}{2.627118in}}%
\pgfpathlineto{\pgfqpoint{5.603247in}{2.627204in}}%
\pgfpathlineto{\pgfqpoint{5.610576in}{2.634627in}}%
\pgfpathlineto{\pgfqpoint{5.617905in}{2.642228in}}%
\pgfpathlineto{\pgfqpoint{5.625234in}{2.650015in}}%
\pgfpathlineto{\pgfqpoint{5.632563in}{2.657993in}}%
\pgfpathlineto{\pgfqpoint{5.618437in}{2.658316in}}%
\pgfpathlineto{\pgfqpoint{5.604320in}{2.658706in}}%
\pgfpathlineto{\pgfqpoint{5.590214in}{2.659161in}}%
\pgfpathlineto{\pgfqpoint{5.576117in}{2.659684in}}%
\pgfpathlineto{\pgfqpoint{5.568764in}{2.651290in}}%
\pgfpathlineto{\pgfqpoint{5.561411in}{2.643093in}}%
\pgfpathlineto{\pgfqpoint{5.554058in}{2.635086in}}%
\pgfpathlineto{\pgfqpoint{5.546705in}{2.627261in}}%
\pgfpathclose%
\pgfusepath{fill}%
\end{pgfscope}%
\begin{pgfscope}%
\pgfpathrectangle{\pgfqpoint{1.150000in}{0.150000in}}{\pgfqpoint{5.700000in}{5.700000in}}%
\pgfusepath{clip}%
\pgfsetbuttcap%
\pgfsetroundjoin%
\definecolor{currentfill}{rgb}{0.260571,0.246922,0.522828}%
\pgfsetfillcolor{currentfill}%
\pgfsetfillopacity{0.700000}%
\pgfsetlinewidth{0.000000pt}%
\definecolor{currentstroke}{rgb}{0.000000,0.000000,0.000000}%
\pgfsetstrokecolor{currentstroke}%
\pgfsetdash{}{0pt}%
\pgfpathmoveto{\pgfqpoint{5.146798in}{2.509095in}}%
\pgfpathlineto{\pgfqpoint{5.160803in}{2.509061in}}%
\pgfpathlineto{\pgfqpoint{5.174817in}{2.509095in}}%
\pgfpathlineto{\pgfqpoint{5.188842in}{2.509198in}}%
\pgfpathlineto{\pgfqpoint{5.202876in}{2.509370in}}%
\pgfpathlineto{\pgfqpoint{5.210349in}{2.516467in}}%
\pgfpathlineto{\pgfqpoint{5.217818in}{2.523663in}}%
\pgfpathlineto{\pgfqpoint{5.225284in}{2.530964in}}%
\pgfpathlineto{\pgfqpoint{5.232747in}{2.538376in}}%
\pgfpathlineto{\pgfqpoint{5.218732in}{2.538532in}}%
\pgfpathlineto{\pgfqpoint{5.204727in}{2.538756in}}%
\pgfpathlineto{\pgfqpoint{5.190732in}{2.539049in}}%
\pgfpathlineto{\pgfqpoint{5.176746in}{2.539410in}}%
\pgfpathlineto{\pgfqpoint{5.169264in}{2.531664in}}%
\pgfpathlineto{\pgfqpoint{5.161778in}{2.524034in}}%
\pgfpathlineto{\pgfqpoint{5.154290in}{2.516513in}}%
\pgfpathlineto{\pgfqpoint{5.146798in}{2.509095in}}%
\pgfpathclose%
\pgfusepath{fill}%
\end{pgfscope}%
\begin{pgfscope}%
\pgfpathrectangle{\pgfqpoint{1.150000in}{0.150000in}}{\pgfqpoint{5.700000in}{5.700000in}}%
\pgfusepath{clip}%
\pgfsetbuttcap%
\pgfsetroundjoin%
\definecolor{currentfill}{rgb}{0.214298,0.355619,0.551184}%
\pgfsetfillcolor{currentfill}%
\pgfsetfillopacity{0.700000}%
\pgfsetlinewidth{0.000000pt}%
\definecolor{currentstroke}{rgb}{0.000000,0.000000,0.000000}%
\pgfsetstrokecolor{currentstroke}%
\pgfsetdash{}{0pt}%
\pgfpathmoveto{\pgfqpoint{5.946850in}{2.753757in}}%
\pgfpathlineto{\pgfqpoint{5.961073in}{2.753202in}}%
\pgfpathlineto{\pgfqpoint{5.975306in}{2.752712in}}%
\pgfpathlineto{\pgfqpoint{5.989551in}{2.752287in}}%
\pgfpathlineto{\pgfqpoint{6.003805in}{2.751927in}}%
\pgfpathlineto{\pgfqpoint{6.011035in}{2.760706in}}%
\pgfpathlineto{\pgfqpoint{6.018270in}{2.769764in}}%
\pgfpathlineto{\pgfqpoint{6.025510in}{2.779110in}}%
\pgfpathlineto{\pgfqpoint{6.032757in}{2.788753in}}%
\pgfpathlineto{\pgfqpoint{6.018532in}{2.789602in}}%
\pgfpathlineto{\pgfqpoint{6.004317in}{2.790516in}}%
\pgfpathlineto{\pgfqpoint{5.990112in}{2.791495in}}%
\pgfpathlineto{\pgfqpoint{5.975918in}{2.792538in}}%
\pgfpathlineto{\pgfqpoint{5.968642in}{2.782399in}}%
\pgfpathlineto{\pgfqpoint{5.961372in}{2.772562in}}%
\pgfpathlineto{\pgfqpoint{5.954108in}{2.763017in}}%
\pgfpathlineto{\pgfqpoint{5.946850in}{2.753757in}}%
\pgfpathclose%
\pgfusepath{fill}%
\end{pgfscope}%
\begin{pgfscope}%
\pgfpathrectangle{\pgfqpoint{1.150000in}{0.150000in}}{\pgfqpoint{5.700000in}{5.700000in}}%
\pgfusepath{clip}%
\pgfsetbuttcap%
\pgfsetroundjoin%
\definecolor{currentfill}{rgb}{0.280255,0.165693,0.476498}%
\pgfsetfillcolor{currentfill}%
\pgfsetfillopacity{0.700000}%
\pgfsetlinewidth{0.000000pt}%
\definecolor{currentstroke}{rgb}{0.000000,0.000000,0.000000}%
\pgfsetstrokecolor{currentstroke}%
\pgfsetdash{}{0pt}%
\pgfpathmoveto{\pgfqpoint{2.394277in}{2.360666in}}%
\pgfpathlineto{\pgfqpoint{2.407826in}{2.349264in}}%
\pgfpathlineto{\pgfqpoint{2.421374in}{2.337997in}}%
\pgfpathlineto{\pgfqpoint{2.434920in}{2.326863in}}%
\pgfpathlineto{\pgfqpoint{2.448464in}{2.315861in}}%
\pgfpathlineto{\pgfqpoint{2.457027in}{2.321926in}}%
\pgfpathlineto{\pgfqpoint{2.465579in}{2.328102in}}%
\pgfpathlineto{\pgfqpoint{2.474121in}{2.334388in}}%
\pgfpathlineto{\pgfqpoint{2.482653in}{2.340782in}}%
\pgfpathlineto{\pgfqpoint{2.469131in}{2.351635in}}%
\pgfpathlineto{\pgfqpoint{2.455608in}{2.362620in}}%
\pgfpathlineto{\pgfqpoint{2.442084in}{2.373738in}}%
\pgfpathlineto{\pgfqpoint{2.428557in}{2.384990in}}%
\pgfpathlineto{\pgfqpoint{2.420003in}{2.378737in}}%
\pgfpathlineto{\pgfqpoint{2.411438in}{2.372598in}}%
\pgfpathlineto{\pgfqpoint{2.402863in}{2.366574in}}%
\pgfpathlineto{\pgfqpoint{2.394277in}{2.360666in}}%
\pgfpathclose%
\pgfusepath{fill}%
\end{pgfscope}%
\begin{pgfscope}%
\pgfpathrectangle{\pgfqpoint{1.150000in}{0.150000in}}{\pgfqpoint{5.700000in}{5.700000in}}%
\pgfusepath{clip}%
\pgfsetbuttcap%
\pgfsetroundjoin%
\definecolor{currentfill}{rgb}{0.275191,0.194905,0.496005}%
\pgfsetfillcolor{currentfill}%
\pgfsetfillopacity{0.700000}%
\pgfsetlinewidth{0.000000pt}%
\definecolor{currentstroke}{rgb}{0.000000,0.000000,0.000000}%
\pgfsetstrokecolor{currentstroke}%
\pgfsetdash{}{0pt}%
\pgfpathmoveto{\pgfqpoint{4.746865in}{2.393259in}}%
\pgfpathlineto{\pgfqpoint{4.760749in}{2.392931in}}%
\pgfpathlineto{\pgfqpoint{4.774642in}{2.392675in}}%
\pgfpathlineto{\pgfqpoint{4.788544in}{2.392489in}}%
\pgfpathlineto{\pgfqpoint{4.802455in}{2.392375in}}%
\pgfpathlineto{\pgfqpoint{4.810086in}{2.399811in}}%
\pgfpathlineto{\pgfqpoint{4.817713in}{2.407289in}}%
\pgfpathlineto{\pgfqpoint{4.825334in}{2.414814in}}%
\pgfpathlineto{\pgfqpoint{4.832950in}{2.422390in}}%
\pgfpathlineto{\pgfqpoint{4.819055in}{2.422749in}}%
\pgfpathlineto{\pgfqpoint{4.805168in}{2.423180in}}%
\pgfpathlineto{\pgfqpoint{4.791290in}{2.423682in}}%
\pgfpathlineto{\pgfqpoint{4.777422in}{2.424254in}}%
\pgfpathlineto{\pgfqpoint{4.769790in}{2.416426in}}%
\pgfpathlineto{\pgfqpoint{4.762153in}{2.408654in}}%
\pgfpathlineto{\pgfqpoint{4.754512in}{2.400933in}}%
\pgfpathlineto{\pgfqpoint{4.746865in}{2.393259in}}%
\pgfpathclose%
\pgfusepath{fill}%
\end{pgfscope}%
\begin{pgfscope}%
\pgfpathrectangle{\pgfqpoint{1.150000in}{0.150000in}}{\pgfqpoint{5.700000in}{5.700000in}}%
\pgfusepath{clip}%
\pgfsetbuttcap%
\pgfsetroundjoin%
\definecolor{currentfill}{rgb}{0.282656,0.100196,0.422160}%
\pgfsetfillcolor{currentfill}%
\pgfsetfillopacity{0.700000}%
\pgfsetlinewidth{0.000000pt}%
\definecolor{currentstroke}{rgb}{0.000000,0.000000,0.000000}%
\pgfsetstrokecolor{currentstroke}%
\pgfsetdash{}{0pt}%
\pgfpathmoveto{\pgfqpoint{4.033082in}{2.201494in}}%
\pgfpathlineto{\pgfqpoint{4.046758in}{2.199628in}}%
\pgfpathlineto{\pgfqpoint{4.060442in}{2.197841in}}%
\pgfpathlineto{\pgfqpoint{4.074133in}{2.196131in}}%
\pgfpathlineto{\pgfqpoint{4.087831in}{2.194499in}}%
\pgfpathlineto{\pgfqpoint{4.095734in}{2.203081in}}%
\pgfpathlineto{\pgfqpoint{4.103630in}{2.211660in}}%
\pgfpathlineto{\pgfqpoint{4.111521in}{2.220238in}}%
\pgfpathlineto{\pgfqpoint{4.119407in}{2.228818in}}%
\pgfpathlineto{\pgfqpoint{4.105720in}{2.230551in}}%
\pgfpathlineto{\pgfqpoint{4.092040in}{2.232362in}}%
\pgfpathlineto{\pgfqpoint{4.078368in}{2.234251in}}%
\pgfpathlineto{\pgfqpoint{4.064702in}{2.236217in}}%
\pgfpathlineto{\pgfqpoint{4.056806in}{2.227529in}}%
\pgfpathlineto{\pgfqpoint{4.048903in}{2.218848in}}%
\pgfpathlineto{\pgfqpoint{4.040996in}{2.210170in}}%
\pgfpathlineto{\pgfqpoint{4.033082in}{2.201494in}}%
\pgfpathclose%
\pgfusepath{fill}%
\end{pgfscope}%
\begin{pgfscope}%
\pgfpathrectangle{\pgfqpoint{1.150000in}{0.150000in}}{\pgfqpoint{5.700000in}{5.700000in}}%
\pgfusepath{clip}%
\pgfsetbuttcap%
\pgfsetroundjoin%
\definecolor{currentfill}{rgb}{0.243113,0.292092,0.538516}%
\pgfsetfillcolor{currentfill}%
\pgfsetfillopacity{0.700000}%
\pgfsetlinewidth{0.000000pt}%
\definecolor{currentstroke}{rgb}{0.000000,0.000000,0.000000}%
\pgfsetstrokecolor{currentstroke}%
\pgfsetdash{}{0pt}%
\pgfpathmoveto{\pgfqpoint{5.460814in}{2.597217in}}%
\pgfpathlineto{\pgfqpoint{5.474916in}{2.597224in}}%
\pgfpathlineto{\pgfqpoint{5.489029in}{2.597298in}}%
\pgfpathlineto{\pgfqpoint{5.503152in}{2.597440in}}%
\pgfpathlineto{\pgfqpoint{5.517285in}{2.597648in}}%
\pgfpathlineto{\pgfqpoint{5.524642in}{2.604812in}}%
\pgfpathlineto{\pgfqpoint{5.531997in}{2.612131in}}%
\pgfpathlineto{\pgfqpoint{5.539351in}{2.619612in}}%
\pgfpathlineto{\pgfqpoint{5.546705in}{2.627261in}}%
\pgfpathlineto{\pgfqpoint{5.532595in}{2.627442in}}%
\pgfpathlineto{\pgfqpoint{5.518495in}{2.627689in}}%
\pgfpathlineto{\pgfqpoint{5.504405in}{2.628003in}}%
\pgfpathlineto{\pgfqpoint{5.490325in}{2.628384in}}%
\pgfpathlineto{\pgfqpoint{5.482949in}{2.620339in}}%
\pgfpathlineto{\pgfqpoint{5.475571in}{2.612468in}}%
\pgfpathlineto{\pgfqpoint{5.468193in}{2.604763in}}%
\pgfpathlineto{\pgfqpoint{5.460814in}{2.597217in}}%
\pgfpathclose%
\pgfusepath{fill}%
\end{pgfscope}%
\begin{pgfscope}%
\pgfpathrectangle{\pgfqpoint{1.150000in}{0.150000in}}{\pgfqpoint{5.700000in}{5.700000in}}%
\pgfusepath{clip}%
\pgfsetbuttcap%
\pgfsetroundjoin%
\definecolor{currentfill}{rgb}{0.282623,0.140926,0.457517}%
\pgfsetfillcolor{currentfill}%
\pgfsetfillopacity{0.700000}%
\pgfsetlinewidth{0.000000pt}%
\definecolor{currentstroke}{rgb}{0.000000,0.000000,0.000000}%
\pgfsetstrokecolor{currentstroke}%
\pgfsetdash{}{0pt}%
\pgfpathmoveto{\pgfqpoint{4.346854in}{2.279785in}}%
\pgfpathlineto{\pgfqpoint{4.360619in}{2.278776in}}%
\pgfpathlineto{\pgfqpoint{4.374392in}{2.277841in}}%
\pgfpathlineto{\pgfqpoint{4.388173in}{2.276980in}}%
\pgfpathlineto{\pgfqpoint{4.401962in}{2.276194in}}%
\pgfpathlineto{\pgfqpoint{4.409750in}{2.284276in}}%
\pgfpathlineto{\pgfqpoint{4.417532in}{2.292366in}}%
\pgfpathlineto{\pgfqpoint{4.425309in}{2.300466in}}%
\pgfpathlineto{\pgfqpoint{4.433080in}{2.308581in}}%
\pgfpathlineto{\pgfqpoint{4.419303in}{2.309530in}}%
\pgfpathlineto{\pgfqpoint{4.405534in}{2.310553in}}%
\pgfpathlineto{\pgfqpoint{4.391773in}{2.311651in}}%
\pgfpathlineto{\pgfqpoint{4.378021in}{2.312823in}}%
\pgfpathlineto{\pgfqpoint{4.370237in}{2.304539in}}%
\pgfpathlineto{\pgfqpoint{4.362448in}{2.296273in}}%
\pgfpathlineto{\pgfqpoint{4.354654in}{2.288023in}}%
\pgfpathlineto{\pgfqpoint{4.346854in}{2.279785in}}%
\pgfpathclose%
\pgfusepath{fill}%
\end{pgfscope}%
\begin{pgfscope}%
\pgfpathrectangle{\pgfqpoint{1.150000in}{0.150000in}}{\pgfqpoint{5.700000in}{5.700000in}}%
\pgfusepath{clip}%
\pgfsetbuttcap%
\pgfsetroundjoin%
\definecolor{currentfill}{rgb}{0.282910,0.105393,0.426902}%
\pgfsetfillcolor{currentfill}%
\pgfsetfillopacity{0.700000}%
\pgfsetlinewidth{0.000000pt}%
\definecolor{currentstroke}{rgb}{0.000000,0.000000,0.000000}%
\pgfsetstrokecolor{currentstroke}%
\pgfsetdash{}{0pt}%
\pgfpathmoveto{\pgfqpoint{2.644834in}{2.220430in}}%
\pgfpathlineto{\pgfqpoint{2.658347in}{2.211193in}}%
\pgfpathlineto{\pgfqpoint{2.671859in}{2.202073in}}%
\pgfpathlineto{\pgfqpoint{2.685373in}{2.193070in}}%
\pgfpathlineto{\pgfqpoint{2.698886in}{2.184182in}}%
\pgfpathlineto{\pgfqpoint{2.707325in}{2.191222in}}%
\pgfpathlineto{\pgfqpoint{2.715756in}{2.198345in}}%
\pgfpathlineto{\pgfqpoint{2.724178in}{2.205550in}}%
\pgfpathlineto{\pgfqpoint{2.732591in}{2.212835in}}%
\pgfpathlineto{\pgfqpoint{2.719096in}{2.221597in}}%
\pgfpathlineto{\pgfqpoint{2.705603in}{2.230473in}}%
\pgfpathlineto{\pgfqpoint{2.692109in}{2.239467in}}%
\pgfpathlineto{\pgfqpoint{2.678616in}{2.248577in}}%
\pgfpathlineto{\pgfqpoint{2.670184in}{2.241410in}}%
\pgfpathlineto{\pgfqpoint{2.661743in}{2.234329in}}%
\pgfpathlineto{\pgfqpoint{2.653293in}{2.227335in}}%
\pgfpathlineto{\pgfqpoint{2.644834in}{2.220430in}}%
\pgfpathclose%
\pgfusepath{fill}%
\end{pgfscope}%
\begin{pgfscope}%
\pgfpathrectangle{\pgfqpoint{1.150000in}{0.150000in}}{\pgfqpoint{5.700000in}{5.700000in}}%
\pgfusepath{clip}%
\pgfsetbuttcap%
\pgfsetroundjoin%
\definecolor{currentfill}{rgb}{0.277018,0.050344,0.375715}%
\pgfsetfillcolor{currentfill}%
\pgfsetfillopacity{0.700000}%
\pgfsetlinewidth{0.000000pt}%
\definecolor{currentstroke}{rgb}{0.000000,0.000000,0.000000}%
\pgfsetstrokecolor{currentstroke}%
\pgfsetdash{}{0pt}%
\pgfpathmoveto{\pgfqpoint{3.491808in}{2.105416in}}%
\pgfpathlineto{\pgfqpoint{3.505365in}{2.101442in}}%
\pgfpathlineto{\pgfqpoint{3.518928in}{2.097555in}}%
\pgfpathlineto{\pgfqpoint{3.532496in}{2.093755in}}%
\pgfpathlineto{\pgfqpoint{3.546069in}{2.090041in}}%
\pgfpathlineto{\pgfqpoint{3.554163in}{2.098940in}}%
\pgfpathlineto{\pgfqpoint{3.562251in}{2.107846in}}%
\pgfpathlineto{\pgfqpoint{3.570333in}{2.116760in}}%
\pgfpathlineto{\pgfqpoint{3.578410in}{2.125684in}}%
\pgfpathlineto{\pgfqpoint{3.564848in}{2.129396in}}%
\pgfpathlineto{\pgfqpoint{3.551292in}{2.133195in}}%
\pgfpathlineto{\pgfqpoint{3.537740in}{2.137080in}}%
\pgfpathlineto{\pgfqpoint{3.524194in}{2.141052in}}%
\pgfpathlineto{\pgfqpoint{3.516107in}{2.132123in}}%
\pgfpathlineto{\pgfqpoint{3.508013in}{2.123208in}}%
\pgfpathlineto{\pgfqpoint{3.499913in}{2.114305in}}%
\pgfpathlineto{\pgfqpoint{3.491808in}{2.105416in}}%
\pgfpathclose%
\pgfusepath{fill}%
\end{pgfscope}%
\begin{pgfscope}%
\pgfpathrectangle{\pgfqpoint{1.150000in}{0.150000in}}{\pgfqpoint{5.700000in}{5.700000in}}%
\pgfusepath{clip}%
\pgfsetbuttcap%
\pgfsetroundjoin%
\definecolor{currentfill}{rgb}{0.220057,0.343307,0.549413}%
\pgfsetfillcolor{currentfill}%
\pgfsetfillopacity{0.700000}%
\pgfsetlinewidth{0.000000pt}%
\definecolor{currentstroke}{rgb}{0.000000,0.000000,0.000000}%
\pgfsetstrokecolor{currentstroke}%
\pgfsetdash{}{0pt}%
\pgfpathmoveto{\pgfqpoint{5.860961in}{2.720383in}}%
\pgfpathlineto{\pgfqpoint{5.875170in}{2.720037in}}%
\pgfpathlineto{\pgfqpoint{5.889390in}{2.719756in}}%
\pgfpathlineto{\pgfqpoint{5.903620in}{2.719541in}}%
\pgfpathlineto{\pgfqpoint{5.917861in}{2.719391in}}%
\pgfpathlineto{\pgfqpoint{5.925102in}{2.727598in}}%
\pgfpathlineto{\pgfqpoint{5.932347in}{2.736056in}}%
\pgfpathlineto{\pgfqpoint{5.939596in}{2.744773in}}%
\pgfpathlineto{\pgfqpoint{5.946850in}{2.753757in}}%
\pgfpathlineto{\pgfqpoint{5.932637in}{2.754377in}}%
\pgfpathlineto{\pgfqpoint{5.918435in}{2.755061in}}%
\pgfpathlineto{\pgfqpoint{5.904243in}{2.755811in}}%
\pgfpathlineto{\pgfqpoint{5.890062in}{2.756626in}}%
\pgfpathlineto{\pgfqpoint{5.882780in}{2.747165in}}%
\pgfpathlineto{\pgfqpoint{5.875503in}{2.737977in}}%
\pgfpathlineto{\pgfqpoint{5.868230in}{2.729052in}}%
\pgfpathlineto{\pgfqpoint{5.860961in}{2.720383in}}%
\pgfpathclose%
\pgfusepath{fill}%
\end{pgfscope}%
\begin{pgfscope}%
\pgfpathrectangle{\pgfqpoint{1.150000in}{0.150000in}}{\pgfqpoint{5.700000in}{5.700000in}}%
\pgfusepath{clip}%
\pgfsetbuttcap%
\pgfsetroundjoin%
\definecolor{currentfill}{rgb}{0.263663,0.237631,0.518762}%
\pgfsetfillcolor{currentfill}%
\pgfsetfillopacity{0.700000}%
\pgfsetlinewidth{0.000000pt}%
\definecolor{currentstroke}{rgb}{0.000000,0.000000,0.000000}%
\pgfsetstrokecolor{currentstroke}%
\pgfsetdash{}{0pt}%
\pgfpathmoveto{\pgfqpoint{5.060796in}{2.479947in}}%
\pgfpathlineto{\pgfqpoint{5.074780in}{2.479944in}}%
\pgfpathlineto{\pgfqpoint{5.088775in}{2.480010in}}%
\pgfpathlineto{\pgfqpoint{5.102779in}{2.480146in}}%
\pgfpathlineto{\pgfqpoint{5.116792in}{2.480350in}}%
\pgfpathlineto{\pgfqpoint{5.124300in}{2.487409in}}%
\pgfpathlineto{\pgfqpoint{5.131803in}{2.494549in}}%
\pgfpathlineto{\pgfqpoint{5.139302in}{2.501776in}}%
\pgfpathlineto{\pgfqpoint{5.146798in}{2.509095in}}%
\pgfpathlineto{\pgfqpoint{5.132803in}{2.509198in}}%
\pgfpathlineto{\pgfqpoint{5.118817in}{2.509370in}}%
\pgfpathlineto{\pgfqpoint{5.104841in}{2.509611in}}%
\pgfpathlineto{\pgfqpoint{5.090875in}{2.509920in}}%
\pgfpathlineto{\pgfqpoint{5.083361in}{2.502287in}}%
\pgfpathlineto{\pgfqpoint{5.075843in}{2.494750in}}%
\pgfpathlineto{\pgfqpoint{5.068321in}{2.487306in}}%
\pgfpathlineto{\pgfqpoint{5.060796in}{2.479947in}}%
\pgfpathclose%
\pgfusepath{fill}%
\end{pgfscope}%
\begin{pgfscope}%
\pgfpathrectangle{\pgfqpoint{1.150000in}{0.150000in}}{\pgfqpoint{5.700000in}{5.700000in}}%
\pgfusepath{clip}%
\pgfsetbuttcap%
\pgfsetroundjoin%
\definecolor{currentfill}{rgb}{0.279566,0.067836,0.391917}%
\pgfsetfillcolor{currentfill}%
\pgfsetfillopacity{0.700000}%
\pgfsetlinewidth{0.000000pt}%
\definecolor{currentstroke}{rgb}{0.000000,0.000000,0.000000}%
\pgfsetstrokecolor{currentstroke}%
\pgfsetdash{}{0pt}%
\pgfpathmoveto{\pgfqpoint{3.719264in}{2.134686in}}%
\pgfpathlineto{\pgfqpoint{3.732867in}{2.131714in}}%
\pgfpathlineto{\pgfqpoint{3.746475in}{2.128824in}}%
\pgfpathlineto{\pgfqpoint{3.760090in}{2.126017in}}%
\pgfpathlineto{\pgfqpoint{3.773711in}{2.123292in}}%
\pgfpathlineto{\pgfqpoint{3.781725in}{2.132175in}}%
\pgfpathlineto{\pgfqpoint{3.789734in}{2.141056in}}%
\pgfpathlineto{\pgfqpoint{3.797736in}{2.149937in}}%
\pgfpathlineto{\pgfqpoint{3.805733in}{2.158818in}}%
\pgfpathlineto{\pgfqpoint{3.792123in}{2.161583in}}%
\pgfpathlineto{\pgfqpoint{3.778520in}{2.164429in}}%
\pgfpathlineto{\pgfqpoint{3.764922in}{2.167358in}}%
\pgfpathlineto{\pgfqpoint{3.751330in}{2.170370in}}%
\pgfpathlineto{\pgfqpoint{3.743322in}{2.161441in}}%
\pgfpathlineto{\pgfqpoint{3.735309in}{2.152518in}}%
\pgfpathlineto{\pgfqpoint{3.727289in}{2.143600in}}%
\pgfpathlineto{\pgfqpoint{3.719264in}{2.134686in}}%
\pgfpathclose%
\pgfusepath{fill}%
\end{pgfscope}%
\begin{pgfscope}%
\pgfpathrectangle{\pgfqpoint{1.150000in}{0.150000in}}{\pgfqpoint{5.700000in}{5.700000in}}%
\pgfusepath{clip}%
\pgfsetbuttcap%
\pgfsetroundjoin%
\definecolor{currentfill}{rgb}{0.277018,0.050344,0.375715}%
\pgfsetfillcolor{currentfill}%
\pgfsetfillopacity{0.700000}%
\pgfsetlinewidth{0.000000pt}%
\definecolor{currentstroke}{rgb}{0.000000,0.000000,0.000000}%
\pgfsetstrokecolor{currentstroke}%
\pgfsetdash{}{0pt}%
\pgfpathmoveto{\pgfqpoint{2.981954in}{2.119282in}}%
\pgfpathlineto{\pgfqpoint{2.995457in}{2.112465in}}%
\pgfpathlineto{\pgfqpoint{3.008963in}{2.105750in}}%
\pgfpathlineto{\pgfqpoint{3.022472in}{2.099135in}}%
\pgfpathlineto{\pgfqpoint{3.035983in}{2.092621in}}%
\pgfpathlineto{\pgfqpoint{3.044272in}{2.100750in}}%
\pgfpathlineto{\pgfqpoint{3.052554in}{2.108926in}}%
\pgfpathlineto{\pgfqpoint{3.060828in}{2.117150in}}%
\pgfpathlineto{\pgfqpoint{3.069096in}{2.125419in}}%
\pgfpathlineto{\pgfqpoint{3.055600in}{2.131850in}}%
\pgfpathlineto{\pgfqpoint{3.042106in}{2.138381in}}%
\pgfpathlineto{\pgfqpoint{3.028616in}{2.145012in}}%
\pgfpathlineto{\pgfqpoint{3.015128in}{2.151746in}}%
\pgfpathlineto{\pgfqpoint{3.006845in}{2.143552in}}%
\pgfpathlineto{\pgfqpoint{2.998555in}{2.135410in}}%
\pgfpathlineto{\pgfqpoint{2.990258in}{2.127320in}}%
\pgfpathlineto{\pgfqpoint{2.981954in}{2.119282in}}%
\pgfpathclose%
\pgfusepath{fill}%
\end{pgfscope}%
\begin{pgfscope}%
\pgfpathrectangle{\pgfqpoint{1.150000in}{0.150000in}}{\pgfqpoint{5.700000in}{5.700000in}}%
\pgfusepath{clip}%
\pgfsetbuttcap%
\pgfsetroundjoin%
\definecolor{currentfill}{rgb}{0.276022,0.044167,0.370164}%
\pgfsetfillcolor{currentfill}%
\pgfsetfillopacity{0.700000}%
\pgfsetlinewidth{0.000000pt}%
\definecolor{currentstroke}{rgb}{0.000000,0.000000,0.000000}%
\pgfsetstrokecolor{currentstroke}%
\pgfsetdash{}{0pt}%
\pgfpathmoveto{\pgfqpoint{3.123110in}{2.100688in}}%
\pgfpathlineto{\pgfqpoint{3.136622in}{2.094750in}}%
\pgfpathlineto{\pgfqpoint{3.150137in}{2.088909in}}%
\pgfpathlineto{\pgfqpoint{3.163655in}{2.083164in}}%
\pgfpathlineto{\pgfqpoint{3.177177in}{2.077515in}}%
\pgfpathlineto{\pgfqpoint{3.185409in}{2.085971in}}%
\pgfpathlineto{\pgfqpoint{3.193634in}{2.094462in}}%
\pgfpathlineto{\pgfqpoint{3.201852in}{2.102987in}}%
\pgfpathlineto{\pgfqpoint{3.210064in}{2.111545in}}%
\pgfpathlineto{\pgfqpoint{3.196556in}{2.117132in}}%
\pgfpathlineto{\pgfqpoint{3.183052in}{2.122814in}}%
\pgfpathlineto{\pgfqpoint{3.169551in}{2.128592in}}%
\pgfpathlineto{\pgfqpoint{3.156053in}{2.134467in}}%
\pgfpathlineto{\pgfqpoint{3.147828in}{2.125964in}}%
\pgfpathlineto{\pgfqpoint{3.139595in}{2.117500in}}%
\pgfpathlineto{\pgfqpoint{3.131356in}{2.109074in}}%
\pgfpathlineto{\pgfqpoint{3.123110in}{2.100688in}}%
\pgfpathclose%
\pgfusepath{fill}%
\end{pgfscope}%
\begin{pgfscope}%
\pgfpathrectangle{\pgfqpoint{1.150000in}{0.150000in}}{\pgfqpoint{5.700000in}{5.700000in}}%
\pgfusepath{clip}%
\pgfsetbuttcap%
\pgfsetroundjoin%
\definecolor{currentfill}{rgb}{0.277134,0.185228,0.489898}%
\pgfsetfillcolor{currentfill}%
\pgfsetfillopacity{0.700000}%
\pgfsetlinewidth{0.000000pt}%
\definecolor{currentstroke}{rgb}{0.000000,0.000000,0.000000}%
\pgfsetstrokecolor{currentstroke}%
\pgfsetdash{}{0pt}%
\pgfpathmoveto{\pgfqpoint{4.660724in}{2.364075in}}%
\pgfpathlineto{\pgfqpoint{4.674586in}{2.363686in}}%
\pgfpathlineto{\pgfqpoint{4.688457in}{2.363369in}}%
\pgfpathlineto{\pgfqpoint{4.702338in}{2.363123in}}%
\pgfpathlineto{\pgfqpoint{4.716227in}{2.362949in}}%
\pgfpathlineto{\pgfqpoint{4.723894in}{2.370477in}}%
\pgfpathlineto{\pgfqpoint{4.731557in}{2.378035in}}%
\pgfpathlineto{\pgfqpoint{4.739214in}{2.385628in}}%
\pgfpathlineto{\pgfqpoint{4.746865in}{2.393259in}}%
\pgfpathlineto{\pgfqpoint{4.732991in}{2.393658in}}%
\pgfpathlineto{\pgfqpoint{4.719125in}{2.394128in}}%
\pgfpathlineto{\pgfqpoint{4.705268in}{2.394670in}}%
\pgfpathlineto{\pgfqpoint{4.691420in}{2.395284in}}%
\pgfpathlineto{\pgfqpoint{4.683754in}{2.387421in}}%
\pgfpathlineto{\pgfqpoint{4.676082in}{2.379601in}}%
\pgfpathlineto{\pgfqpoint{4.668406in}{2.371820in}}%
\pgfpathlineto{\pgfqpoint{4.660724in}{2.364075in}}%
\pgfpathclose%
\pgfusepath{fill}%
\end{pgfscope}%
\begin{pgfscope}%
\pgfpathrectangle{\pgfqpoint{1.150000in}{0.150000in}}{\pgfqpoint{5.700000in}{5.700000in}}%
\pgfusepath{clip}%
\pgfsetbuttcap%
\pgfsetroundjoin%
\definecolor{currentfill}{rgb}{0.281887,0.150881,0.465405}%
\pgfsetfillcolor{currentfill}%
\pgfsetfillopacity{0.700000}%
\pgfsetlinewidth{0.000000pt}%
\definecolor{currentstroke}{rgb}{0.000000,0.000000,0.000000}%
\pgfsetstrokecolor{currentstroke}%
\pgfsetdash{}{0pt}%
\pgfpathmoveto{\pgfqpoint{2.448464in}{2.315861in}}%
\pgfpathlineto{\pgfqpoint{2.462006in}{2.304990in}}%
\pgfpathlineto{\pgfqpoint{2.475547in}{2.294248in}}%
\pgfpathlineto{\pgfqpoint{2.489087in}{2.283635in}}%
\pgfpathlineto{\pgfqpoint{2.502625in}{2.273150in}}%
\pgfpathlineto{\pgfqpoint{2.511165in}{2.279371in}}%
\pgfpathlineto{\pgfqpoint{2.519695in}{2.285699in}}%
\pgfpathlineto{\pgfqpoint{2.528215in}{2.292131in}}%
\pgfpathlineto{\pgfqpoint{2.536725in}{2.298666in}}%
\pgfpathlineto{\pgfqpoint{2.523209in}{2.309003in}}%
\pgfpathlineto{\pgfqpoint{2.509691in}{2.319467in}}%
\pgfpathlineto{\pgfqpoint{2.496173in}{2.330060in}}%
\pgfpathlineto{\pgfqpoint{2.482653in}{2.340782in}}%
\pgfpathlineto{\pgfqpoint{2.474121in}{2.334388in}}%
\pgfpathlineto{\pgfqpoint{2.465579in}{2.328102in}}%
\pgfpathlineto{\pgfqpoint{2.457027in}{2.321926in}}%
\pgfpathlineto{\pgfqpoint{2.448464in}{2.315861in}}%
\pgfpathclose%
\pgfusepath{fill}%
\end{pgfscope}%
\begin{pgfscope}%
\pgfpathrectangle{\pgfqpoint{1.150000in}{0.150000in}}{\pgfqpoint{5.700000in}{5.700000in}}%
\pgfusepath{clip}%
\pgfsetbuttcap%
\pgfsetroundjoin%
\definecolor{currentfill}{rgb}{0.279566,0.067836,0.391917}%
\pgfsetfillcolor{currentfill}%
\pgfsetfillopacity{0.700000}%
\pgfsetlinewidth{0.000000pt}%
\definecolor{currentstroke}{rgb}{0.000000,0.000000,0.000000}%
\pgfsetstrokecolor{currentstroke}%
\pgfsetdash{}{0pt}%
\pgfpathmoveto{\pgfqpoint{2.840578in}{2.146798in}}%
\pgfpathlineto{\pgfqpoint{2.854082in}{2.139039in}}%
\pgfpathlineto{\pgfqpoint{2.867588in}{2.131386in}}%
\pgfpathlineto{\pgfqpoint{2.881095in}{2.123840in}}%
\pgfpathlineto{\pgfqpoint{2.894604in}{2.116400in}}%
\pgfpathlineto{\pgfqpoint{2.902956in}{2.124103in}}%
\pgfpathlineto{\pgfqpoint{2.911299in}{2.131868in}}%
\pgfpathlineto{\pgfqpoint{2.919635in}{2.139695in}}%
\pgfpathlineto{\pgfqpoint{2.927963in}{2.147581in}}%
\pgfpathlineto{\pgfqpoint{2.914470in}{2.154917in}}%
\pgfpathlineto{\pgfqpoint{2.900980in}{2.162358in}}%
\pgfpathlineto{\pgfqpoint{2.887491in}{2.169906in}}%
\pgfpathlineto{\pgfqpoint{2.874004in}{2.177561in}}%
\pgfpathlineto{\pgfqpoint{2.865659in}{2.169772in}}%
\pgfpathlineto{\pgfqpoint{2.857307in}{2.162047in}}%
\pgfpathlineto{\pgfqpoint{2.848946in}{2.154389in}}%
\pgfpathlineto{\pgfqpoint{2.840578in}{2.146798in}}%
\pgfpathclose%
\pgfusepath{fill}%
\end{pgfscope}%
\begin{pgfscope}%
\pgfpathrectangle{\pgfqpoint{1.150000in}{0.150000in}}{\pgfqpoint{5.700000in}{5.700000in}}%
\pgfusepath{clip}%
\pgfsetbuttcap%
\pgfsetroundjoin%
\definecolor{currentfill}{rgb}{0.274952,0.037752,0.364543}%
\pgfsetfillcolor{currentfill}%
\pgfsetfillopacity{0.700000}%
\pgfsetlinewidth{0.000000pt}%
\definecolor{currentstroke}{rgb}{0.000000,0.000000,0.000000}%
\pgfsetstrokecolor{currentstroke}%
\pgfsetdash{}{0pt}%
\pgfpathmoveto{\pgfqpoint{3.264133in}{2.090144in}}%
\pgfpathlineto{\pgfqpoint{3.277660in}{2.085028in}}%
\pgfpathlineto{\pgfqpoint{3.291192in}{2.080004in}}%
\pgfpathlineto{\pgfqpoint{3.304727in}{2.075071in}}%
\pgfpathlineto{\pgfqpoint{3.318267in}{2.070231in}}%
\pgfpathlineto{\pgfqpoint{3.326446in}{2.078922in}}%
\pgfpathlineto{\pgfqpoint{3.334618in}{2.087637in}}%
\pgfpathlineto{\pgfqpoint{3.342785in}{2.096373in}}%
\pgfpathlineto{\pgfqpoint{3.350945in}{2.105132in}}%
\pgfpathlineto{\pgfqpoint{3.337418in}{2.109931in}}%
\pgfpathlineto{\pgfqpoint{3.323895in}{2.114821in}}%
\pgfpathlineto{\pgfqpoint{3.310377in}{2.119802in}}%
\pgfpathlineto{\pgfqpoint{3.296863in}{2.124876in}}%
\pgfpathlineto{\pgfqpoint{3.288690in}{2.116152in}}%
\pgfpathlineto{\pgfqpoint{3.280511in}{2.107455in}}%
\pgfpathlineto{\pgfqpoint{3.272325in}{2.098786in}}%
\pgfpathlineto{\pgfqpoint{3.264133in}{2.090144in}}%
\pgfpathclose%
\pgfusepath{fill}%
\end{pgfscope}%
\begin{pgfscope}%
\pgfpathrectangle{\pgfqpoint{1.150000in}{0.150000in}}{\pgfqpoint{5.700000in}{5.700000in}}%
\pgfusepath{clip}%
\pgfsetbuttcap%
\pgfsetroundjoin%
\definecolor{currentfill}{rgb}{0.248629,0.278775,0.534556}%
\pgfsetfillcolor{currentfill}%
\pgfsetfillopacity{0.700000}%
\pgfsetlinewidth{0.000000pt}%
\definecolor{currentstroke}{rgb}{0.000000,0.000000,0.000000}%
\pgfsetstrokecolor{currentstroke}%
\pgfsetdash{}{0pt}%
\pgfpathmoveto{\pgfqpoint{5.374883in}{2.567665in}}%
\pgfpathlineto{\pgfqpoint{5.388967in}{2.567772in}}%
\pgfpathlineto{\pgfqpoint{5.403061in}{2.567946in}}%
\pgfpathlineto{\pgfqpoint{5.417165in}{2.568188in}}%
\pgfpathlineto{\pgfqpoint{5.431280in}{2.568497in}}%
\pgfpathlineto{\pgfqpoint{5.438666in}{2.575470in}}%
\pgfpathlineto{\pgfqpoint{5.446051in}{2.582578in}}%
\pgfpathlineto{\pgfqpoint{5.453433in}{2.589824in}}%
\pgfpathlineto{\pgfqpoint{5.460814in}{2.597217in}}%
\pgfpathlineto{\pgfqpoint{5.446722in}{2.597277in}}%
\pgfpathlineto{\pgfqpoint{5.432639in}{2.597404in}}%
\pgfpathlineto{\pgfqpoint{5.418567in}{2.597598in}}%
\pgfpathlineto{\pgfqpoint{5.404505in}{2.597860in}}%
\pgfpathlineto{\pgfqpoint{5.397102in}{2.590091in}}%
\pgfpathlineto{\pgfqpoint{5.389698in}{2.582474in}}%
\pgfpathlineto{\pgfqpoint{5.382291in}{2.575000in}}%
\pgfpathlineto{\pgfqpoint{5.374883in}{2.567665in}}%
\pgfpathclose%
\pgfusepath{fill}%
\end{pgfscope}%
\begin{pgfscope}%
\pgfpathrectangle{\pgfqpoint{1.150000in}{0.150000in}}{\pgfqpoint{5.700000in}{5.700000in}}%
\pgfusepath{clip}%
\pgfsetbuttcap%
\pgfsetroundjoin%
\definecolor{currentfill}{rgb}{0.225863,0.330805,0.547314}%
\pgfsetfillcolor{currentfill}%
\pgfsetfillopacity{0.700000}%
\pgfsetlinewidth{0.000000pt}%
\definecolor{currentstroke}{rgb}{0.000000,0.000000,0.000000}%
\pgfsetstrokecolor{currentstroke}%
\pgfsetdash{}{0pt}%
\pgfpathmoveto{\pgfqpoint{5.775074in}{2.688340in}}%
\pgfpathlineto{\pgfqpoint{5.789268in}{2.688182in}}%
\pgfpathlineto{\pgfqpoint{5.803473in}{2.688089in}}%
\pgfpathlineto{\pgfqpoint{5.817688in}{2.688062in}}%
\pgfpathlineto{\pgfqpoint{5.831914in}{2.688100in}}%
\pgfpathlineto{\pgfqpoint{5.839172in}{2.695828in}}%
\pgfpathlineto{\pgfqpoint{5.846432in}{2.703779in}}%
\pgfpathlineto{\pgfqpoint{5.853695in}{2.711961in}}%
\pgfpathlineto{\pgfqpoint{5.860961in}{2.720383in}}%
\pgfpathlineto{\pgfqpoint{5.846762in}{2.720794in}}%
\pgfpathlineto{\pgfqpoint{5.832574in}{2.721271in}}%
\pgfpathlineto{\pgfqpoint{5.818396in}{2.721813in}}%
\pgfpathlineto{\pgfqpoint{5.804228in}{2.722420in}}%
\pgfpathlineto{\pgfqpoint{5.796936in}{2.713542in}}%
\pgfpathlineto{\pgfqpoint{5.789646in}{2.704908in}}%
\pgfpathlineto{\pgfqpoint{5.782359in}{2.696510in}}%
\pgfpathlineto{\pgfqpoint{5.775074in}{2.688340in}}%
\pgfpathclose%
\pgfusepath{fill}%
\end{pgfscope}%
\begin{pgfscope}%
\pgfpathrectangle{\pgfqpoint{1.150000in}{0.150000in}}{\pgfqpoint{5.700000in}{5.700000in}}%
\pgfusepath{clip}%
\pgfsetbuttcap%
\pgfsetroundjoin%
\definecolor{currentfill}{rgb}{0.281924,0.089666,0.412415}%
\pgfsetfillcolor{currentfill}%
\pgfsetfillopacity{0.700000}%
\pgfsetlinewidth{0.000000pt}%
\definecolor{currentstroke}{rgb}{0.000000,0.000000,0.000000}%
\pgfsetstrokecolor{currentstroke}%
\pgfsetdash{}{0pt}%
\pgfpathmoveto{\pgfqpoint{3.946694in}{2.174698in}}%
\pgfpathlineto{\pgfqpoint{3.960353in}{2.172598in}}%
\pgfpathlineto{\pgfqpoint{3.974018in}{2.170577in}}%
\pgfpathlineto{\pgfqpoint{3.987691in}{2.168635in}}%
\pgfpathlineto{\pgfqpoint{4.001372in}{2.166771in}}%
\pgfpathlineto{\pgfqpoint{4.009308in}{2.175459in}}%
\pgfpathlineto{\pgfqpoint{4.017238in}{2.184140in}}%
\pgfpathlineto{\pgfqpoint{4.025163in}{2.192818in}}%
\pgfpathlineto{\pgfqpoint{4.033082in}{2.201494in}}%
\pgfpathlineto{\pgfqpoint{4.019413in}{2.203438in}}%
\pgfpathlineto{\pgfqpoint{4.005751in}{2.205461in}}%
\pgfpathlineto{\pgfqpoint{3.992096in}{2.207562in}}%
\pgfpathlineto{\pgfqpoint{3.978448in}{2.209743in}}%
\pgfpathlineto{\pgfqpoint{3.970518in}{2.200979in}}%
\pgfpathlineto{\pgfqpoint{3.962582in}{2.192218in}}%
\pgfpathlineto{\pgfqpoint{3.954641in}{2.183458in}}%
\pgfpathlineto{\pgfqpoint{3.946694in}{2.174698in}}%
\pgfpathclose%
\pgfusepath{fill}%
\end{pgfscope}%
\begin{pgfscope}%
\pgfpathrectangle{\pgfqpoint{1.150000in}{0.150000in}}{\pgfqpoint{5.700000in}{5.700000in}}%
\pgfusepath{clip}%
\pgfsetbuttcap%
\pgfsetroundjoin%
\definecolor{currentfill}{rgb}{0.283072,0.130895,0.449241}%
\pgfsetfillcolor{currentfill}%
\pgfsetfillopacity{0.700000}%
\pgfsetlinewidth{0.000000pt}%
\definecolor{currentstroke}{rgb}{0.000000,0.000000,0.000000}%
\pgfsetstrokecolor{currentstroke}%
\pgfsetdash{}{0pt}%
\pgfpathmoveto{\pgfqpoint{4.260571in}{2.251117in}}%
\pgfpathlineto{\pgfqpoint{4.274315in}{2.249950in}}%
\pgfpathlineto{\pgfqpoint{4.288068in}{2.248858in}}%
\pgfpathlineto{\pgfqpoint{4.301829in}{2.247841in}}%
\pgfpathlineto{\pgfqpoint{4.315597in}{2.246899in}}%
\pgfpathlineto{\pgfqpoint{4.323420in}{2.255117in}}%
\pgfpathlineto{\pgfqpoint{4.331237in}{2.263335in}}%
\pgfpathlineto{\pgfqpoint{4.339048in}{2.271557in}}%
\pgfpathlineto{\pgfqpoint{4.346854in}{2.279785in}}%
\pgfpathlineto{\pgfqpoint{4.333097in}{2.280870in}}%
\pgfpathlineto{\pgfqpoint{4.319348in}{2.282029in}}%
\pgfpathlineto{\pgfqpoint{4.305608in}{2.283263in}}%
\pgfpathlineto{\pgfqpoint{4.291875in}{2.284572in}}%
\pgfpathlineto{\pgfqpoint{4.284057in}{2.276194in}}%
\pgfpathlineto{\pgfqpoint{4.276234in}{2.267828in}}%
\pgfpathlineto{\pgfqpoint{4.268405in}{2.259469in}}%
\pgfpathlineto{\pgfqpoint{4.260571in}{2.251117in}}%
\pgfpathclose%
\pgfusepath{fill}%
\end{pgfscope}%
\begin{pgfscope}%
\pgfpathrectangle{\pgfqpoint{1.150000in}{0.150000in}}{\pgfqpoint{5.700000in}{5.700000in}}%
\pgfusepath{clip}%
\pgfsetbuttcap%
\pgfsetroundjoin%
\definecolor{currentfill}{rgb}{0.266580,0.228262,0.514349}%
\pgfsetfillcolor{currentfill}%
\pgfsetfillopacity{0.700000}%
\pgfsetlinewidth{0.000000pt}%
\definecolor{currentstroke}{rgb}{0.000000,0.000000,0.000000}%
\pgfsetstrokecolor{currentstroke}%
\pgfsetdash{}{0pt}%
\pgfpathmoveto{\pgfqpoint{4.974739in}{2.450827in}}%
\pgfpathlineto{\pgfqpoint{4.988703in}{2.450833in}}%
\pgfpathlineto{\pgfqpoint{5.002676in}{2.450909in}}%
\pgfpathlineto{\pgfqpoint{5.016659in}{2.451055in}}%
\pgfpathlineto{\pgfqpoint{5.030651in}{2.451270in}}%
\pgfpathlineto{\pgfqpoint{5.038194in}{2.458336in}}%
\pgfpathlineto{\pgfqpoint{5.045732in}{2.465467in}}%
\pgfpathlineto{\pgfqpoint{5.053266in}{2.472669in}}%
\pgfpathlineto{\pgfqpoint{5.060796in}{2.479947in}}%
\pgfpathlineto{\pgfqpoint{5.046821in}{2.480019in}}%
\pgfpathlineto{\pgfqpoint{5.032856in}{2.480161in}}%
\pgfpathlineto{\pgfqpoint{5.018900in}{2.480371in}}%
\pgfpathlineto{\pgfqpoint{5.004954in}{2.480652in}}%
\pgfpathlineto{\pgfqpoint{4.997406in}{2.473080in}}%
\pgfpathlineto{\pgfqpoint{4.989855in}{2.465589in}}%
\pgfpathlineto{\pgfqpoint{4.982299in}{2.458173in}}%
\pgfpathlineto{\pgfqpoint{4.974739in}{2.450827in}}%
\pgfpathclose%
\pgfusepath{fill}%
\end{pgfscope}%
\begin{pgfscope}%
\pgfpathrectangle{\pgfqpoint{1.150000in}{0.150000in}}{\pgfqpoint{5.700000in}{5.700000in}}%
\pgfusepath{clip}%
\pgfsetbuttcap%
\pgfsetroundjoin%
\definecolor{currentfill}{rgb}{0.278826,0.175490,0.483397}%
\pgfsetfillcolor{currentfill}%
\pgfsetfillopacity{0.700000}%
\pgfsetlinewidth{0.000000pt}%
\definecolor{currentstroke}{rgb}{0.000000,0.000000,0.000000}%
\pgfsetstrokecolor{currentstroke}%
\pgfsetdash{}{0pt}%
\pgfpathmoveto{\pgfqpoint{4.574526in}{2.334825in}}%
\pgfpathlineto{\pgfqpoint{4.588367in}{2.334352in}}%
\pgfpathlineto{\pgfqpoint{4.602216in}{2.333951in}}%
\pgfpathlineto{\pgfqpoint{4.616075in}{2.333622in}}%
\pgfpathlineto{\pgfqpoint{4.629942in}{2.333366in}}%
\pgfpathlineto{\pgfqpoint{4.637646in}{2.341010in}}%
\pgfpathlineto{\pgfqpoint{4.645344in}{2.348674in}}%
\pgfpathlineto{\pgfqpoint{4.653037in}{2.356361in}}%
\pgfpathlineto{\pgfqpoint{4.660724in}{2.364075in}}%
\pgfpathlineto{\pgfqpoint{4.646870in}{2.364536in}}%
\pgfpathlineto{\pgfqpoint{4.633026in}{2.365069in}}%
\pgfpathlineto{\pgfqpoint{4.619190in}{2.365675in}}%
\pgfpathlineto{\pgfqpoint{4.605363in}{2.366352in}}%
\pgfpathlineto{\pgfqpoint{4.597662in}{2.358426in}}%
\pgfpathlineto{\pgfqpoint{4.589955in}{2.350532in}}%
\pgfpathlineto{\pgfqpoint{4.582243in}{2.342667in}}%
\pgfpathlineto{\pgfqpoint{4.574526in}{2.334825in}}%
\pgfpathclose%
\pgfusepath{fill}%
\end{pgfscope}%
\begin{pgfscope}%
\pgfpathrectangle{\pgfqpoint{1.150000in}{0.150000in}}{\pgfqpoint{5.700000in}{5.700000in}}%
\pgfusepath{clip}%
\pgfsetbuttcap%
\pgfsetroundjoin%
\definecolor{currentfill}{rgb}{0.277941,0.056324,0.381191}%
\pgfsetfillcolor{currentfill}%
\pgfsetfillopacity{0.700000}%
\pgfsetlinewidth{0.000000pt}%
\definecolor{currentstroke}{rgb}{0.000000,0.000000,0.000000}%
\pgfsetstrokecolor{currentstroke}%
\pgfsetdash{}{0pt}%
\pgfpathmoveto{\pgfqpoint{3.632712in}{2.111688in}}%
\pgfpathlineto{\pgfqpoint{3.646302in}{2.108401in}}%
\pgfpathlineto{\pgfqpoint{3.659898in}{2.105198in}}%
\pgfpathlineto{\pgfqpoint{3.673499in}{2.102079in}}%
\pgfpathlineto{\pgfqpoint{3.687107in}{2.099043in}}%
\pgfpathlineto{\pgfqpoint{3.695155in}{2.107953in}}%
\pgfpathlineto{\pgfqpoint{3.703197in}{2.116863in}}%
\pgfpathlineto{\pgfqpoint{3.711233in}{2.125774in}}%
\pgfpathlineto{\pgfqpoint{3.719264in}{2.134686in}}%
\pgfpathlineto{\pgfqpoint{3.705668in}{2.137741in}}%
\pgfpathlineto{\pgfqpoint{3.692078in}{2.140879in}}%
\pgfpathlineto{\pgfqpoint{3.678493in}{2.144101in}}%
\pgfpathlineto{\pgfqpoint{3.664914in}{2.147407in}}%
\pgfpathlineto{\pgfqpoint{3.656873in}{2.138468in}}%
\pgfpathlineto{\pgfqpoint{3.648825in}{2.129536in}}%
\pgfpathlineto{\pgfqpoint{3.640771in}{2.120610in}}%
\pgfpathlineto{\pgfqpoint{3.632712in}{2.111688in}}%
\pgfpathclose%
\pgfusepath{fill}%
\end{pgfscope}%
\begin{pgfscope}%
\pgfpathrectangle{\pgfqpoint{1.150000in}{0.150000in}}{\pgfqpoint{5.700000in}{5.700000in}}%
\pgfusepath{clip}%
\pgfsetbuttcap%
\pgfsetroundjoin%
\definecolor{currentfill}{rgb}{0.276022,0.044167,0.370164}%
\pgfsetfillcolor{currentfill}%
\pgfsetfillopacity{0.700000}%
\pgfsetlinewidth{0.000000pt}%
\definecolor{currentstroke}{rgb}{0.000000,0.000000,0.000000}%
\pgfsetstrokecolor{currentstroke}%
\pgfsetdash{}{0pt}%
\pgfpathmoveto{\pgfqpoint{3.405097in}{2.086844in}}%
\pgfpathlineto{\pgfqpoint{3.418647in}{2.082496in}}%
\pgfpathlineto{\pgfqpoint{3.432202in}{2.078237in}}%
\pgfpathlineto{\pgfqpoint{3.445761in}{2.074066in}}%
\pgfpathlineto{\pgfqpoint{3.459326in}{2.069983in}}%
\pgfpathlineto{\pgfqpoint{3.467455in}{2.078823in}}%
\pgfpathlineto{\pgfqpoint{3.475579in}{2.087675in}}%
\pgfpathlineto{\pgfqpoint{3.483696in}{2.096539in}}%
\pgfpathlineto{\pgfqpoint{3.491808in}{2.105416in}}%
\pgfpathlineto{\pgfqpoint{3.478255in}{2.109477in}}%
\pgfpathlineto{\pgfqpoint{3.464708in}{2.113627in}}%
\pgfpathlineto{\pgfqpoint{3.451165in}{2.117864in}}%
\pgfpathlineto{\pgfqpoint{3.437628in}{2.122190in}}%
\pgfpathlineto{\pgfqpoint{3.429504in}{2.113328in}}%
\pgfpathlineto{\pgfqpoint{3.421375in}{2.104483in}}%
\pgfpathlineto{\pgfqpoint{3.413239in}{2.095655in}}%
\pgfpathlineto{\pgfqpoint{3.405097in}{2.086844in}}%
\pgfpathclose%
\pgfusepath{fill}%
\end{pgfscope}%
\begin{pgfscope}%
\pgfpathrectangle{\pgfqpoint{1.150000in}{0.150000in}}{\pgfqpoint{5.700000in}{5.700000in}}%
\pgfusepath{clip}%
\pgfsetbuttcap%
\pgfsetroundjoin%
\definecolor{currentfill}{rgb}{0.281924,0.089666,0.412415}%
\pgfsetfillcolor{currentfill}%
\pgfsetfillopacity{0.700000}%
\pgfsetlinewidth{0.000000pt}%
\definecolor{currentstroke}{rgb}{0.000000,0.000000,0.000000}%
\pgfsetstrokecolor{currentstroke}%
\pgfsetdash{}{0pt}%
\pgfpathmoveto{\pgfqpoint{2.698886in}{2.184182in}}%
\pgfpathlineto{\pgfqpoint{2.712400in}{2.175410in}}%
\pgfpathlineto{\pgfqpoint{2.725915in}{2.166751in}}%
\pgfpathlineto{\pgfqpoint{2.739430in}{2.158205in}}%
\pgfpathlineto{\pgfqpoint{2.752946in}{2.149771in}}%
\pgfpathlineto{\pgfqpoint{2.761366in}{2.156943in}}%
\pgfpathlineto{\pgfqpoint{2.769778in}{2.164195in}}%
\pgfpathlineto{\pgfqpoint{2.778181in}{2.171523in}}%
\pgfpathlineto{\pgfqpoint{2.786576in}{2.178927in}}%
\pgfpathlineto{\pgfqpoint{2.773078in}{2.187235in}}%
\pgfpathlineto{\pgfqpoint{2.759581in}{2.195655in}}%
\pgfpathlineto{\pgfqpoint{2.746086in}{2.204188in}}%
\pgfpathlineto{\pgfqpoint{2.732591in}{2.212835in}}%
\pgfpathlineto{\pgfqpoint{2.724178in}{2.205550in}}%
\pgfpathlineto{\pgfqpoint{2.715756in}{2.198345in}}%
\pgfpathlineto{\pgfqpoint{2.707325in}{2.191222in}}%
\pgfpathlineto{\pgfqpoint{2.698886in}{2.184182in}}%
\pgfpathclose%
\pgfusepath{fill}%
\end{pgfscope}%
\begin{pgfscope}%
\pgfpathrectangle{\pgfqpoint{1.150000in}{0.150000in}}{\pgfqpoint{5.700000in}{5.700000in}}%
\pgfusepath{clip}%
\pgfsetbuttcap%
\pgfsetroundjoin%
\definecolor{currentfill}{rgb}{0.229739,0.322361,0.545706}%
\pgfsetfillcolor{currentfill}%
\pgfsetfillopacity{0.700000}%
\pgfsetlinewidth{0.000000pt}%
\definecolor{currentstroke}{rgb}{0.000000,0.000000,0.000000}%
\pgfsetstrokecolor{currentstroke}%
\pgfsetdash{}{0pt}%
\pgfpathmoveto{\pgfqpoint{5.689173in}{2.657362in}}%
\pgfpathlineto{\pgfqpoint{5.703351in}{2.657369in}}%
\pgfpathlineto{\pgfqpoint{5.717540in}{2.657443in}}%
\pgfpathlineto{\pgfqpoint{5.731740in}{2.657582in}}%
\pgfpathlineto{\pgfqpoint{5.745950in}{2.657788in}}%
\pgfpathlineto{\pgfqpoint{5.753229in}{2.665122in}}%
\pgfpathlineto{\pgfqpoint{5.760509in}{2.672654in}}%
\pgfpathlineto{\pgfqpoint{5.767791in}{2.680391in}}%
\pgfpathlineto{\pgfqpoint{5.775074in}{2.688340in}}%
\pgfpathlineto{\pgfqpoint{5.760890in}{2.688564in}}%
\pgfpathlineto{\pgfqpoint{5.746716in}{2.688854in}}%
\pgfpathlineto{\pgfqpoint{5.732553in}{2.689210in}}%
\pgfpathlineto{\pgfqpoint{5.718401in}{2.689632in}}%
\pgfpathlineto{\pgfqpoint{5.711092in}{2.681246in}}%
\pgfpathlineto{\pgfqpoint{5.703784in}{2.673077in}}%
\pgfpathlineto{\pgfqpoint{5.696478in}{2.665118in}}%
\pgfpathlineto{\pgfqpoint{5.689173in}{2.657362in}}%
\pgfpathclose%
\pgfusepath{fill}%
\end{pgfscope}%
\begin{pgfscope}%
\pgfpathrectangle{\pgfqpoint{1.150000in}{0.150000in}}{\pgfqpoint{5.700000in}{5.700000in}}%
\pgfusepath{clip}%
\pgfsetbuttcap%
\pgfsetroundjoin%
\definecolor{currentfill}{rgb}{0.252194,0.269783,0.531579}%
\pgfsetfillcolor{currentfill}%
\pgfsetfillopacity{0.700000}%
\pgfsetlinewidth{0.000000pt}%
\definecolor{currentstroke}{rgb}{0.000000,0.000000,0.000000}%
\pgfsetstrokecolor{currentstroke}%
\pgfsetdash{}{0pt}%
\pgfpathmoveto{\pgfqpoint{5.288905in}{2.538433in}}%
\pgfpathlineto{\pgfqpoint{5.302970in}{2.538617in}}%
\pgfpathlineto{\pgfqpoint{5.317045in}{2.538869in}}%
\pgfpathlineto{\pgfqpoint{5.331129in}{2.539189in}}%
\pgfpathlineto{\pgfqpoint{5.345224in}{2.539577in}}%
\pgfpathlineto{\pgfqpoint{5.352643in}{2.546424in}}%
\pgfpathlineto{\pgfqpoint{5.360059in}{2.553383in}}%
\pgfpathlineto{\pgfqpoint{5.367472in}{2.560461in}}%
\pgfpathlineto{\pgfqpoint{5.374883in}{2.567665in}}%
\pgfpathlineto{\pgfqpoint{5.360809in}{2.567626in}}%
\pgfpathlineto{\pgfqpoint{5.346745in}{2.567654in}}%
\pgfpathlineto{\pgfqpoint{5.332692in}{2.567750in}}%
\pgfpathlineto{\pgfqpoint{5.318648in}{2.567914in}}%
\pgfpathlineto{\pgfqpoint{5.311216in}{2.560355in}}%
\pgfpathlineto{\pgfqpoint{5.303782in}{2.552926in}}%
\pgfpathlineto{\pgfqpoint{5.296345in}{2.545620in}}%
\pgfpathlineto{\pgfqpoint{5.288905in}{2.538433in}}%
\pgfpathclose%
\pgfusepath{fill}%
\end{pgfscope}%
\begin{pgfscope}%
\pgfpathrectangle{\pgfqpoint{1.150000in}{0.150000in}}{\pgfqpoint{5.700000in}{5.700000in}}%
\pgfusepath{clip}%
\pgfsetbuttcap%
\pgfsetroundjoin%
\definecolor{currentfill}{rgb}{0.282884,0.135920,0.453427}%
\pgfsetfillcolor{currentfill}%
\pgfsetfillopacity{0.700000}%
\pgfsetlinewidth{0.000000pt}%
\definecolor{currentstroke}{rgb}{0.000000,0.000000,0.000000}%
\pgfsetstrokecolor{currentstroke}%
\pgfsetdash{}{0pt}%
\pgfpathmoveto{\pgfqpoint{2.502625in}{2.273150in}}%
\pgfpathlineto{\pgfqpoint{2.516162in}{2.262791in}}%
\pgfpathlineto{\pgfqpoint{2.529699in}{2.252558in}}%
\pgfpathlineto{\pgfqpoint{2.543234in}{2.242449in}}%
\pgfpathlineto{\pgfqpoint{2.556769in}{2.232463in}}%
\pgfpathlineto{\pgfqpoint{2.565287in}{2.238840in}}%
\pgfpathlineto{\pgfqpoint{2.573796in}{2.245318in}}%
\pgfpathlineto{\pgfqpoint{2.582294in}{2.251895in}}%
\pgfpathlineto{\pgfqpoint{2.590783in}{2.258570in}}%
\pgfpathlineto{\pgfqpoint{2.577270in}{2.268408in}}%
\pgfpathlineto{\pgfqpoint{2.563756in}{2.278370in}}%
\pgfpathlineto{\pgfqpoint{2.550241in}{2.288455in}}%
\pgfpathlineto{\pgfqpoint{2.536725in}{2.298666in}}%
\pgfpathlineto{\pgfqpoint{2.528215in}{2.292131in}}%
\pgfpathlineto{\pgfqpoint{2.519695in}{2.285699in}}%
\pgfpathlineto{\pgfqpoint{2.511165in}{2.279371in}}%
\pgfpathlineto{\pgfqpoint{2.502625in}{2.273150in}}%
\pgfpathclose%
\pgfusepath{fill}%
\end{pgfscope}%
\begin{pgfscope}%
\pgfpathrectangle{\pgfqpoint{1.150000in}{0.150000in}}{\pgfqpoint{5.700000in}{5.700000in}}%
\pgfusepath{clip}%
\pgfsetbuttcap%
\pgfsetroundjoin%
\definecolor{currentfill}{rgb}{0.283229,0.120777,0.440584}%
\pgfsetfillcolor{currentfill}%
\pgfsetfillopacity{0.700000}%
\pgfsetlinewidth{0.000000pt}%
\definecolor{currentstroke}{rgb}{0.000000,0.000000,0.000000}%
\pgfsetstrokecolor{currentstroke}%
\pgfsetdash{}{0pt}%
\pgfpathmoveto{\pgfqpoint{4.174230in}{2.222655in}}%
\pgfpathlineto{\pgfqpoint{4.187955in}{2.221306in}}%
\pgfpathlineto{\pgfqpoint{4.201688in}{2.220033in}}%
\pgfpathlineto{\pgfqpoint{4.215428in}{2.218836in}}%
\pgfpathlineto{\pgfqpoint{4.229176in}{2.217715in}}%
\pgfpathlineto{\pgfqpoint{4.237033in}{2.226070in}}%
\pgfpathlineto{\pgfqpoint{4.244885in}{2.234420in}}%
\pgfpathlineto{\pgfqpoint{4.252731in}{2.242768in}}%
\pgfpathlineto{\pgfqpoint{4.260571in}{2.251117in}}%
\pgfpathlineto{\pgfqpoint{4.246834in}{2.252360in}}%
\pgfpathlineto{\pgfqpoint{4.233105in}{2.253679in}}%
\pgfpathlineto{\pgfqpoint{4.219384in}{2.255073in}}%
\pgfpathlineto{\pgfqpoint{4.205671in}{2.256544in}}%
\pgfpathlineto{\pgfqpoint{4.197819in}{2.248066in}}%
\pgfpathlineto{\pgfqpoint{4.189962in}{2.239593in}}%
\pgfpathlineto{\pgfqpoint{4.182099in}{2.231124in}}%
\pgfpathlineto{\pgfqpoint{4.174230in}{2.222655in}}%
\pgfpathclose%
\pgfusepath{fill}%
\end{pgfscope}%
\begin{pgfscope}%
\pgfpathrectangle{\pgfqpoint{1.150000in}{0.150000in}}{\pgfqpoint{5.700000in}{5.700000in}}%
\pgfusepath{clip}%
\pgfsetbuttcap%
\pgfsetroundjoin%
\definecolor{currentfill}{rgb}{0.280894,0.078907,0.402329}%
\pgfsetfillcolor{currentfill}%
\pgfsetfillopacity{0.700000}%
\pgfsetlinewidth{0.000000pt}%
\definecolor{currentstroke}{rgb}{0.000000,0.000000,0.000000}%
\pgfsetstrokecolor{currentstroke}%
\pgfsetdash{}{0pt}%
\pgfpathmoveto{\pgfqpoint{3.860238in}{2.148574in}}%
\pgfpathlineto{\pgfqpoint{3.873881in}{2.146215in}}%
\pgfpathlineto{\pgfqpoint{3.887530in}{2.143936in}}%
\pgfpathlineto{\pgfqpoint{3.901186in}{2.141737in}}%
\pgfpathlineto{\pgfqpoint{3.914849in}{2.139618in}}%
\pgfpathlineto{\pgfqpoint{3.922819in}{2.148397in}}%
\pgfpathlineto{\pgfqpoint{3.930783in}{2.157169in}}%
\pgfpathlineto{\pgfqpoint{3.938741in}{2.165935in}}%
\pgfpathlineto{\pgfqpoint{3.946694in}{2.174698in}}%
\pgfpathlineto{\pgfqpoint{3.933042in}{2.176877in}}%
\pgfpathlineto{\pgfqpoint{3.919396in}{2.179136in}}%
\pgfpathlineto{\pgfqpoint{3.905758in}{2.181475in}}%
\pgfpathlineto{\pgfqpoint{3.892126in}{2.183894in}}%
\pgfpathlineto{\pgfqpoint{3.884163in}{2.175064in}}%
\pgfpathlineto{\pgfqpoint{3.876193in}{2.166235in}}%
\pgfpathlineto{\pgfqpoint{3.868219in}{2.157405in}}%
\pgfpathlineto{\pgfqpoint{3.860238in}{2.148574in}}%
\pgfpathclose%
\pgfusepath{fill}%
\end{pgfscope}%
\begin{pgfscope}%
\pgfpathrectangle{\pgfqpoint{1.150000in}{0.150000in}}{\pgfqpoint{5.700000in}{5.700000in}}%
\pgfusepath{clip}%
\pgfsetbuttcap%
\pgfsetroundjoin%
\definecolor{currentfill}{rgb}{0.269308,0.218818,0.509577}%
\pgfsetfillcolor{currentfill}%
\pgfsetfillopacity{0.700000}%
\pgfsetlinewidth{0.000000pt}%
\definecolor{currentstroke}{rgb}{0.000000,0.000000,0.000000}%
\pgfsetstrokecolor{currentstroke}%
\pgfsetdash{}{0pt}%
\pgfpathmoveto{\pgfqpoint{4.888626in}{2.421656in}}%
\pgfpathlineto{\pgfqpoint{4.902568in}{2.421649in}}%
\pgfpathlineto{\pgfqpoint{4.916519in}{2.421711in}}%
\pgfpathlineto{\pgfqpoint{4.930481in}{2.421844in}}%
\pgfpathlineto{\pgfqpoint{4.944451in}{2.422047in}}%
\pgfpathlineto{\pgfqpoint{4.952031in}{2.429162in}}%
\pgfpathlineto{\pgfqpoint{4.959605in}{2.436327in}}%
\pgfpathlineto{\pgfqpoint{4.967174in}{2.443547in}}%
\pgfpathlineto{\pgfqpoint{4.974739in}{2.450827in}}%
\pgfpathlineto{\pgfqpoint{4.960785in}{2.450891in}}%
\pgfpathlineto{\pgfqpoint{4.946840in}{2.451025in}}%
\pgfpathlineto{\pgfqpoint{4.932905in}{2.451228in}}%
\pgfpathlineto{\pgfqpoint{4.918979in}{2.451501in}}%
\pgfpathlineto{\pgfqpoint{4.911398in}{2.443947in}}%
\pgfpathlineto{\pgfqpoint{4.903812in}{2.436458in}}%
\pgfpathlineto{\pgfqpoint{4.896221in}{2.429029in}}%
\pgfpathlineto{\pgfqpoint{4.888626in}{2.421656in}}%
\pgfpathclose%
\pgfusepath{fill}%
\end{pgfscope}%
\begin{pgfscope}%
\pgfpathrectangle{\pgfqpoint{1.150000in}{0.150000in}}{\pgfqpoint{5.700000in}{5.700000in}}%
\pgfusepath{clip}%
\pgfsetbuttcap%
\pgfsetroundjoin%
\definecolor{currentfill}{rgb}{0.276022,0.044167,0.370164}%
\pgfsetfillcolor{currentfill}%
\pgfsetfillopacity{0.700000}%
\pgfsetlinewidth{0.000000pt}%
\definecolor{currentstroke}{rgb}{0.000000,0.000000,0.000000}%
\pgfsetstrokecolor{currentstroke}%
\pgfsetdash{}{0pt}%
\pgfpathmoveto{\pgfqpoint{3.035983in}{2.092621in}}%
\pgfpathlineto{\pgfqpoint{3.049498in}{2.086207in}}%
\pgfpathlineto{\pgfqpoint{3.063015in}{2.079891in}}%
\pgfpathlineto{\pgfqpoint{3.076535in}{2.073675in}}%
\pgfpathlineto{\pgfqpoint{3.090058in}{2.067556in}}%
\pgfpathlineto{\pgfqpoint{3.098331in}{2.075776in}}%
\pgfpathlineto{\pgfqpoint{3.106598in}{2.084038in}}%
\pgfpathlineto{\pgfqpoint{3.114858in}{2.092343in}}%
\pgfpathlineto{\pgfqpoint{3.123110in}{2.100688in}}%
\pgfpathlineto{\pgfqpoint{3.109602in}{2.106724in}}%
\pgfpathlineto{\pgfqpoint{3.096097in}{2.112857in}}%
\pgfpathlineto{\pgfqpoint{3.082595in}{2.119089in}}%
\pgfpathlineto{\pgfqpoint{3.069096in}{2.125419in}}%
\pgfpathlineto{\pgfqpoint{3.060828in}{2.117150in}}%
\pgfpathlineto{\pgfqpoint{3.052554in}{2.108926in}}%
\pgfpathlineto{\pgfqpoint{3.044272in}{2.100750in}}%
\pgfpathlineto{\pgfqpoint{3.035983in}{2.092621in}}%
\pgfpathclose%
\pgfusepath{fill}%
\end{pgfscope}%
\begin{pgfscope}%
\pgfpathrectangle{\pgfqpoint{1.150000in}{0.150000in}}{\pgfqpoint{5.700000in}{5.700000in}}%
\pgfusepath{clip}%
\pgfsetbuttcap%
\pgfsetroundjoin%
\definecolor{currentfill}{rgb}{0.212395,0.359683,0.551710}%
\pgfsetfillcolor{currentfill}%
\pgfsetfillopacity{0.700000}%
\pgfsetlinewidth{0.000000pt}%
\definecolor{currentstroke}{rgb}{0.000000,0.000000,0.000000}%
\pgfsetstrokecolor{currentstroke}%
\pgfsetdash{}{0pt}%
\pgfpathmoveto{\pgfqpoint{6.003805in}{2.751927in}}%
\pgfpathlineto{\pgfqpoint{6.018071in}{2.751632in}}%
\pgfpathlineto{\pgfqpoint{6.032347in}{2.751402in}}%
\pgfpathlineto{\pgfqpoint{6.046634in}{2.751237in}}%
\pgfpathlineto{\pgfqpoint{6.053841in}{2.759653in}}%
\pgfpathlineto{\pgfqpoint{6.061053in}{2.768345in}}%
\pgfpathlineto{\pgfqpoint{6.068272in}{2.777322in}}%
\pgfpathlineto{\pgfqpoint{6.075496in}{2.786593in}}%
\pgfpathlineto{\pgfqpoint{6.061239in}{2.787248in}}%
\pgfpathlineto{\pgfqpoint{6.046993in}{2.787968in}}%
\pgfpathlineto{\pgfqpoint{6.032757in}{2.788753in}}%
\pgfpathlineto{\pgfqpoint{6.025510in}{2.779110in}}%
\pgfpathlineto{\pgfqpoint{6.018270in}{2.769764in}}%
\pgfpathlineto{\pgfqpoint{6.011035in}{2.760706in}}%
\pgfpathlineto{\pgfqpoint{6.003805in}{2.751927in}}%
\pgfpathclose%
\pgfusepath{fill}%
\end{pgfscope}%
\begin{pgfscope}%
\pgfpathrectangle{\pgfqpoint{1.150000in}{0.150000in}}{\pgfqpoint{5.700000in}{5.700000in}}%
\pgfusepath{clip}%
\pgfsetbuttcap%
\pgfsetroundjoin%
\definecolor{currentfill}{rgb}{0.280868,0.160771,0.472899}%
\pgfsetfillcolor{currentfill}%
\pgfsetfillopacity{0.700000}%
\pgfsetlinewidth{0.000000pt}%
\definecolor{currentstroke}{rgb}{0.000000,0.000000,0.000000}%
\pgfsetstrokecolor{currentstroke}%
\pgfsetdash{}{0pt}%
\pgfpathmoveto{\pgfqpoint{4.488272in}{2.305521in}}%
\pgfpathlineto{\pgfqpoint{4.502091in}{2.304940in}}%
\pgfpathlineto{\pgfqpoint{4.515919in}{2.304431in}}%
\pgfpathlineto{\pgfqpoint{4.529756in}{2.303996in}}%
\pgfpathlineto{\pgfqpoint{4.543601in}{2.303634in}}%
\pgfpathlineto{\pgfqpoint{4.551341in}{2.311413in}}%
\pgfpathlineto{\pgfqpoint{4.559075in}{2.319202in}}%
\pgfpathlineto{\pgfqpoint{4.566803in}{2.327005in}}%
\pgfpathlineto{\pgfqpoint{4.574526in}{2.334825in}}%
\pgfpathlineto{\pgfqpoint{4.560694in}{2.335372in}}%
\pgfpathlineto{\pgfqpoint{4.546871in}{2.335991in}}%
\pgfpathlineto{\pgfqpoint{4.533056in}{2.336682in}}%
\pgfpathlineto{\pgfqpoint{4.519250in}{2.337447in}}%
\pgfpathlineto{\pgfqpoint{4.511514in}{2.329436in}}%
\pgfpathlineto{\pgfqpoint{4.503772in}{2.321447in}}%
\pgfpathlineto{\pgfqpoint{4.496025in}{2.313476in}}%
\pgfpathlineto{\pgfqpoint{4.488272in}{2.305521in}}%
\pgfpathclose%
\pgfusepath{fill}%
\end{pgfscope}%
\begin{pgfscope}%
\pgfpathrectangle{\pgfqpoint{1.150000in}{0.150000in}}{\pgfqpoint{5.700000in}{5.700000in}}%
\pgfusepath{clip}%
\pgfsetbuttcap%
\pgfsetroundjoin%
\definecolor{currentfill}{rgb}{0.235526,0.309527,0.542944}%
\pgfsetfillcolor{currentfill}%
\pgfsetfillopacity{0.700000}%
\pgfsetlinewidth{0.000000pt}%
\definecolor{currentstroke}{rgb}{0.000000,0.000000,0.000000}%
\pgfsetstrokecolor{currentstroke}%
\pgfsetdash{}{0pt}%
\pgfpathmoveto{\pgfqpoint{5.603247in}{2.627204in}}%
\pgfpathlineto{\pgfqpoint{5.617408in}{2.627356in}}%
\pgfpathlineto{\pgfqpoint{5.631580in}{2.627575in}}%
\pgfpathlineto{\pgfqpoint{5.645763in}{2.627860in}}%
\pgfpathlineto{\pgfqpoint{5.659956in}{2.628211in}}%
\pgfpathlineto{\pgfqpoint{5.667260in}{2.635232in}}%
\pgfpathlineto{\pgfqpoint{5.674564in}{2.642426in}}%
\pgfpathlineto{\pgfqpoint{5.681869in}{2.649800in}}%
\pgfpathlineto{\pgfqpoint{5.689173in}{2.657362in}}%
\pgfpathlineto{\pgfqpoint{5.675005in}{2.657420in}}%
\pgfpathlineto{\pgfqpoint{5.660847in}{2.657545in}}%
\pgfpathlineto{\pgfqpoint{5.646700in}{2.657736in}}%
\pgfpathlineto{\pgfqpoint{5.632563in}{2.657993in}}%
\pgfpathlineto{\pgfqpoint{5.625234in}{2.650015in}}%
\pgfpathlineto{\pgfqpoint{5.617905in}{2.642228in}}%
\pgfpathlineto{\pgfqpoint{5.610576in}{2.634627in}}%
\pgfpathlineto{\pgfqpoint{5.603247in}{2.627204in}}%
\pgfpathclose%
\pgfusepath{fill}%
\end{pgfscope}%
\begin{pgfscope}%
\pgfpathrectangle{\pgfqpoint{1.150000in}{0.150000in}}{\pgfqpoint{5.700000in}{5.700000in}}%
\pgfusepath{clip}%
\pgfsetbuttcap%
\pgfsetroundjoin%
\definecolor{currentfill}{rgb}{0.274952,0.037752,0.364543}%
\pgfsetfillcolor{currentfill}%
\pgfsetfillopacity{0.700000}%
\pgfsetlinewidth{0.000000pt}%
\definecolor{currentstroke}{rgb}{0.000000,0.000000,0.000000}%
\pgfsetstrokecolor{currentstroke}%
\pgfsetdash{}{0pt}%
\pgfpathmoveto{\pgfqpoint{3.177177in}{2.077515in}}%
\pgfpathlineto{\pgfqpoint{3.190702in}{2.071960in}}%
\pgfpathlineto{\pgfqpoint{3.204231in}{2.066500in}}%
\pgfpathlineto{\pgfqpoint{3.217764in}{2.061135in}}%
\pgfpathlineto{\pgfqpoint{3.231301in}{2.055862in}}%
\pgfpathlineto{\pgfqpoint{3.239519in}{2.064389in}}%
\pgfpathlineto{\pgfqpoint{3.247730in}{2.072945in}}%
\pgfpathlineto{\pgfqpoint{3.255935in}{2.081531in}}%
\pgfpathlineto{\pgfqpoint{3.264133in}{2.090144in}}%
\pgfpathlineto{\pgfqpoint{3.250610in}{2.095354in}}%
\pgfpathlineto{\pgfqpoint{3.237091in}{2.100657in}}%
\pgfpathlineto{\pgfqpoint{3.223575in}{2.106054in}}%
\pgfpathlineto{\pgfqpoint{3.210064in}{2.111545in}}%
\pgfpathlineto{\pgfqpoint{3.201852in}{2.102987in}}%
\pgfpathlineto{\pgfqpoint{3.193634in}{2.094462in}}%
\pgfpathlineto{\pgfqpoint{3.185409in}{2.085971in}}%
\pgfpathlineto{\pgfqpoint{3.177177in}{2.077515in}}%
\pgfpathclose%
\pgfusepath{fill}%
\end{pgfscope}%
\begin{pgfscope}%
\pgfpathrectangle{\pgfqpoint{1.150000in}{0.150000in}}{\pgfqpoint{5.700000in}{5.700000in}}%
\pgfusepath{clip}%
\pgfsetbuttcap%
\pgfsetroundjoin%
\definecolor{currentfill}{rgb}{0.277941,0.056324,0.381191}%
\pgfsetfillcolor{currentfill}%
\pgfsetfillopacity{0.700000}%
\pgfsetlinewidth{0.000000pt}%
\definecolor{currentstroke}{rgb}{0.000000,0.000000,0.000000}%
\pgfsetstrokecolor{currentstroke}%
\pgfsetdash{}{0pt}%
\pgfpathmoveto{\pgfqpoint{2.894604in}{2.116400in}}%
\pgfpathlineto{\pgfqpoint{2.908115in}{2.109065in}}%
\pgfpathlineto{\pgfqpoint{2.921629in}{2.101834in}}%
\pgfpathlineto{\pgfqpoint{2.935144in}{2.094707in}}%
\pgfpathlineto{\pgfqpoint{2.948662in}{2.087683in}}%
\pgfpathlineto{\pgfqpoint{2.956996in}{2.095498in}}%
\pgfpathlineto{\pgfqpoint{2.965323in}{2.103371in}}%
\pgfpathlineto{\pgfqpoint{2.973642in}{2.111299in}}%
\pgfpathlineto{\pgfqpoint{2.981954in}{2.119282in}}%
\pgfpathlineto{\pgfqpoint{2.968453in}{2.126202in}}%
\pgfpathlineto{\pgfqpoint{2.954954in}{2.133225in}}%
\pgfpathlineto{\pgfqpoint{2.941457in}{2.140351in}}%
\pgfpathlineto{\pgfqpoint{2.927963in}{2.147581in}}%
\pgfpathlineto{\pgfqpoint{2.919635in}{2.139695in}}%
\pgfpathlineto{\pgfqpoint{2.911299in}{2.131868in}}%
\pgfpathlineto{\pgfqpoint{2.902956in}{2.124103in}}%
\pgfpathlineto{\pgfqpoint{2.894604in}{2.116400in}}%
\pgfpathclose%
\pgfusepath{fill}%
\end{pgfscope}%
\begin{pgfscope}%
\pgfpathrectangle{\pgfqpoint{1.150000in}{0.150000in}}{\pgfqpoint{5.700000in}{5.700000in}}%
\pgfusepath{clip}%
\pgfsetbuttcap%
\pgfsetroundjoin%
\definecolor{currentfill}{rgb}{0.255645,0.260703,0.528312}%
\pgfsetfillcolor{currentfill}%
\pgfsetfillopacity{0.700000}%
\pgfsetlinewidth{0.000000pt}%
\definecolor{currentstroke}{rgb}{0.000000,0.000000,0.000000}%
\pgfsetstrokecolor{currentstroke}%
\pgfsetdash{}{0pt}%
\pgfpathmoveto{\pgfqpoint{5.202876in}{2.509370in}}%
\pgfpathlineto{\pgfqpoint{5.216921in}{2.509610in}}%
\pgfpathlineto{\pgfqpoint{5.230975in}{2.509918in}}%
\pgfpathlineto{\pgfqpoint{5.245040in}{2.510295in}}%
\pgfpathlineto{\pgfqpoint{5.259115in}{2.510739in}}%
\pgfpathlineto{\pgfqpoint{5.266567in}{2.517516in}}%
\pgfpathlineto{\pgfqpoint{5.274017in}{2.524386in}}%
\pgfpathlineto{\pgfqpoint{5.281463in}{2.531356in}}%
\pgfpathlineto{\pgfqpoint{5.288905in}{2.538433in}}%
\pgfpathlineto{\pgfqpoint{5.274851in}{2.538316in}}%
\pgfpathlineto{\pgfqpoint{5.260806in}{2.538268in}}%
\pgfpathlineto{\pgfqpoint{5.246772in}{2.538288in}}%
\pgfpathlineto{\pgfqpoint{5.232747in}{2.538376in}}%
\pgfpathlineto{\pgfqpoint{5.225284in}{2.530964in}}%
\pgfpathlineto{\pgfqpoint{5.217818in}{2.523663in}}%
\pgfpathlineto{\pgfqpoint{5.210349in}{2.516467in}}%
\pgfpathlineto{\pgfqpoint{5.202876in}{2.509370in}}%
\pgfpathclose%
\pgfusepath{fill}%
\end{pgfscope}%
\begin{pgfscope}%
\pgfpathrectangle{\pgfqpoint{1.150000in}{0.150000in}}{\pgfqpoint{5.700000in}{5.700000in}}%
\pgfusepath{clip}%
\pgfsetbuttcap%
\pgfsetroundjoin%
\definecolor{currentfill}{rgb}{0.277018,0.050344,0.375715}%
\pgfsetfillcolor{currentfill}%
\pgfsetfillopacity{0.700000}%
\pgfsetlinewidth{0.000000pt}%
\definecolor{currentstroke}{rgb}{0.000000,0.000000,0.000000}%
\pgfsetstrokecolor{currentstroke}%
\pgfsetdash{}{0pt}%
\pgfpathmoveto{\pgfqpoint{3.546069in}{2.090041in}}%
\pgfpathlineto{\pgfqpoint{3.559648in}{2.086413in}}%
\pgfpathlineto{\pgfqpoint{3.573232in}{2.082871in}}%
\pgfpathlineto{\pgfqpoint{3.586822in}{2.079413in}}%
\pgfpathlineto{\pgfqpoint{3.600418in}{2.076041in}}%
\pgfpathlineto{\pgfqpoint{3.608500in}{2.084948in}}%
\pgfpathlineto{\pgfqpoint{3.616577in}{2.093858in}}%
\pgfpathlineto{\pgfqpoint{3.624647in}{2.102771in}}%
\pgfpathlineto{\pgfqpoint{3.632712in}{2.111688in}}%
\pgfpathlineto{\pgfqpoint{3.619128in}{2.115060in}}%
\pgfpathlineto{\pgfqpoint{3.605550in}{2.118516in}}%
\pgfpathlineto{\pgfqpoint{3.591977in}{2.122057in}}%
\pgfpathlineto{\pgfqpoint{3.578410in}{2.125684in}}%
\pgfpathlineto{\pgfqpoint{3.570333in}{2.116760in}}%
\pgfpathlineto{\pgfqpoint{3.562251in}{2.107846in}}%
\pgfpathlineto{\pgfqpoint{3.554163in}{2.098940in}}%
\pgfpathlineto{\pgfqpoint{3.546069in}{2.090041in}}%
\pgfpathclose%
\pgfusepath{fill}%
\end{pgfscope}%
\begin{pgfscope}%
\pgfpathrectangle{\pgfqpoint{1.150000in}{0.150000in}}{\pgfqpoint{5.700000in}{5.700000in}}%
\pgfusepath{clip}%
\pgfsetbuttcap%
\pgfsetroundjoin%
\definecolor{currentfill}{rgb}{0.283091,0.110553,0.431554}%
\pgfsetfillcolor{currentfill}%
\pgfsetfillopacity{0.700000}%
\pgfsetlinewidth{0.000000pt}%
\definecolor{currentstroke}{rgb}{0.000000,0.000000,0.000000}%
\pgfsetstrokecolor{currentstroke}%
\pgfsetdash{}{0pt}%
\pgfpathmoveto{\pgfqpoint{4.087831in}{2.194499in}}%
\pgfpathlineto{\pgfqpoint{4.101537in}{2.192944in}}%
\pgfpathlineto{\pgfqpoint{4.115250in}{2.191466in}}%
\pgfpathlineto{\pgfqpoint{4.128971in}{2.190065in}}%
\pgfpathlineto{\pgfqpoint{4.142699in}{2.188741in}}%
\pgfpathlineto{\pgfqpoint{4.150590in}{2.197230in}}%
\pgfpathlineto{\pgfqpoint{4.158476in}{2.205710in}}%
\pgfpathlineto{\pgfqpoint{4.166356in}{2.214184in}}%
\pgfpathlineto{\pgfqpoint{4.174230in}{2.222655in}}%
\pgfpathlineto{\pgfqpoint{4.160513in}{2.224080in}}%
\pgfpathlineto{\pgfqpoint{4.146803in}{2.225583in}}%
\pgfpathlineto{\pgfqpoint{4.133101in}{2.227162in}}%
\pgfpathlineto{\pgfqpoint{4.119407in}{2.228818in}}%
\pgfpathlineto{\pgfqpoint{4.111521in}{2.220238in}}%
\pgfpathlineto{\pgfqpoint{4.103630in}{2.211660in}}%
\pgfpathlineto{\pgfqpoint{4.095734in}{2.203081in}}%
\pgfpathlineto{\pgfqpoint{4.087831in}{2.194499in}}%
\pgfpathclose%
\pgfusepath{fill}%
\end{pgfscope}%
\begin{pgfscope}%
\pgfpathrectangle{\pgfqpoint{1.150000in}{0.150000in}}{\pgfqpoint{5.700000in}{5.700000in}}%
\pgfusepath{clip}%
\pgfsetbuttcap%
\pgfsetroundjoin%
\definecolor{currentfill}{rgb}{0.273006,0.204520,0.501721}%
\pgfsetfillcolor{currentfill}%
\pgfsetfillopacity{0.700000}%
\pgfsetlinewidth{0.000000pt}%
\definecolor{currentstroke}{rgb}{0.000000,0.000000,0.000000}%
\pgfsetstrokecolor{currentstroke}%
\pgfsetdash{}{0pt}%
\pgfpathmoveto{\pgfqpoint{4.802455in}{2.392375in}}%
\pgfpathlineto{\pgfqpoint{4.816375in}{2.392331in}}%
\pgfpathlineto{\pgfqpoint{4.830305in}{2.392359in}}%
\pgfpathlineto{\pgfqpoint{4.844244in}{2.392457in}}%
\pgfpathlineto{\pgfqpoint{4.858193in}{2.392625in}}%
\pgfpathlineto{\pgfqpoint{4.865809in}{2.399823in}}%
\pgfpathlineto{\pgfqpoint{4.873420in}{2.407057in}}%
\pgfpathlineto{\pgfqpoint{4.881025in}{2.414333in}}%
\pgfpathlineto{\pgfqpoint{4.888626in}{2.421656in}}%
\pgfpathlineto{\pgfqpoint{4.874693in}{2.421734in}}%
\pgfpathlineto{\pgfqpoint{4.860769in}{2.421882in}}%
\pgfpathlineto{\pgfqpoint{4.846855in}{2.422100in}}%
\pgfpathlineto{\pgfqpoint{4.832950in}{2.422390in}}%
\pgfpathlineto{\pgfqpoint{4.825334in}{2.414814in}}%
\pgfpathlineto{\pgfqpoint{4.817713in}{2.407289in}}%
\pgfpathlineto{\pgfqpoint{4.810086in}{2.399811in}}%
\pgfpathlineto{\pgfqpoint{4.802455in}{2.392375in}}%
\pgfpathclose%
\pgfusepath{fill}%
\end{pgfscope}%
\begin{pgfscope}%
\pgfpathrectangle{\pgfqpoint{1.150000in}{0.150000in}}{\pgfqpoint{5.700000in}{5.700000in}}%
\pgfusepath{clip}%
\pgfsetbuttcap%
\pgfsetroundjoin%
\definecolor{currentfill}{rgb}{0.280267,0.073417,0.397163}%
\pgfsetfillcolor{currentfill}%
\pgfsetfillopacity{0.700000}%
\pgfsetlinewidth{0.000000pt}%
\definecolor{currentstroke}{rgb}{0.000000,0.000000,0.000000}%
\pgfsetstrokecolor{currentstroke}%
\pgfsetdash{}{0pt}%
\pgfpathmoveto{\pgfqpoint{3.773711in}{2.123292in}}%
\pgfpathlineto{\pgfqpoint{3.787338in}{2.120649in}}%
\pgfpathlineto{\pgfqpoint{3.800972in}{2.118087in}}%
\pgfpathlineto{\pgfqpoint{3.814612in}{2.115607in}}%
\pgfpathlineto{\pgfqpoint{3.828259in}{2.113207in}}%
\pgfpathlineto{\pgfqpoint{3.836262in}{2.122058in}}%
\pgfpathlineto{\pgfqpoint{3.844260in}{2.130901in}}%
\pgfpathlineto{\pgfqpoint{3.852252in}{2.139740in}}%
\pgfpathlineto{\pgfqpoint{3.860238in}{2.148574in}}%
\pgfpathlineto{\pgfqpoint{3.846602in}{2.151014in}}%
\pgfpathlineto{\pgfqpoint{3.832973in}{2.153534in}}%
\pgfpathlineto{\pgfqpoint{3.819350in}{2.156136in}}%
\pgfpathlineto{\pgfqpoint{3.805733in}{2.158818in}}%
\pgfpathlineto{\pgfqpoint{3.797736in}{2.149937in}}%
\pgfpathlineto{\pgfqpoint{3.789734in}{2.141056in}}%
\pgfpathlineto{\pgfqpoint{3.781725in}{2.132175in}}%
\pgfpathlineto{\pgfqpoint{3.773711in}{2.123292in}}%
\pgfpathclose%
\pgfusepath{fill}%
\end{pgfscope}%
\begin{pgfscope}%
\pgfpathrectangle{\pgfqpoint{1.150000in}{0.150000in}}{\pgfqpoint{5.700000in}{5.700000in}}%
\pgfusepath{clip}%
\pgfsetbuttcap%
\pgfsetroundjoin%
\definecolor{currentfill}{rgb}{0.283229,0.120777,0.440584}%
\pgfsetfillcolor{currentfill}%
\pgfsetfillopacity{0.700000}%
\pgfsetlinewidth{0.000000pt}%
\definecolor{currentstroke}{rgb}{0.000000,0.000000,0.000000}%
\pgfsetstrokecolor{currentstroke}%
\pgfsetdash{}{0pt}%
\pgfpathmoveto{\pgfqpoint{2.556769in}{2.232463in}}%
\pgfpathlineto{\pgfqpoint{2.570303in}{2.222599in}}%
\pgfpathlineto{\pgfqpoint{2.583837in}{2.212857in}}%
\pgfpathlineto{\pgfqpoint{2.597371in}{2.203235in}}%
\pgfpathlineto{\pgfqpoint{2.610904in}{2.193733in}}%
\pgfpathlineto{\pgfqpoint{2.619401in}{2.200265in}}%
\pgfpathlineto{\pgfqpoint{2.627888in}{2.206893in}}%
\pgfpathlineto{\pgfqpoint{2.636366in}{2.213615in}}%
\pgfpathlineto{\pgfqpoint{2.644834in}{2.220430in}}%
\pgfpathlineto{\pgfqpoint{2.631322in}{2.229785in}}%
\pgfpathlineto{\pgfqpoint{2.617809in}{2.239260in}}%
\pgfpathlineto{\pgfqpoint{2.604296in}{2.248854in}}%
\pgfpathlineto{\pgfqpoint{2.590783in}{2.258570in}}%
\pgfpathlineto{\pgfqpoint{2.582294in}{2.251895in}}%
\pgfpathlineto{\pgfqpoint{2.573796in}{2.245318in}}%
\pgfpathlineto{\pgfqpoint{2.565287in}{2.238840in}}%
\pgfpathlineto{\pgfqpoint{2.556769in}{2.232463in}}%
\pgfpathclose%
\pgfusepath{fill}%
\end{pgfscope}%
\begin{pgfscope}%
\pgfpathrectangle{\pgfqpoint{1.150000in}{0.150000in}}{\pgfqpoint{5.700000in}{5.700000in}}%
\pgfusepath{clip}%
\pgfsetbuttcap%
\pgfsetroundjoin%
\definecolor{currentfill}{rgb}{0.274952,0.037752,0.364543}%
\pgfsetfillcolor{currentfill}%
\pgfsetfillopacity{0.700000}%
\pgfsetlinewidth{0.000000pt}%
\definecolor{currentstroke}{rgb}{0.000000,0.000000,0.000000}%
\pgfsetstrokecolor{currentstroke}%
\pgfsetdash{}{0pt}%
\pgfpathmoveto{\pgfqpoint{3.318267in}{2.070231in}}%
\pgfpathlineto{\pgfqpoint{3.331811in}{2.065481in}}%
\pgfpathlineto{\pgfqpoint{3.345359in}{2.060822in}}%
\pgfpathlineto{\pgfqpoint{3.358912in}{2.056253in}}%
\pgfpathlineto{\pgfqpoint{3.372470in}{2.051774in}}%
\pgfpathlineto{\pgfqpoint{3.380636in}{2.060515in}}%
\pgfpathlineto{\pgfqpoint{3.388796in}{2.069274in}}%
\pgfpathlineto{\pgfqpoint{3.396950in}{2.078050in}}%
\pgfpathlineto{\pgfqpoint{3.405097in}{2.086844in}}%
\pgfpathlineto{\pgfqpoint{3.391552in}{2.091281in}}%
\pgfpathlineto{\pgfqpoint{3.378012in}{2.095808in}}%
\pgfpathlineto{\pgfqpoint{3.364476in}{2.100425in}}%
\pgfpathlineto{\pgfqpoint{3.350945in}{2.105132in}}%
\pgfpathlineto{\pgfqpoint{3.342785in}{2.096373in}}%
\pgfpathlineto{\pgfqpoint{3.334618in}{2.087637in}}%
\pgfpathlineto{\pgfqpoint{3.326446in}{2.078922in}}%
\pgfpathlineto{\pgfqpoint{3.318267in}{2.070231in}}%
\pgfpathclose%
\pgfusepath{fill}%
\end{pgfscope}%
\begin{pgfscope}%
\pgfpathrectangle{\pgfqpoint{1.150000in}{0.150000in}}{\pgfqpoint{5.700000in}{5.700000in}}%
\pgfusepath{clip}%
\pgfsetbuttcap%
\pgfsetroundjoin%
\definecolor{currentfill}{rgb}{0.280894,0.078907,0.402329}%
\pgfsetfillcolor{currentfill}%
\pgfsetfillopacity{0.700000}%
\pgfsetlinewidth{0.000000pt}%
\definecolor{currentstroke}{rgb}{0.000000,0.000000,0.000000}%
\pgfsetstrokecolor{currentstroke}%
\pgfsetdash{}{0pt}%
\pgfpathmoveto{\pgfqpoint{2.752946in}{2.149771in}}%
\pgfpathlineto{\pgfqpoint{2.766464in}{2.141448in}}%
\pgfpathlineto{\pgfqpoint{2.779982in}{2.133236in}}%
\pgfpathlineto{\pgfqpoint{2.793501in}{2.125134in}}%
\pgfpathlineto{\pgfqpoint{2.807022in}{2.117141in}}%
\pgfpathlineto{\pgfqpoint{2.815423in}{2.124446in}}%
\pgfpathlineto{\pgfqpoint{2.823816in}{2.131826in}}%
\pgfpathlineto{\pgfqpoint{2.832201in}{2.139277in}}%
\pgfpathlineto{\pgfqpoint{2.840578in}{2.146798in}}%
\pgfpathlineto{\pgfqpoint{2.827075in}{2.154667in}}%
\pgfpathlineto{\pgfqpoint{2.813574in}{2.162643in}}%
\pgfpathlineto{\pgfqpoint{2.800074in}{2.170730in}}%
\pgfpathlineto{\pgfqpoint{2.786576in}{2.178927in}}%
\pgfpathlineto{\pgfqpoint{2.778181in}{2.171523in}}%
\pgfpathlineto{\pgfqpoint{2.769778in}{2.164195in}}%
\pgfpathlineto{\pgfqpoint{2.761366in}{2.156943in}}%
\pgfpathlineto{\pgfqpoint{2.752946in}{2.149771in}}%
\pgfpathclose%
\pgfusepath{fill}%
\end{pgfscope}%
\begin{pgfscope}%
\pgfpathrectangle{\pgfqpoint{1.150000in}{0.150000in}}{\pgfqpoint{5.700000in}{5.700000in}}%
\pgfusepath{clip}%
\pgfsetbuttcap%
\pgfsetroundjoin%
\definecolor{currentfill}{rgb}{0.281887,0.150881,0.465405}%
\pgfsetfillcolor{currentfill}%
\pgfsetfillopacity{0.700000}%
\pgfsetlinewidth{0.000000pt}%
\definecolor{currentstroke}{rgb}{0.000000,0.000000,0.000000}%
\pgfsetstrokecolor{currentstroke}%
\pgfsetdash{}{0pt}%
\pgfpathmoveto{\pgfqpoint{4.401962in}{2.276194in}}%
\pgfpathlineto{\pgfqpoint{4.415760in}{2.275481in}}%
\pgfpathlineto{\pgfqpoint{4.429567in}{2.274843in}}%
\pgfpathlineto{\pgfqpoint{4.443382in}{2.274278in}}%
\pgfpathlineto{\pgfqpoint{4.457205in}{2.273788in}}%
\pgfpathlineto{\pgfqpoint{4.464980in}{2.281714in}}%
\pgfpathlineto{\pgfqpoint{4.472750in}{2.289643in}}%
\pgfpathlineto{\pgfqpoint{4.480514in}{2.297578in}}%
\pgfpathlineto{\pgfqpoint{4.488272in}{2.305521in}}%
\pgfpathlineto{\pgfqpoint{4.474461in}{2.306175in}}%
\pgfpathlineto{\pgfqpoint{4.460659in}{2.306903in}}%
\pgfpathlineto{\pgfqpoint{4.446865in}{2.307705in}}%
\pgfpathlineto{\pgfqpoint{4.433080in}{2.308581in}}%
\pgfpathlineto{\pgfqpoint{4.425309in}{2.300466in}}%
\pgfpathlineto{\pgfqpoint{4.417532in}{2.292366in}}%
\pgfpathlineto{\pgfqpoint{4.409750in}{2.284276in}}%
\pgfpathlineto{\pgfqpoint{4.401962in}{2.276194in}}%
\pgfpathclose%
\pgfusepath{fill}%
\end{pgfscope}%
\begin{pgfscope}%
\pgfpathrectangle{\pgfqpoint{1.150000in}{0.150000in}}{\pgfqpoint{5.700000in}{5.700000in}}%
\pgfusepath{clip}%
\pgfsetbuttcap%
\pgfsetroundjoin%
\definecolor{currentfill}{rgb}{0.276194,0.190074,0.493001}%
\pgfsetfillcolor{currentfill}%
\pgfsetfillopacity{0.700000}%
\pgfsetlinewidth{0.000000pt}%
\definecolor{currentstroke}{rgb}{0.000000,0.000000,0.000000}%
\pgfsetstrokecolor{currentstroke}%
\pgfsetdash{}{0pt}%
\pgfpathmoveto{\pgfqpoint{2.305500in}{2.385905in}}%
\pgfpathlineto{\pgfqpoint{2.319084in}{2.373782in}}%
\pgfpathlineto{\pgfqpoint{2.332665in}{2.361799in}}%
\pgfpathlineto{\pgfqpoint{2.346244in}{2.349954in}}%
\pgfpathlineto{\pgfqpoint{2.359821in}{2.338247in}}%
\pgfpathlineto{\pgfqpoint{2.368452in}{2.343665in}}%
\pgfpathlineto{\pgfqpoint{2.377071in}{2.349210in}}%
\pgfpathlineto{\pgfqpoint{2.385680in}{2.354877in}}%
\pgfpathlineto{\pgfqpoint{2.394277in}{2.360666in}}%
\pgfpathlineto{\pgfqpoint{2.380725in}{2.372203in}}%
\pgfpathlineto{\pgfqpoint{2.367171in}{2.383877in}}%
\pgfpathlineto{\pgfqpoint{2.353614in}{2.395689in}}%
\pgfpathlineto{\pgfqpoint{2.340055in}{2.407641in}}%
\pgfpathlineto{\pgfqpoint{2.331434in}{2.402015in}}%
\pgfpathlineto{\pgfqpoint{2.322801in}{2.396516in}}%
\pgfpathlineto{\pgfqpoint{2.314156in}{2.391145in}}%
\pgfpathlineto{\pgfqpoint{2.305500in}{2.385905in}}%
\pgfpathclose%
\pgfusepath{fill}%
\end{pgfscope}%
\begin{pgfscope}%
\pgfpathrectangle{\pgfqpoint{1.150000in}{0.150000in}}{\pgfqpoint{5.700000in}{5.700000in}}%
\pgfusepath{clip}%
\pgfsetbuttcap%
\pgfsetroundjoin%
\definecolor{currentfill}{rgb}{0.239346,0.300855,0.540844}%
\pgfsetfillcolor{currentfill}%
\pgfsetfillopacity{0.700000}%
\pgfsetlinewidth{0.000000pt}%
\definecolor{currentstroke}{rgb}{0.000000,0.000000,0.000000}%
\pgfsetstrokecolor{currentstroke}%
\pgfsetdash{}{0pt}%
\pgfpathmoveto{\pgfqpoint{5.517285in}{2.597648in}}%
\pgfpathlineto{\pgfqpoint{5.531429in}{2.597923in}}%
\pgfpathlineto{\pgfqpoint{5.545583in}{2.598265in}}%
\pgfpathlineto{\pgfqpoint{5.559748in}{2.598674in}}%
\pgfpathlineto{\pgfqpoint{5.573923in}{2.599150in}}%
\pgfpathlineto{\pgfqpoint{5.581256in}{2.605931in}}%
\pgfpathlineto{\pgfqpoint{5.588587in}{2.612863in}}%
\pgfpathlineto{\pgfqpoint{5.595917in}{2.619952in}}%
\pgfpathlineto{\pgfqpoint{5.603247in}{2.627204in}}%
\pgfpathlineto{\pgfqpoint{5.589096in}{2.627118in}}%
\pgfpathlineto{\pgfqpoint{5.574955in}{2.627099in}}%
\pgfpathlineto{\pgfqpoint{5.560825in}{2.627147in}}%
\pgfpathlineto{\pgfqpoint{5.546705in}{2.627261in}}%
\pgfpathlineto{\pgfqpoint{5.539351in}{2.619612in}}%
\pgfpathlineto{\pgfqpoint{5.531997in}{2.612131in}}%
\pgfpathlineto{\pgfqpoint{5.524642in}{2.604812in}}%
\pgfpathlineto{\pgfqpoint{5.517285in}{2.597648in}}%
\pgfpathclose%
\pgfusepath{fill}%
\end{pgfscope}%
\begin{pgfscope}%
\pgfpathrectangle{\pgfqpoint{1.150000in}{0.150000in}}{\pgfqpoint{5.700000in}{5.700000in}}%
\pgfusepath{clip}%
\pgfsetbuttcap%
\pgfsetroundjoin%
\definecolor{currentfill}{rgb}{0.216210,0.351535,0.550627}%
\pgfsetfillcolor{currentfill}%
\pgfsetfillopacity{0.700000}%
\pgfsetlinewidth{0.000000pt}%
\definecolor{currentstroke}{rgb}{0.000000,0.000000,0.000000}%
\pgfsetstrokecolor{currentstroke}%
\pgfsetdash{}{0pt}%
\pgfpathmoveto{\pgfqpoint{5.917861in}{2.719391in}}%
\pgfpathlineto{\pgfqpoint{5.932113in}{2.719306in}}%
\pgfpathlineto{\pgfqpoint{5.946375in}{2.719286in}}%
\pgfpathlineto{\pgfqpoint{5.960649in}{2.719332in}}%
\pgfpathlineto{\pgfqpoint{5.974933in}{2.719443in}}%
\pgfpathlineto{\pgfqpoint{5.982145in}{2.727186in}}%
\pgfpathlineto{\pgfqpoint{5.989361in}{2.735176in}}%
\pgfpathlineto{\pgfqpoint{5.996581in}{2.743420in}}%
\pgfpathlineto{\pgfqpoint{6.003805in}{2.751927in}}%
\pgfpathlineto{\pgfqpoint{5.989551in}{2.752287in}}%
\pgfpathlineto{\pgfqpoint{5.975306in}{2.752712in}}%
\pgfpathlineto{\pgfqpoint{5.961073in}{2.753202in}}%
\pgfpathlineto{\pgfqpoint{5.946850in}{2.753757in}}%
\pgfpathlineto{\pgfqpoint{5.939596in}{2.744773in}}%
\pgfpathlineto{\pgfqpoint{5.932347in}{2.736056in}}%
\pgfpathlineto{\pgfqpoint{5.925102in}{2.727598in}}%
\pgfpathlineto{\pgfqpoint{5.917861in}{2.719391in}}%
\pgfpathclose%
\pgfusepath{fill}%
\end{pgfscope}%
\begin{pgfscope}%
\pgfpathrectangle{\pgfqpoint{1.150000in}{0.150000in}}{\pgfqpoint{5.700000in}{5.700000in}}%
\pgfusepath{clip}%
\pgfsetbuttcap%
\pgfsetroundjoin%
\definecolor{currentfill}{rgb}{0.260571,0.246922,0.522828}%
\pgfsetfillcolor{currentfill}%
\pgfsetfillopacity{0.700000}%
\pgfsetlinewidth{0.000000pt}%
\definecolor{currentstroke}{rgb}{0.000000,0.000000,0.000000}%
\pgfsetstrokecolor{currentstroke}%
\pgfsetdash{}{0pt}%
\pgfpathmoveto{\pgfqpoint{5.116792in}{2.480350in}}%
\pgfpathlineto{\pgfqpoint{5.130816in}{2.480623in}}%
\pgfpathlineto{\pgfqpoint{5.144850in}{2.480965in}}%
\pgfpathlineto{\pgfqpoint{5.158893in}{2.481376in}}%
\pgfpathlineto{\pgfqpoint{5.172947in}{2.481856in}}%
\pgfpathlineto{\pgfqpoint{5.180435in}{2.488614in}}%
\pgfpathlineto{\pgfqpoint{5.187920in}{2.495449in}}%
\pgfpathlineto{\pgfqpoint{5.195400in}{2.502365in}}%
\pgfpathlineto{\pgfqpoint{5.202876in}{2.509370in}}%
\pgfpathlineto{\pgfqpoint{5.188842in}{2.509198in}}%
\pgfpathlineto{\pgfqpoint{5.174817in}{2.509095in}}%
\pgfpathlineto{\pgfqpoint{5.160803in}{2.509061in}}%
\pgfpathlineto{\pgfqpoint{5.146798in}{2.509095in}}%
\pgfpathlineto{\pgfqpoint{5.139302in}{2.501776in}}%
\pgfpathlineto{\pgfqpoint{5.131803in}{2.494549in}}%
\pgfpathlineto{\pgfqpoint{5.124300in}{2.487409in}}%
\pgfpathlineto{\pgfqpoint{5.116792in}{2.480350in}}%
\pgfpathclose%
\pgfusepath{fill}%
\end{pgfscope}%
\begin{pgfscope}%
\pgfpathrectangle{\pgfqpoint{1.150000in}{0.150000in}}{\pgfqpoint{5.700000in}{5.700000in}}%
\pgfusepath{clip}%
\pgfsetbuttcap%
\pgfsetroundjoin%
\definecolor{currentfill}{rgb}{0.275191,0.194905,0.496005}%
\pgfsetfillcolor{currentfill}%
\pgfsetfillopacity{0.700000}%
\pgfsetlinewidth{0.000000pt}%
\definecolor{currentstroke}{rgb}{0.000000,0.000000,0.000000}%
\pgfsetstrokecolor{currentstroke}%
\pgfsetdash{}{0pt}%
\pgfpathmoveto{\pgfqpoint{4.716227in}{2.362949in}}%
\pgfpathlineto{\pgfqpoint{4.730125in}{2.362847in}}%
\pgfpathlineto{\pgfqpoint{4.744033in}{2.362816in}}%
\pgfpathlineto{\pgfqpoint{4.757950in}{2.362856in}}%
\pgfpathlineto{\pgfqpoint{4.771876in}{2.362967in}}%
\pgfpathlineto{\pgfqpoint{4.779529in}{2.370277in}}%
\pgfpathlineto{\pgfqpoint{4.787176in}{2.377612in}}%
\pgfpathlineto{\pgfqpoint{4.794818in}{2.384977in}}%
\pgfpathlineto{\pgfqpoint{4.802455in}{2.392375in}}%
\pgfpathlineto{\pgfqpoint{4.788544in}{2.392489in}}%
\pgfpathlineto{\pgfqpoint{4.774642in}{2.392675in}}%
\pgfpathlineto{\pgfqpoint{4.760749in}{2.392931in}}%
\pgfpathlineto{\pgfqpoint{4.746865in}{2.393259in}}%
\pgfpathlineto{\pgfqpoint{4.739214in}{2.385628in}}%
\pgfpathlineto{\pgfqpoint{4.731557in}{2.378035in}}%
\pgfpathlineto{\pgfqpoint{4.723894in}{2.370477in}}%
\pgfpathlineto{\pgfqpoint{4.716227in}{2.362949in}}%
\pgfpathclose%
\pgfusepath{fill}%
\end{pgfscope}%
\begin{pgfscope}%
\pgfpathrectangle{\pgfqpoint{1.150000in}{0.150000in}}{\pgfqpoint{5.700000in}{5.700000in}}%
\pgfusepath{clip}%
\pgfsetbuttcap%
\pgfsetroundjoin%
\definecolor{currentfill}{rgb}{0.282656,0.100196,0.422160}%
\pgfsetfillcolor{currentfill}%
\pgfsetfillopacity{0.700000}%
\pgfsetlinewidth{0.000000pt}%
\definecolor{currentstroke}{rgb}{0.000000,0.000000,0.000000}%
\pgfsetstrokecolor{currentstroke}%
\pgfsetdash{}{0pt}%
\pgfpathmoveto{\pgfqpoint{4.001372in}{2.166771in}}%
\pgfpathlineto{\pgfqpoint{4.015059in}{2.164986in}}%
\pgfpathlineto{\pgfqpoint{4.028753in}{2.163280in}}%
\pgfpathlineto{\pgfqpoint{4.042455in}{2.161651in}}%
\pgfpathlineto{\pgfqpoint{4.056165in}{2.160100in}}%
\pgfpathlineto{\pgfqpoint{4.064090in}{2.168713in}}%
\pgfpathlineto{\pgfqpoint{4.072009in}{2.177317in}}%
\pgfpathlineto{\pgfqpoint{4.079923in}{2.185911in}}%
\pgfpathlineto{\pgfqpoint{4.087831in}{2.194499in}}%
\pgfpathlineto{\pgfqpoint{4.074133in}{2.196131in}}%
\pgfpathlineto{\pgfqpoint{4.060442in}{2.197841in}}%
\pgfpathlineto{\pgfqpoint{4.046758in}{2.199628in}}%
\pgfpathlineto{\pgfqpoint{4.033082in}{2.201494in}}%
\pgfpathlineto{\pgfqpoint{4.025163in}{2.192818in}}%
\pgfpathlineto{\pgfqpoint{4.017238in}{2.184140in}}%
\pgfpathlineto{\pgfqpoint{4.009308in}{2.175459in}}%
\pgfpathlineto{\pgfqpoint{4.001372in}{2.166771in}}%
\pgfpathclose%
\pgfusepath{fill}%
\end{pgfscope}%
\begin{pgfscope}%
\pgfpathrectangle{\pgfqpoint{1.150000in}{0.150000in}}{\pgfqpoint{5.700000in}{5.700000in}}%
\pgfusepath{clip}%
\pgfsetbuttcap%
\pgfsetroundjoin%
\definecolor{currentfill}{rgb}{0.276022,0.044167,0.370164}%
\pgfsetfillcolor{currentfill}%
\pgfsetfillopacity{0.700000}%
\pgfsetlinewidth{0.000000pt}%
\definecolor{currentstroke}{rgb}{0.000000,0.000000,0.000000}%
\pgfsetstrokecolor{currentstroke}%
\pgfsetdash{}{0pt}%
\pgfpathmoveto{\pgfqpoint{3.459326in}{2.069983in}}%
\pgfpathlineto{\pgfqpoint{3.472895in}{2.065987in}}%
\pgfpathlineto{\pgfqpoint{3.486470in}{2.062079in}}%
\pgfpathlineto{\pgfqpoint{3.500050in}{2.058257in}}%
\pgfpathlineto{\pgfqpoint{3.513635in}{2.054521in}}%
\pgfpathlineto{\pgfqpoint{3.521753in}{2.063391in}}%
\pgfpathlineto{\pgfqpoint{3.529864in}{2.072267in}}%
\pgfpathlineto{\pgfqpoint{3.537970in}{2.081151in}}%
\pgfpathlineto{\pgfqpoint{3.546069in}{2.090041in}}%
\pgfpathlineto{\pgfqpoint{3.532496in}{2.093755in}}%
\pgfpathlineto{\pgfqpoint{3.518928in}{2.097555in}}%
\pgfpathlineto{\pgfqpoint{3.505365in}{2.101442in}}%
\pgfpathlineto{\pgfqpoint{3.491808in}{2.105416in}}%
\pgfpathlineto{\pgfqpoint{3.483696in}{2.096539in}}%
\pgfpathlineto{\pgfqpoint{3.475579in}{2.087675in}}%
\pgfpathlineto{\pgfqpoint{3.467455in}{2.078823in}}%
\pgfpathlineto{\pgfqpoint{3.459326in}{2.069983in}}%
\pgfpathclose%
\pgfusepath{fill}%
\end{pgfscope}%
\begin{pgfscope}%
\pgfpathrectangle{\pgfqpoint{1.150000in}{0.150000in}}{\pgfqpoint{5.700000in}{5.700000in}}%
\pgfusepath{clip}%
\pgfsetbuttcap%
\pgfsetroundjoin%
\definecolor{currentfill}{rgb}{0.282623,0.140926,0.457517}%
\pgfsetfillcolor{currentfill}%
\pgfsetfillopacity{0.700000}%
\pgfsetlinewidth{0.000000pt}%
\definecolor{currentstroke}{rgb}{0.000000,0.000000,0.000000}%
\pgfsetstrokecolor{currentstroke}%
\pgfsetdash{}{0pt}%
\pgfpathmoveto{\pgfqpoint{4.315597in}{2.246899in}}%
\pgfpathlineto{\pgfqpoint{4.329374in}{2.246033in}}%
\pgfpathlineto{\pgfqpoint{4.343159in}{2.245240in}}%
\pgfpathlineto{\pgfqpoint{4.356953in}{2.244523in}}%
\pgfpathlineto{\pgfqpoint{4.370754in}{2.243879in}}%
\pgfpathlineto{\pgfqpoint{4.378565in}{2.251961in}}%
\pgfpathlineto{\pgfqpoint{4.386370in}{2.260039in}}%
\pgfpathlineto{\pgfqpoint{4.394169in}{2.268116in}}%
\pgfpathlineto{\pgfqpoint{4.401962in}{2.276194in}}%
\pgfpathlineto{\pgfqpoint{4.388173in}{2.276980in}}%
\pgfpathlineto{\pgfqpoint{4.374392in}{2.277841in}}%
\pgfpathlineto{\pgfqpoint{4.360619in}{2.278776in}}%
\pgfpathlineto{\pgfqpoint{4.346854in}{2.279785in}}%
\pgfpathlineto{\pgfqpoint{4.339048in}{2.271557in}}%
\pgfpathlineto{\pgfqpoint{4.331237in}{2.263335in}}%
\pgfpathlineto{\pgfqpoint{4.323420in}{2.255117in}}%
\pgfpathlineto{\pgfqpoint{4.315597in}{2.246899in}}%
\pgfpathclose%
\pgfusepath{fill}%
\end{pgfscope}%
\begin{pgfscope}%
\pgfpathrectangle{\pgfqpoint{1.150000in}{0.150000in}}{\pgfqpoint{5.700000in}{5.700000in}}%
\pgfusepath{clip}%
\pgfsetbuttcap%
\pgfsetroundjoin%
\definecolor{currentfill}{rgb}{0.221989,0.339161,0.548752}%
\pgfsetfillcolor{currentfill}%
\pgfsetfillopacity{0.700000}%
\pgfsetlinewidth{0.000000pt}%
\definecolor{currentstroke}{rgb}{0.000000,0.000000,0.000000}%
\pgfsetstrokecolor{currentstroke}%
\pgfsetdash{}{0pt}%
\pgfpathmoveto{\pgfqpoint{5.831914in}{2.688100in}}%
\pgfpathlineto{\pgfqpoint{5.846151in}{2.688204in}}%
\pgfpathlineto{\pgfqpoint{5.860399in}{2.688374in}}%
\pgfpathlineto{\pgfqpoint{5.874657in}{2.688609in}}%
\pgfpathlineto{\pgfqpoint{5.888926in}{2.688910in}}%
\pgfpathlineto{\pgfqpoint{5.896156in}{2.696194in}}%
\pgfpathlineto{\pgfqpoint{5.903389in}{2.703697in}}%
\pgfpathlineto{\pgfqpoint{5.910623in}{2.711426in}}%
\pgfpathlineto{\pgfqpoint{5.917861in}{2.719391in}}%
\pgfpathlineto{\pgfqpoint{5.903620in}{2.719541in}}%
\pgfpathlineto{\pgfqpoint{5.889390in}{2.719756in}}%
\pgfpathlineto{\pgfqpoint{5.875170in}{2.720037in}}%
\pgfpathlineto{\pgfqpoint{5.860961in}{2.720383in}}%
\pgfpathlineto{\pgfqpoint{5.853695in}{2.711961in}}%
\pgfpathlineto{\pgfqpoint{5.846432in}{2.703779in}}%
\pgfpathlineto{\pgfqpoint{5.839172in}{2.695828in}}%
\pgfpathlineto{\pgfqpoint{5.831914in}{2.688100in}}%
\pgfpathclose%
\pgfusepath{fill}%
\end{pgfscope}%
\begin{pgfscope}%
\pgfpathrectangle{\pgfqpoint{1.150000in}{0.150000in}}{\pgfqpoint{5.700000in}{5.700000in}}%
\pgfusepath{clip}%
\pgfsetbuttcap%
\pgfsetroundjoin%
\definecolor{currentfill}{rgb}{0.279574,0.170599,0.479997}%
\pgfsetfillcolor{currentfill}%
\pgfsetfillopacity{0.700000}%
\pgfsetlinewidth{0.000000pt}%
\definecolor{currentstroke}{rgb}{0.000000,0.000000,0.000000}%
\pgfsetstrokecolor{currentstroke}%
\pgfsetdash{}{0pt}%
\pgfpathmoveto{\pgfqpoint{2.359821in}{2.338247in}}%
\pgfpathlineto{\pgfqpoint{2.373395in}{2.326675in}}%
\pgfpathlineto{\pgfqpoint{2.386967in}{2.315237in}}%
\pgfpathlineto{\pgfqpoint{2.400536in}{2.303933in}}%
\pgfpathlineto{\pgfqpoint{2.414104in}{2.292761in}}%
\pgfpathlineto{\pgfqpoint{2.422711in}{2.298358in}}%
\pgfpathlineto{\pgfqpoint{2.431306in}{2.304075in}}%
\pgfpathlineto{\pgfqpoint{2.439890in}{2.309910in}}%
\pgfpathlineto{\pgfqpoint{2.448464in}{2.315861in}}%
\pgfpathlineto{\pgfqpoint{2.434920in}{2.326863in}}%
\pgfpathlineto{\pgfqpoint{2.421374in}{2.337997in}}%
\pgfpathlineto{\pgfqpoint{2.407826in}{2.349264in}}%
\pgfpathlineto{\pgfqpoint{2.394277in}{2.360666in}}%
\pgfpathlineto{\pgfqpoint{2.385680in}{2.354877in}}%
\pgfpathlineto{\pgfqpoint{2.377071in}{2.349210in}}%
\pgfpathlineto{\pgfqpoint{2.368452in}{2.343665in}}%
\pgfpathlineto{\pgfqpoint{2.359821in}{2.338247in}}%
\pgfpathclose%
\pgfusepath{fill}%
\end{pgfscope}%
\begin{pgfscope}%
\pgfpathrectangle{\pgfqpoint{1.150000in}{0.150000in}}{\pgfqpoint{5.700000in}{5.700000in}}%
\pgfusepath{clip}%
\pgfsetbuttcap%
\pgfsetroundjoin%
\definecolor{currentfill}{rgb}{0.278791,0.062145,0.386592}%
\pgfsetfillcolor{currentfill}%
\pgfsetfillopacity{0.700000}%
\pgfsetlinewidth{0.000000pt}%
\definecolor{currentstroke}{rgb}{0.000000,0.000000,0.000000}%
\pgfsetstrokecolor{currentstroke}%
\pgfsetdash{}{0pt}%
\pgfpathmoveto{\pgfqpoint{3.687107in}{2.099043in}}%
\pgfpathlineto{\pgfqpoint{3.700720in}{2.096091in}}%
\pgfpathlineto{\pgfqpoint{3.714340in}{2.093221in}}%
\pgfpathlineto{\pgfqpoint{3.727966in}{2.090433in}}%
\pgfpathlineto{\pgfqpoint{3.741598in}{2.087727in}}%
\pgfpathlineto{\pgfqpoint{3.749634in}{2.096625in}}%
\pgfpathlineto{\pgfqpoint{3.757666in}{2.105518in}}%
\pgfpathlineto{\pgfqpoint{3.765691in}{2.114407in}}%
\pgfpathlineto{\pgfqpoint{3.773711in}{2.123292in}}%
\pgfpathlineto{\pgfqpoint{3.760090in}{2.126017in}}%
\pgfpathlineto{\pgfqpoint{3.746475in}{2.128824in}}%
\pgfpathlineto{\pgfqpoint{3.732867in}{2.131714in}}%
\pgfpathlineto{\pgfqpoint{3.719264in}{2.134686in}}%
\pgfpathlineto{\pgfqpoint{3.711233in}{2.125774in}}%
\pgfpathlineto{\pgfqpoint{3.703197in}{2.116863in}}%
\pgfpathlineto{\pgfqpoint{3.695155in}{2.107953in}}%
\pgfpathlineto{\pgfqpoint{3.687107in}{2.099043in}}%
\pgfpathclose%
\pgfusepath{fill}%
\end{pgfscope}%
\begin{pgfscope}%
\pgfpathrectangle{\pgfqpoint{1.150000in}{0.150000in}}{\pgfqpoint{5.700000in}{5.700000in}}%
\pgfusepath{clip}%
\pgfsetbuttcap%
\pgfsetroundjoin%
\definecolor{currentfill}{rgb}{0.244972,0.287675,0.537260}%
\pgfsetfillcolor{currentfill}%
\pgfsetfillopacity{0.700000}%
\pgfsetlinewidth{0.000000pt}%
\definecolor{currentstroke}{rgb}{0.000000,0.000000,0.000000}%
\pgfsetstrokecolor{currentstroke}%
\pgfsetdash{}{0pt}%
\pgfpathmoveto{\pgfqpoint{5.431280in}{2.568497in}}%
\pgfpathlineto{\pgfqpoint{5.445405in}{2.568873in}}%
\pgfpathlineto{\pgfqpoint{5.459540in}{2.569317in}}%
\pgfpathlineto{\pgfqpoint{5.473686in}{2.569828in}}%
\pgfpathlineto{\pgfqpoint{5.487842in}{2.570406in}}%
\pgfpathlineto{\pgfqpoint{5.495206in}{2.577017in}}%
\pgfpathlineto{\pgfqpoint{5.502568in}{2.583757in}}%
\pgfpathlineto{\pgfqpoint{5.509927in}{2.590632in}}%
\pgfpathlineto{\pgfqpoint{5.517285in}{2.597648in}}%
\pgfpathlineto{\pgfqpoint{5.503152in}{2.597440in}}%
\pgfpathlineto{\pgfqpoint{5.489029in}{2.597298in}}%
\pgfpathlineto{\pgfqpoint{5.474916in}{2.597224in}}%
\pgfpathlineto{\pgfqpoint{5.460814in}{2.597217in}}%
\pgfpathlineto{\pgfqpoint{5.453433in}{2.589824in}}%
\pgfpathlineto{\pgfqpoint{5.446051in}{2.582578in}}%
\pgfpathlineto{\pgfqpoint{5.438666in}{2.575470in}}%
\pgfpathlineto{\pgfqpoint{5.431280in}{2.568497in}}%
\pgfpathclose%
\pgfusepath{fill}%
\end{pgfscope}%
\begin{pgfscope}%
\pgfpathrectangle{\pgfqpoint{1.150000in}{0.150000in}}{\pgfqpoint{5.700000in}{5.700000in}}%
\pgfusepath{clip}%
\pgfsetbuttcap%
\pgfsetroundjoin%
\definecolor{currentfill}{rgb}{0.274952,0.037752,0.364543}%
\pgfsetfillcolor{currentfill}%
\pgfsetfillopacity{0.700000}%
\pgfsetlinewidth{0.000000pt}%
\definecolor{currentstroke}{rgb}{0.000000,0.000000,0.000000}%
\pgfsetstrokecolor{currentstroke}%
\pgfsetdash{}{0pt}%
\pgfpathmoveto{\pgfqpoint{3.090058in}{2.067556in}}%
\pgfpathlineto{\pgfqpoint{3.103584in}{2.061535in}}%
\pgfpathlineto{\pgfqpoint{3.117114in}{2.055610in}}%
\pgfpathlineto{\pgfqpoint{3.130647in}{2.049782in}}%
\pgfpathlineto{\pgfqpoint{3.144183in}{2.044049in}}%
\pgfpathlineto{\pgfqpoint{3.152442in}{2.052360in}}%
\pgfpathlineto{\pgfqpoint{3.160693in}{2.060708in}}%
\pgfpathlineto{\pgfqpoint{3.168938in}{2.069093in}}%
\pgfpathlineto{\pgfqpoint{3.177177in}{2.077515in}}%
\pgfpathlineto{\pgfqpoint{3.163655in}{2.083164in}}%
\pgfpathlineto{\pgfqpoint{3.150137in}{2.088909in}}%
\pgfpathlineto{\pgfqpoint{3.136622in}{2.094750in}}%
\pgfpathlineto{\pgfqpoint{3.123110in}{2.100688in}}%
\pgfpathlineto{\pgfqpoint{3.114858in}{2.092343in}}%
\pgfpathlineto{\pgfqpoint{3.106598in}{2.084038in}}%
\pgfpathlineto{\pgfqpoint{3.098331in}{2.075776in}}%
\pgfpathlineto{\pgfqpoint{3.090058in}{2.067556in}}%
\pgfpathclose%
\pgfusepath{fill}%
\end{pgfscope}%
\begin{pgfscope}%
\pgfpathrectangle{\pgfqpoint{1.150000in}{0.150000in}}{\pgfqpoint{5.700000in}{5.700000in}}%
\pgfusepath{clip}%
\pgfsetbuttcap%
\pgfsetroundjoin%
\definecolor{currentfill}{rgb}{0.277018,0.050344,0.375715}%
\pgfsetfillcolor{currentfill}%
\pgfsetfillopacity{0.700000}%
\pgfsetlinewidth{0.000000pt}%
\definecolor{currentstroke}{rgb}{0.000000,0.000000,0.000000}%
\pgfsetstrokecolor{currentstroke}%
\pgfsetdash{}{0pt}%
\pgfpathmoveto{\pgfqpoint{2.948662in}{2.087683in}}%
\pgfpathlineto{\pgfqpoint{2.962181in}{2.080762in}}%
\pgfpathlineto{\pgfqpoint{2.975704in}{2.073942in}}%
\pgfpathlineto{\pgfqpoint{2.989228in}{2.067224in}}%
\pgfpathlineto{\pgfqpoint{3.002756in}{2.060606in}}%
\pgfpathlineto{\pgfqpoint{3.011074in}{2.068532in}}%
\pgfpathlineto{\pgfqpoint{3.019384in}{2.076511in}}%
\pgfpathlineto{\pgfqpoint{3.027687in}{2.084541in}}%
\pgfpathlineto{\pgfqpoint{3.035983in}{2.092621in}}%
\pgfpathlineto{\pgfqpoint{3.022472in}{2.099135in}}%
\pgfpathlineto{\pgfqpoint{3.008963in}{2.105750in}}%
\pgfpathlineto{\pgfqpoint{2.995457in}{2.112465in}}%
\pgfpathlineto{\pgfqpoint{2.981954in}{2.119282in}}%
\pgfpathlineto{\pgfqpoint{2.973642in}{2.111299in}}%
\pgfpathlineto{\pgfqpoint{2.965323in}{2.103371in}}%
\pgfpathlineto{\pgfqpoint{2.956996in}{2.095498in}}%
\pgfpathlineto{\pgfqpoint{2.948662in}{2.087683in}}%
\pgfpathclose%
\pgfusepath{fill}%
\end{pgfscope}%
\begin{pgfscope}%
\pgfpathrectangle{\pgfqpoint{1.150000in}{0.150000in}}{\pgfqpoint{5.700000in}{5.700000in}}%
\pgfusepath{clip}%
\pgfsetbuttcap%
\pgfsetroundjoin%
\definecolor{currentfill}{rgb}{0.282910,0.105393,0.426902}%
\pgfsetfillcolor{currentfill}%
\pgfsetfillopacity{0.700000}%
\pgfsetlinewidth{0.000000pt}%
\definecolor{currentstroke}{rgb}{0.000000,0.000000,0.000000}%
\pgfsetstrokecolor{currentstroke}%
\pgfsetdash{}{0pt}%
\pgfpathmoveto{\pgfqpoint{2.610904in}{2.193733in}}%
\pgfpathlineto{\pgfqpoint{2.624438in}{2.184349in}}%
\pgfpathlineto{\pgfqpoint{2.637971in}{2.175082in}}%
\pgfpathlineto{\pgfqpoint{2.651504in}{2.165932in}}%
\pgfpathlineto{\pgfqpoint{2.665038in}{2.156898in}}%
\pgfpathlineto{\pgfqpoint{2.673514in}{2.163584in}}%
\pgfpathlineto{\pgfqpoint{2.681980in}{2.170362in}}%
\pgfpathlineto{\pgfqpoint{2.690438in}{2.177229in}}%
\pgfpathlineto{\pgfqpoint{2.698886in}{2.184182in}}%
\pgfpathlineto{\pgfqpoint{2.685373in}{2.193070in}}%
\pgfpathlineto{\pgfqpoint{2.671859in}{2.202073in}}%
\pgfpathlineto{\pgfqpoint{2.658347in}{2.211193in}}%
\pgfpathlineto{\pgfqpoint{2.644834in}{2.220430in}}%
\pgfpathlineto{\pgfqpoint{2.636366in}{2.213615in}}%
\pgfpathlineto{\pgfqpoint{2.627888in}{2.206893in}}%
\pgfpathlineto{\pgfqpoint{2.619401in}{2.200265in}}%
\pgfpathlineto{\pgfqpoint{2.610904in}{2.193733in}}%
\pgfpathclose%
\pgfusepath{fill}%
\end{pgfscope}%
\begin{pgfscope}%
\pgfpathrectangle{\pgfqpoint{1.150000in}{0.150000in}}{\pgfqpoint{5.700000in}{5.700000in}}%
\pgfusepath{clip}%
\pgfsetbuttcap%
\pgfsetroundjoin%
\definecolor{currentfill}{rgb}{0.263663,0.237631,0.518762}%
\pgfsetfillcolor{currentfill}%
\pgfsetfillopacity{0.700000}%
\pgfsetlinewidth{0.000000pt}%
\definecolor{currentstroke}{rgb}{0.000000,0.000000,0.000000}%
\pgfsetstrokecolor{currentstroke}%
\pgfsetdash{}{0pt}%
\pgfpathmoveto{\pgfqpoint{5.030651in}{2.451270in}}%
\pgfpathlineto{\pgfqpoint{5.044653in}{2.451554in}}%
\pgfpathlineto{\pgfqpoint{5.058666in}{2.451908in}}%
\pgfpathlineto{\pgfqpoint{5.072688in}{2.452331in}}%
\pgfpathlineto{\pgfqpoint{5.086720in}{2.452823in}}%
\pgfpathlineto{\pgfqpoint{5.094245in}{2.459609in}}%
\pgfpathlineto{\pgfqpoint{5.101765in}{2.466455in}}%
\pgfpathlineto{\pgfqpoint{5.109281in}{2.473367in}}%
\pgfpathlineto{\pgfqpoint{5.116792in}{2.480350in}}%
\pgfpathlineto{\pgfqpoint{5.102779in}{2.480146in}}%
\pgfpathlineto{\pgfqpoint{5.088775in}{2.480010in}}%
\pgfpathlineto{\pgfqpoint{5.074780in}{2.479944in}}%
\pgfpathlineto{\pgfqpoint{5.060796in}{2.479947in}}%
\pgfpathlineto{\pgfqpoint{5.053266in}{2.472669in}}%
\pgfpathlineto{\pgfqpoint{5.045732in}{2.465467in}}%
\pgfpathlineto{\pgfqpoint{5.038194in}{2.458336in}}%
\pgfpathlineto{\pgfqpoint{5.030651in}{2.451270in}}%
\pgfpathclose%
\pgfusepath{fill}%
\end{pgfscope}%
\begin{pgfscope}%
\pgfpathrectangle{\pgfqpoint{1.150000in}{0.150000in}}{\pgfqpoint{5.700000in}{5.700000in}}%
\pgfusepath{clip}%
\pgfsetbuttcap%
\pgfsetroundjoin%
\definecolor{currentfill}{rgb}{0.277134,0.185228,0.489898}%
\pgfsetfillcolor{currentfill}%
\pgfsetfillopacity{0.700000}%
\pgfsetlinewidth{0.000000pt}%
\definecolor{currentstroke}{rgb}{0.000000,0.000000,0.000000}%
\pgfsetstrokecolor{currentstroke}%
\pgfsetdash{}{0pt}%
\pgfpathmoveto{\pgfqpoint{4.629942in}{2.333366in}}%
\pgfpathlineto{\pgfqpoint{4.643819in}{2.333181in}}%
\pgfpathlineto{\pgfqpoint{4.657704in}{2.333069in}}%
\pgfpathlineto{\pgfqpoint{4.671598in}{2.333029in}}%
\pgfpathlineto{\pgfqpoint{4.685502in}{2.333060in}}%
\pgfpathlineto{\pgfqpoint{4.693192in}{2.340506in}}%
\pgfpathlineto{\pgfqpoint{4.700876in}{2.347968in}}%
\pgfpathlineto{\pgfqpoint{4.708554in}{2.355447in}}%
\pgfpathlineto{\pgfqpoint{4.716227in}{2.362949in}}%
\pgfpathlineto{\pgfqpoint{4.702338in}{2.363123in}}%
\pgfpathlineto{\pgfqpoint{4.688457in}{2.363369in}}%
\pgfpathlineto{\pgfqpoint{4.674586in}{2.363686in}}%
\pgfpathlineto{\pgfqpoint{4.660724in}{2.364075in}}%
\pgfpathlineto{\pgfqpoint{4.653037in}{2.356361in}}%
\pgfpathlineto{\pgfqpoint{4.645344in}{2.348674in}}%
\pgfpathlineto{\pgfqpoint{4.637646in}{2.341010in}}%
\pgfpathlineto{\pgfqpoint{4.629942in}{2.333366in}}%
\pgfpathclose%
\pgfusepath{fill}%
\end{pgfscope}%
\begin{pgfscope}%
\pgfpathrectangle{\pgfqpoint{1.150000in}{0.150000in}}{\pgfqpoint{5.700000in}{5.700000in}}%
\pgfusepath{clip}%
\pgfsetbuttcap%
\pgfsetroundjoin%
\definecolor{currentfill}{rgb}{0.274952,0.037752,0.364543}%
\pgfsetfillcolor{currentfill}%
\pgfsetfillopacity{0.700000}%
\pgfsetlinewidth{0.000000pt}%
\definecolor{currentstroke}{rgb}{0.000000,0.000000,0.000000}%
\pgfsetstrokecolor{currentstroke}%
\pgfsetdash{}{0pt}%
\pgfpathmoveto{\pgfqpoint{3.231301in}{2.055862in}}%
\pgfpathlineto{\pgfqpoint{3.244842in}{2.050683in}}%
\pgfpathlineto{\pgfqpoint{3.258386in}{2.045596in}}%
\pgfpathlineto{\pgfqpoint{3.271935in}{2.040602in}}%
\pgfpathlineto{\pgfqpoint{3.285488in}{2.035698in}}%
\pgfpathlineto{\pgfqpoint{3.293692in}{2.044295in}}%
\pgfpathlineto{\pgfqpoint{3.301890in}{2.052917in}}%
\pgfpathlineto{\pgfqpoint{3.310082in}{2.061562in}}%
\pgfpathlineto{\pgfqpoint{3.318267in}{2.070231in}}%
\pgfpathlineto{\pgfqpoint{3.304727in}{2.075071in}}%
\pgfpathlineto{\pgfqpoint{3.291192in}{2.080004in}}%
\pgfpathlineto{\pgfqpoint{3.277660in}{2.085028in}}%
\pgfpathlineto{\pgfqpoint{3.264133in}{2.090144in}}%
\pgfpathlineto{\pgfqpoint{3.255935in}{2.081531in}}%
\pgfpathlineto{\pgfqpoint{3.247730in}{2.072945in}}%
\pgfpathlineto{\pgfqpoint{3.239519in}{2.064389in}}%
\pgfpathlineto{\pgfqpoint{3.231301in}{2.055862in}}%
\pgfpathclose%
\pgfusepath{fill}%
\end{pgfscope}%
\begin{pgfscope}%
\pgfpathrectangle{\pgfqpoint{1.150000in}{0.150000in}}{\pgfqpoint{5.700000in}{5.700000in}}%
\pgfusepath{clip}%
\pgfsetbuttcap%
\pgfsetroundjoin%
\definecolor{currentfill}{rgb}{0.281924,0.089666,0.412415}%
\pgfsetfillcolor{currentfill}%
\pgfsetfillopacity{0.700000}%
\pgfsetlinewidth{0.000000pt}%
\definecolor{currentstroke}{rgb}{0.000000,0.000000,0.000000}%
\pgfsetstrokecolor{currentstroke}%
\pgfsetdash{}{0pt}%
\pgfpathmoveto{\pgfqpoint{3.914849in}{2.139618in}}%
\pgfpathlineto{\pgfqpoint{3.928519in}{2.137579in}}%
\pgfpathlineto{\pgfqpoint{3.942196in}{2.135618in}}%
\pgfpathlineto{\pgfqpoint{3.955879in}{2.133737in}}%
\pgfpathlineto{\pgfqpoint{3.969571in}{2.131934in}}%
\pgfpathlineto{\pgfqpoint{3.977529in}{2.140660in}}%
\pgfpathlineto{\pgfqpoint{3.985482in}{2.149373in}}%
\pgfpathlineto{\pgfqpoint{3.993430in}{2.158077in}}%
\pgfpathlineto{\pgfqpoint{4.001372in}{2.166771in}}%
\pgfpathlineto{\pgfqpoint{3.987691in}{2.168635in}}%
\pgfpathlineto{\pgfqpoint{3.974018in}{2.170577in}}%
\pgfpathlineto{\pgfqpoint{3.960353in}{2.172598in}}%
\pgfpathlineto{\pgfqpoint{3.946694in}{2.174698in}}%
\pgfpathlineto{\pgfqpoint{3.938741in}{2.165935in}}%
\pgfpathlineto{\pgfqpoint{3.930783in}{2.157169in}}%
\pgfpathlineto{\pgfqpoint{3.922819in}{2.148397in}}%
\pgfpathlineto{\pgfqpoint{3.914849in}{2.139618in}}%
\pgfpathclose%
\pgfusepath{fill}%
\end{pgfscope}%
\begin{pgfscope}%
\pgfpathrectangle{\pgfqpoint{1.150000in}{0.150000in}}{\pgfqpoint{5.700000in}{5.700000in}}%
\pgfusepath{clip}%
\pgfsetbuttcap%
\pgfsetroundjoin%
\definecolor{currentfill}{rgb}{0.279566,0.067836,0.391917}%
\pgfsetfillcolor{currentfill}%
\pgfsetfillopacity{0.700000}%
\pgfsetlinewidth{0.000000pt}%
\definecolor{currentstroke}{rgb}{0.000000,0.000000,0.000000}%
\pgfsetstrokecolor{currentstroke}%
\pgfsetdash{}{0pt}%
\pgfpathmoveto{\pgfqpoint{2.807022in}{2.117141in}}%
\pgfpathlineto{\pgfqpoint{2.820544in}{2.109255in}}%
\pgfpathlineto{\pgfqpoint{2.834068in}{2.101478in}}%
\pgfpathlineto{\pgfqpoint{2.847593in}{2.093807in}}%
\pgfpathlineto{\pgfqpoint{2.861120in}{2.086242in}}%
\pgfpathlineto{\pgfqpoint{2.869503in}{2.093680in}}%
\pgfpathlineto{\pgfqpoint{2.877878in}{2.101187in}}%
\pgfpathlineto{\pgfqpoint{2.886245in}{2.108761in}}%
\pgfpathlineto{\pgfqpoint{2.894604in}{2.116400in}}%
\pgfpathlineto{\pgfqpoint{2.881095in}{2.123840in}}%
\pgfpathlineto{\pgfqpoint{2.867588in}{2.131386in}}%
\pgfpathlineto{\pgfqpoint{2.854082in}{2.139039in}}%
\pgfpathlineto{\pgfqpoint{2.840578in}{2.146798in}}%
\pgfpathlineto{\pgfqpoint{2.832201in}{2.139277in}}%
\pgfpathlineto{\pgfqpoint{2.823816in}{2.131826in}}%
\pgfpathlineto{\pgfqpoint{2.815423in}{2.124446in}}%
\pgfpathlineto{\pgfqpoint{2.807022in}{2.117141in}}%
\pgfpathclose%
\pgfusepath{fill}%
\end{pgfscope}%
\begin{pgfscope}%
\pgfpathrectangle{\pgfqpoint{1.150000in}{0.150000in}}{\pgfqpoint{5.700000in}{5.700000in}}%
\pgfusepath{clip}%
\pgfsetbuttcap%
\pgfsetroundjoin%
\definecolor{currentfill}{rgb}{0.225863,0.330805,0.547314}%
\pgfsetfillcolor{currentfill}%
\pgfsetfillopacity{0.700000}%
\pgfsetlinewidth{0.000000pt}%
\definecolor{currentstroke}{rgb}{0.000000,0.000000,0.000000}%
\pgfsetstrokecolor{currentstroke}%
\pgfsetdash{}{0pt}%
\pgfpathmoveto{\pgfqpoint{5.745950in}{2.657788in}}%
\pgfpathlineto{\pgfqpoint{5.760171in}{2.658060in}}%
\pgfpathlineto{\pgfqpoint{5.774403in}{2.658398in}}%
\pgfpathlineto{\pgfqpoint{5.788645in}{2.658801in}}%
\pgfpathlineto{\pgfqpoint{5.802898in}{2.659271in}}%
\pgfpathlineto{\pgfqpoint{5.810151in}{2.666181in}}%
\pgfpathlineto{\pgfqpoint{5.817404in}{2.673285in}}%
\pgfpathlineto{\pgfqpoint{5.824658in}{2.680588in}}%
\pgfpathlineto{\pgfqpoint{5.831914in}{2.688100in}}%
\pgfpathlineto{\pgfqpoint{5.817688in}{2.688062in}}%
\pgfpathlineto{\pgfqpoint{5.803473in}{2.688089in}}%
\pgfpathlineto{\pgfqpoint{5.789268in}{2.688182in}}%
\pgfpathlineto{\pgfqpoint{5.775074in}{2.688340in}}%
\pgfpathlineto{\pgfqpoint{5.767791in}{2.680391in}}%
\pgfpathlineto{\pgfqpoint{5.760509in}{2.672654in}}%
\pgfpathlineto{\pgfqpoint{5.753229in}{2.665122in}}%
\pgfpathlineto{\pgfqpoint{5.745950in}{2.657788in}}%
\pgfpathclose%
\pgfusepath{fill}%
\end{pgfscope}%
\begin{pgfscope}%
\pgfpathrectangle{\pgfqpoint{1.150000in}{0.150000in}}{\pgfqpoint{5.700000in}{5.700000in}}%
\pgfusepath{clip}%
\pgfsetbuttcap%
\pgfsetroundjoin%
\definecolor{currentfill}{rgb}{0.283187,0.125848,0.444960}%
\pgfsetfillcolor{currentfill}%
\pgfsetfillopacity{0.700000}%
\pgfsetlinewidth{0.000000pt}%
\definecolor{currentstroke}{rgb}{0.000000,0.000000,0.000000}%
\pgfsetstrokecolor{currentstroke}%
\pgfsetdash{}{0pt}%
\pgfpathmoveto{\pgfqpoint{4.229176in}{2.217715in}}%
\pgfpathlineto{\pgfqpoint{4.242933in}{2.216670in}}%
\pgfpathlineto{\pgfqpoint{4.256697in}{2.215701in}}%
\pgfpathlineto{\pgfqpoint{4.270469in}{2.214806in}}%
\pgfpathlineto{\pgfqpoint{4.284250in}{2.213987in}}%
\pgfpathlineto{\pgfqpoint{4.292095in}{2.222227in}}%
\pgfpathlineto{\pgfqpoint{4.299935in}{2.230457in}}%
\pgfpathlineto{\pgfqpoint{4.307769in}{2.238680in}}%
\pgfpathlineto{\pgfqpoint{4.315597in}{2.246899in}}%
\pgfpathlineto{\pgfqpoint{4.301829in}{2.247841in}}%
\pgfpathlineto{\pgfqpoint{4.288068in}{2.248858in}}%
\pgfpathlineto{\pgfqpoint{4.274315in}{2.249950in}}%
\pgfpathlineto{\pgfqpoint{4.260571in}{2.251117in}}%
\pgfpathlineto{\pgfqpoint{4.252731in}{2.242768in}}%
\pgfpathlineto{\pgfqpoint{4.244885in}{2.234420in}}%
\pgfpathlineto{\pgfqpoint{4.237033in}{2.226070in}}%
\pgfpathlineto{\pgfqpoint{4.229176in}{2.217715in}}%
\pgfpathclose%
\pgfusepath{fill}%
\end{pgfscope}%
\begin{pgfscope}%
\pgfpathrectangle{\pgfqpoint{1.150000in}{0.150000in}}{\pgfqpoint{5.700000in}{5.700000in}}%
\pgfusepath{clip}%
\pgfsetbuttcap%
\pgfsetroundjoin%
\definecolor{currentfill}{rgb}{0.248629,0.278775,0.534556}%
\pgfsetfillcolor{currentfill}%
\pgfsetfillopacity{0.700000}%
\pgfsetlinewidth{0.000000pt}%
\definecolor{currentstroke}{rgb}{0.000000,0.000000,0.000000}%
\pgfsetstrokecolor{currentstroke}%
\pgfsetdash{}{0pt}%
\pgfpathmoveto{\pgfqpoint{5.345224in}{2.539577in}}%
\pgfpathlineto{\pgfqpoint{5.359330in}{2.540033in}}%
\pgfpathlineto{\pgfqpoint{5.373445in}{2.540557in}}%
\pgfpathlineto{\pgfqpoint{5.387572in}{2.541148in}}%
\pgfpathlineto{\pgfqpoint{5.401708in}{2.541807in}}%
\pgfpathlineto{\pgfqpoint{5.409105in}{2.548311in}}%
\pgfpathlineto{\pgfqpoint{5.416500in}{2.554923in}}%
\pgfpathlineto{\pgfqpoint{5.423891in}{2.561650in}}%
\pgfpathlineto{\pgfqpoint{5.431280in}{2.568497in}}%
\pgfpathlineto{\pgfqpoint{5.417165in}{2.568188in}}%
\pgfpathlineto{\pgfqpoint{5.403061in}{2.567946in}}%
\pgfpathlineto{\pgfqpoint{5.388967in}{2.567772in}}%
\pgfpathlineto{\pgfqpoint{5.374883in}{2.567665in}}%
\pgfpathlineto{\pgfqpoint{5.367472in}{2.560461in}}%
\pgfpathlineto{\pgfqpoint{5.360059in}{2.553383in}}%
\pgfpathlineto{\pgfqpoint{5.352643in}{2.546424in}}%
\pgfpathlineto{\pgfqpoint{5.345224in}{2.539577in}}%
\pgfpathclose%
\pgfusepath{fill}%
\end{pgfscope}%
\begin{pgfscope}%
\pgfpathrectangle{\pgfqpoint{1.150000in}{0.150000in}}{\pgfqpoint{5.700000in}{5.700000in}}%
\pgfusepath{clip}%
\pgfsetbuttcap%
\pgfsetroundjoin%
\definecolor{currentfill}{rgb}{0.281887,0.150881,0.465405}%
\pgfsetfillcolor{currentfill}%
\pgfsetfillopacity{0.700000}%
\pgfsetlinewidth{0.000000pt}%
\definecolor{currentstroke}{rgb}{0.000000,0.000000,0.000000}%
\pgfsetstrokecolor{currentstroke}%
\pgfsetdash{}{0pt}%
\pgfpathmoveto{\pgfqpoint{2.414104in}{2.292761in}}%
\pgfpathlineto{\pgfqpoint{2.427670in}{2.281720in}}%
\pgfpathlineto{\pgfqpoint{2.441235in}{2.270809in}}%
\pgfpathlineto{\pgfqpoint{2.454798in}{2.260027in}}%
\pgfpathlineto{\pgfqpoint{2.468360in}{2.249373in}}%
\pgfpathlineto{\pgfqpoint{2.476942in}{2.255147in}}%
\pgfpathlineto{\pgfqpoint{2.485514in}{2.261036in}}%
\pgfpathlineto{\pgfqpoint{2.494074in}{2.267038in}}%
\pgfpathlineto{\pgfqpoint{2.502625in}{2.273150in}}%
\pgfpathlineto{\pgfqpoint{2.489087in}{2.283635in}}%
\pgfpathlineto{\pgfqpoint{2.475547in}{2.294248in}}%
\pgfpathlineto{\pgfqpoint{2.462006in}{2.304990in}}%
\pgfpathlineto{\pgfqpoint{2.448464in}{2.315861in}}%
\pgfpathlineto{\pgfqpoint{2.439890in}{2.309910in}}%
\pgfpathlineto{\pgfqpoint{2.431306in}{2.304075in}}%
\pgfpathlineto{\pgfqpoint{2.422711in}{2.298358in}}%
\pgfpathlineto{\pgfqpoint{2.414104in}{2.292761in}}%
\pgfpathclose%
\pgfusepath{fill}%
\end{pgfscope}%
\begin{pgfscope}%
\pgfpathrectangle{\pgfqpoint{1.150000in}{0.150000in}}{\pgfqpoint{5.700000in}{5.700000in}}%
\pgfusepath{clip}%
\pgfsetbuttcap%
\pgfsetroundjoin%
\definecolor{currentfill}{rgb}{0.277018,0.050344,0.375715}%
\pgfsetfillcolor{currentfill}%
\pgfsetfillopacity{0.700000}%
\pgfsetlinewidth{0.000000pt}%
\definecolor{currentstroke}{rgb}{0.000000,0.000000,0.000000}%
\pgfsetstrokecolor{currentstroke}%
\pgfsetdash{}{0pt}%
\pgfpathmoveto{\pgfqpoint{3.600418in}{2.076041in}}%
\pgfpathlineto{\pgfqpoint{3.614019in}{2.072753in}}%
\pgfpathlineto{\pgfqpoint{3.627626in}{2.069549in}}%
\pgfpathlineto{\pgfqpoint{3.641239in}{2.066429in}}%
\pgfpathlineto{\pgfqpoint{3.654857in}{2.063392in}}%
\pgfpathlineto{\pgfqpoint{3.662928in}{2.072308in}}%
\pgfpathlineto{\pgfqpoint{3.670993in}{2.081221in}}%
\pgfpathlineto{\pgfqpoint{3.679053in}{2.090133in}}%
\pgfpathlineto{\pgfqpoint{3.687107in}{2.099043in}}%
\pgfpathlineto{\pgfqpoint{3.673499in}{2.102079in}}%
\pgfpathlineto{\pgfqpoint{3.659898in}{2.105198in}}%
\pgfpathlineto{\pgfqpoint{3.646302in}{2.108401in}}%
\pgfpathlineto{\pgfqpoint{3.632712in}{2.111688in}}%
\pgfpathlineto{\pgfqpoint{3.624647in}{2.102771in}}%
\pgfpathlineto{\pgfqpoint{3.616577in}{2.093858in}}%
\pgfpathlineto{\pgfqpoint{3.608500in}{2.084948in}}%
\pgfpathlineto{\pgfqpoint{3.600418in}{2.076041in}}%
\pgfpathclose%
\pgfusepath{fill}%
\end{pgfscope}%
\begin{pgfscope}%
\pgfpathrectangle{\pgfqpoint{1.150000in}{0.150000in}}{\pgfqpoint{5.700000in}{5.700000in}}%
\pgfusepath{clip}%
\pgfsetbuttcap%
\pgfsetroundjoin%
\definecolor{currentfill}{rgb}{0.266580,0.228262,0.514349}%
\pgfsetfillcolor{currentfill}%
\pgfsetfillopacity{0.700000}%
\pgfsetlinewidth{0.000000pt}%
\definecolor{currentstroke}{rgb}{0.000000,0.000000,0.000000}%
\pgfsetstrokecolor{currentstroke}%
\pgfsetdash{}{0pt}%
\pgfpathmoveto{\pgfqpoint{4.944451in}{2.422047in}}%
\pgfpathlineto{\pgfqpoint{4.958432in}{2.422321in}}%
\pgfpathlineto{\pgfqpoint{4.972422in}{2.422664in}}%
\pgfpathlineto{\pgfqpoint{4.986422in}{2.423076in}}%
\pgfpathlineto{\pgfqpoint{5.000432in}{2.423559in}}%
\pgfpathlineto{\pgfqpoint{5.007994in}{2.430413in}}%
\pgfpathlineto{\pgfqpoint{5.015551in}{2.437313in}}%
\pgfpathlineto{\pgfqpoint{5.023104in}{2.444264in}}%
\pgfpathlineto{\pgfqpoint{5.030651in}{2.451270in}}%
\pgfpathlineto{\pgfqpoint{5.016659in}{2.451055in}}%
\pgfpathlineto{\pgfqpoint{5.002676in}{2.450909in}}%
\pgfpathlineto{\pgfqpoint{4.988703in}{2.450833in}}%
\pgfpathlineto{\pgfqpoint{4.974739in}{2.450827in}}%
\pgfpathlineto{\pgfqpoint{4.967174in}{2.443547in}}%
\pgfpathlineto{\pgfqpoint{4.959605in}{2.436327in}}%
\pgfpathlineto{\pgfqpoint{4.952031in}{2.429162in}}%
\pgfpathlineto{\pgfqpoint{4.944451in}{2.422047in}}%
\pgfpathclose%
\pgfusepath{fill}%
\end{pgfscope}%
\begin{pgfscope}%
\pgfpathrectangle{\pgfqpoint{1.150000in}{0.150000in}}{\pgfqpoint{5.700000in}{5.700000in}}%
\pgfusepath{clip}%
\pgfsetbuttcap%
\pgfsetroundjoin%
\definecolor{currentfill}{rgb}{0.274952,0.037752,0.364543}%
\pgfsetfillcolor{currentfill}%
\pgfsetfillopacity{0.700000}%
\pgfsetlinewidth{0.000000pt}%
\definecolor{currentstroke}{rgb}{0.000000,0.000000,0.000000}%
\pgfsetstrokecolor{currentstroke}%
\pgfsetdash{}{0pt}%
\pgfpathmoveto{\pgfqpoint{3.372470in}{2.051774in}}%
\pgfpathlineto{\pgfqpoint{3.386032in}{2.047384in}}%
\pgfpathlineto{\pgfqpoint{3.399600in}{2.043082in}}%
\pgfpathlineto{\pgfqpoint{3.413171in}{2.038869in}}%
\pgfpathlineto{\pgfqpoint{3.426748in}{2.034744in}}%
\pgfpathlineto{\pgfqpoint{3.434902in}{2.043535in}}%
\pgfpathlineto{\pgfqpoint{3.443049in}{2.052339in}}%
\pgfpathlineto{\pgfqpoint{3.451191in}{2.061155in}}%
\pgfpathlineto{\pgfqpoint{3.459326in}{2.069983in}}%
\pgfpathlineto{\pgfqpoint{3.445761in}{2.074066in}}%
\pgfpathlineto{\pgfqpoint{3.432202in}{2.078237in}}%
\pgfpathlineto{\pgfqpoint{3.418647in}{2.082496in}}%
\pgfpathlineto{\pgfqpoint{3.405097in}{2.086844in}}%
\pgfpathlineto{\pgfqpoint{3.396950in}{2.078050in}}%
\pgfpathlineto{\pgfqpoint{3.388796in}{2.069274in}}%
\pgfpathlineto{\pgfqpoint{3.380636in}{2.060515in}}%
\pgfpathlineto{\pgfqpoint{3.372470in}{2.051774in}}%
\pgfpathclose%
\pgfusepath{fill}%
\end{pgfscope}%
\begin{pgfscope}%
\pgfpathrectangle{\pgfqpoint{1.150000in}{0.150000in}}{\pgfqpoint{5.700000in}{5.700000in}}%
\pgfusepath{clip}%
\pgfsetbuttcap%
\pgfsetroundjoin%
\definecolor{currentfill}{rgb}{0.279574,0.170599,0.479997}%
\pgfsetfillcolor{currentfill}%
\pgfsetfillopacity{0.700000}%
\pgfsetlinewidth{0.000000pt}%
\definecolor{currentstroke}{rgb}{0.000000,0.000000,0.000000}%
\pgfsetstrokecolor{currentstroke}%
\pgfsetdash{}{0pt}%
\pgfpathmoveto{\pgfqpoint{4.543601in}{2.303634in}}%
\pgfpathlineto{\pgfqpoint{4.557456in}{2.303345in}}%
\pgfpathlineto{\pgfqpoint{4.571319in}{2.303128in}}%
\pgfpathlineto{\pgfqpoint{4.585191in}{2.302984in}}%
\pgfpathlineto{\pgfqpoint{4.599072in}{2.302913in}}%
\pgfpathlineto{\pgfqpoint{4.606798in}{2.310514in}}%
\pgfpathlineto{\pgfqpoint{4.614518in}{2.318122in}}%
\pgfpathlineto{\pgfqpoint{4.622233in}{2.325737in}}%
\pgfpathlineto{\pgfqpoint{4.629942in}{2.333366in}}%
\pgfpathlineto{\pgfqpoint{4.616075in}{2.333622in}}%
\pgfpathlineto{\pgfqpoint{4.602216in}{2.333951in}}%
\pgfpathlineto{\pgfqpoint{4.588367in}{2.334352in}}%
\pgfpathlineto{\pgfqpoint{4.574526in}{2.334825in}}%
\pgfpathlineto{\pgfqpoint{4.566803in}{2.327005in}}%
\pgfpathlineto{\pgfqpoint{4.559075in}{2.319202in}}%
\pgfpathlineto{\pgfqpoint{4.551341in}{2.311413in}}%
\pgfpathlineto{\pgfqpoint{4.543601in}{2.303634in}}%
\pgfpathclose%
\pgfusepath{fill}%
\end{pgfscope}%
\begin{pgfscope}%
\pgfpathrectangle{\pgfqpoint{1.150000in}{0.150000in}}{\pgfqpoint{5.700000in}{5.700000in}}%
\pgfusepath{clip}%
\pgfsetbuttcap%
\pgfsetroundjoin%
\definecolor{currentfill}{rgb}{0.231674,0.318106,0.544834}%
\pgfsetfillcolor{currentfill}%
\pgfsetfillopacity{0.700000}%
\pgfsetlinewidth{0.000000pt}%
\definecolor{currentstroke}{rgb}{0.000000,0.000000,0.000000}%
\pgfsetstrokecolor{currentstroke}%
\pgfsetdash{}{0pt}%
\pgfpathmoveto{\pgfqpoint{5.659956in}{2.628211in}}%
\pgfpathlineto{\pgfqpoint{5.674160in}{2.628629in}}%
\pgfpathlineto{\pgfqpoint{5.688375in}{2.629114in}}%
\pgfpathlineto{\pgfqpoint{5.702600in}{2.629665in}}%
\pgfpathlineto{\pgfqpoint{5.716836in}{2.630282in}}%
\pgfpathlineto{\pgfqpoint{5.724115in}{2.636898in}}%
\pgfpathlineto{\pgfqpoint{5.731393in}{2.643684in}}%
\pgfpathlineto{\pgfqpoint{5.738671in}{2.650644in}}%
\pgfpathlineto{\pgfqpoint{5.745950in}{2.657788in}}%
\pgfpathlineto{\pgfqpoint{5.731740in}{2.657582in}}%
\pgfpathlineto{\pgfqpoint{5.717540in}{2.657443in}}%
\pgfpathlineto{\pgfqpoint{5.703351in}{2.657369in}}%
\pgfpathlineto{\pgfqpoint{5.689173in}{2.657362in}}%
\pgfpathlineto{\pgfqpoint{5.681869in}{2.649800in}}%
\pgfpathlineto{\pgfqpoint{5.674564in}{2.642426in}}%
\pgfpathlineto{\pgfqpoint{5.667260in}{2.635232in}}%
\pgfpathlineto{\pgfqpoint{5.659956in}{2.628211in}}%
\pgfpathclose%
\pgfusepath{fill}%
\end{pgfscope}%
\begin{pgfscope}%
\pgfpathrectangle{\pgfqpoint{1.150000in}{0.150000in}}{\pgfqpoint{5.700000in}{5.700000in}}%
\pgfusepath{clip}%
\pgfsetbuttcap%
\pgfsetroundjoin%
\definecolor{currentfill}{rgb}{0.280894,0.078907,0.402329}%
\pgfsetfillcolor{currentfill}%
\pgfsetfillopacity{0.700000}%
\pgfsetlinewidth{0.000000pt}%
\definecolor{currentstroke}{rgb}{0.000000,0.000000,0.000000}%
\pgfsetstrokecolor{currentstroke}%
\pgfsetdash{}{0pt}%
\pgfpathmoveto{\pgfqpoint{3.828259in}{2.113207in}}%
\pgfpathlineto{\pgfqpoint{3.841913in}{2.110888in}}%
\pgfpathlineto{\pgfqpoint{3.855573in}{2.108649in}}%
\pgfpathlineto{\pgfqpoint{3.869240in}{2.106490in}}%
\pgfpathlineto{\pgfqpoint{3.882914in}{2.104411in}}%
\pgfpathlineto{\pgfqpoint{3.890906in}{2.113229in}}%
\pgfpathlineto{\pgfqpoint{3.898893in}{2.122036in}}%
\pgfpathlineto{\pgfqpoint{3.906874in}{2.130832in}}%
\pgfpathlineto{\pgfqpoint{3.914849in}{2.139618in}}%
\pgfpathlineto{\pgfqpoint{3.901186in}{2.141737in}}%
\pgfpathlineto{\pgfqpoint{3.887530in}{2.143936in}}%
\pgfpathlineto{\pgfqpoint{3.873881in}{2.146215in}}%
\pgfpathlineto{\pgfqpoint{3.860238in}{2.148574in}}%
\pgfpathlineto{\pgfqpoint{3.852252in}{2.139740in}}%
\pgfpathlineto{\pgfqpoint{3.844260in}{2.130901in}}%
\pgfpathlineto{\pgfqpoint{3.836262in}{2.122058in}}%
\pgfpathlineto{\pgfqpoint{3.828259in}{2.113207in}}%
\pgfpathclose%
\pgfusepath{fill}%
\end{pgfscope}%
\begin{pgfscope}%
\pgfpathrectangle{\pgfqpoint{1.150000in}{0.150000in}}{\pgfqpoint{5.700000in}{5.700000in}}%
\pgfusepath{clip}%
\pgfsetbuttcap%
\pgfsetroundjoin%
\definecolor{currentfill}{rgb}{0.281924,0.089666,0.412415}%
\pgfsetfillcolor{currentfill}%
\pgfsetfillopacity{0.700000}%
\pgfsetlinewidth{0.000000pt}%
\definecolor{currentstroke}{rgb}{0.000000,0.000000,0.000000}%
\pgfsetstrokecolor{currentstroke}%
\pgfsetdash{}{0pt}%
\pgfpathmoveto{\pgfqpoint{2.665038in}{2.156898in}}%
\pgfpathlineto{\pgfqpoint{2.678572in}{2.147979in}}%
\pgfpathlineto{\pgfqpoint{2.692107in}{2.139173in}}%
\pgfpathlineto{\pgfqpoint{2.705642in}{2.130481in}}%
\pgfpathlineto{\pgfqpoint{2.719178in}{2.121901in}}%
\pgfpathlineto{\pgfqpoint{2.727634in}{2.128741in}}%
\pgfpathlineto{\pgfqpoint{2.736080in}{2.135668in}}%
\pgfpathlineto{\pgfqpoint{2.744518in}{2.142678in}}%
\pgfpathlineto{\pgfqpoint{2.752946in}{2.149771in}}%
\pgfpathlineto{\pgfqpoint{2.739430in}{2.158205in}}%
\pgfpathlineto{\pgfqpoint{2.725915in}{2.166751in}}%
\pgfpathlineto{\pgfqpoint{2.712400in}{2.175410in}}%
\pgfpathlineto{\pgfqpoint{2.698886in}{2.184182in}}%
\pgfpathlineto{\pgfqpoint{2.690438in}{2.177229in}}%
\pgfpathlineto{\pgfqpoint{2.681980in}{2.170362in}}%
\pgfpathlineto{\pgfqpoint{2.673514in}{2.163584in}}%
\pgfpathlineto{\pgfqpoint{2.665038in}{2.156898in}}%
\pgfpathclose%
\pgfusepath{fill}%
\end{pgfscope}%
\begin{pgfscope}%
\pgfpathrectangle{\pgfqpoint{1.150000in}{0.150000in}}{\pgfqpoint{5.700000in}{5.700000in}}%
\pgfusepath{clip}%
\pgfsetbuttcap%
\pgfsetroundjoin%
\definecolor{currentfill}{rgb}{0.283197,0.115680,0.436115}%
\pgfsetfillcolor{currentfill}%
\pgfsetfillopacity{0.700000}%
\pgfsetlinewidth{0.000000pt}%
\definecolor{currentstroke}{rgb}{0.000000,0.000000,0.000000}%
\pgfsetstrokecolor{currentstroke}%
\pgfsetdash{}{0pt}%
\pgfpathmoveto{\pgfqpoint{4.142699in}{2.188741in}}%
\pgfpathlineto{\pgfqpoint{4.156435in}{2.187494in}}%
\pgfpathlineto{\pgfqpoint{4.170179in}{2.186323in}}%
\pgfpathlineto{\pgfqpoint{4.183931in}{2.185228in}}%
\pgfpathlineto{\pgfqpoint{4.197691in}{2.184209in}}%
\pgfpathlineto{\pgfqpoint{4.205571in}{2.192604in}}%
\pgfpathlineto{\pgfqpoint{4.213445in}{2.200985in}}%
\pgfpathlineto{\pgfqpoint{4.221314in}{2.209354in}}%
\pgfpathlineto{\pgfqpoint{4.229176in}{2.217715in}}%
\pgfpathlineto{\pgfqpoint{4.215428in}{2.218836in}}%
\pgfpathlineto{\pgfqpoint{4.201688in}{2.220033in}}%
\pgfpathlineto{\pgfqpoint{4.187955in}{2.221306in}}%
\pgfpathlineto{\pgfqpoint{4.174230in}{2.222655in}}%
\pgfpathlineto{\pgfqpoint{4.166356in}{2.214184in}}%
\pgfpathlineto{\pgfqpoint{4.158476in}{2.205710in}}%
\pgfpathlineto{\pgfqpoint{4.150590in}{2.197230in}}%
\pgfpathlineto{\pgfqpoint{4.142699in}{2.188741in}}%
\pgfpathclose%
\pgfusepath{fill}%
\end{pgfscope}%
\begin{pgfscope}%
\pgfpathrectangle{\pgfqpoint{1.150000in}{0.150000in}}{\pgfqpoint{5.700000in}{5.700000in}}%
\pgfusepath{clip}%
\pgfsetbuttcap%
\pgfsetroundjoin%
\definecolor{currentfill}{rgb}{0.252194,0.269783,0.531579}%
\pgfsetfillcolor{currentfill}%
\pgfsetfillopacity{0.700000}%
\pgfsetlinewidth{0.000000pt}%
\definecolor{currentstroke}{rgb}{0.000000,0.000000,0.000000}%
\pgfsetstrokecolor{currentstroke}%
\pgfsetdash{}{0pt}%
\pgfpathmoveto{\pgfqpoint{5.259115in}{2.510739in}}%
\pgfpathlineto{\pgfqpoint{5.273200in}{2.511253in}}%
\pgfpathlineto{\pgfqpoint{5.287295in}{2.511834in}}%
\pgfpathlineto{\pgfqpoint{5.301400in}{2.512484in}}%
\pgfpathlineto{\pgfqpoint{5.315516in}{2.513202in}}%
\pgfpathlineto{\pgfqpoint{5.322949in}{2.519656in}}%
\pgfpathlineto{\pgfqpoint{5.330377in}{2.526199in}}%
\pgfpathlineto{\pgfqpoint{5.337803in}{2.532838in}}%
\pgfpathlineto{\pgfqpoint{5.345224in}{2.539577in}}%
\pgfpathlineto{\pgfqpoint{5.331129in}{2.539189in}}%
\pgfpathlineto{\pgfqpoint{5.317045in}{2.538869in}}%
\pgfpathlineto{\pgfqpoint{5.302970in}{2.538617in}}%
\pgfpathlineto{\pgfqpoint{5.288905in}{2.538433in}}%
\pgfpathlineto{\pgfqpoint{5.281463in}{2.531356in}}%
\pgfpathlineto{\pgfqpoint{5.274017in}{2.524386in}}%
\pgfpathlineto{\pgfqpoint{5.266567in}{2.517516in}}%
\pgfpathlineto{\pgfqpoint{5.259115in}{2.510739in}}%
\pgfpathclose%
\pgfusepath{fill}%
\end{pgfscope}%
\begin{pgfscope}%
\pgfpathrectangle{\pgfqpoint{1.150000in}{0.150000in}}{\pgfqpoint{5.700000in}{5.700000in}}%
\pgfusepath{clip}%
\pgfsetbuttcap%
\pgfsetroundjoin%
\definecolor{currentfill}{rgb}{0.270595,0.214069,0.507052}%
\pgfsetfillcolor{currentfill}%
\pgfsetfillopacity{0.700000}%
\pgfsetlinewidth{0.000000pt}%
\definecolor{currentstroke}{rgb}{0.000000,0.000000,0.000000}%
\pgfsetstrokecolor{currentstroke}%
\pgfsetdash{}{0pt}%
\pgfpathmoveto{\pgfqpoint{4.858193in}{2.392625in}}%
\pgfpathlineto{\pgfqpoint{4.872151in}{2.392864in}}%
\pgfpathlineto{\pgfqpoint{4.886119in}{2.393174in}}%
\pgfpathlineto{\pgfqpoint{4.900097in}{2.393554in}}%
\pgfpathlineto{\pgfqpoint{4.914084in}{2.394004in}}%
\pgfpathlineto{\pgfqpoint{4.921684in}{2.400962in}}%
\pgfpathlineto{\pgfqpoint{4.929278in}{2.407952in}}%
\pgfpathlineto{\pgfqpoint{4.936867in}{2.414979in}}%
\pgfpathlineto{\pgfqpoint{4.944451in}{2.422047in}}%
\pgfpathlineto{\pgfqpoint{4.930481in}{2.421844in}}%
\pgfpathlineto{\pgfqpoint{4.916519in}{2.421711in}}%
\pgfpathlineto{\pgfqpoint{4.902568in}{2.421649in}}%
\pgfpathlineto{\pgfqpoint{4.888626in}{2.421656in}}%
\pgfpathlineto{\pgfqpoint{4.881025in}{2.414333in}}%
\pgfpathlineto{\pgfqpoint{4.873420in}{2.407057in}}%
\pgfpathlineto{\pgfqpoint{4.865809in}{2.399823in}}%
\pgfpathlineto{\pgfqpoint{4.858193in}{2.392625in}}%
\pgfpathclose%
\pgfusepath{fill}%
\end{pgfscope}%
\begin{pgfscope}%
\pgfpathrectangle{\pgfqpoint{1.150000in}{0.150000in}}{\pgfqpoint{5.700000in}{5.700000in}}%
\pgfusepath{clip}%
\pgfsetbuttcap%
\pgfsetroundjoin%
\definecolor{currentfill}{rgb}{0.212395,0.359683,0.551710}%
\pgfsetfillcolor{currentfill}%
\pgfsetfillopacity{0.700000}%
\pgfsetlinewidth{0.000000pt}%
\definecolor{currentstroke}{rgb}{0.000000,0.000000,0.000000}%
\pgfsetstrokecolor{currentstroke}%
\pgfsetdash{}{0pt}%
\pgfpathmoveto{\pgfqpoint{5.974933in}{2.719443in}}%
\pgfpathlineto{\pgfqpoint{5.989228in}{2.719619in}}%
\pgfpathlineto{\pgfqpoint{6.003534in}{2.719860in}}%
\pgfpathlineto{\pgfqpoint{6.017851in}{2.720166in}}%
\pgfpathlineto{\pgfqpoint{6.025041in}{2.727562in}}%
\pgfpathlineto{\pgfqpoint{6.032234in}{2.735199in}}%
\pgfpathlineto{\pgfqpoint{6.039432in}{2.743088in}}%
\pgfpathlineto{\pgfqpoint{6.046634in}{2.751237in}}%
\pgfpathlineto{\pgfqpoint{6.032347in}{2.751402in}}%
\pgfpathlineto{\pgfqpoint{6.018071in}{2.751632in}}%
\pgfpathlineto{\pgfqpoint{6.003805in}{2.751927in}}%
\pgfpathlineto{\pgfqpoint{5.996581in}{2.743420in}}%
\pgfpathlineto{\pgfqpoint{5.989361in}{2.735176in}}%
\pgfpathlineto{\pgfqpoint{5.982145in}{2.727186in}}%
\pgfpathlineto{\pgfqpoint{5.974933in}{2.719443in}}%
\pgfpathclose%
\pgfusepath{fill}%
\end{pgfscope}%
\begin{pgfscope}%
\pgfpathrectangle{\pgfqpoint{1.150000in}{0.150000in}}{\pgfqpoint{5.700000in}{5.700000in}}%
\pgfusepath{clip}%
\pgfsetbuttcap%
\pgfsetroundjoin%
\definecolor{currentfill}{rgb}{0.282884,0.135920,0.453427}%
\pgfsetfillcolor{currentfill}%
\pgfsetfillopacity{0.700000}%
\pgfsetlinewidth{0.000000pt}%
\definecolor{currentstroke}{rgb}{0.000000,0.000000,0.000000}%
\pgfsetstrokecolor{currentstroke}%
\pgfsetdash{}{0pt}%
\pgfpathmoveto{\pgfqpoint{2.468360in}{2.249373in}}%
\pgfpathlineto{\pgfqpoint{2.481920in}{2.238845in}}%
\pgfpathlineto{\pgfqpoint{2.495480in}{2.228443in}}%
\pgfpathlineto{\pgfqpoint{2.509038in}{2.218165in}}%
\pgfpathlineto{\pgfqpoint{2.522595in}{2.208011in}}%
\pgfpathlineto{\pgfqpoint{2.531154in}{2.213961in}}%
\pgfpathlineto{\pgfqpoint{2.539703in}{2.220021in}}%
\pgfpathlineto{\pgfqpoint{2.548241in}{2.226189in}}%
\pgfpathlineto{\pgfqpoint{2.556769in}{2.232463in}}%
\pgfpathlineto{\pgfqpoint{2.543234in}{2.242449in}}%
\pgfpathlineto{\pgfqpoint{2.529699in}{2.252558in}}%
\pgfpathlineto{\pgfqpoint{2.516162in}{2.262791in}}%
\pgfpathlineto{\pgfqpoint{2.502625in}{2.273150in}}%
\pgfpathlineto{\pgfqpoint{2.494074in}{2.267038in}}%
\pgfpathlineto{\pgfqpoint{2.485514in}{2.261036in}}%
\pgfpathlineto{\pgfqpoint{2.476942in}{2.255147in}}%
\pgfpathlineto{\pgfqpoint{2.468360in}{2.249373in}}%
\pgfpathclose%
\pgfusepath{fill}%
\end{pgfscope}%
\begin{pgfscope}%
\pgfpathrectangle{\pgfqpoint{1.150000in}{0.150000in}}{\pgfqpoint{5.700000in}{5.700000in}}%
\pgfusepath{clip}%
\pgfsetbuttcap%
\pgfsetroundjoin%
\definecolor{currentfill}{rgb}{0.276022,0.044167,0.370164}%
\pgfsetfillcolor{currentfill}%
\pgfsetfillopacity{0.700000}%
\pgfsetlinewidth{0.000000pt}%
\definecolor{currentstroke}{rgb}{0.000000,0.000000,0.000000}%
\pgfsetstrokecolor{currentstroke}%
\pgfsetdash{}{0pt}%
\pgfpathmoveto{\pgfqpoint{3.002756in}{2.060606in}}%
\pgfpathlineto{\pgfqpoint{3.016286in}{2.054088in}}%
\pgfpathlineto{\pgfqpoint{3.029819in}{2.047669in}}%
\pgfpathlineto{\pgfqpoint{3.043354in}{2.041348in}}%
\pgfpathlineto{\pgfqpoint{3.056893in}{2.035126in}}%
\pgfpathlineto{\pgfqpoint{3.065195in}{2.043164in}}%
\pgfpathlineto{\pgfqpoint{3.073490in}{2.051249in}}%
\pgfpathlineto{\pgfqpoint{3.081777in}{2.059380in}}%
\pgfpathlineto{\pgfqpoint{3.090058in}{2.067556in}}%
\pgfpathlineto{\pgfqpoint{3.076535in}{2.073675in}}%
\pgfpathlineto{\pgfqpoint{3.063015in}{2.079891in}}%
\pgfpathlineto{\pgfqpoint{3.049498in}{2.086207in}}%
\pgfpathlineto{\pgfqpoint{3.035983in}{2.092621in}}%
\pgfpathlineto{\pgfqpoint{3.027687in}{2.084541in}}%
\pgfpathlineto{\pgfqpoint{3.019384in}{2.076511in}}%
\pgfpathlineto{\pgfqpoint{3.011074in}{2.068532in}}%
\pgfpathlineto{\pgfqpoint{3.002756in}{2.060606in}}%
\pgfpathclose%
\pgfusepath{fill}%
\end{pgfscope}%
\begin{pgfscope}%
\pgfpathrectangle{\pgfqpoint{1.150000in}{0.150000in}}{\pgfqpoint{5.700000in}{5.700000in}}%
\pgfusepath{clip}%
\pgfsetbuttcap%
\pgfsetroundjoin%
\definecolor{currentfill}{rgb}{0.280868,0.160771,0.472899}%
\pgfsetfillcolor{currentfill}%
\pgfsetfillopacity{0.700000}%
\pgfsetlinewidth{0.000000pt}%
\definecolor{currentstroke}{rgb}{0.000000,0.000000,0.000000}%
\pgfsetstrokecolor{currentstroke}%
\pgfsetdash{}{0pt}%
\pgfpathmoveto{\pgfqpoint{4.457205in}{2.273788in}}%
\pgfpathlineto{\pgfqpoint{4.471037in}{2.273370in}}%
\pgfpathlineto{\pgfqpoint{4.484878in}{2.273026in}}%
\pgfpathlineto{\pgfqpoint{4.498728in}{2.272755in}}%
\pgfpathlineto{\pgfqpoint{4.512586in}{2.272557in}}%
\pgfpathlineto{\pgfqpoint{4.520348in}{2.280327in}}%
\pgfpathlineto{\pgfqpoint{4.528105in}{2.288095in}}%
\pgfpathlineto{\pgfqpoint{4.535856in}{2.295863in}}%
\pgfpathlineto{\pgfqpoint{4.543601in}{2.303634in}}%
\pgfpathlineto{\pgfqpoint{4.529756in}{2.303996in}}%
\pgfpathlineto{\pgfqpoint{4.515919in}{2.304431in}}%
\pgfpathlineto{\pgfqpoint{4.502091in}{2.304940in}}%
\pgfpathlineto{\pgfqpoint{4.488272in}{2.305521in}}%
\pgfpathlineto{\pgfqpoint{4.480514in}{2.297578in}}%
\pgfpathlineto{\pgfqpoint{4.472750in}{2.289643in}}%
\pgfpathlineto{\pgfqpoint{4.464980in}{2.281714in}}%
\pgfpathlineto{\pgfqpoint{4.457205in}{2.273788in}}%
\pgfpathclose%
\pgfusepath{fill}%
\end{pgfscope}%
\begin{pgfscope}%
\pgfpathrectangle{\pgfqpoint{1.150000in}{0.150000in}}{\pgfqpoint{5.700000in}{5.700000in}}%
\pgfusepath{clip}%
\pgfsetbuttcap%
\pgfsetroundjoin%
\definecolor{currentfill}{rgb}{0.273809,0.031497,0.358853}%
\pgfsetfillcolor{currentfill}%
\pgfsetfillopacity{0.700000}%
\pgfsetlinewidth{0.000000pt}%
\definecolor{currentstroke}{rgb}{0.000000,0.000000,0.000000}%
\pgfsetstrokecolor{currentstroke}%
\pgfsetdash{}{0pt}%
\pgfpathmoveto{\pgfqpoint{3.144183in}{2.044049in}}%
\pgfpathlineto{\pgfqpoint{3.157723in}{2.038412in}}%
\pgfpathlineto{\pgfqpoint{3.171266in}{2.032869in}}%
\pgfpathlineto{\pgfqpoint{3.184813in}{2.027420in}}%
\pgfpathlineto{\pgfqpoint{3.198364in}{2.022065in}}%
\pgfpathlineto{\pgfqpoint{3.206608in}{2.030466in}}%
\pgfpathlineto{\pgfqpoint{3.214846in}{2.038900in}}%
\pgfpathlineto{\pgfqpoint{3.223077in}{2.047366in}}%
\pgfpathlineto{\pgfqpoint{3.231301in}{2.055862in}}%
\pgfpathlineto{\pgfqpoint{3.217764in}{2.061135in}}%
\pgfpathlineto{\pgfqpoint{3.204231in}{2.066500in}}%
\pgfpathlineto{\pgfqpoint{3.190702in}{2.071960in}}%
\pgfpathlineto{\pgfqpoint{3.177177in}{2.077515in}}%
\pgfpathlineto{\pgfqpoint{3.168938in}{2.069093in}}%
\pgfpathlineto{\pgfqpoint{3.160693in}{2.060708in}}%
\pgfpathlineto{\pgfqpoint{3.152442in}{2.052360in}}%
\pgfpathlineto{\pgfqpoint{3.144183in}{2.044049in}}%
\pgfpathclose%
\pgfusepath{fill}%
\end{pgfscope}%
\begin{pgfscope}%
\pgfpathrectangle{\pgfqpoint{1.150000in}{0.150000in}}{\pgfqpoint{5.700000in}{5.700000in}}%
\pgfusepath{clip}%
\pgfsetbuttcap%
\pgfsetroundjoin%
\definecolor{currentfill}{rgb}{0.276022,0.044167,0.370164}%
\pgfsetfillcolor{currentfill}%
\pgfsetfillopacity{0.700000}%
\pgfsetlinewidth{0.000000pt}%
\definecolor{currentstroke}{rgb}{0.000000,0.000000,0.000000}%
\pgfsetstrokecolor{currentstroke}%
\pgfsetdash{}{0pt}%
\pgfpathmoveto{\pgfqpoint{3.513635in}{2.054521in}}%
\pgfpathlineto{\pgfqpoint{3.527226in}{2.050872in}}%
\pgfpathlineto{\pgfqpoint{3.540822in}{2.047308in}}%
\pgfpathlineto{\pgfqpoint{3.554423in}{2.043829in}}%
\pgfpathlineto{\pgfqpoint{3.568030in}{2.040436in}}%
\pgfpathlineto{\pgfqpoint{3.576136in}{2.049334in}}%
\pgfpathlineto{\pgfqpoint{3.584236in}{2.058234in}}%
\pgfpathlineto{\pgfqpoint{3.592330in}{2.067136in}}%
\pgfpathlineto{\pgfqpoint{3.600418in}{2.076041in}}%
\pgfpathlineto{\pgfqpoint{3.586822in}{2.079413in}}%
\pgfpathlineto{\pgfqpoint{3.573232in}{2.082871in}}%
\pgfpathlineto{\pgfqpoint{3.559648in}{2.086413in}}%
\pgfpathlineto{\pgfqpoint{3.546069in}{2.090041in}}%
\pgfpathlineto{\pgfqpoint{3.537970in}{2.081151in}}%
\pgfpathlineto{\pgfqpoint{3.529864in}{2.072267in}}%
\pgfpathlineto{\pgfqpoint{3.521753in}{2.063391in}}%
\pgfpathlineto{\pgfqpoint{3.513635in}{2.054521in}}%
\pgfpathclose%
\pgfusepath{fill}%
\end{pgfscope}%
\begin{pgfscope}%
\pgfpathrectangle{\pgfqpoint{1.150000in}{0.150000in}}{\pgfqpoint{5.700000in}{5.700000in}}%
\pgfusepath{clip}%
\pgfsetbuttcap%
\pgfsetroundjoin%
\definecolor{currentfill}{rgb}{0.277941,0.056324,0.381191}%
\pgfsetfillcolor{currentfill}%
\pgfsetfillopacity{0.700000}%
\pgfsetlinewidth{0.000000pt}%
\definecolor{currentstroke}{rgb}{0.000000,0.000000,0.000000}%
\pgfsetstrokecolor{currentstroke}%
\pgfsetdash{}{0pt}%
\pgfpathmoveto{\pgfqpoint{2.861120in}{2.086242in}}%
\pgfpathlineto{\pgfqpoint{2.874649in}{2.078782in}}%
\pgfpathlineto{\pgfqpoint{2.888180in}{2.071426in}}%
\pgfpathlineto{\pgfqpoint{2.901712in}{2.064175in}}%
\pgfpathlineto{\pgfqpoint{2.915247in}{2.057027in}}%
\pgfpathlineto{\pgfqpoint{2.923612in}{2.064597in}}%
\pgfpathlineto{\pgfqpoint{2.931970in}{2.072232in}}%
\pgfpathlineto{\pgfqpoint{2.940320in}{2.079927in}}%
\pgfpathlineto{\pgfqpoint{2.948662in}{2.087683in}}%
\pgfpathlineto{\pgfqpoint{2.935144in}{2.094707in}}%
\pgfpathlineto{\pgfqpoint{2.921629in}{2.101834in}}%
\pgfpathlineto{\pgfqpoint{2.908115in}{2.109065in}}%
\pgfpathlineto{\pgfqpoint{2.894604in}{2.116400in}}%
\pgfpathlineto{\pgfqpoint{2.886245in}{2.108761in}}%
\pgfpathlineto{\pgfqpoint{2.877878in}{2.101187in}}%
\pgfpathlineto{\pgfqpoint{2.869503in}{2.093680in}}%
\pgfpathlineto{\pgfqpoint{2.861120in}{2.086242in}}%
\pgfpathclose%
\pgfusepath{fill}%
\end{pgfscope}%
\begin{pgfscope}%
\pgfpathrectangle{\pgfqpoint{1.150000in}{0.150000in}}{\pgfqpoint{5.700000in}{5.700000in}}%
\pgfusepath{clip}%
\pgfsetbuttcap%
\pgfsetroundjoin%
\definecolor{currentfill}{rgb}{0.235526,0.309527,0.542944}%
\pgfsetfillcolor{currentfill}%
\pgfsetfillopacity{0.700000}%
\pgfsetlinewidth{0.000000pt}%
\definecolor{currentstroke}{rgb}{0.000000,0.000000,0.000000}%
\pgfsetstrokecolor{currentstroke}%
\pgfsetdash{}{0pt}%
\pgfpathmoveto{\pgfqpoint{5.573923in}{2.599150in}}%
\pgfpathlineto{\pgfqpoint{5.588109in}{2.599692in}}%
\pgfpathlineto{\pgfqpoint{5.602305in}{2.600302in}}%
\pgfpathlineto{\pgfqpoint{5.616513in}{2.600978in}}%
\pgfpathlineto{\pgfqpoint{5.630731in}{2.601721in}}%
\pgfpathlineto{\pgfqpoint{5.638039in}{2.608119in}}%
\pgfpathlineto{\pgfqpoint{5.645346in}{2.614662in}}%
\pgfpathlineto{\pgfqpoint{5.652651in}{2.621357in}}%
\pgfpathlineto{\pgfqpoint{5.659956in}{2.628211in}}%
\pgfpathlineto{\pgfqpoint{5.645763in}{2.627860in}}%
\pgfpathlineto{\pgfqpoint{5.631580in}{2.627575in}}%
\pgfpathlineto{\pgfqpoint{5.617408in}{2.627356in}}%
\pgfpathlineto{\pgfqpoint{5.603247in}{2.627204in}}%
\pgfpathlineto{\pgfqpoint{5.595917in}{2.619952in}}%
\pgfpathlineto{\pgfqpoint{5.588587in}{2.612863in}}%
\pgfpathlineto{\pgfqpoint{5.581256in}{2.605931in}}%
\pgfpathlineto{\pgfqpoint{5.573923in}{2.599150in}}%
\pgfpathclose%
\pgfusepath{fill}%
\end{pgfscope}%
\begin{pgfscope}%
\pgfpathrectangle{\pgfqpoint{1.150000in}{0.150000in}}{\pgfqpoint{5.700000in}{5.700000in}}%
\pgfusepath{clip}%
\pgfsetbuttcap%
\pgfsetroundjoin%
\definecolor{currentfill}{rgb}{0.282910,0.105393,0.426902}%
\pgfsetfillcolor{currentfill}%
\pgfsetfillopacity{0.700000}%
\pgfsetlinewidth{0.000000pt}%
\definecolor{currentstroke}{rgb}{0.000000,0.000000,0.000000}%
\pgfsetstrokecolor{currentstroke}%
\pgfsetdash{}{0pt}%
\pgfpathmoveto{\pgfqpoint{4.056165in}{2.160100in}}%
\pgfpathlineto{\pgfqpoint{4.069881in}{2.158626in}}%
\pgfpathlineto{\pgfqpoint{4.083606in}{2.157230in}}%
\pgfpathlineto{\pgfqpoint{4.097338in}{2.155910in}}%
\pgfpathlineto{\pgfqpoint{4.111077in}{2.154668in}}%
\pgfpathlineto{\pgfqpoint{4.118991in}{2.163208in}}%
\pgfpathlineto{\pgfqpoint{4.126900in}{2.171732in}}%
\pgfpathlineto{\pgfqpoint{4.134802in}{2.180243in}}%
\pgfpathlineto{\pgfqpoint{4.142699in}{2.188741in}}%
\pgfpathlineto{\pgfqpoint{4.128971in}{2.190065in}}%
\pgfpathlineto{\pgfqpoint{4.115250in}{2.191466in}}%
\pgfpathlineto{\pgfqpoint{4.101537in}{2.192944in}}%
\pgfpathlineto{\pgfqpoint{4.087831in}{2.194499in}}%
\pgfpathlineto{\pgfqpoint{4.079923in}{2.185911in}}%
\pgfpathlineto{\pgfqpoint{4.072009in}{2.177317in}}%
\pgfpathlineto{\pgfqpoint{4.064090in}{2.168713in}}%
\pgfpathlineto{\pgfqpoint{4.056165in}{2.160100in}}%
\pgfpathclose%
\pgfusepath{fill}%
\end{pgfscope}%
\begin{pgfscope}%
\pgfpathrectangle{\pgfqpoint{1.150000in}{0.150000in}}{\pgfqpoint{5.700000in}{5.700000in}}%
\pgfusepath{clip}%
\pgfsetbuttcap%
\pgfsetroundjoin%
\definecolor{currentfill}{rgb}{0.279566,0.067836,0.391917}%
\pgfsetfillcolor{currentfill}%
\pgfsetfillopacity{0.700000}%
\pgfsetlinewidth{0.000000pt}%
\definecolor{currentstroke}{rgb}{0.000000,0.000000,0.000000}%
\pgfsetstrokecolor{currentstroke}%
\pgfsetdash{}{0pt}%
\pgfpathmoveto{\pgfqpoint{3.741598in}{2.087727in}}%
\pgfpathlineto{\pgfqpoint{3.755236in}{2.085104in}}%
\pgfpathlineto{\pgfqpoint{3.768881in}{2.082562in}}%
\pgfpathlineto{\pgfqpoint{3.782532in}{2.080101in}}%
\pgfpathlineto{\pgfqpoint{3.796190in}{2.077721in}}%
\pgfpathlineto{\pgfqpoint{3.804216in}{2.086607in}}%
\pgfpathlineto{\pgfqpoint{3.812236in}{2.095482in}}%
\pgfpathlineto{\pgfqpoint{3.820250in}{2.104349in}}%
\pgfpathlineto{\pgfqpoint{3.828259in}{2.113207in}}%
\pgfpathlineto{\pgfqpoint{3.814612in}{2.115607in}}%
\pgfpathlineto{\pgfqpoint{3.800972in}{2.118087in}}%
\pgfpathlineto{\pgfqpoint{3.787338in}{2.120649in}}%
\pgfpathlineto{\pgfqpoint{3.773711in}{2.123292in}}%
\pgfpathlineto{\pgfqpoint{3.765691in}{2.114407in}}%
\pgfpathlineto{\pgfqpoint{3.757666in}{2.105518in}}%
\pgfpathlineto{\pgfqpoint{3.749634in}{2.096625in}}%
\pgfpathlineto{\pgfqpoint{3.741598in}{2.087727in}}%
\pgfpathclose%
\pgfusepath{fill}%
\end{pgfscope}%
\begin{pgfscope}%
\pgfpathrectangle{\pgfqpoint{1.150000in}{0.150000in}}{\pgfqpoint{5.700000in}{5.700000in}}%
\pgfusepath{clip}%
\pgfsetbuttcap%
\pgfsetroundjoin%
\definecolor{currentfill}{rgb}{0.255645,0.260703,0.528312}%
\pgfsetfillcolor{currentfill}%
\pgfsetfillopacity{0.700000}%
\pgfsetlinewidth{0.000000pt}%
\definecolor{currentstroke}{rgb}{0.000000,0.000000,0.000000}%
\pgfsetstrokecolor{currentstroke}%
\pgfsetdash{}{0pt}%
\pgfpathmoveto{\pgfqpoint{5.172947in}{2.481856in}}%
\pgfpathlineto{\pgfqpoint{5.187011in}{2.482405in}}%
\pgfpathlineto{\pgfqpoint{5.201085in}{2.483022in}}%
\pgfpathlineto{\pgfqpoint{5.215169in}{2.483708in}}%
\pgfpathlineto{\pgfqpoint{5.229264in}{2.484462in}}%
\pgfpathlineto{\pgfqpoint{5.236733in}{2.490918in}}%
\pgfpathlineto{\pgfqpoint{5.244197in}{2.497446in}}%
\pgfpathlineto{\pgfqpoint{5.251658in}{2.504052in}}%
\pgfpathlineto{\pgfqpoint{5.259115in}{2.510739in}}%
\pgfpathlineto{\pgfqpoint{5.245040in}{2.510295in}}%
\pgfpathlineto{\pgfqpoint{5.230975in}{2.509918in}}%
\pgfpathlineto{\pgfqpoint{5.216921in}{2.509610in}}%
\pgfpathlineto{\pgfqpoint{5.202876in}{2.509370in}}%
\pgfpathlineto{\pgfqpoint{5.195400in}{2.502365in}}%
\pgfpathlineto{\pgfqpoint{5.187920in}{2.495449in}}%
\pgfpathlineto{\pgfqpoint{5.180435in}{2.488614in}}%
\pgfpathlineto{\pgfqpoint{5.172947in}{2.481856in}}%
\pgfpathclose%
\pgfusepath{fill}%
\end{pgfscope}%
\begin{pgfscope}%
\pgfpathrectangle{\pgfqpoint{1.150000in}{0.150000in}}{\pgfqpoint{5.700000in}{5.700000in}}%
\pgfusepath{clip}%
\pgfsetbuttcap%
\pgfsetroundjoin%
\definecolor{currentfill}{rgb}{0.273809,0.031497,0.358853}%
\pgfsetfillcolor{currentfill}%
\pgfsetfillopacity{0.700000}%
\pgfsetlinewidth{0.000000pt}%
\definecolor{currentstroke}{rgb}{0.000000,0.000000,0.000000}%
\pgfsetstrokecolor{currentstroke}%
\pgfsetdash{}{0pt}%
\pgfpathmoveto{\pgfqpoint{3.285488in}{2.035698in}}%
\pgfpathlineto{\pgfqpoint{3.299046in}{2.030886in}}%
\pgfpathlineto{\pgfqpoint{3.312607in}{2.026165in}}%
\pgfpathlineto{\pgfqpoint{3.326174in}{2.021533in}}%
\pgfpathlineto{\pgfqpoint{3.339744in}{2.016992in}}%
\pgfpathlineto{\pgfqpoint{3.347935in}{2.025659in}}%
\pgfpathlineto{\pgfqpoint{3.356120in}{2.034345in}}%
\pgfpathlineto{\pgfqpoint{3.364298in}{2.043050in}}%
\pgfpathlineto{\pgfqpoint{3.372470in}{2.051774in}}%
\pgfpathlineto{\pgfqpoint{3.358912in}{2.056253in}}%
\pgfpathlineto{\pgfqpoint{3.345359in}{2.060822in}}%
\pgfpathlineto{\pgfqpoint{3.331811in}{2.065481in}}%
\pgfpathlineto{\pgfqpoint{3.318267in}{2.070231in}}%
\pgfpathlineto{\pgfqpoint{3.310082in}{2.061562in}}%
\pgfpathlineto{\pgfqpoint{3.301890in}{2.052917in}}%
\pgfpathlineto{\pgfqpoint{3.293692in}{2.044295in}}%
\pgfpathlineto{\pgfqpoint{3.285488in}{2.035698in}}%
\pgfpathclose%
\pgfusepath{fill}%
\end{pgfscope}%
\begin{pgfscope}%
\pgfpathrectangle{\pgfqpoint{1.150000in}{0.150000in}}{\pgfqpoint{5.700000in}{5.700000in}}%
\pgfusepath{clip}%
\pgfsetbuttcap%
\pgfsetroundjoin%
\definecolor{currentfill}{rgb}{0.273006,0.204520,0.501721}%
\pgfsetfillcolor{currentfill}%
\pgfsetfillopacity{0.700000}%
\pgfsetlinewidth{0.000000pt}%
\definecolor{currentstroke}{rgb}{0.000000,0.000000,0.000000}%
\pgfsetstrokecolor{currentstroke}%
\pgfsetdash{}{0pt}%
\pgfpathmoveto{\pgfqpoint{4.771876in}{2.362967in}}%
\pgfpathlineto{\pgfqpoint{4.785812in}{2.363150in}}%
\pgfpathlineto{\pgfqpoint{4.799757in}{2.363403in}}%
\pgfpathlineto{\pgfqpoint{4.813712in}{2.363728in}}%
\pgfpathlineto{\pgfqpoint{4.827676in}{2.364123in}}%
\pgfpathlineto{\pgfqpoint{4.835314in}{2.371214in}}%
\pgfpathlineto{\pgfqpoint{4.842946in}{2.378325in}}%
\pgfpathlineto{\pgfqpoint{4.850572in}{2.385461in}}%
\pgfpathlineto{\pgfqpoint{4.858193in}{2.392625in}}%
\pgfpathlineto{\pgfqpoint{4.844244in}{2.392457in}}%
\pgfpathlineto{\pgfqpoint{4.830305in}{2.392359in}}%
\pgfpathlineto{\pgfqpoint{4.816375in}{2.392331in}}%
\pgfpathlineto{\pgfqpoint{4.802455in}{2.392375in}}%
\pgfpathlineto{\pgfqpoint{4.794818in}{2.384977in}}%
\pgfpathlineto{\pgfqpoint{4.787176in}{2.377612in}}%
\pgfpathlineto{\pgfqpoint{4.779529in}{2.370277in}}%
\pgfpathlineto{\pgfqpoint{4.771876in}{2.362967in}}%
\pgfpathclose%
\pgfusepath{fill}%
\end{pgfscope}%
\begin{pgfscope}%
\pgfpathrectangle{\pgfqpoint{1.150000in}{0.150000in}}{\pgfqpoint{5.700000in}{5.700000in}}%
\pgfusepath{clip}%
\pgfsetbuttcap%
\pgfsetroundjoin%
\definecolor{currentfill}{rgb}{0.281887,0.150881,0.465405}%
\pgfsetfillcolor{currentfill}%
\pgfsetfillopacity{0.700000}%
\pgfsetlinewidth{0.000000pt}%
\definecolor{currentstroke}{rgb}{0.000000,0.000000,0.000000}%
\pgfsetstrokecolor{currentstroke}%
\pgfsetdash{}{0pt}%
\pgfpathmoveto{\pgfqpoint{4.370754in}{2.243879in}}%
\pgfpathlineto{\pgfqpoint{4.384565in}{2.243310in}}%
\pgfpathlineto{\pgfqpoint{4.398383in}{2.242815in}}%
\pgfpathlineto{\pgfqpoint{4.412211in}{2.242394in}}%
\pgfpathlineto{\pgfqpoint{4.426046in}{2.242047in}}%
\pgfpathlineto{\pgfqpoint{4.433845in}{2.249993in}}%
\pgfpathlineto{\pgfqpoint{4.441637in}{2.257930in}}%
\pgfpathlineto{\pgfqpoint{4.449424in}{2.265860in}}%
\pgfpathlineto{\pgfqpoint{4.457205in}{2.273788in}}%
\pgfpathlineto{\pgfqpoint{4.443382in}{2.274278in}}%
\pgfpathlineto{\pgfqpoint{4.429567in}{2.274843in}}%
\pgfpathlineto{\pgfqpoint{4.415760in}{2.275481in}}%
\pgfpathlineto{\pgfqpoint{4.401962in}{2.276194in}}%
\pgfpathlineto{\pgfqpoint{4.394169in}{2.268116in}}%
\pgfpathlineto{\pgfqpoint{4.386370in}{2.260039in}}%
\pgfpathlineto{\pgfqpoint{4.378565in}{2.251961in}}%
\pgfpathlineto{\pgfqpoint{4.370754in}{2.243879in}}%
\pgfpathclose%
\pgfusepath{fill}%
\end{pgfscope}%
\begin{pgfscope}%
\pgfpathrectangle{\pgfqpoint{1.150000in}{0.150000in}}{\pgfqpoint{5.700000in}{5.700000in}}%
\pgfusepath{clip}%
\pgfsetbuttcap%
\pgfsetroundjoin%
\definecolor{currentfill}{rgb}{0.280894,0.078907,0.402329}%
\pgfsetfillcolor{currentfill}%
\pgfsetfillopacity{0.700000}%
\pgfsetlinewidth{0.000000pt}%
\definecolor{currentstroke}{rgb}{0.000000,0.000000,0.000000}%
\pgfsetstrokecolor{currentstroke}%
\pgfsetdash{}{0pt}%
\pgfpathmoveto{\pgfqpoint{2.719178in}{2.121901in}}%
\pgfpathlineto{\pgfqpoint{2.732715in}{2.113432in}}%
\pgfpathlineto{\pgfqpoint{2.746253in}{2.105074in}}%
\pgfpathlineto{\pgfqpoint{2.759791in}{2.096826in}}%
\pgfpathlineto{\pgfqpoint{2.773331in}{2.088687in}}%
\pgfpathlineto{\pgfqpoint{2.781767in}{2.095681in}}%
\pgfpathlineto{\pgfqpoint{2.790194in}{2.102756in}}%
\pgfpathlineto{\pgfqpoint{2.798612in}{2.109910in}}%
\pgfpathlineto{\pgfqpoint{2.807022in}{2.117141in}}%
\pgfpathlineto{\pgfqpoint{2.793501in}{2.125134in}}%
\pgfpathlineto{\pgfqpoint{2.779982in}{2.133236in}}%
\pgfpathlineto{\pgfqpoint{2.766464in}{2.141448in}}%
\pgfpathlineto{\pgfqpoint{2.752946in}{2.149771in}}%
\pgfpathlineto{\pgfqpoint{2.744518in}{2.142678in}}%
\pgfpathlineto{\pgfqpoint{2.736080in}{2.135668in}}%
\pgfpathlineto{\pgfqpoint{2.727634in}{2.128741in}}%
\pgfpathlineto{\pgfqpoint{2.719178in}{2.121901in}}%
\pgfpathclose%
\pgfusepath{fill}%
\end{pgfscope}%
\begin{pgfscope}%
\pgfpathrectangle{\pgfqpoint{1.150000in}{0.150000in}}{\pgfqpoint{5.700000in}{5.700000in}}%
\pgfusepath{clip}%
\pgfsetbuttcap%
\pgfsetroundjoin%
\definecolor{currentfill}{rgb}{0.218130,0.347432,0.550038}%
\pgfsetfillcolor{currentfill}%
\pgfsetfillopacity{0.700000}%
\pgfsetlinewidth{0.000000pt}%
\definecolor{currentstroke}{rgb}{0.000000,0.000000,0.000000}%
\pgfsetstrokecolor{currentstroke}%
\pgfsetdash{}{0pt}%
\pgfpathmoveto{\pgfqpoint{5.888926in}{2.688910in}}%
\pgfpathlineto{\pgfqpoint{5.903207in}{2.689277in}}%
\pgfpathlineto{\pgfqpoint{5.917498in}{2.689709in}}%
\pgfpathlineto{\pgfqpoint{5.931800in}{2.690207in}}%
\pgfpathlineto{\pgfqpoint{5.946114in}{2.690770in}}%
\pgfpathlineto{\pgfqpoint{5.953315in}{2.697609in}}%
\pgfpathlineto{\pgfqpoint{5.960518in}{2.704662in}}%
\pgfpathlineto{\pgfqpoint{5.967724in}{2.711937in}}%
\pgfpathlineto{\pgfqpoint{5.974933in}{2.719443in}}%
\pgfpathlineto{\pgfqpoint{5.960649in}{2.719332in}}%
\pgfpathlineto{\pgfqpoint{5.946375in}{2.719286in}}%
\pgfpathlineto{\pgfqpoint{5.932113in}{2.719306in}}%
\pgfpathlineto{\pgfqpoint{5.917861in}{2.719391in}}%
\pgfpathlineto{\pgfqpoint{5.910623in}{2.711426in}}%
\pgfpathlineto{\pgfqpoint{5.903389in}{2.703697in}}%
\pgfpathlineto{\pgfqpoint{5.896156in}{2.696194in}}%
\pgfpathlineto{\pgfqpoint{5.888926in}{2.688910in}}%
\pgfpathclose%
\pgfusepath{fill}%
\end{pgfscope}%
\begin{pgfscope}%
\pgfpathrectangle{\pgfqpoint{1.150000in}{0.150000in}}{\pgfqpoint{5.700000in}{5.700000in}}%
\pgfusepath{clip}%
\pgfsetbuttcap%
\pgfsetroundjoin%
\definecolor{currentfill}{rgb}{0.283229,0.120777,0.440584}%
\pgfsetfillcolor{currentfill}%
\pgfsetfillopacity{0.700000}%
\pgfsetlinewidth{0.000000pt}%
\definecolor{currentstroke}{rgb}{0.000000,0.000000,0.000000}%
\pgfsetstrokecolor{currentstroke}%
\pgfsetdash{}{0pt}%
\pgfpathmoveto{\pgfqpoint{2.522595in}{2.208011in}}%
\pgfpathlineto{\pgfqpoint{2.536152in}{2.197979in}}%
\pgfpathlineto{\pgfqpoint{2.549708in}{2.188068in}}%
\pgfpathlineto{\pgfqpoint{2.563264in}{2.178279in}}%
\pgfpathlineto{\pgfqpoint{2.576820in}{2.168608in}}%
\pgfpathlineto{\pgfqpoint{2.585356in}{2.174734in}}%
\pgfpathlineto{\pgfqpoint{2.593882in}{2.180965in}}%
\pgfpathlineto{\pgfqpoint{2.602398in}{2.187299in}}%
\pgfpathlineto{\pgfqpoint{2.610904in}{2.193733in}}%
\pgfpathlineto{\pgfqpoint{2.597371in}{2.203235in}}%
\pgfpathlineto{\pgfqpoint{2.583837in}{2.212857in}}%
\pgfpathlineto{\pgfqpoint{2.570303in}{2.222599in}}%
\pgfpathlineto{\pgfqpoint{2.556769in}{2.232463in}}%
\pgfpathlineto{\pgfqpoint{2.548241in}{2.226189in}}%
\pgfpathlineto{\pgfqpoint{2.539703in}{2.220021in}}%
\pgfpathlineto{\pgfqpoint{2.531154in}{2.213961in}}%
\pgfpathlineto{\pgfqpoint{2.522595in}{2.208011in}}%
\pgfpathclose%
\pgfusepath{fill}%
\end{pgfscope}%
\begin{pgfscope}%
\pgfpathrectangle{\pgfqpoint{1.150000in}{0.150000in}}{\pgfqpoint{5.700000in}{5.700000in}}%
\pgfusepath{clip}%
\pgfsetbuttcap%
\pgfsetroundjoin%
\definecolor{currentfill}{rgb}{0.239346,0.300855,0.540844}%
\pgfsetfillcolor{currentfill}%
\pgfsetfillopacity{0.700000}%
\pgfsetlinewidth{0.000000pt}%
\definecolor{currentstroke}{rgb}{0.000000,0.000000,0.000000}%
\pgfsetstrokecolor{currentstroke}%
\pgfsetdash{}{0pt}%
\pgfpathmoveto{\pgfqpoint{5.487842in}{2.570406in}}%
\pgfpathlineto{\pgfqpoint{5.502009in}{2.571052in}}%
\pgfpathlineto{\pgfqpoint{5.516187in}{2.571764in}}%
\pgfpathlineto{\pgfqpoint{5.530375in}{2.572544in}}%
\pgfpathlineto{\pgfqpoint{5.544574in}{2.573391in}}%
\pgfpathlineto{\pgfqpoint{5.551915in}{2.579639in}}%
\pgfpathlineto{\pgfqpoint{5.559253in}{2.586010in}}%
\pgfpathlineto{\pgfqpoint{5.566589in}{2.592512in}}%
\pgfpathlineto{\pgfqpoint{5.573923in}{2.599150in}}%
\pgfpathlineto{\pgfqpoint{5.559748in}{2.598674in}}%
\pgfpathlineto{\pgfqpoint{5.545583in}{2.598265in}}%
\pgfpathlineto{\pgfqpoint{5.531429in}{2.597923in}}%
\pgfpathlineto{\pgfqpoint{5.517285in}{2.597648in}}%
\pgfpathlineto{\pgfqpoint{5.509927in}{2.590632in}}%
\pgfpathlineto{\pgfqpoint{5.502568in}{2.583757in}}%
\pgfpathlineto{\pgfqpoint{5.495206in}{2.577017in}}%
\pgfpathlineto{\pgfqpoint{5.487842in}{2.570406in}}%
\pgfpathclose%
\pgfusepath{fill}%
\end{pgfscope}%
\begin{pgfscope}%
\pgfpathrectangle{\pgfqpoint{1.150000in}{0.150000in}}{\pgfqpoint{5.700000in}{5.700000in}}%
\pgfusepath{clip}%
\pgfsetbuttcap%
\pgfsetroundjoin%
\definecolor{currentfill}{rgb}{0.282327,0.094955,0.417331}%
\pgfsetfillcolor{currentfill}%
\pgfsetfillopacity{0.700000}%
\pgfsetlinewidth{0.000000pt}%
\definecolor{currentstroke}{rgb}{0.000000,0.000000,0.000000}%
\pgfsetstrokecolor{currentstroke}%
\pgfsetdash{}{0pt}%
\pgfpathmoveto{\pgfqpoint{3.969571in}{2.131934in}}%
\pgfpathlineto{\pgfqpoint{3.983269in}{2.130210in}}%
\pgfpathlineto{\pgfqpoint{3.996974in}{2.128564in}}%
\pgfpathlineto{\pgfqpoint{4.010687in}{2.126996in}}%
\pgfpathlineto{\pgfqpoint{4.024407in}{2.125506in}}%
\pgfpathlineto{\pgfqpoint{4.032355in}{2.134178in}}%
\pgfpathlineto{\pgfqpoint{4.040297in}{2.142833in}}%
\pgfpathlineto{\pgfqpoint{4.048234in}{2.151473in}}%
\pgfpathlineto{\pgfqpoint{4.056165in}{2.160100in}}%
\pgfpathlineto{\pgfqpoint{4.042455in}{2.161651in}}%
\pgfpathlineto{\pgfqpoint{4.028753in}{2.163280in}}%
\pgfpathlineto{\pgfqpoint{4.015059in}{2.164986in}}%
\pgfpathlineto{\pgfqpoint{4.001372in}{2.166771in}}%
\pgfpathlineto{\pgfqpoint{3.993430in}{2.158077in}}%
\pgfpathlineto{\pgfqpoint{3.985482in}{2.149373in}}%
\pgfpathlineto{\pgfqpoint{3.977529in}{2.140660in}}%
\pgfpathlineto{\pgfqpoint{3.969571in}{2.131934in}}%
\pgfpathclose%
\pgfusepath{fill}%
\end{pgfscope}%
\begin{pgfscope}%
\pgfpathrectangle{\pgfqpoint{1.150000in}{0.150000in}}{\pgfqpoint{5.700000in}{5.700000in}}%
\pgfusepath{clip}%
\pgfsetbuttcap%
\pgfsetroundjoin%
\definecolor{currentfill}{rgb}{0.260571,0.246922,0.522828}%
\pgfsetfillcolor{currentfill}%
\pgfsetfillopacity{0.700000}%
\pgfsetlinewidth{0.000000pt}%
\definecolor{currentstroke}{rgb}{0.000000,0.000000,0.000000}%
\pgfsetstrokecolor{currentstroke}%
\pgfsetdash{}{0pt}%
\pgfpathmoveto{\pgfqpoint{5.086720in}{2.452823in}}%
\pgfpathlineto{\pgfqpoint{5.100762in}{2.453385in}}%
\pgfpathlineto{\pgfqpoint{5.114814in}{2.454016in}}%
\pgfpathlineto{\pgfqpoint{5.128877in}{2.454715in}}%
\pgfpathlineto{\pgfqpoint{5.142949in}{2.455484in}}%
\pgfpathlineto{\pgfqpoint{5.150456in}{2.461989in}}%
\pgfpathlineto{\pgfqpoint{5.157957in}{2.468549in}}%
\pgfpathlineto{\pgfqpoint{5.165454in}{2.475170in}}%
\pgfpathlineto{\pgfqpoint{5.172947in}{2.481856in}}%
\pgfpathlineto{\pgfqpoint{5.158893in}{2.481376in}}%
\pgfpathlineto{\pgfqpoint{5.144850in}{2.480965in}}%
\pgfpathlineto{\pgfqpoint{5.130816in}{2.480623in}}%
\pgfpathlineto{\pgfqpoint{5.116792in}{2.480350in}}%
\pgfpathlineto{\pgfqpoint{5.109281in}{2.473367in}}%
\pgfpathlineto{\pgfqpoint{5.101765in}{2.466455in}}%
\pgfpathlineto{\pgfqpoint{5.094245in}{2.459609in}}%
\pgfpathlineto{\pgfqpoint{5.086720in}{2.452823in}}%
\pgfpathclose%
\pgfusepath{fill}%
\end{pgfscope}%
\begin{pgfscope}%
\pgfpathrectangle{\pgfqpoint{1.150000in}{0.150000in}}{\pgfqpoint{5.700000in}{5.700000in}}%
\pgfusepath{clip}%
\pgfsetbuttcap%
\pgfsetroundjoin%
\definecolor{currentfill}{rgb}{0.275191,0.194905,0.496005}%
\pgfsetfillcolor{currentfill}%
\pgfsetfillopacity{0.700000}%
\pgfsetlinewidth{0.000000pt}%
\definecolor{currentstroke}{rgb}{0.000000,0.000000,0.000000}%
\pgfsetstrokecolor{currentstroke}%
\pgfsetdash{}{0pt}%
\pgfpathmoveto{\pgfqpoint{2.270753in}{2.366305in}}%
\pgfpathlineto{\pgfqpoint{2.284364in}{2.353989in}}%
\pgfpathlineto{\pgfqpoint{2.297972in}{2.341814in}}%
\pgfpathlineto{\pgfqpoint{2.311577in}{2.329778in}}%
\pgfpathlineto{\pgfqpoint{2.325179in}{2.317878in}}%
\pgfpathlineto{\pgfqpoint{2.333858in}{2.322769in}}%
\pgfpathlineto{\pgfqpoint{2.342524in}{2.327796in}}%
\pgfpathlineto{\pgfqpoint{2.351178in}{2.332956in}}%
\pgfpathlineto{\pgfqpoint{2.359821in}{2.338247in}}%
\pgfpathlineto{\pgfqpoint{2.346244in}{2.349954in}}%
\pgfpathlineto{\pgfqpoint{2.332665in}{2.361799in}}%
\pgfpathlineto{\pgfqpoint{2.319084in}{2.373782in}}%
\pgfpathlineto{\pgfqpoint{2.305500in}{2.385905in}}%
\pgfpathlineto{\pgfqpoint{2.296831in}{2.380799in}}%
\pgfpathlineto{\pgfqpoint{2.288151in}{2.375828in}}%
\pgfpathlineto{\pgfqpoint{2.279458in}{2.370996in}}%
\pgfpathlineto{\pgfqpoint{2.270753in}{2.366305in}}%
\pgfpathclose%
\pgfusepath{fill}%
\end{pgfscope}%
\begin{pgfscope}%
\pgfpathrectangle{\pgfqpoint{1.150000in}{0.150000in}}{\pgfqpoint{5.700000in}{5.700000in}}%
\pgfusepath{clip}%
\pgfsetbuttcap%
\pgfsetroundjoin%
\definecolor{currentfill}{rgb}{0.274952,0.037752,0.364543}%
\pgfsetfillcolor{currentfill}%
\pgfsetfillopacity{0.700000}%
\pgfsetlinewidth{0.000000pt}%
\definecolor{currentstroke}{rgb}{0.000000,0.000000,0.000000}%
\pgfsetstrokecolor{currentstroke}%
\pgfsetdash{}{0pt}%
\pgfpathmoveto{\pgfqpoint{3.426748in}{2.034744in}}%
\pgfpathlineto{\pgfqpoint{3.440330in}{2.030707in}}%
\pgfpathlineto{\pgfqpoint{3.453917in}{2.026757in}}%
\pgfpathlineto{\pgfqpoint{3.467509in}{2.022893in}}%
\pgfpathlineto{\pgfqpoint{3.481106in}{2.019116in}}%
\pgfpathlineto{\pgfqpoint{3.489248in}{2.027956in}}%
\pgfpathlineto{\pgfqpoint{3.497383in}{2.036804in}}%
\pgfpathlineto{\pgfqpoint{3.505512in}{2.045659in}}%
\pgfpathlineto{\pgfqpoint{3.513635in}{2.054521in}}%
\pgfpathlineto{\pgfqpoint{3.500050in}{2.058257in}}%
\pgfpathlineto{\pgfqpoint{3.486470in}{2.062079in}}%
\pgfpathlineto{\pgfqpoint{3.472895in}{2.065987in}}%
\pgfpathlineto{\pgfqpoint{3.459326in}{2.069983in}}%
\pgfpathlineto{\pgfqpoint{3.451191in}{2.061155in}}%
\pgfpathlineto{\pgfqpoint{3.443049in}{2.052339in}}%
\pgfpathlineto{\pgfqpoint{3.434902in}{2.043535in}}%
\pgfpathlineto{\pgfqpoint{3.426748in}{2.034744in}}%
\pgfpathclose%
\pgfusepath{fill}%
\end{pgfscope}%
\begin{pgfscope}%
\pgfpathrectangle{\pgfqpoint{1.150000in}{0.150000in}}{\pgfqpoint{5.700000in}{5.700000in}}%
\pgfusepath{clip}%
\pgfsetbuttcap%
\pgfsetroundjoin%
\definecolor{currentfill}{rgb}{0.275191,0.194905,0.496005}%
\pgfsetfillcolor{currentfill}%
\pgfsetfillopacity{0.700000}%
\pgfsetlinewidth{0.000000pt}%
\definecolor{currentstroke}{rgb}{0.000000,0.000000,0.000000}%
\pgfsetstrokecolor{currentstroke}%
\pgfsetdash{}{0pt}%
\pgfpathmoveto{\pgfqpoint{4.685502in}{2.333060in}}%
\pgfpathlineto{\pgfqpoint{4.699415in}{2.333163in}}%
\pgfpathlineto{\pgfqpoint{4.713337in}{2.333338in}}%
\pgfpathlineto{\pgfqpoint{4.727269in}{2.333584in}}%
\pgfpathlineto{\pgfqpoint{4.741210in}{2.333901in}}%
\pgfpathlineto{\pgfqpoint{4.748885in}{2.341150in}}%
\pgfpathlineto{\pgfqpoint{4.756554in}{2.348407in}}%
\pgfpathlineto{\pgfqpoint{4.764218in}{2.355679in}}%
\pgfpathlineto{\pgfqpoint{4.771876in}{2.362967in}}%
\pgfpathlineto{\pgfqpoint{4.757950in}{2.362856in}}%
\pgfpathlineto{\pgfqpoint{4.744033in}{2.362816in}}%
\pgfpathlineto{\pgfqpoint{4.730125in}{2.362847in}}%
\pgfpathlineto{\pgfqpoint{4.716227in}{2.362949in}}%
\pgfpathlineto{\pgfqpoint{4.708554in}{2.355447in}}%
\pgfpathlineto{\pgfqpoint{4.700876in}{2.347968in}}%
\pgfpathlineto{\pgfqpoint{4.693192in}{2.340506in}}%
\pgfpathlineto{\pgfqpoint{4.685502in}{2.333060in}}%
\pgfpathclose%
\pgfusepath{fill}%
\end{pgfscope}%
\begin{pgfscope}%
\pgfpathrectangle{\pgfqpoint{1.150000in}{0.150000in}}{\pgfqpoint{5.700000in}{5.700000in}}%
\pgfusepath{clip}%
\pgfsetbuttcap%
\pgfsetroundjoin%
\definecolor{currentfill}{rgb}{0.277941,0.056324,0.381191}%
\pgfsetfillcolor{currentfill}%
\pgfsetfillopacity{0.700000}%
\pgfsetlinewidth{0.000000pt}%
\definecolor{currentstroke}{rgb}{0.000000,0.000000,0.000000}%
\pgfsetstrokecolor{currentstroke}%
\pgfsetdash{}{0pt}%
\pgfpathmoveto{\pgfqpoint{3.654857in}{2.063392in}}%
\pgfpathlineto{\pgfqpoint{3.668482in}{2.060438in}}%
\pgfpathlineto{\pgfqpoint{3.682113in}{2.057567in}}%
\pgfpathlineto{\pgfqpoint{3.695750in}{2.054779in}}%
\pgfpathlineto{\pgfqpoint{3.709393in}{2.052073in}}%
\pgfpathlineto{\pgfqpoint{3.717453in}{2.060997in}}%
\pgfpathlineto{\pgfqpoint{3.725507in}{2.069914in}}%
\pgfpathlineto{\pgfqpoint{3.733555in}{2.078824in}}%
\pgfpathlineto{\pgfqpoint{3.741598in}{2.087727in}}%
\pgfpathlineto{\pgfqpoint{3.727966in}{2.090433in}}%
\pgfpathlineto{\pgfqpoint{3.714340in}{2.093221in}}%
\pgfpathlineto{\pgfqpoint{3.700720in}{2.096091in}}%
\pgfpathlineto{\pgfqpoint{3.687107in}{2.099043in}}%
\pgfpathlineto{\pgfqpoint{3.679053in}{2.090133in}}%
\pgfpathlineto{\pgfqpoint{3.670993in}{2.081221in}}%
\pgfpathlineto{\pgfqpoint{3.662928in}{2.072308in}}%
\pgfpathlineto{\pgfqpoint{3.654857in}{2.063392in}}%
\pgfpathclose%
\pgfusepath{fill}%
\end{pgfscope}%
\begin{pgfscope}%
\pgfpathrectangle{\pgfqpoint{1.150000in}{0.150000in}}{\pgfqpoint{5.700000in}{5.700000in}}%
\pgfusepath{clip}%
\pgfsetbuttcap%
\pgfsetroundjoin%
\definecolor{currentfill}{rgb}{0.282884,0.135920,0.453427}%
\pgfsetfillcolor{currentfill}%
\pgfsetfillopacity{0.700000}%
\pgfsetlinewidth{0.000000pt}%
\definecolor{currentstroke}{rgb}{0.000000,0.000000,0.000000}%
\pgfsetstrokecolor{currentstroke}%
\pgfsetdash{}{0pt}%
\pgfpathmoveto{\pgfqpoint{4.284250in}{2.213987in}}%
\pgfpathlineto{\pgfqpoint{4.298038in}{2.213243in}}%
\pgfpathlineto{\pgfqpoint{4.311835in}{2.212574in}}%
\pgfpathlineto{\pgfqpoint{4.325641in}{2.211979in}}%
\pgfpathlineto{\pgfqpoint{4.339454in}{2.211459in}}%
\pgfpathlineto{\pgfqpoint{4.347288in}{2.219583in}}%
\pgfpathlineto{\pgfqpoint{4.355116in}{2.227693in}}%
\pgfpathlineto{\pgfqpoint{4.362938in}{2.235791in}}%
\pgfpathlineto{\pgfqpoint{4.370754in}{2.243879in}}%
\pgfpathlineto{\pgfqpoint{4.356953in}{2.244523in}}%
\pgfpathlineto{\pgfqpoint{4.343159in}{2.245240in}}%
\pgfpathlineto{\pgfqpoint{4.329374in}{2.246033in}}%
\pgfpathlineto{\pgfqpoint{4.315597in}{2.246899in}}%
\pgfpathlineto{\pgfqpoint{4.307769in}{2.238680in}}%
\pgfpathlineto{\pgfqpoint{4.299935in}{2.230457in}}%
\pgfpathlineto{\pgfqpoint{4.292095in}{2.222227in}}%
\pgfpathlineto{\pgfqpoint{4.284250in}{2.213987in}}%
\pgfpathclose%
\pgfusepath{fill}%
\end{pgfscope}%
\begin{pgfscope}%
\pgfpathrectangle{\pgfqpoint{1.150000in}{0.150000in}}{\pgfqpoint{5.700000in}{5.700000in}}%
\pgfusepath{clip}%
\pgfsetbuttcap%
\pgfsetroundjoin%
\definecolor{currentfill}{rgb}{0.221989,0.339161,0.548752}%
\pgfsetfillcolor{currentfill}%
\pgfsetfillopacity{0.700000}%
\pgfsetlinewidth{0.000000pt}%
\definecolor{currentstroke}{rgb}{0.000000,0.000000,0.000000}%
\pgfsetstrokecolor{currentstroke}%
\pgfsetdash{}{0pt}%
\pgfpathmoveto{\pgfqpoint{5.802898in}{2.659271in}}%
\pgfpathlineto{\pgfqpoint{5.817163in}{2.659807in}}%
\pgfpathlineto{\pgfqpoint{5.831438in}{2.660409in}}%
\pgfpathlineto{\pgfqpoint{5.845724in}{2.661076in}}%
\pgfpathlineto{\pgfqpoint{5.860022in}{2.661810in}}%
\pgfpathlineto{\pgfqpoint{5.867246in}{2.668295in}}%
\pgfpathlineto{\pgfqpoint{5.874472in}{2.674968in}}%
\pgfpathlineto{\pgfqpoint{5.881698in}{2.681837in}}%
\pgfpathlineto{\pgfqpoint{5.888926in}{2.688910in}}%
\pgfpathlineto{\pgfqpoint{5.874657in}{2.688609in}}%
\pgfpathlineto{\pgfqpoint{5.860399in}{2.688374in}}%
\pgfpathlineto{\pgfqpoint{5.846151in}{2.688204in}}%
\pgfpathlineto{\pgfqpoint{5.831914in}{2.688100in}}%
\pgfpathlineto{\pgfqpoint{5.824658in}{2.680588in}}%
\pgfpathlineto{\pgfqpoint{5.817404in}{2.673285in}}%
\pgfpathlineto{\pgfqpoint{5.810151in}{2.666181in}}%
\pgfpathlineto{\pgfqpoint{5.802898in}{2.659271in}}%
\pgfpathclose%
\pgfusepath{fill}%
\end{pgfscope}%
\begin{pgfscope}%
\pgfpathrectangle{\pgfqpoint{1.150000in}{0.150000in}}{\pgfqpoint{5.700000in}{5.700000in}}%
\pgfusepath{clip}%
\pgfsetbuttcap%
\pgfsetroundjoin%
\definecolor{currentfill}{rgb}{0.274952,0.037752,0.364543}%
\pgfsetfillcolor{currentfill}%
\pgfsetfillopacity{0.700000}%
\pgfsetlinewidth{0.000000pt}%
\definecolor{currentstroke}{rgb}{0.000000,0.000000,0.000000}%
\pgfsetstrokecolor{currentstroke}%
\pgfsetdash{}{0pt}%
\pgfpathmoveto{\pgfqpoint{3.056893in}{2.035126in}}%
\pgfpathlineto{\pgfqpoint{3.070435in}{2.029001in}}%
\pgfpathlineto{\pgfqpoint{3.083980in}{2.022973in}}%
\pgfpathlineto{\pgfqpoint{3.097528in}{2.017041in}}%
\pgfpathlineto{\pgfqpoint{3.111080in}{2.011205in}}%
\pgfpathlineto{\pgfqpoint{3.119366in}{2.019354in}}%
\pgfpathlineto{\pgfqpoint{3.127645in}{2.027545in}}%
\pgfpathlineto{\pgfqpoint{3.135918in}{2.035777in}}%
\pgfpathlineto{\pgfqpoint{3.144183in}{2.044049in}}%
\pgfpathlineto{\pgfqpoint{3.130647in}{2.049782in}}%
\pgfpathlineto{\pgfqpoint{3.117114in}{2.055610in}}%
\pgfpathlineto{\pgfqpoint{3.103584in}{2.061535in}}%
\pgfpathlineto{\pgfqpoint{3.090058in}{2.067556in}}%
\pgfpathlineto{\pgfqpoint{3.081777in}{2.059380in}}%
\pgfpathlineto{\pgfqpoint{3.073490in}{2.051249in}}%
\pgfpathlineto{\pgfqpoint{3.065195in}{2.043164in}}%
\pgfpathlineto{\pgfqpoint{3.056893in}{2.035126in}}%
\pgfpathclose%
\pgfusepath{fill}%
\end{pgfscope}%
\begin{pgfscope}%
\pgfpathrectangle{\pgfqpoint{1.150000in}{0.150000in}}{\pgfqpoint{5.700000in}{5.700000in}}%
\pgfusepath{clip}%
\pgfsetbuttcap%
\pgfsetroundjoin%
\definecolor{currentfill}{rgb}{0.277018,0.050344,0.375715}%
\pgfsetfillcolor{currentfill}%
\pgfsetfillopacity{0.700000}%
\pgfsetlinewidth{0.000000pt}%
\definecolor{currentstroke}{rgb}{0.000000,0.000000,0.000000}%
\pgfsetstrokecolor{currentstroke}%
\pgfsetdash{}{0pt}%
\pgfpathmoveto{\pgfqpoint{2.915247in}{2.057027in}}%
\pgfpathlineto{\pgfqpoint{2.928784in}{2.049981in}}%
\pgfpathlineto{\pgfqpoint{2.942323in}{2.043037in}}%
\pgfpathlineto{\pgfqpoint{2.955865in}{2.036194in}}%
\pgfpathlineto{\pgfqpoint{2.969409in}{2.029452in}}%
\pgfpathlineto{\pgfqpoint{2.977757in}{2.037154in}}%
\pgfpathlineto{\pgfqpoint{2.986098in}{2.044915in}}%
\pgfpathlineto{\pgfqpoint{2.994430in}{2.052733in}}%
\pgfpathlineto{\pgfqpoint{3.002756in}{2.060606in}}%
\pgfpathlineto{\pgfqpoint{2.989228in}{2.067224in}}%
\pgfpathlineto{\pgfqpoint{2.975704in}{2.073942in}}%
\pgfpathlineto{\pgfqpoint{2.962181in}{2.080762in}}%
\pgfpathlineto{\pgfqpoint{2.948662in}{2.087683in}}%
\pgfpathlineto{\pgfqpoint{2.940320in}{2.079927in}}%
\pgfpathlineto{\pgfqpoint{2.931970in}{2.072232in}}%
\pgfpathlineto{\pgfqpoint{2.923612in}{2.064597in}}%
\pgfpathlineto{\pgfqpoint{2.915247in}{2.057027in}}%
\pgfpathclose%
\pgfusepath{fill}%
\end{pgfscope}%
\begin{pgfscope}%
\pgfpathrectangle{\pgfqpoint{1.150000in}{0.150000in}}{\pgfqpoint{5.700000in}{5.700000in}}%
\pgfusepath{clip}%
\pgfsetbuttcap%
\pgfsetroundjoin%
\definecolor{currentfill}{rgb}{0.243113,0.292092,0.538516}%
\pgfsetfillcolor{currentfill}%
\pgfsetfillopacity{0.700000}%
\pgfsetlinewidth{0.000000pt}%
\definecolor{currentstroke}{rgb}{0.000000,0.000000,0.000000}%
\pgfsetstrokecolor{currentstroke}%
\pgfsetdash{}{0pt}%
\pgfpathmoveto{\pgfqpoint{5.401708in}{2.541807in}}%
\pgfpathlineto{\pgfqpoint{5.415855in}{2.542534in}}%
\pgfpathlineto{\pgfqpoint{5.430013in}{2.543328in}}%
\pgfpathlineto{\pgfqpoint{5.444182in}{2.544190in}}%
\pgfpathlineto{\pgfqpoint{5.458361in}{2.545119in}}%
\pgfpathlineto{\pgfqpoint{5.465735in}{2.551280in}}%
\pgfpathlineto{\pgfqpoint{5.473107in}{2.557543in}}%
\pgfpathlineto{\pgfqpoint{5.480476in}{2.563917in}}%
\pgfpathlineto{\pgfqpoint{5.487842in}{2.570406in}}%
\pgfpathlineto{\pgfqpoint{5.473686in}{2.569828in}}%
\pgfpathlineto{\pgfqpoint{5.459540in}{2.569317in}}%
\pgfpathlineto{\pgfqpoint{5.445405in}{2.568873in}}%
\pgfpathlineto{\pgfqpoint{5.431280in}{2.568497in}}%
\pgfpathlineto{\pgfqpoint{5.423891in}{2.561650in}}%
\pgfpathlineto{\pgfqpoint{5.416500in}{2.554923in}}%
\pgfpathlineto{\pgfqpoint{5.409105in}{2.548311in}}%
\pgfpathlineto{\pgfqpoint{5.401708in}{2.541807in}}%
\pgfpathclose%
\pgfusepath{fill}%
\end{pgfscope}%
\begin{pgfscope}%
\pgfpathrectangle{\pgfqpoint{1.150000in}{0.150000in}}{\pgfqpoint{5.700000in}{5.700000in}}%
\pgfusepath{clip}%
\pgfsetbuttcap%
\pgfsetroundjoin%
\definecolor{currentfill}{rgb}{0.273809,0.031497,0.358853}%
\pgfsetfillcolor{currentfill}%
\pgfsetfillopacity{0.700000}%
\pgfsetlinewidth{0.000000pt}%
\definecolor{currentstroke}{rgb}{0.000000,0.000000,0.000000}%
\pgfsetstrokecolor{currentstroke}%
\pgfsetdash{}{0pt}%
\pgfpathmoveto{\pgfqpoint{3.198364in}{2.022065in}}%
\pgfpathlineto{\pgfqpoint{3.211919in}{2.016803in}}%
\pgfpathlineto{\pgfqpoint{3.225478in}{2.011633in}}%
\pgfpathlineto{\pgfqpoint{3.239041in}{2.006556in}}%
\pgfpathlineto{\pgfqpoint{3.252608in}{2.001570in}}%
\pgfpathlineto{\pgfqpoint{3.260837in}{2.010061in}}%
\pgfpathlineto{\pgfqpoint{3.269061in}{2.018581in}}%
\pgfpathlineto{\pgfqpoint{3.277278in}{2.027127in}}%
\pgfpathlineto{\pgfqpoint{3.285488in}{2.035698in}}%
\pgfpathlineto{\pgfqpoint{3.271935in}{2.040602in}}%
\pgfpathlineto{\pgfqpoint{3.258386in}{2.045596in}}%
\pgfpathlineto{\pgfqpoint{3.244842in}{2.050683in}}%
\pgfpathlineto{\pgfqpoint{3.231301in}{2.055862in}}%
\pgfpathlineto{\pgfqpoint{3.223077in}{2.047366in}}%
\pgfpathlineto{\pgfqpoint{3.214846in}{2.038900in}}%
\pgfpathlineto{\pgfqpoint{3.206608in}{2.030466in}}%
\pgfpathlineto{\pgfqpoint{3.198364in}{2.022065in}}%
\pgfpathclose%
\pgfusepath{fill}%
\end{pgfscope}%
\begin{pgfscope}%
\pgfpathrectangle{\pgfqpoint{1.150000in}{0.150000in}}{\pgfqpoint{5.700000in}{5.700000in}}%
\pgfusepath{clip}%
\pgfsetbuttcap%
\pgfsetroundjoin%
\definecolor{currentfill}{rgb}{0.263663,0.237631,0.518762}%
\pgfsetfillcolor{currentfill}%
\pgfsetfillopacity{0.700000}%
\pgfsetlinewidth{0.000000pt}%
\definecolor{currentstroke}{rgb}{0.000000,0.000000,0.000000}%
\pgfsetstrokecolor{currentstroke}%
\pgfsetdash{}{0pt}%
\pgfpathmoveto{\pgfqpoint{5.000432in}{2.423559in}}%
\pgfpathlineto{\pgfqpoint{5.014452in}{2.424111in}}%
\pgfpathlineto{\pgfqpoint{5.028482in}{2.424733in}}%
\pgfpathlineto{\pgfqpoint{5.042522in}{2.425425in}}%
\pgfpathlineto{\pgfqpoint{5.056572in}{2.426186in}}%
\pgfpathlineto{\pgfqpoint{5.064116in}{2.432779in}}%
\pgfpathlineto{\pgfqpoint{5.071656in}{2.439413in}}%
\pgfpathlineto{\pgfqpoint{5.079190in}{2.446093in}}%
\pgfpathlineto{\pgfqpoint{5.086720in}{2.452823in}}%
\pgfpathlineto{\pgfqpoint{5.072688in}{2.452331in}}%
\pgfpathlineto{\pgfqpoint{5.058666in}{2.451908in}}%
\pgfpathlineto{\pgfqpoint{5.044653in}{2.451554in}}%
\pgfpathlineto{\pgfqpoint{5.030651in}{2.451270in}}%
\pgfpathlineto{\pgfqpoint{5.023104in}{2.444264in}}%
\pgfpathlineto{\pgfqpoint{5.015551in}{2.437313in}}%
\pgfpathlineto{\pgfqpoint{5.007994in}{2.430413in}}%
\pgfpathlineto{\pgfqpoint{5.000432in}{2.423559in}}%
\pgfpathclose%
\pgfusepath{fill}%
\end{pgfscope}%
\begin{pgfscope}%
\pgfpathrectangle{\pgfqpoint{1.150000in}{0.150000in}}{\pgfqpoint{5.700000in}{5.700000in}}%
\pgfusepath{clip}%
\pgfsetbuttcap%
\pgfsetroundjoin%
\definecolor{currentfill}{rgb}{0.278826,0.175490,0.483397}%
\pgfsetfillcolor{currentfill}%
\pgfsetfillopacity{0.700000}%
\pgfsetlinewidth{0.000000pt}%
\definecolor{currentstroke}{rgb}{0.000000,0.000000,0.000000}%
\pgfsetstrokecolor{currentstroke}%
\pgfsetdash{}{0pt}%
\pgfpathmoveto{\pgfqpoint{2.325179in}{2.317878in}}%
\pgfpathlineto{\pgfqpoint{2.338779in}{2.306115in}}%
\pgfpathlineto{\pgfqpoint{2.352377in}{2.294486in}}%
\pgfpathlineto{\pgfqpoint{2.365972in}{2.282991in}}%
\pgfpathlineto{\pgfqpoint{2.379566in}{2.271628in}}%
\pgfpathlineto{\pgfqpoint{2.388218in}{2.276718in}}%
\pgfpathlineto{\pgfqpoint{2.396858in}{2.281939in}}%
\pgfpathlineto{\pgfqpoint{2.405487in}{2.287287in}}%
\pgfpathlineto{\pgfqpoint{2.414104in}{2.292761in}}%
\pgfpathlineto{\pgfqpoint{2.400536in}{2.303933in}}%
\pgfpathlineto{\pgfqpoint{2.386967in}{2.315237in}}%
\pgfpathlineto{\pgfqpoint{2.373395in}{2.326675in}}%
\pgfpathlineto{\pgfqpoint{2.359821in}{2.338247in}}%
\pgfpathlineto{\pgfqpoint{2.351178in}{2.332956in}}%
\pgfpathlineto{\pgfqpoint{2.342524in}{2.327796in}}%
\pgfpathlineto{\pgfqpoint{2.333858in}{2.322769in}}%
\pgfpathlineto{\pgfqpoint{2.325179in}{2.317878in}}%
\pgfpathclose%
\pgfusepath{fill}%
\end{pgfscope}%
\begin{pgfscope}%
\pgfpathrectangle{\pgfqpoint{1.150000in}{0.150000in}}{\pgfqpoint{5.700000in}{5.700000in}}%
\pgfusepath{clip}%
\pgfsetbuttcap%
\pgfsetroundjoin%
\definecolor{currentfill}{rgb}{0.281446,0.084320,0.407414}%
\pgfsetfillcolor{currentfill}%
\pgfsetfillopacity{0.700000}%
\pgfsetlinewidth{0.000000pt}%
\definecolor{currentstroke}{rgb}{0.000000,0.000000,0.000000}%
\pgfsetstrokecolor{currentstroke}%
\pgfsetdash{}{0pt}%
\pgfpathmoveto{\pgfqpoint{3.882914in}{2.104411in}}%
\pgfpathlineto{\pgfqpoint{3.896594in}{2.102412in}}%
\pgfpathlineto{\pgfqpoint{3.910282in}{2.100492in}}%
\pgfpathlineto{\pgfqpoint{3.923977in}{2.098651in}}%
\pgfpathlineto{\pgfqpoint{3.937679in}{2.096889in}}%
\pgfpathlineto{\pgfqpoint{3.945660in}{2.105674in}}%
\pgfpathlineto{\pgfqpoint{3.953636in}{2.114442in}}%
\pgfpathlineto{\pgfqpoint{3.961606in}{2.123195in}}%
\pgfpathlineto{\pgfqpoint{3.969571in}{2.131934in}}%
\pgfpathlineto{\pgfqpoint{3.955879in}{2.133737in}}%
\pgfpathlineto{\pgfqpoint{3.942196in}{2.135618in}}%
\pgfpathlineto{\pgfqpoint{3.928519in}{2.137579in}}%
\pgfpathlineto{\pgfqpoint{3.914849in}{2.139618in}}%
\pgfpathlineto{\pgfqpoint{3.906874in}{2.130832in}}%
\pgfpathlineto{\pgfqpoint{3.898893in}{2.122036in}}%
\pgfpathlineto{\pgfqpoint{3.890906in}{2.113229in}}%
\pgfpathlineto{\pgfqpoint{3.882914in}{2.104411in}}%
\pgfpathclose%
\pgfusepath{fill}%
\end{pgfscope}%
\begin{pgfscope}%
\pgfpathrectangle{\pgfqpoint{1.150000in}{0.150000in}}{\pgfqpoint{5.700000in}{5.700000in}}%
\pgfusepath{clip}%
\pgfsetbuttcap%
\pgfsetroundjoin%
\definecolor{currentfill}{rgb}{0.282910,0.105393,0.426902}%
\pgfsetfillcolor{currentfill}%
\pgfsetfillopacity{0.700000}%
\pgfsetlinewidth{0.000000pt}%
\definecolor{currentstroke}{rgb}{0.000000,0.000000,0.000000}%
\pgfsetstrokecolor{currentstroke}%
\pgfsetdash{}{0pt}%
\pgfpathmoveto{\pgfqpoint{2.576820in}{2.168608in}}%
\pgfpathlineto{\pgfqpoint{2.590375in}{2.159057in}}%
\pgfpathlineto{\pgfqpoint{2.603930in}{2.149622in}}%
\pgfpathlineto{\pgfqpoint{2.617485in}{2.140305in}}%
\pgfpathlineto{\pgfqpoint{2.631040in}{2.131104in}}%
\pgfpathlineto{\pgfqpoint{2.639554in}{2.137405in}}%
\pgfpathlineto{\pgfqpoint{2.648058in}{2.143806in}}%
\pgfpathlineto{\pgfqpoint{2.656553in}{2.150304in}}%
\pgfpathlineto{\pgfqpoint{2.665038in}{2.156898in}}%
\pgfpathlineto{\pgfqpoint{2.651504in}{2.165932in}}%
\pgfpathlineto{\pgfqpoint{2.637971in}{2.175082in}}%
\pgfpathlineto{\pgfqpoint{2.624438in}{2.184349in}}%
\pgfpathlineto{\pgfqpoint{2.610904in}{2.193733in}}%
\pgfpathlineto{\pgfqpoint{2.602398in}{2.187299in}}%
\pgfpathlineto{\pgfqpoint{2.593882in}{2.180965in}}%
\pgfpathlineto{\pgfqpoint{2.585356in}{2.174734in}}%
\pgfpathlineto{\pgfqpoint{2.576820in}{2.168608in}}%
\pgfpathclose%
\pgfusepath{fill}%
\end{pgfscope}%
\begin{pgfscope}%
\pgfpathrectangle{\pgfqpoint{1.150000in}{0.150000in}}{\pgfqpoint{5.700000in}{5.700000in}}%
\pgfusepath{clip}%
\pgfsetbuttcap%
\pgfsetroundjoin%
\definecolor{currentfill}{rgb}{0.278012,0.180367,0.486697}%
\pgfsetfillcolor{currentfill}%
\pgfsetfillopacity{0.700000}%
\pgfsetlinewidth{0.000000pt}%
\definecolor{currentstroke}{rgb}{0.000000,0.000000,0.000000}%
\pgfsetstrokecolor{currentstroke}%
\pgfsetdash{}{0pt}%
\pgfpathmoveto{\pgfqpoint{4.599072in}{2.302913in}}%
\pgfpathlineto{\pgfqpoint{4.612962in}{2.302913in}}%
\pgfpathlineto{\pgfqpoint{4.626861in}{2.302986in}}%
\pgfpathlineto{\pgfqpoint{4.640769in}{2.303131in}}%
\pgfpathlineto{\pgfqpoint{4.654687in}{2.303348in}}%
\pgfpathlineto{\pgfqpoint{4.662399in}{2.310772in}}%
\pgfpathlineto{\pgfqpoint{4.670106in}{2.318196in}}%
\pgfpathlineto{\pgfqpoint{4.677807in}{2.325624in}}%
\pgfpathlineto{\pgfqpoint{4.685502in}{2.333060in}}%
\pgfpathlineto{\pgfqpoint{4.671598in}{2.333029in}}%
\pgfpathlineto{\pgfqpoint{4.657704in}{2.333069in}}%
\pgfpathlineto{\pgfqpoint{4.643819in}{2.333181in}}%
\pgfpathlineto{\pgfqpoint{4.629942in}{2.333366in}}%
\pgfpathlineto{\pgfqpoint{4.622233in}{2.325737in}}%
\pgfpathlineto{\pgfqpoint{4.614518in}{2.318122in}}%
\pgfpathlineto{\pgfqpoint{4.606798in}{2.310514in}}%
\pgfpathlineto{\pgfqpoint{4.599072in}{2.302913in}}%
\pgfpathclose%
\pgfusepath{fill}%
\end{pgfscope}%
\begin{pgfscope}%
\pgfpathrectangle{\pgfqpoint{1.150000in}{0.150000in}}{\pgfqpoint{5.700000in}{5.700000in}}%
\pgfusepath{clip}%
\pgfsetbuttcap%
\pgfsetroundjoin%
\definecolor{currentfill}{rgb}{0.283187,0.125848,0.444960}%
\pgfsetfillcolor{currentfill}%
\pgfsetfillopacity{0.700000}%
\pgfsetlinewidth{0.000000pt}%
\definecolor{currentstroke}{rgb}{0.000000,0.000000,0.000000}%
\pgfsetstrokecolor{currentstroke}%
\pgfsetdash{}{0pt}%
\pgfpathmoveto{\pgfqpoint{4.197691in}{2.184209in}}%
\pgfpathlineto{\pgfqpoint{4.211458in}{2.183266in}}%
\pgfpathlineto{\pgfqpoint{4.225234in}{2.182399in}}%
\pgfpathlineto{\pgfqpoint{4.239018in}{2.181607in}}%
\pgfpathlineto{\pgfqpoint{4.252810in}{2.180890in}}%
\pgfpathlineto{\pgfqpoint{4.260678in}{2.189190in}}%
\pgfpathlineto{\pgfqpoint{4.268541in}{2.197471in}}%
\pgfpathlineto{\pgfqpoint{4.276398in}{2.205736in}}%
\pgfpathlineto{\pgfqpoint{4.284250in}{2.213987in}}%
\pgfpathlineto{\pgfqpoint{4.270469in}{2.214806in}}%
\pgfpathlineto{\pgfqpoint{4.256697in}{2.215701in}}%
\pgfpathlineto{\pgfqpoint{4.242933in}{2.216670in}}%
\pgfpathlineto{\pgfqpoint{4.229176in}{2.217715in}}%
\pgfpathlineto{\pgfqpoint{4.221314in}{2.209354in}}%
\pgfpathlineto{\pgfqpoint{4.213445in}{2.200985in}}%
\pgfpathlineto{\pgfqpoint{4.205571in}{2.192604in}}%
\pgfpathlineto{\pgfqpoint{4.197691in}{2.184209in}}%
\pgfpathclose%
\pgfusepath{fill}%
\end{pgfscope}%
\begin{pgfscope}%
\pgfpathrectangle{\pgfqpoint{1.150000in}{0.150000in}}{\pgfqpoint{5.700000in}{5.700000in}}%
\pgfusepath{clip}%
\pgfsetbuttcap%
\pgfsetroundjoin%
\definecolor{currentfill}{rgb}{0.279566,0.067836,0.391917}%
\pgfsetfillcolor{currentfill}%
\pgfsetfillopacity{0.700000}%
\pgfsetlinewidth{0.000000pt}%
\definecolor{currentstroke}{rgb}{0.000000,0.000000,0.000000}%
\pgfsetstrokecolor{currentstroke}%
\pgfsetdash{}{0pt}%
\pgfpathmoveto{\pgfqpoint{2.773331in}{2.088687in}}%
\pgfpathlineto{\pgfqpoint{2.786873in}{2.080656in}}%
\pgfpathlineto{\pgfqpoint{2.800415in}{2.072733in}}%
\pgfpathlineto{\pgfqpoint{2.813959in}{2.064917in}}%
\pgfpathlineto{\pgfqpoint{2.827505in}{2.057206in}}%
\pgfpathlineto{\pgfqpoint{2.835921in}{2.064354in}}%
\pgfpathlineto{\pgfqpoint{2.844329in}{2.071577in}}%
\pgfpathlineto{\pgfqpoint{2.852729in}{2.078873in}}%
\pgfpathlineto{\pgfqpoint{2.861120in}{2.086242in}}%
\pgfpathlineto{\pgfqpoint{2.847593in}{2.093807in}}%
\pgfpathlineto{\pgfqpoint{2.834068in}{2.101478in}}%
\pgfpathlineto{\pgfqpoint{2.820544in}{2.109255in}}%
\pgfpathlineto{\pgfqpoint{2.807022in}{2.117141in}}%
\pgfpathlineto{\pgfqpoint{2.798612in}{2.109910in}}%
\pgfpathlineto{\pgfqpoint{2.790194in}{2.102756in}}%
\pgfpathlineto{\pgfqpoint{2.781767in}{2.095681in}}%
\pgfpathlineto{\pgfqpoint{2.773331in}{2.088687in}}%
\pgfpathclose%
\pgfusepath{fill}%
\end{pgfscope}%
\begin{pgfscope}%
\pgfpathrectangle{\pgfqpoint{1.150000in}{0.150000in}}{\pgfqpoint{5.700000in}{5.700000in}}%
\pgfusepath{clip}%
\pgfsetbuttcap%
\pgfsetroundjoin%
\definecolor{currentfill}{rgb}{0.225863,0.330805,0.547314}%
\pgfsetfillcolor{currentfill}%
\pgfsetfillopacity{0.700000}%
\pgfsetlinewidth{0.000000pt}%
\definecolor{currentstroke}{rgb}{0.000000,0.000000,0.000000}%
\pgfsetstrokecolor{currentstroke}%
\pgfsetdash{}{0pt}%
\pgfpathmoveto{\pgfqpoint{5.716836in}{2.630282in}}%
\pgfpathlineto{\pgfqpoint{5.731084in}{2.630965in}}%
\pgfpathlineto{\pgfqpoint{5.745342in}{2.631715in}}%
\pgfpathlineto{\pgfqpoint{5.759611in}{2.632531in}}%
\pgfpathlineto{\pgfqpoint{5.773892in}{2.633414in}}%
\pgfpathlineto{\pgfqpoint{5.781144in}{2.639625in}}%
\pgfpathlineto{\pgfqpoint{5.788395in}{2.646000in}}%
\pgfpathlineto{\pgfqpoint{5.795647in}{2.652547in}}%
\pgfpathlineto{\pgfqpoint{5.802898in}{2.659271in}}%
\pgfpathlineto{\pgfqpoint{5.788645in}{2.658801in}}%
\pgfpathlineto{\pgfqpoint{5.774403in}{2.658398in}}%
\pgfpathlineto{\pgfqpoint{5.760171in}{2.658060in}}%
\pgfpathlineto{\pgfqpoint{5.745950in}{2.657788in}}%
\pgfpathlineto{\pgfqpoint{5.738671in}{2.650644in}}%
\pgfpathlineto{\pgfqpoint{5.731393in}{2.643684in}}%
\pgfpathlineto{\pgfqpoint{5.724115in}{2.636898in}}%
\pgfpathlineto{\pgfqpoint{5.716836in}{2.630282in}}%
\pgfpathclose%
\pgfusepath{fill}%
\end{pgfscope}%
\begin{pgfscope}%
\pgfpathrectangle{\pgfqpoint{1.150000in}{0.150000in}}{\pgfqpoint{5.700000in}{5.700000in}}%
\pgfusepath{clip}%
\pgfsetbuttcap%
\pgfsetroundjoin%
\definecolor{currentfill}{rgb}{0.277018,0.050344,0.375715}%
\pgfsetfillcolor{currentfill}%
\pgfsetfillopacity{0.700000}%
\pgfsetlinewidth{0.000000pt}%
\definecolor{currentstroke}{rgb}{0.000000,0.000000,0.000000}%
\pgfsetstrokecolor{currentstroke}%
\pgfsetdash{}{0pt}%
\pgfpathmoveto{\pgfqpoint{3.568030in}{2.040436in}}%
\pgfpathlineto{\pgfqpoint{3.581643in}{2.037126in}}%
\pgfpathlineto{\pgfqpoint{3.595262in}{2.033901in}}%
\pgfpathlineto{\pgfqpoint{3.608886in}{2.030760in}}%
\pgfpathlineto{\pgfqpoint{3.622516in}{2.027702in}}%
\pgfpathlineto{\pgfqpoint{3.630610in}{2.036629in}}%
\pgfpathlineto{\pgfqpoint{3.638698in}{2.045553in}}%
\pgfpathlineto{\pgfqpoint{3.646781in}{2.054474in}}%
\pgfpathlineto{\pgfqpoint{3.654857in}{2.063392in}}%
\pgfpathlineto{\pgfqpoint{3.641239in}{2.066429in}}%
\pgfpathlineto{\pgfqpoint{3.627626in}{2.069549in}}%
\pgfpathlineto{\pgfqpoint{3.614019in}{2.072753in}}%
\pgfpathlineto{\pgfqpoint{3.600418in}{2.076041in}}%
\pgfpathlineto{\pgfqpoint{3.592330in}{2.067136in}}%
\pgfpathlineto{\pgfqpoint{3.584236in}{2.058234in}}%
\pgfpathlineto{\pgfqpoint{3.576136in}{2.049334in}}%
\pgfpathlineto{\pgfqpoint{3.568030in}{2.040436in}}%
\pgfpathclose%
\pgfusepath{fill}%
\end{pgfscope}%
\begin{pgfscope}%
\pgfpathrectangle{\pgfqpoint{1.150000in}{0.150000in}}{\pgfqpoint{5.700000in}{5.700000in}}%
\pgfusepath{clip}%
\pgfsetbuttcap%
\pgfsetroundjoin%
\definecolor{currentfill}{rgb}{0.273809,0.031497,0.358853}%
\pgfsetfillcolor{currentfill}%
\pgfsetfillopacity{0.700000}%
\pgfsetlinewidth{0.000000pt}%
\definecolor{currentstroke}{rgb}{0.000000,0.000000,0.000000}%
\pgfsetstrokecolor{currentstroke}%
\pgfsetdash{}{0pt}%
\pgfpathmoveto{\pgfqpoint{3.339744in}{2.016992in}}%
\pgfpathlineto{\pgfqpoint{3.353319in}{2.012539in}}%
\pgfpathlineto{\pgfqpoint{3.366899in}{2.008176in}}%
\pgfpathlineto{\pgfqpoint{3.380484in}{2.003901in}}%
\pgfpathlineto{\pgfqpoint{3.394074in}{1.999714in}}%
\pgfpathlineto{\pgfqpoint{3.402252in}{2.008450in}}%
\pgfpathlineto{\pgfqpoint{3.410423in}{2.017201in}}%
\pgfpathlineto{\pgfqpoint{3.418589in}{2.025966in}}%
\pgfpathlineto{\pgfqpoint{3.426748in}{2.034744in}}%
\pgfpathlineto{\pgfqpoint{3.413171in}{2.038869in}}%
\pgfpathlineto{\pgfqpoint{3.399600in}{2.043082in}}%
\pgfpathlineto{\pgfqpoint{3.386032in}{2.047384in}}%
\pgfpathlineto{\pgfqpoint{3.372470in}{2.051774in}}%
\pgfpathlineto{\pgfqpoint{3.364298in}{2.043050in}}%
\pgfpathlineto{\pgfqpoint{3.356120in}{2.034345in}}%
\pgfpathlineto{\pgfqpoint{3.347935in}{2.025659in}}%
\pgfpathlineto{\pgfqpoint{3.339744in}{2.016992in}}%
\pgfpathclose%
\pgfusepath{fill}%
\end{pgfscope}%
\begin{pgfscope}%
\pgfpathrectangle{\pgfqpoint{1.150000in}{0.150000in}}{\pgfqpoint{5.700000in}{5.700000in}}%
\pgfusepath{clip}%
\pgfsetbuttcap%
\pgfsetroundjoin%
\definecolor{currentfill}{rgb}{0.248629,0.278775,0.534556}%
\pgfsetfillcolor{currentfill}%
\pgfsetfillopacity{0.700000}%
\pgfsetlinewidth{0.000000pt}%
\definecolor{currentstroke}{rgb}{0.000000,0.000000,0.000000}%
\pgfsetstrokecolor{currentstroke}%
\pgfsetdash{}{0pt}%
\pgfpathmoveto{\pgfqpoint{5.315516in}{2.513202in}}%
\pgfpathlineto{\pgfqpoint{5.329643in}{2.513987in}}%
\pgfpathlineto{\pgfqpoint{5.343780in}{2.514841in}}%
\pgfpathlineto{\pgfqpoint{5.357927in}{2.515763in}}%
\pgfpathlineto{\pgfqpoint{5.372086in}{2.516753in}}%
\pgfpathlineto{\pgfqpoint{5.379497in}{2.522884in}}%
\pgfpathlineto{\pgfqpoint{5.386904in}{2.529099in}}%
\pgfpathlineto{\pgfqpoint{5.394308in}{2.535405in}}%
\pgfpathlineto{\pgfqpoint{5.401708in}{2.541807in}}%
\pgfpathlineto{\pgfqpoint{5.387572in}{2.541148in}}%
\pgfpathlineto{\pgfqpoint{5.373445in}{2.540557in}}%
\pgfpathlineto{\pgfqpoint{5.359330in}{2.540033in}}%
\pgfpathlineto{\pgfqpoint{5.345224in}{2.539577in}}%
\pgfpathlineto{\pgfqpoint{5.337803in}{2.532838in}}%
\pgfpathlineto{\pgfqpoint{5.330377in}{2.526199in}}%
\pgfpathlineto{\pgfqpoint{5.322949in}{2.519656in}}%
\pgfpathlineto{\pgfqpoint{5.315516in}{2.513202in}}%
\pgfpathclose%
\pgfusepath{fill}%
\end{pgfscope}%
\begin{pgfscope}%
\pgfpathrectangle{\pgfqpoint{1.150000in}{0.150000in}}{\pgfqpoint{5.700000in}{5.700000in}}%
\pgfusepath{clip}%
\pgfsetbuttcap%
\pgfsetroundjoin%
\definecolor{currentfill}{rgb}{0.266580,0.228262,0.514349}%
\pgfsetfillcolor{currentfill}%
\pgfsetfillopacity{0.700000}%
\pgfsetlinewidth{0.000000pt}%
\definecolor{currentstroke}{rgb}{0.000000,0.000000,0.000000}%
\pgfsetstrokecolor{currentstroke}%
\pgfsetdash{}{0pt}%
\pgfpathmoveto{\pgfqpoint{4.914084in}{2.394004in}}%
\pgfpathlineto{\pgfqpoint{4.928081in}{2.394525in}}%
\pgfpathlineto{\pgfqpoint{4.942088in}{2.395115in}}%
\pgfpathlineto{\pgfqpoint{4.956105in}{2.395776in}}%
\pgfpathlineto{\pgfqpoint{4.970132in}{2.396507in}}%
\pgfpathlineto{\pgfqpoint{4.977715in}{2.403224in}}%
\pgfpathlineto{\pgfqpoint{4.985293in}{2.409969in}}%
\pgfpathlineto{\pgfqpoint{4.992865in}{2.416746in}}%
\pgfpathlineto{\pgfqpoint{5.000432in}{2.423559in}}%
\pgfpathlineto{\pgfqpoint{4.986422in}{2.423076in}}%
\pgfpathlineto{\pgfqpoint{4.972422in}{2.422664in}}%
\pgfpathlineto{\pgfqpoint{4.958432in}{2.422321in}}%
\pgfpathlineto{\pgfqpoint{4.944451in}{2.422047in}}%
\pgfpathlineto{\pgfqpoint{4.936867in}{2.414979in}}%
\pgfpathlineto{\pgfqpoint{4.929278in}{2.407952in}}%
\pgfpathlineto{\pgfqpoint{4.921684in}{2.400962in}}%
\pgfpathlineto{\pgfqpoint{4.914084in}{2.394004in}}%
\pgfpathclose%
\pgfusepath{fill}%
\end{pgfscope}%
\begin{pgfscope}%
\pgfpathrectangle{\pgfqpoint{1.150000in}{0.150000in}}{\pgfqpoint{5.700000in}{5.700000in}}%
\pgfusepath{clip}%
\pgfsetbuttcap%
\pgfsetroundjoin%
\definecolor{currentfill}{rgb}{0.279574,0.170599,0.479997}%
\pgfsetfillcolor{currentfill}%
\pgfsetfillopacity{0.700000}%
\pgfsetlinewidth{0.000000pt}%
\definecolor{currentstroke}{rgb}{0.000000,0.000000,0.000000}%
\pgfsetstrokecolor{currentstroke}%
\pgfsetdash{}{0pt}%
\pgfpathmoveto{\pgfqpoint{4.512586in}{2.272557in}}%
\pgfpathlineto{\pgfqpoint{4.526453in}{2.272432in}}%
\pgfpathlineto{\pgfqpoint{4.540329in}{2.272380in}}%
\pgfpathlineto{\pgfqpoint{4.554215in}{2.272401in}}%
\pgfpathlineto{\pgfqpoint{4.568109in}{2.272494in}}%
\pgfpathlineto{\pgfqpoint{4.575858in}{2.280107in}}%
\pgfpathlineto{\pgfqpoint{4.583602in}{2.287712in}}%
\pgfpathlineto{\pgfqpoint{4.591340in}{2.295313in}}%
\pgfpathlineto{\pgfqpoint{4.599072in}{2.302913in}}%
\pgfpathlineto{\pgfqpoint{4.585191in}{2.302984in}}%
\pgfpathlineto{\pgfqpoint{4.571319in}{2.303128in}}%
\pgfpathlineto{\pgfqpoint{4.557456in}{2.303345in}}%
\pgfpathlineto{\pgfqpoint{4.543601in}{2.303634in}}%
\pgfpathlineto{\pgfqpoint{4.535856in}{2.295863in}}%
\pgfpathlineto{\pgfqpoint{4.528105in}{2.288095in}}%
\pgfpathlineto{\pgfqpoint{4.520348in}{2.280327in}}%
\pgfpathlineto{\pgfqpoint{4.512586in}{2.272557in}}%
\pgfpathclose%
\pgfusepath{fill}%
\end{pgfscope}%
\begin{pgfscope}%
\pgfpathrectangle{\pgfqpoint{1.150000in}{0.150000in}}{\pgfqpoint{5.700000in}{5.700000in}}%
\pgfusepath{clip}%
\pgfsetbuttcap%
\pgfsetroundjoin%
\definecolor{currentfill}{rgb}{0.281412,0.155834,0.469201}%
\pgfsetfillcolor{currentfill}%
\pgfsetfillopacity{0.700000}%
\pgfsetlinewidth{0.000000pt}%
\definecolor{currentstroke}{rgb}{0.000000,0.000000,0.000000}%
\pgfsetstrokecolor{currentstroke}%
\pgfsetdash{}{0pt}%
\pgfpathmoveto{\pgfqpoint{2.379566in}{2.271628in}}%
\pgfpathlineto{\pgfqpoint{2.393157in}{2.260397in}}%
\pgfpathlineto{\pgfqpoint{2.406747in}{2.249295in}}%
\pgfpathlineto{\pgfqpoint{2.420334in}{2.238323in}}%
\pgfpathlineto{\pgfqpoint{2.433921in}{2.227478in}}%
\pgfpathlineto{\pgfqpoint{2.442547in}{2.232767in}}%
\pgfpathlineto{\pgfqpoint{2.451163in}{2.238180in}}%
\pgfpathlineto{\pgfqpoint{2.459767in}{2.243716in}}%
\pgfpathlineto{\pgfqpoint{2.468360in}{2.249373in}}%
\pgfpathlineto{\pgfqpoint{2.454798in}{2.260027in}}%
\pgfpathlineto{\pgfqpoint{2.441235in}{2.270809in}}%
\pgfpathlineto{\pgfqpoint{2.427670in}{2.281720in}}%
\pgfpathlineto{\pgfqpoint{2.414104in}{2.292761in}}%
\pgfpathlineto{\pgfqpoint{2.405487in}{2.287287in}}%
\pgfpathlineto{\pgfqpoint{2.396858in}{2.281939in}}%
\pgfpathlineto{\pgfqpoint{2.388218in}{2.276718in}}%
\pgfpathlineto{\pgfqpoint{2.379566in}{2.271628in}}%
\pgfpathclose%
\pgfusepath{fill}%
\end{pgfscope}%
\begin{pgfscope}%
\pgfpathrectangle{\pgfqpoint{1.150000in}{0.150000in}}{\pgfqpoint{5.700000in}{5.700000in}}%
\pgfusepath{clip}%
\pgfsetbuttcap%
\pgfsetroundjoin%
\definecolor{currentfill}{rgb}{0.280267,0.073417,0.397163}%
\pgfsetfillcolor{currentfill}%
\pgfsetfillopacity{0.700000}%
\pgfsetlinewidth{0.000000pt}%
\definecolor{currentstroke}{rgb}{0.000000,0.000000,0.000000}%
\pgfsetstrokecolor{currentstroke}%
\pgfsetdash{}{0pt}%
\pgfpathmoveto{\pgfqpoint{3.796190in}{2.077721in}}%
\pgfpathlineto{\pgfqpoint{3.809854in}{2.075421in}}%
\pgfpathlineto{\pgfqpoint{3.823525in}{2.073202in}}%
\pgfpathlineto{\pgfqpoint{3.837203in}{2.071064in}}%
\pgfpathlineto{\pgfqpoint{3.850888in}{2.069005in}}%
\pgfpathlineto{\pgfqpoint{3.858903in}{2.077878in}}%
\pgfpathlineto{\pgfqpoint{3.866912in}{2.086737in}}%
\pgfpathlineto{\pgfqpoint{3.874916in}{2.095581in}}%
\pgfpathlineto{\pgfqpoint{3.882914in}{2.104411in}}%
\pgfpathlineto{\pgfqpoint{3.869240in}{2.106490in}}%
\pgfpathlineto{\pgfqpoint{3.855573in}{2.108649in}}%
\pgfpathlineto{\pgfqpoint{3.841913in}{2.110888in}}%
\pgfpathlineto{\pgfqpoint{3.828259in}{2.113207in}}%
\pgfpathlineto{\pgfqpoint{3.820250in}{2.104349in}}%
\pgfpathlineto{\pgfqpoint{3.812236in}{2.095482in}}%
\pgfpathlineto{\pgfqpoint{3.804216in}{2.086607in}}%
\pgfpathlineto{\pgfqpoint{3.796190in}{2.077721in}}%
\pgfpathclose%
\pgfusepath{fill}%
\end{pgfscope}%
\begin{pgfscope}%
\pgfpathrectangle{\pgfqpoint{1.150000in}{0.150000in}}{\pgfqpoint{5.700000in}{5.700000in}}%
\pgfusepath{clip}%
\pgfsetbuttcap%
\pgfsetroundjoin%
\definecolor{currentfill}{rgb}{0.283197,0.115680,0.436115}%
\pgfsetfillcolor{currentfill}%
\pgfsetfillopacity{0.700000}%
\pgfsetlinewidth{0.000000pt}%
\definecolor{currentstroke}{rgb}{0.000000,0.000000,0.000000}%
\pgfsetstrokecolor{currentstroke}%
\pgfsetdash{}{0pt}%
\pgfpathmoveto{\pgfqpoint{4.111077in}{2.154668in}}%
\pgfpathlineto{\pgfqpoint{4.124824in}{2.153502in}}%
\pgfpathlineto{\pgfqpoint{4.138579in}{2.152413in}}%
\pgfpathlineto{\pgfqpoint{4.152342in}{2.151400in}}%
\pgfpathlineto{\pgfqpoint{4.166113in}{2.150463in}}%
\pgfpathlineto{\pgfqpoint{4.174016in}{2.158929in}}%
\pgfpathlineto{\pgfqpoint{4.181913in}{2.167374in}}%
\pgfpathlineto{\pgfqpoint{4.189805in}{2.175800in}}%
\pgfpathlineto{\pgfqpoint{4.197691in}{2.184209in}}%
\pgfpathlineto{\pgfqpoint{4.183931in}{2.185228in}}%
\pgfpathlineto{\pgfqpoint{4.170179in}{2.186323in}}%
\pgfpathlineto{\pgfqpoint{4.156435in}{2.187494in}}%
\pgfpathlineto{\pgfqpoint{4.142699in}{2.188741in}}%
\pgfpathlineto{\pgfqpoint{4.134802in}{2.180243in}}%
\pgfpathlineto{\pgfqpoint{4.126900in}{2.171732in}}%
\pgfpathlineto{\pgfqpoint{4.118991in}{2.163208in}}%
\pgfpathlineto{\pgfqpoint{4.111077in}{2.154668in}}%
\pgfpathclose%
\pgfusepath{fill}%
\end{pgfscope}%
\begin{pgfscope}%
\pgfpathrectangle{\pgfqpoint{1.150000in}{0.150000in}}{\pgfqpoint{5.700000in}{5.700000in}}%
\pgfusepath{clip}%
\pgfsetbuttcap%
\pgfsetroundjoin%
\definecolor{currentfill}{rgb}{0.231674,0.318106,0.544834}%
\pgfsetfillcolor{currentfill}%
\pgfsetfillopacity{0.700000}%
\pgfsetlinewidth{0.000000pt}%
\definecolor{currentstroke}{rgb}{0.000000,0.000000,0.000000}%
\pgfsetstrokecolor{currentstroke}%
\pgfsetdash{}{0pt}%
\pgfpathmoveto{\pgfqpoint{5.630731in}{2.601721in}}%
\pgfpathlineto{\pgfqpoint{5.644960in}{2.602531in}}%
\pgfpathlineto{\pgfqpoint{5.659200in}{2.603407in}}%
\pgfpathlineto{\pgfqpoint{5.673451in}{2.604350in}}%
\pgfpathlineto{\pgfqpoint{5.687714in}{2.605360in}}%
\pgfpathlineto{\pgfqpoint{5.694996in}{2.611372in}}%
\pgfpathlineto{\pgfqpoint{5.702277in}{2.617526in}}%
\pgfpathlineto{\pgfqpoint{5.709557in}{2.623826in}}%
\pgfpathlineto{\pgfqpoint{5.716836in}{2.630282in}}%
\pgfpathlineto{\pgfqpoint{5.702600in}{2.629665in}}%
\pgfpathlineto{\pgfqpoint{5.688375in}{2.629114in}}%
\pgfpathlineto{\pgfqpoint{5.674160in}{2.628629in}}%
\pgfpathlineto{\pgfqpoint{5.659956in}{2.628211in}}%
\pgfpathlineto{\pgfqpoint{5.652651in}{2.621357in}}%
\pgfpathlineto{\pgfqpoint{5.645346in}{2.614662in}}%
\pgfpathlineto{\pgfqpoint{5.638039in}{2.608119in}}%
\pgfpathlineto{\pgfqpoint{5.630731in}{2.601721in}}%
\pgfpathclose%
\pgfusepath{fill}%
\end{pgfscope}%
\begin{pgfscope}%
\pgfpathrectangle{\pgfqpoint{1.150000in}{0.150000in}}{\pgfqpoint{5.700000in}{5.700000in}}%
\pgfusepath{clip}%
\pgfsetbuttcap%
\pgfsetroundjoin%
\definecolor{currentfill}{rgb}{0.281924,0.089666,0.412415}%
\pgfsetfillcolor{currentfill}%
\pgfsetfillopacity{0.700000}%
\pgfsetlinewidth{0.000000pt}%
\definecolor{currentstroke}{rgb}{0.000000,0.000000,0.000000}%
\pgfsetstrokecolor{currentstroke}%
\pgfsetdash{}{0pt}%
\pgfpathmoveto{\pgfqpoint{2.631040in}{2.131104in}}%
\pgfpathlineto{\pgfqpoint{2.644596in}{2.122017in}}%
\pgfpathlineto{\pgfqpoint{2.658151in}{2.113045in}}%
\pgfpathlineto{\pgfqpoint{2.671708in}{2.104186in}}%
\pgfpathlineto{\pgfqpoint{2.685264in}{2.095439in}}%
\pgfpathlineto{\pgfqpoint{2.693757in}{2.101915in}}%
\pgfpathlineto{\pgfqpoint{2.702240in}{2.108485in}}%
\pgfpathlineto{\pgfqpoint{2.710714in}{2.115148in}}%
\pgfpathlineto{\pgfqpoint{2.719178in}{2.121901in}}%
\pgfpathlineto{\pgfqpoint{2.705642in}{2.130481in}}%
\pgfpathlineto{\pgfqpoint{2.692107in}{2.139173in}}%
\pgfpathlineto{\pgfqpoint{2.678572in}{2.147979in}}%
\pgfpathlineto{\pgfqpoint{2.665038in}{2.156898in}}%
\pgfpathlineto{\pgfqpoint{2.656553in}{2.150304in}}%
\pgfpathlineto{\pgfqpoint{2.648058in}{2.143806in}}%
\pgfpathlineto{\pgfqpoint{2.639554in}{2.137405in}}%
\pgfpathlineto{\pgfqpoint{2.631040in}{2.131104in}}%
\pgfpathclose%
\pgfusepath{fill}%
\end{pgfscope}%
\begin{pgfscope}%
\pgfpathrectangle{\pgfqpoint{1.150000in}{0.150000in}}{\pgfqpoint{5.700000in}{5.700000in}}%
\pgfusepath{clip}%
\pgfsetbuttcap%
\pgfsetroundjoin%
\definecolor{currentfill}{rgb}{0.252194,0.269783,0.531579}%
\pgfsetfillcolor{currentfill}%
\pgfsetfillopacity{0.700000}%
\pgfsetlinewidth{0.000000pt}%
\definecolor{currentstroke}{rgb}{0.000000,0.000000,0.000000}%
\pgfsetstrokecolor{currentstroke}%
\pgfsetdash{}{0pt}%
\pgfpathmoveto{\pgfqpoint{5.229264in}{2.484462in}}%
\pgfpathlineto{\pgfqpoint{5.243369in}{2.485285in}}%
\pgfpathlineto{\pgfqpoint{5.257484in}{2.486177in}}%
\pgfpathlineto{\pgfqpoint{5.271610in}{2.487137in}}%
\pgfpathlineto{\pgfqpoint{5.285747in}{2.488165in}}%
\pgfpathlineto{\pgfqpoint{5.293196in}{2.494318in}}%
\pgfpathlineto{\pgfqpoint{5.300640in}{2.500538in}}%
\pgfpathlineto{\pgfqpoint{5.308080in}{2.506831in}}%
\pgfpathlineto{\pgfqpoint{5.315516in}{2.513202in}}%
\pgfpathlineto{\pgfqpoint{5.301400in}{2.512484in}}%
\pgfpathlineto{\pgfqpoint{5.287295in}{2.511834in}}%
\pgfpathlineto{\pgfqpoint{5.273200in}{2.511253in}}%
\pgfpathlineto{\pgfqpoint{5.259115in}{2.510739in}}%
\pgfpathlineto{\pgfqpoint{5.251658in}{2.504052in}}%
\pgfpathlineto{\pgfqpoint{5.244197in}{2.497446in}}%
\pgfpathlineto{\pgfqpoint{5.236733in}{2.490918in}}%
\pgfpathlineto{\pgfqpoint{5.229264in}{2.484462in}}%
\pgfpathclose%
\pgfusepath{fill}%
\end{pgfscope}%
\begin{pgfscope}%
\pgfpathrectangle{\pgfqpoint{1.150000in}{0.150000in}}{\pgfqpoint{5.700000in}{5.700000in}}%
\pgfusepath{clip}%
\pgfsetbuttcap%
\pgfsetroundjoin%
\definecolor{currentfill}{rgb}{0.214298,0.355619,0.551184}%
\pgfsetfillcolor{currentfill}%
\pgfsetfillopacity{0.700000}%
\pgfsetlinewidth{0.000000pt}%
\definecolor{currentstroke}{rgb}{0.000000,0.000000,0.000000}%
\pgfsetstrokecolor{currentstroke}%
\pgfsetdash{}{0pt}%
\pgfpathmoveto{\pgfqpoint{5.946114in}{2.690770in}}%
\pgfpathlineto{\pgfqpoint{5.960438in}{2.691399in}}%
\pgfpathlineto{\pgfqpoint{5.974774in}{2.692093in}}%
\pgfpathlineto{\pgfqpoint{5.989122in}{2.692853in}}%
\pgfpathlineto{\pgfqpoint{5.996300in}{2.699357in}}%
\pgfpathlineto{\pgfqpoint{6.003481in}{2.706072in}}%
\pgfpathlineto{\pgfqpoint{6.010665in}{2.713006in}}%
\pgfpathlineto{\pgfqpoint{6.017851in}{2.720166in}}%
\pgfpathlineto{\pgfqpoint{6.003534in}{2.719860in}}%
\pgfpathlineto{\pgfqpoint{5.989228in}{2.719619in}}%
\pgfpathlineto{\pgfqpoint{5.974933in}{2.719443in}}%
\pgfpathlineto{\pgfqpoint{5.967724in}{2.711937in}}%
\pgfpathlineto{\pgfqpoint{5.960518in}{2.704662in}}%
\pgfpathlineto{\pgfqpoint{5.953315in}{2.697609in}}%
\pgfpathlineto{\pgfqpoint{5.946114in}{2.690770in}}%
\pgfpathclose%
\pgfusepath{fill}%
\end{pgfscope}%
\begin{pgfscope}%
\pgfpathrectangle{\pgfqpoint{1.150000in}{0.150000in}}{\pgfqpoint{5.700000in}{5.700000in}}%
\pgfusepath{clip}%
\pgfsetbuttcap%
\pgfsetroundjoin%
\definecolor{currentfill}{rgb}{0.274952,0.037752,0.364543}%
\pgfsetfillcolor{currentfill}%
\pgfsetfillopacity{0.700000}%
\pgfsetlinewidth{0.000000pt}%
\definecolor{currentstroke}{rgb}{0.000000,0.000000,0.000000}%
\pgfsetstrokecolor{currentstroke}%
\pgfsetdash{}{0pt}%
\pgfpathmoveto{\pgfqpoint{2.969409in}{2.029452in}}%
\pgfpathlineto{\pgfqpoint{2.982956in}{2.022809in}}%
\pgfpathlineto{\pgfqpoint{2.996505in}{2.016266in}}%
\pgfpathlineto{\pgfqpoint{3.010058in}{2.009822in}}%
\pgfpathlineto{\pgfqpoint{3.023613in}{2.003475in}}%
\pgfpathlineto{\pgfqpoint{3.031944in}{2.011310in}}%
\pgfpathlineto{\pgfqpoint{3.040268in}{2.019197in}}%
\pgfpathlineto{\pgfqpoint{3.048584in}{2.027137in}}%
\pgfpathlineto{\pgfqpoint{3.056893in}{2.035126in}}%
\pgfpathlineto{\pgfqpoint{3.043354in}{2.041348in}}%
\pgfpathlineto{\pgfqpoint{3.029819in}{2.047669in}}%
\pgfpathlineto{\pgfqpoint{3.016286in}{2.054088in}}%
\pgfpathlineto{\pgfqpoint{3.002756in}{2.060606in}}%
\pgfpathlineto{\pgfqpoint{2.994430in}{2.052733in}}%
\pgfpathlineto{\pgfqpoint{2.986098in}{2.044915in}}%
\pgfpathlineto{\pgfqpoint{2.977757in}{2.037154in}}%
\pgfpathlineto{\pgfqpoint{2.969409in}{2.029452in}}%
\pgfpathclose%
\pgfusepath{fill}%
\end{pgfscope}%
\begin{pgfscope}%
\pgfpathrectangle{\pgfqpoint{1.150000in}{0.150000in}}{\pgfqpoint{5.700000in}{5.700000in}}%
\pgfusepath{clip}%
\pgfsetbuttcap%
\pgfsetroundjoin%
\definecolor{currentfill}{rgb}{0.273809,0.031497,0.358853}%
\pgfsetfillcolor{currentfill}%
\pgfsetfillopacity{0.700000}%
\pgfsetlinewidth{0.000000pt}%
\definecolor{currentstroke}{rgb}{0.000000,0.000000,0.000000}%
\pgfsetstrokecolor{currentstroke}%
\pgfsetdash{}{0pt}%
\pgfpathmoveto{\pgfqpoint{3.111080in}{2.011205in}}%
\pgfpathlineto{\pgfqpoint{3.124635in}{2.005464in}}%
\pgfpathlineto{\pgfqpoint{3.138193in}{1.999818in}}%
\pgfpathlineto{\pgfqpoint{3.151755in}{1.994266in}}%
\pgfpathlineto{\pgfqpoint{3.165321in}{1.988808in}}%
\pgfpathlineto{\pgfqpoint{3.173592in}{1.997067in}}%
\pgfpathlineto{\pgfqpoint{3.181856in}{2.005364in}}%
\pgfpathlineto{\pgfqpoint{3.190114in}{2.013697in}}%
\pgfpathlineto{\pgfqpoint{3.198364in}{2.022065in}}%
\pgfpathlineto{\pgfqpoint{3.184813in}{2.027420in}}%
\pgfpathlineto{\pgfqpoint{3.171266in}{2.032869in}}%
\pgfpathlineto{\pgfqpoint{3.157723in}{2.038412in}}%
\pgfpathlineto{\pgfqpoint{3.144183in}{2.044049in}}%
\pgfpathlineto{\pgfqpoint{3.135918in}{2.035777in}}%
\pgfpathlineto{\pgfqpoint{3.127645in}{2.027545in}}%
\pgfpathlineto{\pgfqpoint{3.119366in}{2.019354in}}%
\pgfpathlineto{\pgfqpoint{3.111080in}{2.011205in}}%
\pgfpathclose%
\pgfusepath{fill}%
\end{pgfscope}%
\begin{pgfscope}%
\pgfpathrectangle{\pgfqpoint{1.150000in}{0.150000in}}{\pgfqpoint{5.700000in}{5.700000in}}%
\pgfusepath{clip}%
\pgfsetbuttcap%
\pgfsetroundjoin%
\definecolor{currentfill}{rgb}{0.270595,0.214069,0.507052}%
\pgfsetfillcolor{currentfill}%
\pgfsetfillopacity{0.700000}%
\pgfsetlinewidth{0.000000pt}%
\definecolor{currentstroke}{rgb}{0.000000,0.000000,0.000000}%
\pgfsetstrokecolor{currentstroke}%
\pgfsetdash{}{0pt}%
\pgfpathmoveto{\pgfqpoint{4.827676in}{2.364123in}}%
\pgfpathlineto{\pgfqpoint{4.841650in}{2.364589in}}%
\pgfpathlineto{\pgfqpoint{4.855634in}{2.365126in}}%
\pgfpathlineto{\pgfqpoint{4.869627in}{2.365733in}}%
\pgfpathlineto{\pgfqpoint{4.883630in}{2.366411in}}%
\pgfpathlineto{\pgfqpoint{4.891252in}{2.373282in}}%
\pgfpathlineto{\pgfqpoint{4.898868in}{2.380169in}}%
\pgfpathlineto{\pgfqpoint{4.906479in}{2.387075in}}%
\pgfpathlineto{\pgfqpoint{4.914084in}{2.394004in}}%
\pgfpathlineto{\pgfqpoint{4.900097in}{2.393554in}}%
\pgfpathlineto{\pgfqpoint{4.886119in}{2.393174in}}%
\pgfpathlineto{\pgfqpoint{4.872151in}{2.392864in}}%
\pgfpathlineto{\pgfqpoint{4.858193in}{2.392625in}}%
\pgfpathlineto{\pgfqpoint{4.850572in}{2.385461in}}%
\pgfpathlineto{\pgfqpoint{4.842946in}{2.378325in}}%
\pgfpathlineto{\pgfqpoint{4.835314in}{2.371214in}}%
\pgfpathlineto{\pgfqpoint{4.827676in}{2.364123in}}%
\pgfpathclose%
\pgfusepath{fill}%
\end{pgfscope}%
\begin{pgfscope}%
\pgfpathrectangle{\pgfqpoint{1.150000in}{0.150000in}}{\pgfqpoint{5.700000in}{5.700000in}}%
\pgfusepath{clip}%
\pgfsetbuttcap%
\pgfsetroundjoin%
\definecolor{currentfill}{rgb}{0.274952,0.037752,0.364543}%
\pgfsetfillcolor{currentfill}%
\pgfsetfillopacity{0.700000}%
\pgfsetlinewidth{0.000000pt}%
\definecolor{currentstroke}{rgb}{0.000000,0.000000,0.000000}%
\pgfsetstrokecolor{currentstroke}%
\pgfsetdash{}{0pt}%
\pgfpathmoveto{\pgfqpoint{3.481106in}{2.019116in}}%
\pgfpathlineto{\pgfqpoint{3.494709in}{2.015425in}}%
\pgfpathlineto{\pgfqpoint{3.508317in}{2.011819in}}%
\pgfpathlineto{\pgfqpoint{3.521931in}{2.008299in}}%
\pgfpathlineto{\pgfqpoint{3.535550in}{2.004864in}}%
\pgfpathlineto{\pgfqpoint{3.543678in}{2.013753in}}%
\pgfpathlineto{\pgfqpoint{3.551802in}{2.022645in}}%
\pgfpathlineto{\pgfqpoint{3.559919in}{2.031539in}}%
\pgfpathlineto{\pgfqpoint{3.568030in}{2.040436in}}%
\pgfpathlineto{\pgfqpoint{3.554423in}{2.043829in}}%
\pgfpathlineto{\pgfqpoint{3.540822in}{2.047308in}}%
\pgfpathlineto{\pgfqpoint{3.527226in}{2.050872in}}%
\pgfpathlineto{\pgfqpoint{3.513635in}{2.054521in}}%
\pgfpathlineto{\pgfqpoint{3.505512in}{2.045659in}}%
\pgfpathlineto{\pgfqpoint{3.497383in}{2.036804in}}%
\pgfpathlineto{\pgfqpoint{3.489248in}{2.027956in}}%
\pgfpathlineto{\pgfqpoint{3.481106in}{2.019116in}}%
\pgfpathclose%
\pgfusepath{fill}%
\end{pgfscope}%
\begin{pgfscope}%
\pgfpathrectangle{\pgfqpoint{1.150000in}{0.150000in}}{\pgfqpoint{5.700000in}{5.700000in}}%
\pgfusepath{clip}%
\pgfsetbuttcap%
\pgfsetroundjoin%
\definecolor{currentfill}{rgb}{0.280868,0.160771,0.472899}%
\pgfsetfillcolor{currentfill}%
\pgfsetfillopacity{0.700000}%
\pgfsetlinewidth{0.000000pt}%
\definecolor{currentstroke}{rgb}{0.000000,0.000000,0.000000}%
\pgfsetstrokecolor{currentstroke}%
\pgfsetdash{}{0pt}%
\pgfpathmoveto{\pgfqpoint{4.426046in}{2.242047in}}%
\pgfpathlineto{\pgfqpoint{4.439891in}{2.241773in}}%
\pgfpathlineto{\pgfqpoint{4.453744in}{2.241573in}}%
\pgfpathlineto{\pgfqpoint{4.467606in}{2.241446in}}%
\pgfpathlineto{\pgfqpoint{4.481477in}{2.241393in}}%
\pgfpathlineto{\pgfqpoint{4.489263in}{2.249202in}}%
\pgfpathlineto{\pgfqpoint{4.497043in}{2.256998in}}%
\pgfpathlineto{\pgfqpoint{4.504818in}{2.264782in}}%
\pgfpathlineto{\pgfqpoint{4.512586in}{2.272557in}}%
\pgfpathlineto{\pgfqpoint{4.498728in}{2.272755in}}%
\pgfpathlineto{\pgfqpoint{4.484878in}{2.273026in}}%
\pgfpathlineto{\pgfqpoint{4.471037in}{2.273370in}}%
\pgfpathlineto{\pgfqpoint{4.457205in}{2.273788in}}%
\pgfpathlineto{\pgfqpoint{4.449424in}{2.265860in}}%
\pgfpathlineto{\pgfqpoint{4.441637in}{2.257930in}}%
\pgfpathlineto{\pgfqpoint{4.433845in}{2.249993in}}%
\pgfpathlineto{\pgfqpoint{4.426046in}{2.242047in}}%
\pgfpathclose%
\pgfusepath{fill}%
\end{pgfscope}%
\begin{pgfscope}%
\pgfpathrectangle{\pgfqpoint{1.150000in}{0.150000in}}{\pgfqpoint{5.700000in}{5.700000in}}%
\pgfusepath{clip}%
\pgfsetbuttcap%
\pgfsetroundjoin%
\definecolor{currentfill}{rgb}{0.282656,0.100196,0.422160}%
\pgfsetfillcolor{currentfill}%
\pgfsetfillopacity{0.700000}%
\pgfsetlinewidth{0.000000pt}%
\definecolor{currentstroke}{rgb}{0.000000,0.000000,0.000000}%
\pgfsetstrokecolor{currentstroke}%
\pgfsetdash{}{0pt}%
\pgfpathmoveto{\pgfqpoint{4.024407in}{2.125506in}}%
\pgfpathlineto{\pgfqpoint{4.038135in}{2.124093in}}%
\pgfpathlineto{\pgfqpoint{4.051870in}{2.122758in}}%
\pgfpathlineto{\pgfqpoint{4.065613in}{2.121500in}}%
\pgfpathlineto{\pgfqpoint{4.079363in}{2.120319in}}%
\pgfpathlineto{\pgfqpoint{4.087300in}{2.128937in}}%
\pgfpathlineto{\pgfqpoint{4.095232in}{2.137534in}}%
\pgfpathlineto{\pgfqpoint{4.103157in}{2.146110in}}%
\pgfpathlineto{\pgfqpoint{4.111077in}{2.154668in}}%
\pgfpathlineto{\pgfqpoint{4.097338in}{2.155910in}}%
\pgfpathlineto{\pgfqpoint{4.083606in}{2.157230in}}%
\pgfpathlineto{\pgfqpoint{4.069881in}{2.158626in}}%
\pgfpathlineto{\pgfqpoint{4.056165in}{2.160100in}}%
\pgfpathlineto{\pgfqpoint{4.048234in}{2.151473in}}%
\pgfpathlineto{\pgfqpoint{4.040297in}{2.142833in}}%
\pgfpathlineto{\pgfqpoint{4.032355in}{2.134178in}}%
\pgfpathlineto{\pgfqpoint{4.024407in}{2.125506in}}%
\pgfpathclose%
\pgfusepath{fill}%
\end{pgfscope}%
\begin{pgfscope}%
\pgfpathrectangle{\pgfqpoint{1.150000in}{0.150000in}}{\pgfqpoint{5.700000in}{5.700000in}}%
\pgfusepath{clip}%
\pgfsetbuttcap%
\pgfsetroundjoin%
\definecolor{currentfill}{rgb}{0.282623,0.140926,0.457517}%
\pgfsetfillcolor{currentfill}%
\pgfsetfillopacity{0.700000}%
\pgfsetlinewidth{0.000000pt}%
\definecolor{currentstroke}{rgb}{0.000000,0.000000,0.000000}%
\pgfsetstrokecolor{currentstroke}%
\pgfsetdash{}{0pt}%
\pgfpathmoveto{\pgfqpoint{2.433921in}{2.227478in}}%
\pgfpathlineto{\pgfqpoint{2.447506in}{2.216761in}}%
\pgfpathlineto{\pgfqpoint{2.461089in}{2.206169in}}%
\pgfpathlineto{\pgfqpoint{2.474672in}{2.195702in}}%
\pgfpathlineto{\pgfqpoint{2.488254in}{2.185358in}}%
\pgfpathlineto{\pgfqpoint{2.496855in}{2.190844in}}%
\pgfpathlineto{\pgfqpoint{2.505446in}{2.196450in}}%
\pgfpathlineto{\pgfqpoint{2.514026in}{2.202173in}}%
\pgfpathlineto{\pgfqpoint{2.522595in}{2.208011in}}%
\pgfpathlineto{\pgfqpoint{2.509038in}{2.218165in}}%
\pgfpathlineto{\pgfqpoint{2.495480in}{2.228443in}}%
\pgfpathlineto{\pgfqpoint{2.481920in}{2.238845in}}%
\pgfpathlineto{\pgfqpoint{2.468360in}{2.249373in}}%
\pgfpathlineto{\pgfqpoint{2.459767in}{2.243716in}}%
\pgfpathlineto{\pgfqpoint{2.451163in}{2.238180in}}%
\pgfpathlineto{\pgfqpoint{2.442547in}{2.232767in}}%
\pgfpathlineto{\pgfqpoint{2.433921in}{2.227478in}}%
\pgfpathclose%
\pgfusepath{fill}%
\end{pgfscope}%
\begin{pgfscope}%
\pgfpathrectangle{\pgfqpoint{1.150000in}{0.150000in}}{\pgfqpoint{5.700000in}{5.700000in}}%
\pgfusepath{clip}%
\pgfsetbuttcap%
\pgfsetroundjoin%
\definecolor{currentfill}{rgb}{0.277941,0.056324,0.381191}%
\pgfsetfillcolor{currentfill}%
\pgfsetfillopacity{0.700000}%
\pgfsetlinewidth{0.000000pt}%
\definecolor{currentstroke}{rgb}{0.000000,0.000000,0.000000}%
\pgfsetstrokecolor{currentstroke}%
\pgfsetdash{}{0pt}%
\pgfpathmoveto{\pgfqpoint{2.827505in}{2.057206in}}%
\pgfpathlineto{\pgfqpoint{2.841052in}{2.049601in}}%
\pgfpathlineto{\pgfqpoint{2.854601in}{2.042101in}}%
\pgfpathlineto{\pgfqpoint{2.868152in}{2.034704in}}%
\pgfpathlineto{\pgfqpoint{2.881705in}{2.027411in}}%
\pgfpathlineto{\pgfqpoint{2.890103in}{2.034711in}}%
\pgfpathlineto{\pgfqpoint{2.898492in}{2.042082in}}%
\pgfpathlineto{\pgfqpoint{2.906874in}{2.049521in}}%
\pgfpathlineto{\pgfqpoint{2.915247in}{2.057027in}}%
\pgfpathlineto{\pgfqpoint{2.901712in}{2.064175in}}%
\pgfpathlineto{\pgfqpoint{2.888180in}{2.071426in}}%
\pgfpathlineto{\pgfqpoint{2.874649in}{2.078782in}}%
\pgfpathlineto{\pgfqpoint{2.861120in}{2.086242in}}%
\pgfpathlineto{\pgfqpoint{2.852729in}{2.078873in}}%
\pgfpathlineto{\pgfqpoint{2.844329in}{2.071577in}}%
\pgfpathlineto{\pgfqpoint{2.835921in}{2.064354in}}%
\pgfpathlineto{\pgfqpoint{2.827505in}{2.057206in}}%
\pgfpathclose%
\pgfusepath{fill}%
\end{pgfscope}%
\begin{pgfscope}%
\pgfpathrectangle{\pgfqpoint{1.150000in}{0.150000in}}{\pgfqpoint{5.700000in}{5.700000in}}%
\pgfusepath{clip}%
\pgfsetbuttcap%
\pgfsetroundjoin%
\definecolor{currentfill}{rgb}{0.278791,0.062145,0.386592}%
\pgfsetfillcolor{currentfill}%
\pgfsetfillopacity{0.700000}%
\pgfsetlinewidth{0.000000pt}%
\definecolor{currentstroke}{rgb}{0.000000,0.000000,0.000000}%
\pgfsetstrokecolor{currentstroke}%
\pgfsetdash{}{0pt}%
\pgfpathmoveto{\pgfqpoint{3.709393in}{2.052073in}}%
\pgfpathlineto{\pgfqpoint{3.723043in}{2.049449in}}%
\pgfpathlineto{\pgfqpoint{3.736698in}{2.046906in}}%
\pgfpathlineto{\pgfqpoint{3.750361in}{2.044444in}}%
\pgfpathlineto{\pgfqpoint{3.764030in}{2.042064in}}%
\pgfpathlineto{\pgfqpoint{3.772078in}{2.050996in}}%
\pgfpathlineto{\pgfqpoint{3.780121in}{2.059916in}}%
\pgfpathlineto{\pgfqpoint{3.788158in}{2.068824in}}%
\pgfpathlineto{\pgfqpoint{3.796190in}{2.077721in}}%
\pgfpathlineto{\pgfqpoint{3.782532in}{2.080101in}}%
\pgfpathlineto{\pgfqpoint{3.768881in}{2.082562in}}%
\pgfpathlineto{\pgfqpoint{3.755236in}{2.085104in}}%
\pgfpathlineto{\pgfqpoint{3.741598in}{2.087727in}}%
\pgfpathlineto{\pgfqpoint{3.733555in}{2.078824in}}%
\pgfpathlineto{\pgfqpoint{3.725507in}{2.069914in}}%
\pgfpathlineto{\pgfqpoint{3.717453in}{2.060997in}}%
\pgfpathlineto{\pgfqpoint{3.709393in}{2.052073in}}%
\pgfpathclose%
\pgfusepath{fill}%
\end{pgfscope}%
\begin{pgfscope}%
\pgfpathrectangle{\pgfqpoint{1.150000in}{0.150000in}}{\pgfqpoint{5.700000in}{5.700000in}}%
\pgfusepath{clip}%
\pgfsetbuttcap%
\pgfsetroundjoin%
\definecolor{currentfill}{rgb}{0.273809,0.031497,0.358853}%
\pgfsetfillcolor{currentfill}%
\pgfsetfillopacity{0.700000}%
\pgfsetlinewidth{0.000000pt}%
\definecolor{currentstroke}{rgb}{0.000000,0.000000,0.000000}%
\pgfsetstrokecolor{currentstroke}%
\pgfsetdash{}{0pt}%
\pgfpathmoveto{\pgfqpoint{3.252608in}{2.001570in}}%
\pgfpathlineto{\pgfqpoint{3.266179in}{1.996675in}}%
\pgfpathlineto{\pgfqpoint{3.279754in}{1.991871in}}%
\pgfpathlineto{\pgfqpoint{3.293334in}{1.987157in}}%
\pgfpathlineto{\pgfqpoint{3.306918in}{1.982533in}}%
\pgfpathlineto{\pgfqpoint{3.315134in}{1.991115in}}%
\pgfpathlineto{\pgfqpoint{3.323344in}{1.999719in}}%
\pgfpathlineto{\pgfqpoint{3.331547in}{2.008345in}}%
\pgfpathlineto{\pgfqpoint{3.339744in}{2.016992in}}%
\pgfpathlineto{\pgfqpoint{3.326174in}{2.021533in}}%
\pgfpathlineto{\pgfqpoint{3.312607in}{2.026165in}}%
\pgfpathlineto{\pgfqpoint{3.299046in}{2.030886in}}%
\pgfpathlineto{\pgfqpoint{3.285488in}{2.035698in}}%
\pgfpathlineto{\pgfqpoint{3.277278in}{2.027127in}}%
\pgfpathlineto{\pgfqpoint{3.269061in}{2.018581in}}%
\pgfpathlineto{\pgfqpoint{3.260837in}{2.010061in}}%
\pgfpathlineto{\pgfqpoint{3.252608in}{2.001570in}}%
\pgfpathclose%
\pgfusepath{fill}%
\end{pgfscope}%
\begin{pgfscope}%
\pgfpathrectangle{\pgfqpoint{1.150000in}{0.150000in}}{\pgfqpoint{5.700000in}{5.700000in}}%
\pgfusepath{clip}%
\pgfsetbuttcap%
\pgfsetroundjoin%
\definecolor{currentfill}{rgb}{0.235526,0.309527,0.542944}%
\pgfsetfillcolor{currentfill}%
\pgfsetfillopacity{0.700000}%
\pgfsetlinewidth{0.000000pt}%
\definecolor{currentstroke}{rgb}{0.000000,0.000000,0.000000}%
\pgfsetstrokecolor{currentstroke}%
\pgfsetdash{}{0pt}%
\pgfpathmoveto{\pgfqpoint{5.544574in}{2.573391in}}%
\pgfpathlineto{\pgfqpoint{5.558784in}{2.574306in}}%
\pgfpathlineto{\pgfqpoint{5.573005in}{2.575287in}}%
\pgfpathlineto{\pgfqpoint{5.587237in}{2.576336in}}%
\pgfpathlineto{\pgfqpoint{5.601480in}{2.577451in}}%
\pgfpathlineto{\pgfqpoint{5.608796in}{2.583333in}}%
\pgfpathlineto{\pgfqpoint{5.616110in}{2.589335in}}%
\pgfpathlineto{\pgfqpoint{5.623421in}{2.595462in}}%
\pgfpathlineto{\pgfqpoint{5.630731in}{2.601721in}}%
\pgfpathlineto{\pgfqpoint{5.616513in}{2.600978in}}%
\pgfpathlineto{\pgfqpoint{5.602305in}{2.600302in}}%
\pgfpathlineto{\pgfqpoint{5.588109in}{2.599692in}}%
\pgfpathlineto{\pgfqpoint{5.573923in}{2.599150in}}%
\pgfpathlineto{\pgfqpoint{5.566589in}{2.592512in}}%
\pgfpathlineto{\pgfqpoint{5.559253in}{2.586010in}}%
\pgfpathlineto{\pgfqpoint{5.551915in}{2.579639in}}%
\pgfpathlineto{\pgfqpoint{5.544574in}{2.573391in}}%
\pgfpathclose%
\pgfusepath{fill}%
\end{pgfscope}%
\begin{pgfscope}%
\pgfpathrectangle{\pgfqpoint{1.150000in}{0.150000in}}{\pgfqpoint{5.700000in}{5.700000in}}%
\pgfusepath{clip}%
\pgfsetbuttcap%
\pgfsetroundjoin%
\definecolor{currentfill}{rgb}{0.255645,0.260703,0.528312}%
\pgfsetfillcolor{currentfill}%
\pgfsetfillopacity{0.700000}%
\pgfsetlinewidth{0.000000pt}%
\definecolor{currentstroke}{rgb}{0.000000,0.000000,0.000000}%
\pgfsetstrokecolor{currentstroke}%
\pgfsetdash{}{0pt}%
\pgfpathmoveto{\pgfqpoint{5.142949in}{2.455484in}}%
\pgfpathlineto{\pgfqpoint{5.157032in}{2.456322in}}%
\pgfpathlineto{\pgfqpoint{5.171125in}{2.457229in}}%
\pgfpathlineto{\pgfqpoint{5.185229in}{2.458205in}}%
\pgfpathlineto{\pgfqpoint{5.199343in}{2.459249in}}%
\pgfpathlineto{\pgfqpoint{5.206830in}{2.465471in}}%
\pgfpathlineto{\pgfqpoint{5.214313in}{2.471744in}}%
\pgfpathlineto{\pgfqpoint{5.221791in}{2.478073in}}%
\pgfpathlineto{\pgfqpoint{5.229264in}{2.484462in}}%
\pgfpathlineto{\pgfqpoint{5.215169in}{2.483708in}}%
\pgfpathlineto{\pgfqpoint{5.201085in}{2.483022in}}%
\pgfpathlineto{\pgfqpoint{5.187011in}{2.482405in}}%
\pgfpathlineto{\pgfqpoint{5.172947in}{2.481856in}}%
\pgfpathlineto{\pgfqpoint{5.165454in}{2.475170in}}%
\pgfpathlineto{\pgfqpoint{5.157957in}{2.468549in}}%
\pgfpathlineto{\pgfqpoint{5.150456in}{2.461989in}}%
\pgfpathlineto{\pgfqpoint{5.142949in}{2.455484in}}%
\pgfpathclose%
\pgfusepath{fill}%
\end{pgfscope}%
\begin{pgfscope}%
\pgfpathrectangle{\pgfqpoint{1.150000in}{0.150000in}}{\pgfqpoint{5.700000in}{5.700000in}}%
\pgfusepath{clip}%
\pgfsetbuttcap%
\pgfsetroundjoin%
\definecolor{currentfill}{rgb}{0.273006,0.204520,0.501721}%
\pgfsetfillcolor{currentfill}%
\pgfsetfillopacity{0.700000}%
\pgfsetlinewidth{0.000000pt}%
\definecolor{currentstroke}{rgb}{0.000000,0.000000,0.000000}%
\pgfsetstrokecolor{currentstroke}%
\pgfsetdash{}{0pt}%
\pgfpathmoveto{\pgfqpoint{4.741210in}{2.333901in}}%
\pgfpathlineto{\pgfqpoint{4.755160in}{2.334290in}}%
\pgfpathlineto{\pgfqpoint{4.769121in}{2.334750in}}%
\pgfpathlineto{\pgfqpoint{4.783090in}{2.335282in}}%
\pgfpathlineto{\pgfqpoint{4.797070in}{2.335884in}}%
\pgfpathlineto{\pgfqpoint{4.804730in}{2.342933in}}%
\pgfpathlineto{\pgfqpoint{4.812384in}{2.349986in}}%
\pgfpathlineto{\pgfqpoint{4.820033in}{2.357049in}}%
\pgfpathlineto{\pgfqpoint{4.827676in}{2.364123in}}%
\pgfpathlineto{\pgfqpoint{4.813712in}{2.363728in}}%
\pgfpathlineto{\pgfqpoint{4.799757in}{2.363403in}}%
\pgfpathlineto{\pgfqpoint{4.785812in}{2.363150in}}%
\pgfpathlineto{\pgfqpoint{4.771876in}{2.362967in}}%
\pgfpathlineto{\pgfqpoint{4.764218in}{2.355679in}}%
\pgfpathlineto{\pgfqpoint{4.756554in}{2.348407in}}%
\pgfpathlineto{\pgfqpoint{4.748885in}{2.341150in}}%
\pgfpathlineto{\pgfqpoint{4.741210in}{2.333901in}}%
\pgfpathclose%
\pgfusepath{fill}%
\end{pgfscope}%
\begin{pgfscope}%
\pgfpathrectangle{\pgfqpoint{1.150000in}{0.150000in}}{\pgfqpoint{5.700000in}{5.700000in}}%
\pgfusepath{clip}%
\pgfsetbuttcap%
\pgfsetroundjoin%
\definecolor{currentfill}{rgb}{0.282290,0.145912,0.461510}%
\pgfsetfillcolor{currentfill}%
\pgfsetfillopacity{0.700000}%
\pgfsetlinewidth{0.000000pt}%
\definecolor{currentstroke}{rgb}{0.000000,0.000000,0.000000}%
\pgfsetstrokecolor{currentstroke}%
\pgfsetdash{}{0pt}%
\pgfpathmoveto{\pgfqpoint{4.339454in}{2.211459in}}%
\pgfpathlineto{\pgfqpoint{4.353276in}{2.211013in}}%
\pgfpathlineto{\pgfqpoint{4.367107in}{2.210641in}}%
\pgfpathlineto{\pgfqpoint{4.380946in}{2.210344in}}%
\pgfpathlineto{\pgfqpoint{4.394794in}{2.210120in}}%
\pgfpathlineto{\pgfqpoint{4.402616in}{2.218128in}}%
\pgfpathlineto{\pgfqpoint{4.410432in}{2.226117in}}%
\pgfpathlineto{\pgfqpoint{4.418242in}{2.234089in}}%
\pgfpathlineto{\pgfqpoint{4.426046in}{2.242047in}}%
\pgfpathlineto{\pgfqpoint{4.412211in}{2.242394in}}%
\pgfpathlineto{\pgfqpoint{4.398383in}{2.242815in}}%
\pgfpathlineto{\pgfqpoint{4.384565in}{2.243310in}}%
\pgfpathlineto{\pgfqpoint{4.370754in}{2.243879in}}%
\pgfpathlineto{\pgfqpoint{4.362938in}{2.235791in}}%
\pgfpathlineto{\pgfqpoint{4.355116in}{2.227693in}}%
\pgfpathlineto{\pgfqpoint{4.347288in}{2.219583in}}%
\pgfpathlineto{\pgfqpoint{4.339454in}{2.211459in}}%
\pgfpathclose%
\pgfusepath{fill}%
\end{pgfscope}%
\begin{pgfscope}%
\pgfpathrectangle{\pgfqpoint{1.150000in}{0.150000in}}{\pgfqpoint{5.700000in}{5.700000in}}%
\pgfusepath{clip}%
\pgfsetbuttcap%
\pgfsetroundjoin%
\definecolor{currentfill}{rgb}{0.218130,0.347432,0.550038}%
\pgfsetfillcolor{currentfill}%
\pgfsetfillopacity{0.700000}%
\pgfsetlinewidth{0.000000pt}%
\definecolor{currentstroke}{rgb}{0.000000,0.000000,0.000000}%
\pgfsetstrokecolor{currentstroke}%
\pgfsetdash{}{0pt}%
\pgfpathmoveto{\pgfqpoint{5.860022in}{2.661810in}}%
\pgfpathlineto{\pgfqpoint{5.874330in}{2.662609in}}%
\pgfpathlineto{\pgfqpoint{5.888650in}{2.663475in}}%
\pgfpathlineto{\pgfqpoint{5.902982in}{2.664406in}}%
\pgfpathlineto{\pgfqpoint{5.917324in}{2.665404in}}%
\pgfpathlineto{\pgfqpoint{5.924520in}{2.671462in}}%
\pgfpathlineto{\pgfqpoint{5.931717in}{2.677704in}}%
\pgfpathlineto{\pgfqpoint{5.938915in}{2.684138in}}%
\pgfpathlineto{\pgfqpoint{5.946114in}{2.690770in}}%
\pgfpathlineto{\pgfqpoint{5.931800in}{2.690207in}}%
\pgfpathlineto{\pgfqpoint{5.917498in}{2.689709in}}%
\pgfpathlineto{\pgfqpoint{5.903207in}{2.689277in}}%
\pgfpathlineto{\pgfqpoint{5.888926in}{2.688910in}}%
\pgfpathlineto{\pgfqpoint{5.881698in}{2.681837in}}%
\pgfpathlineto{\pgfqpoint{5.874472in}{2.674968in}}%
\pgfpathlineto{\pgfqpoint{5.867246in}{2.668295in}}%
\pgfpathlineto{\pgfqpoint{5.860022in}{2.661810in}}%
\pgfpathclose%
\pgfusepath{fill}%
\end{pgfscope}%
\begin{pgfscope}%
\pgfpathrectangle{\pgfqpoint{1.150000in}{0.150000in}}{\pgfqpoint{5.700000in}{5.700000in}}%
\pgfusepath{clip}%
\pgfsetbuttcap%
\pgfsetroundjoin%
\definecolor{currentfill}{rgb}{0.281924,0.089666,0.412415}%
\pgfsetfillcolor{currentfill}%
\pgfsetfillopacity{0.700000}%
\pgfsetlinewidth{0.000000pt}%
\definecolor{currentstroke}{rgb}{0.000000,0.000000,0.000000}%
\pgfsetstrokecolor{currentstroke}%
\pgfsetdash{}{0pt}%
\pgfpathmoveto{\pgfqpoint{3.937679in}{2.096889in}}%
\pgfpathlineto{\pgfqpoint{3.951388in}{2.095205in}}%
\pgfpathlineto{\pgfqpoint{3.965104in}{2.093600in}}%
\pgfpathlineto{\pgfqpoint{3.978828in}{2.092072in}}%
\pgfpathlineto{\pgfqpoint{3.992559in}{2.090623in}}%
\pgfpathlineto{\pgfqpoint{4.000529in}{2.099375in}}%
\pgfpathlineto{\pgfqpoint{4.008494in}{2.108106in}}%
\pgfpathlineto{\pgfqpoint{4.016454in}{2.116815in}}%
\pgfpathlineto{\pgfqpoint{4.024407in}{2.125506in}}%
\pgfpathlineto{\pgfqpoint{4.010687in}{2.126996in}}%
\pgfpathlineto{\pgfqpoint{3.996974in}{2.128564in}}%
\pgfpathlineto{\pgfqpoint{3.983269in}{2.130210in}}%
\pgfpathlineto{\pgfqpoint{3.969571in}{2.131934in}}%
\pgfpathlineto{\pgfqpoint{3.961606in}{2.123195in}}%
\pgfpathlineto{\pgfqpoint{3.953636in}{2.114442in}}%
\pgfpathlineto{\pgfqpoint{3.945660in}{2.105674in}}%
\pgfpathlineto{\pgfqpoint{3.937679in}{2.096889in}}%
\pgfpathclose%
\pgfusepath{fill}%
\end{pgfscope}%
\begin{pgfscope}%
\pgfpathrectangle{\pgfqpoint{1.150000in}{0.150000in}}{\pgfqpoint{5.700000in}{5.700000in}}%
\pgfusepath{clip}%
\pgfsetbuttcap%
\pgfsetroundjoin%
\definecolor{currentfill}{rgb}{0.280894,0.078907,0.402329}%
\pgfsetfillcolor{currentfill}%
\pgfsetfillopacity{0.700000}%
\pgfsetlinewidth{0.000000pt}%
\definecolor{currentstroke}{rgb}{0.000000,0.000000,0.000000}%
\pgfsetstrokecolor{currentstroke}%
\pgfsetdash{}{0pt}%
\pgfpathmoveto{\pgfqpoint{2.685264in}{2.095439in}}%
\pgfpathlineto{\pgfqpoint{2.698822in}{2.086804in}}%
\pgfpathlineto{\pgfqpoint{2.712380in}{2.078279in}}%
\pgfpathlineto{\pgfqpoint{2.725940in}{2.069865in}}%
\pgfpathlineto{\pgfqpoint{2.739500in}{2.061560in}}%
\pgfpathlineto{\pgfqpoint{2.747971in}{2.068210in}}%
\pgfpathlineto{\pgfqpoint{2.756434in}{2.074949in}}%
\pgfpathlineto{\pgfqpoint{2.764887in}{2.081776in}}%
\pgfpathlineto{\pgfqpoint{2.773331in}{2.088687in}}%
\pgfpathlineto{\pgfqpoint{2.759791in}{2.096826in}}%
\pgfpathlineto{\pgfqpoint{2.746253in}{2.105074in}}%
\pgfpathlineto{\pgfqpoint{2.732715in}{2.113432in}}%
\pgfpathlineto{\pgfqpoint{2.719178in}{2.121901in}}%
\pgfpathlineto{\pgfqpoint{2.710714in}{2.115148in}}%
\pgfpathlineto{\pgfqpoint{2.702240in}{2.108485in}}%
\pgfpathlineto{\pgfqpoint{2.693757in}{2.101915in}}%
\pgfpathlineto{\pgfqpoint{2.685264in}{2.095439in}}%
\pgfpathclose%
\pgfusepath{fill}%
\end{pgfscope}%
\begin{pgfscope}%
\pgfpathrectangle{\pgfqpoint{1.150000in}{0.150000in}}{\pgfqpoint{5.700000in}{5.700000in}}%
\pgfusepath{clip}%
\pgfsetbuttcap%
\pgfsetroundjoin%
\definecolor{currentfill}{rgb}{0.273809,0.031497,0.358853}%
\pgfsetfillcolor{currentfill}%
\pgfsetfillopacity{0.700000}%
\pgfsetlinewidth{0.000000pt}%
\definecolor{currentstroke}{rgb}{0.000000,0.000000,0.000000}%
\pgfsetstrokecolor{currentstroke}%
\pgfsetdash{}{0pt}%
\pgfpathmoveto{\pgfqpoint{3.394074in}{1.999714in}}%
\pgfpathlineto{\pgfqpoint{3.407668in}{1.995614in}}%
\pgfpathlineto{\pgfqpoint{3.421268in}{1.991602in}}%
\pgfpathlineto{\pgfqpoint{3.434872in}{1.987677in}}%
\pgfpathlineto{\pgfqpoint{3.448482in}{1.983837in}}%
\pgfpathlineto{\pgfqpoint{3.456647in}{1.992644in}}%
\pgfpathlineto{\pgfqpoint{3.464806in}{2.001459in}}%
\pgfpathlineto{\pgfqpoint{3.472959in}{2.010283in}}%
\pgfpathlineto{\pgfqpoint{3.481106in}{2.019116in}}%
\pgfpathlineto{\pgfqpoint{3.467509in}{2.022893in}}%
\pgfpathlineto{\pgfqpoint{3.453917in}{2.026757in}}%
\pgfpathlineto{\pgfqpoint{3.440330in}{2.030707in}}%
\pgfpathlineto{\pgfqpoint{3.426748in}{2.034744in}}%
\pgfpathlineto{\pgfqpoint{3.418589in}{2.025966in}}%
\pgfpathlineto{\pgfqpoint{3.410423in}{2.017201in}}%
\pgfpathlineto{\pgfqpoint{3.402252in}{2.008450in}}%
\pgfpathlineto{\pgfqpoint{3.394074in}{1.999714in}}%
\pgfpathclose%
\pgfusepath{fill}%
\end{pgfscope}%
\begin{pgfscope}%
\pgfpathrectangle{\pgfqpoint{1.150000in}{0.150000in}}{\pgfqpoint{5.700000in}{5.700000in}}%
\pgfusepath{clip}%
\pgfsetbuttcap%
\pgfsetroundjoin%
\definecolor{currentfill}{rgb}{0.277018,0.050344,0.375715}%
\pgfsetfillcolor{currentfill}%
\pgfsetfillopacity{0.700000}%
\pgfsetlinewidth{0.000000pt}%
\definecolor{currentstroke}{rgb}{0.000000,0.000000,0.000000}%
\pgfsetstrokecolor{currentstroke}%
\pgfsetdash{}{0pt}%
\pgfpathmoveto{\pgfqpoint{3.622516in}{2.027702in}}%
\pgfpathlineto{\pgfqpoint{3.636152in}{2.024727in}}%
\pgfpathlineto{\pgfqpoint{3.649794in}{2.021835in}}%
\pgfpathlineto{\pgfqpoint{3.663443in}{2.019026in}}%
\pgfpathlineto{\pgfqpoint{3.677097in}{2.016299in}}%
\pgfpathlineto{\pgfqpoint{3.685180in}{2.025254in}}%
\pgfpathlineto{\pgfqpoint{3.693257in}{2.034202in}}%
\pgfpathlineto{\pgfqpoint{3.701328in}{2.043141in}}%
\pgfpathlineto{\pgfqpoint{3.709393in}{2.052073in}}%
\pgfpathlineto{\pgfqpoint{3.695750in}{2.054779in}}%
\pgfpathlineto{\pgfqpoint{3.682113in}{2.057567in}}%
\pgfpathlineto{\pgfqpoint{3.668482in}{2.060438in}}%
\pgfpathlineto{\pgfqpoint{3.654857in}{2.063392in}}%
\pgfpathlineto{\pgfqpoint{3.646781in}{2.054474in}}%
\pgfpathlineto{\pgfqpoint{3.638698in}{2.045553in}}%
\pgfpathlineto{\pgfqpoint{3.630610in}{2.036629in}}%
\pgfpathlineto{\pgfqpoint{3.622516in}{2.027702in}}%
\pgfpathclose%
\pgfusepath{fill}%
\end{pgfscope}%
\begin{pgfscope}%
\pgfpathrectangle{\pgfqpoint{1.150000in}{0.150000in}}{\pgfqpoint{5.700000in}{5.700000in}}%
\pgfusepath{clip}%
\pgfsetbuttcap%
\pgfsetroundjoin%
\definecolor{currentfill}{rgb}{0.239346,0.300855,0.540844}%
\pgfsetfillcolor{currentfill}%
\pgfsetfillopacity{0.700000}%
\pgfsetlinewidth{0.000000pt}%
\definecolor{currentstroke}{rgb}{0.000000,0.000000,0.000000}%
\pgfsetstrokecolor{currentstroke}%
\pgfsetdash{}{0pt}%
\pgfpathmoveto{\pgfqpoint{5.458361in}{2.545119in}}%
\pgfpathlineto{\pgfqpoint{5.472551in}{2.546116in}}%
\pgfpathlineto{\pgfqpoint{5.486751in}{2.547181in}}%
\pgfpathlineto{\pgfqpoint{5.500963in}{2.548313in}}%
\pgfpathlineto{\pgfqpoint{5.515186in}{2.549512in}}%
\pgfpathlineto{\pgfqpoint{5.522537in}{2.555328in}}%
\pgfpathlineto{\pgfqpoint{5.529886in}{2.561242in}}%
\pgfpathlineto{\pgfqpoint{5.537232in}{2.567261in}}%
\pgfpathlineto{\pgfqpoint{5.544574in}{2.573391in}}%
\pgfpathlineto{\pgfqpoint{5.530375in}{2.572544in}}%
\pgfpathlineto{\pgfqpoint{5.516187in}{2.571764in}}%
\pgfpathlineto{\pgfqpoint{5.502009in}{2.571052in}}%
\pgfpathlineto{\pgfqpoint{5.487842in}{2.570406in}}%
\pgfpathlineto{\pgfqpoint{5.480476in}{2.563917in}}%
\pgfpathlineto{\pgfqpoint{5.473107in}{2.557543in}}%
\pgfpathlineto{\pgfqpoint{5.465735in}{2.551280in}}%
\pgfpathlineto{\pgfqpoint{5.458361in}{2.545119in}}%
\pgfpathclose%
\pgfusepath{fill}%
\end{pgfscope}%
\begin{pgfscope}%
\pgfpathrectangle{\pgfqpoint{1.150000in}{0.150000in}}{\pgfqpoint{5.700000in}{5.700000in}}%
\pgfusepath{clip}%
\pgfsetbuttcap%
\pgfsetroundjoin%
\definecolor{currentfill}{rgb}{0.283187,0.125848,0.444960}%
\pgfsetfillcolor{currentfill}%
\pgfsetfillopacity{0.700000}%
\pgfsetlinewidth{0.000000pt}%
\definecolor{currentstroke}{rgb}{0.000000,0.000000,0.000000}%
\pgfsetstrokecolor{currentstroke}%
\pgfsetdash{}{0pt}%
\pgfpathmoveto{\pgfqpoint{2.488254in}{2.185358in}}%
\pgfpathlineto{\pgfqpoint{2.501834in}{2.175137in}}%
\pgfpathlineto{\pgfqpoint{2.515414in}{2.165038in}}%
\pgfpathlineto{\pgfqpoint{2.528993in}{2.155059in}}%
\pgfpathlineto{\pgfqpoint{2.542572in}{2.145201in}}%
\pgfpathlineto{\pgfqpoint{2.551150in}{2.150883in}}%
\pgfpathlineto{\pgfqpoint{2.559717in}{2.156680in}}%
\pgfpathlineto{\pgfqpoint{2.568273in}{2.162589in}}%
\pgfpathlineto{\pgfqpoint{2.576820in}{2.168608in}}%
\pgfpathlineto{\pgfqpoint{2.563264in}{2.178279in}}%
\pgfpathlineto{\pgfqpoint{2.549708in}{2.188068in}}%
\pgfpathlineto{\pgfqpoint{2.536152in}{2.197979in}}%
\pgfpathlineto{\pgfqpoint{2.522595in}{2.208011in}}%
\pgfpathlineto{\pgfqpoint{2.514026in}{2.202173in}}%
\pgfpathlineto{\pgfqpoint{2.505446in}{2.196450in}}%
\pgfpathlineto{\pgfqpoint{2.496855in}{2.190844in}}%
\pgfpathlineto{\pgfqpoint{2.488254in}{2.185358in}}%
\pgfpathclose%
\pgfusepath{fill}%
\end{pgfscope}%
\begin{pgfscope}%
\pgfpathrectangle{\pgfqpoint{1.150000in}{0.150000in}}{\pgfqpoint{5.700000in}{5.700000in}}%
\pgfusepath{clip}%
\pgfsetbuttcap%
\pgfsetroundjoin%
\definecolor{currentfill}{rgb}{0.282884,0.135920,0.453427}%
\pgfsetfillcolor{currentfill}%
\pgfsetfillopacity{0.700000}%
\pgfsetlinewidth{0.000000pt}%
\definecolor{currentstroke}{rgb}{0.000000,0.000000,0.000000}%
\pgfsetstrokecolor{currentstroke}%
\pgfsetdash{}{0pt}%
\pgfpathmoveto{\pgfqpoint{4.252810in}{2.180890in}}%
\pgfpathlineto{\pgfqpoint{4.266610in}{2.180249in}}%
\pgfpathlineto{\pgfqpoint{4.280419in}{2.179682in}}%
\pgfpathlineto{\pgfqpoint{4.294236in}{2.179190in}}%
\pgfpathlineto{\pgfqpoint{4.308061in}{2.178773in}}%
\pgfpathlineto{\pgfqpoint{4.315918in}{2.186977in}}%
\pgfpathlineto{\pgfqpoint{4.323769in}{2.195158in}}%
\pgfpathlineto{\pgfqpoint{4.331615in}{2.203318in}}%
\pgfpathlineto{\pgfqpoint{4.339454in}{2.211459in}}%
\pgfpathlineto{\pgfqpoint{4.325641in}{2.211979in}}%
\pgfpathlineto{\pgfqpoint{4.311835in}{2.212574in}}%
\pgfpathlineto{\pgfqpoint{4.298038in}{2.213243in}}%
\pgfpathlineto{\pgfqpoint{4.284250in}{2.213987in}}%
\pgfpathlineto{\pgfqpoint{4.276398in}{2.205736in}}%
\pgfpathlineto{\pgfqpoint{4.268541in}{2.197471in}}%
\pgfpathlineto{\pgfqpoint{4.260678in}{2.189190in}}%
\pgfpathlineto{\pgfqpoint{4.252810in}{2.180890in}}%
\pgfpathclose%
\pgfusepath{fill}%
\end{pgfscope}%
\begin{pgfscope}%
\pgfpathrectangle{\pgfqpoint{1.150000in}{0.150000in}}{\pgfqpoint{5.700000in}{5.700000in}}%
\pgfusepath{clip}%
\pgfsetbuttcap%
\pgfsetroundjoin%
\definecolor{currentfill}{rgb}{0.258965,0.251537,0.524736}%
\pgfsetfillcolor{currentfill}%
\pgfsetfillopacity{0.700000}%
\pgfsetlinewidth{0.000000pt}%
\definecolor{currentstroke}{rgb}{0.000000,0.000000,0.000000}%
\pgfsetstrokecolor{currentstroke}%
\pgfsetdash{}{0pt}%
\pgfpathmoveto{\pgfqpoint{5.056572in}{2.426186in}}%
\pgfpathlineto{\pgfqpoint{5.070632in}{2.427016in}}%
\pgfpathlineto{\pgfqpoint{5.084702in}{2.427916in}}%
\pgfpathlineto{\pgfqpoint{5.098783in}{2.428885in}}%
\pgfpathlineto{\pgfqpoint{5.112874in}{2.429924in}}%
\pgfpathlineto{\pgfqpoint{5.120400in}{2.436255in}}%
\pgfpathlineto{\pgfqpoint{5.127922in}{2.442623in}}%
\pgfpathlineto{\pgfqpoint{5.135438in}{2.449031in}}%
\pgfpathlineto{\pgfqpoint{5.142949in}{2.455484in}}%
\pgfpathlineto{\pgfqpoint{5.128877in}{2.454715in}}%
\pgfpathlineto{\pgfqpoint{5.114814in}{2.454016in}}%
\pgfpathlineto{\pgfqpoint{5.100762in}{2.453385in}}%
\pgfpathlineto{\pgfqpoint{5.086720in}{2.452823in}}%
\pgfpathlineto{\pgfqpoint{5.079190in}{2.446093in}}%
\pgfpathlineto{\pgfqpoint{5.071656in}{2.439413in}}%
\pgfpathlineto{\pgfqpoint{5.064116in}{2.432779in}}%
\pgfpathlineto{\pgfqpoint{5.056572in}{2.426186in}}%
\pgfpathclose%
\pgfusepath{fill}%
\end{pgfscope}%
\begin{pgfscope}%
\pgfpathrectangle{\pgfqpoint{1.150000in}{0.150000in}}{\pgfqpoint{5.700000in}{5.700000in}}%
\pgfusepath{clip}%
\pgfsetbuttcap%
\pgfsetroundjoin%
\definecolor{currentfill}{rgb}{0.275191,0.194905,0.496005}%
\pgfsetfillcolor{currentfill}%
\pgfsetfillopacity{0.700000}%
\pgfsetlinewidth{0.000000pt}%
\definecolor{currentstroke}{rgb}{0.000000,0.000000,0.000000}%
\pgfsetstrokecolor{currentstroke}%
\pgfsetdash{}{0pt}%
\pgfpathmoveto{\pgfqpoint{4.654687in}{2.303348in}}%
\pgfpathlineto{\pgfqpoint{4.668614in}{2.303637in}}%
\pgfpathlineto{\pgfqpoint{4.682550in}{2.303998in}}%
\pgfpathlineto{\pgfqpoint{4.696496in}{2.304430in}}%
\pgfpathlineto{\pgfqpoint{4.710451in}{2.304934in}}%
\pgfpathlineto{\pgfqpoint{4.718150in}{2.312179in}}%
\pgfpathlineto{\pgfqpoint{4.725842in}{2.319420in}}%
\pgfpathlineto{\pgfqpoint{4.733529in}{2.326659in}}%
\pgfpathlineto{\pgfqpoint{4.741210in}{2.333901in}}%
\pgfpathlineto{\pgfqpoint{4.727269in}{2.333584in}}%
\pgfpathlineto{\pgfqpoint{4.713337in}{2.333338in}}%
\pgfpathlineto{\pgfqpoint{4.699415in}{2.333163in}}%
\pgfpathlineto{\pgfqpoint{4.685502in}{2.333060in}}%
\pgfpathlineto{\pgfqpoint{4.677807in}{2.325624in}}%
\pgfpathlineto{\pgfqpoint{4.670106in}{2.318196in}}%
\pgfpathlineto{\pgfqpoint{4.662399in}{2.310772in}}%
\pgfpathlineto{\pgfqpoint{4.654687in}{2.303348in}}%
\pgfpathclose%
\pgfusepath{fill}%
\end{pgfscope}%
\begin{pgfscope}%
\pgfpathrectangle{\pgfqpoint{1.150000in}{0.150000in}}{\pgfqpoint{5.700000in}{5.700000in}}%
\pgfusepath{clip}%
\pgfsetbuttcap%
\pgfsetroundjoin%
\definecolor{currentfill}{rgb}{0.273809,0.031497,0.358853}%
\pgfsetfillcolor{currentfill}%
\pgfsetfillopacity{0.700000}%
\pgfsetlinewidth{0.000000pt}%
\definecolor{currentstroke}{rgb}{0.000000,0.000000,0.000000}%
\pgfsetstrokecolor{currentstroke}%
\pgfsetdash{}{0pt}%
\pgfpathmoveto{\pgfqpoint{3.023613in}{2.003475in}}%
\pgfpathlineto{\pgfqpoint{3.037171in}{1.997226in}}%
\pgfpathlineto{\pgfqpoint{3.050732in}{1.991074in}}%
\pgfpathlineto{\pgfqpoint{3.064296in}{1.985019in}}%
\pgfpathlineto{\pgfqpoint{3.077863in}{1.979059in}}%
\pgfpathlineto{\pgfqpoint{3.086178in}{1.987025in}}%
\pgfpathlineto{\pgfqpoint{3.094486in}{1.995039in}}%
\pgfpathlineto{\pgfqpoint{3.102786in}{2.003100in}}%
\pgfpathlineto{\pgfqpoint{3.111080in}{2.011205in}}%
\pgfpathlineto{\pgfqpoint{3.097528in}{2.017041in}}%
\pgfpathlineto{\pgfqpoint{3.083980in}{2.022973in}}%
\pgfpathlineto{\pgfqpoint{3.070435in}{2.029001in}}%
\pgfpathlineto{\pgfqpoint{3.056893in}{2.035126in}}%
\pgfpathlineto{\pgfqpoint{3.048584in}{2.027137in}}%
\pgfpathlineto{\pgfqpoint{3.040268in}{2.019197in}}%
\pgfpathlineto{\pgfqpoint{3.031944in}{2.011310in}}%
\pgfpathlineto{\pgfqpoint{3.023613in}{2.003475in}}%
\pgfpathclose%
\pgfusepath{fill}%
\end{pgfscope}%
\begin{pgfscope}%
\pgfpathrectangle{\pgfqpoint{1.150000in}{0.150000in}}{\pgfqpoint{5.700000in}{5.700000in}}%
\pgfusepath{clip}%
\pgfsetbuttcap%
\pgfsetroundjoin%
\definecolor{currentfill}{rgb}{0.221989,0.339161,0.548752}%
\pgfsetfillcolor{currentfill}%
\pgfsetfillopacity{0.700000}%
\pgfsetlinewidth{0.000000pt}%
\definecolor{currentstroke}{rgb}{0.000000,0.000000,0.000000}%
\pgfsetstrokecolor{currentstroke}%
\pgfsetdash{}{0pt}%
\pgfpathmoveto{\pgfqpoint{5.773892in}{2.633414in}}%
\pgfpathlineto{\pgfqpoint{5.788183in}{2.634362in}}%
\pgfpathlineto{\pgfqpoint{5.802486in}{2.635377in}}%
\pgfpathlineto{\pgfqpoint{5.816800in}{2.636459in}}%
\pgfpathlineto{\pgfqpoint{5.831125in}{2.637607in}}%
\pgfpathlineto{\pgfqpoint{5.838350in}{2.643411in}}%
\pgfpathlineto{\pgfqpoint{5.845574in}{2.649375in}}%
\pgfpathlineto{\pgfqpoint{5.852798in}{2.655506in}}%
\pgfpathlineto{\pgfqpoint{5.860022in}{2.661810in}}%
\pgfpathlineto{\pgfqpoint{5.845724in}{2.661076in}}%
\pgfpathlineto{\pgfqpoint{5.831438in}{2.660409in}}%
\pgfpathlineto{\pgfqpoint{5.817163in}{2.659807in}}%
\pgfpathlineto{\pgfqpoint{5.802898in}{2.659271in}}%
\pgfpathlineto{\pgfqpoint{5.795647in}{2.652547in}}%
\pgfpathlineto{\pgfqpoint{5.788395in}{2.646000in}}%
\pgfpathlineto{\pgfqpoint{5.781144in}{2.639625in}}%
\pgfpathlineto{\pgfqpoint{5.773892in}{2.633414in}}%
\pgfpathclose%
\pgfusepath{fill}%
\end{pgfscope}%
\begin{pgfscope}%
\pgfpathrectangle{\pgfqpoint{1.150000in}{0.150000in}}{\pgfqpoint{5.700000in}{5.700000in}}%
\pgfusepath{clip}%
\pgfsetbuttcap%
\pgfsetroundjoin%
\definecolor{currentfill}{rgb}{0.274128,0.199721,0.498911}%
\pgfsetfillcolor{currentfill}%
\pgfsetfillopacity{0.700000}%
\pgfsetlinewidth{0.000000pt}%
\definecolor{currentstroke}{rgb}{0.000000,0.000000,0.000000}%
\pgfsetstrokecolor{currentstroke}%
\pgfsetdash{}{0pt}%
\pgfpathmoveto{\pgfqpoint{2.235804in}{2.349008in}}%
\pgfpathlineto{\pgfqpoint{2.249443in}{2.336479in}}%
\pgfpathlineto{\pgfqpoint{2.263079in}{2.324090in}}%
\pgfpathlineto{\pgfqpoint{2.276712in}{2.311841in}}%
\pgfpathlineto{\pgfqpoint{2.290342in}{2.299728in}}%
\pgfpathlineto{\pgfqpoint{2.299071in}{2.304048in}}%
\pgfpathlineto{\pgfqpoint{2.307786in}{2.308514in}}%
\pgfpathlineto{\pgfqpoint{2.316489in}{2.313125in}}%
\pgfpathlineto{\pgfqpoint{2.325179in}{2.317878in}}%
\pgfpathlineto{\pgfqpoint{2.311577in}{2.329778in}}%
\pgfpathlineto{\pgfqpoint{2.297972in}{2.341814in}}%
\pgfpathlineto{\pgfqpoint{2.284364in}{2.353989in}}%
\pgfpathlineto{\pgfqpoint{2.270753in}{2.366305in}}%
\pgfpathlineto{\pgfqpoint{2.262035in}{2.361757in}}%
\pgfpathlineto{\pgfqpoint{2.253305in}{2.357357in}}%
\pgfpathlineto{\pgfqpoint{2.244561in}{2.353106in}}%
\pgfpathlineto{\pgfqpoint{2.235804in}{2.349008in}}%
\pgfpathclose%
\pgfusepath{fill}%
\end{pgfscope}%
\begin{pgfscope}%
\pgfpathrectangle{\pgfqpoint{1.150000in}{0.150000in}}{\pgfqpoint{5.700000in}{5.700000in}}%
\pgfusepath{clip}%
\pgfsetbuttcap%
\pgfsetroundjoin%
\definecolor{currentfill}{rgb}{0.276022,0.044167,0.370164}%
\pgfsetfillcolor{currentfill}%
\pgfsetfillopacity{0.700000}%
\pgfsetlinewidth{0.000000pt}%
\definecolor{currentstroke}{rgb}{0.000000,0.000000,0.000000}%
\pgfsetstrokecolor{currentstroke}%
\pgfsetdash{}{0pt}%
\pgfpathmoveto{\pgfqpoint{2.881705in}{2.027411in}}%
\pgfpathlineto{\pgfqpoint{2.895260in}{2.020220in}}%
\pgfpathlineto{\pgfqpoint{2.908818in}{2.013131in}}%
\pgfpathlineto{\pgfqpoint{2.922377in}{2.006144in}}%
\pgfpathlineto{\pgfqpoint{2.935939in}{1.999257in}}%
\pgfpathlineto{\pgfqpoint{2.944318in}{2.006709in}}%
\pgfpathlineto{\pgfqpoint{2.952690in}{2.014227in}}%
\pgfpathlineto{\pgfqpoint{2.961053in}{2.021809in}}%
\pgfpathlineto{\pgfqpoint{2.969409in}{2.029452in}}%
\pgfpathlineto{\pgfqpoint{2.955865in}{2.036194in}}%
\pgfpathlineto{\pgfqpoint{2.942323in}{2.043037in}}%
\pgfpathlineto{\pgfqpoint{2.928784in}{2.049981in}}%
\pgfpathlineto{\pgfqpoint{2.915247in}{2.057027in}}%
\pgfpathlineto{\pgfqpoint{2.906874in}{2.049521in}}%
\pgfpathlineto{\pgfqpoint{2.898492in}{2.042082in}}%
\pgfpathlineto{\pgfqpoint{2.890103in}{2.034711in}}%
\pgfpathlineto{\pgfqpoint{2.881705in}{2.027411in}}%
\pgfpathclose%
\pgfusepath{fill}%
\end{pgfscope}%
\begin{pgfscope}%
\pgfpathrectangle{\pgfqpoint{1.150000in}{0.150000in}}{\pgfqpoint{5.700000in}{5.700000in}}%
\pgfusepath{clip}%
\pgfsetbuttcap%
\pgfsetroundjoin%
\definecolor{currentfill}{rgb}{0.280894,0.078907,0.402329}%
\pgfsetfillcolor{currentfill}%
\pgfsetfillopacity{0.700000}%
\pgfsetlinewidth{0.000000pt}%
\definecolor{currentstroke}{rgb}{0.000000,0.000000,0.000000}%
\pgfsetstrokecolor{currentstroke}%
\pgfsetdash{}{0pt}%
\pgfpathmoveto{\pgfqpoint{3.850888in}{2.069005in}}%
\pgfpathlineto{\pgfqpoint{3.864579in}{2.067025in}}%
\pgfpathlineto{\pgfqpoint{3.878278in}{2.065125in}}%
\pgfpathlineto{\pgfqpoint{3.891984in}{2.063305in}}%
\pgfpathlineto{\pgfqpoint{3.905696in}{2.061563in}}%
\pgfpathlineto{\pgfqpoint{3.913700in}{2.070424in}}%
\pgfpathlineto{\pgfqpoint{3.921699in}{2.079264in}}%
\pgfpathlineto{\pgfqpoint{3.929692in}{2.088086in}}%
\pgfpathlineto{\pgfqpoint{3.937679in}{2.096889in}}%
\pgfpathlineto{\pgfqpoint{3.923977in}{2.098651in}}%
\pgfpathlineto{\pgfqpoint{3.910282in}{2.100492in}}%
\pgfpathlineto{\pgfqpoint{3.896594in}{2.102412in}}%
\pgfpathlineto{\pgfqpoint{3.882914in}{2.104411in}}%
\pgfpathlineto{\pgfqpoint{3.874916in}{2.095581in}}%
\pgfpathlineto{\pgfqpoint{3.866912in}{2.086737in}}%
\pgfpathlineto{\pgfqpoint{3.858903in}{2.077878in}}%
\pgfpathlineto{\pgfqpoint{3.850888in}{2.069005in}}%
\pgfpathclose%
\pgfusepath{fill}%
\end{pgfscope}%
\begin{pgfscope}%
\pgfpathrectangle{\pgfqpoint{1.150000in}{0.150000in}}{\pgfqpoint{5.700000in}{5.700000in}}%
\pgfusepath{clip}%
\pgfsetbuttcap%
\pgfsetroundjoin%
\definecolor{currentfill}{rgb}{0.272594,0.025563,0.353093}%
\pgfsetfillcolor{currentfill}%
\pgfsetfillopacity{0.700000}%
\pgfsetlinewidth{0.000000pt}%
\definecolor{currentstroke}{rgb}{0.000000,0.000000,0.000000}%
\pgfsetstrokecolor{currentstroke}%
\pgfsetdash{}{0pt}%
\pgfpathmoveto{\pgfqpoint{3.165321in}{1.988808in}}%
\pgfpathlineto{\pgfqpoint{3.178890in}{1.983442in}}%
\pgfpathlineto{\pgfqpoint{3.192464in}{1.978170in}}%
\pgfpathlineto{\pgfqpoint{3.206041in}{1.972989in}}%
\pgfpathlineto{\pgfqpoint{3.219622in}{1.967900in}}%
\pgfpathlineto{\pgfqpoint{3.227879in}{1.976271in}}%
\pgfpathlineto{\pgfqpoint{3.236128in}{1.984673in}}%
\pgfpathlineto{\pgfqpoint{3.244371in}{1.993107in}}%
\pgfpathlineto{\pgfqpoint{3.252608in}{2.001570in}}%
\pgfpathlineto{\pgfqpoint{3.239041in}{2.006556in}}%
\pgfpathlineto{\pgfqpoint{3.225478in}{2.011633in}}%
\pgfpathlineto{\pgfqpoint{3.211919in}{2.016803in}}%
\pgfpathlineto{\pgfqpoint{3.198364in}{2.022065in}}%
\pgfpathlineto{\pgfqpoint{3.190114in}{2.013697in}}%
\pgfpathlineto{\pgfqpoint{3.181856in}{2.005364in}}%
\pgfpathlineto{\pgfqpoint{3.173592in}{1.997067in}}%
\pgfpathlineto{\pgfqpoint{3.165321in}{1.988808in}}%
\pgfpathclose%
\pgfusepath{fill}%
\end{pgfscope}%
\begin{pgfscope}%
\pgfpathrectangle{\pgfqpoint{1.150000in}{0.150000in}}{\pgfqpoint{5.700000in}{5.700000in}}%
\pgfusepath{clip}%
\pgfsetbuttcap%
\pgfsetroundjoin%
\definecolor{currentfill}{rgb}{0.243113,0.292092,0.538516}%
\pgfsetfillcolor{currentfill}%
\pgfsetfillopacity{0.700000}%
\pgfsetlinewidth{0.000000pt}%
\definecolor{currentstroke}{rgb}{0.000000,0.000000,0.000000}%
\pgfsetstrokecolor{currentstroke}%
\pgfsetdash{}{0pt}%
\pgfpathmoveto{\pgfqpoint{5.372086in}{2.516753in}}%
\pgfpathlineto{\pgfqpoint{5.386255in}{2.517811in}}%
\pgfpathlineto{\pgfqpoint{5.400434in}{2.518937in}}%
\pgfpathlineto{\pgfqpoint{5.414625in}{2.520131in}}%
\pgfpathlineto{\pgfqpoint{5.428827in}{2.521392in}}%
\pgfpathlineto{\pgfqpoint{5.436216in}{2.527199in}}%
\pgfpathlineto{\pgfqpoint{5.443601in}{2.533085in}}%
\pgfpathlineto{\pgfqpoint{5.450983in}{2.539056in}}%
\pgfpathlineto{\pgfqpoint{5.458361in}{2.545119in}}%
\pgfpathlineto{\pgfqpoint{5.444182in}{2.544190in}}%
\pgfpathlineto{\pgfqpoint{5.430013in}{2.543328in}}%
\pgfpathlineto{\pgfqpoint{5.415855in}{2.542534in}}%
\pgfpathlineto{\pgfqpoint{5.401708in}{2.541807in}}%
\pgfpathlineto{\pgfqpoint{5.394308in}{2.535405in}}%
\pgfpathlineto{\pgfqpoint{5.386904in}{2.529099in}}%
\pgfpathlineto{\pgfqpoint{5.379497in}{2.522884in}}%
\pgfpathlineto{\pgfqpoint{5.372086in}{2.516753in}}%
\pgfpathclose%
\pgfusepath{fill}%
\end{pgfscope}%
\begin{pgfscope}%
\pgfpathrectangle{\pgfqpoint{1.150000in}{0.150000in}}{\pgfqpoint{5.700000in}{5.700000in}}%
\pgfusepath{clip}%
\pgfsetbuttcap%
\pgfsetroundjoin%
\definecolor{currentfill}{rgb}{0.278012,0.180367,0.486697}%
\pgfsetfillcolor{currentfill}%
\pgfsetfillopacity{0.700000}%
\pgfsetlinewidth{0.000000pt}%
\definecolor{currentstroke}{rgb}{0.000000,0.000000,0.000000}%
\pgfsetstrokecolor{currentstroke}%
\pgfsetdash{}{0pt}%
\pgfpathmoveto{\pgfqpoint{4.568109in}{2.272494in}}%
\pgfpathlineto{\pgfqpoint{4.582012in}{2.272660in}}%
\pgfpathlineto{\pgfqpoint{4.595925in}{2.272898in}}%
\pgfpathlineto{\pgfqpoint{4.609847in}{2.273208in}}%
\pgfpathlineto{\pgfqpoint{4.623778in}{2.273591in}}%
\pgfpathlineto{\pgfqpoint{4.631514in}{2.281046in}}%
\pgfpathlineto{\pgfqpoint{4.639244in}{2.288488in}}%
\pgfpathlineto{\pgfqpoint{4.646969in}{2.295921in}}%
\pgfpathlineto{\pgfqpoint{4.654687in}{2.303348in}}%
\pgfpathlineto{\pgfqpoint{4.640769in}{2.303131in}}%
\pgfpathlineto{\pgfqpoint{4.626861in}{2.302986in}}%
\pgfpathlineto{\pgfqpoint{4.612962in}{2.302913in}}%
\pgfpathlineto{\pgfqpoint{4.599072in}{2.302913in}}%
\pgfpathlineto{\pgfqpoint{4.591340in}{2.295313in}}%
\pgfpathlineto{\pgfqpoint{4.583602in}{2.287712in}}%
\pgfpathlineto{\pgfqpoint{4.575858in}{2.280107in}}%
\pgfpathlineto{\pgfqpoint{4.568109in}{2.272494in}}%
\pgfpathclose%
\pgfusepath{fill}%
\end{pgfscope}%
\begin{pgfscope}%
\pgfpathrectangle{\pgfqpoint{1.150000in}{0.150000in}}{\pgfqpoint{5.700000in}{5.700000in}}%
\pgfusepath{clip}%
\pgfsetbuttcap%
\pgfsetroundjoin%
\definecolor{currentfill}{rgb}{0.283229,0.120777,0.440584}%
\pgfsetfillcolor{currentfill}%
\pgfsetfillopacity{0.700000}%
\pgfsetlinewidth{0.000000pt}%
\definecolor{currentstroke}{rgb}{0.000000,0.000000,0.000000}%
\pgfsetstrokecolor{currentstroke}%
\pgfsetdash{}{0pt}%
\pgfpathmoveto{\pgfqpoint{4.166113in}{2.150463in}}%
\pgfpathlineto{\pgfqpoint{4.179892in}{2.149602in}}%
\pgfpathlineto{\pgfqpoint{4.193679in}{2.148817in}}%
\pgfpathlineto{\pgfqpoint{4.207474in}{2.148107in}}%
\pgfpathlineto{\pgfqpoint{4.221277in}{2.147472in}}%
\pgfpathlineto{\pgfqpoint{4.229169in}{2.155864in}}%
\pgfpathlineto{\pgfqpoint{4.237055in}{2.164229in}}%
\pgfpathlineto{\pgfqpoint{4.244935in}{2.172571in}}%
\pgfpathlineto{\pgfqpoint{4.252810in}{2.180890in}}%
\pgfpathlineto{\pgfqpoint{4.239018in}{2.181607in}}%
\pgfpathlineto{\pgfqpoint{4.225234in}{2.182399in}}%
\pgfpathlineto{\pgfqpoint{4.211458in}{2.183266in}}%
\pgfpathlineto{\pgfqpoint{4.197691in}{2.184209in}}%
\pgfpathlineto{\pgfqpoint{4.189805in}{2.175800in}}%
\pgfpathlineto{\pgfqpoint{4.181913in}{2.167374in}}%
\pgfpathlineto{\pgfqpoint{4.174016in}{2.158929in}}%
\pgfpathlineto{\pgfqpoint{4.166113in}{2.150463in}}%
\pgfpathclose%
\pgfusepath{fill}%
\end{pgfscope}%
\begin{pgfscope}%
\pgfpathrectangle{\pgfqpoint{1.150000in}{0.150000in}}{\pgfqpoint{5.700000in}{5.700000in}}%
\pgfusepath{clip}%
\pgfsetbuttcap%
\pgfsetroundjoin%
\definecolor{currentfill}{rgb}{0.263663,0.237631,0.518762}%
\pgfsetfillcolor{currentfill}%
\pgfsetfillopacity{0.700000}%
\pgfsetlinewidth{0.000000pt}%
\definecolor{currentstroke}{rgb}{0.000000,0.000000,0.000000}%
\pgfsetstrokecolor{currentstroke}%
\pgfsetdash{}{0pt}%
\pgfpathmoveto{\pgfqpoint{4.970132in}{2.396507in}}%
\pgfpathlineto{\pgfqpoint{4.984168in}{2.397307in}}%
\pgfpathlineto{\pgfqpoint{4.998215in}{2.398178in}}%
\pgfpathlineto{\pgfqpoint{5.012273in}{2.399118in}}%
\pgfpathlineto{\pgfqpoint{5.026340in}{2.400128in}}%
\pgfpathlineto{\pgfqpoint{5.033906in}{2.406604in}}%
\pgfpathlineto{\pgfqpoint{5.041467in}{2.413103in}}%
\pgfpathlineto{\pgfqpoint{5.049022in}{2.419629in}}%
\pgfpathlineto{\pgfqpoint{5.056572in}{2.426186in}}%
\pgfpathlineto{\pgfqpoint{5.042522in}{2.425425in}}%
\pgfpathlineto{\pgfqpoint{5.028482in}{2.424733in}}%
\pgfpathlineto{\pgfqpoint{5.014452in}{2.424111in}}%
\pgfpathlineto{\pgfqpoint{5.000432in}{2.423559in}}%
\pgfpathlineto{\pgfqpoint{4.992865in}{2.416746in}}%
\pgfpathlineto{\pgfqpoint{4.985293in}{2.409969in}}%
\pgfpathlineto{\pgfqpoint{4.977715in}{2.403224in}}%
\pgfpathlineto{\pgfqpoint{4.970132in}{2.396507in}}%
\pgfpathclose%
\pgfusepath{fill}%
\end{pgfscope}%
\begin{pgfscope}%
\pgfpathrectangle{\pgfqpoint{1.150000in}{0.150000in}}{\pgfqpoint{5.700000in}{5.700000in}}%
\pgfusepath{clip}%
\pgfsetbuttcap%
\pgfsetroundjoin%
\definecolor{currentfill}{rgb}{0.276022,0.044167,0.370164}%
\pgfsetfillcolor{currentfill}%
\pgfsetfillopacity{0.700000}%
\pgfsetlinewidth{0.000000pt}%
\definecolor{currentstroke}{rgb}{0.000000,0.000000,0.000000}%
\pgfsetstrokecolor{currentstroke}%
\pgfsetdash{}{0pt}%
\pgfpathmoveto{\pgfqpoint{3.535550in}{2.004864in}}%
\pgfpathlineto{\pgfqpoint{3.549174in}{2.001513in}}%
\pgfpathlineto{\pgfqpoint{3.562804in}{1.998246in}}%
\pgfpathlineto{\pgfqpoint{3.576440in}{1.995064in}}%
\pgfpathlineto{\pgfqpoint{3.590082in}{1.991964in}}%
\pgfpathlineto{\pgfqpoint{3.598199in}{2.000903in}}%
\pgfpathlineto{\pgfqpoint{3.606311in}{2.009839in}}%
\pgfpathlineto{\pgfqpoint{3.614416in}{2.018772in}}%
\pgfpathlineto{\pgfqpoint{3.622516in}{2.027702in}}%
\pgfpathlineto{\pgfqpoint{3.608886in}{2.030760in}}%
\pgfpathlineto{\pgfqpoint{3.595262in}{2.033901in}}%
\pgfpathlineto{\pgfqpoint{3.581643in}{2.037126in}}%
\pgfpathlineto{\pgfqpoint{3.568030in}{2.040436in}}%
\pgfpathlineto{\pgfqpoint{3.559919in}{2.031539in}}%
\pgfpathlineto{\pgfqpoint{3.551802in}{2.022645in}}%
\pgfpathlineto{\pgfqpoint{3.543678in}{2.013753in}}%
\pgfpathlineto{\pgfqpoint{3.535550in}{2.004864in}}%
\pgfpathclose%
\pgfusepath{fill}%
\end{pgfscope}%
\begin{pgfscope}%
\pgfpathrectangle{\pgfqpoint{1.150000in}{0.150000in}}{\pgfqpoint{5.700000in}{5.700000in}}%
\pgfusepath{clip}%
\pgfsetbuttcap%
\pgfsetroundjoin%
\definecolor{currentfill}{rgb}{0.283091,0.110553,0.431554}%
\pgfsetfillcolor{currentfill}%
\pgfsetfillopacity{0.700000}%
\pgfsetlinewidth{0.000000pt}%
\definecolor{currentstroke}{rgb}{0.000000,0.000000,0.000000}%
\pgfsetstrokecolor{currentstroke}%
\pgfsetdash{}{0pt}%
\pgfpathmoveto{\pgfqpoint{2.542572in}{2.145201in}}%
\pgfpathlineto{\pgfqpoint{2.556150in}{2.135460in}}%
\pgfpathlineto{\pgfqpoint{2.569729in}{2.125838in}}%
\pgfpathlineto{\pgfqpoint{2.583307in}{2.116333in}}%
\pgfpathlineto{\pgfqpoint{2.596884in}{2.106943in}}%
\pgfpathlineto{\pgfqpoint{2.605439in}{2.112822in}}%
\pgfpathlineto{\pgfqpoint{2.613983in}{2.118809in}}%
\pgfpathlineto{\pgfqpoint{2.622516in}{2.124904in}}%
\pgfpathlineto{\pgfqpoint{2.631040in}{2.131104in}}%
\pgfpathlineto{\pgfqpoint{2.617485in}{2.140305in}}%
\pgfpathlineto{\pgfqpoint{2.603930in}{2.149622in}}%
\pgfpathlineto{\pgfqpoint{2.590375in}{2.159057in}}%
\pgfpathlineto{\pgfqpoint{2.576820in}{2.168608in}}%
\pgfpathlineto{\pgfqpoint{2.568273in}{2.162589in}}%
\pgfpathlineto{\pgfqpoint{2.559717in}{2.156680in}}%
\pgfpathlineto{\pgfqpoint{2.551150in}{2.150883in}}%
\pgfpathlineto{\pgfqpoint{2.542572in}{2.145201in}}%
\pgfpathclose%
\pgfusepath{fill}%
\end{pgfscope}%
\begin{pgfscope}%
\pgfpathrectangle{\pgfqpoint{1.150000in}{0.150000in}}{\pgfqpoint{5.700000in}{5.700000in}}%
\pgfusepath{clip}%
\pgfsetbuttcap%
\pgfsetroundjoin%
\definecolor{currentfill}{rgb}{0.272594,0.025563,0.353093}%
\pgfsetfillcolor{currentfill}%
\pgfsetfillopacity{0.700000}%
\pgfsetlinewidth{0.000000pt}%
\definecolor{currentstroke}{rgb}{0.000000,0.000000,0.000000}%
\pgfsetstrokecolor{currentstroke}%
\pgfsetdash{}{0pt}%
\pgfpathmoveto{\pgfqpoint{3.306918in}{1.982533in}}%
\pgfpathlineto{\pgfqpoint{3.320507in}{1.977998in}}%
\pgfpathlineto{\pgfqpoint{3.334100in}{1.973552in}}%
\pgfpathlineto{\pgfqpoint{3.347698in}{1.969195in}}%
\pgfpathlineto{\pgfqpoint{3.361301in}{1.964925in}}%
\pgfpathlineto{\pgfqpoint{3.369503in}{1.973597in}}%
\pgfpathlineto{\pgfqpoint{3.377700in}{1.982286in}}%
\pgfpathlineto{\pgfqpoint{3.385890in}{1.990992in}}%
\pgfpathlineto{\pgfqpoint{3.394074in}{1.999714in}}%
\pgfpathlineto{\pgfqpoint{3.380484in}{2.003901in}}%
\pgfpathlineto{\pgfqpoint{3.366899in}{2.008176in}}%
\pgfpathlineto{\pgfqpoint{3.353319in}{2.012539in}}%
\pgfpathlineto{\pgfqpoint{3.339744in}{2.016992in}}%
\pgfpathlineto{\pgfqpoint{3.331547in}{2.008345in}}%
\pgfpathlineto{\pgfqpoint{3.323344in}{1.999719in}}%
\pgfpathlineto{\pgfqpoint{3.315134in}{1.991115in}}%
\pgfpathlineto{\pgfqpoint{3.306918in}{1.982533in}}%
\pgfpathclose%
\pgfusepath{fill}%
\end{pgfscope}%
\begin{pgfscope}%
\pgfpathrectangle{\pgfqpoint{1.150000in}{0.150000in}}{\pgfqpoint{5.700000in}{5.700000in}}%
\pgfusepath{clip}%
\pgfsetbuttcap%
\pgfsetroundjoin%
\definecolor{currentfill}{rgb}{0.279566,0.067836,0.391917}%
\pgfsetfillcolor{currentfill}%
\pgfsetfillopacity{0.700000}%
\pgfsetlinewidth{0.000000pt}%
\definecolor{currentstroke}{rgb}{0.000000,0.000000,0.000000}%
\pgfsetstrokecolor{currentstroke}%
\pgfsetdash{}{0pt}%
\pgfpathmoveto{\pgfqpoint{2.739500in}{2.061560in}}%
\pgfpathlineto{\pgfqpoint{2.753061in}{2.053363in}}%
\pgfpathlineto{\pgfqpoint{2.766624in}{2.045273in}}%
\pgfpathlineto{\pgfqpoint{2.780188in}{2.037291in}}%
\pgfpathlineto{\pgfqpoint{2.793753in}{2.029415in}}%
\pgfpathlineto{\pgfqpoint{2.802204in}{2.036239in}}%
\pgfpathlineto{\pgfqpoint{2.810647in}{2.043147in}}%
\pgfpathlineto{\pgfqpoint{2.819080in}{2.050137in}}%
\pgfpathlineto{\pgfqpoint{2.827505in}{2.057206in}}%
\pgfpathlineto{\pgfqpoint{2.813959in}{2.064917in}}%
\pgfpathlineto{\pgfqpoint{2.800415in}{2.072733in}}%
\pgfpathlineto{\pgfqpoint{2.786873in}{2.080656in}}%
\pgfpathlineto{\pgfqpoint{2.773331in}{2.088687in}}%
\pgfpathlineto{\pgfqpoint{2.764887in}{2.081776in}}%
\pgfpathlineto{\pgfqpoint{2.756434in}{2.074949in}}%
\pgfpathlineto{\pgfqpoint{2.747971in}{2.068210in}}%
\pgfpathlineto{\pgfqpoint{2.739500in}{2.061560in}}%
\pgfpathclose%
\pgfusepath{fill}%
\end{pgfscope}%
\begin{pgfscope}%
\pgfpathrectangle{\pgfqpoint{1.150000in}{0.150000in}}{\pgfqpoint{5.700000in}{5.700000in}}%
\pgfusepath{clip}%
\pgfsetbuttcap%
\pgfsetroundjoin%
\definecolor{currentfill}{rgb}{0.278012,0.180367,0.486697}%
\pgfsetfillcolor{currentfill}%
\pgfsetfillopacity{0.700000}%
\pgfsetlinewidth{0.000000pt}%
\definecolor{currentstroke}{rgb}{0.000000,0.000000,0.000000}%
\pgfsetstrokecolor{currentstroke}%
\pgfsetdash{}{0pt}%
\pgfpathmoveto{\pgfqpoint{2.290342in}{2.299728in}}%
\pgfpathlineto{\pgfqpoint{2.303970in}{2.287752in}}%
\pgfpathlineto{\pgfqpoint{2.317595in}{2.275911in}}%
\pgfpathlineto{\pgfqpoint{2.331217in}{2.264204in}}%
\pgfpathlineto{\pgfqpoint{2.344837in}{2.252629in}}%
\pgfpathlineto{\pgfqpoint{2.353538in}{2.257169in}}%
\pgfpathlineto{\pgfqpoint{2.362226in}{2.261851in}}%
\pgfpathlineto{\pgfqpoint{2.370902in}{2.266671in}}%
\pgfpathlineto{\pgfqpoint{2.379566in}{2.271628in}}%
\pgfpathlineto{\pgfqpoint{2.365972in}{2.282991in}}%
\pgfpathlineto{\pgfqpoint{2.352377in}{2.294486in}}%
\pgfpathlineto{\pgfqpoint{2.338779in}{2.306115in}}%
\pgfpathlineto{\pgfqpoint{2.325179in}{2.317878in}}%
\pgfpathlineto{\pgfqpoint{2.316489in}{2.313125in}}%
\pgfpathlineto{\pgfqpoint{2.307786in}{2.308514in}}%
\pgfpathlineto{\pgfqpoint{2.299071in}{2.304048in}}%
\pgfpathlineto{\pgfqpoint{2.290342in}{2.299728in}}%
\pgfpathclose%
\pgfusepath{fill}%
\end{pgfscope}%
\begin{pgfscope}%
\pgfpathrectangle{\pgfqpoint{1.150000in}{0.150000in}}{\pgfqpoint{5.700000in}{5.700000in}}%
\pgfusepath{clip}%
\pgfsetbuttcap%
\pgfsetroundjoin%
\definecolor{currentfill}{rgb}{0.225863,0.330805,0.547314}%
\pgfsetfillcolor{currentfill}%
\pgfsetfillopacity{0.700000}%
\pgfsetlinewidth{0.000000pt}%
\definecolor{currentstroke}{rgb}{0.000000,0.000000,0.000000}%
\pgfsetstrokecolor{currentstroke}%
\pgfsetdash{}{0pt}%
\pgfpathmoveto{\pgfqpoint{5.687714in}{2.605360in}}%
\pgfpathlineto{\pgfqpoint{5.701987in}{2.606436in}}%
\pgfpathlineto{\pgfqpoint{5.716271in}{2.607580in}}%
\pgfpathlineto{\pgfqpoint{5.730567in}{2.608790in}}%
\pgfpathlineto{\pgfqpoint{5.744874in}{2.610066in}}%
\pgfpathlineto{\pgfqpoint{5.752131in}{2.615692in}}%
\pgfpathlineto{\pgfqpoint{5.759385in}{2.621454in}}%
\pgfpathlineto{\pgfqpoint{5.766639in}{2.627359in}}%
\pgfpathlineto{\pgfqpoint{5.773892in}{2.633414in}}%
\pgfpathlineto{\pgfqpoint{5.759611in}{2.632531in}}%
\pgfpathlineto{\pgfqpoint{5.745342in}{2.631715in}}%
\pgfpathlineto{\pgfqpoint{5.731084in}{2.630965in}}%
\pgfpathlineto{\pgfqpoint{5.716836in}{2.630282in}}%
\pgfpathlineto{\pgfqpoint{5.709557in}{2.623826in}}%
\pgfpathlineto{\pgfqpoint{5.702277in}{2.617526in}}%
\pgfpathlineto{\pgfqpoint{5.694996in}{2.611372in}}%
\pgfpathlineto{\pgfqpoint{5.687714in}{2.605360in}}%
\pgfpathclose%
\pgfusepath{fill}%
\end{pgfscope}%
\begin{pgfscope}%
\pgfpathrectangle{\pgfqpoint{1.150000in}{0.150000in}}{\pgfqpoint{5.700000in}{5.700000in}}%
\pgfusepath{clip}%
\pgfsetbuttcap%
\pgfsetroundjoin%
\definecolor{currentfill}{rgb}{0.279566,0.067836,0.391917}%
\pgfsetfillcolor{currentfill}%
\pgfsetfillopacity{0.700000}%
\pgfsetlinewidth{0.000000pt}%
\definecolor{currentstroke}{rgb}{0.000000,0.000000,0.000000}%
\pgfsetstrokecolor{currentstroke}%
\pgfsetdash{}{0pt}%
\pgfpathmoveto{\pgfqpoint{3.764030in}{2.042064in}}%
\pgfpathlineto{\pgfqpoint{3.777705in}{2.039764in}}%
\pgfpathlineto{\pgfqpoint{3.791387in}{2.037545in}}%
\pgfpathlineto{\pgfqpoint{3.805076in}{2.035406in}}%
\pgfpathlineto{\pgfqpoint{3.818771in}{2.033347in}}%
\pgfpathlineto{\pgfqpoint{3.826809in}{2.042287in}}%
\pgfpathlineto{\pgfqpoint{3.834841in}{2.051210in}}%
\pgfpathlineto{\pgfqpoint{3.842867in}{2.060115in}}%
\pgfpathlineto{\pgfqpoint{3.850888in}{2.069005in}}%
\pgfpathlineto{\pgfqpoint{3.837203in}{2.071064in}}%
\pgfpathlineto{\pgfqpoint{3.823525in}{2.073202in}}%
\pgfpathlineto{\pgfqpoint{3.809854in}{2.075421in}}%
\pgfpathlineto{\pgfqpoint{3.796190in}{2.077721in}}%
\pgfpathlineto{\pgfqpoint{3.788158in}{2.068824in}}%
\pgfpathlineto{\pgfqpoint{3.780121in}{2.059916in}}%
\pgfpathlineto{\pgfqpoint{3.772078in}{2.050996in}}%
\pgfpathlineto{\pgfqpoint{3.764030in}{2.042064in}}%
\pgfpathclose%
\pgfusepath{fill}%
\end{pgfscope}%
\begin{pgfscope}%
\pgfpathrectangle{\pgfqpoint{1.150000in}{0.150000in}}{\pgfqpoint{5.700000in}{5.700000in}}%
\pgfusepath{clip}%
\pgfsetbuttcap%
\pgfsetroundjoin%
\definecolor{currentfill}{rgb}{0.279574,0.170599,0.479997}%
\pgfsetfillcolor{currentfill}%
\pgfsetfillopacity{0.700000}%
\pgfsetlinewidth{0.000000pt}%
\definecolor{currentstroke}{rgb}{0.000000,0.000000,0.000000}%
\pgfsetstrokecolor{currentstroke}%
\pgfsetdash{}{0pt}%
\pgfpathmoveto{\pgfqpoint{4.481477in}{2.241393in}}%
\pgfpathlineto{\pgfqpoint{4.495357in}{2.241412in}}%
\pgfpathlineto{\pgfqpoint{4.509246in}{2.241505in}}%
\pgfpathlineto{\pgfqpoint{4.523144in}{2.241670in}}%
\pgfpathlineto{\pgfqpoint{4.537052in}{2.241908in}}%
\pgfpathlineto{\pgfqpoint{4.544825in}{2.249581in}}%
\pgfpathlineto{\pgfqpoint{4.552592in}{2.257234in}}%
\pgfpathlineto{\pgfqpoint{4.560353in}{2.264871in}}%
\pgfpathlineto{\pgfqpoint{4.568109in}{2.272494in}}%
\pgfpathlineto{\pgfqpoint{4.554215in}{2.272401in}}%
\pgfpathlineto{\pgfqpoint{4.540329in}{2.272380in}}%
\pgfpathlineto{\pgfqpoint{4.526453in}{2.272432in}}%
\pgfpathlineto{\pgfqpoint{4.512586in}{2.272557in}}%
\pgfpathlineto{\pgfqpoint{4.504818in}{2.264782in}}%
\pgfpathlineto{\pgfqpoint{4.497043in}{2.256998in}}%
\pgfpathlineto{\pgfqpoint{4.489263in}{2.249202in}}%
\pgfpathlineto{\pgfqpoint{4.481477in}{2.241393in}}%
\pgfpathclose%
\pgfusepath{fill}%
\end{pgfscope}%
\begin{pgfscope}%
\pgfpathrectangle{\pgfqpoint{1.150000in}{0.150000in}}{\pgfqpoint{5.700000in}{5.700000in}}%
\pgfusepath{clip}%
\pgfsetbuttcap%
\pgfsetroundjoin%
\definecolor{currentfill}{rgb}{0.246811,0.283237,0.535941}%
\pgfsetfillcolor{currentfill}%
\pgfsetfillopacity{0.700000}%
\pgfsetlinewidth{0.000000pt}%
\definecolor{currentstroke}{rgb}{0.000000,0.000000,0.000000}%
\pgfsetstrokecolor{currentstroke}%
\pgfsetdash{}{0pt}%
\pgfpathmoveto{\pgfqpoint{5.285747in}{2.488165in}}%
\pgfpathlineto{\pgfqpoint{5.299894in}{2.489262in}}%
\pgfpathlineto{\pgfqpoint{5.314052in}{2.490427in}}%
\pgfpathlineto{\pgfqpoint{5.328221in}{2.491660in}}%
\pgfpathlineto{\pgfqpoint{5.342400in}{2.492962in}}%
\pgfpathlineto{\pgfqpoint{5.349828in}{2.498811in}}%
\pgfpathlineto{\pgfqpoint{5.357252in}{2.504722in}}%
\pgfpathlineto{\pgfqpoint{5.364671in}{2.510701in}}%
\pgfpathlineto{\pgfqpoint{5.372086in}{2.516753in}}%
\pgfpathlineto{\pgfqpoint{5.357927in}{2.515763in}}%
\pgfpathlineto{\pgfqpoint{5.343780in}{2.514841in}}%
\pgfpathlineto{\pgfqpoint{5.329643in}{2.513987in}}%
\pgfpathlineto{\pgfqpoint{5.315516in}{2.513202in}}%
\pgfpathlineto{\pgfqpoint{5.308080in}{2.506831in}}%
\pgfpathlineto{\pgfqpoint{5.300640in}{2.500538in}}%
\pgfpathlineto{\pgfqpoint{5.293196in}{2.494318in}}%
\pgfpathlineto{\pgfqpoint{5.285747in}{2.488165in}}%
\pgfpathclose%
\pgfusepath{fill}%
\end{pgfscope}%
\begin{pgfscope}%
\pgfpathrectangle{\pgfqpoint{1.150000in}{0.150000in}}{\pgfqpoint{5.700000in}{5.700000in}}%
\pgfusepath{clip}%
\pgfsetbuttcap%
\pgfsetroundjoin%
\definecolor{currentfill}{rgb}{0.283091,0.110553,0.431554}%
\pgfsetfillcolor{currentfill}%
\pgfsetfillopacity{0.700000}%
\pgfsetlinewidth{0.000000pt}%
\definecolor{currentstroke}{rgb}{0.000000,0.000000,0.000000}%
\pgfsetstrokecolor{currentstroke}%
\pgfsetdash{}{0pt}%
\pgfpathmoveto{\pgfqpoint{4.079363in}{2.120319in}}%
\pgfpathlineto{\pgfqpoint{4.093122in}{2.119214in}}%
\pgfpathlineto{\pgfqpoint{4.106888in}{2.118187in}}%
\pgfpathlineto{\pgfqpoint{4.120662in}{2.117235in}}%
\pgfpathlineto{\pgfqpoint{4.134443in}{2.116360in}}%
\pgfpathlineto{\pgfqpoint{4.142370in}{2.124925in}}%
\pgfpathlineto{\pgfqpoint{4.150290in}{2.133462in}}%
\pgfpathlineto{\pgfqpoint{4.158204in}{2.141975in}}%
\pgfpathlineto{\pgfqpoint{4.166113in}{2.150463in}}%
\pgfpathlineto{\pgfqpoint{4.152342in}{2.151400in}}%
\pgfpathlineto{\pgfqpoint{4.138579in}{2.152413in}}%
\pgfpathlineto{\pgfqpoint{4.124824in}{2.153502in}}%
\pgfpathlineto{\pgfqpoint{4.111077in}{2.154668in}}%
\pgfpathlineto{\pgfqpoint{4.103157in}{2.146110in}}%
\pgfpathlineto{\pgfqpoint{4.095232in}{2.137534in}}%
\pgfpathlineto{\pgfqpoint{4.087300in}{2.128937in}}%
\pgfpathlineto{\pgfqpoint{4.079363in}{2.120319in}}%
\pgfpathclose%
\pgfusepath{fill}%
\end{pgfscope}%
\begin{pgfscope}%
\pgfpathrectangle{\pgfqpoint{1.150000in}{0.150000in}}{\pgfqpoint{5.700000in}{5.700000in}}%
\pgfusepath{clip}%
\pgfsetbuttcap%
\pgfsetroundjoin%
\definecolor{currentfill}{rgb}{0.266580,0.228262,0.514349}%
\pgfsetfillcolor{currentfill}%
\pgfsetfillopacity{0.700000}%
\pgfsetlinewidth{0.000000pt}%
\definecolor{currentstroke}{rgb}{0.000000,0.000000,0.000000}%
\pgfsetstrokecolor{currentstroke}%
\pgfsetdash{}{0pt}%
\pgfpathmoveto{\pgfqpoint{4.883630in}{2.366411in}}%
\pgfpathlineto{\pgfqpoint{4.897644in}{2.367159in}}%
\pgfpathlineto{\pgfqpoint{4.911667in}{2.367978in}}%
\pgfpathlineto{\pgfqpoint{4.925700in}{2.368867in}}%
\pgfpathlineto{\pgfqpoint{4.939743in}{2.369826in}}%
\pgfpathlineto{\pgfqpoint{4.947349in}{2.376476in}}%
\pgfpathlineto{\pgfqpoint{4.954949in}{2.383137in}}%
\pgfpathlineto{\pgfqpoint{4.962543in}{2.389813in}}%
\pgfpathlineto{\pgfqpoint{4.970132in}{2.396507in}}%
\pgfpathlineto{\pgfqpoint{4.956105in}{2.395776in}}%
\pgfpathlineto{\pgfqpoint{4.942088in}{2.395115in}}%
\pgfpathlineto{\pgfqpoint{4.928081in}{2.394525in}}%
\pgfpathlineto{\pgfqpoint{4.914084in}{2.394004in}}%
\pgfpathlineto{\pgfqpoint{4.906479in}{2.387075in}}%
\pgfpathlineto{\pgfqpoint{4.898868in}{2.380169in}}%
\pgfpathlineto{\pgfqpoint{4.891252in}{2.373282in}}%
\pgfpathlineto{\pgfqpoint{4.883630in}{2.366411in}}%
\pgfpathclose%
\pgfusepath{fill}%
\end{pgfscope}%
\begin{pgfscope}%
\pgfpathrectangle{\pgfqpoint{1.150000in}{0.150000in}}{\pgfqpoint{5.700000in}{5.700000in}}%
\pgfusepath{clip}%
\pgfsetbuttcap%
\pgfsetroundjoin%
\definecolor{currentfill}{rgb}{0.280868,0.160771,0.472899}%
\pgfsetfillcolor{currentfill}%
\pgfsetfillopacity{0.700000}%
\pgfsetlinewidth{0.000000pt}%
\definecolor{currentstroke}{rgb}{0.000000,0.000000,0.000000}%
\pgfsetstrokecolor{currentstroke}%
\pgfsetdash{}{0pt}%
\pgfpathmoveto{\pgfqpoint{2.344837in}{2.252629in}}%
\pgfpathlineto{\pgfqpoint{2.358455in}{2.241186in}}%
\pgfpathlineto{\pgfqpoint{2.372072in}{2.229873in}}%
\pgfpathlineto{\pgfqpoint{2.385686in}{2.218690in}}%
\pgfpathlineto{\pgfqpoint{2.399298in}{2.207634in}}%
\pgfpathlineto{\pgfqpoint{2.407972in}{2.212393in}}%
\pgfpathlineto{\pgfqpoint{2.416633in}{2.217289in}}%
\pgfpathlineto{\pgfqpoint{2.425283in}{2.222318in}}%
\pgfpathlineto{\pgfqpoint{2.433921in}{2.227478in}}%
\pgfpathlineto{\pgfqpoint{2.420334in}{2.238323in}}%
\pgfpathlineto{\pgfqpoint{2.406747in}{2.249295in}}%
\pgfpathlineto{\pgfqpoint{2.393157in}{2.260397in}}%
\pgfpathlineto{\pgfqpoint{2.379566in}{2.271628in}}%
\pgfpathlineto{\pgfqpoint{2.370902in}{2.266671in}}%
\pgfpathlineto{\pgfqpoint{2.362226in}{2.261851in}}%
\pgfpathlineto{\pgfqpoint{2.353538in}{2.257169in}}%
\pgfpathlineto{\pgfqpoint{2.344837in}{2.252629in}}%
\pgfpathclose%
\pgfusepath{fill}%
\end{pgfscope}%
\begin{pgfscope}%
\pgfpathrectangle{\pgfqpoint{1.150000in}{0.150000in}}{\pgfqpoint{5.700000in}{5.700000in}}%
\pgfusepath{clip}%
\pgfsetbuttcap%
\pgfsetroundjoin%
\definecolor{currentfill}{rgb}{0.273809,0.031497,0.358853}%
\pgfsetfillcolor{currentfill}%
\pgfsetfillopacity{0.700000}%
\pgfsetlinewidth{0.000000pt}%
\definecolor{currentstroke}{rgb}{0.000000,0.000000,0.000000}%
\pgfsetstrokecolor{currentstroke}%
\pgfsetdash{}{0pt}%
\pgfpathmoveto{\pgfqpoint{3.448482in}{1.983837in}}%
\pgfpathlineto{\pgfqpoint{3.462097in}{1.980084in}}%
\pgfpathlineto{\pgfqpoint{3.475718in}{1.976417in}}%
\pgfpathlineto{\pgfqpoint{3.489343in}{1.972835in}}%
\pgfpathlineto{\pgfqpoint{3.502975in}{1.969338in}}%
\pgfpathlineto{\pgfqpoint{3.511127in}{1.978214in}}%
\pgfpathlineto{\pgfqpoint{3.519274in}{1.987093in}}%
\pgfpathlineto{\pgfqpoint{3.527415in}{1.995977in}}%
\pgfpathlineto{\pgfqpoint{3.535550in}{2.004864in}}%
\pgfpathlineto{\pgfqpoint{3.521931in}{2.008299in}}%
\pgfpathlineto{\pgfqpoint{3.508317in}{2.011819in}}%
\pgfpathlineto{\pgfqpoint{3.494709in}{2.015425in}}%
\pgfpathlineto{\pgfqpoint{3.481106in}{2.019116in}}%
\pgfpathlineto{\pgfqpoint{3.472959in}{2.010283in}}%
\pgfpathlineto{\pgfqpoint{3.464806in}{2.001459in}}%
\pgfpathlineto{\pgfqpoint{3.456647in}{1.992644in}}%
\pgfpathlineto{\pgfqpoint{3.448482in}{1.983837in}}%
\pgfpathclose%
\pgfusepath{fill}%
\end{pgfscope}%
\begin{pgfscope}%
\pgfpathrectangle{\pgfqpoint{1.150000in}{0.150000in}}{\pgfqpoint{5.700000in}{5.700000in}}%
\pgfusepath{clip}%
\pgfsetbuttcap%
\pgfsetroundjoin%
\definecolor{currentfill}{rgb}{0.272594,0.025563,0.353093}%
\pgfsetfillcolor{currentfill}%
\pgfsetfillopacity{0.700000}%
\pgfsetlinewidth{0.000000pt}%
\definecolor{currentstroke}{rgb}{0.000000,0.000000,0.000000}%
\pgfsetstrokecolor{currentstroke}%
\pgfsetdash{}{0pt}%
\pgfpathmoveto{\pgfqpoint{3.077863in}{1.979059in}}%
\pgfpathlineto{\pgfqpoint{3.091434in}{1.973195in}}%
\pgfpathlineto{\pgfqpoint{3.105008in}{1.967425in}}%
\pgfpathlineto{\pgfqpoint{3.118586in}{1.961749in}}%
\pgfpathlineto{\pgfqpoint{3.132167in}{1.956168in}}%
\pgfpathlineto{\pgfqpoint{3.140466in}{1.964265in}}%
\pgfpathlineto{\pgfqpoint{3.148758in}{1.972405in}}%
\pgfpathlineto{\pgfqpoint{3.157043in}{1.980586in}}%
\pgfpathlineto{\pgfqpoint{3.165321in}{1.988808in}}%
\pgfpathlineto{\pgfqpoint{3.151755in}{1.994266in}}%
\pgfpathlineto{\pgfqpoint{3.138193in}{1.999818in}}%
\pgfpathlineto{\pgfqpoint{3.124635in}{2.005464in}}%
\pgfpathlineto{\pgfqpoint{3.111080in}{2.011205in}}%
\pgfpathlineto{\pgfqpoint{3.102786in}{2.003100in}}%
\pgfpathlineto{\pgfqpoint{3.094486in}{1.995039in}}%
\pgfpathlineto{\pgfqpoint{3.086178in}{1.987025in}}%
\pgfpathlineto{\pgfqpoint{3.077863in}{1.979059in}}%
\pgfpathclose%
\pgfusepath{fill}%
\end{pgfscope}%
\begin{pgfscope}%
\pgfpathrectangle{\pgfqpoint{1.150000in}{0.150000in}}{\pgfqpoint{5.700000in}{5.700000in}}%
\pgfusepath{clip}%
\pgfsetbuttcap%
\pgfsetroundjoin%
\definecolor{currentfill}{rgb}{0.229739,0.322361,0.545706}%
\pgfsetfillcolor{currentfill}%
\pgfsetfillopacity{0.700000}%
\pgfsetlinewidth{0.000000pt}%
\definecolor{currentstroke}{rgb}{0.000000,0.000000,0.000000}%
\pgfsetstrokecolor{currentstroke}%
\pgfsetdash{}{0pt}%
\pgfpathmoveto{\pgfqpoint{5.601480in}{2.577451in}}%
\pgfpathlineto{\pgfqpoint{5.615734in}{2.578634in}}%
\pgfpathlineto{\pgfqpoint{5.629999in}{2.579883in}}%
\pgfpathlineto{\pgfqpoint{5.644276in}{2.581200in}}%
\pgfpathlineto{\pgfqpoint{5.658563in}{2.582584in}}%
\pgfpathlineto{\pgfqpoint{5.665854in}{2.588100in}}%
\pgfpathlineto{\pgfqpoint{5.673143in}{2.593730in}}%
\pgfpathlineto{\pgfqpoint{5.680429in}{2.599481in}}%
\pgfpathlineto{\pgfqpoint{5.687714in}{2.605360in}}%
\pgfpathlineto{\pgfqpoint{5.673451in}{2.604350in}}%
\pgfpathlineto{\pgfqpoint{5.659200in}{2.603407in}}%
\pgfpathlineto{\pgfqpoint{5.644960in}{2.602531in}}%
\pgfpathlineto{\pgfqpoint{5.630731in}{2.601721in}}%
\pgfpathlineto{\pgfqpoint{5.623421in}{2.595462in}}%
\pgfpathlineto{\pgfqpoint{5.616110in}{2.589335in}}%
\pgfpathlineto{\pgfqpoint{5.608796in}{2.583333in}}%
\pgfpathlineto{\pgfqpoint{5.601480in}{2.577451in}}%
\pgfpathclose%
\pgfusepath{fill}%
\end{pgfscope}%
\begin{pgfscope}%
\pgfpathrectangle{\pgfqpoint{1.150000in}{0.150000in}}{\pgfqpoint{5.700000in}{5.700000in}}%
\pgfusepath{clip}%
\pgfsetbuttcap%
\pgfsetroundjoin%
\definecolor{currentfill}{rgb}{0.274952,0.037752,0.364543}%
\pgfsetfillcolor{currentfill}%
\pgfsetfillopacity{0.700000}%
\pgfsetlinewidth{0.000000pt}%
\definecolor{currentstroke}{rgb}{0.000000,0.000000,0.000000}%
\pgfsetstrokecolor{currentstroke}%
\pgfsetdash{}{0pt}%
\pgfpathmoveto{\pgfqpoint{2.935939in}{1.999257in}}%
\pgfpathlineto{\pgfqpoint{2.949503in}{1.992470in}}%
\pgfpathlineto{\pgfqpoint{2.963070in}{1.985782in}}%
\pgfpathlineto{\pgfqpoint{2.976640in}{1.979194in}}%
\pgfpathlineto{\pgfqpoint{2.990212in}{1.972703in}}%
\pgfpathlineto{\pgfqpoint{2.998574in}{1.980307in}}%
\pgfpathlineto{\pgfqpoint{3.006928in}{1.987972in}}%
\pgfpathlineto{\pgfqpoint{3.015274in}{1.995695in}}%
\pgfpathlineto{\pgfqpoint{3.023613in}{2.003475in}}%
\pgfpathlineto{\pgfqpoint{3.010058in}{2.009822in}}%
\pgfpathlineto{\pgfqpoint{2.996505in}{2.016266in}}%
\pgfpathlineto{\pgfqpoint{2.982956in}{2.022809in}}%
\pgfpathlineto{\pgfqpoint{2.969409in}{2.029452in}}%
\pgfpathlineto{\pgfqpoint{2.961053in}{2.021809in}}%
\pgfpathlineto{\pgfqpoint{2.952690in}{2.014227in}}%
\pgfpathlineto{\pgfqpoint{2.944318in}{2.006709in}}%
\pgfpathlineto{\pgfqpoint{2.935939in}{1.999257in}}%
\pgfpathclose%
\pgfusepath{fill}%
\end{pgfscope}%
\begin{pgfscope}%
\pgfpathrectangle{\pgfqpoint{1.150000in}{0.150000in}}{\pgfqpoint{5.700000in}{5.700000in}}%
\pgfusepath{clip}%
\pgfsetbuttcap%
\pgfsetroundjoin%
\definecolor{currentfill}{rgb}{0.281412,0.155834,0.469201}%
\pgfsetfillcolor{currentfill}%
\pgfsetfillopacity{0.700000}%
\pgfsetlinewidth{0.000000pt}%
\definecolor{currentstroke}{rgb}{0.000000,0.000000,0.000000}%
\pgfsetstrokecolor{currentstroke}%
\pgfsetdash{}{0pt}%
\pgfpathmoveto{\pgfqpoint{4.394794in}{2.210120in}}%
\pgfpathlineto{\pgfqpoint{4.408651in}{2.209970in}}%
\pgfpathlineto{\pgfqpoint{4.422517in}{2.209894in}}%
\pgfpathlineto{\pgfqpoint{4.436391in}{2.209891in}}%
\pgfpathlineto{\pgfqpoint{4.450274in}{2.209962in}}%
\pgfpathlineto{\pgfqpoint{4.458084in}{2.217853in}}%
\pgfpathlineto{\pgfqpoint{4.465888in}{2.225721in}}%
\pgfpathlineto{\pgfqpoint{4.473686in}{2.233567in}}%
\pgfpathlineto{\pgfqpoint{4.481477in}{2.241393in}}%
\pgfpathlineto{\pgfqpoint{4.467606in}{2.241446in}}%
\pgfpathlineto{\pgfqpoint{4.453744in}{2.241573in}}%
\pgfpathlineto{\pgfqpoint{4.439891in}{2.241773in}}%
\pgfpathlineto{\pgfqpoint{4.426046in}{2.242047in}}%
\pgfpathlineto{\pgfqpoint{4.418242in}{2.234089in}}%
\pgfpathlineto{\pgfqpoint{4.410432in}{2.226117in}}%
\pgfpathlineto{\pgfqpoint{4.402616in}{2.218128in}}%
\pgfpathlineto{\pgfqpoint{4.394794in}{2.210120in}}%
\pgfpathclose%
\pgfusepath{fill}%
\end{pgfscope}%
\begin{pgfscope}%
\pgfpathrectangle{\pgfqpoint{1.150000in}{0.150000in}}{\pgfqpoint{5.700000in}{5.700000in}}%
\pgfusepath{clip}%
\pgfsetbuttcap%
\pgfsetroundjoin%
\definecolor{currentfill}{rgb}{0.282327,0.094955,0.417331}%
\pgfsetfillcolor{currentfill}%
\pgfsetfillopacity{0.700000}%
\pgfsetlinewidth{0.000000pt}%
\definecolor{currentstroke}{rgb}{0.000000,0.000000,0.000000}%
\pgfsetstrokecolor{currentstroke}%
\pgfsetdash{}{0pt}%
\pgfpathmoveto{\pgfqpoint{2.596884in}{2.106943in}}%
\pgfpathlineto{\pgfqpoint{2.610462in}{2.097669in}}%
\pgfpathlineto{\pgfqpoint{2.624041in}{2.088509in}}%
\pgfpathlineto{\pgfqpoint{2.637619in}{2.079462in}}%
\pgfpathlineto{\pgfqpoint{2.651198in}{2.070528in}}%
\pgfpathlineto{\pgfqpoint{2.659729in}{2.076602in}}%
\pgfpathlineto{\pgfqpoint{2.668251in}{2.082780in}}%
\pgfpathlineto{\pgfqpoint{2.676763in}{2.089060in}}%
\pgfpathlineto{\pgfqpoint{2.685264in}{2.095439in}}%
\pgfpathlineto{\pgfqpoint{2.671708in}{2.104186in}}%
\pgfpathlineto{\pgfqpoint{2.658151in}{2.113045in}}%
\pgfpathlineto{\pgfqpoint{2.644596in}{2.122017in}}%
\pgfpathlineto{\pgfqpoint{2.631040in}{2.131104in}}%
\pgfpathlineto{\pgfqpoint{2.622516in}{2.124904in}}%
\pgfpathlineto{\pgfqpoint{2.613983in}{2.118809in}}%
\pgfpathlineto{\pgfqpoint{2.605439in}{2.112822in}}%
\pgfpathlineto{\pgfqpoint{2.596884in}{2.106943in}}%
\pgfpathclose%
\pgfusepath{fill}%
\end{pgfscope}%
\begin{pgfscope}%
\pgfpathrectangle{\pgfqpoint{1.150000in}{0.150000in}}{\pgfqpoint{5.700000in}{5.700000in}}%
\pgfusepath{clip}%
\pgfsetbuttcap%
\pgfsetroundjoin%
\definecolor{currentfill}{rgb}{0.277941,0.056324,0.381191}%
\pgfsetfillcolor{currentfill}%
\pgfsetfillopacity{0.700000}%
\pgfsetlinewidth{0.000000pt}%
\definecolor{currentstroke}{rgb}{0.000000,0.000000,0.000000}%
\pgfsetstrokecolor{currentstroke}%
\pgfsetdash{}{0pt}%
\pgfpathmoveto{\pgfqpoint{3.677097in}{2.016299in}}%
\pgfpathlineto{\pgfqpoint{3.690758in}{2.013654in}}%
\pgfpathlineto{\pgfqpoint{3.704425in}{2.011090in}}%
\pgfpathlineto{\pgfqpoint{3.718099in}{2.008608in}}%
\pgfpathlineto{\pgfqpoint{3.731778in}{2.006207in}}%
\pgfpathlineto{\pgfqpoint{3.739850in}{2.015191in}}%
\pgfpathlineto{\pgfqpoint{3.747915in}{2.024161in}}%
\pgfpathlineto{\pgfqpoint{3.755975in}{2.033119in}}%
\pgfpathlineto{\pgfqpoint{3.764030in}{2.042064in}}%
\pgfpathlineto{\pgfqpoint{3.750361in}{2.044444in}}%
\pgfpathlineto{\pgfqpoint{3.736698in}{2.046906in}}%
\pgfpathlineto{\pgfqpoint{3.723043in}{2.049449in}}%
\pgfpathlineto{\pgfqpoint{3.709393in}{2.052073in}}%
\pgfpathlineto{\pgfqpoint{3.701328in}{2.043141in}}%
\pgfpathlineto{\pgfqpoint{3.693257in}{2.034202in}}%
\pgfpathlineto{\pgfqpoint{3.685180in}{2.025254in}}%
\pgfpathlineto{\pgfqpoint{3.677097in}{2.016299in}}%
\pgfpathclose%
\pgfusepath{fill}%
\end{pgfscope}%
\begin{pgfscope}%
\pgfpathrectangle{\pgfqpoint{1.150000in}{0.150000in}}{\pgfqpoint{5.700000in}{5.700000in}}%
\pgfusepath{clip}%
\pgfsetbuttcap%
\pgfsetroundjoin%
\definecolor{currentfill}{rgb}{0.212395,0.359683,0.551710}%
\pgfsetfillcolor{currentfill}%
\pgfsetfillopacity{0.700000}%
\pgfsetlinewidth{0.000000pt}%
\definecolor{currentstroke}{rgb}{0.000000,0.000000,0.000000}%
\pgfsetstrokecolor{currentstroke}%
\pgfsetdash{}{0pt}%
\pgfpathmoveto{\pgfqpoint{5.917324in}{2.665404in}}%
\pgfpathlineto{\pgfqpoint{5.931678in}{2.666467in}}%
\pgfpathlineto{\pgfqpoint{5.946044in}{2.667596in}}%
\pgfpathlineto{\pgfqpoint{5.960421in}{2.668791in}}%
\pgfpathlineto{\pgfqpoint{5.967595in}{2.674529in}}%
\pgfpathlineto{\pgfqpoint{5.974769in}{2.680447in}}%
\pgfpathlineto{\pgfqpoint{5.981945in}{2.686552in}}%
\pgfpathlineto{\pgfqpoint{5.989122in}{2.692853in}}%
\pgfpathlineto{\pgfqpoint{5.974774in}{2.692093in}}%
\pgfpathlineto{\pgfqpoint{5.960438in}{2.691399in}}%
\pgfpathlineto{\pgfqpoint{5.946114in}{2.690770in}}%
\pgfpathlineto{\pgfqpoint{5.938915in}{2.684138in}}%
\pgfpathlineto{\pgfqpoint{5.931717in}{2.677704in}}%
\pgfpathlineto{\pgfqpoint{5.924520in}{2.671462in}}%
\pgfpathlineto{\pgfqpoint{5.917324in}{2.665404in}}%
\pgfpathclose%
\pgfusepath{fill}%
\end{pgfscope}%
\begin{pgfscope}%
\pgfpathrectangle{\pgfqpoint{1.150000in}{0.150000in}}{\pgfqpoint{5.700000in}{5.700000in}}%
\pgfusepath{clip}%
\pgfsetbuttcap%
\pgfsetroundjoin%
\definecolor{currentfill}{rgb}{0.282327,0.094955,0.417331}%
\pgfsetfillcolor{currentfill}%
\pgfsetfillopacity{0.700000}%
\pgfsetlinewidth{0.000000pt}%
\definecolor{currentstroke}{rgb}{0.000000,0.000000,0.000000}%
\pgfsetstrokecolor{currentstroke}%
\pgfsetdash{}{0pt}%
\pgfpathmoveto{\pgfqpoint{3.992559in}{2.090623in}}%
\pgfpathlineto{\pgfqpoint{4.006297in}{2.089251in}}%
\pgfpathlineto{\pgfqpoint{4.020043in}{2.087957in}}%
\pgfpathlineto{\pgfqpoint{4.033797in}{2.086740in}}%
\pgfpathlineto{\pgfqpoint{4.047558in}{2.085600in}}%
\pgfpathlineto{\pgfqpoint{4.055518in}{2.094319in}}%
\pgfpathlineto{\pgfqpoint{4.063472in}{2.103011in}}%
\pgfpathlineto{\pgfqpoint{4.071421in}{2.111677in}}%
\pgfpathlineto{\pgfqpoint{4.079363in}{2.120319in}}%
\pgfpathlineto{\pgfqpoint{4.065613in}{2.121500in}}%
\pgfpathlineto{\pgfqpoint{4.051870in}{2.122758in}}%
\pgfpathlineto{\pgfqpoint{4.038135in}{2.124093in}}%
\pgfpathlineto{\pgfqpoint{4.024407in}{2.125506in}}%
\pgfpathlineto{\pgfqpoint{4.016454in}{2.116815in}}%
\pgfpathlineto{\pgfqpoint{4.008494in}{2.108106in}}%
\pgfpathlineto{\pgfqpoint{4.000529in}{2.099375in}}%
\pgfpathlineto{\pgfqpoint{3.992559in}{2.090623in}}%
\pgfpathclose%
\pgfusepath{fill}%
\end{pgfscope}%
\begin{pgfscope}%
\pgfpathrectangle{\pgfqpoint{1.150000in}{0.150000in}}{\pgfqpoint{5.700000in}{5.700000in}}%
\pgfusepath{clip}%
\pgfsetbuttcap%
\pgfsetroundjoin%
\definecolor{currentfill}{rgb}{0.269308,0.218818,0.509577}%
\pgfsetfillcolor{currentfill}%
\pgfsetfillopacity{0.700000}%
\pgfsetlinewidth{0.000000pt}%
\definecolor{currentstroke}{rgb}{0.000000,0.000000,0.000000}%
\pgfsetstrokecolor{currentstroke}%
\pgfsetdash{}{0pt}%
\pgfpathmoveto{\pgfqpoint{4.797070in}{2.335884in}}%
\pgfpathlineto{\pgfqpoint{4.811059in}{2.336557in}}%
\pgfpathlineto{\pgfqpoint{4.825058in}{2.337301in}}%
\pgfpathlineto{\pgfqpoint{4.839067in}{2.338116in}}%
\pgfpathlineto{\pgfqpoint{4.853085in}{2.339002in}}%
\pgfpathlineto{\pgfqpoint{4.860730in}{2.345850in}}%
\pgfpathlineto{\pgfqpoint{4.868370in}{2.352699in}}%
\pgfpathlineto{\pgfqpoint{4.876003in}{2.359551in}}%
\pgfpathlineto{\pgfqpoint{4.883630in}{2.366411in}}%
\pgfpathlineto{\pgfqpoint{4.869627in}{2.365733in}}%
\pgfpathlineto{\pgfqpoint{4.855634in}{2.365126in}}%
\pgfpathlineto{\pgfqpoint{4.841650in}{2.364589in}}%
\pgfpathlineto{\pgfqpoint{4.827676in}{2.364123in}}%
\pgfpathlineto{\pgfqpoint{4.820033in}{2.357049in}}%
\pgfpathlineto{\pgfqpoint{4.812384in}{2.349986in}}%
\pgfpathlineto{\pgfqpoint{4.804730in}{2.342933in}}%
\pgfpathlineto{\pgfqpoint{4.797070in}{2.335884in}}%
\pgfpathclose%
\pgfusepath{fill}%
\end{pgfscope}%
\begin{pgfscope}%
\pgfpathrectangle{\pgfqpoint{1.150000in}{0.150000in}}{\pgfqpoint{5.700000in}{5.700000in}}%
\pgfusepath{clip}%
\pgfsetbuttcap%
\pgfsetroundjoin%
\definecolor{currentfill}{rgb}{0.250425,0.274290,0.533103}%
\pgfsetfillcolor{currentfill}%
\pgfsetfillopacity{0.700000}%
\pgfsetlinewidth{0.000000pt}%
\definecolor{currentstroke}{rgb}{0.000000,0.000000,0.000000}%
\pgfsetstrokecolor{currentstroke}%
\pgfsetdash{}{0pt}%
\pgfpathmoveto{\pgfqpoint{5.199343in}{2.459249in}}%
\pgfpathlineto{\pgfqpoint{5.213468in}{2.460363in}}%
\pgfpathlineto{\pgfqpoint{5.227603in}{2.461545in}}%
\pgfpathlineto{\pgfqpoint{5.241749in}{2.462796in}}%
\pgfpathlineto{\pgfqpoint{5.255905in}{2.464115in}}%
\pgfpathlineto{\pgfqpoint{5.263373in}{2.470054in}}%
\pgfpathlineto{\pgfqpoint{5.270836in}{2.476038in}}%
\pgfpathlineto{\pgfqpoint{5.278294in}{2.482073in}}%
\pgfpathlineto{\pgfqpoint{5.285747in}{2.488165in}}%
\pgfpathlineto{\pgfqpoint{5.271610in}{2.487137in}}%
\pgfpathlineto{\pgfqpoint{5.257484in}{2.486177in}}%
\pgfpathlineto{\pgfqpoint{5.243369in}{2.485285in}}%
\pgfpathlineto{\pgfqpoint{5.229264in}{2.484462in}}%
\pgfpathlineto{\pgfqpoint{5.221791in}{2.478073in}}%
\pgfpathlineto{\pgfqpoint{5.214313in}{2.471744in}}%
\pgfpathlineto{\pgfqpoint{5.206830in}{2.465471in}}%
\pgfpathlineto{\pgfqpoint{5.199343in}{2.459249in}}%
\pgfpathclose%
\pgfusepath{fill}%
\end{pgfscope}%
\begin{pgfscope}%
\pgfpathrectangle{\pgfqpoint{1.150000in}{0.150000in}}{\pgfqpoint{5.700000in}{5.700000in}}%
\pgfusepath{clip}%
\pgfsetbuttcap%
\pgfsetroundjoin%
\definecolor{currentfill}{rgb}{0.272594,0.025563,0.353093}%
\pgfsetfillcolor{currentfill}%
\pgfsetfillopacity{0.700000}%
\pgfsetlinewidth{0.000000pt}%
\definecolor{currentstroke}{rgb}{0.000000,0.000000,0.000000}%
\pgfsetstrokecolor{currentstroke}%
\pgfsetdash{}{0pt}%
\pgfpathmoveto{\pgfqpoint{3.219622in}{1.967900in}}%
\pgfpathlineto{\pgfqpoint{3.233208in}{1.962903in}}%
\pgfpathlineto{\pgfqpoint{3.246797in}{1.957996in}}%
\pgfpathlineto{\pgfqpoint{3.260391in}{1.953179in}}%
\pgfpathlineto{\pgfqpoint{3.273989in}{1.948452in}}%
\pgfpathlineto{\pgfqpoint{3.282231in}{1.956933in}}%
\pgfpathlineto{\pgfqpoint{3.290467in}{1.965441in}}%
\pgfpathlineto{\pgfqpoint{3.298696in}{1.973974in}}%
\pgfpathlineto{\pgfqpoint{3.306918in}{1.982533in}}%
\pgfpathlineto{\pgfqpoint{3.293334in}{1.987157in}}%
\pgfpathlineto{\pgfqpoint{3.279754in}{1.991871in}}%
\pgfpathlineto{\pgfqpoint{3.266179in}{1.996675in}}%
\pgfpathlineto{\pgfqpoint{3.252608in}{2.001570in}}%
\pgfpathlineto{\pgfqpoint{3.244371in}{1.993107in}}%
\pgfpathlineto{\pgfqpoint{3.236128in}{1.984673in}}%
\pgfpathlineto{\pgfqpoint{3.227879in}{1.976271in}}%
\pgfpathlineto{\pgfqpoint{3.219622in}{1.967900in}}%
\pgfpathclose%
\pgfusepath{fill}%
\end{pgfscope}%
\begin{pgfscope}%
\pgfpathrectangle{\pgfqpoint{1.150000in}{0.150000in}}{\pgfqpoint{5.700000in}{5.700000in}}%
\pgfusepath{clip}%
\pgfsetbuttcap%
\pgfsetroundjoin%
\definecolor{currentfill}{rgb}{0.277941,0.056324,0.381191}%
\pgfsetfillcolor{currentfill}%
\pgfsetfillopacity{0.700000}%
\pgfsetlinewidth{0.000000pt}%
\definecolor{currentstroke}{rgb}{0.000000,0.000000,0.000000}%
\pgfsetstrokecolor{currentstroke}%
\pgfsetdash{}{0pt}%
\pgfpathmoveto{\pgfqpoint{2.793753in}{2.029415in}}%
\pgfpathlineto{\pgfqpoint{2.807320in}{2.021644in}}%
\pgfpathlineto{\pgfqpoint{2.820889in}{2.013978in}}%
\pgfpathlineto{\pgfqpoint{2.834459in}{2.006416in}}%
\pgfpathlineto{\pgfqpoint{2.848031in}{1.998957in}}%
\pgfpathlineto{\pgfqpoint{2.856462in}{2.005954in}}%
\pgfpathlineto{\pgfqpoint{2.864885in}{2.013030in}}%
\pgfpathlineto{\pgfqpoint{2.873299in}{2.020183in}}%
\pgfpathlineto{\pgfqpoint{2.881705in}{2.027411in}}%
\pgfpathlineto{\pgfqpoint{2.868152in}{2.034704in}}%
\pgfpathlineto{\pgfqpoint{2.854601in}{2.042101in}}%
\pgfpathlineto{\pgfqpoint{2.841052in}{2.049601in}}%
\pgfpathlineto{\pgfqpoint{2.827505in}{2.057206in}}%
\pgfpathlineto{\pgfqpoint{2.819080in}{2.050137in}}%
\pgfpathlineto{\pgfqpoint{2.810647in}{2.043147in}}%
\pgfpathlineto{\pgfqpoint{2.802204in}{2.036239in}}%
\pgfpathlineto{\pgfqpoint{2.793753in}{2.029415in}}%
\pgfpathclose%
\pgfusepath{fill}%
\end{pgfscope}%
\begin{pgfscope}%
\pgfpathrectangle{\pgfqpoint{1.150000in}{0.150000in}}{\pgfqpoint{5.700000in}{5.700000in}}%
\pgfusepath{clip}%
\pgfsetbuttcap%
\pgfsetroundjoin%
\definecolor{currentfill}{rgb}{0.282290,0.145912,0.461510}%
\pgfsetfillcolor{currentfill}%
\pgfsetfillopacity{0.700000}%
\pgfsetlinewidth{0.000000pt}%
\definecolor{currentstroke}{rgb}{0.000000,0.000000,0.000000}%
\pgfsetstrokecolor{currentstroke}%
\pgfsetdash{}{0pt}%
\pgfpathmoveto{\pgfqpoint{4.308061in}{2.178773in}}%
\pgfpathlineto{\pgfqpoint{4.321895in}{2.178430in}}%
\pgfpathlineto{\pgfqpoint{4.335737in}{2.178162in}}%
\pgfpathlineto{\pgfqpoint{4.349588in}{2.177968in}}%
\pgfpathlineto{\pgfqpoint{4.363448in}{2.177847in}}%
\pgfpathlineto{\pgfqpoint{4.371293in}{2.185956in}}%
\pgfpathlineto{\pgfqpoint{4.379133in}{2.194036in}}%
\pgfpathlineto{\pgfqpoint{4.386967in}{2.202090in}}%
\pgfpathlineto{\pgfqpoint{4.394794in}{2.210120in}}%
\pgfpathlineto{\pgfqpoint{4.380946in}{2.210344in}}%
\pgfpathlineto{\pgfqpoint{4.367107in}{2.210641in}}%
\pgfpathlineto{\pgfqpoint{4.353276in}{2.211013in}}%
\pgfpathlineto{\pgfqpoint{4.339454in}{2.211459in}}%
\pgfpathlineto{\pgfqpoint{4.331615in}{2.203318in}}%
\pgfpathlineto{\pgfqpoint{4.323769in}{2.195158in}}%
\pgfpathlineto{\pgfqpoint{4.315918in}{2.186977in}}%
\pgfpathlineto{\pgfqpoint{4.308061in}{2.178773in}}%
\pgfpathclose%
\pgfusepath{fill}%
\end{pgfscope}%
\begin{pgfscope}%
\pgfpathrectangle{\pgfqpoint{1.150000in}{0.150000in}}{\pgfqpoint{5.700000in}{5.700000in}}%
\pgfusepath{clip}%
\pgfsetbuttcap%
\pgfsetroundjoin%
\definecolor{currentfill}{rgb}{0.282290,0.145912,0.461510}%
\pgfsetfillcolor{currentfill}%
\pgfsetfillopacity{0.700000}%
\pgfsetlinewidth{0.000000pt}%
\definecolor{currentstroke}{rgb}{0.000000,0.000000,0.000000}%
\pgfsetstrokecolor{currentstroke}%
\pgfsetdash{}{0pt}%
\pgfpathmoveto{\pgfqpoint{2.399298in}{2.207634in}}%
\pgfpathlineto{\pgfqpoint{2.412909in}{2.196706in}}%
\pgfpathlineto{\pgfqpoint{2.426519in}{2.185903in}}%
\pgfpathlineto{\pgfqpoint{2.440127in}{2.175226in}}%
\pgfpathlineto{\pgfqpoint{2.453734in}{2.164672in}}%
\pgfpathlineto{\pgfqpoint{2.462381in}{2.169649in}}%
\pgfpathlineto{\pgfqpoint{2.471017in}{2.174758in}}%
\pgfpathlineto{\pgfqpoint{2.479641in}{2.179995in}}%
\pgfpathlineto{\pgfqpoint{2.488254in}{2.185358in}}%
\pgfpathlineto{\pgfqpoint{2.474672in}{2.195702in}}%
\pgfpathlineto{\pgfqpoint{2.461089in}{2.206169in}}%
\pgfpathlineto{\pgfqpoint{2.447506in}{2.216761in}}%
\pgfpathlineto{\pgfqpoint{2.433921in}{2.227478in}}%
\pgfpathlineto{\pgfqpoint{2.425283in}{2.222318in}}%
\pgfpathlineto{\pgfqpoint{2.416633in}{2.217289in}}%
\pgfpathlineto{\pgfqpoint{2.407972in}{2.212393in}}%
\pgfpathlineto{\pgfqpoint{2.399298in}{2.207634in}}%
\pgfpathclose%
\pgfusepath{fill}%
\end{pgfscope}%
\begin{pgfscope}%
\pgfpathrectangle{\pgfqpoint{1.150000in}{0.150000in}}{\pgfqpoint{5.700000in}{5.700000in}}%
\pgfusepath{clip}%
\pgfsetbuttcap%
\pgfsetroundjoin%
\definecolor{currentfill}{rgb}{0.233603,0.313828,0.543914}%
\pgfsetfillcolor{currentfill}%
\pgfsetfillopacity{0.700000}%
\pgfsetlinewidth{0.000000pt}%
\definecolor{currentstroke}{rgb}{0.000000,0.000000,0.000000}%
\pgfsetstrokecolor{currentstroke}%
\pgfsetdash{}{0pt}%
\pgfpathmoveto{\pgfqpoint{5.515186in}{2.549512in}}%
\pgfpathlineto{\pgfqpoint{5.529419in}{2.550779in}}%
\pgfpathlineto{\pgfqpoint{5.543664in}{2.552114in}}%
\pgfpathlineto{\pgfqpoint{5.557920in}{2.553516in}}%
\pgfpathlineto{\pgfqpoint{5.572187in}{2.554985in}}%
\pgfpathlineto{\pgfqpoint{5.579515in}{2.560454in}}%
\pgfpathlineto{\pgfqpoint{5.586840in}{2.566018in}}%
\pgfpathlineto{\pgfqpoint{5.594161in}{2.571681in}}%
\pgfpathlineto{\pgfqpoint{5.601480in}{2.577451in}}%
\pgfpathlineto{\pgfqpoint{5.587237in}{2.576336in}}%
\pgfpathlineto{\pgfqpoint{5.573005in}{2.575287in}}%
\pgfpathlineto{\pgfqpoint{5.558784in}{2.574306in}}%
\pgfpathlineto{\pgfqpoint{5.544574in}{2.573391in}}%
\pgfpathlineto{\pgfqpoint{5.537232in}{2.567261in}}%
\pgfpathlineto{\pgfqpoint{5.529886in}{2.561242in}}%
\pgfpathlineto{\pgfqpoint{5.522537in}{2.555328in}}%
\pgfpathlineto{\pgfqpoint{5.515186in}{2.549512in}}%
\pgfpathclose%
\pgfusepath{fill}%
\end{pgfscope}%
\begin{pgfscope}%
\pgfpathrectangle{\pgfqpoint{1.150000in}{0.150000in}}{\pgfqpoint{5.700000in}{5.700000in}}%
\pgfusepath{clip}%
\pgfsetbuttcap%
\pgfsetroundjoin%
\definecolor{currentfill}{rgb}{0.273006,0.204520,0.501721}%
\pgfsetfillcolor{currentfill}%
\pgfsetfillopacity{0.700000}%
\pgfsetlinewidth{0.000000pt}%
\definecolor{currentstroke}{rgb}{0.000000,0.000000,0.000000}%
\pgfsetstrokecolor{currentstroke}%
\pgfsetdash{}{0pt}%
\pgfpathmoveto{\pgfqpoint{4.710451in}{2.304934in}}%
\pgfpathlineto{\pgfqpoint{4.724416in}{2.305509in}}%
\pgfpathlineto{\pgfqpoint{4.738391in}{2.306156in}}%
\pgfpathlineto{\pgfqpoint{4.752375in}{2.306874in}}%
\pgfpathlineto{\pgfqpoint{4.766369in}{2.307663in}}%
\pgfpathlineto{\pgfqpoint{4.774053in}{2.314729in}}%
\pgfpathlineto{\pgfqpoint{4.781731in}{2.321786in}}%
\pgfpathlineto{\pgfqpoint{4.789403in}{2.328836in}}%
\pgfpathlineto{\pgfqpoint{4.797070in}{2.335884in}}%
\pgfpathlineto{\pgfqpoint{4.783090in}{2.335282in}}%
\pgfpathlineto{\pgfqpoint{4.769121in}{2.334750in}}%
\pgfpathlineto{\pgfqpoint{4.755160in}{2.334290in}}%
\pgfpathlineto{\pgfqpoint{4.741210in}{2.333901in}}%
\pgfpathlineto{\pgfqpoint{4.733529in}{2.326659in}}%
\pgfpathlineto{\pgfqpoint{4.725842in}{2.319420in}}%
\pgfpathlineto{\pgfqpoint{4.718150in}{2.312179in}}%
\pgfpathlineto{\pgfqpoint{4.710451in}{2.304934in}}%
\pgfpathclose%
\pgfusepath{fill}%
\end{pgfscope}%
\begin{pgfscope}%
\pgfpathrectangle{\pgfqpoint{1.150000in}{0.150000in}}{\pgfqpoint{5.700000in}{5.700000in}}%
\pgfusepath{clip}%
\pgfsetbuttcap%
\pgfsetroundjoin%
\definecolor{currentfill}{rgb}{0.281446,0.084320,0.407414}%
\pgfsetfillcolor{currentfill}%
\pgfsetfillopacity{0.700000}%
\pgfsetlinewidth{0.000000pt}%
\definecolor{currentstroke}{rgb}{0.000000,0.000000,0.000000}%
\pgfsetstrokecolor{currentstroke}%
\pgfsetdash{}{0pt}%
\pgfpathmoveto{\pgfqpoint{3.905696in}{2.061563in}}%
\pgfpathlineto{\pgfqpoint{3.919416in}{2.059899in}}%
\pgfpathlineto{\pgfqpoint{3.933143in}{2.058314in}}%
\pgfpathlineto{\pgfqpoint{3.946878in}{2.056808in}}%
\pgfpathlineto{\pgfqpoint{3.960619in}{2.055379in}}%
\pgfpathlineto{\pgfqpoint{3.968613in}{2.064227in}}%
\pgfpathlineto{\pgfqpoint{3.976600in}{2.073050in}}%
\pgfpathlineto{\pgfqpoint{3.984582in}{2.081848in}}%
\pgfpathlineto{\pgfqpoint{3.992559in}{2.090623in}}%
\pgfpathlineto{\pgfqpoint{3.978828in}{2.092072in}}%
\pgfpathlineto{\pgfqpoint{3.965104in}{2.093600in}}%
\pgfpathlineto{\pgfqpoint{3.951388in}{2.095205in}}%
\pgfpathlineto{\pgfqpoint{3.937679in}{2.096889in}}%
\pgfpathlineto{\pgfqpoint{3.929692in}{2.088086in}}%
\pgfpathlineto{\pgfqpoint{3.921699in}{2.079264in}}%
\pgfpathlineto{\pgfqpoint{3.913700in}{2.070424in}}%
\pgfpathlineto{\pgfqpoint{3.905696in}{2.061563in}}%
\pgfpathclose%
\pgfusepath{fill}%
\end{pgfscope}%
\begin{pgfscope}%
\pgfpathrectangle{\pgfqpoint{1.150000in}{0.150000in}}{\pgfqpoint{5.700000in}{5.700000in}}%
\pgfusepath{clip}%
\pgfsetbuttcap%
\pgfsetroundjoin%
\definecolor{currentfill}{rgb}{0.255645,0.260703,0.528312}%
\pgfsetfillcolor{currentfill}%
\pgfsetfillopacity{0.700000}%
\pgfsetlinewidth{0.000000pt}%
\definecolor{currentstroke}{rgb}{0.000000,0.000000,0.000000}%
\pgfsetstrokecolor{currentstroke}%
\pgfsetdash{}{0pt}%
\pgfpathmoveto{\pgfqpoint{5.112874in}{2.429924in}}%
\pgfpathlineto{\pgfqpoint{5.126975in}{2.431032in}}%
\pgfpathlineto{\pgfqpoint{5.141087in}{2.432209in}}%
\pgfpathlineto{\pgfqpoint{5.155210in}{2.433455in}}%
\pgfpathlineto{\pgfqpoint{5.169343in}{2.434770in}}%
\pgfpathlineto{\pgfqpoint{5.176851in}{2.440838in}}%
\pgfpathlineto{\pgfqpoint{5.184353in}{2.446938in}}%
\pgfpathlineto{\pgfqpoint{5.191851in}{2.453073in}}%
\pgfpathlineto{\pgfqpoint{5.199343in}{2.459249in}}%
\pgfpathlineto{\pgfqpoint{5.185229in}{2.458205in}}%
\pgfpathlineto{\pgfqpoint{5.171125in}{2.457229in}}%
\pgfpathlineto{\pgfqpoint{5.157032in}{2.456322in}}%
\pgfpathlineto{\pgfqpoint{5.142949in}{2.455484in}}%
\pgfpathlineto{\pgfqpoint{5.135438in}{2.449031in}}%
\pgfpathlineto{\pgfqpoint{5.127922in}{2.442623in}}%
\pgfpathlineto{\pgfqpoint{5.120400in}{2.436255in}}%
\pgfpathlineto{\pgfqpoint{5.112874in}{2.429924in}}%
\pgfpathclose%
\pgfusepath{fill}%
\end{pgfscope}%
\begin{pgfscope}%
\pgfpathrectangle{\pgfqpoint{1.150000in}{0.150000in}}{\pgfqpoint{5.700000in}{5.700000in}}%
\pgfusepath{clip}%
\pgfsetbuttcap%
\pgfsetroundjoin%
\definecolor{currentfill}{rgb}{0.272594,0.025563,0.353093}%
\pgfsetfillcolor{currentfill}%
\pgfsetfillopacity{0.700000}%
\pgfsetlinewidth{0.000000pt}%
\definecolor{currentstroke}{rgb}{0.000000,0.000000,0.000000}%
\pgfsetstrokecolor{currentstroke}%
\pgfsetdash{}{0pt}%
\pgfpathmoveto{\pgfqpoint{3.361301in}{1.964925in}}%
\pgfpathlineto{\pgfqpoint{3.374909in}{1.960743in}}%
\pgfpathlineto{\pgfqpoint{3.388521in}{1.956649in}}%
\pgfpathlineto{\pgfqpoint{3.402139in}{1.952641in}}%
\pgfpathlineto{\pgfqpoint{3.415761in}{1.948720in}}%
\pgfpathlineto{\pgfqpoint{3.423951in}{1.957481in}}%
\pgfpathlineto{\pgfqpoint{3.432134in}{1.966256in}}%
\pgfpathlineto{\pgfqpoint{3.440311in}{1.975041in}}%
\pgfpathlineto{\pgfqpoint{3.448482in}{1.983837in}}%
\pgfpathlineto{\pgfqpoint{3.434872in}{1.987677in}}%
\pgfpathlineto{\pgfqpoint{3.421268in}{1.991602in}}%
\pgfpathlineto{\pgfqpoint{3.407668in}{1.995614in}}%
\pgfpathlineto{\pgfqpoint{3.394074in}{1.999714in}}%
\pgfpathlineto{\pgfqpoint{3.385890in}{1.990992in}}%
\pgfpathlineto{\pgfqpoint{3.377700in}{1.982286in}}%
\pgfpathlineto{\pgfqpoint{3.369503in}{1.973597in}}%
\pgfpathlineto{\pgfqpoint{3.361301in}{1.964925in}}%
\pgfpathclose%
\pgfusepath{fill}%
\end{pgfscope}%
\begin{pgfscope}%
\pgfpathrectangle{\pgfqpoint{1.150000in}{0.150000in}}{\pgfqpoint{5.700000in}{5.700000in}}%
\pgfusepath{clip}%
\pgfsetbuttcap%
\pgfsetroundjoin%
\definecolor{currentfill}{rgb}{0.276022,0.044167,0.370164}%
\pgfsetfillcolor{currentfill}%
\pgfsetfillopacity{0.700000}%
\pgfsetlinewidth{0.000000pt}%
\definecolor{currentstroke}{rgb}{0.000000,0.000000,0.000000}%
\pgfsetstrokecolor{currentstroke}%
\pgfsetdash{}{0pt}%
\pgfpathmoveto{\pgfqpoint{3.590082in}{1.991964in}}%
\pgfpathlineto{\pgfqpoint{3.603730in}{1.988948in}}%
\pgfpathlineto{\pgfqpoint{3.617384in}{1.986015in}}%
\pgfpathlineto{\pgfqpoint{3.631044in}{1.983165in}}%
\pgfpathlineto{\pgfqpoint{3.644710in}{1.980396in}}%
\pgfpathlineto{\pgfqpoint{3.652815in}{1.989384in}}%
\pgfpathlineto{\pgfqpoint{3.660915in}{1.998363in}}%
\pgfpathlineto{\pgfqpoint{3.669009in}{2.007335in}}%
\pgfpathlineto{\pgfqpoint{3.677097in}{2.016299in}}%
\pgfpathlineto{\pgfqpoint{3.663443in}{2.019026in}}%
\pgfpathlineto{\pgfqpoint{3.649794in}{2.021835in}}%
\pgfpathlineto{\pgfqpoint{3.636152in}{2.024727in}}%
\pgfpathlineto{\pgfqpoint{3.622516in}{2.027702in}}%
\pgfpathlineto{\pgfqpoint{3.614416in}{2.018772in}}%
\pgfpathlineto{\pgfqpoint{3.606311in}{2.009839in}}%
\pgfpathlineto{\pgfqpoint{3.598199in}{2.000903in}}%
\pgfpathlineto{\pgfqpoint{3.590082in}{1.991964in}}%
\pgfpathclose%
\pgfusepath{fill}%
\end{pgfscope}%
\begin{pgfscope}%
\pgfpathrectangle{\pgfqpoint{1.150000in}{0.150000in}}{\pgfqpoint{5.700000in}{5.700000in}}%
\pgfusepath{clip}%
\pgfsetbuttcap%
\pgfsetroundjoin%
\definecolor{currentfill}{rgb}{0.216210,0.351535,0.550627}%
\pgfsetfillcolor{currentfill}%
\pgfsetfillopacity{0.700000}%
\pgfsetlinewidth{0.000000pt}%
\definecolor{currentstroke}{rgb}{0.000000,0.000000,0.000000}%
\pgfsetstrokecolor{currentstroke}%
\pgfsetdash{}{0pt}%
\pgfpathmoveto{\pgfqpoint{5.831125in}{2.637607in}}%
\pgfpathlineto{\pgfqpoint{5.845462in}{2.638821in}}%
\pgfpathlineto{\pgfqpoint{5.859811in}{2.640101in}}%
\pgfpathlineto{\pgfqpoint{5.874171in}{2.641448in}}%
\pgfpathlineto{\pgfqpoint{5.888542in}{2.642860in}}%
\pgfpathlineto{\pgfqpoint{5.895738in}{2.648257in}}%
\pgfpathlineto{\pgfqpoint{5.902934in}{2.653809in}}%
\pgfpathlineto{\pgfqpoint{5.910129in}{2.659522in}}%
\pgfpathlineto{\pgfqpoint{5.917324in}{2.665404in}}%
\pgfpathlineto{\pgfqpoint{5.902982in}{2.664406in}}%
\pgfpathlineto{\pgfqpoint{5.888650in}{2.663475in}}%
\pgfpathlineto{\pgfqpoint{5.874330in}{2.662609in}}%
\pgfpathlineto{\pgfqpoint{5.860022in}{2.661810in}}%
\pgfpathlineto{\pgfqpoint{5.852798in}{2.655506in}}%
\pgfpathlineto{\pgfqpoint{5.845574in}{2.649375in}}%
\pgfpathlineto{\pgfqpoint{5.838350in}{2.643411in}}%
\pgfpathlineto{\pgfqpoint{5.831125in}{2.637607in}}%
\pgfpathclose%
\pgfusepath{fill}%
\end{pgfscope}%
\begin{pgfscope}%
\pgfpathrectangle{\pgfqpoint{1.150000in}{0.150000in}}{\pgfqpoint{5.700000in}{5.700000in}}%
\pgfusepath{clip}%
\pgfsetbuttcap%
\pgfsetroundjoin%
\definecolor{currentfill}{rgb}{0.281446,0.084320,0.407414}%
\pgfsetfillcolor{currentfill}%
\pgfsetfillopacity{0.700000}%
\pgfsetlinewidth{0.000000pt}%
\definecolor{currentstroke}{rgb}{0.000000,0.000000,0.000000}%
\pgfsetstrokecolor{currentstroke}%
\pgfsetdash{}{0pt}%
\pgfpathmoveto{\pgfqpoint{2.651198in}{2.070528in}}%
\pgfpathlineto{\pgfqpoint{2.664778in}{2.061706in}}%
\pgfpathlineto{\pgfqpoint{2.678358in}{2.052994in}}%
\pgfpathlineto{\pgfqpoint{2.691939in}{2.044393in}}%
\pgfpathlineto{\pgfqpoint{2.705520in}{2.035900in}}%
\pgfpathlineto{\pgfqpoint{2.714029in}{2.042169in}}%
\pgfpathlineto{\pgfqpoint{2.722529in}{2.048537in}}%
\pgfpathlineto{\pgfqpoint{2.731019in}{2.055001in}}%
\pgfpathlineto{\pgfqpoint{2.739500in}{2.061560in}}%
\pgfpathlineto{\pgfqpoint{2.725940in}{2.069865in}}%
\pgfpathlineto{\pgfqpoint{2.712380in}{2.078279in}}%
\pgfpathlineto{\pgfqpoint{2.698822in}{2.086804in}}%
\pgfpathlineto{\pgfqpoint{2.685264in}{2.095439in}}%
\pgfpathlineto{\pgfqpoint{2.676763in}{2.089060in}}%
\pgfpathlineto{\pgfqpoint{2.668251in}{2.082780in}}%
\pgfpathlineto{\pgfqpoint{2.659729in}{2.076602in}}%
\pgfpathlineto{\pgfqpoint{2.651198in}{2.070528in}}%
\pgfpathclose%
\pgfusepath{fill}%
\end{pgfscope}%
\begin{pgfscope}%
\pgfpathrectangle{\pgfqpoint{1.150000in}{0.150000in}}{\pgfqpoint{5.700000in}{5.700000in}}%
\pgfusepath{clip}%
\pgfsetbuttcap%
\pgfsetroundjoin%
\definecolor{currentfill}{rgb}{0.283072,0.130895,0.449241}%
\pgfsetfillcolor{currentfill}%
\pgfsetfillopacity{0.700000}%
\pgfsetlinewidth{0.000000pt}%
\definecolor{currentstroke}{rgb}{0.000000,0.000000,0.000000}%
\pgfsetstrokecolor{currentstroke}%
\pgfsetdash{}{0pt}%
\pgfpathmoveto{\pgfqpoint{4.221277in}{2.147472in}}%
\pgfpathlineto{\pgfqpoint{4.235089in}{2.146913in}}%
\pgfpathlineto{\pgfqpoint{4.248909in}{2.146429in}}%
\pgfpathlineto{\pgfqpoint{4.262737in}{2.146020in}}%
\pgfpathlineto{\pgfqpoint{4.276574in}{2.145686in}}%
\pgfpathlineto{\pgfqpoint{4.284454in}{2.154002in}}%
\pgfpathlineto{\pgfqpoint{4.292329in}{2.162287in}}%
\pgfpathlineto{\pgfqpoint{4.300198in}{2.170544in}}%
\pgfpathlineto{\pgfqpoint{4.308061in}{2.178773in}}%
\pgfpathlineto{\pgfqpoint{4.294236in}{2.179190in}}%
\pgfpathlineto{\pgfqpoint{4.280419in}{2.179682in}}%
\pgfpathlineto{\pgfqpoint{4.266610in}{2.180249in}}%
\pgfpathlineto{\pgfqpoint{4.252810in}{2.180890in}}%
\pgfpathlineto{\pgfqpoint{4.244935in}{2.172571in}}%
\pgfpathlineto{\pgfqpoint{4.237055in}{2.164229in}}%
\pgfpathlineto{\pgfqpoint{4.229169in}{2.155864in}}%
\pgfpathlineto{\pgfqpoint{4.221277in}{2.147472in}}%
\pgfpathclose%
\pgfusepath{fill}%
\end{pgfscope}%
\begin{pgfscope}%
\pgfpathrectangle{\pgfqpoint{1.150000in}{0.150000in}}{\pgfqpoint{5.700000in}{5.700000in}}%
\pgfusepath{clip}%
\pgfsetbuttcap%
\pgfsetroundjoin%
\definecolor{currentfill}{rgb}{0.275191,0.194905,0.496005}%
\pgfsetfillcolor{currentfill}%
\pgfsetfillopacity{0.700000}%
\pgfsetlinewidth{0.000000pt}%
\definecolor{currentstroke}{rgb}{0.000000,0.000000,0.000000}%
\pgfsetstrokecolor{currentstroke}%
\pgfsetdash{}{0pt}%
\pgfpathmoveto{\pgfqpoint{4.623778in}{2.273591in}}%
\pgfpathlineto{\pgfqpoint{4.637719in}{2.274045in}}%
\pgfpathlineto{\pgfqpoint{4.651669in}{2.274572in}}%
\pgfpathlineto{\pgfqpoint{4.665628in}{2.275170in}}%
\pgfpathlineto{\pgfqpoint{4.679597in}{2.275840in}}%
\pgfpathlineto{\pgfqpoint{4.687320in}{2.283137in}}%
\pgfpathlineto{\pgfqpoint{4.695037in}{2.290416in}}%
\pgfpathlineto{\pgfqpoint{4.702747in}{2.297680in}}%
\pgfpathlineto{\pgfqpoint{4.710451in}{2.304934in}}%
\pgfpathlineto{\pgfqpoint{4.696496in}{2.304430in}}%
\pgfpathlineto{\pgfqpoint{4.682550in}{2.303998in}}%
\pgfpathlineto{\pgfqpoint{4.668614in}{2.303637in}}%
\pgfpathlineto{\pgfqpoint{4.654687in}{2.303348in}}%
\pgfpathlineto{\pgfqpoint{4.646969in}{2.295921in}}%
\pgfpathlineto{\pgfqpoint{4.639244in}{2.288488in}}%
\pgfpathlineto{\pgfqpoint{4.631514in}{2.281046in}}%
\pgfpathlineto{\pgfqpoint{4.623778in}{2.273591in}}%
\pgfpathclose%
\pgfusepath{fill}%
\end{pgfscope}%
\begin{pgfscope}%
\pgfpathrectangle{\pgfqpoint{1.150000in}{0.150000in}}{\pgfqpoint{5.700000in}{5.700000in}}%
\pgfusepath{clip}%
\pgfsetbuttcap%
\pgfsetroundjoin%
\definecolor{currentfill}{rgb}{0.237441,0.305202,0.541921}%
\pgfsetfillcolor{currentfill}%
\pgfsetfillopacity{0.700000}%
\pgfsetlinewidth{0.000000pt}%
\definecolor{currentstroke}{rgb}{0.000000,0.000000,0.000000}%
\pgfsetstrokecolor{currentstroke}%
\pgfsetdash{}{0pt}%
\pgfpathmoveto{\pgfqpoint{5.428827in}{2.521392in}}%
\pgfpathlineto{\pgfqpoint{5.443039in}{2.522722in}}%
\pgfpathlineto{\pgfqpoint{5.457262in}{2.524119in}}%
\pgfpathlineto{\pgfqpoint{5.471497in}{2.525584in}}%
\pgfpathlineto{\pgfqpoint{5.485743in}{2.527117in}}%
\pgfpathlineto{\pgfqpoint{5.493109in}{2.532598in}}%
\pgfpathlineto{\pgfqpoint{5.500472in}{2.538153in}}%
\pgfpathlineto{\pgfqpoint{5.507831in}{2.543789in}}%
\pgfpathlineto{\pgfqpoint{5.515186in}{2.549512in}}%
\pgfpathlineto{\pgfqpoint{5.500963in}{2.548313in}}%
\pgfpathlineto{\pgfqpoint{5.486751in}{2.547181in}}%
\pgfpathlineto{\pgfqpoint{5.472551in}{2.546116in}}%
\pgfpathlineto{\pgfqpoint{5.458361in}{2.545119in}}%
\pgfpathlineto{\pgfqpoint{5.450983in}{2.539056in}}%
\pgfpathlineto{\pgfqpoint{5.443601in}{2.533085in}}%
\pgfpathlineto{\pgfqpoint{5.436216in}{2.527199in}}%
\pgfpathlineto{\pgfqpoint{5.428827in}{2.521392in}}%
\pgfpathclose%
\pgfusepath{fill}%
\end{pgfscope}%
\begin{pgfscope}%
\pgfpathrectangle{\pgfqpoint{1.150000in}{0.150000in}}{\pgfqpoint{5.700000in}{5.700000in}}%
\pgfusepath{clip}%
\pgfsetbuttcap%
\pgfsetroundjoin%
\definecolor{currentfill}{rgb}{0.273809,0.031497,0.358853}%
\pgfsetfillcolor{currentfill}%
\pgfsetfillopacity{0.700000}%
\pgfsetlinewidth{0.000000pt}%
\definecolor{currentstroke}{rgb}{0.000000,0.000000,0.000000}%
\pgfsetstrokecolor{currentstroke}%
\pgfsetdash{}{0pt}%
\pgfpathmoveto{\pgfqpoint{2.990212in}{1.972703in}}%
\pgfpathlineto{\pgfqpoint{3.003787in}{1.966310in}}%
\pgfpathlineto{\pgfqpoint{3.017365in}{1.960014in}}%
\pgfpathlineto{\pgfqpoint{3.030946in}{1.953814in}}%
\pgfpathlineto{\pgfqpoint{3.044530in}{1.947710in}}%
\pgfpathlineto{\pgfqpoint{3.052875in}{1.955466in}}%
\pgfpathlineto{\pgfqpoint{3.061212in}{1.963278in}}%
\pgfpathlineto{\pgfqpoint{3.069541in}{1.971143in}}%
\pgfpathlineto{\pgfqpoint{3.077863in}{1.979059in}}%
\pgfpathlineto{\pgfqpoint{3.064296in}{1.985019in}}%
\pgfpathlineto{\pgfqpoint{3.050732in}{1.991074in}}%
\pgfpathlineto{\pgfqpoint{3.037171in}{1.997226in}}%
\pgfpathlineto{\pgfqpoint{3.023613in}{2.003475in}}%
\pgfpathlineto{\pgfqpoint{3.015274in}{1.995695in}}%
\pgfpathlineto{\pgfqpoint{3.006928in}{1.987972in}}%
\pgfpathlineto{\pgfqpoint{2.998574in}{1.980307in}}%
\pgfpathlineto{\pgfqpoint{2.990212in}{1.972703in}}%
\pgfpathclose%
\pgfusepath{fill}%
\end{pgfscope}%
\begin{pgfscope}%
\pgfpathrectangle{\pgfqpoint{1.150000in}{0.150000in}}{\pgfqpoint{5.700000in}{5.700000in}}%
\pgfusepath{clip}%
\pgfsetbuttcap%
\pgfsetroundjoin%
\definecolor{currentfill}{rgb}{0.280267,0.073417,0.397163}%
\pgfsetfillcolor{currentfill}%
\pgfsetfillopacity{0.700000}%
\pgfsetlinewidth{0.000000pt}%
\definecolor{currentstroke}{rgb}{0.000000,0.000000,0.000000}%
\pgfsetstrokecolor{currentstroke}%
\pgfsetdash{}{0pt}%
\pgfpathmoveto{\pgfqpoint{3.818771in}{2.033347in}}%
\pgfpathlineto{\pgfqpoint{3.832474in}{2.031367in}}%
\pgfpathlineto{\pgfqpoint{3.846183in}{2.029467in}}%
\pgfpathlineto{\pgfqpoint{3.859900in}{2.027646in}}%
\pgfpathlineto{\pgfqpoint{3.873623in}{2.025904in}}%
\pgfpathlineto{\pgfqpoint{3.881650in}{2.034852in}}%
\pgfpathlineto{\pgfqpoint{3.889671in}{2.043777in}}%
\pgfpathlineto{\pgfqpoint{3.897686in}{2.052681in}}%
\pgfpathlineto{\pgfqpoint{3.905696in}{2.061563in}}%
\pgfpathlineto{\pgfqpoint{3.891984in}{2.063305in}}%
\pgfpathlineto{\pgfqpoint{3.878278in}{2.065125in}}%
\pgfpathlineto{\pgfqpoint{3.864579in}{2.067025in}}%
\pgfpathlineto{\pgfqpoint{3.850888in}{2.069005in}}%
\pgfpathlineto{\pgfqpoint{3.842867in}{2.060115in}}%
\pgfpathlineto{\pgfqpoint{3.834841in}{2.051210in}}%
\pgfpathlineto{\pgfqpoint{3.826809in}{2.042287in}}%
\pgfpathlineto{\pgfqpoint{3.818771in}{2.033347in}}%
\pgfpathclose%
\pgfusepath{fill}%
\end{pgfscope}%
\begin{pgfscope}%
\pgfpathrectangle{\pgfqpoint{1.150000in}{0.150000in}}{\pgfqpoint{5.700000in}{5.700000in}}%
\pgfusepath{clip}%
\pgfsetbuttcap%
\pgfsetroundjoin%
\definecolor{currentfill}{rgb}{0.258965,0.251537,0.524736}%
\pgfsetfillcolor{currentfill}%
\pgfsetfillopacity{0.700000}%
\pgfsetlinewidth{0.000000pt}%
\definecolor{currentstroke}{rgb}{0.000000,0.000000,0.000000}%
\pgfsetstrokecolor{currentstroke}%
\pgfsetdash{}{0pt}%
\pgfpathmoveto{\pgfqpoint{5.026340in}{2.400128in}}%
\pgfpathlineto{\pgfqpoint{5.040418in}{2.401208in}}%
\pgfpathlineto{\pgfqpoint{5.054506in}{2.402358in}}%
\pgfpathlineto{\pgfqpoint{5.068604in}{2.403577in}}%
\pgfpathlineto{\pgfqpoint{5.082713in}{2.404865in}}%
\pgfpathlineto{\pgfqpoint{5.090262in}{2.411099in}}%
\pgfpathlineto{\pgfqpoint{5.097804in}{2.417350in}}%
\pgfpathlineto{\pgfqpoint{5.105342in}{2.423624in}}%
\pgfpathlineto{\pgfqpoint{5.112874in}{2.429924in}}%
\pgfpathlineto{\pgfqpoint{5.098783in}{2.428885in}}%
\pgfpathlineto{\pgfqpoint{5.084702in}{2.427916in}}%
\pgfpathlineto{\pgfqpoint{5.070632in}{2.427016in}}%
\pgfpathlineto{\pgfqpoint{5.056572in}{2.426186in}}%
\pgfpathlineto{\pgfqpoint{5.049022in}{2.419629in}}%
\pgfpathlineto{\pgfqpoint{5.041467in}{2.413103in}}%
\pgfpathlineto{\pgfqpoint{5.033906in}{2.406604in}}%
\pgfpathlineto{\pgfqpoint{5.026340in}{2.400128in}}%
\pgfpathclose%
\pgfusepath{fill}%
\end{pgfscope}%
\begin{pgfscope}%
\pgfpathrectangle{\pgfqpoint{1.150000in}{0.150000in}}{\pgfqpoint{5.700000in}{5.700000in}}%
\pgfusepath{clip}%
\pgfsetbuttcap%
\pgfsetroundjoin%
\definecolor{currentfill}{rgb}{0.283072,0.130895,0.449241}%
\pgfsetfillcolor{currentfill}%
\pgfsetfillopacity{0.700000}%
\pgfsetlinewidth{0.000000pt}%
\definecolor{currentstroke}{rgb}{0.000000,0.000000,0.000000}%
\pgfsetstrokecolor{currentstroke}%
\pgfsetdash{}{0pt}%
\pgfpathmoveto{\pgfqpoint{2.453734in}{2.164672in}}%
\pgfpathlineto{\pgfqpoint{2.467340in}{2.154241in}}%
\pgfpathlineto{\pgfqpoint{2.480945in}{2.143932in}}%
\pgfpathlineto{\pgfqpoint{2.494549in}{2.133744in}}%
\pgfpathlineto{\pgfqpoint{2.508153in}{2.123676in}}%
\pgfpathlineto{\pgfqpoint{2.516774in}{2.128870in}}%
\pgfpathlineto{\pgfqpoint{2.525384in}{2.134191in}}%
\pgfpathlineto{\pgfqpoint{2.533984in}{2.139636in}}%
\pgfpathlineto{\pgfqpoint{2.542572in}{2.145201in}}%
\pgfpathlineto{\pgfqpoint{2.528993in}{2.155059in}}%
\pgfpathlineto{\pgfqpoint{2.515414in}{2.165038in}}%
\pgfpathlineto{\pgfqpoint{2.501834in}{2.175137in}}%
\pgfpathlineto{\pgfqpoint{2.488254in}{2.185358in}}%
\pgfpathlineto{\pgfqpoint{2.479641in}{2.179995in}}%
\pgfpathlineto{\pgfqpoint{2.471017in}{2.174758in}}%
\pgfpathlineto{\pgfqpoint{2.462381in}{2.169649in}}%
\pgfpathlineto{\pgfqpoint{2.453734in}{2.164672in}}%
\pgfpathclose%
\pgfusepath{fill}%
\end{pgfscope}%
\begin{pgfscope}%
\pgfpathrectangle{\pgfqpoint{1.150000in}{0.150000in}}{\pgfqpoint{5.700000in}{5.700000in}}%
\pgfusepath{clip}%
\pgfsetbuttcap%
\pgfsetroundjoin%
\definecolor{currentfill}{rgb}{0.272594,0.025563,0.353093}%
\pgfsetfillcolor{currentfill}%
\pgfsetfillopacity{0.700000}%
\pgfsetlinewidth{0.000000pt}%
\definecolor{currentstroke}{rgb}{0.000000,0.000000,0.000000}%
\pgfsetstrokecolor{currentstroke}%
\pgfsetdash{}{0pt}%
\pgfpathmoveto{\pgfqpoint{3.132167in}{1.956168in}}%
\pgfpathlineto{\pgfqpoint{3.145752in}{1.950679in}}%
\pgfpathlineto{\pgfqpoint{3.159341in}{1.945283in}}%
\pgfpathlineto{\pgfqpoint{3.172933in}{1.939979in}}%
\pgfpathlineto{\pgfqpoint{3.186530in}{1.934767in}}%
\pgfpathlineto{\pgfqpoint{3.194813in}{1.942995in}}%
\pgfpathlineto{\pgfqpoint{3.203090in}{1.951261in}}%
\pgfpathlineto{\pgfqpoint{3.211359in}{1.959563in}}%
\pgfpathlineto{\pgfqpoint{3.219622in}{1.967900in}}%
\pgfpathlineto{\pgfqpoint{3.206041in}{1.972989in}}%
\pgfpathlineto{\pgfqpoint{3.192464in}{1.978170in}}%
\pgfpathlineto{\pgfqpoint{3.178890in}{1.983442in}}%
\pgfpathlineto{\pgfqpoint{3.165321in}{1.988808in}}%
\pgfpathlineto{\pgfqpoint{3.157043in}{1.980586in}}%
\pgfpathlineto{\pgfqpoint{3.148758in}{1.972405in}}%
\pgfpathlineto{\pgfqpoint{3.140466in}{1.964265in}}%
\pgfpathlineto{\pgfqpoint{3.132167in}{1.956168in}}%
\pgfpathclose%
\pgfusepath{fill}%
\end{pgfscope}%
\begin{pgfscope}%
\pgfpathrectangle{\pgfqpoint{1.150000in}{0.150000in}}{\pgfqpoint{5.700000in}{5.700000in}}%
\pgfusepath{clip}%
\pgfsetbuttcap%
\pgfsetroundjoin%
\definecolor{currentfill}{rgb}{0.276022,0.044167,0.370164}%
\pgfsetfillcolor{currentfill}%
\pgfsetfillopacity{0.700000}%
\pgfsetlinewidth{0.000000pt}%
\definecolor{currentstroke}{rgb}{0.000000,0.000000,0.000000}%
\pgfsetstrokecolor{currentstroke}%
\pgfsetdash{}{0pt}%
\pgfpathmoveto{\pgfqpoint{2.848031in}{1.998957in}}%
\pgfpathlineto{\pgfqpoint{2.861605in}{1.991601in}}%
\pgfpathlineto{\pgfqpoint{2.875181in}{1.984347in}}%
\pgfpathlineto{\pgfqpoint{2.888760in}{1.977194in}}%
\pgfpathlineto{\pgfqpoint{2.902340in}{1.970142in}}%
\pgfpathlineto{\pgfqpoint{2.910752in}{1.977312in}}%
\pgfpathlineto{\pgfqpoint{2.919156in}{1.984556in}}%
\pgfpathlineto{\pgfqpoint{2.927552in}{1.991872in}}%
\pgfpathlineto{\pgfqpoint{2.935939in}{1.999257in}}%
\pgfpathlineto{\pgfqpoint{2.922377in}{2.006144in}}%
\pgfpathlineto{\pgfqpoint{2.908818in}{2.013131in}}%
\pgfpathlineto{\pgfqpoint{2.895260in}{2.020220in}}%
\pgfpathlineto{\pgfqpoint{2.881705in}{2.027411in}}%
\pgfpathlineto{\pgfqpoint{2.873299in}{2.020183in}}%
\pgfpathlineto{\pgfqpoint{2.864885in}{2.013030in}}%
\pgfpathlineto{\pgfqpoint{2.856462in}{2.005954in}}%
\pgfpathlineto{\pgfqpoint{2.848031in}{1.998957in}}%
\pgfpathclose%
\pgfusepath{fill}%
\end{pgfscope}%
\begin{pgfscope}%
\pgfpathrectangle{\pgfqpoint{1.150000in}{0.150000in}}{\pgfqpoint{5.700000in}{5.700000in}}%
\pgfusepath{clip}%
\pgfsetbuttcap%
\pgfsetroundjoin%
\definecolor{currentfill}{rgb}{0.283229,0.120777,0.440584}%
\pgfsetfillcolor{currentfill}%
\pgfsetfillopacity{0.700000}%
\pgfsetlinewidth{0.000000pt}%
\definecolor{currentstroke}{rgb}{0.000000,0.000000,0.000000}%
\pgfsetstrokecolor{currentstroke}%
\pgfsetdash{}{0pt}%
\pgfpathmoveto{\pgfqpoint{4.134443in}{2.116360in}}%
\pgfpathlineto{\pgfqpoint{4.148233in}{2.115561in}}%
\pgfpathlineto{\pgfqpoint{4.162031in}{2.114837in}}%
\pgfpathlineto{\pgfqpoint{4.175837in}{2.114190in}}%
\pgfpathlineto{\pgfqpoint{4.189652in}{2.113617in}}%
\pgfpathlineto{\pgfqpoint{4.197567in}{2.122128in}}%
\pgfpathlineto{\pgfqpoint{4.205476in}{2.130606in}}%
\pgfpathlineto{\pgfqpoint{4.213380in}{2.139054in}}%
\pgfpathlineto{\pgfqpoint{4.221277in}{2.147472in}}%
\pgfpathlineto{\pgfqpoint{4.207474in}{2.148107in}}%
\pgfpathlineto{\pgfqpoint{4.193679in}{2.148817in}}%
\pgfpathlineto{\pgfqpoint{4.179892in}{2.149602in}}%
\pgfpathlineto{\pgfqpoint{4.166113in}{2.150463in}}%
\pgfpathlineto{\pgfqpoint{4.158204in}{2.141975in}}%
\pgfpathlineto{\pgfqpoint{4.150290in}{2.133462in}}%
\pgfpathlineto{\pgfqpoint{4.142370in}{2.124925in}}%
\pgfpathlineto{\pgfqpoint{4.134443in}{2.116360in}}%
\pgfpathclose%
\pgfusepath{fill}%
\end{pgfscope}%
\begin{pgfscope}%
\pgfpathrectangle{\pgfqpoint{1.150000in}{0.150000in}}{\pgfqpoint{5.700000in}{5.700000in}}%
\pgfusepath{clip}%
\pgfsetbuttcap%
\pgfsetroundjoin%
\definecolor{currentfill}{rgb}{0.274952,0.037752,0.364543}%
\pgfsetfillcolor{currentfill}%
\pgfsetfillopacity{0.700000}%
\pgfsetlinewidth{0.000000pt}%
\definecolor{currentstroke}{rgb}{0.000000,0.000000,0.000000}%
\pgfsetstrokecolor{currentstroke}%
\pgfsetdash{}{0pt}%
\pgfpathmoveto{\pgfqpoint{3.502975in}{1.969338in}}%
\pgfpathlineto{\pgfqpoint{3.516611in}{1.965925in}}%
\pgfpathlineto{\pgfqpoint{3.530254in}{1.962597in}}%
\pgfpathlineto{\pgfqpoint{3.543902in}{1.959353in}}%
\pgfpathlineto{\pgfqpoint{3.557556in}{1.956192in}}%
\pgfpathlineto{\pgfqpoint{3.565696in}{1.965137in}}%
\pgfpathlineto{\pgfqpoint{3.573831in}{1.974081in}}%
\pgfpathlineto{\pgfqpoint{3.581959in}{1.983024in}}%
\pgfpathlineto{\pgfqpoint{3.590082in}{1.991964in}}%
\pgfpathlineto{\pgfqpoint{3.576440in}{1.995064in}}%
\pgfpathlineto{\pgfqpoint{3.562804in}{1.998246in}}%
\pgfpathlineto{\pgfqpoint{3.549174in}{2.001513in}}%
\pgfpathlineto{\pgfqpoint{3.535550in}{2.004864in}}%
\pgfpathlineto{\pgfqpoint{3.527415in}{1.995977in}}%
\pgfpathlineto{\pgfqpoint{3.519274in}{1.987093in}}%
\pgfpathlineto{\pgfqpoint{3.511127in}{1.978214in}}%
\pgfpathlineto{\pgfqpoint{3.502975in}{1.969338in}}%
\pgfpathclose%
\pgfusepath{fill}%
\end{pgfscope}%
\begin{pgfscope}%
\pgfpathrectangle{\pgfqpoint{1.150000in}{0.150000in}}{\pgfqpoint{5.700000in}{5.700000in}}%
\pgfusepath{clip}%
\pgfsetbuttcap%
\pgfsetroundjoin%
\definecolor{currentfill}{rgb}{0.220057,0.343307,0.549413}%
\pgfsetfillcolor{currentfill}%
\pgfsetfillopacity{0.700000}%
\pgfsetlinewidth{0.000000pt}%
\definecolor{currentstroke}{rgb}{0.000000,0.000000,0.000000}%
\pgfsetstrokecolor{currentstroke}%
\pgfsetdash{}{0pt}%
\pgfpathmoveto{\pgfqpoint{5.744874in}{2.610066in}}%
\pgfpathlineto{\pgfqpoint{5.759193in}{2.611409in}}%
\pgfpathlineto{\pgfqpoint{5.773523in}{2.612819in}}%
\pgfpathlineto{\pgfqpoint{5.787864in}{2.614296in}}%
\pgfpathlineto{\pgfqpoint{5.802217in}{2.615839in}}%
\pgfpathlineto{\pgfqpoint{5.809446in}{2.621077in}}%
\pgfpathlineto{\pgfqpoint{5.816674in}{2.626446in}}%
\pgfpathlineto{\pgfqpoint{5.823900in}{2.631954in}}%
\pgfpathlineto{\pgfqpoint{5.831125in}{2.637607in}}%
\pgfpathlineto{\pgfqpoint{5.816800in}{2.636459in}}%
\pgfpathlineto{\pgfqpoint{5.802486in}{2.635377in}}%
\pgfpathlineto{\pgfqpoint{5.788183in}{2.634362in}}%
\pgfpathlineto{\pgfqpoint{5.773892in}{2.633414in}}%
\pgfpathlineto{\pgfqpoint{5.766639in}{2.627359in}}%
\pgfpathlineto{\pgfqpoint{5.759385in}{2.621454in}}%
\pgfpathlineto{\pgfqpoint{5.752131in}{2.615692in}}%
\pgfpathlineto{\pgfqpoint{5.744874in}{2.610066in}}%
\pgfpathclose%
\pgfusepath{fill}%
\end{pgfscope}%
\begin{pgfscope}%
\pgfpathrectangle{\pgfqpoint{1.150000in}{0.150000in}}{\pgfqpoint{5.700000in}{5.700000in}}%
\pgfusepath{clip}%
\pgfsetbuttcap%
\pgfsetroundjoin%
\definecolor{currentfill}{rgb}{0.278012,0.180367,0.486697}%
\pgfsetfillcolor{currentfill}%
\pgfsetfillopacity{0.700000}%
\pgfsetlinewidth{0.000000pt}%
\definecolor{currentstroke}{rgb}{0.000000,0.000000,0.000000}%
\pgfsetstrokecolor{currentstroke}%
\pgfsetdash{}{0pt}%
\pgfpathmoveto{\pgfqpoint{4.537052in}{2.241908in}}%
\pgfpathlineto{\pgfqpoint{4.550968in}{2.242219in}}%
\pgfpathlineto{\pgfqpoint{4.564894in}{2.242603in}}%
\pgfpathlineto{\pgfqpoint{4.578829in}{2.243058in}}%
\pgfpathlineto{\pgfqpoint{4.592773in}{2.243586in}}%
\pgfpathlineto{\pgfqpoint{4.600533in}{2.251121in}}%
\pgfpathlineto{\pgfqpoint{4.608288in}{2.258631in}}%
\pgfpathlineto{\pgfqpoint{4.616036in}{2.266120in}}%
\pgfpathlineto{\pgfqpoint{4.623778in}{2.273591in}}%
\pgfpathlineto{\pgfqpoint{4.609847in}{2.273208in}}%
\pgfpathlineto{\pgfqpoint{4.595925in}{2.272898in}}%
\pgfpathlineto{\pgfqpoint{4.582012in}{2.272660in}}%
\pgfpathlineto{\pgfqpoint{4.568109in}{2.272494in}}%
\pgfpathlineto{\pgfqpoint{4.560353in}{2.264871in}}%
\pgfpathlineto{\pgfqpoint{4.552592in}{2.257234in}}%
\pgfpathlineto{\pgfqpoint{4.544825in}{2.249581in}}%
\pgfpathlineto{\pgfqpoint{4.537052in}{2.241908in}}%
\pgfpathclose%
\pgfusepath{fill}%
\end{pgfscope}%
\begin{pgfscope}%
\pgfpathrectangle{\pgfqpoint{1.150000in}{0.150000in}}{\pgfqpoint{5.700000in}{5.700000in}}%
\pgfusepath{clip}%
\pgfsetbuttcap%
\pgfsetroundjoin%
\definecolor{currentfill}{rgb}{0.272594,0.025563,0.353093}%
\pgfsetfillcolor{currentfill}%
\pgfsetfillopacity{0.700000}%
\pgfsetlinewidth{0.000000pt}%
\definecolor{currentstroke}{rgb}{0.000000,0.000000,0.000000}%
\pgfsetstrokecolor{currentstroke}%
\pgfsetdash{}{0pt}%
\pgfpathmoveto{\pgfqpoint{3.273989in}{1.948452in}}%
\pgfpathlineto{\pgfqpoint{3.287592in}{1.943814in}}%
\pgfpathlineto{\pgfqpoint{3.301199in}{1.939266in}}%
\pgfpathlineto{\pgfqpoint{3.314811in}{1.934805in}}%
\pgfpathlineto{\pgfqpoint{3.328428in}{1.930433in}}%
\pgfpathlineto{\pgfqpoint{3.336656in}{1.939025in}}%
\pgfpathlineto{\pgfqpoint{3.344877in}{1.947638in}}%
\pgfpathlineto{\pgfqpoint{3.353092in}{1.956272in}}%
\pgfpathlineto{\pgfqpoint{3.361301in}{1.964925in}}%
\pgfpathlineto{\pgfqpoint{3.347698in}{1.969195in}}%
\pgfpathlineto{\pgfqpoint{3.334100in}{1.973552in}}%
\pgfpathlineto{\pgfqpoint{3.320507in}{1.977998in}}%
\pgfpathlineto{\pgfqpoint{3.306918in}{1.982533in}}%
\pgfpathlineto{\pgfqpoint{3.298696in}{1.973974in}}%
\pgfpathlineto{\pgfqpoint{3.290467in}{1.965441in}}%
\pgfpathlineto{\pgfqpoint{3.282231in}{1.956933in}}%
\pgfpathlineto{\pgfqpoint{3.273989in}{1.948452in}}%
\pgfpathclose%
\pgfusepath{fill}%
\end{pgfscope}%
\begin{pgfscope}%
\pgfpathrectangle{\pgfqpoint{1.150000in}{0.150000in}}{\pgfqpoint{5.700000in}{5.700000in}}%
\pgfusepath{clip}%
\pgfsetbuttcap%
\pgfsetroundjoin%
\definecolor{currentfill}{rgb}{0.241237,0.296485,0.539709}%
\pgfsetfillcolor{currentfill}%
\pgfsetfillopacity{0.700000}%
\pgfsetlinewidth{0.000000pt}%
\definecolor{currentstroke}{rgb}{0.000000,0.000000,0.000000}%
\pgfsetstrokecolor{currentstroke}%
\pgfsetdash{}{0pt}%
\pgfpathmoveto{\pgfqpoint{5.342400in}{2.492962in}}%
\pgfpathlineto{\pgfqpoint{5.356590in}{2.494332in}}%
\pgfpathlineto{\pgfqpoint{5.370792in}{2.495770in}}%
\pgfpathlineto{\pgfqpoint{5.385004in}{2.497276in}}%
\pgfpathlineto{\pgfqpoint{5.399227in}{2.498851in}}%
\pgfpathlineto{\pgfqpoint{5.406634in}{2.504395in}}%
\pgfpathlineto{\pgfqpoint{5.414036in}{2.509996in}}%
\pgfpathlineto{\pgfqpoint{5.421433in}{2.515660in}}%
\pgfpathlineto{\pgfqpoint{5.428827in}{2.521392in}}%
\pgfpathlineto{\pgfqpoint{5.414625in}{2.520131in}}%
\pgfpathlineto{\pgfqpoint{5.400434in}{2.518937in}}%
\pgfpathlineto{\pgfqpoint{5.386255in}{2.517811in}}%
\pgfpathlineto{\pgfqpoint{5.372086in}{2.516753in}}%
\pgfpathlineto{\pgfqpoint{5.364671in}{2.510701in}}%
\pgfpathlineto{\pgfqpoint{5.357252in}{2.504722in}}%
\pgfpathlineto{\pgfqpoint{5.349828in}{2.498811in}}%
\pgfpathlineto{\pgfqpoint{5.342400in}{2.492962in}}%
\pgfpathclose%
\pgfusepath{fill}%
\end{pgfscope}%
\begin{pgfscope}%
\pgfpathrectangle{\pgfqpoint{1.150000in}{0.150000in}}{\pgfqpoint{5.700000in}{5.700000in}}%
\pgfusepath{clip}%
\pgfsetbuttcap%
\pgfsetroundjoin%
\definecolor{currentfill}{rgb}{0.262138,0.242286,0.520837}%
\pgfsetfillcolor{currentfill}%
\pgfsetfillopacity{0.700000}%
\pgfsetlinewidth{0.000000pt}%
\definecolor{currentstroke}{rgb}{0.000000,0.000000,0.000000}%
\pgfsetstrokecolor{currentstroke}%
\pgfsetdash{}{0pt}%
\pgfpathmoveto{\pgfqpoint{4.939743in}{2.369826in}}%
\pgfpathlineto{\pgfqpoint{4.953796in}{2.370855in}}%
\pgfpathlineto{\pgfqpoint{4.967860in}{2.371955in}}%
\pgfpathlineto{\pgfqpoint{4.981934in}{2.373124in}}%
\pgfpathlineto{\pgfqpoint{4.996018in}{2.374364in}}%
\pgfpathlineto{\pgfqpoint{5.003607in}{2.380792in}}%
\pgfpathlineto{\pgfqpoint{5.011191in}{2.387227in}}%
\pgfpathlineto{\pgfqpoint{5.018768in}{2.393671in}}%
\pgfpathlineto{\pgfqpoint{5.026340in}{2.400128in}}%
\pgfpathlineto{\pgfqpoint{5.012273in}{2.399118in}}%
\pgfpathlineto{\pgfqpoint{4.998215in}{2.398178in}}%
\pgfpathlineto{\pgfqpoint{4.984168in}{2.397307in}}%
\pgfpathlineto{\pgfqpoint{4.970132in}{2.396507in}}%
\pgfpathlineto{\pgfqpoint{4.962543in}{2.389813in}}%
\pgfpathlineto{\pgfqpoint{4.954949in}{2.383137in}}%
\pgfpathlineto{\pgfqpoint{4.947349in}{2.376476in}}%
\pgfpathlineto{\pgfqpoint{4.939743in}{2.369826in}}%
\pgfpathclose%
\pgfusepath{fill}%
\end{pgfscope}%
\begin{pgfscope}%
\pgfpathrectangle{\pgfqpoint{1.150000in}{0.150000in}}{\pgfqpoint{5.700000in}{5.700000in}}%
\pgfusepath{clip}%
\pgfsetbuttcap%
\pgfsetroundjoin%
\definecolor{currentfill}{rgb}{0.271828,0.209303,0.504434}%
\pgfsetfillcolor{currentfill}%
\pgfsetfillopacity{0.700000}%
\pgfsetlinewidth{0.000000pt}%
\definecolor{currentstroke}{rgb}{0.000000,0.000000,0.000000}%
\pgfsetstrokecolor{currentstroke}%
\pgfsetdash{}{0pt}%
\pgfpathmoveto{\pgfqpoint{2.200641in}{2.334206in}}%
\pgfpathlineto{\pgfqpoint{2.214310in}{2.321442in}}%
\pgfpathlineto{\pgfqpoint{2.227976in}{2.308819in}}%
\pgfpathlineto{\pgfqpoint{2.241638in}{2.296334in}}%
\pgfpathlineto{\pgfqpoint{2.255298in}{2.283988in}}%
\pgfpathlineto{\pgfqpoint{2.264079in}{2.287686in}}%
\pgfpathlineto{\pgfqpoint{2.272847in}{2.291544in}}%
\pgfpathlineto{\pgfqpoint{2.281601in}{2.295559in}}%
\pgfpathlineto{\pgfqpoint{2.290342in}{2.299728in}}%
\pgfpathlineto{\pgfqpoint{2.276712in}{2.311841in}}%
\pgfpathlineto{\pgfqpoint{2.263079in}{2.324090in}}%
\pgfpathlineto{\pgfqpoint{2.249443in}{2.336479in}}%
\pgfpathlineto{\pgfqpoint{2.235804in}{2.349008in}}%
\pgfpathlineto{\pgfqpoint{2.227034in}{2.345065in}}%
\pgfpathlineto{\pgfqpoint{2.218250in}{2.341282in}}%
\pgfpathlineto{\pgfqpoint{2.209452in}{2.337661in}}%
\pgfpathlineto{\pgfqpoint{2.200641in}{2.334206in}}%
\pgfpathclose%
\pgfusepath{fill}%
\end{pgfscope}%
\begin{pgfscope}%
\pgfpathrectangle{\pgfqpoint{1.150000in}{0.150000in}}{\pgfqpoint{5.700000in}{5.700000in}}%
\pgfusepath{clip}%
\pgfsetbuttcap%
\pgfsetroundjoin%
\definecolor{currentfill}{rgb}{0.278791,0.062145,0.386592}%
\pgfsetfillcolor{currentfill}%
\pgfsetfillopacity{0.700000}%
\pgfsetlinewidth{0.000000pt}%
\definecolor{currentstroke}{rgb}{0.000000,0.000000,0.000000}%
\pgfsetstrokecolor{currentstroke}%
\pgfsetdash{}{0pt}%
\pgfpathmoveto{\pgfqpoint{3.731778in}{2.006207in}}%
\pgfpathlineto{\pgfqpoint{3.745465in}{2.003886in}}%
\pgfpathlineto{\pgfqpoint{3.759158in}{2.001646in}}%
\pgfpathlineto{\pgfqpoint{3.772858in}{1.999487in}}%
\pgfpathlineto{\pgfqpoint{3.786564in}{1.997407in}}%
\pgfpathlineto{\pgfqpoint{3.794625in}{2.006419in}}%
\pgfpathlineto{\pgfqpoint{3.802679in}{2.015413in}}%
\pgfpathlineto{\pgfqpoint{3.810728in}{2.024389in}}%
\pgfpathlineto{\pgfqpoint{3.818771in}{2.033347in}}%
\pgfpathlineto{\pgfqpoint{3.805076in}{2.035406in}}%
\pgfpathlineto{\pgfqpoint{3.791387in}{2.037545in}}%
\pgfpathlineto{\pgfqpoint{3.777705in}{2.039764in}}%
\pgfpathlineto{\pgfqpoint{3.764030in}{2.042064in}}%
\pgfpathlineto{\pgfqpoint{3.755975in}{2.033119in}}%
\pgfpathlineto{\pgfqpoint{3.747915in}{2.024161in}}%
\pgfpathlineto{\pgfqpoint{3.739850in}{2.015191in}}%
\pgfpathlineto{\pgfqpoint{3.731778in}{2.006207in}}%
\pgfpathclose%
\pgfusepath{fill}%
\end{pgfscope}%
\begin{pgfscope}%
\pgfpathrectangle{\pgfqpoint{1.150000in}{0.150000in}}{\pgfqpoint{5.700000in}{5.700000in}}%
\pgfusepath{clip}%
\pgfsetbuttcap%
\pgfsetroundjoin%
\definecolor{currentfill}{rgb}{0.279566,0.067836,0.391917}%
\pgfsetfillcolor{currentfill}%
\pgfsetfillopacity{0.700000}%
\pgfsetlinewidth{0.000000pt}%
\definecolor{currentstroke}{rgb}{0.000000,0.000000,0.000000}%
\pgfsetstrokecolor{currentstroke}%
\pgfsetdash{}{0pt}%
\pgfpathmoveto{\pgfqpoint{2.705520in}{2.035900in}}%
\pgfpathlineto{\pgfqpoint{2.719103in}{2.027517in}}%
\pgfpathlineto{\pgfqpoint{2.732687in}{2.019241in}}%
\pgfpathlineto{\pgfqpoint{2.746272in}{2.011072in}}%
\pgfpathlineto{\pgfqpoint{2.759858in}{2.003009in}}%
\pgfpathlineto{\pgfqpoint{2.768346in}{2.009472in}}%
\pgfpathlineto{\pgfqpoint{2.776824in}{2.016029in}}%
\pgfpathlineto{\pgfqpoint{2.785293in}{2.022677in}}%
\pgfpathlineto{\pgfqpoint{2.793753in}{2.029415in}}%
\pgfpathlineto{\pgfqpoint{2.780188in}{2.037291in}}%
\pgfpathlineto{\pgfqpoint{2.766624in}{2.045273in}}%
\pgfpathlineto{\pgfqpoint{2.753061in}{2.053363in}}%
\pgfpathlineto{\pgfqpoint{2.739500in}{2.061560in}}%
\pgfpathlineto{\pgfqpoint{2.731019in}{2.055001in}}%
\pgfpathlineto{\pgfqpoint{2.722529in}{2.048537in}}%
\pgfpathlineto{\pgfqpoint{2.714029in}{2.042169in}}%
\pgfpathlineto{\pgfqpoint{2.705520in}{2.035900in}}%
\pgfpathclose%
\pgfusepath{fill}%
\end{pgfscope}%
\begin{pgfscope}%
\pgfpathrectangle{\pgfqpoint{1.150000in}{0.150000in}}{\pgfqpoint{5.700000in}{5.700000in}}%
\pgfusepath{clip}%
\pgfsetbuttcap%
\pgfsetroundjoin%
\definecolor{currentfill}{rgb}{0.282910,0.105393,0.426902}%
\pgfsetfillcolor{currentfill}%
\pgfsetfillopacity{0.700000}%
\pgfsetlinewidth{0.000000pt}%
\definecolor{currentstroke}{rgb}{0.000000,0.000000,0.000000}%
\pgfsetstrokecolor{currentstroke}%
\pgfsetdash{}{0pt}%
\pgfpathmoveto{\pgfqpoint{4.047558in}{2.085600in}}%
\pgfpathlineto{\pgfqpoint{4.061327in}{2.084537in}}%
\pgfpathlineto{\pgfqpoint{4.075104in}{2.083550in}}%
\pgfpathlineto{\pgfqpoint{4.088889in}{2.082640in}}%
\pgfpathlineto{\pgfqpoint{4.102682in}{2.081807in}}%
\pgfpathlineto{\pgfqpoint{4.110631in}{2.090492in}}%
\pgfpathlineto{\pgfqpoint{4.118574in}{2.099145in}}%
\pgfpathlineto{\pgfqpoint{4.126512in}{2.107767in}}%
\pgfpathlineto{\pgfqpoint{4.134443in}{2.116360in}}%
\pgfpathlineto{\pgfqpoint{4.120662in}{2.117235in}}%
\pgfpathlineto{\pgfqpoint{4.106888in}{2.118187in}}%
\pgfpathlineto{\pgfqpoint{4.093122in}{2.119214in}}%
\pgfpathlineto{\pgfqpoint{4.079363in}{2.120319in}}%
\pgfpathlineto{\pgfqpoint{4.071421in}{2.111677in}}%
\pgfpathlineto{\pgfqpoint{4.063472in}{2.103011in}}%
\pgfpathlineto{\pgfqpoint{4.055518in}{2.094319in}}%
\pgfpathlineto{\pgfqpoint{4.047558in}{2.085600in}}%
\pgfpathclose%
\pgfusepath{fill}%
\end{pgfscope}%
\begin{pgfscope}%
\pgfpathrectangle{\pgfqpoint{1.150000in}{0.150000in}}{\pgfqpoint{5.700000in}{5.700000in}}%
\pgfusepath{clip}%
\pgfsetbuttcap%
\pgfsetroundjoin%
\definecolor{currentfill}{rgb}{0.283091,0.110553,0.431554}%
\pgfsetfillcolor{currentfill}%
\pgfsetfillopacity{0.700000}%
\pgfsetlinewidth{0.000000pt}%
\definecolor{currentstroke}{rgb}{0.000000,0.000000,0.000000}%
\pgfsetstrokecolor{currentstroke}%
\pgfsetdash{}{0pt}%
\pgfpathmoveto{\pgfqpoint{2.508153in}{2.123676in}}%
\pgfpathlineto{\pgfqpoint{2.521756in}{2.113726in}}%
\pgfpathlineto{\pgfqpoint{2.535358in}{2.103895in}}%
\pgfpathlineto{\pgfqpoint{2.548960in}{2.094180in}}%
\pgfpathlineto{\pgfqpoint{2.562562in}{2.084582in}}%
\pgfpathlineto{\pgfqpoint{2.571159in}{2.089994in}}%
\pgfpathlineto{\pgfqpoint{2.579745in}{2.095527in}}%
\pgfpathlineto{\pgfqpoint{2.588320in}{2.101177in}}%
\pgfpathlineto{\pgfqpoint{2.596884in}{2.106943in}}%
\pgfpathlineto{\pgfqpoint{2.583307in}{2.116333in}}%
\pgfpathlineto{\pgfqpoint{2.569729in}{2.125838in}}%
\pgfpathlineto{\pgfqpoint{2.556150in}{2.135460in}}%
\pgfpathlineto{\pgfqpoint{2.542572in}{2.145201in}}%
\pgfpathlineto{\pgfqpoint{2.533984in}{2.139636in}}%
\pgfpathlineto{\pgfqpoint{2.525384in}{2.134191in}}%
\pgfpathlineto{\pgfqpoint{2.516774in}{2.128870in}}%
\pgfpathlineto{\pgfqpoint{2.508153in}{2.123676in}}%
\pgfpathclose%
\pgfusepath{fill}%
\end{pgfscope}%
\begin{pgfscope}%
\pgfpathrectangle{\pgfqpoint{1.150000in}{0.150000in}}{\pgfqpoint{5.700000in}{5.700000in}}%
\pgfusepath{clip}%
\pgfsetbuttcap%
\pgfsetroundjoin%
\definecolor{currentfill}{rgb}{0.280255,0.165693,0.476498}%
\pgfsetfillcolor{currentfill}%
\pgfsetfillopacity{0.700000}%
\pgfsetlinewidth{0.000000pt}%
\definecolor{currentstroke}{rgb}{0.000000,0.000000,0.000000}%
\pgfsetstrokecolor{currentstroke}%
\pgfsetdash{}{0pt}%
\pgfpathmoveto{\pgfqpoint{4.450274in}{2.209962in}}%
\pgfpathlineto{\pgfqpoint{4.464167in}{2.210105in}}%
\pgfpathlineto{\pgfqpoint{4.478068in}{2.210322in}}%
\pgfpathlineto{\pgfqpoint{4.491978in}{2.210612in}}%
\pgfpathlineto{\pgfqpoint{4.505898in}{2.210975in}}%
\pgfpathlineto{\pgfqpoint{4.513696in}{2.218750in}}%
\pgfpathlineto{\pgfqpoint{4.521487in}{2.226496in}}%
\pgfpathlineto{\pgfqpoint{4.529272in}{2.234214in}}%
\pgfpathlineto{\pgfqpoint{4.537052in}{2.241908in}}%
\pgfpathlineto{\pgfqpoint{4.523144in}{2.241670in}}%
\pgfpathlineto{\pgfqpoint{4.509246in}{2.241505in}}%
\pgfpathlineto{\pgfqpoint{4.495357in}{2.241412in}}%
\pgfpathlineto{\pgfqpoint{4.481477in}{2.241393in}}%
\pgfpathlineto{\pgfqpoint{4.473686in}{2.233567in}}%
\pgfpathlineto{\pgfqpoint{4.465888in}{2.225721in}}%
\pgfpathlineto{\pgfqpoint{4.458084in}{2.217853in}}%
\pgfpathlineto{\pgfqpoint{4.450274in}{2.209962in}}%
\pgfpathclose%
\pgfusepath{fill}%
\end{pgfscope}%
\begin{pgfscope}%
\pgfpathrectangle{\pgfqpoint{1.150000in}{0.150000in}}{\pgfqpoint{5.700000in}{5.700000in}}%
\pgfusepath{clip}%
\pgfsetbuttcap%
\pgfsetroundjoin%
\definecolor{currentfill}{rgb}{0.223925,0.334994,0.548053}%
\pgfsetfillcolor{currentfill}%
\pgfsetfillopacity{0.700000}%
\pgfsetlinewidth{0.000000pt}%
\definecolor{currentstroke}{rgb}{0.000000,0.000000,0.000000}%
\pgfsetstrokecolor{currentstroke}%
\pgfsetdash{}{0pt}%
\pgfpathmoveto{\pgfqpoint{5.658563in}{2.582584in}}%
\pgfpathlineto{\pgfqpoint{5.672862in}{2.584035in}}%
\pgfpathlineto{\pgfqpoint{5.687173in}{2.585553in}}%
\pgfpathlineto{\pgfqpoint{5.701495in}{2.587138in}}%
\pgfpathlineto{\pgfqpoint{5.715828in}{2.588789in}}%
\pgfpathlineto{\pgfqpoint{5.723093in}{2.593938in}}%
\pgfpathlineto{\pgfqpoint{5.730356in}{2.599196in}}%
\pgfpathlineto{\pgfqpoint{5.737616in}{2.604570in}}%
\pgfpathlineto{\pgfqpoint{5.744874in}{2.610066in}}%
\pgfpathlineto{\pgfqpoint{5.730567in}{2.608790in}}%
\pgfpathlineto{\pgfqpoint{5.716271in}{2.607580in}}%
\pgfpathlineto{\pgfqpoint{5.701987in}{2.606436in}}%
\pgfpathlineto{\pgfqpoint{5.687714in}{2.605360in}}%
\pgfpathlineto{\pgfqpoint{5.680429in}{2.599481in}}%
\pgfpathlineto{\pgfqpoint{5.673143in}{2.593730in}}%
\pgfpathlineto{\pgfqpoint{5.665854in}{2.588100in}}%
\pgfpathlineto{\pgfqpoint{5.658563in}{2.582584in}}%
\pgfpathclose%
\pgfusepath{fill}%
\end{pgfscope}%
\begin{pgfscope}%
\pgfpathrectangle{\pgfqpoint{1.150000in}{0.150000in}}{\pgfqpoint{5.700000in}{5.700000in}}%
\pgfusepath{clip}%
\pgfsetbuttcap%
\pgfsetroundjoin%
\definecolor{currentfill}{rgb}{0.266580,0.228262,0.514349}%
\pgfsetfillcolor{currentfill}%
\pgfsetfillopacity{0.700000}%
\pgfsetlinewidth{0.000000pt}%
\definecolor{currentstroke}{rgb}{0.000000,0.000000,0.000000}%
\pgfsetstrokecolor{currentstroke}%
\pgfsetdash{}{0pt}%
\pgfpathmoveto{\pgfqpoint{4.853085in}{2.339002in}}%
\pgfpathlineto{\pgfqpoint{4.867114in}{2.339958in}}%
\pgfpathlineto{\pgfqpoint{4.881153in}{2.340985in}}%
\pgfpathlineto{\pgfqpoint{4.895202in}{2.342082in}}%
\pgfpathlineto{\pgfqpoint{4.909261in}{2.343250in}}%
\pgfpathlineto{\pgfqpoint{4.916891in}{2.349898in}}%
\pgfpathlineto{\pgfqpoint{4.924514in}{2.356541in}}%
\pgfpathlineto{\pgfqpoint{4.932131in}{2.363182in}}%
\pgfpathlineto{\pgfqpoint{4.939743in}{2.369826in}}%
\pgfpathlineto{\pgfqpoint{4.925700in}{2.368867in}}%
\pgfpathlineto{\pgfqpoint{4.911667in}{2.367978in}}%
\pgfpathlineto{\pgfqpoint{4.897644in}{2.367159in}}%
\pgfpathlineto{\pgfqpoint{4.883630in}{2.366411in}}%
\pgfpathlineto{\pgfqpoint{4.876003in}{2.359551in}}%
\pgfpathlineto{\pgfqpoint{4.868370in}{2.352699in}}%
\pgfpathlineto{\pgfqpoint{4.860730in}{2.345850in}}%
\pgfpathlineto{\pgfqpoint{4.853085in}{2.339002in}}%
\pgfpathclose%
\pgfusepath{fill}%
\end{pgfscope}%
\begin{pgfscope}%
\pgfpathrectangle{\pgfqpoint{1.150000in}{0.150000in}}{\pgfqpoint{5.700000in}{5.700000in}}%
\pgfusepath{clip}%
\pgfsetbuttcap%
\pgfsetroundjoin%
\definecolor{currentfill}{rgb}{0.276194,0.190074,0.493001}%
\pgfsetfillcolor{currentfill}%
\pgfsetfillopacity{0.700000}%
\pgfsetlinewidth{0.000000pt}%
\definecolor{currentstroke}{rgb}{0.000000,0.000000,0.000000}%
\pgfsetstrokecolor{currentstroke}%
\pgfsetdash{}{0pt}%
\pgfpathmoveto{\pgfqpoint{2.255298in}{2.283988in}}%
\pgfpathlineto{\pgfqpoint{2.268954in}{2.271778in}}%
\pgfpathlineto{\pgfqpoint{2.282608in}{2.259703in}}%
\pgfpathlineto{\pgfqpoint{2.296259in}{2.247763in}}%
\pgfpathlineto{\pgfqpoint{2.309908in}{2.235955in}}%
\pgfpathlineto{\pgfqpoint{2.318660in}{2.239894in}}%
\pgfpathlineto{\pgfqpoint{2.327399in}{2.243989in}}%
\pgfpathlineto{\pgfqpoint{2.336124in}{2.248235in}}%
\pgfpathlineto{\pgfqpoint{2.344837in}{2.252629in}}%
\pgfpathlineto{\pgfqpoint{2.331217in}{2.264204in}}%
\pgfpathlineto{\pgfqpoint{2.317595in}{2.275911in}}%
\pgfpathlineto{\pgfqpoint{2.303970in}{2.287752in}}%
\pgfpathlineto{\pgfqpoint{2.290342in}{2.299728in}}%
\pgfpathlineto{\pgfqpoint{2.281601in}{2.295559in}}%
\pgfpathlineto{\pgfqpoint{2.272847in}{2.291544in}}%
\pgfpathlineto{\pgfqpoint{2.264079in}{2.287686in}}%
\pgfpathlineto{\pgfqpoint{2.255298in}{2.283988in}}%
\pgfpathclose%
\pgfusepath{fill}%
\end{pgfscope}%
\begin{pgfscope}%
\pgfpathrectangle{\pgfqpoint{1.150000in}{0.150000in}}{\pgfqpoint{5.700000in}{5.700000in}}%
\pgfusepath{clip}%
\pgfsetbuttcap%
\pgfsetroundjoin%
\definecolor{currentfill}{rgb}{0.246811,0.283237,0.535941}%
\pgfsetfillcolor{currentfill}%
\pgfsetfillopacity{0.700000}%
\pgfsetlinewidth{0.000000pt}%
\definecolor{currentstroke}{rgb}{0.000000,0.000000,0.000000}%
\pgfsetstrokecolor{currentstroke}%
\pgfsetdash{}{0pt}%
\pgfpathmoveto{\pgfqpoint{5.255905in}{2.464115in}}%
\pgfpathlineto{\pgfqpoint{5.270073in}{2.465504in}}%
\pgfpathlineto{\pgfqpoint{5.284251in}{2.466961in}}%
\pgfpathlineto{\pgfqpoint{5.298440in}{2.468486in}}%
\pgfpathlineto{\pgfqpoint{5.312640in}{2.470080in}}%
\pgfpathlineto{\pgfqpoint{5.320088in}{2.475734in}}%
\pgfpathlineto{\pgfqpoint{5.327530in}{2.481429in}}%
\pgfpathlineto{\pgfqpoint{5.334967in}{2.487169in}}%
\pgfpathlineto{\pgfqpoint{5.342400in}{2.492962in}}%
\pgfpathlineto{\pgfqpoint{5.328221in}{2.491660in}}%
\pgfpathlineto{\pgfqpoint{5.314052in}{2.490427in}}%
\pgfpathlineto{\pgfqpoint{5.299894in}{2.489262in}}%
\pgfpathlineto{\pgfqpoint{5.285747in}{2.488165in}}%
\pgfpathlineto{\pgfqpoint{5.278294in}{2.482073in}}%
\pgfpathlineto{\pgfqpoint{5.270836in}{2.476038in}}%
\pgfpathlineto{\pgfqpoint{5.263373in}{2.470054in}}%
\pgfpathlineto{\pgfqpoint{5.255905in}{2.464115in}}%
\pgfpathclose%
\pgfusepath{fill}%
\end{pgfscope}%
\begin{pgfscope}%
\pgfpathrectangle{\pgfqpoint{1.150000in}{0.150000in}}{\pgfqpoint{5.700000in}{5.700000in}}%
\pgfusepath{clip}%
\pgfsetbuttcap%
\pgfsetroundjoin%
\definecolor{currentfill}{rgb}{0.273809,0.031497,0.358853}%
\pgfsetfillcolor{currentfill}%
\pgfsetfillopacity{0.700000}%
\pgfsetlinewidth{0.000000pt}%
\definecolor{currentstroke}{rgb}{0.000000,0.000000,0.000000}%
\pgfsetstrokecolor{currentstroke}%
\pgfsetdash{}{0pt}%
\pgfpathmoveto{\pgfqpoint{3.415761in}{1.948720in}}%
\pgfpathlineto{\pgfqpoint{3.429389in}{1.944885in}}%
\pgfpathlineto{\pgfqpoint{3.443022in}{1.941135in}}%
\pgfpathlineto{\pgfqpoint{3.456661in}{1.937471in}}%
\pgfpathlineto{\pgfqpoint{3.470304in}{1.933892in}}%
\pgfpathlineto{\pgfqpoint{3.478481in}{1.942743in}}%
\pgfpathlineto{\pgfqpoint{3.486652in}{1.951602in}}%
\pgfpathlineto{\pgfqpoint{3.494816in}{1.960467in}}%
\pgfpathlineto{\pgfqpoint{3.502975in}{1.969338in}}%
\pgfpathlineto{\pgfqpoint{3.489343in}{1.972835in}}%
\pgfpathlineto{\pgfqpoint{3.475718in}{1.976417in}}%
\pgfpathlineto{\pgfqpoint{3.462097in}{1.980084in}}%
\pgfpathlineto{\pgfqpoint{3.448482in}{1.983837in}}%
\pgfpathlineto{\pgfqpoint{3.440311in}{1.975041in}}%
\pgfpathlineto{\pgfqpoint{3.432134in}{1.966256in}}%
\pgfpathlineto{\pgfqpoint{3.423951in}{1.957481in}}%
\pgfpathlineto{\pgfqpoint{3.415761in}{1.948720in}}%
\pgfpathclose%
\pgfusepath{fill}%
\end{pgfscope}%
\begin{pgfscope}%
\pgfpathrectangle{\pgfqpoint{1.150000in}{0.150000in}}{\pgfqpoint{5.700000in}{5.700000in}}%
\pgfusepath{clip}%
\pgfsetbuttcap%
\pgfsetroundjoin%
\definecolor{currentfill}{rgb}{0.277018,0.050344,0.375715}%
\pgfsetfillcolor{currentfill}%
\pgfsetfillopacity{0.700000}%
\pgfsetlinewidth{0.000000pt}%
\definecolor{currentstroke}{rgb}{0.000000,0.000000,0.000000}%
\pgfsetstrokecolor{currentstroke}%
\pgfsetdash{}{0pt}%
\pgfpathmoveto{\pgfqpoint{3.644710in}{1.980396in}}%
\pgfpathlineto{\pgfqpoint{3.658382in}{1.977710in}}%
\pgfpathlineto{\pgfqpoint{3.672060in}{1.975105in}}%
\pgfpathlineto{\pgfqpoint{3.685745in}{1.972582in}}%
\pgfpathlineto{\pgfqpoint{3.699436in}{1.970140in}}%
\pgfpathlineto{\pgfqpoint{3.707531in}{1.979176in}}%
\pgfpathlineto{\pgfqpoint{3.715619in}{1.988199in}}%
\pgfpathlineto{\pgfqpoint{3.723702in}{1.997210in}}%
\pgfpathlineto{\pgfqpoint{3.731778in}{2.006207in}}%
\pgfpathlineto{\pgfqpoint{3.718099in}{2.008608in}}%
\pgfpathlineto{\pgfqpoint{3.704425in}{2.011090in}}%
\pgfpathlineto{\pgfqpoint{3.690758in}{2.013654in}}%
\pgfpathlineto{\pgfqpoint{3.677097in}{2.016299in}}%
\pgfpathlineto{\pgfqpoint{3.669009in}{2.007335in}}%
\pgfpathlineto{\pgfqpoint{3.660915in}{1.998363in}}%
\pgfpathlineto{\pgfqpoint{3.652815in}{1.989384in}}%
\pgfpathlineto{\pgfqpoint{3.644710in}{1.980396in}}%
\pgfpathclose%
\pgfusepath{fill}%
\end{pgfscope}%
\begin{pgfscope}%
\pgfpathrectangle{\pgfqpoint{1.150000in}{0.150000in}}{\pgfqpoint{5.700000in}{5.700000in}}%
\pgfusepath{clip}%
\pgfsetbuttcap%
\pgfsetroundjoin%
\definecolor{currentfill}{rgb}{0.272594,0.025563,0.353093}%
\pgfsetfillcolor{currentfill}%
\pgfsetfillopacity{0.700000}%
\pgfsetlinewidth{0.000000pt}%
\definecolor{currentstroke}{rgb}{0.000000,0.000000,0.000000}%
\pgfsetstrokecolor{currentstroke}%
\pgfsetdash{}{0pt}%
\pgfpathmoveto{\pgfqpoint{3.044530in}{1.947710in}}%
\pgfpathlineto{\pgfqpoint{3.058118in}{1.941702in}}%
\pgfpathlineto{\pgfqpoint{3.071708in}{1.935788in}}%
\pgfpathlineto{\pgfqpoint{3.085302in}{1.929968in}}%
\pgfpathlineto{\pgfqpoint{3.098900in}{1.924243in}}%
\pgfpathlineto{\pgfqpoint{3.107228in}{1.932151in}}%
\pgfpathlineto{\pgfqpoint{3.115548in}{1.940109in}}%
\pgfpathlineto{\pgfqpoint{3.123861in}{1.948115in}}%
\pgfpathlineto{\pgfqpoint{3.132167in}{1.956168in}}%
\pgfpathlineto{\pgfqpoint{3.118586in}{1.961749in}}%
\pgfpathlineto{\pgfqpoint{3.105008in}{1.967425in}}%
\pgfpathlineto{\pgfqpoint{3.091434in}{1.973195in}}%
\pgfpathlineto{\pgfqpoint{3.077863in}{1.979059in}}%
\pgfpathlineto{\pgfqpoint{3.069541in}{1.971143in}}%
\pgfpathlineto{\pgfqpoint{3.061212in}{1.963278in}}%
\pgfpathlineto{\pgfqpoint{3.052875in}{1.955466in}}%
\pgfpathlineto{\pgfqpoint{3.044530in}{1.947710in}}%
\pgfpathclose%
\pgfusepath{fill}%
\end{pgfscope}%
\begin{pgfscope}%
\pgfpathrectangle{\pgfqpoint{1.150000in}{0.150000in}}{\pgfqpoint{5.700000in}{5.700000in}}%
\pgfusepath{clip}%
\pgfsetbuttcap%
\pgfsetroundjoin%
\definecolor{currentfill}{rgb}{0.282327,0.094955,0.417331}%
\pgfsetfillcolor{currentfill}%
\pgfsetfillopacity{0.700000}%
\pgfsetlinewidth{0.000000pt}%
\definecolor{currentstroke}{rgb}{0.000000,0.000000,0.000000}%
\pgfsetstrokecolor{currentstroke}%
\pgfsetdash{}{0pt}%
\pgfpathmoveto{\pgfqpoint{3.960619in}{2.055379in}}%
\pgfpathlineto{\pgfqpoint{3.974369in}{2.054028in}}%
\pgfpathlineto{\pgfqpoint{3.988125in}{2.052754in}}%
\pgfpathlineto{\pgfqpoint{4.001890in}{2.051558in}}%
\pgfpathlineto{\pgfqpoint{4.015662in}{2.050439in}}%
\pgfpathlineto{\pgfqpoint{4.023644in}{2.059274in}}%
\pgfpathlineto{\pgfqpoint{4.031621in}{2.068078in}}%
\pgfpathlineto{\pgfqpoint{4.039593in}{2.076854in}}%
\pgfpathlineto{\pgfqpoint{4.047558in}{2.085600in}}%
\pgfpathlineto{\pgfqpoint{4.033797in}{2.086740in}}%
\pgfpathlineto{\pgfqpoint{4.020043in}{2.087957in}}%
\pgfpathlineto{\pgfqpoint{4.006297in}{2.089251in}}%
\pgfpathlineto{\pgfqpoint{3.992559in}{2.090623in}}%
\pgfpathlineto{\pgfqpoint{3.984582in}{2.081848in}}%
\pgfpathlineto{\pgfqpoint{3.976600in}{2.073050in}}%
\pgfpathlineto{\pgfqpoint{3.968613in}{2.064227in}}%
\pgfpathlineto{\pgfqpoint{3.960619in}{2.055379in}}%
\pgfpathclose%
\pgfusepath{fill}%
\end{pgfscope}%
\begin{pgfscope}%
\pgfpathrectangle{\pgfqpoint{1.150000in}{0.150000in}}{\pgfqpoint{5.700000in}{5.700000in}}%
\pgfusepath{clip}%
\pgfsetbuttcap%
\pgfsetroundjoin%
\definecolor{currentfill}{rgb}{0.281412,0.155834,0.469201}%
\pgfsetfillcolor{currentfill}%
\pgfsetfillopacity{0.700000}%
\pgfsetlinewidth{0.000000pt}%
\definecolor{currentstroke}{rgb}{0.000000,0.000000,0.000000}%
\pgfsetstrokecolor{currentstroke}%
\pgfsetdash{}{0pt}%
\pgfpathmoveto{\pgfqpoint{4.363448in}{2.177847in}}%
\pgfpathlineto{\pgfqpoint{4.377316in}{2.177801in}}%
\pgfpathlineto{\pgfqpoint{4.391194in}{2.177829in}}%
\pgfpathlineto{\pgfqpoint{4.405080in}{2.177930in}}%
\pgfpathlineto{\pgfqpoint{4.418975in}{2.178104in}}%
\pgfpathlineto{\pgfqpoint{4.426809in}{2.186116in}}%
\pgfpathlineto{\pgfqpoint{4.434637in}{2.194095in}}%
\pgfpathlineto{\pgfqpoint{4.442459in}{2.202043in}}%
\pgfpathlineto{\pgfqpoint{4.450274in}{2.209962in}}%
\pgfpathlineto{\pgfqpoint{4.436391in}{2.209891in}}%
\pgfpathlineto{\pgfqpoint{4.422517in}{2.209894in}}%
\pgfpathlineto{\pgfqpoint{4.408651in}{2.209970in}}%
\pgfpathlineto{\pgfqpoint{4.394794in}{2.210120in}}%
\pgfpathlineto{\pgfqpoint{4.386967in}{2.202090in}}%
\pgfpathlineto{\pgfqpoint{4.379133in}{2.194036in}}%
\pgfpathlineto{\pgfqpoint{4.371293in}{2.185956in}}%
\pgfpathlineto{\pgfqpoint{4.363448in}{2.177847in}}%
\pgfpathclose%
\pgfusepath{fill}%
\end{pgfscope}%
\begin{pgfscope}%
\pgfpathrectangle{\pgfqpoint{1.150000in}{0.150000in}}{\pgfqpoint{5.700000in}{5.700000in}}%
\pgfusepath{clip}%
\pgfsetbuttcap%
\pgfsetroundjoin%
\definecolor{currentfill}{rgb}{0.274952,0.037752,0.364543}%
\pgfsetfillcolor{currentfill}%
\pgfsetfillopacity{0.700000}%
\pgfsetlinewidth{0.000000pt}%
\definecolor{currentstroke}{rgb}{0.000000,0.000000,0.000000}%
\pgfsetstrokecolor{currentstroke}%
\pgfsetdash{}{0pt}%
\pgfpathmoveto{\pgfqpoint{2.902340in}{1.970142in}}%
\pgfpathlineto{\pgfqpoint{2.915923in}{1.963190in}}%
\pgfpathlineto{\pgfqpoint{2.929508in}{1.956338in}}%
\pgfpathlineto{\pgfqpoint{2.943096in}{1.949584in}}%
\pgfpathlineto{\pgfqpoint{2.956686in}{1.942929in}}%
\pgfpathlineto{\pgfqpoint{2.965080in}{1.950271in}}%
\pgfpathlineto{\pgfqpoint{2.973465in}{1.957683in}}%
\pgfpathlineto{\pgfqpoint{2.981843in}{1.965161in}}%
\pgfpathlineto{\pgfqpoint{2.990212in}{1.972703in}}%
\pgfpathlineto{\pgfqpoint{2.976640in}{1.979194in}}%
\pgfpathlineto{\pgfqpoint{2.963070in}{1.985782in}}%
\pgfpathlineto{\pgfqpoint{2.949503in}{1.992470in}}%
\pgfpathlineto{\pgfqpoint{2.935939in}{1.999257in}}%
\pgfpathlineto{\pgfqpoint{2.927552in}{1.991872in}}%
\pgfpathlineto{\pgfqpoint{2.919156in}{1.984556in}}%
\pgfpathlineto{\pgfqpoint{2.910752in}{1.977312in}}%
\pgfpathlineto{\pgfqpoint{2.902340in}{1.970142in}}%
\pgfpathclose%
\pgfusepath{fill}%
\end{pgfscope}%
\begin{pgfscope}%
\pgfpathrectangle{\pgfqpoint{1.150000in}{0.150000in}}{\pgfqpoint{5.700000in}{5.700000in}}%
\pgfusepath{clip}%
\pgfsetbuttcap%
\pgfsetroundjoin%
\definecolor{currentfill}{rgb}{0.271305,0.019942,0.347269}%
\pgfsetfillcolor{currentfill}%
\pgfsetfillopacity{0.700000}%
\pgfsetlinewidth{0.000000pt}%
\definecolor{currentstroke}{rgb}{0.000000,0.000000,0.000000}%
\pgfsetstrokecolor{currentstroke}%
\pgfsetdash{}{0pt}%
\pgfpathmoveto{\pgfqpoint{3.186530in}{1.934767in}}%
\pgfpathlineto{\pgfqpoint{3.200130in}{1.929646in}}%
\pgfpathlineto{\pgfqpoint{3.213734in}{1.924616in}}%
\pgfpathlineto{\pgfqpoint{3.227343in}{1.919676in}}%
\pgfpathlineto{\pgfqpoint{3.240956in}{1.914826in}}%
\pgfpathlineto{\pgfqpoint{3.249224in}{1.923185in}}%
\pgfpathlineto{\pgfqpoint{3.257486in}{1.931576in}}%
\pgfpathlineto{\pgfqpoint{3.265741in}{1.939999in}}%
\pgfpathlineto{\pgfqpoint{3.273989in}{1.948452in}}%
\pgfpathlineto{\pgfqpoint{3.260391in}{1.953179in}}%
\pgfpathlineto{\pgfqpoint{3.246797in}{1.957996in}}%
\pgfpathlineto{\pgfqpoint{3.233208in}{1.962903in}}%
\pgfpathlineto{\pgfqpoint{3.219622in}{1.967900in}}%
\pgfpathlineto{\pgfqpoint{3.211359in}{1.959563in}}%
\pgfpathlineto{\pgfqpoint{3.203090in}{1.951261in}}%
\pgfpathlineto{\pgfqpoint{3.194813in}{1.942995in}}%
\pgfpathlineto{\pgfqpoint{3.186530in}{1.934767in}}%
\pgfpathclose%
\pgfusepath{fill}%
\end{pgfscope}%
\begin{pgfscope}%
\pgfpathrectangle{\pgfqpoint{1.150000in}{0.150000in}}{\pgfqpoint{5.700000in}{5.700000in}}%
\pgfusepath{clip}%
\pgfsetbuttcap%
\pgfsetroundjoin%
\definecolor{currentfill}{rgb}{0.269308,0.218818,0.509577}%
\pgfsetfillcolor{currentfill}%
\pgfsetfillopacity{0.700000}%
\pgfsetlinewidth{0.000000pt}%
\definecolor{currentstroke}{rgb}{0.000000,0.000000,0.000000}%
\pgfsetstrokecolor{currentstroke}%
\pgfsetdash{}{0pt}%
\pgfpathmoveto{\pgfqpoint{4.766369in}{2.307663in}}%
\pgfpathlineto{\pgfqpoint{4.780373in}{2.308523in}}%
\pgfpathlineto{\pgfqpoint{4.794387in}{2.309455in}}%
\pgfpathlineto{\pgfqpoint{4.808411in}{2.310457in}}%
\pgfpathlineto{\pgfqpoint{4.822444in}{2.311531in}}%
\pgfpathlineto{\pgfqpoint{4.830114in}{2.318417in}}%
\pgfpathlineto{\pgfqpoint{4.837777in}{2.325288in}}%
\pgfpathlineto{\pgfqpoint{4.845434in}{2.332149in}}%
\pgfpathlineto{\pgfqpoint{4.853085in}{2.339002in}}%
\pgfpathlineto{\pgfqpoint{4.839067in}{2.338116in}}%
\pgfpathlineto{\pgfqpoint{4.825058in}{2.337301in}}%
\pgfpathlineto{\pgfqpoint{4.811059in}{2.336557in}}%
\pgfpathlineto{\pgfqpoint{4.797070in}{2.335884in}}%
\pgfpathlineto{\pgfqpoint{4.789403in}{2.328836in}}%
\pgfpathlineto{\pgfqpoint{4.781731in}{2.321786in}}%
\pgfpathlineto{\pgfqpoint{4.774053in}{2.314729in}}%
\pgfpathlineto{\pgfqpoint{4.766369in}{2.307663in}}%
\pgfpathclose%
\pgfusepath{fill}%
\end{pgfscope}%
\begin{pgfscope}%
\pgfpathrectangle{\pgfqpoint{1.150000in}{0.150000in}}{\pgfqpoint{5.700000in}{5.700000in}}%
\pgfusepath{clip}%
\pgfsetbuttcap%
\pgfsetroundjoin%
\definecolor{currentfill}{rgb}{0.227802,0.326594,0.546532}%
\pgfsetfillcolor{currentfill}%
\pgfsetfillopacity{0.700000}%
\pgfsetlinewidth{0.000000pt}%
\definecolor{currentstroke}{rgb}{0.000000,0.000000,0.000000}%
\pgfsetstrokecolor{currentstroke}%
\pgfsetdash{}{0pt}%
\pgfpathmoveto{\pgfqpoint{5.572187in}{2.554985in}}%
\pgfpathlineto{\pgfqpoint{5.586466in}{2.556522in}}%
\pgfpathlineto{\pgfqpoint{5.600756in}{2.558126in}}%
\pgfpathlineto{\pgfqpoint{5.615057in}{2.559797in}}%
\pgfpathlineto{\pgfqpoint{5.629370in}{2.561536in}}%
\pgfpathlineto{\pgfqpoint{5.636673in}{2.566658in}}%
\pgfpathlineto{\pgfqpoint{5.643973in}{2.571869in}}%
\pgfpathlineto{\pgfqpoint{5.651270in}{2.577176in}}%
\pgfpathlineto{\pgfqpoint{5.658563in}{2.582584in}}%
\pgfpathlineto{\pgfqpoint{5.644276in}{2.581200in}}%
\pgfpathlineto{\pgfqpoint{5.629999in}{2.579883in}}%
\pgfpathlineto{\pgfqpoint{5.615734in}{2.578634in}}%
\pgfpathlineto{\pgfqpoint{5.601480in}{2.577451in}}%
\pgfpathlineto{\pgfqpoint{5.594161in}{2.571681in}}%
\pgfpathlineto{\pgfqpoint{5.586840in}{2.566018in}}%
\pgfpathlineto{\pgfqpoint{5.579515in}{2.560454in}}%
\pgfpathlineto{\pgfqpoint{5.572187in}{2.554985in}}%
\pgfpathclose%
\pgfusepath{fill}%
\end{pgfscope}%
\begin{pgfscope}%
\pgfpathrectangle{\pgfqpoint{1.150000in}{0.150000in}}{\pgfqpoint{5.700000in}{5.700000in}}%
\pgfusepath{clip}%
\pgfsetbuttcap%
\pgfsetroundjoin%
\definecolor{currentfill}{rgb}{0.279574,0.170599,0.479997}%
\pgfsetfillcolor{currentfill}%
\pgfsetfillopacity{0.700000}%
\pgfsetlinewidth{0.000000pt}%
\definecolor{currentstroke}{rgb}{0.000000,0.000000,0.000000}%
\pgfsetstrokecolor{currentstroke}%
\pgfsetdash{}{0pt}%
\pgfpathmoveto{\pgfqpoint{2.309908in}{2.235955in}}%
\pgfpathlineto{\pgfqpoint{2.323554in}{2.224279in}}%
\pgfpathlineto{\pgfqpoint{2.337199in}{2.212733in}}%
\pgfpathlineto{\pgfqpoint{2.350841in}{2.201317in}}%
\pgfpathlineto{\pgfqpoint{2.364481in}{2.190030in}}%
\pgfpathlineto{\pgfqpoint{2.373204in}{2.194210in}}%
\pgfpathlineto{\pgfqpoint{2.381915in}{2.198539in}}%
\pgfpathlineto{\pgfqpoint{2.390613in}{2.203015in}}%
\pgfpathlineto{\pgfqpoint{2.399298in}{2.207634in}}%
\pgfpathlineto{\pgfqpoint{2.385686in}{2.218690in}}%
\pgfpathlineto{\pgfqpoint{2.372072in}{2.229873in}}%
\pgfpathlineto{\pgfqpoint{2.358455in}{2.241186in}}%
\pgfpathlineto{\pgfqpoint{2.344837in}{2.252629in}}%
\pgfpathlineto{\pgfqpoint{2.336124in}{2.248235in}}%
\pgfpathlineto{\pgfqpoint{2.327399in}{2.243989in}}%
\pgfpathlineto{\pgfqpoint{2.318660in}{2.239894in}}%
\pgfpathlineto{\pgfqpoint{2.309908in}{2.235955in}}%
\pgfpathclose%
\pgfusepath{fill}%
\end{pgfscope}%
\begin{pgfscope}%
\pgfpathrectangle{\pgfqpoint{1.150000in}{0.150000in}}{\pgfqpoint{5.700000in}{5.700000in}}%
\pgfusepath{clip}%
\pgfsetbuttcap%
\pgfsetroundjoin%
\definecolor{currentfill}{rgb}{0.282656,0.100196,0.422160}%
\pgfsetfillcolor{currentfill}%
\pgfsetfillopacity{0.700000}%
\pgfsetlinewidth{0.000000pt}%
\definecolor{currentstroke}{rgb}{0.000000,0.000000,0.000000}%
\pgfsetstrokecolor{currentstroke}%
\pgfsetdash{}{0pt}%
\pgfpathmoveto{\pgfqpoint{2.562562in}{2.084582in}}%
\pgfpathlineto{\pgfqpoint{2.576164in}{2.075099in}}%
\pgfpathlineto{\pgfqpoint{2.589766in}{2.065731in}}%
\pgfpathlineto{\pgfqpoint{2.603368in}{2.056476in}}%
\pgfpathlineto{\pgfqpoint{2.616970in}{2.047333in}}%
\pgfpathlineto{\pgfqpoint{2.625543in}{2.052961in}}%
\pgfpathlineto{\pgfqpoint{2.634105in}{2.058705in}}%
\pgfpathlineto{\pgfqpoint{2.642657in}{2.064562in}}%
\pgfpathlineto{\pgfqpoint{2.651198in}{2.070528in}}%
\pgfpathlineto{\pgfqpoint{2.637619in}{2.079462in}}%
\pgfpathlineto{\pgfqpoint{2.624041in}{2.088509in}}%
\pgfpathlineto{\pgfqpoint{2.610462in}{2.097669in}}%
\pgfpathlineto{\pgfqpoint{2.596884in}{2.106943in}}%
\pgfpathlineto{\pgfqpoint{2.588320in}{2.101177in}}%
\pgfpathlineto{\pgfqpoint{2.579745in}{2.095527in}}%
\pgfpathlineto{\pgfqpoint{2.571159in}{2.089994in}}%
\pgfpathlineto{\pgfqpoint{2.562562in}{2.084582in}}%
\pgfpathclose%
\pgfusepath{fill}%
\end{pgfscope}%
\begin{pgfscope}%
\pgfpathrectangle{\pgfqpoint{1.150000in}{0.150000in}}{\pgfqpoint{5.700000in}{5.700000in}}%
\pgfusepath{clip}%
\pgfsetbuttcap%
\pgfsetroundjoin%
\definecolor{currentfill}{rgb}{0.212395,0.359683,0.551710}%
\pgfsetfillcolor{currentfill}%
\pgfsetfillopacity{0.700000}%
\pgfsetlinewidth{0.000000pt}%
\definecolor{currentstroke}{rgb}{0.000000,0.000000,0.000000}%
\pgfsetstrokecolor{currentstroke}%
\pgfsetdash{}{0pt}%
\pgfpathmoveto{\pgfqpoint{5.888542in}{2.642860in}}%
\pgfpathlineto{\pgfqpoint{5.902925in}{2.644340in}}%
\pgfpathlineto{\pgfqpoint{5.917320in}{2.645885in}}%
\pgfpathlineto{\pgfqpoint{5.931727in}{2.647497in}}%
\pgfpathlineto{\pgfqpoint{5.938901in}{2.652587in}}%
\pgfpathlineto{\pgfqpoint{5.946075in}{2.657828in}}%
\pgfpathlineto{\pgfqpoint{5.953248in}{2.663227in}}%
\pgfpathlineto{\pgfqpoint{5.960421in}{2.668791in}}%
\pgfpathlineto{\pgfqpoint{5.946044in}{2.667596in}}%
\pgfpathlineto{\pgfqpoint{5.931678in}{2.666467in}}%
\pgfpathlineto{\pgfqpoint{5.917324in}{2.665404in}}%
\pgfpathlineto{\pgfqpoint{5.910129in}{2.659522in}}%
\pgfpathlineto{\pgfqpoint{5.902934in}{2.653809in}}%
\pgfpathlineto{\pgfqpoint{5.895738in}{2.648257in}}%
\pgfpathlineto{\pgfqpoint{5.888542in}{2.642860in}}%
\pgfpathclose%
\pgfusepath{fill}%
\end{pgfscope}%
\begin{pgfscope}%
\pgfpathrectangle{\pgfqpoint{1.150000in}{0.150000in}}{\pgfqpoint{5.700000in}{5.700000in}}%
\pgfusepath{clip}%
\pgfsetbuttcap%
\pgfsetroundjoin%
\definecolor{currentfill}{rgb}{0.250425,0.274290,0.533103}%
\pgfsetfillcolor{currentfill}%
\pgfsetfillopacity{0.700000}%
\pgfsetlinewidth{0.000000pt}%
\definecolor{currentstroke}{rgb}{0.000000,0.000000,0.000000}%
\pgfsetstrokecolor{currentstroke}%
\pgfsetdash{}{0pt}%
\pgfpathmoveto{\pgfqpoint{5.169343in}{2.434770in}}%
\pgfpathlineto{\pgfqpoint{5.183486in}{2.436155in}}%
\pgfpathlineto{\pgfqpoint{5.197641in}{2.437608in}}%
\pgfpathlineto{\pgfqpoint{5.211806in}{2.439130in}}%
\pgfpathlineto{\pgfqpoint{5.225982in}{2.440722in}}%
\pgfpathlineto{\pgfqpoint{5.233471in}{2.446526in}}%
\pgfpathlineto{\pgfqpoint{5.240955in}{2.452357in}}%
\pgfpathlineto{\pgfqpoint{5.248433in}{2.458218in}}%
\pgfpathlineto{\pgfqpoint{5.255905in}{2.464115in}}%
\pgfpathlineto{\pgfqpoint{5.241749in}{2.462796in}}%
\pgfpathlineto{\pgfqpoint{5.227603in}{2.461545in}}%
\pgfpathlineto{\pgfqpoint{5.213468in}{2.460363in}}%
\pgfpathlineto{\pgfqpoint{5.199343in}{2.459249in}}%
\pgfpathlineto{\pgfqpoint{5.191851in}{2.453073in}}%
\pgfpathlineto{\pgfqpoint{5.184353in}{2.446938in}}%
\pgfpathlineto{\pgfqpoint{5.176851in}{2.440838in}}%
\pgfpathlineto{\pgfqpoint{5.169343in}{2.434770in}}%
\pgfpathclose%
\pgfusepath{fill}%
\end{pgfscope}%
\begin{pgfscope}%
\pgfpathrectangle{\pgfqpoint{1.150000in}{0.150000in}}{\pgfqpoint{5.700000in}{5.700000in}}%
\pgfusepath{clip}%
\pgfsetbuttcap%
\pgfsetroundjoin%
\definecolor{currentfill}{rgb}{0.277941,0.056324,0.381191}%
\pgfsetfillcolor{currentfill}%
\pgfsetfillopacity{0.700000}%
\pgfsetlinewidth{0.000000pt}%
\definecolor{currentstroke}{rgb}{0.000000,0.000000,0.000000}%
\pgfsetstrokecolor{currentstroke}%
\pgfsetdash{}{0pt}%
\pgfpathmoveto{\pgfqpoint{2.759858in}{2.003009in}}%
\pgfpathlineto{\pgfqpoint{2.773446in}{1.995052in}}%
\pgfpathlineto{\pgfqpoint{2.787035in}{1.987200in}}%
\pgfpathlineto{\pgfqpoint{2.800625in}{1.979452in}}%
\pgfpathlineto{\pgfqpoint{2.814218in}{1.971807in}}%
\pgfpathlineto{\pgfqpoint{2.822685in}{1.978464in}}%
\pgfpathlineto{\pgfqpoint{2.831142in}{1.985209in}}%
\pgfpathlineto{\pgfqpoint{2.839591in}{1.992041in}}%
\pgfpathlineto{\pgfqpoint{2.848031in}{1.998957in}}%
\pgfpathlineto{\pgfqpoint{2.834459in}{2.006416in}}%
\pgfpathlineto{\pgfqpoint{2.820889in}{2.013978in}}%
\pgfpathlineto{\pgfqpoint{2.807320in}{2.021644in}}%
\pgfpathlineto{\pgfqpoint{2.793753in}{2.029415in}}%
\pgfpathlineto{\pgfqpoint{2.785293in}{2.022677in}}%
\pgfpathlineto{\pgfqpoint{2.776824in}{2.016029in}}%
\pgfpathlineto{\pgfqpoint{2.768346in}{2.009472in}}%
\pgfpathlineto{\pgfqpoint{2.759858in}{2.003009in}}%
\pgfpathclose%
\pgfusepath{fill}%
\end{pgfscope}%
\begin{pgfscope}%
\pgfpathrectangle{\pgfqpoint{1.150000in}{0.150000in}}{\pgfqpoint{5.700000in}{5.700000in}}%
\pgfusepath{clip}%
\pgfsetbuttcap%
\pgfsetroundjoin%
\definecolor{currentfill}{rgb}{0.282623,0.140926,0.457517}%
\pgfsetfillcolor{currentfill}%
\pgfsetfillopacity{0.700000}%
\pgfsetlinewidth{0.000000pt}%
\definecolor{currentstroke}{rgb}{0.000000,0.000000,0.000000}%
\pgfsetstrokecolor{currentstroke}%
\pgfsetdash{}{0pt}%
\pgfpathmoveto{\pgfqpoint{4.276574in}{2.145686in}}%
\pgfpathlineto{\pgfqpoint{4.290419in}{2.145426in}}%
\pgfpathlineto{\pgfqpoint{4.304273in}{2.145240in}}%
\pgfpathlineto{\pgfqpoint{4.318135in}{2.145129in}}%
\pgfpathlineto{\pgfqpoint{4.332006in}{2.145092in}}%
\pgfpathlineto{\pgfqpoint{4.339876in}{2.153333in}}%
\pgfpathlineto{\pgfqpoint{4.347739in}{2.161538in}}%
\pgfpathlineto{\pgfqpoint{4.355596in}{2.169709in}}%
\pgfpathlineto{\pgfqpoint{4.363448in}{2.177847in}}%
\pgfpathlineto{\pgfqpoint{4.349588in}{2.177968in}}%
\pgfpathlineto{\pgfqpoint{4.335737in}{2.178162in}}%
\pgfpathlineto{\pgfqpoint{4.321895in}{2.178430in}}%
\pgfpathlineto{\pgfqpoint{4.308061in}{2.178773in}}%
\pgfpathlineto{\pgfqpoint{4.300198in}{2.170544in}}%
\pgfpathlineto{\pgfqpoint{4.292329in}{2.162287in}}%
\pgfpathlineto{\pgfqpoint{4.284454in}{2.154002in}}%
\pgfpathlineto{\pgfqpoint{4.276574in}{2.145686in}}%
\pgfpathclose%
\pgfusepath{fill}%
\end{pgfscope}%
\begin{pgfscope}%
\pgfpathrectangle{\pgfqpoint{1.150000in}{0.150000in}}{\pgfqpoint{5.700000in}{5.700000in}}%
\pgfusepath{clip}%
\pgfsetbuttcap%
\pgfsetroundjoin%
\definecolor{currentfill}{rgb}{0.280894,0.078907,0.402329}%
\pgfsetfillcolor{currentfill}%
\pgfsetfillopacity{0.700000}%
\pgfsetlinewidth{0.000000pt}%
\definecolor{currentstroke}{rgb}{0.000000,0.000000,0.000000}%
\pgfsetstrokecolor{currentstroke}%
\pgfsetdash{}{0pt}%
\pgfpathmoveto{\pgfqpoint{3.873623in}{2.025904in}}%
\pgfpathlineto{\pgfqpoint{3.887354in}{2.024241in}}%
\pgfpathlineto{\pgfqpoint{3.901092in}{2.022656in}}%
\pgfpathlineto{\pgfqpoint{3.914837in}{2.021149in}}%
\pgfpathlineto{\pgfqpoint{3.928589in}{2.019721in}}%
\pgfpathlineto{\pgfqpoint{3.936605in}{2.028676in}}%
\pgfpathlineto{\pgfqpoint{3.944616in}{2.037604in}}%
\pgfpathlineto{\pgfqpoint{3.952620in}{2.046505in}}%
\pgfpathlineto{\pgfqpoint{3.960619in}{2.055379in}}%
\pgfpathlineto{\pgfqpoint{3.946878in}{2.056808in}}%
\pgfpathlineto{\pgfqpoint{3.933143in}{2.058314in}}%
\pgfpathlineto{\pgfqpoint{3.919416in}{2.059899in}}%
\pgfpathlineto{\pgfqpoint{3.905696in}{2.061563in}}%
\pgfpathlineto{\pgfqpoint{3.897686in}{2.052681in}}%
\pgfpathlineto{\pgfqpoint{3.889671in}{2.043777in}}%
\pgfpathlineto{\pgfqpoint{3.881650in}{2.034852in}}%
\pgfpathlineto{\pgfqpoint{3.873623in}{2.025904in}}%
\pgfpathclose%
\pgfusepath{fill}%
\end{pgfscope}%
\begin{pgfscope}%
\pgfpathrectangle{\pgfqpoint{1.150000in}{0.150000in}}{\pgfqpoint{5.700000in}{5.700000in}}%
\pgfusepath{clip}%
\pgfsetbuttcap%
\pgfsetroundjoin%
\definecolor{currentfill}{rgb}{0.274952,0.037752,0.364543}%
\pgfsetfillcolor{currentfill}%
\pgfsetfillopacity{0.700000}%
\pgfsetlinewidth{0.000000pt}%
\definecolor{currentstroke}{rgb}{0.000000,0.000000,0.000000}%
\pgfsetstrokecolor{currentstroke}%
\pgfsetdash{}{0pt}%
\pgfpathmoveto{\pgfqpoint{3.557556in}{1.956192in}}%
\pgfpathlineto{\pgfqpoint{3.571215in}{1.953114in}}%
\pgfpathlineto{\pgfqpoint{3.584881in}{1.950120in}}%
\pgfpathlineto{\pgfqpoint{3.598553in}{1.947208in}}%
\pgfpathlineto{\pgfqpoint{3.612230in}{1.944378in}}%
\pgfpathlineto{\pgfqpoint{3.620359in}{1.953392in}}%
\pgfpathlineto{\pgfqpoint{3.628482in}{1.962400in}}%
\pgfpathlineto{\pgfqpoint{3.636599in}{1.971402in}}%
\pgfpathlineto{\pgfqpoint{3.644710in}{1.980396in}}%
\pgfpathlineto{\pgfqpoint{3.631044in}{1.983165in}}%
\pgfpathlineto{\pgfqpoint{3.617384in}{1.986015in}}%
\pgfpathlineto{\pgfqpoint{3.603730in}{1.988948in}}%
\pgfpathlineto{\pgfqpoint{3.590082in}{1.991964in}}%
\pgfpathlineto{\pgfqpoint{3.581959in}{1.983024in}}%
\pgfpathlineto{\pgfqpoint{3.573831in}{1.974081in}}%
\pgfpathlineto{\pgfqpoint{3.565696in}{1.965137in}}%
\pgfpathlineto{\pgfqpoint{3.557556in}{1.956192in}}%
\pgfpathclose%
\pgfusepath{fill}%
\end{pgfscope}%
\begin{pgfscope}%
\pgfpathrectangle{\pgfqpoint{1.150000in}{0.150000in}}{\pgfqpoint{5.700000in}{5.700000in}}%
\pgfusepath{clip}%
\pgfsetbuttcap%
\pgfsetroundjoin%
\definecolor{currentfill}{rgb}{0.272594,0.025563,0.353093}%
\pgfsetfillcolor{currentfill}%
\pgfsetfillopacity{0.700000}%
\pgfsetlinewidth{0.000000pt}%
\definecolor{currentstroke}{rgb}{0.000000,0.000000,0.000000}%
\pgfsetstrokecolor{currentstroke}%
\pgfsetdash{}{0pt}%
\pgfpathmoveto{\pgfqpoint{3.328428in}{1.930433in}}%
\pgfpathlineto{\pgfqpoint{3.342049in}{1.926149in}}%
\pgfpathlineto{\pgfqpoint{3.355675in}{1.921952in}}%
\pgfpathlineto{\pgfqpoint{3.369306in}{1.917842in}}%
\pgfpathlineto{\pgfqpoint{3.382942in}{1.913818in}}%
\pgfpathlineto{\pgfqpoint{3.391156in}{1.922519in}}%
\pgfpathlineto{\pgfqpoint{3.399364in}{1.931238in}}%
\pgfpathlineto{\pgfqpoint{3.407566in}{1.939971in}}%
\pgfpathlineto{\pgfqpoint{3.415761in}{1.948720in}}%
\pgfpathlineto{\pgfqpoint{3.402139in}{1.952641in}}%
\pgfpathlineto{\pgfqpoint{3.388521in}{1.956649in}}%
\pgfpathlineto{\pgfqpoint{3.374909in}{1.960743in}}%
\pgfpathlineto{\pgfqpoint{3.361301in}{1.964925in}}%
\pgfpathlineto{\pgfqpoint{3.353092in}{1.956272in}}%
\pgfpathlineto{\pgfqpoint{3.344877in}{1.947638in}}%
\pgfpathlineto{\pgfqpoint{3.336656in}{1.939025in}}%
\pgfpathlineto{\pgfqpoint{3.328428in}{1.930433in}}%
\pgfpathclose%
\pgfusepath{fill}%
\end{pgfscope}%
\begin{pgfscope}%
\pgfpathrectangle{\pgfqpoint{1.150000in}{0.150000in}}{\pgfqpoint{5.700000in}{5.700000in}}%
\pgfusepath{clip}%
\pgfsetbuttcap%
\pgfsetroundjoin%
\definecolor{currentfill}{rgb}{0.273006,0.204520,0.501721}%
\pgfsetfillcolor{currentfill}%
\pgfsetfillopacity{0.700000}%
\pgfsetlinewidth{0.000000pt}%
\definecolor{currentstroke}{rgb}{0.000000,0.000000,0.000000}%
\pgfsetstrokecolor{currentstroke}%
\pgfsetdash{}{0pt}%
\pgfpathmoveto{\pgfqpoint{4.679597in}{2.275840in}}%
\pgfpathlineto{\pgfqpoint{4.693576in}{2.276582in}}%
\pgfpathlineto{\pgfqpoint{4.707565in}{2.277396in}}%
\pgfpathlineto{\pgfqpoint{4.721563in}{2.278280in}}%
\pgfpathlineto{\pgfqpoint{4.735571in}{2.279237in}}%
\pgfpathlineto{\pgfqpoint{4.743280in}{2.286374in}}%
\pgfpathlineto{\pgfqpoint{4.750983in}{2.293489in}}%
\pgfpathlineto{\pgfqpoint{4.758679in}{2.300584in}}%
\pgfpathlineto{\pgfqpoint{4.766369in}{2.307663in}}%
\pgfpathlineto{\pgfqpoint{4.752375in}{2.306874in}}%
\pgfpathlineto{\pgfqpoint{4.738391in}{2.306156in}}%
\pgfpathlineto{\pgfqpoint{4.724416in}{2.305509in}}%
\pgfpathlineto{\pgfqpoint{4.710451in}{2.304934in}}%
\pgfpathlineto{\pgfqpoint{4.702747in}{2.297680in}}%
\pgfpathlineto{\pgfqpoint{4.695037in}{2.290416in}}%
\pgfpathlineto{\pgfqpoint{4.687320in}{2.283137in}}%
\pgfpathlineto{\pgfqpoint{4.679597in}{2.275840in}}%
\pgfpathclose%
\pgfusepath{fill}%
\end{pgfscope}%
\begin{pgfscope}%
\pgfpathrectangle{\pgfqpoint{1.150000in}{0.150000in}}{\pgfqpoint{5.700000in}{5.700000in}}%
\pgfusepath{clip}%
\pgfsetbuttcap%
\pgfsetroundjoin%
\definecolor{currentfill}{rgb}{0.231674,0.318106,0.544834}%
\pgfsetfillcolor{currentfill}%
\pgfsetfillopacity{0.700000}%
\pgfsetlinewidth{0.000000pt}%
\definecolor{currentstroke}{rgb}{0.000000,0.000000,0.000000}%
\pgfsetstrokecolor{currentstroke}%
\pgfsetdash{}{0pt}%
\pgfpathmoveto{\pgfqpoint{5.485743in}{2.527117in}}%
\pgfpathlineto{\pgfqpoint{5.500000in}{2.528718in}}%
\pgfpathlineto{\pgfqpoint{5.514268in}{2.530386in}}%
\pgfpathlineto{\pgfqpoint{5.528547in}{2.532122in}}%
\pgfpathlineto{\pgfqpoint{5.542838in}{2.533926in}}%
\pgfpathlineto{\pgfqpoint{5.550182in}{2.539080in}}%
\pgfpathlineto{\pgfqpoint{5.557521in}{2.544304in}}%
\pgfpathlineto{\pgfqpoint{5.564856in}{2.549603in}}%
\pgfpathlineto{\pgfqpoint{5.572187in}{2.554985in}}%
\pgfpathlineto{\pgfqpoint{5.557920in}{2.553516in}}%
\pgfpathlineto{\pgfqpoint{5.543664in}{2.552114in}}%
\pgfpathlineto{\pgfqpoint{5.529419in}{2.550779in}}%
\pgfpathlineto{\pgfqpoint{5.515186in}{2.549512in}}%
\pgfpathlineto{\pgfqpoint{5.507831in}{2.543789in}}%
\pgfpathlineto{\pgfqpoint{5.500472in}{2.538153in}}%
\pgfpathlineto{\pgfqpoint{5.493109in}{2.532598in}}%
\pgfpathlineto{\pgfqpoint{5.485743in}{2.527117in}}%
\pgfpathclose%
\pgfusepath{fill}%
\end{pgfscope}%
\begin{pgfscope}%
\pgfpathrectangle{\pgfqpoint{1.150000in}{0.150000in}}{\pgfqpoint{5.700000in}{5.700000in}}%
\pgfusepath{clip}%
\pgfsetbuttcap%
\pgfsetroundjoin%
\definecolor{currentfill}{rgb}{0.253935,0.265254,0.529983}%
\pgfsetfillcolor{currentfill}%
\pgfsetfillopacity{0.700000}%
\pgfsetlinewidth{0.000000pt}%
\definecolor{currentstroke}{rgb}{0.000000,0.000000,0.000000}%
\pgfsetstrokecolor{currentstroke}%
\pgfsetdash{}{0pt}%
\pgfpathmoveto{\pgfqpoint{5.082713in}{2.404865in}}%
\pgfpathlineto{\pgfqpoint{5.096832in}{2.406224in}}%
\pgfpathlineto{\pgfqpoint{5.110962in}{2.407651in}}%
\pgfpathlineto{\pgfqpoint{5.125103in}{2.409148in}}%
\pgfpathlineto{\pgfqpoint{5.139255in}{2.410715in}}%
\pgfpathlineto{\pgfqpoint{5.146785in}{2.416705in}}%
\pgfpathlineto{\pgfqpoint{5.154310in}{2.422708in}}%
\pgfpathlineto{\pgfqpoint{5.161829in}{2.428728in}}%
\pgfpathlineto{\pgfqpoint{5.169343in}{2.434770in}}%
\pgfpathlineto{\pgfqpoint{5.155210in}{2.433455in}}%
\pgfpathlineto{\pgfqpoint{5.141087in}{2.432209in}}%
\pgfpathlineto{\pgfqpoint{5.126975in}{2.431032in}}%
\pgfpathlineto{\pgfqpoint{5.112874in}{2.429924in}}%
\pgfpathlineto{\pgfqpoint{5.105342in}{2.423624in}}%
\pgfpathlineto{\pgfqpoint{5.097804in}{2.417350in}}%
\pgfpathlineto{\pgfqpoint{5.090262in}{2.411099in}}%
\pgfpathlineto{\pgfqpoint{5.082713in}{2.404865in}}%
\pgfpathclose%
\pgfusepath{fill}%
\end{pgfscope}%
\begin{pgfscope}%
\pgfpathrectangle{\pgfqpoint{1.150000in}{0.150000in}}{\pgfqpoint{5.700000in}{5.700000in}}%
\pgfusepath{clip}%
\pgfsetbuttcap%
\pgfsetroundjoin%
\definecolor{currentfill}{rgb}{0.281887,0.150881,0.465405}%
\pgfsetfillcolor{currentfill}%
\pgfsetfillopacity{0.700000}%
\pgfsetlinewidth{0.000000pt}%
\definecolor{currentstroke}{rgb}{0.000000,0.000000,0.000000}%
\pgfsetstrokecolor{currentstroke}%
\pgfsetdash{}{0pt}%
\pgfpathmoveto{\pgfqpoint{2.364481in}{2.190030in}}%
\pgfpathlineto{\pgfqpoint{2.378120in}{2.178869in}}%
\pgfpathlineto{\pgfqpoint{2.391757in}{2.167835in}}%
\pgfpathlineto{\pgfqpoint{2.405392in}{2.156926in}}%
\pgfpathlineto{\pgfqpoint{2.419026in}{2.146141in}}%
\pgfpathlineto{\pgfqpoint{2.427721in}{2.150561in}}%
\pgfpathlineto{\pgfqpoint{2.436404in}{2.155125in}}%
\pgfpathlineto{\pgfqpoint{2.445075in}{2.159829in}}%
\pgfpathlineto{\pgfqpoint{2.453734in}{2.164672in}}%
\pgfpathlineto{\pgfqpoint{2.440127in}{2.175226in}}%
\pgfpathlineto{\pgfqpoint{2.426519in}{2.185903in}}%
\pgfpathlineto{\pgfqpoint{2.412909in}{2.196706in}}%
\pgfpathlineto{\pgfqpoint{2.399298in}{2.207634in}}%
\pgfpathlineto{\pgfqpoint{2.390613in}{2.203015in}}%
\pgfpathlineto{\pgfqpoint{2.381915in}{2.198539in}}%
\pgfpathlineto{\pgfqpoint{2.373204in}{2.194210in}}%
\pgfpathlineto{\pgfqpoint{2.364481in}{2.190030in}}%
\pgfpathclose%
\pgfusepath{fill}%
\end{pgfscope}%
\begin{pgfscope}%
\pgfpathrectangle{\pgfqpoint{1.150000in}{0.150000in}}{\pgfqpoint{5.700000in}{5.700000in}}%
\pgfusepath{clip}%
\pgfsetbuttcap%
\pgfsetroundjoin%
\definecolor{currentfill}{rgb}{0.283072,0.130895,0.449241}%
\pgfsetfillcolor{currentfill}%
\pgfsetfillopacity{0.700000}%
\pgfsetlinewidth{0.000000pt}%
\definecolor{currentstroke}{rgb}{0.000000,0.000000,0.000000}%
\pgfsetstrokecolor{currentstroke}%
\pgfsetdash{}{0pt}%
\pgfpathmoveto{\pgfqpoint{4.189652in}{2.113617in}}%
\pgfpathlineto{\pgfqpoint{4.203474in}{2.113121in}}%
\pgfpathlineto{\pgfqpoint{4.217305in}{2.112699in}}%
\pgfpathlineto{\pgfqpoint{4.231144in}{2.112352in}}%
\pgfpathlineto{\pgfqpoint{4.244992in}{2.112080in}}%
\pgfpathlineto{\pgfqpoint{4.252896in}{2.120535in}}%
\pgfpathlineto{\pgfqpoint{4.260795in}{2.128954in}}%
\pgfpathlineto{\pgfqpoint{4.268687in}{2.137337in}}%
\pgfpathlineto{\pgfqpoint{4.276574in}{2.145686in}}%
\pgfpathlineto{\pgfqpoint{4.262737in}{2.146020in}}%
\pgfpathlineto{\pgfqpoint{4.248909in}{2.146429in}}%
\pgfpathlineto{\pgfqpoint{4.235089in}{2.146913in}}%
\pgfpathlineto{\pgfqpoint{4.221277in}{2.147472in}}%
\pgfpathlineto{\pgfqpoint{4.213380in}{2.139054in}}%
\pgfpathlineto{\pgfqpoint{4.205476in}{2.130606in}}%
\pgfpathlineto{\pgfqpoint{4.197567in}{2.122128in}}%
\pgfpathlineto{\pgfqpoint{4.189652in}{2.113617in}}%
\pgfpathclose%
\pgfusepath{fill}%
\end{pgfscope}%
\begin{pgfscope}%
\pgfpathrectangle{\pgfqpoint{1.150000in}{0.150000in}}{\pgfqpoint{5.700000in}{5.700000in}}%
\pgfusepath{clip}%
\pgfsetbuttcap%
\pgfsetroundjoin%
\definecolor{currentfill}{rgb}{0.279566,0.067836,0.391917}%
\pgfsetfillcolor{currentfill}%
\pgfsetfillopacity{0.700000}%
\pgfsetlinewidth{0.000000pt}%
\definecolor{currentstroke}{rgb}{0.000000,0.000000,0.000000}%
\pgfsetstrokecolor{currentstroke}%
\pgfsetdash{}{0pt}%
\pgfpathmoveto{\pgfqpoint{3.786564in}{1.997407in}}%
\pgfpathlineto{\pgfqpoint{3.800278in}{1.995407in}}%
\pgfpathlineto{\pgfqpoint{3.813998in}{1.993487in}}%
\pgfpathlineto{\pgfqpoint{3.827725in}{1.991645in}}%
\pgfpathlineto{\pgfqpoint{3.841460in}{1.989883in}}%
\pgfpathlineto{\pgfqpoint{3.849509in}{1.998923in}}%
\pgfpathlineto{\pgfqpoint{3.857553in}{2.007940in}}%
\pgfpathlineto{\pgfqpoint{3.865591in}{2.016934in}}%
\pgfpathlineto{\pgfqpoint{3.873623in}{2.025904in}}%
\pgfpathlineto{\pgfqpoint{3.859900in}{2.027646in}}%
\pgfpathlineto{\pgfqpoint{3.846183in}{2.029467in}}%
\pgfpathlineto{\pgfqpoint{3.832474in}{2.031367in}}%
\pgfpathlineto{\pgfqpoint{3.818771in}{2.033347in}}%
\pgfpathlineto{\pgfqpoint{3.810728in}{2.024389in}}%
\pgfpathlineto{\pgfqpoint{3.802679in}{2.015413in}}%
\pgfpathlineto{\pgfqpoint{3.794625in}{2.006419in}}%
\pgfpathlineto{\pgfqpoint{3.786564in}{1.997407in}}%
\pgfpathclose%
\pgfusepath{fill}%
\end{pgfscope}%
\begin{pgfscope}%
\pgfpathrectangle{\pgfqpoint{1.150000in}{0.150000in}}{\pgfqpoint{5.700000in}{5.700000in}}%
\pgfusepath{clip}%
\pgfsetbuttcap%
\pgfsetroundjoin%
\definecolor{currentfill}{rgb}{0.276194,0.190074,0.493001}%
\pgfsetfillcolor{currentfill}%
\pgfsetfillopacity{0.700000}%
\pgfsetlinewidth{0.000000pt}%
\definecolor{currentstroke}{rgb}{0.000000,0.000000,0.000000}%
\pgfsetstrokecolor{currentstroke}%
\pgfsetdash{}{0pt}%
\pgfpathmoveto{\pgfqpoint{4.592773in}{2.243586in}}%
\pgfpathlineto{\pgfqpoint{4.606727in}{2.244187in}}%
\pgfpathlineto{\pgfqpoint{4.620690in}{2.244859in}}%
\pgfpathlineto{\pgfqpoint{4.634663in}{2.245603in}}%
\pgfpathlineto{\pgfqpoint{4.648646in}{2.246420in}}%
\pgfpathlineto{\pgfqpoint{4.656393in}{2.253816in}}%
\pgfpathlineto{\pgfqpoint{4.664134in}{2.261183in}}%
\pgfpathlineto{\pgfqpoint{4.671869in}{2.268523in}}%
\pgfpathlineto{\pgfqpoint{4.679597in}{2.275840in}}%
\pgfpathlineto{\pgfqpoint{4.665628in}{2.275170in}}%
\pgfpathlineto{\pgfqpoint{4.651669in}{2.274572in}}%
\pgfpathlineto{\pgfqpoint{4.637719in}{2.274045in}}%
\pgfpathlineto{\pgfqpoint{4.623778in}{2.273591in}}%
\pgfpathlineto{\pgfqpoint{4.616036in}{2.266120in}}%
\pgfpathlineto{\pgfqpoint{4.608288in}{2.258631in}}%
\pgfpathlineto{\pgfqpoint{4.600533in}{2.251121in}}%
\pgfpathlineto{\pgfqpoint{4.592773in}{2.243586in}}%
\pgfpathclose%
\pgfusepath{fill}%
\end{pgfscope}%
\begin{pgfscope}%
\pgfpathrectangle{\pgfqpoint{1.150000in}{0.150000in}}{\pgfqpoint{5.700000in}{5.700000in}}%
\pgfusepath{clip}%
\pgfsetbuttcap%
\pgfsetroundjoin%
\definecolor{currentfill}{rgb}{0.281446,0.084320,0.407414}%
\pgfsetfillcolor{currentfill}%
\pgfsetfillopacity{0.700000}%
\pgfsetlinewidth{0.000000pt}%
\definecolor{currentstroke}{rgb}{0.000000,0.000000,0.000000}%
\pgfsetstrokecolor{currentstroke}%
\pgfsetdash{}{0pt}%
\pgfpathmoveto{\pgfqpoint{2.616970in}{2.047333in}}%
\pgfpathlineto{\pgfqpoint{2.630573in}{2.038303in}}%
\pgfpathlineto{\pgfqpoint{2.644176in}{2.029383in}}%
\pgfpathlineto{\pgfqpoint{2.657780in}{2.020574in}}%
\pgfpathlineto{\pgfqpoint{2.671385in}{2.011874in}}%
\pgfpathlineto{\pgfqpoint{2.679934in}{2.017718in}}%
\pgfpathlineto{\pgfqpoint{2.688472in}{2.023672in}}%
\pgfpathlineto{\pgfqpoint{2.697001in}{2.029734in}}%
\pgfpathlineto{\pgfqpoint{2.705520in}{2.035900in}}%
\pgfpathlineto{\pgfqpoint{2.691939in}{2.044393in}}%
\pgfpathlineto{\pgfqpoint{2.678358in}{2.052994in}}%
\pgfpathlineto{\pgfqpoint{2.664778in}{2.061706in}}%
\pgfpathlineto{\pgfqpoint{2.651198in}{2.070528in}}%
\pgfpathlineto{\pgfqpoint{2.642657in}{2.064562in}}%
\pgfpathlineto{\pgfqpoint{2.634105in}{2.058705in}}%
\pgfpathlineto{\pgfqpoint{2.625543in}{2.052961in}}%
\pgfpathlineto{\pgfqpoint{2.616970in}{2.047333in}}%
\pgfpathclose%
\pgfusepath{fill}%
\end{pgfscope}%
\begin{pgfscope}%
\pgfpathrectangle{\pgfqpoint{1.150000in}{0.150000in}}{\pgfqpoint{5.700000in}{5.700000in}}%
\pgfusepath{clip}%
\pgfsetbuttcap%
\pgfsetroundjoin%
\definecolor{currentfill}{rgb}{0.214298,0.355619,0.551184}%
\pgfsetfillcolor{currentfill}%
\pgfsetfillopacity{0.700000}%
\pgfsetlinewidth{0.000000pt}%
\definecolor{currentstroke}{rgb}{0.000000,0.000000,0.000000}%
\pgfsetstrokecolor{currentstroke}%
\pgfsetdash{}{0pt}%
\pgfpathmoveto{\pgfqpoint{5.802217in}{2.615839in}}%
\pgfpathlineto{\pgfqpoint{5.816582in}{2.617449in}}%
\pgfpathlineto{\pgfqpoint{5.830958in}{2.619125in}}%
\pgfpathlineto{\pgfqpoint{5.845346in}{2.620869in}}%
\pgfpathlineto{\pgfqpoint{5.859746in}{2.622679in}}%
\pgfpathlineto{\pgfqpoint{5.866948in}{2.627527in}}%
\pgfpathlineto{\pgfqpoint{5.874147in}{2.632502in}}%
\pgfpathlineto{\pgfqpoint{5.881345in}{2.637611in}}%
\pgfpathlineto{\pgfqpoint{5.888542in}{2.642860in}}%
\pgfpathlineto{\pgfqpoint{5.874171in}{2.641448in}}%
\pgfpathlineto{\pgfqpoint{5.859811in}{2.640101in}}%
\pgfpathlineto{\pgfqpoint{5.845462in}{2.638821in}}%
\pgfpathlineto{\pgfqpoint{5.831125in}{2.637607in}}%
\pgfpathlineto{\pgfqpoint{5.823900in}{2.631954in}}%
\pgfpathlineto{\pgfqpoint{5.816674in}{2.626446in}}%
\pgfpathlineto{\pgfqpoint{5.809446in}{2.621077in}}%
\pgfpathlineto{\pgfqpoint{5.802217in}{2.615839in}}%
\pgfpathclose%
\pgfusepath{fill}%
\end{pgfscope}%
\begin{pgfscope}%
\pgfpathrectangle{\pgfqpoint{1.150000in}{0.150000in}}{\pgfqpoint{5.700000in}{5.700000in}}%
\pgfusepath{clip}%
\pgfsetbuttcap%
\pgfsetroundjoin%
\definecolor{currentfill}{rgb}{0.273809,0.031497,0.358853}%
\pgfsetfillcolor{currentfill}%
\pgfsetfillopacity{0.700000}%
\pgfsetlinewidth{0.000000pt}%
\definecolor{currentstroke}{rgb}{0.000000,0.000000,0.000000}%
\pgfsetstrokecolor{currentstroke}%
\pgfsetdash{}{0pt}%
\pgfpathmoveto{\pgfqpoint{2.956686in}{1.942929in}}%
\pgfpathlineto{\pgfqpoint{2.970279in}{1.936371in}}%
\pgfpathlineto{\pgfqpoint{2.983875in}{1.929910in}}%
\pgfpathlineto{\pgfqpoint{2.997474in}{1.923546in}}%
\pgfpathlineto{\pgfqpoint{3.011076in}{1.917277in}}%
\pgfpathlineto{\pgfqpoint{3.019451in}{1.924792in}}%
\pgfpathlineto{\pgfqpoint{3.027819in}{1.932371in}}%
\pgfpathlineto{\pgfqpoint{3.036178in}{1.940011in}}%
\pgfpathlineto{\pgfqpoint{3.044530in}{1.947710in}}%
\pgfpathlineto{\pgfqpoint{3.030946in}{1.953814in}}%
\pgfpathlineto{\pgfqpoint{3.017365in}{1.960014in}}%
\pgfpathlineto{\pgfqpoint{3.003787in}{1.966310in}}%
\pgfpathlineto{\pgfqpoint{2.990212in}{1.972703in}}%
\pgfpathlineto{\pgfqpoint{2.981843in}{1.965161in}}%
\pgfpathlineto{\pgfqpoint{2.973465in}{1.957683in}}%
\pgfpathlineto{\pgfqpoint{2.965080in}{1.950271in}}%
\pgfpathlineto{\pgfqpoint{2.956686in}{1.942929in}}%
\pgfpathclose%
\pgfusepath{fill}%
\end{pgfscope}%
\begin{pgfscope}%
\pgfpathrectangle{\pgfqpoint{1.150000in}{0.150000in}}{\pgfqpoint{5.700000in}{5.700000in}}%
\pgfusepath{clip}%
\pgfsetbuttcap%
\pgfsetroundjoin%
\definecolor{currentfill}{rgb}{0.271305,0.019942,0.347269}%
\pgfsetfillcolor{currentfill}%
\pgfsetfillopacity{0.700000}%
\pgfsetlinewidth{0.000000pt}%
\definecolor{currentstroke}{rgb}{0.000000,0.000000,0.000000}%
\pgfsetstrokecolor{currentstroke}%
\pgfsetdash{}{0pt}%
\pgfpathmoveto{\pgfqpoint{3.098900in}{1.924243in}}%
\pgfpathlineto{\pgfqpoint{3.112501in}{1.918610in}}%
\pgfpathlineto{\pgfqpoint{3.126106in}{1.913070in}}%
\pgfpathlineto{\pgfqpoint{3.139714in}{1.907623in}}%
\pgfpathlineto{\pgfqpoint{3.153326in}{1.902267in}}%
\pgfpathlineto{\pgfqpoint{3.161638in}{1.910326in}}%
\pgfpathlineto{\pgfqpoint{3.169942in}{1.918431in}}%
\pgfpathlineto{\pgfqpoint{3.178239in}{1.926578in}}%
\pgfpathlineto{\pgfqpoint{3.186530in}{1.934767in}}%
\pgfpathlineto{\pgfqpoint{3.172933in}{1.939979in}}%
\pgfpathlineto{\pgfqpoint{3.159341in}{1.945283in}}%
\pgfpathlineto{\pgfqpoint{3.145752in}{1.950679in}}%
\pgfpathlineto{\pgfqpoint{3.132167in}{1.956168in}}%
\pgfpathlineto{\pgfqpoint{3.123861in}{1.948115in}}%
\pgfpathlineto{\pgfqpoint{3.115548in}{1.940109in}}%
\pgfpathlineto{\pgfqpoint{3.107228in}{1.932151in}}%
\pgfpathlineto{\pgfqpoint{3.098900in}{1.924243in}}%
\pgfpathclose%
\pgfusepath{fill}%
\end{pgfscope}%
\begin{pgfscope}%
\pgfpathrectangle{\pgfqpoint{1.150000in}{0.150000in}}{\pgfqpoint{5.700000in}{5.700000in}}%
\pgfusepath{clip}%
\pgfsetbuttcap%
\pgfsetroundjoin%
\definecolor{currentfill}{rgb}{0.273809,0.031497,0.358853}%
\pgfsetfillcolor{currentfill}%
\pgfsetfillopacity{0.700000}%
\pgfsetlinewidth{0.000000pt}%
\definecolor{currentstroke}{rgb}{0.000000,0.000000,0.000000}%
\pgfsetstrokecolor{currentstroke}%
\pgfsetdash{}{0pt}%
\pgfpathmoveto{\pgfqpoint{3.470304in}{1.933892in}}%
\pgfpathlineto{\pgfqpoint{3.483954in}{1.930397in}}%
\pgfpathlineto{\pgfqpoint{3.497609in}{1.926987in}}%
\pgfpathlineto{\pgfqpoint{3.511269in}{1.923661in}}%
\pgfpathlineto{\pgfqpoint{3.524935in}{1.920418in}}%
\pgfpathlineto{\pgfqpoint{3.533099in}{1.929359in}}%
\pgfpathlineto{\pgfqpoint{3.541257in}{1.938302in}}%
\pgfpathlineto{\pgfqpoint{3.549409in}{1.947247in}}%
\pgfpathlineto{\pgfqpoint{3.557556in}{1.956192in}}%
\pgfpathlineto{\pgfqpoint{3.543902in}{1.959353in}}%
\pgfpathlineto{\pgfqpoint{3.530254in}{1.962597in}}%
\pgfpathlineto{\pgfqpoint{3.516611in}{1.965925in}}%
\pgfpathlineto{\pgfqpoint{3.502975in}{1.969338in}}%
\pgfpathlineto{\pgfqpoint{3.494816in}{1.960467in}}%
\pgfpathlineto{\pgfqpoint{3.486652in}{1.951602in}}%
\pgfpathlineto{\pgfqpoint{3.478481in}{1.942743in}}%
\pgfpathlineto{\pgfqpoint{3.470304in}{1.933892in}}%
\pgfpathclose%
\pgfusepath{fill}%
\end{pgfscope}%
\begin{pgfscope}%
\pgfpathrectangle{\pgfqpoint{1.150000in}{0.150000in}}{\pgfqpoint{5.700000in}{5.700000in}}%
\pgfusepath{clip}%
\pgfsetbuttcap%
\pgfsetroundjoin%
\definecolor{currentfill}{rgb}{0.258965,0.251537,0.524736}%
\pgfsetfillcolor{currentfill}%
\pgfsetfillopacity{0.700000}%
\pgfsetlinewidth{0.000000pt}%
\definecolor{currentstroke}{rgb}{0.000000,0.000000,0.000000}%
\pgfsetstrokecolor{currentstroke}%
\pgfsetdash{}{0pt}%
\pgfpathmoveto{\pgfqpoint{4.996018in}{2.374364in}}%
\pgfpathlineto{\pgfqpoint{5.010113in}{2.375673in}}%
\pgfpathlineto{\pgfqpoint{5.024218in}{2.377052in}}%
\pgfpathlineto{\pgfqpoint{5.038334in}{2.378502in}}%
\pgfpathlineto{\pgfqpoint{5.052460in}{2.380021in}}%
\pgfpathlineto{\pgfqpoint{5.060032in}{2.386227in}}%
\pgfpathlineto{\pgfqpoint{5.067598in}{2.392434in}}%
\pgfpathlineto{\pgfqpoint{5.075159in}{2.398645in}}%
\pgfpathlineto{\pgfqpoint{5.082713in}{2.404865in}}%
\pgfpathlineto{\pgfqpoint{5.068604in}{2.403577in}}%
\pgfpathlineto{\pgfqpoint{5.054506in}{2.402358in}}%
\pgfpathlineto{\pgfqpoint{5.040418in}{2.401208in}}%
\pgfpathlineto{\pgfqpoint{5.026340in}{2.400128in}}%
\pgfpathlineto{\pgfqpoint{5.018768in}{2.393671in}}%
\pgfpathlineto{\pgfqpoint{5.011191in}{2.387227in}}%
\pgfpathlineto{\pgfqpoint{5.003607in}{2.380792in}}%
\pgfpathlineto{\pgfqpoint{4.996018in}{2.374364in}}%
\pgfpathclose%
\pgfusepath{fill}%
\end{pgfscope}%
\begin{pgfscope}%
\pgfpathrectangle{\pgfqpoint{1.150000in}{0.150000in}}{\pgfqpoint{5.700000in}{5.700000in}}%
\pgfusepath{clip}%
\pgfsetbuttcap%
\pgfsetroundjoin%
\definecolor{currentfill}{rgb}{0.283197,0.115680,0.436115}%
\pgfsetfillcolor{currentfill}%
\pgfsetfillopacity{0.700000}%
\pgfsetlinewidth{0.000000pt}%
\definecolor{currentstroke}{rgb}{0.000000,0.000000,0.000000}%
\pgfsetstrokecolor{currentstroke}%
\pgfsetdash{}{0pt}%
\pgfpathmoveto{\pgfqpoint{4.102682in}{2.081807in}}%
\pgfpathlineto{\pgfqpoint{4.116482in}{2.081049in}}%
\pgfpathlineto{\pgfqpoint{4.130291in}{2.080367in}}%
\pgfpathlineto{\pgfqpoint{4.144108in}{2.079761in}}%
\pgfpathlineto{\pgfqpoint{4.157933in}{2.079231in}}%
\pgfpathlineto{\pgfqpoint{4.165871in}{2.087882in}}%
\pgfpathlineto{\pgfqpoint{4.173804in}{2.096496in}}%
\pgfpathlineto{\pgfqpoint{4.181731in}{2.105074in}}%
\pgfpathlineto{\pgfqpoint{4.189652in}{2.113617in}}%
\pgfpathlineto{\pgfqpoint{4.175837in}{2.114190in}}%
\pgfpathlineto{\pgfqpoint{4.162031in}{2.114837in}}%
\pgfpathlineto{\pgfqpoint{4.148233in}{2.115561in}}%
\pgfpathlineto{\pgfqpoint{4.134443in}{2.116360in}}%
\pgfpathlineto{\pgfqpoint{4.126512in}{2.107767in}}%
\pgfpathlineto{\pgfqpoint{4.118574in}{2.099145in}}%
\pgfpathlineto{\pgfqpoint{4.110631in}{2.090492in}}%
\pgfpathlineto{\pgfqpoint{4.102682in}{2.081807in}}%
\pgfpathclose%
\pgfusepath{fill}%
\end{pgfscope}%
\begin{pgfscope}%
\pgfpathrectangle{\pgfqpoint{1.150000in}{0.150000in}}{\pgfqpoint{5.700000in}{5.700000in}}%
\pgfusepath{clip}%
\pgfsetbuttcap%
\pgfsetroundjoin%
\definecolor{currentfill}{rgb}{0.235526,0.309527,0.542944}%
\pgfsetfillcolor{currentfill}%
\pgfsetfillopacity{0.700000}%
\pgfsetlinewidth{0.000000pt}%
\definecolor{currentstroke}{rgb}{0.000000,0.000000,0.000000}%
\pgfsetstrokecolor{currentstroke}%
\pgfsetdash{}{0pt}%
\pgfpathmoveto{\pgfqpoint{5.399227in}{2.498851in}}%
\pgfpathlineto{\pgfqpoint{5.413462in}{2.500494in}}%
\pgfpathlineto{\pgfqpoint{5.427707in}{2.502204in}}%
\pgfpathlineto{\pgfqpoint{5.441964in}{2.503983in}}%
\pgfpathlineto{\pgfqpoint{5.456233in}{2.505830in}}%
\pgfpathlineto{\pgfqpoint{5.463617in}{2.511068in}}%
\pgfpathlineto{\pgfqpoint{5.470997in}{2.516358in}}%
\pgfpathlineto{\pgfqpoint{5.478372in}{2.521706in}}%
\pgfpathlineto{\pgfqpoint{5.485743in}{2.527117in}}%
\pgfpathlineto{\pgfqpoint{5.471497in}{2.525584in}}%
\pgfpathlineto{\pgfqpoint{5.457262in}{2.524119in}}%
\pgfpathlineto{\pgfqpoint{5.443039in}{2.522722in}}%
\pgfpathlineto{\pgfqpoint{5.428827in}{2.521392in}}%
\pgfpathlineto{\pgfqpoint{5.421433in}{2.515660in}}%
\pgfpathlineto{\pgfqpoint{5.414036in}{2.509996in}}%
\pgfpathlineto{\pgfqpoint{5.406634in}{2.504395in}}%
\pgfpathlineto{\pgfqpoint{5.399227in}{2.498851in}}%
\pgfpathclose%
\pgfusepath{fill}%
\end{pgfscope}%
\begin{pgfscope}%
\pgfpathrectangle{\pgfqpoint{1.150000in}{0.150000in}}{\pgfqpoint{5.700000in}{5.700000in}}%
\pgfusepath{clip}%
\pgfsetbuttcap%
\pgfsetroundjoin%
\definecolor{currentfill}{rgb}{0.277018,0.050344,0.375715}%
\pgfsetfillcolor{currentfill}%
\pgfsetfillopacity{0.700000}%
\pgfsetlinewidth{0.000000pt}%
\definecolor{currentstroke}{rgb}{0.000000,0.000000,0.000000}%
\pgfsetstrokecolor{currentstroke}%
\pgfsetdash{}{0pt}%
\pgfpathmoveto{\pgfqpoint{2.814218in}{1.971807in}}%
\pgfpathlineto{\pgfqpoint{2.827812in}{1.964265in}}%
\pgfpathlineto{\pgfqpoint{2.841408in}{1.956825in}}%
\pgfpathlineto{\pgfqpoint{2.855006in}{1.949487in}}%
\pgfpathlineto{\pgfqpoint{2.868606in}{1.942249in}}%
\pgfpathlineto{\pgfqpoint{2.877053in}{1.949099in}}%
\pgfpathlineto{\pgfqpoint{2.885491in}{1.956033in}}%
\pgfpathlineto{\pgfqpoint{2.893920in}{1.963048in}}%
\pgfpathlineto{\pgfqpoint{2.902340in}{1.970142in}}%
\pgfpathlineto{\pgfqpoint{2.888760in}{1.977194in}}%
\pgfpathlineto{\pgfqpoint{2.875181in}{1.984347in}}%
\pgfpathlineto{\pgfqpoint{2.861605in}{1.991601in}}%
\pgfpathlineto{\pgfqpoint{2.848031in}{1.998957in}}%
\pgfpathlineto{\pgfqpoint{2.839591in}{1.992041in}}%
\pgfpathlineto{\pgfqpoint{2.831142in}{1.985209in}}%
\pgfpathlineto{\pgfqpoint{2.822685in}{1.978464in}}%
\pgfpathlineto{\pgfqpoint{2.814218in}{1.971807in}}%
\pgfpathclose%
\pgfusepath{fill}%
\end{pgfscope}%
\begin{pgfscope}%
\pgfpathrectangle{\pgfqpoint{1.150000in}{0.150000in}}{\pgfqpoint{5.700000in}{5.700000in}}%
\pgfusepath{clip}%
\pgfsetbuttcap%
\pgfsetroundjoin%
\definecolor{currentfill}{rgb}{0.282884,0.135920,0.453427}%
\pgfsetfillcolor{currentfill}%
\pgfsetfillopacity{0.700000}%
\pgfsetlinewidth{0.000000pt}%
\definecolor{currentstroke}{rgb}{0.000000,0.000000,0.000000}%
\pgfsetstrokecolor{currentstroke}%
\pgfsetdash{}{0pt}%
\pgfpathmoveto{\pgfqpoint{2.419026in}{2.146141in}}%
\pgfpathlineto{\pgfqpoint{2.432659in}{2.135479in}}%
\pgfpathlineto{\pgfqpoint{2.446290in}{2.124939in}}%
\pgfpathlineto{\pgfqpoint{2.459921in}{2.114520in}}%
\pgfpathlineto{\pgfqpoint{2.473551in}{2.104221in}}%
\pgfpathlineto{\pgfqpoint{2.482219in}{2.108880in}}%
\pgfpathlineto{\pgfqpoint{2.490875in}{2.113677in}}%
\pgfpathlineto{\pgfqpoint{2.499520in}{2.118610in}}%
\pgfpathlineto{\pgfqpoint{2.508153in}{2.123676in}}%
\pgfpathlineto{\pgfqpoint{2.494549in}{2.133744in}}%
\pgfpathlineto{\pgfqpoint{2.480945in}{2.143932in}}%
\pgfpathlineto{\pgfqpoint{2.467340in}{2.154241in}}%
\pgfpathlineto{\pgfqpoint{2.453734in}{2.164672in}}%
\pgfpathlineto{\pgfqpoint{2.445075in}{2.159829in}}%
\pgfpathlineto{\pgfqpoint{2.436404in}{2.155125in}}%
\pgfpathlineto{\pgfqpoint{2.427721in}{2.150561in}}%
\pgfpathlineto{\pgfqpoint{2.419026in}{2.146141in}}%
\pgfpathclose%
\pgfusepath{fill}%
\end{pgfscope}%
\begin{pgfscope}%
\pgfpathrectangle{\pgfqpoint{1.150000in}{0.150000in}}{\pgfqpoint{5.700000in}{5.700000in}}%
\pgfusepath{clip}%
\pgfsetbuttcap%
\pgfsetroundjoin%
\definecolor{currentfill}{rgb}{0.271305,0.019942,0.347269}%
\pgfsetfillcolor{currentfill}%
\pgfsetfillopacity{0.700000}%
\pgfsetlinewidth{0.000000pt}%
\definecolor{currentstroke}{rgb}{0.000000,0.000000,0.000000}%
\pgfsetstrokecolor{currentstroke}%
\pgfsetdash{}{0pt}%
\pgfpathmoveto{\pgfqpoint{3.240956in}{1.914826in}}%
\pgfpathlineto{\pgfqpoint{3.254573in}{1.910065in}}%
\pgfpathlineto{\pgfqpoint{3.268195in}{1.905393in}}%
\pgfpathlineto{\pgfqpoint{3.281821in}{1.900810in}}%
\pgfpathlineto{\pgfqpoint{3.295452in}{1.896315in}}%
\pgfpathlineto{\pgfqpoint{3.303706in}{1.904805in}}%
\pgfpathlineto{\pgfqpoint{3.311953in}{1.913322in}}%
\pgfpathlineto{\pgfqpoint{3.320193in}{1.921865in}}%
\pgfpathlineto{\pgfqpoint{3.328428in}{1.930433in}}%
\pgfpathlineto{\pgfqpoint{3.314811in}{1.934805in}}%
\pgfpathlineto{\pgfqpoint{3.301199in}{1.939266in}}%
\pgfpathlineto{\pgfqpoint{3.287592in}{1.943814in}}%
\pgfpathlineto{\pgfqpoint{3.273989in}{1.948452in}}%
\pgfpathlineto{\pgfqpoint{3.265741in}{1.939999in}}%
\pgfpathlineto{\pgfqpoint{3.257486in}{1.931576in}}%
\pgfpathlineto{\pgfqpoint{3.249224in}{1.923185in}}%
\pgfpathlineto{\pgfqpoint{3.240956in}{1.914826in}}%
\pgfpathclose%
\pgfusepath{fill}%
\end{pgfscope}%
\begin{pgfscope}%
\pgfpathrectangle{\pgfqpoint{1.150000in}{0.150000in}}{\pgfqpoint{5.700000in}{5.700000in}}%
\pgfusepath{clip}%
\pgfsetbuttcap%
\pgfsetroundjoin%
\definecolor{currentfill}{rgb}{0.278012,0.180367,0.486697}%
\pgfsetfillcolor{currentfill}%
\pgfsetfillopacity{0.700000}%
\pgfsetlinewidth{0.000000pt}%
\definecolor{currentstroke}{rgb}{0.000000,0.000000,0.000000}%
\pgfsetstrokecolor{currentstroke}%
\pgfsetdash{}{0pt}%
\pgfpathmoveto{\pgfqpoint{4.505898in}{2.210975in}}%
\pgfpathlineto{\pgfqpoint{4.519827in}{2.211411in}}%
\pgfpathlineto{\pgfqpoint{4.533765in}{2.211919in}}%
\pgfpathlineto{\pgfqpoint{4.547713in}{2.212500in}}%
\pgfpathlineto{\pgfqpoint{4.561670in}{2.213154in}}%
\pgfpathlineto{\pgfqpoint{4.569455in}{2.220811in}}%
\pgfpathlineto{\pgfqpoint{4.577234in}{2.228434in}}%
\pgfpathlineto{\pgfqpoint{4.585006in}{2.236025in}}%
\pgfpathlineto{\pgfqpoint{4.592773in}{2.243586in}}%
\pgfpathlineto{\pgfqpoint{4.578829in}{2.243058in}}%
\pgfpathlineto{\pgfqpoint{4.564894in}{2.242603in}}%
\pgfpathlineto{\pgfqpoint{4.550968in}{2.242219in}}%
\pgfpathlineto{\pgfqpoint{4.537052in}{2.241908in}}%
\pgfpathlineto{\pgfqpoint{4.529272in}{2.234214in}}%
\pgfpathlineto{\pgfqpoint{4.521487in}{2.226496in}}%
\pgfpathlineto{\pgfqpoint{4.513696in}{2.218750in}}%
\pgfpathlineto{\pgfqpoint{4.505898in}{2.210975in}}%
\pgfpathclose%
\pgfusepath{fill}%
\end{pgfscope}%
\begin{pgfscope}%
\pgfpathrectangle{\pgfqpoint{1.150000in}{0.150000in}}{\pgfqpoint{5.700000in}{5.700000in}}%
\pgfusepath{clip}%
\pgfsetbuttcap%
\pgfsetroundjoin%
\definecolor{currentfill}{rgb}{0.277941,0.056324,0.381191}%
\pgfsetfillcolor{currentfill}%
\pgfsetfillopacity{0.700000}%
\pgfsetlinewidth{0.000000pt}%
\definecolor{currentstroke}{rgb}{0.000000,0.000000,0.000000}%
\pgfsetstrokecolor{currentstroke}%
\pgfsetdash{}{0pt}%
\pgfpathmoveto{\pgfqpoint{3.699436in}{1.970140in}}%
\pgfpathlineto{\pgfqpoint{3.713134in}{1.967779in}}%
\pgfpathlineto{\pgfqpoint{3.726838in}{1.965498in}}%
\pgfpathlineto{\pgfqpoint{3.740549in}{1.963297in}}%
\pgfpathlineto{\pgfqpoint{3.754267in}{1.961177in}}%
\pgfpathlineto{\pgfqpoint{3.762350in}{1.970261in}}%
\pgfpathlineto{\pgfqpoint{3.770427in}{1.979328in}}%
\pgfpathlineto{\pgfqpoint{3.778499in}{1.988377in}}%
\pgfpathlineto{\pgfqpoint{3.786564in}{1.997407in}}%
\pgfpathlineto{\pgfqpoint{3.772858in}{1.999487in}}%
\pgfpathlineto{\pgfqpoint{3.759158in}{2.001646in}}%
\pgfpathlineto{\pgfqpoint{3.745465in}{2.003886in}}%
\pgfpathlineto{\pgfqpoint{3.731778in}{2.006207in}}%
\pgfpathlineto{\pgfqpoint{3.723702in}{1.997210in}}%
\pgfpathlineto{\pgfqpoint{3.715619in}{1.988199in}}%
\pgfpathlineto{\pgfqpoint{3.707531in}{1.979176in}}%
\pgfpathlineto{\pgfqpoint{3.699436in}{1.970140in}}%
\pgfpathclose%
\pgfusepath{fill}%
\end{pgfscope}%
\begin{pgfscope}%
\pgfpathrectangle{\pgfqpoint{1.150000in}{0.150000in}}{\pgfqpoint{5.700000in}{5.700000in}}%
\pgfusepath{clip}%
\pgfsetbuttcap%
\pgfsetroundjoin%
\definecolor{currentfill}{rgb}{0.218130,0.347432,0.550038}%
\pgfsetfillcolor{currentfill}%
\pgfsetfillopacity{0.700000}%
\pgfsetlinewidth{0.000000pt}%
\definecolor{currentstroke}{rgb}{0.000000,0.000000,0.000000}%
\pgfsetstrokecolor{currentstroke}%
\pgfsetdash{}{0pt}%
\pgfpathmoveto{\pgfqpoint{5.715828in}{2.588789in}}%
\pgfpathlineto{\pgfqpoint{5.730173in}{2.590508in}}%
\pgfpathlineto{\pgfqpoint{5.744530in}{2.592294in}}%
\pgfpathlineto{\pgfqpoint{5.758898in}{2.594147in}}%
\pgfpathlineto{\pgfqpoint{5.773278in}{2.596067in}}%
\pgfpathlineto{\pgfqpoint{5.780517in}{2.600846in}}%
\pgfpathlineto{\pgfqpoint{5.787753in}{2.605730in}}%
\pgfpathlineto{\pgfqpoint{5.794986in}{2.610726in}}%
\pgfpathlineto{\pgfqpoint{5.802217in}{2.615839in}}%
\pgfpathlineto{\pgfqpoint{5.787864in}{2.614296in}}%
\pgfpathlineto{\pgfqpoint{5.773523in}{2.612819in}}%
\pgfpathlineto{\pgfqpoint{5.759193in}{2.611409in}}%
\pgfpathlineto{\pgfqpoint{5.744874in}{2.610066in}}%
\pgfpathlineto{\pgfqpoint{5.737616in}{2.604570in}}%
\pgfpathlineto{\pgfqpoint{5.730356in}{2.599196in}}%
\pgfpathlineto{\pgfqpoint{5.723093in}{2.593938in}}%
\pgfpathlineto{\pgfqpoint{5.715828in}{2.588789in}}%
\pgfpathclose%
\pgfusepath{fill}%
\end{pgfscope}%
\begin{pgfscope}%
\pgfpathrectangle{\pgfqpoint{1.150000in}{0.150000in}}{\pgfqpoint{5.700000in}{5.700000in}}%
\pgfusepath{clip}%
\pgfsetbuttcap%
\pgfsetroundjoin%
\definecolor{currentfill}{rgb}{0.282656,0.100196,0.422160}%
\pgfsetfillcolor{currentfill}%
\pgfsetfillopacity{0.700000}%
\pgfsetlinewidth{0.000000pt}%
\definecolor{currentstroke}{rgb}{0.000000,0.000000,0.000000}%
\pgfsetstrokecolor{currentstroke}%
\pgfsetdash{}{0pt}%
\pgfpathmoveto{\pgfqpoint{4.015662in}{2.050439in}}%
\pgfpathlineto{\pgfqpoint{4.029441in}{2.049396in}}%
\pgfpathlineto{\pgfqpoint{4.043229in}{2.048431in}}%
\pgfpathlineto{\pgfqpoint{4.057024in}{2.047542in}}%
\pgfpathlineto{\pgfqpoint{4.070828in}{2.046729in}}%
\pgfpathlineto{\pgfqpoint{4.078800in}{2.055551in}}%
\pgfpathlineto{\pgfqpoint{4.086766in}{2.064337in}}%
\pgfpathlineto{\pgfqpoint{4.094727in}{2.073089in}}%
\pgfpathlineto{\pgfqpoint{4.102682in}{2.081807in}}%
\pgfpathlineto{\pgfqpoint{4.088889in}{2.082640in}}%
\pgfpathlineto{\pgfqpoint{4.075104in}{2.083550in}}%
\pgfpathlineto{\pgfqpoint{4.061327in}{2.084537in}}%
\pgfpathlineto{\pgfqpoint{4.047558in}{2.085600in}}%
\pgfpathlineto{\pgfqpoint{4.039593in}{2.076854in}}%
\pgfpathlineto{\pgfqpoint{4.031621in}{2.068078in}}%
\pgfpathlineto{\pgfqpoint{4.023644in}{2.059274in}}%
\pgfpathlineto{\pgfqpoint{4.015662in}{2.050439in}}%
\pgfpathclose%
\pgfusepath{fill}%
\end{pgfscope}%
\begin{pgfscope}%
\pgfpathrectangle{\pgfqpoint{1.150000in}{0.150000in}}{\pgfqpoint{5.700000in}{5.700000in}}%
\pgfusepath{clip}%
\pgfsetbuttcap%
\pgfsetroundjoin%
\definecolor{currentfill}{rgb}{0.262138,0.242286,0.520837}%
\pgfsetfillcolor{currentfill}%
\pgfsetfillopacity{0.700000}%
\pgfsetlinewidth{0.000000pt}%
\definecolor{currentstroke}{rgb}{0.000000,0.000000,0.000000}%
\pgfsetstrokecolor{currentstroke}%
\pgfsetdash{}{0pt}%
\pgfpathmoveto{\pgfqpoint{4.909261in}{2.343250in}}%
\pgfpathlineto{\pgfqpoint{4.923330in}{2.344488in}}%
\pgfpathlineto{\pgfqpoint{4.937410in}{2.345796in}}%
\pgfpathlineto{\pgfqpoint{4.951500in}{2.347175in}}%
\pgfpathlineto{\pgfqpoint{4.965601in}{2.348624in}}%
\pgfpathlineto{\pgfqpoint{4.973214in}{2.355071in}}%
\pgfpathlineto{\pgfqpoint{4.980822in}{2.361507in}}%
\pgfpathlineto{\pgfqpoint{4.988423in}{2.367936in}}%
\pgfpathlineto{\pgfqpoint{4.996018in}{2.374364in}}%
\pgfpathlineto{\pgfqpoint{4.981934in}{2.373124in}}%
\pgfpathlineto{\pgfqpoint{4.967860in}{2.371955in}}%
\pgfpathlineto{\pgfqpoint{4.953796in}{2.370855in}}%
\pgfpathlineto{\pgfqpoint{4.939743in}{2.369826in}}%
\pgfpathlineto{\pgfqpoint{4.932131in}{2.363182in}}%
\pgfpathlineto{\pgfqpoint{4.924514in}{2.356541in}}%
\pgfpathlineto{\pgfqpoint{4.916891in}{2.349898in}}%
\pgfpathlineto{\pgfqpoint{4.909261in}{2.343250in}}%
\pgfpathclose%
\pgfusepath{fill}%
\end{pgfscope}%
\begin{pgfscope}%
\pgfpathrectangle{\pgfqpoint{1.150000in}{0.150000in}}{\pgfqpoint{5.700000in}{5.700000in}}%
\pgfusepath{clip}%
\pgfsetbuttcap%
\pgfsetroundjoin%
\definecolor{currentfill}{rgb}{0.241237,0.296485,0.539709}%
\pgfsetfillcolor{currentfill}%
\pgfsetfillopacity{0.700000}%
\pgfsetlinewidth{0.000000pt}%
\definecolor{currentstroke}{rgb}{0.000000,0.000000,0.000000}%
\pgfsetstrokecolor{currentstroke}%
\pgfsetdash{}{0pt}%
\pgfpathmoveto{\pgfqpoint{5.312640in}{2.470080in}}%
\pgfpathlineto{\pgfqpoint{5.326851in}{2.471743in}}%
\pgfpathlineto{\pgfqpoint{5.341074in}{2.473474in}}%
\pgfpathlineto{\pgfqpoint{5.355307in}{2.475274in}}%
\pgfpathlineto{\pgfqpoint{5.369552in}{2.477142in}}%
\pgfpathlineto{\pgfqpoint{5.376978in}{2.482510in}}%
\pgfpathlineto{\pgfqpoint{5.384400in}{2.487914in}}%
\pgfpathlineto{\pgfqpoint{5.391816in}{2.493359in}}%
\pgfpathlineto{\pgfqpoint{5.399227in}{2.498851in}}%
\pgfpathlineto{\pgfqpoint{5.385004in}{2.497276in}}%
\pgfpathlineto{\pgfqpoint{5.370792in}{2.495770in}}%
\pgfpathlineto{\pgfqpoint{5.356590in}{2.494332in}}%
\pgfpathlineto{\pgfqpoint{5.342400in}{2.492962in}}%
\pgfpathlineto{\pgfqpoint{5.334967in}{2.487169in}}%
\pgfpathlineto{\pgfqpoint{5.327530in}{2.481429in}}%
\pgfpathlineto{\pgfqpoint{5.320088in}{2.475734in}}%
\pgfpathlineto{\pgfqpoint{5.312640in}{2.470080in}}%
\pgfpathclose%
\pgfusepath{fill}%
\end{pgfscope}%
\begin{pgfscope}%
\pgfpathrectangle{\pgfqpoint{1.150000in}{0.150000in}}{\pgfqpoint{5.700000in}{5.700000in}}%
\pgfusepath{clip}%
\pgfsetbuttcap%
\pgfsetroundjoin%
\definecolor{currentfill}{rgb}{0.280255,0.165693,0.476498}%
\pgfsetfillcolor{currentfill}%
\pgfsetfillopacity{0.700000}%
\pgfsetlinewidth{0.000000pt}%
\definecolor{currentstroke}{rgb}{0.000000,0.000000,0.000000}%
\pgfsetstrokecolor{currentstroke}%
\pgfsetdash{}{0pt}%
\pgfpathmoveto{\pgfqpoint{4.418975in}{2.178104in}}%
\pgfpathlineto{\pgfqpoint{4.432880in}{2.178352in}}%
\pgfpathlineto{\pgfqpoint{4.446793in}{2.178673in}}%
\pgfpathlineto{\pgfqpoint{4.460715in}{2.179068in}}%
\pgfpathlineto{\pgfqpoint{4.474647in}{2.179535in}}%
\pgfpathlineto{\pgfqpoint{4.482469in}{2.187451in}}%
\pgfpathlineto{\pgfqpoint{4.490285in}{2.195328in}}%
\pgfpathlineto{\pgfqpoint{4.498095in}{2.203169in}}%
\pgfpathlineto{\pgfqpoint{4.505898in}{2.210975in}}%
\pgfpathlineto{\pgfqpoint{4.491978in}{2.210612in}}%
\pgfpathlineto{\pgfqpoint{4.478068in}{2.210322in}}%
\pgfpathlineto{\pgfqpoint{4.464167in}{2.210105in}}%
\pgfpathlineto{\pgfqpoint{4.450274in}{2.209962in}}%
\pgfpathlineto{\pgfqpoint{4.442459in}{2.202043in}}%
\pgfpathlineto{\pgfqpoint{4.434637in}{2.194095in}}%
\pgfpathlineto{\pgfqpoint{4.426809in}{2.186116in}}%
\pgfpathlineto{\pgfqpoint{4.418975in}{2.178104in}}%
\pgfpathclose%
\pgfusepath{fill}%
\end{pgfscope}%
\begin{pgfscope}%
\pgfpathrectangle{\pgfqpoint{1.150000in}{0.150000in}}{\pgfqpoint{5.700000in}{5.700000in}}%
\pgfusepath{clip}%
\pgfsetbuttcap%
\pgfsetroundjoin%
\definecolor{currentfill}{rgb}{0.280267,0.073417,0.397163}%
\pgfsetfillcolor{currentfill}%
\pgfsetfillopacity{0.700000}%
\pgfsetlinewidth{0.000000pt}%
\definecolor{currentstroke}{rgb}{0.000000,0.000000,0.000000}%
\pgfsetstrokecolor{currentstroke}%
\pgfsetdash{}{0pt}%
\pgfpathmoveto{\pgfqpoint{2.671385in}{2.011874in}}%
\pgfpathlineto{\pgfqpoint{2.684990in}{2.003283in}}%
\pgfpathlineto{\pgfqpoint{2.698596in}{1.994800in}}%
\pgfpathlineto{\pgfqpoint{2.712203in}{1.986424in}}%
\pgfpathlineto{\pgfqpoint{2.725812in}{1.978154in}}%
\pgfpathlineto{\pgfqpoint{2.734338in}{1.984213in}}%
\pgfpathlineto{\pgfqpoint{2.742854in}{1.990377in}}%
\pgfpathlineto{\pgfqpoint{2.751361in}{1.996643in}}%
\pgfpathlineto{\pgfqpoint{2.759858in}{2.003009in}}%
\pgfpathlineto{\pgfqpoint{2.746272in}{2.011072in}}%
\pgfpathlineto{\pgfqpoint{2.732687in}{2.019241in}}%
\pgfpathlineto{\pgfqpoint{2.719103in}{2.027517in}}%
\pgfpathlineto{\pgfqpoint{2.705520in}{2.035900in}}%
\pgfpathlineto{\pgfqpoint{2.697001in}{2.029734in}}%
\pgfpathlineto{\pgfqpoint{2.688472in}{2.023672in}}%
\pgfpathlineto{\pgfqpoint{2.679934in}{2.017718in}}%
\pgfpathlineto{\pgfqpoint{2.671385in}{2.011874in}}%
\pgfpathclose%
\pgfusepath{fill}%
\end{pgfscope}%
\begin{pgfscope}%
\pgfpathrectangle{\pgfqpoint{1.150000in}{0.150000in}}{\pgfqpoint{5.700000in}{5.700000in}}%
\pgfusepath{clip}%
\pgfsetbuttcap%
\pgfsetroundjoin%
\definecolor{currentfill}{rgb}{0.272594,0.025563,0.353093}%
\pgfsetfillcolor{currentfill}%
\pgfsetfillopacity{0.700000}%
\pgfsetlinewidth{0.000000pt}%
\definecolor{currentstroke}{rgb}{0.000000,0.000000,0.000000}%
\pgfsetstrokecolor{currentstroke}%
\pgfsetdash{}{0pt}%
\pgfpathmoveto{\pgfqpoint{3.382942in}{1.913818in}}%
\pgfpathlineto{\pgfqpoint{3.396583in}{1.909880in}}%
\pgfpathlineto{\pgfqpoint{3.410229in}{1.906028in}}%
\pgfpathlineto{\pgfqpoint{3.423881in}{1.902262in}}%
\pgfpathlineto{\pgfqpoint{3.437538in}{1.898580in}}%
\pgfpathlineto{\pgfqpoint{3.445739in}{1.907392in}}%
\pgfpathlineto{\pgfqpoint{3.453933in}{1.916215in}}%
\pgfpathlineto{\pgfqpoint{3.462122in}{1.925049in}}%
\pgfpathlineto{\pgfqpoint{3.470304in}{1.933892in}}%
\pgfpathlineto{\pgfqpoint{3.456661in}{1.937471in}}%
\pgfpathlineto{\pgfqpoint{3.443022in}{1.941135in}}%
\pgfpathlineto{\pgfqpoint{3.429389in}{1.944885in}}%
\pgfpathlineto{\pgfqpoint{3.415761in}{1.948720in}}%
\pgfpathlineto{\pgfqpoint{3.407566in}{1.939971in}}%
\pgfpathlineto{\pgfqpoint{3.399364in}{1.931238in}}%
\pgfpathlineto{\pgfqpoint{3.391156in}{1.922519in}}%
\pgfpathlineto{\pgfqpoint{3.382942in}{1.913818in}}%
\pgfpathclose%
\pgfusepath{fill}%
\end{pgfscope}%
\begin{pgfscope}%
\pgfpathrectangle{\pgfqpoint{1.150000in}{0.150000in}}{\pgfqpoint{5.700000in}{5.700000in}}%
\pgfusepath{clip}%
\pgfsetbuttcap%
\pgfsetroundjoin%
\definecolor{currentfill}{rgb}{0.276022,0.044167,0.370164}%
\pgfsetfillcolor{currentfill}%
\pgfsetfillopacity{0.700000}%
\pgfsetlinewidth{0.000000pt}%
\definecolor{currentstroke}{rgb}{0.000000,0.000000,0.000000}%
\pgfsetstrokecolor{currentstroke}%
\pgfsetdash{}{0pt}%
\pgfpathmoveto{\pgfqpoint{3.612230in}{1.944378in}}%
\pgfpathlineto{\pgfqpoint{3.625914in}{1.941630in}}%
\pgfpathlineto{\pgfqpoint{3.639604in}{1.938964in}}%
\pgfpathlineto{\pgfqpoint{3.653300in}{1.936379in}}%
\pgfpathlineto{\pgfqpoint{3.667003in}{1.933876in}}%
\pgfpathlineto{\pgfqpoint{3.675120in}{1.942959in}}%
\pgfpathlineto{\pgfqpoint{3.683231in}{1.952031in}}%
\pgfpathlineto{\pgfqpoint{3.691337in}{1.961092in}}%
\pgfpathlineto{\pgfqpoint{3.699436in}{1.970140in}}%
\pgfpathlineto{\pgfqpoint{3.685745in}{1.972582in}}%
\pgfpathlineto{\pgfqpoint{3.672060in}{1.975105in}}%
\pgfpathlineto{\pgfqpoint{3.658382in}{1.977710in}}%
\pgfpathlineto{\pgfqpoint{3.644710in}{1.980396in}}%
\pgfpathlineto{\pgfqpoint{3.636599in}{1.971402in}}%
\pgfpathlineto{\pgfqpoint{3.628482in}{1.962400in}}%
\pgfpathlineto{\pgfqpoint{3.620359in}{1.953392in}}%
\pgfpathlineto{\pgfqpoint{3.612230in}{1.944378in}}%
\pgfpathclose%
\pgfusepath{fill}%
\end{pgfscope}%
\begin{pgfscope}%
\pgfpathrectangle{\pgfqpoint{1.150000in}{0.150000in}}{\pgfqpoint{5.700000in}{5.700000in}}%
\pgfusepath{clip}%
\pgfsetbuttcap%
\pgfsetroundjoin%
\definecolor{currentfill}{rgb}{0.283229,0.120777,0.440584}%
\pgfsetfillcolor{currentfill}%
\pgfsetfillopacity{0.700000}%
\pgfsetlinewidth{0.000000pt}%
\definecolor{currentstroke}{rgb}{0.000000,0.000000,0.000000}%
\pgfsetstrokecolor{currentstroke}%
\pgfsetdash{}{0pt}%
\pgfpathmoveto{\pgfqpoint{2.473551in}{2.104221in}}%
\pgfpathlineto{\pgfqpoint{2.487180in}{2.094042in}}%
\pgfpathlineto{\pgfqpoint{2.500808in}{2.083980in}}%
\pgfpathlineto{\pgfqpoint{2.514436in}{2.074036in}}%
\pgfpathlineto{\pgfqpoint{2.528064in}{2.064208in}}%
\pgfpathlineto{\pgfqpoint{2.536705in}{2.069104in}}%
\pgfpathlineto{\pgfqpoint{2.545336in}{2.074134in}}%
\pgfpathlineto{\pgfqpoint{2.553955in}{2.079294in}}%
\pgfpathlineto{\pgfqpoint{2.562562in}{2.084582in}}%
\pgfpathlineto{\pgfqpoint{2.548960in}{2.094180in}}%
\pgfpathlineto{\pgfqpoint{2.535358in}{2.103895in}}%
\pgfpathlineto{\pgfqpoint{2.521756in}{2.113726in}}%
\pgfpathlineto{\pgfqpoint{2.508153in}{2.123676in}}%
\pgfpathlineto{\pgfqpoint{2.499520in}{2.118610in}}%
\pgfpathlineto{\pgfqpoint{2.490875in}{2.113677in}}%
\pgfpathlineto{\pgfqpoint{2.482219in}{2.108880in}}%
\pgfpathlineto{\pgfqpoint{2.473551in}{2.104221in}}%
\pgfpathclose%
\pgfusepath{fill}%
\end{pgfscope}%
\begin{pgfscope}%
\pgfpathrectangle{\pgfqpoint{1.150000in}{0.150000in}}{\pgfqpoint{5.700000in}{5.700000in}}%
\pgfusepath{clip}%
\pgfsetbuttcap%
\pgfsetroundjoin%
\definecolor{currentfill}{rgb}{0.269308,0.218818,0.509577}%
\pgfsetfillcolor{currentfill}%
\pgfsetfillopacity{0.700000}%
\pgfsetlinewidth{0.000000pt}%
\definecolor{currentstroke}{rgb}{0.000000,0.000000,0.000000}%
\pgfsetstrokecolor{currentstroke}%
\pgfsetdash{}{0pt}%
\pgfpathmoveto{\pgfqpoint{2.165249in}{2.322113in}}%
\pgfpathlineto{\pgfqpoint{2.178950in}{2.309093in}}%
\pgfpathlineto{\pgfqpoint{2.192647in}{2.296213in}}%
\pgfpathlineto{\pgfqpoint{2.206341in}{2.283473in}}%
\pgfpathlineto{\pgfqpoint{2.220032in}{2.270870in}}%
\pgfpathlineto{\pgfqpoint{2.228870in}{2.273891in}}%
\pgfpathlineto{\pgfqpoint{2.237693in}{2.277087in}}%
\pgfpathlineto{\pgfqpoint{2.246502in}{2.280454in}}%
\pgfpathlineto{\pgfqpoint{2.255298in}{2.283988in}}%
\pgfpathlineto{\pgfqpoint{2.241638in}{2.296334in}}%
\pgfpathlineto{\pgfqpoint{2.227976in}{2.308819in}}%
\pgfpathlineto{\pgfqpoint{2.214310in}{2.321442in}}%
\pgfpathlineto{\pgfqpoint{2.200641in}{2.334206in}}%
\pgfpathlineto{\pgfqpoint{2.191815in}{2.330920in}}%
\pgfpathlineto{\pgfqpoint{2.182974in}{2.327806in}}%
\pgfpathlineto{\pgfqpoint{2.174119in}{2.324869in}}%
\pgfpathlineto{\pgfqpoint{2.165249in}{2.322113in}}%
\pgfpathclose%
\pgfusepath{fill}%
\end{pgfscope}%
\begin{pgfscope}%
\pgfpathrectangle{\pgfqpoint{1.150000in}{0.150000in}}{\pgfqpoint{5.700000in}{5.700000in}}%
\pgfusepath{clip}%
\pgfsetbuttcap%
\pgfsetroundjoin%
\definecolor{currentfill}{rgb}{0.281924,0.089666,0.412415}%
\pgfsetfillcolor{currentfill}%
\pgfsetfillopacity{0.700000}%
\pgfsetlinewidth{0.000000pt}%
\definecolor{currentstroke}{rgb}{0.000000,0.000000,0.000000}%
\pgfsetstrokecolor{currentstroke}%
\pgfsetdash{}{0pt}%
\pgfpathmoveto{\pgfqpoint{3.928589in}{2.019721in}}%
\pgfpathlineto{\pgfqpoint{3.942349in}{2.018370in}}%
\pgfpathlineto{\pgfqpoint{3.956117in}{2.017097in}}%
\pgfpathlineto{\pgfqpoint{3.969892in}{2.015901in}}%
\pgfpathlineto{\pgfqpoint{3.983674in}{2.014782in}}%
\pgfpathlineto{\pgfqpoint{3.991680in}{2.023745in}}%
\pgfpathlineto{\pgfqpoint{3.999679in}{2.032675in}}%
\pgfpathlineto{\pgfqpoint{4.007673in}{2.041572in}}%
\pgfpathlineto{\pgfqpoint{4.015662in}{2.050439in}}%
\pgfpathlineto{\pgfqpoint{4.001890in}{2.051558in}}%
\pgfpathlineto{\pgfqpoint{3.988125in}{2.052754in}}%
\pgfpathlineto{\pgfqpoint{3.974369in}{2.054028in}}%
\pgfpathlineto{\pgfqpoint{3.960619in}{2.055379in}}%
\pgfpathlineto{\pgfqpoint{3.952620in}{2.046505in}}%
\pgfpathlineto{\pgfqpoint{3.944616in}{2.037604in}}%
\pgfpathlineto{\pgfqpoint{3.936605in}{2.028676in}}%
\pgfpathlineto{\pgfqpoint{3.928589in}{2.019721in}}%
\pgfpathclose%
\pgfusepath{fill}%
\end{pgfscope}%
\begin{pgfscope}%
\pgfpathrectangle{\pgfqpoint{1.150000in}{0.150000in}}{\pgfqpoint{5.700000in}{5.700000in}}%
\pgfusepath{clip}%
\pgfsetbuttcap%
\pgfsetroundjoin%
\definecolor{currentfill}{rgb}{0.266580,0.228262,0.514349}%
\pgfsetfillcolor{currentfill}%
\pgfsetfillopacity{0.700000}%
\pgfsetlinewidth{0.000000pt}%
\definecolor{currentstroke}{rgb}{0.000000,0.000000,0.000000}%
\pgfsetstrokecolor{currentstroke}%
\pgfsetdash{}{0pt}%
\pgfpathmoveto{\pgfqpoint{4.822444in}{2.311531in}}%
\pgfpathlineto{\pgfqpoint{4.836488in}{2.312675in}}%
\pgfpathlineto{\pgfqpoint{4.850542in}{2.313890in}}%
\pgfpathlineto{\pgfqpoint{4.864606in}{2.315176in}}%
\pgfpathlineto{\pgfqpoint{4.878681in}{2.316532in}}%
\pgfpathlineto{\pgfqpoint{4.886335in}{2.323238in}}%
\pgfpathlineto{\pgfqpoint{4.893983in}{2.329923in}}%
\pgfpathlineto{\pgfqpoint{4.901625in}{2.336593in}}%
\pgfpathlineto{\pgfqpoint{4.909261in}{2.343250in}}%
\pgfpathlineto{\pgfqpoint{4.895202in}{2.342082in}}%
\pgfpathlineto{\pgfqpoint{4.881153in}{2.340985in}}%
\pgfpathlineto{\pgfqpoint{4.867114in}{2.339958in}}%
\pgfpathlineto{\pgfqpoint{4.853085in}{2.339002in}}%
\pgfpathlineto{\pgfqpoint{4.845434in}{2.332149in}}%
\pgfpathlineto{\pgfqpoint{4.837777in}{2.325288in}}%
\pgfpathlineto{\pgfqpoint{4.830114in}{2.318417in}}%
\pgfpathlineto{\pgfqpoint{4.822444in}{2.311531in}}%
\pgfpathclose%
\pgfusepath{fill}%
\end{pgfscope}%
\begin{pgfscope}%
\pgfpathrectangle{\pgfqpoint{1.150000in}{0.150000in}}{\pgfqpoint{5.700000in}{5.700000in}}%
\pgfusepath{clip}%
\pgfsetbuttcap%
\pgfsetroundjoin%
\definecolor{currentfill}{rgb}{0.221989,0.339161,0.548752}%
\pgfsetfillcolor{currentfill}%
\pgfsetfillopacity{0.700000}%
\pgfsetlinewidth{0.000000pt}%
\definecolor{currentstroke}{rgb}{0.000000,0.000000,0.000000}%
\pgfsetstrokecolor{currentstroke}%
\pgfsetdash{}{0pt}%
\pgfpathmoveto{\pgfqpoint{5.629370in}{2.561536in}}%
\pgfpathlineto{\pgfqpoint{5.643694in}{2.563342in}}%
\pgfpathlineto{\pgfqpoint{5.658030in}{2.565216in}}%
\pgfpathlineto{\pgfqpoint{5.672377in}{2.567157in}}%
\pgfpathlineto{\pgfqpoint{5.686737in}{2.569165in}}%
\pgfpathlineto{\pgfqpoint{5.694015in}{2.573939in}}%
\pgfpathlineto{\pgfqpoint{5.701289in}{2.578796in}}%
\pgfpathlineto{\pgfqpoint{5.708560in}{2.583744in}}%
\pgfpathlineto{\pgfqpoint{5.715828in}{2.588789in}}%
\pgfpathlineto{\pgfqpoint{5.701495in}{2.587138in}}%
\pgfpathlineto{\pgfqpoint{5.687173in}{2.585553in}}%
\pgfpathlineto{\pgfqpoint{5.672862in}{2.584035in}}%
\pgfpathlineto{\pgfqpoint{5.658563in}{2.582584in}}%
\pgfpathlineto{\pgfqpoint{5.651270in}{2.577176in}}%
\pgfpathlineto{\pgfqpoint{5.643973in}{2.571869in}}%
\pgfpathlineto{\pgfqpoint{5.636673in}{2.566658in}}%
\pgfpathlineto{\pgfqpoint{5.629370in}{2.561536in}}%
\pgfpathclose%
\pgfusepath{fill}%
\end{pgfscope}%
\begin{pgfscope}%
\pgfpathrectangle{\pgfqpoint{1.150000in}{0.150000in}}{\pgfqpoint{5.700000in}{5.700000in}}%
\pgfusepath{clip}%
\pgfsetbuttcap%
\pgfsetroundjoin%
\definecolor{currentfill}{rgb}{0.281887,0.150881,0.465405}%
\pgfsetfillcolor{currentfill}%
\pgfsetfillopacity{0.700000}%
\pgfsetlinewidth{0.000000pt}%
\definecolor{currentstroke}{rgb}{0.000000,0.000000,0.000000}%
\pgfsetstrokecolor{currentstroke}%
\pgfsetdash{}{0pt}%
\pgfpathmoveto{\pgfqpoint{4.332006in}{2.145092in}}%
\pgfpathlineto{\pgfqpoint{4.345886in}{2.145129in}}%
\pgfpathlineto{\pgfqpoint{4.359775in}{2.145240in}}%
\pgfpathlineto{\pgfqpoint{4.373673in}{2.145425in}}%
\pgfpathlineto{\pgfqpoint{4.387580in}{2.145683in}}%
\pgfpathlineto{\pgfqpoint{4.395438in}{2.153848in}}%
\pgfpathlineto{\pgfqpoint{4.403290in}{2.161972in}}%
\pgfpathlineto{\pgfqpoint{4.411136in}{2.170057in}}%
\pgfpathlineto{\pgfqpoint{4.418975in}{2.178104in}}%
\pgfpathlineto{\pgfqpoint{4.405080in}{2.177930in}}%
\pgfpathlineto{\pgfqpoint{4.391194in}{2.177829in}}%
\pgfpathlineto{\pgfqpoint{4.377316in}{2.177801in}}%
\pgfpathlineto{\pgfqpoint{4.363448in}{2.177847in}}%
\pgfpathlineto{\pgfqpoint{4.355596in}{2.169709in}}%
\pgfpathlineto{\pgfqpoint{4.347739in}{2.161538in}}%
\pgfpathlineto{\pgfqpoint{4.339876in}{2.153333in}}%
\pgfpathlineto{\pgfqpoint{4.332006in}{2.145092in}}%
\pgfpathclose%
\pgfusepath{fill}%
\end{pgfscope}%
\begin{pgfscope}%
\pgfpathrectangle{\pgfqpoint{1.150000in}{0.150000in}}{\pgfqpoint{5.700000in}{5.700000in}}%
\pgfusepath{clip}%
\pgfsetbuttcap%
\pgfsetroundjoin%
\definecolor{currentfill}{rgb}{0.272594,0.025563,0.353093}%
\pgfsetfillcolor{currentfill}%
\pgfsetfillopacity{0.700000}%
\pgfsetlinewidth{0.000000pt}%
\definecolor{currentstroke}{rgb}{0.000000,0.000000,0.000000}%
\pgfsetstrokecolor{currentstroke}%
\pgfsetdash{}{0pt}%
\pgfpathmoveto{\pgfqpoint{3.011076in}{1.917277in}}%
\pgfpathlineto{\pgfqpoint{3.024681in}{1.911104in}}%
\pgfpathlineto{\pgfqpoint{3.038289in}{1.905026in}}%
\pgfpathlineto{\pgfqpoint{3.051900in}{1.899043in}}%
\pgfpathlineto{\pgfqpoint{3.065514in}{1.893153in}}%
\pgfpathlineto{\pgfqpoint{3.073872in}{1.900839in}}%
\pgfpathlineto{\pgfqpoint{3.082222in}{1.908585in}}%
\pgfpathlineto{\pgfqpoint{3.090565in}{1.916387in}}%
\pgfpathlineto{\pgfqpoint{3.098900in}{1.924243in}}%
\pgfpathlineto{\pgfqpoint{3.085302in}{1.929968in}}%
\pgfpathlineto{\pgfqpoint{3.071708in}{1.935788in}}%
\pgfpathlineto{\pgfqpoint{3.058118in}{1.941702in}}%
\pgfpathlineto{\pgfqpoint{3.044530in}{1.947710in}}%
\pgfpathlineto{\pgfqpoint{3.036178in}{1.940011in}}%
\pgfpathlineto{\pgfqpoint{3.027819in}{1.932371in}}%
\pgfpathlineto{\pgfqpoint{3.019451in}{1.924792in}}%
\pgfpathlineto{\pgfqpoint{3.011076in}{1.917277in}}%
\pgfpathclose%
\pgfusepath{fill}%
\end{pgfscope}%
\begin{pgfscope}%
\pgfpathrectangle{\pgfqpoint{1.150000in}{0.150000in}}{\pgfqpoint{5.700000in}{5.700000in}}%
\pgfusepath{clip}%
\pgfsetbuttcap%
\pgfsetroundjoin%
\definecolor{currentfill}{rgb}{0.244972,0.287675,0.537260}%
\pgfsetfillcolor{currentfill}%
\pgfsetfillopacity{0.700000}%
\pgfsetlinewidth{0.000000pt}%
\definecolor{currentstroke}{rgb}{0.000000,0.000000,0.000000}%
\pgfsetstrokecolor{currentstroke}%
\pgfsetdash{}{0pt}%
\pgfpathmoveto{\pgfqpoint{5.225982in}{2.440722in}}%
\pgfpathlineto{\pgfqpoint{5.240169in}{2.442382in}}%
\pgfpathlineto{\pgfqpoint{5.254367in}{2.444111in}}%
\pgfpathlineto{\pgfqpoint{5.268576in}{2.445909in}}%
\pgfpathlineto{\pgfqpoint{5.282797in}{2.447776in}}%
\pgfpathlineto{\pgfqpoint{5.290266in}{2.453316in}}%
\pgfpathlineto{\pgfqpoint{5.297729in}{2.458876in}}%
\pgfpathlineto{\pgfqpoint{5.305188in}{2.464463in}}%
\pgfpathlineto{\pgfqpoint{5.312640in}{2.470080in}}%
\pgfpathlineto{\pgfqpoint{5.298440in}{2.468486in}}%
\pgfpathlineto{\pgfqpoint{5.284251in}{2.466961in}}%
\pgfpathlineto{\pgfqpoint{5.270073in}{2.465504in}}%
\pgfpathlineto{\pgfqpoint{5.255905in}{2.464115in}}%
\pgfpathlineto{\pgfqpoint{5.248433in}{2.458218in}}%
\pgfpathlineto{\pgfqpoint{5.240955in}{2.452357in}}%
\pgfpathlineto{\pgfqpoint{5.233471in}{2.446526in}}%
\pgfpathlineto{\pgfqpoint{5.225982in}{2.440722in}}%
\pgfpathclose%
\pgfusepath{fill}%
\end{pgfscope}%
\begin{pgfscope}%
\pgfpathrectangle{\pgfqpoint{1.150000in}{0.150000in}}{\pgfqpoint{5.700000in}{5.700000in}}%
\pgfusepath{clip}%
\pgfsetbuttcap%
\pgfsetroundjoin%
\definecolor{currentfill}{rgb}{0.274952,0.037752,0.364543}%
\pgfsetfillcolor{currentfill}%
\pgfsetfillopacity{0.700000}%
\pgfsetlinewidth{0.000000pt}%
\definecolor{currentstroke}{rgb}{0.000000,0.000000,0.000000}%
\pgfsetstrokecolor{currentstroke}%
\pgfsetdash{}{0pt}%
\pgfpathmoveto{\pgfqpoint{2.868606in}{1.942249in}}%
\pgfpathlineto{\pgfqpoint{2.882209in}{1.935112in}}%
\pgfpathlineto{\pgfqpoint{2.895813in}{1.928074in}}%
\pgfpathlineto{\pgfqpoint{2.909421in}{1.921135in}}%
\pgfpathlineto{\pgfqpoint{2.923030in}{1.914294in}}%
\pgfpathlineto{\pgfqpoint{2.931457in}{1.921337in}}%
\pgfpathlineto{\pgfqpoint{2.939875in}{1.928459in}}%
\pgfpathlineto{\pgfqpoint{2.948285in}{1.935657in}}%
\pgfpathlineto{\pgfqpoint{2.956686in}{1.942929in}}%
\pgfpathlineto{\pgfqpoint{2.943096in}{1.949584in}}%
\pgfpathlineto{\pgfqpoint{2.929508in}{1.956338in}}%
\pgfpathlineto{\pgfqpoint{2.915923in}{1.963190in}}%
\pgfpathlineto{\pgfqpoint{2.902340in}{1.970142in}}%
\pgfpathlineto{\pgfqpoint{2.893920in}{1.963048in}}%
\pgfpathlineto{\pgfqpoint{2.885491in}{1.956033in}}%
\pgfpathlineto{\pgfqpoint{2.877053in}{1.949099in}}%
\pgfpathlineto{\pgfqpoint{2.868606in}{1.942249in}}%
\pgfpathclose%
\pgfusepath{fill}%
\end{pgfscope}%
\begin{pgfscope}%
\pgfpathrectangle{\pgfqpoint{1.150000in}{0.150000in}}{\pgfqpoint{5.700000in}{5.700000in}}%
\pgfusepath{clip}%
\pgfsetbuttcap%
\pgfsetroundjoin%
\definecolor{currentfill}{rgb}{0.271305,0.019942,0.347269}%
\pgfsetfillcolor{currentfill}%
\pgfsetfillopacity{0.700000}%
\pgfsetlinewidth{0.000000pt}%
\definecolor{currentstroke}{rgb}{0.000000,0.000000,0.000000}%
\pgfsetstrokecolor{currentstroke}%
\pgfsetdash{}{0pt}%
\pgfpathmoveto{\pgfqpoint{3.153326in}{1.902267in}}%
\pgfpathlineto{\pgfqpoint{3.166942in}{1.897002in}}%
\pgfpathlineto{\pgfqpoint{3.180562in}{1.891829in}}%
\pgfpathlineto{\pgfqpoint{3.194186in}{1.886745in}}%
\pgfpathlineto{\pgfqpoint{3.207815in}{1.881752in}}%
\pgfpathlineto{\pgfqpoint{3.216110in}{1.889962in}}%
\pgfpathlineto{\pgfqpoint{3.224399in}{1.898213in}}%
\pgfpathlineto{\pgfqpoint{3.232681in}{1.906501in}}%
\pgfpathlineto{\pgfqpoint{3.240956in}{1.914826in}}%
\pgfpathlineto{\pgfqpoint{3.227343in}{1.919676in}}%
\pgfpathlineto{\pgfqpoint{3.213734in}{1.924616in}}%
\pgfpathlineto{\pgfqpoint{3.200130in}{1.929646in}}%
\pgfpathlineto{\pgfqpoint{3.186530in}{1.934767in}}%
\pgfpathlineto{\pgfqpoint{3.178239in}{1.926578in}}%
\pgfpathlineto{\pgfqpoint{3.169942in}{1.918431in}}%
\pgfpathlineto{\pgfqpoint{3.161638in}{1.910326in}}%
\pgfpathlineto{\pgfqpoint{3.153326in}{1.902267in}}%
\pgfpathclose%
\pgfusepath{fill}%
\end{pgfscope}%
\begin{pgfscope}%
\pgfpathrectangle{\pgfqpoint{1.150000in}{0.150000in}}{\pgfqpoint{5.700000in}{5.700000in}}%
\pgfusepath{clip}%
\pgfsetbuttcap%
\pgfsetroundjoin%
\definecolor{currentfill}{rgb}{0.274128,0.199721,0.498911}%
\pgfsetfillcolor{currentfill}%
\pgfsetfillopacity{0.700000}%
\pgfsetlinewidth{0.000000pt}%
\definecolor{currentstroke}{rgb}{0.000000,0.000000,0.000000}%
\pgfsetstrokecolor{currentstroke}%
\pgfsetdash{}{0pt}%
\pgfpathmoveto{\pgfqpoint{2.220032in}{2.270870in}}%
\pgfpathlineto{\pgfqpoint{2.233720in}{2.258405in}}%
\pgfpathlineto{\pgfqpoint{2.247404in}{2.246075in}}%
\pgfpathlineto{\pgfqpoint{2.261086in}{2.233880in}}%
\pgfpathlineto{\pgfqpoint{2.274765in}{2.221817in}}%
\pgfpathlineto{\pgfqpoint{2.283571in}{2.225101in}}%
\pgfpathlineto{\pgfqpoint{2.292364in}{2.228555in}}%
\pgfpathlineto{\pgfqpoint{2.301143in}{2.232174in}}%
\pgfpathlineto{\pgfqpoint{2.309908in}{2.235955in}}%
\pgfpathlineto{\pgfqpoint{2.296259in}{2.247763in}}%
\pgfpathlineto{\pgfqpoint{2.282608in}{2.259703in}}%
\pgfpathlineto{\pgfqpoint{2.268954in}{2.271778in}}%
\pgfpathlineto{\pgfqpoint{2.255298in}{2.283988in}}%
\pgfpathlineto{\pgfqpoint{2.246502in}{2.280454in}}%
\pgfpathlineto{\pgfqpoint{2.237693in}{2.277087in}}%
\pgfpathlineto{\pgfqpoint{2.228870in}{2.273891in}}%
\pgfpathlineto{\pgfqpoint{2.220032in}{2.270870in}}%
\pgfpathclose%
\pgfusepath{fill}%
\end{pgfscope}%
\begin{pgfscope}%
\pgfpathrectangle{\pgfqpoint{1.150000in}{0.150000in}}{\pgfqpoint{5.700000in}{5.700000in}}%
\pgfusepath{clip}%
\pgfsetbuttcap%
\pgfsetroundjoin%
\definecolor{currentfill}{rgb}{0.269308,0.218818,0.509577}%
\pgfsetfillcolor{currentfill}%
\pgfsetfillopacity{0.700000}%
\pgfsetlinewidth{0.000000pt}%
\definecolor{currentstroke}{rgb}{0.000000,0.000000,0.000000}%
\pgfsetstrokecolor{currentstroke}%
\pgfsetdash{}{0pt}%
\pgfpathmoveto{\pgfqpoint{4.735571in}{2.279237in}}%
\pgfpathlineto{\pgfqpoint{4.749590in}{2.280264in}}%
\pgfpathlineto{\pgfqpoint{4.763618in}{2.281363in}}%
\pgfpathlineto{\pgfqpoint{4.777656in}{2.282533in}}%
\pgfpathlineto{\pgfqpoint{4.791704in}{2.283775in}}%
\pgfpathlineto{\pgfqpoint{4.799399in}{2.290752in}}%
\pgfpathlineto{\pgfqpoint{4.807087in}{2.297702in}}%
\pgfpathlineto{\pgfqpoint{4.814769in}{2.304627in}}%
\pgfpathlineto{\pgfqpoint{4.822444in}{2.311531in}}%
\pgfpathlineto{\pgfqpoint{4.808411in}{2.310457in}}%
\pgfpathlineto{\pgfqpoint{4.794387in}{2.309455in}}%
\pgfpathlineto{\pgfqpoint{4.780373in}{2.308523in}}%
\pgfpathlineto{\pgfqpoint{4.766369in}{2.307663in}}%
\pgfpathlineto{\pgfqpoint{4.758679in}{2.300584in}}%
\pgfpathlineto{\pgfqpoint{4.750983in}{2.293489in}}%
\pgfpathlineto{\pgfqpoint{4.743280in}{2.286374in}}%
\pgfpathlineto{\pgfqpoint{4.735571in}{2.279237in}}%
\pgfpathclose%
\pgfusepath{fill}%
\end{pgfscope}%
\begin{pgfscope}%
\pgfpathrectangle{\pgfqpoint{1.150000in}{0.150000in}}{\pgfqpoint{5.700000in}{5.700000in}}%
\pgfusepath{clip}%
\pgfsetbuttcap%
\pgfsetroundjoin%
\definecolor{currentfill}{rgb}{0.280267,0.073417,0.397163}%
\pgfsetfillcolor{currentfill}%
\pgfsetfillopacity{0.700000}%
\pgfsetlinewidth{0.000000pt}%
\definecolor{currentstroke}{rgb}{0.000000,0.000000,0.000000}%
\pgfsetstrokecolor{currentstroke}%
\pgfsetdash{}{0pt}%
\pgfpathmoveto{\pgfqpoint{3.841460in}{1.989883in}}%
\pgfpathlineto{\pgfqpoint{3.855201in}{1.988199in}}%
\pgfpathlineto{\pgfqpoint{3.868950in}{1.986594in}}%
\pgfpathlineto{\pgfqpoint{3.882706in}{1.985068in}}%
\pgfpathlineto{\pgfqpoint{3.896469in}{1.983619in}}%
\pgfpathlineto{\pgfqpoint{3.904507in}{1.992687in}}%
\pgfpathlineto{\pgfqpoint{3.912540in}{2.001726in}}%
\pgfpathlineto{\pgfqpoint{3.920568in}{2.010738in}}%
\pgfpathlineto{\pgfqpoint{3.928589in}{2.019721in}}%
\pgfpathlineto{\pgfqpoint{3.914837in}{2.021149in}}%
\pgfpathlineto{\pgfqpoint{3.901092in}{2.022656in}}%
\pgfpathlineto{\pgfqpoint{3.887354in}{2.024241in}}%
\pgfpathlineto{\pgfqpoint{3.873623in}{2.025904in}}%
\pgfpathlineto{\pgfqpoint{3.865591in}{2.016934in}}%
\pgfpathlineto{\pgfqpoint{3.857553in}{2.007940in}}%
\pgfpathlineto{\pgfqpoint{3.849509in}{1.998923in}}%
\pgfpathlineto{\pgfqpoint{3.841460in}{1.989883in}}%
\pgfpathclose%
\pgfusepath{fill}%
\end{pgfscope}%
\begin{pgfscope}%
\pgfpathrectangle{\pgfqpoint{1.150000in}{0.150000in}}{\pgfqpoint{5.700000in}{5.700000in}}%
\pgfusepath{clip}%
\pgfsetbuttcap%
\pgfsetroundjoin%
\definecolor{currentfill}{rgb}{0.282623,0.140926,0.457517}%
\pgfsetfillcolor{currentfill}%
\pgfsetfillopacity{0.700000}%
\pgfsetlinewidth{0.000000pt}%
\definecolor{currentstroke}{rgb}{0.000000,0.000000,0.000000}%
\pgfsetstrokecolor{currentstroke}%
\pgfsetdash{}{0pt}%
\pgfpathmoveto{\pgfqpoint{4.244992in}{2.112080in}}%
\pgfpathlineto{\pgfqpoint{4.258848in}{2.111883in}}%
\pgfpathlineto{\pgfqpoint{4.272713in}{2.111760in}}%
\pgfpathlineto{\pgfqpoint{4.286587in}{2.111711in}}%
\pgfpathlineto{\pgfqpoint{4.300469in}{2.111737in}}%
\pgfpathlineto{\pgfqpoint{4.308363in}{2.120138in}}%
\pgfpathlineto{\pgfqpoint{4.316250in}{2.128496in}}%
\pgfpathlineto{\pgfqpoint{4.324131in}{2.136814in}}%
\pgfpathlineto{\pgfqpoint{4.332006in}{2.145092in}}%
\pgfpathlineto{\pgfqpoint{4.318135in}{2.145129in}}%
\pgfpathlineto{\pgfqpoint{4.304273in}{2.145240in}}%
\pgfpathlineto{\pgfqpoint{4.290419in}{2.145426in}}%
\pgfpathlineto{\pgfqpoint{4.276574in}{2.145686in}}%
\pgfpathlineto{\pgfqpoint{4.268687in}{2.137337in}}%
\pgfpathlineto{\pgfqpoint{4.260795in}{2.128954in}}%
\pgfpathlineto{\pgfqpoint{4.252896in}{2.120535in}}%
\pgfpathlineto{\pgfqpoint{4.244992in}{2.112080in}}%
\pgfpathclose%
\pgfusepath{fill}%
\end{pgfscope}%
\begin{pgfscope}%
\pgfpathrectangle{\pgfqpoint{1.150000in}{0.150000in}}{\pgfqpoint{5.700000in}{5.700000in}}%
\pgfusepath{clip}%
\pgfsetbuttcap%
\pgfsetroundjoin%
\definecolor{currentfill}{rgb}{0.274952,0.037752,0.364543}%
\pgfsetfillcolor{currentfill}%
\pgfsetfillopacity{0.700000}%
\pgfsetlinewidth{0.000000pt}%
\definecolor{currentstroke}{rgb}{0.000000,0.000000,0.000000}%
\pgfsetstrokecolor{currentstroke}%
\pgfsetdash{}{0pt}%
\pgfpathmoveto{\pgfqpoint{3.524935in}{1.920418in}}%
\pgfpathlineto{\pgfqpoint{3.538607in}{1.917258in}}%
\pgfpathlineto{\pgfqpoint{3.552285in}{1.914182in}}%
\pgfpathlineto{\pgfqpoint{3.565968in}{1.911188in}}%
\pgfpathlineto{\pgfqpoint{3.579658in}{1.908276in}}%
\pgfpathlineto{\pgfqpoint{3.587810in}{1.917307in}}%
\pgfpathlineto{\pgfqpoint{3.595956in}{1.926335in}}%
\pgfpathlineto{\pgfqpoint{3.604096in}{1.935358in}}%
\pgfpathlineto{\pgfqpoint{3.612230in}{1.944378in}}%
\pgfpathlineto{\pgfqpoint{3.598553in}{1.947208in}}%
\pgfpathlineto{\pgfqpoint{3.584881in}{1.950120in}}%
\pgfpathlineto{\pgfqpoint{3.571215in}{1.953114in}}%
\pgfpathlineto{\pgfqpoint{3.557556in}{1.956192in}}%
\pgfpathlineto{\pgfqpoint{3.549409in}{1.947247in}}%
\pgfpathlineto{\pgfqpoint{3.541257in}{1.938302in}}%
\pgfpathlineto{\pgfqpoint{3.533099in}{1.929359in}}%
\pgfpathlineto{\pgfqpoint{3.524935in}{1.920418in}}%
\pgfpathclose%
\pgfusepath{fill}%
\end{pgfscope}%
\begin{pgfscope}%
\pgfpathrectangle{\pgfqpoint{1.150000in}{0.150000in}}{\pgfqpoint{5.700000in}{5.700000in}}%
\pgfusepath{clip}%
\pgfsetbuttcap%
\pgfsetroundjoin%
\definecolor{currentfill}{rgb}{0.225863,0.330805,0.547314}%
\pgfsetfillcolor{currentfill}%
\pgfsetfillopacity{0.700000}%
\pgfsetlinewidth{0.000000pt}%
\definecolor{currentstroke}{rgb}{0.000000,0.000000,0.000000}%
\pgfsetstrokecolor{currentstroke}%
\pgfsetdash{}{0pt}%
\pgfpathmoveto{\pgfqpoint{5.542838in}{2.533926in}}%
\pgfpathlineto{\pgfqpoint{5.557141in}{2.535798in}}%
\pgfpathlineto{\pgfqpoint{5.571455in}{2.537737in}}%
\pgfpathlineto{\pgfqpoint{5.585780in}{2.539745in}}%
\pgfpathlineto{\pgfqpoint{5.600118in}{2.541820in}}%
\pgfpathlineto{\pgfqpoint{5.607437in}{2.546645in}}%
\pgfpathlineto{\pgfqpoint{5.614752in}{2.551535in}}%
\pgfpathlineto{\pgfqpoint{5.622063in}{2.556497in}}%
\pgfpathlineto{\pgfqpoint{5.629370in}{2.561536in}}%
\pgfpathlineto{\pgfqpoint{5.615057in}{2.559797in}}%
\pgfpathlineto{\pgfqpoint{5.600756in}{2.558126in}}%
\pgfpathlineto{\pgfqpoint{5.586466in}{2.556522in}}%
\pgfpathlineto{\pgfqpoint{5.572187in}{2.554985in}}%
\pgfpathlineto{\pgfqpoint{5.564856in}{2.549603in}}%
\pgfpathlineto{\pgfqpoint{5.557521in}{2.544304in}}%
\pgfpathlineto{\pgfqpoint{5.550182in}{2.539080in}}%
\pgfpathlineto{\pgfqpoint{5.542838in}{2.533926in}}%
\pgfpathclose%
\pgfusepath{fill}%
\end{pgfscope}%
\begin{pgfscope}%
\pgfpathrectangle{\pgfqpoint{1.150000in}{0.150000in}}{\pgfqpoint{5.700000in}{5.700000in}}%
\pgfusepath{clip}%
\pgfsetbuttcap%
\pgfsetroundjoin%
\definecolor{currentfill}{rgb}{0.248629,0.278775,0.534556}%
\pgfsetfillcolor{currentfill}%
\pgfsetfillopacity{0.700000}%
\pgfsetlinewidth{0.000000pt}%
\definecolor{currentstroke}{rgb}{0.000000,0.000000,0.000000}%
\pgfsetstrokecolor{currentstroke}%
\pgfsetdash{}{0pt}%
\pgfpathmoveto{\pgfqpoint{5.139255in}{2.410715in}}%
\pgfpathlineto{\pgfqpoint{5.153417in}{2.412350in}}%
\pgfpathlineto{\pgfqpoint{5.167590in}{2.414055in}}%
\pgfpathlineto{\pgfqpoint{5.181774in}{2.415830in}}%
\pgfpathlineto{\pgfqpoint{5.195969in}{2.417673in}}%
\pgfpathlineto{\pgfqpoint{5.203481in}{2.423419in}}%
\pgfpathlineto{\pgfqpoint{5.210987in}{2.429173in}}%
\pgfpathlineto{\pgfqpoint{5.218488in}{2.434939in}}%
\pgfpathlineto{\pgfqpoint{5.225982in}{2.440722in}}%
\pgfpathlineto{\pgfqpoint{5.211806in}{2.439130in}}%
\pgfpathlineto{\pgfqpoint{5.197641in}{2.437608in}}%
\pgfpathlineto{\pgfqpoint{5.183486in}{2.436155in}}%
\pgfpathlineto{\pgfqpoint{5.169343in}{2.434770in}}%
\pgfpathlineto{\pgfqpoint{5.161829in}{2.428728in}}%
\pgfpathlineto{\pgfqpoint{5.154310in}{2.422708in}}%
\pgfpathlineto{\pgfqpoint{5.146785in}{2.416705in}}%
\pgfpathlineto{\pgfqpoint{5.139255in}{2.410715in}}%
\pgfpathclose%
\pgfusepath{fill}%
\end{pgfscope}%
\begin{pgfscope}%
\pgfpathrectangle{\pgfqpoint{1.150000in}{0.150000in}}{\pgfqpoint{5.700000in}{5.700000in}}%
\pgfusepath{clip}%
\pgfsetbuttcap%
\pgfsetroundjoin%
\definecolor{currentfill}{rgb}{0.271305,0.019942,0.347269}%
\pgfsetfillcolor{currentfill}%
\pgfsetfillopacity{0.700000}%
\pgfsetlinewidth{0.000000pt}%
\definecolor{currentstroke}{rgb}{0.000000,0.000000,0.000000}%
\pgfsetstrokecolor{currentstroke}%
\pgfsetdash{}{0pt}%
\pgfpathmoveto{\pgfqpoint{3.295452in}{1.896315in}}%
\pgfpathlineto{\pgfqpoint{3.309087in}{1.891908in}}%
\pgfpathlineto{\pgfqpoint{3.322727in}{1.887588in}}%
\pgfpathlineto{\pgfqpoint{3.336372in}{1.883355in}}%
\pgfpathlineto{\pgfqpoint{3.350022in}{1.879208in}}%
\pgfpathlineto{\pgfqpoint{3.358262in}{1.887828in}}%
\pgfpathlineto{\pgfqpoint{3.366495in}{1.896471in}}%
\pgfpathlineto{\pgfqpoint{3.374722in}{1.905135in}}%
\pgfpathlineto{\pgfqpoint{3.382942in}{1.913818in}}%
\pgfpathlineto{\pgfqpoint{3.369306in}{1.917842in}}%
\pgfpathlineto{\pgfqpoint{3.355675in}{1.921952in}}%
\pgfpathlineto{\pgfqpoint{3.342049in}{1.926149in}}%
\pgfpathlineto{\pgfqpoint{3.328428in}{1.930433in}}%
\pgfpathlineto{\pgfqpoint{3.320193in}{1.921865in}}%
\pgfpathlineto{\pgfqpoint{3.311953in}{1.913322in}}%
\pgfpathlineto{\pgfqpoint{3.303706in}{1.904805in}}%
\pgfpathlineto{\pgfqpoint{3.295452in}{1.896315in}}%
\pgfpathclose%
\pgfusepath{fill}%
\end{pgfscope}%
\begin{pgfscope}%
\pgfpathrectangle{\pgfqpoint{1.150000in}{0.150000in}}{\pgfqpoint{5.700000in}{5.700000in}}%
\pgfusepath{clip}%
\pgfsetbuttcap%
\pgfsetroundjoin%
\definecolor{currentfill}{rgb}{0.282910,0.105393,0.426902}%
\pgfsetfillcolor{currentfill}%
\pgfsetfillopacity{0.700000}%
\pgfsetlinewidth{0.000000pt}%
\definecolor{currentstroke}{rgb}{0.000000,0.000000,0.000000}%
\pgfsetstrokecolor{currentstroke}%
\pgfsetdash{}{0pt}%
\pgfpathmoveto{\pgfqpoint{2.528064in}{2.064208in}}%
\pgfpathlineto{\pgfqpoint{2.541691in}{2.054495in}}%
\pgfpathlineto{\pgfqpoint{2.555318in}{2.044897in}}%
\pgfpathlineto{\pgfqpoint{2.568945in}{2.035413in}}%
\pgfpathlineto{\pgfqpoint{2.582572in}{2.026042in}}%
\pgfpathlineto{\pgfqpoint{2.591188in}{2.031175in}}%
\pgfpathlineto{\pgfqpoint{2.599793in}{2.036437in}}%
\pgfpathlineto{\pgfqpoint{2.608387in}{2.041824in}}%
\pgfpathlineto{\pgfqpoint{2.616970in}{2.047333in}}%
\pgfpathlineto{\pgfqpoint{2.603368in}{2.056476in}}%
\pgfpathlineto{\pgfqpoint{2.589766in}{2.065731in}}%
\pgfpathlineto{\pgfqpoint{2.576164in}{2.075099in}}%
\pgfpathlineto{\pgfqpoint{2.562562in}{2.084582in}}%
\pgfpathlineto{\pgfqpoint{2.553955in}{2.079294in}}%
\pgfpathlineto{\pgfqpoint{2.545336in}{2.074134in}}%
\pgfpathlineto{\pgfqpoint{2.536705in}{2.069104in}}%
\pgfpathlineto{\pgfqpoint{2.528064in}{2.064208in}}%
\pgfpathclose%
\pgfusepath{fill}%
\end{pgfscope}%
\begin{pgfscope}%
\pgfpathrectangle{\pgfqpoint{1.150000in}{0.150000in}}{\pgfqpoint{5.700000in}{5.700000in}}%
\pgfusepath{clip}%
\pgfsetbuttcap%
\pgfsetroundjoin%
\definecolor{currentfill}{rgb}{0.278791,0.062145,0.386592}%
\pgfsetfillcolor{currentfill}%
\pgfsetfillopacity{0.700000}%
\pgfsetlinewidth{0.000000pt}%
\definecolor{currentstroke}{rgb}{0.000000,0.000000,0.000000}%
\pgfsetstrokecolor{currentstroke}%
\pgfsetdash{}{0pt}%
\pgfpathmoveto{\pgfqpoint{2.725812in}{1.978154in}}%
\pgfpathlineto{\pgfqpoint{2.739421in}{1.969990in}}%
\pgfpathlineto{\pgfqpoint{2.753032in}{1.961931in}}%
\pgfpathlineto{\pgfqpoint{2.766644in}{1.953976in}}%
\pgfpathlineto{\pgfqpoint{2.780258in}{1.946125in}}%
\pgfpathlineto{\pgfqpoint{2.788762in}{1.952398in}}%
\pgfpathlineto{\pgfqpoint{2.797257in}{1.958771in}}%
\pgfpathlineto{\pgfqpoint{2.805742in}{1.965242in}}%
\pgfpathlineto{\pgfqpoint{2.814218in}{1.971807in}}%
\pgfpathlineto{\pgfqpoint{2.800625in}{1.979452in}}%
\pgfpathlineto{\pgfqpoint{2.787035in}{1.987200in}}%
\pgfpathlineto{\pgfqpoint{2.773446in}{1.995052in}}%
\pgfpathlineto{\pgfqpoint{2.759858in}{2.003009in}}%
\pgfpathlineto{\pgfqpoint{2.751361in}{1.996643in}}%
\pgfpathlineto{\pgfqpoint{2.742854in}{1.990377in}}%
\pgfpathlineto{\pgfqpoint{2.734338in}{1.984213in}}%
\pgfpathlineto{\pgfqpoint{2.725812in}{1.978154in}}%
\pgfpathclose%
\pgfusepath{fill}%
\end{pgfscope}%
\begin{pgfscope}%
\pgfpathrectangle{\pgfqpoint{1.150000in}{0.150000in}}{\pgfqpoint{5.700000in}{5.700000in}}%
\pgfusepath{clip}%
\pgfsetbuttcap%
\pgfsetroundjoin%
\definecolor{currentfill}{rgb}{0.210503,0.363727,0.552206}%
\pgfsetfillcolor{currentfill}%
\pgfsetfillopacity{0.700000}%
\pgfsetlinewidth{0.000000pt}%
\definecolor{currentstroke}{rgb}{0.000000,0.000000,0.000000}%
\pgfsetstrokecolor{currentstroke}%
\pgfsetdash{}{0pt}%
\pgfpathmoveto{\pgfqpoint{5.859746in}{2.622679in}}%
\pgfpathlineto{\pgfqpoint{5.874158in}{2.624555in}}%
\pgfpathlineto{\pgfqpoint{5.888582in}{2.626499in}}%
\pgfpathlineto{\pgfqpoint{5.903018in}{2.628509in}}%
\pgfpathlineto{\pgfqpoint{5.910198in}{2.633064in}}%
\pgfpathlineto{\pgfqpoint{5.917375in}{2.637743in}}%
\pgfpathlineto{\pgfqpoint{5.924552in}{2.642551in}}%
\pgfpathlineto{\pgfqpoint{5.931727in}{2.647497in}}%
\pgfpathlineto{\pgfqpoint{5.917320in}{2.645885in}}%
\pgfpathlineto{\pgfqpoint{5.902925in}{2.644340in}}%
\pgfpathlineto{\pgfqpoint{5.888542in}{2.642860in}}%
\pgfpathlineto{\pgfqpoint{5.881345in}{2.637611in}}%
\pgfpathlineto{\pgfqpoint{5.874147in}{2.632502in}}%
\pgfpathlineto{\pgfqpoint{5.866948in}{2.627527in}}%
\pgfpathlineto{\pgfqpoint{5.859746in}{2.622679in}}%
\pgfpathclose%
\pgfusepath{fill}%
\end{pgfscope}%
\begin{pgfscope}%
\pgfpathrectangle{\pgfqpoint{1.150000in}{0.150000in}}{\pgfqpoint{5.700000in}{5.700000in}}%
\pgfusepath{clip}%
\pgfsetbuttcap%
\pgfsetroundjoin%
\definecolor{currentfill}{rgb}{0.273006,0.204520,0.501721}%
\pgfsetfillcolor{currentfill}%
\pgfsetfillopacity{0.700000}%
\pgfsetlinewidth{0.000000pt}%
\definecolor{currentstroke}{rgb}{0.000000,0.000000,0.000000}%
\pgfsetstrokecolor{currentstroke}%
\pgfsetdash{}{0pt}%
\pgfpathmoveto{\pgfqpoint{4.648646in}{2.246420in}}%
\pgfpathlineto{\pgfqpoint{4.662638in}{2.247308in}}%
\pgfpathlineto{\pgfqpoint{4.676640in}{2.248268in}}%
\pgfpathlineto{\pgfqpoint{4.690652in}{2.249299in}}%
\pgfpathlineto{\pgfqpoint{4.704674in}{2.250402in}}%
\pgfpathlineto{\pgfqpoint{4.712408in}{2.257660in}}%
\pgfpathlineto{\pgfqpoint{4.720135in}{2.264882in}}%
\pgfpathlineto{\pgfqpoint{4.727856in}{2.272074in}}%
\pgfpathlineto{\pgfqpoint{4.735571in}{2.279237in}}%
\pgfpathlineto{\pgfqpoint{4.721563in}{2.278280in}}%
\pgfpathlineto{\pgfqpoint{4.707565in}{2.277396in}}%
\pgfpathlineto{\pgfqpoint{4.693576in}{2.276582in}}%
\pgfpathlineto{\pgfqpoint{4.679597in}{2.275840in}}%
\pgfpathlineto{\pgfqpoint{4.671869in}{2.268523in}}%
\pgfpathlineto{\pgfqpoint{4.664134in}{2.261183in}}%
\pgfpathlineto{\pgfqpoint{4.656393in}{2.253816in}}%
\pgfpathlineto{\pgfqpoint{4.648646in}{2.246420in}}%
\pgfpathclose%
\pgfusepath{fill}%
\end{pgfscope}%
\begin{pgfscope}%
\pgfpathrectangle{\pgfqpoint{1.150000in}{0.150000in}}{\pgfqpoint{5.700000in}{5.700000in}}%
\pgfusepath{clip}%
\pgfsetbuttcap%
\pgfsetroundjoin%
\definecolor{currentfill}{rgb}{0.278012,0.180367,0.486697}%
\pgfsetfillcolor{currentfill}%
\pgfsetfillopacity{0.700000}%
\pgfsetlinewidth{0.000000pt}%
\definecolor{currentstroke}{rgb}{0.000000,0.000000,0.000000}%
\pgfsetstrokecolor{currentstroke}%
\pgfsetdash{}{0pt}%
\pgfpathmoveto{\pgfqpoint{2.274765in}{2.221817in}}%
\pgfpathlineto{\pgfqpoint{2.288441in}{2.209887in}}%
\pgfpathlineto{\pgfqpoint{2.302116in}{2.198087in}}%
\pgfpathlineto{\pgfqpoint{2.315787in}{2.186418in}}%
\pgfpathlineto{\pgfqpoint{2.329457in}{2.174877in}}%
\pgfpathlineto{\pgfqpoint{2.338233in}{2.178423in}}%
\pgfpathlineto{\pgfqpoint{2.346996in}{2.182133in}}%
\pgfpathlineto{\pgfqpoint{2.355745in}{2.186003in}}%
\pgfpathlineto{\pgfqpoint{2.364481in}{2.190030in}}%
\pgfpathlineto{\pgfqpoint{2.350841in}{2.201317in}}%
\pgfpathlineto{\pgfqpoint{2.337199in}{2.212733in}}%
\pgfpathlineto{\pgfqpoint{2.323554in}{2.224279in}}%
\pgfpathlineto{\pgfqpoint{2.309908in}{2.235955in}}%
\pgfpathlineto{\pgfqpoint{2.301143in}{2.232174in}}%
\pgfpathlineto{\pgfqpoint{2.292364in}{2.228555in}}%
\pgfpathlineto{\pgfqpoint{2.283571in}{2.225101in}}%
\pgfpathlineto{\pgfqpoint{2.274765in}{2.221817in}}%
\pgfpathclose%
\pgfusepath{fill}%
\end{pgfscope}%
\begin{pgfscope}%
\pgfpathrectangle{\pgfqpoint{1.150000in}{0.150000in}}{\pgfqpoint{5.700000in}{5.700000in}}%
\pgfusepath{clip}%
\pgfsetbuttcap%
\pgfsetroundjoin%
\definecolor{currentfill}{rgb}{0.283187,0.125848,0.444960}%
\pgfsetfillcolor{currentfill}%
\pgfsetfillopacity{0.700000}%
\pgfsetlinewidth{0.000000pt}%
\definecolor{currentstroke}{rgb}{0.000000,0.000000,0.000000}%
\pgfsetstrokecolor{currentstroke}%
\pgfsetdash{}{0pt}%
\pgfpathmoveto{\pgfqpoint{4.157933in}{2.079231in}}%
\pgfpathlineto{\pgfqpoint{4.171766in}{2.078775in}}%
\pgfpathlineto{\pgfqpoint{4.185608in}{2.078396in}}%
\pgfpathlineto{\pgfqpoint{4.199458in}{2.078091in}}%
\pgfpathlineto{\pgfqpoint{4.213317in}{2.077861in}}%
\pgfpathlineto{\pgfqpoint{4.221244in}{2.086478in}}%
\pgfpathlineto{\pgfqpoint{4.229166in}{2.095052in}}%
\pgfpathlineto{\pgfqpoint{4.237082in}{2.103586in}}%
\pgfpathlineto{\pgfqpoint{4.244992in}{2.112080in}}%
\pgfpathlineto{\pgfqpoint{4.231144in}{2.112352in}}%
\pgfpathlineto{\pgfqpoint{4.217305in}{2.112699in}}%
\pgfpathlineto{\pgfqpoint{4.203474in}{2.113121in}}%
\pgfpathlineto{\pgfqpoint{4.189652in}{2.113617in}}%
\pgfpathlineto{\pgfqpoint{4.181731in}{2.105074in}}%
\pgfpathlineto{\pgfqpoint{4.173804in}{2.096496in}}%
\pgfpathlineto{\pgfqpoint{4.165871in}{2.087882in}}%
\pgfpathlineto{\pgfqpoint{4.157933in}{2.079231in}}%
\pgfpathclose%
\pgfusepath{fill}%
\end{pgfscope}%
\begin{pgfscope}%
\pgfpathrectangle{\pgfqpoint{1.150000in}{0.150000in}}{\pgfqpoint{5.700000in}{5.700000in}}%
\pgfusepath{clip}%
\pgfsetbuttcap%
\pgfsetroundjoin%
\definecolor{currentfill}{rgb}{0.278791,0.062145,0.386592}%
\pgfsetfillcolor{currentfill}%
\pgfsetfillopacity{0.700000}%
\pgfsetlinewidth{0.000000pt}%
\definecolor{currentstroke}{rgb}{0.000000,0.000000,0.000000}%
\pgfsetstrokecolor{currentstroke}%
\pgfsetdash{}{0pt}%
\pgfpathmoveto{\pgfqpoint{3.754267in}{1.961177in}}%
\pgfpathlineto{\pgfqpoint{3.767991in}{1.959136in}}%
\pgfpathlineto{\pgfqpoint{3.781723in}{1.957175in}}%
\pgfpathlineto{\pgfqpoint{3.795461in}{1.955293in}}%
\pgfpathlineto{\pgfqpoint{3.809206in}{1.953490in}}%
\pgfpathlineto{\pgfqpoint{3.817278in}{1.962623in}}%
\pgfpathlineto{\pgfqpoint{3.825344in}{1.971733in}}%
\pgfpathlineto{\pgfqpoint{3.833405in}{1.980819in}}%
\pgfpathlineto{\pgfqpoint{3.841460in}{1.989883in}}%
\pgfpathlineto{\pgfqpoint{3.827725in}{1.991645in}}%
\pgfpathlineto{\pgfqpoint{3.813998in}{1.993487in}}%
\pgfpathlineto{\pgfqpoint{3.800278in}{1.995407in}}%
\pgfpathlineto{\pgfqpoint{3.786564in}{1.997407in}}%
\pgfpathlineto{\pgfqpoint{3.778499in}{1.988377in}}%
\pgfpathlineto{\pgfqpoint{3.770427in}{1.979328in}}%
\pgfpathlineto{\pgfqpoint{3.762350in}{1.970261in}}%
\pgfpathlineto{\pgfqpoint{3.754267in}{1.961177in}}%
\pgfpathclose%
\pgfusepath{fill}%
\end{pgfscope}%
\begin{pgfscope}%
\pgfpathrectangle{\pgfqpoint{1.150000in}{0.150000in}}{\pgfqpoint{5.700000in}{5.700000in}}%
\pgfusepath{clip}%
\pgfsetbuttcap%
\pgfsetroundjoin%
\definecolor{currentfill}{rgb}{0.253935,0.265254,0.529983}%
\pgfsetfillcolor{currentfill}%
\pgfsetfillopacity{0.700000}%
\pgfsetlinewidth{0.000000pt}%
\definecolor{currentstroke}{rgb}{0.000000,0.000000,0.000000}%
\pgfsetstrokecolor{currentstroke}%
\pgfsetdash{}{0pt}%
\pgfpathmoveto{\pgfqpoint{5.052460in}{2.380021in}}%
\pgfpathlineto{\pgfqpoint{5.066597in}{2.381609in}}%
\pgfpathlineto{\pgfqpoint{5.080744in}{2.383268in}}%
\pgfpathlineto{\pgfqpoint{5.094903in}{2.384996in}}%
\pgfpathlineto{\pgfqpoint{5.109072in}{2.386794in}}%
\pgfpathlineto{\pgfqpoint{5.116627in}{2.392776in}}%
\pgfpathlineto{\pgfqpoint{5.124176in}{2.398754in}}%
\pgfpathlineto{\pgfqpoint{5.131718in}{2.404733in}}%
\pgfpathlineto{\pgfqpoint{5.139255in}{2.410715in}}%
\pgfpathlineto{\pgfqpoint{5.125103in}{2.409148in}}%
\pgfpathlineto{\pgfqpoint{5.110962in}{2.407651in}}%
\pgfpathlineto{\pgfqpoint{5.096832in}{2.406224in}}%
\pgfpathlineto{\pgfqpoint{5.082713in}{2.404865in}}%
\pgfpathlineto{\pgfqpoint{5.075159in}{2.398645in}}%
\pgfpathlineto{\pgfqpoint{5.067598in}{2.392434in}}%
\pgfpathlineto{\pgfqpoint{5.060032in}{2.386227in}}%
\pgfpathlineto{\pgfqpoint{5.052460in}{2.380021in}}%
\pgfpathclose%
\pgfusepath{fill}%
\end{pgfscope}%
\begin{pgfscope}%
\pgfpathrectangle{\pgfqpoint{1.150000in}{0.150000in}}{\pgfqpoint{5.700000in}{5.700000in}}%
\pgfusepath{clip}%
\pgfsetbuttcap%
\pgfsetroundjoin%
\definecolor{currentfill}{rgb}{0.229739,0.322361,0.545706}%
\pgfsetfillcolor{currentfill}%
\pgfsetfillopacity{0.700000}%
\pgfsetlinewidth{0.000000pt}%
\definecolor{currentstroke}{rgb}{0.000000,0.000000,0.000000}%
\pgfsetstrokecolor{currentstroke}%
\pgfsetdash{}{0pt}%
\pgfpathmoveto{\pgfqpoint{5.456233in}{2.505830in}}%
\pgfpathlineto{\pgfqpoint{5.470512in}{2.507746in}}%
\pgfpathlineto{\pgfqpoint{5.484803in}{2.509729in}}%
\pgfpathlineto{\pgfqpoint{5.499106in}{2.511780in}}%
\pgfpathlineto{\pgfqpoint{5.513420in}{2.513900in}}%
\pgfpathlineto{\pgfqpoint{5.520782in}{2.518829in}}%
\pgfpathlineto{\pgfqpoint{5.528139in}{2.523807in}}%
\pgfpathlineto{\pgfqpoint{5.535491in}{2.528837in}}%
\pgfpathlineto{\pgfqpoint{5.542838in}{2.533926in}}%
\pgfpathlineto{\pgfqpoint{5.528547in}{2.532122in}}%
\pgfpathlineto{\pgfqpoint{5.514268in}{2.530386in}}%
\pgfpathlineto{\pgfqpoint{5.500000in}{2.528718in}}%
\pgfpathlineto{\pgfqpoint{5.485743in}{2.527117in}}%
\pgfpathlineto{\pgfqpoint{5.478372in}{2.521706in}}%
\pgfpathlineto{\pgfqpoint{5.470997in}{2.516358in}}%
\pgfpathlineto{\pgfqpoint{5.463617in}{2.511068in}}%
\pgfpathlineto{\pgfqpoint{5.456233in}{2.505830in}}%
\pgfpathclose%
\pgfusepath{fill}%
\end{pgfscope}%
\begin{pgfscope}%
\pgfpathrectangle{\pgfqpoint{1.150000in}{0.150000in}}{\pgfqpoint{5.700000in}{5.700000in}}%
\pgfusepath{clip}%
\pgfsetbuttcap%
\pgfsetroundjoin%
\definecolor{currentfill}{rgb}{0.276194,0.190074,0.493001}%
\pgfsetfillcolor{currentfill}%
\pgfsetfillopacity{0.700000}%
\pgfsetlinewidth{0.000000pt}%
\definecolor{currentstroke}{rgb}{0.000000,0.000000,0.000000}%
\pgfsetstrokecolor{currentstroke}%
\pgfsetdash{}{0pt}%
\pgfpathmoveto{\pgfqpoint{4.561670in}{2.213154in}}%
\pgfpathlineto{\pgfqpoint{4.575636in}{2.213879in}}%
\pgfpathlineto{\pgfqpoint{4.589613in}{2.214677in}}%
\pgfpathlineto{\pgfqpoint{4.603598in}{2.215547in}}%
\pgfpathlineto{\pgfqpoint{4.617594in}{2.216490in}}%
\pgfpathlineto{\pgfqpoint{4.625366in}{2.224029in}}%
\pgfpathlineto{\pgfqpoint{4.633132in}{2.231529in}}%
\pgfpathlineto{\pgfqpoint{4.640892in}{2.238991in}}%
\pgfpathlineto{\pgfqpoint{4.648646in}{2.246420in}}%
\pgfpathlineto{\pgfqpoint{4.634663in}{2.245603in}}%
\pgfpathlineto{\pgfqpoint{4.620690in}{2.244859in}}%
\pgfpathlineto{\pgfqpoint{4.606727in}{2.244187in}}%
\pgfpathlineto{\pgfqpoint{4.592773in}{2.243586in}}%
\pgfpathlineto{\pgfqpoint{4.585006in}{2.236025in}}%
\pgfpathlineto{\pgfqpoint{4.577234in}{2.228434in}}%
\pgfpathlineto{\pgfqpoint{4.569455in}{2.220811in}}%
\pgfpathlineto{\pgfqpoint{4.561670in}{2.213154in}}%
\pgfpathclose%
\pgfusepath{fill}%
\end{pgfscope}%
\begin{pgfscope}%
\pgfpathrectangle{\pgfqpoint{1.150000in}{0.150000in}}{\pgfqpoint{5.700000in}{5.700000in}}%
\pgfusepath{clip}%
\pgfsetbuttcap%
\pgfsetroundjoin%
\definecolor{currentfill}{rgb}{0.272594,0.025563,0.353093}%
\pgfsetfillcolor{currentfill}%
\pgfsetfillopacity{0.700000}%
\pgfsetlinewidth{0.000000pt}%
\definecolor{currentstroke}{rgb}{0.000000,0.000000,0.000000}%
\pgfsetstrokecolor{currentstroke}%
\pgfsetdash{}{0pt}%
\pgfpathmoveto{\pgfqpoint{3.437538in}{1.898580in}}%
\pgfpathlineto{\pgfqpoint{3.451200in}{1.894983in}}%
\pgfpathlineto{\pgfqpoint{3.464867in}{1.891471in}}%
\pgfpathlineto{\pgfqpoint{3.478541in}{1.888042in}}%
\pgfpathlineto{\pgfqpoint{3.492219in}{1.884697in}}%
\pgfpathlineto{\pgfqpoint{3.500407in}{1.893618in}}%
\pgfpathlineto{\pgfqpoint{3.508589in}{1.902547in}}%
\pgfpathlineto{\pgfqpoint{3.516765in}{1.911480in}}%
\pgfpathlineto{\pgfqpoint{3.524935in}{1.920418in}}%
\pgfpathlineto{\pgfqpoint{3.511269in}{1.923661in}}%
\pgfpathlineto{\pgfqpoint{3.497609in}{1.926987in}}%
\pgfpathlineto{\pgfqpoint{3.483954in}{1.930397in}}%
\pgfpathlineto{\pgfqpoint{3.470304in}{1.933892in}}%
\pgfpathlineto{\pgfqpoint{3.462122in}{1.925049in}}%
\pgfpathlineto{\pgfqpoint{3.453933in}{1.916215in}}%
\pgfpathlineto{\pgfqpoint{3.445739in}{1.907392in}}%
\pgfpathlineto{\pgfqpoint{3.437538in}{1.898580in}}%
\pgfpathclose%
\pgfusepath{fill}%
\end{pgfscope}%
\begin{pgfscope}%
\pgfpathrectangle{\pgfqpoint{1.150000in}{0.150000in}}{\pgfqpoint{5.700000in}{5.700000in}}%
\pgfusepath{clip}%
\pgfsetbuttcap%
\pgfsetroundjoin%
\definecolor{currentfill}{rgb}{0.271305,0.019942,0.347269}%
\pgfsetfillcolor{currentfill}%
\pgfsetfillopacity{0.700000}%
\pgfsetlinewidth{0.000000pt}%
\definecolor{currentstroke}{rgb}{0.000000,0.000000,0.000000}%
\pgfsetstrokecolor{currentstroke}%
\pgfsetdash{}{0pt}%
\pgfpathmoveto{\pgfqpoint{3.065514in}{1.893153in}}%
\pgfpathlineto{\pgfqpoint{3.079132in}{1.887356in}}%
\pgfpathlineto{\pgfqpoint{3.092754in}{1.881652in}}%
\pgfpathlineto{\pgfqpoint{3.106379in}{1.876040in}}%
\pgfpathlineto{\pgfqpoint{3.120008in}{1.870520in}}%
\pgfpathlineto{\pgfqpoint{3.128348in}{1.878379in}}%
\pgfpathlineto{\pgfqpoint{3.136682in}{1.886291in}}%
\pgfpathlineto{\pgfqpoint{3.145007in}{1.894255in}}%
\pgfpathlineto{\pgfqpoint{3.153326in}{1.902267in}}%
\pgfpathlineto{\pgfqpoint{3.139714in}{1.907623in}}%
\pgfpathlineto{\pgfqpoint{3.126106in}{1.913070in}}%
\pgfpathlineto{\pgfqpoint{3.112501in}{1.918610in}}%
\pgfpathlineto{\pgfqpoint{3.098900in}{1.924243in}}%
\pgfpathlineto{\pgfqpoint{3.090565in}{1.916387in}}%
\pgfpathlineto{\pgfqpoint{3.082222in}{1.908585in}}%
\pgfpathlineto{\pgfqpoint{3.073872in}{1.900839in}}%
\pgfpathlineto{\pgfqpoint{3.065514in}{1.893153in}}%
\pgfpathclose%
\pgfusepath{fill}%
\end{pgfscope}%
\begin{pgfscope}%
\pgfpathrectangle{\pgfqpoint{1.150000in}{0.150000in}}{\pgfqpoint{5.700000in}{5.700000in}}%
\pgfusepath{clip}%
\pgfsetbuttcap%
\pgfsetroundjoin%
\definecolor{currentfill}{rgb}{0.283091,0.110553,0.431554}%
\pgfsetfillcolor{currentfill}%
\pgfsetfillopacity{0.700000}%
\pgfsetlinewidth{0.000000pt}%
\definecolor{currentstroke}{rgb}{0.000000,0.000000,0.000000}%
\pgfsetstrokecolor{currentstroke}%
\pgfsetdash{}{0pt}%
\pgfpathmoveto{\pgfqpoint{4.070828in}{2.046729in}}%
\pgfpathlineto{\pgfqpoint{4.084639in}{2.045992in}}%
\pgfpathlineto{\pgfqpoint{4.098458in}{2.045332in}}%
\pgfpathlineto{\pgfqpoint{4.112286in}{2.044747in}}%
\pgfpathlineto{\pgfqpoint{4.126122in}{2.044238in}}%
\pgfpathlineto{\pgfqpoint{4.134083in}{2.053046in}}%
\pgfpathlineto{\pgfqpoint{4.142039in}{2.061814in}}%
\pgfpathlineto{\pgfqpoint{4.149989in}{2.070542in}}%
\pgfpathlineto{\pgfqpoint{4.157933in}{2.079231in}}%
\pgfpathlineto{\pgfqpoint{4.144108in}{2.079761in}}%
\pgfpathlineto{\pgfqpoint{4.130291in}{2.080367in}}%
\pgfpathlineto{\pgfqpoint{4.116482in}{2.081049in}}%
\pgfpathlineto{\pgfqpoint{4.102682in}{2.081807in}}%
\pgfpathlineto{\pgfqpoint{4.094727in}{2.073089in}}%
\pgfpathlineto{\pgfqpoint{4.086766in}{2.064337in}}%
\pgfpathlineto{\pgfqpoint{4.078800in}{2.055551in}}%
\pgfpathlineto{\pgfqpoint{4.070828in}{2.046729in}}%
\pgfpathclose%
\pgfusepath{fill}%
\end{pgfscope}%
\begin{pgfscope}%
\pgfpathrectangle{\pgfqpoint{1.150000in}{0.150000in}}{\pgfqpoint{5.700000in}{5.700000in}}%
\pgfusepath{clip}%
\pgfsetbuttcap%
\pgfsetroundjoin%
\definecolor{currentfill}{rgb}{0.273809,0.031497,0.358853}%
\pgfsetfillcolor{currentfill}%
\pgfsetfillopacity{0.700000}%
\pgfsetlinewidth{0.000000pt}%
\definecolor{currentstroke}{rgb}{0.000000,0.000000,0.000000}%
\pgfsetstrokecolor{currentstroke}%
\pgfsetdash{}{0pt}%
\pgfpathmoveto{\pgfqpoint{2.923030in}{1.914294in}}%
\pgfpathlineto{\pgfqpoint{2.936642in}{1.907551in}}%
\pgfpathlineto{\pgfqpoint{2.950257in}{1.900905in}}%
\pgfpathlineto{\pgfqpoint{2.963874in}{1.894356in}}%
\pgfpathlineto{\pgfqpoint{2.977495in}{1.887902in}}%
\pgfpathlineto{\pgfqpoint{2.985902in}{1.895138in}}%
\pgfpathlineto{\pgfqpoint{2.994301in}{1.902448in}}%
\pgfpathlineto{\pgfqpoint{3.002693in}{1.909828in}}%
\pgfpathlineto{\pgfqpoint{3.011076in}{1.917277in}}%
\pgfpathlineto{\pgfqpoint{2.997474in}{1.923546in}}%
\pgfpathlineto{\pgfqpoint{2.983875in}{1.929910in}}%
\pgfpathlineto{\pgfqpoint{2.970279in}{1.936371in}}%
\pgfpathlineto{\pgfqpoint{2.956686in}{1.942929in}}%
\pgfpathlineto{\pgfqpoint{2.948285in}{1.935657in}}%
\pgfpathlineto{\pgfqpoint{2.939875in}{1.928459in}}%
\pgfpathlineto{\pgfqpoint{2.931457in}{1.921337in}}%
\pgfpathlineto{\pgfqpoint{2.923030in}{1.914294in}}%
\pgfpathclose%
\pgfusepath{fill}%
\end{pgfscope}%
\begin{pgfscope}%
\pgfpathrectangle{\pgfqpoint{1.150000in}{0.150000in}}{\pgfqpoint{5.700000in}{5.700000in}}%
\pgfusepath{clip}%
\pgfsetbuttcap%
\pgfsetroundjoin%
\definecolor{currentfill}{rgb}{0.280868,0.160771,0.472899}%
\pgfsetfillcolor{currentfill}%
\pgfsetfillopacity{0.700000}%
\pgfsetlinewidth{0.000000pt}%
\definecolor{currentstroke}{rgb}{0.000000,0.000000,0.000000}%
\pgfsetstrokecolor{currentstroke}%
\pgfsetdash{}{0pt}%
\pgfpathmoveto{\pgfqpoint{2.329457in}{2.174877in}}%
\pgfpathlineto{\pgfqpoint{2.343125in}{2.163463in}}%
\pgfpathlineto{\pgfqpoint{2.356791in}{2.152176in}}%
\pgfpathlineto{\pgfqpoint{2.370455in}{2.141014in}}%
\pgfpathlineto{\pgfqpoint{2.384118in}{2.129976in}}%
\pgfpathlineto{\pgfqpoint{2.392864in}{2.133783in}}%
\pgfpathlineto{\pgfqpoint{2.401598in}{2.137749in}}%
\pgfpathlineto{\pgfqpoint{2.410318in}{2.141869in}}%
\pgfpathlineto{\pgfqpoint{2.419026in}{2.146141in}}%
\pgfpathlineto{\pgfqpoint{2.405392in}{2.156926in}}%
\pgfpathlineto{\pgfqpoint{2.391757in}{2.167835in}}%
\pgfpathlineto{\pgfqpoint{2.378120in}{2.178869in}}%
\pgfpathlineto{\pgfqpoint{2.364481in}{2.190030in}}%
\pgfpathlineto{\pgfqpoint{2.355745in}{2.186003in}}%
\pgfpathlineto{\pgfqpoint{2.346996in}{2.182133in}}%
\pgfpathlineto{\pgfqpoint{2.338233in}{2.178423in}}%
\pgfpathlineto{\pgfqpoint{2.329457in}{2.174877in}}%
\pgfpathclose%
\pgfusepath{fill}%
\end{pgfscope}%
\begin{pgfscope}%
\pgfpathrectangle{\pgfqpoint{1.150000in}{0.150000in}}{\pgfqpoint{5.700000in}{5.700000in}}%
\pgfusepath{clip}%
\pgfsetbuttcap%
\pgfsetroundjoin%
\definecolor{currentfill}{rgb}{0.281924,0.089666,0.412415}%
\pgfsetfillcolor{currentfill}%
\pgfsetfillopacity{0.700000}%
\pgfsetlinewidth{0.000000pt}%
\definecolor{currentstroke}{rgb}{0.000000,0.000000,0.000000}%
\pgfsetstrokecolor{currentstroke}%
\pgfsetdash{}{0pt}%
\pgfpathmoveto{\pgfqpoint{2.582572in}{2.026042in}}%
\pgfpathlineto{\pgfqpoint{2.596199in}{2.016782in}}%
\pgfpathlineto{\pgfqpoint{2.609827in}{2.007634in}}%
\pgfpathlineto{\pgfqpoint{2.623455in}{1.998596in}}%
\pgfpathlineto{\pgfqpoint{2.637084in}{1.989668in}}%
\pgfpathlineto{\pgfqpoint{2.645675in}{1.995038in}}%
\pgfpathlineto{\pgfqpoint{2.654255in}{2.000531in}}%
\pgfpathlineto{\pgfqpoint{2.662825in}{2.006144in}}%
\pgfpathlineto{\pgfqpoint{2.671385in}{2.011874in}}%
\pgfpathlineto{\pgfqpoint{2.657780in}{2.020574in}}%
\pgfpathlineto{\pgfqpoint{2.644176in}{2.029383in}}%
\pgfpathlineto{\pgfqpoint{2.630573in}{2.038303in}}%
\pgfpathlineto{\pgfqpoint{2.616970in}{2.047333in}}%
\pgfpathlineto{\pgfqpoint{2.608387in}{2.041824in}}%
\pgfpathlineto{\pgfqpoint{2.599793in}{2.036437in}}%
\pgfpathlineto{\pgfqpoint{2.591188in}{2.031175in}}%
\pgfpathlineto{\pgfqpoint{2.582572in}{2.026042in}}%
\pgfpathclose%
\pgfusepath{fill}%
\end{pgfscope}%
\begin{pgfscope}%
\pgfpathrectangle{\pgfqpoint{1.150000in}{0.150000in}}{\pgfqpoint{5.700000in}{5.700000in}}%
\pgfusepath{clip}%
\pgfsetbuttcap%
\pgfsetroundjoin%
\definecolor{currentfill}{rgb}{0.277018,0.050344,0.375715}%
\pgfsetfillcolor{currentfill}%
\pgfsetfillopacity{0.700000}%
\pgfsetlinewidth{0.000000pt}%
\definecolor{currentstroke}{rgb}{0.000000,0.000000,0.000000}%
\pgfsetstrokecolor{currentstroke}%
\pgfsetdash{}{0pt}%
\pgfpathmoveto{\pgfqpoint{3.667003in}{1.933876in}}%
\pgfpathlineto{\pgfqpoint{3.680712in}{1.931453in}}%
\pgfpathlineto{\pgfqpoint{3.694428in}{1.929111in}}%
\pgfpathlineto{\pgfqpoint{3.708150in}{1.926849in}}%
\pgfpathlineto{\pgfqpoint{3.721879in}{1.924667in}}%
\pgfpathlineto{\pgfqpoint{3.729985in}{1.933819in}}%
\pgfpathlineto{\pgfqpoint{3.738084in}{1.942955in}}%
\pgfpathlineto{\pgfqpoint{3.746179in}{1.952075in}}%
\pgfpathlineto{\pgfqpoint{3.754267in}{1.961177in}}%
\pgfpathlineto{\pgfqpoint{3.740549in}{1.963297in}}%
\pgfpathlineto{\pgfqpoint{3.726838in}{1.965498in}}%
\pgfpathlineto{\pgfqpoint{3.713134in}{1.967779in}}%
\pgfpathlineto{\pgfqpoint{3.699436in}{1.970140in}}%
\pgfpathlineto{\pgfqpoint{3.691337in}{1.961092in}}%
\pgfpathlineto{\pgfqpoint{3.683231in}{1.952031in}}%
\pgfpathlineto{\pgfqpoint{3.675120in}{1.942959in}}%
\pgfpathlineto{\pgfqpoint{3.667003in}{1.933876in}}%
\pgfpathclose%
\pgfusepath{fill}%
\end{pgfscope}%
\begin{pgfscope}%
\pgfpathrectangle{\pgfqpoint{1.150000in}{0.150000in}}{\pgfqpoint{5.700000in}{5.700000in}}%
\pgfusepath{clip}%
\pgfsetbuttcap%
\pgfsetroundjoin%
\definecolor{currentfill}{rgb}{0.269944,0.014625,0.341379}%
\pgfsetfillcolor{currentfill}%
\pgfsetfillopacity{0.700000}%
\pgfsetlinewidth{0.000000pt}%
\definecolor{currentstroke}{rgb}{0.000000,0.000000,0.000000}%
\pgfsetstrokecolor{currentstroke}%
\pgfsetdash{}{0pt}%
\pgfpathmoveto{\pgfqpoint{3.207815in}{1.881752in}}%
\pgfpathlineto{\pgfqpoint{3.221447in}{1.876848in}}%
\pgfpathlineto{\pgfqpoint{3.235084in}{1.872032in}}%
\pgfpathlineto{\pgfqpoint{3.248725in}{1.867306in}}%
\pgfpathlineto{\pgfqpoint{3.262371in}{1.862668in}}%
\pgfpathlineto{\pgfqpoint{3.270651in}{1.871029in}}%
\pgfpathlineto{\pgfqpoint{3.278925in}{1.879426in}}%
\pgfpathlineto{\pgfqpoint{3.287192in}{1.887855in}}%
\pgfpathlineto{\pgfqpoint{3.295452in}{1.896315in}}%
\pgfpathlineto{\pgfqpoint{3.281821in}{1.900810in}}%
\pgfpathlineto{\pgfqpoint{3.268195in}{1.905393in}}%
\pgfpathlineto{\pgfqpoint{3.254573in}{1.910065in}}%
\pgfpathlineto{\pgfqpoint{3.240956in}{1.914826in}}%
\pgfpathlineto{\pgfqpoint{3.232681in}{1.906501in}}%
\pgfpathlineto{\pgfqpoint{3.224399in}{1.898213in}}%
\pgfpathlineto{\pgfqpoint{3.216110in}{1.889962in}}%
\pgfpathlineto{\pgfqpoint{3.207815in}{1.881752in}}%
\pgfpathclose%
\pgfusepath{fill}%
\end{pgfscope}%
\begin{pgfscope}%
\pgfpathrectangle{\pgfqpoint{1.150000in}{0.150000in}}{\pgfqpoint{5.700000in}{5.700000in}}%
\pgfusepath{clip}%
\pgfsetbuttcap%
\pgfsetroundjoin%
\definecolor{currentfill}{rgb}{0.212395,0.359683,0.551710}%
\pgfsetfillcolor{currentfill}%
\pgfsetfillopacity{0.700000}%
\pgfsetlinewidth{0.000000pt}%
\definecolor{currentstroke}{rgb}{0.000000,0.000000,0.000000}%
\pgfsetstrokecolor{currentstroke}%
\pgfsetdash{}{0pt}%
\pgfpathmoveto{\pgfqpoint{5.773278in}{2.596067in}}%
\pgfpathlineto{\pgfqpoint{5.787670in}{2.598054in}}%
\pgfpathlineto{\pgfqpoint{5.802074in}{2.600108in}}%
\pgfpathlineto{\pgfqpoint{5.816490in}{2.602230in}}%
\pgfpathlineto{\pgfqpoint{5.830918in}{2.604418in}}%
\pgfpathlineto{\pgfqpoint{5.838129in}{2.608826in}}%
\pgfpathlineto{\pgfqpoint{5.845338in}{2.613335in}}%
\pgfpathlineto{\pgfqpoint{5.852543in}{2.617950in}}%
\pgfpathlineto{\pgfqpoint{5.859746in}{2.622679in}}%
\pgfpathlineto{\pgfqpoint{5.845346in}{2.620869in}}%
\pgfpathlineto{\pgfqpoint{5.830958in}{2.619125in}}%
\pgfpathlineto{\pgfqpoint{5.816582in}{2.617449in}}%
\pgfpathlineto{\pgfqpoint{5.802217in}{2.615839in}}%
\pgfpathlineto{\pgfqpoint{5.794986in}{2.610726in}}%
\pgfpathlineto{\pgfqpoint{5.787753in}{2.605730in}}%
\pgfpathlineto{\pgfqpoint{5.780517in}{2.600846in}}%
\pgfpathlineto{\pgfqpoint{5.773278in}{2.596067in}}%
\pgfpathclose%
\pgfusepath{fill}%
\end{pgfscope}%
\begin{pgfscope}%
\pgfpathrectangle{\pgfqpoint{1.150000in}{0.150000in}}{\pgfqpoint{5.700000in}{5.700000in}}%
\pgfusepath{clip}%
\pgfsetbuttcap%
\pgfsetroundjoin%
\definecolor{currentfill}{rgb}{0.257322,0.256130,0.526563}%
\pgfsetfillcolor{currentfill}%
\pgfsetfillopacity{0.700000}%
\pgfsetlinewidth{0.000000pt}%
\definecolor{currentstroke}{rgb}{0.000000,0.000000,0.000000}%
\pgfsetstrokecolor{currentstroke}%
\pgfsetdash{}{0pt}%
\pgfpathmoveto{\pgfqpoint{4.965601in}{2.348624in}}%
\pgfpathlineto{\pgfqpoint{4.979712in}{2.350143in}}%
\pgfpathlineto{\pgfqpoint{4.993833in}{2.351733in}}%
\pgfpathlineto{\pgfqpoint{5.007966in}{2.353392in}}%
\pgfpathlineto{\pgfqpoint{5.022109in}{2.355121in}}%
\pgfpathlineto{\pgfqpoint{5.029706in}{2.361365in}}%
\pgfpathlineto{\pgfqpoint{5.037297in}{2.367594in}}%
\pgfpathlineto{\pgfqpoint{5.044881in}{2.373811in}}%
\pgfpathlineto{\pgfqpoint{5.052460in}{2.380021in}}%
\pgfpathlineto{\pgfqpoint{5.038334in}{2.378502in}}%
\pgfpathlineto{\pgfqpoint{5.024218in}{2.377052in}}%
\pgfpathlineto{\pgfqpoint{5.010113in}{2.375673in}}%
\pgfpathlineto{\pgfqpoint{4.996018in}{2.374364in}}%
\pgfpathlineto{\pgfqpoint{4.988423in}{2.367936in}}%
\pgfpathlineto{\pgfqpoint{4.980822in}{2.361507in}}%
\pgfpathlineto{\pgfqpoint{4.973214in}{2.355071in}}%
\pgfpathlineto{\pgfqpoint{4.965601in}{2.348624in}}%
\pgfpathclose%
\pgfusepath{fill}%
\end{pgfscope}%
\begin{pgfscope}%
\pgfpathrectangle{\pgfqpoint{1.150000in}{0.150000in}}{\pgfqpoint{5.700000in}{5.700000in}}%
\pgfusepath{clip}%
\pgfsetbuttcap%
\pgfsetroundjoin%
\definecolor{currentfill}{rgb}{0.278012,0.180367,0.486697}%
\pgfsetfillcolor{currentfill}%
\pgfsetfillopacity{0.700000}%
\pgfsetlinewidth{0.000000pt}%
\definecolor{currentstroke}{rgb}{0.000000,0.000000,0.000000}%
\pgfsetstrokecolor{currentstroke}%
\pgfsetdash{}{0pt}%
\pgfpathmoveto{\pgfqpoint{4.474647in}{2.179535in}}%
\pgfpathlineto{\pgfqpoint{4.488588in}{2.180075in}}%
\pgfpathlineto{\pgfqpoint{4.502539in}{2.180689in}}%
\pgfpathlineto{\pgfqpoint{4.516498in}{2.181374in}}%
\pgfpathlineto{\pgfqpoint{4.530468in}{2.182133in}}%
\pgfpathlineto{\pgfqpoint{4.538278in}{2.189951in}}%
\pgfpathlineto{\pgfqpoint{4.546081in}{2.197726in}}%
\pgfpathlineto{\pgfqpoint{4.553879in}{2.205459in}}%
\pgfpathlineto{\pgfqpoint{4.561670in}{2.213154in}}%
\pgfpathlineto{\pgfqpoint{4.547713in}{2.212500in}}%
\pgfpathlineto{\pgfqpoint{4.533765in}{2.211919in}}%
\pgfpathlineto{\pgfqpoint{4.519827in}{2.211411in}}%
\pgfpathlineto{\pgfqpoint{4.505898in}{2.210975in}}%
\pgfpathlineto{\pgfqpoint{4.498095in}{2.203169in}}%
\pgfpathlineto{\pgfqpoint{4.490285in}{2.195328in}}%
\pgfpathlineto{\pgfqpoint{4.482469in}{2.187451in}}%
\pgfpathlineto{\pgfqpoint{4.474647in}{2.179535in}}%
\pgfpathclose%
\pgfusepath{fill}%
\end{pgfscope}%
\begin{pgfscope}%
\pgfpathrectangle{\pgfqpoint{1.150000in}{0.150000in}}{\pgfqpoint{5.700000in}{5.700000in}}%
\pgfusepath{clip}%
\pgfsetbuttcap%
\pgfsetroundjoin%
\definecolor{currentfill}{rgb}{0.277018,0.050344,0.375715}%
\pgfsetfillcolor{currentfill}%
\pgfsetfillopacity{0.700000}%
\pgfsetlinewidth{0.000000pt}%
\definecolor{currentstroke}{rgb}{0.000000,0.000000,0.000000}%
\pgfsetstrokecolor{currentstroke}%
\pgfsetdash{}{0pt}%
\pgfpathmoveto{\pgfqpoint{2.780258in}{1.946125in}}%
\pgfpathlineto{\pgfqpoint{2.793874in}{1.938376in}}%
\pgfpathlineto{\pgfqpoint{2.807491in}{1.930730in}}%
\pgfpathlineto{\pgfqpoint{2.821110in}{1.923185in}}%
\pgfpathlineto{\pgfqpoint{2.834731in}{1.915741in}}%
\pgfpathlineto{\pgfqpoint{2.843214in}{1.922229in}}%
\pgfpathlineto{\pgfqpoint{2.851687in}{1.928811in}}%
\pgfpathlineto{\pgfqpoint{2.860151in}{1.935485in}}%
\pgfpathlineto{\pgfqpoint{2.868606in}{1.942249in}}%
\pgfpathlineto{\pgfqpoint{2.855006in}{1.949487in}}%
\pgfpathlineto{\pgfqpoint{2.841408in}{1.956825in}}%
\pgfpathlineto{\pgfqpoint{2.827812in}{1.964265in}}%
\pgfpathlineto{\pgfqpoint{2.814218in}{1.971807in}}%
\pgfpathlineto{\pgfqpoint{2.805742in}{1.965242in}}%
\pgfpathlineto{\pgfqpoint{2.797257in}{1.958771in}}%
\pgfpathlineto{\pgfqpoint{2.788762in}{1.952398in}}%
\pgfpathlineto{\pgfqpoint{2.780258in}{1.946125in}}%
\pgfpathclose%
\pgfusepath{fill}%
\end{pgfscope}%
\begin{pgfscope}%
\pgfpathrectangle{\pgfqpoint{1.150000in}{0.150000in}}{\pgfqpoint{5.700000in}{5.700000in}}%
\pgfusepath{clip}%
\pgfsetbuttcap%
\pgfsetroundjoin%
\definecolor{currentfill}{rgb}{0.235526,0.309527,0.542944}%
\pgfsetfillcolor{currentfill}%
\pgfsetfillopacity{0.700000}%
\pgfsetlinewidth{0.000000pt}%
\definecolor{currentstroke}{rgb}{0.000000,0.000000,0.000000}%
\pgfsetstrokecolor{currentstroke}%
\pgfsetdash{}{0pt}%
\pgfpathmoveto{\pgfqpoint{5.369552in}{2.477142in}}%
\pgfpathlineto{\pgfqpoint{5.383808in}{2.479078in}}%
\pgfpathlineto{\pgfqpoint{5.398075in}{2.481084in}}%
\pgfpathlineto{\pgfqpoint{5.412354in}{2.483157in}}%
\pgfpathlineto{\pgfqpoint{5.426644in}{2.485299in}}%
\pgfpathlineto{\pgfqpoint{5.434049in}{2.490380in}}%
\pgfpathlineto{\pgfqpoint{5.441449in}{2.495492in}}%
\pgfpathlineto{\pgfqpoint{5.448843in}{2.500640in}}%
\pgfpathlineto{\pgfqpoint{5.456233in}{2.505830in}}%
\pgfpathlineto{\pgfqpoint{5.441964in}{2.503983in}}%
\pgfpathlineto{\pgfqpoint{5.427707in}{2.502204in}}%
\pgfpathlineto{\pgfqpoint{5.413462in}{2.500494in}}%
\pgfpathlineto{\pgfqpoint{5.399227in}{2.498851in}}%
\pgfpathlineto{\pgfqpoint{5.391816in}{2.493359in}}%
\pgfpathlineto{\pgfqpoint{5.384400in}{2.487914in}}%
\pgfpathlineto{\pgfqpoint{5.376978in}{2.482510in}}%
\pgfpathlineto{\pgfqpoint{5.369552in}{2.477142in}}%
\pgfpathclose%
\pgfusepath{fill}%
\end{pgfscope}%
\begin{pgfscope}%
\pgfpathrectangle{\pgfqpoint{1.150000in}{0.150000in}}{\pgfqpoint{5.700000in}{5.700000in}}%
\pgfusepath{clip}%
\pgfsetbuttcap%
\pgfsetroundjoin%
\definecolor{currentfill}{rgb}{0.282656,0.100196,0.422160}%
\pgfsetfillcolor{currentfill}%
\pgfsetfillopacity{0.700000}%
\pgfsetlinewidth{0.000000pt}%
\definecolor{currentstroke}{rgb}{0.000000,0.000000,0.000000}%
\pgfsetstrokecolor{currentstroke}%
\pgfsetdash{}{0pt}%
\pgfpathmoveto{\pgfqpoint{3.983674in}{2.014782in}}%
\pgfpathlineto{\pgfqpoint{3.997465in}{2.013740in}}%
\pgfpathlineto{\pgfqpoint{4.011263in}{2.012775in}}%
\pgfpathlineto{\pgfqpoint{4.025069in}{2.011887in}}%
\pgfpathlineto{\pgfqpoint{4.038882in}{2.011075in}}%
\pgfpathlineto{\pgfqpoint{4.046877in}{2.020045in}}%
\pgfpathlineto{\pgfqpoint{4.054866in}{2.028977in}}%
\pgfpathlineto{\pgfqpoint{4.062850in}{2.037871in}}%
\pgfpathlineto{\pgfqpoint{4.070828in}{2.046729in}}%
\pgfpathlineto{\pgfqpoint{4.057024in}{2.047542in}}%
\pgfpathlineto{\pgfqpoint{4.043229in}{2.048431in}}%
\pgfpathlineto{\pgfqpoint{4.029441in}{2.049396in}}%
\pgfpathlineto{\pgfqpoint{4.015662in}{2.050439in}}%
\pgfpathlineto{\pgfqpoint{4.007673in}{2.041572in}}%
\pgfpathlineto{\pgfqpoint{3.999679in}{2.032675in}}%
\pgfpathlineto{\pgfqpoint{3.991680in}{2.023745in}}%
\pgfpathlineto{\pgfqpoint{3.983674in}{2.014782in}}%
\pgfpathclose%
\pgfusepath{fill}%
\end{pgfscope}%
\begin{pgfscope}%
\pgfpathrectangle{\pgfqpoint{1.150000in}{0.150000in}}{\pgfqpoint{5.700000in}{5.700000in}}%
\pgfusepath{clip}%
\pgfsetbuttcap%
\pgfsetroundjoin%
\definecolor{currentfill}{rgb}{0.282623,0.140926,0.457517}%
\pgfsetfillcolor{currentfill}%
\pgfsetfillopacity{0.700000}%
\pgfsetlinewidth{0.000000pt}%
\definecolor{currentstroke}{rgb}{0.000000,0.000000,0.000000}%
\pgfsetstrokecolor{currentstroke}%
\pgfsetdash{}{0pt}%
\pgfpathmoveto{\pgfqpoint{2.384118in}{2.129976in}}%
\pgfpathlineto{\pgfqpoint{2.397779in}{2.119062in}}%
\pgfpathlineto{\pgfqpoint{2.411439in}{2.108270in}}%
\pgfpathlineto{\pgfqpoint{2.425098in}{2.097599in}}%
\pgfpathlineto{\pgfqpoint{2.438755in}{2.087048in}}%
\pgfpathlineto{\pgfqpoint{2.447473in}{2.091115in}}%
\pgfpathlineto{\pgfqpoint{2.456178in}{2.095336in}}%
\pgfpathlineto{\pgfqpoint{2.464871in}{2.099705in}}%
\pgfpathlineto{\pgfqpoint{2.473551in}{2.104221in}}%
\pgfpathlineto{\pgfqpoint{2.459921in}{2.114520in}}%
\pgfpathlineto{\pgfqpoint{2.446290in}{2.124939in}}%
\pgfpathlineto{\pgfqpoint{2.432659in}{2.135479in}}%
\pgfpathlineto{\pgfqpoint{2.419026in}{2.146141in}}%
\pgfpathlineto{\pgfqpoint{2.410318in}{2.141869in}}%
\pgfpathlineto{\pgfqpoint{2.401598in}{2.137749in}}%
\pgfpathlineto{\pgfqpoint{2.392864in}{2.133783in}}%
\pgfpathlineto{\pgfqpoint{2.384118in}{2.129976in}}%
\pgfpathclose%
\pgfusepath{fill}%
\end{pgfscope}%
\begin{pgfscope}%
\pgfpathrectangle{\pgfqpoint{1.150000in}{0.150000in}}{\pgfqpoint{5.700000in}{5.700000in}}%
\pgfusepath{clip}%
\pgfsetbuttcap%
\pgfsetroundjoin%
\definecolor{currentfill}{rgb}{0.280255,0.165693,0.476498}%
\pgfsetfillcolor{currentfill}%
\pgfsetfillopacity{0.700000}%
\pgfsetlinewidth{0.000000pt}%
\definecolor{currentstroke}{rgb}{0.000000,0.000000,0.000000}%
\pgfsetstrokecolor{currentstroke}%
\pgfsetdash{}{0pt}%
\pgfpathmoveto{\pgfqpoint{4.387580in}{2.145683in}}%
\pgfpathlineto{\pgfqpoint{4.401496in}{2.146015in}}%
\pgfpathlineto{\pgfqpoint{4.415421in}{2.146420in}}%
\pgfpathlineto{\pgfqpoint{4.429355in}{2.146898in}}%
\pgfpathlineto{\pgfqpoint{4.443298in}{2.147450in}}%
\pgfpathlineto{\pgfqpoint{4.451145in}{2.155538in}}%
\pgfpathlineto{\pgfqpoint{4.458985in}{2.163581in}}%
\pgfpathlineto{\pgfqpoint{4.466819in}{2.171579in}}%
\pgfpathlineto{\pgfqpoint{4.474647in}{2.179535in}}%
\pgfpathlineto{\pgfqpoint{4.460715in}{2.179068in}}%
\pgfpathlineto{\pgfqpoint{4.446793in}{2.178673in}}%
\pgfpathlineto{\pgfqpoint{4.432880in}{2.178352in}}%
\pgfpathlineto{\pgfqpoint{4.418975in}{2.178104in}}%
\pgfpathlineto{\pgfqpoint{4.411136in}{2.170057in}}%
\pgfpathlineto{\pgfqpoint{4.403290in}{2.161972in}}%
\pgfpathlineto{\pgfqpoint{4.395438in}{2.153848in}}%
\pgfpathlineto{\pgfqpoint{4.387580in}{2.145683in}}%
\pgfpathclose%
\pgfusepath{fill}%
\end{pgfscope}%
\begin{pgfscope}%
\pgfpathrectangle{\pgfqpoint{1.150000in}{0.150000in}}{\pgfqpoint{5.700000in}{5.700000in}}%
\pgfusepath{clip}%
\pgfsetbuttcap%
\pgfsetroundjoin%
\definecolor{currentfill}{rgb}{0.262138,0.242286,0.520837}%
\pgfsetfillcolor{currentfill}%
\pgfsetfillopacity{0.700000}%
\pgfsetlinewidth{0.000000pt}%
\definecolor{currentstroke}{rgb}{0.000000,0.000000,0.000000}%
\pgfsetstrokecolor{currentstroke}%
\pgfsetdash{}{0pt}%
\pgfpathmoveto{\pgfqpoint{4.878681in}{2.316532in}}%
\pgfpathlineto{\pgfqpoint{4.892766in}{2.317959in}}%
\pgfpathlineto{\pgfqpoint{4.906861in}{2.319457in}}%
\pgfpathlineto{\pgfqpoint{4.920967in}{2.321025in}}%
\pgfpathlineto{\pgfqpoint{4.935083in}{2.322664in}}%
\pgfpathlineto{\pgfqpoint{4.942722in}{2.329187in}}%
\pgfpathlineto{\pgfqpoint{4.950355in}{2.335686in}}%
\pgfpathlineto{\pgfqpoint{4.957981in}{2.342164in}}%
\pgfpathlineto{\pgfqpoint{4.965601in}{2.348624in}}%
\pgfpathlineto{\pgfqpoint{4.951500in}{2.347175in}}%
\pgfpathlineto{\pgfqpoint{4.937410in}{2.345796in}}%
\pgfpathlineto{\pgfqpoint{4.923330in}{2.344488in}}%
\pgfpathlineto{\pgfqpoint{4.909261in}{2.343250in}}%
\pgfpathlineto{\pgfqpoint{4.901625in}{2.336593in}}%
\pgfpathlineto{\pgfqpoint{4.893983in}{2.329923in}}%
\pgfpathlineto{\pgfqpoint{4.886335in}{2.323238in}}%
\pgfpathlineto{\pgfqpoint{4.878681in}{2.316532in}}%
\pgfpathclose%
\pgfusepath{fill}%
\end{pgfscope}%
\begin{pgfscope}%
\pgfpathrectangle{\pgfqpoint{1.150000in}{0.150000in}}{\pgfqpoint{5.700000in}{5.700000in}}%
\pgfusepath{clip}%
\pgfsetbuttcap%
\pgfsetroundjoin%
\definecolor{currentfill}{rgb}{0.271305,0.019942,0.347269}%
\pgfsetfillcolor{currentfill}%
\pgfsetfillopacity{0.700000}%
\pgfsetlinewidth{0.000000pt}%
\definecolor{currentstroke}{rgb}{0.000000,0.000000,0.000000}%
\pgfsetstrokecolor{currentstroke}%
\pgfsetdash{}{0pt}%
\pgfpathmoveto{\pgfqpoint{3.350022in}{1.879208in}}%
\pgfpathlineto{\pgfqpoint{3.363677in}{1.875148in}}%
\pgfpathlineto{\pgfqpoint{3.377337in}{1.871173in}}%
\pgfpathlineto{\pgfqpoint{3.391002in}{1.867283in}}%
\pgfpathlineto{\pgfqpoint{3.404672in}{1.863479in}}%
\pgfpathlineto{\pgfqpoint{3.412898in}{1.872230in}}%
\pgfpathlineto{\pgfqpoint{3.421117in}{1.880998in}}%
\pgfpathlineto{\pgfqpoint{3.429331in}{1.889782in}}%
\pgfpathlineto{\pgfqpoint{3.437538in}{1.898580in}}%
\pgfpathlineto{\pgfqpoint{3.423881in}{1.902262in}}%
\pgfpathlineto{\pgfqpoint{3.410229in}{1.906028in}}%
\pgfpathlineto{\pgfqpoint{3.396583in}{1.909880in}}%
\pgfpathlineto{\pgfqpoint{3.382942in}{1.913818in}}%
\pgfpathlineto{\pgfqpoint{3.374722in}{1.905135in}}%
\pgfpathlineto{\pgfqpoint{3.366495in}{1.896471in}}%
\pgfpathlineto{\pgfqpoint{3.358262in}{1.887828in}}%
\pgfpathlineto{\pgfqpoint{3.350022in}{1.879208in}}%
\pgfpathclose%
\pgfusepath{fill}%
\end{pgfscope}%
\begin{pgfscope}%
\pgfpathrectangle{\pgfqpoint{1.150000in}{0.150000in}}{\pgfqpoint{5.700000in}{5.700000in}}%
\pgfusepath{clip}%
\pgfsetbuttcap%
\pgfsetroundjoin%
\definecolor{currentfill}{rgb}{0.274952,0.037752,0.364543}%
\pgfsetfillcolor{currentfill}%
\pgfsetfillopacity{0.700000}%
\pgfsetlinewidth{0.000000pt}%
\definecolor{currentstroke}{rgb}{0.000000,0.000000,0.000000}%
\pgfsetstrokecolor{currentstroke}%
\pgfsetdash{}{0pt}%
\pgfpathmoveto{\pgfqpoint{3.579658in}{1.908276in}}%
\pgfpathlineto{\pgfqpoint{3.593354in}{1.905447in}}%
\pgfpathlineto{\pgfqpoint{3.607056in}{1.902699in}}%
\pgfpathlineto{\pgfqpoint{3.620764in}{1.900033in}}%
\pgfpathlineto{\pgfqpoint{3.634478in}{1.897447in}}%
\pgfpathlineto{\pgfqpoint{3.642618in}{1.906567in}}%
\pgfpathlineto{\pgfqpoint{3.650752in}{1.915679in}}%
\pgfpathlineto{\pgfqpoint{3.658881in}{1.924782in}}%
\pgfpathlineto{\pgfqpoint{3.667003in}{1.933876in}}%
\pgfpathlineto{\pgfqpoint{3.653300in}{1.936379in}}%
\pgfpathlineto{\pgfqpoint{3.639604in}{1.938964in}}%
\pgfpathlineto{\pgfqpoint{3.625914in}{1.941630in}}%
\pgfpathlineto{\pgfqpoint{3.612230in}{1.944378in}}%
\pgfpathlineto{\pgfqpoint{3.604096in}{1.935358in}}%
\pgfpathlineto{\pgfqpoint{3.595956in}{1.926335in}}%
\pgfpathlineto{\pgfqpoint{3.587810in}{1.917307in}}%
\pgfpathlineto{\pgfqpoint{3.579658in}{1.908276in}}%
\pgfpathclose%
\pgfusepath{fill}%
\end{pgfscope}%
\begin{pgfscope}%
\pgfpathrectangle{\pgfqpoint{1.150000in}{0.150000in}}{\pgfqpoint{5.700000in}{5.700000in}}%
\pgfusepath{clip}%
\pgfsetbuttcap%
\pgfsetroundjoin%
\definecolor{currentfill}{rgb}{0.216210,0.351535,0.550627}%
\pgfsetfillcolor{currentfill}%
\pgfsetfillopacity{0.700000}%
\pgfsetlinewidth{0.000000pt}%
\definecolor{currentstroke}{rgb}{0.000000,0.000000,0.000000}%
\pgfsetstrokecolor{currentstroke}%
\pgfsetdash{}{0pt}%
\pgfpathmoveto{\pgfqpoint{5.686737in}{2.569165in}}%
\pgfpathlineto{\pgfqpoint{5.701108in}{2.571241in}}%
\pgfpathlineto{\pgfqpoint{5.715491in}{2.573384in}}%
\pgfpathlineto{\pgfqpoint{5.729885in}{2.575595in}}%
\pgfpathlineto{\pgfqpoint{5.744292in}{2.577873in}}%
\pgfpathlineto{\pgfqpoint{5.751544in}{2.582296in}}%
\pgfpathlineto{\pgfqpoint{5.758792in}{2.586798in}}%
\pgfpathlineto{\pgfqpoint{5.766037in}{2.591387in}}%
\pgfpathlineto{\pgfqpoint{5.773278in}{2.596067in}}%
\pgfpathlineto{\pgfqpoint{5.758898in}{2.594147in}}%
\pgfpathlineto{\pgfqpoint{5.744530in}{2.592294in}}%
\pgfpathlineto{\pgfqpoint{5.730173in}{2.590508in}}%
\pgfpathlineto{\pgfqpoint{5.715828in}{2.588789in}}%
\pgfpathlineto{\pgfqpoint{5.708560in}{2.583744in}}%
\pgfpathlineto{\pgfqpoint{5.701289in}{2.578796in}}%
\pgfpathlineto{\pgfqpoint{5.694015in}{2.573939in}}%
\pgfpathlineto{\pgfqpoint{5.686737in}{2.569165in}}%
\pgfpathclose%
\pgfusepath{fill}%
\end{pgfscope}%
\begin{pgfscope}%
\pgfpathrectangle{\pgfqpoint{1.150000in}{0.150000in}}{\pgfqpoint{5.700000in}{5.700000in}}%
\pgfusepath{clip}%
\pgfsetbuttcap%
\pgfsetroundjoin%
\definecolor{currentfill}{rgb}{0.281446,0.084320,0.407414}%
\pgfsetfillcolor{currentfill}%
\pgfsetfillopacity{0.700000}%
\pgfsetlinewidth{0.000000pt}%
\definecolor{currentstroke}{rgb}{0.000000,0.000000,0.000000}%
\pgfsetstrokecolor{currentstroke}%
\pgfsetdash{}{0pt}%
\pgfpathmoveto{\pgfqpoint{3.896469in}{1.983619in}}%
\pgfpathlineto{\pgfqpoint{3.910239in}{1.982248in}}%
\pgfpathlineto{\pgfqpoint{3.924017in}{1.980955in}}%
\pgfpathlineto{\pgfqpoint{3.937803in}{1.979739in}}%
\pgfpathlineto{\pgfqpoint{3.951596in}{1.978600in}}%
\pgfpathlineto{\pgfqpoint{3.959624in}{1.987696in}}%
\pgfpathlineto{\pgfqpoint{3.967647in}{1.996758in}}%
\pgfpathlineto{\pgfqpoint{3.975663in}{2.005787in}}%
\pgfpathlineto{\pgfqpoint{3.983674in}{2.014782in}}%
\pgfpathlineto{\pgfqpoint{3.969892in}{2.015901in}}%
\pgfpathlineto{\pgfqpoint{3.956117in}{2.017097in}}%
\pgfpathlineto{\pgfqpoint{3.942349in}{2.018370in}}%
\pgfpathlineto{\pgfqpoint{3.928589in}{2.019721in}}%
\pgfpathlineto{\pgfqpoint{3.920568in}{2.010738in}}%
\pgfpathlineto{\pgfqpoint{3.912540in}{2.001726in}}%
\pgfpathlineto{\pgfqpoint{3.904507in}{1.992687in}}%
\pgfpathlineto{\pgfqpoint{3.896469in}{1.983619in}}%
\pgfpathclose%
\pgfusepath{fill}%
\end{pgfscope}%
\begin{pgfscope}%
\pgfpathrectangle{\pgfqpoint{1.150000in}{0.150000in}}{\pgfqpoint{5.700000in}{5.700000in}}%
\pgfusepath{clip}%
\pgfsetbuttcap%
\pgfsetroundjoin%
\definecolor{currentfill}{rgb}{0.239346,0.300855,0.540844}%
\pgfsetfillcolor{currentfill}%
\pgfsetfillopacity{0.700000}%
\pgfsetlinewidth{0.000000pt}%
\definecolor{currentstroke}{rgb}{0.000000,0.000000,0.000000}%
\pgfsetstrokecolor{currentstroke}%
\pgfsetdash{}{0pt}%
\pgfpathmoveto{\pgfqpoint{5.282797in}{2.447776in}}%
\pgfpathlineto{\pgfqpoint{5.297028in}{2.449712in}}%
\pgfpathlineto{\pgfqpoint{5.311271in}{2.451717in}}%
\pgfpathlineto{\pgfqpoint{5.325525in}{2.453790in}}%
\pgfpathlineto{\pgfqpoint{5.339790in}{2.455933in}}%
\pgfpathlineto{\pgfqpoint{5.347239in}{2.461206in}}%
\pgfpathlineto{\pgfqpoint{5.354682in}{2.466495in}}%
\pgfpathlineto{\pgfqpoint{5.362120in}{2.471805in}}%
\pgfpathlineto{\pgfqpoint{5.369552in}{2.477142in}}%
\pgfpathlineto{\pgfqpoint{5.355307in}{2.475274in}}%
\pgfpathlineto{\pgfqpoint{5.341074in}{2.473474in}}%
\pgfpathlineto{\pgfqpoint{5.326851in}{2.471743in}}%
\pgfpathlineto{\pgfqpoint{5.312640in}{2.470080in}}%
\pgfpathlineto{\pgfqpoint{5.305188in}{2.464463in}}%
\pgfpathlineto{\pgfqpoint{5.297729in}{2.458876in}}%
\pgfpathlineto{\pgfqpoint{5.290266in}{2.453316in}}%
\pgfpathlineto{\pgfqpoint{5.282797in}{2.447776in}}%
\pgfpathclose%
\pgfusepath{fill}%
\end{pgfscope}%
\begin{pgfscope}%
\pgfpathrectangle{\pgfqpoint{1.150000in}{0.150000in}}{\pgfqpoint{5.700000in}{5.700000in}}%
\pgfusepath{clip}%
\pgfsetbuttcap%
\pgfsetroundjoin%
\definecolor{currentfill}{rgb}{0.280894,0.078907,0.402329}%
\pgfsetfillcolor{currentfill}%
\pgfsetfillopacity{0.700000}%
\pgfsetlinewidth{0.000000pt}%
\definecolor{currentstroke}{rgb}{0.000000,0.000000,0.000000}%
\pgfsetstrokecolor{currentstroke}%
\pgfsetdash{}{0pt}%
\pgfpathmoveto{\pgfqpoint{2.637084in}{1.989668in}}%
\pgfpathlineto{\pgfqpoint{2.650713in}{1.980849in}}%
\pgfpathlineto{\pgfqpoint{2.664343in}{1.972137in}}%
\pgfpathlineto{\pgfqpoint{2.677974in}{1.963533in}}%
\pgfpathlineto{\pgfqpoint{2.691605in}{1.955035in}}%
\pgfpathlineto{\pgfqpoint{2.700172in}{1.960641in}}%
\pgfpathlineto{\pgfqpoint{2.708729in}{1.966365in}}%
\pgfpathlineto{\pgfqpoint{2.717275in}{1.972204in}}%
\pgfpathlineto{\pgfqpoint{2.725812in}{1.978154in}}%
\pgfpathlineto{\pgfqpoint{2.712203in}{1.986424in}}%
\pgfpathlineto{\pgfqpoint{2.698596in}{1.994800in}}%
\pgfpathlineto{\pgfqpoint{2.684990in}{2.003283in}}%
\pgfpathlineto{\pgfqpoint{2.671385in}{2.011874in}}%
\pgfpathlineto{\pgfqpoint{2.662825in}{2.006144in}}%
\pgfpathlineto{\pgfqpoint{2.654255in}{2.000531in}}%
\pgfpathlineto{\pgfqpoint{2.645675in}{1.995038in}}%
\pgfpathlineto{\pgfqpoint{2.637084in}{1.989668in}}%
\pgfpathclose%
\pgfusepath{fill}%
\end{pgfscope}%
\begin{pgfscope}%
\pgfpathrectangle{\pgfqpoint{1.150000in}{0.150000in}}{\pgfqpoint{5.700000in}{5.700000in}}%
\pgfusepath{clip}%
\pgfsetbuttcap%
\pgfsetroundjoin%
\definecolor{currentfill}{rgb}{0.281887,0.150881,0.465405}%
\pgfsetfillcolor{currentfill}%
\pgfsetfillopacity{0.700000}%
\pgfsetlinewidth{0.000000pt}%
\definecolor{currentstroke}{rgb}{0.000000,0.000000,0.000000}%
\pgfsetstrokecolor{currentstroke}%
\pgfsetdash{}{0pt}%
\pgfpathmoveto{\pgfqpoint{4.300469in}{2.111737in}}%
\pgfpathlineto{\pgfqpoint{4.314361in}{2.111837in}}%
\pgfpathlineto{\pgfqpoint{4.328261in}{2.112011in}}%
\pgfpathlineto{\pgfqpoint{4.342170in}{2.112259in}}%
\pgfpathlineto{\pgfqpoint{4.356088in}{2.112580in}}%
\pgfpathlineto{\pgfqpoint{4.363970in}{2.120925in}}%
\pgfpathlineto{\pgfqpoint{4.371846in}{2.129223in}}%
\pgfpathlineto{\pgfqpoint{4.379716in}{2.137475in}}%
\pgfpathlineto{\pgfqpoint{4.387580in}{2.145683in}}%
\pgfpathlineto{\pgfqpoint{4.373673in}{2.145425in}}%
\pgfpathlineto{\pgfqpoint{4.359775in}{2.145240in}}%
\pgfpathlineto{\pgfqpoint{4.345886in}{2.145129in}}%
\pgfpathlineto{\pgfqpoint{4.332006in}{2.145092in}}%
\pgfpathlineto{\pgfqpoint{4.324131in}{2.136814in}}%
\pgfpathlineto{\pgfqpoint{4.316250in}{2.128496in}}%
\pgfpathlineto{\pgfqpoint{4.308363in}{2.120138in}}%
\pgfpathlineto{\pgfqpoint{4.300469in}{2.111737in}}%
\pgfpathclose%
\pgfusepath{fill}%
\end{pgfscope}%
\begin{pgfscope}%
\pgfpathrectangle{\pgfqpoint{1.150000in}{0.150000in}}{\pgfqpoint{5.700000in}{5.700000in}}%
\pgfusepath{clip}%
\pgfsetbuttcap%
\pgfsetroundjoin%
\definecolor{currentfill}{rgb}{0.265145,0.232956,0.516599}%
\pgfsetfillcolor{currentfill}%
\pgfsetfillopacity{0.700000}%
\pgfsetlinewidth{0.000000pt}%
\definecolor{currentstroke}{rgb}{0.000000,0.000000,0.000000}%
\pgfsetstrokecolor{currentstroke}%
\pgfsetdash{}{0pt}%
\pgfpathmoveto{\pgfqpoint{4.791704in}{2.283775in}}%
\pgfpathlineto{\pgfqpoint{4.805762in}{2.285087in}}%
\pgfpathlineto{\pgfqpoint{4.819831in}{2.286470in}}%
\pgfpathlineto{\pgfqpoint{4.833910in}{2.287924in}}%
\pgfpathlineto{\pgfqpoint{4.847999in}{2.289449in}}%
\pgfpathlineto{\pgfqpoint{4.855679in}{2.296266in}}%
\pgfpathlineto{\pgfqpoint{4.863353in}{2.303050in}}%
\pgfpathlineto{\pgfqpoint{4.871020in}{2.309804in}}%
\pgfpathlineto{\pgfqpoint{4.878681in}{2.316532in}}%
\pgfpathlineto{\pgfqpoint{4.864606in}{2.315176in}}%
\pgfpathlineto{\pgfqpoint{4.850542in}{2.313890in}}%
\pgfpathlineto{\pgfqpoint{4.836488in}{2.312675in}}%
\pgfpathlineto{\pgfqpoint{4.822444in}{2.311531in}}%
\pgfpathlineto{\pgfqpoint{4.814769in}{2.304627in}}%
\pgfpathlineto{\pgfqpoint{4.807087in}{2.297702in}}%
\pgfpathlineto{\pgfqpoint{4.799399in}{2.290752in}}%
\pgfpathlineto{\pgfqpoint{4.791704in}{2.283775in}}%
\pgfpathclose%
\pgfusepath{fill}%
\end{pgfscope}%
\begin{pgfscope}%
\pgfpathrectangle{\pgfqpoint{1.150000in}{0.150000in}}{\pgfqpoint{5.700000in}{5.700000in}}%
\pgfusepath{clip}%
\pgfsetbuttcap%
\pgfsetroundjoin%
\definecolor{currentfill}{rgb}{0.272594,0.025563,0.353093}%
\pgfsetfillcolor{currentfill}%
\pgfsetfillopacity{0.700000}%
\pgfsetlinewidth{0.000000pt}%
\definecolor{currentstroke}{rgb}{0.000000,0.000000,0.000000}%
\pgfsetstrokecolor{currentstroke}%
\pgfsetdash{}{0pt}%
\pgfpathmoveto{\pgfqpoint{2.977495in}{1.887902in}}%
\pgfpathlineto{\pgfqpoint{2.991118in}{1.881544in}}%
\pgfpathlineto{\pgfqpoint{3.004744in}{1.875281in}}%
\pgfpathlineto{\pgfqpoint{3.018373in}{1.869113in}}%
\pgfpathlineto{\pgfqpoint{3.032006in}{1.863038in}}%
\pgfpathlineto{\pgfqpoint{3.040395in}{1.870467in}}%
\pgfpathlineto{\pgfqpoint{3.048776in}{1.877964in}}%
\pgfpathlineto{\pgfqpoint{3.057149in}{1.885526in}}%
\pgfpathlineto{\pgfqpoint{3.065514in}{1.893153in}}%
\pgfpathlineto{\pgfqpoint{3.051900in}{1.899043in}}%
\pgfpathlineto{\pgfqpoint{3.038289in}{1.905026in}}%
\pgfpathlineto{\pgfqpoint{3.024681in}{1.911104in}}%
\pgfpathlineto{\pgfqpoint{3.011076in}{1.917277in}}%
\pgfpathlineto{\pgfqpoint{3.002693in}{1.909828in}}%
\pgfpathlineto{\pgfqpoint{2.994301in}{1.902448in}}%
\pgfpathlineto{\pgfqpoint{2.985902in}{1.895138in}}%
\pgfpathlineto{\pgfqpoint{2.977495in}{1.887902in}}%
\pgfpathclose%
\pgfusepath{fill}%
\end{pgfscope}%
\begin{pgfscope}%
\pgfpathrectangle{\pgfqpoint{1.150000in}{0.150000in}}{\pgfqpoint{5.700000in}{5.700000in}}%
\pgfusepath{clip}%
\pgfsetbuttcap%
\pgfsetroundjoin%
\definecolor{currentfill}{rgb}{0.269944,0.014625,0.341379}%
\pgfsetfillcolor{currentfill}%
\pgfsetfillopacity{0.700000}%
\pgfsetlinewidth{0.000000pt}%
\definecolor{currentstroke}{rgb}{0.000000,0.000000,0.000000}%
\pgfsetstrokecolor{currentstroke}%
\pgfsetdash{}{0pt}%
\pgfpathmoveto{\pgfqpoint{3.120008in}{1.870520in}}%
\pgfpathlineto{\pgfqpoint{3.133640in}{1.865092in}}%
\pgfpathlineto{\pgfqpoint{3.147277in}{1.859754in}}%
\pgfpathlineto{\pgfqpoint{3.160917in}{1.854507in}}%
\pgfpathlineto{\pgfqpoint{3.174561in}{1.849349in}}%
\pgfpathlineto{\pgfqpoint{3.182885in}{1.857380in}}%
\pgfpathlineto{\pgfqpoint{3.191202in}{1.865458in}}%
\pgfpathlineto{\pgfqpoint{3.199512in}{1.873583in}}%
\pgfpathlineto{\pgfqpoint{3.207815in}{1.881752in}}%
\pgfpathlineto{\pgfqpoint{3.194186in}{1.886745in}}%
\pgfpathlineto{\pgfqpoint{3.180562in}{1.891829in}}%
\pgfpathlineto{\pgfqpoint{3.166942in}{1.897002in}}%
\pgfpathlineto{\pgfqpoint{3.153326in}{1.902267in}}%
\pgfpathlineto{\pgfqpoint{3.145007in}{1.894255in}}%
\pgfpathlineto{\pgfqpoint{3.136682in}{1.886291in}}%
\pgfpathlineto{\pgfqpoint{3.128348in}{1.878379in}}%
\pgfpathlineto{\pgfqpoint{3.120008in}{1.870520in}}%
\pgfpathclose%
\pgfusepath{fill}%
\end{pgfscope}%
\begin{pgfscope}%
\pgfpathrectangle{\pgfqpoint{1.150000in}{0.150000in}}{\pgfqpoint{5.700000in}{5.700000in}}%
\pgfusepath{clip}%
\pgfsetbuttcap%
\pgfsetroundjoin%
\definecolor{currentfill}{rgb}{0.283187,0.125848,0.444960}%
\pgfsetfillcolor{currentfill}%
\pgfsetfillopacity{0.700000}%
\pgfsetlinewidth{0.000000pt}%
\definecolor{currentstroke}{rgb}{0.000000,0.000000,0.000000}%
\pgfsetstrokecolor{currentstroke}%
\pgfsetdash{}{0pt}%
\pgfpathmoveto{\pgfqpoint{2.438755in}{2.087048in}}%
\pgfpathlineto{\pgfqpoint{2.452412in}{2.076617in}}%
\pgfpathlineto{\pgfqpoint{2.466068in}{2.066304in}}%
\pgfpathlineto{\pgfqpoint{2.479723in}{2.056109in}}%
\pgfpathlineto{\pgfqpoint{2.493378in}{2.046030in}}%
\pgfpathlineto{\pgfqpoint{2.502067in}{2.050356in}}%
\pgfpathlineto{\pgfqpoint{2.510745in}{2.054830in}}%
\pgfpathlineto{\pgfqpoint{2.519410in}{2.059449in}}%
\pgfpathlineto{\pgfqpoint{2.528064in}{2.064208in}}%
\pgfpathlineto{\pgfqpoint{2.514436in}{2.074036in}}%
\pgfpathlineto{\pgfqpoint{2.500808in}{2.083980in}}%
\pgfpathlineto{\pgfqpoint{2.487180in}{2.094042in}}%
\pgfpathlineto{\pgfqpoint{2.473551in}{2.104221in}}%
\pgfpathlineto{\pgfqpoint{2.464871in}{2.099705in}}%
\pgfpathlineto{\pgfqpoint{2.456178in}{2.095336in}}%
\pgfpathlineto{\pgfqpoint{2.447473in}{2.091115in}}%
\pgfpathlineto{\pgfqpoint{2.438755in}{2.087048in}}%
\pgfpathclose%
\pgfusepath{fill}%
\end{pgfscope}%
\begin{pgfscope}%
\pgfpathrectangle{\pgfqpoint{1.150000in}{0.150000in}}{\pgfqpoint{5.700000in}{5.700000in}}%
\pgfusepath{clip}%
\pgfsetbuttcap%
\pgfsetroundjoin%
\definecolor{currentfill}{rgb}{0.280267,0.073417,0.397163}%
\pgfsetfillcolor{currentfill}%
\pgfsetfillopacity{0.700000}%
\pgfsetlinewidth{0.000000pt}%
\definecolor{currentstroke}{rgb}{0.000000,0.000000,0.000000}%
\pgfsetstrokecolor{currentstroke}%
\pgfsetdash{}{0pt}%
\pgfpathmoveto{\pgfqpoint{3.809206in}{1.953490in}}%
\pgfpathlineto{\pgfqpoint{3.822958in}{1.951765in}}%
\pgfpathlineto{\pgfqpoint{3.836718in}{1.950120in}}%
\pgfpathlineto{\pgfqpoint{3.850484in}{1.948552in}}%
\pgfpathlineto{\pgfqpoint{3.864258in}{1.947063in}}%
\pgfpathlineto{\pgfqpoint{3.872319in}{1.956244in}}%
\pgfpathlineto{\pgfqpoint{3.880375in}{1.965397in}}%
\pgfpathlineto{\pgfqpoint{3.888425in}{1.974522in}}%
\pgfpathlineto{\pgfqpoint{3.896469in}{1.983619in}}%
\pgfpathlineto{\pgfqpoint{3.882706in}{1.985068in}}%
\pgfpathlineto{\pgfqpoint{3.868950in}{1.986594in}}%
\pgfpathlineto{\pgfqpoint{3.855201in}{1.988199in}}%
\pgfpathlineto{\pgfqpoint{3.841460in}{1.989883in}}%
\pgfpathlineto{\pgfqpoint{3.833405in}{1.980819in}}%
\pgfpathlineto{\pgfqpoint{3.825344in}{1.971733in}}%
\pgfpathlineto{\pgfqpoint{3.817278in}{1.962623in}}%
\pgfpathlineto{\pgfqpoint{3.809206in}{1.953490in}}%
\pgfpathclose%
\pgfusepath{fill}%
\end{pgfscope}%
\begin{pgfscope}%
\pgfpathrectangle{\pgfqpoint{1.150000in}{0.150000in}}{\pgfqpoint{5.700000in}{5.700000in}}%
\pgfusepath{clip}%
\pgfsetbuttcap%
\pgfsetroundjoin%
\definecolor{currentfill}{rgb}{0.274952,0.037752,0.364543}%
\pgfsetfillcolor{currentfill}%
\pgfsetfillopacity{0.700000}%
\pgfsetlinewidth{0.000000pt}%
\definecolor{currentstroke}{rgb}{0.000000,0.000000,0.000000}%
\pgfsetstrokecolor{currentstroke}%
\pgfsetdash{}{0pt}%
\pgfpathmoveto{\pgfqpoint{2.834731in}{1.915741in}}%
\pgfpathlineto{\pgfqpoint{2.848354in}{1.908397in}}%
\pgfpathlineto{\pgfqpoint{2.861979in}{1.901153in}}%
\pgfpathlineto{\pgfqpoint{2.875607in}{1.894008in}}%
\pgfpathlineto{\pgfqpoint{2.889236in}{1.886962in}}%
\pgfpathlineto{\pgfqpoint{2.897698in}{1.893663in}}%
\pgfpathlineto{\pgfqpoint{2.906151in}{1.900454in}}%
\pgfpathlineto{\pgfqpoint{2.914595in}{1.907332in}}%
\pgfpathlineto{\pgfqpoint{2.923030in}{1.914294in}}%
\pgfpathlineto{\pgfqpoint{2.909421in}{1.921135in}}%
\pgfpathlineto{\pgfqpoint{2.895813in}{1.928074in}}%
\pgfpathlineto{\pgfqpoint{2.882209in}{1.935112in}}%
\pgfpathlineto{\pgfqpoint{2.868606in}{1.942249in}}%
\pgfpathlineto{\pgfqpoint{2.860151in}{1.935485in}}%
\pgfpathlineto{\pgfqpoint{2.851687in}{1.928811in}}%
\pgfpathlineto{\pgfqpoint{2.843214in}{1.922229in}}%
\pgfpathlineto{\pgfqpoint{2.834731in}{1.915741in}}%
\pgfpathclose%
\pgfusepath{fill}%
\end{pgfscope}%
\begin{pgfscope}%
\pgfpathrectangle{\pgfqpoint{1.150000in}{0.150000in}}{\pgfqpoint{5.700000in}{5.700000in}}%
\pgfusepath{clip}%
\pgfsetbuttcap%
\pgfsetroundjoin%
\definecolor{currentfill}{rgb}{0.220057,0.343307,0.549413}%
\pgfsetfillcolor{currentfill}%
\pgfsetfillopacity{0.700000}%
\pgfsetlinewidth{0.000000pt}%
\definecolor{currentstroke}{rgb}{0.000000,0.000000,0.000000}%
\pgfsetstrokecolor{currentstroke}%
\pgfsetdash{}{0pt}%
\pgfpathmoveto{\pgfqpoint{5.600118in}{2.541820in}}%
\pgfpathlineto{\pgfqpoint{5.614467in}{2.543962in}}%
\pgfpathlineto{\pgfqpoint{5.628827in}{2.546173in}}%
\pgfpathlineto{\pgfqpoint{5.643200in}{2.548451in}}%
\pgfpathlineto{\pgfqpoint{5.657584in}{2.550796in}}%
\pgfpathlineto{\pgfqpoint{5.664879in}{2.555292in}}%
\pgfpathlineto{\pgfqpoint{5.672169in}{2.559848in}}%
\pgfpathlineto{\pgfqpoint{5.679455in}{2.564471in}}%
\pgfpathlineto{\pgfqpoint{5.686737in}{2.569165in}}%
\pgfpathlineto{\pgfqpoint{5.672377in}{2.567157in}}%
\pgfpathlineto{\pgfqpoint{5.658030in}{2.565216in}}%
\pgfpathlineto{\pgfqpoint{5.643694in}{2.563342in}}%
\pgfpathlineto{\pgfqpoint{5.629370in}{2.561536in}}%
\pgfpathlineto{\pgfqpoint{5.622063in}{2.556497in}}%
\pgfpathlineto{\pgfqpoint{5.614752in}{2.551535in}}%
\pgfpathlineto{\pgfqpoint{5.607437in}{2.546645in}}%
\pgfpathlineto{\pgfqpoint{5.600118in}{2.541820in}}%
\pgfpathclose%
\pgfusepath{fill}%
\end{pgfscope}%
\begin{pgfscope}%
\pgfpathrectangle{\pgfqpoint{1.150000in}{0.150000in}}{\pgfqpoint{5.700000in}{5.700000in}}%
\pgfusepath{clip}%
\pgfsetbuttcap%
\pgfsetroundjoin%
\definecolor{currentfill}{rgb}{0.243113,0.292092,0.538516}%
\pgfsetfillcolor{currentfill}%
\pgfsetfillopacity{0.700000}%
\pgfsetlinewidth{0.000000pt}%
\definecolor{currentstroke}{rgb}{0.000000,0.000000,0.000000}%
\pgfsetstrokecolor{currentstroke}%
\pgfsetdash{}{0pt}%
\pgfpathmoveto{\pgfqpoint{5.195969in}{2.417673in}}%
\pgfpathlineto{\pgfqpoint{5.210175in}{2.419586in}}%
\pgfpathlineto{\pgfqpoint{5.224392in}{2.421568in}}%
\pgfpathlineto{\pgfqpoint{5.238621in}{2.423619in}}%
\pgfpathlineto{\pgfqpoint{5.252861in}{2.425739in}}%
\pgfpathlineto{\pgfqpoint{5.260354in}{2.431240in}}%
\pgfpathlineto{\pgfqpoint{5.267841in}{2.436743in}}%
\pgfpathlineto{\pgfqpoint{5.275322in}{2.442254in}}%
\pgfpathlineto{\pgfqpoint{5.282797in}{2.447776in}}%
\pgfpathlineto{\pgfqpoint{5.268576in}{2.445909in}}%
\pgfpathlineto{\pgfqpoint{5.254367in}{2.444111in}}%
\pgfpathlineto{\pgfqpoint{5.240169in}{2.442382in}}%
\pgfpathlineto{\pgfqpoint{5.225982in}{2.440722in}}%
\pgfpathlineto{\pgfqpoint{5.218488in}{2.434939in}}%
\pgfpathlineto{\pgfqpoint{5.210987in}{2.429173in}}%
\pgfpathlineto{\pgfqpoint{5.203481in}{2.423419in}}%
\pgfpathlineto{\pgfqpoint{5.195969in}{2.417673in}}%
\pgfpathclose%
\pgfusepath{fill}%
\end{pgfscope}%
\begin{pgfscope}%
\pgfpathrectangle{\pgfqpoint{1.150000in}{0.150000in}}{\pgfqpoint{5.700000in}{5.700000in}}%
\pgfusepath{clip}%
\pgfsetbuttcap%
\pgfsetroundjoin%
\definecolor{currentfill}{rgb}{0.282884,0.135920,0.453427}%
\pgfsetfillcolor{currentfill}%
\pgfsetfillopacity{0.700000}%
\pgfsetlinewidth{0.000000pt}%
\definecolor{currentstroke}{rgb}{0.000000,0.000000,0.000000}%
\pgfsetstrokecolor{currentstroke}%
\pgfsetdash{}{0pt}%
\pgfpathmoveto{\pgfqpoint{4.213317in}{2.077861in}}%
\pgfpathlineto{\pgfqpoint{4.227184in}{2.077706in}}%
\pgfpathlineto{\pgfqpoint{4.241060in}{2.077625in}}%
\pgfpathlineto{\pgfqpoint{4.254944in}{2.077619in}}%
\pgfpathlineto{\pgfqpoint{4.268837in}{2.077687in}}%
\pgfpathlineto{\pgfqpoint{4.276754in}{2.086269in}}%
\pgfpathlineto{\pgfqpoint{4.284665in}{2.094804in}}%
\pgfpathlineto{\pgfqpoint{4.292570in}{2.103293in}}%
\pgfpathlineto{\pgfqpoint{4.300469in}{2.111737in}}%
\pgfpathlineto{\pgfqpoint{4.286587in}{2.111711in}}%
\pgfpathlineto{\pgfqpoint{4.272713in}{2.111760in}}%
\pgfpathlineto{\pgfqpoint{4.258848in}{2.111883in}}%
\pgfpathlineto{\pgfqpoint{4.244992in}{2.112080in}}%
\pgfpathlineto{\pgfqpoint{4.237082in}{2.103586in}}%
\pgfpathlineto{\pgfqpoint{4.229166in}{2.095052in}}%
\pgfpathlineto{\pgfqpoint{4.221244in}{2.086478in}}%
\pgfpathlineto{\pgfqpoint{4.213317in}{2.077861in}}%
\pgfpathclose%
\pgfusepath{fill}%
\end{pgfscope}%
\begin{pgfscope}%
\pgfpathrectangle{\pgfqpoint{1.150000in}{0.150000in}}{\pgfqpoint{5.700000in}{5.700000in}}%
\pgfusepath{clip}%
\pgfsetbuttcap%
\pgfsetroundjoin%
\definecolor{currentfill}{rgb}{0.273809,0.031497,0.358853}%
\pgfsetfillcolor{currentfill}%
\pgfsetfillopacity{0.700000}%
\pgfsetlinewidth{0.000000pt}%
\definecolor{currentstroke}{rgb}{0.000000,0.000000,0.000000}%
\pgfsetstrokecolor{currentstroke}%
\pgfsetdash{}{0pt}%
\pgfpathmoveto{\pgfqpoint{3.492219in}{1.884697in}}%
\pgfpathlineto{\pgfqpoint{3.505904in}{1.881435in}}%
\pgfpathlineto{\pgfqpoint{3.519594in}{1.878256in}}%
\pgfpathlineto{\pgfqpoint{3.533290in}{1.875160in}}%
\pgfpathlineto{\pgfqpoint{3.546992in}{1.872147in}}%
\pgfpathlineto{\pgfqpoint{3.555167in}{1.881178in}}%
\pgfpathlineto{\pgfqpoint{3.563337in}{1.890211in}}%
\pgfpathlineto{\pgfqpoint{3.571500in}{1.899244in}}%
\pgfpathlineto{\pgfqpoint{3.579658in}{1.908276in}}%
\pgfpathlineto{\pgfqpoint{3.565968in}{1.911188in}}%
\pgfpathlineto{\pgfqpoint{3.552285in}{1.914182in}}%
\pgfpathlineto{\pgfqpoint{3.538607in}{1.917258in}}%
\pgfpathlineto{\pgfqpoint{3.524935in}{1.920418in}}%
\pgfpathlineto{\pgfqpoint{3.516765in}{1.911480in}}%
\pgfpathlineto{\pgfqpoint{3.508589in}{1.902547in}}%
\pgfpathlineto{\pgfqpoint{3.500407in}{1.893618in}}%
\pgfpathlineto{\pgfqpoint{3.492219in}{1.884697in}}%
\pgfpathclose%
\pgfusepath{fill}%
\end{pgfscope}%
\begin{pgfscope}%
\pgfpathrectangle{\pgfqpoint{1.150000in}{0.150000in}}{\pgfqpoint{5.700000in}{5.700000in}}%
\pgfusepath{clip}%
\pgfsetbuttcap%
\pgfsetroundjoin%
\definecolor{currentfill}{rgb}{0.269308,0.218818,0.509577}%
\pgfsetfillcolor{currentfill}%
\pgfsetfillopacity{0.700000}%
\pgfsetlinewidth{0.000000pt}%
\definecolor{currentstroke}{rgb}{0.000000,0.000000,0.000000}%
\pgfsetstrokecolor{currentstroke}%
\pgfsetdash{}{0pt}%
\pgfpathmoveto{\pgfqpoint{4.704674in}{2.250402in}}%
\pgfpathlineto{\pgfqpoint{4.718706in}{2.251577in}}%
\pgfpathlineto{\pgfqpoint{4.732747in}{2.252823in}}%
\pgfpathlineto{\pgfqpoint{4.746799in}{2.254141in}}%
\pgfpathlineto{\pgfqpoint{4.760862in}{2.255529in}}%
\pgfpathlineto{\pgfqpoint{4.768582in}{2.262647in}}%
\pgfpathlineto{\pgfqpoint{4.776296in}{2.269725in}}%
\pgfpathlineto{\pgfqpoint{4.784003in}{2.276767in}}%
\pgfpathlineto{\pgfqpoint{4.791704in}{2.283775in}}%
\pgfpathlineto{\pgfqpoint{4.777656in}{2.282533in}}%
\pgfpathlineto{\pgfqpoint{4.763618in}{2.281363in}}%
\pgfpathlineto{\pgfqpoint{4.749590in}{2.280264in}}%
\pgfpathlineto{\pgfqpoint{4.735571in}{2.279237in}}%
\pgfpathlineto{\pgfqpoint{4.727856in}{2.272074in}}%
\pgfpathlineto{\pgfqpoint{4.720135in}{2.264882in}}%
\pgfpathlineto{\pgfqpoint{4.712408in}{2.257660in}}%
\pgfpathlineto{\pgfqpoint{4.704674in}{2.250402in}}%
\pgfpathclose%
\pgfusepath{fill}%
\end{pgfscope}%
\begin{pgfscope}%
\pgfpathrectangle{\pgfqpoint{1.150000in}{0.150000in}}{\pgfqpoint{5.700000in}{5.700000in}}%
\pgfusepath{clip}%
\pgfsetbuttcap%
\pgfsetroundjoin%
\definecolor{currentfill}{rgb}{0.266580,0.228262,0.514349}%
\pgfsetfillcolor{currentfill}%
\pgfsetfillopacity{0.700000}%
\pgfsetlinewidth{0.000000pt}%
\definecolor{currentstroke}{rgb}{0.000000,0.000000,0.000000}%
\pgfsetstrokecolor{currentstroke}%
\pgfsetdash{}{0pt}%
\pgfpathmoveto{\pgfqpoint{2.129614in}{2.312967in}}%
\pgfpathlineto{\pgfqpoint{2.143349in}{2.299668in}}%
\pgfpathlineto{\pgfqpoint{2.157080in}{2.286510in}}%
\pgfpathlineto{\pgfqpoint{2.170807in}{2.273492in}}%
\pgfpathlineto{\pgfqpoint{2.184531in}{2.260612in}}%
\pgfpathlineto{\pgfqpoint{2.193429in}{2.262895in}}%
\pgfpathlineto{\pgfqpoint{2.202312in}{2.265368in}}%
\pgfpathlineto{\pgfqpoint{2.211179in}{2.268028in}}%
\pgfpathlineto{\pgfqpoint{2.220032in}{2.270870in}}%
\pgfpathlineto{\pgfqpoint{2.206341in}{2.283473in}}%
\pgfpathlineto{\pgfqpoint{2.192647in}{2.296213in}}%
\pgfpathlineto{\pgfqpoint{2.178950in}{2.309093in}}%
\pgfpathlineto{\pgfqpoint{2.165249in}{2.322113in}}%
\pgfpathlineto{\pgfqpoint{2.156364in}{2.319540in}}%
\pgfpathlineto{\pgfqpoint{2.147463in}{2.317156in}}%
\pgfpathlineto{\pgfqpoint{2.138547in}{2.314963in}}%
\pgfpathlineto{\pgfqpoint{2.129614in}{2.312967in}}%
\pgfpathclose%
\pgfusepath{fill}%
\end{pgfscope}%
\begin{pgfscope}%
\pgfpathrectangle{\pgfqpoint{1.150000in}{0.150000in}}{\pgfqpoint{5.700000in}{5.700000in}}%
\pgfusepath{clip}%
\pgfsetbuttcap%
\pgfsetroundjoin%
\definecolor{currentfill}{rgb}{0.269944,0.014625,0.341379}%
\pgfsetfillcolor{currentfill}%
\pgfsetfillopacity{0.700000}%
\pgfsetlinewidth{0.000000pt}%
\definecolor{currentstroke}{rgb}{0.000000,0.000000,0.000000}%
\pgfsetstrokecolor{currentstroke}%
\pgfsetdash{}{0pt}%
\pgfpathmoveto{\pgfqpoint{3.262371in}{1.862668in}}%
\pgfpathlineto{\pgfqpoint{3.276021in}{1.858117in}}%
\pgfpathlineto{\pgfqpoint{3.289676in}{1.853654in}}%
\pgfpathlineto{\pgfqpoint{3.303335in}{1.849277in}}%
\pgfpathlineto{\pgfqpoint{3.317000in}{1.844987in}}%
\pgfpathlineto{\pgfqpoint{3.325265in}{1.853500in}}%
\pgfpathlineto{\pgfqpoint{3.333524in}{1.862042in}}%
\pgfpathlineto{\pgfqpoint{3.341776in}{1.870612in}}%
\pgfpathlineto{\pgfqpoint{3.350022in}{1.879208in}}%
\pgfpathlineto{\pgfqpoint{3.336372in}{1.883355in}}%
\pgfpathlineto{\pgfqpoint{3.322727in}{1.887588in}}%
\pgfpathlineto{\pgfqpoint{3.309087in}{1.891908in}}%
\pgfpathlineto{\pgfqpoint{3.295452in}{1.896315in}}%
\pgfpathlineto{\pgfqpoint{3.287192in}{1.887855in}}%
\pgfpathlineto{\pgfqpoint{3.278925in}{1.879426in}}%
\pgfpathlineto{\pgfqpoint{3.270651in}{1.871029in}}%
\pgfpathlineto{\pgfqpoint{3.262371in}{1.862668in}}%
\pgfpathclose%
\pgfusepath{fill}%
\end{pgfscope}%
\begin{pgfscope}%
\pgfpathrectangle{\pgfqpoint{1.150000in}{0.150000in}}{\pgfqpoint{5.700000in}{5.700000in}}%
\pgfusepath{clip}%
\pgfsetbuttcap%
\pgfsetroundjoin%
\definecolor{currentfill}{rgb}{0.283229,0.120777,0.440584}%
\pgfsetfillcolor{currentfill}%
\pgfsetfillopacity{0.700000}%
\pgfsetlinewidth{0.000000pt}%
\definecolor{currentstroke}{rgb}{0.000000,0.000000,0.000000}%
\pgfsetstrokecolor{currentstroke}%
\pgfsetdash{}{0pt}%
\pgfpathmoveto{\pgfqpoint{4.126122in}{2.044238in}}%
\pgfpathlineto{\pgfqpoint{4.139966in}{2.043804in}}%
\pgfpathlineto{\pgfqpoint{4.153818in}{2.043446in}}%
\pgfpathlineto{\pgfqpoint{4.167679in}{2.043163in}}%
\pgfpathlineto{\pgfqpoint{4.181548in}{2.042954in}}%
\pgfpathlineto{\pgfqpoint{4.189499in}{2.051749in}}%
\pgfpathlineto{\pgfqpoint{4.197444in}{2.060497in}}%
\pgfpathlineto{\pgfqpoint{4.205383in}{2.069201in}}%
\pgfpathlineto{\pgfqpoint{4.213317in}{2.077861in}}%
\pgfpathlineto{\pgfqpoint{4.199458in}{2.078091in}}%
\pgfpathlineto{\pgfqpoint{4.185608in}{2.078396in}}%
\pgfpathlineto{\pgfqpoint{4.171766in}{2.078775in}}%
\pgfpathlineto{\pgfqpoint{4.157933in}{2.079231in}}%
\pgfpathlineto{\pgfqpoint{4.149989in}{2.070542in}}%
\pgfpathlineto{\pgfqpoint{4.142039in}{2.061814in}}%
\pgfpathlineto{\pgfqpoint{4.134083in}{2.053046in}}%
\pgfpathlineto{\pgfqpoint{4.126122in}{2.044238in}}%
\pgfpathclose%
\pgfusepath{fill}%
\end{pgfscope}%
\begin{pgfscope}%
\pgfpathrectangle{\pgfqpoint{1.150000in}{0.150000in}}{\pgfqpoint{5.700000in}{5.700000in}}%
\pgfusepath{clip}%
\pgfsetbuttcap%
\pgfsetroundjoin%
\definecolor{currentfill}{rgb}{0.277941,0.056324,0.381191}%
\pgfsetfillcolor{currentfill}%
\pgfsetfillopacity{0.700000}%
\pgfsetlinewidth{0.000000pt}%
\definecolor{currentstroke}{rgb}{0.000000,0.000000,0.000000}%
\pgfsetstrokecolor{currentstroke}%
\pgfsetdash{}{0pt}%
\pgfpathmoveto{\pgfqpoint{3.721879in}{1.924667in}}%
\pgfpathlineto{\pgfqpoint{3.735615in}{1.922565in}}%
\pgfpathlineto{\pgfqpoint{3.749357in}{1.920543in}}%
\pgfpathlineto{\pgfqpoint{3.763106in}{1.918600in}}%
\pgfpathlineto{\pgfqpoint{3.776862in}{1.916736in}}%
\pgfpathlineto{\pgfqpoint{3.784957in}{1.925957in}}%
\pgfpathlineto{\pgfqpoint{3.793046in}{1.935156in}}%
\pgfpathlineto{\pgfqpoint{3.801129in}{1.944334in}}%
\pgfpathlineto{\pgfqpoint{3.809206in}{1.953490in}}%
\pgfpathlineto{\pgfqpoint{3.795461in}{1.955293in}}%
\pgfpathlineto{\pgfqpoint{3.781723in}{1.957175in}}%
\pgfpathlineto{\pgfqpoint{3.767991in}{1.959136in}}%
\pgfpathlineto{\pgfqpoint{3.754267in}{1.961177in}}%
\pgfpathlineto{\pgfqpoint{3.746179in}{1.952075in}}%
\pgfpathlineto{\pgfqpoint{3.738084in}{1.942955in}}%
\pgfpathlineto{\pgfqpoint{3.729985in}{1.933819in}}%
\pgfpathlineto{\pgfqpoint{3.721879in}{1.924667in}}%
\pgfpathclose%
\pgfusepath{fill}%
\end{pgfscope}%
\begin{pgfscope}%
\pgfpathrectangle{\pgfqpoint{1.150000in}{0.150000in}}{\pgfqpoint{5.700000in}{5.700000in}}%
\pgfusepath{clip}%
\pgfsetbuttcap%
\pgfsetroundjoin%
\definecolor{currentfill}{rgb}{0.248629,0.278775,0.534556}%
\pgfsetfillcolor{currentfill}%
\pgfsetfillopacity{0.700000}%
\pgfsetlinewidth{0.000000pt}%
\definecolor{currentstroke}{rgb}{0.000000,0.000000,0.000000}%
\pgfsetstrokecolor{currentstroke}%
\pgfsetdash{}{0pt}%
\pgfpathmoveto{\pgfqpoint{5.109072in}{2.386794in}}%
\pgfpathlineto{\pgfqpoint{5.123252in}{2.388661in}}%
\pgfpathlineto{\pgfqpoint{5.137443in}{2.390598in}}%
\pgfpathlineto{\pgfqpoint{5.151646in}{2.392604in}}%
\pgfpathlineto{\pgfqpoint{5.165859in}{2.394680in}}%
\pgfpathlineto{\pgfqpoint{5.173396in}{2.400438in}}%
\pgfpathlineto{\pgfqpoint{5.180927in}{2.406187in}}%
\pgfpathlineto{\pgfqpoint{5.188451in}{2.411931in}}%
\pgfpathlineto{\pgfqpoint{5.195969in}{2.417673in}}%
\pgfpathlineto{\pgfqpoint{5.181774in}{2.415830in}}%
\pgfpathlineto{\pgfqpoint{5.167590in}{2.414055in}}%
\pgfpathlineto{\pgfqpoint{5.153417in}{2.412350in}}%
\pgfpathlineto{\pgfqpoint{5.139255in}{2.410715in}}%
\pgfpathlineto{\pgfqpoint{5.131718in}{2.404733in}}%
\pgfpathlineto{\pgfqpoint{5.124176in}{2.398754in}}%
\pgfpathlineto{\pgfqpoint{5.116627in}{2.392776in}}%
\pgfpathlineto{\pgfqpoint{5.109072in}{2.386794in}}%
\pgfpathclose%
\pgfusepath{fill}%
\end{pgfscope}%
\begin{pgfscope}%
\pgfpathrectangle{\pgfqpoint{1.150000in}{0.150000in}}{\pgfqpoint{5.700000in}{5.700000in}}%
\pgfusepath{clip}%
\pgfsetbuttcap%
\pgfsetroundjoin%
\definecolor{currentfill}{rgb}{0.273006,0.204520,0.501721}%
\pgfsetfillcolor{currentfill}%
\pgfsetfillopacity{0.700000}%
\pgfsetlinewidth{0.000000pt}%
\definecolor{currentstroke}{rgb}{0.000000,0.000000,0.000000}%
\pgfsetstrokecolor{currentstroke}%
\pgfsetdash{}{0pt}%
\pgfpathmoveto{\pgfqpoint{4.617594in}{2.216490in}}%
\pgfpathlineto{\pgfqpoint{4.631599in}{2.217504in}}%
\pgfpathlineto{\pgfqpoint{4.645614in}{2.218590in}}%
\pgfpathlineto{\pgfqpoint{4.659639in}{2.219748in}}%
\pgfpathlineto{\pgfqpoint{4.673674in}{2.220978in}}%
\pgfpathlineto{\pgfqpoint{4.681434in}{2.228398in}}%
\pgfpathlineto{\pgfqpoint{4.689187in}{2.235774in}}%
\pgfpathlineto{\pgfqpoint{4.696933in}{2.243108in}}%
\pgfpathlineto{\pgfqpoint{4.704674in}{2.250402in}}%
\pgfpathlineto{\pgfqpoint{4.690652in}{2.249299in}}%
\pgfpathlineto{\pgfqpoint{4.676640in}{2.248268in}}%
\pgfpathlineto{\pgfqpoint{4.662638in}{2.247308in}}%
\pgfpathlineto{\pgfqpoint{4.648646in}{2.246420in}}%
\pgfpathlineto{\pgfqpoint{4.640892in}{2.238991in}}%
\pgfpathlineto{\pgfqpoint{4.633132in}{2.231529in}}%
\pgfpathlineto{\pgfqpoint{4.625366in}{2.224029in}}%
\pgfpathlineto{\pgfqpoint{4.617594in}{2.216490in}}%
\pgfpathclose%
\pgfusepath{fill}%
\end{pgfscope}%
\begin{pgfscope}%
\pgfpathrectangle{\pgfqpoint{1.150000in}{0.150000in}}{\pgfqpoint{5.700000in}{5.700000in}}%
\pgfusepath{clip}%
\pgfsetbuttcap%
\pgfsetroundjoin%
\definecolor{currentfill}{rgb}{0.278791,0.062145,0.386592}%
\pgfsetfillcolor{currentfill}%
\pgfsetfillopacity{0.700000}%
\pgfsetlinewidth{0.000000pt}%
\definecolor{currentstroke}{rgb}{0.000000,0.000000,0.000000}%
\pgfsetstrokecolor{currentstroke}%
\pgfsetdash{}{0pt}%
\pgfpathmoveto{\pgfqpoint{2.691605in}{1.955035in}}%
\pgfpathlineto{\pgfqpoint{2.705238in}{1.946644in}}%
\pgfpathlineto{\pgfqpoint{2.718872in}{1.938357in}}%
\pgfpathlineto{\pgfqpoint{2.732507in}{1.930174in}}%
\pgfpathlineto{\pgfqpoint{2.746144in}{1.922096in}}%
\pgfpathlineto{\pgfqpoint{2.754687in}{1.927937in}}%
\pgfpathlineto{\pgfqpoint{2.763221in}{1.933891in}}%
\pgfpathlineto{\pgfqpoint{2.771744in}{1.939955in}}%
\pgfpathlineto{\pgfqpoint{2.780258in}{1.946125in}}%
\pgfpathlineto{\pgfqpoint{2.766644in}{1.953976in}}%
\pgfpathlineto{\pgfqpoint{2.753032in}{1.961931in}}%
\pgfpathlineto{\pgfqpoint{2.739421in}{1.969990in}}%
\pgfpathlineto{\pgfqpoint{2.725812in}{1.978154in}}%
\pgfpathlineto{\pgfqpoint{2.717275in}{1.972204in}}%
\pgfpathlineto{\pgfqpoint{2.708729in}{1.966365in}}%
\pgfpathlineto{\pgfqpoint{2.700172in}{1.960641in}}%
\pgfpathlineto{\pgfqpoint{2.691605in}{1.955035in}}%
\pgfpathclose%
\pgfusepath{fill}%
\end{pgfscope}%
\begin{pgfscope}%
\pgfpathrectangle{\pgfqpoint{1.150000in}{0.150000in}}{\pgfqpoint{5.700000in}{5.700000in}}%
\pgfusepath{clip}%
\pgfsetbuttcap%
\pgfsetroundjoin%
\definecolor{currentfill}{rgb}{0.271828,0.209303,0.504434}%
\pgfsetfillcolor{currentfill}%
\pgfsetfillopacity{0.700000}%
\pgfsetlinewidth{0.000000pt}%
\definecolor{currentstroke}{rgb}{0.000000,0.000000,0.000000}%
\pgfsetstrokecolor{currentstroke}%
\pgfsetdash{}{0pt}%
\pgfpathmoveto{\pgfqpoint{2.184531in}{2.260612in}}%
\pgfpathlineto{\pgfqpoint{2.198251in}{2.247870in}}%
\pgfpathlineto{\pgfqpoint{2.211968in}{2.235263in}}%
\pgfpathlineto{\pgfqpoint{2.225682in}{2.222791in}}%
\pgfpathlineto{\pgfqpoint{2.239394in}{2.210452in}}%
\pgfpathlineto{\pgfqpoint{2.248259in}{2.213020in}}%
\pgfpathlineto{\pgfqpoint{2.257109in}{2.215773in}}%
\pgfpathlineto{\pgfqpoint{2.265944in}{2.218706in}}%
\pgfpathlineto{\pgfqpoint{2.274765in}{2.221817in}}%
\pgfpathlineto{\pgfqpoint{2.261086in}{2.233880in}}%
\pgfpathlineto{\pgfqpoint{2.247404in}{2.246075in}}%
\pgfpathlineto{\pgfqpoint{2.233720in}{2.258405in}}%
\pgfpathlineto{\pgfqpoint{2.220032in}{2.270870in}}%
\pgfpathlineto{\pgfqpoint{2.211179in}{2.268028in}}%
\pgfpathlineto{\pgfqpoint{2.202312in}{2.265368in}}%
\pgfpathlineto{\pgfqpoint{2.193429in}{2.262895in}}%
\pgfpathlineto{\pgfqpoint{2.184531in}{2.260612in}}%
\pgfpathclose%
\pgfusepath{fill}%
\end{pgfscope}%
\begin{pgfscope}%
\pgfpathrectangle{\pgfqpoint{1.150000in}{0.150000in}}{\pgfqpoint{5.700000in}{5.700000in}}%
\pgfusepath{clip}%
\pgfsetbuttcap%
\pgfsetroundjoin%
\definecolor{currentfill}{rgb}{0.223925,0.334994,0.548053}%
\pgfsetfillcolor{currentfill}%
\pgfsetfillopacity{0.700000}%
\pgfsetlinewidth{0.000000pt}%
\definecolor{currentstroke}{rgb}{0.000000,0.000000,0.000000}%
\pgfsetstrokecolor{currentstroke}%
\pgfsetdash{}{0pt}%
\pgfpathmoveto{\pgfqpoint{5.513420in}{2.513900in}}%
\pgfpathlineto{\pgfqpoint{5.527746in}{2.516087in}}%
\pgfpathlineto{\pgfqpoint{5.542083in}{2.518343in}}%
\pgfpathlineto{\pgfqpoint{5.556433in}{2.520667in}}%
\pgfpathlineto{\pgfqpoint{5.570794in}{2.523058in}}%
\pgfpathlineto{\pgfqpoint{5.578132in}{2.527679in}}%
\pgfpathlineto{\pgfqpoint{5.585465in}{2.532342in}}%
\pgfpathlineto{\pgfqpoint{5.592794in}{2.537054in}}%
\pgfpathlineto{\pgfqpoint{5.600118in}{2.541820in}}%
\pgfpathlineto{\pgfqpoint{5.585780in}{2.539745in}}%
\pgfpathlineto{\pgfqpoint{5.571455in}{2.537737in}}%
\pgfpathlineto{\pgfqpoint{5.557141in}{2.535798in}}%
\pgfpathlineto{\pgfqpoint{5.542838in}{2.533926in}}%
\pgfpathlineto{\pgfqpoint{5.535491in}{2.528837in}}%
\pgfpathlineto{\pgfqpoint{5.528139in}{2.523807in}}%
\pgfpathlineto{\pgfqpoint{5.520782in}{2.518829in}}%
\pgfpathlineto{\pgfqpoint{5.513420in}{2.513900in}}%
\pgfpathclose%
\pgfusepath{fill}%
\end{pgfscope}%
\begin{pgfscope}%
\pgfpathrectangle{\pgfqpoint{1.150000in}{0.150000in}}{\pgfqpoint{5.700000in}{5.700000in}}%
\pgfusepath{clip}%
\pgfsetbuttcap%
\pgfsetroundjoin%
\definecolor{currentfill}{rgb}{0.283091,0.110553,0.431554}%
\pgfsetfillcolor{currentfill}%
\pgfsetfillopacity{0.700000}%
\pgfsetlinewidth{0.000000pt}%
\definecolor{currentstroke}{rgb}{0.000000,0.000000,0.000000}%
\pgfsetstrokecolor{currentstroke}%
\pgfsetdash{}{0pt}%
\pgfpathmoveto{\pgfqpoint{2.493378in}{2.046030in}}%
\pgfpathlineto{\pgfqpoint{2.507032in}{2.036067in}}%
\pgfpathlineto{\pgfqpoint{2.520685in}{2.026219in}}%
\pgfpathlineto{\pgfqpoint{2.534339in}{2.016484in}}%
\pgfpathlineto{\pgfqpoint{2.547992in}{2.006863in}}%
\pgfpathlineto{\pgfqpoint{2.556655in}{2.011447in}}%
\pgfpathlineto{\pgfqpoint{2.565306in}{2.016174in}}%
\pgfpathlineto{\pgfqpoint{2.573945in}{2.021040in}}%
\pgfpathlineto{\pgfqpoint{2.582572in}{2.026042in}}%
\pgfpathlineto{\pgfqpoint{2.568945in}{2.035413in}}%
\pgfpathlineto{\pgfqpoint{2.555318in}{2.044897in}}%
\pgfpathlineto{\pgfqpoint{2.541691in}{2.054495in}}%
\pgfpathlineto{\pgfqpoint{2.528064in}{2.064208in}}%
\pgfpathlineto{\pgfqpoint{2.519410in}{2.059449in}}%
\pgfpathlineto{\pgfqpoint{2.510745in}{2.054830in}}%
\pgfpathlineto{\pgfqpoint{2.502067in}{2.050356in}}%
\pgfpathlineto{\pgfqpoint{2.493378in}{2.046030in}}%
\pgfpathclose%
\pgfusepath{fill}%
\end{pgfscope}%
\begin{pgfscope}%
\pgfpathrectangle{\pgfqpoint{1.150000in}{0.150000in}}{\pgfqpoint{5.700000in}{5.700000in}}%
\pgfusepath{clip}%
\pgfsetbuttcap%
\pgfsetroundjoin%
\definecolor{currentfill}{rgb}{0.206756,0.371758,0.553117}%
\pgfsetfillcolor{currentfill}%
\pgfsetfillopacity{0.700000}%
\pgfsetlinewidth{0.000000pt}%
\definecolor{currentstroke}{rgb}{0.000000,0.000000,0.000000}%
\pgfsetstrokecolor{currentstroke}%
\pgfsetdash{}{0pt}%
\pgfpathmoveto{\pgfqpoint{5.830918in}{2.604418in}}%
\pgfpathlineto{\pgfqpoint{5.845358in}{2.606673in}}%
\pgfpathlineto{\pgfqpoint{5.859811in}{2.608995in}}%
\pgfpathlineto{\pgfqpoint{5.874275in}{2.611385in}}%
\pgfpathlineto{\pgfqpoint{5.881465in}{2.615514in}}%
\pgfpathlineto{\pgfqpoint{5.888652in}{2.619740in}}%
\pgfpathlineto{\pgfqpoint{5.895836in}{2.624070in}}%
\pgfpathlineto{\pgfqpoint{5.903018in}{2.628509in}}%
\pgfpathlineto{\pgfqpoint{5.888582in}{2.626499in}}%
\pgfpathlineto{\pgfqpoint{5.874158in}{2.624555in}}%
\pgfpathlineto{\pgfqpoint{5.859746in}{2.622679in}}%
\pgfpathlineto{\pgfqpoint{5.852543in}{2.617950in}}%
\pgfpathlineto{\pgfqpoint{5.845338in}{2.613335in}}%
\pgfpathlineto{\pgfqpoint{5.838129in}{2.608826in}}%
\pgfpathlineto{\pgfqpoint{5.830918in}{2.604418in}}%
\pgfpathclose%
\pgfusepath{fill}%
\end{pgfscope}%
\begin{pgfscope}%
\pgfpathrectangle{\pgfqpoint{1.150000in}{0.150000in}}{\pgfqpoint{5.700000in}{5.700000in}}%
\pgfusepath{clip}%
\pgfsetbuttcap%
\pgfsetroundjoin%
\definecolor{currentfill}{rgb}{0.271305,0.019942,0.347269}%
\pgfsetfillcolor{currentfill}%
\pgfsetfillopacity{0.700000}%
\pgfsetlinewidth{0.000000pt}%
\definecolor{currentstroke}{rgb}{0.000000,0.000000,0.000000}%
\pgfsetstrokecolor{currentstroke}%
\pgfsetdash{}{0pt}%
\pgfpathmoveto{\pgfqpoint{3.404672in}{1.863479in}}%
\pgfpathlineto{\pgfqpoint{3.418348in}{1.859760in}}%
\pgfpathlineto{\pgfqpoint{3.432029in}{1.856124in}}%
\pgfpathlineto{\pgfqpoint{3.445715in}{1.852573in}}%
\pgfpathlineto{\pgfqpoint{3.459407in}{1.849105in}}%
\pgfpathlineto{\pgfqpoint{3.467619in}{1.857986in}}%
\pgfpathlineto{\pgfqpoint{3.475825in}{1.866879in}}%
\pgfpathlineto{\pgfqpoint{3.484025in}{1.875784in}}%
\pgfpathlineto{\pgfqpoint{3.492219in}{1.884697in}}%
\pgfpathlineto{\pgfqpoint{3.478541in}{1.888042in}}%
\pgfpathlineto{\pgfqpoint{3.464867in}{1.891471in}}%
\pgfpathlineto{\pgfqpoint{3.451200in}{1.894983in}}%
\pgfpathlineto{\pgfqpoint{3.437538in}{1.898580in}}%
\pgfpathlineto{\pgfqpoint{3.429331in}{1.889782in}}%
\pgfpathlineto{\pgfqpoint{3.421117in}{1.880998in}}%
\pgfpathlineto{\pgfqpoint{3.412898in}{1.872230in}}%
\pgfpathlineto{\pgfqpoint{3.404672in}{1.863479in}}%
\pgfpathclose%
\pgfusepath{fill}%
\end{pgfscope}%
\begin{pgfscope}%
\pgfpathrectangle{\pgfqpoint{1.150000in}{0.150000in}}{\pgfqpoint{5.700000in}{5.700000in}}%
\pgfusepath{clip}%
\pgfsetbuttcap%
\pgfsetroundjoin%
\definecolor{currentfill}{rgb}{0.282910,0.105393,0.426902}%
\pgfsetfillcolor{currentfill}%
\pgfsetfillopacity{0.700000}%
\pgfsetlinewidth{0.000000pt}%
\definecolor{currentstroke}{rgb}{0.000000,0.000000,0.000000}%
\pgfsetstrokecolor{currentstroke}%
\pgfsetdash{}{0pt}%
\pgfpathmoveto{\pgfqpoint{4.038882in}{2.011075in}}%
\pgfpathlineto{\pgfqpoint{4.052704in}{2.010339in}}%
\pgfpathlineto{\pgfqpoint{4.066534in}{2.009679in}}%
\pgfpathlineto{\pgfqpoint{4.080372in}{2.009095in}}%
\pgfpathlineto{\pgfqpoint{4.094218in}{2.008587in}}%
\pgfpathlineto{\pgfqpoint{4.102203in}{2.017564in}}%
\pgfpathlineto{\pgfqpoint{4.110181in}{2.026497in}}%
\pgfpathlineto{\pgfqpoint{4.118154in}{2.035389in}}%
\pgfpathlineto{\pgfqpoint{4.126122in}{2.044238in}}%
\pgfpathlineto{\pgfqpoint{4.112286in}{2.044747in}}%
\pgfpathlineto{\pgfqpoint{4.098458in}{2.045332in}}%
\pgfpathlineto{\pgfqpoint{4.084639in}{2.045992in}}%
\pgfpathlineto{\pgfqpoint{4.070828in}{2.046729in}}%
\pgfpathlineto{\pgfqpoint{4.062850in}{2.037871in}}%
\pgfpathlineto{\pgfqpoint{4.054866in}{2.028977in}}%
\pgfpathlineto{\pgfqpoint{4.046877in}{2.020045in}}%
\pgfpathlineto{\pgfqpoint{4.038882in}{2.011075in}}%
\pgfpathclose%
\pgfusepath{fill}%
\end{pgfscope}%
\begin{pgfscope}%
\pgfpathrectangle{\pgfqpoint{1.150000in}{0.150000in}}{\pgfqpoint{5.700000in}{5.700000in}}%
\pgfusepath{clip}%
\pgfsetbuttcap%
\pgfsetroundjoin%
\definecolor{currentfill}{rgb}{0.276194,0.190074,0.493001}%
\pgfsetfillcolor{currentfill}%
\pgfsetfillopacity{0.700000}%
\pgfsetlinewidth{0.000000pt}%
\definecolor{currentstroke}{rgb}{0.000000,0.000000,0.000000}%
\pgfsetstrokecolor{currentstroke}%
\pgfsetdash{}{0pt}%
\pgfpathmoveto{\pgfqpoint{4.530468in}{2.182133in}}%
\pgfpathlineto{\pgfqpoint{4.544446in}{2.182964in}}%
\pgfpathlineto{\pgfqpoint{4.558435in}{2.183867in}}%
\pgfpathlineto{\pgfqpoint{4.572433in}{2.184843in}}%
\pgfpathlineto{\pgfqpoint{4.586441in}{2.185890in}}%
\pgfpathlineto{\pgfqpoint{4.594239in}{2.193611in}}%
\pgfpathlineto{\pgfqpoint{4.602030in}{2.201283in}}%
\pgfpathlineto{\pgfqpoint{4.609815in}{2.208908in}}%
\pgfpathlineto{\pgfqpoint{4.617594in}{2.216490in}}%
\pgfpathlineto{\pgfqpoint{4.603598in}{2.215547in}}%
\pgfpathlineto{\pgfqpoint{4.589613in}{2.214677in}}%
\pgfpathlineto{\pgfqpoint{4.575636in}{2.213879in}}%
\pgfpathlineto{\pgfqpoint{4.561670in}{2.213154in}}%
\pgfpathlineto{\pgfqpoint{4.553879in}{2.205459in}}%
\pgfpathlineto{\pgfqpoint{4.546081in}{2.197726in}}%
\pgfpathlineto{\pgfqpoint{4.538278in}{2.189951in}}%
\pgfpathlineto{\pgfqpoint{4.530468in}{2.182133in}}%
\pgfpathclose%
\pgfusepath{fill}%
\end{pgfscope}%
\begin{pgfscope}%
\pgfpathrectangle{\pgfqpoint{1.150000in}{0.150000in}}{\pgfqpoint{5.700000in}{5.700000in}}%
\pgfusepath{clip}%
\pgfsetbuttcap%
\pgfsetroundjoin%
\definecolor{currentfill}{rgb}{0.252194,0.269783,0.531579}%
\pgfsetfillcolor{currentfill}%
\pgfsetfillopacity{0.700000}%
\pgfsetlinewidth{0.000000pt}%
\definecolor{currentstroke}{rgb}{0.000000,0.000000,0.000000}%
\pgfsetstrokecolor{currentstroke}%
\pgfsetdash{}{0pt}%
\pgfpathmoveto{\pgfqpoint{5.022109in}{2.355121in}}%
\pgfpathlineto{\pgfqpoint{5.036263in}{2.356921in}}%
\pgfpathlineto{\pgfqpoint{5.050427in}{2.358790in}}%
\pgfpathlineto{\pgfqpoint{5.064603in}{2.360729in}}%
\pgfpathlineto{\pgfqpoint{5.078789in}{2.362739in}}%
\pgfpathlineto{\pgfqpoint{5.086369in}{2.368779in}}%
\pgfpathlineto{\pgfqpoint{5.093943in}{2.374799in}}%
\pgfpathlineto{\pgfqpoint{5.101511in}{2.380802in}}%
\pgfpathlineto{\pgfqpoint{5.109072in}{2.386794in}}%
\pgfpathlineto{\pgfqpoint{5.094903in}{2.384996in}}%
\pgfpathlineto{\pgfqpoint{5.080744in}{2.383268in}}%
\pgfpathlineto{\pgfqpoint{5.066597in}{2.381609in}}%
\pgfpathlineto{\pgfqpoint{5.052460in}{2.380021in}}%
\pgfpathlineto{\pgfqpoint{5.044881in}{2.373811in}}%
\pgfpathlineto{\pgfqpoint{5.037297in}{2.367594in}}%
\pgfpathlineto{\pgfqpoint{5.029706in}{2.361365in}}%
\pgfpathlineto{\pgfqpoint{5.022109in}{2.355121in}}%
\pgfpathclose%
\pgfusepath{fill}%
\end{pgfscope}%
\begin{pgfscope}%
\pgfpathrectangle{\pgfqpoint{1.150000in}{0.150000in}}{\pgfqpoint{5.700000in}{5.700000in}}%
\pgfusepath{clip}%
\pgfsetbuttcap%
\pgfsetroundjoin%
\definecolor{currentfill}{rgb}{0.271305,0.019942,0.347269}%
\pgfsetfillcolor{currentfill}%
\pgfsetfillopacity{0.700000}%
\pgfsetlinewidth{0.000000pt}%
\definecolor{currentstroke}{rgb}{0.000000,0.000000,0.000000}%
\pgfsetstrokecolor{currentstroke}%
\pgfsetdash{}{0pt}%
\pgfpathmoveto{\pgfqpoint{3.032006in}{1.863038in}}%
\pgfpathlineto{\pgfqpoint{3.045642in}{1.857057in}}%
\pgfpathlineto{\pgfqpoint{3.059281in}{1.851168in}}%
\pgfpathlineto{\pgfqpoint{3.072924in}{1.845372in}}%
\pgfpathlineto{\pgfqpoint{3.086570in}{1.839667in}}%
\pgfpathlineto{\pgfqpoint{3.094941in}{1.847288in}}%
\pgfpathlineto{\pgfqpoint{3.103304in}{1.854972in}}%
\pgfpathlineto{\pgfqpoint{3.111660in}{1.862717in}}%
\pgfpathlineto{\pgfqpoint{3.120008in}{1.870520in}}%
\pgfpathlineto{\pgfqpoint{3.106379in}{1.876040in}}%
\pgfpathlineto{\pgfqpoint{3.092754in}{1.881652in}}%
\pgfpathlineto{\pgfqpoint{3.079132in}{1.887356in}}%
\pgfpathlineto{\pgfqpoint{3.065514in}{1.893153in}}%
\pgfpathlineto{\pgfqpoint{3.057149in}{1.885526in}}%
\pgfpathlineto{\pgfqpoint{3.048776in}{1.877964in}}%
\pgfpathlineto{\pgfqpoint{3.040395in}{1.870467in}}%
\pgfpathlineto{\pgfqpoint{3.032006in}{1.863038in}}%
\pgfpathclose%
\pgfusepath{fill}%
\end{pgfscope}%
\begin{pgfscope}%
\pgfpathrectangle{\pgfqpoint{1.150000in}{0.150000in}}{\pgfqpoint{5.700000in}{5.700000in}}%
\pgfusepath{clip}%
\pgfsetbuttcap%
\pgfsetroundjoin%
\definecolor{currentfill}{rgb}{0.276194,0.190074,0.493001}%
\pgfsetfillcolor{currentfill}%
\pgfsetfillopacity{0.700000}%
\pgfsetlinewidth{0.000000pt}%
\definecolor{currentstroke}{rgb}{0.000000,0.000000,0.000000}%
\pgfsetstrokecolor{currentstroke}%
\pgfsetdash{}{0pt}%
\pgfpathmoveto{\pgfqpoint{2.239394in}{2.210452in}}%
\pgfpathlineto{\pgfqpoint{2.253102in}{2.198246in}}%
\pgfpathlineto{\pgfqpoint{2.266808in}{2.186171in}}%
\pgfpathlineto{\pgfqpoint{2.280511in}{2.174226in}}%
\pgfpathlineto{\pgfqpoint{2.294212in}{2.162410in}}%
\pgfpathlineto{\pgfqpoint{2.303045in}{2.165261in}}%
\pgfpathlineto{\pgfqpoint{2.311863in}{2.168292in}}%
\pgfpathlineto{\pgfqpoint{2.320667in}{2.171498in}}%
\pgfpathlineto{\pgfqpoint{2.329457in}{2.174877in}}%
\pgfpathlineto{\pgfqpoint{2.315787in}{2.186418in}}%
\pgfpathlineto{\pgfqpoint{2.302116in}{2.198087in}}%
\pgfpathlineto{\pgfqpoint{2.288441in}{2.209887in}}%
\pgfpathlineto{\pgfqpoint{2.274765in}{2.221817in}}%
\pgfpathlineto{\pgfqpoint{2.265944in}{2.218706in}}%
\pgfpathlineto{\pgfqpoint{2.257109in}{2.215773in}}%
\pgfpathlineto{\pgfqpoint{2.248259in}{2.213020in}}%
\pgfpathlineto{\pgfqpoint{2.239394in}{2.210452in}}%
\pgfpathclose%
\pgfusepath{fill}%
\end{pgfscope}%
\begin{pgfscope}%
\pgfpathrectangle{\pgfqpoint{1.150000in}{0.150000in}}{\pgfqpoint{5.700000in}{5.700000in}}%
\pgfusepath{clip}%
\pgfsetbuttcap%
\pgfsetroundjoin%
\definecolor{currentfill}{rgb}{0.276022,0.044167,0.370164}%
\pgfsetfillcolor{currentfill}%
\pgfsetfillopacity{0.700000}%
\pgfsetlinewidth{0.000000pt}%
\definecolor{currentstroke}{rgb}{0.000000,0.000000,0.000000}%
\pgfsetstrokecolor{currentstroke}%
\pgfsetdash{}{0pt}%
\pgfpathmoveto{\pgfqpoint{3.634478in}{1.897447in}}%
\pgfpathlineto{\pgfqpoint{3.648199in}{1.894943in}}%
\pgfpathlineto{\pgfqpoint{3.661926in}{1.892519in}}%
\pgfpathlineto{\pgfqpoint{3.675660in}{1.890176in}}%
\pgfpathlineto{\pgfqpoint{3.689400in}{1.887912in}}%
\pgfpathlineto{\pgfqpoint{3.697528in}{1.897122in}}%
\pgfpathlineto{\pgfqpoint{3.705651in}{1.906318in}}%
\pgfpathlineto{\pgfqpoint{3.713768in}{1.915500in}}%
\pgfpathlineto{\pgfqpoint{3.721879in}{1.924667in}}%
\pgfpathlineto{\pgfqpoint{3.708150in}{1.926849in}}%
\pgfpathlineto{\pgfqpoint{3.694428in}{1.929111in}}%
\pgfpathlineto{\pgfqpoint{3.680712in}{1.931453in}}%
\pgfpathlineto{\pgfqpoint{3.667003in}{1.933876in}}%
\pgfpathlineto{\pgfqpoint{3.658881in}{1.924782in}}%
\pgfpathlineto{\pgfqpoint{3.650752in}{1.915679in}}%
\pgfpathlineto{\pgfqpoint{3.642618in}{1.906567in}}%
\pgfpathlineto{\pgfqpoint{3.634478in}{1.897447in}}%
\pgfpathclose%
\pgfusepath{fill}%
\end{pgfscope}%
\begin{pgfscope}%
\pgfpathrectangle{\pgfqpoint{1.150000in}{0.150000in}}{\pgfqpoint{5.700000in}{5.700000in}}%
\pgfusepath{clip}%
\pgfsetbuttcap%
\pgfsetroundjoin%
\definecolor{currentfill}{rgb}{0.273809,0.031497,0.358853}%
\pgfsetfillcolor{currentfill}%
\pgfsetfillopacity{0.700000}%
\pgfsetlinewidth{0.000000pt}%
\definecolor{currentstroke}{rgb}{0.000000,0.000000,0.000000}%
\pgfsetstrokecolor{currentstroke}%
\pgfsetdash{}{0pt}%
\pgfpathmoveto{\pgfqpoint{2.889236in}{1.886962in}}%
\pgfpathlineto{\pgfqpoint{2.902868in}{1.880013in}}%
\pgfpathlineto{\pgfqpoint{2.916503in}{1.873161in}}%
\pgfpathlineto{\pgfqpoint{2.930140in}{1.866406in}}%
\pgfpathlineto{\pgfqpoint{2.943780in}{1.859747in}}%
\pgfpathlineto{\pgfqpoint{2.952222in}{1.866662in}}%
\pgfpathlineto{\pgfqpoint{2.960654in}{1.873661in}}%
\pgfpathlineto{\pgfqpoint{2.969079in}{1.880742in}}%
\pgfpathlineto{\pgfqpoint{2.977495in}{1.887902in}}%
\pgfpathlineto{\pgfqpoint{2.963874in}{1.894356in}}%
\pgfpathlineto{\pgfqpoint{2.950257in}{1.900905in}}%
\pgfpathlineto{\pgfqpoint{2.936642in}{1.907551in}}%
\pgfpathlineto{\pgfqpoint{2.923030in}{1.914294in}}%
\pgfpathlineto{\pgfqpoint{2.914595in}{1.907332in}}%
\pgfpathlineto{\pgfqpoint{2.906151in}{1.900454in}}%
\pgfpathlineto{\pgfqpoint{2.897698in}{1.893663in}}%
\pgfpathlineto{\pgfqpoint{2.889236in}{1.886962in}}%
\pgfpathclose%
\pgfusepath{fill}%
\end{pgfscope}%
\begin{pgfscope}%
\pgfpathrectangle{\pgfqpoint{1.150000in}{0.150000in}}{\pgfqpoint{5.700000in}{5.700000in}}%
\pgfusepath{clip}%
\pgfsetbuttcap%
\pgfsetroundjoin%
\definecolor{currentfill}{rgb}{0.227802,0.326594,0.546532}%
\pgfsetfillcolor{currentfill}%
\pgfsetfillopacity{0.700000}%
\pgfsetlinewidth{0.000000pt}%
\definecolor{currentstroke}{rgb}{0.000000,0.000000,0.000000}%
\pgfsetstrokecolor{currentstroke}%
\pgfsetdash{}{0pt}%
\pgfpathmoveto{\pgfqpoint{5.426644in}{2.485299in}}%
\pgfpathlineto{\pgfqpoint{5.440945in}{2.487509in}}%
\pgfpathlineto{\pgfqpoint{5.455259in}{2.489788in}}%
\pgfpathlineto{\pgfqpoint{5.469584in}{2.492135in}}%
\pgfpathlineto{\pgfqpoint{5.483920in}{2.494551in}}%
\pgfpathlineto{\pgfqpoint{5.491304in}{2.499343in}}%
\pgfpathlineto{\pgfqpoint{5.498681in}{2.504162in}}%
\pgfpathlineto{\pgfqpoint{5.506053in}{2.509012in}}%
\pgfpathlineto{\pgfqpoint{5.513420in}{2.513900in}}%
\pgfpathlineto{\pgfqpoint{5.499106in}{2.511780in}}%
\pgfpathlineto{\pgfqpoint{5.484803in}{2.509729in}}%
\pgfpathlineto{\pgfqpoint{5.470512in}{2.507746in}}%
\pgfpathlineto{\pgfqpoint{5.456233in}{2.505830in}}%
\pgfpathlineto{\pgfqpoint{5.448843in}{2.500640in}}%
\pgfpathlineto{\pgfqpoint{5.441449in}{2.495492in}}%
\pgfpathlineto{\pgfqpoint{5.434049in}{2.490380in}}%
\pgfpathlineto{\pgfqpoint{5.426644in}{2.485299in}}%
\pgfpathclose%
\pgfusepath{fill}%
\end{pgfscope}%
\begin{pgfscope}%
\pgfpathrectangle{\pgfqpoint{1.150000in}{0.150000in}}{\pgfqpoint{5.700000in}{5.700000in}}%
\pgfusepath{clip}%
\pgfsetbuttcap%
\pgfsetroundjoin%
\definecolor{currentfill}{rgb}{0.269944,0.014625,0.341379}%
\pgfsetfillcolor{currentfill}%
\pgfsetfillopacity{0.700000}%
\pgfsetlinewidth{0.000000pt}%
\definecolor{currentstroke}{rgb}{0.000000,0.000000,0.000000}%
\pgfsetstrokecolor{currentstroke}%
\pgfsetdash{}{0pt}%
\pgfpathmoveto{\pgfqpoint{3.174561in}{1.849349in}}%
\pgfpathlineto{\pgfqpoint{3.188210in}{1.844281in}}%
\pgfpathlineto{\pgfqpoint{3.201862in}{1.839302in}}%
\pgfpathlineto{\pgfqpoint{3.215519in}{1.834412in}}%
\pgfpathlineto{\pgfqpoint{3.229181in}{1.829610in}}%
\pgfpathlineto{\pgfqpoint{3.237489in}{1.837812in}}%
\pgfpathlineto{\pgfqpoint{3.245789in}{1.846057in}}%
\pgfpathlineto{\pgfqpoint{3.254084in}{1.854343in}}%
\pgfpathlineto{\pgfqpoint{3.262371in}{1.862668in}}%
\pgfpathlineto{\pgfqpoint{3.248725in}{1.867306in}}%
\pgfpathlineto{\pgfqpoint{3.235084in}{1.872032in}}%
\pgfpathlineto{\pgfqpoint{3.221447in}{1.876848in}}%
\pgfpathlineto{\pgfqpoint{3.207815in}{1.881752in}}%
\pgfpathlineto{\pgfqpoint{3.199512in}{1.873583in}}%
\pgfpathlineto{\pgfqpoint{3.191202in}{1.865458in}}%
\pgfpathlineto{\pgfqpoint{3.182885in}{1.857380in}}%
\pgfpathlineto{\pgfqpoint{3.174561in}{1.849349in}}%
\pgfpathclose%
\pgfusepath{fill}%
\end{pgfscope}%
\begin{pgfscope}%
\pgfpathrectangle{\pgfqpoint{1.150000in}{0.150000in}}{\pgfqpoint{5.700000in}{5.700000in}}%
\pgfusepath{clip}%
\pgfsetbuttcap%
\pgfsetroundjoin%
\definecolor{currentfill}{rgb}{0.282327,0.094955,0.417331}%
\pgfsetfillcolor{currentfill}%
\pgfsetfillopacity{0.700000}%
\pgfsetlinewidth{0.000000pt}%
\definecolor{currentstroke}{rgb}{0.000000,0.000000,0.000000}%
\pgfsetstrokecolor{currentstroke}%
\pgfsetdash{}{0pt}%
\pgfpathmoveto{\pgfqpoint{3.951596in}{1.978600in}}%
\pgfpathlineto{\pgfqpoint{3.965397in}{1.977538in}}%
\pgfpathlineto{\pgfqpoint{3.979206in}{1.976553in}}%
\pgfpathlineto{\pgfqpoint{3.993022in}{1.975645in}}%
\pgfpathlineto{\pgfqpoint{4.006846in}{1.974813in}}%
\pgfpathlineto{\pgfqpoint{4.014864in}{1.983937in}}%
\pgfpathlineto{\pgfqpoint{4.022876in}{1.993021in}}%
\pgfpathlineto{\pgfqpoint{4.030882in}{2.002067in}}%
\pgfpathlineto{\pgfqpoint{4.038882in}{2.011075in}}%
\pgfpathlineto{\pgfqpoint{4.025069in}{2.011887in}}%
\pgfpathlineto{\pgfqpoint{4.011263in}{2.012775in}}%
\pgfpathlineto{\pgfqpoint{3.997465in}{2.013740in}}%
\pgfpathlineto{\pgfqpoint{3.983674in}{2.014782in}}%
\pgfpathlineto{\pgfqpoint{3.975663in}{2.005787in}}%
\pgfpathlineto{\pgfqpoint{3.967647in}{1.996758in}}%
\pgfpathlineto{\pgfqpoint{3.959624in}{1.987696in}}%
\pgfpathlineto{\pgfqpoint{3.951596in}{1.978600in}}%
\pgfpathclose%
\pgfusepath{fill}%
\end{pgfscope}%
\begin{pgfscope}%
\pgfpathrectangle{\pgfqpoint{1.150000in}{0.150000in}}{\pgfqpoint{5.700000in}{5.700000in}}%
\pgfusepath{clip}%
\pgfsetbuttcap%
\pgfsetroundjoin%
\definecolor{currentfill}{rgb}{0.278826,0.175490,0.483397}%
\pgfsetfillcolor{currentfill}%
\pgfsetfillopacity{0.700000}%
\pgfsetlinewidth{0.000000pt}%
\definecolor{currentstroke}{rgb}{0.000000,0.000000,0.000000}%
\pgfsetstrokecolor{currentstroke}%
\pgfsetdash{}{0pt}%
\pgfpathmoveto{\pgfqpoint{4.443298in}{2.147450in}}%
\pgfpathlineto{\pgfqpoint{4.457251in}{2.148074in}}%
\pgfpathlineto{\pgfqpoint{4.471213in}{2.148772in}}%
\pgfpathlineto{\pgfqpoint{4.485184in}{2.149542in}}%
\pgfpathlineto{\pgfqpoint{4.499166in}{2.150385in}}%
\pgfpathlineto{\pgfqpoint{4.507000in}{2.158397in}}%
\pgfpathlineto{\pgfqpoint{4.514829in}{2.166358in}}%
\pgfpathlineto{\pgfqpoint{4.522651in}{2.174269in}}%
\pgfpathlineto{\pgfqpoint{4.530468in}{2.182133in}}%
\pgfpathlineto{\pgfqpoint{4.516498in}{2.181374in}}%
\pgfpathlineto{\pgfqpoint{4.502539in}{2.180689in}}%
\pgfpathlineto{\pgfqpoint{4.488588in}{2.180075in}}%
\pgfpathlineto{\pgfqpoint{4.474647in}{2.179535in}}%
\pgfpathlineto{\pgfqpoint{4.466819in}{2.171579in}}%
\pgfpathlineto{\pgfqpoint{4.458985in}{2.163581in}}%
\pgfpathlineto{\pgfqpoint{4.451145in}{2.155538in}}%
\pgfpathlineto{\pgfqpoint{4.443298in}{2.147450in}}%
\pgfpathclose%
\pgfusepath{fill}%
\end{pgfscope}%
\begin{pgfscope}%
\pgfpathrectangle{\pgfqpoint{1.150000in}{0.150000in}}{\pgfqpoint{5.700000in}{5.700000in}}%
\pgfusepath{clip}%
\pgfsetbuttcap%
\pgfsetroundjoin%
\definecolor{currentfill}{rgb}{0.257322,0.256130,0.526563}%
\pgfsetfillcolor{currentfill}%
\pgfsetfillopacity{0.700000}%
\pgfsetlinewidth{0.000000pt}%
\definecolor{currentstroke}{rgb}{0.000000,0.000000,0.000000}%
\pgfsetstrokecolor{currentstroke}%
\pgfsetdash{}{0pt}%
\pgfpathmoveto{\pgfqpoint{4.935083in}{2.322664in}}%
\pgfpathlineto{\pgfqpoint{4.949210in}{2.324373in}}%
\pgfpathlineto{\pgfqpoint{4.963348in}{2.326152in}}%
\pgfpathlineto{\pgfqpoint{4.977496in}{2.328001in}}%
\pgfpathlineto{\pgfqpoint{4.991655in}{2.329921in}}%
\pgfpathlineto{\pgfqpoint{4.999278in}{2.336262in}}%
\pgfpathlineto{\pgfqpoint{5.006895in}{2.342574in}}%
\pgfpathlineto{\pgfqpoint{5.014505in}{2.348859in}}%
\pgfpathlineto{\pgfqpoint{5.022109in}{2.355121in}}%
\pgfpathlineto{\pgfqpoint{5.007966in}{2.353392in}}%
\pgfpathlineto{\pgfqpoint{4.993833in}{2.351733in}}%
\pgfpathlineto{\pgfqpoint{4.979712in}{2.350143in}}%
\pgfpathlineto{\pgfqpoint{4.965601in}{2.348624in}}%
\pgfpathlineto{\pgfqpoint{4.957981in}{2.342164in}}%
\pgfpathlineto{\pgfqpoint{4.950355in}{2.335686in}}%
\pgfpathlineto{\pgfqpoint{4.942722in}{2.329187in}}%
\pgfpathlineto{\pgfqpoint{4.935083in}{2.322664in}}%
\pgfpathclose%
\pgfusepath{fill}%
\end{pgfscope}%
\begin{pgfscope}%
\pgfpathrectangle{\pgfqpoint{1.150000in}{0.150000in}}{\pgfqpoint{5.700000in}{5.700000in}}%
\pgfusepath{clip}%
\pgfsetbuttcap%
\pgfsetroundjoin%
\definecolor{currentfill}{rgb}{0.282327,0.094955,0.417331}%
\pgfsetfillcolor{currentfill}%
\pgfsetfillopacity{0.700000}%
\pgfsetlinewidth{0.000000pt}%
\definecolor{currentstroke}{rgb}{0.000000,0.000000,0.000000}%
\pgfsetstrokecolor{currentstroke}%
\pgfsetdash{}{0pt}%
\pgfpathmoveto{\pgfqpoint{2.547992in}{2.006863in}}%
\pgfpathlineto{\pgfqpoint{2.561646in}{1.997353in}}%
\pgfpathlineto{\pgfqpoint{2.575299in}{1.987955in}}%
\pgfpathlineto{\pgfqpoint{2.588953in}{1.978668in}}%
\pgfpathlineto{\pgfqpoint{2.602607in}{1.969490in}}%
\pgfpathlineto{\pgfqpoint{2.611243in}{1.974332in}}%
\pgfpathlineto{\pgfqpoint{2.619868in}{1.979312in}}%
\pgfpathlineto{\pgfqpoint{2.628481in}{1.984425in}}%
\pgfpathlineto{\pgfqpoint{2.637084in}{1.989668in}}%
\pgfpathlineto{\pgfqpoint{2.623455in}{1.998596in}}%
\pgfpathlineto{\pgfqpoint{2.609827in}{2.007634in}}%
\pgfpathlineto{\pgfqpoint{2.596199in}{2.016782in}}%
\pgfpathlineto{\pgfqpoint{2.582572in}{2.026042in}}%
\pgfpathlineto{\pgfqpoint{2.573945in}{2.021040in}}%
\pgfpathlineto{\pgfqpoint{2.565306in}{2.016174in}}%
\pgfpathlineto{\pgfqpoint{2.556655in}{2.011447in}}%
\pgfpathlineto{\pgfqpoint{2.547992in}{2.006863in}}%
\pgfpathclose%
\pgfusepath{fill}%
\end{pgfscope}%
\begin{pgfscope}%
\pgfpathrectangle{\pgfqpoint{1.150000in}{0.150000in}}{\pgfqpoint{5.700000in}{5.700000in}}%
\pgfusepath{clip}%
\pgfsetbuttcap%
\pgfsetroundjoin%
\definecolor{currentfill}{rgb}{0.277018,0.050344,0.375715}%
\pgfsetfillcolor{currentfill}%
\pgfsetfillopacity{0.700000}%
\pgfsetlinewidth{0.000000pt}%
\definecolor{currentstroke}{rgb}{0.000000,0.000000,0.000000}%
\pgfsetstrokecolor{currentstroke}%
\pgfsetdash{}{0pt}%
\pgfpathmoveto{\pgfqpoint{2.746144in}{1.922096in}}%
\pgfpathlineto{\pgfqpoint{2.759782in}{1.914120in}}%
\pgfpathlineto{\pgfqpoint{2.773421in}{1.906246in}}%
\pgfpathlineto{\pgfqpoint{2.787063in}{1.898474in}}%
\pgfpathlineto{\pgfqpoint{2.800706in}{1.890804in}}%
\pgfpathlineto{\pgfqpoint{2.809227in}{1.896880in}}%
\pgfpathlineto{\pgfqpoint{2.817738in}{1.903064in}}%
\pgfpathlineto{\pgfqpoint{2.826239in}{1.909352in}}%
\pgfpathlineto{\pgfqpoint{2.834731in}{1.915741in}}%
\pgfpathlineto{\pgfqpoint{2.821110in}{1.923185in}}%
\pgfpathlineto{\pgfqpoint{2.807491in}{1.930730in}}%
\pgfpathlineto{\pgfqpoint{2.793874in}{1.938376in}}%
\pgfpathlineto{\pgfqpoint{2.780258in}{1.946125in}}%
\pgfpathlineto{\pgfqpoint{2.771744in}{1.939955in}}%
\pgfpathlineto{\pgfqpoint{2.763221in}{1.933891in}}%
\pgfpathlineto{\pgfqpoint{2.754687in}{1.927937in}}%
\pgfpathlineto{\pgfqpoint{2.746144in}{1.922096in}}%
\pgfpathclose%
\pgfusepath{fill}%
\end{pgfscope}%
\begin{pgfscope}%
\pgfpathrectangle{\pgfqpoint{1.150000in}{0.150000in}}{\pgfqpoint{5.700000in}{5.700000in}}%
\pgfusepath{clip}%
\pgfsetbuttcap%
\pgfsetroundjoin%
\definecolor{currentfill}{rgb}{0.279574,0.170599,0.479997}%
\pgfsetfillcolor{currentfill}%
\pgfsetfillopacity{0.700000}%
\pgfsetlinewidth{0.000000pt}%
\definecolor{currentstroke}{rgb}{0.000000,0.000000,0.000000}%
\pgfsetstrokecolor{currentstroke}%
\pgfsetdash{}{0pt}%
\pgfpathmoveto{\pgfqpoint{2.294212in}{2.162410in}}%
\pgfpathlineto{\pgfqpoint{2.307911in}{2.150721in}}%
\pgfpathlineto{\pgfqpoint{2.321608in}{2.139159in}}%
\pgfpathlineto{\pgfqpoint{2.335303in}{2.127723in}}%
\pgfpathlineto{\pgfqpoint{2.348996in}{2.116411in}}%
\pgfpathlineto{\pgfqpoint{2.357797in}{2.119545in}}%
\pgfpathlineto{\pgfqpoint{2.366585in}{2.122853in}}%
\pgfpathlineto{\pgfqpoint{2.375358in}{2.126331in}}%
\pgfpathlineto{\pgfqpoint{2.384118in}{2.129976in}}%
\pgfpathlineto{\pgfqpoint{2.370455in}{2.141014in}}%
\pgfpathlineto{\pgfqpoint{2.356791in}{2.152176in}}%
\pgfpathlineto{\pgfqpoint{2.343125in}{2.163463in}}%
\pgfpathlineto{\pgfqpoint{2.329457in}{2.174877in}}%
\pgfpathlineto{\pgfqpoint{2.320667in}{2.171498in}}%
\pgfpathlineto{\pgfqpoint{2.311863in}{2.168292in}}%
\pgfpathlineto{\pgfqpoint{2.303045in}{2.165261in}}%
\pgfpathlineto{\pgfqpoint{2.294212in}{2.162410in}}%
\pgfpathclose%
\pgfusepath{fill}%
\end{pgfscope}%
\begin{pgfscope}%
\pgfpathrectangle{\pgfqpoint{1.150000in}{0.150000in}}{\pgfqpoint{5.700000in}{5.700000in}}%
\pgfusepath{clip}%
\pgfsetbuttcap%
\pgfsetroundjoin%
\definecolor{currentfill}{rgb}{0.210503,0.363727,0.552206}%
\pgfsetfillcolor{currentfill}%
\pgfsetfillopacity{0.700000}%
\pgfsetlinewidth{0.000000pt}%
\definecolor{currentstroke}{rgb}{0.000000,0.000000,0.000000}%
\pgfsetstrokecolor{currentstroke}%
\pgfsetdash{}{0pt}%
\pgfpathmoveto{\pgfqpoint{5.744292in}{2.577873in}}%
\pgfpathlineto{\pgfqpoint{5.758711in}{2.580218in}}%
\pgfpathlineto{\pgfqpoint{5.773142in}{2.582631in}}%
\pgfpathlineto{\pgfqpoint{5.787585in}{2.585111in}}%
\pgfpathlineto{\pgfqpoint{5.802040in}{2.587659in}}%
\pgfpathlineto{\pgfqpoint{5.809265in}{2.591730in}}%
\pgfpathlineto{\pgfqpoint{5.816487in}{2.595876in}}%
\pgfpathlineto{\pgfqpoint{5.823704in}{2.600103in}}%
\pgfpathlineto{\pgfqpoint{5.830918in}{2.604418in}}%
\pgfpathlineto{\pgfqpoint{5.816490in}{2.602230in}}%
\pgfpathlineto{\pgfqpoint{5.802074in}{2.600108in}}%
\pgfpathlineto{\pgfqpoint{5.787670in}{2.598054in}}%
\pgfpathlineto{\pgfqpoint{5.773278in}{2.596067in}}%
\pgfpathlineto{\pgfqpoint{5.766037in}{2.591387in}}%
\pgfpathlineto{\pgfqpoint{5.758792in}{2.586798in}}%
\pgfpathlineto{\pgfqpoint{5.751544in}{2.582296in}}%
\pgfpathlineto{\pgfqpoint{5.744292in}{2.577873in}}%
\pgfpathclose%
\pgfusepath{fill}%
\end{pgfscope}%
\begin{pgfscope}%
\pgfpathrectangle{\pgfqpoint{1.150000in}{0.150000in}}{\pgfqpoint{5.700000in}{5.700000in}}%
\pgfusepath{clip}%
\pgfsetbuttcap%
\pgfsetroundjoin%
\definecolor{currentfill}{rgb}{0.280868,0.160771,0.472899}%
\pgfsetfillcolor{currentfill}%
\pgfsetfillopacity{0.700000}%
\pgfsetlinewidth{0.000000pt}%
\definecolor{currentstroke}{rgb}{0.000000,0.000000,0.000000}%
\pgfsetstrokecolor{currentstroke}%
\pgfsetdash{}{0pt}%
\pgfpathmoveto{\pgfqpoint{4.356088in}{2.112580in}}%
\pgfpathlineto{\pgfqpoint{4.370015in}{2.112976in}}%
\pgfpathlineto{\pgfqpoint{4.383951in}{2.113444in}}%
\pgfpathlineto{\pgfqpoint{4.397896in}{2.113986in}}%
\pgfpathlineto{\pgfqpoint{4.411851in}{2.114601in}}%
\pgfpathlineto{\pgfqpoint{4.419722in}{2.122890in}}%
\pgfpathlineto{\pgfqpoint{4.427587in}{2.131127in}}%
\pgfpathlineto{\pgfqpoint{4.435446in}{2.139313in}}%
\pgfpathlineto{\pgfqpoint{4.443298in}{2.147450in}}%
\pgfpathlineto{\pgfqpoint{4.429355in}{2.146898in}}%
\pgfpathlineto{\pgfqpoint{4.415421in}{2.146420in}}%
\pgfpathlineto{\pgfqpoint{4.401496in}{2.146015in}}%
\pgfpathlineto{\pgfqpoint{4.387580in}{2.145683in}}%
\pgfpathlineto{\pgfqpoint{4.379716in}{2.137475in}}%
\pgfpathlineto{\pgfqpoint{4.371846in}{2.129223in}}%
\pgfpathlineto{\pgfqpoint{4.363970in}{2.120925in}}%
\pgfpathlineto{\pgfqpoint{4.356088in}{2.112580in}}%
\pgfpathclose%
\pgfusepath{fill}%
\end{pgfscope}%
\begin{pgfscope}%
\pgfpathrectangle{\pgfqpoint{1.150000in}{0.150000in}}{\pgfqpoint{5.700000in}{5.700000in}}%
\pgfusepath{clip}%
\pgfsetbuttcap%
\pgfsetroundjoin%
\definecolor{currentfill}{rgb}{0.269944,0.014625,0.341379}%
\pgfsetfillcolor{currentfill}%
\pgfsetfillopacity{0.700000}%
\pgfsetlinewidth{0.000000pt}%
\definecolor{currentstroke}{rgb}{0.000000,0.000000,0.000000}%
\pgfsetstrokecolor{currentstroke}%
\pgfsetdash{}{0pt}%
\pgfpathmoveto{\pgfqpoint{3.317000in}{1.844987in}}%
\pgfpathlineto{\pgfqpoint{3.330669in}{1.840784in}}%
\pgfpathlineto{\pgfqpoint{3.344343in}{1.836666in}}%
\pgfpathlineto{\pgfqpoint{3.358022in}{1.832633in}}%
\pgfpathlineto{\pgfqpoint{3.371706in}{1.828686in}}%
\pgfpathlineto{\pgfqpoint{3.379957in}{1.837349in}}%
\pgfpathlineto{\pgfqpoint{3.388202in}{1.846037in}}%
\pgfpathlineto{\pgfqpoint{3.396440in}{1.854748in}}%
\pgfpathlineto{\pgfqpoint{3.404672in}{1.863479in}}%
\pgfpathlineto{\pgfqpoint{3.391002in}{1.867283in}}%
\pgfpathlineto{\pgfqpoint{3.377337in}{1.871173in}}%
\pgfpathlineto{\pgfqpoint{3.363677in}{1.875148in}}%
\pgfpathlineto{\pgfqpoint{3.350022in}{1.879208in}}%
\pgfpathlineto{\pgfqpoint{3.341776in}{1.870612in}}%
\pgfpathlineto{\pgfqpoint{3.333524in}{1.862042in}}%
\pgfpathlineto{\pgfqpoint{3.325265in}{1.853500in}}%
\pgfpathlineto{\pgfqpoint{3.317000in}{1.844987in}}%
\pgfpathclose%
\pgfusepath{fill}%
\end{pgfscope}%
\begin{pgfscope}%
\pgfpathrectangle{\pgfqpoint{1.150000in}{0.150000in}}{\pgfqpoint{5.700000in}{5.700000in}}%
\pgfusepath{clip}%
\pgfsetbuttcap%
\pgfsetroundjoin%
\definecolor{currentfill}{rgb}{0.233603,0.313828,0.543914}%
\pgfsetfillcolor{currentfill}%
\pgfsetfillopacity{0.700000}%
\pgfsetlinewidth{0.000000pt}%
\definecolor{currentstroke}{rgb}{0.000000,0.000000,0.000000}%
\pgfsetstrokecolor{currentstroke}%
\pgfsetdash{}{0pt}%
\pgfpathmoveto{\pgfqpoint{5.339790in}{2.455933in}}%
\pgfpathlineto{\pgfqpoint{5.354067in}{2.458144in}}%
\pgfpathlineto{\pgfqpoint{5.368355in}{2.460423in}}%
\pgfpathlineto{\pgfqpoint{5.382655in}{2.462772in}}%
\pgfpathlineto{\pgfqpoint{5.396966in}{2.465189in}}%
\pgfpathlineto{\pgfqpoint{5.404394in}{2.470195in}}%
\pgfpathlineto{\pgfqpoint{5.411817in}{2.475211in}}%
\pgfpathlineto{\pgfqpoint{5.419233in}{2.480245in}}%
\pgfpathlineto{\pgfqpoint{5.426644in}{2.485299in}}%
\pgfpathlineto{\pgfqpoint{5.412354in}{2.483157in}}%
\pgfpathlineto{\pgfqpoint{5.398075in}{2.481084in}}%
\pgfpathlineto{\pgfqpoint{5.383808in}{2.479078in}}%
\pgfpathlineto{\pgfqpoint{5.369552in}{2.477142in}}%
\pgfpathlineto{\pgfqpoint{5.362120in}{2.471805in}}%
\pgfpathlineto{\pgfqpoint{5.354682in}{2.466495in}}%
\pgfpathlineto{\pgfqpoint{5.347239in}{2.461206in}}%
\pgfpathlineto{\pgfqpoint{5.339790in}{2.455933in}}%
\pgfpathclose%
\pgfusepath{fill}%
\end{pgfscope}%
\begin{pgfscope}%
\pgfpathrectangle{\pgfqpoint{1.150000in}{0.150000in}}{\pgfqpoint{5.700000in}{5.700000in}}%
\pgfusepath{clip}%
\pgfsetbuttcap%
\pgfsetroundjoin%
\definecolor{currentfill}{rgb}{0.273809,0.031497,0.358853}%
\pgfsetfillcolor{currentfill}%
\pgfsetfillopacity{0.700000}%
\pgfsetlinewidth{0.000000pt}%
\definecolor{currentstroke}{rgb}{0.000000,0.000000,0.000000}%
\pgfsetstrokecolor{currentstroke}%
\pgfsetdash{}{0pt}%
\pgfpathmoveto{\pgfqpoint{3.546992in}{1.872147in}}%
\pgfpathlineto{\pgfqpoint{3.560700in}{1.869215in}}%
\pgfpathlineto{\pgfqpoint{3.574414in}{1.866365in}}%
\pgfpathlineto{\pgfqpoint{3.588134in}{1.863596in}}%
\pgfpathlineto{\pgfqpoint{3.601861in}{1.860909in}}%
\pgfpathlineto{\pgfqpoint{3.610024in}{1.870050in}}%
\pgfpathlineto{\pgfqpoint{3.618181in}{1.879188in}}%
\pgfpathlineto{\pgfqpoint{3.626332in}{1.888320in}}%
\pgfpathlineto{\pgfqpoint{3.634478in}{1.897447in}}%
\pgfpathlineto{\pgfqpoint{3.620764in}{1.900033in}}%
\pgfpathlineto{\pgfqpoint{3.607056in}{1.902699in}}%
\pgfpathlineto{\pgfqpoint{3.593354in}{1.905447in}}%
\pgfpathlineto{\pgfqpoint{3.579658in}{1.908276in}}%
\pgfpathlineto{\pgfqpoint{3.571500in}{1.899244in}}%
\pgfpathlineto{\pgfqpoint{3.563337in}{1.890211in}}%
\pgfpathlineto{\pgfqpoint{3.555167in}{1.881178in}}%
\pgfpathlineto{\pgfqpoint{3.546992in}{1.872147in}}%
\pgfpathclose%
\pgfusepath{fill}%
\end{pgfscope}%
\begin{pgfscope}%
\pgfpathrectangle{\pgfqpoint{1.150000in}{0.150000in}}{\pgfqpoint{5.700000in}{5.700000in}}%
\pgfusepath{clip}%
\pgfsetbuttcap%
\pgfsetroundjoin%
\definecolor{currentfill}{rgb}{0.280894,0.078907,0.402329}%
\pgfsetfillcolor{currentfill}%
\pgfsetfillopacity{0.700000}%
\pgfsetlinewidth{0.000000pt}%
\definecolor{currentstroke}{rgb}{0.000000,0.000000,0.000000}%
\pgfsetstrokecolor{currentstroke}%
\pgfsetdash{}{0pt}%
\pgfpathmoveto{\pgfqpoint{3.864258in}{1.947063in}}%
\pgfpathlineto{\pgfqpoint{3.878039in}{1.945651in}}%
\pgfpathlineto{\pgfqpoint{3.891828in}{1.944318in}}%
\pgfpathlineto{\pgfqpoint{3.905624in}{1.943061in}}%
\pgfpathlineto{\pgfqpoint{3.919428in}{1.941882in}}%
\pgfpathlineto{\pgfqpoint{3.927478in}{1.951112in}}%
\pgfpathlineto{\pgfqpoint{3.935523in}{1.960308in}}%
\pgfpathlineto{\pgfqpoint{3.943563in}{1.969471in}}%
\pgfpathlineto{\pgfqpoint{3.951596in}{1.978600in}}%
\pgfpathlineto{\pgfqpoint{3.937803in}{1.979739in}}%
\pgfpathlineto{\pgfqpoint{3.924017in}{1.980955in}}%
\pgfpathlineto{\pgfqpoint{3.910239in}{1.982248in}}%
\pgfpathlineto{\pgfqpoint{3.896469in}{1.983619in}}%
\pgfpathlineto{\pgfqpoint{3.888425in}{1.974522in}}%
\pgfpathlineto{\pgfqpoint{3.880375in}{1.965397in}}%
\pgfpathlineto{\pgfqpoint{3.872319in}{1.956244in}}%
\pgfpathlineto{\pgfqpoint{3.864258in}{1.947063in}}%
\pgfpathclose%
\pgfusepath{fill}%
\end{pgfscope}%
\begin{pgfscope}%
\pgfpathrectangle{\pgfqpoint{1.150000in}{0.150000in}}{\pgfqpoint{5.700000in}{5.700000in}}%
\pgfusepath{clip}%
\pgfsetbuttcap%
\pgfsetroundjoin%
\definecolor{currentfill}{rgb}{0.260571,0.246922,0.522828}%
\pgfsetfillcolor{currentfill}%
\pgfsetfillopacity{0.700000}%
\pgfsetlinewidth{0.000000pt}%
\definecolor{currentstroke}{rgb}{0.000000,0.000000,0.000000}%
\pgfsetstrokecolor{currentstroke}%
\pgfsetdash{}{0pt}%
\pgfpathmoveto{\pgfqpoint{4.847999in}{2.289449in}}%
\pgfpathlineto{\pgfqpoint{4.862099in}{2.291045in}}%
\pgfpathlineto{\pgfqpoint{4.876209in}{2.292711in}}%
\pgfpathlineto{\pgfqpoint{4.890330in}{2.294449in}}%
\pgfpathlineto{\pgfqpoint{4.904461in}{2.296257in}}%
\pgfpathlineto{\pgfqpoint{4.912127in}{2.302912in}}%
\pgfpathlineto{\pgfqpoint{4.919785in}{2.309529in}}%
\pgfpathlineto{\pgfqpoint{4.927438in}{2.316112in}}%
\pgfpathlineto{\pgfqpoint{4.935083in}{2.322664in}}%
\pgfpathlineto{\pgfqpoint{4.920967in}{2.321025in}}%
\pgfpathlineto{\pgfqpoint{4.906861in}{2.319457in}}%
\pgfpathlineto{\pgfqpoint{4.892766in}{2.317959in}}%
\pgfpathlineto{\pgfqpoint{4.878681in}{2.316532in}}%
\pgfpathlineto{\pgfqpoint{4.871020in}{2.309804in}}%
\pgfpathlineto{\pgfqpoint{4.863353in}{2.303050in}}%
\pgfpathlineto{\pgfqpoint{4.855679in}{2.296266in}}%
\pgfpathlineto{\pgfqpoint{4.847999in}{2.289449in}}%
\pgfpathclose%
\pgfusepath{fill}%
\end{pgfscope}%
\begin{pgfscope}%
\pgfpathrectangle{\pgfqpoint{1.150000in}{0.150000in}}{\pgfqpoint{5.700000in}{5.700000in}}%
\pgfusepath{clip}%
\pgfsetbuttcap%
\pgfsetroundjoin%
\definecolor{currentfill}{rgb}{0.282290,0.145912,0.461510}%
\pgfsetfillcolor{currentfill}%
\pgfsetfillopacity{0.700000}%
\pgfsetlinewidth{0.000000pt}%
\definecolor{currentstroke}{rgb}{0.000000,0.000000,0.000000}%
\pgfsetstrokecolor{currentstroke}%
\pgfsetdash{}{0pt}%
\pgfpathmoveto{\pgfqpoint{4.268837in}{2.077687in}}%
\pgfpathlineto{\pgfqpoint{4.282739in}{2.077830in}}%
\pgfpathlineto{\pgfqpoint{4.296650in}{2.078046in}}%
\pgfpathlineto{\pgfqpoint{4.310570in}{2.078337in}}%
\pgfpathlineto{\pgfqpoint{4.324499in}{2.078701in}}%
\pgfpathlineto{\pgfqpoint{4.332405in}{2.087248in}}%
\pgfpathlineto{\pgfqpoint{4.340305in}{2.095743in}}%
\pgfpathlineto{\pgfqpoint{4.348199in}{2.104187in}}%
\pgfpathlineto{\pgfqpoint{4.356088in}{2.112580in}}%
\pgfpathlineto{\pgfqpoint{4.342170in}{2.112259in}}%
\pgfpathlineto{\pgfqpoint{4.328261in}{2.112011in}}%
\pgfpathlineto{\pgfqpoint{4.314361in}{2.111837in}}%
\pgfpathlineto{\pgfqpoint{4.300469in}{2.111737in}}%
\pgfpathlineto{\pgfqpoint{4.292570in}{2.103293in}}%
\pgfpathlineto{\pgfqpoint{4.284665in}{2.094804in}}%
\pgfpathlineto{\pgfqpoint{4.276754in}{2.086269in}}%
\pgfpathlineto{\pgfqpoint{4.268837in}{2.077687in}}%
\pgfpathclose%
\pgfusepath{fill}%
\end{pgfscope}%
\begin{pgfscope}%
\pgfpathrectangle{\pgfqpoint{1.150000in}{0.150000in}}{\pgfqpoint{5.700000in}{5.700000in}}%
\pgfusepath{clip}%
\pgfsetbuttcap%
\pgfsetroundjoin%
\definecolor{currentfill}{rgb}{0.281887,0.150881,0.465405}%
\pgfsetfillcolor{currentfill}%
\pgfsetfillopacity{0.700000}%
\pgfsetlinewidth{0.000000pt}%
\definecolor{currentstroke}{rgb}{0.000000,0.000000,0.000000}%
\pgfsetstrokecolor{currentstroke}%
\pgfsetdash{}{0pt}%
\pgfpathmoveto{\pgfqpoint{2.348996in}{2.116411in}}%
\pgfpathlineto{\pgfqpoint{2.362687in}{2.105223in}}%
\pgfpathlineto{\pgfqpoint{2.376377in}{2.094158in}}%
\pgfpathlineto{\pgfqpoint{2.390066in}{2.083214in}}%
\pgfpathlineto{\pgfqpoint{2.403753in}{2.072390in}}%
\pgfpathlineto{\pgfqpoint{2.412524in}{2.075805in}}%
\pgfpathlineto{\pgfqpoint{2.421281in}{2.079389in}}%
\pgfpathlineto{\pgfqpoint{2.430025in}{2.083138in}}%
\pgfpathlineto{\pgfqpoint{2.438755in}{2.087048in}}%
\pgfpathlineto{\pgfqpoint{2.425098in}{2.097599in}}%
\pgfpathlineto{\pgfqpoint{2.411439in}{2.108270in}}%
\pgfpathlineto{\pgfqpoint{2.397779in}{2.119062in}}%
\pgfpathlineto{\pgfqpoint{2.384118in}{2.129976in}}%
\pgfpathlineto{\pgfqpoint{2.375358in}{2.126331in}}%
\pgfpathlineto{\pgfqpoint{2.366585in}{2.122853in}}%
\pgfpathlineto{\pgfqpoint{2.357797in}{2.119545in}}%
\pgfpathlineto{\pgfqpoint{2.348996in}{2.116411in}}%
\pgfpathclose%
\pgfusepath{fill}%
\end{pgfscope}%
\begin{pgfscope}%
\pgfpathrectangle{\pgfqpoint{1.150000in}{0.150000in}}{\pgfqpoint{5.700000in}{5.700000in}}%
\pgfusepath{clip}%
\pgfsetbuttcap%
\pgfsetroundjoin%
\definecolor{currentfill}{rgb}{0.237441,0.305202,0.541921}%
\pgfsetfillcolor{currentfill}%
\pgfsetfillopacity{0.700000}%
\pgfsetlinewidth{0.000000pt}%
\definecolor{currentstroke}{rgb}{0.000000,0.000000,0.000000}%
\pgfsetstrokecolor{currentstroke}%
\pgfsetdash{}{0pt}%
\pgfpathmoveto{\pgfqpoint{5.252861in}{2.425739in}}%
\pgfpathlineto{\pgfqpoint{5.267111in}{2.427929in}}%
\pgfpathlineto{\pgfqpoint{5.281374in}{2.430187in}}%
\pgfpathlineto{\pgfqpoint{5.295648in}{2.432515in}}%
\pgfpathlineto{\pgfqpoint{5.309933in}{2.434911in}}%
\pgfpathlineto{\pgfqpoint{5.317406in}{2.440165in}}%
\pgfpathlineto{\pgfqpoint{5.324874in}{2.445417in}}%
\pgfpathlineto{\pgfqpoint{5.332335in}{2.450671in}}%
\pgfpathlineto{\pgfqpoint{5.339790in}{2.455933in}}%
\pgfpathlineto{\pgfqpoint{5.325525in}{2.453790in}}%
\pgfpathlineto{\pgfqpoint{5.311271in}{2.451717in}}%
\pgfpathlineto{\pgfqpoint{5.297028in}{2.449712in}}%
\pgfpathlineto{\pgfqpoint{5.282797in}{2.447776in}}%
\pgfpathlineto{\pgfqpoint{5.275322in}{2.442254in}}%
\pgfpathlineto{\pgfqpoint{5.267841in}{2.436743in}}%
\pgfpathlineto{\pgfqpoint{5.260354in}{2.431240in}}%
\pgfpathlineto{\pgfqpoint{5.252861in}{2.425739in}}%
\pgfpathclose%
\pgfusepath{fill}%
\end{pgfscope}%
\begin{pgfscope}%
\pgfpathrectangle{\pgfqpoint{1.150000in}{0.150000in}}{\pgfqpoint{5.700000in}{5.700000in}}%
\pgfusepath{clip}%
\pgfsetbuttcap%
\pgfsetroundjoin%
\definecolor{currentfill}{rgb}{0.214298,0.355619,0.551184}%
\pgfsetfillcolor{currentfill}%
\pgfsetfillopacity{0.700000}%
\pgfsetlinewidth{0.000000pt}%
\definecolor{currentstroke}{rgb}{0.000000,0.000000,0.000000}%
\pgfsetstrokecolor{currentstroke}%
\pgfsetdash{}{0pt}%
\pgfpathmoveto{\pgfqpoint{5.657584in}{2.550796in}}%
\pgfpathlineto{\pgfqpoint{5.671981in}{2.553210in}}%
\pgfpathlineto{\pgfqpoint{5.686389in}{2.555691in}}%
\pgfpathlineto{\pgfqpoint{5.700810in}{2.558240in}}%
\pgfpathlineto{\pgfqpoint{5.715243in}{2.560857in}}%
\pgfpathlineto{\pgfqpoint{5.722512in}{2.565021in}}%
\pgfpathlineto{\pgfqpoint{5.729776in}{2.569242in}}%
\pgfpathlineto{\pgfqpoint{5.737036in}{2.573523in}}%
\pgfpathlineto{\pgfqpoint{5.744292in}{2.577873in}}%
\pgfpathlineto{\pgfqpoint{5.729885in}{2.575595in}}%
\pgfpathlineto{\pgfqpoint{5.715491in}{2.573384in}}%
\pgfpathlineto{\pgfqpoint{5.701108in}{2.571241in}}%
\pgfpathlineto{\pgfqpoint{5.686737in}{2.569165in}}%
\pgfpathlineto{\pgfqpoint{5.679455in}{2.564471in}}%
\pgfpathlineto{\pgfqpoint{5.672169in}{2.559848in}}%
\pgfpathlineto{\pgfqpoint{5.664879in}{2.555292in}}%
\pgfpathlineto{\pgfqpoint{5.657584in}{2.550796in}}%
\pgfpathclose%
\pgfusepath{fill}%
\end{pgfscope}%
\begin{pgfscope}%
\pgfpathrectangle{\pgfqpoint{1.150000in}{0.150000in}}{\pgfqpoint{5.700000in}{5.700000in}}%
\pgfusepath{clip}%
\pgfsetbuttcap%
\pgfsetroundjoin%
\definecolor{currentfill}{rgb}{0.279566,0.067836,0.391917}%
\pgfsetfillcolor{currentfill}%
\pgfsetfillopacity{0.700000}%
\pgfsetlinewidth{0.000000pt}%
\definecolor{currentstroke}{rgb}{0.000000,0.000000,0.000000}%
\pgfsetstrokecolor{currentstroke}%
\pgfsetdash{}{0pt}%
\pgfpathmoveto{\pgfqpoint{3.776862in}{1.916736in}}%
\pgfpathlineto{\pgfqpoint{3.790626in}{1.914950in}}%
\pgfpathlineto{\pgfqpoint{3.804396in}{1.913244in}}%
\pgfpathlineto{\pgfqpoint{3.818173in}{1.911615in}}%
\pgfpathlineto{\pgfqpoint{3.831958in}{1.910065in}}%
\pgfpathlineto{\pgfqpoint{3.840041in}{1.919354in}}%
\pgfpathlineto{\pgfqpoint{3.848119in}{1.928618in}}%
\pgfpathlineto{\pgfqpoint{3.856192in}{1.937854in}}%
\pgfpathlineto{\pgfqpoint{3.864258in}{1.947063in}}%
\pgfpathlineto{\pgfqpoint{3.850484in}{1.948552in}}%
\pgfpathlineto{\pgfqpoint{3.836718in}{1.950120in}}%
\pgfpathlineto{\pgfqpoint{3.822958in}{1.951765in}}%
\pgfpathlineto{\pgfqpoint{3.809206in}{1.953490in}}%
\pgfpathlineto{\pgfqpoint{3.801129in}{1.944334in}}%
\pgfpathlineto{\pgfqpoint{3.793046in}{1.935156in}}%
\pgfpathlineto{\pgfqpoint{3.784957in}{1.925957in}}%
\pgfpathlineto{\pgfqpoint{3.776862in}{1.916736in}}%
\pgfpathclose%
\pgfusepath{fill}%
\end{pgfscope}%
\begin{pgfscope}%
\pgfpathrectangle{\pgfqpoint{1.150000in}{0.150000in}}{\pgfqpoint{5.700000in}{5.700000in}}%
\pgfusepath{clip}%
\pgfsetbuttcap%
\pgfsetroundjoin%
\definecolor{currentfill}{rgb}{0.265145,0.232956,0.516599}%
\pgfsetfillcolor{currentfill}%
\pgfsetfillopacity{0.700000}%
\pgfsetlinewidth{0.000000pt}%
\definecolor{currentstroke}{rgb}{0.000000,0.000000,0.000000}%
\pgfsetstrokecolor{currentstroke}%
\pgfsetdash{}{0pt}%
\pgfpathmoveto{\pgfqpoint{4.760862in}{2.255529in}}%
\pgfpathlineto{\pgfqpoint{4.774934in}{2.256989in}}%
\pgfpathlineto{\pgfqpoint{4.789017in}{2.258520in}}%
\pgfpathlineto{\pgfqpoint{4.803110in}{2.260123in}}%
\pgfpathlineto{\pgfqpoint{4.817213in}{2.261796in}}%
\pgfpathlineto{\pgfqpoint{4.824920in}{2.268773in}}%
\pgfpathlineto{\pgfqpoint{4.832619in}{2.275706in}}%
\pgfpathlineto{\pgfqpoint{4.840313in}{2.282597in}}%
\pgfpathlineto{\pgfqpoint{4.847999in}{2.289449in}}%
\pgfpathlineto{\pgfqpoint{4.833910in}{2.287924in}}%
\pgfpathlineto{\pgfqpoint{4.819831in}{2.286470in}}%
\pgfpathlineto{\pgfqpoint{4.805762in}{2.285087in}}%
\pgfpathlineto{\pgfqpoint{4.791704in}{2.283775in}}%
\pgfpathlineto{\pgfqpoint{4.784003in}{2.276767in}}%
\pgfpathlineto{\pgfqpoint{4.776296in}{2.269725in}}%
\pgfpathlineto{\pgfqpoint{4.768582in}{2.262647in}}%
\pgfpathlineto{\pgfqpoint{4.760862in}{2.255529in}}%
\pgfpathclose%
\pgfusepath{fill}%
\end{pgfscope}%
\begin{pgfscope}%
\pgfpathrectangle{\pgfqpoint{1.150000in}{0.150000in}}{\pgfqpoint{5.700000in}{5.700000in}}%
\pgfusepath{clip}%
\pgfsetbuttcap%
\pgfsetroundjoin%
\definecolor{currentfill}{rgb}{0.272594,0.025563,0.353093}%
\pgfsetfillcolor{currentfill}%
\pgfsetfillopacity{0.700000}%
\pgfsetlinewidth{0.000000pt}%
\definecolor{currentstroke}{rgb}{0.000000,0.000000,0.000000}%
\pgfsetstrokecolor{currentstroke}%
\pgfsetdash{}{0pt}%
\pgfpathmoveto{\pgfqpoint{2.943780in}{1.859747in}}%
\pgfpathlineto{\pgfqpoint{2.957423in}{1.853184in}}%
\pgfpathlineto{\pgfqpoint{2.971068in}{1.846716in}}%
\pgfpathlineto{\pgfqpoint{2.984717in}{1.840342in}}%
\pgfpathlineto{\pgfqpoint{2.998368in}{1.834062in}}%
\pgfpathlineto{\pgfqpoint{3.006790in}{1.841189in}}%
\pgfpathlineto{\pgfqpoint{3.015204in}{1.848397in}}%
\pgfpathlineto{\pgfqpoint{3.023609in}{1.855680in}}%
\pgfpathlineto{\pgfqpoint{3.032006in}{1.863038in}}%
\pgfpathlineto{\pgfqpoint{3.018373in}{1.869113in}}%
\pgfpathlineto{\pgfqpoint{3.004744in}{1.875281in}}%
\pgfpathlineto{\pgfqpoint{2.991118in}{1.881544in}}%
\pgfpathlineto{\pgfqpoint{2.977495in}{1.887902in}}%
\pgfpathlineto{\pgfqpoint{2.969079in}{1.880742in}}%
\pgfpathlineto{\pgfqpoint{2.960654in}{1.873661in}}%
\pgfpathlineto{\pgfqpoint{2.952222in}{1.866662in}}%
\pgfpathlineto{\pgfqpoint{2.943780in}{1.859747in}}%
\pgfpathclose%
\pgfusepath{fill}%
\end{pgfscope}%
\begin{pgfscope}%
\pgfpathrectangle{\pgfqpoint{1.150000in}{0.150000in}}{\pgfqpoint{5.700000in}{5.700000in}}%
\pgfusepath{clip}%
\pgfsetbuttcap%
\pgfsetroundjoin%
\definecolor{currentfill}{rgb}{0.269944,0.014625,0.341379}%
\pgfsetfillcolor{currentfill}%
\pgfsetfillopacity{0.700000}%
\pgfsetlinewidth{0.000000pt}%
\definecolor{currentstroke}{rgb}{0.000000,0.000000,0.000000}%
\pgfsetstrokecolor{currentstroke}%
\pgfsetdash{}{0pt}%
\pgfpathmoveto{\pgfqpoint{3.086570in}{1.839667in}}%
\pgfpathlineto{\pgfqpoint{3.100220in}{1.834054in}}%
\pgfpathlineto{\pgfqpoint{3.113873in}{1.828532in}}%
\pgfpathlineto{\pgfqpoint{3.127531in}{1.823100in}}%
\pgfpathlineto{\pgfqpoint{3.141192in}{1.817758in}}%
\pgfpathlineto{\pgfqpoint{3.149545in}{1.825572in}}%
\pgfpathlineto{\pgfqpoint{3.157891in}{1.833443in}}%
\pgfpathlineto{\pgfqpoint{3.166230in}{1.841369in}}%
\pgfpathlineto{\pgfqpoint{3.174561in}{1.849349in}}%
\pgfpathlineto{\pgfqpoint{3.160917in}{1.854507in}}%
\pgfpathlineto{\pgfqpoint{3.147277in}{1.859754in}}%
\pgfpathlineto{\pgfqpoint{3.133640in}{1.865092in}}%
\pgfpathlineto{\pgfqpoint{3.120008in}{1.870520in}}%
\pgfpathlineto{\pgfqpoint{3.111660in}{1.862717in}}%
\pgfpathlineto{\pgfqpoint{3.103304in}{1.854972in}}%
\pgfpathlineto{\pgfqpoint{3.094941in}{1.847288in}}%
\pgfpathlineto{\pgfqpoint{3.086570in}{1.839667in}}%
\pgfpathclose%
\pgfusepath{fill}%
\end{pgfscope}%
\begin{pgfscope}%
\pgfpathrectangle{\pgfqpoint{1.150000in}{0.150000in}}{\pgfqpoint{5.700000in}{5.700000in}}%
\pgfusepath{clip}%
\pgfsetbuttcap%
\pgfsetroundjoin%
\definecolor{currentfill}{rgb}{0.281446,0.084320,0.407414}%
\pgfsetfillcolor{currentfill}%
\pgfsetfillopacity{0.700000}%
\pgfsetlinewidth{0.000000pt}%
\definecolor{currentstroke}{rgb}{0.000000,0.000000,0.000000}%
\pgfsetstrokecolor{currentstroke}%
\pgfsetdash{}{0pt}%
\pgfpathmoveto{\pgfqpoint{2.602607in}{1.969490in}}%
\pgfpathlineto{\pgfqpoint{2.616262in}{1.960421in}}%
\pgfpathlineto{\pgfqpoint{2.629917in}{1.951461in}}%
\pgfpathlineto{\pgfqpoint{2.643573in}{1.942608in}}%
\pgfpathlineto{\pgfqpoint{2.657229in}{1.933861in}}%
\pgfpathlineto{\pgfqpoint{2.665840in}{1.938960in}}%
\pgfpathlineto{\pgfqpoint{2.674439in}{1.944191in}}%
\pgfpathlineto{\pgfqpoint{2.683028in}{1.949551in}}%
\pgfpathlineto{\pgfqpoint{2.691605in}{1.955035in}}%
\pgfpathlineto{\pgfqpoint{2.677974in}{1.963533in}}%
\pgfpathlineto{\pgfqpoint{2.664343in}{1.972137in}}%
\pgfpathlineto{\pgfqpoint{2.650713in}{1.980849in}}%
\pgfpathlineto{\pgfqpoint{2.637084in}{1.989668in}}%
\pgfpathlineto{\pgfqpoint{2.628481in}{1.984425in}}%
\pgfpathlineto{\pgfqpoint{2.619868in}{1.979312in}}%
\pgfpathlineto{\pgfqpoint{2.611243in}{1.974332in}}%
\pgfpathlineto{\pgfqpoint{2.602607in}{1.969490in}}%
\pgfpathclose%
\pgfusepath{fill}%
\end{pgfscope}%
\begin{pgfscope}%
\pgfpathrectangle{\pgfqpoint{1.150000in}{0.150000in}}{\pgfqpoint{5.700000in}{5.700000in}}%
\pgfusepath{clip}%
\pgfsetbuttcap%
\pgfsetroundjoin%
\definecolor{currentfill}{rgb}{0.272594,0.025563,0.353093}%
\pgfsetfillcolor{currentfill}%
\pgfsetfillopacity{0.700000}%
\pgfsetlinewidth{0.000000pt}%
\definecolor{currentstroke}{rgb}{0.000000,0.000000,0.000000}%
\pgfsetstrokecolor{currentstroke}%
\pgfsetdash{}{0pt}%
\pgfpathmoveto{\pgfqpoint{3.459407in}{1.849105in}}%
\pgfpathlineto{\pgfqpoint{3.473104in}{1.845721in}}%
\pgfpathlineto{\pgfqpoint{3.486808in}{1.842419in}}%
\pgfpathlineto{\pgfqpoint{3.500516in}{1.839201in}}%
\pgfpathlineto{\pgfqpoint{3.514231in}{1.836064in}}%
\pgfpathlineto{\pgfqpoint{3.522430in}{1.845076in}}%
\pgfpathlineto{\pgfqpoint{3.530624in}{1.854094in}}%
\pgfpathlineto{\pgfqpoint{3.538811in}{1.863118in}}%
\pgfpathlineto{\pgfqpoint{3.546992in}{1.872147in}}%
\pgfpathlineto{\pgfqpoint{3.533290in}{1.875160in}}%
\pgfpathlineto{\pgfqpoint{3.519594in}{1.878256in}}%
\pgfpathlineto{\pgfqpoint{3.505904in}{1.881435in}}%
\pgfpathlineto{\pgfqpoint{3.492219in}{1.884697in}}%
\pgfpathlineto{\pgfqpoint{3.484025in}{1.875784in}}%
\pgfpathlineto{\pgfqpoint{3.475825in}{1.866879in}}%
\pgfpathlineto{\pgfqpoint{3.467619in}{1.857986in}}%
\pgfpathlineto{\pgfqpoint{3.459407in}{1.849105in}}%
\pgfpathclose%
\pgfusepath{fill}%
\end{pgfscope}%
\begin{pgfscope}%
\pgfpathrectangle{\pgfqpoint{1.150000in}{0.150000in}}{\pgfqpoint{5.700000in}{5.700000in}}%
\pgfusepath{clip}%
\pgfsetbuttcap%
\pgfsetroundjoin%
\definecolor{currentfill}{rgb}{0.283072,0.130895,0.449241}%
\pgfsetfillcolor{currentfill}%
\pgfsetfillopacity{0.700000}%
\pgfsetlinewidth{0.000000pt}%
\definecolor{currentstroke}{rgb}{0.000000,0.000000,0.000000}%
\pgfsetstrokecolor{currentstroke}%
\pgfsetdash{}{0pt}%
\pgfpathmoveto{\pgfqpoint{4.181548in}{2.042954in}}%
\pgfpathlineto{\pgfqpoint{4.195425in}{2.042821in}}%
\pgfpathlineto{\pgfqpoint{4.209312in}{2.042762in}}%
\pgfpathlineto{\pgfqpoint{4.223207in}{2.042778in}}%
\pgfpathlineto{\pgfqpoint{4.237110in}{2.042868in}}%
\pgfpathlineto{\pgfqpoint{4.245051in}{2.051648in}}%
\pgfpathlineto{\pgfqpoint{4.252986in}{2.060378in}}%
\pgfpathlineto{\pgfqpoint{4.260914in}{2.069057in}}%
\pgfpathlineto{\pgfqpoint{4.268837in}{2.077687in}}%
\pgfpathlineto{\pgfqpoint{4.254944in}{2.077619in}}%
\pgfpathlineto{\pgfqpoint{4.241060in}{2.077625in}}%
\pgfpathlineto{\pgfqpoint{4.227184in}{2.077706in}}%
\pgfpathlineto{\pgfqpoint{4.213317in}{2.077861in}}%
\pgfpathlineto{\pgfqpoint{4.205383in}{2.069201in}}%
\pgfpathlineto{\pgfqpoint{4.197444in}{2.060497in}}%
\pgfpathlineto{\pgfqpoint{4.189499in}{2.051749in}}%
\pgfpathlineto{\pgfqpoint{4.181548in}{2.042954in}}%
\pgfpathclose%
\pgfusepath{fill}%
\end{pgfscope}%
\begin{pgfscope}%
\pgfpathrectangle{\pgfqpoint{1.150000in}{0.150000in}}{\pgfqpoint{5.700000in}{5.700000in}}%
\pgfusepath{clip}%
\pgfsetbuttcap%
\pgfsetroundjoin%
\definecolor{currentfill}{rgb}{0.269944,0.014625,0.341379}%
\pgfsetfillcolor{currentfill}%
\pgfsetfillopacity{0.700000}%
\pgfsetlinewidth{0.000000pt}%
\definecolor{currentstroke}{rgb}{0.000000,0.000000,0.000000}%
\pgfsetstrokecolor{currentstroke}%
\pgfsetdash{}{0pt}%
\pgfpathmoveto{\pgfqpoint{3.229181in}{1.829610in}}%
\pgfpathlineto{\pgfqpoint{3.242846in}{1.824895in}}%
\pgfpathlineto{\pgfqpoint{3.256516in}{1.820268in}}%
\pgfpathlineto{\pgfqpoint{3.270191in}{1.815728in}}%
\pgfpathlineto{\pgfqpoint{3.283871in}{1.811275in}}%
\pgfpathlineto{\pgfqpoint{3.292163in}{1.819648in}}%
\pgfpathlineto{\pgfqpoint{3.300449in}{1.828059in}}%
\pgfpathlineto{\pgfqpoint{3.308727in}{1.836507in}}%
\pgfpathlineto{\pgfqpoint{3.317000in}{1.844987in}}%
\pgfpathlineto{\pgfqpoint{3.303335in}{1.849277in}}%
\pgfpathlineto{\pgfqpoint{3.289676in}{1.853654in}}%
\pgfpathlineto{\pgfqpoint{3.276021in}{1.858117in}}%
\pgfpathlineto{\pgfqpoint{3.262371in}{1.862668in}}%
\pgfpathlineto{\pgfqpoint{3.254084in}{1.854343in}}%
\pgfpathlineto{\pgfqpoint{3.245789in}{1.846057in}}%
\pgfpathlineto{\pgfqpoint{3.237489in}{1.837812in}}%
\pgfpathlineto{\pgfqpoint{3.229181in}{1.829610in}}%
\pgfpathclose%
\pgfusepath{fill}%
\end{pgfscope}%
\begin{pgfscope}%
\pgfpathrectangle{\pgfqpoint{1.150000in}{0.150000in}}{\pgfqpoint{5.700000in}{5.700000in}}%
\pgfusepath{clip}%
\pgfsetbuttcap%
\pgfsetroundjoin%
\definecolor{currentfill}{rgb}{0.276022,0.044167,0.370164}%
\pgfsetfillcolor{currentfill}%
\pgfsetfillopacity{0.700000}%
\pgfsetlinewidth{0.000000pt}%
\definecolor{currentstroke}{rgb}{0.000000,0.000000,0.000000}%
\pgfsetstrokecolor{currentstroke}%
\pgfsetdash{}{0pt}%
\pgfpathmoveto{\pgfqpoint{2.800706in}{1.890804in}}%
\pgfpathlineto{\pgfqpoint{2.814351in}{1.883233in}}%
\pgfpathlineto{\pgfqpoint{2.827998in}{1.875762in}}%
\pgfpathlineto{\pgfqpoint{2.841647in}{1.868390in}}%
\pgfpathlineto{\pgfqpoint{2.855298in}{1.861117in}}%
\pgfpathlineto{\pgfqpoint{2.863796in}{1.867428in}}%
\pgfpathlineto{\pgfqpoint{2.872286in}{1.873841in}}%
\pgfpathlineto{\pgfqpoint{2.880766in}{1.880353in}}%
\pgfpathlineto{\pgfqpoint{2.889236in}{1.886962in}}%
\pgfpathlineto{\pgfqpoint{2.875607in}{1.894008in}}%
\pgfpathlineto{\pgfqpoint{2.861979in}{1.901153in}}%
\pgfpathlineto{\pgfqpoint{2.848354in}{1.908397in}}%
\pgfpathlineto{\pgfqpoint{2.834731in}{1.915741in}}%
\pgfpathlineto{\pgfqpoint{2.826239in}{1.909352in}}%
\pgfpathlineto{\pgfqpoint{2.817738in}{1.903064in}}%
\pgfpathlineto{\pgfqpoint{2.809227in}{1.896880in}}%
\pgfpathlineto{\pgfqpoint{2.800706in}{1.890804in}}%
\pgfpathclose%
\pgfusepath{fill}%
\end{pgfscope}%
\begin{pgfscope}%
\pgfpathrectangle{\pgfqpoint{1.150000in}{0.150000in}}{\pgfqpoint{5.700000in}{5.700000in}}%
\pgfusepath{clip}%
\pgfsetbuttcap%
\pgfsetroundjoin%
\definecolor{currentfill}{rgb}{0.269308,0.218818,0.509577}%
\pgfsetfillcolor{currentfill}%
\pgfsetfillopacity{0.700000}%
\pgfsetlinewidth{0.000000pt}%
\definecolor{currentstroke}{rgb}{0.000000,0.000000,0.000000}%
\pgfsetstrokecolor{currentstroke}%
\pgfsetdash{}{0pt}%
\pgfpathmoveto{\pgfqpoint{4.673674in}{2.220978in}}%
\pgfpathlineto{\pgfqpoint{4.687719in}{2.222279in}}%
\pgfpathlineto{\pgfqpoint{4.701774in}{2.223652in}}%
\pgfpathlineto{\pgfqpoint{4.715839in}{2.225097in}}%
\pgfpathlineto{\pgfqpoint{4.729915in}{2.226613in}}%
\pgfpathlineto{\pgfqpoint{4.737661in}{2.233914in}}%
\pgfpathlineto{\pgfqpoint{4.745401in}{2.241165in}}%
\pgfpathlineto{\pgfqpoint{4.753135in}{2.248370in}}%
\pgfpathlineto{\pgfqpoint{4.760862in}{2.255529in}}%
\pgfpathlineto{\pgfqpoint{4.746799in}{2.254141in}}%
\pgfpathlineto{\pgfqpoint{4.732747in}{2.252823in}}%
\pgfpathlineto{\pgfqpoint{4.718706in}{2.251577in}}%
\pgfpathlineto{\pgfqpoint{4.704674in}{2.250402in}}%
\pgfpathlineto{\pgfqpoint{4.696933in}{2.243108in}}%
\pgfpathlineto{\pgfqpoint{4.689187in}{2.235774in}}%
\pgfpathlineto{\pgfqpoint{4.681434in}{2.228398in}}%
\pgfpathlineto{\pgfqpoint{4.673674in}{2.220978in}}%
\pgfpathclose%
\pgfusepath{fill}%
\end{pgfscope}%
\begin{pgfscope}%
\pgfpathrectangle{\pgfqpoint{1.150000in}{0.150000in}}{\pgfqpoint{5.700000in}{5.700000in}}%
\pgfusepath{clip}%
\pgfsetbuttcap%
\pgfsetroundjoin%
\definecolor{currentfill}{rgb}{0.241237,0.296485,0.539709}%
\pgfsetfillcolor{currentfill}%
\pgfsetfillopacity{0.700000}%
\pgfsetlinewidth{0.000000pt}%
\definecolor{currentstroke}{rgb}{0.000000,0.000000,0.000000}%
\pgfsetstrokecolor{currentstroke}%
\pgfsetdash{}{0pt}%
\pgfpathmoveto{\pgfqpoint{5.165859in}{2.394680in}}%
\pgfpathlineto{\pgfqpoint{5.180083in}{2.396825in}}%
\pgfpathlineto{\pgfqpoint{5.194319in}{2.399040in}}%
\pgfpathlineto{\pgfqpoint{5.208566in}{2.401325in}}%
\pgfpathlineto{\pgfqpoint{5.222825in}{2.403678in}}%
\pgfpathlineto{\pgfqpoint{5.230343in}{2.409211in}}%
\pgfpathlineto{\pgfqpoint{5.237855in}{2.414729in}}%
\pgfpathlineto{\pgfqpoint{5.245361in}{2.420237in}}%
\pgfpathlineto{\pgfqpoint{5.252861in}{2.425739in}}%
\pgfpathlineto{\pgfqpoint{5.238621in}{2.423619in}}%
\pgfpathlineto{\pgfqpoint{5.224392in}{2.421568in}}%
\pgfpathlineto{\pgfqpoint{5.210175in}{2.419586in}}%
\pgfpathlineto{\pgfqpoint{5.195969in}{2.417673in}}%
\pgfpathlineto{\pgfqpoint{5.188451in}{2.411931in}}%
\pgfpathlineto{\pgfqpoint{5.180927in}{2.406187in}}%
\pgfpathlineto{\pgfqpoint{5.173396in}{2.400438in}}%
\pgfpathlineto{\pgfqpoint{5.165859in}{2.394680in}}%
\pgfpathclose%
\pgfusepath{fill}%
\end{pgfscope}%
\begin{pgfscope}%
\pgfpathrectangle{\pgfqpoint{1.150000in}{0.150000in}}{\pgfqpoint{5.700000in}{5.700000in}}%
\pgfusepath{clip}%
\pgfsetbuttcap%
\pgfsetroundjoin%
\definecolor{currentfill}{rgb}{0.277018,0.050344,0.375715}%
\pgfsetfillcolor{currentfill}%
\pgfsetfillopacity{0.700000}%
\pgfsetlinewidth{0.000000pt}%
\definecolor{currentstroke}{rgb}{0.000000,0.000000,0.000000}%
\pgfsetstrokecolor{currentstroke}%
\pgfsetdash{}{0pt}%
\pgfpathmoveto{\pgfqpoint{3.689400in}{1.887912in}}%
\pgfpathlineto{\pgfqpoint{3.703147in}{1.885729in}}%
\pgfpathlineto{\pgfqpoint{3.716901in}{1.883625in}}%
\pgfpathlineto{\pgfqpoint{3.730661in}{1.881600in}}%
\pgfpathlineto{\pgfqpoint{3.744428in}{1.879654in}}%
\pgfpathlineto{\pgfqpoint{3.752545in}{1.888953in}}%
\pgfpathlineto{\pgfqpoint{3.760657in}{1.898233in}}%
\pgfpathlineto{\pgfqpoint{3.768762in}{1.907494in}}%
\pgfpathlineto{\pgfqpoint{3.776862in}{1.916736in}}%
\pgfpathlineto{\pgfqpoint{3.763106in}{1.918600in}}%
\pgfpathlineto{\pgfqpoint{3.749357in}{1.920543in}}%
\pgfpathlineto{\pgfqpoint{3.735615in}{1.922565in}}%
\pgfpathlineto{\pgfqpoint{3.721879in}{1.924667in}}%
\pgfpathlineto{\pgfqpoint{3.713768in}{1.915500in}}%
\pgfpathlineto{\pgfqpoint{3.705651in}{1.906318in}}%
\pgfpathlineto{\pgfqpoint{3.697528in}{1.897122in}}%
\pgfpathlineto{\pgfqpoint{3.689400in}{1.887912in}}%
\pgfpathclose%
\pgfusepath{fill}%
\end{pgfscope}%
\begin{pgfscope}%
\pgfpathrectangle{\pgfqpoint{1.150000in}{0.150000in}}{\pgfqpoint{5.700000in}{5.700000in}}%
\pgfusepath{clip}%
\pgfsetbuttcap%
\pgfsetroundjoin%
\definecolor{currentfill}{rgb}{0.283197,0.115680,0.436115}%
\pgfsetfillcolor{currentfill}%
\pgfsetfillopacity{0.700000}%
\pgfsetlinewidth{0.000000pt}%
\definecolor{currentstroke}{rgb}{0.000000,0.000000,0.000000}%
\pgfsetstrokecolor{currentstroke}%
\pgfsetdash{}{0pt}%
\pgfpathmoveto{\pgfqpoint{4.094218in}{2.008587in}}%
\pgfpathlineto{\pgfqpoint{4.108072in}{2.008155in}}%
\pgfpathlineto{\pgfqpoint{4.121935in}{2.007797in}}%
\pgfpathlineto{\pgfqpoint{4.135806in}{2.007515in}}%
\pgfpathlineto{\pgfqpoint{4.149686in}{2.007308in}}%
\pgfpathlineto{\pgfqpoint{4.157660in}{2.016291in}}%
\pgfpathlineto{\pgfqpoint{4.165628in}{2.025226in}}%
\pgfpathlineto{\pgfqpoint{4.173591in}{2.034114in}}%
\pgfpathlineto{\pgfqpoint{4.181548in}{2.042954in}}%
\pgfpathlineto{\pgfqpoint{4.167679in}{2.043163in}}%
\pgfpathlineto{\pgfqpoint{4.153818in}{2.043446in}}%
\pgfpathlineto{\pgfqpoint{4.139966in}{2.043804in}}%
\pgfpathlineto{\pgfqpoint{4.126122in}{2.044238in}}%
\pgfpathlineto{\pgfqpoint{4.118154in}{2.035389in}}%
\pgfpathlineto{\pgfqpoint{4.110181in}{2.026497in}}%
\pgfpathlineto{\pgfqpoint{4.102203in}{2.017564in}}%
\pgfpathlineto{\pgfqpoint{4.094218in}{2.008587in}}%
\pgfpathclose%
\pgfusepath{fill}%
\end{pgfscope}%
\begin{pgfscope}%
\pgfpathrectangle{\pgfqpoint{1.150000in}{0.150000in}}{\pgfqpoint{5.700000in}{5.700000in}}%
\pgfusepath{clip}%
\pgfsetbuttcap%
\pgfsetroundjoin%
\definecolor{currentfill}{rgb}{0.218130,0.347432,0.550038}%
\pgfsetfillcolor{currentfill}%
\pgfsetfillopacity{0.700000}%
\pgfsetlinewidth{0.000000pt}%
\definecolor{currentstroke}{rgb}{0.000000,0.000000,0.000000}%
\pgfsetstrokecolor{currentstroke}%
\pgfsetdash{}{0pt}%
\pgfpathmoveto{\pgfqpoint{5.570794in}{2.523058in}}%
\pgfpathlineto{\pgfqpoint{5.585167in}{2.525518in}}%
\pgfpathlineto{\pgfqpoint{5.599552in}{2.528046in}}%
\pgfpathlineto{\pgfqpoint{5.613949in}{2.530642in}}%
\pgfpathlineto{\pgfqpoint{5.628358in}{2.533306in}}%
\pgfpathlineto{\pgfqpoint{5.635672in}{2.537616in}}%
\pgfpathlineto{\pgfqpoint{5.642981in}{2.541964in}}%
\pgfpathlineto{\pgfqpoint{5.650285in}{2.546356in}}%
\pgfpathlineto{\pgfqpoint{5.657584in}{2.550796in}}%
\pgfpathlineto{\pgfqpoint{5.643200in}{2.548451in}}%
\pgfpathlineto{\pgfqpoint{5.628827in}{2.546173in}}%
\pgfpathlineto{\pgfqpoint{5.614467in}{2.543962in}}%
\pgfpathlineto{\pgfqpoint{5.600118in}{2.541820in}}%
\pgfpathlineto{\pgfqpoint{5.592794in}{2.537054in}}%
\pgfpathlineto{\pgfqpoint{5.585465in}{2.532342in}}%
\pgfpathlineto{\pgfqpoint{5.578132in}{2.527679in}}%
\pgfpathlineto{\pgfqpoint{5.570794in}{2.523058in}}%
\pgfpathclose%
\pgfusepath{fill}%
\end{pgfscope}%
\begin{pgfscope}%
\pgfpathrectangle{\pgfqpoint{1.150000in}{0.150000in}}{\pgfqpoint{5.700000in}{5.700000in}}%
\pgfusepath{clip}%
\pgfsetbuttcap%
\pgfsetroundjoin%
\definecolor{currentfill}{rgb}{0.282884,0.135920,0.453427}%
\pgfsetfillcolor{currentfill}%
\pgfsetfillopacity{0.700000}%
\pgfsetlinewidth{0.000000pt}%
\definecolor{currentstroke}{rgb}{0.000000,0.000000,0.000000}%
\pgfsetstrokecolor{currentstroke}%
\pgfsetdash{}{0pt}%
\pgfpathmoveto{\pgfqpoint{2.403753in}{2.072390in}}%
\pgfpathlineto{\pgfqpoint{2.417439in}{2.061686in}}%
\pgfpathlineto{\pgfqpoint{2.431124in}{2.051101in}}%
\pgfpathlineto{\pgfqpoint{2.444808in}{2.040633in}}%
\pgfpathlineto{\pgfqpoint{2.458491in}{2.030282in}}%
\pgfpathlineto{\pgfqpoint{2.467232in}{2.033978in}}%
\pgfpathlineto{\pgfqpoint{2.475960in}{2.037837in}}%
\pgfpathlineto{\pgfqpoint{2.484675in}{2.041856in}}%
\pgfpathlineto{\pgfqpoint{2.493378in}{2.046030in}}%
\pgfpathlineto{\pgfqpoint{2.479723in}{2.056109in}}%
\pgfpathlineto{\pgfqpoint{2.466068in}{2.066304in}}%
\pgfpathlineto{\pgfqpoint{2.452412in}{2.076617in}}%
\pgfpathlineto{\pgfqpoint{2.438755in}{2.087048in}}%
\pgfpathlineto{\pgfqpoint{2.430025in}{2.083138in}}%
\pgfpathlineto{\pgfqpoint{2.421281in}{2.079389in}}%
\pgfpathlineto{\pgfqpoint{2.412524in}{2.075805in}}%
\pgfpathlineto{\pgfqpoint{2.403753in}{2.072390in}}%
\pgfpathclose%
\pgfusepath{fill}%
\end{pgfscope}%
\begin{pgfscope}%
\pgfpathrectangle{\pgfqpoint{1.150000in}{0.150000in}}{\pgfqpoint{5.700000in}{5.700000in}}%
\pgfusepath{clip}%
\pgfsetbuttcap%
\pgfsetroundjoin%
\definecolor{currentfill}{rgb}{0.273006,0.204520,0.501721}%
\pgfsetfillcolor{currentfill}%
\pgfsetfillopacity{0.700000}%
\pgfsetlinewidth{0.000000pt}%
\definecolor{currentstroke}{rgb}{0.000000,0.000000,0.000000}%
\pgfsetstrokecolor{currentstroke}%
\pgfsetdash{}{0pt}%
\pgfpathmoveto{\pgfqpoint{4.586441in}{2.185890in}}%
\pgfpathlineto{\pgfqpoint{4.600459in}{2.187010in}}%
\pgfpathlineto{\pgfqpoint{4.614486in}{2.188202in}}%
\pgfpathlineto{\pgfqpoint{4.628524in}{2.189466in}}%
\pgfpathlineto{\pgfqpoint{4.642571in}{2.190802in}}%
\pgfpathlineto{\pgfqpoint{4.650357in}{2.198425in}}%
\pgfpathlineto{\pgfqpoint{4.658136in}{2.205993in}}%
\pgfpathlineto{\pgfqpoint{4.665908in}{2.213510in}}%
\pgfpathlineto{\pgfqpoint{4.673674in}{2.220978in}}%
\pgfpathlineto{\pgfqpoint{4.659639in}{2.219748in}}%
\pgfpathlineto{\pgfqpoint{4.645614in}{2.218590in}}%
\pgfpathlineto{\pgfqpoint{4.631599in}{2.217504in}}%
\pgfpathlineto{\pgfqpoint{4.617594in}{2.216490in}}%
\pgfpathlineto{\pgfqpoint{4.609815in}{2.208908in}}%
\pgfpathlineto{\pgfqpoint{4.602030in}{2.201283in}}%
\pgfpathlineto{\pgfqpoint{4.594239in}{2.193611in}}%
\pgfpathlineto{\pgfqpoint{4.586441in}{2.185890in}}%
\pgfpathclose%
\pgfusepath{fill}%
\end{pgfscope}%
\begin{pgfscope}%
\pgfpathrectangle{\pgfqpoint{1.150000in}{0.150000in}}{\pgfqpoint{5.700000in}{5.700000in}}%
\pgfusepath{clip}%
\pgfsetbuttcap%
\pgfsetroundjoin%
\definecolor{currentfill}{rgb}{0.282656,0.100196,0.422160}%
\pgfsetfillcolor{currentfill}%
\pgfsetfillopacity{0.700000}%
\pgfsetlinewidth{0.000000pt}%
\definecolor{currentstroke}{rgb}{0.000000,0.000000,0.000000}%
\pgfsetstrokecolor{currentstroke}%
\pgfsetdash{}{0pt}%
\pgfpathmoveto{\pgfqpoint{4.006846in}{1.974813in}}%
\pgfpathlineto{\pgfqpoint{4.020678in}{1.974058in}}%
\pgfpathlineto{\pgfqpoint{4.034518in}{1.973378in}}%
\pgfpathlineto{\pgfqpoint{4.048367in}{1.972775in}}%
\pgfpathlineto{\pgfqpoint{4.062223in}{1.972247in}}%
\pgfpathlineto{\pgfqpoint{4.070230in}{1.981398in}}%
\pgfpathlineto{\pgfqpoint{4.078232in}{1.990504in}}%
\pgfpathlineto{\pgfqpoint{4.086228in}{1.999568in}}%
\pgfpathlineto{\pgfqpoint{4.094218in}{2.008587in}}%
\pgfpathlineto{\pgfqpoint{4.080372in}{2.009095in}}%
\pgfpathlineto{\pgfqpoint{4.066534in}{2.009679in}}%
\pgfpathlineto{\pgfqpoint{4.052704in}{2.010339in}}%
\pgfpathlineto{\pgfqpoint{4.038882in}{2.011075in}}%
\pgfpathlineto{\pgfqpoint{4.030882in}{2.002067in}}%
\pgfpathlineto{\pgfqpoint{4.022876in}{1.993021in}}%
\pgfpathlineto{\pgfqpoint{4.014864in}{1.983937in}}%
\pgfpathlineto{\pgfqpoint{4.006846in}{1.974813in}}%
\pgfpathclose%
\pgfusepath{fill}%
\end{pgfscope}%
\begin{pgfscope}%
\pgfpathrectangle{\pgfqpoint{1.150000in}{0.150000in}}{\pgfqpoint{5.700000in}{5.700000in}}%
\pgfusepath{clip}%
\pgfsetbuttcap%
\pgfsetroundjoin%
\definecolor{currentfill}{rgb}{0.271305,0.019942,0.347269}%
\pgfsetfillcolor{currentfill}%
\pgfsetfillopacity{0.700000}%
\pgfsetlinewidth{0.000000pt}%
\definecolor{currentstroke}{rgb}{0.000000,0.000000,0.000000}%
\pgfsetstrokecolor{currentstroke}%
\pgfsetdash{}{0pt}%
\pgfpathmoveto{\pgfqpoint{3.371706in}{1.828686in}}%
\pgfpathlineto{\pgfqpoint{3.385396in}{1.824823in}}%
\pgfpathlineto{\pgfqpoint{3.399090in}{1.821045in}}%
\pgfpathlineto{\pgfqpoint{3.412790in}{1.817350in}}%
\pgfpathlineto{\pgfqpoint{3.426496in}{1.813740in}}%
\pgfpathlineto{\pgfqpoint{3.434733in}{1.822554in}}%
\pgfpathlineto{\pgfqpoint{3.442964in}{1.831387in}}%
\pgfpathlineto{\pgfqpoint{3.451189in}{1.840238in}}%
\pgfpathlineto{\pgfqpoint{3.459407in}{1.849105in}}%
\pgfpathlineto{\pgfqpoint{3.445715in}{1.852573in}}%
\pgfpathlineto{\pgfqpoint{3.432029in}{1.856124in}}%
\pgfpathlineto{\pgfqpoint{3.418348in}{1.859760in}}%
\pgfpathlineto{\pgfqpoint{3.404672in}{1.863479in}}%
\pgfpathlineto{\pgfqpoint{3.396440in}{1.854748in}}%
\pgfpathlineto{\pgfqpoint{3.388202in}{1.846037in}}%
\pgfpathlineto{\pgfqpoint{3.379957in}{1.837349in}}%
\pgfpathlineto{\pgfqpoint{3.371706in}{1.828686in}}%
\pgfpathclose%
\pgfusepath{fill}%
\end{pgfscope}%
\begin{pgfscope}%
\pgfpathrectangle{\pgfqpoint{1.150000in}{0.150000in}}{\pgfqpoint{5.700000in}{5.700000in}}%
\pgfusepath{clip}%
\pgfsetbuttcap%
\pgfsetroundjoin%
\definecolor{currentfill}{rgb}{0.246811,0.283237,0.535941}%
\pgfsetfillcolor{currentfill}%
\pgfsetfillopacity{0.700000}%
\pgfsetlinewidth{0.000000pt}%
\definecolor{currentstroke}{rgb}{0.000000,0.000000,0.000000}%
\pgfsetstrokecolor{currentstroke}%
\pgfsetdash{}{0pt}%
\pgfpathmoveto{\pgfqpoint{5.078789in}{2.362739in}}%
\pgfpathlineto{\pgfqpoint{5.092986in}{2.364818in}}%
\pgfpathlineto{\pgfqpoint{5.107195in}{2.366966in}}%
\pgfpathlineto{\pgfqpoint{5.121415in}{2.369185in}}%
\pgfpathlineto{\pgfqpoint{5.135646in}{2.371473in}}%
\pgfpathlineto{\pgfqpoint{5.143209in}{2.377309in}}%
\pgfpathlineto{\pgfqpoint{5.150766in}{2.383119in}}%
\pgfpathlineto{\pgfqpoint{5.158316in}{2.388908in}}%
\pgfpathlineto{\pgfqpoint{5.165859in}{2.394680in}}%
\pgfpathlineto{\pgfqpoint{5.151646in}{2.392604in}}%
\pgfpathlineto{\pgfqpoint{5.137443in}{2.390598in}}%
\pgfpathlineto{\pgfqpoint{5.123252in}{2.388661in}}%
\pgfpathlineto{\pgfqpoint{5.109072in}{2.386794in}}%
\pgfpathlineto{\pgfqpoint{5.101511in}{2.380802in}}%
\pgfpathlineto{\pgfqpoint{5.093943in}{2.374799in}}%
\pgfpathlineto{\pgfqpoint{5.086369in}{2.368779in}}%
\pgfpathlineto{\pgfqpoint{5.078789in}{2.362739in}}%
\pgfpathclose%
\pgfusepath{fill}%
\end{pgfscope}%
\begin{pgfscope}%
\pgfpathrectangle{\pgfqpoint{1.150000in}{0.150000in}}{\pgfqpoint{5.700000in}{5.700000in}}%
\pgfusepath{clip}%
\pgfsetbuttcap%
\pgfsetroundjoin%
\definecolor{currentfill}{rgb}{0.279566,0.067836,0.391917}%
\pgfsetfillcolor{currentfill}%
\pgfsetfillopacity{0.700000}%
\pgfsetlinewidth{0.000000pt}%
\definecolor{currentstroke}{rgb}{0.000000,0.000000,0.000000}%
\pgfsetstrokecolor{currentstroke}%
\pgfsetdash{}{0pt}%
\pgfpathmoveto{\pgfqpoint{2.657229in}{1.933861in}}%
\pgfpathlineto{\pgfqpoint{2.670887in}{1.925221in}}%
\pgfpathlineto{\pgfqpoint{2.684545in}{1.916685in}}%
\pgfpathlineto{\pgfqpoint{2.698205in}{1.908255in}}%
\pgfpathlineto{\pgfqpoint{2.711865in}{1.899927in}}%
\pgfpathlineto{\pgfqpoint{2.720451in}{1.905283in}}%
\pgfpathlineto{\pgfqpoint{2.729026in}{1.910765in}}%
\pgfpathlineto{\pgfqpoint{2.737590in}{1.916370in}}%
\pgfpathlineto{\pgfqpoint{2.746144in}{1.922096in}}%
\pgfpathlineto{\pgfqpoint{2.732507in}{1.930174in}}%
\pgfpathlineto{\pgfqpoint{2.718872in}{1.938357in}}%
\pgfpathlineto{\pgfqpoint{2.705238in}{1.946644in}}%
\pgfpathlineto{\pgfqpoint{2.691605in}{1.955035in}}%
\pgfpathlineto{\pgfqpoint{2.683028in}{1.949551in}}%
\pgfpathlineto{\pgfqpoint{2.674439in}{1.944191in}}%
\pgfpathlineto{\pgfqpoint{2.665840in}{1.938960in}}%
\pgfpathlineto{\pgfqpoint{2.657229in}{1.933861in}}%
\pgfpathclose%
\pgfusepath{fill}%
\end{pgfscope}%
\begin{pgfscope}%
\pgfpathrectangle{\pgfqpoint{1.150000in}{0.150000in}}{\pgfqpoint{5.700000in}{5.700000in}}%
\pgfusepath{clip}%
\pgfsetbuttcap%
\pgfsetroundjoin%
\definecolor{currentfill}{rgb}{0.262138,0.242286,0.520837}%
\pgfsetfillcolor{currentfill}%
\pgfsetfillopacity{0.700000}%
\pgfsetlinewidth{0.000000pt}%
\definecolor{currentstroke}{rgb}{0.000000,0.000000,0.000000}%
\pgfsetstrokecolor{currentstroke}%
\pgfsetdash{}{0pt}%
\pgfpathmoveto{\pgfqpoint{2.093720in}{2.307028in}}%
\pgfpathlineto{\pgfqpoint{2.107490in}{2.293428in}}%
\pgfpathlineto{\pgfqpoint{2.121256in}{2.279970in}}%
\pgfpathlineto{\pgfqpoint{2.135019in}{2.266653in}}%
\pgfpathlineto{\pgfqpoint{2.148778in}{2.253473in}}%
\pgfpathlineto{\pgfqpoint{2.157741in}{2.254950in}}%
\pgfpathlineto{\pgfqpoint{2.166687in}{2.256636in}}%
\pgfpathlineto{\pgfqpoint{2.175617in}{2.258524in}}%
\pgfpathlineto{\pgfqpoint{2.184531in}{2.260612in}}%
\pgfpathlineto{\pgfqpoint{2.170807in}{2.273492in}}%
\pgfpathlineto{\pgfqpoint{2.157080in}{2.286510in}}%
\pgfpathlineto{\pgfqpoint{2.143349in}{2.299668in}}%
\pgfpathlineto{\pgfqpoint{2.129614in}{2.312967in}}%
\pgfpathlineto{\pgfqpoint{2.120666in}{2.311170in}}%
\pgfpathlineto{\pgfqpoint{2.111701in}{2.309579in}}%
\pgfpathlineto{\pgfqpoint{2.102719in}{2.308197in}}%
\pgfpathlineto{\pgfqpoint{2.093720in}{2.307028in}}%
\pgfpathclose%
\pgfusepath{fill}%
\end{pgfscope}%
\begin{pgfscope}%
\pgfpathrectangle{\pgfqpoint{1.150000in}{0.150000in}}{\pgfqpoint{5.700000in}{5.700000in}}%
\pgfusepath{clip}%
\pgfsetbuttcap%
\pgfsetroundjoin%
\definecolor{currentfill}{rgb}{0.274952,0.037752,0.364543}%
\pgfsetfillcolor{currentfill}%
\pgfsetfillopacity{0.700000}%
\pgfsetlinewidth{0.000000pt}%
\definecolor{currentstroke}{rgb}{0.000000,0.000000,0.000000}%
\pgfsetstrokecolor{currentstroke}%
\pgfsetdash{}{0pt}%
\pgfpathmoveto{\pgfqpoint{3.601861in}{1.860909in}}%
\pgfpathlineto{\pgfqpoint{3.615593in}{1.858303in}}%
\pgfpathlineto{\pgfqpoint{3.629332in}{1.855777in}}%
\pgfpathlineto{\pgfqpoint{3.643078in}{1.853331in}}%
\pgfpathlineto{\pgfqpoint{3.656830in}{1.850966in}}%
\pgfpathlineto{\pgfqpoint{3.664981in}{1.860217in}}%
\pgfpathlineto{\pgfqpoint{3.673126in}{1.869459in}}%
\pgfpathlineto{\pgfqpoint{3.681266in}{1.878691in}}%
\pgfpathlineto{\pgfqpoint{3.689400in}{1.887912in}}%
\pgfpathlineto{\pgfqpoint{3.675660in}{1.890176in}}%
\pgfpathlineto{\pgfqpoint{3.661926in}{1.892519in}}%
\pgfpathlineto{\pgfqpoint{3.648199in}{1.894943in}}%
\pgfpathlineto{\pgfqpoint{3.634478in}{1.897447in}}%
\pgfpathlineto{\pgfqpoint{3.626332in}{1.888320in}}%
\pgfpathlineto{\pgfqpoint{3.618181in}{1.879188in}}%
\pgfpathlineto{\pgfqpoint{3.610024in}{1.870050in}}%
\pgfpathlineto{\pgfqpoint{3.601861in}{1.860909in}}%
\pgfpathclose%
\pgfusepath{fill}%
\end{pgfscope}%
\begin{pgfscope}%
\pgfpathrectangle{\pgfqpoint{1.150000in}{0.150000in}}{\pgfqpoint{5.700000in}{5.700000in}}%
\pgfusepath{clip}%
\pgfsetbuttcap%
\pgfsetroundjoin%
\definecolor{currentfill}{rgb}{0.276194,0.190074,0.493001}%
\pgfsetfillcolor{currentfill}%
\pgfsetfillopacity{0.700000}%
\pgfsetlinewidth{0.000000pt}%
\definecolor{currentstroke}{rgb}{0.000000,0.000000,0.000000}%
\pgfsetstrokecolor{currentstroke}%
\pgfsetdash{}{0pt}%
\pgfpathmoveto{\pgfqpoint{4.499166in}{2.150385in}}%
\pgfpathlineto{\pgfqpoint{4.513156in}{2.151301in}}%
\pgfpathlineto{\pgfqpoint{4.527156in}{2.152289in}}%
\pgfpathlineto{\pgfqpoint{4.541166in}{2.153349in}}%
\pgfpathlineto{\pgfqpoint{4.555186in}{2.154482in}}%
\pgfpathlineto{\pgfqpoint{4.563009in}{2.162417in}}%
\pgfpathlineto{\pgfqpoint{4.570826in}{2.170295in}}%
\pgfpathlineto{\pgfqpoint{4.578637in}{2.178119in}}%
\pgfpathlineto{\pgfqpoint{4.586441in}{2.185890in}}%
\pgfpathlineto{\pgfqpoint{4.572433in}{2.184843in}}%
\pgfpathlineto{\pgfqpoint{4.558435in}{2.183867in}}%
\pgfpathlineto{\pgfqpoint{4.544446in}{2.182964in}}%
\pgfpathlineto{\pgfqpoint{4.530468in}{2.182133in}}%
\pgfpathlineto{\pgfqpoint{4.522651in}{2.174269in}}%
\pgfpathlineto{\pgfqpoint{4.514829in}{2.166358in}}%
\pgfpathlineto{\pgfqpoint{4.507000in}{2.158397in}}%
\pgfpathlineto{\pgfqpoint{4.499166in}{2.150385in}}%
\pgfpathclose%
\pgfusepath{fill}%
\end{pgfscope}%
\begin{pgfscope}%
\pgfpathrectangle{\pgfqpoint{1.150000in}{0.150000in}}{\pgfqpoint{5.700000in}{5.700000in}}%
\pgfusepath{clip}%
\pgfsetbuttcap%
\pgfsetroundjoin%
\definecolor{currentfill}{rgb}{0.221989,0.339161,0.548752}%
\pgfsetfillcolor{currentfill}%
\pgfsetfillopacity{0.700000}%
\pgfsetlinewidth{0.000000pt}%
\definecolor{currentstroke}{rgb}{0.000000,0.000000,0.000000}%
\pgfsetstrokecolor{currentstroke}%
\pgfsetdash{}{0pt}%
\pgfpathmoveto{\pgfqpoint{5.483920in}{2.494551in}}%
\pgfpathlineto{\pgfqpoint{5.498269in}{2.497035in}}%
\pgfpathlineto{\pgfqpoint{5.512629in}{2.499587in}}%
\pgfpathlineto{\pgfqpoint{5.527002in}{2.502208in}}%
\pgfpathlineto{\pgfqpoint{5.541386in}{2.504897in}}%
\pgfpathlineto{\pgfqpoint{5.548746in}{2.509400in}}%
\pgfpathlineto{\pgfqpoint{5.556101in}{2.513924in}}%
\pgfpathlineto{\pgfqpoint{5.563450in}{2.518475in}}%
\pgfpathlineto{\pgfqpoint{5.570794in}{2.523058in}}%
\pgfpathlineto{\pgfqpoint{5.556433in}{2.520667in}}%
\pgfpathlineto{\pgfqpoint{5.542083in}{2.518343in}}%
\pgfpathlineto{\pgfqpoint{5.527746in}{2.516087in}}%
\pgfpathlineto{\pgfqpoint{5.513420in}{2.513900in}}%
\pgfpathlineto{\pgfqpoint{5.506053in}{2.509012in}}%
\pgfpathlineto{\pgfqpoint{5.498681in}{2.504162in}}%
\pgfpathlineto{\pgfqpoint{5.491304in}{2.499343in}}%
\pgfpathlineto{\pgfqpoint{5.483920in}{2.494551in}}%
\pgfpathclose%
\pgfusepath{fill}%
\end{pgfscope}%
\begin{pgfscope}%
\pgfpathrectangle{\pgfqpoint{1.150000in}{0.150000in}}{\pgfqpoint{5.700000in}{5.700000in}}%
\pgfusepath{clip}%
\pgfsetbuttcap%
\pgfsetroundjoin%
\definecolor{currentfill}{rgb}{0.271305,0.019942,0.347269}%
\pgfsetfillcolor{currentfill}%
\pgfsetfillopacity{0.700000}%
\pgfsetlinewidth{0.000000pt}%
\definecolor{currentstroke}{rgb}{0.000000,0.000000,0.000000}%
\pgfsetstrokecolor{currentstroke}%
\pgfsetdash{}{0pt}%
\pgfpathmoveto{\pgfqpoint{2.998368in}{1.834062in}}%
\pgfpathlineto{\pgfqpoint{3.012023in}{1.827875in}}%
\pgfpathlineto{\pgfqpoint{3.025681in}{1.821781in}}%
\pgfpathlineto{\pgfqpoint{3.039342in}{1.815780in}}%
\pgfpathlineto{\pgfqpoint{3.053007in}{1.809870in}}%
\pgfpathlineto{\pgfqpoint{3.061410in}{1.817211in}}%
\pgfpathlineto{\pgfqpoint{3.069804in}{1.824626in}}%
\pgfpathlineto{\pgfqpoint{3.078191in}{1.832112in}}%
\pgfpathlineto{\pgfqpoint{3.086570in}{1.839667in}}%
\pgfpathlineto{\pgfqpoint{3.072924in}{1.845372in}}%
\pgfpathlineto{\pgfqpoint{3.059281in}{1.851168in}}%
\pgfpathlineto{\pgfqpoint{3.045642in}{1.857057in}}%
\pgfpathlineto{\pgfqpoint{3.032006in}{1.863038in}}%
\pgfpathlineto{\pgfqpoint{3.023609in}{1.855680in}}%
\pgfpathlineto{\pgfqpoint{3.015204in}{1.848397in}}%
\pgfpathlineto{\pgfqpoint{3.006790in}{1.841189in}}%
\pgfpathlineto{\pgfqpoint{2.998368in}{1.834062in}}%
\pgfpathclose%
\pgfusepath{fill}%
\end{pgfscope}%
\begin{pgfscope}%
\pgfpathrectangle{\pgfqpoint{1.150000in}{0.150000in}}{\pgfqpoint{5.700000in}{5.700000in}}%
\pgfusepath{clip}%
\pgfsetbuttcap%
\pgfsetroundjoin%
\definecolor{currentfill}{rgb}{0.283197,0.115680,0.436115}%
\pgfsetfillcolor{currentfill}%
\pgfsetfillopacity{0.700000}%
\pgfsetlinewidth{0.000000pt}%
\definecolor{currentstroke}{rgb}{0.000000,0.000000,0.000000}%
\pgfsetstrokecolor{currentstroke}%
\pgfsetdash{}{0pt}%
\pgfpathmoveto{\pgfqpoint{2.458491in}{2.030282in}}%
\pgfpathlineto{\pgfqpoint{2.472174in}{2.020047in}}%
\pgfpathlineto{\pgfqpoint{2.485856in}{2.009927in}}%
\pgfpathlineto{\pgfqpoint{2.499538in}{1.999921in}}%
\pgfpathlineto{\pgfqpoint{2.513219in}{1.990028in}}%
\pgfpathlineto{\pgfqpoint{2.521931in}{1.994003in}}%
\pgfpathlineto{\pgfqpoint{2.530631in}{1.998137in}}%
\pgfpathlineto{\pgfqpoint{2.539318in}{2.002425in}}%
\pgfpathlineto{\pgfqpoint{2.547992in}{2.006863in}}%
\pgfpathlineto{\pgfqpoint{2.534339in}{2.016484in}}%
\pgfpathlineto{\pgfqpoint{2.520685in}{2.026219in}}%
\pgfpathlineto{\pgfqpoint{2.507032in}{2.036067in}}%
\pgfpathlineto{\pgfqpoint{2.493378in}{2.046030in}}%
\pgfpathlineto{\pgfqpoint{2.484675in}{2.041856in}}%
\pgfpathlineto{\pgfqpoint{2.475960in}{2.037837in}}%
\pgfpathlineto{\pgfqpoint{2.467232in}{2.033978in}}%
\pgfpathlineto{\pgfqpoint{2.458491in}{2.030282in}}%
\pgfpathclose%
\pgfusepath{fill}%
\end{pgfscope}%
\begin{pgfscope}%
\pgfpathrectangle{\pgfqpoint{1.150000in}{0.150000in}}{\pgfqpoint{5.700000in}{5.700000in}}%
\pgfusepath{clip}%
\pgfsetbuttcap%
\pgfsetroundjoin%
\definecolor{currentfill}{rgb}{0.204903,0.375746,0.553533}%
\pgfsetfillcolor{currentfill}%
\pgfsetfillopacity{0.700000}%
\pgfsetlinewidth{0.000000pt}%
\definecolor{currentstroke}{rgb}{0.000000,0.000000,0.000000}%
\pgfsetstrokecolor{currentstroke}%
\pgfsetdash{}{0pt}%
\pgfpathmoveto{\pgfqpoint{5.802040in}{2.587659in}}%
\pgfpathlineto{\pgfqpoint{5.816508in}{2.590274in}}%
\pgfpathlineto{\pgfqpoint{5.830988in}{2.592956in}}%
\pgfpathlineto{\pgfqpoint{5.845481in}{2.595706in}}%
\pgfpathlineto{\pgfqpoint{5.852685in}{2.599512in}}%
\pgfpathlineto{\pgfqpoint{5.859885in}{2.603390in}}%
\pgfpathlineto{\pgfqpoint{5.867082in}{2.607346in}}%
\pgfpathlineto{\pgfqpoint{5.874275in}{2.611385in}}%
\pgfpathlineto{\pgfqpoint{5.859811in}{2.608995in}}%
\pgfpathlineto{\pgfqpoint{5.845358in}{2.606673in}}%
\pgfpathlineto{\pgfqpoint{5.830918in}{2.604418in}}%
\pgfpathlineto{\pgfqpoint{5.823704in}{2.600103in}}%
\pgfpathlineto{\pgfqpoint{5.816487in}{2.595876in}}%
\pgfpathlineto{\pgfqpoint{5.809265in}{2.591730in}}%
\pgfpathlineto{\pgfqpoint{5.802040in}{2.587659in}}%
\pgfpathclose%
\pgfusepath{fill}%
\end{pgfscope}%
\begin{pgfscope}%
\pgfpathrectangle{\pgfqpoint{1.150000in}{0.150000in}}{\pgfqpoint{5.700000in}{5.700000in}}%
\pgfusepath{clip}%
\pgfsetbuttcap%
\pgfsetroundjoin%
\definecolor{currentfill}{rgb}{0.281924,0.089666,0.412415}%
\pgfsetfillcolor{currentfill}%
\pgfsetfillopacity{0.700000}%
\pgfsetlinewidth{0.000000pt}%
\definecolor{currentstroke}{rgb}{0.000000,0.000000,0.000000}%
\pgfsetstrokecolor{currentstroke}%
\pgfsetdash{}{0pt}%
\pgfpathmoveto{\pgfqpoint{3.919428in}{1.941882in}}%
\pgfpathlineto{\pgfqpoint{3.933239in}{1.940780in}}%
\pgfpathlineto{\pgfqpoint{3.947058in}{1.939755in}}%
\pgfpathlineto{\pgfqpoint{3.960885in}{1.938806in}}%
\pgfpathlineto{\pgfqpoint{3.974719in}{1.937934in}}%
\pgfpathlineto{\pgfqpoint{3.982759in}{1.947212in}}%
\pgfpathlineto{\pgfqpoint{3.990794in}{1.956451in}}%
\pgfpathlineto{\pgfqpoint{3.998823in}{1.965651in}}%
\pgfpathlineto{\pgfqpoint{4.006846in}{1.974813in}}%
\pgfpathlineto{\pgfqpoint{3.993022in}{1.975645in}}%
\pgfpathlineto{\pgfqpoint{3.979206in}{1.976553in}}%
\pgfpathlineto{\pgfqpoint{3.965397in}{1.977538in}}%
\pgfpathlineto{\pgfqpoint{3.951596in}{1.978600in}}%
\pgfpathlineto{\pgfqpoint{3.943563in}{1.969471in}}%
\pgfpathlineto{\pgfqpoint{3.935523in}{1.960308in}}%
\pgfpathlineto{\pgfqpoint{3.927478in}{1.951112in}}%
\pgfpathlineto{\pgfqpoint{3.919428in}{1.941882in}}%
\pgfpathclose%
\pgfusepath{fill}%
\end{pgfscope}%
\begin{pgfscope}%
\pgfpathrectangle{\pgfqpoint{1.150000in}{0.150000in}}{\pgfqpoint{5.700000in}{5.700000in}}%
\pgfusepath{clip}%
\pgfsetbuttcap%
\pgfsetroundjoin%
\definecolor{currentfill}{rgb}{0.252194,0.269783,0.531579}%
\pgfsetfillcolor{currentfill}%
\pgfsetfillopacity{0.700000}%
\pgfsetlinewidth{0.000000pt}%
\definecolor{currentstroke}{rgb}{0.000000,0.000000,0.000000}%
\pgfsetstrokecolor{currentstroke}%
\pgfsetdash{}{0pt}%
\pgfpathmoveto{\pgfqpoint{4.991655in}{2.329921in}}%
\pgfpathlineto{\pgfqpoint{5.005825in}{2.331911in}}%
\pgfpathlineto{\pgfqpoint{5.020006in}{2.333971in}}%
\pgfpathlineto{\pgfqpoint{5.034198in}{2.336102in}}%
\pgfpathlineto{\pgfqpoint{5.048400in}{2.338302in}}%
\pgfpathlineto{\pgfqpoint{5.056008in}{2.344460in}}%
\pgfpathlineto{\pgfqpoint{5.063608in}{2.350583in}}%
\pgfpathlineto{\pgfqpoint{5.071202in}{2.356674in}}%
\pgfpathlineto{\pgfqpoint{5.078789in}{2.362739in}}%
\pgfpathlineto{\pgfqpoint{5.064603in}{2.360729in}}%
\pgfpathlineto{\pgfqpoint{5.050427in}{2.358790in}}%
\pgfpathlineto{\pgfqpoint{5.036263in}{2.356921in}}%
\pgfpathlineto{\pgfqpoint{5.022109in}{2.355121in}}%
\pgfpathlineto{\pgfqpoint{5.014505in}{2.348859in}}%
\pgfpathlineto{\pgfqpoint{5.006895in}{2.342574in}}%
\pgfpathlineto{\pgfqpoint{4.999278in}{2.336262in}}%
\pgfpathlineto{\pgfqpoint{4.991655in}{2.329921in}}%
\pgfpathclose%
\pgfusepath{fill}%
\end{pgfscope}%
\begin{pgfscope}%
\pgfpathrectangle{\pgfqpoint{1.150000in}{0.150000in}}{\pgfqpoint{5.700000in}{5.700000in}}%
\pgfusepath{clip}%
\pgfsetbuttcap%
\pgfsetroundjoin%
\definecolor{currentfill}{rgb}{0.269944,0.014625,0.341379}%
\pgfsetfillcolor{currentfill}%
\pgfsetfillopacity{0.700000}%
\pgfsetlinewidth{0.000000pt}%
\definecolor{currentstroke}{rgb}{0.000000,0.000000,0.000000}%
\pgfsetstrokecolor{currentstroke}%
\pgfsetdash{}{0pt}%
\pgfpathmoveto{\pgfqpoint{3.141192in}{1.817758in}}%
\pgfpathlineto{\pgfqpoint{3.154857in}{1.812506in}}%
\pgfpathlineto{\pgfqpoint{3.168526in}{1.807343in}}%
\pgfpathlineto{\pgfqpoint{3.182200in}{1.802268in}}%
\pgfpathlineto{\pgfqpoint{3.195877in}{1.797282in}}%
\pgfpathlineto{\pgfqpoint{3.204214in}{1.805287in}}%
\pgfpathlineto{\pgfqpoint{3.212543in}{1.813345in}}%
\pgfpathlineto{\pgfqpoint{3.220866in}{1.821453in}}%
\pgfpathlineto{\pgfqpoint{3.229181in}{1.829610in}}%
\pgfpathlineto{\pgfqpoint{3.215519in}{1.834412in}}%
\pgfpathlineto{\pgfqpoint{3.201862in}{1.839302in}}%
\pgfpathlineto{\pgfqpoint{3.188210in}{1.844281in}}%
\pgfpathlineto{\pgfqpoint{3.174561in}{1.849349in}}%
\pgfpathlineto{\pgfqpoint{3.166230in}{1.841369in}}%
\pgfpathlineto{\pgfqpoint{3.157891in}{1.833443in}}%
\pgfpathlineto{\pgfqpoint{3.149545in}{1.825572in}}%
\pgfpathlineto{\pgfqpoint{3.141192in}{1.817758in}}%
\pgfpathclose%
\pgfusepath{fill}%
\end{pgfscope}%
\begin{pgfscope}%
\pgfpathrectangle{\pgfqpoint{1.150000in}{0.150000in}}{\pgfqpoint{5.700000in}{5.700000in}}%
\pgfusepath{clip}%
\pgfsetbuttcap%
\pgfsetroundjoin%
\definecolor{currentfill}{rgb}{0.273809,0.031497,0.358853}%
\pgfsetfillcolor{currentfill}%
\pgfsetfillopacity{0.700000}%
\pgfsetlinewidth{0.000000pt}%
\definecolor{currentstroke}{rgb}{0.000000,0.000000,0.000000}%
\pgfsetstrokecolor{currentstroke}%
\pgfsetdash{}{0pt}%
\pgfpathmoveto{\pgfqpoint{2.855298in}{1.861117in}}%
\pgfpathlineto{\pgfqpoint{2.868951in}{1.853942in}}%
\pgfpathlineto{\pgfqpoint{2.882607in}{1.846864in}}%
\pgfpathlineto{\pgfqpoint{2.896265in}{1.839883in}}%
\pgfpathlineto{\pgfqpoint{2.909925in}{1.832998in}}%
\pgfpathlineto{\pgfqpoint{2.918403in}{1.839542in}}%
\pgfpathlineto{\pgfqpoint{2.926871in}{1.846184in}}%
\pgfpathlineto{\pgfqpoint{2.935330in}{1.852921in}}%
\pgfpathlineto{\pgfqpoint{2.943780in}{1.859747in}}%
\pgfpathlineto{\pgfqpoint{2.930140in}{1.866406in}}%
\pgfpathlineto{\pgfqpoint{2.916503in}{1.873161in}}%
\pgfpathlineto{\pgfqpoint{2.902868in}{1.880013in}}%
\pgfpathlineto{\pgfqpoint{2.889236in}{1.886962in}}%
\pgfpathlineto{\pgfqpoint{2.880766in}{1.880353in}}%
\pgfpathlineto{\pgfqpoint{2.872286in}{1.873841in}}%
\pgfpathlineto{\pgfqpoint{2.863796in}{1.867428in}}%
\pgfpathlineto{\pgfqpoint{2.855298in}{1.861117in}}%
\pgfpathclose%
\pgfusepath{fill}%
\end{pgfscope}%
\begin{pgfscope}%
\pgfpathrectangle{\pgfqpoint{1.150000in}{0.150000in}}{\pgfqpoint{5.700000in}{5.700000in}}%
\pgfusepath{clip}%
\pgfsetbuttcap%
\pgfsetroundjoin%
\definecolor{currentfill}{rgb}{0.269308,0.218818,0.509577}%
\pgfsetfillcolor{currentfill}%
\pgfsetfillopacity{0.700000}%
\pgfsetlinewidth{0.000000pt}%
\definecolor{currentstroke}{rgb}{0.000000,0.000000,0.000000}%
\pgfsetstrokecolor{currentstroke}%
\pgfsetdash{}{0pt}%
\pgfpathmoveto{\pgfqpoint{2.148778in}{2.253473in}}%
\pgfpathlineto{\pgfqpoint{2.162533in}{2.240432in}}%
\pgfpathlineto{\pgfqpoint{2.176285in}{2.227526in}}%
\pgfpathlineto{\pgfqpoint{2.190033in}{2.214756in}}%
\pgfpathlineto{\pgfqpoint{2.203779in}{2.202119in}}%
\pgfpathlineto{\pgfqpoint{2.212706in}{2.203903in}}%
\pgfpathlineto{\pgfqpoint{2.221618in}{2.205890in}}%
\pgfpathlineto{\pgfqpoint{2.230513in}{2.208074in}}%
\pgfpathlineto{\pgfqpoint{2.239394in}{2.210452in}}%
\pgfpathlineto{\pgfqpoint{2.225682in}{2.222791in}}%
\pgfpathlineto{\pgfqpoint{2.211968in}{2.235263in}}%
\pgfpathlineto{\pgfqpoint{2.198251in}{2.247870in}}%
\pgfpathlineto{\pgfqpoint{2.184531in}{2.260612in}}%
\pgfpathlineto{\pgfqpoint{2.175617in}{2.258524in}}%
\pgfpathlineto{\pgfqpoint{2.166687in}{2.256636in}}%
\pgfpathlineto{\pgfqpoint{2.157741in}{2.254950in}}%
\pgfpathlineto{\pgfqpoint{2.148778in}{2.253473in}}%
\pgfpathclose%
\pgfusepath{fill}%
\end{pgfscope}%
\begin{pgfscope}%
\pgfpathrectangle{\pgfqpoint{1.150000in}{0.150000in}}{\pgfqpoint{5.700000in}{5.700000in}}%
\pgfusepath{clip}%
\pgfsetbuttcap%
\pgfsetroundjoin%
\definecolor{currentfill}{rgb}{0.278826,0.175490,0.483397}%
\pgfsetfillcolor{currentfill}%
\pgfsetfillopacity{0.700000}%
\pgfsetlinewidth{0.000000pt}%
\definecolor{currentstroke}{rgb}{0.000000,0.000000,0.000000}%
\pgfsetstrokecolor{currentstroke}%
\pgfsetdash{}{0pt}%
\pgfpathmoveto{\pgfqpoint{4.411851in}{2.114601in}}%
\pgfpathlineto{\pgfqpoint{4.425815in}{2.115290in}}%
\pgfpathlineto{\pgfqpoint{4.439788in}{2.116051in}}%
\pgfpathlineto{\pgfqpoint{4.453771in}{2.116885in}}%
\pgfpathlineto{\pgfqpoint{4.467763in}{2.117792in}}%
\pgfpathlineto{\pgfqpoint{4.475623in}{2.126025in}}%
\pgfpathlineto{\pgfqpoint{4.483477in}{2.134201in}}%
\pgfpathlineto{\pgfqpoint{4.491324in}{2.142320in}}%
\pgfpathlineto{\pgfqpoint{4.499166in}{2.150385in}}%
\pgfpathlineto{\pgfqpoint{4.485184in}{2.149542in}}%
\pgfpathlineto{\pgfqpoint{4.471213in}{2.148772in}}%
\pgfpathlineto{\pgfqpoint{4.457251in}{2.148074in}}%
\pgfpathlineto{\pgfqpoint{4.443298in}{2.147450in}}%
\pgfpathlineto{\pgfqpoint{4.435446in}{2.139313in}}%
\pgfpathlineto{\pgfqpoint{4.427587in}{2.131127in}}%
\pgfpathlineto{\pgfqpoint{4.419722in}{2.122890in}}%
\pgfpathlineto{\pgfqpoint{4.411851in}{2.114601in}}%
\pgfpathclose%
\pgfusepath{fill}%
\end{pgfscope}%
\begin{pgfscope}%
\pgfpathrectangle{\pgfqpoint{1.150000in}{0.150000in}}{\pgfqpoint{5.700000in}{5.700000in}}%
\pgfusepath{clip}%
\pgfsetbuttcap%
\pgfsetroundjoin%
\definecolor{currentfill}{rgb}{0.225863,0.330805,0.547314}%
\pgfsetfillcolor{currentfill}%
\pgfsetfillopacity{0.700000}%
\pgfsetlinewidth{0.000000pt}%
\definecolor{currentstroke}{rgb}{0.000000,0.000000,0.000000}%
\pgfsetstrokecolor{currentstroke}%
\pgfsetdash{}{0pt}%
\pgfpathmoveto{\pgfqpoint{5.396966in}{2.465189in}}%
\pgfpathlineto{\pgfqpoint{5.411289in}{2.467675in}}%
\pgfpathlineto{\pgfqpoint{5.425624in}{2.470230in}}%
\pgfpathlineto{\pgfqpoint{5.439971in}{2.472853in}}%
\pgfpathlineto{\pgfqpoint{5.454329in}{2.475545in}}%
\pgfpathlineto{\pgfqpoint{5.461736in}{2.480282in}}%
\pgfpathlineto{\pgfqpoint{5.469137in}{2.485025in}}%
\pgfpathlineto{\pgfqpoint{5.476532in}{2.489780in}}%
\pgfpathlineto{\pgfqpoint{5.483920in}{2.494551in}}%
\pgfpathlineto{\pgfqpoint{5.469584in}{2.492135in}}%
\pgfpathlineto{\pgfqpoint{5.455259in}{2.489788in}}%
\pgfpathlineto{\pgfqpoint{5.440945in}{2.487509in}}%
\pgfpathlineto{\pgfqpoint{5.426644in}{2.485299in}}%
\pgfpathlineto{\pgfqpoint{5.419233in}{2.480245in}}%
\pgfpathlineto{\pgfqpoint{5.411817in}{2.475211in}}%
\pgfpathlineto{\pgfqpoint{5.404394in}{2.470195in}}%
\pgfpathlineto{\pgfqpoint{5.396966in}{2.465189in}}%
\pgfpathclose%
\pgfusepath{fill}%
\end{pgfscope}%
\begin{pgfscope}%
\pgfpathrectangle{\pgfqpoint{1.150000in}{0.150000in}}{\pgfqpoint{5.700000in}{5.700000in}}%
\pgfusepath{clip}%
\pgfsetbuttcap%
\pgfsetroundjoin%
\definecolor{currentfill}{rgb}{0.272594,0.025563,0.353093}%
\pgfsetfillcolor{currentfill}%
\pgfsetfillopacity{0.700000}%
\pgfsetlinewidth{0.000000pt}%
\definecolor{currentstroke}{rgb}{0.000000,0.000000,0.000000}%
\pgfsetstrokecolor{currentstroke}%
\pgfsetdash{}{0pt}%
\pgfpathmoveto{\pgfqpoint{3.514231in}{1.836064in}}%
\pgfpathlineto{\pgfqpoint{3.527952in}{1.833010in}}%
\pgfpathlineto{\pgfqpoint{3.541678in}{1.830037in}}%
\pgfpathlineto{\pgfqpoint{3.555411in}{1.827146in}}%
\pgfpathlineto{\pgfqpoint{3.569149in}{1.824336in}}%
\pgfpathlineto{\pgfqpoint{3.577336in}{1.833478in}}%
\pgfpathlineto{\pgfqpoint{3.585517in}{1.842621in}}%
\pgfpathlineto{\pgfqpoint{3.593692in}{1.851766in}}%
\pgfpathlineto{\pgfqpoint{3.601861in}{1.860909in}}%
\pgfpathlineto{\pgfqpoint{3.588134in}{1.863596in}}%
\pgfpathlineto{\pgfqpoint{3.574414in}{1.866365in}}%
\pgfpathlineto{\pgfqpoint{3.560700in}{1.869215in}}%
\pgfpathlineto{\pgfqpoint{3.546992in}{1.872147in}}%
\pgfpathlineto{\pgfqpoint{3.538811in}{1.863118in}}%
\pgfpathlineto{\pgfqpoint{3.530624in}{1.854094in}}%
\pgfpathlineto{\pgfqpoint{3.522430in}{1.845076in}}%
\pgfpathlineto{\pgfqpoint{3.514231in}{1.836064in}}%
\pgfpathclose%
\pgfusepath{fill}%
\end{pgfscope}%
\begin{pgfscope}%
\pgfpathrectangle{\pgfqpoint{1.150000in}{0.150000in}}{\pgfqpoint{5.700000in}{5.700000in}}%
\pgfusepath{clip}%
\pgfsetbuttcap%
\pgfsetroundjoin%
\definecolor{currentfill}{rgb}{0.280267,0.073417,0.397163}%
\pgfsetfillcolor{currentfill}%
\pgfsetfillopacity{0.700000}%
\pgfsetlinewidth{0.000000pt}%
\definecolor{currentstroke}{rgb}{0.000000,0.000000,0.000000}%
\pgfsetstrokecolor{currentstroke}%
\pgfsetdash{}{0pt}%
\pgfpathmoveto{\pgfqpoint{3.831958in}{1.910065in}}%
\pgfpathlineto{\pgfqpoint{3.845750in}{1.908592in}}%
\pgfpathlineto{\pgfqpoint{3.859549in}{1.907197in}}%
\pgfpathlineto{\pgfqpoint{3.873356in}{1.905880in}}%
\pgfpathlineto{\pgfqpoint{3.887170in}{1.904640in}}%
\pgfpathlineto{\pgfqpoint{3.895243in}{1.913998in}}%
\pgfpathlineto{\pgfqpoint{3.903310in}{1.923325in}}%
\pgfpathlineto{\pgfqpoint{3.911372in}{1.932620in}}%
\pgfpathlineto{\pgfqpoint{3.919428in}{1.941882in}}%
\pgfpathlineto{\pgfqpoint{3.905624in}{1.943061in}}%
\pgfpathlineto{\pgfqpoint{3.891828in}{1.944318in}}%
\pgfpathlineto{\pgfqpoint{3.878039in}{1.945651in}}%
\pgfpathlineto{\pgfqpoint{3.864258in}{1.947063in}}%
\pgfpathlineto{\pgfqpoint{3.856192in}{1.937854in}}%
\pgfpathlineto{\pgfqpoint{3.848119in}{1.928618in}}%
\pgfpathlineto{\pgfqpoint{3.840041in}{1.919354in}}%
\pgfpathlineto{\pgfqpoint{3.831958in}{1.910065in}}%
\pgfpathclose%
\pgfusepath{fill}%
\end{pgfscope}%
\begin{pgfscope}%
\pgfpathrectangle{\pgfqpoint{1.150000in}{0.150000in}}{\pgfqpoint{5.700000in}{5.700000in}}%
\pgfusepath{clip}%
\pgfsetbuttcap%
\pgfsetroundjoin%
\definecolor{currentfill}{rgb}{0.255645,0.260703,0.528312}%
\pgfsetfillcolor{currentfill}%
\pgfsetfillopacity{0.700000}%
\pgfsetlinewidth{0.000000pt}%
\definecolor{currentstroke}{rgb}{0.000000,0.000000,0.000000}%
\pgfsetstrokecolor{currentstroke}%
\pgfsetdash{}{0pt}%
\pgfpathmoveto{\pgfqpoint{4.904461in}{2.296257in}}%
\pgfpathlineto{\pgfqpoint{4.918603in}{2.298135in}}%
\pgfpathlineto{\pgfqpoint{4.932756in}{2.300084in}}%
\pgfpathlineto{\pgfqpoint{4.946920in}{2.302104in}}%
\pgfpathlineto{\pgfqpoint{4.961094in}{2.304194in}}%
\pgfpathlineto{\pgfqpoint{4.968745in}{2.310686in}}%
\pgfpathlineto{\pgfqpoint{4.976388in}{2.317137in}}%
\pgfpathlineto{\pgfqpoint{4.984025in}{2.323547in}}%
\pgfpathlineto{\pgfqpoint{4.991655in}{2.329921in}}%
\pgfpathlineto{\pgfqpoint{4.977496in}{2.328001in}}%
\pgfpathlineto{\pgfqpoint{4.963348in}{2.326152in}}%
\pgfpathlineto{\pgfqpoint{4.949210in}{2.324373in}}%
\pgfpathlineto{\pgfqpoint{4.935083in}{2.322664in}}%
\pgfpathlineto{\pgfqpoint{4.927438in}{2.316112in}}%
\pgfpathlineto{\pgfqpoint{4.919785in}{2.309529in}}%
\pgfpathlineto{\pgfqpoint{4.912127in}{2.302912in}}%
\pgfpathlineto{\pgfqpoint{4.904461in}{2.296257in}}%
\pgfpathclose%
\pgfusepath{fill}%
\end{pgfscope}%
\begin{pgfscope}%
\pgfpathrectangle{\pgfqpoint{1.150000in}{0.150000in}}{\pgfqpoint{5.700000in}{5.700000in}}%
\pgfusepath{clip}%
\pgfsetbuttcap%
\pgfsetroundjoin%
\definecolor{currentfill}{rgb}{0.269944,0.014625,0.341379}%
\pgfsetfillcolor{currentfill}%
\pgfsetfillopacity{0.700000}%
\pgfsetlinewidth{0.000000pt}%
\definecolor{currentstroke}{rgb}{0.000000,0.000000,0.000000}%
\pgfsetstrokecolor{currentstroke}%
\pgfsetdash{}{0pt}%
\pgfpathmoveto{\pgfqpoint{3.283871in}{1.811275in}}%
\pgfpathlineto{\pgfqpoint{3.297555in}{1.806908in}}%
\pgfpathlineto{\pgfqpoint{3.311244in}{1.802626in}}%
\pgfpathlineto{\pgfqpoint{3.324938in}{1.798430in}}%
\pgfpathlineto{\pgfqpoint{3.338636in}{1.794319in}}%
\pgfpathlineto{\pgfqpoint{3.346914in}{1.802863in}}%
\pgfpathlineto{\pgfqpoint{3.355185in}{1.811441in}}%
\pgfpathlineto{\pgfqpoint{3.363449in}{1.820049in}}%
\pgfpathlineto{\pgfqpoint{3.371706in}{1.828686in}}%
\pgfpathlineto{\pgfqpoint{3.358022in}{1.832633in}}%
\pgfpathlineto{\pgfqpoint{3.344343in}{1.836666in}}%
\pgfpathlineto{\pgfqpoint{3.330669in}{1.840784in}}%
\pgfpathlineto{\pgfqpoint{3.317000in}{1.844987in}}%
\pgfpathlineto{\pgfqpoint{3.308727in}{1.836507in}}%
\pgfpathlineto{\pgfqpoint{3.300449in}{1.828059in}}%
\pgfpathlineto{\pgfqpoint{3.292163in}{1.819648in}}%
\pgfpathlineto{\pgfqpoint{3.283871in}{1.811275in}}%
\pgfpathclose%
\pgfusepath{fill}%
\end{pgfscope}%
\begin{pgfscope}%
\pgfpathrectangle{\pgfqpoint{1.150000in}{0.150000in}}{\pgfqpoint{5.700000in}{5.700000in}}%
\pgfusepath{clip}%
\pgfsetbuttcap%
\pgfsetroundjoin%
\definecolor{currentfill}{rgb}{0.280868,0.160771,0.472899}%
\pgfsetfillcolor{currentfill}%
\pgfsetfillopacity{0.700000}%
\pgfsetlinewidth{0.000000pt}%
\definecolor{currentstroke}{rgb}{0.000000,0.000000,0.000000}%
\pgfsetstrokecolor{currentstroke}%
\pgfsetdash{}{0pt}%
\pgfpathmoveto{\pgfqpoint{4.324499in}{2.078701in}}%
\pgfpathlineto{\pgfqpoint{4.338437in}{2.079139in}}%
\pgfpathlineto{\pgfqpoint{4.352384in}{2.079651in}}%
\pgfpathlineto{\pgfqpoint{4.366340in}{2.080236in}}%
\pgfpathlineto{\pgfqpoint{4.380305in}{2.080894in}}%
\pgfpathlineto{\pgfqpoint{4.388201in}{2.089406in}}%
\pgfpathlineto{\pgfqpoint{4.396090in}{2.097860in}}%
\pgfpathlineto{\pgfqpoint{4.403974in}{2.106258in}}%
\pgfpathlineto{\pgfqpoint{4.411851in}{2.114601in}}%
\pgfpathlineto{\pgfqpoint{4.397896in}{2.113986in}}%
\pgfpathlineto{\pgfqpoint{4.383951in}{2.113444in}}%
\pgfpathlineto{\pgfqpoint{4.370015in}{2.112976in}}%
\pgfpathlineto{\pgfqpoint{4.356088in}{2.112580in}}%
\pgfpathlineto{\pgfqpoint{4.348199in}{2.104187in}}%
\pgfpathlineto{\pgfqpoint{4.340305in}{2.095743in}}%
\pgfpathlineto{\pgfqpoint{4.332405in}{2.087248in}}%
\pgfpathlineto{\pgfqpoint{4.324499in}{2.078701in}}%
\pgfpathclose%
\pgfusepath{fill}%
\end{pgfscope}%
\begin{pgfscope}%
\pgfpathrectangle{\pgfqpoint{1.150000in}{0.150000in}}{\pgfqpoint{5.700000in}{5.700000in}}%
\pgfusepath{clip}%
\pgfsetbuttcap%
\pgfsetroundjoin%
\definecolor{currentfill}{rgb}{0.274128,0.199721,0.498911}%
\pgfsetfillcolor{currentfill}%
\pgfsetfillopacity{0.700000}%
\pgfsetlinewidth{0.000000pt}%
\definecolor{currentstroke}{rgb}{0.000000,0.000000,0.000000}%
\pgfsetstrokecolor{currentstroke}%
\pgfsetdash{}{0pt}%
\pgfpathmoveto{\pgfqpoint{2.203779in}{2.202119in}}%
\pgfpathlineto{\pgfqpoint{2.217521in}{2.189615in}}%
\pgfpathlineto{\pgfqpoint{2.231261in}{2.177242in}}%
\pgfpathlineto{\pgfqpoint{2.244997in}{2.165000in}}%
\pgfpathlineto{\pgfqpoint{2.258732in}{2.152887in}}%
\pgfpathlineto{\pgfqpoint{2.267625in}{2.154976in}}%
\pgfpathlineto{\pgfqpoint{2.276503in}{2.157263in}}%
\pgfpathlineto{\pgfqpoint{2.285365in}{2.159742in}}%
\pgfpathlineto{\pgfqpoint{2.294212in}{2.162410in}}%
\pgfpathlineto{\pgfqpoint{2.280511in}{2.174226in}}%
\pgfpathlineto{\pgfqpoint{2.266808in}{2.186171in}}%
\pgfpathlineto{\pgfqpoint{2.253102in}{2.198246in}}%
\pgfpathlineto{\pgfqpoint{2.239394in}{2.210452in}}%
\pgfpathlineto{\pgfqpoint{2.230513in}{2.208074in}}%
\pgfpathlineto{\pgfqpoint{2.221618in}{2.205890in}}%
\pgfpathlineto{\pgfqpoint{2.212706in}{2.203903in}}%
\pgfpathlineto{\pgfqpoint{2.203779in}{2.202119in}}%
\pgfpathclose%
\pgfusepath{fill}%
\end{pgfscope}%
\begin{pgfscope}%
\pgfpathrectangle{\pgfqpoint{1.150000in}{0.150000in}}{\pgfqpoint{5.700000in}{5.700000in}}%
\pgfusepath{clip}%
\pgfsetbuttcap%
\pgfsetroundjoin%
\definecolor{currentfill}{rgb}{0.277941,0.056324,0.381191}%
\pgfsetfillcolor{currentfill}%
\pgfsetfillopacity{0.700000}%
\pgfsetlinewidth{0.000000pt}%
\definecolor{currentstroke}{rgb}{0.000000,0.000000,0.000000}%
\pgfsetstrokecolor{currentstroke}%
\pgfsetdash{}{0pt}%
\pgfpathmoveto{\pgfqpoint{2.711865in}{1.899927in}}%
\pgfpathlineto{\pgfqpoint{2.725527in}{1.891704in}}%
\pgfpathlineto{\pgfqpoint{2.739190in}{1.883582in}}%
\pgfpathlineto{\pgfqpoint{2.752855in}{1.875562in}}%
\pgfpathlineto{\pgfqpoint{2.766522in}{1.867644in}}%
\pgfpathlineto{\pgfqpoint{2.775083in}{1.873255in}}%
\pgfpathlineto{\pgfqpoint{2.783634in}{1.878988in}}%
\pgfpathlineto{\pgfqpoint{2.792175in}{1.884838in}}%
\pgfpathlineto{\pgfqpoint{2.800706in}{1.890804in}}%
\pgfpathlineto{\pgfqpoint{2.787063in}{1.898474in}}%
\pgfpathlineto{\pgfqpoint{2.773421in}{1.906246in}}%
\pgfpathlineto{\pgfqpoint{2.759782in}{1.914120in}}%
\pgfpathlineto{\pgfqpoint{2.746144in}{1.922096in}}%
\pgfpathlineto{\pgfqpoint{2.737590in}{1.916370in}}%
\pgfpathlineto{\pgfqpoint{2.729026in}{1.910765in}}%
\pgfpathlineto{\pgfqpoint{2.720451in}{1.905283in}}%
\pgfpathlineto{\pgfqpoint{2.711865in}{1.899927in}}%
\pgfpathclose%
\pgfusepath{fill}%
\end{pgfscope}%
\begin{pgfscope}%
\pgfpathrectangle{\pgfqpoint{1.150000in}{0.150000in}}{\pgfqpoint{5.700000in}{5.700000in}}%
\pgfusepath{clip}%
\pgfsetbuttcap%
\pgfsetroundjoin%
\definecolor{currentfill}{rgb}{0.282910,0.105393,0.426902}%
\pgfsetfillcolor{currentfill}%
\pgfsetfillopacity{0.700000}%
\pgfsetlinewidth{0.000000pt}%
\definecolor{currentstroke}{rgb}{0.000000,0.000000,0.000000}%
\pgfsetstrokecolor{currentstroke}%
\pgfsetdash{}{0pt}%
\pgfpathmoveto{\pgfqpoint{2.513219in}{1.990028in}}%
\pgfpathlineto{\pgfqpoint{2.526900in}{1.980247in}}%
\pgfpathlineto{\pgfqpoint{2.540581in}{1.970579in}}%
\pgfpathlineto{\pgfqpoint{2.554263in}{1.961020in}}%
\pgfpathlineto{\pgfqpoint{2.567944in}{1.951572in}}%
\pgfpathlineto{\pgfqpoint{2.576628in}{1.955826in}}%
\pgfpathlineto{\pgfqpoint{2.585300in}{1.960233in}}%
\pgfpathlineto{\pgfqpoint{2.593959in}{1.964789in}}%
\pgfpathlineto{\pgfqpoint{2.602607in}{1.969490in}}%
\pgfpathlineto{\pgfqpoint{2.588953in}{1.978668in}}%
\pgfpathlineto{\pgfqpoint{2.575299in}{1.987955in}}%
\pgfpathlineto{\pgfqpoint{2.561646in}{1.997353in}}%
\pgfpathlineto{\pgfqpoint{2.547992in}{2.006863in}}%
\pgfpathlineto{\pgfqpoint{2.539318in}{2.002425in}}%
\pgfpathlineto{\pgfqpoint{2.530631in}{1.998137in}}%
\pgfpathlineto{\pgfqpoint{2.521931in}{1.994003in}}%
\pgfpathlineto{\pgfqpoint{2.513219in}{1.990028in}}%
\pgfpathclose%
\pgfusepath{fill}%
\end{pgfscope}%
\begin{pgfscope}%
\pgfpathrectangle{\pgfqpoint{1.150000in}{0.150000in}}{\pgfqpoint{5.700000in}{5.700000in}}%
\pgfusepath{clip}%
\pgfsetbuttcap%
\pgfsetroundjoin%
\definecolor{currentfill}{rgb}{0.282290,0.145912,0.461510}%
\pgfsetfillcolor{currentfill}%
\pgfsetfillopacity{0.700000}%
\pgfsetlinewidth{0.000000pt}%
\definecolor{currentstroke}{rgb}{0.000000,0.000000,0.000000}%
\pgfsetstrokecolor{currentstroke}%
\pgfsetdash{}{0pt}%
\pgfpathmoveto{\pgfqpoint{4.237110in}{2.042868in}}%
\pgfpathlineto{\pgfqpoint{4.251023in}{2.043033in}}%
\pgfpathlineto{\pgfqpoint{4.264944in}{2.043271in}}%
\pgfpathlineto{\pgfqpoint{4.278875in}{2.043584in}}%
\pgfpathlineto{\pgfqpoint{4.292814in}{2.043971in}}%
\pgfpathlineto{\pgfqpoint{4.300744in}{2.052736in}}%
\pgfpathlineto{\pgfqpoint{4.308668in}{2.061446in}}%
\pgfpathlineto{\pgfqpoint{4.316587in}{2.070101in}}%
\pgfpathlineto{\pgfqpoint{4.324499in}{2.078701in}}%
\pgfpathlineto{\pgfqpoint{4.310570in}{2.078337in}}%
\pgfpathlineto{\pgfqpoint{4.296650in}{2.078046in}}%
\pgfpathlineto{\pgfqpoint{4.282739in}{2.077830in}}%
\pgfpathlineto{\pgfqpoint{4.268837in}{2.077687in}}%
\pgfpathlineto{\pgfqpoint{4.260914in}{2.069057in}}%
\pgfpathlineto{\pgfqpoint{4.252986in}{2.060378in}}%
\pgfpathlineto{\pgfqpoint{4.245051in}{2.051648in}}%
\pgfpathlineto{\pgfqpoint{4.237110in}{2.042868in}}%
\pgfpathclose%
\pgfusepath{fill}%
\end{pgfscope}%
\begin{pgfscope}%
\pgfpathrectangle{\pgfqpoint{1.150000in}{0.150000in}}{\pgfqpoint{5.700000in}{5.700000in}}%
\pgfusepath{clip}%
\pgfsetbuttcap%
\pgfsetroundjoin%
\definecolor{currentfill}{rgb}{0.260571,0.246922,0.522828}%
\pgfsetfillcolor{currentfill}%
\pgfsetfillopacity{0.700000}%
\pgfsetlinewidth{0.000000pt}%
\definecolor{currentstroke}{rgb}{0.000000,0.000000,0.000000}%
\pgfsetstrokecolor{currentstroke}%
\pgfsetdash{}{0pt}%
\pgfpathmoveto{\pgfqpoint{4.817213in}{2.261796in}}%
\pgfpathlineto{\pgfqpoint{4.831327in}{2.263540in}}%
\pgfpathlineto{\pgfqpoint{4.845452in}{2.265355in}}%
\pgfpathlineto{\pgfqpoint{4.859587in}{2.267241in}}%
\pgfpathlineto{\pgfqpoint{4.873732in}{2.269198in}}%
\pgfpathlineto{\pgfqpoint{4.881425in}{2.276034in}}%
\pgfpathlineto{\pgfqpoint{4.889110in}{2.282821in}}%
\pgfpathlineto{\pgfqpoint{4.896789in}{2.289561in}}%
\pgfpathlineto{\pgfqpoint{4.904461in}{2.296257in}}%
\pgfpathlineto{\pgfqpoint{4.890330in}{2.294449in}}%
\pgfpathlineto{\pgfqpoint{4.876209in}{2.292711in}}%
\pgfpathlineto{\pgfqpoint{4.862099in}{2.291045in}}%
\pgfpathlineto{\pgfqpoint{4.847999in}{2.289449in}}%
\pgfpathlineto{\pgfqpoint{4.840313in}{2.282597in}}%
\pgfpathlineto{\pgfqpoint{4.832619in}{2.275706in}}%
\pgfpathlineto{\pgfqpoint{4.824920in}{2.268773in}}%
\pgfpathlineto{\pgfqpoint{4.817213in}{2.261796in}}%
\pgfpathclose%
\pgfusepath{fill}%
\end{pgfscope}%
\begin{pgfscope}%
\pgfpathrectangle{\pgfqpoint{1.150000in}{0.150000in}}{\pgfqpoint{5.700000in}{5.700000in}}%
\pgfusepath{clip}%
\pgfsetbuttcap%
\pgfsetroundjoin%
\definecolor{currentfill}{rgb}{0.231674,0.318106,0.544834}%
\pgfsetfillcolor{currentfill}%
\pgfsetfillopacity{0.700000}%
\pgfsetlinewidth{0.000000pt}%
\definecolor{currentstroke}{rgb}{0.000000,0.000000,0.000000}%
\pgfsetstrokecolor{currentstroke}%
\pgfsetdash{}{0pt}%
\pgfpathmoveto{\pgfqpoint{5.309933in}{2.434911in}}%
\pgfpathlineto{\pgfqpoint{5.324230in}{2.437377in}}%
\pgfpathlineto{\pgfqpoint{5.338538in}{2.439912in}}%
\pgfpathlineto{\pgfqpoint{5.352858in}{2.442516in}}%
\pgfpathlineto{\pgfqpoint{5.367190in}{2.445188in}}%
\pgfpathlineto{\pgfqpoint{5.374644in}{2.450195in}}%
\pgfpathlineto{\pgfqpoint{5.382091in}{2.455194in}}%
\pgfpathlineto{\pgfqpoint{5.389532in}{2.460191in}}%
\pgfpathlineto{\pgfqpoint{5.396966in}{2.465189in}}%
\pgfpathlineto{\pgfqpoint{5.382655in}{2.462772in}}%
\pgfpathlineto{\pgfqpoint{5.368355in}{2.460423in}}%
\pgfpathlineto{\pgfqpoint{5.354067in}{2.458144in}}%
\pgfpathlineto{\pgfqpoint{5.339790in}{2.455933in}}%
\pgfpathlineto{\pgfqpoint{5.332335in}{2.450671in}}%
\pgfpathlineto{\pgfqpoint{5.324874in}{2.445417in}}%
\pgfpathlineto{\pgfqpoint{5.317406in}{2.440165in}}%
\pgfpathlineto{\pgfqpoint{5.309933in}{2.434911in}}%
\pgfpathclose%
\pgfusepath{fill}%
\end{pgfscope}%
\begin{pgfscope}%
\pgfpathrectangle{\pgfqpoint{1.150000in}{0.150000in}}{\pgfqpoint{5.700000in}{5.700000in}}%
\pgfusepath{clip}%
\pgfsetbuttcap%
\pgfsetroundjoin%
\definecolor{currentfill}{rgb}{0.206756,0.371758,0.553117}%
\pgfsetfillcolor{currentfill}%
\pgfsetfillopacity{0.700000}%
\pgfsetlinewidth{0.000000pt}%
\definecolor{currentstroke}{rgb}{0.000000,0.000000,0.000000}%
\pgfsetstrokecolor{currentstroke}%
\pgfsetdash{}{0pt}%
\pgfpathmoveto{\pgfqpoint{5.715243in}{2.560857in}}%
\pgfpathlineto{\pgfqpoint{5.729688in}{2.563542in}}%
\pgfpathlineto{\pgfqpoint{5.744145in}{2.566294in}}%
\pgfpathlineto{\pgfqpoint{5.758615in}{2.569114in}}%
\pgfpathlineto{\pgfqpoint{5.773097in}{2.572002in}}%
\pgfpathlineto{\pgfqpoint{5.780340in}{2.575834in}}%
\pgfpathlineto{\pgfqpoint{5.787578in}{2.579717in}}%
\pgfpathlineto{\pgfqpoint{5.794811in}{2.583656in}}%
\pgfpathlineto{\pgfqpoint{5.802040in}{2.587659in}}%
\pgfpathlineto{\pgfqpoint{5.787585in}{2.585111in}}%
\pgfpathlineto{\pgfqpoint{5.773142in}{2.582631in}}%
\pgfpathlineto{\pgfqpoint{5.758711in}{2.580218in}}%
\pgfpathlineto{\pgfqpoint{5.744292in}{2.577873in}}%
\pgfpathlineto{\pgfqpoint{5.737036in}{2.573523in}}%
\pgfpathlineto{\pgfqpoint{5.729776in}{2.569242in}}%
\pgfpathlineto{\pgfqpoint{5.722512in}{2.565021in}}%
\pgfpathlineto{\pgfqpoint{5.715243in}{2.560857in}}%
\pgfpathclose%
\pgfusepath{fill}%
\end{pgfscope}%
\begin{pgfscope}%
\pgfpathrectangle{\pgfqpoint{1.150000in}{0.150000in}}{\pgfqpoint{5.700000in}{5.700000in}}%
\pgfusepath{clip}%
\pgfsetbuttcap%
\pgfsetroundjoin%
\definecolor{currentfill}{rgb}{0.278791,0.062145,0.386592}%
\pgfsetfillcolor{currentfill}%
\pgfsetfillopacity{0.700000}%
\pgfsetlinewidth{0.000000pt}%
\definecolor{currentstroke}{rgb}{0.000000,0.000000,0.000000}%
\pgfsetstrokecolor{currentstroke}%
\pgfsetdash{}{0pt}%
\pgfpathmoveto{\pgfqpoint{3.744428in}{1.879654in}}%
\pgfpathlineto{\pgfqpoint{3.758203in}{1.877787in}}%
\pgfpathlineto{\pgfqpoint{3.771984in}{1.875999in}}%
\pgfpathlineto{\pgfqpoint{3.785772in}{1.874289in}}%
\pgfpathlineto{\pgfqpoint{3.799567in}{1.872657in}}%
\pgfpathlineto{\pgfqpoint{3.807673in}{1.882045in}}%
\pgfpathlineto{\pgfqpoint{3.815774in}{1.891409in}}%
\pgfpathlineto{\pgfqpoint{3.823869in}{1.900749in}}%
\pgfpathlineto{\pgfqpoint{3.831958in}{1.910065in}}%
\pgfpathlineto{\pgfqpoint{3.818173in}{1.911615in}}%
\pgfpathlineto{\pgfqpoint{3.804396in}{1.913244in}}%
\pgfpathlineto{\pgfqpoint{3.790626in}{1.914950in}}%
\pgfpathlineto{\pgfqpoint{3.776862in}{1.916736in}}%
\pgfpathlineto{\pgfqpoint{3.768762in}{1.907494in}}%
\pgfpathlineto{\pgfqpoint{3.760657in}{1.898233in}}%
\pgfpathlineto{\pgfqpoint{3.752545in}{1.888953in}}%
\pgfpathlineto{\pgfqpoint{3.744428in}{1.879654in}}%
\pgfpathclose%
\pgfusepath{fill}%
\end{pgfscope}%
\begin{pgfscope}%
\pgfpathrectangle{\pgfqpoint{1.150000in}{0.150000in}}{\pgfqpoint{5.700000in}{5.700000in}}%
\pgfusepath{clip}%
\pgfsetbuttcap%
\pgfsetroundjoin%
\definecolor{currentfill}{rgb}{0.278012,0.180367,0.486697}%
\pgfsetfillcolor{currentfill}%
\pgfsetfillopacity{0.700000}%
\pgfsetlinewidth{0.000000pt}%
\definecolor{currentstroke}{rgb}{0.000000,0.000000,0.000000}%
\pgfsetstrokecolor{currentstroke}%
\pgfsetdash{}{0pt}%
\pgfpathmoveto{\pgfqpoint{2.258732in}{2.152887in}}%
\pgfpathlineto{\pgfqpoint{2.272464in}{2.140902in}}%
\pgfpathlineto{\pgfqpoint{2.286193in}{2.129044in}}%
\pgfpathlineto{\pgfqpoint{2.299920in}{2.117312in}}%
\pgfpathlineto{\pgfqpoint{2.313646in}{2.105704in}}%
\pgfpathlineto{\pgfqpoint{2.322506in}{2.108098in}}%
\pgfpathlineto{\pgfqpoint{2.331350in}{2.110684in}}%
\pgfpathlineto{\pgfqpoint{2.340181in}{2.113456in}}%
\pgfpathlineto{\pgfqpoint{2.348996in}{2.116411in}}%
\pgfpathlineto{\pgfqpoint{2.335303in}{2.127723in}}%
\pgfpathlineto{\pgfqpoint{2.321608in}{2.139159in}}%
\pgfpathlineto{\pgfqpoint{2.307911in}{2.150721in}}%
\pgfpathlineto{\pgfqpoint{2.294212in}{2.162410in}}%
\pgfpathlineto{\pgfqpoint{2.285365in}{2.159742in}}%
\pgfpathlineto{\pgfqpoint{2.276503in}{2.157263in}}%
\pgfpathlineto{\pgfqpoint{2.267625in}{2.154976in}}%
\pgfpathlineto{\pgfqpoint{2.258732in}{2.152887in}}%
\pgfpathclose%
\pgfusepath{fill}%
\end{pgfscope}%
\begin{pgfscope}%
\pgfpathrectangle{\pgfqpoint{1.150000in}{0.150000in}}{\pgfqpoint{5.700000in}{5.700000in}}%
\pgfusepath{clip}%
\pgfsetbuttcap%
\pgfsetroundjoin%
\definecolor{currentfill}{rgb}{0.271305,0.019942,0.347269}%
\pgfsetfillcolor{currentfill}%
\pgfsetfillopacity{0.700000}%
\pgfsetlinewidth{0.000000pt}%
\definecolor{currentstroke}{rgb}{0.000000,0.000000,0.000000}%
\pgfsetstrokecolor{currentstroke}%
\pgfsetdash{}{0pt}%
\pgfpathmoveto{\pgfqpoint{3.426496in}{1.813740in}}%
\pgfpathlineto{\pgfqpoint{3.440207in}{1.810212in}}%
\pgfpathlineto{\pgfqpoint{3.453923in}{1.806768in}}%
\pgfpathlineto{\pgfqpoint{3.467645in}{1.803406in}}%
\pgfpathlineto{\pgfqpoint{3.481373in}{1.800126in}}%
\pgfpathlineto{\pgfqpoint{3.489597in}{1.809091in}}%
\pgfpathlineto{\pgfqpoint{3.497814in}{1.818070in}}%
\pgfpathlineto{\pgfqpoint{3.506026in}{1.827062in}}%
\pgfpathlineto{\pgfqpoint{3.514231in}{1.836064in}}%
\pgfpathlineto{\pgfqpoint{3.500516in}{1.839201in}}%
\pgfpathlineto{\pgfqpoint{3.486808in}{1.842419in}}%
\pgfpathlineto{\pgfqpoint{3.473104in}{1.845721in}}%
\pgfpathlineto{\pgfqpoint{3.459407in}{1.849105in}}%
\pgfpathlineto{\pgfqpoint{3.451189in}{1.840238in}}%
\pgfpathlineto{\pgfqpoint{3.442964in}{1.831387in}}%
\pgfpathlineto{\pgfqpoint{3.434733in}{1.822554in}}%
\pgfpathlineto{\pgfqpoint{3.426496in}{1.813740in}}%
\pgfpathclose%
\pgfusepath{fill}%
\end{pgfscope}%
\begin{pgfscope}%
\pgfpathrectangle{\pgfqpoint{1.150000in}{0.150000in}}{\pgfqpoint{5.700000in}{5.700000in}}%
\pgfusepath{clip}%
\pgfsetbuttcap%
\pgfsetroundjoin%
\definecolor{currentfill}{rgb}{0.283187,0.125848,0.444960}%
\pgfsetfillcolor{currentfill}%
\pgfsetfillopacity{0.700000}%
\pgfsetlinewidth{0.000000pt}%
\definecolor{currentstroke}{rgb}{0.000000,0.000000,0.000000}%
\pgfsetstrokecolor{currentstroke}%
\pgfsetdash{}{0pt}%
\pgfpathmoveto{\pgfqpoint{4.149686in}{2.007308in}}%
\pgfpathlineto{\pgfqpoint{4.163574in}{2.007176in}}%
\pgfpathlineto{\pgfqpoint{4.177470in}{2.007118in}}%
\pgfpathlineto{\pgfqpoint{4.191376in}{2.007135in}}%
\pgfpathlineto{\pgfqpoint{4.205290in}{2.007227in}}%
\pgfpathlineto{\pgfqpoint{4.213254in}{2.016216in}}%
\pgfpathlineto{\pgfqpoint{4.221212in}{2.025153in}}%
\pgfpathlineto{\pgfqpoint{4.229164in}{2.034037in}}%
\pgfpathlineto{\pgfqpoint{4.237110in}{2.042868in}}%
\pgfpathlineto{\pgfqpoint{4.223207in}{2.042778in}}%
\pgfpathlineto{\pgfqpoint{4.209312in}{2.042762in}}%
\pgfpathlineto{\pgfqpoint{4.195425in}{2.042821in}}%
\pgfpathlineto{\pgfqpoint{4.181548in}{2.042954in}}%
\pgfpathlineto{\pgfqpoint{4.173591in}{2.034114in}}%
\pgfpathlineto{\pgfqpoint{4.165628in}{2.025226in}}%
\pgfpathlineto{\pgfqpoint{4.157660in}{2.016291in}}%
\pgfpathlineto{\pgfqpoint{4.149686in}{2.007308in}}%
\pgfpathclose%
\pgfusepath{fill}%
\end{pgfscope}%
\begin{pgfscope}%
\pgfpathrectangle{\pgfqpoint{1.150000in}{0.150000in}}{\pgfqpoint{5.700000in}{5.700000in}}%
\pgfusepath{clip}%
\pgfsetbuttcap%
\pgfsetroundjoin%
\definecolor{currentfill}{rgb}{0.269944,0.014625,0.341379}%
\pgfsetfillcolor{currentfill}%
\pgfsetfillopacity{0.700000}%
\pgfsetlinewidth{0.000000pt}%
\definecolor{currentstroke}{rgb}{0.000000,0.000000,0.000000}%
\pgfsetstrokecolor{currentstroke}%
\pgfsetdash{}{0pt}%
\pgfpathmoveto{\pgfqpoint{3.053007in}{1.809870in}}%
\pgfpathlineto{\pgfqpoint{3.066675in}{1.804052in}}%
\pgfpathlineto{\pgfqpoint{3.080347in}{1.798325in}}%
\pgfpathlineto{\pgfqpoint{3.094022in}{1.792688in}}%
\pgfpathlineto{\pgfqpoint{3.107701in}{1.787142in}}%
\pgfpathlineto{\pgfqpoint{3.116085in}{1.794695in}}%
\pgfpathlineto{\pgfqpoint{3.124462in}{1.802317in}}%
\pgfpathlineto{\pgfqpoint{3.132831in}{1.810006in}}%
\pgfpathlineto{\pgfqpoint{3.141192in}{1.817758in}}%
\pgfpathlineto{\pgfqpoint{3.127531in}{1.823100in}}%
\pgfpathlineto{\pgfqpoint{3.113873in}{1.828532in}}%
\pgfpathlineto{\pgfqpoint{3.100220in}{1.834054in}}%
\pgfpathlineto{\pgfqpoint{3.086570in}{1.839667in}}%
\pgfpathlineto{\pgfqpoint{3.078191in}{1.832112in}}%
\pgfpathlineto{\pgfqpoint{3.069804in}{1.824626in}}%
\pgfpathlineto{\pgfqpoint{3.061410in}{1.817211in}}%
\pgfpathlineto{\pgfqpoint{3.053007in}{1.809870in}}%
\pgfpathclose%
\pgfusepath{fill}%
\end{pgfscope}%
\begin{pgfscope}%
\pgfpathrectangle{\pgfqpoint{1.150000in}{0.150000in}}{\pgfqpoint{5.700000in}{5.700000in}}%
\pgfusepath{clip}%
\pgfsetbuttcap%
\pgfsetroundjoin%
\definecolor{currentfill}{rgb}{0.265145,0.232956,0.516599}%
\pgfsetfillcolor{currentfill}%
\pgfsetfillopacity{0.700000}%
\pgfsetlinewidth{0.000000pt}%
\definecolor{currentstroke}{rgb}{0.000000,0.000000,0.000000}%
\pgfsetstrokecolor{currentstroke}%
\pgfsetdash{}{0pt}%
\pgfpathmoveto{\pgfqpoint{4.729915in}{2.226613in}}%
\pgfpathlineto{\pgfqpoint{4.744001in}{2.228200in}}%
\pgfpathlineto{\pgfqpoint{4.758097in}{2.229859in}}%
\pgfpathlineto{\pgfqpoint{4.772203in}{2.231589in}}%
\pgfpathlineto{\pgfqpoint{4.786320in}{2.233390in}}%
\pgfpathlineto{\pgfqpoint{4.794054in}{2.240571in}}%
\pgfpathlineto{\pgfqpoint{4.801780in}{2.247698in}}%
\pgfpathlineto{\pgfqpoint{4.809500in}{2.254772in}}%
\pgfpathlineto{\pgfqpoint{4.817213in}{2.261796in}}%
\pgfpathlineto{\pgfqpoint{4.803110in}{2.260123in}}%
\pgfpathlineto{\pgfqpoint{4.789017in}{2.258520in}}%
\pgfpathlineto{\pgfqpoint{4.774934in}{2.256989in}}%
\pgfpathlineto{\pgfqpoint{4.760862in}{2.255529in}}%
\pgfpathlineto{\pgfqpoint{4.753135in}{2.248370in}}%
\pgfpathlineto{\pgfqpoint{4.745401in}{2.241165in}}%
\pgfpathlineto{\pgfqpoint{4.737661in}{2.233914in}}%
\pgfpathlineto{\pgfqpoint{4.729915in}{2.226613in}}%
\pgfpathclose%
\pgfusepath{fill}%
\end{pgfscope}%
\begin{pgfscope}%
\pgfpathrectangle{\pgfqpoint{1.150000in}{0.150000in}}{\pgfqpoint{5.700000in}{5.700000in}}%
\pgfusepath{clip}%
\pgfsetbuttcap%
\pgfsetroundjoin%
\definecolor{currentfill}{rgb}{0.272594,0.025563,0.353093}%
\pgfsetfillcolor{currentfill}%
\pgfsetfillopacity{0.700000}%
\pgfsetlinewidth{0.000000pt}%
\definecolor{currentstroke}{rgb}{0.000000,0.000000,0.000000}%
\pgfsetstrokecolor{currentstroke}%
\pgfsetdash{}{0pt}%
\pgfpathmoveto{\pgfqpoint{2.909925in}{1.832998in}}%
\pgfpathlineto{\pgfqpoint{2.923589in}{1.826208in}}%
\pgfpathlineto{\pgfqpoint{2.937254in}{1.819513in}}%
\pgfpathlineto{\pgfqpoint{2.950923in}{1.812914in}}%
\pgfpathlineto{\pgfqpoint{2.964595in}{1.806408in}}%
\pgfpathlineto{\pgfqpoint{2.973051in}{1.813186in}}%
\pgfpathlineto{\pgfqpoint{2.981499in}{1.820057in}}%
\pgfpathlineto{\pgfqpoint{2.989938in}{1.827016in}}%
\pgfpathlineto{\pgfqpoint{2.998368in}{1.834062in}}%
\pgfpathlineto{\pgfqpoint{2.984717in}{1.840342in}}%
\pgfpathlineto{\pgfqpoint{2.971068in}{1.846716in}}%
\pgfpathlineto{\pgfqpoint{2.957423in}{1.853184in}}%
\pgfpathlineto{\pgfqpoint{2.943780in}{1.859747in}}%
\pgfpathlineto{\pgfqpoint{2.935330in}{1.852921in}}%
\pgfpathlineto{\pgfqpoint{2.926871in}{1.846184in}}%
\pgfpathlineto{\pgfqpoint{2.918403in}{1.839542in}}%
\pgfpathlineto{\pgfqpoint{2.909925in}{1.832998in}}%
\pgfpathclose%
\pgfusepath{fill}%
\end{pgfscope}%
\begin{pgfscope}%
\pgfpathrectangle{\pgfqpoint{1.150000in}{0.150000in}}{\pgfqpoint{5.700000in}{5.700000in}}%
\pgfusepath{clip}%
\pgfsetbuttcap%
\pgfsetroundjoin%
\definecolor{currentfill}{rgb}{0.235526,0.309527,0.542944}%
\pgfsetfillcolor{currentfill}%
\pgfsetfillopacity{0.700000}%
\pgfsetlinewidth{0.000000pt}%
\definecolor{currentstroke}{rgb}{0.000000,0.000000,0.000000}%
\pgfsetstrokecolor{currentstroke}%
\pgfsetdash{}{0pt}%
\pgfpathmoveto{\pgfqpoint{5.222825in}{2.403678in}}%
\pgfpathlineto{\pgfqpoint{5.237095in}{2.406101in}}%
\pgfpathlineto{\pgfqpoint{5.251376in}{2.408594in}}%
\pgfpathlineto{\pgfqpoint{5.265669in}{2.411156in}}%
\pgfpathlineto{\pgfqpoint{5.279973in}{2.413787in}}%
\pgfpathlineto{\pgfqpoint{5.287473in}{2.419093in}}%
\pgfpathlineto{\pgfqpoint{5.294966in}{2.424379in}}%
\pgfpathlineto{\pgfqpoint{5.302453in}{2.429651in}}%
\pgfpathlineto{\pgfqpoint{5.309933in}{2.434911in}}%
\pgfpathlineto{\pgfqpoint{5.295648in}{2.432515in}}%
\pgfpathlineto{\pgfqpoint{5.281374in}{2.430187in}}%
\pgfpathlineto{\pgfqpoint{5.267111in}{2.427929in}}%
\pgfpathlineto{\pgfqpoint{5.252861in}{2.425739in}}%
\pgfpathlineto{\pgfqpoint{5.245361in}{2.420237in}}%
\pgfpathlineto{\pgfqpoint{5.237855in}{2.414729in}}%
\pgfpathlineto{\pgfqpoint{5.230343in}{2.409211in}}%
\pgfpathlineto{\pgfqpoint{5.222825in}{2.403678in}}%
\pgfpathclose%
\pgfusepath{fill}%
\end{pgfscope}%
\begin{pgfscope}%
\pgfpathrectangle{\pgfqpoint{1.150000in}{0.150000in}}{\pgfqpoint{5.700000in}{5.700000in}}%
\pgfusepath{clip}%
\pgfsetbuttcap%
\pgfsetroundjoin%
\definecolor{currentfill}{rgb}{0.268510,0.009605,0.335427}%
\pgfsetfillcolor{currentfill}%
\pgfsetfillopacity{0.700000}%
\pgfsetlinewidth{0.000000pt}%
\definecolor{currentstroke}{rgb}{0.000000,0.000000,0.000000}%
\pgfsetstrokecolor{currentstroke}%
\pgfsetdash{}{0pt}%
\pgfpathmoveto{\pgfqpoint{3.195877in}{1.797282in}}%
\pgfpathlineto{\pgfqpoint{3.209559in}{1.792383in}}%
\pgfpathlineto{\pgfqpoint{3.223245in}{1.787572in}}%
\pgfpathlineto{\pgfqpoint{3.236936in}{1.782848in}}%
\pgfpathlineto{\pgfqpoint{3.250631in}{1.778210in}}%
\pgfpathlineto{\pgfqpoint{3.258952in}{1.786407in}}%
\pgfpathlineto{\pgfqpoint{3.267265in}{1.794652in}}%
\pgfpathlineto{\pgfqpoint{3.275571in}{1.802942in}}%
\pgfpathlineto{\pgfqpoint{3.283871in}{1.811275in}}%
\pgfpathlineto{\pgfqpoint{3.270191in}{1.815728in}}%
\pgfpathlineto{\pgfqpoint{3.256516in}{1.820268in}}%
\pgfpathlineto{\pgfqpoint{3.242846in}{1.824895in}}%
\pgfpathlineto{\pgfqpoint{3.229181in}{1.829610in}}%
\pgfpathlineto{\pgfqpoint{3.220866in}{1.821453in}}%
\pgfpathlineto{\pgfqpoint{3.212543in}{1.813345in}}%
\pgfpathlineto{\pgfqpoint{3.204214in}{1.805287in}}%
\pgfpathlineto{\pgfqpoint{3.195877in}{1.797282in}}%
\pgfpathclose%
\pgfusepath{fill}%
\end{pgfscope}%
\begin{pgfscope}%
\pgfpathrectangle{\pgfqpoint{1.150000in}{0.150000in}}{\pgfqpoint{5.700000in}{5.700000in}}%
\pgfusepath{clip}%
\pgfsetbuttcap%
\pgfsetroundjoin%
\definecolor{currentfill}{rgb}{0.276022,0.044167,0.370164}%
\pgfsetfillcolor{currentfill}%
\pgfsetfillopacity{0.700000}%
\pgfsetlinewidth{0.000000pt}%
\definecolor{currentstroke}{rgb}{0.000000,0.000000,0.000000}%
\pgfsetstrokecolor{currentstroke}%
\pgfsetdash{}{0pt}%
\pgfpathmoveto{\pgfqpoint{3.656830in}{1.850966in}}%
\pgfpathlineto{\pgfqpoint{3.670588in}{1.848680in}}%
\pgfpathlineto{\pgfqpoint{3.684353in}{1.846474in}}%
\pgfpathlineto{\pgfqpoint{3.698125in}{1.844347in}}%
\pgfpathlineto{\pgfqpoint{3.711904in}{1.842299in}}%
\pgfpathlineto{\pgfqpoint{3.720043in}{1.851660in}}%
\pgfpathlineto{\pgfqpoint{3.728177in}{1.861007in}}%
\pgfpathlineto{\pgfqpoint{3.736306in}{1.870338in}}%
\pgfpathlineto{\pgfqpoint{3.744428in}{1.879654in}}%
\pgfpathlineto{\pgfqpoint{3.730661in}{1.881600in}}%
\pgfpathlineto{\pgfqpoint{3.716901in}{1.883625in}}%
\pgfpathlineto{\pgfqpoint{3.703147in}{1.885729in}}%
\pgfpathlineto{\pgfqpoint{3.689400in}{1.887912in}}%
\pgfpathlineto{\pgfqpoint{3.681266in}{1.878691in}}%
\pgfpathlineto{\pgfqpoint{3.673126in}{1.869459in}}%
\pgfpathlineto{\pgfqpoint{3.664981in}{1.860217in}}%
\pgfpathlineto{\pgfqpoint{3.656830in}{1.850966in}}%
\pgfpathclose%
\pgfusepath{fill}%
\end{pgfscope}%
\begin{pgfscope}%
\pgfpathrectangle{\pgfqpoint{1.150000in}{0.150000in}}{\pgfqpoint{5.700000in}{5.700000in}}%
\pgfusepath{clip}%
\pgfsetbuttcap%
\pgfsetroundjoin%
\definecolor{currentfill}{rgb}{0.210503,0.363727,0.552206}%
\pgfsetfillcolor{currentfill}%
\pgfsetfillopacity{0.700000}%
\pgfsetlinewidth{0.000000pt}%
\definecolor{currentstroke}{rgb}{0.000000,0.000000,0.000000}%
\pgfsetstrokecolor{currentstroke}%
\pgfsetdash{}{0pt}%
\pgfpathmoveto{\pgfqpoint{5.628358in}{2.533306in}}%
\pgfpathlineto{\pgfqpoint{5.642779in}{2.536038in}}%
\pgfpathlineto{\pgfqpoint{5.657212in}{2.538838in}}%
\pgfpathlineto{\pgfqpoint{5.671658in}{2.541706in}}%
\pgfpathlineto{\pgfqpoint{5.686116in}{2.544643in}}%
\pgfpathlineto{\pgfqpoint{5.693406in}{2.548641in}}%
\pgfpathlineto{\pgfqpoint{5.700690in}{2.552673in}}%
\pgfpathlineto{\pgfqpoint{5.707969in}{2.556743in}}%
\pgfpathlineto{\pgfqpoint{5.715243in}{2.560857in}}%
\pgfpathlineto{\pgfqpoint{5.700810in}{2.558240in}}%
\pgfpathlineto{\pgfqpoint{5.686389in}{2.555691in}}%
\pgfpathlineto{\pgfqpoint{5.671981in}{2.553210in}}%
\pgfpathlineto{\pgfqpoint{5.657584in}{2.550796in}}%
\pgfpathlineto{\pgfqpoint{5.650285in}{2.546356in}}%
\pgfpathlineto{\pgfqpoint{5.642981in}{2.541964in}}%
\pgfpathlineto{\pgfqpoint{5.635672in}{2.537616in}}%
\pgfpathlineto{\pgfqpoint{5.628358in}{2.533306in}}%
\pgfpathclose%
\pgfusepath{fill}%
\end{pgfscope}%
\begin{pgfscope}%
\pgfpathrectangle{\pgfqpoint{1.150000in}{0.150000in}}{\pgfqpoint{5.700000in}{5.700000in}}%
\pgfusepath{clip}%
\pgfsetbuttcap%
\pgfsetroundjoin%
\definecolor{currentfill}{rgb}{0.283091,0.110553,0.431554}%
\pgfsetfillcolor{currentfill}%
\pgfsetfillopacity{0.700000}%
\pgfsetlinewidth{0.000000pt}%
\definecolor{currentstroke}{rgb}{0.000000,0.000000,0.000000}%
\pgfsetstrokecolor{currentstroke}%
\pgfsetdash{}{0pt}%
\pgfpathmoveto{\pgfqpoint{4.062223in}{1.972247in}}%
\pgfpathlineto{\pgfqpoint{4.076088in}{1.971795in}}%
\pgfpathlineto{\pgfqpoint{4.089961in}{1.971418in}}%
\pgfpathlineto{\pgfqpoint{4.103842in}{1.971116in}}%
\pgfpathlineto{\pgfqpoint{4.117731in}{1.970889in}}%
\pgfpathlineto{\pgfqpoint{4.125729in}{1.980067in}}%
\pgfpathlineto{\pgfqpoint{4.133720in}{1.989196in}}%
\pgfpathlineto{\pgfqpoint{4.141706in}{1.998276in}}%
\pgfpathlineto{\pgfqpoint{4.149686in}{2.007308in}}%
\pgfpathlineto{\pgfqpoint{4.135806in}{2.007515in}}%
\pgfpathlineto{\pgfqpoint{4.121935in}{2.007797in}}%
\pgfpathlineto{\pgfqpoint{4.108072in}{2.008155in}}%
\pgfpathlineto{\pgfqpoint{4.094218in}{2.008587in}}%
\pgfpathlineto{\pgfqpoint{4.086228in}{1.999568in}}%
\pgfpathlineto{\pgfqpoint{4.078232in}{1.990504in}}%
\pgfpathlineto{\pgfqpoint{4.070230in}{1.981398in}}%
\pgfpathlineto{\pgfqpoint{4.062223in}{1.972247in}}%
\pgfpathclose%
\pgfusepath{fill}%
\end{pgfscope}%
\begin{pgfscope}%
\pgfpathrectangle{\pgfqpoint{1.150000in}{0.150000in}}{\pgfqpoint{5.700000in}{5.700000in}}%
\pgfusepath{clip}%
\pgfsetbuttcap%
\pgfsetroundjoin%
\definecolor{currentfill}{rgb}{0.269308,0.218818,0.509577}%
\pgfsetfillcolor{currentfill}%
\pgfsetfillopacity{0.700000}%
\pgfsetlinewidth{0.000000pt}%
\definecolor{currentstroke}{rgb}{0.000000,0.000000,0.000000}%
\pgfsetstrokecolor{currentstroke}%
\pgfsetdash{}{0pt}%
\pgfpathmoveto{\pgfqpoint{4.642571in}{2.190802in}}%
\pgfpathlineto{\pgfqpoint{4.656629in}{2.192210in}}%
\pgfpathlineto{\pgfqpoint{4.670697in}{2.193689in}}%
\pgfpathlineto{\pgfqpoint{4.684774in}{2.195240in}}%
\pgfpathlineto{\pgfqpoint{4.698863in}{2.196863in}}%
\pgfpathlineto{\pgfqpoint{4.706636in}{2.204387in}}%
\pgfpathlineto{\pgfqpoint{4.714402in}{2.211851in}}%
\pgfpathlineto{\pgfqpoint{4.722162in}{2.219259in}}%
\pgfpathlineto{\pgfqpoint{4.729915in}{2.226613in}}%
\pgfpathlineto{\pgfqpoint{4.715839in}{2.225097in}}%
\pgfpathlineto{\pgfqpoint{4.701774in}{2.223652in}}%
\pgfpathlineto{\pgfqpoint{4.687719in}{2.222279in}}%
\pgfpathlineto{\pgfqpoint{4.673674in}{2.220978in}}%
\pgfpathlineto{\pgfqpoint{4.665908in}{2.213510in}}%
\pgfpathlineto{\pgfqpoint{4.658136in}{2.205993in}}%
\pgfpathlineto{\pgfqpoint{4.650357in}{2.198425in}}%
\pgfpathlineto{\pgfqpoint{4.642571in}{2.190802in}}%
\pgfpathclose%
\pgfusepath{fill}%
\end{pgfscope}%
\begin{pgfscope}%
\pgfpathrectangle{\pgfqpoint{1.150000in}{0.150000in}}{\pgfqpoint{5.700000in}{5.700000in}}%
\pgfusepath{clip}%
\pgfsetbuttcap%
\pgfsetroundjoin%
\definecolor{currentfill}{rgb}{0.281924,0.089666,0.412415}%
\pgfsetfillcolor{currentfill}%
\pgfsetfillopacity{0.700000}%
\pgfsetlinewidth{0.000000pt}%
\definecolor{currentstroke}{rgb}{0.000000,0.000000,0.000000}%
\pgfsetstrokecolor{currentstroke}%
\pgfsetdash{}{0pt}%
\pgfpathmoveto{\pgfqpoint{2.567944in}{1.951572in}}%
\pgfpathlineto{\pgfqpoint{2.581625in}{1.942233in}}%
\pgfpathlineto{\pgfqpoint{2.595307in}{1.933002in}}%
\pgfpathlineto{\pgfqpoint{2.608989in}{1.923879in}}%
\pgfpathlineto{\pgfqpoint{2.622672in}{1.914863in}}%
\pgfpathlineto{\pgfqpoint{2.631329in}{1.919395in}}%
\pgfpathlineto{\pgfqpoint{2.639974in}{1.924075in}}%
\pgfpathlineto{\pgfqpoint{2.648607in}{1.928898in}}%
\pgfpathlineto{\pgfqpoint{2.657229in}{1.933861in}}%
\pgfpathlineto{\pgfqpoint{2.643573in}{1.942608in}}%
\pgfpathlineto{\pgfqpoint{2.629917in}{1.951461in}}%
\pgfpathlineto{\pgfqpoint{2.616262in}{1.960421in}}%
\pgfpathlineto{\pgfqpoint{2.602607in}{1.969490in}}%
\pgfpathlineto{\pgfqpoint{2.593959in}{1.964789in}}%
\pgfpathlineto{\pgfqpoint{2.585300in}{1.960233in}}%
\pgfpathlineto{\pgfqpoint{2.576628in}{1.955826in}}%
\pgfpathlineto{\pgfqpoint{2.567944in}{1.951572in}}%
\pgfpathclose%
\pgfusepath{fill}%
\end{pgfscope}%
\begin{pgfscope}%
\pgfpathrectangle{\pgfqpoint{1.150000in}{0.150000in}}{\pgfqpoint{5.700000in}{5.700000in}}%
\pgfusepath{clip}%
\pgfsetbuttcap%
\pgfsetroundjoin%
\definecolor{currentfill}{rgb}{0.280868,0.160771,0.472899}%
\pgfsetfillcolor{currentfill}%
\pgfsetfillopacity{0.700000}%
\pgfsetlinewidth{0.000000pt}%
\definecolor{currentstroke}{rgb}{0.000000,0.000000,0.000000}%
\pgfsetstrokecolor{currentstroke}%
\pgfsetdash{}{0pt}%
\pgfpathmoveto{\pgfqpoint{2.313646in}{2.105704in}}%
\pgfpathlineto{\pgfqpoint{2.327369in}{2.094221in}}%
\pgfpathlineto{\pgfqpoint{2.341091in}{2.082860in}}%
\pgfpathlineto{\pgfqpoint{2.354811in}{2.071621in}}%
\pgfpathlineto{\pgfqpoint{2.368530in}{2.060503in}}%
\pgfpathlineto{\pgfqpoint{2.377357in}{2.063200in}}%
\pgfpathlineto{\pgfqpoint{2.386170in}{2.066083in}}%
\pgfpathlineto{\pgfqpoint{2.394969in}{2.069148in}}%
\pgfpathlineto{\pgfqpoint{2.403753in}{2.072390in}}%
\pgfpathlineto{\pgfqpoint{2.390066in}{2.083214in}}%
\pgfpathlineto{\pgfqpoint{2.376377in}{2.094158in}}%
\pgfpathlineto{\pgfqpoint{2.362687in}{2.105223in}}%
\pgfpathlineto{\pgfqpoint{2.348996in}{2.116411in}}%
\pgfpathlineto{\pgfqpoint{2.340181in}{2.113456in}}%
\pgfpathlineto{\pgfqpoint{2.331350in}{2.110684in}}%
\pgfpathlineto{\pgfqpoint{2.322506in}{2.108098in}}%
\pgfpathlineto{\pgfqpoint{2.313646in}{2.105704in}}%
\pgfpathclose%
\pgfusepath{fill}%
\end{pgfscope}%
\begin{pgfscope}%
\pgfpathrectangle{\pgfqpoint{1.150000in}{0.150000in}}{\pgfqpoint{5.700000in}{5.700000in}}%
\pgfusepath{clip}%
\pgfsetbuttcap%
\pgfsetroundjoin%
\definecolor{currentfill}{rgb}{0.276022,0.044167,0.370164}%
\pgfsetfillcolor{currentfill}%
\pgfsetfillopacity{0.700000}%
\pgfsetlinewidth{0.000000pt}%
\definecolor{currentstroke}{rgb}{0.000000,0.000000,0.000000}%
\pgfsetstrokecolor{currentstroke}%
\pgfsetdash{}{0pt}%
\pgfpathmoveto{\pgfqpoint{2.766522in}{1.867644in}}%
\pgfpathlineto{\pgfqpoint{2.780190in}{1.859825in}}%
\pgfpathlineto{\pgfqpoint{2.793860in}{1.852107in}}%
\pgfpathlineto{\pgfqpoint{2.807531in}{1.844488in}}%
\pgfpathlineto{\pgfqpoint{2.821205in}{1.836967in}}%
\pgfpathlineto{\pgfqpoint{2.829743in}{1.842834in}}%
\pgfpathlineto{\pgfqpoint{2.838271in}{1.848817in}}%
\pgfpathlineto{\pgfqpoint{2.846789in}{1.854912in}}%
\pgfpathlineto{\pgfqpoint{2.855298in}{1.861117in}}%
\pgfpathlineto{\pgfqpoint{2.841647in}{1.868390in}}%
\pgfpathlineto{\pgfqpoint{2.827998in}{1.875762in}}%
\pgfpathlineto{\pgfqpoint{2.814351in}{1.883233in}}%
\pgfpathlineto{\pgfqpoint{2.800706in}{1.890804in}}%
\pgfpathlineto{\pgfqpoint{2.792175in}{1.884838in}}%
\pgfpathlineto{\pgfqpoint{2.783634in}{1.878988in}}%
\pgfpathlineto{\pgfqpoint{2.775083in}{1.873255in}}%
\pgfpathlineto{\pgfqpoint{2.766522in}{1.867644in}}%
\pgfpathclose%
\pgfusepath{fill}%
\end{pgfscope}%
\begin{pgfscope}%
\pgfpathrectangle{\pgfqpoint{1.150000in}{0.150000in}}{\pgfqpoint{5.700000in}{5.700000in}}%
\pgfusepath{clip}%
\pgfsetbuttcap%
\pgfsetroundjoin%
\definecolor{currentfill}{rgb}{0.241237,0.296485,0.539709}%
\pgfsetfillcolor{currentfill}%
\pgfsetfillopacity{0.700000}%
\pgfsetlinewidth{0.000000pt}%
\definecolor{currentstroke}{rgb}{0.000000,0.000000,0.000000}%
\pgfsetstrokecolor{currentstroke}%
\pgfsetdash{}{0pt}%
\pgfpathmoveto{\pgfqpoint{5.135646in}{2.371473in}}%
\pgfpathlineto{\pgfqpoint{5.149888in}{2.373831in}}%
\pgfpathlineto{\pgfqpoint{5.164141in}{2.376259in}}%
\pgfpathlineto{\pgfqpoint{5.178406in}{2.378756in}}%
\pgfpathlineto{\pgfqpoint{5.192683in}{2.381323in}}%
\pgfpathlineto{\pgfqpoint{5.200229in}{2.386954in}}%
\pgfpathlineto{\pgfqpoint{5.207767in}{2.392554in}}%
\pgfpathlineto{\pgfqpoint{5.215299in}{2.398127in}}%
\pgfpathlineto{\pgfqpoint{5.222825in}{2.403678in}}%
\pgfpathlineto{\pgfqpoint{5.208566in}{2.401325in}}%
\pgfpathlineto{\pgfqpoint{5.194319in}{2.399040in}}%
\pgfpathlineto{\pgfqpoint{5.180083in}{2.396825in}}%
\pgfpathlineto{\pgfqpoint{5.165859in}{2.394680in}}%
\pgfpathlineto{\pgfqpoint{5.158316in}{2.388908in}}%
\pgfpathlineto{\pgfqpoint{5.150766in}{2.383119in}}%
\pgfpathlineto{\pgfqpoint{5.143209in}{2.377309in}}%
\pgfpathlineto{\pgfqpoint{5.135646in}{2.371473in}}%
\pgfpathclose%
\pgfusepath{fill}%
\end{pgfscope}%
\begin{pgfscope}%
\pgfpathrectangle{\pgfqpoint{1.150000in}{0.150000in}}{\pgfqpoint{5.700000in}{5.700000in}}%
\pgfusepath{clip}%
\pgfsetbuttcap%
\pgfsetroundjoin%
\definecolor{currentfill}{rgb}{0.282656,0.100196,0.422160}%
\pgfsetfillcolor{currentfill}%
\pgfsetfillopacity{0.700000}%
\pgfsetlinewidth{0.000000pt}%
\definecolor{currentstroke}{rgb}{0.000000,0.000000,0.000000}%
\pgfsetstrokecolor{currentstroke}%
\pgfsetdash{}{0pt}%
\pgfpathmoveto{\pgfqpoint{3.974719in}{1.937934in}}%
\pgfpathlineto{\pgfqpoint{3.988562in}{1.937138in}}%
\pgfpathlineto{\pgfqpoint{4.002412in}{1.936419in}}%
\pgfpathlineto{\pgfqpoint{4.016271in}{1.935775in}}%
\pgfpathlineto{\pgfqpoint{4.030137in}{1.935207in}}%
\pgfpathlineto{\pgfqpoint{4.038167in}{1.944532in}}%
\pgfpathlineto{\pgfqpoint{4.046191in}{1.953814in}}%
\pgfpathlineto{\pgfqpoint{4.054210in}{1.963052in}}%
\pgfpathlineto{\pgfqpoint{4.062223in}{1.972247in}}%
\pgfpathlineto{\pgfqpoint{4.048367in}{1.972775in}}%
\pgfpathlineto{\pgfqpoint{4.034518in}{1.973378in}}%
\pgfpathlineto{\pgfqpoint{4.020678in}{1.974058in}}%
\pgfpathlineto{\pgfqpoint{4.006846in}{1.974813in}}%
\pgfpathlineto{\pgfqpoint{3.998823in}{1.965651in}}%
\pgfpathlineto{\pgfqpoint{3.990794in}{1.956451in}}%
\pgfpathlineto{\pgfqpoint{3.982759in}{1.947212in}}%
\pgfpathlineto{\pgfqpoint{3.974719in}{1.937934in}}%
\pgfpathclose%
\pgfusepath{fill}%
\end{pgfscope}%
\begin{pgfscope}%
\pgfpathrectangle{\pgfqpoint{1.150000in}{0.150000in}}{\pgfqpoint{5.700000in}{5.700000in}}%
\pgfusepath{clip}%
\pgfsetbuttcap%
\pgfsetroundjoin%
\definecolor{currentfill}{rgb}{0.273006,0.204520,0.501721}%
\pgfsetfillcolor{currentfill}%
\pgfsetfillopacity{0.700000}%
\pgfsetlinewidth{0.000000pt}%
\definecolor{currentstroke}{rgb}{0.000000,0.000000,0.000000}%
\pgfsetstrokecolor{currentstroke}%
\pgfsetdash{}{0pt}%
\pgfpathmoveto{\pgfqpoint{4.555186in}{2.154482in}}%
\pgfpathlineto{\pgfqpoint{4.569216in}{2.155688in}}%
\pgfpathlineto{\pgfqpoint{4.583255in}{2.156965in}}%
\pgfpathlineto{\pgfqpoint{4.597305in}{2.158314in}}%
\pgfpathlineto{\pgfqpoint{4.611364in}{2.159736in}}%
\pgfpathlineto{\pgfqpoint{4.619176in}{2.167593in}}%
\pgfpathlineto{\pgfqpoint{4.626981in}{2.175388in}}%
\pgfpathlineto{\pgfqpoint{4.634779in}{2.183124in}}%
\pgfpathlineto{\pgfqpoint{4.642571in}{2.190802in}}%
\pgfpathlineto{\pgfqpoint{4.628524in}{2.189466in}}%
\pgfpathlineto{\pgfqpoint{4.614486in}{2.188202in}}%
\pgfpathlineto{\pgfqpoint{4.600459in}{2.187010in}}%
\pgfpathlineto{\pgfqpoint{4.586441in}{2.185890in}}%
\pgfpathlineto{\pgfqpoint{4.578637in}{2.178119in}}%
\pgfpathlineto{\pgfqpoint{4.570826in}{2.170295in}}%
\pgfpathlineto{\pgfqpoint{4.563009in}{2.162417in}}%
\pgfpathlineto{\pgfqpoint{4.555186in}{2.154482in}}%
\pgfpathclose%
\pgfusepath{fill}%
\end{pgfscope}%
\begin{pgfscope}%
\pgfpathrectangle{\pgfqpoint{1.150000in}{0.150000in}}{\pgfqpoint{5.700000in}{5.700000in}}%
\pgfusepath{clip}%
\pgfsetbuttcap%
\pgfsetroundjoin%
\definecolor{currentfill}{rgb}{0.269944,0.014625,0.341379}%
\pgfsetfillcolor{currentfill}%
\pgfsetfillopacity{0.700000}%
\pgfsetlinewidth{0.000000pt}%
\definecolor{currentstroke}{rgb}{0.000000,0.000000,0.000000}%
\pgfsetstrokecolor{currentstroke}%
\pgfsetdash{}{0pt}%
\pgfpathmoveto{\pgfqpoint{3.338636in}{1.794319in}}%
\pgfpathlineto{\pgfqpoint{3.352340in}{1.790293in}}%
\pgfpathlineto{\pgfqpoint{3.366049in}{1.786351in}}%
\pgfpathlineto{\pgfqpoint{3.379764in}{1.782493in}}%
\pgfpathlineto{\pgfqpoint{3.393483in}{1.778718in}}%
\pgfpathlineto{\pgfqpoint{3.401746in}{1.787434in}}%
\pgfpathlineto{\pgfqpoint{3.410002in}{1.796178in}}%
\pgfpathlineto{\pgfqpoint{3.418252in}{1.804947in}}%
\pgfpathlineto{\pgfqpoint{3.426496in}{1.813740in}}%
\pgfpathlineto{\pgfqpoint{3.412790in}{1.817350in}}%
\pgfpathlineto{\pgfqpoint{3.399090in}{1.821045in}}%
\pgfpathlineto{\pgfqpoint{3.385396in}{1.824823in}}%
\pgfpathlineto{\pgfqpoint{3.371706in}{1.828686in}}%
\pgfpathlineto{\pgfqpoint{3.363449in}{1.820049in}}%
\pgfpathlineto{\pgfqpoint{3.355185in}{1.811441in}}%
\pgfpathlineto{\pgfqpoint{3.346914in}{1.802863in}}%
\pgfpathlineto{\pgfqpoint{3.338636in}{1.794319in}}%
\pgfpathclose%
\pgfusepath{fill}%
\end{pgfscope}%
\begin{pgfscope}%
\pgfpathrectangle{\pgfqpoint{1.150000in}{0.150000in}}{\pgfqpoint{5.700000in}{5.700000in}}%
\pgfusepath{clip}%
\pgfsetbuttcap%
\pgfsetroundjoin%
\definecolor{currentfill}{rgb}{0.273809,0.031497,0.358853}%
\pgfsetfillcolor{currentfill}%
\pgfsetfillopacity{0.700000}%
\pgfsetlinewidth{0.000000pt}%
\definecolor{currentstroke}{rgb}{0.000000,0.000000,0.000000}%
\pgfsetstrokecolor{currentstroke}%
\pgfsetdash{}{0pt}%
\pgfpathmoveto{\pgfqpoint{3.569149in}{1.824336in}}%
\pgfpathlineto{\pgfqpoint{3.582894in}{1.821607in}}%
\pgfpathlineto{\pgfqpoint{3.596645in}{1.818959in}}%
\pgfpathlineto{\pgfqpoint{3.610403in}{1.816391in}}%
\pgfpathlineto{\pgfqpoint{3.624167in}{1.813903in}}%
\pgfpathlineto{\pgfqpoint{3.632341in}{1.823174in}}%
\pgfpathlineto{\pgfqpoint{3.640510in}{1.832443in}}%
\pgfpathlineto{\pgfqpoint{3.648673in}{1.841707in}}%
\pgfpathlineto{\pgfqpoint{3.656830in}{1.850966in}}%
\pgfpathlineto{\pgfqpoint{3.643078in}{1.853331in}}%
\pgfpathlineto{\pgfqpoint{3.629332in}{1.855777in}}%
\pgfpathlineto{\pgfqpoint{3.615593in}{1.858303in}}%
\pgfpathlineto{\pgfqpoint{3.601861in}{1.860909in}}%
\pgfpathlineto{\pgfqpoint{3.593692in}{1.851766in}}%
\pgfpathlineto{\pgfqpoint{3.585517in}{1.842621in}}%
\pgfpathlineto{\pgfqpoint{3.577336in}{1.833478in}}%
\pgfpathlineto{\pgfqpoint{3.569149in}{1.824336in}}%
\pgfpathclose%
\pgfusepath{fill}%
\end{pgfscope}%
\begin{pgfscope}%
\pgfpathrectangle{\pgfqpoint{1.150000in}{0.150000in}}{\pgfqpoint{5.700000in}{5.700000in}}%
\pgfusepath{clip}%
\pgfsetbuttcap%
\pgfsetroundjoin%
\definecolor{currentfill}{rgb}{0.214298,0.355619,0.551184}%
\pgfsetfillcolor{currentfill}%
\pgfsetfillopacity{0.700000}%
\pgfsetlinewidth{0.000000pt}%
\definecolor{currentstroke}{rgb}{0.000000,0.000000,0.000000}%
\pgfsetstrokecolor{currentstroke}%
\pgfsetdash{}{0pt}%
\pgfpathmoveto{\pgfqpoint{5.541386in}{2.504897in}}%
\pgfpathlineto{\pgfqpoint{5.555782in}{2.507655in}}%
\pgfpathlineto{\pgfqpoint{5.570191in}{2.510481in}}%
\pgfpathlineto{\pgfqpoint{5.584612in}{2.513375in}}%
\pgfpathlineto{\pgfqpoint{5.599045in}{2.516338in}}%
\pgfpathlineto{\pgfqpoint{5.606382in}{2.520550in}}%
\pgfpathlineto{\pgfqpoint{5.613713in}{2.524778in}}%
\pgfpathlineto{\pgfqpoint{5.621038in}{2.529028in}}%
\pgfpathlineto{\pgfqpoint{5.628358in}{2.533306in}}%
\pgfpathlineto{\pgfqpoint{5.613949in}{2.530642in}}%
\pgfpathlineto{\pgfqpoint{5.599552in}{2.528046in}}%
\pgfpathlineto{\pgfqpoint{5.585167in}{2.525518in}}%
\pgfpathlineto{\pgfqpoint{5.570794in}{2.523058in}}%
\pgfpathlineto{\pgfqpoint{5.563450in}{2.518475in}}%
\pgfpathlineto{\pgfqpoint{5.556101in}{2.513924in}}%
\pgfpathlineto{\pgfqpoint{5.548746in}{2.509400in}}%
\pgfpathlineto{\pgfqpoint{5.541386in}{2.504897in}}%
\pgfpathclose%
\pgfusepath{fill}%
\end{pgfscope}%
\begin{pgfscope}%
\pgfpathrectangle{\pgfqpoint{1.150000in}{0.150000in}}{\pgfqpoint{5.700000in}{5.700000in}}%
\pgfusepath{clip}%
\pgfsetbuttcap%
\pgfsetroundjoin%
\definecolor{currentfill}{rgb}{0.277134,0.185228,0.489898}%
\pgfsetfillcolor{currentfill}%
\pgfsetfillopacity{0.700000}%
\pgfsetlinewidth{0.000000pt}%
\definecolor{currentstroke}{rgb}{0.000000,0.000000,0.000000}%
\pgfsetstrokecolor{currentstroke}%
\pgfsetdash{}{0pt}%
\pgfpathmoveto{\pgfqpoint{4.467763in}{2.117792in}}%
\pgfpathlineto{\pgfqpoint{4.481765in}{2.118772in}}%
\pgfpathlineto{\pgfqpoint{4.495777in}{2.119825in}}%
\pgfpathlineto{\pgfqpoint{4.509798in}{2.120950in}}%
\pgfpathlineto{\pgfqpoint{4.523830in}{2.122147in}}%
\pgfpathlineto{\pgfqpoint{4.531678in}{2.130324in}}%
\pgfpathlineto{\pgfqpoint{4.539521in}{2.138437in}}%
\pgfpathlineto{\pgfqpoint{4.547357in}{2.146490in}}%
\pgfpathlineto{\pgfqpoint{4.555186in}{2.154482in}}%
\pgfpathlineto{\pgfqpoint{4.541166in}{2.153349in}}%
\pgfpathlineto{\pgfqpoint{4.527156in}{2.152289in}}%
\pgfpathlineto{\pgfqpoint{4.513156in}{2.151301in}}%
\pgfpathlineto{\pgfqpoint{4.499166in}{2.150385in}}%
\pgfpathlineto{\pgfqpoint{4.491324in}{2.142320in}}%
\pgfpathlineto{\pgfqpoint{4.483477in}{2.134201in}}%
\pgfpathlineto{\pgfqpoint{4.475623in}{2.126025in}}%
\pgfpathlineto{\pgfqpoint{4.467763in}{2.117792in}}%
\pgfpathclose%
\pgfusepath{fill}%
\end{pgfscope}%
\begin{pgfscope}%
\pgfpathrectangle{\pgfqpoint{1.150000in}{0.150000in}}{\pgfqpoint{5.700000in}{5.700000in}}%
\pgfusepath{clip}%
\pgfsetbuttcap%
\pgfsetroundjoin%
\definecolor{currentfill}{rgb}{0.282290,0.145912,0.461510}%
\pgfsetfillcolor{currentfill}%
\pgfsetfillopacity{0.700000}%
\pgfsetlinewidth{0.000000pt}%
\definecolor{currentstroke}{rgb}{0.000000,0.000000,0.000000}%
\pgfsetstrokecolor{currentstroke}%
\pgfsetdash{}{0pt}%
\pgfpathmoveto{\pgfqpoint{2.368530in}{2.060503in}}%
\pgfpathlineto{\pgfqpoint{2.382247in}{2.049504in}}%
\pgfpathlineto{\pgfqpoint{2.395963in}{2.038625in}}%
\pgfpathlineto{\pgfqpoint{2.409678in}{2.027863in}}%
\pgfpathlineto{\pgfqpoint{2.423392in}{2.017219in}}%
\pgfpathlineto{\pgfqpoint{2.432187in}{2.020218in}}%
\pgfpathlineto{\pgfqpoint{2.440969in}{2.023398in}}%
\pgfpathlineto{\pgfqpoint{2.449737in}{2.026754in}}%
\pgfpathlineto{\pgfqpoint{2.458491in}{2.030282in}}%
\pgfpathlineto{\pgfqpoint{2.444808in}{2.040633in}}%
\pgfpathlineto{\pgfqpoint{2.431124in}{2.051101in}}%
\pgfpathlineto{\pgfqpoint{2.417439in}{2.061686in}}%
\pgfpathlineto{\pgfqpoint{2.403753in}{2.072390in}}%
\pgfpathlineto{\pgfqpoint{2.394969in}{2.069148in}}%
\pgfpathlineto{\pgfqpoint{2.386170in}{2.066083in}}%
\pgfpathlineto{\pgfqpoint{2.377357in}{2.063200in}}%
\pgfpathlineto{\pgfqpoint{2.368530in}{2.060503in}}%
\pgfpathclose%
\pgfusepath{fill}%
\end{pgfscope}%
\begin{pgfscope}%
\pgfpathrectangle{\pgfqpoint{1.150000in}{0.150000in}}{\pgfqpoint{5.700000in}{5.700000in}}%
\pgfusepath{clip}%
\pgfsetbuttcap%
\pgfsetroundjoin%
\definecolor{currentfill}{rgb}{0.244972,0.287675,0.537260}%
\pgfsetfillcolor{currentfill}%
\pgfsetfillopacity{0.700000}%
\pgfsetlinewidth{0.000000pt}%
\definecolor{currentstroke}{rgb}{0.000000,0.000000,0.000000}%
\pgfsetstrokecolor{currentstroke}%
\pgfsetdash{}{0pt}%
\pgfpathmoveto{\pgfqpoint{5.048400in}{2.338302in}}%
\pgfpathlineto{\pgfqpoint{5.062614in}{2.340573in}}%
\pgfpathlineto{\pgfqpoint{5.076840in}{2.342913in}}%
\pgfpathlineto{\pgfqpoint{5.091076in}{2.345324in}}%
\pgfpathlineto{\pgfqpoint{5.105324in}{2.347804in}}%
\pgfpathlineto{\pgfqpoint{5.112915in}{2.353778in}}%
\pgfpathlineto{\pgfqpoint{5.120499in}{2.359711in}}%
\pgfpathlineto{\pgfqpoint{5.128076in}{2.365608in}}%
\pgfpathlineto{\pgfqpoint{5.135646in}{2.371473in}}%
\pgfpathlineto{\pgfqpoint{5.121415in}{2.369185in}}%
\pgfpathlineto{\pgfqpoint{5.107195in}{2.366966in}}%
\pgfpathlineto{\pgfqpoint{5.092986in}{2.364818in}}%
\pgfpathlineto{\pgfqpoint{5.078789in}{2.362739in}}%
\pgfpathlineto{\pgfqpoint{5.071202in}{2.356674in}}%
\pgfpathlineto{\pgfqpoint{5.063608in}{2.350583in}}%
\pgfpathlineto{\pgfqpoint{5.056008in}{2.344460in}}%
\pgfpathlineto{\pgfqpoint{5.048400in}{2.338302in}}%
\pgfpathclose%
\pgfusepath{fill}%
\end{pgfscope}%
\begin{pgfscope}%
\pgfpathrectangle{\pgfqpoint{1.150000in}{0.150000in}}{\pgfqpoint{5.700000in}{5.700000in}}%
\pgfusepath{clip}%
\pgfsetbuttcap%
\pgfsetroundjoin%
\definecolor{currentfill}{rgb}{0.281446,0.084320,0.407414}%
\pgfsetfillcolor{currentfill}%
\pgfsetfillopacity{0.700000}%
\pgfsetlinewidth{0.000000pt}%
\definecolor{currentstroke}{rgb}{0.000000,0.000000,0.000000}%
\pgfsetstrokecolor{currentstroke}%
\pgfsetdash{}{0pt}%
\pgfpathmoveto{\pgfqpoint{3.887170in}{1.904640in}}%
\pgfpathlineto{\pgfqpoint{3.900992in}{1.903477in}}%
\pgfpathlineto{\pgfqpoint{3.914821in}{1.902391in}}%
\pgfpathlineto{\pgfqpoint{3.928658in}{1.901382in}}%
\pgfpathlineto{\pgfqpoint{3.942503in}{1.900449in}}%
\pgfpathlineto{\pgfqpoint{3.950565in}{1.909875in}}%
\pgfpathlineto{\pgfqpoint{3.958622in}{1.919266in}}%
\pgfpathlineto{\pgfqpoint{3.966674in}{1.928619in}}%
\pgfpathlineto{\pgfqpoint{3.974719in}{1.937934in}}%
\pgfpathlineto{\pgfqpoint{3.960885in}{1.938806in}}%
\pgfpathlineto{\pgfqpoint{3.947058in}{1.939755in}}%
\pgfpathlineto{\pgfqpoint{3.933239in}{1.940780in}}%
\pgfpathlineto{\pgfqpoint{3.919428in}{1.941882in}}%
\pgfpathlineto{\pgfqpoint{3.911372in}{1.932620in}}%
\pgfpathlineto{\pgfqpoint{3.903310in}{1.923325in}}%
\pgfpathlineto{\pgfqpoint{3.895243in}{1.913998in}}%
\pgfpathlineto{\pgfqpoint{3.887170in}{1.904640in}}%
\pgfpathclose%
\pgfusepath{fill}%
\end{pgfscope}%
\begin{pgfscope}%
\pgfpathrectangle{\pgfqpoint{1.150000in}{0.150000in}}{\pgfqpoint{5.700000in}{5.700000in}}%
\pgfusepath{clip}%
\pgfsetbuttcap%
\pgfsetroundjoin%
\definecolor{currentfill}{rgb}{0.271305,0.019942,0.347269}%
\pgfsetfillcolor{currentfill}%
\pgfsetfillopacity{0.700000}%
\pgfsetlinewidth{0.000000pt}%
\definecolor{currentstroke}{rgb}{0.000000,0.000000,0.000000}%
\pgfsetstrokecolor{currentstroke}%
\pgfsetdash{}{0pt}%
\pgfpathmoveto{\pgfqpoint{2.964595in}{1.806408in}}%
\pgfpathlineto{\pgfqpoint{2.978269in}{1.799995in}}%
\pgfpathlineto{\pgfqpoint{2.991947in}{1.793675in}}%
\pgfpathlineto{\pgfqpoint{3.005628in}{1.787448in}}%
\pgfpathlineto{\pgfqpoint{3.019312in}{1.781313in}}%
\pgfpathlineto{\pgfqpoint{3.027748in}{1.788325in}}%
\pgfpathlineto{\pgfqpoint{3.036176in}{1.795424in}}%
\pgfpathlineto{\pgfqpoint{3.044596in}{1.802607in}}%
\pgfpathlineto{\pgfqpoint{3.053007in}{1.809870in}}%
\pgfpathlineto{\pgfqpoint{3.039342in}{1.815780in}}%
\pgfpathlineto{\pgfqpoint{3.025681in}{1.821781in}}%
\pgfpathlineto{\pgfqpoint{3.012023in}{1.827875in}}%
\pgfpathlineto{\pgfqpoint{2.998368in}{1.834062in}}%
\pgfpathlineto{\pgfqpoint{2.989938in}{1.827016in}}%
\pgfpathlineto{\pgfqpoint{2.981499in}{1.820057in}}%
\pgfpathlineto{\pgfqpoint{2.973051in}{1.813186in}}%
\pgfpathlineto{\pgfqpoint{2.964595in}{1.806408in}}%
\pgfpathclose%
\pgfusepath{fill}%
\end{pgfscope}%
\begin{pgfscope}%
\pgfpathrectangle{\pgfqpoint{1.150000in}{0.150000in}}{\pgfqpoint{5.700000in}{5.700000in}}%
\pgfusepath{clip}%
\pgfsetbuttcap%
\pgfsetroundjoin%
\definecolor{currentfill}{rgb}{0.280894,0.078907,0.402329}%
\pgfsetfillcolor{currentfill}%
\pgfsetfillopacity{0.700000}%
\pgfsetlinewidth{0.000000pt}%
\definecolor{currentstroke}{rgb}{0.000000,0.000000,0.000000}%
\pgfsetstrokecolor{currentstroke}%
\pgfsetdash{}{0pt}%
\pgfpathmoveto{\pgfqpoint{2.622672in}{1.914863in}}%
\pgfpathlineto{\pgfqpoint{2.636356in}{1.905952in}}%
\pgfpathlineto{\pgfqpoint{2.650040in}{1.897147in}}%
\pgfpathlineto{\pgfqpoint{2.663726in}{1.888447in}}%
\pgfpathlineto{\pgfqpoint{2.677412in}{1.879851in}}%
\pgfpathlineto{\pgfqpoint{2.686042in}{1.884660in}}%
\pgfpathlineto{\pgfqpoint{2.694661in}{1.889612in}}%
\pgfpathlineto{\pgfqpoint{2.703269in}{1.894703in}}%
\pgfpathlineto{\pgfqpoint{2.711865in}{1.899927in}}%
\pgfpathlineto{\pgfqpoint{2.698205in}{1.908255in}}%
\pgfpathlineto{\pgfqpoint{2.684545in}{1.916685in}}%
\pgfpathlineto{\pgfqpoint{2.670887in}{1.925221in}}%
\pgfpathlineto{\pgfqpoint{2.657229in}{1.933861in}}%
\pgfpathlineto{\pgfqpoint{2.648607in}{1.928898in}}%
\pgfpathlineto{\pgfqpoint{2.639974in}{1.924075in}}%
\pgfpathlineto{\pgfqpoint{2.631329in}{1.919395in}}%
\pgfpathlineto{\pgfqpoint{2.622672in}{1.914863in}}%
\pgfpathclose%
\pgfusepath{fill}%
\end{pgfscope}%
\begin{pgfscope}%
\pgfpathrectangle{\pgfqpoint{1.150000in}{0.150000in}}{\pgfqpoint{5.700000in}{5.700000in}}%
\pgfusepath{clip}%
\pgfsetbuttcap%
\pgfsetroundjoin%
\definecolor{currentfill}{rgb}{0.268510,0.009605,0.335427}%
\pgfsetfillcolor{currentfill}%
\pgfsetfillopacity{0.700000}%
\pgfsetlinewidth{0.000000pt}%
\definecolor{currentstroke}{rgb}{0.000000,0.000000,0.000000}%
\pgfsetstrokecolor{currentstroke}%
\pgfsetdash{}{0pt}%
\pgfpathmoveto{\pgfqpoint{3.107701in}{1.787142in}}%
\pgfpathlineto{\pgfqpoint{3.121384in}{1.781684in}}%
\pgfpathlineto{\pgfqpoint{3.135071in}{1.776316in}}%
\pgfpathlineto{\pgfqpoint{3.148761in}{1.771037in}}%
\pgfpathlineto{\pgfqpoint{3.162456in}{1.765846in}}%
\pgfpathlineto{\pgfqpoint{3.170823in}{1.773612in}}%
\pgfpathlineto{\pgfqpoint{3.179182in}{1.781441in}}%
\pgfpathlineto{\pgfqpoint{3.187533in}{1.789333in}}%
\pgfpathlineto{\pgfqpoint{3.195877in}{1.797282in}}%
\pgfpathlineto{\pgfqpoint{3.182200in}{1.802268in}}%
\pgfpathlineto{\pgfqpoint{3.168526in}{1.807343in}}%
\pgfpathlineto{\pgfqpoint{3.154857in}{1.812506in}}%
\pgfpathlineto{\pgfqpoint{3.141192in}{1.817758in}}%
\pgfpathlineto{\pgfqpoint{3.132831in}{1.810006in}}%
\pgfpathlineto{\pgfqpoint{3.124462in}{1.802317in}}%
\pgfpathlineto{\pgfqpoint{3.116085in}{1.794695in}}%
\pgfpathlineto{\pgfqpoint{3.107701in}{1.787142in}}%
\pgfpathclose%
\pgfusepath{fill}%
\end{pgfscope}%
\begin{pgfscope}%
\pgfpathrectangle{\pgfqpoint{1.150000in}{0.150000in}}{\pgfqpoint{5.700000in}{5.700000in}}%
\pgfusepath{clip}%
\pgfsetbuttcap%
\pgfsetroundjoin%
\definecolor{currentfill}{rgb}{0.279574,0.170599,0.479997}%
\pgfsetfillcolor{currentfill}%
\pgfsetfillopacity{0.700000}%
\pgfsetlinewidth{0.000000pt}%
\definecolor{currentstroke}{rgb}{0.000000,0.000000,0.000000}%
\pgfsetstrokecolor{currentstroke}%
\pgfsetdash{}{0pt}%
\pgfpathmoveto{\pgfqpoint{4.380305in}{2.080894in}}%
\pgfpathlineto{\pgfqpoint{4.394280in}{2.081626in}}%
\pgfpathlineto{\pgfqpoint{4.408264in}{2.082430in}}%
\pgfpathlineto{\pgfqpoint{4.422258in}{2.083308in}}%
\pgfpathlineto{\pgfqpoint{4.436261in}{2.084259in}}%
\pgfpathlineto{\pgfqpoint{4.444146in}{2.092735in}}%
\pgfpathlineto{\pgfqpoint{4.452025in}{2.101148in}}%
\pgfpathlineto{\pgfqpoint{4.459897in}{2.109501in}}%
\pgfpathlineto{\pgfqpoint{4.467763in}{2.117792in}}%
\pgfpathlineto{\pgfqpoint{4.453771in}{2.116885in}}%
\pgfpathlineto{\pgfqpoint{4.439788in}{2.116051in}}%
\pgfpathlineto{\pgfqpoint{4.425815in}{2.115290in}}%
\pgfpathlineto{\pgfqpoint{4.411851in}{2.114601in}}%
\pgfpathlineto{\pgfqpoint{4.403974in}{2.106258in}}%
\pgfpathlineto{\pgfqpoint{4.396090in}{2.097860in}}%
\pgfpathlineto{\pgfqpoint{4.388201in}{2.089406in}}%
\pgfpathlineto{\pgfqpoint{4.380305in}{2.080894in}}%
\pgfpathclose%
\pgfusepath{fill}%
\end{pgfscope}%
\begin{pgfscope}%
\pgfpathrectangle{\pgfqpoint{1.150000in}{0.150000in}}{\pgfqpoint{5.700000in}{5.700000in}}%
\pgfusepath{clip}%
\pgfsetbuttcap%
\pgfsetroundjoin%
\definecolor{currentfill}{rgb}{0.220057,0.343307,0.549413}%
\pgfsetfillcolor{currentfill}%
\pgfsetfillopacity{0.700000}%
\pgfsetlinewidth{0.000000pt}%
\definecolor{currentstroke}{rgb}{0.000000,0.000000,0.000000}%
\pgfsetstrokecolor{currentstroke}%
\pgfsetdash{}{0pt}%
\pgfpathmoveto{\pgfqpoint{5.454329in}{2.475545in}}%
\pgfpathlineto{\pgfqpoint{5.468700in}{2.478306in}}%
\pgfpathlineto{\pgfqpoint{5.483082in}{2.481136in}}%
\pgfpathlineto{\pgfqpoint{5.497477in}{2.484034in}}%
\pgfpathlineto{\pgfqpoint{5.511884in}{2.487001in}}%
\pgfpathlineto{\pgfqpoint{5.519269in}{2.491468in}}%
\pgfpathlineto{\pgfqpoint{5.526647in}{2.495936in}}%
\pgfpathlineto{\pgfqpoint{5.534020in}{2.500411in}}%
\pgfpathlineto{\pgfqpoint{5.541386in}{2.504897in}}%
\pgfpathlineto{\pgfqpoint{5.527002in}{2.502208in}}%
\pgfpathlineto{\pgfqpoint{5.512629in}{2.499587in}}%
\pgfpathlineto{\pgfqpoint{5.498269in}{2.497035in}}%
\pgfpathlineto{\pgfqpoint{5.483920in}{2.494551in}}%
\pgfpathlineto{\pgfqpoint{5.476532in}{2.489780in}}%
\pgfpathlineto{\pgfqpoint{5.469137in}{2.485025in}}%
\pgfpathlineto{\pgfqpoint{5.461736in}{2.480282in}}%
\pgfpathlineto{\pgfqpoint{5.454329in}{2.475545in}}%
\pgfpathclose%
\pgfusepath{fill}%
\end{pgfscope}%
\begin{pgfscope}%
\pgfpathrectangle{\pgfqpoint{1.150000in}{0.150000in}}{\pgfqpoint{5.700000in}{5.700000in}}%
\pgfusepath{clip}%
\pgfsetbuttcap%
\pgfsetroundjoin%
\definecolor{currentfill}{rgb}{0.250425,0.274290,0.533103}%
\pgfsetfillcolor{currentfill}%
\pgfsetfillopacity{0.700000}%
\pgfsetlinewidth{0.000000pt}%
\definecolor{currentstroke}{rgb}{0.000000,0.000000,0.000000}%
\pgfsetstrokecolor{currentstroke}%
\pgfsetdash{}{0pt}%
\pgfpathmoveto{\pgfqpoint{4.961094in}{2.304194in}}%
\pgfpathlineto{\pgfqpoint{4.975280in}{2.306354in}}%
\pgfpathlineto{\pgfqpoint{4.989476in}{2.308585in}}%
\pgfpathlineto{\pgfqpoint{5.003684in}{2.310886in}}%
\pgfpathlineto{\pgfqpoint{5.017902in}{2.313257in}}%
\pgfpathlineto{\pgfqpoint{5.025537in}{2.319587in}}%
\pgfpathlineto{\pgfqpoint{5.033165in}{2.325869in}}%
\pgfpathlineto{\pgfqpoint{5.040786in}{2.332106in}}%
\pgfpathlineto{\pgfqpoint{5.048400in}{2.338302in}}%
\pgfpathlineto{\pgfqpoint{5.034198in}{2.336102in}}%
\pgfpathlineto{\pgfqpoint{5.020006in}{2.333971in}}%
\pgfpathlineto{\pgfqpoint{5.005825in}{2.331911in}}%
\pgfpathlineto{\pgfqpoint{4.991655in}{2.329921in}}%
\pgfpathlineto{\pgfqpoint{4.984025in}{2.323547in}}%
\pgfpathlineto{\pgfqpoint{4.976388in}{2.317137in}}%
\pgfpathlineto{\pgfqpoint{4.968745in}{2.310686in}}%
\pgfpathlineto{\pgfqpoint{4.961094in}{2.304194in}}%
\pgfpathclose%
\pgfusepath{fill}%
\end{pgfscope}%
\begin{pgfscope}%
\pgfpathrectangle{\pgfqpoint{1.150000in}{0.150000in}}{\pgfqpoint{5.700000in}{5.700000in}}%
\pgfusepath{clip}%
\pgfsetbuttcap%
\pgfsetroundjoin%
\definecolor{currentfill}{rgb}{0.279566,0.067836,0.391917}%
\pgfsetfillcolor{currentfill}%
\pgfsetfillopacity{0.700000}%
\pgfsetlinewidth{0.000000pt}%
\definecolor{currentstroke}{rgb}{0.000000,0.000000,0.000000}%
\pgfsetstrokecolor{currentstroke}%
\pgfsetdash{}{0pt}%
\pgfpathmoveto{\pgfqpoint{3.799567in}{1.872657in}}%
\pgfpathlineto{\pgfqpoint{3.813370in}{1.871103in}}%
\pgfpathlineto{\pgfqpoint{3.827180in}{1.869627in}}%
\pgfpathlineto{\pgfqpoint{3.840997in}{1.868228in}}%
\pgfpathlineto{\pgfqpoint{3.854822in}{1.866907in}}%
\pgfpathlineto{\pgfqpoint{3.862917in}{1.876383in}}%
\pgfpathlineto{\pgfqpoint{3.871007in}{1.885832in}}%
\pgfpathlineto{\pgfqpoint{3.879091in}{1.895251in}}%
\pgfpathlineto{\pgfqpoint{3.887170in}{1.904640in}}%
\pgfpathlineto{\pgfqpoint{3.873356in}{1.905880in}}%
\pgfpathlineto{\pgfqpoint{3.859549in}{1.907197in}}%
\pgfpathlineto{\pgfqpoint{3.845750in}{1.908592in}}%
\pgfpathlineto{\pgfqpoint{3.831958in}{1.910065in}}%
\pgfpathlineto{\pgfqpoint{3.823869in}{1.900749in}}%
\pgfpathlineto{\pgfqpoint{3.815774in}{1.891409in}}%
\pgfpathlineto{\pgfqpoint{3.807673in}{1.882045in}}%
\pgfpathlineto{\pgfqpoint{3.799567in}{1.872657in}}%
\pgfpathclose%
\pgfusepath{fill}%
\end{pgfscope}%
\begin{pgfscope}%
\pgfpathrectangle{\pgfqpoint{1.150000in}{0.150000in}}{\pgfqpoint{5.700000in}{5.700000in}}%
\pgfusepath{clip}%
\pgfsetbuttcap%
\pgfsetroundjoin%
\definecolor{currentfill}{rgb}{0.272594,0.025563,0.353093}%
\pgfsetfillcolor{currentfill}%
\pgfsetfillopacity{0.700000}%
\pgfsetlinewidth{0.000000pt}%
\definecolor{currentstroke}{rgb}{0.000000,0.000000,0.000000}%
\pgfsetstrokecolor{currentstroke}%
\pgfsetdash{}{0pt}%
\pgfpathmoveto{\pgfqpoint{3.481373in}{1.800126in}}%
\pgfpathlineto{\pgfqpoint{3.495107in}{1.796929in}}%
\pgfpathlineto{\pgfqpoint{3.508846in}{1.793814in}}%
\pgfpathlineto{\pgfqpoint{3.522592in}{1.790779in}}%
\pgfpathlineto{\pgfqpoint{3.536343in}{1.787826in}}%
\pgfpathlineto{\pgfqpoint{3.544554in}{1.796942in}}%
\pgfpathlineto{\pgfqpoint{3.552758in}{1.806067in}}%
\pgfpathlineto{\pgfqpoint{3.560957in}{1.815199in}}%
\pgfpathlineto{\pgfqpoint{3.569149in}{1.824336in}}%
\pgfpathlineto{\pgfqpoint{3.555411in}{1.827146in}}%
\pgfpathlineto{\pgfqpoint{3.541678in}{1.830037in}}%
\pgfpathlineto{\pgfqpoint{3.527952in}{1.833010in}}%
\pgfpathlineto{\pgfqpoint{3.514231in}{1.836064in}}%
\pgfpathlineto{\pgfqpoint{3.506026in}{1.827062in}}%
\pgfpathlineto{\pgfqpoint{3.497814in}{1.818070in}}%
\pgfpathlineto{\pgfqpoint{3.489597in}{1.809091in}}%
\pgfpathlineto{\pgfqpoint{3.481373in}{1.800126in}}%
\pgfpathclose%
\pgfusepath{fill}%
\end{pgfscope}%
\begin{pgfscope}%
\pgfpathrectangle{\pgfqpoint{1.150000in}{0.150000in}}{\pgfqpoint{5.700000in}{5.700000in}}%
\pgfusepath{clip}%
\pgfsetbuttcap%
\pgfsetroundjoin%
\definecolor{currentfill}{rgb}{0.274952,0.037752,0.364543}%
\pgfsetfillcolor{currentfill}%
\pgfsetfillopacity{0.700000}%
\pgfsetlinewidth{0.000000pt}%
\definecolor{currentstroke}{rgb}{0.000000,0.000000,0.000000}%
\pgfsetstrokecolor{currentstroke}%
\pgfsetdash{}{0pt}%
\pgfpathmoveto{\pgfqpoint{2.821205in}{1.836967in}}%
\pgfpathlineto{\pgfqpoint{2.834881in}{1.829545in}}%
\pgfpathlineto{\pgfqpoint{2.848559in}{1.822219in}}%
\pgfpathlineto{\pgfqpoint{2.862239in}{1.814991in}}%
\pgfpathlineto{\pgfqpoint{2.875922in}{1.807859in}}%
\pgfpathlineto{\pgfqpoint{2.884437in}{1.813980in}}%
\pgfpathlineto{\pgfqpoint{2.892943in}{1.820213in}}%
\pgfpathlineto{\pgfqpoint{2.901439in}{1.826553in}}%
\pgfpathlineto{\pgfqpoint{2.909925in}{1.832998in}}%
\pgfpathlineto{\pgfqpoint{2.896265in}{1.839883in}}%
\pgfpathlineto{\pgfqpoint{2.882607in}{1.846864in}}%
\pgfpathlineto{\pgfqpoint{2.868951in}{1.853942in}}%
\pgfpathlineto{\pgfqpoint{2.855298in}{1.861117in}}%
\pgfpathlineto{\pgfqpoint{2.846789in}{1.854912in}}%
\pgfpathlineto{\pgfqpoint{2.838271in}{1.848817in}}%
\pgfpathlineto{\pgfqpoint{2.829743in}{1.842834in}}%
\pgfpathlineto{\pgfqpoint{2.821205in}{1.836967in}}%
\pgfpathclose%
\pgfusepath{fill}%
\end{pgfscope}%
\begin{pgfscope}%
\pgfpathrectangle{\pgfqpoint{1.150000in}{0.150000in}}{\pgfqpoint{5.700000in}{5.700000in}}%
\pgfusepath{clip}%
\pgfsetbuttcap%
\pgfsetroundjoin%
\definecolor{currentfill}{rgb}{0.281412,0.155834,0.469201}%
\pgfsetfillcolor{currentfill}%
\pgfsetfillopacity{0.700000}%
\pgfsetlinewidth{0.000000pt}%
\definecolor{currentstroke}{rgb}{0.000000,0.000000,0.000000}%
\pgfsetstrokecolor{currentstroke}%
\pgfsetdash{}{0pt}%
\pgfpathmoveto{\pgfqpoint{4.292814in}{2.043971in}}%
\pgfpathlineto{\pgfqpoint{4.306762in}{2.044431in}}%
\pgfpathlineto{\pgfqpoint{4.320720in}{2.044965in}}%
\pgfpathlineto{\pgfqpoint{4.334687in}{2.045572in}}%
\pgfpathlineto{\pgfqpoint{4.348662in}{2.046253in}}%
\pgfpathlineto{\pgfqpoint{4.356582in}{2.055004in}}%
\pgfpathlineto{\pgfqpoint{4.364496in}{2.063694in}}%
\pgfpathlineto{\pgfqpoint{4.372404in}{2.072324in}}%
\pgfpathlineto{\pgfqpoint{4.380305in}{2.080894in}}%
\pgfpathlineto{\pgfqpoint{4.366340in}{2.080236in}}%
\pgfpathlineto{\pgfqpoint{4.352384in}{2.079651in}}%
\pgfpathlineto{\pgfqpoint{4.338437in}{2.079139in}}%
\pgfpathlineto{\pgfqpoint{4.324499in}{2.078701in}}%
\pgfpathlineto{\pgfqpoint{4.316587in}{2.070101in}}%
\pgfpathlineto{\pgfqpoint{4.308668in}{2.061446in}}%
\pgfpathlineto{\pgfqpoint{4.300744in}{2.052736in}}%
\pgfpathlineto{\pgfqpoint{4.292814in}{2.043971in}}%
\pgfpathclose%
\pgfusepath{fill}%
\end{pgfscope}%
\begin{pgfscope}%
\pgfpathrectangle{\pgfqpoint{1.150000in}{0.150000in}}{\pgfqpoint{5.700000in}{5.700000in}}%
\pgfusepath{clip}%
\pgfsetbuttcap%
\pgfsetroundjoin%
\definecolor{currentfill}{rgb}{0.268510,0.009605,0.335427}%
\pgfsetfillcolor{currentfill}%
\pgfsetfillopacity{0.700000}%
\pgfsetlinewidth{0.000000pt}%
\definecolor{currentstroke}{rgb}{0.000000,0.000000,0.000000}%
\pgfsetstrokecolor{currentstroke}%
\pgfsetdash{}{0pt}%
\pgfpathmoveto{\pgfqpoint{3.250631in}{1.778210in}}%
\pgfpathlineto{\pgfqpoint{3.264331in}{1.773659in}}%
\pgfpathlineto{\pgfqpoint{3.278036in}{1.769193in}}%
\pgfpathlineto{\pgfqpoint{3.291745in}{1.764813in}}%
\pgfpathlineto{\pgfqpoint{3.305459in}{1.760518in}}%
\pgfpathlineto{\pgfqpoint{3.313764in}{1.768907in}}%
\pgfpathlineto{\pgfqpoint{3.322062in}{1.777338in}}%
\pgfpathlineto{\pgfqpoint{3.330352in}{1.785810in}}%
\pgfpathlineto{\pgfqpoint{3.338636in}{1.794319in}}%
\pgfpathlineto{\pgfqpoint{3.324938in}{1.798430in}}%
\pgfpathlineto{\pgfqpoint{3.311244in}{1.802626in}}%
\pgfpathlineto{\pgfqpoint{3.297555in}{1.806908in}}%
\pgfpathlineto{\pgfqpoint{3.283871in}{1.811275in}}%
\pgfpathlineto{\pgfqpoint{3.275571in}{1.802942in}}%
\pgfpathlineto{\pgfqpoint{3.267265in}{1.794652in}}%
\pgfpathlineto{\pgfqpoint{3.258952in}{1.786407in}}%
\pgfpathlineto{\pgfqpoint{3.250631in}{1.778210in}}%
\pgfpathclose%
\pgfusepath{fill}%
\end{pgfscope}%
\begin{pgfscope}%
\pgfpathrectangle{\pgfqpoint{1.150000in}{0.150000in}}{\pgfqpoint{5.700000in}{5.700000in}}%
\pgfusepath{clip}%
\pgfsetbuttcap%
\pgfsetroundjoin%
\definecolor{currentfill}{rgb}{0.201239,0.383670,0.554294}%
\pgfsetfillcolor{currentfill}%
\pgfsetfillopacity{0.700000}%
\pgfsetlinewidth{0.000000pt}%
\definecolor{currentstroke}{rgb}{0.000000,0.000000,0.000000}%
\pgfsetstrokecolor{currentstroke}%
\pgfsetdash{}{0pt}%
\pgfpathmoveto{\pgfqpoint{5.773097in}{2.572002in}}%
\pgfpathlineto{\pgfqpoint{5.787592in}{2.574958in}}%
\pgfpathlineto{\pgfqpoint{5.802099in}{2.577981in}}%
\pgfpathlineto{\pgfqpoint{5.816619in}{2.581073in}}%
\pgfpathlineto{\pgfqpoint{5.823841in}{2.584654in}}%
\pgfpathlineto{\pgfqpoint{5.831059in}{2.588283in}}%
\pgfpathlineto{\pgfqpoint{5.838272in}{2.591965in}}%
\pgfpathlineto{\pgfqpoint{5.845481in}{2.595706in}}%
\pgfpathlineto{\pgfqpoint{5.830988in}{2.592956in}}%
\pgfpathlineto{\pgfqpoint{5.816508in}{2.590274in}}%
\pgfpathlineto{\pgfqpoint{5.802040in}{2.587659in}}%
\pgfpathlineto{\pgfqpoint{5.794811in}{2.583656in}}%
\pgfpathlineto{\pgfqpoint{5.787578in}{2.579717in}}%
\pgfpathlineto{\pgfqpoint{5.780340in}{2.575834in}}%
\pgfpathlineto{\pgfqpoint{5.773097in}{2.572002in}}%
\pgfpathclose%
\pgfusepath{fill}%
\end{pgfscope}%
\begin{pgfscope}%
\pgfpathrectangle{\pgfqpoint{1.150000in}{0.150000in}}{\pgfqpoint{5.700000in}{5.700000in}}%
\pgfusepath{clip}%
\pgfsetbuttcap%
\pgfsetroundjoin%
\definecolor{currentfill}{rgb}{0.283187,0.125848,0.444960}%
\pgfsetfillcolor{currentfill}%
\pgfsetfillopacity{0.700000}%
\pgfsetlinewidth{0.000000pt}%
\definecolor{currentstroke}{rgb}{0.000000,0.000000,0.000000}%
\pgfsetstrokecolor{currentstroke}%
\pgfsetdash{}{0pt}%
\pgfpathmoveto{\pgfqpoint{2.423392in}{2.017219in}}%
\pgfpathlineto{\pgfqpoint{2.437105in}{2.006690in}}%
\pgfpathlineto{\pgfqpoint{2.450817in}{1.996277in}}%
\pgfpathlineto{\pgfqpoint{2.464528in}{1.985978in}}%
\pgfpathlineto{\pgfqpoint{2.478239in}{1.975792in}}%
\pgfpathlineto{\pgfqpoint{2.487005in}{1.979093in}}%
\pgfpathlineto{\pgfqpoint{2.495756in}{1.982568in}}%
\pgfpathlineto{\pgfqpoint{2.504494in}{1.986215in}}%
\pgfpathlineto{\pgfqpoint{2.513219in}{1.990028in}}%
\pgfpathlineto{\pgfqpoint{2.499538in}{1.999921in}}%
\pgfpathlineto{\pgfqpoint{2.485856in}{2.009927in}}%
\pgfpathlineto{\pgfqpoint{2.472174in}{2.020047in}}%
\pgfpathlineto{\pgfqpoint{2.458491in}{2.030282in}}%
\pgfpathlineto{\pgfqpoint{2.449737in}{2.026754in}}%
\pgfpathlineto{\pgfqpoint{2.440969in}{2.023398in}}%
\pgfpathlineto{\pgfqpoint{2.432187in}{2.020218in}}%
\pgfpathlineto{\pgfqpoint{2.423392in}{2.017219in}}%
\pgfpathclose%
\pgfusepath{fill}%
\end{pgfscope}%
\begin{pgfscope}%
\pgfpathrectangle{\pgfqpoint{1.150000in}{0.150000in}}{\pgfqpoint{5.700000in}{5.700000in}}%
\pgfusepath{clip}%
\pgfsetbuttcap%
\pgfsetroundjoin%
\definecolor{currentfill}{rgb}{0.257322,0.256130,0.526563}%
\pgfsetfillcolor{currentfill}%
\pgfsetfillopacity{0.700000}%
\pgfsetlinewidth{0.000000pt}%
\definecolor{currentstroke}{rgb}{0.000000,0.000000,0.000000}%
\pgfsetstrokecolor{currentstroke}%
\pgfsetdash{}{0pt}%
\pgfpathmoveto{\pgfqpoint{2.057546in}{2.304582in}}%
\pgfpathlineto{\pgfqpoint{2.071355in}{2.290660in}}%
\pgfpathlineto{\pgfqpoint{2.085159in}{2.276879in}}%
\pgfpathlineto{\pgfqpoint{2.098959in}{2.263239in}}%
\pgfpathlineto{\pgfqpoint{2.112755in}{2.249738in}}%
\pgfpathlineto{\pgfqpoint{2.121787in}{2.250336in}}%
\pgfpathlineto{\pgfqpoint{2.130801in}{2.251161in}}%
\pgfpathlineto{\pgfqpoint{2.139798in}{2.252209in}}%
\pgfpathlineto{\pgfqpoint{2.148778in}{2.253473in}}%
\pgfpathlineto{\pgfqpoint{2.135019in}{2.266653in}}%
\pgfpathlineto{\pgfqpoint{2.121256in}{2.279970in}}%
\pgfpathlineto{\pgfqpoint{2.107490in}{2.293428in}}%
\pgfpathlineto{\pgfqpoint{2.093720in}{2.307028in}}%
\pgfpathlineto{\pgfqpoint{2.084703in}{2.306077in}}%
\pgfpathlineto{\pgfqpoint{2.075669in}{2.305350in}}%
\pgfpathlineto{\pgfqpoint{2.066617in}{2.304849in}}%
\pgfpathlineto{\pgfqpoint{2.057546in}{2.304582in}}%
\pgfpathclose%
\pgfusepath{fill}%
\end{pgfscope}%
\begin{pgfscope}%
\pgfpathrectangle{\pgfqpoint{1.150000in}{0.150000in}}{\pgfqpoint{5.700000in}{5.700000in}}%
\pgfusepath{clip}%
\pgfsetbuttcap%
\pgfsetroundjoin%
\definecolor{currentfill}{rgb}{0.255645,0.260703,0.528312}%
\pgfsetfillcolor{currentfill}%
\pgfsetfillopacity{0.700000}%
\pgfsetlinewidth{0.000000pt}%
\definecolor{currentstroke}{rgb}{0.000000,0.000000,0.000000}%
\pgfsetstrokecolor{currentstroke}%
\pgfsetdash{}{0pt}%
\pgfpathmoveto{\pgfqpoint{4.873732in}{2.269198in}}%
\pgfpathlineto{\pgfqpoint{4.887889in}{2.271226in}}%
\pgfpathlineto{\pgfqpoint{4.902056in}{2.273325in}}%
\pgfpathlineto{\pgfqpoint{4.916235in}{2.275494in}}%
\pgfpathlineto{\pgfqpoint{4.930424in}{2.277733in}}%
\pgfpathlineto{\pgfqpoint{4.938102in}{2.284428in}}%
\pgfpathlineto{\pgfqpoint{4.945773in}{2.291067in}}%
\pgfpathlineto{\pgfqpoint{4.953437in}{2.297655in}}%
\pgfpathlineto{\pgfqpoint{4.961094in}{2.304194in}}%
\pgfpathlineto{\pgfqpoint{4.946920in}{2.302104in}}%
\pgfpathlineto{\pgfqpoint{4.932756in}{2.300084in}}%
\pgfpathlineto{\pgfqpoint{4.918603in}{2.298135in}}%
\pgfpathlineto{\pgfqpoint{4.904461in}{2.296257in}}%
\pgfpathlineto{\pgfqpoint{4.896789in}{2.289561in}}%
\pgfpathlineto{\pgfqpoint{4.889110in}{2.282821in}}%
\pgfpathlineto{\pgfqpoint{4.881425in}{2.276034in}}%
\pgfpathlineto{\pgfqpoint{4.873732in}{2.269198in}}%
\pgfpathclose%
\pgfusepath{fill}%
\end{pgfscope}%
\begin{pgfscope}%
\pgfpathrectangle{\pgfqpoint{1.150000in}{0.150000in}}{\pgfqpoint{5.700000in}{5.700000in}}%
\pgfusepath{clip}%
\pgfsetbuttcap%
\pgfsetroundjoin%
\definecolor{currentfill}{rgb}{0.282623,0.140926,0.457517}%
\pgfsetfillcolor{currentfill}%
\pgfsetfillopacity{0.700000}%
\pgfsetlinewidth{0.000000pt}%
\definecolor{currentstroke}{rgb}{0.000000,0.000000,0.000000}%
\pgfsetstrokecolor{currentstroke}%
\pgfsetdash{}{0pt}%
\pgfpathmoveto{\pgfqpoint{4.205290in}{2.007227in}}%
\pgfpathlineto{\pgfqpoint{4.219212in}{2.007393in}}%
\pgfpathlineto{\pgfqpoint{4.233144in}{2.007633in}}%
\pgfpathlineto{\pgfqpoint{4.247084in}{2.007947in}}%
\pgfpathlineto{\pgfqpoint{4.261034in}{2.008335in}}%
\pgfpathlineto{\pgfqpoint{4.268988in}{2.017331in}}%
\pgfpathlineto{\pgfqpoint{4.276936in}{2.026268in}}%
\pgfpathlineto{\pgfqpoint{4.284878in}{2.035148in}}%
\pgfpathlineto{\pgfqpoint{4.292814in}{2.043971in}}%
\pgfpathlineto{\pgfqpoint{4.278875in}{2.043584in}}%
\pgfpathlineto{\pgfqpoint{4.264944in}{2.043271in}}%
\pgfpathlineto{\pgfqpoint{4.251023in}{2.043033in}}%
\pgfpathlineto{\pgfqpoint{4.237110in}{2.042868in}}%
\pgfpathlineto{\pgfqpoint{4.229164in}{2.034037in}}%
\pgfpathlineto{\pgfqpoint{4.221212in}{2.025153in}}%
\pgfpathlineto{\pgfqpoint{4.213254in}{2.016216in}}%
\pgfpathlineto{\pgfqpoint{4.205290in}{2.007227in}}%
\pgfpathclose%
\pgfusepath{fill}%
\end{pgfscope}%
\begin{pgfscope}%
\pgfpathrectangle{\pgfqpoint{1.150000in}{0.150000in}}{\pgfqpoint{5.700000in}{5.700000in}}%
\pgfusepath{clip}%
\pgfsetbuttcap%
\pgfsetroundjoin%
\definecolor{currentfill}{rgb}{0.223925,0.334994,0.548053}%
\pgfsetfillcolor{currentfill}%
\pgfsetfillopacity{0.700000}%
\pgfsetlinewidth{0.000000pt}%
\definecolor{currentstroke}{rgb}{0.000000,0.000000,0.000000}%
\pgfsetstrokecolor{currentstroke}%
\pgfsetdash{}{0pt}%
\pgfpathmoveto{\pgfqpoint{5.367190in}{2.445188in}}%
\pgfpathlineto{\pgfqpoint{5.381534in}{2.447930in}}%
\pgfpathlineto{\pgfqpoint{5.395890in}{2.450741in}}%
\pgfpathlineto{\pgfqpoint{5.410257in}{2.453621in}}%
\pgfpathlineto{\pgfqpoint{5.424637in}{2.456570in}}%
\pgfpathlineto{\pgfqpoint{5.432070in}{2.461327in}}%
\pgfpathlineto{\pgfqpoint{5.439496in}{2.466073in}}%
\pgfpathlineto{\pgfqpoint{5.446916in}{2.470810in}}%
\pgfpathlineto{\pgfqpoint{5.454329in}{2.475545in}}%
\pgfpathlineto{\pgfqpoint{5.439971in}{2.472853in}}%
\pgfpathlineto{\pgfqpoint{5.425624in}{2.470230in}}%
\pgfpathlineto{\pgfqpoint{5.411289in}{2.467675in}}%
\pgfpathlineto{\pgfqpoint{5.396966in}{2.465189in}}%
\pgfpathlineto{\pgfqpoint{5.389532in}{2.460191in}}%
\pgfpathlineto{\pgfqpoint{5.382091in}{2.455194in}}%
\pgfpathlineto{\pgfqpoint{5.374644in}{2.450195in}}%
\pgfpathlineto{\pgfqpoint{5.367190in}{2.445188in}}%
\pgfpathclose%
\pgfusepath{fill}%
\end{pgfscope}%
\begin{pgfscope}%
\pgfpathrectangle{\pgfqpoint{1.150000in}{0.150000in}}{\pgfqpoint{5.700000in}{5.700000in}}%
\pgfusepath{clip}%
\pgfsetbuttcap%
\pgfsetroundjoin%
\definecolor{currentfill}{rgb}{0.277941,0.056324,0.381191}%
\pgfsetfillcolor{currentfill}%
\pgfsetfillopacity{0.700000}%
\pgfsetlinewidth{0.000000pt}%
\definecolor{currentstroke}{rgb}{0.000000,0.000000,0.000000}%
\pgfsetstrokecolor{currentstroke}%
\pgfsetdash{}{0pt}%
\pgfpathmoveto{\pgfqpoint{3.711904in}{1.842299in}}%
\pgfpathlineto{\pgfqpoint{3.725689in}{1.840330in}}%
\pgfpathlineto{\pgfqpoint{3.739481in}{1.838440in}}%
\pgfpathlineto{\pgfqpoint{3.753281in}{1.836628in}}%
\pgfpathlineto{\pgfqpoint{3.767087in}{1.834894in}}%
\pgfpathlineto{\pgfqpoint{3.775216in}{1.844364in}}%
\pgfpathlineto{\pgfqpoint{3.783339in}{1.853816in}}%
\pgfpathlineto{\pgfqpoint{3.791456in}{1.863247in}}%
\pgfpathlineto{\pgfqpoint{3.799567in}{1.872657in}}%
\pgfpathlineto{\pgfqpoint{3.785772in}{1.874289in}}%
\pgfpathlineto{\pgfqpoint{3.771984in}{1.875999in}}%
\pgfpathlineto{\pgfqpoint{3.758203in}{1.877787in}}%
\pgfpathlineto{\pgfqpoint{3.744428in}{1.879654in}}%
\pgfpathlineto{\pgfqpoint{3.736306in}{1.870338in}}%
\pgfpathlineto{\pgfqpoint{3.728177in}{1.861007in}}%
\pgfpathlineto{\pgfqpoint{3.720043in}{1.851660in}}%
\pgfpathlineto{\pgfqpoint{3.711904in}{1.842299in}}%
\pgfpathclose%
\pgfusepath{fill}%
\end{pgfscope}%
\begin{pgfscope}%
\pgfpathrectangle{\pgfqpoint{1.150000in}{0.150000in}}{\pgfqpoint{5.700000in}{5.700000in}}%
\pgfusepath{clip}%
\pgfsetbuttcap%
\pgfsetroundjoin%
\definecolor{currentfill}{rgb}{0.265145,0.232956,0.516599}%
\pgfsetfillcolor{currentfill}%
\pgfsetfillopacity{0.700000}%
\pgfsetlinewidth{0.000000pt}%
\definecolor{currentstroke}{rgb}{0.000000,0.000000,0.000000}%
\pgfsetstrokecolor{currentstroke}%
\pgfsetdash{}{0pt}%
\pgfpathmoveto{\pgfqpoint{2.112755in}{2.249738in}}%
\pgfpathlineto{\pgfqpoint{2.126547in}{2.236375in}}%
\pgfpathlineto{\pgfqpoint{2.140336in}{2.223148in}}%
\pgfpathlineto{\pgfqpoint{2.154121in}{2.210057in}}%
\pgfpathlineto{\pgfqpoint{2.167902in}{2.197100in}}%
\pgfpathlineto{\pgfqpoint{2.176897in}{2.198028in}}%
\pgfpathlineto{\pgfqpoint{2.185874in}{2.199177in}}%
\pgfpathlineto{\pgfqpoint{2.194835in}{2.200542in}}%
\pgfpathlineto{\pgfqpoint{2.203779in}{2.202119in}}%
\pgfpathlineto{\pgfqpoint{2.190033in}{2.214756in}}%
\pgfpathlineto{\pgfqpoint{2.176285in}{2.227526in}}%
\pgfpathlineto{\pgfqpoint{2.162533in}{2.240432in}}%
\pgfpathlineto{\pgfqpoint{2.148778in}{2.253473in}}%
\pgfpathlineto{\pgfqpoint{2.139798in}{2.252209in}}%
\pgfpathlineto{\pgfqpoint{2.130801in}{2.251161in}}%
\pgfpathlineto{\pgfqpoint{2.121787in}{2.250336in}}%
\pgfpathlineto{\pgfqpoint{2.112755in}{2.249738in}}%
\pgfpathclose%
\pgfusepath{fill}%
\end{pgfscope}%
\begin{pgfscope}%
\pgfpathrectangle{\pgfqpoint{1.150000in}{0.150000in}}{\pgfqpoint{5.700000in}{5.700000in}}%
\pgfusepath{clip}%
\pgfsetbuttcap%
\pgfsetroundjoin%
\definecolor{currentfill}{rgb}{0.260571,0.246922,0.522828}%
\pgfsetfillcolor{currentfill}%
\pgfsetfillopacity{0.700000}%
\pgfsetlinewidth{0.000000pt}%
\definecolor{currentstroke}{rgb}{0.000000,0.000000,0.000000}%
\pgfsetstrokecolor{currentstroke}%
\pgfsetdash{}{0pt}%
\pgfpathmoveto{\pgfqpoint{4.786320in}{2.233390in}}%
\pgfpathlineto{\pgfqpoint{4.800448in}{2.235262in}}%
\pgfpathlineto{\pgfqpoint{4.814586in}{2.237206in}}%
\pgfpathlineto{\pgfqpoint{4.828735in}{2.239220in}}%
\pgfpathlineto{\pgfqpoint{4.842894in}{2.241306in}}%
\pgfpathlineto{\pgfqpoint{4.850614in}{2.248366in}}%
\pgfpathlineto{\pgfqpoint{4.858327in}{2.255367in}}%
\pgfpathlineto{\pgfqpoint{4.866033in}{2.262310in}}%
\pgfpathlineto{\pgfqpoint{4.873732in}{2.269198in}}%
\pgfpathlineto{\pgfqpoint{4.859587in}{2.267241in}}%
\pgfpathlineto{\pgfqpoint{4.845452in}{2.265355in}}%
\pgfpathlineto{\pgfqpoint{4.831327in}{2.263540in}}%
\pgfpathlineto{\pgfqpoint{4.817213in}{2.261796in}}%
\pgfpathlineto{\pgfqpoint{4.809500in}{2.254772in}}%
\pgfpathlineto{\pgfqpoint{4.801780in}{2.247698in}}%
\pgfpathlineto{\pgfqpoint{4.794054in}{2.240571in}}%
\pgfpathlineto{\pgfqpoint{4.786320in}{2.233390in}}%
\pgfpathclose%
\pgfusepath{fill}%
\end{pgfscope}%
\begin{pgfscope}%
\pgfpathrectangle{\pgfqpoint{1.150000in}{0.150000in}}{\pgfqpoint{5.700000in}{5.700000in}}%
\pgfusepath{clip}%
\pgfsetbuttcap%
\pgfsetroundjoin%
\definecolor{currentfill}{rgb}{0.278791,0.062145,0.386592}%
\pgfsetfillcolor{currentfill}%
\pgfsetfillopacity{0.700000}%
\pgfsetlinewidth{0.000000pt}%
\definecolor{currentstroke}{rgb}{0.000000,0.000000,0.000000}%
\pgfsetstrokecolor{currentstroke}%
\pgfsetdash{}{0pt}%
\pgfpathmoveto{\pgfqpoint{2.677412in}{1.879851in}}%
\pgfpathlineto{\pgfqpoint{2.691099in}{1.871358in}}%
\pgfpathlineto{\pgfqpoint{2.704788in}{1.862967in}}%
\pgfpathlineto{\pgfqpoint{2.718477in}{1.854678in}}%
\pgfpathlineto{\pgfqpoint{2.732169in}{1.846491in}}%
\pgfpathlineto{\pgfqpoint{2.740773in}{1.851577in}}%
\pgfpathlineto{\pgfqpoint{2.749367in}{1.856801in}}%
\pgfpathlineto{\pgfqpoint{2.757950in}{1.862158in}}%
\pgfpathlineto{\pgfqpoint{2.766522in}{1.867644in}}%
\pgfpathlineto{\pgfqpoint{2.752855in}{1.875562in}}%
\pgfpathlineto{\pgfqpoint{2.739190in}{1.883582in}}%
\pgfpathlineto{\pgfqpoint{2.725527in}{1.891704in}}%
\pgfpathlineto{\pgfqpoint{2.711865in}{1.899927in}}%
\pgfpathlineto{\pgfqpoint{2.703269in}{1.894703in}}%
\pgfpathlineto{\pgfqpoint{2.694661in}{1.889612in}}%
\pgfpathlineto{\pgfqpoint{2.686042in}{1.884660in}}%
\pgfpathlineto{\pgfqpoint{2.677412in}{1.879851in}}%
\pgfpathclose%
\pgfusepath{fill}%
\end{pgfscope}%
\begin{pgfscope}%
\pgfpathrectangle{\pgfqpoint{1.150000in}{0.150000in}}{\pgfqpoint{5.700000in}{5.700000in}}%
\pgfusepath{clip}%
\pgfsetbuttcap%
\pgfsetroundjoin%
\definecolor{currentfill}{rgb}{0.283187,0.125848,0.444960}%
\pgfsetfillcolor{currentfill}%
\pgfsetfillopacity{0.700000}%
\pgfsetlinewidth{0.000000pt}%
\definecolor{currentstroke}{rgb}{0.000000,0.000000,0.000000}%
\pgfsetstrokecolor{currentstroke}%
\pgfsetdash{}{0pt}%
\pgfpathmoveto{\pgfqpoint{4.117731in}{1.970889in}}%
\pgfpathlineto{\pgfqpoint{4.131630in}{1.970738in}}%
\pgfpathlineto{\pgfqpoint{4.145536in}{1.970661in}}%
\pgfpathlineto{\pgfqpoint{4.159451in}{1.970659in}}%
\pgfpathlineto{\pgfqpoint{4.173375in}{1.970731in}}%
\pgfpathlineto{\pgfqpoint{4.181363in}{1.979936in}}%
\pgfpathlineto{\pgfqpoint{4.189344in}{1.989087in}}%
\pgfpathlineto{\pgfqpoint{4.197320in}{1.998184in}}%
\pgfpathlineto{\pgfqpoint{4.205290in}{2.007227in}}%
\pgfpathlineto{\pgfqpoint{4.191376in}{2.007135in}}%
\pgfpathlineto{\pgfqpoint{4.177470in}{2.007118in}}%
\pgfpathlineto{\pgfqpoint{4.163574in}{2.007176in}}%
\pgfpathlineto{\pgfqpoint{4.149686in}{2.007308in}}%
\pgfpathlineto{\pgfqpoint{4.141706in}{1.998276in}}%
\pgfpathlineto{\pgfqpoint{4.133720in}{1.989196in}}%
\pgfpathlineto{\pgfqpoint{4.125729in}{1.980067in}}%
\pgfpathlineto{\pgfqpoint{4.117731in}{1.970889in}}%
\pgfpathclose%
\pgfusepath{fill}%
\end{pgfscope}%
\begin{pgfscope}%
\pgfpathrectangle{\pgfqpoint{1.150000in}{0.150000in}}{\pgfqpoint{5.700000in}{5.700000in}}%
\pgfusepath{clip}%
\pgfsetbuttcap%
\pgfsetroundjoin%
\definecolor{currentfill}{rgb}{0.269944,0.014625,0.341379}%
\pgfsetfillcolor{currentfill}%
\pgfsetfillopacity{0.700000}%
\pgfsetlinewidth{0.000000pt}%
\definecolor{currentstroke}{rgb}{0.000000,0.000000,0.000000}%
\pgfsetstrokecolor{currentstroke}%
\pgfsetdash{}{0pt}%
\pgfpathmoveto{\pgfqpoint{3.393483in}{1.778718in}}%
\pgfpathlineto{\pgfqpoint{3.407208in}{1.775027in}}%
\pgfpathlineto{\pgfqpoint{3.420939in}{1.771419in}}%
\pgfpathlineto{\pgfqpoint{3.434674in}{1.767894in}}%
\pgfpathlineto{\pgfqpoint{3.448416in}{1.764451in}}%
\pgfpathlineto{\pgfqpoint{3.456665in}{1.773338in}}%
\pgfpathlineto{\pgfqpoint{3.464907in}{1.782248in}}%
\pgfpathlineto{\pgfqpoint{3.473143in}{1.791178in}}%
\pgfpathlineto{\pgfqpoint{3.481373in}{1.800126in}}%
\pgfpathlineto{\pgfqpoint{3.467645in}{1.803406in}}%
\pgfpathlineto{\pgfqpoint{3.453923in}{1.806768in}}%
\pgfpathlineto{\pgfqpoint{3.440207in}{1.810212in}}%
\pgfpathlineto{\pgfqpoint{3.426496in}{1.813740in}}%
\pgfpathlineto{\pgfqpoint{3.418252in}{1.804947in}}%
\pgfpathlineto{\pgfqpoint{3.410002in}{1.796178in}}%
\pgfpathlineto{\pgfqpoint{3.401746in}{1.787434in}}%
\pgfpathlineto{\pgfqpoint{3.393483in}{1.778718in}}%
\pgfpathclose%
\pgfusepath{fill}%
\end{pgfscope}%
\begin{pgfscope}%
\pgfpathrectangle{\pgfqpoint{1.150000in}{0.150000in}}{\pgfqpoint{5.700000in}{5.700000in}}%
\pgfusepath{clip}%
\pgfsetbuttcap%
\pgfsetroundjoin%
\definecolor{currentfill}{rgb}{0.227802,0.326594,0.546532}%
\pgfsetfillcolor{currentfill}%
\pgfsetfillopacity{0.700000}%
\pgfsetlinewidth{0.000000pt}%
\definecolor{currentstroke}{rgb}{0.000000,0.000000,0.000000}%
\pgfsetstrokecolor{currentstroke}%
\pgfsetdash{}{0pt}%
\pgfpathmoveto{\pgfqpoint{5.279973in}{2.413787in}}%
\pgfpathlineto{\pgfqpoint{5.294290in}{2.416487in}}%
\pgfpathlineto{\pgfqpoint{5.308618in}{2.419257in}}%
\pgfpathlineto{\pgfqpoint{5.322957in}{2.422096in}}%
\pgfpathlineto{\pgfqpoint{5.337309in}{2.425005in}}%
\pgfpathlineto{\pgfqpoint{5.344790in}{2.430083in}}%
\pgfpathlineto{\pgfqpoint{5.352263in}{2.435137in}}%
\pgfpathlineto{\pgfqpoint{5.359730in}{2.440171in}}%
\pgfpathlineto{\pgfqpoint{5.367190in}{2.445188in}}%
\pgfpathlineto{\pgfqpoint{5.352858in}{2.442516in}}%
\pgfpathlineto{\pgfqpoint{5.338538in}{2.439912in}}%
\pgfpathlineto{\pgfqpoint{5.324230in}{2.437377in}}%
\pgfpathlineto{\pgfqpoint{5.309933in}{2.434911in}}%
\pgfpathlineto{\pgfqpoint{5.302453in}{2.429651in}}%
\pgfpathlineto{\pgfqpoint{5.294966in}{2.424379in}}%
\pgfpathlineto{\pgfqpoint{5.287473in}{2.419093in}}%
\pgfpathlineto{\pgfqpoint{5.279973in}{2.413787in}}%
\pgfpathclose%
\pgfusepath{fill}%
\end{pgfscope}%
\begin{pgfscope}%
\pgfpathrectangle{\pgfqpoint{1.150000in}{0.150000in}}{\pgfqpoint{5.700000in}{5.700000in}}%
\pgfusepath{clip}%
\pgfsetbuttcap%
\pgfsetroundjoin%
\definecolor{currentfill}{rgb}{0.283091,0.110553,0.431554}%
\pgfsetfillcolor{currentfill}%
\pgfsetfillopacity{0.700000}%
\pgfsetlinewidth{0.000000pt}%
\definecolor{currentstroke}{rgb}{0.000000,0.000000,0.000000}%
\pgfsetstrokecolor{currentstroke}%
\pgfsetdash{}{0pt}%
\pgfpathmoveto{\pgfqpoint{2.478239in}{1.975792in}}%
\pgfpathlineto{\pgfqpoint{2.491950in}{1.965720in}}%
\pgfpathlineto{\pgfqpoint{2.505660in}{1.955758in}}%
\pgfpathlineto{\pgfqpoint{2.519370in}{1.945908in}}%
\pgfpathlineto{\pgfqpoint{2.533081in}{1.936168in}}%
\pgfpathlineto{\pgfqpoint{2.541816in}{1.939768in}}%
\pgfpathlineto{\pgfqpoint{2.550538in}{1.943539in}}%
\pgfpathlineto{\pgfqpoint{2.559247in}{1.947475in}}%
\pgfpathlineto{\pgfqpoint{2.567944in}{1.951572in}}%
\pgfpathlineto{\pgfqpoint{2.554263in}{1.961020in}}%
\pgfpathlineto{\pgfqpoint{2.540581in}{1.970579in}}%
\pgfpathlineto{\pgfqpoint{2.526900in}{1.980247in}}%
\pgfpathlineto{\pgfqpoint{2.513219in}{1.990028in}}%
\pgfpathlineto{\pgfqpoint{2.504494in}{1.986215in}}%
\pgfpathlineto{\pgfqpoint{2.495756in}{1.982568in}}%
\pgfpathlineto{\pgfqpoint{2.487005in}{1.979093in}}%
\pgfpathlineto{\pgfqpoint{2.478239in}{1.975792in}}%
\pgfpathclose%
\pgfusepath{fill}%
\end{pgfscope}%
\begin{pgfscope}%
\pgfpathrectangle{\pgfqpoint{1.150000in}{0.150000in}}{\pgfqpoint{5.700000in}{5.700000in}}%
\pgfusepath{clip}%
\pgfsetbuttcap%
\pgfsetroundjoin%
\definecolor{currentfill}{rgb}{0.269944,0.014625,0.341379}%
\pgfsetfillcolor{currentfill}%
\pgfsetfillopacity{0.700000}%
\pgfsetlinewidth{0.000000pt}%
\definecolor{currentstroke}{rgb}{0.000000,0.000000,0.000000}%
\pgfsetstrokecolor{currentstroke}%
\pgfsetdash{}{0pt}%
\pgfpathmoveto{\pgfqpoint{3.019312in}{1.781313in}}%
\pgfpathlineto{\pgfqpoint{3.032999in}{1.775269in}}%
\pgfpathlineto{\pgfqpoint{3.046690in}{1.769317in}}%
\pgfpathlineto{\pgfqpoint{3.060384in}{1.763454in}}%
\pgfpathlineto{\pgfqpoint{3.074082in}{1.757682in}}%
\pgfpathlineto{\pgfqpoint{3.082499in}{1.764928in}}%
\pgfpathlineto{\pgfqpoint{3.090908in}{1.772255in}}%
\pgfpathlineto{\pgfqpoint{3.099308in}{1.779661in}}%
\pgfpathlineto{\pgfqpoint{3.107701in}{1.787142in}}%
\pgfpathlineto{\pgfqpoint{3.094022in}{1.792688in}}%
\pgfpathlineto{\pgfqpoint{3.080347in}{1.798325in}}%
\pgfpathlineto{\pgfqpoint{3.066675in}{1.804052in}}%
\pgfpathlineto{\pgfqpoint{3.053007in}{1.809870in}}%
\pgfpathlineto{\pgfqpoint{3.044596in}{1.802607in}}%
\pgfpathlineto{\pgfqpoint{3.036176in}{1.795424in}}%
\pgfpathlineto{\pgfqpoint{3.027748in}{1.788325in}}%
\pgfpathlineto{\pgfqpoint{3.019312in}{1.781313in}}%
\pgfpathclose%
\pgfusepath{fill}%
\end{pgfscope}%
\begin{pgfscope}%
\pgfpathrectangle{\pgfqpoint{1.150000in}{0.150000in}}{\pgfqpoint{5.700000in}{5.700000in}}%
\pgfusepath{clip}%
\pgfsetbuttcap%
\pgfsetroundjoin%
\definecolor{currentfill}{rgb}{0.265145,0.232956,0.516599}%
\pgfsetfillcolor{currentfill}%
\pgfsetfillopacity{0.700000}%
\pgfsetlinewidth{0.000000pt}%
\definecolor{currentstroke}{rgb}{0.000000,0.000000,0.000000}%
\pgfsetstrokecolor{currentstroke}%
\pgfsetdash{}{0pt}%
\pgfpathmoveto{\pgfqpoint{4.698863in}{2.196863in}}%
\pgfpathlineto{\pgfqpoint{4.712961in}{2.198557in}}%
\pgfpathlineto{\pgfqpoint{4.727070in}{2.200323in}}%
\pgfpathlineto{\pgfqpoint{4.741190in}{2.202160in}}%
\pgfpathlineto{\pgfqpoint{4.755319in}{2.204069in}}%
\pgfpathlineto{\pgfqpoint{4.763080in}{2.211493in}}%
\pgfpathlineto{\pgfqpoint{4.770833in}{2.218853in}}%
\pgfpathlineto{\pgfqpoint{4.778580in}{2.226151in}}%
\pgfpathlineto{\pgfqpoint{4.786320in}{2.233390in}}%
\pgfpathlineto{\pgfqpoint{4.772203in}{2.231589in}}%
\pgfpathlineto{\pgfqpoint{4.758097in}{2.229859in}}%
\pgfpathlineto{\pgfqpoint{4.744001in}{2.228200in}}%
\pgfpathlineto{\pgfqpoint{4.729915in}{2.226613in}}%
\pgfpathlineto{\pgfqpoint{4.722162in}{2.219259in}}%
\pgfpathlineto{\pgfqpoint{4.714402in}{2.211851in}}%
\pgfpathlineto{\pgfqpoint{4.706636in}{2.204387in}}%
\pgfpathlineto{\pgfqpoint{4.698863in}{2.196863in}}%
\pgfpathclose%
\pgfusepath{fill}%
\end{pgfscope}%
\begin{pgfscope}%
\pgfpathrectangle{\pgfqpoint{1.150000in}{0.150000in}}{\pgfqpoint{5.700000in}{5.700000in}}%
\pgfusepath{clip}%
\pgfsetbuttcap%
\pgfsetroundjoin%
\definecolor{currentfill}{rgb}{0.270595,0.214069,0.507052}%
\pgfsetfillcolor{currentfill}%
\pgfsetfillopacity{0.700000}%
\pgfsetlinewidth{0.000000pt}%
\definecolor{currentstroke}{rgb}{0.000000,0.000000,0.000000}%
\pgfsetstrokecolor{currentstroke}%
\pgfsetdash{}{0pt}%
\pgfpathmoveto{\pgfqpoint{2.167902in}{2.197100in}}%
\pgfpathlineto{\pgfqpoint{2.181681in}{2.184276in}}%
\pgfpathlineto{\pgfqpoint{2.195456in}{2.171584in}}%
\pgfpathlineto{\pgfqpoint{2.209228in}{2.159022in}}%
\pgfpathlineto{\pgfqpoint{2.222998in}{2.146590in}}%
\pgfpathlineto{\pgfqpoint{2.231956in}{2.147846in}}%
\pgfpathlineto{\pgfqpoint{2.240898in}{2.149317in}}%
\pgfpathlineto{\pgfqpoint{2.249823in}{2.150999in}}%
\pgfpathlineto{\pgfqpoint{2.258732in}{2.152887in}}%
\pgfpathlineto{\pgfqpoint{2.244997in}{2.165000in}}%
\pgfpathlineto{\pgfqpoint{2.231261in}{2.177242in}}%
\pgfpathlineto{\pgfqpoint{2.217521in}{2.189615in}}%
\pgfpathlineto{\pgfqpoint{2.203779in}{2.202119in}}%
\pgfpathlineto{\pgfqpoint{2.194835in}{2.200542in}}%
\pgfpathlineto{\pgfqpoint{2.185874in}{2.199177in}}%
\pgfpathlineto{\pgfqpoint{2.176897in}{2.198028in}}%
\pgfpathlineto{\pgfqpoint{2.167902in}{2.197100in}}%
\pgfpathclose%
\pgfusepath{fill}%
\end{pgfscope}%
\begin{pgfscope}%
\pgfpathrectangle{\pgfqpoint{1.150000in}{0.150000in}}{\pgfqpoint{5.700000in}{5.700000in}}%
\pgfusepath{clip}%
\pgfsetbuttcap%
\pgfsetroundjoin%
\definecolor{currentfill}{rgb}{0.274952,0.037752,0.364543}%
\pgfsetfillcolor{currentfill}%
\pgfsetfillopacity{0.700000}%
\pgfsetlinewidth{0.000000pt}%
\definecolor{currentstroke}{rgb}{0.000000,0.000000,0.000000}%
\pgfsetstrokecolor{currentstroke}%
\pgfsetdash{}{0pt}%
\pgfpathmoveto{\pgfqpoint{3.624167in}{1.813903in}}%
\pgfpathlineto{\pgfqpoint{3.637937in}{1.811495in}}%
\pgfpathlineto{\pgfqpoint{3.651714in}{1.809166in}}%
\pgfpathlineto{\pgfqpoint{3.665497in}{1.806917in}}%
\pgfpathlineto{\pgfqpoint{3.679287in}{1.804746in}}%
\pgfpathlineto{\pgfqpoint{3.687450in}{1.814148in}}%
\pgfpathlineto{\pgfqpoint{3.695607in}{1.823542in}}%
\pgfpathlineto{\pgfqpoint{3.703758in}{1.832926in}}%
\pgfpathlineto{\pgfqpoint{3.711904in}{1.842299in}}%
\pgfpathlineto{\pgfqpoint{3.698125in}{1.844347in}}%
\pgfpathlineto{\pgfqpoint{3.684353in}{1.846474in}}%
\pgfpathlineto{\pgfqpoint{3.670588in}{1.848680in}}%
\pgfpathlineto{\pgfqpoint{3.656830in}{1.850966in}}%
\pgfpathlineto{\pgfqpoint{3.648673in}{1.841707in}}%
\pgfpathlineto{\pgfqpoint{3.640510in}{1.832443in}}%
\pgfpathlineto{\pgfqpoint{3.632341in}{1.823174in}}%
\pgfpathlineto{\pgfqpoint{3.624167in}{1.813903in}}%
\pgfpathclose%
\pgfusepath{fill}%
\end{pgfscope}%
\begin{pgfscope}%
\pgfpathrectangle{\pgfqpoint{1.150000in}{0.150000in}}{\pgfqpoint{5.700000in}{5.700000in}}%
\pgfusepath{clip}%
\pgfsetbuttcap%
\pgfsetroundjoin%
\definecolor{currentfill}{rgb}{0.204903,0.375746,0.553533}%
\pgfsetfillcolor{currentfill}%
\pgfsetfillopacity{0.700000}%
\pgfsetlinewidth{0.000000pt}%
\definecolor{currentstroke}{rgb}{0.000000,0.000000,0.000000}%
\pgfsetstrokecolor{currentstroke}%
\pgfsetdash{}{0pt}%
\pgfpathmoveto{\pgfqpoint{5.686116in}{2.544643in}}%
\pgfpathlineto{\pgfqpoint{5.700587in}{2.547647in}}%
\pgfpathlineto{\pgfqpoint{5.715070in}{2.550720in}}%
\pgfpathlineto{\pgfqpoint{5.729565in}{2.553861in}}%
\pgfpathlineto{\pgfqpoint{5.744073in}{2.557070in}}%
\pgfpathlineto{\pgfqpoint{5.751338in}{2.560755in}}%
\pgfpathlineto{\pgfqpoint{5.758596in}{2.564468in}}%
\pgfpathlineto{\pgfqpoint{5.765849in}{2.568215in}}%
\pgfpathlineto{\pgfqpoint{5.773097in}{2.572002in}}%
\pgfpathlineto{\pgfqpoint{5.758615in}{2.569114in}}%
\pgfpathlineto{\pgfqpoint{5.744145in}{2.566294in}}%
\pgfpathlineto{\pgfqpoint{5.729688in}{2.563542in}}%
\pgfpathlineto{\pgfqpoint{5.715243in}{2.560857in}}%
\pgfpathlineto{\pgfqpoint{5.707969in}{2.556743in}}%
\pgfpathlineto{\pgfqpoint{5.700690in}{2.552673in}}%
\pgfpathlineto{\pgfqpoint{5.693406in}{2.548641in}}%
\pgfpathlineto{\pgfqpoint{5.686116in}{2.544643in}}%
\pgfpathclose%
\pgfusepath{fill}%
\end{pgfscope}%
\begin{pgfscope}%
\pgfpathrectangle{\pgfqpoint{1.150000in}{0.150000in}}{\pgfqpoint{5.700000in}{5.700000in}}%
\pgfusepath{clip}%
\pgfsetbuttcap%
\pgfsetroundjoin%
\definecolor{currentfill}{rgb}{0.282910,0.105393,0.426902}%
\pgfsetfillcolor{currentfill}%
\pgfsetfillopacity{0.700000}%
\pgfsetlinewidth{0.000000pt}%
\definecolor{currentstroke}{rgb}{0.000000,0.000000,0.000000}%
\pgfsetstrokecolor{currentstroke}%
\pgfsetdash{}{0pt}%
\pgfpathmoveto{\pgfqpoint{4.030137in}{1.935207in}}%
\pgfpathlineto{\pgfqpoint{4.044012in}{1.934714in}}%
\pgfpathlineto{\pgfqpoint{4.057895in}{1.934297in}}%
\pgfpathlineto{\pgfqpoint{4.071786in}{1.933956in}}%
\pgfpathlineto{\pgfqpoint{4.085686in}{1.933689in}}%
\pgfpathlineto{\pgfqpoint{4.093706in}{1.943062in}}%
\pgfpathlineto{\pgfqpoint{4.101720in}{1.952387in}}%
\pgfpathlineto{\pgfqpoint{4.109728in}{1.961663in}}%
\pgfpathlineto{\pgfqpoint{4.117731in}{1.970889in}}%
\pgfpathlineto{\pgfqpoint{4.103842in}{1.971116in}}%
\pgfpathlineto{\pgfqpoint{4.089961in}{1.971418in}}%
\pgfpathlineto{\pgfqpoint{4.076088in}{1.971795in}}%
\pgfpathlineto{\pgfqpoint{4.062223in}{1.972247in}}%
\pgfpathlineto{\pgfqpoint{4.054210in}{1.963052in}}%
\pgfpathlineto{\pgfqpoint{4.046191in}{1.953814in}}%
\pgfpathlineto{\pgfqpoint{4.038167in}{1.944532in}}%
\pgfpathlineto{\pgfqpoint{4.030137in}{1.935207in}}%
\pgfpathclose%
\pgfusepath{fill}%
\end{pgfscope}%
\begin{pgfscope}%
\pgfpathrectangle{\pgfqpoint{1.150000in}{0.150000in}}{\pgfqpoint{5.700000in}{5.700000in}}%
\pgfusepath{clip}%
\pgfsetbuttcap%
\pgfsetroundjoin%
\definecolor{currentfill}{rgb}{0.268510,0.009605,0.335427}%
\pgfsetfillcolor{currentfill}%
\pgfsetfillopacity{0.700000}%
\pgfsetlinewidth{0.000000pt}%
\definecolor{currentstroke}{rgb}{0.000000,0.000000,0.000000}%
\pgfsetstrokecolor{currentstroke}%
\pgfsetdash{}{0pt}%
\pgfpathmoveto{\pgfqpoint{3.162456in}{1.765846in}}%
\pgfpathlineto{\pgfqpoint{3.176155in}{1.760743in}}%
\pgfpathlineto{\pgfqpoint{3.189858in}{1.755727in}}%
\pgfpathlineto{\pgfqpoint{3.203566in}{1.750798in}}%
\pgfpathlineto{\pgfqpoint{3.217278in}{1.745956in}}%
\pgfpathlineto{\pgfqpoint{3.225627in}{1.753934in}}%
\pgfpathlineto{\pgfqpoint{3.233969in}{1.761971in}}%
\pgfpathlineto{\pgfqpoint{3.242304in}{1.770064in}}%
\pgfpathlineto{\pgfqpoint{3.250631in}{1.778210in}}%
\pgfpathlineto{\pgfqpoint{3.236936in}{1.782848in}}%
\pgfpathlineto{\pgfqpoint{3.223245in}{1.787572in}}%
\pgfpathlineto{\pgfqpoint{3.209559in}{1.792383in}}%
\pgfpathlineto{\pgfqpoint{3.195877in}{1.797282in}}%
\pgfpathlineto{\pgfqpoint{3.187533in}{1.789333in}}%
\pgfpathlineto{\pgfqpoint{3.179182in}{1.781441in}}%
\pgfpathlineto{\pgfqpoint{3.170823in}{1.773612in}}%
\pgfpathlineto{\pgfqpoint{3.162456in}{1.765846in}}%
\pgfpathclose%
\pgfusepath{fill}%
\end{pgfscope}%
\begin{pgfscope}%
\pgfpathrectangle{\pgfqpoint{1.150000in}{0.150000in}}{\pgfqpoint{5.700000in}{5.700000in}}%
\pgfusepath{clip}%
\pgfsetbuttcap%
\pgfsetroundjoin%
\definecolor{currentfill}{rgb}{0.272594,0.025563,0.353093}%
\pgfsetfillcolor{currentfill}%
\pgfsetfillopacity{0.700000}%
\pgfsetlinewidth{0.000000pt}%
\definecolor{currentstroke}{rgb}{0.000000,0.000000,0.000000}%
\pgfsetstrokecolor{currentstroke}%
\pgfsetdash{}{0pt}%
\pgfpathmoveto{\pgfqpoint{2.875922in}{1.807859in}}%
\pgfpathlineto{\pgfqpoint{2.889607in}{1.800823in}}%
\pgfpathlineto{\pgfqpoint{2.903294in}{1.793881in}}%
\pgfpathlineto{\pgfqpoint{2.916985in}{1.787035in}}%
\pgfpathlineto{\pgfqpoint{2.930677in}{1.780282in}}%
\pgfpathlineto{\pgfqpoint{2.939171in}{1.786658in}}%
\pgfpathlineto{\pgfqpoint{2.947655in}{1.793140in}}%
\pgfpathlineto{\pgfqpoint{2.956129in}{1.799724in}}%
\pgfpathlineto{\pgfqpoint{2.964595in}{1.806408in}}%
\pgfpathlineto{\pgfqpoint{2.950923in}{1.812914in}}%
\pgfpathlineto{\pgfqpoint{2.937254in}{1.819513in}}%
\pgfpathlineto{\pgfqpoint{2.923589in}{1.826208in}}%
\pgfpathlineto{\pgfqpoint{2.909925in}{1.832998in}}%
\pgfpathlineto{\pgfqpoint{2.901439in}{1.826553in}}%
\pgfpathlineto{\pgfqpoint{2.892943in}{1.820213in}}%
\pgfpathlineto{\pgfqpoint{2.884437in}{1.813980in}}%
\pgfpathlineto{\pgfqpoint{2.875922in}{1.807859in}}%
\pgfpathclose%
\pgfusepath{fill}%
\end{pgfscope}%
\begin{pgfscope}%
\pgfpathrectangle{\pgfqpoint{1.150000in}{0.150000in}}{\pgfqpoint{5.700000in}{5.700000in}}%
\pgfusepath{clip}%
\pgfsetbuttcap%
\pgfsetroundjoin%
\definecolor{currentfill}{rgb}{0.269308,0.218818,0.509577}%
\pgfsetfillcolor{currentfill}%
\pgfsetfillopacity{0.700000}%
\pgfsetlinewidth{0.000000pt}%
\definecolor{currentstroke}{rgb}{0.000000,0.000000,0.000000}%
\pgfsetstrokecolor{currentstroke}%
\pgfsetdash{}{0pt}%
\pgfpathmoveto{\pgfqpoint{4.611364in}{2.159736in}}%
\pgfpathlineto{\pgfqpoint{4.625434in}{2.161229in}}%
\pgfpathlineto{\pgfqpoint{4.639514in}{2.162795in}}%
\pgfpathlineto{\pgfqpoint{4.653604in}{2.164432in}}%
\pgfpathlineto{\pgfqpoint{4.667704in}{2.166141in}}%
\pgfpathlineto{\pgfqpoint{4.675504in}{2.173919in}}%
\pgfpathlineto{\pgfqpoint{4.683297in}{2.181632in}}%
\pgfpathlineto{\pgfqpoint{4.691083in}{2.189279in}}%
\pgfpathlineto{\pgfqpoint{4.698863in}{2.196863in}}%
\pgfpathlineto{\pgfqpoint{4.684774in}{2.195240in}}%
\pgfpathlineto{\pgfqpoint{4.670697in}{2.193689in}}%
\pgfpathlineto{\pgfqpoint{4.656629in}{2.192210in}}%
\pgfpathlineto{\pgfqpoint{4.642571in}{2.190802in}}%
\pgfpathlineto{\pgfqpoint{4.634779in}{2.183124in}}%
\pgfpathlineto{\pgfqpoint{4.626981in}{2.175388in}}%
\pgfpathlineto{\pgfqpoint{4.619176in}{2.167593in}}%
\pgfpathlineto{\pgfqpoint{4.611364in}{2.159736in}}%
\pgfpathclose%
\pgfusepath{fill}%
\end{pgfscope}%
\begin{pgfscope}%
\pgfpathrectangle{\pgfqpoint{1.150000in}{0.150000in}}{\pgfqpoint{5.700000in}{5.700000in}}%
\pgfusepath{clip}%
\pgfsetbuttcap%
\pgfsetroundjoin%
\definecolor{currentfill}{rgb}{0.233603,0.313828,0.543914}%
\pgfsetfillcolor{currentfill}%
\pgfsetfillopacity{0.700000}%
\pgfsetlinewidth{0.000000pt}%
\definecolor{currentstroke}{rgb}{0.000000,0.000000,0.000000}%
\pgfsetstrokecolor{currentstroke}%
\pgfsetdash{}{0pt}%
\pgfpathmoveto{\pgfqpoint{5.192683in}{2.381323in}}%
\pgfpathlineto{\pgfqpoint{5.206971in}{2.383960in}}%
\pgfpathlineto{\pgfqpoint{5.221270in}{2.386667in}}%
\pgfpathlineto{\pgfqpoint{5.235582in}{2.389443in}}%
\pgfpathlineto{\pgfqpoint{5.249905in}{2.392288in}}%
\pgfpathlineto{\pgfqpoint{5.257432in}{2.397712in}}%
\pgfpathlineto{\pgfqpoint{5.264953in}{2.403100in}}%
\pgfpathlineto{\pgfqpoint{5.272467in}{2.408457in}}%
\pgfpathlineto{\pgfqpoint{5.279973in}{2.413787in}}%
\pgfpathlineto{\pgfqpoint{5.265669in}{2.411156in}}%
\pgfpathlineto{\pgfqpoint{5.251376in}{2.408594in}}%
\pgfpathlineto{\pgfqpoint{5.237095in}{2.406101in}}%
\pgfpathlineto{\pgfqpoint{5.222825in}{2.403678in}}%
\pgfpathlineto{\pgfqpoint{5.215299in}{2.398127in}}%
\pgfpathlineto{\pgfqpoint{5.207767in}{2.392554in}}%
\pgfpathlineto{\pgfqpoint{5.200229in}{2.386954in}}%
\pgfpathlineto{\pgfqpoint{5.192683in}{2.381323in}}%
\pgfpathclose%
\pgfusepath{fill}%
\end{pgfscope}%
\begin{pgfscope}%
\pgfpathrectangle{\pgfqpoint{1.150000in}{0.150000in}}{\pgfqpoint{5.700000in}{5.700000in}}%
\pgfusepath{clip}%
\pgfsetbuttcap%
\pgfsetroundjoin%
\definecolor{currentfill}{rgb}{0.281924,0.089666,0.412415}%
\pgfsetfillcolor{currentfill}%
\pgfsetfillopacity{0.700000}%
\pgfsetlinewidth{0.000000pt}%
\definecolor{currentstroke}{rgb}{0.000000,0.000000,0.000000}%
\pgfsetstrokecolor{currentstroke}%
\pgfsetdash{}{0pt}%
\pgfpathmoveto{\pgfqpoint{3.942503in}{1.900449in}}%
\pgfpathlineto{\pgfqpoint{3.956355in}{1.899592in}}%
\pgfpathlineto{\pgfqpoint{3.970216in}{1.898811in}}%
\pgfpathlineto{\pgfqpoint{3.984085in}{1.898107in}}%
\pgfpathlineto{\pgfqpoint{3.997961in}{1.897478in}}%
\pgfpathlineto{\pgfqpoint{4.006014in}{1.906973in}}%
\pgfpathlineto{\pgfqpoint{4.014060in}{1.916427in}}%
\pgfpathlineto{\pgfqpoint{4.022102in}{1.925838in}}%
\pgfpathlineto{\pgfqpoint{4.030137in}{1.935207in}}%
\pgfpathlineto{\pgfqpoint{4.016271in}{1.935775in}}%
\pgfpathlineto{\pgfqpoint{4.002412in}{1.936419in}}%
\pgfpathlineto{\pgfqpoint{3.988562in}{1.937138in}}%
\pgfpathlineto{\pgfqpoint{3.974719in}{1.937934in}}%
\pgfpathlineto{\pgfqpoint{3.966674in}{1.928619in}}%
\pgfpathlineto{\pgfqpoint{3.958622in}{1.919266in}}%
\pgfpathlineto{\pgfqpoint{3.950565in}{1.909875in}}%
\pgfpathlineto{\pgfqpoint{3.942503in}{1.900449in}}%
\pgfpathclose%
\pgfusepath{fill}%
\end{pgfscope}%
\begin{pgfscope}%
\pgfpathrectangle{\pgfqpoint{1.150000in}{0.150000in}}{\pgfqpoint{5.700000in}{5.700000in}}%
\pgfusepath{clip}%
\pgfsetbuttcap%
\pgfsetroundjoin%
\definecolor{currentfill}{rgb}{0.275191,0.194905,0.496005}%
\pgfsetfillcolor{currentfill}%
\pgfsetfillopacity{0.700000}%
\pgfsetlinewidth{0.000000pt}%
\definecolor{currentstroke}{rgb}{0.000000,0.000000,0.000000}%
\pgfsetstrokecolor{currentstroke}%
\pgfsetdash{}{0pt}%
\pgfpathmoveto{\pgfqpoint{2.222998in}{2.146590in}}%
\pgfpathlineto{\pgfqpoint{2.236765in}{2.134287in}}%
\pgfpathlineto{\pgfqpoint{2.250529in}{2.122110in}}%
\pgfpathlineto{\pgfqpoint{2.264291in}{2.110060in}}%
\pgfpathlineto{\pgfqpoint{2.278051in}{2.098135in}}%
\pgfpathlineto{\pgfqpoint{2.286974in}{2.099717in}}%
\pgfpathlineto{\pgfqpoint{2.295880in}{2.101509in}}%
\pgfpathlineto{\pgfqpoint{2.304771in}{2.103506in}}%
\pgfpathlineto{\pgfqpoint{2.313646in}{2.105704in}}%
\pgfpathlineto{\pgfqpoint{2.299920in}{2.117312in}}%
\pgfpathlineto{\pgfqpoint{2.286193in}{2.129044in}}%
\pgfpathlineto{\pgfqpoint{2.272464in}{2.140902in}}%
\pgfpathlineto{\pgfqpoint{2.258732in}{2.152887in}}%
\pgfpathlineto{\pgfqpoint{2.249823in}{2.150999in}}%
\pgfpathlineto{\pgfqpoint{2.240898in}{2.149317in}}%
\pgfpathlineto{\pgfqpoint{2.231956in}{2.147846in}}%
\pgfpathlineto{\pgfqpoint{2.222998in}{2.146590in}}%
\pgfpathclose%
\pgfusepath{fill}%
\end{pgfscope}%
\begin{pgfscope}%
\pgfpathrectangle{\pgfqpoint{1.150000in}{0.150000in}}{\pgfqpoint{5.700000in}{5.700000in}}%
\pgfusepath{clip}%
\pgfsetbuttcap%
\pgfsetroundjoin%
\definecolor{currentfill}{rgb}{0.268510,0.009605,0.335427}%
\pgfsetfillcolor{currentfill}%
\pgfsetfillopacity{0.700000}%
\pgfsetlinewidth{0.000000pt}%
\definecolor{currentstroke}{rgb}{0.000000,0.000000,0.000000}%
\pgfsetstrokecolor{currentstroke}%
\pgfsetdash{}{0pt}%
\pgfpathmoveto{\pgfqpoint{3.305459in}{1.760518in}}%
\pgfpathlineto{\pgfqpoint{3.319178in}{1.756308in}}%
\pgfpathlineto{\pgfqpoint{3.332903in}{1.752182in}}%
\pgfpathlineto{\pgfqpoint{3.346632in}{1.748140in}}%
\pgfpathlineto{\pgfqpoint{3.360366in}{1.744182in}}%
\pgfpathlineto{\pgfqpoint{3.368655in}{1.752762in}}%
\pgfpathlineto{\pgfqpoint{3.376938in}{1.761380in}}%
\pgfpathlineto{\pgfqpoint{3.385214in}{1.770033in}}%
\pgfpathlineto{\pgfqpoint{3.393483in}{1.778718in}}%
\pgfpathlineto{\pgfqpoint{3.379764in}{1.782493in}}%
\pgfpathlineto{\pgfqpoint{3.366049in}{1.786351in}}%
\pgfpathlineto{\pgfqpoint{3.352340in}{1.790293in}}%
\pgfpathlineto{\pgfqpoint{3.338636in}{1.794319in}}%
\pgfpathlineto{\pgfqpoint{3.330352in}{1.785810in}}%
\pgfpathlineto{\pgfqpoint{3.322062in}{1.777338in}}%
\pgfpathlineto{\pgfqpoint{3.313764in}{1.768907in}}%
\pgfpathlineto{\pgfqpoint{3.305459in}{1.760518in}}%
\pgfpathclose%
\pgfusepath{fill}%
\end{pgfscope}%
\begin{pgfscope}%
\pgfpathrectangle{\pgfqpoint{1.150000in}{0.150000in}}{\pgfqpoint{5.700000in}{5.700000in}}%
\pgfusepath{clip}%
\pgfsetbuttcap%
\pgfsetroundjoin%
\definecolor{currentfill}{rgb}{0.274128,0.199721,0.498911}%
\pgfsetfillcolor{currentfill}%
\pgfsetfillopacity{0.700000}%
\pgfsetlinewidth{0.000000pt}%
\definecolor{currentstroke}{rgb}{0.000000,0.000000,0.000000}%
\pgfsetstrokecolor{currentstroke}%
\pgfsetdash{}{0pt}%
\pgfpathmoveto{\pgfqpoint{4.523830in}{2.122147in}}%
\pgfpathlineto{\pgfqpoint{4.537871in}{2.123417in}}%
\pgfpathlineto{\pgfqpoint{4.551922in}{2.124759in}}%
\pgfpathlineto{\pgfqpoint{4.565982in}{2.126174in}}%
\pgfpathlineto{\pgfqpoint{4.580053in}{2.127660in}}%
\pgfpathlineto{\pgfqpoint{4.587891in}{2.135779in}}%
\pgfpathlineto{\pgfqpoint{4.595722in}{2.143831in}}%
\pgfpathlineto{\pgfqpoint{4.603546in}{2.151816in}}%
\pgfpathlineto{\pgfqpoint{4.611364in}{2.159736in}}%
\pgfpathlineto{\pgfqpoint{4.597305in}{2.158314in}}%
\pgfpathlineto{\pgfqpoint{4.583255in}{2.156965in}}%
\pgfpathlineto{\pgfqpoint{4.569216in}{2.155688in}}%
\pgfpathlineto{\pgfqpoint{4.555186in}{2.154482in}}%
\pgfpathlineto{\pgfqpoint{4.547357in}{2.146490in}}%
\pgfpathlineto{\pgfqpoint{4.539521in}{2.138437in}}%
\pgfpathlineto{\pgfqpoint{4.531678in}{2.130324in}}%
\pgfpathlineto{\pgfqpoint{4.523830in}{2.122147in}}%
\pgfpathclose%
\pgfusepath{fill}%
\end{pgfscope}%
\begin{pgfscope}%
\pgfpathrectangle{\pgfqpoint{1.150000in}{0.150000in}}{\pgfqpoint{5.700000in}{5.700000in}}%
\pgfusepath{clip}%
\pgfsetbuttcap%
\pgfsetroundjoin%
\definecolor{currentfill}{rgb}{0.272594,0.025563,0.353093}%
\pgfsetfillcolor{currentfill}%
\pgfsetfillopacity{0.700000}%
\pgfsetlinewidth{0.000000pt}%
\definecolor{currentstroke}{rgb}{0.000000,0.000000,0.000000}%
\pgfsetstrokecolor{currentstroke}%
\pgfsetdash{}{0pt}%
\pgfpathmoveto{\pgfqpoint{3.536343in}{1.787826in}}%
\pgfpathlineto{\pgfqpoint{3.550100in}{1.784954in}}%
\pgfpathlineto{\pgfqpoint{3.563864in}{1.782163in}}%
\pgfpathlineto{\pgfqpoint{3.577634in}{1.779452in}}%
\pgfpathlineto{\pgfqpoint{3.591410in}{1.776821in}}%
\pgfpathlineto{\pgfqpoint{3.599608in}{1.786087in}}%
\pgfpathlineto{\pgfqpoint{3.607800in}{1.795357in}}%
\pgfpathlineto{\pgfqpoint{3.615986in}{1.804630in}}%
\pgfpathlineto{\pgfqpoint{3.624167in}{1.813903in}}%
\pgfpathlineto{\pgfqpoint{3.610403in}{1.816391in}}%
\pgfpathlineto{\pgfqpoint{3.596645in}{1.818959in}}%
\pgfpathlineto{\pgfqpoint{3.582894in}{1.821607in}}%
\pgfpathlineto{\pgfqpoint{3.569149in}{1.824336in}}%
\pgfpathlineto{\pgfqpoint{3.560957in}{1.815199in}}%
\pgfpathlineto{\pgfqpoint{3.552758in}{1.806067in}}%
\pgfpathlineto{\pgfqpoint{3.544554in}{1.796942in}}%
\pgfpathlineto{\pgfqpoint{3.536343in}{1.787826in}}%
\pgfpathclose%
\pgfusepath{fill}%
\end{pgfscope}%
\begin{pgfscope}%
\pgfpathrectangle{\pgfqpoint{1.150000in}{0.150000in}}{\pgfqpoint{5.700000in}{5.700000in}}%
\pgfusepath{clip}%
\pgfsetbuttcap%
\pgfsetroundjoin%
\definecolor{currentfill}{rgb}{0.208623,0.367752,0.552675}%
\pgfsetfillcolor{currentfill}%
\pgfsetfillopacity{0.700000}%
\pgfsetlinewidth{0.000000pt}%
\definecolor{currentstroke}{rgb}{0.000000,0.000000,0.000000}%
\pgfsetstrokecolor{currentstroke}%
\pgfsetdash{}{0pt}%
\pgfpathmoveto{\pgfqpoint{5.599045in}{2.516338in}}%
\pgfpathlineto{\pgfqpoint{5.613490in}{2.519369in}}%
\pgfpathlineto{\pgfqpoint{5.627947in}{2.522469in}}%
\pgfpathlineto{\pgfqpoint{5.642418in}{2.525637in}}%
\pgfpathlineto{\pgfqpoint{5.656900in}{2.528873in}}%
\pgfpathlineto{\pgfqpoint{5.664213in}{2.532793in}}%
\pgfpathlineto{\pgfqpoint{5.671520in}{2.536724in}}%
\pgfpathlineto{\pgfqpoint{5.678821in}{2.540672in}}%
\pgfpathlineto{\pgfqpoint{5.686116in}{2.544643in}}%
\pgfpathlineto{\pgfqpoint{5.671658in}{2.541706in}}%
\pgfpathlineto{\pgfqpoint{5.657212in}{2.538838in}}%
\pgfpathlineto{\pgfqpoint{5.642779in}{2.536038in}}%
\pgfpathlineto{\pgfqpoint{5.628358in}{2.533306in}}%
\pgfpathlineto{\pgfqpoint{5.621038in}{2.529028in}}%
\pgfpathlineto{\pgfqpoint{5.613713in}{2.524778in}}%
\pgfpathlineto{\pgfqpoint{5.606382in}{2.520550in}}%
\pgfpathlineto{\pgfqpoint{5.599045in}{2.516338in}}%
\pgfpathclose%
\pgfusepath{fill}%
\end{pgfscope}%
\begin{pgfscope}%
\pgfpathrectangle{\pgfqpoint{1.150000in}{0.150000in}}{\pgfqpoint{5.700000in}{5.700000in}}%
\pgfusepath{clip}%
\pgfsetbuttcap%
\pgfsetroundjoin%
\definecolor{currentfill}{rgb}{0.277018,0.050344,0.375715}%
\pgfsetfillcolor{currentfill}%
\pgfsetfillopacity{0.700000}%
\pgfsetlinewidth{0.000000pt}%
\definecolor{currentstroke}{rgb}{0.000000,0.000000,0.000000}%
\pgfsetstrokecolor{currentstroke}%
\pgfsetdash{}{0pt}%
\pgfpathmoveto{\pgfqpoint{2.732169in}{1.846491in}}%
\pgfpathlineto{\pgfqpoint{2.745861in}{1.838404in}}%
\pgfpathlineto{\pgfqpoint{2.759556in}{1.830417in}}%
\pgfpathlineto{\pgfqpoint{2.773252in}{1.822530in}}%
\pgfpathlineto{\pgfqpoint{2.786949in}{1.814741in}}%
\pgfpathlineto{\pgfqpoint{2.795529in}{1.820104in}}%
\pgfpathlineto{\pgfqpoint{2.804098in}{1.825598in}}%
\pgfpathlineto{\pgfqpoint{2.812657in}{1.831221in}}%
\pgfpathlineto{\pgfqpoint{2.821205in}{1.836967in}}%
\pgfpathlineto{\pgfqpoint{2.807531in}{1.844488in}}%
\pgfpathlineto{\pgfqpoint{2.793860in}{1.852107in}}%
\pgfpathlineto{\pgfqpoint{2.780190in}{1.859825in}}%
\pgfpathlineto{\pgfqpoint{2.766522in}{1.867644in}}%
\pgfpathlineto{\pgfqpoint{2.757950in}{1.862158in}}%
\pgfpathlineto{\pgfqpoint{2.749367in}{1.856801in}}%
\pgfpathlineto{\pgfqpoint{2.740773in}{1.851577in}}%
\pgfpathlineto{\pgfqpoint{2.732169in}{1.846491in}}%
\pgfpathclose%
\pgfusepath{fill}%
\end{pgfscope}%
\begin{pgfscope}%
\pgfpathrectangle{\pgfqpoint{1.150000in}{0.150000in}}{\pgfqpoint{5.700000in}{5.700000in}}%
\pgfusepath{clip}%
\pgfsetbuttcap%
\pgfsetroundjoin%
\definecolor{currentfill}{rgb}{0.282656,0.100196,0.422160}%
\pgfsetfillcolor{currentfill}%
\pgfsetfillopacity{0.700000}%
\pgfsetlinewidth{0.000000pt}%
\definecolor{currentstroke}{rgb}{0.000000,0.000000,0.000000}%
\pgfsetstrokecolor{currentstroke}%
\pgfsetdash{}{0pt}%
\pgfpathmoveto{\pgfqpoint{2.533081in}{1.936168in}}%
\pgfpathlineto{\pgfqpoint{2.546791in}{1.926537in}}%
\pgfpathlineto{\pgfqpoint{2.560501in}{1.917015in}}%
\pgfpathlineto{\pgfqpoint{2.574211in}{1.907600in}}%
\pgfpathlineto{\pgfqpoint{2.587922in}{1.898293in}}%
\pgfpathlineto{\pgfqpoint{2.596629in}{1.902193in}}%
\pgfpathlineto{\pgfqpoint{2.605322in}{1.906257in}}%
\pgfpathlineto{\pgfqpoint{2.614003in}{1.910482in}}%
\pgfpathlineto{\pgfqpoint{2.622672in}{1.914863in}}%
\pgfpathlineto{\pgfqpoint{2.608989in}{1.923879in}}%
\pgfpathlineto{\pgfqpoint{2.595307in}{1.933002in}}%
\pgfpathlineto{\pgfqpoint{2.581625in}{1.942233in}}%
\pgfpathlineto{\pgfqpoint{2.567944in}{1.951572in}}%
\pgfpathlineto{\pgfqpoint{2.559247in}{1.947475in}}%
\pgfpathlineto{\pgfqpoint{2.550538in}{1.943539in}}%
\pgfpathlineto{\pgfqpoint{2.541816in}{1.939768in}}%
\pgfpathlineto{\pgfqpoint{2.533081in}{1.936168in}}%
\pgfpathclose%
\pgfusepath{fill}%
\end{pgfscope}%
\begin{pgfscope}%
\pgfpathrectangle{\pgfqpoint{1.150000in}{0.150000in}}{\pgfqpoint{5.700000in}{5.700000in}}%
\pgfusepath{clip}%
\pgfsetbuttcap%
\pgfsetroundjoin%
\definecolor{currentfill}{rgb}{0.239346,0.300855,0.540844}%
\pgfsetfillcolor{currentfill}%
\pgfsetfillopacity{0.700000}%
\pgfsetlinewidth{0.000000pt}%
\definecolor{currentstroke}{rgb}{0.000000,0.000000,0.000000}%
\pgfsetstrokecolor{currentstroke}%
\pgfsetdash{}{0pt}%
\pgfpathmoveto{\pgfqpoint{5.105324in}{2.347804in}}%
\pgfpathlineto{\pgfqpoint{5.119583in}{2.350355in}}%
\pgfpathlineto{\pgfqpoint{5.133854in}{2.352975in}}%
\pgfpathlineto{\pgfqpoint{5.148136in}{2.355665in}}%
\pgfpathlineto{\pgfqpoint{5.162430in}{2.358426in}}%
\pgfpathlineto{\pgfqpoint{5.170004in}{2.364214in}}%
\pgfpathlineto{\pgfqpoint{5.177570in}{2.369957in}}%
\pgfpathlineto{\pgfqpoint{5.185130in}{2.375659in}}%
\pgfpathlineto{\pgfqpoint{5.192683in}{2.381323in}}%
\pgfpathlineto{\pgfqpoint{5.178406in}{2.378756in}}%
\pgfpathlineto{\pgfqpoint{5.164141in}{2.376259in}}%
\pgfpathlineto{\pgfqpoint{5.149888in}{2.373831in}}%
\pgfpathlineto{\pgfqpoint{5.135646in}{2.371473in}}%
\pgfpathlineto{\pgfqpoint{5.128076in}{2.365608in}}%
\pgfpathlineto{\pgfqpoint{5.120499in}{2.359711in}}%
\pgfpathlineto{\pgfqpoint{5.112915in}{2.353778in}}%
\pgfpathlineto{\pgfqpoint{5.105324in}{2.347804in}}%
\pgfpathclose%
\pgfusepath{fill}%
\end{pgfscope}%
\begin{pgfscope}%
\pgfpathrectangle{\pgfqpoint{1.150000in}{0.150000in}}{\pgfqpoint{5.700000in}{5.700000in}}%
\pgfusepath{clip}%
\pgfsetbuttcap%
\pgfsetroundjoin%
\definecolor{currentfill}{rgb}{0.280894,0.078907,0.402329}%
\pgfsetfillcolor{currentfill}%
\pgfsetfillopacity{0.700000}%
\pgfsetlinewidth{0.000000pt}%
\definecolor{currentstroke}{rgb}{0.000000,0.000000,0.000000}%
\pgfsetstrokecolor{currentstroke}%
\pgfsetdash{}{0pt}%
\pgfpathmoveto{\pgfqpoint{3.854822in}{1.866907in}}%
\pgfpathlineto{\pgfqpoint{3.868654in}{1.865662in}}%
\pgfpathlineto{\pgfqpoint{3.882494in}{1.864495in}}%
\pgfpathlineto{\pgfqpoint{3.896342in}{1.863404in}}%
\pgfpathlineto{\pgfqpoint{3.910197in}{1.862389in}}%
\pgfpathlineto{\pgfqpoint{3.918282in}{1.871955in}}%
\pgfpathlineto{\pgfqpoint{3.926361in}{1.881488in}}%
\pgfpathlineto{\pgfqpoint{3.934435in}{1.890986in}}%
\pgfpathlineto{\pgfqpoint{3.942503in}{1.900449in}}%
\pgfpathlineto{\pgfqpoint{3.928658in}{1.901382in}}%
\pgfpathlineto{\pgfqpoint{3.914821in}{1.902391in}}%
\pgfpathlineto{\pgfqpoint{3.900992in}{1.903477in}}%
\pgfpathlineto{\pgfqpoint{3.887170in}{1.904640in}}%
\pgfpathlineto{\pgfqpoint{3.879091in}{1.895251in}}%
\pgfpathlineto{\pgfqpoint{3.871007in}{1.885832in}}%
\pgfpathlineto{\pgfqpoint{3.862917in}{1.876383in}}%
\pgfpathlineto{\pgfqpoint{3.854822in}{1.866907in}}%
\pgfpathclose%
\pgfusepath{fill}%
\end{pgfscope}%
\begin{pgfscope}%
\pgfpathrectangle{\pgfqpoint{1.150000in}{0.150000in}}{\pgfqpoint{5.700000in}{5.700000in}}%
\pgfusepath{clip}%
\pgfsetbuttcap%
\pgfsetroundjoin%
\definecolor{currentfill}{rgb}{0.277134,0.185228,0.489898}%
\pgfsetfillcolor{currentfill}%
\pgfsetfillopacity{0.700000}%
\pgfsetlinewidth{0.000000pt}%
\definecolor{currentstroke}{rgb}{0.000000,0.000000,0.000000}%
\pgfsetstrokecolor{currentstroke}%
\pgfsetdash{}{0pt}%
\pgfpathmoveto{\pgfqpoint{4.436261in}{2.084259in}}%
\pgfpathlineto{\pgfqpoint{4.450274in}{2.085282in}}%
\pgfpathlineto{\pgfqpoint{4.464297in}{2.086378in}}%
\pgfpathlineto{\pgfqpoint{4.478329in}{2.087547in}}%
\pgfpathlineto{\pgfqpoint{4.492371in}{2.088789in}}%
\pgfpathlineto{\pgfqpoint{4.500245in}{2.097229in}}%
\pgfpathlineto{\pgfqpoint{4.508113in}{2.105601in}}%
\pgfpathlineto{\pgfqpoint{4.515974in}{2.113907in}}%
\pgfpathlineto{\pgfqpoint{4.523830in}{2.122147in}}%
\pgfpathlineto{\pgfqpoint{4.509798in}{2.120950in}}%
\pgfpathlineto{\pgfqpoint{4.495777in}{2.119825in}}%
\pgfpathlineto{\pgfqpoint{4.481765in}{2.118772in}}%
\pgfpathlineto{\pgfqpoint{4.467763in}{2.117792in}}%
\pgfpathlineto{\pgfqpoint{4.459897in}{2.109501in}}%
\pgfpathlineto{\pgfqpoint{4.452025in}{2.101148in}}%
\pgfpathlineto{\pgfqpoint{4.444146in}{2.092735in}}%
\pgfpathlineto{\pgfqpoint{4.436261in}{2.084259in}}%
\pgfpathclose%
\pgfusepath{fill}%
\end{pgfscope}%
\begin{pgfscope}%
\pgfpathrectangle{\pgfqpoint{1.150000in}{0.150000in}}{\pgfqpoint{5.700000in}{5.700000in}}%
\pgfusepath{clip}%
\pgfsetbuttcap%
\pgfsetroundjoin%
\definecolor{currentfill}{rgb}{0.278826,0.175490,0.483397}%
\pgfsetfillcolor{currentfill}%
\pgfsetfillopacity{0.700000}%
\pgfsetlinewidth{0.000000pt}%
\definecolor{currentstroke}{rgb}{0.000000,0.000000,0.000000}%
\pgfsetstrokecolor{currentstroke}%
\pgfsetdash{}{0pt}%
\pgfpathmoveto{\pgfqpoint{2.278051in}{2.098135in}}%
\pgfpathlineto{\pgfqpoint{2.291808in}{2.086335in}}%
\pgfpathlineto{\pgfqpoint{2.305564in}{2.074657in}}%
\pgfpathlineto{\pgfqpoint{2.319318in}{2.063101in}}%
\pgfpathlineto{\pgfqpoint{2.333070in}{2.051667in}}%
\pgfpathlineto{\pgfqpoint{2.341958in}{2.053573in}}%
\pgfpathlineto{\pgfqpoint{2.350830in}{2.055685in}}%
\pgfpathlineto{\pgfqpoint{2.359688in}{2.057996in}}%
\pgfpathlineto{\pgfqpoint{2.368530in}{2.060503in}}%
\pgfpathlineto{\pgfqpoint{2.354811in}{2.071621in}}%
\pgfpathlineto{\pgfqpoint{2.341091in}{2.082860in}}%
\pgfpathlineto{\pgfqpoint{2.327369in}{2.094221in}}%
\pgfpathlineto{\pgfqpoint{2.313646in}{2.105704in}}%
\pgfpathlineto{\pgfqpoint{2.304771in}{2.103506in}}%
\pgfpathlineto{\pgfqpoint{2.295880in}{2.101509in}}%
\pgfpathlineto{\pgfqpoint{2.286974in}{2.099717in}}%
\pgfpathlineto{\pgfqpoint{2.278051in}{2.098135in}}%
\pgfpathclose%
\pgfusepath{fill}%
\end{pgfscope}%
\begin{pgfscope}%
\pgfpathrectangle{\pgfqpoint{1.150000in}{0.150000in}}{\pgfqpoint{5.700000in}{5.700000in}}%
\pgfusepath{clip}%
\pgfsetbuttcap%
\pgfsetroundjoin%
\definecolor{currentfill}{rgb}{0.279574,0.170599,0.479997}%
\pgfsetfillcolor{currentfill}%
\pgfsetfillopacity{0.700000}%
\pgfsetlinewidth{0.000000pt}%
\definecolor{currentstroke}{rgb}{0.000000,0.000000,0.000000}%
\pgfsetstrokecolor{currentstroke}%
\pgfsetdash{}{0pt}%
\pgfpathmoveto{\pgfqpoint{4.348662in}{2.046253in}}%
\pgfpathlineto{\pgfqpoint{4.362648in}{2.047007in}}%
\pgfpathlineto{\pgfqpoint{4.376642in}{2.047835in}}%
\pgfpathlineto{\pgfqpoint{4.390647in}{2.048735in}}%
\pgfpathlineto{\pgfqpoint{4.404660in}{2.049709in}}%
\pgfpathlineto{\pgfqpoint{4.412570in}{2.058445in}}%
\pgfpathlineto{\pgfqpoint{4.420473in}{2.067115in}}%
\pgfpathlineto{\pgfqpoint{4.428370in}{2.075719in}}%
\pgfpathlineto{\pgfqpoint{4.436261in}{2.084259in}}%
\pgfpathlineto{\pgfqpoint{4.422258in}{2.083308in}}%
\pgfpathlineto{\pgfqpoint{4.408264in}{2.082430in}}%
\pgfpathlineto{\pgfqpoint{4.394280in}{2.081626in}}%
\pgfpathlineto{\pgfqpoint{4.380305in}{2.080894in}}%
\pgfpathlineto{\pgfqpoint{4.372404in}{2.072324in}}%
\pgfpathlineto{\pgfqpoint{4.364496in}{2.063694in}}%
\pgfpathlineto{\pgfqpoint{4.356582in}{2.055004in}}%
\pgfpathlineto{\pgfqpoint{4.348662in}{2.046253in}}%
\pgfpathclose%
\pgfusepath{fill}%
\end{pgfscope}%
\begin{pgfscope}%
\pgfpathrectangle{\pgfqpoint{1.150000in}{0.150000in}}{\pgfqpoint{5.700000in}{5.700000in}}%
\pgfusepath{clip}%
\pgfsetbuttcap%
\pgfsetroundjoin%
\definecolor{currentfill}{rgb}{0.244972,0.287675,0.537260}%
\pgfsetfillcolor{currentfill}%
\pgfsetfillopacity{0.700000}%
\pgfsetlinewidth{0.000000pt}%
\definecolor{currentstroke}{rgb}{0.000000,0.000000,0.000000}%
\pgfsetstrokecolor{currentstroke}%
\pgfsetdash{}{0pt}%
\pgfpathmoveto{\pgfqpoint{5.017902in}{2.313257in}}%
\pgfpathlineto{\pgfqpoint{5.032132in}{2.315699in}}%
\pgfpathlineto{\pgfqpoint{5.046373in}{2.318211in}}%
\pgfpathlineto{\pgfqpoint{5.060626in}{2.320793in}}%
\pgfpathlineto{\pgfqpoint{5.074890in}{2.323446in}}%
\pgfpathlineto{\pgfqpoint{5.082509in}{2.329612in}}%
\pgfpathlineto{\pgfqpoint{5.090121in}{2.335725in}}%
\pgfpathlineto{\pgfqpoint{5.097726in}{2.341788in}}%
\pgfpathlineto{\pgfqpoint{5.105324in}{2.347804in}}%
\pgfpathlineto{\pgfqpoint{5.091076in}{2.345324in}}%
\pgfpathlineto{\pgfqpoint{5.076840in}{2.342913in}}%
\pgfpathlineto{\pgfqpoint{5.062614in}{2.340573in}}%
\pgfpathlineto{\pgfqpoint{5.048400in}{2.338302in}}%
\pgfpathlineto{\pgfqpoint{5.040786in}{2.332106in}}%
\pgfpathlineto{\pgfqpoint{5.033165in}{2.325869in}}%
\pgfpathlineto{\pgfqpoint{5.025537in}{2.319587in}}%
\pgfpathlineto{\pgfqpoint{5.017902in}{2.313257in}}%
\pgfpathclose%
\pgfusepath{fill}%
\end{pgfscope}%
\begin{pgfscope}%
\pgfpathrectangle{\pgfqpoint{1.150000in}{0.150000in}}{\pgfqpoint{5.700000in}{5.700000in}}%
\pgfusepath{clip}%
\pgfsetbuttcap%
\pgfsetroundjoin%
\definecolor{currentfill}{rgb}{0.212395,0.359683,0.551710}%
\pgfsetfillcolor{currentfill}%
\pgfsetfillopacity{0.700000}%
\pgfsetlinewidth{0.000000pt}%
\definecolor{currentstroke}{rgb}{0.000000,0.000000,0.000000}%
\pgfsetstrokecolor{currentstroke}%
\pgfsetdash{}{0pt}%
\pgfpathmoveto{\pgfqpoint{5.511884in}{2.487001in}}%
\pgfpathlineto{\pgfqpoint{5.526303in}{2.490037in}}%
\pgfpathlineto{\pgfqpoint{5.540734in}{2.493141in}}%
\pgfpathlineto{\pgfqpoint{5.555178in}{2.496314in}}%
\pgfpathlineto{\pgfqpoint{5.569634in}{2.499556in}}%
\pgfpathlineto{\pgfqpoint{5.576996in}{2.503752in}}%
\pgfpathlineto{\pgfqpoint{5.584352in}{2.507944in}}%
\pgfpathlineto{\pgfqpoint{5.591701in}{2.512137in}}%
\pgfpathlineto{\pgfqpoint{5.599045in}{2.516338in}}%
\pgfpathlineto{\pgfqpoint{5.584612in}{2.513375in}}%
\pgfpathlineto{\pgfqpoint{5.570191in}{2.510481in}}%
\pgfpathlineto{\pgfqpoint{5.555782in}{2.507655in}}%
\pgfpathlineto{\pgfqpoint{5.541386in}{2.504897in}}%
\pgfpathlineto{\pgfqpoint{5.534020in}{2.500411in}}%
\pgfpathlineto{\pgfqpoint{5.526647in}{2.495936in}}%
\pgfpathlineto{\pgfqpoint{5.519269in}{2.491468in}}%
\pgfpathlineto{\pgfqpoint{5.511884in}{2.487001in}}%
\pgfpathclose%
\pgfusepath{fill}%
\end{pgfscope}%
\begin{pgfscope}%
\pgfpathrectangle{\pgfqpoint{1.150000in}{0.150000in}}{\pgfqpoint{5.700000in}{5.700000in}}%
\pgfusepath{clip}%
\pgfsetbuttcap%
\pgfsetroundjoin%
\definecolor{currentfill}{rgb}{0.278791,0.062145,0.386592}%
\pgfsetfillcolor{currentfill}%
\pgfsetfillopacity{0.700000}%
\pgfsetlinewidth{0.000000pt}%
\definecolor{currentstroke}{rgb}{0.000000,0.000000,0.000000}%
\pgfsetstrokecolor{currentstroke}%
\pgfsetdash{}{0pt}%
\pgfpathmoveto{\pgfqpoint{3.767087in}{1.834894in}}%
\pgfpathlineto{\pgfqpoint{3.780901in}{1.833238in}}%
\pgfpathlineto{\pgfqpoint{3.794722in}{1.831660in}}%
\pgfpathlineto{\pgfqpoint{3.808550in}{1.830159in}}%
\pgfpathlineto{\pgfqpoint{3.822385in}{1.828735in}}%
\pgfpathlineto{\pgfqpoint{3.830503in}{1.838315in}}%
\pgfpathlineto{\pgfqpoint{3.838615in}{1.847871in}}%
\pgfpathlineto{\pgfqpoint{3.846721in}{1.857402in}}%
\pgfpathlineto{\pgfqpoint{3.854822in}{1.866907in}}%
\pgfpathlineto{\pgfqpoint{3.840997in}{1.868228in}}%
\pgfpathlineto{\pgfqpoint{3.827180in}{1.869627in}}%
\pgfpathlineto{\pgfqpoint{3.813370in}{1.871103in}}%
\pgfpathlineto{\pgfqpoint{3.799567in}{1.872657in}}%
\pgfpathlineto{\pgfqpoint{3.791456in}{1.863247in}}%
\pgfpathlineto{\pgfqpoint{3.783339in}{1.853816in}}%
\pgfpathlineto{\pgfqpoint{3.775216in}{1.844364in}}%
\pgfpathlineto{\pgfqpoint{3.767087in}{1.834894in}}%
\pgfpathclose%
\pgfusepath{fill}%
\end{pgfscope}%
\begin{pgfscope}%
\pgfpathrectangle{\pgfqpoint{1.150000in}{0.150000in}}{\pgfqpoint{5.700000in}{5.700000in}}%
\pgfusepath{clip}%
\pgfsetbuttcap%
\pgfsetroundjoin%
\definecolor{currentfill}{rgb}{0.268510,0.009605,0.335427}%
\pgfsetfillcolor{currentfill}%
\pgfsetfillopacity{0.700000}%
\pgfsetlinewidth{0.000000pt}%
\definecolor{currentstroke}{rgb}{0.000000,0.000000,0.000000}%
\pgfsetstrokecolor{currentstroke}%
\pgfsetdash{}{0pt}%
\pgfpathmoveto{\pgfqpoint{3.074082in}{1.757682in}}%
\pgfpathlineto{\pgfqpoint{3.087784in}{1.752000in}}%
\pgfpathlineto{\pgfqpoint{3.101489in}{1.746406in}}%
\pgfpathlineto{\pgfqpoint{3.115198in}{1.740901in}}%
\pgfpathlineto{\pgfqpoint{3.128911in}{1.735485in}}%
\pgfpathlineto{\pgfqpoint{3.137309in}{1.742964in}}%
\pgfpathlineto{\pgfqpoint{3.145699in}{1.750519in}}%
\pgfpathlineto{\pgfqpoint{3.154082in}{1.758147in}}%
\pgfpathlineto{\pgfqpoint{3.162456in}{1.765846in}}%
\pgfpathlineto{\pgfqpoint{3.148761in}{1.771037in}}%
\pgfpathlineto{\pgfqpoint{3.135071in}{1.776316in}}%
\pgfpathlineto{\pgfqpoint{3.121384in}{1.781684in}}%
\pgfpathlineto{\pgfqpoint{3.107701in}{1.787142in}}%
\pgfpathlineto{\pgfqpoint{3.099308in}{1.779661in}}%
\pgfpathlineto{\pgfqpoint{3.090908in}{1.772255in}}%
\pgfpathlineto{\pgfqpoint{3.082499in}{1.764928in}}%
\pgfpathlineto{\pgfqpoint{3.074082in}{1.757682in}}%
\pgfpathclose%
\pgfusepath{fill}%
\end{pgfscope}%
\begin{pgfscope}%
\pgfpathrectangle{\pgfqpoint{1.150000in}{0.150000in}}{\pgfqpoint{5.700000in}{5.700000in}}%
\pgfusepath{clip}%
\pgfsetbuttcap%
\pgfsetroundjoin%
\definecolor{currentfill}{rgb}{0.271305,0.019942,0.347269}%
\pgfsetfillcolor{currentfill}%
\pgfsetfillopacity{0.700000}%
\pgfsetlinewidth{0.000000pt}%
\definecolor{currentstroke}{rgb}{0.000000,0.000000,0.000000}%
\pgfsetstrokecolor{currentstroke}%
\pgfsetdash{}{0pt}%
\pgfpathmoveto{\pgfqpoint{2.930677in}{1.780282in}}%
\pgfpathlineto{\pgfqpoint{2.944373in}{1.773623in}}%
\pgfpathlineto{\pgfqpoint{2.958072in}{1.767057in}}%
\pgfpathlineto{\pgfqpoint{2.971773in}{1.760583in}}%
\pgfpathlineto{\pgfqpoint{2.985478in}{1.754202in}}%
\pgfpathlineto{\pgfqpoint{2.993950in}{1.760832in}}%
\pgfpathlineto{\pgfqpoint{3.002413in}{1.767563in}}%
\pgfpathlineto{\pgfqpoint{3.010867in}{1.774391in}}%
\pgfpathlineto{\pgfqpoint{3.019312in}{1.781313in}}%
\pgfpathlineto{\pgfqpoint{3.005628in}{1.787448in}}%
\pgfpathlineto{\pgfqpoint{2.991947in}{1.793675in}}%
\pgfpathlineto{\pgfqpoint{2.978269in}{1.799995in}}%
\pgfpathlineto{\pgfqpoint{2.964595in}{1.806408in}}%
\pgfpathlineto{\pgfqpoint{2.956129in}{1.799724in}}%
\pgfpathlineto{\pgfqpoint{2.947655in}{1.793140in}}%
\pgfpathlineto{\pgfqpoint{2.939171in}{1.786658in}}%
\pgfpathlineto{\pgfqpoint{2.930677in}{1.780282in}}%
\pgfpathclose%
\pgfusepath{fill}%
\end{pgfscope}%
\begin{pgfscope}%
\pgfpathrectangle{\pgfqpoint{1.150000in}{0.150000in}}{\pgfqpoint{5.700000in}{5.700000in}}%
\pgfusepath{clip}%
\pgfsetbuttcap%
\pgfsetroundjoin%
\definecolor{currentfill}{rgb}{0.271305,0.019942,0.347269}%
\pgfsetfillcolor{currentfill}%
\pgfsetfillopacity{0.700000}%
\pgfsetlinewidth{0.000000pt}%
\definecolor{currentstroke}{rgb}{0.000000,0.000000,0.000000}%
\pgfsetstrokecolor{currentstroke}%
\pgfsetdash{}{0pt}%
\pgfpathmoveto{\pgfqpoint{3.448416in}{1.764451in}}%
\pgfpathlineto{\pgfqpoint{3.462163in}{1.761090in}}%
\pgfpathlineto{\pgfqpoint{3.475916in}{1.757811in}}%
\pgfpathlineto{\pgfqpoint{3.489675in}{1.754614in}}%
\pgfpathlineto{\pgfqpoint{3.503439in}{1.751497in}}%
\pgfpathlineto{\pgfqpoint{3.511674in}{1.760555in}}%
\pgfpathlineto{\pgfqpoint{3.519903in}{1.769631in}}%
\pgfpathlineto{\pgfqpoint{3.528126in}{1.778722in}}%
\pgfpathlineto{\pgfqpoint{3.536343in}{1.787826in}}%
\pgfpathlineto{\pgfqpoint{3.522592in}{1.790779in}}%
\pgfpathlineto{\pgfqpoint{3.508846in}{1.793814in}}%
\pgfpathlineto{\pgfqpoint{3.495107in}{1.796929in}}%
\pgfpathlineto{\pgfqpoint{3.481373in}{1.800126in}}%
\pgfpathlineto{\pgfqpoint{3.473143in}{1.791178in}}%
\pgfpathlineto{\pgfqpoint{3.464907in}{1.782248in}}%
\pgfpathlineto{\pgfqpoint{3.456665in}{1.773338in}}%
\pgfpathlineto{\pgfqpoint{3.448416in}{1.764451in}}%
\pgfpathclose%
\pgfusepath{fill}%
\end{pgfscope}%
\begin{pgfscope}%
\pgfpathrectangle{\pgfqpoint{1.150000in}{0.150000in}}{\pgfqpoint{5.700000in}{5.700000in}}%
\pgfusepath{clip}%
\pgfsetbuttcap%
\pgfsetroundjoin%
\definecolor{currentfill}{rgb}{0.281887,0.150881,0.465405}%
\pgfsetfillcolor{currentfill}%
\pgfsetfillopacity{0.700000}%
\pgfsetlinewidth{0.000000pt}%
\definecolor{currentstroke}{rgb}{0.000000,0.000000,0.000000}%
\pgfsetstrokecolor{currentstroke}%
\pgfsetdash{}{0pt}%
\pgfpathmoveto{\pgfqpoint{4.261034in}{2.008335in}}%
\pgfpathlineto{\pgfqpoint{4.274992in}{2.008797in}}%
\pgfpathlineto{\pgfqpoint{4.288960in}{2.009333in}}%
\pgfpathlineto{\pgfqpoint{4.302937in}{2.009942in}}%
\pgfpathlineto{\pgfqpoint{4.316923in}{2.010625in}}%
\pgfpathlineto{\pgfqpoint{4.324867in}{2.019626in}}%
\pgfpathlineto{\pgfqpoint{4.332805in}{2.028565in}}%
\pgfpathlineto{\pgfqpoint{4.340737in}{2.037440in}}%
\pgfpathlineto{\pgfqpoint{4.348662in}{2.046253in}}%
\pgfpathlineto{\pgfqpoint{4.334687in}{2.045572in}}%
\pgfpathlineto{\pgfqpoint{4.320720in}{2.044965in}}%
\pgfpathlineto{\pgfqpoint{4.306762in}{2.044431in}}%
\pgfpathlineto{\pgfqpoint{4.292814in}{2.043971in}}%
\pgfpathlineto{\pgfqpoint{4.284878in}{2.035148in}}%
\pgfpathlineto{\pgfqpoint{4.276936in}{2.026268in}}%
\pgfpathlineto{\pgfqpoint{4.268988in}{2.017331in}}%
\pgfpathlineto{\pgfqpoint{4.261034in}{2.008335in}}%
\pgfpathclose%
\pgfusepath{fill}%
\end{pgfscope}%
\begin{pgfscope}%
\pgfpathrectangle{\pgfqpoint{1.150000in}{0.150000in}}{\pgfqpoint{5.700000in}{5.700000in}}%
\pgfusepath{clip}%
\pgfsetbuttcap%
\pgfsetroundjoin%
\definecolor{currentfill}{rgb}{0.268510,0.009605,0.335427}%
\pgfsetfillcolor{currentfill}%
\pgfsetfillopacity{0.700000}%
\pgfsetlinewidth{0.000000pt}%
\definecolor{currentstroke}{rgb}{0.000000,0.000000,0.000000}%
\pgfsetstrokecolor{currentstroke}%
\pgfsetdash{}{0pt}%
\pgfpathmoveto{\pgfqpoint{3.217278in}{1.745956in}}%
\pgfpathlineto{\pgfqpoint{3.230994in}{1.741200in}}%
\pgfpathlineto{\pgfqpoint{3.244715in}{1.736529in}}%
\pgfpathlineto{\pgfqpoint{3.258441in}{1.731945in}}%
\pgfpathlineto{\pgfqpoint{3.272171in}{1.727445in}}%
\pgfpathlineto{\pgfqpoint{3.280504in}{1.735635in}}%
\pgfpathlineto{\pgfqpoint{3.288829in}{1.743880in}}%
\pgfpathlineto{\pgfqpoint{3.297148in}{1.752175in}}%
\pgfpathlineto{\pgfqpoint{3.305459in}{1.760518in}}%
\pgfpathlineto{\pgfqpoint{3.291745in}{1.764813in}}%
\pgfpathlineto{\pgfqpoint{3.278036in}{1.769193in}}%
\pgfpathlineto{\pgfqpoint{3.264331in}{1.773659in}}%
\pgfpathlineto{\pgfqpoint{3.250631in}{1.778210in}}%
\pgfpathlineto{\pgfqpoint{3.242304in}{1.770064in}}%
\pgfpathlineto{\pgfqpoint{3.233969in}{1.761971in}}%
\pgfpathlineto{\pgfqpoint{3.225627in}{1.753934in}}%
\pgfpathlineto{\pgfqpoint{3.217278in}{1.745956in}}%
\pgfpathclose%
\pgfusepath{fill}%
\end{pgfscope}%
\begin{pgfscope}%
\pgfpathrectangle{\pgfqpoint{1.150000in}{0.150000in}}{\pgfqpoint{5.700000in}{5.700000in}}%
\pgfusepath{clip}%
\pgfsetbuttcap%
\pgfsetroundjoin%
\definecolor{currentfill}{rgb}{0.250425,0.274290,0.533103}%
\pgfsetfillcolor{currentfill}%
\pgfsetfillopacity{0.700000}%
\pgfsetlinewidth{0.000000pt}%
\definecolor{currentstroke}{rgb}{0.000000,0.000000,0.000000}%
\pgfsetstrokecolor{currentstroke}%
\pgfsetdash{}{0pt}%
\pgfpathmoveto{\pgfqpoint{4.930424in}{2.277733in}}%
\pgfpathlineto{\pgfqpoint{4.944624in}{2.280044in}}%
\pgfpathlineto{\pgfqpoint{4.958836in}{2.282425in}}%
\pgfpathlineto{\pgfqpoint{4.973058in}{2.284876in}}%
\pgfpathlineto{\pgfqpoint{4.987292in}{2.287398in}}%
\pgfpathlineto{\pgfqpoint{4.994955in}{2.293950in}}%
\pgfpathlineto{\pgfqpoint{5.002611in}{2.300442in}}%
\pgfpathlineto{\pgfqpoint{5.010260in}{2.306876in}}%
\pgfpathlineto{\pgfqpoint{5.017902in}{2.313257in}}%
\pgfpathlineto{\pgfqpoint{5.003684in}{2.310886in}}%
\pgfpathlineto{\pgfqpoint{4.989476in}{2.308585in}}%
\pgfpathlineto{\pgfqpoint{4.975280in}{2.306354in}}%
\pgfpathlineto{\pgfqpoint{4.961094in}{2.304194in}}%
\pgfpathlineto{\pgfqpoint{4.953437in}{2.297655in}}%
\pgfpathlineto{\pgfqpoint{4.945773in}{2.291067in}}%
\pgfpathlineto{\pgfqpoint{4.938102in}{2.284428in}}%
\pgfpathlineto{\pgfqpoint{4.930424in}{2.277733in}}%
\pgfpathclose%
\pgfusepath{fill}%
\end{pgfscope}%
\begin{pgfscope}%
\pgfpathrectangle{\pgfqpoint{1.150000in}{0.150000in}}{\pgfqpoint{5.700000in}{5.700000in}}%
\pgfusepath{clip}%
\pgfsetbuttcap%
\pgfsetroundjoin%
\definecolor{currentfill}{rgb}{0.281412,0.155834,0.469201}%
\pgfsetfillcolor{currentfill}%
\pgfsetfillopacity{0.700000}%
\pgfsetlinewidth{0.000000pt}%
\definecolor{currentstroke}{rgb}{0.000000,0.000000,0.000000}%
\pgfsetstrokecolor{currentstroke}%
\pgfsetdash{}{0pt}%
\pgfpathmoveto{\pgfqpoint{2.333070in}{2.051667in}}%
\pgfpathlineto{\pgfqpoint{2.346820in}{2.040352in}}%
\pgfpathlineto{\pgfqpoint{2.360569in}{2.029157in}}%
\pgfpathlineto{\pgfqpoint{2.374317in}{2.018080in}}%
\pgfpathlineto{\pgfqpoint{2.388063in}{2.007120in}}%
\pgfpathlineto{\pgfqpoint{2.396917in}{2.009350in}}%
\pgfpathlineto{\pgfqpoint{2.405757in}{2.011780in}}%
\pgfpathlineto{\pgfqpoint{2.414582in}{2.014405in}}%
\pgfpathlineto{\pgfqpoint{2.423392in}{2.017219in}}%
\pgfpathlineto{\pgfqpoint{2.409678in}{2.027863in}}%
\pgfpathlineto{\pgfqpoint{2.395963in}{2.038625in}}%
\pgfpathlineto{\pgfqpoint{2.382247in}{2.049504in}}%
\pgfpathlineto{\pgfqpoint{2.368530in}{2.060503in}}%
\pgfpathlineto{\pgfqpoint{2.359688in}{2.057996in}}%
\pgfpathlineto{\pgfqpoint{2.350830in}{2.055685in}}%
\pgfpathlineto{\pgfqpoint{2.341958in}{2.053573in}}%
\pgfpathlineto{\pgfqpoint{2.333070in}{2.051667in}}%
\pgfpathclose%
\pgfusepath{fill}%
\end{pgfscope}%
\begin{pgfscope}%
\pgfpathrectangle{\pgfqpoint{1.150000in}{0.150000in}}{\pgfqpoint{5.700000in}{5.700000in}}%
\pgfusepath{clip}%
\pgfsetbuttcap%
\pgfsetroundjoin%
\definecolor{currentfill}{rgb}{0.281446,0.084320,0.407414}%
\pgfsetfillcolor{currentfill}%
\pgfsetfillopacity{0.700000}%
\pgfsetlinewidth{0.000000pt}%
\definecolor{currentstroke}{rgb}{0.000000,0.000000,0.000000}%
\pgfsetstrokecolor{currentstroke}%
\pgfsetdash{}{0pt}%
\pgfpathmoveto{\pgfqpoint{2.587922in}{1.898293in}}%
\pgfpathlineto{\pgfqpoint{2.601634in}{1.889091in}}%
\pgfpathlineto{\pgfqpoint{2.615346in}{1.879996in}}%
\pgfpathlineto{\pgfqpoint{2.629058in}{1.871005in}}%
\pgfpathlineto{\pgfqpoint{2.642772in}{1.862118in}}%
\pgfpathlineto{\pgfqpoint{2.651450in}{1.866317in}}%
\pgfpathlineto{\pgfqpoint{2.660116in}{1.870675in}}%
\pgfpathlineto{\pgfqpoint{2.668770in}{1.875187in}}%
\pgfpathlineto{\pgfqpoint{2.677412in}{1.879851in}}%
\pgfpathlineto{\pgfqpoint{2.663726in}{1.888447in}}%
\pgfpathlineto{\pgfqpoint{2.650040in}{1.897147in}}%
\pgfpathlineto{\pgfqpoint{2.636356in}{1.905952in}}%
\pgfpathlineto{\pgfqpoint{2.622672in}{1.914863in}}%
\pgfpathlineto{\pgfqpoint{2.614003in}{1.910482in}}%
\pgfpathlineto{\pgfqpoint{2.605322in}{1.906257in}}%
\pgfpathlineto{\pgfqpoint{2.596629in}{1.902193in}}%
\pgfpathlineto{\pgfqpoint{2.587922in}{1.898293in}}%
\pgfpathclose%
\pgfusepath{fill}%
\end{pgfscope}%
\begin{pgfscope}%
\pgfpathrectangle{\pgfqpoint{1.150000in}{0.150000in}}{\pgfqpoint{5.700000in}{5.700000in}}%
\pgfusepath{clip}%
\pgfsetbuttcap%
\pgfsetroundjoin%
\definecolor{currentfill}{rgb}{0.282884,0.135920,0.453427}%
\pgfsetfillcolor{currentfill}%
\pgfsetfillopacity{0.700000}%
\pgfsetlinewidth{0.000000pt}%
\definecolor{currentstroke}{rgb}{0.000000,0.000000,0.000000}%
\pgfsetstrokecolor{currentstroke}%
\pgfsetdash{}{0pt}%
\pgfpathmoveto{\pgfqpoint{4.173375in}{1.970731in}}%
\pgfpathlineto{\pgfqpoint{4.187308in}{1.970878in}}%
\pgfpathlineto{\pgfqpoint{4.201250in}{1.971099in}}%
\pgfpathlineto{\pgfqpoint{4.215200in}{1.971394in}}%
\pgfpathlineto{\pgfqpoint{4.229159in}{1.971763in}}%
\pgfpathlineto{\pgfqpoint{4.237137in}{1.980995in}}%
\pgfpathlineto{\pgfqpoint{4.245108in}{1.990167in}}%
\pgfpathlineto{\pgfqpoint{4.253074in}{1.999281in}}%
\pgfpathlineto{\pgfqpoint{4.261034in}{2.008335in}}%
\pgfpathlineto{\pgfqpoint{4.247084in}{2.007947in}}%
\pgfpathlineto{\pgfqpoint{4.233144in}{2.007633in}}%
\pgfpathlineto{\pgfqpoint{4.219212in}{2.007393in}}%
\pgfpathlineto{\pgfqpoint{4.205290in}{2.007227in}}%
\pgfpathlineto{\pgfqpoint{4.197320in}{1.998184in}}%
\pgfpathlineto{\pgfqpoint{4.189344in}{1.989087in}}%
\pgfpathlineto{\pgfqpoint{4.181363in}{1.979936in}}%
\pgfpathlineto{\pgfqpoint{4.173375in}{1.970731in}}%
\pgfpathclose%
\pgfusepath{fill}%
\end{pgfscope}%
\begin{pgfscope}%
\pgfpathrectangle{\pgfqpoint{1.150000in}{0.150000in}}{\pgfqpoint{5.700000in}{5.700000in}}%
\pgfusepath{clip}%
\pgfsetbuttcap%
\pgfsetroundjoin%
\definecolor{currentfill}{rgb}{0.216210,0.351535,0.550627}%
\pgfsetfillcolor{currentfill}%
\pgfsetfillopacity{0.700000}%
\pgfsetlinewidth{0.000000pt}%
\definecolor{currentstroke}{rgb}{0.000000,0.000000,0.000000}%
\pgfsetstrokecolor{currentstroke}%
\pgfsetdash{}{0pt}%
\pgfpathmoveto{\pgfqpoint{5.424637in}{2.456570in}}%
\pgfpathlineto{\pgfqpoint{5.439029in}{2.459588in}}%
\pgfpathlineto{\pgfqpoint{5.453433in}{2.462675in}}%
\pgfpathlineto{\pgfqpoint{5.467849in}{2.465831in}}%
\pgfpathlineto{\pgfqpoint{5.482277in}{2.469056in}}%
\pgfpathlineto{\pgfqpoint{5.489689in}{2.473563in}}%
\pgfpathlineto{\pgfqpoint{5.497094in}{2.478053in}}%
\pgfpathlineto{\pgfqpoint{5.504492in}{2.482531in}}%
\pgfpathlineto{\pgfqpoint{5.511884in}{2.487001in}}%
\pgfpathlineto{\pgfqpoint{5.497477in}{2.484034in}}%
\pgfpathlineto{\pgfqpoint{5.483082in}{2.481136in}}%
\pgfpathlineto{\pgfqpoint{5.468700in}{2.478306in}}%
\pgfpathlineto{\pgfqpoint{5.454329in}{2.475545in}}%
\pgfpathlineto{\pgfqpoint{5.446916in}{2.470810in}}%
\pgfpathlineto{\pgfqpoint{5.439496in}{2.466073in}}%
\pgfpathlineto{\pgfqpoint{5.432070in}{2.461327in}}%
\pgfpathlineto{\pgfqpoint{5.424637in}{2.456570in}}%
\pgfpathclose%
\pgfusepath{fill}%
\end{pgfscope}%
\begin{pgfscope}%
\pgfpathrectangle{\pgfqpoint{1.150000in}{0.150000in}}{\pgfqpoint{5.700000in}{5.700000in}}%
\pgfusepath{clip}%
\pgfsetbuttcap%
\pgfsetroundjoin%
\definecolor{currentfill}{rgb}{0.276022,0.044167,0.370164}%
\pgfsetfillcolor{currentfill}%
\pgfsetfillopacity{0.700000}%
\pgfsetlinewidth{0.000000pt}%
\definecolor{currentstroke}{rgb}{0.000000,0.000000,0.000000}%
\pgfsetstrokecolor{currentstroke}%
\pgfsetdash{}{0pt}%
\pgfpathmoveto{\pgfqpoint{2.786949in}{1.814741in}}%
\pgfpathlineto{\pgfqpoint{2.800649in}{1.807050in}}%
\pgfpathlineto{\pgfqpoint{2.814351in}{1.799457in}}%
\pgfpathlineto{\pgfqpoint{2.828054in}{1.791960in}}%
\pgfpathlineto{\pgfqpoint{2.841760in}{1.784561in}}%
\pgfpathlineto{\pgfqpoint{2.850316in}{1.790199in}}%
\pgfpathlineto{\pgfqpoint{2.858861in}{1.795965in}}%
\pgfpathlineto{\pgfqpoint{2.867397in}{1.801852in}}%
\pgfpathlineto{\pgfqpoint{2.875922in}{1.807859in}}%
\pgfpathlineto{\pgfqpoint{2.862239in}{1.814991in}}%
\pgfpathlineto{\pgfqpoint{2.848559in}{1.822219in}}%
\pgfpathlineto{\pgfqpoint{2.834881in}{1.829545in}}%
\pgfpathlineto{\pgfqpoint{2.821205in}{1.836967in}}%
\pgfpathlineto{\pgfqpoint{2.812657in}{1.831221in}}%
\pgfpathlineto{\pgfqpoint{2.804098in}{1.825598in}}%
\pgfpathlineto{\pgfqpoint{2.795529in}{1.820104in}}%
\pgfpathlineto{\pgfqpoint{2.786949in}{1.814741in}}%
\pgfpathclose%
\pgfusepath{fill}%
\end{pgfscope}%
\begin{pgfscope}%
\pgfpathrectangle{\pgfqpoint{1.150000in}{0.150000in}}{\pgfqpoint{5.700000in}{5.700000in}}%
\pgfusepath{clip}%
\pgfsetbuttcap%
\pgfsetroundjoin%
\definecolor{currentfill}{rgb}{0.276022,0.044167,0.370164}%
\pgfsetfillcolor{currentfill}%
\pgfsetfillopacity{0.700000}%
\pgfsetlinewidth{0.000000pt}%
\definecolor{currentstroke}{rgb}{0.000000,0.000000,0.000000}%
\pgfsetstrokecolor{currentstroke}%
\pgfsetdash{}{0pt}%
\pgfpathmoveto{\pgfqpoint{3.679287in}{1.804746in}}%
\pgfpathlineto{\pgfqpoint{3.693084in}{1.802655in}}%
\pgfpathlineto{\pgfqpoint{3.706888in}{1.800642in}}%
\pgfpathlineto{\pgfqpoint{3.720699in}{1.798707in}}%
\pgfpathlineto{\pgfqpoint{3.734516in}{1.796851in}}%
\pgfpathlineto{\pgfqpoint{3.742668in}{1.806383in}}%
\pgfpathlineto{\pgfqpoint{3.750813in}{1.815902in}}%
\pgfpathlineto{\pgfqpoint{3.758953in}{1.825406in}}%
\pgfpathlineto{\pgfqpoint{3.767087in}{1.834894in}}%
\pgfpathlineto{\pgfqpoint{3.753281in}{1.836628in}}%
\pgfpathlineto{\pgfqpoint{3.739481in}{1.838440in}}%
\pgfpathlineto{\pgfqpoint{3.725689in}{1.840330in}}%
\pgfpathlineto{\pgfqpoint{3.711904in}{1.842299in}}%
\pgfpathlineto{\pgfqpoint{3.703758in}{1.832926in}}%
\pgfpathlineto{\pgfqpoint{3.695607in}{1.823542in}}%
\pgfpathlineto{\pgfqpoint{3.687450in}{1.814148in}}%
\pgfpathlineto{\pgfqpoint{3.679287in}{1.804746in}}%
\pgfpathclose%
\pgfusepath{fill}%
\end{pgfscope}%
\begin{pgfscope}%
\pgfpathrectangle{\pgfqpoint{1.150000in}{0.150000in}}{\pgfqpoint{5.700000in}{5.700000in}}%
\pgfusepath{clip}%
\pgfsetbuttcap%
\pgfsetroundjoin%
\definecolor{currentfill}{rgb}{0.255645,0.260703,0.528312}%
\pgfsetfillcolor{currentfill}%
\pgfsetfillopacity{0.700000}%
\pgfsetlinewidth{0.000000pt}%
\definecolor{currentstroke}{rgb}{0.000000,0.000000,0.000000}%
\pgfsetstrokecolor{currentstroke}%
\pgfsetdash{}{0pt}%
\pgfpathmoveto{\pgfqpoint{4.842894in}{2.241306in}}%
\pgfpathlineto{\pgfqpoint{4.857065in}{2.243462in}}%
\pgfpathlineto{\pgfqpoint{4.871246in}{2.245689in}}%
\pgfpathlineto{\pgfqpoint{4.885438in}{2.247988in}}%
\pgfpathlineto{\pgfqpoint{4.899642in}{2.250357in}}%
\pgfpathlineto{\pgfqpoint{4.907348in}{2.257296in}}%
\pgfpathlineto{\pgfqpoint{4.915047in}{2.264170in}}%
\pgfpathlineto{\pgfqpoint{4.922739in}{2.270982in}}%
\pgfpathlineto{\pgfqpoint{4.930424in}{2.277733in}}%
\pgfpathlineto{\pgfqpoint{4.916235in}{2.275494in}}%
\pgfpathlineto{\pgfqpoint{4.902056in}{2.273325in}}%
\pgfpathlineto{\pgfqpoint{4.887889in}{2.271226in}}%
\pgfpathlineto{\pgfqpoint{4.873732in}{2.269198in}}%
\pgfpathlineto{\pgfqpoint{4.866033in}{2.262310in}}%
\pgfpathlineto{\pgfqpoint{4.858327in}{2.255367in}}%
\pgfpathlineto{\pgfqpoint{4.850614in}{2.248366in}}%
\pgfpathlineto{\pgfqpoint{4.842894in}{2.241306in}}%
\pgfpathclose%
\pgfusepath{fill}%
\end{pgfscope}%
\begin{pgfscope}%
\pgfpathrectangle{\pgfqpoint{1.150000in}{0.150000in}}{\pgfqpoint{5.700000in}{5.700000in}}%
\pgfusepath{clip}%
\pgfsetbuttcap%
\pgfsetroundjoin%
\definecolor{currentfill}{rgb}{0.283229,0.120777,0.440584}%
\pgfsetfillcolor{currentfill}%
\pgfsetfillopacity{0.700000}%
\pgfsetlinewidth{0.000000pt}%
\definecolor{currentstroke}{rgb}{0.000000,0.000000,0.000000}%
\pgfsetstrokecolor{currentstroke}%
\pgfsetdash{}{0pt}%
\pgfpathmoveto{\pgfqpoint{4.085686in}{1.933689in}}%
\pgfpathlineto{\pgfqpoint{4.099594in}{1.933497in}}%
\pgfpathlineto{\pgfqpoint{4.113510in}{1.933380in}}%
\pgfpathlineto{\pgfqpoint{4.127435in}{1.933338in}}%
\pgfpathlineto{\pgfqpoint{4.141369in}{1.933371in}}%
\pgfpathlineto{\pgfqpoint{4.149379in}{1.942792in}}%
\pgfpathlineto{\pgfqpoint{4.157384in}{1.952159in}}%
\pgfpathlineto{\pgfqpoint{4.165382in}{1.961472in}}%
\pgfpathlineto{\pgfqpoint{4.173375in}{1.970731in}}%
\pgfpathlineto{\pgfqpoint{4.159451in}{1.970659in}}%
\pgfpathlineto{\pgfqpoint{4.145536in}{1.970661in}}%
\pgfpathlineto{\pgfqpoint{4.131630in}{1.970738in}}%
\pgfpathlineto{\pgfqpoint{4.117731in}{1.970889in}}%
\pgfpathlineto{\pgfqpoint{4.109728in}{1.961663in}}%
\pgfpathlineto{\pgfqpoint{4.101720in}{1.952387in}}%
\pgfpathlineto{\pgfqpoint{4.093706in}{1.943062in}}%
\pgfpathlineto{\pgfqpoint{4.085686in}{1.933689in}}%
\pgfpathclose%
\pgfusepath{fill}%
\end{pgfscope}%
\begin{pgfscope}%
\pgfpathrectangle{\pgfqpoint{1.150000in}{0.150000in}}{\pgfqpoint{5.700000in}{5.700000in}}%
\pgfusepath{clip}%
\pgfsetbuttcap%
\pgfsetroundjoin%
\definecolor{currentfill}{rgb}{0.269944,0.014625,0.341379}%
\pgfsetfillcolor{currentfill}%
\pgfsetfillopacity{0.700000}%
\pgfsetlinewidth{0.000000pt}%
\definecolor{currentstroke}{rgb}{0.000000,0.000000,0.000000}%
\pgfsetstrokecolor{currentstroke}%
\pgfsetdash{}{0pt}%
\pgfpathmoveto{\pgfqpoint{3.360366in}{1.744182in}}%
\pgfpathlineto{\pgfqpoint{3.374106in}{1.740307in}}%
\pgfpathlineto{\pgfqpoint{3.387851in}{1.736515in}}%
\pgfpathlineto{\pgfqpoint{3.401601in}{1.732805in}}%
\pgfpathlineto{\pgfqpoint{3.415357in}{1.729178in}}%
\pgfpathlineto{\pgfqpoint{3.423631in}{1.737950in}}%
\pgfpathlineto{\pgfqpoint{3.431899in}{1.746755in}}%
\pgfpathlineto{\pgfqpoint{3.440161in}{1.755589in}}%
\pgfpathlineto{\pgfqpoint{3.448416in}{1.764451in}}%
\pgfpathlineto{\pgfqpoint{3.434674in}{1.767894in}}%
\pgfpathlineto{\pgfqpoint{3.420939in}{1.771419in}}%
\pgfpathlineto{\pgfqpoint{3.407208in}{1.775027in}}%
\pgfpathlineto{\pgfqpoint{3.393483in}{1.778718in}}%
\pgfpathlineto{\pgfqpoint{3.385214in}{1.770033in}}%
\pgfpathlineto{\pgfqpoint{3.376938in}{1.761380in}}%
\pgfpathlineto{\pgfqpoint{3.368655in}{1.752762in}}%
\pgfpathlineto{\pgfqpoint{3.360366in}{1.744182in}}%
\pgfpathclose%
\pgfusepath{fill}%
\end{pgfscope}%
\begin{pgfscope}%
\pgfpathrectangle{\pgfqpoint{1.150000in}{0.150000in}}{\pgfqpoint{5.700000in}{5.700000in}}%
\pgfusepath{clip}%
\pgfsetbuttcap%
\pgfsetroundjoin%
\definecolor{currentfill}{rgb}{0.197636,0.391528,0.554969}%
\pgfsetfillcolor{currentfill}%
\pgfsetfillopacity{0.700000}%
\pgfsetlinewidth{0.000000pt}%
\definecolor{currentstroke}{rgb}{0.000000,0.000000,0.000000}%
\pgfsetstrokecolor{currentstroke}%
\pgfsetdash{}{0pt}%
\pgfpathmoveto{\pgfqpoint{5.744073in}{2.557070in}}%
\pgfpathlineto{\pgfqpoint{5.758594in}{2.560347in}}%
\pgfpathlineto{\pgfqpoint{5.773128in}{2.563693in}}%
\pgfpathlineto{\pgfqpoint{5.787674in}{2.567106in}}%
\pgfpathlineto{\pgfqpoint{5.794919in}{2.570556in}}%
\pgfpathlineto{\pgfqpoint{5.802158in}{2.574029in}}%
\pgfpathlineto{\pgfqpoint{5.809391in}{2.577533in}}%
\pgfpathlineto{\pgfqpoint{5.816619in}{2.581073in}}%
\pgfpathlineto{\pgfqpoint{5.802099in}{2.577981in}}%
\pgfpathlineto{\pgfqpoint{5.787592in}{2.574958in}}%
\pgfpathlineto{\pgfqpoint{5.773097in}{2.572002in}}%
\pgfpathlineto{\pgfqpoint{5.765849in}{2.568215in}}%
\pgfpathlineto{\pgfqpoint{5.758596in}{2.564468in}}%
\pgfpathlineto{\pgfqpoint{5.751338in}{2.560755in}}%
\pgfpathlineto{\pgfqpoint{5.744073in}{2.557070in}}%
\pgfpathclose%
\pgfusepath{fill}%
\end{pgfscope}%
\begin{pgfscope}%
\pgfpathrectangle{\pgfqpoint{1.150000in}{0.150000in}}{\pgfqpoint{5.700000in}{5.700000in}}%
\pgfusepath{clip}%
\pgfsetbuttcap%
\pgfsetroundjoin%
\definecolor{currentfill}{rgb}{0.260571,0.246922,0.522828}%
\pgfsetfillcolor{currentfill}%
\pgfsetfillopacity{0.700000}%
\pgfsetlinewidth{0.000000pt}%
\definecolor{currentstroke}{rgb}{0.000000,0.000000,0.000000}%
\pgfsetstrokecolor{currentstroke}%
\pgfsetdash{}{0pt}%
\pgfpathmoveto{\pgfqpoint{4.755319in}{2.204069in}}%
\pgfpathlineto{\pgfqpoint{4.769460in}{2.206049in}}%
\pgfpathlineto{\pgfqpoint{4.783611in}{2.208100in}}%
\pgfpathlineto{\pgfqpoint{4.797773in}{2.210222in}}%
\pgfpathlineto{\pgfqpoint{4.811946in}{2.212416in}}%
\pgfpathlineto{\pgfqpoint{4.819693in}{2.219740in}}%
\pgfpathlineto{\pgfqpoint{4.827434in}{2.226995in}}%
\pgfpathlineto{\pgfqpoint{4.835168in}{2.234182in}}%
\pgfpathlineto{\pgfqpoint{4.842894in}{2.241306in}}%
\pgfpathlineto{\pgfqpoint{4.828735in}{2.239220in}}%
\pgfpathlineto{\pgfqpoint{4.814586in}{2.237206in}}%
\pgfpathlineto{\pgfqpoint{4.800448in}{2.235262in}}%
\pgfpathlineto{\pgfqpoint{4.786320in}{2.233390in}}%
\pgfpathlineto{\pgfqpoint{4.778580in}{2.226151in}}%
\pgfpathlineto{\pgfqpoint{4.770833in}{2.218853in}}%
\pgfpathlineto{\pgfqpoint{4.763080in}{2.211493in}}%
\pgfpathlineto{\pgfqpoint{4.755319in}{2.204069in}}%
\pgfpathclose%
\pgfusepath{fill}%
\end{pgfscope}%
\begin{pgfscope}%
\pgfpathrectangle{\pgfqpoint{1.150000in}{0.150000in}}{\pgfqpoint{5.700000in}{5.700000in}}%
\pgfusepath{clip}%
\pgfsetbuttcap%
\pgfsetroundjoin%
\definecolor{currentfill}{rgb}{0.221989,0.339161,0.548752}%
\pgfsetfillcolor{currentfill}%
\pgfsetfillopacity{0.700000}%
\pgfsetlinewidth{0.000000pt}%
\definecolor{currentstroke}{rgb}{0.000000,0.000000,0.000000}%
\pgfsetstrokecolor{currentstroke}%
\pgfsetdash{}{0pt}%
\pgfpathmoveto{\pgfqpoint{5.337309in}{2.425005in}}%
\pgfpathlineto{\pgfqpoint{5.351673in}{2.427982in}}%
\pgfpathlineto{\pgfqpoint{5.366048in}{2.431030in}}%
\pgfpathlineto{\pgfqpoint{5.380436in}{2.434146in}}%
\pgfpathlineto{\pgfqpoint{5.394836in}{2.437332in}}%
\pgfpathlineto{\pgfqpoint{5.402297in}{2.442181in}}%
\pgfpathlineto{\pgfqpoint{5.409751in}{2.447001in}}%
\pgfpathlineto{\pgfqpoint{5.417197in}{2.451796in}}%
\pgfpathlineto{\pgfqpoint{5.424637in}{2.456570in}}%
\pgfpathlineto{\pgfqpoint{5.410257in}{2.453621in}}%
\pgfpathlineto{\pgfqpoint{5.395890in}{2.450741in}}%
\pgfpathlineto{\pgfqpoint{5.381534in}{2.447930in}}%
\pgfpathlineto{\pgfqpoint{5.367190in}{2.445188in}}%
\pgfpathlineto{\pgfqpoint{5.359730in}{2.440171in}}%
\pgfpathlineto{\pgfqpoint{5.352263in}{2.435137in}}%
\pgfpathlineto{\pgfqpoint{5.344790in}{2.430083in}}%
\pgfpathlineto{\pgfqpoint{5.337309in}{2.425005in}}%
\pgfpathclose%
\pgfusepath{fill}%
\end{pgfscope}%
\begin{pgfscope}%
\pgfpathrectangle{\pgfqpoint{1.150000in}{0.150000in}}{\pgfqpoint{5.700000in}{5.700000in}}%
\pgfusepath{clip}%
\pgfsetbuttcap%
\pgfsetroundjoin%
\definecolor{currentfill}{rgb}{0.282623,0.140926,0.457517}%
\pgfsetfillcolor{currentfill}%
\pgfsetfillopacity{0.700000}%
\pgfsetlinewidth{0.000000pt}%
\definecolor{currentstroke}{rgb}{0.000000,0.000000,0.000000}%
\pgfsetstrokecolor{currentstroke}%
\pgfsetdash{}{0pt}%
\pgfpathmoveto{\pgfqpoint{2.388063in}{2.007120in}}%
\pgfpathlineto{\pgfqpoint{2.401808in}{1.996276in}}%
\pgfpathlineto{\pgfqpoint{2.415552in}{1.985548in}}%
\pgfpathlineto{\pgfqpoint{2.429295in}{1.974934in}}%
\pgfpathlineto{\pgfqpoint{2.443038in}{1.964435in}}%
\pgfpathlineto{\pgfqpoint{2.451860in}{1.966988in}}%
\pgfpathlineto{\pgfqpoint{2.460667in}{1.969736in}}%
\pgfpathlineto{\pgfqpoint{2.469460in}{1.972672in}}%
\pgfpathlineto{\pgfqpoint{2.478239in}{1.975792in}}%
\pgfpathlineto{\pgfqpoint{2.464528in}{1.985978in}}%
\pgfpathlineto{\pgfqpoint{2.450817in}{1.996277in}}%
\pgfpathlineto{\pgfqpoint{2.437105in}{2.006690in}}%
\pgfpathlineto{\pgfqpoint{2.423392in}{2.017219in}}%
\pgfpathlineto{\pgfqpoint{2.414582in}{2.014405in}}%
\pgfpathlineto{\pgfqpoint{2.405757in}{2.011780in}}%
\pgfpathlineto{\pgfqpoint{2.396917in}{2.009350in}}%
\pgfpathlineto{\pgfqpoint{2.388063in}{2.007120in}}%
\pgfpathclose%
\pgfusepath{fill}%
\end{pgfscope}%
\begin{pgfscope}%
\pgfpathrectangle{\pgfqpoint{1.150000in}{0.150000in}}{\pgfqpoint{5.700000in}{5.700000in}}%
\pgfusepath{clip}%
\pgfsetbuttcap%
\pgfsetroundjoin%
\definecolor{currentfill}{rgb}{0.273809,0.031497,0.358853}%
\pgfsetfillcolor{currentfill}%
\pgfsetfillopacity{0.700000}%
\pgfsetlinewidth{0.000000pt}%
\definecolor{currentstroke}{rgb}{0.000000,0.000000,0.000000}%
\pgfsetstrokecolor{currentstroke}%
\pgfsetdash{}{0pt}%
\pgfpathmoveto{\pgfqpoint{3.591410in}{1.776821in}}%
\pgfpathlineto{\pgfqpoint{3.605193in}{1.774269in}}%
\pgfpathlineto{\pgfqpoint{3.618982in}{1.771798in}}%
\pgfpathlineto{\pgfqpoint{3.632777in}{1.769405in}}%
\pgfpathlineto{\pgfqpoint{3.646579in}{1.767092in}}%
\pgfpathlineto{\pgfqpoint{3.654765in}{1.776509in}}%
\pgfpathlineto{\pgfqpoint{3.662945in}{1.785925in}}%
\pgfpathlineto{\pgfqpoint{3.671119in}{1.795338in}}%
\pgfpathlineto{\pgfqpoint{3.679287in}{1.804746in}}%
\pgfpathlineto{\pgfqpoint{3.665497in}{1.806917in}}%
\pgfpathlineto{\pgfqpoint{3.651714in}{1.809166in}}%
\pgfpathlineto{\pgfqpoint{3.637937in}{1.811495in}}%
\pgfpathlineto{\pgfqpoint{3.624167in}{1.813903in}}%
\pgfpathlineto{\pgfqpoint{3.615986in}{1.804630in}}%
\pgfpathlineto{\pgfqpoint{3.607800in}{1.795357in}}%
\pgfpathlineto{\pgfqpoint{3.599608in}{1.786087in}}%
\pgfpathlineto{\pgfqpoint{3.591410in}{1.776821in}}%
\pgfpathclose%
\pgfusepath{fill}%
\end{pgfscope}%
\begin{pgfscope}%
\pgfpathrectangle{\pgfqpoint{1.150000in}{0.150000in}}{\pgfqpoint{5.700000in}{5.700000in}}%
\pgfusepath{clip}%
\pgfsetbuttcap%
\pgfsetroundjoin%
\definecolor{currentfill}{rgb}{0.282656,0.100196,0.422160}%
\pgfsetfillcolor{currentfill}%
\pgfsetfillopacity{0.700000}%
\pgfsetlinewidth{0.000000pt}%
\definecolor{currentstroke}{rgb}{0.000000,0.000000,0.000000}%
\pgfsetstrokecolor{currentstroke}%
\pgfsetdash{}{0pt}%
\pgfpathmoveto{\pgfqpoint{3.997961in}{1.897478in}}%
\pgfpathlineto{\pgfqpoint{4.011846in}{1.896925in}}%
\pgfpathlineto{\pgfqpoint{4.025739in}{1.896447in}}%
\pgfpathlineto{\pgfqpoint{4.039640in}{1.896045in}}%
\pgfpathlineto{\pgfqpoint{4.053549in}{1.895717in}}%
\pgfpathlineto{\pgfqpoint{4.061592in}{1.905281in}}%
\pgfpathlineto{\pgfqpoint{4.069629in}{1.914798in}}%
\pgfpathlineto{\pgfqpoint{4.077660in}{1.924267in}}%
\pgfpathlineto{\pgfqpoint{4.085686in}{1.933689in}}%
\pgfpathlineto{\pgfqpoint{4.071786in}{1.933956in}}%
\pgfpathlineto{\pgfqpoint{4.057895in}{1.934297in}}%
\pgfpathlineto{\pgfqpoint{4.044012in}{1.934714in}}%
\pgfpathlineto{\pgfqpoint{4.030137in}{1.935207in}}%
\pgfpathlineto{\pgfqpoint{4.022102in}{1.925838in}}%
\pgfpathlineto{\pgfqpoint{4.014060in}{1.916427in}}%
\pgfpathlineto{\pgfqpoint{4.006014in}{1.906973in}}%
\pgfpathlineto{\pgfqpoint{3.997961in}{1.897478in}}%
\pgfpathclose%
\pgfusepath{fill}%
\end{pgfscope}%
\begin{pgfscope}%
\pgfpathrectangle{\pgfqpoint{1.150000in}{0.150000in}}{\pgfqpoint{5.700000in}{5.700000in}}%
\pgfusepath{clip}%
\pgfsetbuttcap%
\pgfsetroundjoin%
\definecolor{currentfill}{rgb}{0.265145,0.232956,0.516599}%
\pgfsetfillcolor{currentfill}%
\pgfsetfillopacity{0.700000}%
\pgfsetlinewidth{0.000000pt}%
\definecolor{currentstroke}{rgb}{0.000000,0.000000,0.000000}%
\pgfsetstrokecolor{currentstroke}%
\pgfsetdash{}{0pt}%
\pgfpathmoveto{\pgfqpoint{4.667704in}{2.166141in}}%
\pgfpathlineto{\pgfqpoint{4.681815in}{2.167921in}}%
\pgfpathlineto{\pgfqpoint{4.695936in}{2.169773in}}%
\pgfpathlineto{\pgfqpoint{4.710068in}{2.171697in}}%
\pgfpathlineto{\pgfqpoint{4.724210in}{2.173692in}}%
\pgfpathlineto{\pgfqpoint{4.731998in}{2.181392in}}%
\pgfpathlineto{\pgfqpoint{4.739778in}{2.189021in}}%
\pgfpathlineto{\pgfqpoint{4.747552in}{2.196579in}}%
\pgfpathlineto{\pgfqpoint{4.755319in}{2.204069in}}%
\pgfpathlineto{\pgfqpoint{4.741190in}{2.202160in}}%
\pgfpathlineto{\pgfqpoint{4.727070in}{2.200323in}}%
\pgfpathlineto{\pgfqpoint{4.712961in}{2.198557in}}%
\pgfpathlineto{\pgfqpoint{4.698863in}{2.196863in}}%
\pgfpathlineto{\pgfqpoint{4.691083in}{2.189279in}}%
\pgfpathlineto{\pgfqpoint{4.683297in}{2.181632in}}%
\pgfpathlineto{\pgfqpoint{4.675504in}{2.173919in}}%
\pgfpathlineto{\pgfqpoint{4.667704in}{2.166141in}}%
\pgfpathclose%
\pgfusepath{fill}%
\end{pgfscope}%
\begin{pgfscope}%
\pgfpathrectangle{\pgfqpoint{1.150000in}{0.150000in}}{\pgfqpoint{5.700000in}{5.700000in}}%
\pgfusepath{clip}%
\pgfsetbuttcap%
\pgfsetroundjoin%
\definecolor{currentfill}{rgb}{0.280267,0.073417,0.397163}%
\pgfsetfillcolor{currentfill}%
\pgfsetfillopacity{0.700000}%
\pgfsetlinewidth{0.000000pt}%
\definecolor{currentstroke}{rgb}{0.000000,0.000000,0.000000}%
\pgfsetstrokecolor{currentstroke}%
\pgfsetdash{}{0pt}%
\pgfpathmoveto{\pgfqpoint{2.642772in}{1.862118in}}%
\pgfpathlineto{\pgfqpoint{2.656486in}{1.853334in}}%
\pgfpathlineto{\pgfqpoint{2.670201in}{1.844654in}}%
\pgfpathlineto{\pgfqpoint{2.683917in}{1.836075in}}%
\pgfpathlineto{\pgfqpoint{2.697635in}{1.827598in}}%
\pgfpathlineto{\pgfqpoint{2.706286in}{1.832095in}}%
\pgfpathlineto{\pgfqpoint{2.714925in}{1.836745in}}%
\pgfpathlineto{\pgfqpoint{2.723553in}{1.841545in}}%
\pgfpathlineto{\pgfqpoint{2.732169in}{1.846491in}}%
\pgfpathlineto{\pgfqpoint{2.718477in}{1.854678in}}%
\pgfpathlineto{\pgfqpoint{2.704788in}{1.862967in}}%
\pgfpathlineto{\pgfqpoint{2.691099in}{1.871358in}}%
\pgfpathlineto{\pgfqpoint{2.677412in}{1.879851in}}%
\pgfpathlineto{\pgfqpoint{2.668770in}{1.875187in}}%
\pgfpathlineto{\pgfqpoint{2.660116in}{1.870675in}}%
\pgfpathlineto{\pgfqpoint{2.651450in}{1.866317in}}%
\pgfpathlineto{\pgfqpoint{2.642772in}{1.862118in}}%
\pgfpathclose%
\pgfusepath{fill}%
\end{pgfscope}%
\begin{pgfscope}%
\pgfpathrectangle{\pgfqpoint{1.150000in}{0.150000in}}{\pgfqpoint{5.700000in}{5.700000in}}%
\pgfusepath{clip}%
\pgfsetbuttcap%
\pgfsetroundjoin%
\definecolor{currentfill}{rgb}{0.269944,0.014625,0.341379}%
\pgfsetfillcolor{currentfill}%
\pgfsetfillopacity{0.700000}%
\pgfsetlinewidth{0.000000pt}%
\definecolor{currentstroke}{rgb}{0.000000,0.000000,0.000000}%
\pgfsetstrokecolor{currentstroke}%
\pgfsetdash{}{0pt}%
\pgfpathmoveto{\pgfqpoint{2.985478in}{1.754202in}}%
\pgfpathlineto{\pgfqpoint{2.999186in}{1.747911in}}%
\pgfpathlineto{\pgfqpoint{3.012897in}{1.741712in}}%
\pgfpathlineto{\pgfqpoint{3.026611in}{1.735604in}}%
\pgfpathlineto{\pgfqpoint{3.040329in}{1.729586in}}%
\pgfpathlineto{\pgfqpoint{3.048780in}{1.736470in}}%
\pgfpathlineto{\pgfqpoint{3.057223in}{1.743450in}}%
\pgfpathlineto{\pgfqpoint{3.065657in}{1.750522in}}%
\pgfpathlineto{\pgfqpoint{3.074082in}{1.757682in}}%
\pgfpathlineto{\pgfqpoint{3.060384in}{1.763454in}}%
\pgfpathlineto{\pgfqpoint{3.046690in}{1.769317in}}%
\pgfpathlineto{\pgfqpoint{3.032999in}{1.775269in}}%
\pgfpathlineto{\pgfqpoint{3.019312in}{1.781313in}}%
\pgfpathlineto{\pgfqpoint{3.010867in}{1.774391in}}%
\pgfpathlineto{\pgfqpoint{3.002413in}{1.767563in}}%
\pgfpathlineto{\pgfqpoint{2.993950in}{1.760832in}}%
\pgfpathlineto{\pgfqpoint{2.985478in}{1.754202in}}%
\pgfpathclose%
\pgfusepath{fill}%
\end{pgfscope}%
\begin{pgfscope}%
\pgfpathrectangle{\pgfqpoint{1.150000in}{0.150000in}}{\pgfqpoint{5.700000in}{5.700000in}}%
\pgfusepath{clip}%
\pgfsetbuttcap%
\pgfsetroundjoin%
\definecolor{currentfill}{rgb}{0.268510,0.009605,0.335427}%
\pgfsetfillcolor{currentfill}%
\pgfsetfillopacity{0.700000}%
\pgfsetlinewidth{0.000000pt}%
\definecolor{currentstroke}{rgb}{0.000000,0.000000,0.000000}%
\pgfsetstrokecolor{currentstroke}%
\pgfsetdash{}{0pt}%
\pgfpathmoveto{\pgfqpoint{3.128911in}{1.735485in}}%
\pgfpathlineto{\pgfqpoint{3.142628in}{1.730156in}}%
\pgfpathlineto{\pgfqpoint{3.156349in}{1.724915in}}%
\pgfpathlineto{\pgfqpoint{3.170074in}{1.719761in}}%
\pgfpathlineto{\pgfqpoint{3.183804in}{1.714694in}}%
\pgfpathlineto{\pgfqpoint{3.192184in}{1.722405in}}%
\pgfpathlineto{\pgfqpoint{3.200556in}{1.730188in}}%
\pgfpathlineto{\pgfqpoint{3.208921in}{1.738039in}}%
\pgfpathlineto{\pgfqpoint{3.217278in}{1.745956in}}%
\pgfpathlineto{\pgfqpoint{3.203566in}{1.750798in}}%
\pgfpathlineto{\pgfqpoint{3.189858in}{1.755727in}}%
\pgfpathlineto{\pgfqpoint{3.176155in}{1.760743in}}%
\pgfpathlineto{\pgfqpoint{3.162456in}{1.765846in}}%
\pgfpathlineto{\pgfqpoint{3.154082in}{1.758147in}}%
\pgfpathlineto{\pgfqpoint{3.145699in}{1.750519in}}%
\pgfpathlineto{\pgfqpoint{3.137309in}{1.742964in}}%
\pgfpathlineto{\pgfqpoint{3.128911in}{1.735485in}}%
\pgfpathclose%
\pgfusepath{fill}%
\end{pgfscope}%
\begin{pgfscope}%
\pgfpathrectangle{\pgfqpoint{1.150000in}{0.150000in}}{\pgfqpoint{5.700000in}{5.700000in}}%
\pgfusepath{clip}%
\pgfsetbuttcap%
\pgfsetroundjoin%
\definecolor{currentfill}{rgb}{0.250425,0.274290,0.533103}%
\pgfsetfillcolor{currentfill}%
\pgfsetfillopacity{0.700000}%
\pgfsetlinewidth{0.000000pt}%
\definecolor{currentstroke}{rgb}{0.000000,0.000000,0.000000}%
\pgfsetstrokecolor{currentstroke}%
\pgfsetdash{}{0pt}%
\pgfpathmoveto{\pgfqpoint{2.021074in}{2.305939in}}%
\pgfpathlineto{\pgfqpoint{2.034923in}{2.291671in}}%
\pgfpathlineto{\pgfqpoint{2.048767in}{2.277545in}}%
\pgfpathlineto{\pgfqpoint{2.062607in}{2.263560in}}%
\pgfpathlineto{\pgfqpoint{2.076443in}{2.249714in}}%
\pgfpathlineto{\pgfqpoint{2.085549in}{2.249354in}}%
\pgfpathlineto{\pgfqpoint{2.094636in}{2.249242in}}%
\pgfpathlineto{\pgfqpoint{2.103705in}{2.249371in}}%
\pgfpathlineto{\pgfqpoint{2.112755in}{2.249738in}}%
\pgfpathlineto{\pgfqpoint{2.098959in}{2.263239in}}%
\pgfpathlineto{\pgfqpoint{2.085159in}{2.276879in}}%
\pgfpathlineto{\pgfqpoint{2.071355in}{2.290660in}}%
\pgfpathlineto{\pgfqpoint{2.057546in}{2.304582in}}%
\pgfpathlineto{\pgfqpoint{2.048457in}{2.304552in}}%
\pgfpathlineto{\pgfqpoint{2.039349in}{2.304765in}}%
\pgfpathlineto{\pgfqpoint{2.030221in}{2.305225in}}%
\pgfpathlineto{\pgfqpoint{2.021074in}{2.305939in}}%
\pgfpathclose%
\pgfusepath{fill}%
\end{pgfscope}%
\begin{pgfscope}%
\pgfpathrectangle{\pgfqpoint{1.150000in}{0.150000in}}{\pgfqpoint{5.700000in}{5.700000in}}%
\pgfusepath{clip}%
\pgfsetbuttcap%
\pgfsetroundjoin%
\definecolor{currentfill}{rgb}{0.225863,0.330805,0.547314}%
\pgfsetfillcolor{currentfill}%
\pgfsetfillopacity{0.700000}%
\pgfsetlinewidth{0.000000pt}%
\definecolor{currentstroke}{rgb}{0.000000,0.000000,0.000000}%
\pgfsetstrokecolor{currentstroke}%
\pgfsetdash{}{0pt}%
\pgfpathmoveto{\pgfqpoint{5.249905in}{2.392288in}}%
\pgfpathlineto{\pgfqpoint{5.264240in}{2.395203in}}%
\pgfpathlineto{\pgfqpoint{5.278586in}{2.398188in}}%
\pgfpathlineto{\pgfqpoint{5.292945in}{2.401243in}}%
\pgfpathlineto{\pgfqpoint{5.307315in}{2.404367in}}%
\pgfpathlineto{\pgfqpoint{5.314825in}{2.409583in}}%
\pgfpathlineto{\pgfqpoint{5.322327in}{2.414759in}}%
\pgfpathlineto{\pgfqpoint{5.329821in}{2.419898in}}%
\pgfpathlineto{\pgfqpoint{5.337309in}{2.425005in}}%
\pgfpathlineto{\pgfqpoint{5.322957in}{2.422096in}}%
\pgfpathlineto{\pgfqpoint{5.308618in}{2.419257in}}%
\pgfpathlineto{\pgfqpoint{5.294290in}{2.416487in}}%
\pgfpathlineto{\pgfqpoint{5.279973in}{2.413787in}}%
\pgfpathlineto{\pgfqpoint{5.272467in}{2.408457in}}%
\pgfpathlineto{\pgfqpoint{5.264953in}{2.403100in}}%
\pgfpathlineto{\pgfqpoint{5.257432in}{2.397712in}}%
\pgfpathlineto{\pgfqpoint{5.249905in}{2.392288in}}%
\pgfpathclose%
\pgfusepath{fill}%
\end{pgfscope}%
\begin{pgfscope}%
\pgfpathrectangle{\pgfqpoint{1.150000in}{0.150000in}}{\pgfqpoint{5.700000in}{5.700000in}}%
\pgfusepath{clip}%
\pgfsetbuttcap%
\pgfsetroundjoin%
\definecolor{currentfill}{rgb}{0.281446,0.084320,0.407414}%
\pgfsetfillcolor{currentfill}%
\pgfsetfillopacity{0.700000}%
\pgfsetlinewidth{0.000000pt}%
\definecolor{currentstroke}{rgb}{0.000000,0.000000,0.000000}%
\pgfsetstrokecolor{currentstroke}%
\pgfsetdash{}{0pt}%
\pgfpathmoveto{\pgfqpoint{3.910197in}{1.862389in}}%
\pgfpathlineto{\pgfqpoint{3.924060in}{1.861451in}}%
\pgfpathlineto{\pgfqpoint{3.937930in}{1.860589in}}%
\pgfpathlineto{\pgfqpoint{3.951809in}{1.859803in}}%
\pgfpathlineto{\pgfqpoint{3.965696in}{1.859093in}}%
\pgfpathlineto{\pgfqpoint{3.973770in}{1.868748in}}%
\pgfpathlineto{\pgfqpoint{3.981839in}{1.878365in}}%
\pgfpathlineto{\pgfqpoint{3.989903in}{1.887941in}}%
\pgfpathlineto{\pgfqpoint{3.997961in}{1.897478in}}%
\pgfpathlineto{\pgfqpoint{3.984085in}{1.898107in}}%
\pgfpathlineto{\pgfqpoint{3.970216in}{1.898811in}}%
\pgfpathlineto{\pgfqpoint{3.956355in}{1.899592in}}%
\pgfpathlineto{\pgfqpoint{3.942503in}{1.900449in}}%
\pgfpathlineto{\pgfqpoint{3.934435in}{1.890986in}}%
\pgfpathlineto{\pgfqpoint{3.926361in}{1.881488in}}%
\pgfpathlineto{\pgfqpoint{3.918282in}{1.871955in}}%
\pgfpathlineto{\pgfqpoint{3.910197in}{1.862389in}}%
\pgfpathclose%
\pgfusepath{fill}%
\end{pgfscope}%
\begin{pgfscope}%
\pgfpathrectangle{\pgfqpoint{1.150000in}{0.150000in}}{\pgfqpoint{5.700000in}{5.700000in}}%
\pgfusepath{clip}%
\pgfsetbuttcap%
\pgfsetroundjoin%
\definecolor{currentfill}{rgb}{0.270595,0.214069,0.507052}%
\pgfsetfillcolor{currentfill}%
\pgfsetfillopacity{0.700000}%
\pgfsetlinewidth{0.000000pt}%
\definecolor{currentstroke}{rgb}{0.000000,0.000000,0.000000}%
\pgfsetstrokecolor{currentstroke}%
\pgfsetdash{}{0pt}%
\pgfpathmoveto{\pgfqpoint{4.580053in}{2.127660in}}%
\pgfpathlineto{\pgfqpoint{4.594135in}{2.129219in}}%
\pgfpathlineto{\pgfqpoint{4.608226in}{2.130849in}}%
\pgfpathlineto{\pgfqpoint{4.622328in}{2.132552in}}%
\pgfpathlineto{\pgfqpoint{4.636439in}{2.134326in}}%
\pgfpathlineto{\pgfqpoint{4.644266in}{2.142388in}}%
\pgfpathlineto{\pgfqpoint{4.652085in}{2.150377in}}%
\pgfpathlineto{\pgfqpoint{4.659898in}{2.158294in}}%
\pgfpathlineto{\pgfqpoint{4.667704in}{2.166141in}}%
\pgfpathlineto{\pgfqpoint{4.653604in}{2.164432in}}%
\pgfpathlineto{\pgfqpoint{4.639514in}{2.162795in}}%
\pgfpathlineto{\pgfqpoint{4.625434in}{2.161229in}}%
\pgfpathlineto{\pgfqpoint{4.611364in}{2.159736in}}%
\pgfpathlineto{\pgfqpoint{4.603546in}{2.151816in}}%
\pgfpathlineto{\pgfqpoint{4.595722in}{2.143831in}}%
\pgfpathlineto{\pgfqpoint{4.587891in}{2.135779in}}%
\pgfpathlineto{\pgfqpoint{4.580053in}{2.127660in}}%
\pgfpathclose%
\pgfusepath{fill}%
\end{pgfscope}%
\begin{pgfscope}%
\pgfpathrectangle{\pgfqpoint{1.150000in}{0.150000in}}{\pgfqpoint{5.700000in}{5.700000in}}%
\pgfusepath{clip}%
\pgfsetbuttcap%
\pgfsetroundjoin%
\definecolor{currentfill}{rgb}{0.273809,0.031497,0.358853}%
\pgfsetfillcolor{currentfill}%
\pgfsetfillopacity{0.700000}%
\pgfsetlinewidth{0.000000pt}%
\definecolor{currentstroke}{rgb}{0.000000,0.000000,0.000000}%
\pgfsetstrokecolor{currentstroke}%
\pgfsetdash{}{0pt}%
\pgfpathmoveto{\pgfqpoint{2.841760in}{1.784561in}}%
\pgfpathlineto{\pgfqpoint{2.855469in}{1.777256in}}%
\pgfpathlineto{\pgfqpoint{2.869179in}{1.770047in}}%
\pgfpathlineto{\pgfqpoint{2.882892in}{1.762933in}}%
\pgfpathlineto{\pgfqpoint{2.896607in}{1.755913in}}%
\pgfpathlineto{\pgfqpoint{2.905140in}{1.761827in}}%
\pgfpathlineto{\pgfqpoint{2.913662in}{1.767863in}}%
\pgfpathlineto{\pgfqpoint{2.922175in}{1.774016in}}%
\pgfpathlineto{\pgfqpoint{2.930677in}{1.780282in}}%
\pgfpathlineto{\pgfqpoint{2.916985in}{1.787035in}}%
\pgfpathlineto{\pgfqpoint{2.903294in}{1.793881in}}%
\pgfpathlineto{\pgfqpoint{2.889607in}{1.800823in}}%
\pgfpathlineto{\pgfqpoint{2.875922in}{1.807859in}}%
\pgfpathlineto{\pgfqpoint{2.867397in}{1.801852in}}%
\pgfpathlineto{\pgfqpoint{2.858861in}{1.795965in}}%
\pgfpathlineto{\pgfqpoint{2.850316in}{1.790199in}}%
\pgfpathlineto{\pgfqpoint{2.841760in}{1.784561in}}%
\pgfpathclose%
\pgfusepath{fill}%
\end{pgfscope}%
\begin{pgfscope}%
\pgfpathrectangle{\pgfqpoint{1.150000in}{0.150000in}}{\pgfqpoint{5.700000in}{5.700000in}}%
\pgfusepath{clip}%
\pgfsetbuttcap%
\pgfsetroundjoin%
\definecolor{currentfill}{rgb}{0.271305,0.019942,0.347269}%
\pgfsetfillcolor{currentfill}%
\pgfsetfillopacity{0.700000}%
\pgfsetlinewidth{0.000000pt}%
\definecolor{currentstroke}{rgb}{0.000000,0.000000,0.000000}%
\pgfsetstrokecolor{currentstroke}%
\pgfsetdash{}{0pt}%
\pgfpathmoveto{\pgfqpoint{3.503439in}{1.751497in}}%
\pgfpathlineto{\pgfqpoint{3.517210in}{1.748462in}}%
\pgfpathlineto{\pgfqpoint{3.530986in}{1.745507in}}%
\pgfpathlineto{\pgfqpoint{3.544769in}{1.742632in}}%
\pgfpathlineto{\pgfqpoint{3.558558in}{1.739837in}}%
\pgfpathlineto{\pgfqpoint{3.566780in}{1.749067in}}%
\pgfpathlineto{\pgfqpoint{3.574996in}{1.758308in}}%
\pgfpathlineto{\pgfqpoint{3.583206in}{1.767561in}}%
\pgfpathlineto{\pgfqpoint{3.591410in}{1.776821in}}%
\pgfpathlineto{\pgfqpoint{3.577634in}{1.779452in}}%
\pgfpathlineto{\pgfqpoint{3.563864in}{1.782163in}}%
\pgfpathlineto{\pgfqpoint{3.550100in}{1.784954in}}%
\pgfpathlineto{\pgfqpoint{3.536343in}{1.787826in}}%
\pgfpathlineto{\pgfqpoint{3.528126in}{1.778722in}}%
\pgfpathlineto{\pgfqpoint{3.519903in}{1.769631in}}%
\pgfpathlineto{\pgfqpoint{3.511674in}{1.760555in}}%
\pgfpathlineto{\pgfqpoint{3.503439in}{1.751497in}}%
\pgfpathclose%
\pgfusepath{fill}%
\end{pgfscope}%
\begin{pgfscope}%
\pgfpathrectangle{\pgfqpoint{1.150000in}{0.150000in}}{\pgfqpoint{5.700000in}{5.700000in}}%
\pgfusepath{clip}%
\pgfsetbuttcap%
\pgfsetroundjoin%
\definecolor{currentfill}{rgb}{0.258965,0.251537,0.524736}%
\pgfsetfillcolor{currentfill}%
\pgfsetfillopacity{0.700000}%
\pgfsetlinewidth{0.000000pt}%
\definecolor{currentstroke}{rgb}{0.000000,0.000000,0.000000}%
\pgfsetstrokecolor{currentstroke}%
\pgfsetdash{}{0pt}%
\pgfpathmoveto{\pgfqpoint{2.076443in}{2.249714in}}%
\pgfpathlineto{\pgfqpoint{2.090274in}{2.236007in}}%
\pgfpathlineto{\pgfqpoint{2.104102in}{2.222437in}}%
\pgfpathlineto{\pgfqpoint{2.117926in}{2.209003in}}%
\pgfpathlineto{\pgfqpoint{2.131746in}{2.195703in}}%
\pgfpathlineto{\pgfqpoint{2.140812in}{2.195695in}}%
\pgfpathlineto{\pgfqpoint{2.149860in}{2.195929in}}%
\pgfpathlineto{\pgfqpoint{2.158890in}{2.196399in}}%
\pgfpathlineto{\pgfqpoint{2.167902in}{2.197100in}}%
\pgfpathlineto{\pgfqpoint{2.154121in}{2.210057in}}%
\pgfpathlineto{\pgfqpoint{2.140336in}{2.223148in}}%
\pgfpathlineto{\pgfqpoint{2.126547in}{2.236375in}}%
\pgfpathlineto{\pgfqpoint{2.112755in}{2.249738in}}%
\pgfpathlineto{\pgfqpoint{2.103705in}{2.249371in}}%
\pgfpathlineto{\pgfqpoint{2.094636in}{2.249242in}}%
\pgfpathlineto{\pgfqpoint{2.085549in}{2.249354in}}%
\pgfpathlineto{\pgfqpoint{2.076443in}{2.249714in}}%
\pgfpathclose%
\pgfusepath{fill}%
\end{pgfscope}%
\begin{pgfscope}%
\pgfpathrectangle{\pgfqpoint{1.150000in}{0.150000in}}{\pgfqpoint{5.700000in}{5.700000in}}%
\pgfusepath{clip}%
\pgfsetbuttcap%
\pgfsetroundjoin%
\definecolor{currentfill}{rgb}{0.283229,0.120777,0.440584}%
\pgfsetfillcolor{currentfill}%
\pgfsetfillopacity{0.700000}%
\pgfsetlinewidth{0.000000pt}%
\definecolor{currentstroke}{rgb}{0.000000,0.000000,0.000000}%
\pgfsetstrokecolor{currentstroke}%
\pgfsetdash{}{0pt}%
\pgfpathmoveto{\pgfqpoint{2.443038in}{1.964435in}}%
\pgfpathlineto{\pgfqpoint{2.456780in}{1.954048in}}%
\pgfpathlineto{\pgfqpoint{2.470521in}{1.943772in}}%
\pgfpathlineto{\pgfqpoint{2.484262in}{1.933609in}}%
\pgfpathlineto{\pgfqpoint{2.498003in}{1.923555in}}%
\pgfpathlineto{\pgfqpoint{2.506793in}{1.926430in}}%
\pgfpathlineto{\pgfqpoint{2.515570in}{1.929494in}}%
\pgfpathlineto{\pgfqpoint{2.524332in}{1.932742in}}%
\pgfpathlineto{\pgfqpoint{2.533081in}{1.936168in}}%
\pgfpathlineto{\pgfqpoint{2.519370in}{1.945908in}}%
\pgfpathlineto{\pgfqpoint{2.505660in}{1.955758in}}%
\pgfpathlineto{\pgfqpoint{2.491950in}{1.965720in}}%
\pgfpathlineto{\pgfqpoint{2.478239in}{1.975792in}}%
\pgfpathlineto{\pgfqpoint{2.469460in}{1.972672in}}%
\pgfpathlineto{\pgfqpoint{2.460667in}{1.969736in}}%
\pgfpathlineto{\pgfqpoint{2.451860in}{1.966988in}}%
\pgfpathlineto{\pgfqpoint{2.443038in}{1.964435in}}%
\pgfpathclose%
\pgfusepath{fill}%
\end{pgfscope}%
\begin{pgfscope}%
\pgfpathrectangle{\pgfqpoint{1.150000in}{0.150000in}}{\pgfqpoint{5.700000in}{5.700000in}}%
\pgfusepath{clip}%
\pgfsetbuttcap%
\pgfsetroundjoin%
\definecolor{currentfill}{rgb}{0.268510,0.009605,0.335427}%
\pgfsetfillcolor{currentfill}%
\pgfsetfillopacity{0.700000}%
\pgfsetlinewidth{0.000000pt}%
\definecolor{currentstroke}{rgb}{0.000000,0.000000,0.000000}%
\pgfsetstrokecolor{currentstroke}%
\pgfsetdash{}{0pt}%
\pgfpathmoveto{\pgfqpoint{3.272171in}{1.727445in}}%
\pgfpathlineto{\pgfqpoint{3.285906in}{1.723030in}}%
\pgfpathlineto{\pgfqpoint{3.299646in}{1.718700in}}%
\pgfpathlineto{\pgfqpoint{3.313391in}{1.714453in}}%
\pgfpathlineto{\pgfqpoint{3.327141in}{1.710290in}}%
\pgfpathlineto{\pgfqpoint{3.335457in}{1.718693in}}%
\pgfpathlineto{\pgfqpoint{3.343767in}{1.727144in}}%
\pgfpathlineto{\pgfqpoint{3.352070in}{1.735641in}}%
\pgfpathlineto{\pgfqpoint{3.360366in}{1.744182in}}%
\pgfpathlineto{\pgfqpoint{3.346632in}{1.748140in}}%
\pgfpathlineto{\pgfqpoint{3.332903in}{1.752182in}}%
\pgfpathlineto{\pgfqpoint{3.319178in}{1.756308in}}%
\pgfpathlineto{\pgfqpoint{3.305459in}{1.760518in}}%
\pgfpathlineto{\pgfqpoint{3.297148in}{1.752175in}}%
\pgfpathlineto{\pgfqpoint{3.288829in}{1.743880in}}%
\pgfpathlineto{\pgfqpoint{3.280504in}{1.735635in}}%
\pgfpathlineto{\pgfqpoint{3.272171in}{1.727445in}}%
\pgfpathclose%
\pgfusepath{fill}%
\end{pgfscope}%
\begin{pgfscope}%
\pgfpathrectangle{\pgfqpoint{1.150000in}{0.150000in}}{\pgfqpoint{5.700000in}{5.700000in}}%
\pgfusepath{clip}%
\pgfsetbuttcap%
\pgfsetroundjoin%
\definecolor{currentfill}{rgb}{0.201239,0.383670,0.554294}%
\pgfsetfillcolor{currentfill}%
\pgfsetfillopacity{0.700000}%
\pgfsetlinewidth{0.000000pt}%
\definecolor{currentstroke}{rgb}{0.000000,0.000000,0.000000}%
\pgfsetstrokecolor{currentstroke}%
\pgfsetdash{}{0pt}%
\pgfpathmoveto{\pgfqpoint{5.656900in}{2.528873in}}%
\pgfpathlineto{\pgfqpoint{5.671395in}{2.532178in}}%
\pgfpathlineto{\pgfqpoint{5.685903in}{2.535552in}}%
\pgfpathlineto{\pgfqpoint{5.700424in}{2.538994in}}%
\pgfpathlineto{\pgfqpoint{5.714957in}{2.542505in}}%
\pgfpathlineto{\pgfqpoint{5.722246in}{2.546131in}}%
\pgfpathlineto{\pgfqpoint{5.729528in}{2.549763in}}%
\pgfpathlineto{\pgfqpoint{5.736804in}{2.553408in}}%
\pgfpathlineto{\pgfqpoint{5.744073in}{2.557070in}}%
\pgfpathlineto{\pgfqpoint{5.729565in}{2.553861in}}%
\pgfpathlineto{\pgfqpoint{5.715070in}{2.550720in}}%
\pgfpathlineto{\pgfqpoint{5.700587in}{2.547647in}}%
\pgfpathlineto{\pgfqpoint{5.686116in}{2.544643in}}%
\pgfpathlineto{\pgfqpoint{5.678821in}{2.540672in}}%
\pgfpathlineto{\pgfqpoint{5.671520in}{2.536724in}}%
\pgfpathlineto{\pgfqpoint{5.664213in}{2.532793in}}%
\pgfpathlineto{\pgfqpoint{5.656900in}{2.528873in}}%
\pgfpathclose%
\pgfusepath{fill}%
\end{pgfscope}%
\begin{pgfscope}%
\pgfpathrectangle{\pgfqpoint{1.150000in}{0.150000in}}{\pgfqpoint{5.700000in}{5.700000in}}%
\pgfusepath{clip}%
\pgfsetbuttcap%
\pgfsetroundjoin%
\definecolor{currentfill}{rgb}{0.274128,0.199721,0.498911}%
\pgfsetfillcolor{currentfill}%
\pgfsetfillopacity{0.700000}%
\pgfsetlinewidth{0.000000pt}%
\definecolor{currentstroke}{rgb}{0.000000,0.000000,0.000000}%
\pgfsetstrokecolor{currentstroke}%
\pgfsetdash{}{0pt}%
\pgfpathmoveto{\pgfqpoint{4.492371in}{2.088789in}}%
\pgfpathlineto{\pgfqpoint{4.506423in}{2.090102in}}%
\pgfpathlineto{\pgfqpoint{4.520485in}{2.091489in}}%
\pgfpathlineto{\pgfqpoint{4.534557in}{2.092947in}}%
\pgfpathlineto{\pgfqpoint{4.548639in}{2.094478in}}%
\pgfpathlineto{\pgfqpoint{4.556502in}{2.102882in}}%
\pgfpathlineto{\pgfqpoint{4.564359in}{2.111213in}}%
\pgfpathlineto{\pgfqpoint{4.572210in}{2.119472in}}%
\pgfpathlineto{\pgfqpoint{4.580053in}{2.127660in}}%
\pgfpathlineto{\pgfqpoint{4.565982in}{2.126174in}}%
\pgfpathlineto{\pgfqpoint{4.551922in}{2.124759in}}%
\pgfpathlineto{\pgfqpoint{4.537871in}{2.123417in}}%
\pgfpathlineto{\pgfqpoint{4.523830in}{2.122147in}}%
\pgfpathlineto{\pgfqpoint{4.515974in}{2.113907in}}%
\pgfpathlineto{\pgfqpoint{4.508113in}{2.105601in}}%
\pgfpathlineto{\pgfqpoint{4.500245in}{2.097229in}}%
\pgfpathlineto{\pgfqpoint{4.492371in}{2.088789in}}%
\pgfpathclose%
\pgfusepath{fill}%
\end{pgfscope}%
\begin{pgfscope}%
\pgfpathrectangle{\pgfqpoint{1.150000in}{0.150000in}}{\pgfqpoint{5.700000in}{5.700000in}}%
\pgfusepath{clip}%
\pgfsetbuttcap%
\pgfsetroundjoin%
\definecolor{currentfill}{rgb}{0.279566,0.067836,0.391917}%
\pgfsetfillcolor{currentfill}%
\pgfsetfillopacity{0.700000}%
\pgfsetlinewidth{0.000000pt}%
\definecolor{currentstroke}{rgb}{0.000000,0.000000,0.000000}%
\pgfsetstrokecolor{currentstroke}%
\pgfsetdash{}{0pt}%
\pgfpathmoveto{\pgfqpoint{3.822385in}{1.828735in}}%
\pgfpathlineto{\pgfqpoint{3.836228in}{1.827389in}}%
\pgfpathlineto{\pgfqpoint{3.850078in}{1.826119in}}%
\pgfpathlineto{\pgfqpoint{3.863936in}{1.824926in}}%
\pgfpathlineto{\pgfqpoint{3.877801in}{1.823810in}}%
\pgfpathlineto{\pgfqpoint{3.885909in}{1.833500in}}%
\pgfpathlineto{\pgfqpoint{3.894010in}{1.843160in}}%
\pgfpathlineto{\pgfqpoint{3.902106in}{1.852791in}}%
\pgfpathlineto{\pgfqpoint{3.910197in}{1.862389in}}%
\pgfpathlineto{\pgfqpoint{3.896342in}{1.863404in}}%
\pgfpathlineto{\pgfqpoint{3.882494in}{1.864495in}}%
\pgfpathlineto{\pgfqpoint{3.868654in}{1.865662in}}%
\pgfpathlineto{\pgfqpoint{3.854822in}{1.866907in}}%
\pgfpathlineto{\pgfqpoint{3.846721in}{1.857402in}}%
\pgfpathlineto{\pgfqpoint{3.838615in}{1.847871in}}%
\pgfpathlineto{\pgfqpoint{3.830503in}{1.838315in}}%
\pgfpathlineto{\pgfqpoint{3.822385in}{1.828735in}}%
\pgfpathclose%
\pgfusepath{fill}%
\end{pgfscope}%
\begin{pgfscope}%
\pgfpathrectangle{\pgfqpoint{1.150000in}{0.150000in}}{\pgfqpoint{5.700000in}{5.700000in}}%
\pgfusepath{clip}%
\pgfsetbuttcap%
\pgfsetroundjoin%
\definecolor{currentfill}{rgb}{0.231674,0.318106,0.544834}%
\pgfsetfillcolor{currentfill}%
\pgfsetfillopacity{0.700000}%
\pgfsetlinewidth{0.000000pt}%
\definecolor{currentstroke}{rgb}{0.000000,0.000000,0.000000}%
\pgfsetstrokecolor{currentstroke}%
\pgfsetdash{}{0pt}%
\pgfpathmoveto{\pgfqpoint{5.162430in}{2.358426in}}%
\pgfpathlineto{\pgfqpoint{5.176735in}{2.361256in}}%
\pgfpathlineto{\pgfqpoint{5.191052in}{2.364156in}}%
\pgfpathlineto{\pgfqpoint{5.205381in}{2.367126in}}%
\pgfpathlineto{\pgfqpoint{5.219721in}{2.370166in}}%
\pgfpathlineto{\pgfqpoint{5.227278in}{2.375768in}}%
\pgfpathlineto{\pgfqpoint{5.234828in}{2.381320in}}%
\pgfpathlineto{\pgfqpoint{5.242370in}{2.386825in}}%
\pgfpathlineto{\pgfqpoint{5.249905in}{2.392288in}}%
\pgfpathlineto{\pgfqpoint{5.235582in}{2.389443in}}%
\pgfpathlineto{\pgfqpoint{5.221270in}{2.386667in}}%
\pgfpathlineto{\pgfqpoint{5.206971in}{2.383960in}}%
\pgfpathlineto{\pgfqpoint{5.192683in}{2.381323in}}%
\pgfpathlineto{\pgfqpoint{5.185130in}{2.375659in}}%
\pgfpathlineto{\pgfqpoint{5.177570in}{2.369957in}}%
\pgfpathlineto{\pgfqpoint{5.170004in}{2.364214in}}%
\pgfpathlineto{\pgfqpoint{5.162430in}{2.358426in}}%
\pgfpathclose%
\pgfusepath{fill}%
\end{pgfscope}%
\begin{pgfscope}%
\pgfpathrectangle{\pgfqpoint{1.150000in}{0.150000in}}{\pgfqpoint{5.700000in}{5.700000in}}%
\pgfusepath{clip}%
\pgfsetbuttcap%
\pgfsetroundjoin%
\definecolor{currentfill}{rgb}{0.278012,0.180367,0.486697}%
\pgfsetfillcolor{currentfill}%
\pgfsetfillopacity{0.700000}%
\pgfsetlinewidth{0.000000pt}%
\definecolor{currentstroke}{rgb}{0.000000,0.000000,0.000000}%
\pgfsetstrokecolor{currentstroke}%
\pgfsetdash{}{0pt}%
\pgfpathmoveto{\pgfqpoint{4.404660in}{2.049709in}}%
\pgfpathlineto{\pgfqpoint{4.418684in}{2.050755in}}%
\pgfpathlineto{\pgfqpoint{4.432717in}{2.051874in}}%
\pgfpathlineto{\pgfqpoint{4.446759in}{2.053066in}}%
\pgfpathlineto{\pgfqpoint{4.460812in}{2.054330in}}%
\pgfpathlineto{\pgfqpoint{4.468711in}{2.063051in}}%
\pgfpathlineto{\pgfqpoint{4.476604in}{2.071701in}}%
\pgfpathlineto{\pgfqpoint{4.484491in}{2.080280in}}%
\pgfpathlineto{\pgfqpoint{4.492371in}{2.088789in}}%
\pgfpathlineto{\pgfqpoint{4.478329in}{2.087547in}}%
\pgfpathlineto{\pgfqpoint{4.464297in}{2.086378in}}%
\pgfpathlineto{\pgfqpoint{4.450274in}{2.085282in}}%
\pgfpathlineto{\pgfqpoint{4.436261in}{2.084259in}}%
\pgfpathlineto{\pgfqpoint{4.428370in}{2.075719in}}%
\pgfpathlineto{\pgfqpoint{4.420473in}{2.067115in}}%
\pgfpathlineto{\pgfqpoint{4.412570in}{2.058445in}}%
\pgfpathlineto{\pgfqpoint{4.404660in}{2.049709in}}%
\pgfpathclose%
\pgfusepath{fill}%
\end{pgfscope}%
\begin{pgfscope}%
\pgfpathrectangle{\pgfqpoint{1.150000in}{0.150000in}}{\pgfqpoint{5.700000in}{5.700000in}}%
\pgfusepath{clip}%
\pgfsetbuttcap%
\pgfsetroundjoin%
\definecolor{currentfill}{rgb}{0.266580,0.228262,0.514349}%
\pgfsetfillcolor{currentfill}%
\pgfsetfillopacity{0.700000}%
\pgfsetlinewidth{0.000000pt}%
\definecolor{currentstroke}{rgb}{0.000000,0.000000,0.000000}%
\pgfsetstrokecolor{currentstroke}%
\pgfsetdash{}{0pt}%
\pgfpathmoveto{\pgfqpoint{2.131746in}{2.195703in}}%
\pgfpathlineto{\pgfqpoint{2.145562in}{2.182537in}}%
\pgfpathlineto{\pgfqpoint{2.159376in}{2.169503in}}%
\pgfpathlineto{\pgfqpoint{2.173186in}{2.156600in}}%
\pgfpathlineto{\pgfqpoint{2.186993in}{2.143826in}}%
\pgfpathlineto{\pgfqpoint{2.196021in}{2.144169in}}%
\pgfpathlineto{\pgfqpoint{2.205031in}{2.144747in}}%
\pgfpathlineto{\pgfqpoint{2.214023in}{2.145556in}}%
\pgfpathlineto{\pgfqpoint{2.222998in}{2.146590in}}%
\pgfpathlineto{\pgfqpoint{2.209228in}{2.159022in}}%
\pgfpathlineto{\pgfqpoint{2.195456in}{2.171584in}}%
\pgfpathlineto{\pgfqpoint{2.181681in}{2.184276in}}%
\pgfpathlineto{\pgfqpoint{2.167902in}{2.197100in}}%
\pgfpathlineto{\pgfqpoint{2.158890in}{2.196399in}}%
\pgfpathlineto{\pgfqpoint{2.149860in}{2.195929in}}%
\pgfpathlineto{\pgfqpoint{2.140812in}{2.195695in}}%
\pgfpathlineto{\pgfqpoint{2.131746in}{2.195703in}}%
\pgfpathclose%
\pgfusepath{fill}%
\end{pgfscope}%
\begin{pgfscope}%
\pgfpathrectangle{\pgfqpoint{1.150000in}{0.150000in}}{\pgfqpoint{5.700000in}{5.700000in}}%
\pgfusepath{clip}%
\pgfsetbuttcap%
\pgfsetroundjoin%
\definecolor{currentfill}{rgb}{0.237441,0.305202,0.541921}%
\pgfsetfillcolor{currentfill}%
\pgfsetfillopacity{0.700000}%
\pgfsetlinewidth{0.000000pt}%
\definecolor{currentstroke}{rgb}{0.000000,0.000000,0.000000}%
\pgfsetstrokecolor{currentstroke}%
\pgfsetdash{}{0pt}%
\pgfpathmoveto{\pgfqpoint{5.074890in}{2.323446in}}%
\pgfpathlineto{\pgfqpoint{5.089165in}{2.326168in}}%
\pgfpathlineto{\pgfqpoint{5.103452in}{2.328961in}}%
\pgfpathlineto{\pgfqpoint{5.117751in}{2.331824in}}%
\pgfpathlineto{\pgfqpoint{5.132061in}{2.334757in}}%
\pgfpathlineto{\pgfqpoint{5.139664in}{2.340758in}}%
\pgfpathlineto{\pgfqpoint{5.147260in}{2.346701in}}%
\pgfpathlineto{\pgfqpoint{5.154848in}{2.352589in}}%
\pgfpathlineto{\pgfqpoint{5.162430in}{2.358426in}}%
\pgfpathlineto{\pgfqpoint{5.148136in}{2.355665in}}%
\pgfpathlineto{\pgfqpoint{5.133854in}{2.352975in}}%
\pgfpathlineto{\pgfqpoint{5.119583in}{2.350355in}}%
\pgfpathlineto{\pgfqpoint{5.105324in}{2.347804in}}%
\pgfpathlineto{\pgfqpoint{5.097726in}{2.341788in}}%
\pgfpathlineto{\pgfqpoint{5.090121in}{2.335725in}}%
\pgfpathlineto{\pgfqpoint{5.082509in}{2.329612in}}%
\pgfpathlineto{\pgfqpoint{5.074890in}{2.323446in}}%
\pgfpathclose%
\pgfusepath{fill}%
\end{pgfscope}%
\begin{pgfscope}%
\pgfpathrectangle{\pgfqpoint{1.150000in}{0.150000in}}{\pgfqpoint{5.700000in}{5.700000in}}%
\pgfusepath{clip}%
\pgfsetbuttcap%
\pgfsetroundjoin%
\definecolor{currentfill}{rgb}{0.278791,0.062145,0.386592}%
\pgfsetfillcolor{currentfill}%
\pgfsetfillopacity{0.700000}%
\pgfsetlinewidth{0.000000pt}%
\definecolor{currentstroke}{rgb}{0.000000,0.000000,0.000000}%
\pgfsetstrokecolor{currentstroke}%
\pgfsetdash{}{0pt}%
\pgfpathmoveto{\pgfqpoint{2.697635in}{1.827598in}}%
\pgfpathlineto{\pgfqpoint{2.711354in}{1.819221in}}%
\pgfpathlineto{\pgfqpoint{2.725074in}{1.810945in}}%
\pgfpathlineto{\pgfqpoint{2.738796in}{1.802768in}}%
\pgfpathlineto{\pgfqpoint{2.752519in}{1.794690in}}%
\pgfpathlineto{\pgfqpoint{2.761144in}{1.799484in}}%
\pgfpathlineto{\pgfqpoint{2.769757in}{1.804427in}}%
\pgfpathlineto{\pgfqpoint{2.778359in}{1.809514in}}%
\pgfpathlineto{\pgfqpoint{2.786949in}{1.814741in}}%
\pgfpathlineto{\pgfqpoint{2.773252in}{1.822530in}}%
\pgfpathlineto{\pgfqpoint{2.759556in}{1.830417in}}%
\pgfpathlineto{\pgfqpoint{2.745861in}{1.838404in}}%
\pgfpathlineto{\pgfqpoint{2.732169in}{1.846491in}}%
\pgfpathlineto{\pgfqpoint{2.723553in}{1.841545in}}%
\pgfpathlineto{\pgfqpoint{2.714925in}{1.836745in}}%
\pgfpathlineto{\pgfqpoint{2.706286in}{1.832095in}}%
\pgfpathlineto{\pgfqpoint{2.697635in}{1.827598in}}%
\pgfpathclose%
\pgfusepath{fill}%
\end{pgfscope}%
\begin{pgfscope}%
\pgfpathrectangle{\pgfqpoint{1.150000in}{0.150000in}}{\pgfqpoint{5.700000in}{5.700000in}}%
\pgfusepath{clip}%
\pgfsetbuttcap%
\pgfsetroundjoin%
\definecolor{currentfill}{rgb}{0.280255,0.165693,0.476498}%
\pgfsetfillcolor{currentfill}%
\pgfsetfillopacity{0.700000}%
\pgfsetlinewidth{0.000000pt}%
\definecolor{currentstroke}{rgb}{0.000000,0.000000,0.000000}%
\pgfsetstrokecolor{currentstroke}%
\pgfsetdash{}{0pt}%
\pgfpathmoveto{\pgfqpoint{4.316923in}{2.010625in}}%
\pgfpathlineto{\pgfqpoint{4.330918in}{2.011381in}}%
\pgfpathlineto{\pgfqpoint{4.344923in}{2.012210in}}%
\pgfpathlineto{\pgfqpoint{4.358937in}{2.013112in}}%
\pgfpathlineto{\pgfqpoint{4.372961in}{2.014088in}}%
\pgfpathlineto{\pgfqpoint{4.380895in}{2.023096in}}%
\pgfpathlineto{\pgfqpoint{4.388823in}{2.032035in}}%
\pgfpathlineto{\pgfqpoint{4.396745in}{2.040905in}}%
\pgfpathlineto{\pgfqpoint{4.404660in}{2.049709in}}%
\pgfpathlineto{\pgfqpoint{4.390647in}{2.048735in}}%
\pgfpathlineto{\pgfqpoint{4.376642in}{2.047835in}}%
\pgfpathlineto{\pgfqpoint{4.362648in}{2.047007in}}%
\pgfpathlineto{\pgfqpoint{4.348662in}{2.046253in}}%
\pgfpathlineto{\pgfqpoint{4.340737in}{2.037440in}}%
\pgfpathlineto{\pgfqpoint{4.332805in}{2.028565in}}%
\pgfpathlineto{\pgfqpoint{4.324867in}{2.019626in}}%
\pgfpathlineto{\pgfqpoint{4.316923in}{2.010625in}}%
\pgfpathclose%
\pgfusepath{fill}%
\end{pgfscope}%
\begin{pgfscope}%
\pgfpathrectangle{\pgfqpoint{1.150000in}{0.150000in}}{\pgfqpoint{5.700000in}{5.700000in}}%
\pgfusepath{clip}%
\pgfsetbuttcap%
\pgfsetroundjoin%
\definecolor{currentfill}{rgb}{0.277941,0.056324,0.381191}%
\pgfsetfillcolor{currentfill}%
\pgfsetfillopacity{0.700000}%
\pgfsetlinewidth{0.000000pt}%
\definecolor{currentstroke}{rgb}{0.000000,0.000000,0.000000}%
\pgfsetstrokecolor{currentstroke}%
\pgfsetdash{}{0pt}%
\pgfpathmoveto{\pgfqpoint{3.734516in}{1.796851in}}%
\pgfpathlineto{\pgfqpoint{3.748341in}{1.795072in}}%
\pgfpathlineto{\pgfqpoint{3.762173in}{1.793371in}}%
\pgfpathlineto{\pgfqpoint{3.776012in}{1.791748in}}%
\pgfpathlineto{\pgfqpoint{3.789858in}{1.790202in}}%
\pgfpathlineto{\pgfqpoint{3.797998in}{1.799864in}}%
\pgfpathlineto{\pgfqpoint{3.806133in}{1.809508in}}%
\pgfpathlineto{\pgfqpoint{3.814262in}{1.819133in}}%
\pgfpathlineto{\pgfqpoint{3.822385in}{1.828735in}}%
\pgfpathlineto{\pgfqpoint{3.808550in}{1.830159in}}%
\pgfpathlineto{\pgfqpoint{3.794722in}{1.831660in}}%
\pgfpathlineto{\pgfqpoint{3.780901in}{1.833238in}}%
\pgfpathlineto{\pgfqpoint{3.767087in}{1.834894in}}%
\pgfpathlineto{\pgfqpoint{3.758953in}{1.825406in}}%
\pgfpathlineto{\pgfqpoint{3.750813in}{1.815902in}}%
\pgfpathlineto{\pgfqpoint{3.742668in}{1.806383in}}%
\pgfpathlineto{\pgfqpoint{3.734516in}{1.796851in}}%
\pgfpathclose%
\pgfusepath{fill}%
\end{pgfscope}%
\begin{pgfscope}%
\pgfpathrectangle{\pgfqpoint{1.150000in}{0.150000in}}{\pgfqpoint{5.700000in}{5.700000in}}%
\pgfusepath{clip}%
\pgfsetbuttcap%
\pgfsetroundjoin%
\definecolor{currentfill}{rgb}{0.204903,0.375746,0.553533}%
\pgfsetfillcolor{currentfill}%
\pgfsetfillopacity{0.700000}%
\pgfsetlinewidth{0.000000pt}%
\definecolor{currentstroke}{rgb}{0.000000,0.000000,0.000000}%
\pgfsetstrokecolor{currentstroke}%
\pgfsetdash{}{0pt}%
\pgfpathmoveto{\pgfqpoint{5.569634in}{2.499556in}}%
\pgfpathlineto{\pgfqpoint{5.584102in}{2.502867in}}%
\pgfpathlineto{\pgfqpoint{5.598583in}{2.506246in}}%
\pgfpathlineto{\pgfqpoint{5.613077in}{2.509695in}}%
\pgfpathlineto{\pgfqpoint{5.627583in}{2.513212in}}%
\pgfpathlineto{\pgfqpoint{5.634922in}{2.517135in}}%
\pgfpathlineto{\pgfqpoint{5.642255in}{2.521049in}}%
\pgfpathlineto{\pgfqpoint{5.649581in}{2.524960in}}%
\pgfpathlineto{\pgfqpoint{5.656900in}{2.528873in}}%
\pgfpathlineto{\pgfqpoint{5.642418in}{2.525637in}}%
\pgfpathlineto{\pgfqpoint{5.627947in}{2.522469in}}%
\pgfpathlineto{\pgfqpoint{5.613490in}{2.519369in}}%
\pgfpathlineto{\pgfqpoint{5.599045in}{2.516338in}}%
\pgfpathlineto{\pgfqpoint{5.591701in}{2.512137in}}%
\pgfpathlineto{\pgfqpoint{5.584352in}{2.507944in}}%
\pgfpathlineto{\pgfqpoint{5.576996in}{2.503752in}}%
\pgfpathlineto{\pgfqpoint{5.569634in}{2.499556in}}%
\pgfpathclose%
\pgfusepath{fill}%
\end{pgfscope}%
\begin{pgfscope}%
\pgfpathrectangle{\pgfqpoint{1.150000in}{0.150000in}}{\pgfqpoint{5.700000in}{5.700000in}}%
\pgfusepath{clip}%
\pgfsetbuttcap%
\pgfsetroundjoin%
\definecolor{currentfill}{rgb}{0.269944,0.014625,0.341379}%
\pgfsetfillcolor{currentfill}%
\pgfsetfillopacity{0.700000}%
\pgfsetlinewidth{0.000000pt}%
\definecolor{currentstroke}{rgb}{0.000000,0.000000,0.000000}%
\pgfsetstrokecolor{currentstroke}%
\pgfsetdash{}{0pt}%
\pgfpathmoveto{\pgfqpoint{3.415357in}{1.729178in}}%
\pgfpathlineto{\pgfqpoint{3.429118in}{1.725634in}}%
\pgfpathlineto{\pgfqpoint{3.442885in}{1.722170in}}%
\pgfpathlineto{\pgfqpoint{3.456658in}{1.718789in}}%
\pgfpathlineto{\pgfqpoint{3.470436in}{1.715488in}}%
\pgfpathlineto{\pgfqpoint{3.478696in}{1.724452in}}%
\pgfpathlineto{\pgfqpoint{3.486950in}{1.733443in}}%
\pgfpathlineto{\pgfqpoint{3.495198in}{1.742459in}}%
\pgfpathlineto{\pgfqpoint{3.503439in}{1.751497in}}%
\pgfpathlineto{\pgfqpoint{3.489675in}{1.754614in}}%
\pgfpathlineto{\pgfqpoint{3.475916in}{1.757811in}}%
\pgfpathlineto{\pgfqpoint{3.462163in}{1.761090in}}%
\pgfpathlineto{\pgfqpoint{3.448416in}{1.764451in}}%
\pgfpathlineto{\pgfqpoint{3.440161in}{1.755589in}}%
\pgfpathlineto{\pgfqpoint{3.431899in}{1.746755in}}%
\pgfpathlineto{\pgfqpoint{3.423631in}{1.737950in}}%
\pgfpathlineto{\pgfqpoint{3.415357in}{1.729178in}}%
\pgfpathclose%
\pgfusepath{fill}%
\end{pgfscope}%
\begin{pgfscope}%
\pgfpathrectangle{\pgfqpoint{1.150000in}{0.150000in}}{\pgfqpoint{5.700000in}{5.700000in}}%
\pgfusepath{clip}%
\pgfsetbuttcap%
\pgfsetroundjoin%
\definecolor{currentfill}{rgb}{0.271828,0.209303,0.504434}%
\pgfsetfillcolor{currentfill}%
\pgfsetfillopacity{0.700000}%
\pgfsetlinewidth{0.000000pt}%
\definecolor{currentstroke}{rgb}{0.000000,0.000000,0.000000}%
\pgfsetstrokecolor{currentstroke}%
\pgfsetdash{}{0pt}%
\pgfpathmoveto{\pgfqpoint{2.186993in}{2.143826in}}%
\pgfpathlineto{\pgfqpoint{2.200797in}{2.131182in}}%
\pgfpathlineto{\pgfqpoint{2.214598in}{2.118665in}}%
\pgfpathlineto{\pgfqpoint{2.228397in}{2.106275in}}%
\pgfpathlineto{\pgfqpoint{2.242193in}{2.094010in}}%
\pgfpathlineto{\pgfqpoint{2.251183in}{2.094701in}}%
\pgfpathlineto{\pgfqpoint{2.260156in}{2.095622in}}%
\pgfpathlineto{\pgfqpoint{2.269112in}{2.096769in}}%
\pgfpathlineto{\pgfqpoint{2.278051in}{2.098135in}}%
\pgfpathlineto{\pgfqpoint{2.264291in}{2.110060in}}%
\pgfpathlineto{\pgfqpoint{2.250529in}{2.122110in}}%
\pgfpathlineto{\pgfqpoint{2.236765in}{2.134287in}}%
\pgfpathlineto{\pgfqpoint{2.222998in}{2.146590in}}%
\pgfpathlineto{\pgfqpoint{2.214023in}{2.145556in}}%
\pgfpathlineto{\pgfqpoint{2.205031in}{2.144747in}}%
\pgfpathlineto{\pgfqpoint{2.196021in}{2.144169in}}%
\pgfpathlineto{\pgfqpoint{2.186993in}{2.143826in}}%
\pgfpathclose%
\pgfusepath{fill}%
\end{pgfscope}%
\begin{pgfscope}%
\pgfpathrectangle{\pgfqpoint{1.150000in}{0.150000in}}{\pgfqpoint{5.700000in}{5.700000in}}%
\pgfusepath{clip}%
\pgfsetbuttcap%
\pgfsetroundjoin%
\definecolor{currentfill}{rgb}{0.283091,0.110553,0.431554}%
\pgfsetfillcolor{currentfill}%
\pgfsetfillopacity{0.700000}%
\pgfsetlinewidth{0.000000pt}%
\definecolor{currentstroke}{rgb}{0.000000,0.000000,0.000000}%
\pgfsetstrokecolor{currentstroke}%
\pgfsetdash{}{0pt}%
\pgfpathmoveto{\pgfqpoint{2.498003in}{1.923555in}}%
\pgfpathlineto{\pgfqpoint{2.511744in}{1.913611in}}%
\pgfpathlineto{\pgfqpoint{2.525484in}{1.903776in}}%
\pgfpathlineto{\pgfqpoint{2.539224in}{1.894048in}}%
\pgfpathlineto{\pgfqpoint{2.552965in}{1.884428in}}%
\pgfpathlineto{\pgfqpoint{2.561725in}{1.887625in}}%
\pgfpathlineto{\pgfqpoint{2.570470in}{1.891004in}}%
\pgfpathlineto{\pgfqpoint{2.579203in}{1.894562in}}%
\pgfpathlineto{\pgfqpoint{2.587922in}{1.898293in}}%
\pgfpathlineto{\pgfqpoint{2.574211in}{1.907600in}}%
\pgfpathlineto{\pgfqpoint{2.560501in}{1.917015in}}%
\pgfpathlineto{\pgfqpoint{2.546791in}{1.926537in}}%
\pgfpathlineto{\pgfqpoint{2.533081in}{1.936168in}}%
\pgfpathlineto{\pgfqpoint{2.524332in}{1.932742in}}%
\pgfpathlineto{\pgfqpoint{2.515570in}{1.929494in}}%
\pgfpathlineto{\pgfqpoint{2.506793in}{1.926430in}}%
\pgfpathlineto{\pgfqpoint{2.498003in}{1.923555in}}%
\pgfpathclose%
\pgfusepath{fill}%
\end{pgfscope}%
\begin{pgfscope}%
\pgfpathrectangle{\pgfqpoint{1.150000in}{0.150000in}}{\pgfqpoint{5.700000in}{5.700000in}}%
\pgfusepath{clip}%
\pgfsetbuttcap%
\pgfsetroundjoin%
\definecolor{currentfill}{rgb}{0.282290,0.145912,0.461510}%
\pgfsetfillcolor{currentfill}%
\pgfsetfillopacity{0.700000}%
\pgfsetlinewidth{0.000000pt}%
\definecolor{currentstroke}{rgb}{0.000000,0.000000,0.000000}%
\pgfsetstrokecolor{currentstroke}%
\pgfsetdash{}{0pt}%
\pgfpathmoveto{\pgfqpoint{4.229159in}{1.971763in}}%
\pgfpathlineto{\pgfqpoint{4.243128in}{1.972206in}}%
\pgfpathlineto{\pgfqpoint{4.257105in}{1.972722in}}%
\pgfpathlineto{\pgfqpoint{4.271092in}{1.973312in}}%
\pgfpathlineto{\pgfqpoint{4.285088in}{1.973976in}}%
\pgfpathlineto{\pgfqpoint{4.293055in}{1.983235in}}%
\pgfpathlineto{\pgfqpoint{4.301017in}{1.992429in}}%
\pgfpathlineto{\pgfqpoint{4.308973in}{2.001559in}}%
\pgfpathlineto{\pgfqpoint{4.316923in}{2.010625in}}%
\pgfpathlineto{\pgfqpoint{4.302937in}{2.009942in}}%
\pgfpathlineto{\pgfqpoint{4.288960in}{2.009333in}}%
\pgfpathlineto{\pgfqpoint{4.274992in}{2.008797in}}%
\pgfpathlineto{\pgfqpoint{4.261034in}{2.008335in}}%
\pgfpathlineto{\pgfqpoint{4.253074in}{1.999281in}}%
\pgfpathlineto{\pgfqpoint{4.245108in}{1.990167in}}%
\pgfpathlineto{\pgfqpoint{4.237137in}{1.980995in}}%
\pgfpathlineto{\pgfqpoint{4.229159in}{1.971763in}}%
\pgfpathclose%
\pgfusepath{fill}%
\end{pgfscope}%
\begin{pgfscope}%
\pgfpathrectangle{\pgfqpoint{1.150000in}{0.150000in}}{\pgfqpoint{5.700000in}{5.700000in}}%
\pgfusepath{clip}%
\pgfsetbuttcap%
\pgfsetroundjoin%
\definecolor{currentfill}{rgb}{0.268510,0.009605,0.335427}%
\pgfsetfillcolor{currentfill}%
\pgfsetfillopacity{0.700000}%
\pgfsetlinewidth{0.000000pt}%
\definecolor{currentstroke}{rgb}{0.000000,0.000000,0.000000}%
\pgfsetstrokecolor{currentstroke}%
\pgfsetdash{}{0pt}%
\pgfpathmoveto{\pgfqpoint{3.040329in}{1.729586in}}%
\pgfpathlineto{\pgfqpoint{3.054050in}{1.723657in}}%
\pgfpathlineto{\pgfqpoint{3.067775in}{1.717817in}}%
\pgfpathlineto{\pgfqpoint{3.081503in}{1.712067in}}%
\pgfpathlineto{\pgfqpoint{3.095236in}{1.706404in}}%
\pgfpathlineto{\pgfqpoint{3.103667in}{1.713542in}}%
\pgfpathlineto{\pgfqpoint{3.112090in}{1.720771in}}%
\pgfpathlineto{\pgfqpoint{3.120505in}{1.728086in}}%
\pgfpathlineto{\pgfqpoint{3.128911in}{1.735485in}}%
\pgfpathlineto{\pgfqpoint{3.115198in}{1.740901in}}%
\pgfpathlineto{\pgfqpoint{3.101489in}{1.746406in}}%
\pgfpathlineto{\pgfqpoint{3.087784in}{1.752000in}}%
\pgfpathlineto{\pgfqpoint{3.074082in}{1.757682in}}%
\pgfpathlineto{\pgfqpoint{3.065657in}{1.750522in}}%
\pgfpathlineto{\pgfqpoint{3.057223in}{1.743450in}}%
\pgfpathlineto{\pgfqpoint{3.048780in}{1.736470in}}%
\pgfpathlineto{\pgfqpoint{3.040329in}{1.729586in}}%
\pgfpathclose%
\pgfusepath{fill}%
\end{pgfscope}%
\begin{pgfscope}%
\pgfpathrectangle{\pgfqpoint{1.150000in}{0.150000in}}{\pgfqpoint{5.700000in}{5.700000in}}%
\pgfusepath{clip}%
\pgfsetbuttcap%
\pgfsetroundjoin%
\definecolor{currentfill}{rgb}{0.243113,0.292092,0.538516}%
\pgfsetfillcolor{currentfill}%
\pgfsetfillopacity{0.700000}%
\pgfsetlinewidth{0.000000pt}%
\definecolor{currentstroke}{rgb}{0.000000,0.000000,0.000000}%
\pgfsetstrokecolor{currentstroke}%
\pgfsetdash{}{0pt}%
\pgfpathmoveto{\pgfqpoint{4.987292in}{2.287398in}}%
\pgfpathlineto{\pgfqpoint{5.001537in}{2.289991in}}%
\pgfpathlineto{\pgfqpoint{5.015793in}{2.292654in}}%
\pgfpathlineto{\pgfqpoint{5.030061in}{2.295387in}}%
\pgfpathlineto{\pgfqpoint{5.044340in}{2.298191in}}%
\pgfpathlineto{\pgfqpoint{5.051988in}{2.304599in}}%
\pgfpathlineto{\pgfqpoint{5.059630in}{2.310942in}}%
\pgfpathlineto{\pgfqpoint{5.067263in}{2.317224in}}%
\pgfpathlineto{\pgfqpoint{5.074890in}{2.323446in}}%
\pgfpathlineto{\pgfqpoint{5.060626in}{2.320793in}}%
\pgfpathlineto{\pgfqpoint{5.046373in}{2.318211in}}%
\pgfpathlineto{\pgfqpoint{5.032132in}{2.315699in}}%
\pgfpathlineto{\pgfqpoint{5.017902in}{2.313257in}}%
\pgfpathlineto{\pgfqpoint{5.010260in}{2.306876in}}%
\pgfpathlineto{\pgfqpoint{5.002611in}{2.300442in}}%
\pgfpathlineto{\pgfqpoint{4.994955in}{2.293950in}}%
\pgfpathlineto{\pgfqpoint{4.987292in}{2.287398in}}%
\pgfpathclose%
\pgfusepath{fill}%
\end{pgfscope}%
\begin{pgfscope}%
\pgfpathrectangle{\pgfqpoint{1.150000in}{0.150000in}}{\pgfqpoint{5.700000in}{5.700000in}}%
\pgfusepath{clip}%
\pgfsetbuttcap%
\pgfsetroundjoin%
\definecolor{currentfill}{rgb}{0.272594,0.025563,0.353093}%
\pgfsetfillcolor{currentfill}%
\pgfsetfillopacity{0.700000}%
\pgfsetlinewidth{0.000000pt}%
\definecolor{currentstroke}{rgb}{0.000000,0.000000,0.000000}%
\pgfsetstrokecolor{currentstroke}%
\pgfsetdash{}{0pt}%
\pgfpathmoveto{\pgfqpoint{2.896607in}{1.755913in}}%
\pgfpathlineto{\pgfqpoint{2.910325in}{1.748986in}}%
\pgfpathlineto{\pgfqpoint{2.924046in}{1.742153in}}%
\pgfpathlineto{\pgfqpoint{2.937770in}{1.735412in}}%
\pgfpathlineto{\pgfqpoint{2.951496in}{1.728763in}}%
\pgfpathlineto{\pgfqpoint{2.960006in}{1.734953in}}%
\pgfpathlineto{\pgfqpoint{2.968506in}{1.741258in}}%
\pgfpathlineto{\pgfqpoint{2.976997in}{1.747676in}}%
\pgfpathlineto{\pgfqpoint{2.985478in}{1.754202in}}%
\pgfpathlineto{\pgfqpoint{2.971773in}{1.760583in}}%
\pgfpathlineto{\pgfqpoint{2.958072in}{1.767057in}}%
\pgfpathlineto{\pgfqpoint{2.944373in}{1.773623in}}%
\pgfpathlineto{\pgfqpoint{2.930677in}{1.780282in}}%
\pgfpathlineto{\pgfqpoint{2.922175in}{1.774016in}}%
\pgfpathlineto{\pgfqpoint{2.913662in}{1.767863in}}%
\pgfpathlineto{\pgfqpoint{2.905140in}{1.761827in}}%
\pgfpathlineto{\pgfqpoint{2.896607in}{1.755913in}}%
\pgfpathclose%
\pgfusepath{fill}%
\end{pgfscope}%
\begin{pgfscope}%
\pgfpathrectangle{\pgfqpoint{1.150000in}{0.150000in}}{\pgfqpoint{5.700000in}{5.700000in}}%
\pgfusepath{clip}%
\pgfsetbuttcap%
\pgfsetroundjoin%
\definecolor{currentfill}{rgb}{0.267004,0.004874,0.329415}%
\pgfsetfillcolor{currentfill}%
\pgfsetfillopacity{0.700000}%
\pgfsetlinewidth{0.000000pt}%
\definecolor{currentstroke}{rgb}{0.000000,0.000000,0.000000}%
\pgfsetstrokecolor{currentstroke}%
\pgfsetdash{}{0pt}%
\pgfpathmoveto{\pgfqpoint{3.183804in}{1.714694in}}%
\pgfpathlineto{\pgfqpoint{3.197538in}{1.709712in}}%
\pgfpathlineto{\pgfqpoint{3.211276in}{1.704817in}}%
\pgfpathlineto{\pgfqpoint{3.225019in}{1.700007in}}%
\pgfpathlineto{\pgfqpoint{3.238766in}{1.695282in}}%
\pgfpathlineto{\pgfqpoint{3.247128in}{1.703227in}}%
\pgfpathlineto{\pgfqpoint{3.255483in}{1.711238in}}%
\pgfpathlineto{\pgfqpoint{3.263831in}{1.719311in}}%
\pgfpathlineto{\pgfqpoint{3.272171in}{1.727445in}}%
\pgfpathlineto{\pgfqpoint{3.258441in}{1.731945in}}%
\pgfpathlineto{\pgfqpoint{3.244715in}{1.736529in}}%
\pgfpathlineto{\pgfqpoint{3.230994in}{1.741200in}}%
\pgfpathlineto{\pgfqpoint{3.217278in}{1.745956in}}%
\pgfpathlineto{\pgfqpoint{3.208921in}{1.738039in}}%
\pgfpathlineto{\pgfqpoint{3.200556in}{1.730188in}}%
\pgfpathlineto{\pgfqpoint{3.192184in}{1.722405in}}%
\pgfpathlineto{\pgfqpoint{3.183804in}{1.714694in}}%
\pgfpathclose%
\pgfusepath{fill}%
\end{pgfscope}%
\begin{pgfscope}%
\pgfpathrectangle{\pgfqpoint{1.150000in}{0.150000in}}{\pgfqpoint{5.700000in}{5.700000in}}%
\pgfusepath{clip}%
\pgfsetbuttcap%
\pgfsetroundjoin%
\definecolor{currentfill}{rgb}{0.283072,0.130895,0.449241}%
\pgfsetfillcolor{currentfill}%
\pgfsetfillopacity{0.700000}%
\pgfsetlinewidth{0.000000pt}%
\definecolor{currentstroke}{rgb}{0.000000,0.000000,0.000000}%
\pgfsetstrokecolor{currentstroke}%
\pgfsetdash{}{0pt}%
\pgfpathmoveto{\pgfqpoint{4.141369in}{1.933371in}}%
\pgfpathlineto{\pgfqpoint{4.155312in}{1.933477in}}%
\pgfpathlineto{\pgfqpoint{4.169263in}{1.933658in}}%
\pgfpathlineto{\pgfqpoint{4.183223in}{1.933914in}}%
\pgfpathlineto{\pgfqpoint{4.197192in}{1.934243in}}%
\pgfpathlineto{\pgfqpoint{4.205192in}{1.943711in}}%
\pgfpathlineto{\pgfqpoint{4.213187in}{1.953121in}}%
\pgfpathlineto{\pgfqpoint{4.221176in}{1.962472in}}%
\pgfpathlineto{\pgfqpoint{4.229159in}{1.971763in}}%
\pgfpathlineto{\pgfqpoint{4.215200in}{1.971394in}}%
\pgfpathlineto{\pgfqpoint{4.201250in}{1.971099in}}%
\pgfpathlineto{\pgfqpoint{4.187308in}{1.970878in}}%
\pgfpathlineto{\pgfqpoint{4.173375in}{1.970731in}}%
\pgfpathlineto{\pgfqpoint{4.165382in}{1.961472in}}%
\pgfpathlineto{\pgfqpoint{4.157384in}{1.952159in}}%
\pgfpathlineto{\pgfqpoint{4.149379in}{1.942792in}}%
\pgfpathlineto{\pgfqpoint{4.141369in}{1.933371in}}%
\pgfpathclose%
\pgfusepath{fill}%
\end{pgfscope}%
\begin{pgfscope}%
\pgfpathrectangle{\pgfqpoint{1.150000in}{0.150000in}}{\pgfqpoint{5.700000in}{5.700000in}}%
\pgfusepath{clip}%
\pgfsetbuttcap%
\pgfsetroundjoin%
\definecolor{currentfill}{rgb}{0.274952,0.037752,0.364543}%
\pgfsetfillcolor{currentfill}%
\pgfsetfillopacity{0.700000}%
\pgfsetlinewidth{0.000000pt}%
\definecolor{currentstroke}{rgb}{0.000000,0.000000,0.000000}%
\pgfsetstrokecolor{currentstroke}%
\pgfsetdash{}{0pt}%
\pgfpathmoveto{\pgfqpoint{3.646579in}{1.767092in}}%
\pgfpathlineto{\pgfqpoint{3.660388in}{1.764857in}}%
\pgfpathlineto{\pgfqpoint{3.674203in}{1.762701in}}%
\pgfpathlineto{\pgfqpoint{3.688026in}{1.760624in}}%
\pgfpathlineto{\pgfqpoint{3.701855in}{1.758624in}}%
\pgfpathlineto{\pgfqpoint{3.710029in}{1.768192in}}%
\pgfpathlineto{\pgfqpoint{3.718197in}{1.777753in}}%
\pgfpathlineto{\pgfqpoint{3.726360in}{1.787307in}}%
\pgfpathlineto{\pgfqpoint{3.734516in}{1.796851in}}%
\pgfpathlineto{\pgfqpoint{3.720699in}{1.798707in}}%
\pgfpathlineto{\pgfqpoint{3.706888in}{1.800642in}}%
\pgfpathlineto{\pgfqpoint{3.693084in}{1.802655in}}%
\pgfpathlineto{\pgfqpoint{3.679287in}{1.804746in}}%
\pgfpathlineto{\pgfqpoint{3.671119in}{1.795338in}}%
\pgfpathlineto{\pgfqpoint{3.662945in}{1.785925in}}%
\pgfpathlineto{\pgfqpoint{3.654765in}{1.776509in}}%
\pgfpathlineto{\pgfqpoint{3.646579in}{1.767092in}}%
\pgfpathclose%
\pgfusepath{fill}%
\end{pgfscope}%
\begin{pgfscope}%
\pgfpathrectangle{\pgfqpoint{1.150000in}{0.150000in}}{\pgfqpoint{5.700000in}{5.700000in}}%
\pgfusepath{clip}%
\pgfsetbuttcap%
\pgfsetroundjoin%
\definecolor{currentfill}{rgb}{0.210503,0.363727,0.552206}%
\pgfsetfillcolor{currentfill}%
\pgfsetfillopacity{0.700000}%
\pgfsetlinewidth{0.000000pt}%
\definecolor{currentstroke}{rgb}{0.000000,0.000000,0.000000}%
\pgfsetstrokecolor{currentstroke}%
\pgfsetdash{}{0pt}%
\pgfpathmoveto{\pgfqpoint{5.482277in}{2.469056in}}%
\pgfpathlineto{\pgfqpoint{5.496718in}{2.472350in}}%
\pgfpathlineto{\pgfqpoint{5.511171in}{2.475713in}}%
\pgfpathlineto{\pgfqpoint{5.525637in}{2.479145in}}%
\pgfpathlineto{\pgfqpoint{5.540115in}{2.482646in}}%
\pgfpathlineto{\pgfqpoint{5.547505in}{2.486902in}}%
\pgfpathlineto{\pgfqpoint{5.554888in}{2.491136in}}%
\pgfpathlineto{\pgfqpoint{5.562264in}{2.495353in}}%
\pgfpathlineto{\pgfqpoint{5.569634in}{2.499556in}}%
\pgfpathlineto{\pgfqpoint{5.555178in}{2.496314in}}%
\pgfpathlineto{\pgfqpoint{5.540734in}{2.493141in}}%
\pgfpathlineto{\pgfqpoint{5.526303in}{2.490037in}}%
\pgfpathlineto{\pgfqpoint{5.511884in}{2.487001in}}%
\pgfpathlineto{\pgfqpoint{5.504492in}{2.482531in}}%
\pgfpathlineto{\pgfqpoint{5.497094in}{2.478053in}}%
\pgfpathlineto{\pgfqpoint{5.489689in}{2.473563in}}%
\pgfpathlineto{\pgfqpoint{5.482277in}{2.469056in}}%
\pgfpathclose%
\pgfusepath{fill}%
\end{pgfscope}%
\begin{pgfscope}%
\pgfpathrectangle{\pgfqpoint{1.150000in}{0.150000in}}{\pgfqpoint{5.700000in}{5.700000in}}%
\pgfusepath{clip}%
\pgfsetbuttcap%
\pgfsetroundjoin%
\definecolor{currentfill}{rgb}{0.248629,0.278775,0.534556}%
\pgfsetfillcolor{currentfill}%
\pgfsetfillopacity{0.700000}%
\pgfsetlinewidth{0.000000pt}%
\definecolor{currentstroke}{rgb}{0.000000,0.000000,0.000000}%
\pgfsetstrokecolor{currentstroke}%
\pgfsetdash{}{0pt}%
\pgfpathmoveto{\pgfqpoint{4.899642in}{2.250357in}}%
\pgfpathlineto{\pgfqpoint{4.913856in}{2.252796in}}%
\pgfpathlineto{\pgfqpoint{4.928081in}{2.255307in}}%
\pgfpathlineto{\pgfqpoint{4.942318in}{2.257888in}}%
\pgfpathlineto{\pgfqpoint{4.956566in}{2.260540in}}%
\pgfpathlineto{\pgfqpoint{4.964258in}{2.267358in}}%
\pgfpathlineto{\pgfqpoint{4.971943in}{2.274105in}}%
\pgfpathlineto{\pgfqpoint{4.979621in}{2.280784in}}%
\pgfpathlineto{\pgfqpoint{4.987292in}{2.287398in}}%
\pgfpathlineto{\pgfqpoint{4.973058in}{2.284876in}}%
\pgfpathlineto{\pgfqpoint{4.958836in}{2.282425in}}%
\pgfpathlineto{\pgfqpoint{4.944624in}{2.280044in}}%
\pgfpathlineto{\pgfqpoint{4.930424in}{2.277733in}}%
\pgfpathlineto{\pgfqpoint{4.922739in}{2.270982in}}%
\pgfpathlineto{\pgfqpoint{4.915047in}{2.264170in}}%
\pgfpathlineto{\pgfqpoint{4.907348in}{2.257296in}}%
\pgfpathlineto{\pgfqpoint{4.899642in}{2.250357in}}%
\pgfpathclose%
\pgfusepath{fill}%
\end{pgfscope}%
\begin{pgfscope}%
\pgfpathrectangle{\pgfqpoint{1.150000in}{0.150000in}}{\pgfqpoint{5.700000in}{5.700000in}}%
\pgfusepath{clip}%
\pgfsetbuttcap%
\pgfsetroundjoin%
\definecolor{currentfill}{rgb}{0.276194,0.190074,0.493001}%
\pgfsetfillcolor{currentfill}%
\pgfsetfillopacity{0.700000}%
\pgfsetlinewidth{0.000000pt}%
\definecolor{currentstroke}{rgb}{0.000000,0.000000,0.000000}%
\pgfsetstrokecolor{currentstroke}%
\pgfsetdash{}{0pt}%
\pgfpathmoveto{\pgfqpoint{2.242193in}{2.094010in}}%
\pgfpathlineto{\pgfqpoint{2.255987in}{2.081870in}}%
\pgfpathlineto{\pgfqpoint{2.269779in}{2.069853in}}%
\pgfpathlineto{\pgfqpoint{2.283568in}{2.057958in}}%
\pgfpathlineto{\pgfqpoint{2.297355in}{2.046185in}}%
\pgfpathlineto{\pgfqpoint{2.306309in}{2.047224in}}%
\pgfpathlineto{\pgfqpoint{2.315246in}{2.048487in}}%
\pgfpathlineto{\pgfqpoint{2.324166in}{2.049969in}}%
\pgfpathlineto{\pgfqpoint{2.333070in}{2.051667in}}%
\pgfpathlineto{\pgfqpoint{2.319318in}{2.063101in}}%
\pgfpathlineto{\pgfqpoint{2.305564in}{2.074657in}}%
\pgfpathlineto{\pgfqpoint{2.291808in}{2.086335in}}%
\pgfpathlineto{\pgfqpoint{2.278051in}{2.098135in}}%
\pgfpathlineto{\pgfqpoint{2.269112in}{2.096769in}}%
\pgfpathlineto{\pgfqpoint{2.260156in}{2.095622in}}%
\pgfpathlineto{\pgfqpoint{2.251183in}{2.094701in}}%
\pgfpathlineto{\pgfqpoint{2.242193in}{2.094010in}}%
\pgfpathclose%
\pgfusepath{fill}%
\end{pgfscope}%
\begin{pgfscope}%
\pgfpathrectangle{\pgfqpoint{1.150000in}{0.150000in}}{\pgfqpoint{5.700000in}{5.700000in}}%
\pgfusepath{clip}%
\pgfsetbuttcap%
\pgfsetroundjoin%
\definecolor{currentfill}{rgb}{0.283197,0.115680,0.436115}%
\pgfsetfillcolor{currentfill}%
\pgfsetfillopacity{0.700000}%
\pgfsetlinewidth{0.000000pt}%
\definecolor{currentstroke}{rgb}{0.000000,0.000000,0.000000}%
\pgfsetstrokecolor{currentstroke}%
\pgfsetdash{}{0pt}%
\pgfpathmoveto{\pgfqpoint{4.053549in}{1.895717in}}%
\pgfpathlineto{\pgfqpoint{4.067467in}{1.895465in}}%
\pgfpathlineto{\pgfqpoint{4.081393in}{1.895287in}}%
\pgfpathlineto{\pgfqpoint{4.095328in}{1.895184in}}%
\pgfpathlineto{\pgfqpoint{4.109272in}{1.895156in}}%
\pgfpathlineto{\pgfqpoint{4.117304in}{1.904788in}}%
\pgfpathlineto{\pgfqpoint{4.125332in}{1.914369in}}%
\pgfpathlineto{\pgfqpoint{4.133353in}{1.923896in}}%
\pgfpathlineto{\pgfqpoint{4.141369in}{1.933371in}}%
\pgfpathlineto{\pgfqpoint{4.127435in}{1.933338in}}%
\pgfpathlineto{\pgfqpoint{4.113510in}{1.933380in}}%
\pgfpathlineto{\pgfqpoint{4.099594in}{1.933497in}}%
\pgfpathlineto{\pgfqpoint{4.085686in}{1.933689in}}%
\pgfpathlineto{\pgfqpoint{4.077660in}{1.924267in}}%
\pgfpathlineto{\pgfqpoint{4.069629in}{1.914798in}}%
\pgfpathlineto{\pgfqpoint{4.061592in}{1.905281in}}%
\pgfpathlineto{\pgfqpoint{4.053549in}{1.895717in}}%
\pgfpathclose%
\pgfusepath{fill}%
\end{pgfscope}%
\begin{pgfscope}%
\pgfpathrectangle{\pgfqpoint{1.150000in}{0.150000in}}{\pgfqpoint{5.700000in}{5.700000in}}%
\pgfusepath{clip}%
\pgfsetbuttcap%
\pgfsetroundjoin%
\definecolor{currentfill}{rgb}{0.255645,0.260703,0.528312}%
\pgfsetfillcolor{currentfill}%
\pgfsetfillopacity{0.700000}%
\pgfsetlinewidth{0.000000pt}%
\definecolor{currentstroke}{rgb}{0.000000,0.000000,0.000000}%
\pgfsetstrokecolor{currentstroke}%
\pgfsetdash{}{0pt}%
\pgfpathmoveto{\pgfqpoint{4.811946in}{2.212416in}}%
\pgfpathlineto{\pgfqpoint{4.826129in}{2.214680in}}%
\pgfpathlineto{\pgfqpoint{4.840324in}{2.217016in}}%
\pgfpathlineto{\pgfqpoint{4.854529in}{2.219422in}}%
\pgfpathlineto{\pgfqpoint{4.868746in}{2.221900in}}%
\pgfpathlineto{\pgfqpoint{4.876481in}{2.229124in}}%
\pgfpathlineto{\pgfqpoint{4.884208in}{2.236273in}}%
\pgfpathlineto{\pgfqpoint{4.891928in}{2.243350in}}%
\pgfpathlineto{\pgfqpoint{4.899642in}{2.250357in}}%
\pgfpathlineto{\pgfqpoint{4.885438in}{2.247988in}}%
\pgfpathlineto{\pgfqpoint{4.871246in}{2.245689in}}%
\pgfpathlineto{\pgfqpoint{4.857065in}{2.243462in}}%
\pgfpathlineto{\pgfqpoint{4.842894in}{2.241306in}}%
\pgfpathlineto{\pgfqpoint{4.835168in}{2.234182in}}%
\pgfpathlineto{\pgfqpoint{4.827434in}{2.226995in}}%
\pgfpathlineto{\pgfqpoint{4.819693in}{2.219740in}}%
\pgfpathlineto{\pgfqpoint{4.811946in}{2.212416in}}%
\pgfpathclose%
\pgfusepath{fill}%
\end{pgfscope}%
\begin{pgfscope}%
\pgfpathrectangle{\pgfqpoint{1.150000in}{0.150000in}}{\pgfqpoint{5.700000in}{5.700000in}}%
\pgfusepath{clip}%
\pgfsetbuttcap%
\pgfsetroundjoin%
\definecolor{currentfill}{rgb}{0.268510,0.009605,0.335427}%
\pgfsetfillcolor{currentfill}%
\pgfsetfillopacity{0.700000}%
\pgfsetlinewidth{0.000000pt}%
\definecolor{currentstroke}{rgb}{0.000000,0.000000,0.000000}%
\pgfsetstrokecolor{currentstroke}%
\pgfsetdash{}{0pt}%
\pgfpathmoveto{\pgfqpoint{3.327141in}{1.710290in}}%
\pgfpathlineto{\pgfqpoint{3.340896in}{1.706211in}}%
\pgfpathlineto{\pgfqpoint{3.354656in}{1.702214in}}%
\pgfpathlineto{\pgfqpoint{3.368421in}{1.698300in}}%
\pgfpathlineto{\pgfqpoint{3.382192in}{1.694469in}}%
\pgfpathlineto{\pgfqpoint{3.390493in}{1.703084in}}%
\pgfpathlineto{\pgfqpoint{3.398788in}{1.711742in}}%
\pgfpathlineto{\pgfqpoint{3.407076in}{1.720441in}}%
\pgfpathlineto{\pgfqpoint{3.415357in}{1.729178in}}%
\pgfpathlineto{\pgfqpoint{3.401601in}{1.732805in}}%
\pgfpathlineto{\pgfqpoint{3.387851in}{1.736515in}}%
\pgfpathlineto{\pgfqpoint{3.374106in}{1.740307in}}%
\pgfpathlineto{\pgfqpoint{3.360366in}{1.744182in}}%
\pgfpathlineto{\pgfqpoint{3.352070in}{1.735641in}}%
\pgfpathlineto{\pgfqpoint{3.343767in}{1.727144in}}%
\pgfpathlineto{\pgfqpoint{3.335457in}{1.718693in}}%
\pgfpathlineto{\pgfqpoint{3.327141in}{1.710290in}}%
\pgfpathclose%
\pgfusepath{fill}%
\end{pgfscope}%
\begin{pgfscope}%
\pgfpathrectangle{\pgfqpoint{1.150000in}{0.150000in}}{\pgfqpoint{5.700000in}{5.700000in}}%
\pgfusepath{clip}%
\pgfsetbuttcap%
\pgfsetroundjoin%
\definecolor{currentfill}{rgb}{0.277018,0.050344,0.375715}%
\pgfsetfillcolor{currentfill}%
\pgfsetfillopacity{0.700000}%
\pgfsetlinewidth{0.000000pt}%
\definecolor{currentstroke}{rgb}{0.000000,0.000000,0.000000}%
\pgfsetstrokecolor{currentstroke}%
\pgfsetdash{}{0pt}%
\pgfpathmoveto{\pgfqpoint{2.752519in}{1.794690in}}%
\pgfpathlineto{\pgfqpoint{2.766244in}{1.786710in}}%
\pgfpathlineto{\pgfqpoint{2.779971in}{1.778827in}}%
\pgfpathlineto{\pgfqpoint{2.793700in}{1.771042in}}%
\pgfpathlineto{\pgfqpoint{2.807430in}{1.763353in}}%
\pgfpathlineto{\pgfqpoint{2.816029in}{1.768444in}}%
\pgfpathlineto{\pgfqpoint{2.824617in}{1.773679in}}%
\pgfpathlineto{\pgfqpoint{2.833194in}{1.779052in}}%
\pgfpathlineto{\pgfqpoint{2.841760in}{1.784561in}}%
\pgfpathlineto{\pgfqpoint{2.828054in}{1.791960in}}%
\pgfpathlineto{\pgfqpoint{2.814351in}{1.799457in}}%
\pgfpathlineto{\pgfqpoint{2.800649in}{1.807050in}}%
\pgfpathlineto{\pgfqpoint{2.786949in}{1.814741in}}%
\pgfpathlineto{\pgfqpoint{2.778359in}{1.809514in}}%
\pgfpathlineto{\pgfqpoint{2.769757in}{1.804427in}}%
\pgfpathlineto{\pgfqpoint{2.761144in}{1.799484in}}%
\pgfpathlineto{\pgfqpoint{2.752519in}{1.794690in}}%
\pgfpathclose%
\pgfusepath{fill}%
\end{pgfscope}%
\begin{pgfscope}%
\pgfpathrectangle{\pgfqpoint{1.150000in}{0.150000in}}{\pgfqpoint{5.700000in}{5.700000in}}%
\pgfusepath{clip}%
\pgfsetbuttcap%
\pgfsetroundjoin%
\definecolor{currentfill}{rgb}{0.214298,0.355619,0.551184}%
\pgfsetfillcolor{currentfill}%
\pgfsetfillopacity{0.700000}%
\pgfsetlinewidth{0.000000pt}%
\definecolor{currentstroke}{rgb}{0.000000,0.000000,0.000000}%
\pgfsetstrokecolor{currentstroke}%
\pgfsetdash{}{0pt}%
\pgfpathmoveto{\pgfqpoint{5.394836in}{2.437332in}}%
\pgfpathlineto{\pgfqpoint{5.409248in}{2.440587in}}%
\pgfpathlineto{\pgfqpoint{5.423673in}{2.443911in}}%
\pgfpathlineto{\pgfqpoint{5.438109in}{2.447305in}}%
\pgfpathlineto{\pgfqpoint{5.452559in}{2.450768in}}%
\pgfpathlineto{\pgfqpoint{5.459999in}{2.455387in}}%
\pgfpathlineto{\pgfqpoint{5.467432in}{2.459972in}}%
\pgfpathlineto{\pgfqpoint{5.474858in}{2.464527in}}%
\pgfpathlineto{\pgfqpoint{5.482277in}{2.469056in}}%
\pgfpathlineto{\pgfqpoint{5.467849in}{2.465831in}}%
\pgfpathlineto{\pgfqpoint{5.453433in}{2.462675in}}%
\pgfpathlineto{\pgfqpoint{5.439029in}{2.459588in}}%
\pgfpathlineto{\pgfqpoint{5.424637in}{2.456570in}}%
\pgfpathlineto{\pgfqpoint{5.417197in}{2.451796in}}%
\pgfpathlineto{\pgfqpoint{5.409751in}{2.447001in}}%
\pgfpathlineto{\pgfqpoint{5.402297in}{2.442181in}}%
\pgfpathlineto{\pgfqpoint{5.394836in}{2.437332in}}%
\pgfpathclose%
\pgfusepath{fill}%
\end{pgfscope}%
\begin{pgfscope}%
\pgfpathrectangle{\pgfqpoint{1.150000in}{0.150000in}}{\pgfqpoint{5.700000in}{5.700000in}}%
\pgfusepath{clip}%
\pgfsetbuttcap%
\pgfsetroundjoin%
\definecolor{currentfill}{rgb}{0.282327,0.094955,0.417331}%
\pgfsetfillcolor{currentfill}%
\pgfsetfillopacity{0.700000}%
\pgfsetlinewidth{0.000000pt}%
\definecolor{currentstroke}{rgb}{0.000000,0.000000,0.000000}%
\pgfsetstrokecolor{currentstroke}%
\pgfsetdash{}{0pt}%
\pgfpathmoveto{\pgfqpoint{2.552965in}{1.884428in}}%
\pgfpathlineto{\pgfqpoint{2.566706in}{1.874914in}}%
\pgfpathlineto{\pgfqpoint{2.580447in}{1.865507in}}%
\pgfpathlineto{\pgfqpoint{2.594189in}{1.856204in}}%
\pgfpathlineto{\pgfqpoint{2.607931in}{1.847005in}}%
\pgfpathlineto{\pgfqpoint{2.616661in}{1.850522in}}%
\pgfpathlineto{\pgfqpoint{2.625377in}{1.854216in}}%
\pgfpathlineto{\pgfqpoint{2.634081in}{1.858083in}}%
\pgfpathlineto{\pgfqpoint{2.642772in}{1.862118in}}%
\pgfpathlineto{\pgfqpoint{2.629058in}{1.871005in}}%
\pgfpathlineto{\pgfqpoint{2.615346in}{1.879996in}}%
\pgfpathlineto{\pgfqpoint{2.601634in}{1.889091in}}%
\pgfpathlineto{\pgfqpoint{2.587922in}{1.898293in}}%
\pgfpathlineto{\pgfqpoint{2.579203in}{1.894562in}}%
\pgfpathlineto{\pgfqpoint{2.570470in}{1.891004in}}%
\pgfpathlineto{\pgfqpoint{2.561725in}{1.887625in}}%
\pgfpathlineto{\pgfqpoint{2.552965in}{1.884428in}}%
\pgfpathclose%
\pgfusepath{fill}%
\end{pgfscope}%
\begin{pgfscope}%
\pgfpathrectangle{\pgfqpoint{1.150000in}{0.150000in}}{\pgfqpoint{5.700000in}{5.700000in}}%
\pgfusepath{clip}%
\pgfsetbuttcap%
\pgfsetroundjoin%
\definecolor{currentfill}{rgb}{0.272594,0.025563,0.353093}%
\pgfsetfillcolor{currentfill}%
\pgfsetfillopacity{0.700000}%
\pgfsetlinewidth{0.000000pt}%
\definecolor{currentstroke}{rgb}{0.000000,0.000000,0.000000}%
\pgfsetstrokecolor{currentstroke}%
\pgfsetdash{}{0pt}%
\pgfpathmoveto{\pgfqpoint{3.558558in}{1.739837in}}%
\pgfpathlineto{\pgfqpoint{3.572353in}{1.737123in}}%
\pgfpathlineto{\pgfqpoint{3.586155in}{1.734487in}}%
\pgfpathlineto{\pgfqpoint{3.599963in}{1.731931in}}%
\pgfpathlineto{\pgfqpoint{3.613777in}{1.729454in}}%
\pgfpathlineto{\pgfqpoint{3.621986in}{1.738855in}}%
\pgfpathlineto{\pgfqpoint{3.630190in}{1.748263in}}%
\pgfpathlineto{\pgfqpoint{3.638387in}{1.757676in}}%
\pgfpathlineto{\pgfqpoint{3.646579in}{1.767092in}}%
\pgfpathlineto{\pgfqpoint{3.632777in}{1.769405in}}%
\pgfpathlineto{\pgfqpoint{3.618982in}{1.771798in}}%
\pgfpathlineto{\pgfqpoint{3.605193in}{1.774269in}}%
\pgfpathlineto{\pgfqpoint{3.591410in}{1.776821in}}%
\pgfpathlineto{\pgfqpoint{3.583206in}{1.767561in}}%
\pgfpathlineto{\pgfqpoint{3.574996in}{1.758308in}}%
\pgfpathlineto{\pgfqpoint{3.566780in}{1.749067in}}%
\pgfpathlineto{\pgfqpoint{3.558558in}{1.739837in}}%
\pgfpathclose%
\pgfusepath{fill}%
\end{pgfscope}%
\begin{pgfscope}%
\pgfpathrectangle{\pgfqpoint{1.150000in}{0.150000in}}{\pgfqpoint{5.700000in}{5.700000in}}%
\pgfusepath{clip}%
\pgfsetbuttcap%
\pgfsetroundjoin%
\definecolor{currentfill}{rgb}{0.282327,0.094955,0.417331}%
\pgfsetfillcolor{currentfill}%
\pgfsetfillopacity{0.700000}%
\pgfsetlinewidth{0.000000pt}%
\definecolor{currentstroke}{rgb}{0.000000,0.000000,0.000000}%
\pgfsetstrokecolor{currentstroke}%
\pgfsetdash{}{0pt}%
\pgfpathmoveto{\pgfqpoint{3.965696in}{1.859093in}}%
\pgfpathlineto{\pgfqpoint{3.979590in}{1.858458in}}%
\pgfpathlineto{\pgfqpoint{3.993493in}{1.857899in}}%
\pgfpathlineto{\pgfqpoint{4.007404in}{1.857415in}}%
\pgfpathlineto{\pgfqpoint{4.021323in}{1.857007in}}%
\pgfpathlineto{\pgfqpoint{4.029388in}{1.866751in}}%
\pgfpathlineto{\pgfqpoint{4.037447in}{1.876451in}}%
\pgfpathlineto{\pgfqpoint{4.045501in}{1.886107in}}%
\pgfpathlineto{\pgfqpoint{4.053549in}{1.895717in}}%
\pgfpathlineto{\pgfqpoint{4.039640in}{1.896045in}}%
\pgfpathlineto{\pgfqpoint{4.025739in}{1.896447in}}%
\pgfpathlineto{\pgfqpoint{4.011846in}{1.896925in}}%
\pgfpathlineto{\pgfqpoint{3.997961in}{1.897478in}}%
\pgfpathlineto{\pgfqpoint{3.989903in}{1.887941in}}%
\pgfpathlineto{\pgfqpoint{3.981839in}{1.878365in}}%
\pgfpathlineto{\pgfqpoint{3.973770in}{1.868748in}}%
\pgfpathlineto{\pgfqpoint{3.965696in}{1.859093in}}%
\pgfpathclose%
\pgfusepath{fill}%
\end{pgfscope}%
\begin{pgfscope}%
\pgfpathrectangle{\pgfqpoint{1.150000in}{0.150000in}}{\pgfqpoint{5.700000in}{5.700000in}}%
\pgfusepath{clip}%
\pgfsetbuttcap%
\pgfsetroundjoin%
\definecolor{currentfill}{rgb}{0.260571,0.246922,0.522828}%
\pgfsetfillcolor{currentfill}%
\pgfsetfillopacity{0.700000}%
\pgfsetlinewidth{0.000000pt}%
\definecolor{currentstroke}{rgb}{0.000000,0.000000,0.000000}%
\pgfsetstrokecolor{currentstroke}%
\pgfsetdash{}{0pt}%
\pgfpathmoveto{\pgfqpoint{4.724210in}{2.173692in}}%
\pgfpathlineto{\pgfqpoint{4.738363in}{2.175759in}}%
\pgfpathlineto{\pgfqpoint{4.752526in}{2.177897in}}%
\pgfpathlineto{\pgfqpoint{4.766701in}{2.180106in}}%
\pgfpathlineto{\pgfqpoint{4.780886in}{2.182387in}}%
\pgfpathlineto{\pgfqpoint{4.788661in}{2.190008in}}%
\pgfpathlineto{\pgfqpoint{4.796430in}{2.197552in}}%
\pgfpathlineto{\pgfqpoint{4.804191in}{2.205020in}}%
\pgfpathlineto{\pgfqpoint{4.811946in}{2.212416in}}%
\pgfpathlineto{\pgfqpoint{4.797773in}{2.210222in}}%
\pgfpathlineto{\pgfqpoint{4.783611in}{2.208100in}}%
\pgfpathlineto{\pgfqpoint{4.769460in}{2.206049in}}%
\pgfpathlineto{\pgfqpoint{4.755319in}{2.204069in}}%
\pgfpathlineto{\pgfqpoint{4.747552in}{2.196579in}}%
\pgfpathlineto{\pgfqpoint{4.739778in}{2.189021in}}%
\pgfpathlineto{\pgfqpoint{4.731998in}{2.181392in}}%
\pgfpathlineto{\pgfqpoint{4.724210in}{2.173692in}}%
\pgfpathclose%
\pgfusepath{fill}%
\end{pgfscope}%
\begin{pgfscope}%
\pgfpathrectangle{\pgfqpoint{1.150000in}{0.150000in}}{\pgfqpoint{5.700000in}{5.700000in}}%
\pgfusepath{clip}%
\pgfsetbuttcap%
\pgfsetroundjoin%
\definecolor{currentfill}{rgb}{0.279574,0.170599,0.479997}%
\pgfsetfillcolor{currentfill}%
\pgfsetfillopacity{0.700000}%
\pgfsetlinewidth{0.000000pt}%
\definecolor{currentstroke}{rgb}{0.000000,0.000000,0.000000}%
\pgfsetstrokecolor{currentstroke}%
\pgfsetdash{}{0pt}%
\pgfpathmoveto{\pgfqpoint{2.297355in}{2.046185in}}%
\pgfpathlineto{\pgfqpoint{2.311141in}{2.034533in}}%
\pgfpathlineto{\pgfqpoint{2.324925in}{2.023000in}}%
\pgfpathlineto{\pgfqpoint{2.338707in}{2.011585in}}%
\pgfpathlineto{\pgfqpoint{2.352488in}{2.000288in}}%
\pgfpathlineto{\pgfqpoint{2.361406in}{2.001672in}}%
\pgfpathlineto{\pgfqpoint{2.370307in}{2.003275in}}%
\pgfpathlineto{\pgfqpoint{2.379193in}{2.005093in}}%
\pgfpathlineto{\pgfqpoint{2.388063in}{2.007120in}}%
\pgfpathlineto{\pgfqpoint{2.374317in}{2.018080in}}%
\pgfpathlineto{\pgfqpoint{2.360569in}{2.029157in}}%
\pgfpathlineto{\pgfqpoint{2.346820in}{2.040352in}}%
\pgfpathlineto{\pgfqpoint{2.333070in}{2.051667in}}%
\pgfpathlineto{\pgfqpoint{2.324166in}{2.049969in}}%
\pgfpathlineto{\pgfqpoint{2.315246in}{2.048487in}}%
\pgfpathlineto{\pgfqpoint{2.306309in}{2.047224in}}%
\pgfpathlineto{\pgfqpoint{2.297355in}{2.046185in}}%
\pgfpathclose%
\pgfusepath{fill}%
\end{pgfscope}%
\begin{pgfscope}%
\pgfpathrectangle{\pgfqpoint{1.150000in}{0.150000in}}{\pgfqpoint{5.700000in}{5.700000in}}%
\pgfusepath{clip}%
\pgfsetbuttcap%
\pgfsetroundjoin%
\definecolor{currentfill}{rgb}{0.280894,0.078907,0.402329}%
\pgfsetfillcolor{currentfill}%
\pgfsetfillopacity{0.700000}%
\pgfsetlinewidth{0.000000pt}%
\definecolor{currentstroke}{rgb}{0.000000,0.000000,0.000000}%
\pgfsetstrokecolor{currentstroke}%
\pgfsetdash{}{0pt}%
\pgfpathmoveto{\pgfqpoint{3.877801in}{1.823810in}}%
\pgfpathlineto{\pgfqpoint{3.891675in}{1.822770in}}%
\pgfpathlineto{\pgfqpoint{3.905556in}{1.821806in}}%
\pgfpathlineto{\pgfqpoint{3.919444in}{1.820918in}}%
\pgfpathlineto{\pgfqpoint{3.933341in}{1.820105in}}%
\pgfpathlineto{\pgfqpoint{3.941438in}{1.829905in}}%
\pgfpathlineto{\pgfqpoint{3.949529in}{1.839670in}}%
\pgfpathlineto{\pgfqpoint{3.957615in}{1.849400in}}%
\pgfpathlineto{\pgfqpoint{3.965696in}{1.859093in}}%
\pgfpathlineto{\pgfqpoint{3.951809in}{1.859803in}}%
\pgfpathlineto{\pgfqpoint{3.937930in}{1.860589in}}%
\pgfpathlineto{\pgfqpoint{3.924060in}{1.861451in}}%
\pgfpathlineto{\pgfqpoint{3.910197in}{1.862389in}}%
\pgfpathlineto{\pgfqpoint{3.902106in}{1.852791in}}%
\pgfpathlineto{\pgfqpoint{3.894010in}{1.843160in}}%
\pgfpathlineto{\pgfqpoint{3.885909in}{1.833500in}}%
\pgfpathlineto{\pgfqpoint{3.877801in}{1.823810in}}%
\pgfpathclose%
\pgfusepath{fill}%
\end{pgfscope}%
\begin{pgfscope}%
\pgfpathrectangle{\pgfqpoint{1.150000in}{0.150000in}}{\pgfqpoint{5.700000in}{5.700000in}}%
\pgfusepath{clip}%
\pgfsetbuttcap%
\pgfsetroundjoin%
\definecolor{currentfill}{rgb}{0.266580,0.228262,0.514349}%
\pgfsetfillcolor{currentfill}%
\pgfsetfillopacity{0.700000}%
\pgfsetlinewidth{0.000000pt}%
\definecolor{currentstroke}{rgb}{0.000000,0.000000,0.000000}%
\pgfsetstrokecolor{currentstroke}%
\pgfsetdash{}{0pt}%
\pgfpathmoveto{\pgfqpoint{4.636439in}{2.134326in}}%
\pgfpathlineto{\pgfqpoint{4.650562in}{2.136172in}}%
\pgfpathlineto{\pgfqpoint{4.664694in}{2.138090in}}%
\pgfpathlineto{\pgfqpoint{4.678838in}{2.140080in}}%
\pgfpathlineto{\pgfqpoint{4.692992in}{2.142141in}}%
\pgfpathlineto{\pgfqpoint{4.700806in}{2.150145in}}%
\pgfpathlineto{\pgfqpoint{4.708614in}{2.158070in}}%
\pgfpathlineto{\pgfqpoint{4.716416in}{2.165919in}}%
\pgfpathlineto{\pgfqpoint{4.724210in}{2.173692in}}%
\pgfpathlineto{\pgfqpoint{4.710068in}{2.171697in}}%
\pgfpathlineto{\pgfqpoint{4.695936in}{2.169773in}}%
\pgfpathlineto{\pgfqpoint{4.681815in}{2.167921in}}%
\pgfpathlineto{\pgfqpoint{4.667704in}{2.166141in}}%
\pgfpathlineto{\pgfqpoint{4.659898in}{2.158294in}}%
\pgfpathlineto{\pgfqpoint{4.652085in}{2.150377in}}%
\pgfpathlineto{\pgfqpoint{4.644266in}{2.142388in}}%
\pgfpathlineto{\pgfqpoint{4.636439in}{2.134326in}}%
\pgfpathclose%
\pgfusepath{fill}%
\end{pgfscope}%
\begin{pgfscope}%
\pgfpathrectangle{\pgfqpoint{1.150000in}{0.150000in}}{\pgfqpoint{5.700000in}{5.700000in}}%
\pgfusepath{clip}%
\pgfsetbuttcap%
\pgfsetroundjoin%
\definecolor{currentfill}{rgb}{0.220057,0.343307,0.549413}%
\pgfsetfillcolor{currentfill}%
\pgfsetfillopacity{0.700000}%
\pgfsetlinewidth{0.000000pt}%
\definecolor{currentstroke}{rgb}{0.000000,0.000000,0.000000}%
\pgfsetstrokecolor{currentstroke}%
\pgfsetdash{}{0pt}%
\pgfpathmoveto{\pgfqpoint{5.307315in}{2.404367in}}%
\pgfpathlineto{\pgfqpoint{5.321698in}{2.407560in}}%
\pgfpathlineto{\pgfqpoint{5.336093in}{2.410823in}}%
\pgfpathlineto{\pgfqpoint{5.350500in}{2.414156in}}%
\pgfpathlineto{\pgfqpoint{5.364919in}{2.417558in}}%
\pgfpathlineto{\pgfqpoint{5.372410in}{2.422566in}}%
\pgfpathlineto{\pgfqpoint{5.379893in}{2.427528in}}%
\pgfpathlineto{\pgfqpoint{5.387368in}{2.432449in}}%
\pgfpathlineto{\pgfqpoint{5.394836in}{2.437332in}}%
\pgfpathlineto{\pgfqpoint{5.380436in}{2.434146in}}%
\pgfpathlineto{\pgfqpoint{5.366048in}{2.431030in}}%
\pgfpathlineto{\pgfqpoint{5.351673in}{2.427982in}}%
\pgfpathlineto{\pgfqpoint{5.337309in}{2.425005in}}%
\pgfpathlineto{\pgfqpoint{5.329821in}{2.419898in}}%
\pgfpathlineto{\pgfqpoint{5.322327in}{2.414759in}}%
\pgfpathlineto{\pgfqpoint{5.314825in}{2.409583in}}%
\pgfpathlineto{\pgfqpoint{5.307315in}{2.404367in}}%
\pgfpathclose%
\pgfusepath{fill}%
\end{pgfscope}%
\begin{pgfscope}%
\pgfpathrectangle{\pgfqpoint{1.150000in}{0.150000in}}{\pgfqpoint{5.700000in}{5.700000in}}%
\pgfusepath{clip}%
\pgfsetbuttcap%
\pgfsetroundjoin%
\definecolor{currentfill}{rgb}{0.195860,0.395433,0.555276}%
\pgfsetfillcolor{currentfill}%
\pgfsetfillopacity{0.700000}%
\pgfsetlinewidth{0.000000pt}%
\definecolor{currentstroke}{rgb}{0.000000,0.000000,0.000000}%
\pgfsetstrokecolor{currentstroke}%
\pgfsetdash{}{0pt}%
\pgfpathmoveto{\pgfqpoint{5.714957in}{2.542505in}}%
\pgfpathlineto{\pgfqpoint{5.729503in}{2.546084in}}%
\pgfpathlineto{\pgfqpoint{5.744062in}{2.549732in}}%
\pgfpathlineto{\pgfqpoint{5.758635in}{2.553448in}}%
\pgfpathlineto{\pgfqpoint{5.765904in}{2.556853in}}%
\pgfpathlineto{\pgfqpoint{5.773167in}{2.560260in}}%
\pgfpathlineto{\pgfqpoint{5.780424in}{2.563677in}}%
\pgfpathlineto{\pgfqpoint{5.787674in}{2.567106in}}%
\pgfpathlineto{\pgfqpoint{5.773128in}{2.563693in}}%
\pgfpathlineto{\pgfqpoint{5.758594in}{2.560347in}}%
\pgfpathlineto{\pgfqpoint{5.744073in}{2.557070in}}%
\pgfpathlineto{\pgfqpoint{5.736804in}{2.553408in}}%
\pgfpathlineto{\pgfqpoint{5.729528in}{2.549763in}}%
\pgfpathlineto{\pgfqpoint{5.722246in}{2.546131in}}%
\pgfpathlineto{\pgfqpoint{5.714957in}{2.542505in}}%
\pgfpathclose%
\pgfusepath{fill}%
\end{pgfscope}%
\begin{pgfscope}%
\pgfpathrectangle{\pgfqpoint{1.150000in}{0.150000in}}{\pgfqpoint{5.700000in}{5.700000in}}%
\pgfusepath{clip}%
\pgfsetbuttcap%
\pgfsetroundjoin%
\definecolor{currentfill}{rgb}{0.268510,0.009605,0.335427}%
\pgfsetfillcolor{currentfill}%
\pgfsetfillopacity{0.700000}%
\pgfsetlinewidth{0.000000pt}%
\definecolor{currentstroke}{rgb}{0.000000,0.000000,0.000000}%
\pgfsetstrokecolor{currentstroke}%
\pgfsetdash{}{0pt}%
\pgfpathmoveto{\pgfqpoint{3.095236in}{1.706404in}}%
\pgfpathlineto{\pgfqpoint{3.108972in}{1.700830in}}%
\pgfpathlineto{\pgfqpoint{3.122712in}{1.695343in}}%
\pgfpathlineto{\pgfqpoint{3.136456in}{1.689943in}}%
\pgfpathlineto{\pgfqpoint{3.150204in}{1.684629in}}%
\pgfpathlineto{\pgfqpoint{3.158616in}{1.692021in}}%
\pgfpathlineto{\pgfqpoint{3.167020in}{1.699498in}}%
\pgfpathlineto{\pgfqpoint{3.175416in}{1.707057in}}%
\pgfpathlineto{\pgfqpoint{3.183804in}{1.714694in}}%
\pgfpathlineto{\pgfqpoint{3.170074in}{1.719761in}}%
\pgfpathlineto{\pgfqpoint{3.156349in}{1.724915in}}%
\pgfpathlineto{\pgfqpoint{3.142628in}{1.730156in}}%
\pgfpathlineto{\pgfqpoint{3.128911in}{1.735485in}}%
\pgfpathlineto{\pgfqpoint{3.120505in}{1.728086in}}%
\pgfpathlineto{\pgfqpoint{3.112090in}{1.720771in}}%
\pgfpathlineto{\pgfqpoint{3.103667in}{1.713542in}}%
\pgfpathlineto{\pgfqpoint{3.095236in}{1.706404in}}%
\pgfpathclose%
\pgfusepath{fill}%
\end{pgfscope}%
\begin{pgfscope}%
\pgfpathrectangle{\pgfqpoint{1.150000in}{0.150000in}}{\pgfqpoint{5.700000in}{5.700000in}}%
\pgfusepath{clip}%
\pgfsetbuttcap%
\pgfsetroundjoin%
\definecolor{currentfill}{rgb}{0.271305,0.019942,0.347269}%
\pgfsetfillcolor{currentfill}%
\pgfsetfillopacity{0.700000}%
\pgfsetlinewidth{0.000000pt}%
\definecolor{currentstroke}{rgb}{0.000000,0.000000,0.000000}%
\pgfsetstrokecolor{currentstroke}%
\pgfsetdash{}{0pt}%
\pgfpathmoveto{\pgfqpoint{2.951496in}{1.728763in}}%
\pgfpathlineto{\pgfqpoint{2.965226in}{1.722206in}}%
\pgfpathlineto{\pgfqpoint{2.978958in}{1.715740in}}%
\pgfpathlineto{\pgfqpoint{2.992694in}{1.709365in}}%
\pgfpathlineto{\pgfqpoint{3.006433in}{1.703079in}}%
\pgfpathlineto{\pgfqpoint{3.014921in}{1.709544in}}%
\pgfpathlineto{\pgfqpoint{3.023399in}{1.716119in}}%
\pgfpathlineto{\pgfqpoint{3.031869in}{1.722801in}}%
\pgfpathlineto{\pgfqpoint{3.040329in}{1.729586in}}%
\pgfpathlineto{\pgfqpoint{3.026611in}{1.735604in}}%
\pgfpathlineto{\pgfqpoint{3.012897in}{1.741712in}}%
\pgfpathlineto{\pgfqpoint{2.999186in}{1.747911in}}%
\pgfpathlineto{\pgfqpoint{2.985478in}{1.754202in}}%
\pgfpathlineto{\pgfqpoint{2.976997in}{1.747676in}}%
\pgfpathlineto{\pgfqpoint{2.968506in}{1.741258in}}%
\pgfpathlineto{\pgfqpoint{2.960006in}{1.734953in}}%
\pgfpathlineto{\pgfqpoint{2.951496in}{1.728763in}}%
\pgfpathclose%
\pgfusepath{fill}%
\end{pgfscope}%
\begin{pgfscope}%
\pgfpathrectangle{\pgfqpoint{1.150000in}{0.150000in}}{\pgfqpoint{5.700000in}{5.700000in}}%
\pgfusepath{clip}%
\pgfsetbuttcap%
\pgfsetroundjoin%
\definecolor{currentfill}{rgb}{0.270595,0.214069,0.507052}%
\pgfsetfillcolor{currentfill}%
\pgfsetfillopacity{0.700000}%
\pgfsetlinewidth{0.000000pt}%
\definecolor{currentstroke}{rgb}{0.000000,0.000000,0.000000}%
\pgfsetstrokecolor{currentstroke}%
\pgfsetdash{}{0pt}%
\pgfpathmoveto{\pgfqpoint{4.548639in}{2.094478in}}%
\pgfpathlineto{\pgfqpoint{4.562731in}{2.096081in}}%
\pgfpathlineto{\pgfqpoint{4.576833in}{2.097756in}}%
\pgfpathlineto{\pgfqpoint{4.590946in}{2.099503in}}%
\pgfpathlineto{\pgfqpoint{4.605069in}{2.101322in}}%
\pgfpathlineto{\pgfqpoint{4.612921in}{2.109689in}}%
\pgfpathlineto{\pgfqpoint{4.620767in}{2.117978in}}%
\pgfpathlineto{\pgfqpoint{4.628607in}{2.126190in}}%
\pgfpathlineto{\pgfqpoint{4.636439in}{2.134326in}}%
\pgfpathlineto{\pgfqpoint{4.622328in}{2.132552in}}%
\pgfpathlineto{\pgfqpoint{4.608226in}{2.130849in}}%
\pgfpathlineto{\pgfqpoint{4.594135in}{2.129219in}}%
\pgfpathlineto{\pgfqpoint{4.580053in}{2.127660in}}%
\pgfpathlineto{\pgfqpoint{4.572210in}{2.119472in}}%
\pgfpathlineto{\pgfqpoint{4.564359in}{2.111213in}}%
\pgfpathlineto{\pgfqpoint{4.556502in}{2.102882in}}%
\pgfpathlineto{\pgfqpoint{4.548639in}{2.094478in}}%
\pgfpathclose%
\pgfusepath{fill}%
\end{pgfscope}%
\begin{pgfscope}%
\pgfpathrectangle{\pgfqpoint{1.150000in}{0.150000in}}{\pgfqpoint{5.700000in}{5.700000in}}%
\pgfusepath{clip}%
\pgfsetbuttcap%
\pgfsetroundjoin%
\definecolor{currentfill}{rgb}{0.271305,0.019942,0.347269}%
\pgfsetfillcolor{currentfill}%
\pgfsetfillopacity{0.700000}%
\pgfsetlinewidth{0.000000pt}%
\definecolor{currentstroke}{rgb}{0.000000,0.000000,0.000000}%
\pgfsetstrokecolor{currentstroke}%
\pgfsetdash{}{0pt}%
\pgfpathmoveto{\pgfqpoint{3.470436in}{1.715488in}}%
\pgfpathlineto{\pgfqpoint{3.484220in}{1.712269in}}%
\pgfpathlineto{\pgfqpoint{3.498010in}{1.709130in}}%
\pgfpathlineto{\pgfqpoint{3.511806in}{1.706071in}}%
\pgfpathlineto{\pgfqpoint{3.525608in}{1.703092in}}%
\pgfpathlineto{\pgfqpoint{3.533855in}{1.712248in}}%
\pgfpathlineto{\pgfqpoint{3.542096in}{1.721425in}}%
\pgfpathlineto{\pgfqpoint{3.550330in}{1.730623in}}%
\pgfpathlineto{\pgfqpoint{3.558558in}{1.739837in}}%
\pgfpathlineto{\pgfqpoint{3.544769in}{1.742632in}}%
\pgfpathlineto{\pgfqpoint{3.530986in}{1.745507in}}%
\pgfpathlineto{\pgfqpoint{3.517210in}{1.748462in}}%
\pgfpathlineto{\pgfqpoint{3.503439in}{1.751497in}}%
\pgfpathlineto{\pgfqpoint{3.495198in}{1.742459in}}%
\pgfpathlineto{\pgfqpoint{3.486950in}{1.733443in}}%
\pgfpathlineto{\pgfqpoint{3.478696in}{1.724452in}}%
\pgfpathlineto{\pgfqpoint{3.470436in}{1.715488in}}%
\pgfpathclose%
\pgfusepath{fill}%
\end{pgfscope}%
\begin{pgfscope}%
\pgfpathrectangle{\pgfqpoint{1.150000in}{0.150000in}}{\pgfqpoint{5.700000in}{5.700000in}}%
\pgfusepath{clip}%
\pgfsetbuttcap%
\pgfsetroundjoin%
\definecolor{currentfill}{rgb}{0.278791,0.062145,0.386592}%
\pgfsetfillcolor{currentfill}%
\pgfsetfillopacity{0.700000}%
\pgfsetlinewidth{0.000000pt}%
\definecolor{currentstroke}{rgb}{0.000000,0.000000,0.000000}%
\pgfsetstrokecolor{currentstroke}%
\pgfsetdash{}{0pt}%
\pgfpathmoveto{\pgfqpoint{3.789858in}{1.790202in}}%
\pgfpathlineto{\pgfqpoint{3.803712in}{1.788733in}}%
\pgfpathlineto{\pgfqpoint{3.817573in}{1.787340in}}%
\pgfpathlineto{\pgfqpoint{3.831441in}{1.786025in}}%
\pgfpathlineto{\pgfqpoint{3.845317in}{1.784786in}}%
\pgfpathlineto{\pgfqpoint{3.853447in}{1.794579in}}%
\pgfpathlineto{\pgfqpoint{3.861570in}{1.804348in}}%
\pgfpathlineto{\pgfqpoint{3.869689in}{1.814092in}}%
\pgfpathlineto{\pgfqpoint{3.877801in}{1.823810in}}%
\pgfpathlineto{\pgfqpoint{3.863936in}{1.824926in}}%
\pgfpathlineto{\pgfqpoint{3.850078in}{1.826119in}}%
\pgfpathlineto{\pgfqpoint{3.836228in}{1.827389in}}%
\pgfpathlineto{\pgfqpoint{3.822385in}{1.828735in}}%
\pgfpathlineto{\pgfqpoint{3.814262in}{1.819133in}}%
\pgfpathlineto{\pgfqpoint{3.806133in}{1.809508in}}%
\pgfpathlineto{\pgfqpoint{3.797998in}{1.799864in}}%
\pgfpathlineto{\pgfqpoint{3.789858in}{1.790202in}}%
\pgfpathclose%
\pgfusepath{fill}%
\end{pgfscope}%
\begin{pgfscope}%
\pgfpathrectangle{\pgfqpoint{1.150000in}{0.150000in}}{\pgfqpoint{5.700000in}{5.700000in}}%
\pgfusepath{clip}%
\pgfsetbuttcap%
\pgfsetroundjoin%
\definecolor{currentfill}{rgb}{0.275191,0.194905,0.496005}%
\pgfsetfillcolor{currentfill}%
\pgfsetfillopacity{0.700000}%
\pgfsetlinewidth{0.000000pt}%
\definecolor{currentstroke}{rgb}{0.000000,0.000000,0.000000}%
\pgfsetstrokecolor{currentstroke}%
\pgfsetdash{}{0pt}%
\pgfpathmoveto{\pgfqpoint{4.460812in}{2.054330in}}%
\pgfpathlineto{\pgfqpoint{4.474874in}{2.055667in}}%
\pgfpathlineto{\pgfqpoint{4.488946in}{2.057077in}}%
\pgfpathlineto{\pgfqpoint{4.503029in}{2.058559in}}%
\pgfpathlineto{\pgfqpoint{4.517121in}{2.060113in}}%
\pgfpathlineto{\pgfqpoint{4.525010in}{2.068818in}}%
\pgfpathlineto{\pgfqpoint{4.532893in}{2.077447in}}%
\pgfpathlineto{\pgfqpoint{4.540769in}{2.086000in}}%
\pgfpathlineto{\pgfqpoint{4.548639in}{2.094478in}}%
\pgfpathlineto{\pgfqpoint{4.534557in}{2.092947in}}%
\pgfpathlineto{\pgfqpoint{4.520485in}{2.091489in}}%
\pgfpathlineto{\pgfqpoint{4.506423in}{2.090102in}}%
\pgfpathlineto{\pgfqpoint{4.492371in}{2.088789in}}%
\pgfpathlineto{\pgfqpoint{4.484491in}{2.080280in}}%
\pgfpathlineto{\pgfqpoint{4.476604in}{2.071701in}}%
\pgfpathlineto{\pgfqpoint{4.468711in}{2.063051in}}%
\pgfpathlineto{\pgfqpoint{4.460812in}{2.054330in}}%
\pgfpathclose%
\pgfusepath{fill}%
\end{pgfscope}%
\begin{pgfscope}%
\pgfpathrectangle{\pgfqpoint{1.150000in}{0.150000in}}{\pgfqpoint{5.700000in}{5.700000in}}%
\pgfusepath{clip}%
\pgfsetbuttcap%
\pgfsetroundjoin%
\definecolor{currentfill}{rgb}{0.267004,0.004874,0.329415}%
\pgfsetfillcolor{currentfill}%
\pgfsetfillopacity{0.700000}%
\pgfsetlinewidth{0.000000pt}%
\definecolor{currentstroke}{rgb}{0.000000,0.000000,0.000000}%
\pgfsetstrokecolor{currentstroke}%
\pgfsetdash{}{0pt}%
\pgfpathmoveto{\pgfqpoint{3.238766in}{1.695282in}}%
\pgfpathlineto{\pgfqpoint{3.252518in}{1.690642in}}%
\pgfpathlineto{\pgfqpoint{3.266274in}{1.686087in}}%
\pgfpathlineto{\pgfqpoint{3.280036in}{1.681615in}}%
\pgfpathlineto{\pgfqpoint{3.293802in}{1.677227in}}%
\pgfpathlineto{\pgfqpoint{3.302147in}{1.685404in}}%
\pgfpathlineto{\pgfqpoint{3.310486in}{1.693643in}}%
\pgfpathlineto{\pgfqpoint{3.318817in}{1.701939in}}%
\pgfpathlineto{\pgfqpoint{3.327141in}{1.710290in}}%
\pgfpathlineto{\pgfqpoint{3.313391in}{1.714453in}}%
\pgfpathlineto{\pgfqpoint{3.299646in}{1.718700in}}%
\pgfpathlineto{\pgfqpoint{3.285906in}{1.723030in}}%
\pgfpathlineto{\pgfqpoint{3.272171in}{1.727445in}}%
\pgfpathlineto{\pgfqpoint{3.263831in}{1.719311in}}%
\pgfpathlineto{\pgfqpoint{3.255483in}{1.711238in}}%
\pgfpathlineto{\pgfqpoint{3.247128in}{1.703227in}}%
\pgfpathlineto{\pgfqpoint{3.238766in}{1.695282in}}%
\pgfpathclose%
\pgfusepath{fill}%
\end{pgfscope}%
\begin{pgfscope}%
\pgfpathrectangle{\pgfqpoint{1.150000in}{0.150000in}}{\pgfqpoint{5.700000in}{5.700000in}}%
\pgfusepath{clip}%
\pgfsetbuttcap%
\pgfsetroundjoin%
\definecolor{currentfill}{rgb}{0.281887,0.150881,0.465405}%
\pgfsetfillcolor{currentfill}%
\pgfsetfillopacity{0.700000}%
\pgfsetlinewidth{0.000000pt}%
\definecolor{currentstroke}{rgb}{0.000000,0.000000,0.000000}%
\pgfsetstrokecolor{currentstroke}%
\pgfsetdash{}{0pt}%
\pgfpathmoveto{\pgfqpoint{2.352488in}{2.000288in}}%
\pgfpathlineto{\pgfqpoint{2.366267in}{1.989107in}}%
\pgfpathlineto{\pgfqpoint{2.380046in}{1.978042in}}%
\pgfpathlineto{\pgfqpoint{2.393823in}{1.967092in}}%
\pgfpathlineto{\pgfqpoint{2.407599in}{1.956256in}}%
\pgfpathlineto{\pgfqpoint{2.416482in}{1.957985in}}%
\pgfpathlineto{\pgfqpoint{2.425349in}{1.959928in}}%
\pgfpathlineto{\pgfqpoint{2.434201in}{1.962080in}}%
\pgfpathlineto{\pgfqpoint{2.443038in}{1.964435in}}%
\pgfpathlineto{\pgfqpoint{2.429295in}{1.974934in}}%
\pgfpathlineto{\pgfqpoint{2.415552in}{1.985548in}}%
\pgfpathlineto{\pgfqpoint{2.401808in}{1.996276in}}%
\pgfpathlineto{\pgfqpoint{2.388063in}{2.007120in}}%
\pgfpathlineto{\pgfqpoint{2.379193in}{2.005093in}}%
\pgfpathlineto{\pgfqpoint{2.370307in}{2.003275in}}%
\pgfpathlineto{\pgfqpoint{2.361406in}{2.001672in}}%
\pgfpathlineto{\pgfqpoint{2.352488in}{2.000288in}}%
\pgfpathclose%
\pgfusepath{fill}%
\end{pgfscope}%
\begin{pgfscope}%
\pgfpathrectangle{\pgfqpoint{1.150000in}{0.150000in}}{\pgfqpoint{5.700000in}{5.700000in}}%
\pgfusepath{clip}%
\pgfsetbuttcap%
\pgfsetroundjoin%
\definecolor{currentfill}{rgb}{0.223925,0.334994,0.548053}%
\pgfsetfillcolor{currentfill}%
\pgfsetfillopacity{0.700000}%
\pgfsetlinewidth{0.000000pt}%
\definecolor{currentstroke}{rgb}{0.000000,0.000000,0.000000}%
\pgfsetstrokecolor{currentstroke}%
\pgfsetdash{}{0pt}%
\pgfpathmoveto{\pgfqpoint{5.219721in}{2.370166in}}%
\pgfpathlineto{\pgfqpoint{5.234074in}{2.373275in}}%
\pgfpathlineto{\pgfqpoint{5.248439in}{2.376455in}}%
\pgfpathlineto{\pgfqpoint{5.262815in}{2.379704in}}%
\pgfpathlineto{\pgfqpoint{5.277204in}{2.383023in}}%
\pgfpathlineto{\pgfqpoint{5.284743in}{2.388438in}}%
\pgfpathlineto{\pgfqpoint{5.292275in}{2.393798in}}%
\pgfpathlineto{\pgfqpoint{5.299799in}{2.399106in}}%
\pgfpathlineto{\pgfqpoint{5.307315in}{2.404367in}}%
\pgfpathlineto{\pgfqpoint{5.292945in}{2.401243in}}%
\pgfpathlineto{\pgfqpoint{5.278586in}{2.398188in}}%
\pgfpathlineto{\pgfqpoint{5.264240in}{2.395203in}}%
\pgfpathlineto{\pgfqpoint{5.249905in}{2.392288in}}%
\pgfpathlineto{\pgfqpoint{5.242370in}{2.386825in}}%
\pgfpathlineto{\pgfqpoint{5.234828in}{2.381320in}}%
\pgfpathlineto{\pgfqpoint{5.227278in}{2.375768in}}%
\pgfpathlineto{\pgfqpoint{5.219721in}{2.370166in}}%
\pgfpathclose%
\pgfusepath{fill}%
\end{pgfscope}%
\begin{pgfscope}%
\pgfpathrectangle{\pgfqpoint{1.150000in}{0.150000in}}{\pgfqpoint{5.700000in}{5.700000in}}%
\pgfusepath{clip}%
\pgfsetbuttcap%
\pgfsetroundjoin%
\definecolor{currentfill}{rgb}{0.280894,0.078907,0.402329}%
\pgfsetfillcolor{currentfill}%
\pgfsetfillopacity{0.700000}%
\pgfsetlinewidth{0.000000pt}%
\definecolor{currentstroke}{rgb}{0.000000,0.000000,0.000000}%
\pgfsetstrokecolor{currentstroke}%
\pgfsetdash{}{0pt}%
\pgfpathmoveto{\pgfqpoint{2.607931in}{1.847005in}}%
\pgfpathlineto{\pgfqpoint{2.621674in}{1.837910in}}%
\pgfpathlineto{\pgfqpoint{2.635418in}{1.828918in}}%
\pgfpathlineto{\pgfqpoint{2.649162in}{1.820028in}}%
\pgfpathlineto{\pgfqpoint{2.662908in}{1.811239in}}%
\pgfpathlineto{\pgfqpoint{2.671609in}{1.815075in}}%
\pgfpathlineto{\pgfqpoint{2.680297in}{1.819084in}}%
\pgfpathlineto{\pgfqpoint{2.688972in}{1.823259in}}%
\pgfpathlineto{\pgfqpoint{2.697635in}{1.827598in}}%
\pgfpathlineto{\pgfqpoint{2.683917in}{1.836075in}}%
\pgfpathlineto{\pgfqpoint{2.670201in}{1.844654in}}%
\pgfpathlineto{\pgfqpoint{2.656486in}{1.853334in}}%
\pgfpathlineto{\pgfqpoint{2.642772in}{1.862118in}}%
\pgfpathlineto{\pgfqpoint{2.634081in}{1.858083in}}%
\pgfpathlineto{\pgfqpoint{2.625377in}{1.854216in}}%
\pgfpathlineto{\pgfqpoint{2.616661in}{1.850522in}}%
\pgfpathlineto{\pgfqpoint{2.607931in}{1.847005in}}%
\pgfpathclose%
\pgfusepath{fill}%
\end{pgfscope}%
\begin{pgfscope}%
\pgfpathrectangle{\pgfqpoint{1.150000in}{0.150000in}}{\pgfqpoint{5.700000in}{5.700000in}}%
\pgfusepath{clip}%
\pgfsetbuttcap%
\pgfsetroundjoin%
\definecolor{currentfill}{rgb}{0.274952,0.037752,0.364543}%
\pgfsetfillcolor{currentfill}%
\pgfsetfillopacity{0.700000}%
\pgfsetlinewidth{0.000000pt}%
\definecolor{currentstroke}{rgb}{0.000000,0.000000,0.000000}%
\pgfsetstrokecolor{currentstroke}%
\pgfsetdash{}{0pt}%
\pgfpathmoveto{\pgfqpoint{2.807430in}{1.763353in}}%
\pgfpathlineto{\pgfqpoint{2.821163in}{1.755760in}}%
\pgfpathlineto{\pgfqpoint{2.834898in}{1.748263in}}%
\pgfpathlineto{\pgfqpoint{2.848635in}{1.740860in}}%
\pgfpathlineto{\pgfqpoint{2.862374in}{1.733551in}}%
\pgfpathlineto{\pgfqpoint{2.870949in}{1.738939in}}%
\pgfpathlineto{\pgfqpoint{2.879512in}{1.744465in}}%
\pgfpathlineto{\pgfqpoint{2.888065in}{1.750124in}}%
\pgfpathlineto{\pgfqpoint{2.896607in}{1.755913in}}%
\pgfpathlineto{\pgfqpoint{2.882892in}{1.762933in}}%
\pgfpathlineto{\pgfqpoint{2.869179in}{1.770047in}}%
\pgfpathlineto{\pgfqpoint{2.855469in}{1.777256in}}%
\pgfpathlineto{\pgfqpoint{2.841760in}{1.784561in}}%
\pgfpathlineto{\pgfqpoint{2.833194in}{1.779052in}}%
\pgfpathlineto{\pgfqpoint{2.824617in}{1.773679in}}%
\pgfpathlineto{\pgfqpoint{2.816029in}{1.768444in}}%
\pgfpathlineto{\pgfqpoint{2.807430in}{1.763353in}}%
\pgfpathclose%
\pgfusepath{fill}%
\end{pgfscope}%
\begin{pgfscope}%
\pgfpathrectangle{\pgfqpoint{1.150000in}{0.150000in}}{\pgfqpoint{5.700000in}{5.700000in}}%
\pgfusepath{clip}%
\pgfsetbuttcap%
\pgfsetroundjoin%
\definecolor{currentfill}{rgb}{0.278012,0.180367,0.486697}%
\pgfsetfillcolor{currentfill}%
\pgfsetfillopacity{0.700000}%
\pgfsetlinewidth{0.000000pt}%
\definecolor{currentstroke}{rgb}{0.000000,0.000000,0.000000}%
\pgfsetstrokecolor{currentstroke}%
\pgfsetdash{}{0pt}%
\pgfpathmoveto{\pgfqpoint{4.372961in}{2.014088in}}%
\pgfpathlineto{\pgfqpoint{4.386994in}{2.015136in}}%
\pgfpathlineto{\pgfqpoint{4.401037in}{2.016258in}}%
\pgfpathlineto{\pgfqpoint{4.415090in}{2.017452in}}%
\pgfpathlineto{\pgfqpoint{4.429152in}{2.018718in}}%
\pgfpathlineto{\pgfqpoint{4.437077in}{2.027732in}}%
\pgfpathlineto{\pgfqpoint{4.444995in}{2.036671in}}%
\pgfpathlineto{\pgfqpoint{4.452906in}{2.045537in}}%
\pgfpathlineto{\pgfqpoint{4.460812in}{2.054330in}}%
\pgfpathlineto{\pgfqpoint{4.446759in}{2.053066in}}%
\pgfpathlineto{\pgfqpoint{4.432717in}{2.051874in}}%
\pgfpathlineto{\pgfqpoint{4.418684in}{2.050755in}}%
\pgfpathlineto{\pgfqpoint{4.404660in}{2.049709in}}%
\pgfpathlineto{\pgfqpoint{4.396745in}{2.040905in}}%
\pgfpathlineto{\pgfqpoint{4.388823in}{2.032035in}}%
\pgfpathlineto{\pgfqpoint{4.380895in}{2.023096in}}%
\pgfpathlineto{\pgfqpoint{4.372961in}{2.014088in}}%
\pgfpathclose%
\pgfusepath{fill}%
\end{pgfscope}%
\begin{pgfscope}%
\pgfpathrectangle{\pgfqpoint{1.150000in}{0.150000in}}{\pgfqpoint{5.700000in}{5.700000in}}%
\pgfusepath{clip}%
\pgfsetbuttcap%
\pgfsetroundjoin%
\definecolor{currentfill}{rgb}{0.277018,0.050344,0.375715}%
\pgfsetfillcolor{currentfill}%
\pgfsetfillopacity{0.700000}%
\pgfsetlinewidth{0.000000pt}%
\definecolor{currentstroke}{rgb}{0.000000,0.000000,0.000000}%
\pgfsetstrokecolor{currentstroke}%
\pgfsetdash{}{0pt}%
\pgfpathmoveto{\pgfqpoint{3.701855in}{1.758624in}}%
\pgfpathlineto{\pgfqpoint{3.715691in}{1.756702in}}%
\pgfpathlineto{\pgfqpoint{3.729534in}{1.754858in}}%
\pgfpathlineto{\pgfqpoint{3.743384in}{1.753092in}}%
\pgfpathlineto{\pgfqpoint{3.757241in}{1.751402in}}%
\pgfpathlineto{\pgfqpoint{3.765404in}{1.761121in}}%
\pgfpathlineto{\pgfqpoint{3.773561in}{1.770828in}}%
\pgfpathlineto{\pgfqpoint{3.781712in}{1.780523in}}%
\pgfpathlineto{\pgfqpoint{3.789858in}{1.790202in}}%
\pgfpathlineto{\pgfqpoint{3.776012in}{1.791748in}}%
\pgfpathlineto{\pgfqpoint{3.762173in}{1.793371in}}%
\pgfpathlineto{\pgfqpoint{3.748341in}{1.795072in}}%
\pgfpathlineto{\pgfqpoint{3.734516in}{1.796851in}}%
\pgfpathlineto{\pgfqpoint{3.726360in}{1.787307in}}%
\pgfpathlineto{\pgfqpoint{3.718197in}{1.777753in}}%
\pgfpathlineto{\pgfqpoint{3.710029in}{1.768192in}}%
\pgfpathlineto{\pgfqpoint{3.701855in}{1.758624in}}%
\pgfpathclose%
\pgfusepath{fill}%
\end{pgfscope}%
\begin{pgfscope}%
\pgfpathrectangle{\pgfqpoint{1.150000in}{0.150000in}}{\pgfqpoint{5.700000in}{5.700000in}}%
\pgfusepath{clip}%
\pgfsetbuttcap%
\pgfsetroundjoin%
\definecolor{currentfill}{rgb}{0.229739,0.322361,0.545706}%
\pgfsetfillcolor{currentfill}%
\pgfsetfillopacity{0.700000}%
\pgfsetlinewidth{0.000000pt}%
\definecolor{currentstroke}{rgb}{0.000000,0.000000,0.000000}%
\pgfsetstrokecolor{currentstroke}%
\pgfsetdash{}{0pt}%
\pgfpathmoveto{\pgfqpoint{5.132061in}{2.334757in}}%
\pgfpathlineto{\pgfqpoint{5.146383in}{2.337760in}}%
\pgfpathlineto{\pgfqpoint{5.160716in}{2.340834in}}%
\pgfpathlineto{\pgfqpoint{5.175062in}{2.343977in}}%
\pgfpathlineto{\pgfqpoint{5.189419in}{2.347191in}}%
\pgfpathlineto{\pgfqpoint{5.197006in}{2.353026in}}%
\pgfpathlineto{\pgfqpoint{5.204586in}{2.358798in}}%
\pgfpathlineto{\pgfqpoint{5.212157in}{2.364510in}}%
\pgfpathlineto{\pgfqpoint{5.219721in}{2.370166in}}%
\pgfpathlineto{\pgfqpoint{5.205381in}{2.367126in}}%
\pgfpathlineto{\pgfqpoint{5.191052in}{2.364156in}}%
\pgfpathlineto{\pgfqpoint{5.176735in}{2.361256in}}%
\pgfpathlineto{\pgfqpoint{5.162430in}{2.358426in}}%
\pgfpathlineto{\pgfqpoint{5.154848in}{2.352589in}}%
\pgfpathlineto{\pgfqpoint{5.147260in}{2.346701in}}%
\pgfpathlineto{\pgfqpoint{5.139664in}{2.340758in}}%
\pgfpathlineto{\pgfqpoint{5.132061in}{2.334757in}}%
\pgfpathclose%
\pgfusepath{fill}%
\end{pgfscope}%
\begin{pgfscope}%
\pgfpathrectangle{\pgfqpoint{1.150000in}{0.150000in}}{\pgfqpoint{5.700000in}{5.700000in}}%
\pgfusepath{clip}%
\pgfsetbuttcap%
\pgfsetroundjoin%
\definecolor{currentfill}{rgb}{0.280868,0.160771,0.472899}%
\pgfsetfillcolor{currentfill}%
\pgfsetfillopacity{0.700000}%
\pgfsetlinewidth{0.000000pt}%
\definecolor{currentstroke}{rgb}{0.000000,0.000000,0.000000}%
\pgfsetstrokecolor{currentstroke}%
\pgfsetdash{}{0pt}%
\pgfpathmoveto{\pgfqpoint{4.285088in}{1.973976in}}%
\pgfpathlineto{\pgfqpoint{4.299093in}{1.974713in}}%
\pgfpathlineto{\pgfqpoint{4.313107in}{1.975524in}}%
\pgfpathlineto{\pgfqpoint{4.327131in}{1.976407in}}%
\pgfpathlineto{\pgfqpoint{4.341165in}{1.977364in}}%
\pgfpathlineto{\pgfqpoint{4.349123in}{1.986649in}}%
\pgfpathlineto{\pgfqpoint{4.357075in}{1.995865in}}%
\pgfpathlineto{\pgfqpoint{4.365021in}{2.005011in}}%
\pgfpathlineto{\pgfqpoint{4.372961in}{2.014088in}}%
\pgfpathlineto{\pgfqpoint{4.358937in}{2.013112in}}%
\pgfpathlineto{\pgfqpoint{4.344923in}{2.012210in}}%
\pgfpathlineto{\pgfqpoint{4.330918in}{2.011381in}}%
\pgfpathlineto{\pgfqpoint{4.316923in}{2.010625in}}%
\pgfpathlineto{\pgfqpoint{4.308973in}{2.001559in}}%
\pgfpathlineto{\pgfqpoint{4.301017in}{1.992429in}}%
\pgfpathlineto{\pgfqpoint{4.293055in}{1.983235in}}%
\pgfpathlineto{\pgfqpoint{4.285088in}{1.973976in}}%
\pgfpathclose%
\pgfusepath{fill}%
\end{pgfscope}%
\begin{pgfscope}%
\pgfpathrectangle{\pgfqpoint{1.150000in}{0.150000in}}{\pgfqpoint{5.700000in}{5.700000in}}%
\pgfusepath{clip}%
\pgfsetbuttcap%
\pgfsetroundjoin%
\definecolor{currentfill}{rgb}{0.197636,0.391528,0.554969}%
\pgfsetfillcolor{currentfill}%
\pgfsetfillopacity{0.700000}%
\pgfsetlinewidth{0.000000pt}%
\definecolor{currentstroke}{rgb}{0.000000,0.000000,0.000000}%
\pgfsetstrokecolor{currentstroke}%
\pgfsetdash{}{0pt}%
\pgfpathmoveto{\pgfqpoint{5.627583in}{2.513212in}}%
\pgfpathlineto{\pgfqpoint{5.642102in}{2.516798in}}%
\pgfpathlineto{\pgfqpoint{5.656634in}{2.520453in}}%
\pgfpathlineto{\pgfqpoint{5.671179in}{2.524176in}}%
\pgfpathlineto{\pgfqpoint{5.685736in}{2.527969in}}%
\pgfpathlineto{\pgfqpoint{5.693052in}{2.531618in}}%
\pgfpathlineto{\pgfqpoint{5.700360in}{2.535253in}}%
\pgfpathlineto{\pgfqpoint{5.707662in}{2.538881in}}%
\pgfpathlineto{\pgfqpoint{5.714957in}{2.542505in}}%
\pgfpathlineto{\pgfqpoint{5.700424in}{2.538994in}}%
\pgfpathlineto{\pgfqpoint{5.685903in}{2.535552in}}%
\pgfpathlineto{\pgfqpoint{5.671395in}{2.532178in}}%
\pgfpathlineto{\pgfqpoint{5.656900in}{2.528873in}}%
\pgfpathlineto{\pgfqpoint{5.649581in}{2.524960in}}%
\pgfpathlineto{\pgfqpoint{5.642255in}{2.521049in}}%
\pgfpathlineto{\pgfqpoint{5.634922in}{2.517135in}}%
\pgfpathlineto{\pgfqpoint{5.627583in}{2.513212in}}%
\pgfpathclose%
\pgfusepath{fill}%
\end{pgfscope}%
\begin{pgfscope}%
\pgfpathrectangle{\pgfqpoint{1.150000in}{0.150000in}}{\pgfqpoint{5.700000in}{5.700000in}}%
\pgfusepath{clip}%
\pgfsetbuttcap%
\pgfsetroundjoin%
\definecolor{currentfill}{rgb}{0.268510,0.009605,0.335427}%
\pgfsetfillcolor{currentfill}%
\pgfsetfillopacity{0.700000}%
\pgfsetlinewidth{0.000000pt}%
\definecolor{currentstroke}{rgb}{0.000000,0.000000,0.000000}%
\pgfsetstrokecolor{currentstroke}%
\pgfsetdash{}{0pt}%
\pgfpathmoveto{\pgfqpoint{3.382192in}{1.694469in}}%
\pgfpathlineto{\pgfqpoint{3.395968in}{1.690720in}}%
\pgfpathlineto{\pgfqpoint{3.409749in}{1.687052in}}%
\pgfpathlineto{\pgfqpoint{3.423537in}{1.683466in}}%
\pgfpathlineto{\pgfqpoint{3.437329in}{1.679961in}}%
\pgfpathlineto{\pgfqpoint{3.445616in}{1.688788in}}%
\pgfpathlineto{\pgfqpoint{3.453896in}{1.697653in}}%
\pgfpathlineto{\pgfqpoint{3.462169in}{1.706554in}}%
\pgfpathlineto{\pgfqpoint{3.470436in}{1.715488in}}%
\pgfpathlineto{\pgfqpoint{3.456658in}{1.718789in}}%
\pgfpathlineto{\pgfqpoint{3.442885in}{1.722170in}}%
\pgfpathlineto{\pgfqpoint{3.429118in}{1.725634in}}%
\pgfpathlineto{\pgfqpoint{3.415357in}{1.729178in}}%
\pgfpathlineto{\pgfqpoint{3.407076in}{1.720441in}}%
\pgfpathlineto{\pgfqpoint{3.398788in}{1.711742in}}%
\pgfpathlineto{\pgfqpoint{3.390493in}{1.703084in}}%
\pgfpathlineto{\pgfqpoint{3.382192in}{1.694469in}}%
\pgfpathclose%
\pgfusepath{fill}%
\end{pgfscope}%
\begin{pgfscope}%
\pgfpathrectangle{\pgfqpoint{1.150000in}{0.150000in}}{\pgfqpoint{5.700000in}{5.700000in}}%
\pgfusepath{clip}%
\pgfsetbuttcap%
\pgfsetroundjoin%
\definecolor{currentfill}{rgb}{0.282623,0.140926,0.457517}%
\pgfsetfillcolor{currentfill}%
\pgfsetfillopacity{0.700000}%
\pgfsetlinewidth{0.000000pt}%
\definecolor{currentstroke}{rgb}{0.000000,0.000000,0.000000}%
\pgfsetstrokecolor{currentstroke}%
\pgfsetdash{}{0pt}%
\pgfpathmoveto{\pgfqpoint{4.197192in}{1.934243in}}%
\pgfpathlineto{\pgfqpoint{4.211170in}{1.934646in}}%
\pgfpathlineto{\pgfqpoint{4.225157in}{1.935123in}}%
\pgfpathlineto{\pgfqpoint{4.239153in}{1.935673in}}%
\pgfpathlineto{\pgfqpoint{4.253158in}{1.936297in}}%
\pgfpathlineto{\pgfqpoint{4.261149in}{1.945813in}}%
\pgfpathlineto{\pgfqpoint{4.269135in}{1.955265in}}%
\pgfpathlineto{\pgfqpoint{4.277114in}{1.964653in}}%
\pgfpathlineto{\pgfqpoint{4.285088in}{1.973976in}}%
\pgfpathlineto{\pgfqpoint{4.271092in}{1.973312in}}%
\pgfpathlineto{\pgfqpoint{4.257105in}{1.972722in}}%
\pgfpathlineto{\pgfqpoint{4.243128in}{1.972206in}}%
\pgfpathlineto{\pgfqpoint{4.229159in}{1.971763in}}%
\pgfpathlineto{\pgfqpoint{4.221176in}{1.962472in}}%
\pgfpathlineto{\pgfqpoint{4.213187in}{1.953121in}}%
\pgfpathlineto{\pgfqpoint{4.205192in}{1.943711in}}%
\pgfpathlineto{\pgfqpoint{4.197192in}{1.934243in}}%
\pgfpathclose%
\pgfusepath{fill}%
\end{pgfscope}%
\begin{pgfscope}%
\pgfpathrectangle{\pgfqpoint{1.150000in}{0.150000in}}{\pgfqpoint{5.700000in}{5.700000in}}%
\pgfusepath{clip}%
\pgfsetbuttcap%
\pgfsetroundjoin%
\definecolor{currentfill}{rgb}{0.282884,0.135920,0.453427}%
\pgfsetfillcolor{currentfill}%
\pgfsetfillopacity{0.700000}%
\pgfsetlinewidth{0.000000pt}%
\definecolor{currentstroke}{rgb}{0.000000,0.000000,0.000000}%
\pgfsetstrokecolor{currentstroke}%
\pgfsetdash{}{0pt}%
\pgfpathmoveto{\pgfqpoint{2.407599in}{1.956256in}}%
\pgfpathlineto{\pgfqpoint{2.421374in}{1.945533in}}%
\pgfpathlineto{\pgfqpoint{2.435148in}{1.934922in}}%
\pgfpathlineto{\pgfqpoint{2.448922in}{1.924423in}}%
\pgfpathlineto{\pgfqpoint{2.462695in}{1.914034in}}%
\pgfpathlineto{\pgfqpoint{2.471545in}{1.916107in}}%
\pgfpathlineto{\pgfqpoint{2.480379in}{1.918388in}}%
\pgfpathlineto{\pgfqpoint{2.489198in}{1.920873in}}%
\pgfpathlineto{\pgfqpoint{2.498003in}{1.923555in}}%
\pgfpathlineto{\pgfqpoint{2.484262in}{1.933609in}}%
\pgfpathlineto{\pgfqpoint{2.470521in}{1.943772in}}%
\pgfpathlineto{\pgfqpoint{2.456780in}{1.954048in}}%
\pgfpathlineto{\pgfqpoint{2.443038in}{1.964435in}}%
\pgfpathlineto{\pgfqpoint{2.434201in}{1.962080in}}%
\pgfpathlineto{\pgfqpoint{2.425349in}{1.959928in}}%
\pgfpathlineto{\pgfqpoint{2.416482in}{1.957985in}}%
\pgfpathlineto{\pgfqpoint{2.407599in}{1.956256in}}%
\pgfpathclose%
\pgfusepath{fill}%
\end{pgfscope}%
\begin{pgfscope}%
\pgfpathrectangle{\pgfqpoint{1.150000in}{0.150000in}}{\pgfqpoint{5.700000in}{5.700000in}}%
\pgfusepath{clip}%
\pgfsetbuttcap%
\pgfsetroundjoin%
\definecolor{currentfill}{rgb}{0.243113,0.292092,0.538516}%
\pgfsetfillcolor{currentfill}%
\pgfsetfillopacity{0.700000}%
\pgfsetlinewidth{0.000000pt}%
\definecolor{currentstroke}{rgb}{0.000000,0.000000,0.000000}%
\pgfsetstrokecolor{currentstroke}%
\pgfsetdash{}{0pt}%
\pgfpathmoveto{\pgfqpoint{1.984281in}{2.311433in}}%
\pgfpathlineto{\pgfqpoint{1.998172in}{2.296796in}}%
\pgfpathlineto{\pgfqpoint{2.012059in}{2.282302in}}%
\pgfpathlineto{\pgfqpoint{2.025942in}{2.267949in}}%
\pgfpathlineto{\pgfqpoint{2.039819in}{2.253737in}}%
\pgfpathlineto{\pgfqpoint{2.049006in}{2.252332in}}%
\pgfpathlineto{\pgfqpoint{2.058171in}{2.251198in}}%
\pgfpathlineto{\pgfqpoint{2.067317in}{2.250327in}}%
\pgfpathlineto{\pgfqpoint{2.076443in}{2.249714in}}%
\pgfpathlineto{\pgfqpoint{2.062607in}{2.263560in}}%
\pgfpathlineto{\pgfqpoint{2.048767in}{2.277545in}}%
\pgfpathlineto{\pgfqpoint{2.034923in}{2.291671in}}%
\pgfpathlineto{\pgfqpoint{2.021074in}{2.305939in}}%
\pgfpathlineto{\pgfqpoint{2.011907in}{2.306911in}}%
\pgfpathlineto{\pgfqpoint{2.002719in}{2.308147in}}%
\pgfpathlineto{\pgfqpoint{1.993510in}{2.309652in}}%
\pgfpathlineto{\pgfqpoint{1.984281in}{2.311433in}}%
\pgfpathclose%
\pgfusepath{fill}%
\end{pgfscope}%
\begin{pgfscope}%
\pgfpathrectangle{\pgfqpoint{1.150000in}{0.150000in}}{\pgfqpoint{5.700000in}{5.700000in}}%
\pgfusepath{clip}%
\pgfsetbuttcap%
\pgfsetroundjoin%
\definecolor{currentfill}{rgb}{0.235526,0.309527,0.542944}%
\pgfsetfillcolor{currentfill}%
\pgfsetfillopacity{0.700000}%
\pgfsetlinewidth{0.000000pt}%
\definecolor{currentstroke}{rgb}{0.000000,0.000000,0.000000}%
\pgfsetstrokecolor{currentstroke}%
\pgfsetdash{}{0pt}%
\pgfpathmoveto{\pgfqpoint{5.044340in}{2.298191in}}%
\pgfpathlineto{\pgfqpoint{5.058631in}{2.301065in}}%
\pgfpathlineto{\pgfqpoint{5.072933in}{2.304010in}}%
\pgfpathlineto{\pgfqpoint{5.087247in}{2.307025in}}%
\pgfpathlineto{\pgfqpoint{5.101573in}{2.310110in}}%
\pgfpathlineto{\pgfqpoint{5.109206in}{2.316374in}}%
\pgfpathlineto{\pgfqpoint{5.116832in}{2.322568in}}%
\pgfpathlineto{\pgfqpoint{5.124450in}{2.328695in}}%
\pgfpathlineto{\pgfqpoint{5.132061in}{2.334757in}}%
\pgfpathlineto{\pgfqpoint{5.117751in}{2.331824in}}%
\pgfpathlineto{\pgfqpoint{5.103452in}{2.328961in}}%
\pgfpathlineto{\pgfqpoint{5.089165in}{2.326168in}}%
\pgfpathlineto{\pgfqpoint{5.074890in}{2.323446in}}%
\pgfpathlineto{\pgfqpoint{5.067263in}{2.317224in}}%
\pgfpathlineto{\pgfqpoint{5.059630in}{2.310942in}}%
\pgfpathlineto{\pgfqpoint{5.051988in}{2.304599in}}%
\pgfpathlineto{\pgfqpoint{5.044340in}{2.298191in}}%
\pgfpathclose%
\pgfusepath{fill}%
\end{pgfscope}%
\begin{pgfscope}%
\pgfpathrectangle{\pgfqpoint{1.150000in}{0.150000in}}{\pgfqpoint{5.700000in}{5.700000in}}%
\pgfusepath{clip}%
\pgfsetbuttcap%
\pgfsetroundjoin%
\definecolor{currentfill}{rgb}{0.283187,0.125848,0.444960}%
\pgfsetfillcolor{currentfill}%
\pgfsetfillopacity{0.700000}%
\pgfsetlinewidth{0.000000pt}%
\definecolor{currentstroke}{rgb}{0.000000,0.000000,0.000000}%
\pgfsetstrokecolor{currentstroke}%
\pgfsetdash{}{0pt}%
\pgfpathmoveto{\pgfqpoint{4.109272in}{1.895156in}}%
\pgfpathlineto{\pgfqpoint{4.123224in}{1.895202in}}%
\pgfpathlineto{\pgfqpoint{4.137184in}{1.895323in}}%
\pgfpathlineto{\pgfqpoint{4.151154in}{1.895517in}}%
\pgfpathlineto{\pgfqpoint{4.165132in}{1.895786in}}%
\pgfpathlineto{\pgfqpoint{4.173156in}{1.905486in}}%
\pgfpathlineto{\pgfqpoint{4.181173in}{1.915130in}}%
\pgfpathlineto{\pgfqpoint{4.189186in}{1.924715in}}%
\pgfpathlineto{\pgfqpoint{4.197192in}{1.934243in}}%
\pgfpathlineto{\pgfqpoint{4.183223in}{1.933914in}}%
\pgfpathlineto{\pgfqpoint{4.169263in}{1.933658in}}%
\pgfpathlineto{\pgfqpoint{4.155312in}{1.933477in}}%
\pgfpathlineto{\pgfqpoint{4.141369in}{1.933371in}}%
\pgfpathlineto{\pgfqpoint{4.133353in}{1.923896in}}%
\pgfpathlineto{\pgfqpoint{4.125332in}{1.914369in}}%
\pgfpathlineto{\pgfqpoint{4.117304in}{1.904788in}}%
\pgfpathlineto{\pgfqpoint{4.109272in}{1.895156in}}%
\pgfpathclose%
\pgfusepath{fill}%
\end{pgfscope}%
\begin{pgfscope}%
\pgfpathrectangle{\pgfqpoint{1.150000in}{0.150000in}}{\pgfqpoint{5.700000in}{5.700000in}}%
\pgfusepath{clip}%
\pgfsetbuttcap%
\pgfsetroundjoin%
\definecolor{currentfill}{rgb}{0.273809,0.031497,0.358853}%
\pgfsetfillcolor{currentfill}%
\pgfsetfillopacity{0.700000}%
\pgfsetlinewidth{0.000000pt}%
\definecolor{currentstroke}{rgb}{0.000000,0.000000,0.000000}%
\pgfsetstrokecolor{currentstroke}%
\pgfsetdash{}{0pt}%
\pgfpathmoveto{\pgfqpoint{3.613777in}{1.729454in}}%
\pgfpathlineto{\pgfqpoint{3.627598in}{1.727056in}}%
\pgfpathlineto{\pgfqpoint{3.641425in}{1.724736in}}%
\pgfpathlineto{\pgfqpoint{3.655259in}{1.722495in}}%
\pgfpathlineto{\pgfqpoint{3.669100in}{1.720332in}}%
\pgfpathlineto{\pgfqpoint{3.677298in}{1.729904in}}%
\pgfpathlineto{\pgfqpoint{3.685489in}{1.739478in}}%
\pgfpathlineto{\pgfqpoint{3.693675in}{1.749052in}}%
\pgfpathlineto{\pgfqpoint{3.701855in}{1.758624in}}%
\pgfpathlineto{\pgfqpoint{3.688026in}{1.760624in}}%
\pgfpathlineto{\pgfqpoint{3.674203in}{1.762701in}}%
\pgfpathlineto{\pgfqpoint{3.660388in}{1.764857in}}%
\pgfpathlineto{\pgfqpoint{3.646579in}{1.767092in}}%
\pgfpathlineto{\pgfqpoint{3.638387in}{1.757676in}}%
\pgfpathlineto{\pgfqpoint{3.630190in}{1.748263in}}%
\pgfpathlineto{\pgfqpoint{3.621986in}{1.738855in}}%
\pgfpathlineto{\pgfqpoint{3.613777in}{1.729454in}}%
\pgfpathclose%
\pgfusepath{fill}%
\end{pgfscope}%
\begin{pgfscope}%
\pgfpathrectangle{\pgfqpoint{1.150000in}{0.150000in}}{\pgfqpoint{5.700000in}{5.700000in}}%
\pgfusepath{clip}%
\pgfsetbuttcap%
\pgfsetroundjoin%
\definecolor{currentfill}{rgb}{0.279566,0.067836,0.391917}%
\pgfsetfillcolor{currentfill}%
\pgfsetfillopacity{0.700000}%
\pgfsetlinewidth{0.000000pt}%
\definecolor{currentstroke}{rgb}{0.000000,0.000000,0.000000}%
\pgfsetstrokecolor{currentstroke}%
\pgfsetdash{}{0pt}%
\pgfpathmoveto{\pgfqpoint{2.662908in}{1.811239in}}%
\pgfpathlineto{\pgfqpoint{2.676655in}{1.802552in}}%
\pgfpathlineto{\pgfqpoint{2.690403in}{1.793964in}}%
\pgfpathlineto{\pgfqpoint{2.704152in}{1.785477in}}%
\pgfpathlineto{\pgfqpoint{2.717902in}{1.777088in}}%
\pgfpathlineto{\pgfqpoint{2.726575in}{1.781243in}}%
\pgfpathlineto{\pgfqpoint{2.735235in}{1.785564in}}%
\pgfpathlineto{\pgfqpoint{2.743883in}{1.790048in}}%
\pgfpathlineto{\pgfqpoint{2.752519in}{1.794690in}}%
\pgfpathlineto{\pgfqpoint{2.738796in}{1.802768in}}%
\pgfpathlineto{\pgfqpoint{2.725074in}{1.810945in}}%
\pgfpathlineto{\pgfqpoint{2.711354in}{1.819221in}}%
\pgfpathlineto{\pgfqpoint{2.697635in}{1.827598in}}%
\pgfpathlineto{\pgfqpoint{2.688972in}{1.823259in}}%
\pgfpathlineto{\pgfqpoint{2.680297in}{1.819084in}}%
\pgfpathlineto{\pgfqpoint{2.671609in}{1.815075in}}%
\pgfpathlineto{\pgfqpoint{2.662908in}{1.811239in}}%
\pgfpathclose%
\pgfusepath{fill}%
\end{pgfscope}%
\begin{pgfscope}%
\pgfpathrectangle{\pgfqpoint{1.150000in}{0.150000in}}{\pgfqpoint{5.700000in}{5.700000in}}%
\pgfusepath{clip}%
\pgfsetbuttcap%
\pgfsetroundjoin%
\definecolor{currentfill}{rgb}{0.269944,0.014625,0.341379}%
\pgfsetfillcolor{currentfill}%
\pgfsetfillopacity{0.700000}%
\pgfsetlinewidth{0.000000pt}%
\definecolor{currentstroke}{rgb}{0.000000,0.000000,0.000000}%
\pgfsetstrokecolor{currentstroke}%
\pgfsetdash{}{0pt}%
\pgfpathmoveto{\pgfqpoint{3.006433in}{1.703079in}}%
\pgfpathlineto{\pgfqpoint{3.020175in}{1.696884in}}%
\pgfpathlineto{\pgfqpoint{3.033921in}{1.690778in}}%
\pgfpathlineto{\pgfqpoint{3.047670in}{1.684760in}}%
\pgfpathlineto{\pgfqpoint{3.061422in}{1.678831in}}%
\pgfpathlineto{\pgfqpoint{3.069889in}{1.685570in}}%
\pgfpathlineto{\pgfqpoint{3.078347in}{1.692414in}}%
\pgfpathlineto{\pgfqpoint{3.086796in}{1.699360in}}%
\pgfpathlineto{\pgfqpoint{3.095236in}{1.706404in}}%
\pgfpathlineto{\pgfqpoint{3.081503in}{1.712067in}}%
\pgfpathlineto{\pgfqpoint{3.067775in}{1.717817in}}%
\pgfpathlineto{\pgfqpoint{3.054050in}{1.723657in}}%
\pgfpathlineto{\pgfqpoint{3.040329in}{1.729586in}}%
\pgfpathlineto{\pgfqpoint{3.031869in}{1.722801in}}%
\pgfpathlineto{\pgfqpoint{3.023399in}{1.716119in}}%
\pgfpathlineto{\pgfqpoint{3.014921in}{1.709544in}}%
\pgfpathlineto{\pgfqpoint{3.006433in}{1.703079in}}%
\pgfpathclose%
\pgfusepath{fill}%
\end{pgfscope}%
\begin{pgfscope}%
\pgfpathrectangle{\pgfqpoint{1.150000in}{0.150000in}}{\pgfqpoint{5.700000in}{5.700000in}}%
\pgfusepath{clip}%
\pgfsetbuttcap%
\pgfsetroundjoin%
\definecolor{currentfill}{rgb}{0.252194,0.269783,0.531579}%
\pgfsetfillcolor{currentfill}%
\pgfsetfillopacity{0.700000}%
\pgfsetlinewidth{0.000000pt}%
\definecolor{currentstroke}{rgb}{0.000000,0.000000,0.000000}%
\pgfsetstrokecolor{currentstroke}%
\pgfsetdash{}{0pt}%
\pgfpathmoveto{\pgfqpoint{2.039819in}{2.253737in}}%
\pgfpathlineto{\pgfqpoint{2.053693in}{2.239663in}}%
\pgfpathlineto{\pgfqpoint{2.067562in}{2.225726in}}%
\pgfpathlineto{\pgfqpoint{2.081427in}{2.211926in}}%
\pgfpathlineto{\pgfqpoint{2.095288in}{2.198261in}}%
\pgfpathlineto{\pgfqpoint{2.104432in}{2.197231in}}%
\pgfpathlineto{\pgfqpoint{2.113556in}{2.196466in}}%
\pgfpathlineto{\pgfqpoint{2.122660in}{2.195958in}}%
\pgfpathlineto{\pgfqpoint{2.131746in}{2.195703in}}%
\pgfpathlineto{\pgfqpoint{2.117926in}{2.209003in}}%
\pgfpathlineto{\pgfqpoint{2.104102in}{2.222437in}}%
\pgfpathlineto{\pgfqpoint{2.090274in}{2.236007in}}%
\pgfpathlineto{\pgfqpoint{2.076443in}{2.249714in}}%
\pgfpathlineto{\pgfqpoint{2.067317in}{2.250327in}}%
\pgfpathlineto{\pgfqpoint{2.058171in}{2.251198in}}%
\pgfpathlineto{\pgfqpoint{2.049006in}{2.252332in}}%
\pgfpathlineto{\pgfqpoint{2.039819in}{2.253737in}}%
\pgfpathclose%
\pgfusepath{fill}%
\end{pgfscope}%
\begin{pgfscope}%
\pgfpathrectangle{\pgfqpoint{1.150000in}{0.150000in}}{\pgfqpoint{5.700000in}{5.700000in}}%
\pgfusepath{clip}%
\pgfsetbuttcap%
\pgfsetroundjoin%
\definecolor{currentfill}{rgb}{0.243113,0.292092,0.538516}%
\pgfsetfillcolor{currentfill}%
\pgfsetfillopacity{0.700000}%
\pgfsetlinewidth{0.000000pt}%
\definecolor{currentstroke}{rgb}{0.000000,0.000000,0.000000}%
\pgfsetstrokecolor{currentstroke}%
\pgfsetdash{}{0pt}%
\pgfpathmoveto{\pgfqpoint{4.956566in}{2.260540in}}%
\pgfpathlineto{\pgfqpoint{4.970825in}{2.263263in}}%
\pgfpathlineto{\pgfqpoint{4.985096in}{2.266057in}}%
\pgfpathlineto{\pgfqpoint{4.999378in}{2.268921in}}%
\pgfpathlineto{\pgfqpoint{5.013672in}{2.271855in}}%
\pgfpathlineto{\pgfqpoint{5.021350in}{2.278550in}}%
\pgfpathlineto{\pgfqpoint{5.029021in}{2.285169in}}%
\pgfpathlineto{\pgfqpoint{5.036684in}{2.291715in}}%
\pgfpathlineto{\pgfqpoint{5.044340in}{2.298191in}}%
\pgfpathlineto{\pgfqpoint{5.030061in}{2.295387in}}%
\pgfpathlineto{\pgfqpoint{5.015793in}{2.292654in}}%
\pgfpathlineto{\pgfqpoint{5.001537in}{2.289991in}}%
\pgfpathlineto{\pgfqpoint{4.987292in}{2.287398in}}%
\pgfpathlineto{\pgfqpoint{4.979621in}{2.280784in}}%
\pgfpathlineto{\pgfqpoint{4.971943in}{2.274105in}}%
\pgfpathlineto{\pgfqpoint{4.964258in}{2.267358in}}%
\pgfpathlineto{\pgfqpoint{4.956566in}{2.260540in}}%
\pgfpathclose%
\pgfusepath{fill}%
\end{pgfscope}%
\begin{pgfscope}%
\pgfpathrectangle{\pgfqpoint{1.150000in}{0.150000in}}{\pgfqpoint{5.700000in}{5.700000in}}%
\pgfusepath{clip}%
\pgfsetbuttcap%
\pgfsetroundjoin%
\definecolor{currentfill}{rgb}{0.203063,0.379716,0.553925}%
\pgfsetfillcolor{currentfill}%
\pgfsetfillopacity{0.700000}%
\pgfsetlinewidth{0.000000pt}%
\definecolor{currentstroke}{rgb}{0.000000,0.000000,0.000000}%
\pgfsetstrokecolor{currentstroke}%
\pgfsetdash{}{0pt}%
\pgfpathmoveto{\pgfqpoint{5.540115in}{2.482646in}}%
\pgfpathlineto{\pgfqpoint{5.554606in}{2.486217in}}%
\pgfpathlineto{\pgfqpoint{5.569110in}{2.489856in}}%
\pgfpathlineto{\pgfqpoint{5.583626in}{2.493565in}}%
\pgfpathlineto{\pgfqpoint{5.598155in}{2.497342in}}%
\pgfpathlineto{\pgfqpoint{5.605523in}{2.501346in}}%
\pgfpathlineto{\pgfqpoint{5.612883in}{2.505322in}}%
\pgfpathlineto{\pgfqpoint{5.620237in}{2.509276in}}%
\pgfpathlineto{\pgfqpoint{5.627583in}{2.513212in}}%
\pgfpathlineto{\pgfqpoint{5.613077in}{2.509695in}}%
\pgfpathlineto{\pgfqpoint{5.598583in}{2.506246in}}%
\pgfpathlineto{\pgfqpoint{5.584102in}{2.502867in}}%
\pgfpathlineto{\pgfqpoint{5.569634in}{2.499556in}}%
\pgfpathlineto{\pgfqpoint{5.562264in}{2.495353in}}%
\pgfpathlineto{\pgfqpoint{5.554888in}{2.491136in}}%
\pgfpathlineto{\pgfqpoint{5.547505in}{2.486902in}}%
\pgfpathlineto{\pgfqpoint{5.540115in}{2.482646in}}%
\pgfpathclose%
\pgfusepath{fill}%
\end{pgfscope}%
\begin{pgfscope}%
\pgfpathrectangle{\pgfqpoint{1.150000in}{0.150000in}}{\pgfqpoint{5.700000in}{5.700000in}}%
\pgfusepath{clip}%
\pgfsetbuttcap%
\pgfsetroundjoin%
\definecolor{currentfill}{rgb}{0.267004,0.004874,0.329415}%
\pgfsetfillcolor{currentfill}%
\pgfsetfillopacity{0.700000}%
\pgfsetlinewidth{0.000000pt}%
\definecolor{currentstroke}{rgb}{0.000000,0.000000,0.000000}%
\pgfsetstrokecolor{currentstroke}%
\pgfsetdash{}{0pt}%
\pgfpathmoveto{\pgfqpoint{3.150204in}{1.684629in}}%
\pgfpathlineto{\pgfqpoint{3.163956in}{1.679402in}}%
\pgfpathlineto{\pgfqpoint{3.177712in}{1.674261in}}%
\pgfpathlineto{\pgfqpoint{3.191473in}{1.669206in}}%
\pgfpathlineto{\pgfqpoint{3.205238in}{1.664235in}}%
\pgfpathlineto{\pgfqpoint{3.213632in}{1.671880in}}%
\pgfpathlineto{\pgfqpoint{3.222018in}{1.679606in}}%
\pgfpathlineto{\pgfqpoint{3.230396in}{1.687408in}}%
\pgfpathlineto{\pgfqpoint{3.238766in}{1.695282in}}%
\pgfpathlineto{\pgfqpoint{3.225019in}{1.700007in}}%
\pgfpathlineto{\pgfqpoint{3.211276in}{1.704817in}}%
\pgfpathlineto{\pgfqpoint{3.197538in}{1.709712in}}%
\pgfpathlineto{\pgfqpoint{3.183804in}{1.714694in}}%
\pgfpathlineto{\pgfqpoint{3.175416in}{1.707057in}}%
\pgfpathlineto{\pgfqpoint{3.167020in}{1.699498in}}%
\pgfpathlineto{\pgfqpoint{3.158616in}{1.692021in}}%
\pgfpathlineto{\pgfqpoint{3.150204in}{1.684629in}}%
\pgfpathclose%
\pgfusepath{fill}%
\end{pgfscope}%
\begin{pgfscope}%
\pgfpathrectangle{\pgfqpoint{1.150000in}{0.150000in}}{\pgfqpoint{5.700000in}{5.700000in}}%
\pgfusepath{clip}%
\pgfsetbuttcap%
\pgfsetroundjoin%
\definecolor{currentfill}{rgb}{0.282910,0.105393,0.426902}%
\pgfsetfillcolor{currentfill}%
\pgfsetfillopacity{0.700000}%
\pgfsetlinewidth{0.000000pt}%
\definecolor{currentstroke}{rgb}{0.000000,0.000000,0.000000}%
\pgfsetstrokecolor{currentstroke}%
\pgfsetdash{}{0pt}%
\pgfpathmoveto{\pgfqpoint{4.021323in}{1.857007in}}%
\pgfpathlineto{\pgfqpoint{4.035251in}{1.856673in}}%
\pgfpathlineto{\pgfqpoint{4.049187in}{1.856414in}}%
\pgfpathlineto{\pgfqpoint{4.063131in}{1.856230in}}%
\pgfpathlineto{\pgfqpoint{4.077084in}{1.856120in}}%
\pgfpathlineto{\pgfqpoint{4.085139in}{1.865953in}}%
\pgfpathlineto{\pgfqpoint{4.093189in}{1.875738in}}%
\pgfpathlineto{\pgfqpoint{4.101233in}{1.885472in}}%
\pgfpathlineto{\pgfqpoint{4.109272in}{1.895156in}}%
\pgfpathlineto{\pgfqpoint{4.095328in}{1.895184in}}%
\pgfpathlineto{\pgfqpoint{4.081393in}{1.895287in}}%
\pgfpathlineto{\pgfqpoint{4.067467in}{1.895465in}}%
\pgfpathlineto{\pgfqpoint{4.053549in}{1.895717in}}%
\pgfpathlineto{\pgfqpoint{4.045501in}{1.886107in}}%
\pgfpathlineto{\pgfqpoint{4.037447in}{1.876451in}}%
\pgfpathlineto{\pgfqpoint{4.029388in}{1.866751in}}%
\pgfpathlineto{\pgfqpoint{4.021323in}{1.857007in}}%
\pgfpathclose%
\pgfusepath{fill}%
\end{pgfscope}%
\begin{pgfscope}%
\pgfpathrectangle{\pgfqpoint{1.150000in}{0.150000in}}{\pgfqpoint{5.700000in}{5.700000in}}%
\pgfusepath{clip}%
\pgfsetbuttcap%
\pgfsetroundjoin%
\definecolor{currentfill}{rgb}{0.273809,0.031497,0.358853}%
\pgfsetfillcolor{currentfill}%
\pgfsetfillopacity{0.700000}%
\pgfsetlinewidth{0.000000pt}%
\definecolor{currentstroke}{rgb}{0.000000,0.000000,0.000000}%
\pgfsetstrokecolor{currentstroke}%
\pgfsetdash{}{0pt}%
\pgfpathmoveto{\pgfqpoint{2.862374in}{1.733551in}}%
\pgfpathlineto{\pgfqpoint{2.876116in}{1.726336in}}%
\pgfpathlineto{\pgfqpoint{2.889861in}{1.719215in}}%
\pgfpathlineto{\pgfqpoint{2.903608in}{1.712186in}}%
\pgfpathlineto{\pgfqpoint{2.917357in}{1.705249in}}%
\pgfpathlineto{\pgfqpoint{2.925907in}{1.710933in}}%
\pgfpathlineto{\pgfqpoint{2.934447in}{1.716749in}}%
\pgfpathlineto{\pgfqpoint{2.942977in}{1.722694in}}%
\pgfpathlineto{\pgfqpoint{2.951496in}{1.728763in}}%
\pgfpathlineto{\pgfqpoint{2.937770in}{1.735412in}}%
\pgfpathlineto{\pgfqpoint{2.924046in}{1.742153in}}%
\pgfpathlineto{\pgfqpoint{2.910325in}{1.748986in}}%
\pgfpathlineto{\pgfqpoint{2.896607in}{1.755913in}}%
\pgfpathlineto{\pgfqpoint{2.888065in}{1.750124in}}%
\pgfpathlineto{\pgfqpoint{2.879512in}{1.744465in}}%
\pgfpathlineto{\pgfqpoint{2.870949in}{1.738939in}}%
\pgfpathlineto{\pgfqpoint{2.862374in}{1.733551in}}%
\pgfpathclose%
\pgfusepath{fill}%
\end{pgfscope}%
\begin{pgfscope}%
\pgfpathrectangle{\pgfqpoint{1.150000in}{0.150000in}}{\pgfqpoint{5.700000in}{5.700000in}}%
\pgfusepath{clip}%
\pgfsetbuttcap%
\pgfsetroundjoin%
\definecolor{currentfill}{rgb}{0.248629,0.278775,0.534556}%
\pgfsetfillcolor{currentfill}%
\pgfsetfillopacity{0.700000}%
\pgfsetlinewidth{0.000000pt}%
\definecolor{currentstroke}{rgb}{0.000000,0.000000,0.000000}%
\pgfsetstrokecolor{currentstroke}%
\pgfsetdash{}{0pt}%
\pgfpathmoveto{\pgfqpoint{4.868746in}{2.221900in}}%
\pgfpathlineto{\pgfqpoint{4.882974in}{2.224449in}}%
\pgfpathlineto{\pgfqpoint{4.897212in}{2.227068in}}%
\pgfpathlineto{\pgfqpoint{4.911463in}{2.229759in}}%
\pgfpathlineto{\pgfqpoint{4.925724in}{2.232520in}}%
\pgfpathlineto{\pgfqpoint{4.933446in}{2.239642in}}%
\pgfpathlineto{\pgfqpoint{4.941160in}{2.246685in}}%
\pgfpathlineto{\pgfqpoint{4.948867in}{2.253650in}}%
\pgfpathlineto{\pgfqpoint{4.956566in}{2.260540in}}%
\pgfpathlineto{\pgfqpoint{4.942318in}{2.257888in}}%
\pgfpathlineto{\pgfqpoint{4.928081in}{2.255307in}}%
\pgfpathlineto{\pgfqpoint{4.913856in}{2.252796in}}%
\pgfpathlineto{\pgfqpoint{4.899642in}{2.250357in}}%
\pgfpathlineto{\pgfqpoint{4.891928in}{2.243350in}}%
\pgfpathlineto{\pgfqpoint{4.884208in}{2.236273in}}%
\pgfpathlineto{\pgfqpoint{4.876481in}{2.229124in}}%
\pgfpathlineto{\pgfqpoint{4.868746in}{2.221900in}}%
\pgfpathclose%
\pgfusepath{fill}%
\end{pgfscope}%
\begin{pgfscope}%
\pgfpathrectangle{\pgfqpoint{1.150000in}{0.150000in}}{\pgfqpoint{5.700000in}{5.700000in}}%
\pgfusepath{clip}%
\pgfsetbuttcap%
\pgfsetroundjoin%
\definecolor{currentfill}{rgb}{0.260571,0.246922,0.522828}%
\pgfsetfillcolor{currentfill}%
\pgfsetfillopacity{0.700000}%
\pgfsetlinewidth{0.000000pt}%
\definecolor{currentstroke}{rgb}{0.000000,0.000000,0.000000}%
\pgfsetstrokecolor{currentstroke}%
\pgfsetdash{}{0pt}%
\pgfpathmoveto{\pgfqpoint{2.095288in}{2.198261in}}%
\pgfpathlineto{\pgfqpoint{2.109145in}{2.184729in}}%
\pgfpathlineto{\pgfqpoint{2.122999in}{2.171330in}}%
\pgfpathlineto{\pgfqpoint{2.136849in}{2.158063in}}%
\pgfpathlineto{\pgfqpoint{2.150696in}{2.144926in}}%
\pgfpathlineto{\pgfqpoint{2.159799in}{2.144270in}}%
\pgfpathlineto{\pgfqpoint{2.168882in}{2.143871in}}%
\pgfpathlineto{\pgfqpoint{2.177947in}{2.143726in}}%
\pgfpathlineto{\pgfqpoint{2.186993in}{2.143826in}}%
\pgfpathlineto{\pgfqpoint{2.173186in}{2.156600in}}%
\pgfpathlineto{\pgfqpoint{2.159376in}{2.169503in}}%
\pgfpathlineto{\pgfqpoint{2.145562in}{2.182537in}}%
\pgfpathlineto{\pgfqpoint{2.131746in}{2.195703in}}%
\pgfpathlineto{\pgfqpoint{2.122660in}{2.195958in}}%
\pgfpathlineto{\pgfqpoint{2.113556in}{2.196466in}}%
\pgfpathlineto{\pgfqpoint{2.104432in}{2.197231in}}%
\pgfpathlineto{\pgfqpoint{2.095288in}{2.198261in}}%
\pgfpathclose%
\pgfusepath{fill}%
\end{pgfscope}%
\begin{pgfscope}%
\pgfpathrectangle{\pgfqpoint{1.150000in}{0.150000in}}{\pgfqpoint{5.700000in}{5.700000in}}%
\pgfusepath{clip}%
\pgfsetbuttcap%
\pgfsetroundjoin%
\definecolor{currentfill}{rgb}{0.283229,0.120777,0.440584}%
\pgfsetfillcolor{currentfill}%
\pgfsetfillopacity{0.700000}%
\pgfsetlinewidth{0.000000pt}%
\definecolor{currentstroke}{rgb}{0.000000,0.000000,0.000000}%
\pgfsetstrokecolor{currentstroke}%
\pgfsetdash{}{0pt}%
\pgfpathmoveto{\pgfqpoint{2.462695in}{1.914034in}}%
\pgfpathlineto{\pgfqpoint{2.476468in}{1.903755in}}%
\pgfpathlineto{\pgfqpoint{2.490241in}{1.893585in}}%
\pgfpathlineto{\pgfqpoint{2.504013in}{1.883524in}}%
\pgfpathlineto{\pgfqpoint{2.517786in}{1.873569in}}%
\pgfpathlineto{\pgfqpoint{2.526602in}{1.875985in}}%
\pgfpathlineto{\pgfqpoint{2.535404in}{1.878603in}}%
\pgfpathlineto{\pgfqpoint{2.544192in}{1.881419in}}%
\pgfpathlineto{\pgfqpoint{2.552965in}{1.884428in}}%
\pgfpathlineto{\pgfqpoint{2.539224in}{1.894048in}}%
\pgfpathlineto{\pgfqpoint{2.525484in}{1.903776in}}%
\pgfpathlineto{\pgfqpoint{2.511744in}{1.913611in}}%
\pgfpathlineto{\pgfqpoint{2.498003in}{1.923555in}}%
\pgfpathlineto{\pgfqpoint{2.489198in}{1.920873in}}%
\pgfpathlineto{\pgfqpoint{2.480379in}{1.918388in}}%
\pgfpathlineto{\pgfqpoint{2.471545in}{1.916107in}}%
\pgfpathlineto{\pgfqpoint{2.462695in}{1.914034in}}%
\pgfpathclose%
\pgfusepath{fill}%
\end{pgfscope}%
\begin{pgfscope}%
\pgfpathrectangle{\pgfqpoint{1.150000in}{0.150000in}}{\pgfqpoint{5.700000in}{5.700000in}}%
\pgfusepath{clip}%
\pgfsetbuttcap%
\pgfsetroundjoin%
\definecolor{currentfill}{rgb}{0.281924,0.089666,0.412415}%
\pgfsetfillcolor{currentfill}%
\pgfsetfillopacity{0.700000}%
\pgfsetlinewidth{0.000000pt}%
\definecolor{currentstroke}{rgb}{0.000000,0.000000,0.000000}%
\pgfsetstrokecolor{currentstroke}%
\pgfsetdash{}{0pt}%
\pgfpathmoveto{\pgfqpoint{3.933341in}{1.820105in}}%
\pgfpathlineto{\pgfqpoint{3.947246in}{1.819369in}}%
\pgfpathlineto{\pgfqpoint{3.961158in}{1.818707in}}%
\pgfpathlineto{\pgfqpoint{3.975079in}{1.818121in}}%
\pgfpathlineto{\pgfqpoint{3.989008in}{1.817611in}}%
\pgfpathlineto{\pgfqpoint{3.997095in}{1.827520in}}%
\pgfpathlineto{\pgfqpoint{4.005177in}{1.837390in}}%
\pgfpathlineto{\pgfqpoint{4.013253in}{1.847219in}}%
\pgfpathlineto{\pgfqpoint{4.021323in}{1.857007in}}%
\pgfpathlineto{\pgfqpoint{4.007404in}{1.857415in}}%
\pgfpathlineto{\pgfqpoint{3.993493in}{1.857899in}}%
\pgfpathlineto{\pgfqpoint{3.979590in}{1.858458in}}%
\pgfpathlineto{\pgfqpoint{3.965696in}{1.859093in}}%
\pgfpathlineto{\pgfqpoint{3.957615in}{1.849400in}}%
\pgfpathlineto{\pgfqpoint{3.949529in}{1.839670in}}%
\pgfpathlineto{\pgfqpoint{3.941438in}{1.829905in}}%
\pgfpathlineto{\pgfqpoint{3.933341in}{1.820105in}}%
\pgfpathclose%
\pgfusepath{fill}%
\end{pgfscope}%
\begin{pgfscope}%
\pgfpathrectangle{\pgfqpoint{1.150000in}{0.150000in}}{\pgfqpoint{5.700000in}{5.700000in}}%
\pgfusepath{clip}%
\pgfsetbuttcap%
\pgfsetroundjoin%
\definecolor{currentfill}{rgb}{0.267004,0.004874,0.329415}%
\pgfsetfillcolor{currentfill}%
\pgfsetfillopacity{0.700000}%
\pgfsetlinewidth{0.000000pt}%
\definecolor{currentstroke}{rgb}{0.000000,0.000000,0.000000}%
\pgfsetstrokecolor{currentstroke}%
\pgfsetdash{}{0pt}%
\pgfpathmoveto{\pgfqpoint{3.293802in}{1.677227in}}%
\pgfpathlineto{\pgfqpoint{3.307573in}{1.672923in}}%
\pgfpathlineto{\pgfqpoint{3.321349in}{1.668701in}}%
\pgfpathlineto{\pgfqpoint{3.335130in}{1.664562in}}%
\pgfpathlineto{\pgfqpoint{3.348917in}{1.660506in}}%
\pgfpathlineto{\pgfqpoint{3.357246in}{1.668916in}}%
\pgfpathlineto{\pgfqpoint{3.365568in}{1.677382in}}%
\pgfpathlineto{\pgfqpoint{3.373884in}{1.685901in}}%
\pgfpathlineto{\pgfqpoint{3.382192in}{1.694469in}}%
\pgfpathlineto{\pgfqpoint{3.368421in}{1.698300in}}%
\pgfpathlineto{\pgfqpoint{3.354656in}{1.702214in}}%
\pgfpathlineto{\pgfqpoint{3.340896in}{1.706211in}}%
\pgfpathlineto{\pgfqpoint{3.327141in}{1.710290in}}%
\pgfpathlineto{\pgfqpoint{3.318817in}{1.701939in}}%
\pgfpathlineto{\pgfqpoint{3.310486in}{1.693643in}}%
\pgfpathlineto{\pgfqpoint{3.302147in}{1.685404in}}%
\pgfpathlineto{\pgfqpoint{3.293802in}{1.677227in}}%
\pgfpathclose%
\pgfusepath{fill}%
\end{pgfscope}%
\begin{pgfscope}%
\pgfpathrectangle{\pgfqpoint{1.150000in}{0.150000in}}{\pgfqpoint{5.700000in}{5.700000in}}%
\pgfusepath{clip}%
\pgfsetbuttcap%
\pgfsetroundjoin%
\definecolor{currentfill}{rgb}{0.271305,0.019942,0.347269}%
\pgfsetfillcolor{currentfill}%
\pgfsetfillopacity{0.700000}%
\pgfsetlinewidth{0.000000pt}%
\definecolor{currentstroke}{rgb}{0.000000,0.000000,0.000000}%
\pgfsetstrokecolor{currentstroke}%
\pgfsetdash{}{0pt}%
\pgfpathmoveto{\pgfqpoint{3.525608in}{1.703092in}}%
\pgfpathlineto{\pgfqpoint{3.539417in}{1.700193in}}%
\pgfpathlineto{\pgfqpoint{3.553231in}{1.697374in}}%
\pgfpathlineto{\pgfqpoint{3.567052in}{1.694634in}}%
\pgfpathlineto{\pgfqpoint{3.580879in}{1.691973in}}%
\pgfpathlineto{\pgfqpoint{3.589113in}{1.701320in}}%
\pgfpathlineto{\pgfqpoint{3.597340in}{1.710684in}}%
\pgfpathlineto{\pgfqpoint{3.605562in}{1.720063in}}%
\pgfpathlineto{\pgfqpoint{3.613777in}{1.729454in}}%
\pgfpathlineto{\pgfqpoint{3.599963in}{1.731931in}}%
\pgfpathlineto{\pgfqpoint{3.586155in}{1.734487in}}%
\pgfpathlineto{\pgfqpoint{3.572353in}{1.737123in}}%
\pgfpathlineto{\pgfqpoint{3.558558in}{1.739837in}}%
\pgfpathlineto{\pgfqpoint{3.550330in}{1.730623in}}%
\pgfpathlineto{\pgfqpoint{3.542096in}{1.721425in}}%
\pgfpathlineto{\pgfqpoint{3.533855in}{1.712248in}}%
\pgfpathlineto{\pgfqpoint{3.525608in}{1.703092in}}%
\pgfpathclose%
\pgfusepath{fill}%
\end{pgfscope}%
\begin{pgfscope}%
\pgfpathrectangle{\pgfqpoint{1.150000in}{0.150000in}}{\pgfqpoint{5.700000in}{5.700000in}}%
\pgfusepath{clip}%
\pgfsetbuttcap%
\pgfsetroundjoin%
\definecolor{currentfill}{rgb}{0.206756,0.371758,0.553117}%
\pgfsetfillcolor{currentfill}%
\pgfsetfillopacity{0.700000}%
\pgfsetlinewidth{0.000000pt}%
\definecolor{currentstroke}{rgb}{0.000000,0.000000,0.000000}%
\pgfsetstrokecolor{currentstroke}%
\pgfsetdash{}{0pt}%
\pgfpathmoveto{\pgfqpoint{5.452559in}{2.450768in}}%
\pgfpathlineto{\pgfqpoint{5.467020in}{2.454300in}}%
\pgfpathlineto{\pgfqpoint{5.481495in}{2.457902in}}%
\pgfpathlineto{\pgfqpoint{5.495981in}{2.461573in}}%
\pgfpathlineto{\pgfqpoint{5.510481in}{2.465313in}}%
\pgfpathlineto{\pgfqpoint{5.517901in}{2.469702in}}%
\pgfpathlineto{\pgfqpoint{5.525313in}{2.474050in}}%
\pgfpathlineto{\pgfqpoint{5.532718in}{2.478364in}}%
\pgfpathlineto{\pgfqpoint{5.540115in}{2.482646in}}%
\pgfpathlineto{\pgfqpoint{5.525637in}{2.479145in}}%
\pgfpathlineto{\pgfqpoint{5.511171in}{2.475713in}}%
\pgfpathlineto{\pgfqpoint{5.496718in}{2.472350in}}%
\pgfpathlineto{\pgfqpoint{5.482277in}{2.469056in}}%
\pgfpathlineto{\pgfqpoint{5.474858in}{2.464527in}}%
\pgfpathlineto{\pgfqpoint{5.467432in}{2.459972in}}%
\pgfpathlineto{\pgfqpoint{5.459999in}{2.455387in}}%
\pgfpathlineto{\pgfqpoint{5.452559in}{2.450768in}}%
\pgfpathclose%
\pgfusepath{fill}%
\end{pgfscope}%
\begin{pgfscope}%
\pgfpathrectangle{\pgfqpoint{1.150000in}{0.150000in}}{\pgfqpoint{5.700000in}{5.700000in}}%
\pgfusepath{clip}%
\pgfsetbuttcap%
\pgfsetroundjoin%
\definecolor{currentfill}{rgb}{0.255645,0.260703,0.528312}%
\pgfsetfillcolor{currentfill}%
\pgfsetfillopacity{0.700000}%
\pgfsetlinewidth{0.000000pt}%
\definecolor{currentstroke}{rgb}{0.000000,0.000000,0.000000}%
\pgfsetstrokecolor{currentstroke}%
\pgfsetdash{}{0pt}%
\pgfpathmoveto{\pgfqpoint{4.780886in}{2.182387in}}%
\pgfpathlineto{\pgfqpoint{4.795082in}{2.184739in}}%
\pgfpathlineto{\pgfqpoint{4.809289in}{2.187162in}}%
\pgfpathlineto{\pgfqpoint{4.823507in}{2.189657in}}%
\pgfpathlineto{\pgfqpoint{4.837736in}{2.192222in}}%
\pgfpathlineto{\pgfqpoint{4.845499in}{2.199763in}}%
\pgfpathlineto{\pgfqpoint{4.853255in}{2.207222in}}%
\pgfpathlineto{\pgfqpoint{4.861004in}{2.214600in}}%
\pgfpathlineto{\pgfqpoint{4.868746in}{2.221900in}}%
\pgfpathlineto{\pgfqpoint{4.854529in}{2.219422in}}%
\pgfpathlineto{\pgfqpoint{4.840324in}{2.217016in}}%
\pgfpathlineto{\pgfqpoint{4.826129in}{2.214680in}}%
\pgfpathlineto{\pgfqpoint{4.811946in}{2.212416in}}%
\pgfpathlineto{\pgfqpoint{4.804191in}{2.205020in}}%
\pgfpathlineto{\pgfqpoint{4.796430in}{2.197552in}}%
\pgfpathlineto{\pgfqpoint{4.788661in}{2.190008in}}%
\pgfpathlineto{\pgfqpoint{4.780886in}{2.182387in}}%
\pgfpathclose%
\pgfusepath{fill}%
\end{pgfscope}%
\begin{pgfscope}%
\pgfpathrectangle{\pgfqpoint{1.150000in}{0.150000in}}{\pgfqpoint{5.700000in}{5.700000in}}%
\pgfusepath{clip}%
\pgfsetbuttcap%
\pgfsetroundjoin%
\definecolor{currentfill}{rgb}{0.266580,0.228262,0.514349}%
\pgfsetfillcolor{currentfill}%
\pgfsetfillopacity{0.700000}%
\pgfsetlinewidth{0.000000pt}%
\definecolor{currentstroke}{rgb}{0.000000,0.000000,0.000000}%
\pgfsetstrokecolor{currentstroke}%
\pgfsetdash{}{0pt}%
\pgfpathmoveto{\pgfqpoint{2.150696in}{2.144926in}}%
\pgfpathlineto{\pgfqpoint{2.164540in}{2.131918in}}%
\pgfpathlineto{\pgfqpoint{2.178380in}{2.119038in}}%
\pgfpathlineto{\pgfqpoint{2.192218in}{2.106285in}}%
\pgfpathlineto{\pgfqpoint{2.206053in}{2.093658in}}%
\pgfpathlineto{\pgfqpoint{2.215116in}{2.093373in}}%
\pgfpathlineto{\pgfqpoint{2.224160in}{2.093341in}}%
\pgfpathlineto{\pgfqpoint{2.233185in}{2.093555in}}%
\pgfpathlineto{\pgfqpoint{2.242193in}{2.094010in}}%
\pgfpathlineto{\pgfqpoint{2.228397in}{2.106275in}}%
\pgfpathlineto{\pgfqpoint{2.214598in}{2.118665in}}%
\pgfpathlineto{\pgfqpoint{2.200797in}{2.131182in}}%
\pgfpathlineto{\pgfqpoint{2.186993in}{2.143826in}}%
\pgfpathlineto{\pgfqpoint{2.177947in}{2.143726in}}%
\pgfpathlineto{\pgfqpoint{2.168882in}{2.143871in}}%
\pgfpathlineto{\pgfqpoint{2.159799in}{2.144270in}}%
\pgfpathlineto{\pgfqpoint{2.150696in}{2.144926in}}%
\pgfpathclose%
\pgfusepath{fill}%
\end{pgfscope}%
\begin{pgfscope}%
\pgfpathrectangle{\pgfqpoint{1.150000in}{0.150000in}}{\pgfqpoint{5.700000in}{5.700000in}}%
\pgfusepath{clip}%
\pgfsetbuttcap%
\pgfsetroundjoin%
\definecolor{currentfill}{rgb}{0.280267,0.073417,0.397163}%
\pgfsetfillcolor{currentfill}%
\pgfsetfillopacity{0.700000}%
\pgfsetlinewidth{0.000000pt}%
\definecolor{currentstroke}{rgb}{0.000000,0.000000,0.000000}%
\pgfsetstrokecolor{currentstroke}%
\pgfsetdash{}{0pt}%
\pgfpathmoveto{\pgfqpoint{3.845317in}{1.784786in}}%
\pgfpathlineto{\pgfqpoint{3.859201in}{1.783623in}}%
\pgfpathlineto{\pgfqpoint{3.873092in}{1.782536in}}%
\pgfpathlineto{\pgfqpoint{3.886991in}{1.781525in}}%
\pgfpathlineto{\pgfqpoint{3.900898in}{1.780590in}}%
\pgfpathlineto{\pgfqpoint{3.909017in}{1.790514in}}%
\pgfpathlineto{\pgfqpoint{3.917130in}{1.800408in}}%
\pgfpathlineto{\pgfqpoint{3.925238in}{1.810273in}}%
\pgfpathlineto{\pgfqpoint{3.933341in}{1.820105in}}%
\pgfpathlineto{\pgfqpoint{3.919444in}{1.820918in}}%
\pgfpathlineto{\pgfqpoint{3.905556in}{1.821806in}}%
\pgfpathlineto{\pgfqpoint{3.891675in}{1.822770in}}%
\pgfpathlineto{\pgfqpoint{3.877801in}{1.823810in}}%
\pgfpathlineto{\pgfqpoint{3.869689in}{1.814092in}}%
\pgfpathlineto{\pgfqpoint{3.861570in}{1.804348in}}%
\pgfpathlineto{\pgfqpoint{3.853447in}{1.794579in}}%
\pgfpathlineto{\pgfqpoint{3.845317in}{1.784786in}}%
\pgfpathclose%
\pgfusepath{fill}%
\end{pgfscope}%
\begin{pgfscope}%
\pgfpathrectangle{\pgfqpoint{1.150000in}{0.150000in}}{\pgfqpoint{5.700000in}{5.700000in}}%
\pgfusepath{clip}%
\pgfsetbuttcap%
\pgfsetroundjoin%
\definecolor{currentfill}{rgb}{0.260571,0.246922,0.522828}%
\pgfsetfillcolor{currentfill}%
\pgfsetfillopacity{0.700000}%
\pgfsetlinewidth{0.000000pt}%
\definecolor{currentstroke}{rgb}{0.000000,0.000000,0.000000}%
\pgfsetstrokecolor{currentstroke}%
\pgfsetdash{}{0pt}%
\pgfpathmoveto{\pgfqpoint{4.692992in}{2.142141in}}%
\pgfpathlineto{\pgfqpoint{4.707156in}{2.144273in}}%
\pgfpathlineto{\pgfqpoint{4.721331in}{2.146478in}}%
\pgfpathlineto{\pgfqpoint{4.735517in}{2.148753in}}%
\pgfpathlineto{\pgfqpoint{4.749714in}{2.151100in}}%
\pgfpathlineto{\pgfqpoint{4.757517in}{2.159046in}}%
\pgfpathlineto{\pgfqpoint{4.765314in}{2.166908in}}%
\pgfpathlineto{\pgfqpoint{4.773103in}{2.174688in}}%
\pgfpathlineto{\pgfqpoint{4.780886in}{2.182387in}}%
\pgfpathlineto{\pgfqpoint{4.766701in}{2.180106in}}%
\pgfpathlineto{\pgfqpoint{4.752526in}{2.177897in}}%
\pgfpathlineto{\pgfqpoint{4.738363in}{2.175759in}}%
\pgfpathlineto{\pgfqpoint{4.724210in}{2.173692in}}%
\pgfpathlineto{\pgfqpoint{4.716416in}{2.165919in}}%
\pgfpathlineto{\pgfqpoint{4.708614in}{2.158070in}}%
\pgfpathlineto{\pgfqpoint{4.700806in}{2.150145in}}%
\pgfpathlineto{\pgfqpoint{4.692992in}{2.142141in}}%
\pgfpathclose%
\pgfusepath{fill}%
\end{pgfscope}%
\begin{pgfscope}%
\pgfpathrectangle{\pgfqpoint{1.150000in}{0.150000in}}{\pgfqpoint{5.700000in}{5.700000in}}%
\pgfusepath{clip}%
\pgfsetbuttcap%
\pgfsetroundjoin%
\definecolor{currentfill}{rgb}{0.277941,0.056324,0.381191}%
\pgfsetfillcolor{currentfill}%
\pgfsetfillopacity{0.700000}%
\pgfsetlinewidth{0.000000pt}%
\definecolor{currentstroke}{rgb}{0.000000,0.000000,0.000000}%
\pgfsetstrokecolor{currentstroke}%
\pgfsetdash{}{0pt}%
\pgfpathmoveto{\pgfqpoint{2.717902in}{1.777088in}}%
\pgfpathlineto{\pgfqpoint{2.731654in}{1.768798in}}%
\pgfpathlineto{\pgfqpoint{2.745408in}{1.760605in}}%
\pgfpathlineto{\pgfqpoint{2.759163in}{1.752510in}}%
\pgfpathlineto{\pgfqpoint{2.772920in}{1.744511in}}%
\pgfpathlineto{\pgfqpoint{2.781565in}{1.748984in}}%
\pgfpathlineto{\pgfqpoint{2.790199in}{1.753618in}}%
\pgfpathlineto{\pgfqpoint{2.798820in}{1.758410in}}%
\pgfpathlineto{\pgfqpoint{2.807430in}{1.763353in}}%
\pgfpathlineto{\pgfqpoint{2.793700in}{1.771042in}}%
\pgfpathlineto{\pgfqpoint{2.779971in}{1.778827in}}%
\pgfpathlineto{\pgfqpoint{2.766244in}{1.786710in}}%
\pgfpathlineto{\pgfqpoint{2.752519in}{1.794690in}}%
\pgfpathlineto{\pgfqpoint{2.743883in}{1.790048in}}%
\pgfpathlineto{\pgfqpoint{2.735235in}{1.785564in}}%
\pgfpathlineto{\pgfqpoint{2.726575in}{1.781243in}}%
\pgfpathlineto{\pgfqpoint{2.717902in}{1.777088in}}%
\pgfpathclose%
\pgfusepath{fill}%
\end{pgfscope}%
\begin{pgfscope}%
\pgfpathrectangle{\pgfqpoint{1.150000in}{0.150000in}}{\pgfqpoint{5.700000in}{5.700000in}}%
\pgfusepath{clip}%
\pgfsetbuttcap%
\pgfsetroundjoin%
\definecolor{currentfill}{rgb}{0.212395,0.359683,0.551710}%
\pgfsetfillcolor{currentfill}%
\pgfsetfillopacity{0.700000}%
\pgfsetlinewidth{0.000000pt}%
\definecolor{currentstroke}{rgb}{0.000000,0.000000,0.000000}%
\pgfsetstrokecolor{currentstroke}%
\pgfsetdash{}{0pt}%
\pgfpathmoveto{\pgfqpoint{5.364919in}{2.417558in}}%
\pgfpathlineto{\pgfqpoint{5.379351in}{2.421030in}}%
\pgfpathlineto{\pgfqpoint{5.393795in}{2.424572in}}%
\pgfpathlineto{\pgfqpoint{5.408252in}{2.428183in}}%
\pgfpathlineto{\pgfqpoint{5.422721in}{2.431864in}}%
\pgfpathlineto{\pgfqpoint{5.430192in}{2.436662in}}%
\pgfpathlineto{\pgfqpoint{5.437655in}{2.441409in}}%
\pgfpathlineto{\pgfqpoint{5.445111in}{2.446110in}}%
\pgfpathlineto{\pgfqpoint{5.452559in}{2.450768in}}%
\pgfpathlineto{\pgfqpoint{5.438109in}{2.447305in}}%
\pgfpathlineto{\pgfqpoint{5.423673in}{2.443911in}}%
\pgfpathlineto{\pgfqpoint{5.409248in}{2.440587in}}%
\pgfpathlineto{\pgfqpoint{5.394836in}{2.437332in}}%
\pgfpathlineto{\pgfqpoint{5.387368in}{2.432449in}}%
\pgfpathlineto{\pgfqpoint{5.379893in}{2.427528in}}%
\pgfpathlineto{\pgfqpoint{5.372410in}{2.422566in}}%
\pgfpathlineto{\pgfqpoint{5.364919in}{2.417558in}}%
\pgfpathclose%
\pgfusepath{fill}%
\end{pgfscope}%
\begin{pgfscope}%
\pgfpathrectangle{\pgfqpoint{1.150000in}{0.150000in}}{\pgfqpoint{5.700000in}{5.700000in}}%
\pgfusepath{clip}%
\pgfsetbuttcap%
\pgfsetroundjoin%
\definecolor{currentfill}{rgb}{0.266580,0.228262,0.514349}%
\pgfsetfillcolor{currentfill}%
\pgfsetfillopacity{0.700000}%
\pgfsetlinewidth{0.000000pt}%
\definecolor{currentstroke}{rgb}{0.000000,0.000000,0.000000}%
\pgfsetstrokecolor{currentstroke}%
\pgfsetdash{}{0pt}%
\pgfpathmoveto{\pgfqpoint{4.605069in}{2.101322in}}%
\pgfpathlineto{\pgfqpoint{4.619202in}{2.103213in}}%
\pgfpathlineto{\pgfqpoint{4.633346in}{2.105175in}}%
\pgfpathlineto{\pgfqpoint{4.647500in}{2.107210in}}%
\pgfpathlineto{\pgfqpoint{4.661665in}{2.109316in}}%
\pgfpathlineto{\pgfqpoint{4.669507in}{2.117646in}}%
\pgfpathlineto{\pgfqpoint{4.677342in}{2.125893in}}%
\pgfpathlineto{\pgfqpoint{4.685170in}{2.134057in}}%
\pgfpathlineto{\pgfqpoint{4.692992in}{2.142141in}}%
\pgfpathlineto{\pgfqpoint{4.678838in}{2.140080in}}%
\pgfpathlineto{\pgfqpoint{4.664694in}{2.138090in}}%
\pgfpathlineto{\pgfqpoint{4.650562in}{2.136172in}}%
\pgfpathlineto{\pgfqpoint{4.636439in}{2.134326in}}%
\pgfpathlineto{\pgfqpoint{4.628607in}{2.126190in}}%
\pgfpathlineto{\pgfqpoint{4.620767in}{2.117978in}}%
\pgfpathlineto{\pgfqpoint{4.612921in}{2.109689in}}%
\pgfpathlineto{\pgfqpoint{4.605069in}{2.101322in}}%
\pgfpathclose%
\pgfusepath{fill}%
\end{pgfscope}%
\begin{pgfscope}%
\pgfpathrectangle{\pgfqpoint{1.150000in}{0.150000in}}{\pgfqpoint{5.700000in}{5.700000in}}%
\pgfusepath{clip}%
\pgfsetbuttcap%
\pgfsetroundjoin%
\definecolor{currentfill}{rgb}{0.273006,0.204520,0.501721}%
\pgfsetfillcolor{currentfill}%
\pgfsetfillopacity{0.700000}%
\pgfsetlinewidth{0.000000pt}%
\definecolor{currentstroke}{rgb}{0.000000,0.000000,0.000000}%
\pgfsetstrokecolor{currentstroke}%
\pgfsetdash{}{0pt}%
\pgfpathmoveto{\pgfqpoint{2.206053in}{2.093658in}}%
\pgfpathlineto{\pgfqpoint{2.219885in}{2.081156in}}%
\pgfpathlineto{\pgfqpoint{2.233715in}{2.068777in}}%
\pgfpathlineto{\pgfqpoint{2.247542in}{2.056522in}}%
\pgfpathlineto{\pgfqpoint{2.261368in}{2.044388in}}%
\pgfpathlineto{\pgfqpoint{2.270391in}{2.044473in}}%
\pgfpathlineto{\pgfqpoint{2.279397in}{2.044805in}}%
\pgfpathlineto{\pgfqpoint{2.288385in}{2.045377in}}%
\pgfpathlineto{\pgfqpoint{2.297355in}{2.046185in}}%
\pgfpathlineto{\pgfqpoint{2.283568in}{2.057958in}}%
\pgfpathlineto{\pgfqpoint{2.269779in}{2.069853in}}%
\pgfpathlineto{\pgfqpoint{2.255987in}{2.081870in}}%
\pgfpathlineto{\pgfqpoint{2.242193in}{2.094010in}}%
\pgfpathlineto{\pgfqpoint{2.233185in}{2.093555in}}%
\pgfpathlineto{\pgfqpoint{2.224160in}{2.093341in}}%
\pgfpathlineto{\pgfqpoint{2.215116in}{2.093373in}}%
\pgfpathlineto{\pgfqpoint{2.206053in}{2.093658in}}%
\pgfpathclose%
\pgfusepath{fill}%
\end{pgfscope}%
\begin{pgfscope}%
\pgfpathrectangle{\pgfqpoint{1.150000in}{0.150000in}}{\pgfqpoint{5.700000in}{5.700000in}}%
\pgfusepath{clip}%
\pgfsetbuttcap%
\pgfsetroundjoin%
\definecolor{currentfill}{rgb}{0.271828,0.209303,0.504434}%
\pgfsetfillcolor{currentfill}%
\pgfsetfillopacity{0.700000}%
\pgfsetlinewidth{0.000000pt}%
\definecolor{currentstroke}{rgb}{0.000000,0.000000,0.000000}%
\pgfsetstrokecolor{currentstroke}%
\pgfsetdash{}{0pt}%
\pgfpathmoveto{\pgfqpoint{4.517121in}{2.060113in}}%
\pgfpathlineto{\pgfqpoint{4.531224in}{2.061739in}}%
\pgfpathlineto{\pgfqpoint{4.545336in}{2.063438in}}%
\pgfpathlineto{\pgfqpoint{4.559459in}{2.065209in}}%
\pgfpathlineto{\pgfqpoint{4.573592in}{2.067051in}}%
\pgfpathlineto{\pgfqpoint{4.581471in}{2.075741in}}%
\pgfpathlineto{\pgfqpoint{4.589344in}{2.084349in}}%
\pgfpathlineto{\pgfqpoint{4.597209in}{2.092875in}}%
\pgfpathlineto{\pgfqpoint{4.605069in}{2.101322in}}%
\pgfpathlineto{\pgfqpoint{4.590946in}{2.099503in}}%
\pgfpathlineto{\pgfqpoint{4.576833in}{2.097756in}}%
\pgfpathlineto{\pgfqpoint{4.562731in}{2.096081in}}%
\pgfpathlineto{\pgfqpoint{4.548639in}{2.094478in}}%
\pgfpathlineto{\pgfqpoint{4.540769in}{2.086000in}}%
\pgfpathlineto{\pgfqpoint{4.532893in}{2.077447in}}%
\pgfpathlineto{\pgfqpoint{4.525010in}{2.068818in}}%
\pgfpathlineto{\pgfqpoint{4.517121in}{2.060113in}}%
\pgfpathclose%
\pgfusepath{fill}%
\end{pgfscope}%
\begin{pgfscope}%
\pgfpathrectangle{\pgfqpoint{1.150000in}{0.150000in}}{\pgfqpoint{5.700000in}{5.700000in}}%
\pgfusepath{clip}%
\pgfsetbuttcap%
\pgfsetroundjoin%
\definecolor{currentfill}{rgb}{0.277941,0.056324,0.381191}%
\pgfsetfillcolor{currentfill}%
\pgfsetfillopacity{0.700000}%
\pgfsetlinewidth{0.000000pt}%
\definecolor{currentstroke}{rgb}{0.000000,0.000000,0.000000}%
\pgfsetstrokecolor{currentstroke}%
\pgfsetdash{}{0pt}%
\pgfpathmoveto{\pgfqpoint{3.757241in}{1.751402in}}%
\pgfpathlineto{\pgfqpoint{3.771106in}{1.749790in}}%
\pgfpathlineto{\pgfqpoint{3.784978in}{1.748255in}}%
\pgfpathlineto{\pgfqpoint{3.798857in}{1.746796in}}%
\pgfpathlineto{\pgfqpoint{3.812743in}{1.745413in}}%
\pgfpathlineto{\pgfqpoint{3.820895in}{1.755283in}}%
\pgfpathlineto{\pgfqpoint{3.829041in}{1.765136in}}%
\pgfpathlineto{\pgfqpoint{3.837182in}{1.774971in}}%
\pgfpathlineto{\pgfqpoint{3.845317in}{1.784786in}}%
\pgfpathlineto{\pgfqpoint{3.831441in}{1.786025in}}%
\pgfpathlineto{\pgfqpoint{3.817573in}{1.787340in}}%
\pgfpathlineto{\pgfqpoint{3.803712in}{1.788733in}}%
\pgfpathlineto{\pgfqpoint{3.789858in}{1.790202in}}%
\pgfpathlineto{\pgfqpoint{3.781712in}{1.780523in}}%
\pgfpathlineto{\pgfqpoint{3.773561in}{1.770828in}}%
\pgfpathlineto{\pgfqpoint{3.765404in}{1.761121in}}%
\pgfpathlineto{\pgfqpoint{3.757241in}{1.751402in}}%
\pgfpathclose%
\pgfusepath{fill}%
\end{pgfscope}%
\begin{pgfscope}%
\pgfpathrectangle{\pgfqpoint{1.150000in}{0.150000in}}{\pgfqpoint{5.700000in}{5.700000in}}%
\pgfusepath{clip}%
\pgfsetbuttcap%
\pgfsetroundjoin%
\definecolor{currentfill}{rgb}{0.269944,0.014625,0.341379}%
\pgfsetfillcolor{currentfill}%
\pgfsetfillopacity{0.700000}%
\pgfsetlinewidth{0.000000pt}%
\definecolor{currentstroke}{rgb}{0.000000,0.000000,0.000000}%
\pgfsetstrokecolor{currentstroke}%
\pgfsetdash{}{0pt}%
\pgfpathmoveto{\pgfqpoint{3.437329in}{1.679961in}}%
\pgfpathlineto{\pgfqpoint{3.451128in}{1.676536in}}%
\pgfpathlineto{\pgfqpoint{3.464932in}{1.673193in}}%
\pgfpathlineto{\pgfqpoint{3.478742in}{1.669929in}}%
\pgfpathlineto{\pgfqpoint{3.492558in}{1.666746in}}%
\pgfpathlineto{\pgfqpoint{3.500830in}{1.675786in}}%
\pgfpathlineto{\pgfqpoint{3.509096in}{1.684858in}}%
\pgfpathlineto{\pgfqpoint{3.517355in}{1.693961in}}%
\pgfpathlineto{\pgfqpoint{3.525608in}{1.703092in}}%
\pgfpathlineto{\pgfqpoint{3.511806in}{1.706071in}}%
\pgfpathlineto{\pgfqpoint{3.498010in}{1.709130in}}%
\pgfpathlineto{\pgfqpoint{3.484220in}{1.712269in}}%
\pgfpathlineto{\pgfqpoint{3.470436in}{1.715488in}}%
\pgfpathlineto{\pgfqpoint{3.462169in}{1.706554in}}%
\pgfpathlineto{\pgfqpoint{3.453896in}{1.697653in}}%
\pgfpathlineto{\pgfqpoint{3.445616in}{1.688788in}}%
\pgfpathlineto{\pgfqpoint{3.437329in}{1.679961in}}%
\pgfpathclose%
\pgfusepath{fill}%
\end{pgfscope}%
\begin{pgfscope}%
\pgfpathrectangle{\pgfqpoint{1.150000in}{0.150000in}}{\pgfqpoint{5.700000in}{5.700000in}}%
\pgfusepath{clip}%
\pgfsetbuttcap%
\pgfsetroundjoin%
\definecolor{currentfill}{rgb}{0.282910,0.105393,0.426902}%
\pgfsetfillcolor{currentfill}%
\pgfsetfillopacity{0.700000}%
\pgfsetlinewidth{0.000000pt}%
\definecolor{currentstroke}{rgb}{0.000000,0.000000,0.000000}%
\pgfsetstrokecolor{currentstroke}%
\pgfsetdash{}{0pt}%
\pgfpathmoveto{\pgfqpoint{2.517786in}{1.873569in}}%
\pgfpathlineto{\pgfqpoint{2.531558in}{1.863721in}}%
\pgfpathlineto{\pgfqpoint{2.545330in}{1.853980in}}%
\pgfpathlineto{\pgfqpoint{2.559103in}{1.844343in}}%
\pgfpathlineto{\pgfqpoint{2.572876in}{1.834811in}}%
\pgfpathlineto{\pgfqpoint{2.581661in}{1.837569in}}%
\pgfpathlineto{\pgfqpoint{2.590431in}{1.840524in}}%
\pgfpathlineto{\pgfqpoint{2.599188in}{1.843671in}}%
\pgfpathlineto{\pgfqpoint{2.607931in}{1.847005in}}%
\pgfpathlineto{\pgfqpoint{2.594189in}{1.856204in}}%
\pgfpathlineto{\pgfqpoint{2.580447in}{1.865507in}}%
\pgfpathlineto{\pgfqpoint{2.566706in}{1.874914in}}%
\pgfpathlineto{\pgfqpoint{2.552965in}{1.884428in}}%
\pgfpathlineto{\pgfqpoint{2.544192in}{1.881419in}}%
\pgfpathlineto{\pgfqpoint{2.535404in}{1.878603in}}%
\pgfpathlineto{\pgfqpoint{2.526602in}{1.875985in}}%
\pgfpathlineto{\pgfqpoint{2.517786in}{1.873569in}}%
\pgfpathclose%
\pgfusepath{fill}%
\end{pgfscope}%
\begin{pgfscope}%
\pgfpathrectangle{\pgfqpoint{1.150000in}{0.150000in}}{\pgfqpoint{5.700000in}{5.700000in}}%
\pgfusepath{clip}%
\pgfsetbuttcap%
\pgfsetroundjoin%
\definecolor{currentfill}{rgb}{0.268510,0.009605,0.335427}%
\pgfsetfillcolor{currentfill}%
\pgfsetfillopacity{0.700000}%
\pgfsetlinewidth{0.000000pt}%
\definecolor{currentstroke}{rgb}{0.000000,0.000000,0.000000}%
\pgfsetstrokecolor{currentstroke}%
\pgfsetdash{}{0pt}%
\pgfpathmoveto{\pgfqpoint{3.061422in}{1.678831in}}%
\pgfpathlineto{\pgfqpoint{3.075179in}{1.672990in}}%
\pgfpathlineto{\pgfqpoint{3.088939in}{1.667236in}}%
\pgfpathlineto{\pgfqpoint{3.102702in}{1.661569in}}%
\pgfpathlineto{\pgfqpoint{3.116470in}{1.655990in}}%
\pgfpathlineto{\pgfqpoint{3.124917in}{1.663003in}}%
\pgfpathlineto{\pgfqpoint{3.133354in}{1.670116in}}%
\pgfpathlineto{\pgfqpoint{3.141783in}{1.677327in}}%
\pgfpathlineto{\pgfqpoint{3.150204in}{1.684629in}}%
\pgfpathlineto{\pgfqpoint{3.136456in}{1.689943in}}%
\pgfpathlineto{\pgfqpoint{3.122712in}{1.695343in}}%
\pgfpathlineto{\pgfqpoint{3.108972in}{1.700830in}}%
\pgfpathlineto{\pgfqpoint{3.095236in}{1.706404in}}%
\pgfpathlineto{\pgfqpoint{3.086796in}{1.699360in}}%
\pgfpathlineto{\pgfqpoint{3.078347in}{1.692414in}}%
\pgfpathlineto{\pgfqpoint{3.069889in}{1.685570in}}%
\pgfpathlineto{\pgfqpoint{3.061422in}{1.678831in}}%
\pgfpathclose%
\pgfusepath{fill}%
\end{pgfscope}%
\begin{pgfscope}%
\pgfpathrectangle{\pgfqpoint{1.150000in}{0.150000in}}{\pgfqpoint{5.700000in}{5.700000in}}%
\pgfusepath{clip}%
\pgfsetbuttcap%
\pgfsetroundjoin%
\definecolor{currentfill}{rgb}{0.275191,0.194905,0.496005}%
\pgfsetfillcolor{currentfill}%
\pgfsetfillopacity{0.700000}%
\pgfsetlinewidth{0.000000pt}%
\definecolor{currentstroke}{rgb}{0.000000,0.000000,0.000000}%
\pgfsetstrokecolor{currentstroke}%
\pgfsetdash{}{0pt}%
\pgfpathmoveto{\pgfqpoint{4.429152in}{2.018718in}}%
\pgfpathlineto{\pgfqpoint{4.443225in}{2.020058in}}%
\pgfpathlineto{\pgfqpoint{4.457307in}{2.021470in}}%
\pgfpathlineto{\pgfqpoint{4.471399in}{2.022954in}}%
\pgfpathlineto{\pgfqpoint{4.485501in}{2.024511in}}%
\pgfpathlineto{\pgfqpoint{4.493416in}{2.033529in}}%
\pgfpathlineto{\pgfqpoint{4.501324in}{2.042469in}}%
\pgfpathlineto{\pgfqpoint{4.509226in}{2.051330in}}%
\pgfpathlineto{\pgfqpoint{4.517121in}{2.060113in}}%
\pgfpathlineto{\pgfqpoint{4.503029in}{2.058559in}}%
\pgfpathlineto{\pgfqpoint{4.488946in}{2.057077in}}%
\pgfpathlineto{\pgfqpoint{4.474874in}{2.055667in}}%
\pgfpathlineto{\pgfqpoint{4.460812in}{2.054330in}}%
\pgfpathlineto{\pgfqpoint{4.452906in}{2.045537in}}%
\pgfpathlineto{\pgfqpoint{4.444995in}{2.036671in}}%
\pgfpathlineto{\pgfqpoint{4.437077in}{2.027732in}}%
\pgfpathlineto{\pgfqpoint{4.429152in}{2.018718in}}%
\pgfpathclose%
\pgfusepath{fill}%
\end{pgfscope}%
\begin{pgfscope}%
\pgfpathrectangle{\pgfqpoint{1.150000in}{0.150000in}}{\pgfqpoint{5.700000in}{5.700000in}}%
\pgfusepath{clip}%
\pgfsetbuttcap%
\pgfsetroundjoin%
\definecolor{currentfill}{rgb}{0.218130,0.347432,0.550038}%
\pgfsetfillcolor{currentfill}%
\pgfsetfillopacity{0.700000}%
\pgfsetlinewidth{0.000000pt}%
\definecolor{currentstroke}{rgb}{0.000000,0.000000,0.000000}%
\pgfsetstrokecolor{currentstroke}%
\pgfsetdash{}{0pt}%
\pgfpathmoveto{\pgfqpoint{5.277204in}{2.383023in}}%
\pgfpathlineto{\pgfqpoint{5.291605in}{2.386412in}}%
\pgfpathlineto{\pgfqpoint{5.306018in}{2.389871in}}%
\pgfpathlineto{\pgfqpoint{5.320443in}{2.393400in}}%
\pgfpathlineto{\pgfqpoint{5.334881in}{2.396999in}}%
\pgfpathlineto{\pgfqpoint{5.342402in}{2.402225in}}%
\pgfpathlineto{\pgfqpoint{5.349916in}{2.407392in}}%
\pgfpathlineto{\pgfqpoint{5.357421in}{2.412501in}}%
\pgfpathlineto{\pgfqpoint{5.364919in}{2.417558in}}%
\pgfpathlineto{\pgfqpoint{5.350500in}{2.414156in}}%
\pgfpathlineto{\pgfqpoint{5.336093in}{2.410823in}}%
\pgfpathlineto{\pgfqpoint{5.321698in}{2.407560in}}%
\pgfpathlineto{\pgfqpoint{5.307315in}{2.404367in}}%
\pgfpathlineto{\pgfqpoint{5.299799in}{2.399106in}}%
\pgfpathlineto{\pgfqpoint{5.292275in}{2.393798in}}%
\pgfpathlineto{\pgfqpoint{5.284743in}{2.388438in}}%
\pgfpathlineto{\pgfqpoint{5.277204in}{2.383023in}}%
\pgfpathclose%
\pgfusepath{fill}%
\end{pgfscope}%
\begin{pgfscope}%
\pgfpathrectangle{\pgfqpoint{1.150000in}{0.150000in}}{\pgfqpoint{5.700000in}{5.700000in}}%
\pgfusepath{clip}%
\pgfsetbuttcap%
\pgfsetroundjoin%
\definecolor{currentfill}{rgb}{0.271305,0.019942,0.347269}%
\pgfsetfillcolor{currentfill}%
\pgfsetfillopacity{0.700000}%
\pgfsetlinewidth{0.000000pt}%
\definecolor{currentstroke}{rgb}{0.000000,0.000000,0.000000}%
\pgfsetstrokecolor{currentstroke}%
\pgfsetdash{}{0pt}%
\pgfpathmoveto{\pgfqpoint{2.917357in}{1.705249in}}%
\pgfpathlineto{\pgfqpoint{2.931110in}{1.698404in}}%
\pgfpathlineto{\pgfqpoint{2.944865in}{1.691650in}}%
\pgfpathlineto{\pgfqpoint{2.958623in}{1.684986in}}%
\pgfpathlineto{\pgfqpoint{2.972385in}{1.678413in}}%
\pgfpathlineto{\pgfqpoint{2.980911in}{1.684393in}}%
\pgfpathlineto{\pgfqpoint{2.989428in}{1.690500in}}%
\pgfpathlineto{\pgfqpoint{2.997935in}{1.696730in}}%
\pgfpathlineto{\pgfqpoint{3.006433in}{1.703079in}}%
\pgfpathlineto{\pgfqpoint{2.992694in}{1.709365in}}%
\pgfpathlineto{\pgfqpoint{2.978958in}{1.715740in}}%
\pgfpathlineto{\pgfqpoint{2.965226in}{1.722206in}}%
\pgfpathlineto{\pgfqpoint{2.951496in}{1.728763in}}%
\pgfpathlineto{\pgfqpoint{2.942977in}{1.722694in}}%
\pgfpathlineto{\pgfqpoint{2.934447in}{1.716749in}}%
\pgfpathlineto{\pgfqpoint{2.925907in}{1.710933in}}%
\pgfpathlineto{\pgfqpoint{2.917357in}{1.705249in}}%
\pgfpathclose%
\pgfusepath{fill}%
\end{pgfscope}%
\begin{pgfscope}%
\pgfpathrectangle{\pgfqpoint{1.150000in}{0.150000in}}{\pgfqpoint{5.700000in}{5.700000in}}%
\pgfusepath{clip}%
\pgfsetbuttcap%
\pgfsetroundjoin%
\definecolor{currentfill}{rgb}{0.267004,0.004874,0.329415}%
\pgfsetfillcolor{currentfill}%
\pgfsetfillopacity{0.700000}%
\pgfsetlinewidth{0.000000pt}%
\definecolor{currentstroke}{rgb}{0.000000,0.000000,0.000000}%
\pgfsetstrokecolor{currentstroke}%
\pgfsetdash{}{0pt}%
\pgfpathmoveto{\pgfqpoint{3.205238in}{1.664235in}}%
\pgfpathlineto{\pgfqpoint{3.219008in}{1.659349in}}%
\pgfpathlineto{\pgfqpoint{3.232782in}{1.654548in}}%
\pgfpathlineto{\pgfqpoint{3.246561in}{1.649830in}}%
\pgfpathlineto{\pgfqpoint{3.260344in}{1.645197in}}%
\pgfpathlineto{\pgfqpoint{3.268720in}{1.653096in}}%
\pgfpathlineto{\pgfqpoint{3.277088in}{1.661069in}}%
\pgfpathlineto{\pgfqpoint{3.285449in}{1.669114in}}%
\pgfpathlineto{\pgfqpoint{3.293802in}{1.677227in}}%
\pgfpathlineto{\pgfqpoint{3.280036in}{1.681615in}}%
\pgfpathlineto{\pgfqpoint{3.266274in}{1.686087in}}%
\pgfpathlineto{\pgfqpoint{3.252518in}{1.690642in}}%
\pgfpathlineto{\pgfqpoint{3.238766in}{1.695282in}}%
\pgfpathlineto{\pgfqpoint{3.230396in}{1.687408in}}%
\pgfpathlineto{\pgfqpoint{3.222018in}{1.679606in}}%
\pgfpathlineto{\pgfqpoint{3.213632in}{1.671880in}}%
\pgfpathlineto{\pgfqpoint{3.205238in}{1.664235in}}%
\pgfpathclose%
\pgfusepath{fill}%
\end{pgfscope}%
\begin{pgfscope}%
\pgfpathrectangle{\pgfqpoint{1.150000in}{0.150000in}}{\pgfqpoint{5.700000in}{5.700000in}}%
\pgfusepath{clip}%
\pgfsetbuttcap%
\pgfsetroundjoin%
\definecolor{currentfill}{rgb}{0.278826,0.175490,0.483397}%
\pgfsetfillcolor{currentfill}%
\pgfsetfillopacity{0.700000}%
\pgfsetlinewidth{0.000000pt}%
\definecolor{currentstroke}{rgb}{0.000000,0.000000,0.000000}%
\pgfsetstrokecolor{currentstroke}%
\pgfsetdash{}{0pt}%
\pgfpathmoveto{\pgfqpoint{4.341165in}{1.977364in}}%
\pgfpathlineto{\pgfqpoint{4.355208in}{1.978394in}}%
\pgfpathlineto{\pgfqpoint{4.369260in}{1.979496in}}%
\pgfpathlineto{\pgfqpoint{4.383322in}{1.980672in}}%
\pgfpathlineto{\pgfqpoint{4.397394in}{1.981920in}}%
\pgfpathlineto{\pgfqpoint{4.405343in}{1.991232in}}%
\pgfpathlineto{\pgfqpoint{4.413286in}{2.000469in}}%
\pgfpathlineto{\pgfqpoint{4.421222in}{2.009631in}}%
\pgfpathlineto{\pgfqpoint{4.429152in}{2.018718in}}%
\pgfpathlineto{\pgfqpoint{4.415090in}{2.017452in}}%
\pgfpathlineto{\pgfqpoint{4.401037in}{2.016258in}}%
\pgfpathlineto{\pgfqpoint{4.386994in}{2.015136in}}%
\pgfpathlineto{\pgfqpoint{4.372961in}{2.014088in}}%
\pgfpathlineto{\pgfqpoint{4.365021in}{2.005011in}}%
\pgfpathlineto{\pgfqpoint{4.357075in}{1.995865in}}%
\pgfpathlineto{\pgfqpoint{4.349123in}{1.986649in}}%
\pgfpathlineto{\pgfqpoint{4.341165in}{1.977364in}}%
\pgfpathclose%
\pgfusepath{fill}%
\end{pgfscope}%
\begin{pgfscope}%
\pgfpathrectangle{\pgfqpoint{1.150000in}{0.150000in}}{\pgfqpoint{5.700000in}{5.700000in}}%
\pgfusepath{clip}%
\pgfsetbuttcap%
\pgfsetroundjoin%
\definecolor{currentfill}{rgb}{0.192357,0.403199,0.555836}%
\pgfsetfillcolor{currentfill}%
\pgfsetfillopacity{0.700000}%
\pgfsetlinewidth{0.000000pt}%
\definecolor{currentstroke}{rgb}{0.000000,0.000000,0.000000}%
\pgfsetstrokecolor{currentstroke}%
\pgfsetdash{}{0pt}%
\pgfpathmoveto{\pgfqpoint{5.685736in}{2.527969in}}%
\pgfpathlineto{\pgfqpoint{5.700307in}{2.531830in}}%
\pgfpathlineto{\pgfqpoint{5.714891in}{2.535761in}}%
\pgfpathlineto{\pgfqpoint{5.729488in}{2.539760in}}%
\pgfpathlineto{\pgfqpoint{5.736785in}{2.543203in}}%
\pgfpathlineto{\pgfqpoint{5.744075in}{2.546628in}}%
\pgfpathlineto{\pgfqpoint{5.751358in}{2.550042in}}%
\pgfpathlineto{\pgfqpoint{5.758635in}{2.553448in}}%
\pgfpathlineto{\pgfqpoint{5.744062in}{2.549732in}}%
\pgfpathlineto{\pgfqpoint{5.729503in}{2.546084in}}%
\pgfpathlineto{\pgfqpoint{5.714957in}{2.542505in}}%
\pgfpathlineto{\pgfqpoint{5.707662in}{2.538881in}}%
\pgfpathlineto{\pgfqpoint{5.700360in}{2.535253in}}%
\pgfpathlineto{\pgfqpoint{5.693052in}{2.531618in}}%
\pgfpathlineto{\pgfqpoint{5.685736in}{2.527969in}}%
\pgfpathclose%
\pgfusepath{fill}%
\end{pgfscope}%
\begin{pgfscope}%
\pgfpathrectangle{\pgfqpoint{1.150000in}{0.150000in}}{\pgfqpoint{5.700000in}{5.700000in}}%
\pgfusepath{clip}%
\pgfsetbuttcap%
\pgfsetroundjoin%
\definecolor{currentfill}{rgb}{0.276022,0.044167,0.370164}%
\pgfsetfillcolor{currentfill}%
\pgfsetfillopacity{0.700000}%
\pgfsetlinewidth{0.000000pt}%
\definecolor{currentstroke}{rgb}{0.000000,0.000000,0.000000}%
\pgfsetstrokecolor{currentstroke}%
\pgfsetdash{}{0pt}%
\pgfpathmoveto{\pgfqpoint{3.669100in}{1.720332in}}%
\pgfpathlineto{\pgfqpoint{3.682948in}{1.718246in}}%
\pgfpathlineto{\pgfqpoint{3.696803in}{1.716239in}}%
\pgfpathlineto{\pgfqpoint{3.710665in}{1.714308in}}%
\pgfpathlineto{\pgfqpoint{3.724533in}{1.712455in}}%
\pgfpathlineto{\pgfqpoint{3.732719in}{1.722198in}}%
\pgfpathlineto{\pgfqpoint{3.740899in}{1.731939in}}%
\pgfpathlineto{\pgfqpoint{3.749073in}{1.741674in}}%
\pgfpathlineto{\pgfqpoint{3.757241in}{1.751402in}}%
\pgfpathlineto{\pgfqpoint{3.743384in}{1.753092in}}%
\pgfpathlineto{\pgfqpoint{3.729534in}{1.754858in}}%
\pgfpathlineto{\pgfqpoint{3.715691in}{1.756702in}}%
\pgfpathlineto{\pgfqpoint{3.701855in}{1.758624in}}%
\pgfpathlineto{\pgfqpoint{3.693675in}{1.749052in}}%
\pgfpathlineto{\pgfqpoint{3.685489in}{1.739478in}}%
\pgfpathlineto{\pgfqpoint{3.677298in}{1.729904in}}%
\pgfpathlineto{\pgfqpoint{3.669100in}{1.720332in}}%
\pgfpathclose%
\pgfusepath{fill}%
\end{pgfscope}%
\begin{pgfscope}%
\pgfpathrectangle{\pgfqpoint{1.150000in}{0.150000in}}{\pgfqpoint{5.700000in}{5.700000in}}%
\pgfusepath{clip}%
\pgfsetbuttcap%
\pgfsetroundjoin%
\definecolor{currentfill}{rgb}{0.277134,0.185228,0.489898}%
\pgfsetfillcolor{currentfill}%
\pgfsetfillopacity{0.700000}%
\pgfsetlinewidth{0.000000pt}%
\definecolor{currentstroke}{rgb}{0.000000,0.000000,0.000000}%
\pgfsetstrokecolor{currentstroke}%
\pgfsetdash{}{0pt}%
\pgfpathmoveto{\pgfqpoint{2.261368in}{2.044388in}}%
\pgfpathlineto{\pgfqpoint{2.275191in}{2.032374in}}%
\pgfpathlineto{\pgfqpoint{2.289012in}{2.020481in}}%
\pgfpathlineto{\pgfqpoint{2.302831in}{2.008706in}}%
\pgfpathlineto{\pgfqpoint{2.316648in}{1.997050in}}%
\pgfpathlineto{\pgfqpoint{2.325634in}{1.997503in}}%
\pgfpathlineto{\pgfqpoint{2.334602in}{1.998198in}}%
\pgfpathlineto{\pgfqpoint{2.343554in}{1.999128in}}%
\pgfpathlineto{\pgfqpoint{2.352488in}{2.000288in}}%
\pgfpathlineto{\pgfqpoint{2.338707in}{2.011585in}}%
\pgfpathlineto{\pgfqpoint{2.324925in}{2.023000in}}%
\pgfpathlineto{\pgfqpoint{2.311141in}{2.034533in}}%
\pgfpathlineto{\pgfqpoint{2.297355in}{2.046185in}}%
\pgfpathlineto{\pgfqpoint{2.288385in}{2.045377in}}%
\pgfpathlineto{\pgfqpoint{2.279397in}{2.044805in}}%
\pgfpathlineto{\pgfqpoint{2.270391in}{2.044473in}}%
\pgfpathlineto{\pgfqpoint{2.261368in}{2.044388in}}%
\pgfpathclose%
\pgfusepath{fill}%
\end{pgfscope}%
\begin{pgfscope}%
\pgfpathrectangle{\pgfqpoint{1.150000in}{0.150000in}}{\pgfqpoint{5.700000in}{5.700000in}}%
\pgfusepath{clip}%
\pgfsetbuttcap%
\pgfsetroundjoin%
\definecolor{currentfill}{rgb}{0.281412,0.155834,0.469201}%
\pgfsetfillcolor{currentfill}%
\pgfsetfillopacity{0.700000}%
\pgfsetlinewidth{0.000000pt}%
\definecolor{currentstroke}{rgb}{0.000000,0.000000,0.000000}%
\pgfsetstrokecolor{currentstroke}%
\pgfsetdash{}{0pt}%
\pgfpathmoveto{\pgfqpoint{4.253158in}{1.936297in}}%
\pgfpathlineto{\pgfqpoint{4.267173in}{1.936994in}}%
\pgfpathlineto{\pgfqpoint{4.281197in}{1.937765in}}%
\pgfpathlineto{\pgfqpoint{4.295230in}{1.938609in}}%
\pgfpathlineto{\pgfqpoint{4.309273in}{1.939526in}}%
\pgfpathlineto{\pgfqpoint{4.317255in}{1.949090in}}%
\pgfpathlineto{\pgfqpoint{4.325230in}{1.958584in}}%
\pgfpathlineto{\pgfqpoint{4.333201in}{1.968009in}}%
\pgfpathlineto{\pgfqpoint{4.341165in}{1.977364in}}%
\pgfpathlineto{\pgfqpoint{4.327131in}{1.976407in}}%
\pgfpathlineto{\pgfqpoint{4.313107in}{1.975524in}}%
\pgfpathlineto{\pgfqpoint{4.299093in}{1.974713in}}%
\pgfpathlineto{\pgfqpoint{4.285088in}{1.973976in}}%
\pgfpathlineto{\pgfqpoint{4.277114in}{1.964653in}}%
\pgfpathlineto{\pgfqpoint{4.269135in}{1.955265in}}%
\pgfpathlineto{\pgfqpoint{4.261149in}{1.945813in}}%
\pgfpathlineto{\pgfqpoint{4.253158in}{1.936297in}}%
\pgfpathclose%
\pgfusepath{fill}%
\end{pgfscope}%
\begin{pgfscope}%
\pgfpathrectangle{\pgfqpoint{1.150000in}{0.150000in}}{\pgfqpoint{5.700000in}{5.700000in}}%
\pgfusepath{clip}%
\pgfsetbuttcap%
\pgfsetroundjoin%
\definecolor{currentfill}{rgb}{0.223925,0.334994,0.548053}%
\pgfsetfillcolor{currentfill}%
\pgfsetfillopacity{0.700000}%
\pgfsetlinewidth{0.000000pt}%
\definecolor{currentstroke}{rgb}{0.000000,0.000000,0.000000}%
\pgfsetstrokecolor{currentstroke}%
\pgfsetdash{}{0pt}%
\pgfpathmoveto{\pgfqpoint{5.189419in}{2.347191in}}%
\pgfpathlineto{\pgfqpoint{5.203789in}{2.350474in}}%
\pgfpathlineto{\pgfqpoint{5.218170in}{2.353828in}}%
\pgfpathlineto{\pgfqpoint{5.232564in}{2.357252in}}%
\pgfpathlineto{\pgfqpoint{5.246970in}{2.360746in}}%
\pgfpathlineto{\pgfqpoint{5.254540in}{2.366415in}}%
\pgfpathlineto{\pgfqpoint{5.262102in}{2.372015in}}%
\pgfpathlineto{\pgfqpoint{5.269657in}{2.377550in}}%
\pgfpathlineto{\pgfqpoint{5.277204in}{2.383023in}}%
\pgfpathlineto{\pgfqpoint{5.262815in}{2.379704in}}%
\pgfpathlineto{\pgfqpoint{5.248439in}{2.376455in}}%
\pgfpathlineto{\pgfqpoint{5.234074in}{2.373275in}}%
\pgfpathlineto{\pgfqpoint{5.219721in}{2.370166in}}%
\pgfpathlineto{\pgfqpoint{5.212157in}{2.364510in}}%
\pgfpathlineto{\pgfqpoint{5.204586in}{2.358798in}}%
\pgfpathlineto{\pgfqpoint{5.197006in}{2.353026in}}%
\pgfpathlineto{\pgfqpoint{5.189419in}{2.347191in}}%
\pgfpathclose%
\pgfusepath{fill}%
\end{pgfscope}%
\begin{pgfscope}%
\pgfpathrectangle{\pgfqpoint{1.150000in}{0.150000in}}{\pgfqpoint{5.700000in}{5.700000in}}%
\pgfusepath{clip}%
\pgfsetbuttcap%
\pgfsetroundjoin%
\definecolor{currentfill}{rgb}{0.282884,0.135920,0.453427}%
\pgfsetfillcolor{currentfill}%
\pgfsetfillopacity{0.700000}%
\pgfsetlinewidth{0.000000pt}%
\definecolor{currentstroke}{rgb}{0.000000,0.000000,0.000000}%
\pgfsetstrokecolor{currentstroke}%
\pgfsetdash{}{0pt}%
\pgfpathmoveto{\pgfqpoint{4.165132in}{1.895786in}}%
\pgfpathlineto{\pgfqpoint{4.179120in}{1.896128in}}%
\pgfpathlineto{\pgfqpoint{4.193116in}{1.896544in}}%
\pgfpathlineto{\pgfqpoint{4.207121in}{1.897034in}}%
\pgfpathlineto{\pgfqpoint{4.221136in}{1.897598in}}%
\pgfpathlineto{\pgfqpoint{4.229150in}{1.907367in}}%
\pgfpathlineto{\pgfqpoint{4.237159in}{1.917073in}}%
\pgfpathlineto{\pgfqpoint{4.245161in}{1.926717in}}%
\pgfpathlineto{\pgfqpoint{4.253158in}{1.936297in}}%
\pgfpathlineto{\pgfqpoint{4.239153in}{1.935673in}}%
\pgfpathlineto{\pgfqpoint{4.225157in}{1.935123in}}%
\pgfpathlineto{\pgfqpoint{4.211170in}{1.934646in}}%
\pgfpathlineto{\pgfqpoint{4.197192in}{1.934243in}}%
\pgfpathlineto{\pgfqpoint{4.189186in}{1.924715in}}%
\pgfpathlineto{\pgfqpoint{4.181173in}{1.915130in}}%
\pgfpathlineto{\pgfqpoint{4.173156in}{1.905486in}}%
\pgfpathlineto{\pgfqpoint{4.165132in}{1.895786in}}%
\pgfpathclose%
\pgfusepath{fill}%
\end{pgfscope}%
\begin{pgfscope}%
\pgfpathrectangle{\pgfqpoint{1.150000in}{0.150000in}}{\pgfqpoint{5.700000in}{5.700000in}}%
\pgfusepath{clip}%
\pgfsetbuttcap%
\pgfsetroundjoin%
\definecolor{currentfill}{rgb}{0.276022,0.044167,0.370164}%
\pgfsetfillcolor{currentfill}%
\pgfsetfillopacity{0.700000}%
\pgfsetlinewidth{0.000000pt}%
\definecolor{currentstroke}{rgb}{0.000000,0.000000,0.000000}%
\pgfsetstrokecolor{currentstroke}%
\pgfsetdash{}{0pt}%
\pgfpathmoveto{\pgfqpoint{2.772920in}{1.744511in}}%
\pgfpathlineto{\pgfqpoint{2.786679in}{1.736608in}}%
\pgfpathlineto{\pgfqpoint{2.800440in}{1.728801in}}%
\pgfpathlineto{\pgfqpoint{2.814202in}{1.721088in}}%
\pgfpathlineto{\pgfqpoint{2.827967in}{1.713470in}}%
\pgfpathlineto{\pgfqpoint{2.836586in}{1.718261in}}%
\pgfpathlineto{\pgfqpoint{2.845193in}{1.723208in}}%
\pgfpathlineto{\pgfqpoint{2.853789in}{1.728306in}}%
\pgfpathlineto{\pgfqpoint{2.862374in}{1.733551in}}%
\pgfpathlineto{\pgfqpoint{2.848635in}{1.740860in}}%
\pgfpathlineto{\pgfqpoint{2.834898in}{1.748263in}}%
\pgfpathlineto{\pgfqpoint{2.821163in}{1.755760in}}%
\pgfpathlineto{\pgfqpoint{2.807430in}{1.763353in}}%
\pgfpathlineto{\pgfqpoint{2.798820in}{1.758410in}}%
\pgfpathlineto{\pgfqpoint{2.790199in}{1.753618in}}%
\pgfpathlineto{\pgfqpoint{2.781565in}{1.748984in}}%
\pgfpathlineto{\pgfqpoint{2.772920in}{1.744511in}}%
\pgfpathclose%
\pgfusepath{fill}%
\end{pgfscope}%
\begin{pgfscope}%
\pgfpathrectangle{\pgfqpoint{1.150000in}{0.150000in}}{\pgfqpoint{5.700000in}{5.700000in}}%
\pgfusepath{clip}%
\pgfsetbuttcap%
\pgfsetroundjoin%
\definecolor{currentfill}{rgb}{0.268510,0.009605,0.335427}%
\pgfsetfillcolor{currentfill}%
\pgfsetfillopacity{0.700000}%
\pgfsetlinewidth{0.000000pt}%
\definecolor{currentstroke}{rgb}{0.000000,0.000000,0.000000}%
\pgfsetstrokecolor{currentstroke}%
\pgfsetdash{}{0pt}%
\pgfpathmoveto{\pgfqpoint{3.348917in}{1.660506in}}%
\pgfpathlineto{\pgfqpoint{3.362709in}{1.656531in}}%
\pgfpathlineto{\pgfqpoint{3.376506in}{1.652638in}}%
\pgfpathlineto{\pgfqpoint{3.390308in}{1.648827in}}%
\pgfpathlineto{\pgfqpoint{3.404116in}{1.645097in}}%
\pgfpathlineto{\pgfqpoint{3.412430in}{1.653740in}}%
\pgfpathlineto{\pgfqpoint{3.420736in}{1.662433in}}%
\pgfpathlineto{\pgfqpoint{3.429036in}{1.671175in}}%
\pgfpathlineto{\pgfqpoint{3.437329in}{1.679961in}}%
\pgfpathlineto{\pgfqpoint{3.423537in}{1.683466in}}%
\pgfpathlineto{\pgfqpoint{3.409749in}{1.687052in}}%
\pgfpathlineto{\pgfqpoint{3.395968in}{1.690720in}}%
\pgfpathlineto{\pgfqpoint{3.382192in}{1.694469in}}%
\pgfpathlineto{\pgfqpoint{3.373884in}{1.685901in}}%
\pgfpathlineto{\pgfqpoint{3.365568in}{1.677382in}}%
\pgfpathlineto{\pgfqpoint{3.357246in}{1.668916in}}%
\pgfpathlineto{\pgfqpoint{3.348917in}{1.660506in}}%
\pgfpathclose%
\pgfusepath{fill}%
\end{pgfscope}%
\begin{pgfscope}%
\pgfpathrectangle{\pgfqpoint{1.150000in}{0.150000in}}{\pgfqpoint{5.700000in}{5.700000in}}%
\pgfusepath{clip}%
\pgfsetbuttcap%
\pgfsetroundjoin%
\definecolor{currentfill}{rgb}{0.281924,0.089666,0.412415}%
\pgfsetfillcolor{currentfill}%
\pgfsetfillopacity{0.700000}%
\pgfsetlinewidth{0.000000pt}%
\definecolor{currentstroke}{rgb}{0.000000,0.000000,0.000000}%
\pgfsetstrokecolor{currentstroke}%
\pgfsetdash{}{0pt}%
\pgfpathmoveto{\pgfqpoint{2.572876in}{1.834811in}}%
\pgfpathlineto{\pgfqpoint{2.586649in}{1.825383in}}%
\pgfpathlineto{\pgfqpoint{2.600423in}{1.816058in}}%
\pgfpathlineto{\pgfqpoint{2.614198in}{1.806835in}}%
\pgfpathlineto{\pgfqpoint{2.627974in}{1.797714in}}%
\pgfpathlineto{\pgfqpoint{2.636727in}{1.800812in}}%
\pgfpathlineto{\pgfqpoint{2.645468in}{1.804103in}}%
\pgfpathlineto{\pgfqpoint{2.654194in}{1.807580in}}%
\pgfpathlineto{\pgfqpoint{2.662908in}{1.811239in}}%
\pgfpathlineto{\pgfqpoint{2.649162in}{1.820028in}}%
\pgfpathlineto{\pgfqpoint{2.635418in}{1.828918in}}%
\pgfpathlineto{\pgfqpoint{2.621674in}{1.837910in}}%
\pgfpathlineto{\pgfqpoint{2.607931in}{1.847005in}}%
\pgfpathlineto{\pgfqpoint{2.599188in}{1.843671in}}%
\pgfpathlineto{\pgfqpoint{2.590431in}{1.840524in}}%
\pgfpathlineto{\pgfqpoint{2.581661in}{1.837569in}}%
\pgfpathlineto{\pgfqpoint{2.572876in}{1.834811in}}%
\pgfpathclose%
\pgfusepath{fill}%
\end{pgfscope}%
\begin{pgfscope}%
\pgfpathrectangle{\pgfqpoint{1.150000in}{0.150000in}}{\pgfqpoint{5.700000in}{5.700000in}}%
\pgfusepath{clip}%
\pgfsetbuttcap%
\pgfsetroundjoin%
\definecolor{currentfill}{rgb}{0.229739,0.322361,0.545706}%
\pgfsetfillcolor{currentfill}%
\pgfsetfillopacity{0.700000}%
\pgfsetlinewidth{0.000000pt}%
\definecolor{currentstroke}{rgb}{0.000000,0.000000,0.000000}%
\pgfsetstrokecolor{currentstroke}%
\pgfsetdash{}{0pt}%
\pgfpathmoveto{\pgfqpoint{5.101573in}{2.310110in}}%
\pgfpathlineto{\pgfqpoint{5.115910in}{2.313266in}}%
\pgfpathlineto{\pgfqpoint{5.130260in}{2.316492in}}%
\pgfpathlineto{\pgfqpoint{5.144621in}{2.319788in}}%
\pgfpathlineto{\pgfqpoint{5.158995in}{2.323155in}}%
\pgfpathlineto{\pgfqpoint{5.166613in}{2.329274in}}%
\pgfpathlineto{\pgfqpoint{5.174223in}{2.335318in}}%
\pgfpathlineto{\pgfqpoint{5.181825in}{2.341289in}}%
\pgfpathlineto{\pgfqpoint{5.189419in}{2.347191in}}%
\pgfpathlineto{\pgfqpoint{5.175062in}{2.343977in}}%
\pgfpathlineto{\pgfqpoint{5.160716in}{2.340834in}}%
\pgfpathlineto{\pgfqpoint{5.146383in}{2.337760in}}%
\pgfpathlineto{\pgfqpoint{5.132061in}{2.334757in}}%
\pgfpathlineto{\pgfqpoint{5.124450in}{2.328695in}}%
\pgfpathlineto{\pgfqpoint{5.116832in}{2.322568in}}%
\pgfpathlineto{\pgfqpoint{5.109206in}{2.316374in}}%
\pgfpathlineto{\pgfqpoint{5.101573in}{2.310110in}}%
\pgfpathclose%
\pgfusepath{fill}%
\end{pgfscope}%
\begin{pgfscope}%
\pgfpathrectangle{\pgfqpoint{1.150000in}{0.150000in}}{\pgfqpoint{5.700000in}{5.700000in}}%
\pgfusepath{clip}%
\pgfsetbuttcap%
\pgfsetroundjoin%
\definecolor{currentfill}{rgb}{0.283229,0.120777,0.440584}%
\pgfsetfillcolor{currentfill}%
\pgfsetfillopacity{0.700000}%
\pgfsetlinewidth{0.000000pt}%
\definecolor{currentstroke}{rgb}{0.000000,0.000000,0.000000}%
\pgfsetstrokecolor{currentstroke}%
\pgfsetdash{}{0pt}%
\pgfpathmoveto{\pgfqpoint{4.077084in}{1.856120in}}%
\pgfpathlineto{\pgfqpoint{4.091045in}{1.856085in}}%
\pgfpathlineto{\pgfqpoint{4.105015in}{1.856124in}}%
\pgfpathlineto{\pgfqpoint{4.118994in}{1.856237in}}%
\pgfpathlineto{\pgfqpoint{4.132982in}{1.856424in}}%
\pgfpathlineto{\pgfqpoint{4.141028in}{1.866347in}}%
\pgfpathlineto{\pgfqpoint{4.149068in}{1.876215in}}%
\pgfpathlineto{\pgfqpoint{4.157103in}{1.886028in}}%
\pgfpathlineto{\pgfqpoint{4.165132in}{1.895786in}}%
\pgfpathlineto{\pgfqpoint{4.151154in}{1.895517in}}%
\pgfpathlineto{\pgfqpoint{4.137184in}{1.895323in}}%
\pgfpathlineto{\pgfqpoint{4.123224in}{1.895202in}}%
\pgfpathlineto{\pgfqpoint{4.109272in}{1.895156in}}%
\pgfpathlineto{\pgfqpoint{4.101233in}{1.885472in}}%
\pgfpathlineto{\pgfqpoint{4.093189in}{1.875738in}}%
\pgfpathlineto{\pgfqpoint{4.085139in}{1.865953in}}%
\pgfpathlineto{\pgfqpoint{4.077084in}{1.856120in}}%
\pgfpathclose%
\pgfusepath{fill}%
\end{pgfscope}%
\begin{pgfscope}%
\pgfpathrectangle{\pgfqpoint{1.150000in}{0.150000in}}{\pgfqpoint{5.700000in}{5.700000in}}%
\pgfusepath{clip}%
\pgfsetbuttcap%
\pgfsetroundjoin%
\definecolor{currentfill}{rgb}{0.272594,0.025563,0.353093}%
\pgfsetfillcolor{currentfill}%
\pgfsetfillopacity{0.700000}%
\pgfsetlinewidth{0.000000pt}%
\definecolor{currentstroke}{rgb}{0.000000,0.000000,0.000000}%
\pgfsetstrokecolor{currentstroke}%
\pgfsetdash{}{0pt}%
\pgfpathmoveto{\pgfqpoint{3.580879in}{1.691973in}}%
\pgfpathlineto{\pgfqpoint{3.594713in}{1.689390in}}%
\pgfpathlineto{\pgfqpoint{3.608553in}{1.686886in}}%
\pgfpathlineto{\pgfqpoint{3.622399in}{1.684461in}}%
\pgfpathlineto{\pgfqpoint{3.636252in}{1.682113in}}%
\pgfpathlineto{\pgfqpoint{3.644473in}{1.691652in}}%
\pgfpathlineto{\pgfqpoint{3.652688in}{1.701204in}}%
\pgfpathlineto{\pgfqpoint{3.660897in}{1.710764in}}%
\pgfpathlineto{\pgfqpoint{3.669100in}{1.720332in}}%
\pgfpathlineto{\pgfqpoint{3.655259in}{1.722495in}}%
\pgfpathlineto{\pgfqpoint{3.641425in}{1.724736in}}%
\pgfpathlineto{\pgfqpoint{3.627598in}{1.727056in}}%
\pgfpathlineto{\pgfqpoint{3.613777in}{1.729454in}}%
\pgfpathlineto{\pgfqpoint{3.605562in}{1.720063in}}%
\pgfpathlineto{\pgfqpoint{3.597340in}{1.710684in}}%
\pgfpathlineto{\pgfqpoint{3.589113in}{1.701320in}}%
\pgfpathlineto{\pgfqpoint{3.580879in}{1.691973in}}%
\pgfpathclose%
\pgfusepath{fill}%
\end{pgfscope}%
\begin{pgfscope}%
\pgfpathrectangle{\pgfqpoint{1.150000in}{0.150000in}}{\pgfqpoint{5.700000in}{5.700000in}}%
\pgfusepath{clip}%
\pgfsetbuttcap%
\pgfsetroundjoin%
\definecolor{currentfill}{rgb}{0.280255,0.165693,0.476498}%
\pgfsetfillcolor{currentfill}%
\pgfsetfillopacity{0.700000}%
\pgfsetlinewidth{0.000000pt}%
\definecolor{currentstroke}{rgb}{0.000000,0.000000,0.000000}%
\pgfsetstrokecolor{currentstroke}%
\pgfsetdash{}{0pt}%
\pgfpathmoveto{\pgfqpoint{2.316648in}{1.997050in}}%
\pgfpathlineto{\pgfqpoint{2.330464in}{1.985510in}}%
\pgfpathlineto{\pgfqpoint{2.344279in}{1.974086in}}%
\pgfpathlineto{\pgfqpoint{2.358092in}{1.962778in}}%
\pgfpathlineto{\pgfqpoint{2.371903in}{1.951583in}}%
\pgfpathlineto{\pgfqpoint{2.380852in}{1.952404in}}%
\pgfpathlineto{\pgfqpoint{2.389784in}{1.953460in}}%
\pgfpathlineto{\pgfqpoint{2.398700in}{1.954746in}}%
\pgfpathlineto{\pgfqpoint{2.407599in}{1.956256in}}%
\pgfpathlineto{\pgfqpoint{2.393823in}{1.967092in}}%
\pgfpathlineto{\pgfqpoint{2.380046in}{1.978042in}}%
\pgfpathlineto{\pgfqpoint{2.366267in}{1.989107in}}%
\pgfpathlineto{\pgfqpoint{2.352488in}{2.000288in}}%
\pgfpathlineto{\pgfqpoint{2.343554in}{1.999128in}}%
\pgfpathlineto{\pgfqpoint{2.334602in}{1.998198in}}%
\pgfpathlineto{\pgfqpoint{2.325634in}{1.997503in}}%
\pgfpathlineto{\pgfqpoint{2.316648in}{1.997050in}}%
\pgfpathclose%
\pgfusepath{fill}%
\end{pgfscope}%
\begin{pgfscope}%
\pgfpathrectangle{\pgfqpoint{1.150000in}{0.150000in}}{\pgfqpoint{5.700000in}{5.700000in}}%
\pgfusepath{clip}%
\pgfsetbuttcap%
\pgfsetroundjoin%
\definecolor{currentfill}{rgb}{0.235526,0.309527,0.542944}%
\pgfsetfillcolor{currentfill}%
\pgfsetfillopacity{0.700000}%
\pgfsetlinewidth{0.000000pt}%
\definecolor{currentstroke}{rgb}{0.000000,0.000000,0.000000}%
\pgfsetstrokecolor{currentstroke}%
\pgfsetdash{}{0pt}%
\pgfpathmoveto{\pgfqpoint{5.013672in}{2.271855in}}%
\pgfpathlineto{\pgfqpoint{5.027977in}{2.274860in}}%
\pgfpathlineto{\pgfqpoint{5.042294in}{2.277936in}}%
\pgfpathlineto{\pgfqpoint{5.056623in}{2.281082in}}%
\pgfpathlineto{\pgfqpoint{5.070964in}{2.284299in}}%
\pgfpathlineto{\pgfqpoint{5.078627in}{2.290871in}}%
\pgfpathlineto{\pgfqpoint{5.086284in}{2.297361in}}%
\pgfpathlineto{\pgfqpoint{5.093932in}{2.303774in}}%
\pgfpathlineto{\pgfqpoint{5.101573in}{2.310110in}}%
\pgfpathlineto{\pgfqpoint{5.087247in}{2.307025in}}%
\pgfpathlineto{\pgfqpoint{5.072933in}{2.304010in}}%
\pgfpathlineto{\pgfqpoint{5.058631in}{2.301065in}}%
\pgfpathlineto{\pgfqpoint{5.044340in}{2.298191in}}%
\pgfpathlineto{\pgfqpoint{5.036684in}{2.291715in}}%
\pgfpathlineto{\pgfqpoint{5.029021in}{2.285169in}}%
\pgfpathlineto{\pgfqpoint{5.021350in}{2.278550in}}%
\pgfpathlineto{\pgfqpoint{5.013672in}{2.271855in}}%
\pgfpathclose%
\pgfusepath{fill}%
\end{pgfscope}%
\begin{pgfscope}%
\pgfpathrectangle{\pgfqpoint{1.150000in}{0.150000in}}{\pgfqpoint{5.700000in}{5.700000in}}%
\pgfusepath{clip}%
\pgfsetbuttcap%
\pgfsetroundjoin%
\definecolor{currentfill}{rgb}{0.282656,0.100196,0.422160}%
\pgfsetfillcolor{currentfill}%
\pgfsetfillopacity{0.700000}%
\pgfsetlinewidth{0.000000pt}%
\definecolor{currentstroke}{rgb}{0.000000,0.000000,0.000000}%
\pgfsetstrokecolor{currentstroke}%
\pgfsetdash{}{0pt}%
\pgfpathmoveto{\pgfqpoint{3.989008in}{1.817611in}}%
\pgfpathlineto{\pgfqpoint{4.002945in}{1.817175in}}%
\pgfpathlineto{\pgfqpoint{4.016891in}{1.816814in}}%
\pgfpathlineto{\pgfqpoint{4.030844in}{1.816527in}}%
\pgfpathlineto{\pgfqpoint{4.044807in}{1.816315in}}%
\pgfpathlineto{\pgfqpoint{4.052884in}{1.826335in}}%
\pgfpathlineto{\pgfqpoint{4.060956in}{1.836309in}}%
\pgfpathlineto{\pgfqpoint{4.069023in}{1.846238in}}%
\pgfpathlineto{\pgfqpoint{4.077084in}{1.856120in}}%
\pgfpathlineto{\pgfqpoint{4.063131in}{1.856230in}}%
\pgfpathlineto{\pgfqpoint{4.049187in}{1.856414in}}%
\pgfpathlineto{\pgfqpoint{4.035251in}{1.856673in}}%
\pgfpathlineto{\pgfqpoint{4.021323in}{1.857007in}}%
\pgfpathlineto{\pgfqpoint{4.013253in}{1.847219in}}%
\pgfpathlineto{\pgfqpoint{4.005177in}{1.837390in}}%
\pgfpathlineto{\pgfqpoint{3.997095in}{1.827520in}}%
\pgfpathlineto{\pgfqpoint{3.989008in}{1.817611in}}%
\pgfpathclose%
\pgfusepath{fill}%
\end{pgfscope}%
\begin{pgfscope}%
\pgfpathrectangle{\pgfqpoint{1.150000in}{0.150000in}}{\pgfqpoint{5.700000in}{5.700000in}}%
\pgfusepath{clip}%
\pgfsetbuttcap%
\pgfsetroundjoin%
\definecolor{currentfill}{rgb}{0.194100,0.399323,0.555565}%
\pgfsetfillcolor{currentfill}%
\pgfsetfillopacity{0.700000}%
\pgfsetlinewidth{0.000000pt}%
\definecolor{currentstroke}{rgb}{0.000000,0.000000,0.000000}%
\pgfsetstrokecolor{currentstroke}%
\pgfsetdash{}{0pt}%
\pgfpathmoveto{\pgfqpoint{5.598155in}{2.497342in}}%
\pgfpathlineto{\pgfqpoint{5.612697in}{2.501189in}}%
\pgfpathlineto{\pgfqpoint{5.627252in}{2.505105in}}%
\pgfpathlineto{\pgfqpoint{5.641820in}{2.509091in}}%
\pgfpathlineto{\pgfqpoint{5.656401in}{2.513145in}}%
\pgfpathlineto{\pgfqpoint{5.663746in}{2.516895in}}%
\pgfpathlineto{\pgfqpoint{5.671084in}{2.520612in}}%
\pgfpathlineto{\pgfqpoint{5.678414in}{2.524302in}}%
\pgfpathlineto{\pgfqpoint{5.685736in}{2.527969in}}%
\pgfpathlineto{\pgfqpoint{5.671179in}{2.524176in}}%
\pgfpathlineto{\pgfqpoint{5.656634in}{2.520453in}}%
\pgfpathlineto{\pgfqpoint{5.642102in}{2.516798in}}%
\pgfpathlineto{\pgfqpoint{5.627583in}{2.513212in}}%
\pgfpathlineto{\pgfqpoint{5.620237in}{2.509276in}}%
\pgfpathlineto{\pgfqpoint{5.612883in}{2.505322in}}%
\pgfpathlineto{\pgfqpoint{5.605523in}{2.501346in}}%
\pgfpathlineto{\pgfqpoint{5.598155in}{2.497342in}}%
\pgfpathclose%
\pgfusepath{fill}%
\end{pgfscope}%
\begin{pgfscope}%
\pgfpathrectangle{\pgfqpoint{1.150000in}{0.150000in}}{\pgfqpoint{5.700000in}{5.700000in}}%
\pgfusepath{clip}%
\pgfsetbuttcap%
\pgfsetroundjoin%
\definecolor{currentfill}{rgb}{0.267004,0.004874,0.329415}%
\pgfsetfillcolor{currentfill}%
\pgfsetfillopacity{0.700000}%
\pgfsetlinewidth{0.000000pt}%
\definecolor{currentstroke}{rgb}{0.000000,0.000000,0.000000}%
\pgfsetstrokecolor{currentstroke}%
\pgfsetdash{}{0pt}%
\pgfpathmoveto{\pgfqpoint{3.116470in}{1.655990in}}%
\pgfpathlineto{\pgfqpoint{3.130242in}{1.650496in}}%
\pgfpathlineto{\pgfqpoint{3.144018in}{1.645088in}}%
\pgfpathlineto{\pgfqpoint{3.157798in}{1.639766in}}%
\pgfpathlineto{\pgfqpoint{3.171582in}{1.634529in}}%
\pgfpathlineto{\pgfqpoint{3.180008in}{1.641817in}}%
\pgfpathlineto{\pgfqpoint{3.188427in}{1.649199in}}%
\pgfpathlineto{\pgfqpoint{3.196837in}{1.656673in}}%
\pgfpathlineto{\pgfqpoint{3.205238in}{1.664235in}}%
\pgfpathlineto{\pgfqpoint{3.191473in}{1.669206in}}%
\pgfpathlineto{\pgfqpoint{3.177712in}{1.674261in}}%
\pgfpathlineto{\pgfqpoint{3.163956in}{1.679402in}}%
\pgfpathlineto{\pgfqpoint{3.150204in}{1.684629in}}%
\pgfpathlineto{\pgfqpoint{3.141783in}{1.677327in}}%
\pgfpathlineto{\pgfqpoint{3.133354in}{1.670116in}}%
\pgfpathlineto{\pgfqpoint{3.124917in}{1.663003in}}%
\pgfpathlineto{\pgfqpoint{3.116470in}{1.655990in}}%
\pgfpathclose%
\pgfusepath{fill}%
\end{pgfscope}%
\begin{pgfscope}%
\pgfpathrectangle{\pgfqpoint{1.150000in}{0.150000in}}{\pgfqpoint{5.700000in}{5.700000in}}%
\pgfusepath{clip}%
\pgfsetbuttcap%
\pgfsetroundjoin%
\definecolor{currentfill}{rgb}{0.241237,0.296485,0.539709}%
\pgfsetfillcolor{currentfill}%
\pgfsetfillopacity{0.700000}%
\pgfsetlinewidth{0.000000pt}%
\definecolor{currentstroke}{rgb}{0.000000,0.000000,0.000000}%
\pgfsetstrokecolor{currentstroke}%
\pgfsetdash{}{0pt}%
\pgfpathmoveto{\pgfqpoint{4.925724in}{2.232520in}}%
\pgfpathlineto{\pgfqpoint{4.939997in}{2.235352in}}%
\pgfpathlineto{\pgfqpoint{4.954281in}{2.238255in}}%
\pgfpathlineto{\pgfqpoint{4.968577in}{2.241229in}}%
\pgfpathlineto{\pgfqpoint{4.982884in}{2.244273in}}%
\pgfpathlineto{\pgfqpoint{4.990593in}{2.251294in}}%
\pgfpathlineto{\pgfqpoint{4.998293in}{2.258229in}}%
\pgfpathlineto{\pgfqpoint{5.005986in}{2.265082in}}%
\pgfpathlineto{\pgfqpoint{5.013672in}{2.271855in}}%
\pgfpathlineto{\pgfqpoint{4.999378in}{2.268921in}}%
\pgfpathlineto{\pgfqpoint{4.985096in}{2.266057in}}%
\pgfpathlineto{\pgfqpoint{4.970825in}{2.263263in}}%
\pgfpathlineto{\pgfqpoint{4.956566in}{2.260540in}}%
\pgfpathlineto{\pgfqpoint{4.948867in}{2.253650in}}%
\pgfpathlineto{\pgfqpoint{4.941160in}{2.246685in}}%
\pgfpathlineto{\pgfqpoint{4.933446in}{2.239642in}}%
\pgfpathlineto{\pgfqpoint{4.925724in}{2.232520in}}%
\pgfpathclose%
\pgfusepath{fill}%
\end{pgfscope}%
\begin{pgfscope}%
\pgfpathrectangle{\pgfqpoint{1.150000in}{0.150000in}}{\pgfqpoint{5.700000in}{5.700000in}}%
\pgfusepath{clip}%
\pgfsetbuttcap%
\pgfsetroundjoin%
\definecolor{currentfill}{rgb}{0.269944,0.014625,0.341379}%
\pgfsetfillcolor{currentfill}%
\pgfsetfillopacity{0.700000}%
\pgfsetlinewidth{0.000000pt}%
\definecolor{currentstroke}{rgb}{0.000000,0.000000,0.000000}%
\pgfsetstrokecolor{currentstroke}%
\pgfsetdash{}{0pt}%
\pgfpathmoveto{\pgfqpoint{2.972385in}{1.678413in}}%
\pgfpathlineto{\pgfqpoint{2.986149in}{1.671930in}}%
\pgfpathlineto{\pgfqpoint{2.999917in}{1.665536in}}%
\pgfpathlineto{\pgfqpoint{3.013688in}{1.659231in}}%
\pgfpathlineto{\pgfqpoint{3.027462in}{1.653014in}}%
\pgfpathlineto{\pgfqpoint{3.035966in}{1.659289in}}%
\pgfpathlineto{\pgfqpoint{3.044461in}{1.665687in}}%
\pgfpathlineto{\pgfqpoint{3.052946in}{1.672202in}}%
\pgfpathlineto{\pgfqpoint{3.061422in}{1.678831in}}%
\pgfpathlineto{\pgfqpoint{3.047670in}{1.684760in}}%
\pgfpathlineto{\pgfqpoint{3.033921in}{1.690778in}}%
\pgfpathlineto{\pgfqpoint{3.020175in}{1.696884in}}%
\pgfpathlineto{\pgfqpoint{3.006433in}{1.703079in}}%
\pgfpathlineto{\pgfqpoint{2.997935in}{1.696730in}}%
\pgfpathlineto{\pgfqpoint{2.989428in}{1.690500in}}%
\pgfpathlineto{\pgfqpoint{2.980911in}{1.684393in}}%
\pgfpathlineto{\pgfqpoint{2.972385in}{1.678413in}}%
\pgfpathclose%
\pgfusepath{fill}%
\end{pgfscope}%
\begin{pgfscope}%
\pgfpathrectangle{\pgfqpoint{1.150000in}{0.150000in}}{\pgfqpoint{5.700000in}{5.700000in}}%
\pgfusepath{clip}%
\pgfsetbuttcap%
\pgfsetroundjoin%
\definecolor{currentfill}{rgb}{0.281446,0.084320,0.407414}%
\pgfsetfillcolor{currentfill}%
\pgfsetfillopacity{0.700000}%
\pgfsetlinewidth{0.000000pt}%
\definecolor{currentstroke}{rgb}{0.000000,0.000000,0.000000}%
\pgfsetstrokecolor{currentstroke}%
\pgfsetdash{}{0pt}%
\pgfpathmoveto{\pgfqpoint{3.900898in}{1.780590in}}%
\pgfpathlineto{\pgfqpoint{3.914812in}{1.779731in}}%
\pgfpathlineto{\pgfqpoint{3.928735in}{1.778947in}}%
\pgfpathlineto{\pgfqpoint{3.942665in}{1.778238in}}%
\pgfpathlineto{\pgfqpoint{3.956604in}{1.777604in}}%
\pgfpathlineto{\pgfqpoint{3.964713in}{1.787658in}}%
\pgfpathlineto{\pgfqpoint{3.972817in}{1.797678in}}%
\pgfpathlineto{\pgfqpoint{3.980915in}{1.807663in}}%
\pgfpathlineto{\pgfqpoint{3.989008in}{1.817611in}}%
\pgfpathlineto{\pgfqpoint{3.975079in}{1.818121in}}%
\pgfpathlineto{\pgfqpoint{3.961158in}{1.818707in}}%
\pgfpathlineto{\pgfqpoint{3.947246in}{1.819369in}}%
\pgfpathlineto{\pgfqpoint{3.933341in}{1.820105in}}%
\pgfpathlineto{\pgfqpoint{3.925238in}{1.810273in}}%
\pgfpathlineto{\pgfqpoint{3.917130in}{1.800408in}}%
\pgfpathlineto{\pgfqpoint{3.909017in}{1.790514in}}%
\pgfpathlineto{\pgfqpoint{3.900898in}{1.780590in}}%
\pgfpathclose%
\pgfusepath{fill}%
\end{pgfscope}%
\begin{pgfscope}%
\pgfpathrectangle{\pgfqpoint{1.150000in}{0.150000in}}{\pgfqpoint{5.700000in}{5.700000in}}%
\pgfusepath{clip}%
\pgfsetbuttcap%
\pgfsetroundjoin%
\definecolor{currentfill}{rgb}{0.269944,0.014625,0.341379}%
\pgfsetfillcolor{currentfill}%
\pgfsetfillopacity{0.700000}%
\pgfsetlinewidth{0.000000pt}%
\definecolor{currentstroke}{rgb}{0.000000,0.000000,0.000000}%
\pgfsetstrokecolor{currentstroke}%
\pgfsetdash{}{0pt}%
\pgfpathmoveto{\pgfqpoint{3.492558in}{1.666746in}}%
\pgfpathlineto{\pgfqpoint{3.506380in}{1.663643in}}%
\pgfpathlineto{\pgfqpoint{3.520208in}{1.660618in}}%
\pgfpathlineto{\pgfqpoint{3.534042in}{1.657674in}}%
\pgfpathlineto{\pgfqpoint{3.547883in}{1.654808in}}%
\pgfpathlineto{\pgfqpoint{3.556141in}{1.664060in}}%
\pgfpathlineto{\pgfqpoint{3.564393in}{1.673340in}}%
\pgfpathlineto{\pgfqpoint{3.572639in}{1.682645in}}%
\pgfpathlineto{\pgfqpoint{3.580879in}{1.691973in}}%
\pgfpathlineto{\pgfqpoint{3.567052in}{1.694634in}}%
\pgfpathlineto{\pgfqpoint{3.553231in}{1.697374in}}%
\pgfpathlineto{\pgfqpoint{3.539417in}{1.700193in}}%
\pgfpathlineto{\pgfqpoint{3.525608in}{1.703092in}}%
\pgfpathlineto{\pgfqpoint{3.517355in}{1.693961in}}%
\pgfpathlineto{\pgfqpoint{3.509096in}{1.684858in}}%
\pgfpathlineto{\pgfqpoint{3.500830in}{1.675786in}}%
\pgfpathlineto{\pgfqpoint{3.492558in}{1.666746in}}%
\pgfpathclose%
\pgfusepath{fill}%
\end{pgfscope}%
\begin{pgfscope}%
\pgfpathrectangle{\pgfqpoint{1.150000in}{0.150000in}}{\pgfqpoint{5.700000in}{5.700000in}}%
\pgfusepath{clip}%
\pgfsetbuttcap%
\pgfsetroundjoin%
\definecolor{currentfill}{rgb}{0.248629,0.278775,0.534556}%
\pgfsetfillcolor{currentfill}%
\pgfsetfillopacity{0.700000}%
\pgfsetlinewidth{0.000000pt}%
\definecolor{currentstroke}{rgb}{0.000000,0.000000,0.000000}%
\pgfsetstrokecolor{currentstroke}%
\pgfsetdash{}{0pt}%
\pgfpathmoveto{\pgfqpoint{4.837736in}{2.192222in}}%
\pgfpathlineto{\pgfqpoint{4.851976in}{2.194859in}}%
\pgfpathlineto{\pgfqpoint{4.866228in}{2.197566in}}%
\pgfpathlineto{\pgfqpoint{4.880490in}{2.200345in}}%
\pgfpathlineto{\pgfqpoint{4.894765in}{2.203195in}}%
\pgfpathlineto{\pgfqpoint{4.902515in}{2.210655in}}%
\pgfpathlineto{\pgfqpoint{4.910259in}{2.218028in}}%
\pgfpathlineto{\pgfqpoint{4.917995in}{2.225316in}}%
\pgfpathlineto{\pgfqpoint{4.925724in}{2.232520in}}%
\pgfpathlineto{\pgfqpoint{4.911463in}{2.229759in}}%
\pgfpathlineto{\pgfqpoint{4.897212in}{2.227068in}}%
\pgfpathlineto{\pgfqpoint{4.882974in}{2.224449in}}%
\pgfpathlineto{\pgfqpoint{4.868746in}{2.221900in}}%
\pgfpathlineto{\pgfqpoint{4.861004in}{2.214600in}}%
\pgfpathlineto{\pgfqpoint{4.853255in}{2.207222in}}%
\pgfpathlineto{\pgfqpoint{4.845499in}{2.199763in}}%
\pgfpathlineto{\pgfqpoint{4.837736in}{2.192222in}}%
\pgfpathclose%
\pgfusepath{fill}%
\end{pgfscope}%
\begin{pgfscope}%
\pgfpathrectangle{\pgfqpoint{1.150000in}{0.150000in}}{\pgfqpoint{5.700000in}{5.700000in}}%
\pgfusepath{clip}%
\pgfsetbuttcap%
\pgfsetroundjoin%
\definecolor{currentfill}{rgb}{0.281887,0.150881,0.465405}%
\pgfsetfillcolor{currentfill}%
\pgfsetfillopacity{0.700000}%
\pgfsetlinewidth{0.000000pt}%
\definecolor{currentstroke}{rgb}{0.000000,0.000000,0.000000}%
\pgfsetstrokecolor{currentstroke}%
\pgfsetdash{}{0pt}%
\pgfpathmoveto{\pgfqpoint{2.371903in}{1.951583in}}%
\pgfpathlineto{\pgfqpoint{2.385714in}{1.940502in}}%
\pgfpathlineto{\pgfqpoint{2.399524in}{1.929534in}}%
\pgfpathlineto{\pgfqpoint{2.413332in}{1.918677in}}%
\pgfpathlineto{\pgfqpoint{2.427140in}{1.907931in}}%
\pgfpathlineto{\pgfqpoint{2.436053in}{1.909118in}}%
\pgfpathlineto{\pgfqpoint{2.444950in}{1.910534in}}%
\pgfpathlineto{\pgfqpoint{2.453831in}{1.912175in}}%
\pgfpathlineto{\pgfqpoint{2.462695in}{1.914034in}}%
\pgfpathlineto{\pgfqpoint{2.448922in}{1.924423in}}%
\pgfpathlineto{\pgfqpoint{2.435148in}{1.934922in}}%
\pgfpathlineto{\pgfqpoint{2.421374in}{1.945533in}}%
\pgfpathlineto{\pgfqpoint{2.407599in}{1.956256in}}%
\pgfpathlineto{\pgfqpoint{2.398700in}{1.954746in}}%
\pgfpathlineto{\pgfqpoint{2.389784in}{1.953460in}}%
\pgfpathlineto{\pgfqpoint{2.380852in}{1.952404in}}%
\pgfpathlineto{\pgfqpoint{2.371903in}{1.951583in}}%
\pgfpathclose%
\pgfusepath{fill}%
\end{pgfscope}%
\begin{pgfscope}%
\pgfpathrectangle{\pgfqpoint{1.150000in}{0.150000in}}{\pgfqpoint{5.700000in}{5.700000in}}%
\pgfusepath{clip}%
\pgfsetbuttcap%
\pgfsetroundjoin%
\definecolor{currentfill}{rgb}{0.267004,0.004874,0.329415}%
\pgfsetfillcolor{currentfill}%
\pgfsetfillopacity{0.700000}%
\pgfsetlinewidth{0.000000pt}%
\definecolor{currentstroke}{rgb}{0.000000,0.000000,0.000000}%
\pgfsetstrokecolor{currentstroke}%
\pgfsetdash{}{0pt}%
\pgfpathmoveto{\pgfqpoint{3.260344in}{1.645197in}}%
\pgfpathlineto{\pgfqpoint{3.274133in}{1.640647in}}%
\pgfpathlineto{\pgfqpoint{3.287926in}{1.636179in}}%
\pgfpathlineto{\pgfqpoint{3.301724in}{1.631795in}}%
\pgfpathlineto{\pgfqpoint{3.315527in}{1.627492in}}%
\pgfpathlineto{\pgfqpoint{3.323886in}{1.635645in}}%
\pgfpathlineto{\pgfqpoint{3.332237in}{1.643867in}}%
\pgfpathlineto{\pgfqpoint{3.340580in}{1.652155in}}%
\pgfpathlineto{\pgfqpoint{3.348917in}{1.660506in}}%
\pgfpathlineto{\pgfqpoint{3.335130in}{1.664562in}}%
\pgfpathlineto{\pgfqpoint{3.321349in}{1.668701in}}%
\pgfpathlineto{\pgfqpoint{3.307573in}{1.672923in}}%
\pgfpathlineto{\pgfqpoint{3.293802in}{1.677227in}}%
\pgfpathlineto{\pgfqpoint{3.285449in}{1.669114in}}%
\pgfpathlineto{\pgfqpoint{3.277088in}{1.661069in}}%
\pgfpathlineto{\pgfqpoint{3.268720in}{1.653096in}}%
\pgfpathlineto{\pgfqpoint{3.260344in}{1.645197in}}%
\pgfpathclose%
\pgfusepath{fill}%
\end{pgfscope}%
\begin{pgfscope}%
\pgfpathrectangle{\pgfqpoint{1.150000in}{0.150000in}}{\pgfqpoint{5.700000in}{5.700000in}}%
\pgfusepath{clip}%
\pgfsetbuttcap%
\pgfsetroundjoin%
\definecolor{currentfill}{rgb}{0.280894,0.078907,0.402329}%
\pgfsetfillcolor{currentfill}%
\pgfsetfillopacity{0.700000}%
\pgfsetlinewidth{0.000000pt}%
\definecolor{currentstroke}{rgb}{0.000000,0.000000,0.000000}%
\pgfsetstrokecolor{currentstroke}%
\pgfsetdash{}{0pt}%
\pgfpathmoveto{\pgfqpoint{2.627974in}{1.797714in}}%
\pgfpathlineto{\pgfqpoint{2.641750in}{1.788694in}}%
\pgfpathlineto{\pgfqpoint{2.655527in}{1.779774in}}%
\pgfpathlineto{\pgfqpoint{2.669305in}{1.770954in}}%
\pgfpathlineto{\pgfqpoint{2.683085in}{1.762234in}}%
\pgfpathlineto{\pgfqpoint{2.691809in}{1.765673in}}%
\pgfpathlineto{\pgfqpoint{2.700519in}{1.769298in}}%
\pgfpathlineto{\pgfqpoint{2.709217in}{1.773105in}}%
\pgfpathlineto{\pgfqpoint{2.717902in}{1.777088in}}%
\pgfpathlineto{\pgfqpoint{2.704152in}{1.785477in}}%
\pgfpathlineto{\pgfqpoint{2.690403in}{1.793964in}}%
\pgfpathlineto{\pgfqpoint{2.676655in}{1.802552in}}%
\pgfpathlineto{\pgfqpoint{2.662908in}{1.811239in}}%
\pgfpathlineto{\pgfqpoint{2.654194in}{1.807580in}}%
\pgfpathlineto{\pgfqpoint{2.645468in}{1.804103in}}%
\pgfpathlineto{\pgfqpoint{2.636727in}{1.800812in}}%
\pgfpathlineto{\pgfqpoint{2.627974in}{1.797714in}}%
\pgfpathclose%
\pgfusepath{fill}%
\end{pgfscope}%
\begin{pgfscope}%
\pgfpathrectangle{\pgfqpoint{1.150000in}{0.150000in}}{\pgfqpoint{5.700000in}{5.700000in}}%
\pgfusepath{clip}%
\pgfsetbuttcap%
\pgfsetroundjoin%
\definecolor{currentfill}{rgb}{0.199430,0.387607,0.554642}%
\pgfsetfillcolor{currentfill}%
\pgfsetfillopacity{0.700000}%
\pgfsetlinewidth{0.000000pt}%
\definecolor{currentstroke}{rgb}{0.000000,0.000000,0.000000}%
\pgfsetstrokecolor{currentstroke}%
\pgfsetdash{}{0pt}%
\pgfpathmoveto{\pgfqpoint{5.510481in}{2.465313in}}%
\pgfpathlineto{\pgfqpoint{5.524993in}{2.469123in}}%
\pgfpathlineto{\pgfqpoint{5.539518in}{2.473003in}}%
\pgfpathlineto{\pgfqpoint{5.554056in}{2.476951in}}%
\pgfpathlineto{\pgfqpoint{5.568607in}{2.480970in}}%
\pgfpathlineto{\pgfqpoint{5.576006in}{2.485125in}}%
\pgfpathlineto{\pgfqpoint{5.583397in}{2.489237in}}%
\pgfpathlineto{\pgfqpoint{5.590780in}{2.493308in}}%
\pgfpathlineto{\pgfqpoint{5.598155in}{2.497342in}}%
\pgfpathlineto{\pgfqpoint{5.583626in}{2.493565in}}%
\pgfpathlineto{\pgfqpoint{5.569110in}{2.489856in}}%
\pgfpathlineto{\pgfqpoint{5.554606in}{2.486217in}}%
\pgfpathlineto{\pgfqpoint{5.540115in}{2.482646in}}%
\pgfpathlineto{\pgfqpoint{5.532718in}{2.478364in}}%
\pgfpathlineto{\pgfqpoint{5.525313in}{2.474050in}}%
\pgfpathlineto{\pgfqpoint{5.517901in}{2.469702in}}%
\pgfpathlineto{\pgfqpoint{5.510481in}{2.465313in}}%
\pgfpathclose%
\pgfusepath{fill}%
\end{pgfscope}%
\begin{pgfscope}%
\pgfpathrectangle{\pgfqpoint{1.150000in}{0.150000in}}{\pgfqpoint{5.700000in}{5.700000in}}%
\pgfusepath{clip}%
\pgfsetbuttcap%
\pgfsetroundjoin%
\definecolor{currentfill}{rgb}{0.279566,0.067836,0.391917}%
\pgfsetfillcolor{currentfill}%
\pgfsetfillopacity{0.700000}%
\pgfsetlinewidth{0.000000pt}%
\definecolor{currentstroke}{rgb}{0.000000,0.000000,0.000000}%
\pgfsetstrokecolor{currentstroke}%
\pgfsetdash{}{0pt}%
\pgfpathmoveto{\pgfqpoint{3.812743in}{1.745413in}}%
\pgfpathlineto{\pgfqpoint{3.826637in}{1.744107in}}%
\pgfpathlineto{\pgfqpoint{3.840539in}{1.742877in}}%
\pgfpathlineto{\pgfqpoint{3.854448in}{1.741723in}}%
\pgfpathlineto{\pgfqpoint{3.868365in}{1.740644in}}%
\pgfpathlineto{\pgfqpoint{3.876507in}{1.750665in}}%
\pgfpathlineto{\pgfqpoint{3.884643in}{1.760664in}}%
\pgfpathlineto{\pgfqpoint{3.892773in}{1.770640in}}%
\pgfpathlineto{\pgfqpoint{3.900898in}{1.780590in}}%
\pgfpathlineto{\pgfqpoint{3.886991in}{1.781525in}}%
\pgfpathlineto{\pgfqpoint{3.873092in}{1.782536in}}%
\pgfpathlineto{\pgfqpoint{3.859201in}{1.783623in}}%
\pgfpathlineto{\pgfqpoint{3.845317in}{1.784786in}}%
\pgfpathlineto{\pgfqpoint{3.837182in}{1.774971in}}%
\pgfpathlineto{\pgfqpoint{3.829041in}{1.765136in}}%
\pgfpathlineto{\pgfqpoint{3.820895in}{1.755283in}}%
\pgfpathlineto{\pgfqpoint{3.812743in}{1.745413in}}%
\pgfpathclose%
\pgfusepath{fill}%
\end{pgfscope}%
\begin{pgfscope}%
\pgfpathrectangle{\pgfqpoint{1.150000in}{0.150000in}}{\pgfqpoint{5.700000in}{5.700000in}}%
\pgfusepath{clip}%
\pgfsetbuttcap%
\pgfsetroundjoin%
\definecolor{currentfill}{rgb}{0.255645,0.260703,0.528312}%
\pgfsetfillcolor{currentfill}%
\pgfsetfillopacity{0.700000}%
\pgfsetlinewidth{0.000000pt}%
\definecolor{currentstroke}{rgb}{0.000000,0.000000,0.000000}%
\pgfsetstrokecolor{currentstroke}%
\pgfsetdash{}{0pt}%
\pgfpathmoveto{\pgfqpoint{4.749714in}{2.151100in}}%
\pgfpathlineto{\pgfqpoint{4.763922in}{2.153519in}}%
\pgfpathlineto{\pgfqpoint{4.778141in}{2.156009in}}%
\pgfpathlineto{\pgfqpoint{4.792371in}{2.158570in}}%
\pgfpathlineto{\pgfqpoint{4.806611in}{2.161202in}}%
\pgfpathlineto{\pgfqpoint{4.814403in}{2.169089in}}%
\pgfpathlineto{\pgfqpoint{4.822188in}{2.176886in}}%
\pgfpathlineto{\pgfqpoint{4.829965in}{2.184597in}}%
\pgfpathlineto{\pgfqpoint{4.837736in}{2.192222in}}%
\pgfpathlineto{\pgfqpoint{4.823507in}{2.189657in}}%
\pgfpathlineto{\pgfqpoint{4.809289in}{2.187162in}}%
\pgfpathlineto{\pgfqpoint{4.795082in}{2.184739in}}%
\pgfpathlineto{\pgfqpoint{4.780886in}{2.182387in}}%
\pgfpathlineto{\pgfqpoint{4.773103in}{2.174688in}}%
\pgfpathlineto{\pgfqpoint{4.765314in}{2.166908in}}%
\pgfpathlineto{\pgfqpoint{4.757517in}{2.159046in}}%
\pgfpathlineto{\pgfqpoint{4.749714in}{2.151100in}}%
\pgfpathclose%
\pgfusepath{fill}%
\end{pgfscope}%
\begin{pgfscope}%
\pgfpathrectangle{\pgfqpoint{1.150000in}{0.150000in}}{\pgfqpoint{5.700000in}{5.700000in}}%
\pgfusepath{clip}%
\pgfsetbuttcap%
\pgfsetroundjoin%
\definecolor{currentfill}{rgb}{0.274952,0.037752,0.364543}%
\pgfsetfillcolor{currentfill}%
\pgfsetfillopacity{0.700000}%
\pgfsetlinewidth{0.000000pt}%
\definecolor{currentstroke}{rgb}{0.000000,0.000000,0.000000}%
\pgfsetstrokecolor{currentstroke}%
\pgfsetdash{}{0pt}%
\pgfpathmoveto{\pgfqpoint{2.827967in}{1.713470in}}%
\pgfpathlineto{\pgfqpoint{2.841734in}{1.705946in}}%
\pgfpathlineto{\pgfqpoint{2.855504in}{1.698515in}}%
\pgfpathlineto{\pgfqpoint{2.869276in}{1.691177in}}%
\pgfpathlineto{\pgfqpoint{2.883050in}{1.683931in}}%
\pgfpathlineto{\pgfqpoint{2.891643in}{1.689038in}}%
\pgfpathlineto{\pgfqpoint{2.900226in}{1.694297in}}%
\pgfpathlineto{\pgfqpoint{2.908797in}{1.699702in}}%
\pgfpathlineto{\pgfqpoint{2.917357in}{1.705249in}}%
\pgfpathlineto{\pgfqpoint{2.903608in}{1.712186in}}%
\pgfpathlineto{\pgfqpoint{2.889861in}{1.719215in}}%
\pgfpathlineto{\pgfqpoint{2.876116in}{1.726336in}}%
\pgfpathlineto{\pgfqpoint{2.862374in}{1.733551in}}%
\pgfpathlineto{\pgfqpoint{2.853789in}{1.728306in}}%
\pgfpathlineto{\pgfqpoint{2.845193in}{1.723208in}}%
\pgfpathlineto{\pgfqpoint{2.836586in}{1.718261in}}%
\pgfpathlineto{\pgfqpoint{2.827967in}{1.713470in}}%
\pgfpathclose%
\pgfusepath{fill}%
\end{pgfscope}%
\begin{pgfscope}%
\pgfpathrectangle{\pgfqpoint{1.150000in}{0.150000in}}{\pgfqpoint{5.700000in}{5.700000in}}%
\pgfusepath{clip}%
\pgfsetbuttcap%
\pgfsetroundjoin%
\definecolor{currentfill}{rgb}{0.262138,0.242286,0.520837}%
\pgfsetfillcolor{currentfill}%
\pgfsetfillopacity{0.700000}%
\pgfsetlinewidth{0.000000pt}%
\definecolor{currentstroke}{rgb}{0.000000,0.000000,0.000000}%
\pgfsetstrokecolor{currentstroke}%
\pgfsetdash{}{0pt}%
\pgfpathmoveto{\pgfqpoint{4.661665in}{2.109316in}}%
\pgfpathlineto{\pgfqpoint{4.675840in}{2.111494in}}%
\pgfpathlineto{\pgfqpoint{4.690026in}{2.113743in}}%
\pgfpathlineto{\pgfqpoint{4.704223in}{2.116064in}}%
\pgfpathlineto{\pgfqpoint{4.718431in}{2.118457in}}%
\pgfpathlineto{\pgfqpoint{4.726262in}{2.126749in}}%
\pgfpathlineto{\pgfqpoint{4.734087in}{2.134953in}}%
\pgfpathlineto{\pgfqpoint{4.741904in}{2.143070in}}%
\pgfpathlineto{\pgfqpoint{4.749714in}{2.151100in}}%
\pgfpathlineto{\pgfqpoint{4.735517in}{2.148753in}}%
\pgfpathlineto{\pgfqpoint{4.721331in}{2.146478in}}%
\pgfpathlineto{\pgfqpoint{4.707156in}{2.144273in}}%
\pgfpathlineto{\pgfqpoint{4.692992in}{2.142141in}}%
\pgfpathlineto{\pgfqpoint{4.685170in}{2.134057in}}%
\pgfpathlineto{\pgfqpoint{4.677342in}{2.125893in}}%
\pgfpathlineto{\pgfqpoint{4.669507in}{2.117646in}}%
\pgfpathlineto{\pgfqpoint{4.661665in}{2.109316in}}%
\pgfpathclose%
\pgfusepath{fill}%
\end{pgfscope}%
\begin{pgfscope}%
\pgfpathrectangle{\pgfqpoint{1.150000in}{0.150000in}}{\pgfqpoint{5.700000in}{5.700000in}}%
\pgfusepath{clip}%
\pgfsetbuttcap%
\pgfsetroundjoin%
\definecolor{currentfill}{rgb}{0.233603,0.313828,0.543914}%
\pgfsetfillcolor{currentfill}%
\pgfsetfillopacity{0.700000}%
\pgfsetlinewidth{0.000000pt}%
\definecolor{currentstroke}{rgb}{0.000000,0.000000,0.000000}%
\pgfsetstrokecolor{currentstroke}%
\pgfsetdash{}{0pt}%
\pgfpathmoveto{\pgfqpoint{1.947142in}{2.321427in}}%
\pgfpathlineto{\pgfqpoint{1.961079in}{2.306398in}}%
\pgfpathlineto{\pgfqpoint{1.975012in}{2.291512in}}%
\pgfpathlineto{\pgfqpoint{1.988939in}{2.276768in}}%
\pgfpathlineto{\pgfqpoint{2.002861in}{2.262164in}}%
\pgfpathlineto{\pgfqpoint{2.012133in}{2.259624in}}%
\pgfpathlineto{\pgfqpoint{2.021384in}{2.257376in}}%
\pgfpathlineto{\pgfqpoint{2.030612in}{2.255416in}}%
\pgfpathlineto{\pgfqpoint{2.039819in}{2.253737in}}%
\pgfpathlineto{\pgfqpoint{2.025942in}{2.267949in}}%
\pgfpathlineto{\pgfqpoint{2.012059in}{2.282302in}}%
\pgfpathlineto{\pgfqpoint{1.998172in}{2.296796in}}%
\pgfpathlineto{\pgfqpoint{1.984281in}{2.311433in}}%
\pgfpathlineto{\pgfqpoint{1.975029in}{2.313495in}}%
\pgfpathlineto{\pgfqpoint{1.965756in}{2.315844in}}%
\pgfpathlineto{\pgfqpoint{1.956460in}{2.318486in}}%
\pgfpathlineto{\pgfqpoint{1.947142in}{2.321427in}}%
\pgfpathclose%
\pgfusepath{fill}%
\end{pgfscope}%
\begin{pgfscope}%
\pgfpathrectangle{\pgfqpoint{1.150000in}{0.150000in}}{\pgfqpoint{5.700000in}{5.700000in}}%
\pgfusepath{clip}%
\pgfsetbuttcap%
\pgfsetroundjoin%
\definecolor{currentfill}{rgb}{0.266580,0.228262,0.514349}%
\pgfsetfillcolor{currentfill}%
\pgfsetfillopacity{0.700000}%
\pgfsetlinewidth{0.000000pt}%
\definecolor{currentstroke}{rgb}{0.000000,0.000000,0.000000}%
\pgfsetstrokecolor{currentstroke}%
\pgfsetdash{}{0pt}%
\pgfpathmoveto{\pgfqpoint{4.573592in}{2.067051in}}%
\pgfpathlineto{\pgfqpoint{4.587736in}{2.068966in}}%
\pgfpathlineto{\pgfqpoint{4.601890in}{2.070952in}}%
\pgfpathlineto{\pgfqpoint{4.616055in}{2.073011in}}%
\pgfpathlineto{\pgfqpoint{4.630230in}{2.075141in}}%
\pgfpathlineto{\pgfqpoint{4.638099in}{2.083815in}}%
\pgfpathlineto{\pgfqpoint{4.645961in}{2.092401in}}%
\pgfpathlineto{\pgfqpoint{4.653816in}{2.100901in}}%
\pgfpathlineto{\pgfqpoint{4.661665in}{2.109316in}}%
\pgfpathlineto{\pgfqpoint{4.647500in}{2.107210in}}%
\pgfpathlineto{\pgfqpoint{4.633346in}{2.105175in}}%
\pgfpathlineto{\pgfqpoint{4.619202in}{2.103213in}}%
\pgfpathlineto{\pgfqpoint{4.605069in}{2.101322in}}%
\pgfpathlineto{\pgfqpoint{4.597209in}{2.092875in}}%
\pgfpathlineto{\pgfqpoint{4.589344in}{2.084349in}}%
\pgfpathlineto{\pgfqpoint{4.581471in}{2.075741in}}%
\pgfpathlineto{\pgfqpoint{4.573592in}{2.067051in}}%
\pgfpathclose%
\pgfusepath{fill}%
\end{pgfscope}%
\begin{pgfscope}%
\pgfpathrectangle{\pgfqpoint{1.150000in}{0.150000in}}{\pgfqpoint{5.700000in}{5.700000in}}%
\pgfusepath{clip}%
\pgfsetbuttcap%
\pgfsetroundjoin%
\definecolor{currentfill}{rgb}{0.204903,0.375746,0.553533}%
\pgfsetfillcolor{currentfill}%
\pgfsetfillopacity{0.700000}%
\pgfsetlinewidth{0.000000pt}%
\definecolor{currentstroke}{rgb}{0.000000,0.000000,0.000000}%
\pgfsetstrokecolor{currentstroke}%
\pgfsetdash{}{0pt}%
\pgfpathmoveto{\pgfqpoint{5.422721in}{2.431864in}}%
\pgfpathlineto{\pgfqpoint{5.437202in}{2.435614in}}%
\pgfpathlineto{\pgfqpoint{5.451697in}{2.439434in}}%
\pgfpathlineto{\pgfqpoint{5.466204in}{2.443324in}}%
\pgfpathlineto{\pgfqpoint{5.480724in}{2.447284in}}%
\pgfpathlineto{\pgfqpoint{5.488175in}{2.451870in}}%
\pgfpathlineto{\pgfqpoint{5.495618in}{2.456402in}}%
\pgfpathlineto{\pgfqpoint{5.503053in}{2.460881in}}%
\pgfpathlineto{\pgfqpoint{5.510481in}{2.465313in}}%
\pgfpathlineto{\pgfqpoint{5.495981in}{2.461573in}}%
\pgfpathlineto{\pgfqpoint{5.481495in}{2.457902in}}%
\pgfpathlineto{\pgfqpoint{5.467020in}{2.454300in}}%
\pgfpathlineto{\pgfqpoint{5.452559in}{2.450768in}}%
\pgfpathlineto{\pgfqpoint{5.445111in}{2.446110in}}%
\pgfpathlineto{\pgfqpoint{5.437655in}{2.441409in}}%
\pgfpathlineto{\pgfqpoint{5.430192in}{2.436662in}}%
\pgfpathlineto{\pgfqpoint{5.422721in}{2.431864in}}%
\pgfpathclose%
\pgfusepath{fill}%
\end{pgfscope}%
\begin{pgfscope}%
\pgfpathrectangle{\pgfqpoint{1.150000in}{0.150000in}}{\pgfqpoint{5.700000in}{5.700000in}}%
\pgfusepath{clip}%
\pgfsetbuttcap%
\pgfsetroundjoin%
\definecolor{currentfill}{rgb}{0.277018,0.050344,0.375715}%
\pgfsetfillcolor{currentfill}%
\pgfsetfillopacity{0.700000}%
\pgfsetlinewidth{0.000000pt}%
\definecolor{currentstroke}{rgb}{0.000000,0.000000,0.000000}%
\pgfsetstrokecolor{currentstroke}%
\pgfsetdash{}{0pt}%
\pgfpathmoveto{\pgfqpoint{3.724533in}{1.712455in}}%
\pgfpathlineto{\pgfqpoint{3.738409in}{1.710679in}}%
\pgfpathlineto{\pgfqpoint{3.752292in}{1.708979in}}%
\pgfpathlineto{\pgfqpoint{3.766182in}{1.707357in}}%
\pgfpathlineto{\pgfqpoint{3.780080in}{1.705810in}}%
\pgfpathlineto{\pgfqpoint{3.788254in}{1.715725in}}%
\pgfpathlineto{\pgfqpoint{3.796423in}{1.725632in}}%
\pgfpathlineto{\pgfqpoint{3.804586in}{1.735529in}}%
\pgfpathlineto{\pgfqpoint{3.812743in}{1.745413in}}%
\pgfpathlineto{\pgfqpoint{3.798857in}{1.746796in}}%
\pgfpathlineto{\pgfqpoint{3.784978in}{1.748255in}}%
\pgfpathlineto{\pgfqpoint{3.771106in}{1.749790in}}%
\pgfpathlineto{\pgfqpoint{3.757241in}{1.751402in}}%
\pgfpathlineto{\pgfqpoint{3.749073in}{1.741674in}}%
\pgfpathlineto{\pgfqpoint{3.740899in}{1.731939in}}%
\pgfpathlineto{\pgfqpoint{3.732719in}{1.722198in}}%
\pgfpathlineto{\pgfqpoint{3.724533in}{1.712455in}}%
\pgfpathclose%
\pgfusepath{fill}%
\end{pgfscope}%
\begin{pgfscope}%
\pgfpathrectangle{\pgfqpoint{1.150000in}{0.150000in}}{\pgfqpoint{5.700000in}{5.700000in}}%
\pgfusepath{clip}%
\pgfsetbuttcap%
\pgfsetroundjoin%
\definecolor{currentfill}{rgb}{0.271828,0.209303,0.504434}%
\pgfsetfillcolor{currentfill}%
\pgfsetfillopacity{0.700000}%
\pgfsetlinewidth{0.000000pt}%
\definecolor{currentstroke}{rgb}{0.000000,0.000000,0.000000}%
\pgfsetstrokecolor{currentstroke}%
\pgfsetdash{}{0pt}%
\pgfpathmoveto{\pgfqpoint{4.485501in}{2.024511in}}%
\pgfpathlineto{\pgfqpoint{4.499614in}{2.026140in}}%
\pgfpathlineto{\pgfqpoint{4.513736in}{2.027841in}}%
\pgfpathlineto{\pgfqpoint{4.527869in}{2.029614in}}%
\pgfpathlineto{\pgfqpoint{4.542012in}{2.031460in}}%
\pgfpathlineto{\pgfqpoint{4.549917in}{2.040484in}}%
\pgfpathlineto{\pgfqpoint{4.557815in}{2.049423in}}%
\pgfpathlineto{\pgfqpoint{4.565707in}{2.058279in}}%
\pgfpathlineto{\pgfqpoint{4.573592in}{2.067051in}}%
\pgfpathlineto{\pgfqpoint{4.559459in}{2.065209in}}%
\pgfpathlineto{\pgfqpoint{4.545336in}{2.063438in}}%
\pgfpathlineto{\pgfqpoint{4.531224in}{2.061739in}}%
\pgfpathlineto{\pgfqpoint{4.517121in}{2.060113in}}%
\pgfpathlineto{\pgfqpoint{4.509226in}{2.051330in}}%
\pgfpathlineto{\pgfqpoint{4.501324in}{2.042469in}}%
\pgfpathlineto{\pgfqpoint{4.493416in}{2.033529in}}%
\pgfpathlineto{\pgfqpoint{4.485501in}{2.024511in}}%
\pgfpathclose%
\pgfusepath{fill}%
\end{pgfscope}%
\begin{pgfscope}%
\pgfpathrectangle{\pgfqpoint{1.150000in}{0.150000in}}{\pgfqpoint{5.700000in}{5.700000in}}%
\pgfusepath{clip}%
\pgfsetbuttcap%
\pgfsetroundjoin%
\definecolor{currentfill}{rgb}{0.268510,0.009605,0.335427}%
\pgfsetfillcolor{currentfill}%
\pgfsetfillopacity{0.700000}%
\pgfsetlinewidth{0.000000pt}%
\definecolor{currentstroke}{rgb}{0.000000,0.000000,0.000000}%
\pgfsetstrokecolor{currentstroke}%
\pgfsetdash{}{0pt}%
\pgfpathmoveto{\pgfqpoint{3.404116in}{1.645097in}}%
\pgfpathlineto{\pgfqpoint{3.417929in}{1.641447in}}%
\pgfpathlineto{\pgfqpoint{3.431748in}{1.637879in}}%
\pgfpathlineto{\pgfqpoint{3.445573in}{1.634390in}}%
\pgfpathlineto{\pgfqpoint{3.459404in}{1.630981in}}%
\pgfpathlineto{\pgfqpoint{3.467702in}{1.639857in}}%
\pgfpathlineto{\pgfqpoint{3.475994in}{1.648779in}}%
\pgfpathlineto{\pgfqpoint{3.484280in}{1.657743in}}%
\pgfpathlineto{\pgfqpoint{3.492558in}{1.666746in}}%
\pgfpathlineto{\pgfqpoint{3.478742in}{1.669929in}}%
\pgfpathlineto{\pgfqpoint{3.464932in}{1.673193in}}%
\pgfpathlineto{\pgfqpoint{3.451128in}{1.676536in}}%
\pgfpathlineto{\pgfqpoint{3.437329in}{1.679961in}}%
\pgfpathlineto{\pgfqpoint{3.429036in}{1.671175in}}%
\pgfpathlineto{\pgfqpoint{3.420736in}{1.662433in}}%
\pgfpathlineto{\pgfqpoint{3.412430in}{1.653740in}}%
\pgfpathlineto{\pgfqpoint{3.404116in}{1.645097in}}%
\pgfpathclose%
\pgfusepath{fill}%
\end{pgfscope}%
\begin{pgfscope}%
\pgfpathrectangle{\pgfqpoint{1.150000in}{0.150000in}}{\pgfqpoint{5.700000in}{5.700000in}}%
\pgfusepath{clip}%
\pgfsetbuttcap%
\pgfsetroundjoin%
\definecolor{currentfill}{rgb}{0.244972,0.287675,0.537260}%
\pgfsetfillcolor{currentfill}%
\pgfsetfillopacity{0.700000}%
\pgfsetlinewidth{0.000000pt}%
\definecolor{currentstroke}{rgb}{0.000000,0.000000,0.000000}%
\pgfsetstrokecolor{currentstroke}%
\pgfsetdash{}{0pt}%
\pgfpathmoveto{\pgfqpoint{2.002861in}{2.262164in}}%
\pgfpathlineto{\pgfqpoint{2.016779in}{2.247700in}}%
\pgfpathlineto{\pgfqpoint{2.030692in}{2.233374in}}%
\pgfpathlineto{\pgfqpoint{2.044601in}{2.219184in}}%
\pgfpathlineto{\pgfqpoint{2.058505in}{2.205130in}}%
\pgfpathlineto{\pgfqpoint{2.067733in}{2.202988in}}%
\pgfpathlineto{\pgfqpoint{2.076939in}{2.201133in}}%
\pgfpathlineto{\pgfqpoint{2.086124in}{2.199559in}}%
\pgfpathlineto{\pgfqpoint{2.095288in}{2.198261in}}%
\pgfpathlineto{\pgfqpoint{2.081427in}{2.211926in}}%
\pgfpathlineto{\pgfqpoint{2.067562in}{2.225726in}}%
\pgfpathlineto{\pgfqpoint{2.053693in}{2.239663in}}%
\pgfpathlineto{\pgfqpoint{2.039819in}{2.253737in}}%
\pgfpathlineto{\pgfqpoint{2.030612in}{2.255416in}}%
\pgfpathlineto{\pgfqpoint{2.021384in}{2.257376in}}%
\pgfpathlineto{\pgfqpoint{2.012133in}{2.259624in}}%
\pgfpathlineto{\pgfqpoint{2.002861in}{2.262164in}}%
\pgfpathclose%
\pgfusepath{fill}%
\end{pgfscope}%
\begin{pgfscope}%
\pgfpathrectangle{\pgfqpoint{1.150000in}{0.150000in}}{\pgfqpoint{5.700000in}{5.700000in}}%
\pgfusepath{clip}%
\pgfsetbuttcap%
\pgfsetroundjoin%
\definecolor{currentfill}{rgb}{0.282884,0.135920,0.453427}%
\pgfsetfillcolor{currentfill}%
\pgfsetfillopacity{0.700000}%
\pgfsetlinewidth{0.000000pt}%
\definecolor{currentstroke}{rgb}{0.000000,0.000000,0.000000}%
\pgfsetstrokecolor{currentstroke}%
\pgfsetdash{}{0pt}%
\pgfpathmoveto{\pgfqpoint{2.427140in}{1.907931in}}%
\pgfpathlineto{\pgfqpoint{2.440948in}{1.897295in}}%
\pgfpathlineto{\pgfqpoint{2.454754in}{1.886768in}}%
\pgfpathlineto{\pgfqpoint{2.468561in}{1.876350in}}%
\pgfpathlineto{\pgfqpoint{2.482367in}{1.866040in}}%
\pgfpathlineto{\pgfqpoint{2.491245in}{1.867591in}}%
\pgfpathlineto{\pgfqpoint{2.500107in}{1.869367in}}%
\pgfpathlineto{\pgfqpoint{2.508954in}{1.871361in}}%
\pgfpathlineto{\pgfqpoint{2.517786in}{1.873569in}}%
\pgfpathlineto{\pgfqpoint{2.504013in}{1.883524in}}%
\pgfpathlineto{\pgfqpoint{2.490241in}{1.893585in}}%
\pgfpathlineto{\pgfqpoint{2.476468in}{1.903755in}}%
\pgfpathlineto{\pgfqpoint{2.462695in}{1.914034in}}%
\pgfpathlineto{\pgfqpoint{2.453831in}{1.912175in}}%
\pgfpathlineto{\pgfqpoint{2.444950in}{1.910534in}}%
\pgfpathlineto{\pgfqpoint{2.436053in}{1.909118in}}%
\pgfpathlineto{\pgfqpoint{2.427140in}{1.907931in}}%
\pgfpathclose%
\pgfusepath{fill}%
\end{pgfscope}%
\begin{pgfscope}%
\pgfpathrectangle{\pgfqpoint{1.150000in}{0.150000in}}{\pgfqpoint{5.700000in}{5.700000in}}%
\pgfusepath{clip}%
\pgfsetbuttcap%
\pgfsetroundjoin%
\definecolor{currentfill}{rgb}{0.276194,0.190074,0.493001}%
\pgfsetfillcolor{currentfill}%
\pgfsetfillopacity{0.700000}%
\pgfsetlinewidth{0.000000pt}%
\definecolor{currentstroke}{rgb}{0.000000,0.000000,0.000000}%
\pgfsetstrokecolor{currentstroke}%
\pgfsetdash{}{0pt}%
\pgfpathmoveto{\pgfqpoint{4.397394in}{1.981920in}}%
\pgfpathlineto{\pgfqpoint{4.411476in}{1.983241in}}%
\pgfpathlineto{\pgfqpoint{4.425568in}{1.984634in}}%
\pgfpathlineto{\pgfqpoint{4.439670in}{1.986100in}}%
\pgfpathlineto{\pgfqpoint{4.453781in}{1.987638in}}%
\pgfpathlineto{\pgfqpoint{4.461721in}{1.996977in}}%
\pgfpathlineto{\pgfqpoint{4.469654in}{2.006235in}}%
\pgfpathlineto{\pgfqpoint{4.477581in}{2.015413in}}%
\pgfpathlineto{\pgfqpoint{4.485501in}{2.024511in}}%
\pgfpathlineto{\pgfqpoint{4.471399in}{2.022954in}}%
\pgfpathlineto{\pgfqpoint{4.457307in}{2.021470in}}%
\pgfpathlineto{\pgfqpoint{4.443225in}{2.020058in}}%
\pgfpathlineto{\pgfqpoint{4.429152in}{2.018718in}}%
\pgfpathlineto{\pgfqpoint{4.421222in}{2.009631in}}%
\pgfpathlineto{\pgfqpoint{4.413286in}{2.000469in}}%
\pgfpathlineto{\pgfqpoint{4.405343in}{1.991232in}}%
\pgfpathlineto{\pgfqpoint{4.397394in}{1.981920in}}%
\pgfpathclose%
\pgfusepath{fill}%
\end{pgfscope}%
\begin{pgfscope}%
\pgfpathrectangle{\pgfqpoint{1.150000in}{0.150000in}}{\pgfqpoint{5.700000in}{5.700000in}}%
\pgfusepath{clip}%
\pgfsetbuttcap%
\pgfsetroundjoin%
\definecolor{currentfill}{rgb}{0.210503,0.363727,0.552206}%
\pgfsetfillcolor{currentfill}%
\pgfsetfillopacity{0.700000}%
\pgfsetlinewidth{0.000000pt}%
\definecolor{currentstroke}{rgb}{0.000000,0.000000,0.000000}%
\pgfsetstrokecolor{currentstroke}%
\pgfsetdash{}{0pt}%
\pgfpathmoveto{\pgfqpoint{5.334881in}{2.396999in}}%
\pgfpathlineto{\pgfqpoint{5.349331in}{2.400667in}}%
\pgfpathlineto{\pgfqpoint{5.363794in}{2.404405in}}%
\pgfpathlineto{\pgfqpoint{5.378269in}{2.408214in}}%
\pgfpathlineto{\pgfqpoint{5.392757in}{2.412092in}}%
\pgfpathlineto{\pgfqpoint{5.400260in}{2.417129in}}%
\pgfpathlineto{\pgfqpoint{5.407755in}{2.422101in}}%
\pgfpathlineto{\pgfqpoint{5.415242in}{2.427012in}}%
\pgfpathlineto{\pgfqpoint{5.422721in}{2.431864in}}%
\pgfpathlineto{\pgfqpoint{5.408252in}{2.428183in}}%
\pgfpathlineto{\pgfqpoint{5.393795in}{2.424572in}}%
\pgfpathlineto{\pgfqpoint{5.379351in}{2.421030in}}%
\pgfpathlineto{\pgfqpoint{5.364919in}{2.417558in}}%
\pgfpathlineto{\pgfqpoint{5.357421in}{2.412501in}}%
\pgfpathlineto{\pgfqpoint{5.349916in}{2.407392in}}%
\pgfpathlineto{\pgfqpoint{5.342402in}{2.402225in}}%
\pgfpathlineto{\pgfqpoint{5.334881in}{2.396999in}}%
\pgfpathclose%
\pgfusepath{fill}%
\end{pgfscope}%
\begin{pgfscope}%
\pgfpathrectangle{\pgfqpoint{1.150000in}{0.150000in}}{\pgfqpoint{5.700000in}{5.700000in}}%
\pgfusepath{clip}%
\pgfsetbuttcap%
\pgfsetroundjoin%
\definecolor{currentfill}{rgb}{0.279574,0.170599,0.479997}%
\pgfsetfillcolor{currentfill}%
\pgfsetfillopacity{0.700000}%
\pgfsetlinewidth{0.000000pt}%
\definecolor{currentstroke}{rgb}{0.000000,0.000000,0.000000}%
\pgfsetstrokecolor{currentstroke}%
\pgfsetdash{}{0pt}%
\pgfpathmoveto{\pgfqpoint{4.309273in}{1.939526in}}%
\pgfpathlineto{\pgfqpoint{4.323325in}{1.940516in}}%
\pgfpathlineto{\pgfqpoint{4.337387in}{1.941579in}}%
\pgfpathlineto{\pgfqpoint{4.351458in}{1.942715in}}%
\pgfpathlineto{\pgfqpoint{4.365539in}{1.943924in}}%
\pgfpathlineto{\pgfqpoint{4.373512in}{1.953535in}}%
\pgfpathlineto{\pgfqpoint{4.381479in}{1.963072in}}%
\pgfpathlineto{\pgfqpoint{4.389440in}{1.972533in}}%
\pgfpathlineto{\pgfqpoint{4.397394in}{1.981920in}}%
\pgfpathlineto{\pgfqpoint{4.383322in}{1.980672in}}%
\pgfpathlineto{\pgfqpoint{4.369260in}{1.979496in}}%
\pgfpathlineto{\pgfqpoint{4.355208in}{1.978394in}}%
\pgfpathlineto{\pgfqpoint{4.341165in}{1.977364in}}%
\pgfpathlineto{\pgfqpoint{4.333201in}{1.968009in}}%
\pgfpathlineto{\pgfqpoint{4.325230in}{1.958584in}}%
\pgfpathlineto{\pgfqpoint{4.317255in}{1.949090in}}%
\pgfpathlineto{\pgfqpoint{4.309273in}{1.939526in}}%
\pgfpathclose%
\pgfusepath{fill}%
\end{pgfscope}%
\begin{pgfscope}%
\pgfpathrectangle{\pgfqpoint{1.150000in}{0.150000in}}{\pgfqpoint{5.700000in}{5.700000in}}%
\pgfusepath{clip}%
\pgfsetbuttcap%
\pgfsetroundjoin%
\definecolor{currentfill}{rgb}{0.253935,0.265254,0.529983}%
\pgfsetfillcolor{currentfill}%
\pgfsetfillopacity{0.700000}%
\pgfsetlinewidth{0.000000pt}%
\definecolor{currentstroke}{rgb}{0.000000,0.000000,0.000000}%
\pgfsetstrokecolor{currentstroke}%
\pgfsetdash{}{0pt}%
\pgfpathmoveto{\pgfqpoint{2.058505in}{2.205130in}}%
\pgfpathlineto{\pgfqpoint{2.072406in}{2.191211in}}%
\pgfpathlineto{\pgfqpoint{2.086303in}{2.177424in}}%
\pgfpathlineto{\pgfqpoint{2.100195in}{2.163769in}}%
\pgfpathlineto{\pgfqpoint{2.114085in}{2.150245in}}%
\pgfpathlineto{\pgfqpoint{2.123268in}{2.148499in}}%
\pgfpathlineto{\pgfqpoint{2.132431in}{2.147035in}}%
\pgfpathlineto{\pgfqpoint{2.141573in}{2.145846in}}%
\pgfpathlineto{\pgfqpoint{2.150696in}{2.144926in}}%
\pgfpathlineto{\pgfqpoint{2.136849in}{2.158063in}}%
\pgfpathlineto{\pgfqpoint{2.122999in}{2.171330in}}%
\pgfpathlineto{\pgfqpoint{2.109145in}{2.184729in}}%
\pgfpathlineto{\pgfqpoint{2.095288in}{2.198261in}}%
\pgfpathlineto{\pgfqpoint{2.086124in}{2.199559in}}%
\pgfpathlineto{\pgfqpoint{2.076939in}{2.201133in}}%
\pgfpathlineto{\pgfqpoint{2.067733in}{2.202988in}}%
\pgfpathlineto{\pgfqpoint{2.058505in}{2.205130in}}%
\pgfpathclose%
\pgfusepath{fill}%
\end{pgfscope}%
\begin{pgfscope}%
\pgfpathrectangle{\pgfqpoint{1.150000in}{0.150000in}}{\pgfqpoint{5.700000in}{5.700000in}}%
\pgfusepath{clip}%
\pgfsetbuttcap%
\pgfsetroundjoin%
\definecolor{currentfill}{rgb}{0.279566,0.067836,0.391917}%
\pgfsetfillcolor{currentfill}%
\pgfsetfillopacity{0.700000}%
\pgfsetlinewidth{0.000000pt}%
\definecolor{currentstroke}{rgb}{0.000000,0.000000,0.000000}%
\pgfsetstrokecolor{currentstroke}%
\pgfsetdash{}{0pt}%
\pgfpathmoveto{\pgfqpoint{2.683085in}{1.762234in}}%
\pgfpathlineto{\pgfqpoint{2.696865in}{1.753612in}}%
\pgfpathlineto{\pgfqpoint{2.710647in}{1.745087in}}%
\pgfpathlineto{\pgfqpoint{2.724431in}{1.736660in}}%
\pgfpathlineto{\pgfqpoint{2.738216in}{1.728330in}}%
\pgfpathlineto{\pgfqpoint{2.746911in}{1.732109in}}%
\pgfpathlineto{\pgfqpoint{2.755593in}{1.736068in}}%
\pgfpathlineto{\pgfqpoint{2.764263in}{1.740204in}}%
\pgfpathlineto{\pgfqpoint{2.772920in}{1.744511in}}%
\pgfpathlineto{\pgfqpoint{2.759163in}{1.752510in}}%
\pgfpathlineto{\pgfqpoint{2.745408in}{1.760605in}}%
\pgfpathlineto{\pgfqpoint{2.731654in}{1.768798in}}%
\pgfpathlineto{\pgfqpoint{2.717902in}{1.777088in}}%
\pgfpathlineto{\pgfqpoint{2.709217in}{1.773105in}}%
\pgfpathlineto{\pgfqpoint{2.700519in}{1.769298in}}%
\pgfpathlineto{\pgfqpoint{2.691809in}{1.765673in}}%
\pgfpathlineto{\pgfqpoint{2.683085in}{1.762234in}}%
\pgfpathclose%
\pgfusepath{fill}%
\end{pgfscope}%
\begin{pgfscope}%
\pgfpathrectangle{\pgfqpoint{1.150000in}{0.150000in}}{\pgfqpoint{5.700000in}{5.700000in}}%
\pgfusepath{clip}%
\pgfsetbuttcap%
\pgfsetroundjoin%
\definecolor{currentfill}{rgb}{0.268510,0.009605,0.335427}%
\pgfsetfillcolor{currentfill}%
\pgfsetfillopacity{0.700000}%
\pgfsetlinewidth{0.000000pt}%
\definecolor{currentstroke}{rgb}{0.000000,0.000000,0.000000}%
\pgfsetstrokecolor{currentstroke}%
\pgfsetdash{}{0pt}%
\pgfpathmoveto{\pgfqpoint{3.027462in}{1.653014in}}%
\pgfpathlineto{\pgfqpoint{3.041240in}{1.646886in}}%
\pgfpathlineto{\pgfqpoint{3.055021in}{1.640844in}}%
\pgfpathlineto{\pgfqpoint{3.068806in}{1.634891in}}%
\pgfpathlineto{\pgfqpoint{3.082595in}{1.629023in}}%
\pgfpathlineto{\pgfqpoint{3.091077in}{1.635594in}}%
\pgfpathlineto{\pgfqpoint{3.099551in}{1.642281in}}%
\pgfpathlineto{\pgfqpoint{3.108015in}{1.649081in}}%
\pgfpathlineto{\pgfqpoint{3.116470in}{1.655990in}}%
\pgfpathlineto{\pgfqpoint{3.102702in}{1.661569in}}%
\pgfpathlineto{\pgfqpoint{3.088939in}{1.667236in}}%
\pgfpathlineto{\pgfqpoint{3.075179in}{1.672990in}}%
\pgfpathlineto{\pgfqpoint{3.061422in}{1.678831in}}%
\pgfpathlineto{\pgfqpoint{3.052946in}{1.672202in}}%
\pgfpathlineto{\pgfqpoint{3.044461in}{1.665687in}}%
\pgfpathlineto{\pgfqpoint{3.035966in}{1.659289in}}%
\pgfpathlineto{\pgfqpoint{3.027462in}{1.653014in}}%
\pgfpathclose%
\pgfusepath{fill}%
\end{pgfscope}%
\begin{pgfscope}%
\pgfpathrectangle{\pgfqpoint{1.150000in}{0.150000in}}{\pgfqpoint{5.700000in}{5.700000in}}%
\pgfusepath{clip}%
\pgfsetbuttcap%
\pgfsetroundjoin%
\definecolor{currentfill}{rgb}{0.273809,0.031497,0.358853}%
\pgfsetfillcolor{currentfill}%
\pgfsetfillopacity{0.700000}%
\pgfsetlinewidth{0.000000pt}%
\definecolor{currentstroke}{rgb}{0.000000,0.000000,0.000000}%
\pgfsetstrokecolor{currentstroke}%
\pgfsetdash{}{0pt}%
\pgfpathmoveto{\pgfqpoint{3.636252in}{1.682113in}}%
\pgfpathlineto{\pgfqpoint{3.650112in}{1.679843in}}%
\pgfpathlineto{\pgfqpoint{3.663979in}{1.677651in}}%
\pgfpathlineto{\pgfqpoint{3.677852in}{1.675537in}}%
\pgfpathlineto{\pgfqpoint{3.691733in}{1.673499in}}%
\pgfpathlineto{\pgfqpoint{3.699942in}{1.683230in}}%
\pgfpathlineto{\pgfqpoint{3.708145in}{1.692969in}}%
\pgfpathlineto{\pgfqpoint{3.716342in}{1.702711in}}%
\pgfpathlineto{\pgfqpoint{3.724533in}{1.712455in}}%
\pgfpathlineto{\pgfqpoint{3.710665in}{1.714308in}}%
\pgfpathlineto{\pgfqpoint{3.696803in}{1.716239in}}%
\pgfpathlineto{\pgfqpoint{3.682948in}{1.718246in}}%
\pgfpathlineto{\pgfqpoint{3.669100in}{1.720332in}}%
\pgfpathlineto{\pgfqpoint{3.660897in}{1.710764in}}%
\pgfpathlineto{\pgfqpoint{3.652688in}{1.701204in}}%
\pgfpathlineto{\pgfqpoint{3.644473in}{1.691652in}}%
\pgfpathlineto{\pgfqpoint{3.636252in}{1.682113in}}%
\pgfpathclose%
\pgfusepath{fill}%
\end{pgfscope}%
\begin{pgfscope}%
\pgfpathrectangle{\pgfqpoint{1.150000in}{0.150000in}}{\pgfqpoint{5.700000in}{5.700000in}}%
\pgfusepath{clip}%
\pgfsetbuttcap%
\pgfsetroundjoin%
\definecolor{currentfill}{rgb}{0.267004,0.004874,0.329415}%
\pgfsetfillcolor{currentfill}%
\pgfsetfillopacity{0.700000}%
\pgfsetlinewidth{0.000000pt}%
\definecolor{currentstroke}{rgb}{0.000000,0.000000,0.000000}%
\pgfsetstrokecolor{currentstroke}%
\pgfsetdash{}{0pt}%
\pgfpathmoveto{\pgfqpoint{3.171582in}{1.634529in}}%
\pgfpathlineto{\pgfqpoint{3.185370in}{1.629377in}}%
\pgfpathlineto{\pgfqpoint{3.199163in}{1.624309in}}%
\pgfpathlineto{\pgfqpoint{3.212960in}{1.619326in}}%
\pgfpathlineto{\pgfqpoint{3.226762in}{1.614426in}}%
\pgfpathlineto{\pgfqpoint{3.235170in}{1.621987in}}%
\pgfpathlineto{\pgfqpoint{3.243569in}{1.629639in}}%
\pgfpathlineto{\pgfqpoint{3.251961in}{1.637377in}}%
\pgfpathlineto{\pgfqpoint{3.260344in}{1.645197in}}%
\pgfpathlineto{\pgfqpoint{3.246561in}{1.649830in}}%
\pgfpathlineto{\pgfqpoint{3.232782in}{1.654548in}}%
\pgfpathlineto{\pgfqpoint{3.219008in}{1.659349in}}%
\pgfpathlineto{\pgfqpoint{3.205238in}{1.664235in}}%
\pgfpathlineto{\pgfqpoint{3.196837in}{1.656673in}}%
\pgfpathlineto{\pgfqpoint{3.188427in}{1.649199in}}%
\pgfpathlineto{\pgfqpoint{3.180008in}{1.641817in}}%
\pgfpathlineto{\pgfqpoint{3.171582in}{1.634529in}}%
\pgfpathclose%
\pgfusepath{fill}%
\end{pgfscope}%
\begin{pgfscope}%
\pgfpathrectangle{\pgfqpoint{1.150000in}{0.150000in}}{\pgfqpoint{5.700000in}{5.700000in}}%
\pgfusepath{clip}%
\pgfsetbuttcap%
\pgfsetroundjoin%
\definecolor{currentfill}{rgb}{0.281887,0.150881,0.465405}%
\pgfsetfillcolor{currentfill}%
\pgfsetfillopacity{0.700000}%
\pgfsetlinewidth{0.000000pt}%
\definecolor{currentstroke}{rgb}{0.000000,0.000000,0.000000}%
\pgfsetstrokecolor{currentstroke}%
\pgfsetdash{}{0pt}%
\pgfpathmoveto{\pgfqpoint{4.221136in}{1.897598in}}%
\pgfpathlineto{\pgfqpoint{4.235159in}{1.898235in}}%
\pgfpathlineto{\pgfqpoint{4.249192in}{1.898945in}}%
\pgfpathlineto{\pgfqpoint{4.263235in}{1.899728in}}%
\pgfpathlineto{\pgfqpoint{4.277286in}{1.900585in}}%
\pgfpathlineto{\pgfqpoint{4.285292in}{1.910422in}}%
\pgfpathlineto{\pgfqpoint{4.293291in}{1.920192in}}%
\pgfpathlineto{\pgfqpoint{4.301285in}{1.929893in}}%
\pgfpathlineto{\pgfqpoint{4.309273in}{1.939526in}}%
\pgfpathlineto{\pgfqpoint{4.295230in}{1.938609in}}%
\pgfpathlineto{\pgfqpoint{4.281197in}{1.937765in}}%
\pgfpathlineto{\pgfqpoint{4.267173in}{1.936994in}}%
\pgfpathlineto{\pgfqpoint{4.253158in}{1.936297in}}%
\pgfpathlineto{\pgfqpoint{4.245161in}{1.926717in}}%
\pgfpathlineto{\pgfqpoint{4.237159in}{1.917073in}}%
\pgfpathlineto{\pgfqpoint{4.229150in}{1.907367in}}%
\pgfpathlineto{\pgfqpoint{4.221136in}{1.897598in}}%
\pgfpathclose%
\pgfusepath{fill}%
\end{pgfscope}%
\begin{pgfscope}%
\pgfpathrectangle{\pgfqpoint{1.150000in}{0.150000in}}{\pgfqpoint{5.700000in}{5.700000in}}%
\pgfusepath{clip}%
\pgfsetbuttcap%
\pgfsetroundjoin%
\definecolor{currentfill}{rgb}{0.216210,0.351535,0.550627}%
\pgfsetfillcolor{currentfill}%
\pgfsetfillopacity{0.700000}%
\pgfsetlinewidth{0.000000pt}%
\definecolor{currentstroke}{rgb}{0.000000,0.000000,0.000000}%
\pgfsetstrokecolor{currentstroke}%
\pgfsetdash{}{0pt}%
\pgfpathmoveto{\pgfqpoint{5.246970in}{2.360746in}}%
\pgfpathlineto{\pgfqpoint{5.261388in}{2.364310in}}%
\pgfpathlineto{\pgfqpoint{5.275819in}{2.367944in}}%
\pgfpathlineto{\pgfqpoint{5.290261in}{2.371648in}}%
\pgfpathlineto{\pgfqpoint{5.304717in}{2.375422in}}%
\pgfpathlineto{\pgfqpoint{5.312270in}{2.380923in}}%
\pgfpathlineto{\pgfqpoint{5.319815in}{2.386351in}}%
\pgfpathlineto{\pgfqpoint{5.327352in}{2.391708in}}%
\pgfpathlineto{\pgfqpoint{5.334881in}{2.396999in}}%
\pgfpathlineto{\pgfqpoint{5.320443in}{2.393400in}}%
\pgfpathlineto{\pgfqpoint{5.306018in}{2.389871in}}%
\pgfpathlineto{\pgfqpoint{5.291605in}{2.386412in}}%
\pgfpathlineto{\pgfqpoint{5.277204in}{2.383023in}}%
\pgfpathlineto{\pgfqpoint{5.269657in}{2.377550in}}%
\pgfpathlineto{\pgfqpoint{5.262102in}{2.372015in}}%
\pgfpathlineto{\pgfqpoint{5.254540in}{2.366415in}}%
\pgfpathlineto{\pgfqpoint{5.246970in}{2.360746in}}%
\pgfpathclose%
\pgfusepath{fill}%
\end{pgfscope}%
\begin{pgfscope}%
\pgfpathrectangle{\pgfqpoint{1.150000in}{0.150000in}}{\pgfqpoint{5.700000in}{5.700000in}}%
\pgfusepath{clip}%
\pgfsetbuttcap%
\pgfsetroundjoin%
\definecolor{currentfill}{rgb}{0.283072,0.130895,0.449241}%
\pgfsetfillcolor{currentfill}%
\pgfsetfillopacity{0.700000}%
\pgfsetlinewidth{0.000000pt}%
\definecolor{currentstroke}{rgb}{0.000000,0.000000,0.000000}%
\pgfsetstrokecolor{currentstroke}%
\pgfsetdash{}{0pt}%
\pgfpathmoveto{\pgfqpoint{4.132982in}{1.856424in}}%
\pgfpathlineto{\pgfqpoint{4.146978in}{1.856685in}}%
\pgfpathlineto{\pgfqpoint{4.160984in}{1.857020in}}%
\pgfpathlineto{\pgfqpoint{4.174998in}{1.857428in}}%
\pgfpathlineto{\pgfqpoint{4.189022in}{1.857910in}}%
\pgfpathlineto{\pgfqpoint{4.197059in}{1.867922in}}%
\pgfpathlineto{\pgfqpoint{4.205090in}{1.877875in}}%
\pgfpathlineto{\pgfqpoint{4.213116in}{1.887767in}}%
\pgfpathlineto{\pgfqpoint{4.221136in}{1.897598in}}%
\pgfpathlineto{\pgfqpoint{4.207121in}{1.897034in}}%
\pgfpathlineto{\pgfqpoint{4.193116in}{1.896544in}}%
\pgfpathlineto{\pgfqpoint{4.179120in}{1.896128in}}%
\pgfpathlineto{\pgfqpoint{4.165132in}{1.895786in}}%
\pgfpathlineto{\pgfqpoint{4.157103in}{1.886028in}}%
\pgfpathlineto{\pgfqpoint{4.149068in}{1.876215in}}%
\pgfpathlineto{\pgfqpoint{4.141028in}{1.866347in}}%
\pgfpathlineto{\pgfqpoint{4.132982in}{1.856424in}}%
\pgfpathclose%
\pgfusepath{fill}%
\end{pgfscope}%
\begin{pgfscope}%
\pgfpathrectangle{\pgfqpoint{1.150000in}{0.150000in}}{\pgfqpoint{5.700000in}{5.700000in}}%
\pgfusepath{clip}%
\pgfsetbuttcap%
\pgfsetroundjoin%
\definecolor{currentfill}{rgb}{0.272594,0.025563,0.353093}%
\pgfsetfillcolor{currentfill}%
\pgfsetfillopacity{0.700000}%
\pgfsetlinewidth{0.000000pt}%
\definecolor{currentstroke}{rgb}{0.000000,0.000000,0.000000}%
\pgfsetstrokecolor{currentstroke}%
\pgfsetdash{}{0pt}%
\pgfpathmoveto{\pgfqpoint{2.883050in}{1.683931in}}%
\pgfpathlineto{\pgfqpoint{2.896827in}{1.676776in}}%
\pgfpathlineto{\pgfqpoint{2.910607in}{1.669713in}}%
\pgfpathlineto{\pgfqpoint{2.924389in}{1.662741in}}%
\pgfpathlineto{\pgfqpoint{2.938174in}{1.655859in}}%
\pgfpathlineto{\pgfqpoint{2.946743in}{1.661284in}}%
\pgfpathlineto{\pgfqpoint{2.955300in}{1.666854in}}%
\pgfpathlineto{\pgfqpoint{2.963848in}{1.672566in}}%
\pgfpathlineto{\pgfqpoint{2.972385in}{1.678413in}}%
\pgfpathlineto{\pgfqpoint{2.958623in}{1.684986in}}%
\pgfpathlineto{\pgfqpoint{2.944865in}{1.691650in}}%
\pgfpathlineto{\pgfqpoint{2.931110in}{1.698404in}}%
\pgfpathlineto{\pgfqpoint{2.917357in}{1.705249in}}%
\pgfpathlineto{\pgfqpoint{2.908797in}{1.699702in}}%
\pgfpathlineto{\pgfqpoint{2.900226in}{1.694297in}}%
\pgfpathlineto{\pgfqpoint{2.891643in}{1.689038in}}%
\pgfpathlineto{\pgfqpoint{2.883050in}{1.683931in}}%
\pgfpathclose%
\pgfusepath{fill}%
\end{pgfscope}%
\begin{pgfscope}%
\pgfpathrectangle{\pgfqpoint{1.150000in}{0.150000in}}{\pgfqpoint{5.700000in}{5.700000in}}%
\pgfusepath{clip}%
\pgfsetbuttcap%
\pgfsetroundjoin%
\definecolor{currentfill}{rgb}{0.260571,0.246922,0.522828}%
\pgfsetfillcolor{currentfill}%
\pgfsetfillopacity{0.700000}%
\pgfsetlinewidth{0.000000pt}%
\definecolor{currentstroke}{rgb}{0.000000,0.000000,0.000000}%
\pgfsetstrokecolor{currentstroke}%
\pgfsetdash{}{0pt}%
\pgfpathmoveto{\pgfqpoint{2.114085in}{2.150245in}}%
\pgfpathlineto{\pgfqpoint{2.127970in}{2.136851in}}%
\pgfpathlineto{\pgfqpoint{2.141853in}{2.123585in}}%
\pgfpathlineto{\pgfqpoint{2.155732in}{2.110446in}}%
\pgfpathlineto{\pgfqpoint{2.169608in}{2.097434in}}%
\pgfpathlineto{\pgfqpoint{2.178749in}{2.096083in}}%
\pgfpathlineto{\pgfqpoint{2.187870in}{2.095007in}}%
\pgfpathlineto{\pgfqpoint{2.196971in}{2.094201in}}%
\pgfpathlineto{\pgfqpoint{2.206053in}{2.093658in}}%
\pgfpathlineto{\pgfqpoint{2.192218in}{2.106285in}}%
\pgfpathlineto{\pgfqpoint{2.178380in}{2.119038in}}%
\pgfpathlineto{\pgfqpoint{2.164540in}{2.131918in}}%
\pgfpathlineto{\pgfqpoint{2.150696in}{2.144926in}}%
\pgfpathlineto{\pgfqpoint{2.141573in}{2.145846in}}%
\pgfpathlineto{\pgfqpoint{2.132431in}{2.147035in}}%
\pgfpathlineto{\pgfqpoint{2.123268in}{2.148499in}}%
\pgfpathlineto{\pgfqpoint{2.114085in}{2.150245in}}%
\pgfpathclose%
\pgfusepath{fill}%
\end{pgfscope}%
\begin{pgfscope}%
\pgfpathrectangle{\pgfqpoint{1.150000in}{0.150000in}}{\pgfqpoint{5.700000in}{5.700000in}}%
\pgfusepath{clip}%
\pgfsetbuttcap%
\pgfsetroundjoin%
\definecolor{currentfill}{rgb}{0.283229,0.120777,0.440584}%
\pgfsetfillcolor{currentfill}%
\pgfsetfillopacity{0.700000}%
\pgfsetlinewidth{0.000000pt}%
\definecolor{currentstroke}{rgb}{0.000000,0.000000,0.000000}%
\pgfsetstrokecolor{currentstroke}%
\pgfsetdash{}{0pt}%
\pgfpathmoveto{\pgfqpoint{2.482367in}{1.866040in}}%
\pgfpathlineto{\pgfqpoint{2.496172in}{1.855836in}}%
\pgfpathlineto{\pgfqpoint{2.509978in}{1.845739in}}%
\pgfpathlineto{\pgfqpoint{2.523784in}{1.835747in}}%
\pgfpathlineto{\pgfqpoint{2.537590in}{1.825860in}}%
\pgfpathlineto{\pgfqpoint{2.546434in}{1.827775in}}%
\pgfpathlineto{\pgfqpoint{2.555263in}{1.829909in}}%
\pgfpathlineto{\pgfqpoint{2.564077in}{1.832256in}}%
\pgfpathlineto{\pgfqpoint{2.572876in}{1.834811in}}%
\pgfpathlineto{\pgfqpoint{2.559103in}{1.844343in}}%
\pgfpathlineto{\pgfqpoint{2.545330in}{1.853980in}}%
\pgfpathlineto{\pgfqpoint{2.531558in}{1.863721in}}%
\pgfpathlineto{\pgfqpoint{2.517786in}{1.873569in}}%
\pgfpathlineto{\pgfqpoint{2.508954in}{1.871361in}}%
\pgfpathlineto{\pgfqpoint{2.500107in}{1.869367in}}%
\pgfpathlineto{\pgfqpoint{2.491245in}{1.867591in}}%
\pgfpathlineto{\pgfqpoint{2.482367in}{1.866040in}}%
\pgfpathclose%
\pgfusepath{fill}%
\end{pgfscope}%
\begin{pgfscope}%
\pgfpathrectangle{\pgfqpoint{1.150000in}{0.150000in}}{\pgfqpoint{5.700000in}{5.700000in}}%
\pgfusepath{clip}%
\pgfsetbuttcap%
\pgfsetroundjoin%
\definecolor{currentfill}{rgb}{0.283091,0.110553,0.431554}%
\pgfsetfillcolor{currentfill}%
\pgfsetfillopacity{0.700000}%
\pgfsetlinewidth{0.000000pt}%
\definecolor{currentstroke}{rgb}{0.000000,0.000000,0.000000}%
\pgfsetstrokecolor{currentstroke}%
\pgfsetdash{}{0pt}%
\pgfpathmoveto{\pgfqpoint{4.044807in}{1.816315in}}%
\pgfpathlineto{\pgfqpoint{4.058778in}{1.816178in}}%
\pgfpathlineto{\pgfqpoint{4.072757in}{1.816115in}}%
\pgfpathlineto{\pgfqpoint{4.086745in}{1.816125in}}%
\pgfpathlineto{\pgfqpoint{4.100742in}{1.816210in}}%
\pgfpathlineto{\pgfqpoint{4.108810in}{1.826340in}}%
\pgfpathlineto{\pgfqpoint{4.116873in}{1.836419in}}%
\pgfpathlineto{\pgfqpoint{4.124930in}{1.846448in}}%
\pgfpathlineto{\pgfqpoint{4.132982in}{1.856424in}}%
\pgfpathlineto{\pgfqpoint{4.118994in}{1.856237in}}%
\pgfpathlineto{\pgfqpoint{4.105015in}{1.856124in}}%
\pgfpathlineto{\pgfqpoint{4.091045in}{1.856085in}}%
\pgfpathlineto{\pgfqpoint{4.077084in}{1.856120in}}%
\pgfpathlineto{\pgfqpoint{4.069023in}{1.846238in}}%
\pgfpathlineto{\pgfqpoint{4.060956in}{1.836309in}}%
\pgfpathlineto{\pgfqpoint{4.052884in}{1.826335in}}%
\pgfpathlineto{\pgfqpoint{4.044807in}{1.816315in}}%
\pgfpathclose%
\pgfusepath{fill}%
\end{pgfscope}%
\begin{pgfscope}%
\pgfpathrectangle{\pgfqpoint{1.150000in}{0.150000in}}{\pgfqpoint{5.700000in}{5.700000in}}%
\pgfusepath{clip}%
\pgfsetbuttcap%
\pgfsetroundjoin%
\definecolor{currentfill}{rgb}{0.267004,0.004874,0.329415}%
\pgfsetfillcolor{currentfill}%
\pgfsetfillopacity{0.700000}%
\pgfsetlinewidth{0.000000pt}%
\definecolor{currentstroke}{rgb}{0.000000,0.000000,0.000000}%
\pgfsetstrokecolor{currentstroke}%
\pgfsetdash{}{0pt}%
\pgfpathmoveto{\pgfqpoint{3.315527in}{1.627492in}}%
\pgfpathlineto{\pgfqpoint{3.329335in}{1.623272in}}%
\pgfpathlineto{\pgfqpoint{3.343148in}{1.619134in}}%
\pgfpathlineto{\pgfqpoint{3.356967in}{1.615077in}}%
\pgfpathlineto{\pgfqpoint{3.370791in}{1.611101in}}%
\pgfpathlineto{\pgfqpoint{3.379133in}{1.619506in}}%
\pgfpathlineto{\pgfqpoint{3.387468in}{1.627977in}}%
\pgfpathlineto{\pgfqpoint{3.395795in}{1.636508in}}%
\pgfpathlineto{\pgfqpoint{3.404116in}{1.645097in}}%
\pgfpathlineto{\pgfqpoint{3.390308in}{1.648827in}}%
\pgfpathlineto{\pgfqpoint{3.376506in}{1.652638in}}%
\pgfpathlineto{\pgfqpoint{3.362709in}{1.656531in}}%
\pgfpathlineto{\pgfqpoint{3.348917in}{1.660506in}}%
\pgfpathlineto{\pgfqpoint{3.340580in}{1.652155in}}%
\pgfpathlineto{\pgfqpoint{3.332237in}{1.643867in}}%
\pgfpathlineto{\pgfqpoint{3.323886in}{1.635645in}}%
\pgfpathlineto{\pgfqpoint{3.315527in}{1.627492in}}%
\pgfpathclose%
\pgfusepath{fill}%
\end{pgfscope}%
\begin{pgfscope}%
\pgfpathrectangle{\pgfqpoint{1.150000in}{0.150000in}}{\pgfqpoint{5.700000in}{5.700000in}}%
\pgfusepath{clip}%
\pgfsetbuttcap%
\pgfsetroundjoin%
\definecolor{currentfill}{rgb}{0.221989,0.339161,0.548752}%
\pgfsetfillcolor{currentfill}%
\pgfsetfillopacity{0.700000}%
\pgfsetlinewidth{0.000000pt}%
\definecolor{currentstroke}{rgb}{0.000000,0.000000,0.000000}%
\pgfsetstrokecolor{currentstroke}%
\pgfsetdash{}{0pt}%
\pgfpathmoveto{\pgfqpoint{5.158995in}{2.323155in}}%
\pgfpathlineto{\pgfqpoint{5.173380in}{2.326592in}}%
\pgfpathlineto{\pgfqpoint{5.187778in}{2.330099in}}%
\pgfpathlineto{\pgfqpoint{5.202188in}{2.333677in}}%
\pgfpathlineto{\pgfqpoint{5.216610in}{2.337325in}}%
\pgfpathlineto{\pgfqpoint{5.224212in}{2.343298in}}%
\pgfpathlineto{\pgfqpoint{5.231806in}{2.349190in}}%
\pgfpathlineto{\pgfqpoint{5.239392in}{2.355005in}}%
\pgfpathlineto{\pgfqpoint{5.246970in}{2.360746in}}%
\pgfpathlineto{\pgfqpoint{5.232564in}{2.357252in}}%
\pgfpathlineto{\pgfqpoint{5.218170in}{2.353828in}}%
\pgfpathlineto{\pgfqpoint{5.203789in}{2.350474in}}%
\pgfpathlineto{\pgfqpoint{5.189419in}{2.347191in}}%
\pgfpathlineto{\pgfqpoint{5.181825in}{2.341289in}}%
\pgfpathlineto{\pgfqpoint{5.174223in}{2.335318in}}%
\pgfpathlineto{\pgfqpoint{5.166613in}{2.329274in}}%
\pgfpathlineto{\pgfqpoint{5.158995in}{2.323155in}}%
\pgfpathclose%
\pgfusepath{fill}%
\end{pgfscope}%
\begin{pgfscope}%
\pgfpathrectangle{\pgfqpoint{1.150000in}{0.150000in}}{\pgfqpoint{5.700000in}{5.700000in}}%
\pgfusepath{clip}%
\pgfsetbuttcap%
\pgfsetroundjoin%
\definecolor{currentfill}{rgb}{0.271305,0.019942,0.347269}%
\pgfsetfillcolor{currentfill}%
\pgfsetfillopacity{0.700000}%
\pgfsetlinewidth{0.000000pt}%
\definecolor{currentstroke}{rgb}{0.000000,0.000000,0.000000}%
\pgfsetstrokecolor{currentstroke}%
\pgfsetdash{}{0pt}%
\pgfpathmoveto{\pgfqpoint{3.547883in}{1.654808in}}%
\pgfpathlineto{\pgfqpoint{3.561729in}{1.652020in}}%
\pgfpathlineto{\pgfqpoint{3.575583in}{1.649312in}}%
\pgfpathlineto{\pgfqpoint{3.589442in}{1.646681in}}%
\pgfpathlineto{\pgfqpoint{3.603308in}{1.644129in}}%
\pgfpathlineto{\pgfqpoint{3.611553in}{1.653593in}}%
\pgfpathlineto{\pgfqpoint{3.619792in}{1.663081in}}%
\pgfpathlineto{\pgfqpoint{3.628025in}{1.672588in}}%
\pgfpathlineto{\pgfqpoint{3.636252in}{1.682113in}}%
\pgfpathlineto{\pgfqpoint{3.622399in}{1.684461in}}%
\pgfpathlineto{\pgfqpoint{3.608553in}{1.686886in}}%
\pgfpathlineto{\pgfqpoint{3.594713in}{1.689390in}}%
\pgfpathlineto{\pgfqpoint{3.580879in}{1.691973in}}%
\pgfpathlineto{\pgfqpoint{3.572639in}{1.682645in}}%
\pgfpathlineto{\pgfqpoint{3.564393in}{1.673340in}}%
\pgfpathlineto{\pgfqpoint{3.556141in}{1.664060in}}%
\pgfpathlineto{\pgfqpoint{3.547883in}{1.654808in}}%
\pgfpathclose%
\pgfusepath{fill}%
\end{pgfscope}%
\begin{pgfscope}%
\pgfpathrectangle{\pgfqpoint{1.150000in}{0.150000in}}{\pgfqpoint{5.700000in}{5.700000in}}%
\pgfusepath{clip}%
\pgfsetbuttcap%
\pgfsetroundjoin%
\definecolor{currentfill}{rgb}{0.282327,0.094955,0.417331}%
\pgfsetfillcolor{currentfill}%
\pgfsetfillopacity{0.700000}%
\pgfsetlinewidth{0.000000pt}%
\definecolor{currentstroke}{rgb}{0.000000,0.000000,0.000000}%
\pgfsetstrokecolor{currentstroke}%
\pgfsetdash{}{0pt}%
\pgfpathmoveto{\pgfqpoint{3.956604in}{1.777604in}}%
\pgfpathlineto{\pgfqpoint{3.970551in}{1.777046in}}%
\pgfpathlineto{\pgfqpoint{3.984506in}{1.776562in}}%
\pgfpathlineto{\pgfqpoint{3.998469in}{1.776152in}}%
\pgfpathlineto{\pgfqpoint{4.012441in}{1.775818in}}%
\pgfpathlineto{\pgfqpoint{4.020541in}{1.786002in}}%
\pgfpathlineto{\pgfqpoint{4.028635in}{1.796148in}}%
\pgfpathlineto{\pgfqpoint{4.036724in}{1.806252in}}%
\pgfpathlineto{\pgfqpoint{4.044807in}{1.816315in}}%
\pgfpathlineto{\pgfqpoint{4.030844in}{1.816527in}}%
\pgfpathlineto{\pgfqpoint{4.016891in}{1.816814in}}%
\pgfpathlineto{\pgfqpoint{4.002945in}{1.817175in}}%
\pgfpathlineto{\pgfqpoint{3.989008in}{1.817611in}}%
\pgfpathlineto{\pgfqpoint{3.980915in}{1.807663in}}%
\pgfpathlineto{\pgfqpoint{3.972817in}{1.797678in}}%
\pgfpathlineto{\pgfqpoint{3.964713in}{1.787658in}}%
\pgfpathlineto{\pgfqpoint{3.956604in}{1.777604in}}%
\pgfpathclose%
\pgfusepath{fill}%
\end{pgfscope}%
\begin{pgfscope}%
\pgfpathrectangle{\pgfqpoint{1.150000in}{0.150000in}}{\pgfqpoint{5.700000in}{5.700000in}}%
\pgfusepath{clip}%
\pgfsetbuttcap%
\pgfsetroundjoin%
\definecolor{currentfill}{rgb}{0.188923,0.410910,0.556326}%
\pgfsetfillcolor{currentfill}%
\pgfsetfillopacity{0.700000}%
\pgfsetlinewidth{0.000000pt}%
\definecolor{currentstroke}{rgb}{0.000000,0.000000,0.000000}%
\pgfsetstrokecolor{currentstroke}%
\pgfsetdash{}{0pt}%
\pgfpathmoveto{\pgfqpoint{5.656401in}{2.513145in}}%
\pgfpathlineto{\pgfqpoint{5.670995in}{2.517269in}}%
\pgfpathlineto{\pgfqpoint{5.685603in}{2.521462in}}%
\pgfpathlineto{\pgfqpoint{5.700223in}{2.525725in}}%
\pgfpathlineto{\pgfqpoint{5.707551in}{2.529283in}}%
\pgfpathlineto{\pgfqpoint{5.714871in}{2.532805in}}%
\pgfpathlineto{\pgfqpoint{5.722183in}{2.536296in}}%
\pgfpathlineto{\pgfqpoint{5.729488in}{2.539760in}}%
\pgfpathlineto{\pgfqpoint{5.714891in}{2.535761in}}%
\pgfpathlineto{\pgfqpoint{5.700307in}{2.531830in}}%
\pgfpathlineto{\pgfqpoint{5.685736in}{2.527969in}}%
\pgfpathlineto{\pgfqpoint{5.678414in}{2.524302in}}%
\pgfpathlineto{\pgfqpoint{5.671084in}{2.520612in}}%
\pgfpathlineto{\pgfqpoint{5.663746in}{2.516895in}}%
\pgfpathlineto{\pgfqpoint{5.656401in}{2.513145in}}%
\pgfpathclose%
\pgfusepath{fill}%
\end{pgfscope}%
\begin{pgfscope}%
\pgfpathrectangle{\pgfqpoint{1.150000in}{0.150000in}}{\pgfqpoint{5.700000in}{5.700000in}}%
\pgfusepath{clip}%
\pgfsetbuttcap%
\pgfsetroundjoin%
\definecolor{currentfill}{rgb}{0.267968,0.223549,0.512008}%
\pgfsetfillcolor{currentfill}%
\pgfsetfillopacity{0.700000}%
\pgfsetlinewidth{0.000000pt}%
\definecolor{currentstroke}{rgb}{0.000000,0.000000,0.000000}%
\pgfsetstrokecolor{currentstroke}%
\pgfsetdash{}{0pt}%
\pgfpathmoveto{\pgfqpoint{2.169608in}{2.097434in}}%
\pgfpathlineto{\pgfqpoint{2.183481in}{2.084547in}}%
\pgfpathlineto{\pgfqpoint{2.197352in}{2.071784in}}%
\pgfpathlineto{\pgfqpoint{2.211220in}{2.059145in}}%
\pgfpathlineto{\pgfqpoint{2.225085in}{2.046627in}}%
\pgfpathlineto{\pgfqpoint{2.234185in}{2.045669in}}%
\pgfpathlineto{\pgfqpoint{2.243265in}{2.044980in}}%
\pgfpathlineto{\pgfqpoint{2.252325in}{2.044555in}}%
\pgfpathlineto{\pgfqpoint{2.261368in}{2.044388in}}%
\pgfpathlineto{\pgfqpoint{2.247542in}{2.056522in}}%
\pgfpathlineto{\pgfqpoint{2.233715in}{2.068777in}}%
\pgfpathlineto{\pgfqpoint{2.219885in}{2.081156in}}%
\pgfpathlineto{\pgfqpoint{2.206053in}{2.093658in}}%
\pgfpathlineto{\pgfqpoint{2.196971in}{2.094201in}}%
\pgfpathlineto{\pgfqpoint{2.187870in}{2.095007in}}%
\pgfpathlineto{\pgfqpoint{2.178749in}{2.096083in}}%
\pgfpathlineto{\pgfqpoint{2.169608in}{2.097434in}}%
\pgfpathclose%
\pgfusepath{fill}%
\end{pgfscope}%
\begin{pgfscope}%
\pgfpathrectangle{\pgfqpoint{1.150000in}{0.150000in}}{\pgfqpoint{5.700000in}{5.700000in}}%
\pgfusepath{clip}%
\pgfsetbuttcap%
\pgfsetroundjoin%
\definecolor{currentfill}{rgb}{0.227802,0.326594,0.546532}%
\pgfsetfillcolor{currentfill}%
\pgfsetfillopacity{0.700000}%
\pgfsetlinewidth{0.000000pt}%
\definecolor{currentstroke}{rgb}{0.000000,0.000000,0.000000}%
\pgfsetstrokecolor{currentstroke}%
\pgfsetdash{}{0pt}%
\pgfpathmoveto{\pgfqpoint{5.070964in}{2.284299in}}%
\pgfpathlineto{\pgfqpoint{5.085316in}{2.287587in}}%
\pgfpathlineto{\pgfqpoint{5.099680in}{2.290945in}}%
\pgfpathlineto{\pgfqpoint{5.114057in}{2.294373in}}%
\pgfpathlineto{\pgfqpoint{5.128445in}{2.297872in}}%
\pgfpathlineto{\pgfqpoint{5.136094in}{2.304319in}}%
\pgfpathlineto{\pgfqpoint{5.143736in}{2.310680in}}%
\pgfpathlineto{\pgfqpoint{5.151369in}{2.316958in}}%
\pgfpathlineto{\pgfqpoint{5.158995in}{2.323155in}}%
\pgfpathlineto{\pgfqpoint{5.144621in}{2.319788in}}%
\pgfpathlineto{\pgfqpoint{5.130260in}{2.316492in}}%
\pgfpathlineto{\pgfqpoint{5.115910in}{2.313266in}}%
\pgfpathlineto{\pgfqpoint{5.101573in}{2.310110in}}%
\pgfpathlineto{\pgfqpoint{5.093932in}{2.303774in}}%
\pgfpathlineto{\pgfqpoint{5.086284in}{2.297361in}}%
\pgfpathlineto{\pgfqpoint{5.078627in}{2.290871in}}%
\pgfpathlineto{\pgfqpoint{5.070964in}{2.284299in}}%
\pgfpathclose%
\pgfusepath{fill}%
\end{pgfscope}%
\begin{pgfscope}%
\pgfpathrectangle{\pgfqpoint{1.150000in}{0.150000in}}{\pgfqpoint{5.700000in}{5.700000in}}%
\pgfusepath{clip}%
\pgfsetbuttcap%
\pgfsetroundjoin%
\definecolor{currentfill}{rgb}{0.277941,0.056324,0.381191}%
\pgfsetfillcolor{currentfill}%
\pgfsetfillopacity{0.700000}%
\pgfsetlinewidth{0.000000pt}%
\definecolor{currentstroke}{rgb}{0.000000,0.000000,0.000000}%
\pgfsetstrokecolor{currentstroke}%
\pgfsetdash{}{0pt}%
\pgfpathmoveto{\pgfqpoint{2.738216in}{1.728330in}}%
\pgfpathlineto{\pgfqpoint{2.752003in}{1.720096in}}%
\pgfpathlineto{\pgfqpoint{2.765791in}{1.711958in}}%
\pgfpathlineto{\pgfqpoint{2.779581in}{1.703915in}}%
\pgfpathlineto{\pgfqpoint{2.793373in}{1.695966in}}%
\pgfpathlineto{\pgfqpoint{2.802040in}{1.700083in}}%
\pgfpathlineto{\pgfqpoint{2.810695in}{1.704376in}}%
\pgfpathlineto{\pgfqpoint{2.819337in}{1.708840in}}%
\pgfpathlineto{\pgfqpoint{2.827967in}{1.713470in}}%
\pgfpathlineto{\pgfqpoint{2.814202in}{1.721088in}}%
\pgfpathlineto{\pgfqpoint{2.800440in}{1.728801in}}%
\pgfpathlineto{\pgfqpoint{2.786679in}{1.736608in}}%
\pgfpathlineto{\pgfqpoint{2.772920in}{1.744511in}}%
\pgfpathlineto{\pgfqpoint{2.764263in}{1.740204in}}%
\pgfpathlineto{\pgfqpoint{2.755593in}{1.736068in}}%
\pgfpathlineto{\pgfqpoint{2.746911in}{1.732109in}}%
\pgfpathlineto{\pgfqpoint{2.738216in}{1.728330in}}%
\pgfpathclose%
\pgfusepath{fill}%
\end{pgfscope}%
\begin{pgfscope}%
\pgfpathrectangle{\pgfqpoint{1.150000in}{0.150000in}}{\pgfqpoint{5.700000in}{5.700000in}}%
\pgfusepath{clip}%
\pgfsetbuttcap%
\pgfsetroundjoin%
\definecolor{currentfill}{rgb}{0.280894,0.078907,0.402329}%
\pgfsetfillcolor{currentfill}%
\pgfsetfillopacity{0.700000}%
\pgfsetlinewidth{0.000000pt}%
\definecolor{currentstroke}{rgb}{0.000000,0.000000,0.000000}%
\pgfsetstrokecolor{currentstroke}%
\pgfsetdash{}{0pt}%
\pgfpathmoveto{\pgfqpoint{3.868365in}{1.740644in}}%
\pgfpathlineto{\pgfqpoint{3.882290in}{1.739642in}}%
\pgfpathlineto{\pgfqpoint{3.896223in}{1.738714in}}%
\pgfpathlineto{\pgfqpoint{3.910164in}{1.737862in}}%
\pgfpathlineto{\pgfqpoint{3.924112in}{1.737084in}}%
\pgfpathlineto{\pgfqpoint{3.932243in}{1.747257in}}%
\pgfpathlineto{\pgfqpoint{3.940369in}{1.757402in}}%
\pgfpathlineto{\pgfqpoint{3.948489in}{1.767518in}}%
\pgfpathlineto{\pgfqpoint{3.956604in}{1.777604in}}%
\pgfpathlineto{\pgfqpoint{3.942665in}{1.778238in}}%
\pgfpathlineto{\pgfqpoint{3.928735in}{1.778947in}}%
\pgfpathlineto{\pgfqpoint{3.914812in}{1.779731in}}%
\pgfpathlineto{\pgfqpoint{3.900898in}{1.780590in}}%
\pgfpathlineto{\pgfqpoint{3.892773in}{1.770640in}}%
\pgfpathlineto{\pgfqpoint{3.884643in}{1.760664in}}%
\pgfpathlineto{\pgfqpoint{3.876507in}{1.750665in}}%
\pgfpathlineto{\pgfqpoint{3.868365in}{1.740644in}}%
\pgfpathclose%
\pgfusepath{fill}%
\end{pgfscope}%
\begin{pgfscope}%
\pgfpathrectangle{\pgfqpoint{1.150000in}{0.150000in}}{\pgfqpoint{5.700000in}{5.700000in}}%
\pgfusepath{clip}%
\pgfsetbuttcap%
\pgfsetroundjoin%
\definecolor{currentfill}{rgb}{0.235526,0.309527,0.542944}%
\pgfsetfillcolor{currentfill}%
\pgfsetfillopacity{0.700000}%
\pgfsetlinewidth{0.000000pt}%
\definecolor{currentstroke}{rgb}{0.000000,0.000000,0.000000}%
\pgfsetstrokecolor{currentstroke}%
\pgfsetdash{}{0pt}%
\pgfpathmoveto{\pgfqpoint{4.982884in}{2.244273in}}%
\pgfpathlineto{\pgfqpoint{4.997203in}{2.247389in}}%
\pgfpathlineto{\pgfqpoint{5.011534in}{2.250575in}}%
\pgfpathlineto{\pgfqpoint{5.025877in}{2.253831in}}%
\pgfpathlineto{\pgfqpoint{5.040231in}{2.257159in}}%
\pgfpathlineto{\pgfqpoint{5.047926in}{2.264077in}}%
\pgfpathlineto{\pgfqpoint{5.055613in}{2.270905in}}%
\pgfpathlineto{\pgfqpoint{5.063292in}{2.277645in}}%
\pgfpathlineto{\pgfqpoint{5.070964in}{2.284299in}}%
\pgfpathlineto{\pgfqpoint{5.056623in}{2.281082in}}%
\pgfpathlineto{\pgfqpoint{5.042294in}{2.277936in}}%
\pgfpathlineto{\pgfqpoint{5.027977in}{2.274860in}}%
\pgfpathlineto{\pgfqpoint{5.013672in}{2.271855in}}%
\pgfpathlineto{\pgfqpoint{5.005986in}{2.265082in}}%
\pgfpathlineto{\pgfqpoint{4.998293in}{2.258229in}}%
\pgfpathlineto{\pgfqpoint{4.990593in}{2.251294in}}%
\pgfpathlineto{\pgfqpoint{4.982884in}{2.244273in}}%
\pgfpathclose%
\pgfusepath{fill}%
\end{pgfscope}%
\begin{pgfscope}%
\pgfpathrectangle{\pgfqpoint{1.150000in}{0.150000in}}{\pgfqpoint{5.700000in}{5.700000in}}%
\pgfusepath{clip}%
\pgfsetbuttcap%
\pgfsetroundjoin%
\definecolor{currentfill}{rgb}{0.273006,0.204520,0.501721}%
\pgfsetfillcolor{currentfill}%
\pgfsetfillopacity{0.700000}%
\pgfsetlinewidth{0.000000pt}%
\definecolor{currentstroke}{rgb}{0.000000,0.000000,0.000000}%
\pgfsetstrokecolor{currentstroke}%
\pgfsetdash{}{0pt}%
\pgfpathmoveto{\pgfqpoint{2.225085in}{2.046627in}}%
\pgfpathlineto{\pgfqpoint{2.238948in}{2.034231in}}%
\pgfpathlineto{\pgfqpoint{2.252809in}{2.021955in}}%
\pgfpathlineto{\pgfqpoint{2.266667in}{2.009797in}}%
\pgfpathlineto{\pgfqpoint{2.280524in}{1.997759in}}%
\pgfpathlineto{\pgfqpoint{2.289583in}{1.997191in}}%
\pgfpathlineto{\pgfqpoint{2.298623in}{1.996888in}}%
\pgfpathlineto{\pgfqpoint{2.307645in}{1.996843in}}%
\pgfpathlineto{\pgfqpoint{2.316648in}{1.997050in}}%
\pgfpathlineto{\pgfqpoint{2.302831in}{2.008706in}}%
\pgfpathlineto{\pgfqpoint{2.289012in}{2.020481in}}%
\pgfpathlineto{\pgfqpoint{2.275191in}{2.032374in}}%
\pgfpathlineto{\pgfqpoint{2.261368in}{2.044388in}}%
\pgfpathlineto{\pgfqpoint{2.252325in}{2.044555in}}%
\pgfpathlineto{\pgfqpoint{2.243265in}{2.044980in}}%
\pgfpathlineto{\pgfqpoint{2.234185in}{2.045669in}}%
\pgfpathlineto{\pgfqpoint{2.225085in}{2.046627in}}%
\pgfpathclose%
\pgfusepath{fill}%
\end{pgfscope}%
\begin{pgfscope}%
\pgfpathrectangle{\pgfqpoint{1.150000in}{0.150000in}}{\pgfqpoint{5.700000in}{5.700000in}}%
\pgfusepath{clip}%
\pgfsetbuttcap%
\pgfsetroundjoin%
\definecolor{currentfill}{rgb}{0.269944,0.014625,0.341379}%
\pgfsetfillcolor{currentfill}%
\pgfsetfillopacity{0.700000}%
\pgfsetlinewidth{0.000000pt}%
\definecolor{currentstroke}{rgb}{0.000000,0.000000,0.000000}%
\pgfsetstrokecolor{currentstroke}%
\pgfsetdash{}{0pt}%
\pgfpathmoveto{\pgfqpoint{3.459404in}{1.630981in}}%
\pgfpathlineto{\pgfqpoint{3.473240in}{1.627653in}}%
\pgfpathlineto{\pgfqpoint{3.487082in}{1.624403in}}%
\pgfpathlineto{\pgfqpoint{3.500931in}{1.621233in}}%
\pgfpathlineto{\pgfqpoint{3.514785in}{1.618142in}}%
\pgfpathlineto{\pgfqpoint{3.523069in}{1.627251in}}%
\pgfpathlineto{\pgfqpoint{3.531347in}{1.636400in}}%
\pgfpathlineto{\pgfqpoint{3.539618in}{1.645587in}}%
\pgfpathlineto{\pgfqpoint{3.547883in}{1.654808in}}%
\pgfpathlineto{\pgfqpoint{3.534042in}{1.657674in}}%
\pgfpathlineto{\pgfqpoint{3.520208in}{1.660618in}}%
\pgfpathlineto{\pgfqpoint{3.506380in}{1.663643in}}%
\pgfpathlineto{\pgfqpoint{3.492558in}{1.666746in}}%
\pgfpathlineto{\pgfqpoint{3.484280in}{1.657743in}}%
\pgfpathlineto{\pgfqpoint{3.475994in}{1.648779in}}%
\pgfpathlineto{\pgfqpoint{3.467702in}{1.639857in}}%
\pgfpathlineto{\pgfqpoint{3.459404in}{1.630981in}}%
\pgfpathclose%
\pgfusepath{fill}%
\end{pgfscope}%
\begin{pgfscope}%
\pgfpathrectangle{\pgfqpoint{1.150000in}{0.150000in}}{\pgfqpoint{5.700000in}{5.700000in}}%
\pgfusepath{clip}%
\pgfsetbuttcap%
\pgfsetroundjoin%
\definecolor{currentfill}{rgb}{0.282910,0.105393,0.426902}%
\pgfsetfillcolor{currentfill}%
\pgfsetfillopacity{0.700000}%
\pgfsetlinewidth{0.000000pt}%
\definecolor{currentstroke}{rgb}{0.000000,0.000000,0.000000}%
\pgfsetstrokecolor{currentstroke}%
\pgfsetdash{}{0pt}%
\pgfpathmoveto{\pgfqpoint{2.537590in}{1.825860in}}%
\pgfpathlineto{\pgfqpoint{2.551395in}{1.816076in}}%
\pgfpathlineto{\pgfqpoint{2.565202in}{1.806396in}}%
\pgfpathlineto{\pgfqpoint{2.579008in}{1.796819in}}%
\pgfpathlineto{\pgfqpoint{2.592816in}{1.787344in}}%
\pgfpathlineto{\pgfqpoint{2.601627in}{1.789622in}}%
\pgfpathlineto{\pgfqpoint{2.610424in}{1.792113in}}%
\pgfpathlineto{\pgfqpoint{2.619206in}{1.794812in}}%
\pgfpathlineto{\pgfqpoint{2.627974in}{1.797714in}}%
\pgfpathlineto{\pgfqpoint{2.614198in}{1.806835in}}%
\pgfpathlineto{\pgfqpoint{2.600423in}{1.816058in}}%
\pgfpathlineto{\pgfqpoint{2.586649in}{1.825383in}}%
\pgfpathlineto{\pgfqpoint{2.572876in}{1.834811in}}%
\pgfpathlineto{\pgfqpoint{2.564077in}{1.832256in}}%
\pgfpathlineto{\pgfqpoint{2.555263in}{1.829909in}}%
\pgfpathlineto{\pgfqpoint{2.546434in}{1.827775in}}%
\pgfpathlineto{\pgfqpoint{2.537590in}{1.825860in}}%
\pgfpathclose%
\pgfusepath{fill}%
\end{pgfscope}%
\begin{pgfscope}%
\pgfpathrectangle{\pgfqpoint{1.150000in}{0.150000in}}{\pgfqpoint{5.700000in}{5.700000in}}%
\pgfusepath{clip}%
\pgfsetbuttcap%
\pgfsetroundjoin%
\definecolor{currentfill}{rgb}{0.241237,0.296485,0.539709}%
\pgfsetfillcolor{currentfill}%
\pgfsetfillopacity{0.700000}%
\pgfsetlinewidth{0.000000pt}%
\definecolor{currentstroke}{rgb}{0.000000,0.000000,0.000000}%
\pgfsetstrokecolor{currentstroke}%
\pgfsetdash{}{0pt}%
\pgfpathmoveto{\pgfqpoint{4.894765in}{2.203195in}}%
\pgfpathlineto{\pgfqpoint{4.909050in}{2.206115in}}%
\pgfpathlineto{\pgfqpoint{4.923347in}{2.209107in}}%
\pgfpathlineto{\pgfqpoint{4.937656in}{2.212170in}}%
\pgfpathlineto{\pgfqpoint{4.951976in}{2.215303in}}%
\pgfpathlineto{\pgfqpoint{4.959715in}{2.222683in}}%
\pgfpathlineto{\pgfqpoint{4.967445in}{2.229970in}}%
\pgfpathlineto{\pgfqpoint{4.975169in}{2.237166in}}%
\pgfpathlineto{\pgfqpoint{4.982884in}{2.244273in}}%
\pgfpathlineto{\pgfqpoint{4.968577in}{2.241229in}}%
\pgfpathlineto{\pgfqpoint{4.954281in}{2.238255in}}%
\pgfpathlineto{\pgfqpoint{4.939997in}{2.235352in}}%
\pgfpathlineto{\pgfqpoint{4.925724in}{2.232520in}}%
\pgfpathlineto{\pgfqpoint{4.917995in}{2.225316in}}%
\pgfpathlineto{\pgfqpoint{4.910259in}{2.218028in}}%
\pgfpathlineto{\pgfqpoint{4.902515in}{2.210655in}}%
\pgfpathlineto{\pgfqpoint{4.894765in}{2.203195in}}%
\pgfpathclose%
\pgfusepath{fill}%
\end{pgfscope}%
\begin{pgfscope}%
\pgfpathrectangle{\pgfqpoint{1.150000in}{0.150000in}}{\pgfqpoint{5.700000in}{5.700000in}}%
\pgfusepath{clip}%
\pgfsetbuttcap%
\pgfsetroundjoin%
\definecolor{currentfill}{rgb}{0.268510,0.009605,0.335427}%
\pgfsetfillcolor{currentfill}%
\pgfsetfillopacity{0.700000}%
\pgfsetlinewidth{0.000000pt}%
\definecolor{currentstroke}{rgb}{0.000000,0.000000,0.000000}%
\pgfsetstrokecolor{currentstroke}%
\pgfsetdash{}{0pt}%
\pgfpathmoveto{\pgfqpoint{3.082595in}{1.629023in}}%
\pgfpathlineto{\pgfqpoint{3.096387in}{1.623242in}}%
\pgfpathlineto{\pgfqpoint{3.110183in}{1.617547in}}%
\pgfpathlineto{\pgfqpoint{3.123984in}{1.611938in}}%
\pgfpathlineto{\pgfqpoint{3.137788in}{1.606414in}}%
\pgfpathlineto{\pgfqpoint{3.146250in}{1.613279in}}%
\pgfpathlineto{\pgfqpoint{3.154703in}{1.620256in}}%
\pgfpathlineto{\pgfqpoint{3.163146in}{1.627341in}}%
\pgfpathlineto{\pgfqpoint{3.171582in}{1.634529in}}%
\pgfpathlineto{\pgfqpoint{3.157798in}{1.639766in}}%
\pgfpathlineto{\pgfqpoint{3.144018in}{1.645088in}}%
\pgfpathlineto{\pgfqpoint{3.130242in}{1.650496in}}%
\pgfpathlineto{\pgfqpoint{3.116470in}{1.655990in}}%
\pgfpathlineto{\pgfqpoint{3.108015in}{1.649081in}}%
\pgfpathlineto{\pgfqpoint{3.099551in}{1.642281in}}%
\pgfpathlineto{\pgfqpoint{3.091077in}{1.635594in}}%
\pgfpathlineto{\pgfqpoint{3.082595in}{1.629023in}}%
\pgfpathclose%
\pgfusepath{fill}%
\end{pgfscope}%
\begin{pgfscope}%
\pgfpathrectangle{\pgfqpoint{1.150000in}{0.150000in}}{\pgfqpoint{5.700000in}{5.700000in}}%
\pgfusepath{clip}%
\pgfsetbuttcap%
\pgfsetroundjoin%
\definecolor{currentfill}{rgb}{0.277941,0.056324,0.381191}%
\pgfsetfillcolor{currentfill}%
\pgfsetfillopacity{0.700000}%
\pgfsetlinewidth{0.000000pt}%
\definecolor{currentstroke}{rgb}{0.000000,0.000000,0.000000}%
\pgfsetstrokecolor{currentstroke}%
\pgfsetdash{}{0pt}%
\pgfpathmoveto{\pgfqpoint{3.780080in}{1.705810in}}%
\pgfpathlineto{\pgfqpoint{3.793985in}{1.704340in}}%
\pgfpathlineto{\pgfqpoint{3.807897in}{1.702946in}}%
\pgfpathlineto{\pgfqpoint{3.821817in}{1.701628in}}%
\pgfpathlineto{\pgfqpoint{3.835744in}{1.700385in}}%
\pgfpathlineto{\pgfqpoint{3.843908in}{1.710472in}}%
\pgfpathlineto{\pgfqpoint{3.852066in}{1.720546in}}%
\pgfpathlineto{\pgfqpoint{3.860219in}{1.730604in}}%
\pgfpathlineto{\pgfqpoint{3.868365in}{1.740644in}}%
\pgfpathlineto{\pgfqpoint{3.854448in}{1.741723in}}%
\pgfpathlineto{\pgfqpoint{3.840539in}{1.742877in}}%
\pgfpathlineto{\pgfqpoint{3.826637in}{1.744107in}}%
\pgfpathlineto{\pgfqpoint{3.812743in}{1.745413in}}%
\pgfpathlineto{\pgfqpoint{3.804586in}{1.735529in}}%
\pgfpathlineto{\pgfqpoint{3.796423in}{1.725632in}}%
\pgfpathlineto{\pgfqpoint{3.788254in}{1.715725in}}%
\pgfpathlineto{\pgfqpoint{3.780080in}{1.705810in}}%
\pgfpathclose%
\pgfusepath{fill}%
\end{pgfscope}%
\begin{pgfscope}%
\pgfpathrectangle{\pgfqpoint{1.150000in}{0.150000in}}{\pgfqpoint{5.700000in}{5.700000in}}%
\pgfusepath{clip}%
\pgfsetbuttcap%
\pgfsetroundjoin%
\definecolor{currentfill}{rgb}{0.271305,0.019942,0.347269}%
\pgfsetfillcolor{currentfill}%
\pgfsetfillopacity{0.700000}%
\pgfsetlinewidth{0.000000pt}%
\definecolor{currentstroke}{rgb}{0.000000,0.000000,0.000000}%
\pgfsetstrokecolor{currentstroke}%
\pgfsetdash{}{0pt}%
\pgfpathmoveto{\pgfqpoint{2.938174in}{1.655859in}}%
\pgfpathlineto{\pgfqpoint{2.951962in}{1.649067in}}%
\pgfpathlineto{\pgfqpoint{2.965753in}{1.642365in}}%
\pgfpathlineto{\pgfqpoint{2.979548in}{1.635751in}}%
\pgfpathlineto{\pgfqpoint{2.993345in}{1.629226in}}%
\pgfpathlineto{\pgfqpoint{3.001890in}{1.634967in}}%
\pgfpathlineto{\pgfqpoint{3.010424in}{1.640849in}}%
\pgfpathlineto{\pgfqpoint{3.018948in}{1.646866in}}%
\pgfpathlineto{\pgfqpoint{3.027462in}{1.653014in}}%
\pgfpathlineto{\pgfqpoint{3.013688in}{1.659231in}}%
\pgfpathlineto{\pgfqpoint{2.999917in}{1.665536in}}%
\pgfpathlineto{\pgfqpoint{2.986149in}{1.671930in}}%
\pgfpathlineto{\pgfqpoint{2.972385in}{1.678413in}}%
\pgfpathlineto{\pgfqpoint{2.963848in}{1.672566in}}%
\pgfpathlineto{\pgfqpoint{2.955300in}{1.666854in}}%
\pgfpathlineto{\pgfqpoint{2.946743in}{1.661284in}}%
\pgfpathlineto{\pgfqpoint{2.938174in}{1.655859in}}%
\pgfpathclose%
\pgfusepath{fill}%
\end{pgfscope}%
\begin{pgfscope}%
\pgfpathrectangle{\pgfqpoint{1.150000in}{0.150000in}}{\pgfqpoint{5.700000in}{5.700000in}}%
\pgfusepath{clip}%
\pgfsetbuttcap%
\pgfsetroundjoin%
\definecolor{currentfill}{rgb}{0.248629,0.278775,0.534556}%
\pgfsetfillcolor{currentfill}%
\pgfsetfillopacity{0.700000}%
\pgfsetlinewidth{0.000000pt}%
\definecolor{currentstroke}{rgb}{0.000000,0.000000,0.000000}%
\pgfsetstrokecolor{currentstroke}%
\pgfsetdash{}{0pt}%
\pgfpathmoveto{\pgfqpoint{4.806611in}{2.161202in}}%
\pgfpathlineto{\pgfqpoint{4.820864in}{2.163906in}}%
\pgfpathlineto{\pgfqpoint{4.835127in}{2.166680in}}%
\pgfpathlineto{\pgfqpoint{4.849402in}{2.169526in}}%
\pgfpathlineto{\pgfqpoint{4.863688in}{2.172443in}}%
\pgfpathlineto{\pgfqpoint{4.871468in}{2.180271in}}%
\pgfpathlineto{\pgfqpoint{4.879241in}{2.188004in}}%
\pgfpathlineto{\pgfqpoint{4.887006in}{2.195645in}}%
\pgfpathlineto{\pgfqpoint{4.894765in}{2.203195in}}%
\pgfpathlineto{\pgfqpoint{4.880490in}{2.200345in}}%
\pgfpathlineto{\pgfqpoint{4.866228in}{2.197566in}}%
\pgfpathlineto{\pgfqpoint{4.851976in}{2.194859in}}%
\pgfpathlineto{\pgfqpoint{4.837736in}{2.192222in}}%
\pgfpathlineto{\pgfqpoint{4.829965in}{2.184597in}}%
\pgfpathlineto{\pgfqpoint{4.822188in}{2.176886in}}%
\pgfpathlineto{\pgfqpoint{4.814403in}{2.169089in}}%
\pgfpathlineto{\pgfqpoint{4.806611in}{2.161202in}}%
\pgfpathclose%
\pgfusepath{fill}%
\end{pgfscope}%
\begin{pgfscope}%
\pgfpathrectangle{\pgfqpoint{1.150000in}{0.150000in}}{\pgfqpoint{5.700000in}{5.700000in}}%
\pgfusepath{clip}%
\pgfsetbuttcap%
\pgfsetroundjoin%
\definecolor{currentfill}{rgb}{0.267004,0.004874,0.329415}%
\pgfsetfillcolor{currentfill}%
\pgfsetfillopacity{0.700000}%
\pgfsetlinewidth{0.000000pt}%
\definecolor{currentstroke}{rgb}{0.000000,0.000000,0.000000}%
\pgfsetstrokecolor{currentstroke}%
\pgfsetdash{}{0pt}%
\pgfpathmoveto{\pgfqpoint{3.226762in}{1.614426in}}%
\pgfpathlineto{\pgfqpoint{3.240568in}{1.609609in}}%
\pgfpathlineto{\pgfqpoint{3.254379in}{1.604875in}}%
\pgfpathlineto{\pgfqpoint{3.268195in}{1.600224in}}%
\pgfpathlineto{\pgfqpoint{3.282016in}{1.595656in}}%
\pgfpathlineto{\pgfqpoint{3.290405in}{1.603491in}}%
\pgfpathlineto{\pgfqpoint{3.298787in}{1.611412in}}%
\pgfpathlineto{\pgfqpoint{3.307161in}{1.619414in}}%
\pgfpathlineto{\pgfqpoint{3.315527in}{1.627492in}}%
\pgfpathlineto{\pgfqpoint{3.301724in}{1.631795in}}%
\pgfpathlineto{\pgfqpoint{3.287926in}{1.636179in}}%
\pgfpathlineto{\pgfqpoint{3.274133in}{1.640647in}}%
\pgfpathlineto{\pgfqpoint{3.260344in}{1.645197in}}%
\pgfpathlineto{\pgfqpoint{3.251961in}{1.637377in}}%
\pgfpathlineto{\pgfqpoint{3.243569in}{1.629639in}}%
\pgfpathlineto{\pgfqpoint{3.235170in}{1.621987in}}%
\pgfpathlineto{\pgfqpoint{3.226762in}{1.614426in}}%
\pgfpathclose%
\pgfusepath{fill}%
\end{pgfscope}%
\begin{pgfscope}%
\pgfpathrectangle{\pgfqpoint{1.150000in}{0.150000in}}{\pgfqpoint{5.700000in}{5.700000in}}%
\pgfusepath{clip}%
\pgfsetbuttcap%
\pgfsetroundjoin%
\definecolor{currentfill}{rgb}{0.192357,0.403199,0.555836}%
\pgfsetfillcolor{currentfill}%
\pgfsetfillopacity{0.700000}%
\pgfsetlinewidth{0.000000pt}%
\definecolor{currentstroke}{rgb}{0.000000,0.000000,0.000000}%
\pgfsetstrokecolor{currentstroke}%
\pgfsetdash{}{0pt}%
\pgfpathmoveto{\pgfqpoint{5.568607in}{2.480970in}}%
\pgfpathlineto{\pgfqpoint{5.583171in}{2.485057in}}%
\pgfpathlineto{\pgfqpoint{5.597748in}{2.489215in}}%
\pgfpathlineto{\pgfqpoint{5.612339in}{2.493442in}}%
\pgfpathlineto{\pgfqpoint{5.626942in}{2.497738in}}%
\pgfpathlineto{\pgfqpoint{5.634319in}{2.501660in}}%
\pgfpathlineto{\pgfqpoint{5.641687in}{2.505532in}}%
\pgfpathlineto{\pgfqpoint{5.649048in}{2.509359in}}%
\pgfpathlineto{\pgfqpoint{5.656401in}{2.513145in}}%
\pgfpathlineto{\pgfqpoint{5.641820in}{2.509091in}}%
\pgfpathlineto{\pgfqpoint{5.627252in}{2.505105in}}%
\pgfpathlineto{\pgfqpoint{5.612697in}{2.501189in}}%
\pgfpathlineto{\pgfqpoint{5.598155in}{2.497342in}}%
\pgfpathlineto{\pgfqpoint{5.590780in}{2.493308in}}%
\pgfpathlineto{\pgfqpoint{5.583397in}{2.489237in}}%
\pgfpathlineto{\pgfqpoint{5.576006in}{2.485125in}}%
\pgfpathlineto{\pgfqpoint{5.568607in}{2.480970in}}%
\pgfpathclose%
\pgfusepath{fill}%
\end{pgfscope}%
\begin{pgfscope}%
\pgfpathrectangle{\pgfqpoint{1.150000in}{0.150000in}}{\pgfqpoint{5.700000in}{5.700000in}}%
\pgfusepath{clip}%
\pgfsetbuttcap%
\pgfsetroundjoin%
\definecolor{currentfill}{rgb}{0.255645,0.260703,0.528312}%
\pgfsetfillcolor{currentfill}%
\pgfsetfillopacity{0.700000}%
\pgfsetlinewidth{0.000000pt}%
\definecolor{currentstroke}{rgb}{0.000000,0.000000,0.000000}%
\pgfsetstrokecolor{currentstroke}%
\pgfsetdash{}{0pt}%
\pgfpathmoveto{\pgfqpoint{4.718431in}{2.118457in}}%
\pgfpathlineto{\pgfqpoint{4.732650in}{2.120921in}}%
\pgfpathlineto{\pgfqpoint{4.746880in}{2.123456in}}%
\pgfpathlineto{\pgfqpoint{4.761121in}{2.126063in}}%
\pgfpathlineto{\pgfqpoint{4.775373in}{2.128741in}}%
\pgfpathlineto{\pgfqpoint{4.783193in}{2.136996in}}%
\pgfpathlineto{\pgfqpoint{4.791006in}{2.145157in}}%
\pgfpathlineto{\pgfqpoint{4.798812in}{2.153225in}}%
\pgfpathlineto{\pgfqpoint{4.806611in}{2.161202in}}%
\pgfpathlineto{\pgfqpoint{4.792371in}{2.158570in}}%
\pgfpathlineto{\pgfqpoint{4.778141in}{2.156009in}}%
\pgfpathlineto{\pgfqpoint{4.763922in}{2.153519in}}%
\pgfpathlineto{\pgfqpoint{4.749714in}{2.151100in}}%
\pgfpathlineto{\pgfqpoint{4.741904in}{2.143070in}}%
\pgfpathlineto{\pgfqpoint{4.734087in}{2.134953in}}%
\pgfpathlineto{\pgfqpoint{4.726262in}{2.126749in}}%
\pgfpathlineto{\pgfqpoint{4.718431in}{2.118457in}}%
\pgfpathclose%
\pgfusepath{fill}%
\end{pgfscope}%
\begin{pgfscope}%
\pgfpathrectangle{\pgfqpoint{1.150000in}{0.150000in}}{\pgfqpoint{5.700000in}{5.700000in}}%
\pgfusepath{clip}%
\pgfsetbuttcap%
\pgfsetroundjoin%
\definecolor{currentfill}{rgb}{0.262138,0.242286,0.520837}%
\pgfsetfillcolor{currentfill}%
\pgfsetfillopacity{0.700000}%
\pgfsetlinewidth{0.000000pt}%
\definecolor{currentstroke}{rgb}{0.000000,0.000000,0.000000}%
\pgfsetstrokecolor{currentstroke}%
\pgfsetdash{}{0pt}%
\pgfpathmoveto{\pgfqpoint{4.630230in}{2.075141in}}%
\pgfpathlineto{\pgfqpoint{4.644416in}{2.077343in}}%
\pgfpathlineto{\pgfqpoint{4.658613in}{2.079617in}}%
\pgfpathlineto{\pgfqpoint{4.672820in}{2.081962in}}%
\pgfpathlineto{\pgfqpoint{4.687038in}{2.084379in}}%
\pgfpathlineto{\pgfqpoint{4.694897in}{2.093036in}}%
\pgfpathlineto{\pgfqpoint{4.702749in}{2.101601in}}%
\pgfpathlineto{\pgfqpoint{4.710593in}{2.110074in}}%
\pgfpathlineto{\pgfqpoint{4.718431in}{2.118457in}}%
\pgfpathlineto{\pgfqpoint{4.704223in}{2.116064in}}%
\pgfpathlineto{\pgfqpoint{4.690026in}{2.113743in}}%
\pgfpathlineto{\pgfqpoint{4.675840in}{2.111494in}}%
\pgfpathlineto{\pgfqpoint{4.661665in}{2.109316in}}%
\pgfpathlineto{\pgfqpoint{4.653816in}{2.100901in}}%
\pgfpathlineto{\pgfqpoint{4.645961in}{2.092401in}}%
\pgfpathlineto{\pgfqpoint{4.638099in}{2.083815in}}%
\pgfpathlineto{\pgfqpoint{4.630230in}{2.075141in}}%
\pgfpathclose%
\pgfusepath{fill}%
\end{pgfscope}%
\begin{pgfscope}%
\pgfpathrectangle{\pgfqpoint{1.150000in}{0.150000in}}{\pgfqpoint{5.700000in}{5.700000in}}%
\pgfusepath{clip}%
\pgfsetbuttcap%
\pgfsetroundjoin%
\definecolor{currentfill}{rgb}{0.277134,0.185228,0.489898}%
\pgfsetfillcolor{currentfill}%
\pgfsetfillopacity{0.700000}%
\pgfsetlinewidth{0.000000pt}%
\definecolor{currentstroke}{rgb}{0.000000,0.000000,0.000000}%
\pgfsetstrokecolor{currentstroke}%
\pgfsetdash{}{0pt}%
\pgfpathmoveto{\pgfqpoint{2.280524in}{1.997759in}}%
\pgfpathlineto{\pgfqpoint{2.294378in}{1.985837in}}%
\pgfpathlineto{\pgfqpoint{2.308231in}{1.974032in}}%
\pgfpathlineto{\pgfqpoint{2.322082in}{1.962342in}}%
\pgfpathlineto{\pgfqpoint{2.335932in}{1.950767in}}%
\pgfpathlineto{\pgfqpoint{2.344952in}{1.950589in}}%
\pgfpathlineto{\pgfqpoint{2.353954in}{1.950670in}}%
\pgfpathlineto{\pgfqpoint{2.362937in}{1.951003in}}%
\pgfpathlineto{\pgfqpoint{2.371903in}{1.951583in}}%
\pgfpathlineto{\pgfqpoint{2.358092in}{1.962778in}}%
\pgfpathlineto{\pgfqpoint{2.344279in}{1.974086in}}%
\pgfpathlineto{\pgfqpoint{2.330464in}{1.985510in}}%
\pgfpathlineto{\pgfqpoint{2.316648in}{1.997050in}}%
\pgfpathlineto{\pgfqpoint{2.307645in}{1.996843in}}%
\pgfpathlineto{\pgfqpoint{2.298623in}{1.996888in}}%
\pgfpathlineto{\pgfqpoint{2.289583in}{1.997191in}}%
\pgfpathlineto{\pgfqpoint{2.280524in}{1.997759in}}%
\pgfpathclose%
\pgfusepath{fill}%
\end{pgfscope}%
\begin{pgfscope}%
\pgfpathrectangle{\pgfqpoint{1.150000in}{0.150000in}}{\pgfqpoint{5.700000in}{5.700000in}}%
\pgfusepath{clip}%
\pgfsetbuttcap%
\pgfsetroundjoin%
\definecolor{currentfill}{rgb}{0.276022,0.044167,0.370164}%
\pgfsetfillcolor{currentfill}%
\pgfsetfillopacity{0.700000}%
\pgfsetlinewidth{0.000000pt}%
\definecolor{currentstroke}{rgb}{0.000000,0.000000,0.000000}%
\pgfsetstrokecolor{currentstroke}%
\pgfsetdash{}{0pt}%
\pgfpathmoveto{\pgfqpoint{3.691733in}{1.673499in}}%
\pgfpathlineto{\pgfqpoint{3.705620in}{1.671538in}}%
\pgfpathlineto{\pgfqpoint{3.719515in}{1.669655in}}%
\pgfpathlineto{\pgfqpoint{3.733416in}{1.667847in}}%
\pgfpathlineto{\pgfqpoint{3.747325in}{1.666116in}}%
\pgfpathlineto{\pgfqpoint{3.755522in}{1.676040in}}%
\pgfpathlineto{\pgfqpoint{3.763714in}{1.685965in}}%
\pgfpathlineto{\pgfqpoint{3.771900in}{1.695890in}}%
\pgfpathlineto{\pgfqpoint{3.780080in}{1.705810in}}%
\pgfpathlineto{\pgfqpoint{3.766182in}{1.707357in}}%
\pgfpathlineto{\pgfqpoint{3.752292in}{1.708979in}}%
\pgfpathlineto{\pgfqpoint{3.738409in}{1.710679in}}%
\pgfpathlineto{\pgfqpoint{3.724533in}{1.712455in}}%
\pgfpathlineto{\pgfqpoint{3.716342in}{1.702711in}}%
\pgfpathlineto{\pgfqpoint{3.708145in}{1.692969in}}%
\pgfpathlineto{\pgfqpoint{3.699942in}{1.683230in}}%
\pgfpathlineto{\pgfqpoint{3.691733in}{1.673499in}}%
\pgfpathclose%
\pgfusepath{fill}%
\end{pgfscope}%
\begin{pgfscope}%
\pgfpathrectangle{\pgfqpoint{1.150000in}{0.150000in}}{\pgfqpoint{5.700000in}{5.700000in}}%
\pgfusepath{clip}%
\pgfsetbuttcap%
\pgfsetroundjoin%
\definecolor{currentfill}{rgb}{0.267968,0.223549,0.512008}%
\pgfsetfillcolor{currentfill}%
\pgfsetfillopacity{0.700000}%
\pgfsetlinewidth{0.000000pt}%
\definecolor{currentstroke}{rgb}{0.000000,0.000000,0.000000}%
\pgfsetstrokecolor{currentstroke}%
\pgfsetdash{}{0pt}%
\pgfpathmoveto{\pgfqpoint{4.542012in}{2.031460in}}%
\pgfpathlineto{\pgfqpoint{4.556166in}{2.033377in}}%
\pgfpathlineto{\pgfqpoint{4.570330in}{2.035367in}}%
\pgfpathlineto{\pgfqpoint{4.584504in}{2.037428in}}%
\pgfpathlineto{\pgfqpoint{4.598689in}{2.039561in}}%
\pgfpathlineto{\pgfqpoint{4.606584in}{2.048590in}}%
\pgfpathlineto{\pgfqpoint{4.614473in}{2.057529in}}%
\pgfpathlineto{\pgfqpoint{4.622355in}{2.066380in}}%
\pgfpathlineto{\pgfqpoint{4.630230in}{2.075141in}}%
\pgfpathlineto{\pgfqpoint{4.616055in}{2.073011in}}%
\pgfpathlineto{\pgfqpoint{4.601890in}{2.070952in}}%
\pgfpathlineto{\pgfqpoint{4.587736in}{2.068966in}}%
\pgfpathlineto{\pgfqpoint{4.573592in}{2.067051in}}%
\pgfpathlineto{\pgfqpoint{4.565707in}{2.058279in}}%
\pgfpathlineto{\pgfqpoint{4.557815in}{2.049423in}}%
\pgfpathlineto{\pgfqpoint{4.549917in}{2.040484in}}%
\pgfpathlineto{\pgfqpoint{4.542012in}{2.031460in}}%
\pgfpathclose%
\pgfusepath{fill}%
\end{pgfscope}%
\begin{pgfscope}%
\pgfpathrectangle{\pgfqpoint{1.150000in}{0.150000in}}{\pgfqpoint{5.700000in}{5.700000in}}%
\pgfusepath{clip}%
\pgfsetbuttcap%
\pgfsetroundjoin%
\definecolor{currentfill}{rgb}{0.195860,0.395433,0.555276}%
\pgfsetfillcolor{currentfill}%
\pgfsetfillopacity{0.700000}%
\pgfsetlinewidth{0.000000pt}%
\definecolor{currentstroke}{rgb}{0.000000,0.000000,0.000000}%
\pgfsetstrokecolor{currentstroke}%
\pgfsetdash{}{0pt}%
\pgfpathmoveto{\pgfqpoint{5.480724in}{2.447284in}}%
\pgfpathlineto{\pgfqpoint{5.495256in}{2.451313in}}%
\pgfpathlineto{\pgfqpoint{5.509802in}{2.455412in}}%
\pgfpathlineto{\pgfqpoint{5.524361in}{2.459580in}}%
\pgfpathlineto{\pgfqpoint{5.538933in}{2.463819in}}%
\pgfpathlineto{\pgfqpoint{5.546364in}{2.468194in}}%
\pgfpathlineto{\pgfqpoint{5.553786in}{2.472508in}}%
\pgfpathlineto{\pgfqpoint{5.561201in}{2.476765in}}%
\pgfpathlineto{\pgfqpoint{5.568607in}{2.480970in}}%
\pgfpathlineto{\pgfqpoint{5.554056in}{2.476951in}}%
\pgfpathlineto{\pgfqpoint{5.539518in}{2.473003in}}%
\pgfpathlineto{\pgfqpoint{5.524993in}{2.469123in}}%
\pgfpathlineto{\pgfqpoint{5.510481in}{2.465313in}}%
\pgfpathlineto{\pgfqpoint{5.503053in}{2.460881in}}%
\pgfpathlineto{\pgfqpoint{5.495618in}{2.456402in}}%
\pgfpathlineto{\pgfqpoint{5.488175in}{2.451870in}}%
\pgfpathlineto{\pgfqpoint{5.480724in}{2.447284in}}%
\pgfpathclose%
\pgfusepath{fill}%
\end{pgfscope}%
\begin{pgfscope}%
\pgfpathrectangle{\pgfqpoint{1.150000in}{0.150000in}}{\pgfqpoint{5.700000in}{5.700000in}}%
\pgfusepath{clip}%
\pgfsetbuttcap%
\pgfsetroundjoin%
\definecolor{currentfill}{rgb}{0.273006,0.204520,0.501721}%
\pgfsetfillcolor{currentfill}%
\pgfsetfillopacity{0.700000}%
\pgfsetlinewidth{0.000000pt}%
\definecolor{currentstroke}{rgb}{0.000000,0.000000,0.000000}%
\pgfsetstrokecolor{currentstroke}%
\pgfsetdash{}{0pt}%
\pgfpathmoveto{\pgfqpoint{4.453781in}{1.987638in}}%
\pgfpathlineto{\pgfqpoint{4.467903in}{1.989249in}}%
\pgfpathlineto{\pgfqpoint{4.482035in}{1.990932in}}%
\pgfpathlineto{\pgfqpoint{4.496177in}{1.992687in}}%
\pgfpathlineto{\pgfqpoint{4.510330in}{1.994514in}}%
\pgfpathlineto{\pgfqpoint{4.518260in}{2.003878in}}%
\pgfpathlineto{\pgfqpoint{4.526184in}{2.013157in}}%
\pgfpathlineto{\pgfqpoint{4.534101in}{2.022351in}}%
\pgfpathlineto{\pgfqpoint{4.542012in}{2.031460in}}%
\pgfpathlineto{\pgfqpoint{4.527869in}{2.029614in}}%
\pgfpathlineto{\pgfqpoint{4.513736in}{2.027841in}}%
\pgfpathlineto{\pgfqpoint{4.499614in}{2.026140in}}%
\pgfpathlineto{\pgfqpoint{4.485501in}{2.024511in}}%
\pgfpathlineto{\pgfqpoint{4.477581in}{2.015413in}}%
\pgfpathlineto{\pgfqpoint{4.469654in}{2.006235in}}%
\pgfpathlineto{\pgfqpoint{4.461721in}{1.996977in}}%
\pgfpathlineto{\pgfqpoint{4.453781in}{1.987638in}}%
\pgfpathclose%
\pgfusepath{fill}%
\end{pgfscope}%
\begin{pgfscope}%
\pgfpathrectangle{\pgfqpoint{1.150000in}{0.150000in}}{\pgfqpoint{5.700000in}{5.700000in}}%
\pgfusepath{clip}%
\pgfsetbuttcap%
\pgfsetroundjoin%
\definecolor{currentfill}{rgb}{0.276022,0.044167,0.370164}%
\pgfsetfillcolor{currentfill}%
\pgfsetfillopacity{0.700000}%
\pgfsetlinewidth{0.000000pt}%
\definecolor{currentstroke}{rgb}{0.000000,0.000000,0.000000}%
\pgfsetstrokecolor{currentstroke}%
\pgfsetdash{}{0pt}%
\pgfpathmoveto{\pgfqpoint{2.793373in}{1.695966in}}%
\pgfpathlineto{\pgfqpoint{2.807167in}{1.688111in}}%
\pgfpathlineto{\pgfqpoint{2.820964in}{1.680349in}}%
\pgfpathlineto{\pgfqpoint{2.834762in}{1.672680in}}%
\pgfpathlineto{\pgfqpoint{2.848563in}{1.665104in}}%
\pgfpathlineto{\pgfqpoint{2.857202in}{1.669560in}}%
\pgfpathlineto{\pgfqpoint{2.865830in}{1.674186in}}%
\pgfpathlineto{\pgfqpoint{2.874446in}{1.678978in}}%
\pgfpathlineto{\pgfqpoint{2.883050in}{1.683931in}}%
\pgfpathlineto{\pgfqpoint{2.869276in}{1.691177in}}%
\pgfpathlineto{\pgfqpoint{2.855504in}{1.698515in}}%
\pgfpathlineto{\pgfqpoint{2.841734in}{1.705946in}}%
\pgfpathlineto{\pgfqpoint{2.827967in}{1.713470in}}%
\pgfpathlineto{\pgfqpoint{2.819337in}{1.708840in}}%
\pgfpathlineto{\pgfqpoint{2.810695in}{1.704376in}}%
\pgfpathlineto{\pgfqpoint{2.802040in}{1.700083in}}%
\pgfpathlineto{\pgfqpoint{2.793373in}{1.695966in}}%
\pgfpathclose%
\pgfusepath{fill}%
\end{pgfscope}%
\begin{pgfscope}%
\pgfpathrectangle{\pgfqpoint{1.150000in}{0.150000in}}{\pgfqpoint{5.700000in}{5.700000in}}%
\pgfusepath{clip}%
\pgfsetbuttcap%
\pgfsetroundjoin%
\definecolor{currentfill}{rgb}{0.267004,0.004874,0.329415}%
\pgfsetfillcolor{currentfill}%
\pgfsetfillopacity{0.700000}%
\pgfsetlinewidth{0.000000pt}%
\definecolor{currentstroke}{rgb}{0.000000,0.000000,0.000000}%
\pgfsetstrokecolor{currentstroke}%
\pgfsetdash{}{0pt}%
\pgfpathmoveto{\pgfqpoint{3.370791in}{1.611101in}}%
\pgfpathlineto{\pgfqpoint{3.384620in}{1.607206in}}%
\pgfpathlineto{\pgfqpoint{3.398455in}{1.603391in}}%
\pgfpathlineto{\pgfqpoint{3.412295in}{1.599657in}}%
\pgfpathlineto{\pgfqpoint{3.426141in}{1.596002in}}%
\pgfpathlineto{\pgfqpoint{3.434467in}{1.604661in}}%
\pgfpathlineto{\pgfqpoint{3.442786in}{1.613380in}}%
\pgfpathlineto{\pgfqpoint{3.451098in}{1.622155in}}%
\pgfpathlineto{\pgfqpoint{3.459404in}{1.630981in}}%
\pgfpathlineto{\pgfqpoint{3.445573in}{1.634390in}}%
\pgfpathlineto{\pgfqpoint{3.431748in}{1.637879in}}%
\pgfpathlineto{\pgfqpoint{3.417929in}{1.641447in}}%
\pgfpathlineto{\pgfqpoint{3.404116in}{1.645097in}}%
\pgfpathlineto{\pgfqpoint{3.395795in}{1.636508in}}%
\pgfpathlineto{\pgfqpoint{3.387468in}{1.627977in}}%
\pgfpathlineto{\pgfqpoint{3.379133in}{1.619506in}}%
\pgfpathlineto{\pgfqpoint{3.370791in}{1.611101in}}%
\pgfpathclose%
\pgfusepath{fill}%
\end{pgfscope}%
\begin{pgfscope}%
\pgfpathrectangle{\pgfqpoint{1.150000in}{0.150000in}}{\pgfqpoint{5.700000in}{5.700000in}}%
\pgfusepath{clip}%
\pgfsetbuttcap%
\pgfsetroundjoin%
\definecolor{currentfill}{rgb}{0.277134,0.185228,0.489898}%
\pgfsetfillcolor{currentfill}%
\pgfsetfillopacity{0.700000}%
\pgfsetlinewidth{0.000000pt}%
\definecolor{currentstroke}{rgb}{0.000000,0.000000,0.000000}%
\pgfsetstrokecolor{currentstroke}%
\pgfsetdash{}{0pt}%
\pgfpathmoveto{\pgfqpoint{4.365539in}{1.943924in}}%
\pgfpathlineto{\pgfqpoint{4.379630in}{1.945205in}}%
\pgfpathlineto{\pgfqpoint{4.393731in}{1.946559in}}%
\pgfpathlineto{\pgfqpoint{4.407842in}{1.947985in}}%
\pgfpathlineto{\pgfqpoint{4.421962in}{1.949484in}}%
\pgfpathlineto{\pgfqpoint{4.429926in}{1.959143in}}%
\pgfpathlineto{\pgfqpoint{4.437884in}{1.968721in}}%
\pgfpathlineto{\pgfqpoint{4.445836in}{1.978220in}}%
\pgfpathlineto{\pgfqpoint{4.453781in}{1.987638in}}%
\pgfpathlineto{\pgfqpoint{4.439670in}{1.986100in}}%
\pgfpathlineto{\pgfqpoint{4.425568in}{1.984634in}}%
\pgfpathlineto{\pgfqpoint{4.411476in}{1.983241in}}%
\pgfpathlineto{\pgfqpoint{4.397394in}{1.981920in}}%
\pgfpathlineto{\pgfqpoint{4.389440in}{1.972533in}}%
\pgfpathlineto{\pgfqpoint{4.381479in}{1.963072in}}%
\pgfpathlineto{\pgfqpoint{4.373512in}{1.953535in}}%
\pgfpathlineto{\pgfqpoint{4.365539in}{1.943924in}}%
\pgfpathclose%
\pgfusepath{fill}%
\end{pgfscope}%
\begin{pgfscope}%
\pgfpathrectangle{\pgfqpoint{1.150000in}{0.150000in}}{\pgfqpoint{5.700000in}{5.700000in}}%
\pgfusepath{clip}%
\pgfsetbuttcap%
\pgfsetroundjoin%
\definecolor{currentfill}{rgb}{0.281924,0.089666,0.412415}%
\pgfsetfillcolor{currentfill}%
\pgfsetfillopacity{0.700000}%
\pgfsetlinewidth{0.000000pt}%
\definecolor{currentstroke}{rgb}{0.000000,0.000000,0.000000}%
\pgfsetstrokecolor{currentstroke}%
\pgfsetdash{}{0pt}%
\pgfpathmoveto{\pgfqpoint{2.592816in}{1.787344in}}%
\pgfpathlineto{\pgfqpoint{2.606623in}{1.777969in}}%
\pgfpathlineto{\pgfqpoint{2.620432in}{1.768696in}}%
\pgfpathlineto{\pgfqpoint{2.634241in}{1.759522in}}%
\pgfpathlineto{\pgfqpoint{2.648052in}{1.750448in}}%
\pgfpathlineto{\pgfqpoint{2.656831in}{1.753088in}}%
\pgfpathlineto{\pgfqpoint{2.665596in}{1.755936in}}%
\pgfpathlineto{\pgfqpoint{2.674347in}{1.758987in}}%
\pgfpathlineto{\pgfqpoint{2.683085in}{1.762234in}}%
\pgfpathlineto{\pgfqpoint{2.669305in}{1.770954in}}%
\pgfpathlineto{\pgfqpoint{2.655527in}{1.779774in}}%
\pgfpathlineto{\pgfqpoint{2.641750in}{1.788694in}}%
\pgfpathlineto{\pgfqpoint{2.627974in}{1.797714in}}%
\pgfpathlineto{\pgfqpoint{2.619206in}{1.794812in}}%
\pgfpathlineto{\pgfqpoint{2.610424in}{1.792113in}}%
\pgfpathlineto{\pgfqpoint{2.601627in}{1.789622in}}%
\pgfpathlineto{\pgfqpoint{2.592816in}{1.787344in}}%
\pgfpathclose%
\pgfusepath{fill}%
\end{pgfscope}%
\begin{pgfscope}%
\pgfpathrectangle{\pgfqpoint{1.150000in}{0.150000in}}{\pgfqpoint{5.700000in}{5.700000in}}%
\pgfusepath{clip}%
\pgfsetbuttcap%
\pgfsetroundjoin%
\definecolor{currentfill}{rgb}{0.280255,0.165693,0.476498}%
\pgfsetfillcolor{currentfill}%
\pgfsetfillopacity{0.700000}%
\pgfsetlinewidth{0.000000pt}%
\definecolor{currentstroke}{rgb}{0.000000,0.000000,0.000000}%
\pgfsetstrokecolor{currentstroke}%
\pgfsetdash{}{0pt}%
\pgfpathmoveto{\pgfqpoint{4.277286in}{1.900585in}}%
\pgfpathlineto{\pgfqpoint{4.291348in}{1.901514in}}%
\pgfpathlineto{\pgfqpoint{4.305418in}{1.902517in}}%
\pgfpathlineto{\pgfqpoint{4.319498in}{1.903592in}}%
\pgfpathlineto{\pgfqpoint{4.333588in}{1.904740in}}%
\pgfpathlineto{\pgfqpoint{4.341585in}{1.914646in}}%
\pgfpathlineto{\pgfqpoint{4.349576in}{1.924479in}}%
\pgfpathlineto{\pgfqpoint{4.357560in}{1.934238in}}%
\pgfpathlineto{\pgfqpoint{4.365539in}{1.943924in}}%
\pgfpathlineto{\pgfqpoint{4.351458in}{1.942715in}}%
\pgfpathlineto{\pgfqpoint{4.337387in}{1.941579in}}%
\pgfpathlineto{\pgfqpoint{4.323325in}{1.940516in}}%
\pgfpathlineto{\pgfqpoint{4.309273in}{1.939526in}}%
\pgfpathlineto{\pgfqpoint{4.301285in}{1.929893in}}%
\pgfpathlineto{\pgfqpoint{4.293291in}{1.920192in}}%
\pgfpathlineto{\pgfqpoint{4.285292in}{1.910422in}}%
\pgfpathlineto{\pgfqpoint{4.277286in}{1.900585in}}%
\pgfpathclose%
\pgfusepath{fill}%
\end{pgfscope}%
\begin{pgfscope}%
\pgfpathrectangle{\pgfqpoint{1.150000in}{0.150000in}}{\pgfqpoint{5.700000in}{5.700000in}}%
\pgfusepath{clip}%
\pgfsetbuttcap%
\pgfsetroundjoin%
\definecolor{currentfill}{rgb}{0.272594,0.025563,0.353093}%
\pgfsetfillcolor{currentfill}%
\pgfsetfillopacity{0.700000}%
\pgfsetlinewidth{0.000000pt}%
\definecolor{currentstroke}{rgb}{0.000000,0.000000,0.000000}%
\pgfsetstrokecolor{currentstroke}%
\pgfsetdash{}{0pt}%
\pgfpathmoveto{\pgfqpoint{3.603308in}{1.644129in}}%
\pgfpathlineto{\pgfqpoint{3.617180in}{1.641654in}}%
\pgfpathlineto{\pgfqpoint{3.631060in}{1.639257in}}%
\pgfpathlineto{\pgfqpoint{3.644945in}{1.636938in}}%
\pgfpathlineto{\pgfqpoint{3.658838in}{1.634695in}}%
\pgfpathlineto{\pgfqpoint{3.667071in}{1.644372in}}%
\pgfpathlineto{\pgfqpoint{3.675297in}{1.654067in}}%
\pgfpathlineto{\pgfqpoint{3.683518in}{1.663777in}}%
\pgfpathlineto{\pgfqpoint{3.691733in}{1.673499in}}%
\pgfpathlineto{\pgfqpoint{3.677852in}{1.675537in}}%
\pgfpathlineto{\pgfqpoint{3.663979in}{1.677651in}}%
\pgfpathlineto{\pgfqpoint{3.650112in}{1.679843in}}%
\pgfpathlineto{\pgfqpoint{3.636252in}{1.682113in}}%
\pgfpathlineto{\pgfqpoint{3.628025in}{1.672588in}}%
\pgfpathlineto{\pgfqpoint{3.619792in}{1.663081in}}%
\pgfpathlineto{\pgfqpoint{3.611553in}{1.653593in}}%
\pgfpathlineto{\pgfqpoint{3.603308in}{1.644129in}}%
\pgfpathclose%
\pgfusepath{fill}%
\end{pgfscope}%
\begin{pgfscope}%
\pgfpathrectangle{\pgfqpoint{1.150000in}{0.150000in}}{\pgfqpoint{5.700000in}{5.700000in}}%
\pgfusepath{clip}%
\pgfsetbuttcap%
\pgfsetroundjoin%
\definecolor{currentfill}{rgb}{0.282290,0.145912,0.461510}%
\pgfsetfillcolor{currentfill}%
\pgfsetfillopacity{0.700000}%
\pgfsetlinewidth{0.000000pt}%
\definecolor{currentstroke}{rgb}{0.000000,0.000000,0.000000}%
\pgfsetstrokecolor{currentstroke}%
\pgfsetdash{}{0pt}%
\pgfpathmoveto{\pgfqpoint{4.189022in}{1.857910in}}%
\pgfpathlineto{\pgfqpoint{4.203054in}{1.858465in}}%
\pgfpathlineto{\pgfqpoint{4.217096in}{1.859094in}}%
\pgfpathlineto{\pgfqpoint{4.231147in}{1.859796in}}%
\pgfpathlineto{\pgfqpoint{4.245208in}{1.860571in}}%
\pgfpathlineto{\pgfqpoint{4.253236in}{1.870672in}}%
\pgfpathlineto{\pgfqpoint{4.261258in}{1.880709in}}%
\pgfpathlineto{\pgfqpoint{4.269275in}{1.890680in}}%
\pgfpathlineto{\pgfqpoint{4.277286in}{1.900585in}}%
\pgfpathlineto{\pgfqpoint{4.263235in}{1.899728in}}%
\pgfpathlineto{\pgfqpoint{4.249192in}{1.898945in}}%
\pgfpathlineto{\pgfqpoint{4.235159in}{1.898235in}}%
\pgfpathlineto{\pgfqpoint{4.221136in}{1.897598in}}%
\pgfpathlineto{\pgfqpoint{4.213116in}{1.887767in}}%
\pgfpathlineto{\pgfqpoint{4.205090in}{1.877875in}}%
\pgfpathlineto{\pgfqpoint{4.197059in}{1.867922in}}%
\pgfpathlineto{\pgfqpoint{4.189022in}{1.857910in}}%
\pgfpathclose%
\pgfusepath{fill}%
\end{pgfscope}%
\begin{pgfscope}%
\pgfpathrectangle{\pgfqpoint{1.150000in}{0.150000in}}{\pgfqpoint{5.700000in}{5.700000in}}%
\pgfusepath{clip}%
\pgfsetbuttcap%
\pgfsetroundjoin%
\definecolor{currentfill}{rgb}{0.201239,0.383670,0.554294}%
\pgfsetfillcolor{currentfill}%
\pgfsetfillopacity{0.700000}%
\pgfsetlinewidth{0.000000pt}%
\definecolor{currentstroke}{rgb}{0.000000,0.000000,0.000000}%
\pgfsetstrokecolor{currentstroke}%
\pgfsetdash{}{0pt}%
\pgfpathmoveto{\pgfqpoint{5.392757in}{2.412092in}}%
\pgfpathlineto{\pgfqpoint{5.407258in}{2.416040in}}%
\pgfpathlineto{\pgfqpoint{5.421772in}{2.420058in}}%
\pgfpathlineto{\pgfqpoint{5.436298in}{2.424146in}}%
\pgfpathlineto{\pgfqpoint{5.450837in}{2.428304in}}%
\pgfpathlineto{\pgfqpoint{5.458321in}{2.433151in}}%
\pgfpathlineto{\pgfqpoint{5.465797in}{2.437927in}}%
\pgfpathlineto{\pgfqpoint{5.473264in}{2.442637in}}%
\pgfpathlineto{\pgfqpoint{5.480724in}{2.447284in}}%
\pgfpathlineto{\pgfqpoint{5.466204in}{2.443324in}}%
\pgfpathlineto{\pgfqpoint{5.451697in}{2.439434in}}%
\pgfpathlineto{\pgfqpoint{5.437202in}{2.435614in}}%
\pgfpathlineto{\pgfqpoint{5.422721in}{2.431864in}}%
\pgfpathlineto{\pgfqpoint{5.415242in}{2.427012in}}%
\pgfpathlineto{\pgfqpoint{5.407755in}{2.422101in}}%
\pgfpathlineto{\pgfqpoint{5.400260in}{2.417129in}}%
\pgfpathlineto{\pgfqpoint{5.392757in}{2.412092in}}%
\pgfpathclose%
\pgfusepath{fill}%
\end{pgfscope}%
\begin{pgfscope}%
\pgfpathrectangle{\pgfqpoint{1.150000in}{0.150000in}}{\pgfqpoint{5.700000in}{5.700000in}}%
\pgfusepath{clip}%
\pgfsetbuttcap%
\pgfsetroundjoin%
\definecolor{currentfill}{rgb}{0.280255,0.165693,0.476498}%
\pgfsetfillcolor{currentfill}%
\pgfsetfillopacity{0.700000}%
\pgfsetlinewidth{0.000000pt}%
\definecolor{currentstroke}{rgb}{0.000000,0.000000,0.000000}%
\pgfsetstrokecolor{currentstroke}%
\pgfsetdash{}{0pt}%
\pgfpathmoveto{\pgfqpoint{2.335932in}{1.950767in}}%
\pgfpathlineto{\pgfqpoint{2.349781in}{1.939306in}}%
\pgfpathlineto{\pgfqpoint{2.363628in}{1.927957in}}%
\pgfpathlineto{\pgfqpoint{2.377474in}{1.916720in}}%
\pgfpathlineto{\pgfqpoint{2.391319in}{1.905595in}}%
\pgfpathlineto{\pgfqpoint{2.400300in}{1.905806in}}%
\pgfpathlineto{\pgfqpoint{2.409264in}{1.906269in}}%
\pgfpathlineto{\pgfqpoint{2.418211in}{1.906980in}}%
\pgfpathlineto{\pgfqpoint{2.427140in}{1.907931in}}%
\pgfpathlineto{\pgfqpoint{2.413332in}{1.918677in}}%
\pgfpathlineto{\pgfqpoint{2.399524in}{1.929534in}}%
\pgfpathlineto{\pgfqpoint{2.385714in}{1.940502in}}%
\pgfpathlineto{\pgfqpoint{2.371903in}{1.951583in}}%
\pgfpathlineto{\pgfqpoint{2.362937in}{1.951003in}}%
\pgfpathlineto{\pgfqpoint{2.353954in}{1.950670in}}%
\pgfpathlineto{\pgfqpoint{2.344952in}{1.950589in}}%
\pgfpathlineto{\pgfqpoint{2.335932in}{1.950767in}}%
\pgfpathclose%
\pgfusepath{fill}%
\end{pgfscope}%
\begin{pgfscope}%
\pgfpathrectangle{\pgfqpoint{1.150000in}{0.150000in}}{\pgfqpoint{5.700000in}{5.700000in}}%
\pgfusepath{clip}%
\pgfsetbuttcap%
\pgfsetroundjoin%
\definecolor{currentfill}{rgb}{0.283187,0.125848,0.444960}%
\pgfsetfillcolor{currentfill}%
\pgfsetfillopacity{0.700000}%
\pgfsetlinewidth{0.000000pt}%
\definecolor{currentstroke}{rgb}{0.000000,0.000000,0.000000}%
\pgfsetstrokecolor{currentstroke}%
\pgfsetdash{}{0pt}%
\pgfpathmoveto{\pgfqpoint{4.100742in}{1.816210in}}%
\pgfpathlineto{\pgfqpoint{4.114747in}{1.816369in}}%
\pgfpathlineto{\pgfqpoint{4.128762in}{1.816602in}}%
\pgfpathlineto{\pgfqpoint{4.142785in}{1.816908in}}%
\pgfpathlineto{\pgfqpoint{4.156817in}{1.817287in}}%
\pgfpathlineto{\pgfqpoint{4.164877in}{1.827527in}}%
\pgfpathlineto{\pgfqpoint{4.172931in}{1.837711in}}%
\pgfpathlineto{\pgfqpoint{4.180979in}{1.847839in}}%
\pgfpathlineto{\pgfqpoint{4.189022in}{1.857910in}}%
\pgfpathlineto{\pgfqpoint{4.174998in}{1.857428in}}%
\pgfpathlineto{\pgfqpoint{4.160984in}{1.857020in}}%
\pgfpathlineto{\pgfqpoint{4.146978in}{1.856685in}}%
\pgfpathlineto{\pgfqpoint{4.132982in}{1.856424in}}%
\pgfpathlineto{\pgfqpoint{4.124930in}{1.846448in}}%
\pgfpathlineto{\pgfqpoint{4.116873in}{1.836419in}}%
\pgfpathlineto{\pgfqpoint{4.108810in}{1.826340in}}%
\pgfpathlineto{\pgfqpoint{4.100742in}{1.816210in}}%
\pgfpathclose%
\pgfusepath{fill}%
\end{pgfscope}%
\begin{pgfscope}%
\pgfpathrectangle{\pgfqpoint{1.150000in}{0.150000in}}{\pgfqpoint{5.700000in}{5.700000in}}%
\pgfusepath{clip}%
\pgfsetbuttcap%
\pgfsetroundjoin%
\definecolor{currentfill}{rgb}{0.267004,0.004874,0.329415}%
\pgfsetfillcolor{currentfill}%
\pgfsetfillopacity{0.700000}%
\pgfsetlinewidth{0.000000pt}%
\definecolor{currentstroke}{rgb}{0.000000,0.000000,0.000000}%
\pgfsetstrokecolor{currentstroke}%
\pgfsetdash{}{0pt}%
\pgfpathmoveto{\pgfqpoint{3.137788in}{1.606414in}}%
\pgfpathlineto{\pgfqpoint{3.151596in}{1.600975in}}%
\pgfpathlineto{\pgfqpoint{3.165409in}{1.595620in}}%
\pgfpathlineto{\pgfqpoint{3.179226in}{1.590349in}}%
\pgfpathlineto{\pgfqpoint{3.193047in}{1.585161in}}%
\pgfpathlineto{\pgfqpoint{3.201489in}{1.592322in}}%
\pgfpathlineto{\pgfqpoint{3.209922in}{1.599589in}}%
\pgfpathlineto{\pgfqpoint{3.218346in}{1.606958in}}%
\pgfpathlineto{\pgfqpoint{3.226762in}{1.614426in}}%
\pgfpathlineto{\pgfqpoint{3.212960in}{1.619326in}}%
\pgfpathlineto{\pgfqpoint{3.199163in}{1.624309in}}%
\pgfpathlineto{\pgfqpoint{3.185370in}{1.629377in}}%
\pgfpathlineto{\pgfqpoint{3.171582in}{1.634529in}}%
\pgfpathlineto{\pgfqpoint{3.163146in}{1.627341in}}%
\pgfpathlineto{\pgfqpoint{3.154703in}{1.620256in}}%
\pgfpathlineto{\pgfqpoint{3.146250in}{1.613279in}}%
\pgfpathlineto{\pgfqpoint{3.137788in}{1.606414in}}%
\pgfpathclose%
\pgfusepath{fill}%
\end{pgfscope}%
\begin{pgfscope}%
\pgfpathrectangle{\pgfqpoint{1.150000in}{0.150000in}}{\pgfqpoint{5.700000in}{5.700000in}}%
\pgfusepath{clip}%
\pgfsetbuttcap%
\pgfsetroundjoin%
\definecolor{currentfill}{rgb}{0.282910,0.105393,0.426902}%
\pgfsetfillcolor{currentfill}%
\pgfsetfillopacity{0.700000}%
\pgfsetlinewidth{0.000000pt}%
\definecolor{currentstroke}{rgb}{0.000000,0.000000,0.000000}%
\pgfsetstrokecolor{currentstroke}%
\pgfsetdash{}{0pt}%
\pgfpathmoveto{\pgfqpoint{4.012441in}{1.775818in}}%
\pgfpathlineto{\pgfqpoint{4.026421in}{1.775557in}}%
\pgfpathlineto{\pgfqpoint{4.040410in}{1.775371in}}%
\pgfpathlineto{\pgfqpoint{4.054407in}{1.775259in}}%
\pgfpathlineto{\pgfqpoint{4.068413in}{1.775221in}}%
\pgfpathlineto{\pgfqpoint{4.076503in}{1.785536in}}%
\pgfpathlineto{\pgfqpoint{4.084588in}{1.795807in}}%
\pgfpathlineto{\pgfqpoint{4.092668in}{1.806032in}}%
\pgfpathlineto{\pgfqpoint{4.100742in}{1.816210in}}%
\pgfpathlineto{\pgfqpoint{4.086745in}{1.816125in}}%
\pgfpathlineto{\pgfqpoint{4.072757in}{1.816115in}}%
\pgfpathlineto{\pgfqpoint{4.058778in}{1.816178in}}%
\pgfpathlineto{\pgfqpoint{4.044807in}{1.816315in}}%
\pgfpathlineto{\pgfqpoint{4.036724in}{1.806252in}}%
\pgfpathlineto{\pgfqpoint{4.028635in}{1.796148in}}%
\pgfpathlineto{\pgfqpoint{4.020541in}{1.786002in}}%
\pgfpathlineto{\pgfqpoint{4.012441in}{1.775818in}}%
\pgfpathclose%
\pgfusepath{fill}%
\end{pgfscope}%
\begin{pgfscope}%
\pgfpathrectangle{\pgfqpoint{1.150000in}{0.150000in}}{\pgfqpoint{5.700000in}{5.700000in}}%
\pgfusepath{clip}%
\pgfsetbuttcap%
\pgfsetroundjoin%
\definecolor{currentfill}{rgb}{0.269944,0.014625,0.341379}%
\pgfsetfillcolor{currentfill}%
\pgfsetfillopacity{0.700000}%
\pgfsetlinewidth{0.000000pt}%
\definecolor{currentstroke}{rgb}{0.000000,0.000000,0.000000}%
\pgfsetstrokecolor{currentstroke}%
\pgfsetdash{}{0pt}%
\pgfpathmoveto{\pgfqpoint{2.993345in}{1.629226in}}%
\pgfpathlineto{\pgfqpoint{3.007146in}{1.622789in}}%
\pgfpathlineto{\pgfqpoint{3.020950in}{1.616439in}}%
\pgfpathlineto{\pgfqpoint{3.034757in}{1.610177in}}%
\pgfpathlineto{\pgfqpoint{3.048568in}{1.604002in}}%
\pgfpathlineto{\pgfqpoint{3.057089in}{1.610059in}}%
\pgfpathlineto{\pgfqpoint{3.065601in}{1.616252in}}%
\pgfpathlineto{\pgfqpoint{3.074103in}{1.622574in}}%
\pgfpathlineto{\pgfqpoint{3.082595in}{1.629023in}}%
\pgfpathlineto{\pgfqpoint{3.068806in}{1.634891in}}%
\pgfpathlineto{\pgfqpoint{3.055021in}{1.640844in}}%
\pgfpathlineto{\pgfqpoint{3.041240in}{1.646886in}}%
\pgfpathlineto{\pgfqpoint{3.027462in}{1.653014in}}%
\pgfpathlineto{\pgfqpoint{3.018948in}{1.646866in}}%
\pgfpathlineto{\pgfqpoint{3.010424in}{1.640849in}}%
\pgfpathlineto{\pgfqpoint{3.001890in}{1.634967in}}%
\pgfpathlineto{\pgfqpoint{2.993345in}{1.629226in}}%
\pgfpathclose%
\pgfusepath{fill}%
\end{pgfscope}%
\begin{pgfscope}%
\pgfpathrectangle{\pgfqpoint{1.150000in}{0.150000in}}{\pgfqpoint{5.700000in}{5.700000in}}%
\pgfusepath{clip}%
\pgfsetbuttcap%
\pgfsetroundjoin%
\definecolor{currentfill}{rgb}{0.206756,0.371758,0.553117}%
\pgfsetfillcolor{currentfill}%
\pgfsetfillopacity{0.700000}%
\pgfsetlinewidth{0.000000pt}%
\definecolor{currentstroke}{rgb}{0.000000,0.000000,0.000000}%
\pgfsetstrokecolor{currentstroke}%
\pgfsetdash{}{0pt}%
\pgfpathmoveto{\pgfqpoint{5.304717in}{2.375422in}}%
\pgfpathlineto{\pgfqpoint{5.319185in}{2.379267in}}%
\pgfpathlineto{\pgfqpoint{5.333665in}{2.383181in}}%
\pgfpathlineto{\pgfqpoint{5.348158in}{2.387166in}}%
\pgfpathlineto{\pgfqpoint{5.362664in}{2.391221in}}%
\pgfpathlineto{\pgfqpoint{5.370200in}{2.396553in}}%
\pgfpathlineto{\pgfqpoint{5.377727in}{2.401807in}}%
\pgfpathlineto{\pgfqpoint{5.385246in}{2.406986in}}%
\pgfpathlineto{\pgfqpoint{5.392757in}{2.412092in}}%
\pgfpathlineto{\pgfqpoint{5.378269in}{2.408214in}}%
\pgfpathlineto{\pgfqpoint{5.363794in}{2.404405in}}%
\pgfpathlineto{\pgfqpoint{5.349331in}{2.400667in}}%
\pgfpathlineto{\pgfqpoint{5.334881in}{2.396999in}}%
\pgfpathlineto{\pgfqpoint{5.327352in}{2.391708in}}%
\pgfpathlineto{\pgfqpoint{5.319815in}{2.386351in}}%
\pgfpathlineto{\pgfqpoint{5.312270in}{2.380923in}}%
\pgfpathlineto{\pgfqpoint{5.304717in}{2.375422in}}%
\pgfpathclose%
\pgfusepath{fill}%
\end{pgfscope}%
\begin{pgfscope}%
\pgfpathrectangle{\pgfqpoint{1.150000in}{0.150000in}}{\pgfqpoint{5.700000in}{5.700000in}}%
\pgfusepath{clip}%
\pgfsetbuttcap%
\pgfsetroundjoin%
\definecolor{currentfill}{rgb}{0.269944,0.014625,0.341379}%
\pgfsetfillcolor{currentfill}%
\pgfsetfillopacity{0.700000}%
\pgfsetlinewidth{0.000000pt}%
\definecolor{currentstroke}{rgb}{0.000000,0.000000,0.000000}%
\pgfsetstrokecolor{currentstroke}%
\pgfsetdash{}{0pt}%
\pgfpathmoveto{\pgfqpoint{3.514785in}{1.618142in}}%
\pgfpathlineto{\pgfqpoint{3.528646in}{1.615129in}}%
\pgfpathlineto{\pgfqpoint{3.542512in}{1.612195in}}%
\pgfpathlineto{\pgfqpoint{3.556385in}{1.609339in}}%
\pgfpathlineto{\pgfqpoint{3.570265in}{1.606562in}}%
\pgfpathlineto{\pgfqpoint{3.578535in}{1.615904in}}%
\pgfpathlineto{\pgfqpoint{3.586799in}{1.625281in}}%
\pgfpathlineto{\pgfqpoint{3.595056in}{1.634690in}}%
\pgfpathlineto{\pgfqpoint{3.603308in}{1.644129in}}%
\pgfpathlineto{\pgfqpoint{3.589442in}{1.646681in}}%
\pgfpathlineto{\pgfqpoint{3.575583in}{1.649312in}}%
\pgfpathlineto{\pgfqpoint{3.561729in}{1.652020in}}%
\pgfpathlineto{\pgfqpoint{3.547883in}{1.654808in}}%
\pgfpathlineto{\pgfqpoint{3.539618in}{1.645587in}}%
\pgfpathlineto{\pgfqpoint{3.531347in}{1.636400in}}%
\pgfpathlineto{\pgfqpoint{3.523069in}{1.627251in}}%
\pgfpathlineto{\pgfqpoint{3.514785in}{1.618142in}}%
\pgfpathclose%
\pgfusepath{fill}%
\end{pgfscope}%
\begin{pgfscope}%
\pgfpathrectangle{\pgfqpoint{1.150000in}{0.150000in}}{\pgfqpoint{5.700000in}{5.700000in}}%
\pgfusepath{clip}%
\pgfsetbuttcap%
\pgfsetroundjoin%
\definecolor{currentfill}{rgb}{0.281446,0.084320,0.407414}%
\pgfsetfillcolor{currentfill}%
\pgfsetfillopacity{0.700000}%
\pgfsetlinewidth{0.000000pt}%
\definecolor{currentstroke}{rgb}{0.000000,0.000000,0.000000}%
\pgfsetstrokecolor{currentstroke}%
\pgfsetdash{}{0pt}%
\pgfpathmoveto{\pgfqpoint{3.924112in}{1.737084in}}%
\pgfpathlineto{\pgfqpoint{3.938069in}{1.736382in}}%
\pgfpathlineto{\pgfqpoint{3.952033in}{1.735755in}}%
\pgfpathlineto{\pgfqpoint{3.966006in}{1.735202in}}%
\pgfpathlineto{\pgfqpoint{3.979987in}{1.734723in}}%
\pgfpathlineto{\pgfqpoint{3.988109in}{1.745047in}}%
\pgfpathlineto{\pgfqpoint{3.996225in}{1.755338in}}%
\pgfpathlineto{\pgfqpoint{4.004336in}{1.765596in}}%
\pgfpathlineto{\pgfqpoint{4.012441in}{1.775818in}}%
\pgfpathlineto{\pgfqpoint{3.998469in}{1.776152in}}%
\pgfpathlineto{\pgfqpoint{3.984506in}{1.776562in}}%
\pgfpathlineto{\pgfqpoint{3.970551in}{1.777046in}}%
\pgfpathlineto{\pgfqpoint{3.956604in}{1.777604in}}%
\pgfpathlineto{\pgfqpoint{3.948489in}{1.767518in}}%
\pgfpathlineto{\pgfqpoint{3.940369in}{1.757402in}}%
\pgfpathlineto{\pgfqpoint{3.932243in}{1.747257in}}%
\pgfpathlineto{\pgfqpoint{3.924112in}{1.737084in}}%
\pgfpathclose%
\pgfusepath{fill}%
\end{pgfscope}%
\begin{pgfscope}%
\pgfpathrectangle{\pgfqpoint{1.150000in}{0.150000in}}{\pgfqpoint{5.700000in}{5.700000in}}%
\pgfusepath{clip}%
\pgfsetbuttcap%
\pgfsetroundjoin%
\definecolor{currentfill}{rgb}{0.267004,0.004874,0.329415}%
\pgfsetfillcolor{currentfill}%
\pgfsetfillopacity{0.700000}%
\pgfsetlinewidth{0.000000pt}%
\definecolor{currentstroke}{rgb}{0.000000,0.000000,0.000000}%
\pgfsetstrokecolor{currentstroke}%
\pgfsetdash{}{0pt}%
\pgfpathmoveto{\pgfqpoint{3.282016in}{1.595656in}}%
\pgfpathlineto{\pgfqpoint{3.295842in}{1.591169in}}%
\pgfpathlineto{\pgfqpoint{3.309672in}{1.586764in}}%
\pgfpathlineto{\pgfqpoint{3.323508in}{1.582441in}}%
\pgfpathlineto{\pgfqpoint{3.337349in}{1.578198in}}%
\pgfpathlineto{\pgfqpoint{3.345720in}{1.586308in}}%
\pgfpathlineto{\pgfqpoint{3.354085in}{1.594498in}}%
\pgfpathlineto{\pgfqpoint{3.362441in}{1.602763in}}%
\pgfpathlineto{\pgfqpoint{3.370791in}{1.611101in}}%
\pgfpathlineto{\pgfqpoint{3.356967in}{1.615077in}}%
\pgfpathlineto{\pgfqpoint{3.343148in}{1.619134in}}%
\pgfpathlineto{\pgfqpoint{3.329335in}{1.623272in}}%
\pgfpathlineto{\pgfqpoint{3.315527in}{1.627492in}}%
\pgfpathlineto{\pgfqpoint{3.307161in}{1.619414in}}%
\pgfpathlineto{\pgfqpoint{3.298787in}{1.611412in}}%
\pgfpathlineto{\pgfqpoint{3.290405in}{1.603491in}}%
\pgfpathlineto{\pgfqpoint{3.282016in}{1.595656in}}%
\pgfpathclose%
\pgfusepath{fill}%
\end{pgfscope}%
\begin{pgfscope}%
\pgfpathrectangle{\pgfqpoint{1.150000in}{0.150000in}}{\pgfqpoint{5.700000in}{5.700000in}}%
\pgfusepath{clip}%
\pgfsetbuttcap%
\pgfsetroundjoin%
\definecolor{currentfill}{rgb}{0.214298,0.355619,0.551184}%
\pgfsetfillcolor{currentfill}%
\pgfsetfillopacity{0.700000}%
\pgfsetlinewidth{0.000000pt}%
\definecolor{currentstroke}{rgb}{0.000000,0.000000,0.000000}%
\pgfsetstrokecolor{currentstroke}%
\pgfsetdash{}{0pt}%
\pgfpathmoveto{\pgfqpoint{5.216610in}{2.337325in}}%
\pgfpathlineto{\pgfqpoint{5.231044in}{2.341043in}}%
\pgfpathlineto{\pgfqpoint{5.245491in}{2.344831in}}%
\pgfpathlineto{\pgfqpoint{5.259951in}{2.348690in}}%
\pgfpathlineto{\pgfqpoint{5.274422in}{2.352619in}}%
\pgfpathlineto{\pgfqpoint{5.282008in}{2.358446in}}%
\pgfpathlineto{\pgfqpoint{5.289586in}{2.364186in}}%
\pgfpathlineto{\pgfqpoint{5.297155in}{2.369844in}}%
\pgfpathlineto{\pgfqpoint{5.304717in}{2.375422in}}%
\pgfpathlineto{\pgfqpoint{5.290261in}{2.371648in}}%
\pgfpathlineto{\pgfqpoint{5.275819in}{2.367944in}}%
\pgfpathlineto{\pgfqpoint{5.261388in}{2.364310in}}%
\pgfpathlineto{\pgfqpoint{5.246970in}{2.360746in}}%
\pgfpathlineto{\pgfqpoint{5.239392in}{2.355005in}}%
\pgfpathlineto{\pgfqpoint{5.231806in}{2.349190in}}%
\pgfpathlineto{\pgfqpoint{5.224212in}{2.343298in}}%
\pgfpathlineto{\pgfqpoint{5.216610in}{2.337325in}}%
\pgfpathclose%
\pgfusepath{fill}%
\end{pgfscope}%
\begin{pgfscope}%
\pgfpathrectangle{\pgfqpoint{1.150000in}{0.150000in}}{\pgfqpoint{5.700000in}{5.700000in}}%
\pgfusepath{clip}%
\pgfsetbuttcap%
\pgfsetroundjoin%
\definecolor{currentfill}{rgb}{0.281887,0.150881,0.465405}%
\pgfsetfillcolor{currentfill}%
\pgfsetfillopacity{0.700000}%
\pgfsetlinewidth{0.000000pt}%
\definecolor{currentstroke}{rgb}{0.000000,0.000000,0.000000}%
\pgfsetstrokecolor{currentstroke}%
\pgfsetdash{}{0pt}%
\pgfpathmoveto{\pgfqpoint{2.391319in}{1.905595in}}%
\pgfpathlineto{\pgfqpoint{2.405162in}{1.894580in}}%
\pgfpathlineto{\pgfqpoint{2.419006in}{1.883674in}}%
\pgfpathlineto{\pgfqpoint{2.432848in}{1.872878in}}%
\pgfpathlineto{\pgfqpoint{2.446690in}{1.862189in}}%
\pgfpathlineto{\pgfqpoint{2.455634in}{1.862787in}}%
\pgfpathlineto{\pgfqpoint{2.464562in}{1.863632in}}%
\pgfpathlineto{\pgfqpoint{2.473472in}{1.864718in}}%
\pgfpathlineto{\pgfqpoint{2.482367in}{1.866040in}}%
\pgfpathlineto{\pgfqpoint{2.468561in}{1.876350in}}%
\pgfpathlineto{\pgfqpoint{2.454754in}{1.886768in}}%
\pgfpathlineto{\pgfqpoint{2.440948in}{1.897295in}}%
\pgfpathlineto{\pgfqpoint{2.427140in}{1.907931in}}%
\pgfpathlineto{\pgfqpoint{2.418211in}{1.906980in}}%
\pgfpathlineto{\pgfqpoint{2.409264in}{1.906269in}}%
\pgfpathlineto{\pgfqpoint{2.400300in}{1.905806in}}%
\pgfpathlineto{\pgfqpoint{2.391319in}{1.905595in}}%
\pgfpathclose%
\pgfusepath{fill}%
\end{pgfscope}%
\begin{pgfscope}%
\pgfpathrectangle{\pgfqpoint{1.150000in}{0.150000in}}{\pgfqpoint{5.700000in}{5.700000in}}%
\pgfusepath{clip}%
\pgfsetbuttcap%
\pgfsetroundjoin%
\definecolor{currentfill}{rgb}{0.280894,0.078907,0.402329}%
\pgfsetfillcolor{currentfill}%
\pgfsetfillopacity{0.700000}%
\pgfsetlinewidth{0.000000pt}%
\definecolor{currentstroke}{rgb}{0.000000,0.000000,0.000000}%
\pgfsetstrokecolor{currentstroke}%
\pgfsetdash{}{0pt}%
\pgfpathmoveto{\pgfqpoint{2.648052in}{1.750448in}}%
\pgfpathlineto{\pgfqpoint{2.661863in}{1.741473in}}%
\pgfpathlineto{\pgfqpoint{2.675675in}{1.732595in}}%
\pgfpathlineto{\pgfqpoint{2.689489in}{1.723815in}}%
\pgfpathlineto{\pgfqpoint{2.703304in}{1.715132in}}%
\pgfpathlineto{\pgfqpoint{2.712052in}{1.718134in}}%
\pgfpathlineto{\pgfqpoint{2.720787in}{1.721338in}}%
\pgfpathlineto{\pgfqpoint{2.729508in}{1.724738in}}%
\pgfpathlineto{\pgfqpoint{2.738216in}{1.728330in}}%
\pgfpathlineto{\pgfqpoint{2.724431in}{1.736660in}}%
\pgfpathlineto{\pgfqpoint{2.710647in}{1.745087in}}%
\pgfpathlineto{\pgfqpoint{2.696865in}{1.753612in}}%
\pgfpathlineto{\pgfqpoint{2.683085in}{1.762234in}}%
\pgfpathlineto{\pgfqpoint{2.674347in}{1.758987in}}%
\pgfpathlineto{\pgfqpoint{2.665596in}{1.755936in}}%
\pgfpathlineto{\pgfqpoint{2.656831in}{1.753088in}}%
\pgfpathlineto{\pgfqpoint{2.648052in}{1.750448in}}%
\pgfpathclose%
\pgfusepath{fill}%
\end{pgfscope}%
\begin{pgfscope}%
\pgfpathrectangle{\pgfqpoint{1.150000in}{0.150000in}}{\pgfqpoint{5.700000in}{5.700000in}}%
\pgfusepath{clip}%
\pgfsetbuttcap%
\pgfsetroundjoin%
\definecolor{currentfill}{rgb}{0.273809,0.031497,0.358853}%
\pgfsetfillcolor{currentfill}%
\pgfsetfillopacity{0.700000}%
\pgfsetlinewidth{0.000000pt}%
\definecolor{currentstroke}{rgb}{0.000000,0.000000,0.000000}%
\pgfsetstrokecolor{currentstroke}%
\pgfsetdash{}{0pt}%
\pgfpathmoveto{\pgfqpoint{2.848563in}{1.665104in}}%
\pgfpathlineto{\pgfqpoint{2.862366in}{1.657620in}}%
\pgfpathlineto{\pgfqpoint{2.876171in}{1.650227in}}%
\pgfpathlineto{\pgfqpoint{2.889979in}{1.642925in}}%
\pgfpathlineto{\pgfqpoint{2.903790in}{1.635713in}}%
\pgfpathlineto{\pgfqpoint{2.912403in}{1.640507in}}%
\pgfpathlineto{\pgfqpoint{2.921005in}{1.645466in}}%
\pgfpathlineto{\pgfqpoint{2.929595in}{1.650585in}}%
\pgfpathlineto{\pgfqpoint{2.938174in}{1.655859in}}%
\pgfpathlineto{\pgfqpoint{2.924389in}{1.662741in}}%
\pgfpathlineto{\pgfqpoint{2.910607in}{1.669713in}}%
\pgfpathlineto{\pgfqpoint{2.896827in}{1.676776in}}%
\pgfpathlineto{\pgfqpoint{2.883050in}{1.683931in}}%
\pgfpathlineto{\pgfqpoint{2.874446in}{1.678978in}}%
\pgfpathlineto{\pgfqpoint{2.865830in}{1.674186in}}%
\pgfpathlineto{\pgfqpoint{2.857202in}{1.669560in}}%
\pgfpathlineto{\pgfqpoint{2.848563in}{1.665104in}}%
\pgfpathclose%
\pgfusepath{fill}%
\end{pgfscope}%
\begin{pgfscope}%
\pgfpathrectangle{\pgfqpoint{1.150000in}{0.150000in}}{\pgfqpoint{5.700000in}{5.700000in}}%
\pgfusepath{clip}%
\pgfsetbuttcap%
\pgfsetroundjoin%
\definecolor{currentfill}{rgb}{0.279566,0.067836,0.391917}%
\pgfsetfillcolor{currentfill}%
\pgfsetfillopacity{0.700000}%
\pgfsetlinewidth{0.000000pt}%
\definecolor{currentstroke}{rgb}{0.000000,0.000000,0.000000}%
\pgfsetstrokecolor{currentstroke}%
\pgfsetdash{}{0pt}%
\pgfpathmoveto{\pgfqpoint{3.835744in}{1.700385in}}%
\pgfpathlineto{\pgfqpoint{3.849680in}{1.699218in}}%
\pgfpathlineto{\pgfqpoint{3.863622in}{1.698127in}}%
\pgfpathlineto{\pgfqpoint{3.877573in}{1.697110in}}%
\pgfpathlineto{\pgfqpoint{3.891532in}{1.696169in}}%
\pgfpathlineto{\pgfqpoint{3.899685in}{1.706427in}}%
\pgfpathlineto{\pgfqpoint{3.907833in}{1.716668in}}%
\pgfpathlineto{\pgfqpoint{3.915975in}{1.726888in}}%
\pgfpathlineto{\pgfqpoint{3.924112in}{1.737084in}}%
\pgfpathlineto{\pgfqpoint{3.910164in}{1.737862in}}%
\pgfpathlineto{\pgfqpoint{3.896223in}{1.738714in}}%
\pgfpathlineto{\pgfqpoint{3.882290in}{1.739642in}}%
\pgfpathlineto{\pgfqpoint{3.868365in}{1.740644in}}%
\pgfpathlineto{\pgfqpoint{3.860219in}{1.730604in}}%
\pgfpathlineto{\pgfqpoint{3.852066in}{1.720546in}}%
\pgfpathlineto{\pgfqpoint{3.843908in}{1.710472in}}%
\pgfpathlineto{\pgfqpoint{3.835744in}{1.700385in}}%
\pgfpathclose%
\pgfusepath{fill}%
\end{pgfscope}%
\begin{pgfscope}%
\pgfpathrectangle{\pgfqpoint{1.150000in}{0.150000in}}{\pgfqpoint{5.700000in}{5.700000in}}%
\pgfusepath{clip}%
\pgfsetbuttcap%
\pgfsetroundjoin%
\definecolor{currentfill}{rgb}{0.220057,0.343307,0.549413}%
\pgfsetfillcolor{currentfill}%
\pgfsetfillopacity{0.700000}%
\pgfsetlinewidth{0.000000pt}%
\definecolor{currentstroke}{rgb}{0.000000,0.000000,0.000000}%
\pgfsetstrokecolor{currentstroke}%
\pgfsetdash{}{0pt}%
\pgfpathmoveto{\pgfqpoint{5.128445in}{2.297872in}}%
\pgfpathlineto{\pgfqpoint{5.142846in}{2.301441in}}%
\pgfpathlineto{\pgfqpoint{5.157259in}{2.305081in}}%
\pgfpathlineto{\pgfqpoint{5.171684in}{2.308792in}}%
\pgfpathlineto{\pgfqpoint{5.186121in}{2.312573in}}%
\pgfpathlineto{\pgfqpoint{5.193755in}{2.318895in}}%
\pgfpathlineto{\pgfqpoint{5.201382in}{2.325126in}}%
\pgfpathlineto{\pgfqpoint{5.209000in}{2.331268in}}%
\pgfpathlineto{\pgfqpoint{5.216610in}{2.337325in}}%
\pgfpathlineto{\pgfqpoint{5.202188in}{2.333677in}}%
\pgfpathlineto{\pgfqpoint{5.187778in}{2.330099in}}%
\pgfpathlineto{\pgfqpoint{5.173380in}{2.326592in}}%
\pgfpathlineto{\pgfqpoint{5.158995in}{2.323155in}}%
\pgfpathlineto{\pgfqpoint{5.151369in}{2.316958in}}%
\pgfpathlineto{\pgfqpoint{5.143736in}{2.310680in}}%
\pgfpathlineto{\pgfqpoint{5.136094in}{2.304319in}}%
\pgfpathlineto{\pgfqpoint{5.128445in}{2.297872in}}%
\pgfpathclose%
\pgfusepath{fill}%
\end{pgfscope}%
\begin{pgfscope}%
\pgfpathrectangle{\pgfqpoint{1.150000in}{0.150000in}}{\pgfqpoint{5.700000in}{5.700000in}}%
\pgfusepath{clip}%
\pgfsetbuttcap%
\pgfsetroundjoin%
\definecolor{currentfill}{rgb}{0.223925,0.334994,0.548053}%
\pgfsetfillcolor{currentfill}%
\pgfsetfillopacity{0.700000}%
\pgfsetlinewidth{0.000000pt}%
\definecolor{currentstroke}{rgb}{0.000000,0.000000,0.000000}%
\pgfsetstrokecolor{currentstroke}%
\pgfsetdash{}{0pt}%
\pgfpathmoveto{\pgfqpoint{1.909631in}{2.336308in}}%
\pgfpathlineto{\pgfqpoint{1.923617in}{2.320862in}}%
\pgfpathlineto{\pgfqpoint{1.937598in}{2.305561in}}%
\pgfpathlineto{\pgfqpoint{1.951573in}{2.290402in}}%
\pgfpathlineto{\pgfqpoint{1.965542in}{2.275384in}}%
\pgfpathlineto{\pgfqpoint{1.974907in}{2.271607in}}%
\pgfpathlineto{\pgfqpoint{1.984249in}{2.268150in}}%
\pgfpathlineto{\pgfqpoint{1.993566in}{2.265004in}}%
\pgfpathlineto{\pgfqpoint{2.002861in}{2.262164in}}%
\pgfpathlineto{\pgfqpoint{1.988939in}{2.276768in}}%
\pgfpathlineto{\pgfqpoint{1.975012in}{2.291512in}}%
\pgfpathlineto{\pgfqpoint{1.961079in}{2.306398in}}%
\pgfpathlineto{\pgfqpoint{1.947142in}{2.321427in}}%
\pgfpathlineto{\pgfqpoint{1.937800in}{2.324673in}}%
\pgfpathlineto{\pgfqpoint{1.928435in}{2.328231in}}%
\pgfpathlineto{\pgfqpoint{1.919045in}{2.332107in}}%
\pgfpathlineto{\pgfqpoint{1.909631in}{2.336308in}}%
\pgfpathclose%
\pgfusepath{fill}%
\end{pgfscope}%
\begin{pgfscope}%
\pgfpathrectangle{\pgfqpoint{1.150000in}{0.150000in}}{\pgfqpoint{5.700000in}{5.700000in}}%
\pgfusepath{clip}%
\pgfsetbuttcap%
\pgfsetroundjoin%
\definecolor{currentfill}{rgb}{0.233603,0.313828,0.543914}%
\pgfsetfillcolor{currentfill}%
\pgfsetfillopacity{0.700000}%
\pgfsetlinewidth{0.000000pt}%
\definecolor{currentstroke}{rgb}{0.000000,0.000000,0.000000}%
\pgfsetstrokecolor{currentstroke}%
\pgfsetdash{}{0pt}%
\pgfpathmoveto{\pgfqpoint{1.965542in}{2.275384in}}%
\pgfpathlineto{\pgfqpoint{1.979507in}{2.260505in}}%
\pgfpathlineto{\pgfqpoint{1.993467in}{2.245766in}}%
\pgfpathlineto{\pgfqpoint{2.007423in}{2.231163in}}%
\pgfpathlineto{\pgfqpoint{2.021374in}{2.216697in}}%
\pgfpathlineto{\pgfqpoint{2.030691in}{2.213343in}}%
\pgfpathlineto{\pgfqpoint{2.039985in}{2.210301in}}%
\pgfpathlineto{\pgfqpoint{2.049256in}{2.207566in}}%
\pgfpathlineto{\pgfqpoint{2.058505in}{2.205130in}}%
\pgfpathlineto{\pgfqpoint{2.044601in}{2.219184in}}%
\pgfpathlineto{\pgfqpoint{2.030692in}{2.233374in}}%
\pgfpathlineto{\pgfqpoint{2.016779in}{2.247700in}}%
\pgfpathlineto{\pgfqpoint{2.002861in}{2.262164in}}%
\pgfpathlineto{\pgfqpoint{1.993566in}{2.265004in}}%
\pgfpathlineto{\pgfqpoint{1.984249in}{2.268150in}}%
\pgfpathlineto{\pgfqpoint{1.974907in}{2.271607in}}%
\pgfpathlineto{\pgfqpoint{1.965542in}{2.275384in}}%
\pgfpathclose%
\pgfusepath{fill}%
\end{pgfscope}%
\begin{pgfscope}%
\pgfpathrectangle{\pgfqpoint{1.150000in}{0.150000in}}{\pgfqpoint{5.700000in}{5.700000in}}%
\pgfusepath{clip}%
\pgfsetbuttcap%
\pgfsetroundjoin%
\definecolor{currentfill}{rgb}{0.227802,0.326594,0.546532}%
\pgfsetfillcolor{currentfill}%
\pgfsetfillopacity{0.700000}%
\pgfsetlinewidth{0.000000pt}%
\definecolor{currentstroke}{rgb}{0.000000,0.000000,0.000000}%
\pgfsetstrokecolor{currentstroke}%
\pgfsetdash{}{0pt}%
\pgfpathmoveto{\pgfqpoint{5.040231in}{2.257159in}}%
\pgfpathlineto{\pgfqpoint{5.054598in}{2.260557in}}%
\pgfpathlineto{\pgfqpoint{5.068976in}{2.264026in}}%
\pgfpathlineto{\pgfqpoint{5.083366in}{2.267566in}}%
\pgfpathlineto{\pgfqpoint{5.097769in}{2.271176in}}%
\pgfpathlineto{\pgfqpoint{5.105450in}{2.277991in}}%
\pgfpathlineto{\pgfqpoint{5.113123in}{2.284710in}}%
\pgfpathlineto{\pgfqpoint{5.120788in}{2.291336in}}%
\pgfpathlineto{\pgfqpoint{5.128445in}{2.297872in}}%
\pgfpathlineto{\pgfqpoint{5.114057in}{2.294373in}}%
\pgfpathlineto{\pgfqpoint{5.099680in}{2.290945in}}%
\pgfpathlineto{\pgfqpoint{5.085316in}{2.287587in}}%
\pgfpathlineto{\pgfqpoint{5.070964in}{2.284299in}}%
\pgfpathlineto{\pgfqpoint{5.063292in}{2.277645in}}%
\pgfpathlineto{\pgfqpoint{5.055613in}{2.270905in}}%
\pgfpathlineto{\pgfqpoint{5.047926in}{2.264077in}}%
\pgfpathlineto{\pgfqpoint{5.040231in}{2.257159in}}%
\pgfpathclose%
\pgfusepath{fill}%
\end{pgfscope}%
\begin{pgfscope}%
\pgfpathrectangle{\pgfqpoint{1.150000in}{0.150000in}}{\pgfqpoint{5.700000in}{5.700000in}}%
\pgfusepath{clip}%
\pgfsetbuttcap%
\pgfsetroundjoin%
\definecolor{currentfill}{rgb}{0.277018,0.050344,0.375715}%
\pgfsetfillcolor{currentfill}%
\pgfsetfillopacity{0.700000}%
\pgfsetlinewidth{0.000000pt}%
\definecolor{currentstroke}{rgb}{0.000000,0.000000,0.000000}%
\pgfsetstrokecolor{currentstroke}%
\pgfsetdash{}{0pt}%
\pgfpathmoveto{\pgfqpoint{3.747325in}{1.666116in}}%
\pgfpathlineto{\pgfqpoint{3.761241in}{1.664462in}}%
\pgfpathlineto{\pgfqpoint{3.775164in}{1.662883in}}%
\pgfpathlineto{\pgfqpoint{3.789095in}{1.661380in}}%
\pgfpathlineto{\pgfqpoint{3.803034in}{1.659953in}}%
\pgfpathlineto{\pgfqpoint{3.811220in}{1.670069in}}%
\pgfpathlineto{\pgfqpoint{3.819400in}{1.680181in}}%
\pgfpathlineto{\pgfqpoint{3.827575in}{1.690288in}}%
\pgfpathlineto{\pgfqpoint{3.835744in}{1.700385in}}%
\pgfpathlineto{\pgfqpoint{3.821817in}{1.701628in}}%
\pgfpathlineto{\pgfqpoint{3.807897in}{1.702946in}}%
\pgfpathlineto{\pgfqpoint{3.793985in}{1.704340in}}%
\pgfpathlineto{\pgfqpoint{3.780080in}{1.705810in}}%
\pgfpathlineto{\pgfqpoint{3.771900in}{1.695890in}}%
\pgfpathlineto{\pgfqpoint{3.763714in}{1.685965in}}%
\pgfpathlineto{\pgfqpoint{3.755522in}{1.676040in}}%
\pgfpathlineto{\pgfqpoint{3.747325in}{1.666116in}}%
\pgfpathclose%
\pgfusepath{fill}%
\end{pgfscope}%
\begin{pgfscope}%
\pgfpathrectangle{\pgfqpoint{1.150000in}{0.150000in}}{\pgfqpoint{5.700000in}{5.700000in}}%
\pgfusepath{clip}%
\pgfsetbuttcap%
\pgfsetroundjoin%
\definecolor{currentfill}{rgb}{0.268510,0.009605,0.335427}%
\pgfsetfillcolor{currentfill}%
\pgfsetfillopacity{0.700000}%
\pgfsetlinewidth{0.000000pt}%
\definecolor{currentstroke}{rgb}{0.000000,0.000000,0.000000}%
\pgfsetstrokecolor{currentstroke}%
\pgfsetdash{}{0pt}%
\pgfpathmoveto{\pgfqpoint{3.426141in}{1.596002in}}%
\pgfpathlineto{\pgfqpoint{3.439993in}{1.592428in}}%
\pgfpathlineto{\pgfqpoint{3.453850in}{1.588932in}}%
\pgfpathlineto{\pgfqpoint{3.467713in}{1.585516in}}%
\pgfpathlineto{\pgfqpoint{3.481582in}{1.582179in}}%
\pgfpathlineto{\pgfqpoint{3.489893in}{1.591092in}}%
\pgfpathlineto{\pgfqpoint{3.498197in}{1.600059in}}%
\pgfpathlineto{\pgfqpoint{3.506494in}{1.609077in}}%
\pgfpathlineto{\pgfqpoint{3.514785in}{1.618142in}}%
\pgfpathlineto{\pgfqpoint{3.500931in}{1.621233in}}%
\pgfpathlineto{\pgfqpoint{3.487082in}{1.624403in}}%
\pgfpathlineto{\pgfqpoint{3.473240in}{1.627653in}}%
\pgfpathlineto{\pgfqpoint{3.459404in}{1.630981in}}%
\pgfpathlineto{\pgfqpoint{3.451098in}{1.622155in}}%
\pgfpathlineto{\pgfqpoint{3.442786in}{1.613380in}}%
\pgfpathlineto{\pgfqpoint{3.434467in}{1.604661in}}%
\pgfpathlineto{\pgfqpoint{3.426141in}{1.596002in}}%
\pgfpathclose%
\pgfusepath{fill}%
\end{pgfscope}%
\begin{pgfscope}%
\pgfpathrectangle{\pgfqpoint{1.150000in}{0.150000in}}{\pgfqpoint{5.700000in}{5.700000in}}%
\pgfusepath{clip}%
\pgfsetbuttcap%
\pgfsetroundjoin%
\definecolor{currentfill}{rgb}{0.235526,0.309527,0.542944}%
\pgfsetfillcolor{currentfill}%
\pgfsetfillopacity{0.700000}%
\pgfsetlinewidth{0.000000pt}%
\definecolor{currentstroke}{rgb}{0.000000,0.000000,0.000000}%
\pgfsetstrokecolor{currentstroke}%
\pgfsetdash{}{0pt}%
\pgfpathmoveto{\pgfqpoint{4.951976in}{2.215303in}}%
\pgfpathlineto{\pgfqpoint{4.966308in}{2.218507in}}%
\pgfpathlineto{\pgfqpoint{4.980652in}{2.221783in}}%
\pgfpathlineto{\pgfqpoint{4.995007in}{2.225129in}}%
\pgfpathlineto{\pgfqpoint{5.009375in}{2.228546in}}%
\pgfpathlineto{\pgfqpoint{5.017101in}{2.235844in}}%
\pgfpathlineto{\pgfqpoint{5.024819in}{2.243045in}}%
\pgfpathlineto{\pgfqpoint{5.032529in}{2.250149in}}%
\pgfpathlineto{\pgfqpoint{5.040231in}{2.257159in}}%
\pgfpathlineto{\pgfqpoint{5.025877in}{2.253831in}}%
\pgfpathlineto{\pgfqpoint{5.011534in}{2.250575in}}%
\pgfpathlineto{\pgfqpoint{4.997203in}{2.247389in}}%
\pgfpathlineto{\pgfqpoint{4.982884in}{2.244273in}}%
\pgfpathlineto{\pgfqpoint{4.975169in}{2.237166in}}%
\pgfpathlineto{\pgfqpoint{4.967445in}{2.229970in}}%
\pgfpathlineto{\pgfqpoint{4.959715in}{2.222683in}}%
\pgfpathlineto{\pgfqpoint{4.951976in}{2.215303in}}%
\pgfpathclose%
\pgfusepath{fill}%
\end{pgfscope}%
\begin{pgfscope}%
\pgfpathrectangle{\pgfqpoint{1.150000in}{0.150000in}}{\pgfqpoint{5.700000in}{5.700000in}}%
\pgfusepath{clip}%
\pgfsetbuttcap%
\pgfsetroundjoin%
\definecolor{currentfill}{rgb}{0.244972,0.287675,0.537260}%
\pgfsetfillcolor{currentfill}%
\pgfsetfillopacity{0.700000}%
\pgfsetlinewidth{0.000000pt}%
\definecolor{currentstroke}{rgb}{0.000000,0.000000,0.000000}%
\pgfsetstrokecolor{currentstroke}%
\pgfsetdash{}{0pt}%
\pgfpathmoveto{\pgfqpoint{2.021374in}{2.216697in}}%
\pgfpathlineto{\pgfqpoint{2.035320in}{2.202366in}}%
\pgfpathlineto{\pgfqpoint{2.049262in}{2.188168in}}%
\pgfpathlineto{\pgfqpoint{2.063200in}{2.174102in}}%
\pgfpathlineto{\pgfqpoint{2.077135in}{2.160168in}}%
\pgfpathlineto{\pgfqpoint{2.086405in}{2.157234in}}%
\pgfpathlineto{\pgfqpoint{2.095654in}{2.154606in}}%
\pgfpathlineto{\pgfqpoint{2.104880in}{2.152279in}}%
\pgfpathlineto{\pgfqpoint{2.114085in}{2.150245in}}%
\pgfpathlineto{\pgfqpoint{2.100195in}{2.163769in}}%
\pgfpathlineto{\pgfqpoint{2.086303in}{2.177424in}}%
\pgfpathlineto{\pgfqpoint{2.072406in}{2.191211in}}%
\pgfpathlineto{\pgfqpoint{2.058505in}{2.205130in}}%
\pgfpathlineto{\pgfqpoint{2.049256in}{2.207566in}}%
\pgfpathlineto{\pgfqpoint{2.039985in}{2.210301in}}%
\pgfpathlineto{\pgfqpoint{2.030691in}{2.213343in}}%
\pgfpathlineto{\pgfqpoint{2.021374in}{2.216697in}}%
\pgfpathclose%
\pgfusepath{fill}%
\end{pgfscope}%
\begin{pgfscope}%
\pgfpathrectangle{\pgfqpoint{1.150000in}{0.150000in}}{\pgfqpoint{5.700000in}{5.700000in}}%
\pgfusepath{clip}%
\pgfsetbuttcap%
\pgfsetroundjoin%
\definecolor{currentfill}{rgb}{0.185556,0.418570,0.556753}%
\pgfsetfillcolor{currentfill}%
\pgfsetfillopacity{0.700000}%
\pgfsetlinewidth{0.000000pt}%
\definecolor{currentstroke}{rgb}{0.000000,0.000000,0.000000}%
\pgfsetstrokecolor{currentstroke}%
\pgfsetdash{}{0pt}%
\pgfpathmoveto{\pgfqpoint{5.626942in}{2.497738in}}%
\pgfpathlineto{\pgfqpoint{5.641559in}{2.502104in}}%
\pgfpathlineto{\pgfqpoint{5.656189in}{2.506540in}}%
\pgfpathlineto{\pgfqpoint{5.670832in}{2.511045in}}%
\pgfpathlineto{\pgfqpoint{5.678192in}{2.514791in}}%
\pgfpathlineto{\pgfqpoint{5.685544in}{2.518483in}}%
\pgfpathlineto{\pgfqpoint{5.692888in}{2.522127in}}%
\pgfpathlineto{\pgfqpoint{5.700223in}{2.525725in}}%
\pgfpathlineto{\pgfqpoint{5.685603in}{2.521462in}}%
\pgfpathlineto{\pgfqpoint{5.670995in}{2.517269in}}%
\pgfpathlineto{\pgfqpoint{5.656401in}{2.513145in}}%
\pgfpathlineto{\pgfqpoint{5.649048in}{2.509359in}}%
\pgfpathlineto{\pgfqpoint{5.641687in}{2.505532in}}%
\pgfpathlineto{\pgfqpoint{5.634319in}{2.501660in}}%
\pgfpathlineto{\pgfqpoint{5.626942in}{2.497738in}}%
\pgfpathclose%
\pgfusepath{fill}%
\end{pgfscope}%
\begin{pgfscope}%
\pgfpathrectangle{\pgfqpoint{1.150000in}{0.150000in}}{\pgfqpoint{5.700000in}{5.700000in}}%
\pgfusepath{clip}%
\pgfsetbuttcap%
\pgfsetroundjoin%
\definecolor{currentfill}{rgb}{0.282884,0.135920,0.453427}%
\pgfsetfillcolor{currentfill}%
\pgfsetfillopacity{0.700000}%
\pgfsetlinewidth{0.000000pt}%
\definecolor{currentstroke}{rgb}{0.000000,0.000000,0.000000}%
\pgfsetstrokecolor{currentstroke}%
\pgfsetdash{}{0pt}%
\pgfpathmoveto{\pgfqpoint{2.446690in}{1.862189in}}%
\pgfpathlineto{\pgfqpoint{2.460531in}{1.851607in}}%
\pgfpathlineto{\pgfqpoint{2.474372in}{1.841132in}}%
\pgfpathlineto{\pgfqpoint{2.488213in}{1.830762in}}%
\pgfpathlineto{\pgfqpoint{2.502054in}{1.820498in}}%
\pgfpathlineto{\pgfqpoint{2.510962in}{1.821482in}}%
\pgfpathlineto{\pgfqpoint{2.519854in}{1.822707in}}%
\pgfpathlineto{\pgfqpoint{2.528730in}{1.824168in}}%
\pgfpathlineto{\pgfqpoint{2.537590in}{1.825860in}}%
\pgfpathlineto{\pgfqpoint{2.523784in}{1.835747in}}%
\pgfpathlineto{\pgfqpoint{2.509978in}{1.845739in}}%
\pgfpathlineto{\pgfqpoint{2.496172in}{1.855836in}}%
\pgfpathlineto{\pgfqpoint{2.482367in}{1.866040in}}%
\pgfpathlineto{\pgfqpoint{2.473472in}{1.864718in}}%
\pgfpathlineto{\pgfqpoint{2.464562in}{1.863632in}}%
\pgfpathlineto{\pgfqpoint{2.455634in}{1.862787in}}%
\pgfpathlineto{\pgfqpoint{2.446690in}{1.862189in}}%
\pgfpathclose%
\pgfusepath{fill}%
\end{pgfscope}%
\begin{pgfscope}%
\pgfpathrectangle{\pgfqpoint{1.150000in}{0.150000in}}{\pgfqpoint{5.700000in}{5.700000in}}%
\pgfusepath{clip}%
\pgfsetbuttcap%
\pgfsetroundjoin%
\definecolor{currentfill}{rgb}{0.241237,0.296485,0.539709}%
\pgfsetfillcolor{currentfill}%
\pgfsetfillopacity{0.700000}%
\pgfsetlinewidth{0.000000pt}%
\definecolor{currentstroke}{rgb}{0.000000,0.000000,0.000000}%
\pgfsetstrokecolor{currentstroke}%
\pgfsetdash{}{0pt}%
\pgfpathmoveto{\pgfqpoint{4.863688in}{2.172443in}}%
\pgfpathlineto{\pgfqpoint{4.877985in}{2.175431in}}%
\pgfpathlineto{\pgfqpoint{4.892294in}{2.178491in}}%
\pgfpathlineto{\pgfqpoint{4.906615in}{2.181621in}}%
\pgfpathlineto{\pgfqpoint{4.920947in}{2.184823in}}%
\pgfpathlineto{\pgfqpoint{4.928716in}{2.192590in}}%
\pgfpathlineto{\pgfqpoint{4.936477in}{2.200258in}}%
\pgfpathlineto{\pgfqpoint{4.944230in}{2.207829in}}%
\pgfpathlineto{\pgfqpoint{4.951976in}{2.215303in}}%
\pgfpathlineto{\pgfqpoint{4.937656in}{2.212170in}}%
\pgfpathlineto{\pgfqpoint{4.923347in}{2.209107in}}%
\pgfpathlineto{\pgfqpoint{4.909050in}{2.206115in}}%
\pgfpathlineto{\pgfqpoint{4.894765in}{2.203195in}}%
\pgfpathlineto{\pgfqpoint{4.887006in}{2.195645in}}%
\pgfpathlineto{\pgfqpoint{4.879241in}{2.188004in}}%
\pgfpathlineto{\pgfqpoint{4.871468in}{2.180271in}}%
\pgfpathlineto{\pgfqpoint{4.863688in}{2.172443in}}%
\pgfpathclose%
\pgfusepath{fill}%
\end{pgfscope}%
\begin{pgfscope}%
\pgfpathrectangle{\pgfqpoint{1.150000in}{0.150000in}}{\pgfqpoint{5.700000in}{5.700000in}}%
\pgfusepath{clip}%
\pgfsetbuttcap%
\pgfsetroundjoin%
\definecolor{currentfill}{rgb}{0.248629,0.278775,0.534556}%
\pgfsetfillcolor{currentfill}%
\pgfsetfillopacity{0.700000}%
\pgfsetlinewidth{0.000000pt}%
\definecolor{currentstroke}{rgb}{0.000000,0.000000,0.000000}%
\pgfsetstrokecolor{currentstroke}%
\pgfsetdash{}{0pt}%
\pgfpathmoveto{\pgfqpoint{4.775373in}{2.128741in}}%
\pgfpathlineto{\pgfqpoint{4.789636in}{2.131490in}}%
\pgfpathlineto{\pgfqpoint{4.803911in}{2.134311in}}%
\pgfpathlineto{\pgfqpoint{4.818197in}{2.137203in}}%
\pgfpathlineto{\pgfqpoint{4.832494in}{2.140167in}}%
\pgfpathlineto{\pgfqpoint{4.840303in}{2.148383in}}%
\pgfpathlineto{\pgfqpoint{4.848105in}{2.156501in}}%
\pgfpathlineto{\pgfqpoint{4.855900in}{2.164521in}}%
\pgfpathlineto{\pgfqpoint{4.863688in}{2.172443in}}%
\pgfpathlineto{\pgfqpoint{4.849402in}{2.169526in}}%
\pgfpathlineto{\pgfqpoint{4.835127in}{2.166680in}}%
\pgfpathlineto{\pgfqpoint{4.820864in}{2.163906in}}%
\pgfpathlineto{\pgfqpoint{4.806611in}{2.161202in}}%
\pgfpathlineto{\pgfqpoint{4.798812in}{2.153225in}}%
\pgfpathlineto{\pgfqpoint{4.791006in}{2.145157in}}%
\pgfpathlineto{\pgfqpoint{4.783193in}{2.136996in}}%
\pgfpathlineto{\pgfqpoint{4.775373in}{2.128741in}}%
\pgfpathclose%
\pgfusepath{fill}%
\end{pgfscope}%
\begin{pgfscope}%
\pgfpathrectangle{\pgfqpoint{1.150000in}{0.150000in}}{\pgfqpoint{5.700000in}{5.700000in}}%
\pgfusepath{clip}%
\pgfsetbuttcap%
\pgfsetroundjoin%
\definecolor{currentfill}{rgb}{0.268510,0.009605,0.335427}%
\pgfsetfillcolor{currentfill}%
\pgfsetfillopacity{0.700000}%
\pgfsetlinewidth{0.000000pt}%
\definecolor{currentstroke}{rgb}{0.000000,0.000000,0.000000}%
\pgfsetstrokecolor{currentstroke}%
\pgfsetdash{}{0pt}%
\pgfpathmoveto{\pgfqpoint{3.048568in}{1.604002in}}%
\pgfpathlineto{\pgfqpoint{3.062383in}{1.597913in}}%
\pgfpathlineto{\pgfqpoint{3.076201in}{1.591909in}}%
\pgfpathlineto{\pgfqpoint{3.090023in}{1.585992in}}%
\pgfpathlineto{\pgfqpoint{3.103848in}{1.580160in}}%
\pgfpathlineto{\pgfqpoint{3.112348in}{1.586533in}}%
\pgfpathlineto{\pgfqpoint{3.120837in}{1.593036in}}%
\pgfpathlineto{\pgfqpoint{3.129317in}{1.599665in}}%
\pgfpathlineto{\pgfqpoint{3.137788in}{1.606414in}}%
\pgfpathlineto{\pgfqpoint{3.123984in}{1.611938in}}%
\pgfpathlineto{\pgfqpoint{3.110183in}{1.617547in}}%
\pgfpathlineto{\pgfqpoint{3.096387in}{1.623242in}}%
\pgfpathlineto{\pgfqpoint{3.082595in}{1.629023in}}%
\pgfpathlineto{\pgfqpoint{3.074103in}{1.622574in}}%
\pgfpathlineto{\pgfqpoint{3.065601in}{1.616252in}}%
\pgfpathlineto{\pgfqpoint{3.057089in}{1.610059in}}%
\pgfpathlineto{\pgfqpoint{3.048568in}{1.604002in}}%
\pgfpathclose%
\pgfusepath{fill}%
\end{pgfscope}%
\begin{pgfscope}%
\pgfpathrectangle{\pgfqpoint{1.150000in}{0.150000in}}{\pgfqpoint{5.700000in}{5.700000in}}%
\pgfusepath{clip}%
\pgfsetbuttcap%
\pgfsetroundjoin%
\definecolor{currentfill}{rgb}{0.279566,0.067836,0.391917}%
\pgfsetfillcolor{currentfill}%
\pgfsetfillopacity{0.700000}%
\pgfsetlinewidth{0.000000pt}%
\definecolor{currentstroke}{rgb}{0.000000,0.000000,0.000000}%
\pgfsetstrokecolor{currentstroke}%
\pgfsetdash{}{0pt}%
\pgfpathmoveto{\pgfqpoint{2.703304in}{1.715132in}}%
\pgfpathlineto{\pgfqpoint{2.717120in}{1.706546in}}%
\pgfpathlineto{\pgfqpoint{2.730938in}{1.698055in}}%
\pgfpathlineto{\pgfqpoint{2.744757in}{1.689659in}}%
\pgfpathlineto{\pgfqpoint{2.758578in}{1.681358in}}%
\pgfpathlineto{\pgfqpoint{2.767297in}{1.684720in}}%
\pgfpathlineto{\pgfqpoint{2.776002in}{1.688279in}}%
\pgfpathlineto{\pgfqpoint{2.784694in}{1.692029in}}%
\pgfpathlineto{\pgfqpoint{2.793373in}{1.695966in}}%
\pgfpathlineto{\pgfqpoint{2.779581in}{1.703915in}}%
\pgfpathlineto{\pgfqpoint{2.765791in}{1.711958in}}%
\pgfpathlineto{\pgfqpoint{2.752003in}{1.720096in}}%
\pgfpathlineto{\pgfqpoint{2.738216in}{1.728330in}}%
\pgfpathlineto{\pgfqpoint{2.729508in}{1.724738in}}%
\pgfpathlineto{\pgfqpoint{2.720787in}{1.721338in}}%
\pgfpathlineto{\pgfqpoint{2.712052in}{1.718134in}}%
\pgfpathlineto{\pgfqpoint{2.703304in}{1.715132in}}%
\pgfpathclose%
\pgfusepath{fill}%
\end{pgfscope}%
\begin{pgfscope}%
\pgfpathrectangle{\pgfqpoint{1.150000in}{0.150000in}}{\pgfqpoint{5.700000in}{5.700000in}}%
\pgfusepath{clip}%
\pgfsetbuttcap%
\pgfsetroundjoin%
\definecolor{currentfill}{rgb}{0.253935,0.265254,0.529983}%
\pgfsetfillcolor{currentfill}%
\pgfsetfillopacity{0.700000}%
\pgfsetlinewidth{0.000000pt}%
\definecolor{currentstroke}{rgb}{0.000000,0.000000,0.000000}%
\pgfsetstrokecolor{currentstroke}%
\pgfsetdash{}{0pt}%
\pgfpathmoveto{\pgfqpoint{2.077135in}{2.160168in}}%
\pgfpathlineto{\pgfqpoint{2.091065in}{2.146363in}}%
\pgfpathlineto{\pgfqpoint{2.104992in}{2.132688in}}%
\pgfpathlineto{\pgfqpoint{2.118915in}{2.119141in}}%
\pgfpathlineto{\pgfqpoint{2.132835in}{2.105720in}}%
\pgfpathlineto{\pgfqpoint{2.142061in}{2.103204in}}%
\pgfpathlineto{\pgfqpoint{2.151264in}{2.100989in}}%
\pgfpathlineto{\pgfqpoint{2.160447in}{2.099067in}}%
\pgfpathlineto{\pgfqpoint{2.169608in}{2.097434in}}%
\pgfpathlineto{\pgfqpoint{2.155732in}{2.110446in}}%
\pgfpathlineto{\pgfqpoint{2.141853in}{2.123585in}}%
\pgfpathlineto{\pgfqpoint{2.127970in}{2.136851in}}%
\pgfpathlineto{\pgfqpoint{2.114085in}{2.150245in}}%
\pgfpathlineto{\pgfqpoint{2.104880in}{2.152279in}}%
\pgfpathlineto{\pgfqpoint{2.095654in}{2.154606in}}%
\pgfpathlineto{\pgfqpoint{2.086405in}{2.157234in}}%
\pgfpathlineto{\pgfqpoint{2.077135in}{2.160168in}}%
\pgfpathclose%
\pgfusepath{fill}%
\end{pgfscope}%
\begin{pgfscope}%
\pgfpathrectangle{\pgfqpoint{1.150000in}{0.150000in}}{\pgfqpoint{5.700000in}{5.700000in}}%
\pgfusepath{clip}%
\pgfsetbuttcap%
\pgfsetroundjoin%
\definecolor{currentfill}{rgb}{0.267004,0.004874,0.329415}%
\pgfsetfillcolor{currentfill}%
\pgfsetfillopacity{0.700000}%
\pgfsetlinewidth{0.000000pt}%
\definecolor{currentstroke}{rgb}{0.000000,0.000000,0.000000}%
\pgfsetstrokecolor{currentstroke}%
\pgfsetdash{}{0pt}%
\pgfpathmoveto{\pgfqpoint{3.193047in}{1.585161in}}%
\pgfpathlineto{\pgfqpoint{3.206873in}{1.580058in}}%
\pgfpathlineto{\pgfqpoint{3.220703in}{1.575037in}}%
\pgfpathlineto{\pgfqpoint{3.234538in}{1.570099in}}%
\pgfpathlineto{\pgfqpoint{3.248377in}{1.565243in}}%
\pgfpathlineto{\pgfqpoint{3.256799in}{1.572698in}}%
\pgfpathlineto{\pgfqpoint{3.265213in}{1.580255in}}%
\pgfpathlineto{\pgfqpoint{3.273618in}{1.587909in}}%
\pgfpathlineto{\pgfqpoint{3.282016in}{1.595656in}}%
\pgfpathlineto{\pgfqpoint{3.268195in}{1.600224in}}%
\pgfpathlineto{\pgfqpoint{3.254379in}{1.604875in}}%
\pgfpathlineto{\pgfqpoint{3.240568in}{1.609609in}}%
\pgfpathlineto{\pgfqpoint{3.226762in}{1.614426in}}%
\pgfpathlineto{\pgfqpoint{3.218346in}{1.606958in}}%
\pgfpathlineto{\pgfqpoint{3.209922in}{1.599589in}}%
\pgfpathlineto{\pgfqpoint{3.201489in}{1.592322in}}%
\pgfpathlineto{\pgfqpoint{3.193047in}{1.585161in}}%
\pgfpathclose%
\pgfusepath{fill}%
\end{pgfscope}%
\begin{pgfscope}%
\pgfpathrectangle{\pgfqpoint{1.150000in}{0.150000in}}{\pgfqpoint{5.700000in}{5.700000in}}%
\pgfusepath{clip}%
\pgfsetbuttcap%
\pgfsetroundjoin%
\definecolor{currentfill}{rgb}{0.274952,0.037752,0.364543}%
\pgfsetfillcolor{currentfill}%
\pgfsetfillopacity{0.700000}%
\pgfsetlinewidth{0.000000pt}%
\definecolor{currentstroke}{rgb}{0.000000,0.000000,0.000000}%
\pgfsetstrokecolor{currentstroke}%
\pgfsetdash{}{0pt}%
\pgfpathmoveto{\pgfqpoint{3.658838in}{1.634695in}}%
\pgfpathlineto{\pgfqpoint{3.672738in}{1.632529in}}%
\pgfpathlineto{\pgfqpoint{3.686644in}{1.630440in}}%
\pgfpathlineto{\pgfqpoint{3.700557in}{1.628428in}}%
\pgfpathlineto{\pgfqpoint{3.714478in}{1.626492in}}%
\pgfpathlineto{\pgfqpoint{3.722698in}{1.636382in}}%
\pgfpathlineto{\pgfqpoint{3.730913in}{1.646284in}}%
\pgfpathlineto{\pgfqpoint{3.739122in}{1.656197in}}%
\pgfpathlineto{\pgfqpoint{3.747325in}{1.666116in}}%
\pgfpathlineto{\pgfqpoint{3.733416in}{1.667847in}}%
\pgfpathlineto{\pgfqpoint{3.719515in}{1.669655in}}%
\pgfpathlineto{\pgfqpoint{3.705620in}{1.671538in}}%
\pgfpathlineto{\pgfqpoint{3.691733in}{1.673499in}}%
\pgfpathlineto{\pgfqpoint{3.683518in}{1.663777in}}%
\pgfpathlineto{\pgfqpoint{3.675297in}{1.654067in}}%
\pgfpathlineto{\pgfqpoint{3.667071in}{1.644372in}}%
\pgfpathlineto{\pgfqpoint{3.658838in}{1.634695in}}%
\pgfpathclose%
\pgfusepath{fill}%
\end{pgfscope}%
\begin{pgfscope}%
\pgfpathrectangle{\pgfqpoint{1.150000in}{0.150000in}}{\pgfqpoint{5.700000in}{5.700000in}}%
\pgfusepath{clip}%
\pgfsetbuttcap%
\pgfsetroundjoin%
\definecolor{currentfill}{rgb}{0.257322,0.256130,0.526563}%
\pgfsetfillcolor{currentfill}%
\pgfsetfillopacity{0.700000}%
\pgfsetlinewidth{0.000000pt}%
\definecolor{currentstroke}{rgb}{0.000000,0.000000,0.000000}%
\pgfsetstrokecolor{currentstroke}%
\pgfsetdash{}{0pt}%
\pgfpathmoveto{\pgfqpoint{4.687038in}{2.084379in}}%
\pgfpathlineto{\pgfqpoint{4.701268in}{2.086867in}}%
\pgfpathlineto{\pgfqpoint{4.715508in}{2.089427in}}%
\pgfpathlineto{\pgfqpoint{4.729759in}{2.092059in}}%
\pgfpathlineto{\pgfqpoint{4.744022in}{2.094762in}}%
\pgfpathlineto{\pgfqpoint{4.751870in}{2.103402in}}%
\pgfpathlineto{\pgfqpoint{4.759711in}{2.111945in}}%
\pgfpathlineto{\pgfqpoint{4.767546in}{2.120391in}}%
\pgfpathlineto{\pgfqpoint{4.775373in}{2.128741in}}%
\pgfpathlineto{\pgfqpoint{4.761121in}{2.126063in}}%
\pgfpathlineto{\pgfqpoint{4.746880in}{2.123456in}}%
\pgfpathlineto{\pgfqpoint{4.732650in}{2.120921in}}%
\pgfpathlineto{\pgfqpoint{4.718431in}{2.118457in}}%
\pgfpathlineto{\pgfqpoint{4.710593in}{2.110074in}}%
\pgfpathlineto{\pgfqpoint{4.702749in}{2.101601in}}%
\pgfpathlineto{\pgfqpoint{4.694897in}{2.093036in}}%
\pgfpathlineto{\pgfqpoint{4.687038in}{2.084379in}}%
\pgfpathclose%
\pgfusepath{fill}%
\end{pgfscope}%
\begin{pgfscope}%
\pgfpathrectangle{\pgfqpoint{1.150000in}{0.150000in}}{\pgfqpoint{5.700000in}{5.700000in}}%
\pgfusepath{clip}%
\pgfsetbuttcap%
\pgfsetroundjoin%
\definecolor{currentfill}{rgb}{0.263663,0.237631,0.518762}%
\pgfsetfillcolor{currentfill}%
\pgfsetfillopacity{0.700000}%
\pgfsetlinewidth{0.000000pt}%
\definecolor{currentstroke}{rgb}{0.000000,0.000000,0.000000}%
\pgfsetstrokecolor{currentstroke}%
\pgfsetdash{}{0pt}%
\pgfpathmoveto{\pgfqpoint{4.598689in}{2.039561in}}%
\pgfpathlineto{\pgfqpoint{4.612885in}{2.041766in}}%
\pgfpathlineto{\pgfqpoint{4.627091in}{2.044043in}}%
\pgfpathlineto{\pgfqpoint{4.641308in}{2.046391in}}%
\pgfpathlineto{\pgfqpoint{4.655537in}{2.048812in}}%
\pgfpathlineto{\pgfqpoint{4.663422in}{2.057845in}}%
\pgfpathlineto{\pgfqpoint{4.671301in}{2.066784in}}%
\pgfpathlineto{\pgfqpoint{4.679173in}{2.075628in}}%
\pgfpathlineto{\pgfqpoint{4.687038in}{2.084379in}}%
\pgfpathlineto{\pgfqpoint{4.672820in}{2.081962in}}%
\pgfpathlineto{\pgfqpoint{4.658613in}{2.079617in}}%
\pgfpathlineto{\pgfqpoint{4.644416in}{2.077343in}}%
\pgfpathlineto{\pgfqpoint{4.630230in}{2.075141in}}%
\pgfpathlineto{\pgfqpoint{4.622355in}{2.066380in}}%
\pgfpathlineto{\pgfqpoint{4.614473in}{2.057529in}}%
\pgfpathlineto{\pgfqpoint{4.606584in}{2.048590in}}%
\pgfpathlineto{\pgfqpoint{4.598689in}{2.039561in}}%
\pgfpathclose%
\pgfusepath{fill}%
\end{pgfscope}%
\begin{pgfscope}%
\pgfpathrectangle{\pgfqpoint{1.150000in}{0.150000in}}{\pgfqpoint{5.700000in}{5.700000in}}%
\pgfusepath{clip}%
\pgfsetbuttcap%
\pgfsetroundjoin%
\definecolor{currentfill}{rgb}{0.269308,0.218818,0.509577}%
\pgfsetfillcolor{currentfill}%
\pgfsetfillopacity{0.700000}%
\pgfsetlinewidth{0.000000pt}%
\definecolor{currentstroke}{rgb}{0.000000,0.000000,0.000000}%
\pgfsetstrokecolor{currentstroke}%
\pgfsetdash{}{0pt}%
\pgfpathmoveto{\pgfqpoint{4.510330in}{1.994514in}}%
\pgfpathlineto{\pgfqpoint{4.524492in}{1.996413in}}%
\pgfpathlineto{\pgfqpoint{4.538666in}{1.998384in}}%
\pgfpathlineto{\pgfqpoint{4.552849in}{2.000428in}}%
\pgfpathlineto{\pgfqpoint{4.567043in}{2.002543in}}%
\pgfpathlineto{\pgfqpoint{4.574965in}{2.011933in}}%
\pgfpathlineto{\pgfqpoint{4.582879in}{2.021233in}}%
\pgfpathlineto{\pgfqpoint{4.590788in}{2.030442in}}%
\pgfpathlineto{\pgfqpoint{4.598689in}{2.039561in}}%
\pgfpathlineto{\pgfqpoint{4.584504in}{2.037428in}}%
\pgfpathlineto{\pgfqpoint{4.570330in}{2.035367in}}%
\pgfpathlineto{\pgfqpoint{4.556166in}{2.033377in}}%
\pgfpathlineto{\pgfqpoint{4.542012in}{2.031460in}}%
\pgfpathlineto{\pgfqpoint{4.534101in}{2.022351in}}%
\pgfpathlineto{\pgfqpoint{4.526184in}{2.013157in}}%
\pgfpathlineto{\pgfqpoint{4.518260in}{2.003878in}}%
\pgfpathlineto{\pgfqpoint{4.510330in}{1.994514in}}%
\pgfpathclose%
\pgfusepath{fill}%
\end{pgfscope}%
\begin{pgfscope}%
\pgfpathrectangle{\pgfqpoint{1.150000in}{0.150000in}}{\pgfqpoint{5.700000in}{5.700000in}}%
\pgfusepath{clip}%
\pgfsetbuttcap%
\pgfsetroundjoin%
\definecolor{currentfill}{rgb}{0.272594,0.025563,0.353093}%
\pgfsetfillcolor{currentfill}%
\pgfsetfillopacity{0.700000}%
\pgfsetlinewidth{0.000000pt}%
\definecolor{currentstroke}{rgb}{0.000000,0.000000,0.000000}%
\pgfsetstrokecolor{currentstroke}%
\pgfsetdash{}{0pt}%
\pgfpathmoveto{\pgfqpoint{2.903790in}{1.635713in}}%
\pgfpathlineto{\pgfqpoint{2.917603in}{1.628591in}}%
\pgfpathlineto{\pgfqpoint{2.931419in}{1.621559in}}%
\pgfpathlineto{\pgfqpoint{2.945238in}{1.614615in}}%
\pgfpathlineto{\pgfqpoint{2.959060in}{1.607761in}}%
\pgfpathlineto{\pgfqpoint{2.967648in}{1.612892in}}%
\pgfpathlineto{\pgfqpoint{2.976224in}{1.618184in}}%
\pgfpathlineto{\pgfqpoint{2.984790in}{1.623630in}}%
\pgfpathlineto{\pgfqpoint{2.993345in}{1.629226in}}%
\pgfpathlineto{\pgfqpoint{2.979548in}{1.635751in}}%
\pgfpathlineto{\pgfqpoint{2.965753in}{1.642365in}}%
\pgfpathlineto{\pgfqpoint{2.951962in}{1.649067in}}%
\pgfpathlineto{\pgfqpoint{2.938174in}{1.655859in}}%
\pgfpathlineto{\pgfqpoint{2.929595in}{1.650585in}}%
\pgfpathlineto{\pgfqpoint{2.921005in}{1.645466in}}%
\pgfpathlineto{\pgfqpoint{2.912403in}{1.640507in}}%
\pgfpathlineto{\pgfqpoint{2.903790in}{1.635713in}}%
\pgfpathclose%
\pgfusepath{fill}%
\end{pgfscope}%
\begin{pgfscope}%
\pgfpathrectangle{\pgfqpoint{1.150000in}{0.150000in}}{\pgfqpoint{5.700000in}{5.700000in}}%
\pgfusepath{clip}%
\pgfsetbuttcap%
\pgfsetroundjoin%
\definecolor{currentfill}{rgb}{0.274128,0.199721,0.498911}%
\pgfsetfillcolor{currentfill}%
\pgfsetfillopacity{0.700000}%
\pgfsetlinewidth{0.000000pt}%
\definecolor{currentstroke}{rgb}{0.000000,0.000000,0.000000}%
\pgfsetstrokecolor{currentstroke}%
\pgfsetdash{}{0pt}%
\pgfpathmoveto{\pgfqpoint{4.421962in}{1.949484in}}%
\pgfpathlineto{\pgfqpoint{4.436093in}{1.951055in}}%
\pgfpathlineto{\pgfqpoint{4.450234in}{1.952699in}}%
\pgfpathlineto{\pgfqpoint{4.464385in}{1.954414in}}%
\pgfpathlineto{\pgfqpoint{4.478546in}{1.956202in}}%
\pgfpathlineto{\pgfqpoint{4.486501in}{1.965908in}}%
\pgfpathlineto{\pgfqpoint{4.494450in}{1.975529in}}%
\pgfpathlineto{\pgfqpoint{4.502393in}{1.985064in}}%
\pgfpathlineto{\pgfqpoint{4.510330in}{1.994514in}}%
\pgfpathlineto{\pgfqpoint{4.496177in}{1.992687in}}%
\pgfpathlineto{\pgfqpoint{4.482035in}{1.990932in}}%
\pgfpathlineto{\pgfqpoint{4.467903in}{1.989249in}}%
\pgfpathlineto{\pgfqpoint{4.453781in}{1.987638in}}%
\pgfpathlineto{\pgfqpoint{4.445836in}{1.978220in}}%
\pgfpathlineto{\pgfqpoint{4.437884in}{1.968721in}}%
\pgfpathlineto{\pgfqpoint{4.429926in}{1.959143in}}%
\pgfpathlineto{\pgfqpoint{4.421962in}{1.949484in}}%
\pgfpathclose%
\pgfusepath{fill}%
\end{pgfscope}%
\begin{pgfscope}%
\pgfpathrectangle{\pgfqpoint{1.150000in}{0.150000in}}{\pgfqpoint{5.700000in}{5.700000in}}%
\pgfusepath{clip}%
\pgfsetbuttcap%
\pgfsetroundjoin%
\definecolor{currentfill}{rgb}{0.278012,0.180367,0.486697}%
\pgfsetfillcolor{currentfill}%
\pgfsetfillopacity{0.700000}%
\pgfsetlinewidth{0.000000pt}%
\definecolor{currentstroke}{rgb}{0.000000,0.000000,0.000000}%
\pgfsetstrokecolor{currentstroke}%
\pgfsetdash{}{0pt}%
\pgfpathmoveto{\pgfqpoint{4.333588in}{1.904740in}}%
\pgfpathlineto{\pgfqpoint{4.347688in}{1.905961in}}%
\pgfpathlineto{\pgfqpoint{4.361797in}{1.907254in}}%
\pgfpathlineto{\pgfqpoint{4.375917in}{1.908620in}}%
\pgfpathlineto{\pgfqpoint{4.390046in}{1.910058in}}%
\pgfpathlineto{\pgfqpoint{4.398034in}{1.920033in}}%
\pgfpathlineto{\pgfqpoint{4.406016in}{1.929929in}}%
\pgfpathlineto{\pgfqpoint{4.413992in}{1.939746in}}%
\pgfpathlineto{\pgfqpoint{4.421962in}{1.949484in}}%
\pgfpathlineto{\pgfqpoint{4.407842in}{1.947985in}}%
\pgfpathlineto{\pgfqpoint{4.393731in}{1.946559in}}%
\pgfpathlineto{\pgfqpoint{4.379630in}{1.945205in}}%
\pgfpathlineto{\pgfqpoint{4.365539in}{1.943924in}}%
\pgfpathlineto{\pgfqpoint{4.357560in}{1.934238in}}%
\pgfpathlineto{\pgfqpoint{4.349576in}{1.924479in}}%
\pgfpathlineto{\pgfqpoint{4.341585in}{1.914646in}}%
\pgfpathlineto{\pgfqpoint{4.333588in}{1.904740in}}%
\pgfpathclose%
\pgfusepath{fill}%
\end{pgfscope}%
\begin{pgfscope}%
\pgfpathrectangle{\pgfqpoint{1.150000in}{0.150000in}}{\pgfqpoint{5.700000in}{5.700000in}}%
\pgfusepath{clip}%
\pgfsetbuttcap%
\pgfsetroundjoin%
\definecolor{currentfill}{rgb}{0.260571,0.246922,0.522828}%
\pgfsetfillcolor{currentfill}%
\pgfsetfillopacity{0.700000}%
\pgfsetlinewidth{0.000000pt}%
\definecolor{currentstroke}{rgb}{0.000000,0.000000,0.000000}%
\pgfsetstrokecolor{currentstroke}%
\pgfsetdash{}{0pt}%
\pgfpathmoveto{\pgfqpoint{2.132835in}{2.105720in}}%
\pgfpathlineto{\pgfqpoint{2.146752in}{2.092425in}}%
\pgfpathlineto{\pgfqpoint{2.160666in}{2.079255in}}%
\pgfpathlineto{\pgfqpoint{2.174577in}{2.066208in}}%
\pgfpathlineto{\pgfqpoint{2.188485in}{2.053284in}}%
\pgfpathlineto{\pgfqpoint{2.197666in}{2.051184in}}%
\pgfpathlineto{\pgfqpoint{2.206826in}{2.049379in}}%
\pgfpathlineto{\pgfqpoint{2.215966in}{2.047862in}}%
\pgfpathlineto{\pgfqpoint{2.225085in}{2.046627in}}%
\pgfpathlineto{\pgfqpoint{2.211220in}{2.059145in}}%
\pgfpathlineto{\pgfqpoint{2.197352in}{2.071784in}}%
\pgfpathlineto{\pgfqpoint{2.183481in}{2.084547in}}%
\pgfpathlineto{\pgfqpoint{2.169608in}{2.097434in}}%
\pgfpathlineto{\pgfqpoint{2.160447in}{2.099067in}}%
\pgfpathlineto{\pgfqpoint{2.151264in}{2.100989in}}%
\pgfpathlineto{\pgfqpoint{2.142061in}{2.103204in}}%
\pgfpathlineto{\pgfqpoint{2.132835in}{2.105720in}}%
\pgfpathclose%
\pgfusepath{fill}%
\end{pgfscope}%
\begin{pgfscope}%
\pgfpathrectangle{\pgfqpoint{1.150000in}{0.150000in}}{\pgfqpoint{5.700000in}{5.700000in}}%
\pgfusepath{clip}%
\pgfsetbuttcap%
\pgfsetroundjoin%
\definecolor{currentfill}{rgb}{0.188923,0.410910,0.556326}%
\pgfsetfillcolor{currentfill}%
\pgfsetfillopacity{0.700000}%
\pgfsetlinewidth{0.000000pt}%
\definecolor{currentstroke}{rgb}{0.000000,0.000000,0.000000}%
\pgfsetstrokecolor{currentstroke}%
\pgfsetdash{}{0pt}%
\pgfpathmoveto{\pgfqpoint{5.538933in}{2.463819in}}%
\pgfpathlineto{\pgfqpoint{5.553517in}{2.468127in}}%
\pgfpathlineto{\pgfqpoint{5.568116in}{2.472505in}}%
\pgfpathlineto{\pgfqpoint{5.582727in}{2.476953in}}%
\pgfpathlineto{\pgfqpoint{5.597352in}{2.481471in}}%
\pgfpathlineto{\pgfqpoint{5.604762in}{2.485632in}}%
\pgfpathlineto{\pgfqpoint{5.612164in}{2.489728in}}%
\pgfpathlineto{\pgfqpoint{5.619557in}{2.493762in}}%
\pgfpathlineto{\pgfqpoint{5.626942in}{2.497738in}}%
\pgfpathlineto{\pgfqpoint{5.612339in}{2.493442in}}%
\pgfpathlineto{\pgfqpoint{5.597748in}{2.489215in}}%
\pgfpathlineto{\pgfqpoint{5.583171in}{2.485057in}}%
\pgfpathlineto{\pgfqpoint{5.568607in}{2.480970in}}%
\pgfpathlineto{\pgfqpoint{5.561201in}{2.476765in}}%
\pgfpathlineto{\pgfqpoint{5.553786in}{2.472508in}}%
\pgfpathlineto{\pgfqpoint{5.546364in}{2.468194in}}%
\pgfpathlineto{\pgfqpoint{5.538933in}{2.463819in}}%
\pgfpathclose%
\pgfusepath{fill}%
\end{pgfscope}%
\begin{pgfscope}%
\pgfpathrectangle{\pgfqpoint{1.150000in}{0.150000in}}{\pgfqpoint{5.700000in}{5.700000in}}%
\pgfusepath{clip}%
\pgfsetbuttcap%
\pgfsetroundjoin%
\definecolor{currentfill}{rgb}{0.281412,0.155834,0.469201}%
\pgfsetfillcolor{currentfill}%
\pgfsetfillopacity{0.700000}%
\pgfsetlinewidth{0.000000pt}%
\definecolor{currentstroke}{rgb}{0.000000,0.000000,0.000000}%
\pgfsetstrokecolor{currentstroke}%
\pgfsetdash{}{0pt}%
\pgfpathmoveto{\pgfqpoint{4.245208in}{1.860571in}}%
\pgfpathlineto{\pgfqpoint{4.259277in}{1.861419in}}%
\pgfpathlineto{\pgfqpoint{4.273357in}{1.862340in}}%
\pgfpathlineto{\pgfqpoint{4.287445in}{1.863334in}}%
\pgfpathlineto{\pgfqpoint{4.301544in}{1.864401in}}%
\pgfpathlineto{\pgfqpoint{4.309564in}{1.874591in}}%
\pgfpathlineto{\pgfqpoint{4.317578in}{1.884712in}}%
\pgfpathlineto{\pgfqpoint{4.325586in}{1.894762in}}%
\pgfpathlineto{\pgfqpoint{4.333588in}{1.904740in}}%
\pgfpathlineto{\pgfqpoint{4.319498in}{1.903592in}}%
\pgfpathlineto{\pgfqpoint{4.305418in}{1.902517in}}%
\pgfpathlineto{\pgfqpoint{4.291348in}{1.901514in}}%
\pgfpathlineto{\pgfqpoint{4.277286in}{1.900585in}}%
\pgfpathlineto{\pgfqpoint{4.269275in}{1.890680in}}%
\pgfpathlineto{\pgfqpoint{4.261258in}{1.880709in}}%
\pgfpathlineto{\pgfqpoint{4.253236in}{1.870672in}}%
\pgfpathlineto{\pgfqpoint{4.245208in}{1.860571in}}%
\pgfpathclose%
\pgfusepath{fill}%
\end{pgfscope}%
\begin{pgfscope}%
\pgfpathrectangle{\pgfqpoint{1.150000in}{0.150000in}}{\pgfqpoint{5.700000in}{5.700000in}}%
\pgfusepath{clip}%
\pgfsetbuttcap%
\pgfsetroundjoin%
\definecolor{currentfill}{rgb}{0.267004,0.004874,0.329415}%
\pgfsetfillcolor{currentfill}%
\pgfsetfillopacity{0.700000}%
\pgfsetlinewidth{0.000000pt}%
\definecolor{currentstroke}{rgb}{0.000000,0.000000,0.000000}%
\pgfsetstrokecolor{currentstroke}%
\pgfsetdash{}{0pt}%
\pgfpathmoveto{\pgfqpoint{3.337349in}{1.578198in}}%
\pgfpathlineto{\pgfqpoint{3.351194in}{1.574037in}}%
\pgfpathlineto{\pgfqpoint{3.365046in}{1.569956in}}%
\pgfpathlineto{\pgfqpoint{3.378902in}{1.565955in}}%
\pgfpathlineto{\pgfqpoint{3.392765in}{1.562034in}}%
\pgfpathlineto{\pgfqpoint{3.401120in}{1.570418in}}%
\pgfpathlineto{\pgfqpoint{3.409467in}{1.578877in}}%
\pgfpathlineto{\pgfqpoint{3.417808in}{1.587406in}}%
\pgfpathlineto{\pgfqpoint{3.426141in}{1.596002in}}%
\pgfpathlineto{\pgfqpoint{3.412295in}{1.599657in}}%
\pgfpathlineto{\pgfqpoint{3.398455in}{1.603391in}}%
\pgfpathlineto{\pgfqpoint{3.384620in}{1.607206in}}%
\pgfpathlineto{\pgfqpoint{3.370791in}{1.611101in}}%
\pgfpathlineto{\pgfqpoint{3.362441in}{1.602763in}}%
\pgfpathlineto{\pgfqpoint{3.354085in}{1.594498in}}%
\pgfpathlineto{\pgfqpoint{3.345720in}{1.586308in}}%
\pgfpathlineto{\pgfqpoint{3.337349in}{1.578198in}}%
\pgfpathclose%
\pgfusepath{fill}%
\end{pgfscope}%
\begin{pgfscope}%
\pgfpathrectangle{\pgfqpoint{1.150000in}{0.150000in}}{\pgfqpoint{5.700000in}{5.700000in}}%
\pgfusepath{clip}%
\pgfsetbuttcap%
\pgfsetroundjoin%
\definecolor{currentfill}{rgb}{0.283229,0.120777,0.440584}%
\pgfsetfillcolor{currentfill}%
\pgfsetfillopacity{0.700000}%
\pgfsetlinewidth{0.000000pt}%
\definecolor{currentstroke}{rgb}{0.000000,0.000000,0.000000}%
\pgfsetstrokecolor{currentstroke}%
\pgfsetdash{}{0pt}%
\pgfpathmoveto{\pgfqpoint{2.502054in}{1.820498in}}%
\pgfpathlineto{\pgfqpoint{2.515894in}{1.810338in}}%
\pgfpathlineto{\pgfqpoint{2.529735in}{1.800281in}}%
\pgfpathlineto{\pgfqpoint{2.543576in}{1.790327in}}%
\pgfpathlineto{\pgfqpoint{2.557417in}{1.780475in}}%
\pgfpathlineto{\pgfqpoint{2.566290in}{1.781844in}}%
\pgfpathlineto{\pgfqpoint{2.575147in}{1.783449in}}%
\pgfpathlineto{\pgfqpoint{2.583989in}{1.785284in}}%
\pgfpathlineto{\pgfqpoint{2.592816in}{1.787344in}}%
\pgfpathlineto{\pgfqpoint{2.579008in}{1.796819in}}%
\pgfpathlineto{\pgfqpoint{2.565202in}{1.806396in}}%
\pgfpathlineto{\pgfqpoint{2.551395in}{1.816076in}}%
\pgfpathlineto{\pgfqpoint{2.537590in}{1.825860in}}%
\pgfpathlineto{\pgfqpoint{2.528730in}{1.824168in}}%
\pgfpathlineto{\pgfqpoint{2.519854in}{1.822707in}}%
\pgfpathlineto{\pgfqpoint{2.510962in}{1.821482in}}%
\pgfpathlineto{\pgfqpoint{2.502054in}{1.820498in}}%
\pgfpathclose%
\pgfusepath{fill}%
\end{pgfscope}%
\begin{pgfscope}%
\pgfpathrectangle{\pgfqpoint{1.150000in}{0.150000in}}{\pgfqpoint{5.700000in}{5.700000in}}%
\pgfusepath{clip}%
\pgfsetbuttcap%
\pgfsetroundjoin%
\definecolor{currentfill}{rgb}{0.282884,0.135920,0.453427}%
\pgfsetfillcolor{currentfill}%
\pgfsetfillopacity{0.700000}%
\pgfsetlinewidth{0.000000pt}%
\definecolor{currentstroke}{rgb}{0.000000,0.000000,0.000000}%
\pgfsetstrokecolor{currentstroke}%
\pgfsetdash{}{0pt}%
\pgfpathmoveto{\pgfqpoint{4.156817in}{1.817287in}}%
\pgfpathlineto{\pgfqpoint{4.170859in}{1.817740in}}%
\pgfpathlineto{\pgfqpoint{4.184909in}{1.818267in}}%
\pgfpathlineto{\pgfqpoint{4.198969in}{1.818866in}}%
\pgfpathlineto{\pgfqpoint{4.213038in}{1.819539in}}%
\pgfpathlineto{\pgfqpoint{4.221089in}{1.829889in}}%
\pgfpathlineto{\pgfqpoint{4.229134in}{1.840178in}}%
\pgfpathlineto{\pgfqpoint{4.237174in}{1.850406in}}%
\pgfpathlineto{\pgfqpoint{4.245208in}{1.860571in}}%
\pgfpathlineto{\pgfqpoint{4.231147in}{1.859796in}}%
\pgfpathlineto{\pgfqpoint{4.217096in}{1.859094in}}%
\pgfpathlineto{\pgfqpoint{4.203054in}{1.858465in}}%
\pgfpathlineto{\pgfqpoint{4.189022in}{1.857910in}}%
\pgfpathlineto{\pgfqpoint{4.180979in}{1.847839in}}%
\pgfpathlineto{\pgfqpoint{4.172931in}{1.837711in}}%
\pgfpathlineto{\pgfqpoint{4.164877in}{1.827527in}}%
\pgfpathlineto{\pgfqpoint{4.156817in}{1.817287in}}%
\pgfpathclose%
\pgfusepath{fill}%
\end{pgfscope}%
\begin{pgfscope}%
\pgfpathrectangle{\pgfqpoint{1.150000in}{0.150000in}}{\pgfqpoint{5.700000in}{5.700000in}}%
\pgfusepath{clip}%
\pgfsetbuttcap%
\pgfsetroundjoin%
\definecolor{currentfill}{rgb}{0.271305,0.019942,0.347269}%
\pgfsetfillcolor{currentfill}%
\pgfsetfillopacity{0.700000}%
\pgfsetlinewidth{0.000000pt}%
\definecolor{currentstroke}{rgb}{0.000000,0.000000,0.000000}%
\pgfsetstrokecolor{currentstroke}%
\pgfsetdash{}{0pt}%
\pgfpathmoveto{\pgfqpoint{3.570265in}{1.606562in}}%
\pgfpathlineto{\pgfqpoint{3.584150in}{1.603861in}}%
\pgfpathlineto{\pgfqpoint{3.598042in}{1.601239in}}%
\pgfpathlineto{\pgfqpoint{3.611941in}{1.598693in}}%
\pgfpathlineto{\pgfqpoint{3.625847in}{1.596225in}}%
\pgfpathlineto{\pgfqpoint{3.634104in}{1.605801in}}%
\pgfpathlineto{\pgfqpoint{3.642355in}{1.615406in}}%
\pgfpathlineto{\pgfqpoint{3.650599in}{1.625039in}}%
\pgfpathlineto{\pgfqpoint{3.658838in}{1.634695in}}%
\pgfpathlineto{\pgfqpoint{3.644945in}{1.636938in}}%
\pgfpathlineto{\pgfqpoint{3.631060in}{1.639257in}}%
\pgfpathlineto{\pgfqpoint{3.617180in}{1.641654in}}%
\pgfpathlineto{\pgfqpoint{3.603308in}{1.644129in}}%
\pgfpathlineto{\pgfqpoint{3.595056in}{1.634690in}}%
\pgfpathlineto{\pgfqpoint{3.586799in}{1.625281in}}%
\pgfpathlineto{\pgfqpoint{3.578535in}{1.615904in}}%
\pgfpathlineto{\pgfqpoint{3.570265in}{1.606562in}}%
\pgfpathclose%
\pgfusepath{fill}%
\end{pgfscope}%
\begin{pgfscope}%
\pgfpathrectangle{\pgfqpoint{1.150000in}{0.150000in}}{\pgfqpoint{5.700000in}{5.700000in}}%
\pgfusepath{clip}%
\pgfsetbuttcap%
\pgfsetroundjoin%
\definecolor{currentfill}{rgb}{0.283197,0.115680,0.436115}%
\pgfsetfillcolor{currentfill}%
\pgfsetfillopacity{0.700000}%
\pgfsetlinewidth{0.000000pt}%
\definecolor{currentstroke}{rgb}{0.000000,0.000000,0.000000}%
\pgfsetstrokecolor{currentstroke}%
\pgfsetdash{}{0pt}%
\pgfpathmoveto{\pgfqpoint{4.068413in}{1.775221in}}%
\pgfpathlineto{\pgfqpoint{4.082427in}{1.775256in}}%
\pgfpathlineto{\pgfqpoint{4.096451in}{1.775366in}}%
\pgfpathlineto{\pgfqpoint{4.110483in}{1.775549in}}%
\pgfpathlineto{\pgfqpoint{4.124524in}{1.775805in}}%
\pgfpathlineto{\pgfqpoint{4.132605in}{1.786251in}}%
\pgfpathlineto{\pgfqpoint{4.140682in}{1.796648in}}%
\pgfpathlineto{\pgfqpoint{4.148752in}{1.806994in}}%
\pgfpathlineto{\pgfqpoint{4.156817in}{1.817287in}}%
\pgfpathlineto{\pgfqpoint{4.142785in}{1.816908in}}%
\pgfpathlineto{\pgfqpoint{4.128762in}{1.816602in}}%
\pgfpathlineto{\pgfqpoint{4.114747in}{1.816369in}}%
\pgfpathlineto{\pgfqpoint{4.100742in}{1.816210in}}%
\pgfpathlineto{\pgfqpoint{4.092668in}{1.806032in}}%
\pgfpathlineto{\pgfqpoint{4.084588in}{1.795807in}}%
\pgfpathlineto{\pgfqpoint{4.076503in}{1.785536in}}%
\pgfpathlineto{\pgfqpoint{4.068413in}{1.775221in}}%
\pgfpathclose%
\pgfusepath{fill}%
\end{pgfscope}%
\begin{pgfscope}%
\pgfpathrectangle{\pgfqpoint{1.150000in}{0.150000in}}{\pgfqpoint{5.700000in}{5.700000in}}%
\pgfusepath{clip}%
\pgfsetbuttcap%
\pgfsetroundjoin%
\definecolor{currentfill}{rgb}{0.267968,0.223549,0.512008}%
\pgfsetfillcolor{currentfill}%
\pgfsetfillopacity{0.700000}%
\pgfsetlinewidth{0.000000pt}%
\definecolor{currentstroke}{rgb}{0.000000,0.000000,0.000000}%
\pgfsetstrokecolor{currentstroke}%
\pgfsetdash{}{0pt}%
\pgfpathmoveto{\pgfqpoint{2.188485in}{2.053284in}}%
\pgfpathlineto{\pgfqpoint{2.202390in}{2.040481in}}%
\pgfpathlineto{\pgfqpoint{2.216293in}{2.027799in}}%
\pgfpathlineto{\pgfqpoint{2.230193in}{2.015236in}}%
\pgfpathlineto{\pgfqpoint{2.244091in}{2.002792in}}%
\pgfpathlineto{\pgfqpoint{2.253230in}{2.001107in}}%
\pgfpathlineto{\pgfqpoint{2.262347in}{1.999710in}}%
\pgfpathlineto{\pgfqpoint{2.271445in}{1.998596in}}%
\pgfpathlineto{\pgfqpoint{2.280524in}{1.997759in}}%
\pgfpathlineto{\pgfqpoint{2.266667in}{2.009797in}}%
\pgfpathlineto{\pgfqpoint{2.252809in}{2.021955in}}%
\pgfpathlineto{\pgfqpoint{2.238948in}{2.034231in}}%
\pgfpathlineto{\pgfqpoint{2.225085in}{2.046627in}}%
\pgfpathlineto{\pgfqpoint{2.215966in}{2.047862in}}%
\pgfpathlineto{\pgfqpoint{2.206826in}{2.049379in}}%
\pgfpathlineto{\pgfqpoint{2.197666in}{2.051184in}}%
\pgfpathlineto{\pgfqpoint{2.188485in}{2.053284in}}%
\pgfpathclose%
\pgfusepath{fill}%
\end{pgfscope}%
\begin{pgfscope}%
\pgfpathrectangle{\pgfqpoint{1.150000in}{0.150000in}}{\pgfqpoint{5.700000in}{5.700000in}}%
\pgfusepath{clip}%
\pgfsetbuttcap%
\pgfsetroundjoin%
\definecolor{currentfill}{rgb}{0.194100,0.399323,0.555565}%
\pgfsetfillcolor{currentfill}%
\pgfsetfillopacity{0.700000}%
\pgfsetlinewidth{0.000000pt}%
\definecolor{currentstroke}{rgb}{0.000000,0.000000,0.000000}%
\pgfsetstrokecolor{currentstroke}%
\pgfsetdash{}{0pt}%
\pgfpathmoveto{\pgfqpoint{5.450837in}{2.428304in}}%
\pgfpathlineto{\pgfqpoint{5.465389in}{2.432532in}}%
\pgfpathlineto{\pgfqpoint{5.479955in}{2.436830in}}%
\pgfpathlineto{\pgfqpoint{5.494533in}{2.441198in}}%
\pgfpathlineto{\pgfqpoint{5.509125in}{2.445636in}}%
\pgfpathlineto{\pgfqpoint{5.516589in}{2.450291in}}%
\pgfpathlineto{\pgfqpoint{5.524046in}{2.454871in}}%
\pgfpathlineto{\pgfqpoint{5.531493in}{2.459379in}}%
\pgfpathlineto{\pgfqpoint{5.538933in}{2.463819in}}%
\pgfpathlineto{\pgfqpoint{5.524361in}{2.459580in}}%
\pgfpathlineto{\pgfqpoint{5.509802in}{2.455412in}}%
\pgfpathlineto{\pgfqpoint{5.495256in}{2.451313in}}%
\pgfpathlineto{\pgfqpoint{5.480724in}{2.447284in}}%
\pgfpathlineto{\pgfqpoint{5.473264in}{2.442637in}}%
\pgfpathlineto{\pgfqpoint{5.465797in}{2.437927in}}%
\pgfpathlineto{\pgfqpoint{5.458321in}{2.433151in}}%
\pgfpathlineto{\pgfqpoint{5.450837in}{2.428304in}}%
\pgfpathclose%
\pgfusepath{fill}%
\end{pgfscope}%
\begin{pgfscope}%
\pgfpathrectangle{\pgfqpoint{1.150000in}{0.150000in}}{\pgfqpoint{5.700000in}{5.700000in}}%
\pgfusepath{clip}%
\pgfsetbuttcap%
\pgfsetroundjoin%
\definecolor{currentfill}{rgb}{0.282656,0.100196,0.422160}%
\pgfsetfillcolor{currentfill}%
\pgfsetfillopacity{0.700000}%
\pgfsetlinewidth{0.000000pt}%
\definecolor{currentstroke}{rgb}{0.000000,0.000000,0.000000}%
\pgfsetstrokecolor{currentstroke}%
\pgfsetdash{}{0pt}%
\pgfpathmoveto{\pgfqpoint{3.979987in}{1.734723in}}%
\pgfpathlineto{\pgfqpoint{3.993977in}{1.734319in}}%
\pgfpathlineto{\pgfqpoint{4.007975in}{1.733989in}}%
\pgfpathlineto{\pgfqpoint{4.021981in}{1.733733in}}%
\pgfpathlineto{\pgfqpoint{4.035996in}{1.733551in}}%
\pgfpathlineto{\pgfqpoint{4.044108in}{1.744026in}}%
\pgfpathlineto{\pgfqpoint{4.052215in}{1.754464in}}%
\pgfpathlineto{\pgfqpoint{4.060317in}{1.764863in}}%
\pgfpathlineto{\pgfqpoint{4.068413in}{1.775221in}}%
\pgfpathlineto{\pgfqpoint{4.054407in}{1.775259in}}%
\pgfpathlineto{\pgfqpoint{4.040410in}{1.775371in}}%
\pgfpathlineto{\pgfqpoint{4.026421in}{1.775557in}}%
\pgfpathlineto{\pgfqpoint{4.012441in}{1.775818in}}%
\pgfpathlineto{\pgfqpoint{4.004336in}{1.765596in}}%
\pgfpathlineto{\pgfqpoint{3.996225in}{1.755338in}}%
\pgfpathlineto{\pgfqpoint{3.988109in}{1.745047in}}%
\pgfpathlineto{\pgfqpoint{3.979987in}{1.734723in}}%
\pgfpathclose%
\pgfusepath{fill}%
\end{pgfscope}%
\begin{pgfscope}%
\pgfpathrectangle{\pgfqpoint{1.150000in}{0.150000in}}{\pgfqpoint{5.700000in}{5.700000in}}%
\pgfusepath{clip}%
\pgfsetbuttcap%
\pgfsetroundjoin%
\definecolor{currentfill}{rgb}{0.277941,0.056324,0.381191}%
\pgfsetfillcolor{currentfill}%
\pgfsetfillopacity{0.700000}%
\pgfsetlinewidth{0.000000pt}%
\definecolor{currentstroke}{rgb}{0.000000,0.000000,0.000000}%
\pgfsetstrokecolor{currentstroke}%
\pgfsetdash{}{0pt}%
\pgfpathmoveto{\pgfqpoint{2.758578in}{1.681358in}}%
\pgfpathlineto{\pgfqpoint{2.772401in}{1.673151in}}%
\pgfpathlineto{\pgfqpoint{2.786226in}{1.665037in}}%
\pgfpathlineto{\pgfqpoint{2.800053in}{1.657017in}}%
\pgfpathlineto{\pgfqpoint{2.813881in}{1.649089in}}%
\pgfpathlineto{\pgfqpoint{2.822571in}{1.652811in}}%
\pgfpathlineto{\pgfqpoint{2.831247in}{1.656724in}}%
\pgfpathlineto{\pgfqpoint{2.839911in}{1.660824in}}%
\pgfpathlineto{\pgfqpoint{2.848563in}{1.665104in}}%
\pgfpathlineto{\pgfqpoint{2.834762in}{1.672680in}}%
\pgfpathlineto{\pgfqpoint{2.820964in}{1.680349in}}%
\pgfpathlineto{\pgfqpoint{2.807167in}{1.688111in}}%
\pgfpathlineto{\pgfqpoint{2.793373in}{1.695966in}}%
\pgfpathlineto{\pgfqpoint{2.784694in}{1.692029in}}%
\pgfpathlineto{\pgfqpoint{2.776002in}{1.688279in}}%
\pgfpathlineto{\pgfqpoint{2.767297in}{1.684720in}}%
\pgfpathlineto{\pgfqpoint{2.758578in}{1.681358in}}%
\pgfpathclose%
\pgfusepath{fill}%
\end{pgfscope}%
\begin{pgfscope}%
\pgfpathrectangle{\pgfqpoint{1.150000in}{0.150000in}}{\pgfqpoint{5.700000in}{5.700000in}}%
\pgfusepath{clip}%
\pgfsetbuttcap%
\pgfsetroundjoin%
\definecolor{currentfill}{rgb}{0.280894,0.078907,0.402329}%
\pgfsetfillcolor{currentfill}%
\pgfsetfillopacity{0.700000}%
\pgfsetlinewidth{0.000000pt}%
\definecolor{currentstroke}{rgb}{0.000000,0.000000,0.000000}%
\pgfsetstrokecolor{currentstroke}%
\pgfsetdash{}{0pt}%
\pgfpathmoveto{\pgfqpoint{3.891532in}{1.696169in}}%
\pgfpathlineto{\pgfqpoint{3.905498in}{1.695302in}}%
\pgfpathlineto{\pgfqpoint{3.919473in}{1.694510in}}%
\pgfpathlineto{\pgfqpoint{3.933455in}{1.693793in}}%
\pgfpathlineto{\pgfqpoint{3.947446in}{1.693150in}}%
\pgfpathlineto{\pgfqpoint{3.955590in}{1.703581in}}%
\pgfpathlineto{\pgfqpoint{3.963728in}{1.713988in}}%
\pgfpathlineto{\pgfqpoint{3.971860in}{1.724369in}}%
\pgfpathlineto{\pgfqpoint{3.979987in}{1.734723in}}%
\pgfpathlineto{\pgfqpoint{3.966006in}{1.735202in}}%
\pgfpathlineto{\pgfqpoint{3.952033in}{1.735755in}}%
\pgfpathlineto{\pgfqpoint{3.938069in}{1.736382in}}%
\pgfpathlineto{\pgfqpoint{3.924112in}{1.737084in}}%
\pgfpathlineto{\pgfqpoint{3.915975in}{1.726888in}}%
\pgfpathlineto{\pgfqpoint{3.907833in}{1.716668in}}%
\pgfpathlineto{\pgfqpoint{3.899685in}{1.706427in}}%
\pgfpathlineto{\pgfqpoint{3.891532in}{1.696169in}}%
\pgfpathclose%
\pgfusepath{fill}%
\end{pgfscope}%
\begin{pgfscope}%
\pgfpathrectangle{\pgfqpoint{1.150000in}{0.150000in}}{\pgfqpoint{5.700000in}{5.700000in}}%
\pgfusepath{clip}%
\pgfsetbuttcap%
\pgfsetroundjoin%
\definecolor{currentfill}{rgb}{0.199430,0.387607,0.554642}%
\pgfsetfillcolor{currentfill}%
\pgfsetfillopacity{0.700000}%
\pgfsetlinewidth{0.000000pt}%
\definecolor{currentstroke}{rgb}{0.000000,0.000000,0.000000}%
\pgfsetstrokecolor{currentstroke}%
\pgfsetdash{}{0pt}%
\pgfpathmoveto{\pgfqpoint{5.362664in}{2.391221in}}%
\pgfpathlineto{\pgfqpoint{5.377183in}{2.395346in}}%
\pgfpathlineto{\pgfqpoint{5.391714in}{2.399541in}}%
\pgfpathlineto{\pgfqpoint{5.406259in}{2.403806in}}%
\pgfpathlineto{\pgfqpoint{5.420816in}{2.408142in}}%
\pgfpathlineto{\pgfqpoint{5.428334in}{2.413305in}}%
\pgfpathlineto{\pgfqpoint{5.435844in}{2.418384in}}%
\pgfpathlineto{\pgfqpoint{5.443345in}{2.423383in}}%
\pgfpathlineto{\pgfqpoint{5.450837in}{2.428304in}}%
\pgfpathlineto{\pgfqpoint{5.436298in}{2.424146in}}%
\pgfpathlineto{\pgfqpoint{5.421772in}{2.420058in}}%
\pgfpathlineto{\pgfqpoint{5.407258in}{2.416040in}}%
\pgfpathlineto{\pgfqpoint{5.392757in}{2.412092in}}%
\pgfpathlineto{\pgfqpoint{5.385246in}{2.406986in}}%
\pgfpathlineto{\pgfqpoint{5.377727in}{2.401807in}}%
\pgfpathlineto{\pgfqpoint{5.370200in}{2.396553in}}%
\pgfpathlineto{\pgfqpoint{5.362664in}{2.391221in}}%
\pgfpathclose%
\pgfusepath{fill}%
\end{pgfscope}%
\begin{pgfscope}%
\pgfpathrectangle{\pgfqpoint{1.150000in}{0.150000in}}{\pgfqpoint{5.700000in}{5.700000in}}%
\pgfusepath{clip}%
\pgfsetbuttcap%
\pgfsetroundjoin%
\definecolor{currentfill}{rgb}{0.269944,0.014625,0.341379}%
\pgfsetfillcolor{currentfill}%
\pgfsetfillopacity{0.700000}%
\pgfsetlinewidth{0.000000pt}%
\definecolor{currentstroke}{rgb}{0.000000,0.000000,0.000000}%
\pgfsetstrokecolor{currentstroke}%
\pgfsetdash{}{0pt}%
\pgfpathmoveto{\pgfqpoint{3.481582in}{1.582179in}}%
\pgfpathlineto{\pgfqpoint{3.495457in}{1.578921in}}%
\pgfpathlineto{\pgfqpoint{3.509338in}{1.575741in}}%
\pgfpathlineto{\pgfqpoint{3.523225in}{1.572639in}}%
\pgfpathlineto{\pgfqpoint{3.537119in}{1.569615in}}%
\pgfpathlineto{\pgfqpoint{3.545415in}{1.578781in}}%
\pgfpathlineto{\pgfqpoint{3.553705in}{1.587997in}}%
\pgfpathlineto{\pgfqpoint{3.561988in}{1.597258in}}%
\pgfpathlineto{\pgfqpoint{3.570265in}{1.606562in}}%
\pgfpathlineto{\pgfqpoint{3.556385in}{1.609339in}}%
\pgfpathlineto{\pgfqpoint{3.542512in}{1.612195in}}%
\pgfpathlineto{\pgfqpoint{3.528646in}{1.615129in}}%
\pgfpathlineto{\pgfqpoint{3.514785in}{1.618142in}}%
\pgfpathlineto{\pgfqpoint{3.506494in}{1.609077in}}%
\pgfpathlineto{\pgfqpoint{3.498197in}{1.600059in}}%
\pgfpathlineto{\pgfqpoint{3.489893in}{1.591092in}}%
\pgfpathlineto{\pgfqpoint{3.481582in}{1.582179in}}%
\pgfpathclose%
\pgfusepath{fill}%
\end{pgfscope}%
\begin{pgfscope}%
\pgfpathrectangle{\pgfqpoint{1.150000in}{0.150000in}}{\pgfqpoint{5.700000in}{5.700000in}}%
\pgfusepath{clip}%
\pgfsetbuttcap%
\pgfsetroundjoin%
\definecolor{currentfill}{rgb}{0.267004,0.004874,0.329415}%
\pgfsetfillcolor{currentfill}%
\pgfsetfillopacity{0.700000}%
\pgfsetlinewidth{0.000000pt}%
\definecolor{currentstroke}{rgb}{0.000000,0.000000,0.000000}%
\pgfsetstrokecolor{currentstroke}%
\pgfsetdash{}{0pt}%
\pgfpathmoveto{\pgfqpoint{3.103848in}{1.580160in}}%
\pgfpathlineto{\pgfqpoint{3.117678in}{1.574412in}}%
\pgfpathlineto{\pgfqpoint{3.131512in}{1.568749in}}%
\pgfpathlineto{\pgfqpoint{3.145349in}{1.563170in}}%
\pgfpathlineto{\pgfqpoint{3.159191in}{1.557675in}}%
\pgfpathlineto{\pgfqpoint{3.167669in}{1.564364in}}%
\pgfpathlineto{\pgfqpoint{3.176137in}{1.571178in}}%
\pgfpathlineto{\pgfqpoint{3.184597in}{1.578112in}}%
\pgfpathlineto{\pgfqpoint{3.193047in}{1.585161in}}%
\pgfpathlineto{\pgfqpoint{3.179226in}{1.590349in}}%
\pgfpathlineto{\pgfqpoint{3.165409in}{1.595620in}}%
\pgfpathlineto{\pgfqpoint{3.151596in}{1.600975in}}%
\pgfpathlineto{\pgfqpoint{3.137788in}{1.606414in}}%
\pgfpathlineto{\pgfqpoint{3.129317in}{1.599665in}}%
\pgfpathlineto{\pgfqpoint{3.120837in}{1.593036in}}%
\pgfpathlineto{\pgfqpoint{3.112348in}{1.586533in}}%
\pgfpathlineto{\pgfqpoint{3.103848in}{1.580160in}}%
\pgfpathclose%
\pgfusepath{fill}%
\end{pgfscope}%
\begin{pgfscope}%
\pgfpathrectangle{\pgfqpoint{1.150000in}{0.150000in}}{\pgfqpoint{5.700000in}{5.700000in}}%
\pgfusepath{clip}%
\pgfsetbuttcap%
\pgfsetroundjoin%
\definecolor{currentfill}{rgb}{0.273006,0.204520,0.501721}%
\pgfsetfillcolor{currentfill}%
\pgfsetfillopacity{0.700000}%
\pgfsetlinewidth{0.000000pt}%
\definecolor{currentstroke}{rgb}{0.000000,0.000000,0.000000}%
\pgfsetstrokecolor{currentstroke}%
\pgfsetdash{}{0pt}%
\pgfpathmoveto{\pgfqpoint{2.244091in}{2.002792in}}%
\pgfpathlineto{\pgfqpoint{2.257987in}{1.990466in}}%
\pgfpathlineto{\pgfqpoint{2.271881in}{1.978257in}}%
\pgfpathlineto{\pgfqpoint{2.285773in}{1.966163in}}%
\pgfpathlineto{\pgfqpoint{2.299663in}{1.954184in}}%
\pgfpathlineto{\pgfqpoint{2.308760in}{1.952912in}}%
\pgfpathlineto{\pgfqpoint{2.317836in}{1.951922in}}%
\pgfpathlineto{\pgfqpoint{2.326894in}{1.951209in}}%
\pgfpathlineto{\pgfqpoint{2.335932in}{1.950767in}}%
\pgfpathlineto{\pgfqpoint{2.322082in}{1.962342in}}%
\pgfpathlineto{\pgfqpoint{2.308231in}{1.974032in}}%
\pgfpathlineto{\pgfqpoint{2.294378in}{1.985837in}}%
\pgfpathlineto{\pgfqpoint{2.280524in}{1.997759in}}%
\pgfpathlineto{\pgfqpoint{2.271445in}{1.998596in}}%
\pgfpathlineto{\pgfqpoint{2.262347in}{1.999710in}}%
\pgfpathlineto{\pgfqpoint{2.253230in}{2.001107in}}%
\pgfpathlineto{\pgfqpoint{2.244091in}{2.002792in}}%
\pgfpathclose%
\pgfusepath{fill}%
\end{pgfscope}%
\begin{pgfscope}%
\pgfpathrectangle{\pgfqpoint{1.150000in}{0.150000in}}{\pgfqpoint{5.700000in}{5.700000in}}%
\pgfusepath{clip}%
\pgfsetbuttcap%
\pgfsetroundjoin%
\definecolor{currentfill}{rgb}{0.282910,0.105393,0.426902}%
\pgfsetfillcolor{currentfill}%
\pgfsetfillopacity{0.700000}%
\pgfsetlinewidth{0.000000pt}%
\definecolor{currentstroke}{rgb}{0.000000,0.000000,0.000000}%
\pgfsetstrokecolor{currentstroke}%
\pgfsetdash{}{0pt}%
\pgfpathmoveto{\pgfqpoint{2.557417in}{1.780475in}}%
\pgfpathlineto{\pgfqpoint{2.571258in}{1.770725in}}%
\pgfpathlineto{\pgfqpoint{2.585100in}{1.761076in}}%
\pgfpathlineto{\pgfqpoint{2.598943in}{1.751526in}}%
\pgfpathlineto{\pgfqpoint{2.612786in}{1.742077in}}%
\pgfpathlineto{\pgfqpoint{2.621625in}{1.743830in}}%
\pgfpathlineto{\pgfqpoint{2.630449in}{1.745813in}}%
\pgfpathlineto{\pgfqpoint{2.639258in}{1.748021in}}%
\pgfpathlineto{\pgfqpoint{2.648052in}{1.750448in}}%
\pgfpathlineto{\pgfqpoint{2.634241in}{1.759522in}}%
\pgfpathlineto{\pgfqpoint{2.620432in}{1.768696in}}%
\pgfpathlineto{\pgfqpoint{2.606623in}{1.777969in}}%
\pgfpathlineto{\pgfqpoint{2.592816in}{1.787344in}}%
\pgfpathlineto{\pgfqpoint{2.583989in}{1.785284in}}%
\pgfpathlineto{\pgfqpoint{2.575147in}{1.783449in}}%
\pgfpathlineto{\pgfqpoint{2.566290in}{1.781844in}}%
\pgfpathlineto{\pgfqpoint{2.557417in}{1.780475in}}%
\pgfpathclose%
\pgfusepath{fill}%
\end{pgfscope}%
\begin{pgfscope}%
\pgfpathrectangle{\pgfqpoint{1.150000in}{0.150000in}}{\pgfqpoint{5.700000in}{5.700000in}}%
\pgfusepath{clip}%
\pgfsetbuttcap%
\pgfsetroundjoin%
\definecolor{currentfill}{rgb}{0.278791,0.062145,0.386592}%
\pgfsetfillcolor{currentfill}%
\pgfsetfillopacity{0.700000}%
\pgfsetlinewidth{0.000000pt}%
\definecolor{currentstroke}{rgb}{0.000000,0.000000,0.000000}%
\pgfsetstrokecolor{currentstroke}%
\pgfsetdash{}{0pt}%
\pgfpathmoveto{\pgfqpoint{3.803034in}{1.659953in}}%
\pgfpathlineto{\pgfqpoint{3.816979in}{1.658601in}}%
\pgfpathlineto{\pgfqpoint{3.830933in}{1.657324in}}%
\pgfpathlineto{\pgfqpoint{3.844894in}{1.656123in}}%
\pgfpathlineto{\pgfqpoint{3.858863in}{1.654996in}}%
\pgfpathlineto{\pgfqpoint{3.867038in}{1.665305in}}%
\pgfpathlineto{\pgfqpoint{3.875208in}{1.675605in}}%
\pgfpathlineto{\pgfqpoint{3.883373in}{1.685894in}}%
\pgfpathlineto{\pgfqpoint{3.891532in}{1.696169in}}%
\pgfpathlineto{\pgfqpoint{3.877573in}{1.697110in}}%
\pgfpathlineto{\pgfqpoint{3.863622in}{1.698127in}}%
\pgfpathlineto{\pgfqpoint{3.849680in}{1.699218in}}%
\pgfpathlineto{\pgfqpoint{3.835744in}{1.700385in}}%
\pgfpathlineto{\pgfqpoint{3.827575in}{1.690288in}}%
\pgfpathlineto{\pgfqpoint{3.819400in}{1.680181in}}%
\pgfpathlineto{\pgfqpoint{3.811220in}{1.670069in}}%
\pgfpathlineto{\pgfqpoint{3.803034in}{1.659953in}}%
\pgfpathclose%
\pgfusepath{fill}%
\end{pgfscope}%
\begin{pgfscope}%
\pgfpathrectangle{\pgfqpoint{1.150000in}{0.150000in}}{\pgfqpoint{5.700000in}{5.700000in}}%
\pgfusepath{clip}%
\pgfsetbuttcap%
\pgfsetroundjoin%
\definecolor{currentfill}{rgb}{0.271305,0.019942,0.347269}%
\pgfsetfillcolor{currentfill}%
\pgfsetfillopacity{0.700000}%
\pgfsetlinewidth{0.000000pt}%
\definecolor{currentstroke}{rgb}{0.000000,0.000000,0.000000}%
\pgfsetstrokecolor{currentstroke}%
\pgfsetdash{}{0pt}%
\pgfpathmoveto{\pgfqpoint{2.959060in}{1.607761in}}%
\pgfpathlineto{\pgfqpoint{2.972885in}{1.600994in}}%
\pgfpathlineto{\pgfqpoint{2.986713in}{1.594315in}}%
\pgfpathlineto{\pgfqpoint{3.000545in}{1.587723in}}%
\pgfpathlineto{\pgfqpoint{3.014379in}{1.581219in}}%
\pgfpathlineto{\pgfqpoint{3.022942in}{1.586688in}}%
\pgfpathlineto{\pgfqpoint{3.031495in}{1.592311in}}%
\pgfpathlineto{\pgfqpoint{3.040036in}{1.598084in}}%
\pgfpathlineto{\pgfqpoint{3.048568in}{1.604002in}}%
\pgfpathlineto{\pgfqpoint{3.034757in}{1.610177in}}%
\pgfpathlineto{\pgfqpoint{3.020950in}{1.616439in}}%
\pgfpathlineto{\pgfqpoint{3.007146in}{1.622789in}}%
\pgfpathlineto{\pgfqpoint{2.993345in}{1.629226in}}%
\pgfpathlineto{\pgfqpoint{2.984790in}{1.623630in}}%
\pgfpathlineto{\pgfqpoint{2.976224in}{1.618184in}}%
\pgfpathlineto{\pgfqpoint{2.967648in}{1.612892in}}%
\pgfpathlineto{\pgfqpoint{2.959060in}{1.607761in}}%
\pgfpathclose%
\pgfusepath{fill}%
\end{pgfscope}%
\begin{pgfscope}%
\pgfpathrectangle{\pgfqpoint{1.150000in}{0.150000in}}{\pgfqpoint{5.700000in}{5.700000in}}%
\pgfusepath{clip}%
\pgfsetbuttcap%
\pgfsetroundjoin%
\definecolor{currentfill}{rgb}{0.204903,0.375746,0.553533}%
\pgfsetfillcolor{currentfill}%
\pgfsetfillopacity{0.700000}%
\pgfsetlinewidth{0.000000pt}%
\definecolor{currentstroke}{rgb}{0.000000,0.000000,0.000000}%
\pgfsetstrokecolor{currentstroke}%
\pgfsetdash{}{0pt}%
\pgfpathmoveto{\pgfqpoint{5.274422in}{2.352619in}}%
\pgfpathlineto{\pgfqpoint{5.288907in}{2.356619in}}%
\pgfpathlineto{\pgfqpoint{5.303404in}{2.360689in}}%
\pgfpathlineto{\pgfqpoint{5.317914in}{2.364829in}}%
\pgfpathlineto{\pgfqpoint{5.332437in}{2.369040in}}%
\pgfpathlineto{\pgfqpoint{5.340006in}{2.374718in}}%
\pgfpathlineto{\pgfqpoint{5.347567in}{2.380306in}}%
\pgfpathlineto{\pgfqpoint{5.355120in}{2.385806in}}%
\pgfpathlineto{\pgfqpoint{5.362664in}{2.391221in}}%
\pgfpathlineto{\pgfqpoint{5.348158in}{2.387166in}}%
\pgfpathlineto{\pgfqpoint{5.333665in}{2.383181in}}%
\pgfpathlineto{\pgfqpoint{5.319185in}{2.379267in}}%
\pgfpathlineto{\pgfqpoint{5.304717in}{2.375422in}}%
\pgfpathlineto{\pgfqpoint{5.297155in}{2.369844in}}%
\pgfpathlineto{\pgfqpoint{5.289586in}{2.364186in}}%
\pgfpathlineto{\pgfqpoint{5.282008in}{2.358446in}}%
\pgfpathlineto{\pgfqpoint{5.274422in}{2.352619in}}%
\pgfpathclose%
\pgfusepath{fill}%
\end{pgfscope}%
\begin{pgfscope}%
\pgfpathrectangle{\pgfqpoint{1.150000in}{0.150000in}}{\pgfqpoint{5.700000in}{5.700000in}}%
\pgfusepath{clip}%
\pgfsetbuttcap%
\pgfsetroundjoin%
\definecolor{currentfill}{rgb}{0.267004,0.004874,0.329415}%
\pgfsetfillcolor{currentfill}%
\pgfsetfillopacity{0.700000}%
\pgfsetlinewidth{0.000000pt}%
\definecolor{currentstroke}{rgb}{0.000000,0.000000,0.000000}%
\pgfsetstrokecolor{currentstroke}%
\pgfsetdash{}{0pt}%
\pgfpathmoveto{\pgfqpoint{3.248377in}{1.565243in}}%
\pgfpathlineto{\pgfqpoint{3.262221in}{1.560469in}}%
\pgfpathlineto{\pgfqpoint{3.276070in}{1.555777in}}%
\pgfpathlineto{\pgfqpoint{3.289924in}{1.551167in}}%
\pgfpathlineto{\pgfqpoint{3.303782in}{1.546637in}}%
\pgfpathlineto{\pgfqpoint{3.312186in}{1.554388in}}%
\pgfpathlineto{\pgfqpoint{3.320581in}{1.562234in}}%
\pgfpathlineto{\pgfqpoint{3.328969in}{1.570172in}}%
\pgfpathlineto{\pgfqpoint{3.337349in}{1.578198in}}%
\pgfpathlineto{\pgfqpoint{3.323508in}{1.582441in}}%
\pgfpathlineto{\pgfqpoint{3.309672in}{1.586764in}}%
\pgfpathlineto{\pgfqpoint{3.295842in}{1.591169in}}%
\pgfpathlineto{\pgfqpoint{3.282016in}{1.595656in}}%
\pgfpathlineto{\pgfqpoint{3.273618in}{1.587909in}}%
\pgfpathlineto{\pgfqpoint{3.265213in}{1.580255in}}%
\pgfpathlineto{\pgfqpoint{3.256799in}{1.572698in}}%
\pgfpathlineto{\pgfqpoint{3.248377in}{1.565243in}}%
\pgfpathclose%
\pgfusepath{fill}%
\end{pgfscope}%
\begin{pgfscope}%
\pgfpathrectangle{\pgfqpoint{1.150000in}{0.150000in}}{\pgfqpoint{5.700000in}{5.700000in}}%
\pgfusepath{clip}%
\pgfsetbuttcap%
\pgfsetroundjoin%
\definecolor{currentfill}{rgb}{0.212395,0.359683,0.551710}%
\pgfsetfillcolor{currentfill}%
\pgfsetfillopacity{0.700000}%
\pgfsetlinewidth{0.000000pt}%
\definecolor{currentstroke}{rgb}{0.000000,0.000000,0.000000}%
\pgfsetstrokecolor{currentstroke}%
\pgfsetdash{}{0pt}%
\pgfpathmoveto{\pgfqpoint{5.186121in}{2.312573in}}%
\pgfpathlineto{\pgfqpoint{5.200571in}{2.316424in}}%
\pgfpathlineto{\pgfqpoint{5.215033in}{2.320346in}}%
\pgfpathlineto{\pgfqpoint{5.229508in}{2.324339in}}%
\pgfpathlineto{\pgfqpoint{5.243995in}{2.328402in}}%
\pgfpathlineto{\pgfqpoint{5.251615in}{2.334598in}}%
\pgfpathlineto{\pgfqpoint{5.259226in}{2.340698in}}%
\pgfpathlineto{\pgfqpoint{5.266828in}{2.346705in}}%
\pgfpathlineto{\pgfqpoint{5.274422in}{2.352619in}}%
\pgfpathlineto{\pgfqpoint{5.259951in}{2.348690in}}%
\pgfpathlineto{\pgfqpoint{5.245491in}{2.344831in}}%
\pgfpathlineto{\pgfqpoint{5.231044in}{2.341043in}}%
\pgfpathlineto{\pgfqpoint{5.216610in}{2.337325in}}%
\pgfpathlineto{\pgfqpoint{5.209000in}{2.331268in}}%
\pgfpathlineto{\pgfqpoint{5.201382in}{2.325126in}}%
\pgfpathlineto{\pgfqpoint{5.193755in}{2.318895in}}%
\pgfpathlineto{\pgfqpoint{5.186121in}{2.312573in}}%
\pgfpathclose%
\pgfusepath{fill}%
\end{pgfscope}%
\begin{pgfscope}%
\pgfpathrectangle{\pgfqpoint{1.150000in}{0.150000in}}{\pgfqpoint{5.700000in}{5.700000in}}%
\pgfusepath{clip}%
\pgfsetbuttcap%
\pgfsetroundjoin%
\definecolor{currentfill}{rgb}{0.276022,0.044167,0.370164}%
\pgfsetfillcolor{currentfill}%
\pgfsetfillopacity{0.700000}%
\pgfsetlinewidth{0.000000pt}%
\definecolor{currentstroke}{rgb}{0.000000,0.000000,0.000000}%
\pgfsetstrokecolor{currentstroke}%
\pgfsetdash{}{0pt}%
\pgfpathmoveto{\pgfqpoint{3.714478in}{1.626492in}}%
\pgfpathlineto{\pgfqpoint{3.728405in}{1.624632in}}%
\pgfpathlineto{\pgfqpoint{3.742340in}{1.622848in}}%
\pgfpathlineto{\pgfqpoint{3.756282in}{1.621139in}}%
\pgfpathlineto{\pgfqpoint{3.770232in}{1.619507in}}%
\pgfpathlineto{\pgfqpoint{3.778441in}{1.629610in}}%
\pgfpathlineto{\pgfqpoint{3.786644in}{1.639720in}}%
\pgfpathlineto{\pgfqpoint{3.794842in}{1.649836in}}%
\pgfpathlineto{\pgfqpoint{3.803034in}{1.659953in}}%
\pgfpathlineto{\pgfqpoint{3.789095in}{1.661380in}}%
\pgfpathlineto{\pgfqpoint{3.775164in}{1.662883in}}%
\pgfpathlineto{\pgfqpoint{3.761241in}{1.664462in}}%
\pgfpathlineto{\pgfqpoint{3.747325in}{1.666116in}}%
\pgfpathlineto{\pgfqpoint{3.739122in}{1.656197in}}%
\pgfpathlineto{\pgfqpoint{3.730913in}{1.646284in}}%
\pgfpathlineto{\pgfqpoint{3.722698in}{1.636382in}}%
\pgfpathlineto{\pgfqpoint{3.714478in}{1.626492in}}%
\pgfpathclose%
\pgfusepath{fill}%
\end{pgfscope}%
\begin{pgfscope}%
\pgfpathrectangle{\pgfqpoint{1.150000in}{0.150000in}}{\pgfqpoint{5.700000in}{5.700000in}}%
\pgfusepath{clip}%
\pgfsetbuttcap%
\pgfsetroundjoin%
\definecolor{currentfill}{rgb}{0.277134,0.185228,0.489898}%
\pgfsetfillcolor{currentfill}%
\pgfsetfillopacity{0.700000}%
\pgfsetlinewidth{0.000000pt}%
\definecolor{currentstroke}{rgb}{0.000000,0.000000,0.000000}%
\pgfsetstrokecolor{currentstroke}%
\pgfsetdash{}{0pt}%
\pgfpathmoveto{\pgfqpoint{2.299663in}{1.954184in}}%
\pgfpathlineto{\pgfqpoint{2.313552in}{1.942320in}}%
\pgfpathlineto{\pgfqpoint{2.327439in}{1.930568in}}%
\pgfpathlineto{\pgfqpoint{2.341324in}{1.918929in}}%
\pgfpathlineto{\pgfqpoint{2.355209in}{1.907402in}}%
\pgfpathlineto{\pgfqpoint{2.364264in}{1.906540in}}%
\pgfpathlineto{\pgfqpoint{2.373301in}{1.905956in}}%
\pgfpathlineto{\pgfqpoint{2.382319in}{1.905643in}}%
\pgfpathlineto{\pgfqpoint{2.391319in}{1.905595in}}%
\pgfpathlineto{\pgfqpoint{2.377474in}{1.916720in}}%
\pgfpathlineto{\pgfqpoint{2.363628in}{1.927957in}}%
\pgfpathlineto{\pgfqpoint{2.349781in}{1.939306in}}%
\pgfpathlineto{\pgfqpoint{2.335932in}{1.950767in}}%
\pgfpathlineto{\pgfqpoint{2.326894in}{1.951209in}}%
\pgfpathlineto{\pgfqpoint{2.317836in}{1.951922in}}%
\pgfpathlineto{\pgfqpoint{2.308760in}{1.952912in}}%
\pgfpathlineto{\pgfqpoint{2.299663in}{1.954184in}}%
\pgfpathclose%
\pgfusepath{fill}%
\end{pgfscope}%
\begin{pgfscope}%
\pgfpathrectangle{\pgfqpoint{1.150000in}{0.150000in}}{\pgfqpoint{5.700000in}{5.700000in}}%
\pgfusepath{clip}%
\pgfsetbuttcap%
\pgfsetroundjoin%
\definecolor{currentfill}{rgb}{0.220057,0.343307,0.549413}%
\pgfsetfillcolor{currentfill}%
\pgfsetfillopacity{0.700000}%
\pgfsetlinewidth{0.000000pt}%
\definecolor{currentstroke}{rgb}{0.000000,0.000000,0.000000}%
\pgfsetstrokecolor{currentstroke}%
\pgfsetdash{}{0pt}%
\pgfpathmoveto{\pgfqpoint{5.097769in}{2.271176in}}%
\pgfpathlineto{\pgfqpoint{5.112184in}{2.274857in}}%
\pgfpathlineto{\pgfqpoint{5.126611in}{2.278609in}}%
\pgfpathlineto{\pgfqpoint{5.141050in}{2.282431in}}%
\pgfpathlineto{\pgfqpoint{5.155502in}{2.286324in}}%
\pgfpathlineto{\pgfqpoint{5.163169in}{2.293035in}}%
\pgfpathlineto{\pgfqpoint{5.170828in}{2.299645in}}%
\pgfpathlineto{\pgfqpoint{5.178479in}{2.306157in}}%
\pgfpathlineto{\pgfqpoint{5.186121in}{2.312573in}}%
\pgfpathlineto{\pgfqpoint{5.171684in}{2.308792in}}%
\pgfpathlineto{\pgfqpoint{5.157259in}{2.305081in}}%
\pgfpathlineto{\pgfqpoint{5.142846in}{2.301441in}}%
\pgfpathlineto{\pgfqpoint{5.128445in}{2.297872in}}%
\pgfpathlineto{\pgfqpoint{5.120788in}{2.291336in}}%
\pgfpathlineto{\pgfqpoint{5.113123in}{2.284710in}}%
\pgfpathlineto{\pgfqpoint{5.105450in}{2.277991in}}%
\pgfpathlineto{\pgfqpoint{5.097769in}{2.271176in}}%
\pgfpathclose%
\pgfusepath{fill}%
\end{pgfscope}%
\begin{pgfscope}%
\pgfpathrectangle{\pgfqpoint{1.150000in}{0.150000in}}{\pgfqpoint{5.700000in}{5.700000in}}%
\pgfusepath{clip}%
\pgfsetbuttcap%
\pgfsetroundjoin%
\definecolor{currentfill}{rgb}{0.267004,0.004874,0.329415}%
\pgfsetfillcolor{currentfill}%
\pgfsetfillopacity{0.700000}%
\pgfsetlinewidth{0.000000pt}%
\definecolor{currentstroke}{rgb}{0.000000,0.000000,0.000000}%
\pgfsetstrokecolor{currentstroke}%
\pgfsetdash{}{0pt}%
\pgfpathmoveto{\pgfqpoint{3.392765in}{1.562034in}}%
\pgfpathlineto{\pgfqpoint{3.406632in}{1.558193in}}%
\pgfpathlineto{\pgfqpoint{3.420505in}{1.554431in}}%
\pgfpathlineto{\pgfqpoint{3.434384in}{1.550749in}}%
\pgfpathlineto{\pgfqpoint{3.448269in}{1.547145in}}%
\pgfpathlineto{\pgfqpoint{3.456608in}{1.555803in}}%
\pgfpathlineto{\pgfqpoint{3.464939in}{1.564531in}}%
\pgfpathlineto{\pgfqpoint{3.473264in}{1.573324in}}%
\pgfpathlineto{\pgfqpoint{3.481582in}{1.582179in}}%
\pgfpathlineto{\pgfqpoint{3.467713in}{1.585516in}}%
\pgfpathlineto{\pgfqpoint{3.453850in}{1.588932in}}%
\pgfpathlineto{\pgfqpoint{3.439993in}{1.592428in}}%
\pgfpathlineto{\pgfqpoint{3.426141in}{1.596002in}}%
\pgfpathlineto{\pgfqpoint{3.417808in}{1.587406in}}%
\pgfpathlineto{\pgfqpoint{3.409467in}{1.578877in}}%
\pgfpathlineto{\pgfqpoint{3.401120in}{1.570418in}}%
\pgfpathlineto{\pgfqpoint{3.392765in}{1.562034in}}%
\pgfpathclose%
\pgfusepath{fill}%
\end{pgfscope}%
\begin{pgfscope}%
\pgfpathrectangle{\pgfqpoint{1.150000in}{0.150000in}}{\pgfqpoint{5.700000in}{5.700000in}}%
\pgfusepath{clip}%
\pgfsetbuttcap%
\pgfsetroundjoin%
\definecolor{currentfill}{rgb}{0.276022,0.044167,0.370164}%
\pgfsetfillcolor{currentfill}%
\pgfsetfillopacity{0.700000}%
\pgfsetlinewidth{0.000000pt}%
\definecolor{currentstroke}{rgb}{0.000000,0.000000,0.000000}%
\pgfsetstrokecolor{currentstroke}%
\pgfsetdash{}{0pt}%
\pgfpathmoveto{\pgfqpoint{2.813881in}{1.649089in}}%
\pgfpathlineto{\pgfqpoint{2.827712in}{1.641253in}}%
\pgfpathlineto{\pgfqpoint{2.841545in}{1.633508in}}%
\pgfpathlineto{\pgfqpoint{2.855380in}{1.625855in}}%
\pgfpathlineto{\pgfqpoint{2.869218in}{1.618292in}}%
\pgfpathlineto{\pgfqpoint{2.877879in}{1.622373in}}%
\pgfpathlineto{\pgfqpoint{2.886528in}{1.626641in}}%
\pgfpathlineto{\pgfqpoint{2.895165in}{1.631089in}}%
\pgfpathlineto{\pgfqpoint{2.903790in}{1.635713in}}%
\pgfpathlineto{\pgfqpoint{2.889979in}{1.642925in}}%
\pgfpathlineto{\pgfqpoint{2.876171in}{1.650227in}}%
\pgfpathlineto{\pgfqpoint{2.862366in}{1.657620in}}%
\pgfpathlineto{\pgfqpoint{2.848563in}{1.665104in}}%
\pgfpathlineto{\pgfqpoint{2.839911in}{1.660824in}}%
\pgfpathlineto{\pgfqpoint{2.831247in}{1.656724in}}%
\pgfpathlineto{\pgfqpoint{2.822571in}{1.652811in}}%
\pgfpathlineto{\pgfqpoint{2.813881in}{1.649089in}}%
\pgfpathclose%
\pgfusepath{fill}%
\end{pgfscope}%
\begin{pgfscope}%
\pgfpathrectangle{\pgfqpoint{1.150000in}{0.150000in}}{\pgfqpoint{5.700000in}{5.700000in}}%
\pgfusepath{clip}%
\pgfsetbuttcap%
\pgfsetroundjoin%
\definecolor{currentfill}{rgb}{0.227802,0.326594,0.546532}%
\pgfsetfillcolor{currentfill}%
\pgfsetfillopacity{0.700000}%
\pgfsetlinewidth{0.000000pt}%
\definecolor{currentstroke}{rgb}{0.000000,0.000000,0.000000}%
\pgfsetstrokecolor{currentstroke}%
\pgfsetdash{}{0pt}%
\pgfpathmoveto{\pgfqpoint{5.009375in}{2.228546in}}%
\pgfpathlineto{\pgfqpoint{5.023754in}{2.232034in}}%
\pgfpathlineto{\pgfqpoint{5.038146in}{2.235593in}}%
\pgfpathlineto{\pgfqpoint{5.052549in}{2.239222in}}%
\pgfpathlineto{\pgfqpoint{5.066965in}{2.242923in}}%
\pgfpathlineto{\pgfqpoint{5.074678in}{2.250139in}}%
\pgfpathlineto{\pgfqpoint{5.082383in}{2.257252in}}%
\pgfpathlineto{\pgfqpoint{5.090080in}{2.264264in}}%
\pgfpathlineto{\pgfqpoint{5.097769in}{2.271176in}}%
\pgfpathlineto{\pgfqpoint{5.083366in}{2.267566in}}%
\pgfpathlineto{\pgfqpoint{5.068976in}{2.264026in}}%
\pgfpathlineto{\pgfqpoint{5.054598in}{2.260557in}}%
\pgfpathlineto{\pgfqpoint{5.040231in}{2.257159in}}%
\pgfpathlineto{\pgfqpoint{5.032529in}{2.250149in}}%
\pgfpathlineto{\pgfqpoint{5.024819in}{2.243045in}}%
\pgfpathlineto{\pgfqpoint{5.017101in}{2.235844in}}%
\pgfpathlineto{\pgfqpoint{5.009375in}{2.228546in}}%
\pgfpathclose%
\pgfusepath{fill}%
\end{pgfscope}%
\begin{pgfscope}%
\pgfpathrectangle{\pgfqpoint{1.150000in}{0.150000in}}{\pgfqpoint{5.700000in}{5.700000in}}%
\pgfusepath{clip}%
\pgfsetbuttcap%
\pgfsetroundjoin%
\definecolor{currentfill}{rgb}{0.281924,0.089666,0.412415}%
\pgfsetfillcolor{currentfill}%
\pgfsetfillopacity{0.700000}%
\pgfsetlinewidth{0.000000pt}%
\definecolor{currentstroke}{rgb}{0.000000,0.000000,0.000000}%
\pgfsetstrokecolor{currentstroke}%
\pgfsetdash{}{0pt}%
\pgfpathmoveto{\pgfqpoint{2.612786in}{1.742077in}}%
\pgfpathlineto{\pgfqpoint{2.626630in}{1.732726in}}%
\pgfpathlineto{\pgfqpoint{2.640474in}{1.723474in}}%
\pgfpathlineto{\pgfqpoint{2.654320in}{1.714319in}}%
\pgfpathlineto{\pgfqpoint{2.668167in}{1.705261in}}%
\pgfpathlineto{\pgfqpoint{2.676973in}{1.707397in}}%
\pgfpathlineto{\pgfqpoint{2.685764in}{1.709758in}}%
\pgfpathlineto{\pgfqpoint{2.694541in}{1.712339in}}%
\pgfpathlineto{\pgfqpoint{2.703304in}{1.715132in}}%
\pgfpathlineto{\pgfqpoint{2.689489in}{1.723815in}}%
\pgfpathlineto{\pgfqpoint{2.675675in}{1.732595in}}%
\pgfpathlineto{\pgfqpoint{2.661863in}{1.741473in}}%
\pgfpathlineto{\pgfqpoint{2.648052in}{1.750448in}}%
\pgfpathlineto{\pgfqpoint{2.639258in}{1.748021in}}%
\pgfpathlineto{\pgfqpoint{2.630449in}{1.745813in}}%
\pgfpathlineto{\pgfqpoint{2.621625in}{1.743830in}}%
\pgfpathlineto{\pgfqpoint{2.612786in}{1.742077in}}%
\pgfpathclose%
\pgfusepath{fill}%
\end{pgfscope}%
\begin{pgfscope}%
\pgfpathrectangle{\pgfqpoint{1.150000in}{0.150000in}}{\pgfqpoint{5.700000in}{5.700000in}}%
\pgfusepath{clip}%
\pgfsetbuttcap%
\pgfsetroundjoin%
\definecolor{currentfill}{rgb}{0.235526,0.309527,0.542944}%
\pgfsetfillcolor{currentfill}%
\pgfsetfillopacity{0.700000}%
\pgfsetlinewidth{0.000000pt}%
\definecolor{currentstroke}{rgb}{0.000000,0.000000,0.000000}%
\pgfsetstrokecolor{currentstroke}%
\pgfsetdash{}{0pt}%
\pgfpathmoveto{\pgfqpoint{4.920947in}{2.184823in}}%
\pgfpathlineto{\pgfqpoint{4.935291in}{2.188095in}}%
\pgfpathlineto{\pgfqpoint{4.949647in}{2.191438in}}%
\pgfpathlineto{\pgfqpoint{4.964015in}{2.194853in}}%
\pgfpathlineto{\pgfqpoint{4.978394in}{2.198338in}}%
\pgfpathlineto{\pgfqpoint{4.986151in}{2.206046in}}%
\pgfpathlineto{\pgfqpoint{4.993900in}{2.213648in}}%
\pgfpathlineto{\pgfqpoint{5.001641in}{2.221148in}}%
\pgfpathlineto{\pgfqpoint{5.009375in}{2.228546in}}%
\pgfpathlineto{\pgfqpoint{4.995007in}{2.225129in}}%
\pgfpathlineto{\pgfqpoint{4.980652in}{2.221783in}}%
\pgfpathlineto{\pgfqpoint{4.966308in}{2.218507in}}%
\pgfpathlineto{\pgfqpoint{4.951976in}{2.215303in}}%
\pgfpathlineto{\pgfqpoint{4.944230in}{2.207829in}}%
\pgfpathlineto{\pgfqpoint{4.936477in}{2.200258in}}%
\pgfpathlineto{\pgfqpoint{4.928716in}{2.192590in}}%
\pgfpathlineto{\pgfqpoint{4.920947in}{2.184823in}}%
\pgfpathclose%
\pgfusepath{fill}%
\end{pgfscope}%
\begin{pgfscope}%
\pgfpathrectangle{\pgfqpoint{1.150000in}{0.150000in}}{\pgfqpoint{5.700000in}{5.700000in}}%
\pgfusepath{clip}%
\pgfsetbuttcap%
\pgfsetroundjoin%
\definecolor{currentfill}{rgb}{0.243113,0.292092,0.538516}%
\pgfsetfillcolor{currentfill}%
\pgfsetfillopacity{0.700000}%
\pgfsetlinewidth{0.000000pt}%
\definecolor{currentstroke}{rgb}{0.000000,0.000000,0.000000}%
\pgfsetstrokecolor{currentstroke}%
\pgfsetdash{}{0pt}%
\pgfpathmoveto{\pgfqpoint{4.832494in}{2.140167in}}%
\pgfpathlineto{\pgfqpoint{4.846803in}{2.143201in}}%
\pgfpathlineto{\pgfqpoint{4.861123in}{2.146307in}}%
\pgfpathlineto{\pgfqpoint{4.875455in}{2.149484in}}%
\pgfpathlineto{\pgfqpoint{4.889798in}{2.152732in}}%
\pgfpathlineto{\pgfqpoint{4.897597in}{2.160910in}}%
\pgfpathlineto{\pgfqpoint{4.905388in}{2.168984in}}%
\pgfpathlineto{\pgfqpoint{4.913171in}{2.176954in}}%
\pgfpathlineto{\pgfqpoint{4.920947in}{2.184823in}}%
\pgfpathlineto{\pgfqpoint{4.906615in}{2.181621in}}%
\pgfpathlineto{\pgfqpoint{4.892294in}{2.178491in}}%
\pgfpathlineto{\pgfqpoint{4.877985in}{2.175431in}}%
\pgfpathlineto{\pgfqpoint{4.863688in}{2.172443in}}%
\pgfpathlineto{\pgfqpoint{4.855900in}{2.164521in}}%
\pgfpathlineto{\pgfqpoint{4.848105in}{2.156501in}}%
\pgfpathlineto{\pgfqpoint{4.840303in}{2.148383in}}%
\pgfpathlineto{\pgfqpoint{4.832494in}{2.140167in}}%
\pgfpathclose%
\pgfusepath{fill}%
\end{pgfscope}%
\begin{pgfscope}%
\pgfpathrectangle{\pgfqpoint{1.150000in}{0.150000in}}{\pgfqpoint{5.700000in}{5.700000in}}%
\pgfusepath{clip}%
\pgfsetbuttcap%
\pgfsetroundjoin%
\definecolor{currentfill}{rgb}{0.272594,0.025563,0.353093}%
\pgfsetfillcolor{currentfill}%
\pgfsetfillopacity{0.700000}%
\pgfsetlinewidth{0.000000pt}%
\definecolor{currentstroke}{rgb}{0.000000,0.000000,0.000000}%
\pgfsetstrokecolor{currentstroke}%
\pgfsetdash{}{0pt}%
\pgfpathmoveto{\pgfqpoint{3.625847in}{1.596225in}}%
\pgfpathlineto{\pgfqpoint{3.639759in}{1.593834in}}%
\pgfpathlineto{\pgfqpoint{3.653678in}{1.591519in}}%
\pgfpathlineto{\pgfqpoint{3.667603in}{1.589281in}}%
\pgfpathlineto{\pgfqpoint{3.681536in}{1.587119in}}%
\pgfpathlineto{\pgfqpoint{3.689781in}{1.596928in}}%
\pgfpathlineto{\pgfqpoint{3.698019in}{1.606762in}}%
\pgfpathlineto{\pgfqpoint{3.706251in}{1.616617in}}%
\pgfpathlineto{\pgfqpoint{3.714478in}{1.626492in}}%
\pgfpathlineto{\pgfqpoint{3.700557in}{1.628428in}}%
\pgfpathlineto{\pgfqpoint{3.686644in}{1.630440in}}%
\pgfpathlineto{\pgfqpoint{3.672738in}{1.632529in}}%
\pgfpathlineto{\pgfqpoint{3.658838in}{1.634695in}}%
\pgfpathlineto{\pgfqpoint{3.650599in}{1.625039in}}%
\pgfpathlineto{\pgfqpoint{3.642355in}{1.615406in}}%
\pgfpathlineto{\pgfqpoint{3.634104in}{1.605801in}}%
\pgfpathlineto{\pgfqpoint{3.625847in}{1.596225in}}%
\pgfpathclose%
\pgfusepath{fill}%
\end{pgfscope}%
\begin{pgfscope}%
\pgfpathrectangle{\pgfqpoint{1.150000in}{0.150000in}}{\pgfqpoint{5.700000in}{5.700000in}}%
\pgfusepath{clip}%
\pgfsetbuttcap%
\pgfsetroundjoin%
\definecolor{currentfill}{rgb}{0.250425,0.274290,0.533103}%
\pgfsetfillcolor{currentfill}%
\pgfsetfillopacity{0.700000}%
\pgfsetlinewidth{0.000000pt}%
\definecolor{currentstroke}{rgb}{0.000000,0.000000,0.000000}%
\pgfsetstrokecolor{currentstroke}%
\pgfsetdash{}{0pt}%
\pgfpathmoveto{\pgfqpoint{4.744022in}{2.094762in}}%
\pgfpathlineto{\pgfqpoint{4.758295in}{2.097536in}}%
\pgfpathlineto{\pgfqpoint{4.772580in}{2.100381in}}%
\pgfpathlineto{\pgfqpoint{4.786876in}{2.103298in}}%
\pgfpathlineto{\pgfqpoint{4.801184in}{2.106287in}}%
\pgfpathlineto{\pgfqpoint{4.809022in}{2.114911in}}%
\pgfpathlineto{\pgfqpoint{4.816853in}{2.123431in}}%
\pgfpathlineto{\pgfqpoint{4.824677in}{2.131850in}}%
\pgfpathlineto{\pgfqpoint{4.832494in}{2.140167in}}%
\pgfpathlineto{\pgfqpoint{4.818197in}{2.137203in}}%
\pgfpathlineto{\pgfqpoint{4.803911in}{2.134311in}}%
\pgfpathlineto{\pgfqpoint{4.789636in}{2.131490in}}%
\pgfpathlineto{\pgfqpoint{4.775373in}{2.128741in}}%
\pgfpathlineto{\pgfqpoint{4.767546in}{2.120391in}}%
\pgfpathlineto{\pgfqpoint{4.759711in}{2.111945in}}%
\pgfpathlineto{\pgfqpoint{4.751870in}{2.103402in}}%
\pgfpathlineto{\pgfqpoint{4.744022in}{2.094762in}}%
\pgfpathclose%
\pgfusepath{fill}%
\end{pgfscope}%
\begin{pgfscope}%
\pgfpathrectangle{\pgfqpoint{1.150000in}{0.150000in}}{\pgfqpoint{5.700000in}{5.700000in}}%
\pgfusepath{clip}%
\pgfsetbuttcap%
\pgfsetroundjoin%
\definecolor{currentfill}{rgb}{0.257322,0.256130,0.526563}%
\pgfsetfillcolor{currentfill}%
\pgfsetfillopacity{0.700000}%
\pgfsetlinewidth{0.000000pt}%
\definecolor{currentstroke}{rgb}{0.000000,0.000000,0.000000}%
\pgfsetstrokecolor{currentstroke}%
\pgfsetdash{}{0pt}%
\pgfpathmoveto{\pgfqpoint{4.655537in}{2.048812in}}%
\pgfpathlineto{\pgfqpoint{4.669775in}{2.051303in}}%
\pgfpathlineto{\pgfqpoint{4.684025in}{2.053867in}}%
\pgfpathlineto{\pgfqpoint{4.698286in}{2.056501in}}%
\pgfpathlineto{\pgfqpoint{4.712559in}{2.059208in}}%
\pgfpathlineto{\pgfqpoint{4.720435in}{2.068246in}}%
\pgfpathlineto{\pgfqpoint{4.728304in}{2.077184in}}%
\pgfpathlineto{\pgfqpoint{4.736166in}{2.086022in}}%
\pgfpathlineto{\pgfqpoint{4.744022in}{2.094762in}}%
\pgfpathlineto{\pgfqpoint{4.729759in}{2.092059in}}%
\pgfpathlineto{\pgfqpoint{4.715508in}{2.089427in}}%
\pgfpathlineto{\pgfqpoint{4.701268in}{2.086867in}}%
\pgfpathlineto{\pgfqpoint{4.687038in}{2.084379in}}%
\pgfpathlineto{\pgfqpoint{4.679173in}{2.075628in}}%
\pgfpathlineto{\pgfqpoint{4.671301in}{2.066784in}}%
\pgfpathlineto{\pgfqpoint{4.663422in}{2.057845in}}%
\pgfpathlineto{\pgfqpoint{4.655537in}{2.048812in}}%
\pgfpathclose%
\pgfusepath{fill}%
\end{pgfscope}%
\begin{pgfscope}%
\pgfpathrectangle{\pgfqpoint{1.150000in}{0.150000in}}{\pgfqpoint{5.700000in}{5.700000in}}%
\pgfusepath{clip}%
\pgfsetbuttcap%
\pgfsetroundjoin%
\definecolor{currentfill}{rgb}{0.279574,0.170599,0.479997}%
\pgfsetfillcolor{currentfill}%
\pgfsetfillopacity{0.700000}%
\pgfsetlinewidth{0.000000pt}%
\definecolor{currentstroke}{rgb}{0.000000,0.000000,0.000000}%
\pgfsetstrokecolor{currentstroke}%
\pgfsetdash{}{0pt}%
\pgfpathmoveto{\pgfqpoint{2.355209in}{1.907402in}}%
\pgfpathlineto{\pgfqpoint{2.369092in}{1.895985in}}%
\pgfpathlineto{\pgfqpoint{2.382974in}{1.884678in}}%
\pgfpathlineto{\pgfqpoint{2.396855in}{1.873480in}}%
\pgfpathlineto{\pgfqpoint{2.410735in}{1.862390in}}%
\pgfpathlineto{\pgfqpoint{2.419751in}{1.861939in}}%
\pgfpathlineto{\pgfqpoint{2.428749in}{1.861759in}}%
\pgfpathlineto{\pgfqpoint{2.437728in}{1.861844in}}%
\pgfpathlineto{\pgfqpoint{2.446690in}{1.862189in}}%
\pgfpathlineto{\pgfqpoint{2.432848in}{1.872878in}}%
\pgfpathlineto{\pgfqpoint{2.419006in}{1.883674in}}%
\pgfpathlineto{\pgfqpoint{2.405162in}{1.894580in}}%
\pgfpathlineto{\pgfqpoint{2.391319in}{1.905595in}}%
\pgfpathlineto{\pgfqpoint{2.382319in}{1.905643in}}%
\pgfpathlineto{\pgfqpoint{2.373301in}{1.905956in}}%
\pgfpathlineto{\pgfqpoint{2.364264in}{1.906540in}}%
\pgfpathlineto{\pgfqpoint{2.355209in}{1.907402in}}%
\pgfpathclose%
\pgfusepath{fill}%
\end{pgfscope}%
\begin{pgfscope}%
\pgfpathrectangle{\pgfqpoint{1.150000in}{0.150000in}}{\pgfqpoint{5.700000in}{5.700000in}}%
\pgfusepath{clip}%
\pgfsetbuttcap%
\pgfsetroundjoin%
\definecolor{currentfill}{rgb}{0.265145,0.232956,0.516599}%
\pgfsetfillcolor{currentfill}%
\pgfsetfillopacity{0.700000}%
\pgfsetlinewidth{0.000000pt}%
\definecolor{currentstroke}{rgb}{0.000000,0.000000,0.000000}%
\pgfsetstrokecolor{currentstroke}%
\pgfsetdash{}{0pt}%
\pgfpathmoveto{\pgfqpoint{4.567043in}{2.002543in}}%
\pgfpathlineto{\pgfqpoint{4.581248in}{2.004730in}}%
\pgfpathlineto{\pgfqpoint{4.595464in}{2.006988in}}%
\pgfpathlineto{\pgfqpoint{4.609690in}{2.009319in}}%
\pgfpathlineto{\pgfqpoint{4.623927in}{2.011721in}}%
\pgfpathlineto{\pgfqpoint{4.631840in}{2.021137in}}%
\pgfpathlineto{\pgfqpoint{4.639745in}{2.030458in}}%
\pgfpathlineto{\pgfqpoint{4.647644in}{2.039683in}}%
\pgfpathlineto{\pgfqpoint{4.655537in}{2.048812in}}%
\pgfpathlineto{\pgfqpoint{4.641308in}{2.046391in}}%
\pgfpathlineto{\pgfqpoint{4.627091in}{2.044043in}}%
\pgfpathlineto{\pgfqpoint{4.612885in}{2.041766in}}%
\pgfpathlineto{\pgfqpoint{4.598689in}{2.039561in}}%
\pgfpathlineto{\pgfqpoint{4.590788in}{2.030442in}}%
\pgfpathlineto{\pgfqpoint{4.582879in}{2.021233in}}%
\pgfpathlineto{\pgfqpoint{4.574965in}{2.011933in}}%
\pgfpathlineto{\pgfqpoint{4.567043in}{2.002543in}}%
\pgfpathclose%
\pgfusepath{fill}%
\end{pgfscope}%
\begin{pgfscope}%
\pgfpathrectangle{\pgfqpoint{1.150000in}{0.150000in}}{\pgfqpoint{5.700000in}{5.700000in}}%
\pgfusepath{clip}%
\pgfsetbuttcap%
\pgfsetroundjoin%
\definecolor{currentfill}{rgb}{0.270595,0.214069,0.507052}%
\pgfsetfillcolor{currentfill}%
\pgfsetfillopacity{0.700000}%
\pgfsetlinewidth{0.000000pt}%
\definecolor{currentstroke}{rgb}{0.000000,0.000000,0.000000}%
\pgfsetstrokecolor{currentstroke}%
\pgfsetdash{}{0pt}%
\pgfpathmoveto{\pgfqpoint{4.478546in}{1.956202in}}%
\pgfpathlineto{\pgfqpoint{4.492718in}{1.958062in}}%
\pgfpathlineto{\pgfqpoint{4.506900in}{1.959994in}}%
\pgfpathlineto{\pgfqpoint{4.521092in}{1.961998in}}%
\pgfpathlineto{\pgfqpoint{4.535295in}{1.964074in}}%
\pgfpathlineto{\pgfqpoint{4.543242in}{1.973827in}}%
\pgfpathlineto{\pgfqpoint{4.551182in}{1.983489in}}%
\pgfpathlineto{\pgfqpoint{4.559116in}{1.993061in}}%
\pgfpathlineto{\pgfqpoint{4.567043in}{2.002543in}}%
\pgfpathlineto{\pgfqpoint{4.552849in}{2.000428in}}%
\pgfpathlineto{\pgfqpoint{4.538666in}{1.998384in}}%
\pgfpathlineto{\pgfqpoint{4.524492in}{1.996413in}}%
\pgfpathlineto{\pgfqpoint{4.510330in}{1.994514in}}%
\pgfpathlineto{\pgfqpoint{4.502393in}{1.985064in}}%
\pgfpathlineto{\pgfqpoint{4.494450in}{1.975529in}}%
\pgfpathlineto{\pgfqpoint{4.486501in}{1.965908in}}%
\pgfpathlineto{\pgfqpoint{4.478546in}{1.956202in}}%
\pgfpathclose%
\pgfusepath{fill}%
\end{pgfscope}%
\begin{pgfscope}%
\pgfpathrectangle{\pgfqpoint{1.150000in}{0.150000in}}{\pgfqpoint{5.700000in}{5.700000in}}%
\pgfusepath{clip}%
\pgfsetbuttcap%
\pgfsetroundjoin%
\definecolor{currentfill}{rgb}{0.275191,0.194905,0.496005}%
\pgfsetfillcolor{currentfill}%
\pgfsetfillopacity{0.700000}%
\pgfsetlinewidth{0.000000pt}%
\definecolor{currentstroke}{rgb}{0.000000,0.000000,0.000000}%
\pgfsetstrokecolor{currentstroke}%
\pgfsetdash{}{0pt}%
\pgfpathmoveto{\pgfqpoint{4.390046in}{1.910058in}}%
\pgfpathlineto{\pgfqpoint{4.404185in}{1.911569in}}%
\pgfpathlineto{\pgfqpoint{4.418335in}{1.913152in}}%
\pgfpathlineto{\pgfqpoint{4.432494in}{1.914807in}}%
\pgfpathlineto{\pgfqpoint{4.446664in}{1.916534in}}%
\pgfpathlineto{\pgfqpoint{4.454643in}{1.926577in}}%
\pgfpathlineto{\pgfqpoint{4.462617in}{1.936536in}}%
\pgfpathlineto{\pgfqpoint{4.470585in}{1.946412in}}%
\pgfpathlineto{\pgfqpoint{4.478546in}{1.956202in}}%
\pgfpathlineto{\pgfqpoint{4.464385in}{1.954414in}}%
\pgfpathlineto{\pgfqpoint{4.450234in}{1.952699in}}%
\pgfpathlineto{\pgfqpoint{4.436093in}{1.951055in}}%
\pgfpathlineto{\pgfqpoint{4.421962in}{1.949484in}}%
\pgfpathlineto{\pgfqpoint{4.413992in}{1.939746in}}%
\pgfpathlineto{\pgfqpoint{4.406016in}{1.929929in}}%
\pgfpathlineto{\pgfqpoint{4.398034in}{1.920033in}}%
\pgfpathlineto{\pgfqpoint{4.390046in}{1.910058in}}%
\pgfpathclose%
\pgfusepath{fill}%
\end{pgfscope}%
\begin{pgfscope}%
\pgfpathrectangle{\pgfqpoint{1.150000in}{0.150000in}}{\pgfqpoint{5.700000in}{5.700000in}}%
\pgfusepath{clip}%
\pgfsetbuttcap%
\pgfsetroundjoin%
\definecolor{currentfill}{rgb}{0.279574,0.170599,0.479997}%
\pgfsetfillcolor{currentfill}%
\pgfsetfillopacity{0.700000}%
\pgfsetlinewidth{0.000000pt}%
\definecolor{currentstroke}{rgb}{0.000000,0.000000,0.000000}%
\pgfsetstrokecolor{currentstroke}%
\pgfsetdash{}{0pt}%
\pgfpathmoveto{\pgfqpoint{4.301544in}{1.864401in}}%
\pgfpathlineto{\pgfqpoint{4.315652in}{1.865540in}}%
\pgfpathlineto{\pgfqpoint{4.329770in}{1.866752in}}%
\pgfpathlineto{\pgfqpoint{4.343897in}{1.868036in}}%
\pgfpathlineto{\pgfqpoint{4.358035in}{1.869393in}}%
\pgfpathlineto{\pgfqpoint{4.366046in}{1.879673in}}%
\pgfpathlineto{\pgfqpoint{4.374052in}{1.889878in}}%
\pgfpathlineto{\pgfqpoint{4.382052in}{1.900006in}}%
\pgfpathlineto{\pgfqpoint{4.390046in}{1.910058in}}%
\pgfpathlineto{\pgfqpoint{4.375917in}{1.908620in}}%
\pgfpathlineto{\pgfqpoint{4.361797in}{1.907254in}}%
\pgfpathlineto{\pgfqpoint{4.347688in}{1.905961in}}%
\pgfpathlineto{\pgfqpoint{4.333588in}{1.904740in}}%
\pgfpathlineto{\pgfqpoint{4.325586in}{1.894762in}}%
\pgfpathlineto{\pgfqpoint{4.317578in}{1.884712in}}%
\pgfpathlineto{\pgfqpoint{4.309564in}{1.874591in}}%
\pgfpathlineto{\pgfqpoint{4.301544in}{1.864401in}}%
\pgfpathclose%
\pgfusepath{fill}%
\end{pgfscope}%
\begin{pgfscope}%
\pgfpathrectangle{\pgfqpoint{1.150000in}{0.150000in}}{\pgfqpoint{5.700000in}{5.700000in}}%
\pgfusepath{clip}%
\pgfsetbuttcap%
\pgfsetroundjoin%
\definecolor{currentfill}{rgb}{0.267004,0.004874,0.329415}%
\pgfsetfillcolor{currentfill}%
\pgfsetfillopacity{0.700000}%
\pgfsetlinewidth{0.000000pt}%
\definecolor{currentstroke}{rgb}{0.000000,0.000000,0.000000}%
\pgfsetstrokecolor{currentstroke}%
\pgfsetdash{}{0pt}%
\pgfpathmoveto{\pgfqpoint{3.159191in}{1.557675in}}%
\pgfpathlineto{\pgfqpoint{3.173037in}{1.552263in}}%
\pgfpathlineto{\pgfqpoint{3.186888in}{1.546935in}}%
\pgfpathlineto{\pgfqpoint{3.200743in}{1.541689in}}%
\pgfpathlineto{\pgfqpoint{3.214602in}{1.536525in}}%
\pgfpathlineto{\pgfqpoint{3.223059in}{1.543530in}}%
\pgfpathlineto{\pgfqpoint{3.231507in}{1.550655in}}%
\pgfpathlineto{\pgfqpoint{3.239946in}{1.557894in}}%
\pgfpathlineto{\pgfqpoint{3.248377in}{1.565243in}}%
\pgfpathlineto{\pgfqpoint{3.234538in}{1.570099in}}%
\pgfpathlineto{\pgfqpoint{3.220703in}{1.575037in}}%
\pgfpathlineto{\pgfqpoint{3.206873in}{1.580058in}}%
\pgfpathlineto{\pgfqpoint{3.193047in}{1.585161in}}%
\pgfpathlineto{\pgfqpoint{3.184597in}{1.578112in}}%
\pgfpathlineto{\pgfqpoint{3.176137in}{1.571178in}}%
\pgfpathlineto{\pgfqpoint{3.167669in}{1.564364in}}%
\pgfpathlineto{\pgfqpoint{3.159191in}{1.557675in}}%
\pgfpathclose%
\pgfusepath{fill}%
\end{pgfscope}%
\begin{pgfscope}%
\pgfpathrectangle{\pgfqpoint{1.150000in}{0.150000in}}{\pgfqpoint{5.700000in}{5.700000in}}%
\pgfusepath{clip}%
\pgfsetbuttcap%
\pgfsetroundjoin%
\definecolor{currentfill}{rgb}{0.182256,0.426184,0.557120}%
\pgfsetfillcolor{currentfill}%
\pgfsetfillopacity{0.700000}%
\pgfsetlinewidth{0.000000pt}%
\definecolor{currentstroke}{rgb}{0.000000,0.000000,0.000000}%
\pgfsetstrokecolor{currentstroke}%
\pgfsetdash{}{0pt}%
\pgfpathmoveto{\pgfqpoint{5.597352in}{2.481471in}}%
\pgfpathlineto{\pgfqpoint{5.611990in}{2.486059in}}%
\pgfpathlineto{\pgfqpoint{5.626642in}{2.490716in}}%
\pgfpathlineto{\pgfqpoint{5.641307in}{2.495444in}}%
\pgfpathlineto{\pgfqpoint{5.648701in}{2.499445in}}%
\pgfpathlineto{\pgfqpoint{5.656087in}{2.503376in}}%
\pgfpathlineto{\pgfqpoint{5.663464in}{2.507242in}}%
\pgfpathlineto{\pgfqpoint{5.670832in}{2.511045in}}%
\pgfpathlineto{\pgfqpoint{5.656189in}{2.506540in}}%
\pgfpathlineto{\pgfqpoint{5.641559in}{2.502104in}}%
\pgfpathlineto{\pgfqpoint{5.626942in}{2.497738in}}%
\pgfpathlineto{\pgfqpoint{5.619557in}{2.493762in}}%
\pgfpathlineto{\pgfqpoint{5.612164in}{2.489728in}}%
\pgfpathlineto{\pgfqpoint{5.604762in}{2.485632in}}%
\pgfpathlineto{\pgfqpoint{5.597352in}{2.481471in}}%
\pgfpathclose%
\pgfusepath{fill}%
\end{pgfscope}%
\begin{pgfscope}%
\pgfpathrectangle{\pgfqpoint{1.150000in}{0.150000in}}{\pgfqpoint{5.700000in}{5.700000in}}%
\pgfusepath{clip}%
\pgfsetbuttcap%
\pgfsetroundjoin%
\definecolor{currentfill}{rgb}{0.281887,0.150881,0.465405}%
\pgfsetfillcolor{currentfill}%
\pgfsetfillopacity{0.700000}%
\pgfsetlinewidth{0.000000pt}%
\definecolor{currentstroke}{rgb}{0.000000,0.000000,0.000000}%
\pgfsetstrokecolor{currentstroke}%
\pgfsetdash{}{0pt}%
\pgfpathmoveto{\pgfqpoint{4.213038in}{1.819539in}}%
\pgfpathlineto{\pgfqpoint{4.227116in}{1.820285in}}%
\pgfpathlineto{\pgfqpoint{4.241203in}{1.821103in}}%
\pgfpathlineto{\pgfqpoint{4.255301in}{1.821994in}}%
\pgfpathlineto{\pgfqpoint{4.269407in}{1.822958in}}%
\pgfpathlineto{\pgfqpoint{4.277450in}{1.833419in}}%
\pgfpathlineto{\pgfqpoint{4.285487in}{1.843813in}}%
\pgfpathlineto{\pgfqpoint{4.293518in}{1.854141in}}%
\pgfpathlineto{\pgfqpoint{4.301544in}{1.864401in}}%
\pgfpathlineto{\pgfqpoint{4.287445in}{1.863334in}}%
\pgfpathlineto{\pgfqpoint{4.273357in}{1.862340in}}%
\pgfpathlineto{\pgfqpoint{4.259277in}{1.861419in}}%
\pgfpathlineto{\pgfqpoint{4.245208in}{1.860571in}}%
\pgfpathlineto{\pgfqpoint{4.237174in}{1.850406in}}%
\pgfpathlineto{\pgfqpoint{4.229134in}{1.840178in}}%
\pgfpathlineto{\pgfqpoint{4.221089in}{1.829889in}}%
\pgfpathlineto{\pgfqpoint{4.213038in}{1.819539in}}%
\pgfpathclose%
\pgfusepath{fill}%
\end{pgfscope}%
\begin{pgfscope}%
\pgfpathrectangle{\pgfqpoint{1.150000in}{0.150000in}}{\pgfqpoint{5.700000in}{5.700000in}}%
\pgfusepath{clip}%
\pgfsetbuttcap%
\pgfsetroundjoin%
\definecolor{currentfill}{rgb}{0.269944,0.014625,0.341379}%
\pgfsetfillcolor{currentfill}%
\pgfsetfillopacity{0.700000}%
\pgfsetlinewidth{0.000000pt}%
\definecolor{currentstroke}{rgb}{0.000000,0.000000,0.000000}%
\pgfsetstrokecolor{currentstroke}%
\pgfsetdash{}{0pt}%
\pgfpathmoveto{\pgfqpoint{3.014379in}{1.581219in}}%
\pgfpathlineto{\pgfqpoint{3.028217in}{1.574800in}}%
\pgfpathlineto{\pgfqpoint{3.042059in}{1.568468in}}%
\pgfpathlineto{\pgfqpoint{3.055904in}{1.562221in}}%
\pgfpathlineto{\pgfqpoint{3.069753in}{1.556060in}}%
\pgfpathlineto{\pgfqpoint{3.078292in}{1.561866in}}%
\pgfpathlineto{\pgfqpoint{3.086821in}{1.567821in}}%
\pgfpathlineto{\pgfqpoint{3.095340in}{1.573921in}}%
\pgfpathlineto{\pgfqpoint{3.103848in}{1.580160in}}%
\pgfpathlineto{\pgfqpoint{3.090023in}{1.585992in}}%
\pgfpathlineto{\pgfqpoint{3.076201in}{1.591909in}}%
\pgfpathlineto{\pgfqpoint{3.062383in}{1.597913in}}%
\pgfpathlineto{\pgfqpoint{3.048568in}{1.604002in}}%
\pgfpathlineto{\pgfqpoint{3.040036in}{1.598084in}}%
\pgfpathlineto{\pgfqpoint{3.031495in}{1.592311in}}%
\pgfpathlineto{\pgfqpoint{3.022942in}{1.586688in}}%
\pgfpathlineto{\pgfqpoint{3.014379in}{1.581219in}}%
\pgfpathclose%
\pgfusepath{fill}%
\end{pgfscope}%
\begin{pgfscope}%
\pgfpathrectangle{\pgfqpoint{1.150000in}{0.150000in}}{\pgfqpoint{5.700000in}{5.700000in}}%
\pgfusepath{clip}%
\pgfsetbuttcap%
\pgfsetroundjoin%
\definecolor{currentfill}{rgb}{0.283072,0.130895,0.449241}%
\pgfsetfillcolor{currentfill}%
\pgfsetfillopacity{0.700000}%
\pgfsetlinewidth{0.000000pt}%
\definecolor{currentstroke}{rgb}{0.000000,0.000000,0.000000}%
\pgfsetstrokecolor{currentstroke}%
\pgfsetdash{}{0pt}%
\pgfpathmoveto{\pgfqpoint{4.124524in}{1.775805in}}%
\pgfpathlineto{\pgfqpoint{4.138574in}{1.776135in}}%
\pgfpathlineto{\pgfqpoint{4.152633in}{1.776538in}}%
\pgfpathlineto{\pgfqpoint{4.166701in}{1.777014in}}%
\pgfpathlineto{\pgfqpoint{4.180778in}{1.777563in}}%
\pgfpathlineto{\pgfqpoint{4.188851in}{1.788141in}}%
\pgfpathlineto{\pgfqpoint{4.196919in}{1.798664in}}%
\pgfpathlineto{\pgfqpoint{4.204981in}{1.809130in}}%
\pgfpathlineto{\pgfqpoint{4.213038in}{1.819539in}}%
\pgfpathlineto{\pgfqpoint{4.198969in}{1.818866in}}%
\pgfpathlineto{\pgfqpoint{4.184909in}{1.818267in}}%
\pgfpathlineto{\pgfqpoint{4.170859in}{1.817740in}}%
\pgfpathlineto{\pgfqpoint{4.156817in}{1.817287in}}%
\pgfpathlineto{\pgfqpoint{4.148752in}{1.806994in}}%
\pgfpathlineto{\pgfqpoint{4.140682in}{1.796648in}}%
\pgfpathlineto{\pgfqpoint{4.132605in}{1.786251in}}%
\pgfpathlineto{\pgfqpoint{4.124524in}{1.775805in}}%
\pgfpathclose%
\pgfusepath{fill}%
\end{pgfscope}%
\begin{pgfscope}%
\pgfpathrectangle{\pgfqpoint{1.150000in}{0.150000in}}{\pgfqpoint{5.700000in}{5.700000in}}%
\pgfusepath{clip}%
\pgfsetbuttcap%
\pgfsetroundjoin%
\definecolor{currentfill}{rgb}{0.283091,0.110553,0.431554}%
\pgfsetfillcolor{currentfill}%
\pgfsetfillopacity{0.700000}%
\pgfsetlinewidth{0.000000pt}%
\definecolor{currentstroke}{rgb}{0.000000,0.000000,0.000000}%
\pgfsetstrokecolor{currentstroke}%
\pgfsetdash{}{0pt}%
\pgfpathmoveto{\pgfqpoint{4.035996in}{1.733551in}}%
\pgfpathlineto{\pgfqpoint{4.050020in}{1.733442in}}%
\pgfpathlineto{\pgfqpoint{4.064052in}{1.733408in}}%
\pgfpathlineto{\pgfqpoint{4.078093in}{1.733447in}}%
\pgfpathlineto{\pgfqpoint{4.092142in}{1.733559in}}%
\pgfpathlineto{\pgfqpoint{4.100246in}{1.744186in}}%
\pgfpathlineto{\pgfqpoint{4.108344in}{1.754771in}}%
\pgfpathlineto{\pgfqpoint{4.116437in}{1.765311in}}%
\pgfpathlineto{\pgfqpoint{4.124524in}{1.775805in}}%
\pgfpathlineto{\pgfqpoint{4.110483in}{1.775549in}}%
\pgfpathlineto{\pgfqpoint{4.096451in}{1.775366in}}%
\pgfpathlineto{\pgfqpoint{4.082427in}{1.775256in}}%
\pgfpathlineto{\pgfqpoint{4.068413in}{1.775221in}}%
\pgfpathlineto{\pgfqpoint{4.060317in}{1.764863in}}%
\pgfpathlineto{\pgfqpoint{4.052215in}{1.754464in}}%
\pgfpathlineto{\pgfqpoint{4.044108in}{1.744026in}}%
\pgfpathlineto{\pgfqpoint{4.035996in}{1.733551in}}%
\pgfpathclose%
\pgfusepath{fill}%
\end{pgfscope}%
\begin{pgfscope}%
\pgfpathrectangle{\pgfqpoint{1.150000in}{0.150000in}}{\pgfqpoint{5.700000in}{5.700000in}}%
\pgfusepath{clip}%
\pgfsetbuttcap%
\pgfsetroundjoin%
\definecolor{currentfill}{rgb}{0.269944,0.014625,0.341379}%
\pgfsetfillcolor{currentfill}%
\pgfsetfillopacity{0.700000}%
\pgfsetlinewidth{0.000000pt}%
\definecolor{currentstroke}{rgb}{0.000000,0.000000,0.000000}%
\pgfsetstrokecolor{currentstroke}%
\pgfsetdash{}{0pt}%
\pgfpathmoveto{\pgfqpoint{3.537119in}{1.569615in}}%
\pgfpathlineto{\pgfqpoint{3.551018in}{1.566668in}}%
\pgfpathlineto{\pgfqpoint{3.564924in}{1.563800in}}%
\pgfpathlineto{\pgfqpoint{3.578837in}{1.561008in}}%
\pgfpathlineto{\pgfqpoint{3.592755in}{1.558293in}}%
\pgfpathlineto{\pgfqpoint{3.601038in}{1.567714in}}%
\pgfpathlineto{\pgfqpoint{3.609314in}{1.577179in}}%
\pgfpathlineto{\pgfqpoint{3.617583in}{1.586683in}}%
\pgfpathlineto{\pgfqpoint{3.625847in}{1.596225in}}%
\pgfpathlineto{\pgfqpoint{3.611941in}{1.598693in}}%
\pgfpathlineto{\pgfqpoint{3.598042in}{1.601239in}}%
\pgfpathlineto{\pgfqpoint{3.584150in}{1.603861in}}%
\pgfpathlineto{\pgfqpoint{3.570265in}{1.606562in}}%
\pgfpathlineto{\pgfqpoint{3.561988in}{1.597258in}}%
\pgfpathlineto{\pgfqpoint{3.553705in}{1.587997in}}%
\pgfpathlineto{\pgfqpoint{3.545415in}{1.578781in}}%
\pgfpathlineto{\pgfqpoint{3.537119in}{1.569615in}}%
\pgfpathclose%
\pgfusepath{fill}%
\end{pgfscope}%
\begin{pgfscope}%
\pgfpathrectangle{\pgfqpoint{1.150000in}{0.150000in}}{\pgfqpoint{5.700000in}{5.700000in}}%
\pgfusepath{clip}%
\pgfsetbuttcap%
\pgfsetroundjoin%
\definecolor{currentfill}{rgb}{0.267004,0.004874,0.329415}%
\pgfsetfillcolor{currentfill}%
\pgfsetfillopacity{0.700000}%
\pgfsetlinewidth{0.000000pt}%
\definecolor{currentstroke}{rgb}{0.000000,0.000000,0.000000}%
\pgfsetstrokecolor{currentstroke}%
\pgfsetdash{}{0pt}%
\pgfpathmoveto{\pgfqpoint{3.303782in}{1.546637in}}%
\pgfpathlineto{\pgfqpoint{3.317646in}{1.542189in}}%
\pgfpathlineto{\pgfqpoint{3.331515in}{1.537821in}}%
\pgfpathlineto{\pgfqpoint{3.345389in}{1.533533in}}%
\pgfpathlineto{\pgfqpoint{3.359268in}{1.529325in}}%
\pgfpathlineto{\pgfqpoint{3.367654in}{1.537370in}}%
\pgfpathlineto{\pgfqpoint{3.376032in}{1.545506in}}%
\pgfpathlineto{\pgfqpoint{3.384402in}{1.553729in}}%
\pgfpathlineto{\pgfqpoint{3.392765in}{1.562034in}}%
\pgfpathlineto{\pgfqpoint{3.378902in}{1.565955in}}%
\pgfpathlineto{\pgfqpoint{3.365046in}{1.569956in}}%
\pgfpathlineto{\pgfqpoint{3.351194in}{1.574037in}}%
\pgfpathlineto{\pgfqpoint{3.337349in}{1.578198in}}%
\pgfpathlineto{\pgfqpoint{3.328969in}{1.570172in}}%
\pgfpathlineto{\pgfqpoint{3.320581in}{1.562234in}}%
\pgfpathlineto{\pgfqpoint{3.312186in}{1.554388in}}%
\pgfpathlineto{\pgfqpoint{3.303782in}{1.546637in}}%
\pgfpathclose%
\pgfusepath{fill}%
\end{pgfscope}%
\begin{pgfscope}%
\pgfpathrectangle{\pgfqpoint{1.150000in}{0.150000in}}{\pgfqpoint{5.700000in}{5.700000in}}%
\pgfusepath{clip}%
\pgfsetbuttcap%
\pgfsetroundjoin%
\definecolor{currentfill}{rgb}{0.281924,0.089666,0.412415}%
\pgfsetfillcolor{currentfill}%
\pgfsetfillopacity{0.700000}%
\pgfsetlinewidth{0.000000pt}%
\definecolor{currentstroke}{rgb}{0.000000,0.000000,0.000000}%
\pgfsetstrokecolor{currentstroke}%
\pgfsetdash{}{0pt}%
\pgfpathmoveto{\pgfqpoint{3.947446in}{1.693150in}}%
\pgfpathlineto{\pgfqpoint{3.961445in}{1.692581in}}%
\pgfpathlineto{\pgfqpoint{3.975452in}{1.692086in}}%
\pgfpathlineto{\pgfqpoint{3.989468in}{1.691666in}}%
\pgfpathlineto{\pgfqpoint{4.003492in}{1.691319in}}%
\pgfpathlineto{\pgfqpoint{4.011626in}{1.701922in}}%
\pgfpathlineto{\pgfqpoint{4.019755in}{1.712497in}}%
\pgfpathlineto{\pgfqpoint{4.027878in}{1.723040in}}%
\pgfpathlineto{\pgfqpoint{4.035996in}{1.733551in}}%
\pgfpathlineto{\pgfqpoint{4.021981in}{1.733733in}}%
\pgfpathlineto{\pgfqpoint{4.007975in}{1.733989in}}%
\pgfpathlineto{\pgfqpoint{3.993977in}{1.734319in}}%
\pgfpathlineto{\pgfqpoint{3.979987in}{1.734723in}}%
\pgfpathlineto{\pgfqpoint{3.971860in}{1.724369in}}%
\pgfpathlineto{\pgfqpoint{3.963728in}{1.713988in}}%
\pgfpathlineto{\pgfqpoint{3.955590in}{1.703581in}}%
\pgfpathlineto{\pgfqpoint{3.947446in}{1.693150in}}%
\pgfpathclose%
\pgfusepath{fill}%
\end{pgfscope}%
\begin{pgfscope}%
\pgfpathrectangle{\pgfqpoint{1.150000in}{0.150000in}}{\pgfqpoint{5.700000in}{5.700000in}}%
\pgfusepath{clip}%
\pgfsetbuttcap%
\pgfsetroundjoin%
\definecolor{currentfill}{rgb}{0.280894,0.078907,0.402329}%
\pgfsetfillcolor{currentfill}%
\pgfsetfillopacity{0.700000}%
\pgfsetlinewidth{0.000000pt}%
\definecolor{currentstroke}{rgb}{0.000000,0.000000,0.000000}%
\pgfsetstrokecolor{currentstroke}%
\pgfsetdash{}{0pt}%
\pgfpathmoveto{\pgfqpoint{2.668167in}{1.705261in}}%
\pgfpathlineto{\pgfqpoint{2.682015in}{1.696300in}}%
\pgfpathlineto{\pgfqpoint{2.695864in}{1.687435in}}%
\pgfpathlineto{\pgfqpoint{2.709715in}{1.678665in}}%
\pgfpathlineto{\pgfqpoint{2.723567in}{1.669990in}}%
\pgfpathlineto{\pgfqpoint{2.732341in}{1.672508in}}%
\pgfpathlineto{\pgfqpoint{2.741101in}{1.675246in}}%
\pgfpathlineto{\pgfqpoint{2.749846in}{1.678198in}}%
\pgfpathlineto{\pgfqpoint{2.758578in}{1.681358in}}%
\pgfpathlineto{\pgfqpoint{2.744757in}{1.689659in}}%
\pgfpathlineto{\pgfqpoint{2.730938in}{1.698055in}}%
\pgfpathlineto{\pgfqpoint{2.717120in}{1.706546in}}%
\pgfpathlineto{\pgfqpoint{2.703304in}{1.715132in}}%
\pgfpathlineto{\pgfqpoint{2.694541in}{1.712339in}}%
\pgfpathlineto{\pgfqpoint{2.685764in}{1.709758in}}%
\pgfpathlineto{\pgfqpoint{2.676973in}{1.707397in}}%
\pgfpathlineto{\pgfqpoint{2.668167in}{1.705261in}}%
\pgfpathclose%
\pgfusepath{fill}%
\end{pgfscope}%
\begin{pgfscope}%
\pgfpathrectangle{\pgfqpoint{1.150000in}{0.150000in}}{\pgfqpoint{5.700000in}{5.700000in}}%
\pgfusepath{clip}%
\pgfsetbuttcap%
\pgfsetroundjoin%
\definecolor{currentfill}{rgb}{0.281887,0.150881,0.465405}%
\pgfsetfillcolor{currentfill}%
\pgfsetfillopacity{0.700000}%
\pgfsetlinewidth{0.000000pt}%
\definecolor{currentstroke}{rgb}{0.000000,0.000000,0.000000}%
\pgfsetstrokecolor{currentstroke}%
\pgfsetdash{}{0pt}%
\pgfpathmoveto{\pgfqpoint{2.410735in}{1.862390in}}%
\pgfpathlineto{\pgfqpoint{2.424614in}{1.851408in}}%
\pgfpathlineto{\pgfqpoint{2.438493in}{1.840533in}}%
\pgfpathlineto{\pgfqpoint{2.452371in}{1.829763in}}%
\pgfpathlineto{\pgfqpoint{2.466249in}{1.819099in}}%
\pgfpathlineto{\pgfqpoint{2.475226in}{1.819056in}}%
\pgfpathlineto{\pgfqpoint{2.484186in}{1.819279in}}%
\pgfpathlineto{\pgfqpoint{2.493129in}{1.819762in}}%
\pgfpathlineto{\pgfqpoint{2.502054in}{1.820498in}}%
\pgfpathlineto{\pgfqpoint{2.488213in}{1.830762in}}%
\pgfpathlineto{\pgfqpoint{2.474372in}{1.841132in}}%
\pgfpathlineto{\pgfqpoint{2.460531in}{1.851607in}}%
\pgfpathlineto{\pgfqpoint{2.446690in}{1.862189in}}%
\pgfpathlineto{\pgfqpoint{2.437728in}{1.861844in}}%
\pgfpathlineto{\pgfqpoint{2.428749in}{1.861759in}}%
\pgfpathlineto{\pgfqpoint{2.419751in}{1.861939in}}%
\pgfpathlineto{\pgfqpoint{2.410735in}{1.862390in}}%
\pgfpathclose%
\pgfusepath{fill}%
\end{pgfscope}%
\begin{pgfscope}%
\pgfpathrectangle{\pgfqpoint{1.150000in}{0.150000in}}{\pgfqpoint{5.700000in}{5.700000in}}%
\pgfusepath{clip}%
\pgfsetbuttcap%
\pgfsetroundjoin%
\definecolor{currentfill}{rgb}{0.273809,0.031497,0.358853}%
\pgfsetfillcolor{currentfill}%
\pgfsetfillopacity{0.700000}%
\pgfsetlinewidth{0.000000pt}%
\definecolor{currentstroke}{rgb}{0.000000,0.000000,0.000000}%
\pgfsetstrokecolor{currentstroke}%
\pgfsetdash{}{0pt}%
\pgfpathmoveto{\pgfqpoint{2.869218in}{1.618292in}}%
\pgfpathlineto{\pgfqpoint{2.883058in}{1.610819in}}%
\pgfpathlineto{\pgfqpoint{2.896901in}{1.603435in}}%
\pgfpathlineto{\pgfqpoint{2.910747in}{1.596141in}}%
\pgfpathlineto{\pgfqpoint{2.924595in}{1.588936in}}%
\pgfpathlineto{\pgfqpoint{2.933229in}{1.593376in}}%
\pgfpathlineto{\pgfqpoint{2.941851in}{1.597997in}}%
\pgfpathlineto{\pgfqpoint{2.950461in}{1.602794in}}%
\pgfpathlineto{\pgfqpoint{2.959060in}{1.607761in}}%
\pgfpathlineto{\pgfqpoint{2.945238in}{1.614615in}}%
\pgfpathlineto{\pgfqpoint{2.931419in}{1.621559in}}%
\pgfpathlineto{\pgfqpoint{2.917603in}{1.628591in}}%
\pgfpathlineto{\pgfqpoint{2.903790in}{1.635713in}}%
\pgfpathlineto{\pgfqpoint{2.895165in}{1.631089in}}%
\pgfpathlineto{\pgfqpoint{2.886528in}{1.626641in}}%
\pgfpathlineto{\pgfqpoint{2.877879in}{1.622373in}}%
\pgfpathlineto{\pgfqpoint{2.869218in}{1.618292in}}%
\pgfpathclose%
\pgfusepath{fill}%
\end{pgfscope}%
\begin{pgfscope}%
\pgfpathrectangle{\pgfqpoint{1.150000in}{0.150000in}}{\pgfqpoint{5.700000in}{5.700000in}}%
\pgfusepath{clip}%
\pgfsetbuttcap%
\pgfsetroundjoin%
\definecolor{currentfill}{rgb}{0.212395,0.359683,0.551710}%
\pgfsetfillcolor{currentfill}%
\pgfsetfillopacity{0.700000}%
\pgfsetlinewidth{0.000000pt}%
\definecolor{currentstroke}{rgb}{0.000000,0.000000,0.000000}%
\pgfsetstrokecolor{currentstroke}%
\pgfsetdash{}{0pt}%
\pgfpathmoveto{\pgfqpoint{1.871720in}{2.356493in}}%
\pgfpathlineto{\pgfqpoint{1.885758in}{2.340606in}}%
\pgfpathlineto{\pgfqpoint{1.899789in}{2.324865in}}%
\pgfpathlineto{\pgfqpoint{1.913815in}{2.309266in}}%
\pgfpathlineto{\pgfqpoint{1.927835in}{2.293809in}}%
\pgfpathlineto{\pgfqpoint{1.937300in}{2.288691in}}%
\pgfpathlineto{\pgfqpoint{1.946739in}{2.283919in}}%
\pgfpathlineto{\pgfqpoint{1.956153in}{2.279485in}}%
\pgfpathlineto{\pgfqpoint{1.965542in}{2.275384in}}%
\pgfpathlineto{\pgfqpoint{1.951573in}{2.290402in}}%
\pgfpathlineto{\pgfqpoint{1.937598in}{2.305561in}}%
\pgfpathlineto{\pgfqpoint{1.923617in}{2.320862in}}%
\pgfpathlineto{\pgfqpoint{1.909631in}{2.336308in}}%
\pgfpathlineto{\pgfqpoint{1.900192in}{2.340840in}}%
\pgfpathlineto{\pgfqpoint{1.890728in}{2.345710in}}%
\pgfpathlineto{\pgfqpoint{1.881237in}{2.350925in}}%
\pgfpathlineto{\pgfqpoint{1.871720in}{2.356493in}}%
\pgfpathclose%
\pgfusepath{fill}%
\end{pgfscope}%
\begin{pgfscope}%
\pgfpathrectangle{\pgfqpoint{1.150000in}{0.150000in}}{\pgfqpoint{5.700000in}{5.700000in}}%
\pgfusepath{clip}%
\pgfsetbuttcap%
\pgfsetroundjoin%
\definecolor{currentfill}{rgb}{0.280267,0.073417,0.397163}%
\pgfsetfillcolor{currentfill}%
\pgfsetfillopacity{0.700000}%
\pgfsetlinewidth{0.000000pt}%
\definecolor{currentstroke}{rgb}{0.000000,0.000000,0.000000}%
\pgfsetstrokecolor{currentstroke}%
\pgfsetdash{}{0pt}%
\pgfpathmoveto{\pgfqpoint{3.858863in}{1.654996in}}%
\pgfpathlineto{\pgfqpoint{3.872839in}{1.653944in}}%
\pgfpathlineto{\pgfqpoint{3.886824in}{1.652967in}}%
\pgfpathlineto{\pgfqpoint{3.900816in}{1.652065in}}%
\pgfpathlineto{\pgfqpoint{3.914817in}{1.651236in}}%
\pgfpathlineto{\pgfqpoint{3.922982in}{1.661738in}}%
\pgfpathlineto{\pgfqpoint{3.931142in}{1.672226in}}%
\pgfpathlineto{\pgfqpoint{3.939297in}{1.682697in}}%
\pgfpathlineto{\pgfqpoint{3.947446in}{1.693150in}}%
\pgfpathlineto{\pgfqpoint{3.933455in}{1.693793in}}%
\pgfpathlineto{\pgfqpoint{3.919473in}{1.694510in}}%
\pgfpathlineto{\pgfqpoint{3.905498in}{1.695302in}}%
\pgfpathlineto{\pgfqpoint{3.891532in}{1.696169in}}%
\pgfpathlineto{\pgfqpoint{3.883373in}{1.685894in}}%
\pgfpathlineto{\pgfqpoint{3.875208in}{1.675605in}}%
\pgfpathlineto{\pgfqpoint{3.867038in}{1.665305in}}%
\pgfpathlineto{\pgfqpoint{3.858863in}{1.654996in}}%
\pgfpathclose%
\pgfusepath{fill}%
\end{pgfscope}%
\begin{pgfscope}%
\pgfpathrectangle{\pgfqpoint{1.150000in}{0.150000in}}{\pgfqpoint{5.700000in}{5.700000in}}%
\pgfusepath{clip}%
\pgfsetbuttcap%
\pgfsetroundjoin%
\definecolor{currentfill}{rgb}{0.187231,0.414746,0.556547}%
\pgfsetfillcolor{currentfill}%
\pgfsetfillopacity{0.700000}%
\pgfsetlinewidth{0.000000pt}%
\definecolor{currentstroke}{rgb}{0.000000,0.000000,0.000000}%
\pgfsetstrokecolor{currentstroke}%
\pgfsetdash{}{0pt}%
\pgfpathmoveto{\pgfqpoint{5.509125in}{2.445636in}}%
\pgfpathlineto{\pgfqpoint{5.523729in}{2.450144in}}%
\pgfpathlineto{\pgfqpoint{5.538348in}{2.454722in}}%
\pgfpathlineto{\pgfqpoint{5.552979in}{2.459370in}}%
\pgfpathlineto{\pgfqpoint{5.567624in}{2.464089in}}%
\pgfpathlineto{\pgfqpoint{5.575069in}{2.468552in}}%
\pgfpathlineto{\pgfqpoint{5.582506in}{2.472934in}}%
\pgfpathlineto{\pgfqpoint{5.589933in}{2.477239in}}%
\pgfpathlineto{\pgfqpoint{5.597352in}{2.481471in}}%
\pgfpathlineto{\pgfqpoint{5.582727in}{2.476953in}}%
\pgfpathlineto{\pgfqpoint{5.568116in}{2.472505in}}%
\pgfpathlineto{\pgfqpoint{5.553517in}{2.468127in}}%
\pgfpathlineto{\pgfqpoint{5.538933in}{2.463819in}}%
\pgfpathlineto{\pgfqpoint{5.531493in}{2.459379in}}%
\pgfpathlineto{\pgfqpoint{5.524046in}{2.454871in}}%
\pgfpathlineto{\pgfqpoint{5.516589in}{2.450291in}}%
\pgfpathlineto{\pgfqpoint{5.509125in}{2.445636in}}%
\pgfpathclose%
\pgfusepath{fill}%
\end{pgfscope}%
\begin{pgfscope}%
\pgfpathrectangle{\pgfqpoint{1.150000in}{0.150000in}}{\pgfqpoint{5.700000in}{5.700000in}}%
\pgfusepath{clip}%
\pgfsetbuttcap%
\pgfsetroundjoin%
\definecolor{currentfill}{rgb}{0.223925,0.334994,0.548053}%
\pgfsetfillcolor{currentfill}%
\pgfsetfillopacity{0.700000}%
\pgfsetlinewidth{0.000000pt}%
\definecolor{currentstroke}{rgb}{0.000000,0.000000,0.000000}%
\pgfsetstrokecolor{currentstroke}%
\pgfsetdash{}{0pt}%
\pgfpathmoveto{\pgfqpoint{1.927835in}{2.293809in}}%
\pgfpathlineto{\pgfqpoint{1.941850in}{2.278493in}}%
\pgfpathlineto{\pgfqpoint{1.955860in}{2.263315in}}%
\pgfpathlineto{\pgfqpoint{1.969865in}{2.248276in}}%
\pgfpathlineto{\pgfqpoint{1.983865in}{2.233373in}}%
\pgfpathlineto{\pgfqpoint{1.993279in}{2.228701in}}%
\pgfpathlineto{\pgfqpoint{2.002668in}{2.224370in}}%
\pgfpathlineto{\pgfqpoint{2.012033in}{2.220370in}}%
\pgfpathlineto{\pgfqpoint{2.021374in}{2.216697in}}%
\pgfpathlineto{\pgfqpoint{2.007423in}{2.231163in}}%
\pgfpathlineto{\pgfqpoint{1.993467in}{2.245766in}}%
\pgfpathlineto{\pgfqpoint{1.979507in}{2.260505in}}%
\pgfpathlineto{\pgfqpoint{1.965542in}{2.275384in}}%
\pgfpathlineto{\pgfqpoint{1.956153in}{2.279485in}}%
\pgfpathlineto{\pgfqpoint{1.946739in}{2.283919in}}%
\pgfpathlineto{\pgfqpoint{1.937300in}{2.288691in}}%
\pgfpathlineto{\pgfqpoint{1.927835in}{2.293809in}}%
\pgfpathclose%
\pgfusepath{fill}%
\end{pgfscope}%
\begin{pgfscope}%
\pgfpathrectangle{\pgfqpoint{1.150000in}{0.150000in}}{\pgfqpoint{5.700000in}{5.700000in}}%
\pgfusepath{clip}%
\pgfsetbuttcap%
\pgfsetroundjoin%
\definecolor{currentfill}{rgb}{0.233603,0.313828,0.543914}%
\pgfsetfillcolor{currentfill}%
\pgfsetfillopacity{0.700000}%
\pgfsetlinewidth{0.000000pt}%
\definecolor{currentstroke}{rgb}{0.000000,0.000000,0.000000}%
\pgfsetstrokecolor{currentstroke}%
\pgfsetdash{}{0pt}%
\pgfpathmoveto{\pgfqpoint{1.983865in}{2.233373in}}%
\pgfpathlineto{\pgfqpoint{1.997860in}{2.218606in}}%
\pgfpathlineto{\pgfqpoint{2.011851in}{2.203972in}}%
\pgfpathlineto{\pgfqpoint{2.025837in}{2.189472in}}%
\pgfpathlineto{\pgfqpoint{2.039819in}{2.175103in}}%
\pgfpathlineto{\pgfqpoint{2.049184in}{2.170876in}}%
\pgfpathlineto{\pgfqpoint{2.058525in}{2.166982in}}%
\pgfpathlineto{\pgfqpoint{2.067841in}{2.163415in}}%
\pgfpathlineto{\pgfqpoint{2.077135in}{2.160168in}}%
\pgfpathlineto{\pgfqpoint{2.063200in}{2.174102in}}%
\pgfpathlineto{\pgfqpoint{2.049262in}{2.188168in}}%
\pgfpathlineto{\pgfqpoint{2.035320in}{2.202366in}}%
\pgfpathlineto{\pgfqpoint{2.021374in}{2.216697in}}%
\pgfpathlineto{\pgfqpoint{2.012033in}{2.220370in}}%
\pgfpathlineto{\pgfqpoint{2.002668in}{2.224370in}}%
\pgfpathlineto{\pgfqpoint{1.993279in}{2.228701in}}%
\pgfpathlineto{\pgfqpoint{1.983865in}{2.233373in}}%
\pgfpathclose%
\pgfusepath{fill}%
\end{pgfscope}%
\begin{pgfscope}%
\pgfpathrectangle{\pgfqpoint{1.150000in}{0.150000in}}{\pgfqpoint{5.700000in}{5.700000in}}%
\pgfusepath{clip}%
\pgfsetbuttcap%
\pgfsetroundjoin%
\definecolor{currentfill}{rgb}{0.268510,0.009605,0.335427}%
\pgfsetfillcolor{currentfill}%
\pgfsetfillopacity{0.700000}%
\pgfsetlinewidth{0.000000pt}%
\definecolor{currentstroke}{rgb}{0.000000,0.000000,0.000000}%
\pgfsetstrokecolor{currentstroke}%
\pgfsetdash{}{0pt}%
\pgfpathmoveto{\pgfqpoint{3.448269in}{1.547145in}}%
\pgfpathlineto{\pgfqpoint{3.462159in}{1.543620in}}%
\pgfpathlineto{\pgfqpoint{3.476055in}{1.540173in}}%
\pgfpathlineto{\pgfqpoint{3.489958in}{1.536804in}}%
\pgfpathlineto{\pgfqpoint{3.503866in}{1.533514in}}%
\pgfpathlineto{\pgfqpoint{3.512189in}{1.542446in}}%
\pgfpathlineto{\pgfqpoint{3.520506in}{1.551443in}}%
\pgfpathlineto{\pgfqpoint{3.528816in}{1.560501in}}%
\pgfpathlineto{\pgfqpoint{3.537119in}{1.569615in}}%
\pgfpathlineto{\pgfqpoint{3.523225in}{1.572639in}}%
\pgfpathlineto{\pgfqpoint{3.509338in}{1.575741in}}%
\pgfpathlineto{\pgfqpoint{3.495457in}{1.578921in}}%
\pgfpathlineto{\pgfqpoint{3.481582in}{1.582179in}}%
\pgfpathlineto{\pgfqpoint{3.473264in}{1.573324in}}%
\pgfpathlineto{\pgfqpoint{3.464939in}{1.564531in}}%
\pgfpathlineto{\pgfqpoint{3.456608in}{1.555803in}}%
\pgfpathlineto{\pgfqpoint{3.448269in}{1.547145in}}%
\pgfpathclose%
\pgfusepath{fill}%
\end{pgfscope}%
\begin{pgfscope}%
\pgfpathrectangle{\pgfqpoint{1.150000in}{0.150000in}}{\pgfqpoint{5.700000in}{5.700000in}}%
\pgfusepath{clip}%
\pgfsetbuttcap%
\pgfsetroundjoin%
\definecolor{currentfill}{rgb}{0.277018,0.050344,0.375715}%
\pgfsetfillcolor{currentfill}%
\pgfsetfillopacity{0.700000}%
\pgfsetlinewidth{0.000000pt}%
\definecolor{currentstroke}{rgb}{0.000000,0.000000,0.000000}%
\pgfsetstrokecolor{currentstroke}%
\pgfsetdash{}{0pt}%
\pgfpathmoveto{\pgfqpoint{3.770232in}{1.619507in}}%
\pgfpathlineto{\pgfqpoint{3.784188in}{1.617949in}}%
\pgfpathlineto{\pgfqpoint{3.798153in}{1.616467in}}%
\pgfpathlineto{\pgfqpoint{3.812125in}{1.615060in}}%
\pgfpathlineto{\pgfqpoint{3.826104in}{1.613728in}}%
\pgfpathlineto{\pgfqpoint{3.834302in}{1.624044in}}%
\pgfpathlineto{\pgfqpoint{3.842495in}{1.634363in}}%
\pgfpathlineto{\pgfqpoint{3.850681in}{1.644681in}}%
\pgfpathlineto{\pgfqpoint{3.858863in}{1.654996in}}%
\pgfpathlineto{\pgfqpoint{3.844894in}{1.656123in}}%
\pgfpathlineto{\pgfqpoint{3.830933in}{1.657324in}}%
\pgfpathlineto{\pgfqpoint{3.816979in}{1.658601in}}%
\pgfpathlineto{\pgfqpoint{3.803034in}{1.659953in}}%
\pgfpathlineto{\pgfqpoint{3.794842in}{1.649836in}}%
\pgfpathlineto{\pgfqpoint{3.786644in}{1.639720in}}%
\pgfpathlineto{\pgfqpoint{3.778441in}{1.629610in}}%
\pgfpathlineto{\pgfqpoint{3.770232in}{1.619507in}}%
\pgfpathclose%
\pgfusepath{fill}%
\end{pgfscope}%
\begin{pgfscope}%
\pgfpathrectangle{\pgfqpoint{1.150000in}{0.150000in}}{\pgfqpoint{5.700000in}{5.700000in}}%
\pgfusepath{clip}%
\pgfsetbuttcap%
\pgfsetroundjoin%
\definecolor{currentfill}{rgb}{0.192357,0.403199,0.555836}%
\pgfsetfillcolor{currentfill}%
\pgfsetfillopacity{0.700000}%
\pgfsetlinewidth{0.000000pt}%
\definecolor{currentstroke}{rgb}{0.000000,0.000000,0.000000}%
\pgfsetstrokecolor{currentstroke}%
\pgfsetdash{}{0pt}%
\pgfpathmoveto{\pgfqpoint{5.420816in}{2.408142in}}%
\pgfpathlineto{\pgfqpoint{5.435387in}{2.412548in}}%
\pgfpathlineto{\pgfqpoint{5.449971in}{2.417024in}}%
\pgfpathlineto{\pgfqpoint{5.464568in}{2.421570in}}%
\pgfpathlineto{\pgfqpoint{5.479178in}{2.426187in}}%
\pgfpathlineto{\pgfqpoint{5.486678in}{2.431180in}}%
\pgfpathlineto{\pgfqpoint{5.494169in}{2.436083in}}%
\pgfpathlineto{\pgfqpoint{5.501651in}{2.440901in}}%
\pgfpathlineto{\pgfqpoint{5.509125in}{2.445636in}}%
\pgfpathlineto{\pgfqpoint{5.494533in}{2.441198in}}%
\pgfpathlineto{\pgfqpoint{5.479955in}{2.436830in}}%
\pgfpathlineto{\pgfqpoint{5.465389in}{2.432532in}}%
\pgfpathlineto{\pgfqpoint{5.450837in}{2.428304in}}%
\pgfpathlineto{\pgfqpoint{5.443345in}{2.423383in}}%
\pgfpathlineto{\pgfqpoint{5.435844in}{2.418384in}}%
\pgfpathlineto{\pgfqpoint{5.428334in}{2.413305in}}%
\pgfpathlineto{\pgfqpoint{5.420816in}{2.408142in}}%
\pgfpathclose%
\pgfusepath{fill}%
\end{pgfscope}%
\begin{pgfscope}%
\pgfpathrectangle{\pgfqpoint{1.150000in}{0.150000in}}{\pgfqpoint{5.700000in}{5.700000in}}%
\pgfusepath{clip}%
\pgfsetbuttcap%
\pgfsetroundjoin%
\definecolor{currentfill}{rgb}{0.243113,0.292092,0.538516}%
\pgfsetfillcolor{currentfill}%
\pgfsetfillopacity{0.700000}%
\pgfsetlinewidth{0.000000pt}%
\definecolor{currentstroke}{rgb}{0.000000,0.000000,0.000000}%
\pgfsetstrokecolor{currentstroke}%
\pgfsetdash{}{0pt}%
\pgfpathmoveto{\pgfqpoint{2.039819in}{2.175103in}}%
\pgfpathlineto{\pgfqpoint{2.053798in}{2.160865in}}%
\pgfpathlineto{\pgfqpoint{2.067772in}{2.146757in}}%
\pgfpathlineto{\pgfqpoint{2.081742in}{2.132777in}}%
\pgfpathlineto{\pgfqpoint{2.095709in}{2.118924in}}%
\pgfpathlineto{\pgfqpoint{2.105025in}{2.115139in}}%
\pgfpathlineto{\pgfqpoint{2.114318in}{2.111681in}}%
\pgfpathlineto{\pgfqpoint{2.123588in}{2.108543in}}%
\pgfpathlineto{\pgfqpoint{2.132835in}{2.105720in}}%
\pgfpathlineto{\pgfqpoint{2.118915in}{2.119141in}}%
\pgfpathlineto{\pgfqpoint{2.104992in}{2.132688in}}%
\pgfpathlineto{\pgfqpoint{2.091065in}{2.146363in}}%
\pgfpathlineto{\pgfqpoint{2.077135in}{2.160168in}}%
\pgfpathlineto{\pgfqpoint{2.067841in}{2.163415in}}%
\pgfpathlineto{\pgfqpoint{2.058525in}{2.166982in}}%
\pgfpathlineto{\pgfqpoint{2.049184in}{2.170876in}}%
\pgfpathlineto{\pgfqpoint{2.039819in}{2.175103in}}%
\pgfpathclose%
\pgfusepath{fill}%
\end{pgfscope}%
\begin{pgfscope}%
\pgfpathrectangle{\pgfqpoint{1.150000in}{0.150000in}}{\pgfqpoint{5.700000in}{5.700000in}}%
\pgfusepath{clip}%
\pgfsetbuttcap%
\pgfsetroundjoin%
\definecolor{currentfill}{rgb}{0.282884,0.135920,0.453427}%
\pgfsetfillcolor{currentfill}%
\pgfsetfillopacity{0.700000}%
\pgfsetlinewidth{0.000000pt}%
\definecolor{currentstroke}{rgb}{0.000000,0.000000,0.000000}%
\pgfsetstrokecolor{currentstroke}%
\pgfsetdash{}{0pt}%
\pgfpathmoveto{\pgfqpoint{2.466249in}{1.819099in}}%
\pgfpathlineto{\pgfqpoint{2.480126in}{1.808539in}}%
\pgfpathlineto{\pgfqpoint{2.494004in}{1.798083in}}%
\pgfpathlineto{\pgfqpoint{2.507881in}{1.787731in}}%
\pgfpathlineto{\pgfqpoint{2.521758in}{1.777480in}}%
\pgfpathlineto{\pgfqpoint{2.530698in}{1.777845in}}%
\pgfpathlineto{\pgfqpoint{2.539621in}{1.778469in}}%
\pgfpathlineto{\pgfqpoint{2.548527in}{1.779348in}}%
\pgfpathlineto{\pgfqpoint{2.557417in}{1.780475in}}%
\pgfpathlineto{\pgfqpoint{2.543576in}{1.790327in}}%
\pgfpathlineto{\pgfqpoint{2.529735in}{1.800281in}}%
\pgfpathlineto{\pgfqpoint{2.515894in}{1.810338in}}%
\pgfpathlineto{\pgfqpoint{2.502054in}{1.820498in}}%
\pgfpathlineto{\pgfqpoint{2.493129in}{1.819762in}}%
\pgfpathlineto{\pgfqpoint{2.484186in}{1.819279in}}%
\pgfpathlineto{\pgfqpoint{2.475226in}{1.819056in}}%
\pgfpathlineto{\pgfqpoint{2.466249in}{1.819099in}}%
\pgfpathclose%
\pgfusepath{fill}%
\end{pgfscope}%
\begin{pgfscope}%
\pgfpathrectangle{\pgfqpoint{1.150000in}{0.150000in}}{\pgfqpoint{5.700000in}{5.700000in}}%
\pgfusepath{clip}%
\pgfsetbuttcap%
\pgfsetroundjoin%
\definecolor{currentfill}{rgb}{0.197636,0.391528,0.554969}%
\pgfsetfillcolor{currentfill}%
\pgfsetfillopacity{0.700000}%
\pgfsetlinewidth{0.000000pt}%
\definecolor{currentstroke}{rgb}{0.000000,0.000000,0.000000}%
\pgfsetstrokecolor{currentstroke}%
\pgfsetdash{}{0pt}%
\pgfpathmoveto{\pgfqpoint{5.332437in}{2.369040in}}%
\pgfpathlineto{\pgfqpoint{5.346973in}{2.373321in}}%
\pgfpathlineto{\pgfqpoint{5.361521in}{2.377672in}}%
\pgfpathlineto{\pgfqpoint{5.376083in}{2.382094in}}%
\pgfpathlineto{\pgfqpoint{5.390658in}{2.386587in}}%
\pgfpathlineto{\pgfqpoint{5.398210in}{2.392117in}}%
\pgfpathlineto{\pgfqpoint{5.405754in}{2.397551in}}%
\pgfpathlineto{\pgfqpoint{5.413290in}{2.402891in}}%
\pgfpathlineto{\pgfqpoint{5.420816in}{2.408142in}}%
\pgfpathlineto{\pgfqpoint{5.406259in}{2.403806in}}%
\pgfpathlineto{\pgfqpoint{5.391714in}{2.399541in}}%
\pgfpathlineto{\pgfqpoint{5.377183in}{2.395346in}}%
\pgfpathlineto{\pgfqpoint{5.362664in}{2.391221in}}%
\pgfpathlineto{\pgfqpoint{5.355120in}{2.385806in}}%
\pgfpathlineto{\pgfqpoint{5.347567in}{2.380306in}}%
\pgfpathlineto{\pgfqpoint{5.340006in}{2.374718in}}%
\pgfpathlineto{\pgfqpoint{5.332437in}{2.369040in}}%
\pgfpathclose%
\pgfusepath{fill}%
\end{pgfscope}%
\begin{pgfscope}%
\pgfpathrectangle{\pgfqpoint{1.150000in}{0.150000in}}{\pgfqpoint{5.700000in}{5.700000in}}%
\pgfusepath{clip}%
\pgfsetbuttcap%
\pgfsetroundjoin%
\definecolor{currentfill}{rgb}{0.268510,0.009605,0.335427}%
\pgfsetfillcolor{currentfill}%
\pgfsetfillopacity{0.700000}%
\pgfsetlinewidth{0.000000pt}%
\definecolor{currentstroke}{rgb}{0.000000,0.000000,0.000000}%
\pgfsetstrokecolor{currentstroke}%
\pgfsetdash{}{0pt}%
\pgfpathmoveto{\pgfqpoint{3.069753in}{1.556060in}}%
\pgfpathlineto{\pgfqpoint{3.083605in}{1.549984in}}%
\pgfpathlineto{\pgfqpoint{3.097461in}{1.543992in}}%
\pgfpathlineto{\pgfqpoint{3.111321in}{1.538084in}}%
\pgfpathlineto{\pgfqpoint{3.125185in}{1.532260in}}%
\pgfpathlineto{\pgfqpoint{3.133701in}{1.538403in}}%
\pgfpathlineto{\pgfqpoint{3.142208in}{1.544689in}}%
\pgfpathlineto{\pgfqpoint{3.150704in}{1.551115in}}%
\pgfpathlineto{\pgfqpoint{3.159191in}{1.557675in}}%
\pgfpathlineto{\pgfqpoint{3.145349in}{1.563170in}}%
\pgfpathlineto{\pgfqpoint{3.131512in}{1.568749in}}%
\pgfpathlineto{\pgfqpoint{3.117678in}{1.574412in}}%
\pgfpathlineto{\pgfqpoint{3.103848in}{1.580160in}}%
\pgfpathlineto{\pgfqpoint{3.095340in}{1.573921in}}%
\pgfpathlineto{\pgfqpoint{3.086821in}{1.567821in}}%
\pgfpathlineto{\pgfqpoint{3.078292in}{1.561866in}}%
\pgfpathlineto{\pgfqpoint{3.069753in}{1.556060in}}%
\pgfpathclose%
\pgfusepath{fill}%
\end{pgfscope}%
\begin{pgfscope}%
\pgfpathrectangle{\pgfqpoint{1.150000in}{0.150000in}}{\pgfqpoint{5.700000in}{5.700000in}}%
\pgfusepath{clip}%
\pgfsetbuttcap%
\pgfsetroundjoin%
\definecolor{currentfill}{rgb}{0.267004,0.004874,0.329415}%
\pgfsetfillcolor{currentfill}%
\pgfsetfillopacity{0.700000}%
\pgfsetlinewidth{0.000000pt}%
\definecolor{currentstroke}{rgb}{0.000000,0.000000,0.000000}%
\pgfsetstrokecolor{currentstroke}%
\pgfsetdash{}{0pt}%
\pgfpathmoveto{\pgfqpoint{3.214602in}{1.536525in}}%
\pgfpathlineto{\pgfqpoint{3.228466in}{1.531444in}}%
\pgfpathlineto{\pgfqpoint{3.242334in}{1.526444in}}%
\pgfpathlineto{\pgfqpoint{3.256207in}{1.521525in}}%
\pgfpathlineto{\pgfqpoint{3.270085in}{1.516688in}}%
\pgfpathlineto{\pgfqpoint{3.278522in}{1.524009in}}%
\pgfpathlineto{\pgfqpoint{3.286951in}{1.531444in}}%
\pgfpathlineto{\pgfqpoint{3.295371in}{1.538988in}}%
\pgfpathlineto{\pgfqpoint{3.303782in}{1.546637in}}%
\pgfpathlineto{\pgfqpoint{3.289924in}{1.551167in}}%
\pgfpathlineto{\pgfqpoint{3.276070in}{1.555777in}}%
\pgfpathlineto{\pgfqpoint{3.262221in}{1.560469in}}%
\pgfpathlineto{\pgfqpoint{3.248377in}{1.565243in}}%
\pgfpathlineto{\pgfqpoint{3.239946in}{1.557894in}}%
\pgfpathlineto{\pgfqpoint{3.231507in}{1.550655in}}%
\pgfpathlineto{\pgfqpoint{3.223059in}{1.543530in}}%
\pgfpathlineto{\pgfqpoint{3.214602in}{1.536525in}}%
\pgfpathclose%
\pgfusepath{fill}%
\end{pgfscope}%
\begin{pgfscope}%
\pgfpathrectangle{\pgfqpoint{1.150000in}{0.150000in}}{\pgfqpoint{5.700000in}{5.700000in}}%
\pgfusepath{clip}%
\pgfsetbuttcap%
\pgfsetroundjoin%
\definecolor{currentfill}{rgb}{0.279566,0.067836,0.391917}%
\pgfsetfillcolor{currentfill}%
\pgfsetfillopacity{0.700000}%
\pgfsetlinewidth{0.000000pt}%
\definecolor{currentstroke}{rgb}{0.000000,0.000000,0.000000}%
\pgfsetstrokecolor{currentstroke}%
\pgfsetdash{}{0pt}%
\pgfpathmoveto{\pgfqpoint{2.723567in}{1.669990in}}%
\pgfpathlineto{\pgfqpoint{2.737420in}{1.661409in}}%
\pgfpathlineto{\pgfqpoint{2.751275in}{1.652922in}}%
\pgfpathlineto{\pgfqpoint{2.765132in}{1.644528in}}%
\pgfpathlineto{\pgfqpoint{2.778991in}{1.636227in}}%
\pgfpathlineto{\pgfqpoint{2.787734in}{1.639127in}}%
\pgfpathlineto{\pgfqpoint{2.796463in}{1.642241in}}%
\pgfpathlineto{\pgfqpoint{2.805179in}{1.645564in}}%
\pgfpathlineto{\pgfqpoint{2.813881in}{1.649089in}}%
\pgfpathlineto{\pgfqpoint{2.800053in}{1.657017in}}%
\pgfpathlineto{\pgfqpoint{2.786226in}{1.665037in}}%
\pgfpathlineto{\pgfqpoint{2.772401in}{1.673151in}}%
\pgfpathlineto{\pgfqpoint{2.758578in}{1.681358in}}%
\pgfpathlineto{\pgfqpoint{2.749846in}{1.678198in}}%
\pgfpathlineto{\pgfqpoint{2.741101in}{1.675246in}}%
\pgfpathlineto{\pgfqpoint{2.732341in}{1.672508in}}%
\pgfpathlineto{\pgfqpoint{2.723567in}{1.669990in}}%
\pgfpathclose%
\pgfusepath{fill}%
\end{pgfscope}%
\begin{pgfscope}%
\pgfpathrectangle{\pgfqpoint{1.150000in}{0.150000in}}{\pgfqpoint{5.700000in}{5.700000in}}%
\pgfusepath{clip}%
\pgfsetbuttcap%
\pgfsetroundjoin%
\definecolor{currentfill}{rgb}{0.274952,0.037752,0.364543}%
\pgfsetfillcolor{currentfill}%
\pgfsetfillopacity{0.700000}%
\pgfsetlinewidth{0.000000pt}%
\definecolor{currentstroke}{rgb}{0.000000,0.000000,0.000000}%
\pgfsetstrokecolor{currentstroke}%
\pgfsetdash{}{0pt}%
\pgfpathmoveto{\pgfqpoint{3.681536in}{1.587119in}}%
\pgfpathlineto{\pgfqpoint{3.695476in}{1.585033in}}%
\pgfpathlineto{\pgfqpoint{3.709422in}{1.583023in}}%
\pgfpathlineto{\pgfqpoint{3.723376in}{1.581088in}}%
\pgfpathlineto{\pgfqpoint{3.737337in}{1.579230in}}%
\pgfpathlineto{\pgfqpoint{3.745570in}{1.589272in}}%
\pgfpathlineto{\pgfqpoint{3.753796in}{1.599335in}}%
\pgfpathlineto{\pgfqpoint{3.762017in}{1.609414in}}%
\pgfpathlineto{\pgfqpoint{3.770232in}{1.619507in}}%
\pgfpathlineto{\pgfqpoint{3.756282in}{1.621139in}}%
\pgfpathlineto{\pgfqpoint{3.742340in}{1.622848in}}%
\pgfpathlineto{\pgfqpoint{3.728405in}{1.624632in}}%
\pgfpathlineto{\pgfqpoint{3.714478in}{1.626492in}}%
\pgfpathlineto{\pgfqpoint{3.706251in}{1.616617in}}%
\pgfpathlineto{\pgfqpoint{3.698019in}{1.606762in}}%
\pgfpathlineto{\pgfqpoint{3.689781in}{1.596928in}}%
\pgfpathlineto{\pgfqpoint{3.681536in}{1.587119in}}%
\pgfpathclose%
\pgfusepath{fill}%
\end{pgfscope}%
\begin{pgfscope}%
\pgfpathrectangle{\pgfqpoint{1.150000in}{0.150000in}}{\pgfqpoint{5.700000in}{5.700000in}}%
\pgfusepath{clip}%
\pgfsetbuttcap%
\pgfsetroundjoin%
\definecolor{currentfill}{rgb}{0.252194,0.269783,0.531579}%
\pgfsetfillcolor{currentfill}%
\pgfsetfillopacity{0.700000}%
\pgfsetlinewidth{0.000000pt}%
\definecolor{currentstroke}{rgb}{0.000000,0.000000,0.000000}%
\pgfsetstrokecolor{currentstroke}%
\pgfsetdash{}{0pt}%
\pgfpathmoveto{\pgfqpoint{2.095709in}{2.118924in}}%
\pgfpathlineto{\pgfqpoint{2.109672in}{2.105197in}}%
\pgfpathlineto{\pgfqpoint{2.123632in}{2.091596in}}%
\pgfpathlineto{\pgfqpoint{2.137588in}{2.078118in}}%
\pgfpathlineto{\pgfqpoint{2.151542in}{2.064764in}}%
\pgfpathlineto{\pgfqpoint{2.160811in}{2.061418in}}%
\pgfpathlineto{\pgfqpoint{2.170058in}{2.058394in}}%
\pgfpathlineto{\pgfqpoint{2.179282in}{2.055685in}}%
\pgfpathlineto{\pgfqpoint{2.188485in}{2.053284in}}%
\pgfpathlineto{\pgfqpoint{2.174577in}{2.066208in}}%
\pgfpathlineto{\pgfqpoint{2.160666in}{2.079255in}}%
\pgfpathlineto{\pgfqpoint{2.146752in}{2.092425in}}%
\pgfpathlineto{\pgfqpoint{2.132835in}{2.105720in}}%
\pgfpathlineto{\pgfqpoint{2.123588in}{2.108543in}}%
\pgfpathlineto{\pgfqpoint{2.114318in}{2.111681in}}%
\pgfpathlineto{\pgfqpoint{2.105025in}{2.115139in}}%
\pgfpathlineto{\pgfqpoint{2.095709in}{2.118924in}}%
\pgfpathclose%
\pgfusepath{fill}%
\end{pgfscope}%
\begin{pgfscope}%
\pgfpathrectangle{\pgfqpoint{1.150000in}{0.150000in}}{\pgfqpoint{5.700000in}{5.700000in}}%
\pgfusepath{clip}%
\pgfsetbuttcap%
\pgfsetroundjoin%
\definecolor{currentfill}{rgb}{0.204903,0.375746,0.553533}%
\pgfsetfillcolor{currentfill}%
\pgfsetfillopacity{0.700000}%
\pgfsetlinewidth{0.000000pt}%
\definecolor{currentstroke}{rgb}{0.000000,0.000000,0.000000}%
\pgfsetstrokecolor{currentstroke}%
\pgfsetdash{}{0pt}%
\pgfpathmoveto{\pgfqpoint{5.243995in}{2.328402in}}%
\pgfpathlineto{\pgfqpoint{5.258496in}{2.332536in}}%
\pgfpathlineto{\pgfqpoint{5.273009in}{2.336740in}}%
\pgfpathlineto{\pgfqpoint{5.287534in}{2.341015in}}%
\pgfpathlineto{\pgfqpoint{5.302073in}{2.345360in}}%
\pgfpathlineto{\pgfqpoint{5.309677in}{2.351430in}}%
\pgfpathlineto{\pgfqpoint{5.317272in}{2.357398in}}%
\pgfpathlineto{\pgfqpoint{5.324859in}{2.363267in}}%
\pgfpathlineto{\pgfqpoint{5.332437in}{2.369040in}}%
\pgfpathlineto{\pgfqpoint{5.317914in}{2.364829in}}%
\pgfpathlineto{\pgfqpoint{5.303404in}{2.360689in}}%
\pgfpathlineto{\pgfqpoint{5.288907in}{2.356619in}}%
\pgfpathlineto{\pgfqpoint{5.274422in}{2.352619in}}%
\pgfpathlineto{\pgfqpoint{5.266828in}{2.346705in}}%
\pgfpathlineto{\pgfqpoint{5.259226in}{2.340698in}}%
\pgfpathlineto{\pgfqpoint{5.251615in}{2.334598in}}%
\pgfpathlineto{\pgfqpoint{5.243995in}{2.328402in}}%
\pgfpathclose%
\pgfusepath{fill}%
\end{pgfscope}%
\begin{pgfscope}%
\pgfpathrectangle{\pgfqpoint{1.150000in}{0.150000in}}{\pgfqpoint{5.700000in}{5.700000in}}%
\pgfusepath{clip}%
\pgfsetbuttcap%
\pgfsetroundjoin%
\definecolor{currentfill}{rgb}{0.272594,0.025563,0.353093}%
\pgfsetfillcolor{currentfill}%
\pgfsetfillopacity{0.700000}%
\pgfsetlinewidth{0.000000pt}%
\definecolor{currentstroke}{rgb}{0.000000,0.000000,0.000000}%
\pgfsetstrokecolor{currentstroke}%
\pgfsetdash{}{0pt}%
\pgfpathmoveto{\pgfqpoint{2.924595in}{1.588936in}}%
\pgfpathlineto{\pgfqpoint{2.938446in}{1.581818in}}%
\pgfpathlineto{\pgfqpoint{2.952300in}{1.574789in}}%
\pgfpathlineto{\pgfqpoint{2.966157in}{1.567847in}}%
\pgfpathlineto{\pgfqpoint{2.980017in}{1.560991in}}%
\pgfpathlineto{\pgfqpoint{2.988625in}{1.565790in}}%
\pgfpathlineto{\pgfqpoint{2.997221in}{1.570765in}}%
\pgfpathlineto{\pgfqpoint{3.005806in}{1.575909in}}%
\pgfpathlineto{\pgfqpoint{3.014379in}{1.581219in}}%
\pgfpathlineto{\pgfqpoint{3.000545in}{1.587723in}}%
\pgfpathlineto{\pgfqpoint{2.986713in}{1.594315in}}%
\pgfpathlineto{\pgfqpoint{2.972885in}{1.600994in}}%
\pgfpathlineto{\pgfqpoint{2.959060in}{1.607761in}}%
\pgfpathlineto{\pgfqpoint{2.950461in}{1.602794in}}%
\pgfpathlineto{\pgfqpoint{2.941851in}{1.597997in}}%
\pgfpathlineto{\pgfqpoint{2.933229in}{1.593376in}}%
\pgfpathlineto{\pgfqpoint{2.924595in}{1.588936in}}%
\pgfpathclose%
\pgfusepath{fill}%
\end{pgfscope}%
\begin{pgfscope}%
\pgfpathrectangle{\pgfqpoint{1.150000in}{0.150000in}}{\pgfqpoint{5.700000in}{5.700000in}}%
\pgfusepath{clip}%
\pgfsetbuttcap%
\pgfsetroundjoin%
\definecolor{currentfill}{rgb}{0.210503,0.363727,0.552206}%
\pgfsetfillcolor{currentfill}%
\pgfsetfillopacity{0.700000}%
\pgfsetlinewidth{0.000000pt}%
\definecolor{currentstroke}{rgb}{0.000000,0.000000,0.000000}%
\pgfsetstrokecolor{currentstroke}%
\pgfsetdash{}{0pt}%
\pgfpathmoveto{\pgfqpoint{5.155502in}{2.286324in}}%
\pgfpathlineto{\pgfqpoint{5.169966in}{2.290288in}}%
\pgfpathlineto{\pgfqpoint{5.184443in}{2.294322in}}%
\pgfpathlineto{\pgfqpoint{5.198932in}{2.298427in}}%
\pgfpathlineto{\pgfqpoint{5.213434in}{2.302603in}}%
\pgfpathlineto{\pgfqpoint{5.221087in}{2.309210in}}%
\pgfpathlineto{\pgfqpoint{5.228732in}{2.315710in}}%
\pgfpathlineto{\pgfqpoint{5.236368in}{2.322107in}}%
\pgfpathlineto{\pgfqpoint{5.243995in}{2.328402in}}%
\pgfpathlineto{\pgfqpoint{5.229508in}{2.324339in}}%
\pgfpathlineto{\pgfqpoint{5.215033in}{2.320346in}}%
\pgfpathlineto{\pgfqpoint{5.200571in}{2.316424in}}%
\pgfpathlineto{\pgfqpoint{5.186121in}{2.312573in}}%
\pgfpathlineto{\pgfqpoint{5.178479in}{2.306157in}}%
\pgfpathlineto{\pgfqpoint{5.170828in}{2.299645in}}%
\pgfpathlineto{\pgfqpoint{5.163169in}{2.293035in}}%
\pgfpathlineto{\pgfqpoint{5.155502in}{2.286324in}}%
\pgfpathclose%
\pgfusepath{fill}%
\end{pgfscope}%
\begin{pgfscope}%
\pgfpathrectangle{\pgfqpoint{1.150000in}{0.150000in}}{\pgfqpoint{5.700000in}{5.700000in}}%
\pgfusepath{clip}%
\pgfsetbuttcap%
\pgfsetroundjoin%
\definecolor{currentfill}{rgb}{0.267004,0.004874,0.329415}%
\pgfsetfillcolor{currentfill}%
\pgfsetfillopacity{0.700000}%
\pgfsetlinewidth{0.000000pt}%
\definecolor{currentstroke}{rgb}{0.000000,0.000000,0.000000}%
\pgfsetstrokecolor{currentstroke}%
\pgfsetdash{}{0pt}%
\pgfpathmoveto{\pgfqpoint{3.359268in}{1.529325in}}%
\pgfpathlineto{\pgfqpoint{3.373153in}{1.525196in}}%
\pgfpathlineto{\pgfqpoint{3.387043in}{1.521147in}}%
\pgfpathlineto{\pgfqpoint{3.400939in}{1.517178in}}%
\pgfpathlineto{\pgfqpoint{3.414840in}{1.513287in}}%
\pgfpathlineto{\pgfqpoint{3.423208in}{1.521626in}}%
\pgfpathlineto{\pgfqpoint{3.431569in}{1.530052in}}%
\pgfpathlineto{\pgfqpoint{3.439923in}{1.538560in}}%
\pgfpathlineto{\pgfqpoint{3.448269in}{1.547145in}}%
\pgfpathlineto{\pgfqpoint{3.434384in}{1.550749in}}%
\pgfpathlineto{\pgfqpoint{3.420505in}{1.554431in}}%
\pgfpathlineto{\pgfqpoint{3.406632in}{1.558193in}}%
\pgfpathlineto{\pgfqpoint{3.392765in}{1.562034in}}%
\pgfpathlineto{\pgfqpoint{3.384402in}{1.553729in}}%
\pgfpathlineto{\pgfqpoint{3.376032in}{1.545506in}}%
\pgfpathlineto{\pgfqpoint{3.367654in}{1.537370in}}%
\pgfpathlineto{\pgfqpoint{3.359268in}{1.529325in}}%
\pgfpathclose%
\pgfusepath{fill}%
\end{pgfscope}%
\begin{pgfscope}%
\pgfpathrectangle{\pgfqpoint{1.150000in}{0.150000in}}{\pgfqpoint{5.700000in}{5.700000in}}%
\pgfusepath{clip}%
\pgfsetbuttcap%
\pgfsetroundjoin%
\definecolor{currentfill}{rgb}{0.260571,0.246922,0.522828}%
\pgfsetfillcolor{currentfill}%
\pgfsetfillopacity{0.700000}%
\pgfsetlinewidth{0.000000pt}%
\definecolor{currentstroke}{rgb}{0.000000,0.000000,0.000000}%
\pgfsetstrokecolor{currentstroke}%
\pgfsetdash{}{0pt}%
\pgfpathmoveto{\pgfqpoint{2.151542in}{2.064764in}}%
\pgfpathlineto{\pgfqpoint{2.165492in}{2.051531in}}%
\pgfpathlineto{\pgfqpoint{2.179440in}{2.038420in}}%
\pgfpathlineto{\pgfqpoint{2.193385in}{2.025428in}}%
\pgfpathlineto{\pgfqpoint{2.207327in}{2.012556in}}%
\pgfpathlineto{\pgfqpoint{2.216551in}{2.009649in}}%
\pgfpathlineto{\pgfqpoint{2.225752in}{2.007057in}}%
\pgfpathlineto{\pgfqpoint{2.234933in}{2.004774in}}%
\pgfpathlineto{\pgfqpoint{2.244091in}{2.002792in}}%
\pgfpathlineto{\pgfqpoint{2.230193in}{2.015236in}}%
\pgfpathlineto{\pgfqpoint{2.216293in}{2.027799in}}%
\pgfpathlineto{\pgfqpoint{2.202390in}{2.040481in}}%
\pgfpathlineto{\pgfqpoint{2.188485in}{2.053284in}}%
\pgfpathlineto{\pgfqpoint{2.179282in}{2.055685in}}%
\pgfpathlineto{\pgfqpoint{2.170058in}{2.058394in}}%
\pgfpathlineto{\pgfqpoint{2.160811in}{2.061418in}}%
\pgfpathlineto{\pgfqpoint{2.151542in}{2.064764in}}%
\pgfpathclose%
\pgfusepath{fill}%
\end{pgfscope}%
\begin{pgfscope}%
\pgfpathrectangle{\pgfqpoint{1.150000in}{0.150000in}}{\pgfqpoint{5.700000in}{5.700000in}}%
\pgfusepath{clip}%
\pgfsetbuttcap%
\pgfsetroundjoin%
\definecolor{currentfill}{rgb}{0.218130,0.347432,0.550038}%
\pgfsetfillcolor{currentfill}%
\pgfsetfillopacity{0.700000}%
\pgfsetlinewidth{0.000000pt}%
\definecolor{currentstroke}{rgb}{0.000000,0.000000,0.000000}%
\pgfsetstrokecolor{currentstroke}%
\pgfsetdash{}{0pt}%
\pgfpathmoveto{\pgfqpoint{5.066965in}{2.242923in}}%
\pgfpathlineto{\pgfqpoint{5.081393in}{2.246694in}}%
\pgfpathlineto{\pgfqpoint{5.095833in}{2.250536in}}%
\pgfpathlineto{\pgfqpoint{5.110286in}{2.254449in}}%
\pgfpathlineto{\pgfqpoint{5.124751in}{2.258433in}}%
\pgfpathlineto{\pgfqpoint{5.132451in}{2.265567in}}%
\pgfpathlineto{\pgfqpoint{5.140143in}{2.272592in}}%
\pgfpathlineto{\pgfqpoint{5.147826in}{2.279510in}}%
\pgfpathlineto{\pgfqpoint{5.155502in}{2.286324in}}%
\pgfpathlineto{\pgfqpoint{5.141050in}{2.282431in}}%
\pgfpathlineto{\pgfqpoint{5.126611in}{2.278609in}}%
\pgfpathlineto{\pgfqpoint{5.112184in}{2.274857in}}%
\pgfpathlineto{\pgfqpoint{5.097769in}{2.271176in}}%
\pgfpathlineto{\pgfqpoint{5.090080in}{2.264264in}}%
\pgfpathlineto{\pgfqpoint{5.082383in}{2.257252in}}%
\pgfpathlineto{\pgfqpoint{5.074678in}{2.250139in}}%
\pgfpathlineto{\pgfqpoint{5.066965in}{2.242923in}}%
\pgfpathclose%
\pgfusepath{fill}%
\end{pgfscope}%
\begin{pgfscope}%
\pgfpathrectangle{\pgfqpoint{1.150000in}{0.150000in}}{\pgfqpoint{5.700000in}{5.700000in}}%
\pgfusepath{clip}%
\pgfsetbuttcap%
\pgfsetroundjoin%
\definecolor{currentfill}{rgb}{0.271305,0.019942,0.347269}%
\pgfsetfillcolor{currentfill}%
\pgfsetfillopacity{0.700000}%
\pgfsetlinewidth{0.000000pt}%
\definecolor{currentstroke}{rgb}{0.000000,0.000000,0.000000}%
\pgfsetstrokecolor{currentstroke}%
\pgfsetdash{}{0pt}%
\pgfpathmoveto{\pgfqpoint{3.592755in}{1.558293in}}%
\pgfpathlineto{\pgfqpoint{3.606681in}{1.555656in}}%
\pgfpathlineto{\pgfqpoint{3.620613in}{1.553095in}}%
\pgfpathlineto{\pgfqpoint{3.634551in}{1.550610in}}%
\pgfpathlineto{\pgfqpoint{3.648497in}{1.548201in}}%
\pgfpathlineto{\pgfqpoint{3.656766in}{1.557876in}}%
\pgfpathlineto{\pgfqpoint{3.665029in}{1.567590in}}%
\pgfpathlineto{\pgfqpoint{3.673285in}{1.577338in}}%
\pgfpathlineto{\pgfqpoint{3.681536in}{1.587119in}}%
\pgfpathlineto{\pgfqpoint{3.667603in}{1.589281in}}%
\pgfpathlineto{\pgfqpoint{3.653678in}{1.591519in}}%
\pgfpathlineto{\pgfqpoint{3.639759in}{1.593834in}}%
\pgfpathlineto{\pgfqpoint{3.625847in}{1.596225in}}%
\pgfpathlineto{\pgfqpoint{3.617583in}{1.586683in}}%
\pgfpathlineto{\pgfqpoint{3.609314in}{1.577179in}}%
\pgfpathlineto{\pgfqpoint{3.601038in}{1.567714in}}%
\pgfpathlineto{\pgfqpoint{3.592755in}{1.558293in}}%
\pgfpathclose%
\pgfusepath{fill}%
\end{pgfscope}%
\begin{pgfscope}%
\pgfpathrectangle{\pgfqpoint{1.150000in}{0.150000in}}{\pgfqpoint{5.700000in}{5.700000in}}%
\pgfusepath{clip}%
\pgfsetbuttcap%
\pgfsetroundjoin%
\definecolor{currentfill}{rgb}{0.283229,0.120777,0.440584}%
\pgfsetfillcolor{currentfill}%
\pgfsetfillopacity{0.700000}%
\pgfsetlinewidth{0.000000pt}%
\definecolor{currentstroke}{rgb}{0.000000,0.000000,0.000000}%
\pgfsetstrokecolor{currentstroke}%
\pgfsetdash{}{0pt}%
\pgfpathmoveto{\pgfqpoint{2.521758in}{1.777480in}}%
\pgfpathlineto{\pgfqpoint{2.535635in}{1.767332in}}%
\pgfpathlineto{\pgfqpoint{2.549513in}{1.757284in}}%
\pgfpathlineto{\pgfqpoint{2.563390in}{1.747337in}}%
\pgfpathlineto{\pgfqpoint{2.577269in}{1.737490in}}%
\pgfpathlineto{\pgfqpoint{2.586173in}{1.738261in}}%
\pgfpathlineto{\pgfqpoint{2.595060in}{1.739286in}}%
\pgfpathlineto{\pgfqpoint{2.603931in}{1.740560in}}%
\pgfpathlineto{\pgfqpoint{2.612786in}{1.742077in}}%
\pgfpathlineto{\pgfqpoint{2.598943in}{1.751526in}}%
\pgfpathlineto{\pgfqpoint{2.585100in}{1.761076in}}%
\pgfpathlineto{\pgfqpoint{2.571258in}{1.770725in}}%
\pgfpathlineto{\pgfqpoint{2.557417in}{1.780475in}}%
\pgfpathlineto{\pgfqpoint{2.548527in}{1.779348in}}%
\pgfpathlineto{\pgfqpoint{2.539621in}{1.778469in}}%
\pgfpathlineto{\pgfqpoint{2.530698in}{1.777845in}}%
\pgfpathlineto{\pgfqpoint{2.521758in}{1.777480in}}%
\pgfpathclose%
\pgfusepath{fill}%
\end{pgfscope}%
\begin{pgfscope}%
\pgfpathrectangle{\pgfqpoint{1.150000in}{0.150000in}}{\pgfqpoint{5.700000in}{5.700000in}}%
\pgfusepath{clip}%
\pgfsetbuttcap%
\pgfsetroundjoin%
\definecolor{currentfill}{rgb}{0.227802,0.326594,0.546532}%
\pgfsetfillcolor{currentfill}%
\pgfsetfillopacity{0.700000}%
\pgfsetlinewidth{0.000000pt}%
\definecolor{currentstroke}{rgb}{0.000000,0.000000,0.000000}%
\pgfsetstrokecolor{currentstroke}%
\pgfsetdash{}{0pt}%
\pgfpathmoveto{\pgfqpoint{4.978394in}{2.198338in}}%
\pgfpathlineto{\pgfqpoint{4.992786in}{2.201895in}}%
\pgfpathlineto{\pgfqpoint{5.007190in}{2.205522in}}%
\pgfpathlineto{\pgfqpoint{5.021605in}{2.209221in}}%
\pgfpathlineto{\pgfqpoint{5.036033in}{2.212990in}}%
\pgfpathlineto{\pgfqpoint{5.043778in}{2.220637in}}%
\pgfpathlineto{\pgfqpoint{5.051515in}{2.228173in}}%
\pgfpathlineto{\pgfqpoint{5.059244in}{2.235601in}}%
\pgfpathlineto{\pgfqpoint{5.066965in}{2.242923in}}%
\pgfpathlineto{\pgfqpoint{5.052549in}{2.239222in}}%
\pgfpathlineto{\pgfqpoint{5.038146in}{2.235593in}}%
\pgfpathlineto{\pgfqpoint{5.023754in}{2.232034in}}%
\pgfpathlineto{\pgfqpoint{5.009375in}{2.228546in}}%
\pgfpathlineto{\pgfqpoint{5.001641in}{2.221148in}}%
\pgfpathlineto{\pgfqpoint{4.993900in}{2.213648in}}%
\pgfpathlineto{\pgfqpoint{4.986151in}{2.206046in}}%
\pgfpathlineto{\pgfqpoint{4.978394in}{2.198338in}}%
\pgfpathclose%
\pgfusepath{fill}%
\end{pgfscope}%
\begin{pgfscope}%
\pgfpathrectangle{\pgfqpoint{1.150000in}{0.150000in}}{\pgfqpoint{5.700000in}{5.700000in}}%
\pgfusepath{clip}%
\pgfsetbuttcap%
\pgfsetroundjoin%
\definecolor{currentfill}{rgb}{0.235526,0.309527,0.542944}%
\pgfsetfillcolor{currentfill}%
\pgfsetfillopacity{0.700000}%
\pgfsetlinewidth{0.000000pt}%
\definecolor{currentstroke}{rgb}{0.000000,0.000000,0.000000}%
\pgfsetstrokecolor{currentstroke}%
\pgfsetdash{}{0pt}%
\pgfpathmoveto{\pgfqpoint{4.889798in}{2.152732in}}%
\pgfpathlineto{\pgfqpoint{4.904153in}{2.156051in}}%
\pgfpathlineto{\pgfqpoint{4.918520in}{2.159442in}}%
\pgfpathlineto{\pgfqpoint{4.932899in}{2.162903in}}%
\pgfpathlineto{\pgfqpoint{4.947290in}{2.166436in}}%
\pgfpathlineto{\pgfqpoint{4.955078in}{2.174575in}}%
\pgfpathlineto{\pgfqpoint{4.962858in}{2.182604in}}%
\pgfpathlineto{\pgfqpoint{4.970630in}{2.190525in}}%
\pgfpathlineto{\pgfqpoint{4.978394in}{2.198338in}}%
\pgfpathlineto{\pgfqpoint{4.964015in}{2.194853in}}%
\pgfpathlineto{\pgfqpoint{4.949647in}{2.191438in}}%
\pgfpathlineto{\pgfqpoint{4.935291in}{2.188095in}}%
\pgfpathlineto{\pgfqpoint{4.920947in}{2.184823in}}%
\pgfpathlineto{\pgfqpoint{4.913171in}{2.176954in}}%
\pgfpathlineto{\pgfqpoint{4.905388in}{2.168984in}}%
\pgfpathlineto{\pgfqpoint{4.897597in}{2.160910in}}%
\pgfpathlineto{\pgfqpoint{4.889798in}{2.152732in}}%
\pgfpathclose%
\pgfusepath{fill}%
\end{pgfscope}%
\begin{pgfscope}%
\pgfpathrectangle{\pgfqpoint{1.150000in}{0.150000in}}{\pgfqpoint{5.700000in}{5.700000in}}%
\pgfusepath{clip}%
\pgfsetbuttcap%
\pgfsetroundjoin%
\definecolor{currentfill}{rgb}{0.243113,0.292092,0.538516}%
\pgfsetfillcolor{currentfill}%
\pgfsetfillopacity{0.700000}%
\pgfsetlinewidth{0.000000pt}%
\definecolor{currentstroke}{rgb}{0.000000,0.000000,0.000000}%
\pgfsetstrokecolor{currentstroke}%
\pgfsetdash{}{0pt}%
\pgfpathmoveto{\pgfqpoint{4.801184in}{2.106287in}}%
\pgfpathlineto{\pgfqpoint{4.815503in}{2.109347in}}%
\pgfpathlineto{\pgfqpoint{4.829834in}{2.112477in}}%
\pgfpathlineto{\pgfqpoint{4.844176in}{2.115680in}}%
\pgfpathlineto{\pgfqpoint{4.858530in}{2.118953in}}%
\pgfpathlineto{\pgfqpoint{4.866358in}{2.127560in}}%
\pgfpathlineto{\pgfqpoint{4.874179in}{2.136058in}}%
\pgfpathlineto{\pgfqpoint{4.881992in}{2.144448in}}%
\pgfpathlineto{\pgfqpoint{4.889798in}{2.152732in}}%
\pgfpathlineto{\pgfqpoint{4.875455in}{2.149484in}}%
\pgfpathlineto{\pgfqpoint{4.861123in}{2.146307in}}%
\pgfpathlineto{\pgfqpoint{4.846803in}{2.143201in}}%
\pgfpathlineto{\pgfqpoint{4.832494in}{2.140167in}}%
\pgfpathlineto{\pgfqpoint{4.824677in}{2.131850in}}%
\pgfpathlineto{\pgfqpoint{4.816853in}{2.123431in}}%
\pgfpathlineto{\pgfqpoint{4.809022in}{2.114911in}}%
\pgfpathlineto{\pgfqpoint{4.801184in}{2.106287in}}%
\pgfpathclose%
\pgfusepath{fill}%
\end{pgfscope}%
\begin{pgfscope}%
\pgfpathrectangle{\pgfqpoint{1.150000in}{0.150000in}}{\pgfqpoint{5.700000in}{5.700000in}}%
\pgfusepath{clip}%
\pgfsetbuttcap%
\pgfsetroundjoin%
\definecolor{currentfill}{rgb}{0.271828,0.209303,0.504434}%
\pgfsetfillcolor{currentfill}%
\pgfsetfillopacity{0.700000}%
\pgfsetlinewidth{0.000000pt}%
\definecolor{currentstroke}{rgb}{0.000000,0.000000,0.000000}%
\pgfsetstrokecolor{currentstroke}%
\pgfsetdash{}{0pt}%
\pgfpathmoveto{\pgfqpoint{4.446664in}{1.916534in}}%
\pgfpathlineto{\pgfqpoint{4.460844in}{1.918334in}}%
\pgfpathlineto{\pgfqpoint{4.475034in}{1.920205in}}%
\pgfpathlineto{\pgfqpoint{4.489234in}{1.922149in}}%
\pgfpathlineto{\pgfqpoint{4.503446in}{1.924164in}}%
\pgfpathlineto{\pgfqpoint{4.511417in}{1.934275in}}%
\pgfpathlineto{\pgfqpoint{4.519383in}{1.944297in}}%
\pgfpathlineto{\pgfqpoint{4.527342in}{1.954230in}}%
\pgfpathlineto{\pgfqpoint{4.535295in}{1.964074in}}%
\pgfpathlineto{\pgfqpoint{4.521092in}{1.961998in}}%
\pgfpathlineto{\pgfqpoint{4.506900in}{1.959994in}}%
\pgfpathlineto{\pgfqpoint{4.492718in}{1.958062in}}%
\pgfpathlineto{\pgfqpoint{4.478546in}{1.956202in}}%
\pgfpathlineto{\pgfqpoint{4.470585in}{1.946412in}}%
\pgfpathlineto{\pgfqpoint{4.462617in}{1.936536in}}%
\pgfpathlineto{\pgfqpoint{4.454643in}{1.926577in}}%
\pgfpathlineto{\pgfqpoint{4.446664in}{1.916534in}}%
\pgfpathclose%
\pgfusepath{fill}%
\end{pgfscope}%
\begin{pgfscope}%
\pgfpathrectangle{\pgfqpoint{1.150000in}{0.150000in}}{\pgfqpoint{5.700000in}{5.700000in}}%
\pgfusepath{clip}%
\pgfsetbuttcap%
\pgfsetroundjoin%
\definecolor{currentfill}{rgb}{0.252194,0.269783,0.531579}%
\pgfsetfillcolor{currentfill}%
\pgfsetfillopacity{0.700000}%
\pgfsetlinewidth{0.000000pt}%
\definecolor{currentstroke}{rgb}{0.000000,0.000000,0.000000}%
\pgfsetstrokecolor{currentstroke}%
\pgfsetdash{}{0pt}%
\pgfpathmoveto{\pgfqpoint{4.712559in}{2.059208in}}%
\pgfpathlineto{\pgfqpoint{4.726842in}{2.061986in}}%
\pgfpathlineto{\pgfqpoint{4.741136in}{2.064835in}}%
\pgfpathlineto{\pgfqpoint{4.755442in}{2.067756in}}%
\pgfpathlineto{\pgfqpoint{4.769760in}{2.070748in}}%
\pgfpathlineto{\pgfqpoint{4.777626in}{2.079790in}}%
\pgfpathlineto{\pgfqpoint{4.785486in}{2.088727in}}%
\pgfpathlineto{\pgfqpoint{4.793339in}{2.097559in}}%
\pgfpathlineto{\pgfqpoint{4.801184in}{2.106287in}}%
\pgfpathlineto{\pgfqpoint{4.786876in}{2.103298in}}%
\pgfpathlineto{\pgfqpoint{4.772580in}{2.100381in}}%
\pgfpathlineto{\pgfqpoint{4.758295in}{2.097536in}}%
\pgfpathlineto{\pgfqpoint{4.744022in}{2.094762in}}%
\pgfpathlineto{\pgfqpoint{4.736166in}{2.086022in}}%
\pgfpathlineto{\pgfqpoint{4.728304in}{2.077184in}}%
\pgfpathlineto{\pgfqpoint{4.720435in}{2.068246in}}%
\pgfpathlineto{\pgfqpoint{4.712559in}{2.059208in}}%
\pgfpathclose%
\pgfusepath{fill}%
\end{pgfscope}%
\begin{pgfscope}%
\pgfpathrectangle{\pgfqpoint{1.150000in}{0.150000in}}{\pgfqpoint{5.700000in}{5.700000in}}%
\pgfusepath{clip}%
\pgfsetbuttcap%
\pgfsetroundjoin%
\definecolor{currentfill}{rgb}{0.266580,0.228262,0.514349}%
\pgfsetfillcolor{currentfill}%
\pgfsetfillopacity{0.700000}%
\pgfsetlinewidth{0.000000pt}%
\definecolor{currentstroke}{rgb}{0.000000,0.000000,0.000000}%
\pgfsetstrokecolor{currentstroke}%
\pgfsetdash{}{0pt}%
\pgfpathmoveto{\pgfqpoint{4.535295in}{1.964074in}}%
\pgfpathlineto{\pgfqpoint{4.549508in}{1.966221in}}%
\pgfpathlineto{\pgfqpoint{4.563733in}{1.968441in}}%
\pgfpathlineto{\pgfqpoint{4.577968in}{1.970732in}}%
\pgfpathlineto{\pgfqpoint{4.592213in}{1.973095in}}%
\pgfpathlineto{\pgfqpoint{4.600152in}{1.982895in}}%
\pgfpathlineto{\pgfqpoint{4.608083in}{1.992600in}}%
\pgfpathlineto{\pgfqpoint{4.616009in}{2.002208in}}%
\pgfpathlineto{\pgfqpoint{4.623927in}{2.011721in}}%
\pgfpathlineto{\pgfqpoint{4.609690in}{2.009319in}}%
\pgfpathlineto{\pgfqpoint{4.595464in}{2.006988in}}%
\pgfpathlineto{\pgfqpoint{4.581248in}{2.004730in}}%
\pgfpathlineto{\pgfqpoint{4.567043in}{2.002543in}}%
\pgfpathlineto{\pgfqpoint{4.559116in}{1.993061in}}%
\pgfpathlineto{\pgfqpoint{4.551182in}{1.983489in}}%
\pgfpathlineto{\pgfqpoint{4.543242in}{1.973827in}}%
\pgfpathlineto{\pgfqpoint{4.535295in}{1.964074in}}%
\pgfpathclose%
\pgfusepath{fill}%
\end{pgfscope}%
\begin{pgfscope}%
\pgfpathrectangle{\pgfqpoint{1.150000in}{0.150000in}}{\pgfqpoint{5.700000in}{5.700000in}}%
\pgfusepath{clip}%
\pgfsetbuttcap%
\pgfsetroundjoin%
\definecolor{currentfill}{rgb}{0.277134,0.185228,0.489898}%
\pgfsetfillcolor{currentfill}%
\pgfsetfillopacity{0.700000}%
\pgfsetlinewidth{0.000000pt}%
\definecolor{currentstroke}{rgb}{0.000000,0.000000,0.000000}%
\pgfsetstrokecolor{currentstroke}%
\pgfsetdash{}{0pt}%
\pgfpathmoveto{\pgfqpoint{4.358035in}{1.869393in}}%
\pgfpathlineto{\pgfqpoint{4.372182in}{1.870822in}}%
\pgfpathlineto{\pgfqpoint{4.386340in}{1.872323in}}%
\pgfpathlineto{\pgfqpoint{4.400507in}{1.873896in}}%
\pgfpathlineto{\pgfqpoint{4.414685in}{1.875542in}}%
\pgfpathlineto{\pgfqpoint{4.422688in}{1.885912in}}%
\pgfpathlineto{\pgfqpoint{4.430686in}{1.896201in}}%
\pgfpathlineto{\pgfqpoint{4.438678in}{1.906409in}}%
\pgfpathlineto{\pgfqpoint{4.446664in}{1.916534in}}%
\pgfpathlineto{\pgfqpoint{4.432494in}{1.914807in}}%
\pgfpathlineto{\pgfqpoint{4.418335in}{1.913152in}}%
\pgfpathlineto{\pgfqpoint{4.404185in}{1.911569in}}%
\pgfpathlineto{\pgfqpoint{4.390046in}{1.910058in}}%
\pgfpathlineto{\pgfqpoint{4.382052in}{1.900006in}}%
\pgfpathlineto{\pgfqpoint{4.374052in}{1.889878in}}%
\pgfpathlineto{\pgfqpoint{4.366046in}{1.879673in}}%
\pgfpathlineto{\pgfqpoint{4.358035in}{1.869393in}}%
\pgfpathclose%
\pgfusepath{fill}%
\end{pgfscope}%
\begin{pgfscope}%
\pgfpathrectangle{\pgfqpoint{1.150000in}{0.150000in}}{\pgfqpoint{5.700000in}{5.700000in}}%
\pgfusepath{clip}%
\pgfsetbuttcap%
\pgfsetroundjoin%
\definecolor{currentfill}{rgb}{0.258965,0.251537,0.524736}%
\pgfsetfillcolor{currentfill}%
\pgfsetfillopacity{0.700000}%
\pgfsetlinewidth{0.000000pt}%
\definecolor{currentstroke}{rgb}{0.000000,0.000000,0.000000}%
\pgfsetstrokecolor{currentstroke}%
\pgfsetdash{}{0pt}%
\pgfpathmoveto{\pgfqpoint{4.623927in}{2.011721in}}%
\pgfpathlineto{\pgfqpoint{4.638175in}{2.014195in}}%
\pgfpathlineto{\pgfqpoint{4.652435in}{2.016740in}}%
\pgfpathlineto{\pgfqpoint{4.666705in}{2.019357in}}%
\pgfpathlineto{\pgfqpoint{4.680986in}{2.022046in}}%
\pgfpathlineto{\pgfqpoint{4.688889in}{2.031488in}}%
\pgfpathlineto{\pgfqpoint{4.696786in}{2.040829in}}%
\pgfpathlineto{\pgfqpoint{4.704676in}{2.050069in}}%
\pgfpathlineto{\pgfqpoint{4.712559in}{2.059208in}}%
\pgfpathlineto{\pgfqpoint{4.698286in}{2.056501in}}%
\pgfpathlineto{\pgfqpoint{4.684025in}{2.053867in}}%
\pgfpathlineto{\pgfqpoint{4.669775in}{2.051303in}}%
\pgfpathlineto{\pgfqpoint{4.655537in}{2.048812in}}%
\pgfpathlineto{\pgfqpoint{4.647644in}{2.039683in}}%
\pgfpathlineto{\pgfqpoint{4.639745in}{2.030458in}}%
\pgfpathlineto{\pgfqpoint{4.631840in}{2.021137in}}%
\pgfpathlineto{\pgfqpoint{4.623927in}{2.011721in}}%
\pgfpathclose%
\pgfusepath{fill}%
\end{pgfscope}%
\begin{pgfscope}%
\pgfpathrectangle{\pgfqpoint{1.150000in}{0.150000in}}{\pgfqpoint{5.700000in}{5.700000in}}%
\pgfusepath{clip}%
\pgfsetbuttcap%
\pgfsetroundjoin%
\definecolor{currentfill}{rgb}{0.280255,0.165693,0.476498}%
\pgfsetfillcolor{currentfill}%
\pgfsetfillopacity{0.700000}%
\pgfsetlinewidth{0.000000pt}%
\definecolor{currentstroke}{rgb}{0.000000,0.000000,0.000000}%
\pgfsetstrokecolor{currentstroke}%
\pgfsetdash{}{0pt}%
\pgfpathmoveto{\pgfqpoint{4.269407in}{1.822958in}}%
\pgfpathlineto{\pgfqpoint{4.283523in}{1.823995in}}%
\pgfpathlineto{\pgfqpoint{4.297649in}{1.825104in}}%
\pgfpathlineto{\pgfqpoint{4.311785in}{1.826286in}}%
\pgfpathlineto{\pgfqpoint{4.325930in}{1.827540in}}%
\pgfpathlineto{\pgfqpoint{4.333965in}{1.838111in}}%
\pgfpathlineto{\pgfqpoint{4.341994in}{1.848610in}}%
\pgfpathlineto{\pgfqpoint{4.350017in}{1.859038in}}%
\pgfpathlineto{\pgfqpoint{4.358035in}{1.869393in}}%
\pgfpathlineto{\pgfqpoint{4.343897in}{1.868036in}}%
\pgfpathlineto{\pgfqpoint{4.329770in}{1.866752in}}%
\pgfpathlineto{\pgfqpoint{4.315652in}{1.865540in}}%
\pgfpathlineto{\pgfqpoint{4.301544in}{1.864401in}}%
\pgfpathlineto{\pgfqpoint{4.293518in}{1.854141in}}%
\pgfpathlineto{\pgfqpoint{4.285487in}{1.843813in}}%
\pgfpathlineto{\pgfqpoint{4.277450in}{1.833419in}}%
\pgfpathlineto{\pgfqpoint{4.269407in}{1.822958in}}%
\pgfpathclose%
\pgfusepath{fill}%
\end{pgfscope}%
\begin{pgfscope}%
\pgfpathrectangle{\pgfqpoint{1.150000in}{0.150000in}}{\pgfqpoint{5.700000in}{5.700000in}}%
\pgfusepath{clip}%
\pgfsetbuttcap%
\pgfsetroundjoin%
\definecolor{currentfill}{rgb}{0.282623,0.140926,0.457517}%
\pgfsetfillcolor{currentfill}%
\pgfsetfillopacity{0.700000}%
\pgfsetlinewidth{0.000000pt}%
\definecolor{currentstroke}{rgb}{0.000000,0.000000,0.000000}%
\pgfsetstrokecolor{currentstroke}%
\pgfsetdash{}{0pt}%
\pgfpathmoveto{\pgfqpoint{4.180778in}{1.777563in}}%
\pgfpathlineto{\pgfqpoint{4.194865in}{1.778185in}}%
\pgfpathlineto{\pgfqpoint{4.208960in}{1.778881in}}%
\pgfpathlineto{\pgfqpoint{4.223066in}{1.779648in}}%
\pgfpathlineto{\pgfqpoint{4.237180in}{1.780489in}}%
\pgfpathlineto{\pgfqpoint{4.245245in}{1.791198in}}%
\pgfpathlineto{\pgfqpoint{4.253305in}{1.801847in}}%
\pgfpathlineto{\pgfqpoint{4.261359in}{1.812434in}}%
\pgfpathlineto{\pgfqpoint{4.269407in}{1.822958in}}%
\pgfpathlineto{\pgfqpoint{4.255301in}{1.821994in}}%
\pgfpathlineto{\pgfqpoint{4.241203in}{1.821103in}}%
\pgfpathlineto{\pgfqpoint{4.227116in}{1.820285in}}%
\pgfpathlineto{\pgfqpoint{4.213038in}{1.819539in}}%
\pgfpathlineto{\pgfqpoint{4.204981in}{1.809130in}}%
\pgfpathlineto{\pgfqpoint{4.196919in}{1.798664in}}%
\pgfpathlineto{\pgfqpoint{4.188851in}{1.788141in}}%
\pgfpathlineto{\pgfqpoint{4.180778in}{1.777563in}}%
\pgfpathclose%
\pgfusepath{fill}%
\end{pgfscope}%
\begin{pgfscope}%
\pgfpathrectangle{\pgfqpoint{1.150000in}{0.150000in}}{\pgfqpoint{5.700000in}{5.700000in}}%
\pgfusepath{clip}%
\pgfsetbuttcap%
\pgfsetroundjoin%
\definecolor{currentfill}{rgb}{0.283229,0.120777,0.440584}%
\pgfsetfillcolor{currentfill}%
\pgfsetfillopacity{0.700000}%
\pgfsetlinewidth{0.000000pt}%
\definecolor{currentstroke}{rgb}{0.000000,0.000000,0.000000}%
\pgfsetstrokecolor{currentstroke}%
\pgfsetdash{}{0pt}%
\pgfpathmoveto{\pgfqpoint{4.092142in}{1.733559in}}%
\pgfpathlineto{\pgfqpoint{4.106201in}{1.733745in}}%
\pgfpathlineto{\pgfqpoint{4.120268in}{1.734004in}}%
\pgfpathlineto{\pgfqpoint{4.134345in}{1.734336in}}%
\pgfpathlineto{\pgfqpoint{4.148430in}{1.734740in}}%
\pgfpathlineto{\pgfqpoint{4.156525in}{1.745519in}}%
\pgfpathlineto{\pgfqpoint{4.164615in}{1.756251in}}%
\pgfpathlineto{\pgfqpoint{4.172699in}{1.766933in}}%
\pgfpathlineto{\pgfqpoint{4.180778in}{1.777563in}}%
\pgfpathlineto{\pgfqpoint{4.166701in}{1.777014in}}%
\pgfpathlineto{\pgfqpoint{4.152633in}{1.776538in}}%
\pgfpathlineto{\pgfqpoint{4.138574in}{1.776135in}}%
\pgfpathlineto{\pgfqpoint{4.124524in}{1.775805in}}%
\pgfpathlineto{\pgfqpoint{4.116437in}{1.765311in}}%
\pgfpathlineto{\pgfqpoint{4.108344in}{1.754771in}}%
\pgfpathlineto{\pgfqpoint{4.100246in}{1.744186in}}%
\pgfpathlineto{\pgfqpoint{4.092142in}{1.733559in}}%
\pgfpathclose%
\pgfusepath{fill}%
\end{pgfscope}%
\begin{pgfscope}%
\pgfpathrectangle{\pgfqpoint{1.150000in}{0.150000in}}{\pgfqpoint{5.700000in}{5.700000in}}%
\pgfusepath{clip}%
\pgfsetbuttcap%
\pgfsetroundjoin%
\definecolor{currentfill}{rgb}{0.266580,0.228262,0.514349}%
\pgfsetfillcolor{currentfill}%
\pgfsetfillopacity{0.700000}%
\pgfsetlinewidth{0.000000pt}%
\definecolor{currentstroke}{rgb}{0.000000,0.000000,0.000000}%
\pgfsetstrokecolor{currentstroke}%
\pgfsetdash{}{0pt}%
\pgfpathmoveto{\pgfqpoint{2.207327in}{2.012556in}}%
\pgfpathlineto{\pgfqpoint{2.221267in}{1.999802in}}%
\pgfpathlineto{\pgfqpoint{2.235205in}{1.987165in}}%
\pgfpathlineto{\pgfqpoint{2.249140in}{1.974644in}}%
\pgfpathlineto{\pgfqpoint{2.263073in}{1.962239in}}%
\pgfpathlineto{\pgfqpoint{2.272252in}{1.959767in}}%
\pgfpathlineto{\pgfqpoint{2.281410in}{1.957606in}}%
\pgfpathlineto{\pgfqpoint{2.290547in}{1.955747in}}%
\pgfpathlineto{\pgfqpoint{2.299663in}{1.954184in}}%
\pgfpathlineto{\pgfqpoint{2.285773in}{1.966163in}}%
\pgfpathlineto{\pgfqpoint{2.271881in}{1.978257in}}%
\pgfpathlineto{\pgfqpoint{2.257987in}{1.990466in}}%
\pgfpathlineto{\pgfqpoint{2.244091in}{2.002792in}}%
\pgfpathlineto{\pgfqpoint{2.234933in}{2.004774in}}%
\pgfpathlineto{\pgfqpoint{2.225752in}{2.007057in}}%
\pgfpathlineto{\pgfqpoint{2.216551in}{2.009649in}}%
\pgfpathlineto{\pgfqpoint{2.207327in}{2.012556in}}%
\pgfpathclose%
\pgfusepath{fill}%
\end{pgfscope}%
\begin{pgfscope}%
\pgfpathrectangle{\pgfqpoint{1.150000in}{0.150000in}}{\pgfqpoint{5.700000in}{5.700000in}}%
\pgfusepath{clip}%
\pgfsetbuttcap%
\pgfsetroundjoin%
\definecolor{currentfill}{rgb}{0.282656,0.100196,0.422160}%
\pgfsetfillcolor{currentfill}%
\pgfsetfillopacity{0.700000}%
\pgfsetlinewidth{0.000000pt}%
\definecolor{currentstroke}{rgb}{0.000000,0.000000,0.000000}%
\pgfsetstrokecolor{currentstroke}%
\pgfsetdash{}{0pt}%
\pgfpathmoveto{\pgfqpoint{4.003492in}{1.691319in}}%
\pgfpathlineto{\pgfqpoint{4.017524in}{1.691046in}}%
\pgfpathlineto{\pgfqpoint{4.031565in}{1.690846in}}%
\pgfpathlineto{\pgfqpoint{4.045615in}{1.690720in}}%
\pgfpathlineto{\pgfqpoint{4.059673in}{1.690668in}}%
\pgfpathlineto{\pgfqpoint{4.067799in}{1.701444in}}%
\pgfpathlineto{\pgfqpoint{4.075919in}{1.712186in}}%
\pgfpathlineto{\pgfqpoint{4.084033in}{1.722892in}}%
\pgfpathlineto{\pgfqpoint{4.092142in}{1.733559in}}%
\pgfpathlineto{\pgfqpoint{4.078093in}{1.733447in}}%
\pgfpathlineto{\pgfqpoint{4.064052in}{1.733408in}}%
\pgfpathlineto{\pgfqpoint{4.050020in}{1.733442in}}%
\pgfpathlineto{\pgfqpoint{4.035996in}{1.733551in}}%
\pgfpathlineto{\pgfqpoint{4.027878in}{1.723040in}}%
\pgfpathlineto{\pgfqpoint{4.019755in}{1.712497in}}%
\pgfpathlineto{\pgfqpoint{4.011626in}{1.701922in}}%
\pgfpathlineto{\pgfqpoint{4.003492in}{1.691319in}}%
\pgfpathclose%
\pgfusepath{fill}%
\end{pgfscope}%
\begin{pgfscope}%
\pgfpathrectangle{\pgfqpoint{1.150000in}{0.150000in}}{\pgfqpoint{5.700000in}{5.700000in}}%
\pgfusepath{clip}%
\pgfsetbuttcap%
\pgfsetroundjoin%
\definecolor{currentfill}{rgb}{0.277941,0.056324,0.381191}%
\pgfsetfillcolor{currentfill}%
\pgfsetfillopacity{0.700000}%
\pgfsetlinewidth{0.000000pt}%
\definecolor{currentstroke}{rgb}{0.000000,0.000000,0.000000}%
\pgfsetstrokecolor{currentstroke}%
\pgfsetdash{}{0pt}%
\pgfpathmoveto{\pgfqpoint{2.778991in}{1.636227in}}%
\pgfpathlineto{\pgfqpoint{2.792851in}{1.628018in}}%
\pgfpathlineto{\pgfqpoint{2.806714in}{1.619900in}}%
\pgfpathlineto{\pgfqpoint{2.820578in}{1.611874in}}%
\pgfpathlineto{\pgfqpoint{2.834445in}{1.603938in}}%
\pgfpathlineto{\pgfqpoint{2.843158in}{1.607219in}}%
\pgfpathlineto{\pgfqpoint{2.851858in}{1.610709in}}%
\pgfpathlineto{\pgfqpoint{2.860544in}{1.614402in}}%
\pgfpathlineto{\pgfqpoint{2.869218in}{1.618292in}}%
\pgfpathlineto{\pgfqpoint{2.855380in}{1.625855in}}%
\pgfpathlineto{\pgfqpoint{2.841545in}{1.633508in}}%
\pgfpathlineto{\pgfqpoint{2.827712in}{1.641253in}}%
\pgfpathlineto{\pgfqpoint{2.813881in}{1.649089in}}%
\pgfpathlineto{\pgfqpoint{2.805179in}{1.645564in}}%
\pgfpathlineto{\pgfqpoint{2.796463in}{1.642241in}}%
\pgfpathlineto{\pgfqpoint{2.787734in}{1.639127in}}%
\pgfpathlineto{\pgfqpoint{2.778991in}{1.636227in}}%
\pgfpathclose%
\pgfusepath{fill}%
\end{pgfscope}%
\begin{pgfscope}%
\pgfpathrectangle{\pgfqpoint{1.150000in}{0.150000in}}{\pgfqpoint{5.700000in}{5.700000in}}%
\pgfusepath{clip}%
\pgfsetbuttcap%
\pgfsetroundjoin%
\definecolor{currentfill}{rgb}{0.268510,0.009605,0.335427}%
\pgfsetfillcolor{currentfill}%
\pgfsetfillopacity{0.700000}%
\pgfsetlinewidth{0.000000pt}%
\definecolor{currentstroke}{rgb}{0.000000,0.000000,0.000000}%
\pgfsetstrokecolor{currentstroke}%
\pgfsetdash{}{0pt}%
\pgfpathmoveto{\pgfqpoint{3.503866in}{1.533514in}}%
\pgfpathlineto{\pgfqpoint{3.517780in}{1.530301in}}%
\pgfpathlineto{\pgfqpoint{3.531700in}{1.527165in}}%
\pgfpathlineto{\pgfqpoint{3.545627in}{1.524107in}}%
\pgfpathlineto{\pgfqpoint{3.559560in}{1.521125in}}%
\pgfpathlineto{\pgfqpoint{3.567869in}{1.530332in}}%
\pgfpathlineto{\pgfqpoint{3.576171in}{1.539599in}}%
\pgfpathlineto{\pgfqpoint{3.584466in}{1.548921in}}%
\pgfpathlineto{\pgfqpoint{3.592755in}{1.558293in}}%
\pgfpathlineto{\pgfqpoint{3.578837in}{1.561008in}}%
\pgfpathlineto{\pgfqpoint{3.564924in}{1.563800in}}%
\pgfpathlineto{\pgfqpoint{3.551018in}{1.566668in}}%
\pgfpathlineto{\pgfqpoint{3.537119in}{1.569615in}}%
\pgfpathlineto{\pgfqpoint{3.528816in}{1.560501in}}%
\pgfpathlineto{\pgfqpoint{3.520506in}{1.551443in}}%
\pgfpathlineto{\pgfqpoint{3.512189in}{1.542446in}}%
\pgfpathlineto{\pgfqpoint{3.503866in}{1.533514in}}%
\pgfpathclose%
\pgfusepath{fill}%
\end{pgfscope}%
\begin{pgfscope}%
\pgfpathrectangle{\pgfqpoint{1.150000in}{0.150000in}}{\pgfqpoint{5.700000in}{5.700000in}}%
\pgfusepath{clip}%
\pgfsetbuttcap%
\pgfsetroundjoin%
\definecolor{currentfill}{rgb}{0.280894,0.078907,0.402329}%
\pgfsetfillcolor{currentfill}%
\pgfsetfillopacity{0.700000}%
\pgfsetlinewidth{0.000000pt}%
\definecolor{currentstroke}{rgb}{0.000000,0.000000,0.000000}%
\pgfsetstrokecolor{currentstroke}%
\pgfsetdash{}{0pt}%
\pgfpathmoveto{\pgfqpoint{3.914817in}{1.651236in}}%
\pgfpathlineto{\pgfqpoint{3.928825in}{1.650482in}}%
\pgfpathlineto{\pgfqpoint{3.942842in}{1.649802in}}%
\pgfpathlineto{\pgfqpoint{3.956867in}{1.649196in}}%
\pgfpathlineto{\pgfqpoint{3.970901in}{1.648664in}}%
\pgfpathlineto{\pgfqpoint{3.979057in}{1.659359in}}%
\pgfpathlineto{\pgfqpoint{3.987207in}{1.670035in}}%
\pgfpathlineto{\pgfqpoint{3.995352in}{1.680689in}}%
\pgfpathlineto{\pgfqpoint{4.003492in}{1.691319in}}%
\pgfpathlineto{\pgfqpoint{3.989468in}{1.691666in}}%
\pgfpathlineto{\pgfqpoint{3.975452in}{1.692086in}}%
\pgfpathlineto{\pgfqpoint{3.961445in}{1.692581in}}%
\pgfpathlineto{\pgfqpoint{3.947446in}{1.693150in}}%
\pgfpathlineto{\pgfqpoint{3.939297in}{1.682697in}}%
\pgfpathlineto{\pgfqpoint{3.931142in}{1.672226in}}%
\pgfpathlineto{\pgfqpoint{3.922982in}{1.661738in}}%
\pgfpathlineto{\pgfqpoint{3.914817in}{1.651236in}}%
\pgfpathclose%
\pgfusepath{fill}%
\end{pgfscope}%
\begin{pgfscope}%
\pgfpathrectangle{\pgfqpoint{1.150000in}{0.150000in}}{\pgfqpoint{5.700000in}{5.700000in}}%
\pgfusepath{clip}%
\pgfsetbuttcap%
\pgfsetroundjoin%
\definecolor{currentfill}{rgb}{0.267004,0.004874,0.329415}%
\pgfsetfillcolor{currentfill}%
\pgfsetfillopacity{0.700000}%
\pgfsetlinewidth{0.000000pt}%
\definecolor{currentstroke}{rgb}{0.000000,0.000000,0.000000}%
\pgfsetstrokecolor{currentstroke}%
\pgfsetdash{}{0pt}%
\pgfpathmoveto{\pgfqpoint{3.125185in}{1.532260in}}%
\pgfpathlineto{\pgfqpoint{3.139053in}{1.526519in}}%
\pgfpathlineto{\pgfqpoint{3.152925in}{1.520862in}}%
\pgfpathlineto{\pgfqpoint{3.166801in}{1.515287in}}%
\pgfpathlineto{\pgfqpoint{3.180682in}{1.509794in}}%
\pgfpathlineto{\pgfqpoint{3.189176in}{1.516274in}}%
\pgfpathlineto{\pgfqpoint{3.197661in}{1.522892in}}%
\pgfpathlineto{\pgfqpoint{3.206136in}{1.529644in}}%
\pgfpathlineto{\pgfqpoint{3.214602in}{1.536525in}}%
\pgfpathlineto{\pgfqpoint{3.200743in}{1.541689in}}%
\pgfpathlineto{\pgfqpoint{3.186888in}{1.546935in}}%
\pgfpathlineto{\pgfqpoint{3.173037in}{1.552263in}}%
\pgfpathlineto{\pgfqpoint{3.159191in}{1.557675in}}%
\pgfpathlineto{\pgfqpoint{3.150704in}{1.551115in}}%
\pgfpathlineto{\pgfqpoint{3.142208in}{1.544689in}}%
\pgfpathlineto{\pgfqpoint{3.133701in}{1.538403in}}%
\pgfpathlineto{\pgfqpoint{3.125185in}{1.532260in}}%
\pgfpathclose%
\pgfusepath{fill}%
\end{pgfscope}%
\begin{pgfscope}%
\pgfpathrectangle{\pgfqpoint{1.150000in}{0.150000in}}{\pgfqpoint{5.700000in}{5.700000in}}%
\pgfusepath{clip}%
\pgfsetbuttcap%
\pgfsetroundjoin%
\definecolor{currentfill}{rgb}{0.282910,0.105393,0.426902}%
\pgfsetfillcolor{currentfill}%
\pgfsetfillopacity{0.700000}%
\pgfsetlinewidth{0.000000pt}%
\definecolor{currentstroke}{rgb}{0.000000,0.000000,0.000000}%
\pgfsetstrokecolor{currentstroke}%
\pgfsetdash{}{0pt}%
\pgfpathmoveto{\pgfqpoint{2.577269in}{1.737490in}}%
\pgfpathlineto{\pgfqpoint{2.591147in}{1.727742in}}%
\pgfpathlineto{\pgfqpoint{2.605027in}{1.718092in}}%
\pgfpathlineto{\pgfqpoint{2.618907in}{1.708541in}}%
\pgfpathlineto{\pgfqpoint{2.632788in}{1.699086in}}%
\pgfpathlineto{\pgfqpoint{2.641656in}{1.700262in}}%
\pgfpathlineto{\pgfqpoint{2.650509in}{1.701688in}}%
\pgfpathlineto{\pgfqpoint{2.659346in}{1.703356in}}%
\pgfpathlineto{\pgfqpoint{2.668167in}{1.705261in}}%
\pgfpathlineto{\pgfqpoint{2.654320in}{1.714319in}}%
\pgfpathlineto{\pgfqpoint{2.640474in}{1.723474in}}%
\pgfpathlineto{\pgfqpoint{2.626630in}{1.732726in}}%
\pgfpathlineto{\pgfqpoint{2.612786in}{1.742077in}}%
\pgfpathlineto{\pgfqpoint{2.603931in}{1.740560in}}%
\pgfpathlineto{\pgfqpoint{2.595060in}{1.739286in}}%
\pgfpathlineto{\pgfqpoint{2.586173in}{1.738261in}}%
\pgfpathlineto{\pgfqpoint{2.577269in}{1.737490in}}%
\pgfpathclose%
\pgfusepath{fill}%
\end{pgfscope}%
\begin{pgfscope}%
\pgfpathrectangle{\pgfqpoint{1.150000in}{0.150000in}}{\pgfqpoint{5.700000in}{5.700000in}}%
\pgfusepath{clip}%
\pgfsetbuttcap%
\pgfsetroundjoin%
\definecolor{currentfill}{rgb}{0.267004,0.004874,0.329415}%
\pgfsetfillcolor{currentfill}%
\pgfsetfillopacity{0.700000}%
\pgfsetlinewidth{0.000000pt}%
\definecolor{currentstroke}{rgb}{0.000000,0.000000,0.000000}%
\pgfsetstrokecolor{currentstroke}%
\pgfsetdash{}{0pt}%
\pgfpathmoveto{\pgfqpoint{3.270085in}{1.516688in}}%
\pgfpathlineto{\pgfqpoint{3.283968in}{1.511931in}}%
\pgfpathlineto{\pgfqpoint{3.297855in}{1.507255in}}%
\pgfpathlineto{\pgfqpoint{3.311748in}{1.502659in}}%
\pgfpathlineto{\pgfqpoint{3.325645in}{1.498144in}}%
\pgfpathlineto{\pgfqpoint{3.334064in}{1.505780in}}%
\pgfpathlineto{\pgfqpoint{3.342473in}{1.513525in}}%
\pgfpathlineto{\pgfqpoint{3.350875in}{1.521375in}}%
\pgfpathlineto{\pgfqpoint{3.359268in}{1.529325in}}%
\pgfpathlineto{\pgfqpoint{3.345389in}{1.533533in}}%
\pgfpathlineto{\pgfqpoint{3.331515in}{1.537821in}}%
\pgfpathlineto{\pgfqpoint{3.317646in}{1.542189in}}%
\pgfpathlineto{\pgfqpoint{3.303782in}{1.546637in}}%
\pgfpathlineto{\pgfqpoint{3.295371in}{1.538988in}}%
\pgfpathlineto{\pgfqpoint{3.286951in}{1.531444in}}%
\pgfpathlineto{\pgfqpoint{3.278522in}{1.524009in}}%
\pgfpathlineto{\pgfqpoint{3.270085in}{1.516688in}}%
\pgfpathclose%
\pgfusepath{fill}%
\end{pgfscope}%
\begin{pgfscope}%
\pgfpathrectangle{\pgfqpoint{1.150000in}{0.150000in}}{\pgfqpoint{5.700000in}{5.700000in}}%
\pgfusepath{clip}%
\pgfsetbuttcap%
\pgfsetroundjoin%
\definecolor{currentfill}{rgb}{0.271828,0.209303,0.504434}%
\pgfsetfillcolor{currentfill}%
\pgfsetfillopacity{0.700000}%
\pgfsetlinewidth{0.000000pt}%
\definecolor{currentstroke}{rgb}{0.000000,0.000000,0.000000}%
\pgfsetstrokecolor{currentstroke}%
\pgfsetdash{}{0pt}%
\pgfpathmoveto{\pgfqpoint{2.263073in}{1.962239in}}%
\pgfpathlineto{\pgfqpoint{2.277004in}{1.949948in}}%
\pgfpathlineto{\pgfqpoint{2.290934in}{1.937770in}}%
\pgfpathlineto{\pgfqpoint{2.304862in}{1.925706in}}%
\pgfpathlineto{\pgfqpoint{2.318788in}{1.913753in}}%
\pgfpathlineto{\pgfqpoint{2.327924in}{1.911716in}}%
\pgfpathlineto{\pgfqpoint{2.337039in}{1.909983in}}%
\pgfpathlineto{\pgfqpoint{2.346134in}{1.908547in}}%
\pgfpathlineto{\pgfqpoint{2.355209in}{1.907402in}}%
\pgfpathlineto{\pgfqpoint{2.341324in}{1.918929in}}%
\pgfpathlineto{\pgfqpoint{2.327439in}{1.930568in}}%
\pgfpathlineto{\pgfqpoint{2.313552in}{1.942320in}}%
\pgfpathlineto{\pgfqpoint{2.299663in}{1.954184in}}%
\pgfpathlineto{\pgfqpoint{2.290547in}{1.955747in}}%
\pgfpathlineto{\pgfqpoint{2.281410in}{1.957606in}}%
\pgfpathlineto{\pgfqpoint{2.272252in}{1.959767in}}%
\pgfpathlineto{\pgfqpoint{2.263073in}{1.962239in}}%
\pgfpathclose%
\pgfusepath{fill}%
\end{pgfscope}%
\begin{pgfscope}%
\pgfpathrectangle{\pgfqpoint{1.150000in}{0.150000in}}{\pgfqpoint{5.700000in}{5.700000in}}%
\pgfusepath{clip}%
\pgfsetbuttcap%
\pgfsetroundjoin%
\definecolor{currentfill}{rgb}{0.271305,0.019942,0.347269}%
\pgfsetfillcolor{currentfill}%
\pgfsetfillopacity{0.700000}%
\pgfsetlinewidth{0.000000pt}%
\definecolor{currentstroke}{rgb}{0.000000,0.000000,0.000000}%
\pgfsetstrokecolor{currentstroke}%
\pgfsetdash{}{0pt}%
\pgfpathmoveto{\pgfqpoint{2.980017in}{1.560991in}}%
\pgfpathlineto{\pgfqpoint{2.993880in}{1.554223in}}%
\pgfpathlineto{\pgfqpoint{3.007746in}{1.547540in}}%
\pgfpathlineto{\pgfqpoint{3.021616in}{1.540943in}}%
\pgfpathlineto{\pgfqpoint{3.035489in}{1.534431in}}%
\pgfpathlineto{\pgfqpoint{3.044071in}{1.539589in}}%
\pgfpathlineto{\pgfqpoint{3.052643in}{1.544916in}}%
\pgfpathlineto{\pgfqpoint{3.061203in}{1.550408in}}%
\pgfpathlineto{\pgfqpoint{3.069753in}{1.556060in}}%
\pgfpathlineto{\pgfqpoint{3.055904in}{1.562221in}}%
\pgfpathlineto{\pgfqpoint{3.042059in}{1.568468in}}%
\pgfpathlineto{\pgfqpoint{3.028217in}{1.574800in}}%
\pgfpathlineto{\pgfqpoint{3.014379in}{1.581219in}}%
\pgfpathlineto{\pgfqpoint{3.005806in}{1.575909in}}%
\pgfpathlineto{\pgfqpoint{2.997221in}{1.570765in}}%
\pgfpathlineto{\pgfqpoint{2.988625in}{1.565790in}}%
\pgfpathlineto{\pgfqpoint{2.980017in}{1.560991in}}%
\pgfpathclose%
\pgfusepath{fill}%
\end{pgfscope}%
\begin{pgfscope}%
\pgfpathrectangle{\pgfqpoint{1.150000in}{0.150000in}}{\pgfqpoint{5.700000in}{5.700000in}}%
\pgfusepath{clip}%
\pgfsetbuttcap%
\pgfsetroundjoin%
\definecolor{currentfill}{rgb}{0.278791,0.062145,0.386592}%
\pgfsetfillcolor{currentfill}%
\pgfsetfillopacity{0.700000}%
\pgfsetlinewidth{0.000000pt}%
\definecolor{currentstroke}{rgb}{0.000000,0.000000,0.000000}%
\pgfsetstrokecolor{currentstroke}%
\pgfsetdash{}{0pt}%
\pgfpathmoveto{\pgfqpoint{3.826104in}{1.613728in}}%
\pgfpathlineto{\pgfqpoint{3.840091in}{1.612470in}}%
\pgfpathlineto{\pgfqpoint{3.854086in}{1.611287in}}%
\pgfpathlineto{\pgfqpoint{3.868089in}{1.610179in}}%
\pgfpathlineto{\pgfqpoint{3.882099in}{1.609145in}}%
\pgfpathlineto{\pgfqpoint{3.890287in}{1.619675in}}%
\pgfpathlineto{\pgfqpoint{3.898469in}{1.630202in}}%
\pgfpathlineto{\pgfqpoint{3.906646in}{1.640723in}}%
\pgfpathlineto{\pgfqpoint{3.914817in}{1.651236in}}%
\pgfpathlineto{\pgfqpoint{3.900816in}{1.652065in}}%
\pgfpathlineto{\pgfqpoint{3.886824in}{1.652967in}}%
\pgfpathlineto{\pgfqpoint{3.872839in}{1.653944in}}%
\pgfpathlineto{\pgfqpoint{3.858863in}{1.654996in}}%
\pgfpathlineto{\pgfqpoint{3.850681in}{1.644681in}}%
\pgfpathlineto{\pgfqpoint{3.842495in}{1.634363in}}%
\pgfpathlineto{\pgfqpoint{3.834302in}{1.624044in}}%
\pgfpathlineto{\pgfqpoint{3.826104in}{1.613728in}}%
\pgfpathclose%
\pgfusepath{fill}%
\end{pgfscope}%
\begin{pgfscope}%
\pgfpathrectangle{\pgfqpoint{1.150000in}{0.150000in}}{\pgfqpoint{5.700000in}{5.700000in}}%
\pgfusepath{clip}%
\pgfsetbuttcap%
\pgfsetroundjoin%
\definecolor{currentfill}{rgb}{0.180629,0.429975,0.557282}%
\pgfsetfillcolor{currentfill}%
\pgfsetfillopacity{0.700000}%
\pgfsetlinewidth{0.000000pt}%
\definecolor{currentstroke}{rgb}{0.000000,0.000000,0.000000}%
\pgfsetstrokecolor{currentstroke}%
\pgfsetdash{}{0pt}%
\pgfpathmoveto{\pgfqpoint{5.567624in}{2.464089in}}%
\pgfpathlineto{\pgfqpoint{5.582283in}{2.468878in}}%
\pgfpathlineto{\pgfqpoint{5.596955in}{2.473737in}}%
\pgfpathlineto{\pgfqpoint{5.611641in}{2.478666in}}%
\pgfpathlineto{\pgfqpoint{5.619071in}{2.482984in}}%
\pgfpathlineto{\pgfqpoint{5.626492in}{2.487217in}}%
\pgfpathlineto{\pgfqpoint{5.633904in}{2.491369in}}%
\pgfpathlineto{\pgfqpoint{5.641307in}{2.495444in}}%
\pgfpathlineto{\pgfqpoint{5.626642in}{2.490716in}}%
\pgfpathlineto{\pgfqpoint{5.611990in}{2.486059in}}%
\pgfpathlineto{\pgfqpoint{5.597352in}{2.481471in}}%
\pgfpathlineto{\pgfqpoint{5.589933in}{2.477239in}}%
\pgfpathlineto{\pgfqpoint{5.582506in}{2.472934in}}%
\pgfpathlineto{\pgfqpoint{5.575069in}{2.468552in}}%
\pgfpathlineto{\pgfqpoint{5.567624in}{2.464089in}}%
\pgfpathclose%
\pgfusepath{fill}%
\end{pgfscope}%
\begin{pgfscope}%
\pgfpathrectangle{\pgfqpoint{1.150000in}{0.150000in}}{\pgfqpoint{5.700000in}{5.700000in}}%
\pgfusepath{clip}%
\pgfsetbuttcap%
\pgfsetroundjoin%
\definecolor{currentfill}{rgb}{0.276022,0.044167,0.370164}%
\pgfsetfillcolor{currentfill}%
\pgfsetfillopacity{0.700000}%
\pgfsetlinewidth{0.000000pt}%
\definecolor{currentstroke}{rgb}{0.000000,0.000000,0.000000}%
\pgfsetstrokecolor{currentstroke}%
\pgfsetdash{}{0pt}%
\pgfpathmoveto{\pgfqpoint{3.737337in}{1.579230in}}%
\pgfpathlineto{\pgfqpoint{3.751305in}{1.577446in}}%
\pgfpathlineto{\pgfqpoint{3.765281in}{1.575738in}}%
\pgfpathlineto{\pgfqpoint{3.779264in}{1.574104in}}%
\pgfpathlineto{\pgfqpoint{3.793254in}{1.572546in}}%
\pgfpathlineto{\pgfqpoint{3.801475in}{1.582822in}}%
\pgfpathlineto{\pgfqpoint{3.809691in}{1.593114in}}%
\pgfpathlineto{\pgfqpoint{3.817900in}{1.603416in}}%
\pgfpathlineto{\pgfqpoint{3.826104in}{1.613728in}}%
\pgfpathlineto{\pgfqpoint{3.812125in}{1.615060in}}%
\pgfpathlineto{\pgfqpoint{3.798153in}{1.616467in}}%
\pgfpathlineto{\pgfqpoint{3.784188in}{1.617949in}}%
\pgfpathlineto{\pgfqpoint{3.770232in}{1.619507in}}%
\pgfpathlineto{\pgfqpoint{3.762017in}{1.609414in}}%
\pgfpathlineto{\pgfqpoint{3.753796in}{1.599335in}}%
\pgfpathlineto{\pgfqpoint{3.745570in}{1.589272in}}%
\pgfpathlineto{\pgfqpoint{3.737337in}{1.579230in}}%
\pgfpathclose%
\pgfusepath{fill}%
\end{pgfscope}%
\begin{pgfscope}%
\pgfpathrectangle{\pgfqpoint{1.150000in}{0.150000in}}{\pgfqpoint{5.700000in}{5.700000in}}%
\pgfusepath{clip}%
\pgfsetbuttcap%
\pgfsetroundjoin%
\definecolor{currentfill}{rgb}{0.267004,0.004874,0.329415}%
\pgfsetfillcolor{currentfill}%
\pgfsetfillopacity{0.700000}%
\pgfsetlinewidth{0.000000pt}%
\definecolor{currentstroke}{rgb}{0.000000,0.000000,0.000000}%
\pgfsetstrokecolor{currentstroke}%
\pgfsetdash{}{0pt}%
\pgfpathmoveto{\pgfqpoint{3.414840in}{1.513287in}}%
\pgfpathlineto{\pgfqpoint{3.428746in}{1.509474in}}%
\pgfpathlineto{\pgfqpoint{3.442659in}{1.505740in}}%
\pgfpathlineto{\pgfqpoint{3.456577in}{1.502084in}}%
\pgfpathlineto{\pgfqpoint{3.470501in}{1.498506in}}%
\pgfpathlineto{\pgfqpoint{3.478853in}{1.507141in}}%
\pgfpathlineto{\pgfqpoint{3.487198in}{1.515857in}}%
\pgfpathlineto{\pgfqpoint{3.495535in}{1.524649in}}%
\pgfpathlineto{\pgfqpoint{3.503866in}{1.533514in}}%
\pgfpathlineto{\pgfqpoint{3.489958in}{1.536804in}}%
\pgfpathlineto{\pgfqpoint{3.476055in}{1.540173in}}%
\pgfpathlineto{\pgfqpoint{3.462159in}{1.543620in}}%
\pgfpathlineto{\pgfqpoint{3.448269in}{1.547145in}}%
\pgfpathlineto{\pgfqpoint{3.439923in}{1.538560in}}%
\pgfpathlineto{\pgfqpoint{3.431569in}{1.530052in}}%
\pgfpathlineto{\pgfqpoint{3.423208in}{1.521626in}}%
\pgfpathlineto{\pgfqpoint{3.414840in}{1.513287in}}%
\pgfpathclose%
\pgfusepath{fill}%
\end{pgfscope}%
\begin{pgfscope}%
\pgfpathrectangle{\pgfqpoint{1.150000in}{0.150000in}}{\pgfqpoint{5.700000in}{5.700000in}}%
\pgfusepath{clip}%
\pgfsetbuttcap%
\pgfsetroundjoin%
\definecolor{currentfill}{rgb}{0.276194,0.190074,0.493001}%
\pgfsetfillcolor{currentfill}%
\pgfsetfillopacity{0.700000}%
\pgfsetlinewidth{0.000000pt}%
\definecolor{currentstroke}{rgb}{0.000000,0.000000,0.000000}%
\pgfsetstrokecolor{currentstroke}%
\pgfsetdash{}{0pt}%
\pgfpathmoveto{\pgfqpoint{2.318788in}{1.913753in}}%
\pgfpathlineto{\pgfqpoint{2.332712in}{1.901911in}}%
\pgfpathlineto{\pgfqpoint{2.346636in}{1.890180in}}%
\pgfpathlineto{\pgfqpoint{2.360558in}{1.878558in}}%
\pgfpathlineto{\pgfqpoint{2.374479in}{1.867044in}}%
\pgfpathlineto{\pgfqpoint{2.383572in}{1.865440in}}%
\pgfpathlineto{\pgfqpoint{2.392646in}{1.864135in}}%
\pgfpathlineto{\pgfqpoint{2.401700in}{1.863120in}}%
\pgfpathlineto{\pgfqpoint{2.410735in}{1.862390in}}%
\pgfpathlineto{\pgfqpoint{2.396855in}{1.873480in}}%
\pgfpathlineto{\pgfqpoint{2.382974in}{1.884678in}}%
\pgfpathlineto{\pgfqpoint{2.369092in}{1.895985in}}%
\pgfpathlineto{\pgfqpoint{2.355209in}{1.907402in}}%
\pgfpathlineto{\pgfqpoint{2.346134in}{1.908547in}}%
\pgfpathlineto{\pgfqpoint{2.337039in}{1.909983in}}%
\pgfpathlineto{\pgfqpoint{2.327924in}{1.911716in}}%
\pgfpathlineto{\pgfqpoint{2.318788in}{1.913753in}}%
\pgfpathclose%
\pgfusepath{fill}%
\end{pgfscope}%
\begin{pgfscope}%
\pgfpathrectangle{\pgfqpoint{1.150000in}{0.150000in}}{\pgfqpoint{5.700000in}{5.700000in}}%
\pgfusepath{clip}%
\pgfsetbuttcap%
\pgfsetroundjoin%
\definecolor{currentfill}{rgb}{0.276022,0.044167,0.370164}%
\pgfsetfillcolor{currentfill}%
\pgfsetfillopacity{0.700000}%
\pgfsetlinewidth{0.000000pt}%
\definecolor{currentstroke}{rgb}{0.000000,0.000000,0.000000}%
\pgfsetstrokecolor{currentstroke}%
\pgfsetdash{}{0pt}%
\pgfpathmoveto{\pgfqpoint{2.834445in}{1.603938in}}%
\pgfpathlineto{\pgfqpoint{2.848314in}{1.596092in}}%
\pgfpathlineto{\pgfqpoint{2.862185in}{1.588337in}}%
\pgfpathlineto{\pgfqpoint{2.876059in}{1.580670in}}%
\pgfpathlineto{\pgfqpoint{2.889935in}{1.573092in}}%
\pgfpathlineto{\pgfqpoint{2.898619in}{1.576754in}}%
\pgfpathlineto{\pgfqpoint{2.907290in}{1.580619in}}%
\pgfpathlineto{\pgfqpoint{2.915949in}{1.584681in}}%
\pgfpathlineto{\pgfqpoint{2.924595in}{1.588936in}}%
\pgfpathlineto{\pgfqpoint{2.910747in}{1.596141in}}%
\pgfpathlineto{\pgfqpoint{2.896901in}{1.603435in}}%
\pgfpathlineto{\pgfqpoint{2.883058in}{1.610819in}}%
\pgfpathlineto{\pgfqpoint{2.869218in}{1.618292in}}%
\pgfpathlineto{\pgfqpoint{2.860544in}{1.614402in}}%
\pgfpathlineto{\pgfqpoint{2.851858in}{1.610709in}}%
\pgfpathlineto{\pgfqpoint{2.843158in}{1.607219in}}%
\pgfpathlineto{\pgfqpoint{2.834445in}{1.603938in}}%
\pgfpathclose%
\pgfusepath{fill}%
\end{pgfscope}%
\begin{pgfscope}%
\pgfpathrectangle{\pgfqpoint{1.150000in}{0.150000in}}{\pgfqpoint{5.700000in}{5.700000in}}%
\pgfusepath{clip}%
\pgfsetbuttcap%
\pgfsetroundjoin%
\definecolor{currentfill}{rgb}{0.183898,0.422383,0.556944}%
\pgfsetfillcolor{currentfill}%
\pgfsetfillopacity{0.700000}%
\pgfsetlinewidth{0.000000pt}%
\definecolor{currentstroke}{rgb}{0.000000,0.000000,0.000000}%
\pgfsetstrokecolor{currentstroke}%
\pgfsetdash{}{0pt}%
\pgfpathmoveto{\pgfqpoint{5.479178in}{2.426187in}}%
\pgfpathlineto{\pgfqpoint{5.493802in}{2.430874in}}%
\pgfpathlineto{\pgfqpoint{5.508439in}{2.435632in}}%
\pgfpathlineto{\pgfqpoint{5.523089in}{2.440460in}}%
\pgfpathlineto{\pgfqpoint{5.537753in}{2.445359in}}%
\pgfpathlineto{\pgfqpoint{5.545235in}{2.450180in}}%
\pgfpathlineto{\pgfqpoint{5.552707in}{2.454906in}}%
\pgfpathlineto{\pgfqpoint{5.560170in}{2.459541in}}%
\pgfpathlineto{\pgfqpoint{5.567624in}{2.464089in}}%
\pgfpathlineto{\pgfqpoint{5.552979in}{2.459370in}}%
\pgfpathlineto{\pgfqpoint{5.538348in}{2.454722in}}%
\pgfpathlineto{\pgfqpoint{5.523729in}{2.450144in}}%
\pgfpathlineto{\pgfqpoint{5.509125in}{2.445636in}}%
\pgfpathlineto{\pgfqpoint{5.501651in}{2.440901in}}%
\pgfpathlineto{\pgfqpoint{5.494169in}{2.436083in}}%
\pgfpathlineto{\pgfqpoint{5.486678in}{2.431180in}}%
\pgfpathlineto{\pgfqpoint{5.479178in}{2.426187in}}%
\pgfpathclose%
\pgfusepath{fill}%
\end{pgfscope}%
\begin{pgfscope}%
\pgfpathrectangle{\pgfqpoint{1.150000in}{0.150000in}}{\pgfqpoint{5.700000in}{5.700000in}}%
\pgfusepath{clip}%
\pgfsetbuttcap%
\pgfsetroundjoin%
\definecolor{currentfill}{rgb}{0.282327,0.094955,0.417331}%
\pgfsetfillcolor{currentfill}%
\pgfsetfillopacity{0.700000}%
\pgfsetlinewidth{0.000000pt}%
\definecolor{currentstroke}{rgb}{0.000000,0.000000,0.000000}%
\pgfsetstrokecolor{currentstroke}%
\pgfsetdash{}{0pt}%
\pgfpathmoveto{\pgfqpoint{2.632788in}{1.699086in}}%
\pgfpathlineto{\pgfqpoint{2.646669in}{1.689729in}}%
\pgfpathlineto{\pgfqpoint{2.660552in}{1.680467in}}%
\pgfpathlineto{\pgfqpoint{2.674436in}{1.671301in}}%
\pgfpathlineto{\pgfqpoint{2.688321in}{1.662230in}}%
\pgfpathlineto{\pgfqpoint{2.697155in}{1.663811in}}%
\pgfpathlineto{\pgfqpoint{2.705974in}{1.665635in}}%
\pgfpathlineto{\pgfqpoint{2.714778in}{1.667697in}}%
\pgfpathlineto{\pgfqpoint{2.723567in}{1.669990in}}%
\pgfpathlineto{\pgfqpoint{2.709715in}{1.678665in}}%
\pgfpathlineto{\pgfqpoint{2.695864in}{1.687435in}}%
\pgfpathlineto{\pgfqpoint{2.682015in}{1.696300in}}%
\pgfpathlineto{\pgfqpoint{2.668167in}{1.705261in}}%
\pgfpathlineto{\pgfqpoint{2.659346in}{1.703356in}}%
\pgfpathlineto{\pgfqpoint{2.650509in}{1.701688in}}%
\pgfpathlineto{\pgfqpoint{2.641656in}{1.700262in}}%
\pgfpathlineto{\pgfqpoint{2.632788in}{1.699086in}}%
\pgfpathclose%
\pgfusepath{fill}%
\end{pgfscope}%
\begin{pgfscope}%
\pgfpathrectangle{\pgfqpoint{1.150000in}{0.150000in}}{\pgfqpoint{5.700000in}{5.700000in}}%
\pgfusepath{clip}%
\pgfsetbuttcap%
\pgfsetroundjoin%
\definecolor{currentfill}{rgb}{0.272594,0.025563,0.353093}%
\pgfsetfillcolor{currentfill}%
\pgfsetfillopacity{0.700000}%
\pgfsetlinewidth{0.000000pt}%
\definecolor{currentstroke}{rgb}{0.000000,0.000000,0.000000}%
\pgfsetstrokecolor{currentstroke}%
\pgfsetdash{}{0pt}%
\pgfpathmoveto{\pgfqpoint{3.648497in}{1.548201in}}%
\pgfpathlineto{\pgfqpoint{3.662449in}{1.545869in}}%
\pgfpathlineto{\pgfqpoint{3.676408in}{1.543612in}}%
\pgfpathlineto{\pgfqpoint{3.690374in}{1.541431in}}%
\pgfpathlineto{\pgfqpoint{3.704347in}{1.539325in}}%
\pgfpathlineto{\pgfqpoint{3.712604in}{1.549255in}}%
\pgfpathlineto{\pgfqpoint{3.720854in}{1.559217in}}%
\pgfpathlineto{\pgfqpoint{3.729099in}{1.569210in}}%
\pgfpathlineto{\pgfqpoint{3.737337in}{1.579230in}}%
\pgfpathlineto{\pgfqpoint{3.723376in}{1.581088in}}%
\pgfpathlineto{\pgfqpoint{3.709422in}{1.583023in}}%
\pgfpathlineto{\pgfqpoint{3.695476in}{1.585033in}}%
\pgfpathlineto{\pgfqpoint{3.681536in}{1.587119in}}%
\pgfpathlineto{\pgfqpoint{3.673285in}{1.577338in}}%
\pgfpathlineto{\pgfqpoint{3.665029in}{1.567590in}}%
\pgfpathlineto{\pgfqpoint{3.656766in}{1.557876in}}%
\pgfpathlineto{\pgfqpoint{3.648497in}{1.548201in}}%
\pgfpathclose%
\pgfusepath{fill}%
\end{pgfscope}%
\begin{pgfscope}%
\pgfpathrectangle{\pgfqpoint{1.150000in}{0.150000in}}{\pgfqpoint{5.700000in}{5.700000in}}%
\pgfusepath{clip}%
\pgfsetbuttcap%
\pgfsetroundjoin%
\definecolor{currentfill}{rgb}{0.190631,0.407061,0.556089}%
\pgfsetfillcolor{currentfill}%
\pgfsetfillopacity{0.700000}%
\pgfsetlinewidth{0.000000pt}%
\definecolor{currentstroke}{rgb}{0.000000,0.000000,0.000000}%
\pgfsetstrokecolor{currentstroke}%
\pgfsetdash{}{0pt}%
\pgfpathmoveto{\pgfqpoint{5.390658in}{2.386587in}}%
\pgfpathlineto{\pgfqpoint{5.405246in}{2.391150in}}%
\pgfpathlineto{\pgfqpoint{5.419847in}{2.395783in}}%
\pgfpathlineto{\pgfqpoint{5.434461in}{2.400487in}}%
\pgfpathlineto{\pgfqpoint{5.449089in}{2.405262in}}%
\pgfpathlineto{\pgfqpoint{5.456625in}{2.410642in}}%
\pgfpathlineto{\pgfqpoint{5.464152in}{2.415922in}}%
\pgfpathlineto{\pgfqpoint{5.471669in}{2.421102in}}%
\pgfpathlineto{\pgfqpoint{5.479178in}{2.426187in}}%
\pgfpathlineto{\pgfqpoint{5.464568in}{2.421570in}}%
\pgfpathlineto{\pgfqpoint{5.449971in}{2.417024in}}%
\pgfpathlineto{\pgfqpoint{5.435387in}{2.412548in}}%
\pgfpathlineto{\pgfqpoint{5.420816in}{2.408142in}}%
\pgfpathlineto{\pgfqpoint{5.413290in}{2.402891in}}%
\pgfpathlineto{\pgfqpoint{5.405754in}{2.397551in}}%
\pgfpathlineto{\pgfqpoint{5.398210in}{2.392117in}}%
\pgfpathlineto{\pgfqpoint{5.390658in}{2.386587in}}%
\pgfpathclose%
\pgfusepath{fill}%
\end{pgfscope}%
\begin{pgfscope}%
\pgfpathrectangle{\pgfqpoint{1.150000in}{0.150000in}}{\pgfqpoint{5.700000in}{5.700000in}}%
\pgfusepath{clip}%
\pgfsetbuttcap%
\pgfsetroundjoin%
\definecolor{currentfill}{rgb}{0.267004,0.004874,0.329415}%
\pgfsetfillcolor{currentfill}%
\pgfsetfillopacity{0.700000}%
\pgfsetlinewidth{0.000000pt}%
\definecolor{currentstroke}{rgb}{0.000000,0.000000,0.000000}%
\pgfsetstrokecolor{currentstroke}%
\pgfsetdash{}{0pt}%
\pgfpathmoveto{\pgfqpoint{3.180682in}{1.509794in}}%
\pgfpathlineto{\pgfqpoint{3.194567in}{1.504384in}}%
\pgfpathlineto{\pgfqpoint{3.208456in}{1.499055in}}%
\pgfpathlineto{\pgfqpoint{3.222349in}{1.493808in}}%
\pgfpathlineto{\pgfqpoint{3.236248in}{1.488642in}}%
\pgfpathlineto{\pgfqpoint{3.244721in}{1.495458in}}%
\pgfpathlineto{\pgfqpoint{3.253184in}{1.502408in}}%
\pgfpathlineto{\pgfqpoint{3.261639in}{1.509486in}}%
\pgfpathlineto{\pgfqpoint{3.270085in}{1.516688in}}%
\pgfpathlineto{\pgfqpoint{3.256207in}{1.521525in}}%
\pgfpathlineto{\pgfqpoint{3.242334in}{1.526444in}}%
\pgfpathlineto{\pgfqpoint{3.228466in}{1.531444in}}%
\pgfpathlineto{\pgfqpoint{3.214602in}{1.536525in}}%
\pgfpathlineto{\pgfqpoint{3.206136in}{1.529644in}}%
\pgfpathlineto{\pgfqpoint{3.197661in}{1.522892in}}%
\pgfpathlineto{\pgfqpoint{3.189176in}{1.516274in}}%
\pgfpathlineto{\pgfqpoint{3.180682in}{1.509794in}}%
\pgfpathclose%
\pgfusepath{fill}%
\end{pgfscope}%
\begin{pgfscope}%
\pgfpathrectangle{\pgfqpoint{1.150000in}{0.150000in}}{\pgfqpoint{5.700000in}{5.700000in}}%
\pgfusepath{clip}%
\pgfsetbuttcap%
\pgfsetroundjoin%
\definecolor{currentfill}{rgb}{0.279574,0.170599,0.479997}%
\pgfsetfillcolor{currentfill}%
\pgfsetfillopacity{0.700000}%
\pgfsetlinewidth{0.000000pt}%
\definecolor{currentstroke}{rgb}{0.000000,0.000000,0.000000}%
\pgfsetstrokecolor{currentstroke}%
\pgfsetdash{}{0pt}%
\pgfpathmoveto{\pgfqpoint{2.374479in}{1.867044in}}%
\pgfpathlineto{\pgfqpoint{2.388399in}{1.855639in}}%
\pgfpathlineto{\pgfqpoint{2.402318in}{1.844340in}}%
\pgfpathlineto{\pgfqpoint{2.416236in}{1.833148in}}%
\pgfpathlineto{\pgfqpoint{2.430153in}{1.822061in}}%
\pgfpathlineto{\pgfqpoint{2.439206in}{1.820889in}}%
\pgfpathlineto{\pgfqpoint{2.448239in}{1.820009in}}%
\pgfpathlineto{\pgfqpoint{2.457253in}{1.819414in}}%
\pgfpathlineto{\pgfqpoint{2.466249in}{1.819099in}}%
\pgfpathlineto{\pgfqpoint{2.452371in}{1.829763in}}%
\pgfpathlineto{\pgfqpoint{2.438493in}{1.840533in}}%
\pgfpathlineto{\pgfqpoint{2.424614in}{1.851408in}}%
\pgfpathlineto{\pgfqpoint{2.410735in}{1.862390in}}%
\pgfpathlineto{\pgfqpoint{2.401700in}{1.863120in}}%
\pgfpathlineto{\pgfqpoint{2.392646in}{1.864135in}}%
\pgfpathlineto{\pgfqpoint{2.383572in}{1.865440in}}%
\pgfpathlineto{\pgfqpoint{2.374479in}{1.867044in}}%
\pgfpathclose%
\pgfusepath{fill}%
\end{pgfscope}%
\begin{pgfscope}%
\pgfpathrectangle{\pgfqpoint{1.150000in}{0.150000in}}{\pgfqpoint{5.700000in}{5.700000in}}%
\pgfusepath{clip}%
\pgfsetbuttcap%
\pgfsetroundjoin%
\definecolor{currentfill}{rgb}{0.195860,0.395433,0.555276}%
\pgfsetfillcolor{currentfill}%
\pgfsetfillopacity{0.700000}%
\pgfsetlinewidth{0.000000pt}%
\definecolor{currentstroke}{rgb}{0.000000,0.000000,0.000000}%
\pgfsetstrokecolor{currentstroke}%
\pgfsetdash{}{0pt}%
\pgfpathmoveto{\pgfqpoint{5.302073in}{2.345360in}}%
\pgfpathlineto{\pgfqpoint{5.316625in}{2.349776in}}%
\pgfpathlineto{\pgfqpoint{5.331189in}{2.354263in}}%
\pgfpathlineto{\pgfqpoint{5.345767in}{2.358821in}}%
\pgfpathlineto{\pgfqpoint{5.360358in}{2.363449in}}%
\pgfpathlineto{\pgfqpoint{5.367946in}{2.369391in}}%
\pgfpathlineto{\pgfqpoint{5.375525in}{2.375227in}}%
\pgfpathlineto{\pgfqpoint{5.383096in}{2.380958in}}%
\pgfpathlineto{\pgfqpoint{5.390658in}{2.386587in}}%
\pgfpathlineto{\pgfqpoint{5.376083in}{2.382094in}}%
\pgfpathlineto{\pgfqpoint{5.361521in}{2.377672in}}%
\pgfpathlineto{\pgfqpoint{5.346973in}{2.373321in}}%
\pgfpathlineto{\pgfqpoint{5.332437in}{2.369040in}}%
\pgfpathlineto{\pgfqpoint{5.324859in}{2.363267in}}%
\pgfpathlineto{\pgfqpoint{5.317272in}{2.357398in}}%
\pgfpathlineto{\pgfqpoint{5.309677in}{2.351430in}}%
\pgfpathlineto{\pgfqpoint{5.302073in}{2.345360in}}%
\pgfpathclose%
\pgfusepath{fill}%
\end{pgfscope}%
\begin{pgfscope}%
\pgfpathrectangle{\pgfqpoint{1.150000in}{0.150000in}}{\pgfqpoint{5.700000in}{5.700000in}}%
\pgfusepath{clip}%
\pgfsetbuttcap%
\pgfsetroundjoin%
\definecolor{currentfill}{rgb}{0.269944,0.014625,0.341379}%
\pgfsetfillcolor{currentfill}%
\pgfsetfillopacity{0.700000}%
\pgfsetlinewidth{0.000000pt}%
\definecolor{currentstroke}{rgb}{0.000000,0.000000,0.000000}%
\pgfsetstrokecolor{currentstroke}%
\pgfsetdash{}{0pt}%
\pgfpathmoveto{\pgfqpoint{3.035489in}{1.534431in}}%
\pgfpathlineto{\pgfqpoint{3.049366in}{1.528005in}}%
\pgfpathlineto{\pgfqpoint{3.063246in}{1.521663in}}%
\pgfpathlineto{\pgfqpoint{3.077130in}{1.515405in}}%
\pgfpathlineto{\pgfqpoint{3.091017in}{1.509231in}}%
\pgfpathlineto{\pgfqpoint{3.099575in}{1.514746in}}%
\pgfpathlineto{\pgfqpoint{3.108122in}{1.520426in}}%
\pgfpathlineto{\pgfqpoint{3.116659in}{1.526266in}}%
\pgfpathlineto{\pgfqpoint{3.125185in}{1.532260in}}%
\pgfpathlineto{\pgfqpoint{3.111321in}{1.538084in}}%
\pgfpathlineto{\pgfqpoint{3.097461in}{1.543992in}}%
\pgfpathlineto{\pgfqpoint{3.083605in}{1.549984in}}%
\pgfpathlineto{\pgfqpoint{3.069753in}{1.556060in}}%
\pgfpathlineto{\pgfqpoint{3.061203in}{1.550408in}}%
\pgfpathlineto{\pgfqpoint{3.052643in}{1.544916in}}%
\pgfpathlineto{\pgfqpoint{3.044071in}{1.539589in}}%
\pgfpathlineto{\pgfqpoint{3.035489in}{1.534431in}}%
\pgfpathclose%
\pgfusepath{fill}%
\end{pgfscope}%
\begin{pgfscope}%
\pgfpathrectangle{\pgfqpoint{1.150000in}{0.150000in}}{\pgfqpoint{5.700000in}{5.700000in}}%
\pgfusepath{clip}%
\pgfsetbuttcap%
\pgfsetroundjoin%
\definecolor{currentfill}{rgb}{0.203063,0.379716,0.553925}%
\pgfsetfillcolor{currentfill}%
\pgfsetfillopacity{0.700000}%
\pgfsetlinewidth{0.000000pt}%
\definecolor{currentstroke}{rgb}{0.000000,0.000000,0.000000}%
\pgfsetstrokecolor{currentstroke}%
\pgfsetdash{}{0pt}%
\pgfpathmoveto{\pgfqpoint{5.213434in}{2.302603in}}%
\pgfpathlineto{\pgfqpoint{5.227949in}{2.306850in}}%
\pgfpathlineto{\pgfqpoint{5.242476in}{2.311167in}}%
\pgfpathlineto{\pgfqpoint{5.257017in}{2.315556in}}%
\pgfpathlineto{\pgfqpoint{5.271570in}{2.320015in}}%
\pgfpathlineto{\pgfqpoint{5.279209in}{2.326516in}}%
\pgfpathlineto{\pgfqpoint{5.286839in}{2.332905in}}%
\pgfpathlineto{\pgfqpoint{5.294460in}{2.339186in}}%
\pgfpathlineto{\pgfqpoint{5.302073in}{2.345360in}}%
\pgfpathlineto{\pgfqpoint{5.287534in}{2.341015in}}%
\pgfpathlineto{\pgfqpoint{5.273009in}{2.336740in}}%
\pgfpathlineto{\pgfqpoint{5.258496in}{2.332536in}}%
\pgfpathlineto{\pgfqpoint{5.243995in}{2.328402in}}%
\pgfpathlineto{\pgfqpoint{5.236368in}{2.322107in}}%
\pgfpathlineto{\pgfqpoint{5.228732in}{2.315710in}}%
\pgfpathlineto{\pgfqpoint{5.221087in}{2.309210in}}%
\pgfpathlineto{\pgfqpoint{5.213434in}{2.302603in}}%
\pgfpathclose%
\pgfusepath{fill}%
\end{pgfscope}%
\begin{pgfscope}%
\pgfpathrectangle{\pgfqpoint{1.150000in}{0.150000in}}{\pgfqpoint{5.700000in}{5.700000in}}%
\pgfusepath{clip}%
\pgfsetbuttcap%
\pgfsetroundjoin%
\definecolor{currentfill}{rgb}{0.269944,0.014625,0.341379}%
\pgfsetfillcolor{currentfill}%
\pgfsetfillopacity{0.700000}%
\pgfsetlinewidth{0.000000pt}%
\definecolor{currentstroke}{rgb}{0.000000,0.000000,0.000000}%
\pgfsetstrokecolor{currentstroke}%
\pgfsetdash{}{0pt}%
\pgfpathmoveto{\pgfqpoint{3.559560in}{1.521125in}}%
\pgfpathlineto{\pgfqpoint{3.573500in}{1.518220in}}%
\pgfpathlineto{\pgfqpoint{3.587445in}{1.515392in}}%
\pgfpathlineto{\pgfqpoint{3.601398in}{1.512640in}}%
\pgfpathlineto{\pgfqpoint{3.615357in}{1.509965in}}%
\pgfpathlineto{\pgfqpoint{3.623651in}{1.519447in}}%
\pgfpathlineto{\pgfqpoint{3.631940in}{1.528983in}}%
\pgfpathlineto{\pgfqpoint{3.640221in}{1.538569in}}%
\pgfpathlineto{\pgfqpoint{3.648497in}{1.548201in}}%
\pgfpathlineto{\pgfqpoint{3.634551in}{1.550610in}}%
\pgfpathlineto{\pgfqpoint{3.620613in}{1.553095in}}%
\pgfpathlineto{\pgfqpoint{3.606681in}{1.555656in}}%
\pgfpathlineto{\pgfqpoint{3.592755in}{1.558293in}}%
\pgfpathlineto{\pgfqpoint{3.584466in}{1.548921in}}%
\pgfpathlineto{\pgfqpoint{3.576171in}{1.539599in}}%
\pgfpathlineto{\pgfqpoint{3.567869in}{1.530332in}}%
\pgfpathlineto{\pgfqpoint{3.559560in}{1.521125in}}%
\pgfpathclose%
\pgfusepath{fill}%
\end{pgfscope}%
\begin{pgfscope}%
\pgfpathrectangle{\pgfqpoint{1.150000in}{0.150000in}}{\pgfqpoint{5.700000in}{5.700000in}}%
\pgfusepath{clip}%
\pgfsetbuttcap%
\pgfsetroundjoin%
\definecolor{currentfill}{rgb}{0.267004,0.004874,0.329415}%
\pgfsetfillcolor{currentfill}%
\pgfsetfillopacity{0.700000}%
\pgfsetlinewidth{0.000000pt}%
\definecolor{currentstroke}{rgb}{0.000000,0.000000,0.000000}%
\pgfsetstrokecolor{currentstroke}%
\pgfsetdash{}{0pt}%
\pgfpathmoveto{\pgfqpoint{3.325645in}{1.498144in}}%
\pgfpathlineto{\pgfqpoint{3.339548in}{1.493707in}}%
\pgfpathlineto{\pgfqpoint{3.353456in}{1.489350in}}%
\pgfpathlineto{\pgfqpoint{3.367370in}{1.485072in}}%
\pgfpathlineto{\pgfqpoint{3.381288in}{1.480873in}}%
\pgfpathlineto{\pgfqpoint{3.389688in}{1.488826in}}%
\pgfpathlineto{\pgfqpoint{3.398080in}{1.496882in}}%
\pgfpathlineto{\pgfqpoint{3.406463in}{1.505037in}}%
\pgfpathlineto{\pgfqpoint{3.414840in}{1.513287in}}%
\pgfpathlineto{\pgfqpoint{3.400939in}{1.517178in}}%
\pgfpathlineto{\pgfqpoint{3.387043in}{1.521147in}}%
\pgfpathlineto{\pgfqpoint{3.373153in}{1.525196in}}%
\pgfpathlineto{\pgfqpoint{3.359268in}{1.529325in}}%
\pgfpathlineto{\pgfqpoint{3.350875in}{1.521375in}}%
\pgfpathlineto{\pgfqpoint{3.342473in}{1.513525in}}%
\pgfpathlineto{\pgfqpoint{3.334064in}{1.505780in}}%
\pgfpathlineto{\pgfqpoint{3.325645in}{1.498144in}}%
\pgfpathclose%
\pgfusepath{fill}%
\end{pgfscope}%
\begin{pgfscope}%
\pgfpathrectangle{\pgfqpoint{1.150000in}{0.150000in}}{\pgfqpoint{5.700000in}{5.700000in}}%
\pgfusepath{clip}%
\pgfsetbuttcap%
\pgfsetroundjoin%
\definecolor{currentfill}{rgb}{0.278012,0.180367,0.486697}%
\pgfsetfillcolor{currentfill}%
\pgfsetfillopacity{0.700000}%
\pgfsetlinewidth{0.000000pt}%
\definecolor{currentstroke}{rgb}{0.000000,0.000000,0.000000}%
\pgfsetstrokecolor{currentstroke}%
\pgfsetdash{}{0pt}%
\pgfpathmoveto{\pgfqpoint{4.325930in}{1.827540in}}%
\pgfpathlineto{\pgfqpoint{4.340085in}{1.828866in}}%
\pgfpathlineto{\pgfqpoint{4.354251in}{1.830265in}}%
\pgfpathlineto{\pgfqpoint{4.368426in}{1.831736in}}%
\pgfpathlineto{\pgfqpoint{4.382611in}{1.833279in}}%
\pgfpathlineto{\pgfqpoint{4.390638in}{1.843960in}}%
\pgfpathlineto{\pgfqpoint{4.398659in}{1.854565in}}%
\pgfpathlineto{\pgfqpoint{4.406675in}{1.865093in}}%
\pgfpathlineto{\pgfqpoint{4.414685in}{1.875542in}}%
\pgfpathlineto{\pgfqpoint{4.400507in}{1.873896in}}%
\pgfpathlineto{\pgfqpoint{4.386340in}{1.872323in}}%
\pgfpathlineto{\pgfqpoint{4.372182in}{1.870822in}}%
\pgfpathlineto{\pgfqpoint{4.358035in}{1.869393in}}%
\pgfpathlineto{\pgfqpoint{4.350017in}{1.859038in}}%
\pgfpathlineto{\pgfqpoint{4.341994in}{1.848610in}}%
\pgfpathlineto{\pgfqpoint{4.333965in}{1.838111in}}%
\pgfpathlineto{\pgfqpoint{4.325930in}{1.827540in}}%
\pgfpathclose%
\pgfusepath{fill}%
\end{pgfscope}%
\begin{pgfscope}%
\pgfpathrectangle{\pgfqpoint{1.150000in}{0.150000in}}{\pgfqpoint{5.700000in}{5.700000in}}%
\pgfusepath{clip}%
\pgfsetbuttcap%
\pgfsetroundjoin%
\definecolor{currentfill}{rgb}{0.281412,0.155834,0.469201}%
\pgfsetfillcolor{currentfill}%
\pgfsetfillopacity{0.700000}%
\pgfsetlinewidth{0.000000pt}%
\definecolor{currentstroke}{rgb}{0.000000,0.000000,0.000000}%
\pgfsetstrokecolor{currentstroke}%
\pgfsetdash{}{0pt}%
\pgfpathmoveto{\pgfqpoint{4.237180in}{1.780489in}}%
\pgfpathlineto{\pgfqpoint{4.251304in}{1.781402in}}%
\pgfpathlineto{\pgfqpoint{4.265438in}{1.782387in}}%
\pgfpathlineto{\pgfqpoint{4.279581in}{1.783445in}}%
\pgfpathlineto{\pgfqpoint{4.293734in}{1.784576in}}%
\pgfpathlineto{\pgfqpoint{4.301792in}{1.795416in}}%
\pgfpathlineto{\pgfqpoint{4.309844in}{1.806191in}}%
\pgfpathlineto{\pgfqpoint{4.317890in}{1.816900in}}%
\pgfpathlineto{\pgfqpoint{4.325930in}{1.827540in}}%
\pgfpathlineto{\pgfqpoint{4.311785in}{1.826286in}}%
\pgfpathlineto{\pgfqpoint{4.297649in}{1.825104in}}%
\pgfpathlineto{\pgfqpoint{4.283523in}{1.823995in}}%
\pgfpathlineto{\pgfqpoint{4.269407in}{1.822958in}}%
\pgfpathlineto{\pgfqpoint{4.261359in}{1.812434in}}%
\pgfpathlineto{\pgfqpoint{4.253305in}{1.801847in}}%
\pgfpathlineto{\pgfqpoint{4.245245in}{1.791198in}}%
\pgfpathlineto{\pgfqpoint{4.237180in}{1.780489in}}%
\pgfpathclose%
\pgfusepath{fill}%
\end{pgfscope}%
\begin{pgfscope}%
\pgfpathrectangle{\pgfqpoint{1.150000in}{0.150000in}}{\pgfqpoint{5.700000in}{5.700000in}}%
\pgfusepath{clip}%
\pgfsetbuttcap%
\pgfsetroundjoin%
\definecolor{currentfill}{rgb}{0.274128,0.199721,0.498911}%
\pgfsetfillcolor{currentfill}%
\pgfsetfillopacity{0.700000}%
\pgfsetlinewidth{0.000000pt}%
\definecolor{currentstroke}{rgb}{0.000000,0.000000,0.000000}%
\pgfsetstrokecolor{currentstroke}%
\pgfsetdash{}{0pt}%
\pgfpathmoveto{\pgfqpoint{4.414685in}{1.875542in}}%
\pgfpathlineto{\pgfqpoint{4.428872in}{1.877260in}}%
\pgfpathlineto{\pgfqpoint{4.443071in}{1.879050in}}%
\pgfpathlineto{\pgfqpoint{4.457279in}{1.880911in}}%
\pgfpathlineto{\pgfqpoint{4.471498in}{1.882845in}}%
\pgfpathlineto{\pgfqpoint{4.479494in}{1.893304in}}%
\pgfpathlineto{\pgfqpoint{4.487484in}{1.903678in}}%
\pgfpathlineto{\pgfqpoint{4.495468in}{1.913964in}}%
\pgfpathlineto{\pgfqpoint{4.503446in}{1.924164in}}%
\pgfpathlineto{\pgfqpoint{4.489234in}{1.922149in}}%
\pgfpathlineto{\pgfqpoint{4.475034in}{1.920205in}}%
\pgfpathlineto{\pgfqpoint{4.460844in}{1.918334in}}%
\pgfpathlineto{\pgfqpoint{4.446664in}{1.916534in}}%
\pgfpathlineto{\pgfqpoint{4.438678in}{1.906409in}}%
\pgfpathlineto{\pgfqpoint{4.430686in}{1.896201in}}%
\pgfpathlineto{\pgfqpoint{4.422688in}{1.885912in}}%
\pgfpathlineto{\pgfqpoint{4.414685in}{1.875542in}}%
\pgfpathclose%
\pgfusepath{fill}%
\end{pgfscope}%
\begin{pgfscope}%
\pgfpathrectangle{\pgfqpoint{1.150000in}{0.150000in}}{\pgfqpoint{5.700000in}{5.700000in}}%
\pgfusepath{clip}%
\pgfsetbuttcap%
\pgfsetroundjoin%
\definecolor{currentfill}{rgb}{0.210503,0.363727,0.552206}%
\pgfsetfillcolor{currentfill}%
\pgfsetfillopacity{0.700000}%
\pgfsetlinewidth{0.000000pt}%
\definecolor{currentstroke}{rgb}{0.000000,0.000000,0.000000}%
\pgfsetstrokecolor{currentstroke}%
\pgfsetdash{}{0pt}%
\pgfpathmoveto{\pgfqpoint{5.124751in}{2.258433in}}%
\pgfpathlineto{\pgfqpoint{5.139229in}{2.262488in}}%
\pgfpathlineto{\pgfqpoint{5.153719in}{2.266614in}}%
\pgfpathlineto{\pgfqpoint{5.168222in}{2.270810in}}%
\pgfpathlineto{\pgfqpoint{5.182737in}{2.275078in}}%
\pgfpathlineto{\pgfqpoint{5.190424in}{2.282128in}}%
\pgfpathlineto{\pgfqpoint{5.198103in}{2.289065in}}%
\pgfpathlineto{\pgfqpoint{5.205773in}{2.295889in}}%
\pgfpathlineto{\pgfqpoint{5.213434in}{2.302603in}}%
\pgfpathlineto{\pgfqpoint{5.198932in}{2.298427in}}%
\pgfpathlineto{\pgfqpoint{5.184443in}{2.294322in}}%
\pgfpathlineto{\pgfqpoint{5.169966in}{2.290288in}}%
\pgfpathlineto{\pgfqpoint{5.155502in}{2.286324in}}%
\pgfpathlineto{\pgfqpoint{5.147826in}{2.279510in}}%
\pgfpathlineto{\pgfqpoint{5.140143in}{2.272592in}}%
\pgfpathlineto{\pgfqpoint{5.132451in}{2.265567in}}%
\pgfpathlineto{\pgfqpoint{5.124751in}{2.258433in}}%
\pgfpathclose%
\pgfusepath{fill}%
\end{pgfscope}%
\begin{pgfscope}%
\pgfpathrectangle{\pgfqpoint{1.150000in}{0.150000in}}{\pgfqpoint{5.700000in}{5.700000in}}%
\pgfusepath{clip}%
\pgfsetbuttcap%
\pgfsetroundjoin%
\definecolor{currentfill}{rgb}{0.282884,0.135920,0.453427}%
\pgfsetfillcolor{currentfill}%
\pgfsetfillopacity{0.700000}%
\pgfsetlinewidth{0.000000pt}%
\definecolor{currentstroke}{rgb}{0.000000,0.000000,0.000000}%
\pgfsetstrokecolor{currentstroke}%
\pgfsetdash{}{0pt}%
\pgfpathmoveto{\pgfqpoint{4.148430in}{1.734740in}}%
\pgfpathlineto{\pgfqpoint{4.162525in}{1.735218in}}%
\pgfpathlineto{\pgfqpoint{4.176629in}{1.735769in}}%
\pgfpathlineto{\pgfqpoint{4.190742in}{1.736392in}}%
\pgfpathlineto{\pgfqpoint{4.204864in}{1.737088in}}%
\pgfpathlineto{\pgfqpoint{4.212952in}{1.748020in}}%
\pgfpathlineto{\pgfqpoint{4.221033in}{1.758898in}}%
\pgfpathlineto{\pgfqpoint{4.229109in}{1.769722in}}%
\pgfpathlineto{\pgfqpoint{4.237180in}{1.780489in}}%
\pgfpathlineto{\pgfqpoint{4.223066in}{1.779648in}}%
\pgfpathlineto{\pgfqpoint{4.208960in}{1.778881in}}%
\pgfpathlineto{\pgfqpoint{4.194865in}{1.778185in}}%
\pgfpathlineto{\pgfqpoint{4.180778in}{1.777563in}}%
\pgfpathlineto{\pgfqpoint{4.172699in}{1.766933in}}%
\pgfpathlineto{\pgfqpoint{4.164615in}{1.756251in}}%
\pgfpathlineto{\pgfqpoint{4.156525in}{1.745519in}}%
\pgfpathlineto{\pgfqpoint{4.148430in}{1.734740in}}%
\pgfpathclose%
\pgfusepath{fill}%
\end{pgfscope}%
\begin{pgfscope}%
\pgfpathrectangle{\pgfqpoint{1.150000in}{0.150000in}}{\pgfqpoint{5.700000in}{5.700000in}}%
\pgfusepath{clip}%
\pgfsetbuttcap%
\pgfsetroundjoin%
\definecolor{currentfill}{rgb}{0.267968,0.223549,0.512008}%
\pgfsetfillcolor{currentfill}%
\pgfsetfillopacity{0.700000}%
\pgfsetlinewidth{0.000000pt}%
\definecolor{currentstroke}{rgb}{0.000000,0.000000,0.000000}%
\pgfsetstrokecolor{currentstroke}%
\pgfsetdash{}{0pt}%
\pgfpathmoveto{\pgfqpoint{4.503446in}{1.924164in}}%
\pgfpathlineto{\pgfqpoint{4.517667in}{1.926251in}}%
\pgfpathlineto{\pgfqpoint{4.531900in}{1.928410in}}%
\pgfpathlineto{\pgfqpoint{4.546142in}{1.930641in}}%
\pgfpathlineto{\pgfqpoint{4.560396in}{1.932943in}}%
\pgfpathlineto{\pgfqpoint{4.568360in}{1.943123in}}%
\pgfpathlineto{\pgfqpoint{4.576317in}{1.953208in}}%
\pgfpathlineto{\pgfqpoint{4.584268in}{1.963199in}}%
\pgfpathlineto{\pgfqpoint{4.592213in}{1.973095in}}%
\pgfpathlineto{\pgfqpoint{4.577968in}{1.970732in}}%
\pgfpathlineto{\pgfqpoint{4.563733in}{1.968441in}}%
\pgfpathlineto{\pgfqpoint{4.549508in}{1.966221in}}%
\pgfpathlineto{\pgfqpoint{4.535295in}{1.964074in}}%
\pgfpathlineto{\pgfqpoint{4.527342in}{1.954230in}}%
\pgfpathlineto{\pgfqpoint{4.519383in}{1.944297in}}%
\pgfpathlineto{\pgfqpoint{4.511417in}{1.934275in}}%
\pgfpathlineto{\pgfqpoint{4.503446in}{1.924164in}}%
\pgfpathclose%
\pgfusepath{fill}%
\end{pgfscope}%
\begin{pgfscope}%
\pgfpathrectangle{\pgfqpoint{1.150000in}{0.150000in}}{\pgfqpoint{5.700000in}{5.700000in}}%
\pgfusepath{clip}%
\pgfsetbuttcap%
\pgfsetroundjoin%
\definecolor{currentfill}{rgb}{0.260571,0.246922,0.522828}%
\pgfsetfillcolor{currentfill}%
\pgfsetfillopacity{0.700000}%
\pgfsetlinewidth{0.000000pt}%
\definecolor{currentstroke}{rgb}{0.000000,0.000000,0.000000}%
\pgfsetstrokecolor{currentstroke}%
\pgfsetdash{}{0pt}%
\pgfpathmoveto{\pgfqpoint{4.592213in}{1.973095in}}%
\pgfpathlineto{\pgfqpoint{4.606470in}{1.975530in}}%
\pgfpathlineto{\pgfqpoint{4.620737in}{1.978036in}}%
\pgfpathlineto{\pgfqpoint{4.635016in}{1.980614in}}%
\pgfpathlineto{\pgfqpoint{4.649305in}{1.983263in}}%
\pgfpathlineto{\pgfqpoint{4.657235in}{1.993111in}}%
\pgfpathlineto{\pgfqpoint{4.665159in}{2.002857in}}%
\pgfpathlineto{\pgfqpoint{4.673076in}{2.012502in}}%
\pgfpathlineto{\pgfqpoint{4.680986in}{2.022046in}}%
\pgfpathlineto{\pgfqpoint{4.666705in}{2.019357in}}%
\pgfpathlineto{\pgfqpoint{4.652435in}{2.016740in}}%
\pgfpathlineto{\pgfqpoint{4.638175in}{2.014195in}}%
\pgfpathlineto{\pgfqpoint{4.623927in}{2.011721in}}%
\pgfpathlineto{\pgfqpoint{4.616009in}{2.002208in}}%
\pgfpathlineto{\pgfqpoint{4.608083in}{1.992600in}}%
\pgfpathlineto{\pgfqpoint{4.600152in}{1.982895in}}%
\pgfpathlineto{\pgfqpoint{4.592213in}{1.973095in}}%
\pgfpathclose%
\pgfusepath{fill}%
\end{pgfscope}%
\begin{pgfscope}%
\pgfpathrectangle{\pgfqpoint{1.150000in}{0.150000in}}{\pgfqpoint{5.700000in}{5.700000in}}%
\pgfusepath{clip}%
\pgfsetbuttcap%
\pgfsetroundjoin%
\definecolor{currentfill}{rgb}{0.283197,0.115680,0.436115}%
\pgfsetfillcolor{currentfill}%
\pgfsetfillopacity{0.700000}%
\pgfsetlinewidth{0.000000pt}%
\definecolor{currentstroke}{rgb}{0.000000,0.000000,0.000000}%
\pgfsetstrokecolor{currentstroke}%
\pgfsetdash{}{0pt}%
\pgfpathmoveto{\pgfqpoint{4.059673in}{1.690668in}}%
\pgfpathlineto{\pgfqpoint{4.073741in}{1.690688in}}%
\pgfpathlineto{\pgfqpoint{4.087816in}{1.690782in}}%
\pgfpathlineto{\pgfqpoint{4.101901in}{1.690949in}}%
\pgfpathlineto{\pgfqpoint{4.115995in}{1.691189in}}%
\pgfpathlineto{\pgfqpoint{4.124112in}{1.702138in}}%
\pgfpathlineto{\pgfqpoint{4.132223in}{1.713047in}}%
\pgfpathlineto{\pgfqpoint{4.140330in}{1.723916in}}%
\pgfpathlineto{\pgfqpoint{4.148430in}{1.734740in}}%
\pgfpathlineto{\pgfqpoint{4.134345in}{1.734336in}}%
\pgfpathlineto{\pgfqpoint{4.120268in}{1.734004in}}%
\pgfpathlineto{\pgfqpoint{4.106201in}{1.733745in}}%
\pgfpathlineto{\pgfqpoint{4.092142in}{1.733559in}}%
\pgfpathlineto{\pgfqpoint{4.084033in}{1.722892in}}%
\pgfpathlineto{\pgfqpoint{4.075919in}{1.712186in}}%
\pgfpathlineto{\pgfqpoint{4.067799in}{1.701444in}}%
\pgfpathlineto{\pgfqpoint{4.059673in}{1.690668in}}%
\pgfpathclose%
\pgfusepath{fill}%
\end{pgfscope}%
\begin{pgfscope}%
\pgfpathrectangle{\pgfqpoint{1.150000in}{0.150000in}}{\pgfqpoint{5.700000in}{5.700000in}}%
\pgfusepath{clip}%
\pgfsetbuttcap%
\pgfsetroundjoin%
\definecolor{currentfill}{rgb}{0.218130,0.347432,0.550038}%
\pgfsetfillcolor{currentfill}%
\pgfsetfillopacity{0.700000}%
\pgfsetlinewidth{0.000000pt}%
\definecolor{currentstroke}{rgb}{0.000000,0.000000,0.000000}%
\pgfsetstrokecolor{currentstroke}%
\pgfsetdash{}{0pt}%
\pgfpathmoveto{\pgfqpoint{5.036033in}{2.212990in}}%
\pgfpathlineto{\pgfqpoint{5.050474in}{2.216831in}}%
\pgfpathlineto{\pgfqpoint{5.064926in}{2.220742in}}%
\pgfpathlineto{\pgfqpoint{5.079391in}{2.224724in}}%
\pgfpathlineto{\pgfqpoint{5.093869in}{2.228778in}}%
\pgfpathlineto{\pgfqpoint{5.101602in}{2.236363in}}%
\pgfpathlineto{\pgfqpoint{5.109326in}{2.243833in}}%
\pgfpathlineto{\pgfqpoint{5.117043in}{2.251189in}}%
\pgfpathlineto{\pgfqpoint{5.124751in}{2.258433in}}%
\pgfpathlineto{\pgfqpoint{5.110286in}{2.254449in}}%
\pgfpathlineto{\pgfqpoint{5.095833in}{2.250536in}}%
\pgfpathlineto{\pgfqpoint{5.081393in}{2.246694in}}%
\pgfpathlineto{\pgfqpoint{5.066965in}{2.242923in}}%
\pgfpathlineto{\pgfqpoint{5.059244in}{2.235601in}}%
\pgfpathlineto{\pgfqpoint{5.051515in}{2.228173in}}%
\pgfpathlineto{\pgfqpoint{5.043778in}{2.220637in}}%
\pgfpathlineto{\pgfqpoint{5.036033in}{2.212990in}}%
\pgfpathclose%
\pgfusepath{fill}%
\end{pgfscope}%
\begin{pgfscope}%
\pgfpathrectangle{\pgfqpoint{1.150000in}{0.150000in}}{\pgfqpoint{5.700000in}{5.700000in}}%
\pgfusepath{clip}%
\pgfsetbuttcap%
\pgfsetroundjoin%
\definecolor{currentfill}{rgb}{0.252194,0.269783,0.531579}%
\pgfsetfillcolor{currentfill}%
\pgfsetfillopacity{0.700000}%
\pgfsetlinewidth{0.000000pt}%
\definecolor{currentstroke}{rgb}{0.000000,0.000000,0.000000}%
\pgfsetstrokecolor{currentstroke}%
\pgfsetdash{}{0pt}%
\pgfpathmoveto{\pgfqpoint{4.680986in}{2.022046in}}%
\pgfpathlineto{\pgfqpoint{4.695278in}{2.024806in}}%
\pgfpathlineto{\pgfqpoint{4.709582in}{2.027637in}}%
\pgfpathlineto{\pgfqpoint{4.723896in}{2.030540in}}%
\pgfpathlineto{\pgfqpoint{4.738223in}{2.033515in}}%
\pgfpathlineto{\pgfqpoint{4.746117in}{2.042983in}}%
\pgfpathlineto{\pgfqpoint{4.754005in}{2.052344in}}%
\pgfpathlineto{\pgfqpoint{4.761886in}{2.061599in}}%
\pgfpathlineto{\pgfqpoint{4.769760in}{2.070748in}}%
\pgfpathlineto{\pgfqpoint{4.755442in}{2.067756in}}%
\pgfpathlineto{\pgfqpoint{4.741136in}{2.064835in}}%
\pgfpathlineto{\pgfqpoint{4.726842in}{2.061986in}}%
\pgfpathlineto{\pgfqpoint{4.712559in}{2.059208in}}%
\pgfpathlineto{\pgfqpoint{4.704676in}{2.050069in}}%
\pgfpathlineto{\pgfqpoint{4.696786in}{2.040829in}}%
\pgfpathlineto{\pgfqpoint{4.688889in}{2.031488in}}%
\pgfpathlineto{\pgfqpoint{4.680986in}{2.022046in}}%
\pgfpathclose%
\pgfusepath{fill}%
\end{pgfscope}%
\begin{pgfscope}%
\pgfpathrectangle{\pgfqpoint{1.150000in}{0.150000in}}{\pgfqpoint{5.700000in}{5.700000in}}%
\pgfusepath{clip}%
\pgfsetbuttcap%
\pgfsetroundjoin%
\definecolor{currentfill}{rgb}{0.227802,0.326594,0.546532}%
\pgfsetfillcolor{currentfill}%
\pgfsetfillopacity{0.700000}%
\pgfsetlinewidth{0.000000pt}%
\definecolor{currentstroke}{rgb}{0.000000,0.000000,0.000000}%
\pgfsetstrokecolor{currentstroke}%
\pgfsetdash{}{0pt}%
\pgfpathmoveto{\pgfqpoint{4.947290in}{2.166436in}}%
\pgfpathlineto{\pgfqpoint{4.961693in}{2.170039in}}%
\pgfpathlineto{\pgfqpoint{4.976108in}{2.173714in}}%
\pgfpathlineto{\pgfqpoint{4.990535in}{2.177460in}}%
\pgfpathlineto{\pgfqpoint{5.004975in}{2.181277in}}%
\pgfpathlineto{\pgfqpoint{5.012751in}{2.189377in}}%
\pgfpathlineto{\pgfqpoint{5.020520in}{2.197362in}}%
\pgfpathlineto{\pgfqpoint{5.028281in}{2.205232in}}%
\pgfpathlineto{\pgfqpoint{5.036033in}{2.212990in}}%
\pgfpathlineto{\pgfqpoint{5.021605in}{2.209221in}}%
\pgfpathlineto{\pgfqpoint{5.007190in}{2.205522in}}%
\pgfpathlineto{\pgfqpoint{4.992786in}{2.201895in}}%
\pgfpathlineto{\pgfqpoint{4.978394in}{2.198338in}}%
\pgfpathlineto{\pgfqpoint{4.970630in}{2.190525in}}%
\pgfpathlineto{\pgfqpoint{4.962858in}{2.182604in}}%
\pgfpathlineto{\pgfqpoint{4.955078in}{2.174575in}}%
\pgfpathlineto{\pgfqpoint{4.947290in}{2.166436in}}%
\pgfpathclose%
\pgfusepath{fill}%
\end{pgfscope}%
\begin{pgfscope}%
\pgfpathrectangle{\pgfqpoint{1.150000in}{0.150000in}}{\pgfqpoint{5.700000in}{5.700000in}}%
\pgfusepath{clip}%
\pgfsetbuttcap%
\pgfsetroundjoin%
\definecolor{currentfill}{rgb}{0.244972,0.287675,0.537260}%
\pgfsetfillcolor{currentfill}%
\pgfsetfillopacity{0.700000}%
\pgfsetlinewidth{0.000000pt}%
\definecolor{currentstroke}{rgb}{0.000000,0.000000,0.000000}%
\pgfsetstrokecolor{currentstroke}%
\pgfsetdash{}{0pt}%
\pgfpathmoveto{\pgfqpoint{4.769760in}{2.070748in}}%
\pgfpathlineto{\pgfqpoint{4.784088in}{2.073811in}}%
\pgfpathlineto{\pgfqpoint{4.798429in}{2.076946in}}%
\pgfpathlineto{\pgfqpoint{4.812780in}{2.080152in}}%
\pgfpathlineto{\pgfqpoint{4.827144in}{2.083429in}}%
\pgfpathlineto{\pgfqpoint{4.835001in}{2.092476in}}%
\pgfpathlineto{\pgfqpoint{4.842851in}{2.101412in}}%
\pgfpathlineto{\pgfqpoint{4.850694in}{2.110238in}}%
\pgfpathlineto{\pgfqpoint{4.858530in}{2.118953in}}%
\pgfpathlineto{\pgfqpoint{4.844176in}{2.115680in}}%
\pgfpathlineto{\pgfqpoint{4.829834in}{2.112477in}}%
\pgfpathlineto{\pgfqpoint{4.815503in}{2.109347in}}%
\pgfpathlineto{\pgfqpoint{4.801184in}{2.106287in}}%
\pgfpathlineto{\pgfqpoint{4.793339in}{2.097559in}}%
\pgfpathlineto{\pgfqpoint{4.785486in}{2.088727in}}%
\pgfpathlineto{\pgfqpoint{4.777626in}{2.079790in}}%
\pgfpathlineto{\pgfqpoint{4.769760in}{2.070748in}}%
\pgfpathclose%
\pgfusepath{fill}%
\end{pgfscope}%
\begin{pgfscope}%
\pgfpathrectangle{\pgfqpoint{1.150000in}{0.150000in}}{\pgfqpoint{5.700000in}{5.700000in}}%
\pgfusepath{clip}%
\pgfsetbuttcap%
\pgfsetroundjoin%
\definecolor{currentfill}{rgb}{0.235526,0.309527,0.542944}%
\pgfsetfillcolor{currentfill}%
\pgfsetfillopacity{0.700000}%
\pgfsetlinewidth{0.000000pt}%
\definecolor{currentstroke}{rgb}{0.000000,0.000000,0.000000}%
\pgfsetstrokecolor{currentstroke}%
\pgfsetdash{}{0pt}%
\pgfpathmoveto{\pgfqpoint{4.858530in}{2.118953in}}%
\pgfpathlineto{\pgfqpoint{4.872895in}{2.122298in}}%
\pgfpathlineto{\pgfqpoint{4.887273in}{2.125714in}}%
\pgfpathlineto{\pgfqpoint{4.901662in}{2.129201in}}%
\pgfpathlineto{\pgfqpoint{4.916064in}{2.132759in}}%
\pgfpathlineto{\pgfqpoint{4.923882in}{2.141348in}}%
\pgfpathlineto{\pgfqpoint{4.931692in}{2.149823in}}%
\pgfpathlineto{\pgfqpoint{4.939495in}{2.158185in}}%
\pgfpathlineto{\pgfqpoint{4.947290in}{2.166436in}}%
\pgfpathlineto{\pgfqpoint{4.932899in}{2.162903in}}%
\pgfpathlineto{\pgfqpoint{4.918520in}{2.159442in}}%
\pgfpathlineto{\pgfqpoint{4.904153in}{2.156051in}}%
\pgfpathlineto{\pgfqpoint{4.889798in}{2.152732in}}%
\pgfpathlineto{\pgfqpoint{4.881992in}{2.144448in}}%
\pgfpathlineto{\pgfqpoint{4.874179in}{2.136058in}}%
\pgfpathlineto{\pgfqpoint{4.866358in}{2.127560in}}%
\pgfpathlineto{\pgfqpoint{4.858530in}{2.118953in}}%
\pgfpathclose%
\pgfusepath{fill}%
\end{pgfscope}%
\begin{pgfscope}%
\pgfpathrectangle{\pgfqpoint{1.150000in}{0.150000in}}{\pgfqpoint{5.700000in}{5.700000in}}%
\pgfusepath{clip}%
\pgfsetbuttcap%
\pgfsetroundjoin%
\definecolor{currentfill}{rgb}{0.282327,0.094955,0.417331}%
\pgfsetfillcolor{currentfill}%
\pgfsetfillopacity{0.700000}%
\pgfsetlinewidth{0.000000pt}%
\definecolor{currentstroke}{rgb}{0.000000,0.000000,0.000000}%
\pgfsetstrokecolor{currentstroke}%
\pgfsetdash{}{0pt}%
\pgfpathmoveto{\pgfqpoint{3.970901in}{1.648664in}}%
\pgfpathlineto{\pgfqpoint{3.984942in}{1.648205in}}%
\pgfpathlineto{\pgfqpoint{3.998992in}{1.647820in}}%
\pgfpathlineto{\pgfqpoint{4.013051in}{1.647509in}}%
\pgfpathlineto{\pgfqpoint{4.027118in}{1.647270in}}%
\pgfpathlineto{\pgfqpoint{4.035265in}{1.658158in}}%
\pgfpathlineto{\pgfqpoint{4.043407in}{1.669023in}}%
\pgfpathlineto{\pgfqpoint{4.051543in}{1.679860in}}%
\pgfpathlineto{\pgfqpoint{4.059673in}{1.690668in}}%
\pgfpathlineto{\pgfqpoint{4.045615in}{1.690720in}}%
\pgfpathlineto{\pgfqpoint{4.031565in}{1.690846in}}%
\pgfpathlineto{\pgfqpoint{4.017524in}{1.691046in}}%
\pgfpathlineto{\pgfqpoint{4.003492in}{1.691319in}}%
\pgfpathlineto{\pgfqpoint{3.995352in}{1.680689in}}%
\pgfpathlineto{\pgfqpoint{3.987207in}{1.670035in}}%
\pgfpathlineto{\pgfqpoint{3.979057in}{1.659359in}}%
\pgfpathlineto{\pgfqpoint{3.970901in}{1.648664in}}%
\pgfpathclose%
\pgfusepath{fill}%
\end{pgfscope}%
\begin{pgfscope}%
\pgfpathrectangle{\pgfqpoint{1.150000in}{0.150000in}}{\pgfqpoint{5.700000in}{5.700000in}}%
\pgfusepath{clip}%
\pgfsetbuttcap%
\pgfsetroundjoin%
\definecolor{currentfill}{rgb}{0.281446,0.084320,0.407414}%
\pgfsetfillcolor{currentfill}%
\pgfsetfillopacity{0.700000}%
\pgfsetlinewidth{0.000000pt}%
\definecolor{currentstroke}{rgb}{0.000000,0.000000,0.000000}%
\pgfsetstrokecolor{currentstroke}%
\pgfsetdash{}{0pt}%
\pgfpathmoveto{\pgfqpoint{2.688321in}{1.662230in}}%
\pgfpathlineto{\pgfqpoint{2.702207in}{1.653254in}}%
\pgfpathlineto{\pgfqpoint{2.716095in}{1.644371in}}%
\pgfpathlineto{\pgfqpoint{2.729984in}{1.635582in}}%
\pgfpathlineto{\pgfqpoint{2.743874in}{1.626885in}}%
\pgfpathlineto{\pgfqpoint{2.752676in}{1.628870in}}%
\pgfpathlineto{\pgfqpoint{2.761462in}{1.631092in}}%
\pgfpathlineto{\pgfqpoint{2.770233in}{1.633546in}}%
\pgfpathlineto{\pgfqpoint{2.778991in}{1.636227in}}%
\pgfpathlineto{\pgfqpoint{2.765132in}{1.644528in}}%
\pgfpathlineto{\pgfqpoint{2.751275in}{1.652922in}}%
\pgfpathlineto{\pgfqpoint{2.737420in}{1.661409in}}%
\pgfpathlineto{\pgfqpoint{2.723567in}{1.669990in}}%
\pgfpathlineto{\pgfqpoint{2.714778in}{1.667697in}}%
\pgfpathlineto{\pgfqpoint{2.705974in}{1.665635in}}%
\pgfpathlineto{\pgfqpoint{2.697155in}{1.663811in}}%
\pgfpathlineto{\pgfqpoint{2.688321in}{1.662230in}}%
\pgfpathclose%
\pgfusepath{fill}%
\end{pgfscope}%
\begin{pgfscope}%
\pgfpathrectangle{\pgfqpoint{1.150000in}{0.150000in}}{\pgfqpoint{5.700000in}{5.700000in}}%
\pgfusepath{clip}%
\pgfsetbuttcap%
\pgfsetroundjoin%
\definecolor{currentfill}{rgb}{0.274952,0.037752,0.364543}%
\pgfsetfillcolor{currentfill}%
\pgfsetfillopacity{0.700000}%
\pgfsetlinewidth{0.000000pt}%
\definecolor{currentstroke}{rgb}{0.000000,0.000000,0.000000}%
\pgfsetstrokecolor{currentstroke}%
\pgfsetdash{}{0pt}%
\pgfpathmoveto{\pgfqpoint{2.889935in}{1.573092in}}%
\pgfpathlineto{\pgfqpoint{2.903814in}{1.565603in}}%
\pgfpathlineto{\pgfqpoint{2.917696in}{1.558201in}}%
\pgfpathlineto{\pgfqpoint{2.931580in}{1.550887in}}%
\pgfpathlineto{\pgfqpoint{2.945467in}{1.543660in}}%
\pgfpathlineto{\pgfqpoint{2.954123in}{1.547702in}}%
\pgfpathlineto{\pgfqpoint{2.962766in}{1.551941in}}%
\pgfpathlineto{\pgfqpoint{2.971398in}{1.556373in}}%
\pgfpathlineto{\pgfqpoint{2.980017in}{1.560991in}}%
\pgfpathlineto{\pgfqpoint{2.966157in}{1.567847in}}%
\pgfpathlineto{\pgfqpoint{2.952300in}{1.574789in}}%
\pgfpathlineto{\pgfqpoint{2.938446in}{1.581818in}}%
\pgfpathlineto{\pgfqpoint{2.924595in}{1.588936in}}%
\pgfpathlineto{\pgfqpoint{2.915949in}{1.584681in}}%
\pgfpathlineto{\pgfqpoint{2.907290in}{1.580619in}}%
\pgfpathlineto{\pgfqpoint{2.898619in}{1.576754in}}%
\pgfpathlineto{\pgfqpoint{2.889935in}{1.573092in}}%
\pgfpathclose%
\pgfusepath{fill}%
\end{pgfscope}%
\begin{pgfscope}%
\pgfpathrectangle{\pgfqpoint{1.150000in}{0.150000in}}{\pgfqpoint{5.700000in}{5.700000in}}%
\pgfusepath{clip}%
\pgfsetbuttcap%
\pgfsetroundjoin%
\definecolor{currentfill}{rgb}{0.281412,0.155834,0.469201}%
\pgfsetfillcolor{currentfill}%
\pgfsetfillopacity{0.700000}%
\pgfsetlinewidth{0.000000pt}%
\definecolor{currentstroke}{rgb}{0.000000,0.000000,0.000000}%
\pgfsetstrokecolor{currentstroke}%
\pgfsetdash{}{0pt}%
\pgfpathmoveto{\pgfqpoint{2.430153in}{1.822061in}}%
\pgfpathlineto{\pgfqpoint{2.444070in}{1.811079in}}%
\pgfpathlineto{\pgfqpoint{2.457986in}{1.800202in}}%
\pgfpathlineto{\pgfqpoint{2.471902in}{1.789427in}}%
\pgfpathlineto{\pgfqpoint{2.485818in}{1.778756in}}%
\pgfpathlineto{\pgfqpoint{2.494831in}{1.778014in}}%
\pgfpathlineto{\pgfqpoint{2.503824in}{1.777558in}}%
\pgfpathlineto{\pgfqpoint{2.512800in}{1.777383in}}%
\pgfpathlineto{\pgfqpoint{2.521758in}{1.777480in}}%
\pgfpathlineto{\pgfqpoint{2.507881in}{1.787731in}}%
\pgfpathlineto{\pgfqpoint{2.494004in}{1.798083in}}%
\pgfpathlineto{\pgfqpoint{2.480126in}{1.808539in}}%
\pgfpathlineto{\pgfqpoint{2.466249in}{1.819099in}}%
\pgfpathlineto{\pgfqpoint{2.457253in}{1.819414in}}%
\pgfpathlineto{\pgfqpoint{2.448239in}{1.820009in}}%
\pgfpathlineto{\pgfqpoint{2.439206in}{1.820889in}}%
\pgfpathlineto{\pgfqpoint{2.430153in}{1.822061in}}%
\pgfpathclose%
\pgfusepath{fill}%
\end{pgfscope}%
\begin{pgfscope}%
\pgfpathrectangle{\pgfqpoint{1.150000in}{0.150000in}}{\pgfqpoint{5.700000in}{5.700000in}}%
\pgfusepath{clip}%
\pgfsetbuttcap%
\pgfsetroundjoin%
\definecolor{currentfill}{rgb}{0.280267,0.073417,0.397163}%
\pgfsetfillcolor{currentfill}%
\pgfsetfillopacity{0.700000}%
\pgfsetlinewidth{0.000000pt}%
\definecolor{currentstroke}{rgb}{0.000000,0.000000,0.000000}%
\pgfsetstrokecolor{currentstroke}%
\pgfsetdash{}{0pt}%
\pgfpathmoveto{\pgfqpoint{3.882099in}{1.609145in}}%
\pgfpathlineto{\pgfqpoint{3.896118in}{1.608184in}}%
\pgfpathlineto{\pgfqpoint{3.910144in}{1.607298in}}%
\pgfpathlineto{\pgfqpoint{3.924179in}{1.606486in}}%
\pgfpathlineto{\pgfqpoint{3.938222in}{1.605747in}}%
\pgfpathlineto{\pgfqpoint{3.946400in}{1.616492in}}%
\pgfpathlineto{\pgfqpoint{3.954572in}{1.627228in}}%
\pgfpathlineto{\pgfqpoint{3.962739in}{1.637953in}}%
\pgfpathlineto{\pgfqpoint{3.970901in}{1.648664in}}%
\pgfpathlineto{\pgfqpoint{3.956867in}{1.649196in}}%
\pgfpathlineto{\pgfqpoint{3.942842in}{1.649802in}}%
\pgfpathlineto{\pgfqpoint{3.928825in}{1.650482in}}%
\pgfpathlineto{\pgfqpoint{3.914817in}{1.651236in}}%
\pgfpathlineto{\pgfqpoint{3.906646in}{1.640723in}}%
\pgfpathlineto{\pgfqpoint{3.898469in}{1.630202in}}%
\pgfpathlineto{\pgfqpoint{3.890287in}{1.619675in}}%
\pgfpathlineto{\pgfqpoint{3.882099in}{1.609145in}}%
\pgfpathclose%
\pgfusepath{fill}%
\end{pgfscope}%
\begin{pgfscope}%
\pgfpathrectangle{\pgfqpoint{1.150000in}{0.150000in}}{\pgfqpoint{5.700000in}{5.700000in}}%
\pgfusepath{clip}%
\pgfsetbuttcap%
\pgfsetroundjoin%
\definecolor{currentfill}{rgb}{0.268510,0.009605,0.335427}%
\pgfsetfillcolor{currentfill}%
\pgfsetfillopacity{0.700000}%
\pgfsetlinewidth{0.000000pt}%
\definecolor{currentstroke}{rgb}{0.000000,0.000000,0.000000}%
\pgfsetstrokecolor{currentstroke}%
\pgfsetdash{}{0pt}%
\pgfpathmoveto{\pgfqpoint{3.470501in}{1.498506in}}%
\pgfpathlineto{\pgfqpoint{3.484431in}{1.495005in}}%
\pgfpathlineto{\pgfqpoint{3.498366in}{1.491582in}}%
\pgfpathlineto{\pgfqpoint{3.512308in}{1.488236in}}%
\pgfpathlineto{\pgfqpoint{3.526256in}{1.484967in}}%
\pgfpathlineto{\pgfqpoint{3.534593in}{1.493898in}}%
\pgfpathlineto{\pgfqpoint{3.542922in}{1.502903in}}%
\pgfpathlineto{\pgfqpoint{3.551245in}{1.511981in}}%
\pgfpathlineto{\pgfqpoint{3.559560in}{1.521125in}}%
\pgfpathlineto{\pgfqpoint{3.545627in}{1.524107in}}%
\pgfpathlineto{\pgfqpoint{3.531700in}{1.527165in}}%
\pgfpathlineto{\pgfqpoint{3.517780in}{1.530301in}}%
\pgfpathlineto{\pgfqpoint{3.503866in}{1.533514in}}%
\pgfpathlineto{\pgfqpoint{3.495535in}{1.524649in}}%
\pgfpathlineto{\pgfqpoint{3.487198in}{1.515857in}}%
\pgfpathlineto{\pgfqpoint{3.478853in}{1.507141in}}%
\pgfpathlineto{\pgfqpoint{3.470501in}{1.498506in}}%
\pgfpathclose%
\pgfusepath{fill}%
\end{pgfscope}%
\begin{pgfscope}%
\pgfpathrectangle{\pgfqpoint{1.150000in}{0.150000in}}{\pgfqpoint{5.700000in}{5.700000in}}%
\pgfusepath{clip}%
\pgfsetbuttcap%
\pgfsetroundjoin%
\definecolor{currentfill}{rgb}{0.277941,0.056324,0.381191}%
\pgfsetfillcolor{currentfill}%
\pgfsetfillopacity{0.700000}%
\pgfsetlinewidth{0.000000pt}%
\definecolor{currentstroke}{rgb}{0.000000,0.000000,0.000000}%
\pgfsetstrokecolor{currentstroke}%
\pgfsetdash{}{0pt}%
\pgfpathmoveto{\pgfqpoint{3.793254in}{1.572546in}}%
\pgfpathlineto{\pgfqpoint{3.807252in}{1.571062in}}%
\pgfpathlineto{\pgfqpoint{3.821258in}{1.569652in}}%
\pgfpathlineto{\pgfqpoint{3.835271in}{1.568317in}}%
\pgfpathlineto{\pgfqpoint{3.849292in}{1.567056in}}%
\pgfpathlineto{\pgfqpoint{3.857502in}{1.577567in}}%
\pgfpathlineto{\pgfqpoint{3.865707in}{1.588088in}}%
\pgfpathlineto{\pgfqpoint{3.873906in}{1.598615in}}%
\pgfpathlineto{\pgfqpoint{3.882099in}{1.609145in}}%
\pgfpathlineto{\pgfqpoint{3.868089in}{1.610179in}}%
\pgfpathlineto{\pgfqpoint{3.854086in}{1.611287in}}%
\pgfpathlineto{\pgfqpoint{3.840091in}{1.612470in}}%
\pgfpathlineto{\pgfqpoint{3.826104in}{1.613728in}}%
\pgfpathlineto{\pgfqpoint{3.817900in}{1.603416in}}%
\pgfpathlineto{\pgfqpoint{3.809691in}{1.593114in}}%
\pgfpathlineto{\pgfqpoint{3.801475in}{1.582822in}}%
\pgfpathlineto{\pgfqpoint{3.793254in}{1.572546in}}%
\pgfpathclose%
\pgfusepath{fill}%
\end{pgfscope}%
\begin{pgfscope}%
\pgfpathrectangle{\pgfqpoint{1.150000in}{0.150000in}}{\pgfqpoint{5.700000in}{5.700000in}}%
\pgfusepath{clip}%
\pgfsetbuttcap%
\pgfsetroundjoin%
\definecolor{currentfill}{rgb}{0.268510,0.009605,0.335427}%
\pgfsetfillcolor{currentfill}%
\pgfsetfillopacity{0.700000}%
\pgfsetlinewidth{0.000000pt}%
\definecolor{currentstroke}{rgb}{0.000000,0.000000,0.000000}%
\pgfsetstrokecolor{currentstroke}%
\pgfsetdash{}{0pt}%
\pgfpathmoveto{\pgfqpoint{3.091017in}{1.509231in}}%
\pgfpathlineto{\pgfqpoint{3.104909in}{1.503140in}}%
\pgfpathlineto{\pgfqpoint{3.118804in}{1.497133in}}%
\pgfpathlineto{\pgfqpoint{3.132703in}{1.491208in}}%
\pgfpathlineto{\pgfqpoint{3.146606in}{1.485366in}}%
\pgfpathlineto{\pgfqpoint{3.155140in}{1.491239in}}%
\pgfpathlineto{\pgfqpoint{3.163664in}{1.497272in}}%
\pgfpathlineto{\pgfqpoint{3.172178in}{1.503459in}}%
\pgfpathlineto{\pgfqpoint{3.180682in}{1.509794in}}%
\pgfpathlineto{\pgfqpoint{3.166801in}{1.515287in}}%
\pgfpathlineto{\pgfqpoint{3.152925in}{1.520862in}}%
\pgfpathlineto{\pgfqpoint{3.139053in}{1.526519in}}%
\pgfpathlineto{\pgfqpoint{3.125185in}{1.532260in}}%
\pgfpathlineto{\pgfqpoint{3.116659in}{1.526266in}}%
\pgfpathlineto{\pgfqpoint{3.108122in}{1.520426in}}%
\pgfpathlineto{\pgfqpoint{3.099575in}{1.514746in}}%
\pgfpathlineto{\pgfqpoint{3.091017in}{1.509231in}}%
\pgfpathclose%
\pgfusepath{fill}%
\end{pgfscope}%
\begin{pgfscope}%
\pgfpathrectangle{\pgfqpoint{1.150000in}{0.150000in}}{\pgfqpoint{5.700000in}{5.700000in}}%
\pgfusepath{clip}%
\pgfsetbuttcap%
\pgfsetroundjoin%
\definecolor{currentfill}{rgb}{0.282623,0.140926,0.457517}%
\pgfsetfillcolor{currentfill}%
\pgfsetfillopacity{0.700000}%
\pgfsetlinewidth{0.000000pt}%
\definecolor{currentstroke}{rgb}{0.000000,0.000000,0.000000}%
\pgfsetstrokecolor{currentstroke}%
\pgfsetdash{}{0pt}%
\pgfpathmoveto{\pgfqpoint{2.485818in}{1.778756in}}%
\pgfpathlineto{\pgfqpoint{2.499733in}{1.768186in}}%
\pgfpathlineto{\pgfqpoint{2.513649in}{1.757718in}}%
\pgfpathlineto{\pgfqpoint{2.527564in}{1.747351in}}%
\pgfpathlineto{\pgfqpoint{2.541480in}{1.737084in}}%
\pgfpathlineto{\pgfqpoint{2.550454in}{1.736771in}}%
\pgfpathlineto{\pgfqpoint{2.559410in}{1.736739in}}%
\pgfpathlineto{\pgfqpoint{2.568348in}{1.736980in}}%
\pgfpathlineto{\pgfqpoint{2.577269in}{1.737490in}}%
\pgfpathlineto{\pgfqpoint{2.563390in}{1.747337in}}%
\pgfpathlineto{\pgfqpoint{2.549513in}{1.757284in}}%
\pgfpathlineto{\pgfqpoint{2.535635in}{1.767332in}}%
\pgfpathlineto{\pgfqpoint{2.521758in}{1.777480in}}%
\pgfpathlineto{\pgfqpoint{2.512800in}{1.777383in}}%
\pgfpathlineto{\pgfqpoint{2.503824in}{1.777558in}}%
\pgfpathlineto{\pgfqpoint{2.494831in}{1.778014in}}%
\pgfpathlineto{\pgfqpoint{2.485818in}{1.778756in}}%
\pgfpathclose%
\pgfusepath{fill}%
\end{pgfscope}%
\begin{pgfscope}%
\pgfpathrectangle{\pgfqpoint{1.150000in}{0.150000in}}{\pgfqpoint{5.700000in}{5.700000in}}%
\pgfusepath{clip}%
\pgfsetbuttcap%
\pgfsetroundjoin%
\definecolor{currentfill}{rgb}{0.267004,0.004874,0.329415}%
\pgfsetfillcolor{currentfill}%
\pgfsetfillopacity{0.700000}%
\pgfsetlinewidth{0.000000pt}%
\definecolor{currentstroke}{rgb}{0.000000,0.000000,0.000000}%
\pgfsetstrokecolor{currentstroke}%
\pgfsetdash{}{0pt}%
\pgfpathmoveto{\pgfqpoint{3.236248in}{1.488642in}}%
\pgfpathlineto{\pgfqpoint{3.250151in}{1.483557in}}%
\pgfpathlineto{\pgfqpoint{3.264058in}{1.478552in}}%
\pgfpathlineto{\pgfqpoint{3.277971in}{1.473627in}}%
\pgfpathlineto{\pgfqpoint{3.291888in}{1.468783in}}%
\pgfpathlineto{\pgfqpoint{3.300340in}{1.475935in}}%
\pgfpathlineto{\pgfqpoint{3.308784in}{1.483216in}}%
\pgfpathlineto{\pgfqpoint{3.317219in}{1.490621in}}%
\pgfpathlineto{\pgfqpoint{3.325645in}{1.498144in}}%
\pgfpathlineto{\pgfqpoint{3.311748in}{1.502659in}}%
\pgfpathlineto{\pgfqpoint{3.297855in}{1.507255in}}%
\pgfpathlineto{\pgfqpoint{3.283968in}{1.511931in}}%
\pgfpathlineto{\pgfqpoint{3.270085in}{1.516688in}}%
\pgfpathlineto{\pgfqpoint{3.261639in}{1.509486in}}%
\pgfpathlineto{\pgfqpoint{3.253184in}{1.502408in}}%
\pgfpathlineto{\pgfqpoint{3.244721in}{1.495458in}}%
\pgfpathlineto{\pgfqpoint{3.236248in}{1.488642in}}%
\pgfpathclose%
\pgfusepath{fill}%
\end{pgfscope}%
\begin{pgfscope}%
\pgfpathrectangle{\pgfqpoint{1.150000in}{0.150000in}}{\pgfqpoint{5.700000in}{5.700000in}}%
\pgfusepath{clip}%
\pgfsetbuttcap%
\pgfsetroundjoin%
\definecolor{currentfill}{rgb}{0.274952,0.037752,0.364543}%
\pgfsetfillcolor{currentfill}%
\pgfsetfillopacity{0.700000}%
\pgfsetlinewidth{0.000000pt}%
\definecolor{currentstroke}{rgb}{0.000000,0.000000,0.000000}%
\pgfsetstrokecolor{currentstroke}%
\pgfsetdash{}{0pt}%
\pgfpathmoveto{\pgfqpoint{3.704347in}{1.539325in}}%
\pgfpathlineto{\pgfqpoint{3.718328in}{1.537295in}}%
\pgfpathlineto{\pgfqpoint{3.732315in}{1.535340in}}%
\pgfpathlineto{\pgfqpoint{3.746310in}{1.533459in}}%
\pgfpathlineto{\pgfqpoint{3.760312in}{1.531654in}}%
\pgfpathlineto{\pgfqpoint{3.768556in}{1.541837in}}%
\pgfpathlineto{\pgfqpoint{3.776795in}{1.552050in}}%
\pgfpathlineto{\pgfqpoint{3.785028in}{1.562287in}}%
\pgfpathlineto{\pgfqpoint{3.793254in}{1.572546in}}%
\pgfpathlineto{\pgfqpoint{3.779264in}{1.574104in}}%
\pgfpathlineto{\pgfqpoint{3.765281in}{1.575738in}}%
\pgfpathlineto{\pgfqpoint{3.751305in}{1.577446in}}%
\pgfpathlineto{\pgfqpoint{3.737337in}{1.579230in}}%
\pgfpathlineto{\pgfqpoint{3.729099in}{1.569210in}}%
\pgfpathlineto{\pgfqpoint{3.720854in}{1.559217in}}%
\pgfpathlineto{\pgfqpoint{3.712604in}{1.549255in}}%
\pgfpathlineto{\pgfqpoint{3.704347in}{1.539325in}}%
\pgfpathclose%
\pgfusepath{fill}%
\end{pgfscope}%
\begin{pgfscope}%
\pgfpathrectangle{\pgfqpoint{1.150000in}{0.150000in}}{\pgfqpoint{5.700000in}{5.700000in}}%
\pgfusepath{clip}%
\pgfsetbuttcap%
\pgfsetroundjoin%
\definecolor{currentfill}{rgb}{0.279566,0.067836,0.391917}%
\pgfsetfillcolor{currentfill}%
\pgfsetfillopacity{0.700000}%
\pgfsetlinewidth{0.000000pt}%
\definecolor{currentstroke}{rgb}{0.000000,0.000000,0.000000}%
\pgfsetstrokecolor{currentstroke}%
\pgfsetdash{}{0pt}%
\pgfpathmoveto{\pgfqpoint{2.743874in}{1.626885in}}%
\pgfpathlineto{\pgfqpoint{2.757767in}{1.618281in}}%
\pgfpathlineto{\pgfqpoint{2.771661in}{1.609769in}}%
\pgfpathlineto{\pgfqpoint{2.785557in}{1.601348in}}%
\pgfpathlineto{\pgfqpoint{2.799454in}{1.593018in}}%
\pgfpathlineto{\pgfqpoint{2.808223in}{1.595405in}}%
\pgfpathlineto{\pgfqpoint{2.816978in}{1.598025in}}%
\pgfpathlineto{\pgfqpoint{2.825718in}{1.600871in}}%
\pgfpathlineto{\pgfqpoint{2.834445in}{1.603938in}}%
\pgfpathlineto{\pgfqpoint{2.820578in}{1.611874in}}%
\pgfpathlineto{\pgfqpoint{2.806714in}{1.619900in}}%
\pgfpathlineto{\pgfqpoint{2.792851in}{1.628018in}}%
\pgfpathlineto{\pgfqpoint{2.778991in}{1.636227in}}%
\pgfpathlineto{\pgfqpoint{2.770233in}{1.633546in}}%
\pgfpathlineto{\pgfqpoint{2.761462in}{1.631092in}}%
\pgfpathlineto{\pgfqpoint{2.752676in}{1.628870in}}%
\pgfpathlineto{\pgfqpoint{2.743874in}{1.626885in}}%
\pgfpathclose%
\pgfusepath{fill}%
\end{pgfscope}%
\begin{pgfscope}%
\pgfpathrectangle{\pgfqpoint{1.150000in}{0.150000in}}{\pgfqpoint{5.700000in}{5.700000in}}%
\pgfusepath{clip}%
\pgfsetbuttcap%
\pgfsetroundjoin%
\definecolor{currentfill}{rgb}{0.272594,0.025563,0.353093}%
\pgfsetfillcolor{currentfill}%
\pgfsetfillopacity{0.700000}%
\pgfsetlinewidth{0.000000pt}%
\definecolor{currentstroke}{rgb}{0.000000,0.000000,0.000000}%
\pgfsetstrokecolor{currentstroke}%
\pgfsetdash{}{0pt}%
\pgfpathmoveto{\pgfqpoint{2.945467in}{1.543660in}}%
\pgfpathlineto{\pgfqpoint{2.959357in}{1.536519in}}%
\pgfpathlineto{\pgfqpoint{2.973250in}{1.529465in}}%
\pgfpathlineto{\pgfqpoint{2.987146in}{1.522497in}}%
\pgfpathlineto{\pgfqpoint{3.001046in}{1.515614in}}%
\pgfpathlineto{\pgfqpoint{3.009674in}{1.520035in}}%
\pgfpathlineto{\pgfqpoint{3.018291in}{1.524649in}}%
\pgfpathlineto{\pgfqpoint{3.026896in}{1.529450in}}%
\pgfpathlineto{\pgfqpoint{3.035489in}{1.534431in}}%
\pgfpathlineto{\pgfqpoint{3.021616in}{1.540943in}}%
\pgfpathlineto{\pgfqpoint{3.007746in}{1.547540in}}%
\pgfpathlineto{\pgfqpoint{2.993880in}{1.554223in}}%
\pgfpathlineto{\pgfqpoint{2.980017in}{1.560991in}}%
\pgfpathlineto{\pgfqpoint{2.971398in}{1.556373in}}%
\pgfpathlineto{\pgfqpoint{2.962766in}{1.551941in}}%
\pgfpathlineto{\pgfqpoint{2.954123in}{1.547702in}}%
\pgfpathlineto{\pgfqpoint{2.945467in}{1.543660in}}%
\pgfpathclose%
\pgfusepath{fill}%
\end{pgfscope}%
\begin{pgfscope}%
\pgfpathrectangle{\pgfqpoint{1.150000in}{0.150000in}}{\pgfqpoint{5.700000in}{5.700000in}}%
\pgfusepath{clip}%
\pgfsetbuttcap%
\pgfsetroundjoin%
\definecolor{currentfill}{rgb}{0.177423,0.437527,0.557565}%
\pgfsetfillcolor{currentfill}%
\pgfsetfillopacity{0.700000}%
\pgfsetlinewidth{0.000000pt}%
\definecolor{currentstroke}{rgb}{0.000000,0.000000,0.000000}%
\pgfsetstrokecolor{currentstroke}%
\pgfsetdash{}{0pt}%
\pgfpathmoveto{\pgfqpoint{5.537753in}{2.445359in}}%
\pgfpathlineto{\pgfqpoint{5.552431in}{2.450328in}}%
\pgfpathlineto{\pgfqpoint{5.567123in}{2.455367in}}%
\pgfpathlineto{\pgfqpoint{5.581828in}{2.460477in}}%
\pgfpathlineto{\pgfqpoint{5.589295in}{2.465169in}}%
\pgfpathlineto{\pgfqpoint{5.596753in}{2.469762in}}%
\pgfpathlineto{\pgfqpoint{5.604201in}{2.474260in}}%
\pgfpathlineto{\pgfqpoint{5.611641in}{2.478666in}}%
\pgfpathlineto{\pgfqpoint{5.596955in}{2.473737in}}%
\pgfpathlineto{\pgfqpoint{5.582283in}{2.468878in}}%
\pgfpathlineto{\pgfqpoint{5.567624in}{2.464089in}}%
\pgfpathlineto{\pgfqpoint{5.560170in}{2.459541in}}%
\pgfpathlineto{\pgfqpoint{5.552707in}{2.454906in}}%
\pgfpathlineto{\pgfqpoint{5.545235in}{2.450180in}}%
\pgfpathlineto{\pgfqpoint{5.537753in}{2.445359in}}%
\pgfpathclose%
\pgfusepath{fill}%
\end{pgfscope}%
\begin{pgfscope}%
\pgfpathrectangle{\pgfqpoint{1.150000in}{0.150000in}}{\pgfqpoint{5.700000in}{5.700000in}}%
\pgfusepath{clip}%
\pgfsetbuttcap%
\pgfsetroundjoin%
\definecolor{currentfill}{rgb}{0.267004,0.004874,0.329415}%
\pgfsetfillcolor{currentfill}%
\pgfsetfillopacity{0.700000}%
\pgfsetlinewidth{0.000000pt}%
\definecolor{currentstroke}{rgb}{0.000000,0.000000,0.000000}%
\pgfsetstrokecolor{currentstroke}%
\pgfsetdash{}{0pt}%
\pgfpathmoveto{\pgfqpoint{3.381288in}{1.480873in}}%
\pgfpathlineto{\pgfqpoint{3.395212in}{1.476753in}}%
\pgfpathlineto{\pgfqpoint{3.409142in}{1.472711in}}%
\pgfpathlineto{\pgfqpoint{3.423077in}{1.468747in}}%
\pgfpathlineto{\pgfqpoint{3.437018in}{1.464861in}}%
\pgfpathlineto{\pgfqpoint{3.445400in}{1.473129in}}%
\pgfpathlineto{\pgfqpoint{3.453774in}{1.481495in}}%
\pgfpathlineto{\pgfqpoint{3.462141in}{1.489956in}}%
\pgfpathlineto{\pgfqpoint{3.470501in}{1.498506in}}%
\pgfpathlineto{\pgfqpoint{3.456577in}{1.502084in}}%
\pgfpathlineto{\pgfqpoint{3.442659in}{1.505740in}}%
\pgfpathlineto{\pgfqpoint{3.428746in}{1.509474in}}%
\pgfpathlineto{\pgfqpoint{3.414840in}{1.513287in}}%
\pgfpathlineto{\pgfqpoint{3.406463in}{1.505037in}}%
\pgfpathlineto{\pgfqpoint{3.398080in}{1.496882in}}%
\pgfpathlineto{\pgfqpoint{3.389688in}{1.488826in}}%
\pgfpathlineto{\pgfqpoint{3.381288in}{1.480873in}}%
\pgfpathclose%
\pgfusepath{fill}%
\end{pgfscope}%
\begin{pgfscope}%
\pgfpathrectangle{\pgfqpoint{1.150000in}{0.150000in}}{\pgfqpoint{5.700000in}{5.700000in}}%
\pgfusepath{clip}%
\pgfsetbuttcap%
\pgfsetroundjoin%
\definecolor{currentfill}{rgb}{0.271305,0.019942,0.347269}%
\pgfsetfillcolor{currentfill}%
\pgfsetfillopacity{0.700000}%
\pgfsetlinewidth{0.000000pt}%
\definecolor{currentstroke}{rgb}{0.000000,0.000000,0.000000}%
\pgfsetstrokecolor{currentstroke}%
\pgfsetdash{}{0pt}%
\pgfpathmoveto{\pgfqpoint{3.615357in}{1.509965in}}%
\pgfpathlineto{\pgfqpoint{3.629322in}{1.507365in}}%
\pgfpathlineto{\pgfqpoint{3.643295in}{1.504841in}}%
\pgfpathlineto{\pgfqpoint{3.657274in}{1.502392in}}%
\pgfpathlineto{\pgfqpoint{3.671260in}{1.500019in}}%
\pgfpathlineto{\pgfqpoint{3.679541in}{1.509777in}}%
\pgfpathlineto{\pgfqpoint{3.687816in}{1.519583in}}%
\pgfpathlineto{\pgfqpoint{3.696085in}{1.529434in}}%
\pgfpathlineto{\pgfqpoint{3.704347in}{1.539325in}}%
\pgfpathlineto{\pgfqpoint{3.690374in}{1.541431in}}%
\pgfpathlineto{\pgfqpoint{3.676408in}{1.543612in}}%
\pgfpathlineto{\pgfqpoint{3.662449in}{1.545869in}}%
\pgfpathlineto{\pgfqpoint{3.648497in}{1.548201in}}%
\pgfpathlineto{\pgfqpoint{3.640221in}{1.538569in}}%
\pgfpathlineto{\pgfqpoint{3.631940in}{1.528983in}}%
\pgfpathlineto{\pgfqpoint{3.623651in}{1.519447in}}%
\pgfpathlineto{\pgfqpoint{3.615357in}{1.509965in}}%
\pgfpathclose%
\pgfusepath{fill}%
\end{pgfscope}%
\begin{pgfscope}%
\pgfpathrectangle{\pgfqpoint{1.150000in}{0.150000in}}{\pgfqpoint{5.700000in}{5.700000in}}%
\pgfusepath{clip}%
\pgfsetbuttcap%
\pgfsetroundjoin%
\definecolor{currentfill}{rgb}{0.283187,0.125848,0.444960}%
\pgfsetfillcolor{currentfill}%
\pgfsetfillopacity{0.700000}%
\pgfsetlinewidth{0.000000pt}%
\definecolor{currentstroke}{rgb}{0.000000,0.000000,0.000000}%
\pgfsetstrokecolor{currentstroke}%
\pgfsetdash{}{0pt}%
\pgfpathmoveto{\pgfqpoint{2.541480in}{1.737084in}}%
\pgfpathlineto{\pgfqpoint{2.555396in}{1.726916in}}%
\pgfpathlineto{\pgfqpoint{2.569312in}{1.716847in}}%
\pgfpathlineto{\pgfqpoint{2.583229in}{1.706876in}}%
\pgfpathlineto{\pgfqpoint{2.597146in}{1.697002in}}%
\pgfpathlineto{\pgfqpoint{2.606082in}{1.697117in}}%
\pgfpathlineto{\pgfqpoint{2.615001in}{1.697507in}}%
\pgfpathlineto{\pgfqpoint{2.623903in}{1.698166in}}%
\pgfpathlineto{\pgfqpoint{2.632788in}{1.699086in}}%
\pgfpathlineto{\pgfqpoint{2.618907in}{1.708541in}}%
\pgfpathlineto{\pgfqpoint{2.605027in}{1.718092in}}%
\pgfpathlineto{\pgfqpoint{2.591147in}{1.727742in}}%
\pgfpathlineto{\pgfqpoint{2.577269in}{1.737490in}}%
\pgfpathlineto{\pgfqpoint{2.568348in}{1.736980in}}%
\pgfpathlineto{\pgfqpoint{2.559410in}{1.736739in}}%
\pgfpathlineto{\pgfqpoint{2.550454in}{1.736771in}}%
\pgfpathlineto{\pgfqpoint{2.541480in}{1.737084in}}%
\pgfpathclose%
\pgfusepath{fill}%
\end{pgfscope}%
\begin{pgfscope}%
\pgfpathrectangle{\pgfqpoint{1.150000in}{0.150000in}}{\pgfqpoint{5.700000in}{5.700000in}}%
\pgfusepath{clip}%
\pgfsetbuttcap%
\pgfsetroundjoin%
\definecolor{currentfill}{rgb}{0.182256,0.426184,0.557120}%
\pgfsetfillcolor{currentfill}%
\pgfsetfillopacity{0.700000}%
\pgfsetlinewidth{0.000000pt}%
\definecolor{currentstroke}{rgb}{0.000000,0.000000,0.000000}%
\pgfsetstrokecolor{currentstroke}%
\pgfsetdash{}{0pt}%
\pgfpathmoveto{\pgfqpoint{5.449089in}{2.405262in}}%
\pgfpathlineto{\pgfqpoint{5.463730in}{2.410107in}}%
\pgfpathlineto{\pgfqpoint{5.478385in}{2.415023in}}%
\pgfpathlineto{\pgfqpoint{5.493053in}{2.420009in}}%
\pgfpathlineto{\pgfqpoint{5.507735in}{2.425067in}}%
\pgfpathlineto{\pgfqpoint{5.515254in}{2.430297in}}%
\pgfpathlineto{\pgfqpoint{5.522763in}{2.435420in}}%
\pgfpathlineto{\pgfqpoint{5.530263in}{2.440440in}}%
\pgfpathlineto{\pgfqpoint{5.537753in}{2.445359in}}%
\pgfpathlineto{\pgfqpoint{5.523089in}{2.440460in}}%
\pgfpathlineto{\pgfqpoint{5.508439in}{2.435632in}}%
\pgfpathlineto{\pgfqpoint{5.493802in}{2.430874in}}%
\pgfpathlineto{\pgfqpoint{5.479178in}{2.426187in}}%
\pgfpathlineto{\pgfqpoint{5.471669in}{2.421102in}}%
\pgfpathlineto{\pgfqpoint{5.464152in}{2.415922in}}%
\pgfpathlineto{\pgfqpoint{5.456625in}{2.410642in}}%
\pgfpathlineto{\pgfqpoint{5.449089in}{2.405262in}}%
\pgfpathclose%
\pgfusepath{fill}%
\end{pgfscope}%
\begin{pgfscope}%
\pgfpathrectangle{\pgfqpoint{1.150000in}{0.150000in}}{\pgfqpoint{5.700000in}{5.700000in}}%
\pgfusepath{clip}%
\pgfsetbuttcap%
\pgfsetroundjoin%
\definecolor{currentfill}{rgb}{0.282290,0.145912,0.461510}%
\pgfsetfillcolor{currentfill}%
\pgfsetfillopacity{0.700000}%
\pgfsetlinewidth{0.000000pt}%
\definecolor{currentstroke}{rgb}{0.000000,0.000000,0.000000}%
\pgfsetstrokecolor{currentstroke}%
\pgfsetdash{}{0pt}%
\pgfpathmoveto{\pgfqpoint{4.204864in}{1.737088in}}%
\pgfpathlineto{\pgfqpoint{4.218996in}{1.737857in}}%
\pgfpathlineto{\pgfqpoint{4.233138in}{1.738698in}}%
\pgfpathlineto{\pgfqpoint{4.247289in}{1.739611in}}%
\pgfpathlineto{\pgfqpoint{4.261449in}{1.740596in}}%
\pgfpathlineto{\pgfqpoint{4.269529in}{1.751680in}}%
\pgfpathlineto{\pgfqpoint{4.277603in}{1.762706in}}%
\pgfpathlineto{\pgfqpoint{4.285671in}{1.773672in}}%
\pgfpathlineto{\pgfqpoint{4.293734in}{1.784576in}}%
\pgfpathlineto{\pgfqpoint{4.279581in}{1.783445in}}%
\pgfpathlineto{\pgfqpoint{4.265438in}{1.782387in}}%
\pgfpathlineto{\pgfqpoint{4.251304in}{1.781402in}}%
\pgfpathlineto{\pgfqpoint{4.237180in}{1.780489in}}%
\pgfpathlineto{\pgfqpoint{4.229109in}{1.769722in}}%
\pgfpathlineto{\pgfqpoint{4.221033in}{1.758898in}}%
\pgfpathlineto{\pgfqpoint{4.212952in}{1.748020in}}%
\pgfpathlineto{\pgfqpoint{4.204864in}{1.737088in}}%
\pgfpathclose%
\pgfusepath{fill}%
\end{pgfscope}%
\begin{pgfscope}%
\pgfpathrectangle{\pgfqpoint{1.150000in}{0.150000in}}{\pgfqpoint{5.700000in}{5.700000in}}%
\pgfusepath{clip}%
\pgfsetbuttcap%
\pgfsetroundjoin%
\definecolor{currentfill}{rgb}{0.279574,0.170599,0.479997}%
\pgfsetfillcolor{currentfill}%
\pgfsetfillopacity{0.700000}%
\pgfsetlinewidth{0.000000pt}%
\definecolor{currentstroke}{rgb}{0.000000,0.000000,0.000000}%
\pgfsetstrokecolor{currentstroke}%
\pgfsetdash{}{0pt}%
\pgfpathmoveto{\pgfqpoint{4.293734in}{1.784576in}}%
\pgfpathlineto{\pgfqpoint{4.307897in}{1.785778in}}%
\pgfpathlineto{\pgfqpoint{4.322070in}{1.787053in}}%
\pgfpathlineto{\pgfqpoint{4.336252in}{1.788400in}}%
\pgfpathlineto{\pgfqpoint{4.350445in}{1.789819in}}%
\pgfpathlineto{\pgfqpoint{4.358495in}{1.800791in}}%
\pgfpathlineto{\pgfqpoint{4.366539in}{1.811693in}}%
\pgfpathlineto{\pgfqpoint{4.374578in}{1.822522in}}%
\pgfpathlineto{\pgfqpoint{4.382611in}{1.833279in}}%
\pgfpathlineto{\pgfqpoint{4.368426in}{1.831736in}}%
\pgfpathlineto{\pgfqpoint{4.354251in}{1.830265in}}%
\pgfpathlineto{\pgfqpoint{4.340085in}{1.828866in}}%
\pgfpathlineto{\pgfqpoint{4.325930in}{1.827540in}}%
\pgfpathlineto{\pgfqpoint{4.317890in}{1.816900in}}%
\pgfpathlineto{\pgfqpoint{4.309844in}{1.806191in}}%
\pgfpathlineto{\pgfqpoint{4.301792in}{1.795416in}}%
\pgfpathlineto{\pgfqpoint{4.293734in}{1.784576in}}%
\pgfpathclose%
\pgfusepath{fill}%
\end{pgfscope}%
\begin{pgfscope}%
\pgfpathrectangle{\pgfqpoint{1.150000in}{0.150000in}}{\pgfqpoint{5.700000in}{5.700000in}}%
\pgfusepath{clip}%
\pgfsetbuttcap%
\pgfsetroundjoin%
\definecolor{currentfill}{rgb}{0.283187,0.125848,0.444960}%
\pgfsetfillcolor{currentfill}%
\pgfsetfillopacity{0.700000}%
\pgfsetlinewidth{0.000000pt}%
\definecolor{currentstroke}{rgb}{0.000000,0.000000,0.000000}%
\pgfsetstrokecolor{currentstroke}%
\pgfsetdash{}{0pt}%
\pgfpathmoveto{\pgfqpoint{4.115995in}{1.691189in}}%
\pgfpathlineto{\pgfqpoint{4.130098in}{1.691501in}}%
\pgfpathlineto{\pgfqpoint{4.144210in}{1.691886in}}%
\pgfpathlineto{\pgfqpoint{4.158331in}{1.692344in}}%
\pgfpathlineto{\pgfqpoint{4.172461in}{1.692875in}}%
\pgfpathlineto{\pgfqpoint{4.180570in}{1.703997in}}%
\pgfpathlineto{\pgfqpoint{4.188674in}{1.715075in}}%
\pgfpathlineto{\pgfqpoint{4.196772in}{1.726106in}}%
\pgfpathlineto{\pgfqpoint{4.204864in}{1.737088in}}%
\pgfpathlineto{\pgfqpoint{4.190742in}{1.736392in}}%
\pgfpathlineto{\pgfqpoint{4.176629in}{1.735769in}}%
\pgfpathlineto{\pgfqpoint{4.162525in}{1.735218in}}%
\pgfpathlineto{\pgfqpoint{4.148430in}{1.734740in}}%
\pgfpathlineto{\pgfqpoint{4.140330in}{1.723916in}}%
\pgfpathlineto{\pgfqpoint{4.132223in}{1.713047in}}%
\pgfpathlineto{\pgfqpoint{4.124112in}{1.702138in}}%
\pgfpathlineto{\pgfqpoint{4.115995in}{1.691189in}}%
\pgfpathclose%
\pgfusepath{fill}%
\end{pgfscope}%
\begin{pgfscope}%
\pgfpathrectangle{\pgfqpoint{1.150000in}{0.150000in}}{\pgfqpoint{5.700000in}{5.700000in}}%
\pgfusepath{clip}%
\pgfsetbuttcap%
\pgfsetroundjoin%
\definecolor{currentfill}{rgb}{0.275191,0.194905,0.496005}%
\pgfsetfillcolor{currentfill}%
\pgfsetfillopacity{0.700000}%
\pgfsetlinewidth{0.000000pt}%
\definecolor{currentstroke}{rgb}{0.000000,0.000000,0.000000}%
\pgfsetstrokecolor{currentstroke}%
\pgfsetdash{}{0pt}%
\pgfpathmoveto{\pgfqpoint{4.382611in}{1.833279in}}%
\pgfpathlineto{\pgfqpoint{4.396806in}{1.834893in}}%
\pgfpathlineto{\pgfqpoint{4.411012in}{1.836580in}}%
\pgfpathlineto{\pgfqpoint{4.425228in}{1.838339in}}%
\pgfpathlineto{\pgfqpoint{4.439454in}{1.840170in}}%
\pgfpathlineto{\pgfqpoint{4.447474in}{1.850962in}}%
\pgfpathlineto{\pgfqpoint{4.455488in}{1.861673in}}%
\pgfpathlineto{\pgfqpoint{4.463496in}{1.872301in}}%
\pgfpathlineto{\pgfqpoint{4.471498in}{1.882845in}}%
\pgfpathlineto{\pgfqpoint{4.457279in}{1.880911in}}%
\pgfpathlineto{\pgfqpoint{4.443071in}{1.879050in}}%
\pgfpathlineto{\pgfqpoint{4.428872in}{1.877260in}}%
\pgfpathlineto{\pgfqpoint{4.414685in}{1.875542in}}%
\pgfpathlineto{\pgfqpoint{4.406675in}{1.865093in}}%
\pgfpathlineto{\pgfqpoint{4.398659in}{1.854565in}}%
\pgfpathlineto{\pgfqpoint{4.390638in}{1.843960in}}%
\pgfpathlineto{\pgfqpoint{4.382611in}{1.833279in}}%
\pgfpathclose%
\pgfusepath{fill}%
\end{pgfscope}%
\begin{pgfscope}%
\pgfpathrectangle{\pgfqpoint{1.150000in}{0.150000in}}{\pgfqpoint{5.700000in}{5.700000in}}%
\pgfusepath{clip}%
\pgfsetbuttcap%
\pgfsetroundjoin%
\definecolor{currentfill}{rgb}{0.278791,0.062145,0.386592}%
\pgfsetfillcolor{currentfill}%
\pgfsetfillopacity{0.700000}%
\pgfsetlinewidth{0.000000pt}%
\definecolor{currentstroke}{rgb}{0.000000,0.000000,0.000000}%
\pgfsetstrokecolor{currentstroke}%
\pgfsetdash{}{0pt}%
\pgfpathmoveto{\pgfqpoint{2.799454in}{1.593018in}}%
\pgfpathlineto{\pgfqpoint{2.813354in}{1.584778in}}%
\pgfpathlineto{\pgfqpoint{2.827256in}{1.576628in}}%
\pgfpathlineto{\pgfqpoint{2.841160in}{1.568567in}}%
\pgfpathlineto{\pgfqpoint{2.855066in}{1.560596in}}%
\pgfpathlineto{\pgfqpoint{2.863804in}{1.563385in}}%
\pgfpathlineto{\pgfqpoint{2.872528in}{1.566402in}}%
\pgfpathlineto{\pgfqpoint{2.881238in}{1.569639in}}%
\pgfpathlineto{\pgfqpoint{2.889935in}{1.573092in}}%
\pgfpathlineto{\pgfqpoint{2.876059in}{1.580670in}}%
\pgfpathlineto{\pgfqpoint{2.862185in}{1.588337in}}%
\pgfpathlineto{\pgfqpoint{2.848314in}{1.596092in}}%
\pgfpathlineto{\pgfqpoint{2.834445in}{1.603938in}}%
\pgfpathlineto{\pgfqpoint{2.825718in}{1.600871in}}%
\pgfpathlineto{\pgfqpoint{2.816978in}{1.598025in}}%
\pgfpathlineto{\pgfqpoint{2.808223in}{1.595405in}}%
\pgfpathlineto{\pgfqpoint{2.799454in}{1.593018in}}%
\pgfpathclose%
\pgfusepath{fill}%
\end{pgfscope}%
\begin{pgfscope}%
\pgfpathrectangle{\pgfqpoint{1.150000in}{0.150000in}}{\pgfqpoint{5.700000in}{5.700000in}}%
\pgfusepath{clip}%
\pgfsetbuttcap%
\pgfsetroundjoin%
\definecolor{currentfill}{rgb}{0.282910,0.105393,0.426902}%
\pgfsetfillcolor{currentfill}%
\pgfsetfillopacity{0.700000}%
\pgfsetlinewidth{0.000000pt}%
\definecolor{currentstroke}{rgb}{0.000000,0.000000,0.000000}%
\pgfsetstrokecolor{currentstroke}%
\pgfsetdash{}{0pt}%
\pgfpathmoveto{\pgfqpoint{4.027118in}{1.647270in}}%
\pgfpathlineto{\pgfqpoint{4.041194in}{1.647105in}}%
\pgfpathlineto{\pgfqpoint{4.055278in}{1.647013in}}%
\pgfpathlineto{\pgfqpoint{4.069372in}{1.646993in}}%
\pgfpathlineto{\pgfqpoint{4.083474in}{1.647047in}}%
\pgfpathlineto{\pgfqpoint{4.091612in}{1.658129in}}%
\pgfpathlineto{\pgfqpoint{4.099745in}{1.669182in}}%
\pgfpathlineto{\pgfqpoint{4.107873in}{1.680202in}}%
\pgfpathlineto{\pgfqpoint{4.115995in}{1.691189in}}%
\pgfpathlineto{\pgfqpoint{4.101901in}{1.690949in}}%
\pgfpathlineto{\pgfqpoint{4.087816in}{1.690782in}}%
\pgfpathlineto{\pgfqpoint{4.073741in}{1.690688in}}%
\pgfpathlineto{\pgfqpoint{4.059673in}{1.690668in}}%
\pgfpathlineto{\pgfqpoint{4.051543in}{1.679860in}}%
\pgfpathlineto{\pgfqpoint{4.043407in}{1.669023in}}%
\pgfpathlineto{\pgfqpoint{4.035265in}{1.658158in}}%
\pgfpathlineto{\pgfqpoint{4.027118in}{1.647270in}}%
\pgfpathclose%
\pgfusepath{fill}%
\end{pgfscope}%
\begin{pgfscope}%
\pgfpathrectangle{\pgfqpoint{1.150000in}{0.150000in}}{\pgfqpoint{5.700000in}{5.700000in}}%
\pgfusepath{clip}%
\pgfsetbuttcap%
\pgfsetroundjoin%
\definecolor{currentfill}{rgb}{0.269308,0.218818,0.509577}%
\pgfsetfillcolor{currentfill}%
\pgfsetfillopacity{0.700000}%
\pgfsetlinewidth{0.000000pt}%
\definecolor{currentstroke}{rgb}{0.000000,0.000000,0.000000}%
\pgfsetstrokecolor{currentstroke}%
\pgfsetdash{}{0pt}%
\pgfpathmoveto{\pgfqpoint{4.471498in}{1.882845in}}%
\pgfpathlineto{\pgfqpoint{4.485727in}{1.884850in}}%
\pgfpathlineto{\pgfqpoint{4.499967in}{1.886927in}}%
\pgfpathlineto{\pgfqpoint{4.514217in}{1.889076in}}%
\pgfpathlineto{\pgfqpoint{4.528479in}{1.891297in}}%
\pgfpathlineto{\pgfqpoint{4.536467in}{1.901846in}}%
\pgfpathlineto{\pgfqpoint{4.544450in}{1.912304in}}%
\pgfpathlineto{\pgfqpoint{4.552426in}{1.922670in}}%
\pgfpathlineto{\pgfqpoint{4.560396in}{1.932943in}}%
\pgfpathlineto{\pgfqpoint{4.546142in}{1.930641in}}%
\pgfpathlineto{\pgfqpoint{4.531900in}{1.928410in}}%
\pgfpathlineto{\pgfqpoint{4.517667in}{1.926251in}}%
\pgfpathlineto{\pgfqpoint{4.503446in}{1.924164in}}%
\pgfpathlineto{\pgfqpoint{4.495468in}{1.913964in}}%
\pgfpathlineto{\pgfqpoint{4.487484in}{1.903678in}}%
\pgfpathlineto{\pgfqpoint{4.479494in}{1.893304in}}%
\pgfpathlineto{\pgfqpoint{4.471498in}{1.882845in}}%
\pgfpathclose%
\pgfusepath{fill}%
\end{pgfscope}%
\begin{pgfscope}%
\pgfpathrectangle{\pgfqpoint{1.150000in}{0.150000in}}{\pgfqpoint{5.700000in}{5.700000in}}%
\pgfusepath{clip}%
\pgfsetbuttcap%
\pgfsetroundjoin%
\definecolor{currentfill}{rgb}{0.188923,0.410910,0.556326}%
\pgfsetfillcolor{currentfill}%
\pgfsetfillopacity{0.700000}%
\pgfsetlinewidth{0.000000pt}%
\definecolor{currentstroke}{rgb}{0.000000,0.000000,0.000000}%
\pgfsetstrokecolor{currentstroke}%
\pgfsetdash{}{0pt}%
\pgfpathmoveto{\pgfqpoint{5.360358in}{2.363449in}}%
\pgfpathlineto{\pgfqpoint{5.374962in}{2.368147in}}%
\pgfpathlineto{\pgfqpoint{5.389579in}{2.372917in}}%
\pgfpathlineto{\pgfqpoint{5.404210in}{2.377757in}}%
\pgfpathlineto{\pgfqpoint{5.418854in}{2.382669in}}%
\pgfpathlineto{\pgfqpoint{5.426427in}{2.388483in}}%
\pgfpathlineto{\pgfqpoint{5.433990in}{2.394185in}}%
\pgfpathlineto{\pgfqpoint{5.441544in}{2.399777in}}%
\pgfpathlineto{\pgfqpoint{5.449089in}{2.405262in}}%
\pgfpathlineto{\pgfqpoint{5.434461in}{2.400487in}}%
\pgfpathlineto{\pgfqpoint{5.419847in}{2.395783in}}%
\pgfpathlineto{\pgfqpoint{5.405246in}{2.391150in}}%
\pgfpathlineto{\pgfqpoint{5.390658in}{2.386587in}}%
\pgfpathlineto{\pgfqpoint{5.383096in}{2.380958in}}%
\pgfpathlineto{\pgfqpoint{5.375525in}{2.375227in}}%
\pgfpathlineto{\pgfqpoint{5.367946in}{2.369391in}}%
\pgfpathlineto{\pgfqpoint{5.360358in}{2.363449in}}%
\pgfpathclose%
\pgfusepath{fill}%
\end{pgfscope}%
\begin{pgfscope}%
\pgfpathrectangle{\pgfqpoint{1.150000in}{0.150000in}}{\pgfqpoint{5.700000in}{5.700000in}}%
\pgfusepath{clip}%
\pgfsetbuttcap%
\pgfsetroundjoin%
\definecolor{currentfill}{rgb}{0.268510,0.009605,0.335427}%
\pgfsetfillcolor{currentfill}%
\pgfsetfillopacity{0.700000}%
\pgfsetlinewidth{0.000000pt}%
\definecolor{currentstroke}{rgb}{0.000000,0.000000,0.000000}%
\pgfsetstrokecolor{currentstroke}%
\pgfsetdash{}{0pt}%
\pgfpathmoveto{\pgfqpoint{3.526256in}{1.484967in}}%
\pgfpathlineto{\pgfqpoint{3.540211in}{1.481775in}}%
\pgfpathlineto{\pgfqpoint{3.554171in}{1.478659in}}%
\pgfpathlineto{\pgfqpoint{3.568138in}{1.475619in}}%
\pgfpathlineto{\pgfqpoint{3.582111in}{1.472656in}}%
\pgfpathlineto{\pgfqpoint{3.590433in}{1.481882in}}%
\pgfpathlineto{\pgfqpoint{3.598747in}{1.491178in}}%
\pgfpathlineto{\pgfqpoint{3.607055in}{1.500540in}}%
\pgfpathlineto{\pgfqpoint{3.615357in}{1.509965in}}%
\pgfpathlineto{\pgfqpoint{3.601398in}{1.512640in}}%
\pgfpathlineto{\pgfqpoint{3.587445in}{1.515392in}}%
\pgfpathlineto{\pgfqpoint{3.573500in}{1.518220in}}%
\pgfpathlineto{\pgfqpoint{3.559560in}{1.521125in}}%
\pgfpathlineto{\pgfqpoint{3.551245in}{1.511981in}}%
\pgfpathlineto{\pgfqpoint{3.542922in}{1.502903in}}%
\pgfpathlineto{\pgfqpoint{3.534593in}{1.493898in}}%
\pgfpathlineto{\pgfqpoint{3.526256in}{1.484967in}}%
\pgfpathclose%
\pgfusepath{fill}%
\end{pgfscope}%
\begin{pgfscope}%
\pgfpathrectangle{\pgfqpoint{1.150000in}{0.150000in}}{\pgfqpoint{5.700000in}{5.700000in}}%
\pgfusepath{clip}%
\pgfsetbuttcap%
\pgfsetroundjoin%
\definecolor{currentfill}{rgb}{0.262138,0.242286,0.520837}%
\pgfsetfillcolor{currentfill}%
\pgfsetfillopacity{0.700000}%
\pgfsetlinewidth{0.000000pt}%
\definecolor{currentstroke}{rgb}{0.000000,0.000000,0.000000}%
\pgfsetstrokecolor{currentstroke}%
\pgfsetdash{}{0pt}%
\pgfpathmoveto{\pgfqpoint{4.560396in}{1.932943in}}%
\pgfpathlineto{\pgfqpoint{4.574661in}{1.935317in}}%
\pgfpathlineto{\pgfqpoint{4.588936in}{1.937763in}}%
\pgfpathlineto{\pgfqpoint{4.603222in}{1.940280in}}%
\pgfpathlineto{\pgfqpoint{4.617520in}{1.942869in}}%
\pgfpathlineto{\pgfqpoint{4.625476in}{1.953117in}}%
\pgfpathlineto{\pgfqpoint{4.633425in}{1.963266in}}%
\pgfpathlineto{\pgfqpoint{4.641369in}{1.973315in}}%
\pgfpathlineto{\pgfqpoint{4.649305in}{1.983263in}}%
\pgfpathlineto{\pgfqpoint{4.635016in}{1.980614in}}%
\pgfpathlineto{\pgfqpoint{4.620737in}{1.978036in}}%
\pgfpathlineto{\pgfqpoint{4.606470in}{1.975530in}}%
\pgfpathlineto{\pgfqpoint{4.592213in}{1.973095in}}%
\pgfpathlineto{\pgfqpoint{4.584268in}{1.963199in}}%
\pgfpathlineto{\pgfqpoint{4.576317in}{1.953208in}}%
\pgfpathlineto{\pgfqpoint{4.568360in}{1.943123in}}%
\pgfpathlineto{\pgfqpoint{4.560396in}{1.932943in}}%
\pgfpathclose%
\pgfusepath{fill}%
\end{pgfscope}%
\begin{pgfscope}%
\pgfpathrectangle{\pgfqpoint{1.150000in}{0.150000in}}{\pgfqpoint{5.700000in}{5.700000in}}%
\pgfusepath{clip}%
\pgfsetbuttcap%
\pgfsetroundjoin%
\definecolor{currentfill}{rgb}{0.267004,0.004874,0.329415}%
\pgfsetfillcolor{currentfill}%
\pgfsetfillopacity{0.700000}%
\pgfsetlinewidth{0.000000pt}%
\definecolor{currentstroke}{rgb}{0.000000,0.000000,0.000000}%
\pgfsetstrokecolor{currentstroke}%
\pgfsetdash{}{0pt}%
\pgfpathmoveto{\pgfqpoint{3.146606in}{1.485366in}}%
\pgfpathlineto{\pgfqpoint{3.160513in}{1.479606in}}%
\pgfpathlineto{\pgfqpoint{3.174425in}{1.473927in}}%
\pgfpathlineto{\pgfqpoint{3.188341in}{1.468330in}}%
\pgfpathlineto{\pgfqpoint{3.202261in}{1.462815in}}%
\pgfpathlineto{\pgfqpoint{3.210772in}{1.469045in}}%
\pgfpathlineto{\pgfqpoint{3.219274in}{1.475430in}}%
\pgfpathlineto{\pgfqpoint{3.227765in}{1.481964in}}%
\pgfpathlineto{\pgfqpoint{3.236248in}{1.488642in}}%
\pgfpathlineto{\pgfqpoint{3.222349in}{1.493808in}}%
\pgfpathlineto{\pgfqpoint{3.208456in}{1.499055in}}%
\pgfpathlineto{\pgfqpoint{3.194567in}{1.504384in}}%
\pgfpathlineto{\pgfqpoint{3.180682in}{1.509794in}}%
\pgfpathlineto{\pgfqpoint{3.172178in}{1.503459in}}%
\pgfpathlineto{\pgfqpoint{3.163664in}{1.497272in}}%
\pgfpathlineto{\pgfqpoint{3.155140in}{1.491239in}}%
\pgfpathlineto{\pgfqpoint{3.146606in}{1.485366in}}%
\pgfpathclose%
\pgfusepath{fill}%
\end{pgfscope}%
\begin{pgfscope}%
\pgfpathrectangle{\pgfqpoint{1.150000in}{0.150000in}}{\pgfqpoint{5.700000in}{5.700000in}}%
\pgfusepath{clip}%
\pgfsetbuttcap%
\pgfsetroundjoin%
\definecolor{currentfill}{rgb}{0.281446,0.084320,0.407414}%
\pgfsetfillcolor{currentfill}%
\pgfsetfillopacity{0.700000}%
\pgfsetlinewidth{0.000000pt}%
\definecolor{currentstroke}{rgb}{0.000000,0.000000,0.000000}%
\pgfsetstrokecolor{currentstroke}%
\pgfsetdash{}{0pt}%
\pgfpathmoveto{\pgfqpoint{3.938222in}{1.605747in}}%
\pgfpathlineto{\pgfqpoint{3.952273in}{1.605082in}}%
\pgfpathlineto{\pgfqpoint{3.966332in}{1.604491in}}%
\pgfpathlineto{\pgfqpoint{3.980400in}{1.603973in}}%
\pgfpathlineto{\pgfqpoint{3.994476in}{1.603528in}}%
\pgfpathlineto{\pgfqpoint{4.002645in}{1.614486in}}%
\pgfpathlineto{\pgfqpoint{4.010808in}{1.625431in}}%
\pgfpathlineto{\pgfqpoint{4.018966in}{1.636360in}}%
\pgfpathlineto{\pgfqpoint{4.027118in}{1.647270in}}%
\pgfpathlineto{\pgfqpoint{4.013051in}{1.647509in}}%
\pgfpathlineto{\pgfqpoint{3.998992in}{1.647820in}}%
\pgfpathlineto{\pgfqpoint{3.984942in}{1.648205in}}%
\pgfpathlineto{\pgfqpoint{3.970901in}{1.648664in}}%
\pgfpathlineto{\pgfqpoint{3.962739in}{1.637953in}}%
\pgfpathlineto{\pgfqpoint{3.954572in}{1.627228in}}%
\pgfpathlineto{\pgfqpoint{3.946400in}{1.616492in}}%
\pgfpathlineto{\pgfqpoint{3.938222in}{1.605747in}}%
\pgfpathclose%
\pgfusepath{fill}%
\end{pgfscope}%
\begin{pgfscope}%
\pgfpathrectangle{\pgfqpoint{1.150000in}{0.150000in}}{\pgfqpoint{5.700000in}{5.700000in}}%
\pgfusepath{clip}%
\pgfsetbuttcap%
\pgfsetroundjoin%
\definecolor{currentfill}{rgb}{0.255645,0.260703,0.528312}%
\pgfsetfillcolor{currentfill}%
\pgfsetfillopacity{0.700000}%
\pgfsetlinewidth{0.000000pt}%
\definecolor{currentstroke}{rgb}{0.000000,0.000000,0.000000}%
\pgfsetstrokecolor{currentstroke}%
\pgfsetdash{}{0pt}%
\pgfpathmoveto{\pgfqpoint{4.649305in}{1.983263in}}%
\pgfpathlineto{\pgfqpoint{4.663606in}{1.985984in}}%
\pgfpathlineto{\pgfqpoint{4.677918in}{1.988776in}}%
\pgfpathlineto{\pgfqpoint{4.692241in}{1.991640in}}%
\pgfpathlineto{\pgfqpoint{4.706575in}{1.994576in}}%
\pgfpathlineto{\pgfqpoint{4.714497in}{2.004470in}}%
\pgfpathlineto{\pgfqpoint{4.722413in}{2.014259in}}%
\pgfpathlineto{\pgfqpoint{4.730321in}{2.023940in}}%
\pgfpathlineto{\pgfqpoint{4.738223in}{2.033515in}}%
\pgfpathlineto{\pgfqpoint{4.723896in}{2.030540in}}%
\pgfpathlineto{\pgfqpoint{4.709582in}{2.027637in}}%
\pgfpathlineto{\pgfqpoint{4.695278in}{2.024806in}}%
\pgfpathlineto{\pgfqpoint{4.680986in}{2.022046in}}%
\pgfpathlineto{\pgfqpoint{4.673076in}{2.012502in}}%
\pgfpathlineto{\pgfqpoint{4.665159in}{2.002857in}}%
\pgfpathlineto{\pgfqpoint{4.657235in}{1.993111in}}%
\pgfpathlineto{\pgfqpoint{4.649305in}{1.983263in}}%
\pgfpathclose%
\pgfusepath{fill}%
\end{pgfscope}%
\begin{pgfscope}%
\pgfpathrectangle{\pgfqpoint{1.150000in}{0.150000in}}{\pgfqpoint{5.700000in}{5.700000in}}%
\pgfusepath{clip}%
\pgfsetbuttcap%
\pgfsetroundjoin%
\definecolor{currentfill}{rgb}{0.195860,0.395433,0.555276}%
\pgfsetfillcolor{currentfill}%
\pgfsetfillopacity{0.700000}%
\pgfsetlinewidth{0.000000pt}%
\definecolor{currentstroke}{rgb}{0.000000,0.000000,0.000000}%
\pgfsetstrokecolor{currentstroke}%
\pgfsetdash{}{0pt}%
\pgfpathmoveto{\pgfqpoint{5.271570in}{2.320015in}}%
\pgfpathlineto{\pgfqpoint{5.286137in}{2.324544in}}%
\pgfpathlineto{\pgfqpoint{5.300716in}{2.329145in}}%
\pgfpathlineto{\pgfqpoint{5.315309in}{2.333817in}}%
\pgfpathlineto{\pgfqpoint{5.329915in}{2.338559in}}%
\pgfpathlineto{\pgfqpoint{5.337539in}{2.344954in}}%
\pgfpathlineto{\pgfqpoint{5.345154in}{2.351232in}}%
\pgfpathlineto{\pgfqpoint{5.352760in}{2.357397in}}%
\pgfpathlineto{\pgfqpoint{5.360358in}{2.363449in}}%
\pgfpathlineto{\pgfqpoint{5.345767in}{2.358821in}}%
\pgfpathlineto{\pgfqpoint{5.331189in}{2.354263in}}%
\pgfpathlineto{\pgfqpoint{5.316625in}{2.349776in}}%
\pgfpathlineto{\pgfqpoint{5.302073in}{2.345360in}}%
\pgfpathlineto{\pgfqpoint{5.294460in}{2.339186in}}%
\pgfpathlineto{\pgfqpoint{5.286839in}{2.332905in}}%
\pgfpathlineto{\pgfqpoint{5.279209in}{2.326516in}}%
\pgfpathlineto{\pgfqpoint{5.271570in}{2.320015in}}%
\pgfpathclose%
\pgfusepath{fill}%
\end{pgfscope}%
\begin{pgfscope}%
\pgfpathrectangle{\pgfqpoint{1.150000in}{0.150000in}}{\pgfqpoint{5.700000in}{5.700000in}}%
\pgfusepath{clip}%
\pgfsetbuttcap%
\pgfsetroundjoin%
\definecolor{currentfill}{rgb}{0.246811,0.283237,0.535941}%
\pgfsetfillcolor{currentfill}%
\pgfsetfillopacity{0.700000}%
\pgfsetlinewidth{0.000000pt}%
\definecolor{currentstroke}{rgb}{0.000000,0.000000,0.000000}%
\pgfsetstrokecolor{currentstroke}%
\pgfsetdash{}{0pt}%
\pgfpathmoveto{\pgfqpoint{4.738223in}{2.033515in}}%
\pgfpathlineto{\pgfqpoint{4.752560in}{2.036561in}}%
\pgfpathlineto{\pgfqpoint{4.766909in}{2.039678in}}%
\pgfpathlineto{\pgfqpoint{4.781270in}{2.042867in}}%
\pgfpathlineto{\pgfqpoint{4.795642in}{2.046126in}}%
\pgfpathlineto{\pgfqpoint{4.803528in}{2.055620in}}%
\pgfpathlineto{\pgfqpoint{4.811407in}{2.065002in}}%
\pgfpathlineto{\pgfqpoint{4.819279in}{2.074271in}}%
\pgfpathlineto{\pgfqpoint{4.827144in}{2.083429in}}%
\pgfpathlineto{\pgfqpoint{4.812780in}{2.080152in}}%
\pgfpathlineto{\pgfqpoint{4.798429in}{2.076946in}}%
\pgfpathlineto{\pgfqpoint{4.784088in}{2.073811in}}%
\pgfpathlineto{\pgfqpoint{4.769760in}{2.070748in}}%
\pgfpathlineto{\pgfqpoint{4.761886in}{2.061599in}}%
\pgfpathlineto{\pgfqpoint{4.754005in}{2.052344in}}%
\pgfpathlineto{\pgfqpoint{4.746117in}{2.042983in}}%
\pgfpathlineto{\pgfqpoint{4.738223in}{2.033515in}}%
\pgfpathclose%
\pgfusepath{fill}%
\end{pgfscope}%
\begin{pgfscope}%
\pgfpathrectangle{\pgfqpoint{1.150000in}{0.150000in}}{\pgfqpoint{5.700000in}{5.700000in}}%
\pgfusepath{clip}%
\pgfsetbuttcap%
\pgfsetroundjoin%
\definecolor{currentfill}{rgb}{0.267004,0.004874,0.329415}%
\pgfsetfillcolor{currentfill}%
\pgfsetfillopacity{0.700000}%
\pgfsetlinewidth{0.000000pt}%
\definecolor{currentstroke}{rgb}{0.000000,0.000000,0.000000}%
\pgfsetstrokecolor{currentstroke}%
\pgfsetdash{}{0pt}%
\pgfpathmoveto{\pgfqpoint{3.291888in}{1.468783in}}%
\pgfpathlineto{\pgfqpoint{3.305810in}{1.464018in}}%
\pgfpathlineto{\pgfqpoint{3.319737in}{1.459332in}}%
\pgfpathlineto{\pgfqpoint{3.333669in}{1.454725in}}%
\pgfpathlineto{\pgfqpoint{3.347607in}{1.450197in}}%
\pgfpathlineto{\pgfqpoint{3.356040in}{1.457687in}}%
\pgfpathlineto{\pgfqpoint{3.364464in}{1.465299in}}%
\pgfpathlineto{\pgfqpoint{3.372880in}{1.473030in}}%
\pgfpathlineto{\pgfqpoint{3.381288in}{1.480873in}}%
\pgfpathlineto{\pgfqpoint{3.367370in}{1.485072in}}%
\pgfpathlineto{\pgfqpoint{3.353456in}{1.489350in}}%
\pgfpathlineto{\pgfqpoint{3.339548in}{1.493707in}}%
\pgfpathlineto{\pgfqpoint{3.325645in}{1.498144in}}%
\pgfpathlineto{\pgfqpoint{3.317219in}{1.490621in}}%
\pgfpathlineto{\pgfqpoint{3.308784in}{1.483216in}}%
\pgfpathlineto{\pgfqpoint{3.300340in}{1.475935in}}%
\pgfpathlineto{\pgfqpoint{3.291888in}{1.468783in}}%
\pgfpathclose%
\pgfusepath{fill}%
\end{pgfscope}%
\begin{pgfscope}%
\pgfpathrectangle{\pgfqpoint{1.150000in}{0.150000in}}{\pgfqpoint{5.700000in}{5.700000in}}%
\pgfusepath{clip}%
\pgfsetbuttcap%
\pgfsetroundjoin%
\definecolor{currentfill}{rgb}{0.203063,0.379716,0.553925}%
\pgfsetfillcolor{currentfill}%
\pgfsetfillopacity{0.700000}%
\pgfsetlinewidth{0.000000pt}%
\definecolor{currentstroke}{rgb}{0.000000,0.000000,0.000000}%
\pgfsetstrokecolor{currentstroke}%
\pgfsetdash{}{0pt}%
\pgfpathmoveto{\pgfqpoint{5.182737in}{2.275078in}}%
\pgfpathlineto{\pgfqpoint{5.197266in}{2.279416in}}%
\pgfpathlineto{\pgfqpoint{5.211807in}{2.283825in}}%
\pgfpathlineto{\pgfqpoint{5.226361in}{2.288306in}}%
\pgfpathlineto{\pgfqpoint{5.240928in}{2.292857in}}%
\pgfpathlineto{\pgfqpoint{5.248602in}{2.299823in}}%
\pgfpathlineto{\pgfqpoint{5.256267in}{2.306670in}}%
\pgfpathlineto{\pgfqpoint{5.263923in}{2.313400in}}%
\pgfpathlineto{\pgfqpoint{5.271570in}{2.320015in}}%
\pgfpathlineto{\pgfqpoint{5.257017in}{2.315556in}}%
\pgfpathlineto{\pgfqpoint{5.242476in}{2.311167in}}%
\pgfpathlineto{\pgfqpoint{5.227949in}{2.306850in}}%
\pgfpathlineto{\pgfqpoint{5.213434in}{2.302603in}}%
\pgfpathlineto{\pgfqpoint{5.205773in}{2.295889in}}%
\pgfpathlineto{\pgfqpoint{5.198103in}{2.289065in}}%
\pgfpathlineto{\pgfqpoint{5.190424in}{2.282128in}}%
\pgfpathlineto{\pgfqpoint{5.182737in}{2.275078in}}%
\pgfpathclose%
\pgfusepath{fill}%
\end{pgfscope}%
\begin{pgfscope}%
\pgfpathrectangle{\pgfqpoint{1.150000in}{0.150000in}}{\pgfqpoint{5.700000in}{5.700000in}}%
\pgfusepath{clip}%
\pgfsetbuttcap%
\pgfsetroundjoin%
\definecolor{currentfill}{rgb}{0.237441,0.305202,0.541921}%
\pgfsetfillcolor{currentfill}%
\pgfsetfillopacity{0.700000}%
\pgfsetlinewidth{0.000000pt}%
\definecolor{currentstroke}{rgb}{0.000000,0.000000,0.000000}%
\pgfsetstrokecolor{currentstroke}%
\pgfsetdash{}{0pt}%
\pgfpathmoveto{\pgfqpoint{4.827144in}{2.083429in}}%
\pgfpathlineto{\pgfqpoint{4.841519in}{2.086778in}}%
\pgfpathlineto{\pgfqpoint{4.855906in}{2.090198in}}%
\pgfpathlineto{\pgfqpoint{4.870305in}{2.093690in}}%
\pgfpathlineto{\pgfqpoint{4.884716in}{2.097252in}}%
\pgfpathlineto{\pgfqpoint{4.892564in}{2.106303in}}%
\pgfpathlineto{\pgfqpoint{4.900405in}{2.115237in}}%
\pgfpathlineto{\pgfqpoint{4.908238in}{2.124056in}}%
\pgfpathlineto{\pgfqpoint{4.916064in}{2.132759in}}%
\pgfpathlineto{\pgfqpoint{4.901662in}{2.129201in}}%
\pgfpathlineto{\pgfqpoint{4.887273in}{2.125714in}}%
\pgfpathlineto{\pgfqpoint{4.872895in}{2.122298in}}%
\pgfpathlineto{\pgfqpoint{4.858530in}{2.118953in}}%
\pgfpathlineto{\pgfqpoint{4.850694in}{2.110238in}}%
\pgfpathlineto{\pgfqpoint{4.842851in}{2.101412in}}%
\pgfpathlineto{\pgfqpoint{4.835001in}{2.092476in}}%
\pgfpathlineto{\pgfqpoint{4.827144in}{2.083429in}}%
\pgfpathclose%
\pgfusepath{fill}%
\end{pgfscope}%
\begin{pgfscope}%
\pgfpathrectangle{\pgfqpoint{1.150000in}{0.150000in}}{\pgfqpoint{5.700000in}{5.700000in}}%
\pgfusepath{clip}%
\pgfsetbuttcap%
\pgfsetroundjoin%
\definecolor{currentfill}{rgb}{0.271305,0.019942,0.347269}%
\pgfsetfillcolor{currentfill}%
\pgfsetfillopacity{0.700000}%
\pgfsetlinewidth{0.000000pt}%
\definecolor{currentstroke}{rgb}{0.000000,0.000000,0.000000}%
\pgfsetstrokecolor{currentstroke}%
\pgfsetdash{}{0pt}%
\pgfpathmoveto{\pgfqpoint{3.001046in}{1.515614in}}%
\pgfpathlineto{\pgfqpoint{3.014948in}{1.508816in}}%
\pgfpathlineto{\pgfqpoint{3.028854in}{1.502102in}}%
\pgfpathlineto{\pgfqpoint{3.042763in}{1.495473in}}%
\pgfpathlineto{\pgfqpoint{3.056676in}{1.488928in}}%
\pgfpathlineto{\pgfqpoint{3.065278in}{1.493729in}}%
\pgfpathlineto{\pgfqpoint{3.073869in}{1.498717in}}%
\pgfpathlineto{\pgfqpoint{3.082449in}{1.503886in}}%
\pgfpathlineto{\pgfqpoint{3.091017in}{1.509231in}}%
\pgfpathlineto{\pgfqpoint{3.077130in}{1.515405in}}%
\pgfpathlineto{\pgfqpoint{3.063246in}{1.521663in}}%
\pgfpathlineto{\pgfqpoint{3.049366in}{1.528005in}}%
\pgfpathlineto{\pgfqpoint{3.035489in}{1.534431in}}%
\pgfpathlineto{\pgfqpoint{3.026896in}{1.529450in}}%
\pgfpathlineto{\pgfqpoint{3.018291in}{1.524649in}}%
\pgfpathlineto{\pgfqpoint{3.009674in}{1.520035in}}%
\pgfpathlineto{\pgfqpoint{3.001046in}{1.515614in}}%
\pgfpathclose%
\pgfusepath{fill}%
\end{pgfscope}%
\begin{pgfscope}%
\pgfpathrectangle{\pgfqpoint{1.150000in}{0.150000in}}{\pgfqpoint{5.700000in}{5.700000in}}%
\pgfusepath{clip}%
\pgfsetbuttcap%
\pgfsetroundjoin%
\definecolor{currentfill}{rgb}{0.210503,0.363727,0.552206}%
\pgfsetfillcolor{currentfill}%
\pgfsetfillopacity{0.700000}%
\pgfsetlinewidth{0.000000pt}%
\definecolor{currentstroke}{rgb}{0.000000,0.000000,0.000000}%
\pgfsetstrokecolor{currentstroke}%
\pgfsetdash{}{0pt}%
\pgfpathmoveto{\pgfqpoint{5.093869in}{2.228778in}}%
\pgfpathlineto{\pgfqpoint{5.108359in}{2.232902in}}%
\pgfpathlineto{\pgfqpoint{5.122861in}{2.237098in}}%
\pgfpathlineto{\pgfqpoint{5.137377in}{2.241364in}}%
\pgfpathlineto{\pgfqpoint{5.151905in}{2.245702in}}%
\pgfpathlineto{\pgfqpoint{5.159626in}{2.253225in}}%
\pgfpathlineto{\pgfqpoint{5.167338in}{2.260628in}}%
\pgfpathlineto{\pgfqpoint{5.175042in}{2.267911in}}%
\pgfpathlineto{\pgfqpoint{5.182737in}{2.275078in}}%
\pgfpathlineto{\pgfqpoint{5.168222in}{2.270810in}}%
\pgfpathlineto{\pgfqpoint{5.153719in}{2.266614in}}%
\pgfpathlineto{\pgfqpoint{5.139229in}{2.262488in}}%
\pgfpathlineto{\pgfqpoint{5.124751in}{2.258433in}}%
\pgfpathlineto{\pgfqpoint{5.117043in}{2.251189in}}%
\pgfpathlineto{\pgfqpoint{5.109326in}{2.243833in}}%
\pgfpathlineto{\pgfqpoint{5.101602in}{2.236363in}}%
\pgfpathlineto{\pgfqpoint{5.093869in}{2.228778in}}%
\pgfpathclose%
\pgfusepath{fill}%
\end{pgfscope}%
\begin{pgfscope}%
\pgfpathrectangle{\pgfqpoint{1.150000in}{0.150000in}}{\pgfqpoint{5.700000in}{5.700000in}}%
\pgfusepath{clip}%
\pgfsetbuttcap%
\pgfsetroundjoin%
\definecolor{currentfill}{rgb}{0.227802,0.326594,0.546532}%
\pgfsetfillcolor{currentfill}%
\pgfsetfillopacity{0.700000}%
\pgfsetlinewidth{0.000000pt}%
\definecolor{currentstroke}{rgb}{0.000000,0.000000,0.000000}%
\pgfsetstrokecolor{currentstroke}%
\pgfsetdash{}{0pt}%
\pgfpathmoveto{\pgfqpoint{4.916064in}{2.132759in}}%
\pgfpathlineto{\pgfqpoint{4.930477in}{2.136389in}}%
\pgfpathlineto{\pgfqpoint{4.944902in}{2.140089in}}%
\pgfpathlineto{\pgfqpoint{4.959340in}{2.143861in}}%
\pgfpathlineto{\pgfqpoint{4.973789in}{2.147704in}}%
\pgfpathlineto{\pgfqpoint{4.981597in}{2.156275in}}%
\pgfpathlineto{\pgfqpoint{4.989398in}{2.164727in}}%
\pgfpathlineto{\pgfqpoint{4.997190in}{2.173061in}}%
\pgfpathlineto{\pgfqpoint{5.004975in}{2.181277in}}%
\pgfpathlineto{\pgfqpoint{4.990535in}{2.177460in}}%
\pgfpathlineto{\pgfqpoint{4.976108in}{2.173714in}}%
\pgfpathlineto{\pgfqpoint{4.961693in}{2.170039in}}%
\pgfpathlineto{\pgfqpoint{4.947290in}{2.166436in}}%
\pgfpathlineto{\pgfqpoint{4.939495in}{2.158185in}}%
\pgfpathlineto{\pgfqpoint{4.931692in}{2.149823in}}%
\pgfpathlineto{\pgfqpoint{4.923882in}{2.141348in}}%
\pgfpathlineto{\pgfqpoint{4.916064in}{2.132759in}}%
\pgfpathclose%
\pgfusepath{fill}%
\end{pgfscope}%
\begin{pgfscope}%
\pgfpathrectangle{\pgfqpoint{1.150000in}{0.150000in}}{\pgfqpoint{5.700000in}{5.700000in}}%
\pgfusepath{clip}%
\pgfsetbuttcap%
\pgfsetroundjoin%
\definecolor{currentfill}{rgb}{0.218130,0.347432,0.550038}%
\pgfsetfillcolor{currentfill}%
\pgfsetfillopacity{0.700000}%
\pgfsetlinewidth{0.000000pt}%
\definecolor{currentstroke}{rgb}{0.000000,0.000000,0.000000}%
\pgfsetstrokecolor{currentstroke}%
\pgfsetdash{}{0pt}%
\pgfpathmoveto{\pgfqpoint{5.004975in}{2.181277in}}%
\pgfpathlineto{\pgfqpoint{5.019426in}{2.185165in}}%
\pgfpathlineto{\pgfqpoint{5.033890in}{2.189124in}}%
\pgfpathlineto{\pgfqpoint{5.048366in}{2.193155in}}%
\pgfpathlineto{\pgfqpoint{5.062855in}{2.197256in}}%
\pgfpathlineto{\pgfqpoint{5.070621in}{2.205316in}}%
\pgfpathlineto{\pgfqpoint{5.078378in}{2.213256in}}%
\pgfpathlineto{\pgfqpoint{5.086128in}{2.221076in}}%
\pgfpathlineto{\pgfqpoint{5.093869in}{2.228778in}}%
\pgfpathlineto{\pgfqpoint{5.079391in}{2.224724in}}%
\pgfpathlineto{\pgfqpoint{5.064926in}{2.220742in}}%
\pgfpathlineto{\pgfqpoint{5.050474in}{2.216831in}}%
\pgfpathlineto{\pgfqpoint{5.036033in}{2.212990in}}%
\pgfpathlineto{\pgfqpoint{5.028281in}{2.205232in}}%
\pgfpathlineto{\pgfqpoint{5.020520in}{2.197362in}}%
\pgfpathlineto{\pgfqpoint{5.012751in}{2.189377in}}%
\pgfpathlineto{\pgfqpoint{5.004975in}{2.181277in}}%
\pgfpathclose%
\pgfusepath{fill}%
\end{pgfscope}%
\begin{pgfscope}%
\pgfpathrectangle{\pgfqpoint{1.150000in}{0.150000in}}{\pgfqpoint{5.700000in}{5.700000in}}%
\pgfusepath{clip}%
\pgfsetbuttcap%
\pgfsetroundjoin%
\definecolor{currentfill}{rgb}{0.278791,0.062145,0.386592}%
\pgfsetfillcolor{currentfill}%
\pgfsetfillopacity{0.700000}%
\pgfsetlinewidth{0.000000pt}%
\definecolor{currentstroke}{rgb}{0.000000,0.000000,0.000000}%
\pgfsetstrokecolor{currentstroke}%
\pgfsetdash{}{0pt}%
\pgfpathmoveto{\pgfqpoint{3.849292in}{1.567056in}}%
\pgfpathlineto{\pgfqpoint{3.863321in}{1.565869in}}%
\pgfpathlineto{\pgfqpoint{3.877358in}{1.564756in}}%
\pgfpathlineto{\pgfqpoint{3.891402in}{1.563717in}}%
\pgfpathlineto{\pgfqpoint{3.905455in}{1.562751in}}%
\pgfpathlineto{\pgfqpoint{3.913655in}{1.573497in}}%
\pgfpathlineto{\pgfqpoint{3.921850in}{1.584247in}}%
\pgfpathlineto{\pgfqpoint{3.930038in}{1.594998in}}%
\pgfpathlineto{\pgfqpoint{3.938222in}{1.605747in}}%
\pgfpathlineto{\pgfqpoint{3.924179in}{1.606486in}}%
\pgfpathlineto{\pgfqpoint{3.910144in}{1.607298in}}%
\pgfpathlineto{\pgfqpoint{3.896118in}{1.608184in}}%
\pgfpathlineto{\pgfqpoint{3.882099in}{1.609145in}}%
\pgfpathlineto{\pgfqpoint{3.873906in}{1.598615in}}%
\pgfpathlineto{\pgfqpoint{3.865707in}{1.588088in}}%
\pgfpathlineto{\pgfqpoint{3.857502in}{1.577567in}}%
\pgfpathlineto{\pgfqpoint{3.849292in}{1.567056in}}%
\pgfpathclose%
\pgfusepath{fill}%
\end{pgfscope}%
\begin{pgfscope}%
\pgfpathrectangle{\pgfqpoint{1.150000in}{0.150000in}}{\pgfqpoint{5.700000in}{5.700000in}}%
\pgfusepath{clip}%
\pgfsetbuttcap%
\pgfsetroundjoin%
\definecolor{currentfill}{rgb}{0.283091,0.110553,0.431554}%
\pgfsetfillcolor{currentfill}%
\pgfsetfillopacity{0.700000}%
\pgfsetlinewidth{0.000000pt}%
\definecolor{currentstroke}{rgb}{0.000000,0.000000,0.000000}%
\pgfsetstrokecolor{currentstroke}%
\pgfsetdash{}{0pt}%
\pgfpathmoveto{\pgfqpoint{2.597146in}{1.697002in}}%
\pgfpathlineto{\pgfqpoint{2.611064in}{1.687226in}}%
\pgfpathlineto{\pgfqpoint{2.624982in}{1.677546in}}%
\pgfpathlineto{\pgfqpoint{2.638902in}{1.667961in}}%
\pgfpathlineto{\pgfqpoint{2.652822in}{1.658472in}}%
\pgfpathlineto{\pgfqpoint{2.661721in}{1.659014in}}%
\pgfpathlineto{\pgfqpoint{2.670604in}{1.659826in}}%
\pgfpathlineto{\pgfqpoint{2.679471in}{1.660900in}}%
\pgfpathlineto{\pgfqpoint{2.688321in}{1.662230in}}%
\pgfpathlineto{\pgfqpoint{2.674436in}{1.671301in}}%
\pgfpathlineto{\pgfqpoint{2.660552in}{1.680467in}}%
\pgfpathlineto{\pgfqpoint{2.646669in}{1.689729in}}%
\pgfpathlineto{\pgfqpoint{2.632788in}{1.699086in}}%
\pgfpathlineto{\pgfqpoint{2.623903in}{1.698166in}}%
\pgfpathlineto{\pgfqpoint{2.615001in}{1.697507in}}%
\pgfpathlineto{\pgfqpoint{2.606082in}{1.697117in}}%
\pgfpathlineto{\pgfqpoint{2.597146in}{1.697002in}}%
\pgfpathclose%
\pgfusepath{fill}%
\end{pgfscope}%
\begin{pgfscope}%
\pgfpathrectangle{\pgfqpoint{1.150000in}{0.150000in}}{\pgfqpoint{5.700000in}{5.700000in}}%
\pgfusepath{clip}%
\pgfsetbuttcap%
\pgfsetroundjoin%
\definecolor{currentfill}{rgb}{0.276022,0.044167,0.370164}%
\pgfsetfillcolor{currentfill}%
\pgfsetfillopacity{0.700000}%
\pgfsetlinewidth{0.000000pt}%
\definecolor{currentstroke}{rgb}{0.000000,0.000000,0.000000}%
\pgfsetstrokecolor{currentstroke}%
\pgfsetdash{}{0pt}%
\pgfpathmoveto{\pgfqpoint{3.760312in}{1.531654in}}%
\pgfpathlineto{\pgfqpoint{3.774321in}{1.529922in}}%
\pgfpathlineto{\pgfqpoint{3.788338in}{1.528266in}}%
\pgfpathlineto{\pgfqpoint{3.802362in}{1.526683in}}%
\pgfpathlineto{\pgfqpoint{3.816394in}{1.525175in}}%
\pgfpathlineto{\pgfqpoint{3.824627in}{1.535614in}}%
\pgfpathlineto{\pgfqpoint{3.832855in}{1.546076in}}%
\pgfpathlineto{\pgfqpoint{3.841076in}{1.556558in}}%
\pgfpathlineto{\pgfqpoint{3.849292in}{1.567056in}}%
\pgfpathlineto{\pgfqpoint{3.835271in}{1.568317in}}%
\pgfpathlineto{\pgfqpoint{3.821258in}{1.569652in}}%
\pgfpathlineto{\pgfqpoint{3.807252in}{1.571062in}}%
\pgfpathlineto{\pgfqpoint{3.793254in}{1.572546in}}%
\pgfpathlineto{\pgfqpoint{3.785028in}{1.562287in}}%
\pgfpathlineto{\pgfqpoint{3.776795in}{1.552050in}}%
\pgfpathlineto{\pgfqpoint{3.768556in}{1.541837in}}%
\pgfpathlineto{\pgfqpoint{3.760312in}{1.531654in}}%
\pgfpathclose%
\pgfusepath{fill}%
\end{pgfscope}%
\begin{pgfscope}%
\pgfpathrectangle{\pgfqpoint{1.150000in}{0.150000in}}{\pgfqpoint{5.700000in}{5.700000in}}%
\pgfusepath{clip}%
\pgfsetbuttcap%
\pgfsetroundjoin%
\definecolor{currentfill}{rgb}{0.267004,0.004874,0.329415}%
\pgfsetfillcolor{currentfill}%
\pgfsetfillopacity{0.700000}%
\pgfsetlinewidth{0.000000pt}%
\definecolor{currentstroke}{rgb}{0.000000,0.000000,0.000000}%
\pgfsetstrokecolor{currentstroke}%
\pgfsetdash{}{0pt}%
\pgfpathmoveto{\pgfqpoint{3.437018in}{1.464861in}}%
\pgfpathlineto{\pgfqpoint{3.450964in}{1.461052in}}%
\pgfpathlineto{\pgfqpoint{3.464916in}{1.457320in}}%
\pgfpathlineto{\pgfqpoint{3.478874in}{1.453666in}}%
\pgfpathlineto{\pgfqpoint{3.492838in}{1.450089in}}%
\pgfpathlineto{\pgfqpoint{3.501204in}{1.458673in}}%
\pgfpathlineto{\pgfqpoint{3.509562in}{1.467350in}}%
\pgfpathlineto{\pgfqpoint{3.517913in}{1.476117in}}%
\pgfpathlineto{\pgfqpoint{3.526256in}{1.484967in}}%
\pgfpathlineto{\pgfqpoint{3.512308in}{1.488236in}}%
\pgfpathlineto{\pgfqpoint{3.498366in}{1.491582in}}%
\pgfpathlineto{\pgfqpoint{3.484431in}{1.495005in}}%
\pgfpathlineto{\pgfqpoint{3.470501in}{1.498506in}}%
\pgfpathlineto{\pgfqpoint{3.462141in}{1.489956in}}%
\pgfpathlineto{\pgfqpoint{3.453774in}{1.481495in}}%
\pgfpathlineto{\pgfqpoint{3.445400in}{1.473129in}}%
\pgfpathlineto{\pgfqpoint{3.437018in}{1.464861in}}%
\pgfpathclose%
\pgfusepath{fill}%
\end{pgfscope}%
\begin{pgfscope}%
\pgfpathrectangle{\pgfqpoint{1.150000in}{0.150000in}}{\pgfqpoint{5.700000in}{5.700000in}}%
\pgfusepath{clip}%
\pgfsetbuttcap%
\pgfsetroundjoin%
\definecolor{currentfill}{rgb}{0.277018,0.050344,0.375715}%
\pgfsetfillcolor{currentfill}%
\pgfsetfillopacity{0.700000}%
\pgfsetlinewidth{0.000000pt}%
\definecolor{currentstroke}{rgb}{0.000000,0.000000,0.000000}%
\pgfsetstrokecolor{currentstroke}%
\pgfsetdash{}{0pt}%
\pgfpathmoveto{\pgfqpoint{2.855066in}{1.560596in}}%
\pgfpathlineto{\pgfqpoint{2.868975in}{1.552712in}}%
\pgfpathlineto{\pgfqpoint{2.882886in}{1.544917in}}%
\pgfpathlineto{\pgfqpoint{2.896799in}{1.537209in}}%
\pgfpathlineto{\pgfqpoint{2.910715in}{1.529589in}}%
\pgfpathlineto{\pgfqpoint{2.919423in}{1.532780in}}%
\pgfpathlineto{\pgfqpoint{2.928117in}{1.536193in}}%
\pgfpathlineto{\pgfqpoint{2.936799in}{1.539822in}}%
\pgfpathlineto{\pgfqpoint{2.945467in}{1.543660in}}%
\pgfpathlineto{\pgfqpoint{2.931580in}{1.550887in}}%
\pgfpathlineto{\pgfqpoint{2.917696in}{1.558201in}}%
\pgfpathlineto{\pgfqpoint{2.903814in}{1.565603in}}%
\pgfpathlineto{\pgfqpoint{2.889935in}{1.573092in}}%
\pgfpathlineto{\pgfqpoint{2.881238in}{1.569639in}}%
\pgfpathlineto{\pgfqpoint{2.872528in}{1.566402in}}%
\pgfpathlineto{\pgfqpoint{2.863804in}{1.563385in}}%
\pgfpathlineto{\pgfqpoint{2.855066in}{1.560596in}}%
\pgfpathclose%
\pgfusepath{fill}%
\end{pgfscope}%
\begin{pgfscope}%
\pgfpathrectangle{\pgfqpoint{1.150000in}{0.150000in}}{\pgfqpoint{5.700000in}{5.700000in}}%
\pgfusepath{clip}%
\pgfsetbuttcap%
\pgfsetroundjoin%
\definecolor{currentfill}{rgb}{0.272594,0.025563,0.353093}%
\pgfsetfillcolor{currentfill}%
\pgfsetfillopacity{0.700000}%
\pgfsetlinewidth{0.000000pt}%
\definecolor{currentstroke}{rgb}{0.000000,0.000000,0.000000}%
\pgfsetstrokecolor{currentstroke}%
\pgfsetdash{}{0pt}%
\pgfpathmoveto{\pgfqpoint{3.671260in}{1.500019in}}%
\pgfpathlineto{\pgfqpoint{3.685252in}{1.497721in}}%
\pgfpathlineto{\pgfqpoint{3.699252in}{1.495498in}}%
\pgfpathlineto{\pgfqpoint{3.713259in}{1.493350in}}%
\pgfpathlineto{\pgfqpoint{3.727273in}{1.491277in}}%
\pgfpathlineto{\pgfqpoint{3.735542in}{1.501310in}}%
\pgfpathlineto{\pgfqpoint{3.743805in}{1.511386in}}%
\pgfpathlineto{\pgfqpoint{3.752061in}{1.521502in}}%
\pgfpathlineto{\pgfqpoint{3.760312in}{1.531654in}}%
\pgfpathlineto{\pgfqpoint{3.746310in}{1.533459in}}%
\pgfpathlineto{\pgfqpoint{3.732315in}{1.535340in}}%
\pgfpathlineto{\pgfqpoint{3.718328in}{1.537295in}}%
\pgfpathlineto{\pgfqpoint{3.704347in}{1.539325in}}%
\pgfpathlineto{\pgfqpoint{3.696085in}{1.529434in}}%
\pgfpathlineto{\pgfqpoint{3.687816in}{1.519583in}}%
\pgfpathlineto{\pgfqpoint{3.679541in}{1.509777in}}%
\pgfpathlineto{\pgfqpoint{3.671260in}{1.500019in}}%
\pgfpathclose%
\pgfusepath{fill}%
\end{pgfscope}%
\begin{pgfscope}%
\pgfpathrectangle{\pgfqpoint{1.150000in}{0.150000in}}{\pgfqpoint{5.700000in}{5.700000in}}%
\pgfusepath{clip}%
\pgfsetbuttcap%
\pgfsetroundjoin%
\definecolor{currentfill}{rgb}{0.282656,0.100196,0.422160}%
\pgfsetfillcolor{currentfill}%
\pgfsetfillopacity{0.700000}%
\pgfsetlinewidth{0.000000pt}%
\definecolor{currentstroke}{rgb}{0.000000,0.000000,0.000000}%
\pgfsetstrokecolor{currentstroke}%
\pgfsetdash{}{0pt}%
\pgfpathmoveto{\pgfqpoint{2.652822in}{1.658472in}}%
\pgfpathlineto{\pgfqpoint{2.666743in}{1.649078in}}%
\pgfpathlineto{\pgfqpoint{2.680666in}{1.639778in}}%
\pgfpathlineto{\pgfqpoint{2.694589in}{1.630571in}}%
\pgfpathlineto{\pgfqpoint{2.708514in}{1.621457in}}%
\pgfpathlineto{\pgfqpoint{2.717378in}{1.622425in}}%
\pgfpathlineto{\pgfqpoint{2.726226in}{1.623657in}}%
\pgfpathlineto{\pgfqpoint{2.735058in}{1.625146in}}%
\pgfpathlineto{\pgfqpoint{2.743874in}{1.626885in}}%
\pgfpathlineto{\pgfqpoint{2.729984in}{1.635582in}}%
\pgfpathlineto{\pgfqpoint{2.716095in}{1.644371in}}%
\pgfpathlineto{\pgfqpoint{2.702207in}{1.653254in}}%
\pgfpathlineto{\pgfqpoint{2.688321in}{1.662230in}}%
\pgfpathlineto{\pgfqpoint{2.679471in}{1.660900in}}%
\pgfpathlineto{\pgfqpoint{2.670604in}{1.659826in}}%
\pgfpathlineto{\pgfqpoint{2.661721in}{1.659014in}}%
\pgfpathlineto{\pgfqpoint{2.652822in}{1.658472in}}%
\pgfpathclose%
\pgfusepath{fill}%
\end{pgfscope}%
\begin{pgfscope}%
\pgfpathrectangle{\pgfqpoint{1.150000in}{0.150000in}}{\pgfqpoint{5.700000in}{5.700000in}}%
\pgfusepath{clip}%
\pgfsetbuttcap%
\pgfsetroundjoin%
\definecolor{currentfill}{rgb}{0.267004,0.004874,0.329415}%
\pgfsetfillcolor{currentfill}%
\pgfsetfillopacity{0.700000}%
\pgfsetlinewidth{0.000000pt}%
\definecolor{currentstroke}{rgb}{0.000000,0.000000,0.000000}%
\pgfsetstrokecolor{currentstroke}%
\pgfsetdash{}{0pt}%
\pgfpathmoveto{\pgfqpoint{3.202261in}{1.462815in}}%
\pgfpathlineto{\pgfqpoint{3.216185in}{1.457380in}}%
\pgfpathlineto{\pgfqpoint{3.230114in}{1.452025in}}%
\pgfpathlineto{\pgfqpoint{3.244048in}{1.446751in}}%
\pgfpathlineto{\pgfqpoint{3.257986in}{1.441556in}}%
\pgfpathlineto{\pgfqpoint{3.266475in}{1.448145in}}%
\pgfpathlineto{\pgfqpoint{3.274955in}{1.454882in}}%
\pgfpathlineto{\pgfqpoint{3.283426in}{1.461763in}}%
\pgfpathlineto{\pgfqpoint{3.291888in}{1.468783in}}%
\pgfpathlineto{\pgfqpoint{3.277971in}{1.473627in}}%
\pgfpathlineto{\pgfqpoint{3.264058in}{1.478552in}}%
\pgfpathlineto{\pgfqpoint{3.250151in}{1.483557in}}%
\pgfpathlineto{\pgfqpoint{3.236248in}{1.488642in}}%
\pgfpathlineto{\pgfqpoint{3.227765in}{1.481964in}}%
\pgfpathlineto{\pgfqpoint{3.219274in}{1.475430in}}%
\pgfpathlineto{\pgfqpoint{3.210772in}{1.469045in}}%
\pgfpathlineto{\pgfqpoint{3.202261in}{1.462815in}}%
\pgfpathclose%
\pgfusepath{fill}%
\end{pgfscope}%
\begin{pgfscope}%
\pgfpathrectangle{\pgfqpoint{1.150000in}{0.150000in}}{\pgfqpoint{5.700000in}{5.700000in}}%
\pgfusepath{clip}%
\pgfsetbuttcap%
\pgfsetroundjoin%
\definecolor{currentfill}{rgb}{0.269944,0.014625,0.341379}%
\pgfsetfillcolor{currentfill}%
\pgfsetfillopacity{0.700000}%
\pgfsetlinewidth{0.000000pt}%
\definecolor{currentstroke}{rgb}{0.000000,0.000000,0.000000}%
\pgfsetstrokecolor{currentstroke}%
\pgfsetdash{}{0pt}%
\pgfpathmoveto{\pgfqpoint{3.056676in}{1.488928in}}%
\pgfpathlineto{\pgfqpoint{3.070592in}{1.482466in}}%
\pgfpathlineto{\pgfqpoint{3.084512in}{1.476088in}}%
\pgfpathlineto{\pgfqpoint{3.098436in}{1.469792in}}%
\pgfpathlineto{\pgfqpoint{3.112363in}{1.463579in}}%
\pgfpathlineto{\pgfqpoint{3.120940in}{1.468758in}}%
\pgfpathlineto{\pgfqpoint{3.129506in}{1.474120in}}%
\pgfpathlineto{\pgfqpoint{3.138062in}{1.479658in}}%
\pgfpathlineto{\pgfqpoint{3.146606in}{1.485366in}}%
\pgfpathlineto{\pgfqpoint{3.132703in}{1.491208in}}%
\pgfpathlineto{\pgfqpoint{3.118804in}{1.497133in}}%
\pgfpathlineto{\pgfqpoint{3.104909in}{1.503140in}}%
\pgfpathlineto{\pgfqpoint{3.091017in}{1.509231in}}%
\pgfpathlineto{\pgfqpoint{3.082449in}{1.503886in}}%
\pgfpathlineto{\pgfqpoint{3.073869in}{1.498717in}}%
\pgfpathlineto{\pgfqpoint{3.065278in}{1.493729in}}%
\pgfpathlineto{\pgfqpoint{3.056676in}{1.488928in}}%
\pgfpathclose%
\pgfusepath{fill}%
\end{pgfscope}%
\begin{pgfscope}%
\pgfpathrectangle{\pgfqpoint{1.150000in}{0.150000in}}{\pgfqpoint{5.700000in}{5.700000in}}%
\pgfusepath{clip}%
\pgfsetbuttcap%
\pgfsetroundjoin%
\definecolor{currentfill}{rgb}{0.269944,0.014625,0.341379}%
\pgfsetfillcolor{currentfill}%
\pgfsetfillopacity{0.700000}%
\pgfsetlinewidth{0.000000pt}%
\definecolor{currentstroke}{rgb}{0.000000,0.000000,0.000000}%
\pgfsetstrokecolor{currentstroke}%
\pgfsetdash{}{0pt}%
\pgfpathmoveto{\pgfqpoint{3.582111in}{1.472656in}}%
\pgfpathlineto{\pgfqpoint{3.596091in}{1.469768in}}%
\pgfpathlineto{\pgfqpoint{3.610077in}{1.466956in}}%
\pgfpathlineto{\pgfqpoint{3.624070in}{1.464220in}}%
\pgfpathlineto{\pgfqpoint{3.638070in}{1.461558in}}%
\pgfpathlineto{\pgfqpoint{3.646377in}{1.471080in}}%
\pgfpathlineto{\pgfqpoint{3.654678in}{1.480667in}}%
\pgfpathlineto{\pgfqpoint{3.662972in}{1.490315in}}%
\pgfpathlineto{\pgfqpoint{3.671260in}{1.500019in}}%
\pgfpathlineto{\pgfqpoint{3.657274in}{1.502392in}}%
\pgfpathlineto{\pgfqpoint{3.643295in}{1.504841in}}%
\pgfpathlineto{\pgfqpoint{3.629322in}{1.507365in}}%
\pgfpathlineto{\pgfqpoint{3.615357in}{1.509965in}}%
\pgfpathlineto{\pgfqpoint{3.607055in}{1.500540in}}%
\pgfpathlineto{\pgfqpoint{3.598747in}{1.491178in}}%
\pgfpathlineto{\pgfqpoint{3.590433in}{1.481882in}}%
\pgfpathlineto{\pgfqpoint{3.582111in}{1.472656in}}%
\pgfpathclose%
\pgfusepath{fill}%
\end{pgfscope}%
\begin{pgfscope}%
\pgfpathrectangle{\pgfqpoint{1.150000in}{0.150000in}}{\pgfqpoint{5.700000in}{5.700000in}}%
\pgfusepath{clip}%
\pgfsetbuttcap%
\pgfsetroundjoin%
\definecolor{currentfill}{rgb}{0.267004,0.004874,0.329415}%
\pgfsetfillcolor{currentfill}%
\pgfsetfillopacity{0.700000}%
\pgfsetlinewidth{0.000000pt}%
\definecolor{currentstroke}{rgb}{0.000000,0.000000,0.000000}%
\pgfsetstrokecolor{currentstroke}%
\pgfsetdash{}{0pt}%
\pgfpathmoveto{\pgfqpoint{3.347607in}{1.450197in}}%
\pgfpathlineto{\pgfqpoint{3.361549in}{1.445748in}}%
\pgfpathlineto{\pgfqpoint{3.375497in}{1.441377in}}%
\pgfpathlineto{\pgfqpoint{3.389450in}{1.437084in}}%
\pgfpathlineto{\pgfqpoint{3.403409in}{1.432869in}}%
\pgfpathlineto{\pgfqpoint{3.411823in}{1.440695in}}%
\pgfpathlineto{\pgfqpoint{3.420229in}{1.448639in}}%
\pgfpathlineto{\pgfqpoint{3.428627in}{1.456696in}}%
\pgfpathlineto{\pgfqpoint{3.437018in}{1.464861in}}%
\pgfpathlineto{\pgfqpoint{3.423077in}{1.468747in}}%
\pgfpathlineto{\pgfqpoint{3.409142in}{1.472711in}}%
\pgfpathlineto{\pgfqpoint{3.395212in}{1.476753in}}%
\pgfpathlineto{\pgfqpoint{3.381288in}{1.480873in}}%
\pgfpathlineto{\pgfqpoint{3.372880in}{1.473030in}}%
\pgfpathlineto{\pgfqpoint{3.364464in}{1.465299in}}%
\pgfpathlineto{\pgfqpoint{3.356040in}{1.457687in}}%
\pgfpathlineto{\pgfqpoint{3.347607in}{1.450197in}}%
\pgfpathclose%
\pgfusepath{fill}%
\end{pgfscope}%
\begin{pgfscope}%
\pgfpathrectangle{\pgfqpoint{1.150000in}{0.150000in}}{\pgfqpoint{5.700000in}{5.700000in}}%
\pgfusepath{clip}%
\pgfsetbuttcap%
\pgfsetroundjoin%
\definecolor{currentfill}{rgb}{0.282623,0.140926,0.457517}%
\pgfsetfillcolor{currentfill}%
\pgfsetfillopacity{0.700000}%
\pgfsetlinewidth{0.000000pt}%
\definecolor{currentstroke}{rgb}{0.000000,0.000000,0.000000}%
\pgfsetstrokecolor{currentstroke}%
\pgfsetdash{}{0pt}%
\pgfpathmoveto{\pgfqpoint{4.172461in}{1.692875in}}%
\pgfpathlineto{\pgfqpoint{4.186601in}{1.693477in}}%
\pgfpathlineto{\pgfqpoint{4.200750in}{1.694153in}}%
\pgfpathlineto{\pgfqpoint{4.214908in}{1.694900in}}%
\pgfpathlineto{\pgfqpoint{4.229076in}{1.695720in}}%
\pgfpathlineto{\pgfqpoint{4.237177in}{1.707016in}}%
\pgfpathlineto{\pgfqpoint{4.245273in}{1.718262in}}%
\pgfpathlineto{\pgfqpoint{4.253364in}{1.729456in}}%
\pgfpathlineto{\pgfqpoint{4.261449in}{1.740596in}}%
\pgfpathlineto{\pgfqpoint{4.247289in}{1.739611in}}%
\pgfpathlineto{\pgfqpoint{4.233138in}{1.738698in}}%
\pgfpathlineto{\pgfqpoint{4.218996in}{1.737857in}}%
\pgfpathlineto{\pgfqpoint{4.204864in}{1.737088in}}%
\pgfpathlineto{\pgfqpoint{4.196772in}{1.726106in}}%
\pgfpathlineto{\pgfqpoint{4.188674in}{1.715075in}}%
\pgfpathlineto{\pgfqpoint{4.180570in}{1.703997in}}%
\pgfpathlineto{\pgfqpoint{4.172461in}{1.692875in}}%
\pgfpathclose%
\pgfusepath{fill}%
\end{pgfscope}%
\begin{pgfscope}%
\pgfpathrectangle{\pgfqpoint{1.150000in}{0.150000in}}{\pgfqpoint{5.700000in}{5.700000in}}%
\pgfusepath{clip}%
\pgfsetbuttcap%
\pgfsetroundjoin%
\definecolor{currentfill}{rgb}{0.280868,0.160771,0.472899}%
\pgfsetfillcolor{currentfill}%
\pgfsetfillopacity{0.700000}%
\pgfsetlinewidth{0.000000pt}%
\definecolor{currentstroke}{rgb}{0.000000,0.000000,0.000000}%
\pgfsetstrokecolor{currentstroke}%
\pgfsetdash{}{0pt}%
\pgfpathmoveto{\pgfqpoint{4.261449in}{1.740596in}}%
\pgfpathlineto{\pgfqpoint{4.275619in}{1.741654in}}%
\pgfpathlineto{\pgfqpoint{4.289799in}{1.742784in}}%
\pgfpathlineto{\pgfqpoint{4.303989in}{1.743986in}}%
\pgfpathlineto{\pgfqpoint{4.318189in}{1.745259in}}%
\pgfpathlineto{\pgfqpoint{4.326261in}{1.756496in}}%
\pgfpathlineto{\pgfqpoint{4.334328in}{1.767670in}}%
\pgfpathlineto{\pgfqpoint{4.342389in}{1.778778in}}%
\pgfpathlineto{\pgfqpoint{4.350445in}{1.789819in}}%
\pgfpathlineto{\pgfqpoint{4.336252in}{1.788400in}}%
\pgfpathlineto{\pgfqpoint{4.322070in}{1.787053in}}%
\pgfpathlineto{\pgfqpoint{4.307897in}{1.785778in}}%
\pgfpathlineto{\pgfqpoint{4.293734in}{1.784576in}}%
\pgfpathlineto{\pgfqpoint{4.285671in}{1.773672in}}%
\pgfpathlineto{\pgfqpoint{4.277603in}{1.762706in}}%
\pgfpathlineto{\pgfqpoint{4.269529in}{1.751680in}}%
\pgfpathlineto{\pgfqpoint{4.261449in}{1.740596in}}%
\pgfpathclose%
\pgfusepath{fill}%
\end{pgfscope}%
\begin{pgfscope}%
\pgfpathrectangle{\pgfqpoint{1.150000in}{0.150000in}}{\pgfqpoint{5.700000in}{5.700000in}}%
\pgfusepath{clip}%
\pgfsetbuttcap%
\pgfsetroundjoin%
\definecolor{currentfill}{rgb}{0.283197,0.115680,0.436115}%
\pgfsetfillcolor{currentfill}%
\pgfsetfillopacity{0.700000}%
\pgfsetlinewidth{0.000000pt}%
\definecolor{currentstroke}{rgb}{0.000000,0.000000,0.000000}%
\pgfsetstrokecolor{currentstroke}%
\pgfsetdash{}{0pt}%
\pgfpathmoveto{\pgfqpoint{4.083474in}{1.647047in}}%
\pgfpathlineto{\pgfqpoint{4.097585in}{1.647173in}}%
\pgfpathlineto{\pgfqpoint{4.111705in}{1.647372in}}%
\pgfpathlineto{\pgfqpoint{4.125834in}{1.647644in}}%
\pgfpathlineto{\pgfqpoint{4.139972in}{1.647988in}}%
\pgfpathlineto{\pgfqpoint{4.148102in}{1.659264in}}%
\pgfpathlineto{\pgfqpoint{4.156227in}{1.670506in}}%
\pgfpathlineto{\pgfqpoint{4.164347in}{1.681710in}}%
\pgfpathlineto{\pgfqpoint{4.172461in}{1.692875in}}%
\pgfpathlineto{\pgfqpoint{4.158331in}{1.692344in}}%
\pgfpathlineto{\pgfqpoint{4.144210in}{1.691886in}}%
\pgfpathlineto{\pgfqpoint{4.130098in}{1.691501in}}%
\pgfpathlineto{\pgfqpoint{4.115995in}{1.691189in}}%
\pgfpathlineto{\pgfqpoint{4.107873in}{1.680202in}}%
\pgfpathlineto{\pgfqpoint{4.099745in}{1.669182in}}%
\pgfpathlineto{\pgfqpoint{4.091612in}{1.658129in}}%
\pgfpathlineto{\pgfqpoint{4.083474in}{1.647047in}}%
\pgfpathclose%
\pgfusepath{fill}%
\end{pgfscope}%
\begin{pgfscope}%
\pgfpathrectangle{\pgfqpoint{1.150000in}{0.150000in}}{\pgfqpoint{5.700000in}{5.700000in}}%
\pgfusepath{clip}%
\pgfsetbuttcap%
\pgfsetroundjoin%
\definecolor{currentfill}{rgb}{0.175841,0.441290,0.557685}%
\pgfsetfillcolor{currentfill}%
\pgfsetfillopacity{0.700000}%
\pgfsetlinewidth{0.000000pt}%
\definecolor{currentstroke}{rgb}{0.000000,0.000000,0.000000}%
\pgfsetstrokecolor{currentstroke}%
\pgfsetdash{}{0pt}%
\pgfpathmoveto{\pgfqpoint{5.507735in}{2.425067in}}%
\pgfpathlineto{\pgfqpoint{5.522431in}{2.430195in}}%
\pgfpathlineto{\pgfqpoint{5.537141in}{2.435394in}}%
\pgfpathlineto{\pgfqpoint{5.551864in}{2.440663in}}%
\pgfpathlineto{\pgfqpoint{5.559369in}{2.445780in}}%
\pgfpathlineto{\pgfqpoint{5.566865in}{2.450786in}}%
\pgfpathlineto{\pgfqpoint{5.574351in}{2.455684in}}%
\pgfpathlineto{\pgfqpoint{5.581828in}{2.460477in}}%
\pgfpathlineto{\pgfqpoint{5.567123in}{2.455367in}}%
\pgfpathlineto{\pgfqpoint{5.552431in}{2.450328in}}%
\pgfpathlineto{\pgfqpoint{5.537753in}{2.445359in}}%
\pgfpathlineto{\pgfqpoint{5.530263in}{2.440440in}}%
\pgfpathlineto{\pgfqpoint{5.522763in}{2.435420in}}%
\pgfpathlineto{\pgfqpoint{5.515254in}{2.430297in}}%
\pgfpathlineto{\pgfqpoint{5.507735in}{2.425067in}}%
\pgfpathclose%
\pgfusepath{fill}%
\end{pgfscope}%
\begin{pgfscope}%
\pgfpathrectangle{\pgfqpoint{1.150000in}{0.150000in}}{\pgfqpoint{5.700000in}{5.700000in}}%
\pgfusepath{clip}%
\pgfsetbuttcap%
\pgfsetroundjoin%
\definecolor{currentfill}{rgb}{0.277134,0.185228,0.489898}%
\pgfsetfillcolor{currentfill}%
\pgfsetfillopacity{0.700000}%
\pgfsetlinewidth{0.000000pt}%
\definecolor{currentstroke}{rgb}{0.000000,0.000000,0.000000}%
\pgfsetstrokecolor{currentstroke}%
\pgfsetdash{}{0pt}%
\pgfpathmoveto{\pgfqpoint{4.350445in}{1.789819in}}%
\pgfpathlineto{\pgfqpoint{4.364648in}{1.791310in}}%
\pgfpathlineto{\pgfqpoint{4.378860in}{1.792872in}}%
\pgfpathlineto{\pgfqpoint{4.393083in}{1.794507in}}%
\pgfpathlineto{\pgfqpoint{4.407316in}{1.796214in}}%
\pgfpathlineto{\pgfqpoint{4.415359in}{1.807318in}}%
\pgfpathlineto{\pgfqpoint{4.423397in}{1.818346in}}%
\pgfpathlineto{\pgfqpoint{4.431428in}{1.829297in}}%
\pgfpathlineto{\pgfqpoint{4.439454in}{1.840170in}}%
\pgfpathlineto{\pgfqpoint{4.425228in}{1.838339in}}%
\pgfpathlineto{\pgfqpoint{4.411012in}{1.836580in}}%
\pgfpathlineto{\pgfqpoint{4.396806in}{1.834893in}}%
\pgfpathlineto{\pgfqpoint{4.382611in}{1.833279in}}%
\pgfpathlineto{\pgfqpoint{4.374578in}{1.822522in}}%
\pgfpathlineto{\pgfqpoint{4.366539in}{1.811693in}}%
\pgfpathlineto{\pgfqpoint{4.358495in}{1.800791in}}%
\pgfpathlineto{\pgfqpoint{4.350445in}{1.789819in}}%
\pgfpathclose%
\pgfusepath{fill}%
\end{pgfscope}%
\begin{pgfscope}%
\pgfpathrectangle{\pgfqpoint{1.150000in}{0.150000in}}{\pgfqpoint{5.700000in}{5.700000in}}%
\pgfusepath{clip}%
\pgfsetbuttcap%
\pgfsetroundjoin%
\definecolor{currentfill}{rgb}{0.282327,0.094955,0.417331}%
\pgfsetfillcolor{currentfill}%
\pgfsetfillopacity{0.700000}%
\pgfsetlinewidth{0.000000pt}%
\definecolor{currentstroke}{rgb}{0.000000,0.000000,0.000000}%
\pgfsetstrokecolor{currentstroke}%
\pgfsetdash{}{0pt}%
\pgfpathmoveto{\pgfqpoint{3.994476in}{1.603528in}}%
\pgfpathlineto{\pgfqpoint{4.008561in}{1.603156in}}%
\pgfpathlineto{\pgfqpoint{4.022654in}{1.602857in}}%
\pgfpathlineto{\pgfqpoint{4.036756in}{1.602630in}}%
\pgfpathlineto{\pgfqpoint{4.050866in}{1.602477in}}%
\pgfpathlineto{\pgfqpoint{4.059026in}{1.613650in}}%
\pgfpathlineto{\pgfqpoint{4.067181in}{1.624805in}}%
\pgfpathlineto{\pgfqpoint{4.075330in}{1.635938in}}%
\pgfpathlineto{\pgfqpoint{4.083474in}{1.647047in}}%
\pgfpathlineto{\pgfqpoint{4.069372in}{1.646993in}}%
\pgfpathlineto{\pgfqpoint{4.055278in}{1.647013in}}%
\pgfpathlineto{\pgfqpoint{4.041194in}{1.647105in}}%
\pgfpathlineto{\pgfqpoint{4.027118in}{1.647270in}}%
\pgfpathlineto{\pgfqpoint{4.018966in}{1.636360in}}%
\pgfpathlineto{\pgfqpoint{4.010808in}{1.625431in}}%
\pgfpathlineto{\pgfqpoint{4.002645in}{1.614486in}}%
\pgfpathlineto{\pgfqpoint{3.994476in}{1.603528in}}%
\pgfpathclose%
\pgfusepath{fill}%
\end{pgfscope}%
\begin{pgfscope}%
\pgfpathrectangle{\pgfqpoint{1.150000in}{0.150000in}}{\pgfqpoint{5.700000in}{5.700000in}}%
\pgfusepath{clip}%
\pgfsetbuttcap%
\pgfsetroundjoin%
\definecolor{currentfill}{rgb}{0.274952,0.037752,0.364543}%
\pgfsetfillcolor{currentfill}%
\pgfsetfillopacity{0.700000}%
\pgfsetlinewidth{0.000000pt}%
\definecolor{currentstroke}{rgb}{0.000000,0.000000,0.000000}%
\pgfsetstrokecolor{currentstroke}%
\pgfsetdash{}{0pt}%
\pgfpathmoveto{\pgfqpoint{2.910715in}{1.529589in}}%
\pgfpathlineto{\pgfqpoint{2.924634in}{1.522055in}}%
\pgfpathlineto{\pgfqpoint{2.938555in}{1.514608in}}%
\pgfpathlineto{\pgfqpoint{2.952480in}{1.507246in}}%
\pgfpathlineto{\pgfqpoint{2.966407in}{1.499970in}}%
\pgfpathlineto{\pgfqpoint{2.975086in}{1.503563in}}%
\pgfpathlineto{\pgfqpoint{2.983752in}{1.507371in}}%
\pgfpathlineto{\pgfqpoint{2.992405in}{1.511390in}}%
\pgfpathlineto{\pgfqpoint{3.001046in}{1.515614in}}%
\pgfpathlineto{\pgfqpoint{2.987146in}{1.522497in}}%
\pgfpathlineto{\pgfqpoint{2.973250in}{1.529465in}}%
\pgfpathlineto{\pgfqpoint{2.959357in}{1.536519in}}%
\pgfpathlineto{\pgfqpoint{2.945467in}{1.543660in}}%
\pgfpathlineto{\pgfqpoint{2.936799in}{1.539822in}}%
\pgfpathlineto{\pgfqpoint{2.928117in}{1.536193in}}%
\pgfpathlineto{\pgfqpoint{2.919423in}{1.532780in}}%
\pgfpathlineto{\pgfqpoint{2.910715in}{1.529589in}}%
\pgfpathclose%
\pgfusepath{fill}%
\end{pgfscope}%
\begin{pgfscope}%
\pgfpathrectangle{\pgfqpoint{1.150000in}{0.150000in}}{\pgfqpoint{5.700000in}{5.700000in}}%
\pgfusepath{clip}%
\pgfsetbuttcap%
\pgfsetroundjoin%
\definecolor{currentfill}{rgb}{0.271828,0.209303,0.504434}%
\pgfsetfillcolor{currentfill}%
\pgfsetfillopacity{0.700000}%
\pgfsetlinewidth{0.000000pt}%
\definecolor{currentstroke}{rgb}{0.000000,0.000000,0.000000}%
\pgfsetstrokecolor{currentstroke}%
\pgfsetdash{}{0pt}%
\pgfpathmoveto{\pgfqpoint{4.439454in}{1.840170in}}%
\pgfpathlineto{\pgfqpoint{4.453690in}{1.842072in}}%
\pgfpathlineto{\pgfqpoint{4.467937in}{1.844046in}}%
\pgfpathlineto{\pgfqpoint{4.482195in}{1.846092in}}%
\pgfpathlineto{\pgfqpoint{4.496463in}{1.848210in}}%
\pgfpathlineto{\pgfqpoint{4.504476in}{1.859113in}}%
\pgfpathlineto{\pgfqpoint{4.512483in}{1.869929in}}%
\pgfpathlineto{\pgfqpoint{4.520484in}{1.880658in}}%
\pgfpathlineto{\pgfqpoint{4.528479in}{1.891297in}}%
\pgfpathlineto{\pgfqpoint{4.514217in}{1.889076in}}%
\pgfpathlineto{\pgfqpoint{4.499967in}{1.886927in}}%
\pgfpathlineto{\pgfqpoint{4.485727in}{1.884850in}}%
\pgfpathlineto{\pgfqpoint{4.471498in}{1.882845in}}%
\pgfpathlineto{\pgfqpoint{4.463496in}{1.872301in}}%
\pgfpathlineto{\pgfqpoint{4.455488in}{1.861673in}}%
\pgfpathlineto{\pgfqpoint{4.447474in}{1.850962in}}%
\pgfpathlineto{\pgfqpoint{4.439454in}{1.840170in}}%
\pgfpathclose%
\pgfusepath{fill}%
\end{pgfscope}%
\begin{pgfscope}%
\pgfpathrectangle{\pgfqpoint{1.150000in}{0.150000in}}{\pgfqpoint{5.700000in}{5.700000in}}%
\pgfusepath{clip}%
\pgfsetbuttcap%
\pgfsetroundjoin%
\definecolor{currentfill}{rgb}{0.281446,0.084320,0.407414}%
\pgfsetfillcolor{currentfill}%
\pgfsetfillopacity{0.700000}%
\pgfsetlinewidth{0.000000pt}%
\definecolor{currentstroke}{rgb}{0.000000,0.000000,0.000000}%
\pgfsetstrokecolor{currentstroke}%
\pgfsetdash{}{0pt}%
\pgfpathmoveto{\pgfqpoint{2.708514in}{1.621457in}}%
\pgfpathlineto{\pgfqpoint{2.722440in}{1.612436in}}%
\pgfpathlineto{\pgfqpoint{2.736368in}{1.603507in}}%
\pgfpathlineto{\pgfqpoint{2.750297in}{1.594669in}}%
\pgfpathlineto{\pgfqpoint{2.764228in}{1.585922in}}%
\pgfpathlineto{\pgfqpoint{2.773058in}{1.587315in}}%
\pgfpathlineto{\pgfqpoint{2.781872in}{1.588967in}}%
\pgfpathlineto{\pgfqpoint{2.790671in}{1.590870in}}%
\pgfpathlineto{\pgfqpoint{2.799454in}{1.593018in}}%
\pgfpathlineto{\pgfqpoint{2.785557in}{1.601348in}}%
\pgfpathlineto{\pgfqpoint{2.771661in}{1.609769in}}%
\pgfpathlineto{\pgfqpoint{2.757767in}{1.618281in}}%
\pgfpathlineto{\pgfqpoint{2.743874in}{1.626885in}}%
\pgfpathlineto{\pgfqpoint{2.735058in}{1.625146in}}%
\pgfpathlineto{\pgfqpoint{2.726226in}{1.623657in}}%
\pgfpathlineto{\pgfqpoint{2.717378in}{1.622425in}}%
\pgfpathlineto{\pgfqpoint{2.708514in}{1.621457in}}%
\pgfpathclose%
\pgfusepath{fill}%
\end{pgfscope}%
\begin{pgfscope}%
\pgfpathrectangle{\pgfqpoint{1.150000in}{0.150000in}}{\pgfqpoint{5.700000in}{5.700000in}}%
\pgfusepath{clip}%
\pgfsetbuttcap%
\pgfsetroundjoin%
\definecolor{currentfill}{rgb}{0.280267,0.073417,0.397163}%
\pgfsetfillcolor{currentfill}%
\pgfsetfillopacity{0.700000}%
\pgfsetlinewidth{0.000000pt}%
\definecolor{currentstroke}{rgb}{0.000000,0.000000,0.000000}%
\pgfsetstrokecolor{currentstroke}%
\pgfsetdash{}{0pt}%
\pgfpathmoveto{\pgfqpoint{3.905455in}{1.562751in}}%
\pgfpathlineto{\pgfqpoint{3.919516in}{1.561859in}}%
\pgfpathlineto{\pgfqpoint{3.933585in}{1.561040in}}%
\pgfpathlineto{\pgfqpoint{3.947662in}{1.560295in}}%
\pgfpathlineto{\pgfqpoint{3.961747in}{1.559622in}}%
\pgfpathlineto{\pgfqpoint{3.969938in}{1.570603in}}%
\pgfpathlineto{\pgfqpoint{3.978123in}{1.581583in}}%
\pgfpathlineto{\pgfqpoint{3.986302in}{1.592559in}}%
\pgfpathlineto{\pgfqpoint{3.994476in}{1.603528in}}%
\pgfpathlineto{\pgfqpoint{3.980400in}{1.603973in}}%
\pgfpathlineto{\pgfqpoint{3.966332in}{1.604491in}}%
\pgfpathlineto{\pgfqpoint{3.952273in}{1.605082in}}%
\pgfpathlineto{\pgfqpoint{3.938222in}{1.605747in}}%
\pgfpathlineto{\pgfqpoint{3.930038in}{1.594998in}}%
\pgfpathlineto{\pgfqpoint{3.921850in}{1.584247in}}%
\pgfpathlineto{\pgfqpoint{3.913655in}{1.573497in}}%
\pgfpathlineto{\pgfqpoint{3.905455in}{1.562751in}}%
\pgfpathclose%
\pgfusepath{fill}%
\end{pgfscope}%
\begin{pgfscope}%
\pgfpathrectangle{\pgfqpoint{1.150000in}{0.150000in}}{\pgfqpoint{5.700000in}{5.700000in}}%
\pgfusepath{clip}%
\pgfsetbuttcap%
\pgfsetroundjoin%
\definecolor{currentfill}{rgb}{0.265145,0.232956,0.516599}%
\pgfsetfillcolor{currentfill}%
\pgfsetfillopacity{0.700000}%
\pgfsetlinewidth{0.000000pt}%
\definecolor{currentstroke}{rgb}{0.000000,0.000000,0.000000}%
\pgfsetstrokecolor{currentstroke}%
\pgfsetdash{}{0pt}%
\pgfpathmoveto{\pgfqpoint{4.528479in}{1.891297in}}%
\pgfpathlineto{\pgfqpoint{4.542750in}{1.893589in}}%
\pgfpathlineto{\pgfqpoint{4.557033in}{1.895953in}}%
\pgfpathlineto{\pgfqpoint{4.571327in}{1.898389in}}%
\pgfpathlineto{\pgfqpoint{4.585631in}{1.900896in}}%
\pgfpathlineto{\pgfqpoint{4.593613in}{1.911535in}}%
\pgfpathlineto{\pgfqpoint{4.601588in}{1.922077in}}%
\pgfpathlineto{\pgfqpoint{4.609557in}{1.932522in}}%
\pgfpathlineto{\pgfqpoint{4.617520in}{1.942869in}}%
\pgfpathlineto{\pgfqpoint{4.603222in}{1.940280in}}%
\pgfpathlineto{\pgfqpoint{4.588936in}{1.937763in}}%
\pgfpathlineto{\pgfqpoint{4.574661in}{1.935317in}}%
\pgfpathlineto{\pgfqpoint{4.560396in}{1.932943in}}%
\pgfpathlineto{\pgfqpoint{4.552426in}{1.922670in}}%
\pgfpathlineto{\pgfqpoint{4.544450in}{1.912304in}}%
\pgfpathlineto{\pgfqpoint{4.536467in}{1.901846in}}%
\pgfpathlineto{\pgfqpoint{4.528479in}{1.891297in}}%
\pgfpathclose%
\pgfusepath{fill}%
\end{pgfscope}%
\begin{pgfscope}%
\pgfpathrectangle{\pgfqpoint{1.150000in}{0.150000in}}{\pgfqpoint{5.700000in}{5.700000in}}%
\pgfusepath{clip}%
\pgfsetbuttcap%
\pgfsetroundjoin%
\definecolor{currentfill}{rgb}{0.268510,0.009605,0.335427}%
\pgfsetfillcolor{currentfill}%
\pgfsetfillopacity{0.700000}%
\pgfsetlinewidth{0.000000pt}%
\definecolor{currentstroke}{rgb}{0.000000,0.000000,0.000000}%
\pgfsetstrokecolor{currentstroke}%
\pgfsetdash{}{0pt}%
\pgfpathmoveto{\pgfqpoint{3.492838in}{1.450089in}}%
\pgfpathlineto{\pgfqpoint{3.506809in}{1.446588in}}%
\pgfpathlineto{\pgfqpoint{3.520785in}{1.443164in}}%
\pgfpathlineto{\pgfqpoint{3.534767in}{1.439816in}}%
\pgfpathlineto{\pgfqpoint{3.548755in}{1.436543in}}%
\pgfpathlineto{\pgfqpoint{3.557105in}{1.445444in}}%
\pgfpathlineto{\pgfqpoint{3.565447in}{1.454432in}}%
\pgfpathlineto{\pgfqpoint{3.573783in}{1.463504in}}%
\pgfpathlineto{\pgfqpoint{3.582111in}{1.472656in}}%
\pgfpathlineto{\pgfqpoint{3.568138in}{1.475619in}}%
\pgfpathlineto{\pgfqpoint{3.554171in}{1.478659in}}%
\pgfpathlineto{\pgfqpoint{3.540211in}{1.481775in}}%
\pgfpathlineto{\pgfqpoint{3.526256in}{1.484967in}}%
\pgfpathlineto{\pgfqpoint{3.517913in}{1.476117in}}%
\pgfpathlineto{\pgfqpoint{3.509562in}{1.467350in}}%
\pgfpathlineto{\pgfqpoint{3.501204in}{1.458673in}}%
\pgfpathlineto{\pgfqpoint{3.492838in}{1.450089in}}%
\pgfpathclose%
\pgfusepath{fill}%
\end{pgfscope}%
\begin{pgfscope}%
\pgfpathrectangle{\pgfqpoint{1.150000in}{0.150000in}}{\pgfqpoint{5.700000in}{5.700000in}}%
\pgfusepath{clip}%
\pgfsetbuttcap%
\pgfsetroundjoin%
\definecolor{currentfill}{rgb}{0.257322,0.256130,0.526563}%
\pgfsetfillcolor{currentfill}%
\pgfsetfillopacity{0.700000}%
\pgfsetlinewidth{0.000000pt}%
\definecolor{currentstroke}{rgb}{0.000000,0.000000,0.000000}%
\pgfsetstrokecolor{currentstroke}%
\pgfsetdash{}{0pt}%
\pgfpathmoveto{\pgfqpoint{4.617520in}{1.942869in}}%
\pgfpathlineto{\pgfqpoint{4.631828in}{1.945530in}}%
\pgfpathlineto{\pgfqpoint{4.646147in}{1.948262in}}%
\pgfpathlineto{\pgfqpoint{4.660478in}{1.951065in}}%
\pgfpathlineto{\pgfqpoint{4.674820in}{1.953940in}}%
\pgfpathlineto{\pgfqpoint{4.682769in}{1.964257in}}%
\pgfpathlineto{\pgfqpoint{4.690711in}{1.974468in}}%
\pgfpathlineto{\pgfqpoint{4.698647in}{1.984575in}}%
\pgfpathlineto{\pgfqpoint{4.706575in}{1.994576in}}%
\pgfpathlineto{\pgfqpoint{4.692241in}{1.991640in}}%
\pgfpathlineto{\pgfqpoint{4.677918in}{1.988776in}}%
\pgfpathlineto{\pgfqpoint{4.663606in}{1.985984in}}%
\pgfpathlineto{\pgfqpoint{4.649305in}{1.983263in}}%
\pgfpathlineto{\pgfqpoint{4.641369in}{1.973315in}}%
\pgfpathlineto{\pgfqpoint{4.633425in}{1.963266in}}%
\pgfpathlineto{\pgfqpoint{4.625476in}{1.953117in}}%
\pgfpathlineto{\pgfqpoint{4.617520in}{1.942869in}}%
\pgfpathclose%
\pgfusepath{fill}%
\end{pgfscope}%
\begin{pgfscope}%
\pgfpathrectangle{\pgfqpoint{1.150000in}{0.150000in}}{\pgfqpoint{5.700000in}{5.700000in}}%
\pgfusepath{clip}%
\pgfsetbuttcap%
\pgfsetroundjoin%
\definecolor{currentfill}{rgb}{0.248629,0.278775,0.534556}%
\pgfsetfillcolor{currentfill}%
\pgfsetfillopacity{0.700000}%
\pgfsetlinewidth{0.000000pt}%
\definecolor{currentstroke}{rgb}{0.000000,0.000000,0.000000}%
\pgfsetstrokecolor{currentstroke}%
\pgfsetdash{}{0pt}%
\pgfpathmoveto{\pgfqpoint{4.706575in}{1.994576in}}%
\pgfpathlineto{\pgfqpoint{4.720921in}{1.997583in}}%
\pgfpathlineto{\pgfqpoint{4.735278in}{2.000661in}}%
\pgfpathlineto{\pgfqpoint{4.749647in}{2.003810in}}%
\pgfpathlineto{\pgfqpoint{4.764028in}{2.007031in}}%
\pgfpathlineto{\pgfqpoint{4.771942in}{2.016973in}}%
\pgfpathlineto{\pgfqpoint{4.779849in}{2.026803in}}%
\pgfpathlineto{\pgfqpoint{4.787749in}{2.036521in}}%
\pgfpathlineto{\pgfqpoint{4.795642in}{2.046126in}}%
\pgfpathlineto{\pgfqpoint{4.781270in}{2.042867in}}%
\pgfpathlineto{\pgfqpoint{4.766909in}{2.039678in}}%
\pgfpathlineto{\pgfqpoint{4.752560in}{2.036561in}}%
\pgfpathlineto{\pgfqpoint{4.738223in}{2.033515in}}%
\pgfpathlineto{\pgfqpoint{4.730321in}{2.023940in}}%
\pgfpathlineto{\pgfqpoint{4.722413in}{2.014259in}}%
\pgfpathlineto{\pgfqpoint{4.714497in}{2.004470in}}%
\pgfpathlineto{\pgfqpoint{4.706575in}{1.994576in}}%
\pgfpathclose%
\pgfusepath{fill}%
\end{pgfscope}%
\begin{pgfscope}%
\pgfpathrectangle{\pgfqpoint{1.150000in}{0.150000in}}{\pgfqpoint{5.700000in}{5.700000in}}%
\pgfusepath{clip}%
\pgfsetbuttcap%
\pgfsetroundjoin%
\definecolor{currentfill}{rgb}{0.277941,0.056324,0.381191}%
\pgfsetfillcolor{currentfill}%
\pgfsetfillopacity{0.700000}%
\pgfsetlinewidth{0.000000pt}%
\definecolor{currentstroke}{rgb}{0.000000,0.000000,0.000000}%
\pgfsetstrokecolor{currentstroke}%
\pgfsetdash{}{0pt}%
\pgfpathmoveto{\pgfqpoint{3.816394in}{1.525175in}}%
\pgfpathlineto{\pgfqpoint{3.830434in}{1.523740in}}%
\pgfpathlineto{\pgfqpoint{3.844481in}{1.522380in}}%
\pgfpathlineto{\pgfqpoint{3.858536in}{1.521093in}}%
\pgfpathlineto{\pgfqpoint{3.872599in}{1.519880in}}%
\pgfpathlineto{\pgfqpoint{3.880821in}{1.530574in}}%
\pgfpathlineto{\pgfqpoint{3.889038in}{1.541286in}}%
\pgfpathlineto{\pgfqpoint{3.897249in}{1.552013in}}%
\pgfpathlineto{\pgfqpoint{3.905455in}{1.562751in}}%
\pgfpathlineto{\pgfqpoint{3.891402in}{1.563717in}}%
\pgfpathlineto{\pgfqpoint{3.877358in}{1.564756in}}%
\pgfpathlineto{\pgfqpoint{3.863321in}{1.565869in}}%
\pgfpathlineto{\pgfqpoint{3.849292in}{1.567056in}}%
\pgfpathlineto{\pgfqpoint{3.841076in}{1.556558in}}%
\pgfpathlineto{\pgfqpoint{3.832855in}{1.546076in}}%
\pgfpathlineto{\pgfqpoint{3.824627in}{1.535614in}}%
\pgfpathlineto{\pgfqpoint{3.816394in}{1.525175in}}%
\pgfpathclose%
\pgfusepath{fill}%
\end{pgfscope}%
\begin{pgfscope}%
\pgfpathrectangle{\pgfqpoint{1.150000in}{0.150000in}}{\pgfqpoint{5.700000in}{5.700000in}}%
\pgfusepath{clip}%
\pgfsetbuttcap%
\pgfsetroundjoin%
\definecolor{currentfill}{rgb}{0.180629,0.429975,0.557282}%
\pgfsetfillcolor{currentfill}%
\pgfsetfillopacity{0.700000}%
\pgfsetlinewidth{0.000000pt}%
\definecolor{currentstroke}{rgb}{0.000000,0.000000,0.000000}%
\pgfsetstrokecolor{currentstroke}%
\pgfsetdash{}{0pt}%
\pgfpathmoveto{\pgfqpoint{5.418854in}{2.382669in}}%
\pgfpathlineto{\pgfqpoint{5.433512in}{2.387651in}}%
\pgfpathlineto{\pgfqpoint{5.448184in}{2.392703in}}%
\pgfpathlineto{\pgfqpoint{5.462869in}{2.397827in}}%
\pgfpathlineto{\pgfqpoint{5.477567in}{2.403022in}}%
\pgfpathlineto{\pgfqpoint{5.485124in}{2.408707in}}%
\pgfpathlineto{\pgfqpoint{5.492670in}{2.414275in}}%
\pgfpathlineto{\pgfqpoint{5.500208in}{2.419727in}}%
\pgfpathlineto{\pgfqpoint{5.507735in}{2.425067in}}%
\pgfpathlineto{\pgfqpoint{5.493053in}{2.420009in}}%
\pgfpathlineto{\pgfqpoint{5.478385in}{2.415023in}}%
\pgfpathlineto{\pgfqpoint{5.463730in}{2.410107in}}%
\pgfpathlineto{\pgfqpoint{5.449089in}{2.405262in}}%
\pgfpathlineto{\pgfqpoint{5.441544in}{2.399777in}}%
\pgfpathlineto{\pgfqpoint{5.433990in}{2.394185in}}%
\pgfpathlineto{\pgfqpoint{5.426427in}{2.388483in}}%
\pgfpathlineto{\pgfqpoint{5.418854in}{2.382669in}}%
\pgfpathclose%
\pgfusepath{fill}%
\end{pgfscope}%
\begin{pgfscope}%
\pgfpathrectangle{\pgfqpoint{1.150000in}{0.150000in}}{\pgfqpoint{5.700000in}{5.700000in}}%
\pgfusepath{clip}%
\pgfsetbuttcap%
\pgfsetroundjoin%
\definecolor{currentfill}{rgb}{0.239346,0.300855,0.540844}%
\pgfsetfillcolor{currentfill}%
\pgfsetfillopacity{0.700000}%
\pgfsetlinewidth{0.000000pt}%
\definecolor{currentstroke}{rgb}{0.000000,0.000000,0.000000}%
\pgfsetstrokecolor{currentstroke}%
\pgfsetdash{}{0pt}%
\pgfpathmoveto{\pgfqpoint{4.795642in}{2.046126in}}%
\pgfpathlineto{\pgfqpoint{4.810026in}{2.049458in}}%
\pgfpathlineto{\pgfqpoint{4.824422in}{2.052860in}}%
\pgfpathlineto{\pgfqpoint{4.838830in}{2.056334in}}%
\pgfpathlineto{\pgfqpoint{4.853250in}{2.059879in}}%
\pgfpathlineto{\pgfqpoint{4.861127in}{2.069399in}}%
\pgfpathlineto{\pgfqpoint{4.868997in}{2.078800in}}%
\pgfpathlineto{\pgfqpoint{4.876860in}{2.088085in}}%
\pgfpathlineto{\pgfqpoint{4.884716in}{2.097252in}}%
\pgfpathlineto{\pgfqpoint{4.870305in}{2.093690in}}%
\pgfpathlineto{\pgfqpoint{4.855906in}{2.090198in}}%
\pgfpathlineto{\pgfqpoint{4.841519in}{2.086778in}}%
\pgfpathlineto{\pgfqpoint{4.827144in}{2.083429in}}%
\pgfpathlineto{\pgfqpoint{4.819279in}{2.074271in}}%
\pgfpathlineto{\pgfqpoint{4.811407in}{2.065002in}}%
\pgfpathlineto{\pgfqpoint{4.803528in}{2.055620in}}%
\pgfpathlineto{\pgfqpoint{4.795642in}{2.046126in}}%
\pgfpathclose%
\pgfusepath{fill}%
\end{pgfscope}%
\begin{pgfscope}%
\pgfpathrectangle{\pgfqpoint{1.150000in}{0.150000in}}{\pgfqpoint{5.700000in}{5.700000in}}%
\pgfusepath{clip}%
\pgfsetbuttcap%
\pgfsetroundjoin%
\definecolor{currentfill}{rgb}{0.187231,0.414746,0.556547}%
\pgfsetfillcolor{currentfill}%
\pgfsetfillopacity{0.700000}%
\pgfsetlinewidth{0.000000pt}%
\definecolor{currentstroke}{rgb}{0.000000,0.000000,0.000000}%
\pgfsetstrokecolor{currentstroke}%
\pgfsetdash{}{0pt}%
\pgfpathmoveto{\pgfqpoint{5.329915in}{2.338559in}}%
\pgfpathlineto{\pgfqpoint{5.344534in}{2.343373in}}%
\pgfpathlineto{\pgfqpoint{5.359167in}{2.348257in}}%
\pgfpathlineto{\pgfqpoint{5.373813in}{2.353212in}}%
\pgfpathlineto{\pgfqpoint{5.388472in}{2.358239in}}%
\pgfpathlineto{\pgfqpoint{5.396082in}{2.364527in}}%
\pgfpathlineto{\pgfqpoint{5.403682in}{2.370693in}}%
\pgfpathlineto{\pgfqpoint{5.411273in}{2.376739in}}%
\pgfpathlineto{\pgfqpoint{5.418854in}{2.382669in}}%
\pgfpathlineto{\pgfqpoint{5.404210in}{2.377757in}}%
\pgfpathlineto{\pgfqpoint{5.389579in}{2.372917in}}%
\pgfpathlineto{\pgfqpoint{5.374962in}{2.368147in}}%
\pgfpathlineto{\pgfqpoint{5.360358in}{2.363449in}}%
\pgfpathlineto{\pgfqpoint{5.352760in}{2.357397in}}%
\pgfpathlineto{\pgfqpoint{5.345154in}{2.351232in}}%
\pgfpathlineto{\pgfqpoint{5.337539in}{2.344954in}}%
\pgfpathlineto{\pgfqpoint{5.329915in}{2.338559in}}%
\pgfpathclose%
\pgfusepath{fill}%
\end{pgfscope}%
\begin{pgfscope}%
\pgfpathrectangle{\pgfqpoint{1.150000in}{0.150000in}}{\pgfqpoint{5.700000in}{5.700000in}}%
\pgfusepath{clip}%
\pgfsetbuttcap%
\pgfsetroundjoin%
\definecolor{currentfill}{rgb}{0.229739,0.322361,0.545706}%
\pgfsetfillcolor{currentfill}%
\pgfsetfillopacity{0.700000}%
\pgfsetlinewidth{0.000000pt}%
\definecolor{currentstroke}{rgb}{0.000000,0.000000,0.000000}%
\pgfsetstrokecolor{currentstroke}%
\pgfsetdash{}{0pt}%
\pgfpathmoveto{\pgfqpoint{4.884716in}{2.097252in}}%
\pgfpathlineto{\pgfqpoint{4.899139in}{2.100886in}}%
\pgfpathlineto{\pgfqpoint{4.913574in}{2.104591in}}%
\pgfpathlineto{\pgfqpoint{4.928021in}{2.108367in}}%
\pgfpathlineto{\pgfqpoint{4.942480in}{2.112215in}}%
\pgfpathlineto{\pgfqpoint{4.950319in}{2.121269in}}%
\pgfpathlineto{\pgfqpoint{4.958150in}{2.130202in}}%
\pgfpathlineto{\pgfqpoint{4.965974in}{2.139013in}}%
\pgfpathlineto{\pgfqpoint{4.973789in}{2.147704in}}%
\pgfpathlineto{\pgfqpoint{4.959340in}{2.143861in}}%
\pgfpathlineto{\pgfqpoint{4.944902in}{2.140089in}}%
\pgfpathlineto{\pgfqpoint{4.930477in}{2.136389in}}%
\pgfpathlineto{\pgfqpoint{4.916064in}{2.132759in}}%
\pgfpathlineto{\pgfqpoint{4.908238in}{2.124056in}}%
\pgfpathlineto{\pgfqpoint{4.900405in}{2.115237in}}%
\pgfpathlineto{\pgfqpoint{4.892564in}{2.106303in}}%
\pgfpathlineto{\pgfqpoint{4.884716in}{2.097252in}}%
\pgfpathclose%
\pgfusepath{fill}%
\end{pgfscope}%
\begin{pgfscope}%
\pgfpathrectangle{\pgfqpoint{1.150000in}{0.150000in}}{\pgfqpoint{5.700000in}{5.700000in}}%
\pgfusepath{clip}%
\pgfsetbuttcap%
\pgfsetroundjoin%
\definecolor{currentfill}{rgb}{0.220057,0.343307,0.549413}%
\pgfsetfillcolor{currentfill}%
\pgfsetfillopacity{0.700000}%
\pgfsetlinewidth{0.000000pt}%
\definecolor{currentstroke}{rgb}{0.000000,0.000000,0.000000}%
\pgfsetstrokecolor{currentstroke}%
\pgfsetdash{}{0pt}%
\pgfpathmoveto{\pgfqpoint{4.973789in}{2.147704in}}%
\pgfpathlineto{\pgfqpoint{4.988252in}{2.151619in}}%
\pgfpathlineto{\pgfqpoint{5.002726in}{2.155604in}}%
\pgfpathlineto{\pgfqpoint{5.017213in}{2.159661in}}%
\pgfpathlineto{\pgfqpoint{5.031712in}{2.163788in}}%
\pgfpathlineto{\pgfqpoint{5.039510in}{2.172341in}}%
\pgfpathlineto{\pgfqpoint{5.047300in}{2.180770in}}%
\pgfpathlineto{\pgfqpoint{5.055082in}{2.189074in}}%
\pgfpathlineto{\pgfqpoint{5.062855in}{2.197256in}}%
\pgfpathlineto{\pgfqpoint{5.048366in}{2.193155in}}%
\pgfpathlineto{\pgfqpoint{5.033890in}{2.189124in}}%
\pgfpathlineto{\pgfqpoint{5.019426in}{2.185165in}}%
\pgfpathlineto{\pgfqpoint{5.004975in}{2.181277in}}%
\pgfpathlineto{\pgfqpoint{4.997190in}{2.173061in}}%
\pgfpathlineto{\pgfqpoint{4.989398in}{2.164727in}}%
\pgfpathlineto{\pgfqpoint{4.981597in}{2.156275in}}%
\pgfpathlineto{\pgfqpoint{4.973789in}{2.147704in}}%
\pgfpathclose%
\pgfusepath{fill}%
\end{pgfscope}%
\begin{pgfscope}%
\pgfpathrectangle{\pgfqpoint{1.150000in}{0.150000in}}{\pgfqpoint{5.700000in}{5.700000in}}%
\pgfusepath{clip}%
\pgfsetbuttcap%
\pgfsetroundjoin%
\definecolor{currentfill}{rgb}{0.194100,0.399323,0.555565}%
\pgfsetfillcolor{currentfill}%
\pgfsetfillopacity{0.700000}%
\pgfsetlinewidth{0.000000pt}%
\definecolor{currentstroke}{rgb}{0.000000,0.000000,0.000000}%
\pgfsetstrokecolor{currentstroke}%
\pgfsetdash{}{0pt}%
\pgfpathmoveto{\pgfqpoint{5.240928in}{2.292857in}}%
\pgfpathlineto{\pgfqpoint{5.255508in}{2.297479in}}%
\pgfpathlineto{\pgfqpoint{5.270102in}{2.302172in}}%
\pgfpathlineto{\pgfqpoint{5.284708in}{2.306936in}}%
\pgfpathlineto{\pgfqpoint{5.299328in}{2.311772in}}%
\pgfpathlineto{\pgfqpoint{5.306988in}{2.318654in}}%
\pgfpathlineto{\pgfqpoint{5.314639in}{2.325411in}}%
\pgfpathlineto{\pgfqpoint{5.322282in}{2.332046in}}%
\pgfpathlineto{\pgfqpoint{5.329915in}{2.338559in}}%
\pgfpathlineto{\pgfqpoint{5.315309in}{2.333817in}}%
\pgfpathlineto{\pgfqpoint{5.300716in}{2.329145in}}%
\pgfpathlineto{\pgfqpoint{5.286137in}{2.324544in}}%
\pgfpathlineto{\pgfqpoint{5.271570in}{2.320015in}}%
\pgfpathlineto{\pgfqpoint{5.263923in}{2.313400in}}%
\pgfpathlineto{\pgfqpoint{5.256267in}{2.306670in}}%
\pgfpathlineto{\pgfqpoint{5.248602in}{2.299823in}}%
\pgfpathlineto{\pgfqpoint{5.240928in}{2.292857in}}%
\pgfpathclose%
\pgfusepath{fill}%
\end{pgfscope}%
\begin{pgfscope}%
\pgfpathrectangle{\pgfqpoint{1.150000in}{0.150000in}}{\pgfqpoint{5.700000in}{5.700000in}}%
\pgfusepath{clip}%
\pgfsetbuttcap%
\pgfsetroundjoin%
\definecolor{currentfill}{rgb}{0.268510,0.009605,0.335427}%
\pgfsetfillcolor{currentfill}%
\pgfsetfillopacity{0.700000}%
\pgfsetlinewidth{0.000000pt}%
\definecolor{currentstroke}{rgb}{0.000000,0.000000,0.000000}%
\pgfsetstrokecolor{currentstroke}%
\pgfsetdash{}{0pt}%
\pgfpathmoveto{\pgfqpoint{3.112363in}{1.463579in}}%
\pgfpathlineto{\pgfqpoint{3.126295in}{1.457447in}}%
\pgfpathlineto{\pgfqpoint{3.140230in}{1.451398in}}%
\pgfpathlineto{\pgfqpoint{3.154169in}{1.445430in}}%
\pgfpathlineto{\pgfqpoint{3.168113in}{1.439544in}}%
\pgfpathlineto{\pgfqpoint{3.176665in}{1.445102in}}%
\pgfpathlineto{\pgfqpoint{3.185208in}{1.450837in}}%
\pgfpathlineto{\pgfqpoint{3.193739in}{1.456743in}}%
\pgfpathlineto{\pgfqpoint{3.202261in}{1.462815in}}%
\pgfpathlineto{\pgfqpoint{3.188341in}{1.468330in}}%
\pgfpathlineto{\pgfqpoint{3.174425in}{1.473927in}}%
\pgfpathlineto{\pgfqpoint{3.160513in}{1.479606in}}%
\pgfpathlineto{\pgfqpoint{3.146606in}{1.485366in}}%
\pgfpathlineto{\pgfqpoint{3.138062in}{1.479658in}}%
\pgfpathlineto{\pgfqpoint{3.129506in}{1.474120in}}%
\pgfpathlineto{\pgfqpoint{3.120940in}{1.468758in}}%
\pgfpathlineto{\pgfqpoint{3.112363in}{1.463579in}}%
\pgfpathclose%
\pgfusepath{fill}%
\end{pgfscope}%
\begin{pgfscope}%
\pgfpathrectangle{\pgfqpoint{1.150000in}{0.150000in}}{\pgfqpoint{5.700000in}{5.700000in}}%
\pgfusepath{clip}%
\pgfsetbuttcap%
\pgfsetroundjoin%
\definecolor{currentfill}{rgb}{0.210503,0.363727,0.552206}%
\pgfsetfillcolor{currentfill}%
\pgfsetfillopacity{0.700000}%
\pgfsetlinewidth{0.000000pt}%
\definecolor{currentstroke}{rgb}{0.000000,0.000000,0.000000}%
\pgfsetstrokecolor{currentstroke}%
\pgfsetdash{}{0pt}%
\pgfpathmoveto{\pgfqpoint{5.062855in}{2.197256in}}%
\pgfpathlineto{\pgfqpoint{5.077357in}{2.201429in}}%
\pgfpathlineto{\pgfqpoint{5.091871in}{2.205672in}}%
\pgfpathlineto{\pgfqpoint{5.106398in}{2.209987in}}%
\pgfpathlineto{\pgfqpoint{5.120937in}{2.214373in}}%
\pgfpathlineto{\pgfqpoint{5.128692in}{2.222393in}}%
\pgfpathlineto{\pgfqpoint{5.136438in}{2.230287in}}%
\pgfpathlineto{\pgfqpoint{5.144175in}{2.238056in}}%
\pgfpathlineto{\pgfqpoint{5.151905in}{2.245702in}}%
\pgfpathlineto{\pgfqpoint{5.137377in}{2.241364in}}%
\pgfpathlineto{\pgfqpoint{5.122861in}{2.237098in}}%
\pgfpathlineto{\pgfqpoint{5.108359in}{2.232902in}}%
\pgfpathlineto{\pgfqpoint{5.093869in}{2.228778in}}%
\pgfpathlineto{\pgfqpoint{5.086128in}{2.221076in}}%
\pgfpathlineto{\pgfqpoint{5.078378in}{2.213256in}}%
\pgfpathlineto{\pgfqpoint{5.070621in}{2.205316in}}%
\pgfpathlineto{\pgfqpoint{5.062855in}{2.197256in}}%
\pgfpathclose%
\pgfusepath{fill}%
\end{pgfscope}%
\begin{pgfscope}%
\pgfpathrectangle{\pgfqpoint{1.150000in}{0.150000in}}{\pgfqpoint{5.700000in}{5.700000in}}%
\pgfusepath{clip}%
\pgfsetbuttcap%
\pgfsetroundjoin%
\definecolor{currentfill}{rgb}{0.267004,0.004874,0.329415}%
\pgfsetfillcolor{currentfill}%
\pgfsetfillopacity{0.700000}%
\pgfsetlinewidth{0.000000pt}%
\definecolor{currentstroke}{rgb}{0.000000,0.000000,0.000000}%
\pgfsetstrokecolor{currentstroke}%
\pgfsetdash{}{0pt}%
\pgfpathmoveto{\pgfqpoint{3.257986in}{1.441556in}}%
\pgfpathlineto{\pgfqpoint{3.271929in}{1.436442in}}%
\pgfpathlineto{\pgfqpoint{3.285876in}{1.431406in}}%
\pgfpathlineto{\pgfqpoint{3.299829in}{1.426450in}}%
\pgfpathlineto{\pgfqpoint{3.313786in}{1.421572in}}%
\pgfpathlineto{\pgfqpoint{3.322255in}{1.428518in}}%
\pgfpathlineto{\pgfqpoint{3.330714in}{1.435608in}}%
\pgfpathlineto{\pgfqpoint{3.339165in}{1.442836in}}%
\pgfpathlineto{\pgfqpoint{3.347607in}{1.450197in}}%
\pgfpathlineto{\pgfqpoint{3.333669in}{1.454725in}}%
\pgfpathlineto{\pgfqpoint{3.319737in}{1.459332in}}%
\pgfpathlineto{\pgfqpoint{3.305810in}{1.464018in}}%
\pgfpathlineto{\pgfqpoint{3.291888in}{1.468783in}}%
\pgfpathlineto{\pgfqpoint{3.283426in}{1.461763in}}%
\pgfpathlineto{\pgfqpoint{3.274955in}{1.454882in}}%
\pgfpathlineto{\pgfqpoint{3.266475in}{1.448145in}}%
\pgfpathlineto{\pgfqpoint{3.257986in}{1.441556in}}%
\pgfpathclose%
\pgfusepath{fill}%
\end{pgfscope}%
\begin{pgfscope}%
\pgfpathrectangle{\pgfqpoint{1.150000in}{0.150000in}}{\pgfqpoint{5.700000in}{5.700000in}}%
\pgfusepath{clip}%
\pgfsetbuttcap%
\pgfsetroundjoin%
\definecolor{currentfill}{rgb}{0.203063,0.379716,0.553925}%
\pgfsetfillcolor{currentfill}%
\pgfsetfillopacity{0.700000}%
\pgfsetlinewidth{0.000000pt}%
\definecolor{currentstroke}{rgb}{0.000000,0.000000,0.000000}%
\pgfsetstrokecolor{currentstroke}%
\pgfsetdash{}{0pt}%
\pgfpathmoveto{\pgfqpoint{5.151905in}{2.245702in}}%
\pgfpathlineto{\pgfqpoint{5.166446in}{2.250110in}}%
\pgfpathlineto{\pgfqpoint{5.180999in}{2.254590in}}%
\pgfpathlineto{\pgfqpoint{5.195566in}{2.259140in}}%
\pgfpathlineto{\pgfqpoint{5.210146in}{2.263762in}}%
\pgfpathlineto{\pgfqpoint{5.217854in}{2.271223in}}%
\pgfpathlineto{\pgfqpoint{5.225554in}{2.278558in}}%
\pgfpathlineto{\pgfqpoint{5.233246in}{2.285769in}}%
\pgfpathlineto{\pgfqpoint{5.240928in}{2.292857in}}%
\pgfpathlineto{\pgfqpoint{5.226361in}{2.288306in}}%
\pgfpathlineto{\pgfqpoint{5.211807in}{2.283825in}}%
\pgfpathlineto{\pgfqpoint{5.197266in}{2.279416in}}%
\pgfpathlineto{\pgfqpoint{5.182737in}{2.275078in}}%
\pgfpathlineto{\pgfqpoint{5.175042in}{2.267911in}}%
\pgfpathlineto{\pgfqpoint{5.167338in}{2.260628in}}%
\pgfpathlineto{\pgfqpoint{5.159626in}{2.253225in}}%
\pgfpathlineto{\pgfqpoint{5.151905in}{2.245702in}}%
\pgfpathclose%
\pgfusepath{fill}%
\end{pgfscope}%
\begin{pgfscope}%
\pgfpathrectangle{\pgfqpoint{1.150000in}{0.150000in}}{\pgfqpoint{5.700000in}{5.700000in}}%
\pgfusepath{clip}%
\pgfsetbuttcap%
\pgfsetroundjoin%
\definecolor{currentfill}{rgb}{0.274952,0.037752,0.364543}%
\pgfsetfillcolor{currentfill}%
\pgfsetfillopacity{0.700000}%
\pgfsetlinewidth{0.000000pt}%
\definecolor{currentstroke}{rgb}{0.000000,0.000000,0.000000}%
\pgfsetstrokecolor{currentstroke}%
\pgfsetdash{}{0pt}%
\pgfpathmoveto{\pgfqpoint{3.727273in}{1.491277in}}%
\pgfpathlineto{\pgfqpoint{3.741295in}{1.489278in}}%
\pgfpathlineto{\pgfqpoint{3.755323in}{1.487353in}}%
\pgfpathlineto{\pgfqpoint{3.769359in}{1.485503in}}%
\pgfpathlineto{\pgfqpoint{3.783403in}{1.483726in}}%
\pgfpathlineto{\pgfqpoint{3.791659in}{1.494035in}}%
\pgfpathlineto{\pgfqpoint{3.799910in}{1.504382in}}%
\pgfpathlineto{\pgfqpoint{3.808155in}{1.514763in}}%
\pgfpathlineto{\pgfqpoint{3.816394in}{1.525175in}}%
\pgfpathlineto{\pgfqpoint{3.802362in}{1.526683in}}%
\pgfpathlineto{\pgfqpoint{3.788338in}{1.528266in}}%
\pgfpathlineto{\pgfqpoint{3.774321in}{1.529922in}}%
\pgfpathlineto{\pgfqpoint{3.760312in}{1.531654in}}%
\pgfpathlineto{\pgfqpoint{3.752061in}{1.521502in}}%
\pgfpathlineto{\pgfqpoint{3.743805in}{1.511386in}}%
\pgfpathlineto{\pgfqpoint{3.735542in}{1.501310in}}%
\pgfpathlineto{\pgfqpoint{3.727273in}{1.491277in}}%
\pgfpathclose%
\pgfusepath{fill}%
\end{pgfscope}%
\begin{pgfscope}%
\pgfpathrectangle{\pgfqpoint{1.150000in}{0.150000in}}{\pgfqpoint{5.700000in}{5.700000in}}%
\pgfusepath{clip}%
\pgfsetbuttcap%
\pgfsetroundjoin%
\definecolor{currentfill}{rgb}{0.280267,0.073417,0.397163}%
\pgfsetfillcolor{currentfill}%
\pgfsetfillopacity{0.700000}%
\pgfsetlinewidth{0.000000pt}%
\definecolor{currentstroke}{rgb}{0.000000,0.000000,0.000000}%
\pgfsetstrokecolor{currentstroke}%
\pgfsetdash{}{0pt}%
\pgfpathmoveto{\pgfqpoint{2.764228in}{1.585922in}}%
\pgfpathlineto{\pgfqpoint{2.778161in}{1.577266in}}%
\pgfpathlineto{\pgfqpoint{2.792095in}{1.568700in}}%
\pgfpathlineto{\pgfqpoint{2.806031in}{1.560223in}}%
\pgfpathlineto{\pgfqpoint{2.819970in}{1.551836in}}%
\pgfpathlineto{\pgfqpoint{2.828766in}{1.553653in}}%
\pgfpathlineto{\pgfqpoint{2.837547in}{1.555723in}}%
\pgfpathlineto{\pgfqpoint{2.846314in}{1.558040in}}%
\pgfpathlineto{\pgfqpoint{2.855066in}{1.560596in}}%
\pgfpathlineto{\pgfqpoint{2.841160in}{1.568567in}}%
\pgfpathlineto{\pgfqpoint{2.827256in}{1.576628in}}%
\pgfpathlineto{\pgfqpoint{2.813354in}{1.584778in}}%
\pgfpathlineto{\pgfqpoint{2.799454in}{1.593018in}}%
\pgfpathlineto{\pgfqpoint{2.790671in}{1.590870in}}%
\pgfpathlineto{\pgfqpoint{2.781872in}{1.588967in}}%
\pgfpathlineto{\pgfqpoint{2.773058in}{1.587315in}}%
\pgfpathlineto{\pgfqpoint{2.764228in}{1.585922in}}%
\pgfpathclose%
\pgfusepath{fill}%
\end{pgfscope}%
\begin{pgfscope}%
\pgfpathrectangle{\pgfqpoint{1.150000in}{0.150000in}}{\pgfqpoint{5.700000in}{5.700000in}}%
\pgfusepath{clip}%
\pgfsetbuttcap%
\pgfsetroundjoin%
\definecolor{currentfill}{rgb}{0.267004,0.004874,0.329415}%
\pgfsetfillcolor{currentfill}%
\pgfsetfillopacity{0.700000}%
\pgfsetlinewidth{0.000000pt}%
\definecolor{currentstroke}{rgb}{0.000000,0.000000,0.000000}%
\pgfsetstrokecolor{currentstroke}%
\pgfsetdash{}{0pt}%
\pgfpathmoveto{\pgfqpoint{3.403409in}{1.432869in}}%
\pgfpathlineto{\pgfqpoint{3.417373in}{1.428731in}}%
\pgfpathlineto{\pgfqpoint{3.431343in}{1.424671in}}%
\pgfpathlineto{\pgfqpoint{3.445318in}{1.420688in}}%
\pgfpathlineto{\pgfqpoint{3.459300in}{1.416781in}}%
\pgfpathlineto{\pgfqpoint{3.467696in}{1.424944in}}%
\pgfpathlineto{\pgfqpoint{3.476085in}{1.433219in}}%
\pgfpathlineto{\pgfqpoint{3.484465in}{1.441603in}}%
\pgfpathlineto{\pgfqpoint{3.492838in}{1.450089in}}%
\pgfpathlineto{\pgfqpoint{3.478874in}{1.453666in}}%
\pgfpathlineto{\pgfqpoint{3.464916in}{1.457320in}}%
\pgfpathlineto{\pgfqpoint{3.450964in}{1.461052in}}%
\pgfpathlineto{\pgfqpoint{3.437018in}{1.464861in}}%
\pgfpathlineto{\pgfqpoint{3.428627in}{1.456696in}}%
\pgfpathlineto{\pgfqpoint{3.420229in}{1.448639in}}%
\pgfpathlineto{\pgfqpoint{3.411823in}{1.440695in}}%
\pgfpathlineto{\pgfqpoint{3.403409in}{1.432869in}}%
\pgfpathclose%
\pgfusepath{fill}%
\end{pgfscope}%
\begin{pgfscope}%
\pgfpathrectangle{\pgfqpoint{1.150000in}{0.150000in}}{\pgfqpoint{5.700000in}{5.700000in}}%
\pgfusepath{clip}%
\pgfsetbuttcap%
\pgfsetroundjoin%
\definecolor{currentfill}{rgb}{0.273809,0.031497,0.358853}%
\pgfsetfillcolor{currentfill}%
\pgfsetfillopacity{0.700000}%
\pgfsetlinewidth{0.000000pt}%
\definecolor{currentstroke}{rgb}{0.000000,0.000000,0.000000}%
\pgfsetstrokecolor{currentstroke}%
\pgfsetdash{}{0pt}%
\pgfpathmoveto{\pgfqpoint{2.966407in}{1.499970in}}%
\pgfpathlineto{\pgfqpoint{2.980337in}{1.492779in}}%
\pgfpathlineto{\pgfqpoint{2.994271in}{1.485673in}}%
\pgfpathlineto{\pgfqpoint{3.008207in}{1.478651in}}%
\pgfpathlineto{\pgfqpoint{3.022147in}{1.471713in}}%
\pgfpathlineto{\pgfqpoint{3.030798in}{1.475707in}}%
\pgfpathlineto{\pgfqpoint{3.039436in}{1.479911in}}%
\pgfpathlineto{\pgfqpoint{3.048062in}{1.484320in}}%
\pgfpathlineto{\pgfqpoint{3.056676in}{1.488928in}}%
\pgfpathlineto{\pgfqpoint{3.042763in}{1.495473in}}%
\pgfpathlineto{\pgfqpoint{3.028854in}{1.502102in}}%
\pgfpathlineto{\pgfqpoint{3.014948in}{1.508816in}}%
\pgfpathlineto{\pgfqpoint{3.001046in}{1.515614in}}%
\pgfpathlineto{\pgfqpoint{2.992405in}{1.511390in}}%
\pgfpathlineto{\pgfqpoint{2.983752in}{1.507371in}}%
\pgfpathlineto{\pgfqpoint{2.975086in}{1.503563in}}%
\pgfpathlineto{\pgfqpoint{2.966407in}{1.499970in}}%
\pgfpathclose%
\pgfusepath{fill}%
\end{pgfscope}%
\begin{pgfscope}%
\pgfpathrectangle{\pgfqpoint{1.150000in}{0.150000in}}{\pgfqpoint{5.700000in}{5.700000in}}%
\pgfusepath{clip}%
\pgfsetbuttcap%
\pgfsetroundjoin%
\definecolor{currentfill}{rgb}{0.271305,0.019942,0.347269}%
\pgfsetfillcolor{currentfill}%
\pgfsetfillopacity{0.700000}%
\pgfsetlinewidth{0.000000pt}%
\definecolor{currentstroke}{rgb}{0.000000,0.000000,0.000000}%
\pgfsetstrokecolor{currentstroke}%
\pgfsetdash{}{0pt}%
\pgfpathmoveto{\pgfqpoint{3.638070in}{1.461558in}}%
\pgfpathlineto{\pgfqpoint{3.652076in}{1.458972in}}%
\pgfpathlineto{\pgfqpoint{3.666089in}{1.456461in}}%
\pgfpathlineto{\pgfqpoint{3.680109in}{1.454024in}}%
\pgfpathlineto{\pgfqpoint{3.694136in}{1.451663in}}%
\pgfpathlineto{\pgfqpoint{3.702430in}{1.461480in}}%
\pgfpathlineto{\pgfqpoint{3.710717in}{1.471358in}}%
\pgfpathlineto{\pgfqpoint{3.718999in}{1.481292in}}%
\pgfpathlineto{\pgfqpoint{3.727273in}{1.491277in}}%
\pgfpathlineto{\pgfqpoint{3.713259in}{1.493350in}}%
\pgfpathlineto{\pgfqpoint{3.699252in}{1.495498in}}%
\pgfpathlineto{\pgfqpoint{3.685252in}{1.497721in}}%
\pgfpathlineto{\pgfqpoint{3.671260in}{1.500019in}}%
\pgfpathlineto{\pgfqpoint{3.662972in}{1.490315in}}%
\pgfpathlineto{\pgfqpoint{3.654678in}{1.480667in}}%
\pgfpathlineto{\pgfqpoint{3.646377in}{1.471080in}}%
\pgfpathlineto{\pgfqpoint{3.638070in}{1.461558in}}%
\pgfpathclose%
\pgfusepath{fill}%
\end{pgfscope}%
\begin{pgfscope}%
\pgfpathrectangle{\pgfqpoint{1.150000in}{0.150000in}}{\pgfqpoint{5.700000in}{5.700000in}}%
\pgfusepath{clip}%
\pgfsetbuttcap%
\pgfsetroundjoin%
\definecolor{currentfill}{rgb}{0.268510,0.009605,0.335427}%
\pgfsetfillcolor{currentfill}%
\pgfsetfillopacity{0.700000}%
\pgfsetlinewidth{0.000000pt}%
\definecolor{currentstroke}{rgb}{0.000000,0.000000,0.000000}%
\pgfsetstrokecolor{currentstroke}%
\pgfsetdash{}{0pt}%
\pgfpathmoveto{\pgfqpoint{3.548755in}{1.436543in}}%
\pgfpathlineto{\pgfqpoint{3.562750in}{1.433347in}}%
\pgfpathlineto{\pgfqpoint{3.576751in}{1.430227in}}%
\pgfpathlineto{\pgfqpoint{3.590759in}{1.427181in}}%
\pgfpathlineto{\pgfqpoint{3.604773in}{1.424211in}}%
\pgfpathlineto{\pgfqpoint{3.613107in}{1.433428in}}%
\pgfpathlineto{\pgfqpoint{3.621435in}{1.442728in}}%
\pgfpathlineto{\pgfqpoint{3.629756in}{1.452106in}}%
\pgfpathlineto{\pgfqpoint{3.638070in}{1.461558in}}%
\pgfpathlineto{\pgfqpoint{3.624070in}{1.464220in}}%
\pgfpathlineto{\pgfqpoint{3.610077in}{1.466956in}}%
\pgfpathlineto{\pgfqpoint{3.596091in}{1.469768in}}%
\pgfpathlineto{\pgfqpoint{3.582111in}{1.472656in}}%
\pgfpathlineto{\pgfqpoint{3.573783in}{1.463504in}}%
\pgfpathlineto{\pgfqpoint{3.565447in}{1.454432in}}%
\pgfpathlineto{\pgfqpoint{3.557105in}{1.445444in}}%
\pgfpathlineto{\pgfqpoint{3.548755in}{1.436543in}}%
\pgfpathclose%
\pgfusepath{fill}%
\end{pgfscope}%
\begin{pgfscope}%
\pgfpathrectangle{\pgfqpoint{1.150000in}{0.150000in}}{\pgfqpoint{5.700000in}{5.700000in}}%
\pgfusepath{clip}%
\pgfsetbuttcap%
\pgfsetroundjoin%
\definecolor{currentfill}{rgb}{0.283072,0.130895,0.449241}%
\pgfsetfillcolor{currentfill}%
\pgfsetfillopacity{0.700000}%
\pgfsetlinewidth{0.000000pt}%
\definecolor{currentstroke}{rgb}{0.000000,0.000000,0.000000}%
\pgfsetstrokecolor{currentstroke}%
\pgfsetdash{}{0pt}%
\pgfpathmoveto{\pgfqpoint{4.139972in}{1.647988in}}%
\pgfpathlineto{\pgfqpoint{4.154119in}{1.648404in}}%
\pgfpathlineto{\pgfqpoint{4.168276in}{1.648892in}}%
\pgfpathlineto{\pgfqpoint{4.182441in}{1.649453in}}%
\pgfpathlineto{\pgfqpoint{4.196617in}{1.650086in}}%
\pgfpathlineto{\pgfqpoint{4.204740in}{1.661557in}}%
\pgfpathlineto{\pgfqpoint{4.212857in}{1.672988in}}%
\pgfpathlineto{\pgfqpoint{4.220969in}{1.684376in}}%
\pgfpathlineto{\pgfqpoint{4.229076in}{1.695720in}}%
\pgfpathlineto{\pgfqpoint{4.214908in}{1.694900in}}%
\pgfpathlineto{\pgfqpoint{4.200750in}{1.694153in}}%
\pgfpathlineto{\pgfqpoint{4.186601in}{1.693477in}}%
\pgfpathlineto{\pgfqpoint{4.172461in}{1.692875in}}%
\pgfpathlineto{\pgfqpoint{4.164347in}{1.681710in}}%
\pgfpathlineto{\pgfqpoint{4.156227in}{1.670506in}}%
\pgfpathlineto{\pgfqpoint{4.148102in}{1.659264in}}%
\pgfpathlineto{\pgfqpoint{4.139972in}{1.647988in}}%
\pgfpathclose%
\pgfusepath{fill}%
\end{pgfscope}%
\begin{pgfscope}%
\pgfpathrectangle{\pgfqpoint{1.150000in}{0.150000in}}{\pgfqpoint{5.700000in}{5.700000in}}%
\pgfusepath{clip}%
\pgfsetbuttcap%
\pgfsetroundjoin%
\definecolor{currentfill}{rgb}{0.281887,0.150881,0.465405}%
\pgfsetfillcolor{currentfill}%
\pgfsetfillopacity{0.700000}%
\pgfsetlinewidth{0.000000pt}%
\definecolor{currentstroke}{rgb}{0.000000,0.000000,0.000000}%
\pgfsetstrokecolor{currentstroke}%
\pgfsetdash{}{0pt}%
\pgfpathmoveto{\pgfqpoint{4.229076in}{1.695720in}}%
\pgfpathlineto{\pgfqpoint{4.243253in}{1.696612in}}%
\pgfpathlineto{\pgfqpoint{4.257441in}{1.697575in}}%
\pgfpathlineto{\pgfqpoint{4.271637in}{1.698611in}}%
\pgfpathlineto{\pgfqpoint{4.285844in}{1.699719in}}%
\pgfpathlineto{\pgfqpoint{4.293938in}{1.711189in}}%
\pgfpathlineto{\pgfqpoint{4.302027in}{1.722603in}}%
\pgfpathlineto{\pgfqpoint{4.310111in}{1.733961in}}%
\pgfpathlineto{\pgfqpoint{4.318189in}{1.745259in}}%
\pgfpathlineto{\pgfqpoint{4.303989in}{1.743986in}}%
\pgfpathlineto{\pgfqpoint{4.289799in}{1.742784in}}%
\pgfpathlineto{\pgfqpoint{4.275619in}{1.741654in}}%
\pgfpathlineto{\pgfqpoint{4.261449in}{1.740596in}}%
\pgfpathlineto{\pgfqpoint{4.253364in}{1.729456in}}%
\pgfpathlineto{\pgfqpoint{4.245273in}{1.718262in}}%
\pgfpathlineto{\pgfqpoint{4.237177in}{1.707016in}}%
\pgfpathlineto{\pgfqpoint{4.229076in}{1.695720in}}%
\pgfpathclose%
\pgfusepath{fill}%
\end{pgfscope}%
\begin{pgfscope}%
\pgfpathrectangle{\pgfqpoint{1.150000in}{0.150000in}}{\pgfqpoint{5.700000in}{5.700000in}}%
\pgfusepath{clip}%
\pgfsetbuttcap%
\pgfsetroundjoin%
\definecolor{currentfill}{rgb}{0.282910,0.105393,0.426902}%
\pgfsetfillcolor{currentfill}%
\pgfsetfillopacity{0.700000}%
\pgfsetlinewidth{0.000000pt}%
\definecolor{currentstroke}{rgb}{0.000000,0.000000,0.000000}%
\pgfsetstrokecolor{currentstroke}%
\pgfsetdash{}{0pt}%
\pgfpathmoveto{\pgfqpoint{4.050866in}{1.602477in}}%
\pgfpathlineto{\pgfqpoint{4.064986in}{1.602396in}}%
\pgfpathlineto{\pgfqpoint{4.079114in}{1.602388in}}%
\pgfpathlineto{\pgfqpoint{4.093251in}{1.602452in}}%
\pgfpathlineto{\pgfqpoint{4.107397in}{1.602589in}}%
\pgfpathlineto{\pgfqpoint{4.115549in}{1.613977in}}%
\pgfpathlineto{\pgfqpoint{4.123695in}{1.625341in}}%
\pgfpathlineto{\pgfqpoint{4.131836in}{1.636679in}}%
\pgfpathlineto{\pgfqpoint{4.139972in}{1.647988in}}%
\pgfpathlineto{\pgfqpoint{4.125834in}{1.647644in}}%
\pgfpathlineto{\pgfqpoint{4.111705in}{1.647372in}}%
\pgfpathlineto{\pgfqpoint{4.097585in}{1.647173in}}%
\pgfpathlineto{\pgfqpoint{4.083474in}{1.647047in}}%
\pgfpathlineto{\pgfqpoint{4.075330in}{1.635938in}}%
\pgfpathlineto{\pgfqpoint{4.067181in}{1.624805in}}%
\pgfpathlineto{\pgfqpoint{4.059026in}{1.613650in}}%
\pgfpathlineto{\pgfqpoint{4.050866in}{1.602477in}}%
\pgfpathclose%
\pgfusepath{fill}%
\end{pgfscope}%
\begin{pgfscope}%
\pgfpathrectangle{\pgfqpoint{1.150000in}{0.150000in}}{\pgfqpoint{5.700000in}{5.700000in}}%
\pgfusepath{clip}%
\pgfsetbuttcap%
\pgfsetroundjoin%
\definecolor{currentfill}{rgb}{0.278791,0.062145,0.386592}%
\pgfsetfillcolor{currentfill}%
\pgfsetfillopacity{0.700000}%
\pgfsetlinewidth{0.000000pt}%
\definecolor{currentstroke}{rgb}{0.000000,0.000000,0.000000}%
\pgfsetstrokecolor{currentstroke}%
\pgfsetdash{}{0pt}%
\pgfpathmoveto{\pgfqpoint{2.819970in}{1.551836in}}%
\pgfpathlineto{\pgfqpoint{2.833910in}{1.543537in}}%
\pgfpathlineto{\pgfqpoint{2.847853in}{1.535326in}}%
\pgfpathlineto{\pgfqpoint{2.861797in}{1.527203in}}%
\pgfpathlineto{\pgfqpoint{2.875745in}{1.519167in}}%
\pgfpathlineto{\pgfqpoint{2.884509in}{1.521408in}}%
\pgfpathlineto{\pgfqpoint{2.893258in}{1.523897in}}%
\pgfpathlineto{\pgfqpoint{2.901994in}{1.526626in}}%
\pgfpathlineto{\pgfqpoint{2.910715in}{1.529589in}}%
\pgfpathlineto{\pgfqpoint{2.896799in}{1.537209in}}%
\pgfpathlineto{\pgfqpoint{2.882886in}{1.544917in}}%
\pgfpathlineto{\pgfqpoint{2.868975in}{1.552712in}}%
\pgfpathlineto{\pgfqpoint{2.855066in}{1.560596in}}%
\pgfpathlineto{\pgfqpoint{2.846314in}{1.558040in}}%
\pgfpathlineto{\pgfqpoint{2.837547in}{1.555723in}}%
\pgfpathlineto{\pgfqpoint{2.828766in}{1.553653in}}%
\pgfpathlineto{\pgfqpoint{2.819970in}{1.551836in}}%
\pgfpathclose%
\pgfusepath{fill}%
\end{pgfscope}%
\begin{pgfscope}%
\pgfpathrectangle{\pgfqpoint{1.150000in}{0.150000in}}{\pgfqpoint{5.700000in}{5.700000in}}%
\pgfusepath{clip}%
\pgfsetbuttcap%
\pgfsetroundjoin%
\definecolor{currentfill}{rgb}{0.268510,0.009605,0.335427}%
\pgfsetfillcolor{currentfill}%
\pgfsetfillopacity{0.700000}%
\pgfsetlinewidth{0.000000pt}%
\definecolor{currentstroke}{rgb}{0.000000,0.000000,0.000000}%
\pgfsetstrokecolor{currentstroke}%
\pgfsetdash{}{0pt}%
\pgfpathmoveto{\pgfqpoint{3.168113in}{1.439544in}}%
\pgfpathlineto{\pgfqpoint{3.182060in}{1.433738in}}%
\pgfpathlineto{\pgfqpoint{3.196012in}{1.428012in}}%
\pgfpathlineto{\pgfqpoint{3.209968in}{1.422367in}}%
\pgfpathlineto{\pgfqpoint{3.223929in}{1.416802in}}%
\pgfpathlineto{\pgfqpoint{3.232458in}{1.422740in}}%
\pgfpathlineto{\pgfqpoint{3.240977in}{1.428848in}}%
\pgfpathlineto{\pgfqpoint{3.249487in}{1.435122in}}%
\pgfpathlineto{\pgfqpoint{3.257986in}{1.441556in}}%
\pgfpathlineto{\pgfqpoint{3.244048in}{1.446751in}}%
\pgfpathlineto{\pgfqpoint{3.230114in}{1.452025in}}%
\pgfpathlineto{\pgfqpoint{3.216185in}{1.457380in}}%
\pgfpathlineto{\pgfqpoint{3.202261in}{1.462815in}}%
\pgfpathlineto{\pgfqpoint{3.193739in}{1.456743in}}%
\pgfpathlineto{\pgfqpoint{3.185208in}{1.450837in}}%
\pgfpathlineto{\pgfqpoint{3.176665in}{1.445102in}}%
\pgfpathlineto{\pgfqpoint{3.168113in}{1.439544in}}%
\pgfpathclose%
\pgfusepath{fill}%
\end{pgfscope}%
\begin{pgfscope}%
\pgfpathrectangle{\pgfqpoint{1.150000in}{0.150000in}}{\pgfqpoint{5.700000in}{5.700000in}}%
\pgfusepath{clip}%
\pgfsetbuttcap%
\pgfsetroundjoin%
\definecolor{currentfill}{rgb}{0.278826,0.175490,0.483397}%
\pgfsetfillcolor{currentfill}%
\pgfsetfillopacity{0.700000}%
\pgfsetlinewidth{0.000000pt}%
\definecolor{currentstroke}{rgb}{0.000000,0.000000,0.000000}%
\pgfsetstrokecolor{currentstroke}%
\pgfsetdash{}{0pt}%
\pgfpathmoveto{\pgfqpoint{4.318189in}{1.745259in}}%
\pgfpathlineto{\pgfqpoint{4.332398in}{1.746605in}}%
\pgfpathlineto{\pgfqpoint{4.346618in}{1.748023in}}%
\pgfpathlineto{\pgfqpoint{4.360847in}{1.749512in}}%
\pgfpathlineto{\pgfqpoint{4.375087in}{1.751073in}}%
\pgfpathlineto{\pgfqpoint{4.383153in}{1.762463in}}%
\pgfpathlineto{\pgfqpoint{4.391213in}{1.773784in}}%
\pgfpathlineto{\pgfqpoint{4.399268in}{1.785035in}}%
\pgfpathlineto{\pgfqpoint{4.407316in}{1.796214in}}%
\pgfpathlineto{\pgfqpoint{4.393083in}{1.794507in}}%
\pgfpathlineto{\pgfqpoint{4.378860in}{1.792872in}}%
\pgfpathlineto{\pgfqpoint{4.364648in}{1.791310in}}%
\pgfpathlineto{\pgfqpoint{4.350445in}{1.789819in}}%
\pgfpathlineto{\pgfqpoint{4.342389in}{1.778778in}}%
\pgfpathlineto{\pgfqpoint{4.334328in}{1.767670in}}%
\pgfpathlineto{\pgfqpoint{4.326261in}{1.756496in}}%
\pgfpathlineto{\pgfqpoint{4.318189in}{1.745259in}}%
\pgfpathclose%
\pgfusepath{fill}%
\end{pgfscope}%
\begin{pgfscope}%
\pgfpathrectangle{\pgfqpoint{1.150000in}{0.150000in}}{\pgfqpoint{5.700000in}{5.700000in}}%
\pgfusepath{clip}%
\pgfsetbuttcap%
\pgfsetroundjoin%
\definecolor{currentfill}{rgb}{0.281446,0.084320,0.407414}%
\pgfsetfillcolor{currentfill}%
\pgfsetfillopacity{0.700000}%
\pgfsetlinewidth{0.000000pt}%
\definecolor{currentstroke}{rgb}{0.000000,0.000000,0.000000}%
\pgfsetstrokecolor{currentstroke}%
\pgfsetdash{}{0pt}%
\pgfpathmoveto{\pgfqpoint{3.961747in}{1.559622in}}%
\pgfpathlineto{\pgfqpoint{3.975841in}{1.559023in}}%
\pgfpathlineto{\pgfqpoint{3.989943in}{1.558496in}}%
\pgfpathlineto{\pgfqpoint{4.004054in}{1.558042in}}%
\pgfpathlineto{\pgfqpoint{4.018173in}{1.557661in}}%
\pgfpathlineto{\pgfqpoint{4.026354in}{1.568877in}}%
\pgfpathlineto{\pgfqpoint{4.034530in}{1.580087in}}%
\pgfpathlineto{\pgfqpoint{4.042701in}{1.591288in}}%
\pgfpathlineto{\pgfqpoint{4.050866in}{1.602477in}}%
\pgfpathlineto{\pgfqpoint{4.036756in}{1.602630in}}%
\pgfpathlineto{\pgfqpoint{4.022654in}{1.602857in}}%
\pgfpathlineto{\pgfqpoint{4.008561in}{1.603156in}}%
\pgfpathlineto{\pgfqpoint{3.994476in}{1.603528in}}%
\pgfpathlineto{\pgfqpoint{3.986302in}{1.592559in}}%
\pgfpathlineto{\pgfqpoint{3.978123in}{1.581583in}}%
\pgfpathlineto{\pgfqpoint{3.969938in}{1.570603in}}%
\pgfpathlineto{\pgfqpoint{3.961747in}{1.559622in}}%
\pgfpathclose%
\pgfusepath{fill}%
\end{pgfscope}%
\begin{pgfscope}%
\pgfpathrectangle{\pgfqpoint{1.150000in}{0.150000in}}{\pgfqpoint{5.700000in}{5.700000in}}%
\pgfusepath{clip}%
\pgfsetbuttcap%
\pgfsetroundjoin%
\definecolor{currentfill}{rgb}{0.267004,0.004874,0.329415}%
\pgfsetfillcolor{currentfill}%
\pgfsetfillopacity{0.700000}%
\pgfsetlinewidth{0.000000pt}%
\definecolor{currentstroke}{rgb}{0.000000,0.000000,0.000000}%
\pgfsetstrokecolor{currentstroke}%
\pgfsetdash{}{0pt}%
\pgfpathmoveto{\pgfqpoint{3.313786in}{1.421572in}}%
\pgfpathlineto{\pgfqpoint{3.327749in}{1.416773in}}%
\pgfpathlineto{\pgfqpoint{3.341716in}{1.412053in}}%
\pgfpathlineto{\pgfqpoint{3.355689in}{1.407410in}}%
\pgfpathlineto{\pgfqpoint{3.369667in}{1.402845in}}%
\pgfpathlineto{\pgfqpoint{3.378115in}{1.410148in}}%
\pgfpathlineto{\pgfqpoint{3.386555in}{1.417590in}}%
\pgfpathlineto{\pgfqpoint{3.394986in}{1.425166in}}%
\pgfpathlineto{\pgfqpoint{3.403409in}{1.432869in}}%
\pgfpathlineto{\pgfqpoint{3.389450in}{1.437084in}}%
\pgfpathlineto{\pgfqpoint{3.375497in}{1.441377in}}%
\pgfpathlineto{\pgfqpoint{3.361549in}{1.445748in}}%
\pgfpathlineto{\pgfqpoint{3.347607in}{1.450197in}}%
\pgfpathlineto{\pgfqpoint{3.339165in}{1.442836in}}%
\pgfpathlineto{\pgfqpoint{3.330714in}{1.435608in}}%
\pgfpathlineto{\pgfqpoint{3.322255in}{1.428518in}}%
\pgfpathlineto{\pgfqpoint{3.313786in}{1.421572in}}%
\pgfpathclose%
\pgfusepath{fill}%
\end{pgfscope}%
\begin{pgfscope}%
\pgfpathrectangle{\pgfqpoint{1.150000in}{0.150000in}}{\pgfqpoint{5.700000in}{5.700000in}}%
\pgfusepath{clip}%
\pgfsetbuttcap%
\pgfsetroundjoin%
\definecolor{currentfill}{rgb}{0.274128,0.199721,0.498911}%
\pgfsetfillcolor{currentfill}%
\pgfsetfillopacity{0.700000}%
\pgfsetlinewidth{0.000000pt}%
\definecolor{currentstroke}{rgb}{0.000000,0.000000,0.000000}%
\pgfsetstrokecolor{currentstroke}%
\pgfsetdash{}{0pt}%
\pgfpathmoveto{\pgfqpoint{4.407316in}{1.796214in}}%
\pgfpathlineto{\pgfqpoint{4.421560in}{1.797992in}}%
\pgfpathlineto{\pgfqpoint{4.435814in}{1.799842in}}%
\pgfpathlineto{\pgfqpoint{4.450078in}{1.801763in}}%
\pgfpathlineto{\pgfqpoint{4.464353in}{1.803756in}}%
\pgfpathlineto{\pgfqpoint{4.472389in}{1.814993in}}%
\pgfpathlineto{\pgfqpoint{4.480420in}{1.826148in}}%
\pgfpathlineto{\pgfqpoint{4.488444in}{1.837221in}}%
\pgfpathlineto{\pgfqpoint{4.496463in}{1.848210in}}%
\pgfpathlineto{\pgfqpoint{4.482195in}{1.846092in}}%
\pgfpathlineto{\pgfqpoint{4.467937in}{1.844046in}}%
\pgfpathlineto{\pgfqpoint{4.453690in}{1.842072in}}%
\pgfpathlineto{\pgfqpoint{4.439454in}{1.840170in}}%
\pgfpathlineto{\pgfqpoint{4.431428in}{1.829297in}}%
\pgfpathlineto{\pgfqpoint{4.423397in}{1.818346in}}%
\pgfpathlineto{\pgfqpoint{4.415359in}{1.807318in}}%
\pgfpathlineto{\pgfqpoint{4.407316in}{1.796214in}}%
\pgfpathclose%
\pgfusepath{fill}%
\end{pgfscope}%
\begin{pgfscope}%
\pgfpathrectangle{\pgfqpoint{1.150000in}{0.150000in}}{\pgfqpoint{5.700000in}{5.700000in}}%
\pgfusepath{clip}%
\pgfsetbuttcap%
\pgfsetroundjoin%
\definecolor{currentfill}{rgb}{0.279566,0.067836,0.391917}%
\pgfsetfillcolor{currentfill}%
\pgfsetfillopacity{0.700000}%
\pgfsetlinewidth{0.000000pt}%
\definecolor{currentstroke}{rgb}{0.000000,0.000000,0.000000}%
\pgfsetstrokecolor{currentstroke}%
\pgfsetdash{}{0pt}%
\pgfpathmoveto{\pgfqpoint{3.872599in}{1.519880in}}%
\pgfpathlineto{\pgfqpoint{3.886670in}{1.518739in}}%
\pgfpathlineto{\pgfqpoint{3.900749in}{1.517673in}}%
\pgfpathlineto{\pgfqpoint{3.914836in}{1.516679in}}%
\pgfpathlineto{\pgfqpoint{3.928931in}{1.515758in}}%
\pgfpathlineto{\pgfqpoint{3.937143in}{1.526708in}}%
\pgfpathlineto{\pgfqpoint{3.945350in}{1.537672in}}%
\pgfpathlineto{\pgfqpoint{3.953551in}{1.548644in}}%
\pgfpathlineto{\pgfqpoint{3.961747in}{1.559622in}}%
\pgfpathlineto{\pgfqpoint{3.947662in}{1.560295in}}%
\pgfpathlineto{\pgfqpoint{3.933585in}{1.561040in}}%
\pgfpathlineto{\pgfqpoint{3.919516in}{1.561859in}}%
\pgfpathlineto{\pgfqpoint{3.905455in}{1.562751in}}%
\pgfpathlineto{\pgfqpoint{3.897249in}{1.552013in}}%
\pgfpathlineto{\pgfqpoint{3.889038in}{1.541286in}}%
\pgfpathlineto{\pgfqpoint{3.880821in}{1.530574in}}%
\pgfpathlineto{\pgfqpoint{3.872599in}{1.519880in}}%
\pgfpathclose%
\pgfusepath{fill}%
\end{pgfscope}%
\begin{pgfscope}%
\pgfpathrectangle{\pgfqpoint{1.150000in}{0.150000in}}{\pgfqpoint{5.700000in}{5.700000in}}%
\pgfusepath{clip}%
\pgfsetbuttcap%
\pgfsetroundjoin%
\definecolor{currentfill}{rgb}{0.266580,0.228262,0.514349}%
\pgfsetfillcolor{currentfill}%
\pgfsetfillopacity{0.700000}%
\pgfsetlinewidth{0.000000pt}%
\definecolor{currentstroke}{rgb}{0.000000,0.000000,0.000000}%
\pgfsetstrokecolor{currentstroke}%
\pgfsetdash{}{0pt}%
\pgfpathmoveto{\pgfqpoint{4.496463in}{1.848210in}}%
\pgfpathlineto{\pgfqpoint{4.510742in}{1.850399in}}%
\pgfpathlineto{\pgfqpoint{4.525032in}{1.852660in}}%
\pgfpathlineto{\pgfqpoint{4.539332in}{1.854992in}}%
\pgfpathlineto{\pgfqpoint{4.553643in}{1.857395in}}%
\pgfpathlineto{\pgfqpoint{4.561650in}{1.868410in}}%
\pgfpathlineto{\pgfqpoint{4.569650in}{1.879332in}}%
\pgfpathlineto{\pgfqpoint{4.577644in}{1.890161in}}%
\pgfpathlineto{\pgfqpoint{4.585631in}{1.900896in}}%
\pgfpathlineto{\pgfqpoint{4.571327in}{1.898389in}}%
\pgfpathlineto{\pgfqpoint{4.557033in}{1.895953in}}%
\pgfpathlineto{\pgfqpoint{4.542750in}{1.893589in}}%
\pgfpathlineto{\pgfqpoint{4.528479in}{1.891297in}}%
\pgfpathlineto{\pgfqpoint{4.520484in}{1.880658in}}%
\pgfpathlineto{\pgfqpoint{4.512483in}{1.869929in}}%
\pgfpathlineto{\pgfqpoint{4.504476in}{1.859113in}}%
\pgfpathlineto{\pgfqpoint{4.496463in}{1.848210in}}%
\pgfpathclose%
\pgfusepath{fill}%
\end{pgfscope}%
\begin{pgfscope}%
\pgfpathrectangle{\pgfqpoint{1.150000in}{0.150000in}}{\pgfqpoint{5.700000in}{5.700000in}}%
\pgfusepath{clip}%
\pgfsetbuttcap%
\pgfsetroundjoin%
\definecolor{currentfill}{rgb}{0.272594,0.025563,0.353093}%
\pgfsetfillcolor{currentfill}%
\pgfsetfillopacity{0.700000}%
\pgfsetlinewidth{0.000000pt}%
\definecolor{currentstroke}{rgb}{0.000000,0.000000,0.000000}%
\pgfsetstrokecolor{currentstroke}%
\pgfsetdash{}{0pt}%
\pgfpathmoveto{\pgfqpoint{3.022147in}{1.471713in}}%
\pgfpathlineto{\pgfqpoint{3.036090in}{1.464859in}}%
\pgfpathlineto{\pgfqpoint{3.050037in}{1.458088in}}%
\pgfpathlineto{\pgfqpoint{3.063987in}{1.451400in}}%
\pgfpathlineto{\pgfqpoint{3.077940in}{1.444794in}}%
\pgfpathlineto{\pgfqpoint{3.086564in}{1.449188in}}%
\pgfpathlineto{\pgfqpoint{3.095175in}{1.453787in}}%
\pgfpathlineto{\pgfqpoint{3.103775in}{1.458586in}}%
\pgfpathlineto{\pgfqpoint{3.112363in}{1.463579in}}%
\pgfpathlineto{\pgfqpoint{3.098436in}{1.469792in}}%
\pgfpathlineto{\pgfqpoint{3.084512in}{1.476088in}}%
\pgfpathlineto{\pgfqpoint{3.070592in}{1.482466in}}%
\pgfpathlineto{\pgfqpoint{3.056676in}{1.488928in}}%
\pgfpathlineto{\pgfqpoint{3.048062in}{1.484320in}}%
\pgfpathlineto{\pgfqpoint{3.039436in}{1.479911in}}%
\pgfpathlineto{\pgfqpoint{3.030798in}{1.475707in}}%
\pgfpathlineto{\pgfqpoint{3.022147in}{1.471713in}}%
\pgfpathclose%
\pgfusepath{fill}%
\end{pgfscope}%
\begin{pgfscope}%
\pgfpathrectangle{\pgfqpoint{1.150000in}{0.150000in}}{\pgfqpoint{5.700000in}{5.700000in}}%
\pgfusepath{clip}%
\pgfsetbuttcap%
\pgfsetroundjoin%
\definecolor{currentfill}{rgb}{0.258965,0.251537,0.524736}%
\pgfsetfillcolor{currentfill}%
\pgfsetfillopacity{0.700000}%
\pgfsetlinewidth{0.000000pt}%
\definecolor{currentstroke}{rgb}{0.000000,0.000000,0.000000}%
\pgfsetstrokecolor{currentstroke}%
\pgfsetdash{}{0pt}%
\pgfpathmoveto{\pgfqpoint{4.585631in}{1.900896in}}%
\pgfpathlineto{\pgfqpoint{4.599947in}{1.903474in}}%
\pgfpathlineto{\pgfqpoint{4.614274in}{1.906124in}}%
\pgfpathlineto{\pgfqpoint{4.628611in}{1.908846in}}%
\pgfpathlineto{\pgfqpoint{4.642960in}{1.911638in}}%
\pgfpathlineto{\pgfqpoint{4.650935in}{1.922367in}}%
\pgfpathlineto{\pgfqpoint{4.658903in}{1.932995in}}%
\pgfpathlineto{\pgfqpoint{4.666865in}{1.943519in}}%
\pgfpathlineto{\pgfqpoint{4.674820in}{1.953940in}}%
\pgfpathlineto{\pgfqpoint{4.660478in}{1.951065in}}%
\pgfpathlineto{\pgfqpoint{4.646147in}{1.948262in}}%
\pgfpathlineto{\pgfqpoint{4.631828in}{1.945530in}}%
\pgfpathlineto{\pgfqpoint{4.617520in}{1.942869in}}%
\pgfpathlineto{\pgfqpoint{4.609557in}{1.932522in}}%
\pgfpathlineto{\pgfqpoint{4.601588in}{1.922077in}}%
\pgfpathlineto{\pgfqpoint{4.593613in}{1.911535in}}%
\pgfpathlineto{\pgfqpoint{4.585631in}{1.900896in}}%
\pgfpathclose%
\pgfusepath{fill}%
\end{pgfscope}%
\begin{pgfscope}%
\pgfpathrectangle{\pgfqpoint{1.150000in}{0.150000in}}{\pgfqpoint{5.700000in}{5.700000in}}%
\pgfusepath{clip}%
\pgfsetbuttcap%
\pgfsetroundjoin%
\definecolor{currentfill}{rgb}{0.276022,0.044167,0.370164}%
\pgfsetfillcolor{currentfill}%
\pgfsetfillopacity{0.700000}%
\pgfsetlinewidth{0.000000pt}%
\definecolor{currentstroke}{rgb}{0.000000,0.000000,0.000000}%
\pgfsetstrokecolor{currentstroke}%
\pgfsetdash{}{0pt}%
\pgfpathmoveto{\pgfqpoint{3.783403in}{1.483726in}}%
\pgfpathlineto{\pgfqpoint{3.797454in}{1.482024in}}%
\pgfpathlineto{\pgfqpoint{3.811512in}{1.480395in}}%
\pgfpathlineto{\pgfqpoint{3.825578in}{1.478839in}}%
\pgfpathlineto{\pgfqpoint{3.839652in}{1.477357in}}%
\pgfpathlineto{\pgfqpoint{3.847897in}{1.487942in}}%
\pgfpathlineto{\pgfqpoint{3.856137in}{1.498560in}}%
\pgfpathlineto{\pgfqpoint{3.864371in}{1.509207in}}%
\pgfpathlineto{\pgfqpoint{3.872599in}{1.519880in}}%
\pgfpathlineto{\pgfqpoint{3.858536in}{1.521093in}}%
\pgfpathlineto{\pgfqpoint{3.844481in}{1.522380in}}%
\pgfpathlineto{\pgfqpoint{3.830434in}{1.523740in}}%
\pgfpathlineto{\pgfqpoint{3.816394in}{1.525175in}}%
\pgfpathlineto{\pgfqpoint{3.808155in}{1.514763in}}%
\pgfpathlineto{\pgfqpoint{3.799910in}{1.504382in}}%
\pgfpathlineto{\pgfqpoint{3.791659in}{1.494035in}}%
\pgfpathlineto{\pgfqpoint{3.783403in}{1.483726in}}%
\pgfpathclose%
\pgfusepath{fill}%
\end{pgfscope}%
\begin{pgfscope}%
\pgfpathrectangle{\pgfqpoint{1.150000in}{0.150000in}}{\pgfqpoint{5.700000in}{5.700000in}}%
\pgfusepath{clip}%
\pgfsetbuttcap%
\pgfsetroundjoin%
\definecolor{currentfill}{rgb}{0.250425,0.274290,0.533103}%
\pgfsetfillcolor{currentfill}%
\pgfsetfillopacity{0.700000}%
\pgfsetlinewidth{0.000000pt}%
\definecolor{currentstroke}{rgb}{0.000000,0.000000,0.000000}%
\pgfsetstrokecolor{currentstroke}%
\pgfsetdash{}{0pt}%
\pgfpathmoveto{\pgfqpoint{4.674820in}{1.953940in}}%
\pgfpathlineto{\pgfqpoint{4.689174in}{1.956886in}}%
\pgfpathlineto{\pgfqpoint{4.703538in}{1.959904in}}%
\pgfpathlineto{\pgfqpoint{4.717915in}{1.962993in}}%
\pgfpathlineto{\pgfqpoint{4.732303in}{1.966153in}}%
\pgfpathlineto{\pgfqpoint{4.740244in}{1.976539in}}%
\pgfpathlineto{\pgfqpoint{4.748179in}{1.986814in}}%
\pgfpathlineto{\pgfqpoint{4.756107in}{1.996978in}}%
\pgfpathlineto{\pgfqpoint{4.764028in}{2.007031in}}%
\pgfpathlineto{\pgfqpoint{4.749647in}{2.003810in}}%
\pgfpathlineto{\pgfqpoint{4.735278in}{2.000661in}}%
\pgfpathlineto{\pgfqpoint{4.720921in}{1.997583in}}%
\pgfpathlineto{\pgfqpoint{4.706575in}{1.994576in}}%
\pgfpathlineto{\pgfqpoint{4.698647in}{1.984575in}}%
\pgfpathlineto{\pgfqpoint{4.690711in}{1.974468in}}%
\pgfpathlineto{\pgfqpoint{4.682769in}{1.964257in}}%
\pgfpathlineto{\pgfqpoint{4.674820in}{1.953940in}}%
\pgfpathclose%
\pgfusepath{fill}%
\end{pgfscope}%
\begin{pgfscope}%
\pgfpathrectangle{\pgfqpoint{1.150000in}{0.150000in}}{\pgfqpoint{5.700000in}{5.700000in}}%
\pgfusepath{clip}%
\pgfsetbuttcap%
\pgfsetroundjoin%
\definecolor{currentfill}{rgb}{0.267004,0.004874,0.329415}%
\pgfsetfillcolor{currentfill}%
\pgfsetfillopacity{0.700000}%
\pgfsetlinewidth{0.000000pt}%
\definecolor{currentstroke}{rgb}{0.000000,0.000000,0.000000}%
\pgfsetstrokecolor{currentstroke}%
\pgfsetdash{}{0pt}%
\pgfpathmoveto{\pgfqpoint{3.459300in}{1.416781in}}%
\pgfpathlineto{\pgfqpoint{3.473287in}{1.412951in}}%
\pgfpathlineto{\pgfqpoint{3.487279in}{1.409198in}}%
\pgfpathlineto{\pgfqpoint{3.501278in}{1.405520in}}%
\pgfpathlineto{\pgfqpoint{3.515283in}{1.401919in}}%
\pgfpathlineto{\pgfqpoint{3.523662in}{1.410419in}}%
\pgfpathlineto{\pgfqpoint{3.532034in}{1.419026in}}%
\pgfpathlineto{\pgfqpoint{3.540398in}{1.427736in}}%
\pgfpathlineto{\pgfqpoint{3.548755in}{1.436543in}}%
\pgfpathlineto{\pgfqpoint{3.534767in}{1.439816in}}%
\pgfpathlineto{\pgfqpoint{3.520785in}{1.443164in}}%
\pgfpathlineto{\pgfqpoint{3.506809in}{1.446588in}}%
\pgfpathlineto{\pgfqpoint{3.492838in}{1.450089in}}%
\pgfpathlineto{\pgfqpoint{3.484465in}{1.441603in}}%
\pgfpathlineto{\pgfqpoint{3.476085in}{1.433219in}}%
\pgfpathlineto{\pgfqpoint{3.467696in}{1.424944in}}%
\pgfpathlineto{\pgfqpoint{3.459300in}{1.416781in}}%
\pgfpathclose%
\pgfusepath{fill}%
\end{pgfscope}%
\begin{pgfscope}%
\pgfpathrectangle{\pgfqpoint{1.150000in}{0.150000in}}{\pgfqpoint{5.700000in}{5.700000in}}%
\pgfusepath{clip}%
\pgfsetbuttcap%
\pgfsetroundjoin%
\definecolor{currentfill}{rgb}{0.174274,0.445044,0.557792}%
\pgfsetfillcolor{currentfill}%
\pgfsetfillopacity{0.700000}%
\pgfsetlinewidth{0.000000pt}%
\definecolor{currentstroke}{rgb}{0.000000,0.000000,0.000000}%
\pgfsetstrokecolor{currentstroke}%
\pgfsetdash{}{0pt}%
\pgfpathmoveto{\pgfqpoint{5.477567in}{2.403022in}}%
\pgfpathlineto{\pgfqpoint{5.492280in}{2.408288in}}%
\pgfpathlineto{\pgfqpoint{5.507006in}{2.413625in}}%
\pgfpathlineto{\pgfqpoint{5.521747in}{2.419033in}}%
\pgfpathlineto{\pgfqpoint{5.529291in}{2.424620in}}%
\pgfpathlineto{\pgfqpoint{5.536825in}{2.430086in}}%
\pgfpathlineto{\pgfqpoint{5.544349in}{2.435433in}}%
\pgfpathlineto{\pgfqpoint{5.551864in}{2.440663in}}%
\pgfpathlineto{\pgfqpoint{5.537141in}{2.435394in}}%
\pgfpathlineto{\pgfqpoint{5.522431in}{2.430195in}}%
\pgfpathlineto{\pgfqpoint{5.507735in}{2.425067in}}%
\pgfpathlineto{\pgfqpoint{5.500208in}{2.419727in}}%
\pgfpathlineto{\pgfqpoint{5.492670in}{2.414275in}}%
\pgfpathlineto{\pgfqpoint{5.485124in}{2.408707in}}%
\pgfpathlineto{\pgfqpoint{5.477567in}{2.403022in}}%
\pgfpathclose%
\pgfusepath{fill}%
\end{pgfscope}%
\begin{pgfscope}%
\pgfpathrectangle{\pgfqpoint{1.150000in}{0.150000in}}{\pgfqpoint{5.700000in}{5.700000in}}%
\pgfusepath{clip}%
\pgfsetbuttcap%
\pgfsetroundjoin%
\definecolor{currentfill}{rgb}{0.241237,0.296485,0.539709}%
\pgfsetfillcolor{currentfill}%
\pgfsetfillopacity{0.700000}%
\pgfsetlinewidth{0.000000pt}%
\definecolor{currentstroke}{rgb}{0.000000,0.000000,0.000000}%
\pgfsetstrokecolor{currentstroke}%
\pgfsetdash{}{0pt}%
\pgfpathmoveto{\pgfqpoint{4.764028in}{2.007031in}}%
\pgfpathlineto{\pgfqpoint{4.778420in}{2.010324in}}%
\pgfpathlineto{\pgfqpoint{4.792824in}{2.013687in}}%
\pgfpathlineto{\pgfqpoint{4.807239in}{2.017122in}}%
\pgfpathlineto{\pgfqpoint{4.821667in}{2.020628in}}%
\pgfpathlineto{\pgfqpoint{4.829573in}{2.030617in}}%
\pgfpathlineto{\pgfqpoint{4.837473in}{2.040489in}}%
\pgfpathlineto{\pgfqpoint{4.845365in}{2.050243in}}%
\pgfpathlineto{\pgfqpoint{4.853250in}{2.059879in}}%
\pgfpathlineto{\pgfqpoint{4.838830in}{2.056334in}}%
\pgfpathlineto{\pgfqpoint{4.824422in}{2.052860in}}%
\pgfpathlineto{\pgfqpoint{4.810026in}{2.049458in}}%
\pgfpathlineto{\pgfqpoint{4.795642in}{2.046126in}}%
\pgfpathlineto{\pgfqpoint{4.787749in}{2.036521in}}%
\pgfpathlineto{\pgfqpoint{4.779849in}{2.026803in}}%
\pgfpathlineto{\pgfqpoint{4.771942in}{2.016973in}}%
\pgfpathlineto{\pgfqpoint{4.764028in}{2.007031in}}%
\pgfpathclose%
\pgfusepath{fill}%
\end{pgfscope}%
\begin{pgfscope}%
\pgfpathrectangle{\pgfqpoint{1.150000in}{0.150000in}}{\pgfqpoint{5.700000in}{5.700000in}}%
\pgfusepath{clip}%
\pgfsetbuttcap%
\pgfsetroundjoin%
\definecolor{currentfill}{rgb}{0.231674,0.318106,0.544834}%
\pgfsetfillcolor{currentfill}%
\pgfsetfillopacity{0.700000}%
\pgfsetlinewidth{0.000000pt}%
\definecolor{currentstroke}{rgb}{0.000000,0.000000,0.000000}%
\pgfsetstrokecolor{currentstroke}%
\pgfsetdash{}{0pt}%
\pgfpathmoveto{\pgfqpoint{4.853250in}{2.059879in}}%
\pgfpathlineto{\pgfqpoint{4.867681in}{2.063496in}}%
\pgfpathlineto{\pgfqpoint{4.882125in}{2.067184in}}%
\pgfpathlineto{\pgfqpoint{4.896581in}{2.070943in}}%
\pgfpathlineto{\pgfqpoint{4.911048in}{2.074773in}}%
\pgfpathlineto{\pgfqpoint{4.918918in}{2.084318in}}%
\pgfpathlineto{\pgfqpoint{4.926779in}{2.093739in}}%
\pgfpathlineto{\pgfqpoint{4.934634in}{2.103038in}}%
\pgfpathlineto{\pgfqpoint{4.942480in}{2.112215in}}%
\pgfpathlineto{\pgfqpoint{4.928021in}{2.108367in}}%
\pgfpathlineto{\pgfqpoint{4.913574in}{2.104591in}}%
\pgfpathlineto{\pgfqpoint{4.899139in}{2.100886in}}%
\pgfpathlineto{\pgfqpoint{4.884716in}{2.097252in}}%
\pgfpathlineto{\pgfqpoint{4.876860in}{2.088085in}}%
\pgfpathlineto{\pgfqpoint{4.868997in}{2.078800in}}%
\pgfpathlineto{\pgfqpoint{4.861127in}{2.069399in}}%
\pgfpathlineto{\pgfqpoint{4.853250in}{2.059879in}}%
\pgfpathclose%
\pgfusepath{fill}%
\end{pgfscope}%
\begin{pgfscope}%
\pgfpathrectangle{\pgfqpoint{1.150000in}{0.150000in}}{\pgfqpoint{5.700000in}{5.700000in}}%
\pgfusepath{clip}%
\pgfsetbuttcap%
\pgfsetroundjoin%
\definecolor{currentfill}{rgb}{0.273809,0.031497,0.358853}%
\pgfsetfillcolor{currentfill}%
\pgfsetfillopacity{0.700000}%
\pgfsetlinewidth{0.000000pt}%
\definecolor{currentstroke}{rgb}{0.000000,0.000000,0.000000}%
\pgfsetstrokecolor{currentstroke}%
\pgfsetdash{}{0pt}%
\pgfpathmoveto{\pgfqpoint{3.694136in}{1.451663in}}%
\pgfpathlineto{\pgfqpoint{3.708170in}{1.449375in}}%
\pgfpathlineto{\pgfqpoint{3.722211in}{1.447162in}}%
\pgfpathlineto{\pgfqpoint{3.736259in}{1.445023in}}%
\pgfpathlineto{\pgfqpoint{3.750315in}{1.442957in}}%
\pgfpathlineto{\pgfqpoint{3.758596in}{1.453072in}}%
\pgfpathlineto{\pgfqpoint{3.766871in}{1.463241in}}%
\pgfpathlineto{\pgfqpoint{3.775140in}{1.473460in}}%
\pgfpathlineto{\pgfqpoint{3.783403in}{1.483726in}}%
\pgfpathlineto{\pgfqpoint{3.769359in}{1.485503in}}%
\pgfpathlineto{\pgfqpoint{3.755323in}{1.487353in}}%
\pgfpathlineto{\pgfqpoint{3.741295in}{1.489278in}}%
\pgfpathlineto{\pgfqpoint{3.727273in}{1.491277in}}%
\pgfpathlineto{\pgfqpoint{3.718999in}{1.481292in}}%
\pgfpathlineto{\pgfqpoint{3.710717in}{1.471358in}}%
\pgfpathlineto{\pgfqpoint{3.702430in}{1.461480in}}%
\pgfpathlineto{\pgfqpoint{3.694136in}{1.451663in}}%
\pgfpathclose%
\pgfusepath{fill}%
\end{pgfscope}%
\begin{pgfscope}%
\pgfpathrectangle{\pgfqpoint{1.150000in}{0.150000in}}{\pgfqpoint{5.700000in}{5.700000in}}%
\pgfusepath{clip}%
\pgfsetbuttcap%
\pgfsetroundjoin%
\definecolor{currentfill}{rgb}{0.221989,0.339161,0.548752}%
\pgfsetfillcolor{currentfill}%
\pgfsetfillopacity{0.700000}%
\pgfsetlinewidth{0.000000pt}%
\definecolor{currentstroke}{rgb}{0.000000,0.000000,0.000000}%
\pgfsetstrokecolor{currentstroke}%
\pgfsetdash{}{0pt}%
\pgfpathmoveto{\pgfqpoint{4.942480in}{2.112215in}}%
\pgfpathlineto{\pgfqpoint{4.956952in}{2.116134in}}%
\pgfpathlineto{\pgfqpoint{4.971436in}{2.120124in}}%
\pgfpathlineto{\pgfqpoint{4.985932in}{2.124185in}}%
\pgfpathlineto{\pgfqpoint{5.000441in}{2.128318in}}%
\pgfpathlineto{\pgfqpoint{5.008270in}{2.137375in}}%
\pgfpathlineto{\pgfqpoint{5.016092in}{2.146306in}}%
\pgfpathlineto{\pgfqpoint{5.023906in}{2.155110in}}%
\pgfpathlineto{\pgfqpoint{5.031712in}{2.163788in}}%
\pgfpathlineto{\pgfqpoint{5.017213in}{2.159661in}}%
\pgfpathlineto{\pgfqpoint{5.002726in}{2.155604in}}%
\pgfpathlineto{\pgfqpoint{4.988252in}{2.151619in}}%
\pgfpathlineto{\pgfqpoint{4.973789in}{2.147704in}}%
\pgfpathlineto{\pgfqpoint{4.965974in}{2.139013in}}%
\pgfpathlineto{\pgfqpoint{4.958150in}{2.130202in}}%
\pgfpathlineto{\pgfqpoint{4.950319in}{2.121269in}}%
\pgfpathlineto{\pgfqpoint{4.942480in}{2.112215in}}%
\pgfpathclose%
\pgfusepath{fill}%
\end{pgfscope}%
\begin{pgfscope}%
\pgfpathrectangle{\pgfqpoint{1.150000in}{0.150000in}}{\pgfqpoint{5.700000in}{5.700000in}}%
\pgfusepath{clip}%
\pgfsetbuttcap%
\pgfsetroundjoin%
\definecolor{currentfill}{rgb}{0.277941,0.056324,0.381191}%
\pgfsetfillcolor{currentfill}%
\pgfsetfillopacity{0.700000}%
\pgfsetlinewidth{0.000000pt}%
\definecolor{currentstroke}{rgb}{0.000000,0.000000,0.000000}%
\pgfsetstrokecolor{currentstroke}%
\pgfsetdash{}{0pt}%
\pgfpathmoveto{\pgfqpoint{2.875745in}{1.519167in}}%
\pgfpathlineto{\pgfqpoint{2.889694in}{1.511218in}}%
\pgfpathlineto{\pgfqpoint{2.903646in}{1.503356in}}%
\pgfpathlineto{\pgfqpoint{2.917601in}{1.495580in}}%
\pgfpathlineto{\pgfqpoint{2.931558in}{1.487889in}}%
\pgfpathlineto{\pgfqpoint{2.940291in}{1.490553in}}%
\pgfpathlineto{\pgfqpoint{2.949010in}{1.493459in}}%
\pgfpathlineto{\pgfqpoint{2.957715in}{1.496600in}}%
\pgfpathlineto{\pgfqpoint{2.966407in}{1.499970in}}%
\pgfpathlineto{\pgfqpoint{2.952480in}{1.507246in}}%
\pgfpathlineto{\pgfqpoint{2.938555in}{1.514608in}}%
\pgfpathlineto{\pgfqpoint{2.924634in}{1.522055in}}%
\pgfpathlineto{\pgfqpoint{2.910715in}{1.529589in}}%
\pgfpathlineto{\pgfqpoint{2.901994in}{1.526626in}}%
\pgfpathlineto{\pgfqpoint{2.893258in}{1.523897in}}%
\pgfpathlineto{\pgfqpoint{2.884509in}{1.521408in}}%
\pgfpathlineto{\pgfqpoint{2.875745in}{1.519167in}}%
\pgfpathclose%
\pgfusepath{fill}%
\end{pgfscope}%
\begin{pgfscope}%
\pgfpathrectangle{\pgfqpoint{1.150000in}{0.150000in}}{\pgfqpoint{5.700000in}{5.700000in}}%
\pgfusepath{clip}%
\pgfsetbuttcap%
\pgfsetroundjoin%
\definecolor{currentfill}{rgb}{0.212395,0.359683,0.551710}%
\pgfsetfillcolor{currentfill}%
\pgfsetfillopacity{0.700000}%
\pgfsetlinewidth{0.000000pt}%
\definecolor{currentstroke}{rgb}{0.000000,0.000000,0.000000}%
\pgfsetstrokecolor{currentstroke}%
\pgfsetdash{}{0pt}%
\pgfpathmoveto{\pgfqpoint{5.031712in}{2.163788in}}%
\pgfpathlineto{\pgfqpoint{5.046224in}{2.167987in}}%
\pgfpathlineto{\pgfqpoint{5.060748in}{2.172258in}}%
\pgfpathlineto{\pgfqpoint{5.075286in}{2.176599in}}%
\pgfpathlineto{\pgfqpoint{5.089836in}{2.181012in}}%
\pgfpathlineto{\pgfqpoint{5.097624in}{2.189546in}}%
\pgfpathlineto{\pgfqpoint{5.105403in}{2.197951in}}%
\pgfpathlineto{\pgfqpoint{5.113174in}{2.206226in}}%
\pgfpathlineto{\pgfqpoint{5.120937in}{2.214373in}}%
\pgfpathlineto{\pgfqpoint{5.106398in}{2.209987in}}%
\pgfpathlineto{\pgfqpoint{5.091871in}{2.205672in}}%
\pgfpathlineto{\pgfqpoint{5.077357in}{2.201429in}}%
\pgfpathlineto{\pgfqpoint{5.062855in}{2.197256in}}%
\pgfpathlineto{\pgfqpoint{5.055082in}{2.189074in}}%
\pgfpathlineto{\pgfqpoint{5.047300in}{2.180770in}}%
\pgfpathlineto{\pgfqpoint{5.039510in}{2.172341in}}%
\pgfpathlineto{\pgfqpoint{5.031712in}{2.163788in}}%
\pgfpathclose%
\pgfusepath{fill}%
\end{pgfscope}%
\begin{pgfscope}%
\pgfpathrectangle{\pgfqpoint{1.150000in}{0.150000in}}{\pgfqpoint{5.700000in}{5.700000in}}%
\pgfusepath{clip}%
\pgfsetbuttcap%
\pgfsetroundjoin%
\definecolor{currentfill}{rgb}{0.179019,0.433756,0.557430}%
\pgfsetfillcolor{currentfill}%
\pgfsetfillopacity{0.700000}%
\pgfsetlinewidth{0.000000pt}%
\definecolor{currentstroke}{rgb}{0.000000,0.000000,0.000000}%
\pgfsetstrokecolor{currentstroke}%
\pgfsetdash{}{0pt}%
\pgfpathmoveto{\pgfqpoint{5.388472in}{2.358239in}}%
\pgfpathlineto{\pgfqpoint{5.403145in}{2.363336in}}%
\pgfpathlineto{\pgfqpoint{5.417832in}{2.368504in}}%
\pgfpathlineto{\pgfqpoint{5.432533in}{2.373744in}}%
\pgfpathlineto{\pgfqpoint{5.447247in}{2.379055in}}%
\pgfpathlineto{\pgfqpoint{5.454841in}{2.385235in}}%
\pgfpathlineto{\pgfqpoint{5.462426in}{2.391289in}}%
\pgfpathlineto{\pgfqpoint{5.470002in}{2.397217in}}%
\pgfpathlineto{\pgfqpoint{5.477567in}{2.403022in}}%
\pgfpathlineto{\pgfqpoint{5.462869in}{2.397827in}}%
\pgfpathlineto{\pgfqpoint{5.448184in}{2.392703in}}%
\pgfpathlineto{\pgfqpoint{5.433512in}{2.387651in}}%
\pgfpathlineto{\pgfqpoint{5.418854in}{2.382669in}}%
\pgfpathlineto{\pgfqpoint{5.411273in}{2.376739in}}%
\pgfpathlineto{\pgfqpoint{5.403682in}{2.370693in}}%
\pgfpathlineto{\pgfqpoint{5.396082in}{2.364527in}}%
\pgfpathlineto{\pgfqpoint{5.388472in}{2.358239in}}%
\pgfpathclose%
\pgfusepath{fill}%
\end{pgfscope}%
\begin{pgfscope}%
\pgfpathrectangle{\pgfqpoint{1.150000in}{0.150000in}}{\pgfqpoint{5.700000in}{5.700000in}}%
\pgfusepath{clip}%
\pgfsetbuttcap%
\pgfsetroundjoin%
\definecolor{currentfill}{rgb}{0.203063,0.379716,0.553925}%
\pgfsetfillcolor{currentfill}%
\pgfsetfillopacity{0.700000}%
\pgfsetlinewidth{0.000000pt}%
\definecolor{currentstroke}{rgb}{0.000000,0.000000,0.000000}%
\pgfsetstrokecolor{currentstroke}%
\pgfsetdash{}{0pt}%
\pgfpathmoveto{\pgfqpoint{5.120937in}{2.214373in}}%
\pgfpathlineto{\pgfqpoint{5.135490in}{2.218830in}}%
\pgfpathlineto{\pgfqpoint{5.150055in}{2.223358in}}%
\pgfpathlineto{\pgfqpoint{5.164633in}{2.227958in}}%
\pgfpathlineto{\pgfqpoint{5.179224in}{2.232629in}}%
\pgfpathlineto{\pgfqpoint{5.186968in}{2.240608in}}%
\pgfpathlineto{\pgfqpoint{5.194702in}{2.248456in}}%
\pgfpathlineto{\pgfqpoint{5.202429in}{2.256174in}}%
\pgfpathlineto{\pgfqpoint{5.210146in}{2.263762in}}%
\pgfpathlineto{\pgfqpoint{5.195566in}{2.259140in}}%
\pgfpathlineto{\pgfqpoint{5.180999in}{2.254590in}}%
\pgfpathlineto{\pgfqpoint{5.166446in}{2.250110in}}%
\pgfpathlineto{\pgfqpoint{5.151905in}{2.245702in}}%
\pgfpathlineto{\pgfqpoint{5.144175in}{2.238056in}}%
\pgfpathlineto{\pgfqpoint{5.136438in}{2.230287in}}%
\pgfpathlineto{\pgfqpoint{5.128692in}{2.222393in}}%
\pgfpathlineto{\pgfqpoint{5.120937in}{2.214373in}}%
\pgfpathclose%
\pgfusepath{fill}%
\end{pgfscope}%
\begin{pgfscope}%
\pgfpathrectangle{\pgfqpoint{1.150000in}{0.150000in}}{\pgfqpoint{5.700000in}{5.700000in}}%
\pgfusepath{clip}%
\pgfsetbuttcap%
\pgfsetroundjoin%
\definecolor{currentfill}{rgb}{0.187231,0.414746,0.556547}%
\pgfsetfillcolor{currentfill}%
\pgfsetfillopacity{0.700000}%
\pgfsetlinewidth{0.000000pt}%
\definecolor{currentstroke}{rgb}{0.000000,0.000000,0.000000}%
\pgfsetstrokecolor{currentstroke}%
\pgfsetdash{}{0pt}%
\pgfpathmoveto{\pgfqpoint{5.299328in}{2.311772in}}%
\pgfpathlineto{\pgfqpoint{5.313961in}{2.316678in}}%
\pgfpathlineto{\pgfqpoint{5.328608in}{2.321656in}}%
\pgfpathlineto{\pgfqpoint{5.343268in}{2.326704in}}%
\pgfpathlineto{\pgfqpoint{5.357941in}{2.331824in}}%
\pgfpathlineto{\pgfqpoint{5.365588in}{2.338621in}}%
\pgfpathlineto{\pgfqpoint{5.373225in}{2.345288in}}%
\pgfpathlineto{\pgfqpoint{5.380853in}{2.351826in}}%
\pgfpathlineto{\pgfqpoint{5.388472in}{2.358239in}}%
\pgfpathlineto{\pgfqpoint{5.373813in}{2.353212in}}%
\pgfpathlineto{\pgfqpoint{5.359167in}{2.348257in}}%
\pgfpathlineto{\pgfqpoint{5.344534in}{2.343373in}}%
\pgfpathlineto{\pgfqpoint{5.329915in}{2.338559in}}%
\pgfpathlineto{\pgfqpoint{5.322282in}{2.332046in}}%
\pgfpathlineto{\pgfqpoint{5.314639in}{2.325411in}}%
\pgfpathlineto{\pgfqpoint{5.306988in}{2.318654in}}%
\pgfpathlineto{\pgfqpoint{5.299328in}{2.311772in}}%
\pgfpathclose%
\pgfusepath{fill}%
\end{pgfscope}%
\begin{pgfscope}%
\pgfpathrectangle{\pgfqpoint{1.150000in}{0.150000in}}{\pgfqpoint{5.700000in}{5.700000in}}%
\pgfusepath{clip}%
\pgfsetbuttcap%
\pgfsetroundjoin%
\definecolor{currentfill}{rgb}{0.194100,0.399323,0.555565}%
\pgfsetfillcolor{currentfill}%
\pgfsetfillopacity{0.700000}%
\pgfsetlinewidth{0.000000pt}%
\definecolor{currentstroke}{rgb}{0.000000,0.000000,0.000000}%
\pgfsetstrokecolor{currentstroke}%
\pgfsetdash{}{0pt}%
\pgfpathmoveto{\pgfqpoint{5.210146in}{2.263762in}}%
\pgfpathlineto{\pgfqpoint{5.224739in}{2.268455in}}%
\pgfpathlineto{\pgfqpoint{5.239345in}{2.273219in}}%
\pgfpathlineto{\pgfqpoint{5.253964in}{2.278055in}}%
\pgfpathlineto{\pgfqpoint{5.268597in}{2.282961in}}%
\pgfpathlineto{\pgfqpoint{5.276293in}{2.290360in}}%
\pgfpathlineto{\pgfqpoint{5.283980in}{2.297626in}}%
\pgfpathlineto{\pgfqpoint{5.291659in}{2.304763in}}%
\pgfpathlineto{\pgfqpoint{5.299328in}{2.311772in}}%
\pgfpathlineto{\pgfqpoint{5.284708in}{2.306936in}}%
\pgfpathlineto{\pgfqpoint{5.270102in}{2.302172in}}%
\pgfpathlineto{\pgfqpoint{5.255508in}{2.297479in}}%
\pgfpathlineto{\pgfqpoint{5.240928in}{2.292857in}}%
\pgfpathlineto{\pgfqpoint{5.233246in}{2.285769in}}%
\pgfpathlineto{\pgfqpoint{5.225554in}{2.278558in}}%
\pgfpathlineto{\pgfqpoint{5.217854in}{2.271223in}}%
\pgfpathlineto{\pgfqpoint{5.210146in}{2.263762in}}%
\pgfpathclose%
\pgfusepath{fill}%
\end{pgfscope}%
\begin{pgfscope}%
\pgfpathrectangle{\pgfqpoint{1.150000in}{0.150000in}}{\pgfqpoint{5.700000in}{5.700000in}}%
\pgfusepath{clip}%
\pgfsetbuttcap%
\pgfsetroundjoin%
\definecolor{currentfill}{rgb}{0.267004,0.004874,0.329415}%
\pgfsetfillcolor{currentfill}%
\pgfsetfillopacity{0.700000}%
\pgfsetlinewidth{0.000000pt}%
\definecolor{currentstroke}{rgb}{0.000000,0.000000,0.000000}%
\pgfsetstrokecolor{currentstroke}%
\pgfsetdash{}{0pt}%
\pgfpathmoveto{\pgfqpoint{3.223929in}{1.416802in}}%
\pgfpathlineto{\pgfqpoint{3.237894in}{1.411317in}}%
\pgfpathlineto{\pgfqpoint{3.251864in}{1.405911in}}%
\pgfpathlineto{\pgfqpoint{3.265838in}{1.400583in}}%
\pgfpathlineto{\pgfqpoint{3.279817in}{1.395335in}}%
\pgfpathlineto{\pgfqpoint{3.288324in}{1.401651in}}%
\pgfpathlineto{\pgfqpoint{3.296821in}{1.408133in}}%
\pgfpathlineto{\pgfqpoint{3.305308in}{1.414775in}}%
\pgfpathlineto{\pgfqpoint{3.313786in}{1.421572in}}%
\pgfpathlineto{\pgfqpoint{3.299829in}{1.426450in}}%
\pgfpathlineto{\pgfqpoint{3.285876in}{1.431406in}}%
\pgfpathlineto{\pgfqpoint{3.271929in}{1.436442in}}%
\pgfpathlineto{\pgfqpoint{3.257986in}{1.441556in}}%
\pgfpathlineto{\pgfqpoint{3.249487in}{1.435122in}}%
\pgfpathlineto{\pgfqpoint{3.240977in}{1.428848in}}%
\pgfpathlineto{\pgfqpoint{3.232458in}{1.422740in}}%
\pgfpathlineto{\pgfqpoint{3.223929in}{1.416802in}}%
\pgfpathclose%
\pgfusepath{fill}%
\end{pgfscope}%
\begin{pgfscope}%
\pgfpathrectangle{\pgfqpoint{1.150000in}{0.150000in}}{\pgfqpoint{5.700000in}{5.700000in}}%
\pgfusepath{clip}%
\pgfsetbuttcap%
\pgfsetroundjoin%
\definecolor{currentfill}{rgb}{0.271305,0.019942,0.347269}%
\pgfsetfillcolor{currentfill}%
\pgfsetfillopacity{0.700000}%
\pgfsetlinewidth{0.000000pt}%
\definecolor{currentstroke}{rgb}{0.000000,0.000000,0.000000}%
\pgfsetstrokecolor{currentstroke}%
\pgfsetdash{}{0pt}%
\pgfpathmoveto{\pgfqpoint{3.077940in}{1.444794in}}%
\pgfpathlineto{\pgfqpoint{3.091897in}{1.438271in}}%
\pgfpathlineto{\pgfqpoint{3.105858in}{1.431829in}}%
\pgfpathlineto{\pgfqpoint{3.119823in}{1.425469in}}%
\pgfpathlineto{\pgfqpoint{3.133791in}{1.419190in}}%
\pgfpathlineto{\pgfqpoint{3.142389in}{1.423984in}}%
\pgfpathlineto{\pgfqpoint{3.150975in}{1.428979in}}%
\pgfpathlineto{\pgfqpoint{3.159549in}{1.434167in}}%
\pgfpathlineto{\pgfqpoint{3.168113in}{1.439544in}}%
\pgfpathlineto{\pgfqpoint{3.154169in}{1.445430in}}%
\pgfpathlineto{\pgfqpoint{3.140230in}{1.451398in}}%
\pgfpathlineto{\pgfqpoint{3.126295in}{1.457447in}}%
\pgfpathlineto{\pgfqpoint{3.112363in}{1.463579in}}%
\pgfpathlineto{\pgfqpoint{3.103775in}{1.458586in}}%
\pgfpathlineto{\pgfqpoint{3.095175in}{1.453787in}}%
\pgfpathlineto{\pgfqpoint{3.086564in}{1.449188in}}%
\pgfpathlineto{\pgfqpoint{3.077940in}{1.444794in}}%
\pgfpathclose%
\pgfusepath{fill}%
\end{pgfscope}%
\begin{pgfscope}%
\pgfpathrectangle{\pgfqpoint{1.150000in}{0.150000in}}{\pgfqpoint{5.700000in}{5.700000in}}%
\pgfusepath{clip}%
\pgfsetbuttcap%
\pgfsetroundjoin%
\definecolor{currentfill}{rgb}{0.269944,0.014625,0.341379}%
\pgfsetfillcolor{currentfill}%
\pgfsetfillopacity{0.700000}%
\pgfsetlinewidth{0.000000pt}%
\definecolor{currentstroke}{rgb}{0.000000,0.000000,0.000000}%
\pgfsetstrokecolor{currentstroke}%
\pgfsetdash{}{0pt}%
\pgfpathmoveto{\pgfqpoint{3.604773in}{1.424211in}}%
\pgfpathlineto{\pgfqpoint{3.618793in}{1.421316in}}%
\pgfpathlineto{\pgfqpoint{3.632821in}{1.418496in}}%
\pgfpathlineto{\pgfqpoint{3.646855in}{1.415750in}}%
\pgfpathlineto{\pgfqpoint{3.660895in}{1.413079in}}%
\pgfpathlineto{\pgfqpoint{3.669215in}{1.422613in}}%
\pgfpathlineto{\pgfqpoint{3.677529in}{1.432224in}}%
\pgfpathlineto{\pgfqpoint{3.685836in}{1.441909in}}%
\pgfpathlineto{\pgfqpoint{3.694136in}{1.451663in}}%
\pgfpathlineto{\pgfqpoint{3.680109in}{1.454024in}}%
\pgfpathlineto{\pgfqpoint{3.666089in}{1.456461in}}%
\pgfpathlineto{\pgfqpoint{3.652076in}{1.458972in}}%
\pgfpathlineto{\pgfqpoint{3.638070in}{1.461558in}}%
\pgfpathlineto{\pgfqpoint{3.629756in}{1.452106in}}%
\pgfpathlineto{\pgfqpoint{3.621435in}{1.442728in}}%
\pgfpathlineto{\pgfqpoint{3.613107in}{1.433428in}}%
\pgfpathlineto{\pgfqpoint{3.604773in}{1.424211in}}%
\pgfpathclose%
\pgfusepath{fill}%
\end{pgfscope}%
\begin{pgfscope}%
\pgfpathrectangle{\pgfqpoint{1.150000in}{0.150000in}}{\pgfqpoint{5.700000in}{5.700000in}}%
\pgfusepath{clip}%
\pgfsetbuttcap%
\pgfsetroundjoin%
\definecolor{currentfill}{rgb}{0.267004,0.004874,0.329415}%
\pgfsetfillcolor{currentfill}%
\pgfsetfillopacity{0.700000}%
\pgfsetlinewidth{0.000000pt}%
\definecolor{currentstroke}{rgb}{0.000000,0.000000,0.000000}%
\pgfsetstrokecolor{currentstroke}%
\pgfsetdash{}{0pt}%
\pgfpathmoveto{\pgfqpoint{3.369667in}{1.402845in}}%
\pgfpathlineto{\pgfqpoint{3.383650in}{1.398357in}}%
\pgfpathlineto{\pgfqpoint{3.397639in}{1.393947in}}%
\pgfpathlineto{\pgfqpoint{3.411633in}{1.389614in}}%
\pgfpathlineto{\pgfqpoint{3.425632in}{1.385357in}}%
\pgfpathlineto{\pgfqpoint{3.434062in}{1.393019in}}%
\pgfpathlineto{\pgfqpoint{3.442483in}{1.400813in}}%
\pgfpathlineto{\pgfqpoint{3.450895in}{1.408736in}}%
\pgfpathlineto{\pgfqpoint{3.459300in}{1.416781in}}%
\pgfpathlineto{\pgfqpoint{3.445318in}{1.420688in}}%
\pgfpathlineto{\pgfqpoint{3.431343in}{1.424671in}}%
\pgfpathlineto{\pgfqpoint{3.417373in}{1.428731in}}%
\pgfpathlineto{\pgfqpoint{3.403409in}{1.432869in}}%
\pgfpathlineto{\pgfqpoint{3.394986in}{1.425166in}}%
\pgfpathlineto{\pgfqpoint{3.386555in}{1.417590in}}%
\pgfpathlineto{\pgfqpoint{3.378115in}{1.410148in}}%
\pgfpathlineto{\pgfqpoint{3.369667in}{1.402845in}}%
\pgfpathclose%
\pgfusepath{fill}%
\end{pgfscope}%
\begin{pgfscope}%
\pgfpathrectangle{\pgfqpoint{1.150000in}{0.150000in}}{\pgfqpoint{5.700000in}{5.700000in}}%
\pgfusepath{clip}%
\pgfsetbuttcap%
\pgfsetroundjoin%
\definecolor{currentfill}{rgb}{0.283229,0.120777,0.440584}%
\pgfsetfillcolor{currentfill}%
\pgfsetfillopacity{0.700000}%
\pgfsetlinewidth{0.000000pt}%
\definecolor{currentstroke}{rgb}{0.000000,0.000000,0.000000}%
\pgfsetstrokecolor{currentstroke}%
\pgfsetdash{}{0pt}%
\pgfpathmoveto{\pgfqpoint{4.107397in}{1.602589in}}%
\pgfpathlineto{\pgfqpoint{4.121552in}{1.602797in}}%
\pgfpathlineto{\pgfqpoint{4.135716in}{1.603078in}}%
\pgfpathlineto{\pgfqpoint{4.149889in}{1.603431in}}%
\pgfpathlineto{\pgfqpoint{4.164072in}{1.603856in}}%
\pgfpathlineto{\pgfqpoint{4.172216in}{1.615460in}}%
\pgfpathlineto{\pgfqpoint{4.180355in}{1.627035in}}%
\pgfpathlineto{\pgfqpoint{4.188488in}{1.638578in}}%
\pgfpathlineto{\pgfqpoint{4.196617in}{1.650086in}}%
\pgfpathlineto{\pgfqpoint{4.182441in}{1.649453in}}%
\pgfpathlineto{\pgfqpoint{4.168276in}{1.648892in}}%
\pgfpathlineto{\pgfqpoint{4.154119in}{1.648404in}}%
\pgfpathlineto{\pgfqpoint{4.139972in}{1.647988in}}%
\pgfpathlineto{\pgfqpoint{4.131836in}{1.636679in}}%
\pgfpathlineto{\pgfqpoint{4.123695in}{1.625341in}}%
\pgfpathlineto{\pgfqpoint{4.115549in}{1.613977in}}%
\pgfpathlineto{\pgfqpoint{4.107397in}{1.602589in}}%
\pgfpathclose%
\pgfusepath{fill}%
\end{pgfscope}%
\begin{pgfscope}%
\pgfpathrectangle{\pgfqpoint{1.150000in}{0.150000in}}{\pgfqpoint{5.700000in}{5.700000in}}%
\pgfusepath{clip}%
\pgfsetbuttcap%
\pgfsetroundjoin%
\definecolor{currentfill}{rgb}{0.282623,0.140926,0.457517}%
\pgfsetfillcolor{currentfill}%
\pgfsetfillopacity{0.700000}%
\pgfsetlinewidth{0.000000pt}%
\definecolor{currentstroke}{rgb}{0.000000,0.000000,0.000000}%
\pgfsetstrokecolor{currentstroke}%
\pgfsetdash{}{0pt}%
\pgfpathmoveto{\pgfqpoint{4.196617in}{1.650086in}}%
\pgfpathlineto{\pgfqpoint{4.210801in}{1.650791in}}%
\pgfpathlineto{\pgfqpoint{4.224995in}{1.651568in}}%
\pgfpathlineto{\pgfqpoint{4.239199in}{1.652416in}}%
\pgfpathlineto{\pgfqpoint{4.253413in}{1.653337in}}%
\pgfpathlineto{\pgfqpoint{4.261529in}{1.665002in}}%
\pgfpathlineto{\pgfqpoint{4.269639in}{1.676623in}}%
\pgfpathlineto{\pgfqpoint{4.277744in}{1.688196in}}%
\pgfpathlineto{\pgfqpoint{4.285844in}{1.699719in}}%
\pgfpathlineto{\pgfqpoint{4.271637in}{1.698611in}}%
\pgfpathlineto{\pgfqpoint{4.257441in}{1.697575in}}%
\pgfpathlineto{\pgfqpoint{4.243253in}{1.696612in}}%
\pgfpathlineto{\pgfqpoint{4.229076in}{1.695720in}}%
\pgfpathlineto{\pgfqpoint{4.220969in}{1.684376in}}%
\pgfpathlineto{\pgfqpoint{4.212857in}{1.672988in}}%
\pgfpathlineto{\pgfqpoint{4.204740in}{1.661557in}}%
\pgfpathlineto{\pgfqpoint{4.196617in}{1.650086in}}%
\pgfpathclose%
\pgfusepath{fill}%
\end{pgfscope}%
\begin{pgfscope}%
\pgfpathrectangle{\pgfqpoint{1.150000in}{0.150000in}}{\pgfqpoint{5.700000in}{5.700000in}}%
\pgfusepath{clip}%
\pgfsetbuttcap%
\pgfsetroundjoin%
\definecolor{currentfill}{rgb}{0.282327,0.094955,0.417331}%
\pgfsetfillcolor{currentfill}%
\pgfsetfillopacity{0.700000}%
\pgfsetlinewidth{0.000000pt}%
\definecolor{currentstroke}{rgb}{0.000000,0.000000,0.000000}%
\pgfsetstrokecolor{currentstroke}%
\pgfsetdash{}{0pt}%
\pgfpathmoveto{\pgfqpoint{4.018173in}{1.557661in}}%
\pgfpathlineto{\pgfqpoint{4.032301in}{1.557352in}}%
\pgfpathlineto{\pgfqpoint{4.046437in}{1.557116in}}%
\pgfpathlineto{\pgfqpoint{4.060583in}{1.556952in}}%
\pgfpathlineto{\pgfqpoint{4.074737in}{1.556860in}}%
\pgfpathlineto{\pgfqpoint{4.082910in}{1.568312in}}%
\pgfpathlineto{\pgfqpoint{4.091077in}{1.579753in}}%
\pgfpathlineto{\pgfqpoint{4.099240in}{1.591180in}}%
\pgfpathlineto{\pgfqpoint{4.107397in}{1.602589in}}%
\pgfpathlineto{\pgfqpoint{4.093251in}{1.602452in}}%
\pgfpathlineto{\pgfqpoint{4.079114in}{1.602388in}}%
\pgfpathlineto{\pgfqpoint{4.064986in}{1.602396in}}%
\pgfpathlineto{\pgfqpoint{4.050866in}{1.602477in}}%
\pgfpathlineto{\pgfqpoint{4.042701in}{1.591288in}}%
\pgfpathlineto{\pgfqpoint{4.034530in}{1.580087in}}%
\pgfpathlineto{\pgfqpoint{4.026354in}{1.568877in}}%
\pgfpathlineto{\pgfqpoint{4.018173in}{1.557661in}}%
\pgfpathclose%
\pgfusepath{fill}%
\end{pgfscope}%
\begin{pgfscope}%
\pgfpathrectangle{\pgfqpoint{1.150000in}{0.150000in}}{\pgfqpoint{5.700000in}{5.700000in}}%
\pgfusepath{clip}%
\pgfsetbuttcap%
\pgfsetroundjoin%
\definecolor{currentfill}{rgb}{0.280255,0.165693,0.476498}%
\pgfsetfillcolor{currentfill}%
\pgfsetfillopacity{0.700000}%
\pgfsetlinewidth{0.000000pt}%
\definecolor{currentstroke}{rgb}{0.000000,0.000000,0.000000}%
\pgfsetstrokecolor{currentstroke}%
\pgfsetdash{}{0pt}%
\pgfpathmoveto{\pgfqpoint{4.285844in}{1.699719in}}%
\pgfpathlineto{\pgfqpoint{4.300060in}{1.700898in}}%
\pgfpathlineto{\pgfqpoint{4.314287in}{1.702149in}}%
\pgfpathlineto{\pgfqpoint{4.328523in}{1.703472in}}%
\pgfpathlineto{\pgfqpoint{4.342769in}{1.704867in}}%
\pgfpathlineto{\pgfqpoint{4.350857in}{1.716511in}}%
\pgfpathlineto{\pgfqpoint{4.358939in}{1.728095in}}%
\pgfpathlineto{\pgfqpoint{4.367016in}{1.739616in}}%
\pgfpathlineto{\pgfqpoint{4.375087in}{1.751073in}}%
\pgfpathlineto{\pgfqpoint{4.360847in}{1.749512in}}%
\pgfpathlineto{\pgfqpoint{4.346618in}{1.748023in}}%
\pgfpathlineto{\pgfqpoint{4.332398in}{1.746605in}}%
\pgfpathlineto{\pgfqpoint{4.318189in}{1.745259in}}%
\pgfpathlineto{\pgfqpoint{4.310111in}{1.733961in}}%
\pgfpathlineto{\pgfqpoint{4.302027in}{1.722603in}}%
\pgfpathlineto{\pgfqpoint{4.293938in}{1.711189in}}%
\pgfpathlineto{\pgfqpoint{4.285844in}{1.699719in}}%
\pgfpathclose%
\pgfusepath{fill}%
\end{pgfscope}%
\begin{pgfscope}%
\pgfpathrectangle{\pgfqpoint{1.150000in}{0.150000in}}{\pgfqpoint{5.700000in}{5.700000in}}%
\pgfusepath{clip}%
\pgfsetbuttcap%
\pgfsetroundjoin%
\definecolor{currentfill}{rgb}{0.276022,0.044167,0.370164}%
\pgfsetfillcolor{currentfill}%
\pgfsetfillopacity{0.700000}%
\pgfsetlinewidth{0.000000pt}%
\definecolor{currentstroke}{rgb}{0.000000,0.000000,0.000000}%
\pgfsetstrokecolor{currentstroke}%
\pgfsetdash{}{0pt}%
\pgfpathmoveto{\pgfqpoint{2.931558in}{1.487889in}}%
\pgfpathlineto{\pgfqpoint{2.945518in}{1.480283in}}%
\pgfpathlineto{\pgfqpoint{2.959481in}{1.472763in}}%
\pgfpathlineto{\pgfqpoint{2.973447in}{1.465326in}}%
\pgfpathlineto{\pgfqpoint{2.987416in}{1.457974in}}%
\pgfpathlineto{\pgfqpoint{2.996118in}{1.461061in}}%
\pgfpathlineto{\pgfqpoint{3.004808in}{1.464384in}}%
\pgfpathlineto{\pgfqpoint{3.013484in}{1.467937in}}%
\pgfpathlineto{\pgfqpoint{3.022147in}{1.471713in}}%
\pgfpathlineto{\pgfqpoint{3.008207in}{1.478651in}}%
\pgfpathlineto{\pgfqpoint{2.994271in}{1.485673in}}%
\pgfpathlineto{\pgfqpoint{2.980337in}{1.492779in}}%
\pgfpathlineto{\pgfqpoint{2.966407in}{1.499970in}}%
\pgfpathlineto{\pgfqpoint{2.957715in}{1.496600in}}%
\pgfpathlineto{\pgfqpoint{2.949010in}{1.493459in}}%
\pgfpathlineto{\pgfqpoint{2.940291in}{1.490553in}}%
\pgfpathlineto{\pgfqpoint{2.931558in}{1.487889in}}%
\pgfpathclose%
\pgfusepath{fill}%
\end{pgfscope}%
\begin{pgfscope}%
\pgfpathrectangle{\pgfqpoint{1.150000in}{0.150000in}}{\pgfqpoint{5.700000in}{5.700000in}}%
\pgfusepath{clip}%
\pgfsetbuttcap%
\pgfsetroundjoin%
\definecolor{currentfill}{rgb}{0.280267,0.073417,0.397163}%
\pgfsetfillcolor{currentfill}%
\pgfsetfillopacity{0.700000}%
\pgfsetlinewidth{0.000000pt}%
\definecolor{currentstroke}{rgb}{0.000000,0.000000,0.000000}%
\pgfsetstrokecolor{currentstroke}%
\pgfsetdash{}{0pt}%
\pgfpathmoveto{\pgfqpoint{3.928931in}{1.515758in}}%
\pgfpathlineto{\pgfqpoint{3.943034in}{1.514911in}}%
\pgfpathlineto{\pgfqpoint{3.957145in}{1.514135in}}%
\pgfpathlineto{\pgfqpoint{3.971265in}{1.513433in}}%
\pgfpathlineto{\pgfqpoint{3.985393in}{1.512803in}}%
\pgfpathlineto{\pgfqpoint{3.993596in}{1.524010in}}%
\pgfpathlineto{\pgfqpoint{4.001794in}{1.535224in}}%
\pgfpathlineto{\pgfqpoint{4.009986in}{1.546442in}}%
\pgfpathlineto{\pgfqpoint{4.018173in}{1.557661in}}%
\pgfpathlineto{\pgfqpoint{4.004054in}{1.558042in}}%
\pgfpathlineto{\pgfqpoint{3.989943in}{1.558496in}}%
\pgfpathlineto{\pgfqpoint{3.975841in}{1.559023in}}%
\pgfpathlineto{\pgfqpoint{3.961747in}{1.559622in}}%
\pgfpathlineto{\pgfqpoint{3.953551in}{1.548644in}}%
\pgfpathlineto{\pgfqpoint{3.945350in}{1.537672in}}%
\pgfpathlineto{\pgfqpoint{3.937143in}{1.526708in}}%
\pgfpathlineto{\pgfqpoint{3.928931in}{1.515758in}}%
\pgfpathclose%
\pgfusepath{fill}%
\end{pgfscope}%
\begin{pgfscope}%
\pgfpathrectangle{\pgfqpoint{1.150000in}{0.150000in}}{\pgfqpoint{5.700000in}{5.700000in}}%
\pgfusepath{clip}%
\pgfsetbuttcap%
\pgfsetroundjoin%
\definecolor{currentfill}{rgb}{0.276194,0.190074,0.493001}%
\pgfsetfillcolor{currentfill}%
\pgfsetfillopacity{0.700000}%
\pgfsetlinewidth{0.000000pt}%
\definecolor{currentstroke}{rgb}{0.000000,0.000000,0.000000}%
\pgfsetstrokecolor{currentstroke}%
\pgfsetdash{}{0pt}%
\pgfpathmoveto{\pgfqpoint{4.375087in}{1.751073in}}%
\pgfpathlineto{\pgfqpoint{4.389337in}{1.752706in}}%
\pgfpathlineto{\pgfqpoint{4.403598in}{1.754410in}}%
\pgfpathlineto{\pgfqpoint{4.417868in}{1.756186in}}%
\pgfpathlineto{\pgfqpoint{4.432150in}{1.758034in}}%
\pgfpathlineto{\pgfqpoint{4.440209in}{1.769577in}}%
\pgfpathlineto{\pgfqpoint{4.448263in}{1.781047in}}%
\pgfpathlineto{\pgfqpoint{4.456310in}{1.792440in}}%
\pgfpathlineto{\pgfqpoint{4.464353in}{1.803756in}}%
\pgfpathlineto{\pgfqpoint{4.450078in}{1.801763in}}%
\pgfpathlineto{\pgfqpoint{4.435814in}{1.799842in}}%
\pgfpathlineto{\pgfqpoint{4.421560in}{1.797992in}}%
\pgfpathlineto{\pgfqpoint{4.407316in}{1.796214in}}%
\pgfpathlineto{\pgfqpoint{4.399268in}{1.785035in}}%
\pgfpathlineto{\pgfqpoint{4.391213in}{1.773784in}}%
\pgfpathlineto{\pgfqpoint{4.383153in}{1.762463in}}%
\pgfpathlineto{\pgfqpoint{4.375087in}{1.751073in}}%
\pgfpathclose%
\pgfusepath{fill}%
\end{pgfscope}%
\begin{pgfscope}%
\pgfpathrectangle{\pgfqpoint{1.150000in}{0.150000in}}{\pgfqpoint{5.700000in}{5.700000in}}%
\pgfusepath{clip}%
\pgfsetbuttcap%
\pgfsetroundjoin%
\definecolor{currentfill}{rgb}{0.268510,0.009605,0.335427}%
\pgfsetfillcolor{currentfill}%
\pgfsetfillopacity{0.700000}%
\pgfsetlinewidth{0.000000pt}%
\definecolor{currentstroke}{rgb}{0.000000,0.000000,0.000000}%
\pgfsetstrokecolor{currentstroke}%
\pgfsetdash{}{0pt}%
\pgfpathmoveto{\pgfqpoint{3.515283in}{1.401919in}}%
\pgfpathlineto{\pgfqpoint{3.529294in}{1.398393in}}%
\pgfpathlineto{\pgfqpoint{3.543311in}{1.394943in}}%
\pgfpathlineto{\pgfqpoint{3.557334in}{1.391568in}}%
\pgfpathlineto{\pgfqpoint{3.571364in}{1.388269in}}%
\pgfpathlineto{\pgfqpoint{3.579727in}{1.397106in}}%
\pgfpathlineto{\pgfqpoint{3.588083in}{1.406045in}}%
\pgfpathlineto{\pgfqpoint{3.596431in}{1.415082in}}%
\pgfpathlineto{\pgfqpoint{3.604773in}{1.424211in}}%
\pgfpathlineto{\pgfqpoint{3.590759in}{1.427181in}}%
\pgfpathlineto{\pgfqpoint{3.576751in}{1.430227in}}%
\pgfpathlineto{\pgfqpoint{3.562750in}{1.433347in}}%
\pgfpathlineto{\pgfqpoint{3.548755in}{1.436543in}}%
\pgfpathlineto{\pgfqpoint{3.540398in}{1.427736in}}%
\pgfpathlineto{\pgfqpoint{3.532034in}{1.419026in}}%
\pgfpathlineto{\pgfqpoint{3.523662in}{1.410419in}}%
\pgfpathlineto{\pgfqpoint{3.515283in}{1.401919in}}%
\pgfpathclose%
\pgfusepath{fill}%
\end{pgfscope}%
\begin{pgfscope}%
\pgfpathrectangle{\pgfqpoint{1.150000in}{0.150000in}}{\pgfqpoint{5.700000in}{5.700000in}}%
\pgfusepath{clip}%
\pgfsetbuttcap%
\pgfsetroundjoin%
\definecolor{currentfill}{rgb}{0.269308,0.218818,0.509577}%
\pgfsetfillcolor{currentfill}%
\pgfsetfillopacity{0.700000}%
\pgfsetlinewidth{0.000000pt}%
\definecolor{currentstroke}{rgb}{0.000000,0.000000,0.000000}%
\pgfsetstrokecolor{currentstroke}%
\pgfsetdash{}{0pt}%
\pgfpathmoveto{\pgfqpoint{4.464353in}{1.803756in}}%
\pgfpathlineto{\pgfqpoint{4.478638in}{1.805821in}}%
\pgfpathlineto{\pgfqpoint{4.492934in}{1.807957in}}%
\pgfpathlineto{\pgfqpoint{4.507241in}{1.810165in}}%
\pgfpathlineto{\pgfqpoint{4.521559in}{1.812444in}}%
\pgfpathlineto{\pgfqpoint{4.529589in}{1.823813in}}%
\pgfpathlineto{\pgfqpoint{4.537613in}{1.835095in}}%
\pgfpathlineto{\pgfqpoint{4.545631in}{1.846290in}}%
\pgfpathlineto{\pgfqpoint{4.553643in}{1.857395in}}%
\pgfpathlineto{\pgfqpoint{4.539332in}{1.854992in}}%
\pgfpathlineto{\pgfqpoint{4.525032in}{1.852660in}}%
\pgfpathlineto{\pgfqpoint{4.510742in}{1.850399in}}%
\pgfpathlineto{\pgfqpoint{4.496463in}{1.848210in}}%
\pgfpathlineto{\pgfqpoint{4.488444in}{1.837221in}}%
\pgfpathlineto{\pgfqpoint{4.480420in}{1.826148in}}%
\pgfpathlineto{\pgfqpoint{4.472389in}{1.814993in}}%
\pgfpathlineto{\pgfqpoint{4.464353in}{1.803756in}}%
\pgfpathclose%
\pgfusepath{fill}%
\end{pgfscope}%
\begin{pgfscope}%
\pgfpathrectangle{\pgfqpoint{1.150000in}{0.150000in}}{\pgfqpoint{5.700000in}{5.700000in}}%
\pgfusepath{clip}%
\pgfsetbuttcap%
\pgfsetroundjoin%
\definecolor{currentfill}{rgb}{0.277941,0.056324,0.381191}%
\pgfsetfillcolor{currentfill}%
\pgfsetfillopacity{0.700000}%
\pgfsetlinewidth{0.000000pt}%
\definecolor{currentstroke}{rgb}{0.000000,0.000000,0.000000}%
\pgfsetstrokecolor{currentstroke}%
\pgfsetdash{}{0pt}%
\pgfpathmoveto{\pgfqpoint{3.839652in}{1.477357in}}%
\pgfpathlineto{\pgfqpoint{3.853733in}{1.475949in}}%
\pgfpathlineto{\pgfqpoint{3.867823in}{1.474613in}}%
\pgfpathlineto{\pgfqpoint{3.881920in}{1.473351in}}%
\pgfpathlineto{\pgfqpoint{3.896025in}{1.472161in}}%
\pgfpathlineto{\pgfqpoint{3.904260in}{1.483022in}}%
\pgfpathlineto{\pgfqpoint{3.912489in}{1.493911in}}%
\pgfpathlineto{\pgfqpoint{3.920713in}{1.504825in}}%
\pgfpathlineto{\pgfqpoint{3.928931in}{1.515758in}}%
\pgfpathlineto{\pgfqpoint{3.914836in}{1.516679in}}%
\pgfpathlineto{\pgfqpoint{3.900749in}{1.517673in}}%
\pgfpathlineto{\pgfqpoint{3.886670in}{1.518739in}}%
\pgfpathlineto{\pgfqpoint{3.872599in}{1.519880in}}%
\pgfpathlineto{\pgfqpoint{3.864371in}{1.509207in}}%
\pgfpathlineto{\pgfqpoint{3.856137in}{1.498560in}}%
\pgfpathlineto{\pgfqpoint{3.847897in}{1.487942in}}%
\pgfpathlineto{\pgfqpoint{3.839652in}{1.477357in}}%
\pgfpathclose%
\pgfusepath{fill}%
\end{pgfscope}%
\begin{pgfscope}%
\pgfpathrectangle{\pgfqpoint{1.150000in}{0.150000in}}{\pgfqpoint{5.700000in}{5.700000in}}%
\pgfusepath{clip}%
\pgfsetbuttcap%
\pgfsetroundjoin%
\definecolor{currentfill}{rgb}{0.262138,0.242286,0.520837}%
\pgfsetfillcolor{currentfill}%
\pgfsetfillopacity{0.700000}%
\pgfsetlinewidth{0.000000pt}%
\definecolor{currentstroke}{rgb}{0.000000,0.000000,0.000000}%
\pgfsetstrokecolor{currentstroke}%
\pgfsetdash{}{0pt}%
\pgfpathmoveto{\pgfqpoint{4.553643in}{1.857395in}}%
\pgfpathlineto{\pgfqpoint{4.567966in}{1.859871in}}%
\pgfpathlineto{\pgfqpoint{4.582299in}{1.862417in}}%
\pgfpathlineto{\pgfqpoint{4.596643in}{1.865035in}}%
\pgfpathlineto{\pgfqpoint{4.610999in}{1.867725in}}%
\pgfpathlineto{\pgfqpoint{4.618998in}{1.878850in}}%
\pgfpathlineto{\pgfqpoint{4.626992in}{1.889879in}}%
\pgfpathlineto{\pgfqpoint{4.634979in}{1.900809in}}%
\pgfpathlineto{\pgfqpoint{4.642960in}{1.911638in}}%
\pgfpathlineto{\pgfqpoint{4.628611in}{1.908846in}}%
\pgfpathlineto{\pgfqpoint{4.614274in}{1.906124in}}%
\pgfpathlineto{\pgfqpoint{4.599947in}{1.903474in}}%
\pgfpathlineto{\pgfqpoint{4.585631in}{1.900896in}}%
\pgfpathlineto{\pgfqpoint{4.577644in}{1.890161in}}%
\pgfpathlineto{\pgfqpoint{4.569650in}{1.879332in}}%
\pgfpathlineto{\pgfqpoint{4.561650in}{1.868410in}}%
\pgfpathlineto{\pgfqpoint{4.553643in}{1.857395in}}%
\pgfpathclose%
\pgfusepath{fill}%
\end{pgfscope}%
\begin{pgfscope}%
\pgfpathrectangle{\pgfqpoint{1.150000in}{0.150000in}}{\pgfqpoint{5.700000in}{5.700000in}}%
\pgfusepath{clip}%
\pgfsetbuttcap%
\pgfsetroundjoin%
\definecolor{currentfill}{rgb}{0.267004,0.004874,0.329415}%
\pgfsetfillcolor{currentfill}%
\pgfsetfillopacity{0.700000}%
\pgfsetlinewidth{0.000000pt}%
\definecolor{currentstroke}{rgb}{0.000000,0.000000,0.000000}%
\pgfsetstrokecolor{currentstroke}%
\pgfsetdash{}{0pt}%
\pgfpathmoveto{\pgfqpoint{3.279817in}{1.395335in}}%
\pgfpathlineto{\pgfqpoint{3.293801in}{1.390165in}}%
\pgfpathlineto{\pgfqpoint{3.307790in}{1.385074in}}%
\pgfpathlineto{\pgfqpoint{3.321783in}{1.380060in}}%
\pgfpathlineto{\pgfqpoint{3.335782in}{1.375125in}}%
\pgfpathlineto{\pgfqpoint{3.344267in}{1.381819in}}%
\pgfpathlineto{\pgfqpoint{3.352743in}{1.388675in}}%
\pgfpathlineto{\pgfqpoint{3.361209in}{1.395685in}}%
\pgfpathlineto{\pgfqpoint{3.369667in}{1.402845in}}%
\pgfpathlineto{\pgfqpoint{3.355689in}{1.407410in}}%
\pgfpathlineto{\pgfqpoint{3.341716in}{1.412053in}}%
\pgfpathlineto{\pgfqpoint{3.327749in}{1.416773in}}%
\pgfpathlineto{\pgfqpoint{3.313786in}{1.421572in}}%
\pgfpathlineto{\pgfqpoint{3.305308in}{1.414775in}}%
\pgfpathlineto{\pgfqpoint{3.296821in}{1.408133in}}%
\pgfpathlineto{\pgfqpoint{3.288324in}{1.401651in}}%
\pgfpathlineto{\pgfqpoint{3.279817in}{1.395335in}}%
\pgfpathclose%
\pgfusepath{fill}%
\end{pgfscope}%
\begin{pgfscope}%
\pgfpathrectangle{\pgfqpoint{1.150000in}{0.150000in}}{\pgfqpoint{5.700000in}{5.700000in}}%
\pgfusepath{clip}%
\pgfsetbuttcap%
\pgfsetroundjoin%
\definecolor{currentfill}{rgb}{0.274952,0.037752,0.364543}%
\pgfsetfillcolor{currentfill}%
\pgfsetfillopacity{0.700000}%
\pgfsetlinewidth{0.000000pt}%
\definecolor{currentstroke}{rgb}{0.000000,0.000000,0.000000}%
\pgfsetstrokecolor{currentstroke}%
\pgfsetdash{}{0pt}%
\pgfpathmoveto{\pgfqpoint{3.750315in}{1.442957in}}%
\pgfpathlineto{\pgfqpoint{3.764378in}{1.440966in}}%
\pgfpathlineto{\pgfqpoint{3.778448in}{1.439048in}}%
\pgfpathlineto{\pgfqpoint{3.792526in}{1.437203in}}%
\pgfpathlineto{\pgfqpoint{3.806611in}{1.435432in}}%
\pgfpathlineto{\pgfqpoint{3.814880in}{1.445843in}}%
\pgfpathlineto{\pgfqpoint{3.823143in}{1.456304in}}%
\pgfpathlineto{\pgfqpoint{3.831400in}{1.466810in}}%
\pgfpathlineto{\pgfqpoint{3.839652in}{1.477357in}}%
\pgfpathlineto{\pgfqpoint{3.825578in}{1.478839in}}%
\pgfpathlineto{\pgfqpoint{3.811512in}{1.480395in}}%
\pgfpathlineto{\pgfqpoint{3.797454in}{1.482024in}}%
\pgfpathlineto{\pgfqpoint{3.783403in}{1.483726in}}%
\pgfpathlineto{\pgfqpoint{3.775140in}{1.473460in}}%
\pgfpathlineto{\pgfqpoint{3.766871in}{1.463241in}}%
\pgfpathlineto{\pgfqpoint{3.758596in}{1.453072in}}%
\pgfpathlineto{\pgfqpoint{3.750315in}{1.442957in}}%
\pgfpathclose%
\pgfusepath{fill}%
\end{pgfscope}%
\begin{pgfscope}%
\pgfpathrectangle{\pgfqpoint{1.150000in}{0.150000in}}{\pgfqpoint{5.700000in}{5.700000in}}%
\pgfusepath{clip}%
\pgfsetbuttcap%
\pgfsetroundjoin%
\definecolor{currentfill}{rgb}{0.253935,0.265254,0.529983}%
\pgfsetfillcolor{currentfill}%
\pgfsetfillopacity{0.700000}%
\pgfsetlinewidth{0.000000pt}%
\definecolor{currentstroke}{rgb}{0.000000,0.000000,0.000000}%
\pgfsetstrokecolor{currentstroke}%
\pgfsetdash{}{0pt}%
\pgfpathmoveto{\pgfqpoint{4.642960in}{1.911638in}}%
\pgfpathlineto{\pgfqpoint{4.657321in}{1.914503in}}%
\pgfpathlineto{\pgfqpoint{4.671692in}{1.917438in}}%
\pgfpathlineto{\pgfqpoint{4.686076in}{1.920445in}}%
\pgfpathlineto{\pgfqpoint{4.700470in}{1.923523in}}%
\pgfpathlineto{\pgfqpoint{4.708438in}{1.934343in}}%
\pgfpathlineto{\pgfqpoint{4.716400in}{1.945055in}}%
\pgfpathlineto{\pgfqpoint{4.724354in}{1.955658in}}%
\pgfpathlineto{\pgfqpoint{4.732303in}{1.966153in}}%
\pgfpathlineto{\pgfqpoint{4.717915in}{1.962993in}}%
\pgfpathlineto{\pgfqpoint{4.703538in}{1.959904in}}%
\pgfpathlineto{\pgfqpoint{4.689174in}{1.956886in}}%
\pgfpathlineto{\pgfqpoint{4.674820in}{1.953940in}}%
\pgfpathlineto{\pgfqpoint{4.666865in}{1.943519in}}%
\pgfpathlineto{\pgfqpoint{4.658903in}{1.932995in}}%
\pgfpathlineto{\pgfqpoint{4.650935in}{1.922367in}}%
\pgfpathlineto{\pgfqpoint{4.642960in}{1.911638in}}%
\pgfpathclose%
\pgfusepath{fill}%
\end{pgfscope}%
\begin{pgfscope}%
\pgfpathrectangle{\pgfqpoint{1.150000in}{0.150000in}}{\pgfqpoint{5.700000in}{5.700000in}}%
\pgfusepath{clip}%
\pgfsetbuttcap%
\pgfsetroundjoin%
\definecolor{currentfill}{rgb}{0.269944,0.014625,0.341379}%
\pgfsetfillcolor{currentfill}%
\pgfsetfillopacity{0.700000}%
\pgfsetlinewidth{0.000000pt}%
\definecolor{currentstroke}{rgb}{0.000000,0.000000,0.000000}%
\pgfsetstrokecolor{currentstroke}%
\pgfsetdash{}{0pt}%
\pgfpathmoveto{\pgfqpoint{3.133791in}{1.419190in}}%
\pgfpathlineto{\pgfqpoint{3.147764in}{1.412992in}}%
\pgfpathlineto{\pgfqpoint{3.161740in}{1.406875in}}%
\pgfpathlineto{\pgfqpoint{3.175721in}{1.400838in}}%
\pgfpathlineto{\pgfqpoint{3.189706in}{1.394881in}}%
\pgfpathlineto{\pgfqpoint{3.198278in}{1.400075in}}%
\pgfpathlineto{\pgfqpoint{3.206839in}{1.405464in}}%
\pgfpathlineto{\pgfqpoint{3.215389in}{1.411042in}}%
\pgfpathlineto{\pgfqpoint{3.223929in}{1.416802in}}%
\pgfpathlineto{\pgfqpoint{3.209968in}{1.422367in}}%
\pgfpathlineto{\pgfqpoint{3.196012in}{1.428012in}}%
\pgfpathlineto{\pgfqpoint{3.182060in}{1.433738in}}%
\pgfpathlineto{\pgfqpoint{3.168113in}{1.439544in}}%
\pgfpathlineto{\pgfqpoint{3.159549in}{1.434167in}}%
\pgfpathlineto{\pgfqpoint{3.150975in}{1.428979in}}%
\pgfpathlineto{\pgfqpoint{3.142389in}{1.423984in}}%
\pgfpathlineto{\pgfqpoint{3.133791in}{1.419190in}}%
\pgfpathclose%
\pgfusepath{fill}%
\end{pgfscope}%
\begin{pgfscope}%
\pgfpathrectangle{\pgfqpoint{1.150000in}{0.150000in}}{\pgfqpoint{5.700000in}{5.700000in}}%
\pgfusepath{clip}%
\pgfsetbuttcap%
\pgfsetroundjoin%
\definecolor{currentfill}{rgb}{0.243113,0.292092,0.538516}%
\pgfsetfillcolor{currentfill}%
\pgfsetfillopacity{0.700000}%
\pgfsetlinewidth{0.000000pt}%
\definecolor{currentstroke}{rgb}{0.000000,0.000000,0.000000}%
\pgfsetstrokecolor{currentstroke}%
\pgfsetdash{}{0pt}%
\pgfpathmoveto{\pgfqpoint{4.732303in}{1.966153in}}%
\pgfpathlineto{\pgfqpoint{4.746702in}{1.969385in}}%
\pgfpathlineto{\pgfqpoint{4.761113in}{1.972688in}}%
\pgfpathlineto{\pgfqpoint{4.775536in}{1.976062in}}%
\pgfpathlineto{\pgfqpoint{4.789971in}{1.979508in}}%
\pgfpathlineto{\pgfqpoint{4.797905in}{1.989962in}}%
\pgfpathlineto{\pgfqpoint{4.805833in}{2.000301in}}%
\pgfpathlineto{\pgfqpoint{4.813753in}{2.010523in}}%
\pgfpathlineto{\pgfqpoint{4.821667in}{2.020628in}}%
\pgfpathlineto{\pgfqpoint{4.807239in}{2.017122in}}%
\pgfpathlineto{\pgfqpoint{4.792824in}{2.013687in}}%
\pgfpathlineto{\pgfqpoint{4.778420in}{2.010324in}}%
\pgfpathlineto{\pgfqpoint{4.764028in}{2.007031in}}%
\pgfpathlineto{\pgfqpoint{4.756107in}{1.996978in}}%
\pgfpathlineto{\pgfqpoint{4.748179in}{1.986814in}}%
\pgfpathlineto{\pgfqpoint{4.740244in}{1.976539in}}%
\pgfpathlineto{\pgfqpoint{4.732303in}{1.966153in}}%
\pgfpathclose%
\pgfusepath{fill}%
\end{pgfscope}%
\begin{pgfscope}%
\pgfpathrectangle{\pgfqpoint{1.150000in}{0.150000in}}{\pgfqpoint{5.700000in}{5.700000in}}%
\pgfusepath{clip}%
\pgfsetbuttcap%
\pgfsetroundjoin%
\definecolor{currentfill}{rgb}{0.267004,0.004874,0.329415}%
\pgfsetfillcolor{currentfill}%
\pgfsetfillopacity{0.700000}%
\pgfsetlinewidth{0.000000pt}%
\definecolor{currentstroke}{rgb}{0.000000,0.000000,0.000000}%
\pgfsetstrokecolor{currentstroke}%
\pgfsetdash{}{0pt}%
\pgfpathmoveto{\pgfqpoint{3.425632in}{1.385357in}}%
\pgfpathlineto{\pgfqpoint{3.439637in}{1.381178in}}%
\pgfpathlineto{\pgfqpoint{3.453648in}{1.377074in}}%
\pgfpathlineto{\pgfqpoint{3.467665in}{1.373047in}}%
\pgfpathlineto{\pgfqpoint{3.481687in}{1.369095in}}%
\pgfpathlineto{\pgfqpoint{3.490098in}{1.377114in}}%
\pgfpathlineto{\pgfqpoint{3.498501in}{1.385261in}}%
\pgfpathlineto{\pgfqpoint{3.506896in}{1.393531in}}%
\pgfpathlineto{\pgfqpoint{3.515283in}{1.401919in}}%
\pgfpathlineto{\pgfqpoint{3.501278in}{1.405520in}}%
\pgfpathlineto{\pgfqpoint{3.487279in}{1.409198in}}%
\pgfpathlineto{\pgfqpoint{3.473287in}{1.412951in}}%
\pgfpathlineto{\pgfqpoint{3.459300in}{1.416781in}}%
\pgfpathlineto{\pgfqpoint{3.450895in}{1.408736in}}%
\pgfpathlineto{\pgfqpoint{3.442483in}{1.400813in}}%
\pgfpathlineto{\pgfqpoint{3.434062in}{1.393019in}}%
\pgfpathlineto{\pgfqpoint{3.425632in}{1.385357in}}%
\pgfpathclose%
\pgfusepath{fill}%
\end{pgfscope}%
\begin{pgfscope}%
\pgfpathrectangle{\pgfqpoint{1.150000in}{0.150000in}}{\pgfqpoint{5.700000in}{5.700000in}}%
\pgfusepath{clip}%
\pgfsetbuttcap%
\pgfsetroundjoin%
\definecolor{currentfill}{rgb}{0.233603,0.313828,0.543914}%
\pgfsetfillcolor{currentfill}%
\pgfsetfillopacity{0.700000}%
\pgfsetlinewidth{0.000000pt}%
\definecolor{currentstroke}{rgb}{0.000000,0.000000,0.000000}%
\pgfsetstrokecolor{currentstroke}%
\pgfsetdash{}{0pt}%
\pgfpathmoveto{\pgfqpoint{4.821667in}{2.020628in}}%
\pgfpathlineto{\pgfqpoint{4.836106in}{2.024206in}}%
\pgfpathlineto{\pgfqpoint{4.850558in}{2.027855in}}%
\pgfpathlineto{\pgfqpoint{4.865022in}{2.031575in}}%
\pgfpathlineto{\pgfqpoint{4.879497in}{2.035367in}}%
\pgfpathlineto{\pgfqpoint{4.887396in}{2.045402in}}%
\pgfpathlineto{\pgfqpoint{4.895288in}{2.055316in}}%
\pgfpathlineto{\pgfqpoint{4.903172in}{2.065106in}}%
\pgfpathlineto{\pgfqpoint{4.911048in}{2.074773in}}%
\pgfpathlineto{\pgfqpoint{4.896581in}{2.070943in}}%
\pgfpathlineto{\pgfqpoint{4.882125in}{2.067184in}}%
\pgfpathlineto{\pgfqpoint{4.867681in}{2.063496in}}%
\pgfpathlineto{\pgfqpoint{4.853250in}{2.059879in}}%
\pgfpathlineto{\pgfqpoint{4.845365in}{2.050243in}}%
\pgfpathlineto{\pgfqpoint{4.837473in}{2.040489in}}%
\pgfpathlineto{\pgfqpoint{4.829573in}{2.030617in}}%
\pgfpathlineto{\pgfqpoint{4.821667in}{2.020628in}}%
\pgfpathclose%
\pgfusepath{fill}%
\end{pgfscope}%
\begin{pgfscope}%
\pgfpathrectangle{\pgfqpoint{1.150000in}{0.150000in}}{\pgfqpoint{5.700000in}{5.700000in}}%
\pgfusepath{clip}%
\pgfsetbuttcap%
\pgfsetroundjoin%
\definecolor{currentfill}{rgb}{0.271305,0.019942,0.347269}%
\pgfsetfillcolor{currentfill}%
\pgfsetfillopacity{0.700000}%
\pgfsetlinewidth{0.000000pt}%
\definecolor{currentstroke}{rgb}{0.000000,0.000000,0.000000}%
\pgfsetstrokecolor{currentstroke}%
\pgfsetdash{}{0pt}%
\pgfpathmoveto{\pgfqpoint{3.660895in}{1.413079in}}%
\pgfpathlineto{\pgfqpoint{3.674943in}{1.410483in}}%
\pgfpathlineto{\pgfqpoint{3.688997in}{1.407960in}}%
\pgfpathlineto{\pgfqpoint{3.703059in}{1.405511in}}%
\pgfpathlineto{\pgfqpoint{3.717127in}{1.403137in}}%
\pgfpathlineto{\pgfqpoint{3.725434in}{1.412987in}}%
\pgfpathlineto{\pgfqpoint{3.733734in}{1.422911in}}%
\pgfpathlineto{\pgfqpoint{3.742028in}{1.432902in}}%
\pgfpathlineto{\pgfqpoint{3.750315in}{1.442957in}}%
\pgfpathlineto{\pgfqpoint{3.736259in}{1.445023in}}%
\pgfpathlineto{\pgfqpoint{3.722211in}{1.447162in}}%
\pgfpathlineto{\pgfqpoint{3.708170in}{1.449375in}}%
\pgfpathlineto{\pgfqpoint{3.694136in}{1.451663in}}%
\pgfpathlineto{\pgfqpoint{3.685836in}{1.441909in}}%
\pgfpathlineto{\pgfqpoint{3.677529in}{1.432224in}}%
\pgfpathlineto{\pgfqpoint{3.669215in}{1.422613in}}%
\pgfpathlineto{\pgfqpoint{3.660895in}{1.413079in}}%
\pgfpathclose%
\pgfusepath{fill}%
\end{pgfscope}%
\begin{pgfscope}%
\pgfpathrectangle{\pgfqpoint{1.150000in}{0.150000in}}{\pgfqpoint{5.700000in}{5.700000in}}%
\pgfusepath{clip}%
\pgfsetbuttcap%
\pgfsetroundjoin%
\definecolor{currentfill}{rgb}{0.274952,0.037752,0.364543}%
\pgfsetfillcolor{currentfill}%
\pgfsetfillopacity{0.700000}%
\pgfsetlinewidth{0.000000pt}%
\definecolor{currentstroke}{rgb}{0.000000,0.000000,0.000000}%
\pgfsetstrokecolor{currentstroke}%
\pgfsetdash{}{0pt}%
\pgfpathmoveto{\pgfqpoint{2.987416in}{1.457974in}}%
\pgfpathlineto{\pgfqpoint{3.001388in}{1.450706in}}%
\pgfpathlineto{\pgfqpoint{3.015363in}{1.443521in}}%
\pgfpathlineto{\pgfqpoint{3.029341in}{1.436418in}}%
\pgfpathlineto{\pgfqpoint{3.043322in}{1.429399in}}%
\pgfpathlineto{\pgfqpoint{3.051996in}{1.432908in}}%
\pgfpathlineto{\pgfqpoint{3.060657in}{1.436648in}}%
\pgfpathlineto{\pgfqpoint{3.069305in}{1.440612in}}%
\pgfpathlineto{\pgfqpoint{3.077940in}{1.444794in}}%
\pgfpathlineto{\pgfqpoint{3.063987in}{1.451400in}}%
\pgfpathlineto{\pgfqpoint{3.050037in}{1.458088in}}%
\pgfpathlineto{\pgfqpoint{3.036090in}{1.464859in}}%
\pgfpathlineto{\pgfqpoint{3.022147in}{1.471713in}}%
\pgfpathlineto{\pgfqpoint{3.013484in}{1.467937in}}%
\pgfpathlineto{\pgfqpoint{3.004808in}{1.464384in}}%
\pgfpathlineto{\pgfqpoint{2.996118in}{1.461061in}}%
\pgfpathlineto{\pgfqpoint{2.987416in}{1.457974in}}%
\pgfpathclose%
\pgfusepath{fill}%
\end{pgfscope}%
\begin{pgfscope}%
\pgfpathrectangle{\pgfqpoint{1.150000in}{0.150000in}}{\pgfqpoint{5.700000in}{5.700000in}}%
\pgfusepath{clip}%
\pgfsetbuttcap%
\pgfsetroundjoin%
\definecolor{currentfill}{rgb}{0.221989,0.339161,0.548752}%
\pgfsetfillcolor{currentfill}%
\pgfsetfillopacity{0.700000}%
\pgfsetlinewidth{0.000000pt}%
\definecolor{currentstroke}{rgb}{0.000000,0.000000,0.000000}%
\pgfsetstrokecolor{currentstroke}%
\pgfsetdash{}{0pt}%
\pgfpathmoveto{\pgfqpoint{4.911048in}{2.074773in}}%
\pgfpathlineto{\pgfqpoint{4.925529in}{2.078675in}}%
\pgfpathlineto{\pgfqpoint{4.940021in}{2.082648in}}%
\pgfpathlineto{\pgfqpoint{4.954526in}{2.086692in}}%
\pgfpathlineto{\pgfqpoint{4.969044in}{2.090808in}}%
\pgfpathlineto{\pgfqpoint{4.976904in}{2.100377in}}%
\pgfpathlineto{\pgfqpoint{4.984758in}{2.109819in}}%
\pgfpathlineto{\pgfqpoint{4.992603in}{2.119132in}}%
\pgfpathlineto{\pgfqpoint{5.000441in}{2.128318in}}%
\pgfpathlineto{\pgfqpoint{4.985932in}{2.124185in}}%
\pgfpathlineto{\pgfqpoint{4.971436in}{2.120124in}}%
\pgfpathlineto{\pgfqpoint{4.956952in}{2.116134in}}%
\pgfpathlineto{\pgfqpoint{4.942480in}{2.112215in}}%
\pgfpathlineto{\pgfqpoint{4.934634in}{2.103038in}}%
\pgfpathlineto{\pgfqpoint{4.926779in}{2.093739in}}%
\pgfpathlineto{\pgfqpoint{4.918918in}{2.084318in}}%
\pgfpathlineto{\pgfqpoint{4.911048in}{2.074773in}}%
\pgfpathclose%
\pgfusepath{fill}%
\end{pgfscope}%
\begin{pgfscope}%
\pgfpathrectangle{\pgfqpoint{1.150000in}{0.150000in}}{\pgfqpoint{5.700000in}{5.700000in}}%
\pgfusepath{clip}%
\pgfsetbuttcap%
\pgfsetroundjoin%
\definecolor{currentfill}{rgb}{0.172719,0.448791,0.557885}%
\pgfsetfillcolor{currentfill}%
\pgfsetfillopacity{0.700000}%
\pgfsetlinewidth{0.000000pt}%
\definecolor{currentstroke}{rgb}{0.000000,0.000000,0.000000}%
\pgfsetstrokecolor{currentstroke}%
\pgfsetdash{}{0pt}%
\pgfpathmoveto{\pgfqpoint{5.447247in}{2.379055in}}%
\pgfpathlineto{\pgfqpoint{5.461975in}{2.384437in}}%
\pgfpathlineto{\pgfqpoint{5.476717in}{2.389890in}}%
\pgfpathlineto{\pgfqpoint{5.491473in}{2.395414in}}%
\pgfpathlineto{\pgfqpoint{5.499056in}{2.401514in}}%
\pgfpathlineto{\pgfqpoint{5.506630in}{2.407482in}}%
\pgfpathlineto{\pgfqpoint{5.514193in}{2.413321in}}%
\pgfpathlineto{\pgfqpoint{5.521747in}{2.419033in}}%
\pgfpathlineto{\pgfqpoint{5.507006in}{2.413625in}}%
\pgfpathlineto{\pgfqpoint{5.492280in}{2.408288in}}%
\pgfpathlineto{\pgfqpoint{5.477567in}{2.403022in}}%
\pgfpathlineto{\pgfqpoint{5.470002in}{2.397217in}}%
\pgfpathlineto{\pgfqpoint{5.462426in}{2.391289in}}%
\pgfpathlineto{\pgfqpoint{5.454841in}{2.385235in}}%
\pgfpathlineto{\pgfqpoint{5.447247in}{2.379055in}}%
\pgfpathclose%
\pgfusepath{fill}%
\end{pgfscope}%
\begin{pgfscope}%
\pgfpathrectangle{\pgfqpoint{1.150000in}{0.150000in}}{\pgfqpoint{5.700000in}{5.700000in}}%
\pgfusepath{clip}%
\pgfsetbuttcap%
\pgfsetroundjoin%
\definecolor{currentfill}{rgb}{0.212395,0.359683,0.551710}%
\pgfsetfillcolor{currentfill}%
\pgfsetfillopacity{0.700000}%
\pgfsetlinewidth{0.000000pt}%
\definecolor{currentstroke}{rgb}{0.000000,0.000000,0.000000}%
\pgfsetstrokecolor{currentstroke}%
\pgfsetdash{}{0pt}%
\pgfpathmoveto{\pgfqpoint{5.000441in}{2.128318in}}%
\pgfpathlineto{\pgfqpoint{5.014962in}{2.132521in}}%
\pgfpathlineto{\pgfqpoint{5.029496in}{2.136797in}}%
\pgfpathlineto{\pgfqpoint{5.044043in}{2.141143in}}%
\pgfpathlineto{\pgfqpoint{5.058602in}{2.145561in}}%
\pgfpathlineto{\pgfqpoint{5.066423in}{2.154622in}}%
\pgfpathlineto{\pgfqpoint{5.074235in}{2.163550in}}%
\pgfpathlineto{\pgfqpoint{5.082040in}{2.172347in}}%
\pgfpathlineto{\pgfqpoint{5.089836in}{2.181012in}}%
\pgfpathlineto{\pgfqpoint{5.075286in}{2.176599in}}%
\pgfpathlineto{\pgfqpoint{5.060748in}{2.172258in}}%
\pgfpathlineto{\pgfqpoint{5.046224in}{2.167987in}}%
\pgfpathlineto{\pgfqpoint{5.031712in}{2.163788in}}%
\pgfpathlineto{\pgfqpoint{5.023906in}{2.155110in}}%
\pgfpathlineto{\pgfqpoint{5.016092in}{2.146306in}}%
\pgfpathlineto{\pgfqpoint{5.008270in}{2.137375in}}%
\pgfpathlineto{\pgfqpoint{5.000441in}{2.128318in}}%
\pgfpathclose%
\pgfusepath{fill}%
\end{pgfscope}%
\begin{pgfscope}%
\pgfpathrectangle{\pgfqpoint{1.150000in}{0.150000in}}{\pgfqpoint{5.700000in}{5.700000in}}%
\pgfusepath{clip}%
\pgfsetbuttcap%
\pgfsetroundjoin%
\definecolor{currentfill}{rgb}{0.203063,0.379716,0.553925}%
\pgfsetfillcolor{currentfill}%
\pgfsetfillopacity{0.700000}%
\pgfsetlinewidth{0.000000pt}%
\definecolor{currentstroke}{rgb}{0.000000,0.000000,0.000000}%
\pgfsetstrokecolor{currentstroke}%
\pgfsetdash{}{0pt}%
\pgfpathmoveto{\pgfqpoint{5.089836in}{2.181012in}}%
\pgfpathlineto{\pgfqpoint{5.104399in}{2.185496in}}%
\pgfpathlineto{\pgfqpoint{5.118974in}{2.190051in}}%
\pgfpathlineto{\pgfqpoint{5.133563in}{2.194678in}}%
\pgfpathlineto{\pgfqpoint{5.148165in}{2.199376in}}%
\pgfpathlineto{\pgfqpoint{5.155943in}{2.207891in}}%
\pgfpathlineto{\pgfqpoint{5.163712in}{2.216271in}}%
\pgfpathlineto{\pgfqpoint{5.171472in}{2.224517in}}%
\pgfpathlineto{\pgfqpoint{5.179224in}{2.232629in}}%
\pgfpathlineto{\pgfqpoint{5.164633in}{2.227958in}}%
\pgfpathlineto{\pgfqpoint{5.150055in}{2.223358in}}%
\pgfpathlineto{\pgfqpoint{5.135490in}{2.218830in}}%
\pgfpathlineto{\pgfqpoint{5.120937in}{2.214373in}}%
\pgfpathlineto{\pgfqpoint{5.113174in}{2.206226in}}%
\pgfpathlineto{\pgfqpoint{5.105403in}{2.197951in}}%
\pgfpathlineto{\pgfqpoint{5.097624in}{2.189546in}}%
\pgfpathlineto{\pgfqpoint{5.089836in}{2.181012in}}%
\pgfpathclose%
\pgfusepath{fill}%
\end{pgfscope}%
\begin{pgfscope}%
\pgfpathrectangle{\pgfqpoint{1.150000in}{0.150000in}}{\pgfqpoint{5.700000in}{5.700000in}}%
\pgfusepath{clip}%
\pgfsetbuttcap%
\pgfsetroundjoin%
\definecolor{currentfill}{rgb}{0.194100,0.399323,0.555565}%
\pgfsetfillcolor{currentfill}%
\pgfsetfillopacity{0.700000}%
\pgfsetlinewidth{0.000000pt}%
\definecolor{currentstroke}{rgb}{0.000000,0.000000,0.000000}%
\pgfsetstrokecolor{currentstroke}%
\pgfsetdash{}{0pt}%
\pgfpathmoveto{\pgfqpoint{5.179224in}{2.232629in}}%
\pgfpathlineto{\pgfqpoint{5.193829in}{2.237371in}}%
\pgfpathlineto{\pgfqpoint{5.208446in}{2.242184in}}%
\pgfpathlineto{\pgfqpoint{5.223077in}{2.247069in}}%
\pgfpathlineto{\pgfqpoint{5.237722in}{2.252025in}}%
\pgfpathlineto{\pgfqpoint{5.245454in}{2.259963in}}%
\pgfpathlineto{\pgfqpoint{5.253177in}{2.267764in}}%
\pgfpathlineto{\pgfqpoint{5.260891in}{2.275430in}}%
\pgfpathlineto{\pgfqpoint{5.268597in}{2.282961in}}%
\pgfpathlineto{\pgfqpoint{5.253964in}{2.278055in}}%
\pgfpathlineto{\pgfqpoint{5.239345in}{2.273219in}}%
\pgfpathlineto{\pgfqpoint{5.224739in}{2.268455in}}%
\pgfpathlineto{\pgfqpoint{5.210146in}{2.263762in}}%
\pgfpathlineto{\pgfqpoint{5.202429in}{2.256174in}}%
\pgfpathlineto{\pgfqpoint{5.194702in}{2.248456in}}%
\pgfpathlineto{\pgfqpoint{5.186968in}{2.240608in}}%
\pgfpathlineto{\pgfqpoint{5.179224in}{2.232629in}}%
\pgfpathclose%
\pgfusepath{fill}%
\end{pgfscope}%
\begin{pgfscope}%
\pgfpathrectangle{\pgfqpoint{1.150000in}{0.150000in}}{\pgfqpoint{5.700000in}{5.700000in}}%
\pgfusepath{clip}%
\pgfsetbuttcap%
\pgfsetroundjoin%
\definecolor{currentfill}{rgb}{0.185556,0.418570,0.556753}%
\pgfsetfillcolor{currentfill}%
\pgfsetfillopacity{0.700000}%
\pgfsetlinewidth{0.000000pt}%
\definecolor{currentstroke}{rgb}{0.000000,0.000000,0.000000}%
\pgfsetstrokecolor{currentstroke}%
\pgfsetdash{}{0pt}%
\pgfpathmoveto{\pgfqpoint{5.268597in}{2.282961in}}%
\pgfpathlineto{\pgfqpoint{5.283243in}{2.287939in}}%
\pgfpathlineto{\pgfqpoint{5.297902in}{2.292988in}}%
\pgfpathlineto{\pgfqpoint{5.312575in}{2.298108in}}%
\pgfpathlineto{\pgfqpoint{5.327261in}{2.303300in}}%
\pgfpathlineto{\pgfqpoint{5.334945in}{2.310635in}}%
\pgfpathlineto{\pgfqpoint{5.342620in}{2.317833in}}%
\pgfpathlineto{\pgfqpoint{5.350285in}{2.324895in}}%
\pgfpathlineto{\pgfqpoint{5.357941in}{2.331824in}}%
\pgfpathlineto{\pgfqpoint{5.343268in}{2.326704in}}%
\pgfpathlineto{\pgfqpoint{5.328608in}{2.321656in}}%
\pgfpathlineto{\pgfqpoint{5.313961in}{2.316678in}}%
\pgfpathlineto{\pgfqpoint{5.299328in}{2.311772in}}%
\pgfpathlineto{\pgfqpoint{5.291659in}{2.304763in}}%
\pgfpathlineto{\pgfqpoint{5.283980in}{2.297626in}}%
\pgfpathlineto{\pgfqpoint{5.276293in}{2.290360in}}%
\pgfpathlineto{\pgfqpoint{5.268597in}{2.282961in}}%
\pgfpathclose%
\pgfusepath{fill}%
\end{pgfscope}%
\begin{pgfscope}%
\pgfpathrectangle{\pgfqpoint{1.150000in}{0.150000in}}{\pgfqpoint{5.700000in}{5.700000in}}%
\pgfusepath{clip}%
\pgfsetbuttcap%
\pgfsetroundjoin%
\definecolor{currentfill}{rgb}{0.179019,0.433756,0.557430}%
\pgfsetfillcolor{currentfill}%
\pgfsetfillopacity{0.700000}%
\pgfsetlinewidth{0.000000pt}%
\definecolor{currentstroke}{rgb}{0.000000,0.000000,0.000000}%
\pgfsetstrokecolor{currentstroke}%
\pgfsetdash{}{0pt}%
\pgfpathmoveto{\pgfqpoint{5.357941in}{2.331824in}}%
\pgfpathlineto{\pgfqpoint{5.372628in}{2.337015in}}%
\pgfpathlineto{\pgfqpoint{5.387329in}{2.342278in}}%
\pgfpathlineto{\pgfqpoint{5.402044in}{2.347611in}}%
\pgfpathlineto{\pgfqpoint{5.416773in}{2.353017in}}%
\pgfpathlineto{\pgfqpoint{5.424406in}{2.359727in}}%
\pgfpathlineto{\pgfqpoint{5.432029in}{2.366303in}}%
\pgfpathlineto{\pgfqpoint{5.439643in}{2.372745in}}%
\pgfpathlineto{\pgfqpoint{5.447247in}{2.379055in}}%
\pgfpathlineto{\pgfqpoint{5.432533in}{2.373744in}}%
\pgfpathlineto{\pgfqpoint{5.417832in}{2.368504in}}%
\pgfpathlineto{\pgfqpoint{5.403145in}{2.363336in}}%
\pgfpathlineto{\pgfqpoint{5.388472in}{2.358239in}}%
\pgfpathlineto{\pgfqpoint{5.380853in}{2.351826in}}%
\pgfpathlineto{\pgfqpoint{5.373225in}{2.345288in}}%
\pgfpathlineto{\pgfqpoint{5.365588in}{2.338621in}}%
\pgfpathlineto{\pgfqpoint{5.357941in}{2.331824in}}%
\pgfpathclose%
\pgfusepath{fill}%
\end{pgfscope}%
\begin{pgfscope}%
\pgfpathrectangle{\pgfqpoint{1.150000in}{0.150000in}}{\pgfqpoint{5.700000in}{5.700000in}}%
\pgfusepath{clip}%
\pgfsetbuttcap%
\pgfsetroundjoin%
\definecolor{currentfill}{rgb}{0.268510,0.009605,0.335427}%
\pgfsetfillcolor{currentfill}%
\pgfsetfillopacity{0.700000}%
\pgfsetlinewidth{0.000000pt}%
\definecolor{currentstroke}{rgb}{0.000000,0.000000,0.000000}%
\pgfsetstrokecolor{currentstroke}%
\pgfsetdash{}{0pt}%
\pgfpathmoveto{\pgfqpoint{3.571364in}{1.388269in}}%
\pgfpathlineto{\pgfqpoint{3.585400in}{1.385044in}}%
\pgfpathlineto{\pgfqpoint{3.599442in}{1.381894in}}%
\pgfpathlineto{\pgfqpoint{3.613491in}{1.378819in}}%
\pgfpathlineto{\pgfqpoint{3.627546in}{1.375818in}}%
\pgfpathlineto{\pgfqpoint{3.635894in}{1.384993in}}%
\pgfpathlineto{\pgfqpoint{3.644235in}{1.394264in}}%
\pgfpathlineto{\pgfqpoint{3.652568in}{1.403628in}}%
\pgfpathlineto{\pgfqpoint{3.660895in}{1.413079in}}%
\pgfpathlineto{\pgfqpoint{3.646855in}{1.415750in}}%
\pgfpathlineto{\pgfqpoint{3.632821in}{1.418496in}}%
\pgfpathlineto{\pgfqpoint{3.618793in}{1.421316in}}%
\pgfpathlineto{\pgfqpoint{3.604773in}{1.424211in}}%
\pgfpathlineto{\pgfqpoint{3.596431in}{1.415082in}}%
\pgfpathlineto{\pgfqpoint{3.588083in}{1.406045in}}%
\pgfpathlineto{\pgfqpoint{3.579727in}{1.397106in}}%
\pgfpathlineto{\pgfqpoint{3.571364in}{1.388269in}}%
\pgfpathclose%
\pgfusepath{fill}%
\end{pgfscope}%
\begin{pgfscope}%
\pgfpathrectangle{\pgfqpoint{1.150000in}{0.150000in}}{\pgfqpoint{5.700000in}{5.700000in}}%
\pgfusepath{clip}%
\pgfsetbuttcap%
\pgfsetroundjoin%
\definecolor{currentfill}{rgb}{0.283091,0.110553,0.431554}%
\pgfsetfillcolor{currentfill}%
\pgfsetfillopacity{0.700000}%
\pgfsetlinewidth{0.000000pt}%
\definecolor{currentstroke}{rgb}{0.000000,0.000000,0.000000}%
\pgfsetstrokecolor{currentstroke}%
\pgfsetdash{}{0pt}%
\pgfpathmoveto{\pgfqpoint{4.074737in}{1.556860in}}%
\pgfpathlineto{\pgfqpoint{4.088900in}{1.556841in}}%
\pgfpathlineto{\pgfqpoint{4.103072in}{1.556893in}}%
\pgfpathlineto{\pgfqpoint{4.117253in}{1.557017in}}%
\pgfpathlineto{\pgfqpoint{4.131443in}{1.557214in}}%
\pgfpathlineto{\pgfqpoint{4.139608in}{1.568902in}}%
\pgfpathlineto{\pgfqpoint{4.147768in}{1.580574in}}%
\pgfpathlineto{\pgfqpoint{4.155923in}{1.592227in}}%
\pgfpathlineto{\pgfqpoint{4.164072in}{1.603856in}}%
\pgfpathlineto{\pgfqpoint{4.149889in}{1.603431in}}%
\pgfpathlineto{\pgfqpoint{4.135716in}{1.603078in}}%
\pgfpathlineto{\pgfqpoint{4.121552in}{1.602797in}}%
\pgfpathlineto{\pgfqpoint{4.107397in}{1.602589in}}%
\pgfpathlineto{\pgfqpoint{4.099240in}{1.591180in}}%
\pgfpathlineto{\pgfqpoint{4.091077in}{1.579753in}}%
\pgfpathlineto{\pgfqpoint{4.082910in}{1.568312in}}%
\pgfpathlineto{\pgfqpoint{4.074737in}{1.556860in}}%
\pgfpathclose%
\pgfusepath{fill}%
\end{pgfscope}%
\begin{pgfscope}%
\pgfpathrectangle{\pgfqpoint{1.150000in}{0.150000in}}{\pgfqpoint{5.700000in}{5.700000in}}%
\pgfusepath{clip}%
\pgfsetbuttcap%
\pgfsetroundjoin%
\definecolor{currentfill}{rgb}{0.283072,0.130895,0.449241}%
\pgfsetfillcolor{currentfill}%
\pgfsetfillopacity{0.700000}%
\pgfsetlinewidth{0.000000pt}%
\definecolor{currentstroke}{rgb}{0.000000,0.000000,0.000000}%
\pgfsetstrokecolor{currentstroke}%
\pgfsetdash{}{0pt}%
\pgfpathmoveto{\pgfqpoint{4.164072in}{1.603856in}}%
\pgfpathlineto{\pgfqpoint{4.178264in}{1.604353in}}%
\pgfpathlineto{\pgfqpoint{4.192465in}{1.604922in}}%
\pgfpathlineto{\pgfqpoint{4.206676in}{1.605562in}}%
\pgfpathlineto{\pgfqpoint{4.220896in}{1.606274in}}%
\pgfpathlineto{\pgfqpoint{4.229033in}{1.618094in}}%
\pgfpathlineto{\pgfqpoint{4.237165in}{1.629880in}}%
\pgfpathlineto{\pgfqpoint{4.245292in}{1.641628in}}%
\pgfpathlineto{\pgfqpoint{4.253413in}{1.653337in}}%
\pgfpathlineto{\pgfqpoint{4.239199in}{1.652416in}}%
\pgfpathlineto{\pgfqpoint{4.224995in}{1.651568in}}%
\pgfpathlineto{\pgfqpoint{4.210801in}{1.650791in}}%
\pgfpathlineto{\pgfqpoint{4.196617in}{1.650086in}}%
\pgfpathlineto{\pgfqpoint{4.188488in}{1.638578in}}%
\pgfpathlineto{\pgfqpoint{4.180355in}{1.627035in}}%
\pgfpathlineto{\pgfqpoint{4.172216in}{1.615460in}}%
\pgfpathlineto{\pgfqpoint{4.164072in}{1.603856in}}%
\pgfpathclose%
\pgfusepath{fill}%
\end{pgfscope}%
\begin{pgfscope}%
\pgfpathrectangle{\pgfqpoint{1.150000in}{0.150000in}}{\pgfqpoint{5.700000in}{5.700000in}}%
\pgfusepath{clip}%
\pgfsetbuttcap%
\pgfsetroundjoin%
\definecolor{currentfill}{rgb}{0.268510,0.009605,0.335427}%
\pgfsetfillcolor{currentfill}%
\pgfsetfillopacity{0.700000}%
\pgfsetlinewidth{0.000000pt}%
\definecolor{currentstroke}{rgb}{0.000000,0.000000,0.000000}%
\pgfsetstrokecolor{currentstroke}%
\pgfsetdash{}{0pt}%
\pgfpathmoveto{\pgfqpoint{3.189706in}{1.394881in}}%
\pgfpathlineto{\pgfqpoint{3.203695in}{1.389003in}}%
\pgfpathlineto{\pgfqpoint{3.217688in}{1.383205in}}%
\pgfpathlineto{\pgfqpoint{3.231686in}{1.377486in}}%
\pgfpathlineto{\pgfqpoint{3.245688in}{1.371846in}}%
\pgfpathlineto{\pgfqpoint{3.254236in}{1.377440in}}%
\pgfpathlineto{\pgfqpoint{3.262773in}{1.383224in}}%
\pgfpathlineto{\pgfqpoint{3.271300in}{1.389191in}}%
\pgfpathlineto{\pgfqpoint{3.279817in}{1.395335in}}%
\pgfpathlineto{\pgfqpoint{3.265838in}{1.400583in}}%
\pgfpathlineto{\pgfqpoint{3.251864in}{1.405911in}}%
\pgfpathlineto{\pgfqpoint{3.237894in}{1.411317in}}%
\pgfpathlineto{\pgfqpoint{3.223929in}{1.416802in}}%
\pgfpathlineto{\pgfqpoint{3.215389in}{1.411042in}}%
\pgfpathlineto{\pgfqpoint{3.206839in}{1.405464in}}%
\pgfpathlineto{\pgfqpoint{3.198278in}{1.400075in}}%
\pgfpathlineto{\pgfqpoint{3.189706in}{1.394881in}}%
\pgfpathclose%
\pgfusepath{fill}%
\end{pgfscope}%
\begin{pgfscope}%
\pgfpathrectangle{\pgfqpoint{1.150000in}{0.150000in}}{\pgfqpoint{5.700000in}{5.700000in}}%
\pgfusepath{clip}%
\pgfsetbuttcap%
\pgfsetroundjoin%
\definecolor{currentfill}{rgb}{0.281924,0.089666,0.412415}%
\pgfsetfillcolor{currentfill}%
\pgfsetfillopacity{0.700000}%
\pgfsetlinewidth{0.000000pt}%
\definecolor{currentstroke}{rgb}{0.000000,0.000000,0.000000}%
\pgfsetstrokecolor{currentstroke}%
\pgfsetdash{}{0pt}%
\pgfpathmoveto{\pgfqpoint{3.985393in}{1.512803in}}%
\pgfpathlineto{\pgfqpoint{3.999530in}{1.512246in}}%
\pgfpathlineto{\pgfqpoint{4.013675in}{1.511761in}}%
\pgfpathlineto{\pgfqpoint{4.027829in}{1.511348in}}%
\pgfpathlineto{\pgfqpoint{4.041992in}{1.511007in}}%
\pgfpathlineto{\pgfqpoint{4.050186in}{1.522470in}}%
\pgfpathlineto{\pgfqpoint{4.058375in}{1.533936in}}%
\pgfpathlineto{\pgfqpoint{4.066559in}{1.545400in}}%
\pgfpathlineto{\pgfqpoint{4.074737in}{1.556860in}}%
\pgfpathlineto{\pgfqpoint{4.060583in}{1.556952in}}%
\pgfpathlineto{\pgfqpoint{4.046437in}{1.557116in}}%
\pgfpathlineto{\pgfqpoint{4.032301in}{1.557352in}}%
\pgfpathlineto{\pgfqpoint{4.018173in}{1.557661in}}%
\pgfpathlineto{\pgfqpoint{4.009986in}{1.546442in}}%
\pgfpathlineto{\pgfqpoint{4.001794in}{1.535224in}}%
\pgfpathlineto{\pgfqpoint{3.993596in}{1.524010in}}%
\pgfpathlineto{\pgfqpoint{3.985393in}{1.512803in}}%
\pgfpathclose%
\pgfusepath{fill}%
\end{pgfscope}%
\begin{pgfscope}%
\pgfpathrectangle{\pgfqpoint{1.150000in}{0.150000in}}{\pgfqpoint{5.700000in}{5.700000in}}%
\pgfusepath{clip}%
\pgfsetbuttcap%
\pgfsetroundjoin%
\definecolor{currentfill}{rgb}{0.281412,0.155834,0.469201}%
\pgfsetfillcolor{currentfill}%
\pgfsetfillopacity{0.700000}%
\pgfsetlinewidth{0.000000pt}%
\definecolor{currentstroke}{rgb}{0.000000,0.000000,0.000000}%
\pgfsetstrokecolor{currentstroke}%
\pgfsetdash{}{0pt}%
\pgfpathmoveto{\pgfqpoint{4.253413in}{1.653337in}}%
\pgfpathlineto{\pgfqpoint{4.267636in}{1.654329in}}%
\pgfpathlineto{\pgfqpoint{4.281869in}{1.655392in}}%
\pgfpathlineto{\pgfqpoint{4.296111in}{1.656528in}}%
\pgfpathlineto{\pgfqpoint{4.310364in}{1.657735in}}%
\pgfpathlineto{\pgfqpoint{4.318474in}{1.669596in}}%
\pgfpathlineto{\pgfqpoint{4.326577in}{1.681407in}}%
\pgfpathlineto{\pgfqpoint{4.334676in}{1.693165in}}%
\pgfpathlineto{\pgfqpoint{4.342769in}{1.704867in}}%
\pgfpathlineto{\pgfqpoint{4.328523in}{1.703472in}}%
\pgfpathlineto{\pgfqpoint{4.314287in}{1.702149in}}%
\pgfpathlineto{\pgfqpoint{4.300060in}{1.700898in}}%
\pgfpathlineto{\pgfqpoint{4.285844in}{1.699719in}}%
\pgfpathlineto{\pgfqpoint{4.277744in}{1.688196in}}%
\pgfpathlineto{\pgfqpoint{4.269639in}{1.676623in}}%
\pgfpathlineto{\pgfqpoint{4.261529in}{1.665002in}}%
\pgfpathlineto{\pgfqpoint{4.253413in}{1.653337in}}%
\pgfpathclose%
\pgfusepath{fill}%
\end{pgfscope}%
\begin{pgfscope}%
\pgfpathrectangle{\pgfqpoint{1.150000in}{0.150000in}}{\pgfqpoint{5.700000in}{5.700000in}}%
\pgfusepath{clip}%
\pgfsetbuttcap%
\pgfsetroundjoin%
\definecolor{currentfill}{rgb}{0.267004,0.004874,0.329415}%
\pgfsetfillcolor{currentfill}%
\pgfsetfillopacity{0.700000}%
\pgfsetlinewidth{0.000000pt}%
\definecolor{currentstroke}{rgb}{0.000000,0.000000,0.000000}%
\pgfsetstrokecolor{currentstroke}%
\pgfsetdash{}{0pt}%
\pgfpathmoveto{\pgfqpoint{3.335782in}{1.375125in}}%
\pgfpathlineto{\pgfqpoint{3.349785in}{1.370266in}}%
\pgfpathlineto{\pgfqpoint{3.363794in}{1.365485in}}%
\pgfpathlineto{\pgfqpoint{3.377808in}{1.360781in}}%
\pgfpathlineto{\pgfqpoint{3.391827in}{1.356154in}}%
\pgfpathlineto{\pgfqpoint{3.400292in}{1.363228in}}%
\pgfpathlineto{\pgfqpoint{3.408748in}{1.370456in}}%
\pgfpathlineto{\pgfqpoint{3.417194in}{1.377835in}}%
\pgfpathlineto{\pgfqpoint{3.425632in}{1.385357in}}%
\pgfpathlineto{\pgfqpoint{3.411633in}{1.389614in}}%
\pgfpathlineto{\pgfqpoint{3.397639in}{1.393947in}}%
\pgfpathlineto{\pgfqpoint{3.383650in}{1.398357in}}%
\pgfpathlineto{\pgfqpoint{3.369667in}{1.402845in}}%
\pgfpathlineto{\pgfqpoint{3.361209in}{1.395685in}}%
\pgfpathlineto{\pgfqpoint{3.352743in}{1.388675in}}%
\pgfpathlineto{\pgfqpoint{3.344267in}{1.381819in}}%
\pgfpathlineto{\pgfqpoint{3.335782in}{1.375125in}}%
\pgfpathclose%
\pgfusepath{fill}%
\end{pgfscope}%
\begin{pgfscope}%
\pgfpathrectangle{\pgfqpoint{1.150000in}{0.150000in}}{\pgfqpoint{5.700000in}{5.700000in}}%
\pgfusepath{clip}%
\pgfsetbuttcap%
\pgfsetroundjoin%
\definecolor{currentfill}{rgb}{0.278012,0.180367,0.486697}%
\pgfsetfillcolor{currentfill}%
\pgfsetfillopacity{0.700000}%
\pgfsetlinewidth{0.000000pt}%
\definecolor{currentstroke}{rgb}{0.000000,0.000000,0.000000}%
\pgfsetstrokecolor{currentstroke}%
\pgfsetdash{}{0pt}%
\pgfpathmoveto{\pgfqpoint{4.342769in}{1.704867in}}%
\pgfpathlineto{\pgfqpoint{4.357026in}{1.706333in}}%
\pgfpathlineto{\pgfqpoint{4.371292in}{1.707870in}}%
\pgfpathlineto{\pgfqpoint{4.385569in}{1.709480in}}%
\pgfpathlineto{\pgfqpoint{4.399856in}{1.711160in}}%
\pgfpathlineto{\pgfqpoint{4.407938in}{1.722979in}}%
\pgfpathlineto{\pgfqpoint{4.416014in}{1.734732in}}%
\pgfpathlineto{\pgfqpoint{4.424085in}{1.746418in}}%
\pgfpathlineto{\pgfqpoint{4.432150in}{1.758034in}}%
\pgfpathlineto{\pgfqpoint{4.417868in}{1.756186in}}%
\pgfpathlineto{\pgfqpoint{4.403598in}{1.754410in}}%
\pgfpathlineto{\pgfqpoint{4.389337in}{1.752706in}}%
\pgfpathlineto{\pgfqpoint{4.375087in}{1.751073in}}%
\pgfpathlineto{\pgfqpoint{4.367016in}{1.739616in}}%
\pgfpathlineto{\pgfqpoint{4.358939in}{1.728095in}}%
\pgfpathlineto{\pgfqpoint{4.350857in}{1.716511in}}%
\pgfpathlineto{\pgfqpoint{4.342769in}{1.704867in}}%
\pgfpathclose%
\pgfusepath{fill}%
\end{pgfscope}%
\begin{pgfscope}%
\pgfpathrectangle{\pgfqpoint{1.150000in}{0.150000in}}{\pgfqpoint{5.700000in}{5.700000in}}%
\pgfusepath{clip}%
\pgfsetbuttcap%
\pgfsetroundjoin%
\definecolor{currentfill}{rgb}{0.279566,0.067836,0.391917}%
\pgfsetfillcolor{currentfill}%
\pgfsetfillopacity{0.700000}%
\pgfsetlinewidth{0.000000pt}%
\definecolor{currentstroke}{rgb}{0.000000,0.000000,0.000000}%
\pgfsetstrokecolor{currentstroke}%
\pgfsetdash{}{0pt}%
\pgfpathmoveto{\pgfqpoint{3.896025in}{1.472161in}}%
\pgfpathlineto{\pgfqpoint{3.910138in}{1.471044in}}%
\pgfpathlineto{\pgfqpoint{3.924259in}{1.470000in}}%
\pgfpathlineto{\pgfqpoint{3.938389in}{1.469028in}}%
\pgfpathlineto{\pgfqpoint{3.952527in}{1.468129in}}%
\pgfpathlineto{\pgfqpoint{3.960752in}{1.479267in}}%
\pgfpathlineto{\pgfqpoint{3.968971in}{1.490428in}}%
\pgfpathlineto{\pgfqpoint{3.977185in}{1.501608in}}%
\pgfpathlineto{\pgfqpoint{3.985393in}{1.512803in}}%
\pgfpathlineto{\pgfqpoint{3.971265in}{1.513433in}}%
\pgfpathlineto{\pgfqpoint{3.957145in}{1.514135in}}%
\pgfpathlineto{\pgfqpoint{3.943034in}{1.514911in}}%
\pgfpathlineto{\pgfqpoint{3.928931in}{1.515758in}}%
\pgfpathlineto{\pgfqpoint{3.920713in}{1.504825in}}%
\pgfpathlineto{\pgfqpoint{3.912489in}{1.493911in}}%
\pgfpathlineto{\pgfqpoint{3.904260in}{1.483022in}}%
\pgfpathlineto{\pgfqpoint{3.896025in}{1.472161in}}%
\pgfpathclose%
\pgfusepath{fill}%
\end{pgfscope}%
\begin{pgfscope}%
\pgfpathrectangle{\pgfqpoint{1.150000in}{0.150000in}}{\pgfqpoint{5.700000in}{5.700000in}}%
\pgfusepath{clip}%
\pgfsetbuttcap%
\pgfsetroundjoin%
\definecolor{currentfill}{rgb}{0.272594,0.025563,0.353093}%
\pgfsetfillcolor{currentfill}%
\pgfsetfillopacity{0.700000}%
\pgfsetlinewidth{0.000000pt}%
\definecolor{currentstroke}{rgb}{0.000000,0.000000,0.000000}%
\pgfsetstrokecolor{currentstroke}%
\pgfsetdash{}{0pt}%
\pgfpathmoveto{\pgfqpoint{3.043322in}{1.429399in}}%
\pgfpathlineto{\pgfqpoint{3.057307in}{1.422462in}}%
\pgfpathlineto{\pgfqpoint{3.071295in}{1.415606in}}%
\pgfpathlineto{\pgfqpoint{3.085287in}{1.408833in}}%
\pgfpathlineto{\pgfqpoint{3.099283in}{1.402140in}}%
\pgfpathlineto{\pgfqpoint{3.107928in}{1.406071in}}%
\pgfpathlineto{\pgfqpoint{3.116561in}{1.410227in}}%
\pgfpathlineto{\pgfqpoint{3.125182in}{1.414603in}}%
\pgfpathlineto{\pgfqpoint{3.133791in}{1.419190in}}%
\pgfpathlineto{\pgfqpoint{3.119823in}{1.425469in}}%
\pgfpathlineto{\pgfqpoint{3.105858in}{1.431829in}}%
\pgfpathlineto{\pgfqpoint{3.091897in}{1.438271in}}%
\pgfpathlineto{\pgfqpoint{3.077940in}{1.444794in}}%
\pgfpathlineto{\pgfqpoint{3.069305in}{1.440612in}}%
\pgfpathlineto{\pgfqpoint{3.060657in}{1.436648in}}%
\pgfpathlineto{\pgfqpoint{3.051996in}{1.432908in}}%
\pgfpathlineto{\pgfqpoint{3.043322in}{1.429399in}}%
\pgfpathclose%
\pgfusepath{fill}%
\end{pgfscope}%
\begin{pgfscope}%
\pgfpathrectangle{\pgfqpoint{1.150000in}{0.150000in}}{\pgfqpoint{5.700000in}{5.700000in}}%
\pgfusepath{clip}%
\pgfsetbuttcap%
\pgfsetroundjoin%
\definecolor{currentfill}{rgb}{0.271828,0.209303,0.504434}%
\pgfsetfillcolor{currentfill}%
\pgfsetfillopacity{0.700000}%
\pgfsetlinewidth{0.000000pt}%
\definecolor{currentstroke}{rgb}{0.000000,0.000000,0.000000}%
\pgfsetstrokecolor{currentstroke}%
\pgfsetdash{}{0pt}%
\pgfpathmoveto{\pgfqpoint{4.432150in}{1.758034in}}%
\pgfpathlineto{\pgfqpoint{4.446441in}{1.759952in}}%
\pgfpathlineto{\pgfqpoint{4.460743in}{1.761943in}}%
\pgfpathlineto{\pgfqpoint{4.475056in}{1.764004in}}%
\pgfpathlineto{\pgfqpoint{4.489380in}{1.766137in}}%
\pgfpathlineto{\pgfqpoint{4.497433in}{1.777835in}}%
\pgfpathlineto{\pgfqpoint{4.505481in}{1.789453in}}%
\pgfpathlineto{\pgfqpoint{4.513522in}{1.800990in}}%
\pgfpathlineto{\pgfqpoint{4.521559in}{1.812444in}}%
\pgfpathlineto{\pgfqpoint{4.507241in}{1.810165in}}%
\pgfpathlineto{\pgfqpoint{4.492934in}{1.807957in}}%
\pgfpathlineto{\pgfqpoint{4.478638in}{1.805821in}}%
\pgfpathlineto{\pgfqpoint{4.464353in}{1.803756in}}%
\pgfpathlineto{\pgfqpoint{4.456310in}{1.792440in}}%
\pgfpathlineto{\pgfqpoint{4.448263in}{1.781047in}}%
\pgfpathlineto{\pgfqpoint{4.440209in}{1.769577in}}%
\pgfpathlineto{\pgfqpoint{4.432150in}{1.758034in}}%
\pgfpathclose%
\pgfusepath{fill}%
\end{pgfscope}%
\begin{pgfscope}%
\pgfpathrectangle{\pgfqpoint{1.150000in}{0.150000in}}{\pgfqpoint{5.700000in}{5.700000in}}%
\pgfusepath{clip}%
\pgfsetbuttcap%
\pgfsetroundjoin%
\definecolor{currentfill}{rgb}{0.276022,0.044167,0.370164}%
\pgfsetfillcolor{currentfill}%
\pgfsetfillopacity{0.700000}%
\pgfsetlinewidth{0.000000pt}%
\definecolor{currentstroke}{rgb}{0.000000,0.000000,0.000000}%
\pgfsetstrokecolor{currentstroke}%
\pgfsetdash{}{0pt}%
\pgfpathmoveto{\pgfqpoint{3.806611in}{1.435432in}}%
\pgfpathlineto{\pgfqpoint{3.820704in}{1.433734in}}%
\pgfpathlineto{\pgfqpoint{3.834804in}{1.432109in}}%
\pgfpathlineto{\pgfqpoint{3.848912in}{1.430557in}}%
\pgfpathlineto{\pgfqpoint{3.863028in}{1.429078in}}%
\pgfpathlineto{\pgfqpoint{3.871286in}{1.439786in}}%
\pgfpathlineto{\pgfqpoint{3.879538in}{1.450539in}}%
\pgfpathlineto{\pgfqpoint{3.887784in}{1.461332in}}%
\pgfpathlineto{\pgfqpoint{3.896025in}{1.472161in}}%
\pgfpathlineto{\pgfqpoint{3.881920in}{1.473351in}}%
\pgfpathlineto{\pgfqpoint{3.867823in}{1.474613in}}%
\pgfpathlineto{\pgfqpoint{3.853733in}{1.475949in}}%
\pgfpathlineto{\pgfqpoint{3.839652in}{1.477357in}}%
\pgfpathlineto{\pgfqpoint{3.831400in}{1.466810in}}%
\pgfpathlineto{\pgfqpoint{3.823143in}{1.456304in}}%
\pgfpathlineto{\pgfqpoint{3.814880in}{1.445843in}}%
\pgfpathlineto{\pgfqpoint{3.806611in}{1.435432in}}%
\pgfpathclose%
\pgfusepath{fill}%
\end{pgfscope}%
\begin{pgfscope}%
\pgfpathrectangle{\pgfqpoint{1.150000in}{0.150000in}}{\pgfqpoint{5.700000in}{5.700000in}}%
\pgfusepath{clip}%
\pgfsetbuttcap%
\pgfsetroundjoin%
\definecolor{currentfill}{rgb}{0.267004,0.004874,0.329415}%
\pgfsetfillcolor{currentfill}%
\pgfsetfillopacity{0.700000}%
\pgfsetlinewidth{0.000000pt}%
\definecolor{currentstroke}{rgb}{0.000000,0.000000,0.000000}%
\pgfsetstrokecolor{currentstroke}%
\pgfsetdash{}{0pt}%
\pgfpathmoveto{\pgfqpoint{3.481687in}{1.369095in}}%
\pgfpathlineto{\pgfqpoint{3.495715in}{1.365219in}}%
\pgfpathlineto{\pgfqpoint{3.509749in}{1.361419in}}%
\pgfpathlineto{\pgfqpoint{3.523789in}{1.357694in}}%
\pgfpathlineto{\pgfqpoint{3.537835in}{1.354044in}}%
\pgfpathlineto{\pgfqpoint{3.546229in}{1.362421in}}%
\pgfpathlineto{\pgfqpoint{3.554615in}{1.370921in}}%
\pgfpathlineto{\pgfqpoint{3.562993in}{1.379539in}}%
\pgfpathlineto{\pgfqpoint{3.571364in}{1.388269in}}%
\pgfpathlineto{\pgfqpoint{3.557334in}{1.391568in}}%
\pgfpathlineto{\pgfqpoint{3.543311in}{1.394943in}}%
\pgfpathlineto{\pgfqpoint{3.529294in}{1.398393in}}%
\pgfpathlineto{\pgfqpoint{3.515283in}{1.401919in}}%
\pgfpathlineto{\pgfqpoint{3.506896in}{1.393531in}}%
\pgfpathlineto{\pgfqpoint{3.498501in}{1.385261in}}%
\pgfpathlineto{\pgfqpoint{3.490098in}{1.377114in}}%
\pgfpathlineto{\pgfqpoint{3.481687in}{1.369095in}}%
\pgfpathclose%
\pgfusepath{fill}%
\end{pgfscope}%
\begin{pgfscope}%
\pgfpathrectangle{\pgfqpoint{1.150000in}{0.150000in}}{\pgfqpoint{5.700000in}{5.700000in}}%
\pgfusepath{clip}%
\pgfsetbuttcap%
\pgfsetroundjoin%
\definecolor{currentfill}{rgb}{0.265145,0.232956,0.516599}%
\pgfsetfillcolor{currentfill}%
\pgfsetfillopacity{0.700000}%
\pgfsetlinewidth{0.000000pt}%
\definecolor{currentstroke}{rgb}{0.000000,0.000000,0.000000}%
\pgfsetstrokecolor{currentstroke}%
\pgfsetdash{}{0pt}%
\pgfpathmoveto{\pgfqpoint{4.521559in}{1.812444in}}%
\pgfpathlineto{\pgfqpoint{4.535887in}{1.814794in}}%
\pgfpathlineto{\pgfqpoint{4.550226in}{1.817216in}}%
\pgfpathlineto{\pgfqpoint{4.564577in}{1.819709in}}%
\pgfpathlineto{\pgfqpoint{4.578938in}{1.822274in}}%
\pgfpathlineto{\pgfqpoint{4.586962in}{1.833775in}}%
\pgfpathlineto{\pgfqpoint{4.594981in}{1.845186in}}%
\pgfpathlineto{\pgfqpoint{4.602993in}{1.856503in}}%
\pgfpathlineto{\pgfqpoint{4.610999in}{1.867725in}}%
\pgfpathlineto{\pgfqpoint{4.596643in}{1.865035in}}%
\pgfpathlineto{\pgfqpoint{4.582299in}{1.862417in}}%
\pgfpathlineto{\pgfqpoint{4.567966in}{1.859871in}}%
\pgfpathlineto{\pgfqpoint{4.553643in}{1.857395in}}%
\pgfpathlineto{\pgfqpoint{4.545631in}{1.846290in}}%
\pgfpathlineto{\pgfqpoint{4.537613in}{1.835095in}}%
\pgfpathlineto{\pgfqpoint{4.529589in}{1.823813in}}%
\pgfpathlineto{\pgfqpoint{4.521559in}{1.812444in}}%
\pgfpathclose%
\pgfusepath{fill}%
\end{pgfscope}%
\begin{pgfscope}%
\pgfpathrectangle{\pgfqpoint{1.150000in}{0.150000in}}{\pgfqpoint{5.700000in}{5.700000in}}%
\pgfusepath{clip}%
\pgfsetbuttcap%
\pgfsetroundjoin%
\definecolor{currentfill}{rgb}{0.255645,0.260703,0.528312}%
\pgfsetfillcolor{currentfill}%
\pgfsetfillopacity{0.700000}%
\pgfsetlinewidth{0.000000pt}%
\definecolor{currentstroke}{rgb}{0.000000,0.000000,0.000000}%
\pgfsetstrokecolor{currentstroke}%
\pgfsetdash{}{0pt}%
\pgfpathmoveto{\pgfqpoint{4.610999in}{1.867725in}}%
\pgfpathlineto{\pgfqpoint{4.625365in}{1.870485in}}%
\pgfpathlineto{\pgfqpoint{4.639743in}{1.873317in}}%
\pgfpathlineto{\pgfqpoint{4.654133in}{1.876221in}}%
\pgfpathlineto{\pgfqpoint{4.668533in}{1.879195in}}%
\pgfpathlineto{\pgfqpoint{4.676527in}{1.890433in}}%
\pgfpathlineto{\pgfqpoint{4.684515in}{1.901567in}}%
\pgfpathlineto{\pgfqpoint{4.692496in}{1.912598in}}%
\pgfpathlineto{\pgfqpoint{4.700470in}{1.923523in}}%
\pgfpathlineto{\pgfqpoint{4.686076in}{1.920445in}}%
\pgfpathlineto{\pgfqpoint{4.671692in}{1.917438in}}%
\pgfpathlineto{\pgfqpoint{4.657321in}{1.914503in}}%
\pgfpathlineto{\pgfqpoint{4.642960in}{1.911638in}}%
\pgfpathlineto{\pgfqpoint{4.634979in}{1.900809in}}%
\pgfpathlineto{\pgfqpoint{4.626992in}{1.889879in}}%
\pgfpathlineto{\pgfqpoint{4.618998in}{1.878850in}}%
\pgfpathlineto{\pgfqpoint{4.610999in}{1.867725in}}%
\pgfpathclose%
\pgfusepath{fill}%
\end{pgfscope}%
\begin{pgfscope}%
\pgfpathrectangle{\pgfqpoint{1.150000in}{0.150000in}}{\pgfqpoint{5.700000in}{5.700000in}}%
\pgfusepath{clip}%
\pgfsetbuttcap%
\pgfsetroundjoin%
\definecolor{currentfill}{rgb}{0.273809,0.031497,0.358853}%
\pgfsetfillcolor{currentfill}%
\pgfsetfillopacity{0.700000}%
\pgfsetlinewidth{0.000000pt}%
\definecolor{currentstroke}{rgb}{0.000000,0.000000,0.000000}%
\pgfsetstrokecolor{currentstroke}%
\pgfsetdash{}{0pt}%
\pgfpathmoveto{\pgfqpoint{3.717127in}{1.403137in}}%
\pgfpathlineto{\pgfqpoint{3.731203in}{1.400836in}}%
\pgfpathlineto{\pgfqpoint{3.745286in}{1.398608in}}%
\pgfpathlineto{\pgfqpoint{3.759376in}{1.396454in}}%
\pgfpathlineto{\pgfqpoint{3.773473in}{1.394373in}}%
\pgfpathlineto{\pgfqpoint{3.781767in}{1.404541in}}%
\pgfpathlineto{\pgfqpoint{3.790054in}{1.414777in}}%
\pgfpathlineto{\pgfqpoint{3.798336in}{1.425075in}}%
\pgfpathlineto{\pgfqpoint{3.806611in}{1.435432in}}%
\pgfpathlineto{\pgfqpoint{3.792526in}{1.437203in}}%
\pgfpathlineto{\pgfqpoint{3.778448in}{1.439048in}}%
\pgfpathlineto{\pgfqpoint{3.764378in}{1.440966in}}%
\pgfpathlineto{\pgfqpoint{3.750315in}{1.442957in}}%
\pgfpathlineto{\pgfqpoint{3.742028in}{1.432902in}}%
\pgfpathlineto{\pgfqpoint{3.733734in}{1.422911in}}%
\pgfpathlineto{\pgfqpoint{3.725434in}{1.412987in}}%
\pgfpathlineto{\pgfqpoint{3.717127in}{1.403137in}}%
\pgfpathclose%
\pgfusepath{fill}%
\end{pgfscope}%
\begin{pgfscope}%
\pgfpathrectangle{\pgfqpoint{1.150000in}{0.150000in}}{\pgfqpoint{5.700000in}{5.700000in}}%
\pgfusepath{clip}%
\pgfsetbuttcap%
\pgfsetroundjoin%
\definecolor{currentfill}{rgb}{0.246811,0.283237,0.535941}%
\pgfsetfillcolor{currentfill}%
\pgfsetfillopacity{0.700000}%
\pgfsetlinewidth{0.000000pt}%
\definecolor{currentstroke}{rgb}{0.000000,0.000000,0.000000}%
\pgfsetstrokecolor{currentstroke}%
\pgfsetdash{}{0pt}%
\pgfpathmoveto{\pgfqpoint{4.700470in}{1.923523in}}%
\pgfpathlineto{\pgfqpoint{4.714876in}{1.926673in}}%
\pgfpathlineto{\pgfqpoint{4.729294in}{1.929894in}}%
\pgfpathlineto{\pgfqpoint{4.743724in}{1.933186in}}%
\pgfpathlineto{\pgfqpoint{4.758165in}{1.936550in}}%
\pgfpathlineto{\pgfqpoint{4.766127in}{1.947459in}}%
\pgfpathlineto{\pgfqpoint{4.774082in}{1.958256in}}%
\pgfpathlineto{\pgfqpoint{4.782030in}{1.968939in}}%
\pgfpathlineto{\pgfqpoint{4.789971in}{1.979508in}}%
\pgfpathlineto{\pgfqpoint{4.775536in}{1.976062in}}%
\pgfpathlineto{\pgfqpoint{4.761113in}{1.972688in}}%
\pgfpathlineto{\pgfqpoint{4.746702in}{1.969385in}}%
\pgfpathlineto{\pgfqpoint{4.732303in}{1.966153in}}%
\pgfpathlineto{\pgfqpoint{4.724354in}{1.955658in}}%
\pgfpathlineto{\pgfqpoint{4.716400in}{1.945055in}}%
\pgfpathlineto{\pgfqpoint{4.708438in}{1.934343in}}%
\pgfpathlineto{\pgfqpoint{4.700470in}{1.923523in}}%
\pgfpathclose%
\pgfusepath{fill}%
\end{pgfscope}%
\begin{pgfscope}%
\pgfpathrectangle{\pgfqpoint{1.150000in}{0.150000in}}{\pgfqpoint{5.700000in}{5.700000in}}%
\pgfusepath{clip}%
\pgfsetbuttcap%
\pgfsetroundjoin%
\definecolor{currentfill}{rgb}{0.235526,0.309527,0.542944}%
\pgfsetfillcolor{currentfill}%
\pgfsetfillopacity{0.700000}%
\pgfsetlinewidth{0.000000pt}%
\definecolor{currentstroke}{rgb}{0.000000,0.000000,0.000000}%
\pgfsetstrokecolor{currentstroke}%
\pgfsetdash{}{0pt}%
\pgfpathmoveto{\pgfqpoint{4.789971in}{1.979508in}}%
\pgfpathlineto{\pgfqpoint{4.804418in}{1.983025in}}%
\pgfpathlineto{\pgfqpoint{4.818876in}{1.986613in}}%
\pgfpathlineto{\pgfqpoint{4.833347in}{1.990273in}}%
\pgfpathlineto{\pgfqpoint{4.847830in}{1.994004in}}%
\pgfpathlineto{\pgfqpoint{4.855757in}{2.004527in}}%
\pgfpathlineto{\pgfqpoint{4.863678in}{2.014928in}}%
\pgfpathlineto{\pgfqpoint{4.871591in}{2.025209in}}%
\pgfpathlineto{\pgfqpoint{4.879497in}{2.035367in}}%
\pgfpathlineto{\pgfqpoint{4.865022in}{2.031575in}}%
\pgfpathlineto{\pgfqpoint{4.850558in}{2.027855in}}%
\pgfpathlineto{\pgfqpoint{4.836106in}{2.024206in}}%
\pgfpathlineto{\pgfqpoint{4.821667in}{2.020628in}}%
\pgfpathlineto{\pgfqpoint{4.813753in}{2.010523in}}%
\pgfpathlineto{\pgfqpoint{4.805833in}{2.000301in}}%
\pgfpathlineto{\pgfqpoint{4.797905in}{1.989962in}}%
\pgfpathlineto{\pgfqpoint{4.789971in}{1.979508in}}%
\pgfpathclose%
\pgfusepath{fill}%
\end{pgfscope}%
\begin{pgfscope}%
\pgfpathrectangle{\pgfqpoint{1.150000in}{0.150000in}}{\pgfqpoint{5.700000in}{5.700000in}}%
\pgfusepath{clip}%
\pgfsetbuttcap%
\pgfsetroundjoin%
\definecolor{currentfill}{rgb}{0.268510,0.009605,0.335427}%
\pgfsetfillcolor{currentfill}%
\pgfsetfillopacity{0.700000}%
\pgfsetlinewidth{0.000000pt}%
\definecolor{currentstroke}{rgb}{0.000000,0.000000,0.000000}%
\pgfsetstrokecolor{currentstroke}%
\pgfsetdash{}{0pt}%
\pgfpathmoveto{\pgfqpoint{3.245688in}{1.371846in}}%
\pgfpathlineto{\pgfqpoint{3.259695in}{1.366284in}}%
\pgfpathlineto{\pgfqpoint{3.273706in}{1.360801in}}%
\pgfpathlineto{\pgfqpoint{3.287722in}{1.355396in}}%
\pgfpathlineto{\pgfqpoint{3.301743in}{1.350068in}}%
\pgfpathlineto{\pgfqpoint{3.310267in}{1.356062in}}%
\pgfpathlineto{\pgfqpoint{3.318782in}{1.362240in}}%
\pgfpathlineto{\pgfqpoint{3.327287in}{1.368596in}}%
\pgfpathlineto{\pgfqpoint{3.335782in}{1.375125in}}%
\pgfpathlineto{\pgfqpoint{3.321783in}{1.380060in}}%
\pgfpathlineto{\pgfqpoint{3.307790in}{1.385074in}}%
\pgfpathlineto{\pgfqpoint{3.293801in}{1.390165in}}%
\pgfpathlineto{\pgfqpoint{3.279817in}{1.395335in}}%
\pgfpathlineto{\pgfqpoint{3.271300in}{1.389191in}}%
\pgfpathlineto{\pgfqpoint{3.262773in}{1.383224in}}%
\pgfpathlineto{\pgfqpoint{3.254236in}{1.377440in}}%
\pgfpathlineto{\pgfqpoint{3.245688in}{1.371846in}}%
\pgfpathclose%
\pgfusepath{fill}%
\end{pgfscope}%
\begin{pgfscope}%
\pgfpathrectangle{\pgfqpoint{1.150000in}{0.150000in}}{\pgfqpoint{5.700000in}{5.700000in}}%
\pgfusepath{clip}%
\pgfsetbuttcap%
\pgfsetroundjoin%
\definecolor{currentfill}{rgb}{0.269944,0.014625,0.341379}%
\pgfsetfillcolor{currentfill}%
\pgfsetfillopacity{0.700000}%
\pgfsetlinewidth{0.000000pt}%
\definecolor{currentstroke}{rgb}{0.000000,0.000000,0.000000}%
\pgfsetstrokecolor{currentstroke}%
\pgfsetdash{}{0pt}%
\pgfpathmoveto{\pgfqpoint{3.627546in}{1.375818in}}%
\pgfpathlineto{\pgfqpoint{3.641608in}{1.372891in}}%
\pgfpathlineto{\pgfqpoint{3.655677in}{1.370039in}}%
\pgfpathlineto{\pgfqpoint{3.669752in}{1.367260in}}%
\pgfpathlineto{\pgfqpoint{3.683835in}{1.364555in}}%
\pgfpathlineto{\pgfqpoint{3.692168in}{1.374068in}}%
\pgfpathlineto{\pgfqpoint{3.700494in}{1.383672in}}%
\pgfpathlineto{\pgfqpoint{3.708814in}{1.393363in}}%
\pgfpathlineto{\pgfqpoint{3.717127in}{1.403137in}}%
\pgfpathlineto{\pgfqpoint{3.703059in}{1.405511in}}%
\pgfpathlineto{\pgfqpoint{3.688997in}{1.407960in}}%
\pgfpathlineto{\pgfqpoint{3.674943in}{1.410483in}}%
\pgfpathlineto{\pgfqpoint{3.660895in}{1.413079in}}%
\pgfpathlineto{\pgfqpoint{3.652568in}{1.403628in}}%
\pgfpathlineto{\pgfqpoint{3.644235in}{1.394264in}}%
\pgfpathlineto{\pgfqpoint{3.635894in}{1.384993in}}%
\pgfpathlineto{\pgfqpoint{3.627546in}{1.375818in}}%
\pgfpathclose%
\pgfusepath{fill}%
\end{pgfscope}%
\begin{pgfscope}%
\pgfpathrectangle{\pgfqpoint{1.150000in}{0.150000in}}{\pgfqpoint{5.700000in}{5.700000in}}%
\pgfusepath{clip}%
\pgfsetbuttcap%
\pgfsetroundjoin%
\definecolor{currentfill}{rgb}{0.223925,0.334994,0.548053}%
\pgfsetfillcolor{currentfill}%
\pgfsetfillopacity{0.700000}%
\pgfsetlinewidth{0.000000pt}%
\definecolor{currentstroke}{rgb}{0.000000,0.000000,0.000000}%
\pgfsetstrokecolor{currentstroke}%
\pgfsetdash{}{0pt}%
\pgfpathmoveto{\pgfqpoint{4.879497in}{2.035367in}}%
\pgfpathlineto{\pgfqpoint{4.893985in}{2.039229in}}%
\pgfpathlineto{\pgfqpoint{4.908486in}{2.043164in}}%
\pgfpathlineto{\pgfqpoint{4.922998in}{2.047169in}}%
\pgfpathlineto{\pgfqpoint{4.937523in}{2.051246in}}%
\pgfpathlineto{\pgfqpoint{4.945415in}{2.061329in}}%
\pgfpathlineto{\pgfqpoint{4.953299in}{2.071283in}}%
\pgfpathlineto{\pgfqpoint{4.961175in}{2.081110in}}%
\pgfpathlineto{\pgfqpoint{4.969044in}{2.090808in}}%
\pgfpathlineto{\pgfqpoint{4.954526in}{2.086692in}}%
\pgfpathlineto{\pgfqpoint{4.940021in}{2.082648in}}%
\pgfpathlineto{\pgfqpoint{4.925529in}{2.078675in}}%
\pgfpathlineto{\pgfqpoint{4.911048in}{2.074773in}}%
\pgfpathlineto{\pgfqpoint{4.903172in}{2.065106in}}%
\pgfpathlineto{\pgfqpoint{4.895288in}{2.055316in}}%
\pgfpathlineto{\pgfqpoint{4.887396in}{2.045402in}}%
\pgfpathlineto{\pgfqpoint{4.879497in}{2.035367in}}%
\pgfpathclose%
\pgfusepath{fill}%
\end{pgfscope}%
\begin{pgfscope}%
\pgfpathrectangle{\pgfqpoint{1.150000in}{0.150000in}}{\pgfqpoint{5.700000in}{5.700000in}}%
\pgfusepath{clip}%
\pgfsetbuttcap%
\pgfsetroundjoin%
\definecolor{currentfill}{rgb}{0.267004,0.004874,0.329415}%
\pgfsetfillcolor{currentfill}%
\pgfsetfillopacity{0.700000}%
\pgfsetlinewidth{0.000000pt}%
\definecolor{currentstroke}{rgb}{0.000000,0.000000,0.000000}%
\pgfsetstrokecolor{currentstroke}%
\pgfsetdash{}{0pt}%
\pgfpathmoveto{\pgfqpoint{3.391827in}{1.356154in}}%
\pgfpathlineto{\pgfqpoint{3.405852in}{1.351603in}}%
\pgfpathlineto{\pgfqpoint{3.419882in}{1.347129in}}%
\pgfpathlineto{\pgfqpoint{3.433917in}{1.342731in}}%
\pgfpathlineto{\pgfqpoint{3.447959in}{1.338408in}}%
\pgfpathlineto{\pgfqpoint{3.456403in}{1.345860in}}%
\pgfpathlineto{\pgfqpoint{3.464840in}{1.353463in}}%
\pgfpathlineto{\pgfqpoint{3.473267in}{1.361209in}}%
\pgfpathlineto{\pgfqpoint{3.481687in}{1.369095in}}%
\pgfpathlineto{\pgfqpoint{3.467665in}{1.373047in}}%
\pgfpathlineto{\pgfqpoint{3.453648in}{1.377074in}}%
\pgfpathlineto{\pgfqpoint{3.439637in}{1.381178in}}%
\pgfpathlineto{\pgfqpoint{3.425632in}{1.385357in}}%
\pgfpathlineto{\pgfqpoint{3.417194in}{1.377835in}}%
\pgfpathlineto{\pgfqpoint{3.408748in}{1.370456in}}%
\pgfpathlineto{\pgfqpoint{3.400292in}{1.363228in}}%
\pgfpathlineto{\pgfqpoint{3.391827in}{1.356154in}}%
\pgfpathclose%
\pgfusepath{fill}%
\end{pgfscope}%
\begin{pgfscope}%
\pgfpathrectangle{\pgfqpoint{1.150000in}{0.150000in}}{\pgfqpoint{5.700000in}{5.700000in}}%
\pgfusepath{clip}%
\pgfsetbuttcap%
\pgfsetroundjoin%
\definecolor{currentfill}{rgb}{0.271305,0.019942,0.347269}%
\pgfsetfillcolor{currentfill}%
\pgfsetfillopacity{0.700000}%
\pgfsetlinewidth{0.000000pt}%
\definecolor{currentstroke}{rgb}{0.000000,0.000000,0.000000}%
\pgfsetstrokecolor{currentstroke}%
\pgfsetdash{}{0pt}%
\pgfpathmoveto{\pgfqpoint{3.099283in}{1.402140in}}%
\pgfpathlineto{\pgfqpoint{3.113282in}{1.395529in}}%
\pgfpathlineto{\pgfqpoint{3.127285in}{1.388998in}}%
\pgfpathlineto{\pgfqpoint{3.141292in}{1.382547in}}%
\pgfpathlineto{\pgfqpoint{3.155302in}{1.376177in}}%
\pgfpathlineto{\pgfqpoint{3.163921in}{1.380529in}}%
\pgfpathlineto{\pgfqpoint{3.172527in}{1.385102in}}%
\pgfpathlineto{\pgfqpoint{3.181122in}{1.389888in}}%
\pgfpathlineto{\pgfqpoint{3.189706in}{1.394881in}}%
\pgfpathlineto{\pgfqpoint{3.175721in}{1.400838in}}%
\pgfpathlineto{\pgfqpoint{3.161740in}{1.406875in}}%
\pgfpathlineto{\pgfqpoint{3.147764in}{1.412992in}}%
\pgfpathlineto{\pgfqpoint{3.133791in}{1.419190in}}%
\pgfpathlineto{\pgfqpoint{3.125182in}{1.414603in}}%
\pgfpathlineto{\pgfqpoint{3.116561in}{1.410227in}}%
\pgfpathlineto{\pgfqpoint{3.107928in}{1.406071in}}%
\pgfpathlineto{\pgfqpoint{3.099283in}{1.402140in}}%
\pgfpathclose%
\pgfusepath{fill}%
\end{pgfscope}%
\begin{pgfscope}%
\pgfpathrectangle{\pgfqpoint{1.150000in}{0.150000in}}{\pgfqpoint{5.700000in}{5.700000in}}%
\pgfusepath{clip}%
\pgfsetbuttcap%
\pgfsetroundjoin%
\definecolor{currentfill}{rgb}{0.214298,0.355619,0.551184}%
\pgfsetfillcolor{currentfill}%
\pgfsetfillopacity{0.700000}%
\pgfsetlinewidth{0.000000pt}%
\definecolor{currentstroke}{rgb}{0.000000,0.000000,0.000000}%
\pgfsetstrokecolor{currentstroke}%
\pgfsetdash{}{0pt}%
\pgfpathmoveto{\pgfqpoint{4.969044in}{2.090808in}}%
\pgfpathlineto{\pgfqpoint{4.983573in}{2.094995in}}%
\pgfpathlineto{\pgfqpoint{4.998116in}{2.099253in}}%
\pgfpathlineto{\pgfqpoint{5.012671in}{2.103582in}}%
\pgfpathlineto{\pgfqpoint{5.027239in}{2.107983in}}%
\pgfpathlineto{\pgfqpoint{5.035092in}{2.117578in}}%
\pgfpathlineto{\pgfqpoint{5.042937in}{2.127039in}}%
\pgfpathlineto{\pgfqpoint{5.050774in}{2.136367in}}%
\pgfpathlineto{\pgfqpoint{5.058602in}{2.145561in}}%
\pgfpathlineto{\pgfqpoint{5.044043in}{2.141143in}}%
\pgfpathlineto{\pgfqpoint{5.029496in}{2.136797in}}%
\pgfpathlineto{\pgfqpoint{5.014962in}{2.132521in}}%
\pgfpathlineto{\pgfqpoint{5.000441in}{2.128318in}}%
\pgfpathlineto{\pgfqpoint{4.992603in}{2.119132in}}%
\pgfpathlineto{\pgfqpoint{4.984758in}{2.109819in}}%
\pgfpathlineto{\pgfqpoint{4.976904in}{2.100377in}}%
\pgfpathlineto{\pgfqpoint{4.969044in}{2.090808in}}%
\pgfpathclose%
\pgfusepath{fill}%
\end{pgfscope}%
\begin{pgfscope}%
\pgfpathrectangle{\pgfqpoint{1.150000in}{0.150000in}}{\pgfqpoint{5.700000in}{5.700000in}}%
\pgfusepath{clip}%
\pgfsetbuttcap%
\pgfsetroundjoin%
\definecolor{currentfill}{rgb}{0.204903,0.375746,0.553533}%
\pgfsetfillcolor{currentfill}%
\pgfsetfillopacity{0.700000}%
\pgfsetlinewidth{0.000000pt}%
\definecolor{currentstroke}{rgb}{0.000000,0.000000,0.000000}%
\pgfsetstrokecolor{currentstroke}%
\pgfsetdash{}{0pt}%
\pgfpathmoveto{\pgfqpoint{5.058602in}{2.145561in}}%
\pgfpathlineto{\pgfqpoint{5.073175in}{2.150050in}}%
\pgfpathlineto{\pgfqpoint{5.087760in}{2.154610in}}%
\pgfpathlineto{\pgfqpoint{5.102358in}{2.159242in}}%
\pgfpathlineto{\pgfqpoint{5.116969in}{2.163945in}}%
\pgfpathlineto{\pgfqpoint{5.124781in}{2.173009in}}%
\pgfpathlineto{\pgfqpoint{5.132584in}{2.181936in}}%
\pgfpathlineto{\pgfqpoint{5.140379in}{2.190724in}}%
\pgfpathlineto{\pgfqpoint{5.148165in}{2.199376in}}%
\pgfpathlineto{\pgfqpoint{5.133563in}{2.194678in}}%
\pgfpathlineto{\pgfqpoint{5.118974in}{2.190051in}}%
\pgfpathlineto{\pgfqpoint{5.104399in}{2.185496in}}%
\pgfpathlineto{\pgfqpoint{5.089836in}{2.181012in}}%
\pgfpathlineto{\pgfqpoint{5.082040in}{2.172347in}}%
\pgfpathlineto{\pgfqpoint{5.074235in}{2.163550in}}%
\pgfpathlineto{\pgfqpoint{5.066423in}{2.154622in}}%
\pgfpathlineto{\pgfqpoint{5.058602in}{2.145561in}}%
\pgfpathclose%
\pgfusepath{fill}%
\end{pgfscope}%
\begin{pgfscope}%
\pgfpathrectangle{\pgfqpoint{1.150000in}{0.150000in}}{\pgfqpoint{5.700000in}{5.700000in}}%
\pgfusepath{clip}%
\pgfsetbuttcap%
\pgfsetroundjoin%
\definecolor{currentfill}{rgb}{0.283229,0.120777,0.440584}%
\pgfsetfillcolor{currentfill}%
\pgfsetfillopacity{0.700000}%
\pgfsetlinewidth{0.000000pt}%
\definecolor{currentstroke}{rgb}{0.000000,0.000000,0.000000}%
\pgfsetstrokecolor{currentstroke}%
\pgfsetdash{}{0pt}%
\pgfpathmoveto{\pgfqpoint{4.131443in}{1.557214in}}%
\pgfpathlineto{\pgfqpoint{4.145642in}{1.557482in}}%
\pgfpathlineto{\pgfqpoint{4.159851in}{1.557821in}}%
\pgfpathlineto{\pgfqpoint{4.174069in}{1.558233in}}%
\pgfpathlineto{\pgfqpoint{4.188296in}{1.558716in}}%
\pgfpathlineto{\pgfqpoint{4.196454in}{1.570641in}}%
\pgfpathlineto{\pgfqpoint{4.204607in}{1.582545in}}%
\pgfpathlineto{\pgfqpoint{4.212754in}{1.594424in}}%
\pgfpathlineto{\pgfqpoint{4.220896in}{1.606274in}}%
\pgfpathlineto{\pgfqpoint{4.206676in}{1.605562in}}%
\pgfpathlineto{\pgfqpoint{4.192465in}{1.604922in}}%
\pgfpathlineto{\pgfqpoint{4.178264in}{1.604353in}}%
\pgfpathlineto{\pgfqpoint{4.164072in}{1.603856in}}%
\pgfpathlineto{\pgfqpoint{4.155923in}{1.592227in}}%
\pgfpathlineto{\pgfqpoint{4.147768in}{1.580574in}}%
\pgfpathlineto{\pgfqpoint{4.139608in}{1.568902in}}%
\pgfpathlineto{\pgfqpoint{4.131443in}{1.557214in}}%
\pgfpathclose%
\pgfusepath{fill}%
\end{pgfscope}%
\begin{pgfscope}%
\pgfpathrectangle{\pgfqpoint{1.150000in}{0.150000in}}{\pgfqpoint{5.700000in}{5.700000in}}%
\pgfusepath{clip}%
\pgfsetbuttcap%
\pgfsetroundjoin%
\definecolor{currentfill}{rgb}{0.171176,0.452530,0.557965}%
\pgfsetfillcolor{currentfill}%
\pgfsetfillopacity{0.700000}%
\pgfsetlinewidth{0.000000pt}%
\definecolor{currentstroke}{rgb}{0.000000,0.000000,0.000000}%
\pgfsetstrokecolor{currentstroke}%
\pgfsetdash{}{0pt}%
\pgfpathmoveto{\pgfqpoint{5.416773in}{2.353017in}}%
\pgfpathlineto{\pgfqpoint{5.431515in}{2.358493in}}%
\pgfpathlineto{\pgfqpoint{5.446272in}{2.364041in}}%
\pgfpathlineto{\pgfqpoint{5.461042in}{2.369660in}}%
\pgfpathlineto{\pgfqpoint{5.468665in}{2.376306in}}%
\pgfpathlineto{\pgfqpoint{5.476277in}{2.382813in}}%
\pgfpathlineto{\pgfqpoint{5.483880in}{2.389181in}}%
\pgfpathlineto{\pgfqpoint{5.491473in}{2.395414in}}%
\pgfpathlineto{\pgfqpoint{5.476717in}{2.389890in}}%
\pgfpathlineto{\pgfqpoint{5.461975in}{2.384437in}}%
\pgfpathlineto{\pgfqpoint{5.447247in}{2.379055in}}%
\pgfpathlineto{\pgfqpoint{5.439643in}{2.372745in}}%
\pgfpathlineto{\pgfqpoint{5.432029in}{2.366303in}}%
\pgfpathlineto{\pgfqpoint{5.424406in}{2.359727in}}%
\pgfpathlineto{\pgfqpoint{5.416773in}{2.353017in}}%
\pgfpathclose%
\pgfusepath{fill}%
\end{pgfscope}%
\begin{pgfscope}%
\pgfpathrectangle{\pgfqpoint{1.150000in}{0.150000in}}{\pgfqpoint{5.700000in}{5.700000in}}%
\pgfusepath{clip}%
\pgfsetbuttcap%
\pgfsetroundjoin%
\definecolor{currentfill}{rgb}{0.282656,0.100196,0.422160}%
\pgfsetfillcolor{currentfill}%
\pgfsetfillopacity{0.700000}%
\pgfsetlinewidth{0.000000pt}%
\definecolor{currentstroke}{rgb}{0.000000,0.000000,0.000000}%
\pgfsetstrokecolor{currentstroke}%
\pgfsetdash{}{0pt}%
\pgfpathmoveto{\pgfqpoint{4.041992in}{1.511007in}}%
\pgfpathlineto{\pgfqpoint{4.056163in}{1.510738in}}%
\pgfpathlineto{\pgfqpoint{4.070343in}{1.510541in}}%
\pgfpathlineto{\pgfqpoint{4.084532in}{1.510416in}}%
\pgfpathlineto{\pgfqpoint{4.098730in}{1.510362in}}%
\pgfpathlineto{\pgfqpoint{4.106916in}{1.522083in}}%
\pgfpathlineto{\pgfqpoint{4.115097in}{1.533801in}}%
\pgfpathlineto{\pgfqpoint{4.123273in}{1.545512in}}%
\pgfpathlineto{\pgfqpoint{4.131443in}{1.557214in}}%
\pgfpathlineto{\pgfqpoint{4.117253in}{1.557017in}}%
\pgfpathlineto{\pgfqpoint{4.103072in}{1.556893in}}%
\pgfpathlineto{\pgfqpoint{4.088900in}{1.556841in}}%
\pgfpathlineto{\pgfqpoint{4.074737in}{1.556860in}}%
\pgfpathlineto{\pgfqpoint{4.066559in}{1.545400in}}%
\pgfpathlineto{\pgfqpoint{4.058375in}{1.533936in}}%
\pgfpathlineto{\pgfqpoint{4.050186in}{1.522470in}}%
\pgfpathlineto{\pgfqpoint{4.041992in}{1.511007in}}%
\pgfpathclose%
\pgfusepath{fill}%
\end{pgfscope}%
\begin{pgfscope}%
\pgfpathrectangle{\pgfqpoint{1.150000in}{0.150000in}}{\pgfqpoint{5.700000in}{5.700000in}}%
\pgfusepath{clip}%
\pgfsetbuttcap%
\pgfsetroundjoin%
\definecolor{currentfill}{rgb}{0.282290,0.145912,0.461510}%
\pgfsetfillcolor{currentfill}%
\pgfsetfillopacity{0.700000}%
\pgfsetlinewidth{0.000000pt}%
\definecolor{currentstroke}{rgb}{0.000000,0.000000,0.000000}%
\pgfsetstrokecolor{currentstroke}%
\pgfsetdash{}{0pt}%
\pgfpathmoveto{\pgfqpoint{4.220896in}{1.606274in}}%
\pgfpathlineto{\pgfqpoint{4.235126in}{1.607058in}}%
\pgfpathlineto{\pgfqpoint{4.249366in}{1.607913in}}%
\pgfpathlineto{\pgfqpoint{4.263615in}{1.608840in}}%
\pgfpathlineto{\pgfqpoint{4.277874in}{1.609838in}}%
\pgfpathlineto{\pgfqpoint{4.286004in}{1.621874in}}%
\pgfpathlineto{\pgfqpoint{4.294130in}{1.633871in}}%
\pgfpathlineto{\pgfqpoint{4.302250in}{1.645825in}}%
\pgfpathlineto{\pgfqpoint{4.310364in}{1.657735in}}%
\pgfpathlineto{\pgfqpoint{4.296111in}{1.656528in}}%
\pgfpathlineto{\pgfqpoint{4.281869in}{1.655392in}}%
\pgfpathlineto{\pgfqpoint{4.267636in}{1.654329in}}%
\pgfpathlineto{\pgfqpoint{4.253413in}{1.653337in}}%
\pgfpathlineto{\pgfqpoint{4.245292in}{1.641628in}}%
\pgfpathlineto{\pgfqpoint{4.237165in}{1.629880in}}%
\pgfpathlineto{\pgfqpoint{4.229033in}{1.618094in}}%
\pgfpathlineto{\pgfqpoint{4.220896in}{1.606274in}}%
\pgfpathclose%
\pgfusepath{fill}%
\end{pgfscope}%
\begin{pgfscope}%
\pgfpathrectangle{\pgfqpoint{1.150000in}{0.150000in}}{\pgfqpoint{5.700000in}{5.700000in}}%
\pgfusepath{clip}%
\pgfsetbuttcap%
\pgfsetroundjoin%
\definecolor{currentfill}{rgb}{0.194100,0.399323,0.555565}%
\pgfsetfillcolor{currentfill}%
\pgfsetfillopacity{0.700000}%
\pgfsetlinewidth{0.000000pt}%
\definecolor{currentstroke}{rgb}{0.000000,0.000000,0.000000}%
\pgfsetstrokecolor{currentstroke}%
\pgfsetdash{}{0pt}%
\pgfpathmoveto{\pgfqpoint{5.148165in}{2.199376in}}%
\pgfpathlineto{\pgfqpoint{5.162780in}{2.204145in}}%
\pgfpathlineto{\pgfqpoint{5.177408in}{2.208986in}}%
\pgfpathlineto{\pgfqpoint{5.192049in}{2.213898in}}%
\pgfpathlineto{\pgfqpoint{5.206704in}{2.218881in}}%
\pgfpathlineto{\pgfqpoint{5.214472in}{2.227378in}}%
\pgfpathlineto{\pgfqpoint{5.222230in}{2.235733in}}%
\pgfpathlineto{\pgfqpoint{5.229980in}{2.243949in}}%
\pgfpathlineto{\pgfqpoint{5.237722in}{2.252025in}}%
\pgfpathlineto{\pgfqpoint{5.223077in}{2.247069in}}%
\pgfpathlineto{\pgfqpoint{5.208446in}{2.242184in}}%
\pgfpathlineto{\pgfqpoint{5.193829in}{2.237371in}}%
\pgfpathlineto{\pgfqpoint{5.179224in}{2.232629in}}%
\pgfpathlineto{\pgfqpoint{5.171472in}{2.224517in}}%
\pgfpathlineto{\pgfqpoint{5.163712in}{2.216271in}}%
\pgfpathlineto{\pgfqpoint{5.155943in}{2.207891in}}%
\pgfpathlineto{\pgfqpoint{5.148165in}{2.199376in}}%
\pgfpathclose%
\pgfusepath{fill}%
\end{pgfscope}%
\begin{pgfscope}%
\pgfpathrectangle{\pgfqpoint{1.150000in}{0.150000in}}{\pgfqpoint{5.700000in}{5.700000in}}%
\pgfusepath{clip}%
\pgfsetbuttcap%
\pgfsetroundjoin%
\definecolor{currentfill}{rgb}{0.280894,0.078907,0.402329}%
\pgfsetfillcolor{currentfill}%
\pgfsetfillopacity{0.700000}%
\pgfsetlinewidth{0.000000pt}%
\definecolor{currentstroke}{rgb}{0.000000,0.000000,0.000000}%
\pgfsetstrokecolor{currentstroke}%
\pgfsetdash{}{0pt}%
\pgfpathmoveto{\pgfqpoint{3.952527in}{1.468129in}}%
\pgfpathlineto{\pgfqpoint{3.966672in}{1.467302in}}%
\pgfpathlineto{\pgfqpoint{3.980827in}{1.466547in}}%
\pgfpathlineto{\pgfqpoint{3.994990in}{1.465865in}}%
\pgfpathlineto{\pgfqpoint{4.009161in}{1.465254in}}%
\pgfpathlineto{\pgfqpoint{4.017377in}{1.476669in}}%
\pgfpathlineto{\pgfqpoint{4.025587in}{1.488103in}}%
\pgfpathlineto{\pgfqpoint{4.033792in}{1.499550in}}%
\pgfpathlineto{\pgfqpoint{4.041992in}{1.511007in}}%
\pgfpathlineto{\pgfqpoint{4.027829in}{1.511348in}}%
\pgfpathlineto{\pgfqpoint{4.013675in}{1.511761in}}%
\pgfpathlineto{\pgfqpoint{3.999530in}{1.512246in}}%
\pgfpathlineto{\pgfqpoint{3.985393in}{1.512803in}}%
\pgfpathlineto{\pgfqpoint{3.977185in}{1.501608in}}%
\pgfpathlineto{\pgfqpoint{3.968971in}{1.490428in}}%
\pgfpathlineto{\pgfqpoint{3.960752in}{1.479267in}}%
\pgfpathlineto{\pgfqpoint{3.952527in}{1.468129in}}%
\pgfpathclose%
\pgfusepath{fill}%
\end{pgfscope}%
\begin{pgfscope}%
\pgfpathrectangle{\pgfqpoint{1.150000in}{0.150000in}}{\pgfqpoint{5.700000in}{5.700000in}}%
\pgfusepath{clip}%
\pgfsetbuttcap%
\pgfsetroundjoin%
\definecolor{currentfill}{rgb}{0.268510,0.009605,0.335427}%
\pgfsetfillcolor{currentfill}%
\pgfsetfillopacity{0.700000}%
\pgfsetlinewidth{0.000000pt}%
\definecolor{currentstroke}{rgb}{0.000000,0.000000,0.000000}%
\pgfsetstrokecolor{currentstroke}%
\pgfsetdash{}{0pt}%
\pgfpathmoveto{\pgfqpoint{3.537835in}{1.354044in}}%
\pgfpathlineto{\pgfqpoint{3.551888in}{1.350469in}}%
\pgfpathlineto{\pgfqpoint{3.565946in}{1.346969in}}%
\pgfpathlineto{\pgfqpoint{3.580011in}{1.343543in}}%
\pgfpathlineto{\pgfqpoint{3.594082in}{1.340191in}}%
\pgfpathlineto{\pgfqpoint{3.602459in}{1.348927in}}%
\pgfpathlineto{\pgfqpoint{3.610829in}{1.357780in}}%
\pgfpathlineto{\pgfqpoint{3.619191in}{1.366745in}}%
\pgfpathlineto{\pgfqpoint{3.627546in}{1.375818in}}%
\pgfpathlineto{\pgfqpoint{3.613491in}{1.378819in}}%
\pgfpathlineto{\pgfqpoint{3.599442in}{1.381894in}}%
\pgfpathlineto{\pgfqpoint{3.585400in}{1.385044in}}%
\pgfpathlineto{\pgfqpoint{3.571364in}{1.388269in}}%
\pgfpathlineto{\pgfqpoint{3.562993in}{1.379539in}}%
\pgfpathlineto{\pgfqpoint{3.554615in}{1.370921in}}%
\pgfpathlineto{\pgfqpoint{3.546229in}{1.362421in}}%
\pgfpathlineto{\pgfqpoint{3.537835in}{1.354044in}}%
\pgfpathclose%
\pgfusepath{fill}%
\end{pgfscope}%
\begin{pgfscope}%
\pgfpathrectangle{\pgfqpoint{1.150000in}{0.150000in}}{\pgfqpoint{5.700000in}{5.700000in}}%
\pgfusepath{clip}%
\pgfsetbuttcap%
\pgfsetroundjoin%
\definecolor{currentfill}{rgb}{0.279574,0.170599,0.479997}%
\pgfsetfillcolor{currentfill}%
\pgfsetfillopacity{0.700000}%
\pgfsetlinewidth{0.000000pt}%
\definecolor{currentstroke}{rgb}{0.000000,0.000000,0.000000}%
\pgfsetstrokecolor{currentstroke}%
\pgfsetdash{}{0pt}%
\pgfpathmoveto{\pgfqpoint{4.310364in}{1.657735in}}%
\pgfpathlineto{\pgfqpoint{4.324627in}{1.659013in}}%
\pgfpathlineto{\pgfqpoint{4.338900in}{1.660363in}}%
\pgfpathlineto{\pgfqpoint{4.353182in}{1.661784in}}%
\pgfpathlineto{\pgfqpoint{4.367475in}{1.663276in}}%
\pgfpathlineto{\pgfqpoint{4.375579in}{1.675333in}}%
\pgfpathlineto{\pgfqpoint{4.383677in}{1.687335in}}%
\pgfpathlineto{\pgfqpoint{4.391769in}{1.699278in}}%
\pgfpathlineto{\pgfqpoint{4.399856in}{1.711160in}}%
\pgfpathlineto{\pgfqpoint{4.385569in}{1.709480in}}%
\pgfpathlineto{\pgfqpoint{4.371292in}{1.707870in}}%
\pgfpathlineto{\pgfqpoint{4.357026in}{1.706333in}}%
\pgfpathlineto{\pgfqpoint{4.342769in}{1.704867in}}%
\pgfpathlineto{\pgfqpoint{4.334676in}{1.693165in}}%
\pgfpathlineto{\pgfqpoint{4.326577in}{1.681407in}}%
\pgfpathlineto{\pgfqpoint{4.318474in}{1.669596in}}%
\pgfpathlineto{\pgfqpoint{4.310364in}{1.657735in}}%
\pgfpathclose%
\pgfusepath{fill}%
\end{pgfscope}%
\begin{pgfscope}%
\pgfpathrectangle{\pgfqpoint{1.150000in}{0.150000in}}{\pgfqpoint{5.700000in}{5.700000in}}%
\pgfusepath{clip}%
\pgfsetbuttcap%
\pgfsetroundjoin%
\definecolor{currentfill}{rgb}{0.185556,0.418570,0.556753}%
\pgfsetfillcolor{currentfill}%
\pgfsetfillopacity{0.700000}%
\pgfsetlinewidth{0.000000pt}%
\definecolor{currentstroke}{rgb}{0.000000,0.000000,0.000000}%
\pgfsetstrokecolor{currentstroke}%
\pgfsetdash{}{0pt}%
\pgfpathmoveto{\pgfqpoint{5.237722in}{2.252025in}}%
\pgfpathlineto{\pgfqpoint{5.252379in}{2.257052in}}%
\pgfpathlineto{\pgfqpoint{5.267050in}{2.262151in}}%
\pgfpathlineto{\pgfqpoint{5.281735in}{2.267321in}}%
\pgfpathlineto{\pgfqpoint{5.296433in}{2.272563in}}%
\pgfpathlineto{\pgfqpoint{5.304154in}{2.280459in}}%
\pgfpathlineto{\pgfqpoint{5.311866in}{2.288214in}}%
\pgfpathlineto{\pgfqpoint{5.319568in}{2.295827in}}%
\pgfpathlineto{\pgfqpoint{5.327261in}{2.303300in}}%
\pgfpathlineto{\pgfqpoint{5.312575in}{2.298108in}}%
\pgfpathlineto{\pgfqpoint{5.297902in}{2.292988in}}%
\pgfpathlineto{\pgfqpoint{5.283243in}{2.287939in}}%
\pgfpathlineto{\pgfqpoint{5.268597in}{2.282961in}}%
\pgfpathlineto{\pgfqpoint{5.260891in}{2.275430in}}%
\pgfpathlineto{\pgfqpoint{5.253177in}{2.267764in}}%
\pgfpathlineto{\pgfqpoint{5.245454in}{2.259963in}}%
\pgfpathlineto{\pgfqpoint{5.237722in}{2.252025in}}%
\pgfpathclose%
\pgfusepath{fill}%
\end{pgfscope}%
\begin{pgfscope}%
\pgfpathrectangle{\pgfqpoint{1.150000in}{0.150000in}}{\pgfqpoint{5.700000in}{5.700000in}}%
\pgfusepath{clip}%
\pgfsetbuttcap%
\pgfsetroundjoin%
\definecolor{currentfill}{rgb}{0.177423,0.437527,0.557565}%
\pgfsetfillcolor{currentfill}%
\pgfsetfillopacity{0.700000}%
\pgfsetlinewidth{0.000000pt}%
\definecolor{currentstroke}{rgb}{0.000000,0.000000,0.000000}%
\pgfsetstrokecolor{currentstroke}%
\pgfsetdash{}{0pt}%
\pgfpathmoveto{\pgfqpoint{5.327261in}{2.303300in}}%
\pgfpathlineto{\pgfqpoint{5.341962in}{2.308563in}}%
\pgfpathlineto{\pgfqpoint{5.356675in}{2.313898in}}%
\pgfpathlineto{\pgfqpoint{5.371403in}{2.319304in}}%
\pgfpathlineto{\pgfqpoint{5.386145in}{2.324781in}}%
\pgfpathlineto{\pgfqpoint{5.393816in}{2.332052in}}%
\pgfpathlineto{\pgfqpoint{5.401478in}{2.339181in}}%
\pgfpathlineto{\pgfqpoint{5.409130in}{2.346168in}}%
\pgfpathlineto{\pgfqpoint{5.416773in}{2.353017in}}%
\pgfpathlineto{\pgfqpoint{5.402044in}{2.347611in}}%
\pgfpathlineto{\pgfqpoint{5.387329in}{2.342278in}}%
\pgfpathlineto{\pgfqpoint{5.372628in}{2.337015in}}%
\pgfpathlineto{\pgfqpoint{5.357941in}{2.331824in}}%
\pgfpathlineto{\pgfqpoint{5.350285in}{2.324895in}}%
\pgfpathlineto{\pgfqpoint{5.342620in}{2.317833in}}%
\pgfpathlineto{\pgfqpoint{5.334945in}{2.310635in}}%
\pgfpathlineto{\pgfqpoint{5.327261in}{2.303300in}}%
\pgfpathclose%
\pgfusepath{fill}%
\end{pgfscope}%
\begin{pgfscope}%
\pgfpathrectangle{\pgfqpoint{1.150000in}{0.150000in}}{\pgfqpoint{5.700000in}{5.700000in}}%
\pgfusepath{clip}%
\pgfsetbuttcap%
\pgfsetroundjoin%
\definecolor{currentfill}{rgb}{0.277941,0.056324,0.381191}%
\pgfsetfillcolor{currentfill}%
\pgfsetfillopacity{0.700000}%
\pgfsetlinewidth{0.000000pt}%
\definecolor{currentstroke}{rgb}{0.000000,0.000000,0.000000}%
\pgfsetstrokecolor{currentstroke}%
\pgfsetdash{}{0pt}%
\pgfpathmoveto{\pgfqpoint{3.863028in}{1.429078in}}%
\pgfpathlineto{\pgfqpoint{3.877152in}{1.427671in}}%
\pgfpathlineto{\pgfqpoint{3.891283in}{1.426337in}}%
\pgfpathlineto{\pgfqpoint{3.905423in}{1.425076in}}%
\pgfpathlineto{\pgfqpoint{3.919571in}{1.423886in}}%
\pgfpathlineto{\pgfqpoint{3.927818in}{1.434892in}}%
\pgfpathlineto{\pgfqpoint{3.936060in}{1.445937in}}%
\pgfpathlineto{\pgfqpoint{3.944296in}{1.457018in}}%
\pgfpathlineto{\pgfqpoint{3.952527in}{1.468129in}}%
\pgfpathlineto{\pgfqpoint{3.938389in}{1.469028in}}%
\pgfpathlineto{\pgfqpoint{3.924259in}{1.470000in}}%
\pgfpathlineto{\pgfqpoint{3.910138in}{1.471044in}}%
\pgfpathlineto{\pgfqpoint{3.896025in}{1.472161in}}%
\pgfpathlineto{\pgfqpoint{3.887784in}{1.461332in}}%
\pgfpathlineto{\pgfqpoint{3.879538in}{1.450539in}}%
\pgfpathlineto{\pgfqpoint{3.871286in}{1.439786in}}%
\pgfpathlineto{\pgfqpoint{3.863028in}{1.429078in}}%
\pgfpathclose%
\pgfusepath{fill}%
\end{pgfscope}%
\begin{pgfscope}%
\pgfpathrectangle{\pgfqpoint{1.150000in}{0.150000in}}{\pgfqpoint{5.700000in}{5.700000in}}%
\pgfusepath{clip}%
\pgfsetbuttcap%
\pgfsetroundjoin%
\definecolor{currentfill}{rgb}{0.274128,0.199721,0.498911}%
\pgfsetfillcolor{currentfill}%
\pgfsetfillopacity{0.700000}%
\pgfsetlinewidth{0.000000pt}%
\definecolor{currentstroke}{rgb}{0.000000,0.000000,0.000000}%
\pgfsetstrokecolor{currentstroke}%
\pgfsetdash{}{0pt}%
\pgfpathmoveto{\pgfqpoint{4.399856in}{1.711160in}}%
\pgfpathlineto{\pgfqpoint{4.414154in}{1.712912in}}%
\pgfpathlineto{\pgfqpoint{4.428462in}{1.714735in}}%
\pgfpathlineto{\pgfqpoint{4.442780in}{1.716630in}}%
\pgfpathlineto{\pgfqpoint{4.457109in}{1.718595in}}%
\pgfpathlineto{\pgfqpoint{4.465185in}{1.730589in}}%
\pgfpathlineto{\pgfqpoint{4.473256in}{1.742513in}}%
\pgfpathlineto{\pgfqpoint{4.481321in}{1.754363in}}%
\pgfpathlineto{\pgfqpoint{4.489380in}{1.766137in}}%
\pgfpathlineto{\pgfqpoint{4.475056in}{1.764004in}}%
\pgfpathlineto{\pgfqpoint{4.460743in}{1.761943in}}%
\pgfpathlineto{\pgfqpoint{4.446441in}{1.759952in}}%
\pgfpathlineto{\pgfqpoint{4.432150in}{1.758034in}}%
\pgfpathlineto{\pgfqpoint{4.424085in}{1.746418in}}%
\pgfpathlineto{\pgfqpoint{4.416014in}{1.734732in}}%
\pgfpathlineto{\pgfqpoint{4.407938in}{1.722979in}}%
\pgfpathlineto{\pgfqpoint{4.399856in}{1.711160in}}%
\pgfpathclose%
\pgfusepath{fill}%
\end{pgfscope}%
\begin{pgfscope}%
\pgfpathrectangle{\pgfqpoint{1.150000in}{0.150000in}}{\pgfqpoint{5.700000in}{5.700000in}}%
\pgfusepath{clip}%
\pgfsetbuttcap%
\pgfsetroundjoin%
\definecolor{currentfill}{rgb}{0.267968,0.223549,0.512008}%
\pgfsetfillcolor{currentfill}%
\pgfsetfillopacity{0.700000}%
\pgfsetlinewidth{0.000000pt}%
\definecolor{currentstroke}{rgb}{0.000000,0.000000,0.000000}%
\pgfsetstrokecolor{currentstroke}%
\pgfsetdash{}{0pt}%
\pgfpathmoveto{\pgfqpoint{4.489380in}{1.766137in}}%
\pgfpathlineto{\pgfqpoint{4.503714in}{1.768342in}}%
\pgfpathlineto{\pgfqpoint{4.518059in}{1.770617in}}%
\pgfpathlineto{\pgfqpoint{4.532415in}{1.772964in}}%
\pgfpathlineto{\pgfqpoint{4.546782in}{1.775382in}}%
\pgfpathlineto{\pgfqpoint{4.554830in}{1.787234in}}%
\pgfpathlineto{\pgfqpoint{4.562872in}{1.799001in}}%
\pgfpathlineto{\pgfqpoint{4.570908in}{1.810681in}}%
\pgfpathlineto{\pgfqpoint{4.578938in}{1.822274in}}%
\pgfpathlineto{\pgfqpoint{4.564577in}{1.819709in}}%
\pgfpathlineto{\pgfqpoint{4.550226in}{1.817216in}}%
\pgfpathlineto{\pgfqpoint{4.535887in}{1.814794in}}%
\pgfpathlineto{\pgfqpoint{4.521559in}{1.812444in}}%
\pgfpathlineto{\pgfqpoint{4.513522in}{1.800990in}}%
\pgfpathlineto{\pgfqpoint{4.505481in}{1.789453in}}%
\pgfpathlineto{\pgfqpoint{4.497433in}{1.777835in}}%
\pgfpathlineto{\pgfqpoint{4.489380in}{1.766137in}}%
\pgfpathclose%
\pgfusepath{fill}%
\end{pgfscope}%
\begin{pgfscope}%
\pgfpathrectangle{\pgfqpoint{1.150000in}{0.150000in}}{\pgfqpoint{5.700000in}{5.700000in}}%
\pgfusepath{clip}%
\pgfsetbuttcap%
\pgfsetroundjoin%
\definecolor{currentfill}{rgb}{0.274952,0.037752,0.364543}%
\pgfsetfillcolor{currentfill}%
\pgfsetfillopacity{0.700000}%
\pgfsetlinewidth{0.000000pt}%
\definecolor{currentstroke}{rgb}{0.000000,0.000000,0.000000}%
\pgfsetstrokecolor{currentstroke}%
\pgfsetdash{}{0pt}%
\pgfpathmoveto{\pgfqpoint{3.773473in}{1.394373in}}%
\pgfpathlineto{\pgfqpoint{3.787578in}{1.392365in}}%
\pgfpathlineto{\pgfqpoint{3.801690in}{1.390430in}}%
\pgfpathlineto{\pgfqpoint{3.815809in}{1.388567in}}%
\pgfpathlineto{\pgfqpoint{3.829937in}{1.386778in}}%
\pgfpathlineto{\pgfqpoint{3.838219in}{1.397264in}}%
\pgfpathlineto{\pgfqpoint{3.846494in}{1.407812in}}%
\pgfpathlineto{\pgfqpoint{3.854764in}{1.418419in}}%
\pgfpathlineto{\pgfqpoint{3.863028in}{1.429078in}}%
\pgfpathlineto{\pgfqpoint{3.848912in}{1.430557in}}%
\pgfpathlineto{\pgfqpoint{3.834804in}{1.432109in}}%
\pgfpathlineto{\pgfqpoint{3.820704in}{1.433734in}}%
\pgfpathlineto{\pgfqpoint{3.806611in}{1.435432in}}%
\pgfpathlineto{\pgfqpoint{3.798336in}{1.425075in}}%
\pgfpathlineto{\pgfqpoint{3.790054in}{1.414777in}}%
\pgfpathlineto{\pgfqpoint{3.781767in}{1.404541in}}%
\pgfpathlineto{\pgfqpoint{3.773473in}{1.394373in}}%
\pgfpathclose%
\pgfusepath{fill}%
\end{pgfscope}%
\begin{pgfscope}%
\pgfpathrectangle{\pgfqpoint{1.150000in}{0.150000in}}{\pgfqpoint{5.700000in}{5.700000in}}%
\pgfusepath{clip}%
\pgfsetbuttcap%
\pgfsetroundjoin%
\definecolor{currentfill}{rgb}{0.267004,0.004874,0.329415}%
\pgfsetfillcolor{currentfill}%
\pgfsetfillopacity{0.700000}%
\pgfsetlinewidth{0.000000pt}%
\definecolor{currentstroke}{rgb}{0.000000,0.000000,0.000000}%
\pgfsetstrokecolor{currentstroke}%
\pgfsetdash{}{0pt}%
\pgfpathmoveto{\pgfqpoint{3.301743in}{1.350068in}}%
\pgfpathlineto{\pgfqpoint{3.315768in}{1.344818in}}%
\pgfpathlineto{\pgfqpoint{3.329799in}{1.339645in}}%
\pgfpathlineto{\pgfqpoint{3.343834in}{1.334549in}}%
\pgfpathlineto{\pgfqpoint{3.357875in}{1.329530in}}%
\pgfpathlineto{\pgfqpoint{3.366377in}{1.335924in}}%
\pgfpathlineto{\pgfqpoint{3.374870in}{1.342496in}}%
\pgfpathlineto{\pgfqpoint{3.383353in}{1.349242in}}%
\pgfpathlineto{\pgfqpoint{3.391827in}{1.356154in}}%
\pgfpathlineto{\pgfqpoint{3.377808in}{1.360781in}}%
\pgfpathlineto{\pgfqpoint{3.363794in}{1.365485in}}%
\pgfpathlineto{\pgfqpoint{3.349785in}{1.370266in}}%
\pgfpathlineto{\pgfqpoint{3.335782in}{1.375125in}}%
\pgfpathlineto{\pgfqpoint{3.327287in}{1.368596in}}%
\pgfpathlineto{\pgfqpoint{3.318782in}{1.362240in}}%
\pgfpathlineto{\pgfqpoint{3.310267in}{1.356062in}}%
\pgfpathlineto{\pgfqpoint{3.301743in}{1.350068in}}%
\pgfpathclose%
\pgfusepath{fill}%
\end{pgfscope}%
\begin{pgfscope}%
\pgfpathrectangle{\pgfqpoint{1.150000in}{0.150000in}}{\pgfqpoint{5.700000in}{5.700000in}}%
\pgfusepath{clip}%
\pgfsetbuttcap%
\pgfsetroundjoin%
\definecolor{currentfill}{rgb}{0.269944,0.014625,0.341379}%
\pgfsetfillcolor{currentfill}%
\pgfsetfillopacity{0.700000}%
\pgfsetlinewidth{0.000000pt}%
\definecolor{currentstroke}{rgb}{0.000000,0.000000,0.000000}%
\pgfsetstrokecolor{currentstroke}%
\pgfsetdash{}{0pt}%
\pgfpathmoveto{\pgfqpoint{3.155302in}{1.376177in}}%
\pgfpathlineto{\pgfqpoint{3.169317in}{1.369886in}}%
\pgfpathlineto{\pgfqpoint{3.183336in}{1.363675in}}%
\pgfpathlineto{\pgfqpoint{3.197359in}{1.357542in}}%
\pgfpathlineto{\pgfqpoint{3.211386in}{1.351489in}}%
\pgfpathlineto{\pgfqpoint{3.219978in}{1.356263in}}%
\pgfpathlineto{\pgfqpoint{3.228559in}{1.361251in}}%
\pgfpathlineto{\pgfqpoint{3.237129in}{1.366448in}}%
\pgfpathlineto{\pgfqpoint{3.245688in}{1.371846in}}%
\pgfpathlineto{\pgfqpoint{3.231686in}{1.377486in}}%
\pgfpathlineto{\pgfqpoint{3.217688in}{1.383205in}}%
\pgfpathlineto{\pgfqpoint{3.203695in}{1.389003in}}%
\pgfpathlineto{\pgfqpoint{3.189706in}{1.394881in}}%
\pgfpathlineto{\pgfqpoint{3.181122in}{1.389888in}}%
\pgfpathlineto{\pgfqpoint{3.172527in}{1.385102in}}%
\pgfpathlineto{\pgfqpoint{3.163921in}{1.380529in}}%
\pgfpathlineto{\pgfqpoint{3.155302in}{1.376177in}}%
\pgfpathclose%
\pgfusepath{fill}%
\end{pgfscope}%
\begin{pgfscope}%
\pgfpathrectangle{\pgfqpoint{1.150000in}{0.150000in}}{\pgfqpoint{5.700000in}{5.700000in}}%
\pgfusepath{clip}%
\pgfsetbuttcap%
\pgfsetroundjoin%
\definecolor{currentfill}{rgb}{0.258965,0.251537,0.524736}%
\pgfsetfillcolor{currentfill}%
\pgfsetfillopacity{0.700000}%
\pgfsetlinewidth{0.000000pt}%
\definecolor{currentstroke}{rgb}{0.000000,0.000000,0.000000}%
\pgfsetstrokecolor{currentstroke}%
\pgfsetdash{}{0pt}%
\pgfpathmoveto{\pgfqpoint{4.578938in}{1.822274in}}%
\pgfpathlineto{\pgfqpoint{4.593311in}{1.824909in}}%
\pgfpathlineto{\pgfqpoint{4.607694in}{1.827616in}}%
\pgfpathlineto{\pgfqpoint{4.622089in}{1.830394in}}%
\pgfpathlineto{\pgfqpoint{4.636496in}{1.833244in}}%
\pgfpathlineto{\pgfqpoint{4.644514in}{1.844879in}}%
\pgfpathlineto{\pgfqpoint{4.652527in}{1.856417in}}%
\pgfpathlineto{\pgfqpoint{4.660533in}{1.867856in}}%
\pgfpathlineto{\pgfqpoint{4.668533in}{1.879195in}}%
\pgfpathlineto{\pgfqpoint{4.654133in}{1.876221in}}%
\pgfpathlineto{\pgfqpoint{4.639743in}{1.873317in}}%
\pgfpathlineto{\pgfqpoint{4.625365in}{1.870485in}}%
\pgfpathlineto{\pgfqpoint{4.610999in}{1.867725in}}%
\pgfpathlineto{\pgfqpoint{4.602993in}{1.856503in}}%
\pgfpathlineto{\pgfqpoint{4.594981in}{1.845186in}}%
\pgfpathlineto{\pgfqpoint{4.586962in}{1.833775in}}%
\pgfpathlineto{\pgfqpoint{4.578938in}{1.822274in}}%
\pgfpathclose%
\pgfusepath{fill}%
\end{pgfscope}%
\begin{pgfscope}%
\pgfpathrectangle{\pgfqpoint{1.150000in}{0.150000in}}{\pgfqpoint{5.700000in}{5.700000in}}%
\pgfusepath{clip}%
\pgfsetbuttcap%
\pgfsetroundjoin%
\definecolor{currentfill}{rgb}{0.267004,0.004874,0.329415}%
\pgfsetfillcolor{currentfill}%
\pgfsetfillopacity{0.700000}%
\pgfsetlinewidth{0.000000pt}%
\definecolor{currentstroke}{rgb}{0.000000,0.000000,0.000000}%
\pgfsetstrokecolor{currentstroke}%
\pgfsetdash{}{0pt}%
\pgfpathmoveto{\pgfqpoint{3.447959in}{1.338408in}}%
\pgfpathlineto{\pgfqpoint{3.462005in}{1.334161in}}%
\pgfpathlineto{\pgfqpoint{3.476058in}{1.329990in}}%
\pgfpathlineto{\pgfqpoint{3.490116in}{1.325894in}}%
\pgfpathlineto{\pgfqpoint{3.504180in}{1.321873in}}%
\pgfpathlineto{\pgfqpoint{3.512606in}{1.329704in}}%
\pgfpathlineto{\pgfqpoint{3.521024in}{1.337680in}}%
\pgfpathlineto{\pgfqpoint{3.529434in}{1.345795in}}%
\pgfpathlineto{\pgfqpoint{3.537835in}{1.354044in}}%
\pgfpathlineto{\pgfqpoint{3.523789in}{1.357694in}}%
\pgfpathlineto{\pgfqpoint{3.509749in}{1.361419in}}%
\pgfpathlineto{\pgfqpoint{3.495715in}{1.365219in}}%
\pgfpathlineto{\pgfqpoint{3.481687in}{1.369095in}}%
\pgfpathlineto{\pgfqpoint{3.473267in}{1.361209in}}%
\pgfpathlineto{\pgfqpoint{3.464840in}{1.353463in}}%
\pgfpathlineto{\pgfqpoint{3.456403in}{1.345860in}}%
\pgfpathlineto{\pgfqpoint{3.447959in}{1.338408in}}%
\pgfpathclose%
\pgfusepath{fill}%
\end{pgfscope}%
\begin{pgfscope}%
\pgfpathrectangle{\pgfqpoint{1.150000in}{0.150000in}}{\pgfqpoint{5.700000in}{5.700000in}}%
\pgfusepath{clip}%
\pgfsetbuttcap%
\pgfsetroundjoin%
\definecolor{currentfill}{rgb}{0.271305,0.019942,0.347269}%
\pgfsetfillcolor{currentfill}%
\pgfsetfillopacity{0.700000}%
\pgfsetlinewidth{0.000000pt}%
\definecolor{currentstroke}{rgb}{0.000000,0.000000,0.000000}%
\pgfsetstrokecolor{currentstroke}%
\pgfsetdash{}{0pt}%
\pgfpathmoveto{\pgfqpoint{3.683835in}{1.364555in}}%
\pgfpathlineto{\pgfqpoint{3.697924in}{1.361924in}}%
\pgfpathlineto{\pgfqpoint{3.712020in}{1.359366in}}%
\pgfpathlineto{\pgfqpoint{3.726123in}{1.356881in}}%
\pgfpathlineto{\pgfqpoint{3.740234in}{1.354469in}}%
\pgfpathlineto{\pgfqpoint{3.748553in}{1.364320in}}%
\pgfpathlineto{\pgfqpoint{3.756866in}{1.374257in}}%
\pgfpathlineto{\pgfqpoint{3.765173in}{1.384276in}}%
\pgfpathlineto{\pgfqpoint{3.773473in}{1.394373in}}%
\pgfpathlineto{\pgfqpoint{3.759376in}{1.396454in}}%
\pgfpathlineto{\pgfqpoint{3.745286in}{1.398608in}}%
\pgfpathlineto{\pgfqpoint{3.731203in}{1.400836in}}%
\pgfpathlineto{\pgfqpoint{3.717127in}{1.403137in}}%
\pgfpathlineto{\pgfqpoint{3.708814in}{1.393363in}}%
\pgfpathlineto{\pgfqpoint{3.700494in}{1.383672in}}%
\pgfpathlineto{\pgfqpoint{3.692168in}{1.374068in}}%
\pgfpathlineto{\pgfqpoint{3.683835in}{1.364555in}}%
\pgfpathclose%
\pgfusepath{fill}%
\end{pgfscope}%
\begin{pgfscope}%
\pgfpathrectangle{\pgfqpoint{1.150000in}{0.150000in}}{\pgfqpoint{5.700000in}{5.700000in}}%
\pgfusepath{clip}%
\pgfsetbuttcap%
\pgfsetroundjoin%
\definecolor{currentfill}{rgb}{0.248629,0.278775,0.534556}%
\pgfsetfillcolor{currentfill}%
\pgfsetfillopacity{0.700000}%
\pgfsetlinewidth{0.000000pt}%
\definecolor{currentstroke}{rgb}{0.000000,0.000000,0.000000}%
\pgfsetstrokecolor{currentstroke}%
\pgfsetdash{}{0pt}%
\pgfpathmoveto{\pgfqpoint{4.668533in}{1.879195in}}%
\pgfpathlineto{\pgfqpoint{4.682946in}{1.882241in}}%
\pgfpathlineto{\pgfqpoint{4.697369in}{1.885358in}}%
\pgfpathlineto{\pgfqpoint{4.711805in}{1.888547in}}%
\pgfpathlineto{\pgfqpoint{4.726252in}{1.891806in}}%
\pgfpathlineto{\pgfqpoint{4.734240in}{1.903156in}}%
\pgfpathlineto{\pgfqpoint{4.742222in}{1.914397in}}%
\pgfpathlineto{\pgfqpoint{4.750197in}{1.925529in}}%
\pgfpathlineto{\pgfqpoint{4.758165in}{1.936550in}}%
\pgfpathlineto{\pgfqpoint{4.743724in}{1.933186in}}%
\pgfpathlineto{\pgfqpoint{4.729294in}{1.929894in}}%
\pgfpathlineto{\pgfqpoint{4.714876in}{1.926673in}}%
\pgfpathlineto{\pgfqpoint{4.700470in}{1.923523in}}%
\pgfpathlineto{\pgfqpoint{4.692496in}{1.912598in}}%
\pgfpathlineto{\pgfqpoint{4.684515in}{1.901567in}}%
\pgfpathlineto{\pgfqpoint{4.676527in}{1.890433in}}%
\pgfpathlineto{\pgfqpoint{4.668533in}{1.879195in}}%
\pgfpathclose%
\pgfusepath{fill}%
\end{pgfscope}%
\begin{pgfscope}%
\pgfpathrectangle{\pgfqpoint{1.150000in}{0.150000in}}{\pgfqpoint{5.700000in}{5.700000in}}%
\pgfusepath{clip}%
\pgfsetbuttcap%
\pgfsetroundjoin%
\definecolor{currentfill}{rgb}{0.239346,0.300855,0.540844}%
\pgfsetfillcolor{currentfill}%
\pgfsetfillopacity{0.700000}%
\pgfsetlinewidth{0.000000pt}%
\definecolor{currentstroke}{rgb}{0.000000,0.000000,0.000000}%
\pgfsetstrokecolor{currentstroke}%
\pgfsetdash{}{0pt}%
\pgfpathmoveto{\pgfqpoint{4.758165in}{1.936550in}}%
\pgfpathlineto{\pgfqpoint{4.772618in}{1.939984in}}%
\pgfpathlineto{\pgfqpoint{4.787083in}{1.943490in}}%
\pgfpathlineto{\pgfqpoint{4.801560in}{1.947068in}}%
\pgfpathlineto{\pgfqpoint{4.816049in}{1.950716in}}%
\pgfpathlineto{\pgfqpoint{4.824005in}{1.961716in}}%
\pgfpathlineto{\pgfqpoint{4.831953in}{1.972598in}}%
\pgfpathlineto{\pgfqpoint{4.839895in}{1.983361in}}%
\pgfpathlineto{\pgfqpoint{4.847830in}{1.994004in}}%
\pgfpathlineto{\pgfqpoint{4.833347in}{1.990273in}}%
\pgfpathlineto{\pgfqpoint{4.818876in}{1.986613in}}%
\pgfpathlineto{\pgfqpoint{4.804418in}{1.983025in}}%
\pgfpathlineto{\pgfqpoint{4.789971in}{1.979508in}}%
\pgfpathlineto{\pgfqpoint{4.782030in}{1.968939in}}%
\pgfpathlineto{\pgfqpoint{4.774082in}{1.958256in}}%
\pgfpathlineto{\pgfqpoint{4.766127in}{1.947459in}}%
\pgfpathlineto{\pgfqpoint{4.758165in}{1.936550in}}%
\pgfpathclose%
\pgfusepath{fill}%
\end{pgfscope}%
\begin{pgfscope}%
\pgfpathrectangle{\pgfqpoint{1.150000in}{0.150000in}}{\pgfqpoint{5.700000in}{5.700000in}}%
\pgfusepath{clip}%
\pgfsetbuttcap%
\pgfsetroundjoin%
\definecolor{currentfill}{rgb}{0.227802,0.326594,0.546532}%
\pgfsetfillcolor{currentfill}%
\pgfsetfillopacity{0.700000}%
\pgfsetlinewidth{0.000000pt}%
\definecolor{currentstroke}{rgb}{0.000000,0.000000,0.000000}%
\pgfsetstrokecolor{currentstroke}%
\pgfsetdash{}{0pt}%
\pgfpathmoveto{\pgfqpoint{4.847830in}{1.994004in}}%
\pgfpathlineto{\pgfqpoint{4.862325in}{1.997806in}}%
\pgfpathlineto{\pgfqpoint{4.876832in}{2.001680in}}%
\pgfpathlineto{\pgfqpoint{4.891352in}{2.005624in}}%
\pgfpathlineto{\pgfqpoint{4.905883in}{2.009641in}}%
\pgfpathlineto{\pgfqpoint{4.913804in}{2.020232in}}%
\pgfpathlineto{\pgfqpoint{4.921718in}{2.030697in}}%
\pgfpathlineto{\pgfqpoint{4.929624in}{2.041035in}}%
\pgfpathlineto{\pgfqpoint{4.937523in}{2.051246in}}%
\pgfpathlineto{\pgfqpoint{4.922998in}{2.047169in}}%
\pgfpathlineto{\pgfqpoint{4.908486in}{2.043164in}}%
\pgfpathlineto{\pgfqpoint{4.893985in}{2.039229in}}%
\pgfpathlineto{\pgfqpoint{4.879497in}{2.035367in}}%
\pgfpathlineto{\pgfqpoint{4.871591in}{2.025209in}}%
\pgfpathlineto{\pgfqpoint{4.863678in}{2.014928in}}%
\pgfpathlineto{\pgfqpoint{4.855757in}{2.004527in}}%
\pgfpathlineto{\pgfqpoint{4.847830in}{1.994004in}}%
\pgfpathclose%
\pgfusepath{fill}%
\end{pgfscope}%
\begin{pgfscope}%
\pgfpathrectangle{\pgfqpoint{1.150000in}{0.150000in}}{\pgfqpoint{5.700000in}{5.700000in}}%
\pgfusepath{clip}%
\pgfsetbuttcap%
\pgfsetroundjoin%
\definecolor{currentfill}{rgb}{0.269944,0.014625,0.341379}%
\pgfsetfillcolor{currentfill}%
\pgfsetfillopacity{0.700000}%
\pgfsetlinewidth{0.000000pt}%
\definecolor{currentstroke}{rgb}{0.000000,0.000000,0.000000}%
\pgfsetstrokecolor{currentstroke}%
\pgfsetdash{}{0pt}%
\pgfpathmoveto{\pgfqpoint{3.594082in}{1.340191in}}%
\pgfpathlineto{\pgfqpoint{3.608160in}{1.336914in}}%
\pgfpathlineto{\pgfqpoint{3.622244in}{1.333711in}}%
\pgfpathlineto{\pgfqpoint{3.636334in}{1.330581in}}%
\pgfpathlineto{\pgfqpoint{3.650432in}{1.327525in}}%
\pgfpathlineto{\pgfqpoint{3.658793in}{1.336619in}}%
\pgfpathlineto{\pgfqpoint{3.667147in}{1.345826in}}%
\pgfpathlineto{\pgfqpoint{3.675494in}{1.355139in}}%
\pgfpathlineto{\pgfqpoint{3.683835in}{1.364555in}}%
\pgfpathlineto{\pgfqpoint{3.669752in}{1.367260in}}%
\pgfpathlineto{\pgfqpoint{3.655677in}{1.370039in}}%
\pgfpathlineto{\pgfqpoint{3.641608in}{1.372891in}}%
\pgfpathlineto{\pgfqpoint{3.627546in}{1.375818in}}%
\pgfpathlineto{\pgfqpoint{3.619191in}{1.366745in}}%
\pgfpathlineto{\pgfqpoint{3.610829in}{1.357780in}}%
\pgfpathlineto{\pgfqpoint{3.602459in}{1.348927in}}%
\pgfpathlineto{\pgfqpoint{3.594082in}{1.340191in}}%
\pgfpathclose%
\pgfusepath{fill}%
\end{pgfscope}%
\begin{pgfscope}%
\pgfpathrectangle{\pgfqpoint{1.150000in}{0.150000in}}{\pgfqpoint{5.700000in}{5.700000in}}%
\pgfusepath{clip}%
\pgfsetbuttcap%
\pgfsetroundjoin%
\definecolor{currentfill}{rgb}{0.216210,0.351535,0.550627}%
\pgfsetfillcolor{currentfill}%
\pgfsetfillopacity{0.700000}%
\pgfsetlinewidth{0.000000pt}%
\definecolor{currentstroke}{rgb}{0.000000,0.000000,0.000000}%
\pgfsetstrokecolor{currentstroke}%
\pgfsetdash{}{0pt}%
\pgfpathmoveto{\pgfqpoint{4.937523in}{2.051246in}}%
\pgfpathlineto{\pgfqpoint{4.952061in}{2.055394in}}%
\pgfpathlineto{\pgfqpoint{4.966611in}{2.059613in}}%
\pgfpathlineto{\pgfqpoint{4.981174in}{2.063904in}}%
\pgfpathlineto{\pgfqpoint{4.995749in}{2.068266in}}%
\pgfpathlineto{\pgfqpoint{5.003634in}{2.078396in}}%
\pgfpathlineto{\pgfqpoint{5.011510in}{2.088392in}}%
\pgfpathlineto{\pgfqpoint{5.019378in}{2.098255in}}%
\pgfpathlineto{\pgfqpoint{5.027239in}{2.107983in}}%
\pgfpathlineto{\pgfqpoint{5.012671in}{2.103582in}}%
\pgfpathlineto{\pgfqpoint{4.998116in}{2.099253in}}%
\pgfpathlineto{\pgfqpoint{4.983573in}{2.094995in}}%
\pgfpathlineto{\pgfqpoint{4.969044in}{2.090808in}}%
\pgfpathlineto{\pgfqpoint{4.961175in}{2.081110in}}%
\pgfpathlineto{\pgfqpoint{4.953299in}{2.071283in}}%
\pgfpathlineto{\pgfqpoint{4.945415in}{2.061329in}}%
\pgfpathlineto{\pgfqpoint{4.937523in}{2.051246in}}%
\pgfpathclose%
\pgfusepath{fill}%
\end{pgfscope}%
\begin{pgfscope}%
\pgfpathrectangle{\pgfqpoint{1.150000in}{0.150000in}}{\pgfqpoint{5.700000in}{5.700000in}}%
\pgfusepath{clip}%
\pgfsetbuttcap%
\pgfsetroundjoin%
\definecolor{currentfill}{rgb}{0.283091,0.110553,0.431554}%
\pgfsetfillcolor{currentfill}%
\pgfsetfillopacity{0.700000}%
\pgfsetlinewidth{0.000000pt}%
\definecolor{currentstroke}{rgb}{0.000000,0.000000,0.000000}%
\pgfsetstrokecolor{currentstroke}%
\pgfsetdash{}{0pt}%
\pgfpathmoveto{\pgfqpoint{4.098730in}{1.510362in}}%
\pgfpathlineto{\pgfqpoint{4.112937in}{1.510381in}}%
\pgfpathlineto{\pgfqpoint{4.127153in}{1.510471in}}%
\pgfpathlineto{\pgfqpoint{4.141378in}{1.510632in}}%
\pgfpathlineto{\pgfqpoint{4.155612in}{1.510865in}}%
\pgfpathlineto{\pgfqpoint{4.163791in}{1.522843in}}%
\pgfpathlineto{\pgfqpoint{4.171964in}{1.534813in}}%
\pgfpathlineto{\pgfqpoint{4.180133in}{1.546772in}}%
\pgfpathlineto{\pgfqpoint{4.188296in}{1.558716in}}%
\pgfpathlineto{\pgfqpoint{4.174069in}{1.558233in}}%
\pgfpathlineto{\pgfqpoint{4.159851in}{1.557821in}}%
\pgfpathlineto{\pgfqpoint{4.145642in}{1.557482in}}%
\pgfpathlineto{\pgfqpoint{4.131443in}{1.557214in}}%
\pgfpathlineto{\pgfqpoint{4.123273in}{1.545512in}}%
\pgfpathlineto{\pgfqpoint{4.115097in}{1.533801in}}%
\pgfpathlineto{\pgfqpoint{4.106916in}{1.522083in}}%
\pgfpathlineto{\pgfqpoint{4.098730in}{1.510362in}}%
\pgfpathclose%
\pgfusepath{fill}%
\end{pgfscope}%
\begin{pgfscope}%
\pgfpathrectangle{\pgfqpoint{1.150000in}{0.150000in}}{\pgfqpoint{5.700000in}{5.700000in}}%
\pgfusepath{clip}%
\pgfsetbuttcap%
\pgfsetroundjoin%
\definecolor{currentfill}{rgb}{0.281924,0.089666,0.412415}%
\pgfsetfillcolor{currentfill}%
\pgfsetfillopacity{0.700000}%
\pgfsetlinewidth{0.000000pt}%
\definecolor{currentstroke}{rgb}{0.000000,0.000000,0.000000}%
\pgfsetstrokecolor{currentstroke}%
\pgfsetdash{}{0pt}%
\pgfpathmoveto{\pgfqpoint{4.009161in}{1.465254in}}%
\pgfpathlineto{\pgfqpoint{4.023341in}{1.464715in}}%
\pgfpathlineto{\pgfqpoint{4.037529in}{1.464248in}}%
\pgfpathlineto{\pgfqpoint{4.051727in}{1.463852in}}%
\pgfpathlineto{\pgfqpoint{4.065932in}{1.463529in}}%
\pgfpathlineto{\pgfqpoint{4.074140in}{1.475222in}}%
\pgfpathlineto{\pgfqpoint{4.082342in}{1.486928in}}%
\pgfpathlineto{\pgfqpoint{4.090539in}{1.498643in}}%
\pgfpathlineto{\pgfqpoint{4.098730in}{1.510362in}}%
\pgfpathlineto{\pgfqpoint{4.084532in}{1.510416in}}%
\pgfpathlineto{\pgfqpoint{4.070343in}{1.510541in}}%
\pgfpathlineto{\pgfqpoint{4.056163in}{1.510738in}}%
\pgfpathlineto{\pgfqpoint{4.041992in}{1.511007in}}%
\pgfpathlineto{\pgfqpoint{4.033792in}{1.499550in}}%
\pgfpathlineto{\pgfqpoint{4.025587in}{1.488103in}}%
\pgfpathlineto{\pgfqpoint{4.017377in}{1.476669in}}%
\pgfpathlineto{\pgfqpoint{4.009161in}{1.465254in}}%
\pgfpathclose%
\pgfusepath{fill}%
\end{pgfscope}%
\begin{pgfscope}%
\pgfpathrectangle{\pgfqpoint{1.150000in}{0.150000in}}{\pgfqpoint{5.700000in}{5.700000in}}%
\pgfusepath{clip}%
\pgfsetbuttcap%
\pgfsetroundjoin%
\definecolor{currentfill}{rgb}{0.282884,0.135920,0.453427}%
\pgfsetfillcolor{currentfill}%
\pgfsetfillopacity{0.700000}%
\pgfsetlinewidth{0.000000pt}%
\definecolor{currentstroke}{rgb}{0.000000,0.000000,0.000000}%
\pgfsetstrokecolor{currentstroke}%
\pgfsetdash{}{0pt}%
\pgfpathmoveto{\pgfqpoint{4.188296in}{1.558716in}}%
\pgfpathlineto{\pgfqpoint{4.202533in}{1.559270in}}%
\pgfpathlineto{\pgfqpoint{4.216779in}{1.559896in}}%
\pgfpathlineto{\pgfqpoint{4.231034in}{1.560593in}}%
\pgfpathlineto{\pgfqpoint{4.245300in}{1.561361in}}%
\pgfpathlineto{\pgfqpoint{4.253451in}{1.573524in}}%
\pgfpathlineto{\pgfqpoint{4.261597in}{1.585660in}}%
\pgfpathlineto{\pgfqpoint{4.269738in}{1.597766in}}%
\pgfpathlineto{\pgfqpoint{4.277874in}{1.609838in}}%
\pgfpathlineto{\pgfqpoint{4.263615in}{1.608840in}}%
\pgfpathlineto{\pgfqpoint{4.249366in}{1.607913in}}%
\pgfpathlineto{\pgfqpoint{4.235126in}{1.607058in}}%
\pgfpathlineto{\pgfqpoint{4.220896in}{1.606274in}}%
\pgfpathlineto{\pgfqpoint{4.212754in}{1.594424in}}%
\pgfpathlineto{\pgfqpoint{4.204607in}{1.582545in}}%
\pgfpathlineto{\pgfqpoint{4.196454in}{1.570641in}}%
\pgfpathlineto{\pgfqpoint{4.188296in}{1.558716in}}%
\pgfpathclose%
\pgfusepath{fill}%
\end{pgfscope}%
\begin{pgfscope}%
\pgfpathrectangle{\pgfqpoint{1.150000in}{0.150000in}}{\pgfqpoint{5.700000in}{5.700000in}}%
\pgfusepath{clip}%
\pgfsetbuttcap%
\pgfsetroundjoin%
\definecolor{currentfill}{rgb}{0.269944,0.014625,0.341379}%
\pgfsetfillcolor{currentfill}%
\pgfsetfillopacity{0.700000}%
\pgfsetlinewidth{0.000000pt}%
\definecolor{currentstroke}{rgb}{0.000000,0.000000,0.000000}%
\pgfsetstrokecolor{currentstroke}%
\pgfsetdash{}{0pt}%
\pgfpathmoveto{\pgfqpoint{3.211386in}{1.351489in}}%
\pgfpathlineto{\pgfqpoint{3.225417in}{1.345514in}}%
\pgfpathlineto{\pgfqpoint{3.239453in}{1.339618in}}%
\pgfpathlineto{\pgfqpoint{3.253493in}{1.333799in}}%
\pgfpathlineto{\pgfqpoint{3.267538in}{1.328058in}}%
\pgfpathlineto{\pgfqpoint{3.276105in}{1.333253in}}%
\pgfpathlineto{\pgfqpoint{3.284662in}{1.338658in}}%
\pgfpathlineto{\pgfqpoint{3.293207in}{1.344265in}}%
\pgfpathlineto{\pgfqpoint{3.301743in}{1.350068in}}%
\pgfpathlineto{\pgfqpoint{3.287722in}{1.355396in}}%
\pgfpathlineto{\pgfqpoint{3.273706in}{1.360801in}}%
\pgfpathlineto{\pgfqpoint{3.259695in}{1.366284in}}%
\pgfpathlineto{\pgfqpoint{3.245688in}{1.371846in}}%
\pgfpathlineto{\pgfqpoint{3.237129in}{1.366448in}}%
\pgfpathlineto{\pgfqpoint{3.228559in}{1.361251in}}%
\pgfpathlineto{\pgfqpoint{3.219978in}{1.356263in}}%
\pgfpathlineto{\pgfqpoint{3.211386in}{1.351489in}}%
\pgfpathclose%
\pgfusepath{fill}%
\end{pgfscope}%
\begin{pgfscope}%
\pgfpathrectangle{\pgfqpoint{1.150000in}{0.150000in}}{\pgfqpoint{5.700000in}{5.700000in}}%
\pgfusepath{clip}%
\pgfsetbuttcap%
\pgfsetroundjoin%
\definecolor{currentfill}{rgb}{0.267004,0.004874,0.329415}%
\pgfsetfillcolor{currentfill}%
\pgfsetfillopacity{0.700000}%
\pgfsetlinewidth{0.000000pt}%
\definecolor{currentstroke}{rgb}{0.000000,0.000000,0.000000}%
\pgfsetstrokecolor{currentstroke}%
\pgfsetdash{}{0pt}%
\pgfpathmoveto{\pgfqpoint{3.357875in}{1.329530in}}%
\pgfpathlineto{\pgfqpoint{3.371920in}{1.324587in}}%
\pgfpathlineto{\pgfqpoint{3.385971in}{1.319721in}}%
\pgfpathlineto{\pgfqpoint{3.400027in}{1.314931in}}%
\pgfpathlineto{\pgfqpoint{3.414089in}{1.310216in}}%
\pgfpathlineto{\pgfqpoint{3.422570in}{1.317010in}}%
\pgfpathlineto{\pgfqpoint{3.431042in}{1.323977in}}%
\pgfpathlineto{\pgfqpoint{3.439505in}{1.331112in}}%
\pgfpathlineto{\pgfqpoint{3.447959in}{1.338408in}}%
\pgfpathlineto{\pgfqpoint{3.433917in}{1.342731in}}%
\pgfpathlineto{\pgfqpoint{3.419882in}{1.347129in}}%
\pgfpathlineto{\pgfqpoint{3.405852in}{1.351603in}}%
\pgfpathlineto{\pgfqpoint{3.391827in}{1.356154in}}%
\pgfpathlineto{\pgfqpoint{3.383353in}{1.349242in}}%
\pgfpathlineto{\pgfqpoint{3.374870in}{1.342496in}}%
\pgfpathlineto{\pgfqpoint{3.366377in}{1.335924in}}%
\pgfpathlineto{\pgfqpoint{3.357875in}{1.329530in}}%
\pgfpathclose%
\pgfusepath{fill}%
\end{pgfscope}%
\begin{pgfscope}%
\pgfpathrectangle{\pgfqpoint{1.150000in}{0.150000in}}{\pgfqpoint{5.700000in}{5.700000in}}%
\pgfusepath{clip}%
\pgfsetbuttcap%
\pgfsetroundjoin%
\definecolor{currentfill}{rgb}{0.279566,0.067836,0.391917}%
\pgfsetfillcolor{currentfill}%
\pgfsetfillopacity{0.700000}%
\pgfsetlinewidth{0.000000pt}%
\definecolor{currentstroke}{rgb}{0.000000,0.000000,0.000000}%
\pgfsetstrokecolor{currentstroke}%
\pgfsetdash{}{0pt}%
\pgfpathmoveto{\pgfqpoint{3.919571in}{1.423886in}}%
\pgfpathlineto{\pgfqpoint{3.933726in}{1.422769in}}%
\pgfpathlineto{\pgfqpoint{3.947890in}{1.421724in}}%
\pgfpathlineto{\pgfqpoint{3.962063in}{1.420750in}}%
\pgfpathlineto{\pgfqpoint{3.976243in}{1.419849in}}%
\pgfpathlineto{\pgfqpoint{3.984481in}{1.431153in}}%
\pgfpathlineto{\pgfqpoint{3.992713in}{1.442492in}}%
\pgfpathlineto{\pgfqpoint{4.000940in}{1.453860in}}%
\pgfpathlineto{\pgfqpoint{4.009161in}{1.465254in}}%
\pgfpathlineto{\pgfqpoint{3.994990in}{1.465865in}}%
\pgfpathlineto{\pgfqpoint{3.980827in}{1.466547in}}%
\pgfpathlineto{\pgfqpoint{3.966672in}{1.467302in}}%
\pgfpathlineto{\pgfqpoint{3.952527in}{1.468129in}}%
\pgfpathlineto{\pgfqpoint{3.944296in}{1.457018in}}%
\pgfpathlineto{\pgfqpoint{3.936060in}{1.445937in}}%
\pgfpathlineto{\pgfqpoint{3.927818in}{1.434892in}}%
\pgfpathlineto{\pgfqpoint{3.919571in}{1.423886in}}%
\pgfpathclose%
\pgfusepath{fill}%
\end{pgfscope}%
\begin{pgfscope}%
\pgfpathrectangle{\pgfqpoint{1.150000in}{0.150000in}}{\pgfqpoint{5.700000in}{5.700000in}}%
\pgfusepath{clip}%
\pgfsetbuttcap%
\pgfsetroundjoin%
\definecolor{currentfill}{rgb}{0.280868,0.160771,0.472899}%
\pgfsetfillcolor{currentfill}%
\pgfsetfillopacity{0.700000}%
\pgfsetlinewidth{0.000000pt}%
\definecolor{currentstroke}{rgb}{0.000000,0.000000,0.000000}%
\pgfsetstrokecolor{currentstroke}%
\pgfsetdash{}{0pt}%
\pgfpathmoveto{\pgfqpoint{4.277874in}{1.609838in}}%
\pgfpathlineto{\pgfqpoint{4.292143in}{1.610908in}}%
\pgfpathlineto{\pgfqpoint{4.306421in}{1.612048in}}%
\pgfpathlineto{\pgfqpoint{4.320710in}{1.613260in}}%
\pgfpathlineto{\pgfqpoint{4.335009in}{1.614543in}}%
\pgfpathlineto{\pgfqpoint{4.343134in}{1.626796in}}%
\pgfpathlineto{\pgfqpoint{4.351253in}{1.639005in}}%
\pgfpathlineto{\pgfqpoint{4.359367in}{1.651166in}}%
\pgfpathlineto{\pgfqpoint{4.367475in}{1.663276in}}%
\pgfpathlineto{\pgfqpoint{4.353182in}{1.661784in}}%
\pgfpathlineto{\pgfqpoint{4.338900in}{1.660363in}}%
\pgfpathlineto{\pgfqpoint{4.324627in}{1.659013in}}%
\pgfpathlineto{\pgfqpoint{4.310364in}{1.657735in}}%
\pgfpathlineto{\pgfqpoint{4.302250in}{1.645825in}}%
\pgfpathlineto{\pgfqpoint{4.294130in}{1.633871in}}%
\pgfpathlineto{\pgfqpoint{4.286004in}{1.621874in}}%
\pgfpathlineto{\pgfqpoint{4.277874in}{1.609838in}}%
\pgfpathclose%
\pgfusepath{fill}%
\end{pgfscope}%
\begin{pgfscope}%
\pgfpathrectangle{\pgfqpoint{1.150000in}{0.150000in}}{\pgfqpoint{5.700000in}{5.700000in}}%
\pgfusepath{clip}%
\pgfsetbuttcap%
\pgfsetroundjoin%
\definecolor{currentfill}{rgb}{0.204903,0.375746,0.553533}%
\pgfsetfillcolor{currentfill}%
\pgfsetfillopacity{0.700000}%
\pgfsetlinewidth{0.000000pt}%
\definecolor{currentstroke}{rgb}{0.000000,0.000000,0.000000}%
\pgfsetstrokecolor{currentstroke}%
\pgfsetdash{}{0pt}%
\pgfpathmoveto{\pgfqpoint{5.027239in}{2.107983in}}%
\pgfpathlineto{\pgfqpoint{5.041820in}{2.112456in}}%
\pgfpathlineto{\pgfqpoint{5.056414in}{2.116999in}}%
\pgfpathlineto{\pgfqpoint{5.071020in}{2.121614in}}%
\pgfpathlineto{\pgfqpoint{5.085640in}{2.126301in}}%
\pgfpathlineto{\pgfqpoint{5.093485in}{2.135921in}}%
\pgfpathlineto{\pgfqpoint{5.101321in}{2.145401in}}%
\pgfpathlineto{\pgfqpoint{5.109149in}{2.154743in}}%
\pgfpathlineto{\pgfqpoint{5.116969in}{2.163945in}}%
\pgfpathlineto{\pgfqpoint{5.102358in}{2.159242in}}%
\pgfpathlineto{\pgfqpoint{5.087760in}{2.154610in}}%
\pgfpathlineto{\pgfqpoint{5.073175in}{2.150050in}}%
\pgfpathlineto{\pgfqpoint{5.058602in}{2.145561in}}%
\pgfpathlineto{\pgfqpoint{5.050774in}{2.136367in}}%
\pgfpathlineto{\pgfqpoint{5.042937in}{2.127039in}}%
\pgfpathlineto{\pgfqpoint{5.035092in}{2.117578in}}%
\pgfpathlineto{\pgfqpoint{5.027239in}{2.107983in}}%
\pgfpathclose%
\pgfusepath{fill}%
\end{pgfscope}%
\begin{pgfscope}%
\pgfpathrectangle{\pgfqpoint{1.150000in}{0.150000in}}{\pgfqpoint{5.700000in}{5.700000in}}%
\pgfusepath{clip}%
\pgfsetbuttcap%
\pgfsetroundjoin%
\definecolor{currentfill}{rgb}{0.277134,0.185228,0.489898}%
\pgfsetfillcolor{currentfill}%
\pgfsetfillopacity{0.700000}%
\pgfsetlinewidth{0.000000pt}%
\definecolor{currentstroke}{rgb}{0.000000,0.000000,0.000000}%
\pgfsetstrokecolor{currentstroke}%
\pgfsetdash{}{0pt}%
\pgfpathmoveto{\pgfqpoint{4.367475in}{1.663276in}}%
\pgfpathlineto{\pgfqpoint{4.381779in}{1.664840in}}%
\pgfpathlineto{\pgfqpoint{4.396092in}{1.666474in}}%
\pgfpathlineto{\pgfqpoint{4.410416in}{1.668180in}}%
\pgfpathlineto{\pgfqpoint{4.424751in}{1.669958in}}%
\pgfpathlineto{\pgfqpoint{4.432849in}{1.682211in}}%
\pgfpathlineto{\pgfqpoint{4.440941in}{1.694404in}}%
\pgfpathlineto{\pgfqpoint{4.449028in}{1.706533in}}%
\pgfpathlineto{\pgfqpoint{4.457109in}{1.718595in}}%
\pgfpathlineto{\pgfqpoint{4.442780in}{1.716630in}}%
\pgfpathlineto{\pgfqpoint{4.428462in}{1.714735in}}%
\pgfpathlineto{\pgfqpoint{4.414154in}{1.712912in}}%
\pgfpathlineto{\pgfqpoint{4.399856in}{1.711160in}}%
\pgfpathlineto{\pgfqpoint{4.391769in}{1.699278in}}%
\pgfpathlineto{\pgfqpoint{4.383677in}{1.687335in}}%
\pgfpathlineto{\pgfqpoint{4.375579in}{1.675333in}}%
\pgfpathlineto{\pgfqpoint{4.367475in}{1.663276in}}%
\pgfpathclose%
\pgfusepath{fill}%
\end{pgfscope}%
\begin{pgfscope}%
\pgfpathrectangle{\pgfqpoint{1.150000in}{0.150000in}}{\pgfqpoint{5.700000in}{5.700000in}}%
\pgfusepath{clip}%
\pgfsetbuttcap%
\pgfsetroundjoin%
\definecolor{currentfill}{rgb}{0.276022,0.044167,0.370164}%
\pgfsetfillcolor{currentfill}%
\pgfsetfillopacity{0.700000}%
\pgfsetlinewidth{0.000000pt}%
\definecolor{currentstroke}{rgb}{0.000000,0.000000,0.000000}%
\pgfsetstrokecolor{currentstroke}%
\pgfsetdash{}{0pt}%
\pgfpathmoveto{\pgfqpoint{3.829937in}{1.386778in}}%
\pgfpathlineto{\pgfqpoint{3.844072in}{1.385060in}}%
\pgfpathlineto{\pgfqpoint{3.858214in}{1.383416in}}%
\pgfpathlineto{\pgfqpoint{3.872365in}{1.381843in}}%
\pgfpathlineto{\pgfqpoint{3.886523in}{1.380343in}}%
\pgfpathlineto{\pgfqpoint{3.894794in}{1.391148in}}%
\pgfpathlineto{\pgfqpoint{3.903058in}{1.402009in}}%
\pgfpathlineto{\pgfqpoint{3.911317in}{1.412924in}}%
\pgfpathlineto{\pgfqpoint{3.919571in}{1.423886in}}%
\pgfpathlineto{\pgfqpoint{3.905423in}{1.425076in}}%
\pgfpathlineto{\pgfqpoint{3.891283in}{1.426337in}}%
\pgfpathlineto{\pgfqpoint{3.877152in}{1.427671in}}%
\pgfpathlineto{\pgfqpoint{3.863028in}{1.429078in}}%
\pgfpathlineto{\pgfqpoint{3.854764in}{1.418419in}}%
\pgfpathlineto{\pgfqpoint{3.846494in}{1.407812in}}%
\pgfpathlineto{\pgfqpoint{3.838219in}{1.397264in}}%
\pgfpathlineto{\pgfqpoint{3.829937in}{1.386778in}}%
\pgfpathclose%
\pgfusepath{fill}%
\end{pgfscope}%
\begin{pgfscope}%
\pgfpathrectangle{\pgfqpoint{1.150000in}{0.150000in}}{\pgfqpoint{5.700000in}{5.700000in}}%
\pgfusepath{clip}%
\pgfsetbuttcap%
\pgfsetroundjoin%
\definecolor{currentfill}{rgb}{0.195860,0.395433,0.555276}%
\pgfsetfillcolor{currentfill}%
\pgfsetfillopacity{0.700000}%
\pgfsetlinewidth{0.000000pt}%
\definecolor{currentstroke}{rgb}{0.000000,0.000000,0.000000}%
\pgfsetstrokecolor{currentstroke}%
\pgfsetdash{}{0pt}%
\pgfpathmoveto{\pgfqpoint{5.116969in}{2.163945in}}%
\pgfpathlineto{\pgfqpoint{5.131594in}{2.168720in}}%
\pgfpathlineto{\pgfqpoint{5.146231in}{2.173566in}}%
\pgfpathlineto{\pgfqpoint{5.160882in}{2.178484in}}%
\pgfpathlineto{\pgfqpoint{5.175546in}{2.183473in}}%
\pgfpathlineto{\pgfqpoint{5.183349in}{2.192540in}}%
\pgfpathlineto{\pgfqpoint{5.191142in}{2.201463in}}%
\pgfpathlineto{\pgfqpoint{5.198928in}{2.210243in}}%
\pgfpathlineto{\pgfqpoint{5.206704in}{2.218881in}}%
\pgfpathlineto{\pgfqpoint{5.192049in}{2.213898in}}%
\pgfpathlineto{\pgfqpoint{5.177408in}{2.208986in}}%
\pgfpathlineto{\pgfqpoint{5.162780in}{2.204145in}}%
\pgfpathlineto{\pgfqpoint{5.148165in}{2.199376in}}%
\pgfpathlineto{\pgfqpoint{5.140379in}{2.190724in}}%
\pgfpathlineto{\pgfqpoint{5.132584in}{2.181936in}}%
\pgfpathlineto{\pgfqpoint{5.124781in}{2.173009in}}%
\pgfpathlineto{\pgfqpoint{5.116969in}{2.163945in}}%
\pgfpathclose%
\pgfusepath{fill}%
\end{pgfscope}%
\begin{pgfscope}%
\pgfpathrectangle{\pgfqpoint{1.150000in}{0.150000in}}{\pgfqpoint{5.700000in}{5.700000in}}%
\pgfusepath{clip}%
\pgfsetbuttcap%
\pgfsetroundjoin%
\definecolor{currentfill}{rgb}{0.267004,0.004874,0.329415}%
\pgfsetfillcolor{currentfill}%
\pgfsetfillopacity{0.700000}%
\pgfsetlinewidth{0.000000pt}%
\definecolor{currentstroke}{rgb}{0.000000,0.000000,0.000000}%
\pgfsetstrokecolor{currentstroke}%
\pgfsetdash{}{0pt}%
\pgfpathmoveto{\pgfqpoint{3.504180in}{1.321873in}}%
\pgfpathlineto{\pgfqpoint{3.518250in}{1.317926in}}%
\pgfpathlineto{\pgfqpoint{3.532326in}{1.314055in}}%
\pgfpathlineto{\pgfqpoint{3.546408in}{1.310257in}}%
\pgfpathlineto{\pgfqpoint{3.560496in}{1.306534in}}%
\pgfpathlineto{\pgfqpoint{3.568904in}{1.314745in}}%
\pgfpathlineto{\pgfqpoint{3.577305in}{1.323095in}}%
\pgfpathlineto{\pgfqpoint{3.585697in}{1.331579in}}%
\pgfpathlineto{\pgfqpoint{3.594082in}{1.340191in}}%
\pgfpathlineto{\pgfqpoint{3.580011in}{1.343543in}}%
\pgfpathlineto{\pgfqpoint{3.565946in}{1.346969in}}%
\pgfpathlineto{\pgfqpoint{3.551888in}{1.350469in}}%
\pgfpathlineto{\pgfqpoint{3.537835in}{1.354044in}}%
\pgfpathlineto{\pgfqpoint{3.529434in}{1.345795in}}%
\pgfpathlineto{\pgfqpoint{3.521024in}{1.337680in}}%
\pgfpathlineto{\pgfqpoint{3.512606in}{1.329704in}}%
\pgfpathlineto{\pgfqpoint{3.504180in}{1.321873in}}%
\pgfpathclose%
\pgfusepath{fill}%
\end{pgfscope}%
\begin{pgfscope}%
\pgfpathrectangle{\pgfqpoint{1.150000in}{0.150000in}}{\pgfqpoint{5.700000in}{5.700000in}}%
\pgfusepath{clip}%
\pgfsetbuttcap%
\pgfsetroundjoin%
\definecolor{currentfill}{rgb}{0.171176,0.452530,0.557965}%
\pgfsetfillcolor{currentfill}%
\pgfsetfillopacity{0.700000}%
\pgfsetlinewidth{0.000000pt}%
\definecolor{currentstroke}{rgb}{0.000000,0.000000,0.000000}%
\pgfsetstrokecolor{currentstroke}%
\pgfsetdash{}{0pt}%
\pgfpathmoveto{\pgfqpoint{5.386145in}{2.324781in}}%
\pgfpathlineto{\pgfqpoint{5.400900in}{2.330330in}}%
\pgfpathlineto{\pgfqpoint{5.415670in}{2.335951in}}%
\pgfpathlineto{\pgfqpoint{5.430454in}{2.341643in}}%
\pgfpathlineto{\pgfqpoint{5.438115in}{2.348866in}}%
\pgfpathlineto{\pgfqpoint{5.445768in}{2.355942in}}%
\pgfpathlineto{\pgfqpoint{5.453410in}{2.362872in}}%
\pgfpathlineto{\pgfqpoint{5.461042in}{2.369660in}}%
\pgfpathlineto{\pgfqpoint{5.446272in}{2.364041in}}%
\pgfpathlineto{\pgfqpoint{5.431515in}{2.358493in}}%
\pgfpathlineto{\pgfqpoint{5.416773in}{2.353017in}}%
\pgfpathlineto{\pgfqpoint{5.409130in}{2.346168in}}%
\pgfpathlineto{\pgfqpoint{5.401478in}{2.339181in}}%
\pgfpathlineto{\pgfqpoint{5.393816in}{2.332052in}}%
\pgfpathlineto{\pgfqpoint{5.386145in}{2.324781in}}%
\pgfpathclose%
\pgfusepath{fill}%
\end{pgfscope}%
\begin{pgfscope}%
\pgfpathrectangle{\pgfqpoint{1.150000in}{0.150000in}}{\pgfqpoint{5.700000in}{5.700000in}}%
\pgfusepath{clip}%
\pgfsetbuttcap%
\pgfsetroundjoin%
\definecolor{currentfill}{rgb}{0.270595,0.214069,0.507052}%
\pgfsetfillcolor{currentfill}%
\pgfsetfillopacity{0.700000}%
\pgfsetlinewidth{0.000000pt}%
\definecolor{currentstroke}{rgb}{0.000000,0.000000,0.000000}%
\pgfsetstrokecolor{currentstroke}%
\pgfsetdash{}{0pt}%
\pgfpathmoveto{\pgfqpoint{4.457109in}{1.718595in}}%
\pgfpathlineto{\pgfqpoint{4.471449in}{1.720632in}}%
\pgfpathlineto{\pgfqpoint{4.485800in}{1.722740in}}%
\pgfpathlineto{\pgfqpoint{4.500161in}{1.724919in}}%
\pgfpathlineto{\pgfqpoint{4.514533in}{1.727170in}}%
\pgfpathlineto{\pgfqpoint{4.522604in}{1.739339in}}%
\pgfpathlineto{\pgfqpoint{4.530669in}{1.751433in}}%
\pgfpathlineto{\pgfqpoint{4.538728in}{1.763448in}}%
\pgfpathlineto{\pgfqpoint{4.546782in}{1.775382in}}%
\pgfpathlineto{\pgfqpoint{4.532415in}{1.772964in}}%
\pgfpathlineto{\pgfqpoint{4.518059in}{1.770617in}}%
\pgfpathlineto{\pgfqpoint{4.503714in}{1.768342in}}%
\pgfpathlineto{\pgfqpoint{4.489380in}{1.766137in}}%
\pgfpathlineto{\pgfqpoint{4.481321in}{1.754363in}}%
\pgfpathlineto{\pgfqpoint{4.473256in}{1.742513in}}%
\pgfpathlineto{\pgfqpoint{4.465185in}{1.730589in}}%
\pgfpathlineto{\pgfqpoint{4.457109in}{1.718595in}}%
\pgfpathclose%
\pgfusepath{fill}%
\end{pgfscope}%
\begin{pgfscope}%
\pgfpathrectangle{\pgfqpoint{1.150000in}{0.150000in}}{\pgfqpoint{5.700000in}{5.700000in}}%
\pgfusepath{clip}%
\pgfsetbuttcap%
\pgfsetroundjoin%
\definecolor{currentfill}{rgb}{0.187231,0.414746,0.556547}%
\pgfsetfillcolor{currentfill}%
\pgfsetfillopacity{0.700000}%
\pgfsetlinewidth{0.000000pt}%
\definecolor{currentstroke}{rgb}{0.000000,0.000000,0.000000}%
\pgfsetstrokecolor{currentstroke}%
\pgfsetdash{}{0pt}%
\pgfpathmoveto{\pgfqpoint{5.206704in}{2.218881in}}%
\pgfpathlineto{\pgfqpoint{5.221372in}{2.223936in}}%
\pgfpathlineto{\pgfqpoint{5.236054in}{2.229063in}}%
\pgfpathlineto{\pgfqpoint{5.250749in}{2.234261in}}%
\pgfpathlineto{\pgfqpoint{5.265458in}{2.239531in}}%
\pgfpathlineto{\pgfqpoint{5.273215in}{2.248007in}}%
\pgfpathlineto{\pgfqpoint{5.280964in}{2.256338in}}%
\pgfpathlineto{\pgfqpoint{5.288703in}{2.264522in}}%
\pgfpathlineto{\pgfqpoint{5.296433in}{2.272563in}}%
\pgfpathlineto{\pgfqpoint{5.281735in}{2.267321in}}%
\pgfpathlineto{\pgfqpoint{5.267050in}{2.262151in}}%
\pgfpathlineto{\pgfqpoint{5.252379in}{2.257052in}}%
\pgfpathlineto{\pgfqpoint{5.237722in}{2.252025in}}%
\pgfpathlineto{\pgfqpoint{5.229980in}{2.243949in}}%
\pgfpathlineto{\pgfqpoint{5.222230in}{2.235733in}}%
\pgfpathlineto{\pgfqpoint{5.214472in}{2.227378in}}%
\pgfpathlineto{\pgfqpoint{5.206704in}{2.218881in}}%
\pgfpathclose%
\pgfusepath{fill}%
\end{pgfscope}%
\begin{pgfscope}%
\pgfpathrectangle{\pgfqpoint{1.150000in}{0.150000in}}{\pgfqpoint{5.700000in}{5.700000in}}%
\pgfusepath{clip}%
\pgfsetbuttcap%
\pgfsetroundjoin%
\definecolor{currentfill}{rgb}{0.177423,0.437527,0.557565}%
\pgfsetfillcolor{currentfill}%
\pgfsetfillopacity{0.700000}%
\pgfsetlinewidth{0.000000pt}%
\definecolor{currentstroke}{rgb}{0.000000,0.000000,0.000000}%
\pgfsetstrokecolor{currentstroke}%
\pgfsetdash{}{0pt}%
\pgfpathmoveto{\pgfqpoint{5.296433in}{2.272563in}}%
\pgfpathlineto{\pgfqpoint{5.311145in}{2.277876in}}%
\pgfpathlineto{\pgfqpoint{5.325871in}{2.283261in}}%
\pgfpathlineto{\pgfqpoint{5.340610in}{2.288717in}}%
\pgfpathlineto{\pgfqpoint{5.355364in}{2.294245in}}%
\pgfpathlineto{\pgfqpoint{5.363073in}{2.302099in}}%
\pgfpathlineto{\pgfqpoint{5.370773in}{2.309806in}}%
\pgfpathlineto{\pgfqpoint{5.378464in}{2.317366in}}%
\pgfpathlineto{\pgfqpoint{5.386145in}{2.324781in}}%
\pgfpathlineto{\pgfqpoint{5.371403in}{2.319304in}}%
\pgfpathlineto{\pgfqpoint{5.356675in}{2.313898in}}%
\pgfpathlineto{\pgfqpoint{5.341962in}{2.308563in}}%
\pgfpathlineto{\pgfqpoint{5.327261in}{2.303300in}}%
\pgfpathlineto{\pgfqpoint{5.319568in}{2.295827in}}%
\pgfpathlineto{\pgfqpoint{5.311866in}{2.288214in}}%
\pgfpathlineto{\pgfqpoint{5.304154in}{2.280459in}}%
\pgfpathlineto{\pgfqpoint{5.296433in}{2.272563in}}%
\pgfpathclose%
\pgfusepath{fill}%
\end{pgfscope}%
\begin{pgfscope}%
\pgfpathrectangle{\pgfqpoint{1.150000in}{0.150000in}}{\pgfqpoint{5.700000in}{5.700000in}}%
\pgfusepath{clip}%
\pgfsetbuttcap%
\pgfsetroundjoin%
\definecolor{currentfill}{rgb}{0.273809,0.031497,0.358853}%
\pgfsetfillcolor{currentfill}%
\pgfsetfillopacity{0.700000}%
\pgfsetlinewidth{0.000000pt}%
\definecolor{currentstroke}{rgb}{0.000000,0.000000,0.000000}%
\pgfsetstrokecolor{currentstroke}%
\pgfsetdash{}{0pt}%
\pgfpathmoveto{\pgfqpoint{3.740234in}{1.354469in}}%
\pgfpathlineto{\pgfqpoint{3.754351in}{1.352130in}}%
\pgfpathlineto{\pgfqpoint{3.768476in}{1.349864in}}%
\pgfpathlineto{\pgfqpoint{3.782608in}{1.347671in}}%
\pgfpathlineto{\pgfqpoint{3.796748in}{1.345550in}}%
\pgfpathlineto{\pgfqpoint{3.805054in}{1.355740in}}%
\pgfpathlineto{\pgfqpoint{3.813355in}{1.366011in}}%
\pgfpathlineto{\pgfqpoint{3.821649in}{1.376358in}}%
\pgfpathlineto{\pgfqpoint{3.829937in}{1.386778in}}%
\pgfpathlineto{\pgfqpoint{3.815809in}{1.388567in}}%
\pgfpathlineto{\pgfqpoint{3.801690in}{1.390430in}}%
\pgfpathlineto{\pgfqpoint{3.787578in}{1.392365in}}%
\pgfpathlineto{\pgfqpoint{3.773473in}{1.394373in}}%
\pgfpathlineto{\pgfqpoint{3.765173in}{1.384276in}}%
\pgfpathlineto{\pgfqpoint{3.756866in}{1.374257in}}%
\pgfpathlineto{\pgfqpoint{3.748553in}{1.364320in}}%
\pgfpathlineto{\pgfqpoint{3.740234in}{1.354469in}}%
\pgfpathclose%
\pgfusepath{fill}%
\end{pgfscope}%
\begin{pgfscope}%
\pgfpathrectangle{\pgfqpoint{1.150000in}{0.150000in}}{\pgfqpoint{5.700000in}{5.700000in}}%
\pgfusepath{clip}%
\pgfsetbuttcap%
\pgfsetroundjoin%
\definecolor{currentfill}{rgb}{0.262138,0.242286,0.520837}%
\pgfsetfillcolor{currentfill}%
\pgfsetfillopacity{0.700000}%
\pgfsetlinewidth{0.000000pt}%
\definecolor{currentstroke}{rgb}{0.000000,0.000000,0.000000}%
\pgfsetstrokecolor{currentstroke}%
\pgfsetdash{}{0pt}%
\pgfpathmoveto{\pgfqpoint{4.546782in}{1.775382in}}%
\pgfpathlineto{\pgfqpoint{4.561160in}{1.777872in}}%
\pgfpathlineto{\pgfqpoint{4.575549in}{1.780432in}}%
\pgfpathlineto{\pgfqpoint{4.589949in}{1.783063in}}%
\pgfpathlineto{\pgfqpoint{4.604361in}{1.785766in}}%
\pgfpathlineto{\pgfqpoint{4.612403in}{1.797772in}}%
\pgfpathlineto{\pgfqpoint{4.620440in}{1.809689in}}%
\pgfpathlineto{\pgfqpoint{4.628471in}{1.821513in}}%
\pgfpathlineto{\pgfqpoint{4.636496in}{1.833244in}}%
\pgfpathlineto{\pgfqpoint{4.622089in}{1.830394in}}%
\pgfpathlineto{\pgfqpoint{4.607694in}{1.827616in}}%
\pgfpathlineto{\pgfqpoint{4.593311in}{1.824909in}}%
\pgfpathlineto{\pgfqpoint{4.578938in}{1.822274in}}%
\pgfpathlineto{\pgfqpoint{4.570908in}{1.810681in}}%
\pgfpathlineto{\pgfqpoint{4.562872in}{1.799001in}}%
\pgfpathlineto{\pgfqpoint{4.554830in}{1.787234in}}%
\pgfpathlineto{\pgfqpoint{4.546782in}{1.775382in}}%
\pgfpathclose%
\pgfusepath{fill}%
\end{pgfscope}%
\begin{pgfscope}%
\pgfpathrectangle{\pgfqpoint{1.150000in}{0.150000in}}{\pgfqpoint{5.700000in}{5.700000in}}%
\pgfusepath{clip}%
\pgfsetbuttcap%
\pgfsetroundjoin%
\definecolor{currentfill}{rgb}{0.252194,0.269783,0.531579}%
\pgfsetfillcolor{currentfill}%
\pgfsetfillopacity{0.700000}%
\pgfsetlinewidth{0.000000pt}%
\definecolor{currentstroke}{rgb}{0.000000,0.000000,0.000000}%
\pgfsetstrokecolor{currentstroke}%
\pgfsetdash{}{0pt}%
\pgfpathmoveto{\pgfqpoint{4.636496in}{1.833244in}}%
\pgfpathlineto{\pgfqpoint{4.650914in}{1.836164in}}%
\pgfpathlineto{\pgfqpoint{4.665343in}{1.839156in}}%
\pgfpathlineto{\pgfqpoint{4.679784in}{1.842219in}}%
\pgfpathlineto{\pgfqpoint{4.694236in}{1.845353in}}%
\pgfpathlineto{\pgfqpoint{4.702249in}{1.857122in}}%
\pgfpathlineto{\pgfqpoint{4.710257in}{1.868788in}}%
\pgfpathlineto{\pgfqpoint{4.718258in}{1.880350in}}%
\pgfpathlineto{\pgfqpoint{4.726252in}{1.891806in}}%
\pgfpathlineto{\pgfqpoint{4.711805in}{1.888547in}}%
\pgfpathlineto{\pgfqpoint{4.697369in}{1.885358in}}%
\pgfpathlineto{\pgfqpoint{4.682946in}{1.882241in}}%
\pgfpathlineto{\pgfqpoint{4.668533in}{1.879195in}}%
\pgfpathlineto{\pgfqpoint{4.660533in}{1.867856in}}%
\pgfpathlineto{\pgfqpoint{4.652527in}{1.856417in}}%
\pgfpathlineto{\pgfqpoint{4.644514in}{1.844879in}}%
\pgfpathlineto{\pgfqpoint{4.636496in}{1.833244in}}%
\pgfpathclose%
\pgfusepath{fill}%
\end{pgfscope}%
\begin{pgfscope}%
\pgfpathrectangle{\pgfqpoint{1.150000in}{0.150000in}}{\pgfqpoint{5.700000in}{5.700000in}}%
\pgfusepath{clip}%
\pgfsetbuttcap%
\pgfsetroundjoin%
\definecolor{currentfill}{rgb}{0.269944,0.014625,0.341379}%
\pgfsetfillcolor{currentfill}%
\pgfsetfillopacity{0.700000}%
\pgfsetlinewidth{0.000000pt}%
\definecolor{currentstroke}{rgb}{0.000000,0.000000,0.000000}%
\pgfsetstrokecolor{currentstroke}%
\pgfsetdash{}{0pt}%
\pgfpathmoveto{\pgfqpoint{3.650432in}{1.327525in}}%
\pgfpathlineto{\pgfqpoint{3.664535in}{1.324542in}}%
\pgfpathlineto{\pgfqpoint{3.678646in}{1.321633in}}%
\pgfpathlineto{\pgfqpoint{3.692764in}{1.318797in}}%
\pgfpathlineto{\pgfqpoint{3.706888in}{1.316034in}}%
\pgfpathlineto{\pgfqpoint{3.715235in}{1.325487in}}%
\pgfpathlineto{\pgfqpoint{3.723575in}{1.335048in}}%
\pgfpathlineto{\pgfqpoint{3.731907in}{1.344710in}}%
\pgfpathlineto{\pgfqpoint{3.740234in}{1.354469in}}%
\pgfpathlineto{\pgfqpoint{3.726123in}{1.356881in}}%
\pgfpathlineto{\pgfqpoint{3.712020in}{1.359366in}}%
\pgfpathlineto{\pgfqpoint{3.697924in}{1.361924in}}%
\pgfpathlineto{\pgfqpoint{3.683835in}{1.364555in}}%
\pgfpathlineto{\pgfqpoint{3.675494in}{1.355139in}}%
\pgfpathlineto{\pgfqpoint{3.667147in}{1.345826in}}%
\pgfpathlineto{\pgfqpoint{3.658793in}{1.336619in}}%
\pgfpathlineto{\pgfqpoint{3.650432in}{1.327525in}}%
\pgfpathclose%
\pgfusepath{fill}%
\end{pgfscope}%
\begin{pgfscope}%
\pgfpathrectangle{\pgfqpoint{1.150000in}{0.150000in}}{\pgfqpoint{5.700000in}{5.700000in}}%
\pgfusepath{clip}%
\pgfsetbuttcap%
\pgfsetroundjoin%
\definecolor{currentfill}{rgb}{0.268510,0.009605,0.335427}%
\pgfsetfillcolor{currentfill}%
\pgfsetfillopacity{0.700000}%
\pgfsetlinewidth{0.000000pt}%
\definecolor{currentstroke}{rgb}{0.000000,0.000000,0.000000}%
\pgfsetstrokecolor{currentstroke}%
\pgfsetdash{}{0pt}%
\pgfpathmoveto{\pgfqpoint{3.267538in}{1.328058in}}%
\pgfpathlineto{\pgfqpoint{3.281587in}{1.322395in}}%
\pgfpathlineto{\pgfqpoint{3.295641in}{1.316809in}}%
\pgfpathlineto{\pgfqpoint{3.309700in}{1.311300in}}%
\pgfpathlineto{\pgfqpoint{3.323763in}{1.305868in}}%
\pgfpathlineto{\pgfqpoint{3.332306in}{1.311484in}}%
\pgfpathlineto{\pgfqpoint{3.340839in}{1.317304in}}%
\pgfpathlineto{\pgfqpoint{3.349362in}{1.323321in}}%
\pgfpathlineto{\pgfqpoint{3.357875in}{1.329530in}}%
\pgfpathlineto{\pgfqpoint{3.343834in}{1.334549in}}%
\pgfpathlineto{\pgfqpoint{3.329799in}{1.339645in}}%
\pgfpathlineto{\pgfqpoint{3.315768in}{1.344818in}}%
\pgfpathlineto{\pgfqpoint{3.301743in}{1.350068in}}%
\pgfpathlineto{\pgfqpoint{3.293207in}{1.344265in}}%
\pgfpathlineto{\pgfqpoint{3.284662in}{1.338658in}}%
\pgfpathlineto{\pgfqpoint{3.276105in}{1.333253in}}%
\pgfpathlineto{\pgfqpoint{3.267538in}{1.328058in}}%
\pgfpathclose%
\pgfusepath{fill}%
\end{pgfscope}%
\begin{pgfscope}%
\pgfpathrectangle{\pgfqpoint{1.150000in}{0.150000in}}{\pgfqpoint{5.700000in}{5.700000in}}%
\pgfusepath{clip}%
\pgfsetbuttcap%
\pgfsetroundjoin%
\definecolor{currentfill}{rgb}{0.241237,0.296485,0.539709}%
\pgfsetfillcolor{currentfill}%
\pgfsetfillopacity{0.700000}%
\pgfsetlinewidth{0.000000pt}%
\definecolor{currentstroke}{rgb}{0.000000,0.000000,0.000000}%
\pgfsetstrokecolor{currentstroke}%
\pgfsetdash{}{0pt}%
\pgfpathmoveto{\pgfqpoint{4.726252in}{1.891806in}}%
\pgfpathlineto{\pgfqpoint{4.740711in}{1.895137in}}%
\pgfpathlineto{\pgfqpoint{4.755182in}{1.898539in}}%
\pgfpathlineto{\pgfqpoint{4.769664in}{1.902012in}}%
\pgfpathlineto{\pgfqpoint{4.784159in}{1.905557in}}%
\pgfpathlineto{\pgfqpoint{4.792142in}{1.917018in}}%
\pgfpathlineto{\pgfqpoint{4.800117in}{1.928366in}}%
\pgfpathlineto{\pgfqpoint{4.808087in}{1.939599in}}%
\pgfpathlineto{\pgfqpoint{4.816049in}{1.950716in}}%
\pgfpathlineto{\pgfqpoint{4.801560in}{1.947068in}}%
\pgfpathlineto{\pgfqpoint{4.787083in}{1.943490in}}%
\pgfpathlineto{\pgfqpoint{4.772618in}{1.939984in}}%
\pgfpathlineto{\pgfqpoint{4.758165in}{1.936550in}}%
\pgfpathlineto{\pgfqpoint{4.750197in}{1.925529in}}%
\pgfpathlineto{\pgfqpoint{4.742222in}{1.914397in}}%
\pgfpathlineto{\pgfqpoint{4.734240in}{1.903156in}}%
\pgfpathlineto{\pgfqpoint{4.726252in}{1.891806in}}%
\pgfpathclose%
\pgfusepath{fill}%
\end{pgfscope}%
\begin{pgfscope}%
\pgfpathrectangle{\pgfqpoint{1.150000in}{0.150000in}}{\pgfqpoint{5.700000in}{5.700000in}}%
\pgfusepath{clip}%
\pgfsetbuttcap%
\pgfsetroundjoin%
\definecolor{currentfill}{rgb}{0.267004,0.004874,0.329415}%
\pgfsetfillcolor{currentfill}%
\pgfsetfillopacity{0.700000}%
\pgfsetlinewidth{0.000000pt}%
\definecolor{currentstroke}{rgb}{0.000000,0.000000,0.000000}%
\pgfsetstrokecolor{currentstroke}%
\pgfsetdash{}{0pt}%
\pgfpathmoveto{\pgfqpoint{3.414089in}{1.310216in}}%
\pgfpathlineto{\pgfqpoint{3.428155in}{1.305577in}}%
\pgfpathlineto{\pgfqpoint{3.442228in}{1.301014in}}%
\pgfpathlineto{\pgfqpoint{3.456305in}{1.296526in}}%
\pgfpathlineto{\pgfqpoint{3.470389in}{1.292112in}}%
\pgfpathlineto{\pgfqpoint{3.478850in}{1.299306in}}%
\pgfpathlineto{\pgfqpoint{3.487302in}{1.306668in}}%
\pgfpathlineto{\pgfqpoint{3.495745in}{1.314192in}}%
\pgfpathlineto{\pgfqpoint{3.504180in}{1.321873in}}%
\pgfpathlineto{\pgfqpoint{3.490116in}{1.325894in}}%
\pgfpathlineto{\pgfqpoint{3.476058in}{1.329990in}}%
\pgfpathlineto{\pgfqpoint{3.462005in}{1.334161in}}%
\pgfpathlineto{\pgfqpoint{3.447959in}{1.338408in}}%
\pgfpathlineto{\pgfqpoint{3.439505in}{1.331112in}}%
\pgfpathlineto{\pgfqpoint{3.431042in}{1.323977in}}%
\pgfpathlineto{\pgfqpoint{3.422570in}{1.317010in}}%
\pgfpathlineto{\pgfqpoint{3.414089in}{1.310216in}}%
\pgfpathclose%
\pgfusepath{fill}%
\end{pgfscope}%
\begin{pgfscope}%
\pgfpathrectangle{\pgfqpoint{1.150000in}{0.150000in}}{\pgfqpoint{5.700000in}{5.700000in}}%
\pgfusepath{clip}%
\pgfsetbuttcap%
\pgfsetroundjoin%
\definecolor{currentfill}{rgb}{0.229739,0.322361,0.545706}%
\pgfsetfillcolor{currentfill}%
\pgfsetfillopacity{0.700000}%
\pgfsetlinewidth{0.000000pt}%
\definecolor{currentstroke}{rgb}{0.000000,0.000000,0.000000}%
\pgfsetstrokecolor{currentstroke}%
\pgfsetdash{}{0pt}%
\pgfpathmoveto{\pgfqpoint{4.816049in}{1.950716in}}%
\pgfpathlineto{\pgfqpoint{4.830550in}{1.954436in}}%
\pgfpathlineto{\pgfqpoint{4.845064in}{1.958227in}}%
\pgfpathlineto{\pgfqpoint{4.859589in}{1.962089in}}%
\pgfpathlineto{\pgfqpoint{4.874127in}{1.966023in}}%
\pgfpathlineto{\pgfqpoint{4.882077in}{1.977113in}}%
\pgfpathlineto{\pgfqpoint{4.890020in}{1.988080in}}%
\pgfpathlineto{\pgfqpoint{4.897955in}{1.998923in}}%
\pgfpathlineto{\pgfqpoint{4.905883in}{2.009641in}}%
\pgfpathlineto{\pgfqpoint{4.891352in}{2.005624in}}%
\pgfpathlineto{\pgfqpoint{4.876832in}{2.001680in}}%
\pgfpathlineto{\pgfqpoint{4.862325in}{1.997806in}}%
\pgfpathlineto{\pgfqpoint{4.847830in}{1.994004in}}%
\pgfpathlineto{\pgfqpoint{4.839895in}{1.983361in}}%
\pgfpathlineto{\pgfqpoint{4.831953in}{1.972598in}}%
\pgfpathlineto{\pgfqpoint{4.824005in}{1.961716in}}%
\pgfpathlineto{\pgfqpoint{4.816049in}{1.950716in}}%
\pgfpathclose%
\pgfusepath{fill}%
\end{pgfscope}%
\begin{pgfscope}%
\pgfpathrectangle{\pgfqpoint{1.150000in}{0.150000in}}{\pgfqpoint{5.700000in}{5.700000in}}%
\pgfusepath{clip}%
\pgfsetbuttcap%
\pgfsetroundjoin%
\definecolor{currentfill}{rgb}{0.282656,0.100196,0.422160}%
\pgfsetfillcolor{currentfill}%
\pgfsetfillopacity{0.700000}%
\pgfsetlinewidth{0.000000pt}%
\definecolor{currentstroke}{rgb}{0.000000,0.000000,0.000000}%
\pgfsetstrokecolor{currentstroke}%
\pgfsetdash{}{0pt}%
\pgfpathmoveto{\pgfqpoint{4.065932in}{1.463529in}}%
\pgfpathlineto{\pgfqpoint{4.080147in}{1.463276in}}%
\pgfpathlineto{\pgfqpoint{4.094371in}{1.463095in}}%
\pgfpathlineto{\pgfqpoint{4.108604in}{1.462986in}}%
\pgfpathlineto{\pgfqpoint{4.122845in}{1.462947in}}%
\pgfpathlineto{\pgfqpoint{4.131045in}{1.474920in}}%
\pgfpathlineto{\pgfqpoint{4.139239in}{1.486899in}}%
\pgfpathlineto{\pgfqpoint{4.147428in}{1.498882in}}%
\pgfpathlineto{\pgfqpoint{4.155612in}{1.510865in}}%
\pgfpathlineto{\pgfqpoint{4.141378in}{1.510632in}}%
\pgfpathlineto{\pgfqpoint{4.127153in}{1.510471in}}%
\pgfpathlineto{\pgfqpoint{4.112937in}{1.510381in}}%
\pgfpathlineto{\pgfqpoint{4.098730in}{1.510362in}}%
\pgfpathlineto{\pgfqpoint{4.090539in}{1.498643in}}%
\pgfpathlineto{\pgfqpoint{4.082342in}{1.486928in}}%
\pgfpathlineto{\pgfqpoint{4.074140in}{1.475222in}}%
\pgfpathlineto{\pgfqpoint{4.065932in}{1.463529in}}%
\pgfpathclose%
\pgfusepath{fill}%
\end{pgfscope}%
\begin{pgfscope}%
\pgfpathrectangle{\pgfqpoint{1.150000in}{0.150000in}}{\pgfqpoint{5.700000in}{5.700000in}}%
\pgfusepath{clip}%
\pgfsetbuttcap%
\pgfsetroundjoin%
\definecolor{currentfill}{rgb}{0.283187,0.125848,0.444960}%
\pgfsetfillcolor{currentfill}%
\pgfsetfillopacity{0.700000}%
\pgfsetlinewidth{0.000000pt}%
\definecolor{currentstroke}{rgb}{0.000000,0.000000,0.000000}%
\pgfsetstrokecolor{currentstroke}%
\pgfsetdash{}{0pt}%
\pgfpathmoveto{\pgfqpoint{4.155612in}{1.510865in}}%
\pgfpathlineto{\pgfqpoint{4.169856in}{1.511169in}}%
\pgfpathlineto{\pgfqpoint{4.184109in}{1.511544in}}%
\pgfpathlineto{\pgfqpoint{4.198371in}{1.511990in}}%
\pgfpathlineto{\pgfqpoint{4.212643in}{1.512508in}}%
\pgfpathlineto{\pgfqpoint{4.220815in}{1.524744in}}%
\pgfpathlineto{\pgfqpoint{4.228982in}{1.536968in}}%
\pgfpathlineto{\pgfqpoint{4.237143in}{1.549174in}}%
\pgfpathlineto{\pgfqpoint{4.245300in}{1.561361in}}%
\pgfpathlineto{\pgfqpoint{4.231034in}{1.560593in}}%
\pgfpathlineto{\pgfqpoint{4.216779in}{1.559896in}}%
\pgfpathlineto{\pgfqpoint{4.202533in}{1.559270in}}%
\pgfpathlineto{\pgfqpoint{4.188296in}{1.558716in}}%
\pgfpathlineto{\pgfqpoint{4.180133in}{1.546772in}}%
\pgfpathlineto{\pgfqpoint{4.171964in}{1.534813in}}%
\pgfpathlineto{\pgfqpoint{4.163791in}{1.522843in}}%
\pgfpathlineto{\pgfqpoint{4.155612in}{1.510865in}}%
\pgfpathclose%
\pgfusepath{fill}%
\end{pgfscope}%
\begin{pgfscope}%
\pgfpathrectangle{\pgfqpoint{1.150000in}{0.150000in}}{\pgfqpoint{5.700000in}{5.700000in}}%
\pgfusepath{clip}%
\pgfsetbuttcap%
\pgfsetroundjoin%
\definecolor{currentfill}{rgb}{0.280894,0.078907,0.402329}%
\pgfsetfillcolor{currentfill}%
\pgfsetfillopacity{0.700000}%
\pgfsetlinewidth{0.000000pt}%
\definecolor{currentstroke}{rgb}{0.000000,0.000000,0.000000}%
\pgfsetstrokecolor{currentstroke}%
\pgfsetdash{}{0pt}%
\pgfpathmoveto{\pgfqpoint{3.976243in}{1.419849in}}%
\pgfpathlineto{\pgfqpoint{3.990432in}{1.419019in}}%
\pgfpathlineto{\pgfqpoint{4.004629in}{1.418261in}}%
\pgfpathlineto{\pgfqpoint{4.018835in}{1.417575in}}%
\pgfpathlineto{\pgfqpoint{4.033050in}{1.416960in}}%
\pgfpathlineto{\pgfqpoint{4.041279in}{1.428563in}}%
\pgfpathlineto{\pgfqpoint{4.049502in}{1.440195in}}%
\pgfpathlineto{\pgfqpoint{4.057720in}{1.451851in}}%
\pgfpathlineto{\pgfqpoint{4.065932in}{1.463529in}}%
\pgfpathlineto{\pgfqpoint{4.051727in}{1.463852in}}%
\pgfpathlineto{\pgfqpoint{4.037529in}{1.464248in}}%
\pgfpathlineto{\pgfqpoint{4.023341in}{1.464715in}}%
\pgfpathlineto{\pgfqpoint{4.009161in}{1.465254in}}%
\pgfpathlineto{\pgfqpoint{4.000940in}{1.453860in}}%
\pgfpathlineto{\pgfqpoint{3.992713in}{1.442492in}}%
\pgfpathlineto{\pgfqpoint{3.984481in}{1.431153in}}%
\pgfpathlineto{\pgfqpoint{3.976243in}{1.419849in}}%
\pgfpathclose%
\pgfusepath{fill}%
\end{pgfscope}%
\begin{pgfscope}%
\pgfpathrectangle{\pgfqpoint{1.150000in}{0.150000in}}{\pgfqpoint{5.700000in}{5.700000in}}%
\pgfusepath{clip}%
\pgfsetbuttcap%
\pgfsetroundjoin%
\definecolor{currentfill}{rgb}{0.268510,0.009605,0.335427}%
\pgfsetfillcolor{currentfill}%
\pgfsetfillopacity{0.700000}%
\pgfsetlinewidth{0.000000pt}%
\definecolor{currentstroke}{rgb}{0.000000,0.000000,0.000000}%
\pgfsetstrokecolor{currentstroke}%
\pgfsetdash{}{0pt}%
\pgfpathmoveto{\pgfqpoint{3.560496in}{1.306534in}}%
\pgfpathlineto{\pgfqpoint{3.574590in}{1.302886in}}%
\pgfpathlineto{\pgfqpoint{3.588691in}{1.299311in}}%
\pgfpathlineto{\pgfqpoint{3.602798in}{1.295809in}}%
\pgfpathlineto{\pgfqpoint{3.616911in}{1.292382in}}%
\pgfpathlineto{\pgfqpoint{3.625303in}{1.300971in}}%
\pgfpathlineto{\pgfqpoint{3.633687in}{1.309696in}}%
\pgfpathlineto{\pgfqpoint{3.642063in}{1.318549in}}%
\pgfpathlineto{\pgfqpoint{3.650432in}{1.327525in}}%
\pgfpathlineto{\pgfqpoint{3.636334in}{1.330581in}}%
\pgfpathlineto{\pgfqpoint{3.622244in}{1.333711in}}%
\pgfpathlineto{\pgfqpoint{3.608160in}{1.336914in}}%
\pgfpathlineto{\pgfqpoint{3.594082in}{1.340191in}}%
\pgfpathlineto{\pgfqpoint{3.585697in}{1.331579in}}%
\pgfpathlineto{\pgfqpoint{3.577305in}{1.323095in}}%
\pgfpathlineto{\pgfqpoint{3.568904in}{1.314745in}}%
\pgfpathlineto{\pgfqpoint{3.560496in}{1.306534in}}%
\pgfpathclose%
\pgfusepath{fill}%
\end{pgfscope}%
\begin{pgfscope}%
\pgfpathrectangle{\pgfqpoint{1.150000in}{0.150000in}}{\pgfqpoint{5.700000in}{5.700000in}}%
\pgfusepath{clip}%
\pgfsetbuttcap%
\pgfsetroundjoin%
\definecolor{currentfill}{rgb}{0.281887,0.150881,0.465405}%
\pgfsetfillcolor{currentfill}%
\pgfsetfillopacity{0.700000}%
\pgfsetlinewidth{0.000000pt}%
\definecolor{currentstroke}{rgb}{0.000000,0.000000,0.000000}%
\pgfsetstrokecolor{currentstroke}%
\pgfsetdash{}{0pt}%
\pgfpathmoveto{\pgfqpoint{4.245300in}{1.561361in}}%
\pgfpathlineto{\pgfqpoint{4.259575in}{1.562201in}}%
\pgfpathlineto{\pgfqpoint{4.273860in}{1.563111in}}%
\pgfpathlineto{\pgfqpoint{4.288154in}{1.564093in}}%
\pgfpathlineto{\pgfqpoint{4.302459in}{1.565146in}}%
\pgfpathlineto{\pgfqpoint{4.310604in}{1.577547in}}%
\pgfpathlineto{\pgfqpoint{4.318744in}{1.589915in}}%
\pgfpathlineto{\pgfqpoint{4.326879in}{1.602249in}}%
\pgfpathlineto{\pgfqpoint{4.335009in}{1.614543in}}%
\pgfpathlineto{\pgfqpoint{4.320710in}{1.613260in}}%
\pgfpathlineto{\pgfqpoint{4.306421in}{1.612048in}}%
\pgfpathlineto{\pgfqpoint{4.292143in}{1.610908in}}%
\pgfpathlineto{\pgfqpoint{4.277874in}{1.609838in}}%
\pgfpathlineto{\pgfqpoint{4.269738in}{1.597766in}}%
\pgfpathlineto{\pgfqpoint{4.261597in}{1.585660in}}%
\pgfpathlineto{\pgfqpoint{4.253451in}{1.573524in}}%
\pgfpathlineto{\pgfqpoint{4.245300in}{1.561361in}}%
\pgfpathclose%
\pgfusepath{fill}%
\end{pgfscope}%
\begin{pgfscope}%
\pgfpathrectangle{\pgfqpoint{1.150000in}{0.150000in}}{\pgfqpoint{5.700000in}{5.700000in}}%
\pgfusepath{clip}%
\pgfsetbuttcap%
\pgfsetroundjoin%
\definecolor{currentfill}{rgb}{0.218130,0.347432,0.550038}%
\pgfsetfillcolor{currentfill}%
\pgfsetfillopacity{0.700000}%
\pgfsetlinewidth{0.000000pt}%
\definecolor{currentstroke}{rgb}{0.000000,0.000000,0.000000}%
\pgfsetstrokecolor{currentstroke}%
\pgfsetdash{}{0pt}%
\pgfpathmoveto{\pgfqpoint{4.905883in}{2.009641in}}%
\pgfpathlineto{\pgfqpoint{4.920428in}{2.013728in}}%
\pgfpathlineto{\pgfqpoint{4.934985in}{2.017887in}}%
\pgfpathlineto{\pgfqpoint{4.949554in}{2.022117in}}%
\pgfpathlineto{\pgfqpoint{4.964136in}{2.026418in}}%
\pgfpathlineto{\pgfqpoint{4.972051in}{2.037079in}}%
\pgfpathlineto{\pgfqpoint{4.979958in}{2.047607in}}%
\pgfpathlineto{\pgfqpoint{4.987858in}{2.058003in}}%
\pgfpathlineto{\pgfqpoint{4.995749in}{2.068266in}}%
\pgfpathlineto{\pgfqpoint{4.981174in}{2.063904in}}%
\pgfpathlineto{\pgfqpoint{4.966611in}{2.059613in}}%
\pgfpathlineto{\pgfqpoint{4.952061in}{2.055394in}}%
\pgfpathlineto{\pgfqpoint{4.937523in}{2.051246in}}%
\pgfpathlineto{\pgfqpoint{4.929624in}{2.041035in}}%
\pgfpathlineto{\pgfqpoint{4.921718in}{2.030697in}}%
\pgfpathlineto{\pgfqpoint{4.913804in}{2.020232in}}%
\pgfpathlineto{\pgfqpoint{4.905883in}{2.009641in}}%
\pgfpathclose%
\pgfusepath{fill}%
\end{pgfscope}%
\begin{pgfscope}%
\pgfpathrectangle{\pgfqpoint{1.150000in}{0.150000in}}{\pgfqpoint{5.700000in}{5.700000in}}%
\pgfusepath{clip}%
\pgfsetbuttcap%
\pgfsetroundjoin%
\definecolor{currentfill}{rgb}{0.277941,0.056324,0.381191}%
\pgfsetfillcolor{currentfill}%
\pgfsetfillopacity{0.700000}%
\pgfsetlinewidth{0.000000pt}%
\definecolor{currentstroke}{rgb}{0.000000,0.000000,0.000000}%
\pgfsetstrokecolor{currentstroke}%
\pgfsetdash{}{0pt}%
\pgfpathmoveto{\pgfqpoint{3.886523in}{1.380343in}}%
\pgfpathlineto{\pgfqpoint{3.900689in}{1.378915in}}%
\pgfpathlineto{\pgfqpoint{3.914864in}{1.377558in}}%
\pgfpathlineto{\pgfqpoint{3.929046in}{1.376274in}}%
\pgfpathlineto{\pgfqpoint{3.943236in}{1.375061in}}%
\pgfpathlineto{\pgfqpoint{3.951497in}{1.386185in}}%
\pgfpathlineto{\pgfqpoint{3.959751in}{1.397360in}}%
\pgfpathlineto{\pgfqpoint{3.968000in}{1.408583in}}%
\pgfpathlineto{\pgfqpoint{3.976243in}{1.419849in}}%
\pgfpathlineto{\pgfqpoint{3.962063in}{1.420750in}}%
\pgfpathlineto{\pgfqpoint{3.947890in}{1.421724in}}%
\pgfpathlineto{\pgfqpoint{3.933726in}{1.422769in}}%
\pgfpathlineto{\pgfqpoint{3.919571in}{1.423886in}}%
\pgfpathlineto{\pgfqpoint{3.911317in}{1.412924in}}%
\pgfpathlineto{\pgfqpoint{3.903058in}{1.402009in}}%
\pgfpathlineto{\pgfqpoint{3.894794in}{1.391148in}}%
\pgfpathlineto{\pgfqpoint{3.886523in}{1.380343in}}%
\pgfpathclose%
\pgfusepath{fill}%
\end{pgfscope}%
\begin{pgfscope}%
\pgfpathrectangle{\pgfqpoint{1.150000in}{0.150000in}}{\pgfqpoint{5.700000in}{5.700000in}}%
\pgfusepath{clip}%
\pgfsetbuttcap%
\pgfsetroundjoin%
\definecolor{currentfill}{rgb}{0.278826,0.175490,0.483397}%
\pgfsetfillcolor{currentfill}%
\pgfsetfillopacity{0.700000}%
\pgfsetlinewidth{0.000000pt}%
\definecolor{currentstroke}{rgb}{0.000000,0.000000,0.000000}%
\pgfsetstrokecolor{currentstroke}%
\pgfsetdash{}{0pt}%
\pgfpathmoveto{\pgfqpoint{4.335009in}{1.614543in}}%
\pgfpathlineto{\pgfqpoint{4.349318in}{1.615898in}}%
\pgfpathlineto{\pgfqpoint{4.363637in}{1.617323in}}%
\pgfpathlineto{\pgfqpoint{4.377966in}{1.618819in}}%
\pgfpathlineto{\pgfqpoint{4.392306in}{1.620387in}}%
\pgfpathlineto{\pgfqpoint{4.400425in}{1.632857in}}%
\pgfpathlineto{\pgfqpoint{4.408539in}{1.645278in}}%
\pgfpathlineto{\pgfqpoint{4.416648in}{1.657645in}}%
\pgfpathlineto{\pgfqpoint{4.424751in}{1.669958in}}%
\pgfpathlineto{\pgfqpoint{4.410416in}{1.668180in}}%
\pgfpathlineto{\pgfqpoint{4.396092in}{1.666474in}}%
\pgfpathlineto{\pgfqpoint{4.381779in}{1.664840in}}%
\pgfpathlineto{\pgfqpoint{4.367475in}{1.663276in}}%
\pgfpathlineto{\pgfqpoint{4.359367in}{1.651166in}}%
\pgfpathlineto{\pgfqpoint{4.351253in}{1.639005in}}%
\pgfpathlineto{\pgfqpoint{4.343134in}{1.626796in}}%
\pgfpathlineto{\pgfqpoint{4.335009in}{1.614543in}}%
\pgfpathclose%
\pgfusepath{fill}%
\end{pgfscope}%
\begin{pgfscope}%
\pgfpathrectangle{\pgfqpoint{1.150000in}{0.150000in}}{\pgfqpoint{5.700000in}{5.700000in}}%
\pgfusepath{clip}%
\pgfsetbuttcap%
\pgfsetroundjoin%
\definecolor{currentfill}{rgb}{0.208623,0.367752,0.552675}%
\pgfsetfillcolor{currentfill}%
\pgfsetfillopacity{0.700000}%
\pgfsetlinewidth{0.000000pt}%
\definecolor{currentstroke}{rgb}{0.000000,0.000000,0.000000}%
\pgfsetstrokecolor{currentstroke}%
\pgfsetdash{}{0pt}%
\pgfpathmoveto{\pgfqpoint{4.995749in}{2.068266in}}%
\pgfpathlineto{\pgfqpoint{5.010338in}{2.072700in}}%
\pgfpathlineto{\pgfqpoint{5.024939in}{2.077205in}}%
\pgfpathlineto{\pgfqpoint{5.039553in}{2.081781in}}%
\pgfpathlineto{\pgfqpoint{5.054180in}{2.086429in}}%
\pgfpathlineto{\pgfqpoint{5.062057in}{2.096606in}}%
\pgfpathlineto{\pgfqpoint{5.069926in}{2.106643in}}%
\pgfpathlineto{\pgfqpoint{5.077787in}{2.116542in}}%
\pgfpathlineto{\pgfqpoint{5.085640in}{2.126301in}}%
\pgfpathlineto{\pgfqpoint{5.071020in}{2.121614in}}%
\pgfpathlineto{\pgfqpoint{5.056414in}{2.116999in}}%
\pgfpathlineto{\pgfqpoint{5.041820in}{2.112456in}}%
\pgfpathlineto{\pgfqpoint{5.027239in}{2.107983in}}%
\pgfpathlineto{\pgfqpoint{5.019378in}{2.098255in}}%
\pgfpathlineto{\pgfqpoint{5.011510in}{2.088392in}}%
\pgfpathlineto{\pgfqpoint{5.003634in}{2.078396in}}%
\pgfpathlineto{\pgfqpoint{4.995749in}{2.068266in}}%
\pgfpathclose%
\pgfusepath{fill}%
\end{pgfscope}%
\begin{pgfscope}%
\pgfpathrectangle{\pgfqpoint{1.150000in}{0.150000in}}{\pgfqpoint{5.700000in}{5.700000in}}%
\pgfusepath{clip}%
\pgfsetbuttcap%
\pgfsetroundjoin%
\definecolor{currentfill}{rgb}{0.273006,0.204520,0.501721}%
\pgfsetfillcolor{currentfill}%
\pgfsetfillopacity{0.700000}%
\pgfsetlinewidth{0.000000pt}%
\definecolor{currentstroke}{rgb}{0.000000,0.000000,0.000000}%
\pgfsetstrokecolor{currentstroke}%
\pgfsetdash{}{0pt}%
\pgfpathmoveto{\pgfqpoint{4.424751in}{1.669958in}}%
\pgfpathlineto{\pgfqpoint{4.439096in}{1.671806in}}%
\pgfpathlineto{\pgfqpoint{4.453451in}{1.673725in}}%
\pgfpathlineto{\pgfqpoint{4.467818in}{1.675715in}}%
\pgfpathlineto{\pgfqpoint{4.482195in}{1.677776in}}%
\pgfpathlineto{\pgfqpoint{4.490288in}{1.690227in}}%
\pgfpathlineto{\pgfqpoint{4.498375in}{1.702611in}}%
\pgfpathlineto{\pgfqpoint{4.506457in}{1.714926in}}%
\pgfpathlineto{\pgfqpoint{4.514533in}{1.727170in}}%
\pgfpathlineto{\pgfqpoint{4.500161in}{1.724919in}}%
\pgfpathlineto{\pgfqpoint{4.485800in}{1.722740in}}%
\pgfpathlineto{\pgfqpoint{4.471449in}{1.720632in}}%
\pgfpathlineto{\pgfqpoint{4.457109in}{1.718595in}}%
\pgfpathlineto{\pgfqpoint{4.449028in}{1.706533in}}%
\pgfpathlineto{\pgfqpoint{4.440941in}{1.694404in}}%
\pgfpathlineto{\pgfqpoint{4.432849in}{1.682211in}}%
\pgfpathlineto{\pgfqpoint{4.424751in}{1.669958in}}%
\pgfpathclose%
\pgfusepath{fill}%
\end{pgfscope}%
\begin{pgfscope}%
\pgfpathrectangle{\pgfqpoint{1.150000in}{0.150000in}}{\pgfqpoint{5.700000in}{5.700000in}}%
\pgfusepath{clip}%
\pgfsetbuttcap%
\pgfsetroundjoin%
\definecolor{currentfill}{rgb}{0.274952,0.037752,0.364543}%
\pgfsetfillcolor{currentfill}%
\pgfsetfillopacity{0.700000}%
\pgfsetlinewidth{0.000000pt}%
\definecolor{currentstroke}{rgb}{0.000000,0.000000,0.000000}%
\pgfsetstrokecolor{currentstroke}%
\pgfsetdash{}{0pt}%
\pgfpathmoveto{\pgfqpoint{3.796748in}{1.345550in}}%
\pgfpathlineto{\pgfqpoint{3.810895in}{1.343502in}}%
\pgfpathlineto{\pgfqpoint{3.825049in}{1.341526in}}%
\pgfpathlineto{\pgfqpoint{3.839211in}{1.339622in}}%
\pgfpathlineto{\pgfqpoint{3.853381in}{1.337790in}}%
\pgfpathlineto{\pgfqpoint{3.861676in}{1.348319in}}%
\pgfpathlineto{\pgfqpoint{3.869964in}{1.358924in}}%
\pgfpathlineto{\pgfqpoint{3.878247in}{1.369600in}}%
\pgfpathlineto{\pgfqpoint{3.886523in}{1.380343in}}%
\pgfpathlineto{\pgfqpoint{3.872365in}{1.381843in}}%
\pgfpathlineto{\pgfqpoint{3.858214in}{1.383416in}}%
\pgfpathlineto{\pgfqpoint{3.844072in}{1.385060in}}%
\pgfpathlineto{\pgfqpoint{3.829937in}{1.386778in}}%
\pgfpathlineto{\pgfqpoint{3.821649in}{1.376358in}}%
\pgfpathlineto{\pgfqpoint{3.813355in}{1.366011in}}%
\pgfpathlineto{\pgfqpoint{3.805054in}{1.355740in}}%
\pgfpathlineto{\pgfqpoint{3.796748in}{1.345550in}}%
\pgfpathclose%
\pgfusepath{fill}%
\end{pgfscope}%
\begin{pgfscope}%
\pgfpathrectangle{\pgfqpoint{1.150000in}{0.150000in}}{\pgfqpoint{5.700000in}{5.700000in}}%
\pgfusepath{clip}%
\pgfsetbuttcap%
\pgfsetroundjoin%
\definecolor{currentfill}{rgb}{0.197636,0.391528,0.554969}%
\pgfsetfillcolor{currentfill}%
\pgfsetfillopacity{0.700000}%
\pgfsetlinewidth{0.000000pt}%
\definecolor{currentstroke}{rgb}{0.000000,0.000000,0.000000}%
\pgfsetstrokecolor{currentstroke}%
\pgfsetdash{}{0pt}%
\pgfpathmoveto{\pgfqpoint{5.085640in}{2.126301in}}%
\pgfpathlineto{\pgfqpoint{5.100273in}{2.131059in}}%
\pgfpathlineto{\pgfqpoint{5.114919in}{2.135888in}}%
\pgfpathlineto{\pgfqpoint{5.129578in}{2.140789in}}%
\pgfpathlineto{\pgfqpoint{5.144250in}{2.145762in}}%
\pgfpathlineto{\pgfqpoint{5.152087in}{2.155407in}}%
\pgfpathlineto{\pgfqpoint{5.159915in}{2.164906in}}%
\pgfpathlineto{\pgfqpoint{5.167735in}{2.174262in}}%
\pgfpathlineto{\pgfqpoint{5.175546in}{2.183473in}}%
\pgfpathlineto{\pgfqpoint{5.160882in}{2.178484in}}%
\pgfpathlineto{\pgfqpoint{5.146231in}{2.173566in}}%
\pgfpathlineto{\pgfqpoint{5.131594in}{2.168720in}}%
\pgfpathlineto{\pgfqpoint{5.116969in}{2.163945in}}%
\pgfpathlineto{\pgfqpoint{5.109149in}{2.154743in}}%
\pgfpathlineto{\pgfqpoint{5.101321in}{2.145401in}}%
\pgfpathlineto{\pgfqpoint{5.093485in}{2.135921in}}%
\pgfpathlineto{\pgfqpoint{5.085640in}{2.126301in}}%
\pgfpathclose%
\pgfusepath{fill}%
\end{pgfscope}%
\begin{pgfscope}%
\pgfpathrectangle{\pgfqpoint{1.150000in}{0.150000in}}{\pgfqpoint{5.700000in}{5.700000in}}%
\pgfusepath{clip}%
\pgfsetbuttcap%
\pgfsetroundjoin%
\definecolor{currentfill}{rgb}{0.268510,0.009605,0.335427}%
\pgfsetfillcolor{currentfill}%
\pgfsetfillopacity{0.700000}%
\pgfsetlinewidth{0.000000pt}%
\definecolor{currentstroke}{rgb}{0.000000,0.000000,0.000000}%
\pgfsetstrokecolor{currentstroke}%
\pgfsetdash{}{0pt}%
\pgfpathmoveto{\pgfqpoint{3.323763in}{1.305868in}}%
\pgfpathlineto{\pgfqpoint{3.337831in}{1.300512in}}%
\pgfpathlineto{\pgfqpoint{3.351904in}{1.295233in}}%
\pgfpathlineto{\pgfqpoint{3.365983in}{1.290029in}}%
\pgfpathlineto{\pgfqpoint{3.380066in}{1.284902in}}%
\pgfpathlineto{\pgfqpoint{3.388587in}{1.290939in}}%
\pgfpathlineto{\pgfqpoint{3.397097in}{1.297175in}}%
\pgfpathlineto{\pgfqpoint{3.405598in}{1.303603in}}%
\pgfpathlineto{\pgfqpoint{3.414089in}{1.310216in}}%
\pgfpathlineto{\pgfqpoint{3.400027in}{1.314931in}}%
\pgfpathlineto{\pgfqpoint{3.385971in}{1.319721in}}%
\pgfpathlineto{\pgfqpoint{3.371920in}{1.324587in}}%
\pgfpathlineto{\pgfqpoint{3.357875in}{1.329530in}}%
\pgfpathlineto{\pgfqpoint{3.349362in}{1.323321in}}%
\pgfpathlineto{\pgfqpoint{3.340839in}{1.317304in}}%
\pgfpathlineto{\pgfqpoint{3.332306in}{1.311484in}}%
\pgfpathlineto{\pgfqpoint{3.323763in}{1.305868in}}%
\pgfpathclose%
\pgfusepath{fill}%
\end{pgfscope}%
\begin{pgfscope}%
\pgfpathrectangle{\pgfqpoint{1.150000in}{0.150000in}}{\pgfqpoint{5.700000in}{5.700000in}}%
\pgfusepath{clip}%
\pgfsetbuttcap%
\pgfsetroundjoin%
\definecolor{currentfill}{rgb}{0.265145,0.232956,0.516599}%
\pgfsetfillcolor{currentfill}%
\pgfsetfillopacity{0.700000}%
\pgfsetlinewidth{0.000000pt}%
\definecolor{currentstroke}{rgb}{0.000000,0.000000,0.000000}%
\pgfsetstrokecolor{currentstroke}%
\pgfsetdash{}{0pt}%
\pgfpathmoveto{\pgfqpoint{4.514533in}{1.727170in}}%
\pgfpathlineto{\pgfqpoint{4.528916in}{1.729491in}}%
\pgfpathlineto{\pgfqpoint{4.543310in}{1.731883in}}%
\pgfpathlineto{\pgfqpoint{4.557715in}{1.734347in}}%
\pgfpathlineto{\pgfqpoint{4.572131in}{1.736881in}}%
\pgfpathlineto{\pgfqpoint{4.580197in}{1.749227in}}%
\pgfpathlineto{\pgfqpoint{4.588258in}{1.761491in}}%
\pgfpathlineto{\pgfqpoint{4.596312in}{1.773672in}}%
\pgfpathlineto{\pgfqpoint{4.604361in}{1.785766in}}%
\pgfpathlineto{\pgfqpoint{4.589949in}{1.783063in}}%
\pgfpathlineto{\pgfqpoint{4.575549in}{1.780432in}}%
\pgfpathlineto{\pgfqpoint{4.561160in}{1.777872in}}%
\pgfpathlineto{\pgfqpoint{4.546782in}{1.775382in}}%
\pgfpathlineto{\pgfqpoint{4.538728in}{1.763448in}}%
\pgfpathlineto{\pgfqpoint{4.530669in}{1.751433in}}%
\pgfpathlineto{\pgfqpoint{4.522604in}{1.739339in}}%
\pgfpathlineto{\pgfqpoint{4.514533in}{1.727170in}}%
\pgfpathclose%
\pgfusepath{fill}%
\end{pgfscope}%
\begin{pgfscope}%
\pgfpathrectangle{\pgfqpoint{1.150000in}{0.150000in}}{\pgfqpoint{5.700000in}{5.700000in}}%
\pgfusepath{clip}%
\pgfsetbuttcap%
\pgfsetroundjoin%
\definecolor{currentfill}{rgb}{0.187231,0.414746,0.556547}%
\pgfsetfillcolor{currentfill}%
\pgfsetfillopacity{0.700000}%
\pgfsetlinewidth{0.000000pt}%
\definecolor{currentstroke}{rgb}{0.000000,0.000000,0.000000}%
\pgfsetstrokecolor{currentstroke}%
\pgfsetdash{}{0pt}%
\pgfpathmoveto{\pgfqpoint{5.175546in}{2.183473in}}%
\pgfpathlineto{\pgfqpoint{5.190224in}{2.188533in}}%
\pgfpathlineto{\pgfqpoint{5.204915in}{2.193665in}}%
\pgfpathlineto{\pgfqpoint{5.219619in}{2.198869in}}%
\pgfpathlineto{\pgfqpoint{5.234337in}{2.204145in}}%
\pgfpathlineto{\pgfqpoint{5.242131in}{2.213214in}}%
\pgfpathlineto{\pgfqpoint{5.249916in}{2.222135in}}%
\pgfpathlineto{\pgfqpoint{5.257691in}{2.230907in}}%
\pgfpathlineto{\pgfqpoint{5.265458in}{2.239531in}}%
\pgfpathlineto{\pgfqpoint{5.250749in}{2.234261in}}%
\pgfpathlineto{\pgfqpoint{5.236054in}{2.229063in}}%
\pgfpathlineto{\pgfqpoint{5.221372in}{2.223936in}}%
\pgfpathlineto{\pgfqpoint{5.206704in}{2.218881in}}%
\pgfpathlineto{\pgfqpoint{5.198928in}{2.210243in}}%
\pgfpathlineto{\pgfqpoint{5.191142in}{2.201463in}}%
\pgfpathlineto{\pgfqpoint{5.183349in}{2.192540in}}%
\pgfpathlineto{\pgfqpoint{5.175546in}{2.183473in}}%
\pgfpathclose%
\pgfusepath{fill}%
\end{pgfscope}%
\begin{pgfscope}%
\pgfpathrectangle{\pgfqpoint{1.150000in}{0.150000in}}{\pgfqpoint{5.700000in}{5.700000in}}%
\pgfusepath{clip}%
\pgfsetbuttcap%
\pgfsetroundjoin%
\definecolor{currentfill}{rgb}{0.267004,0.004874,0.329415}%
\pgfsetfillcolor{currentfill}%
\pgfsetfillopacity{0.700000}%
\pgfsetlinewidth{0.000000pt}%
\definecolor{currentstroke}{rgb}{0.000000,0.000000,0.000000}%
\pgfsetstrokecolor{currentstroke}%
\pgfsetdash{}{0pt}%
\pgfpathmoveto{\pgfqpoint{3.470389in}{1.292112in}}%
\pgfpathlineto{\pgfqpoint{3.484478in}{1.287774in}}%
\pgfpathlineto{\pgfqpoint{3.498573in}{1.283510in}}%
\pgfpathlineto{\pgfqpoint{3.512673in}{1.279321in}}%
\pgfpathlineto{\pgfqpoint{3.526780in}{1.275205in}}%
\pgfpathlineto{\pgfqpoint{3.535222in}{1.282799in}}%
\pgfpathlineto{\pgfqpoint{3.543655in}{1.290556in}}%
\pgfpathlineto{\pgfqpoint{3.552080in}{1.298469in}}%
\pgfpathlineto{\pgfqpoint{3.560496in}{1.306534in}}%
\pgfpathlineto{\pgfqpoint{3.546408in}{1.310257in}}%
\pgfpathlineto{\pgfqpoint{3.532326in}{1.314055in}}%
\pgfpathlineto{\pgfqpoint{3.518250in}{1.317926in}}%
\pgfpathlineto{\pgfqpoint{3.504180in}{1.321873in}}%
\pgfpathlineto{\pgfqpoint{3.495745in}{1.314192in}}%
\pgfpathlineto{\pgfqpoint{3.487302in}{1.306668in}}%
\pgfpathlineto{\pgfqpoint{3.478850in}{1.299306in}}%
\pgfpathlineto{\pgfqpoint{3.470389in}{1.292112in}}%
\pgfpathclose%
\pgfusepath{fill}%
\end{pgfscope}%
\begin{pgfscope}%
\pgfpathrectangle{\pgfqpoint{1.150000in}{0.150000in}}{\pgfqpoint{5.700000in}{5.700000in}}%
\pgfusepath{clip}%
\pgfsetbuttcap%
\pgfsetroundjoin%
\definecolor{currentfill}{rgb}{0.271305,0.019942,0.347269}%
\pgfsetfillcolor{currentfill}%
\pgfsetfillopacity{0.700000}%
\pgfsetlinewidth{0.000000pt}%
\definecolor{currentstroke}{rgb}{0.000000,0.000000,0.000000}%
\pgfsetstrokecolor{currentstroke}%
\pgfsetdash{}{0pt}%
\pgfpathmoveto{\pgfqpoint{3.706888in}{1.316034in}}%
\pgfpathlineto{\pgfqpoint{3.721020in}{1.313344in}}%
\pgfpathlineto{\pgfqpoint{3.735158in}{1.310727in}}%
\pgfpathlineto{\pgfqpoint{3.749304in}{1.308182in}}%
\pgfpathlineto{\pgfqpoint{3.763456in}{1.305709in}}%
\pgfpathlineto{\pgfqpoint{3.771789in}{1.315522in}}%
\pgfpathlineto{\pgfqpoint{3.780115in}{1.325436in}}%
\pgfpathlineto{\pgfqpoint{3.788435in}{1.335448in}}%
\pgfpathlineto{\pgfqpoint{3.796748in}{1.345550in}}%
\pgfpathlineto{\pgfqpoint{3.782608in}{1.347671in}}%
\pgfpathlineto{\pgfqpoint{3.768476in}{1.349864in}}%
\pgfpathlineto{\pgfqpoint{3.754351in}{1.352130in}}%
\pgfpathlineto{\pgfqpoint{3.740234in}{1.354469in}}%
\pgfpathlineto{\pgfqpoint{3.731907in}{1.344710in}}%
\pgfpathlineto{\pgfqpoint{3.723575in}{1.335048in}}%
\pgfpathlineto{\pgfqpoint{3.715235in}{1.325487in}}%
\pgfpathlineto{\pgfqpoint{3.706888in}{1.316034in}}%
\pgfpathclose%
\pgfusepath{fill}%
\end{pgfscope}%
\begin{pgfscope}%
\pgfpathrectangle{\pgfqpoint{1.150000in}{0.150000in}}{\pgfqpoint{5.700000in}{5.700000in}}%
\pgfusepath{clip}%
\pgfsetbuttcap%
\pgfsetroundjoin%
\definecolor{currentfill}{rgb}{0.171176,0.452530,0.557965}%
\pgfsetfillcolor{currentfill}%
\pgfsetfillopacity{0.700000}%
\pgfsetlinewidth{0.000000pt}%
\definecolor{currentstroke}{rgb}{0.000000,0.000000,0.000000}%
\pgfsetstrokecolor{currentstroke}%
\pgfsetdash{}{0pt}%
\pgfpathmoveto{\pgfqpoint{5.355364in}{2.294245in}}%
\pgfpathlineto{\pgfqpoint{5.370131in}{2.299844in}}%
\pgfpathlineto{\pgfqpoint{5.384912in}{2.305516in}}%
\pgfpathlineto{\pgfqpoint{5.399708in}{2.311259in}}%
\pgfpathlineto{\pgfqpoint{5.407409in}{2.319082in}}%
\pgfpathlineto{\pgfqpoint{5.415100in}{2.326752in}}%
\pgfpathlineto{\pgfqpoint{5.422782in}{2.334273in}}%
\pgfpathlineto{\pgfqpoint{5.430454in}{2.341643in}}%
\pgfpathlineto{\pgfqpoint{5.415670in}{2.335951in}}%
\pgfpathlineto{\pgfqpoint{5.400900in}{2.330330in}}%
\pgfpathlineto{\pgfqpoint{5.386145in}{2.324781in}}%
\pgfpathlineto{\pgfqpoint{5.378464in}{2.317366in}}%
\pgfpathlineto{\pgfqpoint{5.370773in}{2.309806in}}%
\pgfpathlineto{\pgfqpoint{5.363073in}{2.302099in}}%
\pgfpathlineto{\pgfqpoint{5.355364in}{2.294245in}}%
\pgfpathclose%
\pgfusepath{fill}%
\end{pgfscope}%
\begin{pgfscope}%
\pgfpathrectangle{\pgfqpoint{1.150000in}{0.150000in}}{\pgfqpoint{5.700000in}{5.700000in}}%
\pgfusepath{clip}%
\pgfsetbuttcap%
\pgfsetroundjoin%
\definecolor{currentfill}{rgb}{0.255645,0.260703,0.528312}%
\pgfsetfillcolor{currentfill}%
\pgfsetfillopacity{0.700000}%
\pgfsetlinewidth{0.000000pt}%
\definecolor{currentstroke}{rgb}{0.000000,0.000000,0.000000}%
\pgfsetstrokecolor{currentstroke}%
\pgfsetdash{}{0pt}%
\pgfpathmoveto{\pgfqpoint{4.604361in}{1.785766in}}%
\pgfpathlineto{\pgfqpoint{4.618783in}{1.788540in}}%
\pgfpathlineto{\pgfqpoint{4.633218in}{1.791385in}}%
\pgfpathlineto{\pgfqpoint{4.647663in}{1.794300in}}%
\pgfpathlineto{\pgfqpoint{4.662120in}{1.797287in}}%
\pgfpathlineto{\pgfqpoint{4.670158in}{1.809449in}}%
\pgfpathlineto{\pgfqpoint{4.678190in}{1.821515in}}%
\pgfpathlineto{\pgfqpoint{4.686216in}{1.833483in}}%
\pgfpathlineto{\pgfqpoint{4.694236in}{1.845353in}}%
\pgfpathlineto{\pgfqpoint{4.679784in}{1.842219in}}%
\pgfpathlineto{\pgfqpoint{4.665343in}{1.839156in}}%
\pgfpathlineto{\pgfqpoint{4.650914in}{1.836164in}}%
\pgfpathlineto{\pgfqpoint{4.636496in}{1.833244in}}%
\pgfpathlineto{\pgfqpoint{4.628471in}{1.821513in}}%
\pgfpathlineto{\pgfqpoint{4.620440in}{1.809689in}}%
\pgfpathlineto{\pgfqpoint{4.612403in}{1.797772in}}%
\pgfpathlineto{\pgfqpoint{4.604361in}{1.785766in}}%
\pgfpathclose%
\pgfusepath{fill}%
\end{pgfscope}%
\begin{pgfscope}%
\pgfpathrectangle{\pgfqpoint{1.150000in}{0.150000in}}{\pgfqpoint{5.700000in}{5.700000in}}%
\pgfusepath{clip}%
\pgfsetbuttcap%
\pgfsetroundjoin%
\definecolor{currentfill}{rgb}{0.179019,0.433756,0.557430}%
\pgfsetfillcolor{currentfill}%
\pgfsetfillopacity{0.700000}%
\pgfsetlinewidth{0.000000pt}%
\definecolor{currentstroke}{rgb}{0.000000,0.000000,0.000000}%
\pgfsetstrokecolor{currentstroke}%
\pgfsetdash{}{0pt}%
\pgfpathmoveto{\pgfqpoint{5.265458in}{2.239531in}}%
\pgfpathlineto{\pgfqpoint{5.280180in}{2.244872in}}%
\pgfpathlineto{\pgfqpoint{5.294916in}{2.250285in}}%
\pgfpathlineto{\pgfqpoint{5.309667in}{2.255769in}}%
\pgfpathlineto{\pgfqpoint{5.324431in}{2.261325in}}%
\pgfpathlineto{\pgfqpoint{5.332178in}{2.269782in}}%
\pgfpathlineto{\pgfqpoint{5.339916in}{2.278087in}}%
\pgfpathlineto{\pgfqpoint{5.347645in}{2.286241in}}%
\pgfpathlineto{\pgfqpoint{5.355364in}{2.294245in}}%
\pgfpathlineto{\pgfqpoint{5.340610in}{2.288717in}}%
\pgfpathlineto{\pgfqpoint{5.325871in}{2.283261in}}%
\pgfpathlineto{\pgfqpoint{5.311145in}{2.277876in}}%
\pgfpathlineto{\pgfqpoint{5.296433in}{2.272563in}}%
\pgfpathlineto{\pgfqpoint{5.288703in}{2.264522in}}%
\pgfpathlineto{\pgfqpoint{5.280964in}{2.256338in}}%
\pgfpathlineto{\pgfqpoint{5.273215in}{2.248007in}}%
\pgfpathlineto{\pgfqpoint{5.265458in}{2.239531in}}%
\pgfpathclose%
\pgfusepath{fill}%
\end{pgfscope}%
\begin{pgfscope}%
\pgfpathrectangle{\pgfqpoint{1.150000in}{0.150000in}}{\pgfqpoint{5.700000in}{5.700000in}}%
\pgfusepath{clip}%
\pgfsetbuttcap%
\pgfsetroundjoin%
\definecolor{currentfill}{rgb}{0.244972,0.287675,0.537260}%
\pgfsetfillcolor{currentfill}%
\pgfsetfillopacity{0.700000}%
\pgfsetlinewidth{0.000000pt}%
\definecolor{currentstroke}{rgb}{0.000000,0.000000,0.000000}%
\pgfsetstrokecolor{currentstroke}%
\pgfsetdash{}{0pt}%
\pgfpathmoveto{\pgfqpoint{4.694236in}{1.845353in}}%
\pgfpathlineto{\pgfqpoint{4.708700in}{1.848558in}}%
\pgfpathlineto{\pgfqpoint{4.723176in}{1.851834in}}%
\pgfpathlineto{\pgfqpoint{4.737664in}{1.855182in}}%
\pgfpathlineto{\pgfqpoint{4.752163in}{1.858600in}}%
\pgfpathlineto{\pgfqpoint{4.760172in}{1.870503in}}%
\pgfpathlineto{\pgfqpoint{4.768174in}{1.882297in}}%
\pgfpathlineto{\pgfqpoint{4.776170in}{1.893983in}}%
\pgfpathlineto{\pgfqpoint{4.784159in}{1.905557in}}%
\pgfpathlineto{\pgfqpoint{4.769664in}{1.902012in}}%
\pgfpathlineto{\pgfqpoint{4.755182in}{1.898539in}}%
\pgfpathlineto{\pgfqpoint{4.740711in}{1.895137in}}%
\pgfpathlineto{\pgfqpoint{4.726252in}{1.891806in}}%
\pgfpathlineto{\pgfqpoint{4.718258in}{1.880350in}}%
\pgfpathlineto{\pgfqpoint{4.710257in}{1.868788in}}%
\pgfpathlineto{\pgfqpoint{4.702249in}{1.857122in}}%
\pgfpathlineto{\pgfqpoint{4.694236in}{1.845353in}}%
\pgfpathclose%
\pgfusepath{fill}%
\end{pgfscope}%
\begin{pgfscope}%
\pgfpathrectangle{\pgfqpoint{1.150000in}{0.150000in}}{\pgfqpoint{5.700000in}{5.700000in}}%
\pgfusepath{clip}%
\pgfsetbuttcap%
\pgfsetroundjoin%
\definecolor{currentfill}{rgb}{0.269944,0.014625,0.341379}%
\pgfsetfillcolor{currentfill}%
\pgfsetfillopacity{0.700000}%
\pgfsetlinewidth{0.000000pt}%
\definecolor{currentstroke}{rgb}{0.000000,0.000000,0.000000}%
\pgfsetstrokecolor{currentstroke}%
\pgfsetdash{}{0pt}%
\pgfpathmoveto{\pgfqpoint{3.616911in}{1.292382in}}%
\pgfpathlineto{\pgfqpoint{3.631031in}{1.289027in}}%
\pgfpathlineto{\pgfqpoint{3.645158in}{1.285746in}}%
\pgfpathlineto{\pgfqpoint{3.659291in}{1.282538in}}%
\pgfpathlineto{\pgfqpoint{3.673430in}{1.279403in}}%
\pgfpathlineto{\pgfqpoint{3.681806in}{1.288373in}}%
\pgfpathlineto{\pgfqpoint{3.690174in}{1.297471in}}%
\pgfpathlineto{\pgfqpoint{3.698534in}{1.306694in}}%
\pgfpathlineto{\pgfqpoint{3.706888in}{1.316034in}}%
\pgfpathlineto{\pgfqpoint{3.692764in}{1.318797in}}%
\pgfpathlineto{\pgfqpoint{3.678646in}{1.321633in}}%
\pgfpathlineto{\pgfqpoint{3.664535in}{1.324542in}}%
\pgfpathlineto{\pgfqpoint{3.650432in}{1.327525in}}%
\pgfpathlineto{\pgfqpoint{3.642063in}{1.318549in}}%
\pgfpathlineto{\pgfqpoint{3.633687in}{1.309696in}}%
\pgfpathlineto{\pgfqpoint{3.625303in}{1.300971in}}%
\pgfpathlineto{\pgfqpoint{3.616911in}{1.292382in}}%
\pgfpathclose%
\pgfusepath{fill}%
\end{pgfscope}%
\begin{pgfscope}%
\pgfpathrectangle{\pgfqpoint{1.150000in}{0.150000in}}{\pgfqpoint{5.700000in}{5.700000in}}%
\pgfusepath{clip}%
\pgfsetbuttcap%
\pgfsetroundjoin%
\definecolor{currentfill}{rgb}{0.233603,0.313828,0.543914}%
\pgfsetfillcolor{currentfill}%
\pgfsetfillopacity{0.700000}%
\pgfsetlinewidth{0.000000pt}%
\definecolor{currentstroke}{rgb}{0.000000,0.000000,0.000000}%
\pgfsetstrokecolor{currentstroke}%
\pgfsetdash{}{0pt}%
\pgfpathmoveto{\pgfqpoint{4.784159in}{1.905557in}}%
\pgfpathlineto{\pgfqpoint{4.798666in}{1.909172in}}%
\pgfpathlineto{\pgfqpoint{4.813184in}{1.912859in}}%
\pgfpathlineto{\pgfqpoint{4.827715in}{1.916617in}}%
\pgfpathlineto{\pgfqpoint{4.842259in}{1.920446in}}%
\pgfpathlineto{\pgfqpoint{4.850236in}{1.932020in}}%
\pgfpathlineto{\pgfqpoint{4.858207in}{1.943475in}}%
\pgfpathlineto{\pgfqpoint{4.866170in}{1.954810in}}%
\pgfpathlineto{\pgfqpoint{4.874127in}{1.966023in}}%
\pgfpathlineto{\pgfqpoint{4.859589in}{1.962089in}}%
\pgfpathlineto{\pgfqpoint{4.845064in}{1.958227in}}%
\pgfpathlineto{\pgfqpoint{4.830550in}{1.954436in}}%
\pgfpathlineto{\pgfqpoint{4.816049in}{1.950716in}}%
\pgfpathlineto{\pgfqpoint{4.808087in}{1.939599in}}%
\pgfpathlineto{\pgfqpoint{4.800117in}{1.928366in}}%
\pgfpathlineto{\pgfqpoint{4.792142in}{1.917018in}}%
\pgfpathlineto{\pgfqpoint{4.784159in}{1.905557in}}%
\pgfpathclose%
\pgfusepath{fill}%
\end{pgfscope}%
\begin{pgfscope}%
\pgfpathrectangle{\pgfqpoint{1.150000in}{0.150000in}}{\pgfqpoint{5.700000in}{5.700000in}}%
\pgfusepath{clip}%
\pgfsetbuttcap%
\pgfsetroundjoin%
\definecolor{currentfill}{rgb}{0.283091,0.110553,0.431554}%
\pgfsetfillcolor{currentfill}%
\pgfsetfillopacity{0.700000}%
\pgfsetlinewidth{0.000000pt}%
\definecolor{currentstroke}{rgb}{0.000000,0.000000,0.000000}%
\pgfsetstrokecolor{currentstroke}%
\pgfsetdash{}{0pt}%
\pgfpathmoveto{\pgfqpoint{4.122845in}{1.462947in}}%
\pgfpathlineto{\pgfqpoint{4.137096in}{1.462980in}}%
\pgfpathlineto{\pgfqpoint{4.151356in}{1.463084in}}%
\pgfpathlineto{\pgfqpoint{4.165626in}{1.463259in}}%
\pgfpathlineto{\pgfqpoint{4.179904in}{1.463505in}}%
\pgfpathlineto{\pgfqpoint{4.188097in}{1.475757in}}%
\pgfpathlineto{\pgfqpoint{4.196284in}{1.488010in}}%
\pgfpathlineto{\pgfqpoint{4.204466in}{1.500262in}}%
\pgfpathlineto{\pgfqpoint{4.212643in}{1.512508in}}%
\pgfpathlineto{\pgfqpoint{4.198371in}{1.511990in}}%
\pgfpathlineto{\pgfqpoint{4.184109in}{1.511544in}}%
\pgfpathlineto{\pgfqpoint{4.169856in}{1.511169in}}%
\pgfpathlineto{\pgfqpoint{4.155612in}{1.510865in}}%
\pgfpathlineto{\pgfqpoint{4.147428in}{1.498882in}}%
\pgfpathlineto{\pgfqpoint{4.139239in}{1.486899in}}%
\pgfpathlineto{\pgfqpoint{4.131045in}{1.474920in}}%
\pgfpathlineto{\pgfqpoint{4.122845in}{1.462947in}}%
\pgfpathclose%
\pgfusepath{fill}%
\end{pgfscope}%
\begin{pgfscope}%
\pgfpathrectangle{\pgfqpoint{1.150000in}{0.150000in}}{\pgfqpoint{5.700000in}{5.700000in}}%
\pgfusepath{clip}%
\pgfsetbuttcap%
\pgfsetroundjoin%
\definecolor{currentfill}{rgb}{0.281924,0.089666,0.412415}%
\pgfsetfillcolor{currentfill}%
\pgfsetfillopacity{0.700000}%
\pgfsetlinewidth{0.000000pt}%
\definecolor{currentstroke}{rgb}{0.000000,0.000000,0.000000}%
\pgfsetstrokecolor{currentstroke}%
\pgfsetdash{}{0pt}%
\pgfpathmoveto{\pgfqpoint{4.033050in}{1.416960in}}%
\pgfpathlineto{\pgfqpoint{4.047273in}{1.416416in}}%
\pgfpathlineto{\pgfqpoint{4.061505in}{1.415944in}}%
\pgfpathlineto{\pgfqpoint{4.075746in}{1.415542in}}%
\pgfpathlineto{\pgfqpoint{4.089995in}{1.415212in}}%
\pgfpathlineto{\pgfqpoint{4.098216in}{1.427115in}}%
\pgfpathlineto{\pgfqpoint{4.106431in}{1.439041in}}%
\pgfpathlineto{\pgfqpoint{4.114641in}{1.450986in}}%
\pgfpathlineto{\pgfqpoint{4.122845in}{1.462947in}}%
\pgfpathlineto{\pgfqpoint{4.108604in}{1.462986in}}%
\pgfpathlineto{\pgfqpoint{4.094371in}{1.463095in}}%
\pgfpathlineto{\pgfqpoint{4.080147in}{1.463276in}}%
\pgfpathlineto{\pgfqpoint{4.065932in}{1.463529in}}%
\pgfpathlineto{\pgfqpoint{4.057720in}{1.451851in}}%
\pgfpathlineto{\pgfqpoint{4.049502in}{1.440195in}}%
\pgfpathlineto{\pgfqpoint{4.041279in}{1.428563in}}%
\pgfpathlineto{\pgfqpoint{4.033050in}{1.416960in}}%
\pgfpathclose%
\pgfusepath{fill}%
\end{pgfscope}%
\begin{pgfscope}%
\pgfpathrectangle{\pgfqpoint{1.150000in}{0.150000in}}{\pgfqpoint{5.700000in}{5.700000in}}%
\pgfusepath{clip}%
\pgfsetbuttcap%
\pgfsetroundjoin%
\definecolor{currentfill}{rgb}{0.282884,0.135920,0.453427}%
\pgfsetfillcolor{currentfill}%
\pgfsetfillopacity{0.700000}%
\pgfsetlinewidth{0.000000pt}%
\definecolor{currentstroke}{rgb}{0.000000,0.000000,0.000000}%
\pgfsetstrokecolor{currentstroke}%
\pgfsetdash{}{0pt}%
\pgfpathmoveto{\pgfqpoint{4.212643in}{1.512508in}}%
\pgfpathlineto{\pgfqpoint{4.226924in}{1.513096in}}%
\pgfpathlineto{\pgfqpoint{4.241215in}{1.513756in}}%
\pgfpathlineto{\pgfqpoint{4.255516in}{1.514487in}}%
\pgfpathlineto{\pgfqpoint{4.269827in}{1.515288in}}%
\pgfpathlineto{\pgfqpoint{4.277992in}{1.527784in}}%
\pgfpathlineto{\pgfqpoint{4.286153in}{1.540261in}}%
\pgfpathlineto{\pgfqpoint{4.294308in}{1.552716in}}%
\pgfpathlineto{\pgfqpoint{4.302459in}{1.565146in}}%
\pgfpathlineto{\pgfqpoint{4.288154in}{1.564093in}}%
\pgfpathlineto{\pgfqpoint{4.273860in}{1.563111in}}%
\pgfpathlineto{\pgfqpoint{4.259575in}{1.562201in}}%
\pgfpathlineto{\pgfqpoint{4.245300in}{1.561361in}}%
\pgfpathlineto{\pgfqpoint{4.237143in}{1.549174in}}%
\pgfpathlineto{\pgfqpoint{4.228982in}{1.536968in}}%
\pgfpathlineto{\pgfqpoint{4.220815in}{1.524744in}}%
\pgfpathlineto{\pgfqpoint{4.212643in}{1.512508in}}%
\pgfpathclose%
\pgfusepath{fill}%
\end{pgfscope}%
\begin{pgfscope}%
\pgfpathrectangle{\pgfqpoint{1.150000in}{0.150000in}}{\pgfqpoint{5.700000in}{5.700000in}}%
\pgfusepath{clip}%
\pgfsetbuttcap%
\pgfsetroundjoin%
\definecolor{currentfill}{rgb}{0.279566,0.067836,0.391917}%
\pgfsetfillcolor{currentfill}%
\pgfsetfillopacity{0.700000}%
\pgfsetlinewidth{0.000000pt}%
\definecolor{currentstroke}{rgb}{0.000000,0.000000,0.000000}%
\pgfsetstrokecolor{currentstroke}%
\pgfsetdash{}{0pt}%
\pgfpathmoveto{\pgfqpoint{3.943236in}{1.375061in}}%
\pgfpathlineto{\pgfqpoint{3.957435in}{1.373920in}}%
\pgfpathlineto{\pgfqpoint{3.971642in}{1.372850in}}%
\pgfpathlineto{\pgfqpoint{3.985857in}{1.371852in}}%
\pgfpathlineto{\pgfqpoint{4.000081in}{1.370925in}}%
\pgfpathlineto{\pgfqpoint{4.008331in}{1.382368in}}%
\pgfpathlineto{\pgfqpoint{4.016576in}{1.393858in}}%
\pgfpathlineto{\pgfqpoint{4.024816in}{1.405390in}}%
\pgfpathlineto{\pgfqpoint{4.033050in}{1.416960in}}%
\pgfpathlineto{\pgfqpoint{4.018835in}{1.417575in}}%
\pgfpathlineto{\pgfqpoint{4.004629in}{1.418261in}}%
\pgfpathlineto{\pgfqpoint{3.990432in}{1.419019in}}%
\pgfpathlineto{\pgfqpoint{3.976243in}{1.419849in}}%
\pgfpathlineto{\pgfqpoint{3.968000in}{1.408583in}}%
\pgfpathlineto{\pgfqpoint{3.959751in}{1.397360in}}%
\pgfpathlineto{\pgfqpoint{3.951497in}{1.386185in}}%
\pgfpathlineto{\pgfqpoint{3.943236in}{1.375061in}}%
\pgfpathclose%
\pgfusepath{fill}%
\end{pgfscope}%
\begin{pgfscope}%
\pgfpathrectangle{\pgfqpoint{1.150000in}{0.150000in}}{\pgfqpoint{5.700000in}{5.700000in}}%
\pgfusepath{clip}%
\pgfsetbuttcap%
\pgfsetroundjoin%
\definecolor{currentfill}{rgb}{0.268510,0.009605,0.335427}%
\pgfsetfillcolor{currentfill}%
\pgfsetfillopacity{0.700000}%
\pgfsetlinewidth{0.000000pt}%
\definecolor{currentstroke}{rgb}{0.000000,0.000000,0.000000}%
\pgfsetstrokecolor{currentstroke}%
\pgfsetdash{}{0pt}%
\pgfpathmoveto{\pgfqpoint{3.380066in}{1.284902in}}%
\pgfpathlineto{\pgfqpoint{3.394154in}{1.279850in}}%
\pgfpathlineto{\pgfqpoint{3.408248in}{1.274873in}}%
\pgfpathlineto{\pgfqpoint{3.422347in}{1.269972in}}%
\pgfpathlineto{\pgfqpoint{3.436451in}{1.265145in}}%
\pgfpathlineto{\pgfqpoint{3.444950in}{1.271604in}}%
\pgfpathlineto{\pgfqpoint{3.453439in}{1.278255in}}%
\pgfpathlineto{\pgfqpoint{3.461919in}{1.285093in}}%
\pgfpathlineto{\pgfqpoint{3.470389in}{1.292112in}}%
\pgfpathlineto{\pgfqpoint{3.456305in}{1.296526in}}%
\pgfpathlineto{\pgfqpoint{3.442228in}{1.301014in}}%
\pgfpathlineto{\pgfqpoint{3.428155in}{1.305577in}}%
\pgfpathlineto{\pgfqpoint{3.414089in}{1.310216in}}%
\pgfpathlineto{\pgfqpoint{3.405598in}{1.303603in}}%
\pgfpathlineto{\pgfqpoint{3.397097in}{1.297175in}}%
\pgfpathlineto{\pgfqpoint{3.388587in}{1.290939in}}%
\pgfpathlineto{\pgfqpoint{3.380066in}{1.284902in}}%
\pgfpathclose%
\pgfusepath{fill}%
\end{pgfscope}%
\begin{pgfscope}%
\pgfpathrectangle{\pgfqpoint{1.150000in}{0.150000in}}{\pgfqpoint{5.700000in}{5.700000in}}%
\pgfusepath{clip}%
\pgfsetbuttcap%
\pgfsetroundjoin%
\definecolor{currentfill}{rgb}{0.280255,0.165693,0.476498}%
\pgfsetfillcolor{currentfill}%
\pgfsetfillopacity{0.700000}%
\pgfsetlinewidth{0.000000pt}%
\definecolor{currentstroke}{rgb}{0.000000,0.000000,0.000000}%
\pgfsetstrokecolor{currentstroke}%
\pgfsetdash{}{0pt}%
\pgfpathmoveto{\pgfqpoint{4.302459in}{1.565146in}}%
\pgfpathlineto{\pgfqpoint{4.316773in}{1.566269in}}%
\pgfpathlineto{\pgfqpoint{4.331098in}{1.567464in}}%
\pgfpathlineto{\pgfqpoint{4.345433in}{1.568730in}}%
\pgfpathlineto{\pgfqpoint{4.359778in}{1.570066in}}%
\pgfpathlineto{\pgfqpoint{4.367918in}{1.582706in}}%
\pgfpathlineto{\pgfqpoint{4.376052in}{1.595308in}}%
\pgfpathlineto{\pgfqpoint{4.384182in}{1.607869in}}%
\pgfpathlineto{\pgfqpoint{4.392306in}{1.620387in}}%
\pgfpathlineto{\pgfqpoint{4.377966in}{1.618819in}}%
\pgfpathlineto{\pgfqpoint{4.363637in}{1.617323in}}%
\pgfpathlineto{\pgfqpoint{4.349318in}{1.615898in}}%
\pgfpathlineto{\pgfqpoint{4.335009in}{1.614543in}}%
\pgfpathlineto{\pgfqpoint{4.326879in}{1.602249in}}%
\pgfpathlineto{\pgfqpoint{4.318744in}{1.589915in}}%
\pgfpathlineto{\pgfqpoint{4.310604in}{1.577547in}}%
\pgfpathlineto{\pgfqpoint{4.302459in}{1.565146in}}%
\pgfpathclose%
\pgfusepath{fill}%
\end{pgfscope}%
\begin{pgfscope}%
\pgfpathrectangle{\pgfqpoint{1.150000in}{0.150000in}}{\pgfqpoint{5.700000in}{5.700000in}}%
\pgfusepath{clip}%
\pgfsetbuttcap%
\pgfsetroundjoin%
\definecolor{currentfill}{rgb}{0.221989,0.339161,0.548752}%
\pgfsetfillcolor{currentfill}%
\pgfsetfillopacity{0.700000}%
\pgfsetlinewidth{0.000000pt}%
\definecolor{currentstroke}{rgb}{0.000000,0.000000,0.000000}%
\pgfsetstrokecolor{currentstroke}%
\pgfsetdash{}{0pt}%
\pgfpathmoveto{\pgfqpoint{4.874127in}{1.966023in}}%
\pgfpathlineto{\pgfqpoint{4.888678in}{1.970028in}}%
\pgfpathlineto{\pgfqpoint{4.903240in}{1.974104in}}%
\pgfpathlineto{\pgfqpoint{4.917816in}{1.978251in}}%
\pgfpathlineto{\pgfqpoint{4.932404in}{1.982470in}}%
\pgfpathlineto{\pgfqpoint{4.940348in}{1.993652in}}%
\pgfpathlineto{\pgfqpoint{4.948285in}{2.004704in}}%
\pgfpathlineto{\pgfqpoint{4.956214in}{2.015626in}}%
\pgfpathlineto{\pgfqpoint{4.964136in}{2.026418in}}%
\pgfpathlineto{\pgfqpoint{4.949554in}{2.022117in}}%
\pgfpathlineto{\pgfqpoint{4.934985in}{2.017887in}}%
\pgfpathlineto{\pgfqpoint{4.920428in}{2.013728in}}%
\pgfpathlineto{\pgfqpoint{4.905883in}{2.009641in}}%
\pgfpathlineto{\pgfqpoint{4.897955in}{1.998923in}}%
\pgfpathlineto{\pgfqpoint{4.890020in}{1.988080in}}%
\pgfpathlineto{\pgfqpoint{4.882077in}{1.977113in}}%
\pgfpathlineto{\pgfqpoint{4.874127in}{1.966023in}}%
\pgfpathclose%
\pgfusepath{fill}%
\end{pgfscope}%
\begin{pgfscope}%
\pgfpathrectangle{\pgfqpoint{1.150000in}{0.150000in}}{\pgfqpoint{5.700000in}{5.700000in}}%
\pgfusepath{clip}%
\pgfsetbuttcap%
\pgfsetroundjoin%
\definecolor{currentfill}{rgb}{0.276022,0.044167,0.370164}%
\pgfsetfillcolor{currentfill}%
\pgfsetfillopacity{0.700000}%
\pgfsetlinewidth{0.000000pt}%
\definecolor{currentstroke}{rgb}{0.000000,0.000000,0.000000}%
\pgfsetstrokecolor{currentstroke}%
\pgfsetdash{}{0pt}%
\pgfpathmoveto{\pgfqpoint{3.853381in}{1.337790in}}%
\pgfpathlineto{\pgfqpoint{3.867559in}{1.336031in}}%
\pgfpathlineto{\pgfqpoint{3.881744in}{1.334342in}}%
\pgfpathlineto{\pgfqpoint{3.895937in}{1.332726in}}%
\pgfpathlineto{\pgfqpoint{3.910138in}{1.331181in}}%
\pgfpathlineto{\pgfqpoint{3.918421in}{1.342049in}}%
\pgfpathlineto{\pgfqpoint{3.926699in}{1.352989in}}%
\pgfpathlineto{\pgfqpoint{3.934970in}{1.363994in}}%
\pgfpathlineto{\pgfqpoint{3.943236in}{1.375061in}}%
\pgfpathlineto{\pgfqpoint{3.929046in}{1.376274in}}%
\pgfpathlineto{\pgfqpoint{3.914864in}{1.377558in}}%
\pgfpathlineto{\pgfqpoint{3.900689in}{1.378915in}}%
\pgfpathlineto{\pgfqpoint{3.886523in}{1.380343in}}%
\pgfpathlineto{\pgfqpoint{3.878247in}{1.369600in}}%
\pgfpathlineto{\pgfqpoint{3.869964in}{1.358924in}}%
\pgfpathlineto{\pgfqpoint{3.861676in}{1.348319in}}%
\pgfpathlineto{\pgfqpoint{3.853381in}{1.337790in}}%
\pgfpathclose%
\pgfusepath{fill}%
\end{pgfscope}%
\begin{pgfscope}%
\pgfpathrectangle{\pgfqpoint{1.150000in}{0.150000in}}{\pgfqpoint{5.700000in}{5.700000in}}%
\pgfusepath{clip}%
\pgfsetbuttcap%
\pgfsetroundjoin%
\definecolor{currentfill}{rgb}{0.276194,0.190074,0.493001}%
\pgfsetfillcolor{currentfill}%
\pgfsetfillopacity{0.700000}%
\pgfsetlinewidth{0.000000pt}%
\definecolor{currentstroke}{rgb}{0.000000,0.000000,0.000000}%
\pgfsetstrokecolor{currentstroke}%
\pgfsetdash{}{0pt}%
\pgfpathmoveto{\pgfqpoint{4.392306in}{1.620387in}}%
\pgfpathlineto{\pgfqpoint{4.406656in}{1.622025in}}%
\pgfpathlineto{\pgfqpoint{4.421017in}{1.623734in}}%
\pgfpathlineto{\pgfqpoint{4.435388in}{1.625514in}}%
\pgfpathlineto{\pgfqpoint{4.449770in}{1.627365in}}%
\pgfpathlineto{\pgfqpoint{4.457884in}{1.640053in}}%
\pgfpathlineto{\pgfqpoint{4.465993in}{1.652687in}}%
\pgfpathlineto{\pgfqpoint{4.474096in}{1.665262in}}%
\pgfpathlineto{\pgfqpoint{4.482195in}{1.677776in}}%
\pgfpathlineto{\pgfqpoint{4.467818in}{1.675715in}}%
\pgfpathlineto{\pgfqpoint{4.453451in}{1.673725in}}%
\pgfpathlineto{\pgfqpoint{4.439096in}{1.671806in}}%
\pgfpathlineto{\pgfqpoint{4.424751in}{1.669958in}}%
\pgfpathlineto{\pgfqpoint{4.416648in}{1.657645in}}%
\pgfpathlineto{\pgfqpoint{4.408539in}{1.645278in}}%
\pgfpathlineto{\pgfqpoint{4.400425in}{1.632857in}}%
\pgfpathlineto{\pgfqpoint{4.392306in}{1.620387in}}%
\pgfpathclose%
\pgfusepath{fill}%
\end{pgfscope}%
\begin{pgfscope}%
\pgfpathrectangle{\pgfqpoint{1.150000in}{0.150000in}}{\pgfqpoint{5.700000in}{5.700000in}}%
\pgfusepath{clip}%
\pgfsetbuttcap%
\pgfsetroundjoin%
\definecolor{currentfill}{rgb}{0.268510,0.009605,0.335427}%
\pgfsetfillcolor{currentfill}%
\pgfsetfillopacity{0.700000}%
\pgfsetlinewidth{0.000000pt}%
\definecolor{currentstroke}{rgb}{0.000000,0.000000,0.000000}%
\pgfsetstrokecolor{currentstroke}%
\pgfsetdash{}{0pt}%
\pgfpathmoveto{\pgfqpoint{3.526780in}{1.275205in}}%
\pgfpathlineto{\pgfqpoint{3.540892in}{1.271164in}}%
\pgfpathlineto{\pgfqpoint{3.555011in}{1.267196in}}%
\pgfpathlineto{\pgfqpoint{3.569136in}{1.263303in}}%
\pgfpathlineto{\pgfqpoint{3.583266in}{1.259482in}}%
\pgfpathlineto{\pgfqpoint{3.591690in}{1.267476in}}%
\pgfpathlineto{\pgfqpoint{3.600105in}{1.275628in}}%
\pgfpathlineto{\pgfqpoint{3.608512in}{1.283932in}}%
\pgfpathlineto{\pgfqpoint{3.616911in}{1.292382in}}%
\pgfpathlineto{\pgfqpoint{3.602798in}{1.295809in}}%
\pgfpathlineto{\pgfqpoint{3.588691in}{1.299311in}}%
\pgfpathlineto{\pgfqpoint{3.574590in}{1.302886in}}%
\pgfpathlineto{\pgfqpoint{3.560496in}{1.306534in}}%
\pgfpathlineto{\pgfqpoint{3.552080in}{1.298469in}}%
\pgfpathlineto{\pgfqpoint{3.543655in}{1.290556in}}%
\pgfpathlineto{\pgfqpoint{3.535222in}{1.282799in}}%
\pgfpathlineto{\pgfqpoint{3.526780in}{1.275205in}}%
\pgfpathclose%
\pgfusepath{fill}%
\end{pgfscope}%
\begin{pgfscope}%
\pgfpathrectangle{\pgfqpoint{1.150000in}{0.150000in}}{\pgfqpoint{5.700000in}{5.700000in}}%
\pgfusepath{clip}%
\pgfsetbuttcap%
\pgfsetroundjoin%
\definecolor{currentfill}{rgb}{0.210503,0.363727,0.552206}%
\pgfsetfillcolor{currentfill}%
\pgfsetfillopacity{0.700000}%
\pgfsetlinewidth{0.000000pt}%
\definecolor{currentstroke}{rgb}{0.000000,0.000000,0.000000}%
\pgfsetstrokecolor{currentstroke}%
\pgfsetdash{}{0pt}%
\pgfpathmoveto{\pgfqpoint{4.964136in}{2.026418in}}%
\pgfpathlineto{\pgfqpoint{4.978731in}{2.030791in}}%
\pgfpathlineto{\pgfqpoint{4.993339in}{2.035235in}}%
\pgfpathlineto{\pgfqpoint{5.007960in}{2.039751in}}%
\pgfpathlineto{\pgfqpoint{5.022593in}{2.044338in}}%
\pgfpathlineto{\pgfqpoint{5.030502in}{2.055067in}}%
\pgfpathlineto{\pgfqpoint{5.038402in}{2.065659in}}%
\pgfpathlineto{\pgfqpoint{5.046295in}{2.076113in}}%
\pgfpathlineto{\pgfqpoint{5.054180in}{2.086429in}}%
\pgfpathlineto{\pgfqpoint{5.039553in}{2.081781in}}%
\pgfpathlineto{\pgfqpoint{5.024939in}{2.077205in}}%
\pgfpathlineto{\pgfqpoint{5.010338in}{2.072700in}}%
\pgfpathlineto{\pgfqpoint{4.995749in}{2.068266in}}%
\pgfpathlineto{\pgfqpoint{4.987858in}{2.058003in}}%
\pgfpathlineto{\pgfqpoint{4.979958in}{2.047607in}}%
\pgfpathlineto{\pgfqpoint{4.972051in}{2.037079in}}%
\pgfpathlineto{\pgfqpoint{4.964136in}{2.026418in}}%
\pgfpathclose%
\pgfusepath{fill}%
\end{pgfscope}%
\begin{pgfscope}%
\pgfpathrectangle{\pgfqpoint{1.150000in}{0.150000in}}{\pgfqpoint{5.700000in}{5.700000in}}%
\pgfusepath{clip}%
\pgfsetbuttcap%
\pgfsetroundjoin%
\definecolor{currentfill}{rgb}{0.273809,0.031497,0.358853}%
\pgfsetfillcolor{currentfill}%
\pgfsetfillopacity{0.700000}%
\pgfsetlinewidth{0.000000pt}%
\definecolor{currentstroke}{rgb}{0.000000,0.000000,0.000000}%
\pgfsetstrokecolor{currentstroke}%
\pgfsetdash{}{0pt}%
\pgfpathmoveto{\pgfqpoint{3.763456in}{1.305709in}}%
\pgfpathlineto{\pgfqpoint{3.777616in}{1.303309in}}%
\pgfpathlineto{\pgfqpoint{3.791784in}{1.300981in}}%
\pgfpathlineto{\pgfqpoint{3.805958in}{1.298725in}}%
\pgfpathlineto{\pgfqpoint{3.820140in}{1.296541in}}%
\pgfpathlineto{\pgfqpoint{3.828460in}{1.306713in}}%
\pgfpathlineto{\pgfqpoint{3.836773in}{1.316982in}}%
\pgfpathlineto{\pgfqpoint{3.845080in}{1.327343in}}%
\pgfpathlineto{\pgfqpoint{3.853381in}{1.337790in}}%
\pgfpathlineto{\pgfqpoint{3.839211in}{1.339622in}}%
\pgfpathlineto{\pgfqpoint{3.825049in}{1.341526in}}%
\pgfpathlineto{\pgfqpoint{3.810895in}{1.343502in}}%
\pgfpathlineto{\pgfqpoint{3.796748in}{1.345550in}}%
\pgfpathlineto{\pgfqpoint{3.788435in}{1.335448in}}%
\pgfpathlineto{\pgfqpoint{3.780115in}{1.325436in}}%
\pgfpathlineto{\pgfqpoint{3.771789in}{1.315522in}}%
\pgfpathlineto{\pgfqpoint{3.763456in}{1.305709in}}%
\pgfpathclose%
\pgfusepath{fill}%
\end{pgfscope}%
\begin{pgfscope}%
\pgfpathrectangle{\pgfqpoint{1.150000in}{0.150000in}}{\pgfqpoint{5.700000in}{5.700000in}}%
\pgfusepath{clip}%
\pgfsetbuttcap%
\pgfsetroundjoin%
\definecolor{currentfill}{rgb}{0.269308,0.218818,0.509577}%
\pgfsetfillcolor{currentfill}%
\pgfsetfillopacity{0.700000}%
\pgfsetlinewidth{0.000000pt}%
\definecolor{currentstroke}{rgb}{0.000000,0.000000,0.000000}%
\pgfsetstrokecolor{currentstroke}%
\pgfsetdash{}{0pt}%
\pgfpathmoveto{\pgfqpoint{4.482195in}{1.677776in}}%
\pgfpathlineto{\pgfqpoint{4.496583in}{1.679908in}}%
\pgfpathlineto{\pgfqpoint{4.510981in}{1.682112in}}%
\pgfpathlineto{\pgfqpoint{4.525391in}{1.684385in}}%
\pgfpathlineto{\pgfqpoint{4.539811in}{1.686730in}}%
\pgfpathlineto{\pgfqpoint{4.547900in}{1.699378in}}%
\pgfpathlineto{\pgfqpoint{4.555983in}{1.711954in}}%
\pgfpathlineto{\pgfqpoint{4.564060in}{1.724456in}}%
\pgfpathlineto{\pgfqpoint{4.572131in}{1.736881in}}%
\pgfpathlineto{\pgfqpoint{4.557715in}{1.734347in}}%
\pgfpathlineto{\pgfqpoint{4.543310in}{1.731883in}}%
\pgfpathlineto{\pgfqpoint{4.528916in}{1.729491in}}%
\pgfpathlineto{\pgfqpoint{4.514533in}{1.727170in}}%
\pgfpathlineto{\pgfqpoint{4.506457in}{1.714926in}}%
\pgfpathlineto{\pgfqpoint{4.498375in}{1.702611in}}%
\pgfpathlineto{\pgfqpoint{4.490288in}{1.690227in}}%
\pgfpathlineto{\pgfqpoint{4.482195in}{1.677776in}}%
\pgfpathclose%
\pgfusepath{fill}%
\end{pgfscope}%
\begin{pgfscope}%
\pgfpathrectangle{\pgfqpoint{1.150000in}{0.150000in}}{\pgfqpoint{5.700000in}{5.700000in}}%
\pgfusepath{clip}%
\pgfsetbuttcap%
\pgfsetroundjoin%
\definecolor{currentfill}{rgb}{0.199430,0.387607,0.554642}%
\pgfsetfillcolor{currentfill}%
\pgfsetfillopacity{0.700000}%
\pgfsetlinewidth{0.000000pt}%
\definecolor{currentstroke}{rgb}{0.000000,0.000000,0.000000}%
\pgfsetstrokecolor{currentstroke}%
\pgfsetdash{}{0pt}%
\pgfpathmoveto{\pgfqpoint{5.054180in}{2.086429in}}%
\pgfpathlineto{\pgfqpoint{5.068820in}{2.091148in}}%
\pgfpathlineto{\pgfqpoint{5.083474in}{2.095939in}}%
\pgfpathlineto{\pgfqpoint{5.098140in}{2.100801in}}%
\pgfpathlineto{\pgfqpoint{5.112820in}{2.105735in}}%
\pgfpathlineto{\pgfqpoint{5.120690in}{2.115959in}}%
\pgfpathlineto{\pgfqpoint{5.128552in}{2.126038in}}%
\pgfpathlineto{\pgfqpoint{5.136405in}{2.135972in}}%
\pgfpathlineto{\pgfqpoint{5.144250in}{2.145762in}}%
\pgfpathlineto{\pgfqpoint{5.129578in}{2.140789in}}%
\pgfpathlineto{\pgfqpoint{5.114919in}{2.135888in}}%
\pgfpathlineto{\pgfqpoint{5.100273in}{2.131059in}}%
\pgfpathlineto{\pgfqpoint{5.085640in}{2.126301in}}%
\pgfpathlineto{\pgfqpoint{5.077787in}{2.116542in}}%
\pgfpathlineto{\pgfqpoint{5.069926in}{2.106643in}}%
\pgfpathlineto{\pgfqpoint{5.062057in}{2.096606in}}%
\pgfpathlineto{\pgfqpoint{5.054180in}{2.086429in}}%
\pgfpathclose%
\pgfusepath{fill}%
\end{pgfscope}%
\begin{pgfscope}%
\pgfpathrectangle{\pgfqpoint{1.150000in}{0.150000in}}{\pgfqpoint{5.700000in}{5.700000in}}%
\pgfusepath{clip}%
\pgfsetbuttcap%
\pgfsetroundjoin%
\definecolor{currentfill}{rgb}{0.260571,0.246922,0.522828}%
\pgfsetfillcolor{currentfill}%
\pgfsetfillopacity{0.700000}%
\pgfsetlinewidth{0.000000pt}%
\definecolor{currentstroke}{rgb}{0.000000,0.000000,0.000000}%
\pgfsetstrokecolor{currentstroke}%
\pgfsetdash{}{0pt}%
\pgfpathmoveto{\pgfqpoint{4.572131in}{1.736881in}}%
\pgfpathlineto{\pgfqpoint{4.586559in}{1.739487in}}%
\pgfpathlineto{\pgfqpoint{4.600998in}{1.742163in}}%
\pgfpathlineto{\pgfqpoint{4.615448in}{1.744910in}}%
\pgfpathlineto{\pgfqpoint{4.629909in}{1.747729in}}%
\pgfpathlineto{\pgfqpoint{4.637971in}{1.760251in}}%
\pgfpathlineto{\pgfqpoint{4.646026in}{1.772686in}}%
\pgfpathlineto{\pgfqpoint{4.654076in}{1.785033in}}%
\pgfpathlineto{\pgfqpoint{4.662120in}{1.797287in}}%
\pgfpathlineto{\pgfqpoint{4.647663in}{1.794300in}}%
\pgfpathlineto{\pgfqpoint{4.633218in}{1.791385in}}%
\pgfpathlineto{\pgfqpoint{4.618783in}{1.788540in}}%
\pgfpathlineto{\pgfqpoint{4.604361in}{1.785766in}}%
\pgfpathlineto{\pgfqpoint{4.596312in}{1.773672in}}%
\pgfpathlineto{\pgfqpoint{4.588258in}{1.761491in}}%
\pgfpathlineto{\pgfqpoint{4.580197in}{1.749227in}}%
\pgfpathlineto{\pgfqpoint{4.572131in}{1.736881in}}%
\pgfpathclose%
\pgfusepath{fill}%
\end{pgfscope}%
\begin{pgfscope}%
\pgfpathrectangle{\pgfqpoint{1.150000in}{0.150000in}}{\pgfqpoint{5.700000in}{5.700000in}}%
\pgfusepath{clip}%
\pgfsetbuttcap%
\pgfsetroundjoin%
\definecolor{currentfill}{rgb}{0.188923,0.410910,0.556326}%
\pgfsetfillcolor{currentfill}%
\pgfsetfillopacity{0.700000}%
\pgfsetlinewidth{0.000000pt}%
\definecolor{currentstroke}{rgb}{0.000000,0.000000,0.000000}%
\pgfsetstrokecolor{currentstroke}%
\pgfsetdash{}{0pt}%
\pgfpathmoveto{\pgfqpoint{5.144250in}{2.145762in}}%
\pgfpathlineto{\pgfqpoint{5.158936in}{2.150806in}}%
\pgfpathlineto{\pgfqpoint{5.173635in}{2.155922in}}%
\pgfpathlineto{\pgfqpoint{5.188348in}{2.161109in}}%
\pgfpathlineto{\pgfqpoint{5.203075in}{2.166368in}}%
\pgfpathlineto{\pgfqpoint{5.210904in}{2.176037in}}%
\pgfpathlineto{\pgfqpoint{5.218724in}{2.185557in}}%
\pgfpathlineto{\pgfqpoint{5.226535in}{2.194925in}}%
\pgfpathlineto{\pgfqpoint{5.234337in}{2.204145in}}%
\pgfpathlineto{\pgfqpoint{5.219619in}{2.198869in}}%
\pgfpathlineto{\pgfqpoint{5.204915in}{2.193665in}}%
\pgfpathlineto{\pgfqpoint{5.190224in}{2.188533in}}%
\pgfpathlineto{\pgfqpoint{5.175546in}{2.183473in}}%
\pgfpathlineto{\pgfqpoint{5.167735in}{2.174262in}}%
\pgfpathlineto{\pgfqpoint{5.159915in}{2.164906in}}%
\pgfpathlineto{\pgfqpoint{5.152087in}{2.155407in}}%
\pgfpathlineto{\pgfqpoint{5.144250in}{2.145762in}}%
\pgfpathclose%
\pgfusepath{fill}%
\end{pgfscope}%
\begin{pgfscope}%
\pgfpathrectangle{\pgfqpoint{1.150000in}{0.150000in}}{\pgfqpoint{5.700000in}{5.700000in}}%
\pgfusepath{clip}%
\pgfsetbuttcap%
\pgfsetroundjoin%
\definecolor{currentfill}{rgb}{0.271305,0.019942,0.347269}%
\pgfsetfillcolor{currentfill}%
\pgfsetfillopacity{0.700000}%
\pgfsetlinewidth{0.000000pt}%
\definecolor{currentstroke}{rgb}{0.000000,0.000000,0.000000}%
\pgfsetstrokecolor{currentstroke}%
\pgfsetdash{}{0pt}%
\pgfpathmoveto{\pgfqpoint{3.673430in}{1.279403in}}%
\pgfpathlineto{\pgfqpoint{3.687577in}{1.276341in}}%
\pgfpathlineto{\pgfqpoint{3.701730in}{1.273351in}}%
\pgfpathlineto{\pgfqpoint{3.715890in}{1.270433in}}%
\pgfpathlineto{\pgfqpoint{3.730057in}{1.267588in}}%
\pgfpathlineto{\pgfqpoint{3.738417in}{1.276938in}}%
\pgfpathlineto{\pgfqpoint{3.746770in}{1.286412in}}%
\pgfpathlineto{\pgfqpoint{3.755117in}{1.296004in}}%
\pgfpathlineto{\pgfqpoint{3.763456in}{1.305709in}}%
\pgfpathlineto{\pgfqpoint{3.749304in}{1.308182in}}%
\pgfpathlineto{\pgfqpoint{3.735158in}{1.310727in}}%
\pgfpathlineto{\pgfqpoint{3.721020in}{1.313344in}}%
\pgfpathlineto{\pgfqpoint{3.706888in}{1.316034in}}%
\pgfpathlineto{\pgfqpoint{3.698534in}{1.306694in}}%
\pgfpathlineto{\pgfqpoint{3.690174in}{1.297471in}}%
\pgfpathlineto{\pgfqpoint{3.681806in}{1.288373in}}%
\pgfpathlineto{\pgfqpoint{3.673430in}{1.279403in}}%
\pgfpathclose%
\pgfusepath{fill}%
\end{pgfscope}%
\begin{pgfscope}%
\pgfpathrectangle{\pgfqpoint{1.150000in}{0.150000in}}{\pgfqpoint{5.700000in}{5.700000in}}%
\pgfusepath{clip}%
\pgfsetbuttcap%
\pgfsetroundjoin%
\definecolor{currentfill}{rgb}{0.171176,0.452530,0.557965}%
\pgfsetfillcolor{currentfill}%
\pgfsetfillopacity{0.700000}%
\pgfsetlinewidth{0.000000pt}%
\definecolor{currentstroke}{rgb}{0.000000,0.000000,0.000000}%
\pgfsetstrokecolor{currentstroke}%
\pgfsetdash{}{0pt}%
\pgfpathmoveto{\pgfqpoint{5.324431in}{2.261325in}}%
\pgfpathlineto{\pgfqpoint{5.339209in}{2.266953in}}%
\pgfpathlineto{\pgfqpoint{5.354001in}{2.272653in}}%
\pgfpathlineto{\pgfqpoint{5.368807in}{2.278425in}}%
\pgfpathlineto{\pgfqpoint{5.376547in}{2.286867in}}%
\pgfpathlineto{\pgfqpoint{5.384277in}{2.295152in}}%
\pgfpathlineto{\pgfqpoint{5.391997in}{2.303283in}}%
\pgfpathlineto{\pgfqpoint{5.399708in}{2.311259in}}%
\pgfpathlineto{\pgfqpoint{5.384912in}{2.305516in}}%
\pgfpathlineto{\pgfqpoint{5.370131in}{2.299844in}}%
\pgfpathlineto{\pgfqpoint{5.355364in}{2.294245in}}%
\pgfpathlineto{\pgfqpoint{5.347645in}{2.286241in}}%
\pgfpathlineto{\pgfqpoint{5.339916in}{2.278087in}}%
\pgfpathlineto{\pgfqpoint{5.332178in}{2.269782in}}%
\pgfpathlineto{\pgfqpoint{5.324431in}{2.261325in}}%
\pgfpathclose%
\pgfusepath{fill}%
\end{pgfscope}%
\begin{pgfscope}%
\pgfpathrectangle{\pgfqpoint{1.150000in}{0.150000in}}{\pgfqpoint{5.700000in}{5.700000in}}%
\pgfusepath{clip}%
\pgfsetbuttcap%
\pgfsetroundjoin%
\definecolor{currentfill}{rgb}{0.250425,0.274290,0.533103}%
\pgfsetfillcolor{currentfill}%
\pgfsetfillopacity{0.700000}%
\pgfsetlinewidth{0.000000pt}%
\definecolor{currentstroke}{rgb}{0.000000,0.000000,0.000000}%
\pgfsetstrokecolor{currentstroke}%
\pgfsetdash{}{0pt}%
\pgfpathmoveto{\pgfqpoint{4.662120in}{1.797287in}}%
\pgfpathlineto{\pgfqpoint{4.676589in}{1.800345in}}%
\pgfpathlineto{\pgfqpoint{4.691070in}{1.803474in}}%
\pgfpathlineto{\pgfqpoint{4.705562in}{1.806674in}}%
\pgfpathlineto{\pgfqpoint{4.720066in}{1.809945in}}%
\pgfpathlineto{\pgfqpoint{4.728099in}{1.822262in}}%
\pgfpathlineto{\pgfqpoint{4.736127in}{1.834478in}}%
\pgfpathlineto{\pgfqpoint{4.744148in}{1.846592in}}%
\pgfpathlineto{\pgfqpoint{4.752163in}{1.858600in}}%
\pgfpathlineto{\pgfqpoint{4.737664in}{1.855182in}}%
\pgfpathlineto{\pgfqpoint{4.723176in}{1.851834in}}%
\pgfpathlineto{\pgfqpoint{4.708700in}{1.848558in}}%
\pgfpathlineto{\pgfqpoint{4.694236in}{1.845353in}}%
\pgfpathlineto{\pgfqpoint{4.686216in}{1.833483in}}%
\pgfpathlineto{\pgfqpoint{4.678190in}{1.821515in}}%
\pgfpathlineto{\pgfqpoint{4.670158in}{1.809449in}}%
\pgfpathlineto{\pgfqpoint{4.662120in}{1.797287in}}%
\pgfpathclose%
\pgfusepath{fill}%
\end{pgfscope}%
\begin{pgfscope}%
\pgfpathrectangle{\pgfqpoint{1.150000in}{0.150000in}}{\pgfqpoint{5.700000in}{5.700000in}}%
\pgfusepath{clip}%
\pgfsetbuttcap%
\pgfsetroundjoin%
\definecolor{currentfill}{rgb}{0.179019,0.433756,0.557430}%
\pgfsetfillcolor{currentfill}%
\pgfsetfillopacity{0.700000}%
\pgfsetlinewidth{0.000000pt}%
\definecolor{currentstroke}{rgb}{0.000000,0.000000,0.000000}%
\pgfsetstrokecolor{currentstroke}%
\pgfsetdash{}{0pt}%
\pgfpathmoveto{\pgfqpoint{5.234337in}{2.204145in}}%
\pgfpathlineto{\pgfqpoint{5.249069in}{2.209492in}}%
\pgfpathlineto{\pgfqpoint{5.263815in}{2.214910in}}%
\pgfpathlineto{\pgfqpoint{5.278574in}{2.220401in}}%
\pgfpathlineto{\pgfqpoint{5.293348in}{2.225963in}}%
\pgfpathlineto{\pgfqpoint{5.301132in}{2.235035in}}%
\pgfpathlineto{\pgfqpoint{5.308908in}{2.243952in}}%
\pgfpathlineto{\pgfqpoint{5.316674in}{2.252716in}}%
\pgfpathlineto{\pgfqpoint{5.324431in}{2.261325in}}%
\pgfpathlineto{\pgfqpoint{5.309667in}{2.255769in}}%
\pgfpathlineto{\pgfqpoint{5.294916in}{2.250285in}}%
\pgfpathlineto{\pgfqpoint{5.280180in}{2.244872in}}%
\pgfpathlineto{\pgfqpoint{5.265458in}{2.239531in}}%
\pgfpathlineto{\pgfqpoint{5.257691in}{2.230907in}}%
\pgfpathlineto{\pgfqpoint{5.249916in}{2.222135in}}%
\pgfpathlineto{\pgfqpoint{5.242131in}{2.213214in}}%
\pgfpathlineto{\pgfqpoint{5.234337in}{2.204145in}}%
\pgfpathclose%
\pgfusepath{fill}%
\end{pgfscope}%
\begin{pgfscope}%
\pgfpathrectangle{\pgfqpoint{1.150000in}{0.150000in}}{\pgfqpoint{5.700000in}{5.700000in}}%
\pgfusepath{clip}%
\pgfsetbuttcap%
\pgfsetroundjoin%
\definecolor{currentfill}{rgb}{0.268510,0.009605,0.335427}%
\pgfsetfillcolor{currentfill}%
\pgfsetfillopacity{0.700000}%
\pgfsetlinewidth{0.000000pt}%
\definecolor{currentstroke}{rgb}{0.000000,0.000000,0.000000}%
\pgfsetstrokecolor{currentstroke}%
\pgfsetdash{}{0pt}%
\pgfpathmoveto{\pgfqpoint{3.436451in}{1.265145in}}%
\pgfpathlineto{\pgfqpoint{3.450561in}{1.260394in}}%
\pgfpathlineto{\pgfqpoint{3.464676in}{1.255716in}}%
\pgfpathlineto{\pgfqpoint{3.478797in}{1.251113in}}%
\pgfpathlineto{\pgfqpoint{3.492924in}{1.246585in}}%
\pgfpathlineto{\pgfqpoint{3.501401in}{1.253464in}}%
\pgfpathlineto{\pgfqpoint{3.509870in}{1.260532in}}%
\pgfpathlineto{\pgfqpoint{3.518329in}{1.267781in}}%
\pgfpathlineto{\pgfqpoint{3.526780in}{1.275205in}}%
\pgfpathlineto{\pgfqpoint{3.512673in}{1.279321in}}%
\pgfpathlineto{\pgfqpoint{3.498573in}{1.283510in}}%
\pgfpathlineto{\pgfqpoint{3.484478in}{1.287774in}}%
\pgfpathlineto{\pgfqpoint{3.470389in}{1.292112in}}%
\pgfpathlineto{\pgfqpoint{3.461919in}{1.285093in}}%
\pgfpathlineto{\pgfqpoint{3.453439in}{1.278255in}}%
\pgfpathlineto{\pgfqpoint{3.444950in}{1.271604in}}%
\pgfpathlineto{\pgfqpoint{3.436451in}{1.265145in}}%
\pgfpathclose%
\pgfusepath{fill}%
\end{pgfscope}%
\begin{pgfscope}%
\pgfpathrectangle{\pgfqpoint{1.150000in}{0.150000in}}{\pgfqpoint{5.700000in}{5.700000in}}%
\pgfusepath{clip}%
\pgfsetbuttcap%
\pgfsetroundjoin%
\definecolor{currentfill}{rgb}{0.237441,0.305202,0.541921}%
\pgfsetfillcolor{currentfill}%
\pgfsetfillopacity{0.700000}%
\pgfsetlinewidth{0.000000pt}%
\definecolor{currentstroke}{rgb}{0.000000,0.000000,0.000000}%
\pgfsetstrokecolor{currentstroke}%
\pgfsetdash{}{0pt}%
\pgfpathmoveto{\pgfqpoint{4.752163in}{1.858600in}}%
\pgfpathlineto{\pgfqpoint{4.766675in}{1.862090in}}%
\pgfpathlineto{\pgfqpoint{4.781198in}{1.865651in}}%
\pgfpathlineto{\pgfqpoint{4.795734in}{1.869283in}}%
\pgfpathlineto{\pgfqpoint{4.810282in}{1.872986in}}%
\pgfpathlineto{\pgfqpoint{4.818286in}{1.885023in}}%
\pgfpathlineto{\pgfqpoint{4.826283in}{1.896946in}}%
\pgfpathlineto{\pgfqpoint{4.834274in}{1.908754in}}%
\pgfpathlineto{\pgfqpoint{4.842259in}{1.920446in}}%
\pgfpathlineto{\pgfqpoint{4.827715in}{1.916617in}}%
\pgfpathlineto{\pgfqpoint{4.813184in}{1.912859in}}%
\pgfpathlineto{\pgfqpoint{4.798666in}{1.909172in}}%
\pgfpathlineto{\pgfqpoint{4.784159in}{1.905557in}}%
\pgfpathlineto{\pgfqpoint{4.776170in}{1.893983in}}%
\pgfpathlineto{\pgfqpoint{4.768174in}{1.882297in}}%
\pgfpathlineto{\pgfqpoint{4.760172in}{1.870503in}}%
\pgfpathlineto{\pgfqpoint{4.752163in}{1.858600in}}%
\pgfpathclose%
\pgfusepath{fill}%
\end{pgfscope}%
\begin{pgfscope}%
\pgfpathrectangle{\pgfqpoint{1.150000in}{0.150000in}}{\pgfqpoint{5.700000in}{5.700000in}}%
\pgfusepath{clip}%
\pgfsetbuttcap%
\pgfsetroundjoin%
\definecolor{currentfill}{rgb}{0.282656,0.100196,0.422160}%
\pgfsetfillcolor{currentfill}%
\pgfsetfillopacity{0.700000}%
\pgfsetlinewidth{0.000000pt}%
\definecolor{currentstroke}{rgb}{0.000000,0.000000,0.000000}%
\pgfsetstrokecolor{currentstroke}%
\pgfsetdash{}{0pt}%
\pgfpathmoveto{\pgfqpoint{4.089995in}{1.415212in}}%
\pgfpathlineto{\pgfqpoint{4.104254in}{1.414953in}}%
\pgfpathlineto{\pgfqpoint{4.118521in}{1.414765in}}%
\pgfpathlineto{\pgfqpoint{4.132798in}{1.414647in}}%
\pgfpathlineto{\pgfqpoint{4.147083in}{1.414601in}}%
\pgfpathlineto{\pgfqpoint{4.155296in}{1.426803in}}%
\pgfpathlineto{\pgfqpoint{4.163504in}{1.439024in}}%
\pgfpathlineto{\pgfqpoint{4.171707in}{1.451260in}}%
\pgfpathlineto{\pgfqpoint{4.179904in}{1.463505in}}%
\pgfpathlineto{\pgfqpoint{4.165626in}{1.463259in}}%
\pgfpathlineto{\pgfqpoint{4.151356in}{1.463084in}}%
\pgfpathlineto{\pgfqpoint{4.137096in}{1.462980in}}%
\pgfpathlineto{\pgfqpoint{4.122845in}{1.462947in}}%
\pgfpathlineto{\pgfqpoint{4.114641in}{1.450986in}}%
\pgfpathlineto{\pgfqpoint{4.106431in}{1.439041in}}%
\pgfpathlineto{\pgfqpoint{4.098216in}{1.427115in}}%
\pgfpathlineto{\pgfqpoint{4.089995in}{1.415212in}}%
\pgfpathclose%
\pgfusepath{fill}%
\end{pgfscope}%
\begin{pgfscope}%
\pgfpathrectangle{\pgfqpoint{1.150000in}{0.150000in}}{\pgfqpoint{5.700000in}{5.700000in}}%
\pgfusepath{clip}%
\pgfsetbuttcap%
\pgfsetroundjoin%
\definecolor{currentfill}{rgb}{0.280894,0.078907,0.402329}%
\pgfsetfillcolor{currentfill}%
\pgfsetfillopacity{0.700000}%
\pgfsetlinewidth{0.000000pt}%
\definecolor{currentstroke}{rgb}{0.000000,0.000000,0.000000}%
\pgfsetstrokecolor{currentstroke}%
\pgfsetdash{}{0pt}%
\pgfpathmoveto{\pgfqpoint{4.000081in}{1.370925in}}%
\pgfpathlineto{\pgfqpoint{4.014313in}{1.370069in}}%
\pgfpathlineto{\pgfqpoint{4.028553in}{1.369284in}}%
\pgfpathlineto{\pgfqpoint{4.042803in}{1.368571in}}%
\pgfpathlineto{\pgfqpoint{4.057061in}{1.367928in}}%
\pgfpathlineto{\pgfqpoint{4.065302in}{1.379691in}}%
\pgfpathlineto{\pgfqpoint{4.073539in}{1.391496in}}%
\pgfpathlineto{\pgfqpoint{4.081770in}{1.403338in}}%
\pgfpathlineto{\pgfqpoint{4.089995in}{1.415212in}}%
\pgfpathlineto{\pgfqpoint{4.075746in}{1.415542in}}%
\pgfpathlineto{\pgfqpoint{4.061505in}{1.415944in}}%
\pgfpathlineto{\pgfqpoint{4.047273in}{1.416416in}}%
\pgfpathlineto{\pgfqpoint{4.033050in}{1.416960in}}%
\pgfpathlineto{\pgfqpoint{4.024816in}{1.405390in}}%
\pgfpathlineto{\pgfqpoint{4.016576in}{1.393858in}}%
\pgfpathlineto{\pgfqpoint{4.008331in}{1.382368in}}%
\pgfpathlineto{\pgfqpoint{4.000081in}{1.370925in}}%
\pgfpathclose%
\pgfusepath{fill}%
\end{pgfscope}%
\begin{pgfscope}%
\pgfpathrectangle{\pgfqpoint{1.150000in}{0.150000in}}{\pgfqpoint{5.700000in}{5.700000in}}%
\pgfusepath{clip}%
\pgfsetbuttcap%
\pgfsetroundjoin%
\definecolor{currentfill}{rgb}{0.283187,0.125848,0.444960}%
\pgfsetfillcolor{currentfill}%
\pgfsetfillopacity{0.700000}%
\pgfsetlinewidth{0.000000pt}%
\definecolor{currentstroke}{rgb}{0.000000,0.000000,0.000000}%
\pgfsetstrokecolor{currentstroke}%
\pgfsetdash{}{0pt}%
\pgfpathmoveto{\pgfqpoint{4.179904in}{1.463505in}}%
\pgfpathlineto{\pgfqpoint{4.194192in}{1.463822in}}%
\pgfpathlineto{\pgfqpoint{4.208490in}{1.464210in}}%
\pgfpathlineto{\pgfqpoint{4.222797in}{1.464668in}}%
\pgfpathlineto{\pgfqpoint{4.237113in}{1.465197in}}%
\pgfpathlineto{\pgfqpoint{4.245299in}{1.477729in}}%
\pgfpathlineto{\pgfqpoint{4.253480in}{1.490257in}}%
\pgfpathlineto{\pgfqpoint{4.261656in}{1.502778in}}%
\pgfpathlineto{\pgfqpoint{4.269827in}{1.515288in}}%
\pgfpathlineto{\pgfqpoint{4.255516in}{1.514487in}}%
\pgfpathlineto{\pgfqpoint{4.241215in}{1.513756in}}%
\pgfpathlineto{\pgfqpoint{4.226924in}{1.513096in}}%
\pgfpathlineto{\pgfqpoint{4.212643in}{1.512508in}}%
\pgfpathlineto{\pgfqpoint{4.204466in}{1.500262in}}%
\pgfpathlineto{\pgfqpoint{4.196284in}{1.488010in}}%
\pgfpathlineto{\pgfqpoint{4.188097in}{1.475757in}}%
\pgfpathlineto{\pgfqpoint{4.179904in}{1.463505in}}%
\pgfpathclose%
\pgfusepath{fill}%
\end{pgfscope}%
\begin{pgfscope}%
\pgfpathrectangle{\pgfqpoint{1.150000in}{0.150000in}}{\pgfqpoint{5.700000in}{5.700000in}}%
\pgfusepath{clip}%
\pgfsetbuttcap%
\pgfsetroundjoin%
\definecolor{currentfill}{rgb}{0.268510,0.009605,0.335427}%
\pgfsetfillcolor{currentfill}%
\pgfsetfillopacity{0.700000}%
\pgfsetlinewidth{0.000000pt}%
\definecolor{currentstroke}{rgb}{0.000000,0.000000,0.000000}%
\pgfsetstrokecolor{currentstroke}%
\pgfsetdash{}{0pt}%
\pgfpathmoveto{\pgfqpoint{3.583266in}{1.259482in}}%
\pgfpathlineto{\pgfqpoint{3.597403in}{1.255735in}}%
\pgfpathlineto{\pgfqpoint{3.611547in}{1.252061in}}%
\pgfpathlineto{\pgfqpoint{3.625696in}{1.248461in}}%
\pgfpathlineto{\pgfqpoint{3.639852in}{1.244932in}}%
\pgfpathlineto{\pgfqpoint{3.648259in}{1.253327in}}%
\pgfpathlineto{\pgfqpoint{3.656657in}{1.261874in}}%
\pgfpathlineto{\pgfqpoint{3.665047in}{1.270568in}}%
\pgfpathlineto{\pgfqpoint{3.673430in}{1.279403in}}%
\pgfpathlineto{\pgfqpoint{3.659291in}{1.282538in}}%
\pgfpathlineto{\pgfqpoint{3.645158in}{1.285746in}}%
\pgfpathlineto{\pgfqpoint{3.631031in}{1.289027in}}%
\pgfpathlineto{\pgfqpoint{3.616911in}{1.292382in}}%
\pgfpathlineto{\pgfqpoint{3.608512in}{1.283932in}}%
\pgfpathlineto{\pgfqpoint{3.600105in}{1.275628in}}%
\pgfpathlineto{\pgfqpoint{3.591690in}{1.267476in}}%
\pgfpathlineto{\pgfqpoint{3.583266in}{1.259482in}}%
\pgfpathclose%
\pgfusepath{fill}%
\end{pgfscope}%
\begin{pgfscope}%
\pgfpathrectangle{\pgfqpoint{1.150000in}{0.150000in}}{\pgfqpoint{5.700000in}{5.700000in}}%
\pgfusepath{clip}%
\pgfsetbuttcap%
\pgfsetroundjoin%
\definecolor{currentfill}{rgb}{0.281887,0.150881,0.465405}%
\pgfsetfillcolor{currentfill}%
\pgfsetfillopacity{0.700000}%
\pgfsetlinewidth{0.000000pt}%
\definecolor{currentstroke}{rgb}{0.000000,0.000000,0.000000}%
\pgfsetstrokecolor{currentstroke}%
\pgfsetdash{}{0pt}%
\pgfpathmoveto{\pgfqpoint{4.269827in}{1.515288in}}%
\pgfpathlineto{\pgfqpoint{4.284147in}{1.516160in}}%
\pgfpathlineto{\pgfqpoint{4.298477in}{1.517103in}}%
\pgfpathlineto{\pgfqpoint{4.312817in}{1.518117in}}%
\pgfpathlineto{\pgfqpoint{4.327167in}{1.519201in}}%
\pgfpathlineto{\pgfqpoint{4.335327in}{1.531957in}}%
\pgfpathlineto{\pgfqpoint{4.343483in}{1.544688in}}%
\pgfpathlineto{\pgfqpoint{4.351633in}{1.557392in}}%
\pgfpathlineto{\pgfqpoint{4.359778in}{1.570066in}}%
\pgfpathlineto{\pgfqpoint{4.345433in}{1.568730in}}%
\pgfpathlineto{\pgfqpoint{4.331098in}{1.567464in}}%
\pgfpathlineto{\pgfqpoint{4.316773in}{1.566269in}}%
\pgfpathlineto{\pgfqpoint{4.302459in}{1.565146in}}%
\pgfpathlineto{\pgfqpoint{4.294308in}{1.552716in}}%
\pgfpathlineto{\pgfqpoint{4.286153in}{1.540261in}}%
\pgfpathlineto{\pgfqpoint{4.277992in}{1.527784in}}%
\pgfpathlineto{\pgfqpoint{4.269827in}{1.515288in}}%
\pgfpathclose%
\pgfusepath{fill}%
\end{pgfscope}%
\begin{pgfscope}%
\pgfpathrectangle{\pgfqpoint{1.150000in}{0.150000in}}{\pgfqpoint{5.700000in}{5.700000in}}%
\pgfusepath{clip}%
\pgfsetbuttcap%
\pgfsetroundjoin%
\definecolor{currentfill}{rgb}{0.277941,0.056324,0.381191}%
\pgfsetfillcolor{currentfill}%
\pgfsetfillopacity{0.700000}%
\pgfsetlinewidth{0.000000pt}%
\definecolor{currentstroke}{rgb}{0.000000,0.000000,0.000000}%
\pgfsetstrokecolor{currentstroke}%
\pgfsetdash{}{0pt}%
\pgfpathmoveto{\pgfqpoint{3.910138in}{1.331181in}}%
\pgfpathlineto{\pgfqpoint{3.924347in}{1.329708in}}%
\pgfpathlineto{\pgfqpoint{3.938564in}{1.328305in}}%
\pgfpathlineto{\pgfqpoint{3.952789in}{1.326975in}}%
\pgfpathlineto{\pgfqpoint{3.967023in}{1.325715in}}%
\pgfpathlineto{\pgfqpoint{3.975296in}{1.336923in}}%
\pgfpathlineto{\pgfqpoint{3.983563in}{1.348198in}}%
\pgfpathlineto{\pgfqpoint{3.991825in}{1.359533in}}%
\pgfpathlineto{\pgfqpoint{4.000081in}{1.370925in}}%
\pgfpathlineto{\pgfqpoint{3.985857in}{1.371852in}}%
\pgfpathlineto{\pgfqpoint{3.971642in}{1.372850in}}%
\pgfpathlineto{\pgfqpoint{3.957435in}{1.373920in}}%
\pgfpathlineto{\pgfqpoint{3.943236in}{1.375061in}}%
\pgfpathlineto{\pgfqpoint{3.934970in}{1.363994in}}%
\pgfpathlineto{\pgfqpoint{3.926699in}{1.352989in}}%
\pgfpathlineto{\pgfqpoint{3.918421in}{1.342049in}}%
\pgfpathlineto{\pgfqpoint{3.910138in}{1.331181in}}%
\pgfpathclose%
\pgfusepath{fill}%
\end{pgfscope}%
\begin{pgfscope}%
\pgfpathrectangle{\pgfqpoint{1.150000in}{0.150000in}}{\pgfqpoint{5.700000in}{5.700000in}}%
\pgfusepath{clip}%
\pgfsetbuttcap%
\pgfsetroundjoin%
\definecolor{currentfill}{rgb}{0.225863,0.330805,0.547314}%
\pgfsetfillcolor{currentfill}%
\pgfsetfillopacity{0.700000}%
\pgfsetlinewidth{0.000000pt}%
\definecolor{currentstroke}{rgb}{0.000000,0.000000,0.000000}%
\pgfsetstrokecolor{currentstroke}%
\pgfsetdash{}{0pt}%
\pgfpathmoveto{\pgfqpoint{4.842259in}{1.920446in}}%
\pgfpathlineto{\pgfqpoint{4.856814in}{1.924347in}}%
\pgfpathlineto{\pgfqpoint{4.871382in}{1.928318in}}%
\pgfpathlineto{\pgfqpoint{4.885963in}{1.932361in}}%
\pgfpathlineto{\pgfqpoint{4.900556in}{1.936476in}}%
\pgfpathlineto{\pgfqpoint{4.908528in}{1.948162in}}%
\pgfpathlineto{\pgfqpoint{4.916494in}{1.959725in}}%
\pgfpathlineto{\pgfqpoint{4.924452in}{1.971161in}}%
\pgfpathlineto{\pgfqpoint{4.932404in}{1.982470in}}%
\pgfpathlineto{\pgfqpoint{4.917816in}{1.978251in}}%
\pgfpathlineto{\pgfqpoint{4.903240in}{1.974104in}}%
\pgfpathlineto{\pgfqpoint{4.888678in}{1.970028in}}%
\pgfpathlineto{\pgfqpoint{4.874127in}{1.966023in}}%
\pgfpathlineto{\pgfqpoint{4.866170in}{1.954810in}}%
\pgfpathlineto{\pgfqpoint{4.858207in}{1.943475in}}%
\pgfpathlineto{\pgfqpoint{4.850236in}{1.932020in}}%
\pgfpathlineto{\pgfqpoint{4.842259in}{1.920446in}}%
\pgfpathclose%
\pgfusepath{fill}%
\end{pgfscope}%
\begin{pgfscope}%
\pgfpathrectangle{\pgfqpoint{1.150000in}{0.150000in}}{\pgfqpoint{5.700000in}{5.700000in}}%
\pgfusepath{clip}%
\pgfsetbuttcap%
\pgfsetroundjoin%
\definecolor{currentfill}{rgb}{0.278012,0.180367,0.486697}%
\pgfsetfillcolor{currentfill}%
\pgfsetfillopacity{0.700000}%
\pgfsetlinewidth{0.000000pt}%
\definecolor{currentstroke}{rgb}{0.000000,0.000000,0.000000}%
\pgfsetstrokecolor{currentstroke}%
\pgfsetdash{}{0pt}%
\pgfpathmoveto{\pgfqpoint{4.359778in}{1.570066in}}%
\pgfpathlineto{\pgfqpoint{4.374133in}{1.571473in}}%
\pgfpathlineto{\pgfqpoint{4.388498in}{1.572951in}}%
\pgfpathlineto{\pgfqpoint{4.402874in}{1.574500in}}%
\pgfpathlineto{\pgfqpoint{4.417260in}{1.576119in}}%
\pgfpathlineto{\pgfqpoint{4.425395in}{1.588998in}}%
\pgfpathlineto{\pgfqpoint{4.433525in}{1.601834in}}%
\pgfpathlineto{\pgfqpoint{4.441650in}{1.614624in}}%
\pgfpathlineto{\pgfqpoint{4.449770in}{1.627365in}}%
\pgfpathlineto{\pgfqpoint{4.435388in}{1.625514in}}%
\pgfpathlineto{\pgfqpoint{4.421017in}{1.623734in}}%
\pgfpathlineto{\pgfqpoint{4.406656in}{1.622025in}}%
\pgfpathlineto{\pgfqpoint{4.392306in}{1.620387in}}%
\pgfpathlineto{\pgfqpoint{4.384182in}{1.607869in}}%
\pgfpathlineto{\pgfqpoint{4.376052in}{1.595308in}}%
\pgfpathlineto{\pgfqpoint{4.367918in}{1.582706in}}%
\pgfpathlineto{\pgfqpoint{4.359778in}{1.570066in}}%
\pgfpathclose%
\pgfusepath{fill}%
\end{pgfscope}%
\begin{pgfscope}%
\pgfpathrectangle{\pgfqpoint{1.150000in}{0.150000in}}{\pgfqpoint{5.700000in}{5.700000in}}%
\pgfusepath{clip}%
\pgfsetbuttcap%
\pgfsetroundjoin%
\definecolor{currentfill}{rgb}{0.274952,0.037752,0.364543}%
\pgfsetfillcolor{currentfill}%
\pgfsetfillopacity{0.700000}%
\pgfsetlinewidth{0.000000pt}%
\definecolor{currentstroke}{rgb}{0.000000,0.000000,0.000000}%
\pgfsetstrokecolor{currentstroke}%
\pgfsetdash{}{0pt}%
\pgfpathmoveto{\pgfqpoint{3.820140in}{1.296541in}}%
\pgfpathlineto{\pgfqpoint{3.834330in}{1.294429in}}%
\pgfpathlineto{\pgfqpoint{3.848527in}{1.292388in}}%
\pgfpathlineto{\pgfqpoint{3.862732in}{1.290419in}}%
\pgfpathlineto{\pgfqpoint{3.876944in}{1.288521in}}%
\pgfpathlineto{\pgfqpoint{3.885252in}{1.299054in}}%
\pgfpathlineto{\pgfqpoint{3.893553in}{1.309678in}}%
\pgfpathlineto{\pgfqpoint{3.901849in}{1.320389in}}%
\pgfpathlineto{\pgfqpoint{3.910138in}{1.331181in}}%
\pgfpathlineto{\pgfqpoint{3.895937in}{1.332726in}}%
\pgfpathlineto{\pgfqpoint{3.881744in}{1.334342in}}%
\pgfpathlineto{\pgfqpoint{3.867559in}{1.336031in}}%
\pgfpathlineto{\pgfqpoint{3.853381in}{1.337790in}}%
\pgfpathlineto{\pgfqpoint{3.845080in}{1.327343in}}%
\pgfpathlineto{\pgfqpoint{3.836773in}{1.316982in}}%
\pgfpathlineto{\pgfqpoint{3.828460in}{1.306713in}}%
\pgfpathlineto{\pgfqpoint{3.820140in}{1.296541in}}%
\pgfpathclose%
\pgfusepath{fill}%
\end{pgfscope}%
\begin{pgfscope}%
\pgfpathrectangle{\pgfqpoint{1.150000in}{0.150000in}}{\pgfqpoint{5.700000in}{5.700000in}}%
\pgfusepath{clip}%
\pgfsetbuttcap%
\pgfsetroundjoin%
\definecolor{currentfill}{rgb}{0.271828,0.209303,0.504434}%
\pgfsetfillcolor{currentfill}%
\pgfsetfillopacity{0.700000}%
\pgfsetlinewidth{0.000000pt}%
\definecolor{currentstroke}{rgb}{0.000000,0.000000,0.000000}%
\pgfsetstrokecolor{currentstroke}%
\pgfsetdash{}{0pt}%
\pgfpathmoveto{\pgfqpoint{4.449770in}{1.627365in}}%
\pgfpathlineto{\pgfqpoint{4.464162in}{1.629286in}}%
\pgfpathlineto{\pgfqpoint{4.478565in}{1.631279in}}%
\pgfpathlineto{\pgfqpoint{4.492979in}{1.633342in}}%
\pgfpathlineto{\pgfqpoint{4.507404in}{1.635476in}}%
\pgfpathlineto{\pgfqpoint{4.515514in}{1.648383in}}%
\pgfpathlineto{\pgfqpoint{4.523618in}{1.661230in}}%
\pgfpathlineto{\pgfqpoint{4.531718in}{1.674013in}}%
\pgfpathlineto{\pgfqpoint{4.539811in}{1.686730in}}%
\pgfpathlineto{\pgfqpoint{4.525391in}{1.684385in}}%
\pgfpathlineto{\pgfqpoint{4.510981in}{1.682112in}}%
\pgfpathlineto{\pgfqpoint{4.496583in}{1.679908in}}%
\pgfpathlineto{\pgfqpoint{4.482195in}{1.677776in}}%
\pgfpathlineto{\pgfqpoint{4.474096in}{1.665262in}}%
\pgfpathlineto{\pgfqpoint{4.465993in}{1.652687in}}%
\pgfpathlineto{\pgfqpoint{4.457884in}{1.640053in}}%
\pgfpathlineto{\pgfqpoint{4.449770in}{1.627365in}}%
\pgfpathclose%
\pgfusepath{fill}%
\end{pgfscope}%
\begin{pgfscope}%
\pgfpathrectangle{\pgfqpoint{1.150000in}{0.150000in}}{\pgfqpoint{5.700000in}{5.700000in}}%
\pgfusepath{clip}%
\pgfsetbuttcap%
\pgfsetroundjoin%
\definecolor{currentfill}{rgb}{0.214298,0.355619,0.551184}%
\pgfsetfillcolor{currentfill}%
\pgfsetfillopacity{0.700000}%
\pgfsetlinewidth{0.000000pt}%
\definecolor{currentstroke}{rgb}{0.000000,0.000000,0.000000}%
\pgfsetstrokecolor{currentstroke}%
\pgfsetdash{}{0pt}%
\pgfpathmoveto{\pgfqpoint{4.932404in}{1.982470in}}%
\pgfpathlineto{\pgfqpoint{4.947004in}{1.986760in}}%
\pgfpathlineto{\pgfqpoint{4.961618in}{1.991122in}}%
\pgfpathlineto{\pgfqpoint{4.976244in}{1.995555in}}%
\pgfpathlineto{\pgfqpoint{4.990883in}{2.000059in}}%
\pgfpathlineto{\pgfqpoint{4.998822in}{2.011331in}}%
\pgfpathlineto{\pgfqpoint{5.006753in}{2.022469in}}%
\pgfpathlineto{\pgfqpoint{5.014677in}{2.033472in}}%
\pgfpathlineto{\pgfqpoint{5.022593in}{2.044338in}}%
\pgfpathlineto{\pgfqpoint{5.007960in}{2.039751in}}%
\pgfpathlineto{\pgfqpoint{4.993339in}{2.035235in}}%
\pgfpathlineto{\pgfqpoint{4.978731in}{2.030791in}}%
\pgfpathlineto{\pgfqpoint{4.964136in}{2.026418in}}%
\pgfpathlineto{\pgfqpoint{4.956214in}{2.015626in}}%
\pgfpathlineto{\pgfqpoint{4.948285in}{2.004704in}}%
\pgfpathlineto{\pgfqpoint{4.940348in}{1.993652in}}%
\pgfpathlineto{\pgfqpoint{4.932404in}{1.982470in}}%
\pgfpathclose%
\pgfusepath{fill}%
\end{pgfscope}%
\begin{pgfscope}%
\pgfpathrectangle{\pgfqpoint{1.150000in}{0.150000in}}{\pgfqpoint{5.700000in}{5.700000in}}%
\pgfusepath{clip}%
\pgfsetbuttcap%
\pgfsetroundjoin%
\definecolor{currentfill}{rgb}{0.272594,0.025563,0.353093}%
\pgfsetfillcolor{currentfill}%
\pgfsetfillopacity{0.700000}%
\pgfsetlinewidth{0.000000pt}%
\definecolor{currentstroke}{rgb}{0.000000,0.000000,0.000000}%
\pgfsetstrokecolor{currentstroke}%
\pgfsetdash{}{0pt}%
\pgfpathmoveto{\pgfqpoint{3.730057in}{1.267588in}}%
\pgfpathlineto{\pgfqpoint{3.744231in}{1.264816in}}%
\pgfpathlineto{\pgfqpoint{3.758412in}{1.262115in}}%
\pgfpathlineto{\pgfqpoint{3.772600in}{1.259486in}}%
\pgfpathlineto{\pgfqpoint{3.786796in}{1.256929in}}%
\pgfpathlineto{\pgfqpoint{3.795142in}{1.266659in}}%
\pgfpathlineto{\pgfqpoint{3.803481in}{1.276508in}}%
\pgfpathlineto{\pgfqpoint{3.811814in}{1.286471in}}%
\pgfpathlineto{\pgfqpoint{3.820140in}{1.296541in}}%
\pgfpathlineto{\pgfqpoint{3.805958in}{1.298725in}}%
\pgfpathlineto{\pgfqpoint{3.791784in}{1.300981in}}%
\pgfpathlineto{\pgfqpoint{3.777616in}{1.303309in}}%
\pgfpathlineto{\pgfqpoint{3.763456in}{1.305709in}}%
\pgfpathlineto{\pgfqpoint{3.755117in}{1.296004in}}%
\pgfpathlineto{\pgfqpoint{3.746770in}{1.286412in}}%
\pgfpathlineto{\pgfqpoint{3.738417in}{1.276938in}}%
\pgfpathlineto{\pgfqpoint{3.730057in}{1.267588in}}%
\pgfpathclose%
\pgfusepath{fill}%
\end{pgfscope}%
\begin{pgfscope}%
\pgfpathrectangle{\pgfqpoint{1.150000in}{0.150000in}}{\pgfqpoint{5.700000in}{5.700000in}}%
\pgfusepath{clip}%
\pgfsetbuttcap%
\pgfsetroundjoin%
\definecolor{currentfill}{rgb}{0.268510,0.009605,0.335427}%
\pgfsetfillcolor{currentfill}%
\pgfsetfillopacity{0.700000}%
\pgfsetlinewidth{0.000000pt}%
\definecolor{currentstroke}{rgb}{0.000000,0.000000,0.000000}%
\pgfsetstrokecolor{currentstroke}%
\pgfsetdash{}{0pt}%
\pgfpathmoveto{\pgfqpoint{3.492924in}{1.246585in}}%
\pgfpathlineto{\pgfqpoint{3.507056in}{1.242130in}}%
\pgfpathlineto{\pgfqpoint{3.521194in}{1.237749in}}%
\pgfpathlineto{\pgfqpoint{3.535338in}{1.233442in}}%
\pgfpathlineto{\pgfqpoint{3.549487in}{1.229208in}}%
\pgfpathlineto{\pgfqpoint{3.557945in}{1.236509in}}%
\pgfpathlineto{\pgfqpoint{3.566394in}{1.243992in}}%
\pgfpathlineto{\pgfqpoint{3.574835in}{1.251652in}}%
\pgfpathlineto{\pgfqpoint{3.583266in}{1.259482in}}%
\pgfpathlineto{\pgfqpoint{3.569136in}{1.263303in}}%
\pgfpathlineto{\pgfqpoint{3.555011in}{1.267196in}}%
\pgfpathlineto{\pgfqpoint{3.540892in}{1.271164in}}%
\pgfpathlineto{\pgfqpoint{3.526780in}{1.275205in}}%
\pgfpathlineto{\pgfqpoint{3.518329in}{1.267781in}}%
\pgfpathlineto{\pgfqpoint{3.509870in}{1.260532in}}%
\pgfpathlineto{\pgfqpoint{3.501401in}{1.253464in}}%
\pgfpathlineto{\pgfqpoint{3.492924in}{1.246585in}}%
\pgfpathclose%
\pgfusepath{fill}%
\end{pgfscope}%
\begin{pgfscope}%
\pgfpathrectangle{\pgfqpoint{1.150000in}{0.150000in}}{\pgfqpoint{5.700000in}{5.700000in}}%
\pgfusepath{clip}%
\pgfsetbuttcap%
\pgfsetroundjoin%
\definecolor{currentfill}{rgb}{0.263663,0.237631,0.518762}%
\pgfsetfillcolor{currentfill}%
\pgfsetfillopacity{0.700000}%
\pgfsetlinewidth{0.000000pt}%
\definecolor{currentstroke}{rgb}{0.000000,0.000000,0.000000}%
\pgfsetstrokecolor{currentstroke}%
\pgfsetdash{}{0pt}%
\pgfpathmoveto{\pgfqpoint{4.539811in}{1.686730in}}%
\pgfpathlineto{\pgfqpoint{4.554243in}{1.689146in}}%
\pgfpathlineto{\pgfqpoint{4.568686in}{1.691632in}}%
\pgfpathlineto{\pgfqpoint{4.583140in}{1.694190in}}%
\pgfpathlineto{\pgfqpoint{4.597605in}{1.696818in}}%
\pgfpathlineto{\pgfqpoint{4.605690in}{1.709664in}}%
\pgfpathlineto{\pgfqpoint{4.613768in}{1.722433in}}%
\pgfpathlineto{\pgfqpoint{4.621842in}{1.735122in}}%
\pgfpathlineto{\pgfqpoint{4.629909in}{1.747729in}}%
\pgfpathlineto{\pgfqpoint{4.615448in}{1.744910in}}%
\pgfpathlineto{\pgfqpoint{4.600998in}{1.742163in}}%
\pgfpathlineto{\pgfqpoint{4.586559in}{1.739487in}}%
\pgfpathlineto{\pgfqpoint{4.572131in}{1.736881in}}%
\pgfpathlineto{\pgfqpoint{4.564060in}{1.724456in}}%
\pgfpathlineto{\pgfqpoint{4.555983in}{1.711954in}}%
\pgfpathlineto{\pgfqpoint{4.547900in}{1.699378in}}%
\pgfpathlineto{\pgfqpoint{4.539811in}{1.686730in}}%
\pgfpathclose%
\pgfusepath{fill}%
\end{pgfscope}%
\begin{pgfscope}%
\pgfpathrectangle{\pgfqpoint{1.150000in}{0.150000in}}{\pgfqpoint{5.700000in}{5.700000in}}%
\pgfusepath{clip}%
\pgfsetbuttcap%
\pgfsetroundjoin%
\definecolor{currentfill}{rgb}{0.201239,0.383670,0.554294}%
\pgfsetfillcolor{currentfill}%
\pgfsetfillopacity{0.700000}%
\pgfsetlinewidth{0.000000pt}%
\definecolor{currentstroke}{rgb}{0.000000,0.000000,0.000000}%
\pgfsetstrokecolor{currentstroke}%
\pgfsetdash{}{0pt}%
\pgfpathmoveto{\pgfqpoint{5.022593in}{2.044338in}}%
\pgfpathlineto{\pgfqpoint{5.037240in}{2.048997in}}%
\pgfpathlineto{\pgfqpoint{5.051899in}{2.053727in}}%
\pgfpathlineto{\pgfqpoint{5.066572in}{2.058528in}}%
\pgfpathlineto{\pgfqpoint{5.081258in}{2.063401in}}%
\pgfpathlineto{\pgfqpoint{5.089161in}{2.074200in}}%
\pgfpathlineto{\pgfqpoint{5.097055in}{2.084855in}}%
\pgfpathlineto{\pgfqpoint{5.104942in}{2.095367in}}%
\pgfpathlineto{\pgfqpoint{5.112820in}{2.105735in}}%
\pgfpathlineto{\pgfqpoint{5.098140in}{2.100801in}}%
\pgfpathlineto{\pgfqpoint{5.083474in}{2.095939in}}%
\pgfpathlineto{\pgfqpoint{5.068820in}{2.091148in}}%
\pgfpathlineto{\pgfqpoint{5.054180in}{2.086429in}}%
\pgfpathlineto{\pgfqpoint{5.046295in}{2.076113in}}%
\pgfpathlineto{\pgfqpoint{5.038402in}{2.065659in}}%
\pgfpathlineto{\pgfqpoint{5.030502in}{2.055067in}}%
\pgfpathlineto{\pgfqpoint{5.022593in}{2.044338in}}%
\pgfpathclose%
\pgfusepath{fill}%
\end{pgfscope}%
\begin{pgfscope}%
\pgfpathrectangle{\pgfqpoint{1.150000in}{0.150000in}}{\pgfqpoint{5.700000in}{5.700000in}}%
\pgfusepath{clip}%
\pgfsetbuttcap%
\pgfsetroundjoin%
\definecolor{currentfill}{rgb}{0.253935,0.265254,0.529983}%
\pgfsetfillcolor{currentfill}%
\pgfsetfillopacity{0.700000}%
\pgfsetlinewidth{0.000000pt}%
\definecolor{currentstroke}{rgb}{0.000000,0.000000,0.000000}%
\pgfsetstrokecolor{currentstroke}%
\pgfsetdash{}{0pt}%
\pgfpathmoveto{\pgfqpoint{4.629909in}{1.747729in}}%
\pgfpathlineto{\pgfqpoint{4.644382in}{1.750618in}}%
\pgfpathlineto{\pgfqpoint{4.658866in}{1.753578in}}%
\pgfpathlineto{\pgfqpoint{4.673362in}{1.756609in}}%
\pgfpathlineto{\pgfqpoint{4.687870in}{1.759711in}}%
\pgfpathlineto{\pgfqpoint{4.695928in}{1.772410in}}%
\pgfpathlineto{\pgfqpoint{4.703980in}{1.785017in}}%
\pgfpathlineto{\pgfqpoint{4.712026in}{1.797530in}}%
\pgfpathlineto{\pgfqpoint{4.720066in}{1.809945in}}%
\pgfpathlineto{\pgfqpoint{4.705562in}{1.806674in}}%
\pgfpathlineto{\pgfqpoint{4.691070in}{1.803474in}}%
\pgfpathlineto{\pgfqpoint{4.676589in}{1.800345in}}%
\pgfpathlineto{\pgfqpoint{4.662120in}{1.797287in}}%
\pgfpathlineto{\pgfqpoint{4.654076in}{1.785033in}}%
\pgfpathlineto{\pgfqpoint{4.646026in}{1.772686in}}%
\pgfpathlineto{\pgfqpoint{4.637971in}{1.760251in}}%
\pgfpathlineto{\pgfqpoint{4.629909in}{1.747729in}}%
\pgfpathclose%
\pgfusepath{fill}%
\end{pgfscope}%
\begin{pgfscope}%
\pgfpathrectangle{\pgfqpoint{1.150000in}{0.150000in}}{\pgfqpoint{5.700000in}{5.700000in}}%
\pgfusepath{clip}%
\pgfsetbuttcap%
\pgfsetroundjoin%
\definecolor{currentfill}{rgb}{0.190631,0.407061,0.556089}%
\pgfsetfillcolor{currentfill}%
\pgfsetfillopacity{0.700000}%
\pgfsetlinewidth{0.000000pt}%
\definecolor{currentstroke}{rgb}{0.000000,0.000000,0.000000}%
\pgfsetstrokecolor{currentstroke}%
\pgfsetdash{}{0pt}%
\pgfpathmoveto{\pgfqpoint{5.112820in}{2.105735in}}%
\pgfpathlineto{\pgfqpoint{5.127513in}{2.110741in}}%
\pgfpathlineto{\pgfqpoint{5.142220in}{2.115818in}}%
\pgfpathlineto{\pgfqpoint{5.156940in}{2.120966in}}%
\pgfpathlineto{\pgfqpoint{5.171673in}{2.126187in}}%
\pgfpathlineto{\pgfqpoint{5.179536in}{2.136458in}}%
\pgfpathlineto{\pgfqpoint{5.187391in}{2.146578in}}%
\pgfpathlineto{\pgfqpoint{5.195237in}{2.156548in}}%
\pgfpathlineto{\pgfqpoint{5.203075in}{2.166368in}}%
\pgfpathlineto{\pgfqpoint{5.188348in}{2.161109in}}%
\pgfpathlineto{\pgfqpoint{5.173635in}{2.155922in}}%
\pgfpathlineto{\pgfqpoint{5.158936in}{2.150806in}}%
\pgfpathlineto{\pgfqpoint{5.144250in}{2.145762in}}%
\pgfpathlineto{\pgfqpoint{5.136405in}{2.135972in}}%
\pgfpathlineto{\pgfqpoint{5.128552in}{2.126038in}}%
\pgfpathlineto{\pgfqpoint{5.120690in}{2.115959in}}%
\pgfpathlineto{\pgfqpoint{5.112820in}{2.105735in}}%
\pgfpathclose%
\pgfusepath{fill}%
\end{pgfscope}%
\begin{pgfscope}%
\pgfpathrectangle{\pgfqpoint{1.150000in}{0.150000in}}{\pgfqpoint{5.700000in}{5.700000in}}%
\pgfusepath{clip}%
\pgfsetbuttcap%
\pgfsetroundjoin%
\definecolor{currentfill}{rgb}{0.269944,0.014625,0.341379}%
\pgfsetfillcolor{currentfill}%
\pgfsetfillopacity{0.700000}%
\pgfsetlinewidth{0.000000pt}%
\definecolor{currentstroke}{rgb}{0.000000,0.000000,0.000000}%
\pgfsetstrokecolor{currentstroke}%
\pgfsetdash{}{0pt}%
\pgfpathmoveto{\pgfqpoint{3.639852in}{1.244932in}}%
\pgfpathlineto{\pgfqpoint{3.654015in}{1.241477in}}%
\pgfpathlineto{\pgfqpoint{3.668184in}{1.238094in}}%
\pgfpathlineto{\pgfqpoint{3.682360in}{1.234784in}}%
\pgfpathlineto{\pgfqpoint{3.696543in}{1.231545in}}%
\pgfpathlineto{\pgfqpoint{3.704932in}{1.240341in}}%
\pgfpathlineto{\pgfqpoint{3.713315in}{1.249284in}}%
\pgfpathlineto{\pgfqpoint{3.721689in}{1.258368in}}%
\pgfpathlineto{\pgfqpoint{3.730057in}{1.267588in}}%
\pgfpathlineto{\pgfqpoint{3.715890in}{1.270433in}}%
\pgfpathlineto{\pgfqpoint{3.701730in}{1.273351in}}%
\pgfpathlineto{\pgfqpoint{3.687577in}{1.276341in}}%
\pgfpathlineto{\pgfqpoint{3.673430in}{1.279403in}}%
\pgfpathlineto{\pgfqpoint{3.665047in}{1.270568in}}%
\pgfpathlineto{\pgfqpoint{3.656657in}{1.261874in}}%
\pgfpathlineto{\pgfqpoint{3.648259in}{1.253327in}}%
\pgfpathlineto{\pgfqpoint{3.639852in}{1.244932in}}%
\pgfpathclose%
\pgfusepath{fill}%
\end{pgfscope}%
\begin{pgfscope}%
\pgfpathrectangle{\pgfqpoint{1.150000in}{0.150000in}}{\pgfqpoint{5.700000in}{5.700000in}}%
\pgfusepath{clip}%
\pgfsetbuttcap%
\pgfsetroundjoin%
\definecolor{currentfill}{rgb}{0.172719,0.448791,0.557885}%
\pgfsetfillcolor{currentfill}%
\pgfsetfillopacity{0.700000}%
\pgfsetlinewidth{0.000000pt}%
\definecolor{currentstroke}{rgb}{0.000000,0.000000,0.000000}%
\pgfsetstrokecolor{currentstroke}%
\pgfsetdash{}{0pt}%
\pgfpathmoveto{\pgfqpoint{5.293348in}{2.225963in}}%
\pgfpathlineto{\pgfqpoint{5.308135in}{2.231597in}}%
\pgfpathlineto{\pgfqpoint{5.322937in}{2.237303in}}%
\pgfpathlineto{\pgfqpoint{5.337752in}{2.243081in}}%
\pgfpathlineto{\pgfqpoint{5.345530in}{2.252155in}}%
\pgfpathlineto{\pgfqpoint{5.353299in}{2.261070in}}%
\pgfpathlineto{\pgfqpoint{5.361057in}{2.269826in}}%
\pgfpathlineto{\pgfqpoint{5.368807in}{2.278425in}}%
\pgfpathlineto{\pgfqpoint{5.354001in}{2.272653in}}%
\pgfpathlineto{\pgfqpoint{5.339209in}{2.266953in}}%
\pgfpathlineto{\pgfqpoint{5.324431in}{2.261325in}}%
\pgfpathlineto{\pgfqpoint{5.316674in}{2.252716in}}%
\pgfpathlineto{\pgfqpoint{5.308908in}{2.243952in}}%
\pgfpathlineto{\pgfqpoint{5.301132in}{2.235035in}}%
\pgfpathlineto{\pgfqpoint{5.293348in}{2.225963in}}%
\pgfpathclose%
\pgfusepath{fill}%
\end{pgfscope}%
\begin{pgfscope}%
\pgfpathrectangle{\pgfqpoint{1.150000in}{0.150000in}}{\pgfqpoint{5.700000in}{5.700000in}}%
\pgfusepath{clip}%
\pgfsetbuttcap%
\pgfsetroundjoin%
\definecolor{currentfill}{rgb}{0.180629,0.429975,0.557282}%
\pgfsetfillcolor{currentfill}%
\pgfsetfillopacity{0.700000}%
\pgfsetlinewidth{0.000000pt}%
\definecolor{currentstroke}{rgb}{0.000000,0.000000,0.000000}%
\pgfsetstrokecolor{currentstroke}%
\pgfsetdash{}{0pt}%
\pgfpathmoveto{\pgfqpoint{5.203075in}{2.166368in}}%
\pgfpathlineto{\pgfqpoint{5.217815in}{2.171699in}}%
\pgfpathlineto{\pgfqpoint{5.232569in}{2.177101in}}%
\pgfpathlineto{\pgfqpoint{5.247337in}{2.182576in}}%
\pgfpathlineto{\pgfqpoint{5.262118in}{2.188122in}}%
\pgfpathlineto{\pgfqpoint{5.269939in}{2.197816in}}%
\pgfpathlineto{\pgfqpoint{5.277751in}{2.207354in}}%
\pgfpathlineto{\pgfqpoint{5.285554in}{2.216736in}}%
\pgfpathlineto{\pgfqpoint{5.293348in}{2.225963in}}%
\pgfpathlineto{\pgfqpoint{5.278574in}{2.220401in}}%
\pgfpathlineto{\pgfqpoint{5.263815in}{2.214910in}}%
\pgfpathlineto{\pgfqpoint{5.249069in}{2.209492in}}%
\pgfpathlineto{\pgfqpoint{5.234337in}{2.204145in}}%
\pgfpathlineto{\pgfqpoint{5.226535in}{2.194925in}}%
\pgfpathlineto{\pgfqpoint{5.218724in}{2.185557in}}%
\pgfpathlineto{\pgfqpoint{5.210904in}{2.176037in}}%
\pgfpathlineto{\pgfqpoint{5.203075in}{2.166368in}}%
\pgfpathclose%
\pgfusepath{fill}%
\end{pgfscope}%
\begin{pgfscope}%
\pgfpathrectangle{\pgfqpoint{1.150000in}{0.150000in}}{\pgfqpoint{5.700000in}{5.700000in}}%
\pgfusepath{clip}%
\pgfsetbuttcap%
\pgfsetroundjoin%
\definecolor{currentfill}{rgb}{0.243113,0.292092,0.538516}%
\pgfsetfillcolor{currentfill}%
\pgfsetfillopacity{0.700000}%
\pgfsetlinewidth{0.000000pt}%
\definecolor{currentstroke}{rgb}{0.000000,0.000000,0.000000}%
\pgfsetstrokecolor{currentstroke}%
\pgfsetdash{}{0pt}%
\pgfpathmoveto{\pgfqpoint{4.720066in}{1.809945in}}%
\pgfpathlineto{\pgfqpoint{4.734581in}{1.813287in}}%
\pgfpathlineto{\pgfqpoint{4.749109in}{1.816700in}}%
\pgfpathlineto{\pgfqpoint{4.763649in}{1.820185in}}%
\pgfpathlineto{\pgfqpoint{4.778200in}{1.823740in}}%
\pgfpathlineto{\pgfqpoint{4.786230in}{1.836212in}}%
\pgfpathlineto{\pgfqpoint{4.794254in}{1.848579in}}%
\pgfpathlineto{\pgfqpoint{4.802271in}{1.860838in}}%
\pgfpathlineto{\pgfqpoint{4.810282in}{1.872986in}}%
\pgfpathlineto{\pgfqpoint{4.795734in}{1.869283in}}%
\pgfpathlineto{\pgfqpoint{4.781198in}{1.865651in}}%
\pgfpathlineto{\pgfqpoint{4.766675in}{1.862090in}}%
\pgfpathlineto{\pgfqpoint{4.752163in}{1.858600in}}%
\pgfpathlineto{\pgfqpoint{4.744148in}{1.846592in}}%
\pgfpathlineto{\pgfqpoint{4.736127in}{1.834478in}}%
\pgfpathlineto{\pgfqpoint{4.728099in}{1.822262in}}%
\pgfpathlineto{\pgfqpoint{4.720066in}{1.809945in}}%
\pgfpathclose%
\pgfusepath{fill}%
\end{pgfscope}%
\begin{pgfscope}%
\pgfpathrectangle{\pgfqpoint{1.150000in}{0.150000in}}{\pgfqpoint{5.700000in}{5.700000in}}%
\pgfusepath{clip}%
\pgfsetbuttcap%
\pgfsetroundjoin%
\definecolor{currentfill}{rgb}{0.281924,0.089666,0.412415}%
\pgfsetfillcolor{currentfill}%
\pgfsetfillopacity{0.700000}%
\pgfsetlinewidth{0.000000pt}%
\definecolor{currentstroke}{rgb}{0.000000,0.000000,0.000000}%
\pgfsetstrokecolor{currentstroke}%
\pgfsetdash{}{0pt}%
\pgfpathmoveto{\pgfqpoint{4.057061in}{1.367928in}}%
\pgfpathlineto{\pgfqpoint{4.071327in}{1.367356in}}%
\pgfpathlineto{\pgfqpoint{4.085603in}{1.366855in}}%
\pgfpathlineto{\pgfqpoint{4.099887in}{1.366424in}}%
\pgfpathlineto{\pgfqpoint{4.114180in}{1.366065in}}%
\pgfpathlineto{\pgfqpoint{4.122414in}{1.378149in}}%
\pgfpathlineto{\pgfqpoint{4.130642in}{1.390269in}}%
\pgfpathlineto{\pgfqpoint{4.138865in}{1.402421in}}%
\pgfpathlineto{\pgfqpoint{4.147083in}{1.414601in}}%
\pgfpathlineto{\pgfqpoint{4.132798in}{1.414647in}}%
\pgfpathlineto{\pgfqpoint{4.118521in}{1.414765in}}%
\pgfpathlineto{\pgfqpoint{4.104254in}{1.414953in}}%
\pgfpathlineto{\pgfqpoint{4.089995in}{1.415212in}}%
\pgfpathlineto{\pgfqpoint{4.081770in}{1.403338in}}%
\pgfpathlineto{\pgfqpoint{4.073539in}{1.391496in}}%
\pgfpathlineto{\pgfqpoint{4.065302in}{1.379691in}}%
\pgfpathlineto{\pgfqpoint{4.057061in}{1.367928in}}%
\pgfpathclose%
\pgfusepath{fill}%
\end{pgfscope}%
\begin{pgfscope}%
\pgfpathrectangle{\pgfqpoint{1.150000in}{0.150000in}}{\pgfqpoint{5.700000in}{5.700000in}}%
\pgfusepath{clip}%
\pgfsetbuttcap%
\pgfsetroundjoin%
\definecolor{currentfill}{rgb}{0.283197,0.115680,0.436115}%
\pgfsetfillcolor{currentfill}%
\pgfsetfillopacity{0.700000}%
\pgfsetlinewidth{0.000000pt}%
\definecolor{currentstroke}{rgb}{0.000000,0.000000,0.000000}%
\pgfsetstrokecolor{currentstroke}%
\pgfsetdash{}{0pt}%
\pgfpathmoveto{\pgfqpoint{4.147083in}{1.414601in}}%
\pgfpathlineto{\pgfqpoint{4.161378in}{1.414625in}}%
\pgfpathlineto{\pgfqpoint{4.175682in}{1.414720in}}%
\pgfpathlineto{\pgfqpoint{4.189996in}{1.414885in}}%
\pgfpathlineto{\pgfqpoint{4.204319in}{1.415121in}}%
\pgfpathlineto{\pgfqpoint{4.212525in}{1.427625in}}%
\pgfpathlineto{\pgfqpoint{4.220726in}{1.440141in}}%
\pgfpathlineto{\pgfqpoint{4.228922in}{1.452667in}}%
\pgfpathlineto{\pgfqpoint{4.237113in}{1.465197in}}%
\pgfpathlineto{\pgfqpoint{4.222797in}{1.464668in}}%
\pgfpathlineto{\pgfqpoint{4.208490in}{1.464210in}}%
\pgfpathlineto{\pgfqpoint{4.194192in}{1.463822in}}%
\pgfpathlineto{\pgfqpoint{4.179904in}{1.463505in}}%
\pgfpathlineto{\pgfqpoint{4.171707in}{1.451260in}}%
\pgfpathlineto{\pgfqpoint{4.163504in}{1.439024in}}%
\pgfpathlineto{\pgfqpoint{4.155296in}{1.426803in}}%
\pgfpathlineto{\pgfqpoint{4.147083in}{1.414601in}}%
\pgfpathclose%
\pgfusepath{fill}%
\end{pgfscope}%
\begin{pgfscope}%
\pgfpathrectangle{\pgfqpoint{1.150000in}{0.150000in}}{\pgfqpoint{5.700000in}{5.700000in}}%
\pgfusepath{clip}%
\pgfsetbuttcap%
\pgfsetroundjoin%
\definecolor{currentfill}{rgb}{0.279566,0.067836,0.391917}%
\pgfsetfillcolor{currentfill}%
\pgfsetfillopacity{0.700000}%
\pgfsetlinewidth{0.000000pt}%
\definecolor{currentstroke}{rgb}{0.000000,0.000000,0.000000}%
\pgfsetstrokecolor{currentstroke}%
\pgfsetdash{}{0pt}%
\pgfpathmoveto{\pgfqpoint{3.967023in}{1.325715in}}%
\pgfpathlineto{\pgfqpoint{3.981264in}{1.324526in}}%
\pgfpathlineto{\pgfqpoint{3.995514in}{1.323408in}}%
\pgfpathlineto{\pgfqpoint{4.009773in}{1.322362in}}%
\pgfpathlineto{\pgfqpoint{4.024040in}{1.321386in}}%
\pgfpathlineto{\pgfqpoint{4.032303in}{1.332935in}}%
\pgfpathlineto{\pgfqpoint{4.040561in}{1.344545in}}%
\pgfpathlineto{\pgfqpoint{4.048814in}{1.356211in}}%
\pgfpathlineto{\pgfqpoint{4.057061in}{1.367928in}}%
\pgfpathlineto{\pgfqpoint{4.042803in}{1.368571in}}%
\pgfpathlineto{\pgfqpoint{4.028553in}{1.369284in}}%
\pgfpathlineto{\pgfqpoint{4.014313in}{1.370069in}}%
\pgfpathlineto{\pgfqpoint{4.000081in}{1.370925in}}%
\pgfpathlineto{\pgfqpoint{3.991825in}{1.359533in}}%
\pgfpathlineto{\pgfqpoint{3.983563in}{1.348198in}}%
\pgfpathlineto{\pgfqpoint{3.975296in}{1.336923in}}%
\pgfpathlineto{\pgfqpoint{3.967023in}{1.325715in}}%
\pgfpathclose%
\pgfusepath{fill}%
\end{pgfscope}%
\begin{pgfscope}%
\pgfpathrectangle{\pgfqpoint{1.150000in}{0.150000in}}{\pgfqpoint{5.700000in}{5.700000in}}%
\pgfusepath{clip}%
\pgfsetbuttcap%
\pgfsetroundjoin%
\definecolor{currentfill}{rgb}{0.282623,0.140926,0.457517}%
\pgfsetfillcolor{currentfill}%
\pgfsetfillopacity{0.700000}%
\pgfsetlinewidth{0.000000pt}%
\definecolor{currentstroke}{rgb}{0.000000,0.000000,0.000000}%
\pgfsetstrokecolor{currentstroke}%
\pgfsetdash{}{0pt}%
\pgfpathmoveto{\pgfqpoint{4.237113in}{1.465197in}}%
\pgfpathlineto{\pgfqpoint{4.251439in}{1.465797in}}%
\pgfpathlineto{\pgfqpoint{4.265775in}{1.466467in}}%
\pgfpathlineto{\pgfqpoint{4.280121in}{1.467208in}}%
\pgfpathlineto{\pgfqpoint{4.294476in}{1.468020in}}%
\pgfpathlineto{\pgfqpoint{4.302656in}{1.480832in}}%
\pgfpathlineto{\pgfqpoint{4.310832in}{1.493635in}}%
\pgfpathlineto{\pgfqpoint{4.319002in}{1.506426in}}%
\pgfpathlineto{\pgfqpoint{4.327167in}{1.519201in}}%
\pgfpathlineto{\pgfqpoint{4.312817in}{1.518117in}}%
\pgfpathlineto{\pgfqpoint{4.298477in}{1.517103in}}%
\pgfpathlineto{\pgfqpoint{4.284147in}{1.516160in}}%
\pgfpathlineto{\pgfqpoint{4.269827in}{1.515288in}}%
\pgfpathlineto{\pgfqpoint{4.261656in}{1.502778in}}%
\pgfpathlineto{\pgfqpoint{4.253480in}{1.490257in}}%
\pgfpathlineto{\pgfqpoint{4.245299in}{1.477729in}}%
\pgfpathlineto{\pgfqpoint{4.237113in}{1.465197in}}%
\pgfpathclose%
\pgfusepath{fill}%
\end{pgfscope}%
\begin{pgfscope}%
\pgfpathrectangle{\pgfqpoint{1.150000in}{0.150000in}}{\pgfqpoint{5.700000in}{5.700000in}}%
\pgfusepath{clip}%
\pgfsetbuttcap%
\pgfsetroundjoin%
\definecolor{currentfill}{rgb}{0.280255,0.165693,0.476498}%
\pgfsetfillcolor{currentfill}%
\pgfsetfillopacity{0.700000}%
\pgfsetlinewidth{0.000000pt}%
\definecolor{currentstroke}{rgb}{0.000000,0.000000,0.000000}%
\pgfsetstrokecolor{currentstroke}%
\pgfsetdash{}{0pt}%
\pgfpathmoveto{\pgfqpoint{4.327167in}{1.519201in}}%
\pgfpathlineto{\pgfqpoint{4.341527in}{1.520356in}}%
\pgfpathlineto{\pgfqpoint{4.355898in}{1.521582in}}%
\pgfpathlineto{\pgfqpoint{4.370278in}{1.522878in}}%
\pgfpathlineto{\pgfqpoint{4.384669in}{1.524245in}}%
\pgfpathlineto{\pgfqpoint{4.392825in}{1.537260in}}%
\pgfpathlineto{\pgfqpoint{4.400975in}{1.550247in}}%
\pgfpathlineto{\pgfqpoint{4.409120in}{1.563201in}}%
\pgfpathlineto{\pgfqpoint{4.417260in}{1.576119in}}%
\pgfpathlineto{\pgfqpoint{4.402874in}{1.574500in}}%
\pgfpathlineto{\pgfqpoint{4.388498in}{1.572951in}}%
\pgfpathlineto{\pgfqpoint{4.374133in}{1.571473in}}%
\pgfpathlineto{\pgfqpoint{4.359778in}{1.570066in}}%
\pgfpathlineto{\pgfqpoint{4.351633in}{1.557392in}}%
\pgfpathlineto{\pgfqpoint{4.343483in}{1.544688in}}%
\pgfpathlineto{\pgfqpoint{4.335327in}{1.531957in}}%
\pgfpathlineto{\pgfqpoint{4.327167in}{1.519201in}}%
\pgfpathclose%
\pgfusepath{fill}%
\end{pgfscope}%
\begin{pgfscope}%
\pgfpathrectangle{\pgfqpoint{1.150000in}{0.150000in}}{\pgfqpoint{5.700000in}{5.700000in}}%
\pgfusepath{clip}%
\pgfsetbuttcap%
\pgfsetroundjoin%
\definecolor{currentfill}{rgb}{0.276022,0.044167,0.370164}%
\pgfsetfillcolor{currentfill}%
\pgfsetfillopacity{0.700000}%
\pgfsetlinewidth{0.000000pt}%
\definecolor{currentstroke}{rgb}{0.000000,0.000000,0.000000}%
\pgfsetstrokecolor{currentstroke}%
\pgfsetdash{}{0pt}%
\pgfpathmoveto{\pgfqpoint{3.876944in}{1.288521in}}%
\pgfpathlineto{\pgfqpoint{3.891165in}{1.286695in}}%
\pgfpathlineto{\pgfqpoint{3.905393in}{1.284939in}}%
\pgfpathlineto{\pgfqpoint{3.919629in}{1.283255in}}%
\pgfpathlineto{\pgfqpoint{3.933873in}{1.281642in}}%
\pgfpathlineto{\pgfqpoint{3.942169in}{1.292536in}}%
\pgfpathlineto{\pgfqpoint{3.950460in}{1.303516in}}%
\pgfpathlineto{\pgfqpoint{3.958744in}{1.314577in}}%
\pgfpathlineto{\pgfqpoint{3.967023in}{1.325715in}}%
\pgfpathlineto{\pgfqpoint{3.952789in}{1.326975in}}%
\pgfpathlineto{\pgfqpoint{3.938564in}{1.328305in}}%
\pgfpathlineto{\pgfqpoint{3.924347in}{1.329708in}}%
\pgfpathlineto{\pgfqpoint{3.910138in}{1.331181in}}%
\pgfpathlineto{\pgfqpoint{3.901849in}{1.320389in}}%
\pgfpathlineto{\pgfqpoint{3.893553in}{1.309678in}}%
\pgfpathlineto{\pgfqpoint{3.885252in}{1.299054in}}%
\pgfpathlineto{\pgfqpoint{3.876944in}{1.288521in}}%
\pgfpathclose%
\pgfusepath{fill}%
\end{pgfscope}%
\begin{pgfscope}%
\pgfpathrectangle{\pgfqpoint{1.150000in}{0.150000in}}{\pgfqpoint{5.700000in}{5.700000in}}%
\pgfusepath{clip}%
\pgfsetbuttcap%
\pgfsetroundjoin%
\definecolor{currentfill}{rgb}{0.229739,0.322361,0.545706}%
\pgfsetfillcolor{currentfill}%
\pgfsetfillopacity{0.700000}%
\pgfsetlinewidth{0.000000pt}%
\definecolor{currentstroke}{rgb}{0.000000,0.000000,0.000000}%
\pgfsetstrokecolor{currentstroke}%
\pgfsetdash{}{0pt}%
\pgfpathmoveto{\pgfqpoint{4.810282in}{1.872986in}}%
\pgfpathlineto{\pgfqpoint{4.824842in}{1.876760in}}%
\pgfpathlineto{\pgfqpoint{4.839414in}{1.880605in}}%
\pgfpathlineto{\pgfqpoint{4.853999in}{1.884522in}}%
\pgfpathlineto{\pgfqpoint{4.868596in}{1.888510in}}%
\pgfpathlineto{\pgfqpoint{4.876596in}{1.900681in}}%
\pgfpathlineto{\pgfqpoint{4.884589in}{1.912733in}}%
\pgfpathlineto{\pgfqpoint{4.892576in}{1.924665in}}%
\pgfpathlineto{\pgfqpoint{4.900556in}{1.936476in}}%
\pgfpathlineto{\pgfqpoint{4.885963in}{1.932361in}}%
\pgfpathlineto{\pgfqpoint{4.871382in}{1.928318in}}%
\pgfpathlineto{\pgfqpoint{4.856814in}{1.924347in}}%
\pgfpathlineto{\pgfqpoint{4.842259in}{1.920446in}}%
\pgfpathlineto{\pgfqpoint{4.834274in}{1.908754in}}%
\pgfpathlineto{\pgfqpoint{4.826283in}{1.896946in}}%
\pgfpathlineto{\pgfqpoint{4.818286in}{1.885023in}}%
\pgfpathlineto{\pgfqpoint{4.810282in}{1.872986in}}%
\pgfpathclose%
\pgfusepath{fill}%
\end{pgfscope}%
\begin{pgfscope}%
\pgfpathrectangle{\pgfqpoint{1.150000in}{0.150000in}}{\pgfqpoint{5.700000in}{5.700000in}}%
\pgfusepath{clip}%
\pgfsetbuttcap%
\pgfsetroundjoin%
\definecolor{currentfill}{rgb}{0.268510,0.009605,0.335427}%
\pgfsetfillcolor{currentfill}%
\pgfsetfillopacity{0.700000}%
\pgfsetlinewidth{0.000000pt}%
\definecolor{currentstroke}{rgb}{0.000000,0.000000,0.000000}%
\pgfsetstrokecolor{currentstroke}%
\pgfsetdash{}{0pt}%
\pgfpathmoveto{\pgfqpoint{3.549487in}{1.229208in}}%
\pgfpathlineto{\pgfqpoint{3.563643in}{1.225047in}}%
\pgfpathlineto{\pgfqpoint{3.577805in}{1.220960in}}%
\pgfpathlineto{\pgfqpoint{3.591973in}{1.216945in}}%
\pgfpathlineto{\pgfqpoint{3.606147in}{1.213003in}}%
\pgfpathlineto{\pgfqpoint{3.614586in}{1.220725in}}%
\pgfpathlineto{\pgfqpoint{3.623016in}{1.228625in}}%
\pgfpathlineto{\pgfqpoint{3.631438in}{1.236696in}}%
\pgfpathlineto{\pgfqpoint{3.639852in}{1.244932in}}%
\pgfpathlineto{\pgfqpoint{3.625696in}{1.248461in}}%
\pgfpathlineto{\pgfqpoint{3.611547in}{1.252061in}}%
\pgfpathlineto{\pgfqpoint{3.597403in}{1.255735in}}%
\pgfpathlineto{\pgfqpoint{3.583266in}{1.259482in}}%
\pgfpathlineto{\pgfqpoint{3.574835in}{1.251652in}}%
\pgfpathlineto{\pgfqpoint{3.566394in}{1.243992in}}%
\pgfpathlineto{\pgfqpoint{3.557945in}{1.236509in}}%
\pgfpathlineto{\pgfqpoint{3.549487in}{1.229208in}}%
\pgfpathclose%
\pgfusepath{fill}%
\end{pgfscope}%
\begin{pgfscope}%
\pgfpathrectangle{\pgfqpoint{1.150000in}{0.150000in}}{\pgfqpoint{5.700000in}{5.700000in}}%
\pgfusepath{clip}%
\pgfsetbuttcap%
\pgfsetroundjoin%
\definecolor{currentfill}{rgb}{0.275191,0.194905,0.496005}%
\pgfsetfillcolor{currentfill}%
\pgfsetfillopacity{0.700000}%
\pgfsetlinewidth{0.000000pt}%
\definecolor{currentstroke}{rgb}{0.000000,0.000000,0.000000}%
\pgfsetstrokecolor{currentstroke}%
\pgfsetdash{}{0pt}%
\pgfpathmoveto{\pgfqpoint{4.417260in}{1.576119in}}%
\pgfpathlineto{\pgfqpoint{4.431657in}{1.577809in}}%
\pgfpathlineto{\pgfqpoint{4.446065in}{1.579569in}}%
\pgfpathlineto{\pgfqpoint{4.460483in}{1.581401in}}%
\pgfpathlineto{\pgfqpoint{4.474911in}{1.583302in}}%
\pgfpathlineto{\pgfqpoint{4.483042in}{1.596421in}}%
\pgfpathlineto{\pgfqpoint{4.491168in}{1.609492in}}%
\pgfpathlineto{\pgfqpoint{4.499289in}{1.622511in}}%
\pgfpathlineto{\pgfqpoint{4.507404in}{1.635476in}}%
\pgfpathlineto{\pgfqpoint{4.492979in}{1.633342in}}%
\pgfpathlineto{\pgfqpoint{4.478565in}{1.631279in}}%
\pgfpathlineto{\pgfqpoint{4.464162in}{1.629286in}}%
\pgfpathlineto{\pgfqpoint{4.449770in}{1.627365in}}%
\pgfpathlineto{\pgfqpoint{4.441650in}{1.614624in}}%
\pgfpathlineto{\pgfqpoint{4.433525in}{1.601834in}}%
\pgfpathlineto{\pgfqpoint{4.425395in}{1.588998in}}%
\pgfpathlineto{\pgfqpoint{4.417260in}{1.576119in}}%
\pgfpathclose%
\pgfusepath{fill}%
\end{pgfscope}%
\begin{pgfscope}%
\pgfpathrectangle{\pgfqpoint{1.150000in}{0.150000in}}{\pgfqpoint{5.700000in}{5.700000in}}%
\pgfusepath{clip}%
\pgfsetbuttcap%
\pgfsetroundjoin%
\definecolor{currentfill}{rgb}{0.273809,0.031497,0.358853}%
\pgfsetfillcolor{currentfill}%
\pgfsetfillopacity{0.700000}%
\pgfsetlinewidth{0.000000pt}%
\definecolor{currentstroke}{rgb}{0.000000,0.000000,0.000000}%
\pgfsetstrokecolor{currentstroke}%
\pgfsetdash{}{0pt}%
\pgfpathmoveto{\pgfqpoint{3.786796in}{1.256929in}}%
\pgfpathlineto{\pgfqpoint{3.800998in}{1.254443in}}%
\pgfpathlineto{\pgfqpoint{3.815209in}{1.252029in}}%
\pgfpathlineto{\pgfqpoint{3.829426in}{1.249687in}}%
\pgfpathlineto{\pgfqpoint{3.843651in}{1.247416in}}%
\pgfpathlineto{\pgfqpoint{3.851984in}{1.257527in}}%
\pgfpathlineto{\pgfqpoint{3.860310in}{1.267752in}}%
\pgfpathlineto{\pgfqpoint{3.868631in}{1.278085in}}%
\pgfpathlineto{\pgfqpoint{3.876944in}{1.288521in}}%
\pgfpathlineto{\pgfqpoint{3.862732in}{1.290419in}}%
\pgfpathlineto{\pgfqpoint{3.848527in}{1.292388in}}%
\pgfpathlineto{\pgfqpoint{3.834330in}{1.294429in}}%
\pgfpathlineto{\pgfqpoint{3.820140in}{1.296541in}}%
\pgfpathlineto{\pgfqpoint{3.811814in}{1.286471in}}%
\pgfpathlineto{\pgfqpoint{3.803481in}{1.276508in}}%
\pgfpathlineto{\pgfqpoint{3.795142in}{1.266659in}}%
\pgfpathlineto{\pgfqpoint{3.786796in}{1.256929in}}%
\pgfpathclose%
\pgfusepath{fill}%
\end{pgfscope}%
\begin{pgfscope}%
\pgfpathrectangle{\pgfqpoint{1.150000in}{0.150000in}}{\pgfqpoint{5.700000in}{5.700000in}}%
\pgfusepath{clip}%
\pgfsetbuttcap%
\pgfsetroundjoin%
\definecolor{currentfill}{rgb}{0.218130,0.347432,0.550038}%
\pgfsetfillcolor{currentfill}%
\pgfsetfillopacity{0.700000}%
\pgfsetlinewidth{0.000000pt}%
\definecolor{currentstroke}{rgb}{0.000000,0.000000,0.000000}%
\pgfsetstrokecolor{currentstroke}%
\pgfsetdash{}{0pt}%
\pgfpathmoveto{\pgfqpoint{4.900556in}{1.936476in}}%
\pgfpathlineto{\pgfqpoint{4.915161in}{1.940661in}}%
\pgfpathlineto{\pgfqpoint{4.929779in}{1.944918in}}%
\pgfpathlineto{\pgfqpoint{4.944410in}{1.949246in}}%
\pgfpathlineto{\pgfqpoint{4.959054in}{1.953645in}}%
\pgfpathlineto{\pgfqpoint{4.967022in}{1.965445in}}%
\pgfpathlineto{\pgfqpoint{4.974983in}{1.977115in}}%
\pgfpathlineto{\pgfqpoint{4.982937in}{1.988653in}}%
\pgfpathlineto{\pgfqpoint{4.990883in}{2.000059in}}%
\pgfpathlineto{\pgfqpoint{4.976244in}{1.995555in}}%
\pgfpathlineto{\pgfqpoint{4.961618in}{1.991122in}}%
\pgfpathlineto{\pgfqpoint{4.947004in}{1.986760in}}%
\pgfpathlineto{\pgfqpoint{4.932404in}{1.982470in}}%
\pgfpathlineto{\pgfqpoint{4.924452in}{1.971161in}}%
\pgfpathlineto{\pgfqpoint{4.916494in}{1.959725in}}%
\pgfpathlineto{\pgfqpoint{4.908528in}{1.948162in}}%
\pgfpathlineto{\pgfqpoint{4.900556in}{1.936476in}}%
\pgfpathclose%
\pgfusepath{fill}%
\end{pgfscope}%
\begin{pgfscope}%
\pgfpathrectangle{\pgfqpoint{1.150000in}{0.150000in}}{\pgfqpoint{5.700000in}{5.700000in}}%
\pgfusepath{clip}%
\pgfsetbuttcap%
\pgfsetroundjoin%
\definecolor{currentfill}{rgb}{0.267968,0.223549,0.512008}%
\pgfsetfillcolor{currentfill}%
\pgfsetfillopacity{0.700000}%
\pgfsetlinewidth{0.000000pt}%
\definecolor{currentstroke}{rgb}{0.000000,0.000000,0.000000}%
\pgfsetstrokecolor{currentstroke}%
\pgfsetdash{}{0pt}%
\pgfpathmoveto{\pgfqpoint{4.507404in}{1.635476in}}%
\pgfpathlineto{\pgfqpoint{4.521840in}{1.637680in}}%
\pgfpathlineto{\pgfqpoint{4.536286in}{1.639956in}}%
\pgfpathlineto{\pgfqpoint{4.550744in}{1.642301in}}%
\pgfpathlineto{\pgfqpoint{4.565213in}{1.644718in}}%
\pgfpathlineto{\pgfqpoint{4.573319in}{1.657845in}}%
\pgfpathlineto{\pgfqpoint{4.581420in}{1.670905in}}%
\pgfpathlineto{\pgfqpoint{4.589515in}{1.683897in}}%
\pgfpathlineto{\pgfqpoint{4.597605in}{1.696818in}}%
\pgfpathlineto{\pgfqpoint{4.583140in}{1.694190in}}%
\pgfpathlineto{\pgfqpoint{4.568686in}{1.691632in}}%
\pgfpathlineto{\pgfqpoint{4.554243in}{1.689146in}}%
\pgfpathlineto{\pgfqpoint{4.539811in}{1.686730in}}%
\pgfpathlineto{\pgfqpoint{4.531718in}{1.674013in}}%
\pgfpathlineto{\pgfqpoint{4.523618in}{1.661230in}}%
\pgfpathlineto{\pgfqpoint{4.515514in}{1.648383in}}%
\pgfpathlineto{\pgfqpoint{4.507404in}{1.635476in}}%
\pgfpathclose%
\pgfusepath{fill}%
\end{pgfscope}%
\begin{pgfscope}%
\pgfpathrectangle{\pgfqpoint{1.150000in}{0.150000in}}{\pgfqpoint{5.700000in}{5.700000in}}%
\pgfusepath{clip}%
\pgfsetbuttcap%
\pgfsetroundjoin%
\definecolor{currentfill}{rgb}{0.204903,0.375746,0.553533}%
\pgfsetfillcolor{currentfill}%
\pgfsetfillopacity{0.700000}%
\pgfsetlinewidth{0.000000pt}%
\definecolor{currentstroke}{rgb}{0.000000,0.000000,0.000000}%
\pgfsetstrokecolor{currentstroke}%
\pgfsetdash{}{0pt}%
\pgfpathmoveto{\pgfqpoint{4.990883in}{2.000059in}}%
\pgfpathlineto{\pgfqpoint{5.005535in}{2.004635in}}%
\pgfpathlineto{\pgfqpoint{5.020200in}{2.009282in}}%
\pgfpathlineto{\pgfqpoint{5.034879in}{2.014000in}}%
\pgfpathlineto{\pgfqpoint{5.049570in}{2.018790in}}%
\pgfpathlineto{\pgfqpoint{5.057504in}{2.030154in}}%
\pgfpathlineto{\pgfqpoint{5.065430in}{2.041377in}}%
\pgfpathlineto{\pgfqpoint{5.073348in}{2.052460in}}%
\pgfpathlineto{\pgfqpoint{5.081258in}{2.063401in}}%
\pgfpathlineto{\pgfqpoint{5.066572in}{2.058528in}}%
\pgfpathlineto{\pgfqpoint{5.051899in}{2.053727in}}%
\pgfpathlineto{\pgfqpoint{5.037240in}{2.048997in}}%
\pgfpathlineto{\pgfqpoint{5.022593in}{2.044338in}}%
\pgfpathlineto{\pgfqpoint{5.014677in}{2.033472in}}%
\pgfpathlineto{\pgfqpoint{5.006753in}{2.022469in}}%
\pgfpathlineto{\pgfqpoint{4.998822in}{2.011331in}}%
\pgfpathlineto{\pgfqpoint{4.990883in}{2.000059in}}%
\pgfpathclose%
\pgfusepath{fill}%
\end{pgfscope}%
\begin{pgfscope}%
\pgfpathrectangle{\pgfqpoint{1.150000in}{0.150000in}}{\pgfqpoint{5.700000in}{5.700000in}}%
\pgfusepath{clip}%
\pgfsetbuttcap%
\pgfsetroundjoin%
\definecolor{currentfill}{rgb}{0.271305,0.019942,0.347269}%
\pgfsetfillcolor{currentfill}%
\pgfsetfillopacity{0.700000}%
\pgfsetlinewidth{0.000000pt}%
\definecolor{currentstroke}{rgb}{0.000000,0.000000,0.000000}%
\pgfsetstrokecolor{currentstroke}%
\pgfsetdash{}{0pt}%
\pgfpathmoveto{\pgfqpoint{3.696543in}{1.231545in}}%
\pgfpathlineto{\pgfqpoint{3.710732in}{1.228379in}}%
\pgfpathlineto{\pgfqpoint{3.724928in}{1.225285in}}%
\pgfpathlineto{\pgfqpoint{3.739131in}{1.222262in}}%
\pgfpathlineto{\pgfqpoint{3.753341in}{1.219311in}}%
\pgfpathlineto{\pgfqpoint{3.761716in}{1.228508in}}%
\pgfpathlineto{\pgfqpoint{3.770083in}{1.237848in}}%
\pgfpathlineto{\pgfqpoint{3.778443in}{1.247323in}}%
\pgfpathlineto{\pgfqpoint{3.786796in}{1.256929in}}%
\pgfpathlineto{\pgfqpoint{3.772600in}{1.259486in}}%
\pgfpathlineto{\pgfqpoint{3.758412in}{1.262115in}}%
\pgfpathlineto{\pgfqpoint{3.744231in}{1.264816in}}%
\pgfpathlineto{\pgfqpoint{3.730057in}{1.267588in}}%
\pgfpathlineto{\pgfqpoint{3.721689in}{1.258368in}}%
\pgfpathlineto{\pgfqpoint{3.713315in}{1.249284in}}%
\pgfpathlineto{\pgfqpoint{3.704932in}{1.240341in}}%
\pgfpathlineto{\pgfqpoint{3.696543in}{1.231545in}}%
\pgfpathclose%
\pgfusepath{fill}%
\end{pgfscope}%
\begin{pgfscope}%
\pgfpathrectangle{\pgfqpoint{1.150000in}{0.150000in}}{\pgfqpoint{5.700000in}{5.700000in}}%
\pgfusepath{clip}%
\pgfsetbuttcap%
\pgfsetroundjoin%
\definecolor{currentfill}{rgb}{0.258965,0.251537,0.524736}%
\pgfsetfillcolor{currentfill}%
\pgfsetfillopacity{0.700000}%
\pgfsetlinewidth{0.000000pt}%
\definecolor{currentstroke}{rgb}{0.000000,0.000000,0.000000}%
\pgfsetstrokecolor{currentstroke}%
\pgfsetdash{}{0pt}%
\pgfpathmoveto{\pgfqpoint{4.597605in}{1.696818in}}%
\pgfpathlineto{\pgfqpoint{4.612082in}{1.699517in}}%
\pgfpathlineto{\pgfqpoint{4.626570in}{1.702286in}}%
\pgfpathlineto{\pgfqpoint{4.641070in}{1.705127in}}%
\pgfpathlineto{\pgfqpoint{4.655581in}{1.708038in}}%
\pgfpathlineto{\pgfqpoint{4.663662in}{1.721082in}}%
\pgfpathlineto{\pgfqpoint{4.671737in}{1.734044in}}%
\pgfpathlineto{\pgfqpoint{4.679807in}{1.746921in}}%
\pgfpathlineto{\pgfqpoint{4.687870in}{1.759711in}}%
\pgfpathlineto{\pgfqpoint{4.673362in}{1.756609in}}%
\pgfpathlineto{\pgfqpoint{4.658866in}{1.753578in}}%
\pgfpathlineto{\pgfqpoint{4.644382in}{1.750618in}}%
\pgfpathlineto{\pgfqpoint{4.629909in}{1.747729in}}%
\pgfpathlineto{\pgfqpoint{4.621842in}{1.735122in}}%
\pgfpathlineto{\pgfqpoint{4.613768in}{1.722433in}}%
\pgfpathlineto{\pgfqpoint{4.605690in}{1.709664in}}%
\pgfpathlineto{\pgfqpoint{4.597605in}{1.696818in}}%
\pgfpathclose%
\pgfusepath{fill}%
\end{pgfscope}%
\begin{pgfscope}%
\pgfpathrectangle{\pgfqpoint{1.150000in}{0.150000in}}{\pgfqpoint{5.700000in}{5.700000in}}%
\pgfusepath{clip}%
\pgfsetbuttcap%
\pgfsetroundjoin%
\definecolor{currentfill}{rgb}{0.194100,0.399323,0.555565}%
\pgfsetfillcolor{currentfill}%
\pgfsetfillopacity{0.700000}%
\pgfsetlinewidth{0.000000pt}%
\definecolor{currentstroke}{rgb}{0.000000,0.000000,0.000000}%
\pgfsetstrokecolor{currentstroke}%
\pgfsetdash{}{0pt}%
\pgfpathmoveto{\pgfqpoint{5.081258in}{2.063401in}}%
\pgfpathlineto{\pgfqpoint{5.095958in}{2.068346in}}%
\pgfpathlineto{\pgfqpoint{5.110670in}{2.073362in}}%
\pgfpathlineto{\pgfqpoint{5.125397in}{2.078450in}}%
\pgfpathlineto{\pgfqpoint{5.140136in}{2.083609in}}%
\pgfpathlineto{\pgfqpoint{5.148033in}{2.094477in}}%
\pgfpathlineto{\pgfqpoint{5.155921in}{2.105196in}}%
\pgfpathlineto{\pgfqpoint{5.163802in}{2.115766in}}%
\pgfpathlineto{\pgfqpoint{5.171673in}{2.126187in}}%
\pgfpathlineto{\pgfqpoint{5.156940in}{2.120966in}}%
\pgfpathlineto{\pgfqpoint{5.142220in}{2.115818in}}%
\pgfpathlineto{\pgfqpoint{5.127513in}{2.110741in}}%
\pgfpathlineto{\pgfqpoint{5.112820in}{2.105735in}}%
\pgfpathlineto{\pgfqpoint{5.104942in}{2.095367in}}%
\pgfpathlineto{\pgfqpoint{5.097055in}{2.084855in}}%
\pgfpathlineto{\pgfqpoint{5.089161in}{2.074200in}}%
\pgfpathlineto{\pgfqpoint{5.081258in}{2.063401in}}%
\pgfpathclose%
\pgfusepath{fill}%
\end{pgfscope}%
\begin{pgfscope}%
\pgfpathrectangle{\pgfqpoint{1.150000in}{0.150000in}}{\pgfqpoint{5.700000in}{5.700000in}}%
\pgfusepath{clip}%
\pgfsetbuttcap%
\pgfsetroundjoin%
\definecolor{currentfill}{rgb}{0.246811,0.283237,0.535941}%
\pgfsetfillcolor{currentfill}%
\pgfsetfillopacity{0.700000}%
\pgfsetlinewidth{0.000000pt}%
\definecolor{currentstroke}{rgb}{0.000000,0.000000,0.000000}%
\pgfsetstrokecolor{currentstroke}%
\pgfsetdash{}{0pt}%
\pgfpathmoveto{\pgfqpoint{4.687870in}{1.759711in}}%
\pgfpathlineto{\pgfqpoint{4.702390in}{1.762883in}}%
\pgfpathlineto{\pgfqpoint{4.716921in}{1.766127in}}%
\pgfpathlineto{\pgfqpoint{4.731464in}{1.769442in}}%
\pgfpathlineto{\pgfqpoint{4.746019in}{1.772827in}}%
\pgfpathlineto{\pgfqpoint{4.754074in}{1.785704in}}%
\pgfpathlineto{\pgfqpoint{4.762122in}{1.798483in}}%
\pgfpathlineto{\pgfqpoint{4.770164in}{1.811163in}}%
\pgfpathlineto{\pgfqpoint{4.778200in}{1.823740in}}%
\pgfpathlineto{\pgfqpoint{4.763649in}{1.820185in}}%
\pgfpathlineto{\pgfqpoint{4.749109in}{1.816700in}}%
\pgfpathlineto{\pgfqpoint{4.734581in}{1.813287in}}%
\pgfpathlineto{\pgfqpoint{4.720066in}{1.809945in}}%
\pgfpathlineto{\pgfqpoint{4.712026in}{1.797530in}}%
\pgfpathlineto{\pgfqpoint{4.703980in}{1.785017in}}%
\pgfpathlineto{\pgfqpoint{4.695928in}{1.772410in}}%
\pgfpathlineto{\pgfqpoint{4.687870in}{1.759711in}}%
\pgfpathclose%
\pgfusepath{fill}%
\end{pgfscope}%
\begin{pgfscope}%
\pgfpathrectangle{\pgfqpoint{1.150000in}{0.150000in}}{\pgfqpoint{5.700000in}{5.700000in}}%
\pgfusepath{clip}%
\pgfsetbuttcap%
\pgfsetroundjoin%
\definecolor{currentfill}{rgb}{0.282656,0.100196,0.422160}%
\pgfsetfillcolor{currentfill}%
\pgfsetfillopacity{0.700000}%
\pgfsetlinewidth{0.000000pt}%
\definecolor{currentstroke}{rgb}{0.000000,0.000000,0.000000}%
\pgfsetstrokecolor{currentstroke}%
\pgfsetdash{}{0pt}%
\pgfpathmoveto{\pgfqpoint{4.114180in}{1.366065in}}%
\pgfpathlineto{\pgfqpoint{4.128482in}{1.365775in}}%
\pgfpathlineto{\pgfqpoint{4.142794in}{1.365557in}}%
\pgfpathlineto{\pgfqpoint{4.157114in}{1.365408in}}%
\pgfpathlineto{\pgfqpoint{4.171444in}{1.365330in}}%
\pgfpathlineto{\pgfqpoint{4.179670in}{1.377736in}}%
\pgfpathlineto{\pgfqpoint{4.187891in}{1.390173in}}%
\pgfpathlineto{\pgfqpoint{4.196108in}{1.402636in}}%
\pgfpathlineto{\pgfqpoint{4.204319in}{1.415121in}}%
\pgfpathlineto{\pgfqpoint{4.189996in}{1.414885in}}%
\pgfpathlineto{\pgfqpoint{4.175682in}{1.414720in}}%
\pgfpathlineto{\pgfqpoint{4.161378in}{1.414625in}}%
\pgfpathlineto{\pgfqpoint{4.147083in}{1.414601in}}%
\pgfpathlineto{\pgfqpoint{4.138865in}{1.402421in}}%
\pgfpathlineto{\pgfqpoint{4.130642in}{1.390269in}}%
\pgfpathlineto{\pgfqpoint{4.122414in}{1.378149in}}%
\pgfpathlineto{\pgfqpoint{4.114180in}{1.366065in}}%
\pgfpathclose%
\pgfusepath{fill}%
\end{pgfscope}%
\begin{pgfscope}%
\pgfpathrectangle{\pgfqpoint{1.150000in}{0.150000in}}{\pgfqpoint{5.700000in}{5.700000in}}%
\pgfusepath{clip}%
\pgfsetbuttcap%
\pgfsetroundjoin%
\definecolor{currentfill}{rgb}{0.280894,0.078907,0.402329}%
\pgfsetfillcolor{currentfill}%
\pgfsetfillopacity{0.700000}%
\pgfsetlinewidth{0.000000pt}%
\definecolor{currentstroke}{rgb}{0.000000,0.000000,0.000000}%
\pgfsetstrokecolor{currentstroke}%
\pgfsetdash{}{0pt}%
\pgfpathmoveto{\pgfqpoint{4.024040in}{1.321386in}}%
\pgfpathlineto{\pgfqpoint{4.038315in}{1.320480in}}%
\pgfpathlineto{\pgfqpoint{4.052599in}{1.319645in}}%
\pgfpathlineto{\pgfqpoint{4.066892in}{1.318881in}}%
\pgfpathlineto{\pgfqpoint{4.081193in}{1.318187in}}%
\pgfpathlineto{\pgfqpoint{4.089448in}{1.330078in}}%
\pgfpathlineto{\pgfqpoint{4.097697in}{1.342025in}}%
\pgfpathlineto{\pgfqpoint{4.105941in}{1.354022in}}%
\pgfpathlineto{\pgfqpoint{4.114180in}{1.366065in}}%
\pgfpathlineto{\pgfqpoint{4.099887in}{1.366424in}}%
\pgfpathlineto{\pgfqpoint{4.085603in}{1.366855in}}%
\pgfpathlineto{\pgfqpoint{4.071327in}{1.367356in}}%
\pgfpathlineto{\pgfqpoint{4.057061in}{1.367928in}}%
\pgfpathlineto{\pgfqpoint{4.048814in}{1.356211in}}%
\pgfpathlineto{\pgfqpoint{4.040561in}{1.344545in}}%
\pgfpathlineto{\pgfqpoint{4.032303in}{1.332935in}}%
\pgfpathlineto{\pgfqpoint{4.024040in}{1.321386in}}%
\pgfpathclose%
\pgfusepath{fill}%
\end{pgfscope}%
\begin{pgfscope}%
\pgfpathrectangle{\pgfqpoint{1.150000in}{0.150000in}}{\pgfqpoint{5.700000in}{5.700000in}}%
\pgfusepath{clip}%
\pgfsetbuttcap%
\pgfsetroundjoin%
\definecolor{currentfill}{rgb}{0.283187,0.125848,0.444960}%
\pgfsetfillcolor{currentfill}%
\pgfsetfillopacity{0.700000}%
\pgfsetlinewidth{0.000000pt}%
\definecolor{currentstroke}{rgb}{0.000000,0.000000,0.000000}%
\pgfsetstrokecolor{currentstroke}%
\pgfsetdash{}{0pt}%
\pgfpathmoveto{\pgfqpoint{4.204319in}{1.415121in}}%
\pgfpathlineto{\pgfqpoint{4.218651in}{1.415428in}}%
\pgfpathlineto{\pgfqpoint{4.232993in}{1.415805in}}%
\pgfpathlineto{\pgfqpoint{4.247344in}{1.416252in}}%
\pgfpathlineto{\pgfqpoint{4.261706in}{1.416769in}}%
\pgfpathlineto{\pgfqpoint{4.269906in}{1.429574in}}%
\pgfpathlineto{\pgfqpoint{4.278101in}{1.442387in}}%
\pgfpathlineto{\pgfqpoint{4.286291in}{1.455203in}}%
\pgfpathlineto{\pgfqpoint{4.294476in}{1.468020in}}%
\pgfpathlineto{\pgfqpoint{4.280121in}{1.467208in}}%
\pgfpathlineto{\pgfqpoint{4.265775in}{1.466467in}}%
\pgfpathlineto{\pgfqpoint{4.251439in}{1.465797in}}%
\pgfpathlineto{\pgfqpoint{4.237113in}{1.465197in}}%
\pgfpathlineto{\pgfqpoint{4.228922in}{1.452667in}}%
\pgfpathlineto{\pgfqpoint{4.220726in}{1.440141in}}%
\pgfpathlineto{\pgfqpoint{4.212525in}{1.427625in}}%
\pgfpathlineto{\pgfqpoint{4.204319in}{1.415121in}}%
\pgfpathclose%
\pgfusepath{fill}%
\end{pgfscope}%
\begin{pgfscope}%
\pgfpathrectangle{\pgfqpoint{1.150000in}{0.150000in}}{\pgfqpoint{5.700000in}{5.700000in}}%
\pgfusepath{clip}%
\pgfsetbuttcap%
\pgfsetroundjoin%
\definecolor{currentfill}{rgb}{0.182256,0.426184,0.557120}%
\pgfsetfillcolor{currentfill}%
\pgfsetfillopacity{0.700000}%
\pgfsetlinewidth{0.000000pt}%
\definecolor{currentstroke}{rgb}{0.000000,0.000000,0.000000}%
\pgfsetstrokecolor{currentstroke}%
\pgfsetdash{}{0pt}%
\pgfpathmoveto{\pgfqpoint{5.171673in}{2.126187in}}%
\pgfpathlineto{\pgfqpoint{5.186420in}{2.131479in}}%
\pgfpathlineto{\pgfqpoint{5.201181in}{2.136843in}}%
\pgfpathlineto{\pgfqpoint{5.215956in}{2.142279in}}%
\pgfpathlineto{\pgfqpoint{5.230745in}{2.147786in}}%
\pgfpathlineto{\pgfqpoint{5.238601in}{2.158104in}}%
\pgfpathlineto{\pgfqpoint{5.246449in}{2.168266in}}%
\pgfpathlineto{\pgfqpoint{5.254288in}{2.178272in}}%
\pgfpathlineto{\pgfqpoint{5.262118in}{2.188122in}}%
\pgfpathlineto{\pgfqpoint{5.247337in}{2.182576in}}%
\pgfpathlineto{\pgfqpoint{5.232569in}{2.177101in}}%
\pgfpathlineto{\pgfqpoint{5.217815in}{2.171699in}}%
\pgfpathlineto{\pgfqpoint{5.203075in}{2.166368in}}%
\pgfpathlineto{\pgfqpoint{5.195237in}{2.156548in}}%
\pgfpathlineto{\pgfqpoint{5.187391in}{2.146578in}}%
\pgfpathlineto{\pgfqpoint{5.179536in}{2.136458in}}%
\pgfpathlineto{\pgfqpoint{5.171673in}{2.126187in}}%
\pgfpathclose%
\pgfusepath{fill}%
\end{pgfscope}%
\begin{pgfscope}%
\pgfpathrectangle{\pgfqpoint{1.150000in}{0.150000in}}{\pgfqpoint{5.700000in}{5.700000in}}%
\pgfusepath{clip}%
\pgfsetbuttcap%
\pgfsetroundjoin%
\definecolor{currentfill}{rgb}{0.269944,0.014625,0.341379}%
\pgfsetfillcolor{currentfill}%
\pgfsetfillopacity{0.700000}%
\pgfsetlinewidth{0.000000pt}%
\definecolor{currentstroke}{rgb}{0.000000,0.000000,0.000000}%
\pgfsetstrokecolor{currentstroke}%
\pgfsetdash{}{0pt}%
\pgfpathmoveto{\pgfqpoint{3.606147in}{1.213003in}}%
\pgfpathlineto{\pgfqpoint{3.620327in}{1.209134in}}%
\pgfpathlineto{\pgfqpoint{3.634514in}{1.205337in}}%
\pgfpathlineto{\pgfqpoint{3.648707in}{1.201612in}}%
\pgfpathlineto{\pgfqpoint{3.662906in}{1.197959in}}%
\pgfpathlineto{\pgfqpoint{3.671327in}{1.206104in}}%
\pgfpathlineto{\pgfqpoint{3.679740in}{1.214421in}}%
\pgfpathlineto{\pgfqpoint{3.688145in}{1.222903in}}%
\pgfpathlineto{\pgfqpoint{3.696543in}{1.231545in}}%
\pgfpathlineto{\pgfqpoint{3.682360in}{1.234784in}}%
\pgfpathlineto{\pgfqpoint{3.668184in}{1.238094in}}%
\pgfpathlineto{\pgfqpoint{3.654015in}{1.241477in}}%
\pgfpathlineto{\pgfqpoint{3.639852in}{1.244932in}}%
\pgfpathlineto{\pgfqpoint{3.631438in}{1.236696in}}%
\pgfpathlineto{\pgfqpoint{3.623016in}{1.228625in}}%
\pgfpathlineto{\pgfqpoint{3.614586in}{1.220725in}}%
\pgfpathlineto{\pgfqpoint{3.606147in}{1.213003in}}%
\pgfpathclose%
\pgfusepath{fill}%
\end{pgfscope}%
\begin{pgfscope}%
\pgfpathrectangle{\pgfqpoint{1.150000in}{0.150000in}}{\pgfqpoint{5.700000in}{5.700000in}}%
\pgfusepath{clip}%
\pgfsetbuttcap%
\pgfsetroundjoin%
\definecolor{currentfill}{rgb}{0.277941,0.056324,0.381191}%
\pgfsetfillcolor{currentfill}%
\pgfsetfillopacity{0.700000}%
\pgfsetlinewidth{0.000000pt}%
\definecolor{currentstroke}{rgb}{0.000000,0.000000,0.000000}%
\pgfsetstrokecolor{currentstroke}%
\pgfsetdash{}{0pt}%
\pgfpathmoveto{\pgfqpoint{3.933873in}{1.281642in}}%
\pgfpathlineto{\pgfqpoint{3.948125in}{1.280100in}}%
\pgfpathlineto{\pgfqpoint{3.962385in}{1.278629in}}%
\pgfpathlineto{\pgfqpoint{3.976654in}{1.277228in}}%
\pgfpathlineto{\pgfqpoint{3.990930in}{1.275898in}}%
\pgfpathlineto{\pgfqpoint{3.999216in}{1.287153in}}%
\pgfpathlineto{\pgfqpoint{4.007496in}{1.298489in}}%
\pgfpathlineto{\pgfqpoint{4.015771in}{1.309902in}}%
\pgfpathlineto{\pgfqpoint{4.024040in}{1.321386in}}%
\pgfpathlineto{\pgfqpoint{4.009773in}{1.322362in}}%
\pgfpathlineto{\pgfqpoint{3.995514in}{1.323408in}}%
\pgfpathlineto{\pgfqpoint{3.981264in}{1.324526in}}%
\pgfpathlineto{\pgfqpoint{3.967023in}{1.325715in}}%
\pgfpathlineto{\pgfqpoint{3.958744in}{1.314577in}}%
\pgfpathlineto{\pgfqpoint{3.950460in}{1.303516in}}%
\pgfpathlineto{\pgfqpoint{3.942169in}{1.292536in}}%
\pgfpathlineto{\pgfqpoint{3.933873in}{1.281642in}}%
\pgfpathclose%
\pgfusepath{fill}%
\end{pgfscope}%
\begin{pgfscope}%
\pgfpathrectangle{\pgfqpoint{1.150000in}{0.150000in}}{\pgfqpoint{5.700000in}{5.700000in}}%
\pgfusepath{clip}%
\pgfsetbuttcap%
\pgfsetroundjoin%
\definecolor{currentfill}{rgb}{0.174274,0.445044,0.557792}%
\pgfsetfillcolor{currentfill}%
\pgfsetfillopacity{0.700000}%
\pgfsetlinewidth{0.000000pt}%
\definecolor{currentstroke}{rgb}{0.000000,0.000000,0.000000}%
\pgfsetstrokecolor{currentstroke}%
\pgfsetdash{}{0pt}%
\pgfpathmoveto{\pgfqpoint{5.262118in}{2.188122in}}%
\pgfpathlineto{\pgfqpoint{5.276914in}{2.193740in}}%
\pgfpathlineto{\pgfqpoint{5.291723in}{2.199430in}}%
\pgfpathlineto{\pgfqpoint{5.306547in}{2.205192in}}%
\pgfpathlineto{\pgfqpoint{5.314362in}{2.214904in}}%
\pgfpathlineto{\pgfqpoint{5.322168in}{2.224456in}}%
\pgfpathlineto{\pgfqpoint{5.329965in}{2.233849in}}%
\pgfpathlineto{\pgfqpoint{5.337752in}{2.243081in}}%
\pgfpathlineto{\pgfqpoint{5.322937in}{2.237303in}}%
\pgfpathlineto{\pgfqpoint{5.308135in}{2.231597in}}%
\pgfpathlineto{\pgfqpoint{5.293348in}{2.225963in}}%
\pgfpathlineto{\pgfqpoint{5.285554in}{2.216736in}}%
\pgfpathlineto{\pgfqpoint{5.277751in}{2.207354in}}%
\pgfpathlineto{\pgfqpoint{5.269939in}{2.197816in}}%
\pgfpathlineto{\pgfqpoint{5.262118in}{2.188122in}}%
\pgfpathclose%
\pgfusepath{fill}%
\end{pgfscope}%
\begin{pgfscope}%
\pgfpathrectangle{\pgfqpoint{1.150000in}{0.150000in}}{\pgfqpoint{5.700000in}{5.700000in}}%
\pgfusepath{clip}%
\pgfsetbuttcap%
\pgfsetroundjoin%
\definecolor{currentfill}{rgb}{0.281887,0.150881,0.465405}%
\pgfsetfillcolor{currentfill}%
\pgfsetfillopacity{0.700000}%
\pgfsetlinewidth{0.000000pt}%
\definecolor{currentstroke}{rgb}{0.000000,0.000000,0.000000}%
\pgfsetstrokecolor{currentstroke}%
\pgfsetdash{}{0pt}%
\pgfpathmoveto{\pgfqpoint{4.294476in}{1.468020in}}%
\pgfpathlineto{\pgfqpoint{4.308842in}{1.468902in}}%
\pgfpathlineto{\pgfqpoint{4.323217in}{1.469854in}}%
\pgfpathlineto{\pgfqpoint{4.337602in}{1.470876in}}%
\pgfpathlineto{\pgfqpoint{4.351998in}{1.471969in}}%
\pgfpathlineto{\pgfqpoint{4.360173in}{1.485063in}}%
\pgfpathlineto{\pgfqpoint{4.368344in}{1.498142in}}%
\pgfpathlineto{\pgfqpoint{4.376509in}{1.511204in}}%
\pgfpathlineto{\pgfqpoint{4.384669in}{1.524245in}}%
\pgfpathlineto{\pgfqpoint{4.370278in}{1.522878in}}%
\pgfpathlineto{\pgfqpoint{4.355898in}{1.521582in}}%
\pgfpathlineto{\pgfqpoint{4.341527in}{1.520356in}}%
\pgfpathlineto{\pgfqpoint{4.327167in}{1.519201in}}%
\pgfpathlineto{\pgfqpoint{4.319002in}{1.506426in}}%
\pgfpathlineto{\pgfqpoint{4.310832in}{1.493635in}}%
\pgfpathlineto{\pgfqpoint{4.302656in}{1.480832in}}%
\pgfpathlineto{\pgfqpoint{4.294476in}{1.468020in}}%
\pgfpathclose%
\pgfusepath{fill}%
\end{pgfscope}%
\begin{pgfscope}%
\pgfpathrectangle{\pgfqpoint{1.150000in}{0.150000in}}{\pgfqpoint{5.700000in}{5.700000in}}%
\pgfusepath{clip}%
\pgfsetbuttcap%
\pgfsetroundjoin%
\definecolor{currentfill}{rgb}{0.233603,0.313828,0.543914}%
\pgfsetfillcolor{currentfill}%
\pgfsetfillopacity{0.700000}%
\pgfsetlinewidth{0.000000pt}%
\definecolor{currentstroke}{rgb}{0.000000,0.000000,0.000000}%
\pgfsetstrokecolor{currentstroke}%
\pgfsetdash{}{0pt}%
\pgfpathmoveto{\pgfqpoint{4.778200in}{1.823740in}}%
\pgfpathlineto{\pgfqpoint{4.792764in}{1.827366in}}%
\pgfpathlineto{\pgfqpoint{4.807341in}{1.831063in}}%
\pgfpathlineto{\pgfqpoint{4.821929in}{1.834831in}}%
\pgfpathlineto{\pgfqpoint{4.836530in}{1.838670in}}%
\pgfpathlineto{\pgfqpoint{4.844556in}{1.851300in}}%
\pgfpathlineto{\pgfqpoint{4.852576in}{1.863817in}}%
\pgfpathlineto{\pgfqpoint{4.860589in}{1.876221in}}%
\pgfpathlineto{\pgfqpoint{4.868596in}{1.888510in}}%
\pgfpathlineto{\pgfqpoint{4.853999in}{1.884522in}}%
\pgfpathlineto{\pgfqpoint{4.839414in}{1.880605in}}%
\pgfpathlineto{\pgfqpoint{4.824842in}{1.876760in}}%
\pgfpathlineto{\pgfqpoint{4.810282in}{1.872986in}}%
\pgfpathlineto{\pgfqpoint{4.802271in}{1.860838in}}%
\pgfpathlineto{\pgfqpoint{4.794254in}{1.848579in}}%
\pgfpathlineto{\pgfqpoint{4.786230in}{1.836212in}}%
\pgfpathlineto{\pgfqpoint{4.778200in}{1.823740in}}%
\pgfpathclose%
\pgfusepath{fill}%
\end{pgfscope}%
\begin{pgfscope}%
\pgfpathrectangle{\pgfqpoint{1.150000in}{0.150000in}}{\pgfqpoint{5.700000in}{5.700000in}}%
\pgfusepath{clip}%
\pgfsetbuttcap%
\pgfsetroundjoin%
\definecolor{currentfill}{rgb}{0.278012,0.180367,0.486697}%
\pgfsetfillcolor{currentfill}%
\pgfsetfillopacity{0.700000}%
\pgfsetlinewidth{0.000000pt}%
\definecolor{currentstroke}{rgb}{0.000000,0.000000,0.000000}%
\pgfsetstrokecolor{currentstroke}%
\pgfsetdash{}{0pt}%
\pgfpathmoveto{\pgfqpoint{4.384669in}{1.524245in}}%
\pgfpathlineto{\pgfqpoint{4.399070in}{1.525682in}}%
\pgfpathlineto{\pgfqpoint{4.413482in}{1.527189in}}%
\pgfpathlineto{\pgfqpoint{4.427904in}{1.528767in}}%
\pgfpathlineto{\pgfqpoint{4.442337in}{1.530416in}}%
\pgfpathlineto{\pgfqpoint{4.450488in}{1.543692in}}%
\pgfpathlineto{\pgfqpoint{4.458634in}{1.556934in}}%
\pgfpathlineto{\pgfqpoint{4.466775in}{1.570139in}}%
\pgfpathlineto{\pgfqpoint{4.474911in}{1.583302in}}%
\pgfpathlineto{\pgfqpoint{4.460483in}{1.581401in}}%
\pgfpathlineto{\pgfqpoint{4.446065in}{1.579569in}}%
\pgfpathlineto{\pgfqpoint{4.431657in}{1.577809in}}%
\pgfpathlineto{\pgfqpoint{4.417260in}{1.576119in}}%
\pgfpathlineto{\pgfqpoint{4.409120in}{1.563201in}}%
\pgfpathlineto{\pgfqpoint{4.400975in}{1.550247in}}%
\pgfpathlineto{\pgfqpoint{4.392825in}{1.537260in}}%
\pgfpathlineto{\pgfqpoint{4.384669in}{1.524245in}}%
\pgfpathclose%
\pgfusepath{fill}%
\end{pgfscope}%
\begin{pgfscope}%
\pgfpathrectangle{\pgfqpoint{1.150000in}{0.150000in}}{\pgfqpoint{5.700000in}{5.700000in}}%
\pgfusepath{clip}%
\pgfsetbuttcap%
\pgfsetroundjoin%
\definecolor{currentfill}{rgb}{0.274952,0.037752,0.364543}%
\pgfsetfillcolor{currentfill}%
\pgfsetfillopacity{0.700000}%
\pgfsetlinewidth{0.000000pt}%
\definecolor{currentstroke}{rgb}{0.000000,0.000000,0.000000}%
\pgfsetstrokecolor{currentstroke}%
\pgfsetdash{}{0pt}%
\pgfpathmoveto{\pgfqpoint{3.843651in}{1.247416in}}%
\pgfpathlineto{\pgfqpoint{3.857884in}{1.245216in}}%
\pgfpathlineto{\pgfqpoint{3.872124in}{1.243087in}}%
\pgfpathlineto{\pgfqpoint{3.886372in}{1.241029in}}%
\pgfpathlineto{\pgfqpoint{3.900627in}{1.239041in}}%
\pgfpathlineto{\pgfqpoint{3.908948in}{1.249535in}}%
\pgfpathlineto{\pgfqpoint{3.917262in}{1.260136in}}%
\pgfpathlineto{\pgfqpoint{3.925571in}{1.270840in}}%
\pgfpathlineto{\pgfqpoint{3.933873in}{1.281642in}}%
\pgfpathlineto{\pgfqpoint{3.919629in}{1.283255in}}%
\pgfpathlineto{\pgfqpoint{3.905393in}{1.284939in}}%
\pgfpathlineto{\pgfqpoint{3.891165in}{1.286695in}}%
\pgfpathlineto{\pgfqpoint{3.876944in}{1.288521in}}%
\pgfpathlineto{\pgfqpoint{3.868631in}{1.278085in}}%
\pgfpathlineto{\pgfqpoint{3.860310in}{1.267752in}}%
\pgfpathlineto{\pgfqpoint{3.851984in}{1.257527in}}%
\pgfpathlineto{\pgfqpoint{3.843651in}{1.247416in}}%
\pgfpathclose%
\pgfusepath{fill}%
\end{pgfscope}%
\begin{pgfscope}%
\pgfpathrectangle{\pgfqpoint{1.150000in}{0.150000in}}{\pgfqpoint{5.700000in}{5.700000in}}%
\pgfusepath{clip}%
\pgfsetbuttcap%
\pgfsetroundjoin%
\definecolor{currentfill}{rgb}{0.221989,0.339161,0.548752}%
\pgfsetfillcolor{currentfill}%
\pgfsetfillopacity{0.700000}%
\pgfsetlinewidth{0.000000pt}%
\definecolor{currentstroke}{rgb}{0.000000,0.000000,0.000000}%
\pgfsetstrokecolor{currentstroke}%
\pgfsetdash{}{0pt}%
\pgfpathmoveto{\pgfqpoint{4.868596in}{1.888510in}}%
\pgfpathlineto{\pgfqpoint{4.883206in}{1.892568in}}%
\pgfpathlineto{\pgfqpoint{4.897828in}{1.896699in}}%
\pgfpathlineto{\pgfqpoint{4.912463in}{1.900900in}}%
\pgfpathlineto{\pgfqpoint{4.927111in}{1.905172in}}%
\pgfpathlineto{\pgfqpoint{4.935107in}{1.917479in}}%
\pgfpathlineto{\pgfqpoint{4.943096in}{1.929660in}}%
\pgfpathlineto{\pgfqpoint{4.951079in}{1.941717in}}%
\pgfpathlineto{\pgfqpoint{4.959054in}{1.953645in}}%
\pgfpathlineto{\pgfqpoint{4.944410in}{1.949246in}}%
\pgfpathlineto{\pgfqpoint{4.929779in}{1.944918in}}%
\pgfpathlineto{\pgfqpoint{4.915161in}{1.940661in}}%
\pgfpathlineto{\pgfqpoint{4.900556in}{1.936476in}}%
\pgfpathlineto{\pgfqpoint{4.892576in}{1.924665in}}%
\pgfpathlineto{\pgfqpoint{4.884589in}{1.912733in}}%
\pgfpathlineto{\pgfqpoint{4.876596in}{1.900681in}}%
\pgfpathlineto{\pgfqpoint{4.868596in}{1.888510in}}%
\pgfpathclose%
\pgfusepath{fill}%
\end{pgfscope}%
\begin{pgfscope}%
\pgfpathrectangle{\pgfqpoint{1.150000in}{0.150000in}}{\pgfqpoint{5.700000in}{5.700000in}}%
\pgfusepath{clip}%
\pgfsetbuttcap%
\pgfsetroundjoin%
\definecolor{currentfill}{rgb}{0.271828,0.209303,0.504434}%
\pgfsetfillcolor{currentfill}%
\pgfsetfillopacity{0.700000}%
\pgfsetlinewidth{0.000000pt}%
\definecolor{currentstroke}{rgb}{0.000000,0.000000,0.000000}%
\pgfsetstrokecolor{currentstroke}%
\pgfsetdash{}{0pt}%
\pgfpathmoveto{\pgfqpoint{4.474911in}{1.583302in}}%
\pgfpathlineto{\pgfqpoint{4.489351in}{1.585274in}}%
\pgfpathlineto{\pgfqpoint{4.503801in}{1.587317in}}%
\pgfpathlineto{\pgfqpoint{4.518263in}{1.589430in}}%
\pgfpathlineto{\pgfqpoint{4.532735in}{1.591614in}}%
\pgfpathlineto{\pgfqpoint{4.540862in}{1.604973in}}%
\pgfpathlineto{\pgfqpoint{4.548985in}{1.618279in}}%
\pgfpathlineto{\pgfqpoint{4.557101in}{1.631528in}}%
\pgfpathlineto{\pgfqpoint{4.565213in}{1.644718in}}%
\pgfpathlineto{\pgfqpoint{4.550744in}{1.642301in}}%
\pgfpathlineto{\pgfqpoint{4.536286in}{1.639956in}}%
\pgfpathlineto{\pgfqpoint{4.521840in}{1.637680in}}%
\pgfpathlineto{\pgfqpoint{4.507404in}{1.635476in}}%
\pgfpathlineto{\pgfqpoint{4.499289in}{1.622511in}}%
\pgfpathlineto{\pgfqpoint{4.491168in}{1.609492in}}%
\pgfpathlineto{\pgfqpoint{4.483042in}{1.596421in}}%
\pgfpathlineto{\pgfqpoint{4.474911in}{1.583302in}}%
\pgfpathclose%
\pgfusepath{fill}%
\end{pgfscope}%
\begin{pgfscope}%
\pgfpathrectangle{\pgfqpoint{1.150000in}{0.150000in}}{\pgfqpoint{5.700000in}{5.700000in}}%
\pgfusepath{clip}%
\pgfsetbuttcap%
\pgfsetroundjoin%
\definecolor{currentfill}{rgb}{0.272594,0.025563,0.353093}%
\pgfsetfillcolor{currentfill}%
\pgfsetfillopacity{0.700000}%
\pgfsetlinewidth{0.000000pt}%
\definecolor{currentstroke}{rgb}{0.000000,0.000000,0.000000}%
\pgfsetstrokecolor{currentstroke}%
\pgfsetdash{}{0pt}%
\pgfpathmoveto{\pgfqpoint{3.753341in}{1.219311in}}%
\pgfpathlineto{\pgfqpoint{3.767558in}{1.216432in}}%
\pgfpathlineto{\pgfqpoint{3.781783in}{1.213624in}}%
\pgfpathlineto{\pgfqpoint{3.796014in}{1.210887in}}%
\pgfpathlineto{\pgfqpoint{3.810253in}{1.208222in}}%
\pgfpathlineto{\pgfqpoint{3.818612in}{1.217821in}}%
\pgfpathlineto{\pgfqpoint{3.826965in}{1.227557in}}%
\pgfpathlineto{\pgfqpoint{3.835312in}{1.237424in}}%
\pgfpathlineto{\pgfqpoint{3.843651in}{1.247416in}}%
\pgfpathlineto{\pgfqpoint{3.829426in}{1.249687in}}%
\pgfpathlineto{\pgfqpoint{3.815209in}{1.252029in}}%
\pgfpathlineto{\pgfqpoint{3.800998in}{1.254443in}}%
\pgfpathlineto{\pgfqpoint{3.786796in}{1.256929in}}%
\pgfpathlineto{\pgfqpoint{3.778443in}{1.247323in}}%
\pgfpathlineto{\pgfqpoint{3.770083in}{1.237848in}}%
\pgfpathlineto{\pgfqpoint{3.761716in}{1.228508in}}%
\pgfpathlineto{\pgfqpoint{3.753341in}{1.219311in}}%
\pgfpathclose%
\pgfusepath{fill}%
\end{pgfscope}%
\begin{pgfscope}%
\pgfpathrectangle{\pgfqpoint{1.150000in}{0.150000in}}{\pgfqpoint{5.700000in}{5.700000in}}%
\pgfusepath{clip}%
\pgfsetbuttcap%
\pgfsetroundjoin%
\definecolor{currentfill}{rgb}{0.262138,0.242286,0.520837}%
\pgfsetfillcolor{currentfill}%
\pgfsetfillopacity{0.700000}%
\pgfsetlinewidth{0.000000pt}%
\definecolor{currentstroke}{rgb}{0.000000,0.000000,0.000000}%
\pgfsetstrokecolor{currentstroke}%
\pgfsetdash{}{0pt}%
\pgfpathmoveto{\pgfqpoint{4.565213in}{1.644718in}}%
\pgfpathlineto{\pgfqpoint{4.579693in}{1.647205in}}%
\pgfpathlineto{\pgfqpoint{4.594185in}{1.649763in}}%
\pgfpathlineto{\pgfqpoint{4.608687in}{1.652391in}}%
\pgfpathlineto{\pgfqpoint{4.623202in}{1.655090in}}%
\pgfpathlineto{\pgfqpoint{4.631305in}{1.668437in}}%
\pgfpathlineto{\pgfqpoint{4.639402in}{1.681712in}}%
\pgfpathlineto{\pgfqpoint{4.647494in}{1.694914in}}%
\pgfpathlineto{\pgfqpoint{4.655581in}{1.708038in}}%
\pgfpathlineto{\pgfqpoint{4.641070in}{1.705127in}}%
\pgfpathlineto{\pgfqpoint{4.626570in}{1.702286in}}%
\pgfpathlineto{\pgfqpoint{4.612082in}{1.699517in}}%
\pgfpathlineto{\pgfqpoint{4.597605in}{1.696818in}}%
\pgfpathlineto{\pgfqpoint{4.589515in}{1.683897in}}%
\pgfpathlineto{\pgfqpoint{4.581420in}{1.670905in}}%
\pgfpathlineto{\pgfqpoint{4.573319in}{1.657845in}}%
\pgfpathlineto{\pgfqpoint{4.565213in}{1.644718in}}%
\pgfpathclose%
\pgfusepath{fill}%
\end{pgfscope}%
\begin{pgfscope}%
\pgfpathrectangle{\pgfqpoint{1.150000in}{0.150000in}}{\pgfqpoint{5.700000in}{5.700000in}}%
\pgfusepath{clip}%
\pgfsetbuttcap%
\pgfsetroundjoin%
\definecolor{currentfill}{rgb}{0.208623,0.367752,0.552675}%
\pgfsetfillcolor{currentfill}%
\pgfsetfillopacity{0.700000}%
\pgfsetlinewidth{0.000000pt}%
\definecolor{currentstroke}{rgb}{0.000000,0.000000,0.000000}%
\pgfsetstrokecolor{currentstroke}%
\pgfsetdash{}{0pt}%
\pgfpathmoveto{\pgfqpoint{4.959054in}{1.953645in}}%
\pgfpathlineto{\pgfqpoint{4.973711in}{1.958116in}}%
\pgfpathlineto{\pgfqpoint{4.988381in}{1.962658in}}%
\pgfpathlineto{\pgfqpoint{5.003063in}{1.967271in}}%
\pgfpathlineto{\pgfqpoint{5.017759in}{1.971956in}}%
\pgfpathlineto{\pgfqpoint{5.025723in}{1.983870in}}%
\pgfpathlineto{\pgfqpoint{5.033680in}{1.995647in}}%
\pgfpathlineto{\pgfqpoint{5.041629in}{2.007288in}}%
\pgfpathlineto{\pgfqpoint{5.049570in}{2.018790in}}%
\pgfpathlineto{\pgfqpoint{5.034879in}{2.014000in}}%
\pgfpathlineto{\pgfqpoint{5.020200in}{2.009282in}}%
\pgfpathlineto{\pgfqpoint{5.005535in}{2.004635in}}%
\pgfpathlineto{\pgfqpoint{4.990883in}{2.000059in}}%
\pgfpathlineto{\pgfqpoint{4.982937in}{1.988653in}}%
\pgfpathlineto{\pgfqpoint{4.974983in}{1.977115in}}%
\pgfpathlineto{\pgfqpoint{4.967022in}{1.965445in}}%
\pgfpathlineto{\pgfqpoint{4.959054in}{1.953645in}}%
\pgfpathclose%
\pgfusepath{fill}%
\end{pgfscope}%
\begin{pgfscope}%
\pgfpathrectangle{\pgfqpoint{1.150000in}{0.150000in}}{\pgfqpoint{5.700000in}{5.700000in}}%
\pgfusepath{clip}%
\pgfsetbuttcap%
\pgfsetroundjoin%
\definecolor{currentfill}{rgb}{0.269944,0.014625,0.341379}%
\pgfsetfillcolor{currentfill}%
\pgfsetfillopacity{0.700000}%
\pgfsetlinewidth{0.000000pt}%
\definecolor{currentstroke}{rgb}{0.000000,0.000000,0.000000}%
\pgfsetstrokecolor{currentstroke}%
\pgfsetdash{}{0pt}%
\pgfpathmoveto{\pgfqpoint{3.662906in}{1.197959in}}%
\pgfpathlineto{\pgfqpoint{3.677112in}{1.194379in}}%
\pgfpathlineto{\pgfqpoint{3.691325in}{1.190870in}}%
\pgfpathlineto{\pgfqpoint{3.705544in}{1.187433in}}%
\pgfpathlineto{\pgfqpoint{3.719770in}{1.184068in}}%
\pgfpathlineto{\pgfqpoint{3.728174in}{1.192634in}}%
\pgfpathlineto{\pgfqpoint{3.736571in}{1.201368in}}%
\pgfpathlineto{\pgfqpoint{3.744960in}{1.210263in}}%
\pgfpathlineto{\pgfqpoint{3.753341in}{1.219311in}}%
\pgfpathlineto{\pgfqpoint{3.739131in}{1.222262in}}%
\pgfpathlineto{\pgfqpoint{3.724928in}{1.225285in}}%
\pgfpathlineto{\pgfqpoint{3.710732in}{1.228379in}}%
\pgfpathlineto{\pgfqpoint{3.696543in}{1.231545in}}%
\pgfpathlineto{\pgfqpoint{3.688145in}{1.222903in}}%
\pgfpathlineto{\pgfqpoint{3.679740in}{1.214421in}}%
\pgfpathlineto{\pgfqpoint{3.671327in}{1.206104in}}%
\pgfpathlineto{\pgfqpoint{3.662906in}{1.197959in}}%
\pgfpathclose%
\pgfusepath{fill}%
\end{pgfscope}%
\begin{pgfscope}%
\pgfpathrectangle{\pgfqpoint{1.150000in}{0.150000in}}{\pgfqpoint{5.700000in}{5.700000in}}%
\pgfusepath{clip}%
\pgfsetbuttcap%
\pgfsetroundjoin%
\definecolor{currentfill}{rgb}{0.252194,0.269783,0.531579}%
\pgfsetfillcolor{currentfill}%
\pgfsetfillopacity{0.700000}%
\pgfsetlinewidth{0.000000pt}%
\definecolor{currentstroke}{rgb}{0.000000,0.000000,0.000000}%
\pgfsetstrokecolor{currentstroke}%
\pgfsetdash{}{0pt}%
\pgfpathmoveto{\pgfqpoint{4.655581in}{1.708038in}}%
\pgfpathlineto{\pgfqpoint{4.670104in}{1.711020in}}%
\pgfpathlineto{\pgfqpoint{4.684638in}{1.714072in}}%
\pgfpathlineto{\pgfqpoint{4.699184in}{1.717196in}}%
\pgfpathlineto{\pgfqpoint{4.713743in}{1.720390in}}%
\pgfpathlineto{\pgfqpoint{4.721820in}{1.733633in}}%
\pgfpathlineto{\pgfqpoint{4.729893in}{1.746789in}}%
\pgfpathlineto{\pgfqpoint{4.737959in}{1.759855in}}%
\pgfpathlineto{\pgfqpoint{4.746019in}{1.772827in}}%
\pgfpathlineto{\pgfqpoint{4.731464in}{1.769442in}}%
\pgfpathlineto{\pgfqpoint{4.716921in}{1.766127in}}%
\pgfpathlineto{\pgfqpoint{4.702390in}{1.762883in}}%
\pgfpathlineto{\pgfqpoint{4.687870in}{1.759711in}}%
\pgfpathlineto{\pgfqpoint{4.679807in}{1.746921in}}%
\pgfpathlineto{\pgfqpoint{4.671737in}{1.734044in}}%
\pgfpathlineto{\pgfqpoint{4.663662in}{1.721082in}}%
\pgfpathlineto{\pgfqpoint{4.655581in}{1.708038in}}%
\pgfpathclose%
\pgfusepath{fill}%
\end{pgfscope}%
\begin{pgfscope}%
\pgfpathrectangle{\pgfqpoint{1.150000in}{0.150000in}}{\pgfqpoint{5.700000in}{5.700000in}}%
\pgfusepath{clip}%
\pgfsetbuttcap%
\pgfsetroundjoin%
\definecolor{currentfill}{rgb}{0.195860,0.395433,0.555276}%
\pgfsetfillcolor{currentfill}%
\pgfsetfillopacity{0.700000}%
\pgfsetlinewidth{0.000000pt}%
\definecolor{currentstroke}{rgb}{0.000000,0.000000,0.000000}%
\pgfsetstrokecolor{currentstroke}%
\pgfsetdash{}{0pt}%
\pgfpathmoveto{\pgfqpoint{5.049570in}{2.018790in}}%
\pgfpathlineto{\pgfqpoint{5.064275in}{2.023652in}}%
\pgfpathlineto{\pgfqpoint{5.078993in}{2.028585in}}%
\pgfpathlineto{\pgfqpoint{5.093724in}{2.033590in}}%
\pgfpathlineto{\pgfqpoint{5.108469in}{2.038666in}}%
\pgfpathlineto{\pgfqpoint{5.116398in}{2.050121in}}%
\pgfpathlineto{\pgfqpoint{5.124319in}{2.061431in}}%
\pgfpathlineto{\pgfqpoint{5.132232in}{2.072594in}}%
\pgfpathlineto{\pgfqpoint{5.140136in}{2.083609in}}%
\pgfpathlineto{\pgfqpoint{5.125397in}{2.078450in}}%
\pgfpathlineto{\pgfqpoint{5.110670in}{2.073362in}}%
\pgfpathlineto{\pgfqpoint{5.095958in}{2.068346in}}%
\pgfpathlineto{\pgfqpoint{5.081258in}{2.063401in}}%
\pgfpathlineto{\pgfqpoint{5.073348in}{2.052460in}}%
\pgfpathlineto{\pgfqpoint{5.065430in}{2.041377in}}%
\pgfpathlineto{\pgfqpoint{5.057504in}{2.030154in}}%
\pgfpathlineto{\pgfqpoint{5.049570in}{2.018790in}}%
\pgfpathclose%
\pgfusepath{fill}%
\end{pgfscope}%
\begin{pgfscope}%
\pgfpathrectangle{\pgfqpoint{1.150000in}{0.150000in}}{\pgfqpoint{5.700000in}{5.700000in}}%
\pgfusepath{clip}%
\pgfsetbuttcap%
\pgfsetroundjoin%
\definecolor{currentfill}{rgb}{0.281924,0.089666,0.412415}%
\pgfsetfillcolor{currentfill}%
\pgfsetfillopacity{0.700000}%
\pgfsetlinewidth{0.000000pt}%
\definecolor{currentstroke}{rgb}{0.000000,0.000000,0.000000}%
\pgfsetstrokecolor{currentstroke}%
\pgfsetdash{}{0pt}%
\pgfpathmoveto{\pgfqpoint{4.081193in}{1.318187in}}%
\pgfpathlineto{\pgfqpoint{4.095503in}{1.317564in}}%
\pgfpathlineto{\pgfqpoint{4.109823in}{1.317011in}}%
\pgfpathlineto{\pgfqpoint{4.124151in}{1.316528in}}%
\pgfpathlineto{\pgfqpoint{4.138488in}{1.316115in}}%
\pgfpathlineto{\pgfqpoint{4.146734in}{1.328348in}}%
\pgfpathlineto{\pgfqpoint{4.154976in}{1.340632in}}%
\pgfpathlineto{\pgfqpoint{4.163212in}{1.352961in}}%
\pgfpathlineto{\pgfqpoint{4.171444in}{1.365330in}}%
\pgfpathlineto{\pgfqpoint{4.157114in}{1.365408in}}%
\pgfpathlineto{\pgfqpoint{4.142794in}{1.365557in}}%
\pgfpathlineto{\pgfqpoint{4.128482in}{1.365775in}}%
\pgfpathlineto{\pgfqpoint{4.114180in}{1.366065in}}%
\pgfpathlineto{\pgfqpoint{4.105941in}{1.354022in}}%
\pgfpathlineto{\pgfqpoint{4.097697in}{1.342025in}}%
\pgfpathlineto{\pgfqpoint{4.089448in}{1.330078in}}%
\pgfpathlineto{\pgfqpoint{4.081193in}{1.318187in}}%
\pgfpathclose%
\pgfusepath{fill}%
\end{pgfscope}%
\begin{pgfscope}%
\pgfpathrectangle{\pgfqpoint{1.150000in}{0.150000in}}{\pgfqpoint{5.700000in}{5.700000in}}%
\pgfusepath{clip}%
\pgfsetbuttcap%
\pgfsetroundjoin%
\definecolor{currentfill}{rgb}{0.283197,0.115680,0.436115}%
\pgfsetfillcolor{currentfill}%
\pgfsetfillopacity{0.700000}%
\pgfsetlinewidth{0.000000pt}%
\definecolor{currentstroke}{rgb}{0.000000,0.000000,0.000000}%
\pgfsetstrokecolor{currentstroke}%
\pgfsetdash{}{0pt}%
\pgfpathmoveto{\pgfqpoint{4.171444in}{1.365330in}}%
\pgfpathlineto{\pgfqpoint{4.185783in}{1.365323in}}%
\pgfpathlineto{\pgfqpoint{4.200131in}{1.365385in}}%
\pgfpathlineto{\pgfqpoint{4.214489in}{1.365518in}}%
\pgfpathlineto{\pgfqpoint{4.228856in}{1.365721in}}%
\pgfpathlineto{\pgfqpoint{4.237076in}{1.378448in}}%
\pgfpathlineto{\pgfqpoint{4.245291in}{1.391202in}}%
\pgfpathlineto{\pgfqpoint{4.253501in}{1.403977in}}%
\pgfpathlineto{\pgfqpoint{4.261706in}{1.416769in}}%
\pgfpathlineto{\pgfqpoint{4.247344in}{1.416252in}}%
\pgfpathlineto{\pgfqpoint{4.232993in}{1.415805in}}%
\pgfpathlineto{\pgfqpoint{4.218651in}{1.415428in}}%
\pgfpathlineto{\pgfqpoint{4.204319in}{1.415121in}}%
\pgfpathlineto{\pgfqpoint{4.196108in}{1.402636in}}%
\pgfpathlineto{\pgfqpoint{4.187891in}{1.390173in}}%
\pgfpathlineto{\pgfqpoint{4.179670in}{1.377736in}}%
\pgfpathlineto{\pgfqpoint{4.171444in}{1.365330in}}%
\pgfpathclose%
\pgfusepath{fill}%
\end{pgfscope}%
\begin{pgfscope}%
\pgfpathrectangle{\pgfqpoint{1.150000in}{0.150000in}}{\pgfqpoint{5.700000in}{5.700000in}}%
\pgfusepath{clip}%
\pgfsetbuttcap%
\pgfsetroundjoin%
\definecolor{currentfill}{rgb}{0.279566,0.067836,0.391917}%
\pgfsetfillcolor{currentfill}%
\pgfsetfillopacity{0.700000}%
\pgfsetlinewidth{0.000000pt}%
\definecolor{currentstroke}{rgb}{0.000000,0.000000,0.000000}%
\pgfsetstrokecolor{currentstroke}%
\pgfsetdash{}{0pt}%
\pgfpathmoveto{\pgfqpoint{3.990930in}{1.275898in}}%
\pgfpathlineto{\pgfqpoint{4.005215in}{1.274638in}}%
\pgfpathlineto{\pgfqpoint{4.019508in}{1.273449in}}%
\pgfpathlineto{\pgfqpoint{4.033810in}{1.272330in}}%
\pgfpathlineto{\pgfqpoint{4.048121in}{1.271282in}}%
\pgfpathlineto{\pgfqpoint{4.056397in}{1.282899in}}%
\pgfpathlineto{\pgfqpoint{4.064668in}{1.294593in}}%
\pgfpathlineto{\pgfqpoint{4.072933in}{1.306357in}}%
\pgfpathlineto{\pgfqpoint{4.081193in}{1.318187in}}%
\pgfpathlineto{\pgfqpoint{4.066892in}{1.318881in}}%
\pgfpathlineto{\pgfqpoint{4.052599in}{1.319645in}}%
\pgfpathlineto{\pgfqpoint{4.038315in}{1.320480in}}%
\pgfpathlineto{\pgfqpoint{4.024040in}{1.321386in}}%
\pgfpathlineto{\pgfqpoint{4.015771in}{1.309902in}}%
\pgfpathlineto{\pgfqpoint{4.007496in}{1.298489in}}%
\pgfpathlineto{\pgfqpoint{3.999216in}{1.287153in}}%
\pgfpathlineto{\pgfqpoint{3.990930in}{1.275898in}}%
\pgfpathclose%
\pgfusepath{fill}%
\end{pgfscope}%
\begin{pgfscope}%
\pgfpathrectangle{\pgfqpoint{1.150000in}{0.150000in}}{\pgfqpoint{5.700000in}{5.700000in}}%
\pgfusepath{clip}%
\pgfsetbuttcap%
\pgfsetroundjoin%
\definecolor{currentfill}{rgb}{0.282623,0.140926,0.457517}%
\pgfsetfillcolor{currentfill}%
\pgfsetfillopacity{0.700000}%
\pgfsetlinewidth{0.000000pt}%
\definecolor{currentstroke}{rgb}{0.000000,0.000000,0.000000}%
\pgfsetstrokecolor{currentstroke}%
\pgfsetdash{}{0pt}%
\pgfpathmoveto{\pgfqpoint{4.261706in}{1.416769in}}%
\pgfpathlineto{\pgfqpoint{4.276076in}{1.417357in}}%
\pgfpathlineto{\pgfqpoint{4.290457in}{1.418015in}}%
\pgfpathlineto{\pgfqpoint{4.304848in}{1.418743in}}%
\pgfpathlineto{\pgfqpoint{4.319248in}{1.419542in}}%
\pgfpathlineto{\pgfqpoint{4.327443in}{1.432648in}}%
\pgfpathlineto{\pgfqpoint{4.335633in}{1.445758in}}%
\pgfpathlineto{\pgfqpoint{4.343818in}{1.458867in}}%
\pgfpathlineto{\pgfqpoint{4.351998in}{1.471969in}}%
\pgfpathlineto{\pgfqpoint{4.337602in}{1.470876in}}%
\pgfpathlineto{\pgfqpoint{4.323217in}{1.469854in}}%
\pgfpathlineto{\pgfqpoint{4.308842in}{1.468902in}}%
\pgfpathlineto{\pgfqpoint{4.294476in}{1.468020in}}%
\pgfpathlineto{\pgfqpoint{4.286291in}{1.455203in}}%
\pgfpathlineto{\pgfqpoint{4.278101in}{1.442387in}}%
\pgfpathlineto{\pgfqpoint{4.269906in}{1.429574in}}%
\pgfpathlineto{\pgfqpoint{4.261706in}{1.416769in}}%
\pgfpathclose%
\pgfusepath{fill}%
\end{pgfscope}%
\begin{pgfscope}%
\pgfpathrectangle{\pgfqpoint{1.150000in}{0.150000in}}{\pgfqpoint{5.700000in}{5.700000in}}%
\pgfusepath{clip}%
\pgfsetbuttcap%
\pgfsetroundjoin%
\definecolor{currentfill}{rgb}{0.239346,0.300855,0.540844}%
\pgfsetfillcolor{currentfill}%
\pgfsetfillopacity{0.700000}%
\pgfsetlinewidth{0.000000pt}%
\definecolor{currentstroke}{rgb}{0.000000,0.000000,0.000000}%
\pgfsetstrokecolor{currentstroke}%
\pgfsetdash{}{0pt}%
\pgfpathmoveto{\pgfqpoint{4.746019in}{1.772827in}}%
\pgfpathlineto{\pgfqpoint{4.760587in}{1.776283in}}%
\pgfpathlineto{\pgfqpoint{4.775166in}{1.779811in}}%
\pgfpathlineto{\pgfqpoint{4.789757in}{1.783409in}}%
\pgfpathlineto{\pgfqpoint{4.804361in}{1.787078in}}%
\pgfpathlineto{\pgfqpoint{4.812413in}{1.800133in}}%
\pgfpathlineto{\pgfqpoint{4.820458in}{1.813085in}}%
\pgfpathlineto{\pgfqpoint{4.828497in}{1.825931in}}%
\pgfpathlineto{\pgfqpoint{4.836530in}{1.838670in}}%
\pgfpathlineto{\pgfqpoint{4.821929in}{1.834831in}}%
\pgfpathlineto{\pgfqpoint{4.807341in}{1.831063in}}%
\pgfpathlineto{\pgfqpoint{4.792764in}{1.827366in}}%
\pgfpathlineto{\pgfqpoint{4.778200in}{1.823740in}}%
\pgfpathlineto{\pgfqpoint{4.770164in}{1.811163in}}%
\pgfpathlineto{\pgfqpoint{4.762122in}{1.798483in}}%
\pgfpathlineto{\pgfqpoint{4.754074in}{1.785704in}}%
\pgfpathlineto{\pgfqpoint{4.746019in}{1.772827in}}%
\pgfpathclose%
\pgfusepath{fill}%
\end{pgfscope}%
\begin{pgfscope}%
\pgfpathrectangle{\pgfqpoint{1.150000in}{0.150000in}}{\pgfqpoint{5.700000in}{5.700000in}}%
\pgfusepath{clip}%
\pgfsetbuttcap%
\pgfsetroundjoin%
\definecolor{currentfill}{rgb}{0.276022,0.044167,0.370164}%
\pgfsetfillcolor{currentfill}%
\pgfsetfillopacity{0.700000}%
\pgfsetlinewidth{0.000000pt}%
\definecolor{currentstroke}{rgb}{0.000000,0.000000,0.000000}%
\pgfsetstrokecolor{currentstroke}%
\pgfsetdash{}{0pt}%
\pgfpathmoveto{\pgfqpoint{3.900627in}{1.239041in}}%
\pgfpathlineto{\pgfqpoint{3.914891in}{1.237125in}}%
\pgfpathlineto{\pgfqpoint{3.929162in}{1.235279in}}%
\pgfpathlineto{\pgfqpoint{3.943441in}{1.233504in}}%
\pgfpathlineto{\pgfqpoint{3.957729in}{1.231799in}}%
\pgfpathlineto{\pgfqpoint{3.966038in}{1.242675in}}%
\pgfpathlineto{\pgfqpoint{3.974341in}{1.253653in}}%
\pgfpathlineto{\pgfqpoint{3.982639in}{1.264729in}}%
\pgfpathlineto{\pgfqpoint{3.990930in}{1.275898in}}%
\pgfpathlineto{\pgfqpoint{3.976654in}{1.277228in}}%
\pgfpathlineto{\pgfqpoint{3.962385in}{1.278629in}}%
\pgfpathlineto{\pgfqpoint{3.948125in}{1.280100in}}%
\pgfpathlineto{\pgfqpoint{3.933873in}{1.281642in}}%
\pgfpathlineto{\pgfqpoint{3.925571in}{1.270840in}}%
\pgfpathlineto{\pgfqpoint{3.917262in}{1.260136in}}%
\pgfpathlineto{\pgfqpoint{3.908948in}{1.249535in}}%
\pgfpathlineto{\pgfqpoint{3.900627in}{1.239041in}}%
\pgfpathclose%
\pgfusepath{fill}%
\end{pgfscope}%
\begin{pgfscope}%
\pgfpathrectangle{\pgfqpoint{1.150000in}{0.150000in}}{\pgfqpoint{5.700000in}{5.700000in}}%
\pgfusepath{clip}%
\pgfsetbuttcap%
\pgfsetroundjoin%
\definecolor{currentfill}{rgb}{0.185556,0.418570,0.556753}%
\pgfsetfillcolor{currentfill}%
\pgfsetfillopacity{0.700000}%
\pgfsetlinewidth{0.000000pt}%
\definecolor{currentstroke}{rgb}{0.000000,0.000000,0.000000}%
\pgfsetstrokecolor{currentstroke}%
\pgfsetdash{}{0pt}%
\pgfpathmoveto{\pgfqpoint{5.140136in}{2.083609in}}%
\pgfpathlineto{\pgfqpoint{5.154890in}{2.088841in}}%
\pgfpathlineto{\pgfqpoint{5.169657in}{2.094144in}}%
\pgfpathlineto{\pgfqpoint{5.184438in}{2.099518in}}%
\pgfpathlineto{\pgfqpoint{5.199232in}{2.104965in}}%
\pgfpathlineto{\pgfqpoint{5.207123in}{2.115902in}}%
\pgfpathlineto{\pgfqpoint{5.215006in}{2.126685in}}%
\pgfpathlineto{\pgfqpoint{5.222880in}{2.137313in}}%
\pgfpathlineto{\pgfqpoint{5.230745in}{2.147786in}}%
\pgfpathlineto{\pgfqpoint{5.215956in}{2.142279in}}%
\pgfpathlineto{\pgfqpoint{5.201181in}{2.136843in}}%
\pgfpathlineto{\pgfqpoint{5.186420in}{2.131479in}}%
\pgfpathlineto{\pgfqpoint{5.171673in}{2.126187in}}%
\pgfpathlineto{\pgfqpoint{5.163802in}{2.115766in}}%
\pgfpathlineto{\pgfqpoint{5.155921in}{2.105196in}}%
\pgfpathlineto{\pgfqpoint{5.148033in}{2.094477in}}%
\pgfpathlineto{\pgfqpoint{5.140136in}{2.083609in}}%
\pgfpathclose%
\pgfusepath{fill}%
\end{pgfscope}%
\begin{pgfscope}%
\pgfpathrectangle{\pgfqpoint{1.150000in}{0.150000in}}{\pgfqpoint{5.700000in}{5.700000in}}%
\pgfusepath{clip}%
\pgfsetbuttcap%
\pgfsetroundjoin%
\definecolor{currentfill}{rgb}{0.280255,0.165693,0.476498}%
\pgfsetfillcolor{currentfill}%
\pgfsetfillopacity{0.700000}%
\pgfsetlinewidth{0.000000pt}%
\definecolor{currentstroke}{rgb}{0.000000,0.000000,0.000000}%
\pgfsetstrokecolor{currentstroke}%
\pgfsetdash{}{0pt}%
\pgfpathmoveto{\pgfqpoint{4.351998in}{1.471969in}}%
\pgfpathlineto{\pgfqpoint{4.366404in}{1.473133in}}%
\pgfpathlineto{\pgfqpoint{4.380820in}{1.474366in}}%
\pgfpathlineto{\pgfqpoint{4.395246in}{1.475670in}}%
\pgfpathlineto{\pgfqpoint{4.409683in}{1.477044in}}%
\pgfpathlineto{\pgfqpoint{4.417854in}{1.490419in}}%
\pgfpathlineto{\pgfqpoint{4.426020in}{1.503775in}}%
\pgfpathlineto{\pgfqpoint{4.434181in}{1.517109in}}%
\pgfpathlineto{\pgfqpoint{4.442337in}{1.530416in}}%
\pgfpathlineto{\pgfqpoint{4.427904in}{1.528767in}}%
\pgfpathlineto{\pgfqpoint{4.413482in}{1.527189in}}%
\pgfpathlineto{\pgfqpoint{4.399070in}{1.525682in}}%
\pgfpathlineto{\pgfqpoint{4.384669in}{1.524245in}}%
\pgfpathlineto{\pgfqpoint{4.376509in}{1.511204in}}%
\pgfpathlineto{\pgfqpoint{4.368344in}{1.498142in}}%
\pgfpathlineto{\pgfqpoint{4.360173in}{1.485063in}}%
\pgfpathlineto{\pgfqpoint{4.351998in}{1.471969in}}%
\pgfpathclose%
\pgfusepath{fill}%
\end{pgfscope}%
\begin{pgfscope}%
\pgfpathrectangle{\pgfqpoint{1.150000in}{0.150000in}}{\pgfqpoint{5.700000in}{5.700000in}}%
\pgfusepath{clip}%
\pgfsetbuttcap%
\pgfsetroundjoin%
\definecolor{currentfill}{rgb}{0.175841,0.441290,0.557685}%
\pgfsetfillcolor{currentfill}%
\pgfsetfillopacity{0.700000}%
\pgfsetlinewidth{0.000000pt}%
\definecolor{currentstroke}{rgb}{0.000000,0.000000,0.000000}%
\pgfsetstrokecolor{currentstroke}%
\pgfsetdash{}{0pt}%
\pgfpathmoveto{\pgfqpoint{5.230745in}{2.147786in}}%
\pgfpathlineto{\pgfqpoint{5.245547in}{2.153366in}}%
\pgfpathlineto{\pgfqpoint{5.260364in}{2.159017in}}%
\pgfpathlineto{\pgfqpoint{5.275195in}{2.164740in}}%
\pgfpathlineto{\pgfqpoint{5.283046in}{2.175093in}}%
\pgfpathlineto{\pgfqpoint{5.290889in}{2.185286in}}%
\pgfpathlineto{\pgfqpoint{5.298723in}{2.195319in}}%
\pgfpathlineto{\pgfqpoint{5.306547in}{2.205192in}}%
\pgfpathlineto{\pgfqpoint{5.291723in}{2.199430in}}%
\pgfpathlineto{\pgfqpoint{5.276914in}{2.193740in}}%
\pgfpathlineto{\pgfqpoint{5.262118in}{2.188122in}}%
\pgfpathlineto{\pgfqpoint{5.254288in}{2.178272in}}%
\pgfpathlineto{\pgfqpoint{5.246449in}{2.168266in}}%
\pgfpathlineto{\pgfqpoint{5.238601in}{2.158104in}}%
\pgfpathlineto{\pgfqpoint{5.230745in}{2.147786in}}%
\pgfpathclose%
\pgfusepath{fill}%
\end{pgfscope}%
\begin{pgfscope}%
\pgfpathrectangle{\pgfqpoint{1.150000in}{0.150000in}}{\pgfqpoint{5.700000in}{5.700000in}}%
\pgfusepath{clip}%
\pgfsetbuttcap%
\pgfsetroundjoin%
\definecolor{currentfill}{rgb}{0.275191,0.194905,0.496005}%
\pgfsetfillcolor{currentfill}%
\pgfsetfillopacity{0.700000}%
\pgfsetlinewidth{0.000000pt}%
\definecolor{currentstroke}{rgb}{0.000000,0.000000,0.000000}%
\pgfsetstrokecolor{currentstroke}%
\pgfsetdash{}{0pt}%
\pgfpathmoveto{\pgfqpoint{4.442337in}{1.530416in}}%
\pgfpathlineto{\pgfqpoint{4.456780in}{1.532134in}}%
\pgfpathlineto{\pgfqpoint{4.471235in}{1.533923in}}%
\pgfpathlineto{\pgfqpoint{4.485699in}{1.535782in}}%
\pgfpathlineto{\pgfqpoint{4.500175in}{1.537712in}}%
\pgfpathlineto{\pgfqpoint{4.508323in}{1.551250in}}%
\pgfpathlineto{\pgfqpoint{4.516465in}{1.564749in}}%
\pgfpathlineto{\pgfqpoint{4.524603in}{1.578205in}}%
\pgfpathlineto{\pgfqpoint{4.532735in}{1.591614in}}%
\pgfpathlineto{\pgfqpoint{4.518263in}{1.589430in}}%
\pgfpathlineto{\pgfqpoint{4.503801in}{1.587317in}}%
\pgfpathlineto{\pgfqpoint{4.489351in}{1.585274in}}%
\pgfpathlineto{\pgfqpoint{4.474911in}{1.583302in}}%
\pgfpathlineto{\pgfqpoint{4.466775in}{1.570139in}}%
\pgfpathlineto{\pgfqpoint{4.458634in}{1.556934in}}%
\pgfpathlineto{\pgfqpoint{4.450488in}{1.543692in}}%
\pgfpathlineto{\pgfqpoint{4.442337in}{1.530416in}}%
\pgfpathclose%
\pgfusepath{fill}%
\end{pgfscope}%
\begin{pgfscope}%
\pgfpathrectangle{\pgfqpoint{1.150000in}{0.150000in}}{\pgfqpoint{5.700000in}{5.700000in}}%
\pgfusepath{clip}%
\pgfsetbuttcap%
\pgfsetroundjoin%
\definecolor{currentfill}{rgb}{0.273809,0.031497,0.358853}%
\pgfsetfillcolor{currentfill}%
\pgfsetfillopacity{0.700000}%
\pgfsetlinewidth{0.000000pt}%
\definecolor{currentstroke}{rgb}{0.000000,0.000000,0.000000}%
\pgfsetstrokecolor{currentstroke}%
\pgfsetdash{}{0pt}%
\pgfpathmoveto{\pgfqpoint{3.810253in}{1.208222in}}%
\pgfpathlineto{\pgfqpoint{3.824499in}{1.205628in}}%
\pgfpathlineto{\pgfqpoint{3.838752in}{1.203104in}}%
\pgfpathlineto{\pgfqpoint{3.853013in}{1.200651in}}%
\pgfpathlineto{\pgfqpoint{3.867281in}{1.198269in}}%
\pgfpathlineto{\pgfqpoint{3.875627in}{1.208271in}}%
\pgfpathlineto{\pgfqpoint{3.883967in}{1.218404in}}%
\pgfpathlineto{\pgfqpoint{3.892300in}{1.228663in}}%
\pgfpathlineto{\pgfqpoint{3.900627in}{1.239041in}}%
\pgfpathlineto{\pgfqpoint{3.886372in}{1.241029in}}%
\pgfpathlineto{\pgfqpoint{3.872124in}{1.243087in}}%
\pgfpathlineto{\pgfqpoint{3.857884in}{1.245216in}}%
\pgfpathlineto{\pgfqpoint{3.843651in}{1.247416in}}%
\pgfpathlineto{\pgfqpoint{3.835312in}{1.237424in}}%
\pgfpathlineto{\pgfqpoint{3.826965in}{1.227557in}}%
\pgfpathlineto{\pgfqpoint{3.818612in}{1.217821in}}%
\pgfpathlineto{\pgfqpoint{3.810253in}{1.208222in}}%
\pgfpathclose%
\pgfusepath{fill}%
\end{pgfscope}%
\begin{pgfscope}%
\pgfpathrectangle{\pgfqpoint{1.150000in}{0.150000in}}{\pgfqpoint{5.700000in}{5.700000in}}%
\pgfusepath{clip}%
\pgfsetbuttcap%
\pgfsetroundjoin%
\definecolor{currentfill}{rgb}{0.225863,0.330805,0.547314}%
\pgfsetfillcolor{currentfill}%
\pgfsetfillopacity{0.700000}%
\pgfsetlinewidth{0.000000pt}%
\definecolor{currentstroke}{rgb}{0.000000,0.000000,0.000000}%
\pgfsetstrokecolor{currentstroke}%
\pgfsetdash{}{0pt}%
\pgfpathmoveto{\pgfqpoint{4.836530in}{1.838670in}}%
\pgfpathlineto{\pgfqpoint{4.851143in}{1.842581in}}%
\pgfpathlineto{\pgfqpoint{4.865769in}{1.846562in}}%
\pgfpathlineto{\pgfqpoint{4.880407in}{1.850615in}}%
\pgfpathlineto{\pgfqpoint{4.895058in}{1.854738in}}%
\pgfpathlineto{\pgfqpoint{4.903081in}{1.867524in}}%
\pgfpathlineto{\pgfqpoint{4.911098in}{1.880193in}}%
\pgfpathlineto{\pgfqpoint{4.919108in}{1.892743in}}%
\pgfpathlineto{\pgfqpoint{4.927111in}{1.905172in}}%
\pgfpathlineto{\pgfqpoint{4.912463in}{1.900900in}}%
\pgfpathlineto{\pgfqpoint{4.897828in}{1.896699in}}%
\pgfpathlineto{\pgfqpoint{4.883206in}{1.892568in}}%
\pgfpathlineto{\pgfqpoint{4.868596in}{1.888510in}}%
\pgfpathlineto{\pgfqpoint{4.860589in}{1.876221in}}%
\pgfpathlineto{\pgfqpoint{4.852576in}{1.863817in}}%
\pgfpathlineto{\pgfqpoint{4.844556in}{1.851300in}}%
\pgfpathlineto{\pgfqpoint{4.836530in}{1.838670in}}%
\pgfpathclose%
\pgfusepath{fill}%
\end{pgfscope}%
\begin{pgfscope}%
\pgfpathrectangle{\pgfqpoint{1.150000in}{0.150000in}}{\pgfqpoint{5.700000in}{5.700000in}}%
\pgfusepath{clip}%
\pgfsetbuttcap%
\pgfsetroundjoin%
\definecolor{currentfill}{rgb}{0.266580,0.228262,0.514349}%
\pgfsetfillcolor{currentfill}%
\pgfsetfillopacity{0.700000}%
\pgfsetlinewidth{0.000000pt}%
\definecolor{currentstroke}{rgb}{0.000000,0.000000,0.000000}%
\pgfsetstrokecolor{currentstroke}%
\pgfsetdash{}{0pt}%
\pgfpathmoveto{\pgfqpoint{4.532735in}{1.591614in}}%
\pgfpathlineto{\pgfqpoint{4.547219in}{1.593868in}}%
\pgfpathlineto{\pgfqpoint{4.561713in}{1.596192in}}%
\pgfpathlineto{\pgfqpoint{4.576219in}{1.598587in}}%
\pgfpathlineto{\pgfqpoint{4.590736in}{1.601053in}}%
\pgfpathlineto{\pgfqpoint{4.598860in}{1.614653in}}%
\pgfpathlineto{\pgfqpoint{4.606979in}{1.628195in}}%
\pgfpathlineto{\pgfqpoint{4.615093in}{1.641675in}}%
\pgfpathlineto{\pgfqpoint{4.623202in}{1.655090in}}%
\pgfpathlineto{\pgfqpoint{4.608687in}{1.652391in}}%
\pgfpathlineto{\pgfqpoint{4.594185in}{1.649763in}}%
\pgfpathlineto{\pgfqpoint{4.579693in}{1.647205in}}%
\pgfpathlineto{\pgfqpoint{4.565213in}{1.644718in}}%
\pgfpathlineto{\pgfqpoint{4.557101in}{1.631528in}}%
\pgfpathlineto{\pgfqpoint{4.548985in}{1.618279in}}%
\pgfpathlineto{\pgfqpoint{4.540862in}{1.604973in}}%
\pgfpathlineto{\pgfqpoint{4.532735in}{1.591614in}}%
\pgfpathclose%
\pgfusepath{fill}%
\end{pgfscope}%
\begin{pgfscope}%
\pgfpathrectangle{\pgfqpoint{1.150000in}{0.150000in}}{\pgfqpoint{5.700000in}{5.700000in}}%
\pgfusepath{clip}%
\pgfsetbuttcap%
\pgfsetroundjoin%
\definecolor{currentfill}{rgb}{0.212395,0.359683,0.551710}%
\pgfsetfillcolor{currentfill}%
\pgfsetfillopacity{0.700000}%
\pgfsetlinewidth{0.000000pt}%
\definecolor{currentstroke}{rgb}{0.000000,0.000000,0.000000}%
\pgfsetstrokecolor{currentstroke}%
\pgfsetdash{}{0pt}%
\pgfpathmoveto{\pgfqpoint{4.927111in}{1.905172in}}%
\pgfpathlineto{\pgfqpoint{4.941771in}{1.909516in}}%
\pgfpathlineto{\pgfqpoint{4.956445in}{1.913931in}}%
\pgfpathlineto{\pgfqpoint{4.971131in}{1.918417in}}%
\pgfpathlineto{\pgfqpoint{4.985830in}{1.922975in}}%
\pgfpathlineto{\pgfqpoint{4.993823in}{1.935417in}}%
\pgfpathlineto{\pgfqpoint{5.001809in}{1.947729in}}%
\pgfpathlineto{\pgfqpoint{5.009788in}{1.959909in}}%
\pgfpathlineto{\pgfqpoint{5.017759in}{1.971956in}}%
\pgfpathlineto{\pgfqpoint{5.003063in}{1.967271in}}%
\pgfpathlineto{\pgfqpoint{4.988381in}{1.962658in}}%
\pgfpathlineto{\pgfqpoint{4.973711in}{1.958116in}}%
\pgfpathlineto{\pgfqpoint{4.959054in}{1.953645in}}%
\pgfpathlineto{\pgfqpoint{4.951079in}{1.941717in}}%
\pgfpathlineto{\pgfqpoint{4.943096in}{1.929660in}}%
\pgfpathlineto{\pgfqpoint{4.935107in}{1.917479in}}%
\pgfpathlineto{\pgfqpoint{4.927111in}{1.905172in}}%
\pgfpathclose%
\pgfusepath{fill}%
\end{pgfscope}%
\begin{pgfscope}%
\pgfpathrectangle{\pgfqpoint{1.150000in}{0.150000in}}{\pgfqpoint{5.700000in}{5.700000in}}%
\pgfusepath{clip}%
\pgfsetbuttcap%
\pgfsetroundjoin%
\definecolor{currentfill}{rgb}{0.271305,0.019942,0.347269}%
\pgfsetfillcolor{currentfill}%
\pgfsetfillopacity{0.700000}%
\pgfsetlinewidth{0.000000pt}%
\definecolor{currentstroke}{rgb}{0.000000,0.000000,0.000000}%
\pgfsetstrokecolor{currentstroke}%
\pgfsetdash{}{0pt}%
\pgfpathmoveto{\pgfqpoint{3.719770in}{1.184068in}}%
\pgfpathlineto{\pgfqpoint{3.734003in}{1.180774in}}%
\pgfpathlineto{\pgfqpoint{3.748242in}{1.177551in}}%
\pgfpathlineto{\pgfqpoint{3.762489in}{1.174399in}}%
\pgfpathlineto{\pgfqpoint{3.776742in}{1.171319in}}%
\pgfpathlineto{\pgfqpoint{3.785131in}{1.180308in}}%
\pgfpathlineto{\pgfqpoint{3.793512in}{1.189459in}}%
\pgfpathlineto{\pgfqpoint{3.801886in}{1.198766in}}%
\pgfpathlineto{\pgfqpoint{3.810253in}{1.208222in}}%
\pgfpathlineto{\pgfqpoint{3.796014in}{1.210887in}}%
\pgfpathlineto{\pgfqpoint{3.781783in}{1.213624in}}%
\pgfpathlineto{\pgfqpoint{3.767558in}{1.216432in}}%
\pgfpathlineto{\pgfqpoint{3.753341in}{1.219311in}}%
\pgfpathlineto{\pgfqpoint{3.744960in}{1.210263in}}%
\pgfpathlineto{\pgfqpoint{3.736571in}{1.201368in}}%
\pgfpathlineto{\pgfqpoint{3.728174in}{1.192634in}}%
\pgfpathlineto{\pgfqpoint{3.719770in}{1.184068in}}%
\pgfpathclose%
\pgfusepath{fill}%
\end{pgfscope}%
\begin{pgfscope}%
\pgfpathrectangle{\pgfqpoint{1.150000in}{0.150000in}}{\pgfqpoint{5.700000in}{5.700000in}}%
\pgfusepath{clip}%
\pgfsetbuttcap%
\pgfsetroundjoin%
\definecolor{currentfill}{rgb}{0.257322,0.256130,0.526563}%
\pgfsetfillcolor{currentfill}%
\pgfsetfillopacity{0.700000}%
\pgfsetlinewidth{0.000000pt}%
\definecolor{currentstroke}{rgb}{0.000000,0.000000,0.000000}%
\pgfsetstrokecolor{currentstroke}%
\pgfsetdash{}{0pt}%
\pgfpathmoveto{\pgfqpoint{4.623202in}{1.655090in}}%
\pgfpathlineto{\pgfqpoint{4.637727in}{1.657860in}}%
\pgfpathlineto{\pgfqpoint{4.652265in}{1.660700in}}%
\pgfpathlineto{\pgfqpoint{4.666814in}{1.663610in}}%
\pgfpathlineto{\pgfqpoint{4.681374in}{1.666591in}}%
\pgfpathlineto{\pgfqpoint{4.689475in}{1.680159in}}%
\pgfpathlineto{\pgfqpoint{4.697570in}{1.693650in}}%
\pgfpathlineto{\pgfqpoint{4.705659in}{1.707061in}}%
\pgfpathlineto{\pgfqpoint{4.713743in}{1.720390in}}%
\pgfpathlineto{\pgfqpoint{4.699184in}{1.717196in}}%
\pgfpathlineto{\pgfqpoint{4.684638in}{1.714072in}}%
\pgfpathlineto{\pgfqpoint{4.670104in}{1.711020in}}%
\pgfpathlineto{\pgfqpoint{4.655581in}{1.708038in}}%
\pgfpathlineto{\pgfqpoint{4.647494in}{1.694914in}}%
\pgfpathlineto{\pgfqpoint{4.639402in}{1.681712in}}%
\pgfpathlineto{\pgfqpoint{4.631305in}{1.668437in}}%
\pgfpathlineto{\pgfqpoint{4.623202in}{1.655090in}}%
\pgfpathclose%
\pgfusepath{fill}%
\end{pgfscope}%
\begin{pgfscope}%
\pgfpathrectangle{\pgfqpoint{1.150000in}{0.150000in}}{\pgfqpoint{5.700000in}{5.700000in}}%
\pgfusepath{clip}%
\pgfsetbuttcap%
\pgfsetroundjoin%
\definecolor{currentfill}{rgb}{0.199430,0.387607,0.554642}%
\pgfsetfillcolor{currentfill}%
\pgfsetfillopacity{0.700000}%
\pgfsetlinewidth{0.000000pt}%
\definecolor{currentstroke}{rgb}{0.000000,0.000000,0.000000}%
\pgfsetstrokecolor{currentstroke}%
\pgfsetdash{}{0pt}%
\pgfpathmoveto{\pgfqpoint{5.017759in}{1.971956in}}%
\pgfpathlineto{\pgfqpoint{5.032468in}{1.976713in}}%
\pgfpathlineto{\pgfqpoint{5.047190in}{1.981541in}}%
\pgfpathlineto{\pgfqpoint{5.061926in}{1.986440in}}%
\pgfpathlineto{\pgfqpoint{5.076675in}{1.991411in}}%
\pgfpathlineto{\pgfqpoint{5.084635in}{2.003438in}}%
\pgfpathlineto{\pgfqpoint{5.092587in}{2.015323in}}%
\pgfpathlineto{\pgfqpoint{5.100532in}{2.027067in}}%
\pgfpathlineto{\pgfqpoint{5.108469in}{2.038666in}}%
\pgfpathlineto{\pgfqpoint{5.093724in}{2.033590in}}%
\pgfpathlineto{\pgfqpoint{5.078993in}{2.028585in}}%
\pgfpathlineto{\pgfqpoint{5.064275in}{2.023652in}}%
\pgfpathlineto{\pgfqpoint{5.049570in}{2.018790in}}%
\pgfpathlineto{\pgfqpoint{5.041629in}{2.007288in}}%
\pgfpathlineto{\pgfqpoint{5.033680in}{1.995647in}}%
\pgfpathlineto{\pgfqpoint{5.025723in}{1.983870in}}%
\pgfpathlineto{\pgfqpoint{5.017759in}{1.971956in}}%
\pgfpathclose%
\pgfusepath{fill}%
\end{pgfscope}%
\begin{pgfscope}%
\pgfpathrectangle{\pgfqpoint{1.150000in}{0.150000in}}{\pgfqpoint{5.700000in}{5.700000in}}%
\pgfusepath{clip}%
\pgfsetbuttcap%
\pgfsetroundjoin%
\definecolor{currentfill}{rgb}{0.282656,0.100196,0.422160}%
\pgfsetfillcolor{currentfill}%
\pgfsetfillopacity{0.700000}%
\pgfsetlinewidth{0.000000pt}%
\definecolor{currentstroke}{rgb}{0.000000,0.000000,0.000000}%
\pgfsetstrokecolor{currentstroke}%
\pgfsetdash{}{0pt}%
\pgfpathmoveto{\pgfqpoint{4.138488in}{1.316115in}}%
\pgfpathlineto{\pgfqpoint{4.152834in}{1.315773in}}%
\pgfpathlineto{\pgfqpoint{4.167189in}{1.315500in}}%
\pgfpathlineto{\pgfqpoint{4.181553in}{1.315298in}}%
\pgfpathlineto{\pgfqpoint{4.195927in}{1.315165in}}%
\pgfpathlineto{\pgfqpoint{4.204167in}{1.327741in}}%
\pgfpathlineto{\pgfqpoint{4.212401in}{1.340362in}}%
\pgfpathlineto{\pgfqpoint{4.220631in}{1.353024in}}%
\pgfpathlineto{\pgfqpoint{4.228856in}{1.365721in}}%
\pgfpathlineto{\pgfqpoint{4.214489in}{1.365518in}}%
\pgfpathlineto{\pgfqpoint{4.200131in}{1.365385in}}%
\pgfpathlineto{\pgfqpoint{4.185783in}{1.365323in}}%
\pgfpathlineto{\pgfqpoint{4.171444in}{1.365330in}}%
\pgfpathlineto{\pgfqpoint{4.163212in}{1.352961in}}%
\pgfpathlineto{\pgfqpoint{4.154976in}{1.340632in}}%
\pgfpathlineto{\pgfqpoint{4.146734in}{1.328348in}}%
\pgfpathlineto{\pgfqpoint{4.138488in}{1.316115in}}%
\pgfpathclose%
\pgfusepath{fill}%
\end{pgfscope}%
\begin{pgfscope}%
\pgfpathrectangle{\pgfqpoint{1.150000in}{0.150000in}}{\pgfqpoint{5.700000in}{5.700000in}}%
\pgfusepath{clip}%
\pgfsetbuttcap%
\pgfsetroundjoin%
\definecolor{currentfill}{rgb}{0.280894,0.078907,0.402329}%
\pgfsetfillcolor{currentfill}%
\pgfsetfillopacity{0.700000}%
\pgfsetlinewidth{0.000000pt}%
\definecolor{currentstroke}{rgb}{0.000000,0.000000,0.000000}%
\pgfsetstrokecolor{currentstroke}%
\pgfsetdash{}{0pt}%
\pgfpathmoveto{\pgfqpoint{4.048121in}{1.271282in}}%
\pgfpathlineto{\pgfqpoint{4.062439in}{1.270304in}}%
\pgfpathlineto{\pgfqpoint{4.076767in}{1.269395in}}%
\pgfpathlineto{\pgfqpoint{4.091103in}{1.268557in}}%
\pgfpathlineto{\pgfqpoint{4.105448in}{1.267789in}}%
\pgfpathlineto{\pgfqpoint{4.113716in}{1.279769in}}%
\pgfpathlineto{\pgfqpoint{4.121978in}{1.291821in}}%
\pgfpathlineto{\pgfqpoint{4.130236in}{1.303938in}}%
\pgfpathlineto{\pgfqpoint{4.138488in}{1.316115in}}%
\pgfpathlineto{\pgfqpoint{4.124151in}{1.316528in}}%
\pgfpathlineto{\pgfqpoint{4.109823in}{1.317011in}}%
\pgfpathlineto{\pgfqpoint{4.095503in}{1.317564in}}%
\pgfpathlineto{\pgfqpoint{4.081193in}{1.318187in}}%
\pgfpathlineto{\pgfqpoint{4.072933in}{1.306357in}}%
\pgfpathlineto{\pgfqpoint{4.064668in}{1.294593in}}%
\pgfpathlineto{\pgfqpoint{4.056397in}{1.282899in}}%
\pgfpathlineto{\pgfqpoint{4.048121in}{1.271282in}}%
\pgfpathclose%
\pgfusepath{fill}%
\end{pgfscope}%
\begin{pgfscope}%
\pgfpathrectangle{\pgfqpoint{1.150000in}{0.150000in}}{\pgfqpoint{5.700000in}{5.700000in}}%
\pgfusepath{clip}%
\pgfsetbuttcap%
\pgfsetroundjoin%
\definecolor{currentfill}{rgb}{0.283187,0.125848,0.444960}%
\pgfsetfillcolor{currentfill}%
\pgfsetfillopacity{0.700000}%
\pgfsetlinewidth{0.000000pt}%
\definecolor{currentstroke}{rgb}{0.000000,0.000000,0.000000}%
\pgfsetstrokecolor{currentstroke}%
\pgfsetdash{}{0pt}%
\pgfpathmoveto{\pgfqpoint{4.228856in}{1.365721in}}%
\pgfpathlineto{\pgfqpoint{4.243233in}{1.365994in}}%
\pgfpathlineto{\pgfqpoint{4.257619in}{1.366336in}}%
\pgfpathlineto{\pgfqpoint{4.272015in}{1.366749in}}%
\pgfpathlineto{\pgfqpoint{4.286421in}{1.367232in}}%
\pgfpathlineto{\pgfqpoint{4.294635in}{1.380283in}}%
\pgfpathlineto{\pgfqpoint{4.302844in}{1.393354in}}%
\pgfpathlineto{\pgfqpoint{4.311049in}{1.406442in}}%
\pgfpathlineto{\pgfqpoint{4.319248in}{1.419542in}}%
\pgfpathlineto{\pgfqpoint{4.304848in}{1.418743in}}%
\pgfpathlineto{\pgfqpoint{4.290457in}{1.418015in}}%
\pgfpathlineto{\pgfqpoint{4.276076in}{1.417357in}}%
\pgfpathlineto{\pgfqpoint{4.261706in}{1.416769in}}%
\pgfpathlineto{\pgfqpoint{4.253501in}{1.403977in}}%
\pgfpathlineto{\pgfqpoint{4.245291in}{1.391202in}}%
\pgfpathlineto{\pgfqpoint{4.237076in}{1.378448in}}%
\pgfpathlineto{\pgfqpoint{4.228856in}{1.365721in}}%
\pgfpathclose%
\pgfusepath{fill}%
\end{pgfscope}%
\begin{pgfscope}%
\pgfpathrectangle{\pgfqpoint{1.150000in}{0.150000in}}{\pgfqpoint{5.700000in}{5.700000in}}%
\pgfusepath{clip}%
\pgfsetbuttcap%
\pgfsetroundjoin%
\definecolor{currentfill}{rgb}{0.277941,0.056324,0.381191}%
\pgfsetfillcolor{currentfill}%
\pgfsetfillopacity{0.700000}%
\pgfsetlinewidth{0.000000pt}%
\definecolor{currentstroke}{rgb}{0.000000,0.000000,0.000000}%
\pgfsetstrokecolor{currentstroke}%
\pgfsetdash{}{0pt}%
\pgfpathmoveto{\pgfqpoint{3.957729in}{1.231799in}}%
\pgfpathlineto{\pgfqpoint{3.972024in}{1.230165in}}%
\pgfpathlineto{\pgfqpoint{3.986327in}{1.228601in}}%
\pgfpathlineto{\pgfqpoint{4.000639in}{1.227107in}}%
\pgfpathlineto{\pgfqpoint{4.014959in}{1.225683in}}%
\pgfpathlineto{\pgfqpoint{4.023258in}{1.236941in}}%
\pgfpathlineto{\pgfqpoint{4.031551in}{1.248297in}}%
\pgfpathlineto{\pgfqpoint{4.039839in}{1.259746in}}%
\pgfpathlineto{\pgfqpoint{4.048121in}{1.271282in}}%
\pgfpathlineto{\pgfqpoint{4.033810in}{1.272330in}}%
\pgfpathlineto{\pgfqpoint{4.019508in}{1.273449in}}%
\pgfpathlineto{\pgfqpoint{4.005215in}{1.274638in}}%
\pgfpathlineto{\pgfqpoint{3.990930in}{1.275898in}}%
\pgfpathlineto{\pgfqpoint{3.982639in}{1.264729in}}%
\pgfpathlineto{\pgfqpoint{3.974341in}{1.253653in}}%
\pgfpathlineto{\pgfqpoint{3.966038in}{1.242675in}}%
\pgfpathlineto{\pgfqpoint{3.957729in}{1.231799in}}%
\pgfpathclose%
\pgfusepath{fill}%
\end{pgfscope}%
\begin{pgfscope}%
\pgfpathrectangle{\pgfqpoint{1.150000in}{0.150000in}}{\pgfqpoint{5.700000in}{5.700000in}}%
\pgfusepath{clip}%
\pgfsetbuttcap%
\pgfsetroundjoin%
\definecolor{currentfill}{rgb}{0.244972,0.287675,0.537260}%
\pgfsetfillcolor{currentfill}%
\pgfsetfillopacity{0.700000}%
\pgfsetlinewidth{0.000000pt}%
\definecolor{currentstroke}{rgb}{0.000000,0.000000,0.000000}%
\pgfsetstrokecolor{currentstroke}%
\pgfsetdash{}{0pt}%
\pgfpathmoveto{\pgfqpoint{4.713743in}{1.720390in}}%
\pgfpathlineto{\pgfqpoint{4.728313in}{1.723655in}}%
\pgfpathlineto{\pgfqpoint{4.742894in}{1.726990in}}%
\pgfpathlineto{\pgfqpoint{4.757489in}{1.730396in}}%
\pgfpathlineto{\pgfqpoint{4.772095in}{1.733874in}}%
\pgfpathlineto{\pgfqpoint{4.780170in}{1.747317in}}%
\pgfpathlineto{\pgfqpoint{4.788240in}{1.760667in}}%
\pgfpathlineto{\pgfqpoint{4.796303in}{1.773922in}}%
\pgfpathlineto{\pgfqpoint{4.804361in}{1.787078in}}%
\pgfpathlineto{\pgfqpoint{4.789757in}{1.783409in}}%
\pgfpathlineto{\pgfqpoint{4.775166in}{1.779811in}}%
\pgfpathlineto{\pgfqpoint{4.760587in}{1.776283in}}%
\pgfpathlineto{\pgfqpoint{4.746019in}{1.772827in}}%
\pgfpathlineto{\pgfqpoint{4.737959in}{1.759855in}}%
\pgfpathlineto{\pgfqpoint{4.729893in}{1.746789in}}%
\pgfpathlineto{\pgfqpoint{4.721820in}{1.733633in}}%
\pgfpathlineto{\pgfqpoint{4.713743in}{1.720390in}}%
\pgfpathclose%
\pgfusepath{fill}%
\end{pgfscope}%
\begin{pgfscope}%
\pgfpathrectangle{\pgfqpoint{1.150000in}{0.150000in}}{\pgfqpoint{5.700000in}{5.700000in}}%
\pgfusepath{clip}%
\pgfsetbuttcap%
\pgfsetroundjoin%
\definecolor{currentfill}{rgb}{0.281412,0.155834,0.469201}%
\pgfsetfillcolor{currentfill}%
\pgfsetfillopacity{0.700000}%
\pgfsetlinewidth{0.000000pt}%
\definecolor{currentstroke}{rgb}{0.000000,0.000000,0.000000}%
\pgfsetstrokecolor{currentstroke}%
\pgfsetdash{}{0pt}%
\pgfpathmoveto{\pgfqpoint{4.319248in}{1.419542in}}%
\pgfpathlineto{\pgfqpoint{4.333659in}{1.420410in}}%
\pgfpathlineto{\pgfqpoint{4.348079in}{1.421348in}}%
\pgfpathlineto{\pgfqpoint{4.362510in}{1.422357in}}%
\pgfpathlineto{\pgfqpoint{4.376951in}{1.423435in}}%
\pgfpathlineto{\pgfqpoint{4.385141in}{1.436845in}}%
\pgfpathlineto{\pgfqpoint{4.393327in}{1.450253in}}%
\pgfpathlineto{\pgfqpoint{4.401507in}{1.463653in}}%
\pgfpathlineto{\pgfqpoint{4.409683in}{1.477044in}}%
\pgfpathlineto{\pgfqpoint{4.395246in}{1.475670in}}%
\pgfpathlineto{\pgfqpoint{4.380820in}{1.474366in}}%
\pgfpathlineto{\pgfqpoint{4.366404in}{1.473133in}}%
\pgfpathlineto{\pgfqpoint{4.351998in}{1.471969in}}%
\pgfpathlineto{\pgfqpoint{4.343818in}{1.458867in}}%
\pgfpathlineto{\pgfqpoint{4.335633in}{1.445758in}}%
\pgfpathlineto{\pgfqpoint{4.327443in}{1.432648in}}%
\pgfpathlineto{\pgfqpoint{4.319248in}{1.419542in}}%
\pgfpathclose%
\pgfusepath{fill}%
\end{pgfscope}%
\begin{pgfscope}%
\pgfpathrectangle{\pgfqpoint{1.150000in}{0.150000in}}{\pgfqpoint{5.700000in}{5.700000in}}%
\pgfusepath{clip}%
\pgfsetbuttcap%
\pgfsetroundjoin%
\definecolor{currentfill}{rgb}{0.187231,0.414746,0.556547}%
\pgfsetfillcolor{currentfill}%
\pgfsetfillopacity{0.700000}%
\pgfsetlinewidth{0.000000pt}%
\definecolor{currentstroke}{rgb}{0.000000,0.000000,0.000000}%
\pgfsetstrokecolor{currentstroke}%
\pgfsetdash{}{0pt}%
\pgfpathmoveto{\pgfqpoint{5.108469in}{2.038666in}}%
\pgfpathlineto{\pgfqpoint{5.123227in}{2.043814in}}%
\pgfpathlineto{\pgfqpoint{5.137999in}{2.049034in}}%
\pgfpathlineto{\pgfqpoint{5.152785in}{2.054325in}}%
\pgfpathlineto{\pgfqpoint{5.167584in}{2.059689in}}%
\pgfpathlineto{\pgfqpoint{5.175509in}{2.071235in}}%
\pgfpathlineto{\pgfqpoint{5.183425in}{2.082631in}}%
\pgfpathlineto{\pgfqpoint{5.191333in}{2.093874in}}%
\pgfpathlineto{\pgfqpoint{5.199232in}{2.104965in}}%
\pgfpathlineto{\pgfqpoint{5.184438in}{2.099518in}}%
\pgfpathlineto{\pgfqpoint{5.169657in}{2.094144in}}%
\pgfpathlineto{\pgfqpoint{5.154890in}{2.088841in}}%
\pgfpathlineto{\pgfqpoint{5.140136in}{2.083609in}}%
\pgfpathlineto{\pgfqpoint{5.132232in}{2.072594in}}%
\pgfpathlineto{\pgfqpoint{5.124319in}{2.061431in}}%
\pgfpathlineto{\pgfqpoint{5.116398in}{2.050121in}}%
\pgfpathlineto{\pgfqpoint{5.108469in}{2.038666in}}%
\pgfpathclose%
\pgfusepath{fill}%
\end{pgfscope}%
\begin{pgfscope}%
\pgfpathrectangle{\pgfqpoint{1.150000in}{0.150000in}}{\pgfqpoint{5.700000in}{5.700000in}}%
\pgfusepath{clip}%
\pgfsetbuttcap%
\pgfsetroundjoin%
\definecolor{currentfill}{rgb}{0.274952,0.037752,0.364543}%
\pgfsetfillcolor{currentfill}%
\pgfsetfillopacity{0.700000}%
\pgfsetlinewidth{0.000000pt}%
\definecolor{currentstroke}{rgb}{0.000000,0.000000,0.000000}%
\pgfsetstrokecolor{currentstroke}%
\pgfsetdash{}{0pt}%
\pgfpathmoveto{\pgfqpoint{3.867281in}{1.198269in}}%
\pgfpathlineto{\pgfqpoint{3.881557in}{1.195958in}}%
\pgfpathlineto{\pgfqpoint{3.895840in}{1.193717in}}%
\pgfpathlineto{\pgfqpoint{3.910131in}{1.191547in}}%
\pgfpathlineto{\pgfqpoint{3.924430in}{1.189446in}}%
\pgfpathlineto{\pgfqpoint{3.932764in}{1.199851in}}%
\pgfpathlineto{\pgfqpoint{3.941092in}{1.210382in}}%
\pgfpathlineto{\pgfqpoint{3.949413in}{1.221033in}}%
\pgfpathlineto{\pgfqpoint{3.957729in}{1.231799in}}%
\pgfpathlineto{\pgfqpoint{3.943441in}{1.233504in}}%
\pgfpathlineto{\pgfqpoint{3.929162in}{1.235279in}}%
\pgfpathlineto{\pgfqpoint{3.914891in}{1.237125in}}%
\pgfpathlineto{\pgfqpoint{3.900627in}{1.239041in}}%
\pgfpathlineto{\pgfqpoint{3.892300in}{1.228663in}}%
\pgfpathlineto{\pgfqpoint{3.883967in}{1.218404in}}%
\pgfpathlineto{\pgfqpoint{3.875627in}{1.208271in}}%
\pgfpathlineto{\pgfqpoint{3.867281in}{1.198269in}}%
\pgfpathclose%
\pgfusepath{fill}%
\end{pgfscope}%
\begin{pgfscope}%
\pgfpathrectangle{\pgfqpoint{1.150000in}{0.150000in}}{\pgfqpoint{5.700000in}{5.700000in}}%
\pgfusepath{clip}%
\pgfsetbuttcap%
\pgfsetroundjoin%
\definecolor{currentfill}{rgb}{0.277134,0.185228,0.489898}%
\pgfsetfillcolor{currentfill}%
\pgfsetfillopacity{0.700000}%
\pgfsetlinewidth{0.000000pt}%
\definecolor{currentstroke}{rgb}{0.000000,0.000000,0.000000}%
\pgfsetstrokecolor{currentstroke}%
\pgfsetdash{}{0pt}%
\pgfpathmoveto{\pgfqpoint{4.409683in}{1.477044in}}%
\pgfpathlineto{\pgfqpoint{4.424130in}{1.478487in}}%
\pgfpathlineto{\pgfqpoint{4.438588in}{1.480002in}}%
\pgfpathlineto{\pgfqpoint{4.453056in}{1.481586in}}%
\pgfpathlineto{\pgfqpoint{4.467535in}{1.483240in}}%
\pgfpathlineto{\pgfqpoint{4.475702in}{1.496898in}}%
\pgfpathlineto{\pgfqpoint{4.483865in}{1.510532in}}%
\pgfpathlineto{\pgfqpoint{4.492022in}{1.524138in}}%
\pgfpathlineto{\pgfqpoint{4.500175in}{1.537712in}}%
\pgfpathlineto{\pgfqpoint{4.485699in}{1.535782in}}%
\pgfpathlineto{\pgfqpoint{4.471235in}{1.533923in}}%
\pgfpathlineto{\pgfqpoint{4.456780in}{1.532134in}}%
\pgfpathlineto{\pgfqpoint{4.442337in}{1.530416in}}%
\pgfpathlineto{\pgfqpoint{4.434181in}{1.517109in}}%
\pgfpathlineto{\pgfqpoint{4.426020in}{1.503775in}}%
\pgfpathlineto{\pgfqpoint{4.417854in}{1.490419in}}%
\pgfpathlineto{\pgfqpoint{4.409683in}{1.477044in}}%
\pgfpathclose%
\pgfusepath{fill}%
\end{pgfscope}%
\begin{pgfscope}%
\pgfpathrectangle{\pgfqpoint{1.150000in}{0.150000in}}{\pgfqpoint{5.700000in}{5.700000in}}%
\pgfusepath{clip}%
\pgfsetbuttcap%
\pgfsetroundjoin%
\definecolor{currentfill}{rgb}{0.177423,0.437527,0.557565}%
\pgfsetfillcolor{currentfill}%
\pgfsetfillopacity{0.700000}%
\pgfsetlinewidth{0.000000pt}%
\definecolor{currentstroke}{rgb}{0.000000,0.000000,0.000000}%
\pgfsetstrokecolor{currentstroke}%
\pgfsetdash{}{0pt}%
\pgfpathmoveto{\pgfqpoint{5.199232in}{2.104965in}}%
\pgfpathlineto{\pgfqpoint{5.214041in}{2.110483in}}%
\pgfpathlineto{\pgfqpoint{5.228863in}{2.116074in}}%
\pgfpathlineto{\pgfqpoint{5.243699in}{2.121736in}}%
\pgfpathlineto{\pgfqpoint{5.251586in}{2.132725in}}%
\pgfpathlineto{\pgfqpoint{5.259465in}{2.143556in}}%
\pgfpathlineto{\pgfqpoint{5.267334in}{2.154228in}}%
\pgfpathlineto{\pgfqpoint{5.275195in}{2.164740in}}%
\pgfpathlineto{\pgfqpoint{5.260364in}{2.159017in}}%
\pgfpathlineto{\pgfqpoint{5.245547in}{2.153366in}}%
\pgfpathlineto{\pgfqpoint{5.230745in}{2.147786in}}%
\pgfpathlineto{\pgfqpoint{5.222880in}{2.137313in}}%
\pgfpathlineto{\pgfqpoint{5.215006in}{2.126685in}}%
\pgfpathlineto{\pgfqpoint{5.207123in}{2.115902in}}%
\pgfpathlineto{\pgfqpoint{5.199232in}{2.104965in}}%
\pgfpathclose%
\pgfusepath{fill}%
\end{pgfscope}%
\begin{pgfscope}%
\pgfpathrectangle{\pgfqpoint{1.150000in}{0.150000in}}{\pgfqpoint{5.700000in}{5.700000in}}%
\pgfusepath{clip}%
\pgfsetbuttcap%
\pgfsetroundjoin%
\definecolor{currentfill}{rgb}{0.231674,0.318106,0.544834}%
\pgfsetfillcolor{currentfill}%
\pgfsetfillopacity{0.700000}%
\pgfsetlinewidth{0.000000pt}%
\definecolor{currentstroke}{rgb}{0.000000,0.000000,0.000000}%
\pgfsetstrokecolor{currentstroke}%
\pgfsetdash{}{0pt}%
\pgfpathmoveto{\pgfqpoint{4.804361in}{1.787078in}}%
\pgfpathlineto{\pgfqpoint{4.818977in}{1.790818in}}%
\pgfpathlineto{\pgfqpoint{4.833605in}{1.794629in}}%
\pgfpathlineto{\pgfqpoint{4.848246in}{1.798510in}}%
\pgfpathlineto{\pgfqpoint{4.862900in}{1.802463in}}%
\pgfpathlineto{\pgfqpoint{4.870949in}{1.815697in}}%
\pgfpathlineto{\pgfqpoint{4.878992in}{1.828823in}}%
\pgfpathlineto{\pgfqpoint{4.887028in}{1.841837in}}%
\pgfpathlineto{\pgfqpoint{4.895058in}{1.854738in}}%
\pgfpathlineto{\pgfqpoint{4.880407in}{1.850615in}}%
\pgfpathlineto{\pgfqpoint{4.865769in}{1.846562in}}%
\pgfpathlineto{\pgfqpoint{4.851143in}{1.842581in}}%
\pgfpathlineto{\pgfqpoint{4.836530in}{1.838670in}}%
\pgfpathlineto{\pgfqpoint{4.828497in}{1.825931in}}%
\pgfpathlineto{\pgfqpoint{4.820458in}{1.813085in}}%
\pgfpathlineto{\pgfqpoint{4.812413in}{1.800133in}}%
\pgfpathlineto{\pgfqpoint{4.804361in}{1.787078in}}%
\pgfpathclose%
\pgfusepath{fill}%
\end{pgfscope}%
\begin{pgfscope}%
\pgfpathrectangle{\pgfqpoint{1.150000in}{0.150000in}}{\pgfqpoint{5.700000in}{5.700000in}}%
\pgfusepath{clip}%
\pgfsetbuttcap%
\pgfsetroundjoin%
\definecolor{currentfill}{rgb}{0.270595,0.214069,0.507052}%
\pgfsetfillcolor{currentfill}%
\pgfsetfillopacity{0.700000}%
\pgfsetlinewidth{0.000000pt}%
\definecolor{currentstroke}{rgb}{0.000000,0.000000,0.000000}%
\pgfsetstrokecolor{currentstroke}%
\pgfsetdash{}{0pt}%
\pgfpathmoveto{\pgfqpoint{4.500175in}{1.537712in}}%
\pgfpathlineto{\pgfqpoint{4.514662in}{1.539712in}}%
\pgfpathlineto{\pgfqpoint{4.529159in}{1.541782in}}%
\pgfpathlineto{\pgfqpoint{4.543668in}{1.543922in}}%
\pgfpathlineto{\pgfqpoint{4.558188in}{1.546132in}}%
\pgfpathlineto{\pgfqpoint{4.566332in}{1.559933in}}%
\pgfpathlineto{\pgfqpoint{4.574472in}{1.573689in}}%
\pgfpathlineto{\pgfqpoint{4.582607in}{1.587397in}}%
\pgfpathlineto{\pgfqpoint{4.590736in}{1.601053in}}%
\pgfpathlineto{\pgfqpoint{4.576219in}{1.598587in}}%
\pgfpathlineto{\pgfqpoint{4.561713in}{1.596192in}}%
\pgfpathlineto{\pgfqpoint{4.547219in}{1.593868in}}%
\pgfpathlineto{\pgfqpoint{4.532735in}{1.591614in}}%
\pgfpathlineto{\pgfqpoint{4.524603in}{1.578205in}}%
\pgfpathlineto{\pgfqpoint{4.516465in}{1.564749in}}%
\pgfpathlineto{\pgfqpoint{4.508323in}{1.551250in}}%
\pgfpathlineto{\pgfqpoint{4.500175in}{1.537712in}}%
\pgfpathclose%
\pgfusepath{fill}%
\end{pgfscope}%
\begin{pgfscope}%
\pgfpathrectangle{\pgfqpoint{1.150000in}{0.150000in}}{\pgfqpoint{5.700000in}{5.700000in}}%
\pgfusepath{clip}%
\pgfsetbuttcap%
\pgfsetroundjoin%
\definecolor{currentfill}{rgb}{0.272594,0.025563,0.353093}%
\pgfsetfillcolor{currentfill}%
\pgfsetfillopacity{0.700000}%
\pgfsetlinewidth{0.000000pt}%
\definecolor{currentstroke}{rgb}{0.000000,0.000000,0.000000}%
\pgfsetstrokecolor{currentstroke}%
\pgfsetdash{}{0pt}%
\pgfpathmoveto{\pgfqpoint{3.776742in}{1.171319in}}%
\pgfpathlineto{\pgfqpoint{3.791003in}{1.168309in}}%
\pgfpathlineto{\pgfqpoint{3.805271in}{1.165371in}}%
\pgfpathlineto{\pgfqpoint{3.819546in}{1.162502in}}%
\pgfpathlineto{\pgfqpoint{3.833828in}{1.159705in}}%
\pgfpathlineto{\pgfqpoint{3.842201in}{1.169117in}}%
\pgfpathlineto{\pgfqpoint{3.850568in}{1.178687in}}%
\pgfpathlineto{\pgfqpoint{3.858928in}{1.188406in}}%
\pgfpathlineto{\pgfqpoint{3.867281in}{1.198269in}}%
\pgfpathlineto{\pgfqpoint{3.853013in}{1.200651in}}%
\pgfpathlineto{\pgfqpoint{3.838752in}{1.203104in}}%
\pgfpathlineto{\pgfqpoint{3.824499in}{1.205628in}}%
\pgfpathlineto{\pgfqpoint{3.810253in}{1.208222in}}%
\pgfpathlineto{\pgfqpoint{3.801886in}{1.198766in}}%
\pgfpathlineto{\pgfqpoint{3.793512in}{1.189459in}}%
\pgfpathlineto{\pgfqpoint{3.785131in}{1.180308in}}%
\pgfpathlineto{\pgfqpoint{3.776742in}{1.171319in}}%
\pgfpathclose%
\pgfusepath{fill}%
\end{pgfscope}%
\begin{pgfscope}%
\pgfpathrectangle{\pgfqpoint{1.150000in}{0.150000in}}{\pgfqpoint{5.700000in}{5.700000in}}%
\pgfusepath{clip}%
\pgfsetbuttcap%
\pgfsetroundjoin%
\definecolor{currentfill}{rgb}{0.218130,0.347432,0.550038}%
\pgfsetfillcolor{currentfill}%
\pgfsetfillopacity{0.700000}%
\pgfsetlinewidth{0.000000pt}%
\definecolor{currentstroke}{rgb}{0.000000,0.000000,0.000000}%
\pgfsetstrokecolor{currentstroke}%
\pgfsetdash{}{0pt}%
\pgfpathmoveto{\pgfqpoint{4.895058in}{1.854738in}}%
\pgfpathlineto{\pgfqpoint{4.909721in}{1.858933in}}%
\pgfpathlineto{\pgfqpoint{4.924398in}{1.863199in}}%
\pgfpathlineto{\pgfqpoint{4.939087in}{1.867536in}}%
\pgfpathlineto{\pgfqpoint{4.953789in}{1.871944in}}%
\pgfpathlineto{\pgfqpoint{4.961810in}{1.884888in}}%
\pgfpathlineto{\pgfqpoint{4.969823in}{1.897709in}}%
\pgfpathlineto{\pgfqpoint{4.977830in}{1.910405in}}%
\pgfpathlineto{\pgfqpoint{4.985830in}{1.922975in}}%
\pgfpathlineto{\pgfqpoint{4.971131in}{1.918417in}}%
\pgfpathlineto{\pgfqpoint{4.956445in}{1.913931in}}%
\pgfpathlineto{\pgfqpoint{4.941771in}{1.909516in}}%
\pgfpathlineto{\pgfqpoint{4.927111in}{1.905172in}}%
\pgfpathlineto{\pgfqpoint{4.919108in}{1.892743in}}%
\pgfpathlineto{\pgfqpoint{4.911098in}{1.880193in}}%
\pgfpathlineto{\pgfqpoint{4.903081in}{1.867524in}}%
\pgfpathlineto{\pgfqpoint{4.895058in}{1.854738in}}%
\pgfpathclose%
\pgfusepath{fill}%
\end{pgfscope}%
\begin{pgfscope}%
\pgfpathrectangle{\pgfqpoint{1.150000in}{0.150000in}}{\pgfqpoint{5.700000in}{5.700000in}}%
\pgfusepath{clip}%
\pgfsetbuttcap%
\pgfsetroundjoin%
\definecolor{currentfill}{rgb}{0.260571,0.246922,0.522828}%
\pgfsetfillcolor{currentfill}%
\pgfsetfillopacity{0.700000}%
\pgfsetlinewidth{0.000000pt}%
\definecolor{currentstroke}{rgb}{0.000000,0.000000,0.000000}%
\pgfsetstrokecolor{currentstroke}%
\pgfsetdash{}{0pt}%
\pgfpathmoveto{\pgfqpoint{4.590736in}{1.601053in}}%
\pgfpathlineto{\pgfqpoint{4.605264in}{1.603588in}}%
\pgfpathlineto{\pgfqpoint{4.619804in}{1.606194in}}%
\pgfpathlineto{\pgfqpoint{4.634356in}{1.608871in}}%
\pgfpathlineto{\pgfqpoint{4.648918in}{1.611618in}}%
\pgfpathlineto{\pgfqpoint{4.657040in}{1.625460in}}%
\pgfpathlineto{\pgfqpoint{4.665157in}{1.639239in}}%
\pgfpathlineto{\pgfqpoint{4.673268in}{1.652950in}}%
\pgfpathlineto{\pgfqpoint{4.681374in}{1.666591in}}%
\pgfpathlineto{\pgfqpoint{4.666814in}{1.663610in}}%
\pgfpathlineto{\pgfqpoint{4.652265in}{1.660700in}}%
\pgfpathlineto{\pgfqpoint{4.637727in}{1.657860in}}%
\pgfpathlineto{\pgfqpoint{4.623202in}{1.655090in}}%
\pgfpathlineto{\pgfqpoint{4.615093in}{1.641675in}}%
\pgfpathlineto{\pgfqpoint{4.606979in}{1.628195in}}%
\pgfpathlineto{\pgfqpoint{4.598860in}{1.614653in}}%
\pgfpathlineto{\pgfqpoint{4.590736in}{1.601053in}}%
\pgfpathclose%
\pgfusepath{fill}%
\end{pgfscope}%
\begin{pgfscope}%
\pgfpathrectangle{\pgfqpoint{1.150000in}{0.150000in}}{\pgfqpoint{5.700000in}{5.700000in}}%
\pgfusepath{clip}%
\pgfsetbuttcap%
\pgfsetroundjoin%
\definecolor{currentfill}{rgb}{0.281924,0.089666,0.412415}%
\pgfsetfillcolor{currentfill}%
\pgfsetfillopacity{0.700000}%
\pgfsetlinewidth{0.000000pt}%
\definecolor{currentstroke}{rgb}{0.000000,0.000000,0.000000}%
\pgfsetstrokecolor{currentstroke}%
\pgfsetdash{}{0pt}%
\pgfpathmoveto{\pgfqpoint{4.105448in}{1.267789in}}%
\pgfpathlineto{\pgfqpoint{4.119802in}{1.267091in}}%
\pgfpathlineto{\pgfqpoint{4.134165in}{1.266463in}}%
\pgfpathlineto{\pgfqpoint{4.148537in}{1.265904in}}%
\pgfpathlineto{\pgfqpoint{4.162917in}{1.265415in}}%
\pgfpathlineto{\pgfqpoint{4.171177in}{1.277759in}}%
\pgfpathlineto{\pgfqpoint{4.179432in}{1.290169in}}%
\pgfpathlineto{\pgfqpoint{4.187682in}{1.302639in}}%
\pgfpathlineto{\pgfqpoint{4.195927in}{1.315165in}}%
\pgfpathlineto{\pgfqpoint{4.181553in}{1.315298in}}%
\pgfpathlineto{\pgfqpoint{4.167189in}{1.315500in}}%
\pgfpathlineto{\pgfqpoint{4.152834in}{1.315773in}}%
\pgfpathlineto{\pgfqpoint{4.138488in}{1.316115in}}%
\pgfpathlineto{\pgfqpoint{4.130236in}{1.303938in}}%
\pgfpathlineto{\pgfqpoint{4.121978in}{1.291821in}}%
\pgfpathlineto{\pgfqpoint{4.113716in}{1.279769in}}%
\pgfpathlineto{\pgfqpoint{4.105448in}{1.267789in}}%
\pgfpathclose%
\pgfusepath{fill}%
\end{pgfscope}%
\begin{pgfscope}%
\pgfpathrectangle{\pgfqpoint{1.150000in}{0.150000in}}{\pgfqpoint{5.700000in}{5.700000in}}%
\pgfusepath{clip}%
\pgfsetbuttcap%
\pgfsetroundjoin%
\definecolor{currentfill}{rgb}{0.204903,0.375746,0.553533}%
\pgfsetfillcolor{currentfill}%
\pgfsetfillopacity{0.700000}%
\pgfsetlinewidth{0.000000pt}%
\definecolor{currentstroke}{rgb}{0.000000,0.000000,0.000000}%
\pgfsetstrokecolor{currentstroke}%
\pgfsetdash{}{0pt}%
\pgfpathmoveto{\pgfqpoint{4.985830in}{1.922975in}}%
\pgfpathlineto{\pgfqpoint{5.000543in}{1.927604in}}%
\pgfpathlineto{\pgfqpoint{5.015268in}{1.932304in}}%
\pgfpathlineto{\pgfqpoint{5.030007in}{1.937076in}}%
\pgfpathlineto{\pgfqpoint{5.044759in}{1.941920in}}%
\pgfpathlineto{\pgfqpoint{5.052749in}{1.954497in}}%
\pgfpathlineto{\pgfqpoint{5.060732in}{1.966939in}}%
\pgfpathlineto{\pgfqpoint{5.068707in}{1.979244in}}%
\pgfpathlineto{\pgfqpoint{5.076675in}{1.991411in}}%
\pgfpathlineto{\pgfqpoint{5.061926in}{1.986440in}}%
\pgfpathlineto{\pgfqpoint{5.047190in}{1.981541in}}%
\pgfpathlineto{\pgfqpoint{5.032468in}{1.976713in}}%
\pgfpathlineto{\pgfqpoint{5.017759in}{1.971956in}}%
\pgfpathlineto{\pgfqpoint{5.009788in}{1.959909in}}%
\pgfpathlineto{\pgfqpoint{5.001809in}{1.947729in}}%
\pgfpathlineto{\pgfqpoint{4.993823in}{1.935417in}}%
\pgfpathlineto{\pgfqpoint{4.985830in}{1.922975in}}%
\pgfpathclose%
\pgfusepath{fill}%
\end{pgfscope}%
\begin{pgfscope}%
\pgfpathrectangle{\pgfqpoint{1.150000in}{0.150000in}}{\pgfqpoint{5.700000in}{5.700000in}}%
\pgfusepath{clip}%
\pgfsetbuttcap%
\pgfsetroundjoin%
\definecolor{currentfill}{rgb}{0.283197,0.115680,0.436115}%
\pgfsetfillcolor{currentfill}%
\pgfsetfillopacity{0.700000}%
\pgfsetlinewidth{0.000000pt}%
\definecolor{currentstroke}{rgb}{0.000000,0.000000,0.000000}%
\pgfsetstrokecolor{currentstroke}%
\pgfsetdash{}{0pt}%
\pgfpathmoveto{\pgfqpoint{4.195927in}{1.315165in}}%
\pgfpathlineto{\pgfqpoint{4.210310in}{1.315102in}}%
\pgfpathlineto{\pgfqpoint{4.224702in}{1.315109in}}%
\pgfpathlineto{\pgfqpoint{4.239104in}{1.315186in}}%
\pgfpathlineto{\pgfqpoint{4.253515in}{1.315332in}}%
\pgfpathlineto{\pgfqpoint{4.261749in}{1.328252in}}%
\pgfpathlineto{\pgfqpoint{4.269978in}{1.341212in}}%
\pgfpathlineto{\pgfqpoint{4.278202in}{1.354207in}}%
\pgfpathlineto{\pgfqpoint{4.286421in}{1.367232in}}%
\pgfpathlineto{\pgfqpoint{4.272015in}{1.366749in}}%
\pgfpathlineto{\pgfqpoint{4.257619in}{1.366336in}}%
\pgfpathlineto{\pgfqpoint{4.243233in}{1.365994in}}%
\pgfpathlineto{\pgfqpoint{4.228856in}{1.365721in}}%
\pgfpathlineto{\pgfqpoint{4.220631in}{1.353024in}}%
\pgfpathlineto{\pgfqpoint{4.212401in}{1.340362in}}%
\pgfpathlineto{\pgfqpoint{4.204167in}{1.327741in}}%
\pgfpathlineto{\pgfqpoint{4.195927in}{1.315165in}}%
\pgfpathclose%
\pgfusepath{fill}%
\end{pgfscope}%
\begin{pgfscope}%
\pgfpathrectangle{\pgfqpoint{1.150000in}{0.150000in}}{\pgfqpoint{5.700000in}{5.700000in}}%
\pgfusepath{clip}%
\pgfsetbuttcap%
\pgfsetroundjoin%
\definecolor{currentfill}{rgb}{0.250425,0.274290,0.533103}%
\pgfsetfillcolor{currentfill}%
\pgfsetfillopacity{0.700000}%
\pgfsetlinewidth{0.000000pt}%
\definecolor{currentstroke}{rgb}{0.000000,0.000000,0.000000}%
\pgfsetstrokecolor{currentstroke}%
\pgfsetdash{}{0pt}%
\pgfpathmoveto{\pgfqpoint{4.681374in}{1.666591in}}%
\pgfpathlineto{\pgfqpoint{4.695947in}{1.669643in}}%
\pgfpathlineto{\pgfqpoint{4.710531in}{1.672766in}}%
\pgfpathlineto{\pgfqpoint{4.725127in}{1.675958in}}%
\pgfpathlineto{\pgfqpoint{4.739735in}{1.679222in}}%
\pgfpathlineto{\pgfqpoint{4.747833in}{1.693011in}}%
\pgfpathlineto{\pgfqpoint{4.755926in}{1.706718in}}%
\pgfpathlineto{\pgfqpoint{4.764013in}{1.720339in}}%
\pgfpathlineto{\pgfqpoint{4.772095in}{1.733874in}}%
\pgfpathlineto{\pgfqpoint{4.757489in}{1.730396in}}%
\pgfpathlineto{\pgfqpoint{4.742894in}{1.726990in}}%
\pgfpathlineto{\pgfqpoint{4.728313in}{1.723655in}}%
\pgfpathlineto{\pgfqpoint{4.713743in}{1.720390in}}%
\pgfpathlineto{\pgfqpoint{4.705659in}{1.707061in}}%
\pgfpathlineto{\pgfqpoint{4.697570in}{1.693650in}}%
\pgfpathlineto{\pgfqpoint{4.689475in}{1.680159in}}%
\pgfpathlineto{\pgfqpoint{4.681374in}{1.666591in}}%
\pgfpathclose%
\pgfusepath{fill}%
\end{pgfscope}%
\begin{pgfscope}%
\pgfpathrectangle{\pgfqpoint{1.150000in}{0.150000in}}{\pgfqpoint{5.700000in}{5.700000in}}%
\pgfusepath{clip}%
\pgfsetbuttcap%
\pgfsetroundjoin%
\definecolor{currentfill}{rgb}{0.279566,0.067836,0.391917}%
\pgfsetfillcolor{currentfill}%
\pgfsetfillopacity{0.700000}%
\pgfsetlinewidth{0.000000pt}%
\definecolor{currentstroke}{rgb}{0.000000,0.000000,0.000000}%
\pgfsetstrokecolor{currentstroke}%
\pgfsetdash{}{0pt}%
\pgfpathmoveto{\pgfqpoint{4.014959in}{1.225683in}}%
\pgfpathlineto{\pgfqpoint{4.029288in}{1.224329in}}%
\pgfpathlineto{\pgfqpoint{4.043624in}{1.223045in}}%
\pgfpathlineto{\pgfqpoint{4.057970in}{1.221831in}}%
\pgfpathlineto{\pgfqpoint{4.072324in}{1.220687in}}%
\pgfpathlineto{\pgfqpoint{4.080613in}{1.232329in}}%
\pgfpathlineto{\pgfqpoint{4.088897in}{1.244063in}}%
\pgfpathlineto{\pgfqpoint{4.097175in}{1.255885in}}%
\pgfpathlineto{\pgfqpoint{4.105448in}{1.267789in}}%
\pgfpathlineto{\pgfqpoint{4.091103in}{1.268557in}}%
\pgfpathlineto{\pgfqpoint{4.076767in}{1.269395in}}%
\pgfpathlineto{\pgfqpoint{4.062439in}{1.270304in}}%
\pgfpathlineto{\pgfqpoint{4.048121in}{1.271282in}}%
\pgfpathlineto{\pgfqpoint{4.039839in}{1.259746in}}%
\pgfpathlineto{\pgfqpoint{4.031551in}{1.248297in}}%
\pgfpathlineto{\pgfqpoint{4.023258in}{1.236941in}}%
\pgfpathlineto{\pgfqpoint{4.014959in}{1.225683in}}%
\pgfpathclose%
\pgfusepath{fill}%
\end{pgfscope}%
\begin{pgfscope}%
\pgfpathrectangle{\pgfqpoint{1.150000in}{0.150000in}}{\pgfqpoint{5.700000in}{5.700000in}}%
\pgfusepath{clip}%
\pgfsetbuttcap%
\pgfsetroundjoin%
\definecolor{currentfill}{rgb}{0.282623,0.140926,0.457517}%
\pgfsetfillcolor{currentfill}%
\pgfsetfillopacity{0.700000}%
\pgfsetlinewidth{0.000000pt}%
\definecolor{currentstroke}{rgb}{0.000000,0.000000,0.000000}%
\pgfsetstrokecolor{currentstroke}%
\pgfsetdash{}{0pt}%
\pgfpathmoveto{\pgfqpoint{4.286421in}{1.367232in}}%
\pgfpathlineto{\pgfqpoint{4.300836in}{1.367785in}}%
\pgfpathlineto{\pgfqpoint{4.315262in}{1.368407in}}%
\pgfpathlineto{\pgfqpoint{4.329697in}{1.369100in}}%
\pgfpathlineto{\pgfqpoint{4.344142in}{1.369862in}}%
\pgfpathlineto{\pgfqpoint{4.352352in}{1.383236in}}%
\pgfpathlineto{\pgfqpoint{4.360556in}{1.396626in}}%
\pgfpathlineto{\pgfqpoint{4.368756in}{1.410028in}}%
\pgfpathlineto{\pgfqpoint{4.376951in}{1.423435in}}%
\pgfpathlineto{\pgfqpoint{4.362510in}{1.422357in}}%
\pgfpathlineto{\pgfqpoint{4.348079in}{1.421348in}}%
\pgfpathlineto{\pgfqpoint{4.333659in}{1.420410in}}%
\pgfpathlineto{\pgfqpoint{4.319248in}{1.419542in}}%
\pgfpathlineto{\pgfqpoint{4.311049in}{1.406442in}}%
\pgfpathlineto{\pgfqpoint{4.302844in}{1.393354in}}%
\pgfpathlineto{\pgfqpoint{4.294635in}{1.380283in}}%
\pgfpathlineto{\pgfqpoint{4.286421in}{1.367232in}}%
\pgfpathclose%
\pgfusepath{fill}%
\end{pgfscope}%
\begin{pgfscope}%
\pgfpathrectangle{\pgfqpoint{1.150000in}{0.150000in}}{\pgfqpoint{5.700000in}{5.700000in}}%
\pgfusepath{clip}%
\pgfsetbuttcap%
\pgfsetroundjoin%
\definecolor{currentfill}{rgb}{0.276022,0.044167,0.370164}%
\pgfsetfillcolor{currentfill}%
\pgfsetfillopacity{0.700000}%
\pgfsetlinewidth{0.000000pt}%
\definecolor{currentstroke}{rgb}{0.000000,0.000000,0.000000}%
\pgfsetstrokecolor{currentstroke}%
\pgfsetdash{}{0pt}%
\pgfpathmoveto{\pgfqpoint{3.924430in}{1.189446in}}%
\pgfpathlineto{\pgfqpoint{3.938737in}{1.187417in}}%
\pgfpathlineto{\pgfqpoint{3.953052in}{1.185457in}}%
\pgfpathlineto{\pgfqpoint{3.967375in}{1.183567in}}%
\pgfpathlineto{\pgfqpoint{3.981706in}{1.181747in}}%
\pgfpathlineto{\pgfqpoint{3.990028in}{1.192555in}}%
\pgfpathlineto{\pgfqpoint{3.998344in}{1.203484in}}%
\pgfpathlineto{\pgfqpoint{4.006655in}{1.214529in}}%
\pgfpathlineto{\pgfqpoint{4.014959in}{1.225683in}}%
\pgfpathlineto{\pgfqpoint{4.000639in}{1.227107in}}%
\pgfpathlineto{\pgfqpoint{3.986327in}{1.228601in}}%
\pgfpathlineto{\pgfqpoint{3.972024in}{1.230165in}}%
\pgfpathlineto{\pgfqpoint{3.957729in}{1.231799in}}%
\pgfpathlineto{\pgfqpoint{3.949413in}{1.221033in}}%
\pgfpathlineto{\pgfqpoint{3.941092in}{1.210382in}}%
\pgfpathlineto{\pgfqpoint{3.932764in}{1.199851in}}%
\pgfpathlineto{\pgfqpoint{3.924430in}{1.189446in}}%
\pgfpathclose%
\pgfusepath{fill}%
\end{pgfscope}%
\begin{pgfscope}%
\pgfpathrectangle{\pgfqpoint{1.150000in}{0.150000in}}{\pgfqpoint{5.700000in}{5.700000in}}%
\pgfusepath{clip}%
\pgfsetbuttcap%
\pgfsetroundjoin%
\definecolor{currentfill}{rgb}{0.279574,0.170599,0.479997}%
\pgfsetfillcolor{currentfill}%
\pgfsetfillopacity{0.700000}%
\pgfsetlinewidth{0.000000pt}%
\definecolor{currentstroke}{rgb}{0.000000,0.000000,0.000000}%
\pgfsetstrokecolor{currentstroke}%
\pgfsetdash{}{0pt}%
\pgfpathmoveto{\pgfqpoint{4.376951in}{1.423435in}}%
\pgfpathlineto{\pgfqpoint{4.391402in}{1.424584in}}%
\pgfpathlineto{\pgfqpoint{4.405864in}{1.425802in}}%
\pgfpathlineto{\pgfqpoint{4.420336in}{1.427090in}}%
\pgfpathlineto{\pgfqpoint{4.434818in}{1.428448in}}%
\pgfpathlineto{\pgfqpoint{4.443004in}{1.442162in}}%
\pgfpathlineto{\pgfqpoint{4.451186in}{1.455868in}}%
\pgfpathlineto{\pgfqpoint{4.459363in}{1.469562in}}%
\pgfpathlineto{\pgfqpoint{4.467535in}{1.483240in}}%
\pgfpathlineto{\pgfqpoint{4.453056in}{1.481586in}}%
\pgfpathlineto{\pgfqpoint{4.438588in}{1.480002in}}%
\pgfpathlineto{\pgfqpoint{4.424130in}{1.478487in}}%
\pgfpathlineto{\pgfqpoint{4.409683in}{1.477044in}}%
\pgfpathlineto{\pgfqpoint{4.401507in}{1.463653in}}%
\pgfpathlineto{\pgfqpoint{4.393327in}{1.450253in}}%
\pgfpathlineto{\pgfqpoint{4.385141in}{1.436845in}}%
\pgfpathlineto{\pgfqpoint{4.376951in}{1.423435in}}%
\pgfpathclose%
\pgfusepath{fill}%
\end{pgfscope}%
\begin{pgfscope}%
\pgfpathrectangle{\pgfqpoint{1.150000in}{0.150000in}}{\pgfqpoint{5.700000in}{5.700000in}}%
\pgfusepath{clip}%
\pgfsetbuttcap%
\pgfsetroundjoin%
\definecolor{currentfill}{rgb}{0.190631,0.407061,0.556089}%
\pgfsetfillcolor{currentfill}%
\pgfsetfillopacity{0.700000}%
\pgfsetlinewidth{0.000000pt}%
\definecolor{currentstroke}{rgb}{0.000000,0.000000,0.000000}%
\pgfsetstrokecolor{currentstroke}%
\pgfsetdash{}{0pt}%
\pgfpathmoveto{\pgfqpoint{5.076675in}{1.991411in}}%
\pgfpathlineto{\pgfqpoint{5.091437in}{1.996454in}}%
\pgfpathlineto{\pgfqpoint{5.106213in}{2.001568in}}%
\pgfpathlineto{\pgfqpoint{5.121003in}{2.006754in}}%
\pgfpathlineto{\pgfqpoint{5.135806in}{2.012011in}}%
\pgfpathlineto{\pgfqpoint{5.143762in}{2.024152in}}%
\pgfpathlineto{\pgfqpoint{5.151711in}{2.036146in}}%
\pgfpathlineto{\pgfqpoint{5.159652in}{2.047992in}}%
\pgfpathlineto{\pgfqpoint{5.167584in}{2.059689in}}%
\pgfpathlineto{\pgfqpoint{5.152785in}{2.054325in}}%
\pgfpathlineto{\pgfqpoint{5.137999in}{2.049034in}}%
\pgfpathlineto{\pgfqpoint{5.123227in}{2.043814in}}%
\pgfpathlineto{\pgfqpoint{5.108469in}{2.038666in}}%
\pgfpathlineto{\pgfqpoint{5.100532in}{2.027067in}}%
\pgfpathlineto{\pgfqpoint{5.092587in}{2.015323in}}%
\pgfpathlineto{\pgfqpoint{5.084635in}{2.003438in}}%
\pgfpathlineto{\pgfqpoint{5.076675in}{1.991411in}}%
\pgfpathclose%
\pgfusepath{fill}%
\end{pgfscope}%
\begin{pgfscope}%
\pgfpathrectangle{\pgfqpoint{1.150000in}{0.150000in}}{\pgfqpoint{5.700000in}{5.700000in}}%
\pgfusepath{clip}%
\pgfsetbuttcap%
\pgfsetroundjoin%
\definecolor{currentfill}{rgb}{0.237441,0.305202,0.541921}%
\pgfsetfillcolor{currentfill}%
\pgfsetfillopacity{0.700000}%
\pgfsetlinewidth{0.000000pt}%
\definecolor{currentstroke}{rgb}{0.000000,0.000000,0.000000}%
\pgfsetstrokecolor{currentstroke}%
\pgfsetdash{}{0pt}%
\pgfpathmoveto{\pgfqpoint{4.772095in}{1.733874in}}%
\pgfpathlineto{\pgfqpoint{4.786713in}{1.737421in}}%
\pgfpathlineto{\pgfqpoint{4.801344in}{1.741040in}}%
\pgfpathlineto{\pgfqpoint{4.815986in}{1.744730in}}%
\pgfpathlineto{\pgfqpoint{4.830642in}{1.748490in}}%
\pgfpathlineto{\pgfqpoint{4.838715in}{1.762134in}}%
\pgfpathlineto{\pgfqpoint{4.846783in}{1.775679in}}%
\pgfpathlineto{\pgfqpoint{4.854844in}{1.789123in}}%
\pgfpathlineto{\pgfqpoint{4.862900in}{1.802463in}}%
\pgfpathlineto{\pgfqpoint{4.848246in}{1.798510in}}%
\pgfpathlineto{\pgfqpoint{4.833605in}{1.794629in}}%
\pgfpathlineto{\pgfqpoint{4.818977in}{1.790818in}}%
\pgfpathlineto{\pgfqpoint{4.804361in}{1.787078in}}%
\pgfpathlineto{\pgfqpoint{4.796303in}{1.773922in}}%
\pgfpathlineto{\pgfqpoint{4.788240in}{1.760667in}}%
\pgfpathlineto{\pgfqpoint{4.780170in}{1.747317in}}%
\pgfpathlineto{\pgfqpoint{4.772095in}{1.733874in}}%
\pgfpathclose%
\pgfusepath{fill}%
\end{pgfscope}%
\begin{pgfscope}%
\pgfpathrectangle{\pgfqpoint{1.150000in}{0.150000in}}{\pgfqpoint{5.700000in}{5.700000in}}%
\pgfusepath{clip}%
\pgfsetbuttcap%
\pgfsetroundjoin%
\definecolor{currentfill}{rgb}{0.180629,0.429975,0.557282}%
\pgfsetfillcolor{currentfill}%
\pgfsetfillopacity{0.700000}%
\pgfsetlinewidth{0.000000pt}%
\definecolor{currentstroke}{rgb}{0.000000,0.000000,0.000000}%
\pgfsetstrokecolor{currentstroke}%
\pgfsetdash{}{0pt}%
\pgfpathmoveto{\pgfqpoint{5.167584in}{2.059689in}}%
\pgfpathlineto{\pgfqpoint{5.182397in}{2.065124in}}%
\pgfpathlineto{\pgfqpoint{5.197224in}{2.070631in}}%
\pgfpathlineto{\pgfqpoint{5.212066in}{2.076210in}}%
\pgfpathlineto{\pgfqpoint{5.219987in}{2.087825in}}%
\pgfpathlineto{\pgfqpoint{5.227899in}{2.099285in}}%
\pgfpathlineto{\pgfqpoint{5.235804in}{2.110589in}}%
\pgfpathlineto{\pgfqpoint{5.243699in}{2.121736in}}%
\pgfpathlineto{\pgfqpoint{5.228863in}{2.116074in}}%
\pgfpathlineto{\pgfqpoint{5.214041in}{2.110483in}}%
\pgfpathlineto{\pgfqpoint{5.199232in}{2.104965in}}%
\pgfpathlineto{\pgfqpoint{5.191333in}{2.093874in}}%
\pgfpathlineto{\pgfqpoint{5.183425in}{2.082631in}}%
\pgfpathlineto{\pgfqpoint{5.175509in}{2.071235in}}%
\pgfpathlineto{\pgfqpoint{5.167584in}{2.059689in}}%
\pgfpathclose%
\pgfusepath{fill}%
\end{pgfscope}%
\begin{pgfscope}%
\pgfpathrectangle{\pgfqpoint{1.150000in}{0.150000in}}{\pgfqpoint{5.700000in}{5.700000in}}%
\pgfusepath{clip}%
\pgfsetbuttcap%
\pgfsetroundjoin%
\definecolor{currentfill}{rgb}{0.273809,0.031497,0.358853}%
\pgfsetfillcolor{currentfill}%
\pgfsetfillopacity{0.700000}%
\pgfsetlinewidth{0.000000pt}%
\definecolor{currentstroke}{rgb}{0.000000,0.000000,0.000000}%
\pgfsetstrokecolor{currentstroke}%
\pgfsetdash{}{0pt}%
\pgfpathmoveto{\pgfqpoint{3.833828in}{1.159705in}}%
\pgfpathlineto{\pgfqpoint{3.848117in}{1.156978in}}%
\pgfpathlineto{\pgfqpoint{3.862414in}{1.154321in}}%
\pgfpathlineto{\pgfqpoint{3.876719in}{1.151735in}}%
\pgfpathlineto{\pgfqpoint{3.891031in}{1.149218in}}%
\pgfpathlineto{\pgfqpoint{3.899390in}{1.159055in}}%
\pgfpathlineto{\pgfqpoint{3.907744in}{1.169042in}}%
\pgfpathlineto{\pgfqpoint{3.916090in}{1.179175in}}%
\pgfpathlineto{\pgfqpoint{3.924430in}{1.189446in}}%
\pgfpathlineto{\pgfqpoint{3.910131in}{1.191547in}}%
\pgfpathlineto{\pgfqpoint{3.895840in}{1.193717in}}%
\pgfpathlineto{\pgfqpoint{3.881557in}{1.195958in}}%
\pgfpathlineto{\pgfqpoint{3.867281in}{1.198269in}}%
\pgfpathlineto{\pgfqpoint{3.858928in}{1.188406in}}%
\pgfpathlineto{\pgfqpoint{3.850568in}{1.178687in}}%
\pgfpathlineto{\pgfqpoint{3.842201in}{1.169117in}}%
\pgfpathlineto{\pgfqpoint{3.833828in}{1.159705in}}%
\pgfpathclose%
\pgfusepath{fill}%
\end{pgfscope}%
\begin{pgfscope}%
\pgfpathrectangle{\pgfqpoint{1.150000in}{0.150000in}}{\pgfqpoint{5.700000in}{5.700000in}}%
\pgfusepath{clip}%
\pgfsetbuttcap%
\pgfsetroundjoin%
\definecolor{currentfill}{rgb}{0.274128,0.199721,0.498911}%
\pgfsetfillcolor{currentfill}%
\pgfsetfillopacity{0.700000}%
\pgfsetlinewidth{0.000000pt}%
\definecolor{currentstroke}{rgb}{0.000000,0.000000,0.000000}%
\pgfsetstrokecolor{currentstroke}%
\pgfsetdash{}{0pt}%
\pgfpathmoveto{\pgfqpoint{4.467535in}{1.483240in}}%
\pgfpathlineto{\pgfqpoint{4.482025in}{1.484964in}}%
\pgfpathlineto{\pgfqpoint{4.496525in}{1.486759in}}%
\pgfpathlineto{\pgfqpoint{4.511037in}{1.488623in}}%
\pgfpathlineto{\pgfqpoint{4.525559in}{1.490557in}}%
\pgfpathlineto{\pgfqpoint{4.533724in}{1.504499in}}%
\pgfpathlineto{\pgfqpoint{4.541883in}{1.518411in}}%
\pgfpathlineto{\pgfqpoint{4.550038in}{1.532290in}}%
\pgfpathlineto{\pgfqpoint{4.558188in}{1.546132in}}%
\pgfpathlineto{\pgfqpoint{4.543668in}{1.543922in}}%
\pgfpathlineto{\pgfqpoint{4.529159in}{1.541782in}}%
\pgfpathlineto{\pgfqpoint{4.514662in}{1.539712in}}%
\pgfpathlineto{\pgfqpoint{4.500175in}{1.537712in}}%
\pgfpathlineto{\pgfqpoint{4.492022in}{1.524138in}}%
\pgfpathlineto{\pgfqpoint{4.483865in}{1.510532in}}%
\pgfpathlineto{\pgfqpoint{4.475702in}{1.496898in}}%
\pgfpathlineto{\pgfqpoint{4.467535in}{1.483240in}}%
\pgfpathclose%
\pgfusepath{fill}%
\end{pgfscope}%
\begin{pgfscope}%
\pgfpathrectangle{\pgfqpoint{1.150000in}{0.150000in}}{\pgfqpoint{5.700000in}{5.700000in}}%
\pgfusepath{clip}%
\pgfsetbuttcap%
\pgfsetroundjoin%
\definecolor{currentfill}{rgb}{0.221989,0.339161,0.548752}%
\pgfsetfillcolor{currentfill}%
\pgfsetfillopacity{0.700000}%
\pgfsetlinewidth{0.000000pt}%
\definecolor{currentstroke}{rgb}{0.000000,0.000000,0.000000}%
\pgfsetstrokecolor{currentstroke}%
\pgfsetdash{}{0pt}%
\pgfpathmoveto{\pgfqpoint{4.862900in}{1.802463in}}%
\pgfpathlineto{\pgfqpoint{4.877566in}{1.806487in}}%
\pgfpathlineto{\pgfqpoint{4.892244in}{1.810582in}}%
\pgfpathlineto{\pgfqpoint{4.906936in}{1.814748in}}%
\pgfpathlineto{\pgfqpoint{4.921640in}{1.818985in}}%
\pgfpathlineto{\pgfqpoint{4.929687in}{1.832398in}}%
\pgfpathlineto{\pgfqpoint{4.937728in}{1.845697in}}%
\pgfpathlineto{\pgfqpoint{4.945762in}{1.858880in}}%
\pgfpathlineto{\pgfqpoint{4.953789in}{1.871944in}}%
\pgfpathlineto{\pgfqpoint{4.939087in}{1.867536in}}%
\pgfpathlineto{\pgfqpoint{4.924398in}{1.863199in}}%
\pgfpathlineto{\pgfqpoint{4.909721in}{1.858933in}}%
\pgfpathlineto{\pgfqpoint{4.895058in}{1.854738in}}%
\pgfpathlineto{\pgfqpoint{4.887028in}{1.841837in}}%
\pgfpathlineto{\pgfqpoint{4.878992in}{1.828823in}}%
\pgfpathlineto{\pgfqpoint{4.870949in}{1.815697in}}%
\pgfpathlineto{\pgfqpoint{4.862900in}{1.802463in}}%
\pgfpathclose%
\pgfusepath{fill}%
\end{pgfscope}%
\begin{pgfscope}%
\pgfpathrectangle{\pgfqpoint{1.150000in}{0.150000in}}{\pgfqpoint{5.700000in}{5.700000in}}%
\pgfusepath{clip}%
\pgfsetbuttcap%
\pgfsetroundjoin%
\definecolor{currentfill}{rgb}{0.266580,0.228262,0.514349}%
\pgfsetfillcolor{currentfill}%
\pgfsetfillopacity{0.700000}%
\pgfsetlinewidth{0.000000pt}%
\definecolor{currentstroke}{rgb}{0.000000,0.000000,0.000000}%
\pgfsetstrokecolor{currentstroke}%
\pgfsetdash{}{0pt}%
\pgfpathmoveto{\pgfqpoint{4.558188in}{1.546132in}}%
\pgfpathlineto{\pgfqpoint{4.572718in}{1.548413in}}%
\pgfpathlineto{\pgfqpoint{4.587261in}{1.550764in}}%
\pgfpathlineto{\pgfqpoint{4.601814in}{1.553185in}}%
\pgfpathlineto{\pgfqpoint{4.616379in}{1.555676in}}%
\pgfpathlineto{\pgfqpoint{4.624521in}{1.569740in}}%
\pgfpathlineto{\pgfqpoint{4.632659in}{1.583754in}}%
\pgfpathlineto{\pgfqpoint{4.640791in}{1.597714in}}%
\pgfpathlineto{\pgfqpoint{4.648918in}{1.611618in}}%
\pgfpathlineto{\pgfqpoint{4.634356in}{1.608871in}}%
\pgfpathlineto{\pgfqpoint{4.619804in}{1.606194in}}%
\pgfpathlineto{\pgfqpoint{4.605264in}{1.603588in}}%
\pgfpathlineto{\pgfqpoint{4.590736in}{1.601053in}}%
\pgfpathlineto{\pgfqpoint{4.582607in}{1.587397in}}%
\pgfpathlineto{\pgfqpoint{4.574472in}{1.573689in}}%
\pgfpathlineto{\pgfqpoint{4.566332in}{1.559933in}}%
\pgfpathlineto{\pgfqpoint{4.558188in}{1.546132in}}%
\pgfpathclose%
\pgfusepath{fill}%
\end{pgfscope}%
\begin{pgfscope}%
\pgfpathrectangle{\pgfqpoint{1.150000in}{0.150000in}}{\pgfqpoint{5.700000in}{5.700000in}}%
\pgfusepath{clip}%
\pgfsetbuttcap%
\pgfsetroundjoin%
\definecolor{currentfill}{rgb}{0.255645,0.260703,0.528312}%
\pgfsetfillcolor{currentfill}%
\pgfsetfillopacity{0.700000}%
\pgfsetlinewidth{0.000000pt}%
\definecolor{currentstroke}{rgb}{0.000000,0.000000,0.000000}%
\pgfsetstrokecolor{currentstroke}%
\pgfsetdash{}{0pt}%
\pgfpathmoveto{\pgfqpoint{4.648918in}{1.611618in}}%
\pgfpathlineto{\pgfqpoint{4.663493in}{1.614435in}}%
\pgfpathlineto{\pgfqpoint{4.678079in}{1.617323in}}%
\pgfpathlineto{\pgfqpoint{4.692677in}{1.620281in}}%
\pgfpathlineto{\pgfqpoint{4.707287in}{1.623309in}}%
\pgfpathlineto{\pgfqpoint{4.715407in}{1.637394in}}%
\pgfpathlineto{\pgfqpoint{4.723522in}{1.651411in}}%
\pgfpathlineto{\pgfqpoint{4.731631in}{1.665354in}}%
\pgfpathlineto{\pgfqpoint{4.739735in}{1.679222in}}%
\pgfpathlineto{\pgfqpoint{4.725127in}{1.675958in}}%
\pgfpathlineto{\pgfqpoint{4.710531in}{1.672766in}}%
\pgfpathlineto{\pgfqpoint{4.695947in}{1.669643in}}%
\pgfpathlineto{\pgfqpoint{4.681374in}{1.666591in}}%
\pgfpathlineto{\pgfqpoint{4.673268in}{1.652950in}}%
\pgfpathlineto{\pgfqpoint{4.665157in}{1.639239in}}%
\pgfpathlineto{\pgfqpoint{4.657040in}{1.625460in}}%
\pgfpathlineto{\pgfqpoint{4.648918in}{1.611618in}}%
\pgfpathclose%
\pgfusepath{fill}%
\end{pgfscope}%
\begin{pgfscope}%
\pgfpathrectangle{\pgfqpoint{1.150000in}{0.150000in}}{\pgfqpoint{5.700000in}{5.700000in}}%
\pgfusepath{clip}%
\pgfsetbuttcap%
\pgfsetroundjoin%
\definecolor{currentfill}{rgb}{0.208623,0.367752,0.552675}%
\pgfsetfillcolor{currentfill}%
\pgfsetfillopacity{0.700000}%
\pgfsetlinewidth{0.000000pt}%
\definecolor{currentstroke}{rgb}{0.000000,0.000000,0.000000}%
\pgfsetstrokecolor{currentstroke}%
\pgfsetdash{}{0pt}%
\pgfpathmoveto{\pgfqpoint{4.953789in}{1.871944in}}%
\pgfpathlineto{\pgfqpoint{4.968504in}{1.876424in}}%
\pgfpathlineto{\pgfqpoint{4.983232in}{1.880975in}}%
\pgfpathlineto{\pgfqpoint{4.997974in}{1.885597in}}%
\pgfpathlineto{\pgfqpoint{5.012728in}{1.890290in}}%
\pgfpathlineto{\pgfqpoint{5.020747in}{1.903392in}}%
\pgfpathlineto{\pgfqpoint{5.028758in}{1.916365in}}%
\pgfpathlineto{\pgfqpoint{5.036762in}{1.929208in}}%
\pgfpathlineto{\pgfqpoint{5.044759in}{1.941920in}}%
\pgfpathlineto{\pgfqpoint{5.030007in}{1.937076in}}%
\pgfpathlineto{\pgfqpoint{5.015268in}{1.932304in}}%
\pgfpathlineto{\pgfqpoint{5.000543in}{1.927604in}}%
\pgfpathlineto{\pgfqpoint{4.985830in}{1.922975in}}%
\pgfpathlineto{\pgfqpoint{4.977830in}{1.910405in}}%
\pgfpathlineto{\pgfqpoint{4.969823in}{1.897709in}}%
\pgfpathlineto{\pgfqpoint{4.961810in}{1.884888in}}%
\pgfpathlineto{\pgfqpoint{4.953789in}{1.871944in}}%
\pgfpathclose%
\pgfusepath{fill}%
\end{pgfscope}%
\begin{pgfscope}%
\pgfpathrectangle{\pgfqpoint{1.150000in}{0.150000in}}{\pgfqpoint{5.700000in}{5.700000in}}%
\pgfusepath{clip}%
\pgfsetbuttcap%
\pgfsetroundjoin%
\definecolor{currentfill}{rgb}{0.282656,0.100196,0.422160}%
\pgfsetfillcolor{currentfill}%
\pgfsetfillopacity{0.700000}%
\pgfsetlinewidth{0.000000pt}%
\definecolor{currentstroke}{rgb}{0.000000,0.000000,0.000000}%
\pgfsetstrokecolor{currentstroke}%
\pgfsetdash{}{0pt}%
\pgfpathmoveto{\pgfqpoint{4.162917in}{1.265415in}}%
\pgfpathlineto{\pgfqpoint{4.177307in}{1.264996in}}%
\pgfpathlineto{\pgfqpoint{4.191706in}{1.264647in}}%
\pgfpathlineto{\pgfqpoint{4.206115in}{1.264366in}}%
\pgfpathlineto{\pgfqpoint{4.220532in}{1.264156in}}%
\pgfpathlineto{\pgfqpoint{4.228786in}{1.276864in}}%
\pgfpathlineto{\pgfqpoint{4.237034in}{1.289633in}}%
\pgfpathlineto{\pgfqpoint{4.245277in}{1.302458in}}%
\pgfpathlineto{\pgfqpoint{4.253515in}{1.315332in}}%
\pgfpathlineto{\pgfqpoint{4.239104in}{1.315186in}}%
\pgfpathlineto{\pgfqpoint{4.224702in}{1.315109in}}%
\pgfpathlineto{\pgfqpoint{4.210310in}{1.315102in}}%
\pgfpathlineto{\pgfqpoint{4.195927in}{1.315165in}}%
\pgfpathlineto{\pgfqpoint{4.187682in}{1.302639in}}%
\pgfpathlineto{\pgfqpoint{4.179432in}{1.290169in}}%
\pgfpathlineto{\pgfqpoint{4.171177in}{1.277759in}}%
\pgfpathlineto{\pgfqpoint{4.162917in}{1.265415in}}%
\pgfpathclose%
\pgfusepath{fill}%
\end{pgfscope}%
\begin{pgfscope}%
\pgfpathrectangle{\pgfqpoint{1.150000in}{0.150000in}}{\pgfqpoint{5.700000in}{5.700000in}}%
\pgfusepath{clip}%
\pgfsetbuttcap%
\pgfsetroundjoin%
\definecolor{currentfill}{rgb}{0.280894,0.078907,0.402329}%
\pgfsetfillcolor{currentfill}%
\pgfsetfillopacity{0.700000}%
\pgfsetlinewidth{0.000000pt}%
\definecolor{currentstroke}{rgb}{0.000000,0.000000,0.000000}%
\pgfsetstrokecolor{currentstroke}%
\pgfsetdash{}{0pt}%
\pgfpathmoveto{\pgfqpoint{4.072324in}{1.220687in}}%
\pgfpathlineto{\pgfqpoint{4.086686in}{1.219613in}}%
\pgfpathlineto{\pgfqpoint{4.101057in}{1.218608in}}%
\pgfpathlineto{\pgfqpoint{4.115437in}{1.217672in}}%
\pgfpathlineto{\pgfqpoint{4.129826in}{1.216807in}}%
\pgfpathlineto{\pgfqpoint{4.138106in}{1.228833in}}%
\pgfpathlineto{\pgfqpoint{4.146382in}{1.240947in}}%
\pgfpathlineto{\pgfqpoint{4.154652in}{1.253143in}}%
\pgfpathlineto{\pgfqpoint{4.162917in}{1.265415in}}%
\pgfpathlineto{\pgfqpoint{4.148537in}{1.265904in}}%
\pgfpathlineto{\pgfqpoint{4.134165in}{1.266463in}}%
\pgfpathlineto{\pgfqpoint{4.119802in}{1.267091in}}%
\pgfpathlineto{\pgfqpoint{4.105448in}{1.267789in}}%
\pgfpathlineto{\pgfqpoint{4.097175in}{1.255885in}}%
\pgfpathlineto{\pgfqpoint{4.088897in}{1.244063in}}%
\pgfpathlineto{\pgfqpoint{4.080613in}{1.232329in}}%
\pgfpathlineto{\pgfqpoint{4.072324in}{1.220687in}}%
\pgfpathclose%
\pgfusepath{fill}%
\end{pgfscope}%
\begin{pgfscope}%
\pgfpathrectangle{\pgfqpoint{1.150000in}{0.150000in}}{\pgfqpoint{5.700000in}{5.700000in}}%
\pgfusepath{clip}%
\pgfsetbuttcap%
\pgfsetroundjoin%
\definecolor{currentfill}{rgb}{0.283187,0.125848,0.444960}%
\pgfsetfillcolor{currentfill}%
\pgfsetfillopacity{0.700000}%
\pgfsetlinewidth{0.000000pt}%
\definecolor{currentstroke}{rgb}{0.000000,0.000000,0.000000}%
\pgfsetstrokecolor{currentstroke}%
\pgfsetdash{}{0pt}%
\pgfpathmoveto{\pgfqpoint{4.253515in}{1.315332in}}%
\pgfpathlineto{\pgfqpoint{4.267936in}{1.315549in}}%
\pgfpathlineto{\pgfqpoint{4.282367in}{1.315834in}}%
\pgfpathlineto{\pgfqpoint{4.296808in}{1.316190in}}%
\pgfpathlineto{\pgfqpoint{4.311258in}{1.316615in}}%
\pgfpathlineto{\pgfqpoint{4.319486in}{1.329879in}}%
\pgfpathlineto{\pgfqpoint{4.327709in}{1.343178in}}%
\pgfpathlineto{\pgfqpoint{4.335928in}{1.356507in}}%
\pgfpathlineto{\pgfqpoint{4.344142in}{1.369862in}}%
\pgfpathlineto{\pgfqpoint{4.329697in}{1.369100in}}%
\pgfpathlineto{\pgfqpoint{4.315262in}{1.368407in}}%
\pgfpathlineto{\pgfqpoint{4.300836in}{1.367785in}}%
\pgfpathlineto{\pgfqpoint{4.286421in}{1.367232in}}%
\pgfpathlineto{\pgfqpoint{4.278202in}{1.354207in}}%
\pgfpathlineto{\pgfqpoint{4.269978in}{1.341212in}}%
\pgfpathlineto{\pgfqpoint{4.261749in}{1.328252in}}%
\pgfpathlineto{\pgfqpoint{4.253515in}{1.315332in}}%
\pgfpathclose%
\pgfusepath{fill}%
\end{pgfscope}%
\begin{pgfscope}%
\pgfpathrectangle{\pgfqpoint{1.150000in}{0.150000in}}{\pgfqpoint{5.700000in}{5.700000in}}%
\pgfusepath{clip}%
\pgfsetbuttcap%
\pgfsetroundjoin%
\definecolor{currentfill}{rgb}{0.277941,0.056324,0.381191}%
\pgfsetfillcolor{currentfill}%
\pgfsetfillopacity{0.700000}%
\pgfsetlinewidth{0.000000pt}%
\definecolor{currentstroke}{rgb}{0.000000,0.000000,0.000000}%
\pgfsetstrokecolor{currentstroke}%
\pgfsetdash{}{0pt}%
\pgfpathmoveto{\pgfqpoint{3.981706in}{1.181747in}}%
\pgfpathlineto{\pgfqpoint{3.996044in}{1.179997in}}%
\pgfpathlineto{\pgfqpoint{4.010392in}{1.178317in}}%
\pgfpathlineto{\pgfqpoint{4.024747in}{1.176706in}}%
\pgfpathlineto{\pgfqpoint{4.039111in}{1.175165in}}%
\pgfpathlineto{\pgfqpoint{4.047422in}{1.186377in}}%
\pgfpathlineto{\pgfqpoint{4.055728in}{1.197705in}}%
\pgfpathlineto{\pgfqpoint{4.064029in}{1.209144in}}%
\pgfpathlineto{\pgfqpoint{4.072324in}{1.220687in}}%
\pgfpathlineto{\pgfqpoint{4.057970in}{1.221831in}}%
\pgfpathlineto{\pgfqpoint{4.043624in}{1.223045in}}%
\pgfpathlineto{\pgfqpoint{4.029288in}{1.224329in}}%
\pgfpathlineto{\pgfqpoint{4.014959in}{1.225683in}}%
\pgfpathlineto{\pgfqpoint{4.006655in}{1.214529in}}%
\pgfpathlineto{\pgfqpoint{3.998344in}{1.203484in}}%
\pgfpathlineto{\pgfqpoint{3.990028in}{1.192555in}}%
\pgfpathlineto{\pgfqpoint{3.981706in}{1.181747in}}%
\pgfpathclose%
\pgfusepath{fill}%
\end{pgfscope}%
\begin{pgfscope}%
\pgfpathrectangle{\pgfqpoint{1.150000in}{0.150000in}}{\pgfqpoint{5.700000in}{5.700000in}}%
\pgfusepath{clip}%
\pgfsetbuttcap%
\pgfsetroundjoin%
\definecolor{currentfill}{rgb}{0.281412,0.155834,0.469201}%
\pgfsetfillcolor{currentfill}%
\pgfsetfillopacity{0.700000}%
\pgfsetlinewidth{0.000000pt}%
\definecolor{currentstroke}{rgb}{0.000000,0.000000,0.000000}%
\pgfsetstrokecolor{currentstroke}%
\pgfsetdash{}{0pt}%
\pgfpathmoveto{\pgfqpoint{4.344142in}{1.369862in}}%
\pgfpathlineto{\pgfqpoint{4.358598in}{1.370693in}}%
\pgfpathlineto{\pgfqpoint{4.373063in}{1.371595in}}%
\pgfpathlineto{\pgfqpoint{4.387539in}{1.372566in}}%
\pgfpathlineto{\pgfqpoint{4.402025in}{1.373607in}}%
\pgfpathlineto{\pgfqpoint{4.410230in}{1.387306in}}%
\pgfpathlineto{\pgfqpoint{4.418431in}{1.401016in}}%
\pgfpathlineto{\pgfqpoint{4.426627in}{1.414731in}}%
\pgfpathlineto{\pgfqpoint{4.434818in}{1.428448in}}%
\pgfpathlineto{\pgfqpoint{4.420336in}{1.427090in}}%
\pgfpathlineto{\pgfqpoint{4.405864in}{1.425802in}}%
\pgfpathlineto{\pgfqpoint{4.391402in}{1.424584in}}%
\pgfpathlineto{\pgfqpoint{4.376951in}{1.423435in}}%
\pgfpathlineto{\pgfqpoint{4.368756in}{1.410028in}}%
\pgfpathlineto{\pgfqpoint{4.360556in}{1.396626in}}%
\pgfpathlineto{\pgfqpoint{4.352352in}{1.383236in}}%
\pgfpathlineto{\pgfqpoint{4.344142in}{1.369862in}}%
\pgfpathclose%
\pgfusepath{fill}%
\end{pgfscope}%
\begin{pgfscope}%
\pgfpathrectangle{\pgfqpoint{1.150000in}{0.150000in}}{\pgfqpoint{5.700000in}{5.700000in}}%
\pgfusepath{clip}%
\pgfsetbuttcap%
\pgfsetroundjoin%
\definecolor{currentfill}{rgb}{0.243113,0.292092,0.538516}%
\pgfsetfillcolor{currentfill}%
\pgfsetfillopacity{0.700000}%
\pgfsetlinewidth{0.000000pt}%
\definecolor{currentstroke}{rgb}{0.000000,0.000000,0.000000}%
\pgfsetstrokecolor{currentstroke}%
\pgfsetdash{}{0pt}%
\pgfpathmoveto{\pgfqpoint{4.739735in}{1.679222in}}%
\pgfpathlineto{\pgfqpoint{4.754355in}{1.682556in}}%
\pgfpathlineto{\pgfqpoint{4.768987in}{1.685961in}}%
\pgfpathlineto{\pgfqpoint{4.783632in}{1.689436in}}%
\pgfpathlineto{\pgfqpoint{4.798289in}{1.692982in}}%
\pgfpathlineto{\pgfqpoint{4.806386in}{1.706993in}}%
\pgfpathlineto{\pgfqpoint{4.814477in}{1.720917in}}%
\pgfpathlineto{\pgfqpoint{4.822562in}{1.734750in}}%
\pgfpathlineto{\pgfqpoint{4.830642in}{1.748490in}}%
\pgfpathlineto{\pgfqpoint{4.815986in}{1.744730in}}%
\pgfpathlineto{\pgfqpoint{4.801344in}{1.741040in}}%
\pgfpathlineto{\pgfqpoint{4.786713in}{1.737421in}}%
\pgfpathlineto{\pgfqpoint{4.772095in}{1.733874in}}%
\pgfpathlineto{\pgfqpoint{4.764013in}{1.720339in}}%
\pgfpathlineto{\pgfqpoint{4.755926in}{1.706718in}}%
\pgfpathlineto{\pgfqpoint{4.747833in}{1.693011in}}%
\pgfpathlineto{\pgfqpoint{4.739735in}{1.679222in}}%
\pgfpathclose%
\pgfusepath{fill}%
\end{pgfscope}%
\begin{pgfscope}%
\pgfpathrectangle{\pgfqpoint{1.150000in}{0.150000in}}{\pgfqpoint{5.700000in}{5.700000in}}%
\pgfusepath{clip}%
\pgfsetbuttcap%
\pgfsetroundjoin%
\definecolor{currentfill}{rgb}{0.195860,0.395433,0.555276}%
\pgfsetfillcolor{currentfill}%
\pgfsetfillopacity{0.700000}%
\pgfsetlinewidth{0.000000pt}%
\definecolor{currentstroke}{rgb}{0.000000,0.000000,0.000000}%
\pgfsetstrokecolor{currentstroke}%
\pgfsetdash{}{0pt}%
\pgfpathmoveto{\pgfqpoint{5.044759in}{1.941920in}}%
\pgfpathlineto{\pgfqpoint{5.059525in}{1.946834in}}%
\pgfpathlineto{\pgfqpoint{5.074304in}{1.951821in}}%
\pgfpathlineto{\pgfqpoint{5.089097in}{1.956878in}}%
\pgfpathlineto{\pgfqpoint{5.103903in}{1.962008in}}%
\pgfpathlineto{\pgfqpoint{5.111890in}{1.974722in}}%
\pgfpathlineto{\pgfqpoint{5.119870in}{1.987295in}}%
\pgfpathlineto{\pgfqpoint{5.127842in}{1.999725in}}%
\pgfpathlineto{\pgfqpoint{5.135806in}{2.012011in}}%
\pgfpathlineto{\pgfqpoint{5.121003in}{2.006754in}}%
\pgfpathlineto{\pgfqpoint{5.106213in}{2.001568in}}%
\pgfpathlineto{\pgfqpoint{5.091437in}{1.996454in}}%
\pgfpathlineto{\pgfqpoint{5.076675in}{1.991411in}}%
\pgfpathlineto{\pgfqpoint{5.068707in}{1.979244in}}%
\pgfpathlineto{\pgfqpoint{5.060732in}{1.966939in}}%
\pgfpathlineto{\pgfqpoint{5.052749in}{1.954497in}}%
\pgfpathlineto{\pgfqpoint{5.044759in}{1.941920in}}%
\pgfpathclose%
\pgfusepath{fill}%
\end{pgfscope}%
\begin{pgfscope}%
\pgfpathrectangle{\pgfqpoint{1.150000in}{0.150000in}}{\pgfqpoint{5.700000in}{5.700000in}}%
\pgfusepath{clip}%
\pgfsetbuttcap%
\pgfsetroundjoin%
\definecolor{currentfill}{rgb}{0.274952,0.037752,0.364543}%
\pgfsetfillcolor{currentfill}%
\pgfsetfillopacity{0.700000}%
\pgfsetlinewidth{0.000000pt}%
\definecolor{currentstroke}{rgb}{0.000000,0.000000,0.000000}%
\pgfsetstrokecolor{currentstroke}%
\pgfsetdash{}{0pt}%
\pgfpathmoveto{\pgfqpoint{3.891031in}{1.149218in}}%
\pgfpathlineto{\pgfqpoint{3.905350in}{1.146772in}}%
\pgfpathlineto{\pgfqpoint{3.919677in}{1.144396in}}%
\pgfpathlineto{\pgfqpoint{3.934012in}{1.142090in}}%
\pgfpathlineto{\pgfqpoint{3.948355in}{1.139853in}}%
\pgfpathlineto{\pgfqpoint{3.956702in}{1.150113in}}%
\pgfpathlineto{\pgfqpoint{3.965043in}{1.160520in}}%
\pgfpathlineto{\pgfqpoint{3.973377in}{1.171067in}}%
\pgfpathlineto{\pgfqpoint{3.981706in}{1.181747in}}%
\pgfpathlineto{\pgfqpoint{3.967375in}{1.183567in}}%
\pgfpathlineto{\pgfqpoint{3.953052in}{1.185457in}}%
\pgfpathlineto{\pgfqpoint{3.938737in}{1.187417in}}%
\pgfpathlineto{\pgfqpoint{3.924430in}{1.189446in}}%
\pgfpathlineto{\pgfqpoint{3.916090in}{1.179175in}}%
\pgfpathlineto{\pgfqpoint{3.907744in}{1.169042in}}%
\pgfpathlineto{\pgfqpoint{3.899390in}{1.159055in}}%
\pgfpathlineto{\pgfqpoint{3.891031in}{1.149218in}}%
\pgfpathclose%
\pgfusepath{fill}%
\end{pgfscope}%
\begin{pgfscope}%
\pgfpathrectangle{\pgfqpoint{1.150000in}{0.150000in}}{\pgfqpoint{5.700000in}{5.700000in}}%
\pgfusepath{clip}%
\pgfsetbuttcap%
\pgfsetroundjoin%
\definecolor{currentfill}{rgb}{0.277134,0.185228,0.489898}%
\pgfsetfillcolor{currentfill}%
\pgfsetfillopacity{0.700000}%
\pgfsetlinewidth{0.000000pt}%
\definecolor{currentstroke}{rgb}{0.000000,0.000000,0.000000}%
\pgfsetstrokecolor{currentstroke}%
\pgfsetdash{}{0pt}%
\pgfpathmoveto{\pgfqpoint{4.434818in}{1.428448in}}%
\pgfpathlineto{\pgfqpoint{4.449311in}{1.429876in}}%
\pgfpathlineto{\pgfqpoint{4.463815in}{1.431373in}}%
\pgfpathlineto{\pgfqpoint{4.478329in}{1.432941in}}%
\pgfpathlineto{\pgfqpoint{4.492854in}{1.434578in}}%
\pgfpathlineto{\pgfqpoint{4.501037in}{1.448596in}}%
\pgfpathlineto{\pgfqpoint{4.509216in}{1.462602in}}%
\pgfpathlineto{\pgfqpoint{4.517390in}{1.476590in}}%
\pgfpathlineto{\pgfqpoint{4.525559in}{1.490557in}}%
\pgfpathlineto{\pgfqpoint{4.511037in}{1.488623in}}%
\pgfpathlineto{\pgfqpoint{4.496525in}{1.486759in}}%
\pgfpathlineto{\pgfqpoint{4.482025in}{1.484964in}}%
\pgfpathlineto{\pgfqpoint{4.467535in}{1.483240in}}%
\pgfpathlineto{\pgfqpoint{4.459363in}{1.469562in}}%
\pgfpathlineto{\pgfqpoint{4.451186in}{1.455868in}}%
\pgfpathlineto{\pgfqpoint{4.443004in}{1.442162in}}%
\pgfpathlineto{\pgfqpoint{4.434818in}{1.428448in}}%
\pgfpathclose%
\pgfusepath{fill}%
\end{pgfscope}%
\begin{pgfscope}%
\pgfpathrectangle{\pgfqpoint{1.150000in}{0.150000in}}{\pgfqpoint{5.700000in}{5.700000in}}%
\pgfusepath{clip}%
\pgfsetbuttcap%
\pgfsetroundjoin%
\definecolor{currentfill}{rgb}{0.183898,0.422383,0.556944}%
\pgfsetfillcolor{currentfill}%
\pgfsetfillopacity{0.700000}%
\pgfsetlinewidth{0.000000pt}%
\definecolor{currentstroke}{rgb}{0.000000,0.000000,0.000000}%
\pgfsetstrokecolor{currentstroke}%
\pgfsetdash{}{0pt}%
\pgfpathmoveto{\pgfqpoint{5.135806in}{2.012011in}}%
\pgfpathlineto{\pgfqpoint{5.150623in}{2.017340in}}%
\pgfpathlineto{\pgfqpoint{5.165454in}{2.022741in}}%
\pgfpathlineto{\pgfqpoint{5.180298in}{2.028214in}}%
\pgfpathlineto{\pgfqpoint{5.188252in}{2.040441in}}%
\pgfpathlineto{\pgfqpoint{5.196198in}{2.052516in}}%
\pgfpathlineto{\pgfqpoint{5.204136in}{2.064440in}}%
\pgfpathlineto{\pgfqpoint{5.212066in}{2.076210in}}%
\pgfpathlineto{\pgfqpoint{5.197224in}{2.070631in}}%
\pgfpathlineto{\pgfqpoint{5.182397in}{2.065124in}}%
\pgfpathlineto{\pgfqpoint{5.167584in}{2.059689in}}%
\pgfpathlineto{\pgfqpoint{5.159652in}{2.047992in}}%
\pgfpathlineto{\pgfqpoint{5.151711in}{2.036146in}}%
\pgfpathlineto{\pgfqpoint{5.143762in}{2.024152in}}%
\pgfpathlineto{\pgfqpoint{5.135806in}{2.012011in}}%
\pgfpathclose%
\pgfusepath{fill}%
\end{pgfscope}%
\begin{pgfscope}%
\pgfpathrectangle{\pgfqpoint{1.150000in}{0.150000in}}{\pgfqpoint{5.700000in}{5.700000in}}%
\pgfusepath{clip}%
\pgfsetbuttcap%
\pgfsetroundjoin%
\definecolor{currentfill}{rgb}{0.270595,0.214069,0.507052}%
\pgfsetfillcolor{currentfill}%
\pgfsetfillopacity{0.700000}%
\pgfsetlinewidth{0.000000pt}%
\definecolor{currentstroke}{rgb}{0.000000,0.000000,0.000000}%
\pgfsetstrokecolor{currentstroke}%
\pgfsetdash{}{0pt}%
\pgfpathmoveto{\pgfqpoint{4.525559in}{1.490557in}}%
\pgfpathlineto{\pgfqpoint{4.540093in}{1.492562in}}%
\pgfpathlineto{\pgfqpoint{4.554637in}{1.494636in}}%
\pgfpathlineto{\pgfqpoint{4.569193in}{1.496780in}}%
\pgfpathlineto{\pgfqpoint{4.583759in}{1.498994in}}%
\pgfpathlineto{\pgfqpoint{4.591922in}{1.513221in}}%
\pgfpathlineto{\pgfqpoint{4.600079in}{1.527412in}}%
\pgfpathlineto{\pgfqpoint{4.608231in}{1.541565in}}%
\pgfpathlineto{\pgfqpoint{4.616379in}{1.555676in}}%
\pgfpathlineto{\pgfqpoint{4.601814in}{1.553185in}}%
\pgfpathlineto{\pgfqpoint{4.587261in}{1.550764in}}%
\pgfpathlineto{\pgfqpoint{4.572718in}{1.548413in}}%
\pgfpathlineto{\pgfqpoint{4.558188in}{1.546132in}}%
\pgfpathlineto{\pgfqpoint{4.550038in}{1.532290in}}%
\pgfpathlineto{\pgfqpoint{4.541883in}{1.518411in}}%
\pgfpathlineto{\pgfqpoint{4.533724in}{1.504499in}}%
\pgfpathlineto{\pgfqpoint{4.525559in}{1.490557in}}%
\pgfpathclose%
\pgfusepath{fill}%
\end{pgfscope}%
\begin{pgfscope}%
\pgfpathrectangle{\pgfqpoint{1.150000in}{0.150000in}}{\pgfqpoint{5.700000in}{5.700000in}}%
\pgfusepath{clip}%
\pgfsetbuttcap%
\pgfsetroundjoin%
\definecolor{currentfill}{rgb}{0.227802,0.326594,0.546532}%
\pgfsetfillcolor{currentfill}%
\pgfsetfillopacity{0.700000}%
\pgfsetlinewidth{0.000000pt}%
\definecolor{currentstroke}{rgb}{0.000000,0.000000,0.000000}%
\pgfsetstrokecolor{currentstroke}%
\pgfsetdash{}{0pt}%
\pgfpathmoveto{\pgfqpoint{4.830642in}{1.748490in}}%
\pgfpathlineto{\pgfqpoint{4.845310in}{1.752321in}}%
\pgfpathlineto{\pgfqpoint{4.859990in}{1.756223in}}%
\pgfpathlineto{\pgfqpoint{4.874683in}{1.760196in}}%
\pgfpathlineto{\pgfqpoint{4.889388in}{1.764239in}}%
\pgfpathlineto{\pgfqpoint{4.897461in}{1.778085in}}%
\pgfpathlineto{\pgfqpoint{4.905527in}{1.791826in}}%
\pgfpathlineto{\pgfqpoint{4.913586in}{1.805460in}}%
\pgfpathlineto{\pgfqpoint{4.921640in}{1.818985in}}%
\pgfpathlineto{\pgfqpoint{4.906936in}{1.814748in}}%
\pgfpathlineto{\pgfqpoint{4.892244in}{1.810582in}}%
\pgfpathlineto{\pgfqpoint{4.877566in}{1.806487in}}%
\pgfpathlineto{\pgfqpoint{4.862900in}{1.802463in}}%
\pgfpathlineto{\pgfqpoint{4.854844in}{1.789123in}}%
\pgfpathlineto{\pgfqpoint{4.846783in}{1.775679in}}%
\pgfpathlineto{\pgfqpoint{4.838715in}{1.762134in}}%
\pgfpathlineto{\pgfqpoint{4.830642in}{1.748490in}}%
\pgfpathclose%
\pgfusepath{fill}%
\end{pgfscope}%
\begin{pgfscope}%
\pgfpathrectangle{\pgfqpoint{1.150000in}{0.150000in}}{\pgfqpoint{5.700000in}{5.700000in}}%
\pgfusepath{clip}%
\pgfsetbuttcap%
\pgfsetroundjoin%
\definecolor{currentfill}{rgb}{0.260571,0.246922,0.522828}%
\pgfsetfillcolor{currentfill}%
\pgfsetfillopacity{0.700000}%
\pgfsetlinewidth{0.000000pt}%
\definecolor{currentstroke}{rgb}{0.000000,0.000000,0.000000}%
\pgfsetstrokecolor{currentstroke}%
\pgfsetdash{}{0pt}%
\pgfpathmoveto{\pgfqpoint{4.616379in}{1.555676in}}%
\pgfpathlineto{\pgfqpoint{4.630955in}{1.558237in}}%
\pgfpathlineto{\pgfqpoint{4.645543in}{1.560868in}}%
\pgfpathlineto{\pgfqpoint{4.660143in}{1.563570in}}%
\pgfpathlineto{\pgfqpoint{4.674754in}{1.566342in}}%
\pgfpathlineto{\pgfqpoint{4.682895in}{1.580670in}}%
\pgfpathlineto{\pgfqpoint{4.691030in}{1.594943in}}%
\pgfpathlineto{\pgfqpoint{4.699161in}{1.609157in}}%
\pgfpathlineto{\pgfqpoint{4.707287in}{1.623309in}}%
\pgfpathlineto{\pgfqpoint{4.692677in}{1.620281in}}%
\pgfpathlineto{\pgfqpoint{4.678079in}{1.617323in}}%
\pgfpathlineto{\pgfqpoint{4.663493in}{1.614435in}}%
\pgfpathlineto{\pgfqpoint{4.648918in}{1.611618in}}%
\pgfpathlineto{\pgfqpoint{4.640791in}{1.597714in}}%
\pgfpathlineto{\pgfqpoint{4.632659in}{1.583754in}}%
\pgfpathlineto{\pgfqpoint{4.624521in}{1.569740in}}%
\pgfpathlineto{\pgfqpoint{4.616379in}{1.555676in}}%
\pgfpathclose%
\pgfusepath{fill}%
\end{pgfscope}%
\begin{pgfscope}%
\pgfpathrectangle{\pgfqpoint{1.150000in}{0.150000in}}{\pgfqpoint{5.700000in}{5.700000in}}%
\pgfusepath{clip}%
\pgfsetbuttcap%
\pgfsetroundjoin%
\definecolor{currentfill}{rgb}{0.214298,0.355619,0.551184}%
\pgfsetfillcolor{currentfill}%
\pgfsetfillopacity{0.700000}%
\pgfsetlinewidth{0.000000pt}%
\definecolor{currentstroke}{rgb}{0.000000,0.000000,0.000000}%
\pgfsetstrokecolor{currentstroke}%
\pgfsetdash{}{0pt}%
\pgfpathmoveto{\pgfqpoint{4.921640in}{1.818985in}}%
\pgfpathlineto{\pgfqpoint{4.936357in}{1.823293in}}%
\pgfpathlineto{\pgfqpoint{4.951087in}{1.827672in}}%
\pgfpathlineto{\pgfqpoint{4.965830in}{1.832122in}}%
\pgfpathlineto{\pgfqpoint{4.980586in}{1.836644in}}%
\pgfpathlineto{\pgfqpoint{4.988632in}{1.850237in}}%
\pgfpathlineto{\pgfqpoint{4.996671in}{1.863711in}}%
\pgfpathlineto{\pgfqpoint{5.004703in}{1.877063in}}%
\pgfpathlineto{\pgfqpoint{5.012728in}{1.890290in}}%
\pgfpathlineto{\pgfqpoint{4.997974in}{1.885597in}}%
\pgfpathlineto{\pgfqpoint{4.983232in}{1.880975in}}%
\pgfpathlineto{\pgfqpoint{4.968504in}{1.876424in}}%
\pgfpathlineto{\pgfqpoint{4.953789in}{1.871944in}}%
\pgfpathlineto{\pgfqpoint{4.945762in}{1.858880in}}%
\pgfpathlineto{\pgfqpoint{4.937728in}{1.845697in}}%
\pgfpathlineto{\pgfqpoint{4.929687in}{1.832398in}}%
\pgfpathlineto{\pgfqpoint{4.921640in}{1.818985in}}%
\pgfpathclose%
\pgfusepath{fill}%
\end{pgfscope}%
\begin{pgfscope}%
\pgfpathrectangle{\pgfqpoint{1.150000in}{0.150000in}}{\pgfqpoint{5.700000in}{5.700000in}}%
\pgfusepath{clip}%
\pgfsetbuttcap%
\pgfsetroundjoin%
\definecolor{currentfill}{rgb}{0.281924,0.089666,0.412415}%
\pgfsetfillcolor{currentfill}%
\pgfsetfillopacity{0.700000}%
\pgfsetlinewidth{0.000000pt}%
\definecolor{currentstroke}{rgb}{0.000000,0.000000,0.000000}%
\pgfsetstrokecolor{currentstroke}%
\pgfsetdash{}{0pt}%
\pgfpathmoveto{\pgfqpoint{4.129826in}{1.216807in}}%
\pgfpathlineto{\pgfqpoint{4.144223in}{1.216011in}}%
\pgfpathlineto{\pgfqpoint{4.158630in}{1.215284in}}%
\pgfpathlineto{\pgfqpoint{4.173046in}{1.214626in}}%
\pgfpathlineto{\pgfqpoint{4.187470in}{1.214038in}}%
\pgfpathlineto{\pgfqpoint{4.195743in}{1.226449in}}%
\pgfpathlineto{\pgfqpoint{4.204011in}{1.238943in}}%
\pgfpathlineto{\pgfqpoint{4.212274in}{1.251514in}}%
\pgfpathlineto{\pgfqpoint{4.220532in}{1.264156in}}%
\pgfpathlineto{\pgfqpoint{4.206115in}{1.264366in}}%
\pgfpathlineto{\pgfqpoint{4.191706in}{1.264647in}}%
\pgfpathlineto{\pgfqpoint{4.177307in}{1.264996in}}%
\pgfpathlineto{\pgfqpoint{4.162917in}{1.265415in}}%
\pgfpathlineto{\pgfqpoint{4.154652in}{1.253143in}}%
\pgfpathlineto{\pgfqpoint{4.146382in}{1.240947in}}%
\pgfpathlineto{\pgfqpoint{4.138106in}{1.228833in}}%
\pgfpathlineto{\pgfqpoint{4.129826in}{1.216807in}}%
\pgfpathclose%
\pgfusepath{fill}%
\end{pgfscope}%
\begin{pgfscope}%
\pgfpathrectangle{\pgfqpoint{1.150000in}{0.150000in}}{\pgfqpoint{5.700000in}{5.700000in}}%
\pgfusepath{clip}%
\pgfsetbuttcap%
\pgfsetroundjoin%
\definecolor{currentfill}{rgb}{0.283091,0.110553,0.431554}%
\pgfsetfillcolor{currentfill}%
\pgfsetfillopacity{0.700000}%
\pgfsetlinewidth{0.000000pt}%
\definecolor{currentstroke}{rgb}{0.000000,0.000000,0.000000}%
\pgfsetstrokecolor{currentstroke}%
\pgfsetdash{}{0pt}%
\pgfpathmoveto{\pgfqpoint{4.220532in}{1.264156in}}%
\pgfpathlineto{\pgfqpoint{4.234959in}{1.264015in}}%
\pgfpathlineto{\pgfqpoint{4.249396in}{1.263943in}}%
\pgfpathlineto{\pgfqpoint{4.263842in}{1.263941in}}%
\pgfpathlineto{\pgfqpoint{4.278298in}{1.264008in}}%
\pgfpathlineto{\pgfqpoint{4.286545in}{1.277082in}}%
\pgfpathlineto{\pgfqpoint{4.294787in}{1.290211in}}%
\pgfpathlineto{\pgfqpoint{4.303025in}{1.303390in}}%
\pgfpathlineto{\pgfqpoint{4.311258in}{1.316615in}}%
\pgfpathlineto{\pgfqpoint{4.296808in}{1.316190in}}%
\pgfpathlineto{\pgfqpoint{4.282367in}{1.315834in}}%
\pgfpathlineto{\pgfqpoint{4.267936in}{1.315549in}}%
\pgfpathlineto{\pgfqpoint{4.253515in}{1.315332in}}%
\pgfpathlineto{\pgfqpoint{4.245277in}{1.302458in}}%
\pgfpathlineto{\pgfqpoint{4.237034in}{1.289633in}}%
\pgfpathlineto{\pgfqpoint{4.228786in}{1.276864in}}%
\pgfpathlineto{\pgfqpoint{4.220532in}{1.264156in}}%
\pgfpathclose%
\pgfusepath{fill}%
\end{pgfscope}%
\begin{pgfscope}%
\pgfpathrectangle{\pgfqpoint{1.150000in}{0.150000in}}{\pgfqpoint{5.700000in}{5.700000in}}%
\pgfusepath{clip}%
\pgfsetbuttcap%
\pgfsetroundjoin%
\definecolor{currentfill}{rgb}{0.279566,0.067836,0.391917}%
\pgfsetfillcolor{currentfill}%
\pgfsetfillopacity{0.700000}%
\pgfsetlinewidth{0.000000pt}%
\definecolor{currentstroke}{rgb}{0.000000,0.000000,0.000000}%
\pgfsetstrokecolor{currentstroke}%
\pgfsetdash{}{0pt}%
\pgfpathmoveto{\pgfqpoint{4.039111in}{1.175165in}}%
\pgfpathlineto{\pgfqpoint{4.053483in}{1.173694in}}%
\pgfpathlineto{\pgfqpoint{4.067863in}{1.172292in}}%
\pgfpathlineto{\pgfqpoint{4.082252in}{1.170959in}}%
\pgfpathlineto{\pgfqpoint{4.096649in}{1.169696in}}%
\pgfpathlineto{\pgfqpoint{4.104952in}{1.181313in}}%
\pgfpathlineto{\pgfqpoint{4.113248in}{1.193041in}}%
\pgfpathlineto{\pgfqpoint{4.121540in}{1.204874in}}%
\pgfpathlineto{\pgfqpoint{4.129826in}{1.216807in}}%
\pgfpathlineto{\pgfqpoint{4.115437in}{1.217672in}}%
\pgfpathlineto{\pgfqpoint{4.101057in}{1.218608in}}%
\pgfpathlineto{\pgfqpoint{4.086686in}{1.219613in}}%
\pgfpathlineto{\pgfqpoint{4.072324in}{1.220687in}}%
\pgfpathlineto{\pgfqpoint{4.064029in}{1.209144in}}%
\pgfpathlineto{\pgfqpoint{4.055728in}{1.197705in}}%
\pgfpathlineto{\pgfqpoint{4.047422in}{1.186377in}}%
\pgfpathlineto{\pgfqpoint{4.039111in}{1.175165in}}%
\pgfpathclose%
\pgfusepath{fill}%
\end{pgfscope}%
\begin{pgfscope}%
\pgfpathrectangle{\pgfqpoint{1.150000in}{0.150000in}}{\pgfqpoint{5.700000in}{5.700000in}}%
\pgfusepath{clip}%
\pgfsetbuttcap%
\pgfsetroundjoin%
\definecolor{currentfill}{rgb}{0.282623,0.140926,0.457517}%
\pgfsetfillcolor{currentfill}%
\pgfsetfillopacity{0.700000}%
\pgfsetlinewidth{0.000000pt}%
\definecolor{currentstroke}{rgb}{0.000000,0.000000,0.000000}%
\pgfsetstrokecolor{currentstroke}%
\pgfsetdash{}{0pt}%
\pgfpathmoveto{\pgfqpoint{4.311258in}{1.316615in}}%
\pgfpathlineto{\pgfqpoint{4.325718in}{1.317109in}}%
\pgfpathlineto{\pgfqpoint{4.340188in}{1.317673in}}%
\pgfpathlineto{\pgfqpoint{4.354668in}{1.318306in}}%
\pgfpathlineto{\pgfqpoint{4.369158in}{1.319009in}}%
\pgfpathlineto{\pgfqpoint{4.377382in}{1.332619in}}%
\pgfpathlineto{\pgfqpoint{4.385601in}{1.346258in}}%
\pgfpathlineto{\pgfqpoint{4.393815in}{1.359923in}}%
\pgfpathlineto{\pgfqpoint{4.402025in}{1.373607in}}%
\pgfpathlineto{\pgfqpoint{4.387539in}{1.372566in}}%
\pgfpathlineto{\pgfqpoint{4.373063in}{1.371595in}}%
\pgfpathlineto{\pgfqpoint{4.358598in}{1.370693in}}%
\pgfpathlineto{\pgfqpoint{4.344142in}{1.369862in}}%
\pgfpathlineto{\pgfqpoint{4.335928in}{1.356507in}}%
\pgfpathlineto{\pgfqpoint{4.327709in}{1.343178in}}%
\pgfpathlineto{\pgfqpoint{4.319486in}{1.329879in}}%
\pgfpathlineto{\pgfqpoint{4.311258in}{1.316615in}}%
\pgfpathclose%
\pgfusepath{fill}%
\end{pgfscope}%
\begin{pgfscope}%
\pgfpathrectangle{\pgfqpoint{1.150000in}{0.150000in}}{\pgfqpoint{5.700000in}{5.700000in}}%
\pgfusepath{clip}%
\pgfsetbuttcap%
\pgfsetroundjoin%
\definecolor{currentfill}{rgb}{0.248629,0.278775,0.534556}%
\pgfsetfillcolor{currentfill}%
\pgfsetfillopacity{0.700000}%
\pgfsetlinewidth{0.000000pt}%
\definecolor{currentstroke}{rgb}{0.000000,0.000000,0.000000}%
\pgfsetstrokecolor{currentstroke}%
\pgfsetdash{}{0pt}%
\pgfpathmoveto{\pgfqpoint{4.707287in}{1.623309in}}%
\pgfpathlineto{\pgfqpoint{4.721908in}{1.626408in}}%
\pgfpathlineto{\pgfqpoint{4.736542in}{1.629577in}}%
\pgfpathlineto{\pgfqpoint{4.751188in}{1.632816in}}%
\pgfpathlineto{\pgfqpoint{4.765845in}{1.636126in}}%
\pgfpathlineto{\pgfqpoint{4.773964in}{1.650456in}}%
\pgfpathlineto{\pgfqpoint{4.782078in}{1.664710in}}%
\pgfpathlineto{\pgfqpoint{4.790186in}{1.678887in}}%
\pgfpathlineto{\pgfqpoint{4.798289in}{1.692982in}}%
\pgfpathlineto{\pgfqpoint{4.783632in}{1.689436in}}%
\pgfpathlineto{\pgfqpoint{4.768987in}{1.685961in}}%
\pgfpathlineto{\pgfqpoint{4.754355in}{1.682556in}}%
\pgfpathlineto{\pgfqpoint{4.739735in}{1.679222in}}%
\pgfpathlineto{\pgfqpoint{4.731631in}{1.665354in}}%
\pgfpathlineto{\pgfqpoint{4.723522in}{1.651411in}}%
\pgfpathlineto{\pgfqpoint{4.715407in}{1.637394in}}%
\pgfpathlineto{\pgfqpoint{4.707287in}{1.623309in}}%
\pgfpathclose%
\pgfusepath{fill}%
\end{pgfscope}%
\begin{pgfscope}%
\pgfpathrectangle{\pgfqpoint{1.150000in}{0.150000in}}{\pgfqpoint{5.700000in}{5.700000in}}%
\pgfusepath{clip}%
\pgfsetbuttcap%
\pgfsetroundjoin%
\definecolor{currentfill}{rgb}{0.276022,0.044167,0.370164}%
\pgfsetfillcolor{currentfill}%
\pgfsetfillopacity{0.700000}%
\pgfsetlinewidth{0.000000pt}%
\definecolor{currentstroke}{rgb}{0.000000,0.000000,0.000000}%
\pgfsetstrokecolor{currentstroke}%
\pgfsetdash{}{0pt}%
\pgfpathmoveto{\pgfqpoint{3.948355in}{1.139853in}}%
\pgfpathlineto{\pgfqpoint{3.962705in}{1.137686in}}%
\pgfpathlineto{\pgfqpoint{3.977064in}{1.135589in}}%
\pgfpathlineto{\pgfqpoint{3.991430in}{1.133561in}}%
\pgfpathlineto{\pgfqpoint{4.005805in}{1.131602in}}%
\pgfpathlineto{\pgfqpoint{4.014140in}{1.142288in}}%
\pgfpathlineto{\pgfqpoint{4.022470in}{1.153114in}}%
\pgfpathlineto{\pgfqpoint{4.030793in}{1.164076in}}%
\pgfpathlineto{\pgfqpoint{4.039111in}{1.175165in}}%
\pgfpathlineto{\pgfqpoint{4.024747in}{1.176706in}}%
\pgfpathlineto{\pgfqpoint{4.010392in}{1.178317in}}%
\pgfpathlineto{\pgfqpoint{3.996044in}{1.179997in}}%
\pgfpathlineto{\pgfqpoint{3.981706in}{1.181747in}}%
\pgfpathlineto{\pgfqpoint{3.973377in}{1.171067in}}%
\pgfpathlineto{\pgfqpoint{3.965043in}{1.160520in}}%
\pgfpathlineto{\pgfqpoint{3.956702in}{1.150113in}}%
\pgfpathlineto{\pgfqpoint{3.948355in}{1.139853in}}%
\pgfpathclose%
\pgfusepath{fill}%
\end{pgfscope}%
\begin{pgfscope}%
\pgfpathrectangle{\pgfqpoint{1.150000in}{0.150000in}}{\pgfqpoint{5.700000in}{5.700000in}}%
\pgfusepath{clip}%
\pgfsetbuttcap%
\pgfsetroundjoin%
\definecolor{currentfill}{rgb}{0.279574,0.170599,0.479997}%
\pgfsetfillcolor{currentfill}%
\pgfsetfillopacity{0.700000}%
\pgfsetlinewidth{0.000000pt}%
\definecolor{currentstroke}{rgb}{0.000000,0.000000,0.000000}%
\pgfsetstrokecolor{currentstroke}%
\pgfsetdash{}{0pt}%
\pgfpathmoveto{\pgfqpoint{4.402025in}{1.373607in}}%
\pgfpathlineto{\pgfqpoint{4.416522in}{1.374717in}}%
\pgfpathlineto{\pgfqpoint{4.431028in}{1.375897in}}%
\pgfpathlineto{\pgfqpoint{4.445546in}{1.377147in}}%
\pgfpathlineto{\pgfqpoint{4.460074in}{1.378466in}}%
\pgfpathlineto{\pgfqpoint{4.468276in}{1.392490in}}%
\pgfpathlineto{\pgfqpoint{4.476473in}{1.406521in}}%
\pgfpathlineto{\pgfqpoint{4.484666in}{1.420551in}}%
\pgfpathlineto{\pgfqpoint{4.492854in}{1.434578in}}%
\pgfpathlineto{\pgfqpoint{4.478329in}{1.432941in}}%
\pgfpathlineto{\pgfqpoint{4.463815in}{1.431373in}}%
\pgfpathlineto{\pgfqpoint{4.449311in}{1.429876in}}%
\pgfpathlineto{\pgfqpoint{4.434818in}{1.428448in}}%
\pgfpathlineto{\pgfqpoint{4.426627in}{1.414731in}}%
\pgfpathlineto{\pgfqpoint{4.418431in}{1.401016in}}%
\pgfpathlineto{\pgfqpoint{4.410230in}{1.387306in}}%
\pgfpathlineto{\pgfqpoint{4.402025in}{1.373607in}}%
\pgfpathclose%
\pgfusepath{fill}%
\end{pgfscope}%
\begin{pgfscope}%
\pgfpathrectangle{\pgfqpoint{1.150000in}{0.150000in}}{\pgfqpoint{5.700000in}{5.700000in}}%
\pgfusepath{clip}%
\pgfsetbuttcap%
\pgfsetroundjoin%
\definecolor{currentfill}{rgb}{0.199430,0.387607,0.554642}%
\pgfsetfillcolor{currentfill}%
\pgfsetfillopacity{0.700000}%
\pgfsetlinewidth{0.000000pt}%
\definecolor{currentstroke}{rgb}{0.000000,0.000000,0.000000}%
\pgfsetstrokecolor{currentstroke}%
\pgfsetdash{}{0pt}%
\pgfpathmoveto{\pgfqpoint{5.012728in}{1.890290in}}%
\pgfpathlineto{\pgfqpoint{5.027496in}{1.895055in}}%
\pgfpathlineto{\pgfqpoint{5.042277in}{1.899891in}}%
\pgfpathlineto{\pgfqpoint{5.057072in}{1.904799in}}%
\pgfpathlineto{\pgfqpoint{5.071880in}{1.909778in}}%
\pgfpathlineto{\pgfqpoint{5.079896in}{1.923038in}}%
\pgfpathlineto{\pgfqpoint{5.087906in}{1.936164in}}%
\pgfpathlineto{\pgfqpoint{5.095908in}{1.949155in}}%
\pgfpathlineto{\pgfqpoint{5.103903in}{1.962008in}}%
\pgfpathlineto{\pgfqpoint{5.089097in}{1.956878in}}%
\pgfpathlineto{\pgfqpoint{5.074304in}{1.951821in}}%
\pgfpathlineto{\pgfqpoint{5.059525in}{1.946834in}}%
\pgfpathlineto{\pgfqpoint{5.044759in}{1.941920in}}%
\pgfpathlineto{\pgfqpoint{5.036762in}{1.929208in}}%
\pgfpathlineto{\pgfqpoint{5.028758in}{1.916365in}}%
\pgfpathlineto{\pgfqpoint{5.020747in}{1.903392in}}%
\pgfpathlineto{\pgfqpoint{5.012728in}{1.890290in}}%
\pgfpathclose%
\pgfusepath{fill}%
\end{pgfscope}%
\begin{pgfscope}%
\pgfpathrectangle{\pgfqpoint{1.150000in}{0.150000in}}{\pgfqpoint{5.700000in}{5.700000in}}%
\pgfusepath{clip}%
\pgfsetbuttcap%
\pgfsetroundjoin%
\definecolor{currentfill}{rgb}{0.188923,0.410910,0.556326}%
\pgfsetfillcolor{currentfill}%
\pgfsetfillopacity{0.700000}%
\pgfsetlinewidth{0.000000pt}%
\definecolor{currentstroke}{rgb}{0.000000,0.000000,0.000000}%
\pgfsetstrokecolor{currentstroke}%
\pgfsetdash{}{0pt}%
\pgfpathmoveto{\pgfqpoint{5.103903in}{1.962008in}}%
\pgfpathlineto{\pgfqpoint{5.118722in}{1.967209in}}%
\pgfpathlineto{\pgfqpoint{5.133556in}{1.972482in}}%
\pgfpathlineto{\pgfqpoint{5.148403in}{1.977826in}}%
\pgfpathlineto{\pgfqpoint{5.156389in}{1.990643in}}%
\pgfpathlineto{\pgfqpoint{5.164366in}{2.003314in}}%
\pgfpathlineto{\pgfqpoint{5.172336in}{2.015838in}}%
\pgfpathlineto{\pgfqpoint{5.180298in}{2.028214in}}%
\pgfpathlineto{\pgfqpoint{5.165454in}{2.022741in}}%
\pgfpathlineto{\pgfqpoint{5.150623in}{2.017340in}}%
\pgfpathlineto{\pgfqpoint{5.135806in}{2.012011in}}%
\pgfpathlineto{\pgfqpoint{5.127842in}{1.999725in}}%
\pgfpathlineto{\pgfqpoint{5.119870in}{1.987295in}}%
\pgfpathlineto{\pgfqpoint{5.111890in}{1.974722in}}%
\pgfpathlineto{\pgfqpoint{5.103903in}{1.962008in}}%
\pgfpathclose%
\pgfusepath{fill}%
\end{pgfscope}%
\begin{pgfscope}%
\pgfpathrectangle{\pgfqpoint{1.150000in}{0.150000in}}{\pgfqpoint{5.700000in}{5.700000in}}%
\pgfusepath{clip}%
\pgfsetbuttcap%
\pgfsetroundjoin%
\definecolor{currentfill}{rgb}{0.274128,0.199721,0.498911}%
\pgfsetfillcolor{currentfill}%
\pgfsetfillopacity{0.700000}%
\pgfsetlinewidth{0.000000pt}%
\definecolor{currentstroke}{rgb}{0.000000,0.000000,0.000000}%
\pgfsetstrokecolor{currentstroke}%
\pgfsetdash{}{0pt}%
\pgfpathmoveto{\pgfqpoint{4.492854in}{1.434578in}}%
\pgfpathlineto{\pgfqpoint{4.507390in}{1.436285in}}%
\pgfpathlineto{\pgfqpoint{4.521936in}{1.438062in}}%
\pgfpathlineto{\pgfqpoint{4.536494in}{1.439908in}}%
\pgfpathlineto{\pgfqpoint{4.551063in}{1.441824in}}%
\pgfpathlineto{\pgfqpoint{4.559244in}{1.456148in}}%
\pgfpathlineto{\pgfqpoint{4.567421in}{1.470454in}}%
\pgfpathlineto{\pgfqpoint{4.575592in}{1.484737in}}%
\pgfpathlineto{\pgfqpoint{4.583759in}{1.498994in}}%
\pgfpathlineto{\pgfqpoint{4.569193in}{1.496780in}}%
\pgfpathlineto{\pgfqpoint{4.554637in}{1.494636in}}%
\pgfpathlineto{\pgfqpoint{4.540093in}{1.492562in}}%
\pgfpathlineto{\pgfqpoint{4.525559in}{1.490557in}}%
\pgfpathlineto{\pgfqpoint{4.517390in}{1.476590in}}%
\pgfpathlineto{\pgfqpoint{4.509216in}{1.462602in}}%
\pgfpathlineto{\pgfqpoint{4.501037in}{1.448596in}}%
\pgfpathlineto{\pgfqpoint{4.492854in}{1.434578in}}%
\pgfpathclose%
\pgfusepath{fill}%
\end{pgfscope}%
\begin{pgfscope}%
\pgfpathrectangle{\pgfqpoint{1.150000in}{0.150000in}}{\pgfqpoint{5.700000in}{5.700000in}}%
\pgfusepath{clip}%
\pgfsetbuttcap%
\pgfsetroundjoin%
\definecolor{currentfill}{rgb}{0.233603,0.313828,0.543914}%
\pgfsetfillcolor{currentfill}%
\pgfsetfillopacity{0.700000}%
\pgfsetlinewidth{0.000000pt}%
\definecolor{currentstroke}{rgb}{0.000000,0.000000,0.000000}%
\pgfsetstrokecolor{currentstroke}%
\pgfsetdash{}{0pt}%
\pgfpathmoveto{\pgfqpoint{4.798289in}{1.692982in}}%
\pgfpathlineto{\pgfqpoint{4.812958in}{1.696599in}}%
\pgfpathlineto{\pgfqpoint{4.827639in}{1.700286in}}%
\pgfpathlineto{\pgfqpoint{4.842333in}{1.704044in}}%
\pgfpathlineto{\pgfqpoint{4.857040in}{1.707872in}}%
\pgfpathlineto{\pgfqpoint{4.865136in}{1.722106in}}%
\pgfpathlineto{\pgfqpoint{4.873226in}{1.736247in}}%
\pgfpathlineto{\pgfqpoint{4.881310in}{1.750293in}}%
\pgfpathlineto{\pgfqpoint{4.889388in}{1.764239in}}%
\pgfpathlineto{\pgfqpoint{4.874683in}{1.760196in}}%
\pgfpathlineto{\pgfqpoint{4.859990in}{1.756223in}}%
\pgfpathlineto{\pgfqpoint{4.845310in}{1.752321in}}%
\pgfpathlineto{\pgfqpoint{4.830642in}{1.748490in}}%
\pgfpathlineto{\pgfqpoint{4.822562in}{1.734750in}}%
\pgfpathlineto{\pgfqpoint{4.814477in}{1.720917in}}%
\pgfpathlineto{\pgfqpoint{4.806386in}{1.706993in}}%
\pgfpathlineto{\pgfqpoint{4.798289in}{1.692982in}}%
\pgfpathclose%
\pgfusepath{fill}%
\end{pgfscope}%
\begin{pgfscope}%
\pgfpathrectangle{\pgfqpoint{1.150000in}{0.150000in}}{\pgfqpoint{5.700000in}{5.700000in}}%
\pgfusepath{clip}%
\pgfsetbuttcap%
\pgfsetroundjoin%
\definecolor{currentfill}{rgb}{0.265145,0.232956,0.516599}%
\pgfsetfillcolor{currentfill}%
\pgfsetfillopacity{0.700000}%
\pgfsetlinewidth{0.000000pt}%
\definecolor{currentstroke}{rgb}{0.000000,0.000000,0.000000}%
\pgfsetstrokecolor{currentstroke}%
\pgfsetdash{}{0pt}%
\pgfpathmoveto{\pgfqpoint{4.583759in}{1.498994in}}%
\pgfpathlineto{\pgfqpoint{4.598337in}{1.501278in}}%
\pgfpathlineto{\pgfqpoint{4.612927in}{1.503633in}}%
\pgfpathlineto{\pgfqpoint{4.627528in}{1.506057in}}%
\pgfpathlineto{\pgfqpoint{4.642140in}{1.508551in}}%
\pgfpathlineto{\pgfqpoint{4.650301in}{1.523062in}}%
\pgfpathlineto{\pgfqpoint{4.658457in}{1.537534in}}%
\pgfpathlineto{\pgfqpoint{4.666608in}{1.551962in}}%
\pgfpathlineto{\pgfqpoint{4.674754in}{1.566342in}}%
\pgfpathlineto{\pgfqpoint{4.660143in}{1.563570in}}%
\pgfpathlineto{\pgfqpoint{4.645543in}{1.560868in}}%
\pgfpathlineto{\pgfqpoint{4.630955in}{1.558237in}}%
\pgfpathlineto{\pgfqpoint{4.616379in}{1.555676in}}%
\pgfpathlineto{\pgfqpoint{4.608231in}{1.541565in}}%
\pgfpathlineto{\pgfqpoint{4.600079in}{1.527412in}}%
\pgfpathlineto{\pgfqpoint{4.591922in}{1.513221in}}%
\pgfpathlineto{\pgfqpoint{4.583759in}{1.498994in}}%
\pgfpathclose%
\pgfusepath{fill}%
\end{pgfscope}%
\begin{pgfscope}%
\pgfpathrectangle{\pgfqpoint{1.150000in}{0.150000in}}{\pgfqpoint{5.700000in}{5.700000in}}%
\pgfusepath{clip}%
\pgfsetbuttcap%
\pgfsetroundjoin%
\definecolor{currentfill}{rgb}{0.220057,0.343307,0.549413}%
\pgfsetfillcolor{currentfill}%
\pgfsetfillopacity{0.700000}%
\pgfsetlinewidth{0.000000pt}%
\definecolor{currentstroke}{rgb}{0.000000,0.000000,0.000000}%
\pgfsetstrokecolor{currentstroke}%
\pgfsetdash{}{0pt}%
\pgfpathmoveto{\pgfqpoint{4.889388in}{1.764239in}}%
\pgfpathlineto{\pgfqpoint{4.904107in}{1.768354in}}%
\pgfpathlineto{\pgfqpoint{4.918838in}{1.772539in}}%
\pgfpathlineto{\pgfqpoint{4.933582in}{1.776796in}}%
\pgfpathlineto{\pgfqpoint{4.948339in}{1.781124in}}%
\pgfpathlineto{\pgfqpoint{4.956410in}{1.795171in}}%
\pgfpathlineto{\pgfqpoint{4.964475in}{1.809108in}}%
\pgfpathlineto{\pgfqpoint{4.972534in}{1.822933in}}%
\pgfpathlineto{\pgfqpoint{4.980586in}{1.836644in}}%
\pgfpathlineto{\pgfqpoint{4.965830in}{1.832122in}}%
\pgfpathlineto{\pgfqpoint{4.951087in}{1.827672in}}%
\pgfpathlineto{\pgfqpoint{4.936357in}{1.823293in}}%
\pgfpathlineto{\pgfqpoint{4.921640in}{1.818985in}}%
\pgfpathlineto{\pgfqpoint{4.913586in}{1.805460in}}%
\pgfpathlineto{\pgfqpoint{4.905527in}{1.791826in}}%
\pgfpathlineto{\pgfqpoint{4.897461in}{1.778085in}}%
\pgfpathlineto{\pgfqpoint{4.889388in}{1.764239in}}%
\pgfpathclose%
\pgfusepath{fill}%
\end{pgfscope}%
\begin{pgfscope}%
\pgfpathrectangle{\pgfqpoint{1.150000in}{0.150000in}}{\pgfqpoint{5.700000in}{5.700000in}}%
\pgfusepath{clip}%
\pgfsetbuttcap%
\pgfsetroundjoin%
\definecolor{currentfill}{rgb}{0.282656,0.100196,0.422160}%
\pgfsetfillcolor{currentfill}%
\pgfsetfillopacity{0.700000}%
\pgfsetlinewidth{0.000000pt}%
\definecolor{currentstroke}{rgb}{0.000000,0.000000,0.000000}%
\pgfsetstrokecolor{currentstroke}%
\pgfsetdash{}{0pt}%
\pgfpathmoveto{\pgfqpoint{4.187470in}{1.214038in}}%
\pgfpathlineto{\pgfqpoint{4.201904in}{1.213519in}}%
\pgfpathlineto{\pgfqpoint{4.216347in}{1.213069in}}%
\pgfpathlineto{\pgfqpoint{4.230799in}{1.212689in}}%
\pgfpathlineto{\pgfqpoint{4.245261in}{1.212377in}}%
\pgfpathlineto{\pgfqpoint{4.253527in}{1.225174in}}%
\pgfpathlineto{\pgfqpoint{4.261789in}{1.238049in}}%
\pgfpathlineto{\pgfqpoint{4.270046in}{1.250995in}}%
\pgfpathlineto{\pgfqpoint{4.278298in}{1.264008in}}%
\pgfpathlineto{\pgfqpoint{4.263842in}{1.263941in}}%
\pgfpathlineto{\pgfqpoint{4.249396in}{1.263943in}}%
\pgfpathlineto{\pgfqpoint{4.234959in}{1.264015in}}%
\pgfpathlineto{\pgfqpoint{4.220532in}{1.264156in}}%
\pgfpathlineto{\pgfqpoint{4.212274in}{1.251514in}}%
\pgfpathlineto{\pgfqpoint{4.204011in}{1.238943in}}%
\pgfpathlineto{\pgfqpoint{4.195743in}{1.226449in}}%
\pgfpathlineto{\pgfqpoint{4.187470in}{1.214038in}}%
\pgfpathclose%
\pgfusepath{fill}%
\end{pgfscope}%
\begin{pgfscope}%
\pgfpathrectangle{\pgfqpoint{1.150000in}{0.150000in}}{\pgfqpoint{5.700000in}{5.700000in}}%
\pgfusepath{clip}%
\pgfsetbuttcap%
\pgfsetroundjoin%
\definecolor{currentfill}{rgb}{0.280894,0.078907,0.402329}%
\pgfsetfillcolor{currentfill}%
\pgfsetfillopacity{0.700000}%
\pgfsetlinewidth{0.000000pt}%
\definecolor{currentstroke}{rgb}{0.000000,0.000000,0.000000}%
\pgfsetstrokecolor{currentstroke}%
\pgfsetdash{}{0pt}%
\pgfpathmoveto{\pgfqpoint{4.096649in}{1.169696in}}%
\pgfpathlineto{\pgfqpoint{4.111056in}{1.168502in}}%
\pgfpathlineto{\pgfqpoint{4.125471in}{1.167377in}}%
\pgfpathlineto{\pgfqpoint{4.139894in}{1.166322in}}%
\pgfpathlineto{\pgfqpoint{4.154327in}{1.165335in}}%
\pgfpathlineto{\pgfqpoint{4.162620in}{1.177358in}}%
\pgfpathlineto{\pgfqpoint{4.170909in}{1.189487in}}%
\pgfpathlineto{\pgfqpoint{4.179192in}{1.201715in}}%
\pgfpathlineto{\pgfqpoint{4.187470in}{1.214038in}}%
\pgfpathlineto{\pgfqpoint{4.173046in}{1.214626in}}%
\pgfpathlineto{\pgfqpoint{4.158630in}{1.215284in}}%
\pgfpathlineto{\pgfqpoint{4.144223in}{1.216011in}}%
\pgfpathlineto{\pgfqpoint{4.129826in}{1.216807in}}%
\pgfpathlineto{\pgfqpoint{4.121540in}{1.204874in}}%
\pgfpathlineto{\pgfqpoint{4.113248in}{1.193041in}}%
\pgfpathlineto{\pgfqpoint{4.104952in}{1.181313in}}%
\pgfpathlineto{\pgfqpoint{4.096649in}{1.169696in}}%
\pgfpathclose%
\pgfusepath{fill}%
\end{pgfscope}%
\begin{pgfscope}%
\pgfpathrectangle{\pgfqpoint{1.150000in}{0.150000in}}{\pgfqpoint{5.700000in}{5.700000in}}%
\pgfusepath{clip}%
\pgfsetbuttcap%
\pgfsetroundjoin%
\definecolor{currentfill}{rgb}{0.283187,0.125848,0.444960}%
\pgfsetfillcolor{currentfill}%
\pgfsetfillopacity{0.700000}%
\pgfsetlinewidth{0.000000pt}%
\definecolor{currentstroke}{rgb}{0.000000,0.000000,0.000000}%
\pgfsetstrokecolor{currentstroke}%
\pgfsetdash{}{0pt}%
\pgfpathmoveto{\pgfqpoint{4.278298in}{1.264008in}}%
\pgfpathlineto{\pgfqpoint{4.292763in}{1.264144in}}%
\pgfpathlineto{\pgfqpoint{4.307238in}{1.264350in}}%
\pgfpathlineto{\pgfqpoint{4.321723in}{1.264625in}}%
\pgfpathlineto{\pgfqpoint{4.336217in}{1.264968in}}%
\pgfpathlineto{\pgfqpoint{4.344459in}{1.278408in}}%
\pgfpathlineto{\pgfqpoint{4.352697in}{1.291899in}}%
\pgfpathlineto{\pgfqpoint{4.360930in}{1.305434in}}%
\pgfpathlineto{\pgfqpoint{4.369158in}{1.319009in}}%
\pgfpathlineto{\pgfqpoint{4.354668in}{1.318306in}}%
\pgfpathlineto{\pgfqpoint{4.340188in}{1.317673in}}%
\pgfpathlineto{\pgfqpoint{4.325718in}{1.317109in}}%
\pgfpathlineto{\pgfqpoint{4.311258in}{1.316615in}}%
\pgfpathlineto{\pgfqpoint{4.303025in}{1.303390in}}%
\pgfpathlineto{\pgfqpoint{4.294787in}{1.290211in}}%
\pgfpathlineto{\pgfqpoint{4.286545in}{1.277082in}}%
\pgfpathlineto{\pgfqpoint{4.278298in}{1.264008in}}%
\pgfpathclose%
\pgfusepath{fill}%
\end{pgfscope}%
\begin{pgfscope}%
\pgfpathrectangle{\pgfqpoint{1.150000in}{0.150000in}}{\pgfqpoint{5.700000in}{5.700000in}}%
\pgfusepath{clip}%
\pgfsetbuttcap%
\pgfsetroundjoin%
\definecolor{currentfill}{rgb}{0.253935,0.265254,0.529983}%
\pgfsetfillcolor{currentfill}%
\pgfsetfillopacity{0.700000}%
\pgfsetlinewidth{0.000000pt}%
\definecolor{currentstroke}{rgb}{0.000000,0.000000,0.000000}%
\pgfsetstrokecolor{currentstroke}%
\pgfsetdash{}{0pt}%
\pgfpathmoveto{\pgfqpoint{4.674754in}{1.566342in}}%
\pgfpathlineto{\pgfqpoint{4.689376in}{1.569184in}}%
\pgfpathlineto{\pgfqpoint{4.704011in}{1.572096in}}%
\pgfpathlineto{\pgfqpoint{4.718658in}{1.575079in}}%
\pgfpathlineto{\pgfqpoint{4.733316in}{1.578131in}}%
\pgfpathlineto{\pgfqpoint{4.741456in}{1.592725in}}%
\pgfpathlineto{\pgfqpoint{4.749591in}{1.607257in}}%
\pgfpathlineto{\pgfqpoint{4.757721in}{1.621726in}}%
\pgfpathlineto{\pgfqpoint{4.765845in}{1.636126in}}%
\pgfpathlineto{\pgfqpoint{4.751188in}{1.632816in}}%
\pgfpathlineto{\pgfqpoint{4.736542in}{1.629577in}}%
\pgfpathlineto{\pgfqpoint{4.721908in}{1.626408in}}%
\pgfpathlineto{\pgfqpoint{4.707287in}{1.623309in}}%
\pgfpathlineto{\pgfqpoint{4.699161in}{1.609157in}}%
\pgfpathlineto{\pgfqpoint{4.691030in}{1.594943in}}%
\pgfpathlineto{\pgfqpoint{4.682895in}{1.580670in}}%
\pgfpathlineto{\pgfqpoint{4.674754in}{1.566342in}}%
\pgfpathclose%
\pgfusepath{fill}%
\end{pgfscope}%
\begin{pgfscope}%
\pgfpathrectangle{\pgfqpoint{1.150000in}{0.150000in}}{\pgfqpoint{5.700000in}{5.700000in}}%
\pgfusepath{clip}%
\pgfsetbuttcap%
\pgfsetroundjoin%
\definecolor{currentfill}{rgb}{0.277941,0.056324,0.381191}%
\pgfsetfillcolor{currentfill}%
\pgfsetfillopacity{0.700000}%
\pgfsetlinewidth{0.000000pt}%
\definecolor{currentstroke}{rgb}{0.000000,0.000000,0.000000}%
\pgfsetstrokecolor{currentstroke}%
\pgfsetdash{}{0pt}%
\pgfpathmoveto{\pgfqpoint{4.005805in}{1.131602in}}%
\pgfpathlineto{\pgfqpoint{4.020187in}{1.129713in}}%
\pgfpathlineto{\pgfqpoint{4.034578in}{1.127894in}}%
\pgfpathlineto{\pgfqpoint{4.048977in}{1.126143in}}%
\pgfpathlineto{\pgfqpoint{4.063385in}{1.124462in}}%
\pgfpathlineto{\pgfqpoint{4.071709in}{1.135573in}}%
\pgfpathlineto{\pgfqpoint{4.080028in}{1.146820in}}%
\pgfpathlineto{\pgfqpoint{4.088342in}{1.158196in}}%
\pgfpathlineto{\pgfqpoint{4.096649in}{1.169696in}}%
\pgfpathlineto{\pgfqpoint{4.082252in}{1.170959in}}%
\pgfpathlineto{\pgfqpoint{4.067863in}{1.172292in}}%
\pgfpathlineto{\pgfqpoint{4.053483in}{1.173694in}}%
\pgfpathlineto{\pgfqpoint{4.039111in}{1.175165in}}%
\pgfpathlineto{\pgfqpoint{4.030793in}{1.164076in}}%
\pgfpathlineto{\pgfqpoint{4.022470in}{1.153114in}}%
\pgfpathlineto{\pgfqpoint{4.014140in}{1.142288in}}%
\pgfpathlineto{\pgfqpoint{4.005805in}{1.131602in}}%
\pgfpathclose%
\pgfusepath{fill}%
\end{pgfscope}%
\begin{pgfscope}%
\pgfpathrectangle{\pgfqpoint{1.150000in}{0.150000in}}{\pgfqpoint{5.700000in}{5.700000in}}%
\pgfusepath{clip}%
\pgfsetbuttcap%
\pgfsetroundjoin%
\definecolor{currentfill}{rgb}{0.281412,0.155834,0.469201}%
\pgfsetfillcolor{currentfill}%
\pgfsetfillopacity{0.700000}%
\pgfsetlinewidth{0.000000pt}%
\definecolor{currentstroke}{rgb}{0.000000,0.000000,0.000000}%
\pgfsetstrokecolor{currentstroke}%
\pgfsetdash{}{0pt}%
\pgfpathmoveto{\pgfqpoint{4.369158in}{1.319009in}}%
\pgfpathlineto{\pgfqpoint{4.383658in}{1.319781in}}%
\pgfpathlineto{\pgfqpoint{4.398169in}{1.320622in}}%
\pgfpathlineto{\pgfqpoint{4.412690in}{1.321533in}}%
\pgfpathlineto{\pgfqpoint{4.427221in}{1.322513in}}%
\pgfpathlineto{\pgfqpoint{4.435441in}{1.336469in}}%
\pgfpathlineto{\pgfqpoint{4.443656in}{1.350450in}}%
\pgfpathlineto{\pgfqpoint{4.451867in}{1.364451in}}%
\pgfpathlineto{\pgfqpoint{4.460074in}{1.378466in}}%
\pgfpathlineto{\pgfqpoint{4.445546in}{1.377147in}}%
\pgfpathlineto{\pgfqpoint{4.431028in}{1.375897in}}%
\pgfpathlineto{\pgfqpoint{4.416522in}{1.374717in}}%
\pgfpathlineto{\pgfqpoint{4.402025in}{1.373607in}}%
\pgfpathlineto{\pgfqpoint{4.393815in}{1.359923in}}%
\pgfpathlineto{\pgfqpoint{4.385601in}{1.346258in}}%
\pgfpathlineto{\pgfqpoint{4.377382in}{1.332619in}}%
\pgfpathlineto{\pgfqpoint{4.369158in}{1.319009in}}%
\pgfpathclose%
\pgfusepath{fill}%
\end{pgfscope}%
\begin{pgfscope}%
\pgfpathrectangle{\pgfqpoint{1.150000in}{0.150000in}}{\pgfqpoint{5.700000in}{5.700000in}}%
\pgfusepath{clip}%
\pgfsetbuttcap%
\pgfsetroundjoin%
\definecolor{currentfill}{rgb}{0.204903,0.375746,0.553533}%
\pgfsetfillcolor{currentfill}%
\pgfsetfillopacity{0.700000}%
\pgfsetlinewidth{0.000000pt}%
\definecolor{currentstroke}{rgb}{0.000000,0.000000,0.000000}%
\pgfsetstrokecolor{currentstroke}%
\pgfsetdash{}{0pt}%
\pgfpathmoveto{\pgfqpoint{4.980586in}{1.836644in}}%
\pgfpathlineto{\pgfqpoint{4.995355in}{1.841237in}}%
\pgfpathlineto{\pgfqpoint{5.010138in}{1.845901in}}%
\pgfpathlineto{\pgfqpoint{5.024934in}{1.850636in}}%
\pgfpathlineto{\pgfqpoint{5.039743in}{1.855442in}}%
\pgfpathlineto{\pgfqpoint{5.047788in}{1.869216in}}%
\pgfpathlineto{\pgfqpoint{5.055825in}{1.882865in}}%
\pgfpathlineto{\pgfqpoint{5.063856in}{1.896386in}}%
\pgfpathlineto{\pgfqpoint{5.071880in}{1.909778in}}%
\pgfpathlineto{\pgfqpoint{5.057072in}{1.904799in}}%
\pgfpathlineto{\pgfqpoint{5.042277in}{1.899891in}}%
\pgfpathlineto{\pgfqpoint{5.027496in}{1.895055in}}%
\pgfpathlineto{\pgfqpoint{5.012728in}{1.890290in}}%
\pgfpathlineto{\pgfqpoint{5.004703in}{1.877063in}}%
\pgfpathlineto{\pgfqpoint{4.996671in}{1.863711in}}%
\pgfpathlineto{\pgfqpoint{4.988632in}{1.850237in}}%
\pgfpathlineto{\pgfqpoint{4.980586in}{1.836644in}}%
\pgfpathclose%
\pgfusepath{fill}%
\end{pgfscope}%
\begin{pgfscope}%
\pgfpathrectangle{\pgfqpoint{1.150000in}{0.150000in}}{\pgfqpoint{5.700000in}{5.700000in}}%
\pgfusepath{clip}%
\pgfsetbuttcap%
\pgfsetroundjoin%
\definecolor{currentfill}{rgb}{0.277134,0.185228,0.489898}%
\pgfsetfillcolor{currentfill}%
\pgfsetfillopacity{0.700000}%
\pgfsetlinewidth{0.000000pt}%
\definecolor{currentstroke}{rgb}{0.000000,0.000000,0.000000}%
\pgfsetstrokecolor{currentstroke}%
\pgfsetdash{}{0pt}%
\pgfpathmoveto{\pgfqpoint{4.460074in}{1.378466in}}%
\pgfpathlineto{\pgfqpoint{4.474612in}{1.379854in}}%
\pgfpathlineto{\pgfqpoint{4.489161in}{1.381312in}}%
\pgfpathlineto{\pgfqpoint{4.503721in}{1.382840in}}%
\pgfpathlineto{\pgfqpoint{4.518292in}{1.384437in}}%
\pgfpathlineto{\pgfqpoint{4.526492in}{1.398788in}}%
\pgfpathlineto{\pgfqpoint{4.534687in}{1.413140in}}%
\pgfpathlineto{\pgfqpoint{4.542877in}{1.427486in}}%
\pgfpathlineto{\pgfqpoint{4.551063in}{1.441824in}}%
\pgfpathlineto{\pgfqpoint{4.536494in}{1.439908in}}%
\pgfpathlineto{\pgfqpoint{4.521936in}{1.438062in}}%
\pgfpathlineto{\pgfqpoint{4.507390in}{1.436285in}}%
\pgfpathlineto{\pgfqpoint{4.492854in}{1.434578in}}%
\pgfpathlineto{\pgfqpoint{4.484666in}{1.420551in}}%
\pgfpathlineto{\pgfqpoint{4.476473in}{1.406521in}}%
\pgfpathlineto{\pgfqpoint{4.468276in}{1.392490in}}%
\pgfpathlineto{\pgfqpoint{4.460074in}{1.378466in}}%
\pgfpathclose%
\pgfusepath{fill}%
\end{pgfscope}%
\begin{pgfscope}%
\pgfpathrectangle{\pgfqpoint{1.150000in}{0.150000in}}{\pgfqpoint{5.700000in}{5.700000in}}%
\pgfusepath{clip}%
\pgfsetbuttcap%
\pgfsetroundjoin%
\definecolor{currentfill}{rgb}{0.241237,0.296485,0.539709}%
\pgfsetfillcolor{currentfill}%
\pgfsetfillopacity{0.700000}%
\pgfsetlinewidth{0.000000pt}%
\definecolor{currentstroke}{rgb}{0.000000,0.000000,0.000000}%
\pgfsetstrokecolor{currentstroke}%
\pgfsetdash{}{0pt}%
\pgfpathmoveto{\pgfqpoint{4.765845in}{1.636126in}}%
\pgfpathlineto{\pgfqpoint{4.780515in}{1.639507in}}%
\pgfpathlineto{\pgfqpoint{4.795197in}{1.642958in}}%
\pgfpathlineto{\pgfqpoint{4.809892in}{1.646479in}}%
\pgfpathlineto{\pgfqpoint{4.824599in}{1.650071in}}%
\pgfpathlineto{\pgfqpoint{4.832717in}{1.664645in}}%
\pgfpathlineto{\pgfqpoint{4.840830in}{1.679138in}}%
\pgfpathlineto{\pgfqpoint{4.848938in}{1.693549in}}%
\pgfpathlineto{\pgfqpoint{4.857040in}{1.707872in}}%
\pgfpathlineto{\pgfqpoint{4.842333in}{1.704044in}}%
\pgfpathlineto{\pgfqpoint{4.827639in}{1.700286in}}%
\pgfpathlineto{\pgfqpoint{4.812958in}{1.696599in}}%
\pgfpathlineto{\pgfqpoint{4.798289in}{1.692982in}}%
\pgfpathlineto{\pgfqpoint{4.790186in}{1.678887in}}%
\pgfpathlineto{\pgfqpoint{4.782078in}{1.664710in}}%
\pgfpathlineto{\pgfqpoint{4.773964in}{1.650456in}}%
\pgfpathlineto{\pgfqpoint{4.765845in}{1.636126in}}%
\pgfpathclose%
\pgfusepath{fill}%
\end{pgfscope}%
\begin{pgfscope}%
\pgfpathrectangle{\pgfqpoint{1.150000in}{0.150000in}}{\pgfqpoint{5.700000in}{5.700000in}}%
\pgfusepath{clip}%
\pgfsetbuttcap%
\pgfsetroundjoin%
\definecolor{currentfill}{rgb}{0.192357,0.403199,0.555836}%
\pgfsetfillcolor{currentfill}%
\pgfsetfillopacity{0.700000}%
\pgfsetlinewidth{0.000000pt}%
\definecolor{currentstroke}{rgb}{0.000000,0.000000,0.000000}%
\pgfsetstrokecolor{currentstroke}%
\pgfsetdash{}{0pt}%
\pgfpathmoveto{\pgfqpoint{5.071880in}{1.909778in}}%
\pgfpathlineto{\pgfqpoint{5.086701in}{1.914829in}}%
\pgfpathlineto{\pgfqpoint{5.101537in}{1.919951in}}%
\pgfpathlineto{\pgfqpoint{5.116386in}{1.925144in}}%
\pgfpathlineto{\pgfqpoint{5.124401in}{1.938524in}}%
\pgfpathlineto{\pgfqpoint{5.132409in}{1.951765in}}%
\pgfpathlineto{\pgfqpoint{5.140410in}{1.964867in}}%
\pgfpathlineto{\pgfqpoint{5.148403in}{1.977826in}}%
\pgfpathlineto{\pgfqpoint{5.133556in}{1.972482in}}%
\pgfpathlineto{\pgfqpoint{5.118722in}{1.967209in}}%
\pgfpathlineto{\pgfqpoint{5.103903in}{1.962008in}}%
\pgfpathlineto{\pgfqpoint{5.095908in}{1.949155in}}%
\pgfpathlineto{\pgfqpoint{5.087906in}{1.936164in}}%
\pgfpathlineto{\pgfqpoint{5.079896in}{1.923038in}}%
\pgfpathlineto{\pgfqpoint{5.071880in}{1.909778in}}%
\pgfpathclose%
\pgfusepath{fill}%
\end{pgfscope}%
\begin{pgfscope}%
\pgfpathrectangle{\pgfqpoint{1.150000in}{0.150000in}}{\pgfqpoint{5.700000in}{5.700000in}}%
\pgfusepath{clip}%
\pgfsetbuttcap%
\pgfsetroundjoin%
\definecolor{currentfill}{rgb}{0.270595,0.214069,0.507052}%
\pgfsetfillcolor{currentfill}%
\pgfsetfillopacity{0.700000}%
\pgfsetlinewidth{0.000000pt}%
\definecolor{currentstroke}{rgb}{0.000000,0.000000,0.000000}%
\pgfsetstrokecolor{currentstroke}%
\pgfsetdash{}{0pt}%
\pgfpathmoveto{\pgfqpoint{4.551063in}{1.441824in}}%
\pgfpathlineto{\pgfqpoint{4.565643in}{1.443810in}}%
\pgfpathlineto{\pgfqpoint{4.580234in}{1.445865in}}%
\pgfpathlineto{\pgfqpoint{4.594836in}{1.447991in}}%
\pgfpathlineto{\pgfqpoint{4.609449in}{1.450186in}}%
\pgfpathlineto{\pgfqpoint{4.617629in}{1.464816in}}%
\pgfpathlineto{\pgfqpoint{4.625804in}{1.479423in}}%
\pgfpathlineto{\pgfqpoint{4.633974in}{1.494003in}}%
\pgfpathlineto{\pgfqpoint{4.642140in}{1.508551in}}%
\pgfpathlineto{\pgfqpoint{4.627528in}{1.506057in}}%
\pgfpathlineto{\pgfqpoint{4.612927in}{1.503633in}}%
\pgfpathlineto{\pgfqpoint{4.598337in}{1.501278in}}%
\pgfpathlineto{\pgfqpoint{4.583759in}{1.498994in}}%
\pgfpathlineto{\pgfqpoint{4.575592in}{1.484737in}}%
\pgfpathlineto{\pgfqpoint{4.567421in}{1.470454in}}%
\pgfpathlineto{\pgfqpoint{4.559244in}{1.456148in}}%
\pgfpathlineto{\pgfqpoint{4.551063in}{1.441824in}}%
\pgfpathclose%
\pgfusepath{fill}%
\end{pgfscope}%
\begin{pgfscope}%
\pgfpathrectangle{\pgfqpoint{1.150000in}{0.150000in}}{\pgfqpoint{5.700000in}{5.700000in}}%
\pgfusepath{clip}%
\pgfsetbuttcap%
\pgfsetroundjoin%
\definecolor{currentfill}{rgb}{0.225863,0.330805,0.547314}%
\pgfsetfillcolor{currentfill}%
\pgfsetfillopacity{0.700000}%
\pgfsetlinewidth{0.000000pt}%
\definecolor{currentstroke}{rgb}{0.000000,0.000000,0.000000}%
\pgfsetstrokecolor{currentstroke}%
\pgfsetdash{}{0pt}%
\pgfpathmoveto{\pgfqpoint{4.857040in}{1.707872in}}%
\pgfpathlineto{\pgfqpoint{4.871759in}{1.711772in}}%
\pgfpathlineto{\pgfqpoint{4.886491in}{1.715742in}}%
\pgfpathlineto{\pgfqpoint{4.901235in}{1.719783in}}%
\pgfpathlineto{\pgfqpoint{4.915992in}{1.723894in}}%
\pgfpathlineto{\pgfqpoint{4.924088in}{1.738352in}}%
\pgfpathlineto{\pgfqpoint{4.932178in}{1.752711in}}%
\pgfpathlineto{\pgfqpoint{4.940262in}{1.766970in}}%
\pgfpathlineto{\pgfqpoint{4.948339in}{1.781124in}}%
\pgfpathlineto{\pgfqpoint{4.933582in}{1.776796in}}%
\pgfpathlineto{\pgfqpoint{4.918838in}{1.772539in}}%
\pgfpathlineto{\pgfqpoint{4.904107in}{1.768354in}}%
\pgfpathlineto{\pgfqpoint{4.889388in}{1.764239in}}%
\pgfpathlineto{\pgfqpoint{4.881310in}{1.750293in}}%
\pgfpathlineto{\pgfqpoint{4.873226in}{1.736247in}}%
\pgfpathlineto{\pgfqpoint{4.865136in}{1.722106in}}%
\pgfpathlineto{\pgfqpoint{4.857040in}{1.707872in}}%
\pgfpathclose%
\pgfusepath{fill}%
\end{pgfscope}%
\begin{pgfscope}%
\pgfpathrectangle{\pgfqpoint{1.150000in}{0.150000in}}{\pgfqpoint{5.700000in}{5.700000in}}%
\pgfusepath{clip}%
\pgfsetbuttcap%
\pgfsetroundjoin%
\definecolor{currentfill}{rgb}{0.258965,0.251537,0.524736}%
\pgfsetfillcolor{currentfill}%
\pgfsetfillopacity{0.700000}%
\pgfsetlinewidth{0.000000pt}%
\definecolor{currentstroke}{rgb}{0.000000,0.000000,0.000000}%
\pgfsetstrokecolor{currentstroke}%
\pgfsetdash{}{0pt}%
\pgfpathmoveto{\pgfqpoint{4.642140in}{1.508551in}}%
\pgfpathlineto{\pgfqpoint{4.656764in}{1.511114in}}%
\pgfpathlineto{\pgfqpoint{4.671399in}{1.513748in}}%
\pgfpathlineto{\pgfqpoint{4.686047in}{1.516452in}}%
\pgfpathlineto{\pgfqpoint{4.700706in}{1.519226in}}%
\pgfpathlineto{\pgfqpoint{4.708866in}{1.534024in}}%
\pgfpathlineto{\pgfqpoint{4.717021in}{1.548777in}}%
\pgfpathlineto{\pgfqpoint{4.725171in}{1.563481in}}%
\pgfpathlineto{\pgfqpoint{4.733316in}{1.578131in}}%
\pgfpathlineto{\pgfqpoint{4.718658in}{1.575079in}}%
\pgfpathlineto{\pgfqpoint{4.704011in}{1.572096in}}%
\pgfpathlineto{\pgfqpoint{4.689376in}{1.569184in}}%
\pgfpathlineto{\pgfqpoint{4.674754in}{1.566342in}}%
\pgfpathlineto{\pgfqpoint{4.666608in}{1.551962in}}%
\pgfpathlineto{\pgfqpoint{4.658457in}{1.537534in}}%
\pgfpathlineto{\pgfqpoint{4.650301in}{1.523062in}}%
\pgfpathlineto{\pgfqpoint{4.642140in}{1.508551in}}%
\pgfpathclose%
\pgfusepath{fill}%
\end{pgfscope}%
\begin{pgfscope}%
\pgfpathrectangle{\pgfqpoint{1.150000in}{0.150000in}}{\pgfqpoint{5.700000in}{5.700000in}}%
\pgfusepath{clip}%
\pgfsetbuttcap%
\pgfsetroundjoin%
\definecolor{currentfill}{rgb}{0.281924,0.089666,0.412415}%
\pgfsetfillcolor{currentfill}%
\pgfsetfillopacity{0.700000}%
\pgfsetlinewidth{0.000000pt}%
\definecolor{currentstroke}{rgb}{0.000000,0.000000,0.000000}%
\pgfsetstrokecolor{currentstroke}%
\pgfsetdash{}{0pt}%
\pgfpathmoveto{\pgfqpoint{4.154327in}{1.165335in}}%
\pgfpathlineto{\pgfqpoint{4.168768in}{1.164418in}}%
\pgfpathlineto{\pgfqpoint{4.183219in}{1.163569in}}%
\pgfpathlineto{\pgfqpoint{4.197678in}{1.162790in}}%
\pgfpathlineto{\pgfqpoint{4.212147in}{1.162079in}}%
\pgfpathlineto{\pgfqpoint{4.220433in}{1.174508in}}%
\pgfpathlineto{\pgfqpoint{4.228714in}{1.187038in}}%
\pgfpathlineto{\pgfqpoint{4.236990in}{1.199663in}}%
\pgfpathlineto{\pgfqpoint{4.245261in}{1.212377in}}%
\pgfpathlineto{\pgfqpoint{4.230799in}{1.212689in}}%
\pgfpathlineto{\pgfqpoint{4.216347in}{1.213069in}}%
\pgfpathlineto{\pgfqpoint{4.201904in}{1.213519in}}%
\pgfpathlineto{\pgfqpoint{4.187470in}{1.214038in}}%
\pgfpathlineto{\pgfqpoint{4.179192in}{1.201715in}}%
\pgfpathlineto{\pgfqpoint{4.170909in}{1.189487in}}%
\pgfpathlineto{\pgfqpoint{4.162620in}{1.177358in}}%
\pgfpathlineto{\pgfqpoint{4.154327in}{1.165335in}}%
\pgfpathclose%
\pgfusepath{fill}%
\end{pgfscope}%
\begin{pgfscope}%
\pgfpathrectangle{\pgfqpoint{1.150000in}{0.150000in}}{\pgfqpoint{5.700000in}{5.700000in}}%
\pgfusepath{clip}%
\pgfsetbuttcap%
\pgfsetroundjoin%
\definecolor{currentfill}{rgb}{0.283091,0.110553,0.431554}%
\pgfsetfillcolor{currentfill}%
\pgfsetfillopacity{0.700000}%
\pgfsetlinewidth{0.000000pt}%
\definecolor{currentstroke}{rgb}{0.000000,0.000000,0.000000}%
\pgfsetstrokecolor{currentstroke}%
\pgfsetdash{}{0pt}%
\pgfpathmoveto{\pgfqpoint{4.245261in}{1.212377in}}%
\pgfpathlineto{\pgfqpoint{4.259732in}{1.212135in}}%
\pgfpathlineto{\pgfqpoint{4.274213in}{1.211961in}}%
\pgfpathlineto{\pgfqpoint{4.288703in}{1.211857in}}%
\pgfpathlineto{\pgfqpoint{4.303203in}{1.211821in}}%
\pgfpathlineto{\pgfqpoint{4.311463in}{1.225005in}}%
\pgfpathlineto{\pgfqpoint{4.319719in}{1.238261in}}%
\pgfpathlineto{\pgfqpoint{4.327970in}{1.251584in}}%
\pgfpathlineto{\pgfqpoint{4.336217in}{1.264968in}}%
\pgfpathlineto{\pgfqpoint{4.321723in}{1.264625in}}%
\pgfpathlineto{\pgfqpoint{4.307238in}{1.264350in}}%
\pgfpathlineto{\pgfqpoint{4.292763in}{1.264144in}}%
\pgfpathlineto{\pgfqpoint{4.278298in}{1.264008in}}%
\pgfpathlineto{\pgfqpoint{4.270046in}{1.250995in}}%
\pgfpathlineto{\pgfqpoint{4.261789in}{1.238049in}}%
\pgfpathlineto{\pgfqpoint{4.253527in}{1.225174in}}%
\pgfpathlineto{\pgfqpoint{4.245261in}{1.212377in}}%
\pgfpathclose%
\pgfusepath{fill}%
\end{pgfscope}%
\begin{pgfscope}%
\pgfpathrectangle{\pgfqpoint{1.150000in}{0.150000in}}{\pgfqpoint{5.700000in}{5.700000in}}%
\pgfusepath{clip}%
\pgfsetbuttcap%
\pgfsetroundjoin%
\definecolor{currentfill}{rgb}{0.279566,0.067836,0.391917}%
\pgfsetfillcolor{currentfill}%
\pgfsetfillopacity{0.700000}%
\pgfsetlinewidth{0.000000pt}%
\definecolor{currentstroke}{rgb}{0.000000,0.000000,0.000000}%
\pgfsetstrokecolor{currentstroke}%
\pgfsetdash{}{0pt}%
\pgfpathmoveto{\pgfqpoint{4.063385in}{1.124462in}}%
\pgfpathlineto{\pgfqpoint{4.077801in}{1.122850in}}%
\pgfpathlineto{\pgfqpoint{4.092225in}{1.121306in}}%
\pgfpathlineto{\pgfqpoint{4.106658in}{1.119832in}}%
\pgfpathlineto{\pgfqpoint{4.121099in}{1.118427in}}%
\pgfpathlineto{\pgfqpoint{4.129414in}{1.129964in}}%
\pgfpathlineto{\pgfqpoint{4.137724in}{1.141632in}}%
\pgfpathlineto{\pgfqpoint{4.146028in}{1.153425in}}%
\pgfpathlineto{\pgfqpoint{4.154327in}{1.165335in}}%
\pgfpathlineto{\pgfqpoint{4.139894in}{1.166322in}}%
\pgfpathlineto{\pgfqpoint{4.125471in}{1.167377in}}%
\pgfpathlineto{\pgfqpoint{4.111056in}{1.168502in}}%
\pgfpathlineto{\pgfqpoint{4.096649in}{1.169696in}}%
\pgfpathlineto{\pgfqpoint{4.088342in}{1.158196in}}%
\pgfpathlineto{\pgfqpoint{4.080028in}{1.146820in}}%
\pgfpathlineto{\pgfqpoint{4.071709in}{1.135573in}}%
\pgfpathlineto{\pgfqpoint{4.063385in}{1.124462in}}%
\pgfpathclose%
\pgfusepath{fill}%
\end{pgfscope}%
\begin{pgfscope}%
\pgfpathrectangle{\pgfqpoint{1.150000in}{0.150000in}}{\pgfqpoint{5.700000in}{5.700000in}}%
\pgfusepath{clip}%
\pgfsetbuttcap%
\pgfsetroundjoin%
\definecolor{currentfill}{rgb}{0.282623,0.140926,0.457517}%
\pgfsetfillcolor{currentfill}%
\pgfsetfillopacity{0.700000}%
\pgfsetlinewidth{0.000000pt}%
\definecolor{currentstroke}{rgb}{0.000000,0.000000,0.000000}%
\pgfsetstrokecolor{currentstroke}%
\pgfsetdash{}{0pt}%
\pgfpathmoveto{\pgfqpoint{4.336217in}{1.264968in}}%
\pgfpathlineto{\pgfqpoint{4.350722in}{1.265382in}}%
\pgfpathlineto{\pgfqpoint{4.365236in}{1.265864in}}%
\pgfpathlineto{\pgfqpoint{4.379761in}{1.266415in}}%
\pgfpathlineto{\pgfqpoint{4.394296in}{1.267035in}}%
\pgfpathlineto{\pgfqpoint{4.402534in}{1.280842in}}%
\pgfpathlineto{\pgfqpoint{4.410767in}{1.294694in}}%
\pgfpathlineto{\pgfqpoint{4.418996in}{1.308586in}}%
\pgfpathlineto{\pgfqpoint{4.427221in}{1.322513in}}%
\pgfpathlineto{\pgfqpoint{4.412690in}{1.321533in}}%
\pgfpathlineto{\pgfqpoint{4.398169in}{1.320622in}}%
\pgfpathlineto{\pgfqpoint{4.383658in}{1.319781in}}%
\pgfpathlineto{\pgfqpoint{4.369158in}{1.319009in}}%
\pgfpathlineto{\pgfqpoint{4.360930in}{1.305434in}}%
\pgfpathlineto{\pgfqpoint{4.352697in}{1.291899in}}%
\pgfpathlineto{\pgfqpoint{4.344459in}{1.278408in}}%
\pgfpathlineto{\pgfqpoint{4.336217in}{1.264968in}}%
\pgfpathclose%
\pgfusepath{fill}%
\end{pgfscope}%
\begin{pgfscope}%
\pgfpathrectangle{\pgfqpoint{1.150000in}{0.150000in}}{\pgfqpoint{5.700000in}{5.700000in}}%
\pgfusepath{clip}%
\pgfsetbuttcap%
\pgfsetroundjoin%
\definecolor{currentfill}{rgb}{0.210503,0.363727,0.552206}%
\pgfsetfillcolor{currentfill}%
\pgfsetfillopacity{0.700000}%
\pgfsetlinewidth{0.000000pt}%
\definecolor{currentstroke}{rgb}{0.000000,0.000000,0.000000}%
\pgfsetstrokecolor{currentstroke}%
\pgfsetdash{}{0pt}%
\pgfpathmoveto{\pgfqpoint{4.948339in}{1.781124in}}%
\pgfpathlineto{\pgfqpoint{4.963109in}{1.785522in}}%
\pgfpathlineto{\pgfqpoint{4.977892in}{1.789992in}}%
\pgfpathlineto{\pgfqpoint{4.992689in}{1.794532in}}%
\pgfpathlineto{\pgfqpoint{5.007498in}{1.799144in}}%
\pgfpathlineto{\pgfqpoint{5.015569in}{1.813394in}}%
\pgfpathlineto{\pgfqpoint{5.023634in}{1.827529in}}%
\pgfpathlineto{\pgfqpoint{5.031692in}{1.841546in}}%
\pgfpathlineto{\pgfqpoint{5.039743in}{1.855442in}}%
\pgfpathlineto{\pgfqpoint{5.024934in}{1.850636in}}%
\pgfpathlineto{\pgfqpoint{5.010138in}{1.845901in}}%
\pgfpathlineto{\pgfqpoint{4.995355in}{1.841237in}}%
\pgfpathlineto{\pgfqpoint{4.980586in}{1.836644in}}%
\pgfpathlineto{\pgfqpoint{4.972534in}{1.822933in}}%
\pgfpathlineto{\pgfqpoint{4.964475in}{1.809108in}}%
\pgfpathlineto{\pgfqpoint{4.956410in}{1.795171in}}%
\pgfpathlineto{\pgfqpoint{4.948339in}{1.781124in}}%
\pgfpathclose%
\pgfusepath{fill}%
\end{pgfscope}%
\begin{pgfscope}%
\pgfpathrectangle{\pgfqpoint{1.150000in}{0.150000in}}{\pgfqpoint{5.700000in}{5.700000in}}%
\pgfusepath{clip}%
\pgfsetbuttcap%
\pgfsetroundjoin%
\definecolor{currentfill}{rgb}{0.246811,0.283237,0.535941}%
\pgfsetfillcolor{currentfill}%
\pgfsetfillopacity{0.700000}%
\pgfsetlinewidth{0.000000pt}%
\definecolor{currentstroke}{rgb}{0.000000,0.000000,0.000000}%
\pgfsetstrokecolor{currentstroke}%
\pgfsetdash{}{0pt}%
\pgfpathmoveto{\pgfqpoint{4.733316in}{1.578131in}}%
\pgfpathlineto{\pgfqpoint{4.747987in}{1.581254in}}%
\pgfpathlineto{\pgfqpoint{4.762669in}{1.584447in}}%
\pgfpathlineto{\pgfqpoint{4.777364in}{1.587710in}}%
\pgfpathlineto{\pgfqpoint{4.792071in}{1.591044in}}%
\pgfpathlineto{\pgfqpoint{4.800211in}{1.605903in}}%
\pgfpathlineto{\pgfqpoint{4.808345in}{1.620697in}}%
\pgfpathlineto{\pgfqpoint{4.816475in}{1.635420in}}%
\pgfpathlineto{\pgfqpoint{4.824599in}{1.650071in}}%
\pgfpathlineto{\pgfqpoint{4.809892in}{1.646479in}}%
\pgfpathlineto{\pgfqpoint{4.795197in}{1.642958in}}%
\pgfpathlineto{\pgfqpoint{4.780515in}{1.639507in}}%
\pgfpathlineto{\pgfqpoint{4.765845in}{1.636126in}}%
\pgfpathlineto{\pgfqpoint{4.757721in}{1.621726in}}%
\pgfpathlineto{\pgfqpoint{4.749591in}{1.607257in}}%
\pgfpathlineto{\pgfqpoint{4.741456in}{1.592725in}}%
\pgfpathlineto{\pgfqpoint{4.733316in}{1.578131in}}%
\pgfpathclose%
\pgfusepath{fill}%
\end{pgfscope}%
\begin{pgfscope}%
\pgfpathrectangle{\pgfqpoint{1.150000in}{0.150000in}}{\pgfqpoint{5.700000in}{5.700000in}}%
\pgfusepath{clip}%
\pgfsetbuttcap%
\pgfsetroundjoin%
\definecolor{currentfill}{rgb}{0.279574,0.170599,0.479997}%
\pgfsetfillcolor{currentfill}%
\pgfsetfillopacity{0.700000}%
\pgfsetlinewidth{0.000000pt}%
\definecolor{currentstroke}{rgb}{0.000000,0.000000,0.000000}%
\pgfsetstrokecolor{currentstroke}%
\pgfsetdash{}{0pt}%
\pgfpathmoveto{\pgfqpoint{4.427221in}{1.322513in}}%
\pgfpathlineto{\pgfqpoint{4.441762in}{1.323563in}}%
\pgfpathlineto{\pgfqpoint{4.456314in}{1.324681in}}%
\pgfpathlineto{\pgfqpoint{4.470877in}{1.325869in}}%
\pgfpathlineto{\pgfqpoint{4.485450in}{1.327126in}}%
\pgfpathlineto{\pgfqpoint{4.493667in}{1.341430in}}%
\pgfpathlineto{\pgfqpoint{4.501880in}{1.355753in}}%
\pgfpathlineto{\pgfqpoint{4.510088in}{1.370090in}}%
\pgfpathlineto{\pgfqpoint{4.518292in}{1.384437in}}%
\pgfpathlineto{\pgfqpoint{4.503721in}{1.382840in}}%
\pgfpathlineto{\pgfqpoint{4.489161in}{1.381312in}}%
\pgfpathlineto{\pgfqpoint{4.474612in}{1.379854in}}%
\pgfpathlineto{\pgfqpoint{4.460074in}{1.378466in}}%
\pgfpathlineto{\pgfqpoint{4.451867in}{1.364451in}}%
\pgfpathlineto{\pgfqpoint{4.443656in}{1.350450in}}%
\pgfpathlineto{\pgfqpoint{4.435441in}{1.336469in}}%
\pgfpathlineto{\pgfqpoint{4.427221in}{1.322513in}}%
\pgfpathclose%
\pgfusepath{fill}%
\end{pgfscope}%
\begin{pgfscope}%
\pgfpathrectangle{\pgfqpoint{1.150000in}{0.150000in}}{\pgfqpoint{5.700000in}{5.700000in}}%
\pgfusepath{clip}%
\pgfsetbuttcap%
\pgfsetroundjoin%
\definecolor{currentfill}{rgb}{0.197636,0.391528,0.554969}%
\pgfsetfillcolor{currentfill}%
\pgfsetfillopacity{0.700000}%
\pgfsetlinewidth{0.000000pt}%
\definecolor{currentstroke}{rgb}{0.000000,0.000000,0.000000}%
\pgfsetstrokecolor{currentstroke}%
\pgfsetdash{}{0pt}%
\pgfpathmoveto{\pgfqpoint{5.039743in}{1.855442in}}%
\pgfpathlineto{\pgfqpoint{5.054566in}{1.860320in}}%
\pgfpathlineto{\pgfqpoint{5.069402in}{1.865269in}}%
\pgfpathlineto{\pgfqpoint{5.084252in}{1.870290in}}%
\pgfpathlineto{\pgfqpoint{5.092296in}{1.884200in}}%
\pgfpathlineto{\pgfqpoint{5.100333in}{1.897980in}}%
\pgfpathlineto{\pgfqpoint{5.108363in}{1.911629in}}%
\pgfpathlineto{\pgfqpoint{5.116386in}{1.925144in}}%
\pgfpathlineto{\pgfqpoint{5.101537in}{1.919951in}}%
\pgfpathlineto{\pgfqpoint{5.086701in}{1.914829in}}%
\pgfpathlineto{\pgfqpoint{5.071880in}{1.909778in}}%
\pgfpathlineto{\pgfqpoint{5.063856in}{1.896386in}}%
\pgfpathlineto{\pgfqpoint{5.055825in}{1.882865in}}%
\pgfpathlineto{\pgfqpoint{5.047788in}{1.869216in}}%
\pgfpathlineto{\pgfqpoint{5.039743in}{1.855442in}}%
\pgfpathclose%
\pgfusepath{fill}%
\end{pgfscope}%
\begin{pgfscope}%
\pgfpathrectangle{\pgfqpoint{1.150000in}{0.150000in}}{\pgfqpoint{5.700000in}{5.700000in}}%
\pgfusepath{clip}%
\pgfsetbuttcap%
\pgfsetroundjoin%
\definecolor{currentfill}{rgb}{0.274128,0.199721,0.498911}%
\pgfsetfillcolor{currentfill}%
\pgfsetfillopacity{0.700000}%
\pgfsetlinewidth{0.000000pt}%
\definecolor{currentstroke}{rgb}{0.000000,0.000000,0.000000}%
\pgfsetstrokecolor{currentstroke}%
\pgfsetdash{}{0pt}%
\pgfpathmoveto{\pgfqpoint{4.518292in}{1.384437in}}%
\pgfpathlineto{\pgfqpoint{4.532874in}{1.386103in}}%
\pgfpathlineto{\pgfqpoint{4.547467in}{1.387839in}}%
\pgfpathlineto{\pgfqpoint{4.562070in}{1.389645in}}%
\pgfpathlineto{\pgfqpoint{4.576685in}{1.391520in}}%
\pgfpathlineto{\pgfqpoint{4.584883in}{1.406199in}}%
\pgfpathlineto{\pgfqpoint{4.593076in}{1.420872in}}%
\pgfpathlineto{\pgfqpoint{4.601265in}{1.435536in}}%
\pgfpathlineto{\pgfqpoint{4.609449in}{1.450186in}}%
\pgfpathlineto{\pgfqpoint{4.594836in}{1.447991in}}%
\pgfpathlineto{\pgfqpoint{4.580234in}{1.445865in}}%
\pgfpathlineto{\pgfqpoint{4.565643in}{1.443810in}}%
\pgfpathlineto{\pgfqpoint{4.551063in}{1.441824in}}%
\pgfpathlineto{\pgfqpoint{4.542877in}{1.427486in}}%
\pgfpathlineto{\pgfqpoint{4.534687in}{1.413140in}}%
\pgfpathlineto{\pgfqpoint{4.526492in}{1.398788in}}%
\pgfpathlineto{\pgfqpoint{4.518292in}{1.384437in}}%
\pgfpathclose%
\pgfusepath{fill}%
\end{pgfscope}%
\begin{pgfscope}%
\pgfpathrectangle{\pgfqpoint{1.150000in}{0.150000in}}{\pgfqpoint{5.700000in}{5.700000in}}%
\pgfusepath{clip}%
\pgfsetbuttcap%
\pgfsetroundjoin%
\definecolor{currentfill}{rgb}{0.231674,0.318106,0.544834}%
\pgfsetfillcolor{currentfill}%
\pgfsetfillopacity{0.700000}%
\pgfsetlinewidth{0.000000pt}%
\definecolor{currentstroke}{rgb}{0.000000,0.000000,0.000000}%
\pgfsetstrokecolor{currentstroke}%
\pgfsetdash{}{0pt}%
\pgfpathmoveto{\pgfqpoint{4.824599in}{1.650071in}}%
\pgfpathlineto{\pgfqpoint{4.839318in}{1.653733in}}%
\pgfpathlineto{\pgfqpoint{4.854050in}{1.657466in}}%
\pgfpathlineto{\pgfqpoint{4.868795in}{1.661270in}}%
\pgfpathlineto{\pgfqpoint{4.883552in}{1.665144in}}%
\pgfpathlineto{\pgfqpoint{4.891670in}{1.679963in}}%
\pgfpathlineto{\pgfqpoint{4.899783in}{1.694697in}}%
\pgfpathlineto{\pgfqpoint{4.907891in}{1.709342in}}%
\pgfpathlineto{\pgfqpoint{4.915992in}{1.723894in}}%
\pgfpathlineto{\pgfqpoint{4.901235in}{1.719783in}}%
\pgfpathlineto{\pgfqpoint{4.886491in}{1.715742in}}%
\pgfpathlineto{\pgfqpoint{4.871759in}{1.711772in}}%
\pgfpathlineto{\pgfqpoint{4.857040in}{1.707872in}}%
\pgfpathlineto{\pgfqpoint{4.848938in}{1.693549in}}%
\pgfpathlineto{\pgfqpoint{4.840830in}{1.679138in}}%
\pgfpathlineto{\pgfqpoint{4.832717in}{1.664645in}}%
\pgfpathlineto{\pgfqpoint{4.824599in}{1.650071in}}%
\pgfpathclose%
\pgfusepath{fill}%
\end{pgfscope}%
\begin{pgfscope}%
\pgfpathrectangle{\pgfqpoint{1.150000in}{0.150000in}}{\pgfqpoint{5.700000in}{5.700000in}}%
\pgfusepath{clip}%
\pgfsetbuttcap%
\pgfsetroundjoin%
\definecolor{currentfill}{rgb}{0.265145,0.232956,0.516599}%
\pgfsetfillcolor{currentfill}%
\pgfsetfillopacity{0.700000}%
\pgfsetlinewidth{0.000000pt}%
\definecolor{currentstroke}{rgb}{0.000000,0.000000,0.000000}%
\pgfsetstrokecolor{currentstroke}%
\pgfsetdash{}{0pt}%
\pgfpathmoveto{\pgfqpoint{4.609449in}{1.450186in}}%
\pgfpathlineto{\pgfqpoint{4.624074in}{1.452450in}}%
\pgfpathlineto{\pgfqpoint{4.638711in}{1.454785in}}%
\pgfpathlineto{\pgfqpoint{4.653359in}{1.457189in}}%
\pgfpathlineto{\pgfqpoint{4.668018in}{1.459662in}}%
\pgfpathlineto{\pgfqpoint{4.676197in}{1.474600in}}%
\pgfpathlineto{\pgfqpoint{4.684371in}{1.489510in}}%
\pgfpathlineto{\pgfqpoint{4.692541in}{1.504386in}}%
\pgfpathlineto{\pgfqpoint{4.700706in}{1.519226in}}%
\pgfpathlineto{\pgfqpoint{4.686047in}{1.516452in}}%
\pgfpathlineto{\pgfqpoint{4.671399in}{1.513748in}}%
\pgfpathlineto{\pgfqpoint{4.656764in}{1.511114in}}%
\pgfpathlineto{\pgfqpoint{4.642140in}{1.508551in}}%
\pgfpathlineto{\pgfqpoint{4.633974in}{1.494003in}}%
\pgfpathlineto{\pgfqpoint{4.625804in}{1.479423in}}%
\pgfpathlineto{\pgfqpoint{4.617629in}{1.464816in}}%
\pgfpathlineto{\pgfqpoint{4.609449in}{1.450186in}}%
\pgfpathclose%
\pgfusepath{fill}%
\end{pgfscope}%
\begin{pgfscope}%
\pgfpathrectangle{\pgfqpoint{1.150000in}{0.150000in}}{\pgfqpoint{5.700000in}{5.700000in}}%
\pgfusepath{clip}%
\pgfsetbuttcap%
\pgfsetroundjoin%
\definecolor{currentfill}{rgb}{0.282656,0.100196,0.422160}%
\pgfsetfillcolor{currentfill}%
\pgfsetfillopacity{0.700000}%
\pgfsetlinewidth{0.000000pt}%
\definecolor{currentstroke}{rgb}{0.000000,0.000000,0.000000}%
\pgfsetstrokecolor{currentstroke}%
\pgfsetdash{}{0pt}%
\pgfpathmoveto{\pgfqpoint{4.212147in}{1.162079in}}%
\pgfpathlineto{\pgfqpoint{4.226624in}{1.161437in}}%
\pgfpathlineto{\pgfqpoint{4.241111in}{1.160864in}}%
\pgfpathlineto{\pgfqpoint{4.255607in}{1.160360in}}%
\pgfpathlineto{\pgfqpoint{4.270113in}{1.159924in}}%
\pgfpathlineto{\pgfqpoint{4.278393in}{1.172761in}}%
\pgfpathlineto{\pgfqpoint{4.286667in}{1.185693in}}%
\pgfpathlineto{\pgfqpoint{4.294937in}{1.198715in}}%
\pgfpathlineto{\pgfqpoint{4.303203in}{1.211821in}}%
\pgfpathlineto{\pgfqpoint{4.288703in}{1.211857in}}%
\pgfpathlineto{\pgfqpoint{4.274213in}{1.211961in}}%
\pgfpathlineto{\pgfqpoint{4.259732in}{1.212135in}}%
\pgfpathlineto{\pgfqpoint{4.245261in}{1.212377in}}%
\pgfpathlineto{\pgfqpoint{4.236990in}{1.199663in}}%
\pgfpathlineto{\pgfqpoint{4.228714in}{1.187038in}}%
\pgfpathlineto{\pgfqpoint{4.220433in}{1.174508in}}%
\pgfpathlineto{\pgfqpoint{4.212147in}{1.162079in}}%
\pgfpathclose%
\pgfusepath{fill}%
\end{pgfscope}%
\begin{pgfscope}%
\pgfpathrectangle{\pgfqpoint{1.150000in}{0.150000in}}{\pgfqpoint{5.700000in}{5.700000in}}%
\pgfusepath{clip}%
\pgfsetbuttcap%
\pgfsetroundjoin%
\definecolor{currentfill}{rgb}{0.280267,0.073417,0.397163}%
\pgfsetfillcolor{currentfill}%
\pgfsetfillopacity{0.700000}%
\pgfsetlinewidth{0.000000pt}%
\definecolor{currentstroke}{rgb}{0.000000,0.000000,0.000000}%
\pgfsetstrokecolor{currentstroke}%
\pgfsetdash{}{0pt}%
\pgfpathmoveto{\pgfqpoint{4.121099in}{1.118427in}}%
\pgfpathlineto{\pgfqpoint{4.135549in}{1.117090in}}%
\pgfpathlineto{\pgfqpoint{4.150008in}{1.115822in}}%
\pgfpathlineto{\pgfqpoint{4.164475in}{1.114623in}}%
\pgfpathlineto{\pgfqpoint{4.178952in}{1.113492in}}%
\pgfpathlineto{\pgfqpoint{4.187258in}{1.125457in}}%
\pgfpathlineto{\pgfqpoint{4.195559in}{1.137547in}}%
\pgfpathlineto{\pgfqpoint{4.203855in}{1.149757in}}%
\pgfpathlineto{\pgfqpoint{4.212147in}{1.162079in}}%
\pgfpathlineto{\pgfqpoint{4.197678in}{1.162790in}}%
\pgfpathlineto{\pgfqpoint{4.183219in}{1.163569in}}%
\pgfpathlineto{\pgfqpoint{4.168768in}{1.164418in}}%
\pgfpathlineto{\pgfqpoint{4.154327in}{1.165335in}}%
\pgfpathlineto{\pgfqpoint{4.146028in}{1.153425in}}%
\pgfpathlineto{\pgfqpoint{4.137724in}{1.141632in}}%
\pgfpathlineto{\pgfqpoint{4.129414in}{1.129964in}}%
\pgfpathlineto{\pgfqpoint{4.121099in}{1.118427in}}%
\pgfpathclose%
\pgfusepath{fill}%
\end{pgfscope}%
\begin{pgfscope}%
\pgfpathrectangle{\pgfqpoint{1.150000in}{0.150000in}}{\pgfqpoint{5.700000in}{5.700000in}}%
\pgfusepath{clip}%
\pgfsetbuttcap%
\pgfsetroundjoin%
\definecolor{currentfill}{rgb}{0.218130,0.347432,0.550038}%
\pgfsetfillcolor{currentfill}%
\pgfsetfillopacity{0.700000}%
\pgfsetlinewidth{0.000000pt}%
\definecolor{currentstroke}{rgb}{0.000000,0.000000,0.000000}%
\pgfsetstrokecolor{currentstroke}%
\pgfsetdash{}{0pt}%
\pgfpathmoveto{\pgfqpoint{4.915992in}{1.723894in}}%
\pgfpathlineto{\pgfqpoint{4.930763in}{1.728077in}}%
\pgfpathlineto{\pgfqpoint{4.945546in}{1.732330in}}%
\pgfpathlineto{\pgfqpoint{4.960342in}{1.736654in}}%
\pgfpathlineto{\pgfqpoint{4.975152in}{1.741050in}}%
\pgfpathlineto{\pgfqpoint{4.983248in}{1.755732in}}%
\pgfpathlineto{\pgfqpoint{4.991337in}{1.770310in}}%
\pgfpathlineto{\pgfqpoint{4.999421in}{1.784782in}}%
\pgfpathlineto{\pgfqpoint{5.007498in}{1.799144in}}%
\pgfpathlineto{\pgfqpoint{4.992689in}{1.794532in}}%
\pgfpathlineto{\pgfqpoint{4.977892in}{1.789992in}}%
\pgfpathlineto{\pgfqpoint{4.963109in}{1.785522in}}%
\pgfpathlineto{\pgfqpoint{4.948339in}{1.781124in}}%
\pgfpathlineto{\pgfqpoint{4.940262in}{1.766970in}}%
\pgfpathlineto{\pgfqpoint{4.932178in}{1.752711in}}%
\pgfpathlineto{\pgfqpoint{4.924088in}{1.738352in}}%
\pgfpathlineto{\pgfqpoint{4.915992in}{1.723894in}}%
\pgfpathclose%
\pgfusepath{fill}%
\end{pgfscope}%
\begin{pgfscope}%
\pgfpathrectangle{\pgfqpoint{1.150000in}{0.150000in}}{\pgfqpoint{5.700000in}{5.700000in}}%
\pgfusepath{clip}%
\pgfsetbuttcap%
\pgfsetroundjoin%
\definecolor{currentfill}{rgb}{0.283187,0.125848,0.444960}%
\pgfsetfillcolor{currentfill}%
\pgfsetfillopacity{0.700000}%
\pgfsetlinewidth{0.000000pt}%
\definecolor{currentstroke}{rgb}{0.000000,0.000000,0.000000}%
\pgfsetstrokecolor{currentstroke}%
\pgfsetdash{}{0pt}%
\pgfpathmoveto{\pgfqpoint{4.303203in}{1.211821in}}%
\pgfpathlineto{\pgfqpoint{4.317712in}{1.211854in}}%
\pgfpathlineto{\pgfqpoint{4.332231in}{1.211957in}}%
\pgfpathlineto{\pgfqpoint{4.346760in}{1.212128in}}%
\pgfpathlineto{\pgfqpoint{4.361299in}{1.212368in}}%
\pgfpathlineto{\pgfqpoint{4.369555in}{1.225939in}}%
\pgfpathlineto{\pgfqpoint{4.377806in}{1.239578in}}%
\pgfpathlineto{\pgfqpoint{4.386053in}{1.253279in}}%
\pgfpathlineto{\pgfqpoint{4.394296in}{1.267035in}}%
\pgfpathlineto{\pgfqpoint{4.379761in}{1.266415in}}%
\pgfpathlineto{\pgfqpoint{4.365236in}{1.265864in}}%
\pgfpathlineto{\pgfqpoint{4.350722in}{1.265382in}}%
\pgfpathlineto{\pgfqpoint{4.336217in}{1.264968in}}%
\pgfpathlineto{\pgfqpoint{4.327970in}{1.251584in}}%
\pgfpathlineto{\pgfqpoint{4.319719in}{1.238261in}}%
\pgfpathlineto{\pgfqpoint{4.311463in}{1.225005in}}%
\pgfpathlineto{\pgfqpoint{4.303203in}{1.211821in}}%
\pgfpathclose%
\pgfusepath{fill}%
\end{pgfscope}%
\begin{pgfscope}%
\pgfpathrectangle{\pgfqpoint{1.150000in}{0.150000in}}{\pgfqpoint{5.700000in}{5.700000in}}%
\pgfusepath{clip}%
\pgfsetbuttcap%
\pgfsetroundjoin%
\definecolor{currentfill}{rgb}{0.253935,0.265254,0.529983}%
\pgfsetfillcolor{currentfill}%
\pgfsetfillopacity{0.700000}%
\pgfsetlinewidth{0.000000pt}%
\definecolor{currentstroke}{rgb}{0.000000,0.000000,0.000000}%
\pgfsetstrokecolor{currentstroke}%
\pgfsetdash{}{0pt}%
\pgfpathmoveto{\pgfqpoint{4.700706in}{1.519226in}}%
\pgfpathlineto{\pgfqpoint{4.715377in}{1.522070in}}%
\pgfpathlineto{\pgfqpoint{4.730059in}{1.524983in}}%
\pgfpathlineto{\pgfqpoint{4.744754in}{1.527967in}}%
\pgfpathlineto{\pgfqpoint{4.759461in}{1.531021in}}%
\pgfpathlineto{\pgfqpoint{4.767621in}{1.546107in}}%
\pgfpathlineto{\pgfqpoint{4.775776in}{1.561141in}}%
\pgfpathlineto{\pgfqpoint{4.783926in}{1.576122in}}%
\pgfpathlineto{\pgfqpoint{4.792071in}{1.591044in}}%
\pgfpathlineto{\pgfqpoint{4.777364in}{1.587710in}}%
\pgfpathlineto{\pgfqpoint{4.762669in}{1.584447in}}%
\pgfpathlineto{\pgfqpoint{4.747987in}{1.581254in}}%
\pgfpathlineto{\pgfqpoint{4.733316in}{1.578131in}}%
\pgfpathlineto{\pgfqpoint{4.725171in}{1.563481in}}%
\pgfpathlineto{\pgfqpoint{4.717021in}{1.548777in}}%
\pgfpathlineto{\pgfqpoint{4.708866in}{1.534024in}}%
\pgfpathlineto{\pgfqpoint{4.700706in}{1.519226in}}%
\pgfpathclose%
\pgfusepath{fill}%
\end{pgfscope}%
\begin{pgfscope}%
\pgfpathrectangle{\pgfqpoint{1.150000in}{0.150000in}}{\pgfqpoint{5.700000in}{5.700000in}}%
\pgfusepath{clip}%
\pgfsetbuttcap%
\pgfsetroundjoin%
\definecolor{currentfill}{rgb}{0.281887,0.150881,0.465405}%
\pgfsetfillcolor{currentfill}%
\pgfsetfillopacity{0.700000}%
\pgfsetlinewidth{0.000000pt}%
\definecolor{currentstroke}{rgb}{0.000000,0.000000,0.000000}%
\pgfsetstrokecolor{currentstroke}%
\pgfsetdash{}{0pt}%
\pgfpathmoveto{\pgfqpoint{4.394296in}{1.267035in}}%
\pgfpathlineto{\pgfqpoint{4.408840in}{1.267725in}}%
\pgfpathlineto{\pgfqpoint{4.423396in}{1.268483in}}%
\pgfpathlineto{\pgfqpoint{4.437961in}{1.269310in}}%
\pgfpathlineto{\pgfqpoint{4.452537in}{1.270206in}}%
\pgfpathlineto{\pgfqpoint{4.460772in}{1.284382in}}%
\pgfpathlineto{\pgfqpoint{4.469002in}{1.298597in}}%
\pgfpathlineto{\pgfqpoint{4.477228in}{1.312847in}}%
\pgfpathlineto{\pgfqpoint{4.485450in}{1.327126in}}%
\pgfpathlineto{\pgfqpoint{4.470877in}{1.325869in}}%
\pgfpathlineto{\pgfqpoint{4.456314in}{1.324681in}}%
\pgfpathlineto{\pgfqpoint{4.441762in}{1.323563in}}%
\pgfpathlineto{\pgfqpoint{4.427221in}{1.322513in}}%
\pgfpathlineto{\pgfqpoint{4.418996in}{1.308586in}}%
\pgfpathlineto{\pgfqpoint{4.410767in}{1.294694in}}%
\pgfpathlineto{\pgfqpoint{4.402534in}{1.280842in}}%
\pgfpathlineto{\pgfqpoint{4.394296in}{1.267035in}}%
\pgfpathclose%
\pgfusepath{fill}%
\end{pgfscope}%
\begin{pgfscope}%
\pgfpathrectangle{\pgfqpoint{1.150000in}{0.150000in}}{\pgfqpoint{5.700000in}{5.700000in}}%
\pgfusepath{clip}%
\pgfsetbuttcap%
\pgfsetroundjoin%
\definecolor{currentfill}{rgb}{0.203063,0.379716,0.553925}%
\pgfsetfillcolor{currentfill}%
\pgfsetfillopacity{0.700000}%
\pgfsetlinewidth{0.000000pt}%
\definecolor{currentstroke}{rgb}{0.000000,0.000000,0.000000}%
\pgfsetstrokecolor{currentstroke}%
\pgfsetdash{}{0pt}%
\pgfpathmoveto{\pgfqpoint{5.007498in}{1.799144in}}%
\pgfpathlineto{\pgfqpoint{5.022321in}{1.803827in}}%
\pgfpathlineto{\pgfqpoint{5.037158in}{1.808582in}}%
\pgfpathlineto{\pgfqpoint{5.052008in}{1.813407in}}%
\pgfpathlineto{\pgfqpoint{5.060079in}{1.827809in}}%
\pgfpathlineto{\pgfqpoint{5.068143in}{1.842092in}}%
\pgfpathlineto{\pgfqpoint{5.076201in}{1.856253in}}%
\pgfpathlineto{\pgfqpoint{5.084252in}{1.870290in}}%
\pgfpathlineto{\pgfqpoint{5.069402in}{1.865269in}}%
\pgfpathlineto{\pgfqpoint{5.054566in}{1.860320in}}%
\pgfpathlineto{\pgfqpoint{5.039743in}{1.855442in}}%
\pgfpathlineto{\pgfqpoint{5.031692in}{1.841546in}}%
\pgfpathlineto{\pgfqpoint{5.023634in}{1.827529in}}%
\pgfpathlineto{\pgfqpoint{5.015569in}{1.813394in}}%
\pgfpathlineto{\pgfqpoint{5.007498in}{1.799144in}}%
\pgfpathclose%
\pgfusepath{fill}%
\end{pgfscope}%
\begin{pgfscope}%
\pgfpathrectangle{\pgfqpoint{1.150000in}{0.150000in}}{\pgfqpoint{5.700000in}{5.700000in}}%
\pgfusepath{clip}%
\pgfsetbuttcap%
\pgfsetroundjoin%
\definecolor{currentfill}{rgb}{0.277134,0.185228,0.489898}%
\pgfsetfillcolor{currentfill}%
\pgfsetfillopacity{0.700000}%
\pgfsetlinewidth{0.000000pt}%
\definecolor{currentstroke}{rgb}{0.000000,0.000000,0.000000}%
\pgfsetstrokecolor{currentstroke}%
\pgfsetdash{}{0pt}%
\pgfpathmoveto{\pgfqpoint{4.485450in}{1.327126in}}%
\pgfpathlineto{\pgfqpoint{4.500034in}{1.328452in}}%
\pgfpathlineto{\pgfqpoint{4.514628in}{1.329848in}}%
\pgfpathlineto{\pgfqpoint{4.529234in}{1.331312in}}%
\pgfpathlineto{\pgfqpoint{4.543850in}{1.332846in}}%
\pgfpathlineto{\pgfqpoint{4.552065in}{1.347498in}}%
\pgfpathlineto{\pgfqpoint{4.560276in}{1.362165in}}%
\pgfpathlineto{\pgfqpoint{4.568483in}{1.376840in}}%
\pgfpathlineto{\pgfqpoint{4.576685in}{1.391520in}}%
\pgfpathlineto{\pgfqpoint{4.562070in}{1.389645in}}%
\pgfpathlineto{\pgfqpoint{4.547467in}{1.387839in}}%
\pgfpathlineto{\pgfqpoint{4.532874in}{1.386103in}}%
\pgfpathlineto{\pgfqpoint{4.518292in}{1.384437in}}%
\pgfpathlineto{\pgfqpoint{4.510088in}{1.370090in}}%
\pgfpathlineto{\pgfqpoint{4.501880in}{1.355753in}}%
\pgfpathlineto{\pgfqpoint{4.493667in}{1.341430in}}%
\pgfpathlineto{\pgfqpoint{4.485450in}{1.327126in}}%
\pgfpathclose%
\pgfusepath{fill}%
\end{pgfscope}%
\begin{pgfscope}%
\pgfpathrectangle{\pgfqpoint{1.150000in}{0.150000in}}{\pgfqpoint{5.700000in}{5.700000in}}%
\pgfusepath{clip}%
\pgfsetbuttcap%
\pgfsetroundjoin%
\definecolor{currentfill}{rgb}{0.239346,0.300855,0.540844}%
\pgfsetfillcolor{currentfill}%
\pgfsetfillopacity{0.700000}%
\pgfsetlinewidth{0.000000pt}%
\definecolor{currentstroke}{rgb}{0.000000,0.000000,0.000000}%
\pgfsetstrokecolor{currentstroke}%
\pgfsetdash{}{0pt}%
\pgfpathmoveto{\pgfqpoint{4.792071in}{1.591044in}}%
\pgfpathlineto{\pgfqpoint{4.806790in}{1.594447in}}%
\pgfpathlineto{\pgfqpoint{4.821522in}{1.597922in}}%
\pgfpathlineto{\pgfqpoint{4.836266in}{1.601466in}}%
\pgfpathlineto{\pgfqpoint{4.851023in}{1.605081in}}%
\pgfpathlineto{\pgfqpoint{4.859163in}{1.620207in}}%
\pgfpathlineto{\pgfqpoint{4.867298in}{1.635262in}}%
\pgfpathlineto{\pgfqpoint{4.875428in}{1.650242in}}%
\pgfpathlineto{\pgfqpoint{4.883552in}{1.665144in}}%
\pgfpathlineto{\pgfqpoint{4.868795in}{1.661270in}}%
\pgfpathlineto{\pgfqpoint{4.854050in}{1.657466in}}%
\pgfpathlineto{\pgfqpoint{4.839318in}{1.653733in}}%
\pgfpathlineto{\pgfqpoint{4.824599in}{1.650071in}}%
\pgfpathlineto{\pgfqpoint{4.816475in}{1.635420in}}%
\pgfpathlineto{\pgfqpoint{4.808345in}{1.620697in}}%
\pgfpathlineto{\pgfqpoint{4.800211in}{1.605903in}}%
\pgfpathlineto{\pgfqpoint{4.792071in}{1.591044in}}%
\pgfpathclose%
\pgfusepath{fill}%
\end{pgfscope}%
\begin{pgfscope}%
\pgfpathrectangle{\pgfqpoint{1.150000in}{0.150000in}}{\pgfqpoint{5.700000in}{5.700000in}}%
\pgfusepath{clip}%
\pgfsetbuttcap%
\pgfsetroundjoin%
\definecolor{currentfill}{rgb}{0.270595,0.214069,0.507052}%
\pgfsetfillcolor{currentfill}%
\pgfsetfillopacity{0.700000}%
\pgfsetlinewidth{0.000000pt}%
\definecolor{currentstroke}{rgb}{0.000000,0.000000,0.000000}%
\pgfsetstrokecolor{currentstroke}%
\pgfsetdash{}{0pt}%
\pgfpathmoveto{\pgfqpoint{4.576685in}{1.391520in}}%
\pgfpathlineto{\pgfqpoint{4.591311in}{1.393464in}}%
\pgfpathlineto{\pgfqpoint{4.605948in}{1.395478in}}%
\pgfpathlineto{\pgfqpoint{4.620597in}{1.397561in}}%
\pgfpathlineto{\pgfqpoint{4.635257in}{1.399713in}}%
\pgfpathlineto{\pgfqpoint{4.643454in}{1.414721in}}%
\pgfpathlineto{\pgfqpoint{4.651646in}{1.429718in}}%
\pgfpathlineto{\pgfqpoint{4.659835in}{1.444700in}}%
\pgfpathlineto{\pgfqpoint{4.668018in}{1.459662in}}%
\pgfpathlineto{\pgfqpoint{4.653359in}{1.457189in}}%
\pgfpathlineto{\pgfqpoint{4.638711in}{1.454785in}}%
\pgfpathlineto{\pgfqpoint{4.624074in}{1.452450in}}%
\pgfpathlineto{\pgfqpoint{4.609449in}{1.450186in}}%
\pgfpathlineto{\pgfqpoint{4.601265in}{1.435536in}}%
\pgfpathlineto{\pgfqpoint{4.593076in}{1.420872in}}%
\pgfpathlineto{\pgfqpoint{4.584883in}{1.406199in}}%
\pgfpathlineto{\pgfqpoint{4.576685in}{1.391520in}}%
\pgfpathclose%
\pgfusepath{fill}%
\end{pgfscope}%
\begin{pgfscope}%
\pgfpathrectangle{\pgfqpoint{1.150000in}{0.150000in}}{\pgfqpoint{5.700000in}{5.700000in}}%
\pgfusepath{clip}%
\pgfsetbuttcap%
\pgfsetroundjoin%
\definecolor{currentfill}{rgb}{0.223925,0.334994,0.548053}%
\pgfsetfillcolor{currentfill}%
\pgfsetfillopacity{0.700000}%
\pgfsetlinewidth{0.000000pt}%
\definecolor{currentstroke}{rgb}{0.000000,0.000000,0.000000}%
\pgfsetstrokecolor{currentstroke}%
\pgfsetdash{}{0pt}%
\pgfpathmoveto{\pgfqpoint{4.883552in}{1.665144in}}%
\pgfpathlineto{\pgfqpoint{4.898322in}{1.669088in}}%
\pgfpathlineto{\pgfqpoint{4.913104in}{1.673104in}}%
\pgfpathlineto{\pgfqpoint{4.927900in}{1.677190in}}%
\pgfpathlineto{\pgfqpoint{4.942709in}{1.681346in}}%
\pgfpathlineto{\pgfqpoint{4.950828in}{1.696412in}}%
\pgfpathlineto{\pgfqpoint{4.958942in}{1.711386in}}%
\pgfpathlineto{\pgfqpoint{4.967050in}{1.726267in}}%
\pgfpathlineto{\pgfqpoint{4.975152in}{1.741050in}}%
\pgfpathlineto{\pgfqpoint{4.960342in}{1.736654in}}%
\pgfpathlineto{\pgfqpoint{4.945546in}{1.732330in}}%
\pgfpathlineto{\pgfqpoint{4.930763in}{1.728077in}}%
\pgfpathlineto{\pgfqpoint{4.915992in}{1.723894in}}%
\pgfpathlineto{\pgfqpoint{4.907891in}{1.709342in}}%
\pgfpathlineto{\pgfqpoint{4.899783in}{1.694697in}}%
\pgfpathlineto{\pgfqpoint{4.891670in}{1.679963in}}%
\pgfpathlineto{\pgfqpoint{4.883552in}{1.665144in}}%
\pgfpathclose%
\pgfusepath{fill}%
\end{pgfscope}%
\begin{pgfscope}%
\pgfpathrectangle{\pgfqpoint{1.150000in}{0.150000in}}{\pgfqpoint{5.700000in}{5.700000in}}%
\pgfusepath{clip}%
\pgfsetbuttcap%
\pgfsetroundjoin%
\definecolor{currentfill}{rgb}{0.258965,0.251537,0.524736}%
\pgfsetfillcolor{currentfill}%
\pgfsetfillopacity{0.700000}%
\pgfsetlinewidth{0.000000pt}%
\definecolor{currentstroke}{rgb}{0.000000,0.000000,0.000000}%
\pgfsetstrokecolor{currentstroke}%
\pgfsetdash{}{0pt}%
\pgfpathmoveto{\pgfqpoint{4.668018in}{1.459662in}}%
\pgfpathlineto{\pgfqpoint{4.682689in}{1.462206in}}%
\pgfpathlineto{\pgfqpoint{4.697372in}{1.464819in}}%
\pgfpathlineto{\pgfqpoint{4.712067in}{1.467502in}}%
\pgfpathlineto{\pgfqpoint{4.726773in}{1.470255in}}%
\pgfpathlineto{\pgfqpoint{4.734952in}{1.485501in}}%
\pgfpathlineto{\pgfqpoint{4.743126in}{1.500714in}}%
\pgfpathlineto{\pgfqpoint{4.751296in}{1.515889in}}%
\pgfpathlineto{\pgfqpoint{4.759461in}{1.531021in}}%
\pgfpathlineto{\pgfqpoint{4.744754in}{1.527967in}}%
\pgfpathlineto{\pgfqpoint{4.730059in}{1.524983in}}%
\pgfpathlineto{\pgfqpoint{4.715377in}{1.522070in}}%
\pgfpathlineto{\pgfqpoint{4.700706in}{1.519226in}}%
\pgfpathlineto{\pgfqpoint{4.692541in}{1.504386in}}%
\pgfpathlineto{\pgfqpoint{4.684371in}{1.489510in}}%
\pgfpathlineto{\pgfqpoint{4.676197in}{1.474600in}}%
\pgfpathlineto{\pgfqpoint{4.668018in}{1.459662in}}%
\pgfpathclose%
\pgfusepath{fill}%
\end{pgfscope}%
\begin{pgfscope}%
\pgfpathrectangle{\pgfqpoint{1.150000in}{0.150000in}}{\pgfqpoint{5.700000in}{5.700000in}}%
\pgfusepath{clip}%
\pgfsetbuttcap%
\pgfsetroundjoin%
\definecolor{currentfill}{rgb}{0.281446,0.084320,0.407414}%
\pgfsetfillcolor{currentfill}%
\pgfsetfillopacity{0.700000}%
\pgfsetlinewidth{0.000000pt}%
\definecolor{currentstroke}{rgb}{0.000000,0.000000,0.000000}%
\pgfsetstrokecolor{currentstroke}%
\pgfsetdash{}{0pt}%
\pgfpathmoveto{\pgfqpoint{4.178952in}{1.113492in}}%
\pgfpathlineto{\pgfqpoint{4.193437in}{1.112431in}}%
\pgfpathlineto{\pgfqpoint{4.207931in}{1.111437in}}%
\pgfpathlineto{\pgfqpoint{4.222435in}{1.110512in}}%
\pgfpathlineto{\pgfqpoint{4.236947in}{1.109656in}}%
\pgfpathlineto{\pgfqpoint{4.245246in}{1.122049in}}%
\pgfpathlineto{\pgfqpoint{4.253540in}{1.134562in}}%
\pgfpathlineto{\pgfqpoint{4.261829in}{1.147189in}}%
\pgfpathlineto{\pgfqpoint{4.270113in}{1.159924in}}%
\pgfpathlineto{\pgfqpoint{4.255607in}{1.160360in}}%
\pgfpathlineto{\pgfqpoint{4.241111in}{1.160864in}}%
\pgfpathlineto{\pgfqpoint{4.226624in}{1.161437in}}%
\pgfpathlineto{\pgfqpoint{4.212147in}{1.162079in}}%
\pgfpathlineto{\pgfqpoint{4.203855in}{1.149757in}}%
\pgfpathlineto{\pgfqpoint{4.195559in}{1.137547in}}%
\pgfpathlineto{\pgfqpoint{4.187258in}{1.125457in}}%
\pgfpathlineto{\pgfqpoint{4.178952in}{1.113492in}}%
\pgfpathclose%
\pgfusepath{fill}%
\end{pgfscope}%
\begin{pgfscope}%
\pgfpathrectangle{\pgfqpoint{1.150000in}{0.150000in}}{\pgfqpoint{5.700000in}{5.700000in}}%
\pgfusepath{clip}%
\pgfsetbuttcap%
\pgfsetroundjoin%
\definecolor{currentfill}{rgb}{0.283091,0.110553,0.431554}%
\pgfsetfillcolor{currentfill}%
\pgfsetfillopacity{0.700000}%
\pgfsetlinewidth{0.000000pt}%
\definecolor{currentstroke}{rgb}{0.000000,0.000000,0.000000}%
\pgfsetstrokecolor{currentstroke}%
\pgfsetdash{}{0pt}%
\pgfpathmoveto{\pgfqpoint{4.270113in}{1.159924in}}%
\pgfpathlineto{\pgfqpoint{4.284628in}{1.159557in}}%
\pgfpathlineto{\pgfqpoint{4.299153in}{1.159259in}}%
\pgfpathlineto{\pgfqpoint{4.313687in}{1.159029in}}%
\pgfpathlineto{\pgfqpoint{4.328231in}{1.158868in}}%
\pgfpathlineto{\pgfqpoint{4.336505in}{1.172113in}}%
\pgfpathlineto{\pgfqpoint{4.344774in}{1.185448in}}%
\pgfpathlineto{\pgfqpoint{4.353039in}{1.198869in}}%
\pgfpathlineto{\pgfqpoint{4.361299in}{1.212368in}}%
\pgfpathlineto{\pgfqpoint{4.346760in}{1.212128in}}%
\pgfpathlineto{\pgfqpoint{4.332231in}{1.211957in}}%
\pgfpathlineto{\pgfqpoint{4.317712in}{1.211854in}}%
\pgfpathlineto{\pgfqpoint{4.303203in}{1.211821in}}%
\pgfpathlineto{\pgfqpoint{4.294937in}{1.198715in}}%
\pgfpathlineto{\pgfqpoint{4.286667in}{1.185693in}}%
\pgfpathlineto{\pgfqpoint{4.278393in}{1.172761in}}%
\pgfpathlineto{\pgfqpoint{4.270113in}{1.159924in}}%
\pgfpathclose%
\pgfusepath{fill}%
\end{pgfscope}%
\begin{pgfscope}%
\pgfpathrectangle{\pgfqpoint{1.150000in}{0.150000in}}{\pgfqpoint{5.700000in}{5.700000in}}%
\pgfusepath{clip}%
\pgfsetbuttcap%
\pgfsetroundjoin%
\definecolor{currentfill}{rgb}{0.282884,0.135920,0.453427}%
\pgfsetfillcolor{currentfill}%
\pgfsetfillopacity{0.700000}%
\pgfsetlinewidth{0.000000pt}%
\definecolor{currentstroke}{rgb}{0.000000,0.000000,0.000000}%
\pgfsetstrokecolor{currentstroke}%
\pgfsetdash{}{0pt}%
\pgfpathmoveto{\pgfqpoint{4.361299in}{1.212368in}}%
\pgfpathlineto{\pgfqpoint{4.375848in}{1.212676in}}%
\pgfpathlineto{\pgfqpoint{4.390407in}{1.213053in}}%
\pgfpathlineto{\pgfqpoint{4.404976in}{1.213500in}}%
\pgfpathlineto{\pgfqpoint{4.419555in}{1.214014in}}%
\pgfpathlineto{\pgfqpoint{4.427807in}{1.227975in}}%
\pgfpathlineto{\pgfqpoint{4.436055in}{1.241998in}}%
\pgfpathlineto{\pgfqpoint{4.444298in}{1.256077in}}%
\pgfpathlineto{\pgfqpoint{4.452537in}{1.270206in}}%
\pgfpathlineto{\pgfqpoint{4.437961in}{1.269310in}}%
\pgfpathlineto{\pgfqpoint{4.423396in}{1.268483in}}%
\pgfpathlineto{\pgfqpoint{4.408840in}{1.267725in}}%
\pgfpathlineto{\pgfqpoint{4.394296in}{1.267035in}}%
\pgfpathlineto{\pgfqpoint{4.386053in}{1.253279in}}%
\pgfpathlineto{\pgfqpoint{4.377806in}{1.239578in}}%
\pgfpathlineto{\pgfqpoint{4.369555in}{1.225939in}}%
\pgfpathlineto{\pgfqpoint{4.361299in}{1.212368in}}%
\pgfpathclose%
\pgfusepath{fill}%
\end{pgfscope}%
\begin{pgfscope}%
\pgfpathrectangle{\pgfqpoint{1.150000in}{0.150000in}}{\pgfqpoint{5.700000in}{5.700000in}}%
\pgfusepath{clip}%
\pgfsetbuttcap%
\pgfsetroundjoin%
\definecolor{currentfill}{rgb}{0.210503,0.363727,0.552206}%
\pgfsetfillcolor{currentfill}%
\pgfsetfillopacity{0.700000}%
\pgfsetlinewidth{0.000000pt}%
\definecolor{currentstroke}{rgb}{0.000000,0.000000,0.000000}%
\pgfsetstrokecolor{currentstroke}%
\pgfsetdash{}{0pt}%
\pgfpathmoveto{\pgfqpoint{4.975152in}{1.741050in}}%
\pgfpathlineto{\pgfqpoint{4.989974in}{1.745516in}}%
\pgfpathlineto{\pgfqpoint{5.004810in}{1.750053in}}%
\pgfpathlineto{\pgfqpoint{5.019659in}{1.754661in}}%
\pgfpathlineto{\pgfqpoint{5.027756in}{1.769512in}}%
\pgfpathlineto{\pgfqpoint{5.035846in}{1.784256in}}%
\pgfpathlineto{\pgfqpoint{5.043930in}{1.798888in}}%
\pgfpathlineto{\pgfqpoint{5.052008in}{1.813407in}}%
\pgfpathlineto{\pgfqpoint{5.037158in}{1.808582in}}%
\pgfpathlineto{\pgfqpoint{5.022321in}{1.803827in}}%
\pgfpathlineto{\pgfqpoint{5.007498in}{1.799144in}}%
\pgfpathlineto{\pgfqpoint{4.999421in}{1.784782in}}%
\pgfpathlineto{\pgfqpoint{4.991337in}{1.770310in}}%
\pgfpathlineto{\pgfqpoint{4.983248in}{1.755732in}}%
\pgfpathlineto{\pgfqpoint{4.975152in}{1.741050in}}%
\pgfpathclose%
\pgfusepath{fill}%
\end{pgfscope}%
\begin{pgfscope}%
\pgfpathrectangle{\pgfqpoint{1.150000in}{0.150000in}}{\pgfqpoint{5.700000in}{5.700000in}}%
\pgfusepath{clip}%
\pgfsetbuttcap%
\pgfsetroundjoin%
\definecolor{currentfill}{rgb}{0.280255,0.165693,0.476498}%
\pgfsetfillcolor{currentfill}%
\pgfsetfillopacity{0.700000}%
\pgfsetlinewidth{0.000000pt}%
\definecolor{currentstroke}{rgb}{0.000000,0.000000,0.000000}%
\pgfsetstrokecolor{currentstroke}%
\pgfsetdash{}{0pt}%
\pgfpathmoveto{\pgfqpoint{4.452537in}{1.270206in}}%
\pgfpathlineto{\pgfqpoint{4.467123in}{1.271172in}}%
\pgfpathlineto{\pgfqpoint{4.481720in}{1.272206in}}%
\pgfpathlineto{\pgfqpoint{4.496328in}{1.273309in}}%
\pgfpathlineto{\pgfqpoint{4.510946in}{1.274481in}}%
\pgfpathlineto{\pgfqpoint{4.519178in}{1.289025in}}%
\pgfpathlineto{\pgfqpoint{4.527406in}{1.303605in}}%
\pgfpathlineto{\pgfqpoint{4.535630in}{1.318213in}}%
\pgfpathlineto{\pgfqpoint{4.543850in}{1.332846in}}%
\pgfpathlineto{\pgfqpoint{4.529234in}{1.331312in}}%
\pgfpathlineto{\pgfqpoint{4.514628in}{1.329848in}}%
\pgfpathlineto{\pgfqpoint{4.500034in}{1.328452in}}%
\pgfpathlineto{\pgfqpoint{4.485450in}{1.327126in}}%
\pgfpathlineto{\pgfqpoint{4.477228in}{1.312847in}}%
\pgfpathlineto{\pgfqpoint{4.469002in}{1.298597in}}%
\pgfpathlineto{\pgfqpoint{4.460772in}{1.284382in}}%
\pgfpathlineto{\pgfqpoint{4.452537in}{1.270206in}}%
\pgfpathclose%
\pgfusepath{fill}%
\end{pgfscope}%
\begin{pgfscope}%
\pgfpathrectangle{\pgfqpoint{1.150000in}{0.150000in}}{\pgfqpoint{5.700000in}{5.700000in}}%
\pgfusepath{clip}%
\pgfsetbuttcap%
\pgfsetroundjoin%
\definecolor{currentfill}{rgb}{0.246811,0.283237,0.535941}%
\pgfsetfillcolor{currentfill}%
\pgfsetfillopacity{0.700000}%
\pgfsetlinewidth{0.000000pt}%
\definecolor{currentstroke}{rgb}{0.000000,0.000000,0.000000}%
\pgfsetstrokecolor{currentstroke}%
\pgfsetdash{}{0pt}%
\pgfpathmoveto{\pgfqpoint{4.759461in}{1.531021in}}%
\pgfpathlineto{\pgfqpoint{4.774180in}{1.534145in}}%
\pgfpathlineto{\pgfqpoint{4.788911in}{1.537338in}}%
\pgfpathlineto{\pgfqpoint{4.803654in}{1.540602in}}%
\pgfpathlineto{\pgfqpoint{4.818410in}{1.543936in}}%
\pgfpathlineto{\pgfqpoint{4.826571in}{1.559310in}}%
\pgfpathlineto{\pgfqpoint{4.834726in}{1.574628in}}%
\pgfpathlineto{\pgfqpoint{4.842877in}{1.589886in}}%
\pgfpathlineto{\pgfqpoint{4.851023in}{1.605081in}}%
\pgfpathlineto{\pgfqpoint{4.836266in}{1.601466in}}%
\pgfpathlineto{\pgfqpoint{4.821522in}{1.597922in}}%
\pgfpathlineto{\pgfqpoint{4.806790in}{1.594447in}}%
\pgfpathlineto{\pgfqpoint{4.792071in}{1.591044in}}%
\pgfpathlineto{\pgfqpoint{4.783926in}{1.576122in}}%
\pgfpathlineto{\pgfqpoint{4.775776in}{1.561141in}}%
\pgfpathlineto{\pgfqpoint{4.767621in}{1.546107in}}%
\pgfpathlineto{\pgfqpoint{4.759461in}{1.531021in}}%
\pgfpathclose%
\pgfusepath{fill}%
\end{pgfscope}%
\begin{pgfscope}%
\pgfpathrectangle{\pgfqpoint{1.150000in}{0.150000in}}{\pgfqpoint{5.700000in}{5.700000in}}%
\pgfusepath{clip}%
\pgfsetbuttcap%
\pgfsetroundjoin%
\definecolor{currentfill}{rgb}{0.274128,0.199721,0.498911}%
\pgfsetfillcolor{currentfill}%
\pgfsetfillopacity{0.700000}%
\pgfsetlinewidth{0.000000pt}%
\definecolor{currentstroke}{rgb}{0.000000,0.000000,0.000000}%
\pgfsetstrokecolor{currentstroke}%
\pgfsetdash{}{0pt}%
\pgfpathmoveto{\pgfqpoint{4.543850in}{1.332846in}}%
\pgfpathlineto{\pgfqpoint{4.558477in}{1.334449in}}%
\pgfpathlineto{\pgfqpoint{4.573116in}{1.336121in}}%
\pgfpathlineto{\pgfqpoint{4.587765in}{1.337863in}}%
\pgfpathlineto{\pgfqpoint{4.602425in}{1.339673in}}%
\pgfpathlineto{\pgfqpoint{4.610640in}{1.354675in}}%
\pgfpathlineto{\pgfqpoint{4.618850in}{1.369685in}}%
\pgfpathlineto{\pgfqpoint{4.627056in}{1.384700in}}%
\pgfpathlineto{\pgfqpoint{4.635257in}{1.399713in}}%
\pgfpathlineto{\pgfqpoint{4.620597in}{1.397561in}}%
\pgfpathlineto{\pgfqpoint{4.605948in}{1.395478in}}%
\pgfpathlineto{\pgfqpoint{4.591311in}{1.393464in}}%
\pgfpathlineto{\pgfqpoint{4.576685in}{1.391520in}}%
\pgfpathlineto{\pgfqpoint{4.568483in}{1.376840in}}%
\pgfpathlineto{\pgfqpoint{4.560276in}{1.362165in}}%
\pgfpathlineto{\pgfqpoint{4.552065in}{1.347498in}}%
\pgfpathlineto{\pgfqpoint{4.543850in}{1.332846in}}%
\pgfpathclose%
\pgfusepath{fill}%
\end{pgfscope}%
\begin{pgfscope}%
\pgfpathrectangle{\pgfqpoint{1.150000in}{0.150000in}}{\pgfqpoint{5.700000in}{5.700000in}}%
\pgfusepath{clip}%
\pgfsetbuttcap%
\pgfsetroundjoin%
\definecolor{currentfill}{rgb}{0.231674,0.318106,0.544834}%
\pgfsetfillcolor{currentfill}%
\pgfsetfillopacity{0.700000}%
\pgfsetlinewidth{0.000000pt}%
\definecolor{currentstroke}{rgb}{0.000000,0.000000,0.000000}%
\pgfsetstrokecolor{currentstroke}%
\pgfsetdash{}{0pt}%
\pgfpathmoveto{\pgfqpoint{4.851023in}{1.605081in}}%
\pgfpathlineto{\pgfqpoint{4.865792in}{1.608766in}}%
\pgfpathlineto{\pgfqpoint{4.880573in}{1.612521in}}%
\pgfpathlineto{\pgfqpoint{4.895368in}{1.616347in}}%
\pgfpathlineto{\pgfqpoint{4.910175in}{1.620244in}}%
\pgfpathlineto{\pgfqpoint{4.918317in}{1.635638in}}%
\pgfpathlineto{\pgfqpoint{4.926453in}{1.650956in}}%
\pgfpathlineto{\pgfqpoint{4.934584in}{1.666193in}}%
\pgfpathlineto{\pgfqpoint{4.942709in}{1.681346in}}%
\pgfpathlineto{\pgfqpoint{4.927900in}{1.677190in}}%
\pgfpathlineto{\pgfqpoint{4.913104in}{1.673104in}}%
\pgfpathlineto{\pgfqpoint{4.898322in}{1.669088in}}%
\pgfpathlineto{\pgfqpoint{4.883552in}{1.665144in}}%
\pgfpathlineto{\pgfqpoint{4.875428in}{1.650242in}}%
\pgfpathlineto{\pgfqpoint{4.867298in}{1.635262in}}%
\pgfpathlineto{\pgfqpoint{4.859163in}{1.620207in}}%
\pgfpathlineto{\pgfqpoint{4.851023in}{1.605081in}}%
\pgfpathclose%
\pgfusepath{fill}%
\end{pgfscope}%
\begin{pgfscope}%
\pgfpathrectangle{\pgfqpoint{1.150000in}{0.150000in}}{\pgfqpoint{5.700000in}{5.700000in}}%
\pgfusepath{clip}%
\pgfsetbuttcap%
\pgfsetroundjoin%
\definecolor{currentfill}{rgb}{0.265145,0.232956,0.516599}%
\pgfsetfillcolor{currentfill}%
\pgfsetfillopacity{0.700000}%
\pgfsetlinewidth{0.000000pt}%
\definecolor{currentstroke}{rgb}{0.000000,0.000000,0.000000}%
\pgfsetstrokecolor{currentstroke}%
\pgfsetdash{}{0pt}%
\pgfpathmoveto{\pgfqpoint{4.635257in}{1.399713in}}%
\pgfpathlineto{\pgfqpoint{4.649928in}{1.401935in}}%
\pgfpathlineto{\pgfqpoint{4.664611in}{1.404227in}}%
\pgfpathlineto{\pgfqpoint{4.679306in}{1.406588in}}%
\pgfpathlineto{\pgfqpoint{4.694012in}{1.409018in}}%
\pgfpathlineto{\pgfqpoint{4.702209in}{1.424356in}}%
\pgfpathlineto{\pgfqpoint{4.710401in}{1.439677in}}%
\pgfpathlineto{\pgfqpoint{4.718589in}{1.454978in}}%
\pgfpathlineto{\pgfqpoint{4.726773in}{1.470255in}}%
\pgfpathlineto{\pgfqpoint{4.712067in}{1.467502in}}%
\pgfpathlineto{\pgfqpoint{4.697372in}{1.464819in}}%
\pgfpathlineto{\pgfqpoint{4.682689in}{1.462206in}}%
\pgfpathlineto{\pgfqpoint{4.668018in}{1.459662in}}%
\pgfpathlineto{\pgfqpoint{4.659835in}{1.444700in}}%
\pgfpathlineto{\pgfqpoint{4.651646in}{1.429718in}}%
\pgfpathlineto{\pgfqpoint{4.643454in}{1.414721in}}%
\pgfpathlineto{\pgfqpoint{4.635257in}{1.399713in}}%
\pgfpathclose%
\pgfusepath{fill}%
\end{pgfscope}%
\begin{pgfscope}%
\pgfpathrectangle{\pgfqpoint{1.150000in}{0.150000in}}{\pgfqpoint{5.700000in}{5.700000in}}%
\pgfusepath{clip}%
\pgfsetbuttcap%
\pgfsetroundjoin%
\definecolor{currentfill}{rgb}{0.282327,0.094955,0.417331}%
\pgfsetfillcolor{currentfill}%
\pgfsetfillopacity{0.700000}%
\pgfsetlinewidth{0.000000pt}%
\definecolor{currentstroke}{rgb}{0.000000,0.000000,0.000000}%
\pgfsetstrokecolor{currentstroke}%
\pgfsetdash{}{0pt}%
\pgfpathmoveto{\pgfqpoint{4.236947in}{1.109656in}}%
\pgfpathlineto{\pgfqpoint{4.251469in}{1.108868in}}%
\pgfpathlineto{\pgfqpoint{4.266000in}{1.108149in}}%
\pgfpathlineto{\pgfqpoint{4.280540in}{1.107497in}}%
\pgfpathlineto{\pgfqpoint{4.295090in}{1.106915in}}%
\pgfpathlineto{\pgfqpoint{4.303382in}{1.119736in}}%
\pgfpathlineto{\pgfqpoint{4.311670in}{1.132673in}}%
\pgfpathlineto{\pgfqpoint{4.319952in}{1.145719in}}%
\pgfpathlineto{\pgfqpoint{4.328231in}{1.158868in}}%
\pgfpathlineto{\pgfqpoint{4.313687in}{1.159029in}}%
\pgfpathlineto{\pgfqpoint{4.299153in}{1.159259in}}%
\pgfpathlineto{\pgfqpoint{4.284628in}{1.159557in}}%
\pgfpathlineto{\pgfqpoint{4.270113in}{1.159924in}}%
\pgfpathlineto{\pgfqpoint{4.261829in}{1.147189in}}%
\pgfpathlineto{\pgfqpoint{4.253540in}{1.134562in}}%
\pgfpathlineto{\pgfqpoint{4.245246in}{1.122049in}}%
\pgfpathlineto{\pgfqpoint{4.236947in}{1.109656in}}%
\pgfpathclose%
\pgfusepath{fill}%
\end{pgfscope}%
\begin{pgfscope}%
\pgfpathrectangle{\pgfqpoint{1.150000in}{0.150000in}}{\pgfqpoint{5.700000in}{5.700000in}}%
\pgfusepath{clip}%
\pgfsetbuttcap%
\pgfsetroundjoin%
\definecolor{currentfill}{rgb}{0.216210,0.351535,0.550627}%
\pgfsetfillcolor{currentfill}%
\pgfsetfillopacity{0.700000}%
\pgfsetlinewidth{0.000000pt}%
\definecolor{currentstroke}{rgb}{0.000000,0.000000,0.000000}%
\pgfsetstrokecolor{currentstroke}%
\pgfsetdash{}{0pt}%
\pgfpathmoveto{\pgfqpoint{4.942709in}{1.681346in}}%
\pgfpathlineto{\pgfqpoint{4.957530in}{1.685574in}}%
\pgfpathlineto{\pgfqpoint{4.972365in}{1.689872in}}%
\pgfpathlineto{\pgfqpoint{4.987213in}{1.694241in}}%
\pgfpathlineto{\pgfqpoint{4.995333in}{1.709492in}}%
\pgfpathlineto{\pgfqpoint{5.003448in}{1.724648in}}%
\pgfpathlineto{\pgfqpoint{5.011557in}{1.739705in}}%
\pgfpathlineto{\pgfqpoint{5.019659in}{1.754661in}}%
\pgfpathlineto{\pgfqpoint{5.004810in}{1.750053in}}%
\pgfpathlineto{\pgfqpoint{4.989974in}{1.745516in}}%
\pgfpathlineto{\pgfqpoint{4.975152in}{1.741050in}}%
\pgfpathlineto{\pgfqpoint{4.967050in}{1.726267in}}%
\pgfpathlineto{\pgfqpoint{4.958942in}{1.711386in}}%
\pgfpathlineto{\pgfqpoint{4.950828in}{1.696412in}}%
\pgfpathlineto{\pgfqpoint{4.942709in}{1.681346in}}%
\pgfpathclose%
\pgfusepath{fill}%
\end{pgfscope}%
\begin{pgfscope}%
\pgfpathrectangle{\pgfqpoint{1.150000in}{0.150000in}}{\pgfqpoint{5.700000in}{5.700000in}}%
\pgfusepath{clip}%
\pgfsetbuttcap%
\pgfsetroundjoin%
\definecolor{currentfill}{rgb}{0.283229,0.120777,0.440584}%
\pgfsetfillcolor{currentfill}%
\pgfsetfillopacity{0.700000}%
\pgfsetlinewidth{0.000000pt}%
\definecolor{currentstroke}{rgb}{0.000000,0.000000,0.000000}%
\pgfsetstrokecolor{currentstroke}%
\pgfsetdash{}{0pt}%
\pgfpathmoveto{\pgfqpoint{4.328231in}{1.158868in}}%
\pgfpathlineto{\pgfqpoint{4.342784in}{1.158775in}}%
\pgfpathlineto{\pgfqpoint{4.357348in}{1.158751in}}%
\pgfpathlineto{\pgfqpoint{4.371921in}{1.158795in}}%
\pgfpathlineto{\pgfqpoint{4.386504in}{1.158908in}}%
\pgfpathlineto{\pgfqpoint{4.394773in}{1.172562in}}%
\pgfpathlineto{\pgfqpoint{4.403038in}{1.186302in}}%
\pgfpathlineto{\pgfqpoint{4.411299in}{1.200122in}}%
\pgfpathlineto{\pgfqpoint{4.419555in}{1.214014in}}%
\pgfpathlineto{\pgfqpoint{4.404976in}{1.213500in}}%
\pgfpathlineto{\pgfqpoint{4.390407in}{1.213053in}}%
\pgfpathlineto{\pgfqpoint{4.375848in}{1.212676in}}%
\pgfpathlineto{\pgfqpoint{4.361299in}{1.212368in}}%
\pgfpathlineto{\pgfqpoint{4.353039in}{1.198869in}}%
\pgfpathlineto{\pgfqpoint{4.344774in}{1.185448in}}%
\pgfpathlineto{\pgfqpoint{4.336505in}{1.172113in}}%
\pgfpathlineto{\pgfqpoint{4.328231in}{1.158868in}}%
\pgfpathclose%
\pgfusepath{fill}%
\end{pgfscope}%
\begin{pgfscope}%
\pgfpathrectangle{\pgfqpoint{1.150000in}{0.150000in}}{\pgfqpoint{5.700000in}{5.700000in}}%
\pgfusepath{clip}%
\pgfsetbuttcap%
\pgfsetroundjoin%
\definecolor{currentfill}{rgb}{0.281887,0.150881,0.465405}%
\pgfsetfillcolor{currentfill}%
\pgfsetfillopacity{0.700000}%
\pgfsetlinewidth{0.000000pt}%
\definecolor{currentstroke}{rgb}{0.000000,0.000000,0.000000}%
\pgfsetstrokecolor{currentstroke}%
\pgfsetdash{}{0pt}%
\pgfpathmoveto{\pgfqpoint{4.419555in}{1.214014in}}%
\pgfpathlineto{\pgfqpoint{4.434144in}{1.214598in}}%
\pgfpathlineto{\pgfqpoint{4.448744in}{1.215250in}}%
\pgfpathlineto{\pgfqpoint{4.463354in}{1.215971in}}%
\pgfpathlineto{\pgfqpoint{4.477975in}{1.216760in}}%
\pgfpathlineto{\pgfqpoint{4.486224in}{1.231111in}}%
\pgfpathlineto{\pgfqpoint{4.494469in}{1.245518in}}%
\pgfpathlineto{\pgfqpoint{4.502709in}{1.259977in}}%
\pgfpathlineto{\pgfqpoint{4.510946in}{1.274481in}}%
\pgfpathlineto{\pgfqpoint{4.496328in}{1.273309in}}%
\pgfpathlineto{\pgfqpoint{4.481720in}{1.272206in}}%
\pgfpathlineto{\pgfqpoint{4.467123in}{1.271172in}}%
\pgfpathlineto{\pgfqpoint{4.452537in}{1.270206in}}%
\pgfpathlineto{\pgfqpoint{4.444298in}{1.256077in}}%
\pgfpathlineto{\pgfqpoint{4.436055in}{1.241998in}}%
\pgfpathlineto{\pgfqpoint{4.427807in}{1.227975in}}%
\pgfpathlineto{\pgfqpoint{4.419555in}{1.214014in}}%
\pgfpathclose%
\pgfusepath{fill}%
\end{pgfscope}%
\begin{pgfscope}%
\pgfpathrectangle{\pgfqpoint{1.150000in}{0.150000in}}{\pgfqpoint{5.700000in}{5.700000in}}%
\pgfusepath{clip}%
\pgfsetbuttcap%
\pgfsetroundjoin%
\definecolor{currentfill}{rgb}{0.252194,0.269783,0.531579}%
\pgfsetfillcolor{currentfill}%
\pgfsetfillopacity{0.700000}%
\pgfsetlinewidth{0.000000pt}%
\definecolor{currentstroke}{rgb}{0.000000,0.000000,0.000000}%
\pgfsetstrokecolor{currentstroke}%
\pgfsetdash{}{0pt}%
\pgfpathmoveto{\pgfqpoint{4.726773in}{1.470255in}}%
\pgfpathlineto{\pgfqpoint{4.741491in}{1.473077in}}%
\pgfpathlineto{\pgfqpoint{4.756222in}{1.475969in}}%
\pgfpathlineto{\pgfqpoint{4.770964in}{1.478931in}}%
\pgfpathlineto{\pgfqpoint{4.785719in}{1.481963in}}%
\pgfpathlineto{\pgfqpoint{4.793899in}{1.497519in}}%
\pgfpathlineto{\pgfqpoint{4.802074in}{1.513037in}}%
\pgfpathlineto{\pgfqpoint{4.810244in}{1.528510in}}%
\pgfpathlineto{\pgfqpoint{4.818410in}{1.543936in}}%
\pgfpathlineto{\pgfqpoint{4.803654in}{1.540602in}}%
\pgfpathlineto{\pgfqpoint{4.788911in}{1.537338in}}%
\pgfpathlineto{\pgfqpoint{4.774180in}{1.534145in}}%
\pgfpathlineto{\pgfqpoint{4.759461in}{1.531021in}}%
\pgfpathlineto{\pgfqpoint{4.751296in}{1.515889in}}%
\pgfpathlineto{\pgfqpoint{4.743126in}{1.500714in}}%
\pgfpathlineto{\pgfqpoint{4.734952in}{1.485501in}}%
\pgfpathlineto{\pgfqpoint{4.726773in}{1.470255in}}%
\pgfpathclose%
\pgfusepath{fill}%
\end{pgfscope}%
\begin{pgfscope}%
\pgfpathrectangle{\pgfqpoint{1.150000in}{0.150000in}}{\pgfqpoint{5.700000in}{5.700000in}}%
\pgfusepath{clip}%
\pgfsetbuttcap%
\pgfsetroundjoin%
\definecolor{currentfill}{rgb}{0.278012,0.180367,0.486697}%
\pgfsetfillcolor{currentfill}%
\pgfsetfillopacity{0.700000}%
\pgfsetlinewidth{0.000000pt}%
\definecolor{currentstroke}{rgb}{0.000000,0.000000,0.000000}%
\pgfsetstrokecolor{currentstroke}%
\pgfsetdash{}{0pt}%
\pgfpathmoveto{\pgfqpoint{4.510946in}{1.274481in}}%
\pgfpathlineto{\pgfqpoint{4.525575in}{1.275722in}}%
\pgfpathlineto{\pgfqpoint{4.540214in}{1.277032in}}%
\pgfpathlineto{\pgfqpoint{4.554865in}{1.278410in}}%
\pgfpathlineto{\pgfqpoint{4.569526in}{1.279858in}}%
\pgfpathlineto{\pgfqpoint{4.577757in}{1.294773in}}%
\pgfpathlineto{\pgfqpoint{4.585984in}{1.309717in}}%
\pgfpathlineto{\pgfqpoint{4.594207in}{1.324686in}}%
\pgfpathlineto{\pgfqpoint{4.602425in}{1.339673in}}%
\pgfpathlineto{\pgfqpoint{4.587765in}{1.337863in}}%
\pgfpathlineto{\pgfqpoint{4.573116in}{1.336121in}}%
\pgfpathlineto{\pgfqpoint{4.558477in}{1.334449in}}%
\pgfpathlineto{\pgfqpoint{4.543850in}{1.332846in}}%
\pgfpathlineto{\pgfqpoint{4.535630in}{1.318213in}}%
\pgfpathlineto{\pgfqpoint{4.527406in}{1.303605in}}%
\pgfpathlineto{\pgfqpoint{4.519178in}{1.289025in}}%
\pgfpathlineto{\pgfqpoint{4.510946in}{1.274481in}}%
\pgfpathclose%
\pgfusepath{fill}%
\end{pgfscope}%
\begin{pgfscope}%
\pgfpathrectangle{\pgfqpoint{1.150000in}{0.150000in}}{\pgfqpoint{5.700000in}{5.700000in}}%
\pgfusepath{clip}%
\pgfsetbuttcap%
\pgfsetroundjoin%
\definecolor{currentfill}{rgb}{0.239346,0.300855,0.540844}%
\pgfsetfillcolor{currentfill}%
\pgfsetfillopacity{0.700000}%
\pgfsetlinewidth{0.000000pt}%
\definecolor{currentstroke}{rgb}{0.000000,0.000000,0.000000}%
\pgfsetstrokecolor{currentstroke}%
\pgfsetdash{}{0pt}%
\pgfpathmoveto{\pgfqpoint{4.818410in}{1.543936in}}%
\pgfpathlineto{\pgfqpoint{4.833178in}{1.547340in}}%
\pgfpathlineto{\pgfqpoint{4.847959in}{1.550814in}}%
\pgfpathlineto{\pgfqpoint{4.862752in}{1.554359in}}%
\pgfpathlineto{\pgfqpoint{4.877557in}{1.557973in}}%
\pgfpathlineto{\pgfqpoint{4.885720in}{1.573637in}}%
\pgfpathlineto{\pgfqpoint{4.893877in}{1.589239in}}%
\pgfpathlineto{\pgfqpoint{4.902029in}{1.604776in}}%
\pgfpathlineto{\pgfqpoint{4.910175in}{1.620244in}}%
\pgfpathlineto{\pgfqpoint{4.895368in}{1.616347in}}%
\pgfpathlineto{\pgfqpoint{4.880573in}{1.612521in}}%
\pgfpathlineto{\pgfqpoint{4.865792in}{1.608766in}}%
\pgfpathlineto{\pgfqpoint{4.851023in}{1.605081in}}%
\pgfpathlineto{\pgfqpoint{4.842877in}{1.589886in}}%
\pgfpathlineto{\pgfqpoint{4.834726in}{1.574628in}}%
\pgfpathlineto{\pgfqpoint{4.826571in}{1.559310in}}%
\pgfpathlineto{\pgfqpoint{4.818410in}{1.543936in}}%
\pgfpathclose%
\pgfusepath{fill}%
\end{pgfscope}%
\begin{pgfscope}%
\pgfpathrectangle{\pgfqpoint{1.150000in}{0.150000in}}{\pgfqpoint{5.700000in}{5.700000in}}%
\pgfusepath{clip}%
\pgfsetbuttcap%
\pgfsetroundjoin%
\definecolor{currentfill}{rgb}{0.270595,0.214069,0.507052}%
\pgfsetfillcolor{currentfill}%
\pgfsetfillopacity{0.700000}%
\pgfsetlinewidth{0.000000pt}%
\definecolor{currentstroke}{rgb}{0.000000,0.000000,0.000000}%
\pgfsetstrokecolor{currentstroke}%
\pgfsetdash{}{0pt}%
\pgfpathmoveto{\pgfqpoint{4.602425in}{1.339673in}}%
\pgfpathlineto{\pgfqpoint{4.617097in}{1.341553in}}%
\pgfpathlineto{\pgfqpoint{4.631780in}{1.343502in}}%
\pgfpathlineto{\pgfqpoint{4.646475in}{1.345520in}}%
\pgfpathlineto{\pgfqpoint{4.661181in}{1.347607in}}%
\pgfpathlineto{\pgfqpoint{4.669395in}{1.362959in}}%
\pgfpathlineto{\pgfqpoint{4.677605in}{1.378315in}}%
\pgfpathlineto{\pgfqpoint{4.685810in}{1.393670in}}%
\pgfpathlineto{\pgfqpoint{4.694012in}{1.409018in}}%
\pgfpathlineto{\pgfqpoint{4.679306in}{1.406588in}}%
\pgfpathlineto{\pgfqpoint{4.664611in}{1.404227in}}%
\pgfpathlineto{\pgfqpoint{4.649928in}{1.401935in}}%
\pgfpathlineto{\pgfqpoint{4.635257in}{1.399713in}}%
\pgfpathlineto{\pgfqpoint{4.627056in}{1.384700in}}%
\pgfpathlineto{\pgfqpoint{4.618850in}{1.369685in}}%
\pgfpathlineto{\pgfqpoint{4.610640in}{1.354675in}}%
\pgfpathlineto{\pgfqpoint{4.602425in}{1.339673in}}%
\pgfpathclose%
\pgfusepath{fill}%
\end{pgfscope}%
\begin{pgfscope}%
\pgfpathrectangle{\pgfqpoint{1.150000in}{0.150000in}}{\pgfqpoint{5.700000in}{5.700000in}}%
\pgfusepath{clip}%
\pgfsetbuttcap%
\pgfsetroundjoin%
\definecolor{currentfill}{rgb}{0.223925,0.334994,0.548053}%
\pgfsetfillcolor{currentfill}%
\pgfsetfillopacity{0.700000}%
\pgfsetlinewidth{0.000000pt}%
\definecolor{currentstroke}{rgb}{0.000000,0.000000,0.000000}%
\pgfsetstrokecolor{currentstroke}%
\pgfsetdash{}{0pt}%
\pgfpathmoveto{\pgfqpoint{4.910175in}{1.620244in}}%
\pgfpathlineto{\pgfqpoint{4.924996in}{1.624210in}}%
\pgfpathlineto{\pgfqpoint{4.939829in}{1.628248in}}%
\pgfpathlineto{\pgfqpoint{4.954675in}{1.632356in}}%
\pgfpathlineto{\pgfqpoint{4.962818in}{1.647952in}}%
\pgfpathlineto{\pgfqpoint{4.970955in}{1.663467in}}%
\pgfpathlineto{\pgfqpoint{4.979087in}{1.678898in}}%
\pgfpathlineto{\pgfqpoint{4.987213in}{1.694241in}}%
\pgfpathlineto{\pgfqpoint{4.972365in}{1.689872in}}%
\pgfpathlineto{\pgfqpoint{4.957530in}{1.685574in}}%
\pgfpathlineto{\pgfqpoint{4.942709in}{1.681346in}}%
\pgfpathlineto{\pgfqpoint{4.934584in}{1.666193in}}%
\pgfpathlineto{\pgfqpoint{4.926453in}{1.650956in}}%
\pgfpathlineto{\pgfqpoint{4.918317in}{1.635638in}}%
\pgfpathlineto{\pgfqpoint{4.910175in}{1.620244in}}%
\pgfpathclose%
\pgfusepath{fill}%
\end{pgfscope}%
\begin{pgfscope}%
\pgfpathrectangle{\pgfqpoint{1.150000in}{0.150000in}}{\pgfqpoint{5.700000in}{5.700000in}}%
\pgfusepath{clip}%
\pgfsetbuttcap%
\pgfsetroundjoin%
\definecolor{currentfill}{rgb}{0.283091,0.110553,0.431554}%
\pgfsetfillcolor{currentfill}%
\pgfsetfillopacity{0.700000}%
\pgfsetlinewidth{0.000000pt}%
\definecolor{currentstroke}{rgb}{0.000000,0.000000,0.000000}%
\pgfsetstrokecolor{currentstroke}%
\pgfsetdash{}{0pt}%
\pgfpathmoveto{\pgfqpoint{4.295090in}{1.106915in}}%
\pgfpathlineto{\pgfqpoint{4.309649in}{1.106400in}}%
\pgfpathlineto{\pgfqpoint{4.324217in}{1.105954in}}%
\pgfpathlineto{\pgfqpoint{4.338796in}{1.105575in}}%
\pgfpathlineto{\pgfqpoint{4.353383in}{1.105266in}}%
\pgfpathlineto{\pgfqpoint{4.361670in}{1.118517in}}%
\pgfpathlineto{\pgfqpoint{4.369952in}{1.131879in}}%
\pgfpathlineto{\pgfqpoint{4.378230in}{1.145344in}}%
\pgfpathlineto{\pgfqpoint{4.386504in}{1.158908in}}%
\pgfpathlineto{\pgfqpoint{4.371921in}{1.158795in}}%
\pgfpathlineto{\pgfqpoint{4.357348in}{1.158751in}}%
\pgfpathlineto{\pgfqpoint{4.342784in}{1.158775in}}%
\pgfpathlineto{\pgfqpoint{4.328231in}{1.158868in}}%
\pgfpathlineto{\pgfqpoint{4.319952in}{1.145719in}}%
\pgfpathlineto{\pgfqpoint{4.311670in}{1.132673in}}%
\pgfpathlineto{\pgfqpoint{4.303382in}{1.119736in}}%
\pgfpathlineto{\pgfqpoint{4.295090in}{1.106915in}}%
\pgfpathclose%
\pgfusepath{fill}%
\end{pgfscope}%
\begin{pgfscope}%
\pgfpathrectangle{\pgfqpoint{1.150000in}{0.150000in}}{\pgfqpoint{5.700000in}{5.700000in}}%
\pgfusepath{clip}%
\pgfsetbuttcap%
\pgfsetroundjoin%
\definecolor{currentfill}{rgb}{0.258965,0.251537,0.524736}%
\pgfsetfillcolor{currentfill}%
\pgfsetfillopacity{0.700000}%
\pgfsetlinewidth{0.000000pt}%
\definecolor{currentstroke}{rgb}{0.000000,0.000000,0.000000}%
\pgfsetstrokecolor{currentstroke}%
\pgfsetdash{}{0pt}%
\pgfpathmoveto{\pgfqpoint{4.694012in}{1.409018in}}%
\pgfpathlineto{\pgfqpoint{4.708730in}{1.411518in}}%
\pgfpathlineto{\pgfqpoint{4.723459in}{1.414088in}}%
\pgfpathlineto{\pgfqpoint{4.738201in}{1.416727in}}%
\pgfpathlineto{\pgfqpoint{4.752954in}{1.419435in}}%
\pgfpathlineto{\pgfqpoint{4.761152in}{1.435103in}}%
\pgfpathlineto{\pgfqpoint{4.769346in}{1.450750in}}%
\pgfpathlineto{\pgfqpoint{4.777535in}{1.466372in}}%
\pgfpathlineto{\pgfqpoint{4.785719in}{1.481963in}}%
\pgfpathlineto{\pgfqpoint{4.770964in}{1.478931in}}%
\pgfpathlineto{\pgfqpoint{4.756222in}{1.475969in}}%
\pgfpathlineto{\pgfqpoint{4.741491in}{1.473077in}}%
\pgfpathlineto{\pgfqpoint{4.726773in}{1.470255in}}%
\pgfpathlineto{\pgfqpoint{4.718589in}{1.454978in}}%
\pgfpathlineto{\pgfqpoint{4.710401in}{1.439677in}}%
\pgfpathlineto{\pgfqpoint{4.702209in}{1.424356in}}%
\pgfpathlineto{\pgfqpoint{4.694012in}{1.409018in}}%
\pgfpathclose%
\pgfusepath{fill}%
\end{pgfscope}%
\begin{pgfscope}%
\pgfpathrectangle{\pgfqpoint{1.150000in}{0.150000in}}{\pgfqpoint{5.700000in}{5.700000in}}%
\pgfusepath{clip}%
\pgfsetbuttcap%
\pgfsetroundjoin%
\definecolor{currentfill}{rgb}{0.282884,0.135920,0.453427}%
\pgfsetfillcolor{currentfill}%
\pgfsetfillopacity{0.700000}%
\pgfsetlinewidth{0.000000pt}%
\definecolor{currentstroke}{rgb}{0.000000,0.000000,0.000000}%
\pgfsetstrokecolor{currentstroke}%
\pgfsetdash{}{0pt}%
\pgfpathmoveto{\pgfqpoint{4.386504in}{1.158908in}}%
\pgfpathlineto{\pgfqpoint{4.401097in}{1.159089in}}%
\pgfpathlineto{\pgfqpoint{4.415700in}{1.159338in}}%
\pgfpathlineto{\pgfqpoint{4.430313in}{1.159656in}}%
\pgfpathlineto{\pgfqpoint{4.444937in}{1.160043in}}%
\pgfpathlineto{\pgfqpoint{4.453202in}{1.174108in}}%
\pgfpathlineto{\pgfqpoint{4.461464in}{1.188253in}}%
\pgfpathlineto{\pgfqpoint{4.469721in}{1.202472in}}%
\pgfpathlineto{\pgfqpoint{4.477975in}{1.216760in}}%
\pgfpathlineto{\pgfqpoint{4.463354in}{1.215971in}}%
\pgfpathlineto{\pgfqpoint{4.448744in}{1.215250in}}%
\pgfpathlineto{\pgfqpoint{4.434144in}{1.214598in}}%
\pgfpathlineto{\pgfqpoint{4.419555in}{1.214014in}}%
\pgfpathlineto{\pgfqpoint{4.411299in}{1.200122in}}%
\pgfpathlineto{\pgfqpoint{4.403038in}{1.186302in}}%
\pgfpathlineto{\pgfqpoint{4.394773in}{1.172562in}}%
\pgfpathlineto{\pgfqpoint{4.386504in}{1.158908in}}%
\pgfpathclose%
\pgfusepath{fill}%
\end{pgfscope}%
\begin{pgfscope}%
\pgfpathrectangle{\pgfqpoint{1.150000in}{0.150000in}}{\pgfqpoint{5.700000in}{5.700000in}}%
\pgfusepath{clip}%
\pgfsetbuttcap%
\pgfsetroundjoin%
\definecolor{currentfill}{rgb}{0.280255,0.165693,0.476498}%
\pgfsetfillcolor{currentfill}%
\pgfsetfillopacity{0.700000}%
\pgfsetlinewidth{0.000000pt}%
\definecolor{currentstroke}{rgb}{0.000000,0.000000,0.000000}%
\pgfsetstrokecolor{currentstroke}%
\pgfsetdash{}{0pt}%
\pgfpathmoveto{\pgfqpoint{4.477975in}{1.216760in}}%
\pgfpathlineto{\pgfqpoint{4.492605in}{1.217618in}}%
\pgfpathlineto{\pgfqpoint{4.507247in}{1.218545in}}%
\pgfpathlineto{\pgfqpoint{4.521899in}{1.219540in}}%
\pgfpathlineto{\pgfqpoint{4.536562in}{1.220604in}}%
\pgfpathlineto{\pgfqpoint{4.544809in}{1.235346in}}%
\pgfpathlineto{\pgfqpoint{4.553052in}{1.250139in}}%
\pgfpathlineto{\pgfqpoint{4.561291in}{1.264978in}}%
\pgfpathlineto{\pgfqpoint{4.569526in}{1.279858in}}%
\pgfpathlineto{\pgfqpoint{4.554865in}{1.278410in}}%
\pgfpathlineto{\pgfqpoint{4.540214in}{1.277032in}}%
\pgfpathlineto{\pgfqpoint{4.525575in}{1.275722in}}%
\pgfpathlineto{\pgfqpoint{4.510946in}{1.274481in}}%
\pgfpathlineto{\pgfqpoint{4.502709in}{1.259977in}}%
\pgfpathlineto{\pgfqpoint{4.494469in}{1.245518in}}%
\pgfpathlineto{\pgfqpoint{4.486224in}{1.231111in}}%
\pgfpathlineto{\pgfqpoint{4.477975in}{1.216760in}}%
\pgfpathclose%
\pgfusepath{fill}%
\end{pgfscope}%
\begin{pgfscope}%
\pgfpathrectangle{\pgfqpoint{1.150000in}{0.150000in}}{\pgfqpoint{5.700000in}{5.700000in}}%
\pgfusepath{clip}%
\pgfsetbuttcap%
\pgfsetroundjoin%
\definecolor{currentfill}{rgb}{0.246811,0.283237,0.535941}%
\pgfsetfillcolor{currentfill}%
\pgfsetfillopacity{0.700000}%
\pgfsetlinewidth{0.000000pt}%
\definecolor{currentstroke}{rgb}{0.000000,0.000000,0.000000}%
\pgfsetstrokecolor{currentstroke}%
\pgfsetdash{}{0pt}%
\pgfpathmoveto{\pgfqpoint{4.785719in}{1.481963in}}%
\pgfpathlineto{\pgfqpoint{4.800486in}{1.485065in}}%
\pgfpathlineto{\pgfqpoint{4.815265in}{1.488236in}}%
\pgfpathlineto{\pgfqpoint{4.830056in}{1.491478in}}%
\pgfpathlineto{\pgfqpoint{4.844860in}{1.494789in}}%
\pgfpathlineto{\pgfqpoint{4.853042in}{1.510656in}}%
\pgfpathlineto{\pgfqpoint{4.861218in}{1.526479in}}%
\pgfpathlineto{\pgfqpoint{4.869390in}{1.542253in}}%
\pgfpathlineto{\pgfqpoint{4.877557in}{1.557973in}}%
\pgfpathlineto{\pgfqpoint{4.862752in}{1.554359in}}%
\pgfpathlineto{\pgfqpoint{4.847959in}{1.550814in}}%
\pgfpathlineto{\pgfqpoint{4.833178in}{1.547340in}}%
\pgfpathlineto{\pgfqpoint{4.818410in}{1.543936in}}%
\pgfpathlineto{\pgfqpoint{4.810244in}{1.528510in}}%
\pgfpathlineto{\pgfqpoint{4.802074in}{1.513037in}}%
\pgfpathlineto{\pgfqpoint{4.793899in}{1.497519in}}%
\pgfpathlineto{\pgfqpoint{4.785719in}{1.481963in}}%
\pgfpathclose%
\pgfusepath{fill}%
\end{pgfscope}%
\begin{pgfscope}%
\pgfpathrectangle{\pgfqpoint{1.150000in}{0.150000in}}{\pgfqpoint{5.700000in}{5.700000in}}%
\pgfusepath{clip}%
\pgfsetbuttcap%
\pgfsetroundjoin%
\definecolor{currentfill}{rgb}{0.274128,0.199721,0.498911}%
\pgfsetfillcolor{currentfill}%
\pgfsetfillopacity{0.700000}%
\pgfsetlinewidth{0.000000pt}%
\definecolor{currentstroke}{rgb}{0.000000,0.000000,0.000000}%
\pgfsetstrokecolor{currentstroke}%
\pgfsetdash{}{0pt}%
\pgfpathmoveto{\pgfqpoint{4.569526in}{1.279858in}}%
\pgfpathlineto{\pgfqpoint{4.584199in}{1.281374in}}%
\pgfpathlineto{\pgfqpoint{4.598883in}{1.282960in}}%
\pgfpathlineto{\pgfqpoint{4.613577in}{1.284614in}}%
\pgfpathlineto{\pgfqpoint{4.628283in}{1.286337in}}%
\pgfpathlineto{\pgfqpoint{4.636514in}{1.301623in}}%
\pgfpathlineto{\pgfqpoint{4.644740in}{1.316934in}}%
\pgfpathlineto{\pgfqpoint{4.652963in}{1.332264in}}%
\pgfpathlineto{\pgfqpoint{4.661181in}{1.347607in}}%
\pgfpathlineto{\pgfqpoint{4.646475in}{1.345520in}}%
\pgfpathlineto{\pgfqpoint{4.631780in}{1.343502in}}%
\pgfpathlineto{\pgfqpoint{4.617097in}{1.341553in}}%
\pgfpathlineto{\pgfqpoint{4.602425in}{1.339673in}}%
\pgfpathlineto{\pgfqpoint{4.594207in}{1.324686in}}%
\pgfpathlineto{\pgfqpoint{4.585984in}{1.309717in}}%
\pgfpathlineto{\pgfqpoint{4.577757in}{1.294773in}}%
\pgfpathlineto{\pgfqpoint{4.569526in}{1.279858in}}%
\pgfpathclose%
\pgfusepath{fill}%
\end{pgfscope}%
\begin{pgfscope}%
\pgfpathrectangle{\pgfqpoint{1.150000in}{0.150000in}}{\pgfqpoint{5.700000in}{5.700000in}}%
\pgfusepath{clip}%
\pgfsetbuttcap%
\pgfsetroundjoin%
\definecolor{currentfill}{rgb}{0.231674,0.318106,0.544834}%
\pgfsetfillcolor{currentfill}%
\pgfsetfillopacity{0.700000}%
\pgfsetlinewidth{0.000000pt}%
\definecolor{currentstroke}{rgb}{0.000000,0.000000,0.000000}%
\pgfsetstrokecolor{currentstroke}%
\pgfsetdash{}{0pt}%
\pgfpathmoveto{\pgfqpoint{4.877557in}{1.557973in}}%
\pgfpathlineto{\pgfqpoint{4.892376in}{1.561658in}}%
\pgfpathlineto{\pgfqpoint{4.907207in}{1.565413in}}%
\pgfpathlineto{\pgfqpoint{4.922051in}{1.569238in}}%
\pgfpathlineto{\pgfqpoint{4.930215in}{1.585120in}}%
\pgfpathlineto{\pgfqpoint{4.938373in}{1.600936in}}%
\pgfpathlineto{\pgfqpoint{4.946527in}{1.616682in}}%
\pgfpathlineto{\pgfqpoint{4.954675in}{1.632356in}}%
\pgfpathlineto{\pgfqpoint{4.939829in}{1.628248in}}%
\pgfpathlineto{\pgfqpoint{4.924996in}{1.624210in}}%
\pgfpathlineto{\pgfqpoint{4.910175in}{1.620244in}}%
\pgfpathlineto{\pgfqpoint{4.902029in}{1.604776in}}%
\pgfpathlineto{\pgfqpoint{4.893877in}{1.589239in}}%
\pgfpathlineto{\pgfqpoint{4.885720in}{1.573637in}}%
\pgfpathlineto{\pgfqpoint{4.877557in}{1.557973in}}%
\pgfpathclose%
\pgfusepath{fill}%
\end{pgfscope}%
\begin{pgfscope}%
\pgfpathrectangle{\pgfqpoint{1.150000in}{0.150000in}}{\pgfqpoint{5.700000in}{5.700000in}}%
\pgfusepath{clip}%
\pgfsetbuttcap%
\pgfsetroundjoin%
\definecolor{currentfill}{rgb}{0.265145,0.232956,0.516599}%
\pgfsetfillcolor{currentfill}%
\pgfsetfillopacity{0.700000}%
\pgfsetlinewidth{0.000000pt}%
\definecolor{currentstroke}{rgb}{0.000000,0.000000,0.000000}%
\pgfsetstrokecolor{currentstroke}%
\pgfsetdash{}{0pt}%
\pgfpathmoveto{\pgfqpoint{4.661181in}{1.347607in}}%
\pgfpathlineto{\pgfqpoint{4.675898in}{1.349764in}}%
\pgfpathlineto{\pgfqpoint{4.690627in}{1.351989in}}%
\pgfpathlineto{\pgfqpoint{4.705368in}{1.354284in}}%
\pgfpathlineto{\pgfqpoint{4.720120in}{1.356648in}}%
\pgfpathlineto{\pgfqpoint{4.728335in}{1.372352in}}%
\pgfpathlineto{\pgfqpoint{4.736546in}{1.388055in}}%
\pgfpathlineto{\pgfqpoint{4.744752in}{1.403751in}}%
\pgfpathlineto{\pgfqpoint{4.752954in}{1.419435in}}%
\pgfpathlineto{\pgfqpoint{4.738201in}{1.416727in}}%
\pgfpathlineto{\pgfqpoint{4.723459in}{1.414088in}}%
\pgfpathlineto{\pgfqpoint{4.708730in}{1.411518in}}%
\pgfpathlineto{\pgfqpoint{4.694012in}{1.409018in}}%
\pgfpathlineto{\pgfqpoint{4.685810in}{1.393670in}}%
\pgfpathlineto{\pgfqpoint{4.677605in}{1.378315in}}%
\pgfpathlineto{\pgfqpoint{4.669395in}{1.362959in}}%
\pgfpathlineto{\pgfqpoint{4.661181in}{1.347607in}}%
\pgfpathclose%
\pgfusepath{fill}%
\end{pgfscope}%
\begin{pgfscope}%
\pgfpathrectangle{\pgfqpoint{1.150000in}{0.150000in}}{\pgfqpoint{5.700000in}{5.700000in}}%
\pgfusepath{clip}%
\pgfsetbuttcap%
\pgfsetroundjoin%
\definecolor{currentfill}{rgb}{0.283229,0.120777,0.440584}%
\pgfsetfillcolor{currentfill}%
\pgfsetfillopacity{0.700000}%
\pgfsetlinewidth{0.000000pt}%
\definecolor{currentstroke}{rgb}{0.000000,0.000000,0.000000}%
\pgfsetstrokecolor{currentstroke}%
\pgfsetdash{}{0pt}%
\pgfpathmoveto{\pgfqpoint{4.353383in}{1.105266in}}%
\pgfpathlineto{\pgfqpoint{4.367981in}{1.105024in}}%
\pgfpathlineto{\pgfqpoint{4.382588in}{1.104850in}}%
\pgfpathlineto{\pgfqpoint{4.397206in}{1.104745in}}%
\pgfpathlineto{\pgfqpoint{4.411833in}{1.104707in}}%
\pgfpathlineto{\pgfqpoint{4.420115in}{1.118390in}}%
\pgfpathlineto{\pgfqpoint{4.428393in}{1.132177in}}%
\pgfpathlineto{\pgfqpoint{4.436667in}{1.146064in}}%
\pgfpathlineto{\pgfqpoint{4.444937in}{1.160043in}}%
\pgfpathlineto{\pgfqpoint{4.430313in}{1.159656in}}%
\pgfpathlineto{\pgfqpoint{4.415700in}{1.159338in}}%
\pgfpathlineto{\pgfqpoint{4.401097in}{1.159089in}}%
\pgfpathlineto{\pgfqpoint{4.386504in}{1.158908in}}%
\pgfpathlineto{\pgfqpoint{4.378230in}{1.145344in}}%
\pgfpathlineto{\pgfqpoint{4.369952in}{1.131879in}}%
\pgfpathlineto{\pgfqpoint{4.361670in}{1.118517in}}%
\pgfpathlineto{\pgfqpoint{4.353383in}{1.105266in}}%
\pgfpathclose%
\pgfusepath{fill}%
\end{pgfscope}%
\begin{pgfscope}%
\pgfpathrectangle{\pgfqpoint{1.150000in}{0.150000in}}{\pgfqpoint{5.700000in}{5.700000in}}%
\pgfusepath{clip}%
\pgfsetbuttcap%
\pgfsetroundjoin%
\definecolor{currentfill}{rgb}{0.281887,0.150881,0.465405}%
\pgfsetfillcolor{currentfill}%
\pgfsetfillopacity{0.700000}%
\pgfsetlinewidth{0.000000pt}%
\definecolor{currentstroke}{rgb}{0.000000,0.000000,0.000000}%
\pgfsetstrokecolor{currentstroke}%
\pgfsetdash{}{0pt}%
\pgfpathmoveto{\pgfqpoint{4.444937in}{1.160043in}}%
\pgfpathlineto{\pgfqpoint{4.459570in}{1.160497in}}%
\pgfpathlineto{\pgfqpoint{4.474214in}{1.161020in}}%
\pgfpathlineto{\pgfqpoint{4.488869in}{1.161611in}}%
\pgfpathlineto{\pgfqpoint{4.503534in}{1.162271in}}%
\pgfpathlineto{\pgfqpoint{4.511797in}{1.176748in}}%
\pgfpathlineto{\pgfqpoint{4.520056in}{1.191299in}}%
\pgfpathlineto{\pgfqpoint{4.528311in}{1.205920in}}%
\pgfpathlineto{\pgfqpoint{4.536562in}{1.220604in}}%
\pgfpathlineto{\pgfqpoint{4.521899in}{1.219540in}}%
\pgfpathlineto{\pgfqpoint{4.507247in}{1.218545in}}%
\pgfpathlineto{\pgfqpoint{4.492605in}{1.217618in}}%
\pgfpathlineto{\pgfqpoint{4.477975in}{1.216760in}}%
\pgfpathlineto{\pgfqpoint{4.469721in}{1.202472in}}%
\pgfpathlineto{\pgfqpoint{4.461464in}{1.188253in}}%
\pgfpathlineto{\pgfqpoint{4.453202in}{1.174108in}}%
\pgfpathlineto{\pgfqpoint{4.444937in}{1.160043in}}%
\pgfpathclose%
\pgfusepath{fill}%
\end{pgfscope}%
\begin{pgfscope}%
\pgfpathrectangle{\pgfqpoint{1.150000in}{0.150000in}}{\pgfqpoint{5.700000in}{5.700000in}}%
\pgfusepath{clip}%
\pgfsetbuttcap%
\pgfsetroundjoin%
\definecolor{currentfill}{rgb}{0.252194,0.269783,0.531579}%
\pgfsetfillcolor{currentfill}%
\pgfsetfillopacity{0.700000}%
\pgfsetlinewidth{0.000000pt}%
\definecolor{currentstroke}{rgb}{0.000000,0.000000,0.000000}%
\pgfsetstrokecolor{currentstroke}%
\pgfsetdash{}{0pt}%
\pgfpathmoveto{\pgfqpoint{4.752954in}{1.419435in}}%
\pgfpathlineto{\pgfqpoint{4.767720in}{1.422213in}}%
\pgfpathlineto{\pgfqpoint{4.782497in}{1.425061in}}%
\pgfpathlineto{\pgfqpoint{4.797287in}{1.427978in}}%
\pgfpathlineto{\pgfqpoint{4.812089in}{1.430965in}}%
\pgfpathlineto{\pgfqpoint{4.820289in}{1.446965in}}%
\pgfpathlineto{\pgfqpoint{4.828484in}{1.462939in}}%
\pgfpathlineto{\pgfqpoint{4.836674in}{1.478881in}}%
\pgfpathlineto{\pgfqpoint{4.844860in}{1.494789in}}%
\pgfpathlineto{\pgfqpoint{4.830056in}{1.491478in}}%
\pgfpathlineto{\pgfqpoint{4.815265in}{1.488236in}}%
\pgfpathlineto{\pgfqpoint{4.800486in}{1.485065in}}%
\pgfpathlineto{\pgfqpoint{4.785719in}{1.481963in}}%
\pgfpathlineto{\pgfqpoint{4.777535in}{1.466372in}}%
\pgfpathlineto{\pgfqpoint{4.769346in}{1.450750in}}%
\pgfpathlineto{\pgfqpoint{4.761152in}{1.435103in}}%
\pgfpathlineto{\pgfqpoint{4.752954in}{1.419435in}}%
\pgfpathclose%
\pgfusepath{fill}%
\end{pgfscope}%
\begin{pgfscope}%
\pgfpathrectangle{\pgfqpoint{1.150000in}{0.150000in}}{\pgfqpoint{5.700000in}{5.700000in}}%
\pgfusepath{clip}%
\pgfsetbuttcap%
\pgfsetroundjoin%
\definecolor{currentfill}{rgb}{0.278012,0.180367,0.486697}%
\pgfsetfillcolor{currentfill}%
\pgfsetfillopacity{0.700000}%
\pgfsetlinewidth{0.000000pt}%
\definecolor{currentstroke}{rgb}{0.000000,0.000000,0.000000}%
\pgfsetstrokecolor{currentstroke}%
\pgfsetdash{}{0pt}%
\pgfpathmoveto{\pgfqpoint{4.536562in}{1.220604in}}%
\pgfpathlineto{\pgfqpoint{4.551236in}{1.221737in}}%
\pgfpathlineto{\pgfqpoint{4.565920in}{1.222938in}}%
\pgfpathlineto{\pgfqpoint{4.580616in}{1.224208in}}%
\pgfpathlineto{\pgfqpoint{4.595322in}{1.225546in}}%
\pgfpathlineto{\pgfqpoint{4.603568in}{1.240680in}}%
\pgfpathlineto{\pgfqpoint{4.611810in}{1.255860in}}%
\pgfpathlineto{\pgfqpoint{4.620049in}{1.271081in}}%
\pgfpathlineto{\pgfqpoint{4.628283in}{1.286337in}}%
\pgfpathlineto{\pgfqpoint{4.613577in}{1.284614in}}%
\pgfpathlineto{\pgfqpoint{4.598883in}{1.282960in}}%
\pgfpathlineto{\pgfqpoint{4.584199in}{1.281374in}}%
\pgfpathlineto{\pgfqpoint{4.569526in}{1.279858in}}%
\pgfpathlineto{\pgfqpoint{4.561291in}{1.264978in}}%
\pgfpathlineto{\pgfqpoint{4.553052in}{1.250139in}}%
\pgfpathlineto{\pgfqpoint{4.544809in}{1.235346in}}%
\pgfpathlineto{\pgfqpoint{4.536562in}{1.220604in}}%
\pgfpathclose%
\pgfusepath{fill}%
\end{pgfscope}%
\begin{pgfscope}%
\pgfpathrectangle{\pgfqpoint{1.150000in}{0.150000in}}{\pgfqpoint{5.700000in}{5.700000in}}%
\pgfusepath{clip}%
\pgfsetbuttcap%
\pgfsetroundjoin%
\definecolor{currentfill}{rgb}{0.239346,0.300855,0.540844}%
\pgfsetfillcolor{currentfill}%
\pgfsetfillopacity{0.700000}%
\pgfsetlinewidth{0.000000pt}%
\definecolor{currentstroke}{rgb}{0.000000,0.000000,0.000000}%
\pgfsetstrokecolor{currentstroke}%
\pgfsetdash{}{0pt}%
\pgfpathmoveto{\pgfqpoint{4.844860in}{1.494789in}}%
\pgfpathlineto{\pgfqpoint{4.859677in}{1.498170in}}%
\pgfpathlineto{\pgfqpoint{4.874506in}{1.501621in}}%
\pgfpathlineto{\pgfqpoint{4.889347in}{1.505142in}}%
\pgfpathlineto{\pgfqpoint{4.897530in}{1.521243in}}%
\pgfpathlineto{\pgfqpoint{4.905709in}{1.537296in}}%
\pgfpathlineto{\pgfqpoint{4.913882in}{1.553296in}}%
\pgfpathlineto{\pgfqpoint{4.922051in}{1.569238in}}%
\pgfpathlineto{\pgfqpoint{4.907207in}{1.565413in}}%
\pgfpathlineto{\pgfqpoint{4.892376in}{1.561658in}}%
\pgfpathlineto{\pgfqpoint{4.877557in}{1.557973in}}%
\pgfpathlineto{\pgfqpoint{4.869390in}{1.542253in}}%
\pgfpathlineto{\pgfqpoint{4.861218in}{1.526479in}}%
\pgfpathlineto{\pgfqpoint{4.853042in}{1.510656in}}%
\pgfpathlineto{\pgfqpoint{4.844860in}{1.494789in}}%
\pgfpathclose%
\pgfusepath{fill}%
\end{pgfscope}%
\begin{pgfscope}%
\pgfpathrectangle{\pgfqpoint{1.150000in}{0.150000in}}{\pgfqpoint{5.700000in}{5.700000in}}%
\pgfusepath{clip}%
\pgfsetbuttcap%
\pgfsetroundjoin%
\definecolor{currentfill}{rgb}{0.270595,0.214069,0.507052}%
\pgfsetfillcolor{currentfill}%
\pgfsetfillopacity{0.700000}%
\pgfsetlinewidth{0.000000pt}%
\definecolor{currentstroke}{rgb}{0.000000,0.000000,0.000000}%
\pgfsetstrokecolor{currentstroke}%
\pgfsetdash{}{0pt}%
\pgfpathmoveto{\pgfqpoint{4.628283in}{1.286337in}}%
\pgfpathlineto{\pgfqpoint{4.643001in}{1.288129in}}%
\pgfpathlineto{\pgfqpoint{4.657729in}{1.289990in}}%
\pgfpathlineto{\pgfqpoint{4.672469in}{1.291920in}}%
\pgfpathlineto{\pgfqpoint{4.687221in}{1.293919in}}%
\pgfpathlineto{\pgfqpoint{4.695452in}{1.309577in}}%
\pgfpathlineto{\pgfqpoint{4.703678in}{1.325255in}}%
\pgfpathlineto{\pgfqpoint{4.711901in}{1.340947in}}%
\pgfpathlineto{\pgfqpoint{4.720120in}{1.356648in}}%
\pgfpathlineto{\pgfqpoint{4.705368in}{1.354284in}}%
\pgfpathlineto{\pgfqpoint{4.690627in}{1.351989in}}%
\pgfpathlineto{\pgfqpoint{4.675898in}{1.349764in}}%
\pgfpathlineto{\pgfqpoint{4.661181in}{1.347607in}}%
\pgfpathlineto{\pgfqpoint{4.652963in}{1.332264in}}%
\pgfpathlineto{\pgfqpoint{4.644740in}{1.316934in}}%
\pgfpathlineto{\pgfqpoint{4.636514in}{1.301623in}}%
\pgfpathlineto{\pgfqpoint{4.628283in}{1.286337in}}%
\pgfpathclose%
\pgfusepath{fill}%
\end{pgfscope}%
\begin{pgfscope}%
\pgfpathrectangle{\pgfqpoint{1.150000in}{0.150000in}}{\pgfqpoint{5.700000in}{5.700000in}}%
\pgfusepath{clip}%
\pgfsetbuttcap%
\pgfsetroundjoin%
\definecolor{currentfill}{rgb}{0.260571,0.246922,0.522828}%
\pgfsetfillcolor{currentfill}%
\pgfsetfillopacity{0.700000}%
\pgfsetlinewidth{0.000000pt}%
\definecolor{currentstroke}{rgb}{0.000000,0.000000,0.000000}%
\pgfsetstrokecolor{currentstroke}%
\pgfsetdash{}{0pt}%
\pgfpathmoveto{\pgfqpoint{4.720120in}{1.356648in}}%
\pgfpathlineto{\pgfqpoint{4.734885in}{1.359081in}}%
\pgfpathlineto{\pgfqpoint{4.749661in}{1.361584in}}%
\pgfpathlineto{\pgfqpoint{4.764449in}{1.364156in}}%
\pgfpathlineto{\pgfqpoint{4.779249in}{1.366797in}}%
\pgfpathlineto{\pgfqpoint{4.787465in}{1.382854in}}%
\pgfpathlineto{\pgfqpoint{4.795677in}{1.398904in}}%
\pgfpathlineto{\pgfqpoint{4.803885in}{1.414943in}}%
\pgfpathlineto{\pgfqpoint{4.812089in}{1.430965in}}%
\pgfpathlineto{\pgfqpoint{4.797287in}{1.427978in}}%
\pgfpathlineto{\pgfqpoint{4.782497in}{1.425061in}}%
\pgfpathlineto{\pgfqpoint{4.767720in}{1.422213in}}%
\pgfpathlineto{\pgfqpoint{4.752954in}{1.419435in}}%
\pgfpathlineto{\pgfqpoint{4.744752in}{1.403751in}}%
\pgfpathlineto{\pgfqpoint{4.736546in}{1.388055in}}%
\pgfpathlineto{\pgfqpoint{4.728335in}{1.372352in}}%
\pgfpathlineto{\pgfqpoint{4.720120in}{1.356648in}}%
\pgfpathclose%
\pgfusepath{fill}%
\end{pgfscope}%
\begin{pgfscope}%
\pgfpathrectangle{\pgfqpoint{1.150000in}{0.150000in}}{\pgfqpoint{5.700000in}{5.700000in}}%
\pgfusepath{clip}%
\pgfsetbuttcap%
\pgfsetroundjoin%
\definecolor{currentfill}{rgb}{0.283072,0.130895,0.449241}%
\pgfsetfillcolor{currentfill}%
\pgfsetfillopacity{0.700000}%
\pgfsetlinewidth{0.000000pt}%
\definecolor{currentstroke}{rgb}{0.000000,0.000000,0.000000}%
\pgfsetstrokecolor{currentstroke}%
\pgfsetdash{}{0pt}%
\pgfpathmoveto{\pgfqpoint{4.411833in}{1.104707in}}%
\pgfpathlineto{\pgfqpoint{4.426470in}{1.104738in}}%
\pgfpathlineto{\pgfqpoint{4.441117in}{1.104837in}}%
\pgfpathlineto{\pgfqpoint{4.455774in}{1.105003in}}%
\pgfpathlineto{\pgfqpoint{4.470442in}{1.105238in}}%
\pgfpathlineto{\pgfqpoint{4.478721in}{1.119353in}}%
\pgfpathlineto{\pgfqpoint{4.486996in}{1.133567in}}%
\pgfpathlineto{\pgfqpoint{4.495267in}{1.147876in}}%
\pgfpathlineto{\pgfqpoint{4.503534in}{1.162271in}}%
\pgfpathlineto{\pgfqpoint{4.488869in}{1.161611in}}%
\pgfpathlineto{\pgfqpoint{4.474214in}{1.161020in}}%
\pgfpathlineto{\pgfqpoint{4.459570in}{1.160497in}}%
\pgfpathlineto{\pgfqpoint{4.444937in}{1.160043in}}%
\pgfpathlineto{\pgfqpoint{4.436667in}{1.146064in}}%
\pgfpathlineto{\pgfqpoint{4.428393in}{1.132177in}}%
\pgfpathlineto{\pgfqpoint{4.420115in}{1.118390in}}%
\pgfpathlineto{\pgfqpoint{4.411833in}{1.104707in}}%
\pgfpathclose%
\pgfusepath{fill}%
\end{pgfscope}%
\begin{pgfscope}%
\pgfpathrectangle{\pgfqpoint{1.150000in}{0.150000in}}{\pgfqpoint{5.700000in}{5.700000in}}%
\pgfusepath{clip}%
\pgfsetbuttcap%
\pgfsetroundjoin%
\definecolor{currentfill}{rgb}{0.280868,0.160771,0.472899}%
\pgfsetfillcolor{currentfill}%
\pgfsetfillopacity{0.700000}%
\pgfsetlinewidth{0.000000pt}%
\definecolor{currentstroke}{rgb}{0.000000,0.000000,0.000000}%
\pgfsetstrokecolor{currentstroke}%
\pgfsetdash{}{0pt}%
\pgfpathmoveto{\pgfqpoint{4.503534in}{1.162271in}}%
\pgfpathlineto{\pgfqpoint{4.518209in}{1.162999in}}%
\pgfpathlineto{\pgfqpoint{4.532895in}{1.163795in}}%
\pgfpathlineto{\pgfqpoint{4.547591in}{1.164660in}}%
\pgfpathlineto{\pgfqpoint{4.562299in}{1.165592in}}%
\pgfpathlineto{\pgfqpoint{4.570560in}{1.180482in}}%
\pgfpathlineto{\pgfqpoint{4.578818in}{1.195441in}}%
\pgfpathlineto{\pgfqpoint{4.587072in}{1.210464in}}%
\pgfpathlineto{\pgfqpoint{4.595322in}{1.225546in}}%
\pgfpathlineto{\pgfqpoint{4.580616in}{1.224208in}}%
\pgfpathlineto{\pgfqpoint{4.565920in}{1.222938in}}%
\pgfpathlineto{\pgfqpoint{4.551236in}{1.221737in}}%
\pgfpathlineto{\pgfqpoint{4.536562in}{1.220604in}}%
\pgfpathlineto{\pgfqpoint{4.528311in}{1.205920in}}%
\pgfpathlineto{\pgfqpoint{4.520056in}{1.191299in}}%
\pgfpathlineto{\pgfqpoint{4.511797in}{1.176748in}}%
\pgfpathlineto{\pgfqpoint{4.503534in}{1.162271in}}%
\pgfpathclose%
\pgfusepath{fill}%
\end{pgfscope}%
\begin{pgfscope}%
\pgfpathrectangle{\pgfqpoint{1.150000in}{0.150000in}}{\pgfqpoint{5.700000in}{5.700000in}}%
\pgfusepath{clip}%
\pgfsetbuttcap%
\pgfsetroundjoin%
\definecolor{currentfill}{rgb}{0.246811,0.283237,0.535941}%
\pgfsetfillcolor{currentfill}%
\pgfsetfillopacity{0.700000}%
\pgfsetlinewidth{0.000000pt}%
\definecolor{currentstroke}{rgb}{0.000000,0.000000,0.000000}%
\pgfsetstrokecolor{currentstroke}%
\pgfsetdash{}{0pt}%
\pgfpathmoveto{\pgfqpoint{4.812089in}{1.430965in}}%
\pgfpathlineto{\pgfqpoint{4.826903in}{1.434021in}}%
\pgfpathlineto{\pgfqpoint{4.841730in}{1.437147in}}%
\pgfpathlineto{\pgfqpoint{4.856569in}{1.440343in}}%
\pgfpathlineto{\pgfqpoint{4.864770in}{1.456593in}}%
\pgfpathlineto{\pgfqpoint{4.872967in}{1.472812in}}%
\pgfpathlineto{\pgfqpoint{4.881159in}{1.488997in}}%
\pgfpathlineto{\pgfqpoint{4.889347in}{1.505142in}}%
\pgfpathlineto{\pgfqpoint{4.874506in}{1.501621in}}%
\pgfpathlineto{\pgfqpoint{4.859677in}{1.498170in}}%
\pgfpathlineto{\pgfqpoint{4.844860in}{1.494789in}}%
\pgfpathlineto{\pgfqpoint{4.836674in}{1.478881in}}%
\pgfpathlineto{\pgfqpoint{4.828484in}{1.462939in}}%
\pgfpathlineto{\pgfqpoint{4.820289in}{1.446965in}}%
\pgfpathlineto{\pgfqpoint{4.812089in}{1.430965in}}%
\pgfpathclose%
\pgfusepath{fill}%
\end{pgfscope}%
\begin{pgfscope}%
\pgfpathrectangle{\pgfqpoint{1.150000in}{0.150000in}}{\pgfqpoint{5.700000in}{5.700000in}}%
\pgfusepath{clip}%
\pgfsetbuttcap%
\pgfsetroundjoin%
\definecolor{currentfill}{rgb}{0.275191,0.194905,0.496005}%
\pgfsetfillcolor{currentfill}%
\pgfsetfillopacity{0.700000}%
\pgfsetlinewidth{0.000000pt}%
\definecolor{currentstroke}{rgb}{0.000000,0.000000,0.000000}%
\pgfsetstrokecolor{currentstroke}%
\pgfsetdash{}{0pt}%
\pgfpathmoveto{\pgfqpoint{4.595322in}{1.225546in}}%
\pgfpathlineto{\pgfqpoint{4.610039in}{1.226953in}}%
\pgfpathlineto{\pgfqpoint{4.624768in}{1.228428in}}%
\pgfpathlineto{\pgfqpoint{4.639508in}{1.229972in}}%
\pgfpathlineto{\pgfqpoint{4.654259in}{1.231585in}}%
\pgfpathlineto{\pgfqpoint{4.662505in}{1.247112in}}%
\pgfpathlineto{\pgfqpoint{4.670747in}{1.262680in}}%
\pgfpathlineto{\pgfqpoint{4.678986in}{1.278284in}}%
\pgfpathlineto{\pgfqpoint{4.687221in}{1.293919in}}%
\pgfpathlineto{\pgfqpoint{4.672469in}{1.291920in}}%
\pgfpathlineto{\pgfqpoint{4.657729in}{1.289990in}}%
\pgfpathlineto{\pgfqpoint{4.643001in}{1.288129in}}%
\pgfpathlineto{\pgfqpoint{4.628283in}{1.286337in}}%
\pgfpathlineto{\pgfqpoint{4.620049in}{1.271081in}}%
\pgfpathlineto{\pgfqpoint{4.611810in}{1.255860in}}%
\pgfpathlineto{\pgfqpoint{4.603568in}{1.240680in}}%
\pgfpathlineto{\pgfqpoint{4.595322in}{1.225546in}}%
\pgfpathclose%
\pgfusepath{fill}%
\end{pgfscope}%
\begin{pgfscope}%
\pgfpathrectangle{\pgfqpoint{1.150000in}{0.150000in}}{\pgfqpoint{5.700000in}{5.700000in}}%
\pgfusepath{clip}%
\pgfsetbuttcap%
\pgfsetroundjoin%
\definecolor{currentfill}{rgb}{0.266580,0.228262,0.514349}%
\pgfsetfillcolor{currentfill}%
\pgfsetfillopacity{0.700000}%
\pgfsetlinewidth{0.000000pt}%
\definecolor{currentstroke}{rgb}{0.000000,0.000000,0.000000}%
\pgfsetstrokecolor{currentstroke}%
\pgfsetdash{}{0pt}%
\pgfpathmoveto{\pgfqpoint{4.687221in}{1.293919in}}%
\pgfpathlineto{\pgfqpoint{4.701984in}{1.295986in}}%
\pgfpathlineto{\pgfqpoint{4.716758in}{1.298123in}}%
\pgfpathlineto{\pgfqpoint{4.731545in}{1.300328in}}%
\pgfpathlineto{\pgfqpoint{4.746343in}{1.302603in}}%
\pgfpathlineto{\pgfqpoint{4.754575in}{1.318636in}}%
\pgfpathlineto{\pgfqpoint{4.762804in}{1.334682in}}%
\pgfpathlineto{\pgfqpoint{4.771028in}{1.350738in}}%
\pgfpathlineto{\pgfqpoint{4.779249in}{1.366797in}}%
\pgfpathlineto{\pgfqpoint{4.764449in}{1.364156in}}%
\pgfpathlineto{\pgfqpoint{4.749661in}{1.361584in}}%
\pgfpathlineto{\pgfqpoint{4.734885in}{1.359081in}}%
\pgfpathlineto{\pgfqpoint{4.720120in}{1.356648in}}%
\pgfpathlineto{\pgfqpoint{4.711901in}{1.340947in}}%
\pgfpathlineto{\pgfqpoint{4.703678in}{1.325255in}}%
\pgfpathlineto{\pgfqpoint{4.695452in}{1.309577in}}%
\pgfpathlineto{\pgfqpoint{4.687221in}{1.293919in}}%
\pgfpathclose%
\pgfusepath{fill}%
\end{pgfscope}%
\begin{pgfscope}%
\pgfpathrectangle{\pgfqpoint{1.150000in}{0.150000in}}{\pgfqpoint{5.700000in}{5.700000in}}%
\pgfusepath{clip}%
\pgfsetbuttcap%
\pgfsetroundjoin%
\definecolor{currentfill}{rgb}{0.253935,0.265254,0.529983}%
\pgfsetfillcolor{currentfill}%
\pgfsetfillopacity{0.700000}%
\pgfsetlinewidth{0.000000pt}%
\definecolor{currentstroke}{rgb}{0.000000,0.000000,0.000000}%
\pgfsetstrokecolor{currentstroke}%
\pgfsetdash{}{0pt}%
\pgfpathmoveto{\pgfqpoint{4.779249in}{1.366797in}}%
\pgfpathlineto{\pgfqpoint{4.794061in}{1.369507in}}%
\pgfpathlineto{\pgfqpoint{4.808885in}{1.372287in}}%
\pgfpathlineto{\pgfqpoint{4.823721in}{1.375136in}}%
\pgfpathlineto{\pgfqpoint{4.831939in}{1.391459in}}%
\pgfpathlineto{\pgfqpoint{4.840153in}{1.407771in}}%
\pgfpathlineto{\pgfqpoint{4.848363in}{1.424067in}}%
\pgfpathlineto{\pgfqpoint{4.856569in}{1.440343in}}%
\pgfpathlineto{\pgfqpoint{4.841730in}{1.437147in}}%
\pgfpathlineto{\pgfqpoint{4.826903in}{1.434021in}}%
\pgfpathlineto{\pgfqpoint{4.812089in}{1.430965in}}%
\pgfpathlineto{\pgfqpoint{4.803885in}{1.414943in}}%
\pgfpathlineto{\pgfqpoint{4.795677in}{1.398904in}}%
\pgfpathlineto{\pgfqpoint{4.787465in}{1.382854in}}%
\pgfpathlineto{\pgfqpoint{4.779249in}{1.366797in}}%
\pgfpathclose%
\pgfusepath{fill}%
\end{pgfscope}%
\begin{pgfscope}%
\pgfpathrectangle{\pgfqpoint{1.150000in}{0.150000in}}{\pgfqpoint{5.700000in}{5.700000in}}%
\pgfusepath{clip}%
\pgfsetbuttcap%
\pgfsetroundjoin%
\definecolor{currentfill}{rgb}{0.282290,0.145912,0.461510}%
\pgfsetfillcolor{currentfill}%
\pgfsetfillopacity{0.700000}%
\pgfsetlinewidth{0.000000pt}%
\definecolor{currentstroke}{rgb}{0.000000,0.000000,0.000000}%
\pgfsetstrokecolor{currentstroke}%
\pgfsetdash{}{0pt}%
\pgfpathmoveto{\pgfqpoint{4.470442in}{1.105238in}}%
\pgfpathlineto{\pgfqpoint{4.485120in}{1.105541in}}%
\pgfpathlineto{\pgfqpoint{4.499808in}{1.105912in}}%
\pgfpathlineto{\pgfqpoint{4.514506in}{1.106350in}}%
\pgfpathlineto{\pgfqpoint{4.529215in}{1.106857in}}%
\pgfpathlineto{\pgfqpoint{4.537492in}{1.121405in}}%
\pgfpathlineto{\pgfqpoint{4.545764in}{1.136048in}}%
\pgfpathlineto{\pgfqpoint{4.554034in}{1.150779in}}%
\pgfpathlineto{\pgfqpoint{4.562299in}{1.165592in}}%
\pgfpathlineto{\pgfqpoint{4.547591in}{1.164660in}}%
\pgfpathlineto{\pgfqpoint{4.532895in}{1.163795in}}%
\pgfpathlineto{\pgfqpoint{4.518209in}{1.162999in}}%
\pgfpathlineto{\pgfqpoint{4.503534in}{1.162271in}}%
\pgfpathlineto{\pgfqpoint{4.495267in}{1.147876in}}%
\pgfpathlineto{\pgfqpoint{4.486996in}{1.133567in}}%
\pgfpathlineto{\pgfqpoint{4.478721in}{1.119353in}}%
\pgfpathlineto{\pgfqpoint{4.470442in}{1.105238in}}%
\pgfpathclose%
\pgfusepath{fill}%
\end{pgfscope}%
\begin{pgfscope}%
\pgfpathrectangle{\pgfqpoint{1.150000in}{0.150000in}}{\pgfqpoint{5.700000in}{5.700000in}}%
\pgfusepath{clip}%
\pgfsetbuttcap%
\pgfsetroundjoin%
\definecolor{currentfill}{rgb}{0.278826,0.175490,0.483397}%
\pgfsetfillcolor{currentfill}%
\pgfsetfillopacity{0.700000}%
\pgfsetlinewidth{0.000000pt}%
\definecolor{currentstroke}{rgb}{0.000000,0.000000,0.000000}%
\pgfsetstrokecolor{currentstroke}%
\pgfsetdash{}{0pt}%
\pgfpathmoveto{\pgfqpoint{4.562299in}{1.165592in}}%
\pgfpathlineto{\pgfqpoint{4.577017in}{1.166594in}}%
\pgfpathlineto{\pgfqpoint{4.591746in}{1.167663in}}%
\pgfpathlineto{\pgfqpoint{4.606486in}{1.168801in}}%
\pgfpathlineto{\pgfqpoint{4.621237in}{1.170007in}}%
\pgfpathlineto{\pgfqpoint{4.629498in}{1.185310in}}%
\pgfpathlineto{\pgfqpoint{4.637755in}{1.200678in}}%
\pgfpathlineto{\pgfqpoint{4.646009in}{1.216105in}}%
\pgfpathlineto{\pgfqpoint{4.654259in}{1.231585in}}%
\pgfpathlineto{\pgfqpoint{4.639508in}{1.229972in}}%
\pgfpathlineto{\pgfqpoint{4.624768in}{1.228428in}}%
\pgfpathlineto{\pgfqpoint{4.610039in}{1.226953in}}%
\pgfpathlineto{\pgfqpoint{4.595322in}{1.225546in}}%
\pgfpathlineto{\pgfqpoint{4.587072in}{1.210464in}}%
\pgfpathlineto{\pgfqpoint{4.578818in}{1.195441in}}%
\pgfpathlineto{\pgfqpoint{4.570560in}{1.180482in}}%
\pgfpathlineto{\pgfqpoint{4.562299in}{1.165592in}}%
\pgfpathclose%
\pgfusepath{fill}%
\end{pgfscope}%
\begin{pgfscope}%
\pgfpathrectangle{\pgfqpoint{1.150000in}{0.150000in}}{\pgfqpoint{5.700000in}{5.700000in}}%
\pgfusepath{clip}%
\pgfsetbuttcap%
\pgfsetroundjoin%
\definecolor{currentfill}{rgb}{0.271828,0.209303,0.504434}%
\pgfsetfillcolor{currentfill}%
\pgfsetfillopacity{0.700000}%
\pgfsetlinewidth{0.000000pt}%
\definecolor{currentstroke}{rgb}{0.000000,0.000000,0.000000}%
\pgfsetstrokecolor{currentstroke}%
\pgfsetdash{}{0pt}%
\pgfpathmoveto{\pgfqpoint{4.654259in}{1.231585in}}%
\pgfpathlineto{\pgfqpoint{4.669021in}{1.233266in}}%
\pgfpathlineto{\pgfqpoint{4.683795in}{1.235016in}}%
\pgfpathlineto{\pgfqpoint{4.698580in}{1.236835in}}%
\pgfpathlineto{\pgfqpoint{4.713376in}{1.238722in}}%
\pgfpathlineto{\pgfqpoint{4.721624in}{1.254644in}}%
\pgfpathlineto{\pgfqpoint{4.729867in}{1.270601in}}%
\pgfpathlineto{\pgfqpoint{4.738107in}{1.286590in}}%
\pgfpathlineto{\pgfqpoint{4.746343in}{1.302603in}}%
\pgfpathlineto{\pgfqpoint{4.731545in}{1.300328in}}%
\pgfpathlineto{\pgfqpoint{4.716758in}{1.298123in}}%
\pgfpathlineto{\pgfqpoint{4.701984in}{1.295986in}}%
\pgfpathlineto{\pgfqpoint{4.687221in}{1.293919in}}%
\pgfpathlineto{\pgfqpoint{4.678986in}{1.278284in}}%
\pgfpathlineto{\pgfqpoint{4.670747in}{1.262680in}}%
\pgfpathlineto{\pgfqpoint{4.662505in}{1.247112in}}%
\pgfpathlineto{\pgfqpoint{4.654259in}{1.231585in}}%
\pgfpathclose%
\pgfusepath{fill}%
\end{pgfscope}%
\begin{pgfscope}%
\pgfpathrectangle{\pgfqpoint{1.150000in}{0.150000in}}{\pgfqpoint{5.700000in}{5.700000in}}%
\pgfusepath{clip}%
\pgfsetbuttcap%
\pgfsetroundjoin%
\definecolor{currentfill}{rgb}{0.260571,0.246922,0.522828}%
\pgfsetfillcolor{currentfill}%
\pgfsetfillopacity{0.700000}%
\pgfsetlinewidth{0.000000pt}%
\definecolor{currentstroke}{rgb}{0.000000,0.000000,0.000000}%
\pgfsetstrokecolor{currentstroke}%
\pgfsetdash{}{0pt}%
\pgfpathmoveto{\pgfqpoint{4.746343in}{1.302603in}}%
\pgfpathlineto{\pgfqpoint{4.761153in}{1.304947in}}%
\pgfpathlineto{\pgfqpoint{4.775975in}{1.307359in}}%
\pgfpathlineto{\pgfqpoint{4.790809in}{1.309841in}}%
\pgfpathlineto{\pgfqpoint{4.799043in}{1.326155in}}%
\pgfpathlineto{\pgfqpoint{4.807273in}{1.342479in}}%
\pgfpathlineto{\pgfqpoint{4.815499in}{1.358808in}}%
\pgfpathlineto{\pgfqpoint{4.823721in}{1.375136in}}%
\pgfpathlineto{\pgfqpoint{4.808885in}{1.372287in}}%
\pgfpathlineto{\pgfqpoint{4.794061in}{1.369507in}}%
\pgfpathlineto{\pgfqpoint{4.779249in}{1.366797in}}%
\pgfpathlineto{\pgfqpoint{4.771028in}{1.350738in}}%
\pgfpathlineto{\pgfqpoint{4.762804in}{1.334682in}}%
\pgfpathlineto{\pgfqpoint{4.754575in}{1.318636in}}%
\pgfpathlineto{\pgfqpoint{4.746343in}{1.302603in}}%
\pgfpathclose%
\pgfusepath{fill}%
\end{pgfscope}%
\begin{pgfscope}%
\pgfpathrectangle{\pgfqpoint{1.150000in}{0.150000in}}{\pgfqpoint{5.700000in}{5.700000in}}%
\pgfusepath{clip}%
\pgfsetbuttcap%
\pgfsetroundjoin%
\definecolor{currentfill}{rgb}{0.280868,0.160771,0.472899}%
\pgfsetfillcolor{currentfill}%
\pgfsetfillopacity{0.700000}%
\pgfsetlinewidth{0.000000pt}%
\definecolor{currentstroke}{rgb}{0.000000,0.000000,0.000000}%
\pgfsetstrokecolor{currentstroke}%
\pgfsetdash{}{0pt}%
\pgfpathmoveto{\pgfqpoint{4.529215in}{1.106857in}}%
\pgfpathlineto{\pgfqpoint{4.543935in}{1.107432in}}%
\pgfpathlineto{\pgfqpoint{4.558665in}{1.108075in}}%
\pgfpathlineto{\pgfqpoint{4.573405in}{1.108785in}}%
\pgfpathlineto{\pgfqpoint{4.588157in}{1.109564in}}%
\pgfpathlineto{\pgfqpoint{4.596432in}{1.124547in}}%
\pgfpathlineto{\pgfqpoint{4.604704in}{1.139619in}}%
\pgfpathlineto{\pgfqpoint{4.612972in}{1.154774in}}%
\pgfpathlineto{\pgfqpoint{4.621237in}{1.170007in}}%
\pgfpathlineto{\pgfqpoint{4.606486in}{1.168801in}}%
\pgfpathlineto{\pgfqpoint{4.591746in}{1.167663in}}%
\pgfpathlineto{\pgfqpoint{4.577017in}{1.166594in}}%
\pgfpathlineto{\pgfqpoint{4.562299in}{1.165592in}}%
\pgfpathlineto{\pgfqpoint{4.554034in}{1.150779in}}%
\pgfpathlineto{\pgfqpoint{4.545764in}{1.136048in}}%
\pgfpathlineto{\pgfqpoint{4.537492in}{1.121405in}}%
\pgfpathlineto{\pgfqpoint{4.529215in}{1.106857in}}%
\pgfpathclose%
\pgfusepath{fill}%
\end{pgfscope}%
\begin{pgfscope}%
\pgfpathrectangle{\pgfqpoint{1.150000in}{0.150000in}}{\pgfqpoint{5.700000in}{5.700000in}}%
\pgfusepath{clip}%
\pgfsetbuttcap%
\pgfsetroundjoin%
\definecolor{currentfill}{rgb}{0.275191,0.194905,0.496005}%
\pgfsetfillcolor{currentfill}%
\pgfsetfillopacity{0.700000}%
\pgfsetlinewidth{0.000000pt}%
\definecolor{currentstroke}{rgb}{0.000000,0.000000,0.000000}%
\pgfsetstrokecolor{currentstroke}%
\pgfsetdash{}{0pt}%
\pgfpathmoveto{\pgfqpoint{4.621237in}{1.170007in}}%
\pgfpathlineto{\pgfqpoint{4.635999in}{1.171281in}}%
\pgfpathlineto{\pgfqpoint{4.650772in}{1.172623in}}%
\pgfpathlineto{\pgfqpoint{4.665556in}{1.174034in}}%
\pgfpathlineto{\pgfqpoint{4.680352in}{1.175513in}}%
\pgfpathlineto{\pgfqpoint{4.688613in}{1.191232in}}%
\pgfpathlineto{\pgfqpoint{4.696871in}{1.207010in}}%
\pgfpathlineto{\pgfqpoint{4.705126in}{1.222842in}}%
\pgfpathlineto{\pgfqpoint{4.713376in}{1.238722in}}%
\pgfpathlineto{\pgfqpoint{4.698580in}{1.236835in}}%
\pgfpathlineto{\pgfqpoint{4.683795in}{1.235016in}}%
\pgfpathlineto{\pgfqpoint{4.669021in}{1.233266in}}%
\pgfpathlineto{\pgfqpoint{4.654259in}{1.231585in}}%
\pgfpathlineto{\pgfqpoint{4.646009in}{1.216105in}}%
\pgfpathlineto{\pgfqpoint{4.637755in}{1.200678in}}%
\pgfpathlineto{\pgfqpoint{4.629498in}{1.185310in}}%
\pgfpathlineto{\pgfqpoint{4.621237in}{1.170007in}}%
\pgfpathclose%
\pgfusepath{fill}%
\end{pgfscope}%
\begin{pgfscope}%
\pgfpathrectangle{\pgfqpoint{1.150000in}{0.150000in}}{\pgfqpoint{5.700000in}{5.700000in}}%
\pgfusepath{clip}%
\pgfsetbuttcap%
\pgfsetroundjoin%
\definecolor{currentfill}{rgb}{0.267968,0.223549,0.512008}%
\pgfsetfillcolor{currentfill}%
\pgfsetfillopacity{0.700000}%
\pgfsetlinewidth{0.000000pt}%
\definecolor{currentstroke}{rgb}{0.000000,0.000000,0.000000}%
\pgfsetstrokecolor{currentstroke}%
\pgfsetdash{}{0pt}%
\pgfpathmoveto{\pgfqpoint{4.713376in}{1.238722in}}%
\pgfpathlineto{\pgfqpoint{4.728185in}{1.240678in}}%
\pgfpathlineto{\pgfqpoint{4.743005in}{1.242702in}}%
\pgfpathlineto{\pgfqpoint{4.757836in}{1.244796in}}%
\pgfpathlineto{\pgfqpoint{4.766085in}{1.261014in}}%
\pgfpathlineto{\pgfqpoint{4.774330in}{1.277265in}}%
\pgfpathlineto{\pgfqpoint{4.782571in}{1.293542in}}%
\pgfpathlineto{\pgfqpoint{4.790809in}{1.309841in}}%
\pgfpathlineto{\pgfqpoint{4.775975in}{1.307359in}}%
\pgfpathlineto{\pgfqpoint{4.761153in}{1.304947in}}%
\pgfpathlineto{\pgfqpoint{4.746343in}{1.302603in}}%
\pgfpathlineto{\pgfqpoint{4.738107in}{1.286590in}}%
\pgfpathlineto{\pgfqpoint{4.729867in}{1.270601in}}%
\pgfpathlineto{\pgfqpoint{4.721624in}{1.254644in}}%
\pgfpathlineto{\pgfqpoint{4.713376in}{1.238722in}}%
\pgfpathclose%
\pgfusepath{fill}%
\end{pgfscope}%
\begin{pgfscope}%
\pgfpathrectangle{\pgfqpoint{1.150000in}{0.150000in}}{\pgfqpoint{5.700000in}{5.700000in}}%
\pgfusepath{clip}%
\pgfsetbuttcap%
\pgfsetroundjoin%
\definecolor{currentfill}{rgb}{0.278826,0.175490,0.483397}%
\pgfsetfillcolor{currentfill}%
\pgfsetfillopacity{0.700000}%
\pgfsetlinewidth{0.000000pt}%
\definecolor{currentstroke}{rgb}{0.000000,0.000000,0.000000}%
\pgfsetstrokecolor{currentstroke}%
\pgfsetdash{}{0pt}%
\pgfpathmoveto{\pgfqpoint{4.588157in}{1.109564in}}%
\pgfpathlineto{\pgfqpoint{4.602919in}{1.110411in}}%
\pgfpathlineto{\pgfqpoint{4.617692in}{1.111325in}}%
\pgfpathlineto{\pgfqpoint{4.632477in}{1.112308in}}%
\pgfpathlineto{\pgfqpoint{4.647272in}{1.113359in}}%
\pgfpathlineto{\pgfqpoint{4.655547in}{1.128777in}}%
\pgfpathlineto{\pgfqpoint{4.663818in}{1.144280in}}%
\pgfpathlineto{\pgfqpoint{4.672087in}{1.159861in}}%
\pgfpathlineto{\pgfqpoint{4.680352in}{1.175513in}}%
\pgfpathlineto{\pgfqpoint{4.665556in}{1.174034in}}%
\pgfpathlineto{\pgfqpoint{4.650772in}{1.172623in}}%
\pgfpathlineto{\pgfqpoint{4.635999in}{1.171281in}}%
\pgfpathlineto{\pgfqpoint{4.621237in}{1.170007in}}%
\pgfpathlineto{\pgfqpoint{4.612972in}{1.154774in}}%
\pgfpathlineto{\pgfqpoint{4.604704in}{1.139619in}}%
\pgfpathlineto{\pgfqpoint{4.596432in}{1.124547in}}%
\pgfpathlineto{\pgfqpoint{4.588157in}{1.109564in}}%
\pgfpathclose%
\pgfusepath{fill}%
\end{pgfscope}%
\begin{pgfscope}%
\pgfpathrectangle{\pgfqpoint{1.150000in}{0.150000in}}{\pgfqpoint{5.700000in}{5.700000in}}%
\pgfusepath{clip}%
\pgfsetbuttcap%
\pgfsetroundjoin%
\definecolor{currentfill}{rgb}{0.273006,0.204520,0.501721}%
\pgfsetfillcolor{currentfill}%
\pgfsetfillopacity{0.700000}%
\pgfsetlinewidth{0.000000pt}%
\definecolor{currentstroke}{rgb}{0.000000,0.000000,0.000000}%
\pgfsetstrokecolor{currentstroke}%
\pgfsetdash{}{0pt}%
\pgfpathmoveto{\pgfqpoint{4.680352in}{1.175513in}}%
\pgfpathlineto{\pgfqpoint{4.695159in}{1.177061in}}%
\pgfpathlineto{\pgfqpoint{4.709977in}{1.178677in}}%
\pgfpathlineto{\pgfqpoint{4.724807in}{1.180361in}}%
\pgfpathlineto{\pgfqpoint{4.733069in}{1.196392in}}%
\pgfpathlineto{\pgfqpoint{4.741329in}{1.212479in}}%
\pgfpathlineto{\pgfqpoint{4.749584in}{1.228615in}}%
\pgfpathlineto{\pgfqpoint{4.757836in}{1.244796in}}%
\pgfpathlineto{\pgfqpoint{4.743005in}{1.242702in}}%
\pgfpathlineto{\pgfqpoint{4.728185in}{1.240678in}}%
\pgfpathlineto{\pgfqpoint{4.713376in}{1.238722in}}%
\pgfpathlineto{\pgfqpoint{4.705126in}{1.222842in}}%
\pgfpathlineto{\pgfqpoint{4.696871in}{1.207010in}}%
\pgfpathlineto{\pgfqpoint{4.688613in}{1.191232in}}%
\pgfpathlineto{\pgfqpoint{4.680352in}{1.175513in}}%
\pgfpathclose%
\pgfusepath{fill}%
\end{pgfscope}%
\begin{pgfscope}%
\pgfpathrectangle{\pgfqpoint{1.150000in}{0.150000in}}{\pgfqpoint{5.700000in}{5.700000in}}%
\pgfusepath{clip}%
\pgfsetbuttcap%
\pgfsetroundjoin%
\definecolor{currentfill}{rgb}{0.277134,0.185228,0.489898}%
\pgfsetfillcolor{currentfill}%
\pgfsetfillopacity{0.700000}%
\pgfsetlinewidth{0.000000pt}%
\definecolor{currentstroke}{rgb}{0.000000,0.000000,0.000000}%
\pgfsetstrokecolor{currentstroke}%
\pgfsetdash{}{0pt}%
\pgfpathmoveto{\pgfqpoint{4.647272in}{1.113359in}}%
\pgfpathlineto{\pgfqpoint{4.662078in}{1.114477in}}%
\pgfpathlineto{\pgfqpoint{4.676895in}{1.115664in}}%
\pgfpathlineto{\pgfqpoint{4.691724in}{1.116919in}}%
\pgfpathlineto{\pgfqpoint{4.699999in}{1.132665in}}%
\pgfpathlineto{\pgfqpoint{4.708272in}{1.148491in}}%
\pgfpathlineto{\pgfqpoint{4.716541in}{1.164392in}}%
\pgfpathlineto{\pgfqpoint{4.724807in}{1.180361in}}%
\pgfpathlineto{\pgfqpoint{4.709977in}{1.178677in}}%
\pgfpathlineto{\pgfqpoint{4.695159in}{1.177061in}}%
\pgfpathlineto{\pgfqpoint{4.680352in}{1.175513in}}%
\pgfpathlineto{\pgfqpoint{4.672087in}{1.159861in}}%
\pgfpathlineto{\pgfqpoint{4.663818in}{1.144280in}}%
\pgfpathlineto{\pgfqpoint{4.655547in}{1.128777in}}%
\pgfpathlineto{\pgfqpoint{4.647272in}{1.113359in}}%
\pgfpathclose%
\pgfusepath{fill}%
\end{pgfscope}%
\end{pgfpicture}%
\makeatother%
\endgroup%
}
    \caption{Your figure caption}
    \label{fig:newton_surface}
\end{figure}
\newpage



\subsection{Metóda združených gradientov (MSG)}

Táto metóda, v skriptách označovaná ako MSG, bola pôvodne navrhnutá pre riešenie sústav lineárnych rovníc s pozitívne definitnou maticou (čo zodpovedá minimalizácii kvadratickej funkcie). Pre kvadratické funkcie v $\mathbb{R}^n$ nájde MSG presné minimum najviac v $n$ krokoch.

Pre všeobecné nekvadratické funkcie (náš prípad) sa metóda modifikuje tak, že nový smer hľadania $h_k$ konštruujeme ako lineárnu kombináciu aktuálneho antigradientu a predchádzajúceho smeru:
$$ h_k = - \nabla f(x^{[k]}) + \beta_{k-1} h_{k-1}, $$
kde pre prvý krok volíme $h_0 = -\nabla f(x^{[0]})$.
Koeficient $\beta_{k-1}$ zabezpečuje, aby boli smery "združené". V našej analýze využijeme \textit{Fletcherov-Reevesov} vzorec:
$$ \beta_{k-1}^{FR} = \frac{\|\nabla f(x^{[k]})\|^2}{\|\nabla f(x^{[k-1]})\|^2}. $$
Metóda sa pri nekvadratických funkciách zvyčajne po $n$ krokoch „reštartuje“ (položí sa $\beta = 0$), aby sa eliminovali kumulované chyby a zachovala konvergencia.

\subsubsection{Analýza metódy pri rôznych hodnotách počiatočnej aproximácie}

Porovnanie vykonáme pre rovnaké počiatočné body ako pri Newtonovej metóde. Očakávame, že MSG bude potrebovať viac krokov než Newtonova metóda, ale menej než metóda najväčšieho spádu.

\vspace{0.5cm}
\noindent \textbf{Počiatočný bod} $x^{[0]} = [0; 0]$ \\
Z tohto bodu sa metóda potrebuje "vymotať" z údolia.

\begin{table}[h!]
    \centering
    \begin{tabular}{cccc}
        \toprule
        \textbf{Iterácia} & \textbf{Bod } $x^{[k]} = [x;y]$ & \textbf{Hodnota } $f(x^{[k]})$ & \textbf{Norma } $\|\nabla f\|$ \\
        \midrule
        0  & $[0.000000;\; 0.000000]$   & 2.000000 & -- \\
        1  & $[0.000000;\; -0.500000]$  & 1.669031 & 1.000000 \\
        2  & $[0.250000;\; -0.553265]$  & 1.607017 & 0.511223 \\
        \dots & \dots & \dots & \dots \\
        13 & $[0.388215;\; -0.741333]$  & 1.568997 & 0.015403 \\
        14 & $[0.388461;\; -0.740832]$  & 1.568997 & 0.002233 \\
        15 & $[0.388004;\; -0.740889]$  & 1.568996 & 0.001887 \\
        \bottomrule
    \end{tabular}
    \caption{Priebeh MSG pre $x^{[0]} = [0;0]$.}
\end{table}


\noindent \textbf{Počiatočný bod} $x^{[0]} = [1.5; 0.5]$ \\
V tomto prípade je funkcia strmšia (vplyvom exponenciály), čo môže ovplyvniť veľkosť kroku v prvej iterácii.

\begin{table}[h!]
    \centering
    \begin{tabular}{cccc}
        \toprule
        \textbf{Iterácia} & \textbf{Bod } $x^{[k]} = [x;y]$ & \textbf{Hodnota } $f(x^{[k]})$ & \textbf{Norma } $\|\nabla f\|$ \\
        \midrule
        0  & $[1.500000;\; 0.500000]$   & 4.934351 & -- \\
        1  & $[-0.138435;\; 0.925639]$  & 4.524030 & 3.385639 \\
        2  & $[0.396305;\; -0.626467]$  & 1.603598 & 6.566560 \\
        \dots & \dots & \dots & \dots \\
        9  & $[0.387814;\; -0.741050]$  & 1.568997 & 0.015270 \\
        10 & $[0.388119;\; -0.740641]$  & 1.568997 & 0.002043 \\
        11 & $[0.388183;\; -0.740998]$  & 1.568996 & 0.001725 \\
        \bottomrule
    \end{tabular}
    \caption{Priebeh MSG pre $x^{[0]} = [1.5;0.5]$.}
\end{table}


\noindent Z výsledkov vidíme, že MSG konverguje pomalšie než Newtonova metóda (ktorá potrebovala cca 3-4 kroky), ale nevyžaduje výpočet inverznej Hessovej matice, čo je jej hlavná výhoda pri úlohách s veľkým počtom premenných. Následuje graf průběhu.

\begin{figure}[H]
    \centering
    \resizebox{0.8\textwidth}{!}{
    %% Creator: Matplotlib, PGF backend
%%
%% To include the figure in your LaTeX document, write
%%   \input{<filename>.pgf}
%%
%% Make sure the required packages are loaded in your preamble
%%   \usepackage{pgf}
%%
%% Also ensure that all the required font packages are loaded; for instance,
%% the lmodern package is sometimes necessary when using math font.
%%   \usepackage{lmodern}
%%
%% Figures using additional raster images can only be included by \input if
%% they are in the same directory as the main LaTeX file. For loading figures
%% from other directories you can use the `import` package
%%   \usepackage{import}
%%
%% and then include the figures with
%%   \import{<path to file>}{<filename>.pgf}
%%
%% Matplotlib used the following preamble
%%   
%%   \usepackage{fontspec}
%%   \setmainfont{DejaVuSerif.ttf}[Path=\detokenize{/home/radimek/Documents/projekt_mat_prog/mat_prog_kernel/lib/python3.12/site-packages/matplotlib/mpl-data/fonts/ttf/}]
%%   \setsansfont{DejaVuSans.ttf}[Path=\detokenize{/home/radimek/Documents/projekt_mat_prog/mat_prog_kernel/lib/python3.12/site-packages/matplotlib/mpl-data/fonts/ttf/}]
%%   \setmonofont{DejaVuSansMono.ttf}[Path=\detokenize{/home/radimek/Documents/projekt_mat_prog/mat_prog_kernel/lib/python3.12/site-packages/matplotlib/mpl-data/fonts/ttf/}]
%%   \makeatletter\@ifpackageloaded{underscore}{}{\usepackage[strings]{underscore}}\makeatother
%%
\begingroup%
\makeatletter%
\begin{pgfpicture}%
\pgfpathrectangle{\pgfpointorigin}{\pgfqpoint{7.000000in}{6.000000in}}%
\pgfusepath{use as bounding box, clip}%
\begin{pgfscope}%
\pgfsetbuttcap%
\pgfsetmiterjoin%
\definecolor{currentfill}{rgb}{1.000000,1.000000,1.000000}%
\pgfsetfillcolor{currentfill}%
\pgfsetlinewidth{0.000000pt}%
\definecolor{currentstroke}{rgb}{1.000000,1.000000,1.000000}%
\pgfsetstrokecolor{currentstroke}%
\pgfsetdash{}{0pt}%
\pgfpathmoveto{\pgfqpoint{0.000000in}{0.000000in}}%
\pgfpathlineto{\pgfqpoint{7.000000in}{0.000000in}}%
\pgfpathlineto{\pgfqpoint{7.000000in}{6.000000in}}%
\pgfpathlineto{\pgfqpoint{0.000000in}{6.000000in}}%
\pgfpathlineto{\pgfqpoint{0.000000in}{0.000000in}}%
\pgfpathclose%
\pgfusepath{fill}%
\end{pgfscope}%
\begin{pgfscope}%
\pgfsetbuttcap%
\pgfsetmiterjoin%
\definecolor{currentfill}{rgb}{1.000000,1.000000,1.000000}%
\pgfsetfillcolor{currentfill}%
\pgfsetlinewidth{0.000000pt}%
\definecolor{currentstroke}{rgb}{0.000000,0.000000,0.000000}%
\pgfsetstrokecolor{currentstroke}%
\pgfsetstrokeopacity{0.000000}%
\pgfsetdash{}{0pt}%
\pgfpathmoveto{\pgfqpoint{0.854460in}{0.571603in}}%
\pgfpathlineto{\pgfqpoint{6.739560in}{0.571603in}}%
\pgfpathlineto{\pgfqpoint{6.739560in}{5.640039in}}%
\pgfpathlineto{\pgfqpoint{0.854460in}{5.640039in}}%
\pgfpathlineto{\pgfqpoint{0.854460in}{0.571603in}}%
\pgfpathclose%
\pgfusepath{fill}%
\end{pgfscope}%
\begin{pgfscope}%
\pgfpathrectangle{\pgfqpoint{0.854460in}{0.571603in}}{\pgfqpoint{5.885100in}{5.068436in}}%
\pgfusepath{clip}%
\pgfsetbuttcap%
\pgfsetroundjoin%
\definecolor{currentfill}{rgb}{0.000000,0.000000,1.000000}%
\pgfsetfillcolor{currentfill}%
\pgfsetlinewidth{1.003750pt}%
\definecolor{currentstroke}{rgb}{0.000000,0.000000,1.000000}%
\pgfsetstrokecolor{currentstroke}%
\pgfsetdash{}{0pt}%
\pgfsys@defobject{currentmarker}{\pgfqpoint{-0.069444in}{-0.069444in}}{\pgfqpoint{0.069444in}{0.069444in}}{%
\pgfpathmoveto{\pgfqpoint{0.000000in}{-0.069444in}}%
\pgfpathcurveto{\pgfqpoint{0.018417in}{-0.069444in}}{\pgfqpoint{0.036082in}{-0.062127in}}{\pgfqpoint{0.049105in}{-0.049105in}}%
\pgfpathcurveto{\pgfqpoint{0.062127in}{-0.036082in}}{\pgfqpoint{0.069444in}{-0.018417in}}{\pgfqpoint{0.069444in}{0.000000in}}%
\pgfpathcurveto{\pgfqpoint{0.069444in}{0.018417in}}{\pgfqpoint{0.062127in}{0.036082in}}{\pgfqpoint{0.049105in}{0.049105in}}%
\pgfpathcurveto{\pgfqpoint{0.036082in}{0.062127in}}{\pgfqpoint{0.018417in}{0.069444in}}{\pgfqpoint{0.000000in}{0.069444in}}%
\pgfpathcurveto{\pgfqpoint{-0.018417in}{0.069444in}}{\pgfqpoint{-0.036082in}{0.062127in}}{\pgfqpoint{-0.049105in}{0.049105in}}%
\pgfpathcurveto{\pgfqpoint{-0.062127in}{0.036082in}}{\pgfqpoint{-0.069444in}{0.018417in}}{\pgfqpoint{-0.069444in}{0.000000in}}%
\pgfpathcurveto{\pgfqpoint{-0.069444in}{-0.018417in}}{\pgfqpoint{-0.062127in}{-0.036082in}}{\pgfqpoint{-0.049105in}{-0.049105in}}%
\pgfpathcurveto{\pgfqpoint{-0.036082in}{-0.062127in}}{\pgfqpoint{-0.018417in}{-0.069444in}}{\pgfqpoint{0.000000in}{-0.069444in}}%
\pgfpathlineto{\pgfqpoint{0.000000in}{-0.069444in}}%
\pgfpathclose%
\pgfusepath{stroke,fill}%
}%
\begin{pgfscope}%
\pgfsys@transformshift{3.577658in}{1.227971in}%
\pgfsys@useobject{currentmarker}{}%
\end{pgfscope}%
\end{pgfscope}%
\begin{pgfscope}%
\pgfsetbuttcap%
\pgfsetroundjoin%
\definecolor{currentfill}{rgb}{0.000000,0.000000,0.000000}%
\pgfsetfillcolor{currentfill}%
\pgfsetlinewidth{0.803000pt}%
\definecolor{currentstroke}{rgb}{0.000000,0.000000,0.000000}%
\pgfsetstrokecolor{currentstroke}%
\pgfsetdash{}{0pt}%
\pgfsys@defobject{currentmarker}{\pgfqpoint{0.000000in}{-0.048611in}}{\pgfqpoint{0.000000in}{0.000000in}}{%
\pgfpathmoveto{\pgfqpoint{0.000000in}{0.000000in}}%
\pgfpathlineto{\pgfqpoint{0.000000in}{-0.048611in}}%
\pgfusepath{stroke,fill}%
}%
\begin{pgfscope}%
\pgfsys@transformshift{0.854460in}{0.571603in}%
\pgfsys@useobject{currentmarker}{}%
\end{pgfscope}%
\end{pgfscope}%
\begin{pgfscope}%
\definecolor{textcolor}{rgb}{0.000000,0.000000,0.000000}%
\pgfsetstrokecolor{textcolor}%
\pgfsetfillcolor{textcolor}%
\pgftext[x=0.854460in,y=0.474381in,,top]{\color{textcolor}\sffamily\fontsize{10.000000}{12.000000}\selectfont \ensuremath{-}1.0}%
\end{pgfscope}%
\begin{pgfscope}%
\pgfsetbuttcap%
\pgfsetroundjoin%
\definecolor{currentfill}{rgb}{0.000000,0.000000,0.000000}%
\pgfsetfillcolor{currentfill}%
\pgfsetlinewidth{0.803000pt}%
\definecolor{currentstroke}{rgb}{0.000000,0.000000,0.000000}%
\pgfsetstrokecolor{currentstroke}%
\pgfsetdash{}{0pt}%
\pgfsys@defobject{currentmarker}{\pgfqpoint{0.000000in}{-0.048611in}}{\pgfqpoint{0.000000in}{0.000000in}}{%
\pgfpathmoveto{\pgfqpoint{0.000000in}{0.000000in}}%
\pgfpathlineto{\pgfqpoint{0.000000in}{-0.048611in}}%
\pgfusepath{stroke,fill}%
}%
\begin{pgfscope}%
\pgfsys@transformshift{1.835310in}{0.571603in}%
\pgfsys@useobject{currentmarker}{}%
\end{pgfscope}%
\end{pgfscope}%
\begin{pgfscope}%
\definecolor{textcolor}{rgb}{0.000000,0.000000,0.000000}%
\pgfsetstrokecolor{textcolor}%
\pgfsetfillcolor{textcolor}%
\pgftext[x=1.835310in,y=0.474381in,,top]{\color{textcolor}\sffamily\fontsize{10.000000}{12.000000}\selectfont \ensuremath{-}0.5}%
\end{pgfscope}%
\begin{pgfscope}%
\pgfsetbuttcap%
\pgfsetroundjoin%
\definecolor{currentfill}{rgb}{0.000000,0.000000,0.000000}%
\pgfsetfillcolor{currentfill}%
\pgfsetlinewidth{0.803000pt}%
\definecolor{currentstroke}{rgb}{0.000000,0.000000,0.000000}%
\pgfsetstrokecolor{currentstroke}%
\pgfsetdash{}{0pt}%
\pgfsys@defobject{currentmarker}{\pgfqpoint{0.000000in}{-0.048611in}}{\pgfqpoint{0.000000in}{0.000000in}}{%
\pgfpathmoveto{\pgfqpoint{0.000000in}{0.000000in}}%
\pgfpathlineto{\pgfqpoint{0.000000in}{-0.048611in}}%
\pgfusepath{stroke,fill}%
}%
\begin{pgfscope}%
\pgfsys@transformshift{2.816160in}{0.571603in}%
\pgfsys@useobject{currentmarker}{}%
\end{pgfscope}%
\end{pgfscope}%
\begin{pgfscope}%
\definecolor{textcolor}{rgb}{0.000000,0.000000,0.000000}%
\pgfsetstrokecolor{textcolor}%
\pgfsetfillcolor{textcolor}%
\pgftext[x=2.816160in,y=0.474381in,,top]{\color{textcolor}\sffamily\fontsize{10.000000}{12.000000}\selectfont 0.0}%
\end{pgfscope}%
\begin{pgfscope}%
\pgfsetbuttcap%
\pgfsetroundjoin%
\definecolor{currentfill}{rgb}{0.000000,0.000000,0.000000}%
\pgfsetfillcolor{currentfill}%
\pgfsetlinewidth{0.803000pt}%
\definecolor{currentstroke}{rgb}{0.000000,0.000000,0.000000}%
\pgfsetstrokecolor{currentstroke}%
\pgfsetdash{}{0pt}%
\pgfsys@defobject{currentmarker}{\pgfqpoint{0.000000in}{-0.048611in}}{\pgfqpoint{0.000000in}{0.000000in}}{%
\pgfpathmoveto{\pgfqpoint{0.000000in}{0.000000in}}%
\pgfpathlineto{\pgfqpoint{0.000000in}{-0.048611in}}%
\pgfusepath{stroke,fill}%
}%
\begin{pgfscope}%
\pgfsys@transformshift{3.797010in}{0.571603in}%
\pgfsys@useobject{currentmarker}{}%
\end{pgfscope}%
\end{pgfscope}%
\begin{pgfscope}%
\definecolor{textcolor}{rgb}{0.000000,0.000000,0.000000}%
\pgfsetstrokecolor{textcolor}%
\pgfsetfillcolor{textcolor}%
\pgftext[x=3.797010in,y=0.474381in,,top]{\color{textcolor}\sffamily\fontsize{10.000000}{12.000000}\selectfont 0.5}%
\end{pgfscope}%
\begin{pgfscope}%
\pgfsetbuttcap%
\pgfsetroundjoin%
\definecolor{currentfill}{rgb}{0.000000,0.000000,0.000000}%
\pgfsetfillcolor{currentfill}%
\pgfsetlinewidth{0.803000pt}%
\definecolor{currentstroke}{rgb}{0.000000,0.000000,0.000000}%
\pgfsetstrokecolor{currentstroke}%
\pgfsetdash{}{0pt}%
\pgfsys@defobject{currentmarker}{\pgfqpoint{0.000000in}{-0.048611in}}{\pgfqpoint{0.000000in}{0.000000in}}{%
\pgfpathmoveto{\pgfqpoint{0.000000in}{0.000000in}}%
\pgfpathlineto{\pgfqpoint{0.000000in}{-0.048611in}}%
\pgfusepath{stroke,fill}%
}%
\begin{pgfscope}%
\pgfsys@transformshift{4.777860in}{0.571603in}%
\pgfsys@useobject{currentmarker}{}%
\end{pgfscope}%
\end{pgfscope}%
\begin{pgfscope}%
\definecolor{textcolor}{rgb}{0.000000,0.000000,0.000000}%
\pgfsetstrokecolor{textcolor}%
\pgfsetfillcolor{textcolor}%
\pgftext[x=4.777860in,y=0.474381in,,top]{\color{textcolor}\sffamily\fontsize{10.000000}{12.000000}\selectfont 1.0}%
\end{pgfscope}%
\begin{pgfscope}%
\pgfsetbuttcap%
\pgfsetroundjoin%
\definecolor{currentfill}{rgb}{0.000000,0.000000,0.000000}%
\pgfsetfillcolor{currentfill}%
\pgfsetlinewidth{0.803000pt}%
\definecolor{currentstroke}{rgb}{0.000000,0.000000,0.000000}%
\pgfsetstrokecolor{currentstroke}%
\pgfsetdash{}{0pt}%
\pgfsys@defobject{currentmarker}{\pgfqpoint{0.000000in}{-0.048611in}}{\pgfqpoint{0.000000in}{0.000000in}}{%
\pgfpathmoveto{\pgfqpoint{0.000000in}{0.000000in}}%
\pgfpathlineto{\pgfqpoint{0.000000in}{-0.048611in}}%
\pgfusepath{stroke,fill}%
}%
\begin{pgfscope}%
\pgfsys@transformshift{5.758710in}{0.571603in}%
\pgfsys@useobject{currentmarker}{}%
\end{pgfscope}%
\end{pgfscope}%
\begin{pgfscope}%
\definecolor{textcolor}{rgb}{0.000000,0.000000,0.000000}%
\pgfsetstrokecolor{textcolor}%
\pgfsetfillcolor{textcolor}%
\pgftext[x=5.758710in,y=0.474381in,,top]{\color{textcolor}\sffamily\fontsize{10.000000}{12.000000}\selectfont 1.5}%
\end{pgfscope}%
\begin{pgfscope}%
\pgfsetbuttcap%
\pgfsetroundjoin%
\definecolor{currentfill}{rgb}{0.000000,0.000000,0.000000}%
\pgfsetfillcolor{currentfill}%
\pgfsetlinewidth{0.803000pt}%
\definecolor{currentstroke}{rgb}{0.000000,0.000000,0.000000}%
\pgfsetstrokecolor{currentstroke}%
\pgfsetdash{}{0pt}%
\pgfsys@defobject{currentmarker}{\pgfqpoint{0.000000in}{-0.048611in}}{\pgfqpoint{0.000000in}{0.000000in}}{%
\pgfpathmoveto{\pgfqpoint{0.000000in}{0.000000in}}%
\pgfpathlineto{\pgfqpoint{0.000000in}{-0.048611in}}%
\pgfusepath{stroke,fill}%
}%
\begin{pgfscope}%
\pgfsys@transformshift{6.739560in}{0.571603in}%
\pgfsys@useobject{currentmarker}{}%
\end{pgfscope}%
\end{pgfscope}%
\begin{pgfscope}%
\definecolor{textcolor}{rgb}{0.000000,0.000000,0.000000}%
\pgfsetstrokecolor{textcolor}%
\pgfsetfillcolor{textcolor}%
\pgftext[x=6.739560in,y=0.474381in,,top]{\color{textcolor}\sffamily\fontsize{10.000000}{12.000000}\selectfont 2.0}%
\end{pgfscope}%
\begin{pgfscope}%
\definecolor{textcolor}{rgb}{0.000000,0.000000,0.000000}%
\pgfsetstrokecolor{textcolor}%
\pgfsetfillcolor{textcolor}%
\pgftext[x=3.797010in,y=0.284413in,,top]{\color{textcolor}\sffamily\fontsize{10.000000}{12.000000}\selectfont x}%
\end{pgfscope}%
\begin{pgfscope}%
\pgfsetbuttcap%
\pgfsetroundjoin%
\definecolor{currentfill}{rgb}{0.000000,0.000000,0.000000}%
\pgfsetfillcolor{currentfill}%
\pgfsetlinewidth{0.803000pt}%
\definecolor{currentstroke}{rgb}{0.000000,0.000000,0.000000}%
\pgfsetstrokecolor{currentstroke}%
\pgfsetdash{}{0pt}%
\pgfsys@defobject{currentmarker}{\pgfqpoint{-0.048611in}{0.000000in}}{\pgfqpoint{-0.000000in}{0.000000in}}{%
\pgfpathmoveto{\pgfqpoint{-0.000000in}{0.000000in}}%
\pgfpathlineto{\pgfqpoint{-0.048611in}{0.000000in}}%
\pgfusepath{stroke,fill}%
}%
\begin{pgfscope}%
\pgfsys@transformshift{0.854460in}{0.571603in}%
\pgfsys@useobject{currentmarker}{}%
\end{pgfscope}%
\end{pgfscope}%
\begin{pgfscope}%
\definecolor{textcolor}{rgb}{0.000000,0.000000,0.000000}%
\pgfsetstrokecolor{textcolor}%
\pgfsetfillcolor{textcolor}%
\pgftext[x=0.339968in, y=0.518842in, left, base]{\color{textcolor}\sffamily\fontsize{10.000000}{12.000000}\selectfont \ensuremath{-}1.00}%
\end{pgfscope}%
\begin{pgfscope}%
\pgfsetbuttcap%
\pgfsetroundjoin%
\definecolor{currentfill}{rgb}{0.000000,0.000000,0.000000}%
\pgfsetfillcolor{currentfill}%
\pgfsetlinewidth{0.803000pt}%
\definecolor{currentstroke}{rgb}{0.000000,0.000000,0.000000}%
\pgfsetstrokecolor{currentstroke}%
\pgfsetdash{}{0pt}%
\pgfsys@defobject{currentmarker}{\pgfqpoint{-0.048611in}{0.000000in}}{\pgfqpoint{-0.000000in}{0.000000in}}{%
\pgfpathmoveto{\pgfqpoint{-0.000000in}{0.000000in}}%
\pgfpathlineto{\pgfqpoint{-0.048611in}{0.000000in}}%
\pgfusepath{stroke,fill}%
}%
\begin{pgfscope}%
\pgfsys@transformshift{0.854460in}{1.205158in}%
\pgfsys@useobject{currentmarker}{}%
\end{pgfscope}%
\end{pgfscope}%
\begin{pgfscope}%
\definecolor{textcolor}{rgb}{0.000000,0.000000,0.000000}%
\pgfsetstrokecolor{textcolor}%
\pgfsetfillcolor{textcolor}%
\pgftext[x=0.339968in, y=1.152396in, left, base]{\color{textcolor}\sffamily\fontsize{10.000000}{12.000000}\selectfont \ensuremath{-}0.75}%
\end{pgfscope}%
\begin{pgfscope}%
\pgfsetbuttcap%
\pgfsetroundjoin%
\definecolor{currentfill}{rgb}{0.000000,0.000000,0.000000}%
\pgfsetfillcolor{currentfill}%
\pgfsetlinewidth{0.803000pt}%
\definecolor{currentstroke}{rgb}{0.000000,0.000000,0.000000}%
\pgfsetstrokecolor{currentstroke}%
\pgfsetdash{}{0pt}%
\pgfsys@defobject{currentmarker}{\pgfqpoint{-0.048611in}{0.000000in}}{\pgfqpoint{-0.000000in}{0.000000in}}{%
\pgfpathmoveto{\pgfqpoint{-0.000000in}{0.000000in}}%
\pgfpathlineto{\pgfqpoint{-0.048611in}{0.000000in}}%
\pgfusepath{stroke,fill}%
}%
\begin{pgfscope}%
\pgfsys@transformshift{0.854460in}{1.838712in}%
\pgfsys@useobject{currentmarker}{}%
\end{pgfscope}%
\end{pgfscope}%
\begin{pgfscope}%
\definecolor{textcolor}{rgb}{0.000000,0.000000,0.000000}%
\pgfsetstrokecolor{textcolor}%
\pgfsetfillcolor{textcolor}%
\pgftext[x=0.339968in, y=1.785951in, left, base]{\color{textcolor}\sffamily\fontsize{10.000000}{12.000000}\selectfont \ensuremath{-}0.50}%
\end{pgfscope}%
\begin{pgfscope}%
\pgfsetbuttcap%
\pgfsetroundjoin%
\definecolor{currentfill}{rgb}{0.000000,0.000000,0.000000}%
\pgfsetfillcolor{currentfill}%
\pgfsetlinewidth{0.803000pt}%
\definecolor{currentstroke}{rgb}{0.000000,0.000000,0.000000}%
\pgfsetstrokecolor{currentstroke}%
\pgfsetdash{}{0pt}%
\pgfsys@defobject{currentmarker}{\pgfqpoint{-0.048611in}{0.000000in}}{\pgfqpoint{-0.000000in}{0.000000in}}{%
\pgfpathmoveto{\pgfqpoint{-0.000000in}{0.000000in}}%
\pgfpathlineto{\pgfqpoint{-0.048611in}{0.000000in}}%
\pgfusepath{stroke,fill}%
}%
\begin{pgfscope}%
\pgfsys@transformshift{0.854460in}{2.472267in}%
\pgfsys@useobject{currentmarker}{}%
\end{pgfscope}%
\end{pgfscope}%
\begin{pgfscope}%
\definecolor{textcolor}{rgb}{0.000000,0.000000,0.000000}%
\pgfsetstrokecolor{textcolor}%
\pgfsetfillcolor{textcolor}%
\pgftext[x=0.339968in, y=2.419505in, left, base]{\color{textcolor}\sffamily\fontsize{10.000000}{12.000000}\selectfont \ensuremath{-}0.25}%
\end{pgfscope}%
\begin{pgfscope}%
\pgfsetbuttcap%
\pgfsetroundjoin%
\definecolor{currentfill}{rgb}{0.000000,0.000000,0.000000}%
\pgfsetfillcolor{currentfill}%
\pgfsetlinewidth{0.803000pt}%
\definecolor{currentstroke}{rgb}{0.000000,0.000000,0.000000}%
\pgfsetstrokecolor{currentstroke}%
\pgfsetdash{}{0pt}%
\pgfsys@defobject{currentmarker}{\pgfqpoint{-0.048611in}{0.000000in}}{\pgfqpoint{-0.000000in}{0.000000in}}{%
\pgfpathmoveto{\pgfqpoint{-0.000000in}{0.000000in}}%
\pgfpathlineto{\pgfqpoint{-0.048611in}{0.000000in}}%
\pgfusepath{stroke,fill}%
}%
\begin{pgfscope}%
\pgfsys@transformshift{0.854460in}{3.105821in}%
\pgfsys@useobject{currentmarker}{}%
\end{pgfscope}%
\end{pgfscope}%
\begin{pgfscope}%
\definecolor{textcolor}{rgb}{0.000000,0.000000,0.000000}%
\pgfsetstrokecolor{textcolor}%
\pgfsetfillcolor{textcolor}%
\pgftext[x=0.447993in, y=3.053060in, left, base]{\color{textcolor}\sffamily\fontsize{10.000000}{12.000000}\selectfont 0.00}%
\end{pgfscope}%
\begin{pgfscope}%
\pgfsetbuttcap%
\pgfsetroundjoin%
\definecolor{currentfill}{rgb}{0.000000,0.000000,0.000000}%
\pgfsetfillcolor{currentfill}%
\pgfsetlinewidth{0.803000pt}%
\definecolor{currentstroke}{rgb}{0.000000,0.000000,0.000000}%
\pgfsetstrokecolor{currentstroke}%
\pgfsetdash{}{0pt}%
\pgfsys@defobject{currentmarker}{\pgfqpoint{-0.048611in}{0.000000in}}{\pgfqpoint{-0.000000in}{0.000000in}}{%
\pgfpathmoveto{\pgfqpoint{-0.000000in}{0.000000in}}%
\pgfpathlineto{\pgfqpoint{-0.048611in}{0.000000in}}%
\pgfusepath{stroke,fill}%
}%
\begin{pgfscope}%
\pgfsys@transformshift{0.854460in}{3.739376in}%
\pgfsys@useobject{currentmarker}{}%
\end{pgfscope}%
\end{pgfscope}%
\begin{pgfscope}%
\definecolor{textcolor}{rgb}{0.000000,0.000000,0.000000}%
\pgfsetstrokecolor{textcolor}%
\pgfsetfillcolor{textcolor}%
\pgftext[x=0.447993in, y=3.686614in, left, base]{\color{textcolor}\sffamily\fontsize{10.000000}{12.000000}\selectfont 0.25}%
\end{pgfscope}%
\begin{pgfscope}%
\pgfsetbuttcap%
\pgfsetroundjoin%
\definecolor{currentfill}{rgb}{0.000000,0.000000,0.000000}%
\pgfsetfillcolor{currentfill}%
\pgfsetlinewidth{0.803000pt}%
\definecolor{currentstroke}{rgb}{0.000000,0.000000,0.000000}%
\pgfsetstrokecolor{currentstroke}%
\pgfsetdash{}{0pt}%
\pgfsys@defobject{currentmarker}{\pgfqpoint{-0.048611in}{0.000000in}}{\pgfqpoint{-0.000000in}{0.000000in}}{%
\pgfpathmoveto{\pgfqpoint{-0.000000in}{0.000000in}}%
\pgfpathlineto{\pgfqpoint{-0.048611in}{0.000000in}}%
\pgfusepath{stroke,fill}%
}%
\begin{pgfscope}%
\pgfsys@transformshift{0.854460in}{4.372930in}%
\pgfsys@useobject{currentmarker}{}%
\end{pgfscope}%
\end{pgfscope}%
\begin{pgfscope}%
\definecolor{textcolor}{rgb}{0.000000,0.000000,0.000000}%
\pgfsetstrokecolor{textcolor}%
\pgfsetfillcolor{textcolor}%
\pgftext[x=0.447993in, y=4.320169in, left, base]{\color{textcolor}\sffamily\fontsize{10.000000}{12.000000}\selectfont 0.50}%
\end{pgfscope}%
\begin{pgfscope}%
\pgfsetbuttcap%
\pgfsetroundjoin%
\definecolor{currentfill}{rgb}{0.000000,0.000000,0.000000}%
\pgfsetfillcolor{currentfill}%
\pgfsetlinewidth{0.803000pt}%
\definecolor{currentstroke}{rgb}{0.000000,0.000000,0.000000}%
\pgfsetstrokecolor{currentstroke}%
\pgfsetdash{}{0pt}%
\pgfsys@defobject{currentmarker}{\pgfqpoint{-0.048611in}{0.000000in}}{\pgfqpoint{-0.000000in}{0.000000in}}{%
\pgfpathmoveto{\pgfqpoint{-0.000000in}{0.000000in}}%
\pgfpathlineto{\pgfqpoint{-0.048611in}{0.000000in}}%
\pgfusepath{stroke,fill}%
}%
\begin{pgfscope}%
\pgfsys@transformshift{0.854460in}{5.006485in}%
\pgfsys@useobject{currentmarker}{}%
\end{pgfscope}%
\end{pgfscope}%
\begin{pgfscope}%
\definecolor{textcolor}{rgb}{0.000000,0.000000,0.000000}%
\pgfsetstrokecolor{textcolor}%
\pgfsetfillcolor{textcolor}%
\pgftext[x=0.447993in, y=4.953723in, left, base]{\color{textcolor}\sffamily\fontsize{10.000000}{12.000000}\selectfont 0.75}%
\end{pgfscope}%
\begin{pgfscope}%
\pgfsetbuttcap%
\pgfsetroundjoin%
\definecolor{currentfill}{rgb}{0.000000,0.000000,0.000000}%
\pgfsetfillcolor{currentfill}%
\pgfsetlinewidth{0.803000pt}%
\definecolor{currentstroke}{rgb}{0.000000,0.000000,0.000000}%
\pgfsetstrokecolor{currentstroke}%
\pgfsetdash{}{0pt}%
\pgfsys@defobject{currentmarker}{\pgfqpoint{-0.048611in}{0.000000in}}{\pgfqpoint{-0.000000in}{0.000000in}}{%
\pgfpathmoveto{\pgfqpoint{-0.000000in}{0.000000in}}%
\pgfpathlineto{\pgfqpoint{-0.048611in}{0.000000in}}%
\pgfusepath{stroke,fill}%
}%
\begin{pgfscope}%
\pgfsys@transformshift{0.854460in}{5.640039in}%
\pgfsys@useobject{currentmarker}{}%
\end{pgfscope}%
\end{pgfscope}%
\begin{pgfscope}%
\definecolor{textcolor}{rgb}{0.000000,0.000000,0.000000}%
\pgfsetstrokecolor{textcolor}%
\pgfsetfillcolor{textcolor}%
\pgftext[x=0.447993in, y=5.587277in, left, base]{\color{textcolor}\sffamily\fontsize{10.000000}{12.000000}\selectfont 1.00}%
\end{pgfscope}%
\begin{pgfscope}%
\definecolor{textcolor}{rgb}{0.000000,0.000000,0.000000}%
\pgfsetstrokecolor{textcolor}%
\pgfsetfillcolor{textcolor}%
\pgftext[x=0.284413in,y=3.105821in,,bottom,rotate=90.000000]{\color{textcolor}\sffamily\fontsize{10.000000}{12.000000}\selectfont y}%
\end{pgfscope}%
\begin{pgfscope}%
\pgfpathrectangle{\pgfqpoint{0.854460in}{0.571603in}}{\pgfqpoint{5.885100in}{5.068436in}}%
\pgfusepath{clip}%
\pgfsetbuttcap%
\pgfsetroundjoin%
\pgfsetlinewidth{1.505625pt}%
\definecolor{currentstroke}{rgb}{0.273809,0.031497,0.358853}%
\pgfsetstrokecolor{currentstroke}%
\pgfsetdash{}{0pt}%
\pgfpathmoveto{\pgfqpoint{4.137104in}{0.594156in}}%
\pgfpathlineto{\pgfqpoint{4.107531in}{0.600389in}}%
\pgfpathlineto{\pgfqpoint{4.048384in}{0.615614in}}%
\pgfpathlineto{\pgfqpoint{4.018811in}{0.624181in}}%
\pgfpathlineto{\pgfqpoint{3.948066in}{0.648012in}}%
\pgfpathlineto{\pgfqpoint{3.881868in}{0.673481in}}%
\pgfpathlineto{\pgfqpoint{3.822205in}{0.698951in}}%
\pgfpathlineto{\pgfqpoint{3.767335in}{0.724420in}}%
\pgfpathlineto{\pgfqpoint{3.716076in}{0.749890in}}%
\pgfpathlineto{\pgfqpoint{3.634357in}{0.794152in}}%
\pgfpathlineto{\pgfqpoint{3.575210in}{0.828820in}}%
\pgfpathlineto{\pgfqpoint{3.499258in}{0.877238in}}%
\pgfpathlineto{\pgfqpoint{3.456917in}{0.906046in}}%
\pgfpathlineto{\pgfqpoint{3.391535in}{0.953646in}}%
\pgfpathlineto{\pgfqpoint{3.327111in}{1.004585in}}%
\pgfpathlineto{\pgfqpoint{3.267775in}{1.055524in}}%
\pgfpathlineto{\pgfqpoint{3.213147in}{1.106463in}}%
\pgfpathlineto{\pgfqpoint{3.161183in}{1.159266in}}%
\pgfpathlineto{\pgfqpoint{3.117242in}{1.208341in}}%
\pgfpathlineto{\pgfqpoint{3.075310in}{1.259281in}}%
\pgfpathlineto{\pgfqpoint{3.042890in}{1.302771in}}%
\pgfpathlineto{\pgfqpoint{3.020151in}{1.335689in}}%
\pgfpathlineto{\pgfqpoint{2.988135in}{1.386628in}}%
\pgfpathlineto{\pgfqpoint{2.973705in}{1.412098in}}%
\pgfpathlineto{\pgfqpoint{2.947533in}{1.463037in}}%
\pgfpathlineto{\pgfqpoint{2.924596in}{1.515674in}}%
\pgfpathlineto{\pgfqpoint{2.907476in}{1.564915in}}%
\pgfpathlineto{\pgfqpoint{2.893639in}{1.615854in}}%
\pgfpathlineto{\pgfqpoint{2.888732in}{1.641323in}}%
\pgfpathlineto{\pgfqpoint{2.885009in}{1.666793in}}%
\pgfpathlineto{\pgfqpoint{2.882585in}{1.692262in}}%
\pgfpathlineto{\pgfqpoint{2.881588in}{1.717732in}}%
\pgfpathlineto{\pgfqpoint{2.882167in}{1.743202in}}%
\pgfpathlineto{\pgfqpoint{2.884491in}{1.768671in}}%
\pgfpathlineto{\pgfqpoint{2.888754in}{1.794141in}}%
\pgfpathlineto{\pgfqpoint{2.895198in}{1.819610in}}%
\pgfpathlineto{\pgfqpoint{2.905163in}{1.845080in}}%
\pgfpathlineto{\pgfqpoint{2.924596in}{1.880337in}}%
\pgfpathlineto{\pgfqpoint{2.936985in}{1.896019in}}%
\pgfpathlineto{\pgfqpoint{2.954169in}{1.913874in}}%
\pgfpathlineto{\pgfqpoint{2.963745in}{1.921488in}}%
\pgfpathlineto{\pgfqpoint{2.983743in}{1.935053in}}%
\pgfpathlineto{\pgfqpoint{3.013316in}{1.948767in}}%
\pgfpathlineto{\pgfqpoint{3.042890in}{1.956682in}}%
\pgfpathlineto{\pgfqpoint{3.072463in}{1.960424in}}%
\pgfpathlineto{\pgfqpoint{3.102036in}{1.960653in}}%
\pgfpathlineto{\pgfqpoint{3.131610in}{1.957897in}}%
\pgfpathlineto{\pgfqpoint{3.161183in}{1.952588in}}%
\pgfpathlineto{\pgfqpoint{3.190756in}{1.945003in}}%
\pgfpathlineto{\pgfqpoint{3.220330in}{1.935236in}}%
\pgfpathlineto{\pgfqpoint{3.255016in}{1.921488in}}%
\pgfpathlineto{\pgfqpoint{3.279476in}{1.910526in}}%
\pgfpathlineto{\pgfqpoint{3.309050in}{1.895913in}}%
\pgfpathlineto{\pgfqpoint{3.368197in}{1.862272in}}%
\pgfpathlineto{\pgfqpoint{3.427343in}{1.823984in}}%
\pgfpathlineto{\pgfqpoint{3.486490in}{1.781515in}}%
\pgfpathlineto{\pgfqpoint{3.545637in}{1.735571in}}%
\pgfpathlineto{\pgfqpoint{3.604783in}{1.686548in}}%
\pgfpathlineto{\pgfqpoint{3.684793in}{1.615854in}}%
\pgfpathlineto{\pgfqpoint{3.739690in}{1.564915in}}%
\pgfpathlineto{\pgfqpoint{3.792774in}{1.513976in}}%
\pgfpathlineto{\pgfqpoint{3.870944in}{1.436234in}}%
\pgfpathlineto{\pgfqpoint{3.967054in}{1.335689in}}%
\pgfpathlineto{\pgfqpoint{4.036956in}{1.259281in}}%
\pgfpathlineto{\pgfqpoint{4.107531in}{1.178977in}}%
\pgfpathlineto{\pgfqpoint{4.168246in}{1.106463in}}%
\pgfpathlineto{\pgfqpoint{4.228536in}{1.030055in}}%
\pgfpathlineto{\pgfqpoint{4.284278in}{0.953646in}}%
\pgfpathlineto{\pgfqpoint{4.318017in}{0.902707in}}%
\pgfpathlineto{\pgfqpoint{4.348371in}{0.851768in}}%
\pgfpathlineto{\pgfqpoint{4.374350in}{0.800829in}}%
\pgfpathlineto{\pgfqpoint{4.384843in}{0.775360in}}%
\pgfpathlineto{\pgfqpoint{4.393739in}{0.749890in}}%
\pgfpathlineto{\pgfqpoint{4.400495in}{0.724420in}}%
\pgfpathlineto{\pgfqpoint{4.404221in}{0.698951in}}%
\pgfpathlineto{\pgfqpoint{4.403264in}{0.670073in}}%
\pgfpathlineto{\pgfqpoint{4.397683in}{0.648012in}}%
\pgfpathlineto{\pgfqpoint{4.382086in}{0.622542in}}%
\pgfpathlineto{\pgfqpoint{4.373691in}{0.614624in}}%
\pgfpathlineto{\pgfqpoint{4.344118in}{0.596246in}}%
\pgfpathlineto{\pgfqpoint{4.314544in}{0.587674in}}%
\pgfpathlineto{\pgfqpoint{4.284971in}{0.583249in}}%
\pgfpathlineto{\pgfqpoint{4.255398in}{0.581830in}}%
\pgfpathlineto{\pgfqpoint{4.225824in}{0.582661in}}%
\pgfpathlineto{\pgfqpoint{4.196251in}{0.585223in}}%
\pgfpathlineto{\pgfqpoint{4.137104in}{0.594156in}}%
\pgfpathlineto{\pgfqpoint{4.137104in}{0.594156in}}%
\pgfusepath{stroke}%
\end{pgfscope}%
\begin{pgfscope}%
\pgfpathrectangle{\pgfqpoint{0.854460in}{0.571603in}}{\pgfqpoint{5.885100in}{5.068436in}}%
\pgfusepath{clip}%
\pgfsetbuttcap%
\pgfsetroundjoin%
\pgfsetlinewidth{1.505625pt}%
\definecolor{currentstroke}{rgb}{0.278791,0.062145,0.386592}%
\pgfsetstrokecolor{currentstroke}%
\pgfsetdash{}{0pt}%
\pgfpathmoveto{\pgfqpoint{3.596928in}{0.571603in}}%
\pgfpathlineto{\pgfqpoint{3.507962in}{0.622542in}}%
\pgfpathlineto{\pgfqpoint{3.424344in}{0.673481in}}%
\pgfpathlineto{\pgfqpoint{3.338623in}{0.729256in}}%
\pgfpathlineto{\pgfqpoint{3.271786in}{0.775360in}}%
\pgfpathlineto{\pgfqpoint{3.190756in}{0.834882in}}%
\pgfpathlineto{\pgfqpoint{3.131610in}{0.880969in}}%
\pgfpathlineto{\pgfqpoint{3.072463in}{0.929707in}}%
\pgfpathlineto{\pgfqpoint{3.013316in}{0.981490in}}%
\pgfpathlineto{\pgfqpoint{2.954169in}{1.036752in}}%
\pgfpathlineto{\pgfqpoint{2.895023in}{1.095981in}}%
\pgfpathlineto{\pgfqpoint{2.861102in}{1.131933in}}%
\pgfpathlineto{\pgfqpoint{2.806303in}{1.194084in}}%
\pgfpathlineto{\pgfqpoint{2.773448in}{1.233811in}}%
\pgfpathlineto{\pgfqpoint{2.734109in}{1.284750in}}%
\pgfpathlineto{\pgfqpoint{2.697497in}{1.335689in}}%
\pgfpathlineto{\pgfqpoint{2.663625in}{1.386628in}}%
\pgfpathlineto{\pgfqpoint{2.628862in}{1.443863in}}%
\pgfpathlineto{\pgfqpoint{2.599289in}{1.497617in}}%
\pgfpathlineto{\pgfqpoint{2.578304in}{1.539445in}}%
\pgfpathlineto{\pgfqpoint{2.555142in}{1.590384in}}%
\pgfpathlineto{\pgfqpoint{2.534494in}{1.641323in}}%
\pgfpathlineto{\pgfqpoint{2.516507in}{1.692262in}}%
\pgfpathlineto{\pgfqpoint{2.501163in}{1.743202in}}%
\pgfpathlineto{\pgfqpoint{2.488396in}{1.794141in}}%
\pgfpathlineto{\pgfqpoint{2.478251in}{1.845080in}}%
\pgfpathlineto{\pgfqpoint{2.471010in}{1.896019in}}%
\pgfpathlineto{\pgfqpoint{2.466512in}{1.946958in}}%
\pgfpathlineto{\pgfqpoint{2.464931in}{1.997897in}}%
\pgfpathlineto{\pgfqpoint{2.466447in}{2.048836in}}%
\pgfpathlineto{\pgfqpoint{2.471242in}{2.099775in}}%
\pgfpathlineto{\pgfqpoint{2.479501in}{2.150714in}}%
\pgfpathlineto{\pgfqpoint{2.485234in}{2.176183in}}%
\pgfpathlineto{\pgfqpoint{2.499924in}{2.227123in}}%
\pgfpathlineto{\pgfqpoint{2.510569in}{2.257053in}}%
\pgfpathlineto{\pgfqpoint{2.531363in}{2.303531in}}%
\pgfpathlineto{\pgfqpoint{2.544884in}{2.329001in}}%
\pgfpathlineto{\pgfqpoint{2.569716in}{2.368332in}}%
\pgfpathlineto{\pgfqpoint{2.578291in}{2.379940in}}%
\pgfpathlineto{\pgfqpoint{2.599289in}{2.406256in}}%
\pgfpathlineto{\pgfqpoint{2.628862in}{2.436872in}}%
\pgfpathlineto{\pgfqpoint{2.658436in}{2.462009in}}%
\pgfpathlineto{\pgfqpoint{2.688009in}{2.482626in}}%
\pgfpathlineto{\pgfqpoint{2.717582in}{2.499232in}}%
\pgfpathlineto{\pgfqpoint{2.747156in}{2.512497in}}%
\pgfpathlineto{\pgfqpoint{2.776729in}{2.522657in}}%
\pgfpathlineto{\pgfqpoint{2.806303in}{2.530082in}}%
\pgfpathlineto{\pgfqpoint{2.835876in}{2.534897in}}%
\pgfpathlineto{\pgfqpoint{2.865449in}{2.537291in}}%
\pgfpathlineto{\pgfqpoint{2.895023in}{2.537447in}}%
\pgfpathlineto{\pgfqpoint{2.924596in}{2.535485in}}%
\pgfpathlineto{\pgfqpoint{2.954169in}{2.531502in}}%
\pgfpathlineto{\pgfqpoint{2.983743in}{2.525562in}}%
\pgfpathlineto{\pgfqpoint{3.013316in}{2.517813in}}%
\pgfpathlineto{\pgfqpoint{3.045689in}{2.507287in}}%
\pgfpathlineto{\pgfqpoint{3.072463in}{2.497161in}}%
\pgfpathlineto{\pgfqpoint{3.107404in}{2.481818in}}%
\pgfpathlineto{\pgfqpoint{3.131610in}{2.470132in}}%
\pgfpathlineto{\pgfqpoint{3.161183in}{2.454434in}}%
\pgfpathlineto{\pgfqpoint{3.201057in}{2.430879in}}%
\pgfpathlineto{\pgfqpoint{3.249903in}{2.399300in}}%
\pgfpathlineto{\pgfqpoint{3.311829in}{2.354470in}}%
\pgfpathlineto{\pgfqpoint{3.368197in}{2.309881in}}%
\pgfpathlineto{\pgfqpoint{3.435131in}{2.252592in}}%
\pgfpathlineto{\pgfqpoint{3.491374in}{2.201653in}}%
\pgfpathlineto{\pgfqpoint{3.545637in}{2.150413in}}%
\pgfpathlineto{\pgfqpoint{3.634357in}{2.062754in}}%
\pgfpathlineto{\pgfqpoint{3.752650in}{1.940697in}}%
\pgfpathlineto{\pgfqpoint{3.930090in}{1.750905in}}%
\pgfpathlineto{\pgfqpoint{4.354678in}{1.284750in}}%
\pgfpathlineto{\pgfqpoint{4.491984in}{1.131763in}}%
\pgfpathlineto{\pgfqpoint{4.648299in}{0.953646in}}%
\pgfpathlineto{\pgfqpoint{4.734868in}{0.851768in}}%
\pgfpathlineto{\pgfqpoint{4.797548in}{0.775360in}}%
\pgfpathlineto{\pgfqpoint{4.857619in}{0.698951in}}%
\pgfpathlineto{\pgfqpoint{4.906012in}{0.633775in}}%
\pgfpathlineto{\pgfqpoint{4.935585in}{0.591734in}}%
\pgfpathlineto{\pgfqpoint{4.949135in}{0.571603in}}%
\pgfpathlineto{\pgfqpoint{4.949135in}{0.571603in}}%
\pgfusepath{stroke}%
\end{pgfscope}%
\begin{pgfscope}%
\pgfpathrectangle{\pgfqpoint{0.854460in}{0.571603in}}{\pgfqpoint{5.885100in}{5.068436in}}%
\pgfusepath{clip}%
\pgfsetbuttcap%
\pgfsetroundjoin%
\pgfsetlinewidth{1.505625pt}%
\definecolor{currentstroke}{rgb}{0.282327,0.094955,0.417331}%
\pgfsetstrokecolor{currentstroke}%
\pgfsetdash{}{0pt}%
\pgfpathmoveto{\pgfqpoint{3.320897in}{0.571603in}}%
\pgfpathlineto{\pgfqpoint{3.241386in}{0.622542in}}%
\pgfpathlineto{\pgfqpoint{3.161183in}{0.676713in}}%
\pgfpathlineto{\pgfqpoint{3.094084in}{0.724420in}}%
\pgfpathlineto{\pgfqpoint{3.013316in}{0.785118in}}%
\pgfpathlineto{\pgfqpoint{2.954169in}{0.831977in}}%
\pgfpathlineto{\pgfqpoint{2.895023in}{0.881216in}}%
\pgfpathlineto{\pgfqpoint{2.835876in}{0.933148in}}%
\pgfpathlineto{\pgfqpoint{2.776729in}{0.988119in}}%
\pgfpathlineto{\pgfqpoint{2.717582in}{1.046510in}}%
\pgfpathlineto{\pgfqpoint{2.684320in}{1.080994in}}%
\pgfpathlineto{\pgfqpoint{2.628862in}{1.141900in}}%
\pgfpathlineto{\pgfqpoint{2.593597in}{1.182872in}}%
\pgfpathlineto{\pgfqpoint{2.552204in}{1.233811in}}%
\pgfpathlineto{\pgfqpoint{2.510569in}{1.288383in}}%
\pgfpathlineto{\pgfqpoint{2.476815in}{1.335689in}}%
\pgfpathlineto{\pgfqpoint{2.442837in}{1.386628in}}%
\pgfpathlineto{\pgfqpoint{2.411190in}{1.437567in}}%
\pgfpathlineto{\pgfqpoint{2.381843in}{1.488506in}}%
\pgfpathlineto{\pgfqpoint{2.354752in}{1.539445in}}%
\pgfpathlineto{\pgfqpoint{2.329861in}{1.590384in}}%
\pgfpathlineto{\pgfqpoint{2.303555in}{1.650142in}}%
\pgfpathlineto{\pgfqpoint{2.286822in}{1.692262in}}%
\pgfpathlineto{\pgfqpoint{2.268455in}{1.743202in}}%
\pgfpathlineto{\pgfqpoint{2.252290in}{1.794141in}}%
\pgfpathlineto{\pgfqpoint{2.238208in}{1.845080in}}%
\pgfpathlineto{\pgfqpoint{2.226295in}{1.896019in}}%
\pgfpathlineto{\pgfqpoint{2.214835in}{1.956276in}}%
\pgfpathlineto{\pgfqpoint{2.208712in}{1.997897in}}%
\pgfpathlineto{\pgfqpoint{2.203184in}{2.048836in}}%
\pgfpathlineto{\pgfqpoint{2.199788in}{2.099775in}}%
\pgfpathlineto{\pgfqpoint{2.198587in}{2.150714in}}%
\pgfpathlineto{\pgfqpoint{2.199638in}{2.201653in}}%
\pgfpathlineto{\pgfqpoint{2.202997in}{2.252592in}}%
\pgfpathlineto{\pgfqpoint{2.208708in}{2.303531in}}%
\pgfpathlineto{\pgfqpoint{2.216899in}{2.354470in}}%
\pgfpathlineto{\pgfqpoint{2.227905in}{2.405409in}}%
\pgfpathlineto{\pgfqpoint{2.244409in}{2.466027in}}%
\pgfpathlineto{\pgfqpoint{2.258291in}{2.507287in}}%
\pgfpathlineto{\pgfqpoint{2.278191in}{2.558226in}}%
\pgfpathlineto{\pgfqpoint{2.303555in}{2.612650in}}%
\pgfpathlineto{\pgfqpoint{2.333129in}{2.665897in}}%
\pgfpathlineto{\pgfqpoint{2.362702in}{2.711442in}}%
\pgfpathlineto{\pgfqpoint{2.392275in}{2.750860in}}%
\pgfpathlineto{\pgfqpoint{2.423604in}{2.787452in}}%
\pgfpathlineto{\pgfqpoint{2.451422in}{2.816138in}}%
\pgfpathlineto{\pgfqpoint{2.480996in}{2.843164in}}%
\pgfpathlineto{\pgfqpoint{2.510569in}{2.867070in}}%
\pgfpathlineto{\pgfqpoint{2.542015in}{2.889330in}}%
\pgfpathlineto{\pgfqpoint{2.584806in}{2.914800in}}%
\pgfpathlineto{\pgfqpoint{2.599289in}{2.922556in}}%
\pgfpathlineto{\pgfqpoint{2.638922in}{2.940269in}}%
\pgfpathlineto{\pgfqpoint{2.658436in}{2.947832in}}%
\pgfpathlineto{\pgfqpoint{2.688009in}{2.957297in}}%
\pgfpathlineto{\pgfqpoint{2.722650in}{2.965739in}}%
\pgfpathlineto{\pgfqpoint{2.747156in}{2.970275in}}%
\pgfpathlineto{\pgfqpoint{2.776729in}{2.973870in}}%
\pgfpathlineto{\pgfqpoint{2.806303in}{2.975610in}}%
\pgfpathlineto{\pgfqpoint{2.835876in}{2.975522in}}%
\pgfpathlineto{\pgfqpoint{2.865449in}{2.973632in}}%
\pgfpathlineto{\pgfqpoint{2.895023in}{2.969967in}}%
\pgfpathlineto{\pgfqpoint{2.924596in}{2.964542in}}%
\pgfpathlineto{\pgfqpoint{2.954169in}{2.957349in}}%
\pgfpathlineto{\pgfqpoint{2.983743in}{2.948454in}}%
\pgfpathlineto{\pgfqpoint{3.013316in}{2.937870in}}%
\pgfpathlineto{\pgfqpoint{3.042890in}{2.925587in}}%
\pgfpathlineto{\pgfqpoint{3.072463in}{2.911675in}}%
\pgfpathlineto{\pgfqpoint{3.113769in}{2.889330in}}%
\pgfpathlineto{\pgfqpoint{3.155599in}{2.863861in}}%
\pgfpathlineto{\pgfqpoint{3.161183in}{2.860344in}}%
\pgfpathlineto{\pgfqpoint{3.193200in}{2.838391in}}%
\pgfpathlineto{\pgfqpoint{3.227570in}{2.812922in}}%
\pgfpathlineto{\pgfqpoint{3.279476in}{2.771264in}}%
\pgfpathlineto{\pgfqpoint{3.338623in}{2.718945in}}%
\pgfpathlineto{\pgfqpoint{3.399841in}{2.660104in}}%
\pgfpathlineto{\pgfqpoint{3.456917in}{2.601708in}}%
\pgfpathlineto{\pgfqpoint{3.545637in}{2.505300in}}%
\pgfpathlineto{\pgfqpoint{3.634357in}{2.403961in}}%
\pgfpathlineto{\pgfqpoint{3.783692in}{2.227123in}}%
\pgfpathlineto{\pgfqpoint{4.060099in}{1.896019in}}%
\pgfpathlineto{\pgfqpoint{4.255874in}{1.666793in}}%
\pgfpathlineto{\pgfqpoint{4.389314in}{1.513976in}}%
\pgfpathlineto{\pgfqpoint{4.580705in}{1.299774in}}%
\pgfpathlineto{\pgfqpoint{4.733419in}{1.131933in}}%
\pgfpathlineto{\pgfqpoint{5.113025in}{0.716429in}}%
\pgfpathlineto{\pgfqpoint{5.218518in}{0.597073in}}%
\pgfpathlineto{\pgfqpoint{5.240530in}{0.571603in}}%
\pgfpathlineto{\pgfqpoint{5.240530in}{0.571603in}}%
\pgfusepath{stroke}%
\end{pgfscope}%
\begin{pgfscope}%
\pgfpathrectangle{\pgfqpoint{0.854460in}{0.571603in}}{\pgfqpoint{5.885100in}{5.068436in}}%
\pgfusepath{clip}%
\pgfsetbuttcap%
\pgfsetroundjoin%
\pgfsetlinewidth{1.505625pt}%
\definecolor{currentstroke}{rgb}{0.283229,0.120777,0.440584}%
\pgfsetstrokecolor{currentstroke}%
\pgfsetdash{}{0pt}%
\pgfpathmoveto{\pgfqpoint{3.111874in}{0.571603in}}%
\pgfpathlineto{\pgfqpoint{3.036974in}{0.622542in}}%
\pgfpathlineto{\pgfqpoint{2.954169in}{0.681864in}}%
\pgfpathlineto{\pgfqpoint{2.895023in}{0.726197in}}%
\pgfpathlineto{\pgfqpoint{2.832351in}{0.775360in}}%
\pgfpathlineto{\pgfqpoint{2.770439in}{0.826299in}}%
\pgfpathlineto{\pgfqpoint{2.711451in}{0.877238in}}%
\pgfpathlineto{\pgfqpoint{2.655289in}{0.928177in}}%
\pgfpathlineto{\pgfqpoint{2.599289in}{0.981703in}}%
\pgfpathlineto{\pgfqpoint{2.540142in}{1.041645in}}%
\pgfpathlineto{\pgfqpoint{2.503165in}{1.080994in}}%
\pgfpathlineto{\pgfqpoint{2.451422in}{1.139073in}}%
\pgfpathlineto{\pgfqpoint{2.414477in}{1.182872in}}%
\pgfpathlineto{\pgfqpoint{2.362702in}{1.248197in}}%
\pgfpathlineto{\pgfqpoint{2.333129in}{1.287748in}}%
\pgfpathlineto{\pgfqpoint{2.299196in}{1.335689in}}%
\pgfpathlineto{\pgfqpoint{2.265319in}{1.386628in}}%
\pgfpathlineto{\pgfqpoint{2.233586in}{1.437567in}}%
\pgfpathlineto{\pgfqpoint{2.203964in}{1.488506in}}%
\pgfpathlineto{\pgfqpoint{2.176410in}{1.539445in}}%
\pgfpathlineto{\pgfqpoint{2.150874in}{1.590384in}}%
\pgfpathlineto{\pgfqpoint{2.126115in}{1.644110in}}%
\pgfpathlineto{\pgfqpoint{2.095749in}{1.717732in}}%
\pgfpathlineto{\pgfqpoint{2.068520in}{1.794141in}}%
\pgfpathlineto{\pgfqpoint{2.052817in}{1.845080in}}%
\pgfpathlineto{\pgfqpoint{2.037395in}{1.901705in}}%
\pgfpathlineto{\pgfqpoint{2.026824in}{1.946958in}}%
\pgfpathlineto{\pgfqpoint{2.016594in}{1.997897in}}%
\pgfpathlineto{\pgfqpoint{2.007822in}{2.050970in}}%
\pgfpathlineto{\pgfqpoint{2.001639in}{2.099775in}}%
\pgfpathlineto{\pgfqpoint{1.996955in}{2.150714in}}%
\pgfpathlineto{\pgfqpoint{1.994088in}{2.201653in}}%
\pgfpathlineto{\pgfqpoint{1.993068in}{2.252592in}}%
\pgfpathlineto{\pgfqpoint{1.993917in}{2.303531in}}%
\pgfpathlineto{\pgfqpoint{1.996657in}{2.354470in}}%
\pgfpathlineto{\pgfqpoint{2.001300in}{2.405409in}}%
\pgfpathlineto{\pgfqpoint{2.007856in}{2.456348in}}%
\pgfpathlineto{\pgfqpoint{2.016639in}{2.507287in}}%
\pgfpathlineto{\pgfqpoint{2.027408in}{2.558226in}}%
\pgfpathlineto{\pgfqpoint{2.040259in}{2.609165in}}%
\pgfpathlineto{\pgfqpoint{2.055541in}{2.660104in}}%
\pgfpathlineto{\pgfqpoint{2.073079in}{2.711044in}}%
\pgfpathlineto{\pgfqpoint{2.096542in}{2.769995in}}%
\pgfpathlineto{\pgfqpoint{2.116068in}{2.812922in}}%
\pgfpathlineto{\pgfqpoint{2.141871in}{2.863861in}}%
\pgfpathlineto{\pgfqpoint{2.170909in}{2.914800in}}%
\pgfpathlineto{\pgfqpoint{2.186612in}{2.940269in}}%
\pgfpathlineto{\pgfqpoint{2.221387in}{2.991208in}}%
\pgfpathlineto{\pgfqpoint{2.260633in}{3.042147in}}%
\pgfpathlineto{\pgfqpoint{2.282138in}{3.067617in}}%
\pgfpathlineto{\pgfqpoint{2.305035in}{3.093086in}}%
\pgfpathlineto{\pgfqpoint{2.333129in}{3.121925in}}%
\pgfpathlineto{\pgfqpoint{2.362702in}{3.149813in}}%
\pgfpathlineto{\pgfqpoint{2.392275in}{3.175419in}}%
\pgfpathlineto{\pgfqpoint{2.421849in}{3.198907in}}%
\pgfpathlineto{\pgfqpoint{2.451447in}{3.220434in}}%
\pgfpathlineto{\pgfqpoint{2.490766in}{3.245904in}}%
\pgfpathlineto{\pgfqpoint{2.510569in}{3.257762in}}%
\pgfpathlineto{\pgfqpoint{2.540142in}{3.273866in}}%
\pgfpathlineto{\pgfqpoint{2.589522in}{3.296843in}}%
\pgfpathlineto{\pgfqpoint{2.599289in}{3.301038in}}%
\pgfpathlineto{\pgfqpoint{2.628862in}{3.312208in}}%
\pgfpathlineto{\pgfqpoint{2.660050in}{3.322312in}}%
\pgfpathlineto{\pgfqpoint{2.688009in}{3.329912in}}%
\pgfpathlineto{\pgfqpoint{2.717582in}{3.336443in}}%
\pgfpathlineto{\pgfqpoint{2.747156in}{3.341456in}}%
\pgfpathlineto{\pgfqpoint{2.776729in}{3.344936in}}%
\pgfpathlineto{\pgfqpoint{2.806303in}{3.346868in}}%
\pgfpathlineto{\pgfqpoint{2.835876in}{3.347237in}}%
\pgfpathlineto{\pgfqpoint{2.865449in}{3.346028in}}%
\pgfpathlineto{\pgfqpoint{2.895023in}{3.343225in}}%
\pgfpathlineto{\pgfqpoint{2.924596in}{3.338811in}}%
\pgfpathlineto{\pgfqpoint{2.954169in}{3.332769in}}%
\pgfpathlineto{\pgfqpoint{2.992540in}{3.322312in}}%
\pgfpathlineto{\pgfqpoint{3.013316in}{3.315686in}}%
\pgfpathlineto{\pgfqpoint{3.060808in}{3.296843in}}%
\pgfpathlineto{\pgfqpoint{3.072463in}{3.291753in}}%
\pgfpathlineto{\pgfqpoint{3.112557in}{3.271373in}}%
\pgfpathlineto{\pgfqpoint{3.131610in}{3.260798in}}%
\pgfpathlineto{\pgfqpoint{3.161183in}{3.242665in}}%
\pgfpathlineto{\pgfqpoint{3.193920in}{3.220434in}}%
\pgfpathlineto{\pgfqpoint{3.228022in}{3.194965in}}%
\pgfpathlineto{\pgfqpoint{3.259437in}{3.169495in}}%
\pgfpathlineto{\pgfqpoint{3.288731in}{3.144025in}}%
\pgfpathlineto{\pgfqpoint{3.316328in}{3.118556in}}%
\pgfpathlineto{\pgfqpoint{3.367628in}{3.067617in}}%
\pgfpathlineto{\pgfqpoint{3.397770in}{3.035531in}}%
\pgfpathlineto{\pgfqpoint{3.459181in}{2.965739in}}%
\pgfpathlineto{\pgfqpoint{3.521898in}{2.889330in}}%
\pgfpathlineto{\pgfqpoint{3.581555in}{2.812922in}}%
\pgfpathlineto{\pgfqpoint{3.663930in}{2.703320in}}%
\pgfpathlineto{\pgfqpoint{4.018811in}{2.218676in}}%
\pgfpathlineto{\pgfqpoint{4.089233in}{2.125244in}}%
\pgfpathlineto{\pgfqpoint{4.207448in}{1.972427in}}%
\pgfpathlineto{\pgfqpoint{4.329657in}{1.819610in}}%
\pgfpathlineto{\pgfqpoint{4.456168in}{1.666793in}}%
\pgfpathlineto{\pgfqpoint{4.586978in}{1.513976in}}%
\pgfpathlineto{\pgfqpoint{4.721985in}{1.361159in}}%
\pgfpathlineto{\pgfqpoint{4.860855in}{1.208341in}}%
\pgfpathlineto{\pgfqpoint{5.024305in}{1.033084in}}%
\pgfpathlineto{\pgfqpoint{5.172387in}{0.877238in}}%
\pgfpathlineto{\pgfqpoint{5.464941in}{0.571603in}}%
\pgfpathlineto{\pgfqpoint{5.464941in}{0.571603in}}%
\pgfusepath{stroke}%
\end{pgfscope}%
\begin{pgfscope}%
\pgfpathrectangle{\pgfqpoint{0.854460in}{0.571603in}}{\pgfqpoint{5.885100in}{5.068436in}}%
\pgfusepath{clip}%
\pgfsetbuttcap%
\pgfsetroundjoin%
\pgfsetlinewidth{1.505625pt}%
\definecolor{currentstroke}{rgb}{0.281887,0.150881,0.465405}%
\pgfsetstrokecolor{currentstroke}%
\pgfsetdash{}{0pt}%
\pgfpathmoveto{\pgfqpoint{2.937449in}{0.571603in}}%
\pgfpathlineto{\pgfqpoint{2.865449in}{0.622567in}}%
\pgfpathlineto{\pgfqpoint{2.776729in}{0.688851in}}%
\pgfpathlineto{\pgfqpoint{2.717582in}{0.735223in}}%
\pgfpathlineto{\pgfqpoint{2.658436in}{0.783590in}}%
\pgfpathlineto{\pgfqpoint{2.599289in}{0.834187in}}%
\pgfpathlineto{\pgfqpoint{2.540142in}{0.887274in}}%
\pgfpathlineto{\pgfqpoint{2.480996in}{0.943131in}}%
\pgfpathlineto{\pgfqpoint{2.421849in}{1.002067in}}%
\pgfpathlineto{\pgfqpoint{2.392275in}{1.032902in}}%
\pgfpathlineto{\pgfqpoint{2.333129in}{1.097625in}}%
\pgfpathlineto{\pgfqpoint{2.303200in}{1.131933in}}%
\pgfpathlineto{\pgfqpoint{2.244409in}{1.203456in}}%
\pgfpathlineto{\pgfqpoint{2.214835in}{1.241616in}}%
\pgfpathlineto{\pgfqpoint{2.182802in}{1.284750in}}%
\pgfpathlineto{\pgfqpoint{2.147058in}{1.335689in}}%
\pgfpathlineto{\pgfqpoint{2.113369in}{1.386628in}}%
\pgfpathlineto{\pgfqpoint{2.081708in}{1.437567in}}%
\pgfpathlineto{\pgfqpoint{2.052039in}{1.488506in}}%
\pgfpathlineto{\pgfqpoint{2.024323in}{1.539445in}}%
\pgfpathlineto{\pgfqpoint{1.998509in}{1.590384in}}%
\pgfpathlineto{\pgfqpoint{1.974542in}{1.641323in}}%
\pgfpathlineto{\pgfqpoint{1.948675in}{1.701494in}}%
\pgfpathlineto{\pgfqpoint{1.922716in}{1.768671in}}%
\pgfpathlineto{\pgfqpoint{1.905159in}{1.819610in}}%
\pgfpathlineto{\pgfqpoint{1.889196in}{1.870549in}}%
\pgfpathlineto{\pgfqpoint{1.868691in}{1.946958in}}%
\pgfpathlineto{\pgfqpoint{1.857039in}{1.997897in}}%
\pgfpathlineto{\pgfqpoint{1.847176in}{2.048836in}}%
\pgfpathlineto{\pgfqpoint{1.838897in}{2.099775in}}%
\pgfpathlineto{\pgfqpoint{1.830381in}{2.168032in}}%
\pgfpathlineto{\pgfqpoint{1.827285in}{2.201653in}}%
\pgfpathlineto{\pgfqpoint{1.824006in}{2.252592in}}%
\pgfpathlineto{\pgfqpoint{1.822355in}{2.303531in}}%
\pgfpathlineto{\pgfqpoint{1.822344in}{2.354470in}}%
\pgfpathlineto{\pgfqpoint{1.823983in}{2.405409in}}%
\pgfpathlineto{\pgfqpoint{1.827276in}{2.456348in}}%
\pgfpathlineto{\pgfqpoint{1.832281in}{2.507287in}}%
\pgfpathlineto{\pgfqpoint{1.839089in}{2.558226in}}%
\pgfpathlineto{\pgfqpoint{1.847595in}{2.609165in}}%
\pgfpathlineto{\pgfqpoint{1.859955in}{2.669787in}}%
\pgfpathlineto{\pgfqpoint{1.869968in}{2.711044in}}%
\pgfpathlineto{\pgfqpoint{1.889528in}{2.780754in}}%
\pgfpathlineto{\pgfqpoint{1.899907in}{2.812922in}}%
\pgfpathlineto{\pgfqpoint{1.919102in}{2.867381in}}%
\pgfpathlineto{\pgfqpoint{1.937976in}{2.914800in}}%
\pgfpathlineto{\pgfqpoint{1.960306in}{2.965739in}}%
\pgfpathlineto{\pgfqpoint{1.984961in}{3.016678in}}%
\pgfpathlineto{\pgfqpoint{2.012147in}{3.067617in}}%
\pgfpathlineto{\pgfqpoint{2.042089in}{3.118556in}}%
\pgfpathlineto{\pgfqpoint{2.075035in}{3.169495in}}%
\pgfpathlineto{\pgfqpoint{2.111252in}{3.220434in}}%
\pgfpathlineto{\pgfqpoint{2.130609in}{3.245904in}}%
\pgfpathlineto{\pgfqpoint{2.172569in}{3.296843in}}%
\pgfpathlineto{\pgfqpoint{2.195145in}{3.322312in}}%
\pgfpathlineto{\pgfqpoint{2.218884in}{3.347782in}}%
\pgfpathlineto{\pgfqpoint{2.244409in}{3.373699in}}%
\pgfpathlineto{\pgfqpoint{2.298991in}{3.424190in}}%
\pgfpathlineto{\pgfqpoint{2.333129in}{3.452949in}}%
\pgfpathlineto{\pgfqpoint{2.362702in}{3.476123in}}%
\pgfpathlineto{\pgfqpoint{2.396372in}{3.500599in}}%
\pgfpathlineto{\pgfqpoint{2.451422in}{3.536843in}}%
\pgfpathlineto{\pgfqpoint{2.480996in}{3.554408in}}%
\pgfpathlineto{\pgfqpoint{2.540142in}{3.585598in}}%
\pgfpathlineto{\pgfqpoint{2.577067in}{3.602477in}}%
\pgfpathlineto{\pgfqpoint{2.628862in}{3.623151in}}%
\pgfpathlineto{\pgfqpoint{2.658436in}{3.633234in}}%
\pgfpathlineto{\pgfqpoint{2.688009in}{3.642096in}}%
\pgfpathlineto{\pgfqpoint{2.734196in}{3.653416in}}%
\pgfpathlineto{\pgfqpoint{2.747156in}{3.656234in}}%
\pgfpathlineto{\pgfqpoint{2.776729in}{3.661421in}}%
\pgfpathlineto{\pgfqpoint{2.806303in}{3.665373in}}%
\pgfpathlineto{\pgfqpoint{2.835876in}{3.668056in}}%
\pgfpathlineto{\pgfqpoint{2.865449in}{3.669438in}}%
\pgfpathlineto{\pgfqpoint{2.895023in}{3.669484in}}%
\pgfpathlineto{\pgfqpoint{2.924596in}{3.668158in}}%
\pgfpathlineto{\pgfqpoint{2.954169in}{3.665422in}}%
\pgfpathlineto{\pgfqpoint{2.983743in}{3.661237in}}%
\pgfpathlineto{\pgfqpoint{3.022177in}{3.653416in}}%
\pgfpathlineto{\pgfqpoint{3.042890in}{3.648290in}}%
\pgfpathlineto{\pgfqpoint{3.072463in}{3.639403in}}%
\pgfpathlineto{\pgfqpoint{3.104313in}{3.627946in}}%
\pgfpathlineto{\pgfqpoint{3.131610in}{3.616554in}}%
\pgfpathlineto{\pgfqpoint{3.161186in}{3.602477in}}%
\pgfpathlineto{\pgfqpoint{3.206521in}{3.577007in}}%
\pgfpathlineto{\pgfqpoint{3.220330in}{3.568564in}}%
\pgfpathlineto{\pgfqpoint{3.249903in}{3.548662in}}%
\pgfpathlineto{\pgfqpoint{3.280268in}{3.526068in}}%
\pgfpathlineto{\pgfqpoint{3.311382in}{3.500599in}}%
\pgfpathlineto{\pgfqpoint{3.340018in}{3.475129in}}%
\pgfpathlineto{\pgfqpoint{3.368197in}{3.448117in}}%
\pgfpathlineto{\pgfqpoint{3.414998in}{3.398721in}}%
\pgfpathlineto{\pgfqpoint{3.437464in}{3.373251in}}%
\pgfpathlineto{\pgfqpoint{3.479636in}{3.322312in}}%
\pgfpathlineto{\pgfqpoint{3.516063in}{3.275320in}}%
\pgfpathlineto{\pgfqpoint{3.575210in}{3.193331in}}%
\pgfpathlineto{\pgfqpoint{3.642602in}{3.093086in}}%
\pgfpathlineto{\pgfqpoint{3.708005in}{2.991208in}}%
\pgfpathlineto{\pgfqpoint{3.787723in}{2.863861in}}%
\pgfpathlineto{\pgfqpoint{3.947075in}{2.609165in}}%
\pgfpathlineto{\pgfqpoint{4.062620in}{2.430879in}}%
\pgfpathlineto{\pgfqpoint{4.166677in}{2.277063in}}%
\pgfpathlineto{\pgfqpoint{4.237321in}{2.176183in}}%
\pgfpathlineto{\pgfqpoint{4.329717in}{2.048836in}}%
\pgfpathlineto{\pgfqpoint{4.432838in}{1.912407in}}%
\pgfpathlineto{\pgfqpoint{4.505314in}{1.819610in}}%
\pgfpathlineto{\pgfqpoint{4.610278in}{1.689776in}}%
\pgfpathlineto{\pgfqpoint{4.698998in}{1.583570in}}%
\pgfpathlineto{\pgfqpoint{4.787718in}{1.480338in}}%
\pgfpathlineto{\pgfqpoint{4.876438in}{1.379776in}}%
\pgfpathlineto{\pgfqpoint{4.965158in}{1.281608in}}%
\pgfpathlineto{\pgfqpoint{5.056368in}{1.182872in}}%
\pgfpathlineto{\pgfqpoint{5.176738in}{1.055524in}}%
\pgfpathlineto{\pgfqpoint{5.324784in}{0.902707in}}%
\pgfpathlineto{\pgfqpoint{5.501385in}{0.724420in}}%
\pgfpathlineto{\pgfqpoint{5.654718in}{0.571603in}}%
\pgfpathlineto{\pgfqpoint{5.654718in}{0.571603in}}%
\pgfusepath{stroke}%
\end{pgfscope}%
\begin{pgfscope}%
\pgfpathrectangle{\pgfqpoint{0.854460in}{0.571603in}}{\pgfqpoint{5.885100in}{5.068436in}}%
\pgfusepath{clip}%
\pgfsetbuttcap%
\pgfsetroundjoin%
\pgfsetlinewidth{1.505625pt}%
\definecolor{currentstroke}{rgb}{0.278826,0.175490,0.483397}%
\pgfsetstrokecolor{currentstroke}%
\pgfsetdash{}{0pt}%
\pgfpathmoveto{\pgfqpoint{2.785093in}{0.571603in}}%
\pgfpathlineto{\pgfqpoint{2.715383in}{0.622542in}}%
\pgfpathlineto{\pgfqpoint{2.628862in}{0.689104in}}%
\pgfpathlineto{\pgfqpoint{2.569716in}{0.736834in}}%
\pgfpathlineto{\pgfqpoint{2.510569in}{0.786620in}}%
\pgfpathlineto{\pgfqpoint{2.451422in}{0.838695in}}%
\pgfpathlineto{\pgfqpoint{2.392275in}{0.893309in}}%
\pgfpathlineto{\pgfqpoint{2.330203in}{0.953646in}}%
\pgfpathlineto{\pgfqpoint{2.273982in}{1.011526in}}%
\pgfpathlineto{\pgfqpoint{2.214835in}{1.075996in}}%
\pgfpathlineto{\pgfqpoint{2.185262in}{1.109811in}}%
\pgfpathlineto{\pgfqpoint{2.124704in}{1.182872in}}%
\pgfpathlineto{\pgfqpoint{2.066968in}{1.258131in}}%
\pgfpathlineto{\pgfqpoint{2.029709in}{1.310220in}}%
\pgfpathlineto{\pgfqpoint{1.995281in}{1.361159in}}%
\pgfpathlineto{\pgfqpoint{1.962806in}{1.412098in}}%
\pgfpathlineto{\pgfqpoint{1.932252in}{1.463037in}}%
\pgfpathlineto{\pgfqpoint{1.903583in}{1.513976in}}%
\pgfpathlineto{\pgfqpoint{1.876752in}{1.564915in}}%
\pgfpathlineto{\pgfqpoint{1.851712in}{1.615854in}}%
\pgfpathlineto{\pgfqpoint{1.828407in}{1.666793in}}%
\pgfpathlineto{\pgfqpoint{1.796785in}{1.743202in}}%
\pgfpathlineto{\pgfqpoint{1.768930in}{1.819610in}}%
\pgfpathlineto{\pgfqpoint{1.744809in}{1.896019in}}%
\pgfpathlineto{\pgfqpoint{1.730795in}{1.946958in}}%
\pgfpathlineto{\pgfqpoint{1.712599in}{2.023366in}}%
\pgfpathlineto{\pgfqpoint{1.702558in}{2.074305in}}%
\pgfpathlineto{\pgfqpoint{1.690277in}{2.150714in}}%
\pgfpathlineto{\pgfqpoint{1.682515in}{2.215869in}}%
\pgfpathlineto{\pgfqpoint{1.679240in}{2.252592in}}%
\pgfpathlineto{\pgfqpoint{1.676049in}{2.303531in}}%
\pgfpathlineto{\pgfqpoint{1.674356in}{2.354470in}}%
\pgfpathlineto{\pgfqpoint{1.674170in}{2.405409in}}%
\pgfpathlineto{\pgfqpoint{1.675492in}{2.456348in}}%
\pgfpathlineto{\pgfqpoint{1.678324in}{2.507287in}}%
\pgfpathlineto{\pgfqpoint{1.682666in}{2.558226in}}%
\pgfpathlineto{\pgfqpoint{1.688672in}{2.609165in}}%
\pgfpathlineto{\pgfqpoint{1.696214in}{2.660104in}}%
\pgfpathlineto{\pgfqpoint{1.705280in}{2.711044in}}%
\pgfpathlineto{\pgfqpoint{1.715965in}{2.761983in}}%
\pgfpathlineto{\pgfqpoint{1.735165in}{2.838391in}}%
\pgfpathlineto{\pgfqpoint{1.750092in}{2.889330in}}%
\pgfpathlineto{\pgfqpoint{1.771235in}{2.953201in}}%
\pgfpathlineto{\pgfqpoint{1.795185in}{3.016678in}}%
\pgfpathlineto{\pgfqpoint{1.816583in}{3.067617in}}%
\pgfpathlineto{\pgfqpoint{1.839933in}{3.118556in}}%
\pgfpathlineto{\pgfqpoint{1.865373in}{3.169495in}}%
\pgfpathlineto{\pgfqpoint{1.893055in}{3.220434in}}%
\pgfpathlineto{\pgfqpoint{1.923135in}{3.271373in}}%
\pgfpathlineto{\pgfqpoint{1.955780in}{3.322312in}}%
\pgfpathlineto{\pgfqpoint{1.991163in}{3.373251in}}%
\pgfpathlineto{\pgfqpoint{2.029464in}{3.424190in}}%
\pgfpathlineto{\pgfqpoint{2.066968in}{3.470383in}}%
\pgfpathlineto{\pgfqpoint{2.096542in}{3.504413in}}%
\pgfpathlineto{\pgfqpoint{2.140567in}{3.551538in}}%
\pgfpathlineto{\pgfqpoint{2.185262in}{3.595902in}}%
\pgfpathlineto{\pgfqpoint{2.220068in}{3.627946in}}%
\pgfpathlineto{\pgfqpoint{2.273982in}{3.673972in}}%
\pgfpathlineto{\pgfqpoint{2.312775in}{3.704355in}}%
\pgfpathlineto{\pgfqpoint{2.362702in}{3.740667in}}%
\pgfpathlineto{\pgfqpoint{2.423815in}{3.780764in}}%
\pgfpathlineto{\pgfqpoint{2.480996in}{3.814210in}}%
\pgfpathlineto{\pgfqpoint{2.540142in}{3.844894in}}%
\pgfpathlineto{\pgfqpoint{2.599289in}{3.871718in}}%
\pgfpathlineto{\pgfqpoint{2.658436in}{3.894733in}}%
\pgfpathlineto{\pgfqpoint{2.717582in}{3.913965in}}%
\pgfpathlineto{\pgfqpoint{2.776729in}{3.929344in}}%
\pgfpathlineto{\pgfqpoint{2.835876in}{3.940752in}}%
\pgfpathlineto{\pgfqpoint{2.895023in}{3.948059in}}%
\pgfpathlineto{\pgfqpoint{2.924596in}{3.950115in}}%
\pgfpathlineto{\pgfqpoint{2.954169in}{3.951053in}}%
\pgfpathlineto{\pgfqpoint{2.983743in}{3.950834in}}%
\pgfpathlineto{\pgfqpoint{3.013316in}{3.949413in}}%
\pgfpathlineto{\pgfqpoint{3.042890in}{3.946745in}}%
\pgfpathlineto{\pgfqpoint{3.072463in}{3.942779in}}%
\pgfpathlineto{\pgfqpoint{3.119233in}{3.933581in}}%
\pgfpathlineto{\pgfqpoint{3.131610in}{3.930700in}}%
\pgfpathlineto{\pgfqpoint{3.161183in}{3.922392in}}%
\pgfpathlineto{\pgfqpoint{3.202287in}{3.908111in}}%
\pgfpathlineto{\pgfqpoint{3.220330in}{3.900974in}}%
\pgfpathlineto{\pgfqpoint{3.259804in}{3.882642in}}%
\pgfpathlineto{\pgfqpoint{3.279476in}{3.872411in}}%
\pgfpathlineto{\pgfqpoint{3.309050in}{3.855217in}}%
\pgfpathlineto{\pgfqpoint{3.344457in}{3.831703in}}%
\pgfpathlineto{\pgfqpoint{3.378316in}{3.806233in}}%
\pgfpathlineto{\pgfqpoint{3.408674in}{3.780764in}}%
\pgfpathlineto{\pgfqpoint{3.436309in}{3.755294in}}%
\pgfpathlineto{\pgfqpoint{3.461792in}{3.729825in}}%
\pgfpathlineto{\pgfqpoint{3.486490in}{3.703280in}}%
\pgfpathlineto{\pgfqpoint{3.528436in}{3.653416in}}%
\pgfpathlineto{\pgfqpoint{3.567260in}{3.602477in}}%
\pgfpathlineto{\pgfqpoint{3.585399in}{3.577007in}}%
\pgfpathlineto{\pgfqpoint{3.619737in}{3.526068in}}%
\pgfpathlineto{\pgfqpoint{3.652007in}{3.475129in}}%
\pgfpathlineto{\pgfqpoint{3.693504in}{3.405884in}}%
\pgfpathlineto{\pgfqpoint{3.754717in}{3.296843in}}%
\pgfpathlineto{\pgfqpoint{3.841370in}{3.134652in}}%
\pgfpathlineto{\pgfqpoint{3.930348in}{2.965739in}}%
\pgfpathlineto{\pgfqpoint{3.998607in}{2.838391in}}%
\pgfpathlineto{\pgfqpoint{4.069081in}{2.711044in}}%
\pgfpathlineto{\pgfqpoint{4.157421in}{2.558226in}}%
\pgfpathlineto{\pgfqpoint{4.234760in}{2.430879in}}%
\pgfpathlineto{\pgfqpoint{4.299312in}{2.329001in}}%
\pgfpathlineto{\pgfqpoint{4.383608in}{2.201653in}}%
\pgfpathlineto{\pgfqpoint{4.462411in}{2.087994in}}%
\pgfpathlineto{\pgfqpoint{4.527177in}{1.997897in}}%
\pgfpathlineto{\pgfqpoint{4.610278in}{1.886599in}}%
\pgfpathlineto{\pgfqpoint{4.681731in}{1.794141in}}%
\pgfpathlineto{\pgfqpoint{4.763219in}{1.692262in}}%
\pgfpathlineto{\pgfqpoint{4.847472in}{1.590384in}}%
\pgfpathlineto{\pgfqpoint{4.935585in}{1.487159in}}%
\pgfpathlineto{\pgfqpoint{5.024305in}{1.386317in}}%
\pgfpathlineto{\pgfqpoint{5.116201in}{1.284750in}}%
\pgfpathlineto{\pgfqpoint{5.231319in}{1.161234in}}%
\pgfpathlineto{\pgfqpoint{5.332531in}{1.055524in}}%
\pgfpathlineto{\pgfqpoint{5.467906in}{0.917862in}}%
\pgfpathlineto{\pgfqpoint{5.586199in}{0.800446in}}%
\pgfpathlineto{\pgfqpoint{5.742861in}{0.648012in}}%
\pgfpathlineto{\pgfqpoint{5.822537in}{0.571603in}}%
\pgfpathlineto{\pgfqpoint{5.822537in}{0.571603in}}%
\pgfusepath{stroke}%
\end{pgfscope}%
\begin{pgfscope}%
\pgfpathrectangle{\pgfqpoint{0.854460in}{0.571603in}}{\pgfqpoint{5.885100in}{5.068436in}}%
\pgfusepath{clip}%
\pgfsetbuttcap%
\pgfsetroundjoin%
\pgfsetlinewidth{1.505625pt}%
\definecolor{currentstroke}{rgb}{0.273006,0.204520,0.501721}%
\pgfsetstrokecolor{currentstroke}%
\pgfsetdash{}{0pt}%
\pgfpathmoveto{\pgfqpoint{2.648461in}{0.571603in}}%
\pgfpathlineto{\pgfqpoint{2.569716in}{0.630911in}}%
\pgfpathlineto{\pgfqpoint{2.510569in}{0.677376in}}%
\pgfpathlineto{\pgfqpoint{2.451422in}{0.725724in}}%
\pgfpathlineto{\pgfqpoint{2.392275in}{0.776164in}}%
\pgfpathlineto{\pgfqpoint{2.333129in}{0.828923in}}%
\pgfpathlineto{\pgfqpoint{2.273982in}{0.884248in}}%
\pgfpathlineto{\pgfqpoint{2.214835in}{0.942409in}}%
\pgfpathlineto{\pgfqpoint{2.154856in}{1.004585in}}%
\pgfpathlineto{\pgfqpoint{2.096542in}{1.068903in}}%
\pgfpathlineto{\pgfqpoint{2.042639in}{1.131933in}}%
\pgfpathlineto{\pgfqpoint{2.001497in}{1.182872in}}%
\pgfpathlineto{\pgfqpoint{1.948675in}{1.252391in}}%
\pgfpathlineto{\pgfqpoint{1.907566in}{1.310220in}}%
\pgfpathlineto{\pgfqpoint{1.873426in}{1.361159in}}%
\pgfpathlineto{\pgfqpoint{1.830381in}{1.429815in}}%
\pgfpathlineto{\pgfqpoint{1.796225in}{1.488506in}}%
\pgfpathlineto{\pgfqpoint{1.755378in}{1.564915in}}%
\pgfpathlineto{\pgfqpoint{1.730293in}{1.615854in}}%
\pgfpathlineto{\pgfqpoint{1.706880in}{1.666793in}}%
\pgfpathlineto{\pgfqpoint{1.674934in}{1.743202in}}%
\pgfpathlineto{\pgfqpoint{1.646612in}{1.819610in}}%
\pgfpathlineto{\pgfqpoint{1.621809in}{1.896019in}}%
\pgfpathlineto{\pgfqpoint{1.600559in}{1.972427in}}%
\pgfpathlineto{\pgfqpoint{1.588247in}{2.023366in}}%
\pgfpathlineto{\pgfqpoint{1.572598in}{2.099775in}}%
\pgfpathlineto{\pgfqpoint{1.563922in}{2.150714in}}%
\pgfpathlineto{\pgfqpoint{1.553806in}{2.227123in}}%
\pgfpathlineto{\pgfqpoint{1.546851in}{2.303531in}}%
\pgfpathlineto{\pgfqpoint{1.543083in}{2.379940in}}%
\pgfpathlineto{\pgfqpoint{1.542519in}{2.456348in}}%
\pgfpathlineto{\pgfqpoint{1.545165in}{2.532757in}}%
\pgfpathlineto{\pgfqpoint{1.551013in}{2.609165in}}%
\pgfpathlineto{\pgfqpoint{1.560041in}{2.685574in}}%
\pgfpathlineto{\pgfqpoint{1.567906in}{2.736513in}}%
\pgfpathlineto{\pgfqpoint{1.582535in}{2.812922in}}%
\pgfpathlineto{\pgfqpoint{1.594046in}{2.863861in}}%
\pgfpathlineto{\pgfqpoint{1.614425in}{2.940269in}}%
\pgfpathlineto{\pgfqpoint{1.629920in}{2.991208in}}%
\pgfpathlineto{\pgfqpoint{1.656227in}{3.067617in}}%
\pgfpathlineto{\pgfqpoint{1.686409in}{3.144025in}}%
\pgfpathlineto{\pgfqpoint{1.712088in}{3.202247in}}%
\pgfpathlineto{\pgfqpoint{1.745810in}{3.271373in}}%
\pgfpathlineto{\pgfqpoint{1.772956in}{3.322312in}}%
\pgfpathlineto{\pgfqpoint{1.802225in}{3.373251in}}%
\pgfpathlineto{\pgfqpoint{1.833741in}{3.424190in}}%
\pgfpathlineto{\pgfqpoint{1.867632in}{3.475129in}}%
\pgfpathlineto{\pgfqpoint{1.904024in}{3.526068in}}%
\pgfpathlineto{\pgfqpoint{1.948675in}{3.584097in}}%
\pgfpathlineto{\pgfqpoint{1.985067in}{3.627946in}}%
\pgfpathlineto{\pgfqpoint{2.037395in}{3.686595in}}%
\pgfpathlineto{\pgfqpoint{2.079126in}{3.729825in}}%
\pgfpathlineto{\pgfqpoint{2.126115in}{3.775451in}}%
\pgfpathlineto{\pgfqpoint{2.159921in}{3.806233in}}%
\pgfpathlineto{\pgfqpoint{2.219853in}{3.857172in}}%
\pgfpathlineto{\pgfqpoint{2.273982in}{3.899486in}}%
\pgfpathlineto{\pgfqpoint{2.333129in}{3.941988in}}%
\pgfpathlineto{\pgfqpoint{2.398148in}{3.984520in}}%
\pgfpathlineto{\pgfqpoint{2.451422in}{4.016351in}}%
\pgfpathlineto{\pgfqpoint{2.510569in}{4.048549in}}%
\pgfpathlineto{\pgfqpoint{2.569716in}{4.077624in}}%
\pgfpathlineto{\pgfqpoint{2.628862in}{4.103621in}}%
\pgfpathlineto{\pgfqpoint{2.688009in}{4.126571in}}%
\pgfpathlineto{\pgfqpoint{2.747156in}{4.146491in}}%
\pgfpathlineto{\pgfqpoint{2.806303in}{4.163386in}}%
\pgfpathlineto{\pgfqpoint{2.865449in}{4.177019in}}%
\pgfpathlineto{\pgfqpoint{2.924596in}{4.187497in}}%
\pgfpathlineto{\pgfqpoint{2.983743in}{4.194449in}}%
\pgfpathlineto{\pgfqpoint{3.042890in}{4.197780in}}%
\pgfpathlineto{\pgfqpoint{3.102036in}{4.197193in}}%
\pgfpathlineto{\pgfqpoint{3.131610in}{4.195322in}}%
\pgfpathlineto{\pgfqpoint{3.161183in}{4.192335in}}%
\pgfpathlineto{\pgfqpoint{3.190756in}{4.188179in}}%
\pgfpathlineto{\pgfqpoint{3.220330in}{4.182695in}}%
\pgfpathlineto{\pgfqpoint{3.249903in}{4.175903in}}%
\pgfpathlineto{\pgfqpoint{3.294791in}{4.162807in}}%
\pgfpathlineto{\pgfqpoint{3.309050in}{4.158043in}}%
\pgfpathlineto{\pgfqpoint{3.360115in}{4.137337in}}%
\pgfpathlineto{\pgfqpoint{3.368197in}{4.133682in}}%
\pgfpathlineto{\pgfqpoint{3.410022in}{4.111867in}}%
\pgfpathlineto{\pgfqpoint{3.427343in}{4.101790in}}%
\pgfpathlineto{\pgfqpoint{3.456917in}{4.082688in}}%
\pgfpathlineto{\pgfqpoint{3.486860in}{4.060928in}}%
\pgfpathlineto{\pgfqpoint{3.517996in}{4.035459in}}%
\pgfpathlineto{\pgfqpoint{3.545997in}{4.009989in}}%
\pgfpathlineto{\pgfqpoint{3.575210in}{3.980505in}}%
\pgfpathlineto{\pgfqpoint{3.616258in}{3.933581in}}%
\pgfpathlineto{\pgfqpoint{3.636564in}{3.908111in}}%
\pgfpathlineto{\pgfqpoint{3.673376in}{3.857172in}}%
\pgfpathlineto{\pgfqpoint{3.706598in}{3.806233in}}%
\pgfpathlineto{\pgfqpoint{3.737052in}{3.755294in}}%
\pgfpathlineto{\pgfqpoint{3.765369in}{3.704355in}}%
\pgfpathlineto{\pgfqpoint{3.804883in}{3.627946in}}%
\pgfpathlineto{\pgfqpoint{3.841370in}{3.552664in}}%
\pgfpathlineto{\pgfqpoint{3.900517in}{3.423225in}}%
\pgfpathlineto{\pgfqpoint{4.058482in}{3.067617in}}%
\pgfpathlineto{\pgfqpoint{4.117793in}{2.940269in}}%
\pgfpathlineto{\pgfqpoint{4.179930in}{2.812922in}}%
\pgfpathlineto{\pgfqpoint{4.245475in}{2.685574in}}%
\pgfpathlineto{\pgfqpoint{4.314893in}{2.558226in}}%
\pgfpathlineto{\pgfqpoint{4.358413in}{2.481818in}}%
\pgfpathlineto{\pgfqpoint{4.418943in}{2.379940in}}%
\pgfpathlineto{\pgfqpoint{4.491984in}{2.263062in}}%
\pgfpathlineto{\pgfqpoint{4.531788in}{2.201653in}}%
\pgfpathlineto{\pgfqpoint{4.600337in}{2.099775in}}%
\pgfpathlineto{\pgfqpoint{4.671938in}{1.997897in}}%
\pgfpathlineto{\pgfqpoint{4.746501in}{1.896019in}}%
\pgfpathlineto{\pgfqpoint{4.824166in}{1.794141in}}%
\pgfpathlineto{\pgfqpoint{4.906012in}{1.690880in}}%
\pgfpathlineto{\pgfqpoint{4.988557in}{1.590384in}}%
\pgfpathlineto{\pgfqpoint{5.083452in}{1.479058in}}%
\pgfpathlineto{\pgfqpoint{5.164829in}{1.386628in}}%
\pgfpathlineto{\pgfqpoint{5.260892in}{1.280871in}}%
\pgfpathlineto{\pgfqpoint{5.352528in}{1.182872in}}%
\pgfpathlineto{\pgfqpoint{5.467906in}{1.063063in}}%
\pgfpathlineto{\pgfqpoint{5.576303in}{0.953646in}}%
\pgfpathlineto{\pgfqpoint{5.705939in}{0.826299in}}%
\pgfpathlineto{\pgfqpoint{5.852359in}{0.686115in}}%
\pgfpathlineto{\pgfqpoint{5.974598in}{0.571603in}}%
\pgfpathlineto{\pgfqpoint{5.974598in}{0.571603in}}%
\pgfusepath{stroke}%
\end{pgfscope}%
\begin{pgfscope}%
\pgfpathrectangle{\pgfqpoint{0.854460in}{0.571603in}}{\pgfqpoint{5.885100in}{5.068436in}}%
\pgfusepath{clip}%
\pgfsetbuttcap%
\pgfsetroundjoin%
\pgfsetlinewidth{1.505625pt}%
\definecolor{currentstroke}{rgb}{0.266580,0.228262,0.514349}%
\pgfsetstrokecolor{currentstroke}%
\pgfsetdash{}{0pt}%
\pgfpathmoveto{\pgfqpoint{2.523741in}{0.571603in}}%
\pgfpathlineto{\pgfqpoint{2.451422in}{0.627154in}}%
\pgfpathlineto{\pgfqpoint{2.392275in}{0.674470in}}%
\pgfpathlineto{\pgfqpoint{2.332323in}{0.724420in}}%
\pgfpathlineto{\pgfqpoint{2.273700in}{0.775360in}}%
\pgfpathlineto{\pgfqpoint{2.214835in}{0.828849in}}%
\pgfpathlineto{\pgfqpoint{2.155689in}{0.885203in}}%
\pgfpathlineto{\pgfqpoint{2.096542in}{0.944431in}}%
\pgfpathlineto{\pgfqpoint{2.039514in}{1.004585in}}%
\pgfpathlineto{\pgfqpoint{1.993724in}{1.055524in}}%
\pgfpathlineto{\pgfqpoint{1.948675in}{1.107994in}}%
\pgfpathlineto{\pgfqpoint{1.888250in}{1.182872in}}%
\pgfpathlineto{\pgfqpoint{1.830381in}{1.260203in}}%
\pgfpathlineto{\pgfqpoint{1.778215in}{1.335689in}}%
\pgfpathlineto{\pgfqpoint{1.741661in}{1.392427in}}%
\pgfpathlineto{\pgfqpoint{1.699324in}{1.463037in}}%
\pgfpathlineto{\pgfqpoint{1.670859in}{1.513976in}}%
\pgfpathlineto{\pgfqpoint{1.644100in}{1.564915in}}%
\pgfpathlineto{\pgfqpoint{1.619002in}{1.615854in}}%
\pgfpathlineto{\pgfqpoint{1.584491in}{1.692262in}}%
\pgfpathlineto{\pgfqpoint{1.553544in}{1.768671in}}%
\pgfpathlineto{\pgfqpoint{1.526073in}{1.845080in}}%
\pgfpathlineto{\pgfqpoint{1.501972in}{1.921488in}}%
\pgfpathlineto{\pgfqpoint{1.481244in}{1.997897in}}%
\pgfpathlineto{\pgfqpoint{1.469223in}{2.048836in}}%
\pgfpathlineto{\pgfqpoint{1.453860in}{2.125244in}}%
\pgfpathlineto{\pgfqpoint{1.441611in}{2.201653in}}%
\pgfpathlineto{\pgfqpoint{1.432530in}{2.278062in}}%
\pgfpathlineto{\pgfqpoint{1.426468in}{2.354470in}}%
\pgfpathlineto{\pgfqpoint{1.423441in}{2.430879in}}%
\pgfpathlineto{\pgfqpoint{1.423456in}{2.507287in}}%
\pgfpathlineto{\pgfqpoint{1.426511in}{2.583696in}}%
\pgfpathlineto{\pgfqpoint{1.432594in}{2.660104in}}%
\pgfpathlineto{\pgfqpoint{1.441680in}{2.736513in}}%
\pgfpathlineto{\pgfqpoint{1.449476in}{2.787452in}}%
\pgfpathlineto{\pgfqpoint{1.463840in}{2.863861in}}%
\pgfpathlineto{\pgfqpoint{1.481311in}{2.940269in}}%
\pgfpathlineto{\pgfqpoint{1.502055in}{3.016678in}}%
\pgfpathlineto{\pgfqpoint{1.517825in}{3.067617in}}%
\pgfpathlineto{\pgfqpoint{1.534977in}{3.118556in}}%
\pgfpathlineto{\pgfqpoint{1.564221in}{3.196000in}}%
\pgfpathlineto{\pgfqpoint{1.596357in}{3.271373in}}%
\pgfpathlineto{\pgfqpoint{1.632833in}{3.347782in}}%
\pgfpathlineto{\pgfqpoint{1.659367in}{3.398721in}}%
\pgfpathlineto{\pgfqpoint{1.687824in}{3.449660in}}%
\pgfpathlineto{\pgfqpoint{1.718303in}{3.500599in}}%
\pgfpathlineto{\pgfqpoint{1.750902in}{3.551538in}}%
\pgfpathlineto{\pgfqpoint{1.785723in}{3.602477in}}%
\pgfpathlineto{\pgfqpoint{1.830381in}{3.663373in}}%
\pgfpathlineto{\pgfqpoint{1.862502in}{3.704355in}}%
\pgfpathlineto{\pgfqpoint{1.904983in}{3.755294in}}%
\pgfpathlineto{\pgfqpoint{1.950213in}{3.806233in}}%
\pgfpathlineto{\pgfqpoint{2.007822in}{3.866299in}}%
\pgfpathlineto{\pgfqpoint{2.066968in}{3.923220in}}%
\pgfpathlineto{\pgfqpoint{2.126115in}{3.975839in}}%
\pgfpathlineto{\pgfqpoint{2.185262in}{4.024562in}}%
\pgfpathlineto{\pgfqpoint{2.244409in}{4.069748in}}%
\pgfpathlineto{\pgfqpoint{2.303790in}{4.111867in}}%
\pgfpathlineto{\pgfqpoint{2.362702in}{4.150489in}}%
\pgfpathlineto{\pgfqpoint{2.424939in}{4.188276in}}%
\pgfpathlineto{\pgfqpoint{2.480996in}{4.219676in}}%
\pgfpathlineto{\pgfqpoint{2.540142in}{4.250227in}}%
\pgfpathlineto{\pgfqpoint{2.599289in}{4.278229in}}%
\pgfpathlineto{\pgfqpoint{2.658436in}{4.303721in}}%
\pgfpathlineto{\pgfqpoint{2.717582in}{4.326729in}}%
\pgfpathlineto{\pgfqpoint{2.776729in}{4.347272in}}%
\pgfpathlineto{\pgfqpoint{2.840371in}{4.366563in}}%
\pgfpathlineto{\pgfqpoint{2.895023in}{4.380757in}}%
\pgfpathlineto{\pgfqpoint{2.954169in}{4.393661in}}%
\pgfpathlineto{\pgfqpoint{3.013316in}{4.403729in}}%
\pgfpathlineto{\pgfqpoint{3.072463in}{4.410989in}}%
\pgfpathlineto{\pgfqpoint{3.131610in}{4.415199in}}%
\pgfpathlineto{\pgfqpoint{3.190756in}{4.416079in}}%
\pgfpathlineto{\pgfqpoint{3.249903in}{4.413310in}}%
\pgfpathlineto{\pgfqpoint{3.309050in}{4.406528in}}%
\pgfpathlineto{\pgfqpoint{3.338623in}{4.401502in}}%
\pgfpathlineto{\pgfqpoint{3.381388in}{4.392032in}}%
\pgfpathlineto{\pgfqpoint{3.397770in}{4.387790in}}%
\pgfpathlineto{\pgfqpoint{3.427343in}{4.378866in}}%
\pgfpathlineto{\pgfqpoint{3.461941in}{4.366563in}}%
\pgfpathlineto{\pgfqpoint{3.486490in}{4.356496in}}%
\pgfpathlineto{\pgfqpoint{3.519392in}{4.341093in}}%
\pgfpathlineto{\pgfqpoint{3.565002in}{4.315624in}}%
\pgfpathlineto{\pgfqpoint{3.575210in}{4.309320in}}%
\pgfpathlineto{\pgfqpoint{3.604783in}{4.289267in}}%
\pgfpathlineto{\pgfqpoint{3.636667in}{4.264685in}}%
\pgfpathlineto{\pgfqpoint{3.665953in}{4.239215in}}%
\pgfpathlineto{\pgfqpoint{3.693504in}{4.212438in}}%
\pgfpathlineto{\pgfqpoint{3.723077in}{4.180220in}}%
\pgfpathlineto{\pgfqpoint{3.757812in}{4.137337in}}%
\pgfpathlineto{\pgfqpoint{3.793851in}{4.086398in}}%
\pgfpathlineto{\pgfqpoint{3.825657in}{4.035459in}}%
\pgfpathlineto{\pgfqpoint{3.854213in}{3.984520in}}%
\pgfpathlineto{\pgfqpoint{3.880260in}{3.933581in}}%
\pgfpathlineto{\pgfqpoint{3.904370in}{3.882642in}}%
\pgfpathlineto{\pgfqpoint{3.937655in}{3.806233in}}%
\pgfpathlineto{\pgfqpoint{3.968576in}{3.729825in}}%
\pgfpathlineto{\pgfqpoint{4.016975in}{3.602477in}}%
\pgfpathlineto{\pgfqpoint{4.054229in}{3.500599in}}%
\pgfpathlineto{\pgfqpoint{4.129014in}{3.296843in}}%
\pgfpathlineto{\pgfqpoint{4.187788in}{3.144025in}}%
\pgfpathlineto{\pgfqpoint{4.239728in}{3.016678in}}%
\pgfpathlineto{\pgfqpoint{4.284971in}{2.912096in}}%
\pgfpathlineto{\pgfqpoint{4.330071in}{2.812922in}}%
\pgfpathlineto{\pgfqpoint{4.379087in}{2.711044in}}%
\pgfpathlineto{\pgfqpoint{4.430876in}{2.609165in}}%
\pgfpathlineto{\pgfqpoint{4.491984in}{2.495726in}}%
\pgfpathlineto{\pgfqpoint{4.528506in}{2.430879in}}%
\pgfpathlineto{\pgfqpoint{4.588514in}{2.329001in}}%
\pgfpathlineto{\pgfqpoint{4.651661in}{2.227123in}}%
\pgfpathlineto{\pgfqpoint{4.718018in}{2.125244in}}%
\pgfpathlineto{\pgfqpoint{4.787718in}{2.023276in}}%
\pgfpathlineto{\pgfqpoint{4.846865in}{1.940209in}}%
\pgfpathlineto{\pgfqpoint{4.906012in}{1.860012in}}%
\pgfpathlineto{\pgfqpoint{4.965158in}{1.782409in}}%
\pgfpathlineto{\pgfqpoint{5.036166in}{1.692262in}}%
\pgfpathlineto{\pgfqpoint{5.119525in}{1.590384in}}%
\pgfpathlineto{\pgfqpoint{5.206055in}{1.488506in}}%
\pgfpathlineto{\pgfqpoint{5.295698in}{1.386628in}}%
\pgfpathlineto{\pgfqpoint{5.388396in}{1.284750in}}%
\pgfpathlineto{\pgfqpoint{5.497479in}{1.168901in}}%
\pgfpathlineto{\pgfqpoint{5.586199in}{1.077473in}}%
\pgfpathlineto{\pgfqpoint{5.684106in}{0.979116in}}%
\pgfpathlineto{\pgfqpoint{5.814594in}{0.851768in}}%
\pgfpathlineto{\pgfqpoint{5.941079in}{0.731792in}}%
\pgfpathlineto{\pgfqpoint{6.059373in}{0.622232in}}%
\pgfpathlineto{\pgfqpoint{6.114810in}{0.571603in}}%
\pgfpathlineto{\pgfqpoint{6.114810in}{0.571603in}}%
\pgfusepath{stroke}%
\end{pgfscope}%
\begin{pgfscope}%
\pgfpathrectangle{\pgfqpoint{0.854460in}{0.571603in}}{\pgfqpoint{5.885100in}{5.068436in}}%
\pgfusepath{clip}%
\pgfsetbuttcap%
\pgfsetroundjoin%
\pgfsetlinewidth{1.505625pt}%
\definecolor{currentstroke}{rgb}{0.257322,0.256130,0.526563}%
\pgfsetstrokecolor{currentstroke}%
\pgfsetdash{}{0pt}%
\pgfpathmoveto{\pgfqpoint{2.408459in}{0.571603in}}%
\pgfpathlineto{\pgfqpoint{2.333129in}{0.630694in}}%
\pgfpathlineto{\pgfqpoint{2.273982in}{0.679082in}}%
\pgfpathlineto{\pgfqpoint{2.214835in}{0.729455in}}%
\pgfpathlineto{\pgfqpoint{2.155689in}{0.782021in}}%
\pgfpathlineto{\pgfqpoint{2.096542in}{0.837003in}}%
\pgfpathlineto{\pgfqpoint{2.029326in}{0.902707in}}%
\pgfpathlineto{\pgfqpoint{1.978248in}{0.955263in}}%
\pgfpathlineto{\pgfqpoint{1.909625in}{1.030055in}}%
\pgfpathlineto{\pgfqpoint{1.859955in}{1.087518in}}%
\pgfpathlineto{\pgfqpoint{1.802978in}{1.157402in}}%
\pgfpathlineto{\pgfqpoint{1.763855in}{1.208341in}}%
\pgfpathlineto{\pgfqpoint{1.712088in}{1.279872in}}%
\pgfpathlineto{\pgfqpoint{1.674253in}{1.335689in}}%
\pgfpathlineto{\pgfqpoint{1.641616in}{1.386628in}}%
\pgfpathlineto{\pgfqpoint{1.610735in}{1.437567in}}%
\pgfpathlineto{\pgfqpoint{1.581573in}{1.488506in}}%
\pgfpathlineto{\pgfqpoint{1.554090in}{1.539445in}}%
\pgfpathlineto{\pgfqpoint{1.528242in}{1.590384in}}%
\pgfpathlineto{\pgfqpoint{1.492531in}{1.666793in}}%
\pgfpathlineto{\pgfqpoint{1.470637in}{1.717732in}}%
\pgfpathlineto{\pgfqpoint{1.440711in}{1.794141in}}%
\pgfpathlineto{\pgfqpoint{1.414113in}{1.870549in}}%
\pgfpathlineto{\pgfqpoint{1.390820in}{1.946958in}}%
\pgfpathlineto{\pgfqpoint{1.377088in}{1.997897in}}%
\pgfpathlineto{\pgfqpoint{1.359004in}{2.074305in}}%
\pgfpathlineto{\pgfqpoint{1.348741in}{2.125244in}}%
\pgfpathlineto{\pgfqpoint{1.335805in}{2.201653in}}%
\pgfpathlineto{\pgfqpoint{1.325816in}{2.278062in}}%
\pgfpathlineto{\pgfqpoint{1.318875in}{2.354470in}}%
\pgfpathlineto{\pgfqpoint{1.314819in}{2.430879in}}%
\pgfpathlineto{\pgfqpoint{1.313656in}{2.507287in}}%
\pgfpathlineto{\pgfqpoint{1.315382in}{2.583696in}}%
\pgfpathlineto{\pgfqpoint{1.319987in}{2.660104in}}%
\pgfpathlineto{\pgfqpoint{1.327634in}{2.738012in}}%
\pgfpathlineto{\pgfqpoint{1.337984in}{2.812922in}}%
\pgfpathlineto{\pgfqpoint{1.351383in}{2.889330in}}%
\pgfpathlineto{\pgfqpoint{1.361996in}{2.940269in}}%
\pgfpathlineto{\pgfqpoint{1.380493in}{3.016678in}}%
\pgfpathlineto{\pgfqpoint{1.394562in}{3.067617in}}%
\pgfpathlineto{\pgfqpoint{1.418274in}{3.144025in}}%
\pgfpathlineto{\pgfqpoint{1.445928in}{3.222040in}}%
\pgfpathlineto{\pgfqpoint{1.475808in}{3.296843in}}%
\pgfpathlineto{\pgfqpoint{1.509930in}{3.373251in}}%
\pgfpathlineto{\pgfqpoint{1.547812in}{3.449660in}}%
\pgfpathlineto{\pgfqpoint{1.575228in}{3.500599in}}%
\pgfpathlineto{\pgfqpoint{1.604490in}{3.551538in}}%
\pgfpathlineto{\pgfqpoint{1.635680in}{3.602477in}}%
\pgfpathlineto{\pgfqpoint{1.668884in}{3.653416in}}%
\pgfpathlineto{\pgfqpoint{1.712088in}{3.715381in}}%
\pgfpathlineto{\pgfqpoint{1.761415in}{3.780764in}}%
\pgfpathlineto{\pgfqpoint{1.802519in}{3.831703in}}%
\pgfpathlineto{\pgfqpoint{1.859955in}{3.897851in}}%
\pgfpathlineto{\pgfqpoint{1.892930in}{3.933581in}}%
\pgfpathlineto{\pgfqpoint{1.948675in}{3.990568in}}%
\pgfpathlineto{\pgfqpoint{2.007822in}{4.046822in}}%
\pgfpathlineto{\pgfqpoint{2.066968in}{4.099226in}}%
\pgfpathlineto{\pgfqpoint{2.126115in}{4.148134in}}%
\pgfpathlineto{\pgfqpoint{2.185262in}{4.193863in}}%
\pgfpathlineto{\pgfqpoint{2.248121in}{4.239215in}}%
\pgfpathlineto{\pgfqpoint{2.303555in}{4.276611in}}%
\pgfpathlineto{\pgfqpoint{2.365231in}{4.315624in}}%
\pgfpathlineto{\pgfqpoint{2.421849in}{4.349035in}}%
\pgfpathlineto{\pgfqpoint{2.480996in}{4.381681in}}%
\pgfpathlineto{\pgfqpoint{2.551170in}{4.417502in}}%
\pgfpathlineto{\pgfqpoint{2.605036in}{4.442971in}}%
\pgfpathlineto{\pgfqpoint{2.663107in}{4.468441in}}%
\pgfpathlineto{\pgfqpoint{2.726489in}{4.493910in}}%
\pgfpathlineto{\pgfqpoint{2.796708in}{4.519380in}}%
\pgfpathlineto{\pgfqpoint{2.835876in}{4.532325in}}%
\pgfpathlineto{\pgfqpoint{2.895023in}{4.550153in}}%
\pgfpathlineto{\pgfqpoint{2.972915in}{4.570319in}}%
\pgfpathlineto{\pgfqpoint{3.013316in}{4.579317in}}%
\pgfpathlineto{\pgfqpoint{3.072463in}{4.590580in}}%
\pgfpathlineto{\pgfqpoint{3.131610in}{4.599469in}}%
\pgfpathlineto{\pgfqpoint{3.190756in}{4.605810in}}%
\pgfpathlineto{\pgfqpoint{3.249903in}{4.609506in}}%
\pgfpathlineto{\pgfqpoint{3.309050in}{4.610308in}}%
\pgfpathlineto{\pgfqpoint{3.368197in}{4.607929in}}%
\pgfpathlineto{\pgfqpoint{3.427343in}{4.602040in}}%
\pgfpathlineto{\pgfqpoint{3.486490in}{4.592165in}}%
\pgfpathlineto{\pgfqpoint{3.516063in}{4.585507in}}%
\pgfpathlineto{\pgfqpoint{3.569663in}{4.570319in}}%
\pgfpathlineto{\pgfqpoint{3.575210in}{4.568541in}}%
\pgfpathlineto{\pgfqpoint{3.604783in}{4.557858in}}%
\pgfpathlineto{\pgfqpoint{3.636289in}{4.544849in}}%
\pgfpathlineto{\pgfqpoint{3.663930in}{4.531712in}}%
\pgfpathlineto{\pgfqpoint{3.693504in}{4.515893in}}%
\pgfpathlineto{\pgfqpoint{3.729076in}{4.493910in}}%
\pgfpathlineto{\pgfqpoint{3.764577in}{4.468441in}}%
\pgfpathlineto{\pgfqpoint{3.795522in}{4.442971in}}%
\pgfpathlineto{\pgfqpoint{3.822915in}{4.417502in}}%
\pgfpathlineto{\pgfqpoint{3.847496in}{4.392032in}}%
\pgfpathlineto{\pgfqpoint{3.870944in}{4.365164in}}%
\pgfpathlineto{\pgfqpoint{3.900517in}{4.326960in}}%
\pgfpathlineto{\pgfqpoint{3.930090in}{4.283549in}}%
\pgfpathlineto{\pgfqpoint{3.959664in}{4.234050in}}%
\pgfpathlineto{\pgfqpoint{3.989237in}{4.177585in}}%
\pgfpathlineto{\pgfqpoint{4.018811in}{4.113390in}}%
\pgfpathlineto{\pgfqpoint{4.040435in}{4.060928in}}%
\pgfpathlineto{\pgfqpoint{4.059863in}{4.009989in}}%
\pgfpathlineto{\pgfqpoint{4.086762in}{3.933581in}}%
\pgfpathlineto{\pgfqpoint{4.119682in}{3.831703in}}%
\pgfpathlineto{\pgfqpoint{4.166677in}{3.676284in}}%
\pgfpathlineto{\pgfqpoint{4.226620in}{3.475129in}}%
\pgfpathlineto{\pgfqpoint{4.266281in}{3.347782in}}%
\pgfpathlineto{\pgfqpoint{4.308516in}{3.220434in}}%
\pgfpathlineto{\pgfqpoint{4.353909in}{3.093086in}}%
\pgfpathlineto{\pgfqpoint{4.392902in}{2.991208in}}%
\pgfpathlineto{\pgfqpoint{4.434557in}{2.889330in}}%
\pgfpathlineto{\pgfqpoint{4.467572in}{2.812922in}}%
\pgfpathlineto{\pgfqpoint{4.514173in}{2.711044in}}%
\pgfpathlineto{\pgfqpoint{4.563818in}{2.609165in}}%
\pgfpathlineto{\pgfqpoint{4.616645in}{2.507287in}}%
\pgfpathlineto{\pgfqpoint{4.669425in}{2.411282in}}%
\pgfpathlineto{\pgfqpoint{4.701965in}{2.354470in}}%
\pgfpathlineto{\pgfqpoint{4.758145in}{2.260528in}}%
\pgfpathlineto{\pgfqpoint{4.794731in}{2.201653in}}%
\pgfpathlineto{\pgfqpoint{4.860797in}{2.099775in}}%
\pgfpathlineto{\pgfqpoint{4.935585in}{1.990419in}}%
\pgfpathlineto{\pgfqpoint{4.994732in}{1.907685in}}%
\pgfpathlineto{\pgfqpoint{5.053878in}{1.827927in}}%
\pgfpathlineto{\pgfqpoint{5.113025in}{1.750846in}}%
\pgfpathlineto{\pgfqpoint{5.172172in}{1.676177in}}%
\pgfpathlineto{\pgfqpoint{5.242315in}{1.590384in}}%
\pgfpathlineto{\pgfqpoint{5.328754in}{1.488506in}}%
\pgfpathlineto{\pgfqpoint{5.418481in}{1.386628in}}%
\pgfpathlineto{\pgfqpoint{5.511446in}{1.284750in}}%
\pgfpathlineto{\pgfqpoint{5.615772in}{1.174389in}}%
\pgfpathlineto{\pgfqpoint{5.706866in}{1.080994in}}%
\pgfpathlineto{\pgfqpoint{5.822786in}{0.965687in}}%
\pgfpathlineto{\pgfqpoint{5.914235in}{0.877238in}}%
\pgfpathlineto{\pgfqpoint{6.049494in}{0.749890in}}%
\pgfpathlineto{\pgfqpoint{6.177666in}{0.632596in}}%
\pgfpathlineto{\pgfqpoint{6.245586in}{0.571603in}}%
\pgfpathlineto{\pgfqpoint{6.245586in}{0.571603in}}%
\pgfusepath{stroke}%
\end{pgfscope}%
\begin{pgfscope}%
\pgfpathrectangle{\pgfqpoint{0.854460in}{0.571603in}}{\pgfqpoint{5.885100in}{5.068436in}}%
\pgfusepath{clip}%
\pgfsetbuttcap%
\pgfsetroundjoin%
\pgfsetlinewidth{1.505625pt}%
\definecolor{currentstroke}{rgb}{0.248629,0.278775,0.534556}%
\pgfsetstrokecolor{currentstroke}%
\pgfsetdash{}{0pt}%
\pgfpathmoveto{\pgfqpoint{2.300862in}{0.571603in}}%
\pgfpathlineto{\pgfqpoint{2.214835in}{0.640546in}}%
\pgfpathlineto{\pgfqpoint{2.145448in}{0.698951in}}%
\pgfpathlineto{\pgfqpoint{2.066968in}{0.768552in}}%
\pgfpathlineto{\pgfqpoint{2.005096in}{0.826299in}}%
\pgfpathlineto{\pgfqpoint{1.948675in}{0.881587in}}%
\pgfpathlineto{\pgfqpoint{1.878991in}{0.953646in}}%
\pgfpathlineto{\pgfqpoint{1.830381in}{1.006687in}}%
\pgfpathlineto{\pgfqpoint{1.766099in}{1.080994in}}%
\pgfpathlineto{\pgfqpoint{1.712088in}{1.147627in}}%
\pgfpathlineto{\pgfqpoint{1.665705in}{1.208341in}}%
\pgfpathlineto{\pgfqpoint{1.623368in}{1.267075in}}%
\pgfpathlineto{\pgfqpoint{1.576976in}{1.335689in}}%
\pgfpathlineto{\pgfqpoint{1.534648in}{1.402849in}}%
\pgfpathlineto{\pgfqpoint{1.499224in}{1.463037in}}%
\pgfpathlineto{\pgfqpoint{1.457630in}{1.539445in}}%
\pgfpathlineto{\pgfqpoint{1.431903in}{1.590384in}}%
\pgfpathlineto{\pgfqpoint{1.407732in}{1.641323in}}%
\pgfpathlineto{\pgfqpoint{1.374381in}{1.717732in}}%
\pgfpathlineto{\pgfqpoint{1.344379in}{1.794141in}}%
\pgfpathlineto{\pgfqpoint{1.317635in}{1.870549in}}%
\pgfpathlineto{\pgfqpoint{1.294045in}{1.946958in}}%
\pgfpathlineto{\pgfqpoint{1.273593in}{2.023366in}}%
\pgfpathlineto{\pgfqpoint{1.261653in}{2.074305in}}%
\pgfpathlineto{\pgfqpoint{1.246213in}{2.150714in}}%
\pgfpathlineto{\pgfqpoint{1.233692in}{2.227123in}}%
\pgfpathlineto{\pgfqpoint{1.224100in}{2.303531in}}%
\pgfpathlineto{\pgfqpoint{1.217308in}{2.379940in}}%
\pgfpathlineto{\pgfqpoint{1.213324in}{2.456348in}}%
\pgfpathlineto{\pgfqpoint{1.212153in}{2.532757in}}%
\pgfpathlineto{\pgfqpoint{1.213789in}{2.609165in}}%
\pgfpathlineto{\pgfqpoint{1.218224in}{2.685574in}}%
\pgfpathlineto{\pgfqpoint{1.225437in}{2.761983in}}%
\pgfpathlineto{\pgfqpoint{1.235403in}{2.838391in}}%
\pgfpathlineto{\pgfqpoint{1.248296in}{2.914800in}}%
\pgfpathlineto{\pgfqpoint{1.264009in}{2.991208in}}%
\pgfpathlineto{\pgfqpoint{1.276157in}{3.042147in}}%
\pgfpathlineto{\pgfqpoint{1.298061in}{3.122887in}}%
\pgfpathlineto{\pgfqpoint{1.320592in}{3.194965in}}%
\pgfpathlineto{\pgfqpoint{1.338197in}{3.245904in}}%
\pgfpathlineto{\pgfqpoint{1.367289in}{3.322312in}}%
\pgfpathlineto{\pgfqpoint{1.399761in}{3.398721in}}%
\pgfpathlineto{\pgfqpoint{1.423327in}{3.449660in}}%
\pgfpathlineto{\pgfqpoint{1.461784in}{3.526068in}}%
\pgfpathlineto{\pgfqpoint{1.489523in}{3.577007in}}%
\pgfpathlineto{\pgfqpoint{1.519033in}{3.627946in}}%
\pgfpathlineto{\pgfqpoint{1.550387in}{3.678886in}}%
\pgfpathlineto{\pgfqpoint{1.593795in}{3.744814in}}%
\pgfpathlineto{\pgfqpoint{1.637408in}{3.806233in}}%
\pgfpathlineto{\pgfqpoint{1.682515in}{3.865648in}}%
\pgfpathlineto{\pgfqpoint{1.738024in}{3.933581in}}%
\pgfpathlineto{\pgfqpoint{1.782622in}{3.984520in}}%
\pgfpathlineto{\pgfqpoint{1.830381in}{4.036106in}}%
\pgfpathlineto{\pgfqpoint{1.889528in}{4.095809in}}%
\pgfpathlineto{\pgfqpoint{1.948675in}{4.151606in}}%
\pgfpathlineto{\pgfqpoint{2.007822in}{4.203858in}}%
\pgfpathlineto{\pgfqpoint{2.066968in}{4.252887in}}%
\pgfpathlineto{\pgfqpoint{2.126115in}{4.298981in}}%
\pgfpathlineto{\pgfqpoint{2.185262in}{4.342401in}}%
\pgfpathlineto{\pgfqpoint{2.257714in}{4.392032in}}%
\pgfpathlineto{\pgfqpoint{2.303555in}{4.421704in}}%
\pgfpathlineto{\pgfqpoint{2.380565in}{4.468441in}}%
\pgfpathlineto{\pgfqpoint{2.425097in}{4.493910in}}%
\pgfpathlineto{\pgfqpoint{2.480996in}{4.524196in}}%
\pgfpathlineto{\pgfqpoint{2.540142in}{4.554342in}}%
\pgfpathlineto{\pgfqpoint{2.599289in}{4.582624in}}%
\pgfpathlineto{\pgfqpoint{2.658436in}{4.609080in}}%
\pgfpathlineto{\pgfqpoint{2.717582in}{4.633741in}}%
\pgfpathlineto{\pgfqpoint{2.776729in}{4.656636in}}%
\pgfpathlineto{\pgfqpoint{2.835876in}{4.677789in}}%
\pgfpathlineto{\pgfqpoint{2.896511in}{4.697667in}}%
\pgfpathlineto{\pgfqpoint{2.983743in}{4.722952in}}%
\pgfpathlineto{\pgfqpoint{3.042890in}{4.737776in}}%
\pgfpathlineto{\pgfqpoint{3.102036in}{4.750858in}}%
\pgfpathlineto{\pgfqpoint{3.161183in}{4.761905in}}%
\pgfpathlineto{\pgfqpoint{3.220330in}{4.771031in}}%
\pgfpathlineto{\pgfqpoint{3.279476in}{4.777972in}}%
\pgfpathlineto{\pgfqpoint{3.338623in}{4.782620in}}%
\pgfpathlineto{\pgfqpoint{3.397770in}{4.784844in}}%
\pgfpathlineto{\pgfqpoint{3.456917in}{4.784407in}}%
\pgfpathlineto{\pgfqpoint{3.516063in}{4.781039in}}%
\pgfpathlineto{\pgfqpoint{3.577472in}{4.774075in}}%
\pgfpathlineto{\pgfqpoint{3.634357in}{4.763918in}}%
\pgfpathlineto{\pgfqpoint{3.693504in}{4.749238in}}%
\pgfpathlineto{\pgfqpoint{3.723077in}{4.739932in}}%
\pgfpathlineto{\pgfqpoint{3.768080in}{4.723136in}}%
\pgfpathlineto{\pgfqpoint{3.782224in}{4.717168in}}%
\pgfpathlineto{\pgfqpoint{3.822770in}{4.697667in}}%
\pgfpathlineto{\pgfqpoint{3.841370in}{4.687559in}}%
\pgfpathlineto{\pgfqpoint{3.870944in}{4.669758in}}%
\pgfpathlineto{\pgfqpoint{3.904254in}{4.646728in}}%
\pgfpathlineto{\pgfqpoint{3.936265in}{4.621258in}}%
\pgfpathlineto{\pgfqpoint{3.964398in}{4.595788in}}%
\pgfpathlineto{\pgfqpoint{3.989449in}{4.570319in}}%
\pgfpathlineto{\pgfqpoint{4.018811in}{4.536316in}}%
\pgfpathlineto{\pgfqpoint{4.050572in}{4.493910in}}%
\pgfpathlineto{\pgfqpoint{4.083043in}{4.442971in}}%
\pgfpathlineto{\pgfqpoint{4.110873in}{4.392032in}}%
\pgfpathlineto{\pgfqpoint{4.137104in}{4.336514in}}%
\pgfpathlineto{\pgfqpoint{4.156342in}{4.290154in}}%
\pgfpathlineto{\pgfqpoint{4.175426in}{4.239215in}}%
\pgfpathlineto{\pgfqpoint{4.200808in}{4.162807in}}%
\pgfpathlineto{\pgfqpoint{4.225824in}{4.077342in}}%
\pgfpathlineto{\pgfqpoint{4.250214in}{3.984520in}}%
\pgfpathlineto{\pgfqpoint{4.275099in}{3.882642in}}%
\pgfpathlineto{\pgfqpoint{4.367028in}{3.500599in}}%
\pgfpathlineto{\pgfqpoint{4.403264in}{3.365873in}}%
\pgfpathlineto{\pgfqpoint{4.430674in}{3.271373in}}%
\pgfpathlineto{\pgfqpoint{4.470805in}{3.144025in}}%
\pgfpathlineto{\pgfqpoint{4.505842in}{3.042147in}}%
\pgfpathlineto{\pgfqpoint{4.543712in}{2.940269in}}%
\pgfpathlineto{\pgfqpoint{4.584560in}{2.838391in}}%
\pgfpathlineto{\pgfqpoint{4.617202in}{2.761983in}}%
\pgfpathlineto{\pgfqpoint{4.651645in}{2.685574in}}%
\pgfpathlineto{\pgfqpoint{4.700500in}{2.583696in}}%
\pgfpathlineto{\pgfqpoint{4.739268in}{2.507287in}}%
\pgfpathlineto{\pgfqpoint{4.794010in}{2.405409in}}%
\pgfpathlineto{\pgfqpoint{4.846865in}{2.312717in}}%
\pgfpathlineto{\pgfqpoint{4.882606in}{2.252592in}}%
\pgfpathlineto{\pgfqpoint{4.935585in}{2.167227in}}%
\pgfpathlineto{\pgfqpoint{4.979094in}{2.099775in}}%
\pgfpathlineto{\pgfqpoint{5.047879in}{1.997897in}}%
\pgfpathlineto{\pgfqpoint{5.120193in}{1.896019in}}%
\pgfpathlineto{\pgfqpoint{5.196066in}{1.794141in}}%
\pgfpathlineto{\pgfqpoint{5.275448in}{1.692262in}}%
\pgfpathlineto{\pgfqpoint{5.358370in}{1.590384in}}%
\pgfpathlineto{\pgfqpoint{5.444771in}{1.488506in}}%
\pgfpathlineto{\pgfqpoint{5.534601in}{1.386628in}}%
\pgfpathlineto{\pgfqpoint{5.627812in}{1.284750in}}%
\pgfpathlineto{\pgfqpoint{5.734066in}{1.172855in}}%
\pgfpathlineto{\pgfqpoint{5.824208in}{1.080994in}}%
\pgfpathlineto{\pgfqpoint{5.927151in}{0.979116in}}%
\pgfpathlineto{\pgfqpoint{6.033240in}{0.877238in}}%
\pgfpathlineto{\pgfqpoint{6.148093in}{0.769988in}}%
\pgfpathlineto{\pgfqpoint{6.266386in}{0.662457in}}%
\pgfpathlineto{\pgfqpoint{6.368678in}{0.571603in}}%
\pgfpathlineto{\pgfqpoint{6.368678in}{0.571603in}}%
\pgfusepath{stroke}%
\end{pgfscope}%
\begin{pgfscope}%
\pgfpathrectangle{\pgfqpoint{0.854460in}{0.571603in}}{\pgfqpoint{5.885100in}{5.068436in}}%
\pgfusepath{clip}%
\pgfsetbuttcap%
\pgfsetroundjoin%
\pgfsetlinewidth{1.505625pt}%
\definecolor{currentstroke}{rgb}{0.239346,0.300855,0.540844}%
\pgfsetstrokecolor{currentstroke}%
\pgfsetdash{}{0pt}%
\pgfpathmoveto{\pgfqpoint{2.199834in}{0.571603in}}%
\pgfpathlineto{\pgfqpoint{2.126115in}{0.631384in}}%
\pgfpathlineto{\pgfqpoint{2.046762in}{0.698951in}}%
\pgfpathlineto{\pgfqpoint{1.978248in}{0.760437in}}%
\pgfpathlineto{\pgfqpoint{1.908475in}{0.826299in}}%
\pgfpathlineto{\pgfqpoint{1.857047in}{0.877238in}}%
\pgfpathlineto{\pgfqpoint{1.800808in}{0.935659in}}%
\pgfpathlineto{\pgfqpoint{1.737889in}{1.004585in}}%
\pgfpathlineto{\pgfqpoint{1.682515in}{1.069018in}}%
\pgfpathlineto{\pgfqpoint{1.631389in}{1.131933in}}%
\pgfpathlineto{\pgfqpoint{1.592106in}{1.182872in}}%
\pgfpathlineto{\pgfqpoint{1.536696in}{1.259281in}}%
\pgfpathlineto{\pgfqpoint{1.502001in}{1.310220in}}%
\pgfpathlineto{\pgfqpoint{1.453222in}{1.386628in}}%
\pgfpathlineto{\pgfqpoint{1.416354in}{1.448740in}}%
\pgfpathlineto{\pgfqpoint{1.380218in}{1.513976in}}%
\pgfpathlineto{\pgfqpoint{1.341235in}{1.590384in}}%
\pgfpathlineto{\pgfqpoint{1.317154in}{1.641323in}}%
\pgfpathlineto{\pgfqpoint{1.283855in}{1.717732in}}%
\pgfpathlineto{\pgfqpoint{1.253832in}{1.794141in}}%
\pgfpathlineto{\pgfqpoint{1.226994in}{1.870549in}}%
\pgfpathlineto{\pgfqpoint{1.203240in}{1.946958in}}%
\pgfpathlineto{\pgfqpoint{1.182508in}{2.023366in}}%
\pgfpathlineto{\pgfqpoint{1.170386in}{2.074305in}}%
\pgfpathlineto{\pgfqpoint{1.154541in}{2.150714in}}%
\pgfpathlineto{\pgfqpoint{1.141602in}{2.227123in}}%
\pgfpathlineto{\pgfqpoint{1.131455in}{2.303531in}}%
\pgfpathlineto{\pgfqpoint{1.124030in}{2.379940in}}%
\pgfpathlineto{\pgfqpoint{1.119360in}{2.456348in}}%
\pgfpathlineto{\pgfqpoint{1.117439in}{2.532757in}}%
\pgfpathlineto{\pgfqpoint{1.118192in}{2.609165in}}%
\pgfpathlineto{\pgfqpoint{1.121631in}{2.685574in}}%
\pgfpathlineto{\pgfqpoint{1.127827in}{2.761983in}}%
\pgfpathlineto{\pgfqpoint{1.136707in}{2.838391in}}%
\pgfpathlineto{\pgfqpoint{1.148241in}{2.914800in}}%
\pgfpathlineto{\pgfqpoint{1.157546in}{2.965739in}}%
\pgfpathlineto{\pgfqpoint{1.173771in}{3.042147in}}%
\pgfpathlineto{\pgfqpoint{1.186178in}{3.093086in}}%
\pgfpathlineto{\pgfqpoint{1.209341in}{3.176792in}}%
\pgfpathlineto{\pgfqpoint{1.231212in}{3.245904in}}%
\pgfpathlineto{\pgfqpoint{1.248919in}{3.296843in}}%
\pgfpathlineto{\pgfqpoint{1.278088in}{3.373251in}}%
\pgfpathlineto{\pgfqpoint{1.310523in}{3.449660in}}%
\pgfpathlineto{\pgfqpoint{1.333993in}{3.500599in}}%
\pgfpathlineto{\pgfqpoint{1.372200in}{3.577007in}}%
\pgfpathlineto{\pgfqpoint{1.399685in}{3.627946in}}%
\pgfpathlineto{\pgfqpoint{1.428867in}{3.678886in}}%
\pgfpathlineto{\pgfqpoint{1.459814in}{3.729825in}}%
\pgfpathlineto{\pgfqpoint{1.492593in}{3.780764in}}%
\pgfpathlineto{\pgfqpoint{1.534648in}{3.842202in}}%
\pgfpathlineto{\pgfqpoint{1.583121in}{3.908111in}}%
\pgfpathlineto{\pgfqpoint{1.623368in}{3.959583in}}%
\pgfpathlineto{\pgfqpoint{1.682515in}{4.030171in}}%
\pgfpathlineto{\pgfqpoint{1.733009in}{4.086398in}}%
\pgfpathlineto{\pgfqpoint{1.781493in}{4.137337in}}%
\pgfpathlineto{\pgfqpoint{1.832725in}{4.188276in}}%
\pgfpathlineto{\pgfqpoint{1.889528in}{4.241483in}}%
\pgfpathlineto{\pgfqpoint{1.948675in}{4.293624in}}%
\pgfpathlineto{\pgfqpoint{2.007822in}{4.342777in}}%
\pgfpathlineto{\pgfqpoint{2.070772in}{4.392032in}}%
\pgfpathlineto{\pgfqpoint{2.140132in}{4.442971in}}%
\pgfpathlineto{\pgfqpoint{2.185262in}{4.474444in}}%
\pgfpathlineto{\pgfqpoint{2.253239in}{4.519380in}}%
\pgfpathlineto{\pgfqpoint{2.333129in}{4.568798in}}%
\pgfpathlineto{\pgfqpoint{2.392275in}{4.603044in}}%
\pgfpathlineto{\pgfqpoint{2.472648in}{4.646728in}}%
\pgfpathlineto{\pgfqpoint{2.540142in}{4.680940in}}%
\pgfpathlineto{\pgfqpoint{2.599289in}{4.709142in}}%
\pgfpathlineto{\pgfqpoint{2.658436in}{4.735739in}}%
\pgfpathlineto{\pgfqpoint{2.717582in}{4.760764in}}%
\pgfpathlineto{\pgfqpoint{2.776729in}{4.784247in}}%
\pgfpathlineto{\pgfqpoint{2.835876in}{4.806215in}}%
\pgfpathlineto{\pgfqpoint{2.924596in}{4.836287in}}%
\pgfpathlineto{\pgfqpoint{2.983743in}{4.854483in}}%
\pgfpathlineto{\pgfqpoint{3.072463in}{4.878854in}}%
\pgfpathlineto{\pgfqpoint{3.161183in}{4.899683in}}%
\pgfpathlineto{\pgfqpoint{3.220330in}{4.911462in}}%
\pgfpathlineto{\pgfqpoint{3.279476in}{4.921616in}}%
\pgfpathlineto{\pgfqpoint{3.338623in}{4.929977in}}%
\pgfpathlineto{\pgfqpoint{3.397770in}{4.936389in}}%
\pgfpathlineto{\pgfqpoint{3.456917in}{4.940826in}}%
\pgfpathlineto{\pgfqpoint{3.516063in}{4.943102in}}%
\pgfpathlineto{\pgfqpoint{3.575210in}{4.943010in}}%
\pgfpathlineto{\pgfqpoint{3.634357in}{4.940307in}}%
\pgfpathlineto{\pgfqpoint{3.693504in}{4.934713in}}%
\pgfpathlineto{\pgfqpoint{3.752650in}{4.925877in}}%
\pgfpathlineto{\pgfqpoint{3.811797in}{4.913099in}}%
\pgfpathlineto{\pgfqpoint{3.853806in}{4.901423in}}%
\pgfpathlineto{\pgfqpoint{3.900517in}{4.885433in}}%
\pgfpathlineto{\pgfqpoint{3.930090in}{4.873507in}}%
\pgfpathlineto{\pgfqpoint{3.977959in}{4.850484in}}%
\pgfpathlineto{\pgfqpoint{3.989237in}{4.844388in}}%
\pgfpathlineto{\pgfqpoint{4.021760in}{4.825014in}}%
\pgfpathlineto{\pgfqpoint{4.058440in}{4.799545in}}%
\pgfpathlineto{\pgfqpoint{4.090096in}{4.774075in}}%
\pgfpathlineto{\pgfqpoint{4.117805in}{4.748606in}}%
\pgfpathlineto{\pgfqpoint{4.142367in}{4.723136in}}%
\pgfpathlineto{\pgfqpoint{4.166677in}{4.694740in}}%
\pgfpathlineto{\pgfqpoint{4.196251in}{4.654960in}}%
\pgfpathlineto{\pgfqpoint{4.218060in}{4.621258in}}%
\pgfpathlineto{\pgfqpoint{4.232928in}{4.595788in}}%
\pgfpathlineto{\pgfqpoint{4.259187in}{4.544849in}}%
\pgfpathlineto{\pgfqpoint{4.284971in}{4.485642in}}%
\pgfpathlineto{\pgfqpoint{4.301007in}{4.442971in}}%
\pgfpathlineto{\pgfqpoint{4.318101in}{4.392032in}}%
\pgfpathlineto{\pgfqpoint{4.340179in}{4.315624in}}%
\pgfpathlineto{\pgfqpoint{4.353074in}{4.264685in}}%
\pgfpathlineto{\pgfqpoint{4.373691in}{4.173452in}}%
\pgfpathlineto{\pgfqpoint{4.395967in}{4.060928in}}%
\pgfpathlineto{\pgfqpoint{4.423776in}{3.908111in}}%
\pgfpathlineto{\pgfqpoint{4.466010in}{3.678886in}}%
\pgfpathlineto{\pgfqpoint{4.491984in}{3.551359in}}%
\pgfpathlineto{\pgfqpoint{4.514644in}{3.449660in}}%
\pgfpathlineto{\pgfqpoint{4.539528in}{3.347782in}}%
\pgfpathlineto{\pgfqpoint{4.566859in}{3.245904in}}%
\pgfpathlineto{\pgfqpoint{4.596864in}{3.144025in}}%
\pgfpathlineto{\pgfqpoint{4.629743in}{3.042147in}}%
\pgfpathlineto{\pgfqpoint{4.665675in}{2.940269in}}%
\pgfpathlineto{\pgfqpoint{4.704746in}{2.838391in}}%
\pgfpathlineto{\pgfqpoint{4.736204in}{2.761983in}}%
\pgfpathlineto{\pgfqpoint{4.769553in}{2.685574in}}%
\pgfpathlineto{\pgfqpoint{4.804840in}{2.609165in}}%
\pgfpathlineto{\pgfqpoint{4.854954in}{2.507287in}}%
\pgfpathlineto{\pgfqpoint{4.894848in}{2.430879in}}%
\pgfpathlineto{\pgfqpoint{4.951184in}{2.329001in}}%
\pgfpathlineto{\pgfqpoint{5.011142in}{2.227123in}}%
\pgfpathlineto{\pgfqpoint{5.074752in}{2.125244in}}%
\pgfpathlineto{\pgfqpoint{5.142599in}{2.022568in}}%
\pgfpathlineto{\pgfqpoint{5.201745in}{1.937307in}}%
\pgfpathlineto{\pgfqpoint{5.260892in}{1.855507in}}%
\pgfpathlineto{\pgfqpoint{5.320039in}{1.776784in}}%
\pgfpathlineto{\pgfqpoint{5.379185in}{1.700812in}}%
\pgfpathlineto{\pgfqpoint{5.438332in}{1.627308in}}%
\pgfpathlineto{\pgfqpoint{5.497479in}{1.556034in}}%
\pgfpathlineto{\pgfqpoint{5.556626in}{1.486786in}}%
\pgfpathlineto{\pgfqpoint{5.645083in}{1.386628in}}%
\pgfpathlineto{\pgfqpoint{5.738526in}{1.284750in}}%
\pgfpathlineto{\pgfqpoint{5.835431in}{1.182872in}}%
\pgfpathlineto{\pgfqpoint{5.941079in}{1.075714in}}%
\pgfpathlineto{\pgfqpoint{6.039406in}{0.979116in}}%
\pgfpathlineto{\pgfqpoint{6.148093in}{0.875578in}}%
\pgfpathlineto{\pgfqpoint{6.266386in}{0.766185in}}%
\pgfpathlineto{\pgfqpoint{6.384680in}{0.659889in}}%
\pgfpathlineto{\pgfqpoint{6.485327in}{0.571603in}}%
\pgfpathlineto{\pgfqpoint{6.485327in}{0.571603in}}%
\pgfusepath{stroke}%
\end{pgfscope}%
\begin{pgfscope}%
\pgfpathrectangle{\pgfqpoint{0.854460in}{0.571603in}}{\pgfqpoint{5.885100in}{5.068436in}}%
\pgfusepath{clip}%
\pgfsetbuttcap%
\pgfsetroundjoin%
\pgfsetlinewidth{1.505625pt}%
\definecolor{currentstroke}{rgb}{0.227802,0.326594,0.546532}%
\pgfsetstrokecolor{currentstroke}%
\pgfsetdash{}{0pt}%
\pgfpathmoveto{\pgfqpoint{2.104286in}{0.571603in}}%
\pgfpathlineto{\pgfqpoint{2.037395in}{0.626521in}}%
\pgfpathlineto{\pgfqpoint{1.978248in}{0.677095in}}%
\pgfpathlineto{\pgfqpoint{1.919102in}{0.729783in}}%
\pgfpathlineto{\pgfqpoint{1.843163in}{0.800829in}}%
\pgfpathlineto{\pgfqpoint{1.771235in}{0.872141in}}%
\pgfpathlineto{\pgfqpoint{1.717617in}{0.928177in}}%
\pgfpathlineto{\pgfqpoint{1.671078in}{0.979116in}}%
\pgfpathlineto{\pgfqpoint{1.623368in}{1.033725in}}%
\pgfpathlineto{\pgfqpoint{1.563287in}{1.106463in}}%
\pgfpathlineto{\pgfqpoint{1.504338in}{1.182872in}}%
\pgfpathlineto{\pgfqpoint{1.449491in}{1.259281in}}%
\pgfpathlineto{\pgfqpoint{1.415088in}{1.310220in}}%
\pgfpathlineto{\pgfqpoint{1.366741in}{1.386628in}}%
\pgfpathlineto{\pgfqpoint{1.327634in}{1.453142in}}%
\pgfpathlineto{\pgfqpoint{1.294223in}{1.513976in}}%
\pgfpathlineto{\pgfqpoint{1.255463in}{1.590384in}}%
\pgfpathlineto{\pgfqpoint{1.220046in}{1.666793in}}%
\pgfpathlineto{\pgfqpoint{1.198266in}{1.717732in}}%
\pgfpathlineto{\pgfqpoint{1.168264in}{1.794141in}}%
\pgfpathlineto{\pgfqpoint{1.141372in}{1.870549in}}%
\pgfpathlineto{\pgfqpoint{1.117491in}{1.946958in}}%
\pgfpathlineto{\pgfqpoint{1.096614in}{2.023366in}}%
\pgfpathlineto{\pgfqpoint{1.078636in}{2.099775in}}%
\pgfpathlineto{\pgfqpoint{1.063407in}{2.176183in}}%
\pgfpathlineto{\pgfqpoint{1.051049in}{2.252592in}}%
\pgfpathlineto{\pgfqpoint{1.041366in}{2.329001in}}%
\pgfpathlineto{\pgfqpoint{1.034333in}{2.405409in}}%
\pgfpathlineto{\pgfqpoint{1.029993in}{2.481818in}}%
\pgfpathlineto{\pgfqpoint{1.028305in}{2.558226in}}%
\pgfpathlineto{\pgfqpoint{1.029217in}{2.634635in}}%
\pgfpathlineto{\pgfqpoint{1.032734in}{2.711044in}}%
\pgfpathlineto{\pgfqpoint{1.038933in}{2.787452in}}%
\pgfpathlineto{\pgfqpoint{1.047737in}{2.863861in}}%
\pgfpathlineto{\pgfqpoint{1.059120in}{2.940269in}}%
\pgfpathlineto{\pgfqpoint{1.073293in}{3.016678in}}%
\pgfpathlineto{\pgfqpoint{1.091047in}{3.097090in}}%
\pgfpathlineto{\pgfqpoint{1.109776in}{3.169495in}}%
\pgfpathlineto{\pgfqpoint{1.132262in}{3.245904in}}%
\pgfpathlineto{\pgfqpoint{1.157660in}{3.322312in}}%
\pgfpathlineto{\pgfqpoint{1.186092in}{3.398721in}}%
\pgfpathlineto{\pgfqpoint{1.217675in}{3.475129in}}%
\pgfpathlineto{\pgfqpoint{1.252523in}{3.551538in}}%
\pgfpathlineto{\pgfqpoint{1.277617in}{3.602477in}}%
\pgfpathlineto{\pgfqpoint{1.318222in}{3.678886in}}%
\pgfpathlineto{\pgfqpoint{1.357208in}{3.746509in}}%
\pgfpathlineto{\pgfqpoint{1.394215in}{3.806233in}}%
\pgfpathlineto{\pgfqpoint{1.427737in}{3.857172in}}%
\pgfpathlineto{\pgfqpoint{1.475501in}{3.925339in}}%
\pgfpathlineto{\pgfqpoint{1.519941in}{3.984520in}}%
\pgfpathlineto{\pgfqpoint{1.564221in}{4.040151in}}%
\pgfpathlineto{\pgfqpoint{1.625315in}{4.111867in}}%
\pgfpathlineto{\pgfqpoint{1.682515in}{4.174390in}}%
\pgfpathlineto{\pgfqpoint{1.741661in}{4.235043in}}%
\pgfpathlineto{\pgfqpoint{1.800808in}{4.292002in}}%
\pgfpathlineto{\pgfqpoint{1.859955in}{4.345618in}}%
\pgfpathlineto{\pgfqpoint{1.919102in}{4.396265in}}%
\pgfpathlineto{\pgfqpoint{1.978248in}{4.444205in}}%
\pgfpathlineto{\pgfqpoint{2.043210in}{4.493910in}}%
\pgfpathlineto{\pgfqpoint{2.113884in}{4.544849in}}%
\pgfpathlineto{\pgfqpoint{2.185262in}{4.593298in}}%
\pgfpathlineto{\pgfqpoint{2.244409in}{4.631190in}}%
\pgfpathlineto{\pgfqpoint{2.311840in}{4.672197in}}%
\pgfpathlineto{\pgfqpoint{2.392275in}{4.718163in}}%
\pgfpathlineto{\pgfqpoint{2.451422in}{4.750047in}}%
\pgfpathlineto{\pgfqpoint{2.540142in}{4.794887in}}%
\pgfpathlineto{\pgfqpoint{2.603777in}{4.825014in}}%
\pgfpathlineto{\pgfqpoint{2.688009in}{4.862294in}}%
\pgfpathlineto{\pgfqpoint{2.776729in}{4.898607in}}%
\pgfpathlineto{\pgfqpoint{2.851808in}{4.926892in}}%
\pgfpathlineto{\pgfqpoint{2.924607in}{4.952362in}}%
\pgfpathlineto{\pgfqpoint{3.013316in}{4.980581in}}%
\pgfpathlineto{\pgfqpoint{3.102036in}{5.005839in}}%
\pgfpathlineto{\pgfqpoint{3.193847in}{5.028770in}}%
\pgfpathlineto{\pgfqpoint{3.279476in}{5.047071in}}%
\pgfpathlineto{\pgfqpoint{3.338623in}{5.057983in}}%
\pgfpathlineto{\pgfqpoint{3.427343in}{5.071405in}}%
\pgfpathlineto{\pgfqpoint{3.500161in}{5.079709in}}%
\pgfpathlineto{\pgfqpoint{3.545637in}{5.083564in}}%
\pgfpathlineto{\pgfqpoint{3.604783in}{5.086865in}}%
\pgfpathlineto{\pgfqpoint{3.663930in}{5.088166in}}%
\pgfpathlineto{\pgfqpoint{3.723077in}{5.087271in}}%
\pgfpathlineto{\pgfqpoint{3.782224in}{5.083956in}}%
\pgfpathlineto{\pgfqpoint{3.841370in}{5.077906in}}%
\pgfpathlineto{\pgfqpoint{3.900517in}{5.068645in}}%
\pgfpathlineto{\pgfqpoint{3.959664in}{5.055929in}}%
\pgfpathlineto{\pgfqpoint{3.989237in}{5.047941in}}%
\pgfpathlineto{\pgfqpoint{4.048384in}{5.028555in}}%
\pgfpathlineto{\pgfqpoint{4.077957in}{5.016674in}}%
\pgfpathlineto{\pgfqpoint{4.107787in}{5.003301in}}%
\pgfpathlineto{\pgfqpoint{4.155426in}{4.977831in}}%
\pgfpathlineto{\pgfqpoint{4.166677in}{4.971066in}}%
\pgfpathlineto{\pgfqpoint{4.196251in}{4.951678in}}%
\pgfpathlineto{\pgfqpoint{4.229038in}{4.926892in}}%
\pgfpathlineto{\pgfqpoint{4.258340in}{4.901423in}}%
\pgfpathlineto{\pgfqpoint{4.284971in}{4.875012in}}%
\pgfpathlineto{\pgfqpoint{4.314544in}{4.841084in}}%
\pgfpathlineto{\pgfqpoint{4.344118in}{4.801478in}}%
\pgfpathlineto{\pgfqpoint{4.361842in}{4.774075in}}%
\pgfpathlineto{\pgfqpoint{4.376852in}{4.748606in}}%
\pgfpathlineto{\pgfqpoint{4.403264in}{4.696895in}}%
\pgfpathlineto{\pgfqpoint{4.424655in}{4.646728in}}%
\pgfpathlineto{\pgfqpoint{4.443142in}{4.595788in}}%
\pgfpathlineto{\pgfqpoint{4.462411in}{4.532870in}}%
\pgfpathlineto{\pgfqpoint{4.472658in}{4.493910in}}%
\pgfpathlineto{\pgfqpoint{4.490084in}{4.417502in}}%
\pgfpathlineto{\pgfqpoint{4.499991in}{4.366563in}}%
\pgfpathlineto{\pgfqpoint{4.513101in}{4.290154in}}%
\pgfpathlineto{\pgfqpoint{4.528253in}{4.188276in}}%
\pgfpathlineto{\pgfqpoint{4.551739in}{4.009989in}}%
\pgfpathlineto{\pgfqpoint{4.575227in}{3.831703in}}%
\pgfpathlineto{\pgfqpoint{4.593905in}{3.704355in}}%
\pgfpathlineto{\pgfqpoint{4.610752in}{3.602477in}}%
\pgfpathlineto{\pgfqpoint{4.629524in}{3.500599in}}%
\pgfpathlineto{\pgfqpoint{4.650674in}{3.398721in}}%
\pgfpathlineto{\pgfqpoint{4.674431in}{3.296843in}}%
\pgfpathlineto{\pgfqpoint{4.698998in}{3.202361in}}%
\pgfpathlineto{\pgfqpoint{4.715330in}{3.144025in}}%
\pgfpathlineto{\pgfqpoint{4.746413in}{3.042147in}}%
\pgfpathlineto{\pgfqpoint{4.780720in}{2.940269in}}%
\pgfpathlineto{\pgfqpoint{4.818378in}{2.838391in}}%
\pgfpathlineto{\pgfqpoint{4.848845in}{2.761983in}}%
\pgfpathlineto{\pgfqpoint{4.881272in}{2.685574in}}%
\pgfpathlineto{\pgfqpoint{4.915700in}{2.609165in}}%
\pgfpathlineto{\pgfqpoint{4.952168in}{2.532757in}}%
\pgfpathlineto{\pgfqpoint{5.004000in}{2.430879in}}%
\pgfpathlineto{\pgfqpoint{5.053878in}{2.339160in}}%
\pgfpathlineto{\pgfqpoint{5.088685in}{2.278062in}}%
\pgfpathlineto{\pgfqpoint{5.142599in}{2.187948in}}%
\pgfpathlineto{\pgfqpoint{5.181750in}{2.125244in}}%
\pgfpathlineto{\pgfqpoint{5.231500in}{2.048836in}}%
\pgfpathlineto{\pgfqpoint{5.290465in}{1.962171in}}%
\pgfpathlineto{\pgfqpoint{5.349612in}{1.878979in}}%
\pgfpathlineto{\pgfqpoint{5.393299in}{1.819610in}}%
\pgfpathlineto{\pgfqpoint{5.471361in}{1.717732in}}%
\pgfpathlineto{\pgfqpoint{5.532333in}{1.641323in}}%
\pgfpathlineto{\pgfqpoint{5.616910in}{1.539445in}}%
\pgfpathlineto{\pgfqpoint{5.704492in}{1.438323in}}%
\pgfpathlineto{\pgfqpoint{5.773700in}{1.361159in}}%
\pgfpathlineto{\pgfqpoint{5.868295in}{1.259281in}}%
\pgfpathlineto{\pgfqpoint{5.970653in}{1.153167in}}%
\pgfpathlineto{\pgfqpoint{6.068131in}{1.055524in}}%
\pgfpathlineto{\pgfqpoint{6.177666in}{0.949442in}}%
\pgfpathlineto{\pgfqpoint{6.281681in}{0.851768in}}%
\pgfpathlineto{\pgfqpoint{6.393411in}{0.749890in}}%
\pgfpathlineto{\pgfqpoint{6.508309in}{0.648012in}}%
\pgfpathlineto{\pgfqpoint{6.596482in}{0.571603in}}%
\pgfpathlineto{\pgfqpoint{6.596482in}{0.571603in}}%
\pgfusepath{stroke}%
\end{pgfscope}%
\begin{pgfscope}%
\pgfpathrectangle{\pgfqpoint{0.854460in}{0.571603in}}{\pgfqpoint{5.885100in}{5.068436in}}%
\pgfusepath{clip}%
\pgfsetbuttcap%
\pgfsetroundjoin%
\pgfsetlinewidth{1.505625pt}%
\definecolor{currentstroke}{rgb}{0.218130,0.347432,0.550038}%
\pgfsetstrokecolor{currentstroke}%
\pgfsetdash{}{0pt}%
\pgfpathmoveto{\pgfqpoint{2.013550in}{0.571603in}}%
\pgfpathlineto{\pgfqpoint{1.948675in}{0.625575in}}%
\pgfpathlineto{\pgfqpoint{1.889528in}{0.676811in}}%
\pgfpathlineto{\pgfqpoint{1.809166in}{0.749890in}}%
\pgfpathlineto{\pgfqpoint{1.741661in}{0.814778in}}%
\pgfpathlineto{\pgfqpoint{1.679854in}{0.877238in}}%
\pgfpathlineto{\pgfqpoint{1.623368in}{0.937421in}}%
\pgfpathlineto{\pgfqpoint{1.563505in}{1.004585in}}%
\pgfpathlineto{\pgfqpoint{1.499613in}{1.080994in}}%
\pgfpathlineto{\pgfqpoint{1.439902in}{1.157402in}}%
\pgfpathlineto{\pgfqpoint{1.384219in}{1.233811in}}%
\pgfpathlineto{\pgfqpoint{1.332476in}{1.310220in}}%
\pgfpathlineto{\pgfqpoint{1.298061in}{1.364395in}}%
\pgfpathlineto{\pgfqpoint{1.254544in}{1.437567in}}%
\pgfpathlineto{\pgfqpoint{1.226152in}{1.488506in}}%
\pgfpathlineto{\pgfqpoint{1.186426in}{1.564915in}}%
\pgfpathlineto{\pgfqpoint{1.161842in}{1.615854in}}%
\pgfpathlineto{\pgfqpoint{1.138700in}{1.666793in}}%
\pgfpathlineto{\pgfqpoint{1.106659in}{1.743202in}}%
\pgfpathlineto{\pgfqpoint{1.077734in}{1.819610in}}%
\pgfpathlineto{\pgfqpoint{1.051830in}{1.896019in}}%
\pgfpathlineto{\pgfqpoint{1.028848in}{1.972427in}}%
\pgfpathlineto{\pgfqpoint{1.008794in}{2.048836in}}%
\pgfpathlineto{\pgfqpoint{0.991544in}{2.125244in}}%
\pgfpathlineto{\pgfqpoint{0.976994in}{2.201653in}}%
\pgfpathlineto{\pgfqpoint{0.965190in}{2.278062in}}%
\pgfpathlineto{\pgfqpoint{0.956035in}{2.354470in}}%
\pgfpathlineto{\pgfqpoint{0.949458in}{2.430879in}}%
\pgfpathlineto{\pgfqpoint{0.945462in}{2.507287in}}%
\pgfpathlineto{\pgfqpoint{0.944046in}{2.583696in}}%
\pgfpathlineto{\pgfqpoint{0.945205in}{2.660104in}}%
\pgfpathlineto{\pgfqpoint{0.948927in}{2.736513in}}%
\pgfpathlineto{\pgfqpoint{0.955198in}{2.812922in}}%
\pgfpathlineto{\pgfqpoint{0.963998in}{2.889330in}}%
\pgfpathlineto{\pgfqpoint{0.975354in}{2.965739in}}%
\pgfpathlineto{\pgfqpoint{0.989412in}{3.042147in}}%
\pgfpathlineto{\pgfqpoint{1.006036in}{3.118556in}}%
\pgfpathlineto{\pgfqpoint{1.025427in}{3.194965in}}%
\pgfpathlineto{\pgfqpoint{1.047595in}{3.271373in}}%
\pgfpathlineto{\pgfqpoint{1.072580in}{3.347782in}}%
\pgfpathlineto{\pgfqpoint{1.100498in}{3.424190in}}%
\pgfpathlineto{\pgfqpoint{1.131463in}{3.500599in}}%
\pgfpathlineto{\pgfqpoint{1.165583in}{3.577007in}}%
\pgfpathlineto{\pgfqpoint{1.190138in}{3.627946in}}%
\pgfpathlineto{\pgfqpoint{1.229814in}{3.704355in}}%
\pgfpathlineto{\pgfqpoint{1.268488in}{3.773045in}}%
\pgfpathlineto{\pgfqpoint{1.303931in}{3.831703in}}%
\pgfpathlineto{\pgfqpoint{1.353521in}{3.908111in}}%
\pgfpathlineto{\pgfqpoint{1.388855in}{3.959050in}}%
\pgfpathlineto{\pgfqpoint{1.445928in}{4.036048in}}%
\pgfpathlineto{\pgfqpoint{1.506726in}{4.111867in}}%
\pgfpathlineto{\pgfqpoint{1.564221in}{4.178531in}}%
\pgfpathlineto{\pgfqpoint{1.619995in}{4.239215in}}%
\pgfpathlineto{\pgfqpoint{1.669622in}{4.290154in}}%
\pgfpathlineto{\pgfqpoint{1.721849in}{4.341093in}}%
\pgfpathlineto{\pgfqpoint{1.776889in}{4.392032in}}%
\pgfpathlineto{\pgfqpoint{1.834963in}{4.442971in}}%
\pgfpathlineto{\pgfqpoint{1.896301in}{4.493910in}}%
\pgfpathlineto{\pgfqpoint{1.961135in}{4.544849in}}%
\pgfpathlineto{\pgfqpoint{2.029706in}{4.595788in}}%
\pgfpathlineto{\pgfqpoint{2.096542in}{4.642736in}}%
\pgfpathlineto{\pgfqpoint{2.155689in}{4.682169in}}%
\pgfpathlineto{\pgfqpoint{2.220144in}{4.723136in}}%
\pgfpathlineto{\pgfqpoint{2.305151in}{4.774075in}}%
\pgfpathlineto{\pgfqpoint{2.396281in}{4.825014in}}%
\pgfpathlineto{\pgfqpoint{2.480996in}{4.869213in}}%
\pgfpathlineto{\pgfqpoint{2.546417in}{4.901423in}}%
\pgfpathlineto{\pgfqpoint{2.628862in}{4.939639in}}%
\pgfpathlineto{\pgfqpoint{2.717582in}{4.978090in}}%
\pgfpathlineto{\pgfqpoint{2.806303in}{5.013622in}}%
\pgfpathlineto{\pgfqpoint{2.895023in}{5.046528in}}%
\pgfpathlineto{\pgfqpoint{2.983743in}{5.076803in}}%
\pgfpathlineto{\pgfqpoint{3.075024in}{5.105179in}}%
\pgfpathlineto{\pgfqpoint{3.165999in}{5.130649in}}%
\pgfpathlineto{\pgfqpoint{3.249903in}{5.151631in}}%
\pgfpathlineto{\pgfqpoint{3.338623in}{5.171121in}}%
\pgfpathlineto{\pgfqpoint{3.427343in}{5.187793in}}%
\pgfpathlineto{\pgfqpoint{3.516063in}{5.201424in}}%
\pgfpathlineto{\pgfqpoint{3.575210in}{5.208760in}}%
\pgfpathlineto{\pgfqpoint{3.634357in}{5.214484in}}%
\pgfpathlineto{\pgfqpoint{3.693504in}{5.218619in}}%
\pgfpathlineto{\pgfqpoint{3.752650in}{5.221024in}}%
\pgfpathlineto{\pgfqpoint{3.811797in}{5.221534in}}%
\pgfpathlineto{\pgfqpoint{3.870944in}{5.219963in}}%
\pgfpathlineto{\pgfqpoint{3.930090in}{5.216096in}}%
\pgfpathlineto{\pgfqpoint{3.989237in}{5.209683in}}%
\pgfpathlineto{\pgfqpoint{4.048384in}{5.200219in}}%
\pgfpathlineto{\pgfqpoint{4.107531in}{5.187378in}}%
\pgfpathlineto{\pgfqpoint{4.137104in}{5.179525in}}%
\pgfpathlineto{\pgfqpoint{4.196251in}{5.160270in}}%
\pgfpathlineto{\pgfqpoint{4.225824in}{5.148659in}}%
\pgfpathlineto{\pgfqpoint{4.265634in}{5.130649in}}%
\pgfpathlineto{\pgfqpoint{4.284971in}{5.120731in}}%
\pgfpathlineto{\pgfqpoint{4.314544in}{5.104053in}}%
\pgfpathlineto{\pgfqpoint{4.351651in}{5.079709in}}%
\pgfpathlineto{\pgfqpoint{4.384920in}{5.054240in}}%
\pgfpathlineto{\pgfqpoint{4.413745in}{5.028770in}}%
\pgfpathlineto{\pgfqpoint{4.439011in}{5.003301in}}%
\pgfpathlineto{\pgfqpoint{4.462411in}{4.976540in}}%
\pgfpathlineto{\pgfqpoint{4.491984in}{4.937024in}}%
\pgfpathlineto{\pgfqpoint{4.514721in}{4.901423in}}%
\pgfpathlineto{\pgfqpoint{4.529049in}{4.875953in}}%
\pgfpathlineto{\pgfqpoint{4.553787in}{4.825014in}}%
\pgfpathlineto{\pgfqpoint{4.574124in}{4.774075in}}%
\pgfpathlineto{\pgfqpoint{4.591037in}{4.723136in}}%
\pgfpathlineto{\pgfqpoint{4.605228in}{4.672197in}}%
\pgfpathlineto{\pgfqpoint{4.617135in}{4.621258in}}%
\pgfpathlineto{\pgfqpoint{4.627209in}{4.570319in}}%
\pgfpathlineto{\pgfqpoint{4.639851in}{4.493070in}}%
\pgfpathlineto{\pgfqpoint{4.649695in}{4.417502in}}%
\pgfpathlineto{\pgfqpoint{4.660514in}{4.315624in}}%
\pgfpathlineto{\pgfqpoint{4.673836in}{4.162807in}}%
\pgfpathlineto{\pgfqpoint{4.695557in}{3.908111in}}%
\pgfpathlineto{\pgfqpoint{4.708738in}{3.780764in}}%
\pgfpathlineto{\pgfqpoint{4.721254in}{3.678886in}}%
\pgfpathlineto{\pgfqpoint{4.735866in}{3.577007in}}%
\pgfpathlineto{\pgfqpoint{4.752885in}{3.475129in}}%
\pgfpathlineto{\pgfqpoint{4.772520in}{3.373251in}}%
\pgfpathlineto{\pgfqpoint{4.795050in}{3.271373in}}%
\pgfpathlineto{\pgfqpoint{4.817291in}{3.182280in}}%
\pgfpathlineto{\pgfqpoint{4.834564in}{3.118556in}}%
\pgfpathlineto{\pgfqpoint{4.857055in}{3.042147in}}%
\pgfpathlineto{\pgfqpoint{4.881449in}{2.965739in}}%
\pgfpathlineto{\pgfqpoint{4.907779in}{2.889330in}}%
\pgfpathlineto{\pgfqpoint{4.936077in}{2.812922in}}%
\pgfpathlineto{\pgfqpoint{4.966376in}{2.736513in}}%
\pgfpathlineto{\pgfqpoint{4.998709in}{2.660104in}}%
\pgfpathlineto{\pgfqpoint{5.033111in}{2.583696in}}%
\pgfpathlineto{\pgfqpoint{5.069618in}{2.507287in}}%
\pgfpathlineto{\pgfqpoint{5.113025in}{2.421837in}}%
\pgfpathlineto{\pgfqpoint{5.149017in}{2.354470in}}%
\pgfpathlineto{\pgfqpoint{5.201745in}{2.261167in}}%
\pgfpathlineto{\pgfqpoint{5.236975in}{2.201653in}}%
\pgfpathlineto{\pgfqpoint{5.290465in}{2.115418in}}%
\pgfpathlineto{\pgfqpoint{5.333553in}{2.048836in}}%
\pgfpathlineto{\pgfqpoint{5.402781in}{1.946958in}}%
\pgfpathlineto{\pgfqpoint{5.475839in}{1.845080in}}%
\pgfpathlineto{\pgfqpoint{5.552762in}{1.743202in}}%
\pgfpathlineto{\pgfqpoint{5.633479in}{1.641323in}}%
\pgfpathlineto{\pgfqpoint{5.718023in}{1.539445in}}%
\pgfpathlineto{\pgfqpoint{5.806348in}{1.437567in}}%
\pgfpathlineto{\pgfqpoint{5.898417in}{1.335689in}}%
\pgfpathlineto{\pgfqpoint{5.994203in}{1.233811in}}%
\pgfpathlineto{\pgfqpoint{6.093634in}{1.131933in}}%
\pgfpathlineto{\pgfqpoint{6.196642in}{1.030055in}}%
\pgfpathlineto{\pgfqpoint{6.303207in}{0.928177in}}%
\pgfpathlineto{\pgfqpoint{6.414253in}{0.825386in}}%
\pgfpathlineto{\pgfqpoint{6.532547in}{0.719185in}}%
\pgfpathlineto{\pgfqpoint{6.650840in}{0.616053in}}%
\pgfpathlineto{\pgfqpoint{6.702832in}{0.571603in}}%
\pgfpathlineto{\pgfqpoint{6.702832in}{0.571603in}}%
\pgfusepath{stroke}%
\end{pgfscope}%
\begin{pgfscope}%
\pgfpathrectangle{\pgfqpoint{0.854460in}{0.571603in}}{\pgfqpoint{5.885100in}{5.068436in}}%
\pgfusepath{clip}%
\pgfsetbuttcap%
\pgfsetroundjoin%
\pgfsetlinewidth{1.505625pt}%
\definecolor{currentstroke}{rgb}{0.206756,0.371758,0.553117}%
\pgfsetstrokecolor{currentstroke}%
\pgfsetdash{}{0pt}%
\pgfpathmoveto{\pgfqpoint{1.927055in}{0.571603in}}%
\pgfpathlineto{\pgfqpoint{1.859955in}{0.628218in}}%
\pgfpathlineto{\pgfqpoint{1.780068in}{0.698951in}}%
\pgfpathlineto{\pgfqpoint{1.712088in}{0.762430in}}%
\pgfpathlineto{\pgfqpoint{1.646978in}{0.826299in}}%
\pgfpathlineto{\pgfqpoint{1.593795in}{0.881083in}}%
\pgfpathlineto{\pgfqpoint{1.526955in}{0.953646in}}%
\pgfpathlineto{\pgfqpoint{1.475501in}{1.012717in}}%
\pgfpathlineto{\pgfqpoint{1.416354in}{1.084495in}}%
\pgfpathlineto{\pgfqpoint{1.357208in}{1.161196in}}%
\pgfpathlineto{\pgfqpoint{1.304912in}{1.233811in}}%
\pgfpathlineto{\pgfqpoint{1.268488in}{1.287451in}}%
\pgfpathlineto{\pgfqpoint{1.221495in}{1.361159in}}%
\pgfpathlineto{\pgfqpoint{1.179767in}{1.431440in}}%
\pgfpathlineto{\pgfqpoint{1.148054in}{1.488506in}}%
\pgfpathlineto{\pgfqpoint{1.108637in}{1.564915in}}%
\pgfpathlineto{\pgfqpoint{1.072470in}{1.641323in}}%
\pgfpathlineto{\pgfqpoint{1.050140in}{1.692262in}}%
\pgfpathlineto{\pgfqpoint{1.019239in}{1.768671in}}%
\pgfpathlineto{\pgfqpoint{0.991365in}{1.845080in}}%
\pgfpathlineto{\pgfqpoint{0.966422in}{1.921488in}}%
\pgfpathlineto{\pgfqpoint{0.943181in}{2.002221in}}%
\pgfpathlineto{\pgfqpoint{0.925123in}{2.074305in}}%
\pgfpathlineto{\pgfqpoint{0.908570in}{2.150714in}}%
\pgfpathlineto{\pgfqpoint{0.894743in}{2.227123in}}%
\pgfpathlineto{\pgfqpoint{0.883463in}{2.303531in}}%
\pgfpathlineto{\pgfqpoint{0.874877in}{2.379940in}}%
\pgfpathlineto{\pgfqpoint{0.868800in}{2.456348in}}%
\pgfpathlineto{\pgfqpoint{0.865230in}{2.532757in}}%
\pgfpathlineto{\pgfqpoint{0.864167in}{2.609165in}}%
\pgfpathlineto{\pgfqpoint{0.865601in}{2.685574in}}%
\pgfpathlineto{\pgfqpoint{0.869523in}{2.761983in}}%
\pgfpathlineto{\pgfqpoint{0.875918in}{2.838391in}}%
\pgfpathlineto{\pgfqpoint{0.884781in}{2.914800in}}%
\pgfpathlineto{\pgfqpoint{0.896280in}{2.991208in}}%
\pgfpathlineto{\pgfqpoint{0.910232in}{3.067617in}}%
\pgfpathlineto{\pgfqpoint{0.926860in}{3.144025in}}%
\pgfpathlineto{\pgfqpoint{0.946045in}{3.220434in}}%
\pgfpathlineto{\pgfqpoint{0.967994in}{3.296843in}}%
\pgfpathlineto{\pgfqpoint{0.984185in}{3.347782in}}%
\pgfpathlineto{\pgfqpoint{1.010784in}{3.424190in}}%
\pgfpathlineto{\pgfqpoint{1.040316in}{3.500599in}}%
\pgfpathlineto{\pgfqpoint{1.072887in}{3.577007in}}%
\pgfpathlineto{\pgfqpoint{1.108602in}{3.653416in}}%
\pgfpathlineto{\pgfqpoint{1.134230in}{3.704355in}}%
\pgfpathlineto{\pgfqpoint{1.161360in}{3.755294in}}%
\pgfpathlineto{\pgfqpoint{1.204960in}{3.831703in}}%
\pgfpathlineto{\pgfqpoint{1.238914in}{3.887159in}}%
\pgfpathlineto{\pgfqpoint{1.286013in}{3.959050in}}%
\pgfpathlineto{\pgfqpoint{1.327634in}{4.018563in}}%
\pgfpathlineto{\pgfqpoint{1.378277in}{4.086398in}}%
\pgfpathlineto{\pgfqpoint{1.418566in}{4.137337in}}%
\pgfpathlineto{\pgfqpoint{1.475501in}{4.205093in}}%
\pgfpathlineto{\pgfqpoint{1.528796in}{4.264685in}}%
\pgfpathlineto{\pgfqpoint{1.576939in}{4.315624in}}%
\pgfpathlineto{\pgfqpoint{1.627503in}{4.366563in}}%
\pgfpathlineto{\pgfqpoint{1.682515in}{4.419173in}}%
\pgfpathlineto{\pgfqpoint{1.741661in}{4.472755in}}%
\pgfpathlineto{\pgfqpoint{1.800808in}{4.523589in}}%
\pgfpathlineto{\pgfqpoint{1.859955in}{4.571912in}}%
\pgfpathlineto{\pgfqpoint{1.923579in}{4.621258in}}%
\pgfpathlineto{\pgfqpoint{1.992897in}{4.672197in}}%
\pgfpathlineto{\pgfqpoint{2.066968in}{4.723742in}}%
\pgfpathlineto{\pgfqpoint{2.155689in}{4.781643in}}%
\pgfpathlineto{\pgfqpoint{2.226165in}{4.825014in}}%
\pgfpathlineto{\pgfqpoint{2.313890in}{4.875953in}}%
\pgfpathlineto{\pgfqpoint{2.392275in}{4.918782in}}%
\pgfpathlineto{\pgfqpoint{2.456952in}{4.952362in}}%
\pgfpathlineto{\pgfqpoint{2.540142in}{4.993224in}}%
\pgfpathlineto{\pgfqpoint{2.628862in}{5.034188in}}%
\pgfpathlineto{\pgfqpoint{2.717582in}{5.072491in}}%
\pgfpathlineto{\pgfqpoint{2.806303in}{5.108300in}}%
\pgfpathlineto{\pgfqpoint{2.895023in}{5.141589in}}%
\pgfpathlineto{\pgfqpoint{2.983743in}{5.172526in}}%
\pgfpathlineto{\pgfqpoint{3.072463in}{5.201109in}}%
\pgfpathlineto{\pgfqpoint{3.161183in}{5.227331in}}%
\pgfpathlineto{\pgfqpoint{3.249903in}{5.251172in}}%
\pgfpathlineto{\pgfqpoint{3.338623in}{5.272606in}}%
\pgfpathlineto{\pgfqpoint{3.427343in}{5.291601in}}%
\pgfpathlineto{\pgfqpoint{3.521176in}{5.308935in}}%
\pgfpathlineto{\pgfqpoint{3.604783in}{5.321741in}}%
\pgfpathlineto{\pgfqpoint{3.693504in}{5.332657in}}%
\pgfpathlineto{\pgfqpoint{3.752650in}{5.338153in}}%
\pgfpathlineto{\pgfqpoint{3.811797in}{5.342164in}}%
\pgfpathlineto{\pgfqpoint{3.870944in}{5.344611in}}%
\pgfpathlineto{\pgfqpoint{3.930090in}{5.345351in}}%
\pgfpathlineto{\pgfqpoint{3.989237in}{5.344218in}}%
\pgfpathlineto{\pgfqpoint{4.048384in}{5.341023in}}%
\pgfpathlineto{\pgfqpoint{4.116872in}{5.334405in}}%
\pgfpathlineto{\pgfqpoint{4.166677in}{5.327313in}}%
\pgfpathlineto{\pgfqpoint{4.225824in}{5.316103in}}%
\pgfpathlineto{\pgfqpoint{4.284971in}{5.301330in}}%
\pgfpathlineto{\pgfqpoint{4.344118in}{5.282366in}}%
\pgfpathlineto{\pgfqpoint{4.373691in}{5.270851in}}%
\pgfpathlineto{\pgfqpoint{4.403459in}{5.257996in}}%
\pgfpathlineto{\pgfqpoint{4.452909in}{5.232527in}}%
\pgfpathlineto{\pgfqpoint{4.462411in}{5.227013in}}%
\pgfpathlineto{\pgfqpoint{4.494010in}{5.207057in}}%
\pgfpathlineto{\pgfqpoint{4.528663in}{5.181588in}}%
\pgfpathlineto{\pgfqpoint{4.558542in}{5.156118in}}%
\pgfpathlineto{\pgfqpoint{4.584594in}{5.130649in}}%
\pgfpathlineto{\pgfqpoint{4.610278in}{5.101774in}}%
\pgfpathlineto{\pgfqpoint{4.639851in}{5.062627in}}%
\pgfpathlineto{\pgfqpoint{4.645565in}{5.054240in}}%
\pgfpathlineto{\pgfqpoint{4.669425in}{5.015103in}}%
\pgfpathlineto{\pgfqpoint{4.688728in}{4.977831in}}%
\pgfpathlineto{\pgfqpoint{4.700349in}{4.952362in}}%
\pgfpathlineto{\pgfqpoint{4.720066in}{4.901423in}}%
\pgfpathlineto{\pgfqpoint{4.736157in}{4.850484in}}%
\pgfpathlineto{\pgfqpoint{4.749260in}{4.799545in}}%
\pgfpathlineto{\pgfqpoint{4.759995in}{4.748606in}}%
\pgfpathlineto{\pgfqpoint{4.768657in}{4.697667in}}%
\pgfpathlineto{\pgfqpoint{4.778797in}{4.621258in}}%
\pgfpathlineto{\pgfqpoint{4.786301in}{4.544849in}}%
\pgfpathlineto{\pgfqpoint{4.791822in}{4.468441in}}%
\pgfpathlineto{\pgfqpoint{4.798351in}{4.341093in}}%
\pgfpathlineto{\pgfqpoint{4.816272in}{3.933581in}}%
\pgfpathlineto{\pgfqpoint{4.825305in}{3.806233in}}%
\pgfpathlineto{\pgfqpoint{4.834760in}{3.704355in}}%
\pgfpathlineto{\pgfqpoint{4.846865in}{3.600279in}}%
\pgfpathlineto{\pgfqpoint{4.860864in}{3.500599in}}%
\pgfpathlineto{\pgfqpoint{4.878034in}{3.398721in}}%
\pgfpathlineto{\pgfqpoint{4.892798in}{3.322312in}}%
\pgfpathlineto{\pgfqpoint{4.909381in}{3.245904in}}%
\pgfpathlineto{\pgfqpoint{4.927738in}{3.169495in}}%
\pgfpathlineto{\pgfqpoint{4.947980in}{3.093086in}}%
\pgfpathlineto{\pgfqpoint{4.970176in}{3.016678in}}%
\pgfpathlineto{\pgfqpoint{4.994732in}{2.939132in}}%
\pgfpathlineto{\pgfqpoint{5.024305in}{2.853342in}}%
\pgfpathlineto{\pgfqpoint{5.048701in}{2.787452in}}%
\pgfpathlineto{\pgfqpoint{5.083452in}{2.700352in}}%
\pgfpathlineto{\pgfqpoint{5.113025in}{2.630996in}}%
\pgfpathlineto{\pgfqpoint{5.145926in}{2.558226in}}%
\pgfpathlineto{\pgfqpoint{5.182583in}{2.481818in}}%
\pgfpathlineto{\pgfqpoint{5.221429in}{2.405409in}}%
\pgfpathlineto{\pgfqpoint{5.262485in}{2.329001in}}%
\pgfpathlineto{\pgfqpoint{5.305670in}{2.252592in}}%
\pgfpathlineto{\pgfqpoint{5.351123in}{2.176183in}}%
\pgfpathlineto{\pgfqpoint{5.398719in}{2.099775in}}%
\pgfpathlineto{\pgfqpoint{5.465683in}{1.997897in}}%
\pgfpathlineto{\pgfqpoint{5.536528in}{1.896019in}}%
\pgfpathlineto{\pgfqpoint{5.592266in}{1.819610in}}%
\pgfpathlineto{\pgfqpoint{5.650203in}{1.743202in}}%
\pgfpathlineto{\pgfqpoint{5.710332in}{1.666793in}}%
\pgfpathlineto{\pgfqpoint{5.793213in}{1.565781in}}%
\pgfpathlineto{\pgfqpoint{5.852359in}{1.496368in}}%
\pgfpathlineto{\pgfqpoint{5.911506in}{1.428955in}}%
\pgfpathlineto{\pgfqpoint{5.972667in}{1.361159in}}%
\pgfpathlineto{\pgfqpoint{6.067736in}{1.259281in}}%
\pgfpathlineto{\pgfqpoint{6.166582in}{1.157402in}}%
\pgfpathlineto{\pgfqpoint{6.269166in}{1.055524in}}%
\pgfpathlineto{\pgfqpoint{6.375358in}{0.953646in}}%
\pgfpathlineto{\pgfqpoint{6.485156in}{0.851768in}}%
\pgfpathlineto{\pgfqpoint{6.598484in}{0.749890in}}%
\pgfpathlineto{\pgfqpoint{6.715240in}{0.648012in}}%
\pgfpathlineto{\pgfqpoint{6.739560in}{0.627156in}}%
\pgfpathlineto{\pgfqpoint{6.739560in}{0.627156in}}%
\pgfusepath{stroke}%
\end{pgfscope}%
\begin{pgfscope}%
\pgfpathrectangle{\pgfqpoint{0.854460in}{0.571603in}}{\pgfqpoint{5.885100in}{5.068436in}}%
\pgfusepath{clip}%
\pgfsetbuttcap%
\pgfsetroundjoin%
\pgfsetlinewidth{1.505625pt}%
\definecolor{currentstroke}{rgb}{0.197636,0.391528,0.554969}%
\pgfsetstrokecolor{currentstroke}%
\pgfsetdash{}{0pt}%
\pgfpathmoveto{\pgfqpoint{1.844312in}{0.571603in}}%
\pgfpathlineto{\pgfqpoint{1.830381in}{0.583332in}}%
\pgfpathlineto{\pgfqpoint{1.814129in}{0.597073in}}%
\pgfpathlineto{\pgfqpoint{1.800808in}{0.608504in}}%
\pgfpathlineto{\pgfqpoint{1.784522in}{0.622542in}}%
\pgfpathlineto{\pgfqpoint{1.771235in}{0.634167in}}%
\pgfpathlineto{\pgfqpoint{1.755483in}{0.648012in}}%
\pgfpathlineto{\pgfqpoint{1.741661in}{0.660344in}}%
\pgfpathlineto{\pgfqpoint{1.727006in}{0.673481in}}%
\pgfpathlineto{\pgfqpoint{1.712088in}{0.687056in}}%
\pgfpathlineto{\pgfqpoint{1.699081in}{0.698951in}}%
\pgfpathlineto{\pgfqpoint{1.682515in}{0.714330in}}%
\pgfpathlineto{\pgfqpoint{1.671701in}{0.724420in}}%
\pgfpathlineto{\pgfqpoint{1.652941in}{0.742191in}}%
\pgfpathlineto{\pgfqpoint{1.644857in}{0.749890in}}%
\pgfpathlineto{\pgfqpoint{1.623368in}{0.770666in}}%
\pgfpathlineto{\pgfqpoint{1.618539in}{0.775360in}}%
\pgfpathlineto{\pgfqpoint{1.593795in}{0.799781in}}%
\pgfpathlineto{\pgfqpoint{1.592739in}{0.800829in}}%
\pgfpathlineto{\pgfqpoint{1.567496in}{0.826299in}}%
\pgfpathlineto{\pgfqpoint{1.564221in}{0.829655in}}%
\pgfpathlineto{\pgfqpoint{1.542776in}{0.851768in}}%
\pgfpathlineto{\pgfqpoint{1.534648in}{0.860278in}}%
\pgfpathlineto{\pgfqpoint{1.518549in}{0.877238in}}%
\pgfpathlineto{\pgfqpoint{1.505074in}{0.891653in}}%
\pgfpathlineto{\pgfqpoint{1.494807in}{0.902707in}}%
\pgfpathlineto{\pgfqpoint{1.475501in}{0.923815in}}%
\pgfpathlineto{\pgfqpoint{1.471538in}{0.928177in}}%
\pgfpathlineto{\pgfqpoint{1.448775in}{0.953646in}}%
\pgfpathlineto{\pgfqpoint{1.445928in}{0.956887in}}%
\pgfpathlineto{\pgfqpoint{1.426533in}{0.979116in}}%
\pgfpathlineto{\pgfqpoint{1.416354in}{0.990965in}}%
\pgfpathlineto{\pgfqpoint{1.404737in}{1.004585in}}%
\pgfpathlineto{\pgfqpoint{1.386781in}{1.025969in}}%
\pgfpathlineto{\pgfqpoint{1.383375in}{1.030055in}}%
\pgfpathlineto{\pgfqpoint{1.362518in}{1.055524in}}%
\pgfpathlineto{\pgfqpoint{1.357208in}{1.062122in}}%
\pgfpathlineto{\pgfqpoint{1.342133in}{1.080994in}}%
\pgfpathlineto{\pgfqpoint{1.327634in}{1.099436in}}%
\pgfpathlineto{\pgfqpoint{1.322152in}{1.106463in}}%
\pgfpathlineto{\pgfqpoint{1.302636in}{1.131933in}}%
\pgfpathlineto{\pgfqpoint{1.298061in}{1.138014in}}%
\pgfpathlineto{\pgfqpoint{1.283593in}{1.157402in}}%
\pgfpathlineto{\pgfqpoint{1.268488in}{1.177977in}}%
\pgfpathlineto{\pgfqpoint{1.264923in}{1.182872in}}%
\pgfpathlineto{\pgfqpoint{1.246738in}{1.208341in}}%
\pgfpathlineto{\pgfqpoint{1.238914in}{1.219498in}}%
\pgfpathlineto{\pgfqpoint{1.228963in}{1.233811in}}%
\pgfpathlineto{\pgfqpoint{1.211563in}{1.259281in}}%
\pgfpathlineto{\pgfqpoint{1.209341in}{1.262603in}}%
\pgfpathlineto{\pgfqpoint{1.194658in}{1.284750in}}%
\pgfpathlineto{\pgfqpoint{1.179767in}{1.307593in}}%
\pgfpathlineto{\pgfqpoint{1.178071in}{1.310220in}}%
\pgfpathlineto{\pgfqpoint{1.161980in}{1.335689in}}%
\pgfpathlineto{\pgfqpoint{1.150194in}{1.354685in}}%
\pgfpathlineto{\pgfqpoint{1.146215in}{1.361159in}}%
\pgfpathlineto{\pgfqpoint{1.130900in}{1.386628in}}%
\pgfpathlineto{\pgfqpoint{1.120621in}{1.404059in}}%
\pgfpathlineto{\pgfqpoint{1.115926in}{1.412098in}}%
\pgfpathlineto{\pgfqpoint{1.101382in}{1.437567in}}%
\pgfpathlineto{\pgfqpoint{1.091047in}{1.456037in}}%
\pgfpathlineto{\pgfqpoint{1.087169in}{1.463037in}}%
\pgfpathlineto{\pgfqpoint{1.073392in}{1.488506in}}%
\pgfpathlineto{\pgfqpoint{1.061474in}{1.510992in}}%
\pgfpathlineto{\pgfqpoint{1.059908in}{1.513976in}}%
\pgfpathlineto{\pgfqpoint{1.046889in}{1.539445in}}%
\pgfpathlineto{\pgfqpoint{1.034140in}{1.564915in}}%
\pgfpathlineto{\pgfqpoint{1.031901in}{1.569510in}}%
\pgfpathlineto{\pgfqpoint{1.021834in}{1.590384in}}%
\pgfpathlineto{\pgfqpoint{1.009837in}{1.615854in}}%
\pgfpathlineto{\pgfqpoint{1.002327in}{1.632202in}}%
\pgfpathlineto{\pgfqpoint{0.998184in}{1.641323in}}%
\pgfpathlineto{\pgfqpoint{0.986928in}{1.666793in}}%
\pgfpathlineto{\pgfqpoint{0.975946in}{1.692262in}}%
\pgfpathlineto{\pgfqpoint{0.972754in}{1.699887in}}%
\pgfpathlineto{\pgfqpoint{0.965367in}{1.717732in}}%
\pgfpathlineto{\pgfqpoint{0.955117in}{1.743202in}}%
\pgfpathlineto{\pgfqpoint{0.945142in}{1.768671in}}%
\pgfpathlineto{\pgfqpoint{0.943181in}{1.773843in}}%
\pgfpathlineto{\pgfqpoint{0.935576in}{1.794141in}}%
\pgfpathlineto{\pgfqpoint{0.926318in}{1.819610in}}%
\pgfpathlineto{\pgfqpoint{0.917338in}{1.845080in}}%
\pgfpathlineto{\pgfqpoint{0.913607in}{1.856029in}}%
\pgfpathlineto{\pgfqpoint{0.908720in}{1.870549in}}%
\pgfpathlineto{\pgfqpoint{0.900442in}{1.896019in}}%
\pgfpathlineto{\pgfqpoint{0.892441in}{1.921488in}}%
\pgfpathlineto{\pgfqpoint{0.884718in}{1.946958in}}%
\pgfpathlineto{\pgfqpoint{0.884034in}{1.949307in}}%
\pgfpathlineto{\pgfqpoint{0.877390in}{1.972427in}}%
\pgfpathlineto{\pgfqpoint{0.870350in}{1.997897in}}%
\pgfpathlineto{\pgfqpoint{0.863588in}{2.023366in}}%
\pgfpathlineto{\pgfqpoint{0.857103in}{2.048836in}}%
\pgfpathlineto{\pgfqpoint{0.854460in}{2.059705in}}%
\pgfusepath{stroke}%
\end{pgfscope}%
\begin{pgfscope}%
\pgfpathrectangle{\pgfqpoint{0.854460in}{0.571603in}}{\pgfqpoint{5.885100in}{5.068436in}}%
\pgfusepath{clip}%
\pgfsetbuttcap%
\pgfsetroundjoin%
\pgfsetlinewidth{1.505625pt}%
\definecolor{currentstroke}{rgb}{0.197636,0.391528,0.554969}%
\pgfsetstrokecolor{currentstroke}%
\pgfsetdash{}{0pt}%
\pgfpathmoveto{\pgfqpoint{0.854460in}{3.180525in}}%
\pgfpathlineto{\pgfqpoint{0.871070in}{3.245904in}}%
\pgfpathlineto{\pgfqpoint{0.892890in}{3.322312in}}%
\pgfpathlineto{\pgfqpoint{0.917409in}{3.398721in}}%
\pgfpathlineto{\pgfqpoint{0.944736in}{3.475129in}}%
\pgfpathlineto{\pgfqpoint{0.974979in}{3.551538in}}%
\pgfpathlineto{\pgfqpoint{1.008241in}{3.627946in}}%
\pgfpathlineto{\pgfqpoint{1.044619in}{3.704355in}}%
\pgfpathlineto{\pgfqpoint{1.084208in}{3.780764in}}%
\pgfpathlineto{\pgfqpoint{1.120621in}{3.845852in}}%
\pgfpathlineto{\pgfqpoint{1.157885in}{3.908111in}}%
\pgfpathlineto{\pgfqpoint{1.206943in}{3.984520in}}%
\pgfpathlineto{\pgfqpoint{1.241838in}{4.035459in}}%
\pgfpathlineto{\pgfqpoint{1.298061in}{4.112515in}}%
\pgfpathlineto{\pgfqpoint{1.357653in}{4.188276in}}%
\pgfpathlineto{\pgfqpoint{1.416354in}{4.257778in}}%
\pgfpathlineto{\pgfqpoint{1.468280in}{4.315624in}}%
\pgfpathlineto{\pgfqpoint{1.516473in}{4.366563in}}%
\pgfpathlineto{\pgfqpoint{1.567031in}{4.417502in}}%
\pgfpathlineto{\pgfqpoint{1.623368in}{4.471396in}}%
\pgfpathlineto{\pgfqpoint{1.682515in}{4.525066in}}%
\pgfpathlineto{\pgfqpoint{1.741661in}{4.576057in}}%
\pgfpathlineto{\pgfqpoint{1.800808in}{4.624600in}}%
\pgfpathlineto{\pgfqpoint{1.861687in}{4.672197in}}%
\pgfpathlineto{\pgfqpoint{1.930308in}{4.723136in}}%
\pgfpathlineto{\pgfqpoint{2.007822in}{4.777652in}}%
\pgfpathlineto{\pgfqpoint{2.096542in}{4.836291in}}%
\pgfpathlineto{\pgfqpoint{2.159813in}{4.875953in}}%
\pgfpathlineto{\pgfqpoint{2.245485in}{4.926892in}}%
\pgfpathlineto{\pgfqpoint{2.336615in}{4.977831in}}%
\pgfpathlineto{\pgfqpoint{2.433832in}{5.028770in}}%
\pgfpathlineto{\pgfqpoint{2.510569in}{5.066663in}}%
\pgfpathlineto{\pgfqpoint{2.599289in}{5.108130in}}%
\pgfpathlineto{\pgfqpoint{2.688009in}{5.147058in}}%
\pgfpathlineto{\pgfqpoint{2.776729in}{5.183712in}}%
\pgfpathlineto{\pgfqpoint{2.865449in}{5.218001in}}%
\pgfpathlineto{\pgfqpoint{2.954169in}{5.250143in}}%
\pgfpathlineto{\pgfqpoint{3.053137in}{5.283466in}}%
\pgfpathlineto{\pgfqpoint{3.134672in}{5.308935in}}%
\pgfpathlineto{\pgfqpoint{3.222709in}{5.334405in}}%
\pgfpathlineto{\pgfqpoint{3.319161in}{5.359874in}}%
\pgfpathlineto{\pgfqpoint{3.397770in}{5.378751in}}%
\pgfpathlineto{\pgfqpoint{3.486490in}{5.397981in}}%
\pgfpathlineto{\pgfqpoint{3.575210in}{5.414993in}}%
\pgfpathlineto{\pgfqpoint{3.663930in}{5.429569in}}%
\pgfpathlineto{\pgfqpoint{3.752650in}{5.441654in}}%
\pgfpathlineto{\pgfqpoint{3.841370in}{5.450999in}}%
\pgfpathlineto{\pgfqpoint{3.930090in}{5.457464in}}%
\pgfpathlineto{\pgfqpoint{3.989237in}{5.459989in}}%
\pgfpathlineto{\pgfqpoint{4.048384in}{5.460932in}}%
\pgfpathlineto{\pgfqpoint{4.107531in}{5.460144in}}%
\pgfpathlineto{\pgfqpoint{4.166677in}{5.457450in}}%
\pgfpathlineto{\pgfqpoint{4.225824in}{5.452652in}}%
\pgfpathlineto{\pgfqpoint{4.284971in}{5.445521in}}%
\pgfpathlineto{\pgfqpoint{4.344118in}{5.435771in}}%
\pgfpathlineto{\pgfqpoint{4.403264in}{5.422664in}}%
\pgfpathlineto{\pgfqpoint{4.462411in}{5.405951in}}%
\pgfpathlineto{\pgfqpoint{4.521558in}{5.384687in}}%
\pgfpathlineto{\pgfqpoint{4.576240in}{5.359874in}}%
\pgfpathlineto{\pgfqpoint{4.610278in}{5.341340in}}%
\pgfpathlineto{\pgfqpoint{4.639851in}{5.322959in}}%
\pgfpathlineto{\pgfqpoint{4.669425in}{5.302085in}}%
\pgfpathlineto{\pgfqpoint{4.698998in}{5.278135in}}%
\pgfpathlineto{\pgfqpoint{4.728571in}{5.250365in}}%
\pgfpathlineto{\pgfqpoint{4.758145in}{5.217810in}}%
\pgfpathlineto{\pgfqpoint{4.766937in}{5.207057in}}%
\pgfpathlineto{\pgfqpoint{4.787718in}{5.179069in}}%
\pgfpathlineto{\pgfqpoint{4.802723in}{5.156118in}}%
\pgfpathlineto{\pgfqpoint{4.817754in}{5.130649in}}%
\pgfpathlineto{\pgfqpoint{4.830972in}{5.105179in}}%
\pgfpathlineto{\pgfqpoint{4.846865in}{5.070370in}}%
\pgfpathlineto{\pgfqpoint{4.862875in}{5.028770in}}%
\pgfpathlineto{\pgfqpoint{4.878890in}{4.977831in}}%
\pgfpathlineto{\pgfqpoint{4.891506in}{4.926892in}}%
\pgfpathlineto{\pgfqpoint{4.901533in}{4.875953in}}%
\pgfpathlineto{\pgfqpoint{4.909337in}{4.825014in}}%
\pgfpathlineto{\pgfqpoint{4.915300in}{4.774075in}}%
\pgfpathlineto{\pgfqpoint{4.921600in}{4.697667in}}%
\pgfpathlineto{\pgfqpoint{4.925464in}{4.621258in}}%
\pgfpathlineto{\pgfqpoint{4.927987in}{4.519380in}}%
\pgfpathlineto{\pgfqpoint{4.928668in}{4.392032in}}%
\pgfpathlineto{\pgfqpoint{4.929266in}{4.060928in}}%
\pgfpathlineto{\pgfqpoint{4.932351in}{3.933581in}}%
\pgfpathlineto{\pgfqpoint{4.936869in}{3.831703in}}%
\pgfpathlineto{\pgfqpoint{4.943531in}{3.729825in}}%
\pgfpathlineto{\pgfqpoint{4.952681in}{3.627946in}}%
\pgfpathlineto{\pgfqpoint{4.964590in}{3.526068in}}%
\pgfpathlineto{\pgfqpoint{4.979334in}{3.424190in}}%
\pgfpathlineto{\pgfqpoint{4.994732in}{3.335848in}}%
\pgfpathlineto{\pgfqpoint{5.007375in}{3.271373in}}%
\pgfpathlineto{\pgfqpoint{5.024305in}{3.194587in}}%
\pgfpathlineto{\pgfqpoint{5.042897in}{3.118556in}}%
\pgfpathlineto{\pgfqpoint{5.063595in}{3.042147in}}%
\pgfpathlineto{\pgfqpoint{5.086330in}{2.965739in}}%
\pgfpathlineto{\pgfqpoint{5.113025in}{2.883746in}}%
\pgfpathlineto{\pgfqpoint{5.137947in}{2.812922in}}%
\pgfpathlineto{\pgfqpoint{5.172172in}{2.723376in}}%
\pgfpathlineto{\pgfqpoint{5.201745in}{2.651468in}}%
\pgfpathlineto{\pgfqpoint{5.231354in}{2.583696in}}%
\pgfpathlineto{\pgfqpoint{5.266804in}{2.507287in}}%
\pgfpathlineto{\pgfqpoint{5.304469in}{2.430879in}}%
\pgfpathlineto{\pgfqpoint{5.349612in}{2.344853in}}%
\pgfpathlineto{\pgfqpoint{5.386504in}{2.278062in}}%
\pgfpathlineto{\pgfqpoint{5.438332in}{2.189255in}}%
\pgfpathlineto{\pgfqpoint{5.477471in}{2.125244in}}%
\pgfpathlineto{\pgfqpoint{5.527052in}{2.047785in}}%
\pgfpathlineto{\pgfqpoint{5.586199in}{1.959727in}}%
\pgfpathlineto{\pgfqpoint{5.630782in}{1.896019in}}%
\pgfpathlineto{\pgfqpoint{5.704492in}{1.795428in}}%
\pgfpathlineto{\pgfqpoint{5.744235in}{1.743202in}}%
\pgfpathlineto{\pgfqpoint{5.824870in}{1.641323in}}%
\pgfpathlineto{\pgfqpoint{5.887912in}{1.564915in}}%
\pgfpathlineto{\pgfqpoint{5.975441in}{1.463037in}}%
\pgfpathlineto{\pgfqpoint{6.066872in}{1.361159in}}%
\pgfpathlineto{\pgfqpoint{6.162185in}{1.259281in}}%
\pgfpathlineto{\pgfqpoint{6.261365in}{1.157402in}}%
\pgfpathlineto{\pgfqpoint{6.364315in}{1.055524in}}%
\pgfpathlineto{\pgfqpoint{6.473400in}{0.951441in}}%
\pgfpathlineto{\pgfqpoint{6.581411in}{0.851768in}}%
\pgfpathlineto{\pgfqpoint{6.695420in}{0.749890in}}%
\pgfpathlineto{\pgfqpoint{6.739560in}{0.711291in}}%
\pgfpathlineto{\pgfqpoint{6.739560in}{0.711291in}}%
\pgfusepath{stroke}%
\end{pgfscope}%
\begin{pgfscope}%
\pgfpathrectangle{\pgfqpoint{0.854460in}{0.571603in}}{\pgfqpoint{5.885100in}{5.068436in}}%
\pgfusepath{clip}%
\pgfsetbuttcap%
\pgfsetroundjoin%
\pgfsetlinewidth{1.505625pt}%
\definecolor{currentstroke}{rgb}{0.187231,0.414746,0.556547}%
\pgfsetstrokecolor{currentstroke}%
\pgfsetdash{}{0pt}%
\pgfpathmoveto{\pgfqpoint{1.764898in}{0.571603in}}%
\pgfpathlineto{\pgfqpoint{1.741661in}{0.591424in}}%
\pgfpathlineto{\pgfqpoint{1.735067in}{0.597073in}}%
\pgfpathlineto{\pgfqpoint{1.712088in}{0.617050in}}%
\pgfpathlineto{\pgfqpoint{1.705799in}{0.622542in}}%
\pgfpathlineto{\pgfqpoint{1.682515in}{0.643180in}}%
\pgfpathlineto{\pgfqpoint{1.677089in}{0.648012in}}%
\pgfpathlineto{\pgfqpoint{1.652941in}{0.669836in}}%
\pgfpathlineto{\pgfqpoint{1.648927in}{0.673481in}}%
\pgfpathlineto{\pgfqpoint{1.623368in}{0.697041in}}%
\pgfpathlineto{\pgfqpoint{1.621306in}{0.698951in}}%
\pgfpathlineto{\pgfqpoint{1.594224in}{0.724420in}}%
\pgfpathlineto{\pgfqpoint{1.593795in}{0.724831in}}%
\pgfpathlineto{\pgfqpoint{1.567705in}{0.749890in}}%
\pgfpathlineto{\pgfqpoint{1.564221in}{0.753287in}}%
\pgfpathlineto{\pgfqpoint{1.541709in}{0.775360in}}%
\pgfpathlineto{\pgfqpoint{1.534648in}{0.782387in}}%
\pgfpathlineto{\pgfqpoint{1.516225in}{0.800829in}}%
\pgfpathlineto{\pgfqpoint{1.505074in}{0.812161in}}%
\pgfpathlineto{\pgfqpoint{1.491244in}{0.826299in}}%
\pgfpathlineto{\pgfqpoint{1.475501in}{0.842637in}}%
\pgfpathlineto{\pgfqpoint{1.466757in}{0.851768in}}%
\pgfpathlineto{\pgfqpoint{1.445928in}{0.873849in}}%
\pgfpathlineto{\pgfqpoint{1.442751in}{0.877238in}}%
\pgfpathlineto{\pgfqpoint{1.419262in}{0.902707in}}%
\pgfpathlineto{\pgfqpoint{1.416354in}{0.905912in}}%
\pgfpathlineto{\pgfqpoint{1.396290in}{0.928177in}}%
\pgfpathlineto{\pgfqpoint{1.386781in}{0.938891in}}%
\pgfpathlineto{\pgfqpoint{1.373775in}{0.953646in}}%
\pgfpathlineto{\pgfqpoint{1.357208in}{0.972733in}}%
\pgfpathlineto{\pgfqpoint{1.351706in}{0.979116in}}%
\pgfpathlineto{\pgfqpoint{1.330109in}{1.004585in}}%
\pgfpathlineto{\pgfqpoint{1.327634in}{1.007556in}}%
\pgfpathlineto{\pgfqpoint{1.309028in}{1.030055in}}%
\pgfpathlineto{\pgfqpoint{1.298061in}{1.043524in}}%
\pgfpathlineto{\pgfqpoint{1.288363in}{1.055524in}}%
\pgfpathlineto{\pgfqpoint{1.268488in}{1.080508in}}%
\pgfpathlineto{\pgfqpoint{1.268104in}{1.080994in}}%
\pgfpathlineto{\pgfqpoint{1.248385in}{1.106463in}}%
\pgfpathlineto{\pgfqpoint{1.238914in}{1.118895in}}%
\pgfpathlineto{\pgfqpoint{1.229059in}{1.131933in}}%
\pgfpathlineto{\pgfqpoint{1.210120in}{1.157402in}}%
\pgfpathlineto{\pgfqpoint{1.209341in}{1.158472in}}%
\pgfpathlineto{\pgfqpoint{1.191705in}{1.182872in}}%
\pgfpathlineto{\pgfqpoint{1.179767in}{1.199660in}}%
\pgfpathlineto{\pgfqpoint{1.173645in}{1.208341in}}%
\pgfpathlineto{\pgfqpoint{1.156020in}{1.233811in}}%
\pgfpathlineto{\pgfqpoint{1.150194in}{1.242391in}}%
\pgfpathlineto{\pgfqpoint{1.138827in}{1.259281in}}%
\pgfpathlineto{\pgfqpoint{1.121977in}{1.284750in}}%
\pgfpathlineto{\pgfqpoint{1.120621in}{1.286846in}}%
\pgfpathlineto{\pgfqpoint{1.105625in}{1.310220in}}%
\pgfpathlineto{\pgfqpoint{1.091047in}{1.333331in}}%
\pgfpathlineto{\pgfqpoint{1.089574in}{1.335689in}}%
\pgfpathlineto{\pgfqpoint{1.074010in}{1.361159in}}%
\pgfpathlineto{\pgfqpoint{1.061474in}{1.382047in}}%
\pgfpathlineto{\pgfqpoint{1.058751in}{1.386628in}}%
\pgfpathlineto{\pgfqpoint{1.043950in}{1.412098in}}%
\pgfpathlineto{\pgfqpoint{1.031901in}{1.433231in}}%
\pgfpathlineto{\pgfqpoint{1.029452in}{1.437567in}}%
\pgfpathlineto{\pgfqpoint{1.015410in}{1.463037in}}%
\pgfpathlineto{\pgfqpoint{1.002327in}{1.487228in}}%
\pgfpathlineto{\pgfqpoint{1.001643in}{1.488506in}}%
\pgfpathlineto{\pgfqpoint{0.988351in}{1.513976in}}%
\pgfpathlineto{\pgfqpoint{0.975326in}{1.539445in}}%
\pgfpathlineto{\pgfqpoint{0.972754in}{1.544605in}}%
\pgfpathlineto{\pgfqpoint{0.962735in}{1.564915in}}%
\pgfpathlineto{\pgfqpoint{0.950454in}{1.590384in}}%
\pgfpathlineto{\pgfqpoint{0.943181in}{1.605842in}}%
\pgfpathlineto{\pgfqpoint{0.938520in}{1.615854in}}%
\pgfpathlineto{\pgfqpoint{0.926974in}{1.641323in}}%
\pgfpathlineto{\pgfqpoint{0.915697in}{1.666793in}}%
\pgfpathlineto{\pgfqpoint{0.913607in}{1.671652in}}%
\pgfpathlineto{\pgfqpoint{0.904840in}{1.692262in}}%
\pgfpathlineto{\pgfqpoint{0.894288in}{1.717732in}}%
\pgfpathlineto{\pgfqpoint{0.884034in}{1.743136in}}%
\pgfpathlineto{\pgfqpoint{0.884008in}{1.743202in}}%
\pgfpathlineto{\pgfqpoint{0.874167in}{1.768671in}}%
\pgfpathlineto{\pgfqpoint{0.864598in}{1.794141in}}%
\pgfpathlineto{\pgfqpoint{0.855302in}{1.819610in}}%
\pgfpathlineto{\pgfqpoint{0.854460in}{1.821997in}}%
\pgfusepath{stroke}%
\end{pgfscope}%
\begin{pgfscope}%
\pgfpathrectangle{\pgfqpoint{0.854460in}{0.571603in}}{\pgfqpoint{5.885100in}{5.068436in}}%
\pgfusepath{clip}%
\pgfsetbuttcap%
\pgfsetroundjoin%
\pgfsetlinewidth{1.505625pt}%
\definecolor{currentstroke}{rgb}{0.187231,0.414746,0.556547}%
\pgfsetstrokecolor{currentstroke}%
\pgfsetdash{}{0pt}%
\pgfpathmoveto{\pgfqpoint{0.854460in}{3.448520in}}%
\pgfpathlineto{\pgfqpoint{0.873372in}{3.500599in}}%
\pgfpathlineto{\pgfqpoint{0.903426in}{3.577007in}}%
\pgfpathlineto{\pgfqpoint{0.936405in}{3.653416in}}%
\pgfpathlineto{\pgfqpoint{0.960110in}{3.704355in}}%
\pgfpathlineto{\pgfqpoint{0.998230in}{3.780764in}}%
\pgfpathlineto{\pgfqpoint{1.031901in}{3.843297in}}%
\pgfpathlineto{\pgfqpoint{1.069215in}{3.908111in}}%
\pgfpathlineto{\pgfqpoint{1.116462in}{3.984520in}}%
\pgfpathlineto{\pgfqpoint{1.150194in}{4.035732in}}%
\pgfpathlineto{\pgfqpoint{1.203682in}{4.111867in}}%
\pgfpathlineto{\pgfqpoint{1.241686in}{4.162807in}}%
\pgfpathlineto{\pgfqpoint{1.302331in}{4.239215in}}%
\pgfpathlineto{\pgfqpoint{1.357208in}{4.303862in}}%
\pgfpathlineto{\pgfqpoint{1.416354in}{4.369441in}}%
\pgfpathlineto{\pgfqpoint{1.475501in}{4.431120in}}%
\pgfpathlineto{\pgfqpoint{1.539295in}{4.493910in}}%
\pgfpathlineto{\pgfqpoint{1.593896in}{4.544849in}}%
\pgfpathlineto{\pgfqpoint{1.652941in}{4.597230in}}%
\pgfpathlineto{\pgfqpoint{1.712088in}{4.647200in}}%
\pgfpathlineto{\pgfqpoint{1.774853in}{4.697667in}}%
\pgfpathlineto{\pgfqpoint{1.841465in}{4.748606in}}%
\pgfpathlineto{\pgfqpoint{1.919102in}{4.804913in}}%
\pgfpathlineto{\pgfqpoint{1.985299in}{4.850484in}}%
\pgfpathlineto{\pgfqpoint{2.066968in}{4.903881in}}%
\pgfpathlineto{\pgfqpoint{2.155689in}{4.958565in}}%
\pgfpathlineto{\pgfqpoint{2.244409in}{5.010128in}}%
\pgfpathlineto{\pgfqpoint{2.333129in}{5.058798in}}%
\pgfpathlineto{\pgfqpoint{2.422646in}{5.105179in}}%
\pgfpathlineto{\pgfqpoint{2.527450in}{5.156118in}}%
\pgfpathlineto{\pgfqpoint{2.599289in}{5.189120in}}%
\pgfpathlineto{\pgfqpoint{2.699077in}{5.232527in}}%
\pgfpathlineto{\pgfqpoint{2.806303in}{5.276074in}}%
\pgfpathlineto{\pgfqpoint{2.895023in}{5.309871in}}%
\pgfpathlineto{\pgfqpoint{3.013316in}{5.351686in}}%
\pgfpathlineto{\pgfqpoint{3.116367in}{5.385344in}}%
\pgfpathlineto{\pgfqpoint{3.220330in}{5.416626in}}%
\pgfpathlineto{\pgfqpoint{3.309050in}{5.441253in}}%
\pgfpathlineto{\pgfqpoint{3.397770in}{5.463970in}}%
\pgfpathlineto{\pgfqpoint{3.516063in}{5.491161in}}%
\pgfpathlineto{\pgfqpoint{3.623094in}{5.512691in}}%
\pgfpathlineto{\pgfqpoint{3.693504in}{5.525143in}}%
\pgfpathlineto{\pgfqpoint{3.782224in}{5.538994in}}%
\pgfpathlineto{\pgfqpoint{3.870944in}{5.550368in}}%
\pgfpathlineto{\pgfqpoint{3.959664in}{5.559367in}}%
\pgfpathlineto{\pgfqpoint{4.048384in}{5.565637in}}%
\pgfpathlineto{\pgfqpoint{4.137104in}{5.568847in}}%
\pgfpathlineto{\pgfqpoint{4.196251in}{5.569158in}}%
\pgfpathlineto{\pgfqpoint{4.255398in}{5.567847in}}%
\pgfpathlineto{\pgfqpoint{4.329285in}{5.563630in}}%
\pgfpathlineto{\pgfqpoint{4.373691in}{5.559558in}}%
\pgfpathlineto{\pgfqpoint{4.432838in}{5.552065in}}%
\pgfpathlineto{\pgfqpoint{4.491984in}{5.542051in}}%
\pgfpathlineto{\pgfqpoint{4.521558in}{5.535926in}}%
\pgfpathlineto{\pgfqpoint{4.580705in}{5.520993in}}%
\pgfpathlineto{\pgfqpoint{4.610278in}{5.512204in}}%
\pgfpathlineto{\pgfqpoint{4.669425in}{5.490929in}}%
\pgfpathlineto{\pgfqpoint{4.698998in}{5.478253in}}%
\pgfpathlineto{\pgfqpoint{4.733190in}{5.461752in}}%
\pgfpathlineto{\pgfqpoint{4.777773in}{5.436283in}}%
\pgfpathlineto{\pgfqpoint{4.787718in}{5.429891in}}%
\pgfpathlineto{\pgfqpoint{4.817291in}{5.409300in}}%
\pgfpathlineto{\pgfqpoint{4.847183in}{5.385344in}}%
\pgfpathlineto{\pgfqpoint{4.876438in}{5.358092in}}%
\pgfpathlineto{\pgfqpoint{4.906012in}{5.325655in}}%
\pgfpathlineto{\pgfqpoint{4.919488in}{5.308935in}}%
\pgfpathlineto{\pgfqpoint{4.937987in}{5.283466in}}%
\pgfpathlineto{\pgfqpoint{4.965158in}{5.238848in}}%
\pgfpathlineto{\pgfqpoint{4.981282in}{5.207057in}}%
\pgfpathlineto{\pgfqpoint{4.994732in}{5.176335in}}%
\pgfpathlineto{\pgfqpoint{5.011351in}{5.130649in}}%
\pgfpathlineto{\pgfqpoint{5.026062in}{5.079709in}}%
\pgfpathlineto{\pgfqpoint{5.037303in}{5.028770in}}%
\pgfpathlineto{\pgfqpoint{5.045871in}{4.977831in}}%
\pgfpathlineto{\pgfqpoint{5.052208in}{4.926892in}}%
\pgfpathlineto{\pgfqpoint{5.056640in}{4.875953in}}%
\pgfpathlineto{\pgfqpoint{5.060495in}{4.799545in}}%
\pgfpathlineto{\pgfqpoint{5.061808in}{4.723136in}}%
\pgfpathlineto{\pgfqpoint{5.060792in}{4.621258in}}%
\pgfpathlineto{\pgfqpoint{5.056856in}{4.493910in}}%
\pgfpathlineto{\pgfqpoint{5.043871in}{4.137337in}}%
\pgfpathlineto{\pgfqpoint{5.042193in}{4.009989in}}%
\pgfpathlineto{\pgfqpoint{5.042923in}{3.908111in}}%
\pgfpathlineto{\pgfqpoint{5.045854in}{3.806233in}}%
\pgfpathlineto{\pgfqpoint{5.051277in}{3.704355in}}%
\pgfpathlineto{\pgfqpoint{5.059383in}{3.602477in}}%
\pgfpathlineto{\pgfqpoint{5.067367in}{3.526068in}}%
\pgfpathlineto{\pgfqpoint{5.077108in}{3.449660in}}%
\pgfpathlineto{\pgfqpoint{5.088635in}{3.373251in}}%
\pgfpathlineto{\pgfqpoint{5.102002in}{3.296843in}}%
\pgfpathlineto{\pgfqpoint{5.117319in}{3.220434in}}%
\pgfpathlineto{\pgfqpoint{5.134573in}{3.144025in}}%
\pgfpathlineto{\pgfqpoint{5.153847in}{3.067617in}}%
\pgfpathlineto{\pgfqpoint{5.175207in}{2.991208in}}%
\pgfpathlineto{\pgfqpoint{5.201745in}{2.905277in}}%
\pgfpathlineto{\pgfqpoint{5.224170in}{2.838391in}}%
\pgfpathlineto{\pgfqpoint{5.251871in}{2.761983in}}%
\pgfpathlineto{\pgfqpoint{5.281756in}{2.685574in}}%
\pgfpathlineto{\pgfqpoint{5.320039in}{2.595128in}}%
\pgfpathlineto{\pgfqpoint{5.349612in}{2.529697in}}%
\pgfpathlineto{\pgfqpoint{5.384698in}{2.456348in}}%
\pgfpathlineto{\pgfqpoint{5.423475in}{2.379940in}}%
\pgfpathlineto{\pgfqpoint{5.467906in}{2.297534in}}%
\pgfpathlineto{\pgfqpoint{5.507858in}{2.227123in}}%
\pgfpathlineto{\pgfqpoint{5.556626in}{2.145630in}}%
\pgfpathlineto{\pgfqpoint{5.601340in}{2.074305in}}%
\pgfpathlineto{\pgfqpoint{5.651521in}{1.997897in}}%
\pgfpathlineto{\pgfqpoint{5.704492in}{1.920787in}}%
\pgfpathlineto{\pgfqpoint{5.763639in}{1.838404in}}%
\pgfpathlineto{\pgfqpoint{5.822786in}{1.759471in}}%
\pgfpathlineto{\pgfqpoint{5.881933in}{1.683589in}}%
\pgfpathlineto{\pgfqpoint{5.941079in}{1.610423in}}%
\pgfpathlineto{\pgfqpoint{6.000434in}{1.539445in}}%
\pgfpathlineto{\pgfqpoint{6.066486in}{1.463037in}}%
\pgfpathlineto{\pgfqpoint{6.158069in}{1.361159in}}%
\pgfpathlineto{\pgfqpoint{6.253608in}{1.259281in}}%
\pgfpathlineto{\pgfqpoint{6.355107in}{1.155391in}}%
\pgfpathlineto{\pgfqpoint{6.456404in}{1.055524in}}%
\pgfpathlineto{\pgfqpoint{6.563597in}{0.953646in}}%
\pgfpathlineto{\pgfqpoint{6.680414in}{0.846461in}}%
\pgfpathlineto{\pgfqpoint{6.739560in}{0.793581in}}%
\pgfpathlineto{\pgfqpoint{6.739560in}{0.793581in}}%
\pgfusepath{stroke}%
\end{pgfscope}%
\begin{pgfscope}%
\pgfpathrectangle{\pgfqpoint{0.854460in}{0.571603in}}{\pgfqpoint{5.885100in}{5.068436in}}%
\pgfusepath{clip}%
\pgfsetbuttcap%
\pgfsetroundjoin%
\pgfsetlinewidth{1.505625pt}%
\definecolor{currentstroke}{rgb}{0.179019,0.433756,0.557430}%
\pgfsetstrokecolor{currentstroke}%
\pgfsetdash{}{0pt}%
\pgfpathmoveto{\pgfqpoint{1.688535in}{0.571603in}}%
\pgfpathlineto{\pgfqpoint{1.682515in}{0.576783in}}%
\pgfpathlineto{\pgfqpoint{1.659034in}{0.597073in}}%
\pgfpathlineto{\pgfqpoint{1.652941in}{0.602416in}}%
\pgfpathlineto{\pgfqpoint{1.630094in}{0.622542in}}%
\pgfpathlineto{\pgfqpoint{1.623368in}{0.628555in}}%
\pgfpathlineto{\pgfqpoint{1.601706in}{0.648012in}}%
\pgfpathlineto{\pgfqpoint{1.593795in}{0.655224in}}%
\pgfpathlineto{\pgfqpoint{1.573864in}{0.673481in}}%
\pgfpathlineto{\pgfqpoint{1.564221in}{0.682446in}}%
\pgfpathlineto{\pgfqpoint{1.546558in}{0.698951in}}%
\pgfpathlineto{\pgfqpoint{1.534648in}{0.710247in}}%
\pgfpathlineto{\pgfqpoint{1.519781in}{0.724420in}}%
\pgfpathlineto{\pgfqpoint{1.505074in}{0.738651in}}%
\pgfpathlineto{\pgfqpoint{1.493523in}{0.749890in}}%
\pgfpathlineto{\pgfqpoint{1.475501in}{0.767687in}}%
\pgfpathlineto{\pgfqpoint{1.467775in}{0.775360in}}%
\pgfpathlineto{\pgfqpoint{1.445928in}{0.797381in}}%
\pgfpathlineto{\pgfqpoint{1.442527in}{0.800829in}}%
\pgfpathlineto{\pgfqpoint{1.417792in}{0.826299in}}%
\pgfpathlineto{\pgfqpoint{1.416354in}{0.827803in}}%
\pgfpathlineto{\pgfqpoint{1.393595in}{0.851768in}}%
\pgfpathlineto{\pgfqpoint{1.386781in}{0.859052in}}%
\pgfpathlineto{\pgfqpoint{1.369876in}{0.877238in}}%
\pgfpathlineto{\pgfqpoint{1.357208in}{0.891073in}}%
\pgfpathlineto{\pgfqpoint{1.346624in}{0.902707in}}%
\pgfpathlineto{\pgfqpoint{1.327634in}{0.923900in}}%
\pgfpathlineto{\pgfqpoint{1.323828in}{0.928177in}}%
\pgfpathlineto{\pgfqpoint{1.301529in}{0.953646in}}%
\pgfpathlineto{\pgfqpoint{1.298061in}{0.957675in}}%
\pgfpathlineto{\pgfqpoint{1.279729in}{0.979116in}}%
\pgfpathlineto{\pgfqpoint{1.268488in}{0.992468in}}%
\pgfpathlineto{\pgfqpoint{1.258358in}{1.004585in}}%
\pgfpathlineto{\pgfqpoint{1.238914in}{1.028208in}}%
\pgfpathlineto{\pgfqpoint{1.237405in}{1.030055in}}%
\pgfpathlineto{\pgfqpoint{1.216973in}{1.055524in}}%
\pgfpathlineto{\pgfqpoint{1.209341in}{1.065195in}}%
\pgfpathlineto{\pgfqpoint{1.196965in}{1.080994in}}%
\pgfpathlineto{\pgfqpoint{1.179767in}{1.103297in}}%
\pgfpathlineto{\pgfqpoint{1.177345in}{1.106463in}}%
\pgfpathlineto{\pgfqpoint{1.158224in}{1.131933in}}%
\pgfpathlineto{\pgfqpoint{1.150194in}{1.142811in}}%
\pgfpathlineto{\pgfqpoint{1.139510in}{1.157402in}}%
\pgfpathlineto{\pgfqpoint{1.121162in}{1.182872in}}%
\pgfpathlineto{\pgfqpoint{1.120621in}{1.183638in}}%
\pgfpathlineto{\pgfqpoint{1.103330in}{1.208341in}}%
\pgfpathlineto{\pgfqpoint{1.091047in}{1.226179in}}%
\pgfpathlineto{\pgfqpoint{1.085837in}{1.233811in}}%
\pgfpathlineto{\pgfqpoint{1.068782in}{1.259281in}}%
\pgfpathlineto{\pgfqpoint{1.061474in}{1.270400in}}%
\pgfpathlineto{\pgfqpoint{1.052126in}{1.284750in}}%
\pgfpathlineto{\pgfqpoint{1.035836in}{1.310220in}}%
\pgfpathlineto{\pgfqpoint{1.031901in}{1.316503in}}%
\pgfpathlineto{\pgfqpoint{1.019992in}{1.335689in}}%
\pgfpathlineto{\pgfqpoint{1.004464in}{1.361159in}}%
\pgfpathlineto{\pgfqpoint{1.002327in}{1.364743in}}%
\pgfpathlineto{\pgfqpoint{0.989402in}{1.386628in}}%
\pgfpathlineto{\pgfqpoint{0.974633in}{1.412098in}}%
\pgfpathlineto{\pgfqpoint{0.972754in}{1.415414in}}%
\pgfpathlineto{\pgfqpoint{0.960324in}{1.437567in}}%
\pgfpathlineto{\pgfqpoint{0.946306in}{1.463037in}}%
\pgfpathlineto{\pgfqpoint{0.943181in}{1.468851in}}%
\pgfpathlineto{\pgfqpoint{0.932720in}{1.488506in}}%
\pgfpathlineto{\pgfqpoint{0.919447in}{1.513976in}}%
\pgfpathlineto{\pgfqpoint{0.913607in}{1.525447in}}%
\pgfpathlineto{\pgfqpoint{0.906553in}{1.539445in}}%
\pgfpathlineto{\pgfqpoint{0.894016in}{1.564915in}}%
\pgfpathlineto{\pgfqpoint{0.884034in}{1.585656in}}%
\pgfpathlineto{\pgfqpoint{0.881782in}{1.590384in}}%
\pgfpathlineto{\pgfqpoint{0.869972in}{1.615854in}}%
\pgfpathlineto{\pgfqpoint{0.858429in}{1.641323in}}%
\pgfpathlineto{\pgfqpoint{0.854460in}{1.650320in}}%
\pgfusepath{stroke}%
\end{pgfscope}%
\begin{pgfscope}%
\pgfpathrectangle{\pgfqpoint{0.854460in}{0.571603in}}{\pgfqpoint{5.885100in}{5.068436in}}%
\pgfusepath{clip}%
\pgfsetbuttcap%
\pgfsetroundjoin%
\pgfsetlinewidth{1.505625pt}%
\definecolor{currentstroke}{rgb}{0.179019,0.433756,0.557430}%
\pgfsetstrokecolor{currentstroke}%
\pgfsetdash{}{0pt}%
\pgfpathmoveto{\pgfqpoint{0.854460in}{3.647376in}}%
\pgfpathlineto{\pgfqpoint{0.884034in}{3.713075in}}%
\pgfpathlineto{\pgfqpoint{0.916851in}{3.780764in}}%
\pgfpathlineto{\pgfqpoint{0.956944in}{3.857172in}}%
\pgfpathlineto{\pgfqpoint{1.000263in}{3.933581in}}%
\pgfpathlineto{\pgfqpoint{1.031901in}{3.985823in}}%
\pgfpathlineto{\pgfqpoint{1.080423in}{4.060928in}}%
\pgfpathlineto{\pgfqpoint{1.120621in}{4.119342in}}%
\pgfpathlineto{\pgfqpoint{1.171091in}{4.188276in}}%
\pgfpathlineto{\pgfqpoint{1.210498in}{4.239215in}}%
\pgfpathlineto{\pgfqpoint{1.273359in}{4.315624in}}%
\pgfpathlineto{\pgfqpoint{1.327634in}{4.377493in}}%
\pgfpathlineto{\pgfqpoint{1.388507in}{4.442971in}}%
\pgfpathlineto{\pgfqpoint{1.445928in}{4.501217in}}%
\pgfpathlineto{\pgfqpoint{1.505074in}{4.558118in}}%
\pgfpathlineto{\pgfqpoint{1.574462in}{4.621258in}}%
\pgfpathlineto{\pgfqpoint{1.633514in}{4.672197in}}%
\pgfpathlineto{\pgfqpoint{1.695484in}{4.723136in}}%
\pgfpathlineto{\pgfqpoint{1.771235in}{4.782220in}}%
\pgfpathlineto{\pgfqpoint{1.830381in}{4.826159in}}%
\pgfpathlineto{\pgfqpoint{1.919102in}{4.888582in}}%
\pgfpathlineto{\pgfqpoint{1.978248in}{4.928217in}}%
\pgfpathlineto{\pgfqpoint{2.066968in}{4.984695in}}%
\pgfpathlineto{\pgfqpoint{2.155689in}{5.038033in}}%
\pgfpathlineto{\pgfqpoint{2.244409in}{5.088461in}}%
\pgfpathlineto{\pgfqpoint{2.333129in}{5.136193in}}%
\pgfpathlineto{\pgfqpoint{2.422174in}{5.181588in}}%
\pgfpathlineto{\pgfqpoint{2.528393in}{5.232527in}}%
\pgfpathlineto{\pgfqpoint{2.628862in}{5.277808in}}%
\pgfpathlineto{\pgfqpoint{2.717582in}{5.315554in}}%
\pgfpathlineto{\pgfqpoint{2.828275in}{5.359874in}}%
\pgfpathlineto{\pgfqpoint{2.924596in}{5.395948in}}%
\pgfpathlineto{\pgfqpoint{3.013316in}{5.427282in}}%
\pgfpathlineto{\pgfqpoint{3.117372in}{5.461752in}}%
\pgfpathlineto{\pgfqpoint{3.220330in}{5.493412in}}%
\pgfpathlineto{\pgfqpoint{3.338623in}{5.526868in}}%
\pgfpathlineto{\pgfqpoint{3.427343in}{5.549928in}}%
\pgfpathlineto{\pgfqpoint{3.516063in}{5.571232in}}%
\pgfpathlineto{\pgfqpoint{3.604783in}{5.590774in}}%
\pgfpathlineto{\pgfqpoint{3.723077in}{5.613898in}}%
\pgfpathlineto{\pgfqpoint{3.811797in}{5.628907in}}%
\pgfpathlineto{\pgfqpoint{3.886176in}{5.640039in}}%
\pgfpathlineto{\pgfqpoint{3.886176in}{5.640039in}}%
\pgfusepath{stroke}%
\end{pgfscope}%
\begin{pgfscope}%
\pgfpathrectangle{\pgfqpoint{0.854460in}{0.571603in}}{\pgfqpoint{5.885100in}{5.068436in}}%
\pgfusepath{clip}%
\pgfsetbuttcap%
\pgfsetroundjoin%
\pgfsetlinewidth{1.505625pt}%
\definecolor{currentstroke}{rgb}{0.179019,0.433756,0.557430}%
\pgfsetstrokecolor{currentstroke}%
\pgfsetdash{}{0pt}%
\pgfpathmoveto{\pgfqpoint{4.649720in}{5.640039in}}%
\pgfpathlineto{\pgfqpoint{4.698998in}{5.628676in}}%
\pgfpathlineto{\pgfqpoint{4.758145in}{5.611895in}}%
\pgfpathlineto{\pgfqpoint{4.817291in}{5.590698in}}%
\pgfpathlineto{\pgfqpoint{4.846865in}{5.578015in}}%
\pgfpathlineto{\pgfqpoint{4.877169in}{5.563630in}}%
\pgfpathlineto{\pgfqpoint{4.922579in}{5.538161in}}%
\pgfpathlineto{\pgfqpoint{4.960719in}{5.512691in}}%
\pgfpathlineto{\pgfqpoint{4.965158in}{5.509403in}}%
\pgfpathlineto{\pgfqpoint{4.994732in}{5.485842in}}%
\pgfpathlineto{\pgfqpoint{5.024305in}{5.458392in}}%
\pgfpathlineto{\pgfqpoint{5.053878in}{5.425965in}}%
\pgfpathlineto{\pgfqpoint{5.066120in}{5.410813in}}%
\pgfpathlineto{\pgfqpoint{5.084649in}{5.385344in}}%
\pgfpathlineto{\pgfqpoint{5.100776in}{5.359874in}}%
\pgfpathlineto{\pgfqpoint{5.115087in}{5.334405in}}%
\pgfpathlineto{\pgfqpoint{5.127560in}{5.308935in}}%
\pgfpathlineto{\pgfqpoint{5.142599in}{5.273272in}}%
\pgfpathlineto{\pgfqpoint{5.156753in}{5.232527in}}%
\pgfpathlineto{\pgfqpoint{5.170667in}{5.181588in}}%
\pgfpathlineto{\pgfqpoint{5.180976in}{5.130649in}}%
\pgfpathlineto{\pgfqpoint{5.188457in}{5.079709in}}%
\pgfpathlineto{\pgfqpoint{5.193598in}{5.028770in}}%
\pgfpathlineto{\pgfqpoint{5.196789in}{4.977831in}}%
\pgfpathlineto{\pgfqpoint{5.198625in}{4.901423in}}%
\pgfpathlineto{\pgfqpoint{5.197722in}{4.825014in}}%
\pgfpathlineto{\pgfqpoint{5.193512in}{4.723136in}}%
\pgfpathlineto{\pgfqpoint{5.185294in}{4.595788in}}%
\pgfpathlineto{\pgfqpoint{5.156138in}{4.188276in}}%
\pgfpathlineto{\pgfqpoint{5.150426in}{4.060928in}}%
\pgfpathlineto{\pgfqpoint{5.148021in}{3.959050in}}%
\pgfpathlineto{\pgfqpoint{5.147877in}{3.857172in}}%
\pgfpathlineto{\pgfqpoint{5.150262in}{3.755294in}}%
\pgfpathlineto{\pgfqpoint{5.155417in}{3.653416in}}%
\pgfpathlineto{\pgfqpoint{5.161236in}{3.577007in}}%
\pgfpathlineto{\pgfqpoint{5.168826in}{3.500599in}}%
\pgfpathlineto{\pgfqpoint{5.178203in}{3.424190in}}%
\pgfpathlineto{\pgfqpoint{5.189459in}{3.347782in}}%
\pgfpathlineto{\pgfqpoint{5.202701in}{3.271373in}}%
\pgfpathlineto{\pgfqpoint{5.217862in}{3.194965in}}%
\pgfpathlineto{\pgfqpoint{5.235115in}{3.118556in}}%
\pgfpathlineto{\pgfqpoint{5.254413in}{3.042147in}}%
\pgfpathlineto{\pgfqpoint{5.275820in}{2.965739in}}%
\pgfpathlineto{\pgfqpoint{5.299392in}{2.889330in}}%
\pgfpathlineto{\pgfqpoint{5.325143in}{2.812922in}}%
\pgfpathlineto{\pgfqpoint{5.353090in}{2.736513in}}%
\pgfpathlineto{\pgfqpoint{5.383253in}{2.660104in}}%
\pgfpathlineto{\pgfqpoint{5.415653in}{2.583696in}}%
\pgfpathlineto{\pgfqpoint{5.450313in}{2.507287in}}%
\pgfpathlineto{\pgfqpoint{5.487261in}{2.430879in}}%
\pgfpathlineto{\pgfqpoint{5.527052in}{2.353485in}}%
\pgfpathlineto{\pgfqpoint{5.568022in}{2.278062in}}%
\pgfpathlineto{\pgfqpoint{5.615772in}{2.195110in}}%
\pgfpathlineto{\pgfqpoint{5.658002in}{2.125244in}}%
\pgfpathlineto{\pgfqpoint{5.706489in}{2.048836in}}%
\pgfpathlineto{\pgfqpoint{5.763639in}{1.963118in}}%
\pgfpathlineto{\pgfqpoint{5.810332in}{1.896019in}}%
\pgfpathlineto{\pgfqpoint{5.881933in}{1.797925in}}%
\pgfpathlineto{\pgfqpoint{5.923462in}{1.743202in}}%
\pgfpathlineto{\pgfqpoint{6.000226in}{1.646052in}}%
\pgfpathlineto{\pgfqpoint{6.059373in}{1.574188in}}%
\pgfpathlineto{\pgfqpoint{6.118520in}{1.504662in}}%
\pgfpathlineto{\pgfqpoint{6.177666in}{1.437249in}}%
\pgfpathlineto{\pgfqpoint{6.266386in}{1.339716in}}%
\pgfpathlineto{\pgfqpoint{6.325533in}{1.276837in}}%
\pgfpathlineto{\pgfqpoint{6.414253in}{1.185399in}}%
\pgfpathlineto{\pgfqpoint{6.493379in}{1.106463in}}%
\pgfpathlineto{\pgfqpoint{6.599030in}{1.004585in}}%
\pgfpathlineto{\pgfqpoint{6.709987in}{0.901421in}}%
\pgfpathlineto{\pgfqpoint{6.739560in}{0.874519in}}%
\pgfpathlineto{\pgfqpoint{6.739560in}{0.874519in}}%
\pgfusepath{stroke}%
\end{pgfscope}%
\begin{pgfscope}%
\pgfpathrectangle{\pgfqpoint{0.854460in}{0.571603in}}{\pgfqpoint{5.885100in}{5.068436in}}%
\pgfusepath{clip}%
\pgfsetbuttcap%
\pgfsetroundjoin%
\pgfsetlinewidth{1.505625pt}%
\definecolor{currentstroke}{rgb}{0.169646,0.456262,0.558030}%
\pgfsetstrokecolor{currentstroke}%
\pgfsetdash{}{0pt}%
\pgfpathmoveto{\pgfqpoint{1.614939in}{0.571603in}}%
\pgfpathlineto{\pgfqpoint{1.593795in}{0.590032in}}%
\pgfpathlineto{\pgfqpoint{1.585752in}{0.597073in}}%
\pgfpathlineto{\pgfqpoint{1.564221in}{0.616199in}}%
\pgfpathlineto{\pgfqpoint{1.557113in}{0.622542in}}%
\pgfpathlineto{\pgfqpoint{1.534648in}{0.642887in}}%
\pgfpathlineto{\pgfqpoint{1.529015in}{0.648012in}}%
\pgfpathlineto{\pgfqpoint{1.505074in}{0.670118in}}%
\pgfpathlineto{\pgfqpoint{1.501451in}{0.673481in}}%
\pgfpathlineto{\pgfqpoint{1.475501in}{0.697918in}}%
\pgfpathlineto{\pgfqpoint{1.474410in}{0.698951in}}%
\pgfpathlineto{\pgfqpoint{1.447914in}{0.724420in}}%
\pgfpathlineto{\pgfqpoint{1.445928in}{0.726359in}}%
\pgfpathlineto{\pgfqpoint{1.421949in}{0.749890in}}%
\pgfpathlineto{\pgfqpoint{1.416354in}{0.755462in}}%
\pgfpathlineto{\pgfqpoint{1.396488in}{0.775360in}}%
\pgfpathlineto{\pgfqpoint{1.386781in}{0.785228in}}%
\pgfpathlineto{\pgfqpoint{1.371524in}{0.800829in}}%
\pgfpathlineto{\pgfqpoint{1.357208in}{0.815687in}}%
\pgfpathlineto{\pgfqpoint{1.347045in}{0.826299in}}%
\pgfpathlineto{\pgfqpoint{1.327634in}{0.846870in}}%
\pgfpathlineto{\pgfqpoint{1.323041in}{0.851768in}}%
\pgfpathlineto{\pgfqpoint{1.299525in}{0.877238in}}%
\pgfpathlineto{\pgfqpoint{1.298061in}{0.878849in}}%
\pgfpathlineto{\pgfqpoint{1.276538in}{0.902707in}}%
\pgfpathlineto{\pgfqpoint{1.268488in}{0.911766in}}%
\pgfpathlineto{\pgfqpoint{1.254002in}{0.928177in}}%
\pgfpathlineto{\pgfqpoint{1.238914in}{0.945530in}}%
\pgfpathlineto{\pgfqpoint{1.231906in}{0.953646in}}%
\pgfpathlineto{\pgfqpoint{1.210252in}{0.979116in}}%
\pgfpathlineto{\pgfqpoint{1.209341in}{0.980207in}}%
\pgfpathlineto{\pgfqpoint{1.189128in}{1.004585in}}%
\pgfpathlineto{\pgfqpoint{1.179767in}{1.016050in}}%
\pgfpathlineto{\pgfqpoint{1.168416in}{1.030055in}}%
\pgfpathlineto{\pgfqpoint{1.150194in}{1.052886in}}%
\pgfpathlineto{\pgfqpoint{1.148105in}{1.055524in}}%
\pgfpathlineto{\pgfqpoint{1.128297in}{1.080994in}}%
\pgfpathlineto{\pgfqpoint{1.120621in}{1.091028in}}%
\pgfpathlineto{\pgfqpoint{1.108904in}{1.106463in}}%
\pgfpathlineto{\pgfqpoint{1.091047in}{1.130361in}}%
\pgfpathlineto{\pgfqpoint{1.089882in}{1.131933in}}%
\pgfpathlineto{\pgfqpoint{1.071368in}{1.157402in}}%
\pgfpathlineto{\pgfqpoint{1.061474in}{1.171239in}}%
\pgfpathlineto{\pgfqpoint{1.053224in}{1.182872in}}%
\pgfpathlineto{\pgfqpoint{1.035474in}{1.208341in}}%
\pgfpathlineto{\pgfqpoint{1.031901in}{1.213570in}}%
\pgfpathlineto{\pgfqpoint{1.018184in}{1.233811in}}%
\pgfpathlineto{\pgfqpoint{1.002327in}{1.257597in}}%
\pgfpathlineto{\pgfqpoint{1.001215in}{1.259281in}}%
\pgfpathlineto{\pgfqpoint{0.984736in}{1.284750in}}%
\pgfpathlineto{\pgfqpoint{0.972754in}{1.303589in}}%
\pgfpathlineto{\pgfqpoint{0.968574in}{1.310220in}}%
\pgfpathlineto{\pgfqpoint{0.952849in}{1.335689in}}%
\pgfpathlineto{\pgfqpoint{0.943181in}{1.351645in}}%
\pgfpathlineto{\pgfqpoint{0.937469in}{1.361159in}}%
\pgfpathlineto{\pgfqpoint{0.922493in}{1.386628in}}%
\pgfpathlineto{\pgfqpoint{0.913607in}{1.402042in}}%
\pgfpathlineto{\pgfqpoint{0.907865in}{1.412098in}}%
\pgfpathlineto{\pgfqpoint{0.893633in}{1.437567in}}%
\pgfpathlineto{\pgfqpoint{0.884034in}{1.455096in}}%
\pgfpathlineto{\pgfqpoint{0.879728in}{1.463037in}}%
\pgfpathlineto{\pgfqpoint{0.866232in}{1.488506in}}%
\pgfpathlineto{\pgfqpoint{0.854460in}{1.511174in}}%
\pgfusepath{stroke}%
\end{pgfscope}%
\begin{pgfscope}%
\pgfpathrectangle{\pgfqpoint{0.854460in}{0.571603in}}{\pgfqpoint{5.885100in}{5.068436in}}%
\pgfusepath{clip}%
\pgfsetbuttcap%
\pgfsetroundjoin%
\pgfsetlinewidth{1.505625pt}%
\definecolor{currentstroke}{rgb}{0.169646,0.456262,0.558030}%
\pgfsetstrokecolor{currentstroke}%
\pgfsetdash{}{0pt}%
\pgfpathmoveto{\pgfqpoint{0.854460in}{3.810981in}}%
\pgfpathlineto{\pgfqpoint{0.891977in}{3.882642in}}%
\pgfpathlineto{\pgfqpoint{0.935116in}{3.959050in}}%
\pgfpathlineto{\pgfqpoint{0.972754in}{4.021268in}}%
\pgfpathlineto{\pgfqpoint{1.014719in}{4.086398in}}%
\pgfpathlineto{\pgfqpoint{1.061474in}{4.154536in}}%
\pgfpathlineto{\pgfqpoint{1.104669in}{4.213746in}}%
\pgfpathlineto{\pgfqpoint{1.150194in}{4.272858in}}%
\pgfpathlineto{\pgfqpoint{1.209341in}{4.345115in}}%
\pgfpathlineto{\pgfqpoint{1.272681in}{4.417502in}}%
\pgfpathlineto{\pgfqpoint{1.327634in}{4.476630in}}%
\pgfpathlineto{\pgfqpoint{1.394801in}{4.544849in}}%
\pgfpathlineto{\pgfqpoint{1.447707in}{4.595788in}}%
\pgfpathlineto{\pgfqpoint{1.505074in}{4.648428in}}%
\pgfpathlineto{\pgfqpoint{1.564221in}{4.700158in}}%
\pgfpathlineto{\pgfqpoint{1.623368in}{4.749551in}}%
\pgfpathlineto{\pgfqpoint{1.686103in}{4.799545in}}%
\pgfpathlineto{\pgfqpoint{1.771235in}{4.863814in}}%
\pgfpathlineto{\pgfqpoint{1.830381in}{4.906309in}}%
\pgfpathlineto{\pgfqpoint{1.919102in}{4.966887in}}%
\pgfpathlineto{\pgfqpoint{1.978248in}{5.005415in}}%
\pgfpathlineto{\pgfqpoint{2.066968in}{5.060458in}}%
\pgfpathlineto{\pgfqpoint{2.155689in}{5.112570in}}%
\pgfpathlineto{\pgfqpoint{2.244409in}{5.161967in}}%
\pgfpathlineto{\pgfqpoint{2.333129in}{5.208850in}}%
\pgfpathlineto{\pgfqpoint{2.431459in}{5.257996in}}%
\pgfpathlineto{\pgfqpoint{2.540142in}{5.309215in}}%
\pgfpathlineto{\pgfqpoint{2.658436in}{5.361392in}}%
\pgfpathlineto{\pgfqpoint{2.778412in}{5.410813in}}%
\pgfpathlineto{\pgfqpoint{2.895023in}{5.455587in}}%
\pgfpathlineto{\pgfqpoint{2.983743in}{5.487703in}}%
\pgfpathlineto{\pgfqpoint{3.102036in}{5.527759in}}%
\pgfpathlineto{\pgfqpoint{3.190756in}{5.555930in}}%
\pgfpathlineto{\pgfqpoint{3.302396in}{5.589100in}}%
\pgfpathlineto{\pgfqpoint{3.397770in}{5.615401in}}%
\pgfpathlineto{\pgfqpoint{3.494076in}{5.640039in}}%
\pgfpathlineto{\pgfqpoint{3.494076in}{5.640039in}}%
\pgfusepath{stroke}%
\end{pgfscope}%
\begin{pgfscope}%
\pgfpathrectangle{\pgfqpoint{0.854460in}{0.571603in}}{\pgfqpoint{5.885100in}{5.068436in}}%
\pgfusepath{clip}%
\pgfsetbuttcap%
\pgfsetroundjoin%
\pgfsetlinewidth{1.505625pt}%
\definecolor{currentstroke}{rgb}{0.169646,0.456262,0.558030}%
\pgfsetstrokecolor{currentstroke}%
\pgfsetdash{}{0pt}%
\pgfpathmoveto{\pgfqpoint{5.057130in}{5.640039in}}%
\pgfpathlineto{\pgfqpoint{5.097244in}{5.614570in}}%
\pgfpathlineto{\pgfqpoint{5.131244in}{5.589100in}}%
\pgfpathlineto{\pgfqpoint{5.160345in}{5.563630in}}%
\pgfpathlineto{\pgfqpoint{5.185468in}{5.538161in}}%
\pgfpathlineto{\pgfqpoint{5.207326in}{5.512691in}}%
\pgfpathlineto{\pgfqpoint{5.231319in}{5.479870in}}%
\pgfpathlineto{\pgfqpoint{5.242946in}{5.461752in}}%
\pgfpathlineto{\pgfqpoint{5.260892in}{5.429783in}}%
\pgfpathlineto{\pgfqpoint{5.270228in}{5.410813in}}%
\pgfpathlineto{\pgfqpoint{5.281351in}{5.385344in}}%
\pgfpathlineto{\pgfqpoint{5.291106in}{5.359874in}}%
\pgfpathlineto{\pgfqpoint{5.306748in}{5.308935in}}%
\pgfpathlineto{\pgfqpoint{5.318399in}{5.257996in}}%
\pgfpathlineto{\pgfqpoint{5.326602in}{5.207057in}}%
\pgfpathlineto{\pgfqpoint{5.332056in}{5.156118in}}%
\pgfpathlineto{\pgfqpoint{5.335238in}{5.105179in}}%
\pgfpathlineto{\pgfqpoint{5.336523in}{5.054240in}}%
\pgfpathlineto{\pgfqpoint{5.335585in}{4.977831in}}%
\pgfpathlineto{\pgfqpoint{5.331986in}{4.901423in}}%
\pgfpathlineto{\pgfqpoint{5.324268in}{4.799545in}}%
\pgfpathlineto{\pgfqpoint{5.311646in}{4.672197in}}%
\pgfpathlineto{\pgfqpoint{5.268436in}{4.264685in}}%
\pgfpathlineto{\pgfqpoint{5.258302in}{4.137337in}}%
\pgfpathlineto{\pgfqpoint{5.252307in}{4.035459in}}%
\pgfpathlineto{\pgfqpoint{5.248580in}{3.933581in}}%
\pgfpathlineto{\pgfqpoint{5.247385in}{3.831703in}}%
\pgfpathlineto{\pgfqpoint{5.248952in}{3.729825in}}%
\pgfpathlineto{\pgfqpoint{5.252069in}{3.653416in}}%
\pgfpathlineto{\pgfqpoint{5.256942in}{3.577007in}}%
\pgfpathlineto{\pgfqpoint{5.263623in}{3.500599in}}%
\pgfpathlineto{\pgfqpoint{5.272155in}{3.424190in}}%
\pgfpathlineto{\pgfqpoint{5.282655in}{3.347782in}}%
\pgfpathlineto{\pgfqpoint{5.295149in}{3.271373in}}%
\pgfpathlineto{\pgfqpoint{5.309656in}{3.194965in}}%
\pgfpathlineto{\pgfqpoint{5.326267in}{3.118556in}}%
\pgfpathlineto{\pgfqpoint{5.344993in}{3.042147in}}%
\pgfpathlineto{\pgfqpoint{5.365848in}{2.965739in}}%
\pgfpathlineto{\pgfqpoint{5.388909in}{2.889330in}}%
\pgfpathlineto{\pgfqpoint{5.414187in}{2.812922in}}%
\pgfpathlineto{\pgfqpoint{5.441698in}{2.736513in}}%
\pgfpathlineto{\pgfqpoint{5.471459in}{2.660104in}}%
\pgfpathlineto{\pgfqpoint{5.503487in}{2.583696in}}%
\pgfpathlineto{\pgfqpoint{5.537806in}{2.507287in}}%
\pgfpathlineto{\pgfqpoint{5.574441in}{2.430879in}}%
\pgfpathlineto{\pgfqpoint{5.615772in}{2.350043in}}%
\pgfpathlineto{\pgfqpoint{5.654677in}{2.278062in}}%
\pgfpathlineto{\pgfqpoint{5.704492in}{2.191206in}}%
\pgfpathlineto{\pgfqpoint{5.744244in}{2.125244in}}%
\pgfpathlineto{\pgfqpoint{5.793213in}{2.047856in}}%
\pgfpathlineto{\pgfqpoint{5.852359in}{1.959033in}}%
\pgfpathlineto{\pgfqpoint{5.896172in}{1.896019in}}%
\pgfpathlineto{\pgfqpoint{5.951509in}{1.819610in}}%
\pgfpathlineto{\pgfqpoint{6.009188in}{1.743202in}}%
\pgfpathlineto{\pgfqpoint{6.069200in}{1.666793in}}%
\pgfpathlineto{\pgfqpoint{6.148093in}{1.570618in}}%
\pgfpathlineto{\pgfqpoint{6.207240in}{1.501266in}}%
\pgfpathlineto{\pgfqpoint{6.266386in}{1.434038in}}%
\pgfpathlineto{\pgfqpoint{6.332500in}{1.361159in}}%
\pgfpathlineto{\pgfqpoint{6.428449in}{1.259281in}}%
\pgfpathlineto{\pgfqpoint{6.528474in}{1.157402in}}%
\pgfpathlineto{\pgfqpoint{6.632478in}{1.055524in}}%
\pgfpathlineto{\pgfqpoint{6.739560in}{0.954520in}}%
\pgfpathlineto{\pgfqpoint{6.739560in}{0.954520in}}%
\pgfusepath{stroke}%
\end{pgfscope}%
\begin{pgfscope}%
\pgfpathrectangle{\pgfqpoint{0.854460in}{0.571603in}}{\pgfqpoint{5.885100in}{5.068436in}}%
\pgfusepath{clip}%
\pgfsetbuttcap%
\pgfsetroundjoin%
\pgfsetlinewidth{1.505625pt}%
\definecolor{currentstroke}{rgb}{0.162142,0.474838,0.558140}%
\pgfsetstrokecolor{currentstroke}%
\pgfsetdash{}{0pt}%
\pgfpathmoveto{\pgfqpoint{1.543857in}{0.571603in}}%
\pgfpathlineto{\pgfqpoint{1.534648in}{0.579699in}}%
\pgfpathlineto{\pgfqpoint{1.514974in}{0.597073in}}%
\pgfpathlineto{\pgfqpoint{1.505074in}{0.605943in}}%
\pgfpathlineto{\pgfqpoint{1.486635in}{0.622542in}}%
\pgfpathlineto{\pgfqpoint{1.475501in}{0.632712in}}%
\pgfpathlineto{\pgfqpoint{1.458833in}{0.648012in}}%
\pgfpathlineto{\pgfqpoint{1.445928in}{0.660031in}}%
\pgfpathlineto{\pgfqpoint{1.431558in}{0.673481in}}%
\pgfpathlineto{\pgfqpoint{1.416354in}{0.687922in}}%
\pgfpathlineto{\pgfqpoint{1.404804in}{0.698951in}}%
\pgfpathlineto{\pgfqpoint{1.386781in}{0.716413in}}%
\pgfpathlineto{\pgfqpoint{1.378561in}{0.724420in}}%
\pgfpathlineto{\pgfqpoint{1.357208in}{0.745528in}}%
\pgfpathlineto{\pgfqpoint{1.352819in}{0.749890in}}%
\pgfpathlineto{\pgfqpoint{1.327634in}{0.775295in}}%
\pgfpathlineto{\pgfqpoint{1.327571in}{0.775360in}}%
\pgfpathlineto{\pgfqpoint{1.302874in}{0.800829in}}%
\pgfpathlineto{\pgfqpoint{1.298061in}{0.805867in}}%
\pgfpathlineto{\pgfqpoint{1.278659in}{0.826299in}}%
\pgfpathlineto{\pgfqpoint{1.268488in}{0.837170in}}%
\pgfpathlineto{\pgfqpoint{1.254914in}{0.851768in}}%
\pgfpathlineto{\pgfqpoint{1.238914in}{0.869234in}}%
\pgfpathlineto{\pgfqpoint{1.231629in}{0.877238in}}%
\pgfpathlineto{\pgfqpoint{1.209341in}{0.902093in}}%
\pgfpathlineto{\pgfqpoint{1.208794in}{0.902707in}}%
\pgfpathlineto{\pgfqpoint{1.186494in}{0.928177in}}%
\pgfpathlineto{\pgfqpoint{1.179767in}{0.935978in}}%
\pgfpathlineto{\pgfqpoint{1.164638in}{0.953646in}}%
\pgfpathlineto{\pgfqpoint{1.150194in}{0.970770in}}%
\pgfpathlineto{\pgfqpoint{1.143204in}{0.979116in}}%
\pgfpathlineto{\pgfqpoint{1.122206in}{1.004585in}}%
\pgfpathlineto{\pgfqpoint{1.120621in}{1.006543in}}%
\pgfpathlineto{\pgfqpoint{1.101719in}{1.030055in}}%
\pgfpathlineto{\pgfqpoint{1.091047in}{1.043535in}}%
\pgfpathlineto{\pgfqpoint{1.081627in}{1.055524in}}%
\pgfpathlineto{\pgfqpoint{1.061925in}{1.080994in}}%
\pgfpathlineto{\pgfqpoint{1.061474in}{1.081588in}}%
\pgfpathlineto{\pgfqpoint{1.042742in}{1.106463in}}%
\pgfpathlineto{\pgfqpoint{1.031901in}{1.121087in}}%
\pgfpathlineto{\pgfqpoint{1.023924in}{1.131933in}}%
\pgfpathlineto{\pgfqpoint{1.005507in}{1.157402in}}%
\pgfpathlineto{\pgfqpoint{1.002327in}{1.161884in}}%
\pgfpathlineto{\pgfqpoint{0.987558in}{1.182872in}}%
\pgfpathlineto{\pgfqpoint{0.972754in}{1.204249in}}%
\pgfpathlineto{\pgfqpoint{0.969944in}{1.208341in}}%
\pgfpathlineto{\pgfqpoint{0.952796in}{1.233811in}}%
\pgfpathlineto{\pgfqpoint{0.943181in}{1.248343in}}%
\pgfpathlineto{\pgfqpoint{0.936006in}{1.259281in}}%
\pgfpathlineto{\pgfqpoint{0.919610in}{1.284750in}}%
\pgfpathlineto{\pgfqpoint{0.913607in}{1.294258in}}%
\pgfpathlineto{\pgfqpoint{0.903619in}{1.310220in}}%
\pgfpathlineto{\pgfqpoint{0.887971in}{1.335689in}}%
\pgfpathlineto{\pgfqpoint{0.884034in}{1.342232in}}%
\pgfpathlineto{\pgfqpoint{0.872751in}{1.361159in}}%
\pgfpathlineto{\pgfqpoint{0.857846in}{1.386628in}}%
\pgfpathlineto{\pgfqpoint{0.854460in}{1.392541in}}%
\pgfusepath{stroke}%
\end{pgfscope}%
\begin{pgfscope}%
\pgfpathrectangle{\pgfqpoint{0.854460in}{0.571603in}}{\pgfqpoint{5.885100in}{5.068436in}}%
\pgfusepath{clip}%
\pgfsetbuttcap%
\pgfsetroundjoin%
\pgfsetlinewidth{1.505625pt}%
\definecolor{currentstroke}{rgb}{0.162142,0.474838,0.558140}%
\pgfsetstrokecolor{currentstroke}%
\pgfsetdash{}{0pt}%
\pgfpathmoveto{\pgfqpoint{0.854460in}{3.951854in}}%
\pgfpathlineto{\pgfqpoint{0.888318in}{4.009989in}}%
\pgfpathlineto{\pgfqpoint{0.935865in}{4.086398in}}%
\pgfpathlineto{\pgfqpoint{0.972754in}{4.142100in}}%
\pgfpathlineto{\pgfqpoint{1.023191in}{4.213746in}}%
\pgfpathlineto{\pgfqpoint{1.061474in}{4.265207in}}%
\pgfpathlineto{\pgfqpoint{1.121413in}{4.341093in}}%
\pgfpathlineto{\pgfqpoint{1.186126in}{4.417502in}}%
\pgfpathlineto{\pgfqpoint{1.238914in}{4.476180in}}%
\pgfpathlineto{\pgfqpoint{1.304317in}{4.544849in}}%
\pgfpathlineto{\pgfqpoint{1.357208in}{4.597470in}}%
\pgfpathlineto{\pgfqpoint{1.416354in}{4.653449in}}%
\pgfpathlineto{\pgfqpoint{1.475501in}{4.706766in}}%
\pgfpathlineto{\pgfqpoint{1.534648in}{4.757643in}}%
\pgfpathlineto{\pgfqpoint{1.593795in}{4.806283in}}%
\pgfpathlineto{\pgfqpoint{1.652941in}{4.852873in}}%
\pgfpathlineto{\pgfqpoint{1.717389in}{4.901423in}}%
\pgfpathlineto{\pgfqpoint{1.800808in}{4.961153in}}%
\pgfpathlineto{\pgfqpoint{1.862472in}{5.003301in}}%
\pgfpathlineto{\pgfqpoint{1.948675in}{5.059420in}}%
\pgfpathlineto{\pgfqpoint{2.037395in}{5.114142in}}%
\pgfpathlineto{\pgfqpoint{2.126115in}{5.166037in}}%
\pgfpathlineto{\pgfqpoint{2.214835in}{5.215314in}}%
\pgfpathlineto{\pgfqpoint{2.303555in}{5.262167in}}%
\pgfpathlineto{\pgfqpoint{2.396774in}{5.308935in}}%
\pgfpathlineto{\pgfqpoint{2.510569in}{5.362819in}}%
\pgfpathlineto{\pgfqpoint{2.628862in}{5.415370in}}%
\pgfpathlineto{\pgfqpoint{2.747156in}{5.464650in}}%
\pgfpathlineto{\pgfqpoint{2.870417in}{5.512691in}}%
\pgfpathlineto{\pgfqpoint{2.983743in}{5.553976in}}%
\pgfpathlineto{\pgfqpoint{3.086081in}{5.589100in}}%
\pgfpathlineto{\pgfqpoint{3.190756in}{5.622830in}}%
\pgfpathlineto{\pgfqpoint{3.246629in}{5.640039in}}%
\pgfpathlineto{\pgfqpoint{3.246629in}{5.640039in}}%
\pgfusepath{stroke}%
\end{pgfscope}%
\begin{pgfscope}%
\pgfpathrectangle{\pgfqpoint{0.854460in}{0.571603in}}{\pgfqpoint{5.885100in}{5.068436in}}%
\pgfusepath{clip}%
\pgfsetbuttcap%
\pgfsetroundjoin%
\pgfsetlinewidth{1.505625pt}%
\definecolor{currentstroke}{rgb}{0.162142,0.474838,0.558140}%
\pgfsetstrokecolor{currentstroke}%
\pgfsetdash{}{0pt}%
\pgfpathmoveto{\pgfqpoint{5.319889in}{5.640039in}}%
\pgfpathlineto{\pgfqpoint{5.343080in}{5.614570in}}%
\pgfpathlineto{\pgfqpoint{5.363234in}{5.589100in}}%
\pgfpathlineto{\pgfqpoint{5.380860in}{5.563630in}}%
\pgfpathlineto{\pgfqpoint{5.396085in}{5.538161in}}%
\pgfpathlineto{\pgfqpoint{5.409467in}{5.512691in}}%
\pgfpathlineto{\pgfqpoint{5.420967in}{5.487222in}}%
\pgfpathlineto{\pgfqpoint{5.438332in}{5.440450in}}%
\pgfpathlineto{\pgfqpoint{5.447080in}{5.410813in}}%
\pgfpathlineto{\pgfqpoint{5.458713in}{5.359874in}}%
\pgfpathlineto{\pgfqpoint{5.466774in}{5.308935in}}%
\pgfpathlineto{\pgfqpoint{5.471784in}{5.257996in}}%
\pgfpathlineto{\pgfqpoint{5.474329in}{5.207057in}}%
\pgfpathlineto{\pgfqpoint{5.474831in}{5.156118in}}%
\pgfpathlineto{\pgfqpoint{5.473630in}{5.105179in}}%
\pgfpathlineto{\pgfqpoint{5.469261in}{5.028770in}}%
\pgfpathlineto{\pgfqpoint{5.462445in}{4.952362in}}%
\pgfpathlineto{\pgfqpoint{5.450740in}{4.850484in}}%
\pgfpathlineto{\pgfqpoint{5.433594in}{4.723136in}}%
\pgfpathlineto{\pgfqpoint{5.383680in}{4.366563in}}%
\pgfpathlineto{\pgfqpoint{5.368694in}{4.239215in}}%
\pgfpathlineto{\pgfqpoint{5.358744in}{4.137337in}}%
\pgfpathlineto{\pgfqpoint{5.350973in}{4.035459in}}%
\pgfpathlineto{\pgfqpoint{5.345599in}{3.933581in}}%
\pgfpathlineto{\pgfqpoint{5.342913in}{3.831703in}}%
\pgfpathlineto{\pgfqpoint{5.342799in}{3.755294in}}%
\pgfpathlineto{\pgfqpoint{5.344404in}{3.678886in}}%
\pgfpathlineto{\pgfqpoint{5.347804in}{3.602477in}}%
\pgfpathlineto{\pgfqpoint{5.353038in}{3.526068in}}%
\pgfpathlineto{\pgfqpoint{5.360175in}{3.449660in}}%
\pgfpathlineto{\pgfqpoint{5.369303in}{3.373251in}}%
\pgfpathlineto{\pgfqpoint{5.380478in}{3.296843in}}%
\pgfpathlineto{\pgfqpoint{5.393661in}{3.220434in}}%
\pgfpathlineto{\pgfqpoint{5.409018in}{3.144025in}}%
\pgfpathlineto{\pgfqpoint{5.426450in}{3.067617in}}%
\pgfpathlineto{\pgfqpoint{5.446103in}{2.991208in}}%
\pgfpathlineto{\pgfqpoint{5.467985in}{2.914800in}}%
\pgfpathlineto{\pgfqpoint{5.492053in}{2.838391in}}%
\pgfpathlineto{\pgfqpoint{5.518394in}{2.761983in}}%
\pgfpathlineto{\pgfqpoint{5.547023in}{2.685574in}}%
\pgfpathlineto{\pgfqpoint{5.577957in}{2.609165in}}%
\pgfpathlineto{\pgfqpoint{5.615772in}{2.522783in}}%
\pgfpathlineto{\pgfqpoint{5.646802in}{2.456348in}}%
\pgfpathlineto{\pgfqpoint{5.684696in}{2.379940in}}%
\pgfpathlineto{\pgfqpoint{5.724967in}{2.303531in}}%
\pgfpathlineto{\pgfqpoint{5.767609in}{2.227123in}}%
\pgfpathlineto{\pgfqpoint{5.812588in}{2.150714in}}%
\pgfpathlineto{\pgfqpoint{5.859957in}{2.074305in}}%
\pgfpathlineto{\pgfqpoint{5.911506in}{1.995231in}}%
\pgfpathlineto{\pgfqpoint{5.970653in}{1.908913in}}%
\pgfpathlineto{\pgfqpoint{6.016274in}{1.845080in}}%
\pgfpathlineto{\pgfqpoint{6.088946in}{1.748015in}}%
\pgfpathlineto{\pgfqpoint{6.132344in}{1.692262in}}%
\pgfpathlineto{\pgfqpoint{6.207240in}{1.599764in}}%
\pgfpathlineto{\pgfqpoint{6.266386in}{1.529518in}}%
\pgfpathlineto{\pgfqpoint{6.325533in}{1.461507in}}%
\pgfpathlineto{\pgfqpoint{6.392778in}{1.386628in}}%
\pgfpathlineto{\pgfqpoint{6.487894in}{1.284750in}}%
\pgfpathlineto{\pgfqpoint{6.587147in}{1.182872in}}%
\pgfpathlineto{\pgfqpoint{6.690454in}{1.080994in}}%
\pgfpathlineto{\pgfqpoint{6.739560in}{1.033950in}}%
\pgfpathlineto{\pgfqpoint{6.739560in}{1.033950in}}%
\pgfusepath{stroke}%
\end{pgfscope}%
\begin{pgfscope}%
\pgfpathrectangle{\pgfqpoint{0.854460in}{0.571603in}}{\pgfqpoint{5.885100in}{5.068436in}}%
\pgfusepath{clip}%
\pgfsetbuttcap%
\pgfsetroundjoin%
\pgfsetlinewidth{1.505625pt}%
\definecolor{currentstroke}{rgb}{0.154815,0.493313,0.557840}%
\pgfsetstrokecolor{currentstroke}%
\pgfsetdash{}{0pt}%
\pgfpathmoveto{\pgfqpoint{1.475048in}{0.571603in}}%
\pgfpathlineto{\pgfqpoint{1.446453in}{0.597073in}}%
\pgfpathlineto{\pgfqpoint{1.445928in}{0.597547in}}%
\pgfpathlineto{\pgfqpoint{1.418403in}{0.622542in}}%
\pgfpathlineto{\pgfqpoint{1.416354in}{0.624430in}}%
\pgfpathlineto{\pgfqpoint{1.390884in}{0.648012in}}%
\pgfpathlineto{\pgfqpoint{1.386781in}{0.651867in}}%
\pgfpathlineto{\pgfqpoint{1.363889in}{0.673481in}}%
\pgfpathlineto{\pgfqpoint{1.357208in}{0.679883in}}%
\pgfpathlineto{\pgfqpoint{1.337409in}{0.698951in}}%
\pgfpathlineto{\pgfqpoint{1.327634in}{0.708503in}}%
\pgfpathlineto{\pgfqpoint{1.311435in}{0.724420in}}%
\pgfpathlineto{\pgfqpoint{1.298061in}{0.737754in}}%
\pgfpathlineto{\pgfqpoint{1.285957in}{0.749890in}}%
\pgfpathlineto{\pgfqpoint{1.268488in}{0.767664in}}%
\pgfpathlineto{\pgfqpoint{1.260967in}{0.775360in}}%
\pgfpathlineto{\pgfqpoint{1.238914in}{0.798260in}}%
\pgfpathlineto{\pgfqpoint{1.236455in}{0.800829in}}%
\pgfpathlineto{\pgfqpoint{1.212455in}{0.826299in}}%
\pgfpathlineto{\pgfqpoint{1.209341in}{0.829655in}}%
\pgfpathlineto{\pgfqpoint{1.188955in}{0.851768in}}%
\pgfpathlineto{\pgfqpoint{1.179767in}{0.861883in}}%
\pgfpathlineto{\pgfqpoint{1.165909in}{0.877238in}}%
\pgfpathlineto{\pgfqpoint{1.150194in}{0.894911in}}%
\pgfpathlineto{\pgfqpoint{1.143308in}{0.902707in}}%
\pgfpathlineto{\pgfqpoint{1.121147in}{0.928177in}}%
\pgfpathlineto{\pgfqpoint{1.120621in}{0.928792in}}%
\pgfpathlineto{\pgfqpoint{1.099515in}{0.953646in}}%
\pgfpathlineto{\pgfqpoint{1.091047in}{0.963769in}}%
\pgfpathlineto{\pgfqpoint{1.078301in}{0.979116in}}%
\pgfpathlineto{\pgfqpoint{1.061474in}{0.999684in}}%
\pgfpathlineto{\pgfqpoint{1.057493in}{1.004585in}}%
\pgfpathlineto{\pgfqpoint{1.037156in}{1.030055in}}%
\pgfpathlineto{\pgfqpoint{1.031901in}{1.036748in}}%
\pgfpathlineto{\pgfqpoint{1.017268in}{1.055524in}}%
\pgfpathlineto{\pgfqpoint{1.002327in}{1.074993in}}%
\pgfpathlineto{\pgfqpoint{0.997757in}{1.080994in}}%
\pgfpathlineto{\pgfqpoint{0.978700in}{1.106463in}}%
\pgfpathlineto{\pgfqpoint{0.972754in}{1.114549in}}%
\pgfpathlineto{\pgfqpoint{0.960071in}{1.131933in}}%
\pgfpathlineto{\pgfqpoint{0.943181in}{1.155450in}}%
\pgfpathlineto{\pgfqpoint{0.941790in}{1.157402in}}%
\pgfpathlineto{\pgfqpoint{0.923998in}{1.182872in}}%
\pgfpathlineto{\pgfqpoint{0.913607in}{1.197993in}}%
\pgfpathlineto{\pgfqpoint{0.906556in}{1.208341in}}%
\pgfpathlineto{\pgfqpoint{0.889513in}{1.233811in}}%
\pgfpathlineto{\pgfqpoint{0.884034in}{1.242155in}}%
\pgfpathlineto{\pgfqpoint{0.872886in}{1.259281in}}%
\pgfpathlineto{\pgfqpoint{0.856588in}{1.284750in}}%
\pgfpathlineto{\pgfqpoint{0.854460in}{1.288145in}}%
\pgfusepath{stroke}%
\end{pgfscope}%
\begin{pgfscope}%
\pgfpathrectangle{\pgfqpoint{0.854460in}{0.571603in}}{\pgfqpoint{5.885100in}{5.068436in}}%
\pgfusepath{clip}%
\pgfsetbuttcap%
\pgfsetroundjoin%
\pgfsetlinewidth{1.505625pt}%
\definecolor{currentstroke}{rgb}{0.154815,0.493313,0.557840}%
\pgfsetstrokecolor{currentstroke}%
\pgfsetdash{}{0pt}%
\pgfpathmoveto{\pgfqpoint{0.854460in}{4.076472in}}%
\pgfpathlineto{\pgfqpoint{0.893449in}{4.137337in}}%
\pgfpathlineto{\pgfqpoint{0.945599in}{4.213746in}}%
\pgfpathlineto{\pgfqpoint{1.002327in}{4.291133in}}%
\pgfpathlineto{\pgfqpoint{1.061603in}{4.366563in}}%
\pgfpathlineto{\pgfqpoint{1.125931in}{4.442971in}}%
\pgfpathlineto{\pgfqpoint{1.179767in}{4.503162in}}%
\pgfpathlineto{\pgfqpoint{1.243284in}{4.570319in}}%
\pgfpathlineto{\pgfqpoint{1.298061in}{4.625175in}}%
\pgfpathlineto{\pgfqpoint{1.357208in}{4.681536in}}%
\pgfpathlineto{\pgfqpoint{1.416354in}{4.735238in}}%
\pgfpathlineto{\pgfqpoint{1.475501in}{4.786499in}}%
\pgfpathlineto{\pgfqpoint{1.534648in}{4.835521in}}%
\pgfpathlineto{\pgfqpoint{1.593795in}{4.882491in}}%
\pgfpathlineto{\pgfqpoint{1.652941in}{4.927581in}}%
\pgfpathlineto{\pgfqpoint{1.741661in}{4.991807in}}%
\pgfpathlineto{\pgfqpoint{1.800808in}{5.032673in}}%
\pgfpathlineto{\pgfqpoint{1.889528in}{5.091116in}}%
\pgfpathlineto{\pgfqpoint{1.978248in}{5.146520in}}%
\pgfpathlineto{\pgfqpoint{2.066968in}{5.199105in}}%
\pgfpathlineto{\pgfqpoint{2.155689in}{5.249079in}}%
\pgfpathlineto{\pgfqpoint{2.244409in}{5.296637in}}%
\pgfpathlineto{\pgfqpoint{2.333129in}{5.341960in}}%
\pgfpathlineto{\pgfqpoint{2.422111in}{5.385344in}}%
\pgfpathlineto{\pgfqpoint{2.540142in}{5.439724in}}%
\pgfpathlineto{\pgfqpoint{2.658436in}{5.490978in}}%
\pgfpathlineto{\pgfqpoint{2.776729in}{5.539188in}}%
\pgfpathlineto{\pgfqpoint{2.907587in}{5.589100in}}%
\pgfpathlineto{\pgfqpoint{3.013316in}{5.626946in}}%
\pgfpathlineto{\pgfqpoint{3.051169in}{5.640039in}}%
\pgfpathlineto{\pgfqpoint{3.051169in}{5.640039in}}%
\pgfusepath{stroke}%
\end{pgfscope}%
\begin{pgfscope}%
\pgfpathrectangle{\pgfqpoint{0.854460in}{0.571603in}}{\pgfqpoint{5.885100in}{5.068436in}}%
\pgfusepath{clip}%
\pgfsetbuttcap%
\pgfsetroundjoin%
\pgfsetlinewidth{1.505625pt}%
\definecolor{currentstroke}{rgb}{0.154815,0.493313,0.557840}%
\pgfsetstrokecolor{currentstroke}%
\pgfsetdash{}{0pt}%
\pgfpathmoveto{\pgfqpoint{5.530714in}{5.640039in}}%
\pgfpathlineto{\pgfqpoint{5.545001in}{5.614570in}}%
\pgfpathlineto{\pgfqpoint{5.557469in}{5.589100in}}%
\pgfpathlineto{\pgfqpoint{5.568107in}{5.563630in}}%
\pgfpathlineto{\pgfqpoint{5.577306in}{5.538161in}}%
\pgfpathlineto{\pgfqpoint{5.586199in}{5.508947in}}%
\pgfpathlineto{\pgfqpoint{5.597305in}{5.461752in}}%
\pgfpathlineto{\pgfqpoint{5.605513in}{5.410813in}}%
\pgfpathlineto{\pgfqpoint{5.610481in}{5.359874in}}%
\pgfpathlineto{\pgfqpoint{5.612741in}{5.308935in}}%
\pgfpathlineto{\pgfqpoint{5.612738in}{5.257996in}}%
\pgfpathlineto{\pgfqpoint{5.610849in}{5.207057in}}%
\pgfpathlineto{\pgfqpoint{5.605162in}{5.130649in}}%
\pgfpathlineto{\pgfqpoint{5.596829in}{5.054240in}}%
\pgfpathlineto{\pgfqpoint{5.582769in}{4.952362in}}%
\pgfpathlineto{\pgfqpoint{5.562141in}{4.825014in}}%
\pgfpathlineto{\pgfqpoint{5.484748in}{4.366563in}}%
\pgfpathlineto{\pgfqpoint{5.467906in}{4.244394in}}%
\pgfpathlineto{\pgfqpoint{5.458170in}{4.162807in}}%
\pgfpathlineto{\pgfqpoint{5.448127in}{4.060928in}}%
\pgfpathlineto{\pgfqpoint{5.440645in}{3.959050in}}%
\pgfpathlineto{\pgfqpoint{5.435910in}{3.857172in}}%
\pgfpathlineto{\pgfqpoint{5.434295in}{3.780764in}}%
\pgfpathlineto{\pgfqpoint{5.434440in}{3.704355in}}%
\pgfpathlineto{\pgfqpoint{5.436415in}{3.627946in}}%
\pgfpathlineto{\pgfqpoint{5.440270in}{3.551538in}}%
\pgfpathlineto{\pgfqpoint{5.446050in}{3.475129in}}%
\pgfpathlineto{\pgfqpoint{5.453842in}{3.398721in}}%
\pgfpathlineto{\pgfqpoint{5.463709in}{3.322312in}}%
\pgfpathlineto{\pgfqpoint{5.475639in}{3.245904in}}%
\pgfpathlineto{\pgfqpoint{5.489712in}{3.169495in}}%
\pgfpathlineto{\pgfqpoint{5.505952in}{3.093086in}}%
\pgfpathlineto{\pgfqpoint{5.527052in}{3.006607in}}%
\pgfpathlineto{\pgfqpoint{5.545066in}{2.940269in}}%
\pgfpathlineto{\pgfqpoint{5.567999in}{2.863861in}}%
\pgfpathlineto{\pgfqpoint{5.593221in}{2.787452in}}%
\pgfpathlineto{\pgfqpoint{5.620744in}{2.711044in}}%
\pgfpathlineto{\pgfqpoint{5.650582in}{2.634635in}}%
\pgfpathlineto{\pgfqpoint{5.682752in}{2.558226in}}%
\pgfpathlineto{\pgfqpoint{5.717274in}{2.481818in}}%
\pgfpathlineto{\pgfqpoint{5.754170in}{2.405409in}}%
\pgfpathlineto{\pgfqpoint{5.793464in}{2.329001in}}%
\pgfpathlineto{\pgfqpoint{5.835080in}{2.252592in}}%
\pgfpathlineto{\pgfqpoint{5.881933in}{2.171515in}}%
\pgfpathlineto{\pgfqpoint{5.925544in}{2.099775in}}%
\pgfpathlineto{\pgfqpoint{5.974388in}{2.023366in}}%
\pgfpathlineto{\pgfqpoint{6.029799in}{1.940909in}}%
\pgfpathlineto{\pgfqpoint{6.088946in}{1.857110in}}%
\pgfpathlineto{\pgfqpoint{6.135230in}{1.794141in}}%
\pgfpathlineto{\pgfqpoint{6.207240in}{1.700440in}}%
\pgfpathlineto{\pgfqpoint{6.266386in}{1.626715in}}%
\pgfpathlineto{\pgfqpoint{6.325533in}{1.555607in}}%
\pgfpathlineto{\pgfqpoint{6.384680in}{1.486844in}}%
\pgfpathlineto{\pgfqpoint{6.451135in}{1.412098in}}%
\pgfpathlineto{\pgfqpoint{6.545379in}{1.310220in}}%
\pgfpathlineto{\pgfqpoint{6.643814in}{1.208341in}}%
\pgfpathlineto{\pgfqpoint{6.739560in}{1.113142in}}%
\pgfpathlineto{\pgfqpoint{6.739560in}{1.113142in}}%
\pgfusepath{stroke}%
\end{pgfscope}%
\begin{pgfscope}%
\pgfpathrectangle{\pgfqpoint{0.854460in}{0.571603in}}{\pgfqpoint{5.885100in}{5.068436in}}%
\pgfusepath{clip}%
\pgfsetbuttcap%
\pgfsetroundjoin%
\pgfsetlinewidth{1.505625pt}%
\definecolor{currentstroke}{rgb}{0.146180,0.515413,0.556823}%
\pgfsetstrokecolor{currentstroke}%
\pgfsetdash{}{0pt}%
\pgfpathmoveto{\pgfqpoint{1.408457in}{0.571603in}}%
\pgfpathlineto{\pgfqpoint{1.386781in}{0.591073in}}%
\pgfpathlineto{\pgfqpoint{1.380134in}{0.597073in}}%
\pgfpathlineto{\pgfqpoint{1.357208in}{0.618062in}}%
\pgfpathlineto{\pgfqpoint{1.352338in}{0.622542in}}%
\pgfpathlineto{\pgfqpoint{1.327634in}{0.645598in}}%
\pgfpathlineto{\pgfqpoint{1.325061in}{0.648012in}}%
\pgfpathlineto{\pgfqpoint{1.298298in}{0.673481in}}%
\pgfpathlineto{\pgfqpoint{1.298061in}{0.673711in}}%
\pgfpathlineto{\pgfqpoint{1.272081in}{0.698951in}}%
\pgfpathlineto{\pgfqpoint{1.268488in}{0.702493in}}%
\pgfpathlineto{\pgfqpoint{1.246364in}{0.724420in}}%
\pgfpathlineto{\pgfqpoint{1.238914in}{0.731912in}}%
\pgfpathlineto{\pgfqpoint{1.221138in}{0.749890in}}%
\pgfpathlineto{\pgfqpoint{1.209341in}{0.761996in}}%
\pgfpathlineto{\pgfqpoint{1.196395in}{0.775360in}}%
\pgfpathlineto{\pgfqpoint{1.179767in}{0.792774in}}%
\pgfpathlineto{\pgfqpoint{1.172124in}{0.800829in}}%
\pgfpathlineto{\pgfqpoint{1.150194in}{0.824277in}}%
\pgfpathlineto{\pgfqpoint{1.148315in}{0.826299in}}%
\pgfpathlineto{\pgfqpoint{1.125020in}{0.851768in}}%
\pgfpathlineto{\pgfqpoint{1.120621in}{0.856653in}}%
\pgfpathlineto{\pgfqpoint{1.102201in}{0.877238in}}%
\pgfpathlineto{\pgfqpoint{1.091047in}{0.889888in}}%
\pgfpathlineto{\pgfqpoint{1.079820in}{0.902707in}}%
\pgfpathlineto{\pgfqpoint{1.061474in}{0.923967in}}%
\pgfpathlineto{\pgfqpoint{1.057866in}{0.928177in}}%
\pgfpathlineto{\pgfqpoint{1.036393in}{0.953646in}}%
\pgfpathlineto{\pgfqpoint{1.031901in}{0.959062in}}%
\pgfpathlineto{\pgfqpoint{1.015384in}{0.979116in}}%
\pgfpathlineto{\pgfqpoint{1.002327in}{0.995209in}}%
\pgfpathlineto{\pgfqpoint{0.994776in}{1.004585in}}%
\pgfpathlineto{\pgfqpoint{0.974583in}{1.030055in}}%
\pgfpathlineto{\pgfqpoint{0.972754in}{1.032403in}}%
\pgfpathlineto{\pgfqpoint{0.954883in}{1.055524in}}%
\pgfpathlineto{\pgfqpoint{0.943181in}{1.070900in}}%
\pgfpathlineto{\pgfqpoint{0.935556in}{1.080994in}}%
\pgfpathlineto{\pgfqpoint{0.916634in}{1.106463in}}%
\pgfpathlineto{\pgfqpoint{0.913607in}{1.110613in}}%
\pgfpathlineto{\pgfqpoint{0.898178in}{1.131933in}}%
\pgfpathlineto{\pgfqpoint{0.884034in}{1.151785in}}%
\pgfpathlineto{\pgfqpoint{0.880064in}{1.157402in}}%
\pgfpathlineto{\pgfqpoint{0.862397in}{1.182872in}}%
\pgfpathlineto{\pgfqpoint{0.854460in}{1.194513in}}%
\pgfusepath{stroke}%
\end{pgfscope}%
\begin{pgfscope}%
\pgfpathrectangle{\pgfqpoint{0.854460in}{0.571603in}}{\pgfqpoint{5.885100in}{5.068436in}}%
\pgfusepath{clip}%
\pgfsetbuttcap%
\pgfsetroundjoin%
\pgfsetlinewidth{1.505625pt}%
\definecolor{currentstroke}{rgb}{0.146180,0.515413,0.556823}%
\pgfsetstrokecolor{currentstroke}%
\pgfsetdash{}{0pt}%
\pgfpathmoveto{\pgfqpoint{0.854460in}{4.188757in}}%
\pgfpathlineto{\pgfqpoint{0.871553in}{4.213746in}}%
\pgfpathlineto{\pgfqpoint{0.884034in}{4.231730in}}%
\pgfpathlineto{\pgfqpoint{0.889296in}{4.239215in}}%
\pgfpathlineto{\pgfqpoint{0.907487in}{4.264685in}}%
\pgfpathlineto{\pgfqpoint{0.913607in}{4.273122in}}%
\pgfpathlineto{\pgfqpoint{0.926120in}{4.290154in}}%
\pgfpathlineto{\pgfqpoint{0.943181in}{4.313056in}}%
\pgfpathlineto{\pgfqpoint{0.945118in}{4.315624in}}%
\pgfpathlineto{\pgfqpoint{0.964640in}{4.341093in}}%
\pgfpathlineto{\pgfqpoint{0.972754in}{4.351530in}}%
\pgfpathlineto{\pgfqpoint{0.984587in}{4.366563in}}%
\pgfpathlineto{\pgfqpoint{1.002327in}{4.388798in}}%
\pgfpathlineto{\pgfqpoint{1.004940in}{4.392032in}}%
\pgfpathlineto{\pgfqpoint{1.025823in}{4.417502in}}%
\pgfpathlineto{\pgfqpoint{1.031901in}{4.424812in}}%
\pgfpathlineto{\pgfqpoint{1.047183in}{4.442971in}}%
\pgfpathlineto{\pgfqpoint{1.061474in}{4.459731in}}%
\pgfpathlineto{\pgfqpoint{1.068990in}{4.468441in}}%
\pgfpathlineto{\pgfqpoint{1.091047in}{4.493669in}}%
\pgfpathlineto{\pgfqpoint{1.091261in}{4.493910in}}%
\pgfpathlineto{\pgfqpoint{1.114127in}{4.519380in}}%
\pgfpathlineto{\pgfqpoint{1.120621in}{4.526520in}}%
\pgfpathlineto{\pgfqpoint{1.137486in}{4.544849in}}%
\pgfpathlineto{\pgfqpoint{1.150194in}{4.558486in}}%
\pgfpathlineto{\pgfqpoint{1.161349in}{4.570319in}}%
\pgfpathlineto{\pgfqpoint{1.179767in}{4.589610in}}%
\pgfpathlineto{\pgfqpoint{1.185733in}{4.595788in}}%
\pgfpathlineto{\pgfqpoint{1.209341in}{4.619929in}}%
\pgfpathlineto{\pgfqpoint{1.210655in}{4.621258in}}%
\pgfpathlineto{\pgfqpoint{1.236180in}{4.646728in}}%
\pgfpathlineto{\pgfqpoint{1.238914in}{4.649421in}}%
\pgfpathlineto{\pgfqpoint{1.262289in}{4.672197in}}%
\pgfpathlineto{\pgfqpoint{1.268488in}{4.678162in}}%
\pgfpathlineto{\pgfqpoint{1.288976in}{4.697667in}}%
\pgfpathlineto{\pgfqpoint{1.298061in}{4.706209in}}%
\pgfpathlineto{\pgfqpoint{1.316256in}{4.723136in}}%
\pgfpathlineto{\pgfqpoint{1.327634in}{4.733591in}}%
\pgfpathlineto{\pgfqpoint{1.344146in}{4.748606in}}%
\pgfpathlineto{\pgfqpoint{1.357208in}{4.760337in}}%
\pgfpathlineto{\pgfqpoint{1.372663in}{4.774075in}}%
\pgfpathlineto{\pgfqpoint{1.386781in}{4.786472in}}%
\pgfpathlineto{\pgfqpoint{1.401822in}{4.799545in}}%
\pgfpathlineto{\pgfqpoint{1.416354in}{4.812022in}}%
\pgfpathlineto{\pgfqpoint{1.431639in}{4.825014in}}%
\pgfpathlineto{\pgfqpoint{1.445928in}{4.837012in}}%
\pgfpathlineto{\pgfqpoint{1.462130in}{4.850484in}}%
\pgfpathlineto{\pgfqpoint{1.475501in}{4.861466in}}%
\pgfpathlineto{\pgfqpoint{1.493312in}{4.875953in}}%
\pgfpathlineto{\pgfqpoint{1.505074in}{4.885405in}}%
\pgfpathlineto{\pgfqpoint{1.525198in}{4.901423in}}%
\pgfpathlineto{\pgfqpoint{1.534648in}{4.908853in}}%
\pgfpathlineto{\pgfqpoint{1.557806in}{4.926892in}}%
\pgfpathlineto{\pgfqpoint{1.564221in}{4.931830in}}%
\pgfpathlineto{\pgfqpoint{1.591148in}{4.952362in}}%
\pgfpathlineto{\pgfqpoint{1.593795in}{4.954355in}}%
\pgfpathlineto{\pgfqpoint{1.623368in}{4.976422in}}%
\pgfpathlineto{\pgfqpoint{1.625276in}{4.977831in}}%
\pgfpathlineto{\pgfqpoint{1.652941in}{4.998027in}}%
\pgfpathlineto{\pgfqpoint{1.660230in}{5.003301in}}%
\pgfpathlineto{\pgfqpoint{1.682515in}{5.019230in}}%
\pgfpathlineto{\pgfqpoint{1.695978in}{5.028770in}}%
\pgfpathlineto{\pgfqpoint{1.712088in}{5.040049in}}%
\pgfpathlineto{\pgfqpoint{1.732531in}{5.054240in}}%
\pgfpathlineto{\pgfqpoint{1.741661in}{5.060501in}}%
\pgfpathlineto{\pgfqpoint{1.769903in}{5.079709in}}%
\pgfpathlineto{\pgfqpoint{1.771235in}{5.080604in}}%
\pgfpathlineto{\pgfqpoint{1.800808in}{5.100274in}}%
\pgfpathlineto{\pgfqpoint{1.808245in}{5.105179in}}%
\pgfpathlineto{\pgfqpoint{1.830381in}{5.119602in}}%
\pgfpathlineto{\pgfqpoint{1.847472in}{5.130649in}}%
\pgfpathlineto{\pgfqpoint{1.859955in}{5.138620in}}%
\pgfpathlineto{\pgfqpoint{1.887569in}{5.156118in}}%
\pgfpathlineto{\pgfqpoint{1.889528in}{5.157345in}}%
\pgfpathlineto{\pgfqpoint{1.919102in}{5.175670in}}%
\pgfpathlineto{\pgfqpoint{1.928728in}{5.181588in}}%
\pgfpathlineto{\pgfqpoint{1.948675in}{5.193700in}}%
\pgfpathlineto{\pgfqpoint{1.970834in}{5.207057in}}%
\pgfpathlineto{\pgfqpoint{1.978248in}{5.211472in}}%
\pgfpathlineto{\pgfqpoint{2.007822in}{5.228928in}}%
\pgfpathlineto{\pgfqpoint{2.013974in}{5.232527in}}%
\pgfpathlineto{\pgfqpoint{2.037395in}{5.246057in}}%
\pgfpathlineto{\pgfqpoint{2.058203in}{5.257996in}}%
\pgfpathlineto{\pgfqpoint{2.066968in}{5.262964in}}%
\pgfpathlineto{\pgfqpoint{2.096542in}{5.279582in}}%
\pgfpathlineto{\pgfqpoint{2.103518in}{5.283466in}}%
\pgfpathlineto{\pgfqpoint{2.126115in}{5.295893in}}%
\pgfpathlineto{\pgfqpoint{2.149982in}{5.308935in}}%
\pgfpathlineto{\pgfqpoint{2.155689in}{5.312015in}}%
\pgfpathlineto{\pgfqpoint{2.185262in}{5.327823in}}%
\pgfpathlineto{\pgfqpoint{2.197671in}{5.334405in}}%
\pgfpathlineto{\pgfqpoint{2.214835in}{5.343396in}}%
\pgfpathlineto{\pgfqpoint{2.244409in}{5.358787in}}%
\pgfpathlineto{\pgfqpoint{2.246521in}{5.359874in}}%
\pgfpathlineto{\pgfqpoint{2.273982in}{5.373835in}}%
\pgfpathlineto{\pgfqpoint{2.296747in}{5.385344in}}%
\pgfpathlineto{\pgfqpoint{2.303555in}{5.388743in}}%
\pgfpathlineto{\pgfqpoint{2.333129in}{5.403364in}}%
\pgfpathlineto{\pgfqpoint{2.348308in}{5.410813in}}%
\pgfpathlineto{\pgfqpoint{2.362702in}{5.417789in}}%
\pgfpathlineto{\pgfqpoint{2.392275in}{5.432011in}}%
\pgfpathlineto{\pgfqpoint{2.401242in}{5.436283in}}%
\pgfpathlineto{\pgfqpoint{2.421849in}{5.445977in}}%
\pgfpathlineto{\pgfqpoint{2.451422in}{5.459805in}}%
\pgfpathlineto{\pgfqpoint{2.455632in}{5.461752in}}%
\pgfpathlineto{\pgfqpoint{2.480996in}{5.473335in}}%
\pgfpathlineto{\pgfqpoint{2.510569in}{5.486774in}}%
\pgfpathlineto{\pgfqpoint{2.511566in}{5.487222in}}%
\pgfpathlineto{\pgfqpoint{2.540142in}{5.499889in}}%
\pgfpathlineto{\pgfqpoint{2.569147in}{5.512691in}}%
\pgfpathlineto{\pgfqpoint{2.569716in}{5.512939in}}%
\pgfpathlineto{\pgfqpoint{2.599289in}{5.525666in}}%
\pgfpathlineto{\pgfqpoint{2.628444in}{5.538161in}}%
\pgfpathlineto{\pgfqpoint{2.628862in}{5.538338in}}%
\pgfpathlineto{\pgfqpoint{2.658436in}{5.550691in}}%
\pgfpathlineto{\pgfqpoint{2.688009in}{5.562991in}}%
\pgfpathlineto{\pgfqpoint{2.689565in}{5.563630in}}%
\pgfpathlineto{\pgfqpoint{2.717582in}{5.574988in}}%
\pgfpathlineto{\pgfqpoint{2.747156in}{5.586917in}}%
\pgfpathlineto{\pgfqpoint{2.752632in}{5.589100in}}%
\pgfpathlineto{\pgfqpoint{2.776729in}{5.598581in}}%
\pgfpathlineto{\pgfqpoint{2.806303in}{5.610143in}}%
\pgfpathlineto{\pgfqpoint{2.817744in}{5.614570in}}%
\pgfpathlineto{\pgfqpoint{2.835876in}{5.621494in}}%
\pgfpathlineto{\pgfqpoint{2.865449in}{5.632692in}}%
\pgfpathlineto{\pgfqpoint{2.885014in}{5.640039in}}%
\pgfusepath{stroke}%
\end{pgfscope}%
\begin{pgfscope}%
\pgfpathrectangle{\pgfqpoint{0.854460in}{0.571603in}}{\pgfqpoint{5.885100in}{5.068436in}}%
\pgfusepath{clip}%
\pgfsetbuttcap%
\pgfsetroundjoin%
\pgfsetlinewidth{1.505625pt}%
\definecolor{currentstroke}{rgb}{0.146180,0.515413,0.556823}%
\pgfsetstrokecolor{currentstroke}%
\pgfsetdash{}{0pt}%
\pgfpathmoveto{\pgfqpoint{5.712248in}{5.640039in}}%
\pgfpathlineto{\pgfqpoint{5.720805in}{5.614570in}}%
\pgfpathlineto{\pgfqpoint{5.728054in}{5.589100in}}%
\pgfpathlineto{\pgfqpoint{5.734110in}{5.563630in}}%
\pgfpathlineto{\pgfqpoint{5.742908in}{5.512691in}}%
\pgfpathlineto{\pgfqpoint{5.748150in}{5.461752in}}%
\pgfpathlineto{\pgfqpoint{5.750421in}{5.410813in}}%
\pgfpathlineto{\pgfqpoint{5.750206in}{5.359874in}}%
\pgfpathlineto{\pgfqpoint{5.747915in}{5.308935in}}%
\pgfpathlineto{\pgfqpoint{5.743893in}{5.257996in}}%
\pgfpathlineto{\pgfqpoint{5.735248in}{5.181588in}}%
\pgfpathlineto{\pgfqpoint{5.724056in}{5.105179in}}%
\pgfpathlineto{\pgfqpoint{5.706472in}{5.003301in}}%
\pgfpathlineto{\pgfqpoint{5.681692in}{4.875953in}}%
\pgfpathlineto{\pgfqpoint{5.595356in}{4.442971in}}%
\pgfpathlineto{\pgfqpoint{5.573702in}{4.315624in}}%
\pgfpathlineto{\pgfqpoint{5.558628in}{4.213746in}}%
\pgfpathlineto{\pgfqpoint{5.545798in}{4.111867in}}%
\pgfpathlineto{\pgfqpoint{5.535562in}{4.009989in}}%
\pgfpathlineto{\pgfqpoint{5.528140in}{3.908111in}}%
\pgfpathlineto{\pgfqpoint{5.524504in}{3.831703in}}%
\pgfpathlineto{\pgfqpoint{5.522630in}{3.755294in}}%
\pgfpathlineto{\pgfqpoint{5.522596in}{3.678886in}}%
\pgfpathlineto{\pgfqpoint{5.524466in}{3.602477in}}%
\pgfpathlineto{\pgfqpoint{5.528291in}{3.526068in}}%
\pgfpathlineto{\pgfqpoint{5.534097in}{3.449660in}}%
\pgfpathlineto{\pgfqpoint{5.541972in}{3.373251in}}%
\pgfpathlineto{\pgfqpoint{5.551975in}{3.296843in}}%
\pgfpathlineto{\pgfqpoint{5.564095in}{3.220434in}}%
\pgfpathlineto{\pgfqpoint{5.578401in}{3.144025in}}%
\pgfpathlineto{\pgfqpoint{5.594919in}{3.067617in}}%
\pgfpathlineto{\pgfqpoint{5.615772in}{2.983436in}}%
\pgfpathlineto{\pgfqpoint{5.634710in}{2.914800in}}%
\pgfpathlineto{\pgfqpoint{5.658033in}{2.838391in}}%
\pgfpathlineto{\pgfqpoint{5.683680in}{2.761983in}}%
\pgfpathlineto{\pgfqpoint{5.711661in}{2.685574in}}%
\pgfpathlineto{\pgfqpoint{5.741988in}{2.609165in}}%
\pgfpathlineto{\pgfqpoint{5.774680in}{2.532757in}}%
\pgfpathlineto{\pgfqpoint{5.809755in}{2.456348in}}%
\pgfpathlineto{\pgfqpoint{5.852359in}{2.369950in}}%
\pgfpathlineto{\pgfqpoint{5.887104in}{2.303531in}}%
\pgfpathlineto{\pgfqpoint{5.929366in}{2.227123in}}%
\pgfpathlineto{\pgfqpoint{5.974072in}{2.150714in}}%
\pgfpathlineto{\pgfqpoint{6.021164in}{2.074305in}}%
\pgfpathlineto{\pgfqpoint{6.070685in}{1.997897in}}%
\pgfpathlineto{\pgfqpoint{6.122646in}{1.921488in}}%
\pgfpathlineto{\pgfqpoint{6.177666in}{1.844207in}}%
\pgfpathlineto{\pgfqpoint{6.236813in}{1.764739in}}%
\pgfpathlineto{\pgfqpoint{6.295960in}{1.688557in}}%
\pgfpathlineto{\pgfqpoint{6.355107in}{1.615282in}}%
\pgfpathlineto{\pgfqpoint{6.418642in}{1.539445in}}%
\pgfpathlineto{\pgfqpoint{6.502973in}{1.442926in}}%
\pgfpathlineto{\pgfqpoint{6.562120in}{1.377689in}}%
\pgfpathlineto{\pgfqpoint{6.625093in}{1.310220in}}%
\pgfpathlineto{\pgfqpoint{6.723709in}{1.208341in}}%
\pgfpathlineto{\pgfqpoint{6.739560in}{1.192399in}}%
\pgfpathlineto{\pgfqpoint{6.739560in}{1.192399in}}%
\pgfusepath{stroke}%
\end{pgfscope}%
\begin{pgfscope}%
\pgfpathrectangle{\pgfqpoint{0.854460in}{0.571603in}}{\pgfqpoint{5.885100in}{5.068436in}}%
\pgfusepath{clip}%
\pgfsetbuttcap%
\pgfsetroundjoin%
\pgfsetlinewidth{1.505625pt}%
\definecolor{currentstroke}{rgb}{0.139147,0.533812,0.555298}%
\pgfsetstrokecolor{currentstroke}%
\pgfsetdash{}{0pt}%
\pgfpathmoveto{\pgfqpoint{1.343822in}{0.571603in}}%
\pgfpathlineto{\pgfqpoint{1.327634in}{0.586270in}}%
\pgfpathlineto{\pgfqpoint{1.315769in}{0.597073in}}%
\pgfpathlineto{\pgfqpoint{1.298061in}{0.613425in}}%
\pgfpathlineto{\pgfqpoint{1.288238in}{0.622542in}}%
\pgfpathlineto{\pgfqpoint{1.268488in}{0.641135in}}%
\pgfpathlineto{\pgfqpoint{1.261221in}{0.648012in}}%
\pgfpathlineto{\pgfqpoint{1.238914in}{0.669423in}}%
\pgfpathlineto{\pgfqpoint{1.234709in}{0.673481in}}%
\pgfpathlineto{\pgfqpoint{1.209341in}{0.698315in}}%
\pgfpathlineto{\pgfqpoint{1.208695in}{0.698951in}}%
\pgfpathlineto{\pgfqpoint{1.183215in}{0.724420in}}%
\pgfpathlineto{\pgfqpoint{1.179767in}{0.727917in}}%
\pgfpathlineto{\pgfqpoint{1.158230in}{0.749890in}}%
\pgfpathlineto{\pgfqpoint{1.150194in}{0.758208in}}%
\pgfpathlineto{\pgfqpoint{1.133721in}{0.775360in}}%
\pgfpathlineto{\pgfqpoint{1.120621in}{0.789200in}}%
\pgfpathlineto{\pgfqpoint{1.109680in}{0.800829in}}%
\pgfpathlineto{\pgfqpoint{1.091047in}{0.820923in}}%
\pgfpathlineto{\pgfqpoint{1.086094in}{0.826299in}}%
\pgfpathlineto{\pgfqpoint{1.062977in}{0.851768in}}%
\pgfpathlineto{\pgfqpoint{1.061474in}{0.853451in}}%
\pgfpathlineto{\pgfqpoint{1.040371in}{0.877238in}}%
\pgfpathlineto{\pgfqpoint{1.031901in}{0.886927in}}%
\pgfpathlineto{\pgfqpoint{1.018198in}{0.902707in}}%
\pgfpathlineto{\pgfqpoint{1.002327in}{0.921257in}}%
\pgfpathlineto{\pgfqpoint{0.996447in}{0.928177in}}%
\pgfpathlineto{\pgfqpoint{0.975140in}{0.953646in}}%
\pgfpathlineto{\pgfqpoint{0.972754in}{0.956547in}}%
\pgfpathlineto{\pgfqpoint{0.954323in}{0.979116in}}%
\pgfpathlineto{\pgfqpoint{0.943181in}{0.992965in}}%
\pgfpathlineto{\pgfqpoint{0.933900in}{1.004585in}}%
\pgfpathlineto{\pgfqpoint{0.913865in}{1.030055in}}%
\pgfpathlineto{\pgfqpoint{0.913607in}{1.030389in}}%
\pgfpathlineto{\pgfqpoint{0.894341in}{1.055524in}}%
\pgfpathlineto{\pgfqpoint{0.884034in}{1.069179in}}%
\pgfpathlineto{\pgfqpoint{0.875184in}{1.080994in}}%
\pgfpathlineto{\pgfqpoint{0.856410in}{1.106463in}}%
\pgfpathlineto{\pgfqpoint{0.854460in}{1.109158in}}%
\pgfusepath{stroke}%
\end{pgfscope}%
\begin{pgfscope}%
\pgfpathrectangle{\pgfqpoint{0.854460in}{0.571603in}}{\pgfqpoint{5.885100in}{5.068436in}}%
\pgfusepath{clip}%
\pgfsetbuttcap%
\pgfsetroundjoin%
\pgfsetlinewidth{1.505625pt}%
\definecolor{currentstroke}{rgb}{0.139147,0.533812,0.555298}%
\pgfsetstrokecolor{currentstroke}%
\pgfsetdash{}{0pt}%
\pgfpathmoveto{\pgfqpoint{0.854460in}{4.291107in}}%
\pgfpathlineto{\pgfqpoint{0.872412in}{4.315624in}}%
\pgfpathlineto{\pgfqpoint{0.884034in}{4.331276in}}%
\pgfpathlineto{\pgfqpoint{0.891414in}{4.341093in}}%
\pgfpathlineto{\pgfqpoint{0.910840in}{4.366563in}}%
\pgfpathlineto{\pgfqpoint{0.913607in}{4.370136in}}%
\pgfpathlineto{\pgfqpoint{0.930776in}{4.392032in}}%
\pgfpathlineto{\pgfqpoint{0.943181in}{4.407641in}}%
\pgfpathlineto{\pgfqpoint{0.951112in}{4.417502in}}%
\pgfpathlineto{\pgfqpoint{0.971880in}{4.442971in}}%
\pgfpathlineto{\pgfqpoint{0.972754in}{4.444027in}}%
\pgfpathlineto{\pgfqpoint{0.993207in}{4.468441in}}%
\pgfpathlineto{\pgfqpoint{1.002327in}{4.479184in}}%
\pgfpathlineto{\pgfqpoint{1.014975in}{4.493910in}}%
\pgfpathlineto{\pgfqpoint{1.031901in}{4.513362in}}%
\pgfpathlineto{\pgfqpoint{1.037199in}{4.519380in}}%
\pgfpathlineto{\pgfqpoint{1.059921in}{4.544849in}}%
\pgfpathlineto{\pgfqpoint{1.061474in}{4.546566in}}%
\pgfpathlineto{\pgfqpoint{1.083214in}{4.570319in}}%
\pgfpathlineto{\pgfqpoint{1.091047in}{4.578769in}}%
\pgfpathlineto{\pgfqpoint{1.107002in}{4.595788in}}%
\pgfpathlineto{\pgfqpoint{1.120621in}{4.610131in}}%
\pgfpathlineto{\pgfqpoint{1.131303in}{4.621258in}}%
\pgfpathlineto{\pgfqpoint{1.150194in}{4.640688in}}%
\pgfpathlineto{\pgfqpoint{1.156130in}{4.646728in}}%
\pgfpathlineto{\pgfqpoint{1.179767in}{4.670474in}}%
\pgfpathlineto{\pgfqpoint{1.181502in}{4.672197in}}%
\pgfpathlineto{\pgfqpoint{1.207465in}{4.697667in}}%
\pgfpathlineto{\pgfqpoint{1.209341in}{4.699483in}}%
\pgfpathlineto{\pgfqpoint{1.234025in}{4.723136in}}%
\pgfpathlineto{\pgfqpoint{1.238914in}{4.727763in}}%
\pgfpathlineto{\pgfqpoint{1.261166in}{4.748606in}}%
\pgfpathlineto{\pgfqpoint{1.268488in}{4.755378in}}%
\pgfpathlineto{\pgfqpoint{1.288905in}{4.774075in}}%
\pgfpathlineto{\pgfqpoint{1.298061in}{4.782356in}}%
\pgfpathlineto{\pgfqpoint{1.317257in}{4.799545in}}%
\pgfpathlineto{\pgfqpoint{1.327634in}{4.808723in}}%
\pgfpathlineto{\pgfqpoint{1.346238in}{4.825014in}}%
\pgfpathlineto{\pgfqpoint{1.357208in}{4.834504in}}%
\pgfpathlineto{\pgfqpoint{1.375862in}{4.850484in}}%
\pgfpathlineto{\pgfqpoint{1.386781in}{4.859723in}}%
\pgfpathlineto{\pgfqpoint{1.406146in}{4.875953in}}%
\pgfpathlineto{\pgfqpoint{1.416354in}{4.884405in}}%
\pgfpathlineto{\pgfqpoint{1.437104in}{4.901423in}}%
\pgfpathlineto{\pgfqpoint{1.445928in}{4.908572in}}%
\pgfpathlineto{\pgfqpoint{1.468751in}{4.926892in}}%
\pgfpathlineto{\pgfqpoint{1.475501in}{4.932245in}}%
\pgfpathlineto{\pgfqpoint{1.501101in}{4.952362in}}%
\pgfpathlineto{\pgfqpoint{1.505074in}{4.955446in}}%
\pgfpathlineto{\pgfqpoint{1.534170in}{4.977831in}}%
\pgfpathlineto{\pgfqpoint{1.534648in}{4.978195in}}%
\pgfpathlineto{\pgfqpoint{1.564221in}{5.000454in}}%
\pgfpathlineto{\pgfqpoint{1.568038in}{5.003301in}}%
\pgfpathlineto{\pgfqpoint{1.593795in}{5.022283in}}%
\pgfpathlineto{\pgfqpoint{1.602673in}{5.028770in}}%
\pgfpathlineto{\pgfqpoint{1.623368in}{5.043709in}}%
\pgfpathlineto{\pgfqpoint{1.638081in}{5.054240in}}%
\pgfpathlineto{\pgfqpoint{1.652941in}{5.064748in}}%
\pgfpathlineto{\pgfqpoint{1.674274in}{5.079709in}}%
\pgfpathlineto{\pgfqpoint{1.682515in}{5.085419in}}%
\pgfpathlineto{\pgfqpoint{1.711265in}{5.105179in}}%
\pgfpathlineto{\pgfqpoint{1.712088in}{5.105738in}}%
\pgfpathlineto{\pgfqpoint{1.741661in}{5.125620in}}%
\pgfpathlineto{\pgfqpoint{1.749201in}{5.130649in}}%
\pgfpathlineto{\pgfqpoint{1.771235in}{5.145165in}}%
\pgfpathlineto{\pgfqpoint{1.787988in}{5.156118in}}%
\pgfpathlineto{\pgfqpoint{1.800808in}{5.164398in}}%
\pgfpathlineto{\pgfqpoint{1.827622in}{5.181588in}}%
\pgfpathlineto{\pgfqpoint{1.830381in}{5.183335in}}%
\pgfpathlineto{\pgfqpoint{1.859955in}{5.201887in}}%
\pgfpathlineto{\pgfqpoint{1.868264in}{5.207057in}}%
\pgfpathlineto{\pgfqpoint{1.889528in}{5.220129in}}%
\pgfpathlineto{\pgfqpoint{1.909840in}{5.232527in}}%
\pgfpathlineto{\pgfqpoint{1.919102in}{5.238111in}}%
\pgfpathlineto{\pgfqpoint{1.948675in}{5.255804in}}%
\pgfpathlineto{\pgfqpoint{1.952375in}{5.257996in}}%
\pgfpathlineto{\pgfqpoint{1.978248in}{5.273143in}}%
\pgfpathlineto{\pgfqpoint{1.996000in}{5.283466in}}%
\pgfpathlineto{\pgfqpoint{2.007822in}{5.290257in}}%
\pgfpathlineto{\pgfqpoint{2.037395in}{5.307121in}}%
\pgfpathlineto{\pgfqpoint{2.040607in}{5.308935in}}%
\pgfpathlineto{\pgfqpoint{2.066968in}{5.323640in}}%
\pgfpathlineto{\pgfqpoint{2.086386in}{5.334405in}}%
\pgfpathlineto{\pgfqpoint{2.096542in}{5.339966in}}%
\pgfpathlineto{\pgfqpoint{2.126115in}{5.356032in}}%
\pgfpathlineto{\pgfqpoint{2.133251in}{5.359874in}}%
\pgfpathlineto{\pgfqpoint{2.155689in}{5.371808in}}%
\pgfpathlineto{\pgfqpoint{2.181285in}{5.385344in}}%
\pgfpathlineto{\pgfqpoint{2.185262in}{5.387421in}}%
\pgfpathlineto{\pgfqpoint{2.214835in}{5.402716in}}%
\pgfpathlineto{\pgfqpoint{2.230596in}{5.410813in}}%
\pgfpathlineto{\pgfqpoint{2.244409in}{5.417822in}}%
\pgfpathlineto{\pgfqpoint{2.273982in}{5.432721in}}%
\pgfpathlineto{\pgfqpoint{2.281118in}{5.436283in}}%
\pgfpathlineto{\pgfqpoint{2.303555in}{5.447344in}}%
\pgfpathlineto{\pgfqpoint{2.332928in}{5.461752in}}%
\pgfpathlineto{\pgfqpoint{2.333129in}{5.461850in}}%
\pgfpathlineto{\pgfqpoint{2.362702in}{5.476015in}}%
\pgfpathlineto{\pgfqpoint{2.386210in}{5.487222in}}%
\pgfpathlineto{\pgfqpoint{2.392275in}{5.490077in}}%
\pgfpathlineto{\pgfqpoint{2.421849in}{5.503863in}}%
\pgfpathlineto{\pgfqpoint{2.440904in}{5.512691in}}%
\pgfpathlineto{\pgfqpoint{2.451422in}{5.517503in}}%
\pgfpathlineto{\pgfqpoint{2.480996in}{5.530915in}}%
\pgfpathlineto{\pgfqpoint{2.497088in}{5.538161in}}%
\pgfpathlineto{\pgfqpoint{2.510569in}{5.544155in}}%
\pgfpathlineto{\pgfqpoint{2.540142in}{5.557197in}}%
\pgfpathlineto{\pgfqpoint{2.554842in}{5.563630in}}%
\pgfpathlineto{\pgfqpoint{2.569716in}{5.570057in}}%
\pgfpathlineto{\pgfqpoint{2.599289in}{5.582735in}}%
\pgfpathlineto{\pgfqpoint{2.614253in}{5.589100in}}%
\pgfpathlineto{\pgfqpoint{2.628862in}{5.595235in}}%
\pgfpathlineto{\pgfqpoint{2.658436in}{5.607553in}}%
\pgfpathlineto{\pgfqpoint{2.675409in}{5.614570in}}%
\pgfpathlineto{\pgfqpoint{2.688009in}{5.619712in}}%
\pgfpathlineto{\pgfqpoint{2.717582in}{5.631674in}}%
\pgfpathlineto{\pgfqpoint{2.738400in}{5.640039in}}%
\pgfusepath{stroke}%
\end{pgfscope}%
\begin{pgfscope}%
\pgfpathrectangle{\pgfqpoint{0.854460in}{0.571603in}}{\pgfqpoint{5.885100in}{5.068436in}}%
\pgfusepath{clip}%
\pgfsetbuttcap%
\pgfsetroundjoin%
\pgfsetlinewidth{1.505625pt}%
\definecolor{currentstroke}{rgb}{0.139147,0.533812,0.555298}%
\pgfsetstrokecolor{currentstroke}%
\pgfsetdash{}{0pt}%
\pgfpathmoveto{\pgfqpoint{5.874289in}{5.640039in}}%
\pgfpathlineto{\pgfqpoint{5.882237in}{5.589100in}}%
\pgfpathlineto{\pgfqpoint{5.886455in}{5.538161in}}%
\pgfpathlineto{\pgfqpoint{5.887663in}{5.487222in}}%
\pgfpathlineto{\pgfqpoint{5.886352in}{5.436283in}}%
\pgfpathlineto{\pgfqpoint{5.881933in}{5.374756in}}%
\pgfpathlineto{\pgfqpoint{5.877700in}{5.334405in}}%
\pgfpathlineto{\pgfqpoint{5.867141in}{5.257996in}}%
\pgfpathlineto{\pgfqpoint{5.852359in}{5.172379in}}%
\pgfpathlineto{\pgfqpoint{5.833778in}{5.079709in}}%
\pgfpathlineto{\pgfqpoint{5.805527in}{4.952362in}}%
\pgfpathlineto{\pgfqpoint{5.701027in}{4.493910in}}%
\pgfpathlineto{\pgfqpoint{5.680705in}{4.392032in}}%
\pgfpathlineto{\pgfqpoint{5.662326in}{4.290154in}}%
\pgfpathlineto{\pgfqpoint{5.645346in}{4.181980in}}%
\pgfpathlineto{\pgfqpoint{5.632620in}{4.086398in}}%
\pgfpathlineto{\pgfqpoint{5.621796in}{3.984520in}}%
\pgfpathlineto{\pgfqpoint{5.615626in}{3.908111in}}%
\pgfpathlineto{\pgfqpoint{5.611160in}{3.831703in}}%
\pgfpathlineto{\pgfqpoint{5.608523in}{3.755294in}}%
\pgfpathlineto{\pgfqpoint{5.607775in}{3.678886in}}%
\pgfpathlineto{\pgfqpoint{5.608976in}{3.602477in}}%
\pgfpathlineto{\pgfqpoint{5.612181in}{3.526068in}}%
\pgfpathlineto{\pgfqpoint{5.617433in}{3.449660in}}%
\pgfpathlineto{\pgfqpoint{5.624751in}{3.373251in}}%
\pgfpathlineto{\pgfqpoint{5.634224in}{3.296843in}}%
\pgfpathlineto{\pgfqpoint{5.645905in}{3.220434in}}%
\pgfpathlineto{\pgfqpoint{5.659734in}{3.144025in}}%
\pgfpathlineto{\pgfqpoint{5.675867in}{3.067617in}}%
\pgfpathlineto{\pgfqpoint{5.694214in}{2.991208in}}%
\pgfpathlineto{\pgfqpoint{5.714884in}{2.914800in}}%
\pgfpathlineto{\pgfqpoint{5.737890in}{2.838391in}}%
\pgfpathlineto{\pgfqpoint{5.763639in}{2.760836in}}%
\pgfpathlineto{\pgfqpoint{5.793213in}{2.679602in}}%
\pgfpathlineto{\pgfqpoint{5.822786in}{2.604832in}}%
\pgfpathlineto{\pgfqpoint{5.853442in}{2.532757in}}%
\pgfpathlineto{\pgfqpoint{5.888283in}{2.456348in}}%
\pgfpathlineto{\pgfqpoint{5.925544in}{2.379940in}}%
\pgfpathlineto{\pgfqpoint{5.970653in}{2.293566in}}%
\pgfpathlineto{\pgfqpoint{6.007368in}{2.227123in}}%
\pgfpathlineto{\pgfqpoint{6.059373in}{2.138418in}}%
\pgfpathlineto{\pgfqpoint{6.098915in}{2.074305in}}%
\pgfpathlineto{\pgfqpoint{6.148378in}{1.997897in}}%
\pgfpathlineto{\pgfqpoint{6.207240in}{1.911506in}}%
\pgfpathlineto{\pgfqpoint{6.254551in}{1.845080in}}%
\pgfpathlineto{\pgfqpoint{6.325533in}{1.750091in}}%
\pgfpathlineto{\pgfqpoint{6.370517in}{1.692262in}}%
\pgfpathlineto{\pgfqpoint{6.443827in}{1.601775in}}%
\pgfpathlineto{\pgfqpoint{6.502973in}{1.531614in}}%
\pgfpathlineto{\pgfqpoint{6.562749in}{1.463037in}}%
\pgfpathlineto{\pgfqpoint{6.631636in}{1.386628in}}%
\pgfpathlineto{\pgfqpoint{6.709987in}{1.302865in}}%
\pgfpathlineto{\pgfqpoint{6.739560in}{1.272014in}}%
\pgfpathlineto{\pgfqpoint{6.739560in}{1.272014in}}%
\pgfusepath{stroke}%
\end{pgfscope}%
\begin{pgfscope}%
\pgfpathrectangle{\pgfqpoint{0.854460in}{0.571603in}}{\pgfqpoint{5.885100in}{5.068436in}}%
\pgfusepath{clip}%
\pgfsetbuttcap%
\pgfsetroundjoin%
\pgfsetlinewidth{1.505625pt}%
\definecolor{currentstroke}{rgb}{0.131172,0.555899,0.552459}%
\pgfsetstrokecolor{currentstroke}%
\pgfsetdash{}{0pt}%
\pgfpathmoveto{\pgfqpoint{1.281021in}{0.571603in}}%
\pgfpathlineto{\pgfqpoint{1.268488in}{0.583058in}}%
\pgfpathlineto{\pgfqpoint{1.253228in}{0.597073in}}%
\pgfpathlineto{\pgfqpoint{1.238914in}{0.610407in}}%
\pgfpathlineto{\pgfqpoint{1.225952in}{0.622542in}}%
\pgfpathlineto{\pgfqpoint{1.209341in}{0.638317in}}%
\pgfpathlineto{\pgfqpoint{1.199185in}{0.648012in}}%
\pgfpathlineto{\pgfqpoint{1.179767in}{0.666813in}}%
\pgfpathlineto{\pgfqpoint{1.172918in}{0.673481in}}%
\pgfpathlineto{\pgfqpoint{1.150194in}{0.695921in}}%
\pgfpathlineto{\pgfqpoint{1.147142in}{0.698951in}}%
\pgfpathlineto{\pgfqpoint{1.121866in}{0.724420in}}%
\pgfpathlineto{\pgfqpoint{1.120621in}{0.725694in}}%
\pgfpathlineto{\pgfqpoint{1.097111in}{0.749890in}}%
\pgfpathlineto{\pgfqpoint{1.091047in}{0.756220in}}%
\pgfpathlineto{\pgfqpoint{1.072826in}{0.775360in}}%
\pgfpathlineto{\pgfqpoint{1.061474in}{0.787456in}}%
\pgfpathlineto{\pgfqpoint{1.049002in}{0.800829in}}%
\pgfpathlineto{\pgfqpoint{1.031901in}{0.819433in}}%
\pgfpathlineto{\pgfqpoint{1.025629in}{0.826299in}}%
\pgfpathlineto{\pgfqpoint{1.002702in}{0.851768in}}%
\pgfpathlineto{\pgfqpoint{1.002327in}{0.852191in}}%
\pgfpathlineto{\pgfqpoint{0.980299in}{0.877238in}}%
\pgfpathlineto{\pgfqpoint{0.972754in}{0.885943in}}%
\pgfpathlineto{\pgfqpoint{0.958322in}{0.902707in}}%
\pgfpathlineto{\pgfqpoint{0.943181in}{0.920556in}}%
\pgfpathlineto{\pgfqpoint{0.936761in}{0.928177in}}%
\pgfpathlineto{\pgfqpoint{0.915633in}{0.953646in}}%
\pgfpathlineto{\pgfqpoint{0.913607in}{0.956130in}}%
\pgfpathlineto{\pgfqpoint{0.894995in}{0.979116in}}%
\pgfpathlineto{\pgfqpoint{0.884034in}{0.992856in}}%
\pgfpathlineto{\pgfqpoint{0.874746in}{1.004585in}}%
\pgfpathlineto{\pgfqpoint{0.854881in}{1.030055in}}%
\pgfpathlineto{\pgfqpoint{0.854460in}{1.030604in}}%
\pgfusepath{stroke}%
\end{pgfscope}%
\begin{pgfscope}%
\pgfpathrectangle{\pgfqpoint{0.854460in}{0.571603in}}{\pgfqpoint{5.885100in}{5.068436in}}%
\pgfusepath{clip}%
\pgfsetbuttcap%
\pgfsetroundjoin%
\pgfsetlinewidth{1.505625pt}%
\definecolor{currentstroke}{rgb}{0.131172,0.555899,0.552459}%
\pgfsetstrokecolor{currentstroke}%
\pgfsetdash{}{0pt}%
\pgfpathmoveto{\pgfqpoint{0.854460in}{4.385335in}}%
\pgfpathlineto{\pgfqpoint{0.859611in}{4.392032in}}%
\pgfpathlineto{\pgfqpoint{0.879488in}{4.417502in}}%
\pgfpathlineto{\pgfqpoint{0.884034in}{4.423242in}}%
\pgfpathlineto{\pgfqpoint{0.899846in}{4.442971in}}%
\pgfpathlineto{\pgfqpoint{0.913607in}{4.459914in}}%
\pgfpathlineto{\pgfqpoint{0.920614in}{4.468441in}}%
\pgfpathlineto{\pgfqpoint{0.941830in}{4.493910in}}%
\pgfpathlineto{\pgfqpoint{0.943181in}{4.495507in}}%
\pgfpathlineto{\pgfqpoint{0.963598in}{4.519380in}}%
\pgfpathlineto{\pgfqpoint{0.972754in}{4.529947in}}%
\pgfpathlineto{\pgfqpoint{0.985814in}{4.544849in}}%
\pgfpathlineto{\pgfqpoint{1.002327in}{4.563449in}}%
\pgfpathlineto{\pgfqpoint{1.008495in}{4.570319in}}%
\pgfpathlineto{\pgfqpoint{1.031660in}{4.595788in}}%
\pgfpathlineto{\pgfqpoint{1.031901in}{4.596049in}}%
\pgfpathlineto{\pgfqpoint{1.055416in}{4.621258in}}%
\pgfpathlineto{\pgfqpoint{1.061474in}{4.627670in}}%
\pgfpathlineto{\pgfqpoint{1.079675in}{4.646728in}}%
\pgfpathlineto{\pgfqpoint{1.091047in}{4.658485in}}%
\pgfpathlineto{\pgfqpoint{1.104452in}{4.672197in}}%
\pgfpathlineto{\pgfqpoint{1.120621in}{4.688529in}}%
\pgfpathlineto{\pgfqpoint{1.129763in}{4.697667in}}%
\pgfpathlineto{\pgfqpoint{1.150194in}{4.717832in}}%
\pgfpathlineto{\pgfqpoint{1.155624in}{4.723136in}}%
\pgfpathlineto{\pgfqpoint{1.179767in}{4.746427in}}%
\pgfpathlineto{\pgfqpoint{1.182049in}{4.748606in}}%
\pgfpathlineto{\pgfqpoint{1.209060in}{4.774075in}}%
\pgfpathlineto{\pgfqpoint{1.209341in}{4.774337in}}%
\pgfpathlineto{\pgfqpoint{1.236695in}{4.799545in}}%
\pgfpathlineto{\pgfqpoint{1.238914in}{4.801565in}}%
\pgfpathlineto{\pgfqpoint{1.264931in}{4.825014in}}%
\pgfpathlineto{\pgfqpoint{1.268488in}{4.828180in}}%
\pgfpathlineto{\pgfqpoint{1.293783in}{4.850484in}}%
\pgfpathlineto{\pgfqpoint{1.298061in}{4.854209in}}%
\pgfpathlineto{\pgfqpoint{1.323266in}{4.875953in}}%
\pgfpathlineto{\pgfqpoint{1.327634in}{4.879676in}}%
\pgfpathlineto{\pgfqpoint{1.353394in}{4.901423in}}%
\pgfpathlineto{\pgfqpoint{1.357208in}{4.904603in}}%
\pgfpathlineto{\pgfqpoint{1.384183in}{4.926892in}}%
\pgfpathlineto{\pgfqpoint{1.386781in}{4.929013in}}%
\pgfpathlineto{\pgfqpoint{1.415645in}{4.952362in}}%
\pgfpathlineto{\pgfqpoint{1.416354in}{4.952929in}}%
\pgfpathlineto{\pgfqpoint{1.445928in}{4.976340in}}%
\pgfpathlineto{\pgfqpoint{1.447828in}{4.977831in}}%
\pgfpathlineto{\pgfqpoint{1.475501in}{4.999277in}}%
\pgfpathlineto{\pgfqpoint{1.480738in}{5.003301in}}%
\pgfpathlineto{\pgfqpoint{1.505074in}{5.021771in}}%
\pgfpathlineto{\pgfqpoint{1.514376in}{5.028770in}}%
\pgfpathlineto{\pgfqpoint{1.534648in}{5.043840in}}%
\pgfpathlineto{\pgfqpoint{1.548755in}{5.054240in}}%
\pgfpathlineto{\pgfqpoint{1.564221in}{5.065504in}}%
\pgfpathlineto{\pgfqpoint{1.583887in}{5.079709in}}%
\pgfpathlineto{\pgfqpoint{1.593795in}{5.086779in}}%
\pgfpathlineto{\pgfqpoint{1.619787in}{5.105179in}}%
\pgfpathlineto{\pgfqpoint{1.623368in}{5.107684in}}%
\pgfpathlineto{\pgfqpoint{1.652941in}{5.128185in}}%
\pgfpathlineto{\pgfqpoint{1.656526in}{5.130649in}}%
\pgfpathlineto{\pgfqpoint{1.682515in}{5.148291in}}%
\pgfpathlineto{\pgfqpoint{1.694133in}{5.156118in}}%
\pgfpathlineto{\pgfqpoint{1.712088in}{5.168067in}}%
\pgfpathlineto{\pgfqpoint{1.732555in}{5.181588in}}%
\pgfpathlineto{\pgfqpoint{1.741661in}{5.187530in}}%
\pgfpathlineto{\pgfqpoint{1.771235in}{5.206687in}}%
\pgfpathlineto{\pgfqpoint{1.771812in}{5.207057in}}%
\pgfpathlineto{\pgfqpoint{1.800808in}{5.225433in}}%
\pgfpathlineto{\pgfqpoint{1.812082in}{5.232527in}}%
\pgfpathlineto{\pgfqpoint{1.830381in}{5.243902in}}%
\pgfpathlineto{\pgfqpoint{1.853210in}{5.257996in}}%
\pgfpathlineto{\pgfqpoint{1.859955in}{5.262109in}}%
\pgfpathlineto{\pgfqpoint{1.889528in}{5.279998in}}%
\pgfpathlineto{\pgfqpoint{1.895309in}{5.283466in}}%
\pgfpathlineto{\pgfqpoint{1.919102in}{5.297562in}}%
\pgfpathlineto{\pgfqpoint{1.938421in}{5.308935in}}%
\pgfpathlineto{\pgfqpoint{1.948675in}{5.314898in}}%
\pgfpathlineto{\pgfqpoint{1.978248in}{5.331969in}}%
\pgfpathlineto{\pgfqpoint{1.982507in}{5.334405in}}%
\pgfpathlineto{\pgfqpoint{2.007822in}{5.348709in}}%
\pgfpathlineto{\pgfqpoint{2.027700in}{5.359874in}}%
\pgfpathlineto{\pgfqpoint{2.037395in}{5.365253in}}%
\pgfpathlineto{\pgfqpoint{2.066968in}{5.381536in}}%
\pgfpathlineto{\pgfqpoint{2.073944in}{5.385344in}}%
\pgfpathlineto{\pgfqpoint{2.096542in}{5.397529in}}%
\pgfpathlineto{\pgfqpoint{2.121313in}{5.410813in}}%
\pgfpathlineto{\pgfqpoint{2.126115in}{5.413357in}}%
\pgfpathlineto{\pgfqpoint{2.155689in}{5.428879in}}%
\pgfpathlineto{\pgfqpoint{2.169890in}{5.436283in}}%
\pgfpathlineto{\pgfqpoint{2.185262in}{5.444199in}}%
\pgfpathlineto{\pgfqpoint{2.214835in}{5.459331in}}%
\pgfpathlineto{\pgfqpoint{2.219612in}{5.461752in}}%
\pgfpathlineto{\pgfqpoint{2.244409in}{5.474167in}}%
\pgfpathlineto{\pgfqpoint{2.270609in}{5.487222in}}%
\pgfpathlineto{\pgfqpoint{2.273982in}{5.488882in}}%
\pgfpathlineto{\pgfqpoint{2.303555in}{5.503291in}}%
\pgfpathlineto{\pgfqpoint{2.322953in}{5.512691in}}%
\pgfpathlineto{\pgfqpoint{2.333129in}{5.517561in}}%
\pgfpathlineto{\pgfqpoint{2.362702in}{5.531599in}}%
\pgfpathlineto{\pgfqpoint{2.376624in}{5.538161in}}%
\pgfpathlineto{\pgfqpoint{2.392275in}{5.545446in}}%
\pgfpathlineto{\pgfqpoint{2.421849in}{5.559118in}}%
\pgfpathlineto{\pgfqpoint{2.431692in}{5.563630in}}%
\pgfpathlineto{\pgfqpoint{2.451422in}{5.572562in}}%
\pgfpathlineto{\pgfqpoint{2.480996in}{5.585873in}}%
\pgfpathlineto{\pgfqpoint{2.488231in}{5.589100in}}%
\pgfpathlineto{\pgfqpoint{2.510569in}{5.598935in}}%
\pgfpathlineto{\pgfqpoint{2.540142in}{5.611891in}}%
\pgfpathlineto{\pgfqpoint{2.546317in}{5.614570in}}%
\pgfpathlineto{\pgfqpoint{2.569716in}{5.624591in}}%
\pgfpathlineto{\pgfqpoint{2.599289in}{5.637195in}}%
\pgfpathlineto{\pgfqpoint{2.606029in}{5.640039in}}%
\pgfusepath{stroke}%
\end{pgfscope}%
\begin{pgfscope}%
\pgfpathrectangle{\pgfqpoint{0.854460in}{0.571603in}}{\pgfqpoint{5.885100in}{5.068436in}}%
\pgfusepath{clip}%
\pgfsetbuttcap%
\pgfsetroundjoin%
\pgfsetlinewidth{1.505625pt}%
\definecolor{currentstroke}{rgb}{0.131172,0.555899,0.552459}%
\pgfsetstrokecolor{currentstroke}%
\pgfsetdash{}{0pt}%
\pgfpathmoveto{\pgfqpoint{6.022114in}{5.640039in}}%
\pgfpathlineto{\pgfqpoint{6.023966in}{5.589100in}}%
\pgfpathlineto{\pgfqpoint{6.022961in}{5.538161in}}%
\pgfpathlineto{\pgfqpoint{6.019566in}{5.487222in}}%
\pgfpathlineto{\pgfqpoint{6.014176in}{5.436283in}}%
\pgfpathlineto{\pgfqpoint{6.003066in}{5.359874in}}%
\pgfpathlineto{\pgfqpoint{5.988991in}{5.283466in}}%
\pgfpathlineto{\pgfqpoint{5.970653in}{5.197681in}}%
\pgfpathlineto{\pgfqpoint{5.948707in}{5.105179in}}%
\pgfpathlineto{\pgfqpoint{5.916297in}{4.977831in}}%
\pgfpathlineto{\pgfqpoint{5.822786in}{4.615830in}}%
\pgfpathlineto{\pgfqpoint{5.793213in}{4.490231in}}%
\pgfpathlineto{\pgfqpoint{5.771943in}{4.392032in}}%
\pgfpathlineto{\pgfqpoint{5.751996in}{4.290154in}}%
\pgfpathlineto{\pgfqpoint{5.734066in}{4.185517in}}%
\pgfpathlineto{\pgfqpoint{5.719585in}{4.086398in}}%
\pgfpathlineto{\pgfqpoint{5.710316in}{4.009989in}}%
\pgfpathlineto{\pgfqpoint{5.702748in}{3.933581in}}%
\pgfpathlineto{\pgfqpoint{5.696923in}{3.857172in}}%
\pgfpathlineto{\pgfqpoint{5.692956in}{3.780764in}}%
\pgfpathlineto{\pgfqpoint{5.690903in}{3.704355in}}%
\pgfpathlineto{\pgfqpoint{5.690819in}{3.627946in}}%
\pgfpathlineto{\pgfqpoint{5.692758in}{3.551538in}}%
\pgfpathlineto{\pgfqpoint{5.696771in}{3.475129in}}%
\pgfpathlineto{\pgfqpoint{5.702912in}{3.398721in}}%
\pgfpathlineto{\pgfqpoint{5.711173in}{3.322312in}}%
\pgfpathlineto{\pgfqpoint{5.721636in}{3.245904in}}%
\pgfpathlineto{\pgfqpoint{5.734366in}{3.169495in}}%
\pgfpathlineto{\pgfqpoint{5.749293in}{3.093086in}}%
\pgfpathlineto{\pgfqpoint{5.766561in}{3.016678in}}%
\pgfpathlineto{\pgfqpoint{5.786108in}{2.940269in}}%
\pgfpathlineto{\pgfqpoint{5.807996in}{2.863861in}}%
\pgfpathlineto{\pgfqpoint{5.832254in}{2.787452in}}%
\pgfpathlineto{\pgfqpoint{5.858889in}{2.711044in}}%
\pgfpathlineto{\pgfqpoint{5.887913in}{2.634635in}}%
\pgfpathlineto{\pgfqpoint{5.919338in}{2.558226in}}%
\pgfpathlineto{\pgfqpoint{5.953180in}{2.481818in}}%
\pgfpathlineto{\pgfqpoint{5.989460in}{2.405409in}}%
\pgfpathlineto{\pgfqpoint{6.029799in}{2.325983in}}%
\pgfpathlineto{\pgfqpoint{6.069345in}{2.252592in}}%
\pgfpathlineto{\pgfqpoint{6.118520in}{2.166840in}}%
\pgfpathlineto{\pgfqpoint{6.159038in}{2.099775in}}%
\pgfpathlineto{\pgfqpoint{6.207603in}{2.023366in}}%
\pgfpathlineto{\pgfqpoint{6.266386in}{1.935647in}}%
\pgfpathlineto{\pgfqpoint{6.312036in}{1.870549in}}%
\pgfpathlineto{\pgfqpoint{6.367962in}{1.794141in}}%
\pgfpathlineto{\pgfqpoint{6.426349in}{1.717732in}}%
\pgfpathlineto{\pgfqpoint{6.502973in}{1.622050in}}%
\pgfpathlineto{\pgfqpoint{6.562120in}{1.551248in}}%
\pgfpathlineto{\pgfqpoint{6.621267in}{1.482827in}}%
\pgfpathlineto{\pgfqpoint{6.684450in}{1.412098in}}%
\pgfpathlineto{\pgfqpoint{6.739560in}{1.352280in}}%
\pgfpathlineto{\pgfqpoint{6.739560in}{1.352280in}}%
\pgfusepath{stroke}%
\end{pgfscope}%
\begin{pgfscope}%
\pgfpathrectangle{\pgfqpoint{0.854460in}{0.571603in}}{\pgfqpoint{5.885100in}{5.068436in}}%
\pgfusepath{clip}%
\pgfsetbuttcap%
\pgfsetroundjoin%
\pgfsetlinewidth{1.505625pt}%
\definecolor{currentstroke}{rgb}{0.125394,0.574318,0.549086}%
\pgfsetstrokecolor{currentstroke}%
\pgfsetdash{}{0pt}%
\pgfpathmoveto{\pgfqpoint{1.219941in}{0.571603in}}%
\pgfpathlineto{\pgfqpoint{1.209341in}{0.581376in}}%
\pgfpathlineto{\pgfqpoint{1.192400in}{0.597073in}}%
\pgfpathlineto{\pgfqpoint{1.179767in}{0.608943in}}%
\pgfpathlineto{\pgfqpoint{1.165370in}{0.622542in}}%
\pgfpathlineto{\pgfqpoint{1.150194in}{0.637080in}}%
\pgfpathlineto{\pgfqpoint{1.138843in}{0.648012in}}%
\pgfpathlineto{\pgfqpoint{1.120621in}{0.665810in}}%
\pgfpathlineto{\pgfqpoint{1.112810in}{0.673481in}}%
\pgfpathlineto{\pgfqpoint{1.091047in}{0.695160in}}%
\pgfpathlineto{\pgfqpoint{1.087263in}{0.698951in}}%
\pgfpathlineto{\pgfqpoint{1.062202in}{0.724420in}}%
\pgfpathlineto{\pgfqpoint{1.061474in}{0.725172in}}%
\pgfpathlineto{\pgfqpoint{1.037667in}{0.749890in}}%
\pgfpathlineto{\pgfqpoint{1.031901in}{0.755963in}}%
\pgfpathlineto{\pgfqpoint{1.013595in}{0.775360in}}%
\pgfpathlineto{\pgfqpoint{1.002327in}{0.787472in}}%
\pgfpathlineto{\pgfqpoint{0.989979in}{0.800829in}}%
\pgfpathlineto{\pgfqpoint{0.972754in}{0.819731in}}%
\pgfpathlineto{\pgfqpoint{0.966807in}{0.826299in}}%
\pgfpathlineto{\pgfqpoint{0.944083in}{0.851768in}}%
\pgfpathlineto{\pgfqpoint{0.943181in}{0.852796in}}%
\pgfpathlineto{\pgfqpoint{0.921870in}{0.877238in}}%
\pgfpathlineto{\pgfqpoint{0.913607in}{0.886854in}}%
\pgfpathlineto{\pgfqpoint{0.900078in}{0.902707in}}%
\pgfpathlineto{\pgfqpoint{0.884034in}{0.921785in}}%
\pgfpathlineto{\pgfqpoint{0.878696in}{0.928177in}}%
\pgfpathlineto{\pgfqpoint{0.857758in}{0.953646in}}%
\pgfpathlineto{\pgfqpoint{0.854460in}{0.957724in}}%
\pgfusepath{stroke}%
\end{pgfscope}%
\begin{pgfscope}%
\pgfpathrectangle{\pgfqpoint{0.854460in}{0.571603in}}{\pgfqpoint{5.885100in}{5.068436in}}%
\pgfusepath{clip}%
\pgfsetbuttcap%
\pgfsetroundjoin%
\pgfsetlinewidth{1.505625pt}%
\definecolor{currentstroke}{rgb}{0.125394,0.574318,0.549086}%
\pgfsetstrokecolor{currentstroke}%
\pgfsetdash{}{0pt}%
\pgfpathmoveto{\pgfqpoint{0.854460in}{4.472844in}}%
\pgfpathlineto{\pgfqpoint{0.871713in}{4.493910in}}%
\pgfpathlineto{\pgfqpoint{0.884034in}{4.508758in}}%
\pgfpathlineto{\pgfqpoint{0.892949in}{4.519380in}}%
\pgfpathlineto{\pgfqpoint{0.913607in}{4.543671in}}%
\pgfpathlineto{\pgfqpoint{0.914620in}{4.544849in}}%
\pgfpathlineto{\pgfqpoint{0.936847in}{4.570319in}}%
\pgfpathlineto{\pgfqpoint{0.943181in}{4.577481in}}%
\pgfpathlineto{\pgfqpoint{0.959548in}{4.595788in}}%
\pgfpathlineto{\pgfqpoint{0.972754in}{4.610370in}}%
\pgfpathlineto{\pgfqpoint{0.982722in}{4.621258in}}%
\pgfpathlineto{\pgfqpoint{1.002327in}{4.642399in}}%
\pgfpathlineto{\pgfqpoint{1.006384in}{4.646728in}}%
\pgfpathlineto{\pgfqpoint{1.030571in}{4.672197in}}%
\pgfpathlineto{\pgfqpoint{1.031901in}{4.673578in}}%
\pgfpathlineto{\pgfqpoint{1.055335in}{4.697667in}}%
\pgfpathlineto{\pgfqpoint{1.061474in}{4.703898in}}%
\pgfpathlineto{\pgfqpoint{1.080624in}{4.723136in}}%
\pgfpathlineto{\pgfqpoint{1.091047in}{4.733477in}}%
\pgfpathlineto{\pgfqpoint{1.106453in}{4.748606in}}%
\pgfpathlineto{\pgfqpoint{1.120621in}{4.762346in}}%
\pgfpathlineto{\pgfqpoint{1.132836in}{4.774075in}}%
\pgfpathlineto{\pgfqpoint{1.150194in}{4.790535in}}%
\pgfpathlineto{\pgfqpoint{1.159790in}{4.799545in}}%
\pgfpathlineto{\pgfqpoint{1.179767in}{4.818070in}}%
\pgfpathlineto{\pgfqpoint{1.187329in}{4.825014in}}%
\pgfpathlineto{\pgfqpoint{1.209341in}{4.844980in}}%
\pgfpathlineto{\pgfqpoint{1.215467in}{4.850484in}}%
\pgfpathlineto{\pgfqpoint{1.238914in}{4.871290in}}%
\pgfpathlineto{\pgfqpoint{1.244220in}{4.875953in}}%
\pgfpathlineto{\pgfqpoint{1.268488in}{4.897024in}}%
\pgfpathlineto{\pgfqpoint{1.273602in}{4.901423in}}%
\pgfpathlineto{\pgfqpoint{1.298061in}{4.922206in}}%
\pgfpathlineto{\pgfqpoint{1.303627in}{4.926892in}}%
\pgfpathlineto{\pgfqpoint{1.327634in}{4.946859in}}%
\pgfpathlineto{\pgfqpoint{1.334310in}{4.952362in}}%
\pgfpathlineto{\pgfqpoint{1.357208in}{4.971006in}}%
\pgfpathlineto{\pgfqpoint{1.365664in}{4.977831in}}%
\pgfpathlineto{\pgfqpoint{1.386781in}{4.994667in}}%
\pgfpathlineto{\pgfqpoint{1.397704in}{5.003301in}}%
\pgfpathlineto{\pgfqpoint{1.416354in}{5.017863in}}%
\pgfpathlineto{\pgfqpoint{1.430443in}{5.028770in}}%
\pgfpathlineto{\pgfqpoint{1.445928in}{5.040613in}}%
\pgfpathlineto{\pgfqpoint{1.463894in}{5.054240in}}%
\pgfpathlineto{\pgfqpoint{1.475501in}{5.062937in}}%
\pgfpathlineto{\pgfqpoint{1.498069in}{5.079709in}}%
\pgfpathlineto{\pgfqpoint{1.505074in}{5.084853in}}%
\pgfpathlineto{\pgfqpoint{1.532982in}{5.105179in}}%
\pgfpathlineto{\pgfqpoint{1.534648in}{5.106378in}}%
\pgfpathlineto{\pgfqpoint{1.564221in}{5.127467in}}%
\pgfpathlineto{\pgfqpoint{1.568719in}{5.130649in}}%
\pgfpathlineto{\pgfqpoint{1.593795in}{5.148169in}}%
\pgfpathlineto{\pgfqpoint{1.605258in}{5.156118in}}%
\pgfpathlineto{\pgfqpoint{1.623368in}{5.168523in}}%
\pgfpathlineto{\pgfqpoint{1.642582in}{5.181588in}}%
\pgfpathlineto{\pgfqpoint{1.652941in}{5.188546in}}%
\pgfpathlineto{\pgfqpoint{1.680701in}{5.207057in}}%
\pgfpathlineto{\pgfqpoint{1.682515in}{5.208252in}}%
\pgfpathlineto{\pgfqpoint{1.712088in}{5.227558in}}%
\pgfpathlineto{\pgfqpoint{1.719756in}{5.232527in}}%
\pgfpathlineto{\pgfqpoint{1.741661in}{5.246548in}}%
\pgfpathlineto{\pgfqpoint{1.759671in}{5.257996in}}%
\pgfpathlineto{\pgfqpoint{1.771235in}{5.265258in}}%
\pgfpathlineto{\pgfqpoint{1.800423in}{5.283466in}}%
\pgfpathlineto{\pgfqpoint{1.800808in}{5.283703in}}%
\pgfpathlineto{\pgfqpoint{1.830381in}{5.301754in}}%
\pgfpathlineto{\pgfqpoint{1.842224in}{5.308935in}}%
\pgfpathlineto{\pgfqpoint{1.859955in}{5.319556in}}%
\pgfpathlineto{\pgfqpoint{1.884900in}{5.334405in}}%
\pgfpathlineto{\pgfqpoint{1.889528in}{5.337126in}}%
\pgfpathlineto{\pgfqpoint{1.919102in}{5.354368in}}%
\pgfpathlineto{\pgfqpoint{1.928614in}{5.359874in}}%
\pgfpathlineto{\pgfqpoint{1.948675in}{5.371343in}}%
\pgfpathlineto{\pgfqpoint{1.973306in}{5.385344in}}%
\pgfpathlineto{\pgfqpoint{1.978248in}{5.388119in}}%
\pgfpathlineto{\pgfqpoint{2.007822in}{5.404581in}}%
\pgfpathlineto{\pgfqpoint{2.019094in}{5.410813in}}%
\pgfpathlineto{\pgfqpoint{2.037395in}{5.420806in}}%
\pgfpathlineto{\pgfqpoint{2.065890in}{5.436283in}}%
\pgfpathlineto{\pgfqpoint{2.066968in}{5.436861in}}%
\pgfpathlineto{\pgfqpoint{2.096542in}{5.452573in}}%
\pgfpathlineto{\pgfqpoint{2.113915in}{5.461752in}}%
\pgfpathlineto{\pgfqpoint{2.126115in}{5.468119in}}%
\pgfpathlineto{\pgfqpoint{2.155689in}{5.483448in}}%
\pgfpathlineto{\pgfqpoint{2.163029in}{5.487222in}}%
\pgfpathlineto{\pgfqpoint{2.185262in}{5.498510in}}%
\pgfpathlineto{\pgfqpoint{2.213322in}{5.512691in}}%
\pgfpathlineto{\pgfqpoint{2.214835in}{5.513447in}}%
\pgfpathlineto{\pgfqpoint{2.244409in}{5.528062in}}%
\pgfpathlineto{\pgfqpoint{2.264940in}{5.538161in}}%
\pgfpathlineto{\pgfqpoint{2.273982in}{5.542553in}}%
\pgfpathlineto{\pgfqpoint{2.303555in}{5.556804in}}%
\pgfpathlineto{\pgfqpoint{2.317815in}{5.563630in}}%
\pgfpathlineto{\pgfqpoint{2.333129in}{5.570870in}}%
\pgfpathlineto{\pgfqpoint{2.362702in}{5.584762in}}%
\pgfpathlineto{\pgfqpoint{2.372011in}{5.589100in}}%
\pgfpathlineto{\pgfqpoint{2.392275in}{5.598425in}}%
\pgfpathlineto{\pgfqpoint{2.421849in}{5.611963in}}%
\pgfpathlineto{\pgfqpoint{2.427594in}{5.614570in}}%
\pgfpathlineto{\pgfqpoint{2.451422in}{5.625242in}}%
\pgfpathlineto{\pgfqpoint{2.480996in}{5.638432in}}%
\pgfpathlineto{\pgfqpoint{2.484635in}{5.640039in}}%
\pgfusepath{stroke}%
\end{pgfscope}%
\begin{pgfscope}%
\pgfpathrectangle{\pgfqpoint{0.854460in}{0.571603in}}{\pgfqpoint{5.885100in}{5.068436in}}%
\pgfusepath{clip}%
\pgfsetbuttcap%
\pgfsetroundjoin%
\pgfsetlinewidth{1.505625pt}%
\definecolor{currentstroke}{rgb}{0.125394,0.574318,0.549086}%
\pgfsetstrokecolor{currentstroke}%
\pgfsetdash{}{0pt}%
\pgfpathmoveto{\pgfqpoint{6.158977in}{5.640039in}}%
\pgfpathlineto{\pgfqpoint{6.155981in}{5.589100in}}%
\pgfpathlineto{\pgfqpoint{6.150734in}{5.538161in}}%
\pgfpathlineto{\pgfqpoint{6.143547in}{5.487222in}}%
\pgfpathlineto{\pgfqpoint{6.129836in}{5.410813in}}%
\pgfpathlineto{\pgfqpoint{6.113391in}{5.334405in}}%
\pgfpathlineto{\pgfqpoint{6.088313in}{5.232527in}}%
\pgfpathlineto{\pgfqpoint{6.053614in}{5.105179in}}%
\pgfpathlineto{\pgfqpoint{5.987486in}{4.875953in}}%
\pgfpathlineto{\pgfqpoint{5.937015in}{4.697667in}}%
\pgfpathlineto{\pgfqpoint{5.903129in}{4.570319in}}%
\pgfpathlineto{\pgfqpoint{5.877958in}{4.468441in}}%
\pgfpathlineto{\pgfqpoint{5.854844in}{4.366563in}}%
\pgfpathlineto{\pgfqpoint{5.834054in}{4.264685in}}%
\pgfpathlineto{\pgfqpoint{5.815902in}{4.162807in}}%
\pgfpathlineto{\pgfqpoint{5.800570in}{4.060928in}}%
\pgfpathlineto{\pgfqpoint{5.791066in}{3.984520in}}%
\pgfpathlineto{\pgfqpoint{5.783304in}{3.908111in}}%
\pgfpathlineto{\pgfqpoint{5.777409in}{3.831703in}}%
\pgfpathlineto{\pgfqpoint{5.773434in}{3.755294in}}%
\pgfpathlineto{\pgfqpoint{5.771429in}{3.678886in}}%
\pgfpathlineto{\pgfqpoint{5.771446in}{3.602477in}}%
\pgfpathlineto{\pgfqpoint{5.773532in}{3.526068in}}%
\pgfpathlineto{\pgfqpoint{5.777737in}{3.449660in}}%
\pgfpathlineto{\pgfqpoint{5.784112in}{3.373251in}}%
\pgfpathlineto{\pgfqpoint{5.793213in}{3.293009in}}%
\pgfpathlineto{\pgfqpoint{5.803480in}{3.220434in}}%
\pgfpathlineto{\pgfqpoint{5.816555in}{3.144025in}}%
\pgfpathlineto{\pgfqpoint{5.831910in}{3.067617in}}%
\pgfpathlineto{\pgfqpoint{5.849608in}{2.991208in}}%
\pgfpathlineto{\pgfqpoint{5.869614in}{2.914800in}}%
\pgfpathlineto{\pgfqpoint{5.892009in}{2.838391in}}%
\pgfpathlineto{\pgfqpoint{5.916796in}{2.761983in}}%
\pgfpathlineto{\pgfqpoint{5.943984in}{2.685574in}}%
\pgfpathlineto{\pgfqpoint{5.973582in}{2.609165in}}%
\pgfpathlineto{\pgfqpoint{6.005604in}{2.532757in}}%
\pgfpathlineto{\pgfqpoint{6.040067in}{2.456348in}}%
\pgfpathlineto{\pgfqpoint{6.076990in}{2.379940in}}%
\pgfpathlineto{\pgfqpoint{6.118520in}{2.299590in}}%
\pgfpathlineto{\pgfqpoint{6.158233in}{2.227123in}}%
\pgfpathlineto{\pgfqpoint{6.207240in}{2.142975in}}%
\pgfpathlineto{\pgfqpoint{6.249365in}{2.074305in}}%
\pgfpathlineto{\pgfqpoint{6.298671in}{1.997897in}}%
\pgfpathlineto{\pgfqpoint{6.355107in}{1.914817in}}%
\pgfpathlineto{\pgfqpoint{6.414253in}{1.832009in}}%
\pgfpathlineto{\pgfqpoint{6.461406in}{1.768671in}}%
\pgfpathlineto{\pgfqpoint{6.532547in}{1.677307in}}%
\pgfpathlineto{\pgfqpoint{6.591693in}{1.604550in}}%
\pgfpathlineto{\pgfqpoint{6.650840in}{1.534396in}}%
\pgfpathlineto{\pgfqpoint{6.713123in}{1.463037in}}%
\pgfpathlineto{\pgfqpoint{6.739560in}{1.433500in}}%
\pgfpathlineto{\pgfqpoint{6.739560in}{1.433500in}}%
\pgfusepath{stroke}%
\end{pgfscope}%
\begin{pgfscope}%
\pgfpathrectangle{\pgfqpoint{0.854460in}{0.571603in}}{\pgfqpoint{5.885100in}{5.068436in}}%
\pgfusepath{clip}%
\pgfsetbuttcap%
\pgfsetroundjoin%
\pgfsetlinewidth{1.505625pt}%
\definecolor{currentstroke}{rgb}{0.120565,0.596422,0.543611}%
\pgfsetstrokecolor{currentstroke}%
\pgfsetdash{}{0pt}%
\pgfpathmoveto{\pgfqpoint{1.160477in}{0.571603in}}%
\pgfpathlineto{\pgfqpoint{1.150194in}{0.581167in}}%
\pgfpathlineto{\pgfqpoint{1.133178in}{0.597073in}}%
\pgfpathlineto{\pgfqpoint{1.120621in}{0.608977in}}%
\pgfpathlineto{\pgfqpoint{1.106385in}{0.622542in}}%
\pgfpathlineto{\pgfqpoint{1.091047in}{0.637364in}}%
\pgfpathlineto{\pgfqpoint{1.080089in}{0.648012in}}%
\pgfpathlineto{\pgfqpoint{1.061474in}{0.666354in}}%
\pgfpathlineto{\pgfqpoint{1.054281in}{0.673481in}}%
\pgfpathlineto{\pgfqpoint{1.031901in}{0.695972in}}%
\pgfpathlineto{\pgfqpoint{1.028953in}{0.698951in}}%
\pgfpathlineto{\pgfqpoint{1.004120in}{0.724420in}}%
\pgfpathlineto{\pgfqpoint{1.002327in}{0.726287in}}%
\pgfpathlineto{\pgfqpoint{0.979793in}{0.749890in}}%
\pgfpathlineto{\pgfqpoint{0.972754in}{0.757369in}}%
\pgfpathlineto{\pgfqpoint{0.955925in}{0.775360in}}%
\pgfpathlineto{\pgfqpoint{0.943181in}{0.789180in}}%
\pgfpathlineto{\pgfqpoint{0.932506in}{0.800829in}}%
\pgfpathlineto{\pgfqpoint{0.913607in}{0.821751in}}%
\pgfpathlineto{\pgfqpoint{0.909525in}{0.826299in}}%
\pgfpathlineto{\pgfqpoint{0.887013in}{0.851768in}}%
\pgfpathlineto{\pgfqpoint{0.884034in}{0.855193in}}%
\pgfpathlineto{\pgfqpoint{0.864981in}{0.877238in}}%
\pgfpathlineto{\pgfqpoint{0.854460in}{0.889589in}}%
\pgfusepath{stroke}%
\end{pgfscope}%
\begin{pgfscope}%
\pgfpathrectangle{\pgfqpoint{0.854460in}{0.571603in}}{\pgfqpoint{5.885100in}{5.068436in}}%
\pgfusepath{clip}%
\pgfsetbuttcap%
\pgfsetroundjoin%
\pgfsetlinewidth{1.505625pt}%
\definecolor{currentstroke}{rgb}{0.120565,0.596422,0.543611}%
\pgfsetstrokecolor{currentstroke}%
\pgfsetdash{}{0pt}%
\pgfpathmoveto{\pgfqpoint{0.854460in}{4.554539in}}%
\pgfpathlineto{\pgfqpoint{0.867984in}{4.570319in}}%
\pgfpathlineto{\pgfqpoint{0.884034in}{4.588803in}}%
\pgfpathlineto{\pgfqpoint{0.890165in}{4.595788in}}%
\pgfpathlineto{\pgfqpoint{0.912817in}{4.621258in}}%
\pgfpathlineto{\pgfqpoint{0.913607in}{4.622133in}}%
\pgfpathlineto{\pgfqpoint{0.936035in}{4.646728in}}%
\pgfpathlineto{\pgfqpoint{0.943181in}{4.654464in}}%
\pgfpathlineto{\pgfqpoint{0.959732in}{4.672197in}}%
\pgfpathlineto{\pgfqpoint{0.972754in}{4.685971in}}%
\pgfpathlineto{\pgfqpoint{0.983926in}{4.697667in}}%
\pgfpathlineto{\pgfqpoint{1.002327in}{4.716688in}}%
\pgfpathlineto{\pgfqpoint{1.008629in}{4.723136in}}%
\pgfpathlineto{\pgfqpoint{1.031901in}{4.746648in}}%
\pgfpathlineto{\pgfqpoint{1.033858in}{4.748606in}}%
\pgfpathlineto{\pgfqpoint{1.059656in}{4.774075in}}%
\pgfpathlineto{\pgfqpoint{1.061474in}{4.775847in}}%
\pgfpathlineto{\pgfqpoint{1.086032in}{4.799545in}}%
\pgfpathlineto{\pgfqpoint{1.091047in}{4.804324in}}%
\pgfpathlineto{\pgfqpoint{1.112968in}{4.825014in}}%
\pgfpathlineto{\pgfqpoint{1.120621in}{4.832148in}}%
\pgfpathlineto{\pgfqpoint{1.140479in}{4.850484in}}%
\pgfpathlineto{\pgfqpoint{1.150194in}{4.859344in}}%
\pgfpathlineto{\pgfqpoint{1.168578in}{4.875953in}}%
\pgfpathlineto{\pgfqpoint{1.179767in}{4.885938in}}%
\pgfpathlineto{\pgfqpoint{1.197281in}{4.901423in}}%
\pgfpathlineto{\pgfqpoint{1.209341in}{4.911955in}}%
\pgfpathlineto{\pgfqpoint{1.226602in}{4.926892in}}%
\pgfpathlineto{\pgfqpoint{1.238914in}{4.937418in}}%
\pgfpathlineto{\pgfqpoint{1.256553in}{4.952362in}}%
\pgfpathlineto{\pgfqpoint{1.268488in}{4.962350in}}%
\pgfpathlineto{\pgfqpoint{1.287149in}{4.977831in}}%
\pgfpathlineto{\pgfqpoint{1.298061in}{4.986773in}}%
\pgfpathlineto{\pgfqpoint{1.318404in}{5.003301in}}%
\pgfpathlineto{\pgfqpoint{1.327634in}{5.010709in}}%
\pgfpathlineto{\pgfqpoint{1.350330in}{5.028770in}}%
\pgfpathlineto{\pgfqpoint{1.357208in}{5.034177in}}%
\pgfpathlineto{\pgfqpoint{1.382941in}{5.054240in}}%
\pgfpathlineto{\pgfqpoint{1.386781in}{5.057197in}}%
\pgfpathlineto{\pgfqpoint{1.416249in}{5.079709in}}%
\pgfpathlineto{\pgfqpoint{1.416354in}{5.079789in}}%
\pgfpathlineto{\pgfqpoint{1.445928in}{5.101906in}}%
\pgfpathlineto{\pgfqpoint{1.450339in}{5.105179in}}%
\pgfpathlineto{\pgfqpoint{1.475501in}{5.123621in}}%
\pgfpathlineto{\pgfqpoint{1.485164in}{5.130649in}}%
\pgfpathlineto{\pgfqpoint{1.505074in}{5.144953in}}%
\pgfpathlineto{\pgfqpoint{1.520734in}{5.156118in}}%
\pgfpathlineto{\pgfqpoint{1.534648in}{5.165919in}}%
\pgfpathlineto{\pgfqpoint{1.557060in}{5.181588in}}%
\pgfpathlineto{\pgfqpoint{1.564221in}{5.186534in}}%
\pgfpathlineto{\pgfqpoint{1.593795in}{5.206810in}}%
\pgfpathlineto{\pgfqpoint{1.594158in}{5.207057in}}%
\pgfpathlineto{\pgfqpoint{1.623368in}{5.226662in}}%
\pgfpathlineto{\pgfqpoint{1.632168in}{5.232527in}}%
\pgfpathlineto{\pgfqpoint{1.652941in}{5.246203in}}%
\pgfpathlineto{\pgfqpoint{1.670978in}{5.257996in}}%
\pgfpathlineto{\pgfqpoint{1.682515in}{5.265448in}}%
\pgfpathlineto{\pgfqpoint{1.710596in}{5.283466in}}%
\pgfpathlineto{\pgfqpoint{1.712088in}{5.284412in}}%
\pgfpathlineto{\pgfqpoint{1.741661in}{5.302990in}}%
\pgfpathlineto{\pgfqpoint{1.751190in}{5.308935in}}%
\pgfpathlineto{\pgfqpoint{1.771235in}{5.321289in}}%
\pgfpathlineto{\pgfqpoint{1.792649in}{5.334405in}}%
\pgfpathlineto{\pgfqpoint{1.800808in}{5.339341in}}%
\pgfpathlineto{\pgfqpoint{1.830381in}{5.357104in}}%
\pgfpathlineto{\pgfqpoint{1.835032in}{5.359874in}}%
\pgfpathlineto{\pgfqpoint{1.859955in}{5.374540in}}%
\pgfpathlineto{\pgfqpoint{1.878421in}{5.385344in}}%
\pgfpathlineto{\pgfqpoint{1.889528in}{5.391762in}}%
\pgfpathlineto{\pgfqpoint{1.919102in}{5.408740in}}%
\pgfpathlineto{\pgfqpoint{1.922744in}{5.410813in}}%
\pgfpathlineto{\pgfqpoint{1.948675in}{5.425392in}}%
\pgfpathlineto{\pgfqpoint{1.968151in}{5.436283in}}%
\pgfpathlineto{\pgfqpoint{1.978248in}{5.441860in}}%
\pgfpathlineto{\pgfqpoint{2.007822in}{5.458081in}}%
\pgfpathlineto{\pgfqpoint{2.014567in}{5.461752in}}%
\pgfpathlineto{\pgfqpoint{2.037395in}{5.474023in}}%
\pgfpathlineto{\pgfqpoint{2.062073in}{5.487222in}}%
\pgfpathlineto{\pgfqpoint{2.066968in}{5.489808in}}%
\pgfpathlineto{\pgfqpoint{2.096542in}{5.505300in}}%
\pgfpathlineto{\pgfqpoint{2.110737in}{5.512691in}}%
\pgfpathlineto{\pgfqpoint{2.126115in}{5.520600in}}%
\pgfpathlineto{\pgfqpoint{2.155689in}{5.535721in}}%
\pgfpathlineto{\pgfqpoint{2.160500in}{5.538161in}}%
\pgfpathlineto{\pgfqpoint{2.185262in}{5.550558in}}%
\pgfpathlineto{\pgfqpoint{2.211481in}{5.563630in}}%
\pgfpathlineto{\pgfqpoint{2.214835in}{5.565282in}}%
\pgfpathlineto{\pgfqpoint{2.244409in}{5.579711in}}%
\pgfpathlineto{\pgfqpoint{2.263744in}{5.589100in}}%
\pgfpathlineto{\pgfqpoint{2.273982in}{5.594009in}}%
\pgfpathlineto{\pgfqpoint{2.303555in}{5.608086in}}%
\pgfpathlineto{\pgfqpoint{2.317263in}{5.614570in}}%
\pgfpathlineto{\pgfqpoint{2.333129in}{5.621979in}}%
\pgfpathlineto{\pgfqpoint{2.362702in}{5.635709in}}%
\pgfpathlineto{\pgfqpoint{2.372101in}{5.640039in}}%
\pgfusepath{stroke}%
\end{pgfscope}%
\begin{pgfscope}%
\pgfpathrectangle{\pgfqpoint{0.854460in}{0.571603in}}{\pgfqpoint{5.885100in}{5.068436in}}%
\pgfusepath{clip}%
\pgfsetbuttcap%
\pgfsetroundjoin%
\pgfsetlinewidth{1.505625pt}%
\definecolor{currentstroke}{rgb}{0.120565,0.596422,0.543611}%
\pgfsetstrokecolor{currentstroke}%
\pgfsetdash{}{0pt}%
\pgfpathmoveto{\pgfqpoint{6.287072in}{5.640039in}}%
\pgfpathlineto{\pgfqpoint{6.280061in}{5.589100in}}%
\pgfpathlineto{\pgfqpoint{6.271247in}{5.538161in}}%
\pgfpathlineto{\pgfqpoint{6.255161in}{5.461752in}}%
\pgfpathlineto{\pgfqpoint{6.236465in}{5.385344in}}%
\pgfpathlineto{\pgfqpoint{6.207240in}{5.279337in}}%
\pgfpathlineto{\pgfqpoint{6.170238in}{5.156118in}}%
\pgfpathlineto{\pgfqpoint{6.082057in}{4.875953in}}%
\pgfpathlineto{\pgfqpoint{6.035356in}{4.723136in}}%
\pgfpathlineto{\pgfqpoint{5.998739in}{4.595788in}}%
\pgfpathlineto{\pgfqpoint{5.970653in}{4.490730in}}%
\pgfpathlineto{\pgfqpoint{5.946318in}{4.392032in}}%
\pgfpathlineto{\pgfqpoint{5.923601in}{4.290154in}}%
\pgfpathlineto{\pgfqpoint{5.903574in}{4.188276in}}%
\pgfpathlineto{\pgfqpoint{5.890431in}{4.111867in}}%
\pgfpathlineto{\pgfqpoint{5.879011in}{4.035459in}}%
\pgfpathlineto{\pgfqpoint{5.869335in}{3.959050in}}%
\pgfpathlineto{\pgfqpoint{5.861535in}{3.882642in}}%
\pgfpathlineto{\pgfqpoint{5.855661in}{3.806233in}}%
\pgfpathlineto{\pgfqpoint{5.851754in}{3.729825in}}%
\pgfpathlineto{\pgfqpoint{5.849856in}{3.653416in}}%
\pgfpathlineto{\pgfqpoint{5.850042in}{3.577007in}}%
\pgfpathlineto{\pgfqpoint{5.852359in}{3.500591in}}%
\pgfpathlineto{\pgfqpoint{5.856814in}{3.424190in}}%
\pgfpathlineto{\pgfqpoint{5.863473in}{3.347782in}}%
\pgfpathlineto{\pgfqpoint{5.872386in}{3.271373in}}%
\pgfpathlineto{\pgfqpoint{5.883588in}{3.194965in}}%
\pgfpathlineto{\pgfqpoint{5.897044in}{3.118556in}}%
\pgfpathlineto{\pgfqpoint{5.912876in}{3.042147in}}%
\pgfpathlineto{\pgfqpoint{5.931012in}{2.965739in}}%
\pgfpathlineto{\pgfqpoint{5.951543in}{2.889330in}}%
\pgfpathlineto{\pgfqpoint{5.974482in}{2.812922in}}%
\pgfpathlineto{\pgfqpoint{6.000226in}{2.735393in}}%
\pgfpathlineto{\pgfqpoint{6.029799in}{2.654369in}}%
\pgfpathlineto{\pgfqpoint{6.059373in}{2.579917in}}%
\pgfpathlineto{\pgfqpoint{6.090452in}{2.507287in}}%
\pgfpathlineto{\pgfqpoint{6.125559in}{2.430879in}}%
\pgfpathlineto{\pgfqpoint{6.163148in}{2.354470in}}%
\pgfpathlineto{\pgfqpoint{6.207240in}{2.270763in}}%
\pgfpathlineto{\pgfqpoint{6.245799in}{2.201653in}}%
\pgfpathlineto{\pgfqpoint{6.295960in}{2.116934in}}%
\pgfpathlineto{\pgfqpoint{6.338412in}{2.048836in}}%
\pgfpathlineto{\pgfqpoint{6.388483in}{1.972427in}}%
\pgfpathlineto{\pgfqpoint{6.443827in}{1.892103in}}%
\pgfpathlineto{\pgfqpoint{6.502973in}{1.810351in}}%
\pgfpathlineto{\pgfqpoint{6.562120in}{1.732280in}}%
\pgfpathlineto{\pgfqpoint{6.621267in}{1.657437in}}%
\pgfpathlineto{\pgfqpoint{6.680414in}{1.585446in}}%
\pgfpathlineto{\pgfqpoint{6.739560in}{1.516005in}}%
\pgfpathlineto{\pgfqpoint{6.739560in}{1.516005in}}%
\pgfusepath{stroke}%
\end{pgfscope}%
\begin{pgfscope}%
\pgfpathrectangle{\pgfqpoint{0.854460in}{0.571603in}}{\pgfqpoint{5.885100in}{5.068436in}}%
\pgfusepath{clip}%
\pgfsetbuttcap%
\pgfsetroundjoin%
\pgfsetlinewidth{1.505625pt}%
\definecolor{currentstroke}{rgb}{0.119483,0.614817,0.537692}%
\pgfsetstrokecolor{currentstroke}%
\pgfsetdash{}{0pt}%
\pgfpathmoveto{\pgfqpoint{1.102532in}{0.571603in}}%
\pgfpathlineto{\pgfqpoint{1.091047in}{0.582378in}}%
\pgfpathlineto{\pgfqpoint{1.075466in}{0.597073in}}%
\pgfpathlineto{\pgfqpoint{1.061474in}{0.610454in}}%
\pgfpathlineto{\pgfqpoint{1.048901in}{0.622542in}}%
\pgfpathlineto{\pgfqpoint{1.031901in}{0.639117in}}%
\pgfpathlineto{\pgfqpoint{1.022827in}{0.648012in}}%
\pgfpathlineto{\pgfqpoint{1.002327in}{0.668391in}}%
\pgfpathlineto{\pgfqpoint{0.997235in}{0.673481in}}%
\pgfpathlineto{\pgfqpoint{0.972754in}{0.698301in}}%
\pgfpathlineto{\pgfqpoint{0.972117in}{0.698951in}}%
\pgfpathlineto{\pgfqpoint{0.947520in}{0.724420in}}%
\pgfpathlineto{\pgfqpoint{0.943181in}{0.728979in}}%
\pgfpathlineto{\pgfqpoint{0.923394in}{0.749890in}}%
\pgfpathlineto{\pgfqpoint{0.913607in}{0.760380in}}%
\pgfpathlineto{\pgfqpoint{0.899719in}{0.775360in}}%
\pgfpathlineto{\pgfqpoint{0.884034in}{0.792520in}}%
\pgfpathlineto{\pgfqpoint{0.876487in}{0.800829in}}%
\pgfpathlineto{\pgfqpoint{0.854460in}{0.825430in}}%
\pgfusepath{stroke}%
\end{pgfscope}%
\begin{pgfscope}%
\pgfpathrectangle{\pgfqpoint{0.854460in}{0.571603in}}{\pgfqpoint{5.885100in}{5.068436in}}%
\pgfusepath{clip}%
\pgfsetbuttcap%
\pgfsetroundjoin%
\pgfsetlinewidth{1.505625pt}%
\definecolor{currentstroke}{rgb}{0.119483,0.614817,0.537692}%
\pgfsetstrokecolor{currentstroke}%
\pgfsetdash{}{0pt}%
\pgfpathmoveto{\pgfqpoint{0.854460in}{4.631264in}}%
\pgfpathlineto{\pgfqpoint{0.868311in}{4.646728in}}%
\pgfpathlineto{\pgfqpoint{0.884034in}{4.664057in}}%
\pgfpathlineto{\pgfqpoint{0.891496in}{4.672197in}}%
\pgfpathlineto{\pgfqpoint{0.913607in}{4.696008in}}%
\pgfpathlineto{\pgfqpoint{0.915163in}{4.697667in}}%
\pgfpathlineto{\pgfqpoint{0.939387in}{4.723136in}}%
\pgfpathlineto{\pgfqpoint{0.943181in}{4.727073in}}%
\pgfpathlineto{\pgfqpoint{0.964138in}{4.748606in}}%
\pgfpathlineto{\pgfqpoint{0.972754in}{4.757347in}}%
\pgfpathlineto{\pgfqpoint{0.989407in}{4.774075in}}%
\pgfpathlineto{\pgfqpoint{1.002327in}{4.786893in}}%
\pgfpathlineto{\pgfqpoint{1.015206in}{4.799545in}}%
\pgfpathlineto{\pgfqpoint{1.031901in}{4.815741in}}%
\pgfpathlineto{\pgfqpoint{1.041552in}{4.825014in}}%
\pgfpathlineto{\pgfqpoint{1.061474in}{4.843919in}}%
\pgfpathlineto{\pgfqpoint{1.068457in}{4.850484in}}%
\pgfpathlineto{\pgfqpoint{1.091047in}{4.871456in}}%
\pgfpathlineto{\pgfqpoint{1.095937in}{4.875953in}}%
\pgfpathlineto{\pgfqpoint{1.120621in}{4.898376in}}%
\pgfpathlineto{\pgfqpoint{1.124005in}{4.901423in}}%
\pgfpathlineto{\pgfqpoint{1.150194in}{4.924706in}}%
\pgfpathlineto{\pgfqpoint{1.152676in}{4.926892in}}%
\pgfpathlineto{\pgfqpoint{1.179767in}{4.950469in}}%
\pgfpathlineto{\pgfqpoint{1.181962in}{4.952362in}}%
\pgfpathlineto{\pgfqpoint{1.209341in}{4.975688in}}%
\pgfpathlineto{\pgfqpoint{1.211878in}{4.977831in}}%
\pgfpathlineto{\pgfqpoint{1.238914in}{5.000386in}}%
\pgfpathlineto{\pgfqpoint{1.242438in}{5.003301in}}%
\pgfpathlineto{\pgfqpoint{1.268488in}{5.024585in}}%
\pgfpathlineto{\pgfqpoint{1.273654in}{5.028770in}}%
\pgfpathlineto{\pgfqpoint{1.298061in}{5.048304in}}%
\pgfpathlineto{\pgfqpoint{1.305539in}{5.054240in}}%
\pgfpathlineto{\pgfqpoint{1.327634in}{5.071564in}}%
\pgfpathlineto{\pgfqpoint{1.338107in}{5.079709in}}%
\pgfpathlineto{\pgfqpoint{1.357208in}{5.094385in}}%
\pgfpathlineto{\pgfqpoint{1.371369in}{5.105179in}}%
\pgfpathlineto{\pgfqpoint{1.386781in}{5.116784in}}%
\pgfpathlineto{\pgfqpoint{1.405338in}{5.130649in}}%
\pgfpathlineto{\pgfqpoint{1.416354in}{5.138779in}}%
\pgfpathlineto{\pgfqpoint{1.440025in}{5.156118in}}%
\pgfpathlineto{\pgfqpoint{1.445928in}{5.160389in}}%
\pgfpathlineto{\pgfqpoint{1.475441in}{5.181588in}}%
\pgfpathlineto{\pgfqpoint{1.475501in}{5.181630in}}%
\pgfpathlineto{\pgfqpoint{1.505074in}{5.202428in}}%
\pgfpathlineto{\pgfqpoint{1.511704in}{5.207057in}}%
\pgfpathlineto{\pgfqpoint{1.534648in}{5.222882in}}%
\pgfpathlineto{\pgfqpoint{1.548731in}{5.232527in}}%
\pgfpathlineto{\pgfqpoint{1.564221in}{5.243007in}}%
\pgfpathlineto{\pgfqpoint{1.586530in}{5.257996in}}%
\pgfpathlineto{\pgfqpoint{1.593795in}{5.262818in}}%
\pgfpathlineto{\pgfqpoint{1.623368in}{5.282308in}}%
\pgfpathlineto{\pgfqpoint{1.625140in}{5.283466in}}%
\pgfpathlineto{\pgfqpoint{1.652941in}{5.301411in}}%
\pgfpathlineto{\pgfqpoint{1.664673in}{5.308935in}}%
\pgfpathlineto{\pgfqpoint{1.682515in}{5.320238in}}%
\pgfpathlineto{\pgfqpoint{1.705018in}{5.334405in}}%
\pgfpathlineto{\pgfqpoint{1.712088in}{5.338801in}}%
\pgfpathlineto{\pgfqpoint{1.741661in}{5.357059in}}%
\pgfpathlineto{\pgfqpoint{1.746257in}{5.359874in}}%
\pgfpathlineto{\pgfqpoint{1.771235in}{5.374987in}}%
\pgfpathlineto{\pgfqpoint{1.788453in}{5.385344in}}%
\pgfpathlineto{\pgfqpoint{1.800808in}{5.392684in}}%
\pgfpathlineto{\pgfqpoint{1.830381in}{5.410150in}}%
\pgfpathlineto{\pgfqpoint{1.831514in}{5.410813in}}%
\pgfpathlineto{\pgfqpoint{1.859955in}{5.427259in}}%
\pgfpathlineto{\pgfqpoint{1.875646in}{5.436283in}}%
\pgfpathlineto{\pgfqpoint{1.889528in}{5.444168in}}%
\pgfpathlineto{\pgfqpoint{1.919102in}{5.460872in}}%
\pgfpathlineto{\pgfqpoint{1.920673in}{5.461752in}}%
\pgfpathlineto{\pgfqpoint{1.948675in}{5.477237in}}%
\pgfpathlineto{\pgfqpoint{1.966821in}{5.487222in}}%
\pgfpathlineto{\pgfqpoint{1.978248in}{5.493432in}}%
\pgfpathlineto{\pgfqpoint{2.007822in}{5.509401in}}%
\pgfpathlineto{\pgfqpoint{2.013962in}{5.512691in}}%
\pgfpathlineto{\pgfqpoint{2.037395in}{5.525093in}}%
\pgfpathlineto{\pgfqpoint{2.062201in}{5.538161in}}%
\pgfpathlineto{\pgfqpoint{2.066968in}{5.540641in}}%
\pgfpathlineto{\pgfqpoint{2.096542in}{5.555903in}}%
\pgfpathlineto{\pgfqpoint{2.111600in}{5.563630in}}%
\pgfpathlineto{\pgfqpoint{2.126115in}{5.570987in}}%
\pgfpathlineto{\pgfqpoint{2.155689in}{5.585890in}}%
\pgfpathlineto{\pgfqpoint{2.162109in}{5.589100in}}%
\pgfpathlineto{\pgfqpoint{2.185262in}{5.600532in}}%
\pgfpathlineto{\pgfqpoint{2.213800in}{5.614570in}}%
\pgfpathlineto{\pgfqpoint{2.214835in}{5.615072in}}%
\pgfpathlineto{\pgfqpoint{2.244409in}{5.629303in}}%
\pgfpathlineto{\pgfqpoint{2.266804in}{5.640039in}}%
\pgfusepath{stroke}%
\end{pgfscope}%
\begin{pgfscope}%
\pgfpathrectangle{\pgfqpoint{0.854460in}{0.571603in}}{\pgfqpoint{5.885100in}{5.068436in}}%
\pgfusepath{clip}%
\pgfsetbuttcap%
\pgfsetroundjoin%
\pgfsetlinewidth{1.505625pt}%
\definecolor{currentstroke}{rgb}{0.119483,0.614817,0.537692}%
\pgfsetstrokecolor{currentstroke}%
\pgfsetdash{}{0pt}%
\pgfpathmoveto{\pgfqpoint{6.407925in}{5.640039in}}%
\pgfpathlineto{\pgfqpoint{6.397530in}{5.589100in}}%
\pgfpathlineto{\pgfqpoint{6.379250in}{5.512691in}}%
\pgfpathlineto{\pgfqpoint{6.355107in}{5.425093in}}%
\pgfpathlineto{\pgfqpoint{6.327588in}{5.334405in}}%
\pgfpathlineto{\pgfqpoint{6.286015in}{5.207057in}}%
\pgfpathlineto{\pgfqpoint{6.118520in}{4.705892in}}%
\pgfpathlineto{\pgfqpoint{6.085166in}{4.595788in}}%
\pgfpathlineto{\pgfqpoint{6.056395in}{4.493910in}}%
\pgfpathlineto{\pgfqpoint{6.029799in}{4.391505in}}%
\pgfpathlineto{\pgfqpoint{6.005982in}{4.290154in}}%
\pgfpathlineto{\pgfqpoint{5.984826in}{4.188276in}}%
\pgfpathlineto{\pgfqpoint{5.970653in}{4.110091in}}%
\pgfpathlineto{\pgfqpoint{5.958767in}{4.035459in}}%
\pgfpathlineto{\pgfqpoint{5.948450in}{3.959050in}}%
\pgfpathlineto{\pgfqpoint{5.940040in}{3.882642in}}%
\pgfpathlineto{\pgfqpoint{5.933545in}{3.806233in}}%
\pgfpathlineto{\pgfqpoint{5.929072in}{3.729825in}}%
\pgfpathlineto{\pgfqpoint{5.926667in}{3.653416in}}%
\pgfpathlineto{\pgfqpoint{5.926370in}{3.577007in}}%
\pgfpathlineto{\pgfqpoint{5.928226in}{3.500599in}}%
\pgfpathlineto{\pgfqpoint{5.932278in}{3.424190in}}%
\pgfpathlineto{\pgfqpoint{5.938571in}{3.347782in}}%
\pgfpathlineto{\pgfqpoint{5.947099in}{3.271373in}}%
\pgfpathlineto{\pgfqpoint{5.957922in}{3.194965in}}%
\pgfpathlineto{\pgfqpoint{5.971105in}{3.118556in}}%
\pgfpathlineto{\pgfqpoint{5.986580in}{3.042147in}}%
\pgfpathlineto{\pgfqpoint{6.004471in}{2.965739in}}%
\pgfpathlineto{\pgfqpoint{6.024737in}{2.889330in}}%
\pgfpathlineto{\pgfqpoint{6.047406in}{2.812922in}}%
\pgfpathlineto{\pgfqpoint{6.072516in}{2.736513in}}%
\pgfpathlineto{\pgfqpoint{6.100075in}{2.660104in}}%
\pgfpathlineto{\pgfqpoint{6.130090in}{2.583696in}}%
\pgfpathlineto{\pgfqpoint{6.162577in}{2.507287in}}%
\pgfpathlineto{\pgfqpoint{6.197550in}{2.430879in}}%
\pgfpathlineto{\pgfqpoint{6.236813in}{2.350997in}}%
\pgfpathlineto{\pgfqpoint{6.274972in}{2.278062in}}%
\pgfpathlineto{\pgfqpoint{6.317437in}{2.201653in}}%
\pgfpathlineto{\pgfqpoint{6.362413in}{2.125244in}}%
\pgfpathlineto{\pgfqpoint{6.414253in}{2.042095in}}%
\pgfpathlineto{\pgfqpoint{6.459898in}{1.972427in}}%
\pgfpathlineto{\pgfqpoint{6.512423in}{1.896019in}}%
\pgfpathlineto{\pgfqpoint{6.567471in}{1.819610in}}%
\pgfpathlineto{\pgfqpoint{6.625032in}{1.743202in}}%
\pgfpathlineto{\pgfqpoint{6.685103in}{1.666793in}}%
\pgfpathlineto{\pgfqpoint{6.739560in}{1.600178in}}%
\pgfpathlineto{\pgfqpoint{6.739560in}{1.600178in}}%
\pgfusepath{stroke}%
\end{pgfscope}%
\begin{pgfscope}%
\pgfpathrectangle{\pgfqpoint{0.854460in}{0.571603in}}{\pgfqpoint{5.885100in}{5.068436in}}%
\pgfusepath{clip}%
\pgfsetbuttcap%
\pgfsetroundjoin%
\pgfsetlinewidth{1.505625pt}%
\definecolor{currentstroke}{rgb}{0.123444,0.636809,0.528763}%
\pgfsetstrokecolor{currentstroke}%
\pgfsetdash{}{0pt}%
\pgfpathmoveto{\pgfqpoint{1.046014in}{0.571603in}}%
\pgfpathlineto{\pgfqpoint{1.031901in}{0.584962in}}%
\pgfpathlineto{\pgfqpoint{1.019173in}{0.597073in}}%
\pgfpathlineto{\pgfqpoint{1.002327in}{0.613327in}}%
\pgfpathlineto{\pgfqpoint{0.992827in}{0.622542in}}%
\pgfpathlineto{\pgfqpoint{0.972754in}{0.642287in}}%
\pgfpathlineto{\pgfqpoint{0.966966in}{0.648012in}}%
\pgfpathlineto{\pgfqpoint{0.943181in}{0.671868in}}%
\pgfpathlineto{\pgfqpoint{0.941581in}{0.673481in}}%
\pgfpathlineto{\pgfqpoint{0.916704in}{0.698951in}}%
\pgfpathlineto{\pgfqpoint{0.913607in}{0.702168in}}%
\pgfpathlineto{\pgfqpoint{0.892313in}{0.724420in}}%
\pgfpathlineto{\pgfqpoint{0.884034in}{0.733194in}}%
\pgfpathlineto{\pgfqpoint{0.868376in}{0.749890in}}%
\pgfpathlineto{\pgfqpoint{0.854460in}{0.764940in}}%
\pgfusepath{stroke}%
\end{pgfscope}%
\begin{pgfscope}%
\pgfpathrectangle{\pgfqpoint{0.854460in}{0.571603in}}{\pgfqpoint{5.885100in}{5.068436in}}%
\pgfusepath{clip}%
\pgfsetbuttcap%
\pgfsetroundjoin%
\pgfsetlinewidth{1.505625pt}%
\definecolor{currentstroke}{rgb}{0.123444,0.636809,0.528763}%
\pgfsetstrokecolor{currentstroke}%
\pgfsetdash{}{0pt}%
\pgfpathmoveto{\pgfqpoint{0.854460in}{4.703637in}}%
\pgfpathlineto{\pgfqpoint{0.872690in}{4.723136in}}%
\pgfpathlineto{\pgfqpoint{0.884034in}{4.735116in}}%
\pgfpathlineto{\pgfqpoint{0.896934in}{4.748606in}}%
\pgfpathlineto{\pgfqpoint{0.913607in}{4.765820in}}%
\pgfpathlineto{\pgfqpoint{0.921682in}{4.774075in}}%
\pgfpathlineto{\pgfqpoint{0.943181in}{4.795779in}}%
\pgfpathlineto{\pgfqpoint{0.946947in}{4.799545in}}%
\pgfpathlineto{\pgfqpoint{0.972744in}{4.825014in}}%
\pgfpathlineto{\pgfqpoint{0.972754in}{4.825024in}}%
\pgfpathlineto{\pgfqpoint{0.999137in}{4.850484in}}%
\pgfpathlineto{\pgfqpoint{1.002327in}{4.853525in}}%
\pgfpathlineto{\pgfqpoint{1.026081in}{4.875953in}}%
\pgfpathlineto{\pgfqpoint{1.031901in}{4.881381in}}%
\pgfpathlineto{\pgfqpoint{1.053590in}{4.901423in}}%
\pgfpathlineto{\pgfqpoint{1.061474in}{4.908618in}}%
\pgfpathlineto{\pgfqpoint{1.081678in}{4.926892in}}%
\pgfpathlineto{\pgfqpoint{1.091047in}{4.935263in}}%
\pgfpathlineto{\pgfqpoint{1.110358in}{4.952362in}}%
\pgfpathlineto{\pgfqpoint{1.120621in}{4.961338in}}%
\pgfpathlineto{\pgfqpoint{1.139643in}{4.977831in}}%
\pgfpathlineto{\pgfqpoint{1.150194in}{4.986868in}}%
\pgfpathlineto{\pgfqpoint{1.169547in}{5.003301in}}%
\pgfpathlineto{\pgfqpoint{1.179767in}{5.011873in}}%
\pgfpathlineto{\pgfqpoint{1.200083in}{5.028770in}}%
\pgfpathlineto{\pgfqpoint{1.209341in}{5.036377in}}%
\pgfpathlineto{\pgfqpoint{1.231263in}{5.054240in}}%
\pgfpathlineto{\pgfqpoint{1.238914in}{5.060399in}}%
\pgfpathlineto{\pgfqpoint{1.263100in}{5.079709in}}%
\pgfpathlineto{\pgfqpoint{1.268488in}{5.083959in}}%
\pgfpathlineto{\pgfqpoint{1.295606in}{5.105179in}}%
\pgfpathlineto{\pgfqpoint{1.298061in}{5.107076in}}%
\pgfpathlineto{\pgfqpoint{1.327634in}{5.129753in}}%
\pgfpathlineto{\pgfqpoint{1.328812in}{5.130649in}}%
\pgfpathlineto{\pgfqpoint{1.357208in}{5.151978in}}%
\pgfpathlineto{\pgfqpoint{1.362761in}{5.156118in}}%
\pgfpathlineto{\pgfqpoint{1.386781in}{5.173807in}}%
\pgfpathlineto{\pgfqpoint{1.397426in}{5.181588in}}%
\pgfpathlineto{\pgfqpoint{1.416354in}{5.195256in}}%
\pgfpathlineto{\pgfqpoint{1.432816in}{5.207057in}}%
\pgfpathlineto{\pgfqpoint{1.445928in}{5.216343in}}%
\pgfpathlineto{\pgfqpoint{1.468943in}{5.232527in}}%
\pgfpathlineto{\pgfqpoint{1.475501in}{5.237083in}}%
\pgfpathlineto{\pgfqpoint{1.505074in}{5.257481in}}%
\pgfpathlineto{\pgfqpoint{1.505828in}{5.257996in}}%
\pgfpathlineto{\pgfqpoint{1.534648in}{5.277467in}}%
\pgfpathlineto{\pgfqpoint{1.543587in}{5.283466in}}%
\pgfpathlineto{\pgfqpoint{1.564221in}{5.297145in}}%
\pgfpathlineto{\pgfqpoint{1.582123in}{5.308935in}}%
\pgfpathlineto{\pgfqpoint{1.593795in}{5.316529in}}%
\pgfpathlineto{\pgfqpoint{1.621445in}{5.334405in}}%
\pgfpathlineto{\pgfqpoint{1.623368in}{5.335633in}}%
\pgfpathlineto{\pgfqpoint{1.652941in}{5.354364in}}%
\pgfpathlineto{\pgfqpoint{1.661700in}{5.359874in}}%
\pgfpathlineto{\pgfqpoint{1.682515in}{5.372811in}}%
\pgfpathlineto{\pgfqpoint{1.702799in}{5.385344in}}%
\pgfpathlineto{\pgfqpoint{1.712088in}{5.391013in}}%
\pgfpathlineto{\pgfqpoint{1.741661in}{5.408944in}}%
\pgfpathlineto{\pgfqpoint{1.744769in}{5.410813in}}%
\pgfpathlineto{\pgfqpoint{1.771235in}{5.426538in}}%
\pgfpathlineto{\pgfqpoint{1.787729in}{5.436283in}}%
\pgfpathlineto{\pgfqpoint{1.800808in}{5.443916in}}%
\pgfpathlineto{\pgfqpoint{1.830381in}{5.461079in}}%
\pgfpathlineto{\pgfqpoint{1.831551in}{5.461752in}}%
\pgfpathlineto{\pgfqpoint{1.859955in}{5.477895in}}%
\pgfpathlineto{\pgfqpoint{1.876450in}{5.487222in}}%
\pgfpathlineto{\pgfqpoint{1.889528in}{5.494526in}}%
\pgfpathlineto{\pgfqpoint{1.919102in}{5.510949in}}%
\pgfpathlineto{\pgfqpoint{1.922265in}{5.512691in}}%
\pgfpathlineto{\pgfqpoint{1.948675in}{5.527059in}}%
\pgfpathlineto{\pgfqpoint{1.969178in}{5.538161in}}%
\pgfpathlineto{\pgfqpoint{1.978248in}{5.543012in}}%
\pgfpathlineto{\pgfqpoint{2.007822in}{5.558722in}}%
\pgfpathlineto{\pgfqpoint{2.017125in}{5.563630in}}%
\pgfpathlineto{\pgfqpoint{2.037395in}{5.574194in}}%
\pgfpathlineto{\pgfqpoint{2.066119in}{5.589100in}}%
\pgfpathlineto{\pgfqpoint{2.066968in}{5.589535in}}%
\pgfpathlineto{\pgfqpoint{2.096542in}{5.604558in}}%
\pgfpathlineto{\pgfqpoint{2.116334in}{5.614570in}}%
\pgfpathlineto{\pgfqpoint{2.126115in}{5.619456in}}%
\pgfpathlineto{\pgfqpoint{2.155689in}{5.634130in}}%
\pgfpathlineto{\pgfqpoint{2.167670in}{5.640039in}}%
\pgfusepath{stroke}%
\end{pgfscope}%
\begin{pgfscope}%
\pgfpathrectangle{\pgfqpoint{0.854460in}{0.571603in}}{\pgfqpoint{5.885100in}{5.068436in}}%
\pgfusepath{clip}%
\pgfsetbuttcap%
\pgfsetroundjoin%
\pgfsetlinewidth{1.505625pt}%
\definecolor{currentstroke}{rgb}{0.123444,0.636809,0.528763}%
\pgfsetstrokecolor{currentstroke}%
\pgfsetdash{}{0pt}%
\pgfpathmoveto{\pgfqpoint{6.522648in}{5.640039in}}%
\pgfpathlineto{\pgfqpoint{6.502284in}{5.563630in}}%
\pgfpathlineto{\pgfqpoint{6.473400in}{5.468523in}}%
\pgfpathlineto{\pgfqpoint{6.445907in}{5.385344in}}%
\pgfpathlineto{\pgfqpoint{6.401092in}{5.257996in}}%
\pgfpathlineto{\pgfqpoint{6.227004in}{4.774075in}}%
\pgfpathlineto{\pgfqpoint{6.185146in}{4.646728in}}%
\pgfpathlineto{\pgfqpoint{6.148093in}{4.525059in}}%
\pgfpathlineto{\pgfqpoint{6.124906in}{4.442971in}}%
\pgfpathlineto{\pgfqpoint{6.098477in}{4.341093in}}%
\pgfpathlineto{\pgfqpoint{6.074830in}{4.239215in}}%
\pgfpathlineto{\pgfqpoint{6.059075in}{4.162807in}}%
\pgfpathlineto{\pgfqpoint{6.045015in}{4.086398in}}%
\pgfpathlineto{\pgfqpoint{6.032833in}{4.009989in}}%
\pgfpathlineto{\pgfqpoint{6.022511in}{3.933581in}}%
\pgfpathlineto{\pgfqpoint{6.014146in}{3.857172in}}%
\pgfpathlineto{\pgfqpoint{6.007806in}{3.780764in}}%
\pgfpathlineto{\pgfqpoint{6.003531in}{3.704355in}}%
\pgfpathlineto{\pgfqpoint{6.001361in}{3.627946in}}%
\pgfpathlineto{\pgfqpoint{6.001335in}{3.551538in}}%
\pgfpathlineto{\pgfqpoint{6.003495in}{3.475129in}}%
\pgfpathlineto{\pgfqpoint{6.007881in}{3.398721in}}%
\pgfpathlineto{\pgfqpoint{6.014537in}{3.322312in}}%
\pgfpathlineto{\pgfqpoint{6.023506in}{3.245904in}}%
\pgfpathlineto{\pgfqpoint{6.034794in}{3.169495in}}%
\pgfpathlineto{\pgfqpoint{6.048420in}{3.093086in}}%
\pgfpathlineto{\pgfqpoint{6.064444in}{3.016678in}}%
\pgfpathlineto{\pgfqpoint{6.082854in}{2.940269in}}%
\pgfpathlineto{\pgfqpoint{6.103674in}{2.863861in}}%
\pgfpathlineto{\pgfqpoint{6.126947in}{2.787452in}}%
\pgfpathlineto{\pgfqpoint{6.152676in}{2.711044in}}%
\pgfpathlineto{\pgfqpoint{6.180867in}{2.634635in}}%
\pgfpathlineto{\pgfqpoint{6.211533in}{2.558226in}}%
\pgfpathlineto{\pgfqpoint{6.244684in}{2.481818in}}%
\pgfpathlineto{\pgfqpoint{6.280339in}{2.405409in}}%
\pgfpathlineto{\pgfqpoint{6.318519in}{2.329001in}}%
\pgfpathlineto{\pgfqpoint{6.359214in}{2.252592in}}%
\pgfpathlineto{\pgfqpoint{6.402411in}{2.176183in}}%
\pgfpathlineto{\pgfqpoint{6.448163in}{2.099775in}}%
\pgfpathlineto{\pgfqpoint{6.502973in}{2.013371in}}%
\pgfpathlineto{\pgfqpoint{6.547219in}{1.946958in}}%
\pgfpathlineto{\pgfqpoint{6.600560in}{1.870549in}}%
\pgfpathlineto{\pgfqpoint{6.656437in}{1.794141in}}%
\pgfpathlineto{\pgfqpoint{6.714843in}{1.717732in}}%
\pgfpathlineto{\pgfqpoint{6.739560in}{1.686393in}}%
\pgfpathlineto{\pgfqpoint{6.739560in}{1.686393in}}%
\pgfusepath{stroke}%
\end{pgfscope}%
\begin{pgfscope}%
\pgfpathrectangle{\pgfqpoint{0.854460in}{0.571603in}}{\pgfqpoint{5.885100in}{5.068436in}}%
\pgfusepath{clip}%
\pgfsetbuttcap%
\pgfsetroundjoin%
\pgfsetlinewidth{1.505625pt}%
\definecolor{currentstroke}{rgb}{0.132268,0.655014,0.519661}%
\pgfsetstrokecolor{currentstroke}%
\pgfsetdash{}{0pt}%
\pgfpathmoveto{\pgfqpoint{0.990838in}{0.571603in}}%
\pgfpathlineto{\pgfqpoint{0.972754in}{0.588875in}}%
\pgfpathlineto{\pgfqpoint{0.964215in}{0.597073in}}%
\pgfpathlineto{\pgfqpoint{0.943181in}{0.617549in}}%
\pgfpathlineto{\pgfqpoint{0.938079in}{0.622542in}}%
\pgfpathlineto{\pgfqpoint{0.913607in}{0.646830in}}%
\pgfpathlineto{\pgfqpoint{0.912423in}{0.648012in}}%
\pgfpathlineto{\pgfqpoint{0.887278in}{0.673481in}}%
\pgfpathlineto{\pgfqpoint{0.884034in}{0.676815in}}%
\pgfpathlineto{\pgfqpoint{0.862616in}{0.698951in}}%
\pgfpathlineto{\pgfqpoint{0.854460in}{0.707499in}}%
\pgfusepath{stroke}%
\end{pgfscope}%
\begin{pgfscope}%
\pgfpathrectangle{\pgfqpoint{0.854460in}{0.571603in}}{\pgfqpoint{5.885100in}{5.068436in}}%
\pgfusepath{clip}%
\pgfsetbuttcap%
\pgfsetroundjoin%
\pgfsetlinewidth{1.505625pt}%
\definecolor{currentstroke}{rgb}{0.132268,0.655014,0.519661}%
\pgfsetstrokecolor{currentstroke}%
\pgfsetdash{}{0pt}%
\pgfpathmoveto{\pgfqpoint{0.854460in}{4.772166in}}%
\pgfpathlineto{\pgfqpoint{0.856296in}{4.774075in}}%
\pgfpathlineto{\pgfqpoint{0.881110in}{4.799545in}}%
\pgfpathlineto{\pgfqpoint{0.884034in}{4.802507in}}%
\pgfpathlineto{\pgfqpoint{0.906462in}{4.825014in}}%
\pgfpathlineto{\pgfqpoint{0.913607in}{4.832095in}}%
\pgfpathlineto{\pgfqpoint{0.932338in}{4.850484in}}%
\pgfpathlineto{\pgfqpoint{0.943181in}{4.860996in}}%
\pgfpathlineto{\pgfqpoint{0.958751in}{4.875953in}}%
\pgfpathlineto{\pgfqpoint{0.972754in}{4.889237in}}%
\pgfpathlineto{\pgfqpoint{0.985716in}{4.901423in}}%
\pgfpathlineto{\pgfqpoint{1.002327in}{4.916845in}}%
\pgfpathlineto{\pgfqpoint{1.013246in}{4.926892in}}%
\pgfpathlineto{\pgfqpoint{1.031901in}{4.943846in}}%
\pgfpathlineto{\pgfqpoint{1.041354in}{4.952362in}}%
\pgfpathlineto{\pgfqpoint{1.061474in}{4.970264in}}%
\pgfpathlineto{\pgfqpoint{1.070054in}{4.977831in}}%
\pgfpathlineto{\pgfqpoint{1.091047in}{4.996122in}}%
\pgfpathlineto{\pgfqpoint{1.099358in}{5.003301in}}%
\pgfpathlineto{\pgfqpoint{1.120621in}{5.021443in}}%
\pgfpathlineto{\pgfqpoint{1.129280in}{5.028770in}}%
\pgfpathlineto{\pgfqpoint{1.150194in}{5.046250in}}%
\pgfpathlineto{\pgfqpoint{1.159833in}{5.054240in}}%
\pgfpathlineto{\pgfqpoint{1.179767in}{5.070563in}}%
\pgfpathlineto{\pgfqpoint{1.191029in}{5.079709in}}%
\pgfpathlineto{\pgfqpoint{1.209341in}{5.094402in}}%
\pgfpathlineto{\pgfqpoint{1.222880in}{5.105179in}}%
\pgfpathlineto{\pgfqpoint{1.238914in}{5.117787in}}%
\pgfpathlineto{\pgfqpoint{1.255398in}{5.130649in}}%
\pgfpathlineto{\pgfqpoint{1.268488in}{5.140737in}}%
\pgfpathlineto{\pgfqpoint{1.288595in}{5.156118in}}%
\pgfpathlineto{\pgfqpoint{1.298061in}{5.163271in}}%
\pgfpathlineto{\pgfqpoint{1.322483in}{5.181588in}}%
\pgfpathlineto{\pgfqpoint{1.327634in}{5.185404in}}%
\pgfpathlineto{\pgfqpoint{1.357072in}{5.207057in}}%
\pgfpathlineto{\pgfqpoint{1.357208in}{5.207156in}}%
\pgfpathlineto{\pgfqpoint{1.386781in}{5.228464in}}%
\pgfpathlineto{\pgfqpoint{1.392460in}{5.232527in}}%
\pgfpathlineto{\pgfqpoint{1.416354in}{5.249413in}}%
\pgfpathlineto{\pgfqpoint{1.428584in}{5.257996in}}%
\pgfpathlineto{\pgfqpoint{1.445928in}{5.270021in}}%
\pgfpathlineto{\pgfqpoint{1.465450in}{5.283466in}}%
\pgfpathlineto{\pgfqpoint{1.475501in}{5.290303in}}%
\pgfpathlineto{\pgfqpoint{1.503069in}{5.308935in}}%
\pgfpathlineto{\pgfqpoint{1.505074in}{5.310274in}}%
\pgfpathlineto{\pgfqpoint{1.534648in}{5.329860in}}%
\pgfpathlineto{\pgfqpoint{1.541556in}{5.334405in}}%
\pgfpathlineto{\pgfqpoint{1.564221in}{5.349132in}}%
\pgfpathlineto{\pgfqpoint{1.580856in}{5.359874in}}%
\pgfpathlineto{\pgfqpoint{1.593795in}{5.368128in}}%
\pgfpathlineto{\pgfqpoint{1.620945in}{5.385344in}}%
\pgfpathlineto{\pgfqpoint{1.623368in}{5.386862in}}%
\pgfpathlineto{\pgfqpoint{1.652941in}{5.405239in}}%
\pgfpathlineto{\pgfqpoint{1.661970in}{5.410813in}}%
\pgfpathlineto{\pgfqpoint{1.682515in}{5.423344in}}%
\pgfpathlineto{\pgfqpoint{1.703848in}{5.436283in}}%
\pgfpathlineto{\pgfqpoint{1.712088in}{5.441220in}}%
\pgfpathlineto{\pgfqpoint{1.741661in}{5.458821in}}%
\pgfpathlineto{\pgfqpoint{1.746624in}{5.461752in}}%
\pgfpathlineto{\pgfqpoint{1.771235in}{5.476114in}}%
\pgfpathlineto{\pgfqpoint{1.790369in}{5.487222in}}%
\pgfpathlineto{\pgfqpoint{1.800808in}{5.493208in}}%
\pgfpathlineto{\pgfqpoint{1.830381in}{5.510063in}}%
\pgfpathlineto{\pgfqpoint{1.835028in}{5.512691in}}%
\pgfpathlineto{\pgfqpoint{1.859955in}{5.526618in}}%
\pgfpathlineto{\pgfqpoint{1.880714in}{5.538161in}}%
\pgfpathlineto{\pgfqpoint{1.889528in}{5.543002in}}%
\pgfpathlineto{\pgfqpoint{1.919102in}{5.559138in}}%
\pgfpathlineto{\pgfqpoint{1.927391in}{5.563630in}}%
\pgfpathlineto{\pgfqpoint{1.948675in}{5.575024in}}%
\pgfpathlineto{\pgfqpoint{1.975084in}{5.589100in}}%
\pgfpathlineto{\pgfqpoint{1.978248in}{5.590766in}}%
\pgfpathlineto{\pgfqpoint{2.007822in}{5.606209in}}%
\pgfpathlineto{\pgfqpoint{2.023909in}{5.614570in}}%
\pgfpathlineto{\pgfqpoint{2.037395in}{5.621491in}}%
\pgfpathlineto{\pgfqpoint{2.066968in}{5.636586in}}%
\pgfpathlineto{\pgfqpoint{2.073783in}{5.640039in}}%
\pgfusepath{stroke}%
\end{pgfscope}%
\begin{pgfscope}%
\pgfpathrectangle{\pgfqpoint{0.854460in}{0.571603in}}{\pgfqpoint{5.885100in}{5.068436in}}%
\pgfusepath{clip}%
\pgfsetbuttcap%
\pgfsetroundjoin%
\pgfsetlinewidth{1.505625pt}%
\definecolor{currentstroke}{rgb}{0.132268,0.655014,0.519661}%
\pgfsetstrokecolor{currentstroke}%
\pgfsetdash{}{0pt}%
\pgfpathmoveto{\pgfqpoint{6.632130in}{5.640039in}}%
\pgfpathlineto{\pgfqpoint{6.608014in}{5.563630in}}%
\pgfpathlineto{\pgfqpoint{6.581840in}{5.487222in}}%
\pgfpathlineto{\pgfqpoint{6.544640in}{5.385344in}}%
\pgfpathlineto{\pgfqpoint{6.495906in}{5.257996in}}%
\pgfpathlineto{\pgfqpoint{6.355107in}{4.893584in}}%
\pgfpathlineto{\pgfqpoint{6.311512in}{4.774075in}}%
\pgfpathlineto{\pgfqpoint{6.267783in}{4.646728in}}%
\pgfpathlineto{\pgfqpoint{6.235165in}{4.544849in}}%
\pgfpathlineto{\pgfqpoint{6.204968in}{4.442971in}}%
\pgfpathlineto{\pgfqpoint{6.177445in}{4.341093in}}%
\pgfpathlineto{\pgfqpoint{6.158647in}{4.264685in}}%
\pgfpathlineto{\pgfqpoint{6.136280in}{4.162807in}}%
\pgfpathlineto{\pgfqpoint{6.118520in}{4.068753in}}%
\pgfpathlineto{\pgfqpoint{6.108787in}{4.009989in}}%
\pgfpathlineto{\pgfqpoint{6.097906in}{3.933581in}}%
\pgfpathlineto{\pgfqpoint{6.088946in}{3.856225in}}%
\pgfpathlineto{\pgfqpoint{6.082168in}{3.780764in}}%
\pgfpathlineto{\pgfqpoint{6.077403in}{3.704355in}}%
\pgfpathlineto{\pgfqpoint{6.074781in}{3.627946in}}%
\pgfpathlineto{\pgfqpoint{6.074341in}{3.551538in}}%
\pgfpathlineto{\pgfqpoint{6.076119in}{3.475129in}}%
\pgfpathlineto{\pgfqpoint{6.080156in}{3.398721in}}%
\pgfpathlineto{\pgfqpoint{6.086492in}{3.322312in}}%
\pgfpathlineto{\pgfqpoint{6.095120in}{3.245904in}}%
\pgfpathlineto{\pgfqpoint{6.106096in}{3.169495in}}%
\pgfpathlineto{\pgfqpoint{6.119477in}{3.093086in}}%
\pgfpathlineto{\pgfqpoint{6.135204in}{3.016678in}}%
\pgfpathlineto{\pgfqpoint{6.153385in}{2.940269in}}%
\pgfpathlineto{\pgfqpoint{6.173994in}{2.863861in}}%
\pgfpathlineto{\pgfqpoint{6.197040in}{2.787452in}}%
\pgfpathlineto{\pgfqpoint{6.222566in}{2.711044in}}%
\pgfpathlineto{\pgfqpoint{6.250578in}{2.634635in}}%
\pgfpathlineto{\pgfqpoint{6.281087in}{2.558226in}}%
\pgfpathlineto{\pgfqpoint{6.314104in}{2.481818in}}%
\pgfpathlineto{\pgfqpoint{6.355107in}{2.394229in}}%
\pgfpathlineto{\pgfqpoint{6.387712in}{2.329001in}}%
\pgfpathlineto{\pgfqpoint{6.428285in}{2.252592in}}%
\pgfpathlineto{\pgfqpoint{6.473400in}{2.172840in}}%
\pgfpathlineto{\pgfqpoint{6.517084in}{2.099775in}}%
\pgfpathlineto{\pgfqpoint{6.565318in}{2.023366in}}%
\pgfpathlineto{\pgfqpoint{6.621267in}{1.939412in}}%
\pgfpathlineto{\pgfqpoint{6.669382in}{1.870549in}}%
\pgfpathlineto{\pgfqpoint{6.725245in}{1.794141in}}%
\pgfpathlineto{\pgfqpoint{6.739560in}{1.775164in}}%
\pgfpathlineto{\pgfqpoint{6.739560in}{1.775164in}}%
\pgfusepath{stroke}%
\end{pgfscope}%
\begin{pgfscope}%
\pgfpathrectangle{\pgfqpoint{0.854460in}{0.571603in}}{\pgfqpoint{5.885100in}{5.068436in}}%
\pgfusepath{clip}%
\pgfsetbuttcap%
\pgfsetroundjoin%
\pgfsetlinewidth{1.505625pt}%
\definecolor{currentstroke}{rgb}{0.146616,0.673050,0.508936}%
\pgfsetstrokecolor{currentstroke}%
\pgfsetdash{}{0pt}%
\pgfpathmoveto{\pgfqpoint{0.936926in}{0.571603in}}%
\pgfpathlineto{\pgfqpoint{0.913607in}{0.594074in}}%
\pgfpathlineto{\pgfqpoint{0.910512in}{0.597073in}}%
\pgfpathlineto{\pgfqpoint{0.884585in}{0.622542in}}%
\pgfpathlineto{\pgfqpoint{0.884034in}{0.623092in}}%
\pgfpathlineto{\pgfqpoint{0.859178in}{0.648012in}}%
\pgfpathlineto{\pgfqpoint{0.854460in}{0.652807in}}%
\pgfusepath{stroke}%
\end{pgfscope}%
\begin{pgfscope}%
\pgfpathrectangle{\pgfqpoint{0.854460in}{0.571603in}}{\pgfqpoint{5.885100in}{5.068436in}}%
\pgfusepath{clip}%
\pgfsetbuttcap%
\pgfsetroundjoin%
\pgfsetlinewidth{1.505625pt}%
\definecolor{currentstroke}{rgb}{0.146616,0.673050,0.508936}%
\pgfsetstrokecolor{currentstroke}%
\pgfsetdash{}{0pt}%
\pgfpathmoveto{\pgfqpoint{0.854460in}{4.837189in}}%
\pgfpathlineto{\pgfqpoint{0.867775in}{4.850484in}}%
\pgfpathlineto{\pgfqpoint{0.884034in}{4.866516in}}%
\pgfpathlineto{\pgfqpoint{0.893694in}{4.875953in}}%
\pgfpathlineto{\pgfqpoint{0.913607in}{4.895167in}}%
\pgfpathlineto{\pgfqpoint{0.920150in}{4.901423in}}%
\pgfpathlineto{\pgfqpoint{0.943181in}{4.923170in}}%
\pgfpathlineto{\pgfqpoint{0.947158in}{4.926892in}}%
\pgfpathlineto{\pgfqpoint{0.972754in}{4.950551in}}%
\pgfpathlineto{\pgfqpoint{0.974730in}{4.952362in}}%
\pgfpathlineto{\pgfqpoint{1.002327in}{4.977335in}}%
\pgfpathlineto{\pgfqpoint{1.002881in}{4.977831in}}%
\pgfpathlineto{\pgfqpoint{1.031627in}{5.003301in}}%
\pgfpathlineto{\pgfqpoint{1.031901in}{5.003540in}}%
\pgfpathlineto{\pgfqpoint{1.060976in}{5.028770in}}%
\pgfpathlineto{\pgfqpoint{1.061474in}{5.029197in}}%
\pgfpathlineto{\pgfqpoint{1.090933in}{5.054240in}}%
\pgfpathlineto{\pgfqpoint{1.091047in}{5.054336in}}%
\pgfpathlineto{\pgfqpoint{1.120621in}{5.078963in}}%
\pgfpathlineto{\pgfqpoint{1.121524in}{5.079709in}}%
\pgfpathlineto{\pgfqpoint{1.150194in}{5.103104in}}%
\pgfpathlineto{\pgfqpoint{1.152758in}{5.105179in}}%
\pgfpathlineto{\pgfqpoint{1.179767in}{5.126778in}}%
\pgfpathlineto{\pgfqpoint{1.184645in}{5.130649in}}%
\pgfpathlineto{\pgfqpoint{1.209341in}{5.150006in}}%
\pgfpathlineto{\pgfqpoint{1.217198in}{5.156118in}}%
\pgfpathlineto{\pgfqpoint{1.238914in}{5.172805in}}%
\pgfpathlineto{\pgfqpoint{1.250428in}{5.181588in}}%
\pgfpathlineto{\pgfqpoint{1.268488in}{5.195195in}}%
\pgfpathlineto{\pgfqpoint{1.284346in}{5.207057in}}%
\pgfpathlineto{\pgfqpoint{1.298061in}{5.217191in}}%
\pgfpathlineto{\pgfqpoint{1.318963in}{5.232527in}}%
\pgfpathlineto{\pgfqpoint{1.327634in}{5.238811in}}%
\pgfpathlineto{\pgfqpoint{1.354289in}{5.257996in}}%
\pgfpathlineto{\pgfqpoint{1.357208in}{5.260072in}}%
\pgfpathlineto{\pgfqpoint{1.386781in}{5.280939in}}%
\pgfpathlineto{\pgfqpoint{1.390388in}{5.283466in}}%
\pgfpathlineto{\pgfqpoint{1.416354in}{5.301431in}}%
\pgfpathlineto{\pgfqpoint{1.427272in}{5.308935in}}%
\pgfpathlineto{\pgfqpoint{1.445928in}{5.321602in}}%
\pgfpathlineto{\pgfqpoint{1.464904in}{5.334405in}}%
\pgfpathlineto{\pgfqpoint{1.475501in}{5.341467in}}%
\pgfpathlineto{\pgfqpoint{1.503293in}{5.359874in}}%
\pgfpathlineto{\pgfqpoint{1.505074in}{5.361039in}}%
\pgfpathlineto{\pgfqpoint{1.534648in}{5.380236in}}%
\pgfpathlineto{\pgfqpoint{1.542568in}{5.385344in}}%
\pgfpathlineto{\pgfqpoint{1.564221in}{5.399138in}}%
\pgfpathlineto{\pgfqpoint{1.582655in}{5.410813in}}%
\pgfpathlineto{\pgfqpoint{1.593795in}{5.417783in}}%
\pgfpathlineto{\pgfqpoint{1.623368in}{5.436180in}}%
\pgfpathlineto{\pgfqpoint{1.623534in}{5.436283in}}%
\pgfpathlineto{\pgfqpoint{1.652941in}{5.454204in}}%
\pgfpathlineto{\pgfqpoint{1.665396in}{5.461752in}}%
\pgfpathlineto{\pgfqpoint{1.682515in}{5.472001in}}%
\pgfpathlineto{\pgfqpoint{1.708073in}{5.487222in}}%
\pgfpathlineto{\pgfqpoint{1.712088in}{5.489584in}}%
\pgfpathlineto{\pgfqpoint{1.741661in}{5.506852in}}%
\pgfpathlineto{\pgfqpoint{1.751722in}{5.512691in}}%
\pgfpathlineto{\pgfqpoint{1.771235in}{5.523878in}}%
\pgfpathlineto{\pgfqpoint{1.796270in}{5.538161in}}%
\pgfpathlineto{\pgfqpoint{1.800808in}{5.540718in}}%
\pgfpathlineto{\pgfqpoint{1.830381in}{5.557261in}}%
\pgfpathlineto{\pgfqpoint{1.841833in}{5.563630in}}%
\pgfpathlineto{\pgfqpoint{1.859955in}{5.573586in}}%
\pgfpathlineto{\pgfqpoint{1.888320in}{5.589100in}}%
\pgfpathlineto{\pgfqpoint{1.889528in}{5.589753in}}%
\pgfpathlineto{\pgfqpoint{1.919102in}{5.605598in}}%
\pgfpathlineto{\pgfqpoint{1.935922in}{5.614570in}}%
\pgfpathlineto{\pgfqpoint{1.948675in}{5.621288in}}%
\pgfpathlineto{\pgfqpoint{1.978248in}{5.636783in}}%
\pgfpathlineto{\pgfqpoint{1.984508in}{5.640039in}}%
\pgfusepath{stroke}%
\end{pgfscope}%
\begin{pgfscope}%
\pgfpathrectangle{\pgfqpoint{0.854460in}{0.571603in}}{\pgfqpoint{5.885100in}{5.068436in}}%
\pgfusepath{clip}%
\pgfsetbuttcap%
\pgfsetroundjoin%
\pgfsetlinewidth{1.505625pt}%
\definecolor{currentstroke}{rgb}{0.146616,0.673050,0.508936}%
\pgfsetstrokecolor{currentstroke}%
\pgfsetdash{}{0pt}%
\pgfpathmoveto{\pgfqpoint{6.737069in}{5.640039in}}%
\pgfpathlineto{\pgfqpoint{6.709679in}{5.563630in}}%
\pgfpathlineto{\pgfqpoint{6.670588in}{5.461752in}}%
\pgfpathlineto{\pgfqpoint{6.619235in}{5.334405in}}%
\pgfpathlineto{\pgfqpoint{6.452348in}{4.926892in}}%
\pgfpathlineto{\pgfqpoint{6.403327in}{4.799545in}}%
\pgfpathlineto{\pgfqpoint{6.366140in}{4.697667in}}%
\pgfpathlineto{\pgfqpoint{6.325533in}{4.578852in}}%
\pgfpathlineto{\pgfqpoint{6.298482in}{4.493910in}}%
\pgfpathlineto{\pgfqpoint{6.268507in}{4.392032in}}%
\pgfpathlineto{\pgfqpoint{6.241394in}{4.290154in}}%
\pgfpathlineto{\pgfqpoint{6.217340in}{4.188276in}}%
\pgfpathlineto{\pgfqpoint{6.201432in}{4.111867in}}%
\pgfpathlineto{\pgfqpoint{6.187395in}{4.035459in}}%
\pgfpathlineto{\pgfqpoint{6.175332in}{3.959050in}}%
\pgfpathlineto{\pgfqpoint{6.165235in}{3.882642in}}%
\pgfpathlineto{\pgfqpoint{6.157219in}{3.806233in}}%
\pgfpathlineto{\pgfqpoint{6.151320in}{3.729825in}}%
\pgfpathlineto{\pgfqpoint{6.147566in}{3.653416in}}%
\pgfpathlineto{\pgfqpoint{6.145989in}{3.577007in}}%
\pgfpathlineto{\pgfqpoint{6.146651in}{3.500599in}}%
\pgfpathlineto{\pgfqpoint{6.149576in}{3.424190in}}%
\pgfpathlineto{\pgfqpoint{6.154786in}{3.347782in}}%
\pgfpathlineto{\pgfqpoint{6.162331in}{3.271373in}}%
\pgfpathlineto{\pgfqpoint{6.172253in}{3.194965in}}%
\pgfpathlineto{\pgfqpoint{6.184543in}{3.118556in}}%
\pgfpathlineto{\pgfqpoint{6.199240in}{3.042147in}}%
\pgfpathlineto{\pgfqpoint{6.216364in}{2.965739in}}%
\pgfpathlineto{\pgfqpoint{6.236813in}{2.886264in}}%
\pgfpathlineto{\pgfqpoint{6.257977in}{2.812922in}}%
\pgfpathlineto{\pgfqpoint{6.282490in}{2.736513in}}%
\pgfpathlineto{\pgfqpoint{6.309503in}{2.660104in}}%
\pgfpathlineto{\pgfqpoint{6.339024in}{2.583696in}}%
\pgfpathlineto{\pgfqpoint{6.371063in}{2.507287in}}%
\pgfpathlineto{\pgfqpoint{6.405636in}{2.430879in}}%
\pgfpathlineto{\pgfqpoint{6.443827in}{2.352375in}}%
\pgfpathlineto{\pgfqpoint{6.482390in}{2.278062in}}%
\pgfpathlineto{\pgfqpoint{6.524592in}{2.201653in}}%
\pgfpathlineto{\pgfqpoint{6.569350in}{2.125244in}}%
\pgfpathlineto{\pgfqpoint{6.621267in}{2.041685in}}%
\pgfpathlineto{\pgfqpoint{6.666536in}{1.972427in}}%
\pgfpathlineto{\pgfqpoint{6.718981in}{1.896019in}}%
\pgfpathlineto{\pgfqpoint{6.739560in}{1.867062in}}%
\pgfpathlineto{\pgfqpoint{6.739560in}{1.867062in}}%
\pgfusepath{stroke}%
\end{pgfscope}%
\begin{pgfscope}%
\pgfpathrectangle{\pgfqpoint{0.854460in}{0.571603in}}{\pgfqpoint{5.885100in}{5.068436in}}%
\pgfusepath{clip}%
\pgfsetbuttcap%
\pgfsetroundjoin%
\pgfsetlinewidth{1.505625pt}%
\definecolor{currentstroke}{rgb}{0.170948,0.694384,0.493803}%
\pgfsetstrokecolor{currentstroke}%
\pgfsetdash{}{0pt}%
\pgfpathmoveto{\pgfqpoint{0.884205in}{0.571603in}}%
\pgfpathlineto{\pgfqpoint{0.884034in}{0.571769in}}%
\pgfpathlineto{\pgfqpoint{0.858034in}{0.597073in}}%
\pgfpathlineto{\pgfqpoint{0.854460in}{0.600599in}}%
\pgfusepath{stroke}%
\end{pgfscope}%
\begin{pgfscope}%
\pgfpathrectangle{\pgfqpoint{0.854460in}{0.571603in}}{\pgfqpoint{5.885100in}{5.068436in}}%
\pgfusepath{clip}%
\pgfsetbuttcap%
\pgfsetroundjoin%
\pgfsetlinewidth{1.505625pt}%
\definecolor{currentstroke}{rgb}{0.170948,0.694384,0.493803}%
\pgfsetstrokecolor{currentstroke}%
\pgfsetdash{}{0pt}%
\pgfpathmoveto{\pgfqpoint{0.854460in}{4.899220in}}%
\pgfpathlineto{\pgfqpoint{0.856726in}{4.901423in}}%
\pgfpathlineto{\pgfqpoint{0.883257in}{4.926892in}}%
\pgfpathlineto{\pgfqpoint{0.884034in}{4.927628in}}%
\pgfpathlineto{\pgfqpoint{0.910365in}{4.952362in}}%
\pgfpathlineto{\pgfqpoint{0.913607in}{4.955369in}}%
\pgfpathlineto{\pgfqpoint{0.938030in}{4.977831in}}%
\pgfpathlineto{\pgfqpoint{0.943181in}{4.982510in}}%
\pgfpathlineto{\pgfqpoint{0.966264in}{5.003301in}}%
\pgfpathlineto{\pgfqpoint{0.972754in}{5.009075in}}%
\pgfpathlineto{\pgfqpoint{0.995080in}{5.028770in}}%
\pgfpathlineto{\pgfqpoint{1.002327in}{5.035086in}}%
\pgfpathlineto{\pgfqpoint{1.024490in}{5.054240in}}%
\pgfpathlineto{\pgfqpoint{1.031901in}{5.060566in}}%
\pgfpathlineto{\pgfqpoint{1.054507in}{5.079709in}}%
\pgfpathlineto{\pgfqpoint{1.061474in}{5.085537in}}%
\pgfpathlineto{\pgfqpoint{1.085143in}{5.105179in}}%
\pgfpathlineto{\pgfqpoint{1.091047in}{5.110019in}}%
\pgfpathlineto{\pgfqpoint{1.116410in}{5.130649in}}%
\pgfpathlineto{\pgfqpoint{1.120621in}{5.134032in}}%
\pgfpathlineto{\pgfqpoint{1.148319in}{5.156118in}}%
\pgfpathlineto{\pgfqpoint{1.150194in}{5.157595in}}%
\pgfpathlineto{\pgfqpoint{1.179767in}{5.180710in}}%
\pgfpathlineto{\pgfqpoint{1.180900in}{5.181588in}}%
\pgfpathlineto{\pgfqpoint{1.209341in}{5.203375in}}%
\pgfpathlineto{\pgfqpoint{1.214183in}{5.207057in}}%
\pgfpathlineto{\pgfqpoint{1.238914in}{5.225636in}}%
\pgfpathlineto{\pgfqpoint{1.248152in}{5.232527in}}%
\pgfpathlineto{\pgfqpoint{1.268488in}{5.247511in}}%
\pgfpathlineto{\pgfqpoint{1.282818in}{5.257996in}}%
\pgfpathlineto{\pgfqpoint{1.298061in}{5.269015in}}%
\pgfpathlineto{\pgfqpoint{1.318189in}{5.283466in}}%
\pgfpathlineto{\pgfqpoint{1.327634in}{5.290164in}}%
\pgfpathlineto{\pgfqpoint{1.354277in}{5.308935in}}%
\pgfpathlineto{\pgfqpoint{1.357208in}{5.310975in}}%
\pgfpathlineto{\pgfqpoint{1.386781in}{5.331405in}}%
\pgfpathlineto{\pgfqpoint{1.391155in}{5.334405in}}%
\pgfpathlineto{\pgfqpoint{1.416354in}{5.351479in}}%
\pgfpathlineto{\pgfqpoint{1.428823in}{5.359874in}}%
\pgfpathlineto{\pgfqpoint{1.445928in}{5.371252in}}%
\pgfpathlineto{\pgfqpoint{1.467243in}{5.385344in}}%
\pgfpathlineto{\pgfqpoint{1.475501in}{5.390737in}}%
\pgfpathlineto{\pgfqpoint{1.505074in}{5.409931in}}%
\pgfpathlineto{\pgfqpoint{1.506444in}{5.410813in}}%
\pgfpathlineto{\pgfqpoint{1.534648in}{5.428754in}}%
\pgfpathlineto{\pgfqpoint{1.546551in}{5.436283in}}%
\pgfpathlineto{\pgfqpoint{1.564221in}{5.447323in}}%
\pgfpathlineto{\pgfqpoint{1.587443in}{5.461752in}}%
\pgfpathlineto{\pgfqpoint{1.593795in}{5.465651in}}%
\pgfpathlineto{\pgfqpoint{1.623368in}{5.483681in}}%
\pgfpathlineto{\pgfqpoint{1.629215in}{5.487222in}}%
\pgfpathlineto{\pgfqpoint{1.652941in}{5.501414in}}%
\pgfpathlineto{\pgfqpoint{1.671893in}{5.512691in}}%
\pgfpathlineto{\pgfqpoint{1.682515in}{5.518935in}}%
\pgfpathlineto{\pgfqpoint{1.712088in}{5.536220in}}%
\pgfpathlineto{\pgfqpoint{1.715434in}{5.538161in}}%
\pgfpathlineto{\pgfqpoint{1.741661in}{5.553190in}}%
\pgfpathlineto{\pgfqpoint{1.759967in}{5.563630in}}%
\pgfpathlineto{\pgfqpoint{1.771235in}{5.569978in}}%
\pgfpathlineto{\pgfqpoint{1.800808in}{5.586546in}}%
\pgfpathlineto{\pgfqpoint{1.805401in}{5.589100in}}%
\pgfpathlineto{\pgfqpoint{1.830381in}{5.602824in}}%
\pgfpathlineto{\pgfqpoint{1.851854in}{5.614570in}}%
\pgfpathlineto{\pgfqpoint{1.859955in}{5.618946in}}%
\pgfpathlineto{\pgfqpoint{1.889528in}{5.634823in}}%
\pgfpathlineto{\pgfqpoint{1.899303in}{5.640039in}}%
\pgfusepath{stroke}%
\end{pgfscope}%
\begin{pgfscope}%
\pgfpathrectangle{\pgfqpoint{0.854460in}{0.571603in}}{\pgfqpoint{5.885100in}{5.068436in}}%
\pgfusepath{clip}%
\pgfsetbuttcap%
\pgfsetroundjoin%
\pgfsetlinewidth{1.505625pt}%
\definecolor{currentstroke}{rgb}{0.170948,0.694384,0.493803}%
\pgfsetstrokecolor{currentstroke}%
\pgfsetdash{}{0pt}%
\pgfpathmoveto{\pgfqpoint{6.739560in}{5.402128in}}%
\pgfpathlineto{\pgfqpoint{6.532547in}{4.921016in}}%
\pgfpathlineto{\pgfqpoint{6.483999in}{4.799545in}}%
\pgfpathlineto{\pgfqpoint{6.443827in}{4.693151in}}%
\pgfpathlineto{\pgfqpoint{6.409186in}{4.595788in}}%
\pgfpathlineto{\pgfqpoint{6.375428in}{4.493910in}}%
\pgfpathlineto{\pgfqpoint{6.344433in}{4.392032in}}%
\pgfpathlineto{\pgfqpoint{6.316403in}{4.290154in}}%
\pgfpathlineto{\pgfqpoint{6.295960in}{4.207315in}}%
\pgfpathlineto{\pgfqpoint{6.280296in}{4.137337in}}%
\pgfpathlineto{\pgfqpoint{6.265092in}{4.060928in}}%
\pgfpathlineto{\pgfqpoint{6.251802in}{3.984520in}}%
\pgfpathlineto{\pgfqpoint{6.240572in}{3.908111in}}%
\pgfpathlineto{\pgfqpoint{6.231391in}{3.831703in}}%
\pgfpathlineto{\pgfqpoint{6.224323in}{3.755294in}}%
\pgfpathlineto{\pgfqpoint{6.219429in}{3.678886in}}%
\pgfpathlineto{\pgfqpoint{6.216740in}{3.602477in}}%
\pgfpathlineto{\pgfqpoint{6.216291in}{3.526068in}}%
\pgfpathlineto{\pgfqpoint{6.218116in}{3.449660in}}%
\pgfpathlineto{\pgfqpoint{6.222251in}{3.373251in}}%
\pgfpathlineto{\pgfqpoint{6.228732in}{3.296843in}}%
\pgfpathlineto{\pgfqpoint{6.237594in}{3.220434in}}%
\pgfpathlineto{\pgfqpoint{6.248805in}{3.144025in}}%
\pgfpathlineto{\pgfqpoint{6.262470in}{3.067617in}}%
\pgfpathlineto{\pgfqpoint{6.278544in}{2.991208in}}%
\pgfpathlineto{\pgfqpoint{6.297112in}{2.914800in}}%
\pgfpathlineto{\pgfqpoint{6.318114in}{2.838391in}}%
\pgfpathlineto{\pgfqpoint{6.341618in}{2.761983in}}%
\pgfpathlineto{\pgfqpoint{6.367633in}{2.685574in}}%
\pgfpathlineto{\pgfqpoint{6.396167in}{2.609165in}}%
\pgfpathlineto{\pgfqpoint{6.427229in}{2.532757in}}%
\pgfpathlineto{\pgfqpoint{6.460831in}{2.456348in}}%
\pgfpathlineto{\pgfqpoint{6.502973in}{2.367892in}}%
\pgfpathlineto{\pgfqpoint{6.535705in}{2.303531in}}%
\pgfpathlineto{\pgfqpoint{6.576957in}{2.227123in}}%
\pgfpathlineto{\pgfqpoint{6.621267in}{2.149966in}}%
\pgfpathlineto{\pgfqpoint{6.667204in}{2.074305in}}%
\pgfpathlineto{\pgfqpoint{6.716197in}{1.997897in}}%
\pgfpathlineto{\pgfqpoint{6.739560in}{1.962865in}}%
\pgfpathlineto{\pgfqpoint{6.739560in}{1.962865in}}%
\pgfusepath{stroke}%
\end{pgfscope}%
\begin{pgfscope}%
\pgfpathrectangle{\pgfqpoint{0.854460in}{0.571603in}}{\pgfqpoint{5.885100in}{5.068436in}}%
\pgfusepath{clip}%
\pgfsetbuttcap%
\pgfsetroundjoin%
\pgfsetlinewidth{1.505625pt}%
\definecolor{currentstroke}{rgb}{0.196571,0.711827,0.479221}%
\pgfsetstrokecolor{currentstroke}%
\pgfsetdash{}{0pt}%
\pgfpathmoveto{\pgfqpoint{0.854460in}{4.958405in}}%
\pgfpathlineto{\pgfqpoint{0.875240in}{4.977831in}}%
\pgfpathlineto{\pgfqpoint{0.884034in}{4.985951in}}%
\pgfpathlineto{\pgfqpoint{0.902984in}{5.003301in}}%
\pgfpathlineto{\pgfqpoint{0.913607in}{5.012907in}}%
\pgfpathlineto{\pgfqpoint{0.931298in}{5.028770in}}%
\pgfpathlineto{\pgfqpoint{0.943181in}{5.039295in}}%
\pgfpathlineto{\pgfqpoint{0.960193in}{5.054240in}}%
\pgfpathlineto{\pgfqpoint{0.972754in}{5.065139in}}%
\pgfpathlineto{\pgfqpoint{0.989683in}{5.079709in}}%
\pgfpathlineto{\pgfqpoint{1.002327in}{5.090460in}}%
\pgfpathlineto{\pgfqpoint{1.019778in}{5.105179in}}%
\pgfpathlineto{\pgfqpoint{1.031901in}{5.115279in}}%
\pgfpathlineto{\pgfqpoint{1.050492in}{5.130649in}}%
\pgfpathlineto{\pgfqpoint{1.061474in}{5.139617in}}%
\pgfpathlineto{\pgfqpoint{1.081835in}{5.156118in}}%
\pgfpathlineto{\pgfqpoint{1.091047in}{5.163493in}}%
\pgfpathlineto{\pgfqpoint{1.113820in}{5.181588in}}%
\pgfpathlineto{\pgfqpoint{1.120621in}{5.186926in}}%
\pgfpathlineto{\pgfqpoint{1.146457in}{5.207057in}}%
\pgfpathlineto{\pgfqpoint{1.150194in}{5.209934in}}%
\pgfpathlineto{\pgfqpoint{1.179757in}{5.232527in}}%
\pgfpathlineto{\pgfqpoint{1.179767in}{5.232534in}}%
\pgfpathlineto{\pgfqpoint{1.209341in}{5.254683in}}%
\pgfpathlineto{\pgfqpoint{1.213796in}{5.257996in}}%
\pgfpathlineto{\pgfqpoint{1.238914in}{5.276450in}}%
\pgfpathlineto{\pgfqpoint{1.248529in}{5.283466in}}%
\pgfpathlineto{\pgfqpoint{1.268488in}{5.297852in}}%
\pgfpathlineto{\pgfqpoint{1.283966in}{5.308935in}}%
\pgfpathlineto{\pgfqpoint{1.298061in}{5.318905in}}%
\pgfpathlineto{\pgfqpoint{1.320116in}{5.334405in}}%
\pgfpathlineto{\pgfqpoint{1.327634in}{5.339625in}}%
\pgfpathlineto{\pgfqpoint{1.356987in}{5.359874in}}%
\pgfpathlineto{\pgfqpoint{1.357208in}{5.360024in}}%
\pgfpathlineto{\pgfqpoint{1.386781in}{5.380019in}}%
\pgfpathlineto{\pgfqpoint{1.394706in}{5.385344in}}%
\pgfpathlineto{\pgfqpoint{1.416354in}{5.399714in}}%
\pgfpathlineto{\pgfqpoint{1.433177in}{5.410813in}}%
\pgfpathlineto{\pgfqpoint{1.445928in}{5.419125in}}%
\pgfpathlineto{\pgfqpoint{1.472404in}{5.436283in}}%
\pgfpathlineto{\pgfqpoint{1.475501in}{5.438265in}}%
\pgfpathlineto{\pgfqpoint{1.505074in}{5.457060in}}%
\pgfpathlineto{\pgfqpoint{1.512505in}{5.461752in}}%
\pgfpathlineto{\pgfqpoint{1.534648in}{5.475566in}}%
\pgfpathlineto{\pgfqpoint{1.553433in}{5.487222in}}%
\pgfpathlineto{\pgfqpoint{1.564221in}{5.493834in}}%
\pgfpathlineto{\pgfqpoint{1.593795in}{5.511861in}}%
\pgfpathlineto{\pgfqpoint{1.595168in}{5.512691in}}%
\pgfpathlineto{\pgfqpoint{1.623368in}{5.529541in}}%
\pgfpathlineto{\pgfqpoint{1.637867in}{5.538161in}}%
\pgfpathlineto{\pgfqpoint{1.652941in}{5.547013in}}%
\pgfpathlineto{\pgfqpoint{1.681374in}{5.563630in}}%
\pgfpathlineto{\pgfqpoint{1.682515in}{5.564289in}}%
\pgfpathlineto{\pgfqpoint{1.712088in}{5.581229in}}%
\pgfpathlineto{\pgfqpoint{1.725897in}{5.589100in}}%
\pgfpathlineto{\pgfqpoint{1.741661in}{5.597976in}}%
\pgfpathlineto{\pgfqpoint{1.771235in}{5.614555in}}%
\pgfpathlineto{\pgfqpoint{1.771262in}{5.614570in}}%
\pgfpathlineto{\pgfqpoint{1.800808in}{5.630798in}}%
\pgfpathlineto{\pgfqpoint{1.817704in}{5.640039in}}%
\pgfusepath{stroke}%
\end{pgfscope}%
\begin{pgfscope}%
\pgfpathrectangle{\pgfqpoint{0.854460in}{0.571603in}}{\pgfqpoint{5.885100in}{5.068436in}}%
\pgfusepath{clip}%
\pgfsetbuttcap%
\pgfsetroundjoin%
\pgfsetlinewidth{1.505625pt}%
\definecolor{currentstroke}{rgb}{0.196571,0.711827,0.479221}%
\pgfsetstrokecolor{currentstroke}%
\pgfsetdash{}{0pt}%
\pgfpathmoveto{\pgfqpoint{6.739560in}{5.206321in}}%
\pgfpathlineto{\pgfqpoint{6.670888in}{5.054240in}}%
\pgfpathlineto{\pgfqpoint{6.615500in}{4.926892in}}%
\pgfpathlineto{\pgfqpoint{6.573075in}{4.825014in}}%
\pgfpathlineto{\pgfqpoint{6.532547in}{4.722658in}}%
\pgfpathlineto{\pgfqpoint{6.494655in}{4.621258in}}%
\pgfpathlineto{\pgfqpoint{6.459174in}{4.519380in}}%
\pgfpathlineto{\pgfqpoint{6.426496in}{4.417502in}}%
\pgfpathlineto{\pgfqpoint{6.396813in}{4.315624in}}%
\pgfpathlineto{\pgfqpoint{6.376631in}{4.239215in}}%
\pgfpathlineto{\pgfqpoint{6.355107in}{4.148425in}}%
\pgfpathlineto{\pgfqpoint{6.341880in}{4.086398in}}%
\pgfpathlineto{\pgfqpoint{6.325533in}{3.998682in}}%
\pgfpathlineto{\pgfqpoint{6.315043in}{3.933581in}}%
\pgfpathlineto{\pgfqpoint{6.304725in}{3.857172in}}%
\pgfpathlineto{\pgfqpoint{6.295960in}{3.773990in}}%
\pgfpathlineto{\pgfqpoint{6.290525in}{3.704355in}}%
\pgfpathlineto{\pgfqpoint{6.286716in}{3.627946in}}%
\pgfpathlineto{\pgfqpoint{6.285164in}{3.551538in}}%
\pgfpathlineto{\pgfqpoint{6.285901in}{3.475129in}}%
\pgfpathlineto{\pgfqpoint{6.288961in}{3.398721in}}%
\pgfpathlineto{\pgfqpoint{6.294380in}{3.322312in}}%
\pgfpathlineto{\pgfqpoint{6.302147in}{3.245904in}}%
\pgfpathlineto{\pgfqpoint{6.312320in}{3.169495in}}%
\pgfpathlineto{\pgfqpoint{6.325533in}{3.089980in}}%
\pgfpathlineto{\pgfqpoint{6.339980in}{3.016678in}}%
\pgfpathlineto{\pgfqpoint{6.357519in}{2.940269in}}%
\pgfpathlineto{\pgfqpoint{6.377510in}{2.863861in}}%
\pgfpathlineto{\pgfqpoint{6.400006in}{2.787452in}}%
\pgfpathlineto{\pgfqpoint{6.425027in}{2.711044in}}%
\pgfpathlineto{\pgfqpoint{6.452575in}{2.634635in}}%
\pgfpathlineto{\pgfqpoint{6.482660in}{2.558226in}}%
\pgfpathlineto{\pgfqpoint{6.515292in}{2.481818in}}%
\pgfpathlineto{\pgfqpoint{6.550487in}{2.405409in}}%
\pgfpathlineto{\pgfqpoint{6.591693in}{2.322376in}}%
\pgfpathlineto{\pgfqpoint{6.628590in}{2.252592in}}%
\pgfpathlineto{\pgfqpoint{6.671502in}{2.176183in}}%
\pgfpathlineto{\pgfqpoint{6.717003in}{2.099775in}}%
\pgfpathlineto{\pgfqpoint{6.739560in}{2.063494in}}%
\pgfpathlineto{\pgfqpoint{6.739560in}{2.063494in}}%
\pgfusepath{stroke}%
\end{pgfscope}%
\begin{pgfscope}%
\pgfpathrectangle{\pgfqpoint{0.854460in}{0.571603in}}{\pgfqpoint{5.885100in}{5.068436in}}%
\pgfusepath{clip}%
\pgfsetbuttcap%
\pgfsetroundjoin%
\pgfsetlinewidth{1.505625pt}%
\definecolor{currentstroke}{rgb}{0.232815,0.732247,0.459277}%
\pgfsetstrokecolor{currentstroke}%
\pgfsetdash{}{0pt}%
\pgfpathmoveto{\pgfqpoint{0.854460in}{5.015079in}}%
\pgfpathlineto{\pgfqpoint{0.869485in}{5.028770in}}%
\pgfpathlineto{\pgfqpoint{0.884034in}{5.041866in}}%
\pgfpathlineto{\pgfqpoint{0.897895in}{5.054240in}}%
\pgfpathlineto{\pgfqpoint{0.913607in}{5.068095in}}%
\pgfpathlineto{\pgfqpoint{0.926887in}{5.079709in}}%
\pgfpathlineto{\pgfqpoint{0.943181in}{5.093787in}}%
\pgfpathlineto{\pgfqpoint{0.956472in}{5.105179in}}%
\pgfpathlineto{\pgfqpoint{0.972754in}{5.118965in}}%
\pgfpathlineto{\pgfqpoint{0.986662in}{5.130649in}}%
\pgfpathlineto{\pgfqpoint{1.002327in}{5.143648in}}%
\pgfpathlineto{\pgfqpoint{1.017470in}{5.156118in}}%
\pgfpathlineto{\pgfqpoint{1.031901in}{5.167857in}}%
\pgfpathlineto{\pgfqpoint{1.048907in}{5.181588in}}%
\pgfpathlineto{\pgfqpoint{1.061474in}{5.191611in}}%
\pgfpathlineto{\pgfqpoint{1.080983in}{5.207057in}}%
\pgfpathlineto{\pgfqpoint{1.091047in}{5.214928in}}%
\pgfpathlineto{\pgfqpoint{1.113711in}{5.232527in}}%
\pgfpathlineto{\pgfqpoint{1.120621in}{5.237827in}}%
\pgfpathlineto{\pgfqpoint{1.147100in}{5.257996in}}%
\pgfpathlineto{\pgfqpoint{1.150194in}{5.260325in}}%
\pgfpathlineto{\pgfqpoint{1.179767in}{5.282418in}}%
\pgfpathlineto{\pgfqpoint{1.181181in}{5.283466in}}%
\pgfpathlineto{\pgfqpoint{1.209341in}{5.304092in}}%
\pgfpathlineto{\pgfqpoint{1.215998in}{5.308935in}}%
\pgfpathlineto{\pgfqpoint{1.238914in}{5.325406in}}%
\pgfpathlineto{\pgfqpoint{1.251517in}{5.334405in}}%
\pgfpathlineto{\pgfqpoint{1.268488in}{5.346376in}}%
\pgfpathlineto{\pgfqpoint{1.287746in}{5.359874in}}%
\pgfpathlineto{\pgfqpoint{1.298061in}{5.367017in}}%
\pgfpathlineto{\pgfqpoint{1.324694in}{5.385344in}}%
\pgfpathlineto{\pgfqpoint{1.327634in}{5.387343in}}%
\pgfpathlineto{\pgfqpoint{1.357208in}{5.407302in}}%
\pgfpathlineto{\pgfqpoint{1.362445in}{5.410813in}}%
\pgfpathlineto{\pgfqpoint{1.386781in}{5.426931in}}%
\pgfpathlineto{\pgfqpoint{1.400984in}{5.436283in}}%
\pgfpathlineto{\pgfqpoint{1.416354in}{5.446281in}}%
\pgfpathlineto{\pgfqpoint{1.440276in}{5.461752in}}%
\pgfpathlineto{\pgfqpoint{1.445928in}{5.465364in}}%
\pgfpathlineto{\pgfqpoint{1.475501in}{5.484135in}}%
\pgfpathlineto{\pgfqpoint{1.480398in}{5.487222in}}%
\pgfpathlineto{\pgfqpoint{1.505074in}{5.502590in}}%
\pgfpathlineto{\pgfqpoint{1.521379in}{5.512691in}}%
\pgfpathlineto{\pgfqpoint{1.534648in}{5.520812in}}%
\pgfpathlineto{\pgfqpoint{1.563143in}{5.538161in}}%
\pgfpathlineto{\pgfqpoint{1.564221in}{5.538810in}}%
\pgfpathlineto{\pgfqpoint{1.593795in}{5.556461in}}%
\pgfpathlineto{\pgfqpoint{1.605867in}{5.563630in}}%
\pgfpathlineto{\pgfqpoint{1.623368in}{5.573896in}}%
\pgfpathlineto{\pgfqpoint{1.649410in}{5.589100in}}%
\pgfpathlineto{\pgfqpoint{1.652941in}{5.591137in}}%
\pgfpathlineto{\pgfqpoint{1.682515in}{5.608071in}}%
\pgfpathlineto{\pgfqpoint{1.693923in}{5.614570in}}%
\pgfpathlineto{\pgfqpoint{1.712088in}{5.624790in}}%
\pgfpathlineto{\pgfqpoint{1.739307in}{5.640039in}}%
\pgfusepath{stroke}%
\end{pgfscope}%
\begin{pgfscope}%
\pgfpathrectangle{\pgfqpoint{0.854460in}{0.571603in}}{\pgfqpoint{5.885100in}{5.068436in}}%
\pgfusepath{clip}%
\pgfsetbuttcap%
\pgfsetroundjoin%
\pgfsetlinewidth{1.505625pt}%
\definecolor{currentstroke}{rgb}{0.232815,0.732247,0.459277}%
\pgfsetstrokecolor{currentstroke}%
\pgfsetdash{}{0pt}%
\pgfpathmoveto{\pgfqpoint{6.739560in}{5.028158in}}%
\pgfpathlineto{\pgfqpoint{6.683068in}{4.901423in}}%
\pgfpathlineto{\pgfqpoint{6.639800in}{4.799545in}}%
\pgfpathlineto{\pgfqpoint{6.598797in}{4.697667in}}%
\pgfpathlineto{\pgfqpoint{6.560301in}{4.595788in}}%
\pgfpathlineto{\pgfqpoint{6.524503in}{4.493910in}}%
\pgfpathlineto{\pgfqpoint{6.491651in}{4.392032in}}%
\pgfpathlineto{\pgfqpoint{6.469066in}{4.315624in}}%
\pgfpathlineto{\pgfqpoint{6.443827in}{4.221688in}}%
\pgfpathlineto{\pgfqpoint{6.429400in}{4.162807in}}%
\pgfpathlineto{\pgfqpoint{6.412495in}{4.086398in}}%
\pgfpathlineto{\pgfqpoint{6.397556in}{4.009989in}}%
\pgfpathlineto{\pgfqpoint{6.384680in}{3.933262in}}%
\pgfpathlineto{\pgfqpoint{6.373956in}{3.857172in}}%
\pgfpathlineto{\pgfqpoint{6.365367in}{3.780764in}}%
\pgfpathlineto{\pgfqpoint{6.358988in}{3.704355in}}%
\pgfpathlineto{\pgfqpoint{6.354844in}{3.627946in}}%
\pgfpathlineto{\pgfqpoint{6.352956in}{3.551538in}}%
\pgfpathlineto{\pgfqpoint{6.353382in}{3.475129in}}%
\pgfpathlineto{\pgfqpoint{6.356147in}{3.398721in}}%
\pgfpathlineto{\pgfqpoint{6.361263in}{3.322312in}}%
\pgfpathlineto{\pgfqpoint{6.368778in}{3.245904in}}%
\pgfpathlineto{\pgfqpoint{6.378729in}{3.169495in}}%
\pgfpathlineto{\pgfqpoint{6.391111in}{3.093086in}}%
\pgfpathlineto{\pgfqpoint{6.405953in}{3.016678in}}%
\pgfpathlineto{\pgfqpoint{6.423277in}{2.940269in}}%
\pgfpathlineto{\pgfqpoint{6.443827in}{2.861362in}}%
\pgfpathlineto{\pgfqpoint{6.465441in}{2.787452in}}%
\pgfpathlineto{\pgfqpoint{6.490302in}{2.711044in}}%
\pgfpathlineto{\pgfqpoint{6.517709in}{2.634635in}}%
\pgfpathlineto{\pgfqpoint{6.547669in}{2.558226in}}%
\pgfpathlineto{\pgfqpoint{6.580193in}{2.481818in}}%
\pgfpathlineto{\pgfqpoint{6.621267in}{2.393046in}}%
\pgfpathlineto{\pgfqpoint{6.652981in}{2.329001in}}%
\pgfpathlineto{\pgfqpoint{6.693223in}{2.252592in}}%
\pgfpathlineto{\pgfqpoint{6.739560in}{2.170249in}}%
\pgfpathlineto{\pgfqpoint{6.739560in}{2.170249in}}%
\pgfusepath{stroke}%
\end{pgfscope}%
\begin{pgfscope}%
\pgfpathrectangle{\pgfqpoint{0.854460in}{0.571603in}}{\pgfqpoint{5.885100in}{5.068436in}}%
\pgfusepath{clip}%
\pgfsetbuttcap%
\pgfsetroundjoin%
\pgfsetlinewidth{1.505625pt}%
\definecolor{currentstroke}{rgb}{0.266941,0.748751,0.440573}%
\pgfsetstrokecolor{currentstroke}%
\pgfsetdash{}{0pt}%
\pgfpathmoveto{\pgfqpoint{6.739560in}{4.855943in}}%
\pgfpathlineto{\pgfqpoint{6.737162in}{4.850484in}}%
\pgfpathlineto{\pgfqpoint{6.726088in}{4.825014in}}%
\pgfpathlineto{\pgfqpoint{6.715190in}{4.799545in}}%
\pgfpathlineto{\pgfqpoint{6.709987in}{4.787224in}}%
\pgfpathlineto{\pgfqpoint{6.704426in}{4.774075in}}%
\pgfpathlineto{\pgfqpoint{6.693803in}{4.748606in}}%
\pgfpathlineto{\pgfqpoint{6.683365in}{4.723136in}}%
\pgfpathlineto{\pgfqpoint{6.680414in}{4.715835in}}%
\pgfpathlineto{\pgfqpoint{6.673056in}{4.697667in}}%
\pgfpathlineto{\pgfqpoint{6.662914in}{4.672197in}}%
\pgfpathlineto{\pgfqpoint{6.652966in}{4.646728in}}%
\pgfpathlineto{\pgfqpoint{6.650840in}{4.641202in}}%
\pgfpathlineto{\pgfqpoint{6.643152in}{4.621258in}}%
\pgfpathlineto{\pgfqpoint{6.633520in}{4.595788in}}%
\pgfpathlineto{\pgfqpoint{6.624088in}{4.570319in}}%
\pgfpathlineto{\pgfqpoint{6.621267in}{4.562564in}}%
\pgfpathlineto{\pgfqpoint{6.614807in}{4.544849in}}%
\pgfpathlineto{\pgfqpoint{6.605710in}{4.519380in}}%
\pgfpathlineto{\pgfqpoint{6.596820in}{4.493910in}}%
\pgfpathlineto{\pgfqpoint{6.591693in}{4.478904in}}%
\pgfpathlineto{\pgfqpoint{6.588110in}{4.468441in}}%
\pgfpathlineto{\pgfqpoint{6.579572in}{4.442971in}}%
\pgfpathlineto{\pgfqpoint{6.571246in}{4.417502in}}%
\pgfpathlineto{\pgfqpoint{6.563136in}{4.392032in}}%
\pgfpathlineto{\pgfqpoint{6.562120in}{4.388770in}}%
\pgfpathlineto{\pgfqpoint{6.555187in}{4.366563in}}%
\pgfpathlineto{\pgfqpoint{6.547448in}{4.341093in}}%
\pgfpathlineto{\pgfqpoint{6.539930in}{4.315624in}}%
\pgfpathlineto{\pgfqpoint{6.532633in}{4.290154in}}%
\pgfpathlineto{\pgfqpoint{6.532547in}{4.289846in}}%
\pgfpathlineto{\pgfqpoint{6.525503in}{4.264685in}}%
\pgfpathlineto{\pgfqpoint{6.518598in}{4.239215in}}%
\pgfpathlineto{\pgfqpoint{6.511920in}{4.213746in}}%
\pgfpathlineto{\pgfqpoint{6.505468in}{4.188276in}}%
\pgfpathlineto{\pgfqpoint{6.502973in}{4.178092in}}%
\pgfpathlineto{\pgfqpoint{6.499216in}{4.162807in}}%
\pgfpathlineto{\pgfqpoint{6.493175in}{4.137337in}}%
\pgfpathlineto{\pgfqpoint{6.487367in}{4.111867in}}%
\pgfpathlineto{\pgfqpoint{6.481791in}{4.086398in}}%
\pgfpathlineto{\pgfqpoint{6.476450in}{4.060928in}}%
\pgfpathlineto{\pgfqpoint{6.473400in}{4.045741in}}%
\pgfpathlineto{\pgfqpoint{6.471327in}{4.035459in}}%
\pgfpathlineto{\pgfqpoint{6.466419in}{4.009989in}}%
\pgfpathlineto{\pgfqpoint{6.461749in}{3.984520in}}%
\pgfpathlineto{\pgfqpoint{6.457319in}{3.959050in}}%
\pgfpathlineto{\pgfqpoint{6.453129in}{3.933581in}}%
\pgfpathlineto{\pgfqpoint{6.449181in}{3.908111in}}%
\pgfpathlineto{\pgfqpoint{6.445474in}{3.882642in}}%
\pgfpathlineto{\pgfqpoint{6.443827in}{3.870541in}}%
\pgfpathlineto{\pgfqpoint{6.441998in}{3.857172in}}%
\pgfpathlineto{\pgfqpoint{6.438754in}{3.831703in}}%
\pgfpathlineto{\pgfqpoint{6.435758in}{3.806233in}}%
\pgfpathlineto{\pgfqpoint{6.433009in}{3.780764in}}%
\pgfpathlineto{\pgfqpoint{6.430509in}{3.755294in}}%
\pgfpathlineto{\pgfqpoint{6.428259in}{3.729825in}}%
\pgfpathlineto{\pgfqpoint{6.426259in}{3.704355in}}%
\pgfpathlineto{\pgfqpoint{6.424512in}{3.678886in}}%
\pgfpathlineto{\pgfqpoint{6.423016in}{3.653416in}}%
\pgfpathlineto{\pgfqpoint{6.421775in}{3.627946in}}%
\pgfpathlineto{\pgfqpoint{6.420788in}{3.602477in}}%
\pgfpathlineto{\pgfqpoint{6.420057in}{3.577007in}}%
\pgfpathlineto{\pgfqpoint{6.419582in}{3.551538in}}%
\pgfpathlineto{\pgfqpoint{6.419366in}{3.526068in}}%
\pgfpathlineto{\pgfqpoint{6.419409in}{3.500599in}}%
\pgfpathlineto{\pgfqpoint{6.419711in}{3.475129in}}%
\pgfpathlineto{\pgfqpoint{6.420275in}{3.449660in}}%
\pgfpathlineto{\pgfqpoint{6.421102in}{3.424190in}}%
\pgfpathlineto{\pgfqpoint{6.422192in}{3.398721in}}%
\pgfpathlineto{\pgfqpoint{6.423547in}{3.373251in}}%
\pgfpathlineto{\pgfqpoint{6.425168in}{3.347782in}}%
\pgfpathlineto{\pgfqpoint{6.427056in}{3.322312in}}%
\pgfpathlineto{\pgfqpoint{6.429213in}{3.296843in}}%
\pgfpathlineto{\pgfqpoint{6.431640in}{3.271373in}}%
\pgfpathlineto{\pgfqpoint{6.434339in}{3.245904in}}%
\pgfpathlineto{\pgfqpoint{6.437310in}{3.220434in}}%
\pgfpathlineto{\pgfqpoint{6.440556in}{3.194965in}}%
\pgfpathlineto{\pgfqpoint{6.443827in}{3.171305in}}%
\pgfpathlineto{\pgfqpoint{6.444075in}{3.169495in}}%
\pgfpathlineto{\pgfqpoint{6.447846in}{3.144025in}}%
\pgfpathlineto{\pgfqpoint{6.451894in}{3.118556in}}%
\pgfpathlineto{\pgfqpoint{6.456220in}{3.093086in}}%
\pgfpathlineto{\pgfqpoint{6.460826in}{3.067617in}}%
\pgfpathlineto{\pgfqpoint{6.465714in}{3.042147in}}%
\pgfpathlineto{\pgfqpoint{6.470884in}{3.016678in}}%
\pgfpathlineto{\pgfqpoint{6.473400in}{3.004930in}}%
\pgfpathlineto{\pgfqpoint{6.476319in}{2.991208in}}%
\pgfpathlineto{\pgfqpoint{6.482020in}{2.965739in}}%
\pgfpathlineto{\pgfqpoint{6.488008in}{2.940269in}}%
\pgfpathlineto{\pgfqpoint{6.494283in}{2.914800in}}%
\pgfpathlineto{\pgfqpoint{6.500849in}{2.889330in}}%
\pgfpathlineto{\pgfqpoint{6.502973in}{2.881436in}}%
\pgfpathlineto{\pgfqpoint{6.507673in}{2.863861in}}%
\pgfpathlineto{\pgfqpoint{6.514774in}{2.838391in}}%
\pgfpathlineto{\pgfqpoint{6.522169in}{2.812922in}}%
\pgfpathlineto{\pgfqpoint{6.529859in}{2.787452in}}%
\pgfpathlineto{\pgfqpoint{6.532547in}{2.778878in}}%
\pgfpathlineto{\pgfqpoint{6.537811in}{2.761983in}}%
\pgfpathlineto{\pgfqpoint{6.546041in}{2.736513in}}%
\pgfpathlineto{\pgfqpoint{6.554571in}{2.711044in}}%
\pgfpathlineto{\pgfqpoint{6.562120in}{2.689268in}}%
\pgfpathlineto{\pgfqpoint{6.563394in}{2.685574in}}%
\pgfpathlineto{\pgfqpoint{6.572466in}{2.660104in}}%
\pgfpathlineto{\pgfqpoint{6.581844in}{2.634635in}}%
\pgfpathlineto{\pgfqpoint{6.591528in}{2.609165in}}%
\pgfpathlineto{\pgfqpoint{6.591693in}{2.608742in}}%
\pgfpathlineto{\pgfqpoint{6.601452in}{2.583696in}}%
\pgfpathlineto{\pgfqpoint{6.611686in}{2.558226in}}%
\pgfpathlineto{\pgfqpoint{6.621267in}{2.535081in}}%
\pgfpathlineto{\pgfqpoint{6.622224in}{2.532757in}}%
\pgfpathlineto{\pgfqpoint{6.633008in}{2.507287in}}%
\pgfpathlineto{\pgfqpoint{6.644107in}{2.481818in}}%
\pgfpathlineto{\pgfqpoint{6.650840in}{2.466783in}}%
\pgfpathlineto{\pgfqpoint{6.655491in}{2.456348in}}%
\pgfpathlineto{\pgfqpoint{6.667146in}{2.430879in}}%
\pgfpathlineto{\pgfqpoint{6.679122in}{2.405409in}}%
\pgfpathlineto{\pgfqpoint{6.680414in}{2.402728in}}%
\pgfpathlineto{\pgfqpoint{6.691345in}{2.379940in}}%
\pgfpathlineto{\pgfqpoint{6.703883in}{2.354470in}}%
\pgfpathlineto{\pgfqpoint{6.709987in}{2.342369in}}%
\pgfpathlineto{\pgfqpoint{6.716701in}{2.329001in}}%
\pgfpathlineto{\pgfqpoint{6.729804in}{2.303531in}}%
\pgfpathlineto{\pgfqpoint{6.739560in}{2.285021in}}%
\pgfusepath{stroke}%
\end{pgfscope}%
\begin{pgfscope}%
\pgfpathrectangle{\pgfqpoint{0.854460in}{0.571603in}}{\pgfqpoint{5.885100in}{5.068436in}}%
\pgfusepath{clip}%
\pgfsetbuttcap%
\pgfsetroundjoin%
\pgfsetlinewidth{1.505625pt}%
\definecolor{currentstroke}{rgb}{0.266941,0.748751,0.440573}%
\pgfsetstrokecolor{currentstroke}%
\pgfsetdash{}{0pt}%
\pgfpathmoveto{\pgfqpoint{0.854460in}{5.069472in}}%
\pgfpathlineto{\pgfqpoint{0.865981in}{5.079709in}}%
\pgfpathlineto{\pgfqpoint{0.884034in}{5.095557in}}%
\pgfpathlineto{\pgfqpoint{0.895083in}{5.105179in}}%
\pgfpathlineto{\pgfqpoint{0.913607in}{5.121113in}}%
\pgfpathlineto{\pgfqpoint{0.924779in}{5.130649in}}%
\pgfpathlineto{\pgfqpoint{0.943181in}{5.146162in}}%
\pgfpathlineto{\pgfqpoint{0.955080in}{5.156118in}}%
\pgfpathlineto{\pgfqpoint{0.972754in}{5.170725in}}%
\pgfpathlineto{\pgfqpoint{0.985997in}{5.181588in}}%
\pgfpathlineto{\pgfqpoint{1.002327in}{5.194819in}}%
\pgfpathlineto{\pgfqpoint{1.017542in}{5.207057in}}%
\pgfpathlineto{\pgfqpoint{1.031901in}{5.218466in}}%
\pgfpathlineto{\pgfqpoint{1.049726in}{5.232527in}}%
\pgfpathlineto{\pgfqpoint{1.061474in}{5.241681in}}%
\pgfpathlineto{\pgfqpoint{1.082559in}{5.257996in}}%
\pgfpathlineto{\pgfqpoint{1.091047in}{5.264485in}}%
\pgfpathlineto{\pgfqpoint{1.116052in}{5.283466in}}%
\pgfpathlineto{\pgfqpoint{1.120621in}{5.286892in}}%
\pgfpathlineto{\pgfqpoint{1.150194in}{5.308920in}}%
\pgfpathlineto{\pgfqpoint{1.150214in}{5.308935in}}%
\pgfpathlineto{\pgfqpoint{1.179767in}{5.330513in}}%
\pgfpathlineto{\pgfqpoint{1.185132in}{5.334405in}}%
\pgfpathlineto{\pgfqpoint{1.209341in}{5.351752in}}%
\pgfpathlineto{\pgfqpoint{1.220749in}{5.359874in}}%
\pgfpathlineto{\pgfqpoint{1.238914in}{5.372651in}}%
\pgfpathlineto{\pgfqpoint{1.257073in}{5.385344in}}%
\pgfpathlineto{\pgfqpoint{1.268488in}{5.393226in}}%
\pgfpathlineto{\pgfqpoint{1.294112in}{5.410813in}}%
\pgfpathlineto{\pgfqpoint{1.298061in}{5.413491in}}%
\pgfpathlineto{\pgfqpoint{1.327634in}{5.433405in}}%
\pgfpathlineto{\pgfqpoint{1.331937in}{5.436283in}}%
\pgfpathlineto{\pgfqpoint{1.357208in}{5.452980in}}%
\pgfpathlineto{\pgfqpoint{1.370560in}{5.461752in}}%
\pgfpathlineto{\pgfqpoint{1.386781in}{5.472280in}}%
\pgfpathlineto{\pgfqpoint{1.409933in}{5.487222in}}%
\pgfpathlineto{\pgfqpoint{1.416354in}{5.491316in}}%
\pgfpathlineto{\pgfqpoint{1.445928in}{5.510054in}}%
\pgfpathlineto{\pgfqpoint{1.450120in}{5.512691in}}%
\pgfpathlineto{\pgfqpoint{1.475501in}{5.528470in}}%
\pgfpathlineto{\pgfqpoint{1.491171in}{5.538161in}}%
\pgfpathlineto{\pgfqpoint{1.505074in}{5.546655in}}%
\pgfpathlineto{\pgfqpoint{1.533000in}{5.563630in}}%
\pgfpathlineto{\pgfqpoint{1.534648in}{5.564620in}}%
\pgfpathlineto{\pgfqpoint{1.564221in}{5.582249in}}%
\pgfpathlineto{\pgfqpoint{1.575774in}{5.589100in}}%
\pgfpathlineto{\pgfqpoint{1.593795in}{5.599657in}}%
\pgfpathlineto{\pgfqpoint{1.619369in}{5.614570in}}%
\pgfpathlineto{\pgfqpoint{1.623368in}{5.616873in}}%
\pgfpathlineto{\pgfqpoint{1.652941in}{5.633792in}}%
\pgfpathlineto{\pgfqpoint{1.663920in}{5.640039in}}%
\pgfusepath{stroke}%
\end{pgfscope}%
\begin{pgfscope}%
\pgfpathrectangle{\pgfqpoint{0.854460in}{0.571603in}}{\pgfqpoint{5.885100in}{5.068436in}}%
\pgfusepath{clip}%
\pgfsetbuttcap%
\pgfsetroundjoin%
\pgfsetlinewidth{1.505625pt}%
\definecolor{currentstroke}{rgb}{0.311925,0.767822,0.415586}%
\pgfsetstrokecolor{currentstroke}%
\pgfsetdash{}{0pt}%
\pgfpathmoveto{\pgfqpoint{6.739560in}{4.682235in}}%
\pgfpathlineto{\pgfqpoint{6.735466in}{4.672197in}}%
\pgfpathlineto{\pgfqpoint{6.725239in}{4.646728in}}%
\pgfpathlineto{\pgfqpoint{6.715214in}{4.621258in}}%
\pgfpathlineto{\pgfqpoint{6.709987in}{4.607740in}}%
\pgfpathlineto{\pgfqpoint{6.705357in}{4.595788in}}%
\pgfpathlineto{\pgfqpoint{6.695666in}{4.570319in}}%
\pgfpathlineto{\pgfqpoint{6.686184in}{4.544849in}}%
\pgfpathlineto{\pgfqpoint{6.680414in}{4.529032in}}%
\pgfpathlineto{\pgfqpoint{6.676884in}{4.519380in}}%
\pgfpathlineto{\pgfqpoint{6.667753in}{4.493910in}}%
\pgfpathlineto{\pgfqpoint{6.658837in}{4.468441in}}%
\pgfpathlineto{\pgfqpoint{6.650840in}{4.445042in}}%
\pgfpathlineto{\pgfqpoint{6.650131in}{4.442971in}}%
\pgfpathlineto{\pgfqpoint{6.641581in}{4.417502in}}%
\pgfpathlineto{\pgfqpoint{6.633252in}{4.392032in}}%
\pgfpathlineto{\pgfqpoint{6.625144in}{4.366563in}}%
\pgfpathlineto{\pgfqpoint{6.621267in}{4.354071in}}%
\pgfpathlineto{\pgfqpoint{6.617227in}{4.341093in}}%
\pgfpathlineto{\pgfqpoint{6.609506in}{4.315624in}}%
\pgfpathlineto{\pgfqpoint{6.602010in}{4.290154in}}%
\pgfpathlineto{\pgfqpoint{6.594741in}{4.264685in}}%
\pgfpathlineto{\pgfqpoint{6.591693in}{4.253686in}}%
\pgfpathlineto{\pgfqpoint{6.587670in}{4.239215in}}%
\pgfpathlineto{\pgfqpoint{6.580807in}{4.213746in}}%
\pgfpathlineto{\pgfqpoint{6.574175in}{4.188276in}}%
\pgfpathlineto{\pgfqpoint{6.567775in}{4.162807in}}%
\pgfpathlineto{\pgfqpoint{6.562120in}{4.139456in}}%
\pgfpathlineto{\pgfqpoint{6.561605in}{4.137337in}}%
\pgfpathlineto{\pgfqpoint{6.555629in}{4.111867in}}%
\pgfpathlineto{\pgfqpoint{6.549889in}{4.086398in}}%
\pgfpathlineto{\pgfqpoint{6.544388in}{4.060928in}}%
\pgfpathlineto{\pgfqpoint{6.539125in}{4.035459in}}%
\pgfpathlineto{\pgfqpoint{6.534101in}{4.009989in}}%
\pgfpathlineto{\pgfqpoint{6.532547in}{4.001732in}}%
\pgfpathlineto{\pgfqpoint{6.529294in}{3.984520in}}%
\pgfpathlineto{\pgfqpoint{6.524718in}{3.959050in}}%
\pgfpathlineto{\pgfqpoint{6.520386in}{3.933581in}}%
\pgfpathlineto{\pgfqpoint{6.516299in}{3.908111in}}%
\pgfpathlineto{\pgfqpoint{6.512457in}{3.882642in}}%
\pgfpathlineto{\pgfqpoint{6.508862in}{3.857172in}}%
\pgfpathlineto{\pgfqpoint{6.505514in}{3.831703in}}%
\pgfpathlineto{\pgfqpoint{6.502973in}{3.810842in}}%
\pgfpathlineto{\pgfqpoint{6.502409in}{3.806233in}}%
\pgfpathlineto{\pgfqpoint{6.499537in}{3.780764in}}%
\pgfpathlineto{\pgfqpoint{6.496917in}{3.755294in}}%
\pgfpathlineto{\pgfqpoint{6.494550in}{3.729825in}}%
\pgfpathlineto{\pgfqpoint{6.492436in}{3.704355in}}%
\pgfpathlineto{\pgfqpoint{6.490576in}{3.678886in}}%
\pgfpathlineto{\pgfqpoint{6.488972in}{3.653416in}}%
\pgfpathlineto{\pgfqpoint{6.487624in}{3.627946in}}%
\pgfpathlineto{\pgfqpoint{6.486533in}{3.602477in}}%
\pgfpathlineto{\pgfqpoint{6.485701in}{3.577007in}}%
\pgfpathlineto{\pgfqpoint{6.485128in}{3.551538in}}%
\pgfpathlineto{\pgfqpoint{6.484816in}{3.526068in}}%
\pgfpathlineto{\pgfqpoint{6.484764in}{3.500599in}}%
\pgfpathlineto{\pgfqpoint{6.484976in}{3.475129in}}%
\pgfpathlineto{\pgfqpoint{6.485450in}{3.449660in}}%
\pgfpathlineto{\pgfqpoint{6.486190in}{3.424190in}}%
\pgfpathlineto{\pgfqpoint{6.487195in}{3.398721in}}%
\pgfpathlineto{\pgfqpoint{6.488468in}{3.373251in}}%
\pgfpathlineto{\pgfqpoint{6.490008in}{3.347782in}}%
\pgfpathlineto{\pgfqpoint{6.491818in}{3.322312in}}%
\pgfpathlineto{\pgfqpoint{6.493899in}{3.296843in}}%
\pgfpathlineto{\pgfqpoint{6.496252in}{3.271373in}}%
\pgfpathlineto{\pgfqpoint{6.498878in}{3.245904in}}%
\pgfpathlineto{\pgfqpoint{6.501779in}{3.220434in}}%
\pgfpathlineto{\pgfqpoint{6.502973in}{3.210859in}}%
\pgfpathlineto{\pgfqpoint{6.504942in}{3.194965in}}%
\pgfpathlineto{\pgfqpoint{6.508372in}{3.169495in}}%
\pgfpathlineto{\pgfqpoint{6.512079in}{3.144025in}}%
\pgfpathlineto{\pgfqpoint{6.516064in}{3.118556in}}%
\pgfpathlineto{\pgfqpoint{6.520329in}{3.093086in}}%
\pgfpathlineto{\pgfqpoint{6.524876in}{3.067617in}}%
\pgfpathlineto{\pgfqpoint{6.529706in}{3.042147in}}%
\pgfpathlineto{\pgfqpoint{6.532547in}{3.028000in}}%
\pgfpathlineto{\pgfqpoint{6.534805in}{3.016678in}}%
\pgfpathlineto{\pgfqpoint{6.540168in}{2.991208in}}%
\pgfpathlineto{\pgfqpoint{6.545818in}{2.965739in}}%
\pgfpathlineto{\pgfqpoint{6.551755in}{2.940269in}}%
\pgfpathlineto{\pgfqpoint{6.557982in}{2.914800in}}%
\pgfpathlineto{\pgfqpoint{6.562120in}{2.898626in}}%
\pgfpathlineto{\pgfqpoint{6.564484in}{2.889330in}}%
\pgfpathlineto{\pgfqpoint{6.571248in}{2.863861in}}%
\pgfpathlineto{\pgfqpoint{6.578306in}{2.838391in}}%
\pgfpathlineto{\pgfqpoint{6.585658in}{2.812922in}}%
\pgfpathlineto{\pgfqpoint{6.591693in}{2.792822in}}%
\pgfpathlineto{\pgfqpoint{6.593296in}{2.787452in}}%
\pgfpathlineto{\pgfqpoint{6.601190in}{2.761983in}}%
\pgfpathlineto{\pgfqpoint{6.609383in}{2.736513in}}%
\pgfpathlineto{\pgfqpoint{6.617877in}{2.711044in}}%
\pgfpathlineto{\pgfqpoint{6.621267in}{2.701221in}}%
\pgfpathlineto{\pgfqpoint{6.626637in}{2.685574in}}%
\pgfpathlineto{\pgfqpoint{6.635677in}{2.660104in}}%
\pgfpathlineto{\pgfqpoint{6.645023in}{2.634635in}}%
\pgfpathlineto{\pgfqpoint{6.650840in}{2.619278in}}%
\pgfpathlineto{\pgfqpoint{6.654650in}{2.609165in}}%
\pgfpathlineto{\pgfqpoint{6.664546in}{2.583696in}}%
\pgfpathlineto{\pgfqpoint{6.674753in}{2.558226in}}%
\pgfpathlineto{\pgfqpoint{6.680414in}{2.544508in}}%
\pgfpathlineto{\pgfqpoint{6.685239in}{2.532757in}}%
\pgfpathlineto{\pgfqpoint{6.695999in}{2.507287in}}%
\pgfpathlineto{\pgfqpoint{6.707076in}{2.481818in}}%
\pgfpathlineto{\pgfqpoint{6.709987in}{2.475300in}}%
\pgfpathlineto{\pgfqpoint{6.718413in}{2.456348in}}%
\pgfpathlineto{\pgfqpoint{6.730049in}{2.430879in}}%
\pgfpathlineto{\pgfqpoint{6.739560in}{2.410609in}}%
\pgfusepath{stroke}%
\end{pgfscope}%
\begin{pgfscope}%
\pgfpathrectangle{\pgfqpoint{0.854460in}{0.571603in}}{\pgfqpoint{5.885100in}{5.068436in}}%
\pgfusepath{clip}%
\pgfsetbuttcap%
\pgfsetroundjoin%
\pgfsetlinewidth{1.505625pt}%
\definecolor{currentstroke}{rgb}{0.311925,0.767822,0.415586}%
\pgfsetstrokecolor{currentstroke}%
\pgfsetdash{}{0pt}%
\pgfpathmoveto{\pgfqpoint{0.854460in}{5.121761in}}%
\pgfpathlineto{\pgfqpoint{0.864712in}{5.130649in}}%
\pgfpathlineto{\pgfqpoint{0.884034in}{5.147196in}}%
\pgfpathlineto{\pgfqpoint{0.894533in}{5.156118in}}%
\pgfpathlineto{\pgfqpoint{0.913607in}{5.172131in}}%
\pgfpathlineto{\pgfqpoint{0.924957in}{5.181588in}}%
\pgfpathlineto{\pgfqpoint{0.943181in}{5.196586in}}%
\pgfpathlineto{\pgfqpoint{0.955998in}{5.207057in}}%
\pgfpathlineto{\pgfqpoint{0.972754in}{5.220580in}}%
\pgfpathlineto{\pgfqpoint{0.987665in}{5.232527in}}%
\pgfpathlineto{\pgfqpoint{1.002327in}{5.244132in}}%
\pgfpathlineto{\pgfqpoint{1.019969in}{5.257996in}}%
\pgfpathlineto{\pgfqpoint{1.031901in}{5.267259in}}%
\pgfpathlineto{\pgfqpoint{1.052922in}{5.283466in}}%
\pgfpathlineto{\pgfqpoint{1.061474in}{5.289979in}}%
\pgfpathlineto{\pgfqpoint{1.086532in}{5.308935in}}%
\pgfpathlineto{\pgfqpoint{1.091047in}{5.312310in}}%
\pgfpathlineto{\pgfqpoint{1.120621in}{5.334263in}}%
\pgfpathlineto{\pgfqpoint{1.120813in}{5.334405in}}%
\pgfpathlineto{\pgfqpoint{1.150194in}{5.355788in}}%
\pgfpathlineto{\pgfqpoint{1.155845in}{5.359874in}}%
\pgfpathlineto{\pgfqpoint{1.179767in}{5.376963in}}%
\pgfpathlineto{\pgfqpoint{1.191573in}{5.385344in}}%
\pgfpathlineto{\pgfqpoint{1.209341in}{5.397804in}}%
\pgfpathlineto{\pgfqpoint{1.228006in}{5.410813in}}%
\pgfpathlineto{\pgfqpoint{1.238914in}{5.418324in}}%
\pgfpathlineto{\pgfqpoint{1.265151in}{5.436283in}}%
\pgfpathlineto{\pgfqpoint{1.268488in}{5.438539in}}%
\pgfpathlineto{\pgfqpoint{1.298061in}{5.458398in}}%
\pgfpathlineto{\pgfqpoint{1.303088in}{5.461752in}}%
\pgfpathlineto{\pgfqpoint{1.327634in}{5.477931in}}%
\pgfpathlineto{\pgfqpoint{1.341810in}{5.487222in}}%
\pgfpathlineto{\pgfqpoint{1.357208in}{5.497191in}}%
\pgfpathlineto{\pgfqpoint{1.381277in}{5.512691in}}%
\pgfpathlineto{\pgfqpoint{1.386781in}{5.516193in}}%
\pgfpathlineto{\pgfqpoint{1.416354in}{5.534886in}}%
\pgfpathlineto{\pgfqpoint{1.421568in}{5.538161in}}%
\pgfpathlineto{\pgfqpoint{1.445928in}{5.553274in}}%
\pgfpathlineto{\pgfqpoint{1.462705in}{5.563630in}}%
\pgfpathlineto{\pgfqpoint{1.475501in}{5.571434in}}%
\pgfpathlineto{\pgfqpoint{1.504613in}{5.589100in}}%
\pgfpathlineto{\pgfqpoint{1.505074in}{5.589376in}}%
\pgfpathlineto{\pgfqpoint{1.534648in}{5.606972in}}%
\pgfpathlineto{\pgfqpoint{1.547478in}{5.614570in}}%
\pgfpathlineto{\pgfqpoint{1.564221in}{5.624363in}}%
\pgfpathlineto{\pgfqpoint{1.591141in}{5.640039in}}%
\pgfusepath{stroke}%
\end{pgfscope}%
\begin{pgfscope}%
\pgfpathrectangle{\pgfqpoint{0.854460in}{0.571603in}}{\pgfqpoint{5.885100in}{5.068436in}}%
\pgfusepath{clip}%
\pgfsetbuttcap%
\pgfsetroundjoin%
\pgfsetlinewidth{1.505625pt}%
\definecolor{currentstroke}{rgb}{0.352360,0.783011,0.392636}%
\pgfsetstrokecolor{currentstroke}%
\pgfsetdash{}{0pt}%
\pgfpathmoveto{\pgfqpoint{6.739560in}{4.499731in}}%
\pgfpathlineto{\pgfqpoint{6.737430in}{4.493910in}}%
\pgfpathlineto{\pgfqpoint{6.728292in}{4.468441in}}%
\pgfpathlineto{\pgfqpoint{6.719376in}{4.442971in}}%
\pgfpathlineto{\pgfqpoint{6.710681in}{4.417502in}}%
\pgfpathlineto{\pgfqpoint{6.709987in}{4.415424in}}%
\pgfpathlineto{\pgfqpoint{6.702151in}{4.392032in}}%
\pgfpathlineto{\pgfqpoint{6.693842in}{4.366563in}}%
\pgfpathlineto{\pgfqpoint{6.685759in}{4.341093in}}%
\pgfpathlineto{\pgfqpoint{6.680414in}{4.323784in}}%
\pgfpathlineto{\pgfqpoint{6.677886in}{4.315624in}}%
\pgfpathlineto{\pgfqpoint{6.670203in}{4.290154in}}%
\pgfpathlineto{\pgfqpoint{6.662751in}{4.264685in}}%
\pgfpathlineto{\pgfqpoint{6.655532in}{4.239215in}}%
\pgfpathlineto{\pgfqpoint{6.650840in}{4.222133in}}%
\pgfpathlineto{\pgfqpoint{6.648529in}{4.213746in}}%
\pgfpathlineto{\pgfqpoint{6.641726in}{4.188276in}}%
\pgfpathlineto{\pgfqpoint{6.635161in}{4.162807in}}%
\pgfpathlineto{\pgfqpoint{6.628832in}{4.137337in}}%
\pgfpathlineto{\pgfqpoint{6.622741in}{4.111867in}}%
\pgfpathlineto{\pgfqpoint{6.621267in}{4.105466in}}%
\pgfpathlineto{\pgfqpoint{6.616858in}{4.086398in}}%
\pgfpathlineto{\pgfqpoint{6.611204in}{4.060928in}}%
\pgfpathlineto{\pgfqpoint{6.605794in}{4.035459in}}%
\pgfpathlineto{\pgfqpoint{6.600626in}{4.009989in}}%
\pgfpathlineto{\pgfqpoint{6.595701in}{3.984520in}}%
\pgfpathlineto{\pgfqpoint{6.591693in}{3.962717in}}%
\pgfpathlineto{\pgfqpoint{6.591017in}{3.959050in}}%
\pgfpathlineto{\pgfqpoint{6.586550in}{3.933581in}}%
\pgfpathlineto{\pgfqpoint{6.582331in}{3.908111in}}%
\pgfpathlineto{\pgfqpoint{6.578361in}{3.882642in}}%
\pgfpathlineto{\pgfqpoint{6.574640in}{3.857172in}}%
\pgfpathlineto{\pgfqpoint{6.571169in}{3.831703in}}%
\pgfpathlineto{\pgfqpoint{6.567950in}{3.806233in}}%
\pgfpathlineto{\pgfqpoint{6.564982in}{3.780764in}}%
\pgfpathlineto{\pgfqpoint{6.562267in}{3.755294in}}%
\pgfpathlineto{\pgfqpoint{6.562120in}{3.753778in}}%
\pgfpathlineto{\pgfqpoint{6.559789in}{3.729825in}}%
\pgfpathlineto{\pgfqpoint{6.557566in}{3.704355in}}%
\pgfpathlineto{\pgfqpoint{6.555599in}{3.678886in}}%
\pgfpathlineto{\pgfqpoint{6.553891in}{3.653416in}}%
\pgfpathlineto{\pgfqpoint{6.552442in}{3.627946in}}%
\pgfpathlineto{\pgfqpoint{6.551253in}{3.602477in}}%
\pgfpathlineto{\pgfqpoint{6.550324in}{3.577007in}}%
\pgfpathlineto{\pgfqpoint{6.549657in}{3.551538in}}%
\pgfpathlineto{\pgfqpoint{6.549252in}{3.526068in}}%
\pgfpathlineto{\pgfqpoint{6.549111in}{3.500599in}}%
\pgfpathlineto{\pgfqpoint{6.549235in}{3.475129in}}%
\pgfpathlineto{\pgfqpoint{6.549624in}{3.449660in}}%
\pgfpathlineto{\pgfqpoint{6.550281in}{3.424190in}}%
\pgfpathlineto{\pgfqpoint{6.551205in}{3.398721in}}%
\pgfpathlineto{\pgfqpoint{6.552398in}{3.373251in}}%
\pgfpathlineto{\pgfqpoint{6.553861in}{3.347782in}}%
\pgfpathlineto{\pgfqpoint{6.555596in}{3.322312in}}%
\pgfpathlineto{\pgfqpoint{6.557603in}{3.296843in}}%
\pgfpathlineto{\pgfqpoint{6.559884in}{3.271373in}}%
\pgfpathlineto{\pgfqpoint{6.562120in}{3.249098in}}%
\pgfpathlineto{\pgfqpoint{6.562438in}{3.245904in}}%
\pgfpathlineto{\pgfqpoint{6.565251in}{3.220434in}}%
\pgfpathlineto{\pgfqpoint{6.568340in}{3.194965in}}%
\pgfpathlineto{\pgfqpoint{6.571707in}{3.169495in}}%
\pgfpathlineto{\pgfqpoint{6.575351in}{3.144025in}}%
\pgfpathlineto{\pgfqpoint{6.579276in}{3.118556in}}%
\pgfpathlineto{\pgfqpoint{6.583483in}{3.093086in}}%
\pgfpathlineto{\pgfqpoint{6.587972in}{3.067617in}}%
\pgfpathlineto{\pgfqpoint{6.591693in}{3.047762in}}%
\pgfpathlineto{\pgfqpoint{6.592739in}{3.042147in}}%
\pgfpathlineto{\pgfqpoint{6.597764in}{3.016678in}}%
\pgfpathlineto{\pgfqpoint{6.603076in}{2.991208in}}%
\pgfpathlineto{\pgfqpoint{6.608675in}{2.965739in}}%
\pgfpathlineto{\pgfqpoint{6.614563in}{2.940269in}}%
\pgfpathlineto{\pgfqpoint{6.620742in}{2.914800in}}%
\pgfpathlineto{\pgfqpoint{6.621267in}{2.912731in}}%
\pgfpathlineto{\pgfqpoint{6.627173in}{2.889330in}}%
\pgfpathlineto{\pgfqpoint{6.633893in}{2.863861in}}%
\pgfpathlineto{\pgfqpoint{6.640907in}{2.838391in}}%
\pgfpathlineto{\pgfqpoint{6.648218in}{2.812922in}}%
\pgfpathlineto{\pgfqpoint{6.650840in}{2.804138in}}%
\pgfpathlineto{\pgfqpoint{6.655793in}{2.787452in}}%
\pgfpathlineto{\pgfqpoint{6.663649in}{2.761983in}}%
\pgfpathlineto{\pgfqpoint{6.671804in}{2.736513in}}%
\pgfpathlineto{\pgfqpoint{6.680262in}{2.711044in}}%
\pgfpathlineto{\pgfqpoint{6.680414in}{2.710603in}}%
\pgfpathlineto{\pgfqpoint{6.688966in}{2.685574in}}%
\pgfpathlineto{\pgfqpoint{6.697974in}{2.660104in}}%
\pgfpathlineto{\pgfqpoint{6.707288in}{2.634635in}}%
\pgfpathlineto{\pgfqpoint{6.709987in}{2.627483in}}%
\pgfpathlineto{\pgfqpoint{6.716864in}{2.609165in}}%
\pgfpathlineto{\pgfqpoint{6.726732in}{2.583696in}}%
\pgfpathlineto{\pgfqpoint{6.736911in}{2.558226in}}%
\pgfpathlineto{\pgfqpoint{6.739560in}{2.551786in}}%
\pgfusepath{stroke}%
\end{pgfscope}%
\begin{pgfscope}%
\pgfpathrectangle{\pgfqpoint{0.854460in}{0.571603in}}{\pgfqpoint{5.885100in}{5.068436in}}%
\pgfusepath{clip}%
\pgfsetbuttcap%
\pgfsetroundjoin%
\pgfsetlinewidth{1.505625pt}%
\definecolor{currentstroke}{rgb}{0.352360,0.783011,0.392636}%
\pgfsetstrokecolor{currentstroke}%
\pgfsetdash{}{0pt}%
\pgfpathmoveto{\pgfqpoint{0.854460in}{5.172109in}}%
\pgfpathlineto{\pgfqpoint{0.865663in}{5.181588in}}%
\pgfpathlineto{\pgfqpoint{0.884034in}{5.196943in}}%
\pgfpathlineto{\pgfqpoint{0.896224in}{5.207057in}}%
\pgfpathlineto{\pgfqpoint{0.913607in}{5.221304in}}%
\pgfpathlineto{\pgfqpoint{0.927400in}{5.232527in}}%
\pgfpathlineto{\pgfqpoint{0.943181in}{5.245210in}}%
\pgfpathlineto{\pgfqpoint{0.959202in}{5.257996in}}%
\pgfpathlineto{\pgfqpoint{0.972754in}{5.268680in}}%
\pgfpathlineto{\pgfqpoint{0.991640in}{5.283466in}}%
\pgfpathlineto{\pgfqpoint{1.002327in}{5.291731in}}%
\pgfpathlineto{\pgfqpoint{1.024724in}{5.308935in}}%
\pgfpathlineto{\pgfqpoint{1.031901in}{5.314381in}}%
\pgfpathlineto{\pgfqpoint{1.058465in}{5.334405in}}%
\pgfpathlineto{\pgfqpoint{1.061474in}{5.336646in}}%
\pgfpathlineto{\pgfqpoint{1.091047in}{5.358516in}}%
\pgfpathlineto{\pgfqpoint{1.092897in}{5.359874in}}%
\pgfpathlineto{\pgfqpoint{1.120621in}{5.379984in}}%
\pgfpathlineto{\pgfqpoint{1.128056in}{5.385344in}}%
\pgfpathlineto{\pgfqpoint{1.150194in}{5.401108in}}%
\pgfpathlineto{\pgfqpoint{1.163909in}{5.410813in}}%
\pgfpathlineto{\pgfqpoint{1.179767in}{5.421901in}}%
\pgfpathlineto{\pgfqpoint{1.200463in}{5.436283in}}%
\pgfpathlineto{\pgfqpoint{1.209341in}{5.442378in}}%
\pgfpathlineto{\pgfqpoint{1.237727in}{5.461752in}}%
\pgfpathlineto{\pgfqpoint{1.238914in}{5.462553in}}%
\pgfpathlineto{\pgfqpoint{1.268488in}{5.482349in}}%
\pgfpathlineto{\pgfqpoint{1.275810in}{5.487222in}}%
\pgfpathlineto{\pgfqpoint{1.298061in}{5.501850in}}%
\pgfpathlineto{\pgfqpoint{1.314644in}{5.512691in}}%
\pgfpathlineto{\pgfqpoint{1.327634in}{5.521082in}}%
\pgfpathlineto{\pgfqpoint{1.354219in}{5.538161in}}%
\pgfpathlineto{\pgfqpoint{1.357208in}{5.540058in}}%
\pgfpathlineto{\pgfqpoint{1.386781in}{5.558699in}}%
\pgfpathlineto{\pgfqpoint{1.394649in}{5.563630in}}%
\pgfpathlineto{\pgfqpoint{1.416354in}{5.577069in}}%
\pgfpathlineto{\pgfqpoint{1.435884in}{5.589100in}}%
\pgfpathlineto{\pgfqpoint{1.445928in}{5.595212in}}%
\pgfpathlineto{\pgfqpoint{1.475501in}{5.613115in}}%
\pgfpathlineto{\pgfqpoint{1.477920in}{5.614570in}}%
\pgfpathlineto{\pgfqpoint{1.505074in}{5.630696in}}%
\pgfpathlineto{\pgfqpoint{1.520878in}{5.640039in}}%
\pgfusepath{stroke}%
\end{pgfscope}%
\begin{pgfscope}%
\pgfpathrectangle{\pgfqpoint{0.854460in}{0.571603in}}{\pgfqpoint{5.885100in}{5.068436in}}%
\pgfusepath{clip}%
\pgfsetbuttcap%
\pgfsetroundjoin%
\pgfsetlinewidth{1.505625pt}%
\definecolor{currentstroke}{rgb}{0.404001,0.800275,0.362552}%
\pgfsetstrokecolor{currentstroke}%
\pgfsetdash{}{0pt}%
\pgfpathmoveto{\pgfqpoint{6.739560in}{4.297611in}}%
\pgfpathlineto{\pgfqpoint{6.737266in}{4.290154in}}%
\pgfpathlineto{\pgfqpoint{6.729642in}{4.264685in}}%
\pgfpathlineto{\pgfqpoint{6.722254in}{4.239215in}}%
\pgfpathlineto{\pgfqpoint{6.715102in}{4.213746in}}%
\pgfpathlineto{\pgfqpoint{6.709987in}{4.194919in}}%
\pgfpathlineto{\pgfqpoint{6.708176in}{4.188276in}}%
\pgfpathlineto{\pgfqpoint{6.701453in}{4.162807in}}%
\pgfpathlineto{\pgfqpoint{6.694970in}{4.137337in}}%
\pgfpathlineto{\pgfqpoint{6.688730in}{4.111867in}}%
\pgfpathlineto{\pgfqpoint{6.682731in}{4.086398in}}%
\pgfpathlineto{\pgfqpoint{6.680414in}{4.076163in}}%
\pgfpathlineto{\pgfqpoint{6.676951in}{4.060928in}}%
\pgfpathlineto{\pgfqpoint{6.671400in}{4.035459in}}%
\pgfpathlineto{\pgfqpoint{6.666095in}{4.009989in}}%
\pgfpathlineto{\pgfqpoint{6.661037in}{3.984520in}}%
\pgfpathlineto{\pgfqpoint{6.656226in}{3.959050in}}%
\pgfpathlineto{\pgfqpoint{6.651664in}{3.933581in}}%
\pgfpathlineto{\pgfqpoint{6.650840in}{3.928726in}}%
\pgfpathlineto{\pgfqpoint{6.647326in}{3.908111in}}%
\pgfpathlineto{\pgfqpoint{6.643234in}{3.882642in}}%
\pgfpathlineto{\pgfqpoint{6.639394in}{3.857172in}}%
\pgfpathlineto{\pgfqpoint{6.635807in}{3.831703in}}%
\pgfpathlineto{\pgfqpoint{6.632474in}{3.806233in}}%
\pgfpathlineto{\pgfqpoint{6.629395in}{3.780764in}}%
\pgfpathlineto{\pgfqpoint{6.626571in}{3.755294in}}%
\pgfpathlineto{\pgfqpoint{6.624003in}{3.729825in}}%
\pgfpathlineto{\pgfqpoint{6.621693in}{3.704355in}}%
\pgfpathlineto{\pgfqpoint{6.621267in}{3.699081in}}%
\pgfpathlineto{\pgfqpoint{6.619628in}{3.678886in}}%
\pgfpathlineto{\pgfqpoint{6.617821in}{3.653416in}}%
\pgfpathlineto{\pgfqpoint{6.616275in}{3.627946in}}%
\pgfpathlineto{\pgfqpoint{6.614991in}{3.602477in}}%
\pgfpathlineto{\pgfqpoint{6.613970in}{3.577007in}}%
\pgfpathlineto{\pgfqpoint{6.613212in}{3.551538in}}%
\pgfpathlineto{\pgfqpoint{6.612720in}{3.526068in}}%
\pgfpathlineto{\pgfqpoint{6.612493in}{3.500599in}}%
\pgfpathlineto{\pgfqpoint{6.612533in}{3.475129in}}%
\pgfpathlineto{\pgfqpoint{6.612840in}{3.449660in}}%
\pgfpathlineto{\pgfqpoint{6.613417in}{3.424190in}}%
\pgfpathlineto{\pgfqpoint{6.614263in}{3.398721in}}%
\pgfpathlineto{\pgfqpoint{6.615380in}{3.373251in}}%
\pgfpathlineto{\pgfqpoint{6.616769in}{3.347782in}}%
\pgfpathlineto{\pgfqpoint{6.618431in}{3.322312in}}%
\pgfpathlineto{\pgfqpoint{6.620368in}{3.296843in}}%
\pgfpathlineto{\pgfqpoint{6.621267in}{3.286497in}}%
\pgfpathlineto{\pgfqpoint{6.622571in}{3.271373in}}%
\pgfpathlineto{\pgfqpoint{6.625043in}{3.245904in}}%
\pgfpathlineto{\pgfqpoint{6.627791in}{3.220434in}}%
\pgfpathlineto{\pgfqpoint{6.630817in}{3.194965in}}%
\pgfpathlineto{\pgfqpoint{6.634121in}{3.169495in}}%
\pgfpathlineto{\pgfqpoint{6.637706in}{3.144025in}}%
\pgfpathlineto{\pgfqpoint{6.641572in}{3.118556in}}%
\pgfpathlineto{\pgfqpoint{6.645721in}{3.093086in}}%
\pgfpathlineto{\pgfqpoint{6.650155in}{3.067617in}}%
\pgfpathlineto{\pgfqpoint{6.650840in}{3.063917in}}%
\pgfpathlineto{\pgfqpoint{6.654847in}{3.042147in}}%
\pgfpathlineto{\pgfqpoint{6.659820in}{3.016678in}}%
\pgfpathlineto{\pgfqpoint{6.665081in}{2.991208in}}%
\pgfpathlineto{\pgfqpoint{6.670630in}{2.965739in}}%
\pgfpathlineto{\pgfqpoint{6.676470in}{2.940269in}}%
\pgfpathlineto{\pgfqpoint{6.680414in}{2.923887in}}%
\pgfpathlineto{\pgfqpoint{6.682587in}{2.914800in}}%
\pgfpathlineto{\pgfqpoint{6.688970in}{2.889330in}}%
\pgfpathlineto{\pgfqpoint{6.695647in}{2.863861in}}%
\pgfpathlineto{\pgfqpoint{6.702619in}{2.838391in}}%
\pgfpathlineto{\pgfqpoint{6.709888in}{2.812922in}}%
\pgfpathlineto{\pgfqpoint{6.709987in}{2.812589in}}%
\pgfpathlineto{\pgfqpoint{6.717407in}{2.787452in}}%
\pgfpathlineto{\pgfqpoint{6.725225in}{2.761983in}}%
\pgfpathlineto{\pgfqpoint{6.733343in}{2.736513in}}%
\pgfpathlineto{\pgfqpoint{6.739560in}{2.717706in}}%
\pgfusepath{stroke}%
\end{pgfscope}%
\begin{pgfscope}%
\pgfpathrectangle{\pgfqpoint{0.854460in}{0.571603in}}{\pgfqpoint{5.885100in}{5.068436in}}%
\pgfusepath{clip}%
\pgfsetbuttcap%
\pgfsetroundjoin%
\pgfsetlinewidth{1.505625pt}%
\definecolor{currentstroke}{rgb}{0.404001,0.800275,0.362552}%
\pgfsetstrokecolor{currentstroke}%
\pgfsetdash{}{0pt}%
\pgfpathmoveto{\pgfqpoint{0.854460in}{5.220668in}}%
\pgfpathlineto{\pgfqpoint{0.868815in}{5.232527in}}%
\pgfpathlineto{\pgfqpoint{0.884034in}{5.244947in}}%
\pgfpathlineto{\pgfqpoint{0.900138in}{5.257996in}}%
\pgfpathlineto{\pgfqpoint{0.913607in}{5.268778in}}%
\pgfpathlineto{\pgfqpoint{0.932086in}{5.283466in}}%
\pgfpathlineto{\pgfqpoint{0.943181in}{5.292178in}}%
\pgfpathlineto{\pgfqpoint{0.964669in}{5.308935in}}%
\pgfpathlineto{\pgfqpoint{0.972754in}{5.315164in}}%
\pgfpathlineto{\pgfqpoint{0.997896in}{5.334405in}}%
\pgfpathlineto{\pgfqpoint{1.002327in}{5.337755in}}%
\pgfpathlineto{\pgfqpoint{1.031779in}{5.359874in}}%
\pgfpathlineto{\pgfqpoint{1.031901in}{5.359965in}}%
\pgfpathlineto{\pgfqpoint{1.061474in}{5.381745in}}%
\pgfpathlineto{\pgfqpoint{1.066392in}{5.385344in}}%
\pgfpathlineto{\pgfqpoint{1.091047in}{5.403167in}}%
\pgfpathlineto{\pgfqpoint{1.101690in}{5.410813in}}%
\pgfpathlineto{\pgfqpoint{1.120621in}{5.424250in}}%
\pgfpathlineto{\pgfqpoint{1.137678in}{5.436283in}}%
\pgfpathlineto{\pgfqpoint{1.150194in}{5.445006in}}%
\pgfpathlineto{\pgfqpoint{1.174366in}{5.461752in}}%
\pgfpathlineto{\pgfqpoint{1.179767in}{5.465450in}}%
\pgfpathlineto{\pgfqpoint{1.209341in}{5.485565in}}%
\pgfpathlineto{\pgfqpoint{1.211794in}{5.487222in}}%
\pgfpathlineto{\pgfqpoint{1.238914in}{5.505321in}}%
\pgfpathlineto{\pgfqpoint{1.250021in}{5.512691in}}%
\pgfpathlineto{\pgfqpoint{1.268488in}{5.524799in}}%
\pgfpathlineto{\pgfqpoint{1.288978in}{5.538161in}}%
\pgfpathlineto{\pgfqpoint{1.298061in}{5.544012in}}%
\pgfpathlineto{\pgfqpoint{1.327634in}{5.562961in}}%
\pgfpathlineto{\pgfqpoint{1.328687in}{5.563630in}}%
\pgfpathlineto{\pgfqpoint{1.357208in}{5.581554in}}%
\pgfpathlineto{\pgfqpoint{1.369275in}{5.589100in}}%
\pgfpathlineto{\pgfqpoint{1.386781in}{5.599914in}}%
\pgfpathlineto{\pgfqpoint{1.410620in}{5.614570in}}%
\pgfpathlineto{\pgfqpoint{1.416354in}{5.618052in}}%
\pgfpathlineto{\pgfqpoint{1.445928in}{5.635902in}}%
\pgfpathlineto{\pgfqpoint{1.452822in}{5.640039in}}%
\pgfusepath{stroke}%
\end{pgfscope}%
\begin{pgfscope}%
\pgfpathrectangle{\pgfqpoint{0.854460in}{0.571603in}}{\pgfqpoint{5.885100in}{5.068436in}}%
\pgfusepath{clip}%
\pgfsetbuttcap%
\pgfsetroundjoin%
\pgfsetlinewidth{1.505625pt}%
\definecolor{currentstroke}{rgb}{0.449368,0.813768,0.335384}%
\pgfsetstrokecolor{currentstroke}%
\pgfsetdash{}{0pt}%
\pgfpathmoveto{\pgfqpoint{6.739560in}{4.051473in}}%
\pgfpathlineto{\pgfqpoint{6.735991in}{4.035459in}}%
\pgfpathlineto{\pgfqpoint{6.730556in}{4.009989in}}%
\pgfpathlineto{\pgfqpoint{6.725371in}{3.984520in}}%
\pgfpathlineto{\pgfqpoint{6.720436in}{3.959050in}}%
\pgfpathlineto{\pgfqpoint{6.715753in}{3.933581in}}%
\pgfpathlineto{\pgfqpoint{6.711321in}{3.908111in}}%
\pgfpathlineto{\pgfqpoint{6.709987in}{3.899997in}}%
\pgfpathlineto{\pgfqpoint{6.707122in}{3.882642in}}%
\pgfpathlineto{\pgfqpoint{6.703168in}{3.857172in}}%
\pgfpathlineto{\pgfqpoint{6.699470in}{3.831703in}}%
\pgfpathlineto{\pgfqpoint{6.696029in}{3.806233in}}%
\pgfpathlineto{\pgfqpoint{6.692844in}{3.780764in}}%
\pgfpathlineto{\pgfqpoint{6.689917in}{3.755294in}}%
\pgfpathlineto{\pgfqpoint{6.687249in}{3.729825in}}%
\pgfpathlineto{\pgfqpoint{6.684839in}{3.704355in}}%
\pgfpathlineto{\pgfqpoint{6.682690in}{3.678886in}}%
\pgfpathlineto{\pgfqpoint{6.680801in}{3.653416in}}%
\pgfpathlineto{\pgfqpoint{6.680414in}{3.647365in}}%
\pgfpathlineto{\pgfqpoint{6.679165in}{3.627946in}}%
\pgfpathlineto{\pgfqpoint{6.677790in}{3.602477in}}%
\pgfpathlineto{\pgfqpoint{6.676681in}{3.577007in}}%
\pgfpathlineto{\pgfqpoint{6.675838in}{3.551538in}}%
\pgfpathlineto{\pgfqpoint{6.675261in}{3.526068in}}%
\pgfpathlineto{\pgfqpoint{6.674952in}{3.500599in}}%
\pgfpathlineto{\pgfqpoint{6.674911in}{3.475129in}}%
\pgfpathlineto{\pgfqpoint{6.675140in}{3.449660in}}%
\pgfpathlineto{\pgfqpoint{6.675640in}{3.424190in}}%
\pgfpathlineto{\pgfqpoint{6.676411in}{3.398721in}}%
\pgfpathlineto{\pgfqpoint{6.677455in}{3.373251in}}%
\pgfpathlineto{\pgfqpoint{6.678773in}{3.347782in}}%
\pgfpathlineto{\pgfqpoint{6.680365in}{3.322312in}}%
\pgfpathlineto{\pgfqpoint{6.680414in}{3.321654in}}%
\pgfpathlineto{\pgfqpoint{6.682221in}{3.296843in}}%
\pgfpathlineto{\pgfqpoint{6.684353in}{3.271373in}}%
\pgfpathlineto{\pgfqpoint{6.686760in}{3.245904in}}%
\pgfpathlineto{\pgfqpoint{6.689446in}{3.220434in}}%
\pgfpathlineto{\pgfqpoint{6.692410in}{3.194965in}}%
\pgfpathlineto{\pgfqpoint{6.695655in}{3.169495in}}%
\pgfpathlineto{\pgfqpoint{6.699181in}{3.144025in}}%
\pgfpathlineto{\pgfqpoint{6.702990in}{3.118556in}}%
\pgfpathlineto{\pgfqpoint{6.707083in}{3.093086in}}%
\pgfpathlineto{\pgfqpoint{6.709987in}{3.076200in}}%
\pgfpathlineto{\pgfqpoint{6.711453in}{3.067617in}}%
\pgfpathlineto{\pgfqpoint{6.716089in}{3.042147in}}%
\pgfpathlineto{\pgfqpoint{6.721011in}{3.016678in}}%
\pgfpathlineto{\pgfqpoint{6.726222in}{2.991208in}}%
\pgfpathlineto{\pgfqpoint{6.731723in}{2.965739in}}%
\pgfpathlineto{\pgfqpoint{6.737516in}{2.940269in}}%
\pgfpathlineto{\pgfqpoint{6.739560in}{2.931709in}}%
\pgfusepath{stroke}%
\end{pgfscope}%
\begin{pgfscope}%
\pgfpathrectangle{\pgfqpoint{0.854460in}{0.571603in}}{\pgfqpoint{5.885100in}{5.068436in}}%
\pgfusepath{clip}%
\pgfsetbuttcap%
\pgfsetroundjoin%
\pgfsetlinewidth{1.505625pt}%
\definecolor{currentstroke}{rgb}{0.449368,0.813768,0.335384}%
\pgfsetstrokecolor{currentstroke}%
\pgfsetdash{}{0pt}%
\pgfpathmoveto{\pgfqpoint{0.854460in}{5.267580in}}%
\pgfpathlineto{\pgfqpoint{0.874148in}{5.283466in}}%
\pgfpathlineto{\pgfqpoint{0.884034in}{5.291346in}}%
\pgfpathlineto{\pgfqpoint{0.906252in}{5.308935in}}%
\pgfpathlineto{\pgfqpoint{0.913607in}{5.314687in}}%
\pgfpathlineto{\pgfqpoint{0.938990in}{5.334405in}}%
\pgfpathlineto{\pgfqpoint{0.943181in}{5.337620in}}%
\pgfpathlineto{\pgfqpoint{0.972372in}{5.359874in}}%
\pgfpathlineto{\pgfqpoint{0.972754in}{5.360162in}}%
\pgfpathlineto{\pgfqpoint{1.002327in}{5.382272in}}%
\pgfpathlineto{\pgfqpoint{1.006462in}{5.385344in}}%
\pgfpathlineto{\pgfqpoint{1.031901in}{5.404011in}}%
\pgfpathlineto{\pgfqpoint{1.041229in}{5.410813in}}%
\pgfpathlineto{\pgfqpoint{1.061474in}{5.425398in}}%
\pgfpathlineto{\pgfqpoint{1.076676in}{5.436283in}}%
\pgfpathlineto{\pgfqpoint{1.091047in}{5.446449in}}%
\pgfpathlineto{\pgfqpoint{1.112811in}{5.461752in}}%
\pgfpathlineto{\pgfqpoint{1.120621in}{5.467178in}}%
\pgfpathlineto{\pgfqpoint{1.149642in}{5.487222in}}%
\pgfpathlineto{\pgfqpoint{1.150194in}{5.487598in}}%
\pgfpathlineto{\pgfqpoint{1.179767in}{5.507632in}}%
\pgfpathlineto{\pgfqpoint{1.187279in}{5.512691in}}%
\pgfpathlineto{\pgfqpoint{1.209341in}{5.527372in}}%
\pgfpathlineto{\pgfqpoint{1.225644in}{5.538161in}}%
\pgfpathlineto{\pgfqpoint{1.238914in}{5.546838in}}%
\pgfpathlineto{\pgfqpoint{1.264735in}{5.563630in}}%
\pgfpathlineto{\pgfqpoint{1.268488in}{5.566041in}}%
\pgfpathlineto{\pgfqpoint{1.298061in}{5.584919in}}%
\pgfpathlineto{\pgfqpoint{1.304649in}{5.589100in}}%
\pgfpathlineto{\pgfqpoint{1.327634in}{5.603510in}}%
\pgfpathlineto{\pgfqpoint{1.345363in}{5.614570in}}%
\pgfpathlineto{\pgfqpoint{1.357208in}{5.621870in}}%
\pgfpathlineto{\pgfqpoint{1.386781in}{5.640009in}}%
\pgfpathlineto{\pgfqpoint{1.386830in}{5.640039in}}%
\pgfusepath{stroke}%
\end{pgfscope}%
\begin{pgfscope}%
\pgfpathrectangle{\pgfqpoint{0.854460in}{0.571603in}}{\pgfqpoint{5.885100in}{5.068436in}}%
\pgfusepath{clip}%
\pgfsetbuttcap%
\pgfsetroundjoin%
\pgfsetlinewidth{1.505625pt}%
\definecolor{currentstroke}{rgb}{0.496615,0.826376,0.306377}%
\pgfsetstrokecolor{currentstroke}%
\pgfsetdash{}{0pt}%
\pgfpathmoveto{\pgfqpoint{6.739560in}{3.599678in}}%
\pgfpathlineto{\pgfqpoint{6.738498in}{3.577007in}}%
\pgfpathlineto{\pgfqpoint{6.737572in}{3.551538in}}%
\pgfpathlineto{\pgfqpoint{6.736914in}{3.526068in}}%
\pgfpathlineto{\pgfqpoint{6.736526in}{3.500599in}}%
\pgfpathlineto{\pgfqpoint{6.736409in}{3.475129in}}%
\pgfpathlineto{\pgfqpoint{6.736562in}{3.449660in}}%
\pgfpathlineto{\pgfqpoint{6.736988in}{3.424190in}}%
\pgfpathlineto{\pgfqpoint{6.737688in}{3.398721in}}%
\pgfpathlineto{\pgfqpoint{6.738661in}{3.373251in}}%
\pgfpathlineto{\pgfqpoint{6.739560in}{3.354921in}}%
\pgfusepath{stroke}%
\end{pgfscope}%
\begin{pgfscope}%
\pgfpathrectangle{\pgfqpoint{0.854460in}{0.571603in}}{\pgfqpoint{5.885100in}{5.068436in}}%
\pgfusepath{clip}%
\pgfsetbuttcap%
\pgfsetroundjoin%
\pgfsetlinewidth{1.505625pt}%
\definecolor{currentstroke}{rgb}{0.496615,0.826376,0.306377}%
\pgfsetstrokecolor{currentstroke}%
\pgfsetdash{}{0pt}%
\pgfpathmoveto{\pgfqpoint{0.854460in}{5.312975in}}%
\pgfpathlineto{\pgfqpoint{0.881642in}{5.334405in}}%
\pgfpathlineto{\pgfqpoint{0.884034in}{5.336268in}}%
\pgfpathlineto{\pgfqpoint{0.913607in}{5.359145in}}%
\pgfpathlineto{\pgfqpoint{0.914557in}{5.359874in}}%
\pgfpathlineto{\pgfqpoint{0.943181in}{5.381593in}}%
\pgfpathlineto{\pgfqpoint{0.948156in}{5.385344in}}%
\pgfpathlineto{\pgfqpoint{0.972754in}{5.403664in}}%
\pgfpathlineto{\pgfqpoint{0.982415in}{5.410813in}}%
\pgfpathlineto{\pgfqpoint{1.002327in}{5.425373in}}%
\pgfpathlineto{\pgfqpoint{1.017342in}{5.436283in}}%
\pgfpathlineto{\pgfqpoint{1.031901in}{5.446734in}}%
\pgfpathlineto{\pgfqpoint{1.052947in}{5.461752in}}%
\pgfpathlineto{\pgfqpoint{1.061474in}{5.467764in}}%
\pgfpathlineto{\pgfqpoint{1.089238in}{5.487222in}}%
\pgfpathlineto{\pgfqpoint{1.091047in}{5.488474in}}%
\pgfpathlineto{\pgfqpoint{1.120621in}{5.508810in}}%
\pgfpathlineto{\pgfqpoint{1.126299in}{5.512691in}}%
\pgfpathlineto{\pgfqpoint{1.150194in}{5.528827in}}%
\pgfpathlineto{\pgfqpoint{1.164094in}{5.538161in}}%
\pgfpathlineto{\pgfqpoint{1.179767in}{5.548559in}}%
\pgfpathlineto{\pgfqpoint{1.202607in}{5.563630in}}%
\pgfpathlineto{\pgfqpoint{1.209341in}{5.568021in}}%
\pgfpathlineto{\pgfqpoint{1.238914in}{5.587189in}}%
\pgfpathlineto{\pgfqpoint{1.241882in}{5.589100in}}%
\pgfpathlineto{\pgfqpoint{1.268488in}{5.606025in}}%
\pgfpathlineto{\pgfqpoint{1.281987in}{5.614570in}}%
\pgfpathlineto{\pgfqpoint{1.298061in}{5.624621in}}%
\pgfpathlineto{\pgfqpoint{1.322837in}{5.640039in}}%
\pgfusepath{stroke}%
\end{pgfscope}%
\begin{pgfscope}%
\pgfpathrectangle{\pgfqpoint{0.854460in}{0.571603in}}{\pgfqpoint{5.885100in}{5.068436in}}%
\pgfusepath{clip}%
\pgfsetbuttcap%
\pgfsetroundjoin%
\pgfsetlinewidth{1.505625pt}%
\definecolor{currentstroke}{rgb}{0.555484,0.840254,0.269281}%
\pgfsetstrokecolor{currentstroke}%
\pgfsetdash{}{0pt}%
\pgfpathmoveto{\pgfqpoint{0.854460in}{5.356923in}}%
\pgfpathlineto{\pgfqpoint{0.858247in}{5.359874in}}%
\pgfpathlineto{\pgfqpoint{0.884034in}{5.379731in}}%
\pgfpathlineto{\pgfqpoint{0.891370in}{5.385344in}}%
\pgfpathlineto{\pgfqpoint{0.913607in}{5.402151in}}%
\pgfpathlineto{\pgfqpoint{0.925143in}{5.410813in}}%
\pgfpathlineto{\pgfqpoint{0.943181in}{5.424197in}}%
\pgfpathlineto{\pgfqpoint{0.959573in}{5.436283in}}%
\pgfpathlineto{\pgfqpoint{0.972754in}{5.445885in}}%
\pgfpathlineto{\pgfqpoint{0.994670in}{5.461752in}}%
\pgfpathlineto{\pgfqpoint{1.002327in}{5.467230in}}%
\pgfpathlineto{\pgfqpoint{1.030443in}{5.487222in}}%
\pgfpathlineto{\pgfqpoint{1.031901in}{5.488246in}}%
\pgfpathlineto{\pgfqpoint{1.061474in}{5.508879in}}%
\pgfpathlineto{\pgfqpoint{1.066971in}{5.512691in}}%
\pgfpathlineto{\pgfqpoint{1.091047in}{5.529187in}}%
\pgfpathlineto{\pgfqpoint{1.104219in}{5.538161in}}%
\pgfpathlineto{\pgfqpoint{1.120621in}{5.549201in}}%
\pgfpathlineto{\pgfqpoint{1.142174in}{5.563630in}}%
\pgfpathlineto{\pgfqpoint{1.150194in}{5.568935in}}%
\pgfpathlineto{\pgfqpoint{1.179767in}{5.588388in}}%
\pgfpathlineto{\pgfqpoint{1.180857in}{5.589100in}}%
\pgfpathlineto{\pgfqpoint{1.209341in}{5.607483in}}%
\pgfpathlineto{\pgfqpoint{1.220377in}{5.614570in}}%
\pgfpathlineto{\pgfqpoint{1.238914in}{5.626330in}}%
\pgfpathlineto{\pgfqpoint{1.260631in}{5.640039in}}%
\pgfusepath{stroke}%
\end{pgfscope}%
\begin{pgfscope}%
\pgfpathrectangle{\pgfqpoint{0.854460in}{0.571603in}}{\pgfqpoint{5.885100in}{5.068436in}}%
\pgfusepath{clip}%
\pgfsetbuttcap%
\pgfsetroundjoin%
\pgfsetlinewidth{1.505625pt}%
\definecolor{currentstroke}{rgb}{0.606045,0.850733,0.236712}%
\pgfsetstrokecolor{currentstroke}%
\pgfsetdash{}{0pt}%
\pgfpathmoveto{\pgfqpoint{0.854460in}{5.399493in}}%
\pgfpathlineto{\pgfqpoint{0.869317in}{5.410813in}}%
\pgfpathlineto{\pgfqpoint{0.884034in}{5.421892in}}%
\pgfpathlineto{\pgfqpoint{0.903271in}{5.436283in}}%
\pgfpathlineto{\pgfqpoint{0.913607in}{5.443922in}}%
\pgfpathlineto{\pgfqpoint{0.937881in}{5.461752in}}%
\pgfpathlineto{\pgfqpoint{0.943181in}{5.465599in}}%
\pgfpathlineto{\pgfqpoint{0.972754in}{5.486931in}}%
\pgfpathlineto{\pgfqpoint{0.973161in}{5.487222in}}%
\pgfpathlineto{\pgfqpoint{1.002327in}{5.507861in}}%
\pgfpathlineto{\pgfqpoint{1.009193in}{5.512691in}}%
\pgfpathlineto{\pgfqpoint{1.031901in}{5.528475in}}%
\pgfpathlineto{\pgfqpoint{1.045915in}{5.538161in}}%
\pgfpathlineto{\pgfqpoint{1.061474in}{5.548786in}}%
\pgfpathlineto{\pgfqpoint{1.083333in}{5.563630in}}%
\pgfpathlineto{\pgfqpoint{1.091047in}{5.568806in}}%
\pgfpathlineto{\pgfqpoint{1.120621in}{5.588539in}}%
\pgfpathlineto{\pgfqpoint{1.121467in}{5.589100in}}%
\pgfpathlineto{\pgfqpoint{1.150194in}{5.607908in}}%
\pgfpathlineto{\pgfqpoint{1.160422in}{5.614570in}}%
\pgfpathlineto{\pgfqpoint{1.179767in}{5.627018in}}%
\pgfpathlineto{\pgfqpoint{1.200103in}{5.640039in}}%
\pgfusepath{stroke}%
\end{pgfscope}%
\begin{pgfscope}%
\pgfpathrectangle{\pgfqpoint{0.854460in}{0.571603in}}{\pgfqpoint{5.885100in}{5.068436in}}%
\pgfusepath{clip}%
\pgfsetbuttcap%
\pgfsetroundjoin%
\pgfsetlinewidth{1.505625pt}%
\definecolor{currentstroke}{rgb}{0.668054,0.861999,0.196293}%
\pgfsetstrokecolor{currentstroke}%
\pgfsetdash{}{0pt}%
\pgfpathmoveto{\pgfqpoint{0.854460in}{5.440868in}}%
\pgfpathlineto{\pgfqpoint{0.882488in}{5.461752in}}%
\pgfpathlineto{\pgfqpoint{0.884034in}{5.462891in}}%
\pgfpathlineto{\pgfqpoint{0.913607in}{5.484515in}}%
\pgfpathlineto{\pgfqpoint{0.917332in}{5.487222in}}%
\pgfpathlineto{\pgfqpoint{0.943181in}{5.505777in}}%
\pgfpathlineto{\pgfqpoint{0.952869in}{5.512691in}}%
\pgfpathlineto{\pgfqpoint{0.972754in}{5.526712in}}%
\pgfpathlineto{\pgfqpoint{0.989085in}{5.538161in}}%
\pgfpathlineto{\pgfqpoint{1.002327in}{5.547334in}}%
\pgfpathlineto{\pgfqpoint{1.025987in}{5.563630in}}%
\pgfpathlineto{\pgfqpoint{1.031901in}{5.567655in}}%
\pgfpathlineto{\pgfqpoint{1.061474in}{5.587663in}}%
\pgfpathlineto{\pgfqpoint{1.063612in}{5.589100in}}%
\pgfpathlineto{\pgfqpoint{1.091047in}{5.607319in}}%
\pgfpathlineto{\pgfqpoint{1.102023in}{5.614570in}}%
\pgfpathlineto{\pgfqpoint{1.120621in}{5.626708in}}%
\pgfpathlineto{\pgfqpoint{1.141151in}{5.640039in}}%
\pgfusepath{stroke}%
\end{pgfscope}%
\begin{pgfscope}%
\pgfpathrectangle{\pgfqpoint{0.854460in}{0.571603in}}{\pgfqpoint{5.885100in}{5.068436in}}%
\pgfusepath{clip}%
\pgfsetbuttcap%
\pgfsetroundjoin%
\pgfsetlinewidth{1.505625pt}%
\definecolor{currentstroke}{rgb}{0.720391,0.870350,0.162603}%
\pgfsetstrokecolor{currentstroke}%
\pgfsetdash{}{0pt}%
\pgfpathmoveto{\pgfqpoint{0.854460in}{5.481038in}}%
\pgfpathlineto{\pgfqpoint{0.862851in}{5.487222in}}%
\pgfpathlineto{\pgfqpoint{0.884034in}{5.502647in}}%
\pgfpathlineto{\pgfqpoint{0.897911in}{5.512691in}}%
\pgfpathlineto{\pgfqpoint{0.913607in}{5.523917in}}%
\pgfpathlineto{\pgfqpoint{0.933640in}{5.538161in}}%
\pgfpathlineto{\pgfqpoint{0.943181in}{5.544863in}}%
\pgfpathlineto{\pgfqpoint{0.970046in}{5.563630in}}%
\pgfpathlineto{\pgfqpoint{0.972754in}{5.565500in}}%
\pgfpathlineto{\pgfqpoint{1.002327in}{5.585780in}}%
\pgfpathlineto{\pgfqpoint{1.007198in}{5.589100in}}%
\pgfpathlineto{\pgfqpoint{1.031901in}{5.605737in}}%
\pgfpathlineto{\pgfqpoint{1.045086in}{5.614570in}}%
\pgfpathlineto{\pgfqpoint{1.061474in}{5.625417in}}%
\pgfpathlineto{\pgfqpoint{1.083680in}{5.640039in}}%
\pgfusepath{stroke}%
\end{pgfscope}%
\begin{pgfscope}%
\pgfpathrectangle{\pgfqpoint{0.854460in}{0.571603in}}{\pgfqpoint{5.885100in}{5.068436in}}%
\pgfusepath{clip}%
\pgfsetbuttcap%
\pgfsetroundjoin%
\pgfsetlinewidth{1.505625pt}%
\definecolor{currentstroke}{rgb}{0.783315,0.879285,0.125405}%
\pgfsetstrokecolor{currentstroke}%
\pgfsetdash{}{0pt}%
\pgfpathmoveto{\pgfqpoint{0.854460in}{5.520108in}}%
\pgfpathlineto{\pgfqpoint{0.879497in}{5.538161in}}%
\pgfpathlineto{\pgfqpoint{0.884034in}{5.541393in}}%
\pgfpathlineto{\pgfqpoint{0.913607in}{5.562335in}}%
\pgfpathlineto{\pgfqpoint{0.915448in}{5.563630in}}%
\pgfpathlineto{\pgfqpoint{0.943181in}{5.582907in}}%
\pgfpathlineto{\pgfqpoint{0.952140in}{5.589100in}}%
\pgfpathlineto{\pgfqpoint{0.972754in}{5.603179in}}%
\pgfpathlineto{\pgfqpoint{0.989523in}{5.614570in}}%
\pgfpathlineto{\pgfqpoint{1.002327in}{5.623164in}}%
\pgfpathlineto{\pgfqpoint{1.027602in}{5.640039in}}%
\pgfusepath{stroke}%
\end{pgfscope}%
\begin{pgfscope}%
\pgfpathrectangle{\pgfqpoint{0.854460in}{0.571603in}}{\pgfqpoint{5.885100in}{5.068436in}}%
\pgfusepath{clip}%
\pgfsetbuttcap%
\pgfsetroundjoin%
\pgfsetlinewidth{1.505625pt}%
\definecolor{currentstroke}{rgb}{0.835270,0.886029,0.102646}%
\pgfsetstrokecolor{currentstroke}%
\pgfsetdash{}{0pt}%
\pgfpathmoveto{\pgfqpoint{0.854460in}{5.558151in}}%
\pgfpathlineto{\pgfqpoint{0.862142in}{5.563630in}}%
\pgfpathlineto{\pgfqpoint{0.884034in}{5.579061in}}%
\pgfpathlineto{\pgfqpoint{0.898357in}{5.589100in}}%
\pgfpathlineto{\pgfqpoint{0.913607in}{5.599662in}}%
\pgfpathlineto{\pgfqpoint{0.935252in}{5.614570in}}%
\pgfpathlineto{\pgfqpoint{0.943181in}{5.619966in}}%
\pgfpathlineto{\pgfqpoint{0.972754in}{5.639985in}}%
\pgfpathlineto{\pgfqpoint{0.972835in}{5.640039in}}%
\pgfusepath{stroke}%
\end{pgfscope}%
\begin{pgfscope}%
\pgfpathrectangle{\pgfqpoint{0.854460in}{0.571603in}}{\pgfqpoint{5.885100in}{5.068436in}}%
\pgfusepath{clip}%
\pgfsetbuttcap%
\pgfsetroundjoin%
\pgfsetlinewidth{1.505625pt}%
\definecolor{currentstroke}{rgb}{0.896320,0.893616,0.096335}%
\pgfsetstrokecolor{currentstroke}%
\pgfsetdash{}{0pt}%
\pgfpathmoveto{\pgfqpoint{0.854460in}{5.595201in}}%
\pgfpathlineto{\pgfqpoint{0.882197in}{5.614570in}}%
\pgfpathlineto{\pgfqpoint{0.884034in}{5.615837in}}%
\pgfpathlineto{\pgfqpoint{0.913607in}{5.636110in}}%
\pgfpathlineto{\pgfqpoint{0.919371in}{5.640039in}}%
\pgfusepath{stroke}%
\end{pgfscope}%
\begin{pgfscope}%
\pgfpathrectangle{\pgfqpoint{0.854460in}{0.571603in}}{\pgfqpoint{5.885100in}{5.068436in}}%
\pgfusepath{clip}%
\pgfsetbuttcap%
\pgfsetroundjoin%
\pgfsetlinewidth{1.505625pt}%
\definecolor{currentstroke}{rgb}{0.945636,0.899815,0.112838}%
\pgfsetstrokecolor{currentstroke}%
\pgfsetdash{}{0pt}%
\pgfpathmoveto{\pgfqpoint{0.854460in}{5.631318in}}%
\pgfpathlineto{\pgfqpoint{0.867082in}{5.640039in}}%
\pgfusepath{stroke}%
\end{pgfscope}%
\begin{pgfscope}%
\pgfpathrectangle{\pgfqpoint{0.854460in}{0.571603in}}{\pgfqpoint{5.885100in}{5.068436in}}%
\pgfusepath{clip}%
\pgfsetrectcap%
\pgfsetroundjoin%
\pgfsetlinewidth{1.505625pt}%
\definecolor{currentstroke}{rgb}{1.000000,0.000000,0.000000}%
\pgfsetstrokecolor{currentstroke}%
\pgfsetdash{}{0pt}%
\pgfpathmoveto{\pgfqpoint{5.758710in}{4.372930in}}%
\pgfpathlineto{\pgfqpoint{2.544593in}{5.451593in}}%
\pgfpathlineto{\pgfqpoint{3.593593in}{1.518217in}}%
\pgfpathlineto{\pgfqpoint{3.439286in}{1.136257in}}%
\pgfpathlineto{\pgfqpoint{3.586714in}{1.409049in}}%
\pgfpathlineto{\pgfqpoint{3.547290in}{1.191602in}}%
\pgfpathlineto{\pgfqpoint{3.588765in}{1.272658in}}%
\pgfpathlineto{\pgfqpoint{3.575674in}{1.222389in}}%
\pgfpathlineto{\pgfqpoint{3.580248in}{1.232175in}}%
\pgfpathlineto{\pgfqpoint{3.576934in}{1.227839in}}%
\pgfpathlineto{\pgfqpoint{3.577534in}{1.228875in}}%
\pgfpathlineto{\pgfqpoint{3.577658in}{1.227971in}}%
\pgfusepath{stroke}%
\end{pgfscope}%
\begin{pgfscope}%
\pgfpathrectangle{\pgfqpoint{0.854460in}{0.571603in}}{\pgfqpoint{5.885100in}{5.068436in}}%
\pgfusepath{clip}%
\pgfsetbuttcap%
\pgfsetroundjoin%
\definecolor{currentfill}{rgb}{1.000000,0.000000,0.000000}%
\pgfsetfillcolor{currentfill}%
\pgfsetlinewidth{1.003750pt}%
\definecolor{currentstroke}{rgb}{1.000000,0.000000,0.000000}%
\pgfsetstrokecolor{currentstroke}%
\pgfsetdash{}{0pt}%
\pgfsys@defobject{currentmarker}{\pgfqpoint{-0.041667in}{-0.041667in}}{\pgfqpoint{0.041667in}{0.041667in}}{%
\pgfpathmoveto{\pgfqpoint{0.000000in}{-0.041667in}}%
\pgfpathcurveto{\pgfqpoint{0.011050in}{-0.041667in}}{\pgfqpoint{0.021649in}{-0.037276in}}{\pgfqpoint{0.029463in}{-0.029463in}}%
\pgfpathcurveto{\pgfqpoint{0.037276in}{-0.021649in}}{\pgfqpoint{0.041667in}{-0.011050in}}{\pgfqpoint{0.041667in}{0.000000in}}%
\pgfpathcurveto{\pgfqpoint{0.041667in}{0.011050in}}{\pgfqpoint{0.037276in}{0.021649in}}{\pgfqpoint{0.029463in}{0.029463in}}%
\pgfpathcurveto{\pgfqpoint{0.021649in}{0.037276in}}{\pgfqpoint{0.011050in}{0.041667in}}{\pgfqpoint{0.000000in}{0.041667in}}%
\pgfpathcurveto{\pgfqpoint{-0.011050in}{0.041667in}}{\pgfqpoint{-0.021649in}{0.037276in}}{\pgfqpoint{-0.029463in}{0.029463in}}%
\pgfpathcurveto{\pgfqpoint{-0.037276in}{0.021649in}}{\pgfqpoint{-0.041667in}{0.011050in}}{\pgfqpoint{-0.041667in}{0.000000in}}%
\pgfpathcurveto{\pgfqpoint{-0.041667in}{-0.011050in}}{\pgfqpoint{-0.037276in}{-0.021649in}}{\pgfqpoint{-0.029463in}{-0.029463in}}%
\pgfpathcurveto{\pgfqpoint{-0.021649in}{-0.037276in}}{\pgfqpoint{-0.011050in}{-0.041667in}}{\pgfqpoint{0.000000in}{-0.041667in}}%
\pgfpathlineto{\pgfqpoint{0.000000in}{-0.041667in}}%
\pgfpathclose%
\pgfusepath{stroke,fill}%
}%
\begin{pgfscope}%
\pgfsys@transformshift{5.758710in}{4.372930in}%
\pgfsys@useobject{currentmarker}{}%
\end{pgfscope}%
\begin{pgfscope}%
\pgfsys@transformshift{2.544593in}{5.451593in}%
\pgfsys@useobject{currentmarker}{}%
\end{pgfscope}%
\begin{pgfscope}%
\pgfsys@transformshift{3.593593in}{1.518217in}%
\pgfsys@useobject{currentmarker}{}%
\end{pgfscope}%
\begin{pgfscope}%
\pgfsys@transformshift{3.439286in}{1.136257in}%
\pgfsys@useobject{currentmarker}{}%
\end{pgfscope}%
\begin{pgfscope}%
\pgfsys@transformshift{3.586714in}{1.409049in}%
\pgfsys@useobject{currentmarker}{}%
\end{pgfscope}%
\begin{pgfscope}%
\pgfsys@transformshift{3.547290in}{1.191602in}%
\pgfsys@useobject{currentmarker}{}%
\end{pgfscope}%
\begin{pgfscope}%
\pgfsys@transformshift{3.588765in}{1.272658in}%
\pgfsys@useobject{currentmarker}{}%
\end{pgfscope}%
\begin{pgfscope}%
\pgfsys@transformshift{3.575674in}{1.222389in}%
\pgfsys@useobject{currentmarker}{}%
\end{pgfscope}%
\begin{pgfscope}%
\pgfsys@transformshift{3.580248in}{1.232175in}%
\pgfsys@useobject{currentmarker}{}%
\end{pgfscope}%
\begin{pgfscope}%
\pgfsys@transformshift{3.576934in}{1.227839in}%
\pgfsys@useobject{currentmarker}{}%
\end{pgfscope}%
\begin{pgfscope}%
\pgfsys@transformshift{3.577534in}{1.228875in}%
\pgfsys@useobject{currentmarker}{}%
\end{pgfscope}%
\begin{pgfscope}%
\pgfsys@transformshift{3.577658in}{1.227971in}%
\pgfsys@useobject{currentmarker}{}%
\end{pgfscope}%
\end{pgfscope}%
\begin{pgfscope}%
\pgfsetrectcap%
\pgfsetmiterjoin%
\pgfsetlinewidth{0.803000pt}%
\definecolor{currentstroke}{rgb}{0.000000,0.000000,0.000000}%
\pgfsetstrokecolor{currentstroke}%
\pgfsetdash{}{0pt}%
\pgfpathmoveto{\pgfqpoint{0.854460in}{0.571603in}}%
\pgfpathlineto{\pgfqpoint{0.854460in}{5.640039in}}%
\pgfusepath{stroke}%
\end{pgfscope}%
\begin{pgfscope}%
\pgfsetrectcap%
\pgfsetmiterjoin%
\pgfsetlinewidth{0.803000pt}%
\definecolor{currentstroke}{rgb}{0.000000,0.000000,0.000000}%
\pgfsetstrokecolor{currentstroke}%
\pgfsetdash{}{0pt}%
\pgfpathmoveto{\pgfqpoint{6.739560in}{0.571603in}}%
\pgfpathlineto{\pgfqpoint{6.739560in}{5.640039in}}%
\pgfusepath{stroke}%
\end{pgfscope}%
\begin{pgfscope}%
\pgfsetrectcap%
\pgfsetmiterjoin%
\pgfsetlinewidth{0.803000pt}%
\definecolor{currentstroke}{rgb}{0.000000,0.000000,0.000000}%
\pgfsetstrokecolor{currentstroke}%
\pgfsetdash{}{0pt}%
\pgfpathmoveto{\pgfqpoint{0.854460in}{0.571603in}}%
\pgfpathlineto{\pgfqpoint{6.739560in}{0.571603in}}%
\pgfusepath{stroke}%
\end{pgfscope}%
\begin{pgfscope}%
\pgfsetrectcap%
\pgfsetmiterjoin%
\pgfsetlinewidth{0.803000pt}%
\definecolor{currentstroke}{rgb}{0.000000,0.000000,0.000000}%
\pgfsetstrokecolor{currentstroke}%
\pgfsetdash{}{0pt}%
\pgfpathmoveto{\pgfqpoint{0.854460in}{5.640039in}}%
\pgfpathlineto{\pgfqpoint{6.739560in}{5.640039in}}%
\pgfusepath{stroke}%
\end{pgfscope}%
\begin{pgfscope}%
\definecolor{textcolor}{rgb}{0.000000,0.000000,0.000000}%
\pgfsetstrokecolor{textcolor}%
\pgfsetfillcolor{textcolor}%
\pgftext[x=3.797010in,y=5.723372in,,base]{\color{textcolor}\sffamily\fontsize{12.000000}{14.400000}\selectfont 2D Contour Plot}%
\end{pgfscope}%
\begin{pgfscope}%
\pgfsetbuttcap%
\pgfsetmiterjoin%
\definecolor{currentfill}{rgb}{1.000000,1.000000,1.000000}%
\pgfsetfillcolor{currentfill}%
\pgfsetfillopacity{0.800000}%
\pgfsetlinewidth{1.003750pt}%
\definecolor{currentstroke}{rgb}{0.800000,0.800000,0.800000}%
\pgfsetstrokecolor{currentstroke}%
\pgfsetstrokeopacity{0.800000}%
\pgfsetdash{}{0pt}%
\pgfpathmoveto{\pgfqpoint{5.536408in}{5.121213in}}%
\pgfpathlineto{\pgfqpoint{6.642338in}{5.121213in}}%
\pgfpathquadraticcurveto{\pgfqpoint{6.670116in}{5.121213in}}{\pgfqpoint{6.670116in}{5.148991in}}%
\pgfpathlineto{\pgfqpoint{6.670116in}{5.542817in}}%
\pgfpathquadraticcurveto{\pgfqpoint{6.670116in}{5.570595in}}{\pgfqpoint{6.642338in}{5.570595in}}%
\pgfpathlineto{\pgfqpoint{5.536408in}{5.570595in}}%
\pgfpathquadraticcurveto{\pgfqpoint{5.508630in}{5.570595in}}{\pgfqpoint{5.508630in}{5.542817in}}%
\pgfpathlineto{\pgfqpoint{5.508630in}{5.148991in}}%
\pgfpathquadraticcurveto{\pgfqpoint{5.508630in}{5.121213in}}{\pgfqpoint{5.536408in}{5.121213in}}%
\pgfpathlineto{\pgfqpoint{5.536408in}{5.121213in}}%
\pgfpathclose%
\pgfusepath{stroke,fill}%
\end{pgfscope}%
\begin{pgfscope}%
\pgfsetrectcap%
\pgfsetroundjoin%
\pgfsetlinewidth{1.505625pt}%
\definecolor{currentstroke}{rgb}{1.000000,0.000000,0.000000}%
\pgfsetstrokecolor{currentstroke}%
\pgfsetdash{}{0pt}%
\pgfpathmoveto{\pgfqpoint{5.564186in}{5.458127in}}%
\pgfpathlineto{\pgfqpoint{5.703075in}{5.458127in}}%
\pgfpathlineto{\pgfqpoint{5.841964in}{5.458127in}}%
\pgfusepath{stroke}%
\end{pgfscope}%
\begin{pgfscope}%
\pgfsetbuttcap%
\pgfsetroundjoin%
\definecolor{currentfill}{rgb}{1.000000,0.000000,0.000000}%
\pgfsetfillcolor{currentfill}%
\pgfsetlinewidth{1.003750pt}%
\definecolor{currentstroke}{rgb}{1.000000,0.000000,0.000000}%
\pgfsetstrokecolor{currentstroke}%
\pgfsetdash{}{0pt}%
\pgfsys@defobject{currentmarker}{\pgfqpoint{-0.041667in}{-0.041667in}}{\pgfqpoint{0.041667in}{0.041667in}}{%
\pgfpathmoveto{\pgfqpoint{0.000000in}{-0.041667in}}%
\pgfpathcurveto{\pgfqpoint{0.011050in}{-0.041667in}}{\pgfqpoint{0.021649in}{-0.037276in}}{\pgfqpoint{0.029463in}{-0.029463in}}%
\pgfpathcurveto{\pgfqpoint{0.037276in}{-0.021649in}}{\pgfqpoint{0.041667in}{-0.011050in}}{\pgfqpoint{0.041667in}{0.000000in}}%
\pgfpathcurveto{\pgfqpoint{0.041667in}{0.011050in}}{\pgfqpoint{0.037276in}{0.021649in}}{\pgfqpoint{0.029463in}{0.029463in}}%
\pgfpathcurveto{\pgfqpoint{0.021649in}{0.037276in}}{\pgfqpoint{0.011050in}{0.041667in}}{\pgfqpoint{0.000000in}{0.041667in}}%
\pgfpathcurveto{\pgfqpoint{-0.011050in}{0.041667in}}{\pgfqpoint{-0.021649in}{0.037276in}}{\pgfqpoint{-0.029463in}{0.029463in}}%
\pgfpathcurveto{\pgfqpoint{-0.037276in}{0.021649in}}{\pgfqpoint{-0.041667in}{0.011050in}}{\pgfqpoint{-0.041667in}{0.000000in}}%
\pgfpathcurveto{\pgfqpoint{-0.041667in}{-0.011050in}}{\pgfqpoint{-0.037276in}{-0.021649in}}{\pgfqpoint{-0.029463in}{-0.029463in}}%
\pgfpathcurveto{\pgfqpoint{-0.021649in}{-0.037276in}}{\pgfqpoint{-0.011050in}{-0.041667in}}{\pgfqpoint{0.000000in}{-0.041667in}}%
\pgfpathlineto{\pgfqpoint{0.000000in}{-0.041667in}}%
\pgfpathclose%
\pgfusepath{stroke,fill}%
}%
\begin{pgfscope}%
\pgfsys@transformshift{5.703075in}{5.458127in}%
\pgfsys@useobject{currentmarker}{}%
\end{pgfscope}%
\end{pgfscope}%
\begin{pgfscope}%
\definecolor{textcolor}{rgb}{0.000000,0.000000,0.000000}%
\pgfsetstrokecolor{textcolor}%
\pgfsetfillcolor{textcolor}%
\pgftext[x=5.953075in,y=5.409516in,left,base]{\color{textcolor}\sffamily\fontsize{10.000000}{12.000000}\selectfont Iterations}%
\end{pgfscope}%
\begin{pgfscope}%
\pgfsetbuttcap%
\pgfsetroundjoin%
\definecolor{currentfill}{rgb}{0.000000,0.000000,1.000000}%
\pgfsetfillcolor{currentfill}%
\pgfsetlinewidth{1.003750pt}%
\definecolor{currentstroke}{rgb}{0.000000,0.000000,1.000000}%
\pgfsetstrokecolor{currentstroke}%
\pgfsetdash{}{0pt}%
\pgfsys@defobject{currentmarker}{\pgfqpoint{-0.069444in}{-0.069444in}}{\pgfqpoint{0.069444in}{0.069444in}}{%
\pgfpathmoveto{\pgfqpoint{0.000000in}{-0.069444in}}%
\pgfpathcurveto{\pgfqpoint{0.018417in}{-0.069444in}}{\pgfqpoint{0.036082in}{-0.062127in}}{\pgfqpoint{0.049105in}{-0.049105in}}%
\pgfpathcurveto{\pgfqpoint{0.062127in}{-0.036082in}}{\pgfqpoint{0.069444in}{-0.018417in}}{\pgfqpoint{0.069444in}{0.000000in}}%
\pgfpathcurveto{\pgfqpoint{0.069444in}{0.018417in}}{\pgfqpoint{0.062127in}{0.036082in}}{\pgfqpoint{0.049105in}{0.049105in}}%
\pgfpathcurveto{\pgfqpoint{0.036082in}{0.062127in}}{\pgfqpoint{0.018417in}{0.069444in}}{\pgfqpoint{0.000000in}{0.069444in}}%
\pgfpathcurveto{\pgfqpoint{-0.018417in}{0.069444in}}{\pgfqpoint{-0.036082in}{0.062127in}}{\pgfqpoint{-0.049105in}{0.049105in}}%
\pgfpathcurveto{\pgfqpoint{-0.062127in}{0.036082in}}{\pgfqpoint{-0.069444in}{0.018417in}}{\pgfqpoint{-0.069444in}{0.000000in}}%
\pgfpathcurveto{\pgfqpoint{-0.069444in}{-0.018417in}}{\pgfqpoint{-0.062127in}{-0.036082in}}{\pgfqpoint{-0.049105in}{-0.049105in}}%
\pgfpathcurveto{\pgfqpoint{-0.036082in}{-0.062127in}}{\pgfqpoint{-0.018417in}{-0.069444in}}{\pgfqpoint{0.000000in}{-0.069444in}}%
\pgfpathlineto{\pgfqpoint{0.000000in}{-0.069444in}}%
\pgfpathclose%
\pgfusepath{stroke,fill}%
}%
\begin{pgfscope}%
\pgfsys@transformshift{5.703075in}{5.242117in}%
\pgfsys@useobject{currentmarker}{}%
\end{pgfscope}%
\end{pgfscope}%
\begin{pgfscope}%
\definecolor{textcolor}{rgb}{0.000000,0.000000,0.000000}%
\pgfsetstrokecolor{textcolor}%
\pgfsetfillcolor{textcolor}%
\pgftext[x=5.953075in,y=5.205659in,left,base]{\color{textcolor}\sffamily\fontsize{10.000000}{12.000000}\selectfont Minimum}%
\end{pgfscope}%
\end{pgfpicture}%
\makeatother%
\endgroup%
}
    \caption{Your figure caption}
    \label{fig:cg_contour}
\end{figure}

\begin{figure}[H]
    \centering
    \resizebox{1\textwidth}{!}{
    %% Creator: Matplotlib, PGF backend
%%
%% To include the figure in your LaTeX document, write
%%   \input{<filename>.pgf}
%%
%% Make sure the required packages are loaded in your preamble
%%   \usepackage{pgf}
%%
%% Also ensure that all the required font packages are loaded; for instance,
%% the lmodern package is sometimes necessary when using math font.
%%   \usepackage{lmodern}
%%
%% Figures using additional raster images can only be included by \input if
%% they are in the same directory as the main LaTeX file. For loading figures
%% from other directories you can use the `import` package
%%   \usepackage{import}
%%
%% and then include the figures with
%%   \import{<path to file>}{<filename>.pgf}
%%
%% Matplotlib used the following preamble
%%   
%%   \usepackage{fontspec}
%%   \setmainfont{DejaVuSerif.ttf}[Path=\detokenize{/home/radimek/Documents/projekt_mat_prog/mat_prog_kernel/lib/python3.12/site-packages/matplotlib/mpl-data/fonts/ttf/}]
%%   \setsansfont{DejaVuSans.ttf}[Path=\detokenize{/home/radimek/Documents/projekt_mat_prog/mat_prog_kernel/lib/python3.12/site-packages/matplotlib/mpl-data/fonts/ttf/}]
%%   \setmonofont{DejaVuSansMono.ttf}[Path=\detokenize{/home/radimek/Documents/projekt_mat_prog/mat_prog_kernel/lib/python3.12/site-packages/matplotlib/mpl-data/fonts/ttf/}]
%%   \makeatletter\@ifpackageloaded{underscore}{}{\usepackage[strings]{underscore}}\makeatother
%%
\begingroup%
\makeatletter%
\begin{pgfpicture}%
\pgfpathrectangle{\pgfpointorigin}{\pgfqpoint{8.000000in}{6.000000in}}%
\pgfusepath{use as bounding box, clip}%
\begin{pgfscope}%
\pgfsetbuttcap%
\pgfsetmiterjoin%
\definecolor{currentfill}{rgb}{1.000000,1.000000,1.000000}%
\pgfsetfillcolor{currentfill}%
\pgfsetlinewidth{0.000000pt}%
\definecolor{currentstroke}{rgb}{1.000000,1.000000,1.000000}%
\pgfsetstrokecolor{currentstroke}%
\pgfsetdash{}{0pt}%
\pgfpathmoveto{\pgfqpoint{0.000000in}{0.000000in}}%
\pgfpathlineto{\pgfqpoint{8.000000in}{0.000000in}}%
\pgfpathlineto{\pgfqpoint{8.000000in}{6.000000in}}%
\pgfpathlineto{\pgfqpoint{0.000000in}{6.000000in}}%
\pgfpathlineto{\pgfqpoint{0.000000in}{0.000000in}}%
\pgfpathclose%
\pgfusepath{fill}%
\end{pgfscope}%
\begin{pgfscope}%
\pgfsetbuttcap%
\pgfsetmiterjoin%
\definecolor{currentfill}{rgb}{1.000000,1.000000,1.000000}%
\pgfsetfillcolor{currentfill}%
\pgfsetlinewidth{0.000000pt}%
\definecolor{currentstroke}{rgb}{0.000000,0.000000,0.000000}%
\pgfsetstrokecolor{currentstroke}%
\pgfsetstrokeopacity{0.000000}%
\pgfsetdash{}{0pt}%
\pgfpathmoveto{\pgfqpoint{1.254980in}{0.150000in}}%
\pgfpathlineto{\pgfqpoint{6.745020in}{0.150000in}}%
\pgfpathlineto{\pgfqpoint{6.745020in}{5.640039in}}%
\pgfpathlineto{\pgfqpoint{1.254980in}{5.640039in}}%
\pgfpathlineto{\pgfqpoint{1.254980in}{0.150000in}}%
\pgfpathclose%
\pgfusepath{fill}%
\end{pgfscope}%
\begin{pgfscope}%
\pgfsetbuttcap%
\pgfsetmiterjoin%
\definecolor{currentfill}{rgb}{0.950000,0.950000,0.950000}%
\pgfsetfillcolor{currentfill}%
\pgfsetfillopacity{0.500000}%
\pgfsetlinewidth{1.003750pt}%
\definecolor{currentstroke}{rgb}{0.950000,0.950000,0.950000}%
\pgfsetstrokecolor{currentstroke}%
\pgfsetstrokeopacity{0.500000}%
\pgfsetdash{}{0pt}%
\pgfpathmoveto{\pgfqpoint{1.669516in}{1.503668in}}%
\pgfpathlineto{\pgfqpoint{3.482506in}{3.023352in}}%
\pgfpathlineto{\pgfqpoint{3.457304in}{5.215008in}}%
\pgfpathlineto{\pgfqpoint{1.557553in}{3.828657in}}%
\pgfusepath{stroke,fill}%
\end{pgfscope}%
\begin{pgfscope}%
\pgfsetbuttcap%
\pgfsetmiterjoin%
\definecolor{currentfill}{rgb}{0.900000,0.900000,0.900000}%
\pgfsetfillcolor{currentfill}%
\pgfsetfillopacity{0.500000}%
\pgfsetlinewidth{1.003750pt}%
\definecolor{currentstroke}{rgb}{0.900000,0.900000,0.900000}%
\pgfsetstrokecolor{currentstroke}%
\pgfsetstrokeopacity{0.500000}%
\pgfsetdash{}{0pt}%
\pgfpathmoveto{\pgfqpoint{3.482506in}{3.023352in}}%
\pgfpathlineto{\pgfqpoint{6.391709in}{2.177762in}}%
\pgfpathlineto{\pgfqpoint{6.495528in}{4.444907in}}%
\pgfpathlineto{\pgfqpoint{3.457304in}{5.215008in}}%
\pgfusepath{stroke,fill}%
\end{pgfscope}%
\begin{pgfscope}%
\pgfsetbuttcap%
\pgfsetmiterjoin%
\definecolor{currentfill}{rgb}{0.925000,0.925000,0.925000}%
\pgfsetfillcolor{currentfill}%
\pgfsetfillopacity{0.500000}%
\pgfsetlinewidth{1.003750pt}%
\definecolor{currentstroke}{rgb}{0.925000,0.925000,0.925000}%
\pgfsetstrokecolor{currentstroke}%
\pgfsetstrokeopacity{0.500000}%
\pgfsetdash{}{0pt}%
\pgfpathmoveto{\pgfqpoint{1.669516in}{1.503668in}}%
\pgfpathlineto{\pgfqpoint{4.753413in}{0.496467in}}%
\pgfpathlineto{\pgfqpoint{6.391709in}{2.177762in}}%
\pgfpathlineto{\pgfqpoint{3.482506in}{3.023352in}}%
\pgfusepath{stroke,fill}%
\end{pgfscope}%
\begin{pgfscope}%
\pgfsetrectcap%
\pgfsetroundjoin%
\pgfsetlinewidth{0.803000pt}%
\definecolor{currentstroke}{rgb}{0.000000,0.000000,0.000000}%
\pgfsetstrokecolor{currentstroke}%
\pgfsetdash{}{0pt}%
\pgfpathmoveto{\pgfqpoint{1.669516in}{1.503668in}}%
\pgfpathlineto{\pgfqpoint{4.753413in}{0.496467in}}%
\pgfusepath{stroke}%
\end{pgfscope}%
\begin{pgfscope}%
\definecolor{textcolor}{rgb}{0.000000,0.000000,0.000000}%
\pgfsetstrokecolor{textcolor}%
\pgfsetfillcolor{textcolor}%
\pgftext[x=2.945156in,y=0.524780in,,]{\color{textcolor}\sffamily\fontsize{10.000000}{12.000000}\selectfont x}%
\end{pgfscope}%
\begin{pgfscope}%
\pgfsetbuttcap%
\pgfsetroundjoin%
\pgfsetlinewidth{0.803000pt}%
\definecolor{currentstroke}{rgb}{0.690196,0.690196,0.690196}%
\pgfsetstrokecolor{currentstroke}%
\pgfsetdash{}{0pt}%
\pgfpathmoveto{\pgfqpoint{1.856293in}{1.442666in}}%
\pgfpathlineto{\pgfqpoint{3.659435in}{2.971926in}}%
\pgfpathlineto{\pgfqpoint{3.641714in}{5.168266in}}%
\pgfusepath{stroke}%
\end{pgfscope}%
\begin{pgfscope}%
\pgfsetbuttcap%
\pgfsetroundjoin%
\pgfsetlinewidth{0.803000pt}%
\definecolor{currentstroke}{rgb}{0.690196,0.690196,0.690196}%
\pgfsetstrokecolor{currentstroke}%
\pgfsetdash{}{0pt}%
\pgfpathmoveto{\pgfqpoint{2.287848in}{1.301721in}}%
\pgfpathlineto{\pgfqpoint{4.067873in}{2.853209in}}%
\pgfpathlineto{\pgfqpoint{4.067601in}{5.060316in}}%
\pgfusepath{stroke}%
\end{pgfscope}%
\begin{pgfscope}%
\pgfsetbuttcap%
\pgfsetroundjoin%
\pgfsetlinewidth{0.803000pt}%
\definecolor{currentstroke}{rgb}{0.690196,0.690196,0.690196}%
\pgfsetstrokecolor{currentstroke}%
\pgfsetdash{}{0pt}%
\pgfpathmoveto{\pgfqpoint{2.725971in}{1.158630in}}%
\pgfpathlineto{\pgfqpoint{4.482010in}{2.732836in}}%
\pgfpathlineto{\pgfqpoint{4.499690in}{4.950794in}}%
\pgfusepath{stroke}%
\end{pgfscope}%
\begin{pgfscope}%
\pgfsetbuttcap%
\pgfsetroundjoin%
\pgfsetlinewidth{0.803000pt}%
\definecolor{currentstroke}{rgb}{0.690196,0.690196,0.690196}%
\pgfsetstrokecolor{currentstroke}%
\pgfsetdash{}{0pt}%
\pgfpathmoveto{\pgfqpoint{3.170814in}{1.013344in}}%
\pgfpathlineto{\pgfqpoint{4.901969in}{2.610770in}}%
\pgfpathlineto{\pgfqpoint{4.938117in}{4.839666in}}%
\pgfusepath{stroke}%
\end{pgfscope}%
\begin{pgfscope}%
\pgfsetbuttcap%
\pgfsetroundjoin%
\pgfsetlinewidth{0.803000pt}%
\definecolor{currentstroke}{rgb}{0.690196,0.690196,0.690196}%
\pgfsetstrokecolor{currentstroke}%
\pgfsetdash{}{0pt}%
\pgfpathmoveto{\pgfqpoint{3.622534in}{0.865812in}}%
\pgfpathlineto{\pgfqpoint{5.327872in}{2.486977in}}%
\pgfpathlineto{\pgfqpoint{5.383022in}{4.726895in}}%
\pgfusepath{stroke}%
\end{pgfscope}%
\begin{pgfscope}%
\pgfsetbuttcap%
\pgfsetroundjoin%
\pgfsetlinewidth{0.803000pt}%
\definecolor{currentstroke}{rgb}{0.690196,0.690196,0.690196}%
\pgfsetstrokecolor{currentstroke}%
\pgfsetdash{}{0pt}%
\pgfpathmoveto{\pgfqpoint{4.081290in}{0.715983in}}%
\pgfpathlineto{\pgfqpoint{5.759846in}{2.361419in}}%
\pgfpathlineto{\pgfqpoint{5.834551in}{4.612446in}}%
\pgfusepath{stroke}%
\end{pgfscope}%
\begin{pgfscope}%
\pgfsetbuttcap%
\pgfsetroundjoin%
\pgfsetlinewidth{0.803000pt}%
\definecolor{currentstroke}{rgb}{0.690196,0.690196,0.690196}%
\pgfsetstrokecolor{currentstroke}%
\pgfsetdash{}{0pt}%
\pgfpathmoveto{\pgfqpoint{4.547248in}{0.563801in}}%
\pgfpathlineto{\pgfqpoint{6.198022in}{2.234059in}}%
\pgfpathlineto{\pgfqpoint{6.292853in}{4.496280in}}%
\pgfusepath{stroke}%
\end{pgfscope}%
\begin{pgfscope}%
\pgfsetrectcap%
\pgfsetroundjoin%
\pgfsetlinewidth{0.803000pt}%
\definecolor{currentstroke}{rgb}{0.000000,0.000000,0.000000}%
\pgfsetstrokecolor{currentstroke}%
\pgfsetdash{}{0pt}%
\pgfpathmoveto{\pgfqpoint{1.871995in}{1.455983in}}%
\pgfpathlineto{\pgfqpoint{1.824823in}{1.415976in}}%
\pgfusepath{stroke}%
\end{pgfscope}%
\begin{pgfscope}%
\definecolor{textcolor}{rgb}{0.000000,0.000000,0.000000}%
\pgfsetstrokecolor{textcolor}%
\pgfsetfillcolor{textcolor}%
\pgftext[x=1.751850in,y=1.224727in,,top]{\color{textcolor}\sffamily\fontsize{10.000000}{12.000000}\selectfont \ensuremath{-}1.0}%
\end{pgfscope}%
\begin{pgfscope}%
\pgfsetrectcap%
\pgfsetroundjoin%
\pgfsetlinewidth{0.803000pt}%
\definecolor{currentstroke}{rgb}{0.000000,0.000000,0.000000}%
\pgfsetstrokecolor{currentstroke}%
\pgfsetdash{}{0pt}%
\pgfpathmoveto{\pgfqpoint{2.303357in}{1.315239in}}%
\pgfpathlineto{\pgfqpoint{2.256761in}{1.274625in}}%
\pgfusepath{stroke}%
\end{pgfscope}%
\begin{pgfscope}%
\definecolor{textcolor}{rgb}{0.000000,0.000000,0.000000}%
\pgfsetstrokecolor{textcolor}%
\pgfsetfillcolor{textcolor}%
\pgftext[x=2.183700in,y=1.081776in,,top]{\color{textcolor}\sffamily\fontsize{10.000000}{12.000000}\selectfont \ensuremath{-}0.5}%
\end{pgfscope}%
\begin{pgfscope}%
\pgfsetrectcap%
\pgfsetroundjoin%
\pgfsetlinewidth{0.803000pt}%
\definecolor{currentstroke}{rgb}{0.000000,0.000000,0.000000}%
\pgfsetstrokecolor{currentstroke}%
\pgfsetdash{}{0pt}%
\pgfpathmoveto{\pgfqpoint{2.741281in}{1.172355in}}%
\pgfpathlineto{\pgfqpoint{2.695283in}{1.131120in}}%
\pgfusepath{stroke}%
\end{pgfscope}%
\begin{pgfscope}%
\definecolor{textcolor}{rgb}{0.000000,0.000000,0.000000}%
\pgfsetstrokecolor{textcolor}%
\pgfsetfillcolor{textcolor}%
\pgftext[x=2.622141in,y=0.936643in,,top]{\color{textcolor}\sffamily\fontsize{10.000000}{12.000000}\selectfont 0.0}%
\end{pgfscope}%
\begin{pgfscope}%
\pgfsetrectcap%
\pgfsetroundjoin%
\pgfsetlinewidth{0.803000pt}%
\definecolor{currentstroke}{rgb}{0.000000,0.000000,0.000000}%
\pgfsetstrokecolor{currentstroke}%
\pgfsetdash{}{0pt}%
\pgfpathmoveto{\pgfqpoint{3.185917in}{1.027280in}}%
\pgfpathlineto{\pgfqpoint{3.140542in}{0.985410in}}%
\pgfusepath{stroke}%
\end{pgfscope}%
\begin{pgfscope}%
\definecolor{textcolor}{rgb}{0.000000,0.000000,0.000000}%
\pgfsetstrokecolor{textcolor}%
\pgfsetfillcolor{textcolor}%
\pgftext[x=3.067323in,y=0.789279in,,top]{\color{textcolor}\sffamily\fontsize{10.000000}{12.000000}\selectfont 0.5}%
\end{pgfscope}%
\begin{pgfscope}%
\pgfsetrectcap%
\pgfsetroundjoin%
\pgfsetlinewidth{0.803000pt}%
\definecolor{currentstroke}{rgb}{0.000000,0.000000,0.000000}%
\pgfsetstrokecolor{currentstroke}%
\pgfsetdash{}{0pt}%
\pgfpathmoveto{\pgfqpoint{3.637421in}{0.879965in}}%
\pgfpathlineto{\pgfqpoint{3.592693in}{0.837445in}}%
\pgfusepath{stroke}%
\end{pgfscope}%
\begin{pgfscope}%
\definecolor{textcolor}{rgb}{0.000000,0.000000,0.000000}%
\pgfsetstrokecolor{textcolor}%
\pgfsetfillcolor{textcolor}%
\pgftext[x=3.519405in,y=0.639631in,,top]{\color{textcolor}\sffamily\fontsize{10.000000}{12.000000}\selectfont 1.0}%
\end{pgfscope}%
\begin{pgfscope}%
\pgfsetrectcap%
\pgfsetroundjoin%
\pgfsetlinewidth{0.803000pt}%
\definecolor{currentstroke}{rgb}{0.000000,0.000000,0.000000}%
\pgfsetstrokecolor{currentstroke}%
\pgfsetdash{}{0pt}%
\pgfpathmoveto{\pgfqpoint{4.095953in}{0.730356in}}%
\pgfpathlineto{\pgfqpoint{4.051898in}{0.687171in}}%
\pgfusepath{stroke}%
\end{pgfscope}%
\begin{pgfscope}%
\definecolor{textcolor}{rgb}{0.000000,0.000000,0.000000}%
\pgfsetstrokecolor{textcolor}%
\pgfsetfillcolor{textcolor}%
\pgftext[x=3.978547in,y=0.487646in,,top]{\color{textcolor}\sffamily\fontsize{10.000000}{12.000000}\selectfont 1.5}%
\end{pgfscope}%
\begin{pgfscope}%
\pgfsetrectcap%
\pgfsetroundjoin%
\pgfsetlinewidth{0.803000pt}%
\definecolor{currentstroke}{rgb}{0.000000,0.000000,0.000000}%
\pgfsetstrokecolor{currentstroke}%
\pgfsetdash{}{0pt}%
\pgfpathmoveto{\pgfqpoint{4.561678in}{0.578401in}}%
\pgfpathlineto{\pgfqpoint{4.518323in}{0.534535in}}%
\pgfusepath{stroke}%
\end{pgfscope}%
\begin{pgfscope}%
\definecolor{textcolor}{rgb}{0.000000,0.000000,0.000000}%
\pgfsetstrokecolor{textcolor}%
\pgfsetfillcolor{textcolor}%
\pgftext[x=4.444916in,y=0.333269in,,top]{\color{textcolor}\sffamily\fontsize{10.000000}{12.000000}\selectfont 2.0}%
\end{pgfscope}%
\begin{pgfscope}%
\pgfsetrectcap%
\pgfsetroundjoin%
\pgfsetlinewidth{0.803000pt}%
\definecolor{currentstroke}{rgb}{0.000000,0.000000,0.000000}%
\pgfsetstrokecolor{currentstroke}%
\pgfsetdash{}{0pt}%
\pgfpathmoveto{\pgfqpoint{6.391709in}{2.177762in}}%
\pgfpathlineto{\pgfqpoint{4.753413in}{0.496467in}}%
\pgfusepath{stroke}%
\end{pgfscope}%
\begin{pgfscope}%
\definecolor{textcolor}{rgb}{0.000000,0.000000,0.000000}%
\pgfsetstrokecolor{textcolor}%
\pgfsetfillcolor{textcolor}%
\pgftext[x=5.983676in,y=0.985873in,,]{\color{textcolor}\sffamily\fontsize{10.000000}{12.000000}\selectfont y}%
\end{pgfscope}%
\begin{pgfscope}%
\pgfsetbuttcap%
\pgfsetroundjoin%
\pgfsetlinewidth{0.803000pt}%
\definecolor{currentstroke}{rgb}{0.690196,0.690196,0.690196}%
\pgfsetstrokecolor{currentstroke}%
\pgfsetdash{}{0pt}%
\pgfpathmoveto{\pgfqpoint{1.688926in}{3.924526in}}%
\pgfpathlineto{\pgfqpoint{1.794447in}{1.608387in}}%
\pgfpathlineto{\pgfqpoint{4.866770in}{0.612800in}}%
\pgfusepath{stroke}%
\end{pgfscope}%
\begin{pgfscope}%
\pgfsetbuttcap%
\pgfsetroundjoin%
\pgfsetlinewidth{0.803000pt}%
\definecolor{currentstroke}{rgb}{0.690196,0.690196,0.690196}%
\pgfsetstrokecolor{currentstroke}%
\pgfsetdash{}{0pt}%
\pgfpathmoveto{\pgfqpoint{1.910570in}{4.086273in}}%
\pgfpathlineto{\pgfqpoint{2.005371in}{1.785188in}}%
\pgfpathlineto{\pgfqpoint{5.057999in}{0.809047in}}%
\pgfusepath{stroke}%
\end{pgfscope}%
\begin{pgfscope}%
\pgfsetbuttcap%
\pgfsetroundjoin%
\pgfsetlinewidth{0.803000pt}%
\definecolor{currentstroke}{rgb}{0.690196,0.690196,0.690196}%
\pgfsetstrokecolor{currentstroke}%
\pgfsetdash{}{0pt}%
\pgfpathmoveto{\pgfqpoint{2.127827in}{4.244817in}}%
\pgfpathlineto{\pgfqpoint{2.212301in}{1.958641in}}%
\pgfpathlineto{\pgfqpoint{5.245415in}{1.001383in}}%
\pgfusepath{stroke}%
\end{pgfscope}%
\begin{pgfscope}%
\pgfsetbuttcap%
\pgfsetroundjoin%
\pgfsetlinewidth{0.803000pt}%
\definecolor{currentstroke}{rgb}{0.690196,0.690196,0.690196}%
\pgfsetstrokecolor{currentstroke}%
\pgfsetdash{}{0pt}%
\pgfpathmoveto{\pgfqpoint{2.340825in}{4.400253in}}%
\pgfpathlineto{\pgfqpoint{2.415349in}{2.128839in}}%
\pgfpathlineto{\pgfqpoint{5.429132in}{1.189921in}}%
\pgfusepath{stroke}%
\end{pgfscope}%
\begin{pgfscope}%
\pgfsetbuttcap%
\pgfsetroundjoin%
\pgfsetlinewidth{0.803000pt}%
\definecolor{currentstroke}{rgb}{0.690196,0.690196,0.690196}%
\pgfsetstrokecolor{currentstroke}%
\pgfsetdash{}{0pt}%
\pgfpathmoveto{\pgfqpoint{2.549688in}{4.552671in}}%
\pgfpathlineto{\pgfqpoint{2.614623in}{2.295875in}}%
\pgfpathlineto{\pgfqpoint{5.609257in}{1.374773in}}%
\pgfusepath{stroke}%
\end{pgfscope}%
\begin{pgfscope}%
\pgfsetbuttcap%
\pgfsetroundjoin%
\pgfsetlinewidth{0.803000pt}%
\definecolor{currentstroke}{rgb}{0.690196,0.690196,0.690196}%
\pgfsetstrokecolor{currentstroke}%
\pgfsetdash{}{0pt}%
\pgfpathmoveto{\pgfqpoint{2.754535in}{4.702159in}}%
\pgfpathlineto{\pgfqpoint{2.810227in}{2.459834in}}%
\pgfpathlineto{\pgfqpoint{5.785895in}{1.556047in}}%
\pgfusepath{stroke}%
\end{pgfscope}%
\begin{pgfscope}%
\pgfsetbuttcap%
\pgfsetroundjoin%
\pgfsetlinewidth{0.803000pt}%
\definecolor{currentstroke}{rgb}{0.690196,0.690196,0.690196}%
\pgfsetstrokecolor{currentstroke}%
\pgfsetdash{}{0pt}%
\pgfpathmoveto{\pgfqpoint{2.955481in}{4.848801in}}%
\pgfpathlineto{\pgfqpoint{3.002262in}{2.620801in}}%
\pgfpathlineto{\pgfqpoint{5.959146in}{1.733846in}}%
\pgfusepath{stroke}%
\end{pgfscope}%
\begin{pgfscope}%
\pgfsetbuttcap%
\pgfsetroundjoin%
\pgfsetlinewidth{0.803000pt}%
\definecolor{currentstroke}{rgb}{0.690196,0.690196,0.690196}%
\pgfsetstrokecolor{currentstroke}%
\pgfsetdash{}{0pt}%
\pgfpathmoveto{\pgfqpoint{3.152636in}{4.992676in}}%
\pgfpathlineto{\pgfqpoint{3.190824in}{2.778858in}}%
\pgfpathlineto{\pgfqpoint{6.129107in}{1.908268in}}%
\pgfusepath{stroke}%
\end{pgfscope}%
\begin{pgfscope}%
\pgfsetbuttcap%
\pgfsetroundjoin%
\pgfsetlinewidth{0.803000pt}%
\definecolor{currentstroke}{rgb}{0.690196,0.690196,0.690196}%
\pgfsetstrokecolor{currentstroke}%
\pgfsetdash{}{0pt}%
\pgfpathmoveto{\pgfqpoint{3.346107in}{5.133862in}}%
\pgfpathlineto{\pgfqpoint{3.376007in}{2.934082in}}%
\pgfpathlineto{\pgfqpoint{6.295871in}{2.079408in}}%
\pgfusepath{stroke}%
\end{pgfscope}%
\begin{pgfscope}%
\pgfsetrectcap%
\pgfsetroundjoin%
\pgfsetlinewidth{0.803000pt}%
\definecolor{currentstroke}{rgb}{0.000000,0.000000,0.000000}%
\pgfsetstrokecolor{currentstroke}%
\pgfsetdash{}{0pt}%
\pgfpathmoveto{\pgfqpoint{4.840880in}{0.621189in}}%
\pgfpathlineto{\pgfqpoint{4.918618in}{0.595998in}}%
\pgfusepath{stroke}%
\end{pgfscope}%
\begin{pgfscope}%
\definecolor{textcolor}{rgb}{0.000000,0.000000,0.000000}%
\pgfsetstrokecolor{textcolor}%
\pgfsetfillcolor{textcolor}%
\pgftext[x=5.045633in,y=0.426401in,,top]{\color{textcolor}\sffamily\fontsize{10.000000}{12.000000}\selectfont \ensuremath{-}1.00}%
\end{pgfscope}%
\begin{pgfscope}%
\pgfsetrectcap%
\pgfsetroundjoin%
\pgfsetlinewidth{0.803000pt}%
\definecolor{currentstroke}{rgb}{0.000000,0.000000,0.000000}%
\pgfsetstrokecolor{currentstroke}%
\pgfsetdash{}{0pt}%
\pgfpathmoveto{\pgfqpoint{5.032288in}{0.817269in}}%
\pgfpathlineto{\pgfqpoint{5.109488in}{0.792583in}}%
\pgfusepath{stroke}%
\end{pgfscope}%
\begin{pgfscope}%
\definecolor{textcolor}{rgb}{0.000000,0.000000,0.000000}%
\pgfsetstrokecolor{textcolor}%
\pgfsetfillcolor{textcolor}%
\pgftext[x=5.235136in,y=0.624605in,,top]{\color{textcolor}\sffamily\fontsize{10.000000}{12.000000}\selectfont \ensuremath{-}0.75}%
\end{pgfscope}%
\begin{pgfscope}%
\pgfsetrectcap%
\pgfsetroundjoin%
\pgfsetlinewidth{0.803000pt}%
\definecolor{currentstroke}{rgb}{0.000000,0.000000,0.000000}%
\pgfsetstrokecolor{currentstroke}%
\pgfsetdash{}{0pt}%
\pgfpathmoveto{\pgfqpoint{5.219881in}{1.009441in}}%
\pgfpathlineto{\pgfqpoint{5.296549in}{0.985245in}}%
\pgfusepath{stroke}%
\end{pgfscope}%
\begin{pgfscope}%
\definecolor{textcolor}{rgb}{0.000000,0.000000,0.000000}%
\pgfsetstrokecolor{textcolor}%
\pgfsetfillcolor{textcolor}%
\pgftext[x=5.420860in,y=0.818856in,,top]{\color{textcolor}\sffamily\fontsize{10.000000}{12.000000}\selectfont \ensuremath{-}0.50}%
\end{pgfscope}%
\begin{pgfscope}%
\pgfsetrectcap%
\pgfsetroundjoin%
\pgfsetlinewidth{0.803000pt}%
\definecolor{currentstroke}{rgb}{0.000000,0.000000,0.000000}%
\pgfsetstrokecolor{currentstroke}%
\pgfsetdash{}{0pt}%
\pgfpathmoveto{\pgfqpoint{5.403772in}{1.197821in}}%
\pgfpathlineto{\pgfqpoint{5.479914in}{1.174100in}}%
\pgfusepath{stroke}%
\end{pgfscope}%
\begin{pgfscope}%
\definecolor{textcolor}{rgb}{0.000000,0.000000,0.000000}%
\pgfsetstrokecolor{textcolor}%
\pgfsetfillcolor{textcolor}%
\pgftext[x=5.602916in,y=1.009270in,,top]{\color{textcolor}\sffamily\fontsize{10.000000}{12.000000}\selectfont \ensuremath{-}0.25}%
\end{pgfscope}%
\begin{pgfscope}%
\pgfsetrectcap%
\pgfsetroundjoin%
\pgfsetlinewidth{0.803000pt}%
\definecolor{currentstroke}{rgb}{0.000000,0.000000,0.000000}%
\pgfsetstrokecolor{currentstroke}%
\pgfsetdash{}{0pt}%
\pgfpathmoveto{\pgfqpoint{5.584071in}{1.382520in}}%
\pgfpathlineto{\pgfqpoint{5.659691in}{1.359261in}}%
\pgfusepath{stroke}%
\end{pgfscope}%
\begin{pgfscope}%
\definecolor{textcolor}{rgb}{0.000000,0.000000,0.000000}%
\pgfsetstrokecolor{textcolor}%
\pgfsetfillcolor{textcolor}%
\pgftext[x=5.781411in,y=1.195961in,,top]{\color{textcolor}\sffamily\fontsize{10.000000}{12.000000}\selectfont 0.00}%
\end{pgfscope}%
\begin{pgfscope}%
\pgfsetrectcap%
\pgfsetroundjoin%
\pgfsetlinewidth{0.803000pt}%
\definecolor{currentstroke}{rgb}{0.000000,0.000000,0.000000}%
\pgfsetstrokecolor{currentstroke}%
\pgfsetdash{}{0pt}%
\pgfpathmoveto{\pgfqpoint{5.760880in}{1.563645in}}%
\pgfpathlineto{\pgfqpoint{5.835985in}{1.540834in}}%
\pgfusepath{stroke}%
\end{pgfscope}%
\begin{pgfscope}%
\definecolor{textcolor}{rgb}{0.000000,0.000000,0.000000}%
\pgfsetstrokecolor{textcolor}%
\pgfsetfillcolor{textcolor}%
\pgftext[x=5.956450in,y=1.379036in,,top]{\color{textcolor}\sffamily\fontsize{10.000000}{12.000000}\selectfont 0.25}%
\end{pgfscope}%
\begin{pgfscope}%
\pgfsetrectcap%
\pgfsetroundjoin%
\pgfsetlinewidth{0.803000pt}%
\definecolor{currentstroke}{rgb}{0.000000,0.000000,0.000000}%
\pgfsetstrokecolor{currentstroke}%
\pgfsetdash{}{0pt}%
\pgfpathmoveto{\pgfqpoint{5.934301in}{1.741299in}}%
\pgfpathlineto{\pgfqpoint{6.008897in}{1.718923in}}%
\pgfusepath{stroke}%
\end{pgfscope}%
\begin{pgfscope}%
\definecolor{textcolor}{rgb}{0.000000,0.000000,0.000000}%
\pgfsetstrokecolor{textcolor}%
\pgfsetfillcolor{textcolor}%
\pgftext[x=6.128132in,y=1.558599in,,top]{\color{textcolor}\sffamily\fontsize{10.000000}{12.000000}\selectfont 0.50}%
\end{pgfscope}%
\begin{pgfscope}%
\pgfsetrectcap%
\pgfsetroundjoin%
\pgfsetlinewidth{0.803000pt}%
\definecolor{currentstroke}{rgb}{0.000000,0.000000,0.000000}%
\pgfsetstrokecolor{currentstroke}%
\pgfsetdash{}{0pt}%
\pgfpathmoveto{\pgfqpoint{6.104430in}{1.915580in}}%
\pgfpathlineto{\pgfqpoint{6.178522in}{1.893627in}}%
\pgfusepath{stroke}%
\end{pgfscope}%
\begin{pgfscope}%
\definecolor{textcolor}{rgb}{0.000000,0.000000,0.000000}%
\pgfsetstrokecolor{textcolor}%
\pgfsetfillcolor{textcolor}%
\pgftext[x=6.296552in,y=1.734751in,,top]{\color{textcolor}\sffamily\fontsize{10.000000}{12.000000}\selectfont 0.75}%
\end{pgfscope}%
\begin{pgfscope}%
\pgfsetrectcap%
\pgfsetroundjoin%
\pgfsetlinewidth{0.803000pt}%
\definecolor{currentstroke}{rgb}{0.000000,0.000000,0.000000}%
\pgfsetstrokecolor{currentstroke}%
\pgfsetdash{}{0pt}%
\pgfpathmoveto{\pgfqpoint{6.271359in}{2.086583in}}%
\pgfpathlineto{\pgfqpoint{6.344953in}{2.065041in}}%
\pgfusepath{stroke}%
\end{pgfscope}%
\begin{pgfscope}%
\definecolor{textcolor}{rgb}{0.000000,0.000000,0.000000}%
\pgfsetstrokecolor{textcolor}%
\pgfsetfillcolor{textcolor}%
\pgftext[x=6.461802in,y=1.907589in,,top]{\color{textcolor}\sffamily\fontsize{10.000000}{12.000000}\selectfont 1.00}%
\end{pgfscope}%
\begin{pgfscope}%
\pgfsetrectcap%
\pgfsetroundjoin%
\pgfsetlinewidth{0.803000pt}%
\definecolor{currentstroke}{rgb}{0.000000,0.000000,0.000000}%
\pgfsetstrokecolor{currentstroke}%
\pgfsetdash{}{0pt}%
\pgfpathmoveto{\pgfqpoint{6.391709in}{2.177762in}}%
\pgfpathlineto{\pgfqpoint{6.495528in}{4.444907in}}%
\pgfusepath{stroke}%
\end{pgfscope}%
\begin{pgfscope}%
\definecolor{textcolor}{rgb}{0.000000,0.000000,0.000000}%
\pgfsetstrokecolor{textcolor}%
\pgfsetfillcolor{textcolor}%
\pgftext[x=7.004475in,y=3.361793in,,,rotate=87.378092]{\color{textcolor}\sffamily\fontsize{10.000000}{12.000000}\selectfont f(x,y)}%
\end{pgfscope}%
\begin{pgfscope}%
\pgfsetbuttcap%
\pgfsetroundjoin%
\pgfsetlinewidth{0.803000pt}%
\definecolor{currentstroke}{rgb}{0.690196,0.690196,0.690196}%
\pgfsetstrokecolor{currentstroke}%
\pgfsetdash{}{0pt}%
\pgfpathmoveto{\pgfqpoint{6.403535in}{2.436011in}}%
\pgfpathlineto{\pgfqpoint{3.479630in}{3.273470in}}%
\pgfpathlineto{\pgfqpoint{1.656782in}{1.768110in}}%
\pgfusepath{stroke}%
\end{pgfscope}%
\begin{pgfscope}%
\pgfsetbuttcap%
\pgfsetroundjoin%
\pgfsetlinewidth{0.803000pt}%
\definecolor{currentstroke}{rgb}{0.690196,0.690196,0.690196}%
\pgfsetstrokecolor{currentstroke}%
\pgfsetdash{}{0pt}%
\pgfpathmoveto{\pgfqpoint{6.416362in}{2.716121in}}%
\pgfpathlineto{\pgfqpoint{3.476512in}{3.544624in}}%
\pgfpathlineto{\pgfqpoint{1.642964in}{2.055053in}}%
\pgfusepath{stroke}%
\end{pgfscope}%
\begin{pgfscope}%
\pgfsetbuttcap%
\pgfsetroundjoin%
\pgfsetlinewidth{0.803000pt}%
\definecolor{currentstroke}{rgb}{0.690196,0.690196,0.690196}%
\pgfsetstrokecolor{currentstroke}%
\pgfsetdash{}{0pt}%
\pgfpathmoveto{\pgfqpoint{6.429331in}{2.999333in}}%
\pgfpathlineto{\pgfqpoint{3.473361in}{3.818637in}}%
\pgfpathlineto{\pgfqpoint{1.628986in}{2.345295in}}%
\pgfusepath{stroke}%
\end{pgfscope}%
\begin{pgfscope}%
\pgfsetbuttcap%
\pgfsetroundjoin%
\pgfsetlinewidth{0.803000pt}%
\definecolor{currentstroke}{rgb}{0.690196,0.690196,0.690196}%
\pgfsetstrokecolor{currentstroke}%
\pgfsetdash{}{0pt}%
\pgfpathmoveto{\pgfqpoint{6.442445in}{3.285700in}}%
\pgfpathlineto{\pgfqpoint{3.470177in}{4.095555in}}%
\pgfpathlineto{\pgfqpoint{1.614848in}{2.638894in}}%
\pgfusepath{stroke}%
\end{pgfscope}%
\begin{pgfscope}%
\pgfsetbuttcap%
\pgfsetroundjoin%
\pgfsetlinewidth{0.803000pt}%
\definecolor{currentstroke}{rgb}{0.690196,0.690196,0.690196}%
\pgfsetstrokecolor{currentstroke}%
\pgfsetdash{}{0pt}%
\pgfpathmoveto{\pgfqpoint{6.455705in}{3.575273in}}%
\pgfpathlineto{\pgfqpoint{3.466959in}{4.375424in}}%
\pgfpathlineto{\pgfqpoint{1.600545in}{2.935907in}}%
\pgfusepath{stroke}%
\end{pgfscope}%
\begin{pgfscope}%
\pgfsetbuttcap%
\pgfsetroundjoin%
\pgfsetlinewidth{0.803000pt}%
\definecolor{currentstroke}{rgb}{0.690196,0.690196,0.690196}%
\pgfsetstrokecolor{currentstroke}%
\pgfsetdash{}{0pt}%
\pgfpathmoveto{\pgfqpoint{6.469115in}{3.868107in}}%
\pgfpathlineto{\pgfqpoint{3.463706in}{4.658291in}}%
\pgfpathlineto{\pgfqpoint{1.586074in}{3.236396in}}%
\pgfusepath{stroke}%
\end{pgfscope}%
\begin{pgfscope}%
\pgfsetbuttcap%
\pgfsetroundjoin%
\pgfsetlinewidth{0.803000pt}%
\definecolor{currentstroke}{rgb}{0.690196,0.690196,0.690196}%
\pgfsetstrokecolor{currentstroke}%
\pgfsetdash{}{0pt}%
\pgfpathmoveto{\pgfqpoint{6.482676in}{4.164257in}}%
\pgfpathlineto{\pgfqpoint{3.460418in}{4.944204in}}%
\pgfpathlineto{\pgfqpoint{1.571434in}{3.540420in}}%
\pgfusepath{stroke}%
\end{pgfscope}%
\begin{pgfscope}%
\pgfsetrectcap%
\pgfsetroundjoin%
\pgfsetlinewidth{0.803000pt}%
\definecolor{currentstroke}{rgb}{0.000000,0.000000,0.000000}%
\pgfsetstrokecolor{currentstroke}%
\pgfsetdash{}{0pt}%
\pgfpathmoveto{\pgfqpoint{6.378989in}{2.443041in}}%
\pgfpathlineto{\pgfqpoint{6.452684in}{2.421934in}}%
\pgfusepath{stroke}%
\end{pgfscope}%
\begin{pgfscope}%
\definecolor{textcolor}{rgb}{0.000000,0.000000,0.000000}%
\pgfsetstrokecolor{textcolor}%
\pgfsetfillcolor{textcolor}%
\pgftext[x=6.658505in,y=2.471887in,,top]{\color{textcolor}\sffamily\fontsize{10.000000}{12.000000}\selectfont 2}%
\end{pgfscope}%
\begin{pgfscope}%
\pgfsetrectcap%
\pgfsetroundjoin%
\pgfsetlinewidth{0.803000pt}%
\definecolor{currentstroke}{rgb}{0.000000,0.000000,0.000000}%
\pgfsetstrokecolor{currentstroke}%
\pgfsetdash{}{0pt}%
\pgfpathmoveto{\pgfqpoint{6.391676in}{2.723078in}}%
\pgfpathlineto{\pgfqpoint{6.465793in}{2.702191in}}%
\pgfusepath{stroke}%
\end{pgfscope}%
\begin{pgfscope}%
\definecolor{textcolor}{rgb}{0.000000,0.000000,0.000000}%
\pgfsetstrokecolor{textcolor}%
\pgfsetfillcolor{textcolor}%
\pgftext[x=6.672707in,y=2.751622in,,top]{\color{textcolor}\sffamily\fontsize{10.000000}{12.000000}\selectfont 3}%
\end{pgfscope}%
\begin{pgfscope}%
\pgfsetrectcap%
\pgfsetroundjoin%
\pgfsetlinewidth{0.803000pt}%
\definecolor{currentstroke}{rgb}{0.000000,0.000000,0.000000}%
\pgfsetstrokecolor{currentstroke}%
\pgfsetdash{}{0pt}%
\pgfpathmoveto{\pgfqpoint{6.404503in}{3.006215in}}%
\pgfpathlineto{\pgfqpoint{6.479046in}{2.985554in}}%
\pgfusepath{stroke}%
\end{pgfscope}%
\begin{pgfscope}%
\definecolor{textcolor}{rgb}{0.000000,0.000000,0.000000}%
\pgfsetstrokecolor{textcolor}%
\pgfsetfillcolor{textcolor}%
\pgftext[x=6.687066in,y=3.034449in,,top]{\color{textcolor}\sffamily\fontsize{10.000000}{12.000000}\selectfont 4}%
\end{pgfscope}%
\begin{pgfscope}%
\pgfsetrectcap%
\pgfsetroundjoin%
\pgfsetlinewidth{0.803000pt}%
\definecolor{currentstroke}{rgb}{0.000000,0.000000,0.000000}%
\pgfsetstrokecolor{currentstroke}%
\pgfsetdash{}{0pt}%
\pgfpathmoveto{\pgfqpoint{6.417473in}{3.292503in}}%
\pgfpathlineto{\pgfqpoint{6.492447in}{3.272075in}}%
\pgfusepath{stroke}%
\end{pgfscope}%
\begin{pgfscope}%
\definecolor{textcolor}{rgb}{0.000000,0.000000,0.000000}%
\pgfsetstrokecolor{textcolor}%
\pgfsetfillcolor{textcolor}%
\pgftext[x=6.701584in,y=3.320419in,,top]{\color{textcolor}\sffamily\fontsize{10.000000}{12.000000}\selectfont 5}%
\end{pgfscope}%
\begin{pgfscope}%
\pgfsetrectcap%
\pgfsetroundjoin%
\pgfsetlinewidth{0.803000pt}%
\definecolor{currentstroke}{rgb}{0.000000,0.000000,0.000000}%
\pgfsetstrokecolor{currentstroke}%
\pgfsetdash{}{0pt}%
\pgfpathmoveto{\pgfqpoint{6.430589in}{3.581997in}}%
\pgfpathlineto{\pgfqpoint{6.505999in}{3.561808in}}%
\pgfusepath{stroke}%
\end{pgfscope}%
\begin{pgfscope}%
\definecolor{textcolor}{rgb}{0.000000,0.000000,0.000000}%
\pgfsetstrokecolor{textcolor}%
\pgfsetfillcolor{textcolor}%
\pgftext[x=6.716265in,y=3.609585in,,top]{\color{textcolor}\sffamily\fontsize{10.000000}{12.000000}\selectfont 6}%
\end{pgfscope}%
\begin{pgfscope}%
\pgfsetrectcap%
\pgfsetroundjoin%
\pgfsetlinewidth{0.803000pt}%
\definecolor{currentstroke}{rgb}{0.000000,0.000000,0.000000}%
\pgfsetstrokecolor{currentstroke}%
\pgfsetdash{}{0pt}%
\pgfpathmoveto{\pgfqpoint{6.443852in}{3.874749in}}%
\pgfpathlineto{\pgfqpoint{6.519703in}{3.854806in}}%
\pgfusepath{stroke}%
\end{pgfscope}%
\begin{pgfscope}%
\definecolor{textcolor}{rgb}{0.000000,0.000000,0.000000}%
\pgfsetstrokecolor{textcolor}%
\pgfsetfillcolor{textcolor}%
\pgftext[x=6.731111in,y=3.902001in,,top]{\color{textcolor}\sffamily\fontsize{10.000000}{12.000000}\selectfont 7}%
\end{pgfscope}%
\begin{pgfscope}%
\pgfsetrectcap%
\pgfsetroundjoin%
\pgfsetlinewidth{0.803000pt}%
\definecolor{currentstroke}{rgb}{0.000000,0.000000,0.000000}%
\pgfsetstrokecolor{currentstroke}%
\pgfsetdash{}{0pt}%
\pgfpathmoveto{\pgfqpoint{6.457265in}{4.170815in}}%
\pgfpathlineto{\pgfqpoint{6.533562in}{4.151125in}}%
\pgfusepath{stroke}%
\end{pgfscope}%
\begin{pgfscope}%
\definecolor{textcolor}{rgb}{0.000000,0.000000,0.000000}%
\pgfsetstrokecolor{textcolor}%
\pgfsetfillcolor{textcolor}%
\pgftext[x=6.746125in,y=4.197721in,,top]{\color{textcolor}\sffamily\fontsize{10.000000}{12.000000}\selectfont 8}%
\end{pgfscope}%
\begin{pgfscope}%
\pgfpathrectangle{\pgfqpoint{1.254980in}{0.150000in}}{\pgfqpoint{5.490039in}{5.490039in}}%
\pgfusepath{clip}%
\pgfsetrectcap%
\pgfsetroundjoin%
\pgfsetlinewidth{1.505625pt}%
\definecolor{currentstroke}{rgb}{1.000000,0.000000,0.000000}%
\pgfsetstrokecolor{currentstroke}%
\pgfsetdash{}{0pt}%
\pgfpathmoveto{\pgfqpoint{5.343452in}{3.025813in}}%
\pgfpathlineto{\pgfqpoint{4.212739in}{3.581910in}}%
\pgfpathlineto{\pgfqpoint{3.496955in}{1.581500in}}%
\pgfpathlineto{\pgfqpoint{3.306558in}{1.489184in}}%
\pgfpathlineto{\pgfqpoint{3.459568in}{1.545357in}}%
\pgfpathlineto{\pgfqpoint{3.373020in}{1.484904in}}%
\pgfpathlineto{\pgfqpoint{3.417461in}{1.502213in}}%
\pgfpathlineto{\pgfqpoint{3.395624in}{1.489273in}}%
\pgfpathlineto{\pgfqpoint{3.400795in}{1.491437in}}%
\pgfpathlineto{\pgfqpoint{3.397922in}{1.490657in}}%
\pgfpathlineto{\pgfqpoint{3.398522in}{1.490870in}}%
\pgfpathlineto{\pgfqpoint{3.398291in}{1.490592in}}%
\pgfusepath{stroke}%
\end{pgfscope}%
\begin{pgfscope}%
\pgfpathrectangle{\pgfqpoint{1.254980in}{0.150000in}}{\pgfqpoint{5.490039in}{5.490039in}}%
\pgfusepath{clip}%
\pgfsetbuttcap%
\pgfsetroundjoin%
\definecolor{currentfill}{rgb}{1.000000,0.000000,0.000000}%
\pgfsetfillcolor{currentfill}%
\pgfsetlinewidth{1.003750pt}%
\definecolor{currentstroke}{rgb}{1.000000,0.000000,0.000000}%
\pgfsetstrokecolor{currentstroke}%
\pgfsetdash{}{0pt}%
\pgfsys@defobject{currentmarker}{\pgfqpoint{-0.041667in}{-0.041667in}}{\pgfqpoint{0.041667in}{0.041667in}}{%
\pgfpathmoveto{\pgfqpoint{0.000000in}{-0.041667in}}%
\pgfpathcurveto{\pgfqpoint{0.011050in}{-0.041667in}}{\pgfqpoint{0.021649in}{-0.037276in}}{\pgfqpoint{0.029463in}{-0.029463in}}%
\pgfpathcurveto{\pgfqpoint{0.037276in}{-0.021649in}}{\pgfqpoint{0.041667in}{-0.011050in}}{\pgfqpoint{0.041667in}{0.000000in}}%
\pgfpathcurveto{\pgfqpoint{0.041667in}{0.011050in}}{\pgfqpoint{0.037276in}{0.021649in}}{\pgfqpoint{0.029463in}{0.029463in}}%
\pgfpathcurveto{\pgfqpoint{0.021649in}{0.037276in}}{\pgfqpoint{0.011050in}{0.041667in}}{\pgfqpoint{0.000000in}{0.041667in}}%
\pgfpathcurveto{\pgfqpoint{-0.011050in}{0.041667in}}{\pgfqpoint{-0.021649in}{0.037276in}}{\pgfqpoint{-0.029463in}{0.029463in}}%
\pgfpathcurveto{\pgfqpoint{-0.037276in}{0.021649in}}{\pgfqpoint{-0.041667in}{0.011050in}}{\pgfqpoint{-0.041667in}{0.000000in}}%
\pgfpathcurveto{\pgfqpoint{-0.041667in}{-0.011050in}}{\pgfqpoint{-0.037276in}{-0.021649in}}{\pgfqpoint{-0.029463in}{-0.029463in}}%
\pgfpathcurveto{\pgfqpoint{-0.021649in}{-0.037276in}}{\pgfqpoint{-0.011050in}{-0.041667in}}{\pgfqpoint{0.000000in}{-0.041667in}}%
\pgfpathlineto{\pgfqpoint{0.000000in}{-0.041667in}}%
\pgfpathclose%
\pgfusepath{stroke,fill}%
}%
\begin{pgfscope}%
\pgfsys@transformshift{5.343452in}{3.025813in}%
\pgfsys@useobject{currentmarker}{}%
\end{pgfscope}%
\begin{pgfscope}%
\pgfsys@transformshift{4.212739in}{3.581910in}%
\pgfsys@useobject{currentmarker}{}%
\end{pgfscope}%
\begin{pgfscope}%
\pgfsys@transformshift{3.496955in}{1.581500in}%
\pgfsys@useobject{currentmarker}{}%
\end{pgfscope}%
\begin{pgfscope}%
\pgfsys@transformshift{3.306558in}{1.489184in}%
\pgfsys@useobject{currentmarker}{}%
\end{pgfscope}%
\begin{pgfscope}%
\pgfsys@transformshift{3.459568in}{1.545357in}%
\pgfsys@useobject{currentmarker}{}%
\end{pgfscope}%
\begin{pgfscope}%
\pgfsys@transformshift{3.373020in}{1.484904in}%
\pgfsys@useobject{currentmarker}{}%
\end{pgfscope}%
\begin{pgfscope}%
\pgfsys@transformshift{3.417461in}{1.502213in}%
\pgfsys@useobject{currentmarker}{}%
\end{pgfscope}%
\begin{pgfscope}%
\pgfsys@transformshift{3.395624in}{1.489273in}%
\pgfsys@useobject{currentmarker}{}%
\end{pgfscope}%
\begin{pgfscope}%
\pgfsys@transformshift{3.400795in}{1.491437in}%
\pgfsys@useobject{currentmarker}{}%
\end{pgfscope}%
\begin{pgfscope}%
\pgfsys@transformshift{3.397922in}{1.490657in}%
\pgfsys@useobject{currentmarker}{}%
\end{pgfscope}%
\begin{pgfscope}%
\pgfsys@transformshift{3.398522in}{1.490870in}%
\pgfsys@useobject{currentmarker}{}%
\end{pgfscope}%
\begin{pgfscope}%
\pgfsys@transformshift{3.398291in}{1.490592in}%
\pgfsys@useobject{currentmarker}{}%
\end{pgfscope}%
\end{pgfscope}%
\begin{pgfscope}%
\pgfpathrectangle{\pgfqpoint{1.254980in}{0.150000in}}{\pgfqpoint{5.490039in}{5.490039in}}%
\pgfusepath{clip}%
\pgfsetbuttcap%
\pgfsetroundjoin%
\definecolor{currentfill}{rgb}{0.000000,0.000000,1.000000}%
\pgfsetfillcolor{currentfill}%
\pgfsetlinewidth{1.003750pt}%
\definecolor{currentstroke}{rgb}{0.000000,0.000000,1.000000}%
\pgfsetstrokecolor{currentstroke}%
\pgfsetdash{}{0pt}%
\pgfsys@defobject{currentmarker}{\pgfqpoint{-0.069444in}{-0.069444in}}{\pgfqpoint{0.069444in}{0.069444in}}{%
\pgfpathmoveto{\pgfqpoint{0.000000in}{-0.069444in}}%
\pgfpathcurveto{\pgfqpoint{0.018417in}{-0.069444in}}{\pgfqpoint{0.036082in}{-0.062127in}}{\pgfqpoint{0.049105in}{-0.049105in}}%
\pgfpathcurveto{\pgfqpoint{0.062127in}{-0.036082in}}{\pgfqpoint{0.069444in}{-0.018417in}}{\pgfqpoint{0.069444in}{0.000000in}}%
\pgfpathcurveto{\pgfqpoint{0.069444in}{0.018417in}}{\pgfqpoint{0.062127in}{0.036082in}}{\pgfqpoint{0.049105in}{0.049105in}}%
\pgfpathcurveto{\pgfqpoint{0.036082in}{0.062127in}}{\pgfqpoint{0.018417in}{0.069444in}}{\pgfqpoint{0.000000in}{0.069444in}}%
\pgfpathcurveto{\pgfqpoint{-0.018417in}{0.069444in}}{\pgfqpoint{-0.036082in}{0.062127in}}{\pgfqpoint{-0.049105in}{0.049105in}}%
\pgfpathcurveto{\pgfqpoint{-0.062127in}{0.036082in}}{\pgfqpoint{-0.069444in}{0.018417in}}{\pgfqpoint{-0.069444in}{0.000000in}}%
\pgfpathcurveto{\pgfqpoint{-0.069444in}{-0.018417in}}{\pgfqpoint{-0.062127in}{-0.036082in}}{\pgfqpoint{-0.049105in}{-0.049105in}}%
\pgfpathcurveto{\pgfqpoint{-0.036082in}{-0.062127in}}{\pgfqpoint{-0.018417in}{-0.069444in}}{\pgfqpoint{0.000000in}{-0.069444in}}%
\pgfpathlineto{\pgfqpoint{0.000000in}{-0.069444in}}%
\pgfpathclose%
\pgfusepath{stroke,fill}%
}%
\begin{pgfscope}%
\pgfsys@transformshift{3.398291in}{1.490592in}%
\pgfsys@useobject{currentmarker}{}%
\end{pgfscope}%
\end{pgfscope}%
\begin{pgfscope}%
\definecolor{textcolor}{rgb}{0.000000,0.000000,0.000000}%
\pgfsetstrokecolor{textcolor}%
\pgfsetfillcolor{textcolor}%
\pgftext[x=4.000000in,y=5.723372in,,base]{\color{textcolor}\sffamily\fontsize{12.000000}{14.400000}\selectfont 3D Surface Plot}%
\end{pgfscope}%
\begin{pgfscope}%
\pgfpathrectangle{\pgfqpoint{1.254980in}{0.150000in}}{\pgfqpoint{5.490039in}{5.490039in}}%
\pgfusepath{clip}%
\pgfsetbuttcap%
\pgfsetroundjoin%
\definecolor{currentfill}{rgb}{0.124395,0.578002,0.548287}%
\pgfsetfillcolor{currentfill}%
\pgfsetfillopacity{0.700000}%
\pgfsetlinewidth{0.000000pt}%
\definecolor{currentstroke}{rgb}{0.000000,0.000000,0.000000}%
\pgfsetstrokecolor{currentstroke}%
\pgfsetdash{}{0pt}%
\pgfpathmoveto{\pgfqpoint{4.098565in}{3.716184in}}%
\pgfpathlineto{\pgfqpoint{4.111361in}{3.701701in}}%
\pgfpathlineto{\pgfqpoint{4.124158in}{3.687428in}}%
\pgfpathlineto{\pgfqpoint{4.136955in}{3.673366in}}%
\pgfpathlineto{\pgfqpoint{4.149753in}{3.659511in}}%
\pgfpathlineto{\pgfqpoint{4.157140in}{3.684405in}}%
\pgfpathlineto{\pgfqpoint{4.164528in}{3.709697in}}%
\pgfpathlineto{\pgfqpoint{4.171916in}{3.735396in}}%
\pgfpathlineto{\pgfqpoint{4.179305in}{3.761509in}}%
\pgfpathlineto{\pgfqpoint{4.166508in}{3.776012in}}%
\pgfpathlineto{\pgfqpoint{4.153712in}{3.790724in}}%
\pgfpathlineto{\pgfqpoint{4.140915in}{3.805646in}}%
\pgfpathlineto{\pgfqpoint{4.128119in}{3.820781in}}%
\pgfpathlineto{\pgfqpoint{4.120730in}{3.794006in}}%
\pgfpathlineto{\pgfqpoint{4.113341in}{3.767653in}}%
\pgfpathlineto{\pgfqpoint{4.105953in}{3.741715in}}%
\pgfpathlineto{\pgfqpoint{4.098565in}{3.716184in}}%
\pgfpathclose%
\pgfusepath{fill}%
\end{pgfscope}%
\begin{pgfscope}%
\pgfpathrectangle{\pgfqpoint{1.254980in}{0.150000in}}{\pgfqpoint{5.490039in}{5.490039in}}%
\pgfusepath{clip}%
\pgfsetbuttcap%
\pgfsetroundjoin%
\definecolor{currentfill}{rgb}{0.128729,0.563265,0.551229}%
\pgfsetfillcolor{currentfill}%
\pgfsetfillopacity{0.700000}%
\pgfsetlinewidth{0.000000pt}%
\definecolor{currentstroke}{rgb}{0.000000,0.000000,0.000000}%
\pgfsetstrokecolor{currentstroke}%
\pgfsetdash{}{0pt}%
\pgfpathmoveto{\pgfqpoint{4.017826in}{3.675567in}}%
\pgfpathlineto{\pgfqpoint{4.030623in}{3.660846in}}%
\pgfpathlineto{\pgfqpoint{4.043420in}{3.646341in}}%
\pgfpathlineto{\pgfqpoint{4.056216in}{3.632051in}}%
\pgfpathlineto{\pgfqpoint{4.069013in}{3.617974in}}%
\pgfpathlineto{\pgfqpoint{4.076402in}{3.641955in}}%
\pgfpathlineto{\pgfqpoint{4.083790in}{3.666312in}}%
\pgfpathlineto{\pgfqpoint{4.091177in}{3.691052in}}%
\pgfpathlineto{\pgfqpoint{4.098565in}{3.716184in}}%
\pgfpathlineto{\pgfqpoint{4.085769in}{3.730881in}}%
\pgfpathlineto{\pgfqpoint{4.072973in}{3.745791in}}%
\pgfpathlineto{\pgfqpoint{4.060177in}{3.760917in}}%
\pgfpathlineto{\pgfqpoint{4.047380in}{3.776261in}}%
\pgfpathlineto{\pgfqpoint{4.039992in}{3.750496in}}%
\pgfpathlineto{\pgfqpoint{4.032604in}{3.725130in}}%
\pgfpathlineto{\pgfqpoint{4.025215in}{3.700157in}}%
\pgfpathlineto{\pgfqpoint{4.017826in}{3.675567in}}%
\pgfpathclose%
\pgfusepath{fill}%
\end{pgfscope}%
\begin{pgfscope}%
\pgfpathrectangle{\pgfqpoint{1.254980in}{0.150000in}}{\pgfqpoint{5.490039in}{5.490039in}}%
\pgfusepath{clip}%
\pgfsetbuttcap%
\pgfsetroundjoin%
\definecolor{currentfill}{rgb}{0.119738,0.603785,0.541400}%
\pgfsetfillcolor{currentfill}%
\pgfsetfillopacity{0.700000}%
\pgfsetlinewidth{0.000000pt}%
\definecolor{currentstroke}{rgb}{0.000000,0.000000,0.000000}%
\pgfsetstrokecolor{currentstroke}%
\pgfsetdash{}{0pt}%
\pgfpathmoveto{\pgfqpoint{4.047380in}{3.776261in}}%
\pgfpathlineto{\pgfqpoint{4.060177in}{3.760917in}}%
\pgfpathlineto{\pgfqpoint{4.072973in}{3.745791in}}%
\pgfpathlineto{\pgfqpoint{4.085769in}{3.730881in}}%
\pgfpathlineto{\pgfqpoint{4.098565in}{3.716184in}}%
\pgfpathlineto{\pgfqpoint{4.105953in}{3.741715in}}%
\pgfpathlineto{\pgfqpoint{4.113341in}{3.767653in}}%
\pgfpathlineto{\pgfqpoint{4.120730in}{3.794006in}}%
\pgfpathlineto{\pgfqpoint{4.128119in}{3.820781in}}%
\pgfpathlineto{\pgfqpoint{4.115323in}{3.836129in}}%
\pgfpathlineto{\pgfqpoint{4.102526in}{3.851691in}}%
\pgfpathlineto{\pgfqpoint{4.089729in}{3.867471in}}%
\pgfpathlineto{\pgfqpoint{4.076932in}{3.883468in}}%
\pgfpathlineto{\pgfqpoint{4.069544in}{3.856028in}}%
\pgfpathlineto{\pgfqpoint{4.062156in}{3.829019in}}%
\pgfpathlineto{\pgfqpoint{4.054768in}{3.802432in}}%
\pgfpathlineto{\pgfqpoint{4.047380in}{3.776261in}}%
\pgfpathclose%
\pgfusepath{fill}%
\end{pgfscope}%
\begin{pgfscope}%
\pgfpathrectangle{\pgfqpoint{1.254980in}{0.150000in}}{\pgfqpoint{5.490039in}{5.490039in}}%
\pgfusepath{clip}%
\pgfsetbuttcap%
\pgfsetroundjoin%
\definecolor{currentfill}{rgb}{0.122606,0.585371,0.546557}%
\pgfsetfillcolor{currentfill}%
\pgfsetfillopacity{0.700000}%
\pgfsetlinewidth{0.000000pt}%
\definecolor{currentstroke}{rgb}{0.000000,0.000000,0.000000}%
\pgfsetstrokecolor{currentstroke}%
\pgfsetdash{}{0pt}%
\pgfpathmoveto{\pgfqpoint{3.966635in}{3.736646in}}%
\pgfpathlineto{\pgfqpoint{3.979434in}{3.721044in}}%
\pgfpathlineto{\pgfqpoint{3.992232in}{3.705665in}}%
\pgfpathlineto{\pgfqpoint{4.005029in}{3.690506in}}%
\pgfpathlineto{\pgfqpoint{4.017826in}{3.675567in}}%
\pgfpathlineto{\pgfqpoint{4.025215in}{3.700157in}}%
\pgfpathlineto{\pgfqpoint{4.032604in}{3.725130in}}%
\pgfpathlineto{\pgfqpoint{4.039992in}{3.750496in}}%
\pgfpathlineto{\pgfqpoint{4.047380in}{3.776261in}}%
\pgfpathlineto{\pgfqpoint{4.034583in}{3.791823in}}%
\pgfpathlineto{\pgfqpoint{4.021785in}{3.807605in}}%
\pgfpathlineto{\pgfqpoint{4.008987in}{3.823609in}}%
\pgfpathlineto{\pgfqpoint{3.996188in}{3.839836in}}%
\pgfpathlineto{\pgfqpoint{3.988800in}{3.813435in}}%
\pgfpathlineto{\pgfqpoint{3.981413in}{3.787442in}}%
\pgfpathlineto{\pgfqpoint{3.974024in}{3.761848in}}%
\pgfpathlineto{\pgfqpoint{3.966635in}{3.736646in}}%
\pgfpathclose%
\pgfusepath{fill}%
\end{pgfscope}%
\begin{pgfscope}%
\pgfpathrectangle{\pgfqpoint{1.254980in}{0.150000in}}{\pgfqpoint{5.490039in}{5.490039in}}%
\pgfusepath{clip}%
\pgfsetbuttcap%
\pgfsetroundjoin%
\definecolor{currentfill}{rgb}{0.136408,0.541173,0.554483}%
\pgfsetfillcolor{currentfill}%
\pgfsetfillopacity{0.700000}%
\pgfsetlinewidth{0.000000pt}%
\definecolor{currentstroke}{rgb}{0.000000,0.000000,0.000000}%
\pgfsetstrokecolor{currentstroke}%
\pgfsetdash{}{0pt}%
\pgfpathmoveto{\pgfqpoint{4.069013in}{3.617974in}}%
\pgfpathlineto{\pgfqpoint{4.081811in}{3.604110in}}%
\pgfpathlineto{\pgfqpoint{4.094608in}{3.590456in}}%
\pgfpathlineto{\pgfqpoint{4.107407in}{3.577011in}}%
\pgfpathlineto{\pgfqpoint{4.120206in}{3.563774in}}%
\pgfpathlineto{\pgfqpoint{4.127593in}{3.587148in}}%
\pgfpathlineto{\pgfqpoint{4.134980in}{3.610890in}}%
\pgfpathlineto{\pgfqpoint{4.142366in}{3.635009in}}%
\pgfpathlineto{\pgfqpoint{4.149753in}{3.659511in}}%
\pgfpathlineto{\pgfqpoint{4.136955in}{3.673366in}}%
\pgfpathlineto{\pgfqpoint{4.124158in}{3.687428in}}%
\pgfpathlineto{\pgfqpoint{4.111361in}{3.701701in}}%
\pgfpathlineto{\pgfqpoint{4.098565in}{3.716184in}}%
\pgfpathlineto{\pgfqpoint{4.091177in}{3.691052in}}%
\pgfpathlineto{\pgfqpoint{4.083790in}{3.666312in}}%
\pgfpathlineto{\pgfqpoint{4.076402in}{3.641955in}}%
\pgfpathlineto{\pgfqpoint{4.069013in}{3.617974in}}%
\pgfpathclose%
\pgfusepath{fill}%
\end{pgfscope}%
\begin{pgfscope}%
\pgfpathrectangle{\pgfqpoint{1.254980in}{0.150000in}}{\pgfqpoint{5.490039in}{5.490039in}}%
\pgfusepath{clip}%
\pgfsetbuttcap%
\pgfsetroundjoin%
\definecolor{currentfill}{rgb}{0.131172,0.555899,0.552459}%
\pgfsetfillcolor{currentfill}%
\pgfsetfillopacity{0.700000}%
\pgfsetlinewidth{0.000000pt}%
\definecolor{currentstroke}{rgb}{0.000000,0.000000,0.000000}%
\pgfsetstrokecolor{currentstroke}%
\pgfsetdash{}{0pt}%
\pgfpathmoveto{\pgfqpoint{4.149753in}{3.659511in}}%
\pgfpathlineto{\pgfqpoint{4.162552in}{3.645864in}}%
\pgfpathlineto{\pgfqpoint{4.175352in}{3.632423in}}%
\pgfpathlineto{\pgfqpoint{4.188152in}{3.619186in}}%
\pgfpathlineto{\pgfqpoint{4.200954in}{3.606152in}}%
\pgfpathlineto{\pgfqpoint{4.208340in}{3.630411in}}%
\pgfpathlineto{\pgfqpoint{4.215726in}{3.655060in}}%
\pgfpathlineto{\pgfqpoint{4.223113in}{3.680109in}}%
\pgfpathlineto{\pgfqpoint{4.230501in}{3.705564in}}%
\pgfpathlineto{\pgfqpoint{4.217701in}{3.719243in}}%
\pgfpathlineto{\pgfqpoint{4.204902in}{3.733126in}}%
\pgfpathlineto{\pgfqpoint{4.192103in}{3.747215in}}%
\pgfpathlineto{\pgfqpoint{4.179305in}{3.761509in}}%
\pgfpathlineto{\pgfqpoint{4.171916in}{3.735396in}}%
\pgfpathlineto{\pgfqpoint{4.164528in}{3.709697in}}%
\pgfpathlineto{\pgfqpoint{4.157140in}{3.684405in}}%
\pgfpathlineto{\pgfqpoint{4.149753in}{3.659511in}}%
\pgfpathclose%
\pgfusepath{fill}%
\end{pgfscope}%
\begin{pgfscope}%
\pgfpathrectangle{\pgfqpoint{1.254980in}{0.150000in}}{\pgfqpoint{5.490039in}{5.490039in}}%
\pgfusepath{clip}%
\pgfsetbuttcap%
\pgfsetroundjoin%
\definecolor{currentfill}{rgb}{0.119483,0.614817,0.537692}%
\pgfsetfillcolor{currentfill}%
\pgfsetfillopacity{0.700000}%
\pgfsetlinewidth{0.000000pt}%
\definecolor{currentstroke}{rgb}{0.000000,0.000000,0.000000}%
\pgfsetstrokecolor{currentstroke}%
\pgfsetdash{}{0pt}%
\pgfpathmoveto{\pgfqpoint{4.128119in}{3.820781in}}%
\pgfpathlineto{\pgfqpoint{4.140915in}{3.805646in}}%
\pgfpathlineto{\pgfqpoint{4.153712in}{3.790724in}}%
\pgfpathlineto{\pgfqpoint{4.166508in}{3.776012in}}%
\pgfpathlineto{\pgfqpoint{4.179305in}{3.761509in}}%
\pgfpathlineto{\pgfqpoint{4.186696in}{3.788045in}}%
\pgfpathlineto{\pgfqpoint{4.194087in}{3.815012in}}%
\pgfpathlineto{\pgfqpoint{4.201480in}{3.842418in}}%
\pgfpathlineto{\pgfqpoint{4.188683in}{3.857428in}}%
\pgfpathlineto{\pgfqpoint{4.175886in}{3.872648in}}%
\pgfpathlineto{\pgfqpoint{4.163089in}{3.888079in}}%
\pgfpathlineto{\pgfqpoint{4.150292in}{3.903723in}}%
\pgfpathlineto{\pgfqpoint{4.142900in}{3.875631in}}%
\pgfpathlineto{\pgfqpoint{4.135509in}{3.847986in}}%
\pgfpathlineto{\pgfqpoint{4.128119in}{3.820781in}}%
\pgfpathclose%
\pgfusepath{fill}%
\end{pgfscope}%
\begin{pgfscope}%
\pgfpathrectangle{\pgfqpoint{1.254980in}{0.150000in}}{\pgfqpoint{5.490039in}{5.490039in}}%
\pgfusepath{clip}%
\pgfsetbuttcap%
\pgfsetroundjoin%
\definecolor{currentfill}{rgb}{0.121148,0.592739,0.544641}%
\pgfsetfillcolor{currentfill}%
\pgfsetfillopacity{0.700000}%
\pgfsetlinewidth{0.000000pt}%
\definecolor{currentstroke}{rgb}{0.000000,0.000000,0.000000}%
\pgfsetstrokecolor{currentstroke}%
\pgfsetdash{}{0pt}%
\pgfpathmoveto{\pgfqpoint{4.179305in}{3.761509in}}%
\pgfpathlineto{\pgfqpoint{4.192103in}{3.747215in}}%
\pgfpathlineto{\pgfqpoint{4.204902in}{3.733126in}}%
\pgfpathlineto{\pgfqpoint{4.217701in}{3.719243in}}%
\pgfpathlineto{\pgfqpoint{4.230501in}{3.705564in}}%
\pgfpathlineto{\pgfqpoint{4.237891in}{3.731433in}}%
\pgfpathlineto{\pgfqpoint{4.245282in}{3.757725in}}%
\pgfpathlineto{\pgfqpoint{4.252674in}{3.784449in}}%
\pgfpathlineto{\pgfqpoint{4.239875in}{3.798633in}}%
\pgfpathlineto{\pgfqpoint{4.227076in}{3.813022in}}%
\pgfpathlineto{\pgfqpoint{4.214277in}{3.827616in}}%
\pgfpathlineto{\pgfqpoint{4.201480in}{3.842418in}}%
\pgfpathlineto{\pgfqpoint{4.194087in}{3.815012in}}%
\pgfpathlineto{\pgfqpoint{4.186696in}{3.788045in}}%
\pgfpathlineto{\pgfqpoint{4.179305in}{3.761509in}}%
\pgfpathclose%
\pgfusepath{fill}%
\end{pgfscope}%
\begin{pgfscope}%
\pgfpathrectangle{\pgfqpoint{1.254980in}{0.150000in}}{\pgfqpoint{5.490039in}{5.490039in}}%
\pgfusepath{clip}%
\pgfsetbuttcap%
\pgfsetroundjoin%
\definecolor{currentfill}{rgb}{0.120638,0.625828,0.533488}%
\pgfsetfillcolor{currentfill}%
\pgfsetfillopacity{0.700000}%
\pgfsetlinewidth{0.000000pt}%
\definecolor{currentstroke}{rgb}{0.000000,0.000000,0.000000}%
\pgfsetstrokecolor{currentstroke}%
\pgfsetdash{}{0pt}%
\pgfpathmoveto{\pgfqpoint{3.996188in}{3.839836in}}%
\pgfpathlineto{\pgfqpoint{4.008987in}{3.823609in}}%
\pgfpathlineto{\pgfqpoint{4.021785in}{3.807605in}}%
\pgfpathlineto{\pgfqpoint{4.034583in}{3.791823in}}%
\pgfpathlineto{\pgfqpoint{4.047380in}{3.776261in}}%
\pgfpathlineto{\pgfqpoint{4.054768in}{3.802432in}}%
\pgfpathlineto{\pgfqpoint{4.062156in}{3.829019in}}%
\pgfpathlineto{\pgfqpoint{4.069544in}{3.856028in}}%
\pgfpathlineto{\pgfqpoint{4.076932in}{3.883468in}}%
\pgfpathlineto{\pgfqpoint{4.064134in}{3.899685in}}%
\pgfpathlineto{\pgfqpoint{4.051336in}{3.916123in}}%
\pgfpathlineto{\pgfqpoint{4.038536in}{3.932784in}}%
\pgfpathlineto{\pgfqpoint{4.025735in}{3.949669in}}%
\pgfpathlineto{\pgfqpoint{4.018348in}{3.921561in}}%
\pgfpathlineto{\pgfqpoint{4.010962in}{3.893891in}}%
\pgfpathlineto{\pgfqpoint{4.003575in}{3.866652in}}%
\pgfpathlineto{\pgfqpoint{3.996188in}{3.839836in}}%
\pgfpathclose%
\pgfusepath{fill}%
\end{pgfscope}%
\begin{pgfscope}%
\pgfpathrectangle{\pgfqpoint{1.254980in}{0.150000in}}{\pgfqpoint{5.490039in}{5.490039in}}%
\pgfusepath{clip}%
\pgfsetbuttcap%
\pgfsetroundjoin%
\definecolor{currentfill}{rgb}{0.133743,0.548535,0.553541}%
\pgfsetfillcolor{currentfill}%
\pgfsetfillopacity{0.700000}%
\pgfsetlinewidth{0.000000pt}%
\definecolor{currentstroke}{rgb}{0.000000,0.000000,0.000000}%
\pgfsetstrokecolor{currentstroke}%
\pgfsetdash{}{0pt}%
\pgfpathmoveto{\pgfqpoint{3.937067in}{3.639615in}}%
\pgfpathlineto{\pgfqpoint{3.949866in}{3.624607in}}%
\pgfpathlineto{\pgfqpoint{3.962664in}{3.609820in}}%
\pgfpathlineto{\pgfqpoint{3.975463in}{3.595253in}}%
\pgfpathlineto{\pgfqpoint{3.988260in}{3.580906in}}%
\pgfpathlineto{\pgfqpoint{3.995654in}{3.604031in}}%
\pgfpathlineto{\pgfqpoint{4.003045in}{3.627511in}}%
\pgfpathlineto{\pgfqpoint{4.010436in}{3.651354in}}%
\pgfpathlineto{\pgfqpoint{4.017826in}{3.675567in}}%
\pgfpathlineto{\pgfqpoint{4.005029in}{3.690506in}}%
\pgfpathlineto{\pgfqpoint{3.992232in}{3.705665in}}%
\pgfpathlineto{\pgfqpoint{3.979434in}{3.721044in}}%
\pgfpathlineto{\pgfqpoint{3.966635in}{3.736646in}}%
\pgfpathlineto{\pgfqpoint{3.959245in}{3.711830in}}%
\pgfpathlineto{\pgfqpoint{3.951853in}{3.687390in}}%
\pgfpathlineto{\pgfqpoint{3.944461in}{3.663321in}}%
\pgfpathlineto{\pgfqpoint{3.937067in}{3.639615in}}%
\pgfpathclose%
\pgfusepath{fill}%
\end{pgfscope}%
\begin{pgfscope}%
\pgfpathrectangle{\pgfqpoint{1.254980in}{0.150000in}}{\pgfqpoint{5.490039in}{5.490039in}}%
\pgfusepath{clip}%
\pgfsetbuttcap%
\pgfsetroundjoin%
\definecolor{currentfill}{rgb}{0.124780,0.640461,0.527068}%
\pgfsetfillcolor{currentfill}%
\pgfsetfillopacity{0.700000}%
\pgfsetlinewidth{0.000000pt}%
\definecolor{currentstroke}{rgb}{0.000000,0.000000,0.000000}%
\pgfsetstrokecolor{currentstroke}%
\pgfsetdash{}{0pt}%
\pgfpathmoveto{\pgfqpoint{4.076932in}{3.883468in}}%
\pgfpathlineto{\pgfqpoint{4.089729in}{3.867471in}}%
\pgfpathlineto{\pgfqpoint{4.102526in}{3.851691in}}%
\pgfpathlineto{\pgfqpoint{4.115323in}{3.836129in}}%
\pgfpathlineto{\pgfqpoint{4.128119in}{3.820781in}}%
\pgfpathlineto{\pgfqpoint{4.135509in}{3.847986in}}%
\pgfpathlineto{\pgfqpoint{4.142900in}{3.875631in}}%
\pgfpathlineto{\pgfqpoint{4.150292in}{3.903723in}}%
\pgfpathlineto{\pgfqpoint{4.137495in}{3.919580in}}%
\pgfpathlineto{\pgfqpoint{4.124698in}{3.935654in}}%
\pgfpathlineto{\pgfqpoint{4.111901in}{3.951945in}}%
\pgfpathlineto{\pgfqpoint{4.099102in}{3.968454in}}%
\pgfpathlineto{\pgfqpoint{4.091711in}{3.939673in}}%
\pgfpathlineto{\pgfqpoint{4.084321in}{3.911347in}}%
\pgfpathlineto{\pgfqpoint{4.076932in}{3.883468in}}%
\pgfpathclose%
\pgfusepath{fill}%
\end{pgfscope}%
\begin{pgfscope}%
\pgfpathrectangle{\pgfqpoint{1.254980in}{0.150000in}}{\pgfqpoint{5.490039in}{5.490039in}}%
\pgfusepath{clip}%
\pgfsetbuttcap%
\pgfsetroundjoin%
\definecolor{currentfill}{rgb}{0.141935,0.526453,0.555991}%
\pgfsetfillcolor{currentfill}%
\pgfsetfillopacity{0.700000}%
\pgfsetlinewidth{0.000000pt}%
\definecolor{currentstroke}{rgb}{0.000000,0.000000,0.000000}%
\pgfsetstrokecolor{currentstroke}%
\pgfsetdash{}{0pt}%
\pgfpathmoveto{\pgfqpoint{3.988260in}{3.580906in}}%
\pgfpathlineto{\pgfqpoint{4.001058in}{3.566775in}}%
\pgfpathlineto{\pgfqpoint{4.013856in}{3.552861in}}%
\pgfpathlineto{\pgfqpoint{4.026654in}{3.539160in}}%
\pgfpathlineto{\pgfqpoint{4.039453in}{3.525672in}}%
\pgfpathlineto{\pgfqpoint{4.046844in}{3.548219in}}%
\pgfpathlineto{\pgfqpoint{4.054235in}{3.571114in}}%
\pgfpathlineto{\pgfqpoint{4.061624in}{3.594363in}}%
\pgfpathlineto{\pgfqpoint{4.069013in}{3.617974in}}%
\pgfpathlineto{\pgfqpoint{4.056216in}{3.632051in}}%
\pgfpathlineto{\pgfqpoint{4.043420in}{3.646341in}}%
\pgfpathlineto{\pgfqpoint{4.030623in}{3.660846in}}%
\pgfpathlineto{\pgfqpoint{4.017826in}{3.675567in}}%
\pgfpathlineto{\pgfqpoint{4.010436in}{3.651354in}}%
\pgfpathlineto{\pgfqpoint{4.003045in}{3.627511in}}%
\pgfpathlineto{\pgfqpoint{3.995654in}{3.604031in}}%
\pgfpathlineto{\pgfqpoint{3.988260in}{3.580906in}}%
\pgfpathclose%
\pgfusepath{fill}%
\end{pgfscope}%
\begin{pgfscope}%
\pgfpathrectangle{\pgfqpoint{1.254980in}{0.150000in}}{\pgfqpoint{5.490039in}{5.490039in}}%
\pgfusepath{clip}%
\pgfsetbuttcap%
\pgfsetroundjoin%
\definecolor{currentfill}{rgb}{0.119423,0.611141,0.538982}%
\pgfsetfillcolor{currentfill}%
\pgfsetfillopacity{0.700000}%
\pgfsetlinewidth{0.000000pt}%
\definecolor{currentstroke}{rgb}{0.000000,0.000000,0.000000}%
\pgfsetstrokecolor{currentstroke}%
\pgfsetdash{}{0pt}%
\pgfpathmoveto{\pgfqpoint{3.915429in}{3.801315in}}%
\pgfpathlineto{\pgfqpoint{3.928232in}{3.784806in}}%
\pgfpathlineto{\pgfqpoint{3.941034in}{3.768525in}}%
\pgfpathlineto{\pgfqpoint{3.953835in}{3.752473in}}%
\pgfpathlineto{\pgfqpoint{3.966635in}{3.736646in}}%
\pgfpathlineto{\pgfqpoint{3.974024in}{3.761848in}}%
\pgfpathlineto{\pgfqpoint{3.981413in}{3.787442in}}%
\pgfpathlineto{\pgfqpoint{3.988800in}{3.813435in}}%
\pgfpathlineto{\pgfqpoint{3.996188in}{3.839836in}}%
\pgfpathlineto{\pgfqpoint{3.983387in}{3.856289in}}%
\pgfpathlineto{\pgfqpoint{3.970586in}{3.872968in}}%
\pgfpathlineto{\pgfqpoint{3.957783in}{3.889876in}}%
\pgfpathlineto{\pgfqpoint{3.944978in}{3.907015in}}%
\pgfpathlineto{\pgfqpoint{3.937592in}{3.879974in}}%
\pgfpathlineto{\pgfqpoint{3.930205in}{3.853349in}}%
\pgfpathlineto{\pgfqpoint{3.922817in}{3.827132in}}%
\pgfpathlineto{\pgfqpoint{3.915429in}{3.801315in}}%
\pgfpathclose%
\pgfusepath{fill}%
\end{pgfscope}%
\begin{pgfscope}%
\pgfpathrectangle{\pgfqpoint{1.254980in}{0.150000in}}{\pgfqpoint{5.490039in}{5.490039in}}%
\pgfusepath{clip}%
\pgfsetbuttcap%
\pgfsetroundjoin%
\definecolor{currentfill}{rgb}{0.126453,0.570633,0.549841}%
\pgfsetfillcolor{currentfill}%
\pgfsetfillopacity{0.700000}%
\pgfsetlinewidth{0.000000pt}%
\definecolor{currentstroke}{rgb}{0.000000,0.000000,0.000000}%
\pgfsetstrokecolor{currentstroke}%
\pgfsetdash{}{0pt}%
\pgfpathmoveto{\pgfqpoint{4.230501in}{3.705564in}}%
\pgfpathlineto{\pgfqpoint{4.243303in}{3.692087in}}%
\pgfpathlineto{\pgfqpoint{4.256106in}{3.678811in}}%
\pgfpathlineto{\pgfqpoint{4.268911in}{3.665735in}}%
\pgfpathlineto{\pgfqpoint{4.281717in}{3.652859in}}%
\pgfpathlineto{\pgfqpoint{4.289105in}{3.678065in}}%
\pgfpathlineto{\pgfqpoint{4.296494in}{3.703685in}}%
\pgfpathlineto{\pgfqpoint{4.303886in}{3.729729in}}%
\pgfpathlineto{\pgfqpoint{4.291081in}{3.743109in}}%
\pgfpathlineto{\pgfqpoint{4.278277in}{3.756688in}}%
\pgfpathlineto{\pgfqpoint{4.265475in}{3.770467in}}%
\pgfpathlineto{\pgfqpoint{4.252674in}{3.784449in}}%
\pgfpathlineto{\pgfqpoint{4.245282in}{3.757725in}}%
\pgfpathlineto{\pgfqpoint{4.237891in}{3.731433in}}%
\pgfpathlineto{\pgfqpoint{4.230501in}{3.705564in}}%
\pgfpathclose%
\pgfusepath{fill}%
\end{pgfscope}%
\begin{pgfscope}%
\pgfpathrectangle{\pgfqpoint{1.254980in}{0.150000in}}{\pgfqpoint{5.490039in}{5.490039in}}%
\pgfusepath{clip}%
\pgfsetbuttcap%
\pgfsetroundjoin%
\definecolor{currentfill}{rgb}{0.144759,0.519093,0.556572}%
\pgfsetfillcolor{currentfill}%
\pgfsetfillopacity{0.700000}%
\pgfsetlinewidth{0.000000pt}%
\definecolor{currentstroke}{rgb}{0.000000,0.000000,0.000000}%
\pgfsetstrokecolor{currentstroke}%
\pgfsetdash{}{0pt}%
\pgfpathmoveto{\pgfqpoint{4.120206in}{3.563774in}}%
\pgfpathlineto{\pgfqpoint{4.133007in}{3.550743in}}%
\pgfpathlineto{\pgfqpoint{4.145808in}{3.537917in}}%
\pgfpathlineto{\pgfqpoint{4.158611in}{3.525296in}}%
\pgfpathlineto{\pgfqpoint{4.171415in}{3.512876in}}%
\pgfpathlineto{\pgfqpoint{4.178800in}{3.535646in}}%
\pgfpathlineto{\pgfqpoint{4.186184in}{3.558778in}}%
\pgfpathlineto{\pgfqpoint{4.193569in}{3.582277in}}%
\pgfpathlineto{\pgfqpoint{4.200954in}{3.606152in}}%
\pgfpathlineto{\pgfqpoint{4.188152in}{3.619186in}}%
\pgfpathlineto{\pgfqpoint{4.175352in}{3.632423in}}%
\pgfpathlineto{\pgfqpoint{4.162552in}{3.645864in}}%
\pgfpathlineto{\pgfqpoint{4.149753in}{3.659511in}}%
\pgfpathlineto{\pgfqpoint{4.142366in}{3.635009in}}%
\pgfpathlineto{\pgfqpoint{4.134980in}{3.610890in}}%
\pgfpathlineto{\pgfqpoint{4.127593in}{3.587148in}}%
\pgfpathlineto{\pgfqpoint{4.120206in}{3.563774in}}%
\pgfpathclose%
\pgfusepath{fill}%
\end{pgfscope}%
\begin{pgfscope}%
\pgfpathrectangle{\pgfqpoint{1.254980in}{0.150000in}}{\pgfqpoint{5.490039in}{5.490039in}}%
\pgfusepath{clip}%
\pgfsetbuttcap%
\pgfsetroundjoin%
\definecolor{currentfill}{rgb}{0.125394,0.574318,0.549086}%
\pgfsetfillcolor{currentfill}%
\pgfsetfillopacity{0.700000}%
\pgfsetlinewidth{0.000000pt}%
\definecolor{currentstroke}{rgb}{0.000000,0.000000,0.000000}%
\pgfsetstrokecolor{currentstroke}%
\pgfsetdash{}{0pt}%
\pgfpathmoveto{\pgfqpoint{3.885861in}{3.701902in}}%
\pgfpathlineto{\pgfqpoint{3.898664in}{3.685989in}}%
\pgfpathlineto{\pgfqpoint{3.911466in}{3.670305in}}%
\pgfpathlineto{\pgfqpoint{3.924267in}{3.654848in}}%
\pgfpathlineto{\pgfqpoint{3.937067in}{3.639615in}}%
\pgfpathlineto{\pgfqpoint{3.944461in}{3.663321in}}%
\pgfpathlineto{\pgfqpoint{3.951853in}{3.687390in}}%
\pgfpathlineto{\pgfqpoint{3.959245in}{3.711830in}}%
\pgfpathlineto{\pgfqpoint{3.966635in}{3.736646in}}%
\pgfpathlineto{\pgfqpoint{3.953835in}{3.752473in}}%
\pgfpathlineto{\pgfqpoint{3.941034in}{3.768525in}}%
\pgfpathlineto{\pgfqpoint{3.928232in}{3.784806in}}%
\pgfpathlineto{\pgfqpoint{3.915429in}{3.801315in}}%
\pgfpathlineto{\pgfqpoint{3.908039in}{3.775891in}}%
\pgfpathlineto{\pgfqpoint{3.900648in}{3.750852in}}%
\pgfpathlineto{\pgfqpoint{3.893255in}{3.726191in}}%
\pgfpathlineto{\pgfqpoint{3.885861in}{3.701902in}}%
\pgfpathclose%
\pgfusepath{fill}%
\end{pgfscope}%
\begin{pgfscope}%
\pgfpathrectangle{\pgfqpoint{1.254980in}{0.150000in}}{\pgfqpoint{5.490039in}{5.490039in}}%
\pgfusepath{clip}%
\pgfsetbuttcap%
\pgfsetroundjoin%
\definecolor{currentfill}{rgb}{0.137770,0.537492,0.554906}%
\pgfsetfillcolor{currentfill}%
\pgfsetfillopacity{0.700000}%
\pgfsetlinewidth{0.000000pt}%
\definecolor{currentstroke}{rgb}{0.000000,0.000000,0.000000}%
\pgfsetstrokecolor{currentstroke}%
\pgfsetdash{}{0pt}%
\pgfpathmoveto{\pgfqpoint{4.200954in}{3.606152in}}%
\pgfpathlineto{\pgfqpoint{4.213758in}{3.593320in}}%
\pgfpathlineto{\pgfqpoint{4.226563in}{3.580689in}}%
\pgfpathlineto{\pgfqpoint{4.239369in}{3.568257in}}%
\pgfpathlineto{\pgfqpoint{4.252178in}{3.556023in}}%
\pgfpathlineto{\pgfqpoint{4.259561in}{3.579649in}}%
\pgfpathlineto{\pgfqpoint{4.266945in}{3.603659in}}%
\pgfpathlineto{\pgfqpoint{4.274330in}{3.628059in}}%
\pgfpathlineto{\pgfqpoint{4.281717in}{3.652859in}}%
\pgfpathlineto{\pgfqpoint{4.268911in}{3.665735in}}%
\pgfpathlineto{\pgfqpoint{4.256106in}{3.678811in}}%
\pgfpathlineto{\pgfqpoint{4.243303in}{3.692087in}}%
\pgfpathlineto{\pgfqpoint{4.230501in}{3.705564in}}%
\pgfpathlineto{\pgfqpoint{4.223113in}{3.680109in}}%
\pgfpathlineto{\pgfqpoint{4.215726in}{3.655060in}}%
\pgfpathlineto{\pgfqpoint{4.208340in}{3.630411in}}%
\pgfpathlineto{\pgfqpoint{4.200954in}{3.606152in}}%
\pgfpathclose%
\pgfusepath{fill}%
\end{pgfscope}%
\begin{pgfscope}%
\pgfpathrectangle{\pgfqpoint{1.254980in}{0.150000in}}{\pgfqpoint{5.490039in}{5.490039in}}%
\pgfusepath{clip}%
\pgfsetbuttcap%
\pgfsetroundjoin%
\definecolor{currentfill}{rgb}{0.150476,0.504369,0.557430}%
\pgfsetfillcolor{currentfill}%
\pgfsetfillopacity{0.700000}%
\pgfsetlinewidth{0.000000pt}%
\definecolor{currentstroke}{rgb}{0.000000,0.000000,0.000000}%
\pgfsetstrokecolor{currentstroke}%
\pgfsetdash{}{0pt}%
\pgfpathmoveto{\pgfqpoint{4.039453in}{3.525672in}}%
\pgfpathlineto{\pgfqpoint{4.052252in}{3.512396in}}%
\pgfpathlineto{\pgfqpoint{4.065051in}{3.499329in}}%
\pgfpathlineto{\pgfqpoint{4.077852in}{3.486471in}}%
\pgfpathlineto{\pgfqpoint{4.090654in}{3.473820in}}%
\pgfpathlineto{\pgfqpoint{4.098043in}{3.495792in}}%
\pgfpathlineto{\pgfqpoint{4.105431in}{3.518103in}}%
\pgfpathlineto{\pgfqpoint{4.112819in}{3.540761in}}%
\pgfpathlineto{\pgfqpoint{4.120206in}{3.563774in}}%
\pgfpathlineto{\pgfqpoint{4.107407in}{3.577011in}}%
\pgfpathlineto{\pgfqpoint{4.094608in}{3.590456in}}%
\pgfpathlineto{\pgfqpoint{4.081811in}{3.604110in}}%
\pgfpathlineto{\pgfqpoint{4.069013in}{3.617974in}}%
\pgfpathlineto{\pgfqpoint{4.061624in}{3.594363in}}%
\pgfpathlineto{\pgfqpoint{4.054235in}{3.571114in}}%
\pgfpathlineto{\pgfqpoint{4.046844in}{3.548219in}}%
\pgfpathlineto{\pgfqpoint{4.039453in}{3.525672in}}%
\pgfpathclose%
\pgfusepath{fill}%
\end{pgfscope}%
\begin{pgfscope}%
\pgfpathrectangle{\pgfqpoint{1.254980in}{0.150000in}}{\pgfqpoint{5.490039in}{5.490039in}}%
\pgfusepath{clip}%
\pgfsetbuttcap%
\pgfsetroundjoin%
\definecolor{currentfill}{rgb}{0.132268,0.655014,0.519661}%
\pgfsetfillcolor{currentfill}%
\pgfsetfillopacity{0.700000}%
\pgfsetlinewidth{0.000000pt}%
\definecolor{currentstroke}{rgb}{0.000000,0.000000,0.000000}%
\pgfsetstrokecolor{currentstroke}%
\pgfsetdash{}{0pt}%
\pgfpathmoveto{\pgfqpoint{3.944978in}{3.907015in}}%
\pgfpathlineto{\pgfqpoint{3.957783in}{3.889876in}}%
\pgfpathlineto{\pgfqpoint{3.970586in}{3.872968in}}%
\pgfpathlineto{\pgfqpoint{3.983387in}{3.856289in}}%
\pgfpathlineto{\pgfqpoint{3.996188in}{3.839836in}}%
\pgfpathlineto{\pgfqpoint{4.003575in}{3.866652in}}%
\pgfpathlineto{\pgfqpoint{4.010962in}{3.893891in}}%
\pgfpathlineto{\pgfqpoint{4.018348in}{3.921561in}}%
\pgfpathlineto{\pgfqpoint{4.025735in}{3.949669in}}%
\pgfpathlineto{\pgfqpoint{4.012933in}{3.966781in}}%
\pgfpathlineto{\pgfqpoint{4.000130in}{3.984120in}}%
\pgfpathlineto{\pgfqpoint{3.987325in}{4.001689in}}%
\pgfpathlineto{\pgfqpoint{3.974518in}{4.019489in}}%
\pgfpathlineto{\pgfqpoint{3.967133in}{3.990708in}}%
\pgfpathlineto{\pgfqpoint{3.959748in}{3.962373in}}%
\pgfpathlineto{\pgfqpoint{3.952363in}{3.934479in}}%
\pgfpathlineto{\pgfqpoint{3.944978in}{3.907015in}}%
\pgfpathclose%
\pgfusepath{fill}%
\end{pgfscope}%
\begin{pgfscope}%
\pgfpathrectangle{\pgfqpoint{1.254980in}{0.150000in}}{\pgfqpoint{5.490039in}{5.490039in}}%
\pgfusepath{clip}%
\pgfsetbuttcap%
\pgfsetroundjoin%
\definecolor{currentfill}{rgb}{0.140210,0.665859,0.513427}%
\pgfsetfillcolor{currentfill}%
\pgfsetfillopacity{0.700000}%
\pgfsetlinewidth{0.000000pt}%
\definecolor{currentstroke}{rgb}{0.000000,0.000000,0.000000}%
\pgfsetstrokecolor{currentstroke}%
\pgfsetdash{}{0pt}%
\pgfpathmoveto{\pgfqpoint{4.025735in}{3.949669in}}%
\pgfpathlineto{\pgfqpoint{4.038536in}{3.932784in}}%
\pgfpathlineto{\pgfqpoint{4.051336in}{3.916123in}}%
\pgfpathlineto{\pgfqpoint{4.064134in}{3.899685in}}%
\pgfpathlineto{\pgfqpoint{4.076932in}{3.883468in}}%
\pgfpathlineto{\pgfqpoint{4.084321in}{3.911347in}}%
\pgfpathlineto{\pgfqpoint{4.091711in}{3.939673in}}%
\pgfpathlineto{\pgfqpoint{4.099102in}{3.968454in}}%
\pgfpathlineto{\pgfqpoint{4.086303in}{3.985184in}}%
\pgfpathlineto{\pgfqpoint{4.073503in}{4.002136in}}%
\pgfpathlineto{\pgfqpoint{4.060702in}{4.019311in}}%
\pgfpathlineto{\pgfqpoint{4.047899in}{4.036711in}}%
\pgfpathlineto{\pgfqpoint{4.040511in}{4.007236in}}%
\pgfpathlineto{\pgfqpoint{4.033123in}{3.978225in}}%
\pgfpathlineto{\pgfqpoint{4.025735in}{3.949669in}}%
\pgfpathclose%
\pgfusepath{fill}%
\end{pgfscope}%
\begin{pgfscope}%
\pgfpathrectangle{\pgfqpoint{1.254980in}{0.150000in}}{\pgfqpoint{5.490039in}{5.490039in}}%
\pgfusepath{clip}%
\pgfsetbuttcap%
\pgfsetroundjoin%
\definecolor{currentfill}{rgb}{0.120092,0.600104,0.542530}%
\pgfsetfillcolor{currentfill}%
\pgfsetfillopacity{0.700000}%
\pgfsetlinewidth{0.000000pt}%
\definecolor{currentstroke}{rgb}{0.000000,0.000000,0.000000}%
\pgfsetstrokecolor{currentstroke}%
\pgfsetdash{}{0pt}%
\pgfpathmoveto{\pgfqpoint{3.834632in}{3.767874in}}%
\pgfpathlineto{\pgfqpoint{3.847442in}{3.751029in}}%
\pgfpathlineto{\pgfqpoint{3.860250in}{3.734420in}}%
\pgfpathlineto{\pgfqpoint{3.873056in}{3.718045in}}%
\pgfpathlineto{\pgfqpoint{3.885861in}{3.701902in}}%
\pgfpathlineto{\pgfqpoint{3.893255in}{3.726191in}}%
\pgfpathlineto{\pgfqpoint{3.900648in}{3.750852in}}%
\pgfpathlineto{\pgfqpoint{3.908039in}{3.775891in}}%
\pgfpathlineto{\pgfqpoint{3.915429in}{3.801315in}}%
\pgfpathlineto{\pgfqpoint{3.902624in}{3.818056in}}%
\pgfpathlineto{\pgfqpoint{3.889817in}{3.835029in}}%
\pgfpathlineto{\pgfqpoint{3.877008in}{3.852238in}}%
\pgfpathlineto{\pgfqpoint{3.864197in}{3.869682in}}%
\pgfpathlineto{\pgfqpoint{3.856808in}{3.843647in}}%
\pgfpathlineto{\pgfqpoint{3.849418in}{3.818006in}}%
\pgfpathlineto{\pgfqpoint{3.842026in}{3.792750in}}%
\pgfpathlineto{\pgfqpoint{3.834632in}{3.767874in}}%
\pgfpathclose%
\pgfusepath{fill}%
\end{pgfscope}%
\begin{pgfscope}%
\pgfpathrectangle{\pgfqpoint{1.254980in}{0.150000in}}{\pgfqpoint{5.490039in}{5.490039in}}%
\pgfusepath{clip}%
\pgfsetbuttcap%
\pgfsetroundjoin%
\definecolor{currentfill}{rgb}{0.132444,0.552216,0.553018}%
\pgfsetfillcolor{currentfill}%
\pgfsetfillopacity{0.700000}%
\pgfsetlinewidth{0.000000pt}%
\definecolor{currentstroke}{rgb}{0.000000,0.000000,0.000000}%
\pgfsetstrokecolor{currentstroke}%
\pgfsetdash{}{0pt}%
\pgfpathmoveto{\pgfqpoint{4.281717in}{3.652859in}}%
\pgfpathlineto{\pgfqpoint{4.294525in}{3.640179in}}%
\pgfpathlineto{\pgfqpoint{4.307334in}{3.627697in}}%
\pgfpathlineto{\pgfqpoint{4.320146in}{3.615409in}}%
\pgfpathlineto{\pgfqpoint{4.332960in}{3.603316in}}%
\pgfpathlineto{\pgfqpoint{4.340346in}{3.627861in}}%
\pgfpathlineto{\pgfqpoint{4.347733in}{3.652813in}}%
\pgfpathlineto{\pgfqpoint{4.355123in}{3.678180in}}%
\pgfpathlineto{\pgfqpoint{4.342310in}{3.690774in}}%
\pgfpathlineto{\pgfqpoint{4.329500in}{3.703563in}}%
\pgfpathlineto{\pgfqpoint{4.316692in}{3.716548in}}%
\pgfpathlineto{\pgfqpoint{4.303886in}{3.729729in}}%
\pgfpathlineto{\pgfqpoint{4.296494in}{3.703685in}}%
\pgfpathlineto{\pgfqpoint{4.289105in}{3.678065in}}%
\pgfpathlineto{\pgfqpoint{4.281717in}{3.652859in}}%
\pgfpathclose%
\pgfusepath{fill}%
\end{pgfscope}%
\begin{pgfscope}%
\pgfpathrectangle{\pgfqpoint{1.254980in}{0.150000in}}{\pgfqpoint{5.490039in}{5.490039in}}%
\pgfusepath{clip}%
\pgfsetbuttcap%
\pgfsetroundjoin%
\definecolor{currentfill}{rgb}{0.147607,0.511733,0.557049}%
\pgfsetfillcolor{currentfill}%
\pgfsetfillopacity{0.700000}%
\pgfsetlinewidth{0.000000pt}%
\definecolor{currentstroke}{rgb}{0.000000,0.000000,0.000000}%
\pgfsetstrokecolor{currentstroke}%
\pgfsetdash{}{0pt}%
\pgfpathmoveto{\pgfqpoint{3.907474in}{3.548285in}}%
\pgfpathlineto{\pgfqpoint{3.920274in}{3.533839in}}%
\pgfpathlineto{\pgfqpoint{3.933074in}{3.519614in}}%
\pgfpathlineto{\pgfqpoint{3.945874in}{3.505609in}}%
\pgfpathlineto{\pgfqpoint{3.958674in}{3.491822in}}%
\pgfpathlineto{\pgfqpoint{3.966073in}{3.513594in}}%
\pgfpathlineto{\pgfqpoint{3.973470in}{3.535694in}}%
\pgfpathlineto{\pgfqpoint{3.980866in}{3.558129in}}%
\pgfpathlineto{\pgfqpoint{3.988260in}{3.580906in}}%
\pgfpathlineto{\pgfqpoint{3.975463in}{3.595253in}}%
\pgfpathlineto{\pgfqpoint{3.962664in}{3.609820in}}%
\pgfpathlineto{\pgfqpoint{3.949866in}{3.624607in}}%
\pgfpathlineto{\pgfqpoint{3.937067in}{3.639615in}}%
\pgfpathlineto{\pgfqpoint{3.929671in}{3.616266in}}%
\pgfpathlineto{\pgfqpoint{3.922274in}{3.593265in}}%
\pgfpathlineto{\pgfqpoint{3.914875in}{3.570607in}}%
\pgfpathlineto{\pgfqpoint{3.907474in}{3.548285in}}%
\pgfpathclose%
\pgfusepath{fill}%
\end{pgfscope}%
\begin{pgfscope}%
\pgfpathrectangle{\pgfqpoint{1.254980in}{0.150000in}}{\pgfqpoint{5.490039in}{5.490039in}}%
\pgfusepath{clip}%
\pgfsetbuttcap%
\pgfsetroundjoin%
\definecolor{currentfill}{rgb}{0.137770,0.537492,0.554906}%
\pgfsetfillcolor{currentfill}%
\pgfsetfillopacity{0.700000}%
\pgfsetlinewidth{0.000000pt}%
\definecolor{currentstroke}{rgb}{0.000000,0.000000,0.000000}%
\pgfsetstrokecolor{currentstroke}%
\pgfsetdash{}{0pt}%
\pgfpathmoveto{\pgfqpoint{3.856265in}{3.608315in}}%
\pgfpathlineto{\pgfqpoint{3.869069in}{3.592967in}}%
\pgfpathlineto{\pgfqpoint{3.881871in}{3.577848in}}%
\pgfpathlineto{\pgfqpoint{3.894673in}{3.562954in}}%
\pgfpathlineto{\pgfqpoint{3.907474in}{3.548285in}}%
\pgfpathlineto{\pgfqpoint{3.914875in}{3.570607in}}%
\pgfpathlineto{\pgfqpoint{3.922274in}{3.593265in}}%
\pgfpathlineto{\pgfqpoint{3.929671in}{3.616266in}}%
\pgfpathlineto{\pgfqpoint{3.937067in}{3.639615in}}%
\pgfpathlineto{\pgfqpoint{3.924267in}{3.654848in}}%
\pgfpathlineto{\pgfqpoint{3.911466in}{3.670305in}}%
\pgfpathlineto{\pgfqpoint{3.898664in}{3.685989in}}%
\pgfpathlineto{\pgfqpoint{3.885861in}{3.701902in}}%
\pgfpathlineto{\pgfqpoint{3.878465in}{3.677976in}}%
\pgfpathlineto{\pgfqpoint{3.871067in}{3.654408in}}%
\pgfpathlineto{\pgfqpoint{3.863667in}{3.631190in}}%
\pgfpathlineto{\pgfqpoint{3.856265in}{3.608315in}}%
\pgfpathclose%
\pgfusepath{fill}%
\end{pgfscope}%
\begin{pgfscope}%
\pgfpathrectangle{\pgfqpoint{1.254980in}{0.150000in}}{\pgfqpoint{5.490039in}{5.490039in}}%
\pgfusepath{clip}%
\pgfsetbuttcap%
\pgfsetroundjoin%
\definecolor{currentfill}{rgb}{0.124780,0.640461,0.527068}%
\pgfsetfillcolor{currentfill}%
\pgfsetfillopacity{0.700000}%
\pgfsetlinewidth{0.000000pt}%
\definecolor{currentstroke}{rgb}{0.000000,0.000000,0.000000}%
\pgfsetstrokecolor{currentstroke}%
\pgfsetdash{}{0pt}%
\pgfpathmoveto{\pgfqpoint{3.864197in}{3.869682in}}%
\pgfpathlineto{\pgfqpoint{3.877008in}{3.852238in}}%
\pgfpathlineto{\pgfqpoint{3.889817in}{3.835029in}}%
\pgfpathlineto{\pgfqpoint{3.902624in}{3.818056in}}%
\pgfpathlineto{\pgfqpoint{3.915429in}{3.801315in}}%
\pgfpathlineto{\pgfqpoint{3.922817in}{3.827132in}}%
\pgfpathlineto{\pgfqpoint{3.930205in}{3.853349in}}%
\pgfpathlineto{\pgfqpoint{3.937592in}{3.879974in}}%
\pgfpathlineto{\pgfqpoint{3.944978in}{3.907015in}}%
\pgfpathlineto{\pgfqpoint{3.932171in}{3.924385in}}%
\pgfpathlineto{\pgfqpoint{3.919363in}{3.941990in}}%
\pgfpathlineto{\pgfqpoint{3.906552in}{3.959830in}}%
\pgfpathlineto{\pgfqpoint{3.893739in}{3.977907in}}%
\pgfpathlineto{\pgfqpoint{3.886355in}{3.950223in}}%
\pgfpathlineto{\pgfqpoint{3.878971in}{3.922963in}}%
\pgfpathlineto{\pgfqpoint{3.871584in}{3.896118in}}%
\pgfpathlineto{\pgfqpoint{3.864197in}{3.869682in}}%
\pgfpathclose%
\pgfusepath{fill}%
\end{pgfscope}%
\begin{pgfscope}%
\pgfpathrectangle{\pgfqpoint{1.254980in}{0.150000in}}{\pgfqpoint{5.490039in}{5.490039in}}%
\pgfusepath{clip}%
\pgfsetbuttcap%
\pgfsetroundjoin%
\definecolor{currentfill}{rgb}{0.151918,0.500685,0.557587}%
\pgfsetfillcolor{currentfill}%
\pgfsetfillopacity{0.700000}%
\pgfsetlinewidth{0.000000pt}%
\definecolor{currentstroke}{rgb}{0.000000,0.000000,0.000000}%
\pgfsetstrokecolor{currentstroke}%
\pgfsetdash{}{0pt}%
\pgfpathmoveto{\pgfqpoint{4.171415in}{3.512876in}}%
\pgfpathlineto{\pgfqpoint{4.184221in}{3.500658in}}%
\pgfpathlineto{\pgfqpoint{4.197029in}{3.488640in}}%
\pgfpathlineto{\pgfqpoint{4.209838in}{3.476821in}}%
\pgfpathlineto{\pgfqpoint{4.222650in}{3.465200in}}%
\pgfpathlineto{\pgfqpoint{4.230031in}{3.487368in}}%
\pgfpathlineto{\pgfqpoint{4.237413in}{3.509890in}}%
\pgfpathlineto{\pgfqpoint{4.244795in}{3.532772in}}%
\pgfpathlineto{\pgfqpoint{4.252178in}{3.556023in}}%
\pgfpathlineto{\pgfqpoint{4.239369in}{3.568257in}}%
\pgfpathlineto{\pgfqpoint{4.226563in}{3.580689in}}%
\pgfpathlineto{\pgfqpoint{4.213758in}{3.593320in}}%
\pgfpathlineto{\pgfqpoint{4.200954in}{3.606152in}}%
\pgfpathlineto{\pgfqpoint{4.193569in}{3.582277in}}%
\pgfpathlineto{\pgfqpoint{4.186184in}{3.558778in}}%
\pgfpathlineto{\pgfqpoint{4.178800in}{3.535646in}}%
\pgfpathlineto{\pgfqpoint{4.171415in}{3.512876in}}%
\pgfpathclose%
\pgfusepath{fill}%
\end{pgfscope}%
\begin{pgfscope}%
\pgfpathrectangle{\pgfqpoint{1.254980in}{0.150000in}}{\pgfqpoint{5.490039in}{5.490039in}}%
\pgfusepath{clip}%
\pgfsetbuttcap%
\pgfsetroundjoin%
\definecolor{currentfill}{rgb}{0.154815,0.493313,0.557840}%
\pgfsetfillcolor{currentfill}%
\pgfsetfillopacity{0.700000}%
\pgfsetlinewidth{0.000000pt}%
\definecolor{currentstroke}{rgb}{0.000000,0.000000,0.000000}%
\pgfsetstrokecolor{currentstroke}%
\pgfsetdash{}{0pt}%
\pgfpathmoveto{\pgfqpoint{3.958674in}{3.491822in}}%
\pgfpathlineto{\pgfqpoint{3.971473in}{3.478252in}}%
\pgfpathlineto{\pgfqpoint{3.984273in}{3.464896in}}%
\pgfpathlineto{\pgfqpoint{3.997073in}{3.451755in}}%
\pgfpathlineto{\pgfqpoint{4.009874in}{3.438825in}}%
\pgfpathlineto{\pgfqpoint{4.017271in}{3.460049in}}%
\pgfpathlineto{\pgfqpoint{4.024666in}{3.481594in}}%
\pgfpathlineto{\pgfqpoint{4.032060in}{3.503466in}}%
\pgfpathlineto{\pgfqpoint{4.039453in}{3.525672in}}%
\pgfpathlineto{\pgfqpoint{4.026654in}{3.539160in}}%
\pgfpathlineto{\pgfqpoint{4.013856in}{3.552861in}}%
\pgfpathlineto{\pgfqpoint{4.001058in}{3.566775in}}%
\pgfpathlineto{\pgfqpoint{3.988260in}{3.580906in}}%
\pgfpathlineto{\pgfqpoint{3.980866in}{3.558129in}}%
\pgfpathlineto{\pgfqpoint{3.973470in}{3.535694in}}%
\pgfpathlineto{\pgfqpoint{3.966073in}{3.513594in}}%
\pgfpathlineto{\pgfqpoint{3.958674in}{3.491822in}}%
\pgfpathclose%
\pgfusepath{fill}%
\end{pgfscope}%
\begin{pgfscope}%
\pgfpathrectangle{\pgfqpoint{1.254980in}{0.150000in}}{\pgfqpoint{5.490039in}{5.490039in}}%
\pgfusepath{clip}%
\pgfsetbuttcap%
\pgfsetroundjoin%
\definecolor{currentfill}{rgb}{0.144759,0.519093,0.556572}%
\pgfsetfillcolor{currentfill}%
\pgfsetfillopacity{0.700000}%
\pgfsetlinewidth{0.000000pt}%
\definecolor{currentstroke}{rgb}{0.000000,0.000000,0.000000}%
\pgfsetstrokecolor{currentstroke}%
\pgfsetdash{}{0pt}%
\pgfpathmoveto{\pgfqpoint{4.252178in}{3.556023in}}%
\pgfpathlineto{\pgfqpoint{4.264988in}{3.543986in}}%
\pgfpathlineto{\pgfqpoint{4.277801in}{3.532145in}}%
\pgfpathlineto{\pgfqpoint{4.290616in}{3.520499in}}%
\pgfpathlineto{\pgfqpoint{4.303433in}{3.509046in}}%
\pgfpathlineto{\pgfqpoint{4.310813in}{3.532042in}}%
\pgfpathlineto{\pgfqpoint{4.318194in}{3.555414in}}%
\pgfpathlineto{\pgfqpoint{4.325577in}{3.579169in}}%
\pgfpathlineto{\pgfqpoint{4.332960in}{3.603316in}}%
\pgfpathlineto{\pgfqpoint{4.320146in}{3.615409in}}%
\pgfpathlineto{\pgfqpoint{4.307334in}{3.627697in}}%
\pgfpathlineto{\pgfqpoint{4.294525in}{3.640179in}}%
\pgfpathlineto{\pgfqpoint{4.281717in}{3.652859in}}%
\pgfpathlineto{\pgfqpoint{4.274330in}{3.628059in}}%
\pgfpathlineto{\pgfqpoint{4.266945in}{3.603659in}}%
\pgfpathlineto{\pgfqpoint{4.259561in}{3.579649in}}%
\pgfpathlineto{\pgfqpoint{4.252178in}{3.556023in}}%
\pgfpathclose%
\pgfusepath{fill}%
\end{pgfscope}%
\begin{pgfscope}%
\pgfpathrectangle{\pgfqpoint{1.254980in}{0.150000in}}{\pgfqpoint{5.490039in}{5.490039in}}%
\pgfusepath{clip}%
\pgfsetbuttcap%
\pgfsetroundjoin%
\definecolor{currentfill}{rgb}{0.157729,0.485932,0.558013}%
\pgfsetfillcolor{currentfill}%
\pgfsetfillopacity{0.700000}%
\pgfsetlinewidth{0.000000pt}%
\definecolor{currentstroke}{rgb}{0.000000,0.000000,0.000000}%
\pgfsetstrokecolor{currentstroke}%
\pgfsetdash{}{0pt}%
\pgfpathmoveto{\pgfqpoint{4.090654in}{3.473820in}}%
\pgfpathlineto{\pgfqpoint{4.103456in}{3.461376in}}%
\pgfpathlineto{\pgfqpoint{4.116261in}{3.449135in}}%
\pgfpathlineto{\pgfqpoint{4.129066in}{3.437098in}}%
\pgfpathlineto{\pgfqpoint{4.141873in}{3.425263in}}%
\pgfpathlineto{\pgfqpoint{4.149260in}{3.446660in}}%
\pgfpathlineto{\pgfqpoint{4.156645in}{3.468391in}}%
\pgfpathlineto{\pgfqpoint{4.164030in}{3.490460in}}%
\pgfpathlineto{\pgfqpoint{4.171415in}{3.512876in}}%
\pgfpathlineto{\pgfqpoint{4.158611in}{3.525296in}}%
\pgfpathlineto{\pgfqpoint{4.145808in}{3.537917in}}%
\pgfpathlineto{\pgfqpoint{4.133007in}{3.550743in}}%
\pgfpathlineto{\pgfqpoint{4.120206in}{3.563774in}}%
\pgfpathlineto{\pgfqpoint{4.112819in}{3.540761in}}%
\pgfpathlineto{\pgfqpoint{4.105431in}{3.518103in}}%
\pgfpathlineto{\pgfqpoint{4.098043in}{3.495792in}}%
\pgfpathlineto{\pgfqpoint{4.090654in}{3.473820in}}%
\pgfpathclose%
\pgfusepath{fill}%
\end{pgfscope}%
\begin{pgfscope}%
\pgfpathrectangle{\pgfqpoint{1.254980in}{0.150000in}}{\pgfqpoint{5.490039in}{5.490039in}}%
\pgfusepath{clip}%
\pgfsetbuttcap%
\pgfsetroundjoin%
\definecolor{currentfill}{rgb}{0.129933,0.559582,0.551864}%
\pgfsetfillcolor{currentfill}%
\pgfsetfillopacity{0.700000}%
\pgfsetlinewidth{0.000000pt}%
\definecolor{currentstroke}{rgb}{0.000000,0.000000,0.000000}%
\pgfsetstrokecolor{currentstroke}%
\pgfsetdash{}{0pt}%
\pgfpathmoveto{\pgfqpoint{3.805035in}{3.672018in}}%
\pgfpathlineto{\pgfqpoint{3.817845in}{3.655742in}}%
\pgfpathlineto{\pgfqpoint{3.830653in}{3.639700in}}%
\pgfpathlineto{\pgfqpoint{3.843460in}{3.623892in}}%
\pgfpathlineto{\pgfqpoint{3.856265in}{3.608315in}}%
\pgfpathlineto{\pgfqpoint{3.863667in}{3.631190in}}%
\pgfpathlineto{\pgfqpoint{3.871067in}{3.654408in}}%
\pgfpathlineto{\pgfqpoint{3.878465in}{3.677976in}}%
\pgfpathlineto{\pgfqpoint{3.885861in}{3.701902in}}%
\pgfpathlineto{\pgfqpoint{3.873056in}{3.718045in}}%
\pgfpathlineto{\pgfqpoint{3.860250in}{3.734420in}}%
\pgfpathlineto{\pgfqpoint{3.847442in}{3.751029in}}%
\pgfpathlineto{\pgfqpoint{3.834632in}{3.767874in}}%
\pgfpathlineto{\pgfqpoint{3.827236in}{3.743369in}}%
\pgfpathlineto{\pgfqpoint{3.819838in}{3.719230in}}%
\pgfpathlineto{\pgfqpoint{3.812438in}{3.695448in}}%
\pgfpathlineto{\pgfqpoint{3.805035in}{3.672018in}}%
\pgfpathclose%
\pgfusepath{fill}%
\end{pgfscope}%
\begin{pgfscope}%
\pgfpathrectangle{\pgfqpoint{1.254980in}{0.150000in}}{\pgfqpoint{5.490039in}{5.490039in}}%
\pgfusepath{clip}%
\pgfsetbuttcap%
\pgfsetroundjoin%
\definecolor{currentfill}{rgb}{0.170948,0.694384,0.493803}%
\pgfsetfillcolor{currentfill}%
\pgfsetfillopacity{0.700000}%
\pgfsetlinewidth{0.000000pt}%
\definecolor{currentstroke}{rgb}{0.000000,0.000000,0.000000}%
\pgfsetstrokecolor{currentstroke}%
\pgfsetdash{}{0pt}%
\pgfpathmoveto{\pgfqpoint{3.974518in}{4.019489in}}%
\pgfpathlineto{\pgfqpoint{3.987325in}{4.001689in}}%
\pgfpathlineto{\pgfqpoint{4.000130in}{3.984120in}}%
\pgfpathlineto{\pgfqpoint{4.012933in}{3.966781in}}%
\pgfpathlineto{\pgfqpoint{4.025735in}{3.949669in}}%
\pgfpathlineto{\pgfqpoint{4.033123in}{3.978225in}}%
\pgfpathlineto{\pgfqpoint{4.040511in}{4.007236in}}%
\pgfpathlineto{\pgfqpoint{4.047899in}{4.036711in}}%
\pgfpathlineto{\pgfqpoint{4.035095in}{4.054339in}}%
\pgfpathlineto{\pgfqpoint{4.022290in}{4.072195in}}%
\pgfpathlineto{\pgfqpoint{4.009483in}{4.090281in}}%
\pgfpathlineto{\pgfqpoint{3.996673in}{4.108600in}}%
\pgfpathlineto{\pgfqpoint{3.989288in}{4.078427in}}%
\pgfpathlineto{\pgfqpoint{3.981903in}{4.048726in}}%
\pgfpathlineto{\pgfqpoint{3.974518in}{4.019489in}}%
\pgfpathclose%
\pgfusepath{fill}%
\end{pgfscope}%
\begin{pgfscope}%
\pgfpathrectangle{\pgfqpoint{1.254980in}{0.150000in}}{\pgfqpoint{5.490039in}{5.490039in}}%
\pgfusepath{clip}%
\pgfsetbuttcap%
\pgfsetroundjoin%
\definecolor{currentfill}{rgb}{0.157851,0.683765,0.501686}%
\pgfsetfillcolor{currentfill}%
\pgfsetfillopacity{0.700000}%
\pgfsetlinewidth{0.000000pt}%
\definecolor{currentstroke}{rgb}{0.000000,0.000000,0.000000}%
\pgfsetstrokecolor{currentstroke}%
\pgfsetdash{}{0pt}%
\pgfpathmoveto{\pgfqpoint{3.893739in}{3.977907in}}%
\pgfpathlineto{\pgfqpoint{3.906552in}{3.959830in}}%
\pgfpathlineto{\pgfqpoint{3.919363in}{3.941990in}}%
\pgfpathlineto{\pgfqpoint{3.932171in}{3.924385in}}%
\pgfpathlineto{\pgfqpoint{3.944978in}{3.907015in}}%
\pgfpathlineto{\pgfqpoint{3.952363in}{3.934479in}}%
\pgfpathlineto{\pgfqpoint{3.959748in}{3.962373in}}%
\pgfpathlineto{\pgfqpoint{3.967133in}{3.990708in}}%
\pgfpathlineto{\pgfqpoint{3.974518in}{4.019489in}}%
\pgfpathlineto{\pgfqpoint{3.961709in}{4.037522in}}%
\pgfpathlineto{\pgfqpoint{3.948898in}{4.055790in}}%
\pgfpathlineto{\pgfqpoint{3.936085in}{4.074295in}}%
\pgfpathlineto{\pgfqpoint{3.923269in}{4.093039in}}%
\pgfpathlineto{\pgfqpoint{3.915887in}{4.063580in}}%
\pgfpathlineto{\pgfqpoint{3.908505in}{4.034578in}}%
\pgfpathlineto{\pgfqpoint{3.901123in}{4.006023in}}%
\pgfpathlineto{\pgfqpoint{3.893739in}{3.977907in}}%
\pgfpathclose%
\pgfusepath{fill}%
\end{pgfscope}%
\begin{pgfscope}%
\pgfpathrectangle{\pgfqpoint{1.254980in}{0.150000in}}{\pgfqpoint{5.490039in}{5.490039in}}%
\pgfusepath{clip}%
\pgfsetbuttcap%
\pgfsetroundjoin%
\definecolor{currentfill}{rgb}{0.120638,0.625828,0.533488}%
\pgfsetfillcolor{currentfill}%
\pgfsetfillopacity{0.700000}%
\pgfsetlinewidth{0.000000pt}%
\definecolor{currentstroke}{rgb}{0.000000,0.000000,0.000000}%
\pgfsetstrokecolor{currentstroke}%
\pgfsetdash{}{0pt}%
\pgfpathmoveto{\pgfqpoint{3.783368in}{3.837648in}}%
\pgfpathlineto{\pgfqpoint{3.796188in}{3.819841in}}%
\pgfpathlineto{\pgfqpoint{3.809005in}{3.802278in}}%
\pgfpathlineto{\pgfqpoint{3.821820in}{3.784956in}}%
\pgfpathlineto{\pgfqpoint{3.834632in}{3.767874in}}%
\pgfpathlineto{\pgfqpoint{3.842026in}{3.792750in}}%
\pgfpathlineto{\pgfqpoint{3.849418in}{3.818006in}}%
\pgfpathlineto{\pgfqpoint{3.856808in}{3.843647in}}%
\pgfpathlineto{\pgfqpoint{3.864197in}{3.869682in}}%
\pgfpathlineto{\pgfqpoint{3.851384in}{3.887366in}}%
\pgfpathlineto{\pgfqpoint{3.838568in}{3.905290in}}%
\pgfpathlineto{\pgfqpoint{3.825750in}{3.923456in}}%
\pgfpathlineto{\pgfqpoint{3.812928in}{3.941866in}}%
\pgfpathlineto{\pgfqpoint{3.805541in}{3.915216in}}%
\pgfpathlineto{\pgfqpoint{3.798152in}{3.888968in}}%
\pgfpathlineto{\pgfqpoint{3.790761in}{3.863114in}}%
\pgfpathlineto{\pgfqpoint{3.783368in}{3.837648in}}%
\pgfpathclose%
\pgfusepath{fill}%
\end{pgfscope}%
\begin{pgfscope}%
\pgfpathrectangle{\pgfqpoint{1.254980in}{0.150000in}}{\pgfqpoint{5.490039in}{5.490039in}}%
\pgfusepath{clip}%
\pgfsetbuttcap%
\pgfsetroundjoin%
\definecolor{currentfill}{rgb}{0.139147,0.533812,0.555298}%
\pgfsetfillcolor{currentfill}%
\pgfsetfillopacity{0.700000}%
\pgfsetlinewidth{0.000000pt}%
\definecolor{currentstroke}{rgb}{0.000000,0.000000,0.000000}%
\pgfsetstrokecolor{currentstroke}%
\pgfsetdash{}{0pt}%
\pgfpathmoveto{\pgfqpoint{4.332960in}{3.603316in}}%
\pgfpathlineto{\pgfqpoint{4.345777in}{3.591415in}}%
\pgfpathlineto{\pgfqpoint{4.358596in}{3.579707in}}%
\pgfpathlineto{\pgfqpoint{4.371417in}{3.568189in}}%
\pgfpathlineto{\pgfqpoint{4.384241in}{3.556862in}}%
\pgfpathlineto{\pgfqpoint{4.391624in}{3.580748in}}%
\pgfpathlineto{\pgfqpoint{4.399008in}{3.605034in}}%
\pgfpathlineto{\pgfqpoint{4.406394in}{3.629727in}}%
\pgfpathlineto{\pgfqpoint{4.393573in}{3.641554in}}%
\pgfpathlineto{\pgfqpoint{4.380753in}{3.653571in}}%
\pgfpathlineto{\pgfqpoint{4.367937in}{3.665779in}}%
\pgfpathlineto{\pgfqpoint{4.355123in}{3.678180in}}%
\pgfpathlineto{\pgfqpoint{4.347733in}{3.652813in}}%
\pgfpathlineto{\pgfqpoint{4.340346in}{3.627861in}}%
\pgfpathlineto{\pgfqpoint{4.332960in}{3.603316in}}%
\pgfpathclose%
\pgfusepath{fill}%
\end{pgfscope}%
\begin{pgfscope}%
\pgfpathrectangle{\pgfqpoint{1.254980in}{0.150000in}}{\pgfqpoint{5.490039in}{5.490039in}}%
\pgfusepath{clip}%
\pgfsetbuttcap%
\pgfsetroundjoin%
\definecolor{currentfill}{rgb}{0.163625,0.471133,0.558148}%
\pgfsetfillcolor{currentfill}%
\pgfsetfillopacity{0.700000}%
\pgfsetlinewidth{0.000000pt}%
\definecolor{currentstroke}{rgb}{0.000000,0.000000,0.000000}%
\pgfsetstrokecolor{currentstroke}%
\pgfsetdash{}{0pt}%
\pgfpathmoveto{\pgfqpoint{4.009874in}{3.438825in}}%
\pgfpathlineto{\pgfqpoint{4.022675in}{3.426106in}}%
\pgfpathlineto{\pgfqpoint{4.035478in}{3.413597in}}%
\pgfpathlineto{\pgfqpoint{4.048281in}{3.401296in}}%
\pgfpathlineto{\pgfqpoint{4.061085in}{3.389201in}}%
\pgfpathlineto{\pgfqpoint{4.068479in}{3.409880in}}%
\pgfpathlineto{\pgfqpoint{4.075872in}{3.430871in}}%
\pgfpathlineto{\pgfqpoint{4.083263in}{3.452183in}}%
\pgfpathlineto{\pgfqpoint{4.090654in}{3.473820in}}%
\pgfpathlineto{\pgfqpoint{4.077852in}{3.486471in}}%
\pgfpathlineto{\pgfqpoint{4.065051in}{3.499329in}}%
\pgfpathlineto{\pgfqpoint{4.052252in}{3.512396in}}%
\pgfpathlineto{\pgfqpoint{4.039453in}{3.525672in}}%
\pgfpathlineto{\pgfqpoint{4.032060in}{3.503466in}}%
\pgfpathlineto{\pgfqpoint{4.024666in}{3.481594in}}%
\pgfpathlineto{\pgfqpoint{4.017271in}{3.460049in}}%
\pgfpathlineto{\pgfqpoint{4.009874in}{3.438825in}}%
\pgfpathclose%
\pgfusepath{fill}%
\end{pgfscope}%
\begin{pgfscope}%
\pgfpathrectangle{\pgfqpoint{1.254980in}{0.150000in}}{\pgfqpoint{5.490039in}{5.490039in}}%
\pgfusepath{clip}%
\pgfsetbuttcap%
\pgfsetroundjoin%
\definecolor{currentfill}{rgb}{0.121831,0.589055,0.545623}%
\pgfsetfillcolor{currentfill}%
\pgfsetfillopacity{0.700000}%
\pgfsetlinewidth{0.000000pt}%
\definecolor{currentstroke}{rgb}{0.000000,0.000000,0.000000}%
\pgfsetstrokecolor{currentstroke}%
\pgfsetdash{}{0pt}%
\pgfpathmoveto{\pgfqpoint{3.753773in}{3.739510in}}%
\pgfpathlineto{\pgfqpoint{3.766592in}{3.722275in}}%
\pgfpathlineto{\pgfqpoint{3.779409in}{3.705283in}}%
\pgfpathlineto{\pgfqpoint{3.792223in}{3.688531in}}%
\pgfpathlineto{\pgfqpoint{3.805035in}{3.672018in}}%
\pgfpathlineto{\pgfqpoint{3.812438in}{3.695448in}}%
\pgfpathlineto{\pgfqpoint{3.819838in}{3.719230in}}%
\pgfpathlineto{\pgfqpoint{3.827236in}{3.743369in}}%
\pgfpathlineto{\pgfqpoint{3.834632in}{3.767874in}}%
\pgfpathlineto{\pgfqpoint{3.821820in}{3.784956in}}%
\pgfpathlineto{\pgfqpoint{3.809005in}{3.802278in}}%
\pgfpathlineto{\pgfqpoint{3.796188in}{3.819841in}}%
\pgfpathlineto{\pgfqpoint{3.783368in}{3.837648in}}%
\pgfpathlineto{\pgfqpoint{3.775973in}{3.812561in}}%
\pgfpathlineto{\pgfqpoint{3.768576in}{3.787847in}}%
\pgfpathlineto{\pgfqpoint{3.761176in}{3.763499in}}%
\pgfpathlineto{\pgfqpoint{3.753773in}{3.739510in}}%
\pgfpathclose%
\pgfusepath{fill}%
\end{pgfscope}%
\begin{pgfscope}%
\pgfpathrectangle{\pgfqpoint{1.254980in}{0.150000in}}{\pgfqpoint{5.490039in}{5.490039in}}%
\pgfusepath{clip}%
\pgfsetbuttcap%
\pgfsetroundjoin%
\definecolor{currentfill}{rgb}{0.150476,0.504369,0.557430}%
\pgfsetfillcolor{currentfill}%
\pgfsetfillopacity{0.700000}%
\pgfsetlinewidth{0.000000pt}%
\definecolor{currentstroke}{rgb}{0.000000,0.000000,0.000000}%
\pgfsetstrokecolor{currentstroke}%
\pgfsetdash{}{0pt}%
\pgfpathmoveto{\pgfqpoint{3.826633in}{3.520120in}}%
\pgfpathlineto{\pgfqpoint{3.839439in}{3.505307in}}%
\pgfpathlineto{\pgfqpoint{3.852243in}{3.490721in}}%
\pgfpathlineto{\pgfqpoint{3.865046in}{3.476361in}}%
\pgfpathlineto{\pgfqpoint{3.877849in}{3.462225in}}%
\pgfpathlineto{\pgfqpoint{3.885259in}{3.483269in}}%
\pgfpathlineto{\pgfqpoint{3.892666in}{3.504622in}}%
\pgfpathlineto{\pgfqpoint{3.900071in}{3.526292in}}%
\pgfpathlineto{\pgfqpoint{3.907474in}{3.548285in}}%
\pgfpathlineto{\pgfqpoint{3.894673in}{3.562954in}}%
\pgfpathlineto{\pgfqpoint{3.881871in}{3.577848in}}%
\pgfpathlineto{\pgfqpoint{3.869069in}{3.592967in}}%
\pgfpathlineto{\pgfqpoint{3.856265in}{3.608315in}}%
\pgfpathlineto{\pgfqpoint{3.848861in}{3.585777in}}%
\pgfpathlineto{\pgfqpoint{3.841454in}{3.563569in}}%
\pgfpathlineto{\pgfqpoint{3.834045in}{3.541686in}}%
\pgfpathlineto{\pgfqpoint{3.826633in}{3.520120in}}%
\pgfpathclose%
\pgfusepath{fill}%
\end{pgfscope}%
\begin{pgfscope}%
\pgfpathrectangle{\pgfqpoint{1.254980in}{0.150000in}}{\pgfqpoint{5.490039in}{5.490039in}}%
\pgfusepath{clip}%
\pgfsetbuttcap%
\pgfsetroundjoin%
\definecolor{currentfill}{rgb}{0.143303,0.669459,0.511215}%
\pgfsetfillcolor{currentfill}%
\pgfsetfillopacity{0.700000}%
\pgfsetlinewidth{0.000000pt}%
\definecolor{currentstroke}{rgb}{0.000000,0.000000,0.000000}%
\pgfsetstrokecolor{currentstroke}%
\pgfsetdash{}{0pt}%
\pgfpathmoveto{\pgfqpoint{3.812928in}{3.941866in}}%
\pgfpathlineto{\pgfqpoint{3.825750in}{3.923456in}}%
\pgfpathlineto{\pgfqpoint{3.838568in}{3.905290in}}%
\pgfpathlineto{\pgfqpoint{3.851384in}{3.887366in}}%
\pgfpathlineto{\pgfqpoint{3.864197in}{3.869682in}}%
\pgfpathlineto{\pgfqpoint{3.871584in}{3.896118in}}%
\pgfpathlineto{\pgfqpoint{3.878971in}{3.922963in}}%
\pgfpathlineto{\pgfqpoint{3.886355in}{3.950223in}}%
\pgfpathlineto{\pgfqpoint{3.893739in}{3.977907in}}%
\pgfpathlineto{\pgfqpoint{3.880924in}{3.996224in}}%
\pgfpathlineto{\pgfqpoint{3.868106in}{4.014783in}}%
\pgfpathlineto{\pgfqpoint{3.855285in}{4.033585in}}%
\pgfpathlineto{\pgfqpoint{3.842461in}{4.052632in}}%
\pgfpathlineto{\pgfqpoint{3.835080in}{4.024300in}}%
\pgfpathlineto{\pgfqpoint{3.827698in}{3.996401in}}%
\pgfpathlineto{\pgfqpoint{3.820314in}{3.968925in}}%
\pgfpathlineto{\pgfqpoint{3.812928in}{3.941866in}}%
\pgfpathclose%
\pgfusepath{fill}%
\end{pgfscope}%
\begin{pgfscope}%
\pgfpathrectangle{\pgfqpoint{1.254980in}{0.150000in}}{\pgfqpoint{5.490039in}{5.490039in}}%
\pgfusepath{clip}%
\pgfsetbuttcap%
\pgfsetroundjoin%
\definecolor{currentfill}{rgb}{0.157729,0.485932,0.558013}%
\pgfsetfillcolor{currentfill}%
\pgfsetfillopacity{0.700000}%
\pgfsetlinewidth{0.000000pt}%
\definecolor{currentstroke}{rgb}{0.000000,0.000000,0.000000}%
\pgfsetstrokecolor{currentstroke}%
\pgfsetdash{}{0pt}%
\pgfpathmoveto{\pgfqpoint{4.222650in}{3.465200in}}%
\pgfpathlineto{\pgfqpoint{4.235463in}{3.453775in}}%
\pgfpathlineto{\pgfqpoint{4.248279in}{3.442545in}}%
\pgfpathlineto{\pgfqpoint{4.261098in}{3.431510in}}%
\pgfpathlineto{\pgfqpoint{4.273919in}{3.420668in}}%
\pgfpathlineto{\pgfqpoint{4.281297in}{3.442236in}}%
\pgfpathlineto{\pgfqpoint{4.288675in}{3.464151in}}%
\pgfpathlineto{\pgfqpoint{4.296054in}{3.486418in}}%
\pgfpathlineto{\pgfqpoint{4.303433in}{3.509046in}}%
\pgfpathlineto{\pgfqpoint{4.290616in}{3.520499in}}%
\pgfpathlineto{\pgfqpoint{4.277801in}{3.532145in}}%
\pgfpathlineto{\pgfqpoint{4.264988in}{3.543986in}}%
\pgfpathlineto{\pgfqpoint{4.252178in}{3.556023in}}%
\pgfpathlineto{\pgfqpoint{4.244795in}{3.532772in}}%
\pgfpathlineto{\pgfqpoint{4.237413in}{3.509890in}}%
\pgfpathlineto{\pgfqpoint{4.230031in}{3.487368in}}%
\pgfpathlineto{\pgfqpoint{4.222650in}{3.465200in}}%
\pgfpathclose%
\pgfusepath{fill}%
\end{pgfscope}%
\begin{pgfscope}%
\pgfpathrectangle{\pgfqpoint{1.254980in}{0.150000in}}{\pgfqpoint{5.490039in}{5.490039in}}%
\pgfusepath{clip}%
\pgfsetbuttcap%
\pgfsetroundjoin%
\definecolor{currentfill}{rgb}{0.141935,0.526453,0.555991}%
\pgfsetfillcolor{currentfill}%
\pgfsetfillopacity{0.700000}%
\pgfsetlinewidth{0.000000pt}%
\definecolor{currentstroke}{rgb}{0.000000,0.000000,0.000000}%
\pgfsetstrokecolor{currentstroke}%
\pgfsetdash{}{0pt}%
\pgfpathmoveto{\pgfqpoint{3.775399in}{3.581678in}}%
\pgfpathlineto{\pgfqpoint{3.788210in}{3.565939in}}%
\pgfpathlineto{\pgfqpoint{3.801019in}{3.550434in}}%
\pgfpathlineto{\pgfqpoint{3.813827in}{3.535162in}}%
\pgfpathlineto{\pgfqpoint{3.826633in}{3.520120in}}%
\pgfpathlineto{\pgfqpoint{3.834045in}{3.541686in}}%
\pgfpathlineto{\pgfqpoint{3.841454in}{3.563569in}}%
\pgfpathlineto{\pgfqpoint{3.848861in}{3.585777in}}%
\pgfpathlineto{\pgfqpoint{3.856265in}{3.608315in}}%
\pgfpathlineto{\pgfqpoint{3.843460in}{3.623892in}}%
\pgfpathlineto{\pgfqpoint{3.830653in}{3.639700in}}%
\pgfpathlineto{\pgfqpoint{3.817845in}{3.655742in}}%
\pgfpathlineto{\pgfqpoint{3.805035in}{3.672018in}}%
\pgfpathlineto{\pgfqpoint{3.797630in}{3.648932in}}%
\pgfpathlineto{\pgfqpoint{3.790223in}{3.626185in}}%
\pgfpathlineto{\pgfqpoint{3.782812in}{3.603769in}}%
\pgfpathlineto{\pgfqpoint{3.775399in}{3.581678in}}%
\pgfpathclose%
\pgfusepath{fill}%
\end{pgfscope}%
\begin{pgfscope}%
\pgfpathrectangle{\pgfqpoint{1.254980in}{0.150000in}}{\pgfqpoint{5.490039in}{5.490039in}}%
\pgfusepath{clip}%
\pgfsetbuttcap%
\pgfsetroundjoin%
\definecolor{currentfill}{rgb}{0.165117,0.467423,0.558141}%
\pgfsetfillcolor{currentfill}%
\pgfsetfillopacity{0.700000}%
\pgfsetlinewidth{0.000000pt}%
\definecolor{currentstroke}{rgb}{0.000000,0.000000,0.000000}%
\pgfsetstrokecolor{currentstroke}%
\pgfsetdash{}{0pt}%
\pgfpathmoveto{\pgfqpoint{4.141873in}{3.425263in}}%
\pgfpathlineto{\pgfqpoint{4.154682in}{3.413629in}}%
\pgfpathlineto{\pgfqpoint{4.167493in}{3.402194in}}%
\pgfpathlineto{\pgfqpoint{4.180306in}{3.390958in}}%
\pgfpathlineto{\pgfqpoint{4.193121in}{3.379919in}}%
\pgfpathlineto{\pgfqpoint{4.200504in}{3.400744in}}%
\pgfpathlineto{\pgfqpoint{4.207886in}{3.421895in}}%
\pgfpathlineto{\pgfqpoint{4.215268in}{3.443378in}}%
\pgfpathlineto{\pgfqpoint{4.222650in}{3.465200in}}%
\pgfpathlineto{\pgfqpoint{4.209838in}{3.476821in}}%
\pgfpathlineto{\pgfqpoint{4.197029in}{3.488640in}}%
\pgfpathlineto{\pgfqpoint{4.184221in}{3.500658in}}%
\pgfpathlineto{\pgfqpoint{4.171415in}{3.512876in}}%
\pgfpathlineto{\pgfqpoint{4.164030in}{3.490460in}}%
\pgfpathlineto{\pgfqpoint{4.156645in}{3.468391in}}%
\pgfpathlineto{\pgfqpoint{4.149260in}{3.446660in}}%
\pgfpathlineto{\pgfqpoint{4.141873in}{3.425263in}}%
\pgfpathclose%
\pgfusepath{fill}%
\end{pgfscope}%
\begin{pgfscope}%
\pgfpathrectangle{\pgfqpoint{1.254980in}{0.150000in}}{\pgfqpoint{5.490039in}{5.490039in}}%
\pgfusepath{clip}%
\pgfsetbuttcap%
\pgfsetroundjoin%
\definecolor{currentfill}{rgb}{0.159194,0.482237,0.558073}%
\pgfsetfillcolor{currentfill}%
\pgfsetfillopacity{0.700000}%
\pgfsetlinewidth{0.000000pt}%
\definecolor{currentstroke}{rgb}{0.000000,0.000000,0.000000}%
\pgfsetstrokecolor{currentstroke}%
\pgfsetdash{}{0pt}%
\pgfpathmoveto{\pgfqpoint{3.877849in}{3.462225in}}%
\pgfpathlineto{\pgfqpoint{3.890651in}{3.448311in}}%
\pgfpathlineto{\pgfqpoint{3.903453in}{3.434617in}}%
\pgfpathlineto{\pgfqpoint{3.916255in}{3.421143in}}%
\pgfpathlineto{\pgfqpoint{3.929057in}{3.407887in}}%
\pgfpathlineto{\pgfqpoint{3.936464in}{3.428410in}}%
\pgfpathlineto{\pgfqpoint{3.943869in}{3.449237in}}%
\pgfpathlineto{\pgfqpoint{3.951272in}{3.470372in}}%
\pgfpathlineto{\pgfqpoint{3.958674in}{3.491822in}}%
\pgfpathlineto{\pgfqpoint{3.945874in}{3.505609in}}%
\pgfpathlineto{\pgfqpoint{3.933074in}{3.519614in}}%
\pgfpathlineto{\pgfqpoint{3.920274in}{3.533839in}}%
\pgfpathlineto{\pgfqpoint{3.907474in}{3.548285in}}%
\pgfpathlineto{\pgfqpoint{3.900071in}{3.526292in}}%
\pgfpathlineto{\pgfqpoint{3.892666in}{3.504622in}}%
\pgfpathlineto{\pgfqpoint{3.885259in}{3.483269in}}%
\pgfpathlineto{\pgfqpoint{3.877849in}{3.462225in}}%
\pgfpathclose%
\pgfusepath{fill}%
\end{pgfscope}%
\begin{pgfscope}%
\pgfpathrectangle{\pgfqpoint{1.254980in}{0.150000in}}{\pgfqpoint{5.490039in}{5.490039in}}%
\pgfusepath{clip}%
\pgfsetbuttcap%
\pgfsetroundjoin%
\definecolor{currentfill}{rgb}{0.150476,0.504369,0.557430}%
\pgfsetfillcolor{currentfill}%
\pgfsetfillopacity{0.700000}%
\pgfsetlinewidth{0.000000pt}%
\definecolor{currentstroke}{rgb}{0.000000,0.000000,0.000000}%
\pgfsetstrokecolor{currentstroke}%
\pgfsetdash{}{0pt}%
\pgfpathmoveto{\pgfqpoint{4.303433in}{3.509046in}}%
\pgfpathlineto{\pgfqpoint{4.316253in}{3.497786in}}%
\pgfpathlineto{\pgfqpoint{4.329075in}{3.486717in}}%
\pgfpathlineto{\pgfqpoint{4.341900in}{3.475839in}}%
\pgfpathlineto{\pgfqpoint{4.354729in}{3.465150in}}%
\pgfpathlineto{\pgfqpoint{4.362105in}{3.487518in}}%
\pgfpathlineto{\pgfqpoint{4.369482in}{3.510254in}}%
\pgfpathlineto{\pgfqpoint{4.376861in}{3.533366in}}%
\pgfpathlineto{\pgfqpoint{4.384241in}{3.556862in}}%
\pgfpathlineto{\pgfqpoint{4.371417in}{3.568189in}}%
\pgfpathlineto{\pgfqpoint{4.358596in}{3.579707in}}%
\pgfpathlineto{\pgfqpoint{4.345777in}{3.591415in}}%
\pgfpathlineto{\pgfqpoint{4.332960in}{3.603316in}}%
\pgfpathlineto{\pgfqpoint{4.325577in}{3.579169in}}%
\pgfpathlineto{\pgfqpoint{4.318194in}{3.555414in}}%
\pgfpathlineto{\pgfqpoint{4.310813in}{3.532042in}}%
\pgfpathlineto{\pgfqpoint{4.303433in}{3.509046in}}%
\pgfpathclose%
\pgfusepath{fill}%
\end{pgfscope}%
\begin{pgfscope}%
\pgfpathrectangle{\pgfqpoint{1.254980in}{0.150000in}}{\pgfqpoint{5.490039in}{5.490039in}}%
\pgfusepath{clip}%
\pgfsetbuttcap%
\pgfsetroundjoin%
\definecolor{currentfill}{rgb}{0.132444,0.552216,0.553018}%
\pgfsetfillcolor{currentfill}%
\pgfsetfillopacity{0.700000}%
\pgfsetlinewidth{0.000000pt}%
\definecolor{currentstroke}{rgb}{0.000000,0.000000,0.000000}%
\pgfsetstrokecolor{currentstroke}%
\pgfsetdash{}{0pt}%
\pgfpathmoveto{\pgfqpoint{3.724135in}{3.647014in}}%
\pgfpathlineto{\pgfqpoint{3.736955in}{3.630319in}}%
\pgfpathlineto{\pgfqpoint{3.749772in}{3.613866in}}%
\pgfpathlineto{\pgfqpoint{3.762586in}{3.597653in}}%
\pgfpathlineto{\pgfqpoint{3.775399in}{3.581678in}}%
\pgfpathlineto{\pgfqpoint{3.782812in}{3.603769in}}%
\pgfpathlineto{\pgfqpoint{3.790223in}{3.626185in}}%
\pgfpathlineto{\pgfqpoint{3.797630in}{3.648932in}}%
\pgfpathlineto{\pgfqpoint{3.805035in}{3.672018in}}%
\pgfpathlineto{\pgfqpoint{3.792223in}{3.688531in}}%
\pgfpathlineto{\pgfqpoint{3.779409in}{3.705283in}}%
\pgfpathlineto{\pgfqpoint{3.766592in}{3.722275in}}%
\pgfpathlineto{\pgfqpoint{3.753773in}{3.739510in}}%
\pgfpathlineto{\pgfqpoint{3.746368in}{3.715874in}}%
\pgfpathlineto{\pgfqpoint{3.738960in}{3.692583in}}%
\pgfpathlineto{\pgfqpoint{3.731549in}{3.669632in}}%
\pgfpathlineto{\pgfqpoint{3.724135in}{3.647014in}}%
\pgfpathclose%
\pgfusepath{fill}%
\end{pgfscope}%
\begin{pgfscope}%
\pgfpathrectangle{\pgfqpoint{1.254980in}{0.150000in}}{\pgfqpoint{5.490039in}{5.490039in}}%
\pgfusepath{clip}%
\pgfsetbuttcap%
\pgfsetroundjoin%
\definecolor{currentfill}{rgb}{0.171176,0.452530,0.557965}%
\pgfsetfillcolor{currentfill}%
\pgfsetfillopacity{0.700000}%
\pgfsetlinewidth{0.000000pt}%
\definecolor{currentstroke}{rgb}{0.000000,0.000000,0.000000}%
\pgfsetstrokecolor{currentstroke}%
\pgfsetdash{}{0pt}%
\pgfpathmoveto{\pgfqpoint{4.061085in}{3.389201in}}%
\pgfpathlineto{\pgfqpoint{4.073891in}{3.377312in}}%
\pgfpathlineto{\pgfqpoint{4.086699in}{3.365627in}}%
\pgfpathlineto{\pgfqpoint{4.099508in}{3.354145in}}%
\pgfpathlineto{\pgfqpoint{4.112318in}{3.342864in}}%
\pgfpathlineto{\pgfqpoint{4.119709in}{3.362999in}}%
\pgfpathlineto{\pgfqpoint{4.127098in}{3.383439in}}%
\pgfpathlineto{\pgfqpoint{4.134486in}{3.404191in}}%
\pgfpathlineto{\pgfqpoint{4.141873in}{3.425263in}}%
\pgfpathlineto{\pgfqpoint{4.129066in}{3.437098in}}%
\pgfpathlineto{\pgfqpoint{4.116261in}{3.449135in}}%
\pgfpathlineto{\pgfqpoint{4.103456in}{3.461376in}}%
\pgfpathlineto{\pgfqpoint{4.090654in}{3.473820in}}%
\pgfpathlineto{\pgfqpoint{4.083263in}{3.452183in}}%
\pgfpathlineto{\pgfqpoint{4.075872in}{3.430871in}}%
\pgfpathlineto{\pgfqpoint{4.068479in}{3.409880in}}%
\pgfpathlineto{\pgfqpoint{4.061085in}{3.389201in}}%
\pgfpathclose%
\pgfusepath{fill}%
\end{pgfscope}%
\begin{pgfscope}%
\pgfpathrectangle{\pgfqpoint{1.254980in}{0.150000in}}{\pgfqpoint{5.490039in}{5.490039in}}%
\pgfusepath{clip}%
\pgfsetbuttcap%
\pgfsetroundjoin%
\definecolor{currentfill}{rgb}{0.168126,0.459988,0.558082}%
\pgfsetfillcolor{currentfill}%
\pgfsetfillopacity{0.700000}%
\pgfsetlinewidth{0.000000pt}%
\definecolor{currentstroke}{rgb}{0.000000,0.000000,0.000000}%
\pgfsetstrokecolor{currentstroke}%
\pgfsetdash{}{0pt}%
\pgfpathmoveto{\pgfqpoint{3.929057in}{3.407887in}}%
\pgfpathlineto{\pgfqpoint{3.941859in}{3.394846in}}%
\pgfpathlineto{\pgfqpoint{3.954661in}{3.382020in}}%
\pgfpathlineto{\pgfqpoint{3.967464in}{3.369407in}}%
\pgfpathlineto{\pgfqpoint{3.980268in}{3.357006in}}%
\pgfpathlineto{\pgfqpoint{3.987672in}{3.377012in}}%
\pgfpathlineto{\pgfqpoint{3.995075in}{3.397313in}}%
\pgfpathlineto{\pgfqpoint{4.002475in}{3.417915in}}%
\pgfpathlineto{\pgfqpoint{4.009874in}{3.438825in}}%
\pgfpathlineto{\pgfqpoint{3.997073in}{3.451755in}}%
\pgfpathlineto{\pgfqpoint{3.984273in}{3.464896in}}%
\pgfpathlineto{\pgfqpoint{3.971473in}{3.478252in}}%
\pgfpathlineto{\pgfqpoint{3.958674in}{3.491822in}}%
\pgfpathlineto{\pgfqpoint{3.951272in}{3.470372in}}%
\pgfpathlineto{\pgfqpoint{3.943869in}{3.449237in}}%
\pgfpathlineto{\pgfqpoint{3.936464in}{3.428410in}}%
\pgfpathlineto{\pgfqpoint{3.929057in}{3.407887in}}%
\pgfpathclose%
\pgfusepath{fill}%
\end{pgfscope}%
\begin{pgfscope}%
\pgfpathrectangle{\pgfqpoint{1.254980in}{0.150000in}}{\pgfqpoint{5.490039in}{5.490039in}}%
\pgfusepath{clip}%
\pgfsetbuttcap%
\pgfsetroundjoin%
\definecolor{currentfill}{rgb}{0.214000,0.722114,0.469588}%
\pgfsetfillcolor{currentfill}%
\pgfsetfillopacity{0.700000}%
\pgfsetlinewidth{0.000000pt}%
\definecolor{currentstroke}{rgb}{0.000000,0.000000,0.000000}%
\pgfsetstrokecolor{currentstroke}%
\pgfsetdash{}{0pt}%
\pgfpathmoveto{\pgfqpoint{3.923269in}{4.093039in}}%
\pgfpathlineto{\pgfqpoint{3.936085in}{4.074295in}}%
\pgfpathlineto{\pgfqpoint{3.948898in}{4.055790in}}%
\pgfpathlineto{\pgfqpoint{3.961709in}{4.037522in}}%
\pgfpathlineto{\pgfqpoint{3.974518in}{4.019489in}}%
\pgfpathlineto{\pgfqpoint{3.981903in}{4.048726in}}%
\pgfpathlineto{\pgfqpoint{3.989288in}{4.078427in}}%
\pgfpathlineto{\pgfqpoint{3.996673in}{4.108600in}}%
\pgfpathlineto{\pgfqpoint{3.983862in}{4.127152in}}%
\pgfpathlineto{\pgfqpoint{3.971048in}{4.145941in}}%
\pgfpathlineto{\pgfqpoint{3.958232in}{4.164967in}}%
\pgfpathlineto{\pgfqpoint{3.945413in}{4.184233in}}%
\pgfpathlineto{\pgfqpoint{3.938032in}{4.153356in}}%
\pgfpathlineto{\pgfqpoint{3.930650in}{4.122961in}}%
\pgfpathlineto{\pgfqpoint{3.923269in}{4.093039in}}%
\pgfpathclose%
\pgfusepath{fill}%
\end{pgfscope}%
\begin{pgfscope}%
\pgfpathrectangle{\pgfqpoint{1.254980in}{0.150000in}}{\pgfqpoint{5.490039in}{5.490039in}}%
\pgfusepath{clip}%
\pgfsetbuttcap%
\pgfsetroundjoin%
\definecolor{currentfill}{rgb}{0.132268,0.655014,0.519661}%
\pgfsetfillcolor{currentfill}%
\pgfsetfillopacity{0.700000}%
\pgfsetlinewidth{0.000000pt}%
\definecolor{currentstroke}{rgb}{0.000000,0.000000,0.000000}%
\pgfsetstrokecolor{currentstroke}%
\pgfsetdash{}{0pt}%
\pgfpathmoveto{\pgfqpoint{3.732058in}{3.911348in}}%
\pgfpathlineto{\pgfqpoint{3.744891in}{3.892548in}}%
\pgfpathlineto{\pgfqpoint{3.757720in}{3.873999in}}%
\pgfpathlineto{\pgfqpoint{3.770546in}{3.855700in}}%
\pgfpathlineto{\pgfqpoint{3.783368in}{3.837648in}}%
\pgfpathlineto{\pgfqpoint{3.790761in}{3.863114in}}%
\pgfpathlineto{\pgfqpoint{3.798152in}{3.888968in}}%
\pgfpathlineto{\pgfqpoint{3.805541in}{3.915216in}}%
\pgfpathlineto{\pgfqpoint{3.812928in}{3.941866in}}%
\pgfpathlineto{\pgfqpoint{3.800104in}{3.960523in}}%
\pgfpathlineto{\pgfqpoint{3.787276in}{3.979428in}}%
\pgfpathlineto{\pgfqpoint{3.774445in}{3.998584in}}%
\pgfpathlineto{\pgfqpoint{3.761611in}{4.017993in}}%
\pgfpathlineto{\pgfqpoint{3.754226in}{3.990724in}}%
\pgfpathlineto{\pgfqpoint{3.746839in}{3.963865in}}%
\pgfpathlineto{\pgfqpoint{3.739450in}{3.937409in}}%
\pgfpathlineto{\pgfqpoint{3.732058in}{3.911348in}}%
\pgfpathclose%
\pgfusepath{fill}%
\end{pgfscope}%
\begin{pgfscope}%
\pgfpathrectangle{\pgfqpoint{1.254980in}{0.150000in}}{\pgfqpoint{5.490039in}{5.490039in}}%
\pgfusepath{clip}%
\pgfsetbuttcap%
\pgfsetroundjoin%
\definecolor{currentfill}{rgb}{0.119699,0.618490,0.536347}%
\pgfsetfillcolor{currentfill}%
\pgfsetfillopacity{0.700000}%
\pgfsetlinewidth{0.000000pt}%
\definecolor{currentstroke}{rgb}{0.000000,0.000000,0.000000}%
\pgfsetstrokecolor{currentstroke}%
\pgfsetdash{}{0pt}%
\pgfpathmoveto{\pgfqpoint{3.702467in}{3.810915in}}%
\pgfpathlineto{\pgfqpoint{3.715298in}{3.792690in}}%
\pgfpathlineto{\pgfqpoint{3.728126in}{3.774715in}}%
\pgfpathlineto{\pgfqpoint{3.740951in}{3.756989in}}%
\pgfpathlineto{\pgfqpoint{3.753773in}{3.739510in}}%
\pgfpathlineto{\pgfqpoint{3.761176in}{3.763499in}}%
\pgfpathlineto{\pgfqpoint{3.768576in}{3.787847in}}%
\pgfpathlineto{\pgfqpoint{3.775973in}{3.812561in}}%
\pgfpathlineto{\pgfqpoint{3.783368in}{3.837648in}}%
\pgfpathlineto{\pgfqpoint{3.770546in}{3.855700in}}%
\pgfpathlineto{\pgfqpoint{3.757720in}{3.873999in}}%
\pgfpathlineto{\pgfqpoint{3.744891in}{3.892548in}}%
\pgfpathlineto{\pgfqpoint{3.732058in}{3.911348in}}%
\pgfpathlineto{\pgfqpoint{3.724665in}{3.885675in}}%
\pgfpathlineto{\pgfqpoint{3.717268in}{3.860383in}}%
\pgfpathlineto{\pgfqpoint{3.709869in}{3.835466in}}%
\pgfpathlineto{\pgfqpoint{3.702467in}{3.810915in}}%
\pgfpathclose%
\pgfusepath{fill}%
\end{pgfscope}%
\begin{pgfscope}%
\pgfpathrectangle{\pgfqpoint{1.254980in}{0.150000in}}{\pgfqpoint{5.490039in}{5.490039in}}%
\pgfusepath{clip}%
\pgfsetbuttcap%
\pgfsetroundjoin%
\definecolor{currentfill}{rgb}{0.144759,0.519093,0.556572}%
\pgfsetfillcolor{currentfill}%
\pgfsetfillopacity{0.700000}%
\pgfsetlinewidth{0.000000pt}%
\definecolor{currentstroke}{rgb}{0.000000,0.000000,0.000000}%
\pgfsetstrokecolor{currentstroke}%
\pgfsetdash{}{0pt}%
\pgfpathmoveto{\pgfqpoint{4.384241in}{3.556862in}}%
\pgfpathlineto{\pgfqpoint{4.397069in}{3.545723in}}%
\pgfpathlineto{\pgfqpoint{4.409899in}{3.534772in}}%
\pgfpathlineto{\pgfqpoint{4.422732in}{3.524007in}}%
\pgfpathlineto{\pgfqpoint{4.435569in}{3.513428in}}%
\pgfpathlineto{\pgfqpoint{4.442947in}{3.536659in}}%
\pgfpathlineto{\pgfqpoint{4.450327in}{3.560281in}}%
\pgfpathlineto{\pgfqpoint{4.457709in}{3.584302in}}%
\pgfpathlineto{\pgfqpoint{4.444876in}{3.595378in}}%
\pgfpathlineto{\pgfqpoint{4.432046in}{3.606640in}}%
\pgfpathlineto{\pgfqpoint{4.419219in}{3.618090in}}%
\pgfpathlineto{\pgfqpoint{4.406394in}{3.629727in}}%
\pgfpathlineto{\pgfqpoint{4.399008in}{3.605034in}}%
\pgfpathlineto{\pgfqpoint{4.391624in}{3.580748in}}%
\pgfpathlineto{\pgfqpoint{4.384241in}{3.556862in}}%
\pgfpathclose%
\pgfusepath{fill}%
\end{pgfscope}%
\begin{pgfscope}%
\pgfpathrectangle{\pgfqpoint{1.254980in}{0.150000in}}{\pgfqpoint{5.490039in}{5.490039in}}%
\pgfusepath{clip}%
\pgfsetbuttcap%
\pgfsetroundjoin%
\definecolor{currentfill}{rgb}{0.196571,0.711827,0.479221}%
\pgfsetfillcolor{currentfill}%
\pgfsetfillopacity{0.700000}%
\pgfsetlinewidth{0.000000pt}%
\definecolor{currentstroke}{rgb}{0.000000,0.000000,0.000000}%
\pgfsetstrokecolor{currentstroke}%
\pgfsetdash{}{0pt}%
\pgfpathmoveto{\pgfqpoint{3.842461in}{4.052632in}}%
\pgfpathlineto{\pgfqpoint{3.855285in}{4.033585in}}%
\pgfpathlineto{\pgfqpoint{3.868106in}{4.014783in}}%
\pgfpathlineto{\pgfqpoint{3.880924in}{3.996224in}}%
\pgfpathlineto{\pgfqpoint{3.893739in}{3.977907in}}%
\pgfpathlineto{\pgfqpoint{3.901123in}{4.006023in}}%
\pgfpathlineto{\pgfqpoint{3.908505in}{4.034578in}}%
\pgfpathlineto{\pgfqpoint{3.915887in}{4.063580in}}%
\pgfpathlineto{\pgfqpoint{3.923269in}{4.093039in}}%
\pgfpathlineto{\pgfqpoint{3.910450in}{4.112023in}}%
\pgfpathlineto{\pgfqpoint{3.897629in}{4.131250in}}%
\pgfpathlineto{\pgfqpoint{3.884804in}{4.150722in}}%
\pgfpathlineto{\pgfqpoint{3.871977in}{4.170440in}}%
\pgfpathlineto{\pgfqpoint{3.864599in}{4.140300in}}%
\pgfpathlineto{\pgfqpoint{3.857221in}{4.110624in}}%
\pgfpathlineto{\pgfqpoint{3.849841in}{4.081404in}}%
\pgfpathlineto{\pgfqpoint{3.842461in}{4.052632in}}%
\pgfpathclose%
\pgfusepath{fill}%
\end{pgfscope}%
\begin{pgfscope}%
\pgfpathrectangle{\pgfqpoint{1.254980in}{0.150000in}}{\pgfqpoint{5.490039in}{5.490039in}}%
\pgfusepath{clip}%
\pgfsetbuttcap%
\pgfsetroundjoin%
\definecolor{currentfill}{rgb}{0.163625,0.471133,0.558148}%
\pgfsetfillcolor{currentfill}%
\pgfsetfillopacity{0.700000}%
\pgfsetlinewidth{0.000000pt}%
\definecolor{currentstroke}{rgb}{0.000000,0.000000,0.000000}%
\pgfsetstrokecolor{currentstroke}%
\pgfsetdash{}{0pt}%
\pgfpathmoveto{\pgfqpoint{4.273919in}{3.420668in}}%
\pgfpathlineto{\pgfqpoint{4.286743in}{3.410017in}}%
\pgfpathlineto{\pgfqpoint{4.299569in}{3.399558in}}%
\pgfpathlineto{\pgfqpoint{4.312399in}{3.389289in}}%
\pgfpathlineto{\pgfqpoint{4.325232in}{3.379208in}}%
\pgfpathlineto{\pgfqpoint{4.332605in}{3.400179in}}%
\pgfpathlineto{\pgfqpoint{4.339979in}{3.421488in}}%
\pgfpathlineto{\pgfqpoint{4.347354in}{3.443142in}}%
\pgfpathlineto{\pgfqpoint{4.354729in}{3.465150in}}%
\pgfpathlineto{\pgfqpoint{4.341900in}{3.475839in}}%
\pgfpathlineto{\pgfqpoint{4.329075in}{3.486717in}}%
\pgfpathlineto{\pgfqpoint{4.316253in}{3.497786in}}%
\pgfpathlineto{\pgfqpoint{4.303433in}{3.509046in}}%
\pgfpathlineto{\pgfqpoint{4.296054in}{3.486418in}}%
\pgfpathlineto{\pgfqpoint{4.288675in}{3.464151in}}%
\pgfpathlineto{\pgfqpoint{4.281297in}{3.442236in}}%
\pgfpathlineto{\pgfqpoint{4.273919in}{3.420668in}}%
\pgfpathclose%
\pgfusepath{fill}%
\end{pgfscope}%
\begin{pgfscope}%
\pgfpathrectangle{\pgfqpoint{1.254980in}{0.150000in}}{\pgfqpoint{5.490039in}{5.490039in}}%
\pgfusepath{clip}%
\pgfsetbuttcap%
\pgfsetroundjoin%
\definecolor{currentfill}{rgb}{0.171176,0.452530,0.557965}%
\pgfsetfillcolor{currentfill}%
\pgfsetfillopacity{0.700000}%
\pgfsetlinewidth{0.000000pt}%
\definecolor{currentstroke}{rgb}{0.000000,0.000000,0.000000}%
\pgfsetstrokecolor{currentstroke}%
\pgfsetdash{}{0pt}%
\pgfpathmoveto{\pgfqpoint{4.193121in}{3.379919in}}%
\pgfpathlineto{\pgfqpoint{4.205939in}{3.369076in}}%
\pgfpathlineto{\pgfqpoint{4.218759in}{3.358427in}}%
\pgfpathlineto{\pgfqpoint{4.231581in}{3.347973in}}%
\pgfpathlineto{\pgfqpoint{4.244407in}{3.337711in}}%
\pgfpathlineto{\pgfqpoint{4.251785in}{3.357967in}}%
\pgfpathlineto{\pgfqpoint{4.259163in}{3.378540in}}%
\pgfpathlineto{\pgfqpoint{4.266541in}{3.399438in}}%
\pgfpathlineto{\pgfqpoint{4.273919in}{3.420668in}}%
\pgfpathlineto{\pgfqpoint{4.261098in}{3.431510in}}%
\pgfpathlineto{\pgfqpoint{4.248279in}{3.442545in}}%
\pgfpathlineto{\pgfqpoint{4.235463in}{3.453775in}}%
\pgfpathlineto{\pgfqpoint{4.222650in}{3.465200in}}%
\pgfpathlineto{\pgfqpoint{4.215268in}{3.443378in}}%
\pgfpathlineto{\pgfqpoint{4.207886in}{3.421895in}}%
\pgfpathlineto{\pgfqpoint{4.200504in}{3.400744in}}%
\pgfpathlineto{\pgfqpoint{4.193121in}{3.379919in}}%
\pgfpathclose%
\pgfusepath{fill}%
\end{pgfscope}%
\begin{pgfscope}%
\pgfpathrectangle{\pgfqpoint{1.254980in}{0.150000in}}{\pgfqpoint{5.490039in}{5.490039in}}%
\pgfusepath{clip}%
\pgfsetbuttcap%
\pgfsetroundjoin%
\definecolor{currentfill}{rgb}{0.123463,0.581687,0.547445}%
\pgfsetfillcolor{currentfill}%
\pgfsetfillopacity{0.700000}%
\pgfsetlinewidth{0.000000pt}%
\definecolor{currentstroke}{rgb}{0.000000,0.000000,0.000000}%
\pgfsetstrokecolor{currentstroke}%
\pgfsetdash{}{0pt}%
\pgfpathmoveto{\pgfqpoint{3.672829in}{3.716250in}}%
\pgfpathlineto{\pgfqpoint{3.685660in}{3.698568in}}%
\pgfpathlineto{\pgfqpoint{3.698488in}{3.681136in}}%
\pgfpathlineto{\pgfqpoint{3.711313in}{3.663952in}}%
\pgfpathlineto{\pgfqpoint{3.724135in}{3.647014in}}%
\pgfpathlineto{\pgfqpoint{3.731549in}{3.669632in}}%
\pgfpathlineto{\pgfqpoint{3.738960in}{3.692583in}}%
\pgfpathlineto{\pgfqpoint{3.746368in}{3.715874in}}%
\pgfpathlineto{\pgfqpoint{3.753773in}{3.739510in}}%
\pgfpathlineto{\pgfqpoint{3.740951in}{3.756989in}}%
\pgfpathlineto{\pgfqpoint{3.728126in}{3.774715in}}%
\pgfpathlineto{\pgfqpoint{3.715298in}{3.792690in}}%
\pgfpathlineto{\pgfqpoint{3.702467in}{3.810915in}}%
\pgfpathlineto{\pgfqpoint{3.695062in}{3.786724in}}%
\pgfpathlineto{\pgfqpoint{3.687654in}{3.762888in}}%
\pgfpathlineto{\pgfqpoint{3.680243in}{3.739399in}}%
\pgfpathlineto{\pgfqpoint{3.672829in}{3.716250in}}%
\pgfpathclose%
\pgfusepath{fill}%
\end{pgfscope}%
\begin{pgfscope}%
\pgfpathrectangle{\pgfqpoint{1.254980in}{0.150000in}}{\pgfqpoint{5.490039in}{5.490039in}}%
\pgfusepath{clip}%
\pgfsetbuttcap%
\pgfsetroundjoin%
\definecolor{currentfill}{rgb}{0.175707,0.697900,0.491033}%
\pgfsetfillcolor{currentfill}%
\pgfsetfillopacity{0.700000}%
\pgfsetlinewidth{0.000000pt}%
\definecolor{currentstroke}{rgb}{0.000000,0.000000,0.000000}%
\pgfsetstrokecolor{currentstroke}%
\pgfsetdash{}{0pt}%
\pgfpathmoveto{\pgfqpoint{3.761611in}{4.017993in}}%
\pgfpathlineto{\pgfqpoint{3.774445in}{3.998584in}}%
\pgfpathlineto{\pgfqpoint{3.787276in}{3.979428in}}%
\pgfpathlineto{\pgfqpoint{3.800104in}{3.960523in}}%
\pgfpathlineto{\pgfqpoint{3.812928in}{3.941866in}}%
\pgfpathlineto{\pgfqpoint{3.820314in}{3.968925in}}%
\pgfpathlineto{\pgfqpoint{3.827698in}{3.996401in}}%
\pgfpathlineto{\pgfqpoint{3.835080in}{4.024300in}}%
\pgfpathlineto{\pgfqpoint{3.842461in}{4.052632in}}%
\pgfpathlineto{\pgfqpoint{3.829634in}{4.071927in}}%
\pgfpathlineto{\pgfqpoint{3.816804in}{4.091471in}}%
\pgfpathlineto{\pgfqpoint{3.803969in}{4.111268in}}%
\pgfpathlineto{\pgfqpoint{3.791131in}{4.131318in}}%
\pgfpathlineto{\pgfqpoint{3.783754in}{4.102333in}}%
\pgfpathlineto{\pgfqpoint{3.776374in}{4.073790in}}%
\pgfpathlineto{\pgfqpoint{3.768993in}{4.045679in}}%
\pgfpathlineto{\pgfqpoint{3.761611in}{4.017993in}}%
\pgfpathclose%
\pgfusepath{fill}%
\end{pgfscope}%
\begin{pgfscope}%
\pgfpathrectangle{\pgfqpoint{1.254980in}{0.150000in}}{\pgfqpoint{5.490039in}{5.490039in}}%
\pgfusepath{clip}%
\pgfsetbuttcap%
\pgfsetroundjoin%
\definecolor{currentfill}{rgb}{0.175841,0.441290,0.557685}%
\pgfsetfillcolor{currentfill}%
\pgfsetfillopacity{0.700000}%
\pgfsetlinewidth{0.000000pt}%
\definecolor{currentstroke}{rgb}{0.000000,0.000000,0.000000}%
\pgfsetstrokecolor{currentstroke}%
\pgfsetdash{}{0pt}%
\pgfpathmoveto{\pgfqpoint{3.980268in}{3.357006in}}%
\pgfpathlineto{\pgfqpoint{3.993073in}{3.344816in}}%
\pgfpathlineto{\pgfqpoint{4.005878in}{3.332834in}}%
\pgfpathlineto{\pgfqpoint{4.018685in}{3.321060in}}%
\pgfpathlineto{\pgfqpoint{4.031493in}{3.309492in}}%
\pgfpathlineto{\pgfqpoint{4.038894in}{3.328981in}}%
\pgfpathlineto{\pgfqpoint{4.046293in}{3.348759in}}%
\pgfpathlineto{\pgfqpoint{4.053690in}{3.368830in}}%
\pgfpathlineto{\pgfqpoint{4.061085in}{3.389201in}}%
\pgfpathlineto{\pgfqpoint{4.048281in}{3.401296in}}%
\pgfpathlineto{\pgfqpoint{4.035478in}{3.413597in}}%
\pgfpathlineto{\pgfqpoint{4.022675in}{3.426106in}}%
\pgfpathlineto{\pgfqpoint{4.009874in}{3.438825in}}%
\pgfpathlineto{\pgfqpoint{4.002475in}{3.417915in}}%
\pgfpathlineto{\pgfqpoint{3.995075in}{3.397313in}}%
\pgfpathlineto{\pgfqpoint{3.987672in}{3.377012in}}%
\pgfpathlineto{\pgfqpoint{3.980268in}{3.357006in}}%
\pgfpathclose%
\pgfusepath{fill}%
\end{pgfscope}%
\begin{pgfscope}%
\pgfpathrectangle{\pgfqpoint{1.254980in}{0.150000in}}{\pgfqpoint{5.490039in}{5.490039in}}%
\pgfusepath{clip}%
\pgfsetbuttcap%
\pgfsetroundjoin%
\definecolor{currentfill}{rgb}{0.154815,0.493313,0.557840}%
\pgfsetfillcolor{currentfill}%
\pgfsetfillopacity{0.700000}%
\pgfsetlinewidth{0.000000pt}%
\definecolor{currentstroke}{rgb}{0.000000,0.000000,0.000000}%
\pgfsetstrokecolor{currentstroke}%
\pgfsetdash{}{0pt}%
\pgfpathmoveto{\pgfqpoint{3.745717in}{3.496443in}}%
\pgfpathlineto{\pgfqpoint{3.758530in}{3.481211in}}%
\pgfpathlineto{\pgfqpoint{3.771341in}{3.466212in}}%
\pgfpathlineto{\pgfqpoint{3.784150in}{3.451445in}}%
\pgfpathlineto{\pgfqpoint{3.796959in}{3.436908in}}%
\pgfpathlineto{\pgfqpoint{3.804382in}{3.457265in}}%
\pgfpathlineto{\pgfqpoint{3.811802in}{3.477916in}}%
\pgfpathlineto{\pgfqpoint{3.819219in}{3.498865in}}%
\pgfpathlineto{\pgfqpoint{3.826633in}{3.520120in}}%
\pgfpathlineto{\pgfqpoint{3.813827in}{3.535162in}}%
\pgfpathlineto{\pgfqpoint{3.801019in}{3.550434in}}%
\pgfpathlineto{\pgfqpoint{3.788210in}{3.565939in}}%
\pgfpathlineto{\pgfqpoint{3.775399in}{3.581678in}}%
\pgfpathlineto{\pgfqpoint{3.767983in}{3.559906in}}%
\pgfpathlineto{\pgfqpoint{3.760564in}{3.538447in}}%
\pgfpathlineto{\pgfqpoint{3.753142in}{3.517295in}}%
\pgfpathlineto{\pgfqpoint{3.745717in}{3.496443in}}%
\pgfpathclose%
\pgfusepath{fill}%
\end{pgfscope}%
\begin{pgfscope}%
\pgfpathrectangle{\pgfqpoint{1.254980in}{0.150000in}}{\pgfqpoint{5.490039in}{5.490039in}}%
\pgfusepath{clip}%
\pgfsetbuttcap%
\pgfsetroundjoin%
\definecolor{currentfill}{rgb}{0.163625,0.471133,0.558148}%
\pgfsetfillcolor{currentfill}%
\pgfsetfillopacity{0.700000}%
\pgfsetlinewidth{0.000000pt}%
\definecolor{currentstroke}{rgb}{0.000000,0.000000,0.000000}%
\pgfsetstrokecolor{currentstroke}%
\pgfsetdash{}{0pt}%
\pgfpathmoveto{\pgfqpoint{3.796959in}{3.436908in}}%
\pgfpathlineto{\pgfqpoint{3.809766in}{3.422599in}}%
\pgfpathlineto{\pgfqpoint{3.822572in}{3.408518in}}%
\pgfpathlineto{\pgfqpoint{3.835378in}{3.394661in}}%
\pgfpathlineto{\pgfqpoint{3.848183in}{3.381027in}}%
\pgfpathlineto{\pgfqpoint{3.855604in}{3.400892in}}%
\pgfpathlineto{\pgfqpoint{3.863022in}{3.421043in}}%
\pgfpathlineto{\pgfqpoint{3.870437in}{3.441485in}}%
\pgfpathlineto{\pgfqpoint{3.877849in}{3.462225in}}%
\pgfpathlineto{\pgfqpoint{3.865046in}{3.476361in}}%
\pgfpathlineto{\pgfqpoint{3.852243in}{3.490721in}}%
\pgfpathlineto{\pgfqpoint{3.839439in}{3.505307in}}%
\pgfpathlineto{\pgfqpoint{3.826633in}{3.520120in}}%
\pgfpathlineto{\pgfqpoint{3.819219in}{3.498865in}}%
\pgfpathlineto{\pgfqpoint{3.811802in}{3.477916in}}%
\pgfpathlineto{\pgfqpoint{3.804382in}{3.457265in}}%
\pgfpathlineto{\pgfqpoint{3.796959in}{3.436908in}}%
\pgfpathclose%
\pgfusepath{fill}%
\end{pgfscope}%
\begin{pgfscope}%
\pgfpathrectangle{\pgfqpoint{1.254980in}{0.150000in}}{\pgfqpoint{5.490039in}{5.490039in}}%
\pgfusepath{clip}%
\pgfsetbuttcap%
\pgfsetroundjoin%
\definecolor{currentfill}{rgb}{0.156270,0.489624,0.557936}%
\pgfsetfillcolor{currentfill}%
\pgfsetfillopacity{0.700000}%
\pgfsetlinewidth{0.000000pt}%
\definecolor{currentstroke}{rgb}{0.000000,0.000000,0.000000}%
\pgfsetstrokecolor{currentstroke}%
\pgfsetdash{}{0pt}%
\pgfpathmoveto{\pgfqpoint{4.354729in}{3.465150in}}%
\pgfpathlineto{\pgfqpoint{4.367560in}{3.454649in}}%
\pgfpathlineto{\pgfqpoint{4.380395in}{3.444336in}}%
\pgfpathlineto{\pgfqpoint{4.393233in}{3.434209in}}%
\pgfpathlineto{\pgfqpoint{4.406074in}{3.424267in}}%
\pgfpathlineto{\pgfqpoint{4.413446in}{3.446009in}}%
\pgfpathlineto{\pgfqpoint{4.420818in}{3.468111in}}%
\pgfpathlineto{\pgfqpoint{4.428193in}{3.490582in}}%
\pgfpathlineto{\pgfqpoint{4.435569in}{3.513428in}}%
\pgfpathlineto{\pgfqpoint{4.422732in}{3.524007in}}%
\pgfpathlineto{\pgfqpoint{4.409899in}{3.534772in}}%
\pgfpathlineto{\pgfqpoint{4.397069in}{3.545723in}}%
\pgfpathlineto{\pgfqpoint{4.384241in}{3.556862in}}%
\pgfpathlineto{\pgfqpoint{4.376861in}{3.533366in}}%
\pgfpathlineto{\pgfqpoint{4.369482in}{3.510254in}}%
\pgfpathlineto{\pgfqpoint{4.362105in}{3.487518in}}%
\pgfpathlineto{\pgfqpoint{4.354729in}{3.465150in}}%
\pgfpathclose%
\pgfusepath{fill}%
\end{pgfscope}%
\begin{pgfscope}%
\pgfpathrectangle{\pgfqpoint{1.254980in}{0.150000in}}{\pgfqpoint{5.490039in}{5.490039in}}%
\pgfusepath{clip}%
\pgfsetbuttcap%
\pgfsetroundjoin%
\definecolor{currentfill}{rgb}{0.177423,0.437527,0.557565}%
\pgfsetfillcolor{currentfill}%
\pgfsetfillopacity{0.700000}%
\pgfsetlinewidth{0.000000pt}%
\definecolor{currentstroke}{rgb}{0.000000,0.000000,0.000000}%
\pgfsetstrokecolor{currentstroke}%
\pgfsetdash{}{0pt}%
\pgfpathmoveto{\pgfqpoint{4.112318in}{3.342864in}}%
\pgfpathlineto{\pgfqpoint{4.125131in}{3.331784in}}%
\pgfpathlineto{\pgfqpoint{4.137946in}{3.320903in}}%
\pgfpathlineto{\pgfqpoint{4.150763in}{3.310220in}}%
\pgfpathlineto{\pgfqpoint{4.163582in}{3.299734in}}%
\pgfpathlineto{\pgfqpoint{4.170968in}{3.319325in}}%
\pgfpathlineto{\pgfqpoint{4.178354in}{3.339216in}}%
\pgfpathlineto{\pgfqpoint{4.185738in}{3.359411in}}%
\pgfpathlineto{\pgfqpoint{4.193121in}{3.379919in}}%
\pgfpathlineto{\pgfqpoint{4.180306in}{3.390958in}}%
\pgfpathlineto{\pgfqpoint{4.167493in}{3.402194in}}%
\pgfpathlineto{\pgfqpoint{4.154682in}{3.413629in}}%
\pgfpathlineto{\pgfqpoint{4.141873in}{3.425263in}}%
\pgfpathlineto{\pgfqpoint{4.134486in}{3.404191in}}%
\pgfpathlineto{\pgfqpoint{4.127098in}{3.383439in}}%
\pgfpathlineto{\pgfqpoint{4.119709in}{3.362999in}}%
\pgfpathlineto{\pgfqpoint{4.112318in}{3.342864in}}%
\pgfpathclose%
\pgfusepath{fill}%
\end{pgfscope}%
\begin{pgfscope}%
\pgfpathrectangle{\pgfqpoint{1.254980in}{0.150000in}}{\pgfqpoint{5.490039in}{5.490039in}}%
\pgfusepath{clip}%
\pgfsetbuttcap%
\pgfsetroundjoin%
\definecolor{currentfill}{rgb}{0.144759,0.519093,0.556572}%
\pgfsetfillcolor{currentfill}%
\pgfsetfillopacity{0.700000}%
\pgfsetlinewidth{0.000000pt}%
\definecolor{currentstroke}{rgb}{0.000000,0.000000,0.000000}%
\pgfsetstrokecolor{currentstroke}%
\pgfsetdash{}{0pt}%
\pgfpathmoveto{\pgfqpoint{3.694447in}{3.559746in}}%
\pgfpathlineto{\pgfqpoint{3.707268in}{3.543561in}}%
\pgfpathlineto{\pgfqpoint{3.720086in}{3.527616in}}%
\pgfpathlineto{\pgfqpoint{3.732903in}{3.511911in}}%
\pgfpathlineto{\pgfqpoint{3.745717in}{3.496443in}}%
\pgfpathlineto{\pgfqpoint{3.753142in}{3.517295in}}%
\pgfpathlineto{\pgfqpoint{3.760564in}{3.538447in}}%
\pgfpathlineto{\pgfqpoint{3.767983in}{3.559906in}}%
\pgfpathlineto{\pgfqpoint{3.775399in}{3.581678in}}%
\pgfpathlineto{\pgfqpoint{3.762586in}{3.597653in}}%
\pgfpathlineto{\pgfqpoint{3.749772in}{3.613866in}}%
\pgfpathlineto{\pgfqpoint{3.736955in}{3.630319in}}%
\pgfpathlineto{\pgfqpoint{3.724135in}{3.647014in}}%
\pgfpathlineto{\pgfqpoint{3.716718in}{3.624722in}}%
\pgfpathlineto{\pgfqpoint{3.709298in}{3.602751in}}%
\pgfpathlineto{\pgfqpoint{3.701874in}{3.581094in}}%
\pgfpathlineto{\pgfqpoint{3.694447in}{3.559746in}}%
\pgfpathclose%
\pgfusepath{fill}%
\end{pgfscope}%
\begin{pgfscope}%
\pgfpathrectangle{\pgfqpoint{1.254980in}{0.150000in}}{\pgfqpoint{5.490039in}{5.490039in}}%
\pgfusepath{clip}%
\pgfsetbuttcap%
\pgfsetroundjoin%
\definecolor{currentfill}{rgb}{0.171176,0.452530,0.557965}%
\pgfsetfillcolor{currentfill}%
\pgfsetfillopacity{0.700000}%
\pgfsetlinewidth{0.000000pt}%
\definecolor{currentstroke}{rgb}{0.000000,0.000000,0.000000}%
\pgfsetstrokecolor{currentstroke}%
\pgfsetdash{}{0pt}%
\pgfpathmoveto{\pgfqpoint{3.848183in}{3.381027in}}%
\pgfpathlineto{\pgfqpoint{3.860988in}{3.367615in}}%
\pgfpathlineto{\pgfqpoint{3.873793in}{3.354424in}}%
\pgfpathlineto{\pgfqpoint{3.886598in}{3.341451in}}%
\pgfpathlineto{\pgfqpoint{3.899403in}{3.328695in}}%
\pgfpathlineto{\pgfqpoint{3.906820in}{3.348069in}}%
\pgfpathlineto{\pgfqpoint{3.914235in}{3.367722in}}%
\pgfpathlineto{\pgfqpoint{3.921647in}{3.387659in}}%
\pgfpathlineto{\pgfqpoint{3.929057in}{3.407887in}}%
\pgfpathlineto{\pgfqpoint{3.916255in}{3.421143in}}%
\pgfpathlineto{\pgfqpoint{3.903453in}{3.434617in}}%
\pgfpathlineto{\pgfqpoint{3.890651in}{3.448311in}}%
\pgfpathlineto{\pgfqpoint{3.877849in}{3.462225in}}%
\pgfpathlineto{\pgfqpoint{3.870437in}{3.441485in}}%
\pgfpathlineto{\pgfqpoint{3.863022in}{3.421043in}}%
\pgfpathlineto{\pgfqpoint{3.855604in}{3.400892in}}%
\pgfpathlineto{\pgfqpoint{3.848183in}{3.381027in}}%
\pgfpathclose%
\pgfusepath{fill}%
\end{pgfscope}%
\begin{pgfscope}%
\pgfpathrectangle{\pgfqpoint{1.254980in}{0.150000in}}{\pgfqpoint{5.490039in}{5.490039in}}%
\pgfusepath{clip}%
\pgfsetbuttcap%
\pgfsetroundjoin%
\definecolor{currentfill}{rgb}{0.266941,0.748751,0.440573}%
\pgfsetfillcolor{currentfill}%
\pgfsetfillopacity{0.700000}%
\pgfsetlinewidth{0.000000pt}%
\definecolor{currentstroke}{rgb}{0.000000,0.000000,0.000000}%
\pgfsetstrokecolor{currentstroke}%
\pgfsetdash{}{0pt}%
\pgfpathmoveto{\pgfqpoint{3.871977in}{4.170440in}}%
\pgfpathlineto{\pgfqpoint{3.884804in}{4.150722in}}%
\pgfpathlineto{\pgfqpoint{3.897629in}{4.131250in}}%
\pgfpathlineto{\pgfqpoint{3.910450in}{4.112023in}}%
\pgfpathlineto{\pgfqpoint{3.923269in}{4.093039in}}%
\pgfpathlineto{\pgfqpoint{3.930650in}{4.122961in}}%
\pgfpathlineto{\pgfqpoint{3.938032in}{4.153356in}}%
\pgfpathlineto{\pgfqpoint{3.945413in}{4.184233in}}%
\pgfpathlineto{\pgfqpoint{3.932592in}{4.203740in}}%
\pgfpathlineto{\pgfqpoint{3.919767in}{4.223491in}}%
\pgfpathlineto{\pgfqpoint{3.906939in}{4.243487in}}%
\pgfpathlineto{\pgfqpoint{3.894108in}{4.263731in}}%
\pgfpathlineto{\pgfqpoint{3.886731in}{4.232147in}}%
\pgfpathlineto{\pgfqpoint{3.879354in}{4.201053in}}%
\pgfpathlineto{\pgfqpoint{3.871977in}{4.170440in}}%
\pgfpathclose%
\pgfusepath{fill}%
\end{pgfscope}%
\begin{pgfscope}%
\pgfpathrectangle{\pgfqpoint{1.254980in}{0.150000in}}{\pgfqpoint{5.490039in}{5.490039in}}%
\pgfusepath{clip}%
\pgfsetbuttcap%
\pgfsetroundjoin%
\definecolor{currentfill}{rgb}{0.128087,0.647749,0.523491}%
\pgfsetfillcolor{currentfill}%
\pgfsetfillopacity{0.700000}%
\pgfsetlinewidth{0.000000pt}%
\definecolor{currentstroke}{rgb}{0.000000,0.000000,0.000000}%
\pgfsetstrokecolor{currentstroke}%
\pgfsetdash{}{0pt}%
\pgfpathmoveto{\pgfqpoint{3.651104in}{3.886365in}}%
\pgfpathlineto{\pgfqpoint{3.663951in}{3.867116in}}%
\pgfpathlineto{\pgfqpoint{3.676793in}{3.848126in}}%
\pgfpathlineto{\pgfqpoint{3.689632in}{3.829393in}}%
\pgfpathlineto{\pgfqpoint{3.702467in}{3.810915in}}%
\pgfpathlineto{\pgfqpoint{3.709869in}{3.835466in}}%
\pgfpathlineto{\pgfqpoint{3.717268in}{3.860383in}}%
\pgfpathlineto{\pgfqpoint{3.724665in}{3.885675in}}%
\pgfpathlineto{\pgfqpoint{3.732058in}{3.911348in}}%
\pgfpathlineto{\pgfqpoint{3.719222in}{3.930403in}}%
\pgfpathlineto{\pgfqpoint{3.706382in}{3.949713in}}%
\pgfpathlineto{\pgfqpoint{3.693538in}{3.969281in}}%
\pgfpathlineto{\pgfqpoint{3.680690in}{3.989110in}}%
\pgfpathlineto{\pgfqpoint{3.673298in}{3.962847in}}%
\pgfpathlineto{\pgfqpoint{3.665903in}{3.936973in}}%
\pgfpathlineto{\pgfqpoint{3.658505in}{3.911482in}}%
\pgfpathlineto{\pgfqpoint{3.651104in}{3.886365in}}%
\pgfpathclose%
\pgfusepath{fill}%
\end{pgfscope}%
\begin{pgfscope}%
\pgfpathrectangle{\pgfqpoint{1.254980in}{0.150000in}}{\pgfqpoint{5.490039in}{5.490039in}}%
\pgfusepath{clip}%
\pgfsetbuttcap%
\pgfsetroundjoin%
\definecolor{currentfill}{rgb}{0.150476,0.504369,0.557430}%
\pgfsetfillcolor{currentfill}%
\pgfsetfillopacity{0.700000}%
\pgfsetlinewidth{0.000000pt}%
\definecolor{currentstroke}{rgb}{0.000000,0.000000,0.000000}%
\pgfsetstrokecolor{currentstroke}%
\pgfsetdash{}{0pt}%
\pgfpathmoveto{\pgfqpoint{4.435569in}{3.513428in}}%
\pgfpathlineto{\pgfqpoint{4.448409in}{3.503035in}}%
\pgfpathlineto{\pgfqpoint{4.461252in}{3.492825in}}%
\pgfpathlineto{\pgfqpoint{4.474100in}{3.482798in}}%
\pgfpathlineto{\pgfqpoint{4.486951in}{3.472953in}}%
\pgfpathlineto{\pgfqpoint{4.494324in}{3.495529in}}%
\pgfpathlineto{\pgfqpoint{4.501699in}{3.518488in}}%
\pgfpathlineto{\pgfqpoint{4.509077in}{3.541840in}}%
\pgfpathlineto{\pgfqpoint{4.496229in}{3.552181in}}%
\pgfpathlineto{\pgfqpoint{4.483386in}{3.562704in}}%
\pgfpathlineto{\pgfqpoint{4.470546in}{3.573411in}}%
\pgfpathlineto{\pgfqpoint{4.457709in}{3.584302in}}%
\pgfpathlineto{\pgfqpoint{4.450327in}{3.560281in}}%
\pgfpathlineto{\pgfqpoint{4.442947in}{3.536659in}}%
\pgfpathlineto{\pgfqpoint{4.435569in}{3.513428in}}%
\pgfpathclose%
\pgfusepath{fill}%
\end{pgfscope}%
\begin{pgfscope}%
\pgfpathrectangle{\pgfqpoint{1.254980in}{0.150000in}}{\pgfqpoint{5.490039in}{5.490039in}}%
\pgfusepath{clip}%
\pgfsetbuttcap%
\pgfsetroundjoin%
\definecolor{currentfill}{rgb}{0.162016,0.687316,0.499129}%
\pgfsetfillcolor{currentfill}%
\pgfsetfillopacity{0.700000}%
\pgfsetlinewidth{0.000000pt}%
\definecolor{currentstroke}{rgb}{0.000000,0.000000,0.000000}%
\pgfsetstrokecolor{currentstroke}%
\pgfsetdash{}{0pt}%
\pgfpathmoveto{\pgfqpoint{3.680690in}{3.989110in}}%
\pgfpathlineto{\pgfqpoint{3.693538in}{3.969281in}}%
\pgfpathlineto{\pgfqpoint{3.706382in}{3.949713in}}%
\pgfpathlineto{\pgfqpoint{3.719222in}{3.930403in}}%
\pgfpathlineto{\pgfqpoint{3.732058in}{3.911348in}}%
\pgfpathlineto{\pgfqpoint{3.739450in}{3.937409in}}%
\pgfpathlineto{\pgfqpoint{3.746839in}{3.963865in}}%
\pgfpathlineto{\pgfqpoint{3.754226in}{3.990724in}}%
\pgfpathlineto{\pgfqpoint{3.761611in}{4.017993in}}%
\pgfpathlineto{\pgfqpoint{3.748772in}{4.037656in}}%
\pgfpathlineto{\pgfqpoint{3.735930in}{4.057576in}}%
\pgfpathlineto{\pgfqpoint{3.723083in}{4.077756in}}%
\pgfpathlineto{\pgfqpoint{3.710232in}{4.098198in}}%
\pgfpathlineto{\pgfqpoint{3.702850in}{4.070306in}}%
\pgfpathlineto{\pgfqpoint{3.695466in}{4.042832in}}%
\pgfpathlineto{\pgfqpoint{3.688079in}{4.015769in}}%
\pgfpathlineto{\pgfqpoint{3.680690in}{3.989110in}}%
\pgfpathclose%
\pgfusepath{fill}%
\end{pgfscope}%
\begin{pgfscope}%
\pgfpathrectangle{\pgfqpoint{1.254980in}{0.150000in}}{\pgfqpoint{5.490039in}{5.490039in}}%
\pgfusepath{clip}%
\pgfsetbuttcap%
\pgfsetroundjoin%
\definecolor{currentfill}{rgb}{0.135066,0.544853,0.554029}%
\pgfsetfillcolor{currentfill}%
\pgfsetfillopacity{0.700000}%
\pgfsetlinewidth{0.000000pt}%
\definecolor{currentstroke}{rgb}{0.000000,0.000000,0.000000}%
\pgfsetstrokecolor{currentstroke}%
\pgfsetdash{}{0pt}%
\pgfpathmoveto{\pgfqpoint{3.643137in}{3.626938in}}%
\pgfpathlineto{\pgfqpoint{3.655969in}{3.609768in}}%
\pgfpathlineto{\pgfqpoint{3.668798in}{3.592848in}}%
\pgfpathlineto{\pgfqpoint{3.681624in}{3.576174in}}%
\pgfpathlineto{\pgfqpoint{3.694447in}{3.559746in}}%
\pgfpathlineto{\pgfqpoint{3.701874in}{3.581094in}}%
\pgfpathlineto{\pgfqpoint{3.709298in}{3.602751in}}%
\pgfpathlineto{\pgfqpoint{3.716718in}{3.624722in}}%
\pgfpathlineto{\pgfqpoint{3.724135in}{3.647014in}}%
\pgfpathlineto{\pgfqpoint{3.711313in}{3.663952in}}%
\pgfpathlineto{\pgfqpoint{3.698488in}{3.681136in}}%
\pgfpathlineto{\pgfqpoint{3.685660in}{3.698568in}}%
\pgfpathlineto{\pgfqpoint{3.672829in}{3.716250in}}%
\pgfpathlineto{\pgfqpoint{3.665411in}{3.693436in}}%
\pgfpathlineto{\pgfqpoint{3.657990in}{3.670950in}}%
\pgfpathlineto{\pgfqpoint{3.650565in}{3.648786in}}%
\pgfpathlineto{\pgfqpoint{3.643137in}{3.626938in}}%
\pgfpathclose%
\pgfusepath{fill}%
\end{pgfscope}%
\begin{pgfscope}%
\pgfpathrectangle{\pgfqpoint{1.254980in}{0.150000in}}{\pgfqpoint{5.490039in}{5.490039in}}%
\pgfusepath{clip}%
\pgfsetbuttcap%
\pgfsetroundjoin%
\definecolor{currentfill}{rgb}{0.182256,0.426184,0.557120}%
\pgfsetfillcolor{currentfill}%
\pgfsetfillopacity{0.700000}%
\pgfsetlinewidth{0.000000pt}%
\definecolor{currentstroke}{rgb}{0.000000,0.000000,0.000000}%
\pgfsetstrokecolor{currentstroke}%
\pgfsetdash{}{0pt}%
\pgfpathmoveto{\pgfqpoint{4.031493in}{3.309492in}}%
\pgfpathlineto{\pgfqpoint{4.044302in}{3.298129in}}%
\pgfpathlineto{\pgfqpoint{4.057113in}{3.286970in}}%
\pgfpathlineto{\pgfqpoint{4.069926in}{3.276013in}}%
\pgfpathlineto{\pgfqpoint{4.082741in}{3.265258in}}%
\pgfpathlineto{\pgfqpoint{4.090138in}{3.284232in}}%
\pgfpathlineto{\pgfqpoint{4.097533in}{3.303488in}}%
\pgfpathlineto{\pgfqpoint{4.104926in}{3.323029in}}%
\pgfpathlineto{\pgfqpoint{4.112318in}{3.342864in}}%
\pgfpathlineto{\pgfqpoint{4.099508in}{3.354145in}}%
\pgfpathlineto{\pgfqpoint{4.086699in}{3.365627in}}%
\pgfpathlineto{\pgfqpoint{4.073891in}{3.377312in}}%
\pgfpathlineto{\pgfqpoint{4.061085in}{3.389201in}}%
\pgfpathlineto{\pgfqpoint{4.053690in}{3.368830in}}%
\pgfpathlineto{\pgfqpoint{4.046293in}{3.348759in}}%
\pgfpathlineto{\pgfqpoint{4.038894in}{3.328981in}}%
\pgfpathlineto{\pgfqpoint{4.031493in}{3.309492in}}%
\pgfpathclose%
\pgfusepath{fill}%
\end{pgfscope}%
\begin{pgfscope}%
\pgfpathrectangle{\pgfqpoint{1.254980in}{0.150000in}}{\pgfqpoint{5.490039in}{5.490039in}}%
\pgfusepath{clip}%
\pgfsetbuttcap%
\pgfsetroundjoin%
\definecolor{currentfill}{rgb}{0.252899,0.742211,0.448284}%
\pgfsetfillcolor{currentfill}%
\pgfsetfillopacity{0.700000}%
\pgfsetlinewidth{0.000000pt}%
\definecolor{currentstroke}{rgb}{0.000000,0.000000,0.000000}%
\pgfsetstrokecolor{currentstroke}%
\pgfsetdash{}{0pt}%
\pgfpathmoveto{\pgfqpoint{3.791131in}{4.131318in}}%
\pgfpathlineto{\pgfqpoint{3.803969in}{4.111268in}}%
\pgfpathlineto{\pgfqpoint{3.816804in}{4.091471in}}%
\pgfpathlineto{\pgfqpoint{3.829634in}{4.071927in}}%
\pgfpathlineto{\pgfqpoint{3.842461in}{4.052632in}}%
\pgfpathlineto{\pgfqpoint{3.849841in}{4.081404in}}%
\pgfpathlineto{\pgfqpoint{3.857221in}{4.110624in}}%
\pgfpathlineto{\pgfqpoint{3.864599in}{4.140300in}}%
\pgfpathlineto{\pgfqpoint{3.871977in}{4.170440in}}%
\pgfpathlineto{\pgfqpoint{3.859146in}{4.190407in}}%
\pgfpathlineto{\pgfqpoint{3.846311in}{4.210624in}}%
\pgfpathlineto{\pgfqpoint{3.833473in}{4.231095in}}%
\pgfpathlineto{\pgfqpoint{3.820630in}{4.251821in}}%
\pgfpathlineto{\pgfqpoint{3.813257in}{4.220994in}}%
\pgfpathlineto{\pgfqpoint{3.805883in}{4.190640in}}%
\pgfpathlineto{\pgfqpoint{3.798508in}{4.160750in}}%
\pgfpathlineto{\pgfqpoint{3.791131in}{4.131318in}}%
\pgfpathclose%
\pgfusepath{fill}%
\end{pgfscope}%
\begin{pgfscope}%
\pgfpathrectangle{\pgfqpoint{1.254980in}{0.150000in}}{\pgfqpoint{5.490039in}{5.490039in}}%
\pgfusepath{clip}%
\pgfsetbuttcap%
\pgfsetroundjoin%
\definecolor{currentfill}{rgb}{0.119423,0.611141,0.538982}%
\pgfsetfillcolor{currentfill}%
\pgfsetfillopacity{0.700000}%
\pgfsetlinewidth{0.000000pt}%
\definecolor{currentstroke}{rgb}{0.000000,0.000000,0.000000}%
\pgfsetstrokecolor{currentstroke}%
\pgfsetdash{}{0pt}%
\pgfpathmoveto{\pgfqpoint{3.621468in}{3.789518in}}%
\pgfpathlineto{\pgfqpoint{3.634314in}{3.770815in}}%
\pgfpathlineto{\pgfqpoint{3.647156in}{3.752371in}}%
\pgfpathlineto{\pgfqpoint{3.659994in}{3.734184in}}%
\pgfpathlineto{\pgfqpoint{3.672829in}{3.716250in}}%
\pgfpathlineto{\pgfqpoint{3.680243in}{3.739399in}}%
\pgfpathlineto{\pgfqpoint{3.687654in}{3.762888in}}%
\pgfpathlineto{\pgfqpoint{3.695062in}{3.786724in}}%
\pgfpathlineto{\pgfqpoint{3.702467in}{3.810915in}}%
\pgfpathlineto{\pgfqpoint{3.689632in}{3.829393in}}%
\pgfpathlineto{\pgfqpoint{3.676793in}{3.848126in}}%
\pgfpathlineto{\pgfqpoint{3.663951in}{3.867116in}}%
\pgfpathlineto{\pgfqpoint{3.651104in}{3.886365in}}%
\pgfpathlineto{\pgfqpoint{3.643700in}{3.861617in}}%
\pgfpathlineto{\pgfqpoint{3.636293in}{3.837231in}}%
\pgfpathlineto{\pgfqpoint{3.628882in}{3.813200in}}%
\pgfpathlineto{\pgfqpoint{3.621468in}{3.789518in}}%
\pgfpathclose%
\pgfusepath{fill}%
\end{pgfscope}%
\begin{pgfscope}%
\pgfpathrectangle{\pgfqpoint{1.254980in}{0.150000in}}{\pgfqpoint{5.490039in}{5.490039in}}%
\pgfusepath{clip}%
\pgfsetbuttcap%
\pgfsetroundjoin%
\definecolor{currentfill}{rgb}{0.180629,0.429975,0.557282}%
\pgfsetfillcolor{currentfill}%
\pgfsetfillopacity{0.700000}%
\pgfsetlinewidth{0.000000pt}%
\definecolor{currentstroke}{rgb}{0.000000,0.000000,0.000000}%
\pgfsetstrokecolor{currentstroke}%
\pgfsetdash{}{0pt}%
\pgfpathmoveto{\pgfqpoint{3.899403in}{3.328695in}}%
\pgfpathlineto{\pgfqpoint{3.912208in}{3.316154in}}%
\pgfpathlineto{\pgfqpoint{3.925014in}{3.303828in}}%
\pgfpathlineto{\pgfqpoint{3.937820in}{3.291715in}}%
\pgfpathlineto{\pgfqpoint{3.950627in}{3.279813in}}%
\pgfpathlineto{\pgfqpoint{3.958041in}{3.298699in}}%
\pgfpathlineto{\pgfqpoint{3.965453in}{3.317856in}}%
\pgfpathlineto{\pgfqpoint{3.972861in}{3.337290in}}%
\pgfpathlineto{\pgfqpoint{3.980268in}{3.357006in}}%
\pgfpathlineto{\pgfqpoint{3.967464in}{3.369407in}}%
\pgfpathlineto{\pgfqpoint{3.954661in}{3.382020in}}%
\pgfpathlineto{\pgfqpoint{3.941859in}{3.394846in}}%
\pgfpathlineto{\pgfqpoint{3.929057in}{3.407887in}}%
\pgfpathlineto{\pgfqpoint{3.921647in}{3.387659in}}%
\pgfpathlineto{\pgfqpoint{3.914235in}{3.367722in}}%
\pgfpathlineto{\pgfqpoint{3.906820in}{3.348069in}}%
\pgfpathlineto{\pgfqpoint{3.899403in}{3.328695in}}%
\pgfpathclose%
\pgfusepath{fill}%
\end{pgfscope}%
\begin{pgfscope}%
\pgfpathrectangle{\pgfqpoint{1.254980in}{0.150000in}}{\pgfqpoint{5.490039in}{5.490039in}}%
\pgfusepath{clip}%
\pgfsetbuttcap%
\pgfsetroundjoin%
\definecolor{currentfill}{rgb}{0.169646,0.456262,0.558030}%
\pgfsetfillcolor{currentfill}%
\pgfsetfillopacity{0.700000}%
\pgfsetlinewidth{0.000000pt}%
\definecolor{currentstroke}{rgb}{0.000000,0.000000,0.000000}%
\pgfsetstrokecolor{currentstroke}%
\pgfsetdash{}{0pt}%
\pgfpathmoveto{\pgfqpoint{4.325232in}{3.379208in}}%
\pgfpathlineto{\pgfqpoint{4.338068in}{3.369316in}}%
\pgfpathlineto{\pgfqpoint{4.350908in}{3.359610in}}%
\pgfpathlineto{\pgfqpoint{4.363751in}{3.350091in}}%
\pgfpathlineto{\pgfqpoint{4.376598in}{3.340756in}}%
\pgfpathlineto{\pgfqpoint{4.383966in}{3.361130in}}%
\pgfpathlineto{\pgfqpoint{4.391335in}{3.381835in}}%
\pgfpathlineto{\pgfqpoint{4.398704in}{3.402878in}}%
\pgfpathlineto{\pgfqpoint{4.406074in}{3.424267in}}%
\pgfpathlineto{\pgfqpoint{4.393233in}{3.434209in}}%
\pgfpathlineto{\pgfqpoint{4.380395in}{3.444336in}}%
\pgfpathlineto{\pgfqpoint{4.367560in}{3.454649in}}%
\pgfpathlineto{\pgfqpoint{4.354729in}{3.465150in}}%
\pgfpathlineto{\pgfqpoint{4.347354in}{3.443142in}}%
\pgfpathlineto{\pgfqpoint{4.339979in}{3.421488in}}%
\pgfpathlineto{\pgfqpoint{4.332605in}{3.400179in}}%
\pgfpathlineto{\pgfqpoint{4.325232in}{3.379208in}}%
\pgfpathclose%
\pgfusepath{fill}%
\end{pgfscope}%
\begin{pgfscope}%
\pgfpathrectangle{\pgfqpoint{1.254980in}{0.150000in}}{\pgfqpoint{5.490039in}{5.490039in}}%
\pgfusepath{clip}%
\pgfsetbuttcap%
\pgfsetroundjoin%
\definecolor{currentfill}{rgb}{0.177423,0.437527,0.557565}%
\pgfsetfillcolor{currentfill}%
\pgfsetfillopacity{0.700000}%
\pgfsetlinewidth{0.000000pt}%
\definecolor{currentstroke}{rgb}{0.000000,0.000000,0.000000}%
\pgfsetstrokecolor{currentstroke}%
\pgfsetdash{}{0pt}%
\pgfpathmoveto{\pgfqpoint{4.244407in}{3.337711in}}%
\pgfpathlineto{\pgfqpoint{4.257235in}{3.327641in}}%
\pgfpathlineto{\pgfqpoint{4.270067in}{3.317762in}}%
\pgfpathlineto{\pgfqpoint{4.282901in}{3.308072in}}%
\pgfpathlineto{\pgfqpoint{4.295739in}{3.298571in}}%
\pgfpathlineto{\pgfqpoint{4.303113in}{3.318258in}}%
\pgfpathlineto{\pgfqpoint{4.310486in}{3.338255in}}%
\pgfpathlineto{\pgfqpoint{4.317859in}{3.358570in}}%
\pgfpathlineto{\pgfqpoint{4.325232in}{3.379208in}}%
\pgfpathlineto{\pgfqpoint{4.312399in}{3.389289in}}%
\pgfpathlineto{\pgfqpoint{4.299569in}{3.399558in}}%
\pgfpathlineto{\pgfqpoint{4.286743in}{3.410017in}}%
\pgfpathlineto{\pgfqpoint{4.273919in}{3.420668in}}%
\pgfpathlineto{\pgfqpoint{4.266541in}{3.399438in}}%
\pgfpathlineto{\pgfqpoint{4.259163in}{3.378540in}}%
\pgfpathlineto{\pgfqpoint{4.251785in}{3.357967in}}%
\pgfpathlineto{\pgfqpoint{4.244407in}{3.337711in}}%
\pgfpathclose%
\pgfusepath{fill}%
\end{pgfscope}%
\begin{pgfscope}%
\pgfpathrectangle{\pgfqpoint{1.254980in}{0.150000in}}{\pgfqpoint{5.490039in}{5.490039in}}%
\pgfusepath{clip}%
\pgfsetbuttcap%
\pgfsetroundjoin%
\definecolor{currentfill}{rgb}{0.226397,0.728888,0.462789}%
\pgfsetfillcolor{currentfill}%
\pgfsetfillopacity{0.700000}%
\pgfsetlinewidth{0.000000pt}%
\definecolor{currentstroke}{rgb}{0.000000,0.000000,0.000000}%
\pgfsetstrokecolor{currentstroke}%
\pgfsetdash{}{0pt}%
\pgfpathmoveto{\pgfqpoint{3.710232in}{4.098198in}}%
\pgfpathlineto{\pgfqpoint{3.723083in}{4.077756in}}%
\pgfpathlineto{\pgfqpoint{3.735930in}{4.057576in}}%
\pgfpathlineto{\pgfqpoint{3.748772in}{4.037656in}}%
\pgfpathlineto{\pgfqpoint{3.761611in}{4.017993in}}%
\pgfpathlineto{\pgfqpoint{3.768993in}{4.045679in}}%
\pgfpathlineto{\pgfqpoint{3.776374in}{4.073790in}}%
\pgfpathlineto{\pgfqpoint{3.783754in}{4.102333in}}%
\pgfpathlineto{\pgfqpoint{3.791131in}{4.131318in}}%
\pgfpathlineto{\pgfqpoint{3.778290in}{4.151624in}}%
\pgfpathlineto{\pgfqpoint{3.765443in}{4.172188in}}%
\pgfpathlineto{\pgfqpoint{3.752593in}{4.193013in}}%
\pgfpathlineto{\pgfqpoint{3.739738in}{4.214100in}}%
\pgfpathlineto{\pgfqpoint{3.732364in}{4.184459in}}%
\pgfpathlineto{\pgfqpoint{3.724989in}{4.155266in}}%
\pgfpathlineto{\pgfqpoint{3.717611in}{4.126515in}}%
\pgfpathlineto{\pgfqpoint{3.710232in}{4.098198in}}%
\pgfpathclose%
\pgfusepath{fill}%
\end{pgfscope}%
\begin{pgfscope}%
\pgfpathrectangle{\pgfqpoint{1.254980in}{0.150000in}}{\pgfqpoint{5.490039in}{5.490039in}}%
\pgfusepath{clip}%
\pgfsetbuttcap%
\pgfsetroundjoin%
\definecolor{currentfill}{rgb}{0.162142,0.474838,0.558140}%
\pgfsetfillcolor{currentfill}%
\pgfsetfillopacity{0.700000}%
\pgfsetlinewidth{0.000000pt}%
\definecolor{currentstroke}{rgb}{0.000000,0.000000,0.000000}%
\pgfsetstrokecolor{currentstroke}%
\pgfsetdash{}{0pt}%
\pgfpathmoveto{\pgfqpoint{4.406074in}{3.424267in}}%
\pgfpathlineto{\pgfqpoint{4.418919in}{3.414509in}}%
\pgfpathlineto{\pgfqpoint{4.431768in}{3.404936in}}%
\pgfpathlineto{\pgfqpoint{4.444621in}{3.395544in}}%
\pgfpathlineto{\pgfqpoint{4.457478in}{3.386335in}}%
\pgfpathlineto{\pgfqpoint{4.464844in}{3.407452in}}%
\pgfpathlineto{\pgfqpoint{4.472211in}{3.428922in}}%
\pgfpathlineto{\pgfqpoint{4.479580in}{3.450753in}}%
\pgfpathlineto{\pgfqpoint{4.486951in}{3.472953in}}%
\pgfpathlineto{\pgfqpoint{4.474100in}{3.482798in}}%
\pgfpathlineto{\pgfqpoint{4.461252in}{3.492825in}}%
\pgfpathlineto{\pgfqpoint{4.448409in}{3.503035in}}%
\pgfpathlineto{\pgfqpoint{4.435569in}{3.513428in}}%
\pgfpathlineto{\pgfqpoint{4.428193in}{3.490582in}}%
\pgfpathlineto{\pgfqpoint{4.420818in}{3.468111in}}%
\pgfpathlineto{\pgfqpoint{4.413446in}{3.446009in}}%
\pgfpathlineto{\pgfqpoint{4.406074in}{3.424267in}}%
\pgfpathclose%
\pgfusepath{fill}%
\end{pgfscope}%
\begin{pgfscope}%
\pgfpathrectangle{\pgfqpoint{1.254980in}{0.150000in}}{\pgfqpoint{5.490039in}{5.490039in}}%
\pgfusepath{clip}%
\pgfsetbuttcap%
\pgfsetroundjoin%
\definecolor{currentfill}{rgb}{0.183898,0.422383,0.556944}%
\pgfsetfillcolor{currentfill}%
\pgfsetfillopacity{0.700000}%
\pgfsetlinewidth{0.000000pt}%
\definecolor{currentstroke}{rgb}{0.000000,0.000000,0.000000}%
\pgfsetstrokecolor{currentstroke}%
\pgfsetdash{}{0pt}%
\pgfpathmoveto{\pgfqpoint{4.163582in}{3.299734in}}%
\pgfpathlineto{\pgfqpoint{4.176404in}{3.289443in}}%
\pgfpathlineto{\pgfqpoint{4.189229in}{3.279347in}}%
\pgfpathlineto{\pgfqpoint{4.202056in}{3.269444in}}%
\pgfpathlineto{\pgfqpoint{4.214886in}{3.259734in}}%
\pgfpathlineto{\pgfqpoint{4.222268in}{3.278785in}}%
\pgfpathlineto{\pgfqpoint{4.229648in}{3.298127in}}%
\pgfpathlineto{\pgfqpoint{4.237028in}{3.317767in}}%
\pgfpathlineto{\pgfqpoint{4.244407in}{3.337711in}}%
\pgfpathlineto{\pgfqpoint{4.231581in}{3.347973in}}%
\pgfpathlineto{\pgfqpoint{4.218759in}{3.358427in}}%
\pgfpathlineto{\pgfqpoint{4.205939in}{3.369076in}}%
\pgfpathlineto{\pgfqpoint{4.193121in}{3.379919in}}%
\pgfpathlineto{\pgfqpoint{4.185738in}{3.359411in}}%
\pgfpathlineto{\pgfqpoint{4.178354in}{3.339216in}}%
\pgfpathlineto{\pgfqpoint{4.170968in}{3.319325in}}%
\pgfpathlineto{\pgfqpoint{4.163582in}{3.299734in}}%
\pgfpathclose%
\pgfusepath{fill}%
\end{pgfscope}%
\begin{pgfscope}%
\pgfpathrectangle{\pgfqpoint{1.254980in}{0.150000in}}{\pgfqpoint{5.490039in}{5.490039in}}%
\pgfusepath{clip}%
\pgfsetbuttcap%
\pgfsetroundjoin%
\definecolor{currentfill}{rgb}{0.165117,0.467423,0.558141}%
\pgfsetfillcolor{currentfill}%
\pgfsetfillopacity{0.700000}%
\pgfsetlinewidth{0.000000pt}%
\definecolor{currentstroke}{rgb}{0.000000,0.000000,0.000000}%
\pgfsetstrokecolor{currentstroke}%
\pgfsetdash{}{0pt}%
\pgfpathmoveto{\pgfqpoint{3.715982in}{3.415926in}}%
\pgfpathlineto{\pgfqpoint{3.728797in}{3.401171in}}%
\pgfpathlineto{\pgfqpoint{3.741611in}{3.386648in}}%
\pgfpathlineto{\pgfqpoint{3.754423in}{3.372356in}}%
\pgfpathlineto{\pgfqpoint{3.767234in}{3.358295in}}%
\pgfpathlineto{\pgfqpoint{3.774670in}{3.377537in}}%
\pgfpathlineto{\pgfqpoint{3.782103in}{3.397050in}}%
\pgfpathlineto{\pgfqpoint{3.789532in}{3.416838in}}%
\pgfpathlineto{\pgfqpoint{3.796959in}{3.436908in}}%
\pgfpathlineto{\pgfqpoint{3.784150in}{3.451445in}}%
\pgfpathlineto{\pgfqpoint{3.771341in}{3.466212in}}%
\pgfpathlineto{\pgfqpoint{3.758530in}{3.481211in}}%
\pgfpathlineto{\pgfqpoint{3.745717in}{3.496443in}}%
\pgfpathlineto{\pgfqpoint{3.738288in}{3.475886in}}%
\pgfpathlineto{\pgfqpoint{3.730856in}{3.455618in}}%
\pgfpathlineto{\pgfqpoint{3.723421in}{3.435634in}}%
\pgfpathlineto{\pgfqpoint{3.715982in}{3.415926in}}%
\pgfpathclose%
\pgfusepath{fill}%
\end{pgfscope}%
\begin{pgfscope}%
\pgfpathrectangle{\pgfqpoint{1.254980in}{0.150000in}}{\pgfqpoint{5.490039in}{5.490039in}}%
\pgfusepath{clip}%
\pgfsetbuttcap%
\pgfsetroundjoin%
\definecolor{currentfill}{rgb}{0.125394,0.574318,0.549086}%
\pgfsetfillcolor{currentfill}%
\pgfsetfillopacity{0.700000}%
\pgfsetlinewidth{0.000000pt}%
\definecolor{currentstroke}{rgb}{0.000000,0.000000,0.000000}%
\pgfsetstrokecolor{currentstroke}%
\pgfsetdash{}{0pt}%
\pgfpathmoveto{\pgfqpoint{3.591774in}{3.698149in}}%
\pgfpathlineto{\pgfqpoint{3.604621in}{3.679962in}}%
\pgfpathlineto{\pgfqpoint{3.617463in}{3.662032in}}%
\pgfpathlineto{\pgfqpoint{3.630302in}{3.644358in}}%
\pgfpathlineto{\pgfqpoint{3.643137in}{3.626938in}}%
\pgfpathlineto{\pgfqpoint{3.650565in}{3.648786in}}%
\pgfpathlineto{\pgfqpoint{3.657990in}{3.670950in}}%
\pgfpathlineto{\pgfqpoint{3.665411in}{3.693436in}}%
\pgfpathlineto{\pgfqpoint{3.672829in}{3.716250in}}%
\pgfpathlineto{\pgfqpoint{3.659994in}{3.734184in}}%
\pgfpathlineto{\pgfqpoint{3.647156in}{3.752371in}}%
\pgfpathlineto{\pgfqpoint{3.634314in}{3.770815in}}%
\pgfpathlineto{\pgfqpoint{3.621468in}{3.789518in}}%
\pgfpathlineto{\pgfqpoint{3.614050in}{3.766178in}}%
\pgfpathlineto{\pgfqpoint{3.606629in}{3.743174in}}%
\pgfpathlineto{\pgfqpoint{3.599203in}{3.720500in}}%
\pgfpathlineto{\pgfqpoint{3.591774in}{3.698149in}}%
\pgfpathclose%
\pgfusepath{fill}%
\end{pgfscope}%
\begin{pgfscope}%
\pgfpathrectangle{\pgfqpoint{1.254980in}{0.150000in}}{\pgfqpoint{5.490039in}{5.490039in}}%
\pgfusepath{clip}%
\pgfsetbuttcap%
\pgfsetroundjoin%
\definecolor{currentfill}{rgb}{0.156270,0.489624,0.557936}%
\pgfsetfillcolor{currentfill}%
\pgfsetfillopacity{0.700000}%
\pgfsetlinewidth{0.000000pt}%
\definecolor{currentstroke}{rgb}{0.000000,0.000000,0.000000}%
\pgfsetstrokecolor{currentstroke}%
\pgfsetdash{}{0pt}%
\pgfpathmoveto{\pgfqpoint{3.664703in}{3.477317in}}%
\pgfpathlineto{\pgfqpoint{3.677526in}{3.461610in}}%
\pgfpathlineto{\pgfqpoint{3.690347in}{3.446144in}}%
\pgfpathlineto{\pgfqpoint{3.703165in}{3.430917in}}%
\pgfpathlineto{\pgfqpoint{3.715982in}{3.415926in}}%
\pgfpathlineto{\pgfqpoint{3.723421in}{3.435634in}}%
\pgfpathlineto{\pgfqpoint{3.730856in}{3.455618in}}%
\pgfpathlineto{\pgfqpoint{3.738288in}{3.475886in}}%
\pgfpathlineto{\pgfqpoint{3.745717in}{3.496443in}}%
\pgfpathlineto{\pgfqpoint{3.732903in}{3.511911in}}%
\pgfpathlineto{\pgfqpoint{3.720086in}{3.527616in}}%
\pgfpathlineto{\pgfqpoint{3.707268in}{3.543561in}}%
\pgfpathlineto{\pgfqpoint{3.694447in}{3.559746in}}%
\pgfpathlineto{\pgfqpoint{3.687017in}{3.538700in}}%
\pgfpathlineto{\pgfqpoint{3.679583in}{3.517950in}}%
\pgfpathlineto{\pgfqpoint{3.672145in}{3.497491in}}%
\pgfpathlineto{\pgfqpoint{3.664703in}{3.477317in}}%
\pgfpathclose%
\pgfusepath{fill}%
\end{pgfscope}%
\begin{pgfscope}%
\pgfpathrectangle{\pgfqpoint{1.254980in}{0.150000in}}{\pgfqpoint{5.490039in}{5.490039in}}%
\pgfusepath{clip}%
\pgfsetbuttcap%
\pgfsetroundjoin%
\definecolor{currentfill}{rgb}{0.174274,0.445044,0.557792}%
\pgfsetfillcolor{currentfill}%
\pgfsetfillopacity{0.700000}%
\pgfsetlinewidth{0.000000pt}%
\definecolor{currentstroke}{rgb}{0.000000,0.000000,0.000000}%
\pgfsetstrokecolor{currentstroke}%
\pgfsetdash{}{0pt}%
\pgfpathmoveto{\pgfqpoint{3.767234in}{3.358295in}}%
\pgfpathlineto{\pgfqpoint{3.780044in}{3.344460in}}%
\pgfpathlineto{\pgfqpoint{3.792854in}{3.330853in}}%
\pgfpathlineto{\pgfqpoint{3.805663in}{3.317469in}}%
\pgfpathlineto{\pgfqpoint{3.818471in}{3.304309in}}%
\pgfpathlineto{\pgfqpoint{3.825904in}{3.323089in}}%
\pgfpathlineto{\pgfqpoint{3.833334in}{3.342131in}}%
\pgfpathlineto{\pgfqpoint{3.840760in}{3.361442in}}%
\pgfpathlineto{\pgfqpoint{3.848183in}{3.381027in}}%
\pgfpathlineto{\pgfqpoint{3.835378in}{3.394661in}}%
\pgfpathlineto{\pgfqpoint{3.822572in}{3.408518in}}%
\pgfpathlineto{\pgfqpoint{3.809766in}{3.422599in}}%
\pgfpathlineto{\pgfqpoint{3.796959in}{3.436908in}}%
\pgfpathlineto{\pgfqpoint{3.789532in}{3.416838in}}%
\pgfpathlineto{\pgfqpoint{3.782103in}{3.397050in}}%
\pgfpathlineto{\pgfqpoint{3.774670in}{3.377537in}}%
\pgfpathlineto{\pgfqpoint{3.767234in}{3.358295in}}%
\pgfpathclose%
\pgfusepath{fill}%
\end{pgfscope}%
\begin{pgfscope}%
\pgfpathrectangle{\pgfqpoint{1.254980in}{0.150000in}}{\pgfqpoint{5.490039in}{5.490039in}}%
\pgfusepath{clip}%
\pgfsetbuttcap%
\pgfsetroundjoin%
\definecolor{currentfill}{rgb}{0.187231,0.414746,0.556547}%
\pgfsetfillcolor{currentfill}%
\pgfsetfillopacity{0.700000}%
\pgfsetlinewidth{0.000000pt}%
\definecolor{currentstroke}{rgb}{0.000000,0.000000,0.000000}%
\pgfsetstrokecolor{currentstroke}%
\pgfsetdash{}{0pt}%
\pgfpathmoveto{\pgfqpoint{3.950627in}{3.279813in}}%
\pgfpathlineto{\pgfqpoint{3.963436in}{3.268121in}}%
\pgfpathlineto{\pgfqpoint{3.976245in}{3.256637in}}%
\pgfpathlineto{\pgfqpoint{3.989056in}{3.245361in}}%
\pgfpathlineto{\pgfqpoint{4.001868in}{3.234291in}}%
\pgfpathlineto{\pgfqpoint{4.009277in}{3.252690in}}%
\pgfpathlineto{\pgfqpoint{4.016685in}{3.271352in}}%
\pgfpathlineto{\pgfqpoint{4.024090in}{3.290284in}}%
\pgfpathlineto{\pgfqpoint{4.031493in}{3.309492in}}%
\pgfpathlineto{\pgfqpoint{4.018685in}{3.321060in}}%
\pgfpathlineto{\pgfqpoint{4.005878in}{3.332834in}}%
\pgfpathlineto{\pgfqpoint{3.993073in}{3.344816in}}%
\pgfpathlineto{\pgfqpoint{3.980268in}{3.357006in}}%
\pgfpathlineto{\pgfqpoint{3.972861in}{3.337290in}}%
\pgfpathlineto{\pgfqpoint{3.965453in}{3.317856in}}%
\pgfpathlineto{\pgfqpoint{3.958041in}{3.298699in}}%
\pgfpathlineto{\pgfqpoint{3.950627in}{3.279813in}}%
\pgfpathclose%
\pgfusepath{fill}%
\end{pgfscope}%
\begin{pgfscope}%
\pgfpathrectangle{\pgfqpoint{1.254980in}{0.150000in}}{\pgfqpoint{5.490039in}{5.490039in}}%
\pgfusepath{clip}%
\pgfsetbuttcap%
\pgfsetroundjoin%
\definecolor{currentfill}{rgb}{0.188923,0.410910,0.556326}%
\pgfsetfillcolor{currentfill}%
\pgfsetfillopacity{0.700000}%
\pgfsetlinewidth{0.000000pt}%
\definecolor{currentstroke}{rgb}{0.000000,0.000000,0.000000}%
\pgfsetstrokecolor{currentstroke}%
\pgfsetdash{}{0pt}%
\pgfpathmoveto{\pgfqpoint{4.082741in}{3.265258in}}%
\pgfpathlineto{\pgfqpoint{4.095558in}{3.254702in}}%
\pgfpathlineto{\pgfqpoint{4.108377in}{3.244345in}}%
\pgfpathlineto{\pgfqpoint{4.121199in}{3.234186in}}%
\pgfpathlineto{\pgfqpoint{4.134023in}{3.224224in}}%
\pgfpathlineto{\pgfqpoint{4.141415in}{3.242685in}}%
\pgfpathlineto{\pgfqpoint{4.148806in}{3.261420in}}%
\pgfpathlineto{\pgfqpoint{4.156195in}{3.280434in}}%
\pgfpathlineto{\pgfqpoint{4.163582in}{3.299734in}}%
\pgfpathlineto{\pgfqpoint{4.150763in}{3.310220in}}%
\pgfpathlineto{\pgfqpoint{4.137946in}{3.320903in}}%
\pgfpathlineto{\pgfqpoint{4.125131in}{3.331784in}}%
\pgfpathlineto{\pgfqpoint{4.112318in}{3.342864in}}%
\pgfpathlineto{\pgfqpoint{4.104926in}{3.323029in}}%
\pgfpathlineto{\pgfqpoint{4.097533in}{3.303488in}}%
\pgfpathlineto{\pgfqpoint{4.090138in}{3.284232in}}%
\pgfpathlineto{\pgfqpoint{4.082741in}{3.265258in}}%
\pgfpathclose%
\pgfusepath{fill}%
\end{pgfscope}%
\begin{pgfscope}%
\pgfpathrectangle{\pgfqpoint{1.254980in}{0.150000in}}{\pgfqpoint{5.490039in}{5.490039in}}%
\pgfusepath{clip}%
\pgfsetbuttcap%
\pgfsetroundjoin%
\definecolor{currentfill}{rgb}{0.146180,0.515413,0.556823}%
\pgfsetfillcolor{currentfill}%
\pgfsetfillopacity{0.700000}%
\pgfsetlinewidth{0.000000pt}%
\definecolor{currentstroke}{rgb}{0.000000,0.000000,0.000000}%
\pgfsetstrokecolor{currentstroke}%
\pgfsetdash{}{0pt}%
\pgfpathmoveto{\pgfqpoint{3.613386in}{3.542587in}}%
\pgfpathlineto{\pgfqpoint{3.626220in}{3.525899in}}%
\pgfpathlineto{\pgfqpoint{3.639050in}{3.509459in}}%
\pgfpathlineto{\pgfqpoint{3.651878in}{3.493266in}}%
\pgfpathlineto{\pgfqpoint{3.664703in}{3.477317in}}%
\pgfpathlineto{\pgfqpoint{3.672145in}{3.497491in}}%
\pgfpathlineto{\pgfqpoint{3.679583in}{3.517950in}}%
\pgfpathlineto{\pgfqpoint{3.687017in}{3.538700in}}%
\pgfpathlineto{\pgfqpoint{3.694447in}{3.559746in}}%
\pgfpathlineto{\pgfqpoint{3.681624in}{3.576174in}}%
\pgfpathlineto{\pgfqpoint{3.668798in}{3.592848in}}%
\pgfpathlineto{\pgfqpoint{3.655969in}{3.609768in}}%
\pgfpathlineto{\pgfqpoint{3.643137in}{3.626938in}}%
\pgfpathlineto{\pgfqpoint{3.635705in}{3.605400in}}%
\pgfpathlineto{\pgfqpoint{3.628269in}{3.584165in}}%
\pgfpathlineto{\pgfqpoint{3.620830in}{3.563230in}}%
\pgfpathlineto{\pgfqpoint{3.613386in}{3.542587in}}%
\pgfpathclose%
\pgfusepath{fill}%
\end{pgfscope}%
\begin{pgfscope}%
\pgfpathrectangle{\pgfqpoint{1.254980in}{0.150000in}}{\pgfqpoint{5.490039in}{5.490039in}}%
\pgfusepath{clip}%
\pgfsetbuttcap%
\pgfsetroundjoin%
\definecolor{currentfill}{rgb}{0.153894,0.680203,0.504172}%
\pgfsetfillcolor{currentfill}%
\pgfsetfillopacity{0.700000}%
\pgfsetlinewidth{0.000000pt}%
\definecolor{currentstroke}{rgb}{0.000000,0.000000,0.000000}%
\pgfsetstrokecolor{currentstroke}%
\pgfsetdash{}{0pt}%
\pgfpathmoveto{\pgfqpoint{3.599672in}{3.966004in}}%
\pgfpathlineto{\pgfqpoint{3.612537in}{3.945694in}}%
\pgfpathlineto{\pgfqpoint{3.625398in}{3.925652in}}%
\pgfpathlineto{\pgfqpoint{3.638253in}{3.905877in}}%
\pgfpathlineto{\pgfqpoint{3.651104in}{3.886365in}}%
\pgfpathlineto{\pgfqpoint{3.658505in}{3.911482in}}%
\pgfpathlineto{\pgfqpoint{3.665903in}{3.936973in}}%
\pgfpathlineto{\pgfqpoint{3.673298in}{3.962847in}}%
\pgfpathlineto{\pgfqpoint{3.680690in}{3.989110in}}%
\pgfpathlineto{\pgfqpoint{3.667837in}{4.009202in}}%
\pgfpathlineto{\pgfqpoint{3.654979in}{4.029559in}}%
\pgfpathlineto{\pgfqpoint{3.642117in}{4.050183in}}%
\pgfpathlineto{\pgfqpoint{3.629249in}{4.071077in}}%
\pgfpathlineto{\pgfqpoint{3.621860in}{4.044220in}}%
\pgfpathlineto{\pgfqpoint{3.614467in}{4.017760in}}%
\pgfpathlineto{\pgfqpoint{3.607071in}{3.991690in}}%
\pgfpathlineto{\pgfqpoint{3.599672in}{3.966004in}}%
\pgfpathclose%
\pgfusepath{fill}%
\end{pgfscope}%
\begin{pgfscope}%
\pgfpathrectangle{\pgfqpoint{1.254980in}{0.150000in}}{\pgfqpoint{5.490039in}{5.490039in}}%
\pgfusepath{clip}%
\pgfsetbuttcap%
\pgfsetroundjoin%
\definecolor{currentfill}{rgb}{0.156270,0.489624,0.557936}%
\pgfsetfillcolor{currentfill}%
\pgfsetfillopacity{0.700000}%
\pgfsetlinewidth{0.000000pt}%
\definecolor{currentstroke}{rgb}{0.000000,0.000000,0.000000}%
\pgfsetstrokecolor{currentstroke}%
\pgfsetdash{}{0pt}%
\pgfpathmoveto{\pgfqpoint{4.486951in}{3.472953in}}%
\pgfpathlineto{\pgfqpoint{4.499806in}{3.463289in}}%
\pgfpathlineto{\pgfqpoint{4.512665in}{3.453806in}}%
\pgfpathlineto{\pgfqpoint{4.525528in}{3.444502in}}%
\pgfpathlineto{\pgfqpoint{4.538396in}{3.435376in}}%
\pgfpathlineto{\pgfqpoint{4.545763in}{3.457299in}}%
\pgfpathlineto{\pgfqpoint{4.553132in}{3.479598in}}%
\pgfpathlineto{\pgfqpoint{4.560504in}{3.502282in}}%
\pgfpathlineto{\pgfqpoint{4.547641in}{3.511902in}}%
\pgfpathlineto{\pgfqpoint{4.534782in}{3.521701in}}%
\pgfpathlineto{\pgfqpoint{4.521927in}{3.531680in}}%
\pgfpathlineto{\pgfqpoint{4.509077in}{3.541840in}}%
\pgfpathlineto{\pgfqpoint{4.501699in}{3.518488in}}%
\pgfpathlineto{\pgfqpoint{4.494324in}{3.495529in}}%
\pgfpathlineto{\pgfqpoint{4.486951in}{3.472953in}}%
\pgfpathclose%
\pgfusepath{fill}%
\end{pgfscope}%
\begin{pgfscope}%
\pgfpathrectangle{\pgfqpoint{1.254980in}{0.150000in}}{\pgfqpoint{5.490039in}{5.490039in}}%
\pgfusepath{clip}%
\pgfsetbuttcap%
\pgfsetroundjoin%
\definecolor{currentfill}{rgb}{0.335885,0.777018,0.402049}%
\pgfsetfillcolor{currentfill}%
\pgfsetfillopacity{0.700000}%
\pgfsetlinewidth{0.000000pt}%
\definecolor{currentstroke}{rgb}{0.000000,0.000000,0.000000}%
\pgfsetstrokecolor{currentstroke}%
\pgfsetdash{}{0pt}%
\pgfpathmoveto{\pgfqpoint{3.820630in}{4.251821in}}%
\pgfpathlineto{\pgfqpoint{3.833473in}{4.231095in}}%
\pgfpathlineto{\pgfqpoint{3.846311in}{4.210624in}}%
\pgfpathlineto{\pgfqpoint{3.859146in}{4.190407in}}%
\pgfpathlineto{\pgfqpoint{3.871977in}{4.170440in}}%
\pgfpathlineto{\pgfqpoint{3.879354in}{4.201053in}}%
\pgfpathlineto{\pgfqpoint{3.886731in}{4.232147in}}%
\pgfpathlineto{\pgfqpoint{3.894108in}{4.263731in}}%
\pgfpathlineto{\pgfqpoint{3.881273in}{4.284225in}}%
\pgfpathlineto{\pgfqpoint{3.868435in}{4.304971in}}%
\pgfpathlineto{\pgfqpoint{3.855592in}{4.325970in}}%
\pgfpathlineto{\pgfqpoint{3.842746in}{4.347226in}}%
\pgfpathlineto{\pgfqpoint{3.835374in}{4.314928in}}%
\pgfpathlineto{\pgfqpoint{3.828003in}{4.283130in}}%
\pgfpathlineto{\pgfqpoint{3.820630in}{4.251821in}}%
\pgfpathclose%
\pgfusepath{fill}%
\end{pgfscope}%
\begin{pgfscope}%
\pgfpathrectangle{\pgfqpoint{1.254980in}{0.150000in}}{\pgfqpoint{5.490039in}{5.490039in}}%
\pgfusepath{clip}%
\pgfsetbuttcap%
\pgfsetroundjoin%
\definecolor{currentfill}{rgb}{0.124780,0.640461,0.527068}%
\pgfsetfillcolor{currentfill}%
\pgfsetfillopacity{0.700000}%
\pgfsetlinewidth{0.000000pt}%
\definecolor{currentstroke}{rgb}{0.000000,0.000000,0.000000}%
\pgfsetstrokecolor{currentstroke}%
\pgfsetdash{}{0pt}%
\pgfpathmoveto{\pgfqpoint{3.570039in}{3.866959in}}%
\pgfpathlineto{\pgfqpoint{3.582904in}{3.847200in}}%
\pgfpathlineto{\pgfqpoint{3.595763in}{3.827708in}}%
\pgfpathlineto{\pgfqpoint{3.608618in}{3.808481in}}%
\pgfpathlineto{\pgfqpoint{3.621468in}{3.789518in}}%
\pgfpathlineto{\pgfqpoint{3.628882in}{3.813200in}}%
\pgfpathlineto{\pgfqpoint{3.636293in}{3.837231in}}%
\pgfpathlineto{\pgfqpoint{3.643700in}{3.861617in}}%
\pgfpathlineto{\pgfqpoint{3.651104in}{3.886365in}}%
\pgfpathlineto{\pgfqpoint{3.638253in}{3.905877in}}%
\pgfpathlineto{\pgfqpoint{3.625398in}{3.925652in}}%
\pgfpathlineto{\pgfqpoint{3.612537in}{3.945694in}}%
\pgfpathlineto{\pgfqpoint{3.599672in}{3.966004in}}%
\pgfpathlineto{\pgfqpoint{3.592269in}{3.940695in}}%
\pgfpathlineto{\pgfqpoint{3.584863in}{3.915755in}}%
\pgfpathlineto{\pgfqpoint{3.577453in}{3.891179in}}%
\pgfpathlineto{\pgfqpoint{3.570039in}{3.866959in}}%
\pgfpathclose%
\pgfusepath{fill}%
\end{pgfscope}%
\begin{pgfscope}%
\pgfpathrectangle{\pgfqpoint{1.254980in}{0.150000in}}{\pgfqpoint{5.490039in}{5.490039in}}%
\pgfusepath{clip}%
\pgfsetbuttcap%
\pgfsetroundjoin%
\definecolor{currentfill}{rgb}{0.183898,0.422383,0.556944}%
\pgfsetfillcolor{currentfill}%
\pgfsetfillopacity{0.700000}%
\pgfsetlinewidth{0.000000pt}%
\definecolor{currentstroke}{rgb}{0.000000,0.000000,0.000000}%
\pgfsetstrokecolor{currentstroke}%
\pgfsetdash{}{0pt}%
\pgfpathmoveto{\pgfqpoint{3.818471in}{3.304309in}}%
\pgfpathlineto{\pgfqpoint{3.831280in}{3.291370in}}%
\pgfpathlineto{\pgfqpoint{3.844088in}{3.278650in}}%
\pgfpathlineto{\pgfqpoint{3.856896in}{3.266149in}}%
\pgfpathlineto{\pgfqpoint{3.869705in}{3.253865in}}%
\pgfpathlineto{\pgfqpoint{3.877134in}{3.272183in}}%
\pgfpathlineto{\pgfqpoint{3.884560in}{3.290757in}}%
\pgfpathlineto{\pgfqpoint{3.891983in}{3.309593in}}%
\pgfpathlineto{\pgfqpoint{3.899403in}{3.328695in}}%
\pgfpathlineto{\pgfqpoint{3.886598in}{3.341451in}}%
\pgfpathlineto{\pgfqpoint{3.873793in}{3.354424in}}%
\pgfpathlineto{\pgfqpoint{3.860988in}{3.367615in}}%
\pgfpathlineto{\pgfqpoint{3.848183in}{3.381027in}}%
\pgfpathlineto{\pgfqpoint{3.840760in}{3.361442in}}%
\pgfpathlineto{\pgfqpoint{3.833334in}{3.342131in}}%
\pgfpathlineto{\pgfqpoint{3.825904in}{3.323089in}}%
\pgfpathlineto{\pgfqpoint{3.818471in}{3.304309in}}%
\pgfpathclose%
\pgfusepath{fill}%
\end{pgfscope}%
\begin{pgfscope}%
\pgfpathrectangle{\pgfqpoint{1.254980in}{0.150000in}}{\pgfqpoint{5.490039in}{5.490039in}}%
\pgfusepath{clip}%
\pgfsetbuttcap%
\pgfsetroundjoin%
\definecolor{currentfill}{rgb}{0.208030,0.718701,0.472873}%
\pgfsetfillcolor{currentfill}%
\pgfsetfillopacity{0.700000}%
\pgfsetlinewidth{0.000000pt}%
\definecolor{currentstroke}{rgb}{0.000000,0.000000,0.000000}%
\pgfsetstrokecolor{currentstroke}%
\pgfsetdash{}{0pt}%
\pgfpathmoveto{\pgfqpoint{3.629249in}{4.071077in}}%
\pgfpathlineto{\pgfqpoint{3.642117in}{4.050183in}}%
\pgfpathlineto{\pgfqpoint{3.654979in}{4.029559in}}%
\pgfpathlineto{\pgfqpoint{3.667837in}{4.009202in}}%
\pgfpathlineto{\pgfqpoint{3.680690in}{3.989110in}}%
\pgfpathlineto{\pgfqpoint{3.688079in}{4.015769in}}%
\pgfpathlineto{\pgfqpoint{3.695466in}{4.042832in}}%
\pgfpathlineto{\pgfqpoint{3.702850in}{4.070306in}}%
\pgfpathlineto{\pgfqpoint{3.710232in}{4.098198in}}%
\pgfpathlineto{\pgfqpoint{3.697376in}{4.118903in}}%
\pgfpathlineto{\pgfqpoint{3.684515in}{4.139874in}}%
\pgfpathlineto{\pgfqpoint{3.671650in}{4.161115in}}%
\pgfpathlineto{\pgfqpoint{3.658779in}{4.182626in}}%
\pgfpathlineto{\pgfqpoint{3.651401in}{4.154106in}}%
\pgfpathlineto{\pgfqpoint{3.644020in}{4.126013in}}%
\pgfpathlineto{\pgfqpoint{3.636636in}{4.098339in}}%
\pgfpathlineto{\pgfqpoint{3.629249in}{4.071077in}}%
\pgfpathclose%
\pgfusepath{fill}%
\end{pgfscope}%
\begin{pgfscope}%
\pgfpathrectangle{\pgfqpoint{1.254980in}{0.150000in}}{\pgfqpoint{5.490039in}{5.490039in}}%
\pgfusepath{clip}%
\pgfsetbuttcap%
\pgfsetroundjoin%
\definecolor{currentfill}{rgb}{0.319809,0.770914,0.411152}%
\pgfsetfillcolor{currentfill}%
\pgfsetfillopacity{0.700000}%
\pgfsetlinewidth{0.000000pt}%
\definecolor{currentstroke}{rgb}{0.000000,0.000000,0.000000}%
\pgfsetstrokecolor{currentstroke}%
\pgfsetdash{}{0pt}%
\pgfpathmoveto{\pgfqpoint{3.739738in}{4.214100in}}%
\pgfpathlineto{\pgfqpoint{3.752593in}{4.193013in}}%
\pgfpathlineto{\pgfqpoint{3.765443in}{4.172188in}}%
\pgfpathlineto{\pgfqpoint{3.778290in}{4.151624in}}%
\pgfpathlineto{\pgfqpoint{3.791131in}{4.131318in}}%
\pgfpathlineto{\pgfqpoint{3.798508in}{4.160750in}}%
\pgfpathlineto{\pgfqpoint{3.805883in}{4.190640in}}%
\pgfpathlineto{\pgfqpoint{3.813257in}{4.220994in}}%
\pgfpathlineto{\pgfqpoint{3.820630in}{4.251821in}}%
\pgfpathlineto{\pgfqpoint{3.807784in}{4.272804in}}%
\pgfpathlineto{\pgfqpoint{3.794933in}{4.294047in}}%
\pgfpathlineto{\pgfqpoint{3.782077in}{4.315551in}}%
\pgfpathlineto{\pgfqpoint{3.769217in}{4.337320in}}%
\pgfpathlineto{\pgfqpoint{3.761849in}{4.305801in}}%
\pgfpathlineto{\pgfqpoint{3.754480in}{4.274763in}}%
\pgfpathlineto{\pgfqpoint{3.747110in}{4.244199in}}%
\pgfpathlineto{\pgfqpoint{3.739738in}{4.214100in}}%
\pgfpathclose%
\pgfusepath{fill}%
\end{pgfscope}%
\begin{pgfscope}%
\pgfpathrectangle{\pgfqpoint{1.254980in}{0.150000in}}{\pgfqpoint{5.490039in}{5.490039in}}%
\pgfusepath{clip}%
\pgfsetbuttcap%
\pgfsetroundjoin%
\definecolor{currentfill}{rgb}{0.136408,0.541173,0.554483}%
\pgfsetfillcolor{currentfill}%
\pgfsetfillopacity{0.700000}%
\pgfsetlinewidth{0.000000pt}%
\definecolor{currentstroke}{rgb}{0.000000,0.000000,0.000000}%
\pgfsetstrokecolor{currentstroke}%
\pgfsetdash{}{0pt}%
\pgfpathmoveto{\pgfqpoint{3.562017in}{3.611865in}}%
\pgfpathlineto{\pgfqpoint{3.574865in}{3.594162in}}%
\pgfpathlineto{\pgfqpoint{3.587709in}{3.576716in}}%
\pgfpathlineto{\pgfqpoint{3.600549in}{3.559525in}}%
\pgfpathlineto{\pgfqpoint{3.613386in}{3.542587in}}%
\pgfpathlineto{\pgfqpoint{3.620830in}{3.563230in}}%
\pgfpathlineto{\pgfqpoint{3.628269in}{3.584165in}}%
\pgfpathlineto{\pgfqpoint{3.635705in}{3.605400in}}%
\pgfpathlineto{\pgfqpoint{3.643137in}{3.626938in}}%
\pgfpathlineto{\pgfqpoint{3.630302in}{3.644358in}}%
\pgfpathlineto{\pgfqpoint{3.617463in}{3.662032in}}%
\pgfpathlineto{\pgfqpoint{3.604621in}{3.679962in}}%
\pgfpathlineto{\pgfqpoint{3.591774in}{3.698149in}}%
\pgfpathlineto{\pgfqpoint{3.584341in}{3.676116in}}%
\pgfpathlineto{\pgfqpoint{3.576904in}{3.654395in}}%
\pgfpathlineto{\pgfqpoint{3.569463in}{3.632980in}}%
\pgfpathlineto{\pgfqpoint{3.562017in}{3.611865in}}%
\pgfpathclose%
\pgfusepath{fill}%
\end{pgfscope}%
\begin{pgfscope}%
\pgfpathrectangle{\pgfqpoint{1.254980in}{0.150000in}}{\pgfqpoint{5.490039in}{5.490039in}}%
\pgfusepath{clip}%
\pgfsetbuttcap%
\pgfsetroundjoin%
\definecolor{currentfill}{rgb}{0.119512,0.607464,0.540218}%
\pgfsetfillcolor{currentfill}%
\pgfsetfillopacity{0.700000}%
\pgfsetlinewidth{0.000000pt}%
\definecolor{currentstroke}{rgb}{0.000000,0.000000,0.000000}%
\pgfsetstrokecolor{currentstroke}%
\pgfsetdash{}{0pt}%
\pgfpathmoveto{\pgfqpoint{3.540345in}{3.773520in}}%
\pgfpathlineto{\pgfqpoint{3.553210in}{3.754280in}}%
\pgfpathlineto{\pgfqpoint{3.566069in}{3.735306in}}%
\pgfpathlineto{\pgfqpoint{3.578924in}{3.716596in}}%
\pgfpathlineto{\pgfqpoint{3.591774in}{3.698149in}}%
\pgfpathlineto{\pgfqpoint{3.599203in}{3.720500in}}%
\pgfpathlineto{\pgfqpoint{3.606629in}{3.743174in}}%
\pgfpathlineto{\pgfqpoint{3.614050in}{3.766178in}}%
\pgfpathlineto{\pgfqpoint{3.621468in}{3.789518in}}%
\pgfpathlineto{\pgfqpoint{3.608618in}{3.808481in}}%
\pgfpathlineto{\pgfqpoint{3.595763in}{3.827708in}}%
\pgfpathlineto{\pgfqpoint{3.582904in}{3.847200in}}%
\pgfpathlineto{\pgfqpoint{3.570039in}{3.866959in}}%
\pgfpathlineto{\pgfqpoint{3.562622in}{3.843090in}}%
\pgfpathlineto{\pgfqpoint{3.555200in}{3.819564in}}%
\pgfpathlineto{\pgfqpoint{3.547775in}{3.796377in}}%
\pgfpathlineto{\pgfqpoint{3.540345in}{3.773520in}}%
\pgfpathclose%
\pgfusepath{fill}%
\end{pgfscope}%
\begin{pgfscope}%
\pgfpathrectangle{\pgfqpoint{1.254980in}{0.150000in}}{\pgfqpoint{5.490039in}{5.490039in}}%
\pgfusepath{clip}%
\pgfsetbuttcap%
\pgfsetroundjoin%
\definecolor{currentfill}{rgb}{0.182256,0.426184,0.557120}%
\pgfsetfillcolor{currentfill}%
\pgfsetfillopacity{0.700000}%
\pgfsetlinewidth{0.000000pt}%
\definecolor{currentstroke}{rgb}{0.000000,0.000000,0.000000}%
\pgfsetstrokecolor{currentstroke}%
\pgfsetdash{}{0pt}%
\pgfpathmoveto{\pgfqpoint{4.295739in}{3.298571in}}%
\pgfpathlineto{\pgfqpoint{4.308581in}{3.289258in}}%
\pgfpathlineto{\pgfqpoint{4.321426in}{3.280131in}}%
\pgfpathlineto{\pgfqpoint{4.334275in}{3.271189in}}%
\pgfpathlineto{\pgfqpoint{4.347128in}{3.262433in}}%
\pgfpathlineto{\pgfqpoint{4.354495in}{3.281552in}}%
\pgfpathlineto{\pgfqpoint{4.361863in}{3.300974in}}%
\pgfpathlineto{\pgfqpoint{4.369230in}{3.320707in}}%
\pgfpathlineto{\pgfqpoint{4.376598in}{3.340756in}}%
\pgfpathlineto{\pgfqpoint{4.363751in}{3.350091in}}%
\pgfpathlineto{\pgfqpoint{4.350908in}{3.359610in}}%
\pgfpathlineto{\pgfqpoint{4.338068in}{3.369316in}}%
\pgfpathlineto{\pgfqpoint{4.325232in}{3.379208in}}%
\pgfpathlineto{\pgfqpoint{4.317859in}{3.358570in}}%
\pgfpathlineto{\pgfqpoint{4.310486in}{3.338255in}}%
\pgfpathlineto{\pgfqpoint{4.303113in}{3.318258in}}%
\pgfpathlineto{\pgfqpoint{4.295739in}{3.298571in}}%
\pgfpathclose%
\pgfusepath{fill}%
\end{pgfscope}%
\begin{pgfscope}%
\pgfpathrectangle{\pgfqpoint{1.254980in}{0.150000in}}{\pgfqpoint{5.490039in}{5.490039in}}%
\pgfusepath{clip}%
\pgfsetbuttcap%
\pgfsetroundjoin%
\definecolor{currentfill}{rgb}{0.174274,0.445044,0.557792}%
\pgfsetfillcolor{currentfill}%
\pgfsetfillopacity{0.700000}%
\pgfsetlinewidth{0.000000pt}%
\definecolor{currentstroke}{rgb}{0.000000,0.000000,0.000000}%
\pgfsetstrokecolor{currentstroke}%
\pgfsetdash{}{0pt}%
\pgfpathmoveto{\pgfqpoint{4.376598in}{3.340756in}}%
\pgfpathlineto{\pgfqpoint{4.389449in}{3.331606in}}%
\pgfpathlineto{\pgfqpoint{4.402303in}{3.322638in}}%
\pgfpathlineto{\pgfqpoint{4.415162in}{3.313853in}}%
\pgfpathlineto{\pgfqpoint{4.428025in}{3.305249in}}%
\pgfpathlineto{\pgfqpoint{4.435387in}{3.325028in}}%
\pgfpathlineto{\pgfqpoint{4.442750in}{3.345130in}}%
\pgfpathlineto{\pgfqpoint{4.450113in}{3.365563in}}%
\pgfpathlineto{\pgfqpoint{4.457478in}{3.386335in}}%
\pgfpathlineto{\pgfqpoint{4.444621in}{3.395544in}}%
\pgfpathlineto{\pgfqpoint{4.431768in}{3.404936in}}%
\pgfpathlineto{\pgfqpoint{4.418919in}{3.414509in}}%
\pgfpathlineto{\pgfqpoint{4.406074in}{3.424267in}}%
\pgfpathlineto{\pgfqpoint{4.398704in}{3.402878in}}%
\pgfpathlineto{\pgfqpoint{4.391335in}{3.381835in}}%
\pgfpathlineto{\pgfqpoint{4.383966in}{3.361130in}}%
\pgfpathlineto{\pgfqpoint{4.376598in}{3.340756in}}%
\pgfpathclose%
\pgfusepath{fill}%
\end{pgfscope}%
\begin{pgfscope}%
\pgfpathrectangle{\pgfqpoint{1.254980in}{0.150000in}}{\pgfqpoint{5.490039in}{5.490039in}}%
\pgfusepath{clip}%
\pgfsetbuttcap%
\pgfsetroundjoin%
\definecolor{currentfill}{rgb}{0.194100,0.399323,0.555565}%
\pgfsetfillcolor{currentfill}%
\pgfsetfillopacity{0.700000}%
\pgfsetlinewidth{0.000000pt}%
\definecolor{currentstroke}{rgb}{0.000000,0.000000,0.000000}%
\pgfsetstrokecolor{currentstroke}%
\pgfsetdash{}{0pt}%
\pgfpathmoveto{\pgfqpoint{4.001868in}{3.234291in}}%
\pgfpathlineto{\pgfqpoint{4.014682in}{3.223425in}}%
\pgfpathlineto{\pgfqpoint{4.027497in}{3.212763in}}%
\pgfpathlineto{\pgfqpoint{4.040314in}{3.202303in}}%
\pgfpathlineto{\pgfqpoint{4.053134in}{3.192043in}}%
\pgfpathlineto{\pgfqpoint{4.060539in}{3.209956in}}%
\pgfpathlineto{\pgfqpoint{4.067942in}{3.228125in}}%
\pgfpathlineto{\pgfqpoint{4.075343in}{3.246557in}}%
\pgfpathlineto{\pgfqpoint{4.082741in}{3.265258in}}%
\pgfpathlineto{\pgfqpoint{4.069926in}{3.276013in}}%
\pgfpathlineto{\pgfqpoint{4.057113in}{3.286970in}}%
\pgfpathlineto{\pgfqpoint{4.044302in}{3.298129in}}%
\pgfpathlineto{\pgfqpoint{4.031493in}{3.309492in}}%
\pgfpathlineto{\pgfqpoint{4.024090in}{3.290284in}}%
\pgfpathlineto{\pgfqpoint{4.016685in}{3.271352in}}%
\pgfpathlineto{\pgfqpoint{4.009277in}{3.252690in}}%
\pgfpathlineto{\pgfqpoint{4.001868in}{3.234291in}}%
\pgfpathclose%
\pgfusepath{fill}%
\end{pgfscope}%
\begin{pgfscope}%
\pgfpathrectangle{\pgfqpoint{1.254980in}{0.150000in}}{\pgfqpoint{5.490039in}{5.490039in}}%
\pgfusepath{clip}%
\pgfsetbuttcap%
\pgfsetroundjoin%
\definecolor{currentfill}{rgb}{0.192357,0.403199,0.555836}%
\pgfsetfillcolor{currentfill}%
\pgfsetfillopacity{0.700000}%
\pgfsetlinewidth{0.000000pt}%
\definecolor{currentstroke}{rgb}{0.000000,0.000000,0.000000}%
\pgfsetstrokecolor{currentstroke}%
\pgfsetdash{}{0pt}%
\pgfpathmoveto{\pgfqpoint{3.869705in}{3.253865in}}%
\pgfpathlineto{\pgfqpoint{3.882514in}{3.241796in}}%
\pgfpathlineto{\pgfqpoint{3.895323in}{3.229941in}}%
\pgfpathlineto{\pgfqpoint{3.908134in}{3.218298in}}%
\pgfpathlineto{\pgfqpoint{3.920945in}{3.206866in}}%
\pgfpathlineto{\pgfqpoint{3.928370in}{3.224724in}}%
\pgfpathlineto{\pgfqpoint{3.935792in}{3.242831in}}%
\pgfpathlineto{\pgfqpoint{3.943211in}{3.261192in}}%
\pgfpathlineto{\pgfqpoint{3.950627in}{3.279813in}}%
\pgfpathlineto{\pgfqpoint{3.937820in}{3.291715in}}%
\pgfpathlineto{\pgfqpoint{3.925014in}{3.303828in}}%
\pgfpathlineto{\pgfqpoint{3.912208in}{3.316154in}}%
\pgfpathlineto{\pgfqpoint{3.899403in}{3.328695in}}%
\pgfpathlineto{\pgfqpoint{3.891983in}{3.309593in}}%
\pgfpathlineto{\pgfqpoint{3.884560in}{3.290757in}}%
\pgfpathlineto{\pgfqpoint{3.877134in}{3.272183in}}%
\pgfpathlineto{\pgfqpoint{3.869705in}{3.253865in}}%
\pgfpathclose%
\pgfusepath{fill}%
\end{pgfscope}%
\begin{pgfscope}%
\pgfpathrectangle{\pgfqpoint{1.254980in}{0.150000in}}{\pgfqpoint{5.490039in}{5.490039in}}%
\pgfusepath{clip}%
\pgfsetbuttcap%
\pgfsetroundjoin%
\definecolor{currentfill}{rgb}{0.188923,0.410910,0.556326}%
\pgfsetfillcolor{currentfill}%
\pgfsetfillopacity{0.700000}%
\pgfsetlinewidth{0.000000pt}%
\definecolor{currentstroke}{rgb}{0.000000,0.000000,0.000000}%
\pgfsetstrokecolor{currentstroke}%
\pgfsetdash{}{0pt}%
\pgfpathmoveto{\pgfqpoint{4.214886in}{3.259734in}}%
\pgfpathlineto{\pgfqpoint{4.227720in}{3.250215in}}%
\pgfpathlineto{\pgfqpoint{4.240556in}{3.240887in}}%
\pgfpathlineto{\pgfqpoint{4.253396in}{3.231748in}}%
\pgfpathlineto{\pgfqpoint{4.266240in}{3.222797in}}%
\pgfpathlineto{\pgfqpoint{4.273616in}{3.241308in}}%
\pgfpathlineto{\pgfqpoint{4.280991in}{3.260103in}}%
\pgfpathlineto{\pgfqpoint{4.288365in}{3.279188in}}%
\pgfpathlineto{\pgfqpoint{4.295739in}{3.298571in}}%
\pgfpathlineto{\pgfqpoint{4.282901in}{3.308072in}}%
\pgfpathlineto{\pgfqpoint{4.270067in}{3.317762in}}%
\pgfpathlineto{\pgfqpoint{4.257235in}{3.327641in}}%
\pgfpathlineto{\pgfqpoint{4.244407in}{3.337711in}}%
\pgfpathlineto{\pgfqpoint{4.237028in}{3.317767in}}%
\pgfpathlineto{\pgfqpoint{4.229648in}{3.298127in}}%
\pgfpathlineto{\pgfqpoint{4.222268in}{3.278785in}}%
\pgfpathlineto{\pgfqpoint{4.214886in}{3.259734in}}%
\pgfpathclose%
\pgfusepath{fill}%
\end{pgfscope}%
\begin{pgfscope}%
\pgfpathrectangle{\pgfqpoint{1.254980in}{0.150000in}}{\pgfqpoint{5.490039in}{5.490039in}}%
\pgfusepath{clip}%
\pgfsetbuttcap%
\pgfsetroundjoin%
\definecolor{currentfill}{rgb}{0.166617,0.463708,0.558119}%
\pgfsetfillcolor{currentfill}%
\pgfsetfillopacity{0.700000}%
\pgfsetlinewidth{0.000000pt}%
\definecolor{currentstroke}{rgb}{0.000000,0.000000,0.000000}%
\pgfsetstrokecolor{currentstroke}%
\pgfsetdash{}{0pt}%
\pgfpathmoveto{\pgfqpoint{4.457478in}{3.386335in}}%
\pgfpathlineto{\pgfqpoint{4.470339in}{3.377306in}}%
\pgfpathlineto{\pgfqpoint{4.483204in}{3.368457in}}%
\pgfpathlineto{\pgfqpoint{4.496074in}{3.359787in}}%
\pgfpathlineto{\pgfqpoint{4.508948in}{3.351295in}}%
\pgfpathlineto{\pgfqpoint{4.516308in}{3.371789in}}%
\pgfpathlineto{\pgfqpoint{4.523669in}{3.392629in}}%
\pgfpathlineto{\pgfqpoint{4.531031in}{3.413822in}}%
\pgfpathlineto{\pgfqpoint{4.538396in}{3.435376in}}%
\pgfpathlineto{\pgfqpoint{4.525528in}{3.444502in}}%
\pgfpathlineto{\pgfqpoint{4.512665in}{3.453806in}}%
\pgfpathlineto{\pgfqpoint{4.499806in}{3.463289in}}%
\pgfpathlineto{\pgfqpoint{4.486951in}{3.472953in}}%
\pgfpathlineto{\pgfqpoint{4.479580in}{3.450753in}}%
\pgfpathlineto{\pgfqpoint{4.472211in}{3.428922in}}%
\pgfpathlineto{\pgfqpoint{4.464844in}{3.407452in}}%
\pgfpathlineto{\pgfqpoint{4.457478in}{3.386335in}}%
\pgfpathclose%
\pgfusepath{fill}%
\end{pgfscope}%
\begin{pgfscope}%
\pgfpathrectangle{\pgfqpoint{1.254980in}{0.150000in}}{\pgfqpoint{5.490039in}{5.490039in}}%
\pgfusepath{clip}%
\pgfsetbuttcap%
\pgfsetroundjoin%
\definecolor{currentfill}{rgb}{0.296479,0.761561,0.424223}%
\pgfsetfillcolor{currentfill}%
\pgfsetfillopacity{0.700000}%
\pgfsetlinewidth{0.000000pt}%
\definecolor{currentstroke}{rgb}{0.000000,0.000000,0.000000}%
\pgfsetstrokecolor{currentstroke}%
\pgfsetdash{}{0pt}%
\pgfpathmoveto{\pgfqpoint{3.658779in}{4.182626in}}%
\pgfpathlineto{\pgfqpoint{3.671650in}{4.161115in}}%
\pgfpathlineto{\pgfqpoint{3.684515in}{4.139874in}}%
\pgfpathlineto{\pgfqpoint{3.697376in}{4.118903in}}%
\pgfpathlineto{\pgfqpoint{3.710232in}{4.098198in}}%
\pgfpathlineto{\pgfqpoint{3.717611in}{4.126515in}}%
\pgfpathlineto{\pgfqpoint{3.724989in}{4.155266in}}%
\pgfpathlineto{\pgfqpoint{3.732364in}{4.184459in}}%
\pgfpathlineto{\pgfqpoint{3.739738in}{4.214100in}}%
\pgfpathlineto{\pgfqpoint{3.726878in}{4.235453in}}%
\pgfpathlineto{\pgfqpoint{3.714013in}{4.257074in}}%
\pgfpathlineto{\pgfqpoint{3.701143in}{4.278964in}}%
\pgfpathlineto{\pgfqpoint{3.688268in}{4.301127in}}%
\pgfpathlineto{\pgfqpoint{3.680899in}{4.270823in}}%
\pgfpathlineto{\pgfqpoint{3.673528in}{4.240977in}}%
\pgfpathlineto{\pgfqpoint{3.666154in}{4.211580in}}%
\pgfpathlineto{\pgfqpoint{3.658779in}{4.182626in}}%
\pgfpathclose%
\pgfusepath{fill}%
\end{pgfscope}%
\begin{pgfscope}%
\pgfpathrectangle{\pgfqpoint{1.254980in}{0.150000in}}{\pgfqpoint{5.490039in}{5.490039in}}%
\pgfusepath{clip}%
\pgfsetbuttcap%
\pgfsetroundjoin%
\definecolor{currentfill}{rgb}{0.168126,0.459988,0.558082}%
\pgfsetfillcolor{currentfill}%
\pgfsetfillopacity{0.700000}%
\pgfsetlinewidth{0.000000pt}%
\definecolor{currentstroke}{rgb}{0.000000,0.000000,0.000000}%
\pgfsetstrokecolor{currentstroke}%
\pgfsetdash{}{0pt}%
\pgfpathmoveto{\pgfqpoint{3.634898in}{3.399362in}}%
\pgfpathlineto{\pgfqpoint{3.647724in}{3.384104in}}%
\pgfpathlineto{\pgfqpoint{3.660547in}{3.369087in}}%
\pgfpathlineto{\pgfqpoint{3.673369in}{3.354307in}}%
\pgfpathlineto{\pgfqpoint{3.686189in}{3.339764in}}%
\pgfpathlineto{\pgfqpoint{3.693643in}{3.358416in}}%
\pgfpathlineto{\pgfqpoint{3.701093in}{3.377323in}}%
\pgfpathlineto{\pgfqpoint{3.708539in}{3.396491in}}%
\pgfpathlineto{\pgfqpoint{3.715982in}{3.415926in}}%
\pgfpathlineto{\pgfqpoint{3.703165in}{3.430917in}}%
\pgfpathlineto{\pgfqpoint{3.690347in}{3.446144in}}%
\pgfpathlineto{\pgfqpoint{3.677526in}{3.461610in}}%
\pgfpathlineto{\pgfqpoint{3.664703in}{3.477317in}}%
\pgfpathlineto{\pgfqpoint{3.657258in}{3.457422in}}%
\pgfpathlineto{\pgfqpoint{3.649809in}{3.437802in}}%
\pgfpathlineto{\pgfqpoint{3.642356in}{3.418450in}}%
\pgfpathlineto{\pgfqpoint{3.634898in}{3.399362in}}%
\pgfpathclose%
\pgfusepath{fill}%
\end{pgfscope}%
\begin{pgfscope}%
\pgfpathrectangle{\pgfqpoint{1.254980in}{0.150000in}}{\pgfqpoint{5.490039in}{5.490039in}}%
\pgfusepath{clip}%
\pgfsetbuttcap%
\pgfsetroundjoin%
\definecolor{currentfill}{rgb}{0.195860,0.395433,0.555276}%
\pgfsetfillcolor{currentfill}%
\pgfsetfillopacity{0.700000}%
\pgfsetlinewidth{0.000000pt}%
\definecolor{currentstroke}{rgb}{0.000000,0.000000,0.000000}%
\pgfsetstrokecolor{currentstroke}%
\pgfsetdash{}{0pt}%
\pgfpathmoveto{\pgfqpoint{4.134023in}{3.224224in}}%
\pgfpathlineto{\pgfqpoint{4.146850in}{3.214457in}}%
\pgfpathlineto{\pgfqpoint{4.159679in}{3.204884in}}%
\pgfpathlineto{\pgfqpoint{4.172512in}{3.195504in}}%
\pgfpathlineto{\pgfqpoint{4.185348in}{3.186317in}}%
\pgfpathlineto{\pgfqpoint{4.192734in}{3.204266in}}%
\pgfpathlineto{\pgfqpoint{4.200120in}{3.222481in}}%
\pgfpathlineto{\pgfqpoint{4.207504in}{3.240968in}}%
\pgfpathlineto{\pgfqpoint{4.214886in}{3.259734in}}%
\pgfpathlineto{\pgfqpoint{4.202056in}{3.269444in}}%
\pgfpathlineto{\pgfqpoint{4.189229in}{3.279347in}}%
\pgfpathlineto{\pgfqpoint{4.176404in}{3.289443in}}%
\pgfpathlineto{\pgfqpoint{4.163582in}{3.299734in}}%
\pgfpathlineto{\pgfqpoint{4.156195in}{3.280434in}}%
\pgfpathlineto{\pgfqpoint{4.148806in}{3.261420in}}%
\pgfpathlineto{\pgfqpoint{4.141415in}{3.242685in}}%
\pgfpathlineto{\pgfqpoint{4.134023in}{3.224224in}}%
\pgfpathclose%
\pgfusepath{fill}%
\end{pgfscope}%
\begin{pgfscope}%
\pgfpathrectangle{\pgfqpoint{1.254980in}{0.150000in}}{\pgfqpoint{5.490039in}{5.490039in}}%
\pgfusepath{clip}%
\pgfsetbuttcap%
\pgfsetroundjoin%
\definecolor{currentfill}{rgb}{0.177423,0.437527,0.557565}%
\pgfsetfillcolor{currentfill}%
\pgfsetfillopacity{0.700000}%
\pgfsetlinewidth{0.000000pt}%
\definecolor{currentstroke}{rgb}{0.000000,0.000000,0.000000}%
\pgfsetstrokecolor{currentstroke}%
\pgfsetdash{}{0pt}%
\pgfpathmoveto{\pgfqpoint{3.686189in}{3.339764in}}%
\pgfpathlineto{\pgfqpoint{3.699007in}{3.325455in}}%
\pgfpathlineto{\pgfqpoint{3.711824in}{3.311379in}}%
\pgfpathlineto{\pgfqpoint{3.724640in}{3.297534in}}%
\pgfpathlineto{\pgfqpoint{3.737454in}{3.283918in}}%
\pgfpathlineto{\pgfqpoint{3.744905in}{3.302134in}}%
\pgfpathlineto{\pgfqpoint{3.752351in}{3.320598in}}%
\pgfpathlineto{\pgfqpoint{3.759795in}{3.339317in}}%
\pgfpathlineto{\pgfqpoint{3.767234in}{3.358295in}}%
\pgfpathlineto{\pgfqpoint{3.754423in}{3.372356in}}%
\pgfpathlineto{\pgfqpoint{3.741611in}{3.386648in}}%
\pgfpathlineto{\pgfqpoint{3.728797in}{3.401171in}}%
\pgfpathlineto{\pgfqpoint{3.715982in}{3.415926in}}%
\pgfpathlineto{\pgfqpoint{3.708539in}{3.396491in}}%
\pgfpathlineto{\pgfqpoint{3.701093in}{3.377323in}}%
\pgfpathlineto{\pgfqpoint{3.693643in}{3.358416in}}%
\pgfpathlineto{\pgfqpoint{3.686189in}{3.339764in}}%
\pgfpathclose%
\pgfusepath{fill}%
\end{pgfscope}%
\begin{pgfscope}%
\pgfpathrectangle{\pgfqpoint{1.254980in}{0.150000in}}{\pgfqpoint{5.490039in}{5.490039in}}%
\pgfusepath{clip}%
\pgfsetbuttcap%
\pgfsetroundjoin%
\definecolor{currentfill}{rgb}{0.157729,0.485932,0.558013}%
\pgfsetfillcolor{currentfill}%
\pgfsetfillopacity{0.700000}%
\pgfsetlinewidth{0.000000pt}%
\definecolor{currentstroke}{rgb}{0.000000,0.000000,0.000000}%
\pgfsetstrokecolor{currentstroke}%
\pgfsetdash{}{0pt}%
\pgfpathmoveto{\pgfqpoint{3.583570in}{3.462830in}}%
\pgfpathlineto{\pgfqpoint{3.596406in}{3.446593in}}%
\pgfpathlineto{\pgfqpoint{3.609240in}{3.430604in}}%
\pgfpathlineto{\pgfqpoint{3.622070in}{3.414861in}}%
\pgfpathlineto{\pgfqpoint{3.634898in}{3.399362in}}%
\pgfpathlineto{\pgfqpoint{3.642356in}{3.418450in}}%
\pgfpathlineto{\pgfqpoint{3.649809in}{3.437802in}}%
\pgfpathlineto{\pgfqpoint{3.657258in}{3.457422in}}%
\pgfpathlineto{\pgfqpoint{3.664703in}{3.477317in}}%
\pgfpathlineto{\pgfqpoint{3.651878in}{3.493266in}}%
\pgfpathlineto{\pgfqpoint{3.639050in}{3.509459in}}%
\pgfpathlineto{\pgfqpoint{3.626220in}{3.525899in}}%
\pgfpathlineto{\pgfqpoint{3.613386in}{3.542587in}}%
\pgfpathlineto{\pgfqpoint{3.605938in}{3.522231in}}%
\pgfpathlineto{\pgfqpoint{3.598487in}{3.502156in}}%
\pgfpathlineto{\pgfqpoint{3.591031in}{3.482358in}}%
\pgfpathlineto{\pgfqpoint{3.583570in}{3.462830in}}%
\pgfpathclose%
\pgfusepath{fill}%
\end{pgfscope}%
\begin{pgfscope}%
\pgfpathrectangle{\pgfqpoint{1.254980in}{0.150000in}}{\pgfqpoint{5.490039in}{5.490039in}}%
\pgfusepath{clip}%
\pgfsetbuttcap%
\pgfsetroundjoin%
\definecolor{currentfill}{rgb}{0.126453,0.570633,0.549841}%
\pgfsetfillcolor{currentfill}%
\pgfsetfillopacity{0.700000}%
\pgfsetlinewidth{0.000000pt}%
\definecolor{currentstroke}{rgb}{0.000000,0.000000,0.000000}%
\pgfsetstrokecolor{currentstroke}%
\pgfsetdash{}{0pt}%
\pgfpathmoveto{\pgfqpoint{3.510584in}{3.685292in}}%
\pgfpathlineto{\pgfqpoint{3.523449in}{3.666538in}}%
\pgfpathlineto{\pgfqpoint{3.536310in}{3.648051in}}%
\pgfpathlineto{\pgfqpoint{3.549166in}{3.629827in}}%
\pgfpathlineto{\pgfqpoint{3.562017in}{3.611865in}}%
\pgfpathlineto{\pgfqpoint{3.569463in}{3.632980in}}%
\pgfpathlineto{\pgfqpoint{3.576904in}{3.654395in}}%
\pgfpathlineto{\pgfqpoint{3.584341in}{3.676116in}}%
\pgfpathlineto{\pgfqpoint{3.591774in}{3.698149in}}%
\pgfpathlineto{\pgfqpoint{3.578924in}{3.716596in}}%
\pgfpathlineto{\pgfqpoint{3.566069in}{3.735306in}}%
\pgfpathlineto{\pgfqpoint{3.553210in}{3.754280in}}%
\pgfpathlineto{\pgfqpoint{3.540345in}{3.773520in}}%
\pgfpathlineto{\pgfqpoint{3.532912in}{3.750990in}}%
\pgfpathlineto{\pgfqpoint{3.525474in}{3.728779in}}%
\pgfpathlineto{\pgfqpoint{3.518031in}{3.706881in}}%
\pgfpathlineto{\pgfqpoint{3.510584in}{3.685292in}}%
\pgfpathclose%
\pgfusepath{fill}%
\end{pgfscope}%
\begin{pgfscope}%
\pgfpathrectangle{\pgfqpoint{1.254980in}{0.150000in}}{\pgfqpoint{5.490039in}{5.490039in}}%
\pgfusepath{clip}%
\pgfsetbuttcap%
\pgfsetroundjoin%
\definecolor{currentfill}{rgb}{0.185556,0.418570,0.556753}%
\pgfsetfillcolor{currentfill}%
\pgfsetfillopacity{0.700000}%
\pgfsetlinewidth{0.000000pt}%
\definecolor{currentstroke}{rgb}{0.000000,0.000000,0.000000}%
\pgfsetstrokecolor{currentstroke}%
\pgfsetdash{}{0pt}%
\pgfpathmoveto{\pgfqpoint{3.737454in}{3.283918in}}%
\pgfpathlineto{\pgfqpoint{3.750268in}{3.270529in}}%
\pgfpathlineto{\pgfqpoint{3.763081in}{3.257366in}}%
\pgfpathlineto{\pgfqpoint{3.775894in}{3.244427in}}%
\pgfpathlineto{\pgfqpoint{3.788706in}{3.231710in}}%
\pgfpathlineto{\pgfqpoint{3.796153in}{3.249493in}}%
\pgfpathlineto{\pgfqpoint{3.803596in}{3.267516in}}%
\pgfpathlineto{\pgfqpoint{3.811035in}{3.285787in}}%
\pgfpathlineto{\pgfqpoint{3.818471in}{3.304309in}}%
\pgfpathlineto{\pgfqpoint{3.805663in}{3.317469in}}%
\pgfpathlineto{\pgfqpoint{3.792854in}{3.330853in}}%
\pgfpathlineto{\pgfqpoint{3.780044in}{3.344460in}}%
\pgfpathlineto{\pgfqpoint{3.767234in}{3.358295in}}%
\pgfpathlineto{\pgfqpoint{3.759795in}{3.339317in}}%
\pgfpathlineto{\pgfqpoint{3.752351in}{3.320598in}}%
\pgfpathlineto{\pgfqpoint{3.744905in}{3.302134in}}%
\pgfpathlineto{\pgfqpoint{3.737454in}{3.283918in}}%
\pgfpathclose%
\pgfusepath{fill}%
\end{pgfscope}%
\begin{pgfscope}%
\pgfpathrectangle{\pgfqpoint{1.254980in}{0.150000in}}{\pgfqpoint{5.490039in}{5.490039in}}%
\pgfusepath{clip}%
\pgfsetbuttcap%
\pgfsetroundjoin%
\definecolor{currentfill}{rgb}{0.160665,0.478540,0.558115}%
\pgfsetfillcolor{currentfill}%
\pgfsetfillopacity{0.700000}%
\pgfsetlinewidth{0.000000pt}%
\definecolor{currentstroke}{rgb}{0.000000,0.000000,0.000000}%
\pgfsetstrokecolor{currentstroke}%
\pgfsetdash{}{0pt}%
\pgfpathmoveto{\pgfqpoint{4.538396in}{3.435376in}}%
\pgfpathlineto{\pgfqpoint{4.551268in}{3.426428in}}%
\pgfpathlineto{\pgfqpoint{4.564145in}{3.417657in}}%
\pgfpathlineto{\pgfqpoint{4.577026in}{3.409063in}}%
\pgfpathlineto{\pgfqpoint{4.589913in}{3.400644in}}%
\pgfpathlineto{\pgfqpoint{4.597273in}{3.421915in}}%
\pgfpathlineto{\pgfqpoint{4.604635in}{3.443555in}}%
\pgfpathlineto{\pgfqpoint{4.612001in}{3.465572in}}%
\pgfpathlineto{\pgfqpoint{4.599120in}{3.474486in}}%
\pgfpathlineto{\pgfqpoint{4.586243in}{3.483574in}}%
\pgfpathlineto{\pgfqpoint{4.573372in}{3.492839in}}%
\pgfpathlineto{\pgfqpoint{4.560504in}{3.502282in}}%
\pgfpathlineto{\pgfqpoint{4.553132in}{3.479598in}}%
\pgfpathlineto{\pgfqpoint{4.545763in}{3.457299in}}%
\pgfpathlineto{\pgfqpoint{4.538396in}{3.435376in}}%
\pgfpathclose%
\pgfusepath{fill}%
\end{pgfscope}%
\begin{pgfscope}%
\pgfpathrectangle{\pgfqpoint{1.254980in}{0.150000in}}{\pgfqpoint{5.490039in}{5.490039in}}%
\pgfusepath{clip}%
\pgfsetbuttcap%
\pgfsetroundjoin%
\definecolor{currentfill}{rgb}{0.150148,0.676631,0.506589}%
\pgfsetfillcolor{currentfill}%
\pgfsetfillopacity{0.700000}%
\pgfsetlinewidth{0.000000pt}%
\definecolor{currentstroke}{rgb}{0.000000,0.000000,0.000000}%
\pgfsetstrokecolor{currentstroke}%
\pgfsetdash{}{0pt}%
\pgfpathmoveto{\pgfqpoint{3.518530in}{3.948726in}}%
\pgfpathlineto{\pgfqpoint{3.531416in}{3.927870in}}%
\pgfpathlineto{\pgfqpoint{3.544296in}{3.907292in}}%
\pgfpathlineto{\pgfqpoint{3.557170in}{3.886989in}}%
\pgfpathlineto{\pgfqpoint{3.570039in}{3.866959in}}%
\pgfpathlineto{\pgfqpoint{3.577453in}{3.891179in}}%
\pgfpathlineto{\pgfqpoint{3.584863in}{3.915755in}}%
\pgfpathlineto{\pgfqpoint{3.592269in}{3.940695in}}%
\pgfpathlineto{\pgfqpoint{3.599672in}{3.966004in}}%
\pgfpathlineto{\pgfqpoint{3.586802in}{3.986586in}}%
\pgfpathlineto{\pgfqpoint{3.573926in}{4.007442in}}%
\pgfpathlineto{\pgfqpoint{3.561044in}{4.028574in}}%
\pgfpathlineto{\pgfqpoint{3.548157in}{4.049985in}}%
\pgfpathlineto{\pgfqpoint{3.540756in}{4.024109in}}%
\pgfpathlineto{\pgfqpoint{3.533351in}{3.998612in}}%
\pgfpathlineto{\pgfqpoint{3.525942in}{3.973487in}}%
\pgfpathlineto{\pgfqpoint{3.518530in}{3.948726in}}%
\pgfpathclose%
\pgfusepath{fill}%
\end{pgfscope}%
\begin{pgfscope}%
\pgfpathrectangle{\pgfqpoint{1.254980in}{0.150000in}}{\pgfqpoint{5.490039in}{5.490039in}}%
\pgfusepath{clip}%
\pgfsetbuttcap%
\pgfsetroundjoin%
\definecolor{currentfill}{rgb}{0.196571,0.711827,0.479221}%
\pgfsetfillcolor{currentfill}%
\pgfsetfillopacity{0.700000}%
\pgfsetlinewidth{0.000000pt}%
\definecolor{currentstroke}{rgb}{0.000000,0.000000,0.000000}%
\pgfsetstrokecolor{currentstroke}%
\pgfsetdash{}{0pt}%
\pgfpathmoveto{\pgfqpoint{3.548157in}{4.049985in}}%
\pgfpathlineto{\pgfqpoint{3.561044in}{4.028574in}}%
\pgfpathlineto{\pgfqpoint{3.573926in}{4.007442in}}%
\pgfpathlineto{\pgfqpoint{3.586802in}{3.986586in}}%
\pgfpathlineto{\pgfqpoint{3.599672in}{3.966004in}}%
\pgfpathlineto{\pgfqpoint{3.607071in}{3.991690in}}%
\pgfpathlineto{\pgfqpoint{3.614467in}{4.017760in}}%
\pgfpathlineto{\pgfqpoint{3.621860in}{4.044220in}}%
\pgfpathlineto{\pgfqpoint{3.629249in}{4.071077in}}%
\pgfpathlineto{\pgfqpoint{3.616377in}{4.092244in}}%
\pgfpathlineto{\pgfqpoint{3.603498in}{4.113685in}}%
\pgfpathlineto{\pgfqpoint{3.590614in}{4.135404in}}%
\pgfpathlineto{\pgfqpoint{3.577724in}{4.157404in}}%
\pgfpathlineto{\pgfqpoint{3.570337in}{4.129947in}}%
\pgfpathlineto{\pgfqpoint{3.562947in}{4.102897in}}%
\pgfpathlineto{\pgfqpoint{3.555554in}{4.076245in}}%
\pgfpathlineto{\pgfqpoint{3.548157in}{4.049985in}}%
\pgfpathclose%
\pgfusepath{fill}%
\end{pgfscope}%
\begin{pgfscope}%
\pgfpathrectangle{\pgfqpoint{1.254980in}{0.150000in}}{\pgfqpoint{5.490039in}{5.490039in}}%
\pgfusepath{clip}%
\pgfsetbuttcap%
\pgfsetroundjoin%
\definecolor{currentfill}{rgb}{0.199430,0.387607,0.554642}%
\pgfsetfillcolor{currentfill}%
\pgfsetfillopacity{0.700000}%
\pgfsetlinewidth{0.000000pt}%
\definecolor{currentstroke}{rgb}{0.000000,0.000000,0.000000}%
\pgfsetstrokecolor{currentstroke}%
\pgfsetdash{}{0pt}%
\pgfpathmoveto{\pgfqpoint{3.920945in}{3.206866in}}%
\pgfpathlineto{\pgfqpoint{3.933758in}{3.195643in}}%
\pgfpathlineto{\pgfqpoint{3.946571in}{3.184629in}}%
\pgfpathlineto{\pgfqpoint{3.959386in}{3.173822in}}%
\pgfpathlineto{\pgfqpoint{3.972203in}{3.163220in}}%
\pgfpathlineto{\pgfqpoint{3.979623in}{3.180620in}}%
\pgfpathlineto{\pgfqpoint{3.987041in}{3.198262in}}%
\pgfpathlineto{\pgfqpoint{3.994456in}{3.216150in}}%
\pgfpathlineto{\pgfqpoint{4.001868in}{3.234291in}}%
\pgfpathlineto{\pgfqpoint{3.989056in}{3.245361in}}%
\pgfpathlineto{\pgfqpoint{3.976245in}{3.256637in}}%
\pgfpathlineto{\pgfqpoint{3.963436in}{3.268121in}}%
\pgfpathlineto{\pgfqpoint{3.950627in}{3.279813in}}%
\pgfpathlineto{\pgfqpoint{3.943211in}{3.261192in}}%
\pgfpathlineto{\pgfqpoint{3.935792in}{3.242831in}}%
\pgfpathlineto{\pgfqpoint{3.928370in}{3.224724in}}%
\pgfpathlineto{\pgfqpoint{3.920945in}{3.206866in}}%
\pgfpathclose%
\pgfusepath{fill}%
\end{pgfscope}%
\begin{pgfscope}%
\pgfpathrectangle{\pgfqpoint{1.254980in}{0.150000in}}{\pgfqpoint{5.490039in}{5.490039in}}%
\pgfusepath{clip}%
\pgfsetbuttcap%
\pgfsetroundjoin%
\definecolor{currentfill}{rgb}{0.421908,0.805774,0.351910}%
\pgfsetfillcolor{currentfill}%
\pgfsetfillopacity{0.700000}%
\pgfsetlinewidth{0.000000pt}%
\definecolor{currentstroke}{rgb}{0.000000,0.000000,0.000000}%
\pgfsetstrokecolor{currentstroke}%
\pgfsetdash{}{0pt}%
\pgfpathmoveto{\pgfqpoint{3.769217in}{4.337320in}}%
\pgfpathlineto{\pgfqpoint{3.782077in}{4.315551in}}%
\pgfpathlineto{\pgfqpoint{3.794933in}{4.294047in}}%
\pgfpathlineto{\pgfqpoint{3.807784in}{4.272804in}}%
\pgfpathlineto{\pgfqpoint{3.820630in}{4.251821in}}%
\pgfpathlineto{\pgfqpoint{3.828003in}{4.283130in}}%
\pgfpathlineto{\pgfqpoint{3.835374in}{4.314928in}}%
\pgfpathlineto{\pgfqpoint{3.842746in}{4.347226in}}%
\pgfpathlineto{\pgfqpoint{3.829895in}{4.368740in}}%
\pgfpathlineto{\pgfqpoint{3.817040in}{4.390515in}}%
\pgfpathlineto{\pgfqpoint{3.804180in}{4.412553in}}%
\pgfpathlineto{\pgfqpoint{3.791315in}{4.434857in}}%
\pgfpathlineto{\pgfqpoint{3.783950in}{4.401840in}}%
\pgfpathlineto{\pgfqpoint{3.776584in}{4.369331in}}%
\pgfpathlineto{\pgfqpoint{3.769217in}{4.337320in}}%
\pgfpathclose%
\pgfusepath{fill}%
\end{pgfscope}%
\begin{pgfscope}%
\pgfpathrectangle{\pgfqpoint{1.254980in}{0.150000in}}{\pgfqpoint{5.490039in}{5.490039in}}%
\pgfusepath{clip}%
\pgfsetbuttcap%
\pgfsetroundjoin%
\definecolor{currentfill}{rgb}{0.201239,0.383670,0.554294}%
\pgfsetfillcolor{currentfill}%
\pgfsetfillopacity{0.700000}%
\pgfsetlinewidth{0.000000pt}%
\definecolor{currentstroke}{rgb}{0.000000,0.000000,0.000000}%
\pgfsetstrokecolor{currentstroke}%
\pgfsetdash{}{0pt}%
\pgfpathmoveto{\pgfqpoint{4.053134in}{3.192043in}}%
\pgfpathlineto{\pgfqpoint{4.065956in}{3.181984in}}%
\pgfpathlineto{\pgfqpoint{4.078780in}{3.172123in}}%
\pgfpathlineto{\pgfqpoint{4.091606in}{3.162459in}}%
\pgfpathlineto{\pgfqpoint{4.104436in}{3.152992in}}%
\pgfpathlineto{\pgfqpoint{4.111836in}{3.170419in}}%
\pgfpathlineto{\pgfqpoint{4.119233in}{3.188097in}}%
\pgfpathlineto{\pgfqpoint{4.126629in}{3.206030in}}%
\pgfpathlineto{\pgfqpoint{4.134023in}{3.224224in}}%
\pgfpathlineto{\pgfqpoint{4.121199in}{3.234186in}}%
\pgfpathlineto{\pgfqpoint{4.108377in}{3.244345in}}%
\pgfpathlineto{\pgfqpoint{4.095558in}{3.254702in}}%
\pgfpathlineto{\pgfqpoint{4.082741in}{3.265258in}}%
\pgfpathlineto{\pgfqpoint{4.075343in}{3.246557in}}%
\pgfpathlineto{\pgfqpoint{4.067942in}{3.228125in}}%
\pgfpathlineto{\pgfqpoint{4.060539in}{3.209956in}}%
\pgfpathlineto{\pgfqpoint{4.053134in}{3.192043in}}%
\pgfpathclose%
\pgfusepath{fill}%
\end{pgfscope}%
\begin{pgfscope}%
\pgfpathrectangle{\pgfqpoint{1.254980in}{0.150000in}}{\pgfqpoint{5.490039in}{5.490039in}}%
\pgfusepath{clip}%
\pgfsetbuttcap%
\pgfsetroundjoin%
\definecolor{currentfill}{rgb}{0.147607,0.511733,0.557049}%
\pgfsetfillcolor{currentfill}%
\pgfsetfillopacity{0.700000}%
\pgfsetlinewidth{0.000000pt}%
\definecolor{currentstroke}{rgb}{0.000000,0.000000,0.000000}%
\pgfsetstrokecolor{currentstroke}%
\pgfsetdash{}{0pt}%
\pgfpathmoveto{\pgfqpoint{3.532192in}{3.530297in}}%
\pgfpathlineto{\pgfqpoint{3.545042in}{3.513048in}}%
\pgfpathlineto{\pgfqpoint{3.557888in}{3.496055in}}%
\pgfpathlineto{\pgfqpoint{3.570731in}{3.479317in}}%
\pgfpathlineto{\pgfqpoint{3.583570in}{3.462830in}}%
\pgfpathlineto{\pgfqpoint{3.591031in}{3.482358in}}%
\pgfpathlineto{\pgfqpoint{3.598487in}{3.502156in}}%
\pgfpathlineto{\pgfqpoint{3.605938in}{3.522231in}}%
\pgfpathlineto{\pgfqpoint{3.613386in}{3.542587in}}%
\pgfpathlineto{\pgfqpoint{3.600549in}{3.559525in}}%
\pgfpathlineto{\pgfqpoint{3.587709in}{3.576716in}}%
\pgfpathlineto{\pgfqpoint{3.574865in}{3.594162in}}%
\pgfpathlineto{\pgfqpoint{3.562017in}{3.611865in}}%
\pgfpathlineto{\pgfqpoint{3.554568in}{3.591045in}}%
\pgfpathlineto{\pgfqpoint{3.547114in}{3.570514in}}%
\pgfpathlineto{\pgfqpoint{3.539655in}{3.550267in}}%
\pgfpathlineto{\pgfqpoint{3.532192in}{3.530297in}}%
\pgfpathclose%
\pgfusepath{fill}%
\end{pgfscope}%
\begin{pgfscope}%
\pgfpathrectangle{\pgfqpoint{1.254980in}{0.150000in}}{\pgfqpoint{5.490039in}{5.490039in}}%
\pgfusepath{clip}%
\pgfsetbuttcap%
\pgfsetroundjoin%
\definecolor{currentfill}{rgb}{0.123444,0.636809,0.528763}%
\pgfsetfillcolor{currentfill}%
\pgfsetfillopacity{0.700000}%
\pgfsetlinewidth{0.000000pt}%
\definecolor{currentstroke}{rgb}{0.000000,0.000000,0.000000}%
\pgfsetstrokecolor{currentstroke}%
\pgfsetdash{}{0pt}%
\pgfpathmoveto{\pgfqpoint{3.488837in}{3.853203in}}%
\pgfpathlineto{\pgfqpoint{3.501722in}{3.832869in}}%
\pgfpathlineto{\pgfqpoint{3.514602in}{3.812813in}}%
\pgfpathlineto{\pgfqpoint{3.527476in}{3.793031in}}%
\pgfpathlineto{\pgfqpoint{3.540345in}{3.773520in}}%
\pgfpathlineto{\pgfqpoint{3.547775in}{3.796377in}}%
\pgfpathlineto{\pgfqpoint{3.555200in}{3.819564in}}%
\pgfpathlineto{\pgfqpoint{3.562622in}{3.843090in}}%
\pgfpathlineto{\pgfqpoint{3.570039in}{3.866959in}}%
\pgfpathlineto{\pgfqpoint{3.557170in}{3.886989in}}%
\pgfpathlineto{\pgfqpoint{3.544296in}{3.907292in}}%
\pgfpathlineto{\pgfqpoint{3.531416in}{3.927870in}}%
\pgfpathlineto{\pgfqpoint{3.518530in}{3.948726in}}%
\pgfpathlineto{\pgfqpoint{3.511113in}{3.924323in}}%
\pgfpathlineto{\pgfqpoint{3.503692in}{3.900273in}}%
\pgfpathlineto{\pgfqpoint{3.496267in}{3.876568in}}%
\pgfpathlineto{\pgfqpoint{3.488837in}{3.853203in}}%
\pgfpathclose%
\pgfusepath{fill}%
\end{pgfscope}%
\begin{pgfscope}%
\pgfpathrectangle{\pgfqpoint{1.254980in}{0.150000in}}{\pgfqpoint{5.490039in}{5.490039in}}%
\pgfusepath{clip}%
\pgfsetbuttcap%
\pgfsetroundjoin%
\definecolor{currentfill}{rgb}{0.274149,0.751988,0.436601}%
\pgfsetfillcolor{currentfill}%
\pgfsetfillopacity{0.700000}%
\pgfsetlinewidth{0.000000pt}%
\definecolor{currentstroke}{rgb}{0.000000,0.000000,0.000000}%
\pgfsetstrokecolor{currentstroke}%
\pgfsetdash{}{0pt}%
\pgfpathmoveto{\pgfqpoint{3.577724in}{4.157404in}}%
\pgfpathlineto{\pgfqpoint{3.590614in}{4.135404in}}%
\pgfpathlineto{\pgfqpoint{3.603498in}{4.113685in}}%
\pgfpathlineto{\pgfqpoint{3.616377in}{4.092244in}}%
\pgfpathlineto{\pgfqpoint{3.629249in}{4.071077in}}%
\pgfpathlineto{\pgfqpoint{3.636636in}{4.098339in}}%
\pgfpathlineto{\pgfqpoint{3.644020in}{4.126013in}}%
\pgfpathlineto{\pgfqpoint{3.651401in}{4.154106in}}%
\pgfpathlineto{\pgfqpoint{3.658779in}{4.182626in}}%
\pgfpathlineto{\pgfqpoint{3.645902in}{4.204411in}}%
\pgfpathlineto{\pgfqpoint{3.633020in}{4.226472in}}%
\pgfpathlineto{\pgfqpoint{3.620132in}{4.248813in}}%
\pgfpathlineto{\pgfqpoint{3.607238in}{4.271434in}}%
\pgfpathlineto{\pgfqpoint{3.599864in}{4.242281in}}%
\pgfpathlineto{\pgfqpoint{3.592487in}{4.213563in}}%
\pgfpathlineto{\pgfqpoint{3.585107in}{4.185273in}}%
\pgfpathlineto{\pgfqpoint{3.577724in}{4.157404in}}%
\pgfpathclose%
\pgfusepath{fill}%
\end{pgfscope}%
\begin{pgfscope}%
\pgfpathrectangle{\pgfqpoint{1.254980in}{0.150000in}}{\pgfqpoint{5.490039in}{5.490039in}}%
\pgfusepath{clip}%
\pgfsetbuttcap%
\pgfsetroundjoin%
\definecolor{currentfill}{rgb}{0.195860,0.395433,0.555276}%
\pgfsetfillcolor{currentfill}%
\pgfsetfillopacity{0.700000}%
\pgfsetlinewidth{0.000000pt}%
\definecolor{currentstroke}{rgb}{0.000000,0.000000,0.000000}%
\pgfsetstrokecolor{currentstroke}%
\pgfsetdash{}{0pt}%
\pgfpathmoveto{\pgfqpoint{3.788706in}{3.231710in}}%
\pgfpathlineto{\pgfqpoint{3.801518in}{3.219215in}}%
\pgfpathlineto{\pgfqpoint{3.814331in}{3.206939in}}%
\pgfpathlineto{\pgfqpoint{3.827143in}{3.194881in}}%
\pgfpathlineto{\pgfqpoint{3.839956in}{3.183039in}}%
\pgfpathlineto{\pgfqpoint{3.847398in}{3.200389in}}%
\pgfpathlineto{\pgfqpoint{3.854837in}{3.217973in}}%
\pgfpathlineto{\pgfqpoint{3.862272in}{3.235797in}}%
\pgfpathlineto{\pgfqpoint{3.869705in}{3.253865in}}%
\pgfpathlineto{\pgfqpoint{3.856896in}{3.266149in}}%
\pgfpathlineto{\pgfqpoint{3.844088in}{3.278650in}}%
\pgfpathlineto{\pgfqpoint{3.831280in}{3.291370in}}%
\pgfpathlineto{\pgfqpoint{3.818471in}{3.304309in}}%
\pgfpathlineto{\pgfqpoint{3.811035in}{3.285787in}}%
\pgfpathlineto{\pgfqpoint{3.803596in}{3.267516in}}%
\pgfpathlineto{\pgfqpoint{3.796153in}{3.249493in}}%
\pgfpathlineto{\pgfqpoint{3.788706in}{3.231710in}}%
\pgfpathclose%
\pgfusepath{fill}%
\end{pgfscope}%
\begin{pgfscope}%
\pgfpathrectangle{\pgfqpoint{1.254980in}{0.150000in}}{\pgfqpoint{5.490039in}{5.490039in}}%
\pgfusepath{clip}%
\pgfsetbuttcap%
\pgfsetroundjoin%
\definecolor{currentfill}{rgb}{0.187231,0.414746,0.556547}%
\pgfsetfillcolor{currentfill}%
\pgfsetfillopacity{0.700000}%
\pgfsetlinewidth{0.000000pt}%
\definecolor{currentstroke}{rgb}{0.000000,0.000000,0.000000}%
\pgfsetstrokecolor{currentstroke}%
\pgfsetdash{}{0pt}%
\pgfpathmoveto{\pgfqpoint{4.347128in}{3.262433in}}%
\pgfpathlineto{\pgfqpoint{4.359984in}{3.253860in}}%
\pgfpathlineto{\pgfqpoint{4.372845in}{3.245470in}}%
\pgfpathlineto{\pgfqpoint{4.385711in}{3.237262in}}%
\pgfpathlineto{\pgfqpoint{4.398580in}{3.229235in}}%
\pgfpathlineto{\pgfqpoint{4.405942in}{3.247788in}}%
\pgfpathlineto{\pgfqpoint{4.413303in}{3.266636in}}%
\pgfpathlineto{\pgfqpoint{4.420664in}{3.285788in}}%
\pgfpathlineto{\pgfqpoint{4.428025in}{3.305249in}}%
\pgfpathlineto{\pgfqpoint{4.415162in}{3.313853in}}%
\pgfpathlineto{\pgfqpoint{4.402303in}{3.322638in}}%
\pgfpathlineto{\pgfqpoint{4.389449in}{3.331606in}}%
\pgfpathlineto{\pgfqpoint{4.376598in}{3.340756in}}%
\pgfpathlineto{\pgfqpoint{4.369230in}{3.320707in}}%
\pgfpathlineto{\pgfqpoint{4.361863in}{3.300974in}}%
\pgfpathlineto{\pgfqpoint{4.354495in}{3.281552in}}%
\pgfpathlineto{\pgfqpoint{4.347128in}{3.262433in}}%
\pgfpathclose%
\pgfusepath{fill}%
\end{pgfscope}%
\begin{pgfscope}%
\pgfpathrectangle{\pgfqpoint{1.254980in}{0.150000in}}{\pgfqpoint{5.490039in}{5.490039in}}%
\pgfusepath{clip}%
\pgfsetbuttcap%
\pgfsetroundjoin%
\definecolor{currentfill}{rgb}{0.404001,0.800275,0.362552}%
\pgfsetfillcolor{currentfill}%
\pgfsetfillopacity{0.700000}%
\pgfsetlinewidth{0.000000pt}%
\definecolor{currentstroke}{rgb}{0.000000,0.000000,0.000000}%
\pgfsetstrokecolor{currentstroke}%
\pgfsetdash{}{0pt}%
\pgfpathmoveto{\pgfqpoint{3.688268in}{4.301127in}}%
\pgfpathlineto{\pgfqpoint{3.701143in}{4.278964in}}%
\pgfpathlineto{\pgfqpoint{3.714013in}{4.257074in}}%
\pgfpathlineto{\pgfqpoint{3.726878in}{4.235453in}}%
\pgfpathlineto{\pgfqpoint{3.739738in}{4.214100in}}%
\pgfpathlineto{\pgfqpoint{3.747110in}{4.244199in}}%
\pgfpathlineto{\pgfqpoint{3.754480in}{4.274763in}}%
\pgfpathlineto{\pgfqpoint{3.761849in}{4.305801in}}%
\pgfpathlineto{\pgfqpoint{3.769217in}{4.337320in}}%
\pgfpathlineto{\pgfqpoint{3.756352in}{4.359356in}}%
\pgfpathlineto{\pgfqpoint{3.743481in}{4.381661in}}%
\pgfpathlineto{\pgfqpoint{3.730606in}{4.404237in}}%
\pgfpathlineto{\pgfqpoint{3.717724in}{4.427087in}}%
\pgfpathlineto{\pgfqpoint{3.710363in}{4.394869in}}%
\pgfpathlineto{\pgfqpoint{3.702999in}{4.363142in}}%
\pgfpathlineto{\pgfqpoint{3.695634in}{4.331898in}}%
\pgfpathlineto{\pgfqpoint{3.688268in}{4.301127in}}%
\pgfpathclose%
\pgfusepath{fill}%
\end{pgfscope}%
\begin{pgfscope}%
\pgfpathrectangle{\pgfqpoint{1.254980in}{0.150000in}}{\pgfqpoint{5.490039in}{5.490039in}}%
\pgfusepath{clip}%
\pgfsetbuttcap%
\pgfsetroundjoin%
\definecolor{currentfill}{rgb}{0.179019,0.433756,0.557430}%
\pgfsetfillcolor{currentfill}%
\pgfsetfillopacity{0.700000}%
\pgfsetlinewidth{0.000000pt}%
\definecolor{currentstroke}{rgb}{0.000000,0.000000,0.000000}%
\pgfsetstrokecolor{currentstroke}%
\pgfsetdash{}{0pt}%
\pgfpathmoveto{\pgfqpoint{4.428025in}{3.305249in}}%
\pgfpathlineto{\pgfqpoint{4.440893in}{3.296826in}}%
\pgfpathlineto{\pgfqpoint{4.453765in}{3.288583in}}%
\pgfpathlineto{\pgfqpoint{4.466641in}{3.280518in}}%
\pgfpathlineto{\pgfqpoint{4.479523in}{3.272631in}}%
\pgfpathlineto{\pgfqpoint{4.486878in}{3.291815in}}%
\pgfpathlineto{\pgfqpoint{4.494234in}{3.311316in}}%
\pgfpathlineto{\pgfqpoint{4.501590in}{3.331140in}}%
\pgfpathlineto{\pgfqpoint{4.508948in}{3.351295in}}%
\pgfpathlineto{\pgfqpoint{4.496074in}{3.359787in}}%
\pgfpathlineto{\pgfqpoint{4.483204in}{3.368457in}}%
\pgfpathlineto{\pgfqpoint{4.470339in}{3.377306in}}%
\pgfpathlineto{\pgfqpoint{4.457478in}{3.386335in}}%
\pgfpathlineto{\pgfqpoint{4.450113in}{3.365563in}}%
\pgfpathlineto{\pgfqpoint{4.442750in}{3.345130in}}%
\pgfpathlineto{\pgfqpoint{4.435387in}{3.325028in}}%
\pgfpathlineto{\pgfqpoint{4.428025in}{3.305249in}}%
\pgfpathclose%
\pgfusepath{fill}%
\end{pgfscope}%
\begin{pgfscope}%
\pgfpathrectangle{\pgfqpoint{1.254980in}{0.150000in}}{\pgfqpoint{5.490039in}{5.490039in}}%
\pgfusepath{clip}%
\pgfsetbuttcap%
\pgfsetroundjoin%
\definecolor{currentfill}{rgb}{0.194100,0.399323,0.555565}%
\pgfsetfillcolor{currentfill}%
\pgfsetfillopacity{0.700000}%
\pgfsetlinewidth{0.000000pt}%
\definecolor{currentstroke}{rgb}{0.000000,0.000000,0.000000}%
\pgfsetstrokecolor{currentstroke}%
\pgfsetdash{}{0pt}%
\pgfpathmoveto{\pgfqpoint{4.266240in}{3.222797in}}%
\pgfpathlineto{\pgfqpoint{4.279087in}{3.214033in}}%
\pgfpathlineto{\pgfqpoint{4.291938in}{3.205456in}}%
\pgfpathlineto{\pgfqpoint{4.304793in}{3.197064in}}%
\pgfpathlineto{\pgfqpoint{4.317652in}{3.188857in}}%
\pgfpathlineto{\pgfqpoint{4.325022in}{3.206829in}}%
\pgfpathlineto{\pgfqpoint{4.332391in}{3.225078in}}%
\pgfpathlineto{\pgfqpoint{4.339760in}{3.243610in}}%
\pgfpathlineto{\pgfqpoint{4.347128in}{3.262433in}}%
\pgfpathlineto{\pgfqpoint{4.334275in}{3.271189in}}%
\pgfpathlineto{\pgfqpoint{4.321426in}{3.280131in}}%
\pgfpathlineto{\pgfqpoint{4.308581in}{3.289258in}}%
\pgfpathlineto{\pgfqpoint{4.295739in}{3.298571in}}%
\pgfpathlineto{\pgfqpoint{4.288365in}{3.279188in}}%
\pgfpathlineto{\pgfqpoint{4.280991in}{3.260103in}}%
\pgfpathlineto{\pgfqpoint{4.273616in}{3.241308in}}%
\pgfpathlineto{\pgfqpoint{4.266240in}{3.222797in}}%
\pgfpathclose%
\pgfusepath{fill}%
\end{pgfscope}%
\begin{pgfscope}%
\pgfpathrectangle{\pgfqpoint{1.254980in}{0.150000in}}{\pgfqpoint{5.490039in}{5.490039in}}%
\pgfusepath{clip}%
\pgfsetbuttcap%
\pgfsetroundjoin%
\definecolor{currentfill}{rgb}{0.171176,0.452530,0.557965}%
\pgfsetfillcolor{currentfill}%
\pgfsetfillopacity{0.700000}%
\pgfsetlinewidth{0.000000pt}%
\definecolor{currentstroke}{rgb}{0.000000,0.000000,0.000000}%
\pgfsetstrokecolor{currentstroke}%
\pgfsetdash{}{0pt}%
\pgfpathmoveto{\pgfqpoint{4.508948in}{3.351295in}}%
\pgfpathlineto{\pgfqpoint{4.521827in}{3.342981in}}%
\pgfpathlineto{\pgfqpoint{4.534711in}{3.334844in}}%
\pgfpathlineto{\pgfqpoint{4.547600in}{3.326882in}}%
\pgfpathlineto{\pgfqpoint{4.560494in}{3.319096in}}%
\pgfpathlineto{\pgfqpoint{4.567846in}{3.338968in}}%
\pgfpathlineto{\pgfqpoint{4.575199in}{3.359178in}}%
\pgfpathlineto{\pgfqpoint{4.582555in}{3.379734in}}%
\pgfpathlineto{\pgfqpoint{4.589913in}{3.400644in}}%
\pgfpathlineto{\pgfqpoint{4.577026in}{3.409063in}}%
\pgfpathlineto{\pgfqpoint{4.564145in}{3.417657in}}%
\pgfpathlineto{\pgfqpoint{4.551268in}{3.426428in}}%
\pgfpathlineto{\pgfqpoint{4.538396in}{3.435376in}}%
\pgfpathlineto{\pgfqpoint{4.531031in}{3.413822in}}%
\pgfpathlineto{\pgfqpoint{4.523669in}{3.392629in}}%
\pgfpathlineto{\pgfqpoint{4.516308in}{3.371789in}}%
\pgfpathlineto{\pgfqpoint{4.508948in}{3.351295in}}%
\pgfpathclose%
\pgfusepath{fill}%
\end{pgfscope}%
\begin{pgfscope}%
\pgfpathrectangle{\pgfqpoint{1.254980in}{0.150000in}}{\pgfqpoint{5.490039in}{5.490039in}}%
\pgfusepath{clip}%
\pgfsetbuttcap%
\pgfsetroundjoin%
\definecolor{currentfill}{rgb}{0.119738,0.603785,0.541400}%
\pgfsetfillcolor{currentfill}%
\pgfsetfillopacity{0.700000}%
\pgfsetlinewidth{0.000000pt}%
\definecolor{currentstroke}{rgb}{0.000000,0.000000,0.000000}%
\pgfsetstrokecolor{currentstroke}%
\pgfsetdash{}{0pt}%
\pgfpathmoveto{\pgfqpoint{3.459073in}{3.763016in}}%
\pgfpathlineto{\pgfqpoint{3.471959in}{3.743174in}}%
\pgfpathlineto{\pgfqpoint{3.484839in}{3.723607in}}%
\pgfpathlineto{\pgfqpoint{3.497714in}{3.704314in}}%
\pgfpathlineto{\pgfqpoint{3.510584in}{3.685292in}}%
\pgfpathlineto{\pgfqpoint{3.518031in}{3.706881in}}%
\pgfpathlineto{\pgfqpoint{3.525474in}{3.728779in}}%
\pgfpathlineto{\pgfqpoint{3.532912in}{3.750990in}}%
\pgfpathlineto{\pgfqpoint{3.540345in}{3.773520in}}%
\pgfpathlineto{\pgfqpoint{3.527476in}{3.793031in}}%
\pgfpathlineto{\pgfqpoint{3.514602in}{3.812813in}}%
\pgfpathlineto{\pgfqpoint{3.501722in}{3.832869in}}%
\pgfpathlineto{\pgfqpoint{3.488837in}{3.853203in}}%
\pgfpathlineto{\pgfqpoint{3.481403in}{3.830171in}}%
\pgfpathlineto{\pgfqpoint{3.473964in}{3.807466in}}%
\pgfpathlineto{\pgfqpoint{3.466521in}{3.785083in}}%
\pgfpathlineto{\pgfqpoint{3.459073in}{3.763016in}}%
\pgfpathclose%
\pgfusepath{fill}%
\end{pgfscope}%
\begin{pgfscope}%
\pgfpathrectangle{\pgfqpoint{1.254980in}{0.150000in}}{\pgfqpoint{5.490039in}{5.490039in}}%
\pgfusepath{clip}%
\pgfsetbuttcap%
\pgfsetroundjoin%
\definecolor{currentfill}{rgb}{0.136408,0.541173,0.554483}%
\pgfsetfillcolor{currentfill}%
\pgfsetfillopacity{0.700000}%
\pgfsetlinewidth{0.000000pt}%
\definecolor{currentstroke}{rgb}{0.000000,0.000000,0.000000}%
\pgfsetstrokecolor{currentstroke}%
\pgfsetdash{}{0pt}%
\pgfpathmoveto{\pgfqpoint{3.480751in}{3.601902in}}%
\pgfpathlineto{\pgfqpoint{3.493618in}{3.583605in}}%
\pgfpathlineto{\pgfqpoint{3.506480in}{3.565574in}}%
\pgfpathlineto{\pgfqpoint{3.519338in}{3.547805in}}%
\pgfpathlineto{\pgfqpoint{3.532192in}{3.530297in}}%
\pgfpathlineto{\pgfqpoint{3.539655in}{3.550267in}}%
\pgfpathlineto{\pgfqpoint{3.547114in}{3.570514in}}%
\pgfpathlineto{\pgfqpoint{3.554568in}{3.591045in}}%
\pgfpathlineto{\pgfqpoint{3.562017in}{3.611865in}}%
\pgfpathlineto{\pgfqpoint{3.549166in}{3.629827in}}%
\pgfpathlineto{\pgfqpoint{3.536310in}{3.648051in}}%
\pgfpathlineto{\pgfqpoint{3.523449in}{3.666538in}}%
\pgfpathlineto{\pgfqpoint{3.510584in}{3.685292in}}%
\pgfpathlineto{\pgfqpoint{3.503133in}{3.664005in}}%
\pgfpathlineto{\pgfqpoint{3.495677in}{3.643015in}}%
\pgfpathlineto{\pgfqpoint{3.488216in}{3.622316in}}%
\pgfpathlineto{\pgfqpoint{3.480751in}{3.601902in}}%
\pgfpathclose%
\pgfusepath{fill}%
\end{pgfscope}%
\begin{pgfscope}%
\pgfpathrectangle{\pgfqpoint{1.254980in}{0.150000in}}{\pgfqpoint{5.490039in}{5.490039in}}%
\pgfusepath{clip}%
\pgfsetbuttcap%
\pgfsetroundjoin%
\definecolor{currentfill}{rgb}{0.201239,0.383670,0.554294}%
\pgfsetfillcolor{currentfill}%
\pgfsetfillopacity{0.700000}%
\pgfsetlinewidth{0.000000pt}%
\definecolor{currentstroke}{rgb}{0.000000,0.000000,0.000000}%
\pgfsetstrokecolor{currentstroke}%
\pgfsetdash{}{0pt}%
\pgfpathmoveto{\pgfqpoint{4.185348in}{3.186317in}}%
\pgfpathlineto{\pgfqpoint{4.198187in}{3.177320in}}%
\pgfpathlineto{\pgfqpoint{4.211029in}{3.168514in}}%
\pgfpathlineto{\pgfqpoint{4.223875in}{3.159897in}}%
\pgfpathlineto{\pgfqpoint{4.236724in}{3.151467in}}%
\pgfpathlineto{\pgfqpoint{4.244105in}{3.168905in}}%
\pgfpathlineto{\pgfqpoint{4.251485in}{3.186602in}}%
\pgfpathlineto{\pgfqpoint{4.258863in}{3.204563in}}%
\pgfpathlineto{\pgfqpoint{4.266240in}{3.222797in}}%
\pgfpathlineto{\pgfqpoint{4.253396in}{3.231748in}}%
\pgfpathlineto{\pgfqpoint{4.240556in}{3.240887in}}%
\pgfpathlineto{\pgfqpoint{4.227720in}{3.250215in}}%
\pgfpathlineto{\pgfqpoint{4.214886in}{3.259734in}}%
\pgfpathlineto{\pgfqpoint{4.207504in}{3.240968in}}%
\pgfpathlineto{\pgfqpoint{4.200120in}{3.222481in}}%
\pgfpathlineto{\pgfqpoint{4.192734in}{3.204266in}}%
\pgfpathlineto{\pgfqpoint{4.185348in}{3.186317in}}%
\pgfpathclose%
\pgfusepath{fill}%
\end{pgfscope}%
\begin{pgfscope}%
\pgfpathrectangle{\pgfqpoint{1.254980in}{0.150000in}}{\pgfqpoint{5.490039in}{5.490039in}}%
\pgfusepath{clip}%
\pgfsetbuttcap%
\pgfsetroundjoin%
\definecolor{currentfill}{rgb}{0.206756,0.371758,0.553117}%
\pgfsetfillcolor{currentfill}%
\pgfsetfillopacity{0.700000}%
\pgfsetlinewidth{0.000000pt}%
\definecolor{currentstroke}{rgb}{0.000000,0.000000,0.000000}%
\pgfsetstrokecolor{currentstroke}%
\pgfsetdash{}{0pt}%
\pgfpathmoveto{\pgfqpoint{3.972203in}{3.163220in}}%
\pgfpathlineto{\pgfqpoint{3.985022in}{3.152823in}}%
\pgfpathlineto{\pgfqpoint{3.997842in}{3.142629in}}%
\pgfpathlineto{\pgfqpoint{4.010664in}{3.132637in}}%
\pgfpathlineto{\pgfqpoint{4.023489in}{3.122845in}}%
\pgfpathlineto{\pgfqpoint{4.030904in}{3.139788in}}%
\pgfpathlineto{\pgfqpoint{4.038316in}{3.156965in}}%
\pgfpathlineto{\pgfqpoint{4.045726in}{3.174381in}}%
\pgfpathlineto{\pgfqpoint{4.053134in}{3.192043in}}%
\pgfpathlineto{\pgfqpoint{4.040314in}{3.202303in}}%
\pgfpathlineto{\pgfqpoint{4.027497in}{3.212763in}}%
\pgfpathlineto{\pgfqpoint{4.014682in}{3.223425in}}%
\pgfpathlineto{\pgfqpoint{4.001868in}{3.234291in}}%
\pgfpathlineto{\pgfqpoint{3.994456in}{3.216150in}}%
\pgfpathlineto{\pgfqpoint{3.987041in}{3.198262in}}%
\pgfpathlineto{\pgfqpoint{3.979623in}{3.180620in}}%
\pgfpathlineto{\pgfqpoint{3.972203in}{3.163220in}}%
\pgfpathclose%
\pgfusepath{fill}%
\end{pgfscope}%
\begin{pgfscope}%
\pgfpathrectangle{\pgfqpoint{1.254980in}{0.150000in}}{\pgfqpoint{5.490039in}{5.490039in}}%
\pgfusepath{clip}%
\pgfsetbuttcap%
\pgfsetroundjoin%
\definecolor{currentfill}{rgb}{0.203063,0.379716,0.553925}%
\pgfsetfillcolor{currentfill}%
\pgfsetfillopacity{0.700000}%
\pgfsetlinewidth{0.000000pt}%
\definecolor{currentstroke}{rgb}{0.000000,0.000000,0.000000}%
\pgfsetstrokecolor{currentstroke}%
\pgfsetdash{}{0pt}%
\pgfpathmoveto{\pgfqpoint{3.839956in}{3.183039in}}%
\pgfpathlineto{\pgfqpoint{3.852769in}{3.171412in}}%
\pgfpathlineto{\pgfqpoint{3.865584in}{3.159999in}}%
\pgfpathlineto{\pgfqpoint{3.878398in}{3.148798in}}%
\pgfpathlineto{\pgfqpoint{3.891215in}{3.137807in}}%
\pgfpathlineto{\pgfqpoint{3.898652in}{3.154726in}}%
\pgfpathlineto{\pgfqpoint{3.906086in}{3.171872in}}%
\pgfpathlineto{\pgfqpoint{3.913517in}{3.189250in}}%
\pgfpathlineto{\pgfqpoint{3.920945in}{3.206866in}}%
\pgfpathlineto{\pgfqpoint{3.908134in}{3.218298in}}%
\pgfpathlineto{\pgfqpoint{3.895323in}{3.229941in}}%
\pgfpathlineto{\pgfqpoint{3.882514in}{3.241796in}}%
\pgfpathlineto{\pgfqpoint{3.869705in}{3.253865in}}%
\pgfpathlineto{\pgfqpoint{3.862272in}{3.235797in}}%
\pgfpathlineto{\pgfqpoint{3.854837in}{3.217973in}}%
\pgfpathlineto{\pgfqpoint{3.847398in}{3.200389in}}%
\pgfpathlineto{\pgfqpoint{3.839956in}{3.183039in}}%
\pgfpathclose%
\pgfusepath{fill}%
\end{pgfscope}%
\begin{pgfscope}%
\pgfpathrectangle{\pgfqpoint{1.254980in}{0.150000in}}{\pgfqpoint{5.490039in}{5.490039in}}%
\pgfusepath{clip}%
\pgfsetbuttcap%
\pgfsetroundjoin%
\definecolor{currentfill}{rgb}{0.377779,0.791781,0.377939}%
\pgfsetfillcolor{currentfill}%
\pgfsetfillopacity{0.700000}%
\pgfsetlinewidth{0.000000pt}%
\definecolor{currentstroke}{rgb}{0.000000,0.000000,0.000000}%
\pgfsetstrokecolor{currentstroke}%
\pgfsetdash{}{0pt}%
\pgfpathmoveto{\pgfqpoint{3.607238in}{4.271434in}}%
\pgfpathlineto{\pgfqpoint{3.620132in}{4.248813in}}%
\pgfpathlineto{\pgfqpoint{3.633020in}{4.226472in}}%
\pgfpathlineto{\pgfqpoint{3.645902in}{4.204411in}}%
\pgfpathlineto{\pgfqpoint{3.658779in}{4.182626in}}%
\pgfpathlineto{\pgfqpoint{3.666154in}{4.211580in}}%
\pgfpathlineto{\pgfqpoint{3.673528in}{4.240977in}}%
\pgfpathlineto{\pgfqpoint{3.680899in}{4.270823in}}%
\pgfpathlineto{\pgfqpoint{3.688268in}{4.301127in}}%
\pgfpathlineto{\pgfqpoint{3.675386in}{4.323566in}}%
\pgfpathlineto{\pgfqpoint{3.662499in}{4.346281in}}%
\pgfpathlineto{\pgfqpoint{3.649606in}{4.369278in}}%
\pgfpathlineto{\pgfqpoint{3.636707in}{4.392557in}}%
\pgfpathlineto{\pgfqpoint{3.629344in}{4.361584in}}%
\pgfpathlineto{\pgfqpoint{3.621978in}{4.331078in}}%
\pgfpathlineto{\pgfqpoint{3.614609in}{4.301031in}}%
\pgfpathlineto{\pgfqpoint{3.607238in}{4.271434in}}%
\pgfpathclose%
\pgfusepath{fill}%
\end{pgfscope}%
\begin{pgfscope}%
\pgfpathrectangle{\pgfqpoint{1.254980in}{0.150000in}}{\pgfqpoint{5.490039in}{5.490039in}}%
\pgfusepath{clip}%
\pgfsetbuttcap%
\pgfsetroundjoin%
\definecolor{currentfill}{rgb}{0.179019,0.433756,0.557430}%
\pgfsetfillcolor{currentfill}%
\pgfsetfillopacity{0.700000}%
\pgfsetlinewidth{0.000000pt}%
\definecolor{currentstroke}{rgb}{0.000000,0.000000,0.000000}%
\pgfsetstrokecolor{currentstroke}%
\pgfsetdash{}{0pt}%
\pgfpathmoveto{\pgfqpoint{3.605027in}{3.325538in}}%
\pgfpathlineto{\pgfqpoint{3.617856in}{3.310700in}}%
\pgfpathlineto{\pgfqpoint{3.630683in}{3.296102in}}%
\pgfpathlineto{\pgfqpoint{3.643508in}{3.281741in}}%
\pgfpathlineto{\pgfqpoint{3.656332in}{3.267616in}}%
\pgfpathlineto{\pgfqpoint{3.663802in}{3.285294in}}%
\pgfpathlineto{\pgfqpoint{3.671269in}{3.303209in}}%
\pgfpathlineto{\pgfqpoint{3.678731in}{3.321364in}}%
\pgfpathlineto{\pgfqpoint{3.686189in}{3.339764in}}%
\pgfpathlineto{\pgfqpoint{3.673369in}{3.354307in}}%
\pgfpathlineto{\pgfqpoint{3.660547in}{3.369087in}}%
\pgfpathlineto{\pgfqpoint{3.647724in}{3.384104in}}%
\pgfpathlineto{\pgfqpoint{3.634898in}{3.399362in}}%
\pgfpathlineto{\pgfqpoint{3.627437in}{3.380531in}}%
\pgfpathlineto{\pgfqpoint{3.619971in}{3.361954in}}%
\pgfpathlineto{\pgfqpoint{3.612501in}{3.343624in}}%
\pgfpathlineto{\pgfqpoint{3.605027in}{3.325538in}}%
\pgfpathclose%
\pgfusepath{fill}%
\end{pgfscope}%
\begin{pgfscope}%
\pgfpathrectangle{\pgfqpoint{1.254980in}{0.150000in}}{\pgfqpoint{5.490039in}{5.490039in}}%
\pgfusepath{clip}%
\pgfsetbuttcap%
\pgfsetroundjoin%
\definecolor{currentfill}{rgb}{0.168126,0.459988,0.558082}%
\pgfsetfillcolor{currentfill}%
\pgfsetfillopacity{0.700000}%
\pgfsetlinewidth{0.000000pt}%
\definecolor{currentstroke}{rgb}{0.000000,0.000000,0.000000}%
\pgfsetstrokecolor{currentstroke}%
\pgfsetdash{}{0pt}%
\pgfpathmoveto{\pgfqpoint{3.553685in}{3.387323in}}%
\pgfpathlineto{\pgfqpoint{3.566524in}{3.371508in}}%
\pgfpathlineto{\pgfqpoint{3.579361in}{3.355940in}}%
\pgfpathlineto{\pgfqpoint{3.592195in}{3.340617in}}%
\pgfpathlineto{\pgfqpoint{3.605027in}{3.325538in}}%
\pgfpathlineto{\pgfqpoint{3.612501in}{3.343624in}}%
\pgfpathlineto{\pgfqpoint{3.619971in}{3.361954in}}%
\pgfpathlineto{\pgfqpoint{3.627437in}{3.380531in}}%
\pgfpathlineto{\pgfqpoint{3.634898in}{3.399362in}}%
\pgfpathlineto{\pgfqpoint{3.622070in}{3.414861in}}%
\pgfpathlineto{\pgfqpoint{3.609240in}{3.430604in}}%
\pgfpathlineto{\pgfqpoint{3.596406in}{3.446593in}}%
\pgfpathlineto{\pgfqpoint{3.583570in}{3.462830in}}%
\pgfpathlineto{\pgfqpoint{3.576106in}{3.443568in}}%
\pgfpathlineto{\pgfqpoint{3.568637in}{3.424566in}}%
\pgfpathlineto{\pgfqpoint{3.561163in}{3.405819in}}%
\pgfpathlineto{\pgfqpoint{3.553685in}{3.387323in}}%
\pgfpathclose%
\pgfusepath{fill}%
\end{pgfscope}%
\begin{pgfscope}%
\pgfpathrectangle{\pgfqpoint{1.254980in}{0.150000in}}{\pgfqpoint{5.490039in}{5.490039in}}%
\pgfusepath{clip}%
\pgfsetbuttcap%
\pgfsetroundjoin%
\definecolor{currentfill}{rgb}{0.187231,0.414746,0.556547}%
\pgfsetfillcolor{currentfill}%
\pgfsetfillopacity{0.700000}%
\pgfsetlinewidth{0.000000pt}%
\definecolor{currentstroke}{rgb}{0.000000,0.000000,0.000000}%
\pgfsetstrokecolor{currentstroke}%
\pgfsetdash{}{0pt}%
\pgfpathmoveto{\pgfqpoint{3.656332in}{3.267616in}}%
\pgfpathlineto{\pgfqpoint{3.669154in}{3.253725in}}%
\pgfpathlineto{\pgfqpoint{3.681975in}{3.240066in}}%
\pgfpathlineto{\pgfqpoint{3.694795in}{3.226638in}}%
\pgfpathlineto{\pgfqpoint{3.707614in}{3.213438in}}%
\pgfpathlineto{\pgfqpoint{3.715080in}{3.230710in}}%
\pgfpathlineto{\pgfqpoint{3.722542in}{3.248211in}}%
\pgfpathlineto{\pgfqpoint{3.730000in}{3.265945in}}%
\pgfpathlineto{\pgfqpoint{3.737454in}{3.283918in}}%
\pgfpathlineto{\pgfqpoint{3.724640in}{3.297534in}}%
\pgfpathlineto{\pgfqpoint{3.711824in}{3.311379in}}%
\pgfpathlineto{\pgfqpoint{3.699007in}{3.325455in}}%
\pgfpathlineto{\pgfqpoint{3.686189in}{3.339764in}}%
\pgfpathlineto{\pgfqpoint{3.678731in}{3.321364in}}%
\pgfpathlineto{\pgfqpoint{3.671269in}{3.303209in}}%
\pgfpathlineto{\pgfqpoint{3.663802in}{3.285294in}}%
\pgfpathlineto{\pgfqpoint{3.656332in}{3.267616in}}%
\pgfpathclose%
\pgfusepath{fill}%
\end{pgfscope}%
\begin{pgfscope}%
\pgfpathrectangle{\pgfqpoint{1.254980in}{0.150000in}}{\pgfqpoint{5.490039in}{5.490039in}}%
\pgfusepath{clip}%
\pgfsetbuttcap%
\pgfsetroundjoin%
\definecolor{currentfill}{rgb}{0.165117,0.467423,0.558141}%
\pgfsetfillcolor{currentfill}%
\pgfsetfillopacity{0.700000}%
\pgfsetlinewidth{0.000000pt}%
\definecolor{currentstroke}{rgb}{0.000000,0.000000,0.000000}%
\pgfsetstrokecolor{currentstroke}%
\pgfsetdash{}{0pt}%
\pgfpathmoveto{\pgfqpoint{4.589913in}{3.400644in}}%
\pgfpathlineto{\pgfqpoint{4.602804in}{3.392399in}}%
\pgfpathlineto{\pgfqpoint{4.615700in}{3.384328in}}%
\pgfpathlineto{\pgfqpoint{4.628602in}{3.376430in}}%
\pgfpathlineto{\pgfqpoint{4.641509in}{3.368705in}}%
\pgfpathlineto{\pgfqpoint{4.648861in}{3.389326in}}%
\pgfpathlineto{\pgfqpoint{4.656216in}{3.410308in}}%
\pgfpathlineto{\pgfqpoint{4.663574in}{3.431661in}}%
\pgfpathlineto{\pgfqpoint{4.650673in}{3.439879in}}%
\pgfpathlineto{\pgfqpoint{4.637777in}{3.448270in}}%
\pgfpathlineto{\pgfqpoint{4.624886in}{3.456834in}}%
\pgfpathlineto{\pgfqpoint{4.612001in}{3.465572in}}%
\pgfpathlineto{\pgfqpoint{4.604635in}{3.443555in}}%
\pgfpathlineto{\pgfqpoint{4.597273in}{3.421915in}}%
\pgfpathlineto{\pgfqpoint{4.589913in}{3.400644in}}%
\pgfpathclose%
\pgfusepath{fill}%
\end{pgfscope}%
\begin{pgfscope}%
\pgfpathrectangle{\pgfqpoint{1.254980in}{0.150000in}}{\pgfqpoint{5.490039in}{5.490039in}}%
\pgfusepath{clip}%
\pgfsetbuttcap%
\pgfsetroundjoin%
\definecolor{currentfill}{rgb}{0.191090,0.708366,0.482284}%
\pgfsetfillcolor{currentfill}%
\pgfsetfillopacity{0.700000}%
\pgfsetlinewidth{0.000000pt}%
\definecolor{currentstroke}{rgb}{0.000000,0.000000,0.000000}%
\pgfsetstrokecolor{currentstroke}%
\pgfsetdash{}{0pt}%
\pgfpathmoveto{\pgfqpoint{3.466924in}{4.034981in}}%
\pgfpathlineto{\pgfqpoint{3.479836in}{4.012987in}}%
\pgfpathlineto{\pgfqpoint{3.492740in}{3.991282in}}%
\pgfpathlineto{\pgfqpoint{3.505638in}{3.969862in}}%
\pgfpathlineto{\pgfqpoint{3.518530in}{3.948726in}}%
\pgfpathlineto{\pgfqpoint{3.525942in}{3.973487in}}%
\pgfpathlineto{\pgfqpoint{3.533351in}{3.998612in}}%
\pgfpathlineto{\pgfqpoint{3.540756in}{4.024109in}}%
\pgfpathlineto{\pgfqpoint{3.548157in}{4.049985in}}%
\pgfpathlineto{\pgfqpoint{3.535263in}{4.071677in}}%
\pgfpathlineto{\pgfqpoint{3.522363in}{4.093654in}}%
\pgfpathlineto{\pgfqpoint{3.509457in}{4.115918in}}%
\pgfpathlineto{\pgfqpoint{3.496544in}{4.138471in}}%
\pgfpathlineto{\pgfqpoint{3.489145in}{4.112025in}}%
\pgfpathlineto{\pgfqpoint{3.481743in}{4.085966in}}%
\pgfpathlineto{\pgfqpoint{3.474336in}{4.060287in}}%
\pgfpathlineto{\pgfqpoint{3.466924in}{4.034981in}}%
\pgfpathclose%
\pgfusepath{fill}%
\end{pgfscope}%
\begin{pgfscope}%
\pgfpathrectangle{\pgfqpoint{1.254980in}{0.150000in}}{\pgfqpoint{5.490039in}{5.490039in}}%
\pgfusepath{clip}%
\pgfsetbuttcap%
\pgfsetroundjoin%
\definecolor{currentfill}{rgb}{0.208623,0.367752,0.552675}%
\pgfsetfillcolor{currentfill}%
\pgfsetfillopacity{0.700000}%
\pgfsetlinewidth{0.000000pt}%
\definecolor{currentstroke}{rgb}{0.000000,0.000000,0.000000}%
\pgfsetstrokecolor{currentstroke}%
\pgfsetdash{}{0pt}%
\pgfpathmoveto{\pgfqpoint{4.104436in}{3.152992in}}%
\pgfpathlineto{\pgfqpoint{4.117268in}{3.143719in}}%
\pgfpathlineto{\pgfqpoint{4.130103in}{3.134641in}}%
\pgfpathlineto{\pgfqpoint{4.142941in}{3.125756in}}%
\pgfpathlineto{\pgfqpoint{4.155782in}{3.117063in}}%
\pgfpathlineto{\pgfqpoint{4.163177in}{3.134007in}}%
\pgfpathlineto{\pgfqpoint{4.170569in}{3.151193in}}%
\pgfpathlineto{\pgfqpoint{4.177959in}{3.168628in}}%
\pgfpathlineto{\pgfqpoint{4.185348in}{3.186317in}}%
\pgfpathlineto{\pgfqpoint{4.172512in}{3.195504in}}%
\pgfpathlineto{\pgfqpoint{4.159679in}{3.204884in}}%
\pgfpathlineto{\pgfqpoint{4.146850in}{3.214457in}}%
\pgfpathlineto{\pgfqpoint{4.134023in}{3.224224in}}%
\pgfpathlineto{\pgfqpoint{4.126629in}{3.206030in}}%
\pgfpathlineto{\pgfqpoint{4.119233in}{3.188097in}}%
\pgfpathlineto{\pgfqpoint{4.111836in}{3.170419in}}%
\pgfpathlineto{\pgfqpoint{4.104436in}{3.152992in}}%
\pgfpathclose%
\pgfusepath{fill}%
\end{pgfscope}%
\begin{pgfscope}%
\pgfpathrectangle{\pgfqpoint{1.254980in}{0.150000in}}{\pgfqpoint{5.490039in}{5.490039in}}%
\pgfusepath{clip}%
\pgfsetbuttcap%
\pgfsetroundjoin%
\definecolor{currentfill}{rgb}{0.266941,0.748751,0.440573}%
\pgfsetfillcolor{currentfill}%
\pgfsetfillopacity{0.700000}%
\pgfsetlinewidth{0.000000pt}%
\definecolor{currentstroke}{rgb}{0.000000,0.000000,0.000000}%
\pgfsetstrokecolor{currentstroke}%
\pgfsetdash{}{0pt}%
\pgfpathmoveto{\pgfqpoint{3.496544in}{4.138471in}}%
\pgfpathlineto{\pgfqpoint{3.509457in}{4.115918in}}%
\pgfpathlineto{\pgfqpoint{3.522363in}{4.093654in}}%
\pgfpathlineto{\pgfqpoint{3.535263in}{4.071677in}}%
\pgfpathlineto{\pgfqpoint{3.548157in}{4.049985in}}%
\pgfpathlineto{\pgfqpoint{3.555554in}{4.076245in}}%
\pgfpathlineto{\pgfqpoint{3.562947in}{4.102897in}}%
\pgfpathlineto{\pgfqpoint{3.570337in}{4.129947in}}%
\pgfpathlineto{\pgfqpoint{3.577724in}{4.157404in}}%
\pgfpathlineto{\pgfqpoint{3.564828in}{4.179686in}}%
\pgfpathlineto{\pgfqpoint{3.551925in}{4.202253in}}%
\pgfpathlineto{\pgfqpoint{3.539015in}{4.225109in}}%
\pgfpathlineto{\pgfqpoint{3.526099in}{4.248256in}}%
\pgfpathlineto{\pgfqpoint{3.518716in}{4.220195in}}%
\pgfpathlineto{\pgfqpoint{3.511329in}{4.192549in}}%
\pgfpathlineto{\pgfqpoint{3.503939in}{4.165310in}}%
\pgfpathlineto{\pgfqpoint{3.496544in}{4.138471in}}%
\pgfpathclose%
\pgfusepath{fill}%
\end{pgfscope}%
\begin{pgfscope}%
\pgfpathrectangle{\pgfqpoint{1.254980in}{0.150000in}}{\pgfqpoint{5.490039in}{5.490039in}}%
\pgfusepath{clip}%
\pgfsetbuttcap%
\pgfsetroundjoin%
\definecolor{currentfill}{rgb}{0.146616,0.673050,0.508936}%
\pgfsetfillcolor{currentfill}%
\pgfsetfillopacity{0.700000}%
\pgfsetlinewidth{0.000000pt}%
\definecolor{currentstroke}{rgb}{0.000000,0.000000,0.000000}%
\pgfsetstrokecolor{currentstroke}%
\pgfsetdash{}{0pt}%
\pgfpathmoveto{\pgfqpoint{3.437235in}{3.937357in}}%
\pgfpathlineto{\pgfqpoint{3.450145in}{3.915890in}}%
\pgfpathlineto{\pgfqpoint{3.463048in}{3.894710in}}%
\pgfpathlineto{\pgfqpoint{3.475946in}{3.873815in}}%
\pgfpathlineto{\pgfqpoint{3.488837in}{3.853203in}}%
\pgfpathlineto{\pgfqpoint{3.496267in}{3.876568in}}%
\pgfpathlineto{\pgfqpoint{3.503692in}{3.900273in}}%
\pgfpathlineto{\pgfqpoint{3.511113in}{3.924323in}}%
\pgfpathlineto{\pgfqpoint{3.518530in}{3.948726in}}%
\pgfpathlineto{\pgfqpoint{3.505638in}{3.969862in}}%
\pgfpathlineto{\pgfqpoint{3.492740in}{3.991282in}}%
\pgfpathlineto{\pgfqpoint{3.479836in}{4.012987in}}%
\pgfpathlineto{\pgfqpoint{3.466924in}{4.034981in}}%
\pgfpathlineto{\pgfqpoint{3.459509in}{4.010041in}}%
\pgfpathlineto{\pgfqpoint{3.452089in}{3.985461in}}%
\pgfpathlineto{\pgfqpoint{3.444664in}{3.961235in}}%
\pgfpathlineto{\pgfqpoint{3.437235in}{3.937357in}}%
\pgfpathclose%
\pgfusepath{fill}%
\end{pgfscope}%
\begin{pgfscope}%
\pgfpathrectangle{\pgfqpoint{1.254980in}{0.150000in}}{\pgfqpoint{5.490039in}{5.490039in}}%
\pgfusepath{clip}%
\pgfsetbuttcap%
\pgfsetroundjoin%
\definecolor{currentfill}{rgb}{0.157729,0.485932,0.558013}%
\pgfsetfillcolor{currentfill}%
\pgfsetfillopacity{0.700000}%
\pgfsetlinewidth{0.000000pt}%
\definecolor{currentstroke}{rgb}{0.000000,0.000000,0.000000}%
\pgfsetstrokecolor{currentstroke}%
\pgfsetdash{}{0pt}%
\pgfpathmoveto{\pgfqpoint{3.502294in}{3.453099in}}%
\pgfpathlineto{\pgfqpoint{3.515147in}{3.436274in}}%
\pgfpathlineto{\pgfqpoint{3.527996in}{3.419704in}}%
\pgfpathlineto{\pgfqpoint{3.540842in}{3.403388in}}%
\pgfpathlineto{\pgfqpoint{3.553685in}{3.387323in}}%
\pgfpathlineto{\pgfqpoint{3.561163in}{3.405819in}}%
\pgfpathlineto{\pgfqpoint{3.568637in}{3.424566in}}%
\pgfpathlineto{\pgfqpoint{3.576106in}{3.443568in}}%
\pgfpathlineto{\pgfqpoint{3.583570in}{3.462830in}}%
\pgfpathlineto{\pgfqpoint{3.570731in}{3.479317in}}%
\pgfpathlineto{\pgfqpoint{3.557888in}{3.496055in}}%
\pgfpathlineto{\pgfqpoint{3.545042in}{3.513048in}}%
\pgfpathlineto{\pgfqpoint{3.532192in}{3.530297in}}%
\pgfpathlineto{\pgfqpoint{3.524725in}{3.510601in}}%
\pgfpathlineto{\pgfqpoint{3.517253in}{3.491173in}}%
\pgfpathlineto{\pgfqpoint{3.509776in}{3.472007in}}%
\pgfpathlineto{\pgfqpoint{3.502294in}{3.453099in}}%
\pgfpathclose%
\pgfusepath{fill}%
\end{pgfscope}%
\begin{pgfscope}%
\pgfpathrectangle{\pgfqpoint{1.254980in}{0.150000in}}{\pgfqpoint{5.490039in}{5.490039in}}%
\pgfusepath{clip}%
\pgfsetbuttcap%
\pgfsetroundjoin%
\definecolor{currentfill}{rgb}{0.197636,0.391528,0.554969}%
\pgfsetfillcolor{currentfill}%
\pgfsetfillopacity{0.700000}%
\pgfsetlinewidth{0.000000pt}%
\definecolor{currentstroke}{rgb}{0.000000,0.000000,0.000000}%
\pgfsetstrokecolor{currentstroke}%
\pgfsetdash{}{0pt}%
\pgfpathmoveto{\pgfqpoint{3.707614in}{3.213438in}}%
\pgfpathlineto{\pgfqpoint{3.720432in}{3.200466in}}%
\pgfpathlineto{\pgfqpoint{3.733249in}{3.187719in}}%
\pgfpathlineto{\pgfqpoint{3.746066in}{3.175195in}}%
\pgfpathlineto{\pgfqpoint{3.758883in}{3.162894in}}%
\pgfpathlineto{\pgfqpoint{3.766345in}{3.179761in}}%
\pgfpathlineto{\pgfqpoint{3.773802in}{3.196850in}}%
\pgfpathlineto{\pgfqpoint{3.781256in}{3.214165in}}%
\pgfpathlineto{\pgfqpoint{3.788706in}{3.231710in}}%
\pgfpathlineto{\pgfqpoint{3.775894in}{3.244427in}}%
\pgfpathlineto{\pgfqpoint{3.763081in}{3.257366in}}%
\pgfpathlineto{\pgfqpoint{3.750268in}{3.270529in}}%
\pgfpathlineto{\pgfqpoint{3.737454in}{3.283918in}}%
\pgfpathlineto{\pgfqpoint{3.730000in}{3.265945in}}%
\pgfpathlineto{\pgfqpoint{3.722542in}{3.248211in}}%
\pgfpathlineto{\pgfqpoint{3.715080in}{3.230710in}}%
\pgfpathlineto{\pgfqpoint{3.707614in}{3.213438in}}%
\pgfpathclose%
\pgfusepath{fill}%
\end{pgfscope}%
\begin{pgfscope}%
\pgfpathrectangle{\pgfqpoint{1.254980in}{0.150000in}}{\pgfqpoint{5.490039in}{5.490039in}}%
\pgfusepath{clip}%
\pgfsetbuttcap%
\pgfsetroundjoin%
\definecolor{currentfill}{rgb}{0.126453,0.570633,0.549841}%
\pgfsetfillcolor{currentfill}%
\pgfsetfillopacity{0.700000}%
\pgfsetlinewidth{0.000000pt}%
\definecolor{currentstroke}{rgb}{0.000000,0.000000,0.000000}%
\pgfsetstrokecolor{currentstroke}%
\pgfsetdash{}{0pt}%
\pgfpathmoveto{\pgfqpoint{3.429233in}{3.677794in}}%
\pgfpathlineto{\pgfqpoint{3.442120in}{3.658411in}}%
\pgfpathlineto{\pgfqpoint{3.455003in}{3.639303in}}%
\pgfpathlineto{\pgfqpoint{3.467879in}{3.620467in}}%
\pgfpathlineto{\pgfqpoint{3.480751in}{3.601902in}}%
\pgfpathlineto{\pgfqpoint{3.488216in}{3.622316in}}%
\pgfpathlineto{\pgfqpoint{3.495677in}{3.643015in}}%
\pgfpathlineto{\pgfqpoint{3.503133in}{3.664005in}}%
\pgfpathlineto{\pgfqpoint{3.510584in}{3.685292in}}%
\pgfpathlineto{\pgfqpoint{3.497714in}{3.704314in}}%
\pgfpathlineto{\pgfqpoint{3.484839in}{3.723607in}}%
\pgfpathlineto{\pgfqpoint{3.471959in}{3.743174in}}%
\pgfpathlineto{\pgfqpoint{3.459073in}{3.763016in}}%
\pgfpathlineto{\pgfqpoint{3.451620in}{3.741259in}}%
\pgfpathlineto{\pgfqpoint{3.444163in}{3.719807in}}%
\pgfpathlineto{\pgfqpoint{3.436700in}{3.698654in}}%
\pgfpathlineto{\pgfqpoint{3.429233in}{3.677794in}}%
\pgfpathclose%
\pgfusepath{fill}%
\end{pgfscope}%
\begin{pgfscope}%
\pgfpathrectangle{\pgfqpoint{1.254980in}{0.150000in}}{\pgfqpoint{5.490039in}{5.490039in}}%
\pgfusepath{clip}%
\pgfsetbuttcap%
\pgfsetroundjoin%
\definecolor{currentfill}{rgb}{0.515992,0.831158,0.294279}%
\pgfsetfillcolor{currentfill}%
\pgfsetfillopacity{0.700000}%
\pgfsetlinewidth{0.000000pt}%
\definecolor{currentstroke}{rgb}{0.000000,0.000000,0.000000}%
\pgfsetstrokecolor{currentstroke}%
\pgfsetdash{}{0pt}%
\pgfpathmoveto{\pgfqpoint{3.717724in}{4.427087in}}%
\pgfpathlineto{\pgfqpoint{3.730606in}{4.404237in}}%
\pgfpathlineto{\pgfqpoint{3.743481in}{4.381661in}}%
\pgfpathlineto{\pgfqpoint{3.756352in}{4.359356in}}%
\pgfpathlineto{\pgfqpoint{3.769217in}{4.337320in}}%
\pgfpathlineto{\pgfqpoint{3.776584in}{4.369331in}}%
\pgfpathlineto{\pgfqpoint{3.783950in}{4.401840in}}%
\pgfpathlineto{\pgfqpoint{3.791315in}{4.434857in}}%
\pgfpathlineto{\pgfqpoint{3.778445in}{4.457428in}}%
\pgfpathlineto{\pgfqpoint{3.765569in}{4.480270in}}%
\pgfpathlineto{\pgfqpoint{3.752688in}{4.503384in}}%
\pgfpathlineto{\pgfqpoint{3.739802in}{4.526773in}}%
\pgfpathlineto{\pgfqpoint{3.732444in}{4.493030in}}%
\pgfpathlineto{\pgfqpoint{3.725085in}{4.459805in}}%
\pgfpathlineto{\pgfqpoint{3.717724in}{4.427087in}}%
\pgfpathclose%
\pgfusepath{fill}%
\end{pgfscope}%
\begin{pgfscope}%
\pgfpathrectangle{\pgfqpoint{1.254980in}{0.150000in}}{\pgfqpoint{5.490039in}{5.490039in}}%
\pgfusepath{clip}%
\pgfsetbuttcap%
\pgfsetroundjoin%
\definecolor{currentfill}{rgb}{0.210503,0.363727,0.552206}%
\pgfsetfillcolor{currentfill}%
\pgfsetfillopacity{0.700000}%
\pgfsetlinewidth{0.000000pt}%
\definecolor{currentstroke}{rgb}{0.000000,0.000000,0.000000}%
\pgfsetstrokecolor{currentstroke}%
\pgfsetdash{}{0pt}%
\pgfpathmoveto{\pgfqpoint{3.891215in}{3.137807in}}%
\pgfpathlineto{\pgfqpoint{3.904032in}{3.127026in}}%
\pgfpathlineto{\pgfqpoint{3.916850in}{3.116452in}}%
\pgfpathlineto{\pgfqpoint{3.929670in}{3.106086in}}%
\pgfpathlineto{\pgfqpoint{3.942492in}{3.095924in}}%
\pgfpathlineto{\pgfqpoint{3.949925in}{3.112413in}}%
\pgfpathlineto{\pgfqpoint{3.957354in}{3.129122in}}%
\pgfpathlineto{\pgfqpoint{3.964780in}{3.146056in}}%
\pgfpathlineto{\pgfqpoint{3.972203in}{3.163220in}}%
\pgfpathlineto{\pgfqpoint{3.959386in}{3.173822in}}%
\pgfpathlineto{\pgfqpoint{3.946571in}{3.184629in}}%
\pgfpathlineto{\pgfqpoint{3.933758in}{3.195643in}}%
\pgfpathlineto{\pgfqpoint{3.920945in}{3.206866in}}%
\pgfpathlineto{\pgfqpoint{3.913517in}{3.189250in}}%
\pgfpathlineto{\pgfqpoint{3.906086in}{3.171872in}}%
\pgfpathlineto{\pgfqpoint{3.898652in}{3.154726in}}%
\pgfpathlineto{\pgfqpoint{3.891215in}{3.137807in}}%
\pgfpathclose%
\pgfusepath{fill}%
\end{pgfscope}%
\begin{pgfscope}%
\pgfpathrectangle{\pgfqpoint{1.254980in}{0.150000in}}{\pgfqpoint{5.490039in}{5.490039in}}%
\pgfusepath{clip}%
\pgfsetbuttcap%
\pgfsetroundjoin%
\definecolor{currentfill}{rgb}{0.192357,0.403199,0.555836}%
\pgfsetfillcolor{currentfill}%
\pgfsetfillopacity{0.700000}%
\pgfsetlinewidth{0.000000pt}%
\definecolor{currentstroke}{rgb}{0.000000,0.000000,0.000000}%
\pgfsetstrokecolor{currentstroke}%
\pgfsetdash{}{0pt}%
\pgfpathmoveto{\pgfqpoint{4.398580in}{3.229235in}}%
\pgfpathlineto{\pgfqpoint{4.411455in}{3.221389in}}%
\pgfpathlineto{\pgfqpoint{4.424334in}{3.213722in}}%
\pgfpathlineto{\pgfqpoint{4.437218in}{3.206234in}}%
\pgfpathlineto{\pgfqpoint{4.450106in}{3.198923in}}%
\pgfpathlineto{\pgfqpoint{4.457460in}{3.216910in}}%
\pgfpathlineto{\pgfqpoint{4.464814in}{3.235186in}}%
\pgfpathlineto{\pgfqpoint{4.472168in}{3.253757in}}%
\pgfpathlineto{\pgfqpoint{4.479523in}{3.272631in}}%
\pgfpathlineto{\pgfqpoint{4.466641in}{3.280518in}}%
\pgfpathlineto{\pgfqpoint{4.453765in}{3.288583in}}%
\pgfpathlineto{\pgfqpoint{4.440893in}{3.296826in}}%
\pgfpathlineto{\pgfqpoint{4.428025in}{3.305249in}}%
\pgfpathlineto{\pgfqpoint{4.420664in}{3.285788in}}%
\pgfpathlineto{\pgfqpoint{4.413303in}{3.266636in}}%
\pgfpathlineto{\pgfqpoint{4.405942in}{3.247788in}}%
\pgfpathlineto{\pgfqpoint{4.398580in}{3.229235in}}%
\pgfpathclose%
\pgfusepath{fill}%
\end{pgfscope}%
\begin{pgfscope}%
\pgfpathrectangle{\pgfqpoint{1.254980in}{0.150000in}}{\pgfqpoint{5.490039in}{5.490039in}}%
\pgfusepath{clip}%
\pgfsetbuttcap%
\pgfsetroundjoin%
\definecolor{currentfill}{rgb}{0.214298,0.355619,0.551184}%
\pgfsetfillcolor{currentfill}%
\pgfsetfillopacity{0.700000}%
\pgfsetlinewidth{0.000000pt}%
\definecolor{currentstroke}{rgb}{0.000000,0.000000,0.000000}%
\pgfsetstrokecolor{currentstroke}%
\pgfsetdash{}{0pt}%
\pgfpathmoveto{\pgfqpoint{4.023489in}{3.122845in}}%
\pgfpathlineto{\pgfqpoint{4.036316in}{3.113253in}}%
\pgfpathlineto{\pgfqpoint{4.049145in}{3.103859in}}%
\pgfpathlineto{\pgfqpoint{4.061977in}{3.094662in}}%
\pgfpathlineto{\pgfqpoint{4.074812in}{3.085662in}}%
\pgfpathlineto{\pgfqpoint{4.082222in}{3.102148in}}%
\pgfpathlineto{\pgfqpoint{4.089629in}{3.118861in}}%
\pgfpathlineto{\pgfqpoint{4.097033in}{3.135807in}}%
\pgfpathlineto{\pgfqpoint{4.104436in}{3.152992in}}%
\pgfpathlineto{\pgfqpoint{4.091606in}{3.162459in}}%
\pgfpathlineto{\pgfqpoint{4.078780in}{3.172123in}}%
\pgfpathlineto{\pgfqpoint{4.065956in}{3.181984in}}%
\pgfpathlineto{\pgfqpoint{4.053134in}{3.192043in}}%
\pgfpathlineto{\pgfqpoint{4.045726in}{3.174381in}}%
\pgfpathlineto{\pgfqpoint{4.038316in}{3.156965in}}%
\pgfpathlineto{\pgfqpoint{4.030904in}{3.139788in}}%
\pgfpathlineto{\pgfqpoint{4.023489in}{3.122845in}}%
\pgfpathclose%
\pgfusepath{fill}%
\end{pgfscope}%
\begin{pgfscope}%
\pgfpathrectangle{\pgfqpoint{1.254980in}{0.150000in}}{\pgfqpoint{5.490039in}{5.490039in}}%
\pgfusepath{clip}%
\pgfsetbuttcap%
\pgfsetroundjoin%
\definecolor{currentfill}{rgb}{0.123444,0.636809,0.528763}%
\pgfsetfillcolor{currentfill}%
\pgfsetfillopacity{0.700000}%
\pgfsetlinewidth{0.000000pt}%
\definecolor{currentstroke}{rgb}{0.000000,0.000000,0.000000}%
\pgfsetstrokecolor{currentstroke}%
\pgfsetdash{}{0pt}%
\pgfpathmoveto{\pgfqpoint{3.407469in}{3.845199in}}%
\pgfpathlineto{\pgfqpoint{3.420379in}{3.824226in}}%
\pgfpathlineto{\pgfqpoint{3.433283in}{3.803540in}}%
\pgfpathlineto{\pgfqpoint{3.446181in}{3.783138in}}%
\pgfpathlineto{\pgfqpoint{3.459073in}{3.763016in}}%
\pgfpathlineto{\pgfqpoint{3.466521in}{3.785083in}}%
\pgfpathlineto{\pgfqpoint{3.473964in}{3.807466in}}%
\pgfpathlineto{\pgfqpoint{3.481403in}{3.830171in}}%
\pgfpathlineto{\pgfqpoint{3.488837in}{3.853203in}}%
\pgfpathlineto{\pgfqpoint{3.475946in}{3.873815in}}%
\pgfpathlineto{\pgfqpoint{3.463048in}{3.894710in}}%
\pgfpathlineto{\pgfqpoint{3.450145in}{3.915890in}}%
\pgfpathlineto{\pgfqpoint{3.437235in}{3.937357in}}%
\pgfpathlineto{\pgfqpoint{3.429800in}{3.913820in}}%
\pgfpathlineto{\pgfqpoint{3.422361in}{3.890619in}}%
\pgfpathlineto{\pgfqpoint{3.414918in}{3.867747in}}%
\pgfpathlineto{\pgfqpoint{3.407469in}{3.845199in}}%
\pgfpathclose%
\pgfusepath{fill}%
\end{pgfscope}%
\begin{pgfscope}%
\pgfpathrectangle{\pgfqpoint{1.254980in}{0.150000in}}{\pgfqpoint{5.490039in}{5.490039in}}%
\pgfusepath{clip}%
\pgfsetbuttcap%
\pgfsetroundjoin%
\definecolor{currentfill}{rgb}{0.360741,0.785964,0.387814}%
\pgfsetfillcolor{currentfill}%
\pgfsetfillopacity{0.700000}%
\pgfsetlinewidth{0.000000pt}%
\definecolor{currentstroke}{rgb}{0.000000,0.000000,0.000000}%
\pgfsetstrokecolor{currentstroke}%
\pgfsetdash{}{0pt}%
\pgfpathmoveto{\pgfqpoint{3.526099in}{4.248256in}}%
\pgfpathlineto{\pgfqpoint{3.539015in}{4.225109in}}%
\pgfpathlineto{\pgfqpoint{3.551925in}{4.202253in}}%
\pgfpathlineto{\pgfqpoint{3.564828in}{4.179686in}}%
\pgfpathlineto{\pgfqpoint{3.577724in}{4.157404in}}%
\pgfpathlineto{\pgfqpoint{3.585107in}{4.185273in}}%
\pgfpathlineto{\pgfqpoint{3.592487in}{4.213563in}}%
\pgfpathlineto{\pgfqpoint{3.599864in}{4.242281in}}%
\pgfpathlineto{\pgfqpoint{3.607238in}{4.271434in}}%
\pgfpathlineto{\pgfqpoint{3.594338in}{4.294340in}}%
\pgfpathlineto{\pgfqpoint{3.581431in}{4.317533in}}%
\pgfpathlineto{\pgfqpoint{3.568517in}{4.341015in}}%
\pgfpathlineto{\pgfqpoint{3.555596in}{4.364790in}}%
\pgfpathlineto{\pgfqpoint{3.548227in}{4.334997in}}%
\pgfpathlineto{\pgfqpoint{3.540855in}{4.305649in}}%
\pgfpathlineto{\pgfqpoint{3.533479in}{4.276738in}}%
\pgfpathlineto{\pgfqpoint{3.526099in}{4.248256in}}%
\pgfpathclose%
\pgfusepath{fill}%
\end{pgfscope}%
\begin{pgfscope}%
\pgfpathrectangle{\pgfqpoint{1.254980in}{0.150000in}}{\pgfqpoint{5.490039in}{5.490039in}}%
\pgfusepath{clip}%
\pgfsetbuttcap%
\pgfsetroundjoin%
\definecolor{currentfill}{rgb}{0.199430,0.387607,0.554642}%
\pgfsetfillcolor{currentfill}%
\pgfsetfillopacity{0.700000}%
\pgfsetlinewidth{0.000000pt}%
\definecolor{currentstroke}{rgb}{0.000000,0.000000,0.000000}%
\pgfsetstrokecolor{currentstroke}%
\pgfsetdash{}{0pt}%
\pgfpathmoveto{\pgfqpoint{4.317652in}{3.188857in}}%
\pgfpathlineto{\pgfqpoint{4.330516in}{3.180833in}}%
\pgfpathlineto{\pgfqpoint{4.343383in}{3.172992in}}%
\pgfpathlineto{\pgfqpoint{4.356255in}{3.165333in}}%
\pgfpathlineto{\pgfqpoint{4.369132in}{3.157855in}}%
\pgfpathlineto{\pgfqpoint{4.376495in}{3.175289in}}%
\pgfpathlineto{\pgfqpoint{4.383857in}{3.192992in}}%
\pgfpathlineto{\pgfqpoint{4.391219in}{3.210973in}}%
\pgfpathlineto{\pgfqpoint{4.398580in}{3.229235in}}%
\pgfpathlineto{\pgfqpoint{4.385711in}{3.237262in}}%
\pgfpathlineto{\pgfqpoint{4.372845in}{3.245470in}}%
\pgfpathlineto{\pgfqpoint{4.359984in}{3.253860in}}%
\pgfpathlineto{\pgfqpoint{4.347128in}{3.262433in}}%
\pgfpathlineto{\pgfqpoint{4.339760in}{3.243610in}}%
\pgfpathlineto{\pgfqpoint{4.332391in}{3.225078in}}%
\pgfpathlineto{\pgfqpoint{4.325022in}{3.206829in}}%
\pgfpathlineto{\pgfqpoint{4.317652in}{3.188857in}}%
\pgfpathclose%
\pgfusepath{fill}%
\end{pgfscope}%
\begin{pgfscope}%
\pgfpathrectangle{\pgfqpoint{1.254980in}{0.150000in}}{\pgfqpoint{5.490039in}{5.490039in}}%
\pgfusepath{clip}%
\pgfsetbuttcap%
\pgfsetroundjoin%
\definecolor{currentfill}{rgb}{0.183898,0.422383,0.556944}%
\pgfsetfillcolor{currentfill}%
\pgfsetfillopacity{0.700000}%
\pgfsetlinewidth{0.000000pt}%
\definecolor{currentstroke}{rgb}{0.000000,0.000000,0.000000}%
\pgfsetstrokecolor{currentstroke}%
\pgfsetdash{}{0pt}%
\pgfpathmoveto{\pgfqpoint{4.479523in}{3.272631in}}%
\pgfpathlineto{\pgfqpoint{4.492409in}{3.264922in}}%
\pgfpathlineto{\pgfqpoint{4.505300in}{3.257389in}}%
\pgfpathlineto{\pgfqpoint{4.518197in}{3.250032in}}%
\pgfpathlineto{\pgfqpoint{4.531099in}{3.242849in}}%
\pgfpathlineto{\pgfqpoint{4.538446in}{3.261439in}}%
\pgfpathlineto{\pgfqpoint{4.545794in}{3.280339in}}%
\pgfpathlineto{\pgfqpoint{4.553143in}{3.299556in}}%
\pgfpathlineto{\pgfqpoint{4.560494in}{3.319096in}}%
\pgfpathlineto{\pgfqpoint{4.547600in}{3.326882in}}%
\pgfpathlineto{\pgfqpoint{4.534711in}{3.334844in}}%
\pgfpathlineto{\pgfqpoint{4.521827in}{3.342981in}}%
\pgfpathlineto{\pgfqpoint{4.508948in}{3.351295in}}%
\pgfpathlineto{\pgfqpoint{4.501590in}{3.331140in}}%
\pgfpathlineto{\pgfqpoint{4.494234in}{3.311316in}}%
\pgfpathlineto{\pgfqpoint{4.486878in}{3.291815in}}%
\pgfpathlineto{\pgfqpoint{4.479523in}{3.272631in}}%
\pgfpathclose%
\pgfusepath{fill}%
\end{pgfscope}%
\begin{pgfscope}%
\pgfpathrectangle{\pgfqpoint{1.254980in}{0.150000in}}{\pgfqpoint{5.490039in}{5.490039in}}%
\pgfusepath{clip}%
\pgfsetbuttcap%
\pgfsetroundjoin%
\definecolor{currentfill}{rgb}{0.147607,0.511733,0.557049}%
\pgfsetfillcolor{currentfill}%
\pgfsetfillopacity{0.700000}%
\pgfsetlinewidth{0.000000pt}%
\definecolor{currentstroke}{rgb}{0.000000,0.000000,0.000000}%
\pgfsetstrokecolor{currentstroke}%
\pgfsetdash{}{0pt}%
\pgfpathmoveto{\pgfqpoint{3.450841in}{3.523004in}}%
\pgfpathlineto{\pgfqpoint{3.463711in}{3.505133in}}%
\pgfpathlineto{\pgfqpoint{3.476576in}{3.487527in}}%
\pgfpathlineto{\pgfqpoint{3.489437in}{3.470183in}}%
\pgfpathlineto{\pgfqpoint{3.502294in}{3.453099in}}%
\pgfpathlineto{\pgfqpoint{3.509776in}{3.472007in}}%
\pgfpathlineto{\pgfqpoint{3.517253in}{3.491173in}}%
\pgfpathlineto{\pgfqpoint{3.524725in}{3.510601in}}%
\pgfpathlineto{\pgfqpoint{3.532192in}{3.530297in}}%
\pgfpathlineto{\pgfqpoint{3.519338in}{3.547805in}}%
\pgfpathlineto{\pgfqpoint{3.506480in}{3.565574in}}%
\pgfpathlineto{\pgfqpoint{3.493618in}{3.583605in}}%
\pgfpathlineto{\pgfqpoint{3.480751in}{3.601902in}}%
\pgfpathlineto{\pgfqpoint{3.473281in}{3.581770in}}%
\pgfpathlineto{\pgfqpoint{3.465806in}{3.561913in}}%
\pgfpathlineto{\pgfqpoint{3.458326in}{3.542326in}}%
\pgfpathlineto{\pgfqpoint{3.450841in}{3.523004in}}%
\pgfpathclose%
\pgfusepath{fill}%
\end{pgfscope}%
\begin{pgfscope}%
\pgfpathrectangle{\pgfqpoint{1.254980in}{0.150000in}}{\pgfqpoint{5.490039in}{5.490039in}}%
\pgfusepath{clip}%
\pgfsetbuttcap%
\pgfsetroundjoin%
\definecolor{currentfill}{rgb}{0.206756,0.371758,0.553117}%
\pgfsetfillcolor{currentfill}%
\pgfsetfillopacity{0.700000}%
\pgfsetlinewidth{0.000000pt}%
\definecolor{currentstroke}{rgb}{0.000000,0.000000,0.000000}%
\pgfsetstrokecolor{currentstroke}%
\pgfsetdash{}{0pt}%
\pgfpathmoveto{\pgfqpoint{3.758883in}{3.162894in}}%
\pgfpathlineto{\pgfqpoint{3.771700in}{3.150814in}}%
\pgfpathlineto{\pgfqpoint{3.784517in}{3.138952in}}%
\pgfpathlineto{\pgfqpoint{3.797334in}{3.127308in}}%
\pgfpathlineto{\pgfqpoint{3.810152in}{3.115881in}}%
\pgfpathlineto{\pgfqpoint{3.817608in}{3.132344in}}%
\pgfpathlineto{\pgfqpoint{3.825061in}{3.149022in}}%
\pgfpathlineto{\pgfqpoint{3.832510in}{3.165918in}}%
\pgfpathlineto{\pgfqpoint{3.839956in}{3.183039in}}%
\pgfpathlineto{\pgfqpoint{3.827143in}{3.194881in}}%
\pgfpathlineto{\pgfqpoint{3.814331in}{3.206939in}}%
\pgfpathlineto{\pgfqpoint{3.801518in}{3.219215in}}%
\pgfpathlineto{\pgfqpoint{3.788706in}{3.231710in}}%
\pgfpathlineto{\pgfqpoint{3.781256in}{3.214165in}}%
\pgfpathlineto{\pgfqpoint{3.773802in}{3.196850in}}%
\pgfpathlineto{\pgfqpoint{3.766345in}{3.179761in}}%
\pgfpathlineto{\pgfqpoint{3.758883in}{3.162894in}}%
\pgfpathclose%
\pgfusepath{fill}%
\end{pgfscope}%
\begin{pgfscope}%
\pgfpathrectangle{\pgfqpoint{1.254980in}{0.150000in}}{\pgfqpoint{5.490039in}{5.490039in}}%
\pgfusepath{clip}%
\pgfsetbuttcap%
\pgfsetroundjoin%
\definecolor{currentfill}{rgb}{0.506271,0.828786,0.300362}%
\pgfsetfillcolor{currentfill}%
\pgfsetfillopacity{0.700000}%
\pgfsetlinewidth{0.000000pt}%
\definecolor{currentstroke}{rgb}{0.000000,0.000000,0.000000}%
\pgfsetstrokecolor{currentstroke}%
\pgfsetdash{}{0pt}%
\pgfpathmoveto{\pgfqpoint{3.636707in}{4.392557in}}%
\pgfpathlineto{\pgfqpoint{3.649606in}{4.369278in}}%
\pgfpathlineto{\pgfqpoint{3.662499in}{4.346281in}}%
\pgfpathlineto{\pgfqpoint{3.675386in}{4.323566in}}%
\pgfpathlineto{\pgfqpoint{3.688268in}{4.301127in}}%
\pgfpathlineto{\pgfqpoint{3.695634in}{4.331898in}}%
\pgfpathlineto{\pgfqpoint{3.702999in}{4.363142in}}%
\pgfpathlineto{\pgfqpoint{3.710363in}{4.394869in}}%
\pgfpathlineto{\pgfqpoint{3.717724in}{4.427087in}}%
\pgfpathlineto{\pgfqpoint{3.704837in}{4.450214in}}%
\pgfpathlineto{\pgfqpoint{3.691944in}{4.473620in}}%
\pgfpathlineto{\pgfqpoint{3.679045in}{4.497308in}}%
\pgfpathlineto{\pgfqpoint{3.666139in}{4.521280in}}%
\pgfpathlineto{\pgfqpoint{3.658784in}{4.488358in}}%
\pgfpathlineto{\pgfqpoint{3.651427in}{4.455935in}}%
\pgfpathlineto{\pgfqpoint{3.644068in}{4.424004in}}%
\pgfpathlineto{\pgfqpoint{3.636707in}{4.392557in}}%
\pgfpathclose%
\pgfusepath{fill}%
\end{pgfscope}%
\begin{pgfscope}%
\pgfpathrectangle{\pgfqpoint{1.254980in}{0.150000in}}{\pgfqpoint{5.490039in}{5.490039in}}%
\pgfusepath{clip}%
\pgfsetbuttcap%
\pgfsetroundjoin%
\definecolor{currentfill}{rgb}{0.206756,0.371758,0.553117}%
\pgfsetfillcolor{currentfill}%
\pgfsetfillopacity{0.700000}%
\pgfsetlinewidth{0.000000pt}%
\definecolor{currentstroke}{rgb}{0.000000,0.000000,0.000000}%
\pgfsetstrokecolor{currentstroke}%
\pgfsetdash{}{0pt}%
\pgfpathmoveto{\pgfqpoint{4.236724in}{3.151467in}}%
\pgfpathlineto{\pgfqpoint{4.249578in}{3.143225in}}%
\pgfpathlineto{\pgfqpoint{4.262435in}{3.135169in}}%
\pgfpathlineto{\pgfqpoint{4.275297in}{3.127299in}}%
\pgfpathlineto{\pgfqpoint{4.288162in}{3.119612in}}%
\pgfpathlineto{\pgfqpoint{4.295537in}{3.136539in}}%
\pgfpathlineto{\pgfqpoint{4.302910in}{3.153718in}}%
\pgfpathlineto{\pgfqpoint{4.310282in}{3.171155in}}%
\pgfpathlineto{\pgfqpoint{4.317652in}{3.188857in}}%
\pgfpathlineto{\pgfqpoint{4.304793in}{3.197064in}}%
\pgfpathlineto{\pgfqpoint{4.291938in}{3.205456in}}%
\pgfpathlineto{\pgfqpoint{4.279087in}{3.214033in}}%
\pgfpathlineto{\pgfqpoint{4.266240in}{3.222797in}}%
\pgfpathlineto{\pgfqpoint{4.258863in}{3.204563in}}%
\pgfpathlineto{\pgfqpoint{4.251485in}{3.186602in}}%
\pgfpathlineto{\pgfqpoint{4.244105in}{3.168905in}}%
\pgfpathlineto{\pgfqpoint{4.236724in}{3.151467in}}%
\pgfpathclose%
\pgfusepath{fill}%
\end{pgfscope}%
\begin{pgfscope}%
\pgfpathrectangle{\pgfqpoint{1.254980in}{0.150000in}}{\pgfqpoint{5.490039in}{5.490039in}}%
\pgfusepath{clip}%
\pgfsetbuttcap%
\pgfsetroundjoin%
\definecolor{currentfill}{rgb}{0.175841,0.441290,0.557685}%
\pgfsetfillcolor{currentfill}%
\pgfsetfillopacity{0.700000}%
\pgfsetlinewidth{0.000000pt}%
\definecolor{currentstroke}{rgb}{0.000000,0.000000,0.000000}%
\pgfsetstrokecolor{currentstroke}%
\pgfsetdash{}{0pt}%
\pgfpathmoveto{\pgfqpoint{4.560494in}{3.319096in}}%
\pgfpathlineto{\pgfqpoint{4.573393in}{3.311484in}}%
\pgfpathlineto{\pgfqpoint{4.586297in}{3.304046in}}%
\pgfpathlineto{\pgfqpoint{4.599207in}{3.296780in}}%
\pgfpathlineto{\pgfqpoint{4.612123in}{3.289687in}}%
\pgfpathlineto{\pgfqpoint{4.619466in}{3.308937in}}%
\pgfpathlineto{\pgfqpoint{4.626812in}{3.328518in}}%
\pgfpathlineto{\pgfqpoint{4.634159in}{3.348438in}}%
\pgfpathlineto{\pgfqpoint{4.641509in}{3.368705in}}%
\pgfpathlineto{\pgfqpoint{4.628602in}{3.376430in}}%
\pgfpathlineto{\pgfqpoint{4.615700in}{3.384328in}}%
\pgfpathlineto{\pgfqpoint{4.602804in}{3.392399in}}%
\pgfpathlineto{\pgfqpoint{4.589913in}{3.400644in}}%
\pgfpathlineto{\pgfqpoint{4.582555in}{3.379734in}}%
\pgfpathlineto{\pgfqpoint{4.575199in}{3.359178in}}%
\pgfpathlineto{\pgfqpoint{4.567846in}{3.338968in}}%
\pgfpathlineto{\pgfqpoint{4.560494in}{3.319096in}}%
\pgfpathclose%
\pgfusepath{fill}%
\end{pgfscope}%
\begin{pgfscope}%
\pgfpathrectangle{\pgfqpoint{1.254980in}{0.150000in}}{\pgfqpoint{5.490039in}{5.490039in}}%
\pgfusepath{clip}%
\pgfsetbuttcap%
\pgfsetroundjoin%
\definecolor{currentfill}{rgb}{0.214298,0.355619,0.551184}%
\pgfsetfillcolor{currentfill}%
\pgfsetfillopacity{0.700000}%
\pgfsetlinewidth{0.000000pt}%
\definecolor{currentstroke}{rgb}{0.000000,0.000000,0.000000}%
\pgfsetstrokecolor{currentstroke}%
\pgfsetdash{}{0pt}%
\pgfpathmoveto{\pgfqpoint{4.155782in}{3.117063in}}%
\pgfpathlineto{\pgfqpoint{4.168627in}{3.108560in}}%
\pgfpathlineto{\pgfqpoint{4.181476in}{3.100248in}}%
\pgfpathlineto{\pgfqpoint{4.194328in}{3.092124in}}%
\pgfpathlineto{\pgfqpoint{4.207184in}{3.084188in}}%
\pgfpathlineto{\pgfqpoint{4.214572in}{3.100649in}}%
\pgfpathlineto{\pgfqpoint{4.221958in}{3.117345in}}%
\pgfpathlineto{\pgfqpoint{4.229342in}{3.134283in}}%
\pgfpathlineto{\pgfqpoint{4.236724in}{3.151467in}}%
\pgfpathlineto{\pgfqpoint{4.223875in}{3.159897in}}%
\pgfpathlineto{\pgfqpoint{4.211029in}{3.168514in}}%
\pgfpathlineto{\pgfqpoint{4.198187in}{3.177320in}}%
\pgfpathlineto{\pgfqpoint{4.185348in}{3.186317in}}%
\pgfpathlineto{\pgfqpoint{4.177959in}{3.168628in}}%
\pgfpathlineto{\pgfqpoint{4.170569in}{3.151193in}}%
\pgfpathlineto{\pgfqpoint{4.163177in}{3.134007in}}%
\pgfpathlineto{\pgfqpoint{4.155782in}{3.117063in}}%
\pgfpathclose%
\pgfusepath{fill}%
\end{pgfscope}%
\begin{pgfscope}%
\pgfpathrectangle{\pgfqpoint{1.254980in}{0.150000in}}{\pgfqpoint{5.490039in}{5.490039in}}%
\pgfusepath{clip}%
\pgfsetbuttcap%
\pgfsetroundjoin%
\definecolor{currentfill}{rgb}{0.119738,0.603785,0.541400}%
\pgfsetfillcolor{currentfill}%
\pgfsetfillopacity{0.700000}%
\pgfsetlinewidth{0.000000pt}%
\definecolor{currentstroke}{rgb}{0.000000,0.000000,0.000000}%
\pgfsetstrokecolor{currentstroke}%
\pgfsetdash{}{0pt}%
\pgfpathmoveto{\pgfqpoint{3.377623in}{3.758130in}}%
\pgfpathlineto{\pgfqpoint{3.390535in}{3.737620in}}%
\pgfpathlineto{\pgfqpoint{3.403440in}{3.717396in}}%
\pgfpathlineto{\pgfqpoint{3.416340in}{3.697455in}}%
\pgfpathlineto{\pgfqpoint{3.429233in}{3.677794in}}%
\pgfpathlineto{\pgfqpoint{3.436700in}{3.698654in}}%
\pgfpathlineto{\pgfqpoint{3.444163in}{3.719807in}}%
\pgfpathlineto{\pgfqpoint{3.451620in}{3.741259in}}%
\pgfpathlineto{\pgfqpoint{3.459073in}{3.763016in}}%
\pgfpathlineto{\pgfqpoint{3.446181in}{3.783138in}}%
\pgfpathlineto{\pgfqpoint{3.433283in}{3.803540in}}%
\pgfpathlineto{\pgfqpoint{3.420379in}{3.824226in}}%
\pgfpathlineto{\pgfqpoint{3.407469in}{3.845199in}}%
\pgfpathlineto{\pgfqpoint{3.400015in}{3.822969in}}%
\pgfpathlineto{\pgfqpoint{3.392556in}{3.801051in}}%
\pgfpathlineto{\pgfqpoint{3.385092in}{3.779440in}}%
\pgfpathlineto{\pgfqpoint{3.377623in}{3.758130in}}%
\pgfpathclose%
\pgfusepath{fill}%
\end{pgfscope}%
\begin{pgfscope}%
\pgfpathrectangle{\pgfqpoint{1.254980in}{0.150000in}}{\pgfqpoint{5.490039in}{5.490039in}}%
\pgfusepath{clip}%
\pgfsetbuttcap%
\pgfsetroundjoin%
\definecolor{currentfill}{rgb}{0.135066,0.544853,0.554029}%
\pgfsetfillcolor{currentfill}%
\pgfsetfillopacity{0.700000}%
\pgfsetlinewidth{0.000000pt}%
\definecolor{currentstroke}{rgb}{0.000000,0.000000,0.000000}%
\pgfsetstrokecolor{currentstroke}%
\pgfsetdash{}{0pt}%
\pgfpathmoveto{\pgfqpoint{3.399312in}{3.597185in}}%
\pgfpathlineto{\pgfqpoint{3.412202in}{3.578230in}}%
\pgfpathlineto{\pgfqpoint{3.425087in}{3.559550in}}%
\pgfpathlineto{\pgfqpoint{3.437967in}{3.541142in}}%
\pgfpathlineto{\pgfqpoint{3.450841in}{3.523004in}}%
\pgfpathlineto{\pgfqpoint{3.458326in}{3.542326in}}%
\pgfpathlineto{\pgfqpoint{3.465806in}{3.561913in}}%
\pgfpathlineto{\pgfqpoint{3.473281in}{3.581770in}}%
\pgfpathlineto{\pgfqpoint{3.480751in}{3.601902in}}%
\pgfpathlineto{\pgfqpoint{3.467879in}{3.620467in}}%
\pgfpathlineto{\pgfqpoint{3.455003in}{3.639303in}}%
\pgfpathlineto{\pgfqpoint{3.442120in}{3.658411in}}%
\pgfpathlineto{\pgfqpoint{3.429233in}{3.677794in}}%
\pgfpathlineto{\pgfqpoint{3.421760in}{3.657222in}}%
\pgfpathlineto{\pgfqpoint{3.414283in}{3.636933in}}%
\pgfpathlineto{\pgfqpoint{3.406800in}{3.616923in}}%
\pgfpathlineto{\pgfqpoint{3.399312in}{3.597185in}}%
\pgfpathclose%
\pgfusepath{fill}%
\end{pgfscope}%
\begin{pgfscope}%
\pgfpathrectangle{\pgfqpoint{1.254980in}{0.150000in}}{\pgfqpoint{5.490039in}{5.490039in}}%
\pgfusepath{clip}%
\pgfsetbuttcap%
\pgfsetroundjoin%
\definecolor{currentfill}{rgb}{0.218130,0.347432,0.550038}%
\pgfsetfillcolor{currentfill}%
\pgfsetfillopacity{0.700000}%
\pgfsetlinewidth{0.000000pt}%
\definecolor{currentstroke}{rgb}{0.000000,0.000000,0.000000}%
\pgfsetstrokecolor{currentstroke}%
\pgfsetdash{}{0pt}%
\pgfpathmoveto{\pgfqpoint{3.942492in}{3.095924in}}%
\pgfpathlineto{\pgfqpoint{3.955316in}{3.085967in}}%
\pgfpathlineto{\pgfqpoint{3.968142in}{3.076213in}}%
\pgfpathlineto{\pgfqpoint{3.980969in}{3.066660in}}%
\pgfpathlineto{\pgfqpoint{3.993799in}{3.057308in}}%
\pgfpathlineto{\pgfqpoint{4.001226in}{3.073368in}}%
\pgfpathlineto{\pgfqpoint{4.008650in}{3.089640in}}%
\pgfpathlineto{\pgfqpoint{4.016071in}{3.106131in}}%
\pgfpathlineto{\pgfqpoint{4.023489in}{3.122845in}}%
\pgfpathlineto{\pgfqpoint{4.010664in}{3.132637in}}%
\pgfpathlineto{\pgfqpoint{3.997842in}{3.142629in}}%
\pgfpathlineto{\pgfqpoint{3.985022in}{3.152823in}}%
\pgfpathlineto{\pgfqpoint{3.972203in}{3.163220in}}%
\pgfpathlineto{\pgfqpoint{3.964780in}{3.146056in}}%
\pgfpathlineto{\pgfqpoint{3.957354in}{3.129122in}}%
\pgfpathlineto{\pgfqpoint{3.949925in}{3.112413in}}%
\pgfpathlineto{\pgfqpoint{3.942492in}{3.095924in}}%
\pgfpathclose%
\pgfusepath{fill}%
\end{pgfscope}%
\begin{pgfscope}%
\pgfpathrectangle{\pgfqpoint{1.254980in}{0.150000in}}{\pgfqpoint{5.490039in}{5.490039in}}%
\pgfusepath{clip}%
\pgfsetbuttcap%
\pgfsetroundjoin%
\definecolor{currentfill}{rgb}{0.168126,0.459988,0.558082}%
\pgfsetfillcolor{currentfill}%
\pgfsetfillopacity{0.700000}%
\pgfsetlinewidth{0.000000pt}%
\definecolor{currentstroke}{rgb}{0.000000,0.000000,0.000000}%
\pgfsetstrokecolor{currentstroke}%
\pgfsetdash{}{0pt}%
\pgfpathmoveto{\pgfqpoint{4.641509in}{3.368705in}}%
\pgfpathlineto{\pgfqpoint{4.654422in}{3.361151in}}%
\pgfpathlineto{\pgfqpoint{4.667340in}{3.353769in}}%
\pgfpathlineto{\pgfqpoint{4.680264in}{3.346556in}}%
\pgfpathlineto{\pgfqpoint{4.693193in}{3.339514in}}%
\pgfpathlineto{\pgfqpoint{4.700537in}{3.359485in}}%
\pgfpathlineto{\pgfqpoint{4.707883in}{3.379811in}}%
\pgfpathlineto{\pgfqpoint{4.715232in}{3.400499in}}%
\pgfpathlineto{\pgfqpoint{4.702309in}{3.408034in}}%
\pgfpathlineto{\pgfqpoint{4.689392in}{3.415739in}}%
\pgfpathlineto{\pgfqpoint{4.676480in}{3.423614in}}%
\pgfpathlineto{\pgfqpoint{4.663574in}{3.431661in}}%
\pgfpathlineto{\pgfqpoint{4.656216in}{3.410308in}}%
\pgfpathlineto{\pgfqpoint{4.648861in}{3.389326in}}%
\pgfpathlineto{\pgfqpoint{4.641509in}{3.368705in}}%
\pgfpathclose%
\pgfusepath{fill}%
\end{pgfscope}%
\begin{pgfscope}%
\pgfpathrectangle{\pgfqpoint{1.254980in}{0.150000in}}{\pgfqpoint{5.490039in}{5.490039in}}%
\pgfusepath{clip}%
\pgfsetbuttcap%
\pgfsetroundjoin%
\definecolor{currentfill}{rgb}{0.477504,0.821444,0.318195}%
\pgfsetfillcolor{currentfill}%
\pgfsetfillopacity{0.700000}%
\pgfsetlinewidth{0.000000pt}%
\definecolor{currentstroke}{rgb}{0.000000,0.000000,0.000000}%
\pgfsetstrokecolor{currentstroke}%
\pgfsetdash{}{0pt}%
\pgfpathmoveto{\pgfqpoint{3.555596in}{4.364790in}}%
\pgfpathlineto{\pgfqpoint{3.568517in}{4.341015in}}%
\pgfpathlineto{\pgfqpoint{3.581431in}{4.317533in}}%
\pgfpathlineto{\pgfqpoint{3.594338in}{4.294340in}}%
\pgfpathlineto{\pgfqpoint{3.607238in}{4.271434in}}%
\pgfpathlineto{\pgfqpoint{3.614609in}{4.301031in}}%
\pgfpathlineto{\pgfqpoint{3.621978in}{4.331078in}}%
\pgfpathlineto{\pgfqpoint{3.629344in}{4.361584in}}%
\pgfpathlineto{\pgfqpoint{3.636707in}{4.392557in}}%
\pgfpathlineto{\pgfqpoint{3.623801in}{4.416121in}}%
\pgfpathlineto{\pgfqpoint{3.610889in}{4.439974in}}%
\pgfpathlineto{\pgfqpoint{3.597969in}{4.464118in}}%
\pgfpathlineto{\pgfqpoint{3.585043in}{4.488557in}}%
\pgfpathlineto{\pgfqpoint{3.577685in}{4.456909in}}%
\pgfpathlineto{\pgfqpoint{3.570325in}{4.425738in}}%
\pgfpathlineto{\pgfqpoint{3.562962in}{4.395034in}}%
\pgfpathlineto{\pgfqpoint{3.555596in}{4.364790in}}%
\pgfpathclose%
\pgfusepath{fill}%
\end{pgfscope}%
\begin{pgfscope}%
\pgfpathrectangle{\pgfqpoint{1.254980in}{0.150000in}}{\pgfqpoint{5.490039in}{5.490039in}}%
\pgfusepath{clip}%
\pgfsetbuttcap%
\pgfsetroundjoin%
\definecolor{currentfill}{rgb}{0.214298,0.355619,0.551184}%
\pgfsetfillcolor{currentfill}%
\pgfsetfillopacity{0.700000}%
\pgfsetlinewidth{0.000000pt}%
\definecolor{currentstroke}{rgb}{0.000000,0.000000,0.000000}%
\pgfsetstrokecolor{currentstroke}%
\pgfsetdash{}{0pt}%
\pgfpathmoveto{\pgfqpoint{3.810152in}{3.115881in}}%
\pgfpathlineto{\pgfqpoint{3.822970in}{3.104667in}}%
\pgfpathlineto{\pgfqpoint{3.835789in}{3.093668in}}%
\pgfpathlineto{\pgfqpoint{3.848609in}{3.082880in}}%
\pgfpathlineto{\pgfqpoint{3.861430in}{3.072302in}}%
\pgfpathlineto{\pgfqpoint{3.868882in}{3.088363in}}%
\pgfpathlineto{\pgfqpoint{3.876329in}{3.104630in}}%
\pgfpathlineto{\pgfqpoint{3.883774in}{3.121110in}}%
\pgfpathlineto{\pgfqpoint{3.891215in}{3.137807in}}%
\pgfpathlineto{\pgfqpoint{3.878398in}{3.148798in}}%
\pgfpathlineto{\pgfqpoint{3.865584in}{3.159999in}}%
\pgfpathlineto{\pgfqpoint{3.852769in}{3.171412in}}%
\pgfpathlineto{\pgfqpoint{3.839956in}{3.183039in}}%
\pgfpathlineto{\pgfqpoint{3.832510in}{3.165918in}}%
\pgfpathlineto{\pgfqpoint{3.825061in}{3.149022in}}%
\pgfpathlineto{\pgfqpoint{3.817608in}{3.132344in}}%
\pgfpathlineto{\pgfqpoint{3.810152in}{3.115881in}}%
\pgfpathclose%
\pgfusepath{fill}%
\end{pgfscope}%
\begin{pgfscope}%
\pgfpathrectangle{\pgfqpoint{1.254980in}{0.150000in}}{\pgfqpoint{5.490039in}{5.490039in}}%
\pgfusepath{clip}%
\pgfsetbuttcap%
\pgfsetroundjoin%
\definecolor{currentfill}{rgb}{0.179019,0.433756,0.557430}%
\pgfsetfillcolor{currentfill}%
\pgfsetfillopacity{0.700000}%
\pgfsetlinewidth{0.000000pt}%
\definecolor{currentstroke}{rgb}{0.000000,0.000000,0.000000}%
\pgfsetstrokecolor{currentstroke}%
\pgfsetdash{}{0pt}%
\pgfpathmoveto{\pgfqpoint{3.523726in}{3.315745in}}%
\pgfpathlineto{\pgfqpoint{3.536569in}{3.300322in}}%
\pgfpathlineto{\pgfqpoint{3.549410in}{3.285146in}}%
\pgfpathlineto{\pgfqpoint{3.562248in}{3.270214in}}%
\pgfpathlineto{\pgfqpoint{3.575084in}{3.255526in}}%
\pgfpathlineto{\pgfqpoint{3.582577in}{3.272688in}}%
\pgfpathlineto{\pgfqpoint{3.590064in}{3.290075in}}%
\pgfpathlineto{\pgfqpoint{3.597548in}{3.307690in}}%
\pgfpathlineto{\pgfqpoint{3.605027in}{3.325538in}}%
\pgfpathlineto{\pgfqpoint{3.592195in}{3.340617in}}%
\pgfpathlineto{\pgfqpoint{3.579361in}{3.355940in}}%
\pgfpathlineto{\pgfqpoint{3.566524in}{3.371508in}}%
\pgfpathlineto{\pgfqpoint{3.553685in}{3.387323in}}%
\pgfpathlineto{\pgfqpoint{3.546202in}{3.369072in}}%
\pgfpathlineto{\pgfqpoint{3.538715in}{3.351062in}}%
\pgfpathlineto{\pgfqpoint{3.531222in}{3.333288in}}%
\pgfpathlineto{\pgfqpoint{3.523726in}{3.315745in}}%
\pgfpathclose%
\pgfusepath{fill}%
\end{pgfscope}%
\begin{pgfscope}%
\pgfpathrectangle{\pgfqpoint{1.254980in}{0.150000in}}{\pgfqpoint{5.490039in}{5.490039in}}%
\pgfusepath{clip}%
\pgfsetbuttcap%
\pgfsetroundjoin%
\definecolor{currentfill}{rgb}{0.188923,0.410910,0.556326}%
\pgfsetfillcolor{currentfill}%
\pgfsetfillopacity{0.700000}%
\pgfsetlinewidth{0.000000pt}%
\definecolor{currentstroke}{rgb}{0.000000,0.000000,0.000000}%
\pgfsetstrokecolor{currentstroke}%
\pgfsetdash{}{0pt}%
\pgfpathmoveto{\pgfqpoint{3.575084in}{3.255526in}}%
\pgfpathlineto{\pgfqpoint{3.587918in}{3.241078in}}%
\pgfpathlineto{\pgfqpoint{3.600749in}{3.226870in}}%
\pgfpathlineto{\pgfqpoint{3.613579in}{3.212899in}}%
\pgfpathlineto{\pgfqpoint{3.626407in}{3.199163in}}%
\pgfpathlineto{\pgfqpoint{3.633895in}{3.215946in}}%
\pgfpathlineto{\pgfqpoint{3.641378in}{3.232947in}}%
\pgfpathlineto{\pgfqpoint{3.648857in}{3.250168in}}%
\pgfpathlineto{\pgfqpoint{3.656332in}{3.267616in}}%
\pgfpathlineto{\pgfqpoint{3.643508in}{3.281741in}}%
\pgfpathlineto{\pgfqpoint{3.630683in}{3.296102in}}%
\pgfpathlineto{\pgfqpoint{3.617856in}{3.310700in}}%
\pgfpathlineto{\pgfqpoint{3.605027in}{3.325538in}}%
\pgfpathlineto{\pgfqpoint{3.597548in}{3.307690in}}%
\pgfpathlineto{\pgfqpoint{3.590064in}{3.290075in}}%
\pgfpathlineto{\pgfqpoint{3.582577in}{3.272688in}}%
\pgfpathlineto{\pgfqpoint{3.575084in}{3.255526in}}%
\pgfpathclose%
\pgfusepath{fill}%
\end{pgfscope}%
\begin{pgfscope}%
\pgfpathrectangle{\pgfqpoint{1.254980in}{0.150000in}}{\pgfqpoint{5.490039in}{5.490039in}}%
\pgfusepath{clip}%
\pgfsetbuttcap%
\pgfsetroundjoin%
\definecolor{currentfill}{rgb}{0.259857,0.745492,0.444467}%
\pgfsetfillcolor{currentfill}%
\pgfsetfillopacity{0.700000}%
\pgfsetlinewidth{0.000000pt}%
\definecolor{currentstroke}{rgb}{0.000000,0.000000,0.000000}%
\pgfsetstrokecolor{currentstroke}%
\pgfsetdash{}{0pt}%
\pgfpathmoveto{\pgfqpoint{3.415209in}{4.125899in}}%
\pgfpathlineto{\pgfqpoint{3.428149in}{4.102722in}}%
\pgfpathlineto{\pgfqpoint{3.441081in}{4.079845in}}%
\pgfpathlineto{\pgfqpoint{3.454006in}{4.057266in}}%
\pgfpathlineto{\pgfqpoint{3.466924in}{4.034981in}}%
\pgfpathlineto{\pgfqpoint{3.474336in}{4.060287in}}%
\pgfpathlineto{\pgfqpoint{3.481743in}{4.085966in}}%
\pgfpathlineto{\pgfqpoint{3.489145in}{4.112025in}}%
\pgfpathlineto{\pgfqpoint{3.496544in}{4.138471in}}%
\pgfpathlineto{\pgfqpoint{3.483624in}{4.161317in}}%
\pgfpathlineto{\pgfqpoint{3.470697in}{4.184459in}}%
\pgfpathlineto{\pgfqpoint{3.457762in}{4.207898in}}%
\pgfpathlineto{\pgfqpoint{3.444819in}{4.231640in}}%
\pgfpathlineto{\pgfqpoint{3.437423in}{4.204619in}}%
\pgfpathlineto{\pgfqpoint{3.430023in}{4.177993in}}%
\pgfpathlineto{\pgfqpoint{3.422618in}{4.151755in}}%
\pgfpathlineto{\pgfqpoint{3.415209in}{4.125899in}}%
\pgfpathclose%
\pgfusepath{fill}%
\end{pgfscope}%
\begin{pgfscope}%
\pgfpathrectangle{\pgfqpoint{1.254980in}{0.150000in}}{\pgfqpoint{5.490039in}{5.490039in}}%
\pgfusepath{clip}%
\pgfsetbuttcap%
\pgfsetroundjoin%
\definecolor{currentfill}{rgb}{0.191090,0.708366,0.482284}%
\pgfsetfillcolor{currentfill}%
\pgfsetfillopacity{0.700000}%
\pgfsetlinewidth{0.000000pt}%
\definecolor{currentstroke}{rgb}{0.000000,0.000000,0.000000}%
\pgfsetstrokecolor{currentstroke}%
\pgfsetdash{}{0pt}%
\pgfpathmoveto{\pgfqpoint{3.385523in}{4.026158in}}%
\pgfpathlineto{\pgfqpoint{3.398462in}{4.003512in}}%
\pgfpathlineto{\pgfqpoint{3.411393in}{3.981165in}}%
\pgfpathlineto{\pgfqpoint{3.424317in}{3.959115in}}%
\pgfpathlineto{\pgfqpoint{3.437235in}{3.937357in}}%
\pgfpathlineto{\pgfqpoint{3.444664in}{3.961235in}}%
\pgfpathlineto{\pgfqpoint{3.452089in}{3.985461in}}%
\pgfpathlineto{\pgfqpoint{3.459509in}{4.010041in}}%
\pgfpathlineto{\pgfqpoint{3.466924in}{4.034981in}}%
\pgfpathlineto{\pgfqpoint{3.454006in}{4.057266in}}%
\pgfpathlineto{\pgfqpoint{3.441081in}{4.079845in}}%
\pgfpathlineto{\pgfqpoint{3.428149in}{4.102722in}}%
\pgfpathlineto{\pgfqpoint{3.415209in}{4.125899in}}%
\pgfpathlineto{\pgfqpoint{3.407794in}{4.100417in}}%
\pgfpathlineto{\pgfqpoint{3.400376in}{4.075304in}}%
\pgfpathlineto{\pgfqpoint{3.392952in}{4.050553in}}%
\pgfpathlineto{\pgfqpoint{3.385523in}{4.026158in}}%
\pgfpathclose%
\pgfusepath{fill}%
\end{pgfscope}%
\begin{pgfscope}%
\pgfpathrectangle{\pgfqpoint{1.254980in}{0.150000in}}{\pgfqpoint{5.490039in}{5.490039in}}%
\pgfusepath{clip}%
\pgfsetbuttcap%
\pgfsetroundjoin%
\definecolor{currentfill}{rgb}{0.168126,0.459988,0.558082}%
\pgfsetfillcolor{currentfill}%
\pgfsetfillopacity{0.700000}%
\pgfsetlinewidth{0.000000pt}%
\definecolor{currentstroke}{rgb}{0.000000,0.000000,0.000000}%
\pgfsetstrokecolor{currentstroke}%
\pgfsetdash{}{0pt}%
\pgfpathmoveto{\pgfqpoint{3.472319in}{3.379949in}}%
\pgfpathlineto{\pgfqpoint{3.485176in}{3.363517in}}%
\pgfpathlineto{\pgfqpoint{3.498029in}{3.347341in}}%
\pgfpathlineto{\pgfqpoint{3.510879in}{3.331417in}}%
\pgfpathlineto{\pgfqpoint{3.523726in}{3.315745in}}%
\pgfpathlineto{\pgfqpoint{3.531222in}{3.333288in}}%
\pgfpathlineto{\pgfqpoint{3.538715in}{3.351062in}}%
\pgfpathlineto{\pgfqpoint{3.546202in}{3.369072in}}%
\pgfpathlineto{\pgfqpoint{3.553685in}{3.387323in}}%
\pgfpathlineto{\pgfqpoint{3.540842in}{3.403388in}}%
\pgfpathlineto{\pgfqpoint{3.527996in}{3.419704in}}%
\pgfpathlineto{\pgfqpoint{3.515147in}{3.436274in}}%
\pgfpathlineto{\pgfqpoint{3.502294in}{3.453099in}}%
\pgfpathlineto{\pgfqpoint{3.494808in}{3.434444in}}%
\pgfpathlineto{\pgfqpoint{3.487316in}{3.416037in}}%
\pgfpathlineto{\pgfqpoint{3.479820in}{3.397874in}}%
\pgfpathlineto{\pgfqpoint{3.472319in}{3.379949in}}%
\pgfpathclose%
\pgfusepath{fill}%
\end{pgfscope}%
\begin{pgfscope}%
\pgfpathrectangle{\pgfqpoint{1.254980in}{0.150000in}}{\pgfqpoint{5.490039in}{5.490039in}}%
\pgfusepath{clip}%
\pgfsetbuttcap%
\pgfsetroundjoin%
\definecolor{currentfill}{rgb}{0.199430,0.387607,0.554642}%
\pgfsetfillcolor{currentfill}%
\pgfsetfillopacity{0.700000}%
\pgfsetlinewidth{0.000000pt}%
\definecolor{currentstroke}{rgb}{0.000000,0.000000,0.000000}%
\pgfsetstrokecolor{currentstroke}%
\pgfsetdash{}{0pt}%
\pgfpathmoveto{\pgfqpoint{3.626407in}{3.199163in}}%
\pgfpathlineto{\pgfqpoint{3.639234in}{3.185661in}}%
\pgfpathlineto{\pgfqpoint{3.652060in}{3.172391in}}%
\pgfpathlineto{\pgfqpoint{3.664884in}{3.159351in}}%
\pgfpathlineto{\pgfqpoint{3.677708in}{3.146539in}}%
\pgfpathlineto{\pgfqpoint{3.685191in}{3.162945in}}%
\pgfpathlineto{\pgfqpoint{3.692669in}{3.179560in}}%
\pgfpathlineto{\pgfqpoint{3.700143in}{3.196390in}}%
\pgfpathlineto{\pgfqpoint{3.707614in}{3.213438in}}%
\pgfpathlineto{\pgfqpoint{3.694795in}{3.226638in}}%
\pgfpathlineto{\pgfqpoint{3.681975in}{3.240066in}}%
\pgfpathlineto{\pgfqpoint{3.669154in}{3.253725in}}%
\pgfpathlineto{\pgfqpoint{3.656332in}{3.267616in}}%
\pgfpathlineto{\pgfqpoint{3.648857in}{3.250168in}}%
\pgfpathlineto{\pgfqpoint{3.641378in}{3.232947in}}%
\pgfpathlineto{\pgfqpoint{3.633895in}{3.215946in}}%
\pgfpathlineto{\pgfqpoint{3.626407in}{3.199163in}}%
\pgfpathclose%
\pgfusepath{fill}%
\end{pgfscope}%
\begin{pgfscope}%
\pgfpathrectangle{\pgfqpoint{1.254980in}{0.150000in}}{\pgfqpoint{5.490039in}{5.490039in}}%
\pgfusepath{clip}%
\pgfsetbuttcap%
\pgfsetroundjoin%
\definecolor{currentfill}{rgb}{0.220057,0.343307,0.549413}%
\pgfsetfillcolor{currentfill}%
\pgfsetfillopacity{0.700000}%
\pgfsetlinewidth{0.000000pt}%
\definecolor{currentstroke}{rgb}{0.000000,0.000000,0.000000}%
\pgfsetstrokecolor{currentstroke}%
\pgfsetdash{}{0pt}%
\pgfpathmoveto{\pgfqpoint{4.074812in}{3.085662in}}%
\pgfpathlineto{\pgfqpoint{4.087650in}{3.076856in}}%
\pgfpathlineto{\pgfqpoint{4.100491in}{3.068244in}}%
\pgfpathlineto{\pgfqpoint{4.113335in}{3.059825in}}%
\pgfpathlineto{\pgfqpoint{4.126183in}{3.051598in}}%
\pgfpathlineto{\pgfqpoint{4.133586in}{3.067629in}}%
\pgfpathlineto{\pgfqpoint{4.140987in}{3.083879in}}%
\pgfpathlineto{\pgfqpoint{4.148386in}{3.100355in}}%
\pgfpathlineto{\pgfqpoint{4.155782in}{3.117063in}}%
\pgfpathlineto{\pgfqpoint{4.142941in}{3.125756in}}%
\pgfpathlineto{\pgfqpoint{4.130103in}{3.134641in}}%
\pgfpathlineto{\pgfqpoint{4.117268in}{3.143719in}}%
\pgfpathlineto{\pgfqpoint{4.104436in}{3.152992in}}%
\pgfpathlineto{\pgfqpoint{4.097033in}{3.135807in}}%
\pgfpathlineto{\pgfqpoint{4.089629in}{3.118861in}}%
\pgfpathlineto{\pgfqpoint{4.082222in}{3.102148in}}%
\pgfpathlineto{\pgfqpoint{4.074812in}{3.085662in}}%
\pgfpathclose%
\pgfusepath{fill}%
\end{pgfscope}%
\begin{pgfscope}%
\pgfpathrectangle{\pgfqpoint{1.254980in}{0.150000in}}{\pgfqpoint{5.490039in}{5.490039in}}%
\pgfusepath{clip}%
\pgfsetbuttcap%
\pgfsetroundjoin%
\definecolor{currentfill}{rgb}{0.344074,0.780029,0.397381}%
\pgfsetfillcolor{currentfill}%
\pgfsetfillopacity{0.700000}%
\pgfsetlinewidth{0.000000pt}%
\definecolor{currentstroke}{rgb}{0.000000,0.000000,0.000000}%
\pgfsetstrokecolor{currentstroke}%
\pgfsetdash{}{0pt}%
\pgfpathmoveto{\pgfqpoint{3.444819in}{4.231640in}}%
\pgfpathlineto{\pgfqpoint{3.457762in}{4.207898in}}%
\pgfpathlineto{\pgfqpoint{3.470697in}{4.184459in}}%
\pgfpathlineto{\pgfqpoint{3.483624in}{4.161317in}}%
\pgfpathlineto{\pgfqpoint{3.496544in}{4.138471in}}%
\pgfpathlineto{\pgfqpoint{3.503939in}{4.165310in}}%
\pgfpathlineto{\pgfqpoint{3.511329in}{4.192549in}}%
\pgfpathlineto{\pgfqpoint{3.518716in}{4.220195in}}%
\pgfpathlineto{\pgfqpoint{3.526099in}{4.248256in}}%
\pgfpathlineto{\pgfqpoint{3.513176in}{4.271696in}}%
\pgfpathlineto{\pgfqpoint{3.500245in}{4.295434in}}%
\pgfpathlineto{\pgfqpoint{3.487306in}{4.319471in}}%
\pgfpathlineto{\pgfqpoint{3.474360in}{4.343811in}}%
\pgfpathlineto{\pgfqpoint{3.466981in}{4.315141in}}%
\pgfpathlineto{\pgfqpoint{3.459598in}{4.286894in}}%
\pgfpathlineto{\pgfqpoint{3.452211in}{4.259062in}}%
\pgfpathlineto{\pgfqpoint{3.444819in}{4.231640in}}%
\pgfpathclose%
\pgfusepath{fill}%
\end{pgfscope}%
\begin{pgfscope}%
\pgfpathrectangle{\pgfqpoint{1.254980in}{0.150000in}}{\pgfqpoint{5.490039in}{5.490039in}}%
\pgfusepath{clip}%
\pgfsetbuttcap%
\pgfsetroundjoin%
\definecolor{currentfill}{rgb}{0.146616,0.673050,0.508936}%
\pgfsetfillcolor{currentfill}%
\pgfsetfillopacity{0.700000}%
\pgfsetlinewidth{0.000000pt}%
\definecolor{currentstroke}{rgb}{0.000000,0.000000,0.000000}%
\pgfsetstrokecolor{currentstroke}%
\pgfsetdash{}{0pt}%
\pgfpathmoveto{\pgfqpoint{3.355757in}{3.932012in}}%
\pgfpathlineto{\pgfqpoint{3.368696in}{3.909865in}}%
\pgfpathlineto{\pgfqpoint{3.381627in}{3.888016in}}%
\pgfpathlineto{\pgfqpoint{3.394552in}{3.866461in}}%
\pgfpathlineto{\pgfqpoint{3.407469in}{3.845199in}}%
\pgfpathlineto{\pgfqpoint{3.414918in}{3.867747in}}%
\pgfpathlineto{\pgfqpoint{3.422361in}{3.890619in}}%
\pgfpathlineto{\pgfqpoint{3.429800in}{3.913820in}}%
\pgfpathlineto{\pgfqpoint{3.437235in}{3.937357in}}%
\pgfpathlineto{\pgfqpoint{3.424317in}{3.959115in}}%
\pgfpathlineto{\pgfqpoint{3.411393in}{3.981165in}}%
\pgfpathlineto{\pgfqpoint{3.398462in}{4.003512in}}%
\pgfpathlineto{\pgfqpoint{3.385523in}{4.026158in}}%
\pgfpathlineto{\pgfqpoint{3.378089in}{4.002112in}}%
\pgfpathlineto{\pgfqpoint{3.370650in}{3.978410in}}%
\pgfpathlineto{\pgfqpoint{3.363206in}{3.955045in}}%
\pgfpathlineto{\pgfqpoint{3.355757in}{3.932012in}}%
\pgfpathclose%
\pgfusepath{fill}%
\end{pgfscope}%
\begin{pgfscope}%
\pgfpathrectangle{\pgfqpoint{1.254980in}{0.150000in}}{\pgfqpoint{5.490039in}{5.490039in}}%
\pgfusepath{clip}%
\pgfsetbuttcap%
\pgfsetroundjoin%
\definecolor{currentfill}{rgb}{0.157729,0.485932,0.558013}%
\pgfsetfillcolor{currentfill}%
\pgfsetfillopacity{0.700000}%
\pgfsetlinewidth{0.000000pt}%
\definecolor{currentstroke}{rgb}{0.000000,0.000000,0.000000}%
\pgfsetstrokecolor{currentstroke}%
\pgfsetdash{}{0pt}%
\pgfpathmoveto{\pgfqpoint{3.420851in}{3.448272in}}%
\pgfpathlineto{\pgfqpoint{3.433724in}{3.430797in}}%
\pgfpathlineto{\pgfqpoint{3.446593in}{3.413586in}}%
\pgfpathlineto{\pgfqpoint{3.459458in}{3.396638in}}%
\pgfpathlineto{\pgfqpoint{3.472319in}{3.379949in}}%
\pgfpathlineto{\pgfqpoint{3.479820in}{3.397874in}}%
\pgfpathlineto{\pgfqpoint{3.487316in}{3.416037in}}%
\pgfpathlineto{\pgfqpoint{3.494808in}{3.434444in}}%
\pgfpathlineto{\pgfqpoint{3.502294in}{3.453099in}}%
\pgfpathlineto{\pgfqpoint{3.489437in}{3.470183in}}%
\pgfpathlineto{\pgfqpoint{3.476576in}{3.487527in}}%
\pgfpathlineto{\pgfqpoint{3.463711in}{3.505133in}}%
\pgfpathlineto{\pgfqpoint{3.450841in}{3.523004in}}%
\pgfpathlineto{\pgfqpoint{3.443351in}{3.503943in}}%
\pgfpathlineto{\pgfqpoint{3.435856in}{3.485137in}}%
\pgfpathlineto{\pgfqpoint{3.428356in}{3.466581in}}%
\pgfpathlineto{\pgfqpoint{3.420851in}{3.448272in}}%
\pgfpathclose%
\pgfusepath{fill}%
\end{pgfscope}%
\begin{pgfscope}%
\pgfpathrectangle{\pgfqpoint{1.254980in}{0.150000in}}{\pgfqpoint{5.490039in}{5.490039in}}%
\pgfusepath{clip}%
\pgfsetbuttcap%
\pgfsetroundjoin%
\definecolor{currentfill}{rgb}{0.626579,0.854645,0.223353}%
\pgfsetfillcolor{currentfill}%
\pgfsetfillopacity{0.700000}%
\pgfsetlinewidth{0.000000pt}%
\definecolor{currentstroke}{rgb}{0.000000,0.000000,0.000000}%
\pgfsetstrokecolor{currentstroke}%
\pgfsetdash{}{0pt}%
\pgfpathmoveto{\pgfqpoint{3.666139in}{4.521280in}}%
\pgfpathlineto{\pgfqpoint{3.679045in}{4.497308in}}%
\pgfpathlineto{\pgfqpoint{3.691944in}{4.473620in}}%
\pgfpathlineto{\pgfqpoint{3.704837in}{4.450214in}}%
\pgfpathlineto{\pgfqpoint{3.717724in}{4.427087in}}%
\pgfpathlineto{\pgfqpoint{3.725085in}{4.459805in}}%
\pgfpathlineto{\pgfqpoint{3.732444in}{4.493030in}}%
\pgfpathlineto{\pgfqpoint{3.739802in}{4.526773in}}%
\pgfpathlineto{\pgfqpoint{3.726909in}{4.550441in}}%
\pgfpathlineto{\pgfqpoint{3.714010in}{4.574389in}}%
\pgfpathlineto{\pgfqpoint{3.701105in}{4.598620in}}%
\pgfpathlineto{\pgfqpoint{3.688193in}{4.623136in}}%
\pgfpathlineto{\pgfqpoint{3.680843in}{4.588661in}}%
\pgfpathlineto{\pgfqpoint{3.673492in}{4.554712in}}%
\pgfpathlineto{\pgfqpoint{3.666139in}{4.521280in}}%
\pgfpathclose%
\pgfusepath{fill}%
\end{pgfscope}%
\begin{pgfscope}%
\pgfpathrectangle{\pgfqpoint{1.254980in}{0.150000in}}{\pgfqpoint{5.490039in}{5.490039in}}%
\pgfusepath{clip}%
\pgfsetbuttcap%
\pgfsetroundjoin%
\definecolor{currentfill}{rgb}{0.195860,0.395433,0.555276}%
\pgfsetfillcolor{currentfill}%
\pgfsetfillopacity{0.700000}%
\pgfsetlinewidth{0.000000pt}%
\definecolor{currentstroke}{rgb}{0.000000,0.000000,0.000000}%
\pgfsetstrokecolor{currentstroke}%
\pgfsetdash{}{0pt}%
\pgfpathmoveto{\pgfqpoint{4.450106in}{3.198923in}}%
\pgfpathlineto{\pgfqpoint{4.463000in}{3.191790in}}%
\pgfpathlineto{\pgfqpoint{4.475899in}{3.184833in}}%
\pgfpathlineto{\pgfqpoint{4.488804in}{3.178051in}}%
\pgfpathlineto{\pgfqpoint{4.501714in}{3.171445in}}%
\pgfpathlineto{\pgfqpoint{4.509060in}{3.188866in}}%
\pgfpathlineto{\pgfqpoint{4.516406in}{3.206569in}}%
\pgfpathlineto{\pgfqpoint{4.523752in}{3.224561in}}%
\pgfpathlineto{\pgfqpoint{4.531099in}{3.242849in}}%
\pgfpathlineto{\pgfqpoint{4.518197in}{3.250032in}}%
\pgfpathlineto{\pgfqpoint{4.505300in}{3.257389in}}%
\pgfpathlineto{\pgfqpoint{4.492409in}{3.264922in}}%
\pgfpathlineto{\pgfqpoint{4.479523in}{3.272631in}}%
\pgfpathlineto{\pgfqpoint{4.472168in}{3.253757in}}%
\pgfpathlineto{\pgfqpoint{4.464814in}{3.235186in}}%
\pgfpathlineto{\pgfqpoint{4.457460in}{3.216910in}}%
\pgfpathlineto{\pgfqpoint{4.450106in}{3.198923in}}%
\pgfpathclose%
\pgfusepath{fill}%
\end{pgfscope}%
\begin{pgfscope}%
\pgfpathrectangle{\pgfqpoint{1.254980in}{0.150000in}}{\pgfqpoint{5.490039in}{5.490039in}}%
\pgfusepath{clip}%
\pgfsetbuttcap%
\pgfsetroundjoin%
\definecolor{currentfill}{rgb}{0.208623,0.367752,0.552675}%
\pgfsetfillcolor{currentfill}%
\pgfsetfillopacity{0.700000}%
\pgfsetlinewidth{0.000000pt}%
\definecolor{currentstroke}{rgb}{0.000000,0.000000,0.000000}%
\pgfsetstrokecolor{currentstroke}%
\pgfsetdash{}{0pt}%
\pgfpathmoveto{\pgfqpoint{3.677708in}{3.146539in}}%
\pgfpathlineto{\pgfqpoint{3.690531in}{3.133954in}}%
\pgfpathlineto{\pgfqpoint{3.703353in}{3.121594in}}%
\pgfpathlineto{\pgfqpoint{3.716175in}{3.109458in}}%
\pgfpathlineto{\pgfqpoint{3.728997in}{3.097543in}}%
\pgfpathlineto{\pgfqpoint{3.736475in}{3.113573in}}%
\pgfpathlineto{\pgfqpoint{3.743948in}{3.129804in}}%
\pgfpathlineto{\pgfqpoint{3.751418in}{3.146243in}}%
\pgfpathlineto{\pgfqpoint{3.758883in}{3.162894in}}%
\pgfpathlineto{\pgfqpoint{3.746066in}{3.175195in}}%
\pgfpathlineto{\pgfqpoint{3.733249in}{3.187719in}}%
\pgfpathlineto{\pgfqpoint{3.720432in}{3.200466in}}%
\pgfpathlineto{\pgfqpoint{3.707614in}{3.213438in}}%
\pgfpathlineto{\pgfqpoint{3.700143in}{3.196390in}}%
\pgfpathlineto{\pgfqpoint{3.692669in}{3.179560in}}%
\pgfpathlineto{\pgfqpoint{3.685191in}{3.162945in}}%
\pgfpathlineto{\pgfqpoint{3.677708in}{3.146539in}}%
\pgfpathclose%
\pgfusepath{fill}%
\end{pgfscope}%
\begin{pgfscope}%
\pgfpathrectangle{\pgfqpoint{1.254980in}{0.150000in}}{\pgfqpoint{5.490039in}{5.490039in}}%
\pgfusepath{clip}%
\pgfsetbuttcap%
\pgfsetroundjoin%
\definecolor{currentfill}{rgb}{0.203063,0.379716,0.553925}%
\pgfsetfillcolor{currentfill}%
\pgfsetfillopacity{0.700000}%
\pgfsetlinewidth{0.000000pt}%
\definecolor{currentstroke}{rgb}{0.000000,0.000000,0.000000}%
\pgfsetstrokecolor{currentstroke}%
\pgfsetdash{}{0pt}%
\pgfpathmoveto{\pgfqpoint{4.369132in}{3.157855in}}%
\pgfpathlineto{\pgfqpoint{4.382014in}{3.150557in}}%
\pgfpathlineto{\pgfqpoint{4.394900in}{3.143438in}}%
\pgfpathlineto{\pgfqpoint{4.407791in}{3.136498in}}%
\pgfpathlineto{\pgfqpoint{4.420688in}{3.129736in}}%
\pgfpathlineto{\pgfqpoint{4.428043in}{3.146632in}}%
\pgfpathlineto{\pgfqpoint{4.435398in}{3.163791in}}%
\pgfpathlineto{\pgfqpoint{4.442752in}{3.181219in}}%
\pgfpathlineto{\pgfqpoint{4.450106in}{3.198923in}}%
\pgfpathlineto{\pgfqpoint{4.437218in}{3.206234in}}%
\pgfpathlineto{\pgfqpoint{4.424334in}{3.213722in}}%
\pgfpathlineto{\pgfqpoint{4.411455in}{3.221389in}}%
\pgfpathlineto{\pgfqpoint{4.398580in}{3.229235in}}%
\pgfpathlineto{\pgfqpoint{4.391219in}{3.210973in}}%
\pgfpathlineto{\pgfqpoint{4.383857in}{3.192992in}}%
\pgfpathlineto{\pgfqpoint{4.376495in}{3.175289in}}%
\pgfpathlineto{\pgfqpoint{4.369132in}{3.157855in}}%
\pgfpathclose%
\pgfusepath{fill}%
\end{pgfscope}%
\begin{pgfscope}%
\pgfpathrectangle{\pgfqpoint{1.254980in}{0.150000in}}{\pgfqpoint{5.490039in}{5.490039in}}%
\pgfusepath{clip}%
\pgfsetbuttcap%
\pgfsetroundjoin%
\definecolor{currentfill}{rgb}{0.187231,0.414746,0.556547}%
\pgfsetfillcolor{currentfill}%
\pgfsetfillopacity{0.700000}%
\pgfsetlinewidth{0.000000pt}%
\definecolor{currentstroke}{rgb}{0.000000,0.000000,0.000000}%
\pgfsetstrokecolor{currentstroke}%
\pgfsetdash{}{0pt}%
\pgfpathmoveto{\pgfqpoint{4.531099in}{3.242849in}}%
\pgfpathlineto{\pgfqpoint{4.544006in}{3.235841in}}%
\pgfpathlineto{\pgfqpoint{4.556918in}{3.229006in}}%
\pgfpathlineto{\pgfqpoint{4.569837in}{3.222344in}}%
\pgfpathlineto{\pgfqpoint{4.582761in}{3.215854in}}%
\pgfpathlineto{\pgfqpoint{4.590100in}{3.233852in}}%
\pgfpathlineto{\pgfqpoint{4.597440in}{3.252151in}}%
\pgfpathlineto{\pgfqpoint{4.604780in}{3.270761in}}%
\pgfpathlineto{\pgfqpoint{4.612123in}{3.289687in}}%
\pgfpathlineto{\pgfqpoint{4.599207in}{3.296780in}}%
\pgfpathlineto{\pgfqpoint{4.586297in}{3.304046in}}%
\pgfpathlineto{\pgfqpoint{4.573393in}{3.311484in}}%
\pgfpathlineto{\pgfqpoint{4.560494in}{3.319096in}}%
\pgfpathlineto{\pgfqpoint{4.553143in}{3.299556in}}%
\pgfpathlineto{\pgfqpoint{4.545794in}{3.280339in}}%
\pgfpathlineto{\pgfqpoint{4.538446in}{3.261439in}}%
\pgfpathlineto{\pgfqpoint{4.531099in}{3.242849in}}%
\pgfpathclose%
\pgfusepath{fill}%
\end{pgfscope}%
\begin{pgfscope}%
\pgfpathrectangle{\pgfqpoint{1.254980in}{0.150000in}}{\pgfqpoint{5.490039in}{5.490039in}}%
\pgfusepath{clip}%
\pgfsetbuttcap%
\pgfsetroundjoin%
\definecolor{currentfill}{rgb}{0.125394,0.574318,0.549086}%
\pgfsetfillcolor{currentfill}%
\pgfsetfillopacity{0.700000}%
\pgfsetlinewidth{0.000000pt}%
\definecolor{currentstroke}{rgb}{0.000000,0.000000,0.000000}%
\pgfsetstrokecolor{currentstroke}%
\pgfsetdash{}{0pt}%
\pgfpathmoveto{\pgfqpoint{3.347692in}{3.675800in}}%
\pgfpathlineto{\pgfqpoint{3.360607in}{3.655721in}}%
\pgfpathlineto{\pgfqpoint{3.373515in}{3.635927in}}%
\pgfpathlineto{\pgfqpoint{3.386416in}{3.616416in}}%
\pgfpathlineto{\pgfqpoint{3.399312in}{3.597185in}}%
\pgfpathlineto{\pgfqpoint{3.406800in}{3.616923in}}%
\pgfpathlineto{\pgfqpoint{3.414283in}{3.636933in}}%
\pgfpathlineto{\pgfqpoint{3.421760in}{3.657222in}}%
\pgfpathlineto{\pgfqpoint{3.429233in}{3.677794in}}%
\pgfpathlineto{\pgfqpoint{3.416340in}{3.697455in}}%
\pgfpathlineto{\pgfqpoint{3.403440in}{3.717396in}}%
\pgfpathlineto{\pgfqpoint{3.390535in}{3.737620in}}%
\pgfpathlineto{\pgfqpoint{3.377623in}{3.758130in}}%
\pgfpathlineto{\pgfqpoint{3.370148in}{3.737117in}}%
\pgfpathlineto{\pgfqpoint{3.362668in}{3.716394in}}%
\pgfpathlineto{\pgfqpoint{3.355183in}{3.695956in}}%
\pgfpathlineto{\pgfqpoint{3.347692in}{3.675800in}}%
\pgfpathclose%
\pgfusepath{fill}%
\end{pgfscope}%
\begin{pgfscope}%
\pgfpathrectangle{\pgfqpoint{1.254980in}{0.150000in}}{\pgfqpoint{5.490039in}{5.490039in}}%
\pgfusepath{clip}%
\pgfsetbuttcap%
\pgfsetroundjoin%
\definecolor{currentfill}{rgb}{0.210503,0.363727,0.552206}%
\pgfsetfillcolor{currentfill}%
\pgfsetfillopacity{0.700000}%
\pgfsetlinewidth{0.000000pt}%
\definecolor{currentstroke}{rgb}{0.000000,0.000000,0.000000}%
\pgfsetstrokecolor{currentstroke}%
\pgfsetdash{}{0pt}%
\pgfpathmoveto{\pgfqpoint{4.288162in}{3.119612in}}%
\pgfpathlineto{\pgfqpoint{4.301032in}{3.112110in}}%
\pgfpathlineto{\pgfqpoint{4.313907in}{3.104789in}}%
\pgfpathlineto{\pgfqpoint{4.326786in}{3.097651in}}%
\pgfpathlineto{\pgfqpoint{4.339670in}{3.090693in}}%
\pgfpathlineto{\pgfqpoint{4.347038in}{3.107110in}}%
\pgfpathlineto{\pgfqpoint{4.354404in}{3.123772in}}%
\pgfpathlineto{\pgfqpoint{4.361768in}{3.140685in}}%
\pgfpathlineto{\pgfqpoint{4.369132in}{3.157855in}}%
\pgfpathlineto{\pgfqpoint{4.356255in}{3.165333in}}%
\pgfpathlineto{\pgfqpoint{4.343383in}{3.172992in}}%
\pgfpathlineto{\pgfqpoint{4.330516in}{3.180833in}}%
\pgfpathlineto{\pgfqpoint{4.317652in}{3.188857in}}%
\pgfpathlineto{\pgfqpoint{4.310282in}{3.171155in}}%
\pgfpathlineto{\pgfqpoint{4.302910in}{3.153718in}}%
\pgfpathlineto{\pgfqpoint{4.295537in}{3.136539in}}%
\pgfpathlineto{\pgfqpoint{4.288162in}{3.119612in}}%
\pgfpathclose%
\pgfusepath{fill}%
\end{pgfscope}%
\begin{pgfscope}%
\pgfpathrectangle{\pgfqpoint{1.254980in}{0.150000in}}{\pgfqpoint{5.490039in}{5.490039in}}%
\pgfusepath{clip}%
\pgfsetbuttcap%
\pgfsetroundjoin%
\definecolor{currentfill}{rgb}{0.221989,0.339161,0.548752}%
\pgfsetfillcolor{currentfill}%
\pgfsetfillopacity{0.700000}%
\pgfsetlinewidth{0.000000pt}%
\definecolor{currentstroke}{rgb}{0.000000,0.000000,0.000000}%
\pgfsetstrokecolor{currentstroke}%
\pgfsetdash{}{0pt}%
\pgfpathmoveto{\pgfqpoint{3.861430in}{3.072302in}}%
\pgfpathlineto{\pgfqpoint{3.874253in}{3.061933in}}%
\pgfpathlineto{\pgfqpoint{3.887077in}{3.051773in}}%
\pgfpathlineto{\pgfqpoint{3.899902in}{3.041818in}}%
\pgfpathlineto{\pgfqpoint{3.912730in}{3.032069in}}%
\pgfpathlineto{\pgfqpoint{3.920175in}{3.047728in}}%
\pgfpathlineto{\pgfqpoint{3.927618in}{3.063587in}}%
\pgfpathlineto{\pgfqpoint{3.935057in}{3.079651in}}%
\pgfpathlineto{\pgfqpoint{3.942492in}{3.095924in}}%
\pgfpathlineto{\pgfqpoint{3.929670in}{3.106086in}}%
\pgfpathlineto{\pgfqpoint{3.916850in}{3.116452in}}%
\pgfpathlineto{\pgfqpoint{3.904032in}{3.127026in}}%
\pgfpathlineto{\pgfqpoint{3.891215in}{3.137807in}}%
\pgfpathlineto{\pgfqpoint{3.883774in}{3.121110in}}%
\pgfpathlineto{\pgfqpoint{3.876329in}{3.104630in}}%
\pgfpathlineto{\pgfqpoint{3.868882in}{3.088363in}}%
\pgfpathlineto{\pgfqpoint{3.861430in}{3.072302in}}%
\pgfpathclose%
\pgfusepath{fill}%
\end{pgfscope}%
\begin{pgfscope}%
\pgfpathrectangle{\pgfqpoint{1.254980in}{0.150000in}}{\pgfqpoint{5.490039in}{5.490039in}}%
\pgfusepath{clip}%
\pgfsetbuttcap%
\pgfsetroundjoin%
\definecolor{currentfill}{rgb}{0.458674,0.816363,0.329727}%
\pgfsetfillcolor{currentfill}%
\pgfsetfillopacity{0.700000}%
\pgfsetlinewidth{0.000000pt}%
\definecolor{currentstroke}{rgb}{0.000000,0.000000,0.000000}%
\pgfsetstrokecolor{currentstroke}%
\pgfsetdash{}{0pt}%
\pgfpathmoveto{\pgfqpoint{3.474360in}{4.343811in}}%
\pgfpathlineto{\pgfqpoint{3.487306in}{4.319471in}}%
\pgfpathlineto{\pgfqpoint{3.500245in}{4.295434in}}%
\pgfpathlineto{\pgfqpoint{3.513176in}{4.271696in}}%
\pgfpathlineto{\pgfqpoint{3.526099in}{4.248256in}}%
\pgfpathlineto{\pgfqpoint{3.533479in}{4.276738in}}%
\pgfpathlineto{\pgfqpoint{3.540855in}{4.305649in}}%
\pgfpathlineto{\pgfqpoint{3.548227in}{4.334997in}}%
\pgfpathlineto{\pgfqpoint{3.555596in}{4.364790in}}%
\pgfpathlineto{\pgfqpoint{3.542668in}{4.388860in}}%
\pgfpathlineto{\pgfqpoint{3.529733in}{4.413228in}}%
\pgfpathlineto{\pgfqpoint{3.516789in}{4.437898in}}%
\pgfpathlineto{\pgfqpoint{3.503838in}{4.462872in}}%
\pgfpathlineto{\pgfqpoint{3.496474in}{4.432435in}}%
\pgfpathlineto{\pgfqpoint{3.489106in}{4.402451in}}%
\pgfpathlineto{\pgfqpoint{3.481735in}{4.372912in}}%
\pgfpathlineto{\pgfqpoint{3.474360in}{4.343811in}}%
\pgfpathclose%
\pgfusepath{fill}%
\end{pgfscope}%
\begin{pgfscope}%
\pgfpathrectangle{\pgfqpoint{1.254980in}{0.150000in}}{\pgfqpoint{5.490039in}{5.490039in}}%
\pgfusepath{clip}%
\pgfsetbuttcap%
\pgfsetroundjoin%
\definecolor{currentfill}{rgb}{0.179019,0.433756,0.557430}%
\pgfsetfillcolor{currentfill}%
\pgfsetfillopacity{0.700000}%
\pgfsetlinewidth{0.000000pt}%
\definecolor{currentstroke}{rgb}{0.000000,0.000000,0.000000}%
\pgfsetstrokecolor{currentstroke}%
\pgfsetdash{}{0pt}%
\pgfpathmoveto{\pgfqpoint{4.612123in}{3.289687in}}%
\pgfpathlineto{\pgfqpoint{4.625044in}{3.282765in}}%
\pgfpathlineto{\pgfqpoint{4.637971in}{3.276014in}}%
\pgfpathlineto{\pgfqpoint{4.650904in}{3.269433in}}%
\pgfpathlineto{\pgfqpoint{4.663843in}{3.263022in}}%
\pgfpathlineto{\pgfqpoint{4.671177in}{3.281651in}}%
\pgfpathlineto{\pgfqpoint{4.678514in}{3.300604in}}%
\pgfpathlineto{\pgfqpoint{4.685852in}{3.319889in}}%
\pgfpathlineto{\pgfqpoint{4.693193in}{3.339514in}}%
\pgfpathlineto{\pgfqpoint{4.680264in}{3.346556in}}%
\pgfpathlineto{\pgfqpoint{4.667340in}{3.353769in}}%
\pgfpathlineto{\pgfqpoint{4.654422in}{3.361151in}}%
\pgfpathlineto{\pgfqpoint{4.641509in}{3.368705in}}%
\pgfpathlineto{\pgfqpoint{4.634159in}{3.348438in}}%
\pgfpathlineto{\pgfqpoint{4.626812in}{3.328518in}}%
\pgfpathlineto{\pgfqpoint{4.619466in}{3.308937in}}%
\pgfpathlineto{\pgfqpoint{4.612123in}{3.289687in}}%
\pgfpathclose%
\pgfusepath{fill}%
\end{pgfscope}%
\begin{pgfscope}%
\pgfpathrectangle{\pgfqpoint{1.254980in}{0.150000in}}{\pgfqpoint{5.490039in}{5.490039in}}%
\pgfusepath{clip}%
\pgfsetbuttcap%
\pgfsetroundjoin%
\definecolor{currentfill}{rgb}{0.223925,0.334994,0.548053}%
\pgfsetfillcolor{currentfill}%
\pgfsetfillopacity{0.700000}%
\pgfsetlinewidth{0.000000pt}%
\definecolor{currentstroke}{rgb}{0.000000,0.000000,0.000000}%
\pgfsetstrokecolor{currentstroke}%
\pgfsetdash{}{0pt}%
\pgfpathmoveto{\pgfqpoint{3.993799in}{3.057308in}}%
\pgfpathlineto{\pgfqpoint{4.006632in}{3.048155in}}%
\pgfpathlineto{\pgfqpoint{4.019467in}{3.039201in}}%
\pgfpathlineto{\pgfqpoint{4.032305in}{3.030443in}}%
\pgfpathlineto{\pgfqpoint{4.045146in}{3.021881in}}%
\pgfpathlineto{\pgfqpoint{4.052567in}{3.037512in}}%
\pgfpathlineto{\pgfqpoint{4.059985in}{3.053349in}}%
\pgfpathlineto{\pgfqpoint{4.067400in}{3.069397in}}%
\pgfpathlineto{\pgfqpoint{4.074812in}{3.085662in}}%
\pgfpathlineto{\pgfqpoint{4.061977in}{3.094662in}}%
\pgfpathlineto{\pgfqpoint{4.049145in}{3.103859in}}%
\pgfpathlineto{\pgfqpoint{4.036316in}{3.113253in}}%
\pgfpathlineto{\pgfqpoint{4.023489in}{3.122845in}}%
\pgfpathlineto{\pgfqpoint{4.016071in}{3.106131in}}%
\pgfpathlineto{\pgfqpoint{4.008650in}{3.089640in}}%
\pgfpathlineto{\pgfqpoint{4.001226in}{3.073368in}}%
\pgfpathlineto{\pgfqpoint{3.993799in}{3.057308in}}%
\pgfpathclose%
\pgfusepath{fill}%
\end{pgfscope}%
\begin{pgfscope}%
\pgfpathrectangle{\pgfqpoint{1.254980in}{0.150000in}}{\pgfqpoint{5.490039in}{5.490039in}}%
\pgfusepath{clip}%
\pgfsetbuttcap%
\pgfsetroundjoin%
\definecolor{currentfill}{rgb}{0.124780,0.640461,0.527068}%
\pgfsetfillcolor{currentfill}%
\pgfsetfillopacity{0.700000}%
\pgfsetlinewidth{0.000000pt}%
\definecolor{currentstroke}{rgb}{0.000000,0.000000,0.000000}%
\pgfsetstrokecolor{currentstroke}%
\pgfsetdash{}{0pt}%
\pgfpathmoveto{\pgfqpoint{3.325906in}{3.843084in}}%
\pgfpathlineto{\pgfqpoint{3.338846in}{3.821403in}}%
\pgfpathlineto{\pgfqpoint{3.351779in}{3.800019in}}%
\pgfpathlineto{\pgfqpoint{3.364704in}{3.778929in}}%
\pgfpathlineto{\pgfqpoint{3.377623in}{3.758130in}}%
\pgfpathlineto{\pgfqpoint{3.385092in}{3.779440in}}%
\pgfpathlineto{\pgfqpoint{3.392556in}{3.801051in}}%
\pgfpathlineto{\pgfqpoint{3.400015in}{3.822969in}}%
\pgfpathlineto{\pgfqpoint{3.407469in}{3.845199in}}%
\pgfpathlineto{\pgfqpoint{3.394552in}{3.866461in}}%
\pgfpathlineto{\pgfqpoint{3.381627in}{3.888016in}}%
\pgfpathlineto{\pgfqpoint{3.368696in}{3.909865in}}%
\pgfpathlineto{\pgfqpoint{3.355757in}{3.932012in}}%
\pgfpathlineto{\pgfqpoint{3.348302in}{3.909305in}}%
\pgfpathlineto{\pgfqpoint{3.340842in}{3.886919in}}%
\pgfpathlineto{\pgfqpoint{3.333377in}{3.864847in}}%
\pgfpathlineto{\pgfqpoint{3.325906in}{3.843084in}}%
\pgfpathclose%
\pgfusepath{fill}%
\end{pgfscope}%
\begin{pgfscope}%
\pgfpathrectangle{\pgfqpoint{1.254980in}{0.150000in}}{\pgfqpoint{5.490039in}{5.490039in}}%
\pgfusepath{clip}%
\pgfsetbuttcap%
\pgfsetroundjoin%
\definecolor{currentfill}{rgb}{0.616293,0.852709,0.230052}%
\pgfsetfillcolor{currentfill}%
\pgfsetfillopacity{0.700000}%
\pgfsetlinewidth{0.000000pt}%
\definecolor{currentstroke}{rgb}{0.000000,0.000000,0.000000}%
\pgfsetstrokecolor{currentstroke}%
\pgfsetdash{}{0pt}%
\pgfpathmoveto{\pgfqpoint{3.585043in}{4.488557in}}%
\pgfpathlineto{\pgfqpoint{3.597969in}{4.464118in}}%
\pgfpathlineto{\pgfqpoint{3.610889in}{4.439974in}}%
\pgfpathlineto{\pgfqpoint{3.623801in}{4.416121in}}%
\pgfpathlineto{\pgfqpoint{3.636707in}{4.392557in}}%
\pgfpathlineto{\pgfqpoint{3.644068in}{4.424004in}}%
\pgfpathlineto{\pgfqpoint{3.651427in}{4.455935in}}%
\pgfpathlineto{\pgfqpoint{3.658784in}{4.488358in}}%
\pgfpathlineto{\pgfqpoint{3.666139in}{4.521280in}}%
\pgfpathlineto{\pgfqpoint{3.653226in}{4.545540in}}%
\pgfpathlineto{\pgfqpoint{3.640307in}{4.570090in}}%
\pgfpathlineto{\pgfqpoint{3.627380in}{4.594933in}}%
\pgfpathlineto{\pgfqpoint{3.614446in}{4.620071in}}%
\pgfpathlineto{\pgfqpoint{3.607099in}{4.586437in}}%
\pgfpathlineto{\pgfqpoint{3.599749in}{4.553312in}}%
\pgfpathlineto{\pgfqpoint{3.592397in}{4.520688in}}%
\pgfpathlineto{\pgfqpoint{3.585043in}{4.488557in}}%
\pgfpathclose%
\pgfusepath{fill}%
\end{pgfscope}%
\begin{pgfscope}%
\pgfpathrectangle{\pgfqpoint{1.254980in}{0.150000in}}{\pgfqpoint{5.490039in}{5.490039in}}%
\pgfusepath{clip}%
\pgfsetbuttcap%
\pgfsetroundjoin%
\definecolor{currentfill}{rgb}{0.218130,0.347432,0.550038}%
\pgfsetfillcolor{currentfill}%
\pgfsetfillopacity{0.700000}%
\pgfsetlinewidth{0.000000pt}%
\definecolor{currentstroke}{rgb}{0.000000,0.000000,0.000000}%
\pgfsetstrokecolor{currentstroke}%
\pgfsetdash{}{0pt}%
\pgfpathmoveto{\pgfqpoint{4.207184in}{3.084188in}}%
\pgfpathlineto{\pgfqpoint{4.220044in}{3.076440in}}%
\pgfpathlineto{\pgfqpoint{4.232908in}{3.068877in}}%
\pgfpathlineto{\pgfqpoint{4.245776in}{3.061499in}}%
\pgfpathlineto{\pgfqpoint{4.258649in}{3.054306in}}%
\pgfpathlineto{\pgfqpoint{4.266030in}{3.070284in}}%
\pgfpathlineto{\pgfqpoint{4.273409in}{3.086491in}}%
\pgfpathlineto{\pgfqpoint{4.280787in}{3.102932in}}%
\pgfpathlineto{\pgfqpoint{4.288162in}{3.119612in}}%
\pgfpathlineto{\pgfqpoint{4.275297in}{3.127299in}}%
\pgfpathlineto{\pgfqpoint{4.262435in}{3.135169in}}%
\pgfpathlineto{\pgfqpoint{4.249578in}{3.143225in}}%
\pgfpathlineto{\pgfqpoint{4.236724in}{3.151467in}}%
\pgfpathlineto{\pgfqpoint{4.229342in}{3.134283in}}%
\pgfpathlineto{\pgfqpoint{4.221958in}{3.117345in}}%
\pgfpathlineto{\pgfqpoint{4.214572in}{3.100649in}}%
\pgfpathlineto{\pgfqpoint{4.207184in}{3.084188in}}%
\pgfpathclose%
\pgfusepath{fill}%
\end{pgfscope}%
\begin{pgfscope}%
\pgfpathrectangle{\pgfqpoint{1.254980in}{0.150000in}}{\pgfqpoint{5.490039in}{5.490039in}}%
\pgfusepath{clip}%
\pgfsetbuttcap%
\pgfsetroundjoin%
\definecolor{currentfill}{rgb}{0.216210,0.351535,0.550627}%
\pgfsetfillcolor{currentfill}%
\pgfsetfillopacity{0.700000}%
\pgfsetlinewidth{0.000000pt}%
\definecolor{currentstroke}{rgb}{0.000000,0.000000,0.000000}%
\pgfsetstrokecolor{currentstroke}%
\pgfsetdash{}{0pt}%
\pgfpathmoveto{\pgfqpoint{3.728997in}{3.097543in}}%
\pgfpathlineto{\pgfqpoint{3.741819in}{3.085849in}}%
\pgfpathlineto{\pgfqpoint{3.754641in}{3.074374in}}%
\pgfpathlineto{\pgfqpoint{3.767463in}{3.063116in}}%
\pgfpathlineto{\pgfqpoint{3.780286in}{3.052074in}}%
\pgfpathlineto{\pgfqpoint{3.787759in}{3.067728in}}%
\pgfpathlineto{\pgfqpoint{3.795227in}{3.083577in}}%
\pgfpathlineto{\pgfqpoint{3.802691in}{3.099627in}}%
\pgfpathlineto{\pgfqpoint{3.810152in}{3.115881in}}%
\pgfpathlineto{\pgfqpoint{3.797334in}{3.127308in}}%
\pgfpathlineto{\pgfqpoint{3.784517in}{3.138952in}}%
\pgfpathlineto{\pgfqpoint{3.771700in}{3.150814in}}%
\pgfpathlineto{\pgfqpoint{3.758883in}{3.162894in}}%
\pgfpathlineto{\pgfqpoint{3.751418in}{3.146243in}}%
\pgfpathlineto{\pgfqpoint{3.743948in}{3.129804in}}%
\pgfpathlineto{\pgfqpoint{3.736475in}{3.113573in}}%
\pgfpathlineto{\pgfqpoint{3.728997in}{3.097543in}}%
\pgfpathclose%
\pgfusepath{fill}%
\end{pgfscope}%
\begin{pgfscope}%
\pgfpathrectangle{\pgfqpoint{1.254980in}{0.150000in}}{\pgfqpoint{5.490039in}{5.490039in}}%
\pgfusepath{clip}%
\pgfsetbuttcap%
\pgfsetroundjoin%
\definecolor{currentfill}{rgb}{0.146180,0.515413,0.556823}%
\pgfsetfillcolor{currentfill}%
\pgfsetfillopacity{0.700000}%
\pgfsetlinewidth{0.000000pt}%
\definecolor{currentstroke}{rgb}{0.000000,0.000000,0.000000}%
\pgfsetstrokecolor{currentstroke}%
\pgfsetdash{}{0pt}%
\pgfpathmoveto{\pgfqpoint{3.369307in}{3.520863in}}%
\pgfpathlineto{\pgfqpoint{3.382201in}{3.502307in}}%
\pgfpathlineto{\pgfqpoint{3.395089in}{3.484024in}}%
\pgfpathlineto{\pgfqpoint{3.407972in}{3.466014in}}%
\pgfpathlineto{\pgfqpoint{3.420851in}{3.448272in}}%
\pgfpathlineto{\pgfqpoint{3.428356in}{3.466581in}}%
\pgfpathlineto{\pgfqpoint{3.435856in}{3.485137in}}%
\pgfpathlineto{\pgfqpoint{3.443351in}{3.503943in}}%
\pgfpathlineto{\pgfqpoint{3.450841in}{3.523004in}}%
\pgfpathlineto{\pgfqpoint{3.437967in}{3.541142in}}%
\pgfpathlineto{\pgfqpoint{3.425087in}{3.559550in}}%
\pgfpathlineto{\pgfqpoint{3.412202in}{3.578230in}}%
\pgfpathlineto{\pgfqpoint{3.399312in}{3.597185in}}%
\pgfpathlineto{\pgfqpoint{3.391819in}{3.577714in}}%
\pgfpathlineto{\pgfqpoint{3.384320in}{3.558507in}}%
\pgfpathlineto{\pgfqpoint{3.376817in}{3.539558in}}%
\pgfpathlineto{\pgfqpoint{3.369307in}{3.520863in}}%
\pgfpathclose%
\pgfusepath{fill}%
\end{pgfscope}%
\begin{pgfscope}%
\pgfpathrectangle{\pgfqpoint{1.254980in}{0.150000in}}{\pgfqpoint{5.490039in}{5.490039in}}%
\pgfusepath{clip}%
\pgfsetbuttcap%
\pgfsetroundjoin%
\definecolor{currentfill}{rgb}{0.171176,0.452530,0.557965}%
\pgfsetfillcolor{currentfill}%
\pgfsetfillopacity{0.700000}%
\pgfsetlinewidth{0.000000pt}%
\definecolor{currentstroke}{rgb}{0.000000,0.000000,0.000000}%
\pgfsetstrokecolor{currentstroke}%
\pgfsetdash{}{0pt}%
\pgfpathmoveto{\pgfqpoint{4.693193in}{3.339514in}}%
\pgfpathlineto{\pgfqpoint{4.706129in}{3.332640in}}%
\pgfpathlineto{\pgfqpoint{4.719071in}{3.325935in}}%
\pgfpathlineto{\pgfqpoint{4.732019in}{3.319397in}}%
\pgfpathlineto{\pgfqpoint{4.744973in}{3.313027in}}%
\pgfpathlineto{\pgfqpoint{4.752306in}{3.332349in}}%
\pgfpathlineto{\pgfqpoint{4.759643in}{3.352019in}}%
\pgfpathlineto{\pgfqpoint{4.766983in}{3.372044in}}%
\pgfpathlineto{\pgfqpoint{4.754036in}{3.378906in}}%
\pgfpathlineto{\pgfqpoint{4.741095in}{3.385936in}}%
\pgfpathlineto{\pgfqpoint{4.728160in}{3.393133in}}%
\pgfpathlineto{\pgfqpoint{4.715232in}{3.400499in}}%
\pgfpathlineto{\pgfqpoint{4.707883in}{3.379811in}}%
\pgfpathlineto{\pgfqpoint{4.700537in}{3.359485in}}%
\pgfpathlineto{\pgfqpoint{4.693193in}{3.339514in}}%
\pgfpathclose%
\pgfusepath{fill}%
\end{pgfscope}%
\begin{pgfscope}%
\pgfpathrectangle{\pgfqpoint{1.254980in}{0.150000in}}{\pgfqpoint{5.490039in}{5.490039in}}%
\pgfusepath{clip}%
\pgfsetbuttcap%
\pgfsetroundjoin%
\definecolor{currentfill}{rgb}{0.223925,0.334994,0.548053}%
\pgfsetfillcolor{currentfill}%
\pgfsetfillopacity{0.700000}%
\pgfsetlinewidth{0.000000pt}%
\definecolor{currentstroke}{rgb}{0.000000,0.000000,0.000000}%
\pgfsetstrokecolor{currentstroke}%
\pgfsetdash{}{0pt}%
\pgfpathmoveto{\pgfqpoint{4.126183in}{3.051598in}}%
\pgfpathlineto{\pgfqpoint{4.139034in}{3.043562in}}%
\pgfpathlineto{\pgfqpoint{4.151889in}{3.035715in}}%
\pgfpathlineto{\pgfqpoint{4.164747in}{3.028057in}}%
\pgfpathlineto{\pgfqpoint{4.177610in}{3.020587in}}%
\pgfpathlineto{\pgfqpoint{4.185007in}{3.036162in}}%
\pgfpathlineto{\pgfqpoint{4.192401in}{3.051950in}}%
\pgfpathlineto{\pgfqpoint{4.199794in}{3.067957in}}%
\pgfpathlineto{\pgfqpoint{4.207184in}{3.084188in}}%
\pgfpathlineto{\pgfqpoint{4.194328in}{3.092124in}}%
\pgfpathlineto{\pgfqpoint{4.181476in}{3.100248in}}%
\pgfpathlineto{\pgfqpoint{4.168627in}{3.108560in}}%
\pgfpathlineto{\pgfqpoint{4.155782in}{3.117063in}}%
\pgfpathlineto{\pgfqpoint{4.148386in}{3.100355in}}%
\pgfpathlineto{\pgfqpoint{4.140987in}{3.083879in}}%
\pgfpathlineto{\pgfqpoint{4.133586in}{3.067629in}}%
\pgfpathlineto{\pgfqpoint{4.126183in}{3.051598in}}%
\pgfpathclose%
\pgfusepath{fill}%
\end{pgfscope}%
\begin{pgfscope}%
\pgfpathrectangle{\pgfqpoint{1.254980in}{0.150000in}}{\pgfqpoint{5.490039in}{5.490039in}}%
\pgfusepath{clip}%
\pgfsetbuttcap%
\pgfsetroundjoin%
\definecolor{currentfill}{rgb}{0.188923,0.410910,0.556326}%
\pgfsetfillcolor{currentfill}%
\pgfsetfillopacity{0.700000}%
\pgfsetlinewidth{0.000000pt}%
\definecolor{currentstroke}{rgb}{0.000000,0.000000,0.000000}%
\pgfsetstrokecolor{currentstroke}%
\pgfsetdash{}{0pt}%
\pgfpathmoveto{\pgfqpoint{3.493688in}{3.247796in}}%
\pgfpathlineto{\pgfqpoint{3.506537in}{3.232736in}}%
\pgfpathlineto{\pgfqpoint{3.519383in}{3.217922in}}%
\pgfpathlineto{\pgfqpoint{3.532226in}{3.203353in}}%
\pgfpathlineto{\pgfqpoint{3.545067in}{3.189026in}}%
\pgfpathlineto{\pgfqpoint{3.552578in}{3.205337in}}%
\pgfpathlineto{\pgfqpoint{3.560085in}{3.221855in}}%
\pgfpathlineto{\pgfqpoint{3.567587in}{3.238583in}}%
\pgfpathlineto{\pgfqpoint{3.575084in}{3.255526in}}%
\pgfpathlineto{\pgfqpoint{3.562248in}{3.270214in}}%
\pgfpathlineto{\pgfqpoint{3.549410in}{3.285146in}}%
\pgfpathlineto{\pgfqpoint{3.536569in}{3.300322in}}%
\pgfpathlineto{\pgfqpoint{3.523726in}{3.315745in}}%
\pgfpathlineto{\pgfqpoint{3.516224in}{3.298429in}}%
\pgfpathlineto{\pgfqpoint{3.508717in}{3.281335in}}%
\pgfpathlineto{\pgfqpoint{3.501205in}{3.264459in}}%
\pgfpathlineto{\pgfqpoint{3.493688in}{3.247796in}}%
\pgfpathclose%
\pgfusepath{fill}%
\end{pgfscope}%
\begin{pgfscope}%
\pgfpathrectangle{\pgfqpoint{1.254980in}{0.150000in}}{\pgfqpoint{5.490039in}{5.490039in}}%
\pgfusepath{clip}%
\pgfsetbuttcap%
\pgfsetroundjoin%
\definecolor{currentfill}{rgb}{0.259857,0.745492,0.444467}%
\pgfsetfillcolor{currentfill}%
\pgfsetfillopacity{0.700000}%
\pgfsetlinewidth{0.000000pt}%
\definecolor{currentstroke}{rgb}{0.000000,0.000000,0.000000}%
\pgfsetstrokecolor{currentstroke}%
\pgfsetdash{}{0pt}%
\pgfpathmoveto{\pgfqpoint{3.333686in}{4.119791in}}%
\pgfpathlineto{\pgfqpoint{3.346658in}{4.095919in}}%
\pgfpathlineto{\pgfqpoint{3.359621in}{4.072358in}}%
\pgfpathlineto{\pgfqpoint{3.372576in}{4.049105in}}%
\pgfpathlineto{\pgfqpoint{3.385523in}{4.026158in}}%
\pgfpathlineto{\pgfqpoint{3.392952in}{4.050553in}}%
\pgfpathlineto{\pgfqpoint{3.400376in}{4.075304in}}%
\pgfpathlineto{\pgfqpoint{3.407794in}{4.100417in}}%
\pgfpathlineto{\pgfqpoint{3.415209in}{4.125899in}}%
\pgfpathlineto{\pgfqpoint{3.402261in}{4.149379in}}%
\pgfpathlineto{\pgfqpoint{3.389304in}{4.173165in}}%
\pgfpathlineto{\pgfqpoint{3.376340in}{4.197260in}}%
\pgfpathlineto{\pgfqpoint{3.363366in}{4.221668in}}%
\pgfpathlineto{\pgfqpoint{3.355954in}{4.195640in}}%
\pgfpathlineto{\pgfqpoint{3.348537in}{4.169989in}}%
\pgfpathlineto{\pgfqpoint{3.341114in}{4.144708in}}%
\pgfpathlineto{\pgfqpoint{3.333686in}{4.119791in}}%
\pgfpathclose%
\pgfusepath{fill}%
\end{pgfscope}%
\begin{pgfscope}%
\pgfpathrectangle{\pgfqpoint{1.254980in}{0.150000in}}{\pgfqpoint{5.490039in}{5.490039in}}%
\pgfusepath{clip}%
\pgfsetbuttcap%
\pgfsetroundjoin%
\definecolor{currentfill}{rgb}{0.199430,0.387607,0.554642}%
\pgfsetfillcolor{currentfill}%
\pgfsetfillopacity{0.700000}%
\pgfsetlinewidth{0.000000pt}%
\definecolor{currentstroke}{rgb}{0.000000,0.000000,0.000000}%
\pgfsetstrokecolor{currentstroke}%
\pgfsetdash{}{0pt}%
\pgfpathmoveto{\pgfqpoint{3.545067in}{3.189026in}}%
\pgfpathlineto{\pgfqpoint{3.557906in}{3.174940in}}%
\pgfpathlineto{\pgfqpoint{3.570742in}{3.161093in}}%
\pgfpathlineto{\pgfqpoint{3.583577in}{3.147483in}}%
\pgfpathlineto{\pgfqpoint{3.596411in}{3.134108in}}%
\pgfpathlineto{\pgfqpoint{3.603917in}{3.150069in}}%
\pgfpathlineto{\pgfqpoint{3.611418in}{3.166228in}}%
\pgfpathlineto{\pgfqpoint{3.618915in}{3.182592in}}%
\pgfpathlineto{\pgfqpoint{3.626407in}{3.199163in}}%
\pgfpathlineto{\pgfqpoint{3.613579in}{3.212899in}}%
\pgfpathlineto{\pgfqpoint{3.600749in}{3.226870in}}%
\pgfpathlineto{\pgfqpoint{3.587918in}{3.241078in}}%
\pgfpathlineto{\pgfqpoint{3.575084in}{3.255526in}}%
\pgfpathlineto{\pgfqpoint{3.567587in}{3.238583in}}%
\pgfpathlineto{\pgfqpoint{3.560085in}{3.221855in}}%
\pgfpathlineto{\pgfqpoint{3.552578in}{3.205337in}}%
\pgfpathlineto{\pgfqpoint{3.545067in}{3.189026in}}%
\pgfpathclose%
\pgfusepath{fill}%
\end{pgfscope}%
\begin{pgfscope}%
\pgfpathrectangle{\pgfqpoint{1.254980in}{0.150000in}}{\pgfqpoint{5.490039in}{5.490039in}}%
\pgfusepath{clip}%
\pgfsetbuttcap%
\pgfsetroundjoin%
\definecolor{currentfill}{rgb}{0.177423,0.437527,0.557565}%
\pgfsetfillcolor{currentfill}%
\pgfsetfillopacity{0.700000}%
\pgfsetlinewidth{0.000000pt}%
\definecolor{currentstroke}{rgb}{0.000000,0.000000,0.000000}%
\pgfsetstrokecolor{currentstroke}%
\pgfsetdash{}{0pt}%
\pgfpathmoveto{\pgfqpoint{3.442263in}{3.310543in}}%
\pgfpathlineto{\pgfqpoint{3.455124in}{3.294476in}}%
\pgfpathlineto{\pgfqpoint{3.467982in}{3.278664in}}%
\pgfpathlineto{\pgfqpoint{3.480837in}{3.263105in}}%
\pgfpathlineto{\pgfqpoint{3.493688in}{3.247796in}}%
\pgfpathlineto{\pgfqpoint{3.501205in}{3.264459in}}%
\pgfpathlineto{\pgfqpoint{3.508717in}{3.281335in}}%
\pgfpathlineto{\pgfqpoint{3.516224in}{3.298429in}}%
\pgfpathlineto{\pgfqpoint{3.523726in}{3.315745in}}%
\pgfpathlineto{\pgfqpoint{3.510879in}{3.331417in}}%
\pgfpathlineto{\pgfqpoint{3.498029in}{3.347341in}}%
\pgfpathlineto{\pgfqpoint{3.485176in}{3.363517in}}%
\pgfpathlineto{\pgfqpoint{3.472319in}{3.379949in}}%
\pgfpathlineto{\pgfqpoint{3.464813in}{3.362257in}}%
\pgfpathlineto{\pgfqpoint{3.457301in}{3.344796in}}%
\pgfpathlineto{\pgfqpoint{3.449784in}{3.327559in}}%
\pgfpathlineto{\pgfqpoint{3.442263in}{3.310543in}}%
\pgfpathclose%
\pgfusepath{fill}%
\end{pgfscope}%
\begin{pgfscope}%
\pgfpathrectangle{\pgfqpoint{1.254980in}{0.150000in}}{\pgfqpoint{5.490039in}{5.490039in}}%
\pgfusepath{clip}%
\pgfsetbuttcap%
\pgfsetroundjoin%
\definecolor{currentfill}{rgb}{0.344074,0.780029,0.397381}%
\pgfsetfillcolor{currentfill}%
\pgfsetfillopacity{0.700000}%
\pgfsetlinewidth{0.000000pt}%
\definecolor{currentstroke}{rgb}{0.000000,0.000000,0.000000}%
\pgfsetstrokecolor{currentstroke}%
\pgfsetdash{}{0pt}%
\pgfpathmoveto{\pgfqpoint{3.363366in}{4.221668in}}%
\pgfpathlineto{\pgfqpoint{3.376340in}{4.197260in}}%
\pgfpathlineto{\pgfqpoint{3.389304in}{4.173165in}}%
\pgfpathlineto{\pgfqpoint{3.402261in}{4.149379in}}%
\pgfpathlineto{\pgfqpoint{3.415209in}{4.125899in}}%
\pgfpathlineto{\pgfqpoint{3.422618in}{4.151755in}}%
\pgfpathlineto{\pgfqpoint{3.430023in}{4.177993in}}%
\pgfpathlineto{\pgfqpoint{3.437423in}{4.204619in}}%
\pgfpathlineto{\pgfqpoint{3.444819in}{4.231640in}}%
\pgfpathlineto{\pgfqpoint{3.431868in}{4.255685in}}%
\pgfpathlineto{\pgfqpoint{3.418910in}{4.280039in}}%
\pgfpathlineto{\pgfqpoint{3.405942in}{4.304703in}}%
\pgfpathlineto{\pgfqpoint{3.392966in}{4.329680in}}%
\pgfpathlineto{\pgfqpoint{3.385573in}{4.302079in}}%
\pgfpathlineto{\pgfqpoint{3.378176in}{4.274881in}}%
\pgfpathlineto{\pgfqpoint{3.370774in}{4.248080in}}%
\pgfpathlineto{\pgfqpoint{3.363366in}{4.221668in}}%
\pgfpathclose%
\pgfusepath{fill}%
\end{pgfscope}%
\begin{pgfscope}%
\pgfpathrectangle{\pgfqpoint{1.254980in}{0.150000in}}{\pgfqpoint{5.490039in}{5.490039in}}%
\pgfusepath{clip}%
\pgfsetbuttcap%
\pgfsetroundjoin%
\definecolor{currentfill}{rgb}{0.229739,0.322361,0.545706}%
\pgfsetfillcolor{currentfill}%
\pgfsetfillopacity{0.700000}%
\pgfsetlinewidth{0.000000pt}%
\definecolor{currentstroke}{rgb}{0.000000,0.000000,0.000000}%
\pgfsetstrokecolor{currentstroke}%
\pgfsetdash{}{0pt}%
\pgfpathmoveto{\pgfqpoint{3.912730in}{3.032069in}}%
\pgfpathlineto{\pgfqpoint{3.925559in}{3.022524in}}%
\pgfpathlineto{\pgfqpoint{3.938390in}{3.013182in}}%
\pgfpathlineto{\pgfqpoint{3.951224in}{3.004041in}}%
\pgfpathlineto{\pgfqpoint{3.964060in}{2.995100in}}%
\pgfpathlineto{\pgfqpoint{3.971500in}{3.010358in}}%
\pgfpathlineto{\pgfqpoint{3.978936in}{3.025808in}}%
\pgfpathlineto{\pgfqpoint{3.986370in}{3.041457in}}%
\pgfpathlineto{\pgfqpoint{3.993799in}{3.057308in}}%
\pgfpathlineto{\pgfqpoint{3.980969in}{3.066660in}}%
\pgfpathlineto{\pgfqpoint{3.968142in}{3.076213in}}%
\pgfpathlineto{\pgfqpoint{3.955316in}{3.085967in}}%
\pgfpathlineto{\pgfqpoint{3.942492in}{3.095924in}}%
\pgfpathlineto{\pgfqpoint{3.935057in}{3.079651in}}%
\pgfpathlineto{\pgfqpoint{3.927618in}{3.063587in}}%
\pgfpathlineto{\pgfqpoint{3.920175in}{3.047728in}}%
\pgfpathlineto{\pgfqpoint{3.912730in}{3.032069in}}%
\pgfpathclose%
\pgfusepath{fill}%
\end{pgfscope}%
\begin{pgfscope}%
\pgfpathrectangle{\pgfqpoint{1.254980in}{0.150000in}}{\pgfqpoint{5.490039in}{5.490039in}}%
\pgfusepath{clip}%
\pgfsetbuttcap%
\pgfsetroundjoin%
\definecolor{currentfill}{rgb}{0.585678,0.846661,0.249897}%
\pgfsetfillcolor{currentfill}%
\pgfsetfillopacity{0.700000}%
\pgfsetlinewidth{0.000000pt}%
\definecolor{currentstroke}{rgb}{0.000000,0.000000,0.000000}%
\pgfsetstrokecolor{currentstroke}%
\pgfsetdash{}{0pt}%
\pgfpathmoveto{\pgfqpoint{3.503838in}{4.462872in}}%
\pgfpathlineto{\pgfqpoint{3.516789in}{4.437898in}}%
\pgfpathlineto{\pgfqpoint{3.529733in}{4.413228in}}%
\pgfpathlineto{\pgfqpoint{3.542668in}{4.388860in}}%
\pgfpathlineto{\pgfqpoint{3.555596in}{4.364790in}}%
\pgfpathlineto{\pgfqpoint{3.562962in}{4.395034in}}%
\pgfpathlineto{\pgfqpoint{3.570325in}{4.425738in}}%
\pgfpathlineto{\pgfqpoint{3.577685in}{4.456909in}}%
\pgfpathlineto{\pgfqpoint{3.585043in}{4.488557in}}%
\pgfpathlineto{\pgfqpoint{3.572109in}{4.513292in}}%
\pgfpathlineto{\pgfqpoint{3.559167in}{4.538327in}}%
\pgfpathlineto{\pgfqpoint{3.546217in}{4.563666in}}%
\pgfpathlineto{\pgfqpoint{3.533259in}{4.589310in}}%
\pgfpathlineto{\pgfqpoint{3.525909in}{4.556981in}}%
\pgfpathlineto{\pgfqpoint{3.518555in}{4.525137in}}%
\pgfpathlineto{\pgfqpoint{3.511198in}{4.493770in}}%
\pgfpathlineto{\pgfqpoint{3.503838in}{4.462872in}}%
\pgfpathclose%
\pgfusepath{fill}%
\end{pgfscope}%
\begin{pgfscope}%
\pgfpathrectangle{\pgfqpoint{1.254980in}{0.150000in}}{\pgfqpoint{5.490039in}{5.490039in}}%
\pgfusepath{clip}%
\pgfsetbuttcap%
\pgfsetroundjoin%
\definecolor{currentfill}{rgb}{0.225863,0.330805,0.547314}%
\pgfsetfillcolor{currentfill}%
\pgfsetfillopacity{0.700000}%
\pgfsetlinewidth{0.000000pt}%
\definecolor{currentstroke}{rgb}{0.000000,0.000000,0.000000}%
\pgfsetstrokecolor{currentstroke}%
\pgfsetdash{}{0pt}%
\pgfpathmoveto{\pgfqpoint{3.780286in}{3.052074in}}%
\pgfpathlineto{\pgfqpoint{3.793110in}{3.041247in}}%
\pgfpathlineto{\pgfqpoint{3.805935in}{3.030632in}}%
\pgfpathlineto{\pgfqpoint{3.818760in}{3.020229in}}%
\pgfpathlineto{\pgfqpoint{3.831587in}{3.010037in}}%
\pgfpathlineto{\pgfqpoint{3.839054in}{3.025316in}}%
\pgfpathlineto{\pgfqpoint{3.846516in}{3.040783in}}%
\pgfpathlineto{\pgfqpoint{3.853975in}{3.056444in}}%
\pgfpathlineto{\pgfqpoint{3.861430in}{3.072302in}}%
\pgfpathlineto{\pgfqpoint{3.848609in}{3.082880in}}%
\pgfpathlineto{\pgfqpoint{3.835789in}{3.093668in}}%
\pgfpathlineto{\pgfqpoint{3.822970in}{3.104667in}}%
\pgfpathlineto{\pgfqpoint{3.810152in}{3.115881in}}%
\pgfpathlineto{\pgfqpoint{3.802691in}{3.099627in}}%
\pgfpathlineto{\pgfqpoint{3.795227in}{3.083577in}}%
\pgfpathlineto{\pgfqpoint{3.787759in}{3.067728in}}%
\pgfpathlineto{\pgfqpoint{3.780286in}{3.052074in}}%
\pgfpathclose%
\pgfusepath{fill}%
\end{pgfscope}%
\begin{pgfscope}%
\pgfpathrectangle{\pgfqpoint{1.254980in}{0.150000in}}{\pgfqpoint{5.490039in}{5.490039in}}%
\pgfusepath{clip}%
\pgfsetbuttcap%
\pgfsetroundjoin%
\definecolor{currentfill}{rgb}{0.119423,0.611141,0.538982}%
\pgfsetfillcolor{currentfill}%
\pgfsetfillopacity{0.700000}%
\pgfsetlinewidth{0.000000pt}%
\definecolor{currentstroke}{rgb}{0.000000,0.000000,0.000000}%
\pgfsetstrokecolor{currentstroke}%
\pgfsetdash{}{0pt}%
\pgfpathmoveto{\pgfqpoint{3.295966in}{3.759019in}}%
\pgfpathlineto{\pgfqpoint{3.308909in}{3.737773in}}%
\pgfpathlineto{\pgfqpoint{3.321844in}{3.716823in}}%
\pgfpathlineto{\pgfqpoint{3.334771in}{3.696166in}}%
\pgfpathlineto{\pgfqpoint{3.347692in}{3.675800in}}%
\pgfpathlineto{\pgfqpoint{3.355183in}{3.695956in}}%
\pgfpathlineto{\pgfqpoint{3.362668in}{3.716394in}}%
\pgfpathlineto{\pgfqpoint{3.370148in}{3.737117in}}%
\pgfpathlineto{\pgfqpoint{3.377623in}{3.758130in}}%
\pgfpathlineto{\pgfqpoint{3.364704in}{3.778929in}}%
\pgfpathlineto{\pgfqpoint{3.351779in}{3.800019in}}%
\pgfpathlineto{\pgfqpoint{3.338846in}{3.821403in}}%
\pgfpathlineto{\pgfqpoint{3.325906in}{3.843084in}}%
\pgfpathlineto{\pgfqpoint{3.318430in}{3.821625in}}%
\pgfpathlineto{\pgfqpoint{3.310948in}{3.800465in}}%
\pgfpathlineto{\pgfqpoint{3.303460in}{3.779598in}}%
\pgfpathlineto{\pgfqpoint{3.295966in}{3.759019in}}%
\pgfpathclose%
\pgfusepath{fill}%
\end{pgfscope}%
\begin{pgfscope}%
\pgfpathrectangle{\pgfqpoint{1.254980in}{0.150000in}}{\pgfqpoint{5.490039in}{5.490039in}}%
\pgfusepath{clip}%
\pgfsetbuttcap%
\pgfsetroundjoin%
\definecolor{currentfill}{rgb}{0.133743,0.548535,0.553541}%
\pgfsetfillcolor{currentfill}%
\pgfsetfillopacity{0.700000}%
\pgfsetlinewidth{0.000000pt}%
\definecolor{currentstroke}{rgb}{0.000000,0.000000,0.000000}%
\pgfsetstrokecolor{currentstroke}%
\pgfsetdash{}{0pt}%
\pgfpathmoveto{\pgfqpoint{3.317673in}{3.597878in}}%
\pgfpathlineto{\pgfqpoint{3.330591in}{3.578200in}}%
\pgfpathlineto{\pgfqpoint{3.343502in}{3.558807in}}%
\pgfpathlineto{\pgfqpoint{3.356408in}{3.539695in}}%
\pgfpathlineto{\pgfqpoint{3.369307in}{3.520863in}}%
\pgfpathlineto{\pgfqpoint{3.376817in}{3.539558in}}%
\pgfpathlineto{\pgfqpoint{3.384320in}{3.558507in}}%
\pgfpathlineto{\pgfqpoint{3.391819in}{3.577714in}}%
\pgfpathlineto{\pgfqpoint{3.399312in}{3.597185in}}%
\pgfpathlineto{\pgfqpoint{3.386416in}{3.616416in}}%
\pgfpathlineto{\pgfqpoint{3.373515in}{3.635927in}}%
\pgfpathlineto{\pgfqpoint{3.360607in}{3.655721in}}%
\pgfpathlineto{\pgfqpoint{3.347692in}{3.675800in}}%
\pgfpathlineto{\pgfqpoint{3.340196in}{3.655918in}}%
\pgfpathlineto{\pgfqpoint{3.332694in}{3.636307in}}%
\pgfpathlineto{\pgfqpoint{3.325187in}{3.616962in}}%
\pgfpathlineto{\pgfqpoint{3.317673in}{3.597878in}}%
\pgfpathclose%
\pgfusepath{fill}%
\end{pgfscope}%
\begin{pgfscope}%
\pgfpathrectangle{\pgfqpoint{1.254980in}{0.150000in}}{\pgfqpoint{5.490039in}{5.490039in}}%
\pgfusepath{clip}%
\pgfsetbuttcap%
\pgfsetroundjoin%
\definecolor{currentfill}{rgb}{0.196571,0.711827,0.479221}%
\pgfsetfillcolor{currentfill}%
\pgfsetfillopacity{0.700000}%
\pgfsetlinewidth{0.000000pt}%
\definecolor{currentstroke}{rgb}{0.000000,0.000000,0.000000}%
\pgfsetstrokecolor{currentstroke}%
\pgfsetdash{}{0pt}%
\pgfpathmoveto{\pgfqpoint{3.303921in}{4.023641in}}%
\pgfpathlineto{\pgfqpoint{3.316893in}{4.000272in}}%
\pgfpathlineto{\pgfqpoint{3.329856in}{3.977213in}}%
\pgfpathlineto{\pgfqpoint{3.342810in}{3.954460in}}%
\pgfpathlineto{\pgfqpoint{3.355757in}{3.932012in}}%
\pgfpathlineto{\pgfqpoint{3.363206in}{3.955045in}}%
\pgfpathlineto{\pgfqpoint{3.370650in}{3.978410in}}%
\pgfpathlineto{\pgfqpoint{3.378089in}{4.002112in}}%
\pgfpathlineto{\pgfqpoint{3.385523in}{4.026158in}}%
\pgfpathlineto{\pgfqpoint{3.372576in}{4.049105in}}%
\pgfpathlineto{\pgfqpoint{3.359621in}{4.072358in}}%
\pgfpathlineto{\pgfqpoint{3.346658in}{4.095919in}}%
\pgfpathlineto{\pgfqpoint{3.333686in}{4.119791in}}%
\pgfpathlineto{\pgfqpoint{3.326253in}{4.095232in}}%
\pgfpathlineto{\pgfqpoint{3.318815in}{4.071024in}}%
\pgfpathlineto{\pgfqpoint{3.311371in}{4.047163in}}%
\pgfpathlineto{\pgfqpoint{3.303921in}{4.023641in}}%
\pgfpathclose%
\pgfusepath{fill}%
\end{pgfscope}%
\begin{pgfscope}%
\pgfpathrectangle{\pgfqpoint{1.254980in}{0.150000in}}{\pgfqpoint{5.490039in}{5.490039in}}%
\pgfusepath{clip}%
\pgfsetbuttcap%
\pgfsetroundjoin%
\definecolor{currentfill}{rgb}{0.208623,0.367752,0.552675}%
\pgfsetfillcolor{currentfill}%
\pgfsetfillopacity{0.700000}%
\pgfsetlinewidth{0.000000pt}%
\definecolor{currentstroke}{rgb}{0.000000,0.000000,0.000000}%
\pgfsetstrokecolor{currentstroke}%
\pgfsetdash{}{0pt}%
\pgfpathmoveto{\pgfqpoint{3.596411in}{3.134108in}}%
\pgfpathlineto{\pgfqpoint{3.609243in}{3.120966in}}%
\pgfpathlineto{\pgfqpoint{3.622074in}{3.108056in}}%
\pgfpathlineto{\pgfqpoint{3.634903in}{3.095376in}}%
\pgfpathlineto{\pgfqpoint{3.647732in}{3.082923in}}%
\pgfpathlineto{\pgfqpoint{3.655233in}{3.098535in}}%
\pgfpathlineto{\pgfqpoint{3.662729in}{3.114338in}}%
\pgfpathlineto{\pgfqpoint{3.670221in}{3.130338in}}%
\pgfpathlineto{\pgfqpoint{3.677708in}{3.146539in}}%
\pgfpathlineto{\pgfqpoint{3.664884in}{3.159351in}}%
\pgfpathlineto{\pgfqpoint{3.652060in}{3.172391in}}%
\pgfpathlineto{\pgfqpoint{3.639234in}{3.185661in}}%
\pgfpathlineto{\pgfqpoint{3.626407in}{3.199163in}}%
\pgfpathlineto{\pgfqpoint{3.618915in}{3.182592in}}%
\pgfpathlineto{\pgfqpoint{3.611418in}{3.166228in}}%
\pgfpathlineto{\pgfqpoint{3.603917in}{3.150069in}}%
\pgfpathlineto{\pgfqpoint{3.596411in}{3.134108in}}%
\pgfpathclose%
\pgfusepath{fill}%
\end{pgfscope}%
\begin{pgfscope}%
\pgfpathrectangle{\pgfqpoint{1.254980in}{0.150000in}}{\pgfqpoint{5.490039in}{5.490039in}}%
\pgfusepath{clip}%
\pgfsetbuttcap%
\pgfsetroundjoin%
\definecolor{currentfill}{rgb}{0.166617,0.463708,0.558119}%
\pgfsetfillcolor{currentfill}%
\pgfsetfillopacity{0.700000}%
\pgfsetlinewidth{0.000000pt}%
\definecolor{currentstroke}{rgb}{0.000000,0.000000,0.000000}%
\pgfsetstrokecolor{currentstroke}%
\pgfsetdash{}{0pt}%
\pgfpathmoveto{\pgfqpoint{3.390776in}{3.377404in}}%
\pgfpathlineto{\pgfqpoint{3.403654in}{3.360295in}}%
\pgfpathlineto{\pgfqpoint{3.416528in}{3.343451in}}%
\pgfpathlineto{\pgfqpoint{3.429397in}{3.326867in}}%
\pgfpathlineto{\pgfqpoint{3.442263in}{3.310543in}}%
\pgfpathlineto{\pgfqpoint{3.449784in}{3.327559in}}%
\pgfpathlineto{\pgfqpoint{3.457301in}{3.344796in}}%
\pgfpathlineto{\pgfqpoint{3.464813in}{3.362257in}}%
\pgfpathlineto{\pgfqpoint{3.472319in}{3.379949in}}%
\pgfpathlineto{\pgfqpoint{3.459458in}{3.396638in}}%
\pgfpathlineto{\pgfqpoint{3.446593in}{3.413586in}}%
\pgfpathlineto{\pgfqpoint{3.433724in}{3.430797in}}%
\pgfpathlineto{\pgfqpoint{3.420851in}{3.448272in}}%
\pgfpathlineto{\pgfqpoint{3.413340in}{3.430204in}}%
\pgfpathlineto{\pgfqpoint{3.405824in}{3.412373in}}%
\pgfpathlineto{\pgfqpoint{3.398303in}{3.394775in}}%
\pgfpathlineto{\pgfqpoint{3.390776in}{3.377404in}}%
\pgfpathclose%
\pgfusepath{fill}%
\end{pgfscope}%
\begin{pgfscope}%
\pgfpathrectangle{\pgfqpoint{1.254980in}{0.150000in}}{\pgfqpoint{5.490039in}{5.490039in}}%
\pgfusepath{clip}%
\pgfsetbuttcap%
\pgfsetroundjoin%
\definecolor{currentfill}{rgb}{0.449368,0.813768,0.335384}%
\pgfsetfillcolor{currentfill}%
\pgfsetfillopacity{0.700000}%
\pgfsetlinewidth{0.000000pt}%
\definecolor{currentstroke}{rgb}{0.000000,0.000000,0.000000}%
\pgfsetstrokecolor{currentstroke}%
\pgfsetdash{}{0pt}%
\pgfpathmoveto{\pgfqpoint{3.392966in}{4.329680in}}%
\pgfpathlineto{\pgfqpoint{3.405942in}{4.304703in}}%
\pgfpathlineto{\pgfqpoint{3.418910in}{4.280039in}}%
\pgfpathlineto{\pgfqpoint{3.431868in}{4.255685in}}%
\pgfpathlineto{\pgfqpoint{3.444819in}{4.231640in}}%
\pgfpathlineto{\pgfqpoint{3.452211in}{4.259062in}}%
\pgfpathlineto{\pgfqpoint{3.459598in}{4.286894in}}%
\pgfpathlineto{\pgfqpoint{3.466981in}{4.315141in}}%
\pgfpathlineto{\pgfqpoint{3.474360in}{4.343811in}}%
\pgfpathlineto{\pgfqpoint{3.461406in}{4.368457in}}%
\pgfpathlineto{\pgfqpoint{3.448443in}{4.393412in}}%
\pgfpathlineto{\pgfqpoint{3.435471in}{4.418679in}}%
\pgfpathlineto{\pgfqpoint{3.422491in}{4.444262in}}%
\pgfpathlineto{\pgfqpoint{3.415116in}{4.414976in}}%
\pgfpathlineto{\pgfqpoint{3.407738in}{4.386122in}}%
\pgfpathlineto{\pgfqpoint{3.400354in}{4.357692in}}%
\pgfpathlineto{\pgfqpoint{3.392966in}{4.329680in}}%
\pgfpathclose%
\pgfusepath{fill}%
\end{pgfscope}%
\begin{pgfscope}%
\pgfpathrectangle{\pgfqpoint{1.254980in}{0.150000in}}{\pgfqpoint{5.490039in}{5.490039in}}%
\pgfusepath{clip}%
\pgfsetbuttcap%
\pgfsetroundjoin%
\definecolor{currentfill}{rgb}{0.199430,0.387607,0.554642}%
\pgfsetfillcolor{currentfill}%
\pgfsetfillopacity{0.700000}%
\pgfsetlinewidth{0.000000pt}%
\definecolor{currentstroke}{rgb}{0.000000,0.000000,0.000000}%
\pgfsetstrokecolor{currentstroke}%
\pgfsetdash{}{0pt}%
\pgfpathmoveto{\pgfqpoint{4.501714in}{3.171445in}}%
\pgfpathlineto{\pgfqpoint{4.514630in}{3.165012in}}%
\pgfpathlineto{\pgfqpoint{4.527551in}{3.158753in}}%
\pgfpathlineto{\pgfqpoint{4.540478in}{3.152666in}}%
\pgfpathlineto{\pgfqpoint{4.553411in}{3.146751in}}%
\pgfpathlineto{\pgfqpoint{4.560748in}{3.163608in}}%
\pgfpathlineto{\pgfqpoint{4.568086in}{3.180739in}}%
\pgfpathlineto{\pgfqpoint{4.575423in}{3.198152in}}%
\pgfpathlineto{\pgfqpoint{4.582761in}{3.215854in}}%
\pgfpathlineto{\pgfqpoint{4.569837in}{3.222344in}}%
\pgfpathlineto{\pgfqpoint{4.556918in}{3.229006in}}%
\pgfpathlineto{\pgfqpoint{4.544006in}{3.235841in}}%
\pgfpathlineto{\pgfqpoint{4.531099in}{3.242849in}}%
\pgfpathlineto{\pgfqpoint{4.523752in}{3.224561in}}%
\pgfpathlineto{\pgfqpoint{4.516406in}{3.206569in}}%
\pgfpathlineto{\pgfqpoint{4.509060in}{3.188866in}}%
\pgfpathlineto{\pgfqpoint{4.501714in}{3.171445in}}%
\pgfpathclose%
\pgfusepath{fill}%
\end{pgfscope}%
\begin{pgfscope}%
\pgfpathrectangle{\pgfqpoint{1.254980in}{0.150000in}}{\pgfqpoint{5.490039in}{5.490039in}}%
\pgfusepath{clip}%
\pgfsetbuttcap%
\pgfsetroundjoin%
\definecolor{currentfill}{rgb}{0.206756,0.371758,0.553117}%
\pgfsetfillcolor{currentfill}%
\pgfsetfillopacity{0.700000}%
\pgfsetlinewidth{0.000000pt}%
\definecolor{currentstroke}{rgb}{0.000000,0.000000,0.000000}%
\pgfsetstrokecolor{currentstroke}%
\pgfsetdash{}{0pt}%
\pgfpathmoveto{\pgfqpoint{4.420688in}{3.129736in}}%
\pgfpathlineto{\pgfqpoint{4.433590in}{3.123150in}}%
\pgfpathlineto{\pgfqpoint{4.446497in}{3.116741in}}%
\pgfpathlineto{\pgfqpoint{4.459410in}{3.110507in}}%
\pgfpathlineto{\pgfqpoint{4.472328in}{3.104448in}}%
\pgfpathlineto{\pgfqpoint{4.479675in}{3.120807in}}%
\pgfpathlineto{\pgfqpoint{4.487022in}{3.137421in}}%
\pgfpathlineto{\pgfqpoint{4.494368in}{3.154299in}}%
\pgfpathlineto{\pgfqpoint{4.501714in}{3.171445in}}%
\pgfpathlineto{\pgfqpoint{4.488804in}{3.178051in}}%
\pgfpathlineto{\pgfqpoint{4.475899in}{3.184833in}}%
\pgfpathlineto{\pgfqpoint{4.463000in}{3.191790in}}%
\pgfpathlineto{\pgfqpoint{4.450106in}{3.198923in}}%
\pgfpathlineto{\pgfqpoint{4.442752in}{3.181219in}}%
\pgfpathlineto{\pgfqpoint{4.435398in}{3.163791in}}%
\pgfpathlineto{\pgfqpoint{4.428043in}{3.146632in}}%
\pgfpathlineto{\pgfqpoint{4.420688in}{3.129736in}}%
\pgfpathclose%
\pgfusepath{fill}%
\end{pgfscope}%
\begin{pgfscope}%
\pgfpathrectangle{\pgfqpoint{1.254980in}{0.150000in}}{\pgfqpoint{5.490039in}{5.490039in}}%
\pgfusepath{clip}%
\pgfsetbuttcap%
\pgfsetroundjoin%
\definecolor{currentfill}{rgb}{0.229739,0.322361,0.545706}%
\pgfsetfillcolor{currentfill}%
\pgfsetfillopacity{0.700000}%
\pgfsetlinewidth{0.000000pt}%
\definecolor{currentstroke}{rgb}{0.000000,0.000000,0.000000}%
\pgfsetstrokecolor{currentstroke}%
\pgfsetdash{}{0pt}%
\pgfpathmoveto{\pgfqpoint{4.045146in}{3.021881in}}%
\pgfpathlineto{\pgfqpoint{4.057990in}{3.013514in}}%
\pgfpathlineto{\pgfqpoint{4.070837in}{3.005341in}}%
\pgfpathlineto{\pgfqpoint{4.083688in}{2.997360in}}%
\pgfpathlineto{\pgfqpoint{4.096542in}{2.989571in}}%
\pgfpathlineto{\pgfqpoint{4.103956in}{3.004774in}}%
\pgfpathlineto{\pgfqpoint{4.111368in}{3.020176in}}%
\pgfpathlineto{\pgfqpoint{4.118777in}{3.035783in}}%
\pgfpathlineto{\pgfqpoint{4.126183in}{3.051598in}}%
\pgfpathlineto{\pgfqpoint{4.113335in}{3.059825in}}%
\pgfpathlineto{\pgfqpoint{4.100491in}{3.068244in}}%
\pgfpathlineto{\pgfqpoint{4.087650in}{3.076856in}}%
\pgfpathlineto{\pgfqpoint{4.074812in}{3.085662in}}%
\pgfpathlineto{\pgfqpoint{4.067400in}{3.069397in}}%
\pgfpathlineto{\pgfqpoint{4.059985in}{3.053349in}}%
\pgfpathlineto{\pgfqpoint{4.052567in}{3.037512in}}%
\pgfpathlineto{\pgfqpoint{4.045146in}{3.021881in}}%
\pgfpathclose%
\pgfusepath{fill}%
\end{pgfscope}%
\begin{pgfscope}%
\pgfpathrectangle{\pgfqpoint{1.254980in}{0.150000in}}{\pgfqpoint{5.490039in}{5.490039in}}%
\pgfusepath{clip}%
\pgfsetbuttcap%
\pgfsetroundjoin%
\definecolor{currentfill}{rgb}{0.190631,0.407061,0.556089}%
\pgfsetfillcolor{currentfill}%
\pgfsetfillopacity{0.700000}%
\pgfsetlinewidth{0.000000pt}%
\definecolor{currentstroke}{rgb}{0.000000,0.000000,0.000000}%
\pgfsetstrokecolor{currentstroke}%
\pgfsetdash{}{0pt}%
\pgfpathmoveto{\pgfqpoint{4.582761in}{3.215854in}}%
\pgfpathlineto{\pgfqpoint{4.595691in}{3.209536in}}%
\pgfpathlineto{\pgfqpoint{4.608627in}{3.203388in}}%
\pgfpathlineto{\pgfqpoint{4.621570in}{3.197410in}}%
\pgfpathlineto{\pgfqpoint{4.634518in}{3.191602in}}%
\pgfpathlineto{\pgfqpoint{4.641848in}{3.209006in}}%
\pgfpathlineto{\pgfqpoint{4.649178in}{3.226707in}}%
\pgfpathlineto{\pgfqpoint{4.656510in}{3.244709in}}%
\pgfpathlineto{\pgfqpoint{4.663843in}{3.263022in}}%
\pgfpathlineto{\pgfqpoint{4.650904in}{3.269433in}}%
\pgfpathlineto{\pgfqpoint{4.637971in}{3.276014in}}%
\pgfpathlineto{\pgfqpoint{4.625044in}{3.282765in}}%
\pgfpathlineto{\pgfqpoint{4.612123in}{3.289687in}}%
\pgfpathlineto{\pgfqpoint{4.604780in}{3.270761in}}%
\pgfpathlineto{\pgfqpoint{4.597440in}{3.252151in}}%
\pgfpathlineto{\pgfqpoint{4.590100in}{3.233852in}}%
\pgfpathlineto{\pgfqpoint{4.582761in}{3.215854in}}%
\pgfpathclose%
\pgfusepath{fill}%
\end{pgfscope}%
\begin{pgfscope}%
\pgfpathrectangle{\pgfqpoint{1.254980in}{0.150000in}}{\pgfqpoint{5.490039in}{5.490039in}}%
\pgfusepath{clip}%
\pgfsetbuttcap%
\pgfsetroundjoin%
\definecolor{currentfill}{rgb}{0.214298,0.355619,0.551184}%
\pgfsetfillcolor{currentfill}%
\pgfsetfillopacity{0.700000}%
\pgfsetlinewidth{0.000000pt}%
\definecolor{currentstroke}{rgb}{0.000000,0.000000,0.000000}%
\pgfsetstrokecolor{currentstroke}%
\pgfsetdash{}{0pt}%
\pgfpathmoveto{\pgfqpoint{4.339670in}{3.090693in}}%
\pgfpathlineto{\pgfqpoint{4.352559in}{3.083915in}}%
\pgfpathlineto{\pgfqpoint{4.365453in}{3.077317in}}%
\pgfpathlineto{\pgfqpoint{4.378353in}{3.070897in}}%
\pgfpathlineto{\pgfqpoint{4.391257in}{3.064655in}}%
\pgfpathlineto{\pgfqpoint{4.398617in}{3.080562in}}%
\pgfpathlineto{\pgfqpoint{4.405975in}{3.096707in}}%
\pgfpathlineto{\pgfqpoint{4.413332in}{3.113096in}}%
\pgfpathlineto{\pgfqpoint{4.420688in}{3.129736in}}%
\pgfpathlineto{\pgfqpoint{4.407791in}{3.136498in}}%
\pgfpathlineto{\pgfqpoint{4.394900in}{3.143438in}}%
\pgfpathlineto{\pgfqpoint{4.382014in}{3.150557in}}%
\pgfpathlineto{\pgfqpoint{4.369132in}{3.157855in}}%
\pgfpathlineto{\pgfqpoint{4.361768in}{3.140685in}}%
\pgfpathlineto{\pgfqpoint{4.354404in}{3.123772in}}%
\pgfpathlineto{\pgfqpoint{4.347038in}{3.107110in}}%
\pgfpathlineto{\pgfqpoint{4.339670in}{3.090693in}}%
\pgfpathclose%
\pgfusepath{fill}%
\end{pgfscope}%
\begin{pgfscope}%
\pgfpathrectangle{\pgfqpoint{1.254980in}{0.150000in}}{\pgfqpoint{5.490039in}{5.490039in}}%
\pgfusepath{clip}%
\pgfsetbuttcap%
\pgfsetroundjoin%
\definecolor{currentfill}{rgb}{0.751884,0.874951,0.143228}%
\pgfsetfillcolor{currentfill}%
\pgfsetfillopacity{0.700000}%
\pgfsetlinewidth{0.000000pt}%
\definecolor{currentstroke}{rgb}{0.000000,0.000000,0.000000}%
\pgfsetstrokecolor{currentstroke}%
\pgfsetdash{}{0pt}%
\pgfpathmoveto{\pgfqpoint{3.614446in}{4.620071in}}%
\pgfpathlineto{\pgfqpoint{3.627380in}{4.594933in}}%
\pgfpathlineto{\pgfqpoint{3.640307in}{4.570090in}}%
\pgfpathlineto{\pgfqpoint{3.653226in}{4.545540in}}%
\pgfpathlineto{\pgfqpoint{3.666139in}{4.521280in}}%
\pgfpathlineto{\pgfqpoint{3.673492in}{4.554712in}}%
\pgfpathlineto{\pgfqpoint{3.680843in}{4.588661in}}%
\pgfpathlineto{\pgfqpoint{3.688193in}{4.623136in}}%
\pgfpathlineto{\pgfqpoint{3.675275in}{4.647942in}}%
\pgfpathlineto{\pgfqpoint{3.662349in}{4.673039in}}%
\pgfpathlineto{\pgfqpoint{3.649416in}{4.698430in}}%
\pgfpathlineto{\pgfqpoint{3.636476in}{4.724119in}}%
\pgfpathlineto{\pgfqpoint{3.629135in}{4.688903in}}%
\pgfpathlineto{\pgfqpoint{3.621791in}{4.654224in}}%
\pgfpathlineto{\pgfqpoint{3.614446in}{4.620071in}}%
\pgfpathclose%
\pgfusepath{fill}%
\end{pgfscope}%
\begin{pgfscope}%
\pgfpathrectangle{\pgfqpoint{1.254980in}{0.150000in}}{\pgfqpoint{5.490039in}{5.490039in}}%
\pgfusepath{clip}%
\pgfsetbuttcap%
\pgfsetroundjoin%
\definecolor{currentfill}{rgb}{0.150148,0.676631,0.506589}%
\pgfsetfillcolor{currentfill}%
\pgfsetfillopacity{0.700000}%
\pgfsetlinewidth{0.000000pt}%
\definecolor{currentstroke}{rgb}{0.000000,0.000000,0.000000}%
\pgfsetstrokecolor{currentstroke}%
\pgfsetdash{}{0pt}%
\pgfpathmoveto{\pgfqpoint{3.274067in}{3.932839in}}%
\pgfpathlineto{\pgfqpoint{3.287039in}{3.909939in}}%
\pgfpathlineto{\pgfqpoint{3.300003in}{3.887349in}}%
\pgfpathlineto{\pgfqpoint{3.312958in}{3.865065in}}%
\pgfpathlineto{\pgfqpoint{3.325906in}{3.843084in}}%
\pgfpathlineto{\pgfqpoint{3.333377in}{3.864847in}}%
\pgfpathlineto{\pgfqpoint{3.340842in}{3.886919in}}%
\pgfpathlineto{\pgfqpoint{3.348302in}{3.909305in}}%
\pgfpathlineto{\pgfqpoint{3.355757in}{3.932012in}}%
\pgfpathlineto{\pgfqpoint{3.342810in}{3.954460in}}%
\pgfpathlineto{\pgfqpoint{3.329856in}{3.977213in}}%
\pgfpathlineto{\pgfqpoint{3.316893in}{4.000272in}}%
\pgfpathlineto{\pgfqpoint{3.303921in}{4.023641in}}%
\pgfpathlineto{\pgfqpoint{3.296466in}{4.000454in}}%
\pgfpathlineto{\pgfqpoint{3.289005in}{3.977594in}}%
\pgfpathlineto{\pgfqpoint{3.281539in}{3.955058in}}%
\pgfpathlineto{\pgfqpoint{3.274067in}{3.932839in}}%
\pgfpathclose%
\pgfusepath{fill}%
\end{pgfscope}%
\begin{pgfscope}%
\pgfpathrectangle{\pgfqpoint{1.254980in}{0.150000in}}{\pgfqpoint{5.490039in}{5.490039in}}%
\pgfusepath{clip}%
\pgfsetbuttcap%
\pgfsetroundjoin%
\definecolor{currentfill}{rgb}{0.218130,0.347432,0.550038}%
\pgfsetfillcolor{currentfill}%
\pgfsetfillopacity{0.700000}%
\pgfsetlinewidth{0.000000pt}%
\definecolor{currentstroke}{rgb}{0.000000,0.000000,0.000000}%
\pgfsetstrokecolor{currentstroke}%
\pgfsetdash{}{0pt}%
\pgfpathmoveto{\pgfqpoint{3.647732in}{3.082923in}}%
\pgfpathlineto{\pgfqpoint{3.660561in}{3.070698in}}%
\pgfpathlineto{\pgfqpoint{3.673389in}{3.058697in}}%
\pgfpathlineto{\pgfqpoint{3.686216in}{3.046919in}}%
\pgfpathlineto{\pgfqpoint{3.699044in}{3.035363in}}%
\pgfpathlineto{\pgfqpoint{3.706538in}{3.050626in}}%
\pgfpathlineto{\pgfqpoint{3.714029in}{3.066075in}}%
\pgfpathlineto{\pgfqpoint{3.721515in}{3.081712in}}%
\pgfpathlineto{\pgfqpoint{3.728997in}{3.097543in}}%
\pgfpathlineto{\pgfqpoint{3.716175in}{3.109458in}}%
\pgfpathlineto{\pgfqpoint{3.703353in}{3.121594in}}%
\pgfpathlineto{\pgfqpoint{3.690531in}{3.133954in}}%
\pgfpathlineto{\pgfqpoint{3.677708in}{3.146539in}}%
\pgfpathlineto{\pgfqpoint{3.670221in}{3.130338in}}%
\pgfpathlineto{\pgfqpoint{3.662729in}{3.114338in}}%
\pgfpathlineto{\pgfqpoint{3.655233in}{3.098535in}}%
\pgfpathlineto{\pgfqpoint{3.647732in}{3.082923in}}%
\pgfpathclose%
\pgfusepath{fill}%
\end{pgfscope}%
\begin{pgfscope}%
\pgfpathrectangle{\pgfqpoint{1.254980in}{0.150000in}}{\pgfqpoint{5.490039in}{5.490039in}}%
\pgfusepath{clip}%
\pgfsetbuttcap%
\pgfsetroundjoin%
\definecolor{currentfill}{rgb}{0.180629,0.429975,0.557282}%
\pgfsetfillcolor{currentfill}%
\pgfsetfillopacity{0.700000}%
\pgfsetlinewidth{0.000000pt}%
\definecolor{currentstroke}{rgb}{0.000000,0.000000,0.000000}%
\pgfsetstrokecolor{currentstroke}%
\pgfsetdash{}{0pt}%
\pgfpathmoveto{\pgfqpoint{4.663843in}{3.263022in}}%
\pgfpathlineto{\pgfqpoint{4.676788in}{3.256779in}}%
\pgfpathlineto{\pgfqpoint{4.689740in}{3.250705in}}%
\pgfpathlineto{\pgfqpoint{4.702698in}{3.244798in}}%
\pgfpathlineto{\pgfqpoint{4.715662in}{3.239059in}}%
\pgfpathlineto{\pgfqpoint{4.722987in}{3.257067in}}%
\pgfpathlineto{\pgfqpoint{4.730313in}{3.275393in}}%
\pgfpathlineto{\pgfqpoint{4.737642in}{3.294044in}}%
\pgfpathlineto{\pgfqpoint{4.744973in}{3.313027in}}%
\pgfpathlineto{\pgfqpoint{4.732019in}{3.319397in}}%
\pgfpathlineto{\pgfqpoint{4.719071in}{3.325935in}}%
\pgfpathlineto{\pgfqpoint{4.706129in}{3.332640in}}%
\pgfpathlineto{\pgfqpoint{4.693193in}{3.339514in}}%
\pgfpathlineto{\pgfqpoint{4.685852in}{3.319889in}}%
\pgfpathlineto{\pgfqpoint{4.678514in}{3.300604in}}%
\pgfpathlineto{\pgfqpoint{4.671177in}{3.281651in}}%
\pgfpathlineto{\pgfqpoint{4.663843in}{3.263022in}}%
\pgfpathclose%
\pgfusepath{fill}%
\end{pgfscope}%
\begin{pgfscope}%
\pgfpathrectangle{\pgfqpoint{1.254980in}{0.150000in}}{\pgfqpoint{5.490039in}{5.490039in}}%
\pgfusepath{clip}%
\pgfsetbuttcap%
\pgfsetroundjoin%
\definecolor{currentfill}{rgb}{0.156270,0.489624,0.557936}%
\pgfsetfillcolor{currentfill}%
\pgfsetfillopacity{0.700000}%
\pgfsetlinewidth{0.000000pt}%
\definecolor{currentstroke}{rgb}{0.000000,0.000000,0.000000}%
\pgfsetstrokecolor{currentstroke}%
\pgfsetdash{}{0pt}%
\pgfpathmoveto{\pgfqpoint{3.339214in}{3.448524in}}%
\pgfpathlineto{\pgfqpoint{3.352112in}{3.430336in}}%
\pgfpathlineto{\pgfqpoint{3.365005in}{3.412422in}}%
\pgfpathlineto{\pgfqpoint{3.377893in}{3.394779in}}%
\pgfpathlineto{\pgfqpoint{3.390776in}{3.377404in}}%
\pgfpathlineto{\pgfqpoint{3.398303in}{3.394775in}}%
\pgfpathlineto{\pgfqpoint{3.405824in}{3.412373in}}%
\pgfpathlineto{\pgfqpoint{3.413340in}{3.430204in}}%
\pgfpathlineto{\pgfqpoint{3.420851in}{3.448272in}}%
\pgfpathlineto{\pgfqpoint{3.407972in}{3.466014in}}%
\pgfpathlineto{\pgfqpoint{3.395089in}{3.484024in}}%
\pgfpathlineto{\pgfqpoint{3.382201in}{3.502307in}}%
\pgfpathlineto{\pgfqpoint{3.369307in}{3.520863in}}%
\pgfpathlineto{\pgfqpoint{3.361792in}{3.502416in}}%
\pgfpathlineto{\pgfqpoint{3.354272in}{3.484214in}}%
\pgfpathlineto{\pgfqpoint{3.346746in}{3.466251in}}%
\pgfpathlineto{\pgfqpoint{3.339214in}{3.448524in}}%
\pgfpathclose%
\pgfusepath{fill}%
\end{pgfscope}%
\begin{pgfscope}%
\pgfpathrectangle{\pgfqpoint{1.254980in}{0.150000in}}{\pgfqpoint{5.490039in}{5.490039in}}%
\pgfusepath{clip}%
\pgfsetbuttcap%
\pgfsetroundjoin%
\definecolor{currentfill}{rgb}{0.221989,0.339161,0.548752}%
\pgfsetfillcolor{currentfill}%
\pgfsetfillopacity{0.700000}%
\pgfsetlinewidth{0.000000pt}%
\definecolor{currentstroke}{rgb}{0.000000,0.000000,0.000000}%
\pgfsetstrokecolor{currentstroke}%
\pgfsetdash{}{0pt}%
\pgfpathmoveto{\pgfqpoint{4.258649in}{3.054306in}}%
\pgfpathlineto{\pgfqpoint{4.271526in}{3.047296in}}%
\pgfpathlineto{\pgfqpoint{4.284408in}{3.040469in}}%
\pgfpathlineto{\pgfqpoint{4.297294in}{3.033823in}}%
\pgfpathlineto{\pgfqpoint{4.310186in}{3.027358in}}%
\pgfpathlineto{\pgfqpoint{4.317560in}{3.042853in}}%
\pgfpathlineto{\pgfqpoint{4.324932in}{3.058571in}}%
\pgfpathlineto{\pgfqpoint{4.332302in}{3.074515in}}%
\pgfpathlineto{\pgfqpoint{4.339670in}{3.090693in}}%
\pgfpathlineto{\pgfqpoint{4.326786in}{3.097651in}}%
\pgfpathlineto{\pgfqpoint{4.313907in}{3.104789in}}%
\pgfpathlineto{\pgfqpoint{4.301032in}{3.112110in}}%
\pgfpathlineto{\pgfqpoint{4.288162in}{3.119612in}}%
\pgfpathlineto{\pgfqpoint{4.280787in}{3.102932in}}%
\pgfpathlineto{\pgfqpoint{4.273409in}{3.086491in}}%
\pgfpathlineto{\pgfqpoint{4.266030in}{3.070284in}}%
\pgfpathlineto{\pgfqpoint{4.258649in}{3.054306in}}%
\pgfpathclose%
\pgfusepath{fill}%
\end{pgfscope}%
\begin{pgfscope}%
\pgfpathrectangle{\pgfqpoint{1.254980in}{0.150000in}}{\pgfqpoint{5.490039in}{5.490039in}}%
\pgfusepath{clip}%
\pgfsetbuttcap%
\pgfsetroundjoin%
\definecolor{currentfill}{rgb}{0.231674,0.318106,0.544834}%
\pgfsetfillcolor{currentfill}%
\pgfsetfillopacity{0.700000}%
\pgfsetlinewidth{0.000000pt}%
\definecolor{currentstroke}{rgb}{0.000000,0.000000,0.000000}%
\pgfsetstrokecolor{currentstroke}%
\pgfsetdash{}{0pt}%
\pgfpathmoveto{\pgfqpoint{3.831587in}{3.010037in}}%
\pgfpathlineto{\pgfqpoint{3.844415in}{3.000053in}}%
\pgfpathlineto{\pgfqpoint{3.857245in}{2.990277in}}%
\pgfpathlineto{\pgfqpoint{3.870077in}{2.980707in}}%
\pgfpathlineto{\pgfqpoint{3.882910in}{2.971342in}}%
\pgfpathlineto{\pgfqpoint{3.890370in}{2.986247in}}%
\pgfpathlineto{\pgfqpoint{3.897827in}{3.001333in}}%
\pgfpathlineto{\pgfqpoint{3.905280in}{3.016606in}}%
\pgfpathlineto{\pgfqpoint{3.912730in}{3.032069in}}%
\pgfpathlineto{\pgfqpoint{3.899902in}{3.041818in}}%
\pgfpathlineto{\pgfqpoint{3.887077in}{3.051773in}}%
\pgfpathlineto{\pgfqpoint{3.874253in}{3.061933in}}%
\pgfpathlineto{\pgfqpoint{3.861430in}{3.072302in}}%
\pgfpathlineto{\pgfqpoint{3.853975in}{3.056444in}}%
\pgfpathlineto{\pgfqpoint{3.846516in}{3.040783in}}%
\pgfpathlineto{\pgfqpoint{3.839054in}{3.025316in}}%
\pgfpathlineto{\pgfqpoint{3.831587in}{3.010037in}}%
\pgfpathclose%
\pgfusepath{fill}%
\end{pgfscope}%
\begin{pgfscope}%
\pgfpathrectangle{\pgfqpoint{1.254980in}{0.150000in}}{\pgfqpoint{5.490039in}{5.490039in}}%
\pgfusepath{clip}%
\pgfsetbuttcap%
\pgfsetroundjoin%
\definecolor{currentfill}{rgb}{0.565498,0.842430,0.262877}%
\pgfsetfillcolor{currentfill}%
\pgfsetfillopacity{0.700000}%
\pgfsetlinewidth{0.000000pt}%
\definecolor{currentstroke}{rgb}{0.000000,0.000000,0.000000}%
\pgfsetstrokecolor{currentstroke}%
\pgfsetdash{}{0pt}%
\pgfpathmoveto{\pgfqpoint{3.422491in}{4.444262in}}%
\pgfpathlineto{\pgfqpoint{3.435471in}{4.418679in}}%
\pgfpathlineto{\pgfqpoint{3.448443in}{4.393412in}}%
\pgfpathlineto{\pgfqpoint{3.461406in}{4.368457in}}%
\pgfpathlineto{\pgfqpoint{3.474360in}{4.343811in}}%
\pgfpathlineto{\pgfqpoint{3.481735in}{4.372912in}}%
\pgfpathlineto{\pgfqpoint{3.489106in}{4.402451in}}%
\pgfpathlineto{\pgfqpoint{3.496474in}{4.432435in}}%
\pgfpathlineto{\pgfqpoint{3.503838in}{4.462872in}}%
\pgfpathlineto{\pgfqpoint{3.490878in}{4.488154in}}%
\pgfpathlineto{\pgfqpoint{3.477910in}{4.513746in}}%
\pgfpathlineto{\pgfqpoint{3.464933in}{4.539652in}}%
\pgfpathlineto{\pgfqpoint{3.451947in}{4.565874in}}%
\pgfpathlineto{\pgfqpoint{3.444589in}{4.534786in}}%
\pgfpathlineto{\pgfqpoint{3.437227in}{4.504159in}}%
\pgfpathlineto{\pgfqpoint{3.429861in}{4.473987in}}%
\pgfpathlineto{\pgfqpoint{3.422491in}{4.444262in}}%
\pgfpathclose%
\pgfusepath{fill}%
\end{pgfscope}%
\begin{pgfscope}%
\pgfpathrectangle{\pgfqpoint{1.254980in}{0.150000in}}{\pgfqpoint{5.490039in}{5.490039in}}%
\pgfusepath{clip}%
\pgfsetbuttcap%
\pgfsetroundjoin%
\definecolor{currentfill}{rgb}{0.235526,0.309527,0.542944}%
\pgfsetfillcolor{currentfill}%
\pgfsetfillopacity{0.700000}%
\pgfsetlinewidth{0.000000pt}%
\definecolor{currentstroke}{rgb}{0.000000,0.000000,0.000000}%
\pgfsetstrokecolor{currentstroke}%
\pgfsetdash{}{0pt}%
\pgfpathmoveto{\pgfqpoint{3.964060in}{2.995100in}}%
\pgfpathlineto{\pgfqpoint{3.976899in}{2.986359in}}%
\pgfpathlineto{\pgfqpoint{3.989740in}{2.977815in}}%
\pgfpathlineto{\pgfqpoint{4.002584in}{2.969469in}}%
\pgfpathlineto{\pgfqpoint{4.015431in}{2.961318in}}%
\pgfpathlineto{\pgfqpoint{4.022865in}{2.976174in}}%
\pgfpathlineto{\pgfqpoint{4.030295in}{2.991217in}}%
\pgfpathlineto{\pgfqpoint{4.037722in}{3.006451in}}%
\pgfpathlineto{\pgfqpoint{4.045146in}{3.021881in}}%
\pgfpathlineto{\pgfqpoint{4.032305in}{3.030443in}}%
\pgfpathlineto{\pgfqpoint{4.019467in}{3.039201in}}%
\pgfpathlineto{\pgfqpoint{4.006632in}{3.048155in}}%
\pgfpathlineto{\pgfqpoint{3.993799in}{3.057308in}}%
\pgfpathlineto{\pgfqpoint{3.986370in}{3.041457in}}%
\pgfpathlineto{\pgfqpoint{3.978936in}{3.025808in}}%
\pgfpathlineto{\pgfqpoint{3.971500in}{3.010358in}}%
\pgfpathlineto{\pgfqpoint{3.964060in}{2.995100in}}%
\pgfpathclose%
\pgfusepath{fill}%
\end{pgfscope}%
\begin{pgfscope}%
\pgfpathrectangle{\pgfqpoint{1.254980in}{0.150000in}}{\pgfqpoint{5.490039in}{5.490039in}}%
\pgfusepath{clip}%
\pgfsetbuttcap%
\pgfsetroundjoin%
\definecolor{currentfill}{rgb}{0.123463,0.581687,0.547445}%
\pgfsetfillcolor{currentfill}%
\pgfsetfillopacity{0.700000}%
\pgfsetlinewidth{0.000000pt}%
\definecolor{currentstroke}{rgb}{0.000000,0.000000,0.000000}%
\pgfsetstrokecolor{currentstroke}%
\pgfsetdash{}{0pt}%
\pgfpathmoveto{\pgfqpoint{3.265934in}{3.679487in}}%
\pgfpathlineto{\pgfqpoint{3.278880in}{3.658644in}}%
\pgfpathlineto{\pgfqpoint{3.291818in}{3.638097in}}%
\pgfpathlineto{\pgfqpoint{3.304749in}{3.617842in}}%
\pgfpathlineto{\pgfqpoint{3.317673in}{3.597878in}}%
\pgfpathlineto{\pgfqpoint{3.325187in}{3.616962in}}%
\pgfpathlineto{\pgfqpoint{3.332694in}{3.636307in}}%
\pgfpathlineto{\pgfqpoint{3.340196in}{3.655918in}}%
\pgfpathlineto{\pgfqpoint{3.347692in}{3.675800in}}%
\pgfpathlineto{\pgfqpoint{3.334771in}{3.696166in}}%
\pgfpathlineto{\pgfqpoint{3.321844in}{3.716823in}}%
\pgfpathlineto{\pgfqpoint{3.308909in}{3.737773in}}%
\pgfpathlineto{\pgfqpoint{3.295966in}{3.759019in}}%
\pgfpathlineto{\pgfqpoint{3.288467in}{3.738723in}}%
\pgfpathlineto{\pgfqpoint{3.280962in}{3.718706in}}%
\pgfpathlineto{\pgfqpoint{3.273451in}{3.698962in}}%
\pgfpathlineto{\pgfqpoint{3.265934in}{3.679487in}}%
\pgfpathclose%
\pgfusepath{fill}%
\end{pgfscope}%
\begin{pgfscope}%
\pgfpathrectangle{\pgfqpoint{1.254980in}{0.150000in}}{\pgfqpoint{5.490039in}{5.490039in}}%
\pgfusepath{clip}%
\pgfsetbuttcap%
\pgfsetroundjoin%
\definecolor{currentfill}{rgb}{0.227802,0.326594,0.546532}%
\pgfsetfillcolor{currentfill}%
\pgfsetfillopacity{0.700000}%
\pgfsetlinewidth{0.000000pt}%
\definecolor{currentstroke}{rgb}{0.000000,0.000000,0.000000}%
\pgfsetstrokecolor{currentstroke}%
\pgfsetdash{}{0pt}%
\pgfpathmoveto{\pgfqpoint{4.177610in}{3.020587in}}%
\pgfpathlineto{\pgfqpoint{4.190477in}{3.013304in}}%
\pgfpathlineto{\pgfqpoint{4.203348in}{3.006207in}}%
\pgfpathlineto{\pgfqpoint{4.216223in}{2.999294in}}%
\pgfpathlineto{\pgfqpoint{4.229103in}{2.992566in}}%
\pgfpathlineto{\pgfqpoint{4.236493in}{3.007686in}}%
\pgfpathlineto{\pgfqpoint{4.243880in}{3.023013in}}%
\pgfpathlineto{\pgfqpoint{4.251265in}{3.038551in}}%
\pgfpathlineto{\pgfqpoint{4.258649in}{3.054306in}}%
\pgfpathlineto{\pgfqpoint{4.245776in}{3.061499in}}%
\pgfpathlineto{\pgfqpoint{4.232908in}{3.068877in}}%
\pgfpathlineto{\pgfqpoint{4.220044in}{3.076440in}}%
\pgfpathlineto{\pgfqpoint{4.207184in}{3.084188in}}%
\pgfpathlineto{\pgfqpoint{4.199794in}{3.067957in}}%
\pgfpathlineto{\pgfqpoint{4.192401in}{3.051950in}}%
\pgfpathlineto{\pgfqpoint{4.185007in}{3.036162in}}%
\pgfpathlineto{\pgfqpoint{4.177610in}{3.020587in}}%
\pgfpathclose%
\pgfusepath{fill}%
\end{pgfscope}%
\begin{pgfscope}%
\pgfpathrectangle{\pgfqpoint{1.254980in}{0.150000in}}{\pgfqpoint{5.490039in}{5.490039in}}%
\pgfusepath{clip}%
\pgfsetbuttcap%
\pgfsetroundjoin%
\definecolor{currentfill}{rgb}{0.730889,0.871916,0.156029}%
\pgfsetfillcolor{currentfill}%
\pgfsetfillopacity{0.700000}%
\pgfsetlinewidth{0.000000pt}%
\definecolor{currentstroke}{rgb}{0.000000,0.000000,0.000000}%
\pgfsetstrokecolor{currentstroke}%
\pgfsetdash{}{0pt}%
\pgfpathmoveto{\pgfqpoint{3.533259in}{4.589310in}}%
\pgfpathlineto{\pgfqpoint{3.546217in}{4.563666in}}%
\pgfpathlineto{\pgfqpoint{3.559167in}{4.538327in}}%
\pgfpathlineto{\pgfqpoint{3.572109in}{4.513292in}}%
\pgfpathlineto{\pgfqpoint{3.585043in}{4.488557in}}%
\pgfpathlineto{\pgfqpoint{3.592397in}{4.520688in}}%
\pgfpathlineto{\pgfqpoint{3.599749in}{4.553312in}}%
\pgfpathlineto{\pgfqpoint{3.607099in}{4.586437in}}%
\pgfpathlineto{\pgfqpoint{3.614446in}{4.620071in}}%
\pgfpathlineto{\pgfqpoint{3.601505in}{4.645509in}}%
\pgfpathlineto{\pgfqpoint{3.588555in}{4.671248in}}%
\pgfpathlineto{\pgfqpoint{3.575598in}{4.697291in}}%
\pgfpathlineto{\pgfqpoint{3.562632in}{4.723643in}}%
\pgfpathlineto{\pgfqpoint{3.555293in}{4.689290in}}%
\pgfpathlineto{\pgfqpoint{3.547951in}{4.655456in}}%
\pgfpathlineto{\pgfqpoint{3.540607in}{4.622132in}}%
\pgfpathlineto{\pgfqpoint{3.533259in}{4.589310in}}%
\pgfpathclose%
\pgfusepath{fill}%
\end{pgfscope}%
\begin{pgfscope}%
\pgfpathrectangle{\pgfqpoint{1.254980in}{0.150000in}}{\pgfqpoint{5.490039in}{5.490039in}}%
\pgfusepath{clip}%
\pgfsetbuttcap%
\pgfsetroundjoin%
\definecolor{currentfill}{rgb}{0.174274,0.445044,0.557792}%
\pgfsetfillcolor{currentfill}%
\pgfsetfillopacity{0.700000}%
\pgfsetlinewidth{0.000000pt}%
\definecolor{currentstroke}{rgb}{0.000000,0.000000,0.000000}%
\pgfsetstrokecolor{currentstroke}%
\pgfsetdash{}{0pt}%
\pgfpathmoveto{\pgfqpoint{4.744973in}{3.313027in}}%
\pgfpathlineto{\pgfqpoint{4.757934in}{3.306823in}}%
\pgfpathlineto{\pgfqpoint{4.770901in}{3.300785in}}%
\pgfpathlineto{\pgfqpoint{4.783875in}{3.294912in}}%
\pgfpathlineto{\pgfqpoint{4.796856in}{3.289204in}}%
\pgfpathlineto{\pgfqpoint{4.804179in}{3.307878in}}%
\pgfpathlineto{\pgfqpoint{4.811505in}{3.326893in}}%
\pgfpathlineto{\pgfqpoint{4.818834in}{3.346255in}}%
\pgfpathlineto{\pgfqpoint{4.805861in}{3.352454in}}%
\pgfpathlineto{\pgfqpoint{4.792895in}{3.358819in}}%
\pgfpathlineto{\pgfqpoint{4.779936in}{3.365348in}}%
\pgfpathlineto{\pgfqpoint{4.766983in}{3.372044in}}%
\pgfpathlineto{\pgfqpoint{4.759643in}{3.352019in}}%
\pgfpathlineto{\pgfqpoint{4.752306in}{3.332349in}}%
\pgfpathlineto{\pgfqpoint{4.744973in}{3.313027in}}%
\pgfpathclose%
\pgfusepath{fill}%
\end{pgfscope}%
\begin{pgfscope}%
\pgfpathrectangle{\pgfqpoint{1.254980in}{0.150000in}}{\pgfqpoint{5.490039in}{5.490039in}}%
\pgfusepath{clip}%
\pgfsetbuttcap%
\pgfsetroundjoin%
\definecolor{currentfill}{rgb}{0.227802,0.326594,0.546532}%
\pgfsetfillcolor{currentfill}%
\pgfsetfillopacity{0.700000}%
\pgfsetlinewidth{0.000000pt}%
\definecolor{currentstroke}{rgb}{0.000000,0.000000,0.000000}%
\pgfsetstrokecolor{currentstroke}%
\pgfsetdash{}{0pt}%
\pgfpathmoveto{\pgfqpoint{3.699044in}{3.035363in}}%
\pgfpathlineto{\pgfqpoint{3.711871in}{3.024028in}}%
\pgfpathlineto{\pgfqpoint{3.724699in}{3.012910in}}%
\pgfpathlineto{\pgfqpoint{3.737527in}{3.002011in}}%
\pgfpathlineto{\pgfqpoint{3.750356in}{2.991326in}}%
\pgfpathlineto{\pgfqpoint{3.757845in}{3.006242in}}%
\pgfpathlineto{\pgfqpoint{3.765330in}{3.021336in}}%
\pgfpathlineto{\pgfqpoint{3.772810in}{3.036612in}}%
\pgfpathlineto{\pgfqpoint{3.780286in}{3.052074in}}%
\pgfpathlineto{\pgfqpoint{3.767463in}{3.063116in}}%
\pgfpathlineto{\pgfqpoint{3.754641in}{3.074374in}}%
\pgfpathlineto{\pgfqpoint{3.741819in}{3.085849in}}%
\pgfpathlineto{\pgfqpoint{3.728997in}{3.097543in}}%
\pgfpathlineto{\pgfqpoint{3.721515in}{3.081712in}}%
\pgfpathlineto{\pgfqpoint{3.714029in}{3.066075in}}%
\pgfpathlineto{\pgfqpoint{3.706538in}{3.050626in}}%
\pgfpathlineto{\pgfqpoint{3.699044in}{3.035363in}}%
\pgfpathclose%
\pgfusepath{fill}%
\end{pgfscope}%
\begin{pgfscope}%
\pgfpathrectangle{\pgfqpoint{1.254980in}{0.150000in}}{\pgfqpoint{5.490039in}{5.490039in}}%
\pgfusepath{clip}%
\pgfsetbuttcap%
\pgfsetroundjoin%
\definecolor{currentfill}{rgb}{0.128087,0.647749,0.523491}%
\pgfsetfillcolor{currentfill}%
\pgfsetfillopacity{0.700000}%
\pgfsetlinewidth{0.000000pt}%
\definecolor{currentstroke}{rgb}{0.000000,0.000000,0.000000}%
\pgfsetstrokecolor{currentstroke}%
\pgfsetdash{}{0pt}%
\pgfpathmoveto{\pgfqpoint{3.244118in}{3.847026in}}%
\pgfpathlineto{\pgfqpoint{3.257093in}{3.824565in}}%
\pgfpathlineto{\pgfqpoint{3.270059in}{3.802412in}}%
\pgfpathlineto{\pgfqpoint{3.283016in}{3.780564in}}%
\pgfpathlineto{\pgfqpoint{3.295966in}{3.759019in}}%
\pgfpathlineto{\pgfqpoint{3.303460in}{3.779598in}}%
\pgfpathlineto{\pgfqpoint{3.310948in}{3.800465in}}%
\pgfpathlineto{\pgfqpoint{3.318430in}{3.821625in}}%
\pgfpathlineto{\pgfqpoint{3.325906in}{3.843084in}}%
\pgfpathlineto{\pgfqpoint{3.312958in}{3.865065in}}%
\pgfpathlineto{\pgfqpoint{3.300003in}{3.887349in}}%
\pgfpathlineto{\pgfqpoint{3.287039in}{3.909939in}}%
\pgfpathlineto{\pgfqpoint{3.274067in}{3.932839in}}%
\pgfpathlineto{\pgfqpoint{3.266588in}{3.910931in}}%
\pgfpathlineto{\pgfqpoint{3.259104in}{3.889330in}}%
\pgfpathlineto{\pgfqpoint{3.251614in}{3.868030in}}%
\pgfpathlineto{\pgfqpoint{3.244118in}{3.847026in}}%
\pgfpathclose%
\pgfusepath{fill}%
\end{pgfscope}%
\begin{pgfscope}%
\pgfpathrectangle{\pgfqpoint{1.254980in}{0.150000in}}{\pgfqpoint{5.490039in}{5.490039in}}%
\pgfusepath{clip}%
\pgfsetbuttcap%
\pgfsetroundjoin%
\definecolor{currentfill}{rgb}{0.143343,0.522773,0.556295}%
\pgfsetfillcolor{currentfill}%
\pgfsetfillopacity{0.700000}%
\pgfsetlinewidth{0.000000pt}%
\definecolor{currentstroke}{rgb}{0.000000,0.000000,0.000000}%
\pgfsetstrokecolor{currentstroke}%
\pgfsetdash{}{0pt}%
\pgfpathmoveto{\pgfqpoint{3.287562in}{3.524060in}}%
\pgfpathlineto{\pgfqpoint{3.300484in}{3.504753in}}%
\pgfpathlineto{\pgfqpoint{3.313400in}{3.485729in}}%
\pgfpathlineto{\pgfqpoint{3.326310in}{3.466987in}}%
\pgfpathlineto{\pgfqpoint{3.339214in}{3.448524in}}%
\pgfpathlineto{\pgfqpoint{3.346746in}{3.466251in}}%
\pgfpathlineto{\pgfqpoint{3.354272in}{3.484214in}}%
\pgfpathlineto{\pgfqpoint{3.361792in}{3.502416in}}%
\pgfpathlineto{\pgfqpoint{3.369307in}{3.520863in}}%
\pgfpathlineto{\pgfqpoint{3.356408in}{3.539695in}}%
\pgfpathlineto{\pgfqpoint{3.343502in}{3.558807in}}%
\pgfpathlineto{\pgfqpoint{3.330591in}{3.578200in}}%
\pgfpathlineto{\pgfqpoint{3.317673in}{3.597878in}}%
\pgfpathlineto{\pgfqpoint{3.310154in}{3.579050in}}%
\pgfpathlineto{\pgfqpoint{3.302630in}{3.560474in}}%
\pgfpathlineto{\pgfqpoint{3.295099in}{3.542145in}}%
\pgfpathlineto{\pgfqpoint{3.287562in}{3.524060in}}%
\pgfpathclose%
\pgfusepath{fill}%
\end{pgfscope}%
\begin{pgfscope}%
\pgfpathrectangle{\pgfqpoint{1.254980in}{0.150000in}}{\pgfqpoint{5.490039in}{5.490039in}}%
\pgfusepath{clip}%
\pgfsetbuttcap%
\pgfsetroundjoin%
\definecolor{currentfill}{rgb}{0.197636,0.391528,0.554969}%
\pgfsetfillcolor{currentfill}%
\pgfsetfillopacity{0.700000}%
\pgfsetlinewidth{0.000000pt}%
\definecolor{currentstroke}{rgb}{0.000000,0.000000,0.000000}%
\pgfsetstrokecolor{currentstroke}%
\pgfsetdash{}{0pt}%
\pgfpathmoveto{\pgfqpoint{3.463570in}{3.183198in}}%
\pgfpathlineto{\pgfqpoint{3.476424in}{3.168472in}}%
\pgfpathlineto{\pgfqpoint{3.489276in}{3.153992in}}%
\pgfpathlineto{\pgfqpoint{3.502125in}{3.139756in}}%
\pgfpathlineto{\pgfqpoint{3.514971in}{3.125763in}}%
\pgfpathlineto{\pgfqpoint{3.522503in}{3.141290in}}%
\pgfpathlineto{\pgfqpoint{3.530029in}{3.157007in}}%
\pgfpathlineto{\pgfqpoint{3.537550in}{3.172918in}}%
\pgfpathlineto{\pgfqpoint{3.545067in}{3.189026in}}%
\pgfpathlineto{\pgfqpoint{3.532226in}{3.203353in}}%
\pgfpathlineto{\pgfqpoint{3.519383in}{3.217922in}}%
\pgfpathlineto{\pgfqpoint{3.506537in}{3.232736in}}%
\pgfpathlineto{\pgfqpoint{3.493688in}{3.247796in}}%
\pgfpathlineto{\pgfqpoint{3.486167in}{3.231343in}}%
\pgfpathlineto{\pgfqpoint{3.478640in}{3.215095in}}%
\pgfpathlineto{\pgfqpoint{3.471108in}{3.199048in}}%
\pgfpathlineto{\pgfqpoint{3.463570in}{3.183198in}}%
\pgfpathclose%
\pgfusepath{fill}%
\end{pgfscope}%
\begin{pgfscope}%
\pgfpathrectangle{\pgfqpoint{1.254980in}{0.150000in}}{\pgfqpoint{5.490039in}{5.490039in}}%
\pgfusepath{clip}%
\pgfsetbuttcap%
\pgfsetroundjoin%
\definecolor{currentfill}{rgb}{0.187231,0.414746,0.556547}%
\pgfsetfillcolor{currentfill}%
\pgfsetfillopacity{0.700000}%
\pgfsetlinewidth{0.000000pt}%
\definecolor{currentstroke}{rgb}{0.000000,0.000000,0.000000}%
\pgfsetstrokecolor{currentstroke}%
\pgfsetdash{}{0pt}%
\pgfpathmoveto{\pgfqpoint{3.412122in}{3.244605in}}%
\pgfpathlineto{\pgfqpoint{3.424989in}{3.228873in}}%
\pgfpathlineto{\pgfqpoint{3.437853in}{3.213396in}}%
\pgfpathlineto{\pgfqpoint{3.450713in}{3.198172in}}%
\pgfpathlineto{\pgfqpoint{3.463570in}{3.183198in}}%
\pgfpathlineto{\pgfqpoint{3.471108in}{3.199048in}}%
\pgfpathlineto{\pgfqpoint{3.478640in}{3.215095in}}%
\pgfpathlineto{\pgfqpoint{3.486167in}{3.231343in}}%
\pgfpathlineto{\pgfqpoint{3.493688in}{3.247796in}}%
\pgfpathlineto{\pgfqpoint{3.480837in}{3.263105in}}%
\pgfpathlineto{\pgfqpoint{3.467982in}{3.278664in}}%
\pgfpathlineto{\pgfqpoint{3.455124in}{3.294476in}}%
\pgfpathlineto{\pgfqpoint{3.442263in}{3.310543in}}%
\pgfpathlineto{\pgfqpoint{3.434735in}{3.293744in}}%
\pgfpathlineto{\pgfqpoint{3.427203in}{3.277157in}}%
\pgfpathlineto{\pgfqpoint{3.419665in}{3.260779in}}%
\pgfpathlineto{\pgfqpoint{3.412122in}{3.244605in}}%
\pgfpathclose%
\pgfusepath{fill}%
\end{pgfscope}%
\begin{pgfscope}%
\pgfpathrectangle{\pgfqpoint{1.254980in}{0.150000in}}{\pgfqpoint{5.490039in}{5.490039in}}%
\pgfusepath{clip}%
\pgfsetbuttcap%
\pgfsetroundjoin%
\definecolor{currentfill}{rgb}{0.233603,0.313828,0.543914}%
\pgfsetfillcolor{currentfill}%
\pgfsetfillopacity{0.700000}%
\pgfsetlinewidth{0.000000pt}%
\definecolor{currentstroke}{rgb}{0.000000,0.000000,0.000000}%
\pgfsetstrokecolor{currentstroke}%
\pgfsetdash{}{0pt}%
\pgfpathmoveto{\pgfqpoint{4.096542in}{2.989571in}}%
\pgfpathlineto{\pgfqpoint{4.109400in}{2.981973in}}%
\pgfpathlineto{\pgfqpoint{4.122261in}{2.974565in}}%
\pgfpathlineto{\pgfqpoint{4.135126in}{2.967345in}}%
\pgfpathlineto{\pgfqpoint{4.147996in}{2.960312in}}%
\pgfpathlineto{\pgfqpoint{4.155404in}{2.975088in}}%
\pgfpathlineto{\pgfqpoint{4.162808in}{2.990055in}}%
\pgfpathlineto{\pgfqpoint{4.170210in}{3.005220in}}%
\pgfpathlineto{\pgfqpoint{4.177610in}{3.020587in}}%
\pgfpathlineto{\pgfqpoint{4.164747in}{3.028057in}}%
\pgfpathlineto{\pgfqpoint{4.151889in}{3.035715in}}%
\pgfpathlineto{\pgfqpoint{4.139034in}{3.043562in}}%
\pgfpathlineto{\pgfqpoint{4.126183in}{3.051598in}}%
\pgfpathlineto{\pgfqpoint{4.118777in}{3.035783in}}%
\pgfpathlineto{\pgfqpoint{4.111368in}{3.020176in}}%
\pgfpathlineto{\pgfqpoint{4.103956in}{3.004774in}}%
\pgfpathlineto{\pgfqpoint{4.096542in}{2.989571in}}%
\pgfpathclose%
\pgfusepath{fill}%
\end{pgfscope}%
\begin{pgfscope}%
\pgfpathrectangle{\pgfqpoint{1.254980in}{0.150000in}}{\pgfqpoint{5.490039in}{5.490039in}}%
\pgfusepath{clip}%
\pgfsetbuttcap%
\pgfsetroundjoin%
\definecolor{currentfill}{rgb}{0.208623,0.367752,0.552675}%
\pgfsetfillcolor{currentfill}%
\pgfsetfillopacity{0.700000}%
\pgfsetlinewidth{0.000000pt}%
\definecolor{currentstroke}{rgb}{0.000000,0.000000,0.000000}%
\pgfsetstrokecolor{currentstroke}%
\pgfsetdash{}{0pt}%
\pgfpathmoveto{\pgfqpoint{3.514971in}{3.125763in}}%
\pgfpathlineto{\pgfqpoint{3.527816in}{3.112010in}}%
\pgfpathlineto{\pgfqpoint{3.540658in}{3.098495in}}%
\pgfpathlineto{\pgfqpoint{3.553499in}{3.085218in}}%
\pgfpathlineto{\pgfqpoint{3.566338in}{3.072175in}}%
\pgfpathlineto{\pgfqpoint{3.573864in}{3.087380in}}%
\pgfpathlineto{\pgfqpoint{3.581384in}{3.102768in}}%
\pgfpathlineto{\pgfqpoint{3.588900in}{3.118342in}}%
\pgfpathlineto{\pgfqpoint{3.596411in}{3.134108in}}%
\pgfpathlineto{\pgfqpoint{3.583577in}{3.147483in}}%
\pgfpathlineto{\pgfqpoint{3.570742in}{3.161093in}}%
\pgfpathlineto{\pgfqpoint{3.557906in}{3.174940in}}%
\pgfpathlineto{\pgfqpoint{3.545067in}{3.189026in}}%
\pgfpathlineto{\pgfqpoint{3.537550in}{3.172918in}}%
\pgfpathlineto{\pgfqpoint{3.530029in}{3.157007in}}%
\pgfpathlineto{\pgfqpoint{3.522503in}{3.141290in}}%
\pgfpathlineto{\pgfqpoint{3.514971in}{3.125763in}}%
\pgfpathclose%
\pgfusepath{fill}%
\end{pgfscope}%
\begin{pgfscope}%
\pgfpathrectangle{\pgfqpoint{1.254980in}{0.150000in}}{\pgfqpoint{5.490039in}{5.490039in}}%
\pgfusepath{clip}%
\pgfsetbuttcap%
\pgfsetroundjoin%
\definecolor{currentfill}{rgb}{0.175841,0.441290,0.557685}%
\pgfsetfillcolor{currentfill}%
\pgfsetfillopacity{0.700000}%
\pgfsetlinewidth{0.000000pt}%
\definecolor{currentstroke}{rgb}{0.000000,0.000000,0.000000}%
\pgfsetstrokecolor{currentstroke}%
\pgfsetdash{}{0pt}%
\pgfpathmoveto{\pgfqpoint{3.360613in}{3.310119in}}%
\pgfpathlineto{\pgfqpoint{3.373497in}{3.293348in}}%
\pgfpathlineto{\pgfqpoint{3.386376in}{3.276840in}}%
\pgfpathlineto{\pgfqpoint{3.399251in}{3.260593in}}%
\pgfpathlineto{\pgfqpoint{3.412122in}{3.244605in}}%
\pgfpathlineto{\pgfqpoint{3.419665in}{3.260779in}}%
\pgfpathlineto{\pgfqpoint{3.427203in}{3.277157in}}%
\pgfpathlineto{\pgfqpoint{3.434735in}{3.293744in}}%
\pgfpathlineto{\pgfqpoint{3.442263in}{3.310543in}}%
\pgfpathlineto{\pgfqpoint{3.429397in}{3.326867in}}%
\pgfpathlineto{\pgfqpoint{3.416528in}{3.343451in}}%
\pgfpathlineto{\pgfqpoint{3.403654in}{3.360295in}}%
\pgfpathlineto{\pgfqpoint{3.390776in}{3.377404in}}%
\pgfpathlineto{\pgfqpoint{3.383243in}{3.360257in}}%
\pgfpathlineto{\pgfqpoint{3.375706in}{3.343330in}}%
\pgfpathlineto{\pgfqpoint{3.368162in}{3.326619in}}%
\pgfpathlineto{\pgfqpoint{3.360613in}{3.310119in}}%
\pgfpathclose%
\pgfusepath{fill}%
\end{pgfscope}%
\begin{pgfscope}%
\pgfpathrectangle{\pgfqpoint{1.254980in}{0.150000in}}{\pgfqpoint{5.490039in}{5.490039in}}%
\pgfusepath{clip}%
\pgfsetbuttcap%
\pgfsetroundjoin%
\definecolor{currentfill}{rgb}{0.239346,0.300855,0.540844}%
\pgfsetfillcolor{currentfill}%
\pgfsetfillopacity{0.700000}%
\pgfsetlinewidth{0.000000pt}%
\definecolor{currentstroke}{rgb}{0.000000,0.000000,0.000000}%
\pgfsetstrokecolor{currentstroke}%
\pgfsetdash{}{0pt}%
\pgfpathmoveto{\pgfqpoint{3.882910in}{2.971342in}}%
\pgfpathlineto{\pgfqpoint{3.895745in}{2.962181in}}%
\pgfpathlineto{\pgfqpoint{3.908583in}{2.953223in}}%
\pgfpathlineto{\pgfqpoint{3.921423in}{2.944465in}}%
\pgfpathlineto{\pgfqpoint{3.934265in}{2.935909in}}%
\pgfpathlineto{\pgfqpoint{3.941719in}{2.950440in}}%
\pgfpathlineto{\pgfqpoint{3.949170in}{2.965146in}}%
\pgfpathlineto{\pgfqpoint{3.956617in}{2.980031in}}%
\pgfpathlineto{\pgfqpoint{3.964060in}{2.995100in}}%
\pgfpathlineto{\pgfqpoint{3.951224in}{3.004041in}}%
\pgfpathlineto{\pgfqpoint{3.938390in}{3.013182in}}%
\pgfpathlineto{\pgfqpoint{3.925559in}{3.022524in}}%
\pgfpathlineto{\pgfqpoint{3.912730in}{3.032069in}}%
\pgfpathlineto{\pgfqpoint{3.905280in}{3.016606in}}%
\pgfpathlineto{\pgfqpoint{3.897827in}{3.001333in}}%
\pgfpathlineto{\pgfqpoint{3.890370in}{2.986247in}}%
\pgfpathlineto{\pgfqpoint{3.882910in}{2.971342in}}%
\pgfpathclose%
\pgfusepath{fill}%
\end{pgfscope}%
\begin{pgfscope}%
\pgfpathrectangle{\pgfqpoint{1.254980in}{0.150000in}}{\pgfqpoint{5.490039in}{5.490039in}}%
\pgfusepath{clip}%
\pgfsetbuttcap%
\pgfsetroundjoin%
\definecolor{currentfill}{rgb}{0.709898,0.868751,0.169257}%
\pgfsetfillcolor{currentfill}%
\pgfsetfillopacity{0.700000}%
\pgfsetlinewidth{0.000000pt}%
\definecolor{currentstroke}{rgb}{0.000000,0.000000,0.000000}%
\pgfsetstrokecolor{currentstroke}%
\pgfsetdash{}{0pt}%
\pgfpathmoveto{\pgfqpoint{3.451947in}{4.565874in}}%
\pgfpathlineto{\pgfqpoint{3.464933in}{4.539652in}}%
\pgfpathlineto{\pgfqpoint{3.477910in}{4.513746in}}%
\pgfpathlineto{\pgfqpoint{3.490878in}{4.488154in}}%
\pgfpathlineto{\pgfqpoint{3.503838in}{4.462872in}}%
\pgfpathlineto{\pgfqpoint{3.511198in}{4.493770in}}%
\pgfpathlineto{\pgfqpoint{3.518555in}{4.525137in}}%
\pgfpathlineto{\pgfqpoint{3.525909in}{4.556981in}}%
\pgfpathlineto{\pgfqpoint{3.533259in}{4.589310in}}%
\pgfpathlineto{\pgfqpoint{3.520293in}{4.615263in}}%
\pgfpathlineto{\pgfqpoint{3.507318in}{4.641529in}}%
\pgfpathlineto{\pgfqpoint{3.494334in}{4.668110in}}%
\pgfpathlineto{\pgfqpoint{3.481341in}{4.695010in}}%
\pgfpathlineto{\pgfqpoint{3.473998in}{4.661993in}}%
\pgfpathlineto{\pgfqpoint{3.466651in}{4.629470in}}%
\pgfpathlineto{\pgfqpoint{3.459301in}{4.597433in}}%
\pgfpathlineto{\pgfqpoint{3.451947in}{4.565874in}}%
\pgfpathclose%
\pgfusepath{fill}%
\end{pgfscope}%
\begin{pgfscope}%
\pgfpathrectangle{\pgfqpoint{1.254980in}{0.150000in}}{\pgfqpoint{5.490039in}{5.490039in}}%
\pgfusepath{clip}%
\pgfsetbuttcap%
\pgfsetroundjoin%
\definecolor{currentfill}{rgb}{0.210503,0.363727,0.552206}%
\pgfsetfillcolor{currentfill}%
\pgfsetfillopacity{0.700000}%
\pgfsetlinewidth{0.000000pt}%
\definecolor{currentstroke}{rgb}{0.000000,0.000000,0.000000}%
\pgfsetstrokecolor{currentstroke}%
\pgfsetdash{}{0pt}%
\pgfpathmoveto{\pgfqpoint{4.472328in}{3.104448in}}%
\pgfpathlineto{\pgfqpoint{4.485253in}{3.098563in}}%
\pgfpathlineto{\pgfqpoint{4.498183in}{3.092851in}}%
\pgfpathlineto{\pgfqpoint{4.511119in}{3.087312in}}%
\pgfpathlineto{\pgfqpoint{4.524062in}{3.081944in}}%
\pgfpathlineto{\pgfqpoint{4.531400in}{3.097766in}}%
\pgfpathlineto{\pgfqpoint{4.538737in}{3.113837in}}%
\pgfpathlineto{\pgfqpoint{4.546074in}{3.130163in}}%
\pgfpathlineto{\pgfqpoint{4.553411in}{3.146751in}}%
\pgfpathlineto{\pgfqpoint{4.540478in}{3.152666in}}%
\pgfpathlineto{\pgfqpoint{4.527551in}{3.158753in}}%
\pgfpathlineto{\pgfqpoint{4.514630in}{3.165012in}}%
\pgfpathlineto{\pgfqpoint{4.501714in}{3.171445in}}%
\pgfpathlineto{\pgfqpoint{4.494368in}{3.154299in}}%
\pgfpathlineto{\pgfqpoint{4.487022in}{3.137421in}}%
\pgfpathlineto{\pgfqpoint{4.479675in}{3.120807in}}%
\pgfpathlineto{\pgfqpoint{4.472328in}{3.104448in}}%
\pgfpathclose%
\pgfusepath{fill}%
\end{pgfscope}%
\begin{pgfscope}%
\pgfpathrectangle{\pgfqpoint{1.254980in}{0.150000in}}{\pgfqpoint{5.490039in}{5.490039in}}%
\pgfusepath{clip}%
\pgfsetbuttcap%
\pgfsetroundjoin%
\definecolor{currentfill}{rgb}{0.201239,0.383670,0.554294}%
\pgfsetfillcolor{currentfill}%
\pgfsetfillopacity{0.700000}%
\pgfsetlinewidth{0.000000pt}%
\definecolor{currentstroke}{rgb}{0.000000,0.000000,0.000000}%
\pgfsetstrokecolor{currentstroke}%
\pgfsetdash{}{0pt}%
\pgfpathmoveto{\pgfqpoint{4.553411in}{3.146751in}}%
\pgfpathlineto{\pgfqpoint{4.566351in}{3.141008in}}%
\pgfpathlineto{\pgfqpoint{4.579296in}{3.135435in}}%
\pgfpathlineto{\pgfqpoint{4.592248in}{3.130032in}}%
\pgfpathlineto{\pgfqpoint{4.605207in}{3.124799in}}%
\pgfpathlineto{\pgfqpoint{4.612534in}{3.141091in}}%
\pgfpathlineto{\pgfqpoint{4.619862in}{3.157651in}}%
\pgfpathlineto{\pgfqpoint{4.627190in}{3.174485in}}%
\pgfpathlineto{\pgfqpoint{4.634518in}{3.191602in}}%
\pgfpathlineto{\pgfqpoint{4.621570in}{3.197410in}}%
\pgfpathlineto{\pgfqpoint{4.608627in}{3.203388in}}%
\pgfpathlineto{\pgfqpoint{4.595691in}{3.209536in}}%
\pgfpathlineto{\pgfqpoint{4.582761in}{3.215854in}}%
\pgfpathlineto{\pgfqpoint{4.575423in}{3.198152in}}%
\pgfpathlineto{\pgfqpoint{4.568086in}{3.180739in}}%
\pgfpathlineto{\pgfqpoint{4.560748in}{3.163608in}}%
\pgfpathlineto{\pgfqpoint{4.553411in}{3.146751in}}%
\pgfpathclose%
\pgfusepath{fill}%
\end{pgfscope}%
\begin{pgfscope}%
\pgfpathrectangle{\pgfqpoint{1.254980in}{0.150000in}}{\pgfqpoint{5.490039in}{5.490039in}}%
\pgfusepath{clip}%
\pgfsetbuttcap%
\pgfsetroundjoin%
\definecolor{currentfill}{rgb}{0.235526,0.309527,0.542944}%
\pgfsetfillcolor{currentfill}%
\pgfsetfillopacity{0.700000}%
\pgfsetlinewidth{0.000000pt}%
\definecolor{currentstroke}{rgb}{0.000000,0.000000,0.000000}%
\pgfsetstrokecolor{currentstroke}%
\pgfsetdash{}{0pt}%
\pgfpathmoveto{\pgfqpoint{3.750356in}{2.991326in}}%
\pgfpathlineto{\pgfqpoint{3.763186in}{2.980856in}}%
\pgfpathlineto{\pgfqpoint{3.776016in}{2.970599in}}%
\pgfpathlineto{\pgfqpoint{3.788848in}{2.960553in}}%
\pgfpathlineto{\pgfqpoint{3.801681in}{2.950718in}}%
\pgfpathlineto{\pgfqpoint{3.809164in}{2.965286in}}%
\pgfpathlineto{\pgfqpoint{3.816642in}{2.980026in}}%
\pgfpathlineto{\pgfqpoint{3.824117in}{2.994942in}}%
\pgfpathlineto{\pgfqpoint{3.831587in}{3.010037in}}%
\pgfpathlineto{\pgfqpoint{3.818760in}{3.020229in}}%
\pgfpathlineto{\pgfqpoint{3.805935in}{3.030632in}}%
\pgfpathlineto{\pgfqpoint{3.793110in}{3.041247in}}%
\pgfpathlineto{\pgfqpoint{3.780286in}{3.052074in}}%
\pgfpathlineto{\pgfqpoint{3.772810in}{3.036612in}}%
\pgfpathlineto{\pgfqpoint{3.765330in}{3.021336in}}%
\pgfpathlineto{\pgfqpoint{3.757845in}{3.006242in}}%
\pgfpathlineto{\pgfqpoint{3.750356in}{2.991326in}}%
\pgfpathclose%
\pgfusepath{fill}%
\end{pgfscope}%
\begin{pgfscope}%
\pgfpathrectangle{\pgfqpoint{1.254980in}{0.150000in}}{\pgfqpoint{5.490039in}{5.490039in}}%
\pgfusepath{clip}%
\pgfsetbuttcap%
\pgfsetroundjoin%
\definecolor{currentfill}{rgb}{0.218130,0.347432,0.550038}%
\pgfsetfillcolor{currentfill}%
\pgfsetfillopacity{0.700000}%
\pgfsetlinewidth{0.000000pt}%
\definecolor{currentstroke}{rgb}{0.000000,0.000000,0.000000}%
\pgfsetstrokecolor{currentstroke}%
\pgfsetdash{}{0pt}%
\pgfpathmoveto{\pgfqpoint{4.391257in}{3.064655in}}%
\pgfpathlineto{\pgfqpoint{4.404167in}{3.058589in}}%
\pgfpathlineto{\pgfqpoint{4.417083in}{3.052700in}}%
\pgfpathlineto{\pgfqpoint{4.430004in}{3.046986in}}%
\pgfpathlineto{\pgfqpoint{4.442931in}{3.041447in}}%
\pgfpathlineto{\pgfqpoint{4.450282in}{3.056844in}}%
\pgfpathlineto{\pgfqpoint{4.457632in}{3.072473in}}%
\pgfpathlineto{\pgfqpoint{4.464981in}{3.088338in}}%
\pgfpathlineto{\pgfqpoint{4.472328in}{3.104448in}}%
\pgfpathlineto{\pgfqpoint{4.459410in}{3.110507in}}%
\pgfpathlineto{\pgfqpoint{4.446497in}{3.116741in}}%
\pgfpathlineto{\pgfqpoint{4.433590in}{3.123150in}}%
\pgfpathlineto{\pgfqpoint{4.420688in}{3.129736in}}%
\pgfpathlineto{\pgfqpoint{4.413332in}{3.113096in}}%
\pgfpathlineto{\pgfqpoint{4.405975in}{3.096707in}}%
\pgfpathlineto{\pgfqpoint{4.398617in}{3.080562in}}%
\pgfpathlineto{\pgfqpoint{4.391257in}{3.064655in}}%
\pgfpathclose%
\pgfusepath{fill}%
\end{pgfscope}%
\begin{pgfscope}%
\pgfpathrectangle{\pgfqpoint{1.254980in}{0.150000in}}{\pgfqpoint{5.490039in}{5.490039in}}%
\pgfusepath{clip}%
\pgfsetbuttcap%
\pgfsetroundjoin%
\definecolor{currentfill}{rgb}{0.220057,0.343307,0.549413}%
\pgfsetfillcolor{currentfill}%
\pgfsetfillopacity{0.700000}%
\pgfsetlinewidth{0.000000pt}%
\definecolor{currentstroke}{rgb}{0.000000,0.000000,0.000000}%
\pgfsetstrokecolor{currentstroke}%
\pgfsetdash{}{0pt}%
\pgfpathmoveto{\pgfqpoint{3.566338in}{3.072175in}}%
\pgfpathlineto{\pgfqpoint{3.579176in}{3.059365in}}%
\pgfpathlineto{\pgfqpoint{3.592013in}{3.046786in}}%
\pgfpathlineto{\pgfqpoint{3.604849in}{3.034437in}}%
\pgfpathlineto{\pgfqpoint{3.617684in}{3.022317in}}%
\pgfpathlineto{\pgfqpoint{3.625203in}{3.037201in}}%
\pgfpathlineto{\pgfqpoint{3.632717in}{3.052260in}}%
\pgfpathlineto{\pgfqpoint{3.640227in}{3.067500in}}%
\pgfpathlineto{\pgfqpoint{3.647732in}{3.082923in}}%
\pgfpathlineto{\pgfqpoint{3.634903in}{3.095376in}}%
\pgfpathlineto{\pgfqpoint{3.622074in}{3.108056in}}%
\pgfpathlineto{\pgfqpoint{3.609243in}{3.120966in}}%
\pgfpathlineto{\pgfqpoint{3.596411in}{3.134108in}}%
\pgfpathlineto{\pgfqpoint{3.588900in}{3.118342in}}%
\pgfpathlineto{\pgfqpoint{3.581384in}{3.102768in}}%
\pgfpathlineto{\pgfqpoint{3.573864in}{3.087380in}}%
\pgfpathlineto{\pgfqpoint{3.566338in}{3.072175in}}%
\pgfpathclose%
\pgfusepath{fill}%
\end{pgfscope}%
\begin{pgfscope}%
\pgfpathrectangle{\pgfqpoint{1.254980in}{0.150000in}}{\pgfqpoint{5.490039in}{5.490039in}}%
\pgfusepath{clip}%
\pgfsetbuttcap%
\pgfsetroundjoin%
\definecolor{currentfill}{rgb}{0.192357,0.403199,0.555836}%
\pgfsetfillcolor{currentfill}%
\pgfsetfillopacity{0.700000}%
\pgfsetlinewidth{0.000000pt}%
\definecolor{currentstroke}{rgb}{0.000000,0.000000,0.000000}%
\pgfsetstrokecolor{currentstroke}%
\pgfsetdash{}{0pt}%
\pgfpathmoveto{\pgfqpoint{4.634518in}{3.191602in}}%
\pgfpathlineto{\pgfqpoint{4.647474in}{3.185962in}}%
\pgfpathlineto{\pgfqpoint{4.660435in}{3.180491in}}%
\pgfpathlineto{\pgfqpoint{4.673404in}{3.175187in}}%
\pgfpathlineto{\pgfqpoint{4.686379in}{3.170050in}}%
\pgfpathlineto{\pgfqpoint{4.693698in}{3.186862in}}%
\pgfpathlineto{\pgfqpoint{4.701018in}{3.203963in}}%
\pgfpathlineto{\pgfqpoint{4.708339in}{3.221360in}}%
\pgfpathlineto{\pgfqpoint{4.715662in}{3.239059in}}%
\pgfpathlineto{\pgfqpoint{4.702698in}{3.244798in}}%
\pgfpathlineto{\pgfqpoint{4.689740in}{3.250705in}}%
\pgfpathlineto{\pgfqpoint{4.676788in}{3.256779in}}%
\pgfpathlineto{\pgfqpoint{4.663843in}{3.263022in}}%
\pgfpathlineto{\pgfqpoint{4.656510in}{3.244709in}}%
\pgfpathlineto{\pgfqpoint{4.649178in}{3.226707in}}%
\pgfpathlineto{\pgfqpoint{4.641848in}{3.209006in}}%
\pgfpathlineto{\pgfqpoint{4.634518in}{3.191602in}}%
\pgfpathclose%
\pgfusepath{fill}%
\end{pgfscope}%
\begin{pgfscope}%
\pgfpathrectangle{\pgfqpoint{1.254980in}{0.150000in}}{\pgfqpoint{5.490039in}{5.490039in}}%
\pgfusepath{clip}%
\pgfsetbuttcap%
\pgfsetroundjoin%
\definecolor{currentfill}{rgb}{0.165117,0.467423,0.558141}%
\pgfsetfillcolor{currentfill}%
\pgfsetfillopacity{0.700000}%
\pgfsetlinewidth{0.000000pt}%
\definecolor{currentstroke}{rgb}{0.000000,0.000000,0.000000}%
\pgfsetstrokecolor{currentstroke}%
\pgfsetdash{}{0pt}%
\pgfpathmoveto{\pgfqpoint{3.309030in}{3.379886in}}%
\pgfpathlineto{\pgfqpoint{3.321933in}{3.362037in}}%
\pgfpathlineto{\pgfqpoint{3.334832in}{3.344461in}}%
\pgfpathlineto{\pgfqpoint{3.347725in}{3.327156in}}%
\pgfpathlineto{\pgfqpoint{3.360613in}{3.310119in}}%
\pgfpathlineto{\pgfqpoint{3.368162in}{3.326619in}}%
\pgfpathlineto{\pgfqpoint{3.375706in}{3.343330in}}%
\pgfpathlineto{\pgfqpoint{3.383243in}{3.360257in}}%
\pgfpathlineto{\pgfqpoint{3.390776in}{3.377404in}}%
\pgfpathlineto{\pgfqpoint{3.377893in}{3.394779in}}%
\pgfpathlineto{\pgfqpoint{3.365005in}{3.412422in}}%
\pgfpathlineto{\pgfqpoint{3.352112in}{3.430336in}}%
\pgfpathlineto{\pgfqpoint{3.339214in}{3.448524in}}%
\pgfpathlineto{\pgfqpoint{3.331677in}{3.431028in}}%
\pgfpathlineto{\pgfqpoint{3.324133in}{3.413759in}}%
\pgfpathlineto{\pgfqpoint{3.316584in}{3.396713in}}%
\pgfpathlineto{\pgfqpoint{3.309030in}{3.379886in}}%
\pgfpathclose%
\pgfusepath{fill}%
\end{pgfscope}%
\begin{pgfscope}%
\pgfpathrectangle{\pgfqpoint{1.254980in}{0.150000in}}{\pgfqpoint{5.490039in}{5.490039in}}%
\pgfusepath{clip}%
\pgfsetbuttcap%
\pgfsetroundjoin%
\definecolor{currentfill}{rgb}{0.119699,0.618490,0.536347}%
\pgfsetfillcolor{currentfill}%
\pgfsetfillopacity{0.700000}%
\pgfsetlinewidth{0.000000pt}%
\definecolor{currentstroke}{rgb}{0.000000,0.000000,0.000000}%
\pgfsetstrokecolor{currentstroke}%
\pgfsetdash{}{0pt}%
\pgfpathmoveto{\pgfqpoint{3.214073in}{3.765871in}}%
\pgfpathlineto{\pgfqpoint{3.227050in}{3.743817in}}%
\pgfpathlineto{\pgfqpoint{3.240020in}{3.722070in}}%
\pgfpathlineto{\pgfqpoint{3.252981in}{3.700628in}}%
\pgfpathlineto{\pgfqpoint{3.265934in}{3.679487in}}%
\pgfpathlineto{\pgfqpoint{3.273451in}{3.698962in}}%
\pgfpathlineto{\pgfqpoint{3.280962in}{3.718706in}}%
\pgfpathlineto{\pgfqpoint{3.288467in}{3.738723in}}%
\pgfpathlineto{\pgfqpoint{3.295966in}{3.759019in}}%
\pgfpathlineto{\pgfqpoint{3.283016in}{3.780564in}}%
\pgfpathlineto{\pgfqpoint{3.270059in}{3.802412in}}%
\pgfpathlineto{\pgfqpoint{3.257093in}{3.824565in}}%
\pgfpathlineto{\pgfqpoint{3.244118in}{3.847026in}}%
\pgfpathlineto{\pgfqpoint{3.236616in}{3.826313in}}%
\pgfpathlineto{\pgfqpoint{3.229108in}{3.805886in}}%
\pgfpathlineto{\pgfqpoint{3.221593in}{3.785740in}}%
\pgfpathlineto{\pgfqpoint{3.214073in}{3.765871in}}%
\pgfpathclose%
\pgfusepath{fill}%
\end{pgfscope}%
\begin{pgfscope}%
\pgfpathrectangle{\pgfqpoint{1.254980in}{0.150000in}}{\pgfqpoint{5.490039in}{5.490039in}}%
\pgfusepath{clip}%
\pgfsetbuttcap%
\pgfsetroundjoin%
\definecolor{currentfill}{rgb}{0.225863,0.330805,0.547314}%
\pgfsetfillcolor{currentfill}%
\pgfsetfillopacity{0.700000}%
\pgfsetlinewidth{0.000000pt}%
\definecolor{currentstroke}{rgb}{0.000000,0.000000,0.000000}%
\pgfsetstrokecolor{currentstroke}%
\pgfsetdash{}{0pt}%
\pgfpathmoveto{\pgfqpoint{4.310186in}{3.027358in}}%
\pgfpathlineto{\pgfqpoint{4.323083in}{3.021073in}}%
\pgfpathlineto{\pgfqpoint{4.335985in}{3.014967in}}%
\pgfpathlineto{\pgfqpoint{4.348892in}{3.009039in}}%
\pgfpathlineto{\pgfqpoint{4.361805in}{3.003289in}}%
\pgfpathlineto{\pgfqpoint{4.369170in}{3.018303in}}%
\pgfpathlineto{\pgfqpoint{4.376534in}{3.033531in}}%
\pgfpathlineto{\pgfqpoint{4.383896in}{3.048980in}}%
\pgfpathlineto{\pgfqpoint{4.391257in}{3.064655in}}%
\pgfpathlineto{\pgfqpoint{4.378353in}{3.070897in}}%
\pgfpathlineto{\pgfqpoint{4.365453in}{3.077317in}}%
\pgfpathlineto{\pgfqpoint{4.352559in}{3.083915in}}%
\pgfpathlineto{\pgfqpoint{4.339670in}{3.090693in}}%
\pgfpathlineto{\pgfqpoint{4.332302in}{3.074515in}}%
\pgfpathlineto{\pgfqpoint{4.324932in}{3.058571in}}%
\pgfpathlineto{\pgfqpoint{4.317560in}{3.042853in}}%
\pgfpathlineto{\pgfqpoint{4.310186in}{3.027358in}}%
\pgfpathclose%
\pgfusepath{fill}%
\end{pgfscope}%
\begin{pgfscope}%
\pgfpathrectangle{\pgfqpoint{1.254980in}{0.150000in}}{\pgfqpoint{5.490039in}{5.490039in}}%
\pgfusepath{clip}%
\pgfsetbuttcap%
\pgfsetroundjoin%
\definecolor{currentfill}{rgb}{0.131172,0.555899,0.552459}%
\pgfsetfillcolor{currentfill}%
\pgfsetfillopacity{0.700000}%
\pgfsetlinewidth{0.000000pt}%
\definecolor{currentstroke}{rgb}{0.000000,0.000000,0.000000}%
\pgfsetstrokecolor{currentstroke}%
\pgfsetdash{}{0pt}%
\pgfpathmoveto{\pgfqpoint{3.235805in}{3.604179in}}%
\pgfpathlineto{\pgfqpoint{3.248755in}{3.583710in}}%
\pgfpathlineto{\pgfqpoint{3.261698in}{3.563535in}}%
\pgfpathlineto{\pgfqpoint{3.274634in}{3.543653in}}%
\pgfpathlineto{\pgfqpoint{3.287562in}{3.524060in}}%
\pgfpathlineto{\pgfqpoint{3.295099in}{3.542145in}}%
\pgfpathlineto{\pgfqpoint{3.302630in}{3.560474in}}%
\pgfpathlineto{\pgfqpoint{3.310154in}{3.579050in}}%
\pgfpathlineto{\pgfqpoint{3.317673in}{3.597878in}}%
\pgfpathlineto{\pgfqpoint{3.304749in}{3.617842in}}%
\pgfpathlineto{\pgfqpoint{3.291818in}{3.638097in}}%
\pgfpathlineto{\pgfqpoint{3.278880in}{3.658644in}}%
\pgfpathlineto{\pgfqpoint{3.265934in}{3.679487in}}%
\pgfpathlineto{\pgfqpoint{3.258411in}{3.660275in}}%
\pgfpathlineto{\pgfqpoint{3.250882in}{3.641323in}}%
\pgfpathlineto{\pgfqpoint{3.243347in}{3.622626in}}%
\pgfpathlineto{\pgfqpoint{3.235805in}{3.604179in}}%
\pgfpathclose%
\pgfusepath{fill}%
\end{pgfscope}%
\begin{pgfscope}%
\pgfpathrectangle{\pgfqpoint{1.254980in}{0.150000in}}{\pgfqpoint{5.490039in}{5.490039in}}%
\pgfusepath{clip}%
\pgfsetbuttcap%
\pgfsetroundjoin%
\definecolor{currentfill}{rgb}{0.239346,0.300855,0.540844}%
\pgfsetfillcolor{currentfill}%
\pgfsetfillopacity{0.700000}%
\pgfsetlinewidth{0.000000pt}%
\definecolor{currentstroke}{rgb}{0.000000,0.000000,0.000000}%
\pgfsetstrokecolor{currentstroke}%
\pgfsetdash{}{0pt}%
\pgfpathmoveto{\pgfqpoint{4.015431in}{2.961318in}}%
\pgfpathlineto{\pgfqpoint{4.028282in}{2.953362in}}%
\pgfpathlineto{\pgfqpoint{4.041135in}{2.945599in}}%
\pgfpathlineto{\pgfqpoint{4.053993in}{2.938029in}}%
\pgfpathlineto{\pgfqpoint{4.066853in}{2.930651in}}%
\pgfpathlineto{\pgfqpoint{4.074280in}{2.945107in}}%
\pgfpathlineto{\pgfqpoint{4.081704in}{2.959743in}}%
\pgfpathlineto{\pgfqpoint{4.089124in}{2.974563in}}%
\pgfpathlineto{\pgfqpoint{4.096542in}{2.989571in}}%
\pgfpathlineto{\pgfqpoint{4.083688in}{2.997360in}}%
\pgfpathlineto{\pgfqpoint{4.070837in}{3.005341in}}%
\pgfpathlineto{\pgfqpoint{4.057990in}{3.013514in}}%
\pgfpathlineto{\pgfqpoint{4.045146in}{3.021881in}}%
\pgfpathlineto{\pgfqpoint{4.037722in}{3.006451in}}%
\pgfpathlineto{\pgfqpoint{4.030295in}{2.991217in}}%
\pgfpathlineto{\pgfqpoint{4.022865in}{2.976174in}}%
\pgfpathlineto{\pgfqpoint{4.015431in}{2.961318in}}%
\pgfpathclose%
\pgfusepath{fill}%
\end{pgfscope}%
\begin{pgfscope}%
\pgfpathrectangle{\pgfqpoint{1.254980in}{0.150000in}}{\pgfqpoint{5.490039in}{5.490039in}}%
\pgfusepath{clip}%
\pgfsetbuttcap%
\pgfsetroundjoin%
\definecolor{currentfill}{rgb}{0.183898,0.422383,0.556944}%
\pgfsetfillcolor{currentfill}%
\pgfsetfillopacity{0.700000}%
\pgfsetlinewidth{0.000000pt}%
\definecolor{currentstroke}{rgb}{0.000000,0.000000,0.000000}%
\pgfsetstrokecolor{currentstroke}%
\pgfsetdash{}{0pt}%
\pgfpathmoveto{\pgfqpoint{4.715662in}{3.239059in}}%
\pgfpathlineto{\pgfqpoint{4.728634in}{3.233485in}}%
\pgfpathlineto{\pgfqpoint{4.741612in}{3.228078in}}%
\pgfpathlineto{\pgfqpoint{4.754597in}{3.222836in}}%
\pgfpathlineto{\pgfqpoint{4.767589in}{3.217759in}}%
\pgfpathlineto{\pgfqpoint{4.774903in}{3.235147in}}%
\pgfpathlineto{\pgfqpoint{4.782218in}{3.252846in}}%
\pgfpathlineto{\pgfqpoint{4.789536in}{3.270863in}}%
\pgfpathlineto{\pgfqpoint{4.796856in}{3.289204in}}%
\pgfpathlineto{\pgfqpoint{4.783875in}{3.294912in}}%
\pgfpathlineto{\pgfqpoint{4.770901in}{3.300785in}}%
\pgfpathlineto{\pgfqpoint{4.757934in}{3.306823in}}%
\pgfpathlineto{\pgfqpoint{4.744973in}{3.313027in}}%
\pgfpathlineto{\pgfqpoint{4.737642in}{3.294044in}}%
\pgfpathlineto{\pgfqpoint{4.730313in}{3.275393in}}%
\pgfpathlineto{\pgfqpoint{4.722987in}{3.257067in}}%
\pgfpathlineto{\pgfqpoint{4.715662in}{3.239059in}}%
\pgfpathclose%
\pgfusepath{fill}%
\end{pgfscope}%
\begin{pgfscope}%
\pgfpathrectangle{\pgfqpoint{1.254980in}{0.150000in}}{\pgfqpoint{5.490039in}{5.490039in}}%
\pgfusepath{clip}%
\pgfsetbuttcap%
\pgfsetroundjoin%
\definecolor{currentfill}{rgb}{0.876168,0.891125,0.095250}%
\pgfsetfillcolor{currentfill}%
\pgfsetfillopacity{0.700000}%
\pgfsetlinewidth{0.000000pt}%
\definecolor{currentstroke}{rgb}{0.000000,0.000000,0.000000}%
\pgfsetstrokecolor{currentstroke}%
\pgfsetdash{}{0pt}%
\pgfpathmoveto{\pgfqpoint{3.562632in}{4.723643in}}%
\pgfpathlineto{\pgfqpoint{3.575598in}{4.697291in}}%
\pgfpathlineto{\pgfqpoint{3.588555in}{4.671248in}}%
\pgfpathlineto{\pgfqpoint{3.601505in}{4.645509in}}%
\pgfpathlineto{\pgfqpoint{3.614446in}{4.620071in}}%
\pgfpathlineto{\pgfqpoint{3.621791in}{4.654224in}}%
\pgfpathlineto{\pgfqpoint{3.629135in}{4.688903in}}%
\pgfpathlineto{\pgfqpoint{3.636476in}{4.724119in}}%
\pgfpathlineto{\pgfqpoint{3.623528in}{4.750107in}}%
\pgfpathlineto{\pgfqpoint{3.610572in}{4.776399in}}%
\pgfpathlineto{\pgfqpoint{3.597608in}{4.802997in}}%
\pgfpathlineto{\pgfqpoint{3.584635in}{4.829905in}}%
\pgfpathlineto{\pgfqpoint{3.577303in}{4.793942in}}%
\pgfpathlineto{\pgfqpoint{3.569969in}{4.758524in}}%
\pgfpathlineto{\pgfqpoint{3.562632in}{4.723643in}}%
\pgfpathclose%
\pgfusepath{fill}%
\end{pgfscope}%
\begin{pgfscope}%
\pgfpathrectangle{\pgfqpoint{1.254980in}{0.150000in}}{\pgfqpoint{5.490039in}{5.490039in}}%
\pgfusepath{clip}%
\pgfsetbuttcap%
\pgfsetroundjoin%
\definecolor{currentfill}{rgb}{0.227802,0.326594,0.546532}%
\pgfsetfillcolor{currentfill}%
\pgfsetfillopacity{0.700000}%
\pgfsetlinewidth{0.000000pt}%
\definecolor{currentstroke}{rgb}{0.000000,0.000000,0.000000}%
\pgfsetstrokecolor{currentstroke}%
\pgfsetdash{}{0pt}%
\pgfpathmoveto{\pgfqpoint{3.617684in}{3.022317in}}%
\pgfpathlineto{\pgfqpoint{3.630518in}{3.010422in}}%
\pgfpathlineto{\pgfqpoint{3.643352in}{2.998752in}}%
\pgfpathlineto{\pgfqpoint{3.656186in}{2.987305in}}%
\pgfpathlineto{\pgfqpoint{3.669019in}{2.976079in}}%
\pgfpathlineto{\pgfqpoint{3.676532in}{2.990643in}}%
\pgfpathlineto{\pgfqpoint{3.684041in}{3.005376in}}%
\pgfpathlineto{\pgfqpoint{3.691544in}{3.020281in}}%
\pgfpathlineto{\pgfqpoint{3.699044in}{3.035363in}}%
\pgfpathlineto{\pgfqpoint{3.686216in}{3.046919in}}%
\pgfpathlineto{\pgfqpoint{3.673389in}{3.058697in}}%
\pgfpathlineto{\pgfqpoint{3.660561in}{3.070698in}}%
\pgfpathlineto{\pgfqpoint{3.647732in}{3.082923in}}%
\pgfpathlineto{\pgfqpoint{3.640227in}{3.067500in}}%
\pgfpathlineto{\pgfqpoint{3.632717in}{3.052260in}}%
\pgfpathlineto{\pgfqpoint{3.625203in}{3.037201in}}%
\pgfpathlineto{\pgfqpoint{3.617684in}{3.022317in}}%
\pgfpathclose%
\pgfusepath{fill}%
\end{pgfscope}%
\begin{pgfscope}%
\pgfpathrectangle{\pgfqpoint{1.254980in}{0.150000in}}{\pgfqpoint{5.490039in}{5.490039in}}%
\pgfusepath{clip}%
\pgfsetbuttcap%
\pgfsetroundjoin%
\definecolor{currentfill}{rgb}{0.231674,0.318106,0.544834}%
\pgfsetfillcolor{currentfill}%
\pgfsetfillopacity{0.700000}%
\pgfsetlinewidth{0.000000pt}%
\definecolor{currentstroke}{rgb}{0.000000,0.000000,0.000000}%
\pgfsetstrokecolor{currentstroke}%
\pgfsetdash{}{0pt}%
\pgfpathmoveto{\pgfqpoint{4.229103in}{2.992566in}}%
\pgfpathlineto{\pgfqpoint{4.241987in}{2.986022in}}%
\pgfpathlineto{\pgfqpoint{4.254877in}{2.979659in}}%
\pgfpathlineto{\pgfqpoint{4.267771in}{2.973478in}}%
\pgfpathlineto{\pgfqpoint{4.280671in}{2.967478in}}%
\pgfpathlineto{\pgfqpoint{4.288053in}{2.982144in}}%
\pgfpathlineto{\pgfqpoint{4.295433in}{2.997008in}}%
\pgfpathlineto{\pgfqpoint{4.302810in}{3.012078in}}%
\pgfpathlineto{\pgfqpoint{4.310186in}{3.027358in}}%
\pgfpathlineto{\pgfqpoint{4.297294in}{3.033823in}}%
\pgfpathlineto{\pgfqpoint{4.284408in}{3.040469in}}%
\pgfpathlineto{\pgfqpoint{4.271526in}{3.047296in}}%
\pgfpathlineto{\pgfqpoint{4.258649in}{3.054306in}}%
\pgfpathlineto{\pgfqpoint{4.251265in}{3.038551in}}%
\pgfpathlineto{\pgfqpoint{4.243880in}{3.023013in}}%
\pgfpathlineto{\pgfqpoint{4.236493in}{3.007686in}}%
\pgfpathlineto{\pgfqpoint{4.229103in}{2.992566in}}%
\pgfpathclose%
\pgfusepath{fill}%
\end{pgfscope}%
\begin{pgfscope}%
\pgfpathrectangle{\pgfqpoint{1.254980in}{0.150000in}}{\pgfqpoint{5.490039in}{5.490039in}}%
\pgfusepath{clip}%
\pgfsetbuttcap%
\pgfsetroundjoin%
\definecolor{currentfill}{rgb}{0.175841,0.441290,0.557685}%
\pgfsetfillcolor{currentfill}%
\pgfsetfillopacity{0.700000}%
\pgfsetlinewidth{0.000000pt}%
\definecolor{currentstroke}{rgb}{0.000000,0.000000,0.000000}%
\pgfsetstrokecolor{currentstroke}%
\pgfsetdash{}{0pt}%
\pgfpathmoveto{\pgfqpoint{4.796856in}{3.289204in}}%
\pgfpathlineto{\pgfqpoint{4.809844in}{3.283661in}}%
\pgfpathlineto{\pgfqpoint{4.822839in}{3.278281in}}%
\pgfpathlineto{\pgfqpoint{4.835841in}{3.273064in}}%
\pgfpathlineto{\pgfqpoint{4.848850in}{3.268010in}}%
\pgfpathlineto{\pgfqpoint{4.856162in}{3.286036in}}%
\pgfpathlineto{\pgfqpoint{4.863476in}{3.304395in}}%
\pgfpathlineto{\pgfqpoint{4.870793in}{3.323095in}}%
\pgfpathlineto{\pgfqpoint{4.857793in}{3.328641in}}%
\pgfpathlineto{\pgfqpoint{4.844799in}{3.334349in}}%
\pgfpathlineto{\pgfqpoint{4.831813in}{3.340220in}}%
\pgfpathlineto{\pgfqpoint{4.818834in}{3.346255in}}%
\pgfpathlineto{\pgfqpoint{4.811505in}{3.326893in}}%
\pgfpathlineto{\pgfqpoint{4.804179in}{3.307878in}}%
\pgfpathlineto{\pgfqpoint{4.796856in}{3.289204in}}%
\pgfpathclose%
\pgfusepath{fill}%
\end{pgfscope}%
\begin{pgfscope}%
\pgfpathrectangle{\pgfqpoint{1.254980in}{0.150000in}}{\pgfqpoint{5.490039in}{5.490039in}}%
\pgfusepath{clip}%
\pgfsetbuttcap%
\pgfsetroundjoin%
\definecolor{currentfill}{rgb}{0.153364,0.497000,0.557724}%
\pgfsetfillcolor{currentfill}%
\pgfsetfillopacity{0.700000}%
\pgfsetlinewidth{0.000000pt}%
\definecolor{currentstroke}{rgb}{0.000000,0.000000,0.000000}%
\pgfsetstrokecolor{currentstroke}%
\pgfsetdash{}{0pt}%
\pgfpathmoveto{\pgfqpoint{3.257356in}{3.454061in}}%
\pgfpathlineto{\pgfqpoint{3.270284in}{3.435095in}}%
\pgfpathlineto{\pgfqpoint{3.283205in}{3.416412in}}%
\pgfpathlineto{\pgfqpoint{3.296120in}{3.398010in}}%
\pgfpathlineto{\pgfqpoint{3.309030in}{3.379886in}}%
\pgfpathlineto{\pgfqpoint{3.316584in}{3.396713in}}%
\pgfpathlineto{\pgfqpoint{3.324133in}{3.413759in}}%
\pgfpathlineto{\pgfqpoint{3.331677in}{3.431028in}}%
\pgfpathlineto{\pgfqpoint{3.339214in}{3.448524in}}%
\pgfpathlineto{\pgfqpoint{3.326310in}{3.466987in}}%
\pgfpathlineto{\pgfqpoint{3.313400in}{3.485729in}}%
\pgfpathlineto{\pgfqpoint{3.300484in}{3.504753in}}%
\pgfpathlineto{\pgfqpoint{3.287562in}{3.524060in}}%
\pgfpathlineto{\pgfqpoint{3.280020in}{3.506212in}}%
\pgfpathlineto{\pgfqpoint{3.272471in}{3.488600in}}%
\pgfpathlineto{\pgfqpoint{3.264917in}{3.471217in}}%
\pgfpathlineto{\pgfqpoint{3.257356in}{3.454061in}}%
\pgfpathclose%
\pgfusepath{fill}%
\end{pgfscope}%
\begin{pgfscope}%
\pgfpathrectangle{\pgfqpoint{1.254980in}{0.150000in}}{\pgfqpoint{5.490039in}{5.490039in}}%
\pgfusepath{clip}%
\pgfsetbuttcap%
\pgfsetroundjoin%
\definecolor{currentfill}{rgb}{0.241237,0.296485,0.539709}%
\pgfsetfillcolor{currentfill}%
\pgfsetfillopacity{0.700000}%
\pgfsetlinewidth{0.000000pt}%
\definecolor{currentstroke}{rgb}{0.000000,0.000000,0.000000}%
\pgfsetstrokecolor{currentstroke}%
\pgfsetdash{}{0pt}%
\pgfpathmoveto{\pgfqpoint{3.801681in}{2.950718in}}%
\pgfpathlineto{\pgfqpoint{3.814515in}{2.941091in}}%
\pgfpathlineto{\pgfqpoint{3.827351in}{2.931671in}}%
\pgfpathlineto{\pgfqpoint{3.840189in}{2.922458in}}%
\pgfpathlineto{\pgfqpoint{3.853029in}{2.913450in}}%
\pgfpathlineto{\pgfqpoint{3.860505in}{2.927672in}}%
\pgfpathlineto{\pgfqpoint{3.867977in}{2.942059in}}%
\pgfpathlineto{\pgfqpoint{3.875446in}{2.956614in}}%
\pgfpathlineto{\pgfqpoint{3.882910in}{2.971342in}}%
\pgfpathlineto{\pgfqpoint{3.870077in}{2.980707in}}%
\pgfpathlineto{\pgfqpoint{3.857245in}{2.990277in}}%
\pgfpathlineto{\pgfqpoint{3.844415in}{3.000053in}}%
\pgfpathlineto{\pgfqpoint{3.831587in}{3.010037in}}%
\pgfpathlineto{\pgfqpoint{3.824117in}{2.994942in}}%
\pgfpathlineto{\pgfqpoint{3.816642in}{2.980026in}}%
\pgfpathlineto{\pgfqpoint{3.809164in}{2.965286in}}%
\pgfpathlineto{\pgfqpoint{3.801681in}{2.950718in}}%
\pgfpathclose%
\pgfusepath{fill}%
\end{pgfscope}%
\begin{pgfscope}%
\pgfpathrectangle{\pgfqpoint{1.254980in}{0.150000in}}{\pgfqpoint{5.490039in}{5.490039in}}%
\pgfusepath{clip}%
\pgfsetbuttcap%
\pgfsetroundjoin%
\definecolor{currentfill}{rgb}{0.243113,0.292092,0.538516}%
\pgfsetfillcolor{currentfill}%
\pgfsetfillopacity{0.700000}%
\pgfsetlinewidth{0.000000pt}%
\definecolor{currentstroke}{rgb}{0.000000,0.000000,0.000000}%
\pgfsetstrokecolor{currentstroke}%
\pgfsetdash{}{0pt}%
\pgfpathmoveto{\pgfqpoint{3.934265in}{2.935909in}}%
\pgfpathlineto{\pgfqpoint{3.947110in}{2.927551in}}%
\pgfpathlineto{\pgfqpoint{3.959958in}{2.919391in}}%
\pgfpathlineto{\pgfqpoint{3.972809in}{2.911428in}}%
\pgfpathlineto{\pgfqpoint{3.985662in}{2.903660in}}%
\pgfpathlineto{\pgfqpoint{3.993110in}{2.917818in}}%
\pgfpathlineto{\pgfqpoint{4.000554in}{2.932144in}}%
\pgfpathlineto{\pgfqpoint{4.007994in}{2.946643in}}%
\pgfpathlineto{\pgfqpoint{4.015431in}{2.961318in}}%
\pgfpathlineto{\pgfqpoint{4.002584in}{2.969469in}}%
\pgfpathlineto{\pgfqpoint{3.989740in}{2.977815in}}%
\pgfpathlineto{\pgfqpoint{3.976899in}{2.986359in}}%
\pgfpathlineto{\pgfqpoint{3.964060in}{2.995100in}}%
\pgfpathlineto{\pgfqpoint{3.956617in}{2.980031in}}%
\pgfpathlineto{\pgfqpoint{3.949170in}{2.965146in}}%
\pgfpathlineto{\pgfqpoint{3.941719in}{2.950440in}}%
\pgfpathlineto{\pgfqpoint{3.934265in}{2.935909in}}%
\pgfpathclose%
\pgfusepath{fill}%
\end{pgfscope}%
\begin{pgfscope}%
\pgfpathrectangle{\pgfqpoint{1.254980in}{0.150000in}}{\pgfqpoint{5.490039in}{5.490039in}}%
\pgfusepath{clip}%
\pgfsetbuttcap%
\pgfsetroundjoin%
\definecolor{currentfill}{rgb}{0.237441,0.305202,0.541921}%
\pgfsetfillcolor{currentfill}%
\pgfsetfillopacity{0.700000}%
\pgfsetlinewidth{0.000000pt}%
\definecolor{currentstroke}{rgb}{0.000000,0.000000,0.000000}%
\pgfsetstrokecolor{currentstroke}%
\pgfsetdash{}{0pt}%
\pgfpathmoveto{\pgfqpoint{4.147996in}{2.960312in}}%
\pgfpathlineto{\pgfqpoint{4.160870in}{2.953467in}}%
\pgfpathlineto{\pgfqpoint{4.173748in}{2.946808in}}%
\pgfpathlineto{\pgfqpoint{4.186631in}{2.940333in}}%
\pgfpathlineto{\pgfqpoint{4.199518in}{2.934043in}}%
\pgfpathlineto{\pgfqpoint{4.206918in}{2.948391in}}%
\pgfpathlineto{\pgfqpoint{4.214316in}{2.962924in}}%
\pgfpathlineto{\pgfqpoint{4.221711in}{2.977647in}}%
\pgfpathlineto{\pgfqpoint{4.229103in}{2.992566in}}%
\pgfpathlineto{\pgfqpoint{4.216223in}{2.999294in}}%
\pgfpathlineto{\pgfqpoint{4.203348in}{3.006207in}}%
\pgfpathlineto{\pgfqpoint{4.190477in}{3.013304in}}%
\pgfpathlineto{\pgfqpoint{4.177610in}{3.020587in}}%
\pgfpathlineto{\pgfqpoint{4.170210in}{3.005220in}}%
\pgfpathlineto{\pgfqpoint{4.162808in}{2.990055in}}%
\pgfpathlineto{\pgfqpoint{4.155404in}{2.975088in}}%
\pgfpathlineto{\pgfqpoint{4.147996in}{2.960312in}}%
\pgfpathclose%
\pgfusepath{fill}%
\end{pgfscope}%
\begin{pgfscope}%
\pgfpathrectangle{\pgfqpoint{1.254980in}{0.150000in}}{\pgfqpoint{5.490039in}{5.490039in}}%
\pgfusepath{clip}%
\pgfsetbuttcap%
\pgfsetroundjoin%
\definecolor{currentfill}{rgb}{0.237441,0.305202,0.541921}%
\pgfsetfillcolor{currentfill}%
\pgfsetfillopacity{0.700000}%
\pgfsetlinewidth{0.000000pt}%
\definecolor{currentstroke}{rgb}{0.000000,0.000000,0.000000}%
\pgfsetstrokecolor{currentstroke}%
\pgfsetdash{}{0pt}%
\pgfpathmoveto{\pgfqpoint{3.669019in}{2.976079in}}%
\pgfpathlineto{\pgfqpoint{3.681853in}{2.965074in}}%
\pgfpathlineto{\pgfqpoint{3.694687in}{2.954287in}}%
\pgfpathlineto{\pgfqpoint{3.707522in}{2.943717in}}%
\pgfpathlineto{\pgfqpoint{3.720357in}{2.933363in}}%
\pgfpathlineto{\pgfqpoint{3.727864in}{2.947607in}}%
\pgfpathlineto{\pgfqpoint{3.735365in}{2.962013in}}%
\pgfpathlineto{\pgfqpoint{3.742863in}{2.976585in}}%
\pgfpathlineto{\pgfqpoint{3.750356in}{2.991326in}}%
\pgfpathlineto{\pgfqpoint{3.737527in}{3.002011in}}%
\pgfpathlineto{\pgfqpoint{3.724699in}{3.012910in}}%
\pgfpathlineto{\pgfqpoint{3.711871in}{3.024028in}}%
\pgfpathlineto{\pgfqpoint{3.699044in}{3.035363in}}%
\pgfpathlineto{\pgfqpoint{3.691544in}{3.020281in}}%
\pgfpathlineto{\pgfqpoint{3.684041in}{3.005376in}}%
\pgfpathlineto{\pgfqpoint{3.676532in}{2.990643in}}%
\pgfpathlineto{\pgfqpoint{3.669019in}{2.976079in}}%
\pgfpathclose%
\pgfusepath{fill}%
\end{pgfscope}%
\begin{pgfscope}%
\pgfpathrectangle{\pgfqpoint{1.254980in}{0.150000in}}{\pgfqpoint{5.490039in}{5.490039in}}%
\pgfusepath{clip}%
\pgfsetbuttcap%
\pgfsetroundjoin%
\definecolor{currentfill}{rgb}{0.866013,0.889868,0.095953}%
\pgfsetfillcolor{currentfill}%
\pgfsetfillopacity{0.700000}%
\pgfsetlinewidth{0.000000pt}%
\definecolor{currentstroke}{rgb}{0.000000,0.000000,0.000000}%
\pgfsetstrokecolor{currentstroke}%
\pgfsetdash{}{0pt}%
\pgfpathmoveto{\pgfqpoint{3.481341in}{4.695010in}}%
\pgfpathlineto{\pgfqpoint{3.494334in}{4.668110in}}%
\pgfpathlineto{\pgfqpoint{3.507318in}{4.641529in}}%
\pgfpathlineto{\pgfqpoint{3.520293in}{4.615263in}}%
\pgfpathlineto{\pgfqpoint{3.533259in}{4.589310in}}%
\pgfpathlineto{\pgfqpoint{3.540607in}{4.622132in}}%
\pgfpathlineto{\pgfqpoint{3.547951in}{4.655456in}}%
\pgfpathlineto{\pgfqpoint{3.555293in}{4.689290in}}%
\pgfpathlineto{\pgfqpoint{3.562632in}{4.723643in}}%
\pgfpathlineto{\pgfqpoint{3.549658in}{4.750306in}}%
\pgfpathlineto{\pgfqpoint{3.536674in}{4.777283in}}%
\pgfpathlineto{\pgfqpoint{3.523682in}{4.804577in}}%
\pgfpathlineto{\pgfqpoint{3.510681in}{4.832193in}}%
\pgfpathlineto{\pgfqpoint{3.503350in}{4.797113in}}%
\pgfpathlineto{\pgfqpoint{3.496017in}{4.762562in}}%
\pgfpathlineto{\pgfqpoint{3.488681in}{4.728531in}}%
\pgfpathlineto{\pgfqpoint{3.481341in}{4.695010in}}%
\pgfpathclose%
\pgfusepath{fill}%
\end{pgfscope}%
\begin{pgfscope}%
\pgfpathrectangle{\pgfqpoint{1.254980in}{0.150000in}}{\pgfqpoint{5.490039in}{5.490039in}}%
\pgfusepath{clip}%
\pgfsetbuttcap%
\pgfsetroundjoin%
\definecolor{currentfill}{rgb}{0.121831,0.589055,0.545623}%
\pgfsetfillcolor{currentfill}%
\pgfsetfillopacity{0.700000}%
\pgfsetlinewidth{0.000000pt}%
\definecolor{currentstroke}{rgb}{0.000000,0.000000,0.000000}%
\pgfsetstrokecolor{currentstroke}%
\pgfsetdash{}{0pt}%
\pgfpathmoveto{\pgfqpoint{3.183927in}{3.689065in}}%
\pgfpathlineto{\pgfqpoint{3.196909in}{3.667386in}}%
\pgfpathlineto{\pgfqpoint{3.209882in}{3.646015in}}%
\pgfpathlineto{\pgfqpoint{3.222848in}{3.624947in}}%
\pgfpathlineto{\pgfqpoint{3.235805in}{3.604179in}}%
\pgfpathlineto{\pgfqpoint{3.243347in}{3.622626in}}%
\pgfpathlineto{\pgfqpoint{3.250882in}{3.641323in}}%
\pgfpathlineto{\pgfqpoint{3.258411in}{3.660275in}}%
\pgfpathlineto{\pgfqpoint{3.265934in}{3.679487in}}%
\pgfpathlineto{\pgfqpoint{3.252981in}{3.700628in}}%
\pgfpathlineto{\pgfqpoint{3.240020in}{3.722070in}}%
\pgfpathlineto{\pgfqpoint{3.227050in}{3.743817in}}%
\pgfpathlineto{\pgfqpoint{3.214073in}{3.765871in}}%
\pgfpathlineto{\pgfqpoint{3.206546in}{3.746273in}}%
\pgfpathlineto{\pgfqpoint{3.199013in}{3.726943in}}%
\pgfpathlineto{\pgfqpoint{3.191473in}{3.707875in}}%
\pgfpathlineto{\pgfqpoint{3.183927in}{3.689065in}}%
\pgfpathclose%
\pgfusepath{fill}%
\end{pgfscope}%
\begin{pgfscope}%
\pgfpathrectangle{\pgfqpoint{1.254980in}{0.150000in}}{\pgfqpoint{5.490039in}{5.490039in}}%
\pgfusepath{clip}%
\pgfsetbuttcap%
\pgfsetroundjoin%
\definecolor{currentfill}{rgb}{0.195860,0.395433,0.555276}%
\pgfsetfillcolor{currentfill}%
\pgfsetfillopacity{0.700000}%
\pgfsetlinewidth{0.000000pt}%
\definecolor{currentstroke}{rgb}{0.000000,0.000000,0.000000}%
\pgfsetstrokecolor{currentstroke}%
\pgfsetdash{}{0pt}%
\pgfpathmoveto{\pgfqpoint{3.381894in}{3.181873in}}%
\pgfpathlineto{\pgfqpoint{3.394768in}{3.166449in}}%
\pgfpathlineto{\pgfqpoint{3.407638in}{3.151278in}}%
\pgfpathlineto{\pgfqpoint{3.420504in}{3.136360in}}%
\pgfpathlineto{\pgfqpoint{3.433368in}{3.121692in}}%
\pgfpathlineto{\pgfqpoint{3.440926in}{3.136792in}}%
\pgfpathlineto{\pgfqpoint{3.448480in}{3.152073in}}%
\pgfpathlineto{\pgfqpoint{3.456028in}{3.167541in}}%
\pgfpathlineto{\pgfqpoint{3.463570in}{3.183198in}}%
\pgfpathlineto{\pgfqpoint{3.450713in}{3.198172in}}%
\pgfpathlineto{\pgfqpoint{3.437853in}{3.213396in}}%
\pgfpathlineto{\pgfqpoint{3.424989in}{3.228873in}}%
\pgfpathlineto{\pgfqpoint{3.412122in}{3.244605in}}%
\pgfpathlineto{\pgfqpoint{3.404573in}{3.228631in}}%
\pgfpathlineto{\pgfqpoint{3.397019in}{3.212853in}}%
\pgfpathlineto{\pgfqpoint{3.389459in}{3.197269in}}%
\pgfpathlineto{\pgfqpoint{3.381894in}{3.181873in}}%
\pgfpathclose%
\pgfusepath{fill}%
\end{pgfscope}%
\begin{pgfscope}%
\pgfpathrectangle{\pgfqpoint{1.254980in}{0.150000in}}{\pgfqpoint{5.490039in}{5.490039in}}%
\pgfusepath{clip}%
\pgfsetbuttcap%
\pgfsetroundjoin%
\definecolor{currentfill}{rgb}{0.208623,0.367752,0.552675}%
\pgfsetfillcolor{currentfill}%
\pgfsetfillopacity{0.700000}%
\pgfsetlinewidth{0.000000pt}%
\definecolor{currentstroke}{rgb}{0.000000,0.000000,0.000000}%
\pgfsetstrokecolor{currentstroke}%
\pgfsetdash{}{0pt}%
\pgfpathmoveto{\pgfqpoint{3.433368in}{3.121692in}}%
\pgfpathlineto{\pgfqpoint{3.446228in}{3.107271in}}%
\pgfpathlineto{\pgfqpoint{3.459086in}{3.093097in}}%
\pgfpathlineto{\pgfqpoint{3.471941in}{3.079166in}}%
\pgfpathlineto{\pgfqpoint{3.484794in}{3.065478in}}%
\pgfpathlineto{\pgfqpoint{3.492346in}{3.080283in}}%
\pgfpathlineto{\pgfqpoint{3.499893in}{3.095263in}}%
\pgfpathlineto{\pgfqpoint{3.507435in}{3.110422in}}%
\pgfpathlineto{\pgfqpoint{3.514971in}{3.125763in}}%
\pgfpathlineto{\pgfqpoint{3.502125in}{3.139756in}}%
\pgfpathlineto{\pgfqpoint{3.489276in}{3.153992in}}%
\pgfpathlineto{\pgfqpoint{3.476424in}{3.168472in}}%
\pgfpathlineto{\pgfqpoint{3.463570in}{3.183198in}}%
\pgfpathlineto{\pgfqpoint{3.456028in}{3.167541in}}%
\pgfpathlineto{\pgfqpoint{3.448480in}{3.152073in}}%
\pgfpathlineto{\pgfqpoint{3.440926in}{3.136792in}}%
\pgfpathlineto{\pgfqpoint{3.433368in}{3.121692in}}%
\pgfpathclose%
\pgfusepath{fill}%
\end{pgfscope}%
\begin{pgfscope}%
\pgfpathrectangle{\pgfqpoint{1.254980in}{0.150000in}}{\pgfqpoint{5.490039in}{5.490039in}}%
\pgfusepath{clip}%
\pgfsetbuttcap%
\pgfsetroundjoin%
\definecolor{currentfill}{rgb}{0.185556,0.418570,0.556753}%
\pgfsetfillcolor{currentfill}%
\pgfsetfillopacity{0.700000}%
\pgfsetlinewidth{0.000000pt}%
\definecolor{currentstroke}{rgb}{0.000000,0.000000,0.000000}%
\pgfsetstrokecolor{currentstroke}%
\pgfsetdash{}{0pt}%
\pgfpathmoveto{\pgfqpoint{3.330360in}{3.246157in}}%
\pgfpathlineto{\pgfqpoint{3.343250in}{3.229694in}}%
\pgfpathlineto{\pgfqpoint{3.356135in}{3.213494in}}%
\pgfpathlineto{\pgfqpoint{3.369017in}{3.197554in}}%
\pgfpathlineto{\pgfqpoint{3.381894in}{3.181873in}}%
\pgfpathlineto{\pgfqpoint{3.389459in}{3.197269in}}%
\pgfpathlineto{\pgfqpoint{3.397019in}{3.212853in}}%
\pgfpathlineto{\pgfqpoint{3.404573in}{3.228631in}}%
\pgfpathlineto{\pgfqpoint{3.412122in}{3.244605in}}%
\pgfpathlineto{\pgfqpoint{3.399251in}{3.260593in}}%
\pgfpathlineto{\pgfqpoint{3.386376in}{3.276840in}}%
\pgfpathlineto{\pgfqpoint{3.373497in}{3.293348in}}%
\pgfpathlineto{\pgfqpoint{3.360613in}{3.310119in}}%
\pgfpathlineto{\pgfqpoint{3.353058in}{3.293827in}}%
\pgfpathlineto{\pgfqpoint{3.345498in}{3.277738in}}%
\pgfpathlineto{\pgfqpoint{3.337932in}{3.261849in}}%
\pgfpathlineto{\pgfqpoint{3.330360in}{3.246157in}}%
\pgfpathclose%
\pgfusepath{fill}%
\end{pgfscope}%
\begin{pgfscope}%
\pgfpathrectangle{\pgfqpoint{1.254980in}{0.150000in}}{\pgfqpoint{5.490039in}{5.490039in}}%
\pgfusepath{clip}%
\pgfsetbuttcap%
\pgfsetroundjoin%
\definecolor{currentfill}{rgb}{0.212395,0.359683,0.551710}%
\pgfsetfillcolor{currentfill}%
\pgfsetfillopacity{0.700000}%
\pgfsetlinewidth{0.000000pt}%
\definecolor{currentstroke}{rgb}{0.000000,0.000000,0.000000}%
\pgfsetstrokecolor{currentstroke}%
\pgfsetdash{}{0pt}%
\pgfpathmoveto{\pgfqpoint{4.524062in}{3.081944in}}%
\pgfpathlineto{\pgfqpoint{4.537011in}{3.076748in}}%
\pgfpathlineto{\pgfqpoint{4.549966in}{3.071723in}}%
\pgfpathlineto{\pgfqpoint{4.562928in}{3.066868in}}%
\pgfpathlineto{\pgfqpoint{4.575896in}{3.062182in}}%
\pgfpathlineto{\pgfqpoint{4.583224in}{3.077467in}}%
\pgfpathlineto{\pgfqpoint{4.590552in}{3.092994in}}%
\pgfpathlineto{\pgfqpoint{4.597880in}{3.108769in}}%
\pgfpathlineto{\pgfqpoint{4.605207in}{3.124799in}}%
\pgfpathlineto{\pgfqpoint{4.592248in}{3.130032in}}%
\pgfpathlineto{\pgfqpoint{4.579296in}{3.135435in}}%
\pgfpathlineto{\pgfqpoint{4.566351in}{3.141008in}}%
\pgfpathlineto{\pgfqpoint{4.553411in}{3.146751in}}%
\pgfpathlineto{\pgfqpoint{4.546074in}{3.130163in}}%
\pgfpathlineto{\pgfqpoint{4.538737in}{3.113837in}}%
\pgfpathlineto{\pgfqpoint{4.531400in}{3.097766in}}%
\pgfpathlineto{\pgfqpoint{4.524062in}{3.081944in}}%
\pgfpathclose%
\pgfusepath{fill}%
\end{pgfscope}%
\begin{pgfscope}%
\pgfpathrectangle{\pgfqpoint{1.254980in}{0.150000in}}{\pgfqpoint{5.490039in}{5.490039in}}%
\pgfusepath{clip}%
\pgfsetbuttcap%
\pgfsetroundjoin%
\definecolor{currentfill}{rgb}{0.243113,0.292092,0.538516}%
\pgfsetfillcolor{currentfill}%
\pgfsetfillopacity{0.700000}%
\pgfsetlinewidth{0.000000pt}%
\definecolor{currentstroke}{rgb}{0.000000,0.000000,0.000000}%
\pgfsetstrokecolor{currentstroke}%
\pgfsetdash{}{0pt}%
\pgfpathmoveto{\pgfqpoint{4.066853in}{2.930651in}}%
\pgfpathlineto{\pgfqpoint{4.079718in}{2.923463in}}%
\pgfpathlineto{\pgfqpoint{4.092586in}{2.916465in}}%
\pgfpathlineto{\pgfqpoint{4.105459in}{2.909656in}}%
\pgfpathlineto{\pgfqpoint{4.118336in}{2.903034in}}%
\pgfpathlineto{\pgfqpoint{4.125755in}{2.917090in}}%
\pgfpathlineto{\pgfqpoint{4.133172in}{2.931319in}}%
\pgfpathlineto{\pgfqpoint{4.140586in}{2.945724in}}%
\pgfpathlineto{\pgfqpoint{4.147996in}{2.960312in}}%
\pgfpathlineto{\pgfqpoint{4.135126in}{2.967345in}}%
\pgfpathlineto{\pgfqpoint{4.122261in}{2.974565in}}%
\pgfpathlineto{\pgfqpoint{4.109400in}{2.981973in}}%
\pgfpathlineto{\pgfqpoint{4.096542in}{2.989571in}}%
\pgfpathlineto{\pgfqpoint{4.089124in}{2.974563in}}%
\pgfpathlineto{\pgfqpoint{4.081704in}{2.959743in}}%
\pgfpathlineto{\pgfqpoint{4.074280in}{2.945107in}}%
\pgfpathlineto{\pgfqpoint{4.066853in}{2.930651in}}%
\pgfpathclose%
\pgfusepath{fill}%
\end{pgfscope}%
\begin{pgfscope}%
\pgfpathrectangle{\pgfqpoint{1.254980in}{0.150000in}}{\pgfqpoint{5.490039in}{5.490039in}}%
\pgfusepath{clip}%
\pgfsetbuttcap%
\pgfsetroundjoin%
\definecolor{currentfill}{rgb}{0.218130,0.347432,0.550038}%
\pgfsetfillcolor{currentfill}%
\pgfsetfillopacity{0.700000}%
\pgfsetlinewidth{0.000000pt}%
\definecolor{currentstroke}{rgb}{0.000000,0.000000,0.000000}%
\pgfsetstrokecolor{currentstroke}%
\pgfsetdash{}{0pt}%
\pgfpathmoveto{\pgfqpoint{3.484794in}{3.065478in}}%
\pgfpathlineto{\pgfqpoint{3.497645in}{3.052030in}}%
\pgfpathlineto{\pgfqpoint{3.510494in}{3.038820in}}%
\pgfpathlineto{\pgfqpoint{3.523342in}{3.025846in}}%
\pgfpathlineto{\pgfqpoint{3.536187in}{3.013107in}}%
\pgfpathlineto{\pgfqpoint{3.543733in}{3.027619in}}%
\pgfpathlineto{\pgfqpoint{3.551273in}{3.042298in}}%
\pgfpathlineto{\pgfqpoint{3.558808in}{3.057149in}}%
\pgfpathlineto{\pgfqpoint{3.566338in}{3.072175in}}%
\pgfpathlineto{\pgfqpoint{3.553499in}{3.085218in}}%
\pgfpathlineto{\pgfqpoint{3.540658in}{3.098495in}}%
\pgfpathlineto{\pgfqpoint{3.527816in}{3.112010in}}%
\pgfpathlineto{\pgfqpoint{3.514971in}{3.125763in}}%
\pgfpathlineto{\pgfqpoint{3.507435in}{3.110422in}}%
\pgfpathlineto{\pgfqpoint{3.499893in}{3.095263in}}%
\pgfpathlineto{\pgfqpoint{3.492346in}{3.080283in}}%
\pgfpathlineto{\pgfqpoint{3.484794in}{3.065478in}}%
\pgfpathclose%
\pgfusepath{fill}%
\end{pgfscope}%
\begin{pgfscope}%
\pgfpathrectangle{\pgfqpoint{1.254980in}{0.150000in}}{\pgfqpoint{5.490039in}{5.490039in}}%
\pgfusepath{clip}%
\pgfsetbuttcap%
\pgfsetroundjoin%
\definecolor{currentfill}{rgb}{0.140536,0.530132,0.555659}%
\pgfsetfillcolor{currentfill}%
\pgfsetfillopacity{0.700000}%
\pgfsetlinewidth{0.000000pt}%
\definecolor{currentstroke}{rgb}{0.000000,0.000000,0.000000}%
\pgfsetstrokecolor{currentstroke}%
\pgfsetdash{}{0pt}%
\pgfpathmoveto{\pgfqpoint{3.205578in}{3.532812in}}%
\pgfpathlineto{\pgfqpoint{3.218533in}{3.512686in}}%
\pgfpathlineto{\pgfqpoint{3.231481in}{3.492854in}}%
\pgfpathlineto{\pgfqpoint{3.244422in}{3.473313in}}%
\pgfpathlineto{\pgfqpoint{3.257356in}{3.454061in}}%
\pgfpathlineto{\pgfqpoint{3.264917in}{3.471217in}}%
\pgfpathlineto{\pgfqpoint{3.272471in}{3.488600in}}%
\pgfpathlineto{\pgfqpoint{3.280020in}{3.506212in}}%
\pgfpathlineto{\pgfqpoint{3.287562in}{3.524060in}}%
\pgfpathlineto{\pgfqpoint{3.274634in}{3.543653in}}%
\pgfpathlineto{\pgfqpoint{3.261698in}{3.563535in}}%
\pgfpathlineto{\pgfqpoint{3.248755in}{3.583710in}}%
\pgfpathlineto{\pgfqpoint{3.235805in}{3.604179in}}%
\pgfpathlineto{\pgfqpoint{3.228258in}{3.585979in}}%
\pgfpathlineto{\pgfqpoint{3.220704in}{3.568020in}}%
\pgfpathlineto{\pgfqpoint{3.213144in}{3.550300in}}%
\pgfpathlineto{\pgfqpoint{3.205578in}{3.532812in}}%
\pgfpathclose%
\pgfusepath{fill}%
\end{pgfscope}%
\begin{pgfscope}%
\pgfpathrectangle{\pgfqpoint{1.254980in}{0.150000in}}{\pgfqpoint{5.490039in}{5.490039in}}%
\pgfusepath{clip}%
\pgfsetbuttcap%
\pgfsetroundjoin%
\definecolor{currentfill}{rgb}{0.203063,0.379716,0.553925}%
\pgfsetfillcolor{currentfill}%
\pgfsetfillopacity{0.700000}%
\pgfsetlinewidth{0.000000pt}%
\definecolor{currentstroke}{rgb}{0.000000,0.000000,0.000000}%
\pgfsetstrokecolor{currentstroke}%
\pgfsetdash{}{0pt}%
\pgfpathmoveto{\pgfqpoint{4.605207in}{3.124799in}}%
\pgfpathlineto{\pgfqpoint{4.618172in}{3.119735in}}%
\pgfpathlineto{\pgfqpoint{4.631144in}{3.114838in}}%
\pgfpathlineto{\pgfqpoint{4.644123in}{3.110109in}}%
\pgfpathlineto{\pgfqpoint{4.657109in}{3.105548in}}%
\pgfpathlineto{\pgfqpoint{4.664426in}{3.121275in}}%
\pgfpathlineto{\pgfqpoint{4.671743in}{3.137263in}}%
\pgfpathlineto{\pgfqpoint{4.679060in}{3.153519in}}%
\pgfpathlineto{\pgfqpoint{4.686379in}{3.170050in}}%
\pgfpathlineto{\pgfqpoint{4.673404in}{3.175187in}}%
\pgfpathlineto{\pgfqpoint{4.660435in}{3.180491in}}%
\pgfpathlineto{\pgfqpoint{4.647474in}{3.185962in}}%
\pgfpathlineto{\pgfqpoint{4.634518in}{3.191602in}}%
\pgfpathlineto{\pgfqpoint{4.627190in}{3.174485in}}%
\pgfpathlineto{\pgfqpoint{4.619862in}{3.157651in}}%
\pgfpathlineto{\pgfqpoint{4.612534in}{3.141091in}}%
\pgfpathlineto{\pgfqpoint{4.605207in}{3.124799in}}%
\pgfpathclose%
\pgfusepath{fill}%
\end{pgfscope}%
\begin{pgfscope}%
\pgfpathrectangle{\pgfqpoint{1.254980in}{0.150000in}}{\pgfqpoint{5.490039in}{5.490039in}}%
\pgfusepath{clip}%
\pgfsetbuttcap%
\pgfsetroundjoin%
\definecolor{currentfill}{rgb}{0.220057,0.343307,0.549413}%
\pgfsetfillcolor{currentfill}%
\pgfsetfillopacity{0.700000}%
\pgfsetlinewidth{0.000000pt}%
\definecolor{currentstroke}{rgb}{0.000000,0.000000,0.000000}%
\pgfsetstrokecolor{currentstroke}%
\pgfsetdash{}{0pt}%
\pgfpathmoveto{\pgfqpoint{4.442931in}{3.041447in}}%
\pgfpathlineto{\pgfqpoint{4.455864in}{3.036081in}}%
\pgfpathlineto{\pgfqpoint{4.468804in}{3.030890in}}%
\pgfpathlineto{\pgfqpoint{4.481749in}{3.025870in}}%
\pgfpathlineto{\pgfqpoint{4.494701in}{3.021023in}}%
\pgfpathlineto{\pgfqpoint{4.502043in}{3.035911in}}%
\pgfpathlineto{\pgfqpoint{4.509383in}{3.051023in}}%
\pgfpathlineto{\pgfqpoint{4.516723in}{3.066365in}}%
\pgfpathlineto{\pgfqpoint{4.524062in}{3.081944in}}%
\pgfpathlineto{\pgfqpoint{4.511119in}{3.087312in}}%
\pgfpathlineto{\pgfqpoint{4.498183in}{3.092851in}}%
\pgfpathlineto{\pgfqpoint{4.485253in}{3.098563in}}%
\pgfpathlineto{\pgfqpoint{4.472328in}{3.104448in}}%
\pgfpathlineto{\pgfqpoint{4.464981in}{3.088338in}}%
\pgfpathlineto{\pgfqpoint{4.457632in}{3.072473in}}%
\pgfpathlineto{\pgfqpoint{4.450282in}{3.056844in}}%
\pgfpathlineto{\pgfqpoint{4.442931in}{3.041447in}}%
\pgfpathclose%
\pgfusepath{fill}%
\end{pgfscope}%
\begin{pgfscope}%
\pgfpathrectangle{\pgfqpoint{1.254980in}{0.150000in}}{\pgfqpoint{5.490039in}{5.490039in}}%
\pgfusepath{clip}%
\pgfsetbuttcap%
\pgfsetroundjoin%
\definecolor{currentfill}{rgb}{0.194100,0.399323,0.555565}%
\pgfsetfillcolor{currentfill}%
\pgfsetfillopacity{0.700000}%
\pgfsetlinewidth{0.000000pt}%
\definecolor{currentstroke}{rgb}{0.000000,0.000000,0.000000}%
\pgfsetstrokecolor{currentstroke}%
\pgfsetdash{}{0pt}%
\pgfpathmoveto{\pgfqpoint{4.686379in}{3.170050in}}%
\pgfpathlineto{\pgfqpoint{4.699361in}{3.165080in}}%
\pgfpathlineto{\pgfqpoint{4.712350in}{3.160275in}}%
\pgfpathlineto{\pgfqpoint{4.725346in}{3.155636in}}%
\pgfpathlineto{\pgfqpoint{4.738350in}{3.151161in}}%
\pgfpathlineto{\pgfqpoint{4.745658in}{3.167381in}}%
\pgfpathlineto{\pgfqpoint{4.752967in}{3.183883in}}%
\pgfpathlineto{\pgfqpoint{4.760277in}{3.200673in}}%
\pgfpathlineto{\pgfqpoint{4.767589in}{3.217759in}}%
\pgfpathlineto{\pgfqpoint{4.754597in}{3.222836in}}%
\pgfpathlineto{\pgfqpoint{4.741612in}{3.228078in}}%
\pgfpathlineto{\pgfqpoint{4.728634in}{3.233485in}}%
\pgfpathlineto{\pgfqpoint{4.715662in}{3.239059in}}%
\pgfpathlineto{\pgfqpoint{4.708339in}{3.221360in}}%
\pgfpathlineto{\pgfqpoint{4.701018in}{3.203963in}}%
\pgfpathlineto{\pgfqpoint{4.693698in}{3.186862in}}%
\pgfpathlineto{\pgfqpoint{4.686379in}{3.170050in}}%
\pgfpathclose%
\pgfusepath{fill}%
\end{pgfscope}%
\begin{pgfscope}%
\pgfpathrectangle{\pgfqpoint{1.254980in}{0.150000in}}{\pgfqpoint{5.490039in}{5.490039in}}%
\pgfusepath{clip}%
\pgfsetbuttcap%
\pgfsetroundjoin%
\definecolor{currentfill}{rgb}{0.227802,0.326594,0.546532}%
\pgfsetfillcolor{currentfill}%
\pgfsetfillopacity{0.700000}%
\pgfsetlinewidth{0.000000pt}%
\definecolor{currentstroke}{rgb}{0.000000,0.000000,0.000000}%
\pgfsetstrokecolor{currentstroke}%
\pgfsetdash{}{0pt}%
\pgfpathmoveto{\pgfqpoint{4.361805in}{3.003289in}}%
\pgfpathlineto{\pgfqpoint{4.374723in}{2.997716in}}%
\pgfpathlineto{\pgfqpoint{4.387647in}{2.992319in}}%
\pgfpathlineto{\pgfqpoint{4.400577in}{2.987098in}}%
\pgfpathlineto{\pgfqpoint{4.413513in}{2.982051in}}%
\pgfpathlineto{\pgfqpoint{4.420870in}{2.996583in}}%
\pgfpathlineto{\pgfqpoint{4.428225in}{3.011322in}}%
\pgfpathlineto{\pgfqpoint{4.435579in}{3.026275in}}%
\pgfpathlineto{\pgfqpoint{4.442931in}{3.041447in}}%
\pgfpathlineto{\pgfqpoint{4.430004in}{3.046986in}}%
\pgfpathlineto{\pgfqpoint{4.417083in}{3.052700in}}%
\pgfpathlineto{\pgfqpoint{4.404167in}{3.058589in}}%
\pgfpathlineto{\pgfqpoint{4.391257in}{3.064655in}}%
\pgfpathlineto{\pgfqpoint{4.383896in}{3.048980in}}%
\pgfpathlineto{\pgfqpoint{4.376534in}{3.033531in}}%
\pgfpathlineto{\pgfqpoint{4.369170in}{3.018303in}}%
\pgfpathlineto{\pgfqpoint{4.361805in}{3.003289in}}%
\pgfpathclose%
\pgfusepath{fill}%
\end{pgfscope}%
\begin{pgfscope}%
\pgfpathrectangle{\pgfqpoint{1.254980in}{0.150000in}}{\pgfqpoint{5.490039in}{5.490039in}}%
\pgfusepath{clip}%
\pgfsetbuttcap%
\pgfsetroundjoin%
\definecolor{currentfill}{rgb}{0.172719,0.448791,0.557885}%
\pgfsetfillcolor{currentfill}%
\pgfsetfillopacity{0.700000}%
\pgfsetlinewidth{0.000000pt}%
\definecolor{currentstroke}{rgb}{0.000000,0.000000,0.000000}%
\pgfsetstrokecolor{currentstroke}%
\pgfsetdash{}{0pt}%
\pgfpathmoveto{\pgfqpoint{3.278751in}{3.314688in}}%
\pgfpathlineto{\pgfqpoint{3.291661in}{3.297149in}}%
\pgfpathlineto{\pgfqpoint{3.304566in}{3.279882in}}%
\pgfpathlineto{\pgfqpoint{3.317465in}{3.262886in}}%
\pgfpathlineto{\pgfqpoint{3.330360in}{3.246157in}}%
\pgfpathlineto{\pgfqpoint{3.337932in}{3.261849in}}%
\pgfpathlineto{\pgfqpoint{3.345498in}{3.277738in}}%
\pgfpathlineto{\pgfqpoint{3.353058in}{3.293827in}}%
\pgfpathlineto{\pgfqpoint{3.360613in}{3.310119in}}%
\pgfpathlineto{\pgfqpoint{3.347725in}{3.327156in}}%
\pgfpathlineto{\pgfqpoint{3.334832in}{3.344461in}}%
\pgfpathlineto{\pgfqpoint{3.321933in}{3.362037in}}%
\pgfpathlineto{\pgfqpoint{3.309030in}{3.379886in}}%
\pgfpathlineto{\pgfqpoint{3.301469in}{3.363274in}}%
\pgfpathlineto{\pgfqpoint{3.293902in}{3.346872in}}%
\pgfpathlineto{\pgfqpoint{3.286330in}{3.330678in}}%
\pgfpathlineto{\pgfqpoint{3.278751in}{3.314688in}}%
\pgfpathclose%
\pgfusepath{fill}%
\end{pgfscope}%
\begin{pgfscope}%
\pgfpathrectangle{\pgfqpoint{1.254980in}{0.150000in}}{\pgfqpoint{5.490039in}{5.490039in}}%
\pgfusepath{clip}%
\pgfsetbuttcap%
\pgfsetroundjoin%
\definecolor{currentfill}{rgb}{0.246811,0.283237,0.535941}%
\pgfsetfillcolor{currentfill}%
\pgfsetfillopacity{0.700000}%
\pgfsetlinewidth{0.000000pt}%
\definecolor{currentstroke}{rgb}{0.000000,0.000000,0.000000}%
\pgfsetstrokecolor{currentstroke}%
\pgfsetdash{}{0pt}%
\pgfpathmoveto{\pgfqpoint{3.853029in}{2.913450in}}%
\pgfpathlineto{\pgfqpoint{3.865871in}{2.904645in}}%
\pgfpathlineto{\pgfqpoint{3.878715in}{2.896043in}}%
\pgfpathlineto{\pgfqpoint{3.891561in}{2.887642in}}%
\pgfpathlineto{\pgfqpoint{3.904410in}{2.879441in}}%
\pgfpathlineto{\pgfqpoint{3.911880in}{2.893318in}}%
\pgfpathlineto{\pgfqpoint{3.919345in}{2.907352in}}%
\pgfpathlineto{\pgfqpoint{3.926807in}{2.921547in}}%
\pgfpathlineto{\pgfqpoint{3.934265in}{2.935909in}}%
\pgfpathlineto{\pgfqpoint{3.921423in}{2.944465in}}%
\pgfpathlineto{\pgfqpoint{3.908583in}{2.953223in}}%
\pgfpathlineto{\pgfqpoint{3.895745in}{2.962181in}}%
\pgfpathlineto{\pgfqpoint{3.882910in}{2.971342in}}%
\pgfpathlineto{\pgfqpoint{3.875446in}{2.956614in}}%
\pgfpathlineto{\pgfqpoint{3.867977in}{2.942059in}}%
\pgfpathlineto{\pgfqpoint{3.860505in}{2.927672in}}%
\pgfpathlineto{\pgfqpoint{3.853029in}{2.913450in}}%
\pgfpathclose%
\pgfusepath{fill}%
\end{pgfscope}%
\begin{pgfscope}%
\pgfpathrectangle{\pgfqpoint{1.254980in}{0.150000in}}{\pgfqpoint{5.490039in}{5.490039in}}%
\pgfusepath{clip}%
\pgfsetbuttcap%
\pgfsetroundjoin%
\definecolor{currentfill}{rgb}{0.243113,0.292092,0.538516}%
\pgfsetfillcolor{currentfill}%
\pgfsetfillopacity{0.700000}%
\pgfsetlinewidth{0.000000pt}%
\definecolor{currentstroke}{rgb}{0.000000,0.000000,0.000000}%
\pgfsetstrokecolor{currentstroke}%
\pgfsetdash{}{0pt}%
\pgfpathmoveto{\pgfqpoint{3.720357in}{2.933363in}}%
\pgfpathlineto{\pgfqpoint{3.733193in}{2.923222in}}%
\pgfpathlineto{\pgfqpoint{3.746030in}{2.913295in}}%
\pgfpathlineto{\pgfqpoint{3.758868in}{2.903578in}}%
\pgfpathlineto{\pgfqpoint{3.771708in}{2.894072in}}%
\pgfpathlineto{\pgfqpoint{3.779208in}{2.907997in}}%
\pgfpathlineto{\pgfqpoint{3.786703in}{2.922077in}}%
\pgfpathlineto{\pgfqpoint{3.794194in}{2.936316in}}%
\pgfpathlineto{\pgfqpoint{3.801681in}{2.950718in}}%
\pgfpathlineto{\pgfqpoint{3.788848in}{2.960553in}}%
\pgfpathlineto{\pgfqpoint{3.776016in}{2.970599in}}%
\pgfpathlineto{\pgfqpoint{3.763186in}{2.980856in}}%
\pgfpathlineto{\pgfqpoint{3.750356in}{2.991326in}}%
\pgfpathlineto{\pgfqpoint{3.742863in}{2.976585in}}%
\pgfpathlineto{\pgfqpoint{3.735365in}{2.962013in}}%
\pgfpathlineto{\pgfqpoint{3.727864in}{2.947607in}}%
\pgfpathlineto{\pgfqpoint{3.720357in}{2.933363in}}%
\pgfpathclose%
\pgfusepath{fill}%
\end{pgfscope}%
\begin{pgfscope}%
\pgfpathrectangle{\pgfqpoint{1.254980in}{0.150000in}}{\pgfqpoint{5.490039in}{5.490039in}}%
\pgfusepath{clip}%
\pgfsetbuttcap%
\pgfsetroundjoin%
\definecolor{currentfill}{rgb}{0.227802,0.326594,0.546532}%
\pgfsetfillcolor{currentfill}%
\pgfsetfillopacity{0.700000}%
\pgfsetlinewidth{0.000000pt}%
\definecolor{currentstroke}{rgb}{0.000000,0.000000,0.000000}%
\pgfsetstrokecolor{currentstroke}%
\pgfsetdash{}{0pt}%
\pgfpathmoveto{\pgfqpoint{3.536187in}{3.013107in}}%
\pgfpathlineto{\pgfqpoint{3.549032in}{3.000601in}}%
\pgfpathlineto{\pgfqpoint{3.561875in}{2.988326in}}%
\pgfpathlineto{\pgfqpoint{3.574717in}{2.976281in}}%
\pgfpathlineto{\pgfqpoint{3.587559in}{2.964463in}}%
\pgfpathlineto{\pgfqpoint{3.595097in}{2.978682in}}%
\pgfpathlineto{\pgfqpoint{3.602631in}{2.993061in}}%
\pgfpathlineto{\pgfqpoint{3.610160in}{3.007605in}}%
\pgfpathlineto{\pgfqpoint{3.617684in}{3.022317in}}%
\pgfpathlineto{\pgfqpoint{3.604849in}{3.034437in}}%
\pgfpathlineto{\pgfqpoint{3.592013in}{3.046786in}}%
\pgfpathlineto{\pgfqpoint{3.579176in}{3.059365in}}%
\pgfpathlineto{\pgfqpoint{3.566338in}{3.072175in}}%
\pgfpathlineto{\pgfqpoint{3.558808in}{3.057149in}}%
\pgfpathlineto{\pgfqpoint{3.551273in}{3.042298in}}%
\pgfpathlineto{\pgfqpoint{3.543733in}{3.027619in}}%
\pgfpathlineto{\pgfqpoint{3.536187in}{3.013107in}}%
\pgfpathclose%
\pgfusepath{fill}%
\end{pgfscope}%
\begin{pgfscope}%
\pgfpathrectangle{\pgfqpoint{1.254980in}{0.150000in}}{\pgfqpoint{5.490039in}{5.490039in}}%
\pgfusepath{clip}%
\pgfsetbuttcap%
\pgfsetroundjoin%
\definecolor{currentfill}{rgb}{0.185556,0.418570,0.556753}%
\pgfsetfillcolor{currentfill}%
\pgfsetfillopacity{0.700000}%
\pgfsetlinewidth{0.000000pt}%
\definecolor{currentstroke}{rgb}{0.000000,0.000000,0.000000}%
\pgfsetstrokecolor{currentstroke}%
\pgfsetdash{}{0pt}%
\pgfpathmoveto{\pgfqpoint{4.767589in}{3.217759in}}%
\pgfpathlineto{\pgfqpoint{4.780588in}{3.212846in}}%
\pgfpathlineto{\pgfqpoint{4.793595in}{3.208096in}}%
\pgfpathlineto{\pgfqpoint{4.806609in}{3.203510in}}%
\pgfpathlineto{\pgfqpoint{4.819631in}{3.199086in}}%
\pgfpathlineto{\pgfqpoint{4.826932in}{3.215855in}}%
\pgfpathlineto{\pgfqpoint{4.834236in}{3.232927in}}%
\pgfpathlineto{\pgfqpoint{4.841542in}{3.250309in}}%
\pgfpathlineto{\pgfqpoint{4.848850in}{3.268010in}}%
\pgfpathlineto{\pgfqpoint{4.835841in}{3.273064in}}%
\pgfpathlineto{\pgfqpoint{4.822839in}{3.278281in}}%
\pgfpathlineto{\pgfqpoint{4.809844in}{3.283661in}}%
\pgfpathlineto{\pgfqpoint{4.796856in}{3.289204in}}%
\pgfpathlineto{\pgfqpoint{4.789536in}{3.270863in}}%
\pgfpathlineto{\pgfqpoint{4.782218in}{3.252846in}}%
\pgfpathlineto{\pgfqpoint{4.774903in}{3.235147in}}%
\pgfpathlineto{\pgfqpoint{4.767589in}{3.217759in}}%
\pgfpathclose%
\pgfusepath{fill}%
\end{pgfscope}%
\begin{pgfscope}%
\pgfpathrectangle{\pgfqpoint{1.254980in}{0.150000in}}{\pgfqpoint{5.490039in}{5.490039in}}%
\pgfusepath{clip}%
\pgfsetbuttcap%
\pgfsetroundjoin%
\definecolor{currentfill}{rgb}{0.235526,0.309527,0.542944}%
\pgfsetfillcolor{currentfill}%
\pgfsetfillopacity{0.700000}%
\pgfsetlinewidth{0.000000pt}%
\definecolor{currentstroke}{rgb}{0.000000,0.000000,0.000000}%
\pgfsetstrokecolor{currentstroke}%
\pgfsetdash{}{0pt}%
\pgfpathmoveto{\pgfqpoint{4.280671in}{2.967478in}}%
\pgfpathlineto{\pgfqpoint{4.293575in}{2.961658in}}%
\pgfpathlineto{\pgfqpoint{4.306485in}{2.956018in}}%
\pgfpathlineto{\pgfqpoint{4.319401in}{2.950555in}}%
\pgfpathlineto{\pgfqpoint{4.332322in}{2.945270in}}%
\pgfpathlineto{\pgfqpoint{4.339696in}{2.959481in}}%
\pgfpathlineto{\pgfqpoint{4.347067in}{2.973884in}}%
\pgfpathlineto{\pgfqpoint{4.354437in}{2.988485in}}%
\pgfpathlineto{\pgfqpoint{4.361805in}{3.003289in}}%
\pgfpathlineto{\pgfqpoint{4.348892in}{3.009039in}}%
\pgfpathlineto{\pgfqpoint{4.335985in}{3.014967in}}%
\pgfpathlineto{\pgfqpoint{4.323083in}{3.021073in}}%
\pgfpathlineto{\pgfqpoint{4.310186in}{3.027358in}}%
\pgfpathlineto{\pgfqpoint{4.302810in}{3.012078in}}%
\pgfpathlineto{\pgfqpoint{4.295433in}{2.997008in}}%
\pgfpathlineto{\pgfqpoint{4.288053in}{2.982144in}}%
\pgfpathlineto{\pgfqpoint{4.280671in}{2.967478in}}%
\pgfpathclose%
\pgfusepath{fill}%
\end{pgfscope}%
\begin{pgfscope}%
\pgfpathrectangle{\pgfqpoint{1.254980in}{0.150000in}}{\pgfqpoint{5.490039in}{5.490039in}}%
\pgfusepath{clip}%
\pgfsetbuttcap%
\pgfsetroundjoin%
\definecolor{currentfill}{rgb}{0.248629,0.278775,0.534556}%
\pgfsetfillcolor{currentfill}%
\pgfsetfillopacity{0.700000}%
\pgfsetlinewidth{0.000000pt}%
\definecolor{currentstroke}{rgb}{0.000000,0.000000,0.000000}%
\pgfsetstrokecolor{currentstroke}%
\pgfsetdash{}{0pt}%
\pgfpathmoveto{\pgfqpoint{3.985662in}{2.903660in}}%
\pgfpathlineto{\pgfqpoint{3.998520in}{2.896087in}}%
\pgfpathlineto{\pgfqpoint{4.011380in}{2.888708in}}%
\pgfpathlineto{\pgfqpoint{4.024244in}{2.881521in}}%
\pgfpathlineto{\pgfqpoint{4.037112in}{2.874526in}}%
\pgfpathlineto{\pgfqpoint{4.044553in}{2.888312in}}%
\pgfpathlineto{\pgfqpoint{4.051990in}{2.902258in}}%
\pgfpathlineto{\pgfqpoint{4.059423in}{2.916369in}}%
\pgfpathlineto{\pgfqpoint{4.066853in}{2.930651in}}%
\pgfpathlineto{\pgfqpoint{4.053993in}{2.938029in}}%
\pgfpathlineto{\pgfqpoint{4.041135in}{2.945599in}}%
\pgfpathlineto{\pgfqpoint{4.028282in}{2.953362in}}%
\pgfpathlineto{\pgfqpoint{4.015431in}{2.961318in}}%
\pgfpathlineto{\pgfqpoint{4.007994in}{2.946643in}}%
\pgfpathlineto{\pgfqpoint{4.000554in}{2.932144in}}%
\pgfpathlineto{\pgfqpoint{3.993110in}{2.917818in}}%
\pgfpathlineto{\pgfqpoint{3.985662in}{2.903660in}}%
\pgfpathclose%
\pgfusepath{fill}%
\end{pgfscope}%
\begin{pgfscope}%
\pgfpathrectangle{\pgfqpoint{1.254980in}{0.150000in}}{\pgfqpoint{5.490039in}{5.490039in}}%
\pgfusepath{clip}%
\pgfsetbuttcap%
\pgfsetroundjoin%
\definecolor{currentfill}{rgb}{0.162142,0.474838,0.558140}%
\pgfsetfillcolor{currentfill}%
\pgfsetfillopacity{0.700000}%
\pgfsetlinewidth{0.000000pt}%
\definecolor{currentstroke}{rgb}{0.000000,0.000000,0.000000}%
\pgfsetstrokecolor{currentstroke}%
\pgfsetdash{}{0pt}%
\pgfpathmoveto{\pgfqpoint{3.227052in}{3.387620in}}%
\pgfpathlineto{\pgfqpoint{3.239986in}{3.368966in}}%
\pgfpathlineto{\pgfqpoint{3.252914in}{3.350594in}}%
\pgfpathlineto{\pgfqpoint{3.265835in}{3.332502in}}%
\pgfpathlineto{\pgfqpoint{3.278751in}{3.314688in}}%
\pgfpathlineto{\pgfqpoint{3.286330in}{3.330678in}}%
\pgfpathlineto{\pgfqpoint{3.293902in}{3.346872in}}%
\pgfpathlineto{\pgfqpoint{3.301469in}{3.363274in}}%
\pgfpathlineto{\pgfqpoint{3.309030in}{3.379886in}}%
\pgfpathlineto{\pgfqpoint{3.296120in}{3.398010in}}%
\pgfpathlineto{\pgfqpoint{3.283205in}{3.416412in}}%
\pgfpathlineto{\pgfqpoint{3.270284in}{3.435095in}}%
\pgfpathlineto{\pgfqpoint{3.257356in}{3.454061in}}%
\pgfpathlineto{\pgfqpoint{3.249790in}{3.437127in}}%
\pgfpathlineto{\pgfqpoint{3.242217in}{3.420412in}}%
\pgfpathlineto{\pgfqpoint{3.234638in}{3.403911in}}%
\pgfpathlineto{\pgfqpoint{3.227052in}{3.387620in}}%
\pgfpathclose%
\pgfusepath{fill}%
\end{pgfscope}%
\begin{pgfscope}%
\pgfpathrectangle{\pgfqpoint{1.254980in}{0.150000in}}{\pgfqpoint{5.490039in}{5.490039in}}%
\pgfusepath{clip}%
\pgfsetbuttcap%
\pgfsetroundjoin%
\definecolor{currentfill}{rgb}{0.237441,0.305202,0.541921}%
\pgfsetfillcolor{currentfill}%
\pgfsetfillopacity{0.700000}%
\pgfsetlinewidth{0.000000pt}%
\definecolor{currentstroke}{rgb}{0.000000,0.000000,0.000000}%
\pgfsetstrokecolor{currentstroke}%
\pgfsetdash{}{0pt}%
\pgfpathmoveto{\pgfqpoint{3.587559in}{2.964463in}}%
\pgfpathlineto{\pgfqpoint{3.600400in}{2.952871in}}%
\pgfpathlineto{\pgfqpoint{3.613240in}{2.941504in}}%
\pgfpathlineto{\pgfqpoint{3.626081in}{2.930360in}}%
\pgfpathlineto{\pgfqpoint{3.638921in}{2.919437in}}%
\pgfpathlineto{\pgfqpoint{3.646453in}{2.933363in}}%
\pgfpathlineto{\pgfqpoint{3.653980in}{2.947443in}}%
\pgfpathlineto{\pgfqpoint{3.661502in}{2.961681in}}%
\pgfpathlineto{\pgfqpoint{3.669019in}{2.976079in}}%
\pgfpathlineto{\pgfqpoint{3.656186in}{2.987305in}}%
\pgfpathlineto{\pgfqpoint{3.643352in}{2.998752in}}%
\pgfpathlineto{\pgfqpoint{3.630518in}{3.010422in}}%
\pgfpathlineto{\pgfqpoint{3.617684in}{3.022317in}}%
\pgfpathlineto{\pgfqpoint{3.610160in}{3.007605in}}%
\pgfpathlineto{\pgfqpoint{3.602631in}{2.993061in}}%
\pgfpathlineto{\pgfqpoint{3.595097in}{2.978682in}}%
\pgfpathlineto{\pgfqpoint{3.587559in}{2.964463in}}%
\pgfpathclose%
\pgfusepath{fill}%
\end{pgfscope}%
\begin{pgfscope}%
\pgfpathrectangle{\pgfqpoint{1.254980in}{0.150000in}}{\pgfqpoint{5.490039in}{5.490039in}}%
\pgfusepath{clip}%
\pgfsetbuttcap%
\pgfsetroundjoin%
\definecolor{currentfill}{rgb}{0.177423,0.437527,0.557565}%
\pgfsetfillcolor{currentfill}%
\pgfsetfillopacity{0.700000}%
\pgfsetlinewidth{0.000000pt}%
\definecolor{currentstroke}{rgb}{0.000000,0.000000,0.000000}%
\pgfsetstrokecolor{currentstroke}%
\pgfsetdash{}{0pt}%
\pgfpathmoveto{\pgfqpoint{4.848850in}{3.268010in}}%
\pgfpathlineto{\pgfqpoint{4.861867in}{3.263118in}}%
\pgfpathlineto{\pgfqpoint{4.874892in}{3.258387in}}%
\pgfpathlineto{\pgfqpoint{4.887924in}{3.253818in}}%
\pgfpathlineto{\pgfqpoint{4.900964in}{3.249410in}}%
\pgfpathlineto{\pgfqpoint{4.908262in}{3.266788in}}%
\pgfpathlineto{\pgfqpoint{4.915564in}{3.284492in}}%
\pgfpathlineto{\pgfqpoint{4.922868in}{3.302530in}}%
\pgfpathlineto{\pgfqpoint{4.909838in}{3.307430in}}%
\pgfpathlineto{\pgfqpoint{4.896816in}{3.312491in}}%
\pgfpathlineto{\pgfqpoint{4.883801in}{3.317712in}}%
\pgfpathlineto{\pgfqpoint{4.870793in}{3.323095in}}%
\pgfpathlineto{\pgfqpoint{4.863476in}{3.304395in}}%
\pgfpathlineto{\pgfqpoint{4.856162in}{3.286036in}}%
\pgfpathlineto{\pgfqpoint{4.848850in}{3.268010in}}%
\pgfpathclose%
\pgfusepath{fill}%
\end{pgfscope}%
\begin{pgfscope}%
\pgfpathrectangle{\pgfqpoint{1.254980in}{0.150000in}}{\pgfqpoint{5.490039in}{5.490039in}}%
\pgfusepath{clip}%
\pgfsetbuttcap%
\pgfsetroundjoin%
\definecolor{currentfill}{rgb}{0.993248,0.906157,0.143936}%
\pgfsetfillcolor{currentfill}%
\pgfsetfillopacity{0.700000}%
\pgfsetlinewidth{0.000000pt}%
\definecolor{currentstroke}{rgb}{0.000000,0.000000,0.000000}%
\pgfsetstrokecolor{currentstroke}%
\pgfsetdash{}{0pt}%
\pgfpathmoveto{\pgfqpoint{3.510681in}{4.832193in}}%
\pgfpathlineto{\pgfqpoint{3.523682in}{4.804577in}}%
\pgfpathlineto{\pgfqpoint{3.536674in}{4.777283in}}%
\pgfpathlineto{\pgfqpoint{3.549658in}{4.750306in}}%
\pgfpathlineto{\pgfqpoint{3.562632in}{4.723643in}}%
\pgfpathlineto{\pgfqpoint{3.569969in}{4.758524in}}%
\pgfpathlineto{\pgfqpoint{3.577303in}{4.793942in}}%
\pgfpathlineto{\pgfqpoint{3.584635in}{4.829905in}}%
\pgfpathlineto{\pgfqpoint{3.571653in}{4.857125in}}%
\pgfpathlineto{\pgfqpoint{3.558663in}{4.884661in}}%
\pgfpathlineto{\pgfqpoint{3.545663in}{4.912516in}}%
\pgfpathlineto{\pgfqpoint{3.532654in}{4.940694in}}%
\pgfpathlineto{\pgfqpoint{3.525332in}{4.903974in}}%
\pgfpathlineto{\pgfqpoint{3.518008in}{4.867810in}}%
\pgfpathlineto{\pgfqpoint{3.510681in}{4.832193in}}%
\pgfpathclose%
\pgfusepath{fill}%
\end{pgfscope}%
\begin{pgfscope}%
\pgfpathrectangle{\pgfqpoint{1.254980in}{0.150000in}}{\pgfqpoint{5.490039in}{5.490039in}}%
\pgfusepath{clip}%
\pgfsetbuttcap%
\pgfsetroundjoin%
\definecolor{currentfill}{rgb}{0.241237,0.296485,0.539709}%
\pgfsetfillcolor{currentfill}%
\pgfsetfillopacity{0.700000}%
\pgfsetlinewidth{0.000000pt}%
\definecolor{currentstroke}{rgb}{0.000000,0.000000,0.000000}%
\pgfsetstrokecolor{currentstroke}%
\pgfsetdash{}{0pt}%
\pgfpathmoveto{\pgfqpoint{4.199518in}{2.934043in}}%
\pgfpathlineto{\pgfqpoint{4.212410in}{2.927936in}}%
\pgfpathlineto{\pgfqpoint{4.225307in}{2.922011in}}%
\pgfpathlineto{\pgfqpoint{4.238209in}{2.916268in}}%
\pgfpathlineto{\pgfqpoint{4.251117in}{2.910706in}}%
\pgfpathlineto{\pgfqpoint{4.258509in}{2.924626in}}%
\pgfpathlineto{\pgfqpoint{4.265899in}{2.938725in}}%
\pgfpathlineto{\pgfqpoint{4.273286in}{2.953007in}}%
\pgfpathlineto{\pgfqpoint{4.280671in}{2.967478in}}%
\pgfpathlineto{\pgfqpoint{4.267771in}{2.973478in}}%
\pgfpathlineto{\pgfqpoint{4.254877in}{2.979659in}}%
\pgfpathlineto{\pgfqpoint{4.241987in}{2.986022in}}%
\pgfpathlineto{\pgfqpoint{4.229103in}{2.992566in}}%
\pgfpathlineto{\pgfqpoint{4.221711in}{2.977647in}}%
\pgfpathlineto{\pgfqpoint{4.214316in}{2.962924in}}%
\pgfpathlineto{\pgfqpoint{4.206918in}{2.948391in}}%
\pgfpathlineto{\pgfqpoint{4.199518in}{2.934043in}}%
\pgfpathclose%
\pgfusepath{fill}%
\end{pgfscope}%
\begin{pgfscope}%
\pgfpathrectangle{\pgfqpoint{1.254980in}{0.150000in}}{\pgfqpoint{5.490039in}{5.490039in}}%
\pgfusepath{clip}%
\pgfsetbuttcap%
\pgfsetroundjoin%
\definecolor{currentfill}{rgb}{0.128729,0.563265,0.551229}%
\pgfsetfillcolor{currentfill}%
\pgfsetfillopacity{0.700000}%
\pgfsetlinewidth{0.000000pt}%
\definecolor{currentstroke}{rgb}{0.000000,0.000000,0.000000}%
\pgfsetstrokecolor{currentstroke}%
\pgfsetdash{}{0pt}%
\pgfpathmoveto{\pgfqpoint{3.153678in}{3.616320in}}%
\pgfpathlineto{\pgfqpoint{3.166665in}{3.594987in}}%
\pgfpathlineto{\pgfqpoint{3.179644in}{3.573960in}}%
\pgfpathlineto{\pgfqpoint{3.192615in}{3.553236in}}%
\pgfpathlineto{\pgfqpoint{3.205578in}{3.532812in}}%
\pgfpathlineto{\pgfqpoint{3.213144in}{3.550300in}}%
\pgfpathlineto{\pgfqpoint{3.220704in}{3.568020in}}%
\pgfpathlineto{\pgfqpoint{3.228258in}{3.585979in}}%
\pgfpathlineto{\pgfqpoint{3.235805in}{3.604179in}}%
\pgfpathlineto{\pgfqpoint{3.222848in}{3.624947in}}%
\pgfpathlineto{\pgfqpoint{3.209882in}{3.646015in}}%
\pgfpathlineto{\pgfqpoint{3.196909in}{3.667386in}}%
\pgfpathlineto{\pgfqpoint{3.183927in}{3.689065in}}%
\pgfpathlineto{\pgfqpoint{3.176374in}{3.670508in}}%
\pgfpathlineto{\pgfqpoint{3.168815in}{3.652202in}}%
\pgfpathlineto{\pgfqpoint{3.161250in}{3.634140in}}%
\pgfpathlineto{\pgfqpoint{3.153678in}{3.616320in}}%
\pgfpathclose%
\pgfusepath{fill}%
\end{pgfscope}%
\begin{pgfscope}%
\pgfpathrectangle{\pgfqpoint{1.254980in}{0.150000in}}{\pgfqpoint{5.490039in}{5.490039in}}%
\pgfusepath{clip}%
\pgfsetbuttcap%
\pgfsetroundjoin%
\definecolor{currentfill}{rgb}{0.250425,0.274290,0.533103}%
\pgfsetfillcolor{currentfill}%
\pgfsetfillopacity{0.700000}%
\pgfsetlinewidth{0.000000pt}%
\definecolor{currentstroke}{rgb}{0.000000,0.000000,0.000000}%
\pgfsetstrokecolor{currentstroke}%
\pgfsetdash{}{0pt}%
\pgfpathmoveto{\pgfqpoint{3.771708in}{2.894072in}}%
\pgfpathlineto{\pgfqpoint{3.784549in}{2.884774in}}%
\pgfpathlineto{\pgfqpoint{3.797391in}{2.875684in}}%
\pgfpathlineto{\pgfqpoint{3.810236in}{2.866800in}}%
\pgfpathlineto{\pgfqpoint{3.823082in}{2.858120in}}%
\pgfpathlineto{\pgfqpoint{3.830575in}{2.871727in}}%
\pgfpathlineto{\pgfqpoint{3.838064in}{2.885481in}}%
\pgfpathlineto{\pgfqpoint{3.845548in}{2.899387in}}%
\pgfpathlineto{\pgfqpoint{3.853029in}{2.913450in}}%
\pgfpathlineto{\pgfqpoint{3.840189in}{2.922458in}}%
\pgfpathlineto{\pgfqpoint{3.827351in}{2.931671in}}%
\pgfpathlineto{\pgfqpoint{3.814515in}{2.941091in}}%
\pgfpathlineto{\pgfqpoint{3.801681in}{2.950718in}}%
\pgfpathlineto{\pgfqpoint{3.794194in}{2.936316in}}%
\pgfpathlineto{\pgfqpoint{3.786703in}{2.922077in}}%
\pgfpathlineto{\pgfqpoint{3.779208in}{2.907997in}}%
\pgfpathlineto{\pgfqpoint{3.771708in}{2.894072in}}%
\pgfpathclose%
\pgfusepath{fill}%
\end{pgfscope}%
\begin{pgfscope}%
\pgfpathrectangle{\pgfqpoint{1.254980in}{0.150000in}}{\pgfqpoint{5.490039in}{5.490039in}}%
\pgfusepath{clip}%
\pgfsetbuttcap%
\pgfsetroundjoin%
\definecolor{currentfill}{rgb}{0.246811,0.283237,0.535941}%
\pgfsetfillcolor{currentfill}%
\pgfsetfillopacity{0.700000}%
\pgfsetlinewidth{0.000000pt}%
\definecolor{currentstroke}{rgb}{0.000000,0.000000,0.000000}%
\pgfsetstrokecolor{currentstroke}%
\pgfsetdash{}{0pt}%
\pgfpathmoveto{\pgfqpoint{4.118336in}{2.903034in}}%
\pgfpathlineto{\pgfqpoint{4.131217in}{2.896599in}}%
\pgfpathlineto{\pgfqpoint{4.144102in}{2.890350in}}%
\pgfpathlineto{\pgfqpoint{4.156993in}{2.884286in}}%
\pgfpathlineto{\pgfqpoint{4.169888in}{2.878406in}}%
\pgfpathlineto{\pgfqpoint{4.177300in}{2.892062in}}%
\pgfpathlineto{\pgfqpoint{4.184709in}{2.905883in}}%
\pgfpathlineto{\pgfqpoint{4.192115in}{2.919875in}}%
\pgfpathlineto{\pgfqpoint{4.199518in}{2.934043in}}%
\pgfpathlineto{\pgfqpoint{4.186631in}{2.940333in}}%
\pgfpathlineto{\pgfqpoint{4.173748in}{2.946808in}}%
\pgfpathlineto{\pgfqpoint{4.160870in}{2.953467in}}%
\pgfpathlineto{\pgfqpoint{4.147996in}{2.960312in}}%
\pgfpathlineto{\pgfqpoint{4.140586in}{2.945724in}}%
\pgfpathlineto{\pgfqpoint{4.133172in}{2.931319in}}%
\pgfpathlineto{\pgfqpoint{4.125755in}{2.917090in}}%
\pgfpathlineto{\pgfqpoint{4.118336in}{2.903034in}}%
\pgfpathclose%
\pgfusepath{fill}%
\end{pgfscope}%
\begin{pgfscope}%
\pgfpathrectangle{\pgfqpoint{1.254980in}{0.150000in}}{\pgfqpoint{5.490039in}{5.490039in}}%
\pgfusepath{clip}%
\pgfsetbuttcap%
\pgfsetroundjoin%
\definecolor{currentfill}{rgb}{0.252194,0.269783,0.531579}%
\pgfsetfillcolor{currentfill}%
\pgfsetfillopacity{0.700000}%
\pgfsetlinewidth{0.000000pt}%
\definecolor{currentstroke}{rgb}{0.000000,0.000000,0.000000}%
\pgfsetstrokecolor{currentstroke}%
\pgfsetdash{}{0pt}%
\pgfpathmoveto{\pgfqpoint{3.904410in}{2.879441in}}%
\pgfpathlineto{\pgfqpoint{3.917262in}{2.871440in}}%
\pgfpathlineto{\pgfqpoint{3.930116in}{2.863636in}}%
\pgfpathlineto{\pgfqpoint{3.942974in}{2.856029in}}%
\pgfpathlineto{\pgfqpoint{3.955835in}{2.848617in}}%
\pgfpathlineto{\pgfqpoint{3.963297in}{2.862148in}}%
\pgfpathlineto{\pgfqpoint{3.970756in}{2.875829in}}%
\pgfpathlineto{\pgfqpoint{3.978211in}{2.889665in}}%
\pgfpathlineto{\pgfqpoint{3.985662in}{2.903660in}}%
\pgfpathlineto{\pgfqpoint{3.972809in}{2.911428in}}%
\pgfpathlineto{\pgfqpoint{3.959958in}{2.919391in}}%
\pgfpathlineto{\pgfqpoint{3.947110in}{2.927551in}}%
\pgfpathlineto{\pgfqpoint{3.934265in}{2.935909in}}%
\pgfpathlineto{\pgfqpoint{3.926807in}{2.921547in}}%
\pgfpathlineto{\pgfqpoint{3.919345in}{2.907352in}}%
\pgfpathlineto{\pgfqpoint{3.911880in}{2.893318in}}%
\pgfpathlineto{\pgfqpoint{3.904410in}{2.879441in}}%
\pgfpathclose%
\pgfusepath{fill}%
\end{pgfscope}%
\begin{pgfscope}%
\pgfpathrectangle{\pgfqpoint{1.254980in}{0.150000in}}{\pgfqpoint{5.490039in}{5.490039in}}%
\pgfusepath{clip}%
\pgfsetbuttcap%
\pgfsetroundjoin%
\definecolor{currentfill}{rgb}{0.149039,0.508051,0.557250}%
\pgfsetfillcolor{currentfill}%
\pgfsetfillopacity{0.700000}%
\pgfsetlinewidth{0.000000pt}%
\definecolor{currentstroke}{rgb}{0.000000,0.000000,0.000000}%
\pgfsetstrokecolor{currentstroke}%
\pgfsetdash{}{0pt}%
\pgfpathmoveto{\pgfqpoint{3.175248in}{3.465122in}}%
\pgfpathlineto{\pgfqpoint{3.188210in}{3.445309in}}%
\pgfpathlineto{\pgfqpoint{3.201165in}{3.425789in}}%
\pgfpathlineto{\pgfqpoint{3.214112in}{3.406561in}}%
\pgfpathlineto{\pgfqpoint{3.227052in}{3.387620in}}%
\pgfpathlineto{\pgfqpoint{3.234638in}{3.403911in}}%
\pgfpathlineto{\pgfqpoint{3.242217in}{3.420412in}}%
\pgfpathlineto{\pgfqpoint{3.249790in}{3.437127in}}%
\pgfpathlineto{\pgfqpoint{3.257356in}{3.454061in}}%
\pgfpathlineto{\pgfqpoint{3.244422in}{3.473313in}}%
\pgfpathlineto{\pgfqpoint{3.231481in}{3.492854in}}%
\pgfpathlineto{\pgfqpoint{3.218533in}{3.512686in}}%
\pgfpathlineto{\pgfqpoint{3.205578in}{3.532812in}}%
\pgfpathlineto{\pgfqpoint{3.198005in}{3.515555in}}%
\pgfpathlineto{\pgfqpoint{3.190426in}{3.498523in}}%
\pgfpathlineto{\pgfqpoint{3.182840in}{3.481713in}}%
\pgfpathlineto{\pgfqpoint{3.175248in}{3.465122in}}%
\pgfpathclose%
\pgfusepath{fill}%
\end{pgfscope}%
\begin{pgfscope}%
\pgfpathrectangle{\pgfqpoint{1.254980in}{0.150000in}}{\pgfqpoint{5.490039in}{5.490039in}}%
\pgfusepath{clip}%
\pgfsetbuttcap%
\pgfsetroundjoin%
\definecolor{currentfill}{rgb}{0.244972,0.287675,0.537260}%
\pgfsetfillcolor{currentfill}%
\pgfsetfillopacity{0.700000}%
\pgfsetlinewidth{0.000000pt}%
\definecolor{currentstroke}{rgb}{0.000000,0.000000,0.000000}%
\pgfsetstrokecolor{currentstroke}%
\pgfsetdash{}{0pt}%
\pgfpathmoveto{\pgfqpoint{3.638921in}{2.919437in}}%
\pgfpathlineto{\pgfqpoint{3.651762in}{2.908734in}}%
\pgfpathlineto{\pgfqpoint{3.664602in}{2.898250in}}%
\pgfpathlineto{\pgfqpoint{3.677444in}{2.887982in}}%
\pgfpathlineto{\pgfqpoint{3.690286in}{2.877930in}}%
\pgfpathlineto{\pgfqpoint{3.697811in}{2.891564in}}%
\pgfpathlineto{\pgfqpoint{3.705331in}{2.905345in}}%
\pgfpathlineto{\pgfqpoint{3.712846in}{2.919277in}}%
\pgfpathlineto{\pgfqpoint{3.720357in}{2.933363in}}%
\pgfpathlineto{\pgfqpoint{3.707522in}{2.943717in}}%
\pgfpathlineto{\pgfqpoint{3.694687in}{2.954287in}}%
\pgfpathlineto{\pgfqpoint{3.681853in}{2.965074in}}%
\pgfpathlineto{\pgfqpoint{3.669019in}{2.976079in}}%
\pgfpathlineto{\pgfqpoint{3.661502in}{2.961681in}}%
\pgfpathlineto{\pgfqpoint{3.653980in}{2.947443in}}%
\pgfpathlineto{\pgfqpoint{3.646453in}{2.933363in}}%
\pgfpathlineto{\pgfqpoint{3.638921in}{2.919437in}}%
\pgfpathclose%
\pgfusepath{fill}%
\end{pgfscope}%
\begin{pgfscope}%
\pgfpathrectangle{\pgfqpoint{1.254980in}{0.150000in}}{\pgfqpoint{5.490039in}{5.490039in}}%
\pgfusepath{clip}%
\pgfsetbuttcap%
\pgfsetroundjoin%
\definecolor{currentfill}{rgb}{0.204903,0.375746,0.553533}%
\pgfsetfillcolor{currentfill}%
\pgfsetfillopacity{0.700000}%
\pgfsetlinewidth{0.000000pt}%
\definecolor{currentstroke}{rgb}{0.000000,0.000000,0.000000}%
\pgfsetstrokecolor{currentstroke}%
\pgfsetdash{}{0pt}%
\pgfpathmoveto{\pgfqpoint{3.351576in}{3.122111in}}%
\pgfpathlineto{\pgfqpoint{3.364457in}{3.106965in}}%
\pgfpathlineto{\pgfqpoint{3.377334in}{3.092073in}}%
\pgfpathlineto{\pgfqpoint{3.390208in}{3.077432in}}%
\pgfpathlineto{\pgfqpoint{3.403078in}{3.063041in}}%
\pgfpathlineto{\pgfqpoint{3.410659in}{3.077448in}}%
\pgfpathlineto{\pgfqpoint{3.418234in}{3.092024in}}%
\pgfpathlineto{\pgfqpoint{3.425804in}{3.106770in}}%
\pgfpathlineto{\pgfqpoint{3.433368in}{3.121692in}}%
\pgfpathlineto{\pgfqpoint{3.420504in}{3.136360in}}%
\pgfpathlineto{\pgfqpoint{3.407638in}{3.151278in}}%
\pgfpathlineto{\pgfqpoint{3.394768in}{3.166449in}}%
\pgfpathlineto{\pgfqpoint{3.381894in}{3.181873in}}%
\pgfpathlineto{\pgfqpoint{3.374323in}{3.166663in}}%
\pgfpathlineto{\pgfqpoint{3.366747in}{3.151635in}}%
\pgfpathlineto{\pgfqpoint{3.359164in}{3.136786in}}%
\pgfpathlineto{\pgfqpoint{3.351576in}{3.122111in}}%
\pgfpathclose%
\pgfusepath{fill}%
\end{pgfscope}%
\begin{pgfscope}%
\pgfpathrectangle{\pgfqpoint{1.254980in}{0.150000in}}{\pgfqpoint{5.490039in}{5.490039in}}%
\pgfusepath{clip}%
\pgfsetbuttcap%
\pgfsetroundjoin%
\definecolor{currentfill}{rgb}{0.216210,0.351535,0.550627}%
\pgfsetfillcolor{currentfill}%
\pgfsetfillopacity{0.700000}%
\pgfsetlinewidth{0.000000pt}%
\definecolor{currentstroke}{rgb}{0.000000,0.000000,0.000000}%
\pgfsetstrokecolor{currentstroke}%
\pgfsetdash{}{0pt}%
\pgfpathmoveto{\pgfqpoint{3.403078in}{3.063041in}}%
\pgfpathlineto{\pgfqpoint{3.415946in}{3.048898in}}%
\pgfpathlineto{\pgfqpoint{3.428811in}{3.035001in}}%
\pgfpathlineto{\pgfqpoint{3.441673in}{3.021347in}}%
\pgfpathlineto{\pgfqpoint{3.454533in}{3.007936in}}%
\pgfpathlineto{\pgfqpoint{3.462107in}{3.022076in}}%
\pgfpathlineto{\pgfqpoint{3.469675in}{3.036378in}}%
\pgfpathlineto{\pgfqpoint{3.477237in}{3.050844in}}%
\pgfpathlineto{\pgfqpoint{3.484794in}{3.065478in}}%
\pgfpathlineto{\pgfqpoint{3.471941in}{3.079166in}}%
\pgfpathlineto{\pgfqpoint{3.459086in}{3.093097in}}%
\pgfpathlineto{\pgfqpoint{3.446228in}{3.107271in}}%
\pgfpathlineto{\pgfqpoint{3.433368in}{3.121692in}}%
\pgfpathlineto{\pgfqpoint{3.425804in}{3.106770in}}%
\pgfpathlineto{\pgfqpoint{3.418234in}{3.092024in}}%
\pgfpathlineto{\pgfqpoint{3.410659in}{3.077448in}}%
\pgfpathlineto{\pgfqpoint{3.403078in}{3.063041in}}%
\pgfpathclose%
\pgfusepath{fill}%
\end{pgfscope}%
\begin{pgfscope}%
\pgfpathrectangle{\pgfqpoint{1.254980in}{0.150000in}}{\pgfqpoint{5.490039in}{5.490039in}}%
\pgfusepath{clip}%
\pgfsetbuttcap%
\pgfsetroundjoin%
\definecolor{currentfill}{rgb}{0.194100,0.399323,0.555565}%
\pgfsetfillcolor{currentfill}%
\pgfsetfillopacity{0.700000}%
\pgfsetlinewidth{0.000000pt}%
\definecolor{currentstroke}{rgb}{0.000000,0.000000,0.000000}%
\pgfsetstrokecolor{currentstroke}%
\pgfsetdash{}{0pt}%
\pgfpathmoveto{\pgfqpoint{3.300013in}{3.185279in}}%
\pgfpathlineto{\pgfqpoint{3.312911in}{3.169095in}}%
\pgfpathlineto{\pgfqpoint{3.325803in}{3.153174in}}%
\pgfpathlineto{\pgfqpoint{3.338692in}{3.137514in}}%
\pgfpathlineto{\pgfqpoint{3.351576in}{3.122111in}}%
\pgfpathlineto{\pgfqpoint{3.359164in}{3.136786in}}%
\pgfpathlineto{\pgfqpoint{3.366747in}{3.151635in}}%
\pgfpathlineto{\pgfqpoint{3.374323in}{3.166663in}}%
\pgfpathlineto{\pgfqpoint{3.381894in}{3.181873in}}%
\pgfpathlineto{\pgfqpoint{3.369017in}{3.197554in}}%
\pgfpathlineto{\pgfqpoint{3.356135in}{3.213494in}}%
\pgfpathlineto{\pgfqpoint{3.343250in}{3.229694in}}%
\pgfpathlineto{\pgfqpoint{3.330360in}{3.246157in}}%
\pgfpathlineto{\pgfqpoint{3.322782in}{3.230657in}}%
\pgfpathlineto{\pgfqpoint{3.315199in}{3.215347in}}%
\pgfpathlineto{\pgfqpoint{3.307609in}{3.200222in}}%
\pgfpathlineto{\pgfqpoint{3.300013in}{3.185279in}}%
\pgfpathclose%
\pgfusepath{fill}%
\end{pgfscope}%
\begin{pgfscope}%
\pgfpathrectangle{\pgfqpoint{1.254980in}{0.150000in}}{\pgfqpoint{5.490039in}{5.490039in}}%
\pgfusepath{clip}%
\pgfsetbuttcap%
\pgfsetroundjoin%
\definecolor{currentfill}{rgb}{0.212395,0.359683,0.551710}%
\pgfsetfillcolor{currentfill}%
\pgfsetfillopacity{0.700000}%
\pgfsetlinewidth{0.000000pt}%
\definecolor{currentstroke}{rgb}{0.000000,0.000000,0.000000}%
\pgfsetstrokecolor{currentstroke}%
\pgfsetdash{}{0pt}%
\pgfpathmoveto{\pgfqpoint{4.575896in}{3.062182in}}%
\pgfpathlineto{\pgfqpoint{4.588872in}{3.057665in}}%
\pgfpathlineto{\pgfqpoint{4.601854in}{3.053316in}}%
\pgfpathlineto{\pgfqpoint{4.614843in}{3.049135in}}%
\pgfpathlineto{\pgfqpoint{4.627840in}{3.045121in}}%
\pgfpathlineto{\pgfqpoint{4.635158in}{3.059868in}}%
\pgfpathlineto{\pgfqpoint{4.642475in}{3.074851in}}%
\pgfpathlineto{\pgfqpoint{4.649792in}{3.090075in}}%
\pgfpathlineto{\pgfqpoint{4.657109in}{3.105548in}}%
\pgfpathlineto{\pgfqpoint{4.644123in}{3.110109in}}%
\pgfpathlineto{\pgfqpoint{4.631144in}{3.114838in}}%
\pgfpathlineto{\pgfqpoint{4.618172in}{3.119735in}}%
\pgfpathlineto{\pgfqpoint{4.605207in}{3.124799in}}%
\pgfpathlineto{\pgfqpoint{4.597880in}{3.108769in}}%
\pgfpathlineto{\pgfqpoint{4.590552in}{3.092994in}}%
\pgfpathlineto{\pgfqpoint{4.583224in}{3.077467in}}%
\pgfpathlineto{\pgfqpoint{4.575896in}{3.062182in}}%
\pgfpathclose%
\pgfusepath{fill}%
\end{pgfscope}%
\begin{pgfscope}%
\pgfpathrectangle{\pgfqpoint{1.254980in}{0.150000in}}{\pgfqpoint{5.490039in}{5.490039in}}%
\pgfusepath{clip}%
\pgfsetbuttcap%
\pgfsetroundjoin%
\definecolor{currentfill}{rgb}{0.204903,0.375746,0.553533}%
\pgfsetfillcolor{currentfill}%
\pgfsetfillopacity{0.700000}%
\pgfsetlinewidth{0.000000pt}%
\definecolor{currentstroke}{rgb}{0.000000,0.000000,0.000000}%
\pgfsetstrokecolor{currentstroke}%
\pgfsetdash{}{0pt}%
\pgfpathmoveto{\pgfqpoint{4.657109in}{3.105548in}}%
\pgfpathlineto{\pgfqpoint{4.670102in}{3.101152in}}%
\pgfpathlineto{\pgfqpoint{4.683102in}{3.096923in}}%
\pgfpathlineto{\pgfqpoint{4.696110in}{3.092859in}}%
\pgfpathlineto{\pgfqpoint{4.709125in}{3.088960in}}%
\pgfpathlineto{\pgfqpoint{4.716430in}{3.104122in}}%
\pgfpathlineto{\pgfqpoint{4.723736in}{3.119538in}}%
\pgfpathlineto{\pgfqpoint{4.731043in}{3.135216in}}%
\pgfpathlineto{\pgfqpoint{4.738350in}{3.151161in}}%
\pgfpathlineto{\pgfqpoint{4.725346in}{3.155636in}}%
\pgfpathlineto{\pgfqpoint{4.712350in}{3.160275in}}%
\pgfpathlineto{\pgfqpoint{4.699361in}{3.165080in}}%
\pgfpathlineto{\pgfqpoint{4.686379in}{3.170050in}}%
\pgfpathlineto{\pgfqpoint{4.679060in}{3.153519in}}%
\pgfpathlineto{\pgfqpoint{4.671743in}{3.137263in}}%
\pgfpathlineto{\pgfqpoint{4.664426in}{3.121275in}}%
\pgfpathlineto{\pgfqpoint{4.657109in}{3.105548in}}%
\pgfpathclose%
\pgfusepath{fill}%
\end{pgfscope}%
\begin{pgfscope}%
\pgfpathrectangle{\pgfqpoint{1.254980in}{0.150000in}}{\pgfqpoint{5.490039in}{5.490039in}}%
\pgfusepath{clip}%
\pgfsetbuttcap%
\pgfsetroundjoin%
\definecolor{currentfill}{rgb}{0.221989,0.339161,0.548752}%
\pgfsetfillcolor{currentfill}%
\pgfsetfillopacity{0.700000}%
\pgfsetlinewidth{0.000000pt}%
\definecolor{currentstroke}{rgb}{0.000000,0.000000,0.000000}%
\pgfsetstrokecolor{currentstroke}%
\pgfsetdash{}{0pt}%
\pgfpathmoveto{\pgfqpoint{4.494701in}{3.021023in}}%
\pgfpathlineto{\pgfqpoint{4.507660in}{3.016347in}}%
\pgfpathlineto{\pgfqpoint{4.520625in}{3.011842in}}%
\pgfpathlineto{\pgfqpoint{4.533596in}{3.007507in}}%
\pgfpathlineto{\pgfqpoint{4.546575in}{3.003341in}}%
\pgfpathlineto{\pgfqpoint{4.553907in}{3.017719in}}%
\pgfpathlineto{\pgfqpoint{4.561238in}{3.032314in}}%
\pgfpathlineto{\pgfqpoint{4.568567in}{3.047133in}}%
\pgfpathlineto{\pgfqpoint{4.575896in}{3.062182in}}%
\pgfpathlineto{\pgfqpoint{4.562928in}{3.066868in}}%
\pgfpathlineto{\pgfqpoint{4.549966in}{3.071723in}}%
\pgfpathlineto{\pgfqpoint{4.537011in}{3.076748in}}%
\pgfpathlineto{\pgfqpoint{4.524062in}{3.081944in}}%
\pgfpathlineto{\pgfqpoint{4.516723in}{3.066365in}}%
\pgfpathlineto{\pgfqpoint{4.509383in}{3.051023in}}%
\pgfpathlineto{\pgfqpoint{4.502043in}{3.035911in}}%
\pgfpathlineto{\pgfqpoint{4.494701in}{3.021023in}}%
\pgfpathclose%
\pgfusepath{fill}%
\end{pgfscope}%
\begin{pgfscope}%
\pgfpathrectangle{\pgfqpoint{1.254980in}{0.150000in}}{\pgfqpoint{5.490039in}{5.490039in}}%
\pgfusepath{clip}%
\pgfsetbuttcap%
\pgfsetroundjoin%
\definecolor{currentfill}{rgb}{0.195860,0.395433,0.555276}%
\pgfsetfillcolor{currentfill}%
\pgfsetfillopacity{0.700000}%
\pgfsetlinewidth{0.000000pt}%
\definecolor{currentstroke}{rgb}{0.000000,0.000000,0.000000}%
\pgfsetstrokecolor{currentstroke}%
\pgfsetdash{}{0pt}%
\pgfpathmoveto{\pgfqpoint{4.738350in}{3.151161in}}%
\pgfpathlineto{\pgfqpoint{4.751361in}{3.146851in}}%
\pgfpathlineto{\pgfqpoint{4.764379in}{3.142704in}}%
\pgfpathlineto{\pgfqpoint{4.777406in}{3.138721in}}%
\pgfpathlineto{\pgfqpoint{4.790439in}{3.134900in}}%
\pgfpathlineto{\pgfqpoint{4.797735in}{3.150528in}}%
\pgfpathlineto{\pgfqpoint{4.805032in}{3.166430in}}%
\pgfpathlineto{\pgfqpoint{4.812331in}{3.182614in}}%
\pgfpathlineto{\pgfqpoint{4.819631in}{3.199086in}}%
\pgfpathlineto{\pgfqpoint{4.806609in}{3.203510in}}%
\pgfpathlineto{\pgfqpoint{4.793595in}{3.208096in}}%
\pgfpathlineto{\pgfqpoint{4.780588in}{3.212846in}}%
\pgfpathlineto{\pgfqpoint{4.767589in}{3.217759in}}%
\pgfpathlineto{\pgfqpoint{4.760277in}{3.200673in}}%
\pgfpathlineto{\pgfqpoint{4.752967in}{3.183883in}}%
\pgfpathlineto{\pgfqpoint{4.745658in}{3.167381in}}%
\pgfpathlineto{\pgfqpoint{4.738350in}{3.151161in}}%
\pgfpathclose%
\pgfusepath{fill}%
\end{pgfscope}%
\begin{pgfscope}%
\pgfpathrectangle{\pgfqpoint{1.254980in}{0.150000in}}{\pgfqpoint{5.490039in}{5.490039in}}%
\pgfusepath{clip}%
\pgfsetbuttcap%
\pgfsetroundjoin%
\definecolor{currentfill}{rgb}{0.229739,0.322361,0.545706}%
\pgfsetfillcolor{currentfill}%
\pgfsetfillopacity{0.700000}%
\pgfsetlinewidth{0.000000pt}%
\definecolor{currentstroke}{rgb}{0.000000,0.000000,0.000000}%
\pgfsetstrokecolor{currentstroke}%
\pgfsetdash{}{0pt}%
\pgfpathmoveto{\pgfqpoint{4.413513in}{2.982051in}}%
\pgfpathlineto{\pgfqpoint{4.426455in}{2.977179in}}%
\pgfpathlineto{\pgfqpoint{4.439404in}{2.972479in}}%
\pgfpathlineto{\pgfqpoint{4.452359in}{2.967952in}}%
\pgfpathlineto{\pgfqpoint{4.465320in}{2.963598in}}%
\pgfpathlineto{\pgfqpoint{4.472668in}{2.977647in}}%
\pgfpathlineto{\pgfqpoint{4.480014in}{2.991897in}}%
\pgfpathlineto{\pgfqpoint{4.487358in}{3.006354in}}%
\pgfpathlineto{\pgfqpoint{4.494701in}{3.021023in}}%
\pgfpathlineto{\pgfqpoint{4.481749in}{3.025870in}}%
\pgfpathlineto{\pgfqpoint{4.468804in}{3.030890in}}%
\pgfpathlineto{\pgfqpoint{4.455864in}{3.036081in}}%
\pgfpathlineto{\pgfqpoint{4.442931in}{3.041447in}}%
\pgfpathlineto{\pgfqpoint{4.435579in}{3.026275in}}%
\pgfpathlineto{\pgfqpoint{4.428225in}{3.011322in}}%
\pgfpathlineto{\pgfqpoint{4.420870in}{2.996583in}}%
\pgfpathlineto{\pgfqpoint{4.413513in}{2.982051in}}%
\pgfpathclose%
\pgfusepath{fill}%
\end{pgfscope}%
\begin{pgfscope}%
\pgfpathrectangle{\pgfqpoint{1.254980in}{0.150000in}}{\pgfqpoint{5.490039in}{5.490039in}}%
\pgfusepath{clip}%
\pgfsetbuttcap%
\pgfsetroundjoin%
\definecolor{currentfill}{rgb}{0.227802,0.326594,0.546532}%
\pgfsetfillcolor{currentfill}%
\pgfsetfillopacity{0.700000}%
\pgfsetlinewidth{0.000000pt}%
\definecolor{currentstroke}{rgb}{0.000000,0.000000,0.000000}%
\pgfsetstrokecolor{currentstroke}%
\pgfsetdash{}{0pt}%
\pgfpathmoveto{\pgfqpoint{3.454533in}{3.007936in}}%
\pgfpathlineto{\pgfqpoint{3.467391in}{2.994764in}}%
\pgfpathlineto{\pgfqpoint{3.480247in}{2.981830in}}%
\pgfpathlineto{\pgfqpoint{3.493102in}{2.969133in}}%
\pgfpathlineto{\pgfqpoint{3.505955in}{2.956670in}}%
\pgfpathlineto{\pgfqpoint{3.513521in}{2.970545in}}%
\pgfpathlineto{\pgfqpoint{3.521081in}{2.984574in}}%
\pgfpathlineto{\pgfqpoint{3.528637in}{2.998760in}}%
\pgfpathlineto{\pgfqpoint{3.536187in}{3.013107in}}%
\pgfpathlineto{\pgfqpoint{3.523342in}{3.025846in}}%
\pgfpathlineto{\pgfqpoint{3.510494in}{3.038820in}}%
\pgfpathlineto{\pgfqpoint{3.497645in}{3.052030in}}%
\pgfpathlineto{\pgfqpoint{3.484794in}{3.065478in}}%
\pgfpathlineto{\pgfqpoint{3.477237in}{3.050844in}}%
\pgfpathlineto{\pgfqpoint{3.469675in}{3.036378in}}%
\pgfpathlineto{\pgfqpoint{3.462107in}{3.022076in}}%
\pgfpathlineto{\pgfqpoint{3.454533in}{3.007936in}}%
\pgfpathclose%
\pgfusepath{fill}%
\end{pgfscope}%
\begin{pgfscope}%
\pgfpathrectangle{\pgfqpoint{1.254980in}{0.150000in}}{\pgfqpoint{5.490039in}{5.490039in}}%
\pgfusepath{clip}%
\pgfsetbuttcap%
\pgfsetroundjoin%
\definecolor{currentfill}{rgb}{0.182256,0.426184,0.557120}%
\pgfsetfillcolor{currentfill}%
\pgfsetfillopacity{0.700000}%
\pgfsetlinewidth{0.000000pt}%
\definecolor{currentstroke}{rgb}{0.000000,0.000000,0.000000}%
\pgfsetstrokecolor{currentstroke}%
\pgfsetdash{}{0pt}%
\pgfpathmoveto{\pgfqpoint{3.248376in}{3.252689in}}%
\pgfpathlineto{\pgfqpoint{3.261293in}{3.235431in}}%
\pgfpathlineto{\pgfqpoint{3.274205in}{3.218445in}}%
\pgfpathlineto{\pgfqpoint{3.287112in}{3.201728in}}%
\pgfpathlineto{\pgfqpoint{3.300013in}{3.185279in}}%
\pgfpathlineto{\pgfqpoint{3.307609in}{3.200222in}}%
\pgfpathlineto{\pgfqpoint{3.315199in}{3.215347in}}%
\pgfpathlineto{\pgfqpoint{3.322782in}{3.230657in}}%
\pgfpathlineto{\pgfqpoint{3.330360in}{3.246157in}}%
\pgfpathlineto{\pgfqpoint{3.317465in}{3.262886in}}%
\pgfpathlineto{\pgfqpoint{3.304566in}{3.279882in}}%
\pgfpathlineto{\pgfqpoint{3.291661in}{3.297149in}}%
\pgfpathlineto{\pgfqpoint{3.278751in}{3.314688in}}%
\pgfpathlineto{\pgfqpoint{3.271166in}{3.298897in}}%
\pgfpathlineto{\pgfqpoint{3.263576in}{3.283303in}}%
\pgfpathlineto{\pgfqpoint{3.255979in}{3.267901in}}%
\pgfpathlineto{\pgfqpoint{3.248376in}{3.252689in}}%
\pgfpathclose%
\pgfusepath{fill}%
\end{pgfscope}%
\begin{pgfscope}%
\pgfpathrectangle{\pgfqpoint{1.254980in}{0.150000in}}{\pgfqpoint{5.490039in}{5.490039in}}%
\pgfusepath{clip}%
\pgfsetbuttcap%
\pgfsetroundjoin%
\definecolor{currentfill}{rgb}{0.252194,0.269783,0.531579}%
\pgfsetfillcolor{currentfill}%
\pgfsetfillopacity{0.700000}%
\pgfsetlinewidth{0.000000pt}%
\definecolor{currentstroke}{rgb}{0.000000,0.000000,0.000000}%
\pgfsetstrokecolor{currentstroke}%
\pgfsetdash{}{0pt}%
\pgfpathmoveto{\pgfqpoint{4.037112in}{2.874526in}}%
\pgfpathlineto{\pgfqpoint{4.049984in}{2.867722in}}%
\pgfpathlineto{\pgfqpoint{4.062859in}{2.861107in}}%
\pgfpathlineto{\pgfqpoint{4.075739in}{2.854680in}}%
\pgfpathlineto{\pgfqpoint{4.088623in}{2.848442in}}%
\pgfpathlineto{\pgfqpoint{4.096057in}{2.861854in}}%
\pgfpathlineto{\pgfqpoint{4.103486in}{2.875421in}}%
\pgfpathlineto{\pgfqpoint{4.110913in}{2.889146in}}%
\pgfpathlineto{\pgfqpoint{4.118336in}{2.903034in}}%
\pgfpathlineto{\pgfqpoint{4.105459in}{2.909656in}}%
\pgfpathlineto{\pgfqpoint{4.092586in}{2.916465in}}%
\pgfpathlineto{\pgfqpoint{4.079718in}{2.923463in}}%
\pgfpathlineto{\pgfqpoint{4.066853in}{2.930651in}}%
\pgfpathlineto{\pgfqpoint{4.059423in}{2.916369in}}%
\pgfpathlineto{\pgfqpoint{4.051990in}{2.902258in}}%
\pgfpathlineto{\pgfqpoint{4.044553in}{2.888312in}}%
\pgfpathlineto{\pgfqpoint{4.037112in}{2.874526in}}%
\pgfpathclose%
\pgfusepath{fill}%
\end{pgfscope}%
\begin{pgfscope}%
\pgfpathrectangle{\pgfqpoint{1.254980in}{0.150000in}}{\pgfqpoint{5.490039in}{5.490039in}}%
\pgfusepath{clip}%
\pgfsetbuttcap%
\pgfsetroundjoin%
\definecolor{currentfill}{rgb}{0.185556,0.418570,0.556753}%
\pgfsetfillcolor{currentfill}%
\pgfsetfillopacity{0.700000}%
\pgfsetlinewidth{0.000000pt}%
\definecolor{currentstroke}{rgb}{0.000000,0.000000,0.000000}%
\pgfsetstrokecolor{currentstroke}%
\pgfsetdash{}{0pt}%
\pgfpathmoveto{\pgfqpoint{4.819631in}{3.199086in}}%
\pgfpathlineto{\pgfqpoint{4.832660in}{3.194825in}}%
\pgfpathlineto{\pgfqpoint{4.845697in}{3.190725in}}%
\pgfpathlineto{\pgfqpoint{4.858742in}{3.186787in}}%
\pgfpathlineto{\pgfqpoint{4.871795in}{3.183009in}}%
\pgfpathlineto{\pgfqpoint{4.879084in}{3.199157in}}%
\pgfpathlineto{\pgfqpoint{4.886375in}{3.215602in}}%
\pgfpathlineto{\pgfqpoint{4.893668in}{3.232350in}}%
\pgfpathlineto{\pgfqpoint{4.900964in}{3.249410in}}%
\pgfpathlineto{\pgfqpoint{4.887924in}{3.253818in}}%
\pgfpathlineto{\pgfqpoint{4.874892in}{3.258387in}}%
\pgfpathlineto{\pgfqpoint{4.861867in}{3.263118in}}%
\pgfpathlineto{\pgfqpoint{4.848850in}{3.268010in}}%
\pgfpathlineto{\pgfqpoint{4.841542in}{3.250309in}}%
\pgfpathlineto{\pgfqpoint{4.834236in}{3.232927in}}%
\pgfpathlineto{\pgfqpoint{4.826932in}{3.215855in}}%
\pgfpathlineto{\pgfqpoint{4.819631in}{3.199086in}}%
\pgfpathclose%
\pgfusepath{fill}%
\end{pgfscope}%
\begin{pgfscope}%
\pgfpathrectangle{\pgfqpoint{1.254980in}{0.150000in}}{\pgfqpoint{5.490039in}{5.490039in}}%
\pgfusepath{clip}%
\pgfsetbuttcap%
\pgfsetroundjoin%
\definecolor{currentfill}{rgb}{0.237441,0.305202,0.541921}%
\pgfsetfillcolor{currentfill}%
\pgfsetfillopacity{0.700000}%
\pgfsetlinewidth{0.000000pt}%
\definecolor{currentstroke}{rgb}{0.000000,0.000000,0.000000}%
\pgfsetstrokecolor{currentstroke}%
\pgfsetdash{}{0pt}%
\pgfpathmoveto{\pgfqpoint{4.332322in}{2.945270in}}%
\pgfpathlineto{\pgfqpoint{4.345249in}{2.940162in}}%
\pgfpathlineto{\pgfqpoint{4.358181in}{2.935231in}}%
\pgfpathlineto{\pgfqpoint{4.371120in}{2.930474in}}%
\pgfpathlineto{\pgfqpoint{4.384065in}{2.925893in}}%
\pgfpathlineto{\pgfqpoint{4.391430in}{2.939648in}}%
\pgfpathlineto{\pgfqpoint{4.398793in}{2.953590in}}%
\pgfpathlineto{\pgfqpoint{4.406154in}{2.967722in}}%
\pgfpathlineto{\pgfqpoint{4.413513in}{2.982051in}}%
\pgfpathlineto{\pgfqpoint{4.400577in}{2.987098in}}%
\pgfpathlineto{\pgfqpoint{4.387647in}{2.992319in}}%
\pgfpathlineto{\pgfqpoint{4.374723in}{2.997716in}}%
\pgfpathlineto{\pgfqpoint{4.361805in}{3.003289in}}%
\pgfpathlineto{\pgfqpoint{4.354437in}{2.988485in}}%
\pgfpathlineto{\pgfqpoint{4.347067in}{2.973884in}}%
\pgfpathlineto{\pgfqpoint{4.339696in}{2.959481in}}%
\pgfpathlineto{\pgfqpoint{4.332322in}{2.945270in}}%
\pgfpathclose%
\pgfusepath{fill}%
\end{pgfscope}%
\begin{pgfscope}%
\pgfpathrectangle{\pgfqpoint{1.254980in}{0.150000in}}{\pgfqpoint{5.490039in}{5.490039in}}%
\pgfusepath{clip}%
\pgfsetbuttcap%
\pgfsetroundjoin%
\definecolor{currentfill}{rgb}{0.253935,0.265254,0.529983}%
\pgfsetfillcolor{currentfill}%
\pgfsetfillopacity{0.700000}%
\pgfsetlinewidth{0.000000pt}%
\definecolor{currentstroke}{rgb}{0.000000,0.000000,0.000000}%
\pgfsetstrokecolor{currentstroke}%
\pgfsetdash{}{0pt}%
\pgfpathmoveto{\pgfqpoint{3.823082in}{2.858120in}}%
\pgfpathlineto{\pgfqpoint{3.835931in}{2.849645in}}%
\pgfpathlineto{\pgfqpoint{3.848782in}{2.841371in}}%
\pgfpathlineto{\pgfqpoint{3.861635in}{2.833299in}}%
\pgfpathlineto{\pgfqpoint{3.874491in}{2.825427in}}%
\pgfpathlineto{\pgfqpoint{3.881977in}{2.838715in}}%
\pgfpathlineto{\pgfqpoint{3.889459in}{2.852144in}}%
\pgfpathlineto{\pgfqpoint{3.896936in}{2.865718in}}%
\pgfpathlineto{\pgfqpoint{3.904410in}{2.879441in}}%
\pgfpathlineto{\pgfqpoint{3.891561in}{2.887642in}}%
\pgfpathlineto{\pgfqpoint{3.878715in}{2.896043in}}%
\pgfpathlineto{\pgfqpoint{3.865871in}{2.904645in}}%
\pgfpathlineto{\pgfqpoint{3.853029in}{2.913450in}}%
\pgfpathlineto{\pgfqpoint{3.845548in}{2.899387in}}%
\pgfpathlineto{\pgfqpoint{3.838064in}{2.885481in}}%
\pgfpathlineto{\pgfqpoint{3.830575in}{2.871727in}}%
\pgfpathlineto{\pgfqpoint{3.823082in}{2.858120in}}%
\pgfpathclose%
\pgfusepath{fill}%
\end{pgfscope}%
\begin{pgfscope}%
\pgfpathrectangle{\pgfqpoint{1.254980in}{0.150000in}}{\pgfqpoint{5.490039in}{5.490039in}}%
\pgfusepath{clip}%
\pgfsetbuttcap%
\pgfsetroundjoin%
\definecolor{currentfill}{rgb}{0.237441,0.305202,0.541921}%
\pgfsetfillcolor{currentfill}%
\pgfsetfillopacity{0.700000}%
\pgfsetlinewidth{0.000000pt}%
\definecolor{currentstroke}{rgb}{0.000000,0.000000,0.000000}%
\pgfsetstrokecolor{currentstroke}%
\pgfsetdash{}{0pt}%
\pgfpathmoveto{\pgfqpoint{3.505955in}{2.956670in}}%
\pgfpathlineto{\pgfqpoint{3.518806in}{2.944439in}}%
\pgfpathlineto{\pgfqpoint{3.531657in}{2.932440in}}%
\pgfpathlineto{\pgfqpoint{3.544506in}{2.920670in}}%
\pgfpathlineto{\pgfqpoint{3.557355in}{2.909128in}}%
\pgfpathlineto{\pgfqpoint{3.564913in}{2.922738in}}%
\pgfpathlineto{\pgfqpoint{3.572467in}{2.936495in}}%
\pgfpathlineto{\pgfqpoint{3.580015in}{2.950402in}}%
\pgfpathlineto{\pgfqpoint{3.587559in}{2.964463in}}%
\pgfpathlineto{\pgfqpoint{3.574717in}{2.976281in}}%
\pgfpathlineto{\pgfqpoint{3.561875in}{2.988326in}}%
\pgfpathlineto{\pgfqpoint{3.549032in}{3.000601in}}%
\pgfpathlineto{\pgfqpoint{3.536187in}{3.013107in}}%
\pgfpathlineto{\pgfqpoint{3.528637in}{2.998760in}}%
\pgfpathlineto{\pgfqpoint{3.521081in}{2.984574in}}%
\pgfpathlineto{\pgfqpoint{3.513521in}{2.970545in}}%
\pgfpathlineto{\pgfqpoint{3.505955in}{2.956670in}}%
\pgfpathclose%
\pgfusepath{fill}%
\end{pgfscope}%
\begin{pgfscope}%
\pgfpathrectangle{\pgfqpoint{1.254980in}{0.150000in}}{\pgfqpoint{5.490039in}{5.490039in}}%
\pgfusepath{clip}%
\pgfsetbuttcap%
\pgfsetroundjoin%
\definecolor{currentfill}{rgb}{0.136408,0.541173,0.554483}%
\pgfsetfillcolor{currentfill}%
\pgfsetfillopacity{0.700000}%
\pgfsetlinewidth{0.000000pt}%
\definecolor{currentstroke}{rgb}{0.000000,0.000000,0.000000}%
\pgfsetstrokecolor{currentstroke}%
\pgfsetdash{}{0pt}%
\pgfpathmoveto{\pgfqpoint{3.123322in}{3.547372in}}%
\pgfpathlineto{\pgfqpoint{3.136316in}{3.526354in}}%
\pgfpathlineto{\pgfqpoint{3.149302in}{3.505641in}}%
\pgfpathlineto{\pgfqpoint{3.162279in}{3.485232in}}%
\pgfpathlineto{\pgfqpoint{3.175248in}{3.465122in}}%
\pgfpathlineto{\pgfqpoint{3.182840in}{3.481713in}}%
\pgfpathlineto{\pgfqpoint{3.190426in}{3.498523in}}%
\pgfpathlineto{\pgfqpoint{3.198005in}{3.515555in}}%
\pgfpathlineto{\pgfqpoint{3.205578in}{3.532812in}}%
\pgfpathlineto{\pgfqpoint{3.192615in}{3.553236in}}%
\pgfpathlineto{\pgfqpoint{3.179644in}{3.573960in}}%
\pgfpathlineto{\pgfqpoint{3.166665in}{3.594987in}}%
\pgfpathlineto{\pgfqpoint{3.153678in}{3.616320in}}%
\pgfpathlineto{\pgfqpoint{3.146099in}{3.598737in}}%
\pgfpathlineto{\pgfqpoint{3.138513in}{3.581387in}}%
\pgfpathlineto{\pgfqpoint{3.130921in}{3.564266in}}%
\pgfpathlineto{\pgfqpoint{3.123322in}{3.547372in}}%
\pgfpathclose%
\pgfusepath{fill}%
\end{pgfscope}%
\begin{pgfscope}%
\pgfpathrectangle{\pgfqpoint{1.254980in}{0.150000in}}{\pgfqpoint{5.490039in}{5.490039in}}%
\pgfusepath{clip}%
\pgfsetbuttcap%
\pgfsetroundjoin%
\definecolor{currentfill}{rgb}{0.252194,0.269783,0.531579}%
\pgfsetfillcolor{currentfill}%
\pgfsetfillopacity{0.700000}%
\pgfsetlinewidth{0.000000pt}%
\definecolor{currentstroke}{rgb}{0.000000,0.000000,0.000000}%
\pgfsetstrokecolor{currentstroke}%
\pgfsetdash{}{0pt}%
\pgfpathmoveto{\pgfqpoint{3.690286in}{2.877930in}}%
\pgfpathlineto{\pgfqpoint{3.703129in}{2.868091in}}%
\pgfpathlineto{\pgfqpoint{3.715973in}{2.858466in}}%
\pgfpathlineto{\pgfqpoint{3.728818in}{2.849051in}}%
\pgfpathlineto{\pgfqpoint{3.741664in}{2.839846in}}%
\pgfpathlineto{\pgfqpoint{3.749182in}{2.853189in}}%
\pgfpathlineto{\pgfqpoint{3.756695in}{2.866672in}}%
\pgfpathlineto{\pgfqpoint{3.764204in}{2.880298in}}%
\pgfpathlineto{\pgfqpoint{3.771708in}{2.894072in}}%
\pgfpathlineto{\pgfqpoint{3.758868in}{2.903578in}}%
\pgfpathlineto{\pgfqpoint{3.746030in}{2.913295in}}%
\pgfpathlineto{\pgfqpoint{3.733193in}{2.923222in}}%
\pgfpathlineto{\pgfqpoint{3.720357in}{2.933363in}}%
\pgfpathlineto{\pgfqpoint{3.712846in}{2.919277in}}%
\pgfpathlineto{\pgfqpoint{3.705331in}{2.905345in}}%
\pgfpathlineto{\pgfqpoint{3.697811in}{2.891564in}}%
\pgfpathlineto{\pgfqpoint{3.690286in}{2.877930in}}%
\pgfpathclose%
\pgfusepath{fill}%
\end{pgfscope}%
\begin{pgfscope}%
\pgfpathrectangle{\pgfqpoint{1.254980in}{0.150000in}}{\pgfqpoint{5.490039in}{5.490039in}}%
\pgfusepath{clip}%
\pgfsetbuttcap%
\pgfsetroundjoin%
\definecolor{currentfill}{rgb}{0.169646,0.456262,0.558030}%
\pgfsetfillcolor{currentfill}%
\pgfsetfillopacity{0.700000}%
\pgfsetlinewidth{0.000000pt}%
\definecolor{currentstroke}{rgb}{0.000000,0.000000,0.000000}%
\pgfsetstrokecolor{currentstroke}%
\pgfsetdash{}{0pt}%
\pgfpathmoveto{\pgfqpoint{3.196648in}{3.324496in}}%
\pgfpathlineto{\pgfqpoint{3.209589in}{3.306123in}}%
\pgfpathlineto{\pgfqpoint{3.222524in}{3.288033in}}%
\pgfpathlineto{\pgfqpoint{3.235453in}{3.270222in}}%
\pgfpathlineto{\pgfqpoint{3.248376in}{3.252689in}}%
\pgfpathlineto{\pgfqpoint{3.255979in}{3.267901in}}%
\pgfpathlineto{\pgfqpoint{3.263576in}{3.283303in}}%
\pgfpathlineto{\pgfqpoint{3.271166in}{3.298897in}}%
\pgfpathlineto{\pgfqpoint{3.278751in}{3.314688in}}%
\pgfpathlineto{\pgfqpoint{3.265835in}{3.332502in}}%
\pgfpathlineto{\pgfqpoint{3.252914in}{3.350594in}}%
\pgfpathlineto{\pgfqpoint{3.239986in}{3.368966in}}%
\pgfpathlineto{\pgfqpoint{3.227052in}{3.387620in}}%
\pgfpathlineto{\pgfqpoint{3.219461in}{3.371537in}}%
\pgfpathlineto{\pgfqpoint{3.211863in}{3.355658in}}%
\pgfpathlineto{\pgfqpoint{3.204259in}{3.339979in}}%
\pgfpathlineto{\pgfqpoint{3.196648in}{3.324496in}}%
\pgfpathclose%
\pgfusepath{fill}%
\end{pgfscope}%
\begin{pgfscope}%
\pgfpathrectangle{\pgfqpoint{1.254980in}{0.150000in}}{\pgfqpoint{5.490039in}{5.490039in}}%
\pgfusepath{clip}%
\pgfsetbuttcap%
\pgfsetroundjoin%
\definecolor{currentfill}{rgb}{0.243113,0.292092,0.538516}%
\pgfsetfillcolor{currentfill}%
\pgfsetfillopacity{0.700000}%
\pgfsetlinewidth{0.000000pt}%
\definecolor{currentstroke}{rgb}{0.000000,0.000000,0.000000}%
\pgfsetstrokecolor{currentstroke}%
\pgfsetdash{}{0pt}%
\pgfpathmoveto{\pgfqpoint{4.251117in}{2.910706in}}%
\pgfpathlineto{\pgfqpoint{4.264029in}{2.905323in}}%
\pgfpathlineto{\pgfqpoint{4.276948in}{2.900120in}}%
\pgfpathlineto{\pgfqpoint{4.289871in}{2.895096in}}%
\pgfpathlineto{\pgfqpoint{4.302801in}{2.890249in}}%
\pgfpathlineto{\pgfqpoint{4.310185in}{2.903741in}}%
\pgfpathlineto{\pgfqpoint{4.317567in}{2.917406in}}%
\pgfpathlineto{\pgfqpoint{4.324945in}{2.931247in}}%
\pgfpathlineto{\pgfqpoint{4.332322in}{2.945270in}}%
\pgfpathlineto{\pgfqpoint{4.319401in}{2.950555in}}%
\pgfpathlineto{\pgfqpoint{4.306485in}{2.956018in}}%
\pgfpathlineto{\pgfqpoint{4.293575in}{2.961658in}}%
\pgfpathlineto{\pgfqpoint{4.280671in}{2.967478in}}%
\pgfpathlineto{\pgfqpoint{4.273286in}{2.953007in}}%
\pgfpathlineto{\pgfqpoint{4.265899in}{2.938725in}}%
\pgfpathlineto{\pgfqpoint{4.258509in}{2.924626in}}%
\pgfpathlineto{\pgfqpoint{4.251117in}{2.910706in}}%
\pgfpathclose%
\pgfusepath{fill}%
\end{pgfscope}%
\begin{pgfscope}%
\pgfpathrectangle{\pgfqpoint{1.254980in}{0.150000in}}{\pgfqpoint{5.490039in}{5.490039in}}%
\pgfusepath{clip}%
\pgfsetbuttcap%
\pgfsetroundjoin%
\definecolor{currentfill}{rgb}{0.177423,0.437527,0.557565}%
\pgfsetfillcolor{currentfill}%
\pgfsetfillopacity{0.700000}%
\pgfsetlinewidth{0.000000pt}%
\definecolor{currentstroke}{rgb}{0.000000,0.000000,0.000000}%
\pgfsetstrokecolor{currentstroke}%
\pgfsetdash{}{0pt}%
\pgfpathmoveto{\pgfqpoint{4.900964in}{3.249410in}}%
\pgfpathlineto{\pgfqpoint{4.914012in}{3.245162in}}%
\pgfpathlineto{\pgfqpoint{4.927067in}{3.241073in}}%
\pgfpathlineto{\pgfqpoint{4.940131in}{3.237144in}}%
\pgfpathlineto{\pgfqpoint{4.953204in}{3.233374in}}%
\pgfpathlineto{\pgfqpoint{4.960488in}{3.250104in}}%
\pgfpathlineto{\pgfqpoint{4.967776in}{3.267153in}}%
\pgfpathlineto{\pgfqpoint{4.975067in}{3.284529in}}%
\pgfpathlineto{\pgfqpoint{4.962006in}{3.288791in}}%
\pgfpathlineto{\pgfqpoint{4.948952in}{3.293211in}}%
\pgfpathlineto{\pgfqpoint{4.935906in}{3.297791in}}%
\pgfpathlineto{\pgfqpoint{4.922868in}{3.302530in}}%
\pgfpathlineto{\pgfqpoint{4.915564in}{3.284492in}}%
\pgfpathlineto{\pgfqpoint{4.908262in}{3.266788in}}%
\pgfpathlineto{\pgfqpoint{4.900964in}{3.249410in}}%
\pgfpathclose%
\pgfusepath{fill}%
\end{pgfscope}%
\begin{pgfscope}%
\pgfpathrectangle{\pgfqpoint{1.254980in}{0.150000in}}{\pgfqpoint{5.490039in}{5.490039in}}%
\pgfusepath{clip}%
\pgfsetbuttcap%
\pgfsetroundjoin%
\definecolor{currentfill}{rgb}{0.255645,0.260703,0.528312}%
\pgfsetfillcolor{currentfill}%
\pgfsetfillopacity{0.700000}%
\pgfsetlinewidth{0.000000pt}%
\definecolor{currentstroke}{rgb}{0.000000,0.000000,0.000000}%
\pgfsetstrokecolor{currentstroke}%
\pgfsetdash{}{0pt}%
\pgfpathmoveto{\pgfqpoint{3.955835in}{2.848617in}}%
\pgfpathlineto{\pgfqpoint{3.968699in}{2.841400in}}%
\pgfpathlineto{\pgfqpoint{3.981567in}{2.834377in}}%
\pgfpathlineto{\pgfqpoint{3.994438in}{2.827546in}}%
\pgfpathlineto{\pgfqpoint{4.007313in}{2.820907in}}%
\pgfpathlineto{\pgfqpoint{4.014768in}{2.834092in}}%
\pgfpathlineto{\pgfqpoint{4.022220in}{2.847421in}}%
\pgfpathlineto{\pgfqpoint{4.029668in}{2.860897in}}%
\pgfpathlineto{\pgfqpoint{4.037112in}{2.874526in}}%
\pgfpathlineto{\pgfqpoint{4.024244in}{2.881521in}}%
\pgfpathlineto{\pgfqpoint{4.011380in}{2.888708in}}%
\pgfpathlineto{\pgfqpoint{3.998520in}{2.896087in}}%
\pgfpathlineto{\pgfqpoint{3.985662in}{2.903660in}}%
\pgfpathlineto{\pgfqpoint{3.978211in}{2.889665in}}%
\pgfpathlineto{\pgfqpoint{3.970756in}{2.875829in}}%
\pgfpathlineto{\pgfqpoint{3.963297in}{2.862148in}}%
\pgfpathlineto{\pgfqpoint{3.955835in}{2.848617in}}%
\pgfpathclose%
\pgfusepath{fill}%
\end{pgfscope}%
\begin{pgfscope}%
\pgfpathrectangle{\pgfqpoint{1.254980in}{0.150000in}}{\pgfqpoint{5.490039in}{5.490039in}}%
\pgfusepath{clip}%
\pgfsetbuttcap%
\pgfsetroundjoin%
\definecolor{currentfill}{rgb}{0.244972,0.287675,0.537260}%
\pgfsetfillcolor{currentfill}%
\pgfsetfillopacity{0.700000}%
\pgfsetlinewidth{0.000000pt}%
\definecolor{currentstroke}{rgb}{0.000000,0.000000,0.000000}%
\pgfsetstrokecolor{currentstroke}%
\pgfsetdash{}{0pt}%
\pgfpathmoveto{\pgfqpoint{3.557355in}{2.909128in}}%
\pgfpathlineto{\pgfqpoint{3.570203in}{2.897811in}}%
\pgfpathlineto{\pgfqpoint{3.583051in}{2.886719in}}%
\pgfpathlineto{\pgfqpoint{3.595898in}{2.875850in}}%
\pgfpathlineto{\pgfqpoint{3.608746in}{2.865202in}}%
\pgfpathlineto{\pgfqpoint{3.616297in}{2.878548in}}%
\pgfpathlineto{\pgfqpoint{3.623843in}{2.892033in}}%
\pgfpathlineto{\pgfqpoint{3.631385in}{2.905662in}}%
\pgfpathlineto{\pgfqpoint{3.638921in}{2.919437in}}%
\pgfpathlineto{\pgfqpoint{3.626081in}{2.930360in}}%
\pgfpathlineto{\pgfqpoint{3.613240in}{2.941504in}}%
\pgfpathlineto{\pgfqpoint{3.600400in}{2.952871in}}%
\pgfpathlineto{\pgfqpoint{3.587559in}{2.964463in}}%
\pgfpathlineto{\pgfqpoint{3.580015in}{2.950402in}}%
\pgfpathlineto{\pgfqpoint{3.572467in}{2.936495in}}%
\pgfpathlineto{\pgfqpoint{3.564913in}{2.922738in}}%
\pgfpathlineto{\pgfqpoint{3.557355in}{2.909128in}}%
\pgfpathclose%
\pgfusepath{fill}%
\end{pgfscope}%
\begin{pgfscope}%
\pgfpathrectangle{\pgfqpoint{1.254980in}{0.150000in}}{\pgfqpoint{5.490039in}{5.490039in}}%
\pgfusepath{clip}%
\pgfsetbuttcap%
\pgfsetroundjoin%
\definecolor{currentfill}{rgb}{0.248629,0.278775,0.534556}%
\pgfsetfillcolor{currentfill}%
\pgfsetfillopacity{0.700000}%
\pgfsetlinewidth{0.000000pt}%
\definecolor{currentstroke}{rgb}{0.000000,0.000000,0.000000}%
\pgfsetstrokecolor{currentstroke}%
\pgfsetdash{}{0pt}%
\pgfpathmoveto{\pgfqpoint{4.169888in}{2.878406in}}%
\pgfpathlineto{\pgfqpoint{4.182787in}{2.872709in}}%
\pgfpathlineto{\pgfqpoint{4.195692in}{2.867195in}}%
\pgfpathlineto{\pgfqpoint{4.208602in}{2.861862in}}%
\pgfpathlineto{\pgfqpoint{4.221518in}{2.856711in}}%
\pgfpathlineto{\pgfqpoint{4.228922in}{2.869966in}}%
\pgfpathlineto{\pgfqpoint{4.236323in}{2.883381in}}%
\pgfpathlineto{\pgfqpoint{4.243721in}{2.896959in}}%
\pgfpathlineto{\pgfqpoint{4.251117in}{2.910706in}}%
\pgfpathlineto{\pgfqpoint{4.238209in}{2.916268in}}%
\pgfpathlineto{\pgfqpoint{4.225307in}{2.922011in}}%
\pgfpathlineto{\pgfqpoint{4.212410in}{2.927936in}}%
\pgfpathlineto{\pgfqpoint{4.199518in}{2.934043in}}%
\pgfpathlineto{\pgfqpoint{4.192115in}{2.919875in}}%
\pgfpathlineto{\pgfqpoint{4.184709in}{2.905883in}}%
\pgfpathlineto{\pgfqpoint{4.177300in}{2.892062in}}%
\pgfpathlineto{\pgfqpoint{4.169888in}{2.878406in}}%
\pgfpathclose%
\pgfusepath{fill}%
\end{pgfscope}%
\begin{pgfscope}%
\pgfpathrectangle{\pgfqpoint{1.254980in}{0.150000in}}{\pgfqpoint{5.490039in}{5.490039in}}%
\pgfusepath{clip}%
\pgfsetbuttcap%
\pgfsetroundjoin%
\definecolor{currentfill}{rgb}{0.157729,0.485932,0.558013}%
\pgfsetfillcolor{currentfill}%
\pgfsetfillopacity{0.700000}%
\pgfsetlinewidth{0.000000pt}%
\definecolor{currentstroke}{rgb}{0.000000,0.000000,0.000000}%
\pgfsetstrokecolor{currentstroke}%
\pgfsetdash{}{0pt}%
\pgfpathmoveto{\pgfqpoint{3.144814in}{3.400865in}}%
\pgfpathlineto{\pgfqpoint{3.157784in}{3.381336in}}%
\pgfpathlineto{\pgfqpoint{3.170745in}{3.362099in}}%
\pgfpathlineto{\pgfqpoint{3.183700in}{3.343154in}}%
\pgfpathlineto{\pgfqpoint{3.196648in}{3.324496in}}%
\pgfpathlineto{\pgfqpoint{3.204259in}{3.339979in}}%
\pgfpathlineto{\pgfqpoint{3.211863in}{3.355658in}}%
\pgfpathlineto{\pgfqpoint{3.219461in}{3.371537in}}%
\pgfpathlineto{\pgfqpoint{3.227052in}{3.387620in}}%
\pgfpathlineto{\pgfqpoint{3.214112in}{3.406561in}}%
\pgfpathlineto{\pgfqpoint{3.201165in}{3.425789in}}%
\pgfpathlineto{\pgfqpoint{3.188210in}{3.445309in}}%
\pgfpathlineto{\pgfqpoint{3.175248in}{3.465122in}}%
\pgfpathlineto{\pgfqpoint{3.167650in}{3.448745in}}%
\pgfpathlineto{\pgfqpoint{3.160045in}{3.432579in}}%
\pgfpathlineto{\pgfqpoint{3.152433in}{3.416620in}}%
\pgfpathlineto{\pgfqpoint{3.144814in}{3.400865in}}%
\pgfpathclose%
\pgfusepath{fill}%
\end{pgfscope}%
\begin{pgfscope}%
\pgfpathrectangle{\pgfqpoint{1.254980in}{0.150000in}}{\pgfqpoint{5.490039in}{5.490039in}}%
\pgfusepath{clip}%
\pgfsetbuttcap%
\pgfsetroundjoin%
\definecolor{currentfill}{rgb}{0.257322,0.256130,0.526563}%
\pgfsetfillcolor{currentfill}%
\pgfsetfillopacity{0.700000}%
\pgfsetlinewidth{0.000000pt}%
\definecolor{currentstroke}{rgb}{0.000000,0.000000,0.000000}%
\pgfsetstrokecolor{currentstroke}%
\pgfsetdash{}{0pt}%
\pgfpathmoveto{\pgfqpoint{3.741664in}{2.839846in}}%
\pgfpathlineto{\pgfqpoint{3.754512in}{2.830850in}}%
\pgfpathlineto{\pgfqpoint{3.767362in}{2.822062in}}%
\pgfpathlineto{\pgfqpoint{3.780213in}{2.813479in}}%
\pgfpathlineto{\pgfqpoint{3.793067in}{2.805101in}}%
\pgfpathlineto{\pgfqpoint{3.800577in}{2.818153in}}%
\pgfpathlineto{\pgfqpoint{3.808083in}{2.831337in}}%
\pgfpathlineto{\pgfqpoint{3.815585in}{2.844659in}}%
\pgfpathlineto{\pgfqpoint{3.823082in}{2.858120in}}%
\pgfpathlineto{\pgfqpoint{3.810236in}{2.866800in}}%
\pgfpathlineto{\pgfqpoint{3.797391in}{2.875684in}}%
\pgfpathlineto{\pgfqpoint{3.784549in}{2.884774in}}%
\pgfpathlineto{\pgfqpoint{3.771708in}{2.894072in}}%
\pgfpathlineto{\pgfqpoint{3.764204in}{2.880298in}}%
\pgfpathlineto{\pgfqpoint{3.756695in}{2.866672in}}%
\pgfpathlineto{\pgfqpoint{3.749182in}{2.853189in}}%
\pgfpathlineto{\pgfqpoint{3.741664in}{2.839846in}}%
\pgfpathclose%
\pgfusepath{fill}%
\end{pgfscope}%
\begin{pgfscope}%
\pgfpathrectangle{\pgfqpoint{1.254980in}{0.150000in}}{\pgfqpoint{5.490039in}{5.490039in}}%
\pgfusepath{clip}%
\pgfsetbuttcap%
\pgfsetroundjoin%
\definecolor{currentfill}{rgb}{0.253935,0.265254,0.529983}%
\pgfsetfillcolor{currentfill}%
\pgfsetfillopacity{0.700000}%
\pgfsetlinewidth{0.000000pt}%
\definecolor{currentstroke}{rgb}{0.000000,0.000000,0.000000}%
\pgfsetstrokecolor{currentstroke}%
\pgfsetdash{}{0pt}%
\pgfpathmoveto{\pgfqpoint{4.088623in}{2.848442in}}%
\pgfpathlineto{\pgfqpoint{4.101512in}{2.842390in}}%
\pgfpathlineto{\pgfqpoint{4.114405in}{2.836524in}}%
\pgfpathlineto{\pgfqpoint{4.127303in}{2.830843in}}%
\pgfpathlineto{\pgfqpoint{4.140206in}{2.825346in}}%
\pgfpathlineto{\pgfqpoint{4.147631in}{2.838386in}}%
\pgfpathlineto{\pgfqpoint{4.155053in}{2.851573in}}%
\pgfpathlineto{\pgfqpoint{4.162472in}{2.864911in}}%
\pgfpathlineto{\pgfqpoint{4.169888in}{2.878406in}}%
\pgfpathlineto{\pgfqpoint{4.156993in}{2.884286in}}%
\pgfpathlineto{\pgfqpoint{4.144102in}{2.890350in}}%
\pgfpathlineto{\pgfqpoint{4.131217in}{2.896599in}}%
\pgfpathlineto{\pgfqpoint{4.118336in}{2.903034in}}%
\pgfpathlineto{\pgfqpoint{4.110913in}{2.889146in}}%
\pgfpathlineto{\pgfqpoint{4.103486in}{2.875421in}}%
\pgfpathlineto{\pgfqpoint{4.096057in}{2.861854in}}%
\pgfpathlineto{\pgfqpoint{4.088623in}{2.848442in}}%
\pgfpathclose%
\pgfusepath{fill}%
\end{pgfscope}%
\begin{pgfscope}%
\pgfpathrectangle{\pgfqpoint{1.254980in}{0.150000in}}{\pgfqpoint{5.490039in}{5.490039in}}%
\pgfusepath{clip}%
\pgfsetbuttcap%
\pgfsetroundjoin%
\definecolor{currentfill}{rgb}{0.258965,0.251537,0.524736}%
\pgfsetfillcolor{currentfill}%
\pgfsetfillopacity{0.700000}%
\pgfsetlinewidth{0.000000pt}%
\definecolor{currentstroke}{rgb}{0.000000,0.000000,0.000000}%
\pgfsetstrokecolor{currentstroke}%
\pgfsetdash{}{0pt}%
\pgfpathmoveto{\pgfqpoint{3.874491in}{2.825427in}}%
\pgfpathlineto{\pgfqpoint{3.887350in}{2.817754in}}%
\pgfpathlineto{\pgfqpoint{3.900211in}{2.810279in}}%
\pgfpathlineto{\pgfqpoint{3.913076in}{2.803000in}}%
\pgfpathlineto{\pgfqpoint{3.925944in}{2.795918in}}%
\pgfpathlineto{\pgfqpoint{3.933423in}{2.808887in}}%
\pgfpathlineto{\pgfqpoint{3.940898in}{2.821991in}}%
\pgfpathlineto{\pgfqpoint{3.948368in}{2.835233in}}%
\pgfpathlineto{\pgfqpoint{3.955835in}{2.848617in}}%
\pgfpathlineto{\pgfqpoint{3.942974in}{2.856029in}}%
\pgfpathlineto{\pgfqpoint{3.930116in}{2.863636in}}%
\pgfpathlineto{\pgfqpoint{3.917262in}{2.871440in}}%
\pgfpathlineto{\pgfqpoint{3.904410in}{2.879441in}}%
\pgfpathlineto{\pgfqpoint{3.896936in}{2.865718in}}%
\pgfpathlineto{\pgfqpoint{3.889459in}{2.852144in}}%
\pgfpathlineto{\pgfqpoint{3.881977in}{2.838715in}}%
\pgfpathlineto{\pgfqpoint{3.874491in}{2.825427in}}%
\pgfpathclose%
\pgfusepath{fill}%
\end{pgfscope}%
\begin{pgfscope}%
\pgfpathrectangle{\pgfqpoint{1.254980in}{0.150000in}}{\pgfqpoint{5.490039in}{5.490039in}}%
\pgfusepath{clip}%
\pgfsetbuttcap%
\pgfsetroundjoin%
\definecolor{currentfill}{rgb}{0.214298,0.355619,0.551184}%
\pgfsetfillcolor{currentfill}%
\pgfsetfillopacity{0.700000}%
\pgfsetlinewidth{0.000000pt}%
\definecolor{currentstroke}{rgb}{0.000000,0.000000,0.000000}%
\pgfsetstrokecolor{currentstroke}%
\pgfsetdash{}{0pt}%
\pgfpathmoveto{\pgfqpoint{3.321166in}{3.065100in}}%
\pgfpathlineto{\pgfqpoint{3.334054in}{3.050204in}}%
\pgfpathlineto{\pgfqpoint{3.346939in}{3.035561in}}%
\pgfpathlineto{\pgfqpoint{3.359821in}{3.021170in}}%
\pgfpathlineto{\pgfqpoint{3.372699in}{3.007029in}}%
\pgfpathlineto{\pgfqpoint{3.380303in}{3.020796in}}%
\pgfpathlineto{\pgfqpoint{3.387900in}{3.034718in}}%
\pgfpathlineto{\pgfqpoint{3.395492in}{3.048799in}}%
\pgfpathlineto{\pgfqpoint{3.403078in}{3.063041in}}%
\pgfpathlineto{\pgfqpoint{3.390208in}{3.077432in}}%
\pgfpathlineto{\pgfqpoint{3.377334in}{3.092073in}}%
\pgfpathlineto{\pgfqpoint{3.364457in}{3.106965in}}%
\pgfpathlineto{\pgfqpoint{3.351576in}{3.122111in}}%
\pgfpathlineto{\pgfqpoint{3.343982in}{3.107609in}}%
\pgfpathlineto{\pgfqpoint{3.336383in}{3.093275in}}%
\pgfpathlineto{\pgfqpoint{3.328777in}{3.079106in}}%
\pgfpathlineto{\pgfqpoint{3.321166in}{3.065100in}}%
\pgfpathclose%
\pgfusepath{fill}%
\end{pgfscope}%
\begin{pgfscope}%
\pgfpathrectangle{\pgfqpoint{1.254980in}{0.150000in}}{\pgfqpoint{5.490039in}{5.490039in}}%
\pgfusepath{clip}%
\pgfsetbuttcap%
\pgfsetroundjoin%
\definecolor{currentfill}{rgb}{0.214298,0.355619,0.551184}%
\pgfsetfillcolor{currentfill}%
\pgfsetfillopacity{0.700000}%
\pgfsetlinewidth{0.000000pt}%
\definecolor{currentstroke}{rgb}{0.000000,0.000000,0.000000}%
\pgfsetstrokecolor{currentstroke}%
\pgfsetdash{}{0pt}%
\pgfpathmoveto{\pgfqpoint{4.627840in}{3.045121in}}%
\pgfpathlineto{\pgfqpoint{4.640844in}{3.041273in}}%
\pgfpathlineto{\pgfqpoint{4.653855in}{3.037592in}}%
\pgfpathlineto{\pgfqpoint{4.666874in}{3.034075in}}%
\pgfpathlineto{\pgfqpoint{4.679901in}{3.030724in}}%
\pgfpathlineto{\pgfqpoint{4.687208in}{3.044934in}}%
\pgfpathlineto{\pgfqpoint{4.694513in}{3.059372in}}%
\pgfpathlineto{\pgfqpoint{4.701819in}{3.074045in}}%
\pgfpathlineto{\pgfqpoint{4.709125in}{3.088960in}}%
\pgfpathlineto{\pgfqpoint{4.696110in}{3.092859in}}%
\pgfpathlineto{\pgfqpoint{4.683102in}{3.096923in}}%
\pgfpathlineto{\pgfqpoint{4.670102in}{3.101152in}}%
\pgfpathlineto{\pgfqpoint{4.657109in}{3.105548in}}%
\pgfpathlineto{\pgfqpoint{4.649792in}{3.090075in}}%
\pgfpathlineto{\pgfqpoint{4.642475in}{3.074851in}}%
\pgfpathlineto{\pgfqpoint{4.635158in}{3.059868in}}%
\pgfpathlineto{\pgfqpoint{4.627840in}{3.045121in}}%
\pgfpathclose%
\pgfusepath{fill}%
\end{pgfscope}%
\begin{pgfscope}%
\pgfpathrectangle{\pgfqpoint{1.254980in}{0.150000in}}{\pgfqpoint{5.490039in}{5.490039in}}%
\pgfusepath{clip}%
\pgfsetbuttcap%
\pgfsetroundjoin%
\definecolor{currentfill}{rgb}{0.221989,0.339161,0.548752}%
\pgfsetfillcolor{currentfill}%
\pgfsetfillopacity{0.700000}%
\pgfsetlinewidth{0.000000pt}%
\definecolor{currentstroke}{rgb}{0.000000,0.000000,0.000000}%
\pgfsetstrokecolor{currentstroke}%
\pgfsetdash{}{0pt}%
\pgfpathmoveto{\pgfqpoint{4.546575in}{3.003341in}}%
\pgfpathlineto{\pgfqpoint{4.559561in}{2.999344in}}%
\pgfpathlineto{\pgfqpoint{4.572553in}{2.995515in}}%
\pgfpathlineto{\pgfqpoint{4.585554in}{2.991855in}}%
\pgfpathlineto{\pgfqpoint{4.598561in}{2.988361in}}%
\pgfpathlineto{\pgfqpoint{4.605882in}{3.002229in}}%
\pgfpathlineto{\pgfqpoint{4.613202in}{3.016307in}}%
\pgfpathlineto{\pgfqpoint{4.620522in}{3.030603in}}%
\pgfpathlineto{\pgfqpoint{4.627840in}{3.045121in}}%
\pgfpathlineto{\pgfqpoint{4.614843in}{3.049135in}}%
\pgfpathlineto{\pgfqpoint{4.601854in}{3.053316in}}%
\pgfpathlineto{\pgfqpoint{4.588872in}{3.057665in}}%
\pgfpathlineto{\pgfqpoint{4.575896in}{3.062182in}}%
\pgfpathlineto{\pgfqpoint{4.568567in}{3.047133in}}%
\pgfpathlineto{\pgfqpoint{4.561238in}{3.032314in}}%
\pgfpathlineto{\pgfqpoint{4.553907in}{3.017719in}}%
\pgfpathlineto{\pgfqpoint{4.546575in}{3.003341in}}%
\pgfpathclose%
\pgfusepath{fill}%
\end{pgfscope}%
\begin{pgfscope}%
\pgfpathrectangle{\pgfqpoint{1.254980in}{0.150000in}}{\pgfqpoint{5.490039in}{5.490039in}}%
\pgfusepath{clip}%
\pgfsetbuttcap%
\pgfsetroundjoin%
\definecolor{currentfill}{rgb}{0.203063,0.379716,0.553925}%
\pgfsetfillcolor{currentfill}%
\pgfsetfillopacity{0.700000}%
\pgfsetlinewidth{0.000000pt}%
\definecolor{currentstroke}{rgb}{0.000000,0.000000,0.000000}%
\pgfsetstrokecolor{currentstroke}%
\pgfsetdash{}{0pt}%
\pgfpathmoveto{\pgfqpoint{3.269571in}{3.127267in}}%
\pgfpathlineto{\pgfqpoint{3.282476in}{3.111333in}}%
\pgfpathlineto{\pgfqpoint{3.295377in}{3.095663in}}%
\pgfpathlineto{\pgfqpoint{3.308273in}{3.080253in}}%
\pgfpathlineto{\pgfqpoint{3.321166in}{3.065100in}}%
\pgfpathlineto{\pgfqpoint{3.328777in}{3.079106in}}%
\pgfpathlineto{\pgfqpoint{3.336383in}{3.093275in}}%
\pgfpathlineto{\pgfqpoint{3.343982in}{3.107609in}}%
\pgfpathlineto{\pgfqpoint{3.351576in}{3.122111in}}%
\pgfpathlineto{\pgfqpoint{3.338692in}{3.137514in}}%
\pgfpathlineto{\pgfqpoint{3.325803in}{3.153174in}}%
\pgfpathlineto{\pgfqpoint{3.312911in}{3.169095in}}%
\pgfpathlineto{\pgfqpoint{3.300013in}{3.185279in}}%
\pgfpathlineto{\pgfqpoint{3.292412in}{3.170516in}}%
\pgfpathlineto{\pgfqpoint{3.284804in}{3.155928in}}%
\pgfpathlineto{\pgfqpoint{3.277191in}{3.141512in}}%
\pgfpathlineto{\pgfqpoint{3.269571in}{3.127267in}}%
\pgfpathclose%
\pgfusepath{fill}%
\end{pgfscope}%
\begin{pgfscope}%
\pgfpathrectangle{\pgfqpoint{1.254980in}{0.150000in}}{\pgfqpoint{5.490039in}{5.490039in}}%
\pgfusepath{clip}%
\pgfsetbuttcap%
\pgfsetroundjoin%
\definecolor{currentfill}{rgb}{0.204903,0.375746,0.553533}%
\pgfsetfillcolor{currentfill}%
\pgfsetfillopacity{0.700000}%
\pgfsetlinewidth{0.000000pt}%
\definecolor{currentstroke}{rgb}{0.000000,0.000000,0.000000}%
\pgfsetstrokecolor{currentstroke}%
\pgfsetdash{}{0pt}%
\pgfpathmoveto{\pgfqpoint{4.709125in}{3.088960in}}%
\pgfpathlineto{\pgfqpoint{4.722147in}{3.085225in}}%
\pgfpathlineto{\pgfqpoint{4.735178in}{3.081653in}}%
\pgfpathlineto{\pgfqpoint{4.748216in}{3.078246in}}%
\pgfpathlineto{\pgfqpoint{4.761263in}{3.075001in}}%
\pgfpathlineto{\pgfqpoint{4.768556in}{3.089598in}}%
\pgfpathlineto{\pgfqpoint{4.775850in}{3.104442in}}%
\pgfpathlineto{\pgfqpoint{4.783144in}{3.119541in}}%
\pgfpathlineto{\pgfqpoint{4.790439in}{3.134900in}}%
\pgfpathlineto{\pgfqpoint{4.777406in}{3.138721in}}%
\pgfpathlineto{\pgfqpoint{4.764379in}{3.142704in}}%
\pgfpathlineto{\pgfqpoint{4.751361in}{3.146851in}}%
\pgfpathlineto{\pgfqpoint{4.738350in}{3.151161in}}%
\pgfpathlineto{\pgfqpoint{4.731043in}{3.135216in}}%
\pgfpathlineto{\pgfqpoint{4.723736in}{3.119538in}}%
\pgfpathlineto{\pgfqpoint{4.716430in}{3.104122in}}%
\pgfpathlineto{\pgfqpoint{4.709125in}{3.088960in}}%
\pgfpathclose%
\pgfusepath{fill}%
\end{pgfscope}%
\begin{pgfscope}%
\pgfpathrectangle{\pgfqpoint{1.254980in}{0.150000in}}{\pgfqpoint{5.490039in}{5.490039in}}%
\pgfusepath{clip}%
\pgfsetbuttcap%
\pgfsetroundjoin%
\definecolor{currentfill}{rgb}{0.252194,0.269783,0.531579}%
\pgfsetfillcolor{currentfill}%
\pgfsetfillopacity{0.700000}%
\pgfsetlinewidth{0.000000pt}%
\definecolor{currentstroke}{rgb}{0.000000,0.000000,0.000000}%
\pgfsetstrokecolor{currentstroke}%
\pgfsetdash{}{0pt}%
\pgfpathmoveto{\pgfqpoint{3.608746in}{2.865202in}}%
\pgfpathlineto{\pgfqpoint{3.621593in}{2.854774in}}%
\pgfpathlineto{\pgfqpoint{3.634442in}{2.844564in}}%
\pgfpathlineto{\pgfqpoint{3.647290in}{2.834571in}}%
\pgfpathlineto{\pgfqpoint{3.660139in}{2.824793in}}%
\pgfpathlineto{\pgfqpoint{3.667683in}{2.837874in}}%
\pgfpathlineto{\pgfqpoint{3.675222in}{2.851088in}}%
\pgfpathlineto{\pgfqpoint{3.682756in}{2.864439in}}%
\pgfpathlineto{\pgfqpoint{3.690286in}{2.877930in}}%
\pgfpathlineto{\pgfqpoint{3.677444in}{2.887982in}}%
\pgfpathlineto{\pgfqpoint{3.664602in}{2.898250in}}%
\pgfpathlineto{\pgfqpoint{3.651762in}{2.908734in}}%
\pgfpathlineto{\pgfqpoint{3.638921in}{2.919437in}}%
\pgfpathlineto{\pgfqpoint{3.631385in}{2.905662in}}%
\pgfpathlineto{\pgfqpoint{3.623843in}{2.892033in}}%
\pgfpathlineto{\pgfqpoint{3.616297in}{2.878548in}}%
\pgfpathlineto{\pgfqpoint{3.608746in}{2.865202in}}%
\pgfpathclose%
\pgfusepath{fill}%
\end{pgfscope}%
\begin{pgfscope}%
\pgfpathrectangle{\pgfqpoint{1.254980in}{0.150000in}}{\pgfqpoint{5.490039in}{5.490039in}}%
\pgfusepath{clip}%
\pgfsetbuttcap%
\pgfsetroundjoin%
\definecolor{currentfill}{rgb}{0.225863,0.330805,0.547314}%
\pgfsetfillcolor{currentfill}%
\pgfsetfillopacity{0.700000}%
\pgfsetlinewidth{0.000000pt}%
\definecolor{currentstroke}{rgb}{0.000000,0.000000,0.000000}%
\pgfsetstrokecolor{currentstroke}%
\pgfsetdash{}{0pt}%
\pgfpathmoveto{\pgfqpoint{3.372699in}{3.007029in}}%
\pgfpathlineto{\pgfqpoint{3.385575in}{2.993135in}}%
\pgfpathlineto{\pgfqpoint{3.398447in}{2.979487in}}%
\pgfpathlineto{\pgfqpoint{3.411318in}{2.966082in}}%
\pgfpathlineto{\pgfqpoint{3.424186in}{2.952919in}}%
\pgfpathlineto{\pgfqpoint{3.431781in}{2.966447in}}%
\pgfpathlineto{\pgfqpoint{3.439370in}{2.980124in}}%
\pgfpathlineto{\pgfqpoint{3.446955in}{2.993953in}}%
\pgfpathlineto{\pgfqpoint{3.454533in}{3.007936in}}%
\pgfpathlineto{\pgfqpoint{3.441673in}{3.021347in}}%
\pgfpathlineto{\pgfqpoint{3.428811in}{3.035001in}}%
\pgfpathlineto{\pgfqpoint{3.415946in}{3.048898in}}%
\pgfpathlineto{\pgfqpoint{3.403078in}{3.063041in}}%
\pgfpathlineto{\pgfqpoint{3.395492in}{3.048799in}}%
\pgfpathlineto{\pgfqpoint{3.387900in}{3.034718in}}%
\pgfpathlineto{\pgfqpoint{3.380303in}{3.020796in}}%
\pgfpathlineto{\pgfqpoint{3.372699in}{3.007029in}}%
\pgfpathclose%
\pgfusepath{fill}%
\end{pgfscope}%
\begin{pgfscope}%
\pgfpathrectangle{\pgfqpoint{1.254980in}{0.150000in}}{\pgfqpoint{5.490039in}{5.490039in}}%
\pgfusepath{clip}%
\pgfsetbuttcap%
\pgfsetroundjoin%
\definecolor{currentfill}{rgb}{0.229739,0.322361,0.545706}%
\pgfsetfillcolor{currentfill}%
\pgfsetfillopacity{0.700000}%
\pgfsetlinewidth{0.000000pt}%
\definecolor{currentstroke}{rgb}{0.000000,0.000000,0.000000}%
\pgfsetstrokecolor{currentstroke}%
\pgfsetdash{}{0pt}%
\pgfpathmoveto{\pgfqpoint{4.465320in}{2.963598in}}%
\pgfpathlineto{\pgfqpoint{4.478288in}{2.959415in}}%
\pgfpathlineto{\pgfqpoint{4.491263in}{2.955402in}}%
\pgfpathlineto{\pgfqpoint{4.504245in}{2.951560in}}%
\pgfpathlineto{\pgfqpoint{4.517233in}{2.947887in}}%
\pgfpathlineto{\pgfqpoint{4.524571in}{2.961453in}}%
\pgfpathlineto{\pgfqpoint{4.531907in}{2.975214in}}%
\pgfpathlineto{\pgfqpoint{4.539242in}{2.989174in}}%
\pgfpathlineto{\pgfqpoint{4.546575in}{3.003341in}}%
\pgfpathlineto{\pgfqpoint{4.533596in}{3.007507in}}%
\pgfpathlineto{\pgfqpoint{4.520625in}{3.011842in}}%
\pgfpathlineto{\pgfqpoint{4.507660in}{3.016347in}}%
\pgfpathlineto{\pgfqpoint{4.494701in}{3.021023in}}%
\pgfpathlineto{\pgfqpoint{4.487358in}{3.006354in}}%
\pgfpathlineto{\pgfqpoint{4.480014in}{2.991897in}}%
\pgfpathlineto{\pgfqpoint{4.472668in}{2.977647in}}%
\pgfpathlineto{\pgfqpoint{4.465320in}{2.963598in}}%
\pgfpathclose%
\pgfusepath{fill}%
\end{pgfscope}%
\begin{pgfscope}%
\pgfpathrectangle{\pgfqpoint{1.254980in}{0.150000in}}{\pgfqpoint{5.490039in}{5.490039in}}%
\pgfusepath{clip}%
\pgfsetbuttcap%
\pgfsetroundjoin%
\definecolor{currentfill}{rgb}{0.195860,0.395433,0.555276}%
\pgfsetfillcolor{currentfill}%
\pgfsetfillopacity{0.700000}%
\pgfsetlinewidth{0.000000pt}%
\definecolor{currentstroke}{rgb}{0.000000,0.000000,0.000000}%
\pgfsetstrokecolor{currentstroke}%
\pgfsetdash{}{0pt}%
\pgfpathmoveto{\pgfqpoint{4.790439in}{3.134900in}}%
\pgfpathlineto{\pgfqpoint{4.803481in}{3.131242in}}%
\pgfpathlineto{\pgfqpoint{4.816531in}{3.127745in}}%
\pgfpathlineto{\pgfqpoint{4.829589in}{3.124410in}}%
\pgfpathlineto{\pgfqpoint{4.842656in}{3.121236in}}%
\pgfpathlineto{\pgfqpoint{4.849939in}{3.136270in}}%
\pgfpathlineto{\pgfqpoint{4.857223in}{3.151573in}}%
\pgfpathlineto{\pgfqpoint{4.864508in}{3.167150in}}%
\pgfpathlineto{\pgfqpoint{4.871795in}{3.183009in}}%
\pgfpathlineto{\pgfqpoint{4.858742in}{3.186787in}}%
\pgfpathlineto{\pgfqpoint{4.845697in}{3.190725in}}%
\pgfpathlineto{\pgfqpoint{4.832660in}{3.194825in}}%
\pgfpathlineto{\pgfqpoint{4.819631in}{3.199086in}}%
\pgfpathlineto{\pgfqpoint{4.812331in}{3.182614in}}%
\pgfpathlineto{\pgfqpoint{4.805032in}{3.166430in}}%
\pgfpathlineto{\pgfqpoint{4.797735in}{3.150528in}}%
\pgfpathlineto{\pgfqpoint{4.790439in}{3.134900in}}%
\pgfpathclose%
\pgfusepath{fill}%
\end{pgfscope}%
\begin{pgfscope}%
\pgfpathrectangle{\pgfqpoint{1.254980in}{0.150000in}}{\pgfqpoint{5.490039in}{5.490039in}}%
\pgfusepath{clip}%
\pgfsetbuttcap%
\pgfsetroundjoin%
\definecolor{currentfill}{rgb}{0.190631,0.407061,0.556089}%
\pgfsetfillcolor{currentfill}%
\pgfsetfillopacity{0.700000}%
\pgfsetlinewidth{0.000000pt}%
\definecolor{currentstroke}{rgb}{0.000000,0.000000,0.000000}%
\pgfsetstrokecolor{currentstroke}%
\pgfsetdash{}{0pt}%
\pgfpathmoveto{\pgfqpoint{3.217901in}{3.193671in}}%
\pgfpathlineto{\pgfqpoint{3.230826in}{3.176664in}}%
\pgfpathlineto{\pgfqpoint{3.243746in}{3.159930in}}%
\pgfpathlineto{\pgfqpoint{3.256661in}{3.143464in}}%
\pgfpathlineto{\pgfqpoint{3.269571in}{3.127267in}}%
\pgfpathlineto{\pgfqpoint{3.277191in}{3.141512in}}%
\pgfpathlineto{\pgfqpoint{3.284804in}{3.155928in}}%
\pgfpathlineto{\pgfqpoint{3.292412in}{3.170516in}}%
\pgfpathlineto{\pgfqpoint{3.300013in}{3.185279in}}%
\pgfpathlineto{\pgfqpoint{3.287112in}{3.201728in}}%
\pgfpathlineto{\pgfqpoint{3.274205in}{3.218445in}}%
\pgfpathlineto{\pgfqpoint{3.261293in}{3.235431in}}%
\pgfpathlineto{\pgfqpoint{3.248376in}{3.252689in}}%
\pgfpathlineto{\pgfqpoint{3.240766in}{3.237663in}}%
\pgfpathlineto{\pgfqpoint{3.233151in}{3.222820in}}%
\pgfpathlineto{\pgfqpoint{3.225529in}{3.208157in}}%
\pgfpathlineto{\pgfqpoint{3.217901in}{3.193671in}}%
\pgfpathclose%
\pgfusepath{fill}%
\end{pgfscope}%
\begin{pgfscope}%
\pgfpathrectangle{\pgfqpoint{1.254980in}{0.150000in}}{\pgfqpoint{5.490039in}{5.490039in}}%
\pgfusepath{clip}%
\pgfsetbuttcap%
\pgfsetroundjoin%
\definecolor{currentfill}{rgb}{0.237441,0.305202,0.541921}%
\pgfsetfillcolor{currentfill}%
\pgfsetfillopacity{0.700000}%
\pgfsetlinewidth{0.000000pt}%
\definecolor{currentstroke}{rgb}{0.000000,0.000000,0.000000}%
\pgfsetstrokecolor{currentstroke}%
\pgfsetdash{}{0pt}%
\pgfpathmoveto{\pgfqpoint{4.384065in}{2.925893in}}%
\pgfpathlineto{\pgfqpoint{4.397016in}{2.921485in}}%
\pgfpathlineto{\pgfqpoint{4.409974in}{2.917251in}}%
\pgfpathlineto{\pgfqpoint{4.422938in}{2.913190in}}%
\pgfpathlineto{\pgfqpoint{4.435909in}{2.909301in}}%
\pgfpathlineto{\pgfqpoint{4.443265in}{2.922601in}}%
\pgfpathlineto{\pgfqpoint{4.450619in}{2.936080in}}%
\pgfpathlineto{\pgfqpoint{4.457970in}{2.949744in}}%
\pgfpathlineto{\pgfqpoint{4.465320in}{2.963598in}}%
\pgfpathlineto{\pgfqpoint{4.452359in}{2.967952in}}%
\pgfpathlineto{\pgfqpoint{4.439404in}{2.972479in}}%
\pgfpathlineto{\pgfqpoint{4.426455in}{2.977179in}}%
\pgfpathlineto{\pgfqpoint{4.413513in}{2.982051in}}%
\pgfpathlineto{\pgfqpoint{4.406154in}{2.967722in}}%
\pgfpathlineto{\pgfqpoint{4.398793in}{2.953590in}}%
\pgfpathlineto{\pgfqpoint{4.391430in}{2.939648in}}%
\pgfpathlineto{\pgfqpoint{4.384065in}{2.925893in}}%
\pgfpathclose%
\pgfusepath{fill}%
\end{pgfscope}%
\begin{pgfscope}%
\pgfpathrectangle{\pgfqpoint{1.254980in}{0.150000in}}{\pgfqpoint{5.490039in}{5.490039in}}%
\pgfusepath{clip}%
\pgfsetbuttcap%
\pgfsetroundjoin%
\definecolor{currentfill}{rgb}{0.144759,0.519093,0.556572}%
\pgfsetfillcolor{currentfill}%
\pgfsetfillopacity{0.700000}%
\pgfsetlinewidth{0.000000pt}%
\definecolor{currentstroke}{rgb}{0.000000,0.000000,0.000000}%
\pgfsetstrokecolor{currentstroke}%
\pgfsetdash{}{0pt}%
\pgfpathmoveto{\pgfqpoint{3.092858in}{3.481977in}}%
\pgfpathlineto{\pgfqpoint{3.105860in}{3.461244in}}%
\pgfpathlineto{\pgfqpoint{3.118853in}{3.440817in}}%
\pgfpathlineto{\pgfqpoint{3.131837in}{3.420691in}}%
\pgfpathlineto{\pgfqpoint{3.144814in}{3.400865in}}%
\pgfpathlineto{\pgfqpoint{3.152433in}{3.416620in}}%
\pgfpathlineto{\pgfqpoint{3.160045in}{3.432579in}}%
\pgfpathlineto{\pgfqpoint{3.167650in}{3.448745in}}%
\pgfpathlineto{\pgfqpoint{3.175248in}{3.465122in}}%
\pgfpathlineto{\pgfqpoint{3.162279in}{3.485232in}}%
\pgfpathlineto{\pgfqpoint{3.149302in}{3.505641in}}%
\pgfpathlineto{\pgfqpoint{3.136316in}{3.526354in}}%
\pgfpathlineto{\pgfqpoint{3.123322in}{3.547372in}}%
\pgfpathlineto{\pgfqpoint{3.115717in}{3.530699in}}%
\pgfpathlineto{\pgfqpoint{3.108104in}{3.514244in}}%
\pgfpathlineto{\pgfqpoint{3.100485in}{3.498005in}}%
\pgfpathlineto{\pgfqpoint{3.092858in}{3.481977in}}%
\pgfpathclose%
\pgfusepath{fill}%
\end{pgfscope}%
\begin{pgfscope}%
\pgfpathrectangle{\pgfqpoint{1.254980in}{0.150000in}}{\pgfqpoint{5.490039in}{5.490039in}}%
\pgfusepath{clip}%
\pgfsetbuttcap%
\pgfsetroundjoin%
\definecolor{currentfill}{rgb}{0.235526,0.309527,0.542944}%
\pgfsetfillcolor{currentfill}%
\pgfsetfillopacity{0.700000}%
\pgfsetlinewidth{0.000000pt}%
\definecolor{currentstroke}{rgb}{0.000000,0.000000,0.000000}%
\pgfsetstrokecolor{currentstroke}%
\pgfsetdash{}{0pt}%
\pgfpathmoveto{\pgfqpoint{3.424186in}{2.952919in}}%
\pgfpathlineto{\pgfqpoint{3.437051in}{2.939995in}}%
\pgfpathlineto{\pgfqpoint{3.449915in}{2.927310in}}%
\pgfpathlineto{\pgfqpoint{3.462777in}{2.914861in}}%
\pgfpathlineto{\pgfqpoint{3.475638in}{2.902646in}}%
\pgfpathlineto{\pgfqpoint{3.483225in}{2.915937in}}%
\pgfpathlineto{\pgfqpoint{3.490807in}{2.929369in}}%
\pgfpathlineto{\pgfqpoint{3.498383in}{2.942946in}}%
\pgfpathlineto{\pgfqpoint{3.505955in}{2.956670in}}%
\pgfpathlineto{\pgfqpoint{3.493102in}{2.969133in}}%
\pgfpathlineto{\pgfqpoint{3.480247in}{2.981830in}}%
\pgfpathlineto{\pgfqpoint{3.467391in}{2.994764in}}%
\pgfpathlineto{\pgfqpoint{3.454533in}{3.007936in}}%
\pgfpathlineto{\pgfqpoint{3.446955in}{2.993953in}}%
\pgfpathlineto{\pgfqpoint{3.439370in}{2.980124in}}%
\pgfpathlineto{\pgfqpoint{3.431781in}{2.966447in}}%
\pgfpathlineto{\pgfqpoint{3.424186in}{2.952919in}}%
\pgfpathclose%
\pgfusepath{fill}%
\end{pgfscope}%
\begin{pgfscope}%
\pgfpathrectangle{\pgfqpoint{1.254980in}{0.150000in}}{\pgfqpoint{5.490039in}{5.490039in}}%
\pgfusepath{clip}%
\pgfsetbuttcap%
\pgfsetroundjoin%
\definecolor{currentfill}{rgb}{0.185556,0.418570,0.556753}%
\pgfsetfillcolor{currentfill}%
\pgfsetfillopacity{0.700000}%
\pgfsetlinewidth{0.000000pt}%
\definecolor{currentstroke}{rgb}{0.000000,0.000000,0.000000}%
\pgfsetstrokecolor{currentstroke}%
\pgfsetdash{}{0pt}%
\pgfpathmoveto{\pgfqpoint{4.871795in}{3.183009in}}%
\pgfpathlineto{\pgfqpoint{4.884856in}{3.179392in}}%
\pgfpathlineto{\pgfqpoint{4.897926in}{3.175934in}}%
\pgfpathlineto{\pgfqpoint{4.911004in}{3.172636in}}%
\pgfpathlineto{\pgfqpoint{4.924090in}{3.169497in}}%
\pgfpathlineto{\pgfqpoint{4.931365in}{3.185024in}}%
\pgfpathlineto{\pgfqpoint{4.938642in}{3.200841in}}%
\pgfpathlineto{\pgfqpoint{4.945922in}{3.216955in}}%
\pgfpathlineto{\pgfqpoint{4.953204in}{3.233374in}}%
\pgfpathlineto{\pgfqpoint{4.940131in}{3.237144in}}%
\pgfpathlineto{\pgfqpoint{4.927067in}{3.241073in}}%
\pgfpathlineto{\pgfqpoint{4.914012in}{3.245162in}}%
\pgfpathlineto{\pgfqpoint{4.900964in}{3.249410in}}%
\pgfpathlineto{\pgfqpoint{4.893668in}{3.232350in}}%
\pgfpathlineto{\pgfqpoint{4.886375in}{3.215602in}}%
\pgfpathlineto{\pgfqpoint{4.879084in}{3.199157in}}%
\pgfpathlineto{\pgfqpoint{4.871795in}{3.183009in}}%
\pgfpathclose%
\pgfusepath{fill}%
\end{pgfscope}%
\begin{pgfscope}%
\pgfpathrectangle{\pgfqpoint{1.254980in}{0.150000in}}{\pgfqpoint{5.490039in}{5.490039in}}%
\pgfusepath{clip}%
\pgfsetbuttcap%
\pgfsetroundjoin%
\definecolor{currentfill}{rgb}{0.257322,0.256130,0.526563}%
\pgfsetfillcolor{currentfill}%
\pgfsetfillopacity{0.700000}%
\pgfsetlinewidth{0.000000pt}%
\definecolor{currentstroke}{rgb}{0.000000,0.000000,0.000000}%
\pgfsetstrokecolor{currentstroke}%
\pgfsetdash{}{0pt}%
\pgfpathmoveto{\pgfqpoint{4.007313in}{2.820907in}}%
\pgfpathlineto{\pgfqpoint{4.020192in}{2.814458in}}%
\pgfpathlineto{\pgfqpoint{4.033075in}{2.808199in}}%
\pgfpathlineto{\pgfqpoint{4.045962in}{2.802128in}}%
\pgfpathlineto{\pgfqpoint{4.058854in}{2.796246in}}%
\pgfpathlineto{\pgfqpoint{4.066302in}{2.809085in}}%
\pgfpathlineto{\pgfqpoint{4.073746in}{2.822062in}}%
\pgfpathlineto{\pgfqpoint{4.081187in}{2.835179in}}%
\pgfpathlineto{\pgfqpoint{4.088623in}{2.848442in}}%
\pgfpathlineto{\pgfqpoint{4.075739in}{2.854680in}}%
\pgfpathlineto{\pgfqpoint{4.062859in}{2.861107in}}%
\pgfpathlineto{\pgfqpoint{4.049984in}{2.867722in}}%
\pgfpathlineto{\pgfqpoint{4.037112in}{2.874526in}}%
\pgfpathlineto{\pgfqpoint{4.029668in}{2.860897in}}%
\pgfpathlineto{\pgfqpoint{4.022220in}{2.847421in}}%
\pgfpathlineto{\pgfqpoint{4.014768in}{2.834092in}}%
\pgfpathlineto{\pgfqpoint{4.007313in}{2.820907in}}%
\pgfpathclose%
\pgfusepath{fill}%
\end{pgfscope}%
\begin{pgfscope}%
\pgfpathrectangle{\pgfqpoint{1.254980in}{0.150000in}}{\pgfqpoint{5.490039in}{5.490039in}}%
\pgfusepath{clip}%
\pgfsetbuttcap%
\pgfsetroundjoin%
\definecolor{currentfill}{rgb}{0.244972,0.287675,0.537260}%
\pgfsetfillcolor{currentfill}%
\pgfsetfillopacity{0.700000}%
\pgfsetlinewidth{0.000000pt}%
\definecolor{currentstroke}{rgb}{0.000000,0.000000,0.000000}%
\pgfsetstrokecolor{currentstroke}%
\pgfsetdash{}{0pt}%
\pgfpathmoveto{\pgfqpoint{4.302801in}{2.890249in}}%
\pgfpathlineto{\pgfqpoint{4.315737in}{2.885579in}}%
\pgfpathlineto{\pgfqpoint{4.328678in}{2.881085in}}%
\pgfpathlineto{\pgfqpoint{4.341626in}{2.876766in}}%
\pgfpathlineto{\pgfqpoint{4.354580in}{2.872623in}}%
\pgfpathlineto{\pgfqpoint{4.361955in}{2.885688in}}%
\pgfpathlineto{\pgfqpoint{4.369328in}{2.898918in}}%
\pgfpathlineto{\pgfqpoint{4.376698in}{2.912318in}}%
\pgfpathlineto{\pgfqpoint{4.384065in}{2.925893in}}%
\pgfpathlineto{\pgfqpoint{4.371120in}{2.930474in}}%
\pgfpathlineto{\pgfqpoint{4.358181in}{2.935231in}}%
\pgfpathlineto{\pgfqpoint{4.345249in}{2.940162in}}%
\pgfpathlineto{\pgfqpoint{4.332322in}{2.945270in}}%
\pgfpathlineto{\pgfqpoint{4.324945in}{2.931247in}}%
\pgfpathlineto{\pgfqpoint{4.317567in}{2.917406in}}%
\pgfpathlineto{\pgfqpoint{4.310185in}{2.903741in}}%
\pgfpathlineto{\pgfqpoint{4.302801in}{2.890249in}}%
\pgfpathclose%
\pgfusepath{fill}%
\end{pgfscope}%
\begin{pgfscope}%
\pgfpathrectangle{\pgfqpoint{1.254980in}{0.150000in}}{\pgfqpoint{5.490039in}{5.490039in}}%
\pgfusepath{clip}%
\pgfsetbuttcap%
\pgfsetroundjoin%
\definecolor{currentfill}{rgb}{0.177423,0.437527,0.557565}%
\pgfsetfillcolor{currentfill}%
\pgfsetfillopacity{0.700000}%
\pgfsetlinewidth{0.000000pt}%
\definecolor{currentstroke}{rgb}{0.000000,0.000000,0.000000}%
\pgfsetstrokecolor{currentstroke}%
\pgfsetdash{}{0pt}%
\pgfpathmoveto{\pgfqpoint{3.166140in}{3.264467in}}%
\pgfpathlineto{\pgfqpoint{3.179090in}{3.246348in}}%
\pgfpathlineto{\pgfqpoint{3.192033in}{3.228510in}}%
\pgfpathlineto{\pgfqpoint{3.204970in}{3.210952in}}%
\pgfpathlineto{\pgfqpoint{3.217901in}{3.193671in}}%
\pgfpathlineto{\pgfqpoint{3.225529in}{3.208157in}}%
\pgfpathlineto{\pgfqpoint{3.233151in}{3.222820in}}%
\pgfpathlineto{\pgfqpoint{3.240766in}{3.237663in}}%
\pgfpathlineto{\pgfqpoint{3.248376in}{3.252689in}}%
\pgfpathlineto{\pgfqpoint{3.235453in}{3.270222in}}%
\pgfpathlineto{\pgfqpoint{3.222524in}{3.288033in}}%
\pgfpathlineto{\pgfqpoint{3.209589in}{3.306123in}}%
\pgfpathlineto{\pgfqpoint{3.196648in}{3.324496in}}%
\pgfpathlineto{\pgfqpoint{3.189031in}{3.309207in}}%
\pgfpathlineto{\pgfqpoint{3.181407in}{3.294108in}}%
\pgfpathlineto{\pgfqpoint{3.173777in}{3.279196in}}%
\pgfpathlineto{\pgfqpoint{3.166140in}{3.264467in}}%
\pgfpathclose%
\pgfusepath{fill}%
\end{pgfscope}%
\begin{pgfscope}%
\pgfpathrectangle{\pgfqpoint{1.254980in}{0.150000in}}{\pgfqpoint{5.490039in}{5.490039in}}%
\pgfusepath{clip}%
\pgfsetbuttcap%
\pgfsetroundjoin%
\definecolor{currentfill}{rgb}{0.244972,0.287675,0.537260}%
\pgfsetfillcolor{currentfill}%
\pgfsetfillopacity{0.700000}%
\pgfsetlinewidth{0.000000pt}%
\definecolor{currentstroke}{rgb}{0.000000,0.000000,0.000000}%
\pgfsetstrokecolor{currentstroke}%
\pgfsetdash{}{0pt}%
\pgfpathmoveto{\pgfqpoint{3.475638in}{2.902646in}}%
\pgfpathlineto{\pgfqpoint{3.488497in}{2.890664in}}%
\pgfpathlineto{\pgfqpoint{3.501355in}{2.878912in}}%
\pgfpathlineto{\pgfqpoint{3.514212in}{2.867390in}}%
\pgfpathlineto{\pgfqpoint{3.527069in}{2.856095in}}%
\pgfpathlineto{\pgfqpoint{3.534648in}{2.869149in}}%
\pgfpathlineto{\pgfqpoint{3.542222in}{2.882336in}}%
\pgfpathlineto{\pgfqpoint{3.549791in}{2.895662in}}%
\pgfpathlineto{\pgfqpoint{3.557355in}{2.909128in}}%
\pgfpathlineto{\pgfqpoint{3.544506in}{2.920670in}}%
\pgfpathlineto{\pgfqpoint{3.531657in}{2.932440in}}%
\pgfpathlineto{\pgfqpoint{3.518806in}{2.944439in}}%
\pgfpathlineto{\pgfqpoint{3.505955in}{2.956670in}}%
\pgfpathlineto{\pgfqpoint{3.498383in}{2.942946in}}%
\pgfpathlineto{\pgfqpoint{3.490807in}{2.929369in}}%
\pgfpathlineto{\pgfqpoint{3.483225in}{2.915937in}}%
\pgfpathlineto{\pgfqpoint{3.475638in}{2.902646in}}%
\pgfpathclose%
\pgfusepath{fill}%
\end{pgfscope}%
\begin{pgfscope}%
\pgfpathrectangle{\pgfqpoint{1.254980in}{0.150000in}}{\pgfqpoint{5.490039in}{5.490039in}}%
\pgfusepath{clip}%
\pgfsetbuttcap%
\pgfsetroundjoin%
\definecolor{currentfill}{rgb}{0.260571,0.246922,0.522828}%
\pgfsetfillcolor{currentfill}%
\pgfsetfillopacity{0.700000}%
\pgfsetlinewidth{0.000000pt}%
\definecolor{currentstroke}{rgb}{0.000000,0.000000,0.000000}%
\pgfsetstrokecolor{currentstroke}%
\pgfsetdash{}{0pt}%
\pgfpathmoveto{\pgfqpoint{3.793067in}{2.805101in}}%
\pgfpathlineto{\pgfqpoint{3.805922in}{2.796927in}}%
\pgfpathlineto{\pgfqpoint{3.818781in}{2.788955in}}%
\pgfpathlineto{\pgfqpoint{3.831641in}{2.781184in}}%
\pgfpathlineto{\pgfqpoint{3.844504in}{2.773614in}}%
\pgfpathlineto{\pgfqpoint{3.852008in}{2.786374in}}%
\pgfpathlineto{\pgfqpoint{3.859506in}{2.799261in}}%
\pgfpathlineto{\pgfqpoint{3.867001in}{2.812277in}}%
\pgfpathlineto{\pgfqpoint{3.874491in}{2.825427in}}%
\pgfpathlineto{\pgfqpoint{3.861635in}{2.833299in}}%
\pgfpathlineto{\pgfqpoint{3.848782in}{2.841371in}}%
\pgfpathlineto{\pgfqpoint{3.835931in}{2.849645in}}%
\pgfpathlineto{\pgfqpoint{3.823082in}{2.858120in}}%
\pgfpathlineto{\pgfqpoint{3.815585in}{2.844659in}}%
\pgfpathlineto{\pgfqpoint{3.808083in}{2.831337in}}%
\pgfpathlineto{\pgfqpoint{3.800577in}{2.818153in}}%
\pgfpathlineto{\pgfqpoint{3.793067in}{2.805101in}}%
\pgfpathclose%
\pgfusepath{fill}%
\end{pgfscope}%
\begin{pgfscope}%
\pgfpathrectangle{\pgfqpoint{1.254980in}{0.150000in}}{\pgfqpoint{5.490039in}{5.490039in}}%
\pgfusepath{clip}%
\pgfsetbuttcap%
\pgfsetroundjoin%
\definecolor{currentfill}{rgb}{0.257322,0.256130,0.526563}%
\pgfsetfillcolor{currentfill}%
\pgfsetfillopacity{0.700000}%
\pgfsetlinewidth{0.000000pt}%
\definecolor{currentstroke}{rgb}{0.000000,0.000000,0.000000}%
\pgfsetstrokecolor{currentstroke}%
\pgfsetdash{}{0pt}%
\pgfpathmoveto{\pgfqpoint{3.660139in}{2.824793in}}%
\pgfpathlineto{\pgfqpoint{3.672989in}{2.815229in}}%
\pgfpathlineto{\pgfqpoint{3.685841in}{2.805878in}}%
\pgfpathlineto{\pgfqpoint{3.698693in}{2.796738in}}%
\pgfpathlineto{\pgfqpoint{3.711547in}{2.787808in}}%
\pgfpathlineto{\pgfqpoint{3.719083in}{2.800624in}}%
\pgfpathlineto{\pgfqpoint{3.726615in}{2.813568in}}%
\pgfpathlineto{\pgfqpoint{3.734142in}{2.826640in}}%
\pgfpathlineto{\pgfqpoint{3.741664in}{2.839846in}}%
\pgfpathlineto{\pgfqpoint{3.728818in}{2.849051in}}%
\pgfpathlineto{\pgfqpoint{3.715973in}{2.858466in}}%
\pgfpathlineto{\pgfqpoint{3.703129in}{2.868091in}}%
\pgfpathlineto{\pgfqpoint{3.690286in}{2.877930in}}%
\pgfpathlineto{\pgfqpoint{3.682756in}{2.864439in}}%
\pgfpathlineto{\pgfqpoint{3.675222in}{2.851088in}}%
\pgfpathlineto{\pgfqpoint{3.667683in}{2.837874in}}%
\pgfpathlineto{\pgfqpoint{3.660139in}{2.824793in}}%
\pgfpathclose%
\pgfusepath{fill}%
\end{pgfscope}%
\begin{pgfscope}%
\pgfpathrectangle{\pgfqpoint{1.254980in}{0.150000in}}{\pgfqpoint{5.490039in}{5.490039in}}%
\pgfusepath{clip}%
\pgfsetbuttcap%
\pgfsetroundjoin%
\definecolor{currentfill}{rgb}{0.177423,0.437527,0.557565}%
\pgfsetfillcolor{currentfill}%
\pgfsetfillopacity{0.700000}%
\pgfsetlinewidth{0.000000pt}%
\definecolor{currentstroke}{rgb}{0.000000,0.000000,0.000000}%
\pgfsetstrokecolor{currentstroke}%
\pgfsetdash{}{0pt}%
\pgfpathmoveto{\pgfqpoint{4.953204in}{3.233374in}}%
\pgfpathlineto{\pgfqpoint{4.966284in}{3.229762in}}%
\pgfpathlineto{\pgfqpoint{4.979373in}{3.226308in}}%
\pgfpathlineto{\pgfqpoint{4.992471in}{3.223012in}}%
\pgfpathlineto{\pgfqpoint{5.005578in}{3.219874in}}%
\pgfpathlineto{\pgfqpoint{5.012848in}{3.235955in}}%
\pgfpathlineto{\pgfqpoint{5.020121in}{3.252349in}}%
\pgfpathlineto{\pgfqpoint{5.027397in}{3.269063in}}%
\pgfpathlineto{\pgfqpoint{5.014302in}{3.272693in}}%
\pgfpathlineto{\pgfqpoint{5.001215in}{3.276481in}}%
\pgfpathlineto{\pgfqpoint{4.988137in}{3.280426in}}%
\pgfpathlineto{\pgfqpoint{4.975067in}{3.284529in}}%
\pgfpathlineto{\pgfqpoint{4.967776in}{3.267153in}}%
\pgfpathlineto{\pgfqpoint{4.960488in}{3.250104in}}%
\pgfpathlineto{\pgfqpoint{4.953204in}{3.233374in}}%
\pgfpathclose%
\pgfusepath{fill}%
\end{pgfscope}%
\begin{pgfscope}%
\pgfpathrectangle{\pgfqpoint{1.254980in}{0.150000in}}{\pgfqpoint{5.490039in}{5.490039in}}%
\pgfusepath{clip}%
\pgfsetbuttcap%
\pgfsetroundjoin%
\definecolor{currentfill}{rgb}{0.250425,0.274290,0.533103}%
\pgfsetfillcolor{currentfill}%
\pgfsetfillopacity{0.700000}%
\pgfsetlinewidth{0.000000pt}%
\definecolor{currentstroke}{rgb}{0.000000,0.000000,0.000000}%
\pgfsetstrokecolor{currentstroke}%
\pgfsetdash{}{0pt}%
\pgfpathmoveto{\pgfqpoint{4.221518in}{2.856711in}}%
\pgfpathlineto{\pgfqpoint{4.234439in}{2.851739in}}%
\pgfpathlineto{\pgfqpoint{4.247365in}{2.846946in}}%
\pgfpathlineto{\pgfqpoint{4.260298in}{2.842332in}}%
\pgfpathlineto{\pgfqpoint{4.273236in}{2.837896in}}%
\pgfpathlineto{\pgfqpoint{4.280632in}{2.850751in}}%
\pgfpathlineto{\pgfqpoint{4.288024in}{2.863758in}}%
\pgfpathlineto{\pgfqpoint{4.295414in}{2.876923in}}%
\pgfpathlineto{\pgfqpoint{4.302801in}{2.890249in}}%
\pgfpathlineto{\pgfqpoint{4.289871in}{2.895096in}}%
\pgfpathlineto{\pgfqpoint{4.276948in}{2.900120in}}%
\pgfpathlineto{\pgfqpoint{4.264029in}{2.905323in}}%
\pgfpathlineto{\pgfqpoint{4.251117in}{2.910706in}}%
\pgfpathlineto{\pgfqpoint{4.243721in}{2.896959in}}%
\pgfpathlineto{\pgfqpoint{4.236323in}{2.883381in}}%
\pgfpathlineto{\pgfqpoint{4.228922in}{2.869966in}}%
\pgfpathlineto{\pgfqpoint{4.221518in}{2.856711in}}%
\pgfpathclose%
\pgfusepath{fill}%
\end{pgfscope}%
\begin{pgfscope}%
\pgfpathrectangle{\pgfqpoint{1.254980in}{0.150000in}}{\pgfqpoint{5.490039in}{5.490039in}}%
\pgfusepath{clip}%
\pgfsetbuttcap%
\pgfsetroundjoin%
\definecolor{currentfill}{rgb}{0.165117,0.467423,0.558141}%
\pgfsetfillcolor{currentfill}%
\pgfsetfillopacity{0.700000}%
\pgfsetlinewidth{0.000000pt}%
\definecolor{currentstroke}{rgb}{0.000000,0.000000,0.000000}%
\pgfsetstrokecolor{currentstroke}%
\pgfsetdash{}{0pt}%
\pgfpathmoveto{\pgfqpoint{3.114273in}{3.339822in}}%
\pgfpathlineto{\pgfqpoint{3.127251in}{3.320546in}}%
\pgfpathlineto{\pgfqpoint{3.140221in}{3.301564in}}%
\pgfpathlineto{\pgfqpoint{3.153184in}{3.282872in}}%
\pgfpathlineto{\pgfqpoint{3.166140in}{3.264467in}}%
\pgfpathlineto{\pgfqpoint{3.173777in}{3.279196in}}%
\pgfpathlineto{\pgfqpoint{3.181407in}{3.294108in}}%
\pgfpathlineto{\pgfqpoint{3.189031in}{3.309207in}}%
\pgfpathlineto{\pgfqpoint{3.196648in}{3.324496in}}%
\pgfpathlineto{\pgfqpoint{3.183700in}{3.343154in}}%
\pgfpathlineto{\pgfqpoint{3.170745in}{3.362099in}}%
\pgfpathlineto{\pgfqpoint{3.157784in}{3.381336in}}%
\pgfpathlineto{\pgfqpoint{3.144814in}{3.400865in}}%
\pgfpathlineto{\pgfqpoint{3.137189in}{3.385312in}}%
\pgfpathlineto{\pgfqpoint{3.129557in}{3.369955in}}%
\pgfpathlineto{\pgfqpoint{3.121919in}{3.354793in}}%
\pgfpathlineto{\pgfqpoint{3.114273in}{3.339822in}}%
\pgfpathclose%
\pgfusepath{fill}%
\end{pgfscope}%
\begin{pgfscope}%
\pgfpathrectangle{\pgfqpoint{1.254980in}{0.150000in}}{\pgfqpoint{5.490039in}{5.490039in}}%
\pgfusepath{clip}%
\pgfsetbuttcap%
\pgfsetroundjoin%
\definecolor{currentfill}{rgb}{0.262138,0.242286,0.520837}%
\pgfsetfillcolor{currentfill}%
\pgfsetfillopacity{0.700000}%
\pgfsetlinewidth{0.000000pt}%
\definecolor{currentstroke}{rgb}{0.000000,0.000000,0.000000}%
\pgfsetstrokecolor{currentstroke}%
\pgfsetdash{}{0pt}%
\pgfpathmoveto{\pgfqpoint{3.925944in}{2.795918in}}%
\pgfpathlineto{\pgfqpoint{3.938816in}{2.789029in}}%
\pgfpathlineto{\pgfqpoint{3.951691in}{2.782334in}}%
\pgfpathlineto{\pgfqpoint{3.964569in}{2.775832in}}%
\pgfpathlineto{\pgfqpoint{3.977452in}{2.769522in}}%
\pgfpathlineto{\pgfqpoint{3.984923in}{2.782173in}}%
\pgfpathlineto{\pgfqpoint{3.992390in}{2.794951in}}%
\pgfpathlineto{\pgfqpoint{3.999854in}{2.807861in}}%
\pgfpathlineto{\pgfqpoint{4.007313in}{2.820907in}}%
\pgfpathlineto{\pgfqpoint{3.994438in}{2.827546in}}%
\pgfpathlineto{\pgfqpoint{3.981567in}{2.834377in}}%
\pgfpathlineto{\pgfqpoint{3.968699in}{2.841400in}}%
\pgfpathlineto{\pgfqpoint{3.955835in}{2.848617in}}%
\pgfpathlineto{\pgfqpoint{3.948368in}{2.835233in}}%
\pgfpathlineto{\pgfqpoint{3.940898in}{2.821991in}}%
\pgfpathlineto{\pgfqpoint{3.933423in}{2.808887in}}%
\pgfpathlineto{\pgfqpoint{3.925944in}{2.795918in}}%
\pgfpathclose%
\pgfusepath{fill}%
\end{pgfscope}%
\begin{pgfscope}%
\pgfpathrectangle{\pgfqpoint{1.254980in}{0.150000in}}{\pgfqpoint{5.490039in}{5.490039in}}%
\pgfusepath{clip}%
\pgfsetbuttcap%
\pgfsetroundjoin%
\definecolor{currentfill}{rgb}{0.252194,0.269783,0.531579}%
\pgfsetfillcolor{currentfill}%
\pgfsetfillopacity{0.700000}%
\pgfsetlinewidth{0.000000pt}%
\definecolor{currentstroke}{rgb}{0.000000,0.000000,0.000000}%
\pgfsetstrokecolor{currentstroke}%
\pgfsetdash{}{0pt}%
\pgfpathmoveto{\pgfqpoint{3.527069in}{2.856095in}}%
\pgfpathlineto{\pgfqpoint{3.539925in}{2.845026in}}%
\pgfpathlineto{\pgfqpoint{3.552780in}{2.834182in}}%
\pgfpathlineto{\pgfqpoint{3.565635in}{2.823560in}}%
\pgfpathlineto{\pgfqpoint{3.578491in}{2.813159in}}%
\pgfpathlineto{\pgfqpoint{3.586062in}{2.825976in}}%
\pgfpathlineto{\pgfqpoint{3.593628in}{2.838920in}}%
\pgfpathlineto{\pgfqpoint{3.601190in}{2.851994in}}%
\pgfpathlineto{\pgfqpoint{3.608746in}{2.865202in}}%
\pgfpathlineto{\pgfqpoint{3.595898in}{2.875850in}}%
\pgfpathlineto{\pgfqpoint{3.583051in}{2.886719in}}%
\pgfpathlineto{\pgfqpoint{3.570203in}{2.897811in}}%
\pgfpathlineto{\pgfqpoint{3.557355in}{2.909128in}}%
\pgfpathlineto{\pgfqpoint{3.549791in}{2.895662in}}%
\pgfpathlineto{\pgfqpoint{3.542222in}{2.882336in}}%
\pgfpathlineto{\pgfqpoint{3.534648in}{2.869149in}}%
\pgfpathlineto{\pgfqpoint{3.527069in}{2.856095in}}%
\pgfpathclose%
\pgfusepath{fill}%
\end{pgfscope}%
\begin{pgfscope}%
\pgfpathrectangle{\pgfqpoint{1.254980in}{0.150000in}}{\pgfqpoint{5.490039in}{5.490039in}}%
\pgfusepath{clip}%
\pgfsetbuttcap%
\pgfsetroundjoin%
\definecolor{currentfill}{rgb}{0.255645,0.260703,0.528312}%
\pgfsetfillcolor{currentfill}%
\pgfsetfillopacity{0.700000}%
\pgfsetlinewidth{0.000000pt}%
\definecolor{currentstroke}{rgb}{0.000000,0.000000,0.000000}%
\pgfsetstrokecolor{currentstroke}%
\pgfsetdash{}{0pt}%
\pgfpathmoveto{\pgfqpoint{4.140206in}{2.825346in}}%
\pgfpathlineto{\pgfqpoint{4.153114in}{2.820033in}}%
\pgfpathlineto{\pgfqpoint{4.166026in}{2.814902in}}%
\pgfpathlineto{\pgfqpoint{4.178945in}{2.809953in}}%
\pgfpathlineto{\pgfqpoint{4.191868in}{2.805184in}}%
\pgfpathlineto{\pgfqpoint{4.199286in}{2.817850in}}%
\pgfpathlineto{\pgfqpoint{4.206700in}{2.830657in}}%
\pgfpathlineto{\pgfqpoint{4.214110in}{2.843609in}}%
\pgfpathlineto{\pgfqpoint{4.221518in}{2.856711in}}%
\pgfpathlineto{\pgfqpoint{4.208602in}{2.861862in}}%
\pgfpathlineto{\pgfqpoint{4.195692in}{2.867195in}}%
\pgfpathlineto{\pgfqpoint{4.182787in}{2.872709in}}%
\pgfpathlineto{\pgfqpoint{4.169888in}{2.878406in}}%
\pgfpathlineto{\pgfqpoint{4.162472in}{2.864911in}}%
\pgfpathlineto{\pgfqpoint{4.155053in}{2.851573in}}%
\pgfpathlineto{\pgfqpoint{4.147631in}{2.838386in}}%
\pgfpathlineto{\pgfqpoint{4.140206in}{2.825346in}}%
\pgfpathclose%
\pgfusepath{fill}%
\end{pgfscope}%
\begin{pgfscope}%
\pgfpathrectangle{\pgfqpoint{1.254980in}{0.150000in}}{\pgfqpoint{5.490039in}{5.490039in}}%
\pgfusepath{clip}%
\pgfsetbuttcap%
\pgfsetroundjoin%
\definecolor{currentfill}{rgb}{0.212395,0.359683,0.551710}%
\pgfsetfillcolor{currentfill}%
\pgfsetfillopacity{0.700000}%
\pgfsetlinewidth{0.000000pt}%
\definecolor{currentstroke}{rgb}{0.000000,0.000000,0.000000}%
\pgfsetstrokecolor{currentstroke}%
\pgfsetdash{}{0pt}%
\pgfpathmoveto{\pgfqpoint{4.679901in}{3.030724in}}%
\pgfpathlineto{\pgfqpoint{4.692936in}{3.027537in}}%
\pgfpathlineto{\pgfqpoint{4.705978in}{3.024515in}}%
\pgfpathlineto{\pgfqpoint{4.719029in}{3.021655in}}%
\pgfpathlineto{\pgfqpoint{4.732088in}{3.018959in}}%
\pgfpathlineto{\pgfqpoint{4.739382in}{3.032630in}}%
\pgfpathlineto{\pgfqpoint{4.746676in}{3.046523in}}%
\pgfpathlineto{\pgfqpoint{4.753969in}{3.060645in}}%
\pgfpathlineto{\pgfqpoint{4.761263in}{3.075001in}}%
\pgfpathlineto{\pgfqpoint{4.748216in}{3.078246in}}%
\pgfpathlineto{\pgfqpoint{4.735178in}{3.081653in}}%
\pgfpathlineto{\pgfqpoint{4.722147in}{3.085225in}}%
\pgfpathlineto{\pgfqpoint{4.709125in}{3.088960in}}%
\pgfpathlineto{\pgfqpoint{4.701819in}{3.074045in}}%
\pgfpathlineto{\pgfqpoint{4.694513in}{3.059372in}}%
\pgfpathlineto{\pgfqpoint{4.687208in}{3.044934in}}%
\pgfpathlineto{\pgfqpoint{4.679901in}{3.030724in}}%
\pgfpathclose%
\pgfusepath{fill}%
\end{pgfscope}%
\begin{pgfscope}%
\pgfpathrectangle{\pgfqpoint{1.254980in}{0.150000in}}{\pgfqpoint{5.490039in}{5.490039in}}%
\pgfusepath{clip}%
\pgfsetbuttcap%
\pgfsetroundjoin%
\definecolor{currentfill}{rgb}{0.221989,0.339161,0.548752}%
\pgfsetfillcolor{currentfill}%
\pgfsetfillopacity{0.700000}%
\pgfsetlinewidth{0.000000pt}%
\definecolor{currentstroke}{rgb}{0.000000,0.000000,0.000000}%
\pgfsetstrokecolor{currentstroke}%
\pgfsetdash{}{0pt}%
\pgfpathmoveto{\pgfqpoint{4.598561in}{2.988361in}}%
\pgfpathlineto{\pgfqpoint{4.611576in}{2.985034in}}%
\pgfpathlineto{\pgfqpoint{4.624599in}{2.981873in}}%
\pgfpathlineto{\pgfqpoint{4.637629in}{2.978878in}}%
\pgfpathlineto{\pgfqpoint{4.650668in}{2.976047in}}%
\pgfpathlineto{\pgfqpoint{4.657977in}{2.989404in}}%
\pgfpathlineto{\pgfqpoint{4.665286in}{3.002966in}}%
\pgfpathlineto{\pgfqpoint{4.672594in}{3.016737in}}%
\pgfpathlineto{\pgfqpoint{4.679901in}{3.030724in}}%
\pgfpathlineto{\pgfqpoint{4.666874in}{3.034075in}}%
\pgfpathlineto{\pgfqpoint{4.653855in}{3.037592in}}%
\pgfpathlineto{\pgfqpoint{4.640844in}{3.041273in}}%
\pgfpathlineto{\pgfqpoint{4.627840in}{3.045121in}}%
\pgfpathlineto{\pgfqpoint{4.620522in}{3.030603in}}%
\pgfpathlineto{\pgfqpoint{4.613202in}{3.016307in}}%
\pgfpathlineto{\pgfqpoint{4.605882in}{3.002229in}}%
\pgfpathlineto{\pgfqpoint{4.598561in}{2.988361in}}%
\pgfpathclose%
\pgfusepath{fill}%
\end{pgfscope}%
\begin{pgfscope}%
\pgfpathrectangle{\pgfqpoint{1.254980in}{0.150000in}}{\pgfqpoint{5.490039in}{5.490039in}}%
\pgfusepath{clip}%
\pgfsetbuttcap%
\pgfsetroundjoin%
\definecolor{currentfill}{rgb}{0.262138,0.242286,0.520837}%
\pgfsetfillcolor{currentfill}%
\pgfsetfillopacity{0.700000}%
\pgfsetlinewidth{0.000000pt}%
\definecolor{currentstroke}{rgb}{0.000000,0.000000,0.000000}%
\pgfsetstrokecolor{currentstroke}%
\pgfsetdash{}{0pt}%
\pgfpathmoveto{\pgfqpoint{3.711547in}{2.787808in}}%
\pgfpathlineto{\pgfqpoint{3.724402in}{2.779086in}}%
\pgfpathlineto{\pgfqpoint{3.737259in}{2.770571in}}%
\pgfpathlineto{\pgfqpoint{3.750118in}{2.762263in}}%
\pgfpathlineto{\pgfqpoint{3.762979in}{2.754159in}}%
\pgfpathlineto{\pgfqpoint{3.770508in}{2.766712in}}%
\pgfpathlineto{\pgfqpoint{3.778032in}{2.779384in}}%
\pgfpathlineto{\pgfqpoint{3.785552in}{2.792180in}}%
\pgfpathlineto{\pgfqpoint{3.793067in}{2.805101in}}%
\pgfpathlineto{\pgfqpoint{3.780213in}{2.813479in}}%
\pgfpathlineto{\pgfqpoint{3.767362in}{2.822062in}}%
\pgfpathlineto{\pgfqpoint{3.754512in}{2.830850in}}%
\pgfpathlineto{\pgfqpoint{3.741664in}{2.839846in}}%
\pgfpathlineto{\pgfqpoint{3.734142in}{2.826640in}}%
\pgfpathlineto{\pgfqpoint{3.726615in}{2.813568in}}%
\pgfpathlineto{\pgfqpoint{3.719083in}{2.800624in}}%
\pgfpathlineto{\pgfqpoint{3.711547in}{2.787808in}}%
\pgfpathclose%
\pgfusepath{fill}%
\end{pgfscope}%
\begin{pgfscope}%
\pgfpathrectangle{\pgfqpoint{1.254980in}{0.150000in}}{\pgfqpoint{5.490039in}{5.490039in}}%
\pgfusepath{clip}%
\pgfsetbuttcap%
\pgfsetroundjoin%
\definecolor{currentfill}{rgb}{0.204903,0.375746,0.553533}%
\pgfsetfillcolor{currentfill}%
\pgfsetfillopacity{0.700000}%
\pgfsetlinewidth{0.000000pt}%
\definecolor{currentstroke}{rgb}{0.000000,0.000000,0.000000}%
\pgfsetstrokecolor{currentstroke}%
\pgfsetdash{}{0pt}%
\pgfpathmoveto{\pgfqpoint{4.761263in}{3.075001in}}%
\pgfpathlineto{\pgfqpoint{4.774317in}{3.071918in}}%
\pgfpathlineto{\pgfqpoint{4.787380in}{3.068997in}}%
\pgfpathlineto{\pgfqpoint{4.800451in}{3.066238in}}%
\pgfpathlineto{\pgfqpoint{4.813531in}{3.063640in}}%
\pgfpathlineto{\pgfqpoint{4.820811in}{3.077671in}}%
\pgfpathlineto{\pgfqpoint{4.828092in}{3.091943in}}%
\pgfpathlineto{\pgfqpoint{4.835374in}{3.106462in}}%
\pgfpathlineto{\pgfqpoint{4.842656in}{3.121236in}}%
\pgfpathlineto{\pgfqpoint{4.829589in}{3.124410in}}%
\pgfpathlineto{\pgfqpoint{4.816531in}{3.127745in}}%
\pgfpathlineto{\pgfqpoint{4.803481in}{3.131242in}}%
\pgfpathlineto{\pgfqpoint{4.790439in}{3.134900in}}%
\pgfpathlineto{\pgfqpoint{4.783144in}{3.119541in}}%
\pgfpathlineto{\pgfqpoint{4.775850in}{3.104442in}}%
\pgfpathlineto{\pgfqpoint{4.768556in}{3.089598in}}%
\pgfpathlineto{\pgfqpoint{4.761263in}{3.075001in}}%
\pgfpathclose%
\pgfusepath{fill}%
\end{pgfscope}%
\begin{pgfscope}%
\pgfpathrectangle{\pgfqpoint{1.254980in}{0.150000in}}{\pgfqpoint{5.490039in}{5.490039in}}%
\pgfusepath{clip}%
\pgfsetbuttcap%
\pgfsetroundjoin%
\definecolor{currentfill}{rgb}{0.221989,0.339161,0.548752}%
\pgfsetfillcolor{currentfill}%
\pgfsetfillopacity{0.700000}%
\pgfsetlinewidth{0.000000pt}%
\definecolor{currentstroke}{rgb}{0.000000,0.000000,0.000000}%
\pgfsetstrokecolor{currentstroke}%
\pgfsetdash{}{0pt}%
\pgfpathmoveto{\pgfqpoint{3.290660in}{3.010643in}}%
\pgfpathlineto{\pgfqpoint{3.303558in}{2.995968in}}%
\pgfpathlineto{\pgfqpoint{3.316451in}{2.981547in}}%
\pgfpathlineto{\pgfqpoint{3.329341in}{2.967377in}}%
\pgfpathlineto{\pgfqpoint{3.342228in}{2.953457in}}%
\pgfpathlineto{\pgfqpoint{3.349855in}{2.966631in}}%
\pgfpathlineto{\pgfqpoint{3.357475in}{2.979949in}}%
\pgfpathlineto{\pgfqpoint{3.365090in}{2.993414in}}%
\pgfpathlineto{\pgfqpoint{3.372699in}{3.007029in}}%
\pgfpathlineto{\pgfqpoint{3.359821in}{3.021170in}}%
\pgfpathlineto{\pgfqpoint{3.346939in}{3.035561in}}%
\pgfpathlineto{\pgfqpoint{3.334054in}{3.050204in}}%
\pgfpathlineto{\pgfqpoint{3.321166in}{3.065100in}}%
\pgfpathlineto{\pgfqpoint{3.313548in}{3.051254in}}%
\pgfpathlineto{\pgfqpoint{3.305925in}{3.037564in}}%
\pgfpathlineto{\pgfqpoint{3.298296in}{3.024028in}}%
\pgfpathlineto{\pgfqpoint{3.290660in}{3.010643in}}%
\pgfpathclose%
\pgfusepath{fill}%
\end{pgfscope}%
\begin{pgfscope}%
\pgfpathrectangle{\pgfqpoint{1.254980in}{0.150000in}}{\pgfqpoint{5.490039in}{5.490039in}}%
\pgfusepath{clip}%
\pgfsetbuttcap%
\pgfsetroundjoin%
\definecolor{currentfill}{rgb}{0.229739,0.322361,0.545706}%
\pgfsetfillcolor{currentfill}%
\pgfsetfillopacity{0.700000}%
\pgfsetlinewidth{0.000000pt}%
\definecolor{currentstroke}{rgb}{0.000000,0.000000,0.000000}%
\pgfsetstrokecolor{currentstroke}%
\pgfsetdash{}{0pt}%
\pgfpathmoveto{\pgfqpoint{4.517233in}{2.947887in}}%
\pgfpathlineto{\pgfqpoint{4.530230in}{2.944383in}}%
\pgfpathlineto{\pgfqpoint{4.543233in}{2.941048in}}%
\pgfpathlineto{\pgfqpoint{4.556244in}{2.937880in}}%
\pgfpathlineto{\pgfqpoint{4.569262in}{2.934880in}}%
\pgfpathlineto{\pgfqpoint{4.576589in}{2.947963in}}%
\pgfpathlineto{\pgfqpoint{4.583915in}{2.961234in}}%
\pgfpathlineto{\pgfqpoint{4.591239in}{2.974698in}}%
\pgfpathlineto{\pgfqpoint{4.598561in}{2.988361in}}%
\pgfpathlineto{\pgfqpoint{4.585554in}{2.991855in}}%
\pgfpathlineto{\pgfqpoint{4.572553in}{2.995515in}}%
\pgfpathlineto{\pgfqpoint{4.559561in}{2.999344in}}%
\pgfpathlineto{\pgfqpoint{4.546575in}{3.003341in}}%
\pgfpathlineto{\pgfqpoint{4.539242in}{2.989174in}}%
\pgfpathlineto{\pgfqpoint{4.531907in}{2.975214in}}%
\pgfpathlineto{\pgfqpoint{4.524571in}{2.961453in}}%
\pgfpathlineto{\pgfqpoint{4.517233in}{2.947887in}}%
\pgfpathclose%
\pgfusepath{fill}%
\end{pgfscope}%
\begin{pgfscope}%
\pgfpathrectangle{\pgfqpoint{1.254980in}{0.150000in}}{\pgfqpoint{5.490039in}{5.490039in}}%
\pgfusepath{clip}%
\pgfsetbuttcap%
\pgfsetroundjoin%
\definecolor{currentfill}{rgb}{0.210503,0.363727,0.552206}%
\pgfsetfillcolor{currentfill}%
\pgfsetfillopacity{0.700000}%
\pgfsetlinewidth{0.000000pt}%
\definecolor{currentstroke}{rgb}{0.000000,0.000000,0.000000}%
\pgfsetstrokecolor{currentstroke}%
\pgfsetdash{}{0pt}%
\pgfpathmoveto{\pgfqpoint{3.239031in}{3.071920in}}%
\pgfpathlineto{\pgfqpoint{3.251945in}{3.056210in}}%
\pgfpathlineto{\pgfqpoint{3.264854in}{3.040761in}}%
\pgfpathlineto{\pgfqpoint{3.277759in}{3.025573in}}%
\pgfpathlineto{\pgfqpoint{3.290660in}{3.010643in}}%
\pgfpathlineto{\pgfqpoint{3.298296in}{3.024028in}}%
\pgfpathlineto{\pgfqpoint{3.305925in}{3.037564in}}%
\pgfpathlineto{\pgfqpoint{3.313548in}{3.051254in}}%
\pgfpathlineto{\pgfqpoint{3.321166in}{3.065100in}}%
\pgfpathlineto{\pgfqpoint{3.308273in}{3.080253in}}%
\pgfpathlineto{\pgfqpoint{3.295377in}{3.095663in}}%
\pgfpathlineto{\pgfqpoint{3.282476in}{3.111333in}}%
\pgfpathlineto{\pgfqpoint{3.269571in}{3.127267in}}%
\pgfpathlineto{\pgfqpoint{3.261945in}{3.113187in}}%
\pgfpathlineto{\pgfqpoint{3.254313in}{3.099272in}}%
\pgfpathlineto{\pgfqpoint{3.246675in}{3.085517in}}%
\pgfpathlineto{\pgfqpoint{3.239031in}{3.071920in}}%
\pgfpathclose%
\pgfusepath{fill}%
\end{pgfscope}%
\begin{pgfscope}%
\pgfpathrectangle{\pgfqpoint{1.254980in}{0.150000in}}{\pgfqpoint{5.490039in}{5.490039in}}%
\pgfusepath{clip}%
\pgfsetbuttcap%
\pgfsetroundjoin%
\definecolor{currentfill}{rgb}{0.151918,0.500685,0.557587}%
\pgfsetfillcolor{currentfill}%
\pgfsetfillopacity{0.700000}%
\pgfsetlinewidth{0.000000pt}%
\definecolor{currentstroke}{rgb}{0.000000,0.000000,0.000000}%
\pgfsetstrokecolor{currentstroke}%
\pgfsetdash{}{0pt}%
\pgfpathmoveto{\pgfqpoint{3.062283in}{3.419912in}}%
\pgfpathlineto{\pgfqpoint{3.075293in}{3.399435in}}%
\pgfpathlineto{\pgfqpoint{3.088295in}{3.379263in}}%
\pgfpathlineto{\pgfqpoint{3.101288in}{3.359393in}}%
\pgfpathlineto{\pgfqpoint{3.114273in}{3.339822in}}%
\pgfpathlineto{\pgfqpoint{3.121919in}{3.354793in}}%
\pgfpathlineto{\pgfqpoint{3.129557in}{3.369955in}}%
\pgfpathlineto{\pgfqpoint{3.137189in}{3.385312in}}%
\pgfpathlineto{\pgfqpoint{3.144814in}{3.400865in}}%
\pgfpathlineto{\pgfqpoint{3.131837in}{3.420691in}}%
\pgfpathlineto{\pgfqpoint{3.118853in}{3.440817in}}%
\pgfpathlineto{\pgfqpoint{3.105860in}{3.461244in}}%
\pgfpathlineto{\pgfqpoint{3.092858in}{3.481977in}}%
\pgfpathlineto{\pgfqpoint{3.085225in}{3.466156in}}%
\pgfpathlineto{\pgfqpoint{3.077585in}{3.450541in}}%
\pgfpathlineto{\pgfqpoint{3.069937in}{3.435127in}}%
\pgfpathlineto{\pgfqpoint{3.062283in}{3.419912in}}%
\pgfpathclose%
\pgfusepath{fill}%
\end{pgfscope}%
\begin{pgfscope}%
\pgfpathrectangle{\pgfqpoint{1.254980in}{0.150000in}}{\pgfqpoint{5.490039in}{5.490039in}}%
\pgfusepath{clip}%
\pgfsetbuttcap%
\pgfsetroundjoin%
\definecolor{currentfill}{rgb}{0.260571,0.246922,0.522828}%
\pgfsetfillcolor{currentfill}%
\pgfsetfillopacity{0.700000}%
\pgfsetlinewidth{0.000000pt}%
\definecolor{currentstroke}{rgb}{0.000000,0.000000,0.000000}%
\pgfsetstrokecolor{currentstroke}%
\pgfsetdash{}{0pt}%
\pgfpathmoveto{\pgfqpoint{4.058854in}{2.796246in}}%
\pgfpathlineto{\pgfqpoint{4.071750in}{2.790550in}}%
\pgfpathlineto{\pgfqpoint{4.084651in}{2.785040in}}%
\pgfpathlineto{\pgfqpoint{4.097557in}{2.779715in}}%
\pgfpathlineto{\pgfqpoint{4.110468in}{2.774574in}}%
\pgfpathlineto{\pgfqpoint{4.117908in}{2.787068in}}%
\pgfpathlineto{\pgfqpoint{4.125344in}{2.799692in}}%
\pgfpathlineto{\pgfqpoint{4.132777in}{2.812450in}}%
\pgfpathlineto{\pgfqpoint{4.140206in}{2.825346in}}%
\pgfpathlineto{\pgfqpoint{4.127303in}{2.830843in}}%
\pgfpathlineto{\pgfqpoint{4.114405in}{2.836524in}}%
\pgfpathlineto{\pgfqpoint{4.101512in}{2.842390in}}%
\pgfpathlineto{\pgfqpoint{4.088623in}{2.848442in}}%
\pgfpathlineto{\pgfqpoint{4.081187in}{2.835179in}}%
\pgfpathlineto{\pgfqpoint{4.073746in}{2.822062in}}%
\pgfpathlineto{\pgfqpoint{4.066302in}{2.809085in}}%
\pgfpathlineto{\pgfqpoint{4.058854in}{2.796246in}}%
\pgfpathclose%
\pgfusepath{fill}%
\end{pgfscope}%
\begin{pgfscope}%
\pgfpathrectangle{\pgfqpoint{1.254980in}{0.150000in}}{\pgfqpoint{5.490039in}{5.490039in}}%
\pgfusepath{clip}%
\pgfsetbuttcap%
\pgfsetroundjoin%
\definecolor{currentfill}{rgb}{0.263663,0.237631,0.518762}%
\pgfsetfillcolor{currentfill}%
\pgfsetfillopacity{0.700000}%
\pgfsetlinewidth{0.000000pt}%
\definecolor{currentstroke}{rgb}{0.000000,0.000000,0.000000}%
\pgfsetstrokecolor{currentstroke}%
\pgfsetdash{}{0pt}%
\pgfpathmoveto{\pgfqpoint{3.844504in}{2.773614in}}%
\pgfpathlineto{\pgfqpoint{3.857370in}{2.766242in}}%
\pgfpathlineto{\pgfqpoint{3.870240in}{2.759068in}}%
\pgfpathlineto{\pgfqpoint{3.883112in}{2.752091in}}%
\pgfpathlineto{\pgfqpoint{3.895987in}{2.745310in}}%
\pgfpathlineto{\pgfqpoint{3.903483in}{2.757778in}}%
\pgfpathlineto{\pgfqpoint{3.910974in}{2.770367in}}%
\pgfpathlineto{\pgfqpoint{3.918461in}{2.783079in}}%
\pgfpathlineto{\pgfqpoint{3.925944in}{2.795918in}}%
\pgfpathlineto{\pgfqpoint{3.913076in}{2.803000in}}%
\pgfpathlineto{\pgfqpoint{3.900211in}{2.810279in}}%
\pgfpathlineto{\pgfqpoint{3.887350in}{2.817754in}}%
\pgfpathlineto{\pgfqpoint{3.874491in}{2.825427in}}%
\pgfpathlineto{\pgfqpoint{3.867001in}{2.812277in}}%
\pgfpathlineto{\pgfqpoint{3.859506in}{2.799261in}}%
\pgfpathlineto{\pgfqpoint{3.852008in}{2.786374in}}%
\pgfpathlineto{\pgfqpoint{3.844504in}{2.773614in}}%
\pgfpathclose%
\pgfusepath{fill}%
\end{pgfscope}%
\begin{pgfscope}%
\pgfpathrectangle{\pgfqpoint{1.254980in}{0.150000in}}{\pgfqpoint{5.490039in}{5.490039in}}%
\pgfusepath{clip}%
\pgfsetbuttcap%
\pgfsetroundjoin%
\definecolor{currentfill}{rgb}{0.195860,0.395433,0.555276}%
\pgfsetfillcolor{currentfill}%
\pgfsetfillopacity{0.700000}%
\pgfsetlinewidth{0.000000pt}%
\definecolor{currentstroke}{rgb}{0.000000,0.000000,0.000000}%
\pgfsetstrokecolor{currentstroke}%
\pgfsetdash{}{0pt}%
\pgfpathmoveto{\pgfqpoint{4.842656in}{3.121236in}}%
\pgfpathlineto{\pgfqpoint{4.855730in}{3.118222in}}%
\pgfpathlineto{\pgfqpoint{4.868814in}{3.115367in}}%
\pgfpathlineto{\pgfqpoint{4.881906in}{3.112673in}}%
\pgfpathlineto{\pgfqpoint{4.895006in}{3.110138in}}%
\pgfpathlineto{\pgfqpoint{4.902275in}{3.124579in}}%
\pgfpathlineto{\pgfqpoint{4.909546in}{3.139281in}}%
\pgfpathlineto{\pgfqpoint{4.916817in}{3.154251in}}%
\pgfpathlineto{\pgfqpoint{4.924090in}{3.169497in}}%
\pgfpathlineto{\pgfqpoint{4.911004in}{3.172636in}}%
\pgfpathlineto{\pgfqpoint{4.897926in}{3.175934in}}%
\pgfpathlineto{\pgfqpoint{4.884856in}{3.179392in}}%
\pgfpathlineto{\pgfqpoint{4.871795in}{3.183009in}}%
\pgfpathlineto{\pgfqpoint{4.864508in}{3.167150in}}%
\pgfpathlineto{\pgfqpoint{4.857223in}{3.151573in}}%
\pgfpathlineto{\pgfqpoint{4.849939in}{3.136270in}}%
\pgfpathlineto{\pgfqpoint{4.842656in}{3.121236in}}%
\pgfpathclose%
\pgfusepath{fill}%
\end{pgfscope}%
\begin{pgfscope}%
\pgfpathrectangle{\pgfqpoint{1.254980in}{0.150000in}}{\pgfqpoint{5.490039in}{5.490039in}}%
\pgfusepath{clip}%
\pgfsetbuttcap%
\pgfsetroundjoin%
\definecolor{currentfill}{rgb}{0.233603,0.313828,0.543914}%
\pgfsetfillcolor{currentfill}%
\pgfsetfillopacity{0.700000}%
\pgfsetlinewidth{0.000000pt}%
\definecolor{currentstroke}{rgb}{0.000000,0.000000,0.000000}%
\pgfsetstrokecolor{currentstroke}%
\pgfsetdash{}{0pt}%
\pgfpathmoveto{\pgfqpoint{3.342228in}{2.953457in}}%
\pgfpathlineto{\pgfqpoint{3.355112in}{2.939784in}}%
\pgfpathlineto{\pgfqpoint{3.367994in}{2.926357in}}%
\pgfpathlineto{\pgfqpoint{3.380872in}{2.913173in}}%
\pgfpathlineto{\pgfqpoint{3.393749in}{2.900230in}}%
\pgfpathlineto{\pgfqpoint{3.401366in}{2.913194in}}%
\pgfpathlineto{\pgfqpoint{3.408978in}{2.926295in}}%
\pgfpathlineto{\pgfqpoint{3.416585in}{2.939536in}}%
\pgfpathlineto{\pgfqpoint{3.424186in}{2.952919in}}%
\pgfpathlineto{\pgfqpoint{3.411318in}{2.966082in}}%
\pgfpathlineto{\pgfqpoint{3.398447in}{2.979487in}}%
\pgfpathlineto{\pgfqpoint{3.385575in}{2.993135in}}%
\pgfpathlineto{\pgfqpoint{3.372699in}{3.007029in}}%
\pgfpathlineto{\pgfqpoint{3.365090in}{2.993414in}}%
\pgfpathlineto{\pgfqpoint{3.357475in}{2.979949in}}%
\pgfpathlineto{\pgfqpoint{3.349855in}{2.966631in}}%
\pgfpathlineto{\pgfqpoint{3.342228in}{2.953457in}}%
\pgfpathclose%
\pgfusepath{fill}%
\end{pgfscope}%
\begin{pgfscope}%
\pgfpathrectangle{\pgfqpoint{1.254980in}{0.150000in}}{\pgfqpoint{5.490039in}{5.490039in}}%
\pgfusepath{clip}%
\pgfsetbuttcap%
\pgfsetroundjoin%
\definecolor{currentfill}{rgb}{0.237441,0.305202,0.541921}%
\pgfsetfillcolor{currentfill}%
\pgfsetfillopacity{0.700000}%
\pgfsetlinewidth{0.000000pt}%
\definecolor{currentstroke}{rgb}{0.000000,0.000000,0.000000}%
\pgfsetstrokecolor{currentstroke}%
\pgfsetdash{}{0pt}%
\pgfpathmoveto{\pgfqpoint{4.435909in}{2.909301in}}%
\pgfpathlineto{\pgfqpoint{4.448887in}{2.905583in}}%
\pgfpathlineto{\pgfqpoint{4.461872in}{2.902036in}}%
\pgfpathlineto{\pgfqpoint{4.474864in}{2.898660in}}%
\pgfpathlineto{\pgfqpoint{4.487863in}{2.895453in}}%
\pgfpathlineto{\pgfqpoint{4.495209in}{2.908298in}}%
\pgfpathlineto{\pgfqpoint{4.502552in}{2.921315in}}%
\pgfpathlineto{\pgfqpoint{4.509894in}{2.934509in}}%
\pgfpathlineto{\pgfqpoint{4.517233in}{2.947887in}}%
\pgfpathlineto{\pgfqpoint{4.504245in}{2.951560in}}%
\pgfpathlineto{\pgfqpoint{4.491263in}{2.955402in}}%
\pgfpathlineto{\pgfqpoint{4.478288in}{2.959415in}}%
\pgfpathlineto{\pgfqpoint{4.465320in}{2.963598in}}%
\pgfpathlineto{\pgfqpoint{4.457970in}{2.949744in}}%
\pgfpathlineto{\pgfqpoint{4.450619in}{2.936080in}}%
\pgfpathlineto{\pgfqpoint{4.443265in}{2.922601in}}%
\pgfpathlineto{\pgfqpoint{4.435909in}{2.909301in}}%
\pgfpathclose%
\pgfusepath{fill}%
\end{pgfscope}%
\begin{pgfscope}%
\pgfpathrectangle{\pgfqpoint{1.254980in}{0.150000in}}{\pgfqpoint{5.490039in}{5.490039in}}%
\pgfusepath{clip}%
\pgfsetbuttcap%
\pgfsetroundjoin%
\definecolor{currentfill}{rgb}{0.197636,0.391528,0.554969}%
\pgfsetfillcolor{currentfill}%
\pgfsetfillopacity{0.700000}%
\pgfsetlinewidth{0.000000pt}%
\definecolor{currentstroke}{rgb}{0.000000,0.000000,0.000000}%
\pgfsetstrokecolor{currentstroke}%
\pgfsetdash{}{0pt}%
\pgfpathmoveto{\pgfqpoint{3.187325in}{3.137433in}}%
\pgfpathlineto{\pgfqpoint{3.200259in}{3.120650in}}%
\pgfpathlineto{\pgfqpoint{3.213188in}{3.104138in}}%
\pgfpathlineto{\pgfqpoint{3.226112in}{3.087896in}}%
\pgfpathlineto{\pgfqpoint{3.239031in}{3.071920in}}%
\pgfpathlineto{\pgfqpoint{3.246675in}{3.085517in}}%
\pgfpathlineto{\pgfqpoint{3.254313in}{3.099272in}}%
\pgfpathlineto{\pgfqpoint{3.261945in}{3.113187in}}%
\pgfpathlineto{\pgfqpoint{3.269571in}{3.127267in}}%
\pgfpathlineto{\pgfqpoint{3.256661in}{3.143464in}}%
\pgfpathlineto{\pgfqpoint{3.243746in}{3.159930in}}%
\pgfpathlineto{\pgfqpoint{3.230826in}{3.176664in}}%
\pgfpathlineto{\pgfqpoint{3.217901in}{3.193671in}}%
\pgfpathlineto{\pgfqpoint{3.210267in}{3.179358in}}%
\pgfpathlineto{\pgfqpoint{3.202626in}{3.165216in}}%
\pgfpathlineto{\pgfqpoint{3.194979in}{3.151242in}}%
\pgfpathlineto{\pgfqpoint{3.187325in}{3.137433in}}%
\pgfpathclose%
\pgfusepath{fill}%
\end{pgfscope}%
\begin{pgfscope}%
\pgfpathrectangle{\pgfqpoint{1.254980in}{0.150000in}}{\pgfqpoint{5.490039in}{5.490039in}}%
\pgfusepath{clip}%
\pgfsetbuttcap%
\pgfsetroundjoin%
\definecolor{currentfill}{rgb}{0.258965,0.251537,0.524736}%
\pgfsetfillcolor{currentfill}%
\pgfsetfillopacity{0.700000}%
\pgfsetlinewidth{0.000000pt}%
\definecolor{currentstroke}{rgb}{0.000000,0.000000,0.000000}%
\pgfsetstrokecolor{currentstroke}%
\pgfsetdash{}{0pt}%
\pgfpathmoveto{\pgfqpoint{3.578491in}{2.813159in}}%
\pgfpathlineto{\pgfqpoint{3.591346in}{2.802978in}}%
\pgfpathlineto{\pgfqpoint{3.604202in}{2.793016in}}%
\pgfpathlineto{\pgfqpoint{3.617058in}{2.783270in}}%
\pgfpathlineto{\pgfqpoint{3.629915in}{2.773739in}}%
\pgfpathlineto{\pgfqpoint{3.637479in}{2.786318in}}%
\pgfpathlineto{\pgfqpoint{3.645037in}{2.799018in}}%
\pgfpathlineto{\pgfqpoint{3.652591in}{2.811842in}}%
\pgfpathlineto{\pgfqpoint{3.660139in}{2.824793in}}%
\pgfpathlineto{\pgfqpoint{3.647290in}{2.834571in}}%
\pgfpathlineto{\pgfqpoint{3.634442in}{2.844564in}}%
\pgfpathlineto{\pgfqpoint{3.621593in}{2.854774in}}%
\pgfpathlineto{\pgfqpoint{3.608746in}{2.865202in}}%
\pgfpathlineto{\pgfqpoint{3.601190in}{2.851994in}}%
\pgfpathlineto{\pgfqpoint{3.593628in}{2.838920in}}%
\pgfpathlineto{\pgfqpoint{3.586062in}{2.825976in}}%
\pgfpathlineto{\pgfqpoint{3.578491in}{2.813159in}}%
\pgfpathclose%
\pgfusepath{fill}%
\end{pgfscope}%
\begin{pgfscope}%
\pgfpathrectangle{\pgfqpoint{1.254980in}{0.150000in}}{\pgfqpoint{5.490039in}{5.490039in}}%
\pgfusepath{clip}%
\pgfsetbuttcap%
\pgfsetroundjoin%
\definecolor{currentfill}{rgb}{0.185556,0.418570,0.556753}%
\pgfsetfillcolor{currentfill}%
\pgfsetfillopacity{0.700000}%
\pgfsetlinewidth{0.000000pt}%
\definecolor{currentstroke}{rgb}{0.000000,0.000000,0.000000}%
\pgfsetstrokecolor{currentstroke}%
\pgfsetdash{}{0pt}%
\pgfpathmoveto{\pgfqpoint{4.924090in}{3.169497in}}%
\pgfpathlineto{\pgfqpoint{4.937185in}{3.166516in}}%
\pgfpathlineto{\pgfqpoint{4.950289in}{3.163694in}}%
\pgfpathlineto{\pgfqpoint{4.963402in}{3.161029in}}%
\pgfpathlineto{\pgfqpoint{4.976524in}{3.158522in}}%
\pgfpathlineto{\pgfqpoint{4.983784in}{3.173429in}}%
\pgfpathlineto{\pgfqpoint{4.991046in}{3.188618in}}%
\pgfpathlineto{\pgfqpoint{4.998311in}{3.204097in}}%
\pgfpathlineto{\pgfqpoint{5.005578in}{3.219874in}}%
\pgfpathlineto{\pgfqpoint{4.992471in}{3.223012in}}%
\pgfpathlineto{\pgfqpoint{4.979373in}{3.226308in}}%
\pgfpathlineto{\pgfqpoint{4.966284in}{3.229762in}}%
\pgfpathlineto{\pgfqpoint{4.953204in}{3.233374in}}%
\pgfpathlineto{\pgfqpoint{4.945922in}{3.216955in}}%
\pgfpathlineto{\pgfqpoint{4.938642in}{3.200841in}}%
\pgfpathlineto{\pgfqpoint{4.931365in}{3.185024in}}%
\pgfpathlineto{\pgfqpoint{4.924090in}{3.169497in}}%
\pgfpathclose%
\pgfusepath{fill}%
\end{pgfscope}%
\begin{pgfscope}%
\pgfpathrectangle{\pgfqpoint{1.254980in}{0.150000in}}{\pgfqpoint{5.490039in}{5.490039in}}%
\pgfusepath{clip}%
\pgfsetbuttcap%
\pgfsetroundjoin%
\definecolor{currentfill}{rgb}{0.244972,0.287675,0.537260}%
\pgfsetfillcolor{currentfill}%
\pgfsetfillopacity{0.700000}%
\pgfsetlinewidth{0.000000pt}%
\definecolor{currentstroke}{rgb}{0.000000,0.000000,0.000000}%
\pgfsetstrokecolor{currentstroke}%
\pgfsetdash{}{0pt}%
\pgfpathmoveto{\pgfqpoint{4.354580in}{2.872623in}}%
\pgfpathlineto{\pgfqpoint{4.367540in}{2.868654in}}%
\pgfpathlineto{\pgfqpoint{4.380507in}{2.864858in}}%
\pgfpathlineto{\pgfqpoint{4.393481in}{2.861235in}}%
\pgfpathlineto{\pgfqpoint{4.406461in}{2.857785in}}%
\pgfpathlineto{\pgfqpoint{4.413827in}{2.870421in}}%
\pgfpathlineto{\pgfqpoint{4.421190in}{2.883216in}}%
\pgfpathlineto{\pgfqpoint{4.428551in}{2.896174in}}%
\pgfpathlineto{\pgfqpoint{4.435909in}{2.909301in}}%
\pgfpathlineto{\pgfqpoint{4.422938in}{2.913190in}}%
\pgfpathlineto{\pgfqpoint{4.409974in}{2.917251in}}%
\pgfpathlineto{\pgfqpoint{4.397016in}{2.921485in}}%
\pgfpathlineto{\pgfqpoint{4.384065in}{2.925893in}}%
\pgfpathlineto{\pgfqpoint{4.376698in}{2.912318in}}%
\pgfpathlineto{\pgfqpoint{4.369328in}{2.898918in}}%
\pgfpathlineto{\pgfqpoint{4.361955in}{2.885688in}}%
\pgfpathlineto{\pgfqpoint{4.354580in}{2.872623in}}%
\pgfpathclose%
\pgfusepath{fill}%
\end{pgfscope}%
\begin{pgfscope}%
\pgfpathrectangle{\pgfqpoint{1.254980in}{0.150000in}}{\pgfqpoint{5.490039in}{5.490039in}}%
\pgfusepath{clip}%
\pgfsetbuttcap%
\pgfsetroundjoin%
\definecolor{currentfill}{rgb}{0.243113,0.292092,0.538516}%
\pgfsetfillcolor{currentfill}%
\pgfsetfillopacity{0.700000}%
\pgfsetlinewidth{0.000000pt}%
\definecolor{currentstroke}{rgb}{0.000000,0.000000,0.000000}%
\pgfsetstrokecolor{currentstroke}%
\pgfsetdash{}{0pt}%
\pgfpathmoveto{\pgfqpoint{3.393749in}{2.900230in}}%
\pgfpathlineto{\pgfqpoint{3.406623in}{2.887528in}}%
\pgfpathlineto{\pgfqpoint{3.419495in}{2.875063in}}%
\pgfpathlineto{\pgfqpoint{3.432366in}{2.862834in}}%
\pgfpathlineto{\pgfqpoint{3.445235in}{2.850840in}}%
\pgfpathlineto{\pgfqpoint{3.452844in}{2.863593in}}%
\pgfpathlineto{\pgfqpoint{3.460447in}{2.876477in}}%
\pgfpathlineto{\pgfqpoint{3.468045in}{2.889494in}}%
\pgfpathlineto{\pgfqpoint{3.475638in}{2.902646in}}%
\pgfpathlineto{\pgfqpoint{3.462777in}{2.914861in}}%
\pgfpathlineto{\pgfqpoint{3.449915in}{2.927310in}}%
\pgfpathlineto{\pgfqpoint{3.437051in}{2.939995in}}%
\pgfpathlineto{\pgfqpoint{3.424186in}{2.952919in}}%
\pgfpathlineto{\pgfqpoint{3.416585in}{2.939536in}}%
\pgfpathlineto{\pgfqpoint{3.408978in}{2.926295in}}%
\pgfpathlineto{\pgfqpoint{3.401366in}{2.913194in}}%
\pgfpathlineto{\pgfqpoint{3.393749in}{2.900230in}}%
\pgfpathclose%
\pgfusepath{fill}%
\end{pgfscope}%
\begin{pgfscope}%
\pgfpathrectangle{\pgfqpoint{1.254980in}{0.150000in}}{\pgfqpoint{5.490039in}{5.490039in}}%
\pgfusepath{clip}%
\pgfsetbuttcap%
\pgfsetroundjoin%
\definecolor{currentfill}{rgb}{0.185556,0.418570,0.556753}%
\pgfsetfillcolor{currentfill}%
\pgfsetfillopacity{0.700000}%
\pgfsetlinewidth{0.000000pt}%
\definecolor{currentstroke}{rgb}{0.000000,0.000000,0.000000}%
\pgfsetstrokecolor{currentstroke}%
\pgfsetdash{}{0pt}%
\pgfpathmoveto{\pgfqpoint{3.135528in}{3.207334in}}%
\pgfpathlineto{\pgfqpoint{3.148486in}{3.189439in}}%
\pgfpathlineto{\pgfqpoint{3.161439in}{3.171825in}}%
\pgfpathlineto{\pgfqpoint{3.174385in}{3.154491in}}%
\pgfpathlineto{\pgfqpoint{3.187325in}{3.137433in}}%
\pgfpathlineto{\pgfqpoint{3.194979in}{3.151242in}}%
\pgfpathlineto{\pgfqpoint{3.202626in}{3.165216in}}%
\pgfpathlineto{\pgfqpoint{3.210267in}{3.179358in}}%
\pgfpathlineto{\pgfqpoint{3.217901in}{3.193671in}}%
\pgfpathlineto{\pgfqpoint{3.204970in}{3.210952in}}%
\pgfpathlineto{\pgfqpoint{3.192033in}{3.228510in}}%
\pgfpathlineto{\pgfqpoint{3.179090in}{3.246348in}}%
\pgfpathlineto{\pgfqpoint{3.166140in}{3.264467in}}%
\pgfpathlineto{\pgfqpoint{3.158497in}{3.249920in}}%
\pgfpathlineto{\pgfqpoint{3.150847in}{3.235551in}}%
\pgfpathlineto{\pgfqpoint{3.143191in}{3.221356in}}%
\pgfpathlineto{\pgfqpoint{3.135528in}{3.207334in}}%
\pgfpathclose%
\pgfusepath{fill}%
\end{pgfscope}%
\begin{pgfscope}%
\pgfpathrectangle{\pgfqpoint{1.254980in}{0.150000in}}{\pgfqpoint{5.490039in}{5.490039in}}%
\pgfusepath{clip}%
\pgfsetbuttcap%
\pgfsetroundjoin%
\definecolor{currentfill}{rgb}{0.263663,0.237631,0.518762}%
\pgfsetfillcolor{currentfill}%
\pgfsetfillopacity{0.700000}%
\pgfsetlinewidth{0.000000pt}%
\definecolor{currentstroke}{rgb}{0.000000,0.000000,0.000000}%
\pgfsetstrokecolor{currentstroke}%
\pgfsetdash{}{0pt}%
\pgfpathmoveto{\pgfqpoint{3.977452in}{2.769522in}}%
\pgfpathlineto{\pgfqpoint{3.990339in}{2.763402in}}%
\pgfpathlineto{\pgfqpoint{4.003229in}{2.757471in}}%
\pgfpathlineto{\pgfqpoint{4.016124in}{2.751729in}}%
\pgfpathlineto{\pgfqpoint{4.029024in}{2.746175in}}%
\pgfpathlineto{\pgfqpoint{4.036487in}{2.758508in}}%
\pgfpathlineto{\pgfqpoint{4.043947in}{2.770961in}}%
\pgfpathlineto{\pgfqpoint{4.051402in}{2.783539in}}%
\pgfpathlineto{\pgfqpoint{4.058854in}{2.796246in}}%
\pgfpathlineto{\pgfqpoint{4.045962in}{2.802128in}}%
\pgfpathlineto{\pgfqpoint{4.033075in}{2.808199in}}%
\pgfpathlineto{\pgfqpoint{4.020192in}{2.814458in}}%
\pgfpathlineto{\pgfqpoint{4.007313in}{2.820907in}}%
\pgfpathlineto{\pgfqpoint{3.999854in}{2.807861in}}%
\pgfpathlineto{\pgfqpoint{3.992390in}{2.794951in}}%
\pgfpathlineto{\pgfqpoint{3.984923in}{2.782173in}}%
\pgfpathlineto{\pgfqpoint{3.977452in}{2.769522in}}%
\pgfpathclose%
\pgfusepath{fill}%
\end{pgfscope}%
\begin{pgfscope}%
\pgfpathrectangle{\pgfqpoint{1.254980in}{0.150000in}}{\pgfqpoint{5.490039in}{5.490039in}}%
\pgfusepath{clip}%
\pgfsetbuttcap%
\pgfsetroundjoin%
\definecolor{currentfill}{rgb}{0.250425,0.274290,0.533103}%
\pgfsetfillcolor{currentfill}%
\pgfsetfillopacity{0.700000}%
\pgfsetlinewidth{0.000000pt}%
\definecolor{currentstroke}{rgb}{0.000000,0.000000,0.000000}%
\pgfsetstrokecolor{currentstroke}%
\pgfsetdash{}{0pt}%
\pgfpathmoveto{\pgfqpoint{4.273236in}{2.837896in}}%
\pgfpathlineto{\pgfqpoint{4.286180in}{2.833637in}}%
\pgfpathlineto{\pgfqpoint{4.299130in}{2.829554in}}%
\pgfpathlineto{\pgfqpoint{4.312087in}{2.825646in}}%
\pgfpathlineto{\pgfqpoint{4.325050in}{2.821914in}}%
\pgfpathlineto{\pgfqpoint{4.332437in}{2.834368in}}%
\pgfpathlineto{\pgfqpoint{4.339821in}{2.846968in}}%
\pgfpathlineto{\pgfqpoint{4.347202in}{2.859718in}}%
\pgfpathlineto{\pgfqpoint{4.354580in}{2.872623in}}%
\pgfpathlineto{\pgfqpoint{4.341626in}{2.876766in}}%
\pgfpathlineto{\pgfqpoint{4.328678in}{2.881085in}}%
\pgfpathlineto{\pgfqpoint{4.315737in}{2.885579in}}%
\pgfpathlineto{\pgfqpoint{4.302801in}{2.890249in}}%
\pgfpathlineto{\pgfqpoint{4.295414in}{2.876923in}}%
\pgfpathlineto{\pgfqpoint{4.288024in}{2.863758in}}%
\pgfpathlineto{\pgfqpoint{4.280632in}{2.850751in}}%
\pgfpathlineto{\pgfqpoint{4.273236in}{2.837896in}}%
\pgfpathclose%
\pgfusepath{fill}%
\end{pgfscope}%
\begin{pgfscope}%
\pgfpathrectangle{\pgfqpoint{1.254980in}{0.150000in}}{\pgfqpoint{5.490039in}{5.490039in}}%
\pgfusepath{clip}%
\pgfsetbuttcap%
\pgfsetroundjoin%
\definecolor{currentfill}{rgb}{0.177423,0.437527,0.557565}%
\pgfsetfillcolor{currentfill}%
\pgfsetfillopacity{0.700000}%
\pgfsetlinewidth{0.000000pt}%
\definecolor{currentstroke}{rgb}{0.000000,0.000000,0.000000}%
\pgfsetstrokecolor{currentstroke}%
\pgfsetdash{}{0pt}%
\pgfpathmoveto{\pgfqpoint{5.005578in}{3.219874in}}%
\pgfpathlineto{\pgfqpoint{5.018693in}{3.216892in}}%
\pgfpathlineto{\pgfqpoint{5.031817in}{3.214066in}}%
\pgfpathlineto{\pgfqpoint{5.044951in}{3.211397in}}%
\pgfpathlineto{\pgfqpoint{5.058094in}{3.208884in}}%
\pgfpathlineto{\pgfqpoint{5.065348in}{3.224317in}}%
\pgfpathlineto{\pgfqpoint{5.072605in}{3.240055in}}%
\pgfpathlineto{\pgfqpoint{5.079865in}{3.256106in}}%
\pgfpathlineto{\pgfqpoint{5.066735in}{3.259111in}}%
\pgfpathlineto{\pgfqpoint{5.053613in}{3.262272in}}%
\pgfpathlineto{\pgfqpoint{5.040501in}{3.265590in}}%
\pgfpathlineto{\pgfqpoint{5.027397in}{3.269063in}}%
\pgfpathlineto{\pgfqpoint{5.020121in}{3.252349in}}%
\pgfpathlineto{\pgfqpoint{5.012848in}{3.235955in}}%
\pgfpathlineto{\pgfqpoint{5.005578in}{3.219874in}}%
\pgfpathclose%
\pgfusepath{fill}%
\end{pgfscope}%
\begin{pgfscope}%
\pgfpathrectangle{\pgfqpoint{1.254980in}{0.150000in}}{\pgfqpoint{5.490039in}{5.490039in}}%
\pgfusepath{clip}%
\pgfsetbuttcap%
\pgfsetroundjoin%
\definecolor{currentfill}{rgb}{0.250425,0.274290,0.533103}%
\pgfsetfillcolor{currentfill}%
\pgfsetfillopacity{0.700000}%
\pgfsetlinewidth{0.000000pt}%
\definecolor{currentstroke}{rgb}{0.000000,0.000000,0.000000}%
\pgfsetstrokecolor{currentstroke}%
\pgfsetdash{}{0pt}%
\pgfpathmoveto{\pgfqpoint{3.445235in}{2.850840in}}%
\pgfpathlineto{\pgfqpoint{3.458102in}{2.839077in}}%
\pgfpathlineto{\pgfqpoint{3.470969in}{2.827546in}}%
\pgfpathlineto{\pgfqpoint{3.483834in}{2.816244in}}%
\pgfpathlineto{\pgfqpoint{3.496699in}{2.805169in}}%
\pgfpathlineto{\pgfqpoint{3.504299in}{2.817713in}}%
\pgfpathlineto{\pgfqpoint{3.511894in}{2.830380in}}%
\pgfpathlineto{\pgfqpoint{3.519484in}{2.843174in}}%
\pgfpathlineto{\pgfqpoint{3.527069in}{2.856095in}}%
\pgfpathlineto{\pgfqpoint{3.514212in}{2.867390in}}%
\pgfpathlineto{\pgfqpoint{3.501355in}{2.878912in}}%
\pgfpathlineto{\pgfqpoint{3.488497in}{2.890664in}}%
\pgfpathlineto{\pgfqpoint{3.475638in}{2.902646in}}%
\pgfpathlineto{\pgfqpoint{3.468045in}{2.889494in}}%
\pgfpathlineto{\pgfqpoint{3.460447in}{2.876477in}}%
\pgfpathlineto{\pgfqpoint{3.452844in}{2.863593in}}%
\pgfpathlineto{\pgfqpoint{3.445235in}{2.850840in}}%
\pgfpathclose%
\pgfusepath{fill}%
\end{pgfscope}%
\begin{pgfscope}%
\pgfpathrectangle{\pgfqpoint{1.254980in}{0.150000in}}{\pgfqpoint{5.490039in}{5.490039in}}%
\pgfusepath{clip}%
\pgfsetbuttcap%
\pgfsetroundjoin%
\definecolor{currentfill}{rgb}{0.266580,0.228262,0.514349}%
\pgfsetfillcolor{currentfill}%
\pgfsetfillopacity{0.700000}%
\pgfsetlinewidth{0.000000pt}%
\definecolor{currentstroke}{rgb}{0.000000,0.000000,0.000000}%
\pgfsetstrokecolor{currentstroke}%
\pgfsetdash{}{0pt}%
\pgfpathmoveto{\pgfqpoint{3.762979in}{2.754159in}}%
\pgfpathlineto{\pgfqpoint{3.775842in}{2.746259in}}%
\pgfpathlineto{\pgfqpoint{3.788708in}{2.738561in}}%
\pgfpathlineto{\pgfqpoint{3.801576in}{2.731065in}}%
\pgfpathlineto{\pgfqpoint{3.814447in}{2.723768in}}%
\pgfpathlineto{\pgfqpoint{3.821968in}{2.736057in}}%
\pgfpathlineto{\pgfqpoint{3.829485in}{2.748459in}}%
\pgfpathlineto{\pgfqpoint{3.836997in}{2.760977in}}%
\pgfpathlineto{\pgfqpoint{3.844504in}{2.773614in}}%
\pgfpathlineto{\pgfqpoint{3.831641in}{2.781184in}}%
\pgfpathlineto{\pgfqpoint{3.818781in}{2.788955in}}%
\pgfpathlineto{\pgfqpoint{3.805922in}{2.796927in}}%
\pgfpathlineto{\pgfqpoint{3.793067in}{2.805101in}}%
\pgfpathlineto{\pgfqpoint{3.785552in}{2.792180in}}%
\pgfpathlineto{\pgfqpoint{3.778032in}{2.779384in}}%
\pgfpathlineto{\pgfqpoint{3.770508in}{2.766712in}}%
\pgfpathlineto{\pgfqpoint{3.762979in}{2.754159in}}%
\pgfpathclose%
\pgfusepath{fill}%
\end{pgfscope}%
\begin{pgfscope}%
\pgfpathrectangle{\pgfqpoint{1.254980in}{0.150000in}}{\pgfqpoint{5.490039in}{5.490039in}}%
\pgfusepath{clip}%
\pgfsetbuttcap%
\pgfsetroundjoin%
\definecolor{currentfill}{rgb}{0.171176,0.452530,0.557965}%
\pgfsetfillcolor{currentfill}%
\pgfsetfillopacity{0.700000}%
\pgfsetlinewidth{0.000000pt}%
\definecolor{currentstroke}{rgb}{0.000000,0.000000,0.000000}%
\pgfsetstrokecolor{currentstroke}%
\pgfsetdash{}{0pt}%
\pgfpathmoveto{\pgfqpoint{3.083623in}{3.281790in}}%
\pgfpathlineto{\pgfqpoint{3.096610in}{3.262739in}}%
\pgfpathlineto{\pgfqpoint{3.109590in}{3.243982in}}%
\pgfpathlineto{\pgfqpoint{3.122562in}{3.225515in}}%
\pgfpathlineto{\pgfqpoint{3.135528in}{3.207334in}}%
\pgfpathlineto{\pgfqpoint{3.143191in}{3.221356in}}%
\pgfpathlineto{\pgfqpoint{3.150847in}{3.235551in}}%
\pgfpathlineto{\pgfqpoint{3.158497in}{3.249920in}}%
\pgfpathlineto{\pgfqpoint{3.166140in}{3.264467in}}%
\pgfpathlineto{\pgfqpoint{3.153184in}{3.282872in}}%
\pgfpathlineto{\pgfqpoint{3.140221in}{3.301564in}}%
\pgfpathlineto{\pgfqpoint{3.127251in}{3.320546in}}%
\pgfpathlineto{\pgfqpoint{3.114273in}{3.339822in}}%
\pgfpathlineto{\pgfqpoint{3.106621in}{3.325039in}}%
\pgfpathlineto{\pgfqpoint{3.098962in}{3.310441in}}%
\pgfpathlineto{\pgfqpoint{3.091296in}{3.296025in}}%
\pgfpathlineto{\pgfqpoint{3.083623in}{3.281790in}}%
\pgfpathclose%
\pgfusepath{fill}%
\end{pgfscope}%
\begin{pgfscope}%
\pgfpathrectangle{\pgfqpoint{1.254980in}{0.150000in}}{\pgfqpoint{5.490039in}{5.490039in}}%
\pgfusepath{clip}%
\pgfsetbuttcap%
\pgfsetroundjoin%
\definecolor{currentfill}{rgb}{0.263663,0.237631,0.518762}%
\pgfsetfillcolor{currentfill}%
\pgfsetfillopacity{0.700000}%
\pgfsetlinewidth{0.000000pt}%
\definecolor{currentstroke}{rgb}{0.000000,0.000000,0.000000}%
\pgfsetstrokecolor{currentstroke}%
\pgfsetdash{}{0pt}%
\pgfpathmoveto{\pgfqpoint{3.629915in}{2.773739in}}%
\pgfpathlineto{\pgfqpoint{3.642773in}{2.764422in}}%
\pgfpathlineto{\pgfqpoint{3.655632in}{2.755318in}}%
\pgfpathlineto{\pgfqpoint{3.668492in}{2.746424in}}%
\pgfpathlineto{\pgfqpoint{3.681354in}{2.737741in}}%
\pgfpathlineto{\pgfqpoint{3.688909in}{2.750084in}}%
\pgfpathlineto{\pgfqpoint{3.696460in}{2.762540in}}%
\pgfpathlineto{\pgfqpoint{3.704006in}{2.775114in}}%
\pgfpathlineto{\pgfqpoint{3.711547in}{2.787808in}}%
\pgfpathlineto{\pgfqpoint{3.698693in}{2.796738in}}%
\pgfpathlineto{\pgfqpoint{3.685841in}{2.805878in}}%
\pgfpathlineto{\pgfqpoint{3.672989in}{2.815229in}}%
\pgfpathlineto{\pgfqpoint{3.660139in}{2.824793in}}%
\pgfpathlineto{\pgfqpoint{3.652591in}{2.811842in}}%
\pgfpathlineto{\pgfqpoint{3.645037in}{2.799018in}}%
\pgfpathlineto{\pgfqpoint{3.637479in}{2.786318in}}%
\pgfpathlineto{\pgfqpoint{3.629915in}{2.773739in}}%
\pgfpathclose%
\pgfusepath{fill}%
\end{pgfscope}%
\begin{pgfscope}%
\pgfpathrectangle{\pgfqpoint{1.254980in}{0.150000in}}{\pgfqpoint{5.490039in}{5.490039in}}%
\pgfusepath{clip}%
\pgfsetbuttcap%
\pgfsetroundjoin%
\definecolor{currentfill}{rgb}{0.257322,0.256130,0.526563}%
\pgfsetfillcolor{currentfill}%
\pgfsetfillopacity{0.700000}%
\pgfsetlinewidth{0.000000pt}%
\definecolor{currentstroke}{rgb}{0.000000,0.000000,0.000000}%
\pgfsetstrokecolor{currentstroke}%
\pgfsetdash{}{0pt}%
\pgfpathmoveto{\pgfqpoint{4.191868in}{2.805184in}}%
\pgfpathlineto{\pgfqpoint{4.204798in}{2.800596in}}%
\pgfpathlineto{\pgfqpoint{4.217732in}{2.796187in}}%
\pgfpathlineto{\pgfqpoint{4.230673in}{2.791956in}}%
\pgfpathlineto{\pgfqpoint{4.243620in}{2.787904in}}%
\pgfpathlineto{\pgfqpoint{4.251029in}{2.800197in}}%
\pgfpathlineto{\pgfqpoint{4.258435in}{2.812623in}}%
\pgfpathlineto{\pgfqpoint{4.265837in}{2.825188in}}%
\pgfpathlineto{\pgfqpoint{4.273236in}{2.837896in}}%
\pgfpathlineto{\pgfqpoint{4.260298in}{2.842332in}}%
\pgfpathlineto{\pgfqpoint{4.247365in}{2.846946in}}%
\pgfpathlineto{\pgfqpoint{4.234439in}{2.851739in}}%
\pgfpathlineto{\pgfqpoint{4.221518in}{2.856711in}}%
\pgfpathlineto{\pgfqpoint{4.214110in}{2.843609in}}%
\pgfpathlineto{\pgfqpoint{4.206700in}{2.830657in}}%
\pgfpathlineto{\pgfqpoint{4.199286in}{2.817850in}}%
\pgfpathlineto{\pgfqpoint{4.191868in}{2.805184in}}%
\pgfpathclose%
\pgfusepath{fill}%
\end{pgfscope}%
\begin{pgfscope}%
\pgfpathrectangle{\pgfqpoint{1.254980in}{0.150000in}}{\pgfqpoint{5.490039in}{5.490039in}}%
\pgfusepath{clip}%
\pgfsetbuttcap%
\pgfsetroundjoin%
\definecolor{currentfill}{rgb}{0.266580,0.228262,0.514349}%
\pgfsetfillcolor{currentfill}%
\pgfsetfillopacity{0.700000}%
\pgfsetlinewidth{0.000000pt}%
\definecolor{currentstroke}{rgb}{0.000000,0.000000,0.000000}%
\pgfsetstrokecolor{currentstroke}%
\pgfsetdash{}{0pt}%
\pgfpathmoveto{\pgfqpoint{3.895987in}{2.745310in}}%
\pgfpathlineto{\pgfqpoint{3.908866in}{2.738723in}}%
\pgfpathlineto{\pgfqpoint{3.921749in}{2.732329in}}%
\pgfpathlineto{\pgfqpoint{3.934635in}{2.726128in}}%
\pgfpathlineto{\pgfqpoint{3.947525in}{2.720119in}}%
\pgfpathlineto{\pgfqpoint{3.955013in}{2.732297in}}%
\pgfpathlineto{\pgfqpoint{3.962497in}{2.744588in}}%
\pgfpathlineto{\pgfqpoint{3.969977in}{2.756995in}}%
\pgfpathlineto{\pgfqpoint{3.977452in}{2.769522in}}%
\pgfpathlineto{\pgfqpoint{3.964569in}{2.775832in}}%
\pgfpathlineto{\pgfqpoint{3.951691in}{2.782334in}}%
\pgfpathlineto{\pgfqpoint{3.938816in}{2.789029in}}%
\pgfpathlineto{\pgfqpoint{3.925944in}{2.795918in}}%
\pgfpathlineto{\pgfqpoint{3.918461in}{2.783079in}}%
\pgfpathlineto{\pgfqpoint{3.910974in}{2.770367in}}%
\pgfpathlineto{\pgfqpoint{3.903483in}{2.757778in}}%
\pgfpathlineto{\pgfqpoint{3.895987in}{2.745310in}}%
\pgfpathclose%
\pgfusepath{fill}%
\end{pgfscope}%
\begin{pgfscope}%
\pgfpathrectangle{\pgfqpoint{1.254980in}{0.150000in}}{\pgfqpoint{5.490039in}{5.490039in}}%
\pgfusepath{clip}%
\pgfsetbuttcap%
\pgfsetroundjoin%
\definecolor{currentfill}{rgb}{0.258965,0.251537,0.524736}%
\pgfsetfillcolor{currentfill}%
\pgfsetfillopacity{0.700000}%
\pgfsetlinewidth{0.000000pt}%
\definecolor{currentstroke}{rgb}{0.000000,0.000000,0.000000}%
\pgfsetstrokecolor{currentstroke}%
\pgfsetdash{}{0pt}%
\pgfpathmoveto{\pgfqpoint{3.496699in}{2.805169in}}%
\pgfpathlineto{\pgfqpoint{3.509563in}{2.794320in}}%
\pgfpathlineto{\pgfqpoint{3.522427in}{2.783696in}}%
\pgfpathlineto{\pgfqpoint{3.535290in}{2.773294in}}%
\pgfpathlineto{\pgfqpoint{3.548154in}{2.763113in}}%
\pgfpathlineto{\pgfqpoint{3.555746in}{2.775447in}}%
\pgfpathlineto{\pgfqpoint{3.563332in}{2.787898in}}%
\pgfpathlineto{\pgfqpoint{3.570914in}{2.800468in}}%
\pgfpathlineto{\pgfqpoint{3.578491in}{2.813159in}}%
\pgfpathlineto{\pgfqpoint{3.565635in}{2.823560in}}%
\pgfpathlineto{\pgfqpoint{3.552780in}{2.834182in}}%
\pgfpathlineto{\pgfqpoint{3.539925in}{2.845026in}}%
\pgfpathlineto{\pgfqpoint{3.527069in}{2.856095in}}%
\pgfpathlineto{\pgfqpoint{3.519484in}{2.843174in}}%
\pgfpathlineto{\pgfqpoint{3.511894in}{2.830380in}}%
\pgfpathlineto{\pgfqpoint{3.504299in}{2.817713in}}%
\pgfpathlineto{\pgfqpoint{3.496699in}{2.805169in}}%
\pgfpathclose%
\pgfusepath{fill}%
\end{pgfscope}%
\begin{pgfscope}%
\pgfpathrectangle{\pgfqpoint{1.254980in}{0.150000in}}{\pgfqpoint{5.490039in}{5.490039in}}%
\pgfusepath{clip}%
\pgfsetbuttcap%
\pgfsetroundjoin%
\definecolor{currentfill}{rgb}{0.262138,0.242286,0.520837}%
\pgfsetfillcolor{currentfill}%
\pgfsetfillopacity{0.700000}%
\pgfsetlinewidth{0.000000pt}%
\definecolor{currentstroke}{rgb}{0.000000,0.000000,0.000000}%
\pgfsetstrokecolor{currentstroke}%
\pgfsetdash{}{0pt}%
\pgfpathmoveto{\pgfqpoint{4.110468in}{2.774574in}}%
\pgfpathlineto{\pgfqpoint{4.123383in}{2.769617in}}%
\pgfpathlineto{\pgfqpoint{4.136304in}{2.764842in}}%
\pgfpathlineto{\pgfqpoint{4.149231in}{2.760249in}}%
\pgfpathlineto{\pgfqpoint{4.162163in}{2.755837in}}%
\pgfpathlineto{\pgfqpoint{4.169595in}{2.767984in}}%
\pgfpathlineto{\pgfqpoint{4.177023in}{2.780255in}}%
\pgfpathlineto{\pgfqpoint{4.184447in}{2.792654in}}%
\pgfpathlineto{\pgfqpoint{4.191868in}{2.805184in}}%
\pgfpathlineto{\pgfqpoint{4.178945in}{2.809953in}}%
\pgfpathlineto{\pgfqpoint{4.166026in}{2.814902in}}%
\pgfpathlineto{\pgfqpoint{4.153114in}{2.820033in}}%
\pgfpathlineto{\pgfqpoint{4.140206in}{2.825346in}}%
\pgfpathlineto{\pgfqpoint{4.132777in}{2.812450in}}%
\pgfpathlineto{\pgfqpoint{4.125344in}{2.799692in}}%
\pgfpathlineto{\pgfqpoint{4.117908in}{2.787068in}}%
\pgfpathlineto{\pgfqpoint{4.110468in}{2.774574in}}%
\pgfpathclose%
\pgfusepath{fill}%
\end{pgfscope}%
\begin{pgfscope}%
\pgfpathrectangle{\pgfqpoint{1.254980in}{0.150000in}}{\pgfqpoint{5.490039in}{5.490039in}}%
\pgfusepath{clip}%
\pgfsetbuttcap%
\pgfsetroundjoin%
\definecolor{currentfill}{rgb}{0.159194,0.482237,0.558073}%
\pgfsetfillcolor{currentfill}%
\pgfsetfillopacity{0.700000}%
\pgfsetlinewidth{0.000000pt}%
\definecolor{currentstroke}{rgb}{0.000000,0.000000,0.000000}%
\pgfsetstrokecolor{currentstroke}%
\pgfsetdash{}{0pt}%
\pgfpathmoveto{\pgfqpoint{3.031594in}{3.360976in}}%
\pgfpathlineto{\pgfqpoint{3.044614in}{3.340726in}}%
\pgfpathlineto{\pgfqpoint{3.057625in}{3.320780in}}%
\pgfpathlineto{\pgfqpoint{3.070628in}{3.301135in}}%
\pgfpathlineto{\pgfqpoint{3.083623in}{3.281790in}}%
\pgfpathlineto{\pgfqpoint{3.091296in}{3.296025in}}%
\pgfpathlineto{\pgfqpoint{3.098962in}{3.310441in}}%
\pgfpathlineto{\pgfqpoint{3.106621in}{3.325039in}}%
\pgfpathlineto{\pgfqpoint{3.114273in}{3.339822in}}%
\pgfpathlineto{\pgfqpoint{3.101288in}{3.359393in}}%
\pgfpathlineto{\pgfqpoint{3.088295in}{3.379263in}}%
\pgfpathlineto{\pgfqpoint{3.075293in}{3.399435in}}%
\pgfpathlineto{\pgfqpoint{3.062283in}{3.419912in}}%
\pgfpathlineto{\pgfqpoint{3.054622in}{3.404892in}}%
\pgfpathlineto{\pgfqpoint{3.046953in}{3.390065in}}%
\pgfpathlineto{\pgfqpoint{3.039277in}{3.375427in}}%
\pgfpathlineto{\pgfqpoint{3.031594in}{3.360976in}}%
\pgfpathclose%
\pgfusepath{fill}%
\end{pgfscope}%
\begin{pgfscope}%
\pgfpathrectangle{\pgfqpoint{1.254980in}{0.150000in}}{\pgfqpoint{5.490039in}{5.490039in}}%
\pgfusepath{clip}%
\pgfsetbuttcap%
\pgfsetroundjoin%
\definecolor{currentfill}{rgb}{0.221989,0.339161,0.548752}%
\pgfsetfillcolor{currentfill}%
\pgfsetfillopacity{0.700000}%
\pgfsetlinewidth{0.000000pt}%
\definecolor{currentstroke}{rgb}{0.000000,0.000000,0.000000}%
\pgfsetstrokecolor{currentstroke}%
\pgfsetdash{}{0pt}%
\pgfpathmoveto{\pgfqpoint{4.650668in}{2.976047in}}%
\pgfpathlineto{\pgfqpoint{4.663714in}{2.973382in}}%
\pgfpathlineto{\pgfqpoint{4.676768in}{2.970880in}}%
\pgfpathlineto{\pgfqpoint{4.689831in}{2.968542in}}%
\pgfpathlineto{\pgfqpoint{4.702903in}{2.966367in}}%
\pgfpathlineto{\pgfqpoint{4.710200in}{2.979213in}}%
\pgfpathlineto{\pgfqpoint{4.717497in}{2.992256in}}%
\pgfpathlineto{\pgfqpoint{4.724793in}{3.005503in}}%
\pgfpathlineto{\pgfqpoint{4.732088in}{3.018959in}}%
\pgfpathlineto{\pgfqpoint{4.719029in}{3.021655in}}%
\pgfpathlineto{\pgfqpoint{4.705978in}{3.024515in}}%
\pgfpathlineto{\pgfqpoint{4.692936in}{3.027537in}}%
\pgfpathlineto{\pgfqpoint{4.679901in}{3.030724in}}%
\pgfpathlineto{\pgfqpoint{4.672594in}{3.016737in}}%
\pgfpathlineto{\pgfqpoint{4.665286in}{3.002966in}}%
\pgfpathlineto{\pgfqpoint{4.657977in}{2.989404in}}%
\pgfpathlineto{\pgfqpoint{4.650668in}{2.976047in}}%
\pgfpathclose%
\pgfusepath{fill}%
\end{pgfscope}%
\begin{pgfscope}%
\pgfpathrectangle{\pgfqpoint{1.254980in}{0.150000in}}{\pgfqpoint{5.490039in}{5.490039in}}%
\pgfusepath{clip}%
\pgfsetbuttcap%
\pgfsetroundjoin%
\definecolor{currentfill}{rgb}{0.212395,0.359683,0.551710}%
\pgfsetfillcolor{currentfill}%
\pgfsetfillopacity{0.700000}%
\pgfsetlinewidth{0.000000pt}%
\definecolor{currentstroke}{rgb}{0.000000,0.000000,0.000000}%
\pgfsetstrokecolor{currentstroke}%
\pgfsetdash{}{0pt}%
\pgfpathmoveto{\pgfqpoint{4.732088in}{3.018959in}}%
\pgfpathlineto{\pgfqpoint{4.745155in}{3.016425in}}%
\pgfpathlineto{\pgfqpoint{4.758230in}{3.014053in}}%
\pgfpathlineto{\pgfqpoint{4.771315in}{3.011843in}}%
\pgfpathlineto{\pgfqpoint{4.784408in}{3.009794in}}%
\pgfpathlineto{\pgfqpoint{4.791689in}{3.022926in}}%
\pgfpathlineto{\pgfqpoint{4.798970in}{3.036274in}}%
\pgfpathlineto{\pgfqpoint{4.806250in}{3.049843in}}%
\pgfpathlineto{\pgfqpoint{4.813531in}{3.063640in}}%
\pgfpathlineto{\pgfqpoint{4.800451in}{3.066238in}}%
\pgfpathlineto{\pgfqpoint{4.787380in}{3.068997in}}%
\pgfpathlineto{\pgfqpoint{4.774317in}{3.071918in}}%
\pgfpathlineto{\pgfqpoint{4.761263in}{3.075001in}}%
\pgfpathlineto{\pgfqpoint{4.753969in}{3.060645in}}%
\pgfpathlineto{\pgfqpoint{4.746676in}{3.046523in}}%
\pgfpathlineto{\pgfqpoint{4.739382in}{3.032630in}}%
\pgfpathlineto{\pgfqpoint{4.732088in}{3.018959in}}%
\pgfpathclose%
\pgfusepath{fill}%
\end{pgfscope}%
\begin{pgfscope}%
\pgfpathrectangle{\pgfqpoint{1.254980in}{0.150000in}}{\pgfqpoint{5.490039in}{5.490039in}}%
\pgfusepath{clip}%
\pgfsetbuttcap%
\pgfsetroundjoin%
\definecolor{currentfill}{rgb}{0.229739,0.322361,0.545706}%
\pgfsetfillcolor{currentfill}%
\pgfsetfillopacity{0.700000}%
\pgfsetlinewidth{0.000000pt}%
\definecolor{currentstroke}{rgb}{0.000000,0.000000,0.000000}%
\pgfsetstrokecolor{currentstroke}%
\pgfsetdash{}{0pt}%
\pgfpathmoveto{\pgfqpoint{4.569262in}{2.934880in}}%
\pgfpathlineto{\pgfqpoint{4.582288in}{2.932047in}}%
\pgfpathlineto{\pgfqpoint{4.595322in}{2.929379in}}%
\pgfpathlineto{\pgfqpoint{4.608364in}{2.926878in}}%
\pgfpathlineto{\pgfqpoint{4.621414in}{2.924542in}}%
\pgfpathlineto{\pgfqpoint{4.628730in}{2.937141in}}%
\pgfpathlineto{\pgfqpoint{4.636044in}{2.949921in}}%
\pgfpathlineto{\pgfqpoint{4.643356in}{2.962888in}}%
\pgfpathlineto{\pgfqpoint{4.650668in}{2.976047in}}%
\pgfpathlineto{\pgfqpoint{4.637629in}{2.978878in}}%
\pgfpathlineto{\pgfqpoint{4.624599in}{2.981873in}}%
\pgfpathlineto{\pgfqpoint{4.611576in}{2.985034in}}%
\pgfpathlineto{\pgfqpoint{4.598561in}{2.988361in}}%
\pgfpathlineto{\pgfqpoint{4.591239in}{2.974698in}}%
\pgfpathlineto{\pgfqpoint{4.583915in}{2.961234in}}%
\pgfpathlineto{\pgfqpoint{4.576589in}{2.947963in}}%
\pgfpathlineto{\pgfqpoint{4.569262in}{2.934880in}}%
\pgfpathclose%
\pgfusepath{fill}%
\end{pgfscope}%
\begin{pgfscope}%
\pgfpathrectangle{\pgfqpoint{1.254980in}{0.150000in}}{\pgfqpoint{5.490039in}{5.490039in}}%
\pgfusepath{clip}%
\pgfsetbuttcap%
\pgfsetroundjoin%
\definecolor{currentfill}{rgb}{0.203063,0.379716,0.553925}%
\pgfsetfillcolor{currentfill}%
\pgfsetfillopacity{0.700000}%
\pgfsetlinewidth{0.000000pt}%
\definecolor{currentstroke}{rgb}{0.000000,0.000000,0.000000}%
\pgfsetstrokecolor{currentstroke}%
\pgfsetdash{}{0pt}%
\pgfpathmoveto{\pgfqpoint{4.813531in}{3.063640in}}%
\pgfpathlineto{\pgfqpoint{4.826619in}{3.061202in}}%
\pgfpathlineto{\pgfqpoint{4.839716in}{3.058924in}}%
\pgfpathlineto{\pgfqpoint{4.852822in}{3.056807in}}%
\pgfpathlineto{\pgfqpoint{4.865937in}{3.054848in}}%
\pgfpathlineto{\pgfqpoint{4.873204in}{3.068313in}}%
\pgfpathlineto{\pgfqpoint{4.880471in}{3.082012in}}%
\pgfpathlineto{\pgfqpoint{4.887738in}{3.095951in}}%
\pgfpathlineto{\pgfqpoint{4.895006in}{3.110138in}}%
\pgfpathlineto{\pgfqpoint{4.881906in}{3.112673in}}%
\pgfpathlineto{\pgfqpoint{4.868814in}{3.115367in}}%
\pgfpathlineto{\pgfqpoint{4.855730in}{3.118222in}}%
\pgfpathlineto{\pgfqpoint{4.842656in}{3.121236in}}%
\pgfpathlineto{\pgfqpoint{4.835374in}{3.106462in}}%
\pgfpathlineto{\pgfqpoint{4.828092in}{3.091943in}}%
\pgfpathlineto{\pgfqpoint{4.820811in}{3.077671in}}%
\pgfpathlineto{\pgfqpoint{4.813531in}{3.063640in}}%
\pgfpathclose%
\pgfusepath{fill}%
\end{pgfscope}%
\begin{pgfscope}%
\pgfpathrectangle{\pgfqpoint{1.254980in}{0.150000in}}{\pgfqpoint{5.490039in}{5.490039in}}%
\pgfusepath{clip}%
\pgfsetbuttcap%
\pgfsetroundjoin%
\definecolor{currentfill}{rgb}{0.218130,0.347432,0.550038}%
\pgfsetfillcolor{currentfill}%
\pgfsetfillopacity{0.700000}%
\pgfsetlinewidth{0.000000pt}%
\definecolor{currentstroke}{rgb}{0.000000,0.000000,0.000000}%
\pgfsetstrokecolor{currentstroke}%
\pgfsetdash{}{0pt}%
\pgfpathmoveto{\pgfqpoint{3.208390in}{3.019061in}}%
\pgfpathlineto{\pgfqpoint{3.221314in}{3.003545in}}%
\pgfpathlineto{\pgfqpoint{3.234233in}{2.988290in}}%
\pgfpathlineto{\pgfqpoint{3.247147in}{2.973296in}}%
\pgfpathlineto{\pgfqpoint{3.260058in}{2.958559in}}%
\pgfpathlineto{\pgfqpoint{3.267718in}{2.971366in}}%
\pgfpathlineto{\pgfqpoint{3.275371in}{2.984314in}}%
\pgfpathlineto{\pgfqpoint{3.283019in}{2.997406in}}%
\pgfpathlineto{\pgfqpoint{3.290660in}{3.010643in}}%
\pgfpathlineto{\pgfqpoint{3.277759in}{3.025573in}}%
\pgfpathlineto{\pgfqpoint{3.264854in}{3.040761in}}%
\pgfpathlineto{\pgfqpoint{3.251945in}{3.056210in}}%
\pgfpathlineto{\pgfqpoint{3.239031in}{3.071920in}}%
\pgfpathlineto{\pgfqpoint{3.231380in}{3.058479in}}%
\pgfpathlineto{\pgfqpoint{3.223723in}{3.045190in}}%
\pgfpathlineto{\pgfqpoint{3.216060in}{3.032052in}}%
\pgfpathlineto{\pgfqpoint{3.208390in}{3.019061in}}%
\pgfpathclose%
\pgfusepath{fill}%
\end{pgfscope}%
\begin{pgfscope}%
\pgfpathrectangle{\pgfqpoint{1.254980in}{0.150000in}}{\pgfqpoint{5.490039in}{5.490039in}}%
\pgfusepath{clip}%
\pgfsetbuttcap%
\pgfsetroundjoin%
\definecolor{currentfill}{rgb}{0.237441,0.305202,0.541921}%
\pgfsetfillcolor{currentfill}%
\pgfsetfillopacity{0.700000}%
\pgfsetlinewidth{0.000000pt}%
\definecolor{currentstroke}{rgb}{0.000000,0.000000,0.000000}%
\pgfsetstrokecolor{currentstroke}%
\pgfsetdash{}{0pt}%
\pgfpathmoveto{\pgfqpoint{4.487863in}{2.895453in}}%
\pgfpathlineto{\pgfqpoint{4.500869in}{2.892415in}}%
\pgfpathlineto{\pgfqpoint{4.513883in}{2.889546in}}%
\pgfpathlineto{\pgfqpoint{4.526904in}{2.886845in}}%
\pgfpathlineto{\pgfqpoint{4.539934in}{2.884312in}}%
\pgfpathlineto{\pgfqpoint{4.547269in}{2.896700in}}%
\pgfpathlineto{\pgfqpoint{4.554602in}{2.909254in}}%
\pgfpathlineto{\pgfqpoint{4.561933in}{2.921979in}}%
\pgfpathlineto{\pgfqpoint{4.569262in}{2.934880in}}%
\pgfpathlineto{\pgfqpoint{4.556244in}{2.937880in}}%
\pgfpathlineto{\pgfqpoint{4.543233in}{2.941048in}}%
\pgfpathlineto{\pgfqpoint{4.530230in}{2.944383in}}%
\pgfpathlineto{\pgfqpoint{4.517233in}{2.947887in}}%
\pgfpathlineto{\pgfqpoint{4.509894in}{2.934509in}}%
\pgfpathlineto{\pgfqpoint{4.502552in}{2.921315in}}%
\pgfpathlineto{\pgfqpoint{4.495209in}{2.908298in}}%
\pgfpathlineto{\pgfqpoint{4.487863in}{2.895453in}}%
\pgfpathclose%
\pgfusepath{fill}%
\end{pgfscope}%
\begin{pgfscope}%
\pgfpathrectangle{\pgfqpoint{1.254980in}{0.150000in}}{\pgfqpoint{5.490039in}{5.490039in}}%
\pgfusepath{clip}%
\pgfsetbuttcap%
\pgfsetroundjoin%
\definecolor{currentfill}{rgb}{0.229739,0.322361,0.545706}%
\pgfsetfillcolor{currentfill}%
\pgfsetfillopacity{0.700000}%
\pgfsetlinewidth{0.000000pt}%
\definecolor{currentstroke}{rgb}{0.000000,0.000000,0.000000}%
\pgfsetstrokecolor{currentstroke}%
\pgfsetdash{}{0pt}%
\pgfpathmoveto{\pgfqpoint{3.260058in}{2.958559in}}%
\pgfpathlineto{\pgfqpoint{3.272965in}{2.944078in}}%
\pgfpathlineto{\pgfqpoint{3.285868in}{2.929850in}}%
\pgfpathlineto{\pgfqpoint{3.298767in}{2.915874in}}%
\pgfpathlineto{\pgfqpoint{3.311663in}{2.902147in}}%
\pgfpathlineto{\pgfqpoint{3.319314in}{2.914772in}}%
\pgfpathlineto{\pgfqpoint{3.326958in}{2.927530in}}%
\pgfpathlineto{\pgfqpoint{3.334596in}{2.940424in}}%
\pgfpathlineto{\pgfqpoint{3.342228in}{2.953457in}}%
\pgfpathlineto{\pgfqpoint{3.329341in}{2.967377in}}%
\pgfpathlineto{\pgfqpoint{3.316451in}{2.981547in}}%
\pgfpathlineto{\pgfqpoint{3.303558in}{2.995968in}}%
\pgfpathlineto{\pgfqpoint{3.290660in}{3.010643in}}%
\pgfpathlineto{\pgfqpoint{3.283019in}{2.997406in}}%
\pgfpathlineto{\pgfqpoint{3.275371in}{2.984314in}}%
\pgfpathlineto{\pgfqpoint{3.267718in}{2.971366in}}%
\pgfpathlineto{\pgfqpoint{3.260058in}{2.958559in}}%
\pgfpathclose%
\pgfusepath{fill}%
\end{pgfscope}%
\begin{pgfscope}%
\pgfpathrectangle{\pgfqpoint{1.254980in}{0.150000in}}{\pgfqpoint{5.490039in}{5.490039in}}%
\pgfusepath{clip}%
\pgfsetbuttcap%
\pgfsetroundjoin%
\definecolor{currentfill}{rgb}{0.267968,0.223549,0.512008}%
\pgfsetfillcolor{currentfill}%
\pgfsetfillopacity{0.700000}%
\pgfsetlinewidth{0.000000pt}%
\definecolor{currentstroke}{rgb}{0.000000,0.000000,0.000000}%
\pgfsetstrokecolor{currentstroke}%
\pgfsetdash{}{0pt}%
\pgfpathmoveto{\pgfqpoint{3.681354in}{2.737741in}}%
\pgfpathlineto{\pgfqpoint{3.694217in}{2.729266in}}%
\pgfpathlineto{\pgfqpoint{3.707082in}{2.720998in}}%
\pgfpathlineto{\pgfqpoint{3.719948in}{2.712937in}}%
\pgfpathlineto{\pgfqpoint{3.732817in}{2.705080in}}%
\pgfpathlineto{\pgfqpoint{3.740365in}{2.717186in}}%
\pgfpathlineto{\pgfqpoint{3.747908in}{2.729399in}}%
\pgfpathlineto{\pgfqpoint{3.755446in}{2.741723in}}%
\pgfpathlineto{\pgfqpoint{3.762979in}{2.754159in}}%
\pgfpathlineto{\pgfqpoint{3.750118in}{2.762263in}}%
\pgfpathlineto{\pgfqpoint{3.737259in}{2.770571in}}%
\pgfpathlineto{\pgfqpoint{3.724402in}{2.779086in}}%
\pgfpathlineto{\pgfqpoint{3.711547in}{2.787808in}}%
\pgfpathlineto{\pgfqpoint{3.704006in}{2.775114in}}%
\pgfpathlineto{\pgfqpoint{3.696460in}{2.762540in}}%
\pgfpathlineto{\pgfqpoint{3.688909in}{2.750084in}}%
\pgfpathlineto{\pgfqpoint{3.681354in}{2.737741in}}%
\pgfpathclose%
\pgfusepath{fill}%
\end{pgfscope}%
\begin{pgfscope}%
\pgfpathrectangle{\pgfqpoint{1.254980in}{0.150000in}}{\pgfqpoint{5.490039in}{5.490039in}}%
\pgfusepath{clip}%
\pgfsetbuttcap%
\pgfsetroundjoin%
\definecolor{currentfill}{rgb}{0.194100,0.399323,0.555565}%
\pgfsetfillcolor{currentfill}%
\pgfsetfillopacity{0.700000}%
\pgfsetlinewidth{0.000000pt}%
\definecolor{currentstroke}{rgb}{0.000000,0.000000,0.000000}%
\pgfsetstrokecolor{currentstroke}%
\pgfsetdash{}{0pt}%
\pgfpathmoveto{\pgfqpoint{4.895006in}{3.110138in}}%
\pgfpathlineto{\pgfqpoint{4.908116in}{3.107761in}}%
\pgfpathlineto{\pgfqpoint{4.921234in}{3.105543in}}%
\pgfpathlineto{\pgfqpoint{4.934362in}{3.103483in}}%
\pgfpathlineto{\pgfqpoint{4.947499in}{3.101580in}}%
\pgfpathlineto{\pgfqpoint{4.954753in}{3.115427in}}%
\pgfpathlineto{\pgfqpoint{4.962009in}{3.129529in}}%
\pgfpathlineto{\pgfqpoint{4.969265in}{3.143891in}}%
\pgfpathlineto{\pgfqpoint{4.976524in}{3.158522in}}%
\pgfpathlineto{\pgfqpoint{4.963402in}{3.161029in}}%
\pgfpathlineto{\pgfqpoint{4.950289in}{3.163694in}}%
\pgfpathlineto{\pgfqpoint{4.937185in}{3.166516in}}%
\pgfpathlineto{\pgfqpoint{4.924090in}{3.169497in}}%
\pgfpathlineto{\pgfqpoint{4.916817in}{3.154251in}}%
\pgfpathlineto{\pgfqpoint{4.909546in}{3.139281in}}%
\pgfpathlineto{\pgfqpoint{4.902275in}{3.124579in}}%
\pgfpathlineto{\pgfqpoint{4.895006in}{3.110138in}}%
\pgfpathclose%
\pgfusepath{fill}%
\end{pgfscope}%
\begin{pgfscope}%
\pgfpathrectangle{\pgfqpoint{1.254980in}{0.150000in}}{\pgfqpoint{5.490039in}{5.490039in}}%
\pgfusepath{clip}%
\pgfsetbuttcap%
\pgfsetroundjoin%
\definecolor{currentfill}{rgb}{0.204903,0.375746,0.553533}%
\pgfsetfillcolor{currentfill}%
\pgfsetfillopacity{0.700000}%
\pgfsetlinewidth{0.000000pt}%
\definecolor{currentstroke}{rgb}{0.000000,0.000000,0.000000}%
\pgfsetstrokecolor{currentstroke}%
\pgfsetdash{}{0pt}%
\pgfpathmoveto{\pgfqpoint{3.156645in}{3.083796in}}%
\pgfpathlineto{\pgfqpoint{3.169589in}{3.067207in}}%
\pgfpathlineto{\pgfqpoint{3.182528in}{3.050890in}}%
\pgfpathlineto{\pgfqpoint{3.195461in}{3.034842in}}%
\pgfpathlineto{\pgfqpoint{3.208390in}{3.019061in}}%
\pgfpathlineto{\pgfqpoint{3.216060in}{3.032052in}}%
\pgfpathlineto{\pgfqpoint{3.223723in}{3.045190in}}%
\pgfpathlineto{\pgfqpoint{3.231380in}{3.058479in}}%
\pgfpathlineto{\pgfqpoint{3.239031in}{3.071920in}}%
\pgfpathlineto{\pgfqpoint{3.226112in}{3.087896in}}%
\pgfpathlineto{\pgfqpoint{3.213188in}{3.104138in}}%
\pgfpathlineto{\pgfqpoint{3.200259in}{3.120650in}}%
\pgfpathlineto{\pgfqpoint{3.187325in}{3.137433in}}%
\pgfpathlineto{\pgfqpoint{3.179665in}{3.123786in}}%
\pgfpathlineto{\pgfqpoint{3.171998in}{3.110300in}}%
\pgfpathlineto{\pgfqpoint{3.164325in}{3.096970in}}%
\pgfpathlineto{\pgfqpoint{3.156645in}{3.083796in}}%
\pgfpathclose%
\pgfusepath{fill}%
\end{pgfscope}%
\begin{pgfscope}%
\pgfpathrectangle{\pgfqpoint{1.254980in}{0.150000in}}{\pgfqpoint{5.490039in}{5.490039in}}%
\pgfusepath{clip}%
\pgfsetbuttcap%
\pgfsetroundjoin%
\definecolor{currentfill}{rgb}{0.265145,0.232956,0.516599}%
\pgfsetfillcolor{currentfill}%
\pgfsetfillopacity{0.700000}%
\pgfsetlinewidth{0.000000pt}%
\definecolor{currentstroke}{rgb}{0.000000,0.000000,0.000000}%
\pgfsetstrokecolor{currentstroke}%
\pgfsetdash{}{0pt}%
\pgfpathmoveto{\pgfqpoint{4.029024in}{2.746175in}}%
\pgfpathlineto{\pgfqpoint{4.041928in}{2.740808in}}%
\pgfpathlineto{\pgfqpoint{4.054837in}{2.735627in}}%
\pgfpathlineto{\pgfqpoint{4.067750in}{2.730631in}}%
\pgfpathlineto{\pgfqpoint{4.080669in}{2.725819in}}%
\pgfpathlineto{\pgfqpoint{4.088125in}{2.737833in}}%
\pgfpathlineto{\pgfqpoint{4.095576in}{2.749961in}}%
\pgfpathlineto{\pgfqpoint{4.103024in}{2.762207in}}%
\pgfpathlineto{\pgfqpoint{4.110468in}{2.774574in}}%
\pgfpathlineto{\pgfqpoint{4.097557in}{2.779715in}}%
\pgfpathlineto{\pgfqpoint{4.084651in}{2.785040in}}%
\pgfpathlineto{\pgfqpoint{4.071750in}{2.790550in}}%
\pgfpathlineto{\pgfqpoint{4.058854in}{2.796246in}}%
\pgfpathlineto{\pgfqpoint{4.051402in}{2.783539in}}%
\pgfpathlineto{\pgfqpoint{4.043947in}{2.770961in}}%
\pgfpathlineto{\pgfqpoint{4.036487in}{2.758508in}}%
\pgfpathlineto{\pgfqpoint{4.029024in}{2.746175in}}%
\pgfpathclose%
\pgfusepath{fill}%
\end{pgfscope}%
\begin{pgfscope}%
\pgfpathrectangle{\pgfqpoint{1.254980in}{0.150000in}}{\pgfqpoint{5.490039in}{5.490039in}}%
\pgfusepath{clip}%
\pgfsetbuttcap%
\pgfsetroundjoin%
\definecolor{currentfill}{rgb}{0.239346,0.300855,0.540844}%
\pgfsetfillcolor{currentfill}%
\pgfsetfillopacity{0.700000}%
\pgfsetlinewidth{0.000000pt}%
\definecolor{currentstroke}{rgb}{0.000000,0.000000,0.000000}%
\pgfsetstrokecolor{currentstroke}%
\pgfsetdash{}{0pt}%
\pgfpathmoveto{\pgfqpoint{3.311663in}{2.902147in}}%
\pgfpathlineto{\pgfqpoint{3.324557in}{2.888668in}}%
\pgfpathlineto{\pgfqpoint{3.337447in}{2.875433in}}%
\pgfpathlineto{\pgfqpoint{3.350335in}{2.862442in}}%
\pgfpathlineto{\pgfqpoint{3.363221in}{2.849693in}}%
\pgfpathlineto{\pgfqpoint{3.370861in}{2.862135in}}%
\pgfpathlineto{\pgfqpoint{3.378496in}{2.874703in}}%
\pgfpathlineto{\pgfqpoint{3.386125in}{2.887401in}}%
\pgfpathlineto{\pgfqpoint{3.393749in}{2.900230in}}%
\pgfpathlineto{\pgfqpoint{3.380872in}{2.913173in}}%
\pgfpathlineto{\pgfqpoint{3.367994in}{2.926357in}}%
\pgfpathlineto{\pgfqpoint{3.355112in}{2.939784in}}%
\pgfpathlineto{\pgfqpoint{3.342228in}{2.953457in}}%
\pgfpathlineto{\pgfqpoint{3.334596in}{2.940424in}}%
\pgfpathlineto{\pgfqpoint{3.326958in}{2.927530in}}%
\pgfpathlineto{\pgfqpoint{3.319314in}{2.914772in}}%
\pgfpathlineto{\pgfqpoint{3.311663in}{2.902147in}}%
\pgfpathclose%
\pgfusepath{fill}%
\end{pgfscope}%
\begin{pgfscope}%
\pgfpathrectangle{\pgfqpoint{1.254980in}{0.150000in}}{\pgfqpoint{5.490039in}{5.490039in}}%
\pgfusepath{clip}%
\pgfsetbuttcap%
\pgfsetroundjoin%
\definecolor{currentfill}{rgb}{0.244972,0.287675,0.537260}%
\pgfsetfillcolor{currentfill}%
\pgfsetfillopacity{0.700000}%
\pgfsetlinewidth{0.000000pt}%
\definecolor{currentstroke}{rgb}{0.000000,0.000000,0.000000}%
\pgfsetstrokecolor{currentstroke}%
\pgfsetdash{}{0pt}%
\pgfpathmoveto{\pgfqpoint{4.406461in}{2.857785in}}%
\pgfpathlineto{\pgfqpoint{4.419449in}{2.854506in}}%
\pgfpathlineto{\pgfqpoint{4.432443in}{2.851398in}}%
\pgfpathlineto{\pgfqpoint{4.445445in}{2.848460in}}%
\pgfpathlineto{\pgfqpoint{4.458455in}{2.845692in}}%
\pgfpathlineto{\pgfqpoint{4.465810in}{2.857900in}}%
\pgfpathlineto{\pgfqpoint{4.473164in}{2.870259in}}%
\pgfpathlineto{\pgfqpoint{4.480514in}{2.882775in}}%
\pgfpathlineto{\pgfqpoint{4.487863in}{2.895453in}}%
\pgfpathlineto{\pgfqpoint{4.474864in}{2.898660in}}%
\pgfpathlineto{\pgfqpoint{4.461872in}{2.902036in}}%
\pgfpathlineto{\pgfqpoint{4.448887in}{2.905583in}}%
\pgfpathlineto{\pgfqpoint{4.435909in}{2.909301in}}%
\pgfpathlineto{\pgfqpoint{4.428551in}{2.896174in}}%
\pgfpathlineto{\pgfqpoint{4.421190in}{2.883216in}}%
\pgfpathlineto{\pgfqpoint{4.413827in}{2.870421in}}%
\pgfpathlineto{\pgfqpoint{4.406461in}{2.857785in}}%
\pgfpathclose%
\pgfusepath{fill}%
\end{pgfscope}%
\begin{pgfscope}%
\pgfpathrectangle{\pgfqpoint{1.254980in}{0.150000in}}{\pgfqpoint{5.490039in}{5.490039in}}%
\pgfusepath{clip}%
\pgfsetbuttcap%
\pgfsetroundjoin%
\definecolor{currentfill}{rgb}{0.269308,0.218818,0.509577}%
\pgfsetfillcolor{currentfill}%
\pgfsetfillopacity{0.700000}%
\pgfsetlinewidth{0.000000pt}%
\definecolor{currentstroke}{rgb}{0.000000,0.000000,0.000000}%
\pgfsetstrokecolor{currentstroke}%
\pgfsetdash{}{0pt}%
\pgfpathmoveto{\pgfqpoint{3.814447in}{2.723768in}}%
\pgfpathlineto{\pgfqpoint{3.827321in}{2.716671in}}%
\pgfpathlineto{\pgfqpoint{3.840197in}{2.709771in}}%
\pgfpathlineto{\pgfqpoint{3.853077in}{2.703068in}}%
\pgfpathlineto{\pgfqpoint{3.865961in}{2.696561in}}%
\pgfpathlineto{\pgfqpoint{3.873474in}{2.708585in}}%
\pgfpathlineto{\pgfqpoint{3.880983in}{2.720716in}}%
\pgfpathlineto{\pgfqpoint{3.888487in}{2.732957in}}%
\pgfpathlineto{\pgfqpoint{3.895987in}{2.745310in}}%
\pgfpathlineto{\pgfqpoint{3.883112in}{2.752091in}}%
\pgfpathlineto{\pgfqpoint{3.870240in}{2.759068in}}%
\pgfpathlineto{\pgfqpoint{3.857370in}{2.766242in}}%
\pgfpathlineto{\pgfqpoint{3.844504in}{2.773614in}}%
\pgfpathlineto{\pgfqpoint{3.836997in}{2.760977in}}%
\pgfpathlineto{\pgfqpoint{3.829485in}{2.748459in}}%
\pgfpathlineto{\pgfqpoint{3.821968in}{2.736057in}}%
\pgfpathlineto{\pgfqpoint{3.814447in}{2.723768in}}%
\pgfpathclose%
\pgfusepath{fill}%
\end{pgfscope}%
\begin{pgfscope}%
\pgfpathrectangle{\pgfqpoint{1.254980in}{0.150000in}}{\pgfqpoint{5.490039in}{5.490039in}}%
\pgfusepath{clip}%
\pgfsetbuttcap%
\pgfsetroundjoin%
\definecolor{currentfill}{rgb}{0.185556,0.418570,0.556753}%
\pgfsetfillcolor{currentfill}%
\pgfsetfillopacity{0.700000}%
\pgfsetlinewidth{0.000000pt}%
\definecolor{currentstroke}{rgb}{0.000000,0.000000,0.000000}%
\pgfsetstrokecolor{currentstroke}%
\pgfsetdash{}{0pt}%
\pgfpathmoveto{\pgfqpoint{4.976524in}{3.158522in}}%
\pgfpathlineto{\pgfqpoint{4.989654in}{3.156172in}}%
\pgfpathlineto{\pgfqpoint{5.002795in}{3.153979in}}%
\pgfpathlineto{\pgfqpoint{5.015944in}{3.151942in}}%
\pgfpathlineto{\pgfqpoint{5.029103in}{3.150061in}}%
\pgfpathlineto{\pgfqpoint{5.036347in}{3.164345in}}%
\pgfpathlineto{\pgfqpoint{5.043594in}{3.178906in}}%
\pgfpathlineto{\pgfqpoint{5.050842in}{3.193750in}}%
\pgfpathlineto{\pgfqpoint{5.058094in}{3.208884in}}%
\pgfpathlineto{\pgfqpoint{5.044951in}{3.211397in}}%
\pgfpathlineto{\pgfqpoint{5.031817in}{3.214066in}}%
\pgfpathlineto{\pgfqpoint{5.018693in}{3.216892in}}%
\pgfpathlineto{\pgfqpoint{5.005578in}{3.219874in}}%
\pgfpathlineto{\pgfqpoint{4.998311in}{3.204097in}}%
\pgfpathlineto{\pgfqpoint{4.991046in}{3.188618in}}%
\pgfpathlineto{\pgfqpoint{4.983784in}{3.173429in}}%
\pgfpathlineto{\pgfqpoint{4.976524in}{3.158522in}}%
\pgfpathclose%
\pgfusepath{fill}%
\end{pgfscope}%
\begin{pgfscope}%
\pgfpathrectangle{\pgfqpoint{1.254980in}{0.150000in}}{\pgfqpoint{5.490039in}{5.490039in}}%
\pgfusepath{clip}%
\pgfsetbuttcap%
\pgfsetroundjoin%
\definecolor{currentfill}{rgb}{0.263663,0.237631,0.518762}%
\pgfsetfillcolor{currentfill}%
\pgfsetfillopacity{0.700000}%
\pgfsetlinewidth{0.000000pt}%
\definecolor{currentstroke}{rgb}{0.000000,0.000000,0.000000}%
\pgfsetstrokecolor{currentstroke}%
\pgfsetdash{}{0pt}%
\pgfpathmoveto{\pgfqpoint{3.548154in}{2.763113in}}%
\pgfpathlineto{\pgfqpoint{3.561017in}{2.753151in}}%
\pgfpathlineto{\pgfqpoint{3.573881in}{2.743408in}}%
\pgfpathlineto{\pgfqpoint{3.586746in}{2.733882in}}%
\pgfpathlineto{\pgfqpoint{3.599611in}{2.724571in}}%
\pgfpathlineto{\pgfqpoint{3.607195in}{2.736696in}}%
\pgfpathlineto{\pgfqpoint{3.614773in}{2.748930in}}%
\pgfpathlineto{\pgfqpoint{3.622347in}{2.761277in}}%
\pgfpathlineto{\pgfqpoint{3.629915in}{2.773739in}}%
\pgfpathlineto{\pgfqpoint{3.617058in}{2.783270in}}%
\pgfpathlineto{\pgfqpoint{3.604202in}{2.793016in}}%
\pgfpathlineto{\pgfqpoint{3.591346in}{2.802978in}}%
\pgfpathlineto{\pgfqpoint{3.578491in}{2.813159in}}%
\pgfpathlineto{\pgfqpoint{3.570914in}{2.800468in}}%
\pgfpathlineto{\pgfqpoint{3.563332in}{2.787898in}}%
\pgfpathlineto{\pgfqpoint{3.555746in}{2.775447in}}%
\pgfpathlineto{\pgfqpoint{3.548154in}{2.763113in}}%
\pgfpathclose%
\pgfusepath{fill}%
\end{pgfscope}%
\begin{pgfscope}%
\pgfpathrectangle{\pgfqpoint{1.254980in}{0.150000in}}{\pgfqpoint{5.490039in}{5.490039in}}%
\pgfusepath{clip}%
\pgfsetbuttcap%
\pgfsetroundjoin%
\definecolor{currentfill}{rgb}{0.192357,0.403199,0.555836}%
\pgfsetfillcolor{currentfill}%
\pgfsetfillopacity{0.700000}%
\pgfsetlinewidth{0.000000pt}%
\definecolor{currentstroke}{rgb}{0.000000,0.000000,0.000000}%
\pgfsetstrokecolor{currentstroke}%
\pgfsetdash{}{0pt}%
\pgfpathmoveto{\pgfqpoint{3.104807in}{3.152916in}}%
\pgfpathlineto{\pgfqpoint{3.117776in}{3.135216in}}%
\pgfpathlineto{\pgfqpoint{3.130739in}{3.117798in}}%
\pgfpathlineto{\pgfqpoint{3.143695in}{3.100659in}}%
\pgfpathlineto{\pgfqpoint{3.156645in}{3.083796in}}%
\pgfpathlineto{\pgfqpoint{3.164325in}{3.096970in}}%
\pgfpathlineto{\pgfqpoint{3.171998in}{3.110300in}}%
\pgfpathlineto{\pgfqpoint{3.179665in}{3.123786in}}%
\pgfpathlineto{\pgfqpoint{3.187325in}{3.137433in}}%
\pgfpathlineto{\pgfqpoint{3.174385in}{3.154491in}}%
\pgfpathlineto{\pgfqpoint{3.161439in}{3.171825in}}%
\pgfpathlineto{\pgfqpoint{3.148486in}{3.189439in}}%
\pgfpathlineto{\pgfqpoint{3.135528in}{3.207334in}}%
\pgfpathlineto{\pgfqpoint{3.127858in}{3.193482in}}%
\pgfpathlineto{\pgfqpoint{3.120181in}{3.179797in}}%
\pgfpathlineto{\pgfqpoint{3.112498in}{3.166275in}}%
\pgfpathlineto{\pgfqpoint{3.104807in}{3.152916in}}%
\pgfpathclose%
\pgfusepath{fill}%
\end{pgfscope}%
\begin{pgfscope}%
\pgfpathrectangle{\pgfqpoint{1.254980in}{0.150000in}}{\pgfqpoint{5.490039in}{5.490039in}}%
\pgfusepath{clip}%
\pgfsetbuttcap%
\pgfsetroundjoin%
\definecolor{currentfill}{rgb}{0.252194,0.269783,0.531579}%
\pgfsetfillcolor{currentfill}%
\pgfsetfillopacity{0.700000}%
\pgfsetlinewidth{0.000000pt}%
\definecolor{currentstroke}{rgb}{0.000000,0.000000,0.000000}%
\pgfsetstrokecolor{currentstroke}%
\pgfsetdash{}{0pt}%
\pgfpathmoveto{\pgfqpoint{4.325050in}{2.821914in}}%
\pgfpathlineto{\pgfqpoint{4.338020in}{2.818356in}}%
\pgfpathlineto{\pgfqpoint{4.350996in}{2.814972in}}%
\pgfpathlineto{\pgfqpoint{4.363979in}{2.811760in}}%
\pgfpathlineto{\pgfqpoint{4.376969in}{2.808721in}}%
\pgfpathlineto{\pgfqpoint{4.384347in}{2.820774in}}%
\pgfpathlineto{\pgfqpoint{4.391721in}{2.832966in}}%
\pgfpathlineto{\pgfqpoint{4.399093in}{2.845301in}}%
\pgfpathlineto{\pgfqpoint{4.406461in}{2.857785in}}%
\pgfpathlineto{\pgfqpoint{4.393481in}{2.861235in}}%
\pgfpathlineto{\pgfqpoint{4.380507in}{2.864858in}}%
\pgfpathlineto{\pgfqpoint{4.367540in}{2.868654in}}%
\pgfpathlineto{\pgfqpoint{4.354580in}{2.872623in}}%
\pgfpathlineto{\pgfqpoint{4.347202in}{2.859718in}}%
\pgfpathlineto{\pgfqpoint{4.339821in}{2.846968in}}%
\pgfpathlineto{\pgfqpoint{4.332437in}{2.834368in}}%
\pgfpathlineto{\pgfqpoint{4.325050in}{2.821914in}}%
\pgfpathclose%
\pgfusepath{fill}%
\end{pgfscope}%
\begin{pgfscope}%
\pgfpathrectangle{\pgfqpoint{1.254980in}{0.150000in}}{\pgfqpoint{5.490039in}{5.490039in}}%
\pgfusepath{clip}%
\pgfsetbuttcap%
\pgfsetroundjoin%
\definecolor{currentfill}{rgb}{0.248629,0.278775,0.534556}%
\pgfsetfillcolor{currentfill}%
\pgfsetfillopacity{0.700000}%
\pgfsetlinewidth{0.000000pt}%
\definecolor{currentstroke}{rgb}{0.000000,0.000000,0.000000}%
\pgfsetstrokecolor{currentstroke}%
\pgfsetdash{}{0pt}%
\pgfpathmoveto{\pgfqpoint{3.363221in}{2.849693in}}%
\pgfpathlineto{\pgfqpoint{3.376104in}{2.837183in}}%
\pgfpathlineto{\pgfqpoint{3.388985in}{2.824911in}}%
\pgfpathlineto{\pgfqpoint{3.401865in}{2.812875in}}%
\pgfpathlineto{\pgfqpoint{3.414743in}{2.801073in}}%
\pgfpathlineto{\pgfqpoint{3.422374in}{2.813332in}}%
\pgfpathlineto{\pgfqpoint{3.430000in}{2.825712in}}%
\pgfpathlineto{\pgfqpoint{3.437620in}{2.838213in}}%
\pgfpathlineto{\pgfqpoint{3.445235in}{2.850840in}}%
\pgfpathlineto{\pgfqpoint{3.432366in}{2.862834in}}%
\pgfpathlineto{\pgfqpoint{3.419495in}{2.875063in}}%
\pgfpathlineto{\pgfqpoint{3.406623in}{2.887528in}}%
\pgfpathlineto{\pgfqpoint{3.393749in}{2.900230in}}%
\pgfpathlineto{\pgfqpoint{3.386125in}{2.887401in}}%
\pgfpathlineto{\pgfqpoint{3.378496in}{2.874703in}}%
\pgfpathlineto{\pgfqpoint{3.370861in}{2.862135in}}%
\pgfpathlineto{\pgfqpoint{3.363221in}{2.849693in}}%
\pgfpathclose%
\pgfusepath{fill}%
\end{pgfscope}%
\begin{pgfscope}%
\pgfpathrectangle{\pgfqpoint{1.254980in}{0.150000in}}{\pgfqpoint{5.490039in}{5.490039in}}%
\pgfusepath{clip}%
\pgfsetbuttcap%
\pgfsetroundjoin%
\definecolor{currentfill}{rgb}{0.177423,0.437527,0.557565}%
\pgfsetfillcolor{currentfill}%
\pgfsetfillopacity{0.700000}%
\pgfsetlinewidth{0.000000pt}%
\definecolor{currentstroke}{rgb}{0.000000,0.000000,0.000000}%
\pgfsetstrokecolor{currentstroke}%
\pgfsetdash{}{0pt}%
\pgfpathmoveto{\pgfqpoint{5.058094in}{3.208884in}}%
\pgfpathlineto{\pgfqpoint{5.071246in}{3.206526in}}%
\pgfpathlineto{\pgfqpoint{5.084407in}{3.204324in}}%
\pgfpathlineto{\pgfqpoint{5.097578in}{3.202276in}}%
\pgfpathlineto{\pgfqpoint{5.110759in}{3.200382in}}%
\pgfpathlineto{\pgfqpoint{5.117996in}{3.215165in}}%
\pgfpathlineto{\pgfqpoint{5.125236in}{3.230247in}}%
\pgfpathlineto{\pgfqpoint{5.132480in}{3.245635in}}%
\pgfpathlineto{\pgfqpoint{5.119312in}{3.248021in}}%
\pgfpathlineto{\pgfqpoint{5.106154in}{3.250561in}}%
\pgfpathlineto{\pgfqpoint{5.093005in}{3.253256in}}%
\pgfpathlineto{\pgfqpoint{5.079865in}{3.256106in}}%
\pgfpathlineto{\pgfqpoint{5.072605in}{3.240055in}}%
\pgfpathlineto{\pgfqpoint{5.065348in}{3.224317in}}%
\pgfpathlineto{\pgfqpoint{5.058094in}{3.208884in}}%
\pgfpathclose%
\pgfusepath{fill}%
\end{pgfscope}%
\begin{pgfscope}%
\pgfpathrectangle{\pgfqpoint{1.254980in}{0.150000in}}{\pgfqpoint{5.490039in}{5.490039in}}%
\pgfusepath{clip}%
\pgfsetbuttcap%
\pgfsetroundjoin%
\definecolor{currentfill}{rgb}{0.267968,0.223549,0.512008}%
\pgfsetfillcolor{currentfill}%
\pgfsetfillopacity{0.700000}%
\pgfsetlinewidth{0.000000pt}%
\definecolor{currentstroke}{rgb}{0.000000,0.000000,0.000000}%
\pgfsetstrokecolor{currentstroke}%
\pgfsetdash{}{0pt}%
\pgfpathmoveto{\pgfqpoint{3.947525in}{2.720119in}}%
\pgfpathlineto{\pgfqpoint{3.960420in}{2.714301in}}%
\pgfpathlineto{\pgfqpoint{3.973318in}{2.708672in}}%
\pgfpathlineto{\pgfqpoint{3.986221in}{2.703232in}}%
\pgfpathlineto{\pgfqpoint{3.999128in}{2.697979in}}%
\pgfpathlineto{\pgfqpoint{4.006608in}{2.709866in}}%
\pgfpathlineto{\pgfqpoint{4.014084in}{2.721858in}}%
\pgfpathlineto{\pgfqpoint{4.021556in}{2.733960in}}%
\pgfpathlineto{\pgfqpoint{4.029024in}{2.746175in}}%
\pgfpathlineto{\pgfqpoint{4.016124in}{2.751729in}}%
\pgfpathlineto{\pgfqpoint{4.003229in}{2.757471in}}%
\pgfpathlineto{\pgfqpoint{3.990339in}{2.763402in}}%
\pgfpathlineto{\pgfqpoint{3.977452in}{2.769522in}}%
\pgfpathlineto{\pgfqpoint{3.969977in}{2.756995in}}%
\pgfpathlineto{\pgfqpoint{3.962497in}{2.744588in}}%
\pgfpathlineto{\pgfqpoint{3.955013in}{2.732297in}}%
\pgfpathlineto{\pgfqpoint{3.947525in}{2.720119in}}%
\pgfpathclose%
\pgfusepath{fill}%
\end{pgfscope}%
\begin{pgfscope}%
\pgfpathrectangle{\pgfqpoint{1.254980in}{0.150000in}}{\pgfqpoint{5.490039in}{5.490039in}}%
\pgfusepath{clip}%
\pgfsetbuttcap%
\pgfsetroundjoin%
\definecolor{currentfill}{rgb}{0.257322,0.256130,0.526563}%
\pgfsetfillcolor{currentfill}%
\pgfsetfillopacity{0.700000}%
\pgfsetlinewidth{0.000000pt}%
\definecolor{currentstroke}{rgb}{0.000000,0.000000,0.000000}%
\pgfsetstrokecolor{currentstroke}%
\pgfsetdash{}{0pt}%
\pgfpathmoveto{\pgfqpoint{4.243620in}{2.787904in}}%
\pgfpathlineto{\pgfqpoint{4.256573in}{2.784028in}}%
\pgfpathlineto{\pgfqpoint{4.269532in}{2.780329in}}%
\pgfpathlineto{\pgfqpoint{4.282498in}{2.776806in}}%
\pgfpathlineto{\pgfqpoint{4.295470in}{2.773458in}}%
\pgfpathlineto{\pgfqpoint{4.302870in}{2.785377in}}%
\pgfpathlineto{\pgfqpoint{4.310267in}{2.797423in}}%
\pgfpathlineto{\pgfqpoint{4.317660in}{2.809601in}}%
\pgfpathlineto{\pgfqpoint{4.325050in}{2.821914in}}%
\pgfpathlineto{\pgfqpoint{4.312087in}{2.825646in}}%
\pgfpathlineto{\pgfqpoint{4.299130in}{2.829554in}}%
\pgfpathlineto{\pgfqpoint{4.286180in}{2.833637in}}%
\pgfpathlineto{\pgfqpoint{4.273236in}{2.837896in}}%
\pgfpathlineto{\pgfqpoint{4.265837in}{2.825188in}}%
\pgfpathlineto{\pgfqpoint{4.258435in}{2.812623in}}%
\pgfpathlineto{\pgfqpoint{4.251029in}{2.800197in}}%
\pgfpathlineto{\pgfqpoint{4.243620in}{2.787904in}}%
\pgfpathclose%
\pgfusepath{fill}%
\end{pgfscope}%
\begin{pgfscope}%
\pgfpathrectangle{\pgfqpoint{1.254980in}{0.150000in}}{\pgfqpoint{5.490039in}{5.490039in}}%
\pgfusepath{clip}%
\pgfsetbuttcap%
\pgfsetroundjoin%
\definecolor{currentfill}{rgb}{0.179019,0.433756,0.557430}%
\pgfsetfillcolor{currentfill}%
\pgfsetfillopacity{0.700000}%
\pgfsetlinewidth{0.000000pt}%
\definecolor{currentstroke}{rgb}{0.000000,0.000000,0.000000}%
\pgfsetstrokecolor{currentstroke}%
\pgfsetdash{}{0pt}%
\pgfpathmoveto{\pgfqpoint{3.052861in}{3.226588in}}%
\pgfpathlineto{\pgfqpoint{3.065859in}{3.207734in}}%
\pgfpathlineto{\pgfqpoint{3.078849in}{3.189172in}}%
\pgfpathlineto{\pgfqpoint{3.091832in}{3.170901in}}%
\pgfpathlineto{\pgfqpoint{3.104807in}{3.152916in}}%
\pgfpathlineto{\pgfqpoint{3.112498in}{3.166275in}}%
\pgfpathlineto{\pgfqpoint{3.120181in}{3.179797in}}%
\pgfpathlineto{\pgfqpoint{3.127858in}{3.193482in}}%
\pgfpathlineto{\pgfqpoint{3.135528in}{3.207334in}}%
\pgfpathlineto{\pgfqpoint{3.122562in}{3.225515in}}%
\pgfpathlineto{\pgfqpoint{3.109590in}{3.243982in}}%
\pgfpathlineto{\pgfqpoint{3.096610in}{3.262739in}}%
\pgfpathlineto{\pgfqpoint{3.083623in}{3.281790in}}%
\pgfpathlineto{\pgfqpoint{3.075943in}{3.267730in}}%
\pgfpathlineto{\pgfqpoint{3.068256in}{3.253845in}}%
\pgfpathlineto{\pgfqpoint{3.060562in}{3.240132in}}%
\pgfpathlineto{\pgfqpoint{3.052861in}{3.226588in}}%
\pgfpathclose%
\pgfusepath{fill}%
\end{pgfscope}%
\begin{pgfscope}%
\pgfpathrectangle{\pgfqpoint{1.254980in}{0.150000in}}{\pgfqpoint{5.490039in}{5.490039in}}%
\pgfusepath{clip}%
\pgfsetbuttcap%
\pgfsetroundjoin%
\definecolor{currentfill}{rgb}{0.257322,0.256130,0.526563}%
\pgfsetfillcolor{currentfill}%
\pgfsetfillopacity{0.700000}%
\pgfsetlinewidth{0.000000pt}%
\definecolor{currentstroke}{rgb}{0.000000,0.000000,0.000000}%
\pgfsetstrokecolor{currentstroke}%
\pgfsetdash{}{0pt}%
\pgfpathmoveto{\pgfqpoint{3.414743in}{2.801073in}}%
\pgfpathlineto{\pgfqpoint{3.427619in}{2.789503in}}%
\pgfpathlineto{\pgfqpoint{3.440495in}{2.778165in}}%
\pgfpathlineto{\pgfqpoint{3.453369in}{2.767055in}}%
\pgfpathlineto{\pgfqpoint{3.466242in}{2.756173in}}%
\pgfpathlineto{\pgfqpoint{3.473865in}{2.768250in}}%
\pgfpathlineto{\pgfqpoint{3.481481in}{2.780440in}}%
\pgfpathlineto{\pgfqpoint{3.489093in}{2.792746in}}%
\pgfpathlineto{\pgfqpoint{3.496699in}{2.805169in}}%
\pgfpathlineto{\pgfqpoint{3.483834in}{2.816244in}}%
\pgfpathlineto{\pgfqpoint{3.470969in}{2.827546in}}%
\pgfpathlineto{\pgfqpoint{3.458102in}{2.839077in}}%
\pgfpathlineto{\pgfqpoint{3.445235in}{2.850840in}}%
\pgfpathlineto{\pgfqpoint{3.437620in}{2.838213in}}%
\pgfpathlineto{\pgfqpoint{3.430000in}{2.825712in}}%
\pgfpathlineto{\pgfqpoint{3.422374in}{2.813332in}}%
\pgfpathlineto{\pgfqpoint{3.414743in}{2.801073in}}%
\pgfpathclose%
\pgfusepath{fill}%
\end{pgfscope}%
\begin{pgfscope}%
\pgfpathrectangle{\pgfqpoint{1.254980in}{0.150000in}}{\pgfqpoint{5.490039in}{5.490039in}}%
\pgfusepath{clip}%
\pgfsetbuttcap%
\pgfsetroundjoin%
\definecolor{currentfill}{rgb}{0.270595,0.214069,0.507052}%
\pgfsetfillcolor{currentfill}%
\pgfsetfillopacity{0.700000}%
\pgfsetlinewidth{0.000000pt}%
\definecolor{currentstroke}{rgb}{0.000000,0.000000,0.000000}%
\pgfsetstrokecolor{currentstroke}%
\pgfsetdash{}{0pt}%
\pgfpathmoveto{\pgfqpoint{3.732817in}{2.705080in}}%
\pgfpathlineto{\pgfqpoint{3.745688in}{2.697427in}}%
\pgfpathlineto{\pgfqpoint{3.758562in}{2.689976in}}%
\pgfpathlineto{\pgfqpoint{3.771438in}{2.682726in}}%
\pgfpathlineto{\pgfqpoint{3.784316in}{2.675677in}}%
\pgfpathlineto{\pgfqpoint{3.791856in}{2.687546in}}%
\pgfpathlineto{\pgfqpoint{3.799391in}{2.699516in}}%
\pgfpathlineto{\pgfqpoint{3.806921in}{2.711589in}}%
\pgfpathlineto{\pgfqpoint{3.814447in}{2.723768in}}%
\pgfpathlineto{\pgfqpoint{3.801576in}{2.731065in}}%
\pgfpathlineto{\pgfqpoint{3.788708in}{2.738561in}}%
\pgfpathlineto{\pgfqpoint{3.775842in}{2.746259in}}%
\pgfpathlineto{\pgfqpoint{3.762979in}{2.754159in}}%
\pgfpathlineto{\pgfqpoint{3.755446in}{2.741723in}}%
\pgfpathlineto{\pgfqpoint{3.747908in}{2.729399in}}%
\pgfpathlineto{\pgfqpoint{3.740365in}{2.717186in}}%
\pgfpathlineto{\pgfqpoint{3.732817in}{2.705080in}}%
\pgfpathclose%
\pgfusepath{fill}%
\end{pgfscope}%
\begin{pgfscope}%
\pgfpathrectangle{\pgfqpoint{1.254980in}{0.150000in}}{\pgfqpoint{5.490039in}{5.490039in}}%
\pgfusepath{clip}%
\pgfsetbuttcap%
\pgfsetroundjoin%
\definecolor{currentfill}{rgb}{0.267968,0.223549,0.512008}%
\pgfsetfillcolor{currentfill}%
\pgfsetfillopacity{0.700000}%
\pgfsetlinewidth{0.000000pt}%
\definecolor{currentstroke}{rgb}{0.000000,0.000000,0.000000}%
\pgfsetstrokecolor{currentstroke}%
\pgfsetdash{}{0pt}%
\pgfpathmoveto{\pgfqpoint{3.599611in}{2.724571in}}%
\pgfpathlineto{\pgfqpoint{3.612477in}{2.715474in}}%
\pgfpathlineto{\pgfqpoint{3.625344in}{2.706589in}}%
\pgfpathlineto{\pgfqpoint{3.638212in}{2.697915in}}%
\pgfpathlineto{\pgfqpoint{3.651082in}{2.689451in}}%
\pgfpathlineto{\pgfqpoint{3.658657in}{2.701367in}}%
\pgfpathlineto{\pgfqpoint{3.666228in}{2.713385in}}%
\pgfpathlineto{\pgfqpoint{3.673793in}{2.725509in}}%
\pgfpathlineto{\pgfqpoint{3.681354in}{2.737741in}}%
\pgfpathlineto{\pgfqpoint{3.668492in}{2.746424in}}%
\pgfpathlineto{\pgfqpoint{3.655632in}{2.755318in}}%
\pgfpathlineto{\pgfqpoint{3.642773in}{2.764422in}}%
\pgfpathlineto{\pgfqpoint{3.629915in}{2.773739in}}%
\pgfpathlineto{\pgfqpoint{3.622347in}{2.761277in}}%
\pgfpathlineto{\pgfqpoint{3.614773in}{2.748930in}}%
\pgfpathlineto{\pgfqpoint{3.607195in}{2.736696in}}%
\pgfpathlineto{\pgfqpoint{3.599611in}{2.724571in}}%
\pgfpathclose%
\pgfusepath{fill}%
\end{pgfscope}%
\begin{pgfscope}%
\pgfpathrectangle{\pgfqpoint{1.254980in}{0.150000in}}{\pgfqpoint{5.490039in}{5.490039in}}%
\pgfusepath{clip}%
\pgfsetbuttcap%
\pgfsetroundjoin%
\definecolor{currentfill}{rgb}{0.262138,0.242286,0.520837}%
\pgfsetfillcolor{currentfill}%
\pgfsetfillopacity{0.700000}%
\pgfsetlinewidth{0.000000pt}%
\definecolor{currentstroke}{rgb}{0.000000,0.000000,0.000000}%
\pgfsetstrokecolor{currentstroke}%
\pgfsetdash{}{0pt}%
\pgfpathmoveto{\pgfqpoint{4.162163in}{2.755837in}}%
\pgfpathlineto{\pgfqpoint{4.175100in}{2.751605in}}%
\pgfpathlineto{\pgfqpoint{4.188044in}{2.747553in}}%
\pgfpathlineto{\pgfqpoint{4.200993in}{2.743679in}}%
\pgfpathlineto{\pgfqpoint{4.213949in}{2.739983in}}%
\pgfpathlineto{\pgfqpoint{4.221372in}{2.751784in}}%
\pgfpathlineto{\pgfqpoint{4.228792in}{2.763702in}}%
\pgfpathlineto{\pgfqpoint{4.236208in}{2.775740in}}%
\pgfpathlineto{\pgfqpoint{4.243620in}{2.787904in}}%
\pgfpathlineto{\pgfqpoint{4.230673in}{2.791956in}}%
\pgfpathlineto{\pgfqpoint{4.217732in}{2.796187in}}%
\pgfpathlineto{\pgfqpoint{4.204798in}{2.800596in}}%
\pgfpathlineto{\pgfqpoint{4.191868in}{2.805184in}}%
\pgfpathlineto{\pgfqpoint{4.184447in}{2.792654in}}%
\pgfpathlineto{\pgfqpoint{4.177023in}{2.780255in}}%
\pgfpathlineto{\pgfqpoint{4.169595in}{2.767984in}}%
\pgfpathlineto{\pgfqpoint{4.162163in}{2.755837in}}%
\pgfpathclose%
\pgfusepath{fill}%
\end{pgfscope}%
\begin{pgfscope}%
\pgfpathrectangle{\pgfqpoint{1.254980in}{0.150000in}}{\pgfqpoint{5.490039in}{5.490039in}}%
\pgfusepath{clip}%
\pgfsetbuttcap%
\pgfsetroundjoin%
\definecolor{currentfill}{rgb}{0.165117,0.467423,0.558141}%
\pgfsetfillcolor{currentfill}%
\pgfsetfillopacity{0.700000}%
\pgfsetlinewidth{0.000000pt}%
\definecolor{currentstroke}{rgb}{0.000000,0.000000,0.000000}%
\pgfsetstrokecolor{currentstroke}%
\pgfsetdash{}{0pt}%
\pgfpathmoveto{\pgfqpoint{3.000790in}{3.304987in}}%
\pgfpathlineto{\pgfqpoint{3.013820in}{3.284933in}}%
\pgfpathlineto{\pgfqpoint{3.026842in}{3.265185in}}%
\pgfpathlineto{\pgfqpoint{3.039856in}{3.245737in}}%
\pgfpathlineto{\pgfqpoint{3.052861in}{3.226588in}}%
\pgfpathlineto{\pgfqpoint{3.060562in}{3.240132in}}%
\pgfpathlineto{\pgfqpoint{3.068256in}{3.253845in}}%
\pgfpathlineto{\pgfqpoint{3.075943in}{3.267730in}}%
\pgfpathlineto{\pgfqpoint{3.083623in}{3.281790in}}%
\pgfpathlineto{\pgfqpoint{3.070628in}{3.301135in}}%
\pgfpathlineto{\pgfqpoint{3.057625in}{3.320780in}}%
\pgfpathlineto{\pgfqpoint{3.044614in}{3.340726in}}%
\pgfpathlineto{\pgfqpoint{3.031594in}{3.360976in}}%
\pgfpathlineto{\pgfqpoint{3.023904in}{3.346709in}}%
\pgfpathlineto{\pgfqpoint{3.016207in}{3.332624in}}%
\pgfpathlineto{\pgfqpoint{3.008502in}{3.318717in}}%
\pgfpathlineto{\pgfqpoint{3.000790in}{3.304987in}}%
\pgfpathclose%
\pgfusepath{fill}%
\end{pgfscope}%
\begin{pgfscope}%
\pgfpathrectangle{\pgfqpoint{1.254980in}{0.150000in}}{\pgfqpoint{5.490039in}{5.490039in}}%
\pgfusepath{clip}%
\pgfsetbuttcap%
\pgfsetroundjoin%
\definecolor{currentfill}{rgb}{0.270595,0.214069,0.507052}%
\pgfsetfillcolor{currentfill}%
\pgfsetfillopacity{0.700000}%
\pgfsetlinewidth{0.000000pt}%
\definecolor{currentstroke}{rgb}{0.000000,0.000000,0.000000}%
\pgfsetstrokecolor{currentstroke}%
\pgfsetdash{}{0pt}%
\pgfpathmoveto{\pgfqpoint{3.865961in}{2.696561in}}%
\pgfpathlineto{\pgfqpoint{3.878847in}{2.690248in}}%
\pgfpathlineto{\pgfqpoint{3.891738in}{2.684129in}}%
\pgfpathlineto{\pgfqpoint{3.904632in}{2.678202in}}%
\pgfpathlineto{\pgfqpoint{3.917530in}{2.672468in}}%
\pgfpathlineto{\pgfqpoint{3.925035in}{2.684228in}}%
\pgfpathlineto{\pgfqpoint{3.932536in}{2.696088in}}%
\pgfpathlineto{\pgfqpoint{3.940033in}{2.708051in}}%
\pgfpathlineto{\pgfqpoint{3.947525in}{2.720119in}}%
\pgfpathlineto{\pgfqpoint{3.934635in}{2.726128in}}%
\pgfpathlineto{\pgfqpoint{3.921749in}{2.732329in}}%
\pgfpathlineto{\pgfqpoint{3.908866in}{2.738723in}}%
\pgfpathlineto{\pgfqpoint{3.895987in}{2.745310in}}%
\pgfpathlineto{\pgfqpoint{3.888487in}{2.732957in}}%
\pgfpathlineto{\pgfqpoint{3.880983in}{2.720716in}}%
\pgfpathlineto{\pgfqpoint{3.873474in}{2.708585in}}%
\pgfpathlineto{\pgfqpoint{3.865961in}{2.696561in}}%
\pgfpathclose%
\pgfusepath{fill}%
\end{pgfscope}%
\begin{pgfscope}%
\pgfpathrectangle{\pgfqpoint{1.254980in}{0.150000in}}{\pgfqpoint{5.490039in}{5.490039in}}%
\pgfusepath{clip}%
\pgfsetbuttcap%
\pgfsetroundjoin%
\definecolor{currentfill}{rgb}{0.263663,0.237631,0.518762}%
\pgfsetfillcolor{currentfill}%
\pgfsetfillopacity{0.700000}%
\pgfsetlinewidth{0.000000pt}%
\definecolor{currentstroke}{rgb}{0.000000,0.000000,0.000000}%
\pgfsetstrokecolor{currentstroke}%
\pgfsetdash{}{0pt}%
\pgfpathmoveto{\pgfqpoint{3.466242in}{2.756173in}}%
\pgfpathlineto{\pgfqpoint{3.479115in}{2.745516in}}%
\pgfpathlineto{\pgfqpoint{3.491988in}{2.735084in}}%
\pgfpathlineto{\pgfqpoint{3.504860in}{2.724874in}}%
\pgfpathlineto{\pgfqpoint{3.517732in}{2.714885in}}%
\pgfpathlineto{\pgfqpoint{3.525346in}{2.726781in}}%
\pgfpathlineto{\pgfqpoint{3.532954in}{2.738782in}}%
\pgfpathlineto{\pgfqpoint{3.540556in}{2.750892in}}%
\pgfpathlineto{\pgfqpoint{3.548154in}{2.763113in}}%
\pgfpathlineto{\pgfqpoint{3.535290in}{2.773294in}}%
\pgfpathlineto{\pgfqpoint{3.522427in}{2.783696in}}%
\pgfpathlineto{\pgfqpoint{3.509563in}{2.794320in}}%
\pgfpathlineto{\pgfqpoint{3.496699in}{2.805169in}}%
\pgfpathlineto{\pgfqpoint{3.489093in}{2.792746in}}%
\pgfpathlineto{\pgfqpoint{3.481481in}{2.780440in}}%
\pgfpathlineto{\pgfqpoint{3.473865in}{2.768250in}}%
\pgfpathlineto{\pgfqpoint{3.466242in}{2.756173in}}%
\pgfpathclose%
\pgfusepath{fill}%
\end{pgfscope}%
\begin{pgfscope}%
\pgfpathrectangle{\pgfqpoint{1.254980in}{0.150000in}}{\pgfqpoint{5.490039in}{5.490039in}}%
\pgfusepath{clip}%
\pgfsetbuttcap%
\pgfsetroundjoin%
\definecolor{currentfill}{rgb}{0.266580,0.228262,0.514349}%
\pgfsetfillcolor{currentfill}%
\pgfsetfillopacity{0.700000}%
\pgfsetlinewidth{0.000000pt}%
\definecolor{currentstroke}{rgb}{0.000000,0.000000,0.000000}%
\pgfsetstrokecolor{currentstroke}%
\pgfsetdash{}{0pt}%
\pgfpathmoveto{\pgfqpoint{4.080669in}{2.725819in}}%
\pgfpathlineto{\pgfqpoint{4.093593in}{2.721191in}}%
\pgfpathlineto{\pgfqpoint{4.106522in}{2.716746in}}%
\pgfpathlineto{\pgfqpoint{4.119457in}{2.712482in}}%
\pgfpathlineto{\pgfqpoint{4.132397in}{2.708399in}}%
\pgfpathlineto{\pgfqpoint{4.139844in}{2.720094in}}%
\pgfpathlineto{\pgfqpoint{4.147288in}{2.731895in}}%
\pgfpathlineto{\pgfqpoint{4.154727in}{2.743809in}}%
\pgfpathlineto{\pgfqpoint{4.162163in}{2.755837in}}%
\pgfpathlineto{\pgfqpoint{4.149231in}{2.760249in}}%
\pgfpathlineto{\pgfqpoint{4.136304in}{2.764842in}}%
\pgfpathlineto{\pgfqpoint{4.123383in}{2.769617in}}%
\pgfpathlineto{\pgfqpoint{4.110468in}{2.774574in}}%
\pgfpathlineto{\pgfqpoint{4.103024in}{2.762207in}}%
\pgfpathlineto{\pgfqpoint{4.095576in}{2.749961in}}%
\pgfpathlineto{\pgfqpoint{4.088125in}{2.737833in}}%
\pgfpathlineto{\pgfqpoint{4.080669in}{2.725819in}}%
\pgfpathclose%
\pgfusepath{fill}%
\end{pgfscope}%
\begin{pgfscope}%
\pgfpathrectangle{\pgfqpoint{1.254980in}{0.150000in}}{\pgfqpoint{5.490039in}{5.490039in}}%
\pgfusepath{clip}%
\pgfsetbuttcap%
\pgfsetroundjoin%
\definecolor{currentfill}{rgb}{0.220057,0.343307,0.549413}%
\pgfsetfillcolor{currentfill}%
\pgfsetfillopacity{0.700000}%
\pgfsetlinewidth{0.000000pt}%
\definecolor{currentstroke}{rgb}{0.000000,0.000000,0.000000}%
\pgfsetstrokecolor{currentstroke}%
\pgfsetdash{}{0pt}%
\pgfpathmoveto{\pgfqpoint{4.702903in}{2.966367in}}%
\pgfpathlineto{\pgfqpoint{4.715982in}{2.964355in}}%
\pgfpathlineto{\pgfqpoint{4.729071in}{2.962505in}}%
\pgfpathlineto{\pgfqpoint{4.742168in}{2.960818in}}%
\pgfpathlineto{\pgfqpoint{4.755274in}{2.959291in}}%
\pgfpathlineto{\pgfqpoint{4.762559in}{2.971625in}}%
\pgfpathlineto{\pgfqpoint{4.769843in}{2.984149in}}%
\pgfpathlineto{\pgfqpoint{4.777126in}{2.996870in}}%
\pgfpathlineto{\pgfqpoint{4.784408in}{3.009794in}}%
\pgfpathlineto{\pgfqpoint{4.771315in}{3.011843in}}%
\pgfpathlineto{\pgfqpoint{4.758230in}{3.014053in}}%
\pgfpathlineto{\pgfqpoint{4.745155in}{3.016425in}}%
\pgfpathlineto{\pgfqpoint{4.732088in}{3.018959in}}%
\pgfpathlineto{\pgfqpoint{4.724793in}{3.005503in}}%
\pgfpathlineto{\pgfqpoint{4.717497in}{2.992256in}}%
\pgfpathlineto{\pgfqpoint{4.710200in}{2.979213in}}%
\pgfpathlineto{\pgfqpoint{4.702903in}{2.966367in}}%
\pgfpathclose%
\pgfusepath{fill}%
\end{pgfscope}%
\begin{pgfscope}%
\pgfpathrectangle{\pgfqpoint{1.254980in}{0.150000in}}{\pgfqpoint{5.490039in}{5.490039in}}%
\pgfusepath{clip}%
\pgfsetbuttcap%
\pgfsetroundjoin%
\definecolor{currentfill}{rgb}{0.210503,0.363727,0.552206}%
\pgfsetfillcolor{currentfill}%
\pgfsetfillopacity{0.700000}%
\pgfsetlinewidth{0.000000pt}%
\definecolor{currentstroke}{rgb}{0.000000,0.000000,0.000000}%
\pgfsetstrokecolor{currentstroke}%
\pgfsetdash{}{0pt}%
\pgfpathmoveto{\pgfqpoint{4.784408in}{3.009794in}}%
\pgfpathlineto{\pgfqpoint{4.797509in}{3.007905in}}%
\pgfpathlineto{\pgfqpoint{4.810620in}{3.006178in}}%
\pgfpathlineto{\pgfqpoint{4.823740in}{3.004610in}}%
\pgfpathlineto{\pgfqpoint{4.836869in}{3.003202in}}%
\pgfpathlineto{\pgfqpoint{4.844136in}{3.015794in}}%
\pgfpathlineto{\pgfqpoint{4.851403in}{3.028595in}}%
\pgfpathlineto{\pgfqpoint{4.858670in}{3.041611in}}%
\pgfpathlineto{\pgfqpoint{4.865937in}{3.054848in}}%
\pgfpathlineto{\pgfqpoint{4.852822in}{3.056807in}}%
\pgfpathlineto{\pgfqpoint{4.839716in}{3.058924in}}%
\pgfpathlineto{\pgfqpoint{4.826619in}{3.061202in}}%
\pgfpathlineto{\pgfqpoint{4.813531in}{3.063640in}}%
\pgfpathlineto{\pgfqpoint{4.806250in}{3.049843in}}%
\pgfpathlineto{\pgfqpoint{4.798970in}{3.036274in}}%
\pgfpathlineto{\pgfqpoint{4.791689in}{3.022926in}}%
\pgfpathlineto{\pgfqpoint{4.784408in}{3.009794in}}%
\pgfpathclose%
\pgfusepath{fill}%
\end{pgfscope}%
\begin{pgfscope}%
\pgfpathrectangle{\pgfqpoint{1.254980in}{0.150000in}}{\pgfqpoint{5.490039in}{5.490039in}}%
\pgfusepath{clip}%
\pgfsetbuttcap%
\pgfsetroundjoin%
\definecolor{currentfill}{rgb}{0.227802,0.326594,0.546532}%
\pgfsetfillcolor{currentfill}%
\pgfsetfillopacity{0.700000}%
\pgfsetlinewidth{0.000000pt}%
\definecolor{currentstroke}{rgb}{0.000000,0.000000,0.000000}%
\pgfsetstrokecolor{currentstroke}%
\pgfsetdash{}{0pt}%
\pgfpathmoveto{\pgfqpoint{4.621414in}{2.924542in}}%
\pgfpathlineto{\pgfqpoint{4.634472in}{2.922370in}}%
\pgfpathlineto{\pgfqpoint{4.647538in}{2.920363in}}%
\pgfpathlineto{\pgfqpoint{4.660613in}{2.918520in}}%
\pgfpathlineto{\pgfqpoint{4.673697in}{2.916840in}}%
\pgfpathlineto{\pgfqpoint{4.681001in}{2.928955in}}%
\pgfpathlineto{\pgfqpoint{4.688303in}{2.941244in}}%
\pgfpathlineto{\pgfqpoint{4.695603in}{2.953713in}}%
\pgfpathlineto{\pgfqpoint{4.702903in}{2.966367in}}%
\pgfpathlineto{\pgfqpoint{4.689831in}{2.968542in}}%
\pgfpathlineto{\pgfqpoint{4.676768in}{2.970880in}}%
\pgfpathlineto{\pgfqpoint{4.663714in}{2.973382in}}%
\pgfpathlineto{\pgfqpoint{4.650668in}{2.976047in}}%
\pgfpathlineto{\pgfqpoint{4.643356in}{2.962888in}}%
\pgfpathlineto{\pgfqpoint{4.636044in}{2.949921in}}%
\pgfpathlineto{\pgfqpoint{4.628730in}{2.937141in}}%
\pgfpathlineto{\pgfqpoint{4.621414in}{2.924542in}}%
\pgfpathclose%
\pgfusepath{fill}%
\end{pgfscope}%
\begin{pgfscope}%
\pgfpathrectangle{\pgfqpoint{1.254980in}{0.150000in}}{\pgfqpoint{5.490039in}{5.490039in}}%
\pgfusepath{clip}%
\pgfsetbuttcap%
\pgfsetroundjoin%
\definecolor{currentfill}{rgb}{0.201239,0.383670,0.554294}%
\pgfsetfillcolor{currentfill}%
\pgfsetfillopacity{0.700000}%
\pgfsetlinewidth{0.000000pt}%
\definecolor{currentstroke}{rgb}{0.000000,0.000000,0.000000}%
\pgfsetstrokecolor{currentstroke}%
\pgfsetdash{}{0pt}%
\pgfpathmoveto{\pgfqpoint{4.865937in}{3.054848in}}%
\pgfpathlineto{\pgfqpoint{4.879061in}{3.053049in}}%
\pgfpathlineto{\pgfqpoint{4.892194in}{3.051408in}}%
\pgfpathlineto{\pgfqpoint{4.905336in}{3.049926in}}%
\pgfpathlineto{\pgfqpoint{4.918488in}{3.048601in}}%
\pgfpathlineto{\pgfqpoint{4.925741in}{3.061498in}}%
\pgfpathlineto{\pgfqpoint{4.932993in}{3.074623in}}%
\pgfpathlineto{\pgfqpoint{4.940245in}{3.087981in}}%
\pgfpathlineto{\pgfqpoint{4.947499in}{3.101580in}}%
\pgfpathlineto{\pgfqpoint{4.934362in}{3.103483in}}%
\pgfpathlineto{\pgfqpoint{4.921234in}{3.105543in}}%
\pgfpathlineto{\pgfqpoint{4.908116in}{3.107761in}}%
\pgfpathlineto{\pgfqpoint{4.895006in}{3.110138in}}%
\pgfpathlineto{\pgfqpoint{4.887738in}{3.095951in}}%
\pgfpathlineto{\pgfqpoint{4.880471in}{3.082012in}}%
\pgfpathlineto{\pgfqpoint{4.873204in}{3.068313in}}%
\pgfpathlineto{\pgfqpoint{4.865937in}{3.054848in}}%
\pgfpathclose%
\pgfusepath{fill}%
\end{pgfscope}%
\begin{pgfscope}%
\pgfpathrectangle{\pgfqpoint{1.254980in}{0.150000in}}{\pgfqpoint{5.490039in}{5.490039in}}%
\pgfusepath{clip}%
\pgfsetbuttcap%
\pgfsetroundjoin%
\definecolor{currentfill}{rgb}{0.235526,0.309527,0.542944}%
\pgfsetfillcolor{currentfill}%
\pgfsetfillopacity{0.700000}%
\pgfsetlinewidth{0.000000pt}%
\definecolor{currentstroke}{rgb}{0.000000,0.000000,0.000000}%
\pgfsetstrokecolor{currentstroke}%
\pgfsetdash{}{0pt}%
\pgfpathmoveto{\pgfqpoint{4.539934in}{2.884312in}}%
\pgfpathlineto{\pgfqpoint{4.552971in}{2.881945in}}%
\pgfpathlineto{\pgfqpoint{4.566016in}{2.879745in}}%
\pgfpathlineto{\pgfqpoint{4.579069in}{2.877710in}}%
\pgfpathlineto{\pgfqpoint{4.592130in}{2.875842in}}%
\pgfpathlineto{\pgfqpoint{4.599454in}{2.887773in}}%
\pgfpathlineto{\pgfqpoint{4.606776in}{2.899863in}}%
\pgfpathlineto{\pgfqpoint{4.614096in}{2.912118in}}%
\pgfpathlineto{\pgfqpoint{4.621414in}{2.924542in}}%
\pgfpathlineto{\pgfqpoint{4.608364in}{2.926878in}}%
\pgfpathlineto{\pgfqpoint{4.595322in}{2.929379in}}%
\pgfpathlineto{\pgfqpoint{4.582288in}{2.932047in}}%
\pgfpathlineto{\pgfqpoint{4.569262in}{2.934880in}}%
\pgfpathlineto{\pgfqpoint{4.561933in}{2.921979in}}%
\pgfpathlineto{\pgfqpoint{4.554602in}{2.909254in}}%
\pgfpathlineto{\pgfqpoint{4.547269in}{2.896700in}}%
\pgfpathlineto{\pgfqpoint{4.539934in}{2.884312in}}%
\pgfpathclose%
\pgfusepath{fill}%
\end{pgfscope}%
\begin{pgfscope}%
\pgfpathrectangle{\pgfqpoint{1.254980in}{0.150000in}}{\pgfqpoint{5.490039in}{5.490039in}}%
\pgfusepath{clip}%
\pgfsetbuttcap%
\pgfsetroundjoin%
\definecolor{currentfill}{rgb}{0.223925,0.334994,0.548053}%
\pgfsetfillcolor{currentfill}%
\pgfsetfillopacity{0.700000}%
\pgfsetlinewidth{0.000000pt}%
\definecolor{currentstroke}{rgb}{0.000000,0.000000,0.000000}%
\pgfsetstrokecolor{currentstroke}%
\pgfsetdash{}{0pt}%
\pgfpathmoveto{\pgfqpoint{3.177647in}{2.968529in}}%
\pgfpathlineto{\pgfqpoint{3.190581in}{2.953179in}}%
\pgfpathlineto{\pgfqpoint{3.203510in}{2.938090in}}%
\pgfpathlineto{\pgfqpoint{3.216435in}{2.923262in}}%
\pgfpathlineto{\pgfqpoint{3.229356in}{2.908691in}}%
\pgfpathlineto{\pgfqpoint{3.237041in}{2.920958in}}%
\pgfpathlineto{\pgfqpoint{3.244720in}{2.933357in}}%
\pgfpathlineto{\pgfqpoint{3.252392in}{2.945890in}}%
\pgfpathlineto{\pgfqpoint{3.260058in}{2.958559in}}%
\pgfpathlineto{\pgfqpoint{3.247147in}{2.973296in}}%
\pgfpathlineto{\pgfqpoint{3.234233in}{2.988290in}}%
\pgfpathlineto{\pgfqpoint{3.221314in}{3.003545in}}%
\pgfpathlineto{\pgfqpoint{3.208390in}{3.019061in}}%
\pgfpathlineto{\pgfqpoint{3.200714in}{3.006216in}}%
\pgfpathlineto{\pgfqpoint{3.193031in}{2.993514in}}%
\pgfpathlineto{\pgfqpoint{3.185342in}{2.980952in}}%
\pgfpathlineto{\pgfqpoint{3.177647in}{2.968529in}}%
\pgfpathclose%
\pgfusepath{fill}%
\end{pgfscope}%
\begin{pgfscope}%
\pgfpathrectangle{\pgfqpoint{1.254980in}{0.150000in}}{\pgfqpoint{5.490039in}{5.490039in}}%
\pgfusepath{clip}%
\pgfsetbuttcap%
\pgfsetroundjoin%
\definecolor{currentfill}{rgb}{0.192357,0.403199,0.555836}%
\pgfsetfillcolor{currentfill}%
\pgfsetfillopacity{0.700000}%
\pgfsetlinewidth{0.000000pt}%
\definecolor{currentstroke}{rgb}{0.000000,0.000000,0.000000}%
\pgfsetstrokecolor{currentstroke}%
\pgfsetdash{}{0pt}%
\pgfpathmoveto{\pgfqpoint{4.947499in}{3.101580in}}%
\pgfpathlineto{\pgfqpoint{4.960645in}{3.099835in}}%
\pgfpathlineto{\pgfqpoint{4.973801in}{3.098247in}}%
\pgfpathlineto{\pgfqpoint{4.986966in}{3.096815in}}%
\pgfpathlineto{\pgfqpoint{5.000141in}{3.095540in}}%
\pgfpathlineto{\pgfqpoint{5.007380in}{3.108792in}}%
\pgfpathlineto{\pgfqpoint{5.014619in}{3.122291in}}%
\pgfpathlineto{\pgfqpoint{5.021860in}{3.136045in}}%
\pgfpathlineto{\pgfqpoint{5.029103in}{3.150061in}}%
\pgfpathlineto{\pgfqpoint{5.015944in}{3.151942in}}%
\pgfpathlineto{\pgfqpoint{5.002795in}{3.153979in}}%
\pgfpathlineto{\pgfqpoint{4.989654in}{3.156172in}}%
\pgfpathlineto{\pgfqpoint{4.976524in}{3.158522in}}%
\pgfpathlineto{\pgfqpoint{4.969265in}{3.143891in}}%
\pgfpathlineto{\pgfqpoint{4.962009in}{3.129529in}}%
\pgfpathlineto{\pgfqpoint{4.954753in}{3.115427in}}%
\pgfpathlineto{\pgfqpoint{4.947499in}{3.101580in}}%
\pgfpathclose%
\pgfusepath{fill}%
\end{pgfscope}%
\begin{pgfscope}%
\pgfpathrectangle{\pgfqpoint{1.254980in}{0.150000in}}{\pgfqpoint{5.490039in}{5.490039in}}%
\pgfusepath{clip}%
\pgfsetbuttcap%
\pgfsetroundjoin%
\definecolor{currentfill}{rgb}{0.235526,0.309527,0.542944}%
\pgfsetfillcolor{currentfill}%
\pgfsetfillopacity{0.700000}%
\pgfsetlinewidth{0.000000pt}%
\definecolor{currentstroke}{rgb}{0.000000,0.000000,0.000000}%
\pgfsetstrokecolor{currentstroke}%
\pgfsetdash{}{0pt}%
\pgfpathmoveto{\pgfqpoint{3.229356in}{2.908691in}}%
\pgfpathlineto{\pgfqpoint{3.242273in}{2.894375in}}%
\pgfpathlineto{\pgfqpoint{3.255186in}{2.880313in}}%
\pgfpathlineto{\pgfqpoint{3.268096in}{2.866502in}}%
\pgfpathlineto{\pgfqpoint{3.281003in}{2.852941in}}%
\pgfpathlineto{\pgfqpoint{3.288677in}{2.865053in}}%
\pgfpathlineto{\pgfqpoint{3.296345in}{2.877290in}}%
\pgfpathlineto{\pgfqpoint{3.304007in}{2.889654in}}%
\pgfpathlineto{\pgfqpoint{3.311663in}{2.902147in}}%
\pgfpathlineto{\pgfqpoint{3.298767in}{2.915874in}}%
\pgfpathlineto{\pgfqpoint{3.285868in}{2.929850in}}%
\pgfpathlineto{\pgfqpoint{3.272965in}{2.944078in}}%
\pgfpathlineto{\pgfqpoint{3.260058in}{2.958559in}}%
\pgfpathlineto{\pgfqpoint{3.252392in}{2.945890in}}%
\pgfpathlineto{\pgfqpoint{3.244720in}{2.933357in}}%
\pgfpathlineto{\pgfqpoint{3.237041in}{2.920958in}}%
\pgfpathlineto{\pgfqpoint{3.229356in}{2.908691in}}%
\pgfpathclose%
\pgfusepath{fill}%
\end{pgfscope}%
\begin{pgfscope}%
\pgfpathrectangle{\pgfqpoint{1.254980in}{0.150000in}}{\pgfqpoint{5.490039in}{5.490039in}}%
\pgfusepath{clip}%
\pgfsetbuttcap%
\pgfsetroundjoin%
\definecolor{currentfill}{rgb}{0.243113,0.292092,0.538516}%
\pgfsetfillcolor{currentfill}%
\pgfsetfillopacity{0.700000}%
\pgfsetlinewidth{0.000000pt}%
\definecolor{currentstroke}{rgb}{0.000000,0.000000,0.000000}%
\pgfsetstrokecolor{currentstroke}%
\pgfsetdash{}{0pt}%
\pgfpathmoveto{\pgfqpoint{4.458455in}{2.845692in}}%
\pgfpathlineto{\pgfqpoint{4.471471in}{2.843094in}}%
\pgfpathlineto{\pgfqpoint{4.484495in}{2.840664in}}%
\pgfpathlineto{\pgfqpoint{4.497528in}{2.838403in}}%
\pgfpathlineto{\pgfqpoint{4.510567in}{2.836309in}}%
\pgfpathlineto{\pgfqpoint{4.517913in}{2.848087in}}%
\pgfpathlineto{\pgfqpoint{4.525256in}{2.860010in}}%
\pgfpathlineto{\pgfqpoint{4.532596in}{2.872083in}}%
\pgfpathlineto{\pgfqpoint{4.539934in}{2.884312in}}%
\pgfpathlineto{\pgfqpoint{4.526904in}{2.886845in}}%
\pgfpathlineto{\pgfqpoint{4.513883in}{2.889546in}}%
\pgfpathlineto{\pgfqpoint{4.500869in}{2.892415in}}%
\pgfpathlineto{\pgfqpoint{4.487863in}{2.895453in}}%
\pgfpathlineto{\pgfqpoint{4.480514in}{2.882775in}}%
\pgfpathlineto{\pgfqpoint{4.473164in}{2.870259in}}%
\pgfpathlineto{\pgfqpoint{4.465810in}{2.857900in}}%
\pgfpathlineto{\pgfqpoint{4.458455in}{2.845692in}}%
\pgfpathclose%
\pgfusepath{fill}%
\end{pgfscope}%
\begin{pgfscope}%
\pgfpathrectangle{\pgfqpoint{1.254980in}{0.150000in}}{\pgfqpoint{5.490039in}{5.490039in}}%
\pgfusepath{clip}%
\pgfsetbuttcap%
\pgfsetroundjoin%
\definecolor{currentfill}{rgb}{0.212395,0.359683,0.551710}%
\pgfsetfillcolor{currentfill}%
\pgfsetfillopacity{0.700000}%
\pgfsetlinewidth{0.000000pt}%
\definecolor{currentstroke}{rgb}{0.000000,0.000000,0.000000}%
\pgfsetstrokecolor{currentstroke}%
\pgfsetdash{}{0pt}%
\pgfpathmoveto{\pgfqpoint{3.125859in}{3.032599in}}%
\pgfpathlineto{\pgfqpoint{3.138814in}{3.016177in}}%
\pgfpathlineto{\pgfqpoint{3.151763in}{3.000026in}}%
\pgfpathlineto{\pgfqpoint{3.164708in}{2.984144in}}%
\pgfpathlineto{\pgfqpoint{3.177647in}{2.968529in}}%
\pgfpathlineto{\pgfqpoint{3.185342in}{2.980952in}}%
\pgfpathlineto{\pgfqpoint{3.193031in}{2.993514in}}%
\pgfpathlineto{\pgfqpoint{3.200714in}{3.006216in}}%
\pgfpathlineto{\pgfqpoint{3.208390in}{3.019061in}}%
\pgfpathlineto{\pgfqpoint{3.195461in}{3.034842in}}%
\pgfpathlineto{\pgfqpoint{3.182528in}{3.050890in}}%
\pgfpathlineto{\pgfqpoint{3.169589in}{3.067207in}}%
\pgfpathlineto{\pgfqpoint{3.156645in}{3.083796in}}%
\pgfpathlineto{\pgfqpoint{3.148958in}{3.070774in}}%
\pgfpathlineto{\pgfqpoint{3.141265in}{3.057902in}}%
\pgfpathlineto{\pgfqpoint{3.133565in}{3.045178in}}%
\pgfpathlineto{\pgfqpoint{3.125859in}{3.032599in}}%
\pgfpathclose%
\pgfusepath{fill}%
\end{pgfscope}%
\begin{pgfscope}%
\pgfpathrectangle{\pgfqpoint{1.254980in}{0.150000in}}{\pgfqpoint{5.490039in}{5.490039in}}%
\pgfusepath{clip}%
\pgfsetbuttcap%
\pgfsetroundjoin%
\definecolor{currentfill}{rgb}{0.271828,0.209303,0.504434}%
\pgfsetfillcolor{currentfill}%
\pgfsetfillopacity{0.700000}%
\pgfsetlinewidth{0.000000pt}%
\definecolor{currentstroke}{rgb}{0.000000,0.000000,0.000000}%
\pgfsetstrokecolor{currentstroke}%
\pgfsetdash{}{0pt}%
\pgfpathmoveto{\pgfqpoint{3.651082in}{2.689451in}}%
\pgfpathlineto{\pgfqpoint{3.663953in}{2.681196in}}%
\pgfpathlineto{\pgfqpoint{3.676826in}{2.673148in}}%
\pgfpathlineto{\pgfqpoint{3.689701in}{2.665306in}}%
\pgfpathlineto{\pgfqpoint{3.702578in}{2.657669in}}%
\pgfpathlineto{\pgfqpoint{3.710145in}{2.669375in}}%
\pgfpathlineto{\pgfqpoint{3.717707in}{2.681177in}}%
\pgfpathlineto{\pgfqpoint{3.725265in}{2.693078in}}%
\pgfpathlineto{\pgfqpoint{3.732817in}{2.705080in}}%
\pgfpathlineto{\pgfqpoint{3.719948in}{2.712937in}}%
\pgfpathlineto{\pgfqpoint{3.707082in}{2.720998in}}%
\pgfpathlineto{\pgfqpoint{3.694217in}{2.729266in}}%
\pgfpathlineto{\pgfqpoint{3.681354in}{2.737741in}}%
\pgfpathlineto{\pgfqpoint{3.673793in}{2.725509in}}%
\pgfpathlineto{\pgfqpoint{3.666228in}{2.713385in}}%
\pgfpathlineto{\pgfqpoint{3.658657in}{2.701367in}}%
\pgfpathlineto{\pgfqpoint{3.651082in}{2.689451in}}%
\pgfpathclose%
\pgfusepath{fill}%
\end{pgfscope}%
\begin{pgfscope}%
\pgfpathrectangle{\pgfqpoint{1.254980in}{0.150000in}}{\pgfqpoint{5.490039in}{5.490039in}}%
\pgfusepath{clip}%
\pgfsetbuttcap%
\pgfsetroundjoin%
\definecolor{currentfill}{rgb}{0.183898,0.422383,0.556944}%
\pgfsetfillcolor{currentfill}%
\pgfsetfillopacity{0.700000}%
\pgfsetlinewidth{0.000000pt}%
\definecolor{currentstroke}{rgb}{0.000000,0.000000,0.000000}%
\pgfsetstrokecolor{currentstroke}%
\pgfsetdash{}{0pt}%
\pgfpathmoveto{\pgfqpoint{5.029103in}{3.150061in}}%
\pgfpathlineto{\pgfqpoint{5.042271in}{3.148336in}}%
\pgfpathlineto{\pgfqpoint{5.055450in}{3.146766in}}%
\pgfpathlineto{\pgfqpoint{5.068638in}{3.145351in}}%
\pgfpathlineto{\pgfqpoint{5.081836in}{3.144091in}}%
\pgfpathlineto{\pgfqpoint{5.089063in}{3.157752in}}%
\pgfpathlineto{\pgfqpoint{5.096293in}{3.171683in}}%
\pgfpathlineto{\pgfqpoint{5.103524in}{3.185891in}}%
\pgfpathlineto{\pgfqpoint{5.110759in}{3.200382in}}%
\pgfpathlineto{\pgfqpoint{5.097578in}{3.202276in}}%
\pgfpathlineto{\pgfqpoint{5.084407in}{3.204324in}}%
\pgfpathlineto{\pgfqpoint{5.071246in}{3.206526in}}%
\pgfpathlineto{\pgfqpoint{5.058094in}{3.208884in}}%
\pgfpathlineto{\pgfqpoint{5.050842in}{3.193750in}}%
\pgfpathlineto{\pgfqpoint{5.043594in}{3.178906in}}%
\pgfpathlineto{\pgfqpoint{5.036347in}{3.164345in}}%
\pgfpathlineto{\pgfqpoint{5.029103in}{3.150061in}}%
\pgfpathclose%
\pgfusepath{fill}%
\end{pgfscope}%
\begin{pgfscope}%
\pgfpathrectangle{\pgfqpoint{1.254980in}{0.150000in}}{\pgfqpoint{5.490039in}{5.490039in}}%
\pgfusepath{clip}%
\pgfsetbuttcap%
\pgfsetroundjoin%
\definecolor{currentfill}{rgb}{0.246811,0.283237,0.535941}%
\pgfsetfillcolor{currentfill}%
\pgfsetfillopacity{0.700000}%
\pgfsetlinewidth{0.000000pt}%
\definecolor{currentstroke}{rgb}{0.000000,0.000000,0.000000}%
\pgfsetstrokecolor{currentstroke}%
\pgfsetdash{}{0pt}%
\pgfpathmoveto{\pgfqpoint{3.281003in}{2.852941in}}%
\pgfpathlineto{\pgfqpoint{3.293906in}{2.839627in}}%
\pgfpathlineto{\pgfqpoint{3.306806in}{2.826557in}}%
\pgfpathlineto{\pgfqpoint{3.319704in}{2.813732in}}%
\pgfpathlineto{\pgfqpoint{3.332599in}{2.801147in}}%
\pgfpathlineto{\pgfqpoint{3.340263in}{2.813105in}}%
\pgfpathlineto{\pgfqpoint{3.347922in}{2.825180in}}%
\pgfpathlineto{\pgfqpoint{3.355574in}{2.837376in}}%
\pgfpathlineto{\pgfqpoint{3.363221in}{2.849693in}}%
\pgfpathlineto{\pgfqpoint{3.350335in}{2.862442in}}%
\pgfpathlineto{\pgfqpoint{3.337447in}{2.875433in}}%
\pgfpathlineto{\pgfqpoint{3.324557in}{2.888668in}}%
\pgfpathlineto{\pgfqpoint{3.311663in}{2.902147in}}%
\pgfpathlineto{\pgfqpoint{3.304007in}{2.889654in}}%
\pgfpathlineto{\pgfqpoint{3.296345in}{2.877290in}}%
\pgfpathlineto{\pgfqpoint{3.288677in}{2.865053in}}%
\pgfpathlineto{\pgfqpoint{3.281003in}{2.852941in}}%
\pgfpathclose%
\pgfusepath{fill}%
\end{pgfscope}%
\begin{pgfscope}%
\pgfpathrectangle{\pgfqpoint{1.254980in}{0.150000in}}{\pgfqpoint{5.490039in}{5.490039in}}%
\pgfusepath{clip}%
\pgfsetbuttcap%
\pgfsetroundjoin%
\definecolor{currentfill}{rgb}{0.269308,0.218818,0.509577}%
\pgfsetfillcolor{currentfill}%
\pgfsetfillopacity{0.700000}%
\pgfsetlinewidth{0.000000pt}%
\definecolor{currentstroke}{rgb}{0.000000,0.000000,0.000000}%
\pgfsetstrokecolor{currentstroke}%
\pgfsetdash{}{0pt}%
\pgfpathmoveto{\pgfqpoint{3.999128in}{2.697979in}}%
\pgfpathlineto{\pgfqpoint{4.012040in}{2.692914in}}%
\pgfpathlineto{\pgfqpoint{4.024957in}{2.688035in}}%
\pgfpathlineto{\pgfqpoint{4.037879in}{2.683341in}}%
\pgfpathlineto{\pgfqpoint{4.050806in}{2.678831in}}%
\pgfpathlineto{\pgfqpoint{4.058278in}{2.690426in}}%
\pgfpathlineto{\pgfqpoint{4.065746in}{2.702119in}}%
\pgfpathlineto{\pgfqpoint{4.073209in}{2.713916in}}%
\pgfpathlineto{\pgfqpoint{4.080669in}{2.725819in}}%
\pgfpathlineto{\pgfqpoint{4.067750in}{2.730631in}}%
\pgfpathlineto{\pgfqpoint{4.054837in}{2.735627in}}%
\pgfpathlineto{\pgfqpoint{4.041928in}{2.740808in}}%
\pgfpathlineto{\pgfqpoint{4.029024in}{2.746175in}}%
\pgfpathlineto{\pgfqpoint{4.021556in}{2.733960in}}%
\pgfpathlineto{\pgfqpoint{4.014084in}{2.721858in}}%
\pgfpathlineto{\pgfqpoint{4.006608in}{2.709866in}}%
\pgfpathlineto{\pgfqpoint{3.999128in}{2.697979in}}%
\pgfpathclose%
\pgfusepath{fill}%
\end{pgfscope}%
\begin{pgfscope}%
\pgfpathrectangle{\pgfqpoint{1.254980in}{0.150000in}}{\pgfqpoint{5.490039in}{5.490039in}}%
\pgfusepath{clip}%
\pgfsetbuttcap%
\pgfsetroundjoin%
\definecolor{currentfill}{rgb}{0.250425,0.274290,0.533103}%
\pgfsetfillcolor{currentfill}%
\pgfsetfillopacity{0.700000}%
\pgfsetlinewidth{0.000000pt}%
\definecolor{currentstroke}{rgb}{0.000000,0.000000,0.000000}%
\pgfsetstrokecolor{currentstroke}%
\pgfsetdash{}{0pt}%
\pgfpathmoveto{\pgfqpoint{4.376969in}{2.808721in}}%
\pgfpathlineto{\pgfqpoint{4.389966in}{2.805854in}}%
\pgfpathlineto{\pgfqpoint{4.402971in}{2.803158in}}%
\pgfpathlineto{\pgfqpoint{4.415983in}{2.800633in}}%
\pgfpathlineto{\pgfqpoint{4.429002in}{2.798277in}}%
\pgfpathlineto{\pgfqpoint{4.436370in}{2.809928in}}%
\pgfpathlineto{\pgfqpoint{4.443734in}{2.821711in}}%
\pgfpathlineto{\pgfqpoint{4.451096in}{2.833631in}}%
\pgfpathlineto{\pgfqpoint{4.458455in}{2.845692in}}%
\pgfpathlineto{\pgfqpoint{4.445445in}{2.848460in}}%
\pgfpathlineto{\pgfqpoint{4.432443in}{2.851398in}}%
\pgfpathlineto{\pgfqpoint{4.419449in}{2.854506in}}%
\pgfpathlineto{\pgfqpoint{4.406461in}{2.857785in}}%
\pgfpathlineto{\pgfqpoint{4.399093in}{2.845301in}}%
\pgfpathlineto{\pgfqpoint{4.391721in}{2.832966in}}%
\pgfpathlineto{\pgfqpoint{4.384347in}{2.820774in}}%
\pgfpathlineto{\pgfqpoint{4.376969in}{2.808721in}}%
\pgfpathclose%
\pgfusepath{fill}%
\end{pgfscope}%
\begin{pgfscope}%
\pgfpathrectangle{\pgfqpoint{1.254980in}{0.150000in}}{\pgfqpoint{5.490039in}{5.490039in}}%
\pgfusepath{clip}%
\pgfsetbuttcap%
\pgfsetroundjoin%
\definecolor{currentfill}{rgb}{0.273006,0.204520,0.501721}%
\pgfsetfillcolor{currentfill}%
\pgfsetfillopacity{0.700000}%
\pgfsetlinewidth{0.000000pt}%
\definecolor{currentstroke}{rgb}{0.000000,0.000000,0.000000}%
\pgfsetstrokecolor{currentstroke}%
\pgfsetdash{}{0pt}%
\pgfpathmoveto{\pgfqpoint{3.784316in}{2.675677in}}%
\pgfpathlineto{\pgfqpoint{3.797198in}{2.668826in}}%
\pgfpathlineto{\pgfqpoint{3.810083in}{2.662173in}}%
\pgfpathlineto{\pgfqpoint{3.822970in}{2.655717in}}%
\pgfpathlineto{\pgfqpoint{3.835862in}{2.649457in}}%
\pgfpathlineto{\pgfqpoint{3.843393in}{2.661090in}}%
\pgfpathlineto{\pgfqpoint{3.850920in}{2.672816in}}%
\pgfpathlineto{\pgfqpoint{3.858443in}{2.684638in}}%
\pgfpathlineto{\pgfqpoint{3.865961in}{2.696561in}}%
\pgfpathlineto{\pgfqpoint{3.853077in}{2.703068in}}%
\pgfpathlineto{\pgfqpoint{3.840197in}{2.709771in}}%
\pgfpathlineto{\pgfqpoint{3.827321in}{2.716671in}}%
\pgfpathlineto{\pgfqpoint{3.814447in}{2.723768in}}%
\pgfpathlineto{\pgfqpoint{3.806921in}{2.711589in}}%
\pgfpathlineto{\pgfqpoint{3.799391in}{2.699516in}}%
\pgfpathlineto{\pgfqpoint{3.791856in}{2.687546in}}%
\pgfpathlineto{\pgfqpoint{3.784316in}{2.675677in}}%
\pgfpathclose%
\pgfusepath{fill}%
\end{pgfscope}%
\begin{pgfscope}%
\pgfpathrectangle{\pgfqpoint{1.254980in}{0.150000in}}{\pgfqpoint{5.490039in}{5.490039in}}%
\pgfusepath{clip}%
\pgfsetbuttcap%
\pgfsetroundjoin%
\definecolor{currentfill}{rgb}{0.197636,0.391528,0.554969}%
\pgfsetfillcolor{currentfill}%
\pgfsetfillopacity{0.700000}%
\pgfsetlinewidth{0.000000pt}%
\definecolor{currentstroke}{rgb}{0.000000,0.000000,0.000000}%
\pgfsetstrokecolor{currentstroke}%
\pgfsetdash{}{0pt}%
\pgfpathmoveto{\pgfqpoint{3.073977in}{3.101053in}}%
\pgfpathlineto{\pgfqpoint{3.086957in}{3.083520in}}%
\pgfpathlineto{\pgfqpoint{3.099931in}{3.066268in}}%
\pgfpathlineto{\pgfqpoint{3.112898in}{3.049295in}}%
\pgfpathlineto{\pgfqpoint{3.125859in}{3.032599in}}%
\pgfpathlineto{\pgfqpoint{3.133565in}{3.045178in}}%
\pgfpathlineto{\pgfqpoint{3.141265in}{3.057902in}}%
\pgfpathlineto{\pgfqpoint{3.148958in}{3.070774in}}%
\pgfpathlineto{\pgfqpoint{3.156645in}{3.083796in}}%
\pgfpathlineto{\pgfqpoint{3.143695in}{3.100659in}}%
\pgfpathlineto{\pgfqpoint{3.130739in}{3.117798in}}%
\pgfpathlineto{\pgfqpoint{3.117776in}{3.135216in}}%
\pgfpathlineto{\pgfqpoint{3.104807in}{3.152916in}}%
\pgfpathlineto{\pgfqpoint{3.097110in}{3.139717in}}%
\pgfpathlineto{\pgfqpoint{3.089406in}{3.126675in}}%
\pgfpathlineto{\pgfqpoint{3.081695in}{3.113787in}}%
\pgfpathlineto{\pgfqpoint{3.073977in}{3.101053in}}%
\pgfpathclose%
\pgfusepath{fill}%
\end{pgfscope}%
\begin{pgfscope}%
\pgfpathrectangle{\pgfqpoint{1.254980in}{0.150000in}}{\pgfqpoint{5.490039in}{5.490039in}}%
\pgfusepath{clip}%
\pgfsetbuttcap%
\pgfsetroundjoin%
\definecolor{currentfill}{rgb}{0.267968,0.223549,0.512008}%
\pgfsetfillcolor{currentfill}%
\pgfsetfillopacity{0.700000}%
\pgfsetlinewidth{0.000000pt}%
\definecolor{currentstroke}{rgb}{0.000000,0.000000,0.000000}%
\pgfsetstrokecolor{currentstroke}%
\pgfsetdash{}{0pt}%
\pgfpathmoveto{\pgfqpoint{3.517732in}{2.714885in}}%
\pgfpathlineto{\pgfqpoint{3.530605in}{2.705117in}}%
\pgfpathlineto{\pgfqpoint{3.543477in}{2.695566in}}%
\pgfpathlineto{\pgfqpoint{3.556351in}{2.686232in}}%
\pgfpathlineto{\pgfqpoint{3.569224in}{2.677113in}}%
\pgfpathlineto{\pgfqpoint{3.576829in}{2.688826in}}%
\pgfpathlineto{\pgfqpoint{3.584428in}{2.700638in}}%
\pgfpathlineto{\pgfqpoint{3.592022in}{2.712552in}}%
\pgfpathlineto{\pgfqpoint{3.599611in}{2.724571in}}%
\pgfpathlineto{\pgfqpoint{3.586746in}{2.733882in}}%
\pgfpathlineto{\pgfqpoint{3.573881in}{2.743408in}}%
\pgfpathlineto{\pgfqpoint{3.561017in}{2.753151in}}%
\pgfpathlineto{\pgfqpoint{3.548154in}{2.763113in}}%
\pgfpathlineto{\pgfqpoint{3.540556in}{2.750892in}}%
\pgfpathlineto{\pgfqpoint{3.532954in}{2.738782in}}%
\pgfpathlineto{\pgfqpoint{3.525346in}{2.726781in}}%
\pgfpathlineto{\pgfqpoint{3.517732in}{2.714885in}}%
\pgfpathclose%
\pgfusepath{fill}%
\end{pgfscope}%
\begin{pgfscope}%
\pgfpathrectangle{\pgfqpoint{1.254980in}{0.150000in}}{\pgfqpoint{5.490039in}{5.490039in}}%
\pgfusepath{clip}%
\pgfsetbuttcap%
\pgfsetroundjoin%
\definecolor{currentfill}{rgb}{0.257322,0.256130,0.526563}%
\pgfsetfillcolor{currentfill}%
\pgfsetfillopacity{0.700000}%
\pgfsetlinewidth{0.000000pt}%
\definecolor{currentstroke}{rgb}{0.000000,0.000000,0.000000}%
\pgfsetstrokecolor{currentstroke}%
\pgfsetdash{}{0pt}%
\pgfpathmoveto{\pgfqpoint{4.295470in}{2.773458in}}%
\pgfpathlineto{\pgfqpoint{4.308449in}{2.770284in}}%
\pgfpathlineto{\pgfqpoint{4.321434in}{2.767284in}}%
\pgfpathlineto{\pgfqpoint{4.334427in}{2.764458in}}%
\pgfpathlineto{\pgfqpoint{4.347426in}{2.761803in}}%
\pgfpathlineto{\pgfqpoint{4.354817in}{2.773348in}}%
\pgfpathlineto{\pgfqpoint{4.362204in}{2.785013in}}%
\pgfpathlineto{\pgfqpoint{4.369588in}{2.796802in}}%
\pgfpathlineto{\pgfqpoint{4.376969in}{2.808721in}}%
\pgfpathlineto{\pgfqpoint{4.363979in}{2.811760in}}%
\pgfpathlineto{\pgfqpoint{4.350996in}{2.814972in}}%
\pgfpathlineto{\pgfqpoint{4.338020in}{2.818356in}}%
\pgfpathlineto{\pgfqpoint{4.325050in}{2.821914in}}%
\pgfpathlineto{\pgfqpoint{4.317660in}{2.809601in}}%
\pgfpathlineto{\pgfqpoint{4.310267in}{2.797423in}}%
\pgfpathlineto{\pgfqpoint{4.302870in}{2.785377in}}%
\pgfpathlineto{\pgfqpoint{4.295470in}{2.773458in}}%
\pgfpathclose%
\pgfusepath{fill}%
\end{pgfscope}%
\begin{pgfscope}%
\pgfpathrectangle{\pgfqpoint{1.254980in}{0.150000in}}{\pgfqpoint{5.490039in}{5.490039in}}%
\pgfusepath{clip}%
\pgfsetbuttcap%
\pgfsetroundjoin%
\definecolor{currentfill}{rgb}{0.255645,0.260703,0.528312}%
\pgfsetfillcolor{currentfill}%
\pgfsetfillopacity{0.700000}%
\pgfsetlinewidth{0.000000pt}%
\definecolor{currentstroke}{rgb}{0.000000,0.000000,0.000000}%
\pgfsetstrokecolor{currentstroke}%
\pgfsetdash{}{0pt}%
\pgfpathmoveto{\pgfqpoint{3.332599in}{2.801147in}}%
\pgfpathlineto{\pgfqpoint{3.345492in}{2.788803in}}%
\pgfpathlineto{\pgfqpoint{3.358383in}{2.776696in}}%
\pgfpathlineto{\pgfqpoint{3.371273in}{2.764825in}}%
\pgfpathlineto{\pgfqpoint{3.384160in}{2.753188in}}%
\pgfpathlineto{\pgfqpoint{3.391814in}{2.764991in}}%
\pgfpathlineto{\pgfqpoint{3.399463in}{2.776904in}}%
\pgfpathlineto{\pgfqpoint{3.407106in}{2.788931in}}%
\pgfpathlineto{\pgfqpoint{3.414743in}{2.801073in}}%
\pgfpathlineto{\pgfqpoint{3.401865in}{2.812875in}}%
\pgfpathlineto{\pgfqpoint{3.388985in}{2.824911in}}%
\pgfpathlineto{\pgfqpoint{3.376104in}{2.837183in}}%
\pgfpathlineto{\pgfqpoint{3.363221in}{2.849693in}}%
\pgfpathlineto{\pgfqpoint{3.355574in}{2.837376in}}%
\pgfpathlineto{\pgfqpoint{3.347922in}{2.825180in}}%
\pgfpathlineto{\pgfqpoint{3.340263in}{2.813105in}}%
\pgfpathlineto{\pgfqpoint{3.332599in}{2.801147in}}%
\pgfpathclose%
\pgfusepath{fill}%
\end{pgfscope}%
\begin{pgfscope}%
\pgfpathrectangle{\pgfqpoint{1.254980in}{0.150000in}}{\pgfqpoint{5.490039in}{5.490039in}}%
\pgfusepath{clip}%
\pgfsetbuttcap%
\pgfsetroundjoin%
\definecolor{currentfill}{rgb}{0.175841,0.441290,0.557685}%
\pgfsetfillcolor{currentfill}%
\pgfsetfillopacity{0.700000}%
\pgfsetlinewidth{0.000000pt}%
\definecolor{currentstroke}{rgb}{0.000000,0.000000,0.000000}%
\pgfsetstrokecolor{currentstroke}%
\pgfsetdash{}{0pt}%
\pgfpathmoveto{\pgfqpoint{5.110759in}{3.200382in}}%
\pgfpathlineto{\pgfqpoint{5.123949in}{3.198643in}}%
\pgfpathlineto{\pgfqpoint{5.137150in}{3.197058in}}%
\pgfpathlineto{\pgfqpoint{5.150360in}{3.195626in}}%
\pgfpathlineto{\pgfqpoint{5.163581in}{3.194347in}}%
\pgfpathlineto{\pgfqpoint{5.170800in}{3.208479in}}%
\pgfpathlineto{\pgfqpoint{5.178023in}{3.222904in}}%
\pgfpathlineto{\pgfqpoint{5.185248in}{3.237628in}}%
\pgfpathlineto{\pgfqpoint{5.172041in}{3.239400in}}%
\pgfpathlineto{\pgfqpoint{5.158844in}{3.241324in}}%
\pgfpathlineto{\pgfqpoint{5.145657in}{3.243403in}}%
\pgfpathlineto{\pgfqpoint{5.132480in}{3.245635in}}%
\pgfpathlineto{\pgfqpoint{5.125236in}{3.230247in}}%
\pgfpathlineto{\pgfqpoint{5.117996in}{3.215165in}}%
\pgfpathlineto{\pgfqpoint{5.110759in}{3.200382in}}%
\pgfpathclose%
\pgfusepath{fill}%
\end{pgfscope}%
\begin{pgfscope}%
\pgfpathrectangle{\pgfqpoint{1.254980in}{0.150000in}}{\pgfqpoint{5.490039in}{5.490039in}}%
\pgfusepath{clip}%
\pgfsetbuttcap%
\pgfsetroundjoin%
\definecolor{currentfill}{rgb}{0.185556,0.418570,0.556753}%
\pgfsetfillcolor{currentfill}%
\pgfsetfillopacity{0.700000}%
\pgfsetlinewidth{0.000000pt}%
\definecolor{currentstroke}{rgb}{0.000000,0.000000,0.000000}%
\pgfsetstrokecolor{currentstroke}%
\pgfsetdash{}{0pt}%
\pgfpathmoveto{\pgfqpoint{3.021986in}{3.174055in}}%
\pgfpathlineto{\pgfqpoint{3.034995in}{3.155368in}}%
\pgfpathlineto{\pgfqpoint{3.047996in}{3.136975in}}%
\pgfpathlineto{\pgfqpoint{3.060990in}{3.118870in}}%
\pgfpathlineto{\pgfqpoint{3.073977in}{3.101053in}}%
\pgfpathlineto{\pgfqpoint{3.081695in}{3.113787in}}%
\pgfpathlineto{\pgfqpoint{3.089406in}{3.126675in}}%
\pgfpathlineto{\pgfqpoint{3.097110in}{3.139717in}}%
\pgfpathlineto{\pgfqpoint{3.104807in}{3.152916in}}%
\pgfpathlineto{\pgfqpoint{3.091832in}{3.170901in}}%
\pgfpathlineto{\pgfqpoint{3.078849in}{3.189172in}}%
\pgfpathlineto{\pgfqpoint{3.065859in}{3.207734in}}%
\pgfpathlineto{\pgfqpoint{3.052861in}{3.226588in}}%
\pgfpathlineto{\pgfqpoint{3.045153in}{3.213210in}}%
\pgfpathlineto{\pgfqpoint{3.037438in}{3.199997in}}%
\pgfpathlineto{\pgfqpoint{3.029715in}{3.186946in}}%
\pgfpathlineto{\pgfqpoint{3.021986in}{3.174055in}}%
\pgfpathclose%
\pgfusepath{fill}%
\end{pgfscope}%
\begin{pgfscope}%
\pgfpathrectangle{\pgfqpoint{1.254980in}{0.150000in}}{\pgfqpoint{5.490039in}{5.490039in}}%
\pgfusepath{clip}%
\pgfsetbuttcap%
\pgfsetroundjoin%
\definecolor{currentfill}{rgb}{0.271828,0.209303,0.504434}%
\pgfsetfillcolor{currentfill}%
\pgfsetfillopacity{0.700000}%
\pgfsetlinewidth{0.000000pt}%
\definecolor{currentstroke}{rgb}{0.000000,0.000000,0.000000}%
\pgfsetstrokecolor{currentstroke}%
\pgfsetdash{}{0pt}%
\pgfpathmoveto{\pgfqpoint{3.917530in}{2.672468in}}%
\pgfpathlineto{\pgfqpoint{3.930432in}{2.666923in}}%
\pgfpathlineto{\pgfqpoint{3.943338in}{2.661569in}}%
\pgfpathlineto{\pgfqpoint{3.956249in}{2.656404in}}%
\pgfpathlineto{\pgfqpoint{3.969165in}{2.651426in}}%
\pgfpathlineto{\pgfqpoint{3.976662in}{2.662922in}}%
\pgfpathlineto{\pgfqpoint{3.984155in}{2.674511in}}%
\pgfpathlineto{\pgfqpoint{3.991644in}{2.686196in}}%
\pgfpathlineto{\pgfqpoint{3.999128in}{2.697979in}}%
\pgfpathlineto{\pgfqpoint{3.986221in}{2.703232in}}%
\pgfpathlineto{\pgfqpoint{3.973318in}{2.708672in}}%
\pgfpathlineto{\pgfqpoint{3.960420in}{2.714301in}}%
\pgfpathlineto{\pgfqpoint{3.947525in}{2.720119in}}%
\pgfpathlineto{\pgfqpoint{3.940033in}{2.708051in}}%
\pgfpathlineto{\pgfqpoint{3.932536in}{2.696088in}}%
\pgfpathlineto{\pgfqpoint{3.925035in}{2.684228in}}%
\pgfpathlineto{\pgfqpoint{3.917530in}{2.672468in}}%
\pgfpathclose%
\pgfusepath{fill}%
\end{pgfscope}%
\begin{pgfscope}%
\pgfpathrectangle{\pgfqpoint{1.254980in}{0.150000in}}{\pgfqpoint{5.490039in}{5.490039in}}%
\pgfusepath{clip}%
\pgfsetbuttcap%
\pgfsetroundjoin%
\definecolor{currentfill}{rgb}{0.262138,0.242286,0.520837}%
\pgfsetfillcolor{currentfill}%
\pgfsetfillopacity{0.700000}%
\pgfsetlinewidth{0.000000pt}%
\definecolor{currentstroke}{rgb}{0.000000,0.000000,0.000000}%
\pgfsetstrokecolor{currentstroke}%
\pgfsetdash{}{0pt}%
\pgfpathmoveto{\pgfqpoint{4.213949in}{2.739983in}}%
\pgfpathlineto{\pgfqpoint{4.226910in}{2.736464in}}%
\pgfpathlineto{\pgfqpoint{4.239878in}{2.733122in}}%
\pgfpathlineto{\pgfqpoint{4.252853in}{2.729956in}}%
\pgfpathlineto{\pgfqpoint{4.265834in}{2.726966in}}%
\pgfpathlineto{\pgfqpoint{4.273248in}{2.738420in}}%
\pgfpathlineto{\pgfqpoint{4.280659in}{2.749984in}}%
\pgfpathlineto{\pgfqpoint{4.288066in}{2.761662in}}%
\pgfpathlineto{\pgfqpoint{4.295470in}{2.773458in}}%
\pgfpathlineto{\pgfqpoint{4.282498in}{2.776806in}}%
\pgfpathlineto{\pgfqpoint{4.269532in}{2.780329in}}%
\pgfpathlineto{\pgfqpoint{4.256573in}{2.784028in}}%
\pgfpathlineto{\pgfqpoint{4.243620in}{2.787904in}}%
\pgfpathlineto{\pgfqpoint{4.236208in}{2.775740in}}%
\pgfpathlineto{\pgfqpoint{4.228792in}{2.763702in}}%
\pgfpathlineto{\pgfqpoint{4.221372in}{2.751784in}}%
\pgfpathlineto{\pgfqpoint{4.213949in}{2.739983in}}%
\pgfpathclose%
\pgfusepath{fill}%
\end{pgfscope}%
\begin{pgfscope}%
\pgfpathrectangle{\pgfqpoint{1.254980in}{0.150000in}}{\pgfqpoint{5.490039in}{5.490039in}}%
\pgfusepath{clip}%
\pgfsetbuttcap%
\pgfsetroundjoin%
\definecolor{currentfill}{rgb}{0.262138,0.242286,0.520837}%
\pgfsetfillcolor{currentfill}%
\pgfsetfillopacity{0.700000}%
\pgfsetlinewidth{0.000000pt}%
\definecolor{currentstroke}{rgb}{0.000000,0.000000,0.000000}%
\pgfsetstrokecolor{currentstroke}%
\pgfsetdash{}{0pt}%
\pgfpathmoveto{\pgfqpoint{3.384160in}{2.753188in}}%
\pgfpathlineto{\pgfqpoint{3.397046in}{2.741783in}}%
\pgfpathlineto{\pgfqpoint{3.409931in}{2.730609in}}%
\pgfpathlineto{\pgfqpoint{3.422815in}{2.719665in}}%
\pgfpathlineto{\pgfqpoint{3.435698in}{2.708947in}}%
\pgfpathlineto{\pgfqpoint{3.443342in}{2.720595in}}%
\pgfpathlineto{\pgfqpoint{3.450981in}{2.732348in}}%
\pgfpathlineto{\pgfqpoint{3.458615in}{2.744206in}}%
\pgfpathlineto{\pgfqpoint{3.466242in}{2.756173in}}%
\pgfpathlineto{\pgfqpoint{3.453369in}{2.767055in}}%
\pgfpathlineto{\pgfqpoint{3.440495in}{2.778165in}}%
\pgfpathlineto{\pgfqpoint{3.427619in}{2.789503in}}%
\pgfpathlineto{\pgfqpoint{3.414743in}{2.801073in}}%
\pgfpathlineto{\pgfqpoint{3.407106in}{2.788931in}}%
\pgfpathlineto{\pgfqpoint{3.399463in}{2.776904in}}%
\pgfpathlineto{\pgfqpoint{3.391814in}{2.764991in}}%
\pgfpathlineto{\pgfqpoint{3.384160in}{2.753188in}}%
\pgfpathclose%
\pgfusepath{fill}%
\end{pgfscope}%
\begin{pgfscope}%
\pgfpathrectangle{\pgfqpoint{1.254980in}{0.150000in}}{\pgfqpoint{5.490039in}{5.490039in}}%
\pgfusepath{clip}%
\pgfsetbuttcap%
\pgfsetroundjoin%
\definecolor{currentfill}{rgb}{0.274128,0.199721,0.498911}%
\pgfsetfillcolor{currentfill}%
\pgfsetfillopacity{0.700000}%
\pgfsetlinewidth{0.000000pt}%
\definecolor{currentstroke}{rgb}{0.000000,0.000000,0.000000}%
\pgfsetstrokecolor{currentstroke}%
\pgfsetdash{}{0pt}%
\pgfpathmoveto{\pgfqpoint{3.702578in}{2.657669in}}%
\pgfpathlineto{\pgfqpoint{3.715457in}{2.650235in}}%
\pgfpathlineto{\pgfqpoint{3.728339in}{2.643004in}}%
\pgfpathlineto{\pgfqpoint{3.741223in}{2.635974in}}%
\pgfpathlineto{\pgfqpoint{3.754110in}{2.629144in}}%
\pgfpathlineto{\pgfqpoint{3.761669in}{2.640641in}}%
\pgfpathlineto{\pgfqpoint{3.769223in}{2.652227in}}%
\pgfpathlineto{\pgfqpoint{3.776772in}{2.663905in}}%
\pgfpathlineto{\pgfqpoint{3.784316in}{2.675677in}}%
\pgfpathlineto{\pgfqpoint{3.771438in}{2.682726in}}%
\pgfpathlineto{\pgfqpoint{3.758562in}{2.689976in}}%
\pgfpathlineto{\pgfqpoint{3.745688in}{2.697427in}}%
\pgfpathlineto{\pgfqpoint{3.732817in}{2.705080in}}%
\pgfpathlineto{\pgfqpoint{3.725265in}{2.693078in}}%
\pgfpathlineto{\pgfqpoint{3.717707in}{2.681177in}}%
\pgfpathlineto{\pgfqpoint{3.710145in}{2.669375in}}%
\pgfpathlineto{\pgfqpoint{3.702578in}{2.657669in}}%
\pgfpathclose%
\pgfusepath{fill}%
\end{pgfscope}%
\begin{pgfscope}%
\pgfpathrectangle{\pgfqpoint{1.254980in}{0.150000in}}{\pgfqpoint{5.490039in}{5.490039in}}%
\pgfusepath{clip}%
\pgfsetbuttcap%
\pgfsetroundjoin%
\definecolor{currentfill}{rgb}{0.271828,0.209303,0.504434}%
\pgfsetfillcolor{currentfill}%
\pgfsetfillopacity{0.700000}%
\pgfsetlinewidth{0.000000pt}%
\definecolor{currentstroke}{rgb}{0.000000,0.000000,0.000000}%
\pgfsetstrokecolor{currentstroke}%
\pgfsetdash{}{0pt}%
\pgfpathmoveto{\pgfqpoint{3.569224in}{2.677113in}}%
\pgfpathlineto{\pgfqpoint{3.582099in}{2.668208in}}%
\pgfpathlineto{\pgfqpoint{3.594975in}{2.659515in}}%
\pgfpathlineto{\pgfqpoint{3.607852in}{2.651034in}}%
\pgfpathlineto{\pgfqpoint{3.620730in}{2.642763in}}%
\pgfpathlineto{\pgfqpoint{3.628326in}{2.654294in}}%
\pgfpathlineto{\pgfqpoint{3.635916in}{2.665917in}}%
\pgfpathlineto{\pgfqpoint{3.643502in}{2.677636in}}%
\pgfpathlineto{\pgfqpoint{3.651082in}{2.689451in}}%
\pgfpathlineto{\pgfqpoint{3.638212in}{2.697915in}}%
\pgfpathlineto{\pgfqpoint{3.625344in}{2.706589in}}%
\pgfpathlineto{\pgfqpoint{3.612477in}{2.715474in}}%
\pgfpathlineto{\pgfqpoint{3.599611in}{2.724571in}}%
\pgfpathlineto{\pgfqpoint{3.592022in}{2.712552in}}%
\pgfpathlineto{\pgfqpoint{3.584428in}{2.700638in}}%
\pgfpathlineto{\pgfqpoint{3.576829in}{2.688826in}}%
\pgfpathlineto{\pgfqpoint{3.569224in}{2.677113in}}%
\pgfpathclose%
\pgfusepath{fill}%
\end{pgfscope}%
\begin{pgfscope}%
\pgfpathrectangle{\pgfqpoint{1.254980in}{0.150000in}}{\pgfqpoint{5.490039in}{5.490039in}}%
\pgfusepath{clip}%
\pgfsetbuttcap%
\pgfsetroundjoin%
\definecolor{currentfill}{rgb}{0.266580,0.228262,0.514349}%
\pgfsetfillcolor{currentfill}%
\pgfsetfillopacity{0.700000}%
\pgfsetlinewidth{0.000000pt}%
\definecolor{currentstroke}{rgb}{0.000000,0.000000,0.000000}%
\pgfsetstrokecolor{currentstroke}%
\pgfsetdash{}{0pt}%
\pgfpathmoveto{\pgfqpoint{4.132397in}{2.708399in}}%
\pgfpathlineto{\pgfqpoint{4.145343in}{2.704497in}}%
\pgfpathlineto{\pgfqpoint{4.158295in}{2.700774in}}%
\pgfpathlineto{\pgfqpoint{4.171253in}{2.697230in}}%
\pgfpathlineto{\pgfqpoint{4.184217in}{2.693864in}}%
\pgfpathlineto{\pgfqpoint{4.191656in}{2.705239in}}%
\pgfpathlineto{\pgfqpoint{4.199091in}{2.716714in}}%
\pgfpathlineto{\pgfqpoint{4.206522in}{2.728294in}}%
\pgfpathlineto{\pgfqpoint{4.213949in}{2.739983in}}%
\pgfpathlineto{\pgfqpoint{4.200993in}{2.743679in}}%
\pgfpathlineto{\pgfqpoint{4.188044in}{2.747553in}}%
\pgfpathlineto{\pgfqpoint{4.175100in}{2.751605in}}%
\pgfpathlineto{\pgfqpoint{4.162163in}{2.755837in}}%
\pgfpathlineto{\pgfqpoint{4.154727in}{2.743809in}}%
\pgfpathlineto{\pgfqpoint{4.147288in}{2.731895in}}%
\pgfpathlineto{\pgfqpoint{4.139844in}{2.720094in}}%
\pgfpathlineto{\pgfqpoint{4.132397in}{2.708399in}}%
\pgfpathclose%
\pgfusepath{fill}%
\end{pgfscope}%
\begin{pgfscope}%
\pgfpathrectangle{\pgfqpoint{1.254980in}{0.150000in}}{\pgfqpoint{5.490039in}{5.490039in}}%
\pgfusepath{clip}%
\pgfsetbuttcap%
\pgfsetroundjoin%
\definecolor{currentfill}{rgb}{0.171176,0.452530,0.557965}%
\pgfsetfillcolor{currentfill}%
\pgfsetfillopacity{0.700000}%
\pgfsetlinewidth{0.000000pt}%
\definecolor{currentstroke}{rgb}{0.000000,0.000000,0.000000}%
\pgfsetstrokecolor{currentstroke}%
\pgfsetdash{}{0pt}%
\pgfpathmoveto{\pgfqpoint{2.969867in}{3.251782in}}%
\pgfpathlineto{\pgfqpoint{2.982910in}{3.231897in}}%
\pgfpathlineto{\pgfqpoint{2.995943in}{3.212316in}}%
\pgfpathlineto{\pgfqpoint{3.008969in}{3.193036in}}%
\pgfpathlineto{\pgfqpoint{3.021986in}{3.174055in}}%
\pgfpathlineto{\pgfqpoint{3.029715in}{3.186946in}}%
\pgfpathlineto{\pgfqpoint{3.037438in}{3.199997in}}%
\pgfpathlineto{\pgfqpoint{3.045153in}{3.213210in}}%
\pgfpathlineto{\pgfqpoint{3.052861in}{3.226588in}}%
\pgfpathlineto{\pgfqpoint{3.039856in}{3.245737in}}%
\pgfpathlineto{\pgfqpoint{3.026842in}{3.265185in}}%
\pgfpathlineto{\pgfqpoint{3.013820in}{3.284933in}}%
\pgfpathlineto{\pgfqpoint{3.000790in}{3.304987in}}%
\pgfpathlineto{\pgfqpoint{2.993070in}{3.291430in}}%
\pgfpathlineto{\pgfqpoint{2.985343in}{3.278046in}}%
\pgfpathlineto{\pgfqpoint{2.977609in}{3.264830in}}%
\pgfpathlineto{\pgfqpoint{2.969867in}{3.251782in}}%
\pgfpathclose%
\pgfusepath{fill}%
\end{pgfscope}%
\begin{pgfscope}%
\pgfpathrectangle{\pgfqpoint{1.254980in}{0.150000in}}{\pgfqpoint{5.490039in}{5.490039in}}%
\pgfusepath{clip}%
\pgfsetbuttcap%
\pgfsetroundjoin%
\definecolor{currentfill}{rgb}{0.274128,0.199721,0.498911}%
\pgfsetfillcolor{currentfill}%
\pgfsetfillopacity{0.700000}%
\pgfsetlinewidth{0.000000pt}%
\definecolor{currentstroke}{rgb}{0.000000,0.000000,0.000000}%
\pgfsetstrokecolor{currentstroke}%
\pgfsetdash{}{0pt}%
\pgfpathmoveto{\pgfqpoint{3.835862in}{2.649457in}}%
\pgfpathlineto{\pgfqpoint{3.848756in}{2.643392in}}%
\pgfpathlineto{\pgfqpoint{3.861655in}{2.637520in}}%
\pgfpathlineto{\pgfqpoint{3.874557in}{2.631840in}}%
\pgfpathlineto{\pgfqpoint{3.887463in}{2.626353in}}%
\pgfpathlineto{\pgfqpoint{3.894987in}{2.637748in}}%
\pgfpathlineto{\pgfqpoint{3.902505in}{2.649231in}}%
\pgfpathlineto{\pgfqpoint{3.910020in}{2.660803in}}%
\pgfpathlineto{\pgfqpoint{3.917530in}{2.672468in}}%
\pgfpathlineto{\pgfqpoint{3.904632in}{2.678202in}}%
\pgfpathlineto{\pgfqpoint{3.891738in}{2.684129in}}%
\pgfpathlineto{\pgfqpoint{3.878847in}{2.690248in}}%
\pgfpathlineto{\pgfqpoint{3.865961in}{2.696561in}}%
\pgfpathlineto{\pgfqpoint{3.858443in}{2.684638in}}%
\pgfpathlineto{\pgfqpoint{3.850920in}{2.672816in}}%
\pgfpathlineto{\pgfqpoint{3.843393in}{2.661090in}}%
\pgfpathlineto{\pgfqpoint{3.835862in}{2.649457in}}%
\pgfpathclose%
\pgfusepath{fill}%
\end{pgfscope}%
\begin{pgfscope}%
\pgfpathrectangle{\pgfqpoint{1.254980in}{0.150000in}}{\pgfqpoint{5.490039in}{5.490039in}}%
\pgfusepath{clip}%
\pgfsetbuttcap%
\pgfsetroundjoin%
\definecolor{currentfill}{rgb}{0.218130,0.347432,0.550038}%
\pgfsetfillcolor{currentfill}%
\pgfsetfillopacity{0.700000}%
\pgfsetlinewidth{0.000000pt}%
\definecolor{currentstroke}{rgb}{0.000000,0.000000,0.000000}%
\pgfsetstrokecolor{currentstroke}%
\pgfsetdash{}{0pt}%
\pgfpathmoveto{\pgfqpoint{4.755274in}{2.959291in}}%
\pgfpathlineto{\pgfqpoint{4.768389in}{2.957926in}}%
\pgfpathlineto{\pgfqpoint{4.781513in}{2.956721in}}%
\pgfpathlineto{\pgfqpoint{4.794647in}{2.955676in}}%
\pgfpathlineto{\pgfqpoint{4.807790in}{2.954792in}}%
\pgfpathlineto{\pgfqpoint{4.815061in}{2.966612in}}%
\pgfpathlineto{\pgfqpoint{4.822331in}{2.978617in}}%
\pgfpathlineto{\pgfqpoint{4.829600in}{2.990811in}}%
\pgfpathlineto{\pgfqpoint{4.836869in}{3.003202in}}%
\pgfpathlineto{\pgfqpoint{4.823740in}{3.004610in}}%
\pgfpathlineto{\pgfqpoint{4.810620in}{3.006178in}}%
\pgfpathlineto{\pgfqpoint{4.797509in}{3.007905in}}%
\pgfpathlineto{\pgfqpoint{4.784408in}{3.009794in}}%
\pgfpathlineto{\pgfqpoint{4.777126in}{2.996870in}}%
\pgfpathlineto{\pgfqpoint{4.769843in}{2.984149in}}%
\pgfpathlineto{\pgfqpoint{4.762559in}{2.971625in}}%
\pgfpathlineto{\pgfqpoint{4.755274in}{2.959291in}}%
\pgfpathclose%
\pgfusepath{fill}%
\end{pgfscope}%
\begin{pgfscope}%
\pgfpathrectangle{\pgfqpoint{1.254980in}{0.150000in}}{\pgfqpoint{5.490039in}{5.490039in}}%
\pgfusepath{clip}%
\pgfsetbuttcap%
\pgfsetroundjoin%
\definecolor{currentfill}{rgb}{0.267968,0.223549,0.512008}%
\pgfsetfillcolor{currentfill}%
\pgfsetfillopacity{0.700000}%
\pgfsetlinewidth{0.000000pt}%
\definecolor{currentstroke}{rgb}{0.000000,0.000000,0.000000}%
\pgfsetstrokecolor{currentstroke}%
\pgfsetdash{}{0pt}%
\pgfpathmoveto{\pgfqpoint{3.435698in}{2.708947in}}%
\pgfpathlineto{\pgfqpoint{3.448580in}{2.698456in}}%
\pgfpathlineto{\pgfqpoint{3.461462in}{2.688188in}}%
\pgfpathlineto{\pgfqpoint{3.474344in}{2.678143in}}%
\pgfpathlineto{\pgfqpoint{3.487225in}{2.668320in}}%
\pgfpathlineto{\pgfqpoint{3.494860in}{2.679813in}}%
\pgfpathlineto{\pgfqpoint{3.502490in}{2.691404in}}%
\pgfpathlineto{\pgfqpoint{3.510114in}{2.703094in}}%
\pgfpathlineto{\pgfqpoint{3.517732in}{2.714885in}}%
\pgfpathlineto{\pgfqpoint{3.504860in}{2.724874in}}%
\pgfpathlineto{\pgfqpoint{3.491988in}{2.735084in}}%
\pgfpathlineto{\pgfqpoint{3.479115in}{2.745516in}}%
\pgfpathlineto{\pgfqpoint{3.466242in}{2.756173in}}%
\pgfpathlineto{\pgfqpoint{3.458615in}{2.744206in}}%
\pgfpathlineto{\pgfqpoint{3.450981in}{2.732348in}}%
\pgfpathlineto{\pgfqpoint{3.443342in}{2.720595in}}%
\pgfpathlineto{\pgfqpoint{3.435698in}{2.708947in}}%
\pgfpathclose%
\pgfusepath{fill}%
\end{pgfscope}%
\begin{pgfscope}%
\pgfpathrectangle{\pgfqpoint{1.254980in}{0.150000in}}{\pgfqpoint{5.490039in}{5.490039in}}%
\pgfusepath{clip}%
\pgfsetbuttcap%
\pgfsetroundjoin%
\definecolor{currentfill}{rgb}{0.225863,0.330805,0.547314}%
\pgfsetfillcolor{currentfill}%
\pgfsetfillopacity{0.700000}%
\pgfsetlinewidth{0.000000pt}%
\definecolor{currentstroke}{rgb}{0.000000,0.000000,0.000000}%
\pgfsetstrokecolor{currentstroke}%
\pgfsetdash{}{0pt}%
\pgfpathmoveto{\pgfqpoint{4.673697in}{2.916840in}}%
\pgfpathlineto{\pgfqpoint{4.686789in}{2.915323in}}%
\pgfpathlineto{\pgfqpoint{4.699890in}{2.913969in}}%
\pgfpathlineto{\pgfqpoint{4.713000in}{2.912777in}}%
\pgfpathlineto{\pgfqpoint{4.726119in}{2.911746in}}%
\pgfpathlineto{\pgfqpoint{4.733410in}{2.923375in}}%
\pgfpathlineto{\pgfqpoint{4.740700in}{2.935172in}}%
\pgfpathlineto{\pgfqpoint{4.747988in}{2.947142in}}%
\pgfpathlineto{\pgfqpoint{4.755274in}{2.959291in}}%
\pgfpathlineto{\pgfqpoint{4.742168in}{2.960818in}}%
\pgfpathlineto{\pgfqpoint{4.729071in}{2.962505in}}%
\pgfpathlineto{\pgfqpoint{4.715982in}{2.964355in}}%
\pgfpathlineto{\pgfqpoint{4.702903in}{2.966367in}}%
\pgfpathlineto{\pgfqpoint{4.695603in}{2.953713in}}%
\pgfpathlineto{\pgfqpoint{4.688303in}{2.941244in}}%
\pgfpathlineto{\pgfqpoint{4.681001in}{2.928955in}}%
\pgfpathlineto{\pgfqpoint{4.673697in}{2.916840in}}%
\pgfpathclose%
\pgfusepath{fill}%
\end{pgfscope}%
\begin{pgfscope}%
\pgfpathrectangle{\pgfqpoint{1.254980in}{0.150000in}}{\pgfqpoint{5.490039in}{5.490039in}}%
\pgfusepath{clip}%
\pgfsetbuttcap%
\pgfsetroundjoin%
\definecolor{currentfill}{rgb}{0.208623,0.367752,0.552675}%
\pgfsetfillcolor{currentfill}%
\pgfsetfillopacity{0.700000}%
\pgfsetlinewidth{0.000000pt}%
\definecolor{currentstroke}{rgb}{0.000000,0.000000,0.000000}%
\pgfsetstrokecolor{currentstroke}%
\pgfsetdash{}{0pt}%
\pgfpathmoveto{\pgfqpoint{4.836869in}{3.003202in}}%
\pgfpathlineto{\pgfqpoint{4.850007in}{3.001953in}}%
\pgfpathlineto{\pgfqpoint{4.863155in}{3.000862in}}%
\pgfpathlineto{\pgfqpoint{4.876312in}{2.999931in}}%
\pgfpathlineto{\pgfqpoint{4.889479in}{2.999157in}}%
\pgfpathlineto{\pgfqpoint{4.896732in}{3.011209in}}%
\pgfpathlineto{\pgfqpoint{4.903984in}{3.023463in}}%
\pgfpathlineto{\pgfqpoint{4.911237in}{3.035925in}}%
\pgfpathlineto{\pgfqpoint{4.918488in}{3.048601in}}%
\pgfpathlineto{\pgfqpoint{4.905336in}{3.049926in}}%
\pgfpathlineto{\pgfqpoint{4.892194in}{3.051408in}}%
\pgfpathlineto{\pgfqpoint{4.879061in}{3.053049in}}%
\pgfpathlineto{\pgfqpoint{4.865937in}{3.054848in}}%
\pgfpathlineto{\pgfqpoint{4.858670in}{3.041611in}}%
\pgfpathlineto{\pgfqpoint{4.851403in}{3.028595in}}%
\pgfpathlineto{\pgfqpoint{4.844136in}{3.015794in}}%
\pgfpathlineto{\pgfqpoint{4.836869in}{3.003202in}}%
\pgfpathclose%
\pgfusepath{fill}%
\end{pgfscope}%
\begin{pgfscope}%
\pgfpathrectangle{\pgfqpoint{1.254980in}{0.150000in}}{\pgfqpoint{5.490039in}{5.490039in}}%
\pgfusepath{clip}%
\pgfsetbuttcap%
\pgfsetroundjoin%
\definecolor{currentfill}{rgb}{0.270595,0.214069,0.507052}%
\pgfsetfillcolor{currentfill}%
\pgfsetfillopacity{0.700000}%
\pgfsetlinewidth{0.000000pt}%
\definecolor{currentstroke}{rgb}{0.000000,0.000000,0.000000}%
\pgfsetstrokecolor{currentstroke}%
\pgfsetdash{}{0pt}%
\pgfpathmoveto{\pgfqpoint{4.050806in}{2.678831in}}%
\pgfpathlineto{\pgfqpoint{4.063738in}{2.674505in}}%
\pgfpathlineto{\pgfqpoint{4.076675in}{2.670362in}}%
\pgfpathlineto{\pgfqpoint{4.089618in}{2.666400in}}%
\pgfpathlineto{\pgfqpoint{4.102567in}{2.662620in}}%
\pgfpathlineto{\pgfqpoint{4.110031in}{2.673922in}}%
\pgfpathlineto{\pgfqpoint{4.117490in}{2.685317in}}%
\pgfpathlineto{\pgfqpoint{4.124946in}{2.696808in}}%
\pgfpathlineto{\pgfqpoint{4.132397in}{2.708399in}}%
\pgfpathlineto{\pgfqpoint{4.119457in}{2.712482in}}%
\pgfpathlineto{\pgfqpoint{4.106522in}{2.716746in}}%
\pgfpathlineto{\pgfqpoint{4.093593in}{2.721191in}}%
\pgfpathlineto{\pgfqpoint{4.080669in}{2.725819in}}%
\pgfpathlineto{\pgfqpoint{4.073209in}{2.713916in}}%
\pgfpathlineto{\pgfqpoint{4.065746in}{2.702119in}}%
\pgfpathlineto{\pgfqpoint{4.058278in}{2.690426in}}%
\pgfpathlineto{\pgfqpoint{4.050806in}{2.678831in}}%
\pgfpathclose%
\pgfusepath{fill}%
\end{pgfscope}%
\begin{pgfscope}%
\pgfpathrectangle{\pgfqpoint{1.254980in}{0.150000in}}{\pgfqpoint{5.490039in}{5.490039in}}%
\pgfusepath{clip}%
\pgfsetbuttcap%
\pgfsetroundjoin%
\definecolor{currentfill}{rgb}{0.233603,0.313828,0.543914}%
\pgfsetfillcolor{currentfill}%
\pgfsetfillopacity{0.700000}%
\pgfsetlinewidth{0.000000pt}%
\definecolor{currentstroke}{rgb}{0.000000,0.000000,0.000000}%
\pgfsetstrokecolor{currentstroke}%
\pgfsetdash{}{0pt}%
\pgfpathmoveto{\pgfqpoint{4.592130in}{2.875842in}}%
\pgfpathlineto{\pgfqpoint{4.605200in}{2.874138in}}%
\pgfpathlineto{\pgfqpoint{4.618278in}{2.872599in}}%
\pgfpathlineto{\pgfqpoint{4.631365in}{2.871223in}}%
\pgfpathlineto{\pgfqpoint{4.644461in}{2.870012in}}%
\pgfpathlineto{\pgfqpoint{4.651773in}{2.881485in}}%
\pgfpathlineto{\pgfqpoint{4.659083in}{2.893111in}}%
\pgfpathlineto{\pgfqpoint{4.666391in}{2.904894in}}%
\pgfpathlineto{\pgfqpoint{4.673697in}{2.916840in}}%
\pgfpathlineto{\pgfqpoint{4.660613in}{2.918520in}}%
\pgfpathlineto{\pgfqpoint{4.647538in}{2.920363in}}%
\pgfpathlineto{\pgfqpoint{4.634472in}{2.922370in}}%
\pgfpathlineto{\pgfqpoint{4.621414in}{2.924542in}}%
\pgfpathlineto{\pgfqpoint{4.614096in}{2.912118in}}%
\pgfpathlineto{\pgfqpoint{4.606776in}{2.899863in}}%
\pgfpathlineto{\pgfqpoint{4.599454in}{2.887773in}}%
\pgfpathlineto{\pgfqpoint{4.592130in}{2.875842in}}%
\pgfpathclose%
\pgfusepath{fill}%
\end{pgfscope}%
\begin{pgfscope}%
\pgfpathrectangle{\pgfqpoint{1.254980in}{0.150000in}}{\pgfqpoint{5.490039in}{5.490039in}}%
\pgfusepath{clip}%
\pgfsetbuttcap%
\pgfsetroundjoin%
\definecolor{currentfill}{rgb}{0.199430,0.387607,0.554642}%
\pgfsetfillcolor{currentfill}%
\pgfsetfillopacity{0.700000}%
\pgfsetlinewidth{0.000000pt}%
\definecolor{currentstroke}{rgb}{0.000000,0.000000,0.000000}%
\pgfsetstrokecolor{currentstroke}%
\pgfsetdash{}{0pt}%
\pgfpathmoveto{\pgfqpoint{4.918488in}{3.048601in}}%
\pgfpathlineto{\pgfqpoint{4.931650in}{3.047434in}}%
\pgfpathlineto{\pgfqpoint{4.944821in}{3.046424in}}%
\pgfpathlineto{\pgfqpoint{4.958002in}{3.045571in}}%
\pgfpathlineto{\pgfqpoint{4.971194in}{3.044874in}}%
\pgfpathlineto{\pgfqpoint{4.978430in}{3.057203in}}%
\pgfpathlineto{\pgfqpoint{4.985667in}{3.069752in}}%
\pgfpathlineto{\pgfqpoint{4.992904in}{3.082529in}}%
\pgfpathlineto{\pgfqpoint{5.000141in}{3.095540in}}%
\pgfpathlineto{\pgfqpoint{4.986966in}{3.096815in}}%
\pgfpathlineto{\pgfqpoint{4.973801in}{3.098247in}}%
\pgfpathlineto{\pgfqpoint{4.960645in}{3.099835in}}%
\pgfpathlineto{\pgfqpoint{4.947499in}{3.101580in}}%
\pgfpathlineto{\pgfqpoint{4.940245in}{3.087981in}}%
\pgfpathlineto{\pgfqpoint{4.932993in}{3.074623in}}%
\pgfpathlineto{\pgfqpoint{4.925741in}{3.061498in}}%
\pgfpathlineto{\pgfqpoint{4.918488in}{3.048601in}}%
\pgfpathclose%
\pgfusepath{fill}%
\end{pgfscope}%
\begin{pgfscope}%
\pgfpathrectangle{\pgfqpoint{1.254980in}{0.150000in}}{\pgfqpoint{5.490039in}{5.490039in}}%
\pgfusepath{clip}%
\pgfsetbuttcap%
\pgfsetroundjoin%
\definecolor{currentfill}{rgb}{0.243113,0.292092,0.538516}%
\pgfsetfillcolor{currentfill}%
\pgfsetfillopacity{0.700000}%
\pgfsetlinewidth{0.000000pt}%
\definecolor{currentstroke}{rgb}{0.000000,0.000000,0.000000}%
\pgfsetstrokecolor{currentstroke}%
\pgfsetdash{}{0pt}%
\pgfpathmoveto{\pgfqpoint{4.510567in}{2.836309in}}%
\pgfpathlineto{\pgfqpoint{4.523615in}{2.834382in}}%
\pgfpathlineto{\pgfqpoint{4.536671in}{2.832622in}}%
\pgfpathlineto{\pgfqpoint{4.549736in}{2.831028in}}%
\pgfpathlineto{\pgfqpoint{4.562808in}{2.829600in}}%
\pgfpathlineto{\pgfqpoint{4.570143in}{2.840948in}}%
\pgfpathlineto{\pgfqpoint{4.577475in}{2.852435in}}%
\pgfpathlineto{\pgfqpoint{4.584804in}{2.864064in}}%
\pgfpathlineto{\pgfqpoint{4.592130in}{2.875842in}}%
\pgfpathlineto{\pgfqpoint{4.579069in}{2.877710in}}%
\pgfpathlineto{\pgfqpoint{4.566016in}{2.879745in}}%
\pgfpathlineto{\pgfqpoint{4.552971in}{2.881945in}}%
\pgfpathlineto{\pgfqpoint{4.539934in}{2.884312in}}%
\pgfpathlineto{\pgfqpoint{4.532596in}{2.872083in}}%
\pgfpathlineto{\pgfqpoint{4.525256in}{2.860010in}}%
\pgfpathlineto{\pgfqpoint{4.517913in}{2.848087in}}%
\pgfpathlineto{\pgfqpoint{4.510567in}{2.836309in}}%
\pgfpathclose%
\pgfusepath{fill}%
\end{pgfscope}%
\begin{pgfscope}%
\pgfpathrectangle{\pgfqpoint{1.254980in}{0.150000in}}{\pgfqpoint{5.490039in}{5.490039in}}%
\pgfusepath{clip}%
\pgfsetbuttcap%
\pgfsetroundjoin%
\definecolor{currentfill}{rgb}{0.229739,0.322361,0.545706}%
\pgfsetfillcolor{currentfill}%
\pgfsetfillopacity{0.700000}%
\pgfsetlinewidth{0.000000pt}%
\definecolor{currentstroke}{rgb}{0.000000,0.000000,0.000000}%
\pgfsetstrokecolor{currentstroke}%
\pgfsetdash{}{0pt}%
\pgfpathmoveto{\pgfqpoint{3.146799in}{2.920185in}}%
\pgfpathlineto{\pgfqpoint{3.159744in}{2.904972in}}%
\pgfpathlineto{\pgfqpoint{3.172685in}{2.890021in}}%
\pgfpathlineto{\pgfqpoint{3.185621in}{2.875330in}}%
\pgfpathlineto{\pgfqpoint{3.198553in}{2.860897in}}%
\pgfpathlineto{\pgfqpoint{3.206264in}{2.872658in}}%
\pgfpathlineto{\pgfqpoint{3.213968in}{2.884543in}}%
\pgfpathlineto{\pgfqpoint{3.221665in}{2.896553in}}%
\pgfpathlineto{\pgfqpoint{3.229356in}{2.908691in}}%
\pgfpathlineto{\pgfqpoint{3.216435in}{2.923262in}}%
\pgfpathlineto{\pgfqpoint{3.203510in}{2.938090in}}%
\pgfpathlineto{\pgfqpoint{3.190581in}{2.953179in}}%
\pgfpathlineto{\pgfqpoint{3.177647in}{2.968529in}}%
\pgfpathlineto{\pgfqpoint{3.169945in}{2.956243in}}%
\pgfpathlineto{\pgfqpoint{3.162236in}{2.944092in}}%
\pgfpathlineto{\pgfqpoint{3.154521in}{2.932073in}}%
\pgfpathlineto{\pgfqpoint{3.146799in}{2.920185in}}%
\pgfpathclose%
\pgfusepath{fill}%
\end{pgfscope}%
\begin{pgfscope}%
\pgfpathrectangle{\pgfqpoint{1.254980in}{0.150000in}}{\pgfqpoint{5.490039in}{5.490039in}}%
\pgfusepath{clip}%
\pgfsetbuttcap%
\pgfsetroundjoin%
\definecolor{currentfill}{rgb}{0.190631,0.407061,0.556089}%
\pgfsetfillcolor{currentfill}%
\pgfsetfillopacity{0.700000}%
\pgfsetlinewidth{0.000000pt}%
\definecolor{currentstroke}{rgb}{0.000000,0.000000,0.000000}%
\pgfsetstrokecolor{currentstroke}%
\pgfsetdash{}{0pt}%
\pgfpathmoveto{\pgfqpoint{5.000141in}{3.095540in}}%
\pgfpathlineto{\pgfqpoint{5.013326in}{3.094420in}}%
\pgfpathlineto{\pgfqpoint{5.026521in}{3.093457in}}%
\pgfpathlineto{\pgfqpoint{5.039726in}{3.092648in}}%
\pgfpathlineto{\pgfqpoint{5.052941in}{3.091995in}}%
\pgfpathlineto{\pgfqpoint{5.060163in}{3.104650in}}%
\pgfpathlineto{\pgfqpoint{5.067386in}{3.117547in}}%
\pgfpathlineto{\pgfqpoint{5.074610in}{3.130691in}}%
\pgfpathlineto{\pgfqpoint{5.081836in}{3.144091in}}%
\pgfpathlineto{\pgfqpoint{5.068638in}{3.145351in}}%
\pgfpathlineto{\pgfqpoint{5.055450in}{3.146766in}}%
\pgfpathlineto{\pgfqpoint{5.042271in}{3.148336in}}%
\pgfpathlineto{\pgfqpoint{5.029103in}{3.150061in}}%
\pgfpathlineto{\pgfqpoint{5.021860in}{3.136045in}}%
\pgfpathlineto{\pgfqpoint{5.014619in}{3.122291in}}%
\pgfpathlineto{\pgfqpoint{5.007380in}{3.108792in}}%
\pgfpathlineto{\pgfqpoint{5.000141in}{3.095540in}}%
\pgfpathclose%
\pgfusepath{fill}%
\end{pgfscope}%
\begin{pgfscope}%
\pgfpathrectangle{\pgfqpoint{1.254980in}{0.150000in}}{\pgfqpoint{5.490039in}{5.490039in}}%
\pgfusepath{clip}%
\pgfsetbuttcap%
\pgfsetroundjoin%
\definecolor{currentfill}{rgb}{0.218130,0.347432,0.550038}%
\pgfsetfillcolor{currentfill}%
\pgfsetfillopacity{0.700000}%
\pgfsetlinewidth{0.000000pt}%
\definecolor{currentstroke}{rgb}{0.000000,0.000000,0.000000}%
\pgfsetstrokecolor{currentstroke}%
\pgfsetdash{}{0pt}%
\pgfpathmoveto{\pgfqpoint{3.094964in}{2.983702in}}%
\pgfpathlineto{\pgfqpoint{3.107931in}{2.967418in}}%
\pgfpathlineto{\pgfqpoint{3.120892in}{2.951405in}}%
\pgfpathlineto{\pgfqpoint{3.133848in}{2.935662in}}%
\pgfpathlineto{\pgfqpoint{3.146799in}{2.920185in}}%
\pgfpathlineto{\pgfqpoint{3.154521in}{2.932073in}}%
\pgfpathlineto{\pgfqpoint{3.162236in}{2.944092in}}%
\pgfpathlineto{\pgfqpoint{3.169945in}{2.956243in}}%
\pgfpathlineto{\pgfqpoint{3.177647in}{2.968529in}}%
\pgfpathlineto{\pgfqpoint{3.164708in}{2.984144in}}%
\pgfpathlineto{\pgfqpoint{3.151763in}{3.000026in}}%
\pgfpathlineto{\pgfqpoint{3.138814in}{3.016177in}}%
\pgfpathlineto{\pgfqpoint{3.125859in}{3.032599in}}%
\pgfpathlineto{\pgfqpoint{3.118145in}{3.020164in}}%
\pgfpathlineto{\pgfqpoint{3.110425in}{3.007871in}}%
\pgfpathlineto{\pgfqpoint{3.102698in}{2.995718in}}%
\pgfpathlineto{\pgfqpoint{3.094964in}{2.983702in}}%
\pgfpathclose%
\pgfusepath{fill}%
\end{pgfscope}%
\begin{pgfscope}%
\pgfpathrectangle{\pgfqpoint{1.254980in}{0.150000in}}{\pgfqpoint{5.490039in}{5.490039in}}%
\pgfusepath{clip}%
\pgfsetbuttcap%
\pgfsetroundjoin%
\definecolor{currentfill}{rgb}{0.241237,0.296485,0.539709}%
\pgfsetfillcolor{currentfill}%
\pgfsetfillopacity{0.700000}%
\pgfsetlinewidth{0.000000pt}%
\definecolor{currentstroke}{rgb}{0.000000,0.000000,0.000000}%
\pgfsetstrokecolor{currentstroke}%
\pgfsetdash{}{0pt}%
\pgfpathmoveto{\pgfqpoint{3.198553in}{2.860897in}}%
\pgfpathlineto{\pgfqpoint{3.211481in}{2.846719in}}%
\pgfpathlineto{\pgfqpoint{3.224406in}{2.832795in}}%
\pgfpathlineto{\pgfqpoint{3.237326in}{2.819122in}}%
\pgfpathlineto{\pgfqpoint{3.250243in}{2.805698in}}%
\pgfpathlineto{\pgfqpoint{3.257943in}{2.817331in}}%
\pgfpathlineto{\pgfqpoint{3.265635in}{2.829082in}}%
\pgfpathlineto{\pgfqpoint{3.273322in}{2.840951in}}%
\pgfpathlineto{\pgfqpoint{3.281003in}{2.852941in}}%
\pgfpathlineto{\pgfqpoint{3.268096in}{2.866502in}}%
\pgfpathlineto{\pgfqpoint{3.255186in}{2.880313in}}%
\pgfpathlineto{\pgfqpoint{3.242273in}{2.894375in}}%
\pgfpathlineto{\pgfqpoint{3.229356in}{2.908691in}}%
\pgfpathlineto{\pgfqpoint{3.221665in}{2.896553in}}%
\pgfpathlineto{\pgfqpoint{3.213968in}{2.884543in}}%
\pgfpathlineto{\pgfqpoint{3.206264in}{2.872658in}}%
\pgfpathlineto{\pgfqpoint{3.198553in}{2.860897in}}%
\pgfpathclose%
\pgfusepath{fill}%
\end{pgfscope}%
\begin{pgfscope}%
\pgfpathrectangle{\pgfqpoint{1.254980in}{0.150000in}}{\pgfqpoint{5.490039in}{5.490039in}}%
\pgfusepath{clip}%
\pgfsetbuttcap%
\pgfsetroundjoin%
\definecolor{currentfill}{rgb}{0.248629,0.278775,0.534556}%
\pgfsetfillcolor{currentfill}%
\pgfsetfillopacity{0.700000}%
\pgfsetlinewidth{0.000000pt}%
\definecolor{currentstroke}{rgb}{0.000000,0.000000,0.000000}%
\pgfsetstrokecolor{currentstroke}%
\pgfsetdash{}{0pt}%
\pgfpathmoveto{\pgfqpoint{4.429002in}{2.798277in}}%
\pgfpathlineto{\pgfqpoint{4.442029in}{2.796091in}}%
\pgfpathlineto{\pgfqpoint{4.455063in}{2.794074in}}%
\pgfpathlineto{\pgfqpoint{4.468106in}{2.792226in}}%
\pgfpathlineto{\pgfqpoint{4.481156in}{2.790545in}}%
\pgfpathlineto{\pgfqpoint{4.488514in}{2.801793in}}%
\pgfpathlineto{\pgfqpoint{4.495868in}{2.813167in}}%
\pgfpathlineto{\pgfqpoint{4.503219in}{2.824670in}}%
\pgfpathlineto{\pgfqpoint{4.510567in}{2.836309in}}%
\pgfpathlineto{\pgfqpoint{4.497528in}{2.838403in}}%
\pgfpathlineto{\pgfqpoint{4.484495in}{2.840664in}}%
\pgfpathlineto{\pgfqpoint{4.471471in}{2.843094in}}%
\pgfpathlineto{\pgfqpoint{4.458455in}{2.845692in}}%
\pgfpathlineto{\pgfqpoint{4.451096in}{2.833631in}}%
\pgfpathlineto{\pgfqpoint{4.443734in}{2.821711in}}%
\pgfpathlineto{\pgfqpoint{4.436370in}{2.809928in}}%
\pgfpathlineto{\pgfqpoint{4.429002in}{2.798277in}}%
\pgfpathclose%
\pgfusepath{fill}%
\end{pgfscope}%
\begin{pgfscope}%
\pgfpathrectangle{\pgfqpoint{1.254980in}{0.150000in}}{\pgfqpoint{5.490039in}{5.490039in}}%
\pgfusepath{clip}%
\pgfsetbuttcap%
\pgfsetroundjoin%
\definecolor{currentfill}{rgb}{0.182256,0.426184,0.557120}%
\pgfsetfillcolor{currentfill}%
\pgfsetfillopacity{0.700000}%
\pgfsetlinewidth{0.000000pt}%
\definecolor{currentstroke}{rgb}{0.000000,0.000000,0.000000}%
\pgfsetstrokecolor{currentstroke}%
\pgfsetdash{}{0pt}%
\pgfpathmoveto{\pgfqpoint{5.081836in}{3.144091in}}%
\pgfpathlineto{\pgfqpoint{5.095044in}{3.142985in}}%
\pgfpathlineto{\pgfqpoint{5.108262in}{3.142034in}}%
\pgfpathlineto{\pgfqpoint{5.121490in}{3.141236in}}%
\pgfpathlineto{\pgfqpoint{5.134729in}{3.140592in}}%
\pgfpathlineto{\pgfqpoint{5.141939in}{3.153630in}}%
\pgfpathlineto{\pgfqpoint{5.149151in}{3.166930in}}%
\pgfpathlineto{\pgfqpoint{5.156364in}{3.180500in}}%
\pgfpathlineto{\pgfqpoint{5.163581in}{3.194347in}}%
\pgfpathlineto{\pgfqpoint{5.150360in}{3.195626in}}%
\pgfpathlineto{\pgfqpoint{5.137150in}{3.197058in}}%
\pgfpathlineto{\pgfqpoint{5.123949in}{3.198643in}}%
\pgfpathlineto{\pgfqpoint{5.110759in}{3.200382in}}%
\pgfpathlineto{\pgfqpoint{5.103524in}{3.185891in}}%
\pgfpathlineto{\pgfqpoint{5.096293in}{3.171683in}}%
\pgfpathlineto{\pgfqpoint{5.089063in}{3.157752in}}%
\pgfpathlineto{\pgfqpoint{5.081836in}{3.144091in}}%
\pgfpathclose%
\pgfusepath{fill}%
\end{pgfscope}%
\begin{pgfscope}%
\pgfpathrectangle{\pgfqpoint{1.254980in}{0.150000in}}{\pgfqpoint{5.490039in}{5.490039in}}%
\pgfusepath{clip}%
\pgfsetbuttcap%
\pgfsetroundjoin%
\definecolor{currentfill}{rgb}{0.275191,0.194905,0.496005}%
\pgfsetfillcolor{currentfill}%
\pgfsetfillopacity{0.700000}%
\pgfsetlinewidth{0.000000pt}%
\definecolor{currentstroke}{rgb}{0.000000,0.000000,0.000000}%
\pgfsetstrokecolor{currentstroke}%
\pgfsetdash{}{0pt}%
\pgfpathmoveto{\pgfqpoint{3.620730in}{2.642763in}}%
\pgfpathlineto{\pgfqpoint{3.633610in}{2.634700in}}%
\pgfpathlineto{\pgfqpoint{3.646491in}{2.626844in}}%
\pgfpathlineto{\pgfqpoint{3.659375in}{2.619194in}}%
\pgfpathlineto{\pgfqpoint{3.672260in}{2.611750in}}%
\pgfpathlineto{\pgfqpoint{3.679847in}{2.623099in}}%
\pgfpathlineto{\pgfqpoint{3.687429in}{2.634533in}}%
\pgfpathlineto{\pgfqpoint{3.695006in}{2.646056in}}%
\pgfpathlineto{\pgfqpoint{3.702578in}{2.657669in}}%
\pgfpathlineto{\pgfqpoint{3.689701in}{2.665306in}}%
\pgfpathlineto{\pgfqpoint{3.676826in}{2.673148in}}%
\pgfpathlineto{\pgfqpoint{3.663953in}{2.681196in}}%
\pgfpathlineto{\pgfqpoint{3.651082in}{2.689451in}}%
\pgfpathlineto{\pgfqpoint{3.643502in}{2.677636in}}%
\pgfpathlineto{\pgfqpoint{3.635916in}{2.665917in}}%
\pgfpathlineto{\pgfqpoint{3.628326in}{2.654294in}}%
\pgfpathlineto{\pgfqpoint{3.620730in}{2.642763in}}%
\pgfpathclose%
\pgfusepath{fill}%
\end{pgfscope}%
\begin{pgfscope}%
\pgfpathrectangle{\pgfqpoint{1.254980in}{0.150000in}}{\pgfqpoint{5.490039in}{5.490039in}}%
\pgfusepath{clip}%
\pgfsetbuttcap%
\pgfsetroundjoin%
\definecolor{currentfill}{rgb}{0.204903,0.375746,0.553533}%
\pgfsetfillcolor{currentfill}%
\pgfsetfillopacity{0.700000}%
\pgfsetlinewidth{0.000000pt}%
\definecolor{currentstroke}{rgb}{0.000000,0.000000,0.000000}%
\pgfsetstrokecolor{currentstroke}%
\pgfsetdash{}{0pt}%
\pgfpathmoveto{\pgfqpoint{3.043035in}{3.051602in}}%
\pgfpathlineto{\pgfqpoint{3.056027in}{3.034208in}}%
\pgfpathlineto{\pgfqpoint{3.069012in}{3.017094in}}%
\pgfpathlineto{\pgfqpoint{3.081991in}{3.000260in}}%
\pgfpathlineto{\pgfqpoint{3.094964in}{2.983702in}}%
\pgfpathlineto{\pgfqpoint{3.102698in}{2.995718in}}%
\pgfpathlineto{\pgfqpoint{3.110425in}{3.007871in}}%
\pgfpathlineto{\pgfqpoint{3.118145in}{3.020164in}}%
\pgfpathlineto{\pgfqpoint{3.125859in}{3.032599in}}%
\pgfpathlineto{\pgfqpoint{3.112898in}{3.049295in}}%
\pgfpathlineto{\pgfqpoint{3.099931in}{3.066268in}}%
\pgfpathlineto{\pgfqpoint{3.086957in}{3.083520in}}%
\pgfpathlineto{\pgfqpoint{3.073977in}{3.101053in}}%
\pgfpathlineto{\pgfqpoint{3.066252in}{3.088469in}}%
\pgfpathlineto{\pgfqpoint{3.058520in}{3.076034in}}%
\pgfpathlineto{\pgfqpoint{3.050781in}{3.063746in}}%
\pgfpathlineto{\pgfqpoint{3.043035in}{3.051602in}}%
\pgfpathclose%
\pgfusepath{fill}%
\end{pgfscope}%
\begin{pgfscope}%
\pgfpathrectangle{\pgfqpoint{1.254980in}{0.150000in}}{\pgfqpoint{5.490039in}{5.490039in}}%
\pgfusepath{clip}%
\pgfsetbuttcap%
\pgfsetroundjoin%
\definecolor{currentfill}{rgb}{0.252194,0.269783,0.531579}%
\pgfsetfillcolor{currentfill}%
\pgfsetfillopacity{0.700000}%
\pgfsetlinewidth{0.000000pt}%
\definecolor{currentstroke}{rgb}{0.000000,0.000000,0.000000}%
\pgfsetstrokecolor{currentstroke}%
\pgfsetdash{}{0pt}%
\pgfpathmoveto{\pgfqpoint{3.250243in}{2.805698in}}%
\pgfpathlineto{\pgfqpoint{3.263158in}{2.792521in}}%
\pgfpathlineto{\pgfqpoint{3.276069in}{2.779590in}}%
\pgfpathlineto{\pgfqpoint{3.288977in}{2.766901in}}%
\pgfpathlineto{\pgfqpoint{3.301883in}{2.754455in}}%
\pgfpathlineto{\pgfqpoint{3.309571in}{2.765961in}}%
\pgfpathlineto{\pgfqpoint{3.317253in}{2.777577in}}%
\pgfpathlineto{\pgfqpoint{3.324929in}{2.789306in}}%
\pgfpathlineto{\pgfqpoint{3.332599in}{2.801147in}}%
\pgfpathlineto{\pgfqpoint{3.319704in}{2.813732in}}%
\pgfpathlineto{\pgfqpoint{3.306806in}{2.826557in}}%
\pgfpathlineto{\pgfqpoint{3.293906in}{2.839627in}}%
\pgfpathlineto{\pgfqpoint{3.281003in}{2.852941in}}%
\pgfpathlineto{\pgfqpoint{3.273322in}{2.840951in}}%
\pgfpathlineto{\pgfqpoint{3.265635in}{2.829082in}}%
\pgfpathlineto{\pgfqpoint{3.257943in}{2.817331in}}%
\pgfpathlineto{\pgfqpoint{3.250243in}{2.805698in}}%
\pgfpathclose%
\pgfusepath{fill}%
\end{pgfscope}%
\begin{pgfscope}%
\pgfpathrectangle{\pgfqpoint{1.254980in}{0.150000in}}{\pgfqpoint{5.490039in}{5.490039in}}%
\pgfusepath{clip}%
\pgfsetbuttcap%
\pgfsetroundjoin%
\definecolor{currentfill}{rgb}{0.255645,0.260703,0.528312}%
\pgfsetfillcolor{currentfill}%
\pgfsetfillopacity{0.700000}%
\pgfsetlinewidth{0.000000pt}%
\definecolor{currentstroke}{rgb}{0.000000,0.000000,0.000000}%
\pgfsetstrokecolor{currentstroke}%
\pgfsetdash{}{0pt}%
\pgfpathmoveto{\pgfqpoint{4.347426in}{2.761803in}}%
\pgfpathlineto{\pgfqpoint{4.360433in}{2.759321in}}%
\pgfpathlineto{\pgfqpoint{4.373447in}{2.757010in}}%
\pgfpathlineto{\pgfqpoint{4.386469in}{2.754870in}}%
\pgfpathlineto{\pgfqpoint{4.399498in}{2.752900in}}%
\pgfpathlineto{\pgfqpoint{4.406879in}{2.764069in}}%
\pgfpathlineto{\pgfqpoint{4.414257in}{2.775352in}}%
\pgfpathlineto{\pgfqpoint{4.421631in}{2.786753in}}%
\pgfpathlineto{\pgfqpoint{4.429002in}{2.798277in}}%
\pgfpathlineto{\pgfqpoint{4.415983in}{2.800633in}}%
\pgfpathlineto{\pgfqpoint{4.402971in}{2.803158in}}%
\pgfpathlineto{\pgfqpoint{4.389966in}{2.805854in}}%
\pgfpathlineto{\pgfqpoint{4.376969in}{2.808721in}}%
\pgfpathlineto{\pgfqpoint{4.369588in}{2.796802in}}%
\pgfpathlineto{\pgfqpoint{4.362204in}{2.785013in}}%
\pgfpathlineto{\pgfqpoint{4.354817in}{2.773348in}}%
\pgfpathlineto{\pgfqpoint{4.347426in}{2.761803in}}%
\pgfpathclose%
\pgfusepath{fill}%
\end{pgfscope}%
\begin{pgfscope}%
\pgfpathrectangle{\pgfqpoint{1.254980in}{0.150000in}}{\pgfqpoint{5.490039in}{5.490039in}}%
\pgfusepath{clip}%
\pgfsetbuttcap%
\pgfsetroundjoin%
\definecolor{currentfill}{rgb}{0.273006,0.204520,0.501721}%
\pgfsetfillcolor{currentfill}%
\pgfsetfillopacity{0.700000}%
\pgfsetlinewidth{0.000000pt}%
\definecolor{currentstroke}{rgb}{0.000000,0.000000,0.000000}%
\pgfsetstrokecolor{currentstroke}%
\pgfsetdash{}{0pt}%
\pgfpathmoveto{\pgfqpoint{3.969165in}{2.651426in}}%
\pgfpathlineto{\pgfqpoint{3.982085in}{2.646635in}}%
\pgfpathlineto{\pgfqpoint{3.995010in}{2.642031in}}%
\pgfpathlineto{\pgfqpoint{4.007940in}{2.637612in}}%
\pgfpathlineto{\pgfqpoint{4.020875in}{2.633377in}}%
\pgfpathlineto{\pgfqpoint{4.028364in}{2.644609in}}%
\pgfpathlineto{\pgfqpoint{4.035849in}{2.655926in}}%
\pgfpathlineto{\pgfqpoint{4.043329in}{2.667332in}}%
\pgfpathlineto{\pgfqpoint{4.050806in}{2.678831in}}%
\pgfpathlineto{\pgfqpoint{4.037879in}{2.683341in}}%
\pgfpathlineto{\pgfqpoint{4.024957in}{2.688035in}}%
\pgfpathlineto{\pgfqpoint{4.012040in}{2.692914in}}%
\pgfpathlineto{\pgfqpoint{3.999128in}{2.697979in}}%
\pgfpathlineto{\pgfqpoint{3.991644in}{2.686196in}}%
\pgfpathlineto{\pgfqpoint{3.984155in}{2.674511in}}%
\pgfpathlineto{\pgfqpoint{3.976662in}{2.662922in}}%
\pgfpathlineto{\pgfqpoint{3.969165in}{2.651426in}}%
\pgfpathclose%
\pgfusepath{fill}%
\end{pgfscope}%
\begin{pgfscope}%
\pgfpathrectangle{\pgfqpoint{1.254980in}{0.150000in}}{\pgfqpoint{5.490039in}{5.490039in}}%
\pgfusepath{clip}%
\pgfsetbuttcap%
\pgfsetroundjoin%
\definecolor{currentfill}{rgb}{0.276194,0.190074,0.493001}%
\pgfsetfillcolor{currentfill}%
\pgfsetfillopacity{0.700000}%
\pgfsetlinewidth{0.000000pt}%
\definecolor{currentstroke}{rgb}{0.000000,0.000000,0.000000}%
\pgfsetstrokecolor{currentstroke}%
\pgfsetdash{}{0pt}%
\pgfpathmoveto{\pgfqpoint{3.754110in}{2.629144in}}%
\pgfpathlineto{\pgfqpoint{3.767000in}{2.622514in}}%
\pgfpathlineto{\pgfqpoint{3.779893in}{2.616081in}}%
\pgfpathlineto{\pgfqpoint{3.792789in}{2.609845in}}%
\pgfpathlineto{\pgfqpoint{3.805688in}{2.603804in}}%
\pgfpathlineto{\pgfqpoint{3.813238in}{2.615092in}}%
\pgfpathlineto{\pgfqpoint{3.820784in}{2.626461in}}%
\pgfpathlineto{\pgfqpoint{3.828325in}{2.637915in}}%
\pgfpathlineto{\pgfqpoint{3.835862in}{2.649457in}}%
\pgfpathlineto{\pgfqpoint{3.822970in}{2.655717in}}%
\pgfpathlineto{\pgfqpoint{3.810083in}{2.662173in}}%
\pgfpathlineto{\pgfqpoint{3.797198in}{2.668826in}}%
\pgfpathlineto{\pgfqpoint{3.784316in}{2.675677in}}%
\pgfpathlineto{\pgfqpoint{3.776772in}{2.663905in}}%
\pgfpathlineto{\pgfqpoint{3.769223in}{2.652227in}}%
\pgfpathlineto{\pgfqpoint{3.761669in}{2.640641in}}%
\pgfpathlineto{\pgfqpoint{3.754110in}{2.629144in}}%
\pgfpathclose%
\pgfusepath{fill}%
\end{pgfscope}%
\begin{pgfscope}%
\pgfpathrectangle{\pgfqpoint{1.254980in}{0.150000in}}{\pgfqpoint{5.490039in}{5.490039in}}%
\pgfusepath{clip}%
\pgfsetbuttcap%
\pgfsetroundjoin%
\definecolor{currentfill}{rgb}{0.271828,0.209303,0.504434}%
\pgfsetfillcolor{currentfill}%
\pgfsetfillopacity{0.700000}%
\pgfsetlinewidth{0.000000pt}%
\definecolor{currentstroke}{rgb}{0.000000,0.000000,0.000000}%
\pgfsetstrokecolor{currentstroke}%
\pgfsetdash{}{0pt}%
\pgfpathmoveto{\pgfqpoint{3.487225in}{2.668320in}}%
\pgfpathlineto{\pgfqpoint{3.500107in}{2.658716in}}%
\pgfpathlineto{\pgfqpoint{3.512989in}{2.649330in}}%
\pgfpathlineto{\pgfqpoint{3.525871in}{2.640161in}}%
\pgfpathlineto{\pgfqpoint{3.538754in}{2.631207in}}%
\pgfpathlineto{\pgfqpoint{3.546380in}{2.642546in}}%
\pgfpathlineto{\pgfqpoint{3.554000in}{2.653975in}}%
\pgfpathlineto{\pgfqpoint{3.561615in}{2.665496in}}%
\pgfpathlineto{\pgfqpoint{3.569224in}{2.677113in}}%
\pgfpathlineto{\pgfqpoint{3.556351in}{2.686232in}}%
\pgfpathlineto{\pgfqpoint{3.543477in}{2.695566in}}%
\pgfpathlineto{\pgfqpoint{3.530605in}{2.705117in}}%
\pgfpathlineto{\pgfqpoint{3.517732in}{2.714885in}}%
\pgfpathlineto{\pgfqpoint{3.510114in}{2.703094in}}%
\pgfpathlineto{\pgfqpoint{3.502490in}{2.691404in}}%
\pgfpathlineto{\pgfqpoint{3.494860in}{2.679813in}}%
\pgfpathlineto{\pgfqpoint{3.487225in}{2.668320in}}%
\pgfpathclose%
\pgfusepath{fill}%
\end{pgfscope}%
\begin{pgfscope}%
\pgfpathrectangle{\pgfqpoint{1.254980in}{0.150000in}}{\pgfqpoint{5.490039in}{5.490039in}}%
\pgfusepath{clip}%
\pgfsetbuttcap%
\pgfsetroundjoin%
\definecolor{currentfill}{rgb}{0.262138,0.242286,0.520837}%
\pgfsetfillcolor{currentfill}%
\pgfsetfillopacity{0.700000}%
\pgfsetlinewidth{0.000000pt}%
\definecolor{currentstroke}{rgb}{0.000000,0.000000,0.000000}%
\pgfsetstrokecolor{currentstroke}%
\pgfsetdash{}{0pt}%
\pgfpathmoveto{\pgfqpoint{4.265834in}{2.726966in}}%
\pgfpathlineto{\pgfqpoint{4.278822in}{2.724149in}}%
\pgfpathlineto{\pgfqpoint{4.291817in}{2.721507in}}%
\pgfpathlineto{\pgfqpoint{4.304819in}{2.719038in}}%
\pgfpathlineto{\pgfqpoint{4.317828in}{2.716742in}}%
\pgfpathlineto{\pgfqpoint{4.325233in}{2.727848in}}%
\pgfpathlineto{\pgfqpoint{4.332634in}{2.739058in}}%
\pgfpathlineto{\pgfqpoint{4.340032in}{2.750375in}}%
\pgfpathlineto{\pgfqpoint{4.347426in}{2.761803in}}%
\pgfpathlineto{\pgfqpoint{4.334427in}{2.764458in}}%
\pgfpathlineto{\pgfqpoint{4.321434in}{2.767284in}}%
\pgfpathlineto{\pgfqpoint{4.308449in}{2.770284in}}%
\pgfpathlineto{\pgfqpoint{4.295470in}{2.773458in}}%
\pgfpathlineto{\pgfqpoint{4.288066in}{2.761662in}}%
\pgfpathlineto{\pgfqpoint{4.280659in}{2.749984in}}%
\pgfpathlineto{\pgfqpoint{4.273248in}{2.738420in}}%
\pgfpathlineto{\pgfqpoint{4.265834in}{2.726966in}}%
\pgfpathclose%
\pgfusepath{fill}%
\end{pgfscope}%
\begin{pgfscope}%
\pgfpathrectangle{\pgfqpoint{1.254980in}{0.150000in}}{\pgfqpoint{5.490039in}{5.490039in}}%
\pgfusepath{clip}%
\pgfsetbuttcap%
\pgfsetroundjoin%
\definecolor{currentfill}{rgb}{0.260571,0.246922,0.522828}%
\pgfsetfillcolor{currentfill}%
\pgfsetfillopacity{0.700000}%
\pgfsetlinewidth{0.000000pt}%
\definecolor{currentstroke}{rgb}{0.000000,0.000000,0.000000}%
\pgfsetstrokecolor{currentstroke}%
\pgfsetdash{}{0pt}%
\pgfpathmoveto{\pgfqpoint{3.301883in}{2.754455in}}%
\pgfpathlineto{\pgfqpoint{3.314786in}{2.742247in}}%
\pgfpathlineto{\pgfqpoint{3.327688in}{2.730278in}}%
\pgfpathlineto{\pgfqpoint{3.340587in}{2.718544in}}%
\pgfpathlineto{\pgfqpoint{3.353485in}{2.707045in}}%
\pgfpathlineto{\pgfqpoint{3.361163in}{2.718424in}}%
\pgfpathlineto{\pgfqpoint{3.368834in}{2.729906in}}%
\pgfpathlineto{\pgfqpoint{3.376500in}{2.741494in}}%
\pgfpathlineto{\pgfqpoint{3.384160in}{2.753188in}}%
\pgfpathlineto{\pgfqpoint{3.371273in}{2.764825in}}%
\pgfpathlineto{\pgfqpoint{3.358383in}{2.776696in}}%
\pgfpathlineto{\pgfqpoint{3.345492in}{2.788803in}}%
\pgfpathlineto{\pgfqpoint{3.332599in}{2.801147in}}%
\pgfpathlineto{\pgfqpoint{3.324929in}{2.789306in}}%
\pgfpathlineto{\pgfqpoint{3.317253in}{2.777577in}}%
\pgfpathlineto{\pgfqpoint{3.309571in}{2.765961in}}%
\pgfpathlineto{\pgfqpoint{3.301883in}{2.754455in}}%
\pgfpathclose%
\pgfusepath{fill}%
\end{pgfscope}%
\begin{pgfscope}%
\pgfpathrectangle{\pgfqpoint{1.254980in}{0.150000in}}{\pgfqpoint{5.490039in}{5.490039in}}%
\pgfusepath{clip}%
\pgfsetbuttcap%
\pgfsetroundjoin%
\definecolor{currentfill}{rgb}{0.174274,0.445044,0.557792}%
\pgfsetfillcolor{currentfill}%
\pgfsetfillopacity{0.700000}%
\pgfsetlinewidth{0.000000pt}%
\definecolor{currentstroke}{rgb}{0.000000,0.000000,0.000000}%
\pgfsetstrokecolor{currentstroke}%
\pgfsetdash{}{0pt}%
\pgfpathmoveto{\pgfqpoint{5.163581in}{3.194347in}}%
\pgfpathlineto{\pgfqpoint{5.176812in}{3.193222in}}%
\pgfpathlineto{\pgfqpoint{5.190053in}{3.192249in}}%
\pgfpathlineto{\pgfqpoint{5.203305in}{3.191429in}}%
\pgfpathlineto{\pgfqpoint{5.216568in}{3.190761in}}%
\pgfpathlineto{\pgfqpoint{5.223768in}{3.204242in}}%
\pgfpathlineto{\pgfqpoint{5.230971in}{3.218007in}}%
\pgfpathlineto{\pgfqpoint{5.238178in}{3.232066in}}%
\pgfpathlineto{\pgfqpoint{5.224930in}{3.233228in}}%
\pgfpathlineto{\pgfqpoint{5.211692in}{3.234542in}}%
\pgfpathlineto{\pgfqpoint{5.198465in}{3.236009in}}%
\pgfpathlineto{\pgfqpoint{5.185248in}{3.237628in}}%
\pgfpathlineto{\pgfqpoint{5.178023in}{3.222904in}}%
\pgfpathlineto{\pgfqpoint{5.170800in}{3.208479in}}%
\pgfpathlineto{\pgfqpoint{5.163581in}{3.194347in}}%
\pgfpathclose%
\pgfusepath{fill}%
\end{pgfscope}%
\begin{pgfscope}%
\pgfpathrectangle{\pgfqpoint{1.254980in}{0.150000in}}{\pgfqpoint{5.490039in}{5.490039in}}%
\pgfusepath{clip}%
\pgfsetbuttcap%
\pgfsetroundjoin%
\definecolor{currentfill}{rgb}{0.190631,0.407061,0.556089}%
\pgfsetfillcolor{currentfill}%
\pgfsetfillopacity{0.700000}%
\pgfsetlinewidth{0.000000pt}%
\definecolor{currentstroke}{rgb}{0.000000,0.000000,0.000000}%
\pgfsetstrokecolor{currentstroke}%
\pgfsetdash{}{0pt}%
\pgfpathmoveto{\pgfqpoint{2.990994in}{3.124049in}}%
\pgfpathlineto{\pgfqpoint{3.004015in}{3.105502in}}%
\pgfpathlineto{\pgfqpoint{3.017029in}{3.087247in}}%
\pgfpathlineto{\pgfqpoint{3.030036in}{3.069281in}}%
\pgfpathlineto{\pgfqpoint{3.043035in}{3.051602in}}%
\pgfpathlineto{\pgfqpoint{3.050781in}{3.063746in}}%
\pgfpathlineto{\pgfqpoint{3.058520in}{3.076034in}}%
\pgfpathlineto{\pgfqpoint{3.066252in}{3.088469in}}%
\pgfpathlineto{\pgfqpoint{3.073977in}{3.101053in}}%
\pgfpathlineto{\pgfqpoint{3.060990in}{3.118870in}}%
\pgfpathlineto{\pgfqpoint{3.047996in}{3.136975in}}%
\pgfpathlineto{\pgfqpoint{3.034995in}{3.155368in}}%
\pgfpathlineto{\pgfqpoint{3.021986in}{3.174055in}}%
\pgfpathlineto{\pgfqpoint{3.014249in}{3.161322in}}%
\pgfpathlineto{\pgfqpoint{3.006504in}{3.148744in}}%
\pgfpathlineto{\pgfqpoint{2.998753in}{3.136321in}}%
\pgfpathlineto{\pgfqpoint{2.990994in}{3.124049in}}%
\pgfpathclose%
\pgfusepath{fill}%
\end{pgfscope}%
\begin{pgfscope}%
\pgfpathrectangle{\pgfqpoint{1.254980in}{0.150000in}}{\pgfqpoint{5.490039in}{5.490039in}}%
\pgfusepath{clip}%
\pgfsetbuttcap%
\pgfsetroundjoin%
\definecolor{currentfill}{rgb}{0.275191,0.194905,0.496005}%
\pgfsetfillcolor{currentfill}%
\pgfsetfillopacity{0.700000}%
\pgfsetlinewidth{0.000000pt}%
\definecolor{currentstroke}{rgb}{0.000000,0.000000,0.000000}%
\pgfsetstrokecolor{currentstroke}%
\pgfsetdash{}{0pt}%
\pgfpathmoveto{\pgfqpoint{3.887463in}{2.626353in}}%
\pgfpathlineto{\pgfqpoint{3.900373in}{2.621056in}}%
\pgfpathlineto{\pgfqpoint{3.913288in}{2.615949in}}%
\pgfpathlineto{\pgfqpoint{3.926207in}{2.611031in}}%
\pgfpathlineto{\pgfqpoint{3.939130in}{2.606301in}}%
\pgfpathlineto{\pgfqpoint{3.946646in}{2.617460in}}%
\pgfpathlineto{\pgfqpoint{3.954157in}{2.628698in}}%
\pgfpathlineto{\pgfqpoint{3.961663in}{2.640019in}}%
\pgfpathlineto{\pgfqpoint{3.969165in}{2.651426in}}%
\pgfpathlineto{\pgfqpoint{3.956249in}{2.656404in}}%
\pgfpathlineto{\pgfqpoint{3.943338in}{2.661569in}}%
\pgfpathlineto{\pgfqpoint{3.930432in}{2.666923in}}%
\pgfpathlineto{\pgfqpoint{3.917530in}{2.672468in}}%
\pgfpathlineto{\pgfqpoint{3.910020in}{2.660803in}}%
\pgfpathlineto{\pgfqpoint{3.902505in}{2.649231in}}%
\pgfpathlineto{\pgfqpoint{3.894987in}{2.637748in}}%
\pgfpathlineto{\pgfqpoint{3.887463in}{2.626353in}}%
\pgfpathclose%
\pgfusepath{fill}%
\end{pgfscope}%
\begin{pgfscope}%
\pgfpathrectangle{\pgfqpoint{1.254980in}{0.150000in}}{\pgfqpoint{5.490039in}{5.490039in}}%
\pgfusepath{clip}%
\pgfsetbuttcap%
\pgfsetroundjoin%
\definecolor{currentfill}{rgb}{0.266580,0.228262,0.514349}%
\pgfsetfillcolor{currentfill}%
\pgfsetfillopacity{0.700000}%
\pgfsetlinewidth{0.000000pt}%
\definecolor{currentstroke}{rgb}{0.000000,0.000000,0.000000}%
\pgfsetstrokecolor{currentstroke}%
\pgfsetdash{}{0pt}%
\pgfpathmoveto{\pgfqpoint{4.184217in}{2.693864in}}%
\pgfpathlineto{\pgfqpoint{4.197187in}{2.690675in}}%
\pgfpathlineto{\pgfqpoint{4.210164in}{2.687664in}}%
\pgfpathlineto{\pgfqpoint{4.223148in}{2.684828in}}%
\pgfpathlineto{\pgfqpoint{4.236138in}{2.682168in}}%
\pgfpathlineto{\pgfqpoint{4.243568in}{2.693222in}}%
\pgfpathlineto{\pgfqpoint{4.250994in}{2.704371in}}%
\pgfpathlineto{\pgfqpoint{4.258416in}{2.715617in}}%
\pgfpathlineto{\pgfqpoint{4.265834in}{2.726966in}}%
\pgfpathlineto{\pgfqpoint{4.252853in}{2.729956in}}%
\pgfpathlineto{\pgfqpoint{4.239878in}{2.733122in}}%
\pgfpathlineto{\pgfqpoint{4.226910in}{2.736464in}}%
\pgfpathlineto{\pgfqpoint{4.213949in}{2.739983in}}%
\pgfpathlineto{\pgfqpoint{4.206522in}{2.728294in}}%
\pgfpathlineto{\pgfqpoint{4.199091in}{2.716714in}}%
\pgfpathlineto{\pgfqpoint{4.191656in}{2.705239in}}%
\pgfpathlineto{\pgfqpoint{4.184217in}{2.693864in}}%
\pgfpathclose%
\pgfusepath{fill}%
\end{pgfscope}%
\begin{pgfscope}%
\pgfpathrectangle{\pgfqpoint{1.254980in}{0.150000in}}{\pgfqpoint{5.490039in}{5.490039in}}%
\pgfusepath{clip}%
\pgfsetbuttcap%
\pgfsetroundjoin%
\definecolor{currentfill}{rgb}{0.266580,0.228262,0.514349}%
\pgfsetfillcolor{currentfill}%
\pgfsetfillopacity{0.700000}%
\pgfsetlinewidth{0.000000pt}%
\definecolor{currentstroke}{rgb}{0.000000,0.000000,0.000000}%
\pgfsetstrokecolor{currentstroke}%
\pgfsetdash{}{0pt}%
\pgfpathmoveto{\pgfqpoint{3.353485in}{2.707045in}}%
\pgfpathlineto{\pgfqpoint{3.366382in}{2.695778in}}%
\pgfpathlineto{\pgfqpoint{3.379277in}{2.684741in}}%
\pgfpathlineto{\pgfqpoint{3.392171in}{2.673934in}}%
\pgfpathlineto{\pgfqpoint{3.405064in}{2.663354in}}%
\pgfpathlineto{\pgfqpoint{3.412731in}{2.674606in}}%
\pgfpathlineto{\pgfqpoint{3.420392in}{2.685955in}}%
\pgfpathlineto{\pgfqpoint{3.428048in}{2.697401in}}%
\pgfpathlineto{\pgfqpoint{3.435698in}{2.708947in}}%
\pgfpathlineto{\pgfqpoint{3.422815in}{2.719665in}}%
\pgfpathlineto{\pgfqpoint{3.409931in}{2.730609in}}%
\pgfpathlineto{\pgfqpoint{3.397046in}{2.741783in}}%
\pgfpathlineto{\pgfqpoint{3.384160in}{2.753188in}}%
\pgfpathlineto{\pgfqpoint{3.376500in}{2.741494in}}%
\pgfpathlineto{\pgfqpoint{3.368834in}{2.729906in}}%
\pgfpathlineto{\pgfqpoint{3.361163in}{2.718424in}}%
\pgfpathlineto{\pgfqpoint{3.353485in}{2.707045in}}%
\pgfpathclose%
\pgfusepath{fill}%
\end{pgfscope}%
\begin{pgfscope}%
\pgfpathrectangle{\pgfqpoint{1.254980in}{0.150000in}}{\pgfqpoint{5.490039in}{5.490039in}}%
\pgfusepath{clip}%
\pgfsetbuttcap%
\pgfsetroundjoin%
\definecolor{currentfill}{rgb}{0.177423,0.437527,0.557565}%
\pgfsetfillcolor{currentfill}%
\pgfsetfillopacity{0.700000}%
\pgfsetlinewidth{0.000000pt}%
\definecolor{currentstroke}{rgb}{0.000000,0.000000,0.000000}%
\pgfsetstrokecolor{currentstroke}%
\pgfsetdash{}{0pt}%
\pgfpathmoveto{\pgfqpoint{2.938824in}{3.201220in}}%
\pgfpathlineto{\pgfqpoint{2.951879in}{3.181474in}}%
\pgfpathlineto{\pgfqpoint{2.964926in}{3.162033in}}%
\pgfpathlineto{\pgfqpoint{2.977964in}{3.142892in}}%
\pgfpathlineto{\pgfqpoint{2.990994in}{3.124049in}}%
\pgfpathlineto{\pgfqpoint{2.998753in}{3.136321in}}%
\pgfpathlineto{\pgfqpoint{3.006504in}{3.148744in}}%
\pgfpathlineto{\pgfqpoint{3.014249in}{3.161322in}}%
\pgfpathlineto{\pgfqpoint{3.021986in}{3.174055in}}%
\pgfpathlineto{\pgfqpoint{3.008969in}{3.193036in}}%
\pgfpathlineto{\pgfqpoint{2.995943in}{3.212316in}}%
\pgfpathlineto{\pgfqpoint{2.982910in}{3.231897in}}%
\pgfpathlineto{\pgfqpoint{2.969867in}{3.251782in}}%
\pgfpathlineto{\pgfqpoint{2.962118in}{3.238899in}}%
\pgfpathlineto{\pgfqpoint{2.954361in}{3.226179in}}%
\pgfpathlineto{\pgfqpoint{2.946596in}{3.213620in}}%
\pgfpathlineto{\pgfqpoint{2.938824in}{3.201220in}}%
\pgfpathclose%
\pgfusepath{fill}%
\end{pgfscope}%
\begin{pgfscope}%
\pgfpathrectangle{\pgfqpoint{1.254980in}{0.150000in}}{\pgfqpoint{5.490039in}{5.490039in}}%
\pgfusepath{clip}%
\pgfsetbuttcap%
\pgfsetroundjoin%
\definecolor{currentfill}{rgb}{0.277134,0.185228,0.489898}%
\pgfsetfillcolor{currentfill}%
\pgfsetfillopacity{0.700000}%
\pgfsetlinewidth{0.000000pt}%
\definecolor{currentstroke}{rgb}{0.000000,0.000000,0.000000}%
\pgfsetstrokecolor{currentstroke}%
\pgfsetdash{}{0pt}%
\pgfpathmoveto{\pgfqpoint{3.672260in}{2.611750in}}%
\pgfpathlineto{\pgfqpoint{3.685148in}{2.604509in}}%
\pgfpathlineto{\pgfqpoint{3.698038in}{2.597470in}}%
\pgfpathlineto{\pgfqpoint{3.710931in}{2.590633in}}%
\pgfpathlineto{\pgfqpoint{3.723826in}{2.583995in}}%
\pgfpathlineto{\pgfqpoint{3.731404in}{2.595162in}}%
\pgfpathlineto{\pgfqpoint{3.738978in}{2.606407in}}%
\pgfpathlineto{\pgfqpoint{3.746546in}{2.617734in}}%
\pgfpathlineto{\pgfqpoint{3.754110in}{2.629144in}}%
\pgfpathlineto{\pgfqpoint{3.741223in}{2.635974in}}%
\pgfpathlineto{\pgfqpoint{3.728339in}{2.643004in}}%
\pgfpathlineto{\pgfqpoint{3.715457in}{2.650235in}}%
\pgfpathlineto{\pgfqpoint{3.702578in}{2.657669in}}%
\pgfpathlineto{\pgfqpoint{3.695006in}{2.646056in}}%
\pgfpathlineto{\pgfqpoint{3.687429in}{2.634533in}}%
\pgfpathlineto{\pgfqpoint{3.679847in}{2.623099in}}%
\pgfpathlineto{\pgfqpoint{3.672260in}{2.611750in}}%
\pgfpathclose%
\pgfusepath{fill}%
\end{pgfscope}%
\begin{pgfscope}%
\pgfpathrectangle{\pgfqpoint{1.254980in}{0.150000in}}{\pgfqpoint{5.490039in}{5.490039in}}%
\pgfusepath{clip}%
\pgfsetbuttcap%
\pgfsetroundjoin%
\definecolor{currentfill}{rgb}{0.275191,0.194905,0.496005}%
\pgfsetfillcolor{currentfill}%
\pgfsetfillopacity{0.700000}%
\pgfsetlinewidth{0.000000pt}%
\definecolor{currentstroke}{rgb}{0.000000,0.000000,0.000000}%
\pgfsetstrokecolor{currentstroke}%
\pgfsetdash{}{0pt}%
\pgfpathmoveto{\pgfqpoint{3.538754in}{2.631207in}}%
\pgfpathlineto{\pgfqpoint{3.551638in}{2.622467in}}%
\pgfpathlineto{\pgfqpoint{3.564522in}{2.613939in}}%
\pgfpathlineto{\pgfqpoint{3.577408in}{2.605623in}}%
\pgfpathlineto{\pgfqpoint{3.590296in}{2.597517in}}%
\pgfpathlineto{\pgfqpoint{3.597912in}{2.608701in}}%
\pgfpathlineto{\pgfqpoint{3.605523in}{2.619968in}}%
\pgfpathlineto{\pgfqpoint{3.613129in}{2.631322in}}%
\pgfpathlineto{\pgfqpoint{3.620730in}{2.642763in}}%
\pgfpathlineto{\pgfqpoint{3.607852in}{2.651034in}}%
\pgfpathlineto{\pgfqpoint{3.594975in}{2.659515in}}%
\pgfpathlineto{\pgfqpoint{3.582099in}{2.668208in}}%
\pgfpathlineto{\pgfqpoint{3.569224in}{2.677113in}}%
\pgfpathlineto{\pgfqpoint{3.561615in}{2.665496in}}%
\pgfpathlineto{\pgfqpoint{3.554000in}{2.653975in}}%
\pgfpathlineto{\pgfqpoint{3.546380in}{2.642546in}}%
\pgfpathlineto{\pgfqpoint{3.538754in}{2.631207in}}%
\pgfpathclose%
\pgfusepath{fill}%
\end{pgfscope}%
\begin{pgfscope}%
\pgfpathrectangle{\pgfqpoint{1.254980in}{0.150000in}}{\pgfqpoint{5.490039in}{5.490039in}}%
\pgfusepath{clip}%
\pgfsetbuttcap%
\pgfsetroundjoin%
\definecolor{currentfill}{rgb}{0.270595,0.214069,0.507052}%
\pgfsetfillcolor{currentfill}%
\pgfsetfillopacity{0.700000}%
\pgfsetlinewidth{0.000000pt}%
\definecolor{currentstroke}{rgb}{0.000000,0.000000,0.000000}%
\pgfsetstrokecolor{currentstroke}%
\pgfsetdash{}{0pt}%
\pgfpathmoveto{\pgfqpoint{4.102567in}{2.662620in}}%
\pgfpathlineto{\pgfqpoint{4.115521in}{2.659020in}}%
\pgfpathlineto{\pgfqpoint{4.128482in}{2.655600in}}%
\pgfpathlineto{\pgfqpoint{4.141448in}{2.652359in}}%
\pgfpathlineto{\pgfqpoint{4.154421in}{2.649296in}}%
\pgfpathlineto{\pgfqpoint{4.161876in}{2.660305in}}%
\pgfpathlineto{\pgfqpoint{4.169327in}{2.671401in}}%
\pgfpathlineto{\pgfqpoint{4.176774in}{2.682586in}}%
\pgfpathlineto{\pgfqpoint{4.184217in}{2.693864in}}%
\pgfpathlineto{\pgfqpoint{4.171253in}{2.697230in}}%
\pgfpathlineto{\pgfqpoint{4.158295in}{2.700774in}}%
\pgfpathlineto{\pgfqpoint{4.145343in}{2.704497in}}%
\pgfpathlineto{\pgfqpoint{4.132397in}{2.708399in}}%
\pgfpathlineto{\pgfqpoint{4.124946in}{2.696808in}}%
\pgfpathlineto{\pgfqpoint{4.117490in}{2.685317in}}%
\pgfpathlineto{\pgfqpoint{4.110031in}{2.673922in}}%
\pgfpathlineto{\pgfqpoint{4.102567in}{2.662620in}}%
\pgfpathclose%
\pgfusepath{fill}%
\end{pgfscope}%
\begin{pgfscope}%
\pgfpathrectangle{\pgfqpoint{1.254980in}{0.150000in}}{\pgfqpoint{5.490039in}{5.490039in}}%
\pgfusepath{clip}%
\pgfsetbuttcap%
\pgfsetroundjoin%
\definecolor{currentfill}{rgb}{0.277134,0.185228,0.489898}%
\pgfsetfillcolor{currentfill}%
\pgfsetfillopacity{0.700000}%
\pgfsetlinewidth{0.000000pt}%
\definecolor{currentstroke}{rgb}{0.000000,0.000000,0.000000}%
\pgfsetstrokecolor{currentstroke}%
\pgfsetdash{}{0pt}%
\pgfpathmoveto{\pgfqpoint{3.805688in}{2.603804in}}%
\pgfpathlineto{\pgfqpoint{3.818591in}{2.597959in}}%
\pgfpathlineto{\pgfqpoint{3.831497in}{2.592307in}}%
\pgfpathlineto{\pgfqpoint{3.844408in}{2.586848in}}%
\pgfpathlineto{\pgfqpoint{3.857322in}{2.581581in}}%
\pgfpathlineto{\pgfqpoint{3.864864in}{2.592658in}}%
\pgfpathlineto{\pgfqpoint{3.872402in}{2.603811in}}%
\pgfpathlineto{\pgfqpoint{3.879935in}{2.615041in}}%
\pgfpathlineto{\pgfqpoint{3.887463in}{2.626353in}}%
\pgfpathlineto{\pgfqpoint{3.874557in}{2.631840in}}%
\pgfpathlineto{\pgfqpoint{3.861655in}{2.637520in}}%
\pgfpathlineto{\pgfqpoint{3.848756in}{2.643392in}}%
\pgfpathlineto{\pgfqpoint{3.835862in}{2.649457in}}%
\pgfpathlineto{\pgfqpoint{3.828325in}{2.637915in}}%
\pgfpathlineto{\pgfqpoint{3.820784in}{2.626461in}}%
\pgfpathlineto{\pgfqpoint{3.813238in}{2.615092in}}%
\pgfpathlineto{\pgfqpoint{3.805688in}{2.603804in}}%
\pgfpathclose%
\pgfusepath{fill}%
\end{pgfscope}%
\begin{pgfscope}%
\pgfpathrectangle{\pgfqpoint{1.254980in}{0.150000in}}{\pgfqpoint{5.490039in}{5.490039in}}%
\pgfusepath{clip}%
\pgfsetbuttcap%
\pgfsetroundjoin%
\definecolor{currentfill}{rgb}{0.214298,0.355619,0.551184}%
\pgfsetfillcolor{currentfill}%
\pgfsetfillopacity{0.700000}%
\pgfsetlinewidth{0.000000pt}%
\definecolor{currentstroke}{rgb}{0.000000,0.000000,0.000000}%
\pgfsetstrokecolor{currentstroke}%
\pgfsetdash{}{0pt}%
\pgfpathmoveto{\pgfqpoint{4.807790in}{2.954792in}}%
\pgfpathlineto{\pgfqpoint{4.820942in}{2.954066in}}%
\pgfpathlineto{\pgfqpoint{4.834104in}{2.953500in}}%
\pgfpathlineto{\pgfqpoint{4.847276in}{2.953093in}}%
\pgfpathlineto{\pgfqpoint{4.860458in}{2.952845in}}%
\pgfpathlineto{\pgfqpoint{4.867715in}{2.964151in}}%
\pgfpathlineto{\pgfqpoint{4.874971in}{2.975634in}}%
\pgfpathlineto{\pgfqpoint{4.882225in}{2.987301in}}%
\pgfpathlineto{\pgfqpoint{4.889479in}{2.999157in}}%
\pgfpathlineto{\pgfqpoint{4.876312in}{2.999931in}}%
\pgfpathlineto{\pgfqpoint{4.863155in}{3.000862in}}%
\pgfpathlineto{\pgfqpoint{4.850007in}{3.001953in}}%
\pgfpathlineto{\pgfqpoint{4.836869in}{3.003202in}}%
\pgfpathlineto{\pgfqpoint{4.829600in}{2.990811in}}%
\pgfpathlineto{\pgfqpoint{4.822331in}{2.978617in}}%
\pgfpathlineto{\pgfqpoint{4.815061in}{2.966612in}}%
\pgfpathlineto{\pgfqpoint{4.807790in}{2.954792in}}%
\pgfpathclose%
\pgfusepath{fill}%
\end{pgfscope}%
\begin{pgfscope}%
\pgfpathrectangle{\pgfqpoint{1.254980in}{0.150000in}}{\pgfqpoint{5.490039in}{5.490039in}}%
\pgfusepath{clip}%
\pgfsetbuttcap%
\pgfsetroundjoin%
\definecolor{currentfill}{rgb}{0.223925,0.334994,0.548053}%
\pgfsetfillcolor{currentfill}%
\pgfsetfillopacity{0.700000}%
\pgfsetlinewidth{0.000000pt}%
\definecolor{currentstroke}{rgb}{0.000000,0.000000,0.000000}%
\pgfsetstrokecolor{currentstroke}%
\pgfsetdash{}{0pt}%
\pgfpathmoveto{\pgfqpoint{4.726119in}{2.911746in}}%
\pgfpathlineto{\pgfqpoint{4.739247in}{2.910877in}}%
\pgfpathlineto{\pgfqpoint{4.752385in}{2.910169in}}%
\pgfpathlineto{\pgfqpoint{4.765532in}{2.909621in}}%
\pgfpathlineto{\pgfqpoint{4.778689in}{2.909233in}}%
\pgfpathlineto{\pgfqpoint{4.785967in}{2.920376in}}%
\pgfpathlineto{\pgfqpoint{4.793243in}{2.931679in}}%
\pgfpathlineto{\pgfqpoint{4.800517in}{2.943149in}}%
\pgfpathlineto{\pgfqpoint{4.807790in}{2.954792in}}%
\pgfpathlineto{\pgfqpoint{4.794647in}{2.955676in}}%
\pgfpathlineto{\pgfqpoint{4.781513in}{2.956721in}}%
\pgfpathlineto{\pgfqpoint{4.768389in}{2.957926in}}%
\pgfpathlineto{\pgfqpoint{4.755274in}{2.959291in}}%
\pgfpathlineto{\pgfqpoint{4.747988in}{2.947142in}}%
\pgfpathlineto{\pgfqpoint{4.740700in}{2.935172in}}%
\pgfpathlineto{\pgfqpoint{4.733410in}{2.923375in}}%
\pgfpathlineto{\pgfqpoint{4.726119in}{2.911746in}}%
\pgfpathclose%
\pgfusepath{fill}%
\end{pgfscope}%
\begin{pgfscope}%
\pgfpathrectangle{\pgfqpoint{1.254980in}{0.150000in}}{\pgfqpoint{5.490039in}{5.490039in}}%
\pgfusepath{clip}%
\pgfsetbuttcap%
\pgfsetroundjoin%
\definecolor{currentfill}{rgb}{0.204903,0.375746,0.553533}%
\pgfsetfillcolor{currentfill}%
\pgfsetfillopacity{0.700000}%
\pgfsetlinewidth{0.000000pt}%
\definecolor{currentstroke}{rgb}{0.000000,0.000000,0.000000}%
\pgfsetstrokecolor{currentstroke}%
\pgfsetdash{}{0pt}%
\pgfpathmoveto{\pgfqpoint{4.889479in}{2.999157in}}%
\pgfpathlineto{\pgfqpoint{4.902656in}{2.998542in}}%
\pgfpathlineto{\pgfqpoint{4.915842in}{2.998084in}}%
\pgfpathlineto{\pgfqpoint{4.929039in}{2.997783in}}%
\pgfpathlineto{\pgfqpoint{4.942246in}{2.997639in}}%
\pgfpathlineto{\pgfqpoint{4.949484in}{3.009149in}}%
\pgfpathlineto{\pgfqpoint{4.956721in}{3.020854in}}%
\pgfpathlineto{\pgfqpoint{4.963957in}{3.032760in}}%
\pgfpathlineto{\pgfqpoint{4.971194in}{3.044874in}}%
\pgfpathlineto{\pgfqpoint{4.958002in}{3.045571in}}%
\pgfpathlineto{\pgfqpoint{4.944821in}{3.046424in}}%
\pgfpathlineto{\pgfqpoint{4.931650in}{3.047434in}}%
\pgfpathlineto{\pgfqpoint{4.918488in}{3.048601in}}%
\pgfpathlineto{\pgfqpoint{4.911237in}{3.035925in}}%
\pgfpathlineto{\pgfqpoint{4.903984in}{3.023463in}}%
\pgfpathlineto{\pgfqpoint{4.896732in}{3.011209in}}%
\pgfpathlineto{\pgfqpoint{4.889479in}{2.999157in}}%
\pgfpathclose%
\pgfusepath{fill}%
\end{pgfscope}%
\begin{pgfscope}%
\pgfpathrectangle{\pgfqpoint{1.254980in}{0.150000in}}{\pgfqpoint{5.490039in}{5.490039in}}%
\pgfusepath{clip}%
\pgfsetbuttcap%
\pgfsetroundjoin%
\definecolor{currentfill}{rgb}{0.231674,0.318106,0.544834}%
\pgfsetfillcolor{currentfill}%
\pgfsetfillopacity{0.700000}%
\pgfsetlinewidth{0.000000pt}%
\definecolor{currentstroke}{rgb}{0.000000,0.000000,0.000000}%
\pgfsetstrokecolor{currentstroke}%
\pgfsetdash{}{0pt}%
\pgfpathmoveto{\pgfqpoint{4.644461in}{2.870012in}}%
\pgfpathlineto{\pgfqpoint{4.657565in}{2.868963in}}%
\pgfpathlineto{\pgfqpoint{4.670679in}{2.868078in}}%
\pgfpathlineto{\pgfqpoint{4.683801in}{2.867355in}}%
\pgfpathlineto{\pgfqpoint{4.696933in}{2.866793in}}%
\pgfpathlineto{\pgfqpoint{4.704233in}{2.877808in}}%
\pgfpathlineto{\pgfqpoint{4.711531in}{2.888968in}}%
\pgfpathlineto{\pgfqpoint{4.718826in}{2.900279in}}%
\pgfpathlineto{\pgfqpoint{4.726119in}{2.911746in}}%
\pgfpathlineto{\pgfqpoint{4.713000in}{2.912777in}}%
\pgfpathlineto{\pgfqpoint{4.699890in}{2.913969in}}%
\pgfpathlineto{\pgfqpoint{4.686789in}{2.915323in}}%
\pgfpathlineto{\pgfqpoint{4.673697in}{2.916840in}}%
\pgfpathlineto{\pgfqpoint{4.666391in}{2.904894in}}%
\pgfpathlineto{\pgfqpoint{4.659083in}{2.893111in}}%
\pgfpathlineto{\pgfqpoint{4.651773in}{2.881485in}}%
\pgfpathlineto{\pgfqpoint{4.644461in}{2.870012in}}%
\pgfpathclose%
\pgfusepath{fill}%
\end{pgfscope}%
\begin{pgfscope}%
\pgfpathrectangle{\pgfqpoint{1.254980in}{0.150000in}}{\pgfqpoint{5.490039in}{5.490039in}}%
\pgfusepath{clip}%
\pgfsetbuttcap%
\pgfsetroundjoin%
\definecolor{currentfill}{rgb}{0.271828,0.209303,0.504434}%
\pgfsetfillcolor{currentfill}%
\pgfsetfillopacity{0.700000}%
\pgfsetlinewidth{0.000000pt}%
\definecolor{currentstroke}{rgb}{0.000000,0.000000,0.000000}%
\pgfsetstrokecolor{currentstroke}%
\pgfsetdash{}{0pt}%
\pgfpathmoveto{\pgfqpoint{3.405064in}{2.663354in}}%
\pgfpathlineto{\pgfqpoint{3.417956in}{2.653000in}}%
\pgfpathlineto{\pgfqpoint{3.430848in}{2.642870in}}%
\pgfpathlineto{\pgfqpoint{3.443739in}{2.632963in}}%
\pgfpathlineto{\pgfqpoint{3.456630in}{2.623277in}}%
\pgfpathlineto{\pgfqpoint{3.464287in}{2.634402in}}%
\pgfpathlineto{\pgfqpoint{3.471939in}{2.645616in}}%
\pgfpathlineto{\pgfqpoint{3.479585in}{2.656921in}}%
\pgfpathlineto{\pgfqpoint{3.487225in}{2.668320in}}%
\pgfpathlineto{\pgfqpoint{3.474344in}{2.678143in}}%
\pgfpathlineto{\pgfqpoint{3.461462in}{2.688188in}}%
\pgfpathlineto{\pgfqpoint{3.448580in}{2.698456in}}%
\pgfpathlineto{\pgfqpoint{3.435698in}{2.708947in}}%
\pgfpathlineto{\pgfqpoint{3.428048in}{2.697401in}}%
\pgfpathlineto{\pgfqpoint{3.420392in}{2.685955in}}%
\pgfpathlineto{\pgfqpoint{3.412731in}{2.674606in}}%
\pgfpathlineto{\pgfqpoint{3.405064in}{2.663354in}}%
\pgfpathclose%
\pgfusepath{fill}%
\end{pgfscope}%
\begin{pgfscope}%
\pgfpathrectangle{\pgfqpoint{1.254980in}{0.150000in}}{\pgfqpoint{5.490039in}{5.490039in}}%
\pgfusepath{clip}%
\pgfsetbuttcap%
\pgfsetroundjoin%
\definecolor{currentfill}{rgb}{0.197636,0.391528,0.554969}%
\pgfsetfillcolor{currentfill}%
\pgfsetfillopacity{0.700000}%
\pgfsetlinewidth{0.000000pt}%
\definecolor{currentstroke}{rgb}{0.000000,0.000000,0.000000}%
\pgfsetstrokecolor{currentstroke}%
\pgfsetdash{}{0pt}%
\pgfpathmoveto{\pgfqpoint{4.971194in}{3.044874in}}%
\pgfpathlineto{\pgfqpoint{4.984395in}{3.044334in}}%
\pgfpathlineto{\pgfqpoint{4.997606in}{3.043950in}}%
\pgfpathlineto{\pgfqpoint{5.010828in}{3.043721in}}%
\pgfpathlineto{\pgfqpoint{5.024060in}{3.043648in}}%
\pgfpathlineto{\pgfqpoint{5.031280in}{3.055407in}}%
\pgfpathlineto{\pgfqpoint{5.038500in}{3.067380in}}%
\pgfpathlineto{\pgfqpoint{5.045720in}{3.079573in}}%
\pgfpathlineto{\pgfqpoint{5.052941in}{3.091995in}}%
\pgfpathlineto{\pgfqpoint{5.039726in}{3.092648in}}%
\pgfpathlineto{\pgfqpoint{5.026521in}{3.093457in}}%
\pgfpathlineto{\pgfqpoint{5.013326in}{3.094420in}}%
\pgfpathlineto{\pgfqpoint{5.000141in}{3.095540in}}%
\pgfpathlineto{\pgfqpoint{4.992904in}{3.082529in}}%
\pgfpathlineto{\pgfqpoint{4.985667in}{3.069752in}}%
\pgfpathlineto{\pgfqpoint{4.978430in}{3.057203in}}%
\pgfpathlineto{\pgfqpoint{4.971194in}{3.044874in}}%
\pgfpathclose%
\pgfusepath{fill}%
\end{pgfscope}%
\begin{pgfscope}%
\pgfpathrectangle{\pgfqpoint{1.254980in}{0.150000in}}{\pgfqpoint{5.490039in}{5.490039in}}%
\pgfusepath{clip}%
\pgfsetbuttcap%
\pgfsetroundjoin%
\definecolor{currentfill}{rgb}{0.239346,0.300855,0.540844}%
\pgfsetfillcolor{currentfill}%
\pgfsetfillopacity{0.700000}%
\pgfsetlinewidth{0.000000pt}%
\definecolor{currentstroke}{rgb}{0.000000,0.000000,0.000000}%
\pgfsetstrokecolor{currentstroke}%
\pgfsetdash{}{0pt}%
\pgfpathmoveto{\pgfqpoint{4.562808in}{2.829600in}}%
\pgfpathlineto{\pgfqpoint{4.575890in}{2.828337in}}%
\pgfpathlineto{\pgfqpoint{4.588980in}{2.827239in}}%
\pgfpathlineto{\pgfqpoint{4.602078in}{2.826306in}}%
\pgfpathlineto{\pgfqpoint{4.615186in}{2.825536in}}%
\pgfpathlineto{\pgfqpoint{4.622509in}{2.836452in}}%
\pgfpathlineto{\pgfqpoint{4.629829in}{2.847501in}}%
\pgfpathlineto{\pgfqpoint{4.637146in}{2.858685in}}%
\pgfpathlineto{\pgfqpoint{4.644461in}{2.870012in}}%
\pgfpathlineto{\pgfqpoint{4.631365in}{2.871223in}}%
\pgfpathlineto{\pgfqpoint{4.618278in}{2.872599in}}%
\pgfpathlineto{\pgfqpoint{4.605200in}{2.874138in}}%
\pgfpathlineto{\pgfqpoint{4.592130in}{2.875842in}}%
\pgfpathlineto{\pgfqpoint{4.584804in}{2.864064in}}%
\pgfpathlineto{\pgfqpoint{4.577475in}{2.852435in}}%
\pgfpathlineto{\pgfqpoint{4.570143in}{2.840948in}}%
\pgfpathlineto{\pgfqpoint{4.562808in}{2.829600in}}%
\pgfpathclose%
\pgfusepath{fill}%
\end{pgfscope}%
\begin{pgfscope}%
\pgfpathrectangle{\pgfqpoint{1.254980in}{0.150000in}}{\pgfqpoint{5.490039in}{5.490039in}}%
\pgfusepath{clip}%
\pgfsetbuttcap%
\pgfsetroundjoin%
\definecolor{currentfill}{rgb}{0.273006,0.204520,0.501721}%
\pgfsetfillcolor{currentfill}%
\pgfsetfillopacity{0.700000}%
\pgfsetlinewidth{0.000000pt}%
\definecolor{currentstroke}{rgb}{0.000000,0.000000,0.000000}%
\pgfsetstrokecolor{currentstroke}%
\pgfsetdash{}{0pt}%
\pgfpathmoveto{\pgfqpoint{4.020875in}{2.633377in}}%
\pgfpathlineto{\pgfqpoint{4.033815in}{2.629326in}}%
\pgfpathlineto{\pgfqpoint{4.046761in}{2.625458in}}%
\pgfpathlineto{\pgfqpoint{4.059712in}{2.621772in}}%
\pgfpathlineto{\pgfqpoint{4.072669in}{2.618267in}}%
\pgfpathlineto{\pgfqpoint{4.080150in}{2.629234in}}%
\pgfpathlineto{\pgfqpoint{4.087627in}{2.640279in}}%
\pgfpathlineto{\pgfqpoint{4.095099in}{2.651407in}}%
\pgfpathlineto{\pgfqpoint{4.102567in}{2.662620in}}%
\pgfpathlineto{\pgfqpoint{4.089618in}{2.666400in}}%
\pgfpathlineto{\pgfqpoint{4.076675in}{2.670362in}}%
\pgfpathlineto{\pgfqpoint{4.063738in}{2.674505in}}%
\pgfpathlineto{\pgfqpoint{4.050806in}{2.678831in}}%
\pgfpathlineto{\pgfqpoint{4.043329in}{2.667332in}}%
\pgfpathlineto{\pgfqpoint{4.035849in}{2.655926in}}%
\pgfpathlineto{\pgfqpoint{4.028364in}{2.644609in}}%
\pgfpathlineto{\pgfqpoint{4.020875in}{2.633377in}}%
\pgfpathclose%
\pgfusepath{fill}%
\end{pgfscope}%
\begin{pgfscope}%
\pgfpathrectangle{\pgfqpoint{1.254980in}{0.150000in}}{\pgfqpoint{5.490039in}{5.490039in}}%
\pgfusepath{clip}%
\pgfsetbuttcap%
\pgfsetroundjoin%
\definecolor{currentfill}{rgb}{0.187231,0.414746,0.556547}%
\pgfsetfillcolor{currentfill}%
\pgfsetfillopacity{0.700000}%
\pgfsetlinewidth{0.000000pt}%
\definecolor{currentstroke}{rgb}{0.000000,0.000000,0.000000}%
\pgfsetstrokecolor{currentstroke}%
\pgfsetdash{}{0pt}%
\pgfpathmoveto{\pgfqpoint{5.052941in}{3.091995in}}%
\pgfpathlineto{\pgfqpoint{5.066166in}{3.091496in}}%
\pgfpathlineto{\pgfqpoint{5.079402in}{3.091152in}}%
\pgfpathlineto{\pgfqpoint{5.092649in}{3.090962in}}%
\pgfpathlineto{\pgfqpoint{5.105906in}{3.090926in}}%
\pgfpathlineto{\pgfqpoint{5.113110in}{3.102984in}}%
\pgfpathlineto{\pgfqpoint{5.120315in}{3.115276in}}%
\pgfpathlineto{\pgfqpoint{5.127521in}{3.127810in}}%
\pgfpathlineto{\pgfqpoint{5.134729in}{3.140592in}}%
\pgfpathlineto{\pgfqpoint{5.121490in}{3.141236in}}%
\pgfpathlineto{\pgfqpoint{5.108262in}{3.142034in}}%
\pgfpathlineto{\pgfqpoint{5.095044in}{3.142985in}}%
\pgfpathlineto{\pgfqpoint{5.081836in}{3.144091in}}%
\pgfpathlineto{\pgfqpoint{5.074610in}{3.130691in}}%
\pgfpathlineto{\pgfqpoint{5.067386in}{3.117547in}}%
\pgfpathlineto{\pgfqpoint{5.060163in}{3.104650in}}%
\pgfpathlineto{\pgfqpoint{5.052941in}{3.091995in}}%
\pgfpathclose%
\pgfusepath{fill}%
\end{pgfscope}%
\begin{pgfscope}%
\pgfpathrectangle{\pgfqpoint{1.254980in}{0.150000in}}{\pgfqpoint{5.490039in}{5.490039in}}%
\pgfusepath{clip}%
\pgfsetbuttcap%
\pgfsetroundjoin%
\definecolor{currentfill}{rgb}{0.246811,0.283237,0.535941}%
\pgfsetfillcolor{currentfill}%
\pgfsetfillopacity{0.700000}%
\pgfsetlinewidth{0.000000pt}%
\definecolor{currentstroke}{rgb}{0.000000,0.000000,0.000000}%
\pgfsetstrokecolor{currentstroke}%
\pgfsetdash{}{0pt}%
\pgfpathmoveto{\pgfqpoint{4.481156in}{2.790545in}}%
\pgfpathlineto{\pgfqpoint{4.494215in}{2.789032in}}%
\pgfpathlineto{\pgfqpoint{4.507282in}{2.787685in}}%
\pgfpathlineto{\pgfqpoint{4.520357in}{2.786505in}}%
\pgfpathlineto{\pgfqpoint{4.533441in}{2.785491in}}%
\pgfpathlineto{\pgfqpoint{4.540788in}{2.796336in}}%
\pgfpathlineto{\pgfqpoint{4.548131in}{2.807299in}}%
\pgfpathlineto{\pgfqpoint{4.555471in}{2.818385in}}%
\pgfpathlineto{\pgfqpoint{4.562808in}{2.829600in}}%
\pgfpathlineto{\pgfqpoint{4.549736in}{2.831028in}}%
\pgfpathlineto{\pgfqpoint{4.536671in}{2.832622in}}%
\pgfpathlineto{\pgfqpoint{4.523615in}{2.834382in}}%
\pgfpathlineto{\pgfqpoint{4.510567in}{2.836309in}}%
\pgfpathlineto{\pgfqpoint{4.503219in}{2.824670in}}%
\pgfpathlineto{\pgfqpoint{4.495868in}{2.813167in}}%
\pgfpathlineto{\pgfqpoint{4.488514in}{2.801793in}}%
\pgfpathlineto{\pgfqpoint{4.481156in}{2.790545in}}%
\pgfpathclose%
\pgfusepath{fill}%
\end{pgfscope}%
\begin{pgfscope}%
\pgfpathrectangle{\pgfqpoint{1.254980in}{0.150000in}}{\pgfqpoint{5.490039in}{5.490039in}}%
\pgfusepath{clip}%
\pgfsetbuttcap%
\pgfsetroundjoin%
\definecolor{currentfill}{rgb}{0.235526,0.309527,0.542944}%
\pgfsetfillcolor{currentfill}%
\pgfsetfillopacity{0.700000}%
\pgfsetlinewidth{0.000000pt}%
\definecolor{currentstroke}{rgb}{0.000000,0.000000,0.000000}%
\pgfsetstrokecolor{currentstroke}%
\pgfsetdash{}{0pt}%
\pgfpathmoveto{\pgfqpoint{3.115844in}{2.873905in}}%
\pgfpathlineto{\pgfqpoint{3.128802in}{2.858802in}}%
\pgfpathlineto{\pgfqpoint{3.141755in}{2.843962in}}%
\pgfpathlineto{\pgfqpoint{3.154703in}{2.829381in}}%
\pgfpathlineto{\pgfqpoint{3.167647in}{2.815057in}}%
\pgfpathlineto{\pgfqpoint{3.175384in}{2.826340in}}%
\pgfpathlineto{\pgfqpoint{3.183113in}{2.837740in}}%
\pgfpathlineto{\pgfqpoint{3.190837in}{2.849258in}}%
\pgfpathlineto{\pgfqpoint{3.198553in}{2.860897in}}%
\pgfpathlineto{\pgfqpoint{3.185621in}{2.875330in}}%
\pgfpathlineto{\pgfqpoint{3.172685in}{2.890021in}}%
\pgfpathlineto{\pgfqpoint{3.159744in}{2.904972in}}%
\pgfpathlineto{\pgfqpoint{3.146799in}{2.920185in}}%
\pgfpathlineto{\pgfqpoint{3.139070in}{2.908425in}}%
\pgfpathlineto{\pgfqpoint{3.131335in}{2.896794in}}%
\pgfpathlineto{\pgfqpoint{3.123593in}{2.885287in}}%
\pgfpathlineto{\pgfqpoint{3.115844in}{2.873905in}}%
\pgfpathclose%
\pgfusepath{fill}%
\end{pgfscope}%
\begin{pgfscope}%
\pgfpathrectangle{\pgfqpoint{1.254980in}{0.150000in}}{\pgfqpoint{5.490039in}{5.490039in}}%
\pgfusepath{clip}%
\pgfsetbuttcap%
\pgfsetroundjoin%
\definecolor{currentfill}{rgb}{0.223925,0.334994,0.548053}%
\pgfsetfillcolor{currentfill}%
\pgfsetfillopacity{0.700000}%
\pgfsetlinewidth{0.000000pt}%
\definecolor{currentstroke}{rgb}{0.000000,0.000000,0.000000}%
\pgfsetstrokecolor{currentstroke}%
\pgfsetdash{}{0pt}%
\pgfpathmoveto{\pgfqpoint{3.063959in}{2.936983in}}%
\pgfpathlineto{\pgfqpoint{3.076939in}{2.920809in}}%
\pgfpathlineto{\pgfqpoint{3.089913in}{2.904906in}}%
\pgfpathlineto{\pgfqpoint{3.102881in}{2.889272in}}%
\pgfpathlineto{\pgfqpoint{3.115844in}{2.873905in}}%
\pgfpathlineto{\pgfqpoint{3.123593in}{2.885287in}}%
\pgfpathlineto{\pgfqpoint{3.131335in}{2.896794in}}%
\pgfpathlineto{\pgfqpoint{3.139070in}{2.908425in}}%
\pgfpathlineto{\pgfqpoint{3.146799in}{2.920185in}}%
\pgfpathlineto{\pgfqpoint{3.133848in}{2.935662in}}%
\pgfpathlineto{\pgfqpoint{3.120892in}{2.951405in}}%
\pgfpathlineto{\pgfqpoint{3.107931in}{2.967418in}}%
\pgfpathlineto{\pgfqpoint{3.094964in}{2.983702in}}%
\pgfpathlineto{\pgfqpoint{3.087223in}{2.971822in}}%
\pgfpathlineto{\pgfqpoint{3.079476in}{2.960077in}}%
\pgfpathlineto{\pgfqpoint{3.071721in}{2.948465in}}%
\pgfpathlineto{\pgfqpoint{3.063959in}{2.936983in}}%
\pgfpathclose%
\pgfusepath{fill}%
\end{pgfscope}%
\begin{pgfscope}%
\pgfpathrectangle{\pgfqpoint{1.254980in}{0.150000in}}{\pgfqpoint{5.490039in}{5.490039in}}%
\pgfusepath{clip}%
\pgfsetbuttcap%
\pgfsetroundjoin%
\definecolor{currentfill}{rgb}{0.246811,0.283237,0.535941}%
\pgfsetfillcolor{currentfill}%
\pgfsetfillopacity{0.700000}%
\pgfsetlinewidth{0.000000pt}%
\definecolor{currentstroke}{rgb}{0.000000,0.000000,0.000000}%
\pgfsetstrokecolor{currentstroke}%
\pgfsetdash{}{0pt}%
\pgfpathmoveto{\pgfqpoint{3.167647in}{2.815057in}}%
\pgfpathlineto{\pgfqpoint{3.180587in}{2.800989in}}%
\pgfpathlineto{\pgfqpoint{3.193523in}{2.787175in}}%
\pgfpathlineto{\pgfqpoint{3.206456in}{2.773611in}}%
\pgfpathlineto{\pgfqpoint{3.219384in}{2.760297in}}%
\pgfpathlineto{\pgfqpoint{3.227109in}{2.771480in}}%
\pgfpathlineto{\pgfqpoint{3.234827in}{2.782774in}}%
\pgfpathlineto{\pgfqpoint{3.242538in}{2.794179in}}%
\pgfpathlineto{\pgfqpoint{3.250243in}{2.805698in}}%
\pgfpathlineto{\pgfqpoint{3.237326in}{2.819122in}}%
\pgfpathlineto{\pgfqpoint{3.224406in}{2.832795in}}%
\pgfpathlineto{\pgfqpoint{3.211481in}{2.846719in}}%
\pgfpathlineto{\pgfqpoint{3.198553in}{2.860897in}}%
\pgfpathlineto{\pgfqpoint{3.190837in}{2.849258in}}%
\pgfpathlineto{\pgfqpoint{3.183113in}{2.837740in}}%
\pgfpathlineto{\pgfqpoint{3.175384in}{2.826340in}}%
\pgfpathlineto{\pgfqpoint{3.167647in}{2.815057in}}%
\pgfpathclose%
\pgfusepath{fill}%
\end{pgfscope}%
\begin{pgfscope}%
\pgfpathrectangle{\pgfqpoint{1.254980in}{0.150000in}}{\pgfqpoint{5.490039in}{5.490039in}}%
\pgfusepath{clip}%
\pgfsetbuttcap%
\pgfsetroundjoin%
\definecolor{currentfill}{rgb}{0.253935,0.265254,0.529983}%
\pgfsetfillcolor{currentfill}%
\pgfsetfillopacity{0.700000}%
\pgfsetlinewidth{0.000000pt}%
\definecolor{currentstroke}{rgb}{0.000000,0.000000,0.000000}%
\pgfsetstrokecolor{currentstroke}%
\pgfsetdash{}{0pt}%
\pgfpathmoveto{\pgfqpoint{4.399498in}{2.752900in}}%
\pgfpathlineto{\pgfqpoint{4.412535in}{2.751100in}}%
\pgfpathlineto{\pgfqpoint{4.425580in}{2.749469in}}%
\pgfpathlineto{\pgfqpoint{4.438633in}{2.748006in}}%
\pgfpathlineto{\pgfqpoint{4.451694in}{2.746712in}}%
\pgfpathlineto{\pgfqpoint{4.459065in}{2.757505in}}%
\pgfpathlineto{\pgfqpoint{4.466432in}{2.768406in}}%
\pgfpathlineto{\pgfqpoint{4.473796in}{2.779417in}}%
\pgfpathlineto{\pgfqpoint{4.481156in}{2.790545in}}%
\pgfpathlineto{\pgfqpoint{4.468106in}{2.792226in}}%
\pgfpathlineto{\pgfqpoint{4.455063in}{2.794074in}}%
\pgfpathlineto{\pgfqpoint{4.442029in}{2.796091in}}%
\pgfpathlineto{\pgfqpoint{4.429002in}{2.798277in}}%
\pgfpathlineto{\pgfqpoint{4.421631in}{2.786753in}}%
\pgfpathlineto{\pgfqpoint{4.414257in}{2.775352in}}%
\pgfpathlineto{\pgfqpoint{4.406879in}{2.764069in}}%
\pgfpathlineto{\pgfqpoint{4.399498in}{2.752900in}}%
\pgfpathclose%
\pgfusepath{fill}%
\end{pgfscope}%
\begin{pgfscope}%
\pgfpathrectangle{\pgfqpoint{1.254980in}{0.150000in}}{\pgfqpoint{5.490039in}{5.490039in}}%
\pgfusepath{clip}%
\pgfsetbuttcap%
\pgfsetroundjoin%
\definecolor{currentfill}{rgb}{0.179019,0.433756,0.557430}%
\pgfsetfillcolor{currentfill}%
\pgfsetfillopacity{0.700000}%
\pgfsetlinewidth{0.000000pt}%
\definecolor{currentstroke}{rgb}{0.000000,0.000000,0.000000}%
\pgfsetstrokecolor{currentstroke}%
\pgfsetdash{}{0pt}%
\pgfpathmoveto{\pgfqpoint{5.134729in}{3.140592in}}%
\pgfpathlineto{\pgfqpoint{5.147979in}{3.140101in}}%
\pgfpathlineto{\pgfqpoint{5.161239in}{3.139764in}}%
\pgfpathlineto{\pgfqpoint{5.174510in}{3.139579in}}%
\pgfpathlineto{\pgfqpoint{5.187792in}{3.139547in}}%
\pgfpathlineto{\pgfqpoint{5.194983in}{3.151959in}}%
\pgfpathlineto{\pgfqpoint{5.202175in}{3.164627in}}%
\pgfpathlineto{\pgfqpoint{5.209370in}{3.177559in}}%
\pgfpathlineto{\pgfqpoint{5.216568in}{3.190761in}}%
\pgfpathlineto{\pgfqpoint{5.203305in}{3.191429in}}%
\pgfpathlineto{\pgfqpoint{5.190053in}{3.192249in}}%
\pgfpathlineto{\pgfqpoint{5.176812in}{3.193222in}}%
\pgfpathlineto{\pgfqpoint{5.163581in}{3.194347in}}%
\pgfpathlineto{\pgfqpoint{5.156364in}{3.180500in}}%
\pgfpathlineto{\pgfqpoint{5.149151in}{3.166930in}}%
\pgfpathlineto{\pgfqpoint{5.141939in}{3.153630in}}%
\pgfpathlineto{\pgfqpoint{5.134729in}{3.140592in}}%
\pgfpathclose%
\pgfusepath{fill}%
\end{pgfscope}%
\begin{pgfscope}%
\pgfpathrectangle{\pgfqpoint{1.254980in}{0.150000in}}{\pgfqpoint{5.490039in}{5.490039in}}%
\pgfusepath{clip}%
\pgfsetbuttcap%
\pgfsetroundjoin%
\definecolor{currentfill}{rgb}{0.210503,0.363727,0.552206}%
\pgfsetfillcolor{currentfill}%
\pgfsetfillopacity{0.700000}%
\pgfsetlinewidth{0.000000pt}%
\definecolor{currentstroke}{rgb}{0.000000,0.000000,0.000000}%
\pgfsetstrokecolor{currentstroke}%
\pgfsetdash{}{0pt}%
\pgfpathmoveto{\pgfqpoint{3.011978in}{3.004443in}}%
\pgfpathlineto{\pgfqpoint{3.024983in}{2.987158in}}%
\pgfpathlineto{\pgfqpoint{3.037982in}{2.970155in}}%
\pgfpathlineto{\pgfqpoint{3.050974in}{2.953431in}}%
\pgfpathlineto{\pgfqpoint{3.063959in}{2.936983in}}%
\pgfpathlineto{\pgfqpoint{3.071721in}{2.948465in}}%
\pgfpathlineto{\pgfqpoint{3.079476in}{2.960077in}}%
\pgfpathlineto{\pgfqpoint{3.087223in}{2.971822in}}%
\pgfpathlineto{\pgfqpoint{3.094964in}{2.983702in}}%
\pgfpathlineto{\pgfqpoint{3.081991in}{3.000260in}}%
\pgfpathlineto{\pgfqpoint{3.069012in}{3.017094in}}%
\pgfpathlineto{\pgfqpoint{3.056027in}{3.034208in}}%
\pgfpathlineto{\pgfqpoint{3.043035in}{3.051602in}}%
\pgfpathlineto{\pgfqpoint{3.035281in}{3.039602in}}%
\pgfpathlineto{\pgfqpoint{3.027521in}{3.027743in}}%
\pgfpathlineto{\pgfqpoint{3.019753in}{3.016024in}}%
\pgfpathlineto{\pgfqpoint{3.011978in}{3.004443in}}%
\pgfpathclose%
\pgfusepath{fill}%
\end{pgfscope}%
\begin{pgfscope}%
\pgfpathrectangle{\pgfqpoint{1.254980in}{0.150000in}}{\pgfqpoint{5.490039in}{5.490039in}}%
\pgfusepath{clip}%
\pgfsetbuttcap%
\pgfsetroundjoin%
\definecolor{currentfill}{rgb}{0.278012,0.180367,0.486697}%
\pgfsetfillcolor{currentfill}%
\pgfsetfillopacity{0.700000}%
\pgfsetlinewidth{0.000000pt}%
\definecolor{currentstroke}{rgb}{0.000000,0.000000,0.000000}%
\pgfsetstrokecolor{currentstroke}%
\pgfsetdash{}{0pt}%
\pgfpathmoveto{\pgfqpoint{3.590296in}{2.597517in}}%
\pgfpathlineto{\pgfqpoint{3.603184in}{2.589619in}}%
\pgfpathlineto{\pgfqpoint{3.616075in}{2.581929in}}%
\pgfpathlineto{\pgfqpoint{3.628967in}{2.574444in}}%
\pgfpathlineto{\pgfqpoint{3.641862in}{2.567165in}}%
\pgfpathlineto{\pgfqpoint{3.649469in}{2.578194in}}%
\pgfpathlineto{\pgfqpoint{3.657071in}{2.589300in}}%
\pgfpathlineto{\pgfqpoint{3.664668in}{2.600484in}}%
\pgfpathlineto{\pgfqpoint{3.672260in}{2.611750in}}%
\pgfpathlineto{\pgfqpoint{3.659375in}{2.619194in}}%
\pgfpathlineto{\pgfqpoint{3.646491in}{2.626844in}}%
\pgfpathlineto{\pgfqpoint{3.633610in}{2.634700in}}%
\pgfpathlineto{\pgfqpoint{3.620730in}{2.642763in}}%
\pgfpathlineto{\pgfqpoint{3.613129in}{2.631322in}}%
\pgfpathlineto{\pgfqpoint{3.605523in}{2.619968in}}%
\pgfpathlineto{\pgfqpoint{3.597912in}{2.608701in}}%
\pgfpathlineto{\pgfqpoint{3.590296in}{2.597517in}}%
\pgfpathclose%
\pgfusepath{fill}%
\end{pgfscope}%
\begin{pgfscope}%
\pgfpathrectangle{\pgfqpoint{1.254980in}{0.150000in}}{\pgfqpoint{5.490039in}{5.490039in}}%
\pgfusepath{clip}%
\pgfsetbuttcap%
\pgfsetroundjoin%
\definecolor{currentfill}{rgb}{0.257322,0.256130,0.526563}%
\pgfsetfillcolor{currentfill}%
\pgfsetfillopacity{0.700000}%
\pgfsetlinewidth{0.000000pt}%
\definecolor{currentstroke}{rgb}{0.000000,0.000000,0.000000}%
\pgfsetstrokecolor{currentstroke}%
\pgfsetdash{}{0pt}%
\pgfpathmoveto{\pgfqpoint{3.219384in}{2.760297in}}%
\pgfpathlineto{\pgfqpoint{3.232310in}{2.747230in}}%
\pgfpathlineto{\pgfqpoint{3.245233in}{2.734409in}}%
\pgfpathlineto{\pgfqpoint{3.258152in}{2.721830in}}%
\pgfpathlineto{\pgfqpoint{3.271069in}{2.709493in}}%
\pgfpathlineto{\pgfqpoint{3.278782in}{2.720577in}}%
\pgfpathlineto{\pgfqpoint{3.286488in}{2.731764in}}%
\pgfpathlineto{\pgfqpoint{3.294189in}{2.743056in}}%
\pgfpathlineto{\pgfqpoint{3.301883in}{2.754455in}}%
\pgfpathlineto{\pgfqpoint{3.288977in}{2.766901in}}%
\pgfpathlineto{\pgfqpoint{3.276069in}{2.779590in}}%
\pgfpathlineto{\pgfqpoint{3.263158in}{2.792521in}}%
\pgfpathlineto{\pgfqpoint{3.250243in}{2.805698in}}%
\pgfpathlineto{\pgfqpoint{3.242538in}{2.794179in}}%
\pgfpathlineto{\pgfqpoint{3.234827in}{2.782774in}}%
\pgfpathlineto{\pgfqpoint{3.227109in}{2.771480in}}%
\pgfpathlineto{\pgfqpoint{3.219384in}{2.760297in}}%
\pgfpathclose%
\pgfusepath{fill}%
\end{pgfscope}%
\begin{pgfscope}%
\pgfpathrectangle{\pgfqpoint{1.254980in}{0.150000in}}{\pgfqpoint{5.490039in}{5.490039in}}%
\pgfusepath{clip}%
\pgfsetbuttcap%
\pgfsetroundjoin%
\definecolor{currentfill}{rgb}{0.260571,0.246922,0.522828}%
\pgfsetfillcolor{currentfill}%
\pgfsetfillopacity{0.700000}%
\pgfsetlinewidth{0.000000pt}%
\definecolor{currentstroke}{rgb}{0.000000,0.000000,0.000000}%
\pgfsetstrokecolor{currentstroke}%
\pgfsetdash{}{0pt}%
\pgfpathmoveto{\pgfqpoint{4.317828in}{2.716742in}}%
\pgfpathlineto{\pgfqpoint{4.330844in}{2.714618in}}%
\pgfpathlineto{\pgfqpoint{4.343868in}{2.712665in}}%
\pgfpathlineto{\pgfqpoint{4.356899in}{2.710884in}}%
\pgfpathlineto{\pgfqpoint{4.369938in}{2.709273in}}%
\pgfpathlineto{\pgfqpoint{4.377334in}{2.720031in}}%
\pgfpathlineto{\pgfqpoint{4.384726in}{2.730885in}}%
\pgfpathlineto{\pgfqpoint{4.392114in}{2.741840in}}%
\pgfpathlineto{\pgfqpoint{4.399498in}{2.752900in}}%
\pgfpathlineto{\pgfqpoint{4.386469in}{2.754870in}}%
\pgfpathlineto{\pgfqpoint{4.373447in}{2.757010in}}%
\pgfpathlineto{\pgfqpoint{4.360433in}{2.759321in}}%
\pgfpathlineto{\pgfqpoint{4.347426in}{2.761803in}}%
\pgfpathlineto{\pgfqpoint{4.340032in}{2.750375in}}%
\pgfpathlineto{\pgfqpoint{4.332634in}{2.739058in}}%
\pgfpathlineto{\pgfqpoint{4.325233in}{2.727848in}}%
\pgfpathlineto{\pgfqpoint{4.317828in}{2.716742in}}%
\pgfpathclose%
\pgfusepath{fill}%
\end{pgfscope}%
\begin{pgfscope}%
\pgfpathrectangle{\pgfqpoint{1.254980in}{0.150000in}}{\pgfqpoint{5.490039in}{5.490039in}}%
\pgfusepath{clip}%
\pgfsetbuttcap%
\pgfsetroundjoin%
\definecolor{currentfill}{rgb}{0.276194,0.190074,0.493001}%
\pgfsetfillcolor{currentfill}%
\pgfsetfillopacity{0.700000}%
\pgfsetlinewidth{0.000000pt}%
\definecolor{currentstroke}{rgb}{0.000000,0.000000,0.000000}%
\pgfsetstrokecolor{currentstroke}%
\pgfsetdash{}{0pt}%
\pgfpathmoveto{\pgfqpoint{3.939130in}{2.606301in}}%
\pgfpathlineto{\pgfqpoint{3.952059in}{2.601759in}}%
\pgfpathlineto{\pgfqpoint{3.964992in}{2.597402in}}%
\pgfpathlineto{\pgfqpoint{3.977930in}{2.593231in}}%
\pgfpathlineto{\pgfqpoint{3.990873in}{2.589244in}}%
\pgfpathlineto{\pgfqpoint{3.998380in}{2.600165in}}%
\pgfpathlineto{\pgfqpoint{4.005883in}{2.611158in}}%
\pgfpathlineto{\pgfqpoint{4.013381in}{2.622228in}}%
\pgfpathlineto{\pgfqpoint{4.020875in}{2.633377in}}%
\pgfpathlineto{\pgfqpoint{4.007940in}{2.637612in}}%
\pgfpathlineto{\pgfqpoint{3.995010in}{2.642031in}}%
\pgfpathlineto{\pgfqpoint{3.982085in}{2.646635in}}%
\pgfpathlineto{\pgfqpoint{3.969165in}{2.651426in}}%
\pgfpathlineto{\pgfqpoint{3.961663in}{2.640019in}}%
\pgfpathlineto{\pgfqpoint{3.954157in}{2.628698in}}%
\pgfpathlineto{\pgfqpoint{3.946646in}{2.617460in}}%
\pgfpathlineto{\pgfqpoint{3.939130in}{2.606301in}}%
\pgfpathclose%
\pgfusepath{fill}%
\end{pgfscope}%
\begin{pgfscope}%
\pgfpathrectangle{\pgfqpoint{1.254980in}{0.150000in}}{\pgfqpoint{5.490039in}{5.490039in}}%
\pgfusepath{clip}%
\pgfsetbuttcap%
\pgfsetroundjoin%
\definecolor{currentfill}{rgb}{0.278012,0.180367,0.486697}%
\pgfsetfillcolor{currentfill}%
\pgfsetfillopacity{0.700000}%
\pgfsetlinewidth{0.000000pt}%
\definecolor{currentstroke}{rgb}{0.000000,0.000000,0.000000}%
\pgfsetstrokecolor{currentstroke}%
\pgfsetdash{}{0pt}%
\pgfpathmoveto{\pgfqpoint{3.723826in}{2.583995in}}%
\pgfpathlineto{\pgfqpoint{3.736724in}{2.577557in}}%
\pgfpathlineto{\pgfqpoint{3.749625in}{2.571317in}}%
\pgfpathlineto{\pgfqpoint{3.762530in}{2.565274in}}%
\pgfpathlineto{\pgfqpoint{3.775438in}{2.559427in}}%
\pgfpathlineto{\pgfqpoint{3.783008in}{2.570410in}}%
\pgfpathlineto{\pgfqpoint{3.790573in}{2.581466in}}%
\pgfpathlineto{\pgfqpoint{3.798133in}{2.592597in}}%
\pgfpathlineto{\pgfqpoint{3.805688in}{2.603804in}}%
\pgfpathlineto{\pgfqpoint{3.792789in}{2.609845in}}%
\pgfpathlineto{\pgfqpoint{3.779893in}{2.616081in}}%
\pgfpathlineto{\pgfqpoint{3.767000in}{2.622514in}}%
\pgfpathlineto{\pgfqpoint{3.754110in}{2.629144in}}%
\pgfpathlineto{\pgfqpoint{3.746546in}{2.617734in}}%
\pgfpathlineto{\pgfqpoint{3.738978in}{2.606407in}}%
\pgfpathlineto{\pgfqpoint{3.731404in}{2.595162in}}%
\pgfpathlineto{\pgfqpoint{3.723826in}{2.583995in}}%
\pgfpathclose%
\pgfusepath{fill}%
\end{pgfscope}%
\begin{pgfscope}%
\pgfpathrectangle{\pgfqpoint{1.254980in}{0.150000in}}{\pgfqpoint{5.490039in}{5.490039in}}%
\pgfusepath{clip}%
\pgfsetbuttcap%
\pgfsetroundjoin%
\definecolor{currentfill}{rgb}{0.275191,0.194905,0.496005}%
\pgfsetfillcolor{currentfill}%
\pgfsetfillopacity{0.700000}%
\pgfsetlinewidth{0.000000pt}%
\definecolor{currentstroke}{rgb}{0.000000,0.000000,0.000000}%
\pgfsetstrokecolor{currentstroke}%
\pgfsetdash{}{0pt}%
\pgfpathmoveto{\pgfqpoint{3.456630in}{2.623277in}}%
\pgfpathlineto{\pgfqpoint{3.469522in}{2.613810in}}%
\pgfpathlineto{\pgfqpoint{3.482413in}{2.604562in}}%
\pgfpathlineto{\pgfqpoint{3.495305in}{2.595530in}}%
\pgfpathlineto{\pgfqpoint{3.508198in}{2.586714in}}%
\pgfpathlineto{\pgfqpoint{3.515845in}{2.597712in}}%
\pgfpathlineto{\pgfqpoint{3.523487in}{2.608792in}}%
\pgfpathlineto{\pgfqpoint{3.531123in}{2.619956in}}%
\pgfpathlineto{\pgfqpoint{3.538754in}{2.631207in}}%
\pgfpathlineto{\pgfqpoint{3.525871in}{2.640161in}}%
\pgfpathlineto{\pgfqpoint{3.512989in}{2.649330in}}%
\pgfpathlineto{\pgfqpoint{3.500107in}{2.658716in}}%
\pgfpathlineto{\pgfqpoint{3.487225in}{2.668320in}}%
\pgfpathlineto{\pgfqpoint{3.479585in}{2.656921in}}%
\pgfpathlineto{\pgfqpoint{3.471939in}{2.645616in}}%
\pgfpathlineto{\pgfqpoint{3.464287in}{2.634402in}}%
\pgfpathlineto{\pgfqpoint{3.456630in}{2.623277in}}%
\pgfpathclose%
\pgfusepath{fill}%
\end{pgfscope}%
\begin{pgfscope}%
\pgfpathrectangle{\pgfqpoint{1.254980in}{0.150000in}}{\pgfqpoint{5.490039in}{5.490039in}}%
\pgfusepath{clip}%
\pgfsetbuttcap%
\pgfsetroundjoin%
\definecolor{currentfill}{rgb}{0.195860,0.395433,0.555276}%
\pgfsetfillcolor{currentfill}%
\pgfsetfillopacity{0.700000}%
\pgfsetlinewidth{0.000000pt}%
\definecolor{currentstroke}{rgb}{0.000000,0.000000,0.000000}%
\pgfsetstrokecolor{currentstroke}%
\pgfsetdash{}{0pt}%
\pgfpathmoveto{\pgfqpoint{2.959883in}{3.076449in}}%
\pgfpathlineto{\pgfqpoint{2.972918in}{3.058012in}}%
\pgfpathlineto{\pgfqpoint{2.985946in}{3.039867in}}%
\pgfpathlineto{\pgfqpoint{2.998965in}{3.022012in}}%
\pgfpathlineto{\pgfqpoint{3.011978in}{3.004443in}}%
\pgfpathlineto{\pgfqpoint{3.019753in}{3.016024in}}%
\pgfpathlineto{\pgfqpoint{3.027521in}{3.027743in}}%
\pgfpathlineto{\pgfqpoint{3.035281in}{3.039602in}}%
\pgfpathlineto{\pgfqpoint{3.043035in}{3.051602in}}%
\pgfpathlineto{\pgfqpoint{3.030036in}{3.069281in}}%
\pgfpathlineto{\pgfqpoint{3.017029in}{3.087247in}}%
\pgfpathlineto{\pgfqpoint{3.004015in}{3.105502in}}%
\pgfpathlineto{\pgfqpoint{2.990994in}{3.124049in}}%
\pgfpathlineto{\pgfqpoint{2.983227in}{3.111928in}}%
\pgfpathlineto{\pgfqpoint{2.975453in}{3.099956in}}%
\pgfpathlineto{\pgfqpoint{2.967672in}{3.088130in}}%
\pgfpathlineto{\pgfqpoint{2.959883in}{3.076449in}}%
\pgfpathclose%
\pgfusepath{fill}%
\end{pgfscope}%
\begin{pgfscope}%
\pgfpathrectangle{\pgfqpoint{1.254980in}{0.150000in}}{\pgfqpoint{5.490039in}{5.490039in}}%
\pgfusepath{clip}%
\pgfsetbuttcap%
\pgfsetroundjoin%
\definecolor{currentfill}{rgb}{0.171176,0.452530,0.557965}%
\pgfsetfillcolor{currentfill}%
\pgfsetfillopacity{0.700000}%
\pgfsetlinewidth{0.000000pt}%
\definecolor{currentstroke}{rgb}{0.000000,0.000000,0.000000}%
\pgfsetstrokecolor{currentstroke}%
\pgfsetdash{}{0pt}%
\pgfpathmoveto{\pgfqpoint{5.216568in}{3.190761in}}%
\pgfpathlineto{\pgfqpoint{5.229841in}{3.190245in}}%
\pgfpathlineto{\pgfqpoint{5.243125in}{3.189881in}}%
\pgfpathlineto{\pgfqpoint{5.256420in}{3.189669in}}%
\pgfpathlineto{\pgfqpoint{5.269726in}{3.189608in}}%
\pgfpathlineto{\pgfqpoint{5.276906in}{3.202435in}}%
\pgfpathlineto{\pgfqpoint{5.284089in}{3.215541in}}%
\pgfpathlineto{\pgfqpoint{5.291276in}{3.228933in}}%
\pgfpathlineto{\pgfqpoint{5.277985in}{3.229490in}}%
\pgfpathlineto{\pgfqpoint{5.264705in}{3.230197in}}%
\pgfpathlineto{\pgfqpoint{5.251436in}{3.231056in}}%
\pgfpathlineto{\pgfqpoint{5.238178in}{3.232066in}}%
\pgfpathlineto{\pgfqpoint{5.230971in}{3.218007in}}%
\pgfpathlineto{\pgfqpoint{5.223768in}{3.204242in}}%
\pgfpathlineto{\pgfqpoint{5.216568in}{3.190761in}}%
\pgfpathclose%
\pgfusepath{fill}%
\end{pgfscope}%
\begin{pgfscope}%
\pgfpathrectangle{\pgfqpoint{1.254980in}{0.150000in}}{\pgfqpoint{5.490039in}{5.490039in}}%
\pgfusepath{clip}%
\pgfsetbuttcap%
\pgfsetroundjoin%
\definecolor{currentfill}{rgb}{0.263663,0.237631,0.518762}%
\pgfsetfillcolor{currentfill}%
\pgfsetfillopacity{0.700000}%
\pgfsetlinewidth{0.000000pt}%
\definecolor{currentstroke}{rgb}{0.000000,0.000000,0.000000}%
\pgfsetstrokecolor{currentstroke}%
\pgfsetdash{}{0pt}%
\pgfpathmoveto{\pgfqpoint{3.271069in}{2.709493in}}%
\pgfpathlineto{\pgfqpoint{3.283984in}{2.697396in}}%
\pgfpathlineto{\pgfqpoint{3.296897in}{2.685536in}}%
\pgfpathlineto{\pgfqpoint{3.309807in}{2.673912in}}%
\pgfpathlineto{\pgfqpoint{3.322716in}{2.662523in}}%
\pgfpathlineto{\pgfqpoint{3.330417in}{2.673507in}}%
\pgfpathlineto{\pgfqpoint{3.338113in}{2.684588in}}%
\pgfpathlineto{\pgfqpoint{3.345802in}{2.695766in}}%
\pgfpathlineto{\pgfqpoint{3.353485in}{2.707045in}}%
\pgfpathlineto{\pgfqpoint{3.340587in}{2.718544in}}%
\pgfpathlineto{\pgfqpoint{3.327688in}{2.730278in}}%
\pgfpathlineto{\pgfqpoint{3.314786in}{2.742247in}}%
\pgfpathlineto{\pgfqpoint{3.301883in}{2.754455in}}%
\pgfpathlineto{\pgfqpoint{3.294189in}{2.743056in}}%
\pgfpathlineto{\pgfqpoint{3.286488in}{2.731764in}}%
\pgfpathlineto{\pgfqpoint{3.278782in}{2.720577in}}%
\pgfpathlineto{\pgfqpoint{3.271069in}{2.709493in}}%
\pgfpathclose%
\pgfusepath{fill}%
\end{pgfscope}%
\begin{pgfscope}%
\pgfpathrectangle{\pgfqpoint{1.254980in}{0.150000in}}{\pgfqpoint{5.490039in}{5.490039in}}%
\pgfusepath{clip}%
\pgfsetbuttcap%
\pgfsetroundjoin%
\definecolor{currentfill}{rgb}{0.265145,0.232956,0.516599}%
\pgfsetfillcolor{currentfill}%
\pgfsetfillopacity{0.700000}%
\pgfsetlinewidth{0.000000pt}%
\definecolor{currentstroke}{rgb}{0.000000,0.000000,0.000000}%
\pgfsetstrokecolor{currentstroke}%
\pgfsetdash{}{0pt}%
\pgfpathmoveto{\pgfqpoint{4.236138in}{2.682168in}}%
\pgfpathlineto{\pgfqpoint{4.249135in}{2.679682in}}%
\pgfpathlineto{\pgfqpoint{4.262139in}{2.677370in}}%
\pgfpathlineto{\pgfqpoint{4.275150in}{2.675233in}}%
\pgfpathlineto{\pgfqpoint{4.288168in}{2.673267in}}%
\pgfpathlineto{\pgfqpoint{4.295589in}{2.684001in}}%
\pgfpathlineto{\pgfqpoint{4.303006in}{2.694822in}}%
\pgfpathlineto{\pgfqpoint{4.310419in}{2.705735in}}%
\pgfpathlineto{\pgfqpoint{4.317828in}{2.716742in}}%
\pgfpathlineto{\pgfqpoint{4.304819in}{2.719038in}}%
\pgfpathlineto{\pgfqpoint{4.291817in}{2.721507in}}%
\pgfpathlineto{\pgfqpoint{4.278822in}{2.724149in}}%
\pgfpathlineto{\pgfqpoint{4.265834in}{2.726966in}}%
\pgfpathlineto{\pgfqpoint{4.258416in}{2.715617in}}%
\pgfpathlineto{\pgfqpoint{4.250994in}{2.704371in}}%
\pgfpathlineto{\pgfqpoint{4.243568in}{2.693222in}}%
\pgfpathlineto{\pgfqpoint{4.236138in}{2.682168in}}%
\pgfpathclose%
\pgfusepath{fill}%
\end{pgfscope}%
\begin{pgfscope}%
\pgfpathrectangle{\pgfqpoint{1.254980in}{0.150000in}}{\pgfqpoint{5.490039in}{5.490039in}}%
\pgfusepath{clip}%
\pgfsetbuttcap%
\pgfsetroundjoin%
\definecolor{currentfill}{rgb}{0.278012,0.180367,0.486697}%
\pgfsetfillcolor{currentfill}%
\pgfsetfillopacity{0.700000}%
\pgfsetlinewidth{0.000000pt}%
\definecolor{currentstroke}{rgb}{0.000000,0.000000,0.000000}%
\pgfsetstrokecolor{currentstroke}%
\pgfsetdash{}{0pt}%
\pgfpathmoveto{\pgfqpoint{3.857322in}{2.581581in}}%
\pgfpathlineto{\pgfqpoint{3.870241in}{2.576505in}}%
\pgfpathlineto{\pgfqpoint{3.883164in}{2.571618in}}%
\pgfpathlineto{\pgfqpoint{3.896091in}{2.566921in}}%
\pgfpathlineto{\pgfqpoint{3.909023in}{2.562411in}}%
\pgfpathlineto{\pgfqpoint{3.916557in}{2.573278in}}%
\pgfpathlineto{\pgfqpoint{3.924086in}{2.584214in}}%
\pgfpathlineto{\pgfqpoint{3.931610in}{2.595220in}}%
\pgfpathlineto{\pgfqpoint{3.939130in}{2.606301in}}%
\pgfpathlineto{\pgfqpoint{3.926207in}{2.611031in}}%
\pgfpathlineto{\pgfqpoint{3.913288in}{2.615949in}}%
\pgfpathlineto{\pgfqpoint{3.900373in}{2.621056in}}%
\pgfpathlineto{\pgfqpoint{3.887463in}{2.626353in}}%
\pgfpathlineto{\pgfqpoint{3.879935in}{2.615041in}}%
\pgfpathlineto{\pgfqpoint{3.872402in}{2.603811in}}%
\pgfpathlineto{\pgfqpoint{3.864864in}{2.592658in}}%
\pgfpathlineto{\pgfqpoint{3.857322in}{2.581581in}}%
\pgfpathclose%
\pgfusepath{fill}%
\end{pgfscope}%
\begin{pgfscope}%
\pgfpathrectangle{\pgfqpoint{1.254980in}{0.150000in}}{\pgfqpoint{5.490039in}{5.490039in}}%
\pgfusepath{clip}%
\pgfsetbuttcap%
\pgfsetroundjoin%
\definecolor{currentfill}{rgb}{0.269308,0.218818,0.509577}%
\pgfsetfillcolor{currentfill}%
\pgfsetfillopacity{0.700000}%
\pgfsetlinewidth{0.000000pt}%
\definecolor{currentstroke}{rgb}{0.000000,0.000000,0.000000}%
\pgfsetstrokecolor{currentstroke}%
\pgfsetdash{}{0pt}%
\pgfpathmoveto{\pgfqpoint{4.154421in}{2.649296in}}%
\pgfpathlineto{\pgfqpoint{4.167400in}{2.646410in}}%
\pgfpathlineto{\pgfqpoint{4.180386in}{2.643702in}}%
\pgfpathlineto{\pgfqpoint{4.193378in}{2.641170in}}%
\pgfpathlineto{\pgfqpoint{4.206377in}{2.638813in}}%
\pgfpathlineto{\pgfqpoint{4.213823in}{2.649529in}}%
\pgfpathlineto{\pgfqpoint{4.221266in}{2.660325in}}%
\pgfpathlineto{\pgfqpoint{4.228704in}{2.671203in}}%
\pgfpathlineto{\pgfqpoint{4.236138in}{2.682168in}}%
\pgfpathlineto{\pgfqpoint{4.223148in}{2.684828in}}%
\pgfpathlineto{\pgfqpoint{4.210164in}{2.687664in}}%
\pgfpathlineto{\pgfqpoint{4.197187in}{2.690675in}}%
\pgfpathlineto{\pgfqpoint{4.184217in}{2.693864in}}%
\pgfpathlineto{\pgfqpoint{4.176774in}{2.682586in}}%
\pgfpathlineto{\pgfqpoint{4.169327in}{2.671401in}}%
\pgfpathlineto{\pgfqpoint{4.161876in}{2.660305in}}%
\pgfpathlineto{\pgfqpoint{4.154421in}{2.649296in}}%
\pgfpathclose%
\pgfusepath{fill}%
\end{pgfscope}%
\begin{pgfscope}%
\pgfpathrectangle{\pgfqpoint{1.254980in}{0.150000in}}{\pgfqpoint{5.490039in}{5.490039in}}%
\pgfusepath{clip}%
\pgfsetbuttcap%
\pgfsetroundjoin%
\definecolor{currentfill}{rgb}{0.182256,0.426184,0.557120}%
\pgfsetfillcolor{currentfill}%
\pgfsetfillopacity{0.700000}%
\pgfsetlinewidth{0.000000pt}%
\definecolor{currentstroke}{rgb}{0.000000,0.000000,0.000000}%
\pgfsetstrokecolor{currentstroke}%
\pgfsetdash{}{0pt}%
\pgfpathmoveto{\pgfqpoint{2.907658in}{3.153178in}}%
\pgfpathlineto{\pgfqpoint{2.920727in}{3.133543in}}%
\pgfpathlineto{\pgfqpoint{2.933788in}{3.114212in}}%
\pgfpathlineto{\pgfqpoint{2.946840in}{3.095181in}}%
\pgfpathlineto{\pgfqpoint{2.959883in}{3.076449in}}%
\pgfpathlineto{\pgfqpoint{2.967672in}{3.088130in}}%
\pgfpathlineto{\pgfqpoint{2.975453in}{3.099956in}}%
\pgfpathlineto{\pgfqpoint{2.983227in}{3.111928in}}%
\pgfpathlineto{\pgfqpoint{2.990994in}{3.124049in}}%
\pgfpathlineto{\pgfqpoint{2.977964in}{3.142892in}}%
\pgfpathlineto{\pgfqpoint{2.964926in}{3.162033in}}%
\pgfpathlineto{\pgfqpoint{2.951879in}{3.181474in}}%
\pgfpathlineto{\pgfqpoint{2.938824in}{3.201220in}}%
\pgfpathlineto{\pgfqpoint{2.931044in}{3.188977in}}%
\pgfpathlineto{\pgfqpoint{2.923256in}{3.176891in}}%
\pgfpathlineto{\pgfqpoint{2.915461in}{3.164958in}}%
\pgfpathlineto{\pgfqpoint{2.907658in}{3.153178in}}%
\pgfpathclose%
\pgfusepath{fill}%
\end{pgfscope}%
\begin{pgfscope}%
\pgfpathrectangle{\pgfqpoint{1.254980in}{0.150000in}}{\pgfqpoint{5.490039in}{5.490039in}}%
\pgfusepath{clip}%
\pgfsetbuttcap%
\pgfsetroundjoin%
\definecolor{currentfill}{rgb}{0.269308,0.218818,0.509577}%
\pgfsetfillcolor{currentfill}%
\pgfsetfillopacity{0.700000}%
\pgfsetlinewidth{0.000000pt}%
\definecolor{currentstroke}{rgb}{0.000000,0.000000,0.000000}%
\pgfsetstrokecolor{currentstroke}%
\pgfsetdash{}{0pt}%
\pgfpathmoveto{\pgfqpoint{3.322716in}{2.662523in}}%
\pgfpathlineto{\pgfqpoint{3.335623in}{2.651366in}}%
\pgfpathlineto{\pgfqpoint{3.348529in}{2.640439in}}%
\pgfpathlineto{\pgfqpoint{3.361434in}{2.629742in}}%
\pgfpathlineto{\pgfqpoint{3.374338in}{2.619272in}}%
\pgfpathlineto{\pgfqpoint{3.382028in}{2.630157in}}%
\pgfpathlineto{\pgfqpoint{3.389712in}{2.641131in}}%
\pgfpathlineto{\pgfqpoint{3.397391in}{2.652196in}}%
\pgfpathlineto{\pgfqpoint{3.405064in}{2.663354in}}%
\pgfpathlineto{\pgfqpoint{3.392171in}{2.673934in}}%
\pgfpathlineto{\pgfqpoint{3.379277in}{2.684741in}}%
\pgfpathlineto{\pgfqpoint{3.366382in}{2.695778in}}%
\pgfpathlineto{\pgfqpoint{3.353485in}{2.707045in}}%
\pgfpathlineto{\pgfqpoint{3.345802in}{2.695766in}}%
\pgfpathlineto{\pgfqpoint{3.338113in}{2.684588in}}%
\pgfpathlineto{\pgfqpoint{3.330417in}{2.673507in}}%
\pgfpathlineto{\pgfqpoint{3.322716in}{2.662523in}}%
\pgfpathclose%
\pgfusepath{fill}%
\end{pgfscope}%
\begin{pgfscope}%
\pgfpathrectangle{\pgfqpoint{1.254980in}{0.150000in}}{\pgfqpoint{5.490039in}{5.490039in}}%
\pgfusepath{clip}%
\pgfsetbuttcap%
\pgfsetroundjoin%
\definecolor{currentfill}{rgb}{0.278826,0.175490,0.483397}%
\pgfsetfillcolor{currentfill}%
\pgfsetfillopacity{0.700000}%
\pgfsetlinewidth{0.000000pt}%
\definecolor{currentstroke}{rgb}{0.000000,0.000000,0.000000}%
\pgfsetstrokecolor{currentstroke}%
\pgfsetdash{}{0pt}%
\pgfpathmoveto{\pgfqpoint{3.641862in}{2.567165in}}%
\pgfpathlineto{\pgfqpoint{3.654758in}{2.560089in}}%
\pgfpathlineto{\pgfqpoint{3.667657in}{2.553216in}}%
\pgfpathlineto{\pgfqpoint{3.680558in}{2.546544in}}%
\pgfpathlineto{\pgfqpoint{3.693463in}{2.540072in}}%
\pgfpathlineto{\pgfqpoint{3.701061in}{2.550946in}}%
\pgfpathlineto{\pgfqpoint{3.708654in}{2.561890in}}%
\pgfpathlineto{\pgfqpoint{3.716243in}{2.572905in}}%
\pgfpathlineto{\pgfqpoint{3.723826in}{2.583995in}}%
\pgfpathlineto{\pgfqpoint{3.710931in}{2.590633in}}%
\pgfpathlineto{\pgfqpoint{3.698038in}{2.597470in}}%
\pgfpathlineto{\pgfqpoint{3.685148in}{2.604509in}}%
\pgfpathlineto{\pgfqpoint{3.672260in}{2.611750in}}%
\pgfpathlineto{\pgfqpoint{3.664668in}{2.600484in}}%
\pgfpathlineto{\pgfqpoint{3.657071in}{2.589300in}}%
\pgfpathlineto{\pgfqpoint{3.649469in}{2.578194in}}%
\pgfpathlineto{\pgfqpoint{3.641862in}{2.567165in}}%
\pgfpathclose%
\pgfusepath{fill}%
\end{pgfscope}%
\begin{pgfscope}%
\pgfpathrectangle{\pgfqpoint{1.254980in}{0.150000in}}{\pgfqpoint{5.490039in}{5.490039in}}%
\pgfusepath{clip}%
\pgfsetbuttcap%
\pgfsetroundjoin%
\definecolor{currentfill}{rgb}{0.278012,0.180367,0.486697}%
\pgfsetfillcolor{currentfill}%
\pgfsetfillopacity{0.700000}%
\pgfsetlinewidth{0.000000pt}%
\definecolor{currentstroke}{rgb}{0.000000,0.000000,0.000000}%
\pgfsetstrokecolor{currentstroke}%
\pgfsetdash{}{0pt}%
\pgfpathmoveto{\pgfqpoint{3.508198in}{2.586714in}}%
\pgfpathlineto{\pgfqpoint{3.521091in}{2.578112in}}%
\pgfpathlineto{\pgfqpoint{3.533985in}{2.569722in}}%
\pgfpathlineto{\pgfqpoint{3.546881in}{2.561544in}}%
\pgfpathlineto{\pgfqpoint{3.559778in}{2.553575in}}%
\pgfpathlineto{\pgfqpoint{3.567415in}{2.564445in}}%
\pgfpathlineto{\pgfqpoint{3.575047in}{2.575391in}}%
\pgfpathlineto{\pgfqpoint{3.582674in}{2.586414in}}%
\pgfpathlineto{\pgfqpoint{3.590296in}{2.597517in}}%
\pgfpathlineto{\pgfqpoint{3.577408in}{2.605623in}}%
\pgfpathlineto{\pgfqpoint{3.564522in}{2.613939in}}%
\pgfpathlineto{\pgfqpoint{3.551638in}{2.622467in}}%
\pgfpathlineto{\pgfqpoint{3.538754in}{2.631207in}}%
\pgfpathlineto{\pgfqpoint{3.531123in}{2.619956in}}%
\pgfpathlineto{\pgfqpoint{3.523487in}{2.608792in}}%
\pgfpathlineto{\pgfqpoint{3.515845in}{2.597712in}}%
\pgfpathlineto{\pgfqpoint{3.508198in}{2.586714in}}%
\pgfpathclose%
\pgfusepath{fill}%
\end{pgfscope}%
\begin{pgfscope}%
\pgfpathrectangle{\pgfqpoint{1.254980in}{0.150000in}}{\pgfqpoint{5.490039in}{5.490039in}}%
\pgfusepath{clip}%
\pgfsetbuttcap%
\pgfsetroundjoin%
\definecolor{currentfill}{rgb}{0.273006,0.204520,0.501721}%
\pgfsetfillcolor{currentfill}%
\pgfsetfillopacity{0.700000}%
\pgfsetlinewidth{0.000000pt}%
\definecolor{currentstroke}{rgb}{0.000000,0.000000,0.000000}%
\pgfsetstrokecolor{currentstroke}%
\pgfsetdash{}{0pt}%
\pgfpathmoveto{\pgfqpoint{4.072669in}{2.618267in}}%
\pgfpathlineto{\pgfqpoint{4.085632in}{2.614943in}}%
\pgfpathlineto{\pgfqpoint{4.098601in}{2.611799in}}%
\pgfpathlineto{\pgfqpoint{4.111576in}{2.608834in}}%
\pgfpathlineto{\pgfqpoint{4.124558in}{2.606047in}}%
\pgfpathlineto{\pgfqpoint{4.132030in}{2.616747in}}%
\pgfpathlineto{\pgfqpoint{4.139498in}{2.627520in}}%
\pgfpathlineto{\pgfqpoint{4.146962in}{2.638369in}}%
\pgfpathlineto{\pgfqpoint{4.154421in}{2.649296in}}%
\pgfpathlineto{\pgfqpoint{4.141448in}{2.652359in}}%
\pgfpathlineto{\pgfqpoint{4.128482in}{2.655600in}}%
\pgfpathlineto{\pgfqpoint{4.115521in}{2.659020in}}%
\pgfpathlineto{\pgfqpoint{4.102567in}{2.662620in}}%
\pgfpathlineto{\pgfqpoint{4.095099in}{2.651407in}}%
\pgfpathlineto{\pgfqpoint{4.087627in}{2.640279in}}%
\pgfpathlineto{\pgfqpoint{4.080150in}{2.629234in}}%
\pgfpathlineto{\pgfqpoint{4.072669in}{2.618267in}}%
\pgfpathclose%
\pgfusepath{fill}%
\end{pgfscope}%
\begin{pgfscope}%
\pgfpathrectangle{\pgfqpoint{1.254980in}{0.150000in}}{\pgfqpoint{5.490039in}{5.490039in}}%
\pgfusepath{clip}%
\pgfsetbuttcap%
\pgfsetroundjoin%
\definecolor{currentfill}{rgb}{0.220057,0.343307,0.549413}%
\pgfsetfillcolor{currentfill}%
\pgfsetfillopacity{0.700000}%
\pgfsetlinewidth{0.000000pt}%
\definecolor{currentstroke}{rgb}{0.000000,0.000000,0.000000}%
\pgfsetstrokecolor{currentstroke}%
\pgfsetdash{}{0pt}%
\pgfpathmoveto{\pgfqpoint{4.778689in}{2.909233in}}%
\pgfpathlineto{\pgfqpoint{4.791855in}{2.909005in}}%
\pgfpathlineto{\pgfqpoint{4.805031in}{2.908937in}}%
\pgfpathlineto{\pgfqpoint{4.818218in}{2.909028in}}%
\pgfpathlineto{\pgfqpoint{4.831414in}{2.909277in}}%
\pgfpathlineto{\pgfqpoint{4.838678in}{2.919932in}}%
\pgfpathlineto{\pgfqpoint{4.845939in}{2.930741in}}%
\pgfpathlineto{\pgfqpoint{4.853199in}{2.941710in}}%
\pgfpathlineto{\pgfqpoint{4.860458in}{2.952845in}}%
\pgfpathlineto{\pgfqpoint{4.847276in}{2.953093in}}%
\pgfpathlineto{\pgfqpoint{4.834104in}{2.953500in}}%
\pgfpathlineto{\pgfqpoint{4.820942in}{2.954066in}}%
\pgfpathlineto{\pgfqpoint{4.807790in}{2.954792in}}%
\pgfpathlineto{\pgfqpoint{4.800517in}{2.943149in}}%
\pgfpathlineto{\pgfqpoint{4.793243in}{2.931679in}}%
\pgfpathlineto{\pgfqpoint{4.785967in}{2.920376in}}%
\pgfpathlineto{\pgfqpoint{4.778689in}{2.909233in}}%
\pgfpathclose%
\pgfusepath{fill}%
\end{pgfscope}%
\begin{pgfscope}%
\pgfpathrectangle{\pgfqpoint{1.254980in}{0.150000in}}{\pgfqpoint{5.490039in}{5.490039in}}%
\pgfusepath{clip}%
\pgfsetbuttcap%
\pgfsetroundjoin%
\definecolor{currentfill}{rgb}{0.210503,0.363727,0.552206}%
\pgfsetfillcolor{currentfill}%
\pgfsetfillopacity{0.700000}%
\pgfsetlinewidth{0.000000pt}%
\definecolor{currentstroke}{rgb}{0.000000,0.000000,0.000000}%
\pgfsetstrokecolor{currentstroke}%
\pgfsetdash{}{0pt}%
\pgfpathmoveto{\pgfqpoint{4.860458in}{2.952845in}}%
\pgfpathlineto{\pgfqpoint{4.873650in}{2.952754in}}%
\pgfpathlineto{\pgfqpoint{4.886851in}{2.952821in}}%
\pgfpathlineto{\pgfqpoint{4.900064in}{2.953046in}}%
\pgfpathlineto{\pgfqpoint{4.913286in}{2.953428in}}%
\pgfpathlineto{\pgfqpoint{4.920528in}{2.964219in}}%
\pgfpathlineto{\pgfqpoint{4.927768in}{2.975180in}}%
\pgfpathlineto{\pgfqpoint{4.935008in}{2.986318in}}%
\pgfpathlineto{\pgfqpoint{4.942246in}{2.997639in}}%
\pgfpathlineto{\pgfqpoint{4.929039in}{2.997783in}}%
\pgfpathlineto{\pgfqpoint{4.915842in}{2.998084in}}%
\pgfpathlineto{\pgfqpoint{4.902656in}{2.998542in}}%
\pgfpathlineto{\pgfqpoint{4.889479in}{2.999157in}}%
\pgfpathlineto{\pgfqpoint{4.882225in}{2.987301in}}%
\pgfpathlineto{\pgfqpoint{4.874971in}{2.975634in}}%
\pgfpathlineto{\pgfqpoint{4.867715in}{2.964151in}}%
\pgfpathlineto{\pgfqpoint{4.860458in}{2.952845in}}%
\pgfpathclose%
\pgfusepath{fill}%
\end{pgfscope}%
\begin{pgfscope}%
\pgfpathrectangle{\pgfqpoint{1.254980in}{0.150000in}}{\pgfqpoint{5.490039in}{5.490039in}}%
\pgfusepath{clip}%
\pgfsetbuttcap%
\pgfsetroundjoin%
\definecolor{currentfill}{rgb}{0.227802,0.326594,0.546532}%
\pgfsetfillcolor{currentfill}%
\pgfsetfillopacity{0.700000}%
\pgfsetlinewidth{0.000000pt}%
\definecolor{currentstroke}{rgb}{0.000000,0.000000,0.000000}%
\pgfsetstrokecolor{currentstroke}%
\pgfsetdash{}{0pt}%
\pgfpathmoveto{\pgfqpoint{4.696933in}{2.866793in}}%
\pgfpathlineto{\pgfqpoint{4.710075in}{2.866394in}}%
\pgfpathlineto{\pgfqpoint{4.723225in}{2.866155in}}%
\pgfpathlineto{\pgfqpoint{4.736386in}{2.866078in}}%
\pgfpathlineto{\pgfqpoint{4.749556in}{2.866161in}}%
\pgfpathlineto{\pgfqpoint{4.756843in}{2.876715in}}%
\pgfpathlineto{\pgfqpoint{4.764127in}{2.887408in}}%
\pgfpathlineto{\pgfqpoint{4.771409in}{2.898246in}}%
\pgfpathlineto{\pgfqpoint{4.778689in}{2.909233in}}%
\pgfpathlineto{\pgfqpoint{4.765532in}{2.909621in}}%
\pgfpathlineto{\pgfqpoint{4.752385in}{2.910169in}}%
\pgfpathlineto{\pgfqpoint{4.739247in}{2.910877in}}%
\pgfpathlineto{\pgfqpoint{4.726119in}{2.911746in}}%
\pgfpathlineto{\pgfqpoint{4.718826in}{2.900279in}}%
\pgfpathlineto{\pgfqpoint{4.711531in}{2.888968in}}%
\pgfpathlineto{\pgfqpoint{4.704233in}{2.877808in}}%
\pgfpathlineto{\pgfqpoint{4.696933in}{2.866793in}}%
\pgfpathclose%
\pgfusepath{fill}%
\end{pgfscope}%
\begin{pgfscope}%
\pgfpathrectangle{\pgfqpoint{1.254980in}{0.150000in}}{\pgfqpoint{5.490039in}{5.490039in}}%
\pgfusepath{clip}%
\pgfsetbuttcap%
\pgfsetroundjoin%
\definecolor{currentfill}{rgb}{0.279574,0.170599,0.479997}%
\pgfsetfillcolor{currentfill}%
\pgfsetfillopacity{0.700000}%
\pgfsetlinewidth{0.000000pt}%
\definecolor{currentstroke}{rgb}{0.000000,0.000000,0.000000}%
\pgfsetstrokecolor{currentstroke}%
\pgfsetdash{}{0pt}%
\pgfpathmoveto{\pgfqpoint{3.775438in}{2.559427in}}%
\pgfpathlineto{\pgfqpoint{3.788349in}{2.553774in}}%
\pgfpathlineto{\pgfqpoint{3.801264in}{2.548315in}}%
\pgfpathlineto{\pgfqpoint{3.814183in}{2.543049in}}%
\pgfpathlineto{\pgfqpoint{3.827106in}{2.537975in}}%
\pgfpathlineto{\pgfqpoint{3.834667in}{2.548776in}}%
\pgfpathlineto{\pgfqpoint{3.842224in}{2.559643in}}%
\pgfpathlineto{\pgfqpoint{3.849775in}{2.570577in}}%
\pgfpathlineto{\pgfqpoint{3.857322in}{2.581581in}}%
\pgfpathlineto{\pgfqpoint{3.844408in}{2.586848in}}%
\pgfpathlineto{\pgfqpoint{3.831497in}{2.592307in}}%
\pgfpathlineto{\pgfqpoint{3.818591in}{2.597959in}}%
\pgfpathlineto{\pgfqpoint{3.805688in}{2.603804in}}%
\pgfpathlineto{\pgfqpoint{3.798133in}{2.592597in}}%
\pgfpathlineto{\pgfqpoint{3.790573in}{2.581466in}}%
\pgfpathlineto{\pgfqpoint{3.783008in}{2.570410in}}%
\pgfpathlineto{\pgfqpoint{3.775438in}{2.559427in}}%
\pgfpathclose%
\pgfusepath{fill}%
\end{pgfscope}%
\begin{pgfscope}%
\pgfpathrectangle{\pgfqpoint{1.254980in}{0.150000in}}{\pgfqpoint{5.490039in}{5.490039in}}%
\pgfusepath{clip}%
\pgfsetbuttcap%
\pgfsetroundjoin%
\definecolor{currentfill}{rgb}{0.203063,0.379716,0.553925}%
\pgfsetfillcolor{currentfill}%
\pgfsetfillopacity{0.700000}%
\pgfsetlinewidth{0.000000pt}%
\definecolor{currentstroke}{rgb}{0.000000,0.000000,0.000000}%
\pgfsetstrokecolor{currentstroke}%
\pgfsetdash{}{0pt}%
\pgfpathmoveto{\pgfqpoint{4.942246in}{2.997639in}}%
\pgfpathlineto{\pgfqpoint{4.955463in}{2.997651in}}%
\pgfpathlineto{\pgfqpoint{4.968691in}{2.997820in}}%
\pgfpathlineto{\pgfqpoint{4.981929in}{2.998145in}}%
\pgfpathlineto{\pgfqpoint{4.995178in}{2.998626in}}%
\pgfpathlineto{\pgfqpoint{5.002399in}{3.009592in}}%
\pgfpathlineto{\pgfqpoint{5.009620in}{3.020747in}}%
\pgfpathlineto{\pgfqpoint{5.016840in}{3.032097in}}%
\pgfpathlineto{\pgfqpoint{5.024060in}{3.043648in}}%
\pgfpathlineto{\pgfqpoint{5.010828in}{3.043721in}}%
\pgfpathlineto{\pgfqpoint{4.997606in}{3.043950in}}%
\pgfpathlineto{\pgfqpoint{4.984395in}{3.044334in}}%
\pgfpathlineto{\pgfqpoint{4.971194in}{3.044874in}}%
\pgfpathlineto{\pgfqpoint{4.963957in}{3.032760in}}%
\pgfpathlineto{\pgfqpoint{4.956721in}{3.020854in}}%
\pgfpathlineto{\pgfqpoint{4.949484in}{3.009149in}}%
\pgfpathlineto{\pgfqpoint{4.942246in}{2.997639in}}%
\pgfpathclose%
\pgfusepath{fill}%
\end{pgfscope}%
\begin{pgfscope}%
\pgfpathrectangle{\pgfqpoint{1.254980in}{0.150000in}}{\pgfqpoint{5.490039in}{5.490039in}}%
\pgfusepath{clip}%
\pgfsetbuttcap%
\pgfsetroundjoin%
\definecolor{currentfill}{rgb}{0.237441,0.305202,0.541921}%
\pgfsetfillcolor{currentfill}%
\pgfsetfillopacity{0.700000}%
\pgfsetlinewidth{0.000000pt}%
\definecolor{currentstroke}{rgb}{0.000000,0.000000,0.000000}%
\pgfsetstrokecolor{currentstroke}%
\pgfsetdash{}{0pt}%
\pgfpathmoveto{\pgfqpoint{4.615186in}{2.825536in}}%
\pgfpathlineto{\pgfqpoint{4.628303in}{2.824929in}}%
\pgfpathlineto{\pgfqpoint{4.641428in}{2.824486in}}%
\pgfpathlineto{\pgfqpoint{4.654563in}{2.824205in}}%
\pgfpathlineto{\pgfqpoint{4.667708in}{2.824087in}}%
\pgfpathlineto{\pgfqpoint{4.675018in}{2.834571in}}%
\pgfpathlineto{\pgfqpoint{4.682326in}{2.845180in}}%
\pgfpathlineto{\pgfqpoint{4.689631in}{2.855919in}}%
\pgfpathlineto{\pgfqpoint{4.696933in}{2.866793in}}%
\pgfpathlineto{\pgfqpoint{4.683801in}{2.867355in}}%
\pgfpathlineto{\pgfqpoint{4.670679in}{2.868078in}}%
\pgfpathlineto{\pgfqpoint{4.657565in}{2.868963in}}%
\pgfpathlineto{\pgfqpoint{4.644461in}{2.870012in}}%
\pgfpathlineto{\pgfqpoint{4.637146in}{2.858685in}}%
\pgfpathlineto{\pgfqpoint{4.629829in}{2.847501in}}%
\pgfpathlineto{\pgfqpoint{4.622509in}{2.836452in}}%
\pgfpathlineto{\pgfqpoint{4.615186in}{2.825536in}}%
\pgfpathclose%
\pgfusepath{fill}%
\end{pgfscope}%
\begin{pgfscope}%
\pgfpathrectangle{\pgfqpoint{1.254980in}{0.150000in}}{\pgfqpoint{5.490039in}{5.490039in}}%
\pgfusepath{clip}%
\pgfsetbuttcap%
\pgfsetroundjoin%
\definecolor{currentfill}{rgb}{0.274128,0.199721,0.498911}%
\pgfsetfillcolor{currentfill}%
\pgfsetfillopacity{0.700000}%
\pgfsetlinewidth{0.000000pt}%
\definecolor{currentstroke}{rgb}{0.000000,0.000000,0.000000}%
\pgfsetstrokecolor{currentstroke}%
\pgfsetdash{}{0pt}%
\pgfpathmoveto{\pgfqpoint{3.374338in}{2.619272in}}%
\pgfpathlineto{\pgfqpoint{3.387240in}{2.609028in}}%
\pgfpathlineto{\pgfqpoint{3.400143in}{2.599008in}}%
\pgfpathlineto{\pgfqpoint{3.413045in}{2.589211in}}%
\pgfpathlineto{\pgfqpoint{3.425946in}{2.579635in}}%
\pgfpathlineto{\pgfqpoint{3.433626in}{2.590420in}}%
\pgfpathlineto{\pgfqpoint{3.441300in}{2.601288in}}%
\pgfpathlineto{\pgfqpoint{3.448968in}{2.612239in}}%
\pgfpathlineto{\pgfqpoint{3.456630in}{2.623277in}}%
\pgfpathlineto{\pgfqpoint{3.443739in}{2.632963in}}%
\pgfpathlineto{\pgfqpoint{3.430848in}{2.642870in}}%
\pgfpathlineto{\pgfqpoint{3.417956in}{2.653000in}}%
\pgfpathlineto{\pgfqpoint{3.405064in}{2.663354in}}%
\pgfpathlineto{\pgfqpoint{3.397391in}{2.652196in}}%
\pgfpathlineto{\pgfqpoint{3.389712in}{2.641131in}}%
\pgfpathlineto{\pgfqpoint{3.382028in}{2.630157in}}%
\pgfpathlineto{\pgfqpoint{3.374338in}{2.619272in}}%
\pgfpathclose%
\pgfusepath{fill}%
\end{pgfscope}%
\begin{pgfscope}%
\pgfpathrectangle{\pgfqpoint{1.254980in}{0.150000in}}{\pgfqpoint{5.490039in}{5.490039in}}%
\pgfusepath{clip}%
\pgfsetbuttcap%
\pgfsetroundjoin%
\definecolor{currentfill}{rgb}{0.194100,0.399323,0.555565}%
\pgfsetfillcolor{currentfill}%
\pgfsetfillopacity{0.700000}%
\pgfsetlinewidth{0.000000pt}%
\definecolor{currentstroke}{rgb}{0.000000,0.000000,0.000000}%
\pgfsetstrokecolor{currentstroke}%
\pgfsetdash{}{0pt}%
\pgfpathmoveto{\pgfqpoint{5.024060in}{3.043648in}}%
\pgfpathlineto{\pgfqpoint{5.037303in}{3.043730in}}%
\pgfpathlineto{\pgfqpoint{5.050556in}{3.043966in}}%
\pgfpathlineto{\pgfqpoint{5.063820in}{3.044358in}}%
\pgfpathlineto{\pgfqpoint{5.077095in}{3.044903in}}%
\pgfpathlineto{\pgfqpoint{5.084298in}{3.056091in}}%
\pgfpathlineto{\pgfqpoint{5.091500in}{3.067486in}}%
\pgfpathlineto{\pgfqpoint{5.098703in}{3.079095in}}%
\pgfpathlineto{\pgfqpoint{5.105906in}{3.090926in}}%
\pgfpathlineto{\pgfqpoint{5.092649in}{3.090962in}}%
\pgfpathlineto{\pgfqpoint{5.079402in}{3.091152in}}%
\pgfpathlineto{\pgfqpoint{5.066166in}{3.091496in}}%
\pgfpathlineto{\pgfqpoint{5.052941in}{3.091995in}}%
\pgfpathlineto{\pgfqpoint{5.045720in}{3.079573in}}%
\pgfpathlineto{\pgfqpoint{5.038500in}{3.067380in}}%
\pgfpathlineto{\pgfqpoint{5.031280in}{3.055407in}}%
\pgfpathlineto{\pgfqpoint{5.024060in}{3.043648in}}%
\pgfpathclose%
\pgfusepath{fill}%
\end{pgfscope}%
\begin{pgfscope}%
\pgfpathrectangle{\pgfqpoint{1.254980in}{0.150000in}}{\pgfqpoint{5.490039in}{5.490039in}}%
\pgfusepath{clip}%
\pgfsetbuttcap%
\pgfsetroundjoin%
\definecolor{currentfill}{rgb}{0.244972,0.287675,0.537260}%
\pgfsetfillcolor{currentfill}%
\pgfsetfillopacity{0.700000}%
\pgfsetlinewidth{0.000000pt}%
\definecolor{currentstroke}{rgb}{0.000000,0.000000,0.000000}%
\pgfsetstrokecolor{currentstroke}%
\pgfsetdash{}{0pt}%
\pgfpathmoveto{\pgfqpoint{4.533441in}{2.785491in}}%
\pgfpathlineto{\pgfqpoint{4.546534in}{2.784642in}}%
\pgfpathlineto{\pgfqpoint{4.559635in}{2.783959in}}%
\pgfpathlineto{\pgfqpoint{4.572745in}{2.783440in}}%
\pgfpathlineto{\pgfqpoint{4.585865in}{2.783085in}}%
\pgfpathlineto{\pgfqpoint{4.593200in}{2.793525in}}%
\pgfpathlineto{\pgfqpoint{4.600532in}{2.804077in}}%
\pgfpathlineto{\pgfqpoint{4.607860in}{2.814745in}}%
\pgfpathlineto{\pgfqpoint{4.615186in}{2.825536in}}%
\pgfpathlineto{\pgfqpoint{4.602078in}{2.826306in}}%
\pgfpathlineto{\pgfqpoint{4.588980in}{2.827239in}}%
\pgfpathlineto{\pgfqpoint{4.575890in}{2.828337in}}%
\pgfpathlineto{\pgfqpoint{4.562808in}{2.829600in}}%
\pgfpathlineto{\pgfqpoint{4.555471in}{2.818385in}}%
\pgfpathlineto{\pgfqpoint{4.548131in}{2.807299in}}%
\pgfpathlineto{\pgfqpoint{4.540788in}{2.796336in}}%
\pgfpathlineto{\pgfqpoint{4.533441in}{2.785491in}}%
\pgfpathclose%
\pgfusepath{fill}%
\end{pgfscope}%
\begin{pgfscope}%
\pgfpathrectangle{\pgfqpoint{1.254980in}{0.150000in}}{\pgfqpoint{5.490039in}{5.490039in}}%
\pgfusepath{clip}%
\pgfsetbuttcap%
\pgfsetroundjoin%
\definecolor{currentfill}{rgb}{0.276194,0.190074,0.493001}%
\pgfsetfillcolor{currentfill}%
\pgfsetfillopacity{0.700000}%
\pgfsetlinewidth{0.000000pt}%
\definecolor{currentstroke}{rgb}{0.000000,0.000000,0.000000}%
\pgfsetstrokecolor{currentstroke}%
\pgfsetdash{}{0pt}%
\pgfpathmoveto{\pgfqpoint{3.990873in}{2.589244in}}%
\pgfpathlineto{\pgfqpoint{4.003822in}{2.585442in}}%
\pgfpathlineto{\pgfqpoint{4.016776in}{2.581822in}}%
\pgfpathlineto{\pgfqpoint{4.029736in}{2.578384in}}%
\pgfpathlineto{\pgfqpoint{4.042701in}{2.575128in}}%
\pgfpathlineto{\pgfqpoint{4.050200in}{2.585810in}}%
\pgfpathlineto{\pgfqpoint{4.057694in}{2.596559in}}%
\pgfpathlineto{\pgfqpoint{4.065184in}{2.607377in}}%
\pgfpathlineto{\pgfqpoint{4.072669in}{2.618267in}}%
\pgfpathlineto{\pgfqpoint{4.059712in}{2.621772in}}%
\pgfpathlineto{\pgfqpoint{4.046761in}{2.625458in}}%
\pgfpathlineto{\pgfqpoint{4.033815in}{2.629326in}}%
\pgfpathlineto{\pgfqpoint{4.020875in}{2.633377in}}%
\pgfpathlineto{\pgfqpoint{4.013381in}{2.622228in}}%
\pgfpathlineto{\pgfqpoint{4.005883in}{2.611158in}}%
\pgfpathlineto{\pgfqpoint{3.998380in}{2.600165in}}%
\pgfpathlineto{\pgfqpoint{3.990873in}{2.589244in}}%
\pgfpathclose%
\pgfusepath{fill}%
\end{pgfscope}%
\begin{pgfscope}%
\pgfpathrectangle{\pgfqpoint{1.254980in}{0.150000in}}{\pgfqpoint{5.490039in}{5.490039in}}%
\pgfusepath{clip}%
\pgfsetbuttcap%
\pgfsetroundjoin%
\definecolor{currentfill}{rgb}{0.183898,0.422383,0.556944}%
\pgfsetfillcolor{currentfill}%
\pgfsetfillopacity{0.700000}%
\pgfsetlinewidth{0.000000pt}%
\definecolor{currentstroke}{rgb}{0.000000,0.000000,0.000000}%
\pgfsetstrokecolor{currentstroke}%
\pgfsetdash{}{0pt}%
\pgfpathmoveto{\pgfqpoint{5.105906in}{3.090926in}}%
\pgfpathlineto{\pgfqpoint{5.119174in}{3.091043in}}%
\pgfpathlineto{\pgfqpoint{5.132453in}{3.091314in}}%
\pgfpathlineto{\pgfqpoint{5.145743in}{3.091739in}}%
\pgfpathlineto{\pgfqpoint{5.159044in}{3.092316in}}%
\pgfpathlineto{\pgfqpoint{5.166229in}{3.103775in}}%
\pgfpathlineto{\pgfqpoint{5.173415in}{3.115462in}}%
\pgfpathlineto{\pgfqpoint{5.180603in}{3.127383in}}%
\pgfpathlineto{\pgfqpoint{5.187792in}{3.139547in}}%
\pgfpathlineto{\pgfqpoint{5.174510in}{3.139579in}}%
\pgfpathlineto{\pgfqpoint{5.161239in}{3.139764in}}%
\pgfpathlineto{\pgfqpoint{5.147979in}{3.140101in}}%
\pgfpathlineto{\pgfqpoint{5.134729in}{3.140592in}}%
\pgfpathlineto{\pgfqpoint{5.127521in}{3.127810in}}%
\pgfpathlineto{\pgfqpoint{5.120315in}{3.115276in}}%
\pgfpathlineto{\pgfqpoint{5.113110in}{3.102984in}}%
\pgfpathlineto{\pgfqpoint{5.105906in}{3.090926in}}%
\pgfpathclose%
\pgfusepath{fill}%
\end{pgfscope}%
\begin{pgfscope}%
\pgfpathrectangle{\pgfqpoint{1.254980in}{0.150000in}}{\pgfqpoint{5.490039in}{5.490039in}}%
\pgfusepath{clip}%
\pgfsetbuttcap%
\pgfsetroundjoin%
\definecolor{currentfill}{rgb}{0.241237,0.296485,0.539709}%
\pgfsetfillcolor{currentfill}%
\pgfsetfillopacity{0.700000}%
\pgfsetlinewidth{0.000000pt}%
\definecolor{currentstroke}{rgb}{0.000000,0.000000,0.000000}%
\pgfsetstrokecolor{currentstroke}%
\pgfsetdash{}{0pt}%
\pgfpathmoveto{\pgfqpoint{3.084780in}{2.829589in}}%
\pgfpathlineto{\pgfqpoint{3.097751in}{2.814568in}}%
\pgfpathlineto{\pgfqpoint{3.110717in}{2.799809in}}%
\pgfpathlineto{\pgfqpoint{3.123679in}{2.785310in}}%
\pgfpathlineto{\pgfqpoint{3.136636in}{2.771069in}}%
\pgfpathlineto{\pgfqpoint{3.144399in}{2.781897in}}%
\pgfpathlineto{\pgfqpoint{3.152155in}{2.792837in}}%
\pgfpathlineto{\pgfqpoint{3.159904in}{2.803890in}}%
\pgfpathlineto{\pgfqpoint{3.167647in}{2.815057in}}%
\pgfpathlineto{\pgfqpoint{3.154703in}{2.829381in}}%
\pgfpathlineto{\pgfqpoint{3.141755in}{2.843962in}}%
\pgfpathlineto{\pgfqpoint{3.128802in}{2.858802in}}%
\pgfpathlineto{\pgfqpoint{3.115844in}{2.873905in}}%
\pgfpathlineto{\pgfqpoint{3.108088in}{2.862646in}}%
\pgfpathlineto{\pgfqpoint{3.100326in}{2.851507in}}%
\pgfpathlineto{\pgfqpoint{3.092556in}{2.840489in}}%
\pgfpathlineto{\pgfqpoint{3.084780in}{2.829589in}}%
\pgfpathclose%
\pgfusepath{fill}%
\end{pgfscope}%
\begin{pgfscope}%
\pgfpathrectangle{\pgfqpoint{1.254980in}{0.150000in}}{\pgfqpoint{5.490039in}{5.490039in}}%
\pgfusepath{clip}%
\pgfsetbuttcap%
\pgfsetroundjoin%
\definecolor{currentfill}{rgb}{0.252194,0.269783,0.531579}%
\pgfsetfillcolor{currentfill}%
\pgfsetfillopacity{0.700000}%
\pgfsetlinewidth{0.000000pt}%
\definecolor{currentstroke}{rgb}{0.000000,0.000000,0.000000}%
\pgfsetstrokecolor{currentstroke}%
\pgfsetdash{}{0pt}%
\pgfpathmoveto{\pgfqpoint{4.451694in}{2.746712in}}%
\pgfpathlineto{\pgfqpoint{4.464763in}{2.745585in}}%
\pgfpathlineto{\pgfqpoint{4.477841in}{2.744626in}}%
\pgfpathlineto{\pgfqpoint{4.490927in}{2.743833in}}%
\pgfpathlineto{\pgfqpoint{4.504022in}{2.743206in}}%
\pgfpathlineto{\pgfqpoint{4.511382in}{2.753622in}}%
\pgfpathlineto{\pgfqpoint{4.518738in}{2.764139in}}%
\pgfpathlineto{\pgfqpoint{4.526092in}{2.774760in}}%
\pgfpathlineto{\pgfqpoint{4.533441in}{2.785491in}}%
\pgfpathlineto{\pgfqpoint{4.520357in}{2.786505in}}%
\pgfpathlineto{\pgfqpoint{4.507282in}{2.787685in}}%
\pgfpathlineto{\pgfqpoint{4.494215in}{2.789032in}}%
\pgfpathlineto{\pgfqpoint{4.481156in}{2.790545in}}%
\pgfpathlineto{\pgfqpoint{4.473796in}{2.779417in}}%
\pgfpathlineto{\pgfqpoint{4.466432in}{2.768406in}}%
\pgfpathlineto{\pgfqpoint{4.459065in}{2.757505in}}%
\pgfpathlineto{\pgfqpoint{4.451694in}{2.746712in}}%
\pgfpathclose%
\pgfusepath{fill}%
\end{pgfscope}%
\begin{pgfscope}%
\pgfpathrectangle{\pgfqpoint{1.254980in}{0.150000in}}{\pgfqpoint{5.490039in}{5.490039in}}%
\pgfusepath{clip}%
\pgfsetbuttcap%
\pgfsetroundjoin%
\definecolor{currentfill}{rgb}{0.229739,0.322361,0.545706}%
\pgfsetfillcolor{currentfill}%
\pgfsetfillopacity{0.700000}%
\pgfsetlinewidth{0.000000pt}%
\definecolor{currentstroke}{rgb}{0.000000,0.000000,0.000000}%
\pgfsetstrokecolor{currentstroke}%
\pgfsetdash{}{0pt}%
\pgfpathmoveto{\pgfqpoint{3.032841in}{2.892339in}}%
\pgfpathlineto{\pgfqpoint{3.045835in}{2.876247in}}%
\pgfpathlineto{\pgfqpoint{3.058822in}{2.860426in}}%
\pgfpathlineto{\pgfqpoint{3.071804in}{2.844874in}}%
\pgfpathlineto{\pgfqpoint{3.084780in}{2.829589in}}%
\pgfpathlineto{\pgfqpoint{3.092556in}{2.840489in}}%
\pgfpathlineto{\pgfqpoint{3.100326in}{2.851507in}}%
\pgfpathlineto{\pgfqpoint{3.108088in}{2.862646in}}%
\pgfpathlineto{\pgfqpoint{3.115844in}{2.873905in}}%
\pgfpathlineto{\pgfqpoint{3.102881in}{2.889272in}}%
\pgfpathlineto{\pgfqpoint{3.089913in}{2.904906in}}%
\pgfpathlineto{\pgfqpoint{3.076939in}{2.920809in}}%
\pgfpathlineto{\pgfqpoint{3.063959in}{2.936983in}}%
\pgfpathlineto{\pgfqpoint{3.056191in}{2.925631in}}%
\pgfpathlineto{\pgfqpoint{3.048415in}{2.914407in}}%
\pgfpathlineto{\pgfqpoint{3.040632in}{2.903310in}}%
\pgfpathlineto{\pgfqpoint{3.032841in}{2.892339in}}%
\pgfpathclose%
\pgfusepath{fill}%
\end{pgfscope}%
\begin{pgfscope}%
\pgfpathrectangle{\pgfqpoint{1.254980in}{0.150000in}}{\pgfqpoint{5.490039in}{5.490039in}}%
\pgfusepath{clip}%
\pgfsetbuttcap%
\pgfsetroundjoin%
\definecolor{currentfill}{rgb}{0.252194,0.269783,0.531579}%
\pgfsetfillcolor{currentfill}%
\pgfsetfillopacity{0.700000}%
\pgfsetlinewidth{0.000000pt}%
\definecolor{currentstroke}{rgb}{0.000000,0.000000,0.000000}%
\pgfsetstrokecolor{currentstroke}%
\pgfsetdash{}{0pt}%
\pgfpathmoveto{\pgfqpoint{3.136636in}{2.771069in}}%
\pgfpathlineto{\pgfqpoint{3.149588in}{2.757083in}}%
\pgfpathlineto{\pgfqpoint{3.162537in}{2.743350in}}%
\pgfpathlineto{\pgfqpoint{3.175482in}{2.729869in}}%
\pgfpathlineto{\pgfqpoint{3.188423in}{2.716637in}}%
\pgfpathlineto{\pgfqpoint{3.196173in}{2.727394in}}%
\pgfpathlineto{\pgfqpoint{3.203917in}{2.738255in}}%
\pgfpathlineto{\pgfqpoint{3.211654in}{2.749222in}}%
\pgfpathlineto{\pgfqpoint{3.219384in}{2.760297in}}%
\pgfpathlineto{\pgfqpoint{3.206456in}{2.773611in}}%
\pgfpathlineto{\pgfqpoint{3.193523in}{2.787175in}}%
\pgfpathlineto{\pgfqpoint{3.180587in}{2.800989in}}%
\pgfpathlineto{\pgfqpoint{3.167647in}{2.815057in}}%
\pgfpathlineto{\pgfqpoint{3.159904in}{2.803890in}}%
\pgfpathlineto{\pgfqpoint{3.152155in}{2.792837in}}%
\pgfpathlineto{\pgfqpoint{3.144399in}{2.781897in}}%
\pgfpathlineto{\pgfqpoint{3.136636in}{2.771069in}}%
\pgfpathclose%
\pgfusepath{fill}%
\end{pgfscope}%
\begin{pgfscope}%
\pgfpathrectangle{\pgfqpoint{1.254980in}{0.150000in}}{\pgfqpoint{5.490039in}{5.490039in}}%
\pgfusepath{clip}%
\pgfsetbuttcap%
\pgfsetroundjoin%
\definecolor{currentfill}{rgb}{0.175841,0.441290,0.557685}%
\pgfsetfillcolor{currentfill}%
\pgfsetfillopacity{0.700000}%
\pgfsetlinewidth{0.000000pt}%
\definecolor{currentstroke}{rgb}{0.000000,0.000000,0.000000}%
\pgfsetstrokecolor{currentstroke}%
\pgfsetdash{}{0pt}%
\pgfpathmoveto{\pgfqpoint{5.187792in}{3.139547in}}%
\pgfpathlineto{\pgfqpoint{5.201085in}{3.139667in}}%
\pgfpathlineto{\pgfqpoint{5.214389in}{3.139939in}}%
\pgfpathlineto{\pgfqpoint{5.227704in}{3.140364in}}%
\pgfpathlineto{\pgfqpoint{5.241031in}{3.140940in}}%
\pgfpathlineto{\pgfqpoint{5.248202in}{3.152725in}}%
\pgfpathlineto{\pgfqpoint{5.255374in}{3.164760in}}%
\pgfpathlineto{\pgfqpoint{5.262549in}{3.177052in}}%
\pgfpathlineto{\pgfqpoint{5.269726in}{3.189608in}}%
\pgfpathlineto{\pgfqpoint{5.256420in}{3.189669in}}%
\pgfpathlineto{\pgfqpoint{5.243125in}{3.189881in}}%
\pgfpathlineto{\pgfqpoint{5.229841in}{3.190245in}}%
\pgfpathlineto{\pgfqpoint{5.216568in}{3.190761in}}%
\pgfpathlineto{\pgfqpoint{5.209370in}{3.177559in}}%
\pgfpathlineto{\pgfqpoint{5.202175in}{3.164627in}}%
\pgfpathlineto{\pgfqpoint{5.194983in}{3.151959in}}%
\pgfpathlineto{\pgfqpoint{5.187792in}{3.139547in}}%
\pgfpathclose%
\pgfusepath{fill}%
\end{pgfscope}%
\begin{pgfscope}%
\pgfpathrectangle{\pgfqpoint{1.254980in}{0.150000in}}{\pgfqpoint{5.490039in}{5.490039in}}%
\pgfusepath{clip}%
\pgfsetbuttcap%
\pgfsetroundjoin%
\definecolor{currentfill}{rgb}{0.216210,0.351535,0.550627}%
\pgfsetfillcolor{currentfill}%
\pgfsetfillopacity{0.700000}%
\pgfsetlinewidth{0.000000pt}%
\definecolor{currentstroke}{rgb}{0.000000,0.000000,0.000000}%
\pgfsetstrokecolor{currentstroke}%
\pgfsetdash{}{0pt}%
\pgfpathmoveto{\pgfqpoint{2.980804in}{2.959472in}}%
\pgfpathlineto{\pgfqpoint{2.993823in}{2.942269in}}%
\pgfpathlineto{\pgfqpoint{3.006836in}{2.925348in}}%
\pgfpathlineto{\pgfqpoint{3.019842in}{2.908705in}}%
\pgfpathlineto{\pgfqpoint{3.032841in}{2.892339in}}%
\pgfpathlineto{\pgfqpoint{3.040632in}{2.903310in}}%
\pgfpathlineto{\pgfqpoint{3.048415in}{2.914407in}}%
\pgfpathlineto{\pgfqpoint{3.056191in}{2.925631in}}%
\pgfpathlineto{\pgfqpoint{3.063959in}{2.936983in}}%
\pgfpathlineto{\pgfqpoint{3.050974in}{2.953431in}}%
\pgfpathlineto{\pgfqpoint{3.037982in}{2.970155in}}%
\pgfpathlineto{\pgfqpoint{3.024983in}{2.987158in}}%
\pgfpathlineto{\pgfqpoint{3.011978in}{3.004443in}}%
\pgfpathlineto{\pgfqpoint{3.004195in}{2.992999in}}%
\pgfpathlineto{\pgfqpoint{2.996406in}{2.981690in}}%
\pgfpathlineto{\pgfqpoint{2.988609in}{2.970514in}}%
\pgfpathlineto{\pgfqpoint{2.980804in}{2.959472in}}%
\pgfpathclose%
\pgfusepath{fill}%
\end{pgfscope}%
\begin{pgfscope}%
\pgfpathrectangle{\pgfqpoint{1.254980in}{0.150000in}}{\pgfqpoint{5.490039in}{5.490039in}}%
\pgfusepath{clip}%
\pgfsetbuttcap%
\pgfsetroundjoin%
\definecolor{currentfill}{rgb}{0.258965,0.251537,0.524736}%
\pgfsetfillcolor{currentfill}%
\pgfsetfillopacity{0.700000}%
\pgfsetlinewidth{0.000000pt}%
\definecolor{currentstroke}{rgb}{0.000000,0.000000,0.000000}%
\pgfsetstrokecolor{currentstroke}%
\pgfsetdash{}{0pt}%
\pgfpathmoveto{\pgfqpoint{4.369938in}{2.709273in}}%
\pgfpathlineto{\pgfqpoint{4.382985in}{2.707831in}}%
\pgfpathlineto{\pgfqpoint{4.396040in}{2.706560in}}%
\pgfpathlineto{\pgfqpoint{4.409103in}{2.705457in}}%
\pgfpathlineto{\pgfqpoint{4.422174in}{2.704522in}}%
\pgfpathlineto{\pgfqpoint{4.429560in}{2.714931in}}%
\pgfpathlineto{\pgfqpoint{4.436942in}{2.725429in}}%
\pgfpathlineto{\pgfqpoint{4.444320in}{2.736021in}}%
\pgfpathlineto{\pgfqpoint{4.451694in}{2.746712in}}%
\pgfpathlineto{\pgfqpoint{4.438633in}{2.748006in}}%
\pgfpathlineto{\pgfqpoint{4.425580in}{2.749469in}}%
\pgfpathlineto{\pgfqpoint{4.412535in}{2.751100in}}%
\pgfpathlineto{\pgfqpoint{4.399498in}{2.752900in}}%
\pgfpathlineto{\pgfqpoint{4.392114in}{2.741840in}}%
\pgfpathlineto{\pgfqpoint{4.384726in}{2.730885in}}%
\pgfpathlineto{\pgfqpoint{4.377334in}{2.720031in}}%
\pgfpathlineto{\pgfqpoint{4.369938in}{2.709273in}}%
\pgfpathclose%
\pgfusepath{fill}%
\end{pgfscope}%
\begin{pgfscope}%
\pgfpathrectangle{\pgfqpoint{1.254980in}{0.150000in}}{\pgfqpoint{5.490039in}{5.490039in}}%
\pgfusepath{clip}%
\pgfsetbuttcap%
\pgfsetroundjoin%
\definecolor{currentfill}{rgb}{0.279574,0.170599,0.479997}%
\pgfsetfillcolor{currentfill}%
\pgfsetfillopacity{0.700000}%
\pgfsetlinewidth{0.000000pt}%
\definecolor{currentstroke}{rgb}{0.000000,0.000000,0.000000}%
\pgfsetstrokecolor{currentstroke}%
\pgfsetdash{}{0pt}%
\pgfpathmoveto{\pgfqpoint{3.559778in}{2.553575in}}%
\pgfpathlineto{\pgfqpoint{3.572676in}{2.545815in}}%
\pgfpathlineto{\pgfqpoint{3.585576in}{2.538263in}}%
\pgfpathlineto{\pgfqpoint{3.598477in}{2.530917in}}%
\pgfpathlineto{\pgfqpoint{3.611381in}{2.523775in}}%
\pgfpathlineto{\pgfqpoint{3.619009in}{2.534517in}}%
\pgfpathlineto{\pgfqpoint{3.626632in}{2.545328in}}%
\pgfpathlineto{\pgfqpoint{3.634249in}{2.556210in}}%
\pgfpathlineto{\pgfqpoint{3.641862in}{2.567165in}}%
\pgfpathlineto{\pgfqpoint{3.628967in}{2.574444in}}%
\pgfpathlineto{\pgfqpoint{3.616075in}{2.581929in}}%
\pgfpathlineto{\pgfqpoint{3.603184in}{2.589619in}}%
\pgfpathlineto{\pgfqpoint{3.590296in}{2.597517in}}%
\pgfpathlineto{\pgfqpoint{3.582674in}{2.586414in}}%
\pgfpathlineto{\pgfqpoint{3.575047in}{2.575391in}}%
\pgfpathlineto{\pgfqpoint{3.567415in}{2.564445in}}%
\pgfpathlineto{\pgfqpoint{3.559778in}{2.553575in}}%
\pgfpathclose%
\pgfusepath{fill}%
\end{pgfscope}%
\begin{pgfscope}%
\pgfpathrectangle{\pgfqpoint{1.254980in}{0.150000in}}{\pgfqpoint{5.490039in}{5.490039in}}%
\pgfusepath{clip}%
\pgfsetbuttcap%
\pgfsetroundjoin%
\definecolor{currentfill}{rgb}{0.260571,0.246922,0.522828}%
\pgfsetfillcolor{currentfill}%
\pgfsetfillopacity{0.700000}%
\pgfsetlinewidth{0.000000pt}%
\definecolor{currentstroke}{rgb}{0.000000,0.000000,0.000000}%
\pgfsetstrokecolor{currentstroke}%
\pgfsetdash{}{0pt}%
\pgfpathmoveto{\pgfqpoint{3.188423in}{2.716637in}}%
\pgfpathlineto{\pgfqpoint{3.201361in}{2.703652in}}%
\pgfpathlineto{\pgfqpoint{3.214296in}{2.690912in}}%
\pgfpathlineto{\pgfqpoint{3.227228in}{2.678416in}}%
\pgfpathlineto{\pgfqpoint{3.240157in}{2.666161in}}%
\pgfpathlineto{\pgfqpoint{3.247894in}{2.676847in}}%
\pgfpathlineto{\pgfqpoint{3.255626in}{2.687629in}}%
\pgfpathlineto{\pgfqpoint{3.263351in}{2.698511in}}%
\pgfpathlineto{\pgfqpoint{3.271069in}{2.709493in}}%
\pgfpathlineto{\pgfqpoint{3.258152in}{2.721830in}}%
\pgfpathlineto{\pgfqpoint{3.245233in}{2.734409in}}%
\pgfpathlineto{\pgfqpoint{3.232310in}{2.747230in}}%
\pgfpathlineto{\pgfqpoint{3.219384in}{2.760297in}}%
\pgfpathlineto{\pgfqpoint{3.211654in}{2.749222in}}%
\pgfpathlineto{\pgfqpoint{3.203917in}{2.738255in}}%
\pgfpathlineto{\pgfqpoint{3.196173in}{2.727394in}}%
\pgfpathlineto{\pgfqpoint{3.188423in}{2.716637in}}%
\pgfpathclose%
\pgfusepath{fill}%
\end{pgfscope}%
\begin{pgfscope}%
\pgfpathrectangle{\pgfqpoint{1.254980in}{0.150000in}}{\pgfqpoint{5.490039in}{5.490039in}}%
\pgfusepath{clip}%
\pgfsetbuttcap%
\pgfsetroundjoin%
\definecolor{currentfill}{rgb}{0.278012,0.180367,0.486697}%
\pgfsetfillcolor{currentfill}%
\pgfsetfillopacity{0.700000}%
\pgfsetlinewidth{0.000000pt}%
\definecolor{currentstroke}{rgb}{0.000000,0.000000,0.000000}%
\pgfsetstrokecolor{currentstroke}%
\pgfsetdash{}{0pt}%
\pgfpathmoveto{\pgfqpoint{3.909023in}{2.562411in}}%
\pgfpathlineto{\pgfqpoint{3.921959in}{2.558089in}}%
\pgfpathlineto{\pgfqpoint{3.934901in}{2.553954in}}%
\pgfpathlineto{\pgfqpoint{3.947847in}{2.550003in}}%
\pgfpathlineto{\pgfqpoint{3.960799in}{2.546238in}}%
\pgfpathlineto{\pgfqpoint{3.968325in}{2.556894in}}%
\pgfpathlineto{\pgfqpoint{3.975845in}{2.567612in}}%
\pgfpathlineto{\pgfqpoint{3.983362in}{2.578394in}}%
\pgfpathlineto{\pgfqpoint{3.990873in}{2.589244in}}%
\pgfpathlineto{\pgfqpoint{3.977930in}{2.593231in}}%
\pgfpathlineto{\pgfqpoint{3.964992in}{2.597402in}}%
\pgfpathlineto{\pgfqpoint{3.952059in}{2.601759in}}%
\pgfpathlineto{\pgfqpoint{3.939130in}{2.606301in}}%
\pgfpathlineto{\pgfqpoint{3.931610in}{2.595220in}}%
\pgfpathlineto{\pgfqpoint{3.924086in}{2.584214in}}%
\pgfpathlineto{\pgfqpoint{3.916557in}{2.573278in}}%
\pgfpathlineto{\pgfqpoint{3.909023in}{2.562411in}}%
\pgfpathclose%
\pgfusepath{fill}%
\end{pgfscope}%
\begin{pgfscope}%
\pgfpathrectangle{\pgfqpoint{1.254980in}{0.150000in}}{\pgfqpoint{5.490039in}{5.490039in}}%
\pgfusepath{clip}%
\pgfsetbuttcap%
\pgfsetroundjoin%
\definecolor{currentfill}{rgb}{0.263663,0.237631,0.518762}%
\pgfsetfillcolor{currentfill}%
\pgfsetfillopacity{0.700000}%
\pgfsetlinewidth{0.000000pt}%
\definecolor{currentstroke}{rgb}{0.000000,0.000000,0.000000}%
\pgfsetstrokecolor{currentstroke}%
\pgfsetdash{}{0pt}%
\pgfpathmoveto{\pgfqpoint{4.288168in}{2.673267in}}%
\pgfpathlineto{\pgfqpoint{4.301194in}{2.671475in}}%
\pgfpathlineto{\pgfqpoint{4.314227in}{2.669854in}}%
\pgfpathlineto{\pgfqpoint{4.327268in}{2.668404in}}%
\pgfpathlineto{\pgfqpoint{4.340317in}{2.667125in}}%
\pgfpathlineto{\pgfqpoint{4.347728in}{2.677537in}}%
\pgfpathlineto{\pgfqpoint{4.355136in}{2.688030in}}%
\pgfpathlineto{\pgfqpoint{4.362539in}{2.698607in}}%
\pgfpathlineto{\pgfqpoint{4.369938in}{2.709273in}}%
\pgfpathlineto{\pgfqpoint{4.356899in}{2.710884in}}%
\pgfpathlineto{\pgfqpoint{4.343868in}{2.712665in}}%
\pgfpathlineto{\pgfqpoint{4.330844in}{2.714618in}}%
\pgfpathlineto{\pgfqpoint{4.317828in}{2.716742in}}%
\pgfpathlineto{\pgfqpoint{4.310419in}{2.705735in}}%
\pgfpathlineto{\pgfqpoint{4.303006in}{2.694822in}}%
\pgfpathlineto{\pgfqpoint{4.295589in}{2.684001in}}%
\pgfpathlineto{\pgfqpoint{4.288168in}{2.673267in}}%
\pgfpathclose%
\pgfusepath{fill}%
\end{pgfscope}%
\begin{pgfscope}%
\pgfpathrectangle{\pgfqpoint{1.254980in}{0.150000in}}{\pgfqpoint{5.490039in}{5.490039in}}%
\pgfusepath{clip}%
\pgfsetbuttcap%
\pgfsetroundjoin%
\definecolor{currentfill}{rgb}{0.280255,0.165693,0.476498}%
\pgfsetfillcolor{currentfill}%
\pgfsetfillopacity{0.700000}%
\pgfsetlinewidth{0.000000pt}%
\definecolor{currentstroke}{rgb}{0.000000,0.000000,0.000000}%
\pgfsetstrokecolor{currentstroke}%
\pgfsetdash{}{0pt}%
\pgfpathmoveto{\pgfqpoint{3.693463in}{2.540072in}}%
\pgfpathlineto{\pgfqpoint{3.706370in}{2.533800in}}%
\pgfpathlineto{\pgfqpoint{3.719280in}{2.527725in}}%
\pgfpathlineto{\pgfqpoint{3.732193in}{2.521848in}}%
\pgfpathlineto{\pgfqpoint{3.745109in}{2.516166in}}%
\pgfpathlineto{\pgfqpoint{3.752699in}{2.526884in}}%
\pgfpathlineto{\pgfqpoint{3.760283in}{2.537666in}}%
\pgfpathlineto{\pgfqpoint{3.767863in}{2.548512in}}%
\pgfpathlineto{\pgfqpoint{3.775438in}{2.559427in}}%
\pgfpathlineto{\pgfqpoint{3.762530in}{2.565274in}}%
\pgfpathlineto{\pgfqpoint{3.749625in}{2.571317in}}%
\pgfpathlineto{\pgfqpoint{3.736724in}{2.577557in}}%
\pgfpathlineto{\pgfqpoint{3.723826in}{2.583995in}}%
\pgfpathlineto{\pgfqpoint{3.716243in}{2.572905in}}%
\pgfpathlineto{\pgfqpoint{3.708654in}{2.561890in}}%
\pgfpathlineto{\pgfqpoint{3.701061in}{2.550946in}}%
\pgfpathlineto{\pgfqpoint{3.693463in}{2.540072in}}%
\pgfpathclose%
\pgfusepath{fill}%
\end{pgfscope}%
\begin{pgfscope}%
\pgfpathrectangle{\pgfqpoint{1.254980in}{0.150000in}}{\pgfqpoint{5.490039in}{5.490039in}}%
\pgfusepath{clip}%
\pgfsetbuttcap%
\pgfsetroundjoin%
\definecolor{currentfill}{rgb}{0.201239,0.383670,0.554294}%
\pgfsetfillcolor{currentfill}%
\pgfsetfillopacity{0.700000}%
\pgfsetlinewidth{0.000000pt}%
\definecolor{currentstroke}{rgb}{0.000000,0.000000,0.000000}%
\pgfsetstrokecolor{currentstroke}%
\pgfsetdash{}{0pt}%
\pgfpathmoveto{\pgfqpoint{2.928651in}{3.031151in}}%
\pgfpathlineto{\pgfqpoint{2.941701in}{3.012795in}}%
\pgfpathlineto{\pgfqpoint{2.954743in}{2.994732in}}%
\pgfpathlineto{\pgfqpoint{2.967777in}{2.976959in}}%
\pgfpathlineto{\pgfqpoint{2.980804in}{2.959472in}}%
\pgfpathlineto{\pgfqpoint{2.988609in}{2.970514in}}%
\pgfpathlineto{\pgfqpoint{2.996406in}{2.981690in}}%
\pgfpathlineto{\pgfqpoint{3.004195in}{2.992999in}}%
\pgfpathlineto{\pgfqpoint{3.011978in}{3.004443in}}%
\pgfpathlineto{\pgfqpoint{2.998965in}{3.022012in}}%
\pgfpathlineto{\pgfqpoint{2.985946in}{3.039867in}}%
\pgfpathlineto{\pgfqpoint{2.972918in}{3.058012in}}%
\pgfpathlineto{\pgfqpoint{2.959883in}{3.076449in}}%
\pgfpathlineto{\pgfqpoint{2.952087in}{3.064912in}}%
\pgfpathlineto{\pgfqpoint{2.944282in}{3.053518in}}%
\pgfpathlineto{\pgfqpoint{2.936471in}{3.042264in}}%
\pgfpathlineto{\pgfqpoint{2.928651in}{3.031151in}}%
\pgfpathclose%
\pgfusepath{fill}%
\end{pgfscope}%
\begin{pgfscope}%
\pgfpathrectangle{\pgfqpoint{1.254980in}{0.150000in}}{\pgfqpoint{5.490039in}{5.490039in}}%
\pgfusepath{clip}%
\pgfsetbuttcap%
\pgfsetroundjoin%
\definecolor{currentfill}{rgb}{0.277134,0.185228,0.489898}%
\pgfsetfillcolor{currentfill}%
\pgfsetfillopacity{0.700000}%
\pgfsetlinewidth{0.000000pt}%
\definecolor{currentstroke}{rgb}{0.000000,0.000000,0.000000}%
\pgfsetstrokecolor{currentstroke}%
\pgfsetdash{}{0pt}%
\pgfpathmoveto{\pgfqpoint{3.425946in}{2.579635in}}%
\pgfpathlineto{\pgfqpoint{3.438848in}{2.570279in}}%
\pgfpathlineto{\pgfqpoint{3.451750in}{2.561141in}}%
\pgfpathlineto{\pgfqpoint{3.464652in}{2.552220in}}%
\pgfpathlineto{\pgfqpoint{3.477555in}{2.543514in}}%
\pgfpathlineto{\pgfqpoint{3.485224in}{2.554199in}}%
\pgfpathlineto{\pgfqpoint{3.492887in}{2.564959in}}%
\pgfpathlineto{\pgfqpoint{3.500545in}{2.575797in}}%
\pgfpathlineto{\pgfqpoint{3.508198in}{2.586714in}}%
\pgfpathlineto{\pgfqpoint{3.495305in}{2.595530in}}%
\pgfpathlineto{\pgfqpoint{3.482413in}{2.604562in}}%
\pgfpathlineto{\pgfqpoint{3.469522in}{2.613810in}}%
\pgfpathlineto{\pgfqpoint{3.456630in}{2.623277in}}%
\pgfpathlineto{\pgfqpoint{3.448968in}{2.612239in}}%
\pgfpathlineto{\pgfqpoint{3.441300in}{2.601288in}}%
\pgfpathlineto{\pgfqpoint{3.433626in}{2.590420in}}%
\pgfpathlineto{\pgfqpoint{3.425946in}{2.579635in}}%
\pgfpathclose%
\pgfusepath{fill}%
\end{pgfscope}%
\begin{pgfscope}%
\pgfpathrectangle{\pgfqpoint{1.254980in}{0.150000in}}{\pgfqpoint{5.490039in}{5.490039in}}%
\pgfusepath{clip}%
\pgfsetbuttcap%
\pgfsetroundjoin%
\definecolor{currentfill}{rgb}{0.168126,0.459988,0.558082}%
\pgfsetfillcolor{currentfill}%
\pgfsetfillopacity{0.700000}%
\pgfsetlinewidth{0.000000pt}%
\definecolor{currentstroke}{rgb}{0.000000,0.000000,0.000000}%
\pgfsetstrokecolor{currentstroke}%
\pgfsetdash{}{0pt}%
\pgfpathmoveto{\pgfqpoint{5.269726in}{3.189608in}}%
\pgfpathlineto{\pgfqpoint{5.283044in}{3.189698in}}%
\pgfpathlineto{\pgfqpoint{5.296372in}{3.189938in}}%
\pgfpathlineto{\pgfqpoint{5.309713in}{3.190330in}}%
\pgfpathlineto{\pgfqpoint{5.323065in}{3.190872in}}%
\pgfpathlineto{\pgfqpoint{5.330223in}{3.203045in}}%
\pgfpathlineto{\pgfqpoint{5.337385in}{3.215490in}}%
\pgfpathlineto{\pgfqpoint{5.344550in}{3.228214in}}%
\pgfpathlineto{\pgfqpoint{5.331214in}{3.228169in}}%
\pgfpathlineto{\pgfqpoint{5.317890in}{3.228273in}}%
\pgfpathlineto{\pgfqpoint{5.304577in}{3.228528in}}%
\pgfpathlineto{\pgfqpoint{5.291276in}{3.228933in}}%
\pgfpathlineto{\pgfqpoint{5.284089in}{3.215541in}}%
\pgfpathlineto{\pgfqpoint{5.276906in}{3.202435in}}%
\pgfpathlineto{\pgfqpoint{5.269726in}{3.189608in}}%
\pgfpathclose%
\pgfusepath{fill}%
\end{pgfscope}%
\begin{pgfscope}%
\pgfpathrectangle{\pgfqpoint{1.254980in}{0.150000in}}{\pgfqpoint{5.490039in}{5.490039in}}%
\pgfusepath{clip}%
\pgfsetbuttcap%
\pgfsetroundjoin%
\definecolor{currentfill}{rgb}{0.267968,0.223549,0.512008}%
\pgfsetfillcolor{currentfill}%
\pgfsetfillopacity{0.700000}%
\pgfsetlinewidth{0.000000pt}%
\definecolor{currentstroke}{rgb}{0.000000,0.000000,0.000000}%
\pgfsetstrokecolor{currentstroke}%
\pgfsetdash{}{0pt}%
\pgfpathmoveto{\pgfqpoint{3.240157in}{2.666161in}}%
\pgfpathlineto{\pgfqpoint{3.253084in}{2.654146in}}%
\pgfpathlineto{\pgfqpoint{3.266008in}{2.642369in}}%
\pgfpathlineto{\pgfqpoint{3.278930in}{2.630828in}}%
\pgfpathlineto{\pgfqpoint{3.291851in}{2.619520in}}%
\pgfpathlineto{\pgfqpoint{3.299576in}{2.630133in}}%
\pgfpathlineto{\pgfqpoint{3.307296in}{2.640837in}}%
\pgfpathlineto{\pgfqpoint{3.315009in}{2.651633in}}%
\pgfpathlineto{\pgfqpoint{3.322716in}{2.662523in}}%
\pgfpathlineto{\pgfqpoint{3.309807in}{2.673912in}}%
\pgfpathlineto{\pgfqpoint{3.296897in}{2.685536in}}%
\pgfpathlineto{\pgfqpoint{3.283984in}{2.697396in}}%
\pgfpathlineto{\pgfqpoint{3.271069in}{2.709493in}}%
\pgfpathlineto{\pgfqpoint{3.263351in}{2.698511in}}%
\pgfpathlineto{\pgfqpoint{3.255626in}{2.687629in}}%
\pgfpathlineto{\pgfqpoint{3.247894in}{2.676847in}}%
\pgfpathlineto{\pgfqpoint{3.240157in}{2.666161in}}%
\pgfpathclose%
\pgfusepath{fill}%
\end{pgfscope}%
\begin{pgfscope}%
\pgfpathrectangle{\pgfqpoint{1.254980in}{0.150000in}}{\pgfqpoint{5.490039in}{5.490039in}}%
\pgfusepath{clip}%
\pgfsetbuttcap%
\pgfsetroundjoin%
\definecolor{currentfill}{rgb}{0.269308,0.218818,0.509577}%
\pgfsetfillcolor{currentfill}%
\pgfsetfillopacity{0.700000}%
\pgfsetlinewidth{0.000000pt}%
\definecolor{currentstroke}{rgb}{0.000000,0.000000,0.000000}%
\pgfsetstrokecolor{currentstroke}%
\pgfsetdash{}{0pt}%
\pgfpathmoveto{\pgfqpoint{4.206377in}{2.638813in}}%
\pgfpathlineto{\pgfqpoint{4.219383in}{2.636631in}}%
\pgfpathlineto{\pgfqpoint{4.232396in}{2.634624in}}%
\pgfpathlineto{\pgfqpoint{4.245416in}{2.632790in}}%
\pgfpathlineto{\pgfqpoint{4.258444in}{2.631129in}}%
\pgfpathlineto{\pgfqpoint{4.265881in}{2.641552in}}%
\pgfpathlineto{\pgfqpoint{4.273314in}{2.652046in}}%
\pgfpathlineto{\pgfqpoint{4.280743in}{2.662617in}}%
\pgfpathlineto{\pgfqpoint{4.288168in}{2.673267in}}%
\pgfpathlineto{\pgfqpoint{4.275150in}{2.675233in}}%
\pgfpathlineto{\pgfqpoint{4.262139in}{2.677370in}}%
\pgfpathlineto{\pgfqpoint{4.249135in}{2.679682in}}%
\pgfpathlineto{\pgfqpoint{4.236138in}{2.682168in}}%
\pgfpathlineto{\pgfqpoint{4.228704in}{2.671203in}}%
\pgfpathlineto{\pgfqpoint{4.221266in}{2.660325in}}%
\pgfpathlineto{\pgfqpoint{4.213823in}{2.649529in}}%
\pgfpathlineto{\pgfqpoint{4.206377in}{2.638813in}}%
\pgfpathclose%
\pgfusepath{fill}%
\end{pgfscope}%
\begin{pgfscope}%
\pgfpathrectangle{\pgfqpoint{1.254980in}{0.150000in}}{\pgfqpoint{5.490039in}{5.490039in}}%
\pgfusepath{clip}%
\pgfsetbuttcap%
\pgfsetroundjoin%
\definecolor{currentfill}{rgb}{0.187231,0.414746,0.556547}%
\pgfsetfillcolor{currentfill}%
\pgfsetfillopacity{0.700000}%
\pgfsetlinewidth{0.000000pt}%
\definecolor{currentstroke}{rgb}{0.000000,0.000000,0.000000}%
\pgfsetstrokecolor{currentstroke}%
\pgfsetdash{}{0pt}%
\pgfpathmoveto{\pgfqpoint{2.876365in}{3.107552in}}%
\pgfpathlineto{\pgfqpoint{2.889450in}{3.087999in}}%
\pgfpathlineto{\pgfqpoint{2.902526in}{3.068750in}}%
\pgfpathlineto{\pgfqpoint{2.915593in}{3.049801in}}%
\pgfpathlineto{\pgfqpoint{2.928651in}{3.031151in}}%
\pgfpathlineto{\pgfqpoint{2.936471in}{3.042264in}}%
\pgfpathlineto{\pgfqpoint{2.944282in}{3.053518in}}%
\pgfpathlineto{\pgfqpoint{2.952087in}{3.064912in}}%
\pgfpathlineto{\pgfqpoint{2.959883in}{3.076449in}}%
\pgfpathlineto{\pgfqpoint{2.946840in}{3.095181in}}%
\pgfpathlineto{\pgfqpoint{2.933788in}{3.114212in}}%
\pgfpathlineto{\pgfqpoint{2.920727in}{3.133543in}}%
\pgfpathlineto{\pgfqpoint{2.907658in}{3.153178in}}%
\pgfpathlineto{\pgfqpoint{2.899846in}{3.141548in}}%
\pgfpathlineto{\pgfqpoint{2.892027in}{3.130068in}}%
\pgfpathlineto{\pgfqpoint{2.884200in}{3.118737in}}%
\pgfpathlineto{\pgfqpoint{2.876365in}{3.107552in}}%
\pgfpathclose%
\pgfusepath{fill}%
\end{pgfscope}%
\begin{pgfscope}%
\pgfpathrectangle{\pgfqpoint{1.254980in}{0.150000in}}{\pgfqpoint{5.490039in}{5.490039in}}%
\pgfusepath{clip}%
\pgfsetbuttcap%
\pgfsetroundjoin%
\definecolor{currentfill}{rgb}{0.279574,0.170599,0.479997}%
\pgfsetfillcolor{currentfill}%
\pgfsetfillopacity{0.700000}%
\pgfsetlinewidth{0.000000pt}%
\definecolor{currentstroke}{rgb}{0.000000,0.000000,0.000000}%
\pgfsetstrokecolor{currentstroke}%
\pgfsetdash{}{0pt}%
\pgfpathmoveto{\pgfqpoint{3.827106in}{2.537975in}}%
\pgfpathlineto{\pgfqpoint{3.840033in}{2.533092in}}%
\pgfpathlineto{\pgfqpoint{3.852964in}{2.528399in}}%
\pgfpathlineto{\pgfqpoint{3.865900in}{2.523896in}}%
\pgfpathlineto{\pgfqpoint{3.878840in}{2.519580in}}%
\pgfpathlineto{\pgfqpoint{3.886393in}{2.530197in}}%
\pgfpathlineto{\pgfqpoint{3.893941in}{2.540874in}}%
\pgfpathlineto{\pgfqpoint{3.901484in}{2.551611in}}%
\pgfpathlineto{\pgfqpoint{3.909023in}{2.562411in}}%
\pgfpathlineto{\pgfqpoint{3.896091in}{2.566921in}}%
\pgfpathlineto{\pgfqpoint{3.883164in}{2.571618in}}%
\pgfpathlineto{\pgfqpoint{3.870241in}{2.576505in}}%
\pgfpathlineto{\pgfqpoint{3.857322in}{2.581581in}}%
\pgfpathlineto{\pgfqpoint{3.849775in}{2.570577in}}%
\pgfpathlineto{\pgfqpoint{3.842224in}{2.559643in}}%
\pgfpathlineto{\pgfqpoint{3.834667in}{2.548776in}}%
\pgfpathlineto{\pgfqpoint{3.827106in}{2.537975in}}%
\pgfpathclose%
\pgfusepath{fill}%
\end{pgfscope}%
\begin{pgfscope}%
\pgfpathrectangle{\pgfqpoint{1.254980in}{0.150000in}}{\pgfqpoint{5.490039in}{5.490039in}}%
\pgfusepath{clip}%
\pgfsetbuttcap%
\pgfsetroundjoin%
\definecolor{currentfill}{rgb}{0.273006,0.204520,0.501721}%
\pgfsetfillcolor{currentfill}%
\pgfsetfillopacity{0.700000}%
\pgfsetlinewidth{0.000000pt}%
\definecolor{currentstroke}{rgb}{0.000000,0.000000,0.000000}%
\pgfsetstrokecolor{currentstroke}%
\pgfsetdash{}{0pt}%
\pgfpathmoveto{\pgfqpoint{4.124558in}{2.606047in}}%
\pgfpathlineto{\pgfqpoint{4.137545in}{2.603438in}}%
\pgfpathlineto{\pgfqpoint{4.150540in}{2.601005in}}%
\pgfpathlineto{\pgfqpoint{4.163541in}{2.598750in}}%
\pgfpathlineto{\pgfqpoint{4.176549in}{2.596670in}}%
\pgfpathlineto{\pgfqpoint{4.184012in}{2.607104in}}%
\pgfpathlineto{\pgfqpoint{4.191472in}{2.617604in}}%
\pgfpathlineto{\pgfqpoint{4.198926in}{2.628172in}}%
\pgfpathlineto{\pgfqpoint{4.206377in}{2.638813in}}%
\pgfpathlineto{\pgfqpoint{4.193378in}{2.641170in}}%
\pgfpathlineto{\pgfqpoint{4.180386in}{2.643702in}}%
\pgfpathlineto{\pgfqpoint{4.167400in}{2.646410in}}%
\pgfpathlineto{\pgfqpoint{4.154421in}{2.649296in}}%
\pgfpathlineto{\pgfqpoint{4.146962in}{2.638369in}}%
\pgfpathlineto{\pgfqpoint{4.139498in}{2.627520in}}%
\pgfpathlineto{\pgfqpoint{4.132030in}{2.616747in}}%
\pgfpathlineto{\pgfqpoint{4.124558in}{2.606047in}}%
\pgfpathclose%
\pgfusepath{fill}%
\end{pgfscope}%
\begin{pgfscope}%
\pgfpathrectangle{\pgfqpoint{1.254980in}{0.150000in}}{\pgfqpoint{5.490039in}{5.490039in}}%
\pgfusepath{clip}%
\pgfsetbuttcap%
\pgfsetroundjoin%
\definecolor{currentfill}{rgb}{0.273006,0.204520,0.501721}%
\pgfsetfillcolor{currentfill}%
\pgfsetfillopacity{0.700000}%
\pgfsetlinewidth{0.000000pt}%
\definecolor{currentstroke}{rgb}{0.000000,0.000000,0.000000}%
\pgfsetstrokecolor{currentstroke}%
\pgfsetdash{}{0pt}%
\pgfpathmoveto{\pgfqpoint{3.291851in}{2.619520in}}%
\pgfpathlineto{\pgfqpoint{3.304770in}{2.608446in}}%
\pgfpathlineto{\pgfqpoint{3.317687in}{2.597602in}}%
\pgfpathlineto{\pgfqpoint{3.330603in}{2.586987in}}%
\pgfpathlineto{\pgfqpoint{3.343518in}{2.576600in}}%
\pgfpathlineto{\pgfqpoint{3.351232in}{2.587141in}}%
\pgfpathlineto{\pgfqpoint{3.358940in}{2.597765in}}%
\pgfpathlineto{\pgfqpoint{3.366641in}{2.608476in}}%
\pgfpathlineto{\pgfqpoint{3.374338in}{2.619272in}}%
\pgfpathlineto{\pgfqpoint{3.361434in}{2.629742in}}%
\pgfpathlineto{\pgfqpoint{3.348529in}{2.640439in}}%
\pgfpathlineto{\pgfqpoint{3.335623in}{2.651366in}}%
\pgfpathlineto{\pgfqpoint{3.322716in}{2.662523in}}%
\pgfpathlineto{\pgfqpoint{3.315009in}{2.651633in}}%
\pgfpathlineto{\pgfqpoint{3.307296in}{2.640837in}}%
\pgfpathlineto{\pgfqpoint{3.299576in}{2.630133in}}%
\pgfpathlineto{\pgfqpoint{3.291851in}{2.619520in}}%
\pgfpathclose%
\pgfusepath{fill}%
\end{pgfscope}%
\begin{pgfscope}%
\pgfpathrectangle{\pgfqpoint{1.254980in}{0.150000in}}{\pgfqpoint{5.490039in}{5.490039in}}%
\pgfusepath{clip}%
\pgfsetbuttcap%
\pgfsetroundjoin%
\definecolor{currentfill}{rgb}{0.280868,0.160771,0.472899}%
\pgfsetfillcolor{currentfill}%
\pgfsetfillopacity{0.700000}%
\pgfsetlinewidth{0.000000pt}%
\definecolor{currentstroke}{rgb}{0.000000,0.000000,0.000000}%
\pgfsetstrokecolor{currentstroke}%
\pgfsetdash{}{0pt}%
\pgfpathmoveto{\pgfqpoint{3.611381in}{2.523775in}}%
\pgfpathlineto{\pgfqpoint{3.624286in}{2.516838in}}%
\pgfpathlineto{\pgfqpoint{3.637195in}{2.510102in}}%
\pgfpathlineto{\pgfqpoint{3.650105in}{2.503569in}}%
\pgfpathlineto{\pgfqpoint{3.663018in}{2.497236in}}%
\pgfpathlineto{\pgfqpoint{3.670637in}{2.507850in}}%
\pgfpathlineto{\pgfqpoint{3.678251in}{2.518526in}}%
\pgfpathlineto{\pgfqpoint{3.685859in}{2.529266in}}%
\pgfpathlineto{\pgfqpoint{3.693463in}{2.540072in}}%
\pgfpathlineto{\pgfqpoint{3.680558in}{2.546544in}}%
\pgfpathlineto{\pgfqpoint{3.667657in}{2.553216in}}%
\pgfpathlineto{\pgfqpoint{3.654758in}{2.560089in}}%
\pgfpathlineto{\pgfqpoint{3.641862in}{2.567165in}}%
\pgfpathlineto{\pgfqpoint{3.634249in}{2.556210in}}%
\pgfpathlineto{\pgfqpoint{3.626632in}{2.545328in}}%
\pgfpathlineto{\pgfqpoint{3.619009in}{2.534517in}}%
\pgfpathlineto{\pgfqpoint{3.611381in}{2.523775in}}%
\pgfpathclose%
\pgfusepath{fill}%
\end{pgfscope}%
\begin{pgfscope}%
\pgfpathrectangle{\pgfqpoint{1.254980in}{0.150000in}}{\pgfqpoint{5.490039in}{5.490039in}}%
\pgfusepath{clip}%
\pgfsetbuttcap%
\pgfsetroundjoin%
\definecolor{currentfill}{rgb}{0.279574,0.170599,0.479997}%
\pgfsetfillcolor{currentfill}%
\pgfsetfillopacity{0.700000}%
\pgfsetlinewidth{0.000000pt}%
\definecolor{currentstroke}{rgb}{0.000000,0.000000,0.000000}%
\pgfsetstrokecolor{currentstroke}%
\pgfsetdash{}{0pt}%
\pgfpathmoveto{\pgfqpoint{3.477555in}{2.543514in}}%
\pgfpathlineto{\pgfqpoint{3.490458in}{2.535022in}}%
\pgfpathlineto{\pgfqpoint{3.503362in}{2.526743in}}%
\pgfpathlineto{\pgfqpoint{3.516267in}{2.518675in}}%
\pgfpathlineto{\pgfqpoint{3.529174in}{2.510817in}}%
\pgfpathlineto{\pgfqpoint{3.536833in}{2.521402in}}%
\pgfpathlineto{\pgfqpoint{3.544487in}{2.532055in}}%
\pgfpathlineto{\pgfqpoint{3.552135in}{2.542779in}}%
\pgfpathlineto{\pgfqpoint{3.559778in}{2.553575in}}%
\pgfpathlineto{\pgfqpoint{3.546881in}{2.561544in}}%
\pgfpathlineto{\pgfqpoint{3.533985in}{2.569722in}}%
\pgfpathlineto{\pgfqpoint{3.521091in}{2.578112in}}%
\pgfpathlineto{\pgfqpoint{3.508198in}{2.586714in}}%
\pgfpathlineto{\pgfqpoint{3.500545in}{2.575797in}}%
\pgfpathlineto{\pgfqpoint{3.492887in}{2.564959in}}%
\pgfpathlineto{\pgfqpoint{3.485224in}{2.554199in}}%
\pgfpathlineto{\pgfqpoint{3.477555in}{2.543514in}}%
\pgfpathclose%
\pgfusepath{fill}%
\end{pgfscope}%
\begin{pgfscope}%
\pgfpathrectangle{\pgfqpoint{1.254980in}{0.150000in}}{\pgfqpoint{5.490039in}{5.490039in}}%
\pgfusepath{clip}%
\pgfsetbuttcap%
\pgfsetroundjoin%
\definecolor{currentfill}{rgb}{0.276194,0.190074,0.493001}%
\pgfsetfillcolor{currentfill}%
\pgfsetfillopacity{0.700000}%
\pgfsetlinewidth{0.000000pt}%
\definecolor{currentstroke}{rgb}{0.000000,0.000000,0.000000}%
\pgfsetstrokecolor{currentstroke}%
\pgfsetdash{}{0pt}%
\pgfpathmoveto{\pgfqpoint{4.042701in}{2.575128in}}%
\pgfpathlineto{\pgfqpoint{4.055673in}{2.572053in}}%
\pgfpathlineto{\pgfqpoint{4.068650in}{2.569157in}}%
\pgfpathlineto{\pgfqpoint{4.081634in}{2.566441in}}%
\pgfpathlineto{\pgfqpoint{4.094624in}{2.563903in}}%
\pgfpathlineto{\pgfqpoint{4.102114in}{2.574346in}}%
\pgfpathlineto{\pgfqpoint{4.109600in}{2.584849in}}%
\pgfpathlineto{\pgfqpoint{4.117081in}{2.595415in}}%
\pgfpathlineto{\pgfqpoint{4.124558in}{2.606047in}}%
\pgfpathlineto{\pgfqpoint{4.111576in}{2.608834in}}%
\pgfpathlineto{\pgfqpoint{4.098601in}{2.611799in}}%
\pgfpathlineto{\pgfqpoint{4.085632in}{2.614943in}}%
\pgfpathlineto{\pgfqpoint{4.072669in}{2.618267in}}%
\pgfpathlineto{\pgfqpoint{4.065184in}{2.607377in}}%
\pgfpathlineto{\pgfqpoint{4.057694in}{2.596559in}}%
\pgfpathlineto{\pgfqpoint{4.050200in}{2.585810in}}%
\pgfpathlineto{\pgfqpoint{4.042701in}{2.575128in}}%
\pgfpathclose%
\pgfusepath{fill}%
\end{pgfscope}%
\begin{pgfscope}%
\pgfpathrectangle{\pgfqpoint{1.254980in}{0.150000in}}{\pgfqpoint{5.490039in}{5.490039in}}%
\pgfusepath{clip}%
\pgfsetbuttcap%
\pgfsetroundjoin%
\definecolor{currentfill}{rgb}{0.216210,0.351535,0.550627}%
\pgfsetfillcolor{currentfill}%
\pgfsetfillopacity{0.700000}%
\pgfsetlinewidth{0.000000pt}%
\definecolor{currentstroke}{rgb}{0.000000,0.000000,0.000000}%
\pgfsetstrokecolor{currentstroke}%
\pgfsetdash{}{0pt}%
\pgfpathmoveto{\pgfqpoint{4.831414in}{2.909277in}}%
\pgfpathlineto{\pgfqpoint{4.844620in}{2.909686in}}%
\pgfpathlineto{\pgfqpoint{4.857837in}{2.910252in}}%
\pgfpathlineto{\pgfqpoint{4.871064in}{2.910976in}}%
\pgfpathlineto{\pgfqpoint{4.884302in}{2.911858in}}%
\pgfpathlineto{\pgfqpoint{4.891551in}{2.922023in}}%
\pgfpathlineto{\pgfqpoint{4.898798in}{2.932336in}}%
\pgfpathlineto{\pgfqpoint{4.906043in}{2.942803in}}%
\pgfpathlineto{\pgfqpoint{4.913286in}{2.953428in}}%
\pgfpathlineto{\pgfqpoint{4.900064in}{2.953046in}}%
\pgfpathlineto{\pgfqpoint{4.886851in}{2.952821in}}%
\pgfpathlineto{\pgfqpoint{4.873650in}{2.952754in}}%
\pgfpathlineto{\pgfqpoint{4.860458in}{2.952845in}}%
\pgfpathlineto{\pgfqpoint{4.853199in}{2.941710in}}%
\pgfpathlineto{\pgfqpoint{4.845939in}{2.930741in}}%
\pgfpathlineto{\pgfqpoint{4.838678in}{2.919932in}}%
\pgfpathlineto{\pgfqpoint{4.831414in}{2.909277in}}%
\pgfpathclose%
\pgfusepath{fill}%
\end{pgfscope}%
\begin{pgfscope}%
\pgfpathrectangle{\pgfqpoint{1.254980in}{0.150000in}}{\pgfqpoint{5.490039in}{5.490039in}}%
\pgfusepath{clip}%
\pgfsetbuttcap%
\pgfsetroundjoin%
\definecolor{currentfill}{rgb}{0.223925,0.334994,0.548053}%
\pgfsetfillcolor{currentfill}%
\pgfsetfillopacity{0.700000}%
\pgfsetlinewidth{0.000000pt}%
\definecolor{currentstroke}{rgb}{0.000000,0.000000,0.000000}%
\pgfsetstrokecolor{currentstroke}%
\pgfsetdash{}{0pt}%
\pgfpathmoveto{\pgfqpoint{4.749556in}{2.866161in}}%
\pgfpathlineto{\pgfqpoint{4.762736in}{2.866404in}}%
\pgfpathlineto{\pgfqpoint{4.775926in}{2.866807in}}%
\pgfpathlineto{\pgfqpoint{4.789126in}{2.867369in}}%
\pgfpathlineto{\pgfqpoint{4.802337in}{2.868091in}}%
\pgfpathlineto{\pgfqpoint{4.809610in}{2.878183in}}%
\pgfpathlineto{\pgfqpoint{4.816880in}{2.888409in}}%
\pgfpathlineto{\pgfqpoint{4.824148in}{2.898771in}}%
\pgfpathlineto{\pgfqpoint{4.831414in}{2.909277in}}%
\pgfpathlineto{\pgfqpoint{4.818218in}{2.909028in}}%
\pgfpathlineto{\pgfqpoint{4.805031in}{2.908937in}}%
\pgfpathlineto{\pgfqpoint{4.791855in}{2.909005in}}%
\pgfpathlineto{\pgfqpoint{4.778689in}{2.909233in}}%
\pgfpathlineto{\pgfqpoint{4.771409in}{2.898246in}}%
\pgfpathlineto{\pgfqpoint{4.764127in}{2.887408in}}%
\pgfpathlineto{\pgfqpoint{4.756843in}{2.876715in}}%
\pgfpathlineto{\pgfqpoint{4.749556in}{2.866161in}}%
\pgfpathclose%
\pgfusepath{fill}%
\end{pgfscope}%
\begin{pgfscope}%
\pgfpathrectangle{\pgfqpoint{1.254980in}{0.150000in}}{\pgfqpoint{5.490039in}{5.490039in}}%
\pgfusepath{clip}%
\pgfsetbuttcap%
\pgfsetroundjoin%
\definecolor{currentfill}{rgb}{0.206756,0.371758,0.553117}%
\pgfsetfillcolor{currentfill}%
\pgfsetfillopacity{0.700000}%
\pgfsetlinewidth{0.000000pt}%
\definecolor{currentstroke}{rgb}{0.000000,0.000000,0.000000}%
\pgfsetstrokecolor{currentstroke}%
\pgfsetdash{}{0pt}%
\pgfpathmoveto{\pgfqpoint{4.913286in}{2.953428in}}%
\pgfpathlineto{\pgfqpoint{4.926519in}{2.953967in}}%
\pgfpathlineto{\pgfqpoint{4.939763in}{2.954663in}}%
\pgfpathlineto{\pgfqpoint{4.953017in}{2.955515in}}%
\pgfpathlineto{\pgfqpoint{4.966283in}{2.956523in}}%
\pgfpathlineto{\pgfqpoint{4.973508in}{2.966796in}}%
\pgfpathlineto{\pgfqpoint{4.980733in}{2.977234in}}%
\pgfpathlineto{\pgfqpoint{4.987956in}{2.987842in}}%
\pgfpathlineto{\pgfqpoint{4.995178in}{2.998626in}}%
\pgfpathlineto{\pgfqpoint{4.981929in}{2.998145in}}%
\pgfpathlineto{\pgfqpoint{4.968691in}{2.997820in}}%
\pgfpathlineto{\pgfqpoint{4.955463in}{2.997651in}}%
\pgfpathlineto{\pgfqpoint{4.942246in}{2.997639in}}%
\pgfpathlineto{\pgfqpoint{4.935008in}{2.986318in}}%
\pgfpathlineto{\pgfqpoint{4.927768in}{2.975180in}}%
\pgfpathlineto{\pgfqpoint{4.920528in}{2.964219in}}%
\pgfpathlineto{\pgfqpoint{4.913286in}{2.953428in}}%
\pgfpathclose%
\pgfusepath{fill}%
\end{pgfscope}%
\begin{pgfscope}%
\pgfpathrectangle{\pgfqpoint{1.254980in}{0.150000in}}{\pgfqpoint{5.490039in}{5.490039in}}%
\pgfusepath{clip}%
\pgfsetbuttcap%
\pgfsetroundjoin%
\definecolor{currentfill}{rgb}{0.233603,0.313828,0.543914}%
\pgfsetfillcolor{currentfill}%
\pgfsetfillopacity{0.700000}%
\pgfsetlinewidth{0.000000pt}%
\definecolor{currentstroke}{rgb}{0.000000,0.000000,0.000000}%
\pgfsetstrokecolor{currentstroke}%
\pgfsetdash{}{0pt}%
\pgfpathmoveto{\pgfqpoint{4.667708in}{2.824087in}}%
\pgfpathlineto{\pgfqpoint{4.680862in}{2.824130in}}%
\pgfpathlineto{\pgfqpoint{4.694026in}{2.824335in}}%
\pgfpathlineto{\pgfqpoint{4.707199in}{2.824701in}}%
\pgfpathlineto{\pgfqpoint{4.720383in}{2.825228in}}%
\pgfpathlineto{\pgfqpoint{4.727680in}{2.835279in}}%
\pgfpathlineto{\pgfqpoint{4.734975in}{2.845448in}}%
\pgfpathlineto{\pgfqpoint{4.742267in}{2.855740in}}%
\pgfpathlineto{\pgfqpoint{4.749556in}{2.866161in}}%
\pgfpathlineto{\pgfqpoint{4.736386in}{2.866078in}}%
\pgfpathlineto{\pgfqpoint{4.723225in}{2.866155in}}%
\pgfpathlineto{\pgfqpoint{4.710075in}{2.866394in}}%
\pgfpathlineto{\pgfqpoint{4.696933in}{2.866793in}}%
\pgfpathlineto{\pgfqpoint{4.689631in}{2.855919in}}%
\pgfpathlineto{\pgfqpoint{4.682326in}{2.845180in}}%
\pgfpathlineto{\pgfqpoint{4.675018in}{2.834571in}}%
\pgfpathlineto{\pgfqpoint{4.667708in}{2.824087in}}%
\pgfpathclose%
\pgfusepath{fill}%
\end{pgfscope}%
\begin{pgfscope}%
\pgfpathrectangle{\pgfqpoint{1.254980in}{0.150000in}}{\pgfqpoint{5.490039in}{5.490039in}}%
\pgfusepath{clip}%
\pgfsetbuttcap%
\pgfsetroundjoin%
\definecolor{currentfill}{rgb}{0.280868,0.160771,0.472899}%
\pgfsetfillcolor{currentfill}%
\pgfsetfillopacity{0.700000}%
\pgfsetlinewidth{0.000000pt}%
\definecolor{currentstroke}{rgb}{0.000000,0.000000,0.000000}%
\pgfsetstrokecolor{currentstroke}%
\pgfsetdash{}{0pt}%
\pgfpathmoveto{\pgfqpoint{3.745109in}{2.516166in}}%
\pgfpathlineto{\pgfqpoint{3.758029in}{2.510679in}}%
\pgfpathlineto{\pgfqpoint{3.770953in}{2.505387in}}%
\pgfpathlineto{\pgfqpoint{3.783880in}{2.500287in}}%
\pgfpathlineto{\pgfqpoint{3.796812in}{2.495379in}}%
\pgfpathlineto{\pgfqpoint{3.804393in}{2.505941in}}%
\pgfpathlineto{\pgfqpoint{3.811969in}{2.516560in}}%
\pgfpathlineto{\pgfqpoint{3.819540in}{2.527237in}}%
\pgfpathlineto{\pgfqpoint{3.827106in}{2.537975in}}%
\pgfpathlineto{\pgfqpoint{3.814183in}{2.543049in}}%
\pgfpathlineto{\pgfqpoint{3.801264in}{2.548315in}}%
\pgfpathlineto{\pgfqpoint{3.788349in}{2.553774in}}%
\pgfpathlineto{\pgfqpoint{3.775438in}{2.559427in}}%
\pgfpathlineto{\pgfqpoint{3.767863in}{2.548512in}}%
\pgfpathlineto{\pgfqpoint{3.760283in}{2.537666in}}%
\pgfpathlineto{\pgfqpoint{3.752699in}{2.526884in}}%
\pgfpathlineto{\pgfqpoint{3.745109in}{2.516166in}}%
\pgfpathclose%
\pgfusepath{fill}%
\end{pgfscope}%
\begin{pgfscope}%
\pgfpathrectangle{\pgfqpoint{1.254980in}{0.150000in}}{\pgfqpoint{5.490039in}{5.490039in}}%
\pgfusepath{clip}%
\pgfsetbuttcap%
\pgfsetroundjoin%
\definecolor{currentfill}{rgb}{0.197636,0.391528,0.554969}%
\pgfsetfillcolor{currentfill}%
\pgfsetfillopacity{0.700000}%
\pgfsetlinewidth{0.000000pt}%
\definecolor{currentstroke}{rgb}{0.000000,0.000000,0.000000}%
\pgfsetstrokecolor{currentstroke}%
\pgfsetdash{}{0pt}%
\pgfpathmoveto{\pgfqpoint{4.995178in}{2.998626in}}%
\pgfpathlineto{\pgfqpoint{5.008438in}{2.999262in}}%
\pgfpathlineto{\pgfqpoint{5.021708in}{3.000053in}}%
\pgfpathlineto{\pgfqpoint{5.034990in}{3.000999in}}%
\pgfpathlineto{\pgfqpoint{5.048283in}{3.002100in}}%
\pgfpathlineto{\pgfqpoint{5.055487in}{3.012522in}}%
\pgfpathlineto{\pgfqpoint{5.062690in}{3.023125in}}%
\pgfpathlineto{\pgfqpoint{5.069893in}{3.033917in}}%
\pgfpathlineto{\pgfqpoint{5.077095in}{3.044903in}}%
\pgfpathlineto{\pgfqpoint{5.063820in}{3.044358in}}%
\pgfpathlineto{\pgfqpoint{5.050556in}{3.043966in}}%
\pgfpathlineto{\pgfqpoint{5.037303in}{3.043730in}}%
\pgfpathlineto{\pgfqpoint{5.024060in}{3.043648in}}%
\pgfpathlineto{\pgfqpoint{5.016840in}{3.032097in}}%
\pgfpathlineto{\pgfqpoint{5.009620in}{3.020747in}}%
\pgfpathlineto{\pgfqpoint{5.002399in}{3.009592in}}%
\pgfpathlineto{\pgfqpoint{4.995178in}{2.998626in}}%
\pgfpathclose%
\pgfusepath{fill}%
\end{pgfscope}%
\begin{pgfscope}%
\pgfpathrectangle{\pgfqpoint{1.254980in}{0.150000in}}{\pgfqpoint{5.490039in}{5.490039in}}%
\pgfusepath{clip}%
\pgfsetbuttcap%
\pgfsetroundjoin%
\definecolor{currentfill}{rgb}{0.241237,0.296485,0.539709}%
\pgfsetfillcolor{currentfill}%
\pgfsetfillopacity{0.700000}%
\pgfsetlinewidth{0.000000pt}%
\definecolor{currentstroke}{rgb}{0.000000,0.000000,0.000000}%
\pgfsetstrokecolor{currentstroke}%
\pgfsetdash{}{0pt}%
\pgfpathmoveto{\pgfqpoint{4.585865in}{2.783085in}}%
\pgfpathlineto{\pgfqpoint{4.598993in}{2.782894in}}%
\pgfpathlineto{\pgfqpoint{4.612131in}{2.782866in}}%
\pgfpathlineto{\pgfqpoint{4.625278in}{2.783001in}}%
\pgfpathlineto{\pgfqpoint{4.638435in}{2.783299in}}%
\pgfpathlineto{\pgfqpoint{4.645758in}{2.793333in}}%
\pgfpathlineto{\pgfqpoint{4.653078in}{2.803473in}}%
\pgfpathlineto{\pgfqpoint{4.660394in}{2.813722in}}%
\pgfpathlineto{\pgfqpoint{4.667708in}{2.824087in}}%
\pgfpathlineto{\pgfqpoint{4.654563in}{2.824205in}}%
\pgfpathlineto{\pgfqpoint{4.641428in}{2.824486in}}%
\pgfpathlineto{\pgfqpoint{4.628303in}{2.824929in}}%
\pgfpathlineto{\pgfqpoint{4.615186in}{2.825536in}}%
\pgfpathlineto{\pgfqpoint{4.607860in}{2.814745in}}%
\pgfpathlineto{\pgfqpoint{4.600532in}{2.804077in}}%
\pgfpathlineto{\pgfqpoint{4.593200in}{2.793525in}}%
\pgfpathlineto{\pgfqpoint{4.585865in}{2.783085in}}%
\pgfpathclose%
\pgfusepath{fill}%
\end{pgfscope}%
\begin{pgfscope}%
\pgfpathrectangle{\pgfqpoint{1.254980in}{0.150000in}}{\pgfqpoint{5.490039in}{5.490039in}}%
\pgfusepath{clip}%
\pgfsetbuttcap%
\pgfsetroundjoin%
\definecolor{currentfill}{rgb}{0.277134,0.185228,0.489898}%
\pgfsetfillcolor{currentfill}%
\pgfsetfillopacity{0.700000}%
\pgfsetlinewidth{0.000000pt}%
\definecolor{currentstroke}{rgb}{0.000000,0.000000,0.000000}%
\pgfsetstrokecolor{currentstroke}%
\pgfsetdash{}{0pt}%
\pgfpathmoveto{\pgfqpoint{3.343518in}{2.576600in}}%
\pgfpathlineto{\pgfqpoint{3.356432in}{2.566438in}}%
\pgfpathlineto{\pgfqpoint{3.369345in}{2.556501in}}%
\pgfpathlineto{\pgfqpoint{3.382258in}{2.546787in}}%
\pgfpathlineto{\pgfqpoint{3.395171in}{2.537294in}}%
\pgfpathlineto{\pgfqpoint{3.402873in}{2.547762in}}%
\pgfpathlineto{\pgfqpoint{3.410570in}{2.558308in}}%
\pgfpathlineto{\pgfqpoint{3.418261in}{2.568932in}}%
\pgfpathlineto{\pgfqpoint{3.425946in}{2.579635in}}%
\pgfpathlineto{\pgfqpoint{3.413045in}{2.589211in}}%
\pgfpathlineto{\pgfqpoint{3.400143in}{2.599008in}}%
\pgfpathlineto{\pgfqpoint{3.387240in}{2.609028in}}%
\pgfpathlineto{\pgfqpoint{3.374338in}{2.619272in}}%
\pgfpathlineto{\pgfqpoint{3.366641in}{2.608476in}}%
\pgfpathlineto{\pgfqpoint{3.358940in}{2.597765in}}%
\pgfpathlineto{\pgfqpoint{3.351232in}{2.587141in}}%
\pgfpathlineto{\pgfqpoint{3.343518in}{2.576600in}}%
\pgfpathclose%
\pgfusepath{fill}%
\end{pgfscope}%
\begin{pgfscope}%
\pgfpathrectangle{\pgfqpoint{1.254980in}{0.150000in}}{\pgfqpoint{5.490039in}{5.490039in}}%
\pgfusepath{clip}%
\pgfsetbuttcap%
\pgfsetroundjoin%
\definecolor{currentfill}{rgb}{0.188923,0.410910,0.556326}%
\pgfsetfillcolor{currentfill}%
\pgfsetfillopacity{0.700000}%
\pgfsetlinewidth{0.000000pt}%
\definecolor{currentstroke}{rgb}{0.000000,0.000000,0.000000}%
\pgfsetstrokecolor{currentstroke}%
\pgfsetdash{}{0pt}%
\pgfpathmoveto{\pgfqpoint{5.077095in}{3.044903in}}%
\pgfpathlineto{\pgfqpoint{5.090381in}{3.045603in}}%
\pgfpathlineto{\pgfqpoint{5.103679in}{3.046456in}}%
\pgfpathlineto{\pgfqpoint{5.116987in}{3.047464in}}%
\pgfpathlineto{\pgfqpoint{5.130307in}{3.048624in}}%
\pgfpathlineto{\pgfqpoint{5.137491in}{3.059239in}}%
\pgfpathlineto{\pgfqpoint{5.144675in}{3.070055in}}%
\pgfpathlineto{\pgfqpoint{5.151859in}{3.081078in}}%
\pgfpathlineto{\pgfqpoint{5.159044in}{3.092316in}}%
\pgfpathlineto{\pgfqpoint{5.145743in}{3.091739in}}%
\pgfpathlineto{\pgfqpoint{5.132453in}{3.091314in}}%
\pgfpathlineto{\pgfqpoint{5.119174in}{3.091043in}}%
\pgfpathlineto{\pgfqpoint{5.105906in}{3.090926in}}%
\pgfpathlineto{\pgfqpoint{5.098703in}{3.079095in}}%
\pgfpathlineto{\pgfqpoint{5.091500in}{3.067486in}}%
\pgfpathlineto{\pgfqpoint{5.084298in}{3.056091in}}%
\pgfpathlineto{\pgfqpoint{5.077095in}{3.044903in}}%
\pgfpathclose%
\pgfusepath{fill}%
\end{pgfscope}%
\begin{pgfscope}%
\pgfpathrectangle{\pgfqpoint{1.254980in}{0.150000in}}{\pgfqpoint{5.490039in}{5.490039in}}%
\pgfusepath{clip}%
\pgfsetbuttcap%
\pgfsetroundjoin%
\definecolor{currentfill}{rgb}{0.248629,0.278775,0.534556}%
\pgfsetfillcolor{currentfill}%
\pgfsetfillopacity{0.700000}%
\pgfsetlinewidth{0.000000pt}%
\definecolor{currentstroke}{rgb}{0.000000,0.000000,0.000000}%
\pgfsetstrokecolor{currentstroke}%
\pgfsetdash{}{0pt}%
\pgfpathmoveto{\pgfqpoint{4.504022in}{2.743206in}}%
\pgfpathlineto{\pgfqpoint{4.517125in}{2.742745in}}%
\pgfpathlineto{\pgfqpoint{4.530238in}{2.742449in}}%
\pgfpathlineto{\pgfqpoint{4.543360in}{2.742319in}}%
\pgfpathlineto{\pgfqpoint{4.556490in}{2.742352in}}%
\pgfpathlineto{\pgfqpoint{4.563839in}{2.752391in}}%
\pgfpathlineto{\pgfqpoint{4.571185in}{2.762522in}}%
\pgfpathlineto{\pgfqpoint{4.578526in}{2.772752in}}%
\pgfpathlineto{\pgfqpoint{4.585865in}{2.783085in}}%
\pgfpathlineto{\pgfqpoint{4.572745in}{2.783440in}}%
\pgfpathlineto{\pgfqpoint{4.559635in}{2.783959in}}%
\pgfpathlineto{\pgfqpoint{4.546534in}{2.784642in}}%
\pgfpathlineto{\pgfqpoint{4.533441in}{2.785491in}}%
\pgfpathlineto{\pgfqpoint{4.526092in}{2.774760in}}%
\pgfpathlineto{\pgfqpoint{4.518738in}{2.764139in}}%
\pgfpathlineto{\pgfqpoint{4.511382in}{2.753622in}}%
\pgfpathlineto{\pgfqpoint{4.504022in}{2.743206in}}%
\pgfpathclose%
\pgfusepath{fill}%
\end{pgfscope}%
\begin{pgfscope}%
\pgfpathrectangle{\pgfqpoint{1.254980in}{0.150000in}}{\pgfqpoint{5.490039in}{5.490039in}}%
\pgfusepath{clip}%
\pgfsetbuttcap%
\pgfsetroundjoin%
\definecolor{currentfill}{rgb}{0.278012,0.180367,0.486697}%
\pgfsetfillcolor{currentfill}%
\pgfsetfillopacity{0.700000}%
\pgfsetlinewidth{0.000000pt}%
\definecolor{currentstroke}{rgb}{0.000000,0.000000,0.000000}%
\pgfsetstrokecolor{currentstroke}%
\pgfsetdash{}{0pt}%
\pgfpathmoveto{\pgfqpoint{3.960799in}{2.546238in}}%
\pgfpathlineto{\pgfqpoint{3.973756in}{2.542656in}}%
\pgfpathlineto{\pgfqpoint{3.986719in}{2.539258in}}%
\pgfpathlineto{\pgfqpoint{3.999687in}{2.536042in}}%
\pgfpathlineto{\pgfqpoint{4.012661in}{2.533007in}}%
\pgfpathlineto{\pgfqpoint{4.020178in}{2.543452in}}%
\pgfpathlineto{\pgfqpoint{4.027690in}{2.553952in}}%
\pgfpathlineto{\pgfqpoint{4.035198in}{2.564510in}}%
\pgfpathlineto{\pgfqpoint{4.042701in}{2.575128in}}%
\pgfpathlineto{\pgfqpoint{4.029736in}{2.578384in}}%
\pgfpathlineto{\pgfqpoint{4.016776in}{2.581822in}}%
\pgfpathlineto{\pgfqpoint{4.003822in}{2.585442in}}%
\pgfpathlineto{\pgfqpoint{3.990873in}{2.589244in}}%
\pgfpathlineto{\pgfqpoint{3.983362in}{2.578394in}}%
\pgfpathlineto{\pgfqpoint{3.975845in}{2.567612in}}%
\pgfpathlineto{\pgfqpoint{3.968325in}{2.556894in}}%
\pgfpathlineto{\pgfqpoint{3.960799in}{2.546238in}}%
\pgfpathclose%
\pgfusepath{fill}%
\end{pgfscope}%
\begin{pgfscope}%
\pgfpathrectangle{\pgfqpoint{1.254980in}{0.150000in}}{\pgfqpoint{5.490039in}{5.490039in}}%
\pgfusepath{clip}%
\pgfsetbuttcap%
\pgfsetroundjoin%
\definecolor{currentfill}{rgb}{0.180629,0.429975,0.557282}%
\pgfsetfillcolor{currentfill}%
\pgfsetfillopacity{0.700000}%
\pgfsetlinewidth{0.000000pt}%
\definecolor{currentstroke}{rgb}{0.000000,0.000000,0.000000}%
\pgfsetstrokecolor{currentstroke}%
\pgfsetdash{}{0pt}%
\pgfpathmoveto{\pgfqpoint{5.159044in}{3.092316in}}%
\pgfpathlineto{\pgfqpoint{5.172356in}{3.093046in}}%
\pgfpathlineto{\pgfqpoint{5.185680in}{3.093929in}}%
\pgfpathlineto{\pgfqpoint{5.199015in}{3.094964in}}%
\pgfpathlineto{\pgfqpoint{5.212362in}{3.096151in}}%
\pgfpathlineto{\pgfqpoint{5.219528in}{3.107009in}}%
\pgfpathlineto{\pgfqpoint{5.226694in}{3.118089in}}%
\pgfpathlineto{\pgfqpoint{5.233862in}{3.129396in}}%
\pgfpathlineto{\pgfqpoint{5.241031in}{3.140940in}}%
\pgfpathlineto{\pgfqpoint{5.227704in}{3.140364in}}%
\pgfpathlineto{\pgfqpoint{5.214389in}{3.139939in}}%
\pgfpathlineto{\pgfqpoint{5.201085in}{3.139667in}}%
\pgfpathlineto{\pgfqpoint{5.187792in}{3.139547in}}%
\pgfpathlineto{\pgfqpoint{5.180603in}{3.127383in}}%
\pgfpathlineto{\pgfqpoint{5.173415in}{3.115462in}}%
\pgfpathlineto{\pgfqpoint{5.166229in}{3.103775in}}%
\pgfpathlineto{\pgfqpoint{5.159044in}{3.092316in}}%
\pgfpathclose%
\pgfusepath{fill}%
\end{pgfscope}%
\begin{pgfscope}%
\pgfpathrectangle{\pgfqpoint{1.254980in}{0.150000in}}{\pgfqpoint{5.490039in}{5.490039in}}%
\pgfusepath{clip}%
\pgfsetbuttcap%
\pgfsetroundjoin%
\definecolor{currentfill}{rgb}{0.244972,0.287675,0.537260}%
\pgfsetfillcolor{currentfill}%
\pgfsetfillopacity{0.700000}%
\pgfsetlinewidth{0.000000pt}%
\definecolor{currentstroke}{rgb}{0.000000,0.000000,0.000000}%
\pgfsetstrokecolor{currentstroke}%
\pgfsetdash{}{0pt}%
\pgfpathmoveto{\pgfqpoint{3.053605in}{2.787152in}}%
\pgfpathlineto{\pgfqpoint{3.066590in}{2.772185in}}%
\pgfpathlineto{\pgfqpoint{3.079570in}{2.757480in}}%
\pgfpathlineto{\pgfqpoint{3.092545in}{2.743035in}}%
\pgfpathlineto{\pgfqpoint{3.105516in}{2.728848in}}%
\pgfpathlineto{\pgfqpoint{3.113306in}{2.739241in}}%
\pgfpathlineto{\pgfqpoint{3.121089in}{2.749742in}}%
\pgfpathlineto{\pgfqpoint{3.128866in}{2.760351in}}%
\pgfpathlineto{\pgfqpoint{3.136636in}{2.771069in}}%
\pgfpathlineto{\pgfqpoint{3.123679in}{2.785310in}}%
\pgfpathlineto{\pgfqpoint{3.110717in}{2.799809in}}%
\pgfpathlineto{\pgfqpoint{3.097751in}{2.814568in}}%
\pgfpathlineto{\pgfqpoint{3.084780in}{2.829589in}}%
\pgfpathlineto{\pgfqpoint{3.076997in}{2.818806in}}%
\pgfpathlineto{\pgfqpoint{3.069206in}{2.808140in}}%
\pgfpathlineto{\pgfqpoint{3.061409in}{2.797589in}}%
\pgfpathlineto{\pgfqpoint{3.053605in}{2.787152in}}%
\pgfpathclose%
\pgfusepath{fill}%
\end{pgfscope}%
\begin{pgfscope}%
\pgfpathrectangle{\pgfqpoint{1.254980in}{0.150000in}}{\pgfqpoint{5.490039in}{5.490039in}}%
\pgfusepath{clip}%
\pgfsetbuttcap%
\pgfsetroundjoin%
\definecolor{currentfill}{rgb}{0.233603,0.313828,0.543914}%
\pgfsetfillcolor{currentfill}%
\pgfsetfillopacity{0.700000}%
\pgfsetlinewidth{0.000000pt}%
\definecolor{currentstroke}{rgb}{0.000000,0.000000,0.000000}%
\pgfsetstrokecolor{currentstroke}%
\pgfsetdash{}{0pt}%
\pgfpathmoveto{\pgfqpoint{3.001608in}{2.849687in}}%
\pgfpathlineto{\pgfqpoint{3.014616in}{2.833648in}}%
\pgfpathlineto{\pgfqpoint{3.027618in}{2.817881in}}%
\pgfpathlineto{\pgfqpoint{3.040614in}{2.802383in}}%
\pgfpathlineto{\pgfqpoint{3.053605in}{2.787152in}}%
\pgfpathlineto{\pgfqpoint{3.061409in}{2.797589in}}%
\pgfpathlineto{\pgfqpoint{3.069206in}{2.808140in}}%
\pgfpathlineto{\pgfqpoint{3.076997in}{2.818806in}}%
\pgfpathlineto{\pgfqpoint{3.084780in}{2.829589in}}%
\pgfpathlineto{\pgfqpoint{3.071804in}{2.844874in}}%
\pgfpathlineto{\pgfqpoint{3.058822in}{2.860426in}}%
\pgfpathlineto{\pgfqpoint{3.045835in}{2.876247in}}%
\pgfpathlineto{\pgfqpoint{3.032841in}{2.892339in}}%
\pgfpathlineto{\pgfqpoint{3.025044in}{2.881492in}}%
\pgfpathlineto{\pgfqpoint{3.017239in}{2.870769in}}%
\pgfpathlineto{\pgfqpoint{3.009427in}{2.860167in}}%
\pgfpathlineto{\pgfqpoint{3.001608in}{2.849687in}}%
\pgfpathclose%
\pgfusepath{fill}%
\end{pgfscope}%
\begin{pgfscope}%
\pgfpathrectangle{\pgfqpoint{1.254980in}{0.150000in}}{\pgfqpoint{5.490039in}{5.490039in}}%
\pgfusepath{clip}%
\pgfsetbuttcap%
\pgfsetroundjoin%
\definecolor{currentfill}{rgb}{0.255645,0.260703,0.528312}%
\pgfsetfillcolor{currentfill}%
\pgfsetfillopacity{0.700000}%
\pgfsetlinewidth{0.000000pt}%
\definecolor{currentstroke}{rgb}{0.000000,0.000000,0.000000}%
\pgfsetstrokecolor{currentstroke}%
\pgfsetdash{}{0pt}%
\pgfpathmoveto{\pgfqpoint{4.422174in}{2.704522in}}%
\pgfpathlineto{\pgfqpoint{4.435254in}{2.703755in}}%
\pgfpathlineto{\pgfqpoint{4.448342in}{2.703156in}}%
\pgfpathlineto{\pgfqpoint{4.461439in}{2.702723in}}%
\pgfpathlineto{\pgfqpoint{4.474545in}{2.702457in}}%
\pgfpathlineto{\pgfqpoint{4.481920in}{2.712515in}}%
\pgfpathlineto{\pgfqpoint{4.489291in}{2.722657in}}%
\pgfpathlineto{\pgfqpoint{4.496658in}{2.732886in}}%
\pgfpathlineto{\pgfqpoint{4.504022in}{2.743206in}}%
\pgfpathlineto{\pgfqpoint{4.490927in}{2.743833in}}%
\pgfpathlineto{\pgfqpoint{4.477841in}{2.744626in}}%
\pgfpathlineto{\pgfqpoint{4.464763in}{2.745585in}}%
\pgfpathlineto{\pgfqpoint{4.451694in}{2.746712in}}%
\pgfpathlineto{\pgfqpoint{4.444320in}{2.736021in}}%
\pgfpathlineto{\pgfqpoint{4.436942in}{2.725429in}}%
\pgfpathlineto{\pgfqpoint{4.429560in}{2.714931in}}%
\pgfpathlineto{\pgfqpoint{4.422174in}{2.704522in}}%
\pgfpathclose%
\pgfusepath{fill}%
\end{pgfscope}%
\begin{pgfscope}%
\pgfpathrectangle{\pgfqpoint{1.254980in}{0.150000in}}{\pgfqpoint{5.490039in}{5.490039in}}%
\pgfusepath{clip}%
\pgfsetbuttcap%
\pgfsetroundjoin%
\definecolor{currentfill}{rgb}{0.255645,0.260703,0.528312}%
\pgfsetfillcolor{currentfill}%
\pgfsetfillopacity{0.700000}%
\pgfsetlinewidth{0.000000pt}%
\definecolor{currentstroke}{rgb}{0.000000,0.000000,0.000000}%
\pgfsetstrokecolor{currentstroke}%
\pgfsetdash{}{0pt}%
\pgfpathmoveto{\pgfqpoint{3.105516in}{2.728848in}}%
\pgfpathlineto{\pgfqpoint{3.118482in}{2.714916in}}%
\pgfpathlineto{\pgfqpoint{3.131445in}{2.701238in}}%
\pgfpathlineto{\pgfqpoint{3.144403in}{2.687811in}}%
\pgfpathlineto{\pgfqpoint{3.157357in}{2.674633in}}%
\pgfpathlineto{\pgfqpoint{3.165134in}{2.684982in}}%
\pgfpathlineto{\pgfqpoint{3.172903in}{2.695432in}}%
\pgfpathlineto{\pgfqpoint{3.180667in}{2.705983in}}%
\pgfpathlineto{\pgfqpoint{3.188423in}{2.716637in}}%
\pgfpathlineto{\pgfqpoint{3.175482in}{2.729869in}}%
\pgfpathlineto{\pgfqpoint{3.162537in}{2.743350in}}%
\pgfpathlineto{\pgfqpoint{3.149588in}{2.757083in}}%
\pgfpathlineto{\pgfqpoint{3.136636in}{2.771069in}}%
\pgfpathlineto{\pgfqpoint{3.128866in}{2.760351in}}%
\pgfpathlineto{\pgfqpoint{3.121089in}{2.749742in}}%
\pgfpathlineto{\pgfqpoint{3.113306in}{2.739241in}}%
\pgfpathlineto{\pgfqpoint{3.105516in}{2.728848in}}%
\pgfpathclose%
\pgfusepath{fill}%
\end{pgfscope}%
\begin{pgfscope}%
\pgfpathrectangle{\pgfqpoint{1.254980in}{0.150000in}}{\pgfqpoint{5.490039in}{5.490039in}}%
\pgfusepath{clip}%
\pgfsetbuttcap%
\pgfsetroundjoin%
\definecolor{currentfill}{rgb}{0.172719,0.448791,0.557885}%
\pgfsetfillcolor{currentfill}%
\pgfsetfillopacity{0.700000}%
\pgfsetlinewidth{0.000000pt}%
\definecolor{currentstroke}{rgb}{0.000000,0.000000,0.000000}%
\pgfsetstrokecolor{currentstroke}%
\pgfsetdash{}{0pt}%
\pgfpathmoveto{\pgfqpoint{5.241031in}{3.140940in}}%
\pgfpathlineto{\pgfqpoint{5.254369in}{3.141667in}}%
\pgfpathlineto{\pgfqpoint{5.267719in}{3.142546in}}%
\pgfpathlineto{\pgfqpoint{5.281081in}{3.143576in}}%
\pgfpathlineto{\pgfqpoint{5.294454in}{3.144757in}}%
\pgfpathlineto{\pgfqpoint{5.301604in}{3.155914in}}%
\pgfpathlineto{\pgfqpoint{5.308755in}{3.167314in}}%
\pgfpathlineto{\pgfqpoint{5.315909in}{3.178964in}}%
\pgfpathlineto{\pgfqpoint{5.323065in}{3.190872in}}%
\pgfpathlineto{\pgfqpoint{5.309713in}{3.190330in}}%
\pgfpathlineto{\pgfqpoint{5.296372in}{3.189938in}}%
\pgfpathlineto{\pgfqpoint{5.283044in}{3.189698in}}%
\pgfpathlineto{\pgfqpoint{5.269726in}{3.189608in}}%
\pgfpathlineto{\pgfqpoint{5.262549in}{3.177052in}}%
\pgfpathlineto{\pgfqpoint{5.255374in}{3.164760in}}%
\pgfpathlineto{\pgfqpoint{5.248202in}{3.152725in}}%
\pgfpathlineto{\pgfqpoint{5.241031in}{3.140940in}}%
\pgfpathclose%
\pgfusepath{fill}%
\end{pgfscope}%
\begin{pgfscope}%
\pgfpathrectangle{\pgfqpoint{1.254980in}{0.150000in}}{\pgfqpoint{5.490039in}{5.490039in}}%
\pgfusepath{clip}%
\pgfsetbuttcap%
\pgfsetroundjoin%
\definecolor{currentfill}{rgb}{0.220057,0.343307,0.549413}%
\pgfsetfillcolor{currentfill}%
\pgfsetfillopacity{0.700000}%
\pgfsetlinewidth{0.000000pt}%
\definecolor{currentstroke}{rgb}{0.000000,0.000000,0.000000}%
\pgfsetstrokecolor{currentstroke}%
\pgfsetdash{}{0pt}%
\pgfpathmoveto{\pgfqpoint{2.949510in}{2.916605in}}%
\pgfpathlineto{\pgfqpoint{2.962545in}{2.899456in}}%
\pgfpathlineto{\pgfqpoint{2.975573in}{2.882588in}}%
\pgfpathlineto{\pgfqpoint{2.988594in}{2.865999in}}%
\pgfpathlineto{\pgfqpoint{3.001608in}{2.849687in}}%
\pgfpathlineto{\pgfqpoint{3.009427in}{2.860167in}}%
\pgfpathlineto{\pgfqpoint{3.017239in}{2.870769in}}%
\pgfpathlineto{\pgfqpoint{3.025044in}{2.881492in}}%
\pgfpathlineto{\pgfqpoint{3.032841in}{2.892339in}}%
\pgfpathlineto{\pgfqpoint{3.019842in}{2.908705in}}%
\pgfpathlineto{\pgfqpoint{3.006836in}{2.925348in}}%
\pgfpathlineto{\pgfqpoint{2.993823in}{2.942269in}}%
\pgfpathlineto{\pgfqpoint{2.980804in}{2.959472in}}%
\pgfpathlineto{\pgfqpoint{2.972992in}{2.948561in}}%
\pgfpathlineto{\pgfqpoint{2.965172in}{2.937780in}}%
\pgfpathlineto{\pgfqpoint{2.957345in}{2.927129in}}%
\pgfpathlineto{\pgfqpoint{2.949510in}{2.916605in}}%
\pgfpathclose%
\pgfusepath{fill}%
\end{pgfscope}%
\begin{pgfscope}%
\pgfpathrectangle{\pgfqpoint{1.254980in}{0.150000in}}{\pgfqpoint{5.490039in}{5.490039in}}%
\pgfusepath{clip}%
\pgfsetbuttcap%
\pgfsetroundjoin%
\definecolor{currentfill}{rgb}{0.262138,0.242286,0.520837}%
\pgfsetfillcolor{currentfill}%
\pgfsetfillopacity{0.700000}%
\pgfsetlinewidth{0.000000pt}%
\definecolor{currentstroke}{rgb}{0.000000,0.000000,0.000000}%
\pgfsetstrokecolor{currentstroke}%
\pgfsetdash{}{0pt}%
\pgfpathmoveto{\pgfqpoint{4.340317in}{2.667125in}}%
\pgfpathlineto{\pgfqpoint{4.353374in}{2.666016in}}%
\pgfpathlineto{\pgfqpoint{4.366439in}{2.665077in}}%
\pgfpathlineto{\pgfqpoint{4.379512in}{2.664306in}}%
\pgfpathlineto{\pgfqpoint{4.392593in}{2.663705in}}%
\pgfpathlineto{\pgfqpoint{4.399994in}{2.673794in}}%
\pgfpathlineto{\pgfqpoint{4.407392in}{2.683958in}}%
\pgfpathlineto{\pgfqpoint{4.414785in}{2.694199in}}%
\pgfpathlineto{\pgfqpoint{4.422174in}{2.704522in}}%
\pgfpathlineto{\pgfqpoint{4.409103in}{2.705457in}}%
\pgfpathlineto{\pgfqpoint{4.396040in}{2.706560in}}%
\pgfpathlineto{\pgfqpoint{4.382985in}{2.707831in}}%
\pgfpathlineto{\pgfqpoint{4.369938in}{2.709273in}}%
\pgfpathlineto{\pgfqpoint{4.362539in}{2.698607in}}%
\pgfpathlineto{\pgfqpoint{4.355136in}{2.688030in}}%
\pgfpathlineto{\pgfqpoint{4.347728in}{2.677537in}}%
\pgfpathlineto{\pgfqpoint{4.340317in}{2.667125in}}%
\pgfpathclose%
\pgfusepath{fill}%
\end{pgfscope}%
\begin{pgfscope}%
\pgfpathrectangle{\pgfqpoint{1.254980in}{0.150000in}}{\pgfqpoint{5.490039in}{5.490039in}}%
\pgfusepath{clip}%
\pgfsetbuttcap%
\pgfsetroundjoin%
\definecolor{currentfill}{rgb}{0.280868,0.160771,0.472899}%
\pgfsetfillcolor{currentfill}%
\pgfsetfillopacity{0.700000}%
\pgfsetlinewidth{0.000000pt}%
\definecolor{currentstroke}{rgb}{0.000000,0.000000,0.000000}%
\pgfsetstrokecolor{currentstroke}%
\pgfsetdash{}{0pt}%
\pgfpathmoveto{\pgfqpoint{3.529174in}{2.510817in}}%
\pgfpathlineto{\pgfqpoint{3.542082in}{2.503168in}}%
\pgfpathlineto{\pgfqpoint{3.554992in}{2.495726in}}%
\pgfpathlineto{\pgfqpoint{3.567903in}{2.488491in}}%
\pgfpathlineto{\pgfqpoint{3.580816in}{2.481460in}}%
\pgfpathlineto{\pgfqpoint{3.588465in}{2.491944in}}%
\pgfpathlineto{\pgfqpoint{3.596109in}{2.502490in}}%
\pgfpathlineto{\pgfqpoint{3.603747in}{2.513100in}}%
\pgfpathlineto{\pgfqpoint{3.611381in}{2.523775in}}%
\pgfpathlineto{\pgfqpoint{3.598477in}{2.530917in}}%
\pgfpathlineto{\pgfqpoint{3.585576in}{2.538263in}}%
\pgfpathlineto{\pgfqpoint{3.572676in}{2.545815in}}%
\pgfpathlineto{\pgfqpoint{3.559778in}{2.553575in}}%
\pgfpathlineto{\pgfqpoint{3.552135in}{2.542779in}}%
\pgfpathlineto{\pgfqpoint{3.544487in}{2.532055in}}%
\pgfpathlineto{\pgfqpoint{3.536833in}{2.521402in}}%
\pgfpathlineto{\pgfqpoint{3.529174in}{2.510817in}}%
\pgfpathclose%
\pgfusepath{fill}%
\end{pgfscope}%
\begin{pgfscope}%
\pgfpathrectangle{\pgfqpoint{1.254980in}{0.150000in}}{\pgfqpoint{5.490039in}{5.490039in}}%
\pgfusepath{clip}%
\pgfsetbuttcap%
\pgfsetroundjoin%
\definecolor{currentfill}{rgb}{0.263663,0.237631,0.518762}%
\pgfsetfillcolor{currentfill}%
\pgfsetfillopacity{0.700000}%
\pgfsetlinewidth{0.000000pt}%
\definecolor{currentstroke}{rgb}{0.000000,0.000000,0.000000}%
\pgfsetstrokecolor{currentstroke}%
\pgfsetdash{}{0pt}%
\pgfpathmoveto{\pgfqpoint{3.157357in}{2.674633in}}%
\pgfpathlineto{\pgfqpoint{3.170309in}{2.661703in}}%
\pgfpathlineto{\pgfqpoint{3.183256in}{2.649018in}}%
\pgfpathlineto{\pgfqpoint{3.196201in}{2.636576in}}%
\pgfpathlineto{\pgfqpoint{3.209143in}{2.624376in}}%
\pgfpathlineto{\pgfqpoint{3.216906in}{2.634681in}}%
\pgfpathlineto{\pgfqpoint{3.224663in}{2.645080in}}%
\pgfpathlineto{\pgfqpoint{3.232413in}{2.655573in}}%
\pgfpathlineto{\pgfqpoint{3.240157in}{2.666161in}}%
\pgfpathlineto{\pgfqpoint{3.227228in}{2.678416in}}%
\pgfpathlineto{\pgfqpoint{3.214296in}{2.690912in}}%
\pgfpathlineto{\pgfqpoint{3.201361in}{2.703652in}}%
\pgfpathlineto{\pgfqpoint{3.188423in}{2.716637in}}%
\pgfpathlineto{\pgfqpoint{3.180667in}{2.705983in}}%
\pgfpathlineto{\pgfqpoint{3.172903in}{2.695432in}}%
\pgfpathlineto{\pgfqpoint{3.165134in}{2.684982in}}%
\pgfpathlineto{\pgfqpoint{3.157357in}{2.674633in}}%
\pgfpathclose%
\pgfusepath{fill}%
\end{pgfscope}%
\begin{pgfscope}%
\pgfpathrectangle{\pgfqpoint{1.254980in}{0.150000in}}{\pgfqpoint{5.490039in}{5.490039in}}%
\pgfusepath{clip}%
\pgfsetbuttcap%
\pgfsetroundjoin%
\definecolor{currentfill}{rgb}{0.280255,0.165693,0.476498}%
\pgfsetfillcolor{currentfill}%
\pgfsetfillopacity{0.700000}%
\pgfsetlinewidth{0.000000pt}%
\definecolor{currentstroke}{rgb}{0.000000,0.000000,0.000000}%
\pgfsetstrokecolor{currentstroke}%
\pgfsetdash{}{0pt}%
\pgfpathmoveto{\pgfqpoint{3.878840in}{2.519580in}}%
\pgfpathlineto{\pgfqpoint{3.891785in}{2.515451in}}%
\pgfpathlineto{\pgfqpoint{3.904735in}{2.511510in}}%
\pgfpathlineto{\pgfqpoint{3.917690in}{2.507753in}}%
\pgfpathlineto{\pgfqpoint{3.930650in}{2.504182in}}%
\pgfpathlineto{\pgfqpoint{3.938194in}{2.514616in}}%
\pgfpathlineto{\pgfqpoint{3.945734in}{2.525101in}}%
\pgfpathlineto{\pgfqpoint{3.953269in}{2.535641in}}%
\pgfpathlineto{\pgfqpoint{3.960799in}{2.546238in}}%
\pgfpathlineto{\pgfqpoint{3.947847in}{2.550003in}}%
\pgfpathlineto{\pgfqpoint{3.934901in}{2.553954in}}%
\pgfpathlineto{\pgfqpoint{3.921959in}{2.558089in}}%
\pgfpathlineto{\pgfqpoint{3.909023in}{2.562411in}}%
\pgfpathlineto{\pgfqpoint{3.901484in}{2.551611in}}%
\pgfpathlineto{\pgfqpoint{3.893941in}{2.540874in}}%
\pgfpathlineto{\pgfqpoint{3.886393in}{2.530197in}}%
\pgfpathlineto{\pgfqpoint{3.878840in}{2.519580in}}%
\pgfpathclose%
\pgfusepath{fill}%
\end{pgfscope}%
\begin{pgfscope}%
\pgfpathrectangle{\pgfqpoint{1.254980in}{0.150000in}}{\pgfqpoint{5.490039in}{5.490039in}}%
\pgfusepath{clip}%
\pgfsetbuttcap%
\pgfsetroundjoin%
\definecolor{currentfill}{rgb}{0.206756,0.371758,0.553117}%
\pgfsetfillcolor{currentfill}%
\pgfsetfillopacity{0.700000}%
\pgfsetlinewidth{0.000000pt}%
\definecolor{currentstroke}{rgb}{0.000000,0.000000,0.000000}%
\pgfsetstrokecolor{currentstroke}%
\pgfsetdash{}{0pt}%
\pgfpathmoveto{\pgfqpoint{2.897295in}{2.988071in}}%
\pgfpathlineto{\pgfqpoint{2.910361in}{2.969769in}}%
\pgfpathlineto{\pgfqpoint{2.923418in}{2.951759in}}%
\pgfpathlineto{\pgfqpoint{2.936468in}{2.934039in}}%
\pgfpathlineto{\pgfqpoint{2.949510in}{2.916605in}}%
\pgfpathlineto{\pgfqpoint{2.957345in}{2.927129in}}%
\pgfpathlineto{\pgfqpoint{2.965172in}{2.937780in}}%
\pgfpathlineto{\pgfqpoint{2.972992in}{2.948561in}}%
\pgfpathlineto{\pgfqpoint{2.980804in}{2.959472in}}%
\pgfpathlineto{\pgfqpoint{2.967777in}{2.976959in}}%
\pgfpathlineto{\pgfqpoint{2.954743in}{2.994732in}}%
\pgfpathlineto{\pgfqpoint{2.941701in}{3.012795in}}%
\pgfpathlineto{\pgfqpoint{2.928651in}{3.031151in}}%
\pgfpathlineto{\pgfqpoint{2.920824in}{3.020176in}}%
\pgfpathlineto{\pgfqpoint{2.912989in}{3.009338in}}%
\pgfpathlineto{\pgfqpoint{2.905146in}{2.998637in}}%
\pgfpathlineto{\pgfqpoint{2.897295in}{2.988071in}}%
\pgfpathclose%
\pgfusepath{fill}%
\end{pgfscope}%
\begin{pgfscope}%
\pgfpathrectangle{\pgfqpoint{1.254980in}{0.150000in}}{\pgfqpoint{5.490039in}{5.490039in}}%
\pgfusepath{clip}%
\pgfsetbuttcap%
\pgfsetroundjoin%
\definecolor{currentfill}{rgb}{0.266580,0.228262,0.514349}%
\pgfsetfillcolor{currentfill}%
\pgfsetfillopacity{0.700000}%
\pgfsetlinewidth{0.000000pt}%
\definecolor{currentstroke}{rgb}{0.000000,0.000000,0.000000}%
\pgfsetstrokecolor{currentstroke}%
\pgfsetdash{}{0pt}%
\pgfpathmoveto{\pgfqpoint{4.258444in}{2.631129in}}%
\pgfpathlineto{\pgfqpoint{4.271479in}{2.629641in}}%
\pgfpathlineto{\pgfqpoint{4.284522in}{2.628325in}}%
\pgfpathlineto{\pgfqpoint{4.297572in}{2.627180in}}%
\pgfpathlineto{\pgfqpoint{4.310630in}{2.626206in}}%
\pgfpathlineto{\pgfqpoint{4.318058in}{2.636334in}}%
\pgfpathlineto{\pgfqpoint{4.325482in}{2.646527in}}%
\pgfpathlineto{\pgfqpoint{4.332902in}{2.656789in}}%
\pgfpathlineto{\pgfqpoint{4.340317in}{2.667125in}}%
\pgfpathlineto{\pgfqpoint{4.327268in}{2.668404in}}%
\pgfpathlineto{\pgfqpoint{4.314227in}{2.669854in}}%
\pgfpathlineto{\pgfqpoint{4.301194in}{2.671475in}}%
\pgfpathlineto{\pgfqpoint{4.288168in}{2.673267in}}%
\pgfpathlineto{\pgfqpoint{4.280743in}{2.662617in}}%
\pgfpathlineto{\pgfqpoint{4.273314in}{2.652046in}}%
\pgfpathlineto{\pgfqpoint{4.265881in}{2.641552in}}%
\pgfpathlineto{\pgfqpoint{4.258444in}{2.631129in}}%
\pgfpathclose%
\pgfusepath{fill}%
\end{pgfscope}%
\begin{pgfscope}%
\pgfpathrectangle{\pgfqpoint{1.254980in}{0.150000in}}{\pgfqpoint{5.490039in}{5.490039in}}%
\pgfusepath{clip}%
\pgfsetbuttcap%
\pgfsetroundjoin%
\definecolor{currentfill}{rgb}{0.281412,0.155834,0.469201}%
\pgfsetfillcolor{currentfill}%
\pgfsetfillopacity{0.700000}%
\pgfsetlinewidth{0.000000pt}%
\definecolor{currentstroke}{rgb}{0.000000,0.000000,0.000000}%
\pgfsetstrokecolor{currentstroke}%
\pgfsetdash{}{0pt}%
\pgfpathmoveto{\pgfqpoint{3.663018in}{2.497236in}}%
\pgfpathlineto{\pgfqpoint{3.675934in}{2.491101in}}%
\pgfpathlineto{\pgfqpoint{3.688853in}{2.485165in}}%
\pgfpathlineto{\pgfqpoint{3.701775in}{2.479427in}}%
\pgfpathlineto{\pgfqpoint{3.714701in}{2.473884in}}%
\pgfpathlineto{\pgfqpoint{3.722311in}{2.484369in}}%
\pgfpathlineto{\pgfqpoint{3.729915in}{2.494910in}}%
\pgfpathlineto{\pgfqpoint{3.737515in}{2.505509in}}%
\pgfpathlineto{\pgfqpoint{3.745109in}{2.516166in}}%
\pgfpathlineto{\pgfqpoint{3.732193in}{2.521848in}}%
\pgfpathlineto{\pgfqpoint{3.719280in}{2.527725in}}%
\pgfpathlineto{\pgfqpoint{3.706370in}{2.533800in}}%
\pgfpathlineto{\pgfqpoint{3.693463in}{2.540072in}}%
\pgfpathlineto{\pgfqpoint{3.685859in}{2.529266in}}%
\pgfpathlineto{\pgfqpoint{3.678251in}{2.518526in}}%
\pgfpathlineto{\pgfqpoint{3.670637in}{2.507850in}}%
\pgfpathlineto{\pgfqpoint{3.663018in}{2.497236in}}%
\pgfpathclose%
\pgfusepath{fill}%
\end{pgfscope}%
\begin{pgfscope}%
\pgfpathrectangle{\pgfqpoint{1.254980in}{0.150000in}}{\pgfqpoint{5.490039in}{5.490039in}}%
\pgfusepath{clip}%
\pgfsetbuttcap%
\pgfsetroundjoin%
\definecolor{currentfill}{rgb}{0.279574,0.170599,0.479997}%
\pgfsetfillcolor{currentfill}%
\pgfsetfillopacity{0.700000}%
\pgfsetlinewidth{0.000000pt}%
\definecolor{currentstroke}{rgb}{0.000000,0.000000,0.000000}%
\pgfsetstrokecolor{currentstroke}%
\pgfsetdash{}{0pt}%
\pgfpathmoveto{\pgfqpoint{3.395171in}{2.537294in}}%
\pgfpathlineto{\pgfqpoint{3.408084in}{2.528021in}}%
\pgfpathlineto{\pgfqpoint{3.420996in}{2.518966in}}%
\pgfpathlineto{\pgfqpoint{3.433909in}{2.510127in}}%
\pgfpathlineto{\pgfqpoint{3.446822in}{2.501505in}}%
\pgfpathlineto{\pgfqpoint{3.454514in}{2.511900in}}%
\pgfpathlineto{\pgfqpoint{3.462200in}{2.522366in}}%
\pgfpathlineto{\pgfqpoint{3.469880in}{2.532904in}}%
\pgfpathlineto{\pgfqpoint{3.477555in}{2.543514in}}%
\pgfpathlineto{\pgfqpoint{3.464652in}{2.552220in}}%
\pgfpathlineto{\pgfqpoint{3.451750in}{2.561141in}}%
\pgfpathlineto{\pgfqpoint{3.438848in}{2.570279in}}%
\pgfpathlineto{\pgfqpoint{3.425946in}{2.579635in}}%
\pgfpathlineto{\pgfqpoint{3.418261in}{2.568932in}}%
\pgfpathlineto{\pgfqpoint{3.410570in}{2.558308in}}%
\pgfpathlineto{\pgfqpoint{3.402873in}{2.547762in}}%
\pgfpathlineto{\pgfqpoint{3.395171in}{2.537294in}}%
\pgfpathclose%
\pgfusepath{fill}%
\end{pgfscope}%
\begin{pgfscope}%
\pgfpathrectangle{\pgfqpoint{1.254980in}{0.150000in}}{\pgfqpoint{5.490039in}{5.490039in}}%
\pgfusepath{clip}%
\pgfsetbuttcap%
\pgfsetroundjoin%
\definecolor{currentfill}{rgb}{0.165117,0.467423,0.558141}%
\pgfsetfillcolor{currentfill}%
\pgfsetfillopacity{0.700000}%
\pgfsetlinewidth{0.000000pt}%
\definecolor{currentstroke}{rgb}{0.000000,0.000000,0.000000}%
\pgfsetstrokecolor{currentstroke}%
\pgfsetdash{}{0pt}%
\pgfpathmoveto{\pgfqpoint{5.323065in}{3.190872in}}%
\pgfpathlineto{\pgfqpoint{5.336428in}{3.191565in}}%
\pgfpathlineto{\pgfqpoint{5.349803in}{3.192407in}}%
\pgfpathlineto{\pgfqpoint{5.363191in}{3.193400in}}%
\pgfpathlineto{\pgfqpoint{5.376590in}{3.194542in}}%
\pgfpathlineto{\pgfqpoint{5.383726in}{3.206059in}}%
\pgfpathlineto{\pgfqpoint{5.390866in}{3.217841in}}%
\pgfpathlineto{\pgfqpoint{5.398008in}{3.229896in}}%
\pgfpathlineto{\pgfqpoint{5.384626in}{3.229251in}}%
\pgfpathlineto{\pgfqpoint{5.371255in}{3.228756in}}%
\pgfpathlineto{\pgfqpoint{5.357897in}{3.228410in}}%
\pgfpathlineto{\pgfqpoint{5.344550in}{3.228214in}}%
\pgfpathlineto{\pgfqpoint{5.337385in}{3.215490in}}%
\pgfpathlineto{\pgfqpoint{5.330223in}{3.203045in}}%
\pgfpathlineto{\pgfqpoint{5.323065in}{3.190872in}}%
\pgfpathclose%
\pgfusepath{fill}%
\end{pgfscope}%
\begin{pgfscope}%
\pgfpathrectangle{\pgfqpoint{1.254980in}{0.150000in}}{\pgfqpoint{5.490039in}{5.490039in}}%
\pgfusepath{clip}%
\pgfsetbuttcap%
\pgfsetroundjoin%
\definecolor{currentfill}{rgb}{0.270595,0.214069,0.507052}%
\pgfsetfillcolor{currentfill}%
\pgfsetfillopacity{0.700000}%
\pgfsetlinewidth{0.000000pt}%
\definecolor{currentstroke}{rgb}{0.000000,0.000000,0.000000}%
\pgfsetstrokecolor{currentstroke}%
\pgfsetdash{}{0pt}%
\pgfpathmoveto{\pgfqpoint{3.209143in}{2.624376in}}%
\pgfpathlineto{\pgfqpoint{3.222083in}{2.612415in}}%
\pgfpathlineto{\pgfqpoint{3.235020in}{2.600693in}}%
\pgfpathlineto{\pgfqpoint{3.247954in}{2.589206in}}%
\pgfpathlineto{\pgfqpoint{3.260887in}{2.577954in}}%
\pgfpathlineto{\pgfqpoint{3.268637in}{2.588215in}}%
\pgfpathlineto{\pgfqpoint{3.276381in}{2.598562in}}%
\pgfpathlineto{\pgfqpoint{3.284119in}{2.608997in}}%
\pgfpathlineto{\pgfqpoint{3.291851in}{2.619520in}}%
\pgfpathlineto{\pgfqpoint{3.278930in}{2.630828in}}%
\pgfpathlineto{\pgfqpoint{3.266008in}{2.642369in}}%
\pgfpathlineto{\pgfqpoint{3.253084in}{2.654146in}}%
\pgfpathlineto{\pgfqpoint{3.240157in}{2.666161in}}%
\pgfpathlineto{\pgfqpoint{3.232413in}{2.655573in}}%
\pgfpathlineto{\pgfqpoint{3.224663in}{2.645080in}}%
\pgfpathlineto{\pgfqpoint{3.216906in}{2.634681in}}%
\pgfpathlineto{\pgfqpoint{3.209143in}{2.624376in}}%
\pgfpathclose%
\pgfusepath{fill}%
\end{pgfscope}%
\begin{pgfscope}%
\pgfpathrectangle{\pgfqpoint{1.254980in}{0.150000in}}{\pgfqpoint{5.490039in}{5.490039in}}%
\pgfusepath{clip}%
\pgfsetbuttcap%
\pgfsetroundjoin%
\definecolor{currentfill}{rgb}{0.271828,0.209303,0.504434}%
\pgfsetfillcolor{currentfill}%
\pgfsetfillopacity{0.700000}%
\pgfsetlinewidth{0.000000pt}%
\definecolor{currentstroke}{rgb}{0.000000,0.000000,0.000000}%
\pgfsetstrokecolor{currentstroke}%
\pgfsetdash{}{0pt}%
\pgfpathmoveto{\pgfqpoint{4.176549in}{2.596670in}}%
\pgfpathlineto{\pgfqpoint{4.189564in}{2.594765in}}%
\pgfpathlineto{\pgfqpoint{4.202586in}{2.593035in}}%
\pgfpathlineto{\pgfqpoint{4.215615in}{2.591478in}}%
\pgfpathlineto{\pgfqpoint{4.228651in}{2.590095in}}%
\pgfpathlineto{\pgfqpoint{4.236106in}{2.600262in}}%
\pgfpathlineto{\pgfqpoint{4.243556in}{2.610488in}}%
\pgfpathlineto{\pgfqpoint{4.251002in}{2.620776in}}%
\pgfpathlineto{\pgfqpoint{4.258444in}{2.631129in}}%
\pgfpathlineto{\pgfqpoint{4.245416in}{2.632790in}}%
\pgfpathlineto{\pgfqpoint{4.232396in}{2.634624in}}%
\pgfpathlineto{\pgfqpoint{4.219383in}{2.636631in}}%
\pgfpathlineto{\pgfqpoint{4.206377in}{2.638813in}}%
\pgfpathlineto{\pgfqpoint{4.198926in}{2.628172in}}%
\pgfpathlineto{\pgfqpoint{4.191472in}{2.617604in}}%
\pgfpathlineto{\pgfqpoint{4.184012in}{2.607104in}}%
\pgfpathlineto{\pgfqpoint{4.176549in}{2.596670in}}%
\pgfpathclose%
\pgfusepath{fill}%
\end{pgfscope}%
\begin{pgfscope}%
\pgfpathrectangle{\pgfqpoint{1.254980in}{0.150000in}}{\pgfqpoint{5.490039in}{5.490039in}}%
\pgfusepath{clip}%
\pgfsetbuttcap%
\pgfsetroundjoin%
\definecolor{currentfill}{rgb}{0.190631,0.407061,0.556089}%
\pgfsetfillcolor{currentfill}%
\pgfsetfillopacity{0.700000}%
\pgfsetlinewidth{0.000000pt}%
\definecolor{currentstroke}{rgb}{0.000000,0.000000,0.000000}%
\pgfsetstrokecolor{currentstroke}%
\pgfsetdash{}{0pt}%
\pgfpathmoveto{\pgfqpoint{2.844944in}{3.064259in}}%
\pgfpathlineto{\pgfqpoint{2.858046in}{3.044759in}}%
\pgfpathlineto{\pgfqpoint{2.871138in}{3.025563in}}%
\pgfpathlineto{\pgfqpoint{2.884221in}{3.006668in}}%
\pgfpathlineto{\pgfqpoint{2.897295in}{2.988071in}}%
\pgfpathlineto{\pgfqpoint{2.905146in}{2.998637in}}%
\pgfpathlineto{\pgfqpoint{2.912989in}{3.009338in}}%
\pgfpathlineto{\pgfqpoint{2.920824in}{3.020176in}}%
\pgfpathlineto{\pgfqpoint{2.928651in}{3.031151in}}%
\pgfpathlineto{\pgfqpoint{2.915593in}{3.049801in}}%
\pgfpathlineto{\pgfqpoint{2.902526in}{3.068750in}}%
\pgfpathlineto{\pgfqpoint{2.889450in}{3.087999in}}%
\pgfpathlineto{\pgfqpoint{2.876365in}{3.107552in}}%
\pgfpathlineto{\pgfqpoint{2.868522in}{3.096513in}}%
\pgfpathlineto{\pgfqpoint{2.860671in}{3.085618in}}%
\pgfpathlineto{\pgfqpoint{2.852812in}{3.074867in}}%
\pgfpathlineto{\pgfqpoint{2.844944in}{3.064259in}}%
\pgfpathclose%
\pgfusepath{fill}%
\end{pgfscope}%
\begin{pgfscope}%
\pgfpathrectangle{\pgfqpoint{1.254980in}{0.150000in}}{\pgfqpoint{5.490039in}{5.490039in}}%
\pgfusepath{clip}%
\pgfsetbuttcap%
\pgfsetroundjoin%
\definecolor{currentfill}{rgb}{0.281412,0.155834,0.469201}%
\pgfsetfillcolor{currentfill}%
\pgfsetfillopacity{0.700000}%
\pgfsetlinewidth{0.000000pt}%
\definecolor{currentstroke}{rgb}{0.000000,0.000000,0.000000}%
\pgfsetstrokecolor{currentstroke}%
\pgfsetdash{}{0pt}%
\pgfpathmoveto{\pgfqpoint{3.796812in}{2.495379in}}%
\pgfpathlineto{\pgfqpoint{3.809747in}{2.490662in}}%
\pgfpathlineto{\pgfqpoint{3.822687in}{2.486136in}}%
\pgfpathlineto{\pgfqpoint{3.835631in}{2.481798in}}%
\pgfpathlineto{\pgfqpoint{3.848580in}{2.477649in}}%
\pgfpathlineto{\pgfqpoint{3.856152in}{2.488055in}}%
\pgfpathlineto{\pgfqpoint{3.863720in}{2.498511in}}%
\pgfpathlineto{\pgfqpoint{3.871282in}{2.509018in}}%
\pgfpathlineto{\pgfqpoint{3.878840in}{2.519580in}}%
\pgfpathlineto{\pgfqpoint{3.865900in}{2.523896in}}%
\pgfpathlineto{\pgfqpoint{3.852964in}{2.528399in}}%
\pgfpathlineto{\pgfqpoint{3.840033in}{2.533092in}}%
\pgfpathlineto{\pgfqpoint{3.827106in}{2.537975in}}%
\pgfpathlineto{\pgfqpoint{3.819540in}{2.527237in}}%
\pgfpathlineto{\pgfqpoint{3.811969in}{2.516560in}}%
\pgfpathlineto{\pgfqpoint{3.804393in}{2.505941in}}%
\pgfpathlineto{\pgfqpoint{3.796812in}{2.495379in}}%
\pgfpathclose%
\pgfusepath{fill}%
\end{pgfscope}%
\begin{pgfscope}%
\pgfpathrectangle{\pgfqpoint{1.254980in}{0.150000in}}{\pgfqpoint{5.490039in}{5.490039in}}%
\pgfusepath{clip}%
\pgfsetbuttcap%
\pgfsetroundjoin%
\definecolor{currentfill}{rgb}{0.275191,0.194905,0.496005}%
\pgfsetfillcolor{currentfill}%
\pgfsetfillopacity{0.700000}%
\pgfsetlinewidth{0.000000pt}%
\definecolor{currentstroke}{rgb}{0.000000,0.000000,0.000000}%
\pgfsetstrokecolor{currentstroke}%
\pgfsetdash{}{0pt}%
\pgfpathmoveto{\pgfqpoint{4.094624in}{2.563903in}}%
\pgfpathlineto{\pgfqpoint{4.107620in}{2.561543in}}%
\pgfpathlineto{\pgfqpoint{4.120623in}{2.559361in}}%
\pgfpathlineto{\pgfqpoint{4.133633in}{2.557355in}}%
\pgfpathlineto{\pgfqpoint{4.146650in}{2.555525in}}%
\pgfpathlineto{\pgfqpoint{4.154131in}{2.565728in}}%
\pgfpathlineto{\pgfqpoint{4.161608in}{2.575985in}}%
\pgfpathlineto{\pgfqpoint{4.169081in}{2.586298in}}%
\pgfpathlineto{\pgfqpoint{4.176549in}{2.596670in}}%
\pgfpathlineto{\pgfqpoint{4.163541in}{2.598750in}}%
\pgfpathlineto{\pgfqpoint{4.150540in}{2.601005in}}%
\pgfpathlineto{\pgfqpoint{4.137545in}{2.603438in}}%
\pgfpathlineto{\pgfqpoint{4.124558in}{2.606047in}}%
\pgfpathlineto{\pgfqpoint{4.117081in}{2.595415in}}%
\pgfpathlineto{\pgfqpoint{4.109600in}{2.584849in}}%
\pgfpathlineto{\pgfqpoint{4.102114in}{2.574346in}}%
\pgfpathlineto{\pgfqpoint{4.094624in}{2.563903in}}%
\pgfpathclose%
\pgfusepath{fill}%
\end{pgfscope}%
\begin{pgfscope}%
\pgfpathrectangle{\pgfqpoint{1.254980in}{0.150000in}}{\pgfqpoint{5.490039in}{5.490039in}}%
\pgfusepath{clip}%
\pgfsetbuttcap%
\pgfsetroundjoin%
\definecolor{currentfill}{rgb}{0.275191,0.194905,0.496005}%
\pgfsetfillcolor{currentfill}%
\pgfsetfillopacity{0.700000}%
\pgfsetlinewidth{0.000000pt}%
\definecolor{currentstroke}{rgb}{0.000000,0.000000,0.000000}%
\pgfsetstrokecolor{currentstroke}%
\pgfsetdash{}{0pt}%
\pgfpathmoveto{\pgfqpoint{3.260887in}{2.577954in}}%
\pgfpathlineto{\pgfqpoint{3.273818in}{2.566934in}}%
\pgfpathlineto{\pgfqpoint{3.286748in}{2.556145in}}%
\pgfpathlineto{\pgfqpoint{3.299676in}{2.545586in}}%
\pgfpathlineto{\pgfqpoint{3.312603in}{2.535254in}}%
\pgfpathlineto{\pgfqpoint{3.320341in}{2.545470in}}%
\pgfpathlineto{\pgfqpoint{3.328072in}{2.555765in}}%
\pgfpathlineto{\pgfqpoint{3.335798in}{2.566142in}}%
\pgfpathlineto{\pgfqpoint{3.343518in}{2.576600in}}%
\pgfpathlineto{\pgfqpoint{3.330603in}{2.586987in}}%
\pgfpathlineto{\pgfqpoint{3.317687in}{2.597602in}}%
\pgfpathlineto{\pgfqpoint{3.304770in}{2.608446in}}%
\pgfpathlineto{\pgfqpoint{3.291851in}{2.619520in}}%
\pgfpathlineto{\pgfqpoint{3.284119in}{2.608997in}}%
\pgfpathlineto{\pgfqpoint{3.276381in}{2.598562in}}%
\pgfpathlineto{\pgfqpoint{3.268637in}{2.588215in}}%
\pgfpathlineto{\pgfqpoint{3.260887in}{2.577954in}}%
\pgfpathclose%
\pgfusepath{fill}%
\end{pgfscope}%
\begin{pgfscope}%
\pgfpathrectangle{\pgfqpoint{1.254980in}{0.150000in}}{\pgfqpoint{5.490039in}{5.490039in}}%
\pgfusepath{clip}%
\pgfsetbuttcap%
\pgfsetroundjoin%
\definecolor{currentfill}{rgb}{0.281887,0.150881,0.465405}%
\pgfsetfillcolor{currentfill}%
\pgfsetfillopacity{0.700000}%
\pgfsetlinewidth{0.000000pt}%
\definecolor{currentstroke}{rgb}{0.000000,0.000000,0.000000}%
\pgfsetstrokecolor{currentstroke}%
\pgfsetdash{}{0pt}%
\pgfpathmoveto{\pgfqpoint{3.580816in}{2.481460in}}%
\pgfpathlineto{\pgfqpoint{3.593731in}{2.474634in}}%
\pgfpathlineto{\pgfqpoint{3.606649in}{2.468010in}}%
\pgfpathlineto{\pgfqpoint{3.619569in}{2.461587in}}%
\pgfpathlineto{\pgfqpoint{3.632492in}{2.455365in}}%
\pgfpathlineto{\pgfqpoint{3.640131in}{2.465748in}}%
\pgfpathlineto{\pgfqpoint{3.647765in}{2.476186in}}%
\pgfpathlineto{\pgfqpoint{3.655394in}{2.486682in}}%
\pgfpathlineto{\pgfqpoint{3.663018in}{2.497236in}}%
\pgfpathlineto{\pgfqpoint{3.650105in}{2.503569in}}%
\pgfpathlineto{\pgfqpoint{3.637195in}{2.510102in}}%
\pgfpathlineto{\pgfqpoint{3.624286in}{2.516838in}}%
\pgfpathlineto{\pgfqpoint{3.611381in}{2.523775in}}%
\pgfpathlineto{\pgfqpoint{3.603747in}{2.513100in}}%
\pgfpathlineto{\pgfqpoint{3.596109in}{2.502490in}}%
\pgfpathlineto{\pgfqpoint{3.588465in}{2.491944in}}%
\pgfpathlineto{\pgfqpoint{3.580816in}{2.481460in}}%
\pgfpathclose%
\pgfusepath{fill}%
\end{pgfscope}%
\begin{pgfscope}%
\pgfpathrectangle{\pgfqpoint{1.254980in}{0.150000in}}{\pgfqpoint{5.490039in}{5.490039in}}%
\pgfusepath{clip}%
\pgfsetbuttcap%
\pgfsetroundjoin%
\definecolor{currentfill}{rgb}{0.280868,0.160771,0.472899}%
\pgfsetfillcolor{currentfill}%
\pgfsetfillopacity{0.700000}%
\pgfsetlinewidth{0.000000pt}%
\definecolor{currentstroke}{rgb}{0.000000,0.000000,0.000000}%
\pgfsetstrokecolor{currentstroke}%
\pgfsetdash{}{0pt}%
\pgfpathmoveto{\pgfqpoint{3.446822in}{2.501505in}}%
\pgfpathlineto{\pgfqpoint{3.459736in}{2.493096in}}%
\pgfpathlineto{\pgfqpoint{3.472651in}{2.484900in}}%
\pgfpathlineto{\pgfqpoint{3.485567in}{2.476916in}}%
\pgfpathlineto{\pgfqpoint{3.498484in}{2.469141in}}%
\pgfpathlineto{\pgfqpoint{3.506164in}{2.479464in}}%
\pgfpathlineto{\pgfqpoint{3.513840in}{2.489850in}}%
\pgfpathlineto{\pgfqpoint{3.521510in}{2.500300in}}%
\pgfpathlineto{\pgfqpoint{3.529174in}{2.510817in}}%
\pgfpathlineto{\pgfqpoint{3.516267in}{2.518675in}}%
\pgfpathlineto{\pgfqpoint{3.503362in}{2.526743in}}%
\pgfpathlineto{\pgfqpoint{3.490458in}{2.535022in}}%
\pgfpathlineto{\pgfqpoint{3.477555in}{2.543514in}}%
\pgfpathlineto{\pgfqpoint{3.469880in}{2.532904in}}%
\pgfpathlineto{\pgfqpoint{3.462200in}{2.522366in}}%
\pgfpathlineto{\pgfqpoint{3.454514in}{2.511900in}}%
\pgfpathlineto{\pgfqpoint{3.446822in}{2.501505in}}%
\pgfpathclose%
\pgfusepath{fill}%
\end{pgfscope}%
\begin{pgfscope}%
\pgfpathrectangle{\pgfqpoint{1.254980in}{0.150000in}}{\pgfqpoint{5.490039in}{5.490039in}}%
\pgfusepath{clip}%
\pgfsetbuttcap%
\pgfsetroundjoin%
\definecolor{currentfill}{rgb}{0.278012,0.180367,0.486697}%
\pgfsetfillcolor{currentfill}%
\pgfsetfillopacity{0.700000}%
\pgfsetlinewidth{0.000000pt}%
\definecolor{currentstroke}{rgb}{0.000000,0.000000,0.000000}%
\pgfsetstrokecolor{currentstroke}%
\pgfsetdash{}{0pt}%
\pgfpathmoveto{\pgfqpoint{4.012661in}{2.533007in}}%
\pgfpathlineto{\pgfqpoint{4.025641in}{2.530154in}}%
\pgfpathlineto{\pgfqpoint{4.038627in}{2.527480in}}%
\pgfpathlineto{\pgfqpoint{4.051619in}{2.524986in}}%
\pgfpathlineto{\pgfqpoint{4.064617in}{2.522671in}}%
\pgfpathlineto{\pgfqpoint{4.072126in}{2.532903in}}%
\pgfpathlineto{\pgfqpoint{4.079630in}{2.543184in}}%
\pgfpathlineto{\pgfqpoint{4.087129in}{2.553517in}}%
\pgfpathlineto{\pgfqpoint{4.094624in}{2.563903in}}%
\pgfpathlineto{\pgfqpoint{4.081634in}{2.566441in}}%
\pgfpathlineto{\pgfqpoint{4.068650in}{2.569157in}}%
\pgfpathlineto{\pgfqpoint{4.055673in}{2.572053in}}%
\pgfpathlineto{\pgfqpoint{4.042701in}{2.575128in}}%
\pgfpathlineto{\pgfqpoint{4.035198in}{2.564510in}}%
\pgfpathlineto{\pgfqpoint{4.027690in}{2.553952in}}%
\pgfpathlineto{\pgfqpoint{4.020178in}{2.543452in}}%
\pgfpathlineto{\pgfqpoint{4.012661in}{2.533007in}}%
\pgfpathclose%
\pgfusepath{fill}%
\end{pgfscope}%
\begin{pgfscope}%
\pgfpathrectangle{\pgfqpoint{1.254980in}{0.150000in}}{\pgfqpoint{5.490039in}{5.490039in}}%
\pgfusepath{clip}%
\pgfsetbuttcap%
\pgfsetroundjoin%
\definecolor{currentfill}{rgb}{0.220057,0.343307,0.549413}%
\pgfsetfillcolor{currentfill}%
\pgfsetfillopacity{0.700000}%
\pgfsetlinewidth{0.000000pt}%
\definecolor{currentstroke}{rgb}{0.000000,0.000000,0.000000}%
\pgfsetstrokecolor{currentstroke}%
\pgfsetdash{}{0pt}%
\pgfpathmoveto{\pgfqpoint{4.802337in}{2.868091in}}%
\pgfpathlineto{\pgfqpoint{4.815558in}{2.868971in}}%
\pgfpathlineto{\pgfqpoint{4.828789in}{2.870010in}}%
\pgfpathlineto{\pgfqpoint{4.842031in}{2.871207in}}%
\pgfpathlineto{\pgfqpoint{4.855284in}{2.872562in}}%
\pgfpathlineto{\pgfqpoint{4.862542in}{2.882192in}}%
\pgfpathlineto{\pgfqpoint{4.869798in}{2.891948in}}%
\pgfpathlineto{\pgfqpoint{4.877051in}{2.901835in}}%
\pgfpathlineto{\pgfqpoint{4.884302in}{2.911858in}}%
\pgfpathlineto{\pgfqpoint{4.871064in}{2.910976in}}%
\pgfpathlineto{\pgfqpoint{4.857837in}{2.910252in}}%
\pgfpathlineto{\pgfqpoint{4.844620in}{2.909686in}}%
\pgfpathlineto{\pgfqpoint{4.831414in}{2.909277in}}%
\pgfpathlineto{\pgfqpoint{4.824148in}{2.898771in}}%
\pgfpathlineto{\pgfqpoint{4.816880in}{2.888409in}}%
\pgfpathlineto{\pgfqpoint{4.809610in}{2.878183in}}%
\pgfpathlineto{\pgfqpoint{4.802337in}{2.868091in}}%
\pgfpathclose%
\pgfusepath{fill}%
\end{pgfscope}%
\begin{pgfscope}%
\pgfpathrectangle{\pgfqpoint{1.254980in}{0.150000in}}{\pgfqpoint{5.490039in}{5.490039in}}%
\pgfusepath{clip}%
\pgfsetbuttcap%
\pgfsetroundjoin%
\definecolor{currentfill}{rgb}{0.210503,0.363727,0.552206}%
\pgfsetfillcolor{currentfill}%
\pgfsetfillopacity{0.700000}%
\pgfsetlinewidth{0.000000pt}%
\definecolor{currentstroke}{rgb}{0.000000,0.000000,0.000000}%
\pgfsetstrokecolor{currentstroke}%
\pgfsetdash{}{0pt}%
\pgfpathmoveto{\pgfqpoint{4.884302in}{2.911858in}}%
\pgfpathlineto{\pgfqpoint{4.897551in}{2.912897in}}%
\pgfpathlineto{\pgfqpoint{4.910810in}{2.914093in}}%
\pgfpathlineto{\pgfqpoint{4.924081in}{2.915446in}}%
\pgfpathlineto{\pgfqpoint{4.937362in}{2.916956in}}%
\pgfpathlineto{\pgfqpoint{4.944595in}{2.926630in}}%
\pgfpathlineto{\pgfqpoint{4.951826in}{2.936446in}}%
\pgfpathlineto{\pgfqpoint{4.959055in}{2.946408in}}%
\pgfpathlineto{\pgfqpoint{4.966283in}{2.956523in}}%
\pgfpathlineto{\pgfqpoint{4.953017in}{2.955515in}}%
\pgfpathlineto{\pgfqpoint{4.939763in}{2.954663in}}%
\pgfpathlineto{\pgfqpoint{4.926519in}{2.953967in}}%
\pgfpathlineto{\pgfqpoint{4.913286in}{2.953428in}}%
\pgfpathlineto{\pgfqpoint{4.906043in}{2.942803in}}%
\pgfpathlineto{\pgfqpoint{4.898798in}{2.932336in}}%
\pgfpathlineto{\pgfqpoint{4.891551in}{2.922023in}}%
\pgfpathlineto{\pgfqpoint{4.884302in}{2.911858in}}%
\pgfpathclose%
\pgfusepath{fill}%
\end{pgfscope}%
\begin{pgfscope}%
\pgfpathrectangle{\pgfqpoint{1.254980in}{0.150000in}}{\pgfqpoint{5.490039in}{5.490039in}}%
\pgfusepath{clip}%
\pgfsetbuttcap%
\pgfsetroundjoin%
\definecolor{currentfill}{rgb}{0.227802,0.326594,0.546532}%
\pgfsetfillcolor{currentfill}%
\pgfsetfillopacity{0.700000}%
\pgfsetlinewidth{0.000000pt}%
\definecolor{currentstroke}{rgb}{0.000000,0.000000,0.000000}%
\pgfsetstrokecolor{currentstroke}%
\pgfsetdash{}{0pt}%
\pgfpathmoveto{\pgfqpoint{4.720383in}{2.825228in}}%
\pgfpathlineto{\pgfqpoint{4.733576in}{2.825916in}}%
\pgfpathlineto{\pgfqpoint{4.746780in}{2.826764in}}%
\pgfpathlineto{\pgfqpoint{4.759994in}{2.827771in}}%
\pgfpathlineto{\pgfqpoint{4.773218in}{2.828938in}}%
\pgfpathlineto{\pgfqpoint{4.780502in}{2.838554in}}%
\pgfpathlineto{\pgfqpoint{4.787783in}{2.848281in}}%
\pgfpathlineto{\pgfqpoint{4.795061in}{2.858125in}}%
\pgfpathlineto{\pgfqpoint{4.802337in}{2.868091in}}%
\pgfpathlineto{\pgfqpoint{4.789126in}{2.867369in}}%
\pgfpathlineto{\pgfqpoint{4.775926in}{2.866807in}}%
\pgfpathlineto{\pgfqpoint{4.762736in}{2.866404in}}%
\pgfpathlineto{\pgfqpoint{4.749556in}{2.866161in}}%
\pgfpathlineto{\pgfqpoint{4.742267in}{2.855740in}}%
\pgfpathlineto{\pgfqpoint{4.734975in}{2.845448in}}%
\pgfpathlineto{\pgfqpoint{4.727680in}{2.835279in}}%
\pgfpathlineto{\pgfqpoint{4.720383in}{2.825228in}}%
\pgfpathclose%
\pgfusepath{fill}%
\end{pgfscope}%
\begin{pgfscope}%
\pgfpathrectangle{\pgfqpoint{1.254980in}{0.150000in}}{\pgfqpoint{5.490039in}{5.490039in}}%
\pgfusepath{clip}%
\pgfsetbuttcap%
\pgfsetroundjoin%
\definecolor{currentfill}{rgb}{0.281887,0.150881,0.465405}%
\pgfsetfillcolor{currentfill}%
\pgfsetfillopacity{0.700000}%
\pgfsetlinewidth{0.000000pt}%
\definecolor{currentstroke}{rgb}{0.000000,0.000000,0.000000}%
\pgfsetstrokecolor{currentstroke}%
\pgfsetdash{}{0pt}%
\pgfpathmoveto{\pgfqpoint{3.714701in}{2.473884in}}%
\pgfpathlineto{\pgfqpoint{3.727630in}{2.468536in}}%
\pgfpathlineto{\pgfqpoint{3.740562in}{2.463382in}}%
\pgfpathlineto{\pgfqpoint{3.753499in}{2.458421in}}%
\pgfpathlineto{\pgfqpoint{3.766439in}{2.453652in}}%
\pgfpathlineto{\pgfqpoint{3.774040in}{2.464009in}}%
\pgfpathlineto{\pgfqpoint{3.781635in}{2.474415in}}%
\pgfpathlineto{\pgfqpoint{3.789226in}{2.484871in}}%
\pgfpathlineto{\pgfqpoint{3.796812in}{2.495379in}}%
\pgfpathlineto{\pgfqpoint{3.783880in}{2.500287in}}%
\pgfpathlineto{\pgfqpoint{3.770953in}{2.505387in}}%
\pgfpathlineto{\pgfqpoint{3.758029in}{2.510679in}}%
\pgfpathlineto{\pgfqpoint{3.745109in}{2.516166in}}%
\pgfpathlineto{\pgfqpoint{3.737515in}{2.505509in}}%
\pgfpathlineto{\pgfqpoint{3.729915in}{2.494910in}}%
\pgfpathlineto{\pgfqpoint{3.722311in}{2.484369in}}%
\pgfpathlineto{\pgfqpoint{3.714701in}{2.473884in}}%
\pgfpathclose%
\pgfusepath{fill}%
\end{pgfscope}%
\begin{pgfscope}%
\pgfpathrectangle{\pgfqpoint{1.254980in}{0.150000in}}{\pgfqpoint{5.490039in}{5.490039in}}%
\pgfusepath{clip}%
\pgfsetbuttcap%
\pgfsetroundjoin%
\definecolor{currentfill}{rgb}{0.203063,0.379716,0.553925}%
\pgfsetfillcolor{currentfill}%
\pgfsetfillopacity{0.700000}%
\pgfsetlinewidth{0.000000pt}%
\definecolor{currentstroke}{rgb}{0.000000,0.000000,0.000000}%
\pgfsetstrokecolor{currentstroke}%
\pgfsetdash{}{0pt}%
\pgfpathmoveto{\pgfqpoint{4.966283in}{2.956523in}}%
\pgfpathlineto{\pgfqpoint{4.979559in}{2.957687in}}%
\pgfpathlineto{\pgfqpoint{4.992846in}{2.959007in}}%
\pgfpathlineto{\pgfqpoint{5.006145in}{2.960482in}}%
\pgfpathlineto{\pgfqpoint{5.019455in}{2.962112in}}%
\pgfpathlineto{\pgfqpoint{5.026664in}{2.971867in}}%
\pgfpathlineto{\pgfqpoint{5.033872in}{2.981779in}}%
\pgfpathlineto{\pgfqpoint{5.041078in}{2.991855in}}%
\pgfpathlineto{\pgfqpoint{5.048283in}{3.002100in}}%
\pgfpathlineto{\pgfqpoint{5.034990in}{3.000999in}}%
\pgfpathlineto{\pgfqpoint{5.021708in}{3.000053in}}%
\pgfpathlineto{\pgfqpoint{5.008438in}{2.999262in}}%
\pgfpathlineto{\pgfqpoint{4.995178in}{2.998626in}}%
\pgfpathlineto{\pgfqpoint{4.987956in}{2.987842in}}%
\pgfpathlineto{\pgfqpoint{4.980733in}{2.977234in}}%
\pgfpathlineto{\pgfqpoint{4.973508in}{2.966796in}}%
\pgfpathlineto{\pgfqpoint{4.966283in}{2.956523in}}%
\pgfpathclose%
\pgfusepath{fill}%
\end{pgfscope}%
\begin{pgfscope}%
\pgfpathrectangle{\pgfqpoint{1.254980in}{0.150000in}}{\pgfqpoint{5.490039in}{5.490039in}}%
\pgfusepath{clip}%
\pgfsetbuttcap%
\pgfsetroundjoin%
\definecolor{currentfill}{rgb}{0.237441,0.305202,0.541921}%
\pgfsetfillcolor{currentfill}%
\pgfsetfillopacity{0.700000}%
\pgfsetlinewidth{0.000000pt}%
\definecolor{currentstroke}{rgb}{0.000000,0.000000,0.000000}%
\pgfsetstrokecolor{currentstroke}%
\pgfsetdash{}{0pt}%
\pgfpathmoveto{\pgfqpoint{4.638435in}{2.783299in}}%
\pgfpathlineto{\pgfqpoint{4.651602in}{2.783759in}}%
\pgfpathlineto{\pgfqpoint{4.664778in}{2.784381in}}%
\pgfpathlineto{\pgfqpoint{4.677964in}{2.785165in}}%
\pgfpathlineto{\pgfqpoint{4.691160in}{2.786109in}}%
\pgfpathlineto{\pgfqpoint{4.698471in}{2.795736in}}%
\pgfpathlineto{\pgfqpoint{4.705778in}{2.805462in}}%
\pgfpathlineto{\pgfqpoint{4.713082in}{2.815291in}}%
\pgfpathlineto{\pgfqpoint{4.720383in}{2.825228in}}%
\pgfpathlineto{\pgfqpoint{4.707199in}{2.824701in}}%
\pgfpathlineto{\pgfqpoint{4.694026in}{2.824335in}}%
\pgfpathlineto{\pgfqpoint{4.680862in}{2.824130in}}%
\pgfpathlineto{\pgfqpoint{4.667708in}{2.824087in}}%
\pgfpathlineto{\pgfqpoint{4.660394in}{2.813722in}}%
\pgfpathlineto{\pgfqpoint{4.653078in}{2.803473in}}%
\pgfpathlineto{\pgfqpoint{4.645758in}{2.793333in}}%
\pgfpathlineto{\pgfqpoint{4.638435in}{2.783299in}}%
\pgfpathclose%
\pgfusepath{fill}%
\end{pgfscope}%
\begin{pgfscope}%
\pgfpathrectangle{\pgfqpoint{1.254980in}{0.150000in}}{\pgfqpoint{5.490039in}{5.490039in}}%
\pgfusepath{clip}%
\pgfsetbuttcap%
\pgfsetroundjoin%
\definecolor{currentfill}{rgb}{0.194100,0.399323,0.555565}%
\pgfsetfillcolor{currentfill}%
\pgfsetfillopacity{0.700000}%
\pgfsetlinewidth{0.000000pt}%
\definecolor{currentstroke}{rgb}{0.000000,0.000000,0.000000}%
\pgfsetstrokecolor{currentstroke}%
\pgfsetdash{}{0pt}%
\pgfpathmoveto{\pgfqpoint{5.048283in}{3.002100in}}%
\pgfpathlineto{\pgfqpoint{5.061587in}{3.003356in}}%
\pgfpathlineto{\pgfqpoint{5.074902in}{3.004765in}}%
\pgfpathlineto{\pgfqpoint{5.088229in}{3.006329in}}%
\pgfpathlineto{\pgfqpoint{5.101568in}{3.008047in}}%
\pgfpathlineto{\pgfqpoint{5.108754in}{3.017922in}}%
\pgfpathlineto{\pgfqpoint{5.115939in}{3.027972in}}%
\pgfpathlineto{\pgfqpoint{5.123123in}{3.038204in}}%
\pgfpathlineto{\pgfqpoint{5.130307in}{3.048624in}}%
\pgfpathlineto{\pgfqpoint{5.116987in}{3.047464in}}%
\pgfpathlineto{\pgfqpoint{5.103679in}{3.046456in}}%
\pgfpathlineto{\pgfqpoint{5.090381in}{3.045603in}}%
\pgfpathlineto{\pgfqpoint{5.077095in}{3.044903in}}%
\pgfpathlineto{\pgfqpoint{5.069893in}{3.033917in}}%
\pgfpathlineto{\pgfqpoint{5.062690in}{3.023125in}}%
\pgfpathlineto{\pgfqpoint{5.055487in}{3.012522in}}%
\pgfpathlineto{\pgfqpoint{5.048283in}{3.002100in}}%
\pgfpathclose%
\pgfusepath{fill}%
\end{pgfscope}%
\begin{pgfscope}%
\pgfpathrectangle{\pgfqpoint{1.254980in}{0.150000in}}{\pgfqpoint{5.490039in}{5.490039in}}%
\pgfusepath{clip}%
\pgfsetbuttcap%
\pgfsetroundjoin%
\definecolor{currentfill}{rgb}{0.244972,0.287675,0.537260}%
\pgfsetfillcolor{currentfill}%
\pgfsetfillopacity{0.700000}%
\pgfsetlinewidth{0.000000pt}%
\definecolor{currentstroke}{rgb}{0.000000,0.000000,0.000000}%
\pgfsetstrokecolor{currentstroke}%
\pgfsetdash{}{0pt}%
\pgfpathmoveto{\pgfqpoint{4.556490in}{2.742352in}}%
\pgfpathlineto{\pgfqpoint{4.569630in}{2.742550in}}%
\pgfpathlineto{\pgfqpoint{4.582780in}{2.742911in}}%
\pgfpathlineto{\pgfqpoint{4.595939in}{2.743436in}}%
\pgfpathlineto{\pgfqpoint{4.609108in}{2.744124in}}%
\pgfpathlineto{\pgfqpoint{4.616445in}{2.753783in}}%
\pgfpathlineto{\pgfqpoint{4.623779in}{2.763529in}}%
\pgfpathlineto{\pgfqpoint{4.631109in}{2.773366in}}%
\pgfpathlineto{\pgfqpoint{4.638435in}{2.783299in}}%
\pgfpathlineto{\pgfqpoint{4.625278in}{2.783001in}}%
\pgfpathlineto{\pgfqpoint{4.612131in}{2.782866in}}%
\pgfpathlineto{\pgfqpoint{4.598993in}{2.782894in}}%
\pgfpathlineto{\pgfqpoint{4.585865in}{2.783085in}}%
\pgfpathlineto{\pgfqpoint{4.578526in}{2.772752in}}%
\pgfpathlineto{\pgfqpoint{4.571185in}{2.762522in}}%
\pgfpathlineto{\pgfqpoint{4.563839in}{2.752391in}}%
\pgfpathlineto{\pgfqpoint{4.556490in}{2.742352in}}%
\pgfpathclose%
\pgfusepath{fill}%
\end{pgfscope}%
\begin{pgfscope}%
\pgfpathrectangle{\pgfqpoint{1.254980in}{0.150000in}}{\pgfqpoint{5.490039in}{5.490039in}}%
\pgfusepath{clip}%
\pgfsetbuttcap%
\pgfsetroundjoin%
\definecolor{currentfill}{rgb}{0.278826,0.175490,0.483397}%
\pgfsetfillcolor{currentfill}%
\pgfsetfillopacity{0.700000}%
\pgfsetlinewidth{0.000000pt}%
\definecolor{currentstroke}{rgb}{0.000000,0.000000,0.000000}%
\pgfsetstrokecolor{currentstroke}%
\pgfsetdash{}{0pt}%
\pgfpathmoveto{\pgfqpoint{3.312603in}{2.535254in}}%
\pgfpathlineto{\pgfqpoint{3.325529in}{2.525148in}}%
\pgfpathlineto{\pgfqpoint{3.338454in}{2.515266in}}%
\pgfpathlineto{\pgfqpoint{3.351379in}{2.505607in}}%
\pgfpathlineto{\pgfqpoint{3.364303in}{2.496169in}}%
\pgfpathlineto{\pgfqpoint{3.372029in}{2.506340in}}%
\pgfpathlineto{\pgfqpoint{3.379749in}{2.516584in}}%
\pgfpathlineto{\pgfqpoint{3.387463in}{2.526902in}}%
\pgfpathlineto{\pgfqpoint{3.395171in}{2.537294in}}%
\pgfpathlineto{\pgfqpoint{3.382258in}{2.546787in}}%
\pgfpathlineto{\pgfqpoint{3.369345in}{2.556501in}}%
\pgfpathlineto{\pgfqpoint{3.356432in}{2.566438in}}%
\pgfpathlineto{\pgfqpoint{3.343518in}{2.576600in}}%
\pgfpathlineto{\pgfqpoint{3.335798in}{2.566142in}}%
\pgfpathlineto{\pgfqpoint{3.328072in}{2.555765in}}%
\pgfpathlineto{\pgfqpoint{3.320341in}{2.545470in}}%
\pgfpathlineto{\pgfqpoint{3.312603in}{2.535254in}}%
\pgfpathclose%
\pgfusepath{fill}%
\end{pgfscope}%
\begin{pgfscope}%
\pgfpathrectangle{\pgfqpoint{1.254980in}{0.150000in}}{\pgfqpoint{5.490039in}{5.490039in}}%
\pgfusepath{clip}%
\pgfsetbuttcap%
\pgfsetroundjoin%
\definecolor{currentfill}{rgb}{0.185556,0.418570,0.556753}%
\pgfsetfillcolor{currentfill}%
\pgfsetfillopacity{0.700000}%
\pgfsetlinewidth{0.000000pt}%
\definecolor{currentstroke}{rgb}{0.000000,0.000000,0.000000}%
\pgfsetstrokecolor{currentstroke}%
\pgfsetdash{}{0pt}%
\pgfpathmoveto{\pgfqpoint{5.130307in}{3.048624in}}%
\pgfpathlineto{\pgfqpoint{5.143639in}{3.049938in}}%
\pgfpathlineto{\pgfqpoint{5.156982in}{3.051405in}}%
\pgfpathlineto{\pgfqpoint{5.170337in}{3.053024in}}%
\pgfpathlineto{\pgfqpoint{5.183704in}{3.054796in}}%
\pgfpathlineto{\pgfqpoint{5.190868in}{3.064836in}}%
\pgfpathlineto{\pgfqpoint{5.198033in}{3.075071in}}%
\pgfpathlineto{\pgfqpoint{5.205197in}{3.085507in}}%
\pgfpathlineto{\pgfqpoint{5.212362in}{3.096151in}}%
\pgfpathlineto{\pgfqpoint{5.199015in}{3.094964in}}%
\pgfpathlineto{\pgfqpoint{5.185680in}{3.093929in}}%
\pgfpathlineto{\pgfqpoint{5.172356in}{3.093046in}}%
\pgfpathlineto{\pgfqpoint{5.159044in}{3.092316in}}%
\pgfpathlineto{\pgfqpoint{5.151859in}{3.081078in}}%
\pgfpathlineto{\pgfqpoint{5.144675in}{3.070055in}}%
\pgfpathlineto{\pgfqpoint{5.137491in}{3.059239in}}%
\pgfpathlineto{\pgfqpoint{5.130307in}{3.048624in}}%
\pgfpathclose%
\pgfusepath{fill}%
\end{pgfscope}%
\begin{pgfscope}%
\pgfpathrectangle{\pgfqpoint{1.254980in}{0.150000in}}{\pgfqpoint{5.490039in}{5.490039in}}%
\pgfusepath{clip}%
\pgfsetbuttcap%
\pgfsetroundjoin%
\definecolor{currentfill}{rgb}{0.279574,0.170599,0.479997}%
\pgfsetfillcolor{currentfill}%
\pgfsetfillopacity{0.700000}%
\pgfsetlinewidth{0.000000pt}%
\definecolor{currentstroke}{rgb}{0.000000,0.000000,0.000000}%
\pgfsetstrokecolor{currentstroke}%
\pgfsetdash{}{0pt}%
\pgfpathmoveto{\pgfqpoint{3.930650in}{2.504182in}}%
\pgfpathlineto{\pgfqpoint{3.943615in}{2.500795in}}%
\pgfpathlineto{\pgfqpoint{3.956586in}{2.497591in}}%
\pgfpathlineto{\pgfqpoint{3.969563in}{2.494569in}}%
\pgfpathlineto{\pgfqpoint{3.982546in}{2.491729in}}%
\pgfpathlineto{\pgfqpoint{3.990081in}{2.501979in}}%
\pgfpathlineto{\pgfqpoint{3.997613in}{2.512273in}}%
\pgfpathlineto{\pgfqpoint{4.005139in}{2.522615in}}%
\pgfpathlineto{\pgfqpoint{4.012661in}{2.533007in}}%
\pgfpathlineto{\pgfqpoint{3.999687in}{2.536042in}}%
\pgfpathlineto{\pgfqpoint{3.986719in}{2.539258in}}%
\pgfpathlineto{\pgfqpoint{3.973756in}{2.542656in}}%
\pgfpathlineto{\pgfqpoint{3.960799in}{2.546238in}}%
\pgfpathlineto{\pgfqpoint{3.953269in}{2.535641in}}%
\pgfpathlineto{\pgfqpoint{3.945734in}{2.525101in}}%
\pgfpathlineto{\pgfqpoint{3.938194in}{2.514616in}}%
\pgfpathlineto{\pgfqpoint{3.930650in}{2.504182in}}%
\pgfpathclose%
\pgfusepath{fill}%
\end{pgfscope}%
\begin{pgfscope}%
\pgfpathrectangle{\pgfqpoint{1.254980in}{0.150000in}}{\pgfqpoint{5.490039in}{5.490039in}}%
\pgfusepath{clip}%
\pgfsetbuttcap%
\pgfsetroundjoin%
\definecolor{currentfill}{rgb}{0.252194,0.269783,0.531579}%
\pgfsetfillcolor{currentfill}%
\pgfsetfillopacity{0.700000}%
\pgfsetlinewidth{0.000000pt}%
\definecolor{currentstroke}{rgb}{0.000000,0.000000,0.000000}%
\pgfsetstrokecolor{currentstroke}%
\pgfsetdash{}{0pt}%
\pgfpathmoveto{\pgfqpoint{4.474545in}{2.702457in}}%
\pgfpathlineto{\pgfqpoint{4.487659in}{2.702357in}}%
\pgfpathlineto{\pgfqpoint{4.500782in}{2.702423in}}%
\pgfpathlineto{\pgfqpoint{4.513915in}{2.702653in}}%
\pgfpathlineto{\pgfqpoint{4.527057in}{2.703049in}}%
\pgfpathlineto{\pgfqpoint{4.534421in}{2.712756in}}%
\pgfpathlineto{\pgfqpoint{4.541781in}{2.722539in}}%
\pgfpathlineto{\pgfqpoint{4.549138in}{2.732403in}}%
\pgfpathlineto{\pgfqpoint{4.556490in}{2.742352in}}%
\pgfpathlineto{\pgfqpoint{4.543360in}{2.742319in}}%
\pgfpathlineto{\pgfqpoint{4.530238in}{2.742449in}}%
\pgfpathlineto{\pgfqpoint{4.517125in}{2.742745in}}%
\pgfpathlineto{\pgfqpoint{4.504022in}{2.743206in}}%
\pgfpathlineto{\pgfqpoint{4.496658in}{2.732886in}}%
\pgfpathlineto{\pgfqpoint{4.489291in}{2.722657in}}%
\pgfpathlineto{\pgfqpoint{4.481920in}{2.712515in}}%
\pgfpathlineto{\pgfqpoint{4.474545in}{2.702457in}}%
\pgfpathclose%
\pgfusepath{fill}%
\end{pgfscope}%
\begin{pgfscope}%
\pgfpathrectangle{\pgfqpoint{1.254980in}{0.150000in}}{\pgfqpoint{5.490039in}{5.490039in}}%
\pgfusepath{clip}%
\pgfsetbuttcap%
\pgfsetroundjoin%
\definecolor{currentfill}{rgb}{0.250425,0.274290,0.533103}%
\pgfsetfillcolor{currentfill}%
\pgfsetfillopacity{0.700000}%
\pgfsetlinewidth{0.000000pt}%
\definecolor{currentstroke}{rgb}{0.000000,0.000000,0.000000}%
\pgfsetstrokecolor{currentstroke}%
\pgfsetdash{}{0pt}%
\pgfpathmoveto{\pgfqpoint{3.022315in}{2.746530in}}%
\pgfpathlineto{\pgfqpoint{3.035315in}{2.731589in}}%
\pgfpathlineto{\pgfqpoint{3.048310in}{2.716910in}}%
\pgfpathlineto{\pgfqpoint{3.061301in}{2.702491in}}%
\pgfpathlineto{\pgfqpoint{3.074286in}{2.688330in}}%
\pgfpathlineto{\pgfqpoint{3.082104in}{2.698303in}}%
\pgfpathlineto{\pgfqpoint{3.089915in}{2.708379in}}%
\pgfpathlineto{\pgfqpoint{3.097719in}{2.718561in}}%
\pgfpathlineto{\pgfqpoint{3.105516in}{2.728848in}}%
\pgfpathlineto{\pgfqpoint{3.092545in}{2.743035in}}%
\pgfpathlineto{\pgfqpoint{3.079570in}{2.757480in}}%
\pgfpathlineto{\pgfqpoint{3.066590in}{2.772185in}}%
\pgfpathlineto{\pgfqpoint{3.053605in}{2.787152in}}%
\pgfpathlineto{\pgfqpoint{3.045793in}{2.776828in}}%
\pgfpathlineto{\pgfqpoint{3.037974in}{2.766617in}}%
\pgfpathlineto{\pgfqpoint{3.030148in}{2.756518in}}%
\pgfpathlineto{\pgfqpoint{3.022315in}{2.746530in}}%
\pgfpathclose%
\pgfusepath{fill}%
\end{pgfscope}%
\begin{pgfscope}%
\pgfpathrectangle{\pgfqpoint{1.254980in}{0.150000in}}{\pgfqpoint{5.490039in}{5.490039in}}%
\pgfusepath{clip}%
\pgfsetbuttcap%
\pgfsetroundjoin%
\definecolor{currentfill}{rgb}{0.237441,0.305202,0.541921}%
\pgfsetfillcolor{currentfill}%
\pgfsetfillopacity{0.700000}%
\pgfsetlinewidth{0.000000pt}%
\definecolor{currentstroke}{rgb}{0.000000,0.000000,0.000000}%
\pgfsetstrokecolor{currentstroke}%
\pgfsetdash{}{0pt}%
\pgfpathmoveto{\pgfqpoint{2.970256in}{2.808962in}}%
\pgfpathlineto{\pgfqpoint{2.983280in}{2.792949in}}%
\pgfpathlineto{\pgfqpoint{2.996298in}{2.777207in}}%
\pgfpathlineto{\pgfqpoint{3.009309in}{2.761735in}}%
\pgfpathlineto{\pgfqpoint{3.022315in}{2.746530in}}%
\pgfpathlineto{\pgfqpoint{3.030148in}{2.756518in}}%
\pgfpathlineto{\pgfqpoint{3.037974in}{2.766617in}}%
\pgfpathlineto{\pgfqpoint{3.045793in}{2.776828in}}%
\pgfpathlineto{\pgfqpoint{3.053605in}{2.787152in}}%
\pgfpathlineto{\pgfqpoint{3.040614in}{2.802383in}}%
\pgfpathlineto{\pgfqpoint{3.027618in}{2.817881in}}%
\pgfpathlineto{\pgfqpoint{3.014616in}{2.833648in}}%
\pgfpathlineto{\pgfqpoint{3.001608in}{2.849687in}}%
\pgfpathlineto{\pgfqpoint{2.993781in}{2.839327in}}%
\pgfpathlineto{\pgfqpoint{2.985947in}{2.829087in}}%
\pgfpathlineto{\pgfqpoint{2.978105in}{2.818965in}}%
\pgfpathlineto{\pgfqpoint{2.970256in}{2.808962in}}%
\pgfpathclose%
\pgfusepath{fill}%
\end{pgfscope}%
\begin{pgfscope}%
\pgfpathrectangle{\pgfqpoint{1.254980in}{0.150000in}}{\pgfqpoint{5.490039in}{5.490039in}}%
\pgfusepath{clip}%
\pgfsetbuttcap%
\pgfsetroundjoin%
\definecolor{currentfill}{rgb}{0.177423,0.437527,0.557565}%
\pgfsetfillcolor{currentfill}%
\pgfsetfillopacity{0.700000}%
\pgfsetlinewidth{0.000000pt}%
\definecolor{currentstroke}{rgb}{0.000000,0.000000,0.000000}%
\pgfsetstrokecolor{currentstroke}%
\pgfsetdash{}{0pt}%
\pgfpathmoveto{\pgfqpoint{5.212362in}{3.096151in}}%
\pgfpathlineto{\pgfqpoint{5.225721in}{3.097490in}}%
\pgfpathlineto{\pgfqpoint{5.239092in}{3.098981in}}%
\pgfpathlineto{\pgfqpoint{5.252474in}{3.100623in}}%
\pgfpathlineto{\pgfqpoint{5.265869in}{3.102417in}}%
\pgfpathlineto{\pgfqpoint{5.273014in}{3.112673in}}%
\pgfpathlineto{\pgfqpoint{5.280159in}{3.123144in}}%
\pgfpathlineto{\pgfqpoint{5.287306in}{3.133836in}}%
\pgfpathlineto{\pgfqpoint{5.294454in}{3.144757in}}%
\pgfpathlineto{\pgfqpoint{5.281081in}{3.143576in}}%
\pgfpathlineto{\pgfqpoint{5.267719in}{3.142546in}}%
\pgfpathlineto{\pgfqpoint{5.254369in}{3.141667in}}%
\pgfpathlineto{\pgfqpoint{5.241031in}{3.140940in}}%
\pgfpathlineto{\pgfqpoint{5.233862in}{3.129396in}}%
\pgfpathlineto{\pgfqpoint{5.226694in}{3.118089in}}%
\pgfpathlineto{\pgfqpoint{5.219528in}{3.107009in}}%
\pgfpathlineto{\pgfqpoint{5.212362in}{3.096151in}}%
\pgfpathclose%
\pgfusepath{fill}%
\end{pgfscope}%
\begin{pgfscope}%
\pgfpathrectangle{\pgfqpoint{1.254980in}{0.150000in}}{\pgfqpoint{5.490039in}{5.490039in}}%
\pgfusepath{clip}%
\pgfsetbuttcap%
\pgfsetroundjoin%
\definecolor{currentfill}{rgb}{0.258965,0.251537,0.524736}%
\pgfsetfillcolor{currentfill}%
\pgfsetfillopacity{0.700000}%
\pgfsetlinewidth{0.000000pt}%
\definecolor{currentstroke}{rgb}{0.000000,0.000000,0.000000}%
\pgfsetstrokecolor{currentstroke}%
\pgfsetdash{}{0pt}%
\pgfpathmoveto{\pgfqpoint{4.392593in}{2.663705in}}%
\pgfpathlineto{\pgfqpoint{4.405683in}{2.663271in}}%
\pgfpathlineto{\pgfqpoint{4.418781in}{2.663005in}}%
\pgfpathlineto{\pgfqpoint{4.431888in}{2.662906in}}%
\pgfpathlineto{\pgfqpoint{4.445004in}{2.662974in}}%
\pgfpathlineto{\pgfqpoint{4.452395in}{2.672740in}}%
\pgfpathlineto{\pgfqpoint{4.459783in}{2.682574in}}%
\pgfpathlineto{\pgfqpoint{4.467166in}{2.692478in}}%
\pgfpathlineto{\pgfqpoint{4.474545in}{2.702457in}}%
\pgfpathlineto{\pgfqpoint{4.461439in}{2.702723in}}%
\pgfpathlineto{\pgfqpoint{4.448342in}{2.703156in}}%
\pgfpathlineto{\pgfqpoint{4.435254in}{2.703755in}}%
\pgfpathlineto{\pgfqpoint{4.422174in}{2.704522in}}%
\pgfpathlineto{\pgfqpoint{4.414785in}{2.694199in}}%
\pgfpathlineto{\pgfqpoint{4.407392in}{2.683958in}}%
\pgfpathlineto{\pgfqpoint{4.399994in}{2.673794in}}%
\pgfpathlineto{\pgfqpoint{4.392593in}{2.663705in}}%
\pgfpathclose%
\pgfusepath{fill}%
\end{pgfscope}%
\begin{pgfscope}%
\pgfpathrectangle{\pgfqpoint{1.254980in}{0.150000in}}{\pgfqpoint{5.490039in}{5.490039in}}%
\pgfusepath{clip}%
\pgfsetbuttcap%
\pgfsetroundjoin%
\definecolor{currentfill}{rgb}{0.258965,0.251537,0.524736}%
\pgfsetfillcolor{currentfill}%
\pgfsetfillopacity{0.700000}%
\pgfsetlinewidth{0.000000pt}%
\definecolor{currentstroke}{rgb}{0.000000,0.000000,0.000000}%
\pgfsetstrokecolor{currentstroke}%
\pgfsetdash{}{0pt}%
\pgfpathmoveto{\pgfqpoint{3.074286in}{2.688330in}}%
\pgfpathlineto{\pgfqpoint{3.087267in}{2.674425in}}%
\pgfpathlineto{\pgfqpoint{3.100244in}{2.660773in}}%
\pgfpathlineto{\pgfqpoint{3.113216in}{2.647372in}}%
\pgfpathlineto{\pgfqpoint{3.126185in}{2.634221in}}%
\pgfpathlineto{\pgfqpoint{3.133988in}{2.644178in}}%
\pgfpathlineto{\pgfqpoint{3.141785in}{2.654231in}}%
\pgfpathlineto{\pgfqpoint{3.149574in}{2.664383in}}%
\pgfpathlineto{\pgfqpoint{3.157357in}{2.674633in}}%
\pgfpathlineto{\pgfqpoint{3.144403in}{2.687811in}}%
\pgfpathlineto{\pgfqpoint{3.131445in}{2.701238in}}%
\pgfpathlineto{\pgfqpoint{3.118482in}{2.714916in}}%
\pgfpathlineto{\pgfqpoint{3.105516in}{2.728848in}}%
\pgfpathlineto{\pgfqpoint{3.097719in}{2.718561in}}%
\pgfpathlineto{\pgfqpoint{3.089915in}{2.708379in}}%
\pgfpathlineto{\pgfqpoint{3.082104in}{2.698303in}}%
\pgfpathlineto{\pgfqpoint{3.074286in}{2.688330in}}%
\pgfpathclose%
\pgfusepath{fill}%
\end{pgfscope}%
\begin{pgfscope}%
\pgfpathrectangle{\pgfqpoint{1.254980in}{0.150000in}}{\pgfqpoint{5.490039in}{5.490039in}}%
\pgfusepath{clip}%
\pgfsetbuttcap%
\pgfsetroundjoin%
\definecolor{currentfill}{rgb}{0.223925,0.334994,0.548053}%
\pgfsetfillcolor{currentfill}%
\pgfsetfillopacity{0.700000}%
\pgfsetlinewidth{0.000000pt}%
\definecolor{currentstroke}{rgb}{0.000000,0.000000,0.000000}%
\pgfsetstrokecolor{currentstroke}%
\pgfsetdash{}{0pt}%
\pgfpathmoveto{\pgfqpoint{2.918094in}{2.875778in}}%
\pgfpathlineto{\pgfqpoint{2.931145in}{2.858654in}}%
\pgfpathlineto{\pgfqpoint{2.944189in}{2.841812in}}%
\pgfpathlineto{\pgfqpoint{2.957226in}{2.825249in}}%
\pgfpathlineto{\pgfqpoint{2.970256in}{2.808962in}}%
\pgfpathlineto{\pgfqpoint{2.978105in}{2.818965in}}%
\pgfpathlineto{\pgfqpoint{2.985947in}{2.829087in}}%
\pgfpathlineto{\pgfqpoint{2.993781in}{2.839327in}}%
\pgfpathlineto{\pgfqpoint{3.001608in}{2.849687in}}%
\pgfpathlineto{\pgfqpoint{2.988594in}{2.865999in}}%
\pgfpathlineto{\pgfqpoint{2.975573in}{2.882588in}}%
\pgfpathlineto{\pgfqpoint{2.962545in}{2.899456in}}%
\pgfpathlineto{\pgfqpoint{2.949510in}{2.916605in}}%
\pgfpathlineto{\pgfqpoint{2.941668in}{2.906209in}}%
\pgfpathlineto{\pgfqpoint{2.933818in}{2.895940in}}%
\pgfpathlineto{\pgfqpoint{2.925960in}{2.885797in}}%
\pgfpathlineto{\pgfqpoint{2.918094in}{2.875778in}}%
\pgfpathclose%
\pgfusepath{fill}%
\end{pgfscope}%
\begin{pgfscope}%
\pgfpathrectangle{\pgfqpoint{1.254980in}{0.150000in}}{\pgfqpoint{5.490039in}{5.490039in}}%
\pgfusepath{clip}%
\pgfsetbuttcap%
\pgfsetroundjoin%
\definecolor{currentfill}{rgb}{0.168126,0.459988,0.558082}%
\pgfsetfillcolor{currentfill}%
\pgfsetfillopacity{0.700000}%
\pgfsetlinewidth{0.000000pt}%
\definecolor{currentstroke}{rgb}{0.000000,0.000000,0.000000}%
\pgfsetstrokecolor{currentstroke}%
\pgfsetdash{}{0pt}%
\pgfpathmoveto{\pgfqpoint{5.294454in}{3.144757in}}%
\pgfpathlineto{\pgfqpoint{5.307839in}{3.146089in}}%
\pgfpathlineto{\pgfqpoint{5.321237in}{3.147572in}}%
\pgfpathlineto{\pgfqpoint{5.334647in}{3.149205in}}%
\pgfpathlineto{\pgfqpoint{5.348069in}{3.150988in}}%
\pgfpathlineto{\pgfqpoint{5.355196in}{3.161515in}}%
\pgfpathlineto{\pgfqpoint{5.362325in}{3.172278in}}%
\pgfpathlineto{\pgfqpoint{5.369457in}{3.183285in}}%
\pgfpathlineto{\pgfqpoint{5.376590in}{3.194542in}}%
\pgfpathlineto{\pgfqpoint{5.363191in}{3.193400in}}%
\pgfpathlineto{\pgfqpoint{5.349803in}{3.192407in}}%
\pgfpathlineto{\pgfqpoint{5.336428in}{3.191565in}}%
\pgfpathlineto{\pgfqpoint{5.323065in}{3.190872in}}%
\pgfpathlineto{\pgfqpoint{5.315909in}{3.178964in}}%
\pgfpathlineto{\pgfqpoint{5.308755in}{3.167314in}}%
\pgfpathlineto{\pgfqpoint{5.301604in}{3.155914in}}%
\pgfpathlineto{\pgfqpoint{5.294454in}{3.144757in}}%
\pgfpathclose%
\pgfusepath{fill}%
\end{pgfscope}%
\begin{pgfscope}%
\pgfpathrectangle{\pgfqpoint{1.254980in}{0.150000in}}{\pgfqpoint{5.490039in}{5.490039in}}%
\pgfusepath{clip}%
\pgfsetbuttcap%
\pgfsetroundjoin%
\definecolor{currentfill}{rgb}{0.265145,0.232956,0.516599}%
\pgfsetfillcolor{currentfill}%
\pgfsetfillopacity{0.700000}%
\pgfsetlinewidth{0.000000pt}%
\definecolor{currentstroke}{rgb}{0.000000,0.000000,0.000000}%
\pgfsetstrokecolor{currentstroke}%
\pgfsetdash{}{0pt}%
\pgfpathmoveto{\pgfqpoint{4.310630in}{2.626206in}}%
\pgfpathlineto{\pgfqpoint{4.323697in}{2.625403in}}%
\pgfpathlineto{\pgfqpoint{4.336771in}{2.624769in}}%
\pgfpathlineto{\pgfqpoint{4.349854in}{2.624305in}}%
\pgfpathlineto{\pgfqpoint{4.362945in}{2.624009in}}%
\pgfpathlineto{\pgfqpoint{4.370363in}{2.633841in}}%
\pgfpathlineto{\pgfqpoint{4.377777in}{2.643732in}}%
\pgfpathlineto{\pgfqpoint{4.385187in}{2.653685in}}%
\pgfpathlineto{\pgfqpoint{4.392593in}{2.663705in}}%
\pgfpathlineto{\pgfqpoint{4.379512in}{2.664306in}}%
\pgfpathlineto{\pgfqpoint{4.366439in}{2.665077in}}%
\pgfpathlineto{\pgfqpoint{4.353374in}{2.666016in}}%
\pgfpathlineto{\pgfqpoint{4.340317in}{2.667125in}}%
\pgfpathlineto{\pgfqpoint{4.332902in}{2.656789in}}%
\pgfpathlineto{\pgfqpoint{4.325482in}{2.646527in}}%
\pgfpathlineto{\pgfqpoint{4.318058in}{2.636334in}}%
\pgfpathlineto{\pgfqpoint{4.310630in}{2.626206in}}%
\pgfpathclose%
\pgfusepath{fill}%
\end{pgfscope}%
\begin{pgfscope}%
\pgfpathrectangle{\pgfqpoint{1.254980in}{0.150000in}}{\pgfqpoint{5.490039in}{5.490039in}}%
\pgfusepath{clip}%
\pgfsetbuttcap%
\pgfsetroundjoin%
\definecolor{currentfill}{rgb}{0.281887,0.150881,0.465405}%
\pgfsetfillcolor{currentfill}%
\pgfsetfillopacity{0.700000}%
\pgfsetlinewidth{0.000000pt}%
\definecolor{currentstroke}{rgb}{0.000000,0.000000,0.000000}%
\pgfsetstrokecolor{currentstroke}%
\pgfsetdash{}{0pt}%
\pgfpathmoveto{\pgfqpoint{3.498484in}{2.469141in}}%
\pgfpathlineto{\pgfqpoint{3.511402in}{2.461575in}}%
\pgfpathlineto{\pgfqpoint{3.524322in}{2.454217in}}%
\pgfpathlineto{\pgfqpoint{3.537243in}{2.447065in}}%
\pgfpathlineto{\pgfqpoint{3.550166in}{2.440119in}}%
\pgfpathlineto{\pgfqpoint{3.557837in}{2.450368in}}%
\pgfpathlineto{\pgfqpoint{3.565502in}{2.460674in}}%
\pgfpathlineto{\pgfqpoint{3.573162in}{2.471037in}}%
\pgfpathlineto{\pgfqpoint{3.580816in}{2.481460in}}%
\pgfpathlineto{\pgfqpoint{3.567903in}{2.488491in}}%
\pgfpathlineto{\pgfqpoint{3.554992in}{2.495726in}}%
\pgfpathlineto{\pgfqpoint{3.542082in}{2.503168in}}%
\pgfpathlineto{\pgfqpoint{3.529174in}{2.510817in}}%
\pgfpathlineto{\pgfqpoint{3.521510in}{2.500300in}}%
\pgfpathlineto{\pgfqpoint{3.513840in}{2.489850in}}%
\pgfpathlineto{\pgfqpoint{3.506164in}{2.479464in}}%
\pgfpathlineto{\pgfqpoint{3.498484in}{2.469141in}}%
\pgfpathclose%
\pgfusepath{fill}%
\end{pgfscope}%
\begin{pgfscope}%
\pgfpathrectangle{\pgfqpoint{1.254980in}{0.150000in}}{\pgfqpoint{5.490039in}{5.490039in}}%
\pgfusepath{clip}%
\pgfsetbuttcap%
\pgfsetroundjoin%
\definecolor{currentfill}{rgb}{0.266580,0.228262,0.514349}%
\pgfsetfillcolor{currentfill}%
\pgfsetfillopacity{0.700000}%
\pgfsetlinewidth{0.000000pt}%
\definecolor{currentstroke}{rgb}{0.000000,0.000000,0.000000}%
\pgfsetstrokecolor{currentstroke}%
\pgfsetdash{}{0pt}%
\pgfpathmoveto{\pgfqpoint{3.126185in}{2.634221in}}%
\pgfpathlineto{\pgfqpoint{3.139150in}{2.621318in}}%
\pgfpathlineto{\pgfqpoint{3.152112in}{2.608659in}}%
\pgfpathlineto{\pgfqpoint{3.165071in}{2.596245in}}%
\pgfpathlineto{\pgfqpoint{3.178026in}{2.584072in}}%
\pgfpathlineto{\pgfqpoint{3.185815in}{2.594012in}}%
\pgfpathlineto{\pgfqpoint{3.193598in}{2.604042in}}%
\pgfpathlineto{\pgfqpoint{3.201374in}{2.614163in}}%
\pgfpathlineto{\pgfqpoint{3.209143in}{2.624376in}}%
\pgfpathlineto{\pgfqpoint{3.196201in}{2.636576in}}%
\pgfpathlineto{\pgfqpoint{3.183256in}{2.649018in}}%
\pgfpathlineto{\pgfqpoint{3.170309in}{2.661703in}}%
\pgfpathlineto{\pgfqpoint{3.157357in}{2.674633in}}%
\pgfpathlineto{\pgfqpoint{3.149574in}{2.664383in}}%
\pgfpathlineto{\pgfqpoint{3.141785in}{2.654231in}}%
\pgfpathlineto{\pgfqpoint{3.133988in}{2.644178in}}%
\pgfpathlineto{\pgfqpoint{3.126185in}{2.634221in}}%
\pgfpathclose%
\pgfusepath{fill}%
\end{pgfscope}%
\begin{pgfscope}%
\pgfpathrectangle{\pgfqpoint{1.254980in}{0.150000in}}{\pgfqpoint{5.490039in}{5.490039in}}%
\pgfusepath{clip}%
\pgfsetbuttcap%
\pgfsetroundjoin%
\definecolor{currentfill}{rgb}{0.281412,0.155834,0.469201}%
\pgfsetfillcolor{currentfill}%
\pgfsetfillopacity{0.700000}%
\pgfsetlinewidth{0.000000pt}%
\definecolor{currentstroke}{rgb}{0.000000,0.000000,0.000000}%
\pgfsetstrokecolor{currentstroke}%
\pgfsetdash{}{0pt}%
\pgfpathmoveto{\pgfqpoint{3.848580in}{2.477649in}}%
\pgfpathlineto{\pgfqpoint{3.861534in}{2.473687in}}%
\pgfpathlineto{\pgfqpoint{3.874492in}{2.469912in}}%
\pgfpathlineto{\pgfqpoint{3.887456in}{2.466323in}}%
\pgfpathlineto{\pgfqpoint{3.900425in}{2.462919in}}%
\pgfpathlineto{\pgfqpoint{3.907988in}{2.473168in}}%
\pgfpathlineto{\pgfqpoint{3.915547in}{2.483460in}}%
\pgfpathlineto{\pgfqpoint{3.923101in}{2.493798in}}%
\pgfpathlineto{\pgfqpoint{3.930650in}{2.504182in}}%
\pgfpathlineto{\pgfqpoint{3.917690in}{2.507753in}}%
\pgfpathlineto{\pgfqpoint{3.904735in}{2.511510in}}%
\pgfpathlineto{\pgfqpoint{3.891785in}{2.515451in}}%
\pgfpathlineto{\pgfqpoint{3.878840in}{2.519580in}}%
\pgfpathlineto{\pgfqpoint{3.871282in}{2.509018in}}%
\pgfpathlineto{\pgfqpoint{3.863720in}{2.498511in}}%
\pgfpathlineto{\pgfqpoint{3.856152in}{2.488055in}}%
\pgfpathlineto{\pgfqpoint{3.848580in}{2.477649in}}%
\pgfpathclose%
\pgfusepath{fill}%
\end{pgfscope}%
\begin{pgfscope}%
\pgfpathrectangle{\pgfqpoint{1.254980in}{0.150000in}}{\pgfqpoint{5.490039in}{5.490039in}}%
\pgfusepath{clip}%
\pgfsetbuttcap%
\pgfsetroundjoin%
\definecolor{currentfill}{rgb}{0.210503,0.363727,0.552206}%
\pgfsetfillcolor{currentfill}%
\pgfsetfillopacity{0.700000}%
\pgfsetlinewidth{0.000000pt}%
\definecolor{currentstroke}{rgb}{0.000000,0.000000,0.000000}%
\pgfsetstrokecolor{currentstroke}%
\pgfsetdash{}{0pt}%
\pgfpathmoveto{\pgfqpoint{2.865812in}{2.947144in}}%
\pgfpathlineto{\pgfqpoint{2.878894in}{2.928866in}}%
\pgfpathlineto{\pgfqpoint{2.891969in}{2.910882in}}%
\pgfpathlineto{\pgfqpoint{2.905035in}{2.893187in}}%
\pgfpathlineto{\pgfqpoint{2.918094in}{2.875778in}}%
\pgfpathlineto{\pgfqpoint{2.925960in}{2.885797in}}%
\pgfpathlineto{\pgfqpoint{2.933818in}{2.895940in}}%
\pgfpathlineto{\pgfqpoint{2.941668in}{2.906209in}}%
\pgfpathlineto{\pgfqpoint{2.949510in}{2.916605in}}%
\pgfpathlineto{\pgfqpoint{2.936468in}{2.934039in}}%
\pgfpathlineto{\pgfqpoint{2.923418in}{2.951759in}}%
\pgfpathlineto{\pgfqpoint{2.910361in}{2.969769in}}%
\pgfpathlineto{\pgfqpoint{2.897295in}{2.988071in}}%
\pgfpathlineto{\pgfqpoint{2.889436in}{2.977639in}}%
\pgfpathlineto{\pgfqpoint{2.881569in}{2.967341in}}%
\pgfpathlineto{\pgfqpoint{2.873695in}{2.957176in}}%
\pgfpathlineto{\pgfqpoint{2.865812in}{2.947144in}}%
\pgfpathclose%
\pgfusepath{fill}%
\end{pgfscope}%
\begin{pgfscope}%
\pgfpathrectangle{\pgfqpoint{1.254980in}{0.150000in}}{\pgfqpoint{5.490039in}{5.490039in}}%
\pgfusepath{clip}%
\pgfsetbuttcap%
\pgfsetroundjoin%
\definecolor{currentfill}{rgb}{0.282290,0.145912,0.461510}%
\pgfsetfillcolor{currentfill}%
\pgfsetfillopacity{0.700000}%
\pgfsetlinewidth{0.000000pt}%
\definecolor{currentstroke}{rgb}{0.000000,0.000000,0.000000}%
\pgfsetstrokecolor{currentstroke}%
\pgfsetdash{}{0pt}%
\pgfpathmoveto{\pgfqpoint{3.632492in}{2.455365in}}%
\pgfpathlineto{\pgfqpoint{3.645417in}{2.449342in}}%
\pgfpathlineto{\pgfqpoint{3.658345in}{2.443518in}}%
\pgfpathlineto{\pgfqpoint{3.671277in}{2.437890in}}%
\pgfpathlineto{\pgfqpoint{3.684212in}{2.432459in}}%
\pgfpathlineto{\pgfqpoint{3.691842in}{2.442741in}}%
\pgfpathlineto{\pgfqpoint{3.699466in}{2.453071in}}%
\pgfpathlineto{\pgfqpoint{3.707086in}{2.463451in}}%
\pgfpathlineto{\pgfqpoint{3.714701in}{2.473884in}}%
\pgfpathlineto{\pgfqpoint{3.701775in}{2.479427in}}%
\pgfpathlineto{\pgfqpoint{3.688853in}{2.485165in}}%
\pgfpathlineto{\pgfqpoint{3.675934in}{2.491101in}}%
\pgfpathlineto{\pgfqpoint{3.663018in}{2.497236in}}%
\pgfpathlineto{\pgfqpoint{3.655394in}{2.486682in}}%
\pgfpathlineto{\pgfqpoint{3.647765in}{2.476186in}}%
\pgfpathlineto{\pgfqpoint{3.640131in}{2.465748in}}%
\pgfpathlineto{\pgfqpoint{3.632492in}{2.455365in}}%
\pgfpathclose%
\pgfusepath{fill}%
\end{pgfscope}%
\begin{pgfscope}%
\pgfpathrectangle{\pgfqpoint{1.254980in}{0.150000in}}{\pgfqpoint{5.490039in}{5.490039in}}%
\pgfusepath{clip}%
\pgfsetbuttcap%
\pgfsetroundjoin%
\definecolor{currentfill}{rgb}{0.269308,0.218818,0.509577}%
\pgfsetfillcolor{currentfill}%
\pgfsetfillopacity{0.700000}%
\pgfsetlinewidth{0.000000pt}%
\definecolor{currentstroke}{rgb}{0.000000,0.000000,0.000000}%
\pgfsetstrokecolor{currentstroke}%
\pgfsetdash{}{0pt}%
\pgfpathmoveto{\pgfqpoint{4.228651in}{2.590095in}}%
\pgfpathlineto{\pgfqpoint{4.241696in}{2.588885in}}%
\pgfpathlineto{\pgfqpoint{4.254748in}{2.587846in}}%
\pgfpathlineto{\pgfqpoint{4.267807in}{2.586980in}}%
\pgfpathlineto{\pgfqpoint{4.280875in}{2.586284in}}%
\pgfpathlineto{\pgfqpoint{4.288320in}{2.596183in}}%
\pgfpathlineto{\pgfqpoint{4.295761in}{2.606134in}}%
\pgfpathlineto{\pgfqpoint{4.303198in}{2.616141in}}%
\pgfpathlineto{\pgfqpoint{4.310630in}{2.626206in}}%
\pgfpathlineto{\pgfqpoint{4.297572in}{2.627180in}}%
\pgfpathlineto{\pgfqpoint{4.284522in}{2.628325in}}%
\pgfpathlineto{\pgfqpoint{4.271479in}{2.629641in}}%
\pgfpathlineto{\pgfqpoint{4.258444in}{2.631129in}}%
\pgfpathlineto{\pgfqpoint{4.251002in}{2.620776in}}%
\pgfpathlineto{\pgfqpoint{4.243556in}{2.610488in}}%
\pgfpathlineto{\pgfqpoint{4.236106in}{2.600262in}}%
\pgfpathlineto{\pgfqpoint{4.228651in}{2.590095in}}%
\pgfpathclose%
\pgfusepath{fill}%
\end{pgfscope}%
\begin{pgfscope}%
\pgfpathrectangle{\pgfqpoint{1.254980in}{0.150000in}}{\pgfqpoint{5.490039in}{5.490039in}}%
\pgfusepath{clip}%
\pgfsetbuttcap%
\pgfsetroundjoin%
\definecolor{currentfill}{rgb}{0.280868,0.160771,0.472899}%
\pgfsetfillcolor{currentfill}%
\pgfsetfillopacity{0.700000}%
\pgfsetlinewidth{0.000000pt}%
\definecolor{currentstroke}{rgb}{0.000000,0.000000,0.000000}%
\pgfsetstrokecolor{currentstroke}%
\pgfsetdash{}{0pt}%
\pgfpathmoveto{\pgfqpoint{3.364303in}{2.496169in}}%
\pgfpathlineto{\pgfqpoint{3.377227in}{2.486951in}}%
\pgfpathlineto{\pgfqpoint{3.390151in}{2.477952in}}%
\pgfpathlineto{\pgfqpoint{3.403075in}{2.469170in}}%
\pgfpathlineto{\pgfqpoint{3.416000in}{2.460603in}}%
\pgfpathlineto{\pgfqpoint{3.423714in}{2.470728in}}%
\pgfpathlineto{\pgfqpoint{3.431422in}{2.480920in}}%
\pgfpathlineto{\pgfqpoint{3.439125in}{2.491178in}}%
\pgfpathlineto{\pgfqpoint{3.446822in}{2.501505in}}%
\pgfpathlineto{\pgfqpoint{3.433909in}{2.510127in}}%
\pgfpathlineto{\pgfqpoint{3.420996in}{2.518966in}}%
\pgfpathlineto{\pgfqpoint{3.408084in}{2.528021in}}%
\pgfpathlineto{\pgfqpoint{3.395171in}{2.537294in}}%
\pgfpathlineto{\pgfqpoint{3.387463in}{2.526902in}}%
\pgfpathlineto{\pgfqpoint{3.379749in}{2.516584in}}%
\pgfpathlineto{\pgfqpoint{3.372029in}{2.506340in}}%
\pgfpathlineto{\pgfqpoint{3.364303in}{2.496169in}}%
\pgfpathclose%
\pgfusepath{fill}%
\end{pgfscope}%
\begin{pgfscope}%
\pgfpathrectangle{\pgfqpoint{1.254980in}{0.150000in}}{\pgfqpoint{5.490039in}{5.490039in}}%
\pgfusepath{clip}%
\pgfsetbuttcap%
\pgfsetroundjoin%
\definecolor{currentfill}{rgb}{0.160665,0.478540,0.558115}%
\pgfsetfillcolor{currentfill}%
\pgfsetfillopacity{0.700000}%
\pgfsetlinewidth{0.000000pt}%
\definecolor{currentstroke}{rgb}{0.000000,0.000000,0.000000}%
\pgfsetstrokecolor{currentstroke}%
\pgfsetdash{}{0pt}%
\pgfpathmoveto{\pgfqpoint{5.376590in}{3.194542in}}%
\pgfpathlineto{\pgfqpoint{5.390002in}{3.195835in}}%
\pgfpathlineto{\pgfqpoint{5.403426in}{3.197276in}}%
\pgfpathlineto{\pgfqpoint{5.416862in}{3.198868in}}%
\pgfpathlineto{\pgfqpoint{5.430311in}{3.200608in}}%
\pgfpathlineto{\pgfqpoint{5.437423in}{3.211466in}}%
\pgfpathlineto{\pgfqpoint{5.444539in}{3.222583in}}%
\pgfpathlineto{\pgfqpoint{5.451657in}{3.233967in}}%
\pgfpathlineto{\pgfqpoint{5.438227in}{3.232726in}}%
\pgfpathlineto{\pgfqpoint{5.424808in}{3.231634in}}%
\pgfpathlineto{\pgfqpoint{5.411402in}{3.230690in}}%
\pgfpathlineto{\pgfqpoint{5.398008in}{3.229896in}}%
\pgfpathlineto{\pgfqpoint{5.390866in}{3.217841in}}%
\pgfpathlineto{\pgfqpoint{5.383726in}{3.206059in}}%
\pgfpathlineto{\pgfqpoint{5.376590in}{3.194542in}}%
\pgfpathclose%
\pgfusepath{fill}%
\end{pgfscope}%
\begin{pgfscope}%
\pgfpathrectangle{\pgfqpoint{1.254980in}{0.150000in}}{\pgfqpoint{5.490039in}{5.490039in}}%
\pgfusepath{clip}%
\pgfsetbuttcap%
\pgfsetroundjoin%
\definecolor{currentfill}{rgb}{0.273006,0.204520,0.501721}%
\pgfsetfillcolor{currentfill}%
\pgfsetfillopacity{0.700000}%
\pgfsetlinewidth{0.000000pt}%
\definecolor{currentstroke}{rgb}{0.000000,0.000000,0.000000}%
\pgfsetstrokecolor{currentstroke}%
\pgfsetdash{}{0pt}%
\pgfpathmoveto{\pgfqpoint{3.178026in}{2.584072in}}%
\pgfpathlineto{\pgfqpoint{3.190979in}{2.572138in}}%
\pgfpathlineto{\pgfqpoint{3.203930in}{2.560443in}}%
\pgfpathlineto{\pgfqpoint{3.216878in}{2.548983in}}%
\pgfpathlineto{\pgfqpoint{3.229824in}{2.537759in}}%
\pgfpathlineto{\pgfqpoint{3.237599in}{2.547682in}}%
\pgfpathlineto{\pgfqpoint{3.245368in}{2.557688in}}%
\pgfpathlineto{\pgfqpoint{3.253131in}{2.567779in}}%
\pgfpathlineto{\pgfqpoint{3.260887in}{2.577954in}}%
\pgfpathlineto{\pgfqpoint{3.247954in}{2.589206in}}%
\pgfpathlineto{\pgfqpoint{3.235020in}{2.600693in}}%
\pgfpathlineto{\pgfqpoint{3.222083in}{2.612415in}}%
\pgfpathlineto{\pgfqpoint{3.209143in}{2.624376in}}%
\pgfpathlineto{\pgfqpoint{3.201374in}{2.614163in}}%
\pgfpathlineto{\pgfqpoint{3.193598in}{2.604042in}}%
\pgfpathlineto{\pgfqpoint{3.185815in}{2.594012in}}%
\pgfpathlineto{\pgfqpoint{3.178026in}{2.584072in}}%
\pgfpathclose%
\pgfusepath{fill}%
\end{pgfscope}%
\begin{pgfscope}%
\pgfpathrectangle{\pgfqpoint{1.254980in}{0.150000in}}{\pgfqpoint{5.490039in}{5.490039in}}%
\pgfusepath{clip}%
\pgfsetbuttcap%
\pgfsetroundjoin%
\definecolor{currentfill}{rgb}{0.274128,0.199721,0.498911}%
\pgfsetfillcolor{currentfill}%
\pgfsetfillopacity{0.700000}%
\pgfsetlinewidth{0.000000pt}%
\definecolor{currentstroke}{rgb}{0.000000,0.000000,0.000000}%
\pgfsetstrokecolor{currentstroke}%
\pgfsetdash{}{0pt}%
\pgfpathmoveto{\pgfqpoint{4.146650in}{2.555525in}}%
\pgfpathlineto{\pgfqpoint{4.159673in}{2.553870in}}%
\pgfpathlineto{\pgfqpoint{4.172704in}{2.552390in}}%
\pgfpathlineto{\pgfqpoint{4.185742in}{2.551084in}}%
\pgfpathlineto{\pgfqpoint{4.198788in}{2.549951in}}%
\pgfpathlineto{\pgfqpoint{4.206261in}{2.559915in}}%
\pgfpathlineto{\pgfqpoint{4.213729in}{2.569924in}}%
\pgfpathlineto{\pgfqpoint{4.221192in}{2.579984in}}%
\pgfpathlineto{\pgfqpoint{4.228651in}{2.590095in}}%
\pgfpathlineto{\pgfqpoint{4.215615in}{2.591478in}}%
\pgfpathlineto{\pgfqpoint{4.202586in}{2.593035in}}%
\pgfpathlineto{\pgfqpoint{4.189564in}{2.594765in}}%
\pgfpathlineto{\pgfqpoint{4.176549in}{2.596670in}}%
\pgfpathlineto{\pgfqpoint{4.169081in}{2.586298in}}%
\pgfpathlineto{\pgfqpoint{4.161608in}{2.575985in}}%
\pgfpathlineto{\pgfqpoint{4.154131in}{2.565728in}}%
\pgfpathlineto{\pgfqpoint{4.146650in}{2.555525in}}%
\pgfpathclose%
\pgfusepath{fill}%
\end{pgfscope}%
\begin{pgfscope}%
\pgfpathrectangle{\pgfqpoint{1.254980in}{0.150000in}}{\pgfqpoint{5.490039in}{5.490039in}}%
\pgfusepath{clip}%
\pgfsetbuttcap%
\pgfsetroundjoin%
\definecolor{currentfill}{rgb}{0.195860,0.395433,0.555276}%
\pgfsetfillcolor{currentfill}%
\pgfsetfillopacity{0.700000}%
\pgfsetlinewidth{0.000000pt}%
\definecolor{currentstroke}{rgb}{0.000000,0.000000,0.000000}%
\pgfsetstrokecolor{currentstroke}%
\pgfsetdash{}{0pt}%
\pgfpathmoveto{\pgfqpoint{2.813391in}{3.023233in}}%
\pgfpathlineto{\pgfqpoint{2.826510in}{3.003758in}}%
\pgfpathlineto{\pgfqpoint{2.839620in}{2.984586in}}%
\pgfpathlineto{\pgfqpoint{2.852720in}{2.965716in}}%
\pgfpathlineto{\pgfqpoint{2.865812in}{2.947144in}}%
\pgfpathlineto{\pgfqpoint{2.873695in}{2.957176in}}%
\pgfpathlineto{\pgfqpoint{2.881569in}{2.967341in}}%
\pgfpathlineto{\pgfqpoint{2.889436in}{2.977639in}}%
\pgfpathlineto{\pgfqpoint{2.897295in}{2.988071in}}%
\pgfpathlineto{\pgfqpoint{2.884221in}{3.006668in}}%
\pgfpathlineto{\pgfqpoint{2.871138in}{3.025563in}}%
\pgfpathlineto{\pgfqpoint{2.858046in}{3.044759in}}%
\pgfpathlineto{\pgfqpoint{2.844944in}{3.064259in}}%
\pgfpathlineto{\pgfqpoint{2.837069in}{3.053792in}}%
\pgfpathlineto{\pgfqpoint{2.829184in}{3.043466in}}%
\pgfpathlineto{\pgfqpoint{2.821292in}{3.033280in}}%
\pgfpathlineto{\pgfqpoint{2.813391in}{3.023233in}}%
\pgfpathclose%
\pgfusepath{fill}%
\end{pgfscope}%
\begin{pgfscope}%
\pgfpathrectangle{\pgfqpoint{1.254980in}{0.150000in}}{\pgfqpoint{5.490039in}{5.490039in}}%
\pgfusepath{clip}%
\pgfsetbuttcap%
\pgfsetroundjoin%
\definecolor{currentfill}{rgb}{0.282290,0.145912,0.461510}%
\pgfsetfillcolor{currentfill}%
\pgfsetfillopacity{0.700000}%
\pgfsetlinewidth{0.000000pt}%
\definecolor{currentstroke}{rgb}{0.000000,0.000000,0.000000}%
\pgfsetstrokecolor{currentstroke}%
\pgfsetdash{}{0pt}%
\pgfpathmoveto{\pgfqpoint{3.766439in}{2.453652in}}%
\pgfpathlineto{\pgfqpoint{3.779383in}{2.449075in}}%
\pgfpathlineto{\pgfqpoint{3.792332in}{2.444688in}}%
\pgfpathlineto{\pgfqpoint{3.805285in}{2.440489in}}%
\pgfpathlineto{\pgfqpoint{3.818243in}{2.436480in}}%
\pgfpathlineto{\pgfqpoint{3.825834in}{2.446708in}}%
\pgfpathlineto{\pgfqpoint{3.833421in}{2.456977in}}%
\pgfpathlineto{\pgfqpoint{3.841003in}{2.467290in}}%
\pgfpathlineto{\pgfqpoint{3.848580in}{2.477649in}}%
\pgfpathlineto{\pgfqpoint{3.835631in}{2.481798in}}%
\pgfpathlineto{\pgfqpoint{3.822687in}{2.486136in}}%
\pgfpathlineto{\pgfqpoint{3.809747in}{2.490662in}}%
\pgfpathlineto{\pgfqpoint{3.796812in}{2.495379in}}%
\pgfpathlineto{\pgfqpoint{3.789226in}{2.484871in}}%
\pgfpathlineto{\pgfqpoint{3.781635in}{2.474415in}}%
\pgfpathlineto{\pgfqpoint{3.774040in}{2.464009in}}%
\pgfpathlineto{\pgfqpoint{3.766439in}{2.453652in}}%
\pgfpathclose%
\pgfusepath{fill}%
\end{pgfscope}%
\begin{pgfscope}%
\pgfpathrectangle{\pgfqpoint{1.254980in}{0.150000in}}{\pgfqpoint{5.490039in}{5.490039in}}%
\pgfusepath{clip}%
\pgfsetbuttcap%
\pgfsetroundjoin%
\definecolor{currentfill}{rgb}{0.277134,0.185228,0.489898}%
\pgfsetfillcolor{currentfill}%
\pgfsetfillopacity{0.700000}%
\pgfsetlinewidth{0.000000pt}%
\definecolor{currentstroke}{rgb}{0.000000,0.000000,0.000000}%
\pgfsetstrokecolor{currentstroke}%
\pgfsetdash{}{0pt}%
\pgfpathmoveto{\pgfqpoint{4.064617in}{2.522671in}}%
\pgfpathlineto{\pgfqpoint{4.077622in}{2.520533in}}%
\pgfpathlineto{\pgfqpoint{4.090634in}{2.518573in}}%
\pgfpathlineto{\pgfqpoint{4.103652in}{2.516790in}}%
\pgfpathlineto{\pgfqpoint{4.116677in}{2.515183in}}%
\pgfpathlineto{\pgfqpoint{4.124177in}{2.525203in}}%
\pgfpathlineto{\pgfqpoint{4.131673in}{2.535265in}}%
\pgfpathlineto{\pgfqpoint{4.139163in}{2.545371in}}%
\pgfpathlineto{\pgfqpoint{4.146650in}{2.555525in}}%
\pgfpathlineto{\pgfqpoint{4.133633in}{2.557355in}}%
\pgfpathlineto{\pgfqpoint{4.120623in}{2.559361in}}%
\pgfpathlineto{\pgfqpoint{4.107620in}{2.561543in}}%
\pgfpathlineto{\pgfqpoint{4.094624in}{2.563903in}}%
\pgfpathlineto{\pgfqpoint{4.087129in}{2.553517in}}%
\pgfpathlineto{\pgfqpoint{4.079630in}{2.543184in}}%
\pgfpathlineto{\pgfqpoint{4.072126in}{2.532903in}}%
\pgfpathlineto{\pgfqpoint{4.064617in}{2.522671in}}%
\pgfpathclose%
\pgfusepath{fill}%
\end{pgfscope}%
\begin{pgfscope}%
\pgfpathrectangle{\pgfqpoint{1.254980in}{0.150000in}}{\pgfqpoint{5.490039in}{5.490039in}}%
\pgfusepath{clip}%
\pgfsetbuttcap%
\pgfsetroundjoin%
\definecolor{currentfill}{rgb}{0.277134,0.185228,0.489898}%
\pgfsetfillcolor{currentfill}%
\pgfsetfillopacity{0.700000}%
\pgfsetlinewidth{0.000000pt}%
\definecolor{currentstroke}{rgb}{0.000000,0.000000,0.000000}%
\pgfsetstrokecolor{currentstroke}%
\pgfsetdash{}{0pt}%
\pgfpathmoveto{\pgfqpoint{3.229824in}{2.537759in}}%
\pgfpathlineto{\pgfqpoint{3.242768in}{2.526766in}}%
\pgfpathlineto{\pgfqpoint{3.255710in}{2.516005in}}%
\pgfpathlineto{\pgfqpoint{3.268651in}{2.505473in}}%
\pgfpathlineto{\pgfqpoint{3.281591in}{2.495169in}}%
\pgfpathlineto{\pgfqpoint{3.289353in}{2.505074in}}%
\pgfpathlineto{\pgfqpoint{3.297109in}{2.515057in}}%
\pgfpathlineto{\pgfqpoint{3.304859in}{2.525116in}}%
\pgfpathlineto{\pgfqpoint{3.312603in}{2.535254in}}%
\pgfpathlineto{\pgfqpoint{3.299676in}{2.545586in}}%
\pgfpathlineto{\pgfqpoint{3.286748in}{2.556145in}}%
\pgfpathlineto{\pgfqpoint{3.273818in}{2.566934in}}%
\pgfpathlineto{\pgfqpoint{3.260887in}{2.577954in}}%
\pgfpathlineto{\pgfqpoint{3.253131in}{2.567779in}}%
\pgfpathlineto{\pgfqpoint{3.245368in}{2.557688in}}%
\pgfpathlineto{\pgfqpoint{3.237599in}{2.547682in}}%
\pgfpathlineto{\pgfqpoint{3.229824in}{2.537759in}}%
\pgfpathclose%
\pgfusepath{fill}%
\end{pgfscope}%
\begin{pgfscope}%
\pgfpathrectangle{\pgfqpoint{1.254980in}{0.150000in}}{\pgfqpoint{5.490039in}{5.490039in}}%
\pgfusepath{clip}%
\pgfsetbuttcap%
\pgfsetroundjoin%
\definecolor{currentfill}{rgb}{0.282623,0.140926,0.457517}%
\pgfsetfillcolor{currentfill}%
\pgfsetfillopacity{0.700000}%
\pgfsetlinewidth{0.000000pt}%
\definecolor{currentstroke}{rgb}{0.000000,0.000000,0.000000}%
\pgfsetstrokecolor{currentstroke}%
\pgfsetdash{}{0pt}%
\pgfpathmoveto{\pgfqpoint{3.550166in}{2.440119in}}%
\pgfpathlineto{\pgfqpoint{3.563092in}{2.433376in}}%
\pgfpathlineto{\pgfqpoint{3.576019in}{2.426836in}}%
\pgfpathlineto{\pgfqpoint{3.588949in}{2.420497in}}%
\pgfpathlineto{\pgfqpoint{3.601882in}{2.414359in}}%
\pgfpathlineto{\pgfqpoint{3.609542in}{2.424534in}}%
\pgfpathlineto{\pgfqpoint{3.617197in}{2.434760in}}%
\pgfpathlineto{\pgfqpoint{3.624847in}{2.445036in}}%
\pgfpathlineto{\pgfqpoint{3.632492in}{2.455365in}}%
\pgfpathlineto{\pgfqpoint{3.619569in}{2.461587in}}%
\pgfpathlineto{\pgfqpoint{3.606649in}{2.468010in}}%
\pgfpathlineto{\pgfqpoint{3.593731in}{2.474634in}}%
\pgfpathlineto{\pgfqpoint{3.580816in}{2.481460in}}%
\pgfpathlineto{\pgfqpoint{3.573162in}{2.471037in}}%
\pgfpathlineto{\pgfqpoint{3.565502in}{2.460674in}}%
\pgfpathlineto{\pgfqpoint{3.557837in}{2.450368in}}%
\pgfpathlineto{\pgfqpoint{3.550166in}{2.440119in}}%
\pgfpathclose%
\pgfusepath{fill}%
\end{pgfscope}%
\begin{pgfscope}%
\pgfpathrectangle{\pgfqpoint{1.254980in}{0.150000in}}{\pgfqpoint{5.490039in}{5.490039in}}%
\pgfusepath{clip}%
\pgfsetbuttcap%
\pgfsetroundjoin%
\definecolor{currentfill}{rgb}{0.281887,0.150881,0.465405}%
\pgfsetfillcolor{currentfill}%
\pgfsetfillopacity{0.700000}%
\pgfsetlinewidth{0.000000pt}%
\definecolor{currentstroke}{rgb}{0.000000,0.000000,0.000000}%
\pgfsetstrokecolor{currentstroke}%
\pgfsetdash{}{0pt}%
\pgfpathmoveto{\pgfqpoint{3.416000in}{2.460603in}}%
\pgfpathlineto{\pgfqpoint{3.428925in}{2.452250in}}%
\pgfpathlineto{\pgfqpoint{3.441851in}{2.444110in}}%
\pgfpathlineto{\pgfqpoint{3.454777in}{2.436181in}}%
\pgfpathlineto{\pgfqpoint{3.467705in}{2.428463in}}%
\pgfpathlineto{\pgfqpoint{3.475408in}{2.438543in}}%
\pgfpathlineto{\pgfqpoint{3.483105in}{2.448682in}}%
\pgfpathlineto{\pgfqpoint{3.490797in}{2.458881in}}%
\pgfpathlineto{\pgfqpoint{3.498484in}{2.469141in}}%
\pgfpathlineto{\pgfqpoint{3.485567in}{2.476916in}}%
\pgfpathlineto{\pgfqpoint{3.472651in}{2.484900in}}%
\pgfpathlineto{\pgfqpoint{3.459736in}{2.493096in}}%
\pgfpathlineto{\pgfqpoint{3.446822in}{2.501505in}}%
\pgfpathlineto{\pgfqpoint{3.439125in}{2.491178in}}%
\pgfpathlineto{\pgfqpoint{3.431422in}{2.480920in}}%
\pgfpathlineto{\pgfqpoint{3.423714in}{2.470728in}}%
\pgfpathlineto{\pgfqpoint{3.416000in}{2.460603in}}%
\pgfpathclose%
\pgfusepath{fill}%
\end{pgfscope}%
\begin{pgfscope}%
\pgfpathrectangle{\pgfqpoint{1.254980in}{0.150000in}}{\pgfqpoint{5.490039in}{5.490039in}}%
\pgfusepath{clip}%
\pgfsetbuttcap%
\pgfsetroundjoin%
\definecolor{currentfill}{rgb}{0.279574,0.170599,0.479997}%
\pgfsetfillcolor{currentfill}%
\pgfsetfillopacity{0.700000}%
\pgfsetlinewidth{0.000000pt}%
\definecolor{currentstroke}{rgb}{0.000000,0.000000,0.000000}%
\pgfsetstrokecolor{currentstroke}%
\pgfsetdash{}{0pt}%
\pgfpathmoveto{\pgfqpoint{3.982546in}{2.491729in}}%
\pgfpathlineto{\pgfqpoint{3.995534in}{2.489070in}}%
\pgfpathlineto{\pgfqpoint{4.008528in}{2.486592in}}%
\pgfpathlineto{\pgfqpoint{4.021529in}{2.484292in}}%
\pgfpathlineto{\pgfqpoint{4.034536in}{2.482172in}}%
\pgfpathlineto{\pgfqpoint{4.042063in}{2.492237in}}%
\pgfpathlineto{\pgfqpoint{4.049586in}{2.502340in}}%
\pgfpathlineto{\pgfqpoint{4.057104in}{2.512484in}}%
\pgfpathlineto{\pgfqpoint{4.064617in}{2.522671in}}%
\pgfpathlineto{\pgfqpoint{4.051619in}{2.524986in}}%
\pgfpathlineto{\pgfqpoint{4.038627in}{2.527480in}}%
\pgfpathlineto{\pgfqpoint{4.025641in}{2.530154in}}%
\pgfpathlineto{\pgfqpoint{4.012661in}{2.533007in}}%
\pgfpathlineto{\pgfqpoint{4.005139in}{2.522615in}}%
\pgfpathlineto{\pgfqpoint{3.997613in}{2.512273in}}%
\pgfpathlineto{\pgfqpoint{3.990081in}{2.501979in}}%
\pgfpathlineto{\pgfqpoint{3.982546in}{2.491729in}}%
\pgfpathclose%
\pgfusepath{fill}%
\end{pgfscope}%
\begin{pgfscope}%
\pgfpathrectangle{\pgfqpoint{1.254980in}{0.150000in}}{\pgfqpoint{5.490039in}{5.490039in}}%
\pgfusepath{clip}%
\pgfsetbuttcap%
\pgfsetroundjoin%
\definecolor{currentfill}{rgb}{0.282623,0.140926,0.457517}%
\pgfsetfillcolor{currentfill}%
\pgfsetfillopacity{0.700000}%
\pgfsetlinewidth{0.000000pt}%
\definecolor{currentstroke}{rgb}{0.000000,0.000000,0.000000}%
\pgfsetstrokecolor{currentstroke}%
\pgfsetdash{}{0pt}%
\pgfpathmoveto{\pgfqpoint{3.684212in}{2.432459in}}%
\pgfpathlineto{\pgfqpoint{3.697150in}{2.427223in}}%
\pgfpathlineto{\pgfqpoint{3.710091in}{2.422181in}}%
\pgfpathlineto{\pgfqpoint{3.723037in}{2.417332in}}%
\pgfpathlineto{\pgfqpoint{3.735986in}{2.412675in}}%
\pgfpathlineto{\pgfqpoint{3.743607in}{2.422855in}}%
\pgfpathlineto{\pgfqpoint{3.751223in}{2.433077in}}%
\pgfpathlineto{\pgfqpoint{3.758833in}{2.443342in}}%
\pgfpathlineto{\pgfqpoint{3.766439in}{2.453652in}}%
\pgfpathlineto{\pgfqpoint{3.753499in}{2.458421in}}%
\pgfpathlineto{\pgfqpoint{3.740562in}{2.463382in}}%
\pgfpathlineto{\pgfqpoint{3.727630in}{2.468536in}}%
\pgfpathlineto{\pgfqpoint{3.714701in}{2.473884in}}%
\pgfpathlineto{\pgfqpoint{3.707086in}{2.463451in}}%
\pgfpathlineto{\pgfqpoint{3.699466in}{2.453071in}}%
\pgfpathlineto{\pgfqpoint{3.691842in}{2.442741in}}%
\pgfpathlineto{\pgfqpoint{3.684212in}{2.432459in}}%
\pgfpathclose%
\pgfusepath{fill}%
\end{pgfscope}%
\begin{pgfscope}%
\pgfpathrectangle{\pgfqpoint{1.254980in}{0.150000in}}{\pgfqpoint{5.490039in}{5.490039in}}%
\pgfusepath{clip}%
\pgfsetbuttcap%
\pgfsetroundjoin%
\definecolor{currentfill}{rgb}{0.223925,0.334994,0.548053}%
\pgfsetfillcolor{currentfill}%
\pgfsetfillopacity{0.700000}%
\pgfsetlinewidth{0.000000pt}%
\definecolor{currentstroke}{rgb}{0.000000,0.000000,0.000000}%
\pgfsetstrokecolor{currentstroke}%
\pgfsetdash{}{0pt}%
\pgfpathmoveto{\pgfqpoint{4.773218in}{2.828938in}}%
\pgfpathlineto{\pgfqpoint{4.786453in}{2.830264in}}%
\pgfpathlineto{\pgfqpoint{4.799699in}{2.831750in}}%
\pgfpathlineto{\pgfqpoint{4.812955in}{2.833394in}}%
\pgfpathlineto{\pgfqpoint{4.826222in}{2.835196in}}%
\pgfpathlineto{\pgfqpoint{4.833492in}{2.844375in}}%
\pgfpathlineto{\pgfqpoint{4.840759in}{2.853659in}}%
\pgfpathlineto{\pgfqpoint{4.848023in}{2.863053in}}%
\pgfpathlineto{\pgfqpoint{4.855284in}{2.872562in}}%
\pgfpathlineto{\pgfqpoint{4.842031in}{2.871207in}}%
\pgfpathlineto{\pgfqpoint{4.828789in}{2.870010in}}%
\pgfpathlineto{\pgfqpoint{4.815558in}{2.868971in}}%
\pgfpathlineto{\pgfqpoint{4.802337in}{2.868091in}}%
\pgfpathlineto{\pgfqpoint{4.795061in}{2.858125in}}%
\pgfpathlineto{\pgfqpoint{4.787783in}{2.848281in}}%
\pgfpathlineto{\pgfqpoint{4.780502in}{2.838554in}}%
\pgfpathlineto{\pgfqpoint{4.773218in}{2.828938in}}%
\pgfpathclose%
\pgfusepath{fill}%
\end{pgfscope}%
\begin{pgfscope}%
\pgfpathrectangle{\pgfqpoint{1.254980in}{0.150000in}}{\pgfqpoint{5.490039in}{5.490039in}}%
\pgfusepath{clip}%
\pgfsetbuttcap%
\pgfsetroundjoin%
\definecolor{currentfill}{rgb}{0.214298,0.355619,0.551184}%
\pgfsetfillcolor{currentfill}%
\pgfsetfillopacity{0.700000}%
\pgfsetlinewidth{0.000000pt}%
\definecolor{currentstroke}{rgb}{0.000000,0.000000,0.000000}%
\pgfsetstrokecolor{currentstroke}%
\pgfsetdash{}{0pt}%
\pgfpathmoveto{\pgfqpoint{4.855284in}{2.872562in}}%
\pgfpathlineto{\pgfqpoint{4.868548in}{2.874075in}}%
\pgfpathlineto{\pgfqpoint{4.881823in}{2.875746in}}%
\pgfpathlineto{\pgfqpoint{4.895108in}{2.877573in}}%
\pgfpathlineto{\pgfqpoint{4.908406in}{2.879558in}}%
\pgfpathlineto{\pgfqpoint{4.915649in}{2.888723in}}%
\pgfpathlineto{\pgfqpoint{4.922889in}{2.898008in}}%
\pgfpathlineto{\pgfqpoint{4.930127in}{2.907417in}}%
\pgfpathlineto{\pgfqpoint{4.937362in}{2.916956in}}%
\pgfpathlineto{\pgfqpoint{4.924081in}{2.915446in}}%
\pgfpathlineto{\pgfqpoint{4.910810in}{2.914093in}}%
\pgfpathlineto{\pgfqpoint{4.897551in}{2.912897in}}%
\pgfpathlineto{\pgfqpoint{4.884302in}{2.911858in}}%
\pgfpathlineto{\pgfqpoint{4.877051in}{2.901835in}}%
\pgfpathlineto{\pgfqpoint{4.869798in}{2.891948in}}%
\pgfpathlineto{\pgfqpoint{4.862542in}{2.882192in}}%
\pgfpathlineto{\pgfqpoint{4.855284in}{2.872562in}}%
\pgfpathclose%
\pgfusepath{fill}%
\end{pgfscope}%
\begin{pgfscope}%
\pgfpathrectangle{\pgfqpoint{1.254980in}{0.150000in}}{\pgfqpoint{5.490039in}{5.490039in}}%
\pgfusepath{clip}%
\pgfsetbuttcap%
\pgfsetroundjoin%
\definecolor{currentfill}{rgb}{0.231674,0.318106,0.544834}%
\pgfsetfillcolor{currentfill}%
\pgfsetfillopacity{0.700000}%
\pgfsetlinewidth{0.000000pt}%
\definecolor{currentstroke}{rgb}{0.000000,0.000000,0.000000}%
\pgfsetstrokecolor{currentstroke}%
\pgfsetdash{}{0pt}%
\pgfpathmoveto{\pgfqpoint{4.691160in}{2.786109in}}%
\pgfpathlineto{\pgfqpoint{4.704367in}{2.787215in}}%
\pgfpathlineto{\pgfqpoint{4.717584in}{2.788481in}}%
\pgfpathlineto{\pgfqpoint{4.730811in}{2.789907in}}%
\pgfpathlineto{\pgfqpoint{4.744049in}{2.791493in}}%
\pgfpathlineto{\pgfqpoint{4.751346in}{2.800712in}}%
\pgfpathlineto{\pgfqpoint{4.758640in}{2.810022in}}%
\pgfpathlineto{\pgfqpoint{4.765931in}{2.819429in}}%
\pgfpathlineto{\pgfqpoint{4.773218in}{2.828938in}}%
\pgfpathlineto{\pgfqpoint{4.759994in}{2.827771in}}%
\pgfpathlineto{\pgfqpoint{4.746780in}{2.826764in}}%
\pgfpathlineto{\pgfqpoint{4.733576in}{2.825916in}}%
\pgfpathlineto{\pgfqpoint{4.720383in}{2.825228in}}%
\pgfpathlineto{\pgfqpoint{4.713082in}{2.815291in}}%
\pgfpathlineto{\pgfqpoint{4.705778in}{2.805462in}}%
\pgfpathlineto{\pgfqpoint{4.698471in}{2.795736in}}%
\pgfpathlineto{\pgfqpoint{4.691160in}{2.786109in}}%
\pgfpathclose%
\pgfusepath{fill}%
\end{pgfscope}%
\begin{pgfscope}%
\pgfpathrectangle{\pgfqpoint{1.254980in}{0.150000in}}{\pgfqpoint{5.490039in}{5.490039in}}%
\pgfusepath{clip}%
\pgfsetbuttcap%
\pgfsetroundjoin%
\definecolor{currentfill}{rgb}{0.206756,0.371758,0.553117}%
\pgfsetfillcolor{currentfill}%
\pgfsetfillopacity{0.700000}%
\pgfsetlinewidth{0.000000pt}%
\definecolor{currentstroke}{rgb}{0.000000,0.000000,0.000000}%
\pgfsetstrokecolor{currentstroke}%
\pgfsetdash{}{0pt}%
\pgfpathmoveto{\pgfqpoint{4.937362in}{2.916956in}}%
\pgfpathlineto{\pgfqpoint{4.950655in}{2.918621in}}%
\pgfpathlineto{\pgfqpoint{4.963959in}{2.920443in}}%
\pgfpathlineto{\pgfqpoint{4.977274in}{2.922421in}}%
\pgfpathlineto{\pgfqpoint{4.990601in}{2.924554in}}%
\pgfpathlineto{\pgfqpoint{4.997818in}{2.933736in}}%
\pgfpathlineto{\pgfqpoint{5.005032in}{2.943053in}}%
\pgfpathlineto{\pgfqpoint{5.012245in}{2.952509in}}%
\pgfpathlineto{\pgfqpoint{5.019455in}{2.962112in}}%
\pgfpathlineto{\pgfqpoint{5.006145in}{2.960482in}}%
\pgfpathlineto{\pgfqpoint{4.992846in}{2.959007in}}%
\pgfpathlineto{\pgfqpoint{4.979559in}{2.957687in}}%
\pgfpathlineto{\pgfqpoint{4.966283in}{2.956523in}}%
\pgfpathlineto{\pgfqpoint{4.959055in}{2.946408in}}%
\pgfpathlineto{\pgfqpoint{4.951826in}{2.936446in}}%
\pgfpathlineto{\pgfqpoint{4.944595in}{2.926630in}}%
\pgfpathlineto{\pgfqpoint{4.937362in}{2.916956in}}%
\pgfpathclose%
\pgfusepath{fill}%
\end{pgfscope}%
\begin{pgfscope}%
\pgfpathrectangle{\pgfqpoint{1.254980in}{0.150000in}}{\pgfqpoint{5.490039in}{5.490039in}}%
\pgfusepath{clip}%
\pgfsetbuttcap%
\pgfsetroundjoin%
\definecolor{currentfill}{rgb}{0.241237,0.296485,0.539709}%
\pgfsetfillcolor{currentfill}%
\pgfsetfillopacity{0.700000}%
\pgfsetlinewidth{0.000000pt}%
\definecolor{currentstroke}{rgb}{0.000000,0.000000,0.000000}%
\pgfsetstrokecolor{currentstroke}%
\pgfsetdash{}{0pt}%
\pgfpathmoveto{\pgfqpoint{4.609108in}{2.744124in}}%
\pgfpathlineto{\pgfqpoint{4.622286in}{2.744974in}}%
\pgfpathlineto{\pgfqpoint{4.635475in}{2.745986in}}%
\pgfpathlineto{\pgfqpoint{4.648674in}{2.747160in}}%
\pgfpathlineto{\pgfqpoint{4.661882in}{2.748496in}}%
\pgfpathlineto{\pgfqpoint{4.669208in}{2.757774in}}%
\pgfpathlineto{\pgfqpoint{4.676529in}{2.767133in}}%
\pgfpathlineto{\pgfqpoint{4.683846in}{2.776577in}}%
\pgfpathlineto{\pgfqpoint{4.691160in}{2.786109in}}%
\pgfpathlineto{\pgfqpoint{4.677964in}{2.785165in}}%
\pgfpathlineto{\pgfqpoint{4.664778in}{2.784381in}}%
\pgfpathlineto{\pgfqpoint{4.651602in}{2.783759in}}%
\pgfpathlineto{\pgfqpoint{4.638435in}{2.783299in}}%
\pgfpathlineto{\pgfqpoint{4.631109in}{2.773366in}}%
\pgfpathlineto{\pgfqpoint{4.623779in}{2.763529in}}%
\pgfpathlineto{\pgfqpoint{4.616445in}{2.753783in}}%
\pgfpathlineto{\pgfqpoint{4.609108in}{2.744124in}}%
\pgfpathclose%
\pgfusepath{fill}%
\end{pgfscope}%
\begin{pgfscope}%
\pgfpathrectangle{\pgfqpoint{1.254980in}{0.150000in}}{\pgfqpoint{5.490039in}{5.490039in}}%
\pgfusepath{clip}%
\pgfsetbuttcap%
\pgfsetroundjoin%
\definecolor{currentfill}{rgb}{0.197636,0.391528,0.554969}%
\pgfsetfillcolor{currentfill}%
\pgfsetfillopacity{0.700000}%
\pgfsetlinewidth{0.000000pt}%
\definecolor{currentstroke}{rgb}{0.000000,0.000000,0.000000}%
\pgfsetstrokecolor{currentstroke}%
\pgfsetdash{}{0pt}%
\pgfpathmoveto{\pgfqpoint{5.019455in}{2.962112in}}%
\pgfpathlineto{\pgfqpoint{5.032777in}{2.963897in}}%
\pgfpathlineto{\pgfqpoint{5.046110in}{2.965837in}}%
\pgfpathlineto{\pgfqpoint{5.059455in}{2.967932in}}%
\pgfpathlineto{\pgfqpoint{5.072812in}{2.970180in}}%
\pgfpathlineto{\pgfqpoint{5.080003in}{2.979414in}}%
\pgfpathlineto{\pgfqpoint{5.087193in}{2.988799in}}%
\pgfpathlineto{\pgfqpoint{5.094381in}{2.998342in}}%
\pgfpathlineto{\pgfqpoint{5.101568in}{3.008047in}}%
\pgfpathlineto{\pgfqpoint{5.088229in}{3.006329in}}%
\pgfpathlineto{\pgfqpoint{5.074902in}{3.004765in}}%
\pgfpathlineto{\pgfqpoint{5.061587in}{3.003356in}}%
\pgfpathlineto{\pgfqpoint{5.048283in}{3.002100in}}%
\pgfpathlineto{\pgfqpoint{5.041078in}{2.991855in}}%
\pgfpathlineto{\pgfqpoint{5.033872in}{2.981779in}}%
\pgfpathlineto{\pgfqpoint{5.026664in}{2.971867in}}%
\pgfpathlineto{\pgfqpoint{5.019455in}{2.962112in}}%
\pgfpathclose%
\pgfusepath{fill}%
\end{pgfscope}%
\begin{pgfscope}%
\pgfpathrectangle{\pgfqpoint{1.254980in}{0.150000in}}{\pgfqpoint{5.490039in}{5.490039in}}%
\pgfusepath{clip}%
\pgfsetbuttcap%
\pgfsetroundjoin%
\definecolor{currentfill}{rgb}{0.280255,0.165693,0.476498}%
\pgfsetfillcolor{currentfill}%
\pgfsetfillopacity{0.700000}%
\pgfsetlinewidth{0.000000pt}%
\definecolor{currentstroke}{rgb}{0.000000,0.000000,0.000000}%
\pgfsetstrokecolor{currentstroke}%
\pgfsetdash{}{0pt}%
\pgfpathmoveto{\pgfqpoint{3.281591in}{2.495169in}}%
\pgfpathlineto{\pgfqpoint{3.294529in}{2.485090in}}%
\pgfpathlineto{\pgfqpoint{3.307467in}{2.475236in}}%
\pgfpathlineto{\pgfqpoint{3.320404in}{2.465605in}}%
\pgfpathlineto{\pgfqpoint{3.333340in}{2.456196in}}%
\pgfpathlineto{\pgfqpoint{3.341090in}{2.466084in}}%
\pgfpathlineto{\pgfqpoint{3.348834in}{2.476042in}}%
\pgfpathlineto{\pgfqpoint{3.356571in}{2.486070in}}%
\pgfpathlineto{\pgfqpoint{3.364303in}{2.496169in}}%
\pgfpathlineto{\pgfqpoint{3.351379in}{2.505607in}}%
\pgfpathlineto{\pgfqpoint{3.338454in}{2.515266in}}%
\pgfpathlineto{\pgfqpoint{3.325529in}{2.525148in}}%
\pgfpathlineto{\pgfqpoint{3.312603in}{2.535254in}}%
\pgfpathlineto{\pgfqpoint{3.304859in}{2.525116in}}%
\pgfpathlineto{\pgfqpoint{3.297109in}{2.515057in}}%
\pgfpathlineto{\pgfqpoint{3.289353in}{2.505074in}}%
\pgfpathlineto{\pgfqpoint{3.281591in}{2.495169in}}%
\pgfpathclose%
\pgfusepath{fill}%
\end{pgfscope}%
\begin{pgfscope}%
\pgfpathrectangle{\pgfqpoint{1.254980in}{0.150000in}}{\pgfqpoint{5.490039in}{5.490039in}}%
\pgfusepath{clip}%
\pgfsetbuttcap%
\pgfsetroundjoin%
\definecolor{currentfill}{rgb}{0.248629,0.278775,0.534556}%
\pgfsetfillcolor{currentfill}%
\pgfsetfillopacity{0.700000}%
\pgfsetlinewidth{0.000000pt}%
\definecolor{currentstroke}{rgb}{0.000000,0.000000,0.000000}%
\pgfsetstrokecolor{currentstroke}%
\pgfsetdash{}{0pt}%
\pgfpathmoveto{\pgfqpoint{4.527057in}{2.703049in}}%
\pgfpathlineto{\pgfqpoint{4.540208in}{2.703609in}}%
\pgfpathlineto{\pgfqpoint{4.553369in}{2.704333in}}%
\pgfpathlineto{\pgfqpoint{4.566540in}{2.705220in}}%
\pgfpathlineto{\pgfqpoint{4.579720in}{2.706271in}}%
\pgfpathlineto{\pgfqpoint{4.587073in}{2.715625in}}%
\pgfpathlineto{\pgfqpoint{4.594422in}{2.725049in}}%
\pgfpathlineto{\pgfqpoint{4.601767in}{2.734547in}}%
\pgfpathlineto{\pgfqpoint{4.609108in}{2.744124in}}%
\pgfpathlineto{\pgfqpoint{4.595939in}{2.743436in}}%
\pgfpathlineto{\pgfqpoint{4.582780in}{2.742911in}}%
\pgfpathlineto{\pgfqpoint{4.569630in}{2.742550in}}%
\pgfpathlineto{\pgfqpoint{4.556490in}{2.742352in}}%
\pgfpathlineto{\pgfqpoint{4.549138in}{2.732403in}}%
\pgfpathlineto{\pgfqpoint{4.541781in}{2.722539in}}%
\pgfpathlineto{\pgfqpoint{4.534421in}{2.712756in}}%
\pgfpathlineto{\pgfqpoint{4.527057in}{2.703049in}}%
\pgfpathclose%
\pgfusepath{fill}%
\end{pgfscope}%
\begin{pgfscope}%
\pgfpathrectangle{\pgfqpoint{1.254980in}{0.150000in}}{\pgfqpoint{5.490039in}{5.490039in}}%
\pgfusepath{clip}%
\pgfsetbuttcap%
\pgfsetroundjoin%
\definecolor{currentfill}{rgb}{0.188923,0.410910,0.556326}%
\pgfsetfillcolor{currentfill}%
\pgfsetfillopacity{0.700000}%
\pgfsetlinewidth{0.000000pt}%
\definecolor{currentstroke}{rgb}{0.000000,0.000000,0.000000}%
\pgfsetstrokecolor{currentstroke}%
\pgfsetdash{}{0pt}%
\pgfpathmoveto{\pgfqpoint{5.101568in}{3.008047in}}%
\pgfpathlineto{\pgfqpoint{5.114918in}{3.009918in}}%
\pgfpathlineto{\pgfqpoint{5.128280in}{3.011943in}}%
\pgfpathlineto{\pgfqpoint{5.141655in}{3.014121in}}%
\pgfpathlineto{\pgfqpoint{5.155041in}{3.016453in}}%
\pgfpathlineto{\pgfqpoint{5.162208in}{3.025779in}}%
\pgfpathlineto{\pgfqpoint{5.169374in}{3.035274in}}%
\pgfpathlineto{\pgfqpoint{5.176539in}{3.044944in}}%
\pgfpathlineto{\pgfqpoint{5.183704in}{3.054796in}}%
\pgfpathlineto{\pgfqpoint{5.170337in}{3.053024in}}%
\pgfpathlineto{\pgfqpoint{5.156982in}{3.051405in}}%
\pgfpathlineto{\pgfqpoint{5.143639in}{3.049938in}}%
\pgfpathlineto{\pgfqpoint{5.130307in}{3.048624in}}%
\pgfpathlineto{\pgfqpoint{5.123123in}{3.038204in}}%
\pgfpathlineto{\pgfqpoint{5.115939in}{3.027972in}}%
\pgfpathlineto{\pgfqpoint{5.108754in}{3.017922in}}%
\pgfpathlineto{\pgfqpoint{5.101568in}{3.008047in}}%
\pgfpathclose%
\pgfusepath{fill}%
\end{pgfscope}%
\begin{pgfscope}%
\pgfpathrectangle{\pgfqpoint{1.254980in}{0.150000in}}{\pgfqpoint{5.490039in}{5.490039in}}%
\pgfusepath{clip}%
\pgfsetbuttcap%
\pgfsetroundjoin%
\definecolor{currentfill}{rgb}{0.280868,0.160771,0.472899}%
\pgfsetfillcolor{currentfill}%
\pgfsetfillopacity{0.700000}%
\pgfsetlinewidth{0.000000pt}%
\definecolor{currentstroke}{rgb}{0.000000,0.000000,0.000000}%
\pgfsetstrokecolor{currentstroke}%
\pgfsetdash{}{0pt}%
\pgfpathmoveto{\pgfqpoint{3.900425in}{2.462919in}}%
\pgfpathlineto{\pgfqpoint{3.913399in}{2.459699in}}%
\pgfpathlineto{\pgfqpoint{3.926378in}{2.456662in}}%
\pgfpathlineto{\pgfqpoint{3.939363in}{2.453808in}}%
\pgfpathlineto{\pgfqpoint{3.952354in}{2.451135in}}%
\pgfpathlineto{\pgfqpoint{3.959909in}{2.461228in}}%
\pgfpathlineto{\pgfqpoint{3.967459in}{2.471356in}}%
\pgfpathlineto{\pgfqpoint{3.975005in}{2.481522in}}%
\pgfpathlineto{\pgfqpoint{3.982546in}{2.491729in}}%
\pgfpathlineto{\pgfqpoint{3.969563in}{2.494569in}}%
\pgfpathlineto{\pgfqpoint{3.956586in}{2.497591in}}%
\pgfpathlineto{\pgfqpoint{3.943615in}{2.500795in}}%
\pgfpathlineto{\pgfqpoint{3.930650in}{2.504182in}}%
\pgfpathlineto{\pgfqpoint{3.923101in}{2.493798in}}%
\pgfpathlineto{\pgfqpoint{3.915547in}{2.483460in}}%
\pgfpathlineto{\pgfqpoint{3.907988in}{2.473168in}}%
\pgfpathlineto{\pgfqpoint{3.900425in}{2.462919in}}%
\pgfpathclose%
\pgfusepath{fill}%
\end{pgfscope}%
\begin{pgfscope}%
\pgfpathrectangle{\pgfqpoint{1.254980in}{0.150000in}}{\pgfqpoint{5.490039in}{5.490039in}}%
\pgfusepath{clip}%
\pgfsetbuttcap%
\pgfsetroundjoin%
\definecolor{currentfill}{rgb}{0.180629,0.429975,0.557282}%
\pgfsetfillcolor{currentfill}%
\pgfsetfillopacity{0.700000}%
\pgfsetlinewidth{0.000000pt}%
\definecolor{currentstroke}{rgb}{0.000000,0.000000,0.000000}%
\pgfsetstrokecolor{currentstroke}%
\pgfsetdash{}{0pt}%
\pgfpathmoveto{\pgfqpoint{5.183704in}{3.054796in}}%
\pgfpathlineto{\pgfqpoint{5.197083in}{3.056721in}}%
\pgfpathlineto{\pgfqpoint{5.210474in}{3.058798in}}%
\pgfpathlineto{\pgfqpoint{5.223877in}{3.061027in}}%
\pgfpathlineto{\pgfqpoint{5.237293in}{3.063408in}}%
\pgfpathlineto{\pgfqpoint{5.244437in}{3.072872in}}%
\pgfpathlineto{\pgfqpoint{5.251581in}{3.082523in}}%
\pgfpathlineto{\pgfqpoint{5.258725in}{3.092370in}}%
\pgfpathlineto{\pgfqpoint{5.265869in}{3.102417in}}%
\pgfpathlineto{\pgfqpoint{5.252474in}{3.100623in}}%
\pgfpathlineto{\pgfqpoint{5.239092in}{3.098981in}}%
\pgfpathlineto{\pgfqpoint{5.225721in}{3.097490in}}%
\pgfpathlineto{\pgfqpoint{5.212362in}{3.096151in}}%
\pgfpathlineto{\pgfqpoint{5.205197in}{3.085507in}}%
\pgfpathlineto{\pgfqpoint{5.198033in}{3.075071in}}%
\pgfpathlineto{\pgfqpoint{5.190868in}{3.064836in}}%
\pgfpathlineto{\pgfqpoint{5.183704in}{3.054796in}}%
\pgfpathclose%
\pgfusepath{fill}%
\end{pgfscope}%
\begin{pgfscope}%
\pgfpathrectangle{\pgfqpoint{1.254980in}{0.150000in}}{\pgfqpoint{5.490039in}{5.490039in}}%
\pgfusepath{clip}%
\pgfsetbuttcap%
\pgfsetroundjoin%
\definecolor{currentfill}{rgb}{0.255645,0.260703,0.528312}%
\pgfsetfillcolor{currentfill}%
\pgfsetfillopacity{0.700000}%
\pgfsetlinewidth{0.000000pt}%
\definecolor{currentstroke}{rgb}{0.000000,0.000000,0.000000}%
\pgfsetstrokecolor{currentstroke}%
\pgfsetdash{}{0pt}%
\pgfpathmoveto{\pgfqpoint{4.445004in}{2.662974in}}%
\pgfpathlineto{\pgfqpoint{4.458129in}{2.663208in}}%
\pgfpathlineto{\pgfqpoint{4.471263in}{2.663609in}}%
\pgfpathlineto{\pgfqpoint{4.484407in}{2.664174in}}%
\pgfpathlineto{\pgfqpoint{4.497560in}{2.664905in}}%
\pgfpathlineto{\pgfqpoint{4.504940in}{2.674346in}}%
\pgfpathlineto{\pgfqpoint{4.512317in}{2.683848in}}%
\pgfpathlineto{\pgfqpoint{4.519689in}{2.693414in}}%
\pgfpathlineto{\pgfqpoint{4.527057in}{2.703049in}}%
\pgfpathlineto{\pgfqpoint{4.513915in}{2.702653in}}%
\pgfpathlineto{\pgfqpoint{4.500782in}{2.702423in}}%
\pgfpathlineto{\pgfqpoint{4.487659in}{2.702357in}}%
\pgfpathlineto{\pgfqpoint{4.474545in}{2.702457in}}%
\pgfpathlineto{\pgfqpoint{4.467166in}{2.692478in}}%
\pgfpathlineto{\pgfqpoint{4.459783in}{2.682574in}}%
\pgfpathlineto{\pgfqpoint{4.452395in}{2.672740in}}%
\pgfpathlineto{\pgfqpoint{4.445004in}{2.662974in}}%
\pgfpathclose%
\pgfusepath{fill}%
\end{pgfscope}%
\begin{pgfscope}%
\pgfpathrectangle{\pgfqpoint{1.254980in}{0.150000in}}{\pgfqpoint{5.490039in}{5.490039in}}%
\pgfusepath{clip}%
\pgfsetbuttcap%
\pgfsetroundjoin%
\definecolor{currentfill}{rgb}{0.252194,0.269783,0.531579}%
\pgfsetfillcolor{currentfill}%
\pgfsetfillopacity{0.700000}%
\pgfsetlinewidth{0.000000pt}%
\definecolor{currentstroke}{rgb}{0.000000,0.000000,0.000000}%
\pgfsetstrokecolor{currentstroke}%
\pgfsetdash{}{0pt}%
\pgfpathmoveto{\pgfqpoint{2.990908in}{2.707676in}}%
\pgfpathlineto{\pgfqpoint{3.003925in}{2.692733in}}%
\pgfpathlineto{\pgfqpoint{3.016936in}{2.678053in}}%
\pgfpathlineto{\pgfqpoint{3.029942in}{2.663632in}}%
\pgfpathlineto{\pgfqpoint{3.042943in}{2.649469in}}%
\pgfpathlineto{\pgfqpoint{3.050790in}{2.659031in}}%
\pgfpathlineto{\pgfqpoint{3.058629in}{2.668695in}}%
\pgfpathlineto{\pgfqpoint{3.066461in}{2.678461in}}%
\pgfpathlineto{\pgfqpoint{3.074286in}{2.688330in}}%
\pgfpathlineto{\pgfqpoint{3.061301in}{2.702491in}}%
\pgfpathlineto{\pgfqpoint{3.048310in}{2.716910in}}%
\pgfpathlineto{\pgfqpoint{3.035315in}{2.731589in}}%
\pgfpathlineto{\pgfqpoint{3.022315in}{2.746530in}}%
\pgfpathlineto{\pgfqpoint{3.014474in}{2.736652in}}%
\pgfpathlineto{\pgfqpoint{3.006626in}{2.726884in}}%
\pgfpathlineto{\pgfqpoint{2.998771in}{2.717226in}}%
\pgfpathlineto{\pgfqpoint{2.990908in}{2.707676in}}%
\pgfpathclose%
\pgfusepath{fill}%
\end{pgfscope}%
\begin{pgfscope}%
\pgfpathrectangle{\pgfqpoint{1.254980in}{0.150000in}}{\pgfqpoint{5.490039in}{5.490039in}}%
\pgfusepath{clip}%
\pgfsetbuttcap%
\pgfsetroundjoin%
\definecolor{currentfill}{rgb}{0.241237,0.296485,0.539709}%
\pgfsetfillcolor{currentfill}%
\pgfsetfillopacity{0.700000}%
\pgfsetlinewidth{0.000000pt}%
\definecolor{currentstroke}{rgb}{0.000000,0.000000,0.000000}%
\pgfsetstrokecolor{currentstroke}%
\pgfsetdash{}{0pt}%
\pgfpathmoveto{\pgfqpoint{2.938783in}{2.770117in}}%
\pgfpathlineto{\pgfqpoint{2.951824in}{2.754102in}}%
\pgfpathlineto{\pgfqpoint{2.964858in}{2.738358in}}%
\pgfpathlineto{\pgfqpoint{2.977886in}{2.722884in}}%
\pgfpathlineto{\pgfqpoint{2.990908in}{2.707676in}}%
\pgfpathlineto{\pgfqpoint{2.998771in}{2.717226in}}%
\pgfpathlineto{\pgfqpoint{3.006626in}{2.726884in}}%
\pgfpathlineto{\pgfqpoint{3.014474in}{2.736652in}}%
\pgfpathlineto{\pgfqpoint{3.022315in}{2.746530in}}%
\pgfpathlineto{\pgfqpoint{3.009309in}{2.761735in}}%
\pgfpathlineto{\pgfqpoint{2.996298in}{2.777207in}}%
\pgfpathlineto{\pgfqpoint{2.983280in}{2.792949in}}%
\pgfpathlineto{\pgfqpoint{2.970256in}{2.808962in}}%
\pgfpathlineto{\pgfqpoint{2.962400in}{2.799076in}}%
\pgfpathlineto{\pgfqpoint{2.954535in}{2.789307in}}%
\pgfpathlineto{\pgfqpoint{2.946663in}{2.779654in}}%
\pgfpathlineto{\pgfqpoint{2.938783in}{2.770117in}}%
\pgfpathclose%
\pgfusepath{fill}%
\end{pgfscope}%
\begin{pgfscope}%
\pgfpathrectangle{\pgfqpoint{1.254980in}{0.150000in}}{\pgfqpoint{5.490039in}{5.490039in}}%
\pgfusepath{clip}%
\pgfsetbuttcap%
\pgfsetroundjoin%
\definecolor{currentfill}{rgb}{0.172719,0.448791,0.557885}%
\pgfsetfillcolor{currentfill}%
\pgfsetfillopacity{0.700000}%
\pgfsetlinewidth{0.000000pt}%
\definecolor{currentstroke}{rgb}{0.000000,0.000000,0.000000}%
\pgfsetstrokecolor{currentstroke}%
\pgfsetdash{}{0pt}%
\pgfpathmoveto{\pgfqpoint{5.265869in}{3.102417in}}%
\pgfpathlineto{\pgfqpoint{5.279276in}{3.104363in}}%
\pgfpathlineto{\pgfqpoint{5.292695in}{3.106459in}}%
\pgfpathlineto{\pgfqpoint{5.306127in}{3.108706in}}%
\pgfpathlineto{\pgfqpoint{5.319572in}{3.111105in}}%
\pgfpathlineto{\pgfqpoint{5.326695in}{3.120756in}}%
\pgfpathlineto{\pgfqpoint{5.333818in}{3.130616in}}%
\pgfpathlineto{\pgfqpoint{5.340943in}{3.140691in}}%
\pgfpathlineto{\pgfqpoint{5.348069in}{3.150988in}}%
\pgfpathlineto{\pgfqpoint{5.334647in}{3.149205in}}%
\pgfpathlineto{\pgfqpoint{5.321237in}{3.147572in}}%
\pgfpathlineto{\pgfqpoint{5.307839in}{3.146089in}}%
\pgfpathlineto{\pgfqpoint{5.294454in}{3.144757in}}%
\pgfpathlineto{\pgfqpoint{5.287306in}{3.133836in}}%
\pgfpathlineto{\pgfqpoint{5.280159in}{3.123144in}}%
\pgfpathlineto{\pgfqpoint{5.273014in}{3.112673in}}%
\pgfpathlineto{\pgfqpoint{5.265869in}{3.102417in}}%
\pgfpathclose%
\pgfusepath{fill}%
\end{pgfscope}%
\begin{pgfscope}%
\pgfpathrectangle{\pgfqpoint{1.254980in}{0.150000in}}{\pgfqpoint{5.490039in}{5.490039in}}%
\pgfusepath{clip}%
\pgfsetbuttcap%
\pgfsetroundjoin%
\definecolor{currentfill}{rgb}{0.262138,0.242286,0.520837}%
\pgfsetfillcolor{currentfill}%
\pgfsetfillopacity{0.700000}%
\pgfsetlinewidth{0.000000pt}%
\definecolor{currentstroke}{rgb}{0.000000,0.000000,0.000000}%
\pgfsetstrokecolor{currentstroke}%
\pgfsetdash{}{0pt}%
\pgfpathmoveto{\pgfqpoint{4.362945in}{2.624009in}}%
\pgfpathlineto{\pgfqpoint{4.376045in}{2.623882in}}%
\pgfpathlineto{\pgfqpoint{4.389153in}{2.623923in}}%
\pgfpathlineto{\pgfqpoint{4.402270in}{2.624131in}}%
\pgfpathlineto{\pgfqpoint{4.415397in}{2.624506in}}%
\pgfpathlineto{\pgfqpoint{4.422805in}{2.634041in}}%
\pgfpathlineto{\pgfqpoint{4.430209in}{2.643628in}}%
\pgfpathlineto{\pgfqpoint{4.437609in}{2.653271in}}%
\pgfpathlineto{\pgfqpoint{4.445004in}{2.662974in}}%
\pgfpathlineto{\pgfqpoint{4.431888in}{2.662906in}}%
\pgfpathlineto{\pgfqpoint{4.418781in}{2.663005in}}%
\pgfpathlineto{\pgfqpoint{4.405683in}{2.663271in}}%
\pgfpathlineto{\pgfqpoint{4.392593in}{2.663705in}}%
\pgfpathlineto{\pgfqpoint{4.385187in}{2.653685in}}%
\pgfpathlineto{\pgfqpoint{4.377777in}{2.643732in}}%
\pgfpathlineto{\pgfqpoint{4.370363in}{2.633841in}}%
\pgfpathlineto{\pgfqpoint{4.362945in}{2.624009in}}%
\pgfpathclose%
\pgfusepath{fill}%
\end{pgfscope}%
\begin{pgfscope}%
\pgfpathrectangle{\pgfqpoint{1.254980in}{0.150000in}}{\pgfqpoint{5.490039in}{5.490039in}}%
\pgfusepath{clip}%
\pgfsetbuttcap%
\pgfsetroundjoin%
\definecolor{currentfill}{rgb}{0.262138,0.242286,0.520837}%
\pgfsetfillcolor{currentfill}%
\pgfsetfillopacity{0.700000}%
\pgfsetlinewidth{0.000000pt}%
\definecolor{currentstroke}{rgb}{0.000000,0.000000,0.000000}%
\pgfsetstrokecolor{currentstroke}%
\pgfsetdash{}{0pt}%
\pgfpathmoveto{\pgfqpoint{3.042943in}{2.649469in}}%
\pgfpathlineto{\pgfqpoint{3.055940in}{2.635563in}}%
\pgfpathlineto{\pgfqpoint{3.068932in}{2.621909in}}%
\pgfpathlineto{\pgfqpoint{3.081919in}{2.608508in}}%
\pgfpathlineto{\pgfqpoint{3.094903in}{2.595355in}}%
\pgfpathlineto{\pgfqpoint{3.102734in}{2.604929in}}%
\pgfpathlineto{\pgfqpoint{3.110558in}{2.614597in}}%
\pgfpathlineto{\pgfqpoint{3.118375in}{2.624361in}}%
\pgfpathlineto{\pgfqpoint{3.126185in}{2.634221in}}%
\pgfpathlineto{\pgfqpoint{3.113216in}{2.647372in}}%
\pgfpathlineto{\pgfqpoint{3.100244in}{2.660773in}}%
\pgfpathlineto{\pgfqpoint{3.087267in}{2.674425in}}%
\pgfpathlineto{\pgfqpoint{3.074286in}{2.688330in}}%
\pgfpathlineto{\pgfqpoint{3.066461in}{2.678461in}}%
\pgfpathlineto{\pgfqpoint{3.058629in}{2.668695in}}%
\pgfpathlineto{\pgfqpoint{3.050790in}{2.659031in}}%
\pgfpathlineto{\pgfqpoint{3.042943in}{2.649469in}}%
\pgfpathclose%
\pgfusepath{fill}%
\end{pgfscope}%
\begin{pgfscope}%
\pgfpathrectangle{\pgfqpoint{1.254980in}{0.150000in}}{\pgfqpoint{5.490039in}{5.490039in}}%
\pgfusepath{clip}%
\pgfsetbuttcap%
\pgfsetroundjoin%
\definecolor{currentfill}{rgb}{0.227802,0.326594,0.546532}%
\pgfsetfillcolor{currentfill}%
\pgfsetfillopacity{0.700000}%
\pgfsetlinewidth{0.000000pt}%
\definecolor{currentstroke}{rgb}{0.000000,0.000000,0.000000}%
\pgfsetstrokecolor{currentstroke}%
\pgfsetdash{}{0pt}%
\pgfpathmoveto{\pgfqpoint{2.886552in}{2.836945in}}%
\pgfpathlineto{\pgfqpoint{2.899621in}{2.819818in}}%
\pgfpathlineto{\pgfqpoint{2.912682in}{2.802973in}}%
\pgfpathlineto{\pgfqpoint{2.925736in}{2.786407in}}%
\pgfpathlineto{\pgfqpoint{2.938783in}{2.770117in}}%
\pgfpathlineto{\pgfqpoint{2.946663in}{2.779654in}}%
\pgfpathlineto{\pgfqpoint{2.954535in}{2.789307in}}%
\pgfpathlineto{\pgfqpoint{2.962400in}{2.799076in}}%
\pgfpathlineto{\pgfqpoint{2.970256in}{2.808962in}}%
\pgfpathlineto{\pgfqpoint{2.957226in}{2.825249in}}%
\pgfpathlineto{\pgfqpoint{2.944189in}{2.841812in}}%
\pgfpathlineto{\pgfqpoint{2.931145in}{2.858654in}}%
\pgfpathlineto{\pgfqpoint{2.918094in}{2.875778in}}%
\pgfpathlineto{\pgfqpoint{2.910221in}{2.865885in}}%
\pgfpathlineto{\pgfqpoint{2.902339in}{2.856115in}}%
\pgfpathlineto{\pgfqpoint{2.894449in}{2.846468in}}%
\pgfpathlineto{\pgfqpoint{2.886552in}{2.836945in}}%
\pgfpathclose%
\pgfusepath{fill}%
\end{pgfscope}%
\begin{pgfscope}%
\pgfpathrectangle{\pgfqpoint{1.254980in}{0.150000in}}{\pgfqpoint{5.490039in}{5.490039in}}%
\pgfusepath{clip}%
\pgfsetbuttcap%
\pgfsetroundjoin%
\definecolor{currentfill}{rgb}{0.163625,0.471133,0.558148}%
\pgfsetfillcolor{currentfill}%
\pgfsetfillopacity{0.700000}%
\pgfsetlinewidth{0.000000pt}%
\definecolor{currentstroke}{rgb}{0.000000,0.000000,0.000000}%
\pgfsetstrokecolor{currentstroke}%
\pgfsetdash{}{0pt}%
\pgfpathmoveto{\pgfqpoint{5.348069in}{3.150988in}}%
\pgfpathlineto{\pgfqpoint{5.361503in}{3.152921in}}%
\pgfpathlineto{\pgfqpoint{5.374951in}{3.155005in}}%
\pgfpathlineto{\pgfqpoint{5.388411in}{3.157239in}}%
\pgfpathlineto{\pgfqpoint{5.401883in}{3.159622in}}%
\pgfpathlineto{\pgfqpoint{5.408987in}{3.169516in}}%
\pgfpathlineto{\pgfqpoint{5.416093in}{3.179640in}}%
\pgfpathlineto{\pgfqpoint{5.423201in}{3.190002in}}%
\pgfpathlineto{\pgfqpoint{5.430311in}{3.200608in}}%
\pgfpathlineto{\pgfqpoint{5.416862in}{3.198868in}}%
\pgfpathlineto{\pgfqpoint{5.403426in}{3.197276in}}%
\pgfpathlineto{\pgfqpoint{5.390002in}{3.195835in}}%
\pgfpathlineto{\pgfqpoint{5.376590in}{3.194542in}}%
\pgfpathlineto{\pgfqpoint{5.369457in}{3.183285in}}%
\pgfpathlineto{\pgfqpoint{5.362325in}{3.172278in}}%
\pgfpathlineto{\pgfqpoint{5.355196in}{3.161515in}}%
\pgfpathlineto{\pgfqpoint{5.348069in}{3.150988in}}%
\pgfpathclose%
\pgfusepath{fill}%
\end{pgfscope}%
\begin{pgfscope}%
\pgfpathrectangle{\pgfqpoint{1.254980in}{0.150000in}}{\pgfqpoint{5.490039in}{5.490039in}}%
\pgfusepath{clip}%
\pgfsetbuttcap%
\pgfsetroundjoin%
\definecolor{currentfill}{rgb}{0.267968,0.223549,0.512008}%
\pgfsetfillcolor{currentfill}%
\pgfsetfillopacity{0.700000}%
\pgfsetlinewidth{0.000000pt}%
\definecolor{currentstroke}{rgb}{0.000000,0.000000,0.000000}%
\pgfsetstrokecolor{currentstroke}%
\pgfsetdash{}{0pt}%
\pgfpathmoveto{\pgfqpoint{4.280875in}{2.586284in}}%
\pgfpathlineto{\pgfqpoint{4.293951in}{2.585759in}}%
\pgfpathlineto{\pgfqpoint{4.307034in}{2.585404in}}%
\pgfpathlineto{\pgfqpoint{4.320127in}{2.585219in}}%
\pgfpathlineto{\pgfqpoint{4.333227in}{2.585203in}}%
\pgfpathlineto{\pgfqpoint{4.340664in}{2.594833in}}%
\pgfpathlineto{\pgfqpoint{4.348095in}{2.604509in}}%
\pgfpathlineto{\pgfqpoint{4.355522in}{2.614233in}}%
\pgfpathlineto{\pgfqpoint{4.362945in}{2.624009in}}%
\pgfpathlineto{\pgfqpoint{4.349854in}{2.624305in}}%
\pgfpathlineto{\pgfqpoint{4.336771in}{2.624769in}}%
\pgfpathlineto{\pgfqpoint{4.323697in}{2.625403in}}%
\pgfpathlineto{\pgfqpoint{4.310630in}{2.626206in}}%
\pgfpathlineto{\pgfqpoint{4.303198in}{2.616141in}}%
\pgfpathlineto{\pgfqpoint{4.295761in}{2.606134in}}%
\pgfpathlineto{\pgfqpoint{4.288320in}{2.596183in}}%
\pgfpathlineto{\pgfqpoint{4.280875in}{2.586284in}}%
\pgfpathclose%
\pgfusepath{fill}%
\end{pgfscope}%
\begin{pgfscope}%
\pgfpathrectangle{\pgfqpoint{1.254980in}{0.150000in}}{\pgfqpoint{5.490039in}{5.490039in}}%
\pgfusepath{clip}%
\pgfsetbuttcap%
\pgfsetroundjoin%
\definecolor{currentfill}{rgb}{0.282623,0.140926,0.457517}%
\pgfsetfillcolor{currentfill}%
\pgfsetfillopacity{0.700000}%
\pgfsetlinewidth{0.000000pt}%
\definecolor{currentstroke}{rgb}{0.000000,0.000000,0.000000}%
\pgfsetstrokecolor{currentstroke}%
\pgfsetdash{}{0pt}%
\pgfpathmoveto{\pgfqpoint{3.467705in}{2.428463in}}%
\pgfpathlineto{\pgfqpoint{3.480634in}{2.420953in}}%
\pgfpathlineto{\pgfqpoint{3.493564in}{2.413652in}}%
\pgfpathlineto{\pgfqpoint{3.506496in}{2.406556in}}%
\pgfpathlineto{\pgfqpoint{3.519430in}{2.399666in}}%
\pgfpathlineto{\pgfqpoint{3.527122in}{2.409700in}}%
\pgfpathlineto{\pgfqpoint{3.534809in}{2.419786in}}%
\pgfpathlineto{\pgfqpoint{3.542490in}{2.429925in}}%
\pgfpathlineto{\pgfqpoint{3.550166in}{2.440119in}}%
\pgfpathlineto{\pgfqpoint{3.537243in}{2.447065in}}%
\pgfpathlineto{\pgfqpoint{3.524322in}{2.454217in}}%
\pgfpathlineto{\pgfqpoint{3.511402in}{2.461575in}}%
\pgfpathlineto{\pgfqpoint{3.498484in}{2.469141in}}%
\pgfpathlineto{\pgfqpoint{3.490797in}{2.458881in}}%
\pgfpathlineto{\pgfqpoint{3.483105in}{2.448682in}}%
\pgfpathlineto{\pgfqpoint{3.475408in}{2.438543in}}%
\pgfpathlineto{\pgfqpoint{3.467705in}{2.428463in}}%
\pgfpathclose%
\pgfusepath{fill}%
\end{pgfscope}%
\begin{pgfscope}%
\pgfpathrectangle{\pgfqpoint{1.254980in}{0.150000in}}{\pgfqpoint{5.490039in}{5.490039in}}%
\pgfusepath{clip}%
\pgfsetbuttcap%
\pgfsetroundjoin%
\definecolor{currentfill}{rgb}{0.269308,0.218818,0.509577}%
\pgfsetfillcolor{currentfill}%
\pgfsetfillopacity{0.700000}%
\pgfsetlinewidth{0.000000pt}%
\definecolor{currentstroke}{rgb}{0.000000,0.000000,0.000000}%
\pgfsetstrokecolor{currentstroke}%
\pgfsetdash{}{0pt}%
\pgfpathmoveto{\pgfqpoint{3.094903in}{2.595355in}}%
\pgfpathlineto{\pgfqpoint{3.107883in}{2.582451in}}%
\pgfpathlineto{\pgfqpoint{3.120860in}{2.569792in}}%
\pgfpathlineto{\pgfqpoint{3.133833in}{2.557376in}}%
\pgfpathlineto{\pgfqpoint{3.146803in}{2.545202in}}%
\pgfpathlineto{\pgfqpoint{3.154619in}{2.554787in}}%
\pgfpathlineto{\pgfqpoint{3.162428in}{2.564460in}}%
\pgfpathlineto{\pgfqpoint{3.170231in}{2.574221in}}%
\pgfpathlineto{\pgfqpoint{3.178026in}{2.584072in}}%
\pgfpathlineto{\pgfqpoint{3.165071in}{2.596245in}}%
\pgfpathlineto{\pgfqpoint{3.152112in}{2.608659in}}%
\pgfpathlineto{\pgfqpoint{3.139150in}{2.621318in}}%
\pgfpathlineto{\pgfqpoint{3.126185in}{2.634221in}}%
\pgfpathlineto{\pgfqpoint{3.118375in}{2.624361in}}%
\pgfpathlineto{\pgfqpoint{3.110558in}{2.614597in}}%
\pgfpathlineto{\pgfqpoint{3.102734in}{2.604929in}}%
\pgfpathlineto{\pgfqpoint{3.094903in}{2.595355in}}%
\pgfpathclose%
\pgfusepath{fill}%
\end{pgfscope}%
\begin{pgfscope}%
\pgfpathrectangle{\pgfqpoint{1.254980in}{0.150000in}}{\pgfqpoint{5.490039in}{5.490039in}}%
\pgfusepath{clip}%
\pgfsetbuttcap%
\pgfsetroundjoin%
\definecolor{currentfill}{rgb}{0.282290,0.145912,0.461510}%
\pgfsetfillcolor{currentfill}%
\pgfsetfillopacity{0.700000}%
\pgfsetlinewidth{0.000000pt}%
\definecolor{currentstroke}{rgb}{0.000000,0.000000,0.000000}%
\pgfsetstrokecolor{currentstroke}%
\pgfsetdash{}{0pt}%
\pgfpathmoveto{\pgfqpoint{3.818243in}{2.436480in}}%
\pgfpathlineto{\pgfqpoint{3.831205in}{2.432658in}}%
\pgfpathlineto{\pgfqpoint{3.844172in}{2.429023in}}%
\pgfpathlineto{\pgfqpoint{3.857144in}{2.425573in}}%
\pgfpathlineto{\pgfqpoint{3.870121in}{2.422309in}}%
\pgfpathlineto{\pgfqpoint{3.877705in}{2.432407in}}%
\pgfpathlineto{\pgfqpoint{3.885283in}{2.442540in}}%
\pgfpathlineto{\pgfqpoint{3.892856in}{2.452710in}}%
\pgfpathlineto{\pgfqpoint{3.900425in}{2.462919in}}%
\pgfpathlineto{\pgfqpoint{3.887456in}{2.466323in}}%
\pgfpathlineto{\pgfqpoint{3.874492in}{2.469912in}}%
\pgfpathlineto{\pgfqpoint{3.861534in}{2.473687in}}%
\pgfpathlineto{\pgfqpoint{3.848580in}{2.477649in}}%
\pgfpathlineto{\pgfqpoint{3.841003in}{2.467290in}}%
\pgfpathlineto{\pgfqpoint{3.833421in}{2.456977in}}%
\pgfpathlineto{\pgfqpoint{3.825834in}{2.446708in}}%
\pgfpathlineto{\pgfqpoint{3.818243in}{2.436480in}}%
\pgfpathclose%
\pgfusepath{fill}%
\end{pgfscope}%
\begin{pgfscope}%
\pgfpathrectangle{\pgfqpoint{1.254980in}{0.150000in}}{\pgfqpoint{5.490039in}{5.490039in}}%
\pgfusepath{clip}%
\pgfsetbuttcap%
\pgfsetroundjoin%
\definecolor{currentfill}{rgb}{0.214298,0.355619,0.551184}%
\pgfsetfillcolor{currentfill}%
\pgfsetfillopacity{0.700000}%
\pgfsetlinewidth{0.000000pt}%
\definecolor{currentstroke}{rgb}{0.000000,0.000000,0.000000}%
\pgfsetstrokecolor{currentstroke}%
\pgfsetdash{}{0pt}%
\pgfpathmoveto{\pgfqpoint{2.834198in}{2.908324in}}%
\pgfpathlineto{\pgfqpoint{2.847299in}{2.890043in}}%
\pgfpathlineto{\pgfqpoint{2.860391in}{2.872055in}}%
\pgfpathlineto{\pgfqpoint{2.873476in}{2.854357in}}%
\pgfpathlineto{\pgfqpoint{2.886552in}{2.836945in}}%
\pgfpathlineto{\pgfqpoint{2.894449in}{2.846468in}}%
\pgfpathlineto{\pgfqpoint{2.902339in}{2.856115in}}%
\pgfpathlineto{\pgfqpoint{2.910221in}{2.865885in}}%
\pgfpathlineto{\pgfqpoint{2.918094in}{2.875778in}}%
\pgfpathlineto{\pgfqpoint{2.905035in}{2.893187in}}%
\pgfpathlineto{\pgfqpoint{2.891969in}{2.910882in}}%
\pgfpathlineto{\pgfqpoint{2.878894in}{2.928866in}}%
\pgfpathlineto{\pgfqpoint{2.865812in}{2.947144in}}%
\pgfpathlineto{\pgfqpoint{2.857921in}{2.937242in}}%
\pgfpathlineto{\pgfqpoint{2.850021in}{2.927472in}}%
\pgfpathlineto{\pgfqpoint{2.842114in}{2.917833in}}%
\pgfpathlineto{\pgfqpoint{2.834198in}{2.908324in}}%
\pgfpathclose%
\pgfusepath{fill}%
\end{pgfscope}%
\begin{pgfscope}%
\pgfpathrectangle{\pgfqpoint{1.254980in}{0.150000in}}{\pgfqpoint{5.490039in}{5.490039in}}%
\pgfusepath{clip}%
\pgfsetbuttcap%
\pgfsetroundjoin%
\definecolor{currentfill}{rgb}{0.282884,0.135920,0.453427}%
\pgfsetfillcolor{currentfill}%
\pgfsetfillopacity{0.700000}%
\pgfsetlinewidth{0.000000pt}%
\definecolor{currentstroke}{rgb}{0.000000,0.000000,0.000000}%
\pgfsetstrokecolor{currentstroke}%
\pgfsetdash{}{0pt}%
\pgfpathmoveto{\pgfqpoint{3.601882in}{2.414359in}}%
\pgfpathlineto{\pgfqpoint{3.614817in}{2.408420in}}%
\pgfpathlineto{\pgfqpoint{3.627755in}{2.402680in}}%
\pgfpathlineto{\pgfqpoint{3.640696in}{2.397137in}}%
\pgfpathlineto{\pgfqpoint{3.653640in}{2.391790in}}%
\pgfpathlineto{\pgfqpoint{3.661291in}{2.401891in}}%
\pgfpathlineto{\pgfqpoint{3.668936in}{2.412036in}}%
\pgfpathlineto{\pgfqpoint{3.676576in}{2.422225in}}%
\pgfpathlineto{\pgfqpoint{3.684212in}{2.432459in}}%
\pgfpathlineto{\pgfqpoint{3.671277in}{2.437890in}}%
\pgfpathlineto{\pgfqpoint{3.658345in}{2.443518in}}%
\pgfpathlineto{\pgfqpoint{3.645417in}{2.449342in}}%
\pgfpathlineto{\pgfqpoint{3.632492in}{2.455365in}}%
\pgfpathlineto{\pgfqpoint{3.624847in}{2.445036in}}%
\pgfpathlineto{\pgfqpoint{3.617197in}{2.434760in}}%
\pgfpathlineto{\pgfqpoint{3.609542in}{2.424534in}}%
\pgfpathlineto{\pgfqpoint{3.601882in}{2.414359in}}%
\pgfpathclose%
\pgfusepath{fill}%
\end{pgfscope}%
\begin{pgfscope}%
\pgfpathrectangle{\pgfqpoint{1.254980in}{0.150000in}}{\pgfqpoint{5.490039in}{5.490039in}}%
\pgfusepath{clip}%
\pgfsetbuttcap%
\pgfsetroundjoin%
\definecolor{currentfill}{rgb}{0.271828,0.209303,0.504434}%
\pgfsetfillcolor{currentfill}%
\pgfsetfillopacity{0.700000}%
\pgfsetlinewidth{0.000000pt}%
\definecolor{currentstroke}{rgb}{0.000000,0.000000,0.000000}%
\pgfsetstrokecolor{currentstroke}%
\pgfsetdash{}{0pt}%
\pgfpathmoveto{\pgfqpoint{4.198788in}{2.549951in}}%
\pgfpathlineto{\pgfqpoint{4.211841in}{2.548992in}}%
\pgfpathlineto{\pgfqpoint{4.224902in}{2.548204in}}%
\pgfpathlineto{\pgfqpoint{4.237971in}{2.547589in}}%
\pgfpathlineto{\pgfqpoint{4.251047in}{2.547145in}}%
\pgfpathlineto{\pgfqpoint{4.258511in}{2.556867in}}%
\pgfpathlineto{\pgfqpoint{4.265970in}{2.566629in}}%
\pgfpathlineto{\pgfqpoint{4.273425in}{2.576434in}}%
\pgfpathlineto{\pgfqpoint{4.280875in}{2.586284in}}%
\pgfpathlineto{\pgfqpoint{4.267807in}{2.586980in}}%
\pgfpathlineto{\pgfqpoint{4.254748in}{2.587846in}}%
\pgfpathlineto{\pgfqpoint{4.241696in}{2.588885in}}%
\pgfpathlineto{\pgfqpoint{4.228651in}{2.590095in}}%
\pgfpathlineto{\pgfqpoint{4.221192in}{2.579984in}}%
\pgfpathlineto{\pgfqpoint{4.213729in}{2.569924in}}%
\pgfpathlineto{\pgfqpoint{4.206261in}{2.559915in}}%
\pgfpathlineto{\pgfqpoint{4.198788in}{2.549951in}}%
\pgfpathclose%
\pgfusepath{fill}%
\end{pgfscope}%
\begin{pgfscope}%
\pgfpathrectangle{\pgfqpoint{1.254980in}{0.150000in}}{\pgfqpoint{5.490039in}{5.490039in}}%
\pgfusepath{clip}%
\pgfsetbuttcap%
\pgfsetroundjoin%
\definecolor{currentfill}{rgb}{0.281887,0.150881,0.465405}%
\pgfsetfillcolor{currentfill}%
\pgfsetfillopacity{0.700000}%
\pgfsetlinewidth{0.000000pt}%
\definecolor{currentstroke}{rgb}{0.000000,0.000000,0.000000}%
\pgfsetstrokecolor{currentstroke}%
\pgfsetdash{}{0pt}%
\pgfpathmoveto{\pgfqpoint{3.333340in}{2.456196in}}%
\pgfpathlineto{\pgfqpoint{3.346277in}{2.447006in}}%
\pgfpathlineto{\pgfqpoint{3.359213in}{2.438035in}}%
\pgfpathlineto{\pgfqpoint{3.372149in}{2.429281in}}%
\pgfpathlineto{\pgfqpoint{3.385085in}{2.420743in}}%
\pgfpathlineto{\pgfqpoint{3.392822in}{2.430613in}}%
\pgfpathlineto{\pgfqpoint{3.400554in}{2.440546in}}%
\pgfpathlineto{\pgfqpoint{3.408280in}{2.450542in}}%
\pgfpathlineto{\pgfqpoint{3.416000in}{2.460603in}}%
\pgfpathlineto{\pgfqpoint{3.403075in}{2.469170in}}%
\pgfpathlineto{\pgfqpoint{3.390151in}{2.477952in}}%
\pgfpathlineto{\pgfqpoint{3.377227in}{2.486951in}}%
\pgfpathlineto{\pgfqpoint{3.364303in}{2.496169in}}%
\pgfpathlineto{\pgfqpoint{3.356571in}{2.486070in}}%
\pgfpathlineto{\pgfqpoint{3.348834in}{2.476042in}}%
\pgfpathlineto{\pgfqpoint{3.341090in}{2.466084in}}%
\pgfpathlineto{\pgfqpoint{3.333340in}{2.456196in}}%
\pgfpathclose%
\pgfusepath{fill}%
\end{pgfscope}%
\begin{pgfscope}%
\pgfpathrectangle{\pgfqpoint{1.254980in}{0.150000in}}{\pgfqpoint{5.490039in}{5.490039in}}%
\pgfusepath{clip}%
\pgfsetbuttcap%
\pgfsetroundjoin%
\definecolor{currentfill}{rgb}{0.156270,0.489624,0.557936}%
\pgfsetfillcolor{currentfill}%
\pgfsetfillopacity{0.700000}%
\pgfsetlinewidth{0.000000pt}%
\definecolor{currentstroke}{rgb}{0.000000,0.000000,0.000000}%
\pgfsetstrokecolor{currentstroke}%
\pgfsetdash{}{0pt}%
\pgfpathmoveto{\pgfqpoint{5.430311in}{3.200608in}}%
\pgfpathlineto{\pgfqpoint{5.443772in}{3.202498in}}%
\pgfpathlineto{\pgfqpoint{5.457247in}{3.204536in}}%
\pgfpathlineto{\pgfqpoint{5.470734in}{3.206724in}}%
\pgfpathlineto{\pgfqpoint{5.484234in}{3.209060in}}%
\pgfpathlineto{\pgfqpoint{5.491322in}{3.219258in}}%
\pgfpathlineto{\pgfqpoint{5.498412in}{3.229708in}}%
\pgfpathlineto{\pgfqpoint{5.505506in}{3.240419in}}%
\pgfpathlineto{\pgfqpoint{5.492025in}{3.238584in}}%
\pgfpathlineto{\pgfqpoint{5.478556in}{3.236896in}}%
\pgfpathlineto{\pgfqpoint{5.465100in}{3.235358in}}%
\pgfpathlineto{\pgfqpoint{5.451657in}{3.233967in}}%
\pgfpathlineto{\pgfqpoint{5.444539in}{3.222583in}}%
\pgfpathlineto{\pgfqpoint{5.437423in}{3.211466in}}%
\pgfpathlineto{\pgfqpoint{5.430311in}{3.200608in}}%
\pgfpathclose%
\pgfusepath{fill}%
\end{pgfscope}%
\begin{pgfscope}%
\pgfpathrectangle{\pgfqpoint{1.254980in}{0.150000in}}{\pgfqpoint{5.490039in}{5.490039in}}%
\pgfusepath{clip}%
\pgfsetbuttcap%
\pgfsetroundjoin%
\definecolor{currentfill}{rgb}{0.275191,0.194905,0.496005}%
\pgfsetfillcolor{currentfill}%
\pgfsetfillopacity{0.700000}%
\pgfsetlinewidth{0.000000pt}%
\definecolor{currentstroke}{rgb}{0.000000,0.000000,0.000000}%
\pgfsetstrokecolor{currentstroke}%
\pgfsetdash{}{0pt}%
\pgfpathmoveto{\pgfqpoint{3.146803in}{2.545202in}}%
\pgfpathlineto{\pgfqpoint{3.159771in}{2.533268in}}%
\pgfpathlineto{\pgfqpoint{3.172735in}{2.521573in}}%
\pgfpathlineto{\pgfqpoint{3.185698in}{2.510113in}}%
\pgfpathlineto{\pgfqpoint{3.198658in}{2.498888in}}%
\pgfpathlineto{\pgfqpoint{3.206459in}{2.508483in}}%
\pgfpathlineto{\pgfqpoint{3.214254in}{2.518160in}}%
\pgfpathlineto{\pgfqpoint{3.222042in}{2.527918in}}%
\pgfpathlineto{\pgfqpoint{3.229824in}{2.537759in}}%
\pgfpathlineto{\pgfqpoint{3.216878in}{2.548983in}}%
\pgfpathlineto{\pgfqpoint{3.203930in}{2.560443in}}%
\pgfpathlineto{\pgfqpoint{3.190979in}{2.572138in}}%
\pgfpathlineto{\pgfqpoint{3.178026in}{2.584072in}}%
\pgfpathlineto{\pgfqpoint{3.170231in}{2.574221in}}%
\pgfpathlineto{\pgfqpoint{3.162428in}{2.564460in}}%
\pgfpathlineto{\pgfqpoint{3.154619in}{2.554787in}}%
\pgfpathlineto{\pgfqpoint{3.146803in}{2.545202in}}%
\pgfpathclose%
\pgfusepath{fill}%
\end{pgfscope}%
\begin{pgfscope}%
\pgfpathrectangle{\pgfqpoint{1.254980in}{0.150000in}}{\pgfqpoint{5.490039in}{5.490039in}}%
\pgfusepath{clip}%
\pgfsetbuttcap%
\pgfsetroundjoin%
\definecolor{currentfill}{rgb}{0.276194,0.190074,0.493001}%
\pgfsetfillcolor{currentfill}%
\pgfsetfillopacity{0.700000}%
\pgfsetlinewidth{0.000000pt}%
\definecolor{currentstroke}{rgb}{0.000000,0.000000,0.000000}%
\pgfsetstrokecolor{currentstroke}%
\pgfsetdash{}{0pt}%
\pgfpathmoveto{\pgfqpoint{4.116677in}{2.515183in}}%
\pgfpathlineto{\pgfqpoint{4.129710in}{2.513751in}}%
\pgfpathlineto{\pgfqpoint{4.142749in}{2.512494in}}%
\pgfpathlineto{\pgfqpoint{4.155796in}{2.511412in}}%
\pgfpathlineto{\pgfqpoint{4.168851in}{2.510503in}}%
\pgfpathlineto{\pgfqpoint{4.176342in}{2.520310in}}%
\pgfpathlineto{\pgfqpoint{4.183829in}{2.530152in}}%
\pgfpathlineto{\pgfqpoint{4.191311in}{2.540031in}}%
\pgfpathlineto{\pgfqpoint{4.198788in}{2.549951in}}%
\pgfpathlineto{\pgfqpoint{4.185742in}{2.551084in}}%
\pgfpathlineto{\pgfqpoint{4.172704in}{2.552390in}}%
\pgfpathlineto{\pgfqpoint{4.159673in}{2.553870in}}%
\pgfpathlineto{\pgfqpoint{4.146650in}{2.555525in}}%
\pgfpathlineto{\pgfqpoint{4.139163in}{2.545371in}}%
\pgfpathlineto{\pgfqpoint{4.131673in}{2.535265in}}%
\pgfpathlineto{\pgfqpoint{4.124177in}{2.525203in}}%
\pgfpathlineto{\pgfqpoint{4.116677in}{2.515183in}}%
\pgfpathclose%
\pgfusepath{fill}%
\end{pgfscope}%
\begin{pgfscope}%
\pgfpathrectangle{\pgfqpoint{1.254980in}{0.150000in}}{\pgfqpoint{5.490039in}{5.490039in}}%
\pgfusepath{clip}%
\pgfsetbuttcap%
\pgfsetroundjoin%
\definecolor{currentfill}{rgb}{0.199430,0.387607,0.554642}%
\pgfsetfillcolor{currentfill}%
\pgfsetfillopacity{0.700000}%
\pgfsetlinewidth{0.000000pt}%
\definecolor{currentstroke}{rgb}{0.000000,0.000000,0.000000}%
\pgfsetstrokecolor{currentstroke}%
\pgfsetdash{}{0pt}%
\pgfpathmoveto{\pgfqpoint{2.781702in}{2.984429in}}%
\pgfpathlineto{\pgfqpoint{2.794840in}{2.964950in}}%
\pgfpathlineto{\pgfqpoint{2.807969in}{2.945774in}}%
\pgfpathlineto{\pgfqpoint{2.821088in}{2.926900in}}%
\pgfpathlineto{\pgfqpoint{2.834198in}{2.908324in}}%
\pgfpathlineto{\pgfqpoint{2.842114in}{2.917833in}}%
\pgfpathlineto{\pgfqpoint{2.850021in}{2.927472in}}%
\pgfpathlineto{\pgfqpoint{2.857921in}{2.937242in}}%
\pgfpathlineto{\pgfqpoint{2.865812in}{2.947144in}}%
\pgfpathlineto{\pgfqpoint{2.852720in}{2.965716in}}%
\pgfpathlineto{\pgfqpoint{2.839620in}{2.984586in}}%
\pgfpathlineto{\pgfqpoint{2.826510in}{3.003758in}}%
\pgfpathlineto{\pgfqpoint{2.813391in}{3.023233in}}%
\pgfpathlineto{\pgfqpoint{2.805482in}{3.013325in}}%
\pgfpathlineto{\pgfqpoint{2.797564in}{3.003556in}}%
\pgfpathlineto{\pgfqpoint{2.789637in}{2.993924in}}%
\pgfpathlineto{\pgfqpoint{2.781702in}{2.984429in}}%
\pgfpathclose%
\pgfusepath{fill}%
\end{pgfscope}%
\begin{pgfscope}%
\pgfpathrectangle{\pgfqpoint{1.254980in}{0.150000in}}{\pgfqpoint{5.490039in}{5.490039in}}%
\pgfusepath{clip}%
\pgfsetbuttcap%
\pgfsetroundjoin%
\definecolor{currentfill}{rgb}{0.282884,0.135920,0.453427}%
\pgfsetfillcolor{currentfill}%
\pgfsetfillopacity{0.700000}%
\pgfsetlinewidth{0.000000pt}%
\definecolor{currentstroke}{rgb}{0.000000,0.000000,0.000000}%
\pgfsetstrokecolor{currentstroke}%
\pgfsetdash{}{0pt}%
\pgfpathmoveto{\pgfqpoint{3.735986in}{2.412675in}}%
\pgfpathlineto{\pgfqpoint{3.748939in}{2.408210in}}%
\pgfpathlineto{\pgfqpoint{3.761897in}{2.403934in}}%
\pgfpathlineto{\pgfqpoint{3.774859in}{2.399849in}}%
\pgfpathlineto{\pgfqpoint{3.787825in}{2.395952in}}%
\pgfpathlineto{\pgfqpoint{3.795437in}{2.406029in}}%
\pgfpathlineto{\pgfqpoint{3.803044in}{2.416142in}}%
\pgfpathlineto{\pgfqpoint{3.810646in}{2.426292in}}%
\pgfpathlineto{\pgfqpoint{3.818243in}{2.436480in}}%
\pgfpathlineto{\pgfqpoint{3.805285in}{2.440489in}}%
\pgfpathlineto{\pgfqpoint{3.792332in}{2.444688in}}%
\pgfpathlineto{\pgfqpoint{3.779383in}{2.449075in}}%
\pgfpathlineto{\pgfqpoint{3.766439in}{2.453652in}}%
\pgfpathlineto{\pgfqpoint{3.758833in}{2.443342in}}%
\pgfpathlineto{\pgfqpoint{3.751223in}{2.433077in}}%
\pgfpathlineto{\pgfqpoint{3.743607in}{2.422855in}}%
\pgfpathlineto{\pgfqpoint{3.735986in}{2.412675in}}%
\pgfpathclose%
\pgfusepath{fill}%
\end{pgfscope}%
\begin{pgfscope}%
\pgfpathrectangle{\pgfqpoint{1.254980in}{0.150000in}}{\pgfqpoint{5.490039in}{5.490039in}}%
\pgfusepath{clip}%
\pgfsetbuttcap%
\pgfsetroundjoin%
\definecolor{currentfill}{rgb}{0.278826,0.175490,0.483397}%
\pgfsetfillcolor{currentfill}%
\pgfsetfillopacity{0.700000}%
\pgfsetlinewidth{0.000000pt}%
\definecolor{currentstroke}{rgb}{0.000000,0.000000,0.000000}%
\pgfsetstrokecolor{currentstroke}%
\pgfsetdash{}{0pt}%
\pgfpathmoveto{\pgfqpoint{4.034536in}{2.482172in}}%
\pgfpathlineto{\pgfqpoint{4.047549in}{2.480230in}}%
\pgfpathlineto{\pgfqpoint{4.060570in}{2.478466in}}%
\pgfpathlineto{\pgfqpoint{4.073597in}{2.476878in}}%
\pgfpathlineto{\pgfqpoint{4.086630in}{2.475467in}}%
\pgfpathlineto{\pgfqpoint{4.094149in}{2.485346in}}%
\pgfpathlineto{\pgfqpoint{4.101663in}{2.495257in}}%
\pgfpathlineto{\pgfqpoint{4.109173in}{2.505202in}}%
\pgfpathlineto{\pgfqpoint{4.116677in}{2.515183in}}%
\pgfpathlineto{\pgfqpoint{4.103652in}{2.516790in}}%
\pgfpathlineto{\pgfqpoint{4.090634in}{2.518573in}}%
\pgfpathlineto{\pgfqpoint{4.077622in}{2.520533in}}%
\pgfpathlineto{\pgfqpoint{4.064617in}{2.522671in}}%
\pgfpathlineto{\pgfqpoint{4.057104in}{2.512484in}}%
\pgfpathlineto{\pgfqpoint{4.049586in}{2.502340in}}%
\pgfpathlineto{\pgfqpoint{4.042063in}{2.492237in}}%
\pgfpathlineto{\pgfqpoint{4.034536in}{2.482172in}}%
\pgfpathclose%
\pgfusepath{fill}%
\end{pgfscope}%
\begin{pgfscope}%
\pgfpathrectangle{\pgfqpoint{1.254980in}{0.150000in}}{\pgfqpoint{5.490039in}{5.490039in}}%
\pgfusepath{clip}%
\pgfsetbuttcap%
\pgfsetroundjoin%
\definecolor{currentfill}{rgb}{0.278826,0.175490,0.483397}%
\pgfsetfillcolor{currentfill}%
\pgfsetfillopacity{0.700000}%
\pgfsetlinewidth{0.000000pt}%
\definecolor{currentstroke}{rgb}{0.000000,0.000000,0.000000}%
\pgfsetstrokecolor{currentstroke}%
\pgfsetdash{}{0pt}%
\pgfpathmoveto{\pgfqpoint{3.198658in}{2.498888in}}%
\pgfpathlineto{\pgfqpoint{3.211616in}{2.487895in}}%
\pgfpathlineto{\pgfqpoint{3.224572in}{2.477134in}}%
\pgfpathlineto{\pgfqpoint{3.237526in}{2.466602in}}%
\pgfpathlineto{\pgfqpoint{3.250479in}{2.456298in}}%
\pgfpathlineto{\pgfqpoint{3.258267in}{2.465904in}}%
\pgfpathlineto{\pgfqpoint{3.266048in}{2.475584in}}%
\pgfpathlineto{\pgfqpoint{3.273822in}{2.485338in}}%
\pgfpathlineto{\pgfqpoint{3.281591in}{2.495169in}}%
\pgfpathlineto{\pgfqpoint{3.268651in}{2.505473in}}%
\pgfpathlineto{\pgfqpoint{3.255710in}{2.516005in}}%
\pgfpathlineto{\pgfqpoint{3.242768in}{2.526766in}}%
\pgfpathlineto{\pgfqpoint{3.229824in}{2.537759in}}%
\pgfpathlineto{\pgfqpoint{3.222042in}{2.527918in}}%
\pgfpathlineto{\pgfqpoint{3.214254in}{2.518160in}}%
\pgfpathlineto{\pgfqpoint{3.206459in}{2.508483in}}%
\pgfpathlineto{\pgfqpoint{3.198658in}{2.498888in}}%
\pgfpathclose%
\pgfusepath{fill}%
\end{pgfscope}%
\begin{pgfscope}%
\pgfpathrectangle{\pgfqpoint{1.254980in}{0.150000in}}{\pgfqpoint{5.490039in}{5.490039in}}%
\pgfusepath{clip}%
\pgfsetbuttcap%
\pgfsetroundjoin%
\definecolor{currentfill}{rgb}{0.283072,0.130895,0.449241}%
\pgfsetfillcolor{currentfill}%
\pgfsetfillopacity{0.700000}%
\pgfsetlinewidth{0.000000pt}%
\definecolor{currentstroke}{rgb}{0.000000,0.000000,0.000000}%
\pgfsetstrokecolor{currentstroke}%
\pgfsetdash{}{0pt}%
\pgfpathmoveto{\pgfqpoint{3.519430in}{2.399666in}}%
\pgfpathlineto{\pgfqpoint{3.532366in}{2.392980in}}%
\pgfpathlineto{\pgfqpoint{3.545304in}{2.386496in}}%
\pgfpathlineto{\pgfqpoint{3.558244in}{2.380214in}}%
\pgfpathlineto{\pgfqpoint{3.571187in}{2.374133in}}%
\pgfpathlineto{\pgfqpoint{3.578868in}{2.384120in}}%
\pgfpathlineto{\pgfqpoint{3.586545in}{2.394153in}}%
\pgfpathlineto{\pgfqpoint{3.594216in}{2.404232in}}%
\pgfpathlineto{\pgfqpoint{3.601882in}{2.414359in}}%
\pgfpathlineto{\pgfqpoint{3.588949in}{2.420497in}}%
\pgfpathlineto{\pgfqpoint{3.576019in}{2.426836in}}%
\pgfpathlineto{\pgfqpoint{3.563092in}{2.433376in}}%
\pgfpathlineto{\pgfqpoint{3.550166in}{2.440119in}}%
\pgfpathlineto{\pgfqpoint{3.542490in}{2.429925in}}%
\pgfpathlineto{\pgfqpoint{3.534809in}{2.419786in}}%
\pgfpathlineto{\pgfqpoint{3.527122in}{2.409700in}}%
\pgfpathlineto{\pgfqpoint{3.519430in}{2.399666in}}%
\pgfpathclose%
\pgfusepath{fill}%
\end{pgfscope}%
\begin{pgfscope}%
\pgfpathrectangle{\pgfqpoint{1.254980in}{0.150000in}}{\pgfqpoint{5.490039in}{5.490039in}}%
\pgfusepath{clip}%
\pgfsetbuttcap%
\pgfsetroundjoin%
\definecolor{currentfill}{rgb}{0.282623,0.140926,0.457517}%
\pgfsetfillcolor{currentfill}%
\pgfsetfillopacity{0.700000}%
\pgfsetlinewidth{0.000000pt}%
\definecolor{currentstroke}{rgb}{0.000000,0.000000,0.000000}%
\pgfsetstrokecolor{currentstroke}%
\pgfsetdash{}{0pt}%
\pgfpathmoveto{\pgfqpoint{3.385085in}{2.420743in}}%
\pgfpathlineto{\pgfqpoint{3.398022in}{2.412418in}}%
\pgfpathlineto{\pgfqpoint{3.410959in}{2.404307in}}%
\pgfpathlineto{\pgfqpoint{3.423897in}{2.396407in}}%
\pgfpathlineto{\pgfqpoint{3.436836in}{2.388718in}}%
\pgfpathlineto{\pgfqpoint{3.444562in}{2.398570in}}%
\pgfpathlineto{\pgfqpoint{3.452282in}{2.408477in}}%
\pgfpathlineto{\pgfqpoint{3.459996in}{2.418441in}}%
\pgfpathlineto{\pgfqpoint{3.467705in}{2.428463in}}%
\pgfpathlineto{\pgfqpoint{3.454777in}{2.436181in}}%
\pgfpathlineto{\pgfqpoint{3.441851in}{2.444110in}}%
\pgfpathlineto{\pgfqpoint{3.428925in}{2.452250in}}%
\pgfpathlineto{\pgfqpoint{3.416000in}{2.460603in}}%
\pgfpathlineto{\pgfqpoint{3.408280in}{2.450542in}}%
\pgfpathlineto{\pgfqpoint{3.400554in}{2.440546in}}%
\pgfpathlineto{\pgfqpoint{3.392822in}{2.430613in}}%
\pgfpathlineto{\pgfqpoint{3.385085in}{2.420743in}}%
\pgfpathclose%
\pgfusepath{fill}%
\end{pgfscope}%
\begin{pgfscope}%
\pgfpathrectangle{\pgfqpoint{1.254980in}{0.150000in}}{\pgfqpoint{5.490039in}{5.490039in}}%
\pgfusepath{clip}%
\pgfsetbuttcap%
\pgfsetroundjoin%
\definecolor{currentfill}{rgb}{0.280868,0.160771,0.472899}%
\pgfsetfillcolor{currentfill}%
\pgfsetfillopacity{0.700000}%
\pgfsetlinewidth{0.000000pt}%
\definecolor{currentstroke}{rgb}{0.000000,0.000000,0.000000}%
\pgfsetstrokecolor{currentstroke}%
\pgfsetdash{}{0pt}%
\pgfpathmoveto{\pgfqpoint{3.952354in}{2.451135in}}%
\pgfpathlineto{\pgfqpoint{3.965351in}{2.448644in}}%
\pgfpathlineto{\pgfqpoint{3.978354in}{2.446333in}}%
\pgfpathlineto{\pgfqpoint{3.991363in}{2.444202in}}%
\pgfpathlineto{\pgfqpoint{4.004379in}{2.442250in}}%
\pgfpathlineto{\pgfqpoint{4.011925in}{2.452185in}}%
\pgfpathlineto{\pgfqpoint{4.019467in}{2.462148in}}%
\pgfpathlineto{\pgfqpoint{4.027004in}{2.472144in}}%
\pgfpathlineto{\pgfqpoint{4.034536in}{2.482172in}}%
\pgfpathlineto{\pgfqpoint{4.021529in}{2.484292in}}%
\pgfpathlineto{\pgfqpoint{4.008528in}{2.486592in}}%
\pgfpathlineto{\pgfqpoint{3.995534in}{2.489070in}}%
\pgfpathlineto{\pgfqpoint{3.982546in}{2.491729in}}%
\pgfpathlineto{\pgfqpoint{3.975005in}{2.481522in}}%
\pgfpathlineto{\pgfqpoint{3.967459in}{2.471356in}}%
\pgfpathlineto{\pgfqpoint{3.959909in}{2.461228in}}%
\pgfpathlineto{\pgfqpoint{3.952354in}{2.451135in}}%
\pgfpathclose%
\pgfusepath{fill}%
\end{pgfscope}%
\begin{pgfscope}%
\pgfpathrectangle{\pgfqpoint{1.254980in}{0.150000in}}{\pgfqpoint{5.490039in}{5.490039in}}%
\pgfusepath{clip}%
\pgfsetbuttcap%
\pgfsetroundjoin%
\definecolor{currentfill}{rgb}{0.283072,0.130895,0.449241}%
\pgfsetfillcolor{currentfill}%
\pgfsetfillopacity{0.700000}%
\pgfsetlinewidth{0.000000pt}%
\definecolor{currentstroke}{rgb}{0.000000,0.000000,0.000000}%
\pgfsetstrokecolor{currentstroke}%
\pgfsetdash{}{0pt}%
\pgfpathmoveto{\pgfqpoint{3.653640in}{2.391790in}}%
\pgfpathlineto{\pgfqpoint{3.666588in}{2.386638in}}%
\pgfpathlineto{\pgfqpoint{3.679539in}{2.381681in}}%
\pgfpathlineto{\pgfqpoint{3.692493in}{2.376917in}}%
\pgfpathlineto{\pgfqpoint{3.705452in}{2.372345in}}%
\pgfpathlineto{\pgfqpoint{3.713093in}{2.382372in}}%
\pgfpathlineto{\pgfqpoint{3.720729in}{2.392435in}}%
\pgfpathlineto{\pgfqpoint{3.728360in}{2.402536in}}%
\pgfpathlineto{\pgfqpoint{3.735986in}{2.412675in}}%
\pgfpathlineto{\pgfqpoint{3.723037in}{2.417332in}}%
\pgfpathlineto{\pgfqpoint{3.710091in}{2.422181in}}%
\pgfpathlineto{\pgfqpoint{3.697150in}{2.427223in}}%
\pgfpathlineto{\pgfqpoint{3.684212in}{2.432459in}}%
\pgfpathlineto{\pgfqpoint{3.676576in}{2.422225in}}%
\pgfpathlineto{\pgfqpoint{3.668936in}{2.412036in}}%
\pgfpathlineto{\pgfqpoint{3.661291in}{2.401891in}}%
\pgfpathlineto{\pgfqpoint{3.653640in}{2.391790in}}%
\pgfpathclose%
\pgfusepath{fill}%
\end{pgfscope}%
\begin{pgfscope}%
\pgfpathrectangle{\pgfqpoint{1.254980in}{0.150000in}}{\pgfqpoint{5.490039in}{5.490039in}}%
\pgfusepath{clip}%
\pgfsetbuttcap%
\pgfsetroundjoin%
\definecolor{currentfill}{rgb}{0.225863,0.330805,0.547314}%
\pgfsetfillcolor{currentfill}%
\pgfsetfillopacity{0.700000}%
\pgfsetlinewidth{0.000000pt}%
\definecolor{currentstroke}{rgb}{0.000000,0.000000,0.000000}%
\pgfsetstrokecolor{currentstroke}%
\pgfsetdash{}{0pt}%
\pgfpathmoveto{\pgfqpoint{4.744049in}{2.791493in}}%
\pgfpathlineto{\pgfqpoint{4.757298in}{2.793239in}}%
\pgfpathlineto{\pgfqpoint{4.770557in}{2.795144in}}%
\pgfpathlineto{\pgfqpoint{4.783827in}{2.797208in}}%
\pgfpathlineto{\pgfqpoint{4.797109in}{2.799431in}}%
\pgfpathlineto{\pgfqpoint{4.804393in}{2.808239in}}%
\pgfpathlineto{\pgfqpoint{4.811673in}{2.817133in}}%
\pgfpathlineto{\pgfqpoint{4.818949in}{2.826117in}}%
\pgfpathlineto{\pgfqpoint{4.826222in}{2.835196in}}%
\pgfpathlineto{\pgfqpoint{4.812955in}{2.833394in}}%
\pgfpathlineto{\pgfqpoint{4.799699in}{2.831750in}}%
\pgfpathlineto{\pgfqpoint{4.786453in}{2.830264in}}%
\pgfpathlineto{\pgfqpoint{4.773218in}{2.828938in}}%
\pgfpathlineto{\pgfqpoint{4.765931in}{2.819429in}}%
\pgfpathlineto{\pgfqpoint{4.758640in}{2.810022in}}%
\pgfpathlineto{\pgfqpoint{4.751346in}{2.800712in}}%
\pgfpathlineto{\pgfqpoint{4.744049in}{2.791493in}}%
\pgfpathclose%
\pgfusepath{fill}%
\end{pgfscope}%
\begin{pgfscope}%
\pgfpathrectangle{\pgfqpoint{1.254980in}{0.150000in}}{\pgfqpoint{5.490039in}{5.490039in}}%
\pgfusepath{clip}%
\pgfsetbuttcap%
\pgfsetroundjoin%
\definecolor{currentfill}{rgb}{0.218130,0.347432,0.550038}%
\pgfsetfillcolor{currentfill}%
\pgfsetfillopacity{0.700000}%
\pgfsetlinewidth{0.000000pt}%
\definecolor{currentstroke}{rgb}{0.000000,0.000000,0.000000}%
\pgfsetstrokecolor{currentstroke}%
\pgfsetdash{}{0pt}%
\pgfpathmoveto{\pgfqpoint{4.826222in}{2.835196in}}%
\pgfpathlineto{\pgfqpoint{4.839501in}{2.837156in}}%
\pgfpathlineto{\pgfqpoint{4.852790in}{2.839275in}}%
\pgfpathlineto{\pgfqpoint{4.866091in}{2.841551in}}%
\pgfpathlineto{\pgfqpoint{4.879404in}{2.843984in}}%
\pgfpathlineto{\pgfqpoint{4.886659in}{2.852725in}}%
\pgfpathlineto{\pgfqpoint{4.893911in}{2.861564in}}%
\pgfpathlineto{\pgfqpoint{4.901160in}{2.870507in}}%
\pgfpathlineto{\pgfqpoint{4.908406in}{2.879558in}}%
\pgfpathlineto{\pgfqpoint{4.895108in}{2.877573in}}%
\pgfpathlineto{\pgfqpoint{4.881823in}{2.875746in}}%
\pgfpathlineto{\pgfqpoint{4.868548in}{2.874075in}}%
\pgfpathlineto{\pgfqpoint{4.855284in}{2.872562in}}%
\pgfpathlineto{\pgfqpoint{4.848023in}{2.863053in}}%
\pgfpathlineto{\pgfqpoint{4.840759in}{2.853659in}}%
\pgfpathlineto{\pgfqpoint{4.833492in}{2.844375in}}%
\pgfpathlineto{\pgfqpoint{4.826222in}{2.835196in}}%
\pgfpathclose%
\pgfusepath{fill}%
\end{pgfscope}%
\begin{pgfscope}%
\pgfpathrectangle{\pgfqpoint{1.254980in}{0.150000in}}{\pgfqpoint{5.490039in}{5.490039in}}%
\pgfusepath{clip}%
\pgfsetbuttcap%
\pgfsetroundjoin%
\definecolor{currentfill}{rgb}{0.235526,0.309527,0.542944}%
\pgfsetfillcolor{currentfill}%
\pgfsetfillopacity{0.700000}%
\pgfsetlinewidth{0.000000pt}%
\definecolor{currentstroke}{rgb}{0.000000,0.000000,0.000000}%
\pgfsetstrokecolor{currentstroke}%
\pgfsetdash{}{0pt}%
\pgfpathmoveto{\pgfqpoint{4.661882in}{2.748496in}}%
\pgfpathlineto{\pgfqpoint{4.675102in}{2.749993in}}%
\pgfpathlineto{\pgfqpoint{4.688331in}{2.751651in}}%
\pgfpathlineto{\pgfqpoint{4.701572in}{2.753469in}}%
\pgfpathlineto{\pgfqpoint{4.714823in}{2.755448in}}%
\pgfpathlineto{\pgfqpoint{4.722135in}{2.764344in}}%
\pgfpathlineto{\pgfqpoint{4.729443in}{2.773314in}}%
\pgfpathlineto{\pgfqpoint{4.736748in}{2.782362in}}%
\pgfpathlineto{\pgfqpoint{4.744049in}{2.791493in}}%
\pgfpathlineto{\pgfqpoint{4.730811in}{2.789907in}}%
\pgfpathlineto{\pgfqpoint{4.717584in}{2.788481in}}%
\pgfpathlineto{\pgfqpoint{4.704367in}{2.787215in}}%
\pgfpathlineto{\pgfqpoint{4.691160in}{2.786109in}}%
\pgfpathlineto{\pgfqpoint{4.683846in}{2.776577in}}%
\pgfpathlineto{\pgfqpoint{4.676529in}{2.767133in}}%
\pgfpathlineto{\pgfqpoint{4.669208in}{2.757774in}}%
\pgfpathlineto{\pgfqpoint{4.661882in}{2.748496in}}%
\pgfpathclose%
\pgfusepath{fill}%
\end{pgfscope}%
\begin{pgfscope}%
\pgfpathrectangle{\pgfqpoint{1.254980in}{0.150000in}}{\pgfqpoint{5.490039in}{5.490039in}}%
\pgfusepath{clip}%
\pgfsetbuttcap%
\pgfsetroundjoin%
\definecolor{currentfill}{rgb}{0.208623,0.367752,0.552675}%
\pgfsetfillcolor{currentfill}%
\pgfsetfillopacity{0.700000}%
\pgfsetlinewidth{0.000000pt}%
\definecolor{currentstroke}{rgb}{0.000000,0.000000,0.000000}%
\pgfsetstrokecolor{currentstroke}%
\pgfsetdash{}{0pt}%
\pgfpathmoveto{\pgfqpoint{4.908406in}{2.879558in}}%
\pgfpathlineto{\pgfqpoint{4.921714in}{2.881699in}}%
\pgfpathlineto{\pgfqpoint{4.935034in}{2.883997in}}%
\pgfpathlineto{\pgfqpoint{4.948366in}{2.886451in}}%
\pgfpathlineto{\pgfqpoint{4.961709in}{2.889062in}}%
\pgfpathlineto{\pgfqpoint{4.968936in}{2.897760in}}%
\pgfpathlineto{\pgfqpoint{4.976160in}{2.906572in}}%
\pgfpathlineto{\pgfqpoint{4.983382in}{2.915501in}}%
\pgfpathlineto{\pgfqpoint{4.990601in}{2.924554in}}%
\pgfpathlineto{\pgfqpoint{4.977274in}{2.922421in}}%
\pgfpathlineto{\pgfqpoint{4.963959in}{2.920443in}}%
\pgfpathlineto{\pgfqpoint{4.950655in}{2.918621in}}%
\pgfpathlineto{\pgfqpoint{4.937362in}{2.916956in}}%
\pgfpathlineto{\pgfqpoint{4.930127in}{2.907417in}}%
\pgfpathlineto{\pgfqpoint{4.922889in}{2.898008in}}%
\pgfpathlineto{\pgfqpoint{4.915649in}{2.888723in}}%
\pgfpathlineto{\pgfqpoint{4.908406in}{2.879558in}}%
\pgfpathclose%
\pgfusepath{fill}%
\end{pgfscope}%
\begin{pgfscope}%
\pgfpathrectangle{\pgfqpoint{1.254980in}{0.150000in}}{\pgfqpoint{5.490039in}{5.490039in}}%
\pgfusepath{clip}%
\pgfsetbuttcap%
\pgfsetroundjoin%
\definecolor{currentfill}{rgb}{0.199430,0.387607,0.554642}%
\pgfsetfillcolor{currentfill}%
\pgfsetfillopacity{0.700000}%
\pgfsetlinewidth{0.000000pt}%
\definecolor{currentstroke}{rgb}{0.000000,0.000000,0.000000}%
\pgfsetstrokecolor{currentstroke}%
\pgfsetdash{}{0pt}%
\pgfpathmoveto{\pgfqpoint{4.990601in}{2.924554in}}%
\pgfpathlineto{\pgfqpoint{5.003940in}{2.926843in}}%
\pgfpathlineto{\pgfqpoint{5.017290in}{2.929287in}}%
\pgfpathlineto{\pgfqpoint{5.030653in}{2.931886in}}%
\pgfpathlineto{\pgfqpoint{5.044027in}{2.934640in}}%
\pgfpathlineto{\pgfqpoint{5.051227in}{2.943327in}}%
\pgfpathlineto{\pgfqpoint{5.058424in}{2.952143in}}%
\pgfpathlineto{\pgfqpoint{5.065619in}{2.961092in}}%
\pgfpathlineto{\pgfqpoint{5.072812in}{2.970180in}}%
\pgfpathlineto{\pgfqpoint{5.059455in}{2.967932in}}%
\pgfpathlineto{\pgfqpoint{5.046110in}{2.965837in}}%
\pgfpathlineto{\pgfqpoint{5.032777in}{2.963897in}}%
\pgfpathlineto{\pgfqpoint{5.019455in}{2.962112in}}%
\pgfpathlineto{\pgfqpoint{5.012245in}{2.952509in}}%
\pgfpathlineto{\pgfqpoint{5.005032in}{2.943053in}}%
\pgfpathlineto{\pgfqpoint{4.997818in}{2.933736in}}%
\pgfpathlineto{\pgfqpoint{4.990601in}{2.924554in}}%
\pgfpathclose%
\pgfusepath{fill}%
\end{pgfscope}%
\begin{pgfscope}%
\pgfpathrectangle{\pgfqpoint{1.254980in}{0.150000in}}{\pgfqpoint{5.490039in}{5.490039in}}%
\pgfusepath{clip}%
\pgfsetbuttcap%
\pgfsetroundjoin%
\definecolor{currentfill}{rgb}{0.243113,0.292092,0.538516}%
\pgfsetfillcolor{currentfill}%
\pgfsetfillopacity{0.700000}%
\pgfsetlinewidth{0.000000pt}%
\definecolor{currentstroke}{rgb}{0.000000,0.000000,0.000000}%
\pgfsetstrokecolor{currentstroke}%
\pgfsetdash{}{0pt}%
\pgfpathmoveto{\pgfqpoint{4.579720in}{2.706271in}}%
\pgfpathlineto{\pgfqpoint{4.592911in}{2.707484in}}%
\pgfpathlineto{\pgfqpoint{4.606111in}{2.708860in}}%
\pgfpathlineto{\pgfqpoint{4.619322in}{2.710399in}}%
\pgfpathlineto{\pgfqpoint{4.632543in}{2.712099in}}%
\pgfpathlineto{\pgfqpoint{4.639884in}{2.721099in}}%
\pgfpathlineto{\pgfqpoint{4.647221in}{2.730163in}}%
\pgfpathlineto{\pgfqpoint{4.654554in}{2.739294in}}%
\pgfpathlineto{\pgfqpoint{4.661882in}{2.748496in}}%
\pgfpathlineto{\pgfqpoint{4.648674in}{2.747160in}}%
\pgfpathlineto{\pgfqpoint{4.635475in}{2.745986in}}%
\pgfpathlineto{\pgfqpoint{4.622286in}{2.744974in}}%
\pgfpathlineto{\pgfqpoint{4.609108in}{2.744124in}}%
\pgfpathlineto{\pgfqpoint{4.601767in}{2.734547in}}%
\pgfpathlineto{\pgfqpoint{4.594422in}{2.725049in}}%
\pgfpathlineto{\pgfqpoint{4.587073in}{2.715625in}}%
\pgfpathlineto{\pgfqpoint{4.579720in}{2.706271in}}%
\pgfpathclose%
\pgfusepath{fill}%
\end{pgfscope}%
\begin{pgfscope}%
\pgfpathrectangle{\pgfqpoint{1.254980in}{0.150000in}}{\pgfqpoint{5.490039in}{5.490039in}}%
\pgfusepath{clip}%
\pgfsetbuttcap%
\pgfsetroundjoin%
\definecolor{currentfill}{rgb}{0.280868,0.160771,0.472899}%
\pgfsetfillcolor{currentfill}%
\pgfsetfillopacity{0.700000}%
\pgfsetlinewidth{0.000000pt}%
\definecolor{currentstroke}{rgb}{0.000000,0.000000,0.000000}%
\pgfsetstrokecolor{currentstroke}%
\pgfsetdash{}{0pt}%
\pgfpathmoveto{\pgfqpoint{3.250479in}{2.456298in}}%
\pgfpathlineto{\pgfqpoint{3.263431in}{2.446220in}}%
\pgfpathlineto{\pgfqpoint{3.276382in}{2.436367in}}%
\pgfpathlineto{\pgfqpoint{3.289332in}{2.426736in}}%
\pgfpathlineto{\pgfqpoint{3.302282in}{2.417328in}}%
\pgfpathlineto{\pgfqpoint{3.310055in}{2.426943in}}%
\pgfpathlineto{\pgfqpoint{3.317823in}{2.436626in}}%
\pgfpathlineto{\pgfqpoint{3.325585in}{2.446377in}}%
\pgfpathlineto{\pgfqpoint{3.333340in}{2.456196in}}%
\pgfpathlineto{\pgfqpoint{3.320404in}{2.465605in}}%
\pgfpathlineto{\pgfqpoint{3.307467in}{2.475236in}}%
\pgfpathlineto{\pgfqpoint{3.294529in}{2.485090in}}%
\pgfpathlineto{\pgfqpoint{3.281591in}{2.495169in}}%
\pgfpathlineto{\pgfqpoint{3.273822in}{2.485338in}}%
\pgfpathlineto{\pgfqpoint{3.266048in}{2.475584in}}%
\pgfpathlineto{\pgfqpoint{3.258267in}{2.465904in}}%
\pgfpathlineto{\pgfqpoint{3.250479in}{2.456298in}}%
\pgfpathclose%
\pgfusepath{fill}%
\end{pgfscope}%
\begin{pgfscope}%
\pgfpathrectangle{\pgfqpoint{1.254980in}{0.150000in}}{\pgfqpoint{5.490039in}{5.490039in}}%
\pgfusepath{clip}%
\pgfsetbuttcap%
\pgfsetroundjoin%
\definecolor{currentfill}{rgb}{0.281887,0.150881,0.465405}%
\pgfsetfillcolor{currentfill}%
\pgfsetfillopacity{0.700000}%
\pgfsetlinewidth{0.000000pt}%
\definecolor{currentstroke}{rgb}{0.000000,0.000000,0.000000}%
\pgfsetstrokecolor{currentstroke}%
\pgfsetdash{}{0pt}%
\pgfpathmoveto{\pgfqpoint{3.870121in}{2.422309in}}%
\pgfpathlineto{\pgfqpoint{3.883104in}{2.419229in}}%
\pgfpathlineto{\pgfqpoint{3.896092in}{2.416332in}}%
\pgfpathlineto{\pgfqpoint{3.909086in}{2.413619in}}%
\pgfpathlineto{\pgfqpoint{3.922085in}{2.411087in}}%
\pgfpathlineto{\pgfqpoint{3.929660in}{2.421055in}}%
\pgfpathlineto{\pgfqpoint{3.937229in}{2.431051in}}%
\pgfpathlineto{\pgfqpoint{3.944794in}{2.441077in}}%
\pgfpathlineto{\pgfqpoint{3.952354in}{2.451135in}}%
\pgfpathlineto{\pgfqpoint{3.939363in}{2.453808in}}%
\pgfpathlineto{\pgfqpoint{3.926378in}{2.456662in}}%
\pgfpathlineto{\pgfqpoint{3.913399in}{2.459699in}}%
\pgfpathlineto{\pgfqpoint{3.900425in}{2.462919in}}%
\pgfpathlineto{\pgfqpoint{3.892856in}{2.452710in}}%
\pgfpathlineto{\pgfqpoint{3.885283in}{2.442540in}}%
\pgfpathlineto{\pgfqpoint{3.877705in}{2.432407in}}%
\pgfpathlineto{\pgfqpoint{3.870121in}{2.422309in}}%
\pgfpathclose%
\pgfusepath{fill}%
\end{pgfscope}%
\begin{pgfscope}%
\pgfpathrectangle{\pgfqpoint{1.254980in}{0.150000in}}{\pgfqpoint{5.490039in}{5.490039in}}%
\pgfusepath{clip}%
\pgfsetbuttcap%
\pgfsetroundjoin%
\definecolor{currentfill}{rgb}{0.192357,0.403199,0.555836}%
\pgfsetfillcolor{currentfill}%
\pgfsetfillopacity{0.700000}%
\pgfsetlinewidth{0.000000pt}%
\definecolor{currentstroke}{rgb}{0.000000,0.000000,0.000000}%
\pgfsetstrokecolor{currentstroke}%
\pgfsetdash{}{0pt}%
\pgfpathmoveto{\pgfqpoint{5.072812in}{2.970180in}}%
\pgfpathlineto{\pgfqpoint{5.086181in}{2.972583in}}%
\pgfpathlineto{\pgfqpoint{5.099561in}{2.975140in}}%
\pgfpathlineto{\pgfqpoint{5.112955in}{2.977851in}}%
\pgfpathlineto{\pgfqpoint{5.126360in}{2.980715in}}%
\pgfpathlineto{\pgfqpoint{5.133533in}{2.989427in}}%
\pgfpathlineto{\pgfqpoint{5.140704in}{2.998283in}}%
\pgfpathlineto{\pgfqpoint{5.147873in}{3.007289in}}%
\pgfpathlineto{\pgfqpoint{5.155041in}{3.016453in}}%
\pgfpathlineto{\pgfqpoint{5.141655in}{3.014121in}}%
\pgfpathlineto{\pgfqpoint{5.128280in}{3.011943in}}%
\pgfpathlineto{\pgfqpoint{5.114918in}{3.009918in}}%
\pgfpathlineto{\pgfqpoint{5.101568in}{3.008047in}}%
\pgfpathlineto{\pgfqpoint{5.094381in}{2.998342in}}%
\pgfpathlineto{\pgfqpoint{5.087193in}{2.988799in}}%
\pgfpathlineto{\pgfqpoint{5.080003in}{2.979414in}}%
\pgfpathlineto{\pgfqpoint{5.072812in}{2.970180in}}%
\pgfpathclose%
\pgfusepath{fill}%
\end{pgfscope}%
\begin{pgfscope}%
\pgfpathrectangle{\pgfqpoint{1.254980in}{0.150000in}}{\pgfqpoint{5.490039in}{5.490039in}}%
\pgfusepath{clip}%
\pgfsetbuttcap%
\pgfsetroundjoin%
\definecolor{currentfill}{rgb}{0.250425,0.274290,0.533103}%
\pgfsetfillcolor{currentfill}%
\pgfsetfillopacity{0.700000}%
\pgfsetlinewidth{0.000000pt}%
\definecolor{currentstroke}{rgb}{0.000000,0.000000,0.000000}%
\pgfsetstrokecolor{currentstroke}%
\pgfsetdash{}{0pt}%
\pgfpathmoveto{\pgfqpoint{4.497560in}{2.664905in}}%
\pgfpathlineto{\pgfqpoint{4.510722in}{2.665800in}}%
\pgfpathlineto{\pgfqpoint{4.523894in}{2.666860in}}%
\pgfpathlineto{\pgfqpoint{4.537076in}{2.668083in}}%
\pgfpathlineto{\pgfqpoint{4.550267in}{2.669470in}}%
\pgfpathlineto{\pgfqpoint{4.557637in}{2.678586in}}%
\pgfpathlineto{\pgfqpoint{4.565002in}{2.687755in}}%
\pgfpathlineto{\pgfqpoint{4.572363in}{2.696982in}}%
\pgfpathlineto{\pgfqpoint{4.579720in}{2.706271in}}%
\pgfpathlineto{\pgfqpoint{4.566540in}{2.705220in}}%
\pgfpathlineto{\pgfqpoint{4.553369in}{2.704333in}}%
\pgfpathlineto{\pgfqpoint{4.540208in}{2.703609in}}%
\pgfpathlineto{\pgfqpoint{4.527057in}{2.703049in}}%
\pgfpathlineto{\pgfqpoint{4.519689in}{2.693414in}}%
\pgfpathlineto{\pgfqpoint{4.512317in}{2.683848in}}%
\pgfpathlineto{\pgfqpoint{4.504940in}{2.674346in}}%
\pgfpathlineto{\pgfqpoint{4.497560in}{2.664905in}}%
\pgfpathclose%
\pgfusepath{fill}%
\end{pgfscope}%
\begin{pgfscope}%
\pgfpathrectangle{\pgfqpoint{1.254980in}{0.150000in}}{\pgfqpoint{5.490039in}{5.490039in}}%
\pgfusepath{clip}%
\pgfsetbuttcap%
\pgfsetroundjoin%
\definecolor{currentfill}{rgb}{0.183898,0.422383,0.556944}%
\pgfsetfillcolor{currentfill}%
\pgfsetfillopacity{0.700000}%
\pgfsetlinewidth{0.000000pt}%
\definecolor{currentstroke}{rgb}{0.000000,0.000000,0.000000}%
\pgfsetstrokecolor{currentstroke}%
\pgfsetdash{}{0pt}%
\pgfpathmoveto{\pgfqpoint{5.155041in}{3.016453in}}%
\pgfpathlineto{\pgfqpoint{5.168440in}{3.018937in}}%
\pgfpathlineto{\pgfqpoint{5.181851in}{3.021574in}}%
\pgfpathlineto{\pgfqpoint{5.195274in}{3.024364in}}%
\pgfpathlineto{\pgfqpoint{5.208711in}{3.027306in}}%
\pgfpathlineto{\pgfqpoint{5.215858in}{3.036081in}}%
\pgfpathlineto{\pgfqpoint{5.223003in}{3.045019in}}%
\pgfpathlineto{\pgfqpoint{5.230148in}{3.054126in}}%
\pgfpathlineto{\pgfqpoint{5.237293in}{3.063408in}}%
\pgfpathlineto{\pgfqpoint{5.223877in}{3.061027in}}%
\pgfpathlineto{\pgfqpoint{5.210474in}{3.058798in}}%
\pgfpathlineto{\pgfqpoint{5.197083in}{3.056721in}}%
\pgfpathlineto{\pgfqpoint{5.183704in}{3.054796in}}%
\pgfpathlineto{\pgfqpoint{5.176539in}{3.044944in}}%
\pgfpathlineto{\pgfqpoint{5.169374in}{3.035274in}}%
\pgfpathlineto{\pgfqpoint{5.162208in}{3.025779in}}%
\pgfpathlineto{\pgfqpoint{5.155041in}{3.016453in}}%
\pgfpathclose%
\pgfusepath{fill}%
\end{pgfscope}%
\begin{pgfscope}%
\pgfpathrectangle{\pgfqpoint{1.254980in}{0.150000in}}{\pgfqpoint{5.490039in}{5.490039in}}%
\pgfusepath{clip}%
\pgfsetbuttcap%
\pgfsetroundjoin%
\definecolor{currentfill}{rgb}{0.258965,0.251537,0.524736}%
\pgfsetfillcolor{currentfill}%
\pgfsetfillopacity{0.700000}%
\pgfsetlinewidth{0.000000pt}%
\definecolor{currentstroke}{rgb}{0.000000,0.000000,0.000000}%
\pgfsetstrokecolor{currentstroke}%
\pgfsetdash{}{0pt}%
\pgfpathmoveto{\pgfqpoint{4.415397in}{2.624506in}}%
\pgfpathlineto{\pgfqpoint{4.428532in}{2.625048in}}%
\pgfpathlineto{\pgfqpoint{4.441676in}{2.625755in}}%
\pgfpathlineto{\pgfqpoint{4.454830in}{2.626629in}}%
\pgfpathlineto{\pgfqpoint{4.467993in}{2.627668in}}%
\pgfpathlineto{\pgfqpoint{4.475392in}{2.636905in}}%
\pgfpathlineto{\pgfqpoint{4.482785in}{2.646188in}}%
\pgfpathlineto{\pgfqpoint{4.490175in}{2.655520in}}%
\pgfpathlineto{\pgfqpoint{4.497560in}{2.664905in}}%
\pgfpathlineto{\pgfqpoint{4.484407in}{2.664174in}}%
\pgfpathlineto{\pgfqpoint{4.471263in}{2.663609in}}%
\pgfpathlineto{\pgfqpoint{4.458129in}{2.663208in}}%
\pgfpathlineto{\pgfqpoint{4.445004in}{2.662974in}}%
\pgfpathlineto{\pgfqpoint{4.437609in}{2.653271in}}%
\pgfpathlineto{\pgfqpoint{4.430209in}{2.643628in}}%
\pgfpathlineto{\pgfqpoint{4.422805in}{2.634041in}}%
\pgfpathlineto{\pgfqpoint{4.415397in}{2.624506in}}%
\pgfpathclose%
\pgfusepath{fill}%
\end{pgfscope}%
\begin{pgfscope}%
\pgfpathrectangle{\pgfqpoint{1.254980in}{0.150000in}}{\pgfqpoint{5.490039in}{5.490039in}}%
\pgfusepath{clip}%
\pgfsetbuttcap%
\pgfsetroundjoin%
\definecolor{currentfill}{rgb}{0.255645,0.260703,0.528312}%
\pgfsetfillcolor{currentfill}%
\pgfsetfillopacity{0.700000}%
\pgfsetlinewidth{0.000000pt}%
\definecolor{currentstroke}{rgb}{0.000000,0.000000,0.000000}%
\pgfsetstrokecolor{currentstroke}%
\pgfsetdash{}{0pt}%
\pgfpathmoveto{\pgfqpoint{2.959381in}{2.670564in}}%
\pgfpathlineto{\pgfqpoint{2.972415in}{2.655591in}}%
\pgfpathlineto{\pgfqpoint{2.985443in}{2.640881in}}%
\pgfpathlineto{\pgfqpoint{2.998466in}{2.626430in}}%
\pgfpathlineto{\pgfqpoint{3.011484in}{2.612238in}}%
\pgfpathlineto{\pgfqpoint{3.019360in}{2.621394in}}%
\pgfpathlineto{\pgfqpoint{3.027228in}{2.630651in}}%
\pgfpathlineto{\pgfqpoint{3.035089in}{2.640009in}}%
\pgfpathlineto{\pgfqpoint{3.042943in}{2.649469in}}%
\pgfpathlineto{\pgfqpoint{3.029942in}{2.663632in}}%
\pgfpathlineto{\pgfqpoint{3.016936in}{2.678053in}}%
\pgfpathlineto{\pgfqpoint{3.003925in}{2.692733in}}%
\pgfpathlineto{\pgfqpoint{2.990908in}{2.707676in}}%
\pgfpathlineto{\pgfqpoint{2.983038in}{2.698236in}}%
\pgfpathlineto{\pgfqpoint{2.975160in}{2.688903in}}%
\pgfpathlineto{\pgfqpoint{2.967274in}{2.679680in}}%
\pgfpathlineto{\pgfqpoint{2.959381in}{2.670564in}}%
\pgfpathclose%
\pgfusepath{fill}%
\end{pgfscope}%
\begin{pgfscope}%
\pgfpathrectangle{\pgfqpoint{1.254980in}{0.150000in}}{\pgfqpoint{5.490039in}{5.490039in}}%
\pgfusepath{clip}%
\pgfsetbuttcap%
\pgfsetroundjoin%
\definecolor{currentfill}{rgb}{0.244972,0.287675,0.537260}%
\pgfsetfillcolor{currentfill}%
\pgfsetfillopacity{0.700000}%
\pgfsetlinewidth{0.000000pt}%
\definecolor{currentstroke}{rgb}{0.000000,0.000000,0.000000}%
\pgfsetstrokecolor{currentstroke}%
\pgfsetdash{}{0pt}%
\pgfpathmoveto{\pgfqpoint{2.907185in}{2.733127in}}%
\pgfpathlineto{\pgfqpoint{2.920244in}{2.717081in}}%
\pgfpathlineto{\pgfqpoint{2.933296in}{2.701306in}}%
\pgfpathlineto{\pgfqpoint{2.946341in}{2.685802in}}%
\pgfpathlineto{\pgfqpoint{2.959381in}{2.670564in}}%
\pgfpathlineto{\pgfqpoint{2.967274in}{2.679680in}}%
\pgfpathlineto{\pgfqpoint{2.975160in}{2.688903in}}%
\pgfpathlineto{\pgfqpoint{2.983038in}{2.698236in}}%
\pgfpathlineto{\pgfqpoint{2.990908in}{2.707676in}}%
\pgfpathlineto{\pgfqpoint{2.977886in}{2.722884in}}%
\pgfpathlineto{\pgfqpoint{2.964858in}{2.738358in}}%
\pgfpathlineto{\pgfqpoint{2.951824in}{2.754102in}}%
\pgfpathlineto{\pgfqpoint{2.938783in}{2.770117in}}%
\pgfpathlineto{\pgfqpoint{2.930895in}{2.760697in}}%
\pgfpathlineto{\pgfqpoint{2.923000in}{2.751391in}}%
\pgfpathlineto{\pgfqpoint{2.915096in}{2.742201in}}%
\pgfpathlineto{\pgfqpoint{2.907185in}{2.733127in}}%
\pgfpathclose%
\pgfusepath{fill}%
\end{pgfscope}%
\begin{pgfscope}%
\pgfpathrectangle{\pgfqpoint{1.254980in}{0.150000in}}{\pgfqpoint{5.490039in}{5.490039in}}%
\pgfusepath{clip}%
\pgfsetbuttcap%
\pgfsetroundjoin%
\definecolor{currentfill}{rgb}{0.175841,0.441290,0.557685}%
\pgfsetfillcolor{currentfill}%
\pgfsetfillopacity{0.700000}%
\pgfsetlinewidth{0.000000pt}%
\definecolor{currentstroke}{rgb}{0.000000,0.000000,0.000000}%
\pgfsetstrokecolor{currentstroke}%
\pgfsetdash{}{0pt}%
\pgfpathmoveto{\pgfqpoint{5.237293in}{3.063408in}}%
\pgfpathlineto{\pgfqpoint{5.250721in}{3.065941in}}%
\pgfpathlineto{\pgfqpoint{5.264162in}{3.068625in}}%
\pgfpathlineto{\pgfqpoint{5.277616in}{3.071461in}}%
\pgfpathlineto{\pgfqpoint{5.291082in}{3.074449in}}%
\pgfpathlineto{\pgfqpoint{5.298205in}{3.083334in}}%
\pgfpathlineto{\pgfqpoint{5.305327in}{3.092400in}}%
\pgfpathlineto{\pgfqpoint{5.312449in}{3.101655in}}%
\pgfpathlineto{\pgfqpoint{5.319572in}{3.111105in}}%
\pgfpathlineto{\pgfqpoint{5.306127in}{3.108706in}}%
\pgfpathlineto{\pgfqpoint{5.292695in}{3.106459in}}%
\pgfpathlineto{\pgfqpoint{5.279276in}{3.104363in}}%
\pgfpathlineto{\pgfqpoint{5.265869in}{3.102417in}}%
\pgfpathlineto{\pgfqpoint{5.258725in}{3.092370in}}%
\pgfpathlineto{\pgfqpoint{5.251581in}{3.082523in}}%
\pgfpathlineto{\pgfqpoint{5.244437in}{3.072872in}}%
\pgfpathlineto{\pgfqpoint{5.237293in}{3.063408in}}%
\pgfpathclose%
\pgfusepath{fill}%
\end{pgfscope}%
\begin{pgfscope}%
\pgfpathrectangle{\pgfqpoint{1.254980in}{0.150000in}}{\pgfqpoint{5.490039in}{5.490039in}}%
\pgfusepath{clip}%
\pgfsetbuttcap%
\pgfsetroundjoin%
\definecolor{currentfill}{rgb}{0.265145,0.232956,0.516599}%
\pgfsetfillcolor{currentfill}%
\pgfsetfillopacity{0.700000}%
\pgfsetlinewidth{0.000000pt}%
\definecolor{currentstroke}{rgb}{0.000000,0.000000,0.000000}%
\pgfsetstrokecolor{currentstroke}%
\pgfsetdash{}{0pt}%
\pgfpathmoveto{\pgfqpoint{4.333227in}{2.585203in}}%
\pgfpathlineto{\pgfqpoint{4.346337in}{2.585355in}}%
\pgfpathlineto{\pgfqpoint{4.359455in}{2.585676in}}%
\pgfpathlineto{\pgfqpoint{4.372582in}{2.586164in}}%
\pgfpathlineto{\pgfqpoint{4.385718in}{2.586820in}}%
\pgfpathlineto{\pgfqpoint{4.393144in}{2.596180in}}%
\pgfpathlineto{\pgfqpoint{4.400566in}{2.605579in}}%
\pgfpathlineto{\pgfqpoint{4.407984in}{2.615020in}}%
\pgfpathlineto{\pgfqpoint{4.415397in}{2.624506in}}%
\pgfpathlineto{\pgfqpoint{4.402270in}{2.624131in}}%
\pgfpathlineto{\pgfqpoint{4.389153in}{2.623923in}}%
\pgfpathlineto{\pgfqpoint{4.376045in}{2.623882in}}%
\pgfpathlineto{\pgfqpoint{4.362945in}{2.624009in}}%
\pgfpathlineto{\pgfqpoint{4.355522in}{2.614233in}}%
\pgfpathlineto{\pgfqpoint{4.348095in}{2.604509in}}%
\pgfpathlineto{\pgfqpoint{4.340664in}{2.594833in}}%
\pgfpathlineto{\pgfqpoint{4.333227in}{2.585203in}}%
\pgfpathclose%
\pgfusepath{fill}%
\end{pgfscope}%
\begin{pgfscope}%
\pgfpathrectangle{\pgfqpoint{1.254980in}{0.150000in}}{\pgfqpoint{5.490039in}{5.490039in}}%
\pgfusepath{clip}%
\pgfsetbuttcap%
\pgfsetroundjoin%
\definecolor{currentfill}{rgb}{0.265145,0.232956,0.516599}%
\pgfsetfillcolor{currentfill}%
\pgfsetfillopacity{0.700000}%
\pgfsetlinewidth{0.000000pt}%
\definecolor{currentstroke}{rgb}{0.000000,0.000000,0.000000}%
\pgfsetstrokecolor{currentstroke}%
\pgfsetdash{}{0pt}%
\pgfpathmoveto{\pgfqpoint{3.011484in}{2.612238in}}%
\pgfpathlineto{\pgfqpoint{3.024497in}{2.598302in}}%
\pgfpathlineto{\pgfqpoint{3.037505in}{2.584619in}}%
\pgfpathlineto{\pgfqpoint{3.050509in}{2.571189in}}%
\pgfpathlineto{\pgfqpoint{3.063509in}{2.558008in}}%
\pgfpathlineto{\pgfqpoint{3.071369in}{2.567203in}}%
\pgfpathlineto{\pgfqpoint{3.079221in}{2.576493in}}%
\pgfpathlineto{\pgfqpoint{3.087065in}{2.585877in}}%
\pgfpathlineto{\pgfqpoint{3.094903in}{2.595355in}}%
\pgfpathlineto{\pgfqpoint{3.081919in}{2.608508in}}%
\pgfpathlineto{\pgfqpoint{3.068932in}{2.621909in}}%
\pgfpathlineto{\pgfqpoint{3.055940in}{2.635563in}}%
\pgfpathlineto{\pgfqpoint{3.042943in}{2.649469in}}%
\pgfpathlineto{\pgfqpoint{3.035089in}{2.640009in}}%
\pgfpathlineto{\pgfqpoint{3.027228in}{2.630651in}}%
\pgfpathlineto{\pgfqpoint{3.019360in}{2.621394in}}%
\pgfpathlineto{\pgfqpoint{3.011484in}{2.612238in}}%
\pgfpathclose%
\pgfusepath{fill}%
\end{pgfscope}%
\begin{pgfscope}%
\pgfpathrectangle{\pgfqpoint{1.254980in}{0.150000in}}{\pgfqpoint{5.490039in}{5.490039in}}%
\pgfusepath{clip}%
\pgfsetbuttcap%
\pgfsetroundjoin%
\definecolor{currentfill}{rgb}{0.166617,0.463708,0.558119}%
\pgfsetfillcolor{currentfill}%
\pgfsetfillopacity{0.700000}%
\pgfsetlinewidth{0.000000pt}%
\definecolor{currentstroke}{rgb}{0.000000,0.000000,0.000000}%
\pgfsetstrokecolor{currentstroke}%
\pgfsetdash{}{0pt}%
\pgfpathmoveto{\pgfqpoint{5.319572in}{3.111105in}}%
\pgfpathlineto{\pgfqpoint{5.333029in}{3.113654in}}%
\pgfpathlineto{\pgfqpoint{5.346499in}{3.116353in}}%
\pgfpathlineto{\pgfqpoint{5.359982in}{3.119204in}}%
\pgfpathlineto{\pgfqpoint{5.373478in}{3.122204in}}%
\pgfpathlineto{\pgfqpoint{5.380579in}{3.131249in}}%
\pgfpathlineto{\pgfqpoint{5.387679in}{3.140495in}}%
\pgfpathlineto{\pgfqpoint{5.394781in}{3.149951in}}%
\pgfpathlineto{\pgfqpoint{5.401883in}{3.159622in}}%
\pgfpathlineto{\pgfqpoint{5.388411in}{3.157239in}}%
\pgfpathlineto{\pgfqpoint{5.374951in}{3.155005in}}%
\pgfpathlineto{\pgfqpoint{5.361503in}{3.152921in}}%
\pgfpathlineto{\pgfqpoint{5.348069in}{3.150988in}}%
\pgfpathlineto{\pgfqpoint{5.340943in}{3.140691in}}%
\pgfpathlineto{\pgfqpoint{5.333818in}{3.130616in}}%
\pgfpathlineto{\pgfqpoint{5.326695in}{3.120756in}}%
\pgfpathlineto{\pgfqpoint{5.319572in}{3.111105in}}%
\pgfpathclose%
\pgfusepath{fill}%
\end{pgfscope}%
\begin{pgfscope}%
\pgfpathrectangle{\pgfqpoint{1.254980in}{0.150000in}}{\pgfqpoint{5.490039in}{5.490039in}}%
\pgfusepath{clip}%
\pgfsetbuttcap%
\pgfsetroundjoin%
\definecolor{currentfill}{rgb}{0.231674,0.318106,0.544834}%
\pgfsetfillcolor{currentfill}%
\pgfsetfillopacity{0.700000}%
\pgfsetlinewidth{0.000000pt}%
\definecolor{currentstroke}{rgb}{0.000000,0.000000,0.000000}%
\pgfsetstrokecolor{currentstroke}%
\pgfsetdash{}{0pt}%
\pgfpathmoveto{\pgfqpoint{2.854880in}{2.800079in}}%
\pgfpathlineto{\pgfqpoint{2.867967in}{2.782921in}}%
\pgfpathlineto{\pgfqpoint{2.881047in}{2.766044in}}%
\pgfpathlineto{\pgfqpoint{2.894120in}{2.749447in}}%
\pgfpathlineto{\pgfqpoint{2.907185in}{2.733127in}}%
\pgfpathlineto{\pgfqpoint{2.915096in}{2.742201in}}%
\pgfpathlineto{\pgfqpoint{2.923000in}{2.751391in}}%
\pgfpathlineto{\pgfqpoint{2.930895in}{2.760697in}}%
\pgfpathlineto{\pgfqpoint{2.938783in}{2.770117in}}%
\pgfpathlineto{\pgfqpoint{2.925736in}{2.786407in}}%
\pgfpathlineto{\pgfqpoint{2.912682in}{2.802973in}}%
\pgfpathlineto{\pgfqpoint{2.899621in}{2.819818in}}%
\pgfpathlineto{\pgfqpoint{2.886552in}{2.836945in}}%
\pgfpathlineto{\pgfqpoint{2.878646in}{2.827545in}}%
\pgfpathlineto{\pgfqpoint{2.870732in}{2.818267in}}%
\pgfpathlineto{\pgfqpoint{2.862810in}{2.809112in}}%
\pgfpathlineto{\pgfqpoint{2.854880in}{2.800079in}}%
\pgfpathclose%
\pgfusepath{fill}%
\end{pgfscope}%
\begin{pgfscope}%
\pgfpathrectangle{\pgfqpoint{1.254980in}{0.150000in}}{\pgfqpoint{5.490039in}{5.490039in}}%
\pgfusepath{clip}%
\pgfsetbuttcap%
\pgfsetroundjoin%
\definecolor{currentfill}{rgb}{0.283072,0.130895,0.449241}%
\pgfsetfillcolor{currentfill}%
\pgfsetfillopacity{0.700000}%
\pgfsetlinewidth{0.000000pt}%
\definecolor{currentstroke}{rgb}{0.000000,0.000000,0.000000}%
\pgfsetstrokecolor{currentstroke}%
\pgfsetdash{}{0pt}%
\pgfpathmoveto{\pgfqpoint{3.436836in}{2.388718in}}%
\pgfpathlineto{\pgfqpoint{3.449777in}{2.381237in}}%
\pgfpathlineto{\pgfqpoint{3.462718in}{2.373964in}}%
\pgfpathlineto{\pgfqpoint{3.475661in}{2.366898in}}%
\pgfpathlineto{\pgfqpoint{3.488606in}{2.360037in}}%
\pgfpathlineto{\pgfqpoint{3.496321in}{2.369870in}}%
\pgfpathlineto{\pgfqpoint{3.504029in}{2.379752in}}%
\pgfpathlineto{\pgfqpoint{3.511733in}{2.389684in}}%
\pgfpathlineto{\pgfqpoint{3.519430in}{2.399666in}}%
\pgfpathlineto{\pgfqpoint{3.506496in}{2.406556in}}%
\pgfpathlineto{\pgfqpoint{3.493564in}{2.413652in}}%
\pgfpathlineto{\pgfqpoint{3.480634in}{2.420953in}}%
\pgfpathlineto{\pgfqpoint{3.467705in}{2.428463in}}%
\pgfpathlineto{\pgfqpoint{3.459996in}{2.418441in}}%
\pgfpathlineto{\pgfqpoint{3.452282in}{2.408477in}}%
\pgfpathlineto{\pgfqpoint{3.444562in}{2.398570in}}%
\pgfpathlineto{\pgfqpoint{3.436836in}{2.388718in}}%
\pgfpathclose%
\pgfusepath{fill}%
\end{pgfscope}%
\begin{pgfscope}%
\pgfpathrectangle{\pgfqpoint{1.254980in}{0.150000in}}{\pgfqpoint{5.490039in}{5.490039in}}%
\pgfusepath{clip}%
\pgfsetbuttcap%
\pgfsetroundjoin%
\definecolor{currentfill}{rgb}{0.269308,0.218818,0.509577}%
\pgfsetfillcolor{currentfill}%
\pgfsetfillopacity{0.700000}%
\pgfsetlinewidth{0.000000pt}%
\definecolor{currentstroke}{rgb}{0.000000,0.000000,0.000000}%
\pgfsetstrokecolor{currentstroke}%
\pgfsetdash{}{0pt}%
\pgfpathmoveto{\pgfqpoint{4.251047in}{2.547145in}}%
\pgfpathlineto{\pgfqpoint{4.264132in}{2.546872in}}%
\pgfpathlineto{\pgfqpoint{4.277225in}{2.546769in}}%
\pgfpathlineto{\pgfqpoint{4.290327in}{2.546836in}}%
\pgfpathlineto{\pgfqpoint{4.303437in}{2.547072in}}%
\pgfpathlineto{\pgfqpoint{4.310892in}{2.556552in}}%
\pgfpathlineto{\pgfqpoint{4.318342in}{2.566066in}}%
\pgfpathlineto{\pgfqpoint{4.325787in}{2.575615in}}%
\pgfpathlineto{\pgfqpoint{4.333227in}{2.585203in}}%
\pgfpathlineto{\pgfqpoint{4.320127in}{2.585219in}}%
\pgfpathlineto{\pgfqpoint{4.307034in}{2.585404in}}%
\pgfpathlineto{\pgfqpoint{4.293951in}{2.585759in}}%
\pgfpathlineto{\pgfqpoint{4.280875in}{2.586284in}}%
\pgfpathlineto{\pgfqpoint{4.273425in}{2.576434in}}%
\pgfpathlineto{\pgfqpoint{4.265970in}{2.566629in}}%
\pgfpathlineto{\pgfqpoint{4.258511in}{2.556867in}}%
\pgfpathlineto{\pgfqpoint{4.251047in}{2.547145in}}%
\pgfpathclose%
\pgfusepath{fill}%
\end{pgfscope}%
\begin{pgfscope}%
\pgfpathrectangle{\pgfqpoint{1.254980in}{0.150000in}}{\pgfqpoint{5.490039in}{5.490039in}}%
\pgfusepath{clip}%
\pgfsetbuttcap%
\pgfsetroundjoin%
\definecolor{currentfill}{rgb}{0.159194,0.482237,0.558073}%
\pgfsetfillcolor{currentfill}%
\pgfsetfillopacity{0.700000}%
\pgfsetlinewidth{0.000000pt}%
\definecolor{currentstroke}{rgb}{0.000000,0.000000,0.000000}%
\pgfsetstrokecolor{currentstroke}%
\pgfsetdash{}{0pt}%
\pgfpathmoveto{\pgfqpoint{5.401883in}{3.159622in}}%
\pgfpathlineto{\pgfqpoint{5.415369in}{3.162155in}}%
\pgfpathlineto{\pgfqpoint{5.428868in}{3.164838in}}%
\pgfpathlineto{\pgfqpoint{5.442380in}{3.167670in}}%
\pgfpathlineto{\pgfqpoint{5.455905in}{3.170651in}}%
\pgfpathlineto{\pgfqpoint{5.462984in}{3.179911in}}%
\pgfpathlineto{\pgfqpoint{5.470066in}{3.189394in}}%
\pgfpathlineto{\pgfqpoint{5.477149in}{3.199108in}}%
\pgfpathlineto{\pgfqpoint{5.484234in}{3.209060in}}%
\pgfpathlineto{\pgfqpoint{5.470734in}{3.206724in}}%
\pgfpathlineto{\pgfqpoint{5.457247in}{3.204536in}}%
\pgfpathlineto{\pgfqpoint{5.443772in}{3.202498in}}%
\pgfpathlineto{\pgfqpoint{5.430311in}{3.200608in}}%
\pgfpathlineto{\pgfqpoint{5.423201in}{3.190002in}}%
\pgfpathlineto{\pgfqpoint{5.416093in}{3.179640in}}%
\pgfpathlineto{\pgfqpoint{5.408987in}{3.169516in}}%
\pgfpathlineto{\pgfqpoint{5.401883in}{3.159622in}}%
\pgfpathclose%
\pgfusepath{fill}%
\end{pgfscope}%
\begin{pgfscope}%
\pgfpathrectangle{\pgfqpoint{1.254980in}{0.150000in}}{\pgfqpoint{5.490039in}{5.490039in}}%
\pgfusepath{clip}%
\pgfsetbuttcap%
\pgfsetroundjoin%
\definecolor{currentfill}{rgb}{0.282884,0.135920,0.453427}%
\pgfsetfillcolor{currentfill}%
\pgfsetfillopacity{0.700000}%
\pgfsetlinewidth{0.000000pt}%
\definecolor{currentstroke}{rgb}{0.000000,0.000000,0.000000}%
\pgfsetstrokecolor{currentstroke}%
\pgfsetdash{}{0pt}%
\pgfpathmoveto{\pgfqpoint{3.787825in}{2.395952in}}%
\pgfpathlineto{\pgfqpoint{3.800797in}{2.392242in}}%
\pgfpathlineto{\pgfqpoint{3.813773in}{2.388719in}}%
\pgfpathlineto{\pgfqpoint{3.826753in}{2.385383in}}%
\pgfpathlineto{\pgfqpoint{3.839740in}{2.382231in}}%
\pgfpathlineto{\pgfqpoint{3.847342in}{2.392207in}}%
\pgfpathlineto{\pgfqpoint{3.854940in}{2.402210in}}%
\pgfpathlineto{\pgfqpoint{3.862533in}{2.412244in}}%
\pgfpathlineto{\pgfqpoint{3.870121in}{2.422309in}}%
\pgfpathlineto{\pgfqpoint{3.857144in}{2.425573in}}%
\pgfpathlineto{\pgfqpoint{3.844172in}{2.429023in}}%
\pgfpathlineto{\pgfqpoint{3.831205in}{2.432658in}}%
\pgfpathlineto{\pgfqpoint{3.818243in}{2.436480in}}%
\pgfpathlineto{\pgfqpoint{3.810646in}{2.426292in}}%
\pgfpathlineto{\pgfqpoint{3.803044in}{2.416142in}}%
\pgfpathlineto{\pgfqpoint{3.795437in}{2.406029in}}%
\pgfpathlineto{\pgfqpoint{3.787825in}{2.395952in}}%
\pgfpathclose%
\pgfusepath{fill}%
\end{pgfscope}%
\begin{pgfscope}%
\pgfpathrectangle{\pgfqpoint{1.254980in}{0.150000in}}{\pgfqpoint{5.490039in}{5.490039in}}%
\pgfusepath{clip}%
\pgfsetbuttcap%
\pgfsetroundjoin%
\definecolor{currentfill}{rgb}{0.271828,0.209303,0.504434}%
\pgfsetfillcolor{currentfill}%
\pgfsetfillopacity{0.700000}%
\pgfsetlinewidth{0.000000pt}%
\definecolor{currentstroke}{rgb}{0.000000,0.000000,0.000000}%
\pgfsetstrokecolor{currentstroke}%
\pgfsetdash{}{0pt}%
\pgfpathmoveto{\pgfqpoint{3.063509in}{2.558008in}}%
\pgfpathlineto{\pgfqpoint{3.076505in}{2.545074in}}%
\pgfpathlineto{\pgfqpoint{3.089498in}{2.532386in}}%
\pgfpathlineto{\pgfqpoint{3.102487in}{2.519942in}}%
\pgfpathlineto{\pgfqpoint{3.115472in}{2.507740in}}%
\pgfpathlineto{\pgfqpoint{3.123315in}{2.516975in}}%
\pgfpathlineto{\pgfqpoint{3.131152in}{2.526296in}}%
\pgfpathlineto{\pgfqpoint{3.138981in}{2.535706in}}%
\pgfpathlineto{\pgfqpoint{3.146803in}{2.545202in}}%
\pgfpathlineto{\pgfqpoint{3.133833in}{2.557376in}}%
\pgfpathlineto{\pgfqpoint{3.120860in}{2.569792in}}%
\pgfpathlineto{\pgfqpoint{3.107883in}{2.582451in}}%
\pgfpathlineto{\pgfqpoint{3.094903in}{2.595355in}}%
\pgfpathlineto{\pgfqpoint{3.087065in}{2.585877in}}%
\pgfpathlineto{\pgfqpoint{3.079221in}{2.576493in}}%
\pgfpathlineto{\pgfqpoint{3.071369in}{2.567203in}}%
\pgfpathlineto{\pgfqpoint{3.063509in}{2.558008in}}%
\pgfpathclose%
\pgfusepath{fill}%
\end{pgfscope}%
\begin{pgfscope}%
\pgfpathrectangle{\pgfqpoint{1.254980in}{0.150000in}}{\pgfqpoint{5.490039in}{5.490039in}}%
\pgfusepath{clip}%
\pgfsetbuttcap%
\pgfsetroundjoin%
\definecolor{currentfill}{rgb}{0.283187,0.125848,0.444960}%
\pgfsetfillcolor{currentfill}%
\pgfsetfillopacity{0.700000}%
\pgfsetlinewidth{0.000000pt}%
\definecolor{currentstroke}{rgb}{0.000000,0.000000,0.000000}%
\pgfsetstrokecolor{currentstroke}%
\pgfsetdash{}{0pt}%
\pgfpathmoveto{\pgfqpoint{3.571187in}{2.374133in}}%
\pgfpathlineto{\pgfqpoint{3.584132in}{2.368251in}}%
\pgfpathlineto{\pgfqpoint{3.597080in}{2.362568in}}%
\pgfpathlineto{\pgfqpoint{3.610031in}{2.357082in}}%
\pgfpathlineto{\pgfqpoint{3.622985in}{2.351792in}}%
\pgfpathlineto{\pgfqpoint{3.630657in}{2.361733in}}%
\pgfpathlineto{\pgfqpoint{3.638323in}{2.371712in}}%
\pgfpathlineto{\pgfqpoint{3.645984in}{2.381731in}}%
\pgfpathlineto{\pgfqpoint{3.653640in}{2.391790in}}%
\pgfpathlineto{\pgfqpoint{3.640696in}{2.397137in}}%
\pgfpathlineto{\pgfqpoint{3.627755in}{2.402680in}}%
\pgfpathlineto{\pgfqpoint{3.614817in}{2.408420in}}%
\pgfpathlineto{\pgfqpoint{3.601882in}{2.414359in}}%
\pgfpathlineto{\pgfqpoint{3.594216in}{2.404232in}}%
\pgfpathlineto{\pgfqpoint{3.586545in}{2.394153in}}%
\pgfpathlineto{\pgfqpoint{3.578868in}{2.384120in}}%
\pgfpathlineto{\pgfqpoint{3.571187in}{2.374133in}}%
\pgfpathclose%
\pgfusepath{fill}%
\end{pgfscope}%
\begin{pgfscope}%
\pgfpathrectangle{\pgfqpoint{1.254980in}{0.150000in}}{\pgfqpoint{5.490039in}{5.490039in}}%
\pgfusepath{clip}%
\pgfsetbuttcap%
\pgfsetroundjoin%
\definecolor{currentfill}{rgb}{0.216210,0.351535,0.550627}%
\pgfsetfillcolor{currentfill}%
\pgfsetfillopacity{0.700000}%
\pgfsetlinewidth{0.000000pt}%
\definecolor{currentstroke}{rgb}{0.000000,0.000000,0.000000}%
\pgfsetstrokecolor{currentstroke}%
\pgfsetdash{}{0pt}%
\pgfpathmoveto{\pgfqpoint{2.802449in}{2.871585in}}%
\pgfpathlineto{\pgfqpoint{2.815569in}{2.853272in}}%
\pgfpathlineto{\pgfqpoint{2.828681in}{2.835252in}}%
\pgfpathlineto{\pgfqpoint{2.841785in}{2.817522in}}%
\pgfpathlineto{\pgfqpoint{2.854880in}{2.800079in}}%
\pgfpathlineto{\pgfqpoint{2.862810in}{2.809112in}}%
\pgfpathlineto{\pgfqpoint{2.870732in}{2.818267in}}%
\pgfpathlineto{\pgfqpoint{2.878646in}{2.827545in}}%
\pgfpathlineto{\pgfqpoint{2.886552in}{2.836945in}}%
\pgfpathlineto{\pgfqpoint{2.873476in}{2.854357in}}%
\pgfpathlineto{\pgfqpoint{2.860391in}{2.872055in}}%
\pgfpathlineto{\pgfqpoint{2.847299in}{2.890043in}}%
\pgfpathlineto{\pgfqpoint{2.834198in}{2.908324in}}%
\pgfpathlineto{\pgfqpoint{2.826273in}{2.898945in}}%
\pgfpathlineto{\pgfqpoint{2.818340in}{2.889695in}}%
\pgfpathlineto{\pgfqpoint{2.810399in}{2.880575in}}%
\pgfpathlineto{\pgfqpoint{2.802449in}{2.871585in}}%
\pgfpathclose%
\pgfusepath{fill}%
\end{pgfscope}%
\begin{pgfscope}%
\pgfpathrectangle{\pgfqpoint{1.254980in}{0.150000in}}{\pgfqpoint{5.490039in}{5.490039in}}%
\pgfusepath{clip}%
\pgfsetbuttcap%
\pgfsetroundjoin%
\definecolor{currentfill}{rgb}{0.274128,0.199721,0.498911}%
\pgfsetfillcolor{currentfill}%
\pgfsetfillopacity{0.700000}%
\pgfsetlinewidth{0.000000pt}%
\definecolor{currentstroke}{rgb}{0.000000,0.000000,0.000000}%
\pgfsetstrokecolor{currentstroke}%
\pgfsetdash{}{0pt}%
\pgfpathmoveto{\pgfqpoint{4.168851in}{2.510503in}}%
\pgfpathlineto{\pgfqpoint{4.181913in}{2.509767in}}%
\pgfpathlineto{\pgfqpoint{4.194982in}{2.509204in}}%
\pgfpathlineto{\pgfqpoint{4.208060in}{2.508813in}}%
\pgfpathlineto{\pgfqpoint{4.221146in}{2.508593in}}%
\pgfpathlineto{\pgfqpoint{4.228628in}{2.518186in}}%
\pgfpathlineto{\pgfqpoint{4.236106in}{2.527807in}}%
\pgfpathlineto{\pgfqpoint{4.243579in}{2.537459in}}%
\pgfpathlineto{\pgfqpoint{4.251047in}{2.547145in}}%
\pgfpathlineto{\pgfqpoint{4.237971in}{2.547589in}}%
\pgfpathlineto{\pgfqpoint{4.224902in}{2.548204in}}%
\pgfpathlineto{\pgfqpoint{4.211841in}{2.548992in}}%
\pgfpathlineto{\pgfqpoint{4.198788in}{2.549951in}}%
\pgfpathlineto{\pgfqpoint{4.191311in}{2.540031in}}%
\pgfpathlineto{\pgfqpoint{4.183829in}{2.530152in}}%
\pgfpathlineto{\pgfqpoint{4.176342in}{2.520310in}}%
\pgfpathlineto{\pgfqpoint{4.168851in}{2.510503in}}%
\pgfpathclose%
\pgfusepath{fill}%
\end{pgfscope}%
\begin{pgfscope}%
\pgfpathrectangle{\pgfqpoint{1.254980in}{0.150000in}}{\pgfqpoint{5.490039in}{5.490039in}}%
\pgfusepath{clip}%
\pgfsetbuttcap%
\pgfsetroundjoin%
\definecolor{currentfill}{rgb}{0.282290,0.145912,0.461510}%
\pgfsetfillcolor{currentfill}%
\pgfsetfillopacity{0.700000}%
\pgfsetlinewidth{0.000000pt}%
\definecolor{currentstroke}{rgb}{0.000000,0.000000,0.000000}%
\pgfsetstrokecolor{currentstroke}%
\pgfsetdash{}{0pt}%
\pgfpathmoveto{\pgfqpoint{3.302282in}{2.417328in}}%
\pgfpathlineto{\pgfqpoint{3.315230in}{2.408139in}}%
\pgfpathlineto{\pgfqpoint{3.328179in}{2.399169in}}%
\pgfpathlineto{\pgfqpoint{3.341128in}{2.390415in}}%
\pgfpathlineto{\pgfqpoint{3.354077in}{2.381878in}}%
\pgfpathlineto{\pgfqpoint{3.361838in}{2.391503in}}%
\pgfpathlineto{\pgfqpoint{3.369593in}{2.401188in}}%
\pgfpathlineto{\pgfqpoint{3.377342in}{2.410935in}}%
\pgfpathlineto{\pgfqpoint{3.385085in}{2.420743in}}%
\pgfpathlineto{\pgfqpoint{3.372149in}{2.429281in}}%
\pgfpathlineto{\pgfqpoint{3.359213in}{2.438035in}}%
\pgfpathlineto{\pgfqpoint{3.346277in}{2.447006in}}%
\pgfpathlineto{\pgfqpoint{3.333340in}{2.456196in}}%
\pgfpathlineto{\pgfqpoint{3.325585in}{2.446377in}}%
\pgfpathlineto{\pgfqpoint{3.317823in}{2.436626in}}%
\pgfpathlineto{\pgfqpoint{3.310055in}{2.426943in}}%
\pgfpathlineto{\pgfqpoint{3.302282in}{2.417328in}}%
\pgfpathclose%
\pgfusepath{fill}%
\end{pgfscope}%
\begin{pgfscope}%
\pgfpathrectangle{\pgfqpoint{1.254980in}{0.150000in}}{\pgfqpoint{5.490039in}{5.490039in}}%
\pgfusepath{clip}%
\pgfsetbuttcap%
\pgfsetroundjoin%
\definecolor{currentfill}{rgb}{0.277134,0.185228,0.489898}%
\pgfsetfillcolor{currentfill}%
\pgfsetfillopacity{0.700000}%
\pgfsetlinewidth{0.000000pt}%
\definecolor{currentstroke}{rgb}{0.000000,0.000000,0.000000}%
\pgfsetstrokecolor{currentstroke}%
\pgfsetdash{}{0pt}%
\pgfpathmoveto{\pgfqpoint{4.086630in}{2.475467in}}%
\pgfpathlineto{\pgfqpoint{4.099671in}{2.474231in}}%
\pgfpathlineto{\pgfqpoint{4.112720in}{2.473171in}}%
\pgfpathlineto{\pgfqpoint{4.125775in}{2.472285in}}%
\pgfpathlineto{\pgfqpoint{4.138838in}{2.471573in}}%
\pgfpathlineto{\pgfqpoint{4.146348in}{2.481265in}}%
\pgfpathlineto{\pgfqpoint{4.153854in}{2.490983in}}%
\pgfpathlineto{\pgfqpoint{4.161355in}{2.500728in}}%
\pgfpathlineto{\pgfqpoint{4.168851in}{2.510503in}}%
\pgfpathlineto{\pgfqpoint{4.155796in}{2.511412in}}%
\pgfpathlineto{\pgfqpoint{4.142749in}{2.512494in}}%
\pgfpathlineto{\pgfqpoint{4.129710in}{2.513751in}}%
\pgfpathlineto{\pgfqpoint{4.116677in}{2.515183in}}%
\pgfpathlineto{\pgfqpoint{4.109173in}{2.505202in}}%
\pgfpathlineto{\pgfqpoint{4.101663in}{2.495257in}}%
\pgfpathlineto{\pgfqpoint{4.094149in}{2.485346in}}%
\pgfpathlineto{\pgfqpoint{4.086630in}{2.475467in}}%
\pgfpathclose%
\pgfusepath{fill}%
\end{pgfscope}%
\begin{pgfscope}%
\pgfpathrectangle{\pgfqpoint{1.254980in}{0.150000in}}{\pgfqpoint{5.490039in}{5.490039in}}%
\pgfusepath{clip}%
\pgfsetbuttcap%
\pgfsetroundjoin%
\definecolor{currentfill}{rgb}{0.151918,0.500685,0.557587}%
\pgfsetfillcolor{currentfill}%
\pgfsetfillopacity{0.700000}%
\pgfsetlinewidth{0.000000pt}%
\definecolor{currentstroke}{rgb}{0.000000,0.000000,0.000000}%
\pgfsetstrokecolor{currentstroke}%
\pgfsetdash{}{0pt}%
\pgfpathmoveto{\pgfqpoint{5.484234in}{3.209060in}}%
\pgfpathlineto{\pgfqpoint{5.497747in}{3.211545in}}%
\pgfpathlineto{\pgfqpoint{5.511274in}{3.214179in}}%
\pgfpathlineto{\pgfqpoint{5.524814in}{3.216961in}}%
\pgfpathlineto{\pgfqpoint{5.538367in}{3.219892in}}%
\pgfpathlineto{\pgfqpoint{5.545429in}{3.229427in}}%
\pgfpathlineto{\pgfqpoint{5.552493in}{3.239209in}}%
\pgfpathlineto{\pgfqpoint{5.559561in}{3.249244in}}%
\pgfpathlineto{\pgfqpoint{5.546027in}{3.246816in}}%
\pgfpathlineto{\pgfqpoint{5.532507in}{3.244536in}}%
\pgfpathlineto{\pgfqpoint{5.519000in}{3.242403in}}%
\pgfpathlineto{\pgfqpoint{5.505506in}{3.240419in}}%
\pgfpathlineto{\pgfqpoint{5.498412in}{3.229708in}}%
\pgfpathlineto{\pgfqpoint{5.491322in}{3.219258in}}%
\pgfpathlineto{\pgfqpoint{5.484234in}{3.209060in}}%
\pgfpathclose%
\pgfusepath{fill}%
\end{pgfscope}%
\begin{pgfscope}%
\pgfpathrectangle{\pgfqpoint{1.254980in}{0.150000in}}{\pgfqpoint{5.490039in}{5.490039in}}%
\pgfusepath{clip}%
\pgfsetbuttcap%
\pgfsetroundjoin%
\definecolor{currentfill}{rgb}{0.276194,0.190074,0.493001}%
\pgfsetfillcolor{currentfill}%
\pgfsetfillopacity{0.700000}%
\pgfsetlinewidth{0.000000pt}%
\definecolor{currentstroke}{rgb}{0.000000,0.000000,0.000000}%
\pgfsetstrokecolor{currentstroke}%
\pgfsetdash{}{0pt}%
\pgfpathmoveto{\pgfqpoint{3.115472in}{2.507740in}}%
\pgfpathlineto{\pgfqpoint{3.128455in}{2.495778in}}%
\pgfpathlineto{\pgfqpoint{3.141435in}{2.484054in}}%
\pgfpathlineto{\pgfqpoint{3.154412in}{2.472567in}}%
\pgfpathlineto{\pgfqpoint{3.167387in}{2.461314in}}%
\pgfpathlineto{\pgfqpoint{3.175214in}{2.470587in}}%
\pgfpathlineto{\pgfqpoint{3.183036in}{2.479940in}}%
\pgfpathlineto{\pgfqpoint{3.190850in}{2.489373in}}%
\pgfpathlineto{\pgfqpoint{3.198658in}{2.498888in}}%
\pgfpathlineto{\pgfqpoint{3.185698in}{2.510113in}}%
\pgfpathlineto{\pgfqpoint{3.172735in}{2.521573in}}%
\pgfpathlineto{\pgfqpoint{3.159771in}{2.533268in}}%
\pgfpathlineto{\pgfqpoint{3.146803in}{2.545202in}}%
\pgfpathlineto{\pgfqpoint{3.138981in}{2.535706in}}%
\pgfpathlineto{\pgfqpoint{3.131152in}{2.526296in}}%
\pgfpathlineto{\pgfqpoint{3.123315in}{2.516975in}}%
\pgfpathlineto{\pgfqpoint{3.115472in}{2.507740in}}%
\pgfpathclose%
\pgfusepath{fill}%
\end{pgfscope}%
\begin{pgfscope}%
\pgfpathrectangle{\pgfqpoint{1.254980in}{0.150000in}}{\pgfqpoint{5.490039in}{5.490039in}}%
\pgfusepath{clip}%
\pgfsetbuttcap%
\pgfsetroundjoin%
\definecolor{currentfill}{rgb}{0.201239,0.383670,0.554294}%
\pgfsetfillcolor{currentfill}%
\pgfsetfillopacity{0.700000}%
\pgfsetlinewidth{0.000000pt}%
\definecolor{currentstroke}{rgb}{0.000000,0.000000,0.000000}%
\pgfsetstrokecolor{currentstroke}%
\pgfsetdash{}{0pt}%
\pgfpathmoveto{\pgfqpoint{2.749873in}{2.947820in}}%
\pgfpathlineto{\pgfqpoint{2.763032in}{2.928308in}}%
\pgfpathlineto{\pgfqpoint{2.776180in}{2.909100in}}%
\pgfpathlineto{\pgfqpoint{2.789319in}{2.890193in}}%
\pgfpathlineto{\pgfqpoint{2.802449in}{2.871585in}}%
\pgfpathlineto{\pgfqpoint{2.810399in}{2.880575in}}%
\pgfpathlineto{\pgfqpoint{2.818340in}{2.889695in}}%
\pgfpathlineto{\pgfqpoint{2.826273in}{2.898945in}}%
\pgfpathlineto{\pgfqpoint{2.834198in}{2.908324in}}%
\pgfpathlineto{\pgfqpoint{2.821088in}{2.926900in}}%
\pgfpathlineto{\pgfqpoint{2.807969in}{2.945774in}}%
\pgfpathlineto{\pgfqpoint{2.794840in}{2.964950in}}%
\pgfpathlineto{\pgfqpoint{2.781702in}{2.984429in}}%
\pgfpathlineto{\pgfqpoint{2.773758in}{2.975072in}}%
\pgfpathlineto{\pgfqpoint{2.765806in}{2.965851in}}%
\pgfpathlineto{\pgfqpoint{2.757844in}{2.956768in}}%
\pgfpathlineto{\pgfqpoint{2.749873in}{2.947820in}}%
\pgfpathclose%
\pgfusepath{fill}%
\end{pgfscope}%
\begin{pgfscope}%
\pgfpathrectangle{\pgfqpoint{1.254980in}{0.150000in}}{\pgfqpoint{5.490039in}{5.490039in}}%
\pgfusepath{clip}%
\pgfsetbuttcap%
\pgfsetroundjoin%
\definecolor{currentfill}{rgb}{0.283187,0.125848,0.444960}%
\pgfsetfillcolor{currentfill}%
\pgfsetfillopacity{0.700000}%
\pgfsetlinewidth{0.000000pt}%
\definecolor{currentstroke}{rgb}{0.000000,0.000000,0.000000}%
\pgfsetstrokecolor{currentstroke}%
\pgfsetdash{}{0pt}%
\pgfpathmoveto{\pgfqpoint{3.705452in}{2.372345in}}%
\pgfpathlineto{\pgfqpoint{3.718415in}{2.367964in}}%
\pgfpathlineto{\pgfqpoint{3.731381in}{2.363774in}}%
\pgfpathlineto{\pgfqpoint{3.744353in}{2.359773in}}%
\pgfpathlineto{\pgfqpoint{3.757328in}{2.355962in}}%
\pgfpathlineto{\pgfqpoint{3.764960in}{2.365913in}}%
\pgfpathlineto{\pgfqpoint{3.772587in}{2.375895in}}%
\pgfpathlineto{\pgfqpoint{3.780209in}{2.385907in}}%
\pgfpathlineto{\pgfqpoint{3.787825in}{2.395952in}}%
\pgfpathlineto{\pgfqpoint{3.774859in}{2.399849in}}%
\pgfpathlineto{\pgfqpoint{3.761897in}{2.403934in}}%
\pgfpathlineto{\pgfqpoint{3.748939in}{2.408210in}}%
\pgfpathlineto{\pgfqpoint{3.735986in}{2.412675in}}%
\pgfpathlineto{\pgfqpoint{3.728360in}{2.402536in}}%
\pgfpathlineto{\pgfqpoint{3.720729in}{2.392435in}}%
\pgfpathlineto{\pgfqpoint{3.713093in}{2.382372in}}%
\pgfpathlineto{\pgfqpoint{3.705452in}{2.372345in}}%
\pgfpathclose%
\pgfusepath{fill}%
\end{pgfscope}%
\begin{pgfscope}%
\pgfpathrectangle{\pgfqpoint{1.254980in}{0.150000in}}{\pgfqpoint{5.490039in}{5.490039in}}%
\pgfusepath{clip}%
\pgfsetbuttcap%
\pgfsetroundjoin%
\definecolor{currentfill}{rgb}{0.280255,0.165693,0.476498}%
\pgfsetfillcolor{currentfill}%
\pgfsetfillopacity{0.700000}%
\pgfsetlinewidth{0.000000pt}%
\definecolor{currentstroke}{rgb}{0.000000,0.000000,0.000000}%
\pgfsetstrokecolor{currentstroke}%
\pgfsetdash{}{0pt}%
\pgfpathmoveto{\pgfqpoint{4.004379in}{2.442250in}}%
\pgfpathlineto{\pgfqpoint{4.017401in}{2.440477in}}%
\pgfpathlineto{\pgfqpoint{4.030429in}{2.438881in}}%
\pgfpathlineto{\pgfqpoint{4.043465in}{2.437462in}}%
\pgfpathlineto{\pgfqpoint{4.056507in}{2.436220in}}%
\pgfpathlineto{\pgfqpoint{4.064045in}{2.445996in}}%
\pgfpathlineto{\pgfqpoint{4.071578in}{2.455794in}}%
\pgfpathlineto{\pgfqpoint{4.079107in}{2.465617in}}%
\pgfpathlineto{\pgfqpoint{4.086630in}{2.475467in}}%
\pgfpathlineto{\pgfqpoint{4.073597in}{2.476878in}}%
\pgfpathlineto{\pgfqpoint{4.060570in}{2.478466in}}%
\pgfpathlineto{\pgfqpoint{4.047549in}{2.480230in}}%
\pgfpathlineto{\pgfqpoint{4.034536in}{2.482172in}}%
\pgfpathlineto{\pgfqpoint{4.027004in}{2.472144in}}%
\pgfpathlineto{\pgfqpoint{4.019467in}{2.462148in}}%
\pgfpathlineto{\pgfqpoint{4.011925in}{2.452185in}}%
\pgfpathlineto{\pgfqpoint{4.004379in}{2.442250in}}%
\pgfpathclose%
\pgfusepath{fill}%
\end{pgfscope}%
\begin{pgfscope}%
\pgfpathrectangle{\pgfqpoint{1.254980in}{0.150000in}}{\pgfqpoint{5.490039in}{5.490039in}}%
\pgfusepath{clip}%
\pgfsetbuttcap%
\pgfsetroundjoin%
\definecolor{currentfill}{rgb}{0.279574,0.170599,0.479997}%
\pgfsetfillcolor{currentfill}%
\pgfsetfillopacity{0.700000}%
\pgfsetlinewidth{0.000000pt}%
\definecolor{currentstroke}{rgb}{0.000000,0.000000,0.000000}%
\pgfsetstrokecolor{currentstroke}%
\pgfsetdash{}{0pt}%
\pgfpathmoveto{\pgfqpoint{3.167387in}{2.461314in}}%
\pgfpathlineto{\pgfqpoint{3.180359in}{2.450294in}}%
\pgfpathlineto{\pgfqpoint{3.193330in}{2.439505in}}%
\pgfpathlineto{\pgfqpoint{3.206299in}{2.428946in}}%
\pgfpathlineto{\pgfqpoint{3.219266in}{2.418614in}}%
\pgfpathlineto{\pgfqpoint{3.227079in}{2.427925in}}%
\pgfpathlineto{\pgfqpoint{3.234886in}{2.437309in}}%
\pgfpathlineto{\pgfqpoint{3.242686in}{2.446767in}}%
\pgfpathlineto{\pgfqpoint{3.250479in}{2.456298in}}%
\pgfpathlineto{\pgfqpoint{3.237526in}{2.466602in}}%
\pgfpathlineto{\pgfqpoint{3.224572in}{2.477134in}}%
\pgfpathlineto{\pgfqpoint{3.211616in}{2.487895in}}%
\pgfpathlineto{\pgfqpoint{3.198658in}{2.498888in}}%
\pgfpathlineto{\pgfqpoint{3.190850in}{2.489373in}}%
\pgfpathlineto{\pgfqpoint{3.183036in}{2.479940in}}%
\pgfpathlineto{\pgfqpoint{3.175214in}{2.470587in}}%
\pgfpathlineto{\pgfqpoint{3.167387in}{2.461314in}}%
\pgfpathclose%
\pgfusepath{fill}%
\end{pgfscope}%
\begin{pgfscope}%
\pgfpathrectangle{\pgfqpoint{1.254980in}{0.150000in}}{\pgfqpoint{5.490039in}{5.490039in}}%
\pgfusepath{clip}%
\pgfsetbuttcap%
\pgfsetroundjoin%
\definecolor{currentfill}{rgb}{0.283229,0.120777,0.440584}%
\pgfsetfillcolor{currentfill}%
\pgfsetfillopacity{0.700000}%
\pgfsetlinewidth{0.000000pt}%
\definecolor{currentstroke}{rgb}{0.000000,0.000000,0.000000}%
\pgfsetstrokecolor{currentstroke}%
\pgfsetdash{}{0pt}%
\pgfpathmoveto{\pgfqpoint{3.488606in}{2.360037in}}%
\pgfpathlineto{\pgfqpoint{3.501553in}{2.353380in}}%
\pgfpathlineto{\pgfqpoint{3.514502in}{2.346926in}}%
\pgfpathlineto{\pgfqpoint{3.527452in}{2.340674in}}%
\pgfpathlineto{\pgfqpoint{3.540406in}{2.334622in}}%
\pgfpathlineto{\pgfqpoint{3.548109in}{2.344436in}}%
\pgfpathlineto{\pgfqpoint{3.555807in}{2.354292in}}%
\pgfpathlineto{\pgfqpoint{3.563500in}{2.364190in}}%
\pgfpathlineto{\pgfqpoint{3.571187in}{2.374133in}}%
\pgfpathlineto{\pgfqpoint{3.558244in}{2.380214in}}%
\pgfpathlineto{\pgfqpoint{3.545304in}{2.386496in}}%
\pgfpathlineto{\pgfqpoint{3.532366in}{2.392980in}}%
\pgfpathlineto{\pgfqpoint{3.519430in}{2.399666in}}%
\pgfpathlineto{\pgfqpoint{3.511733in}{2.389684in}}%
\pgfpathlineto{\pgfqpoint{3.504029in}{2.379752in}}%
\pgfpathlineto{\pgfqpoint{3.496321in}{2.369870in}}%
\pgfpathlineto{\pgfqpoint{3.488606in}{2.360037in}}%
\pgfpathclose%
\pgfusepath{fill}%
\end{pgfscope}%
\begin{pgfscope}%
\pgfpathrectangle{\pgfqpoint{1.254980in}{0.150000in}}{\pgfqpoint{5.490039in}{5.490039in}}%
\pgfusepath{clip}%
\pgfsetbuttcap%
\pgfsetroundjoin%
\definecolor{currentfill}{rgb}{0.283072,0.130895,0.449241}%
\pgfsetfillcolor{currentfill}%
\pgfsetfillopacity{0.700000}%
\pgfsetlinewidth{0.000000pt}%
\definecolor{currentstroke}{rgb}{0.000000,0.000000,0.000000}%
\pgfsetstrokecolor{currentstroke}%
\pgfsetdash{}{0pt}%
\pgfpathmoveto{\pgfqpoint{3.354077in}{2.381878in}}%
\pgfpathlineto{\pgfqpoint{3.367026in}{2.373555in}}%
\pgfpathlineto{\pgfqpoint{3.379975in}{2.365445in}}%
\pgfpathlineto{\pgfqpoint{3.392925in}{2.357547in}}%
\pgfpathlineto{\pgfqpoint{3.405877in}{2.349859in}}%
\pgfpathlineto{\pgfqpoint{3.413625in}{2.359492in}}%
\pgfpathlineto{\pgfqpoint{3.421368in}{2.369180in}}%
\pgfpathlineto{\pgfqpoint{3.429105in}{2.378921in}}%
\pgfpathlineto{\pgfqpoint{3.436836in}{2.388718in}}%
\pgfpathlineto{\pgfqpoint{3.423897in}{2.396407in}}%
\pgfpathlineto{\pgfqpoint{3.410959in}{2.404307in}}%
\pgfpathlineto{\pgfqpoint{3.398022in}{2.412418in}}%
\pgfpathlineto{\pgfqpoint{3.385085in}{2.420743in}}%
\pgfpathlineto{\pgfqpoint{3.377342in}{2.410935in}}%
\pgfpathlineto{\pgfqpoint{3.369593in}{2.401188in}}%
\pgfpathlineto{\pgfqpoint{3.361838in}{2.391503in}}%
\pgfpathlineto{\pgfqpoint{3.354077in}{2.381878in}}%
\pgfpathclose%
\pgfusepath{fill}%
\end{pgfscope}%
\begin{pgfscope}%
\pgfpathrectangle{\pgfqpoint{1.254980in}{0.150000in}}{\pgfqpoint{5.490039in}{5.490039in}}%
\pgfusepath{clip}%
\pgfsetbuttcap%
\pgfsetroundjoin%
\definecolor{currentfill}{rgb}{0.281887,0.150881,0.465405}%
\pgfsetfillcolor{currentfill}%
\pgfsetfillopacity{0.700000}%
\pgfsetlinewidth{0.000000pt}%
\definecolor{currentstroke}{rgb}{0.000000,0.000000,0.000000}%
\pgfsetstrokecolor{currentstroke}%
\pgfsetdash{}{0pt}%
\pgfpathmoveto{\pgfqpoint{3.922085in}{2.411087in}}%
\pgfpathlineto{\pgfqpoint{3.935091in}{2.408736in}}%
\pgfpathlineto{\pgfqpoint{3.948102in}{2.406566in}}%
\pgfpathlineto{\pgfqpoint{3.961120in}{2.404576in}}%
\pgfpathlineto{\pgfqpoint{3.974144in}{2.402765in}}%
\pgfpathlineto{\pgfqpoint{3.981710in}{2.412603in}}%
\pgfpathlineto{\pgfqpoint{3.989271in}{2.422461in}}%
\pgfpathlineto{\pgfqpoint{3.996827in}{2.432343in}}%
\pgfpathlineto{\pgfqpoint{4.004379in}{2.442250in}}%
\pgfpathlineto{\pgfqpoint{3.991363in}{2.444202in}}%
\pgfpathlineto{\pgfqpoint{3.978354in}{2.446333in}}%
\pgfpathlineto{\pgfqpoint{3.965351in}{2.448644in}}%
\pgfpathlineto{\pgfqpoint{3.952354in}{2.451135in}}%
\pgfpathlineto{\pgfqpoint{3.944794in}{2.441077in}}%
\pgfpathlineto{\pgfqpoint{3.937229in}{2.431051in}}%
\pgfpathlineto{\pgfqpoint{3.929660in}{2.421055in}}%
\pgfpathlineto{\pgfqpoint{3.922085in}{2.411087in}}%
\pgfpathclose%
\pgfusepath{fill}%
\end{pgfscope}%
\begin{pgfscope}%
\pgfpathrectangle{\pgfqpoint{1.254980in}{0.150000in}}{\pgfqpoint{5.490039in}{5.490039in}}%
\pgfusepath{clip}%
\pgfsetbuttcap%
\pgfsetroundjoin%
\definecolor{currentfill}{rgb}{0.283229,0.120777,0.440584}%
\pgfsetfillcolor{currentfill}%
\pgfsetfillopacity{0.700000}%
\pgfsetlinewidth{0.000000pt}%
\definecolor{currentstroke}{rgb}{0.000000,0.000000,0.000000}%
\pgfsetstrokecolor{currentstroke}%
\pgfsetdash{}{0pt}%
\pgfpathmoveto{\pgfqpoint{3.622985in}{2.351792in}}%
\pgfpathlineto{\pgfqpoint{3.635942in}{2.346698in}}%
\pgfpathlineto{\pgfqpoint{3.648903in}{2.341798in}}%
\pgfpathlineto{\pgfqpoint{3.661868in}{2.337091in}}%
\pgfpathlineto{\pgfqpoint{3.674836in}{2.332577in}}%
\pgfpathlineto{\pgfqpoint{3.682498in}{2.342470in}}%
\pgfpathlineto{\pgfqpoint{3.690154in}{2.352395in}}%
\pgfpathlineto{\pgfqpoint{3.697806in}{2.362353in}}%
\pgfpathlineto{\pgfqpoint{3.705452in}{2.372345in}}%
\pgfpathlineto{\pgfqpoint{3.692493in}{2.376917in}}%
\pgfpathlineto{\pgfqpoint{3.679539in}{2.381681in}}%
\pgfpathlineto{\pgfqpoint{3.666588in}{2.386638in}}%
\pgfpathlineto{\pgfqpoint{3.653640in}{2.391790in}}%
\pgfpathlineto{\pgfqpoint{3.645984in}{2.381731in}}%
\pgfpathlineto{\pgfqpoint{3.638323in}{2.371712in}}%
\pgfpathlineto{\pgfqpoint{3.630657in}{2.361733in}}%
\pgfpathlineto{\pgfqpoint{3.622985in}{2.351792in}}%
\pgfpathclose%
\pgfusepath{fill}%
\end{pgfscope}%
\begin{pgfscope}%
\pgfpathrectangle{\pgfqpoint{1.254980in}{0.150000in}}{\pgfqpoint{5.490039in}{5.490039in}}%
\pgfusepath{clip}%
\pgfsetbuttcap%
\pgfsetroundjoin%
\definecolor{currentfill}{rgb}{0.282623,0.140926,0.457517}%
\pgfsetfillcolor{currentfill}%
\pgfsetfillopacity{0.700000}%
\pgfsetlinewidth{0.000000pt}%
\definecolor{currentstroke}{rgb}{0.000000,0.000000,0.000000}%
\pgfsetstrokecolor{currentstroke}%
\pgfsetdash{}{0pt}%
\pgfpathmoveto{\pgfqpoint{3.839740in}{2.382231in}}%
\pgfpathlineto{\pgfqpoint{3.852731in}{2.379264in}}%
\pgfpathlineto{\pgfqpoint{3.865728in}{2.376481in}}%
\pgfpathlineto{\pgfqpoint{3.878730in}{2.373880in}}%
\pgfpathlineto{\pgfqpoint{3.891738in}{2.371462in}}%
\pgfpathlineto{\pgfqpoint{3.899332in}{2.381334in}}%
\pgfpathlineto{\pgfqpoint{3.906922in}{2.391228in}}%
\pgfpathlineto{\pgfqpoint{3.914506in}{2.401145in}}%
\pgfpathlineto{\pgfqpoint{3.922085in}{2.411087in}}%
\pgfpathlineto{\pgfqpoint{3.909086in}{2.413619in}}%
\pgfpathlineto{\pgfqpoint{3.896092in}{2.416332in}}%
\pgfpathlineto{\pgfqpoint{3.883104in}{2.419229in}}%
\pgfpathlineto{\pgfqpoint{3.870121in}{2.422309in}}%
\pgfpathlineto{\pgfqpoint{3.862533in}{2.412244in}}%
\pgfpathlineto{\pgfqpoint{3.854940in}{2.402210in}}%
\pgfpathlineto{\pgfqpoint{3.847342in}{2.392207in}}%
\pgfpathlineto{\pgfqpoint{3.839740in}{2.382231in}}%
\pgfpathclose%
\pgfusepath{fill}%
\end{pgfscope}%
\begin{pgfscope}%
\pgfpathrectangle{\pgfqpoint{1.254980in}{0.150000in}}{\pgfqpoint{5.490039in}{5.490039in}}%
\pgfusepath{clip}%
\pgfsetbuttcap%
\pgfsetroundjoin%
\definecolor{currentfill}{rgb}{0.229739,0.322361,0.545706}%
\pgfsetfillcolor{currentfill}%
\pgfsetfillopacity{0.700000}%
\pgfsetlinewidth{0.000000pt}%
\definecolor{currentstroke}{rgb}{0.000000,0.000000,0.000000}%
\pgfsetstrokecolor{currentstroke}%
\pgfsetdash{}{0pt}%
\pgfpathmoveto{\pgfqpoint{4.714823in}{2.755448in}}%
\pgfpathlineto{\pgfqpoint{4.728084in}{2.757586in}}%
\pgfpathlineto{\pgfqpoint{4.741357in}{2.759885in}}%
\pgfpathlineto{\pgfqpoint{4.754641in}{2.762343in}}%
\pgfpathlineto{\pgfqpoint{4.767936in}{2.764960in}}%
\pgfpathlineto{\pgfqpoint{4.775235in}{2.773473in}}%
\pgfpathlineto{\pgfqpoint{4.782530in}{2.782052in}}%
\pgfpathlineto{\pgfqpoint{4.789822in}{2.790704in}}%
\pgfpathlineto{\pgfqpoint{4.797109in}{2.799431in}}%
\pgfpathlineto{\pgfqpoint{4.783827in}{2.797208in}}%
\pgfpathlineto{\pgfqpoint{4.770557in}{2.795144in}}%
\pgfpathlineto{\pgfqpoint{4.757298in}{2.793239in}}%
\pgfpathlineto{\pgfqpoint{4.744049in}{2.791493in}}%
\pgfpathlineto{\pgfqpoint{4.736748in}{2.782362in}}%
\pgfpathlineto{\pgfqpoint{4.729443in}{2.773314in}}%
\pgfpathlineto{\pgfqpoint{4.722135in}{2.764344in}}%
\pgfpathlineto{\pgfqpoint{4.714823in}{2.755448in}}%
\pgfpathclose%
\pgfusepath{fill}%
\end{pgfscope}%
\begin{pgfscope}%
\pgfpathrectangle{\pgfqpoint{1.254980in}{0.150000in}}{\pgfqpoint{5.490039in}{5.490039in}}%
\pgfusepath{clip}%
\pgfsetbuttcap%
\pgfsetroundjoin%
\definecolor{currentfill}{rgb}{0.220057,0.343307,0.549413}%
\pgfsetfillcolor{currentfill}%
\pgfsetfillopacity{0.700000}%
\pgfsetlinewidth{0.000000pt}%
\definecolor{currentstroke}{rgb}{0.000000,0.000000,0.000000}%
\pgfsetstrokecolor{currentstroke}%
\pgfsetdash{}{0pt}%
\pgfpathmoveto{\pgfqpoint{4.797109in}{2.799431in}}%
\pgfpathlineto{\pgfqpoint{4.810402in}{2.801813in}}%
\pgfpathlineto{\pgfqpoint{4.823706in}{2.804353in}}%
\pgfpathlineto{\pgfqpoint{4.837021in}{2.807050in}}%
\pgfpathlineto{\pgfqpoint{4.850348in}{2.809906in}}%
\pgfpathlineto{\pgfqpoint{4.857618in}{2.818303in}}%
\pgfpathlineto{\pgfqpoint{4.864883in}{2.826778in}}%
\pgfpathlineto{\pgfqpoint{4.872145in}{2.835337in}}%
\pgfpathlineto{\pgfqpoint{4.879404in}{2.843984in}}%
\pgfpathlineto{\pgfqpoint{4.866091in}{2.841551in}}%
\pgfpathlineto{\pgfqpoint{4.852790in}{2.839275in}}%
\pgfpathlineto{\pgfqpoint{4.839501in}{2.837156in}}%
\pgfpathlineto{\pgfqpoint{4.826222in}{2.835196in}}%
\pgfpathlineto{\pgfqpoint{4.818949in}{2.826117in}}%
\pgfpathlineto{\pgfqpoint{4.811673in}{2.817133in}}%
\pgfpathlineto{\pgfqpoint{4.804393in}{2.808239in}}%
\pgfpathlineto{\pgfqpoint{4.797109in}{2.799431in}}%
\pgfpathclose%
\pgfusepath{fill}%
\end{pgfscope}%
\begin{pgfscope}%
\pgfpathrectangle{\pgfqpoint{1.254980in}{0.150000in}}{\pgfqpoint{5.490039in}{5.490039in}}%
\pgfusepath{clip}%
\pgfsetbuttcap%
\pgfsetroundjoin%
\definecolor{currentfill}{rgb}{0.281887,0.150881,0.465405}%
\pgfsetfillcolor{currentfill}%
\pgfsetfillopacity{0.700000}%
\pgfsetlinewidth{0.000000pt}%
\definecolor{currentstroke}{rgb}{0.000000,0.000000,0.000000}%
\pgfsetstrokecolor{currentstroke}%
\pgfsetdash{}{0pt}%
\pgfpathmoveto{\pgfqpoint{3.219266in}{2.418614in}}%
\pgfpathlineto{\pgfqpoint{3.232232in}{2.408509in}}%
\pgfpathlineto{\pgfqpoint{3.245197in}{2.398629in}}%
\pgfpathlineto{\pgfqpoint{3.258161in}{2.388972in}}%
\pgfpathlineto{\pgfqpoint{3.271124in}{2.379536in}}%
\pgfpathlineto{\pgfqpoint{3.278923in}{2.388884in}}%
\pgfpathlineto{\pgfqpoint{3.286715in}{2.398298in}}%
\pgfpathlineto{\pgfqpoint{3.294501in}{2.407779in}}%
\pgfpathlineto{\pgfqpoint{3.302282in}{2.417328in}}%
\pgfpathlineto{\pgfqpoint{3.289332in}{2.426736in}}%
\pgfpathlineto{\pgfqpoint{3.276382in}{2.436367in}}%
\pgfpathlineto{\pgfqpoint{3.263431in}{2.446220in}}%
\pgfpathlineto{\pgfqpoint{3.250479in}{2.456298in}}%
\pgfpathlineto{\pgfqpoint{3.242686in}{2.446767in}}%
\pgfpathlineto{\pgfqpoint{3.234886in}{2.437309in}}%
\pgfpathlineto{\pgfqpoint{3.227079in}{2.427925in}}%
\pgfpathlineto{\pgfqpoint{3.219266in}{2.418614in}}%
\pgfpathclose%
\pgfusepath{fill}%
\end{pgfscope}%
\begin{pgfscope}%
\pgfpathrectangle{\pgfqpoint{1.254980in}{0.150000in}}{\pgfqpoint{5.490039in}{5.490039in}}%
\pgfusepath{clip}%
\pgfsetbuttcap%
\pgfsetroundjoin%
\definecolor{currentfill}{rgb}{0.237441,0.305202,0.541921}%
\pgfsetfillcolor{currentfill}%
\pgfsetfillopacity{0.700000}%
\pgfsetlinewidth{0.000000pt}%
\definecolor{currentstroke}{rgb}{0.000000,0.000000,0.000000}%
\pgfsetstrokecolor{currentstroke}%
\pgfsetdash{}{0pt}%
\pgfpathmoveto{\pgfqpoint{4.632543in}{2.712099in}}%
\pgfpathlineto{\pgfqpoint{4.645774in}{2.713960in}}%
\pgfpathlineto{\pgfqpoint{4.659016in}{2.715983in}}%
\pgfpathlineto{\pgfqpoint{4.672269in}{2.718167in}}%
\pgfpathlineto{\pgfqpoint{4.685532in}{2.720512in}}%
\pgfpathlineto{\pgfqpoint{4.692861in}{2.729157in}}%
\pgfpathlineto{\pgfqpoint{4.700186in}{2.737859in}}%
\pgfpathlineto{\pgfqpoint{4.707506in}{2.746621in}}%
\pgfpathlineto{\pgfqpoint{4.714823in}{2.755448in}}%
\pgfpathlineto{\pgfqpoint{4.701572in}{2.753469in}}%
\pgfpathlineto{\pgfqpoint{4.688331in}{2.751651in}}%
\pgfpathlineto{\pgfqpoint{4.675102in}{2.749993in}}%
\pgfpathlineto{\pgfqpoint{4.661882in}{2.748496in}}%
\pgfpathlineto{\pgfqpoint{4.654554in}{2.739294in}}%
\pgfpathlineto{\pgfqpoint{4.647221in}{2.730163in}}%
\pgfpathlineto{\pgfqpoint{4.639884in}{2.721099in}}%
\pgfpathlineto{\pgfqpoint{4.632543in}{2.712099in}}%
\pgfpathclose%
\pgfusepath{fill}%
\end{pgfscope}%
\begin{pgfscope}%
\pgfpathrectangle{\pgfqpoint{1.254980in}{0.150000in}}{\pgfqpoint{5.490039in}{5.490039in}}%
\pgfusepath{clip}%
\pgfsetbuttcap%
\pgfsetroundjoin%
\definecolor{currentfill}{rgb}{0.210503,0.363727,0.552206}%
\pgfsetfillcolor{currentfill}%
\pgfsetfillopacity{0.700000}%
\pgfsetlinewidth{0.000000pt}%
\definecolor{currentstroke}{rgb}{0.000000,0.000000,0.000000}%
\pgfsetstrokecolor{currentstroke}%
\pgfsetdash{}{0pt}%
\pgfpathmoveto{\pgfqpoint{4.879404in}{2.843984in}}%
\pgfpathlineto{\pgfqpoint{4.892728in}{2.846575in}}%
\pgfpathlineto{\pgfqpoint{4.906063in}{2.849322in}}%
\pgfpathlineto{\pgfqpoint{4.919411in}{2.852226in}}%
\pgfpathlineto{\pgfqpoint{4.932770in}{2.855287in}}%
\pgfpathlineto{\pgfqpoint{4.940010in}{2.863588in}}%
\pgfpathlineto{\pgfqpoint{4.947246in}{2.871981in}}%
\pgfpathlineto{\pgfqpoint{4.954479in}{2.880470in}}%
\pgfpathlineto{\pgfqpoint{4.961709in}{2.889062in}}%
\pgfpathlineto{\pgfqpoint{4.948366in}{2.886451in}}%
\pgfpathlineto{\pgfqpoint{4.935034in}{2.883997in}}%
\pgfpathlineto{\pgfqpoint{4.921714in}{2.881699in}}%
\pgfpathlineto{\pgfqpoint{4.908406in}{2.879558in}}%
\pgfpathlineto{\pgfqpoint{4.901160in}{2.870507in}}%
\pgfpathlineto{\pgfqpoint{4.893911in}{2.861564in}}%
\pgfpathlineto{\pgfqpoint{4.886659in}{2.852725in}}%
\pgfpathlineto{\pgfqpoint{4.879404in}{2.843984in}}%
\pgfpathclose%
\pgfusepath{fill}%
\end{pgfscope}%
\begin{pgfscope}%
\pgfpathrectangle{\pgfqpoint{1.254980in}{0.150000in}}{\pgfqpoint{5.490039in}{5.490039in}}%
\pgfusepath{clip}%
\pgfsetbuttcap%
\pgfsetroundjoin%
\definecolor{currentfill}{rgb}{0.246811,0.283237,0.535941}%
\pgfsetfillcolor{currentfill}%
\pgfsetfillopacity{0.700000}%
\pgfsetlinewidth{0.000000pt}%
\definecolor{currentstroke}{rgb}{0.000000,0.000000,0.000000}%
\pgfsetstrokecolor{currentstroke}%
\pgfsetdash{}{0pt}%
\pgfpathmoveto{\pgfqpoint{4.550267in}{2.669470in}}%
\pgfpathlineto{\pgfqpoint{4.563469in}{2.671020in}}%
\pgfpathlineto{\pgfqpoint{4.576681in}{2.672734in}}%
\pgfpathlineto{\pgfqpoint{4.589903in}{2.674609in}}%
\pgfpathlineto{\pgfqpoint{4.603136in}{2.676647in}}%
\pgfpathlineto{\pgfqpoint{4.610494in}{2.685436in}}%
\pgfpathlineto{\pgfqpoint{4.617848in}{2.694271in}}%
\pgfpathlineto{\pgfqpoint{4.625197in}{2.703157in}}%
\pgfpathlineto{\pgfqpoint{4.632543in}{2.712099in}}%
\pgfpathlineto{\pgfqpoint{4.619322in}{2.710399in}}%
\pgfpathlineto{\pgfqpoint{4.606111in}{2.708860in}}%
\pgfpathlineto{\pgfqpoint{4.592911in}{2.707484in}}%
\pgfpathlineto{\pgfqpoint{4.579720in}{2.706271in}}%
\pgfpathlineto{\pgfqpoint{4.572363in}{2.696982in}}%
\pgfpathlineto{\pgfqpoint{4.565002in}{2.687755in}}%
\pgfpathlineto{\pgfqpoint{4.557637in}{2.678586in}}%
\pgfpathlineto{\pgfqpoint{4.550267in}{2.669470in}}%
\pgfpathclose%
\pgfusepath{fill}%
\end{pgfscope}%
\begin{pgfscope}%
\pgfpathrectangle{\pgfqpoint{1.254980in}{0.150000in}}{\pgfqpoint{5.490039in}{5.490039in}}%
\pgfusepath{clip}%
\pgfsetbuttcap%
\pgfsetroundjoin%
\definecolor{currentfill}{rgb}{0.203063,0.379716,0.553925}%
\pgfsetfillcolor{currentfill}%
\pgfsetfillopacity{0.700000}%
\pgfsetlinewidth{0.000000pt}%
\definecolor{currentstroke}{rgb}{0.000000,0.000000,0.000000}%
\pgfsetstrokecolor{currentstroke}%
\pgfsetdash{}{0pt}%
\pgfpathmoveto{\pgfqpoint{4.961709in}{2.889062in}}%
\pgfpathlineto{\pgfqpoint{4.975065in}{2.891828in}}%
\pgfpathlineto{\pgfqpoint{4.988432in}{2.894749in}}%
\pgfpathlineto{\pgfqpoint{5.001812in}{2.897827in}}%
\pgfpathlineto{\pgfqpoint{5.015203in}{2.901059in}}%
\pgfpathlineto{\pgfqpoint{5.022414in}{2.909290in}}%
\pgfpathlineto{\pgfqpoint{5.029621in}{2.917627in}}%
\pgfpathlineto{\pgfqpoint{5.036825in}{2.926075in}}%
\pgfpathlineto{\pgfqpoint{5.044027in}{2.934640in}}%
\pgfpathlineto{\pgfqpoint{5.030653in}{2.931886in}}%
\pgfpathlineto{\pgfqpoint{5.017290in}{2.929287in}}%
\pgfpathlineto{\pgfqpoint{5.003940in}{2.926843in}}%
\pgfpathlineto{\pgfqpoint{4.990601in}{2.924554in}}%
\pgfpathlineto{\pgfqpoint{4.983382in}{2.915501in}}%
\pgfpathlineto{\pgfqpoint{4.976160in}{2.906572in}}%
\pgfpathlineto{\pgfqpoint{4.968936in}{2.897760in}}%
\pgfpathlineto{\pgfqpoint{4.961709in}{2.889062in}}%
\pgfpathclose%
\pgfusepath{fill}%
\end{pgfscope}%
\begin{pgfscope}%
\pgfpathrectangle{\pgfqpoint{1.254980in}{0.150000in}}{\pgfqpoint{5.490039in}{5.490039in}}%
\pgfusepath{clip}%
\pgfsetbuttcap%
\pgfsetroundjoin%
\definecolor{currentfill}{rgb}{0.253935,0.265254,0.529983}%
\pgfsetfillcolor{currentfill}%
\pgfsetfillopacity{0.700000}%
\pgfsetlinewidth{0.000000pt}%
\definecolor{currentstroke}{rgb}{0.000000,0.000000,0.000000}%
\pgfsetstrokecolor{currentstroke}%
\pgfsetdash{}{0pt}%
\pgfpathmoveto{\pgfqpoint{4.467993in}{2.627668in}}%
\pgfpathlineto{\pgfqpoint{4.481166in}{2.628872in}}%
\pgfpathlineto{\pgfqpoint{4.494349in}{2.630241in}}%
\pgfpathlineto{\pgfqpoint{4.507542in}{2.631773in}}%
\pgfpathlineto{\pgfqpoint{4.520744in}{2.633470in}}%
\pgfpathlineto{\pgfqpoint{4.528132in}{2.642408in}}%
\pgfpathlineto{\pgfqpoint{4.535515in}{2.651385in}}%
\pgfpathlineto{\pgfqpoint{4.542893in}{2.660404in}}%
\pgfpathlineto{\pgfqpoint{4.550267in}{2.669470in}}%
\pgfpathlineto{\pgfqpoint{4.537076in}{2.668083in}}%
\pgfpathlineto{\pgfqpoint{4.523894in}{2.666860in}}%
\pgfpathlineto{\pgfqpoint{4.510722in}{2.665800in}}%
\pgfpathlineto{\pgfqpoint{4.497560in}{2.664905in}}%
\pgfpathlineto{\pgfqpoint{4.490175in}{2.655520in}}%
\pgfpathlineto{\pgfqpoint{4.482785in}{2.646188in}}%
\pgfpathlineto{\pgfqpoint{4.475392in}{2.636905in}}%
\pgfpathlineto{\pgfqpoint{4.467993in}{2.627668in}}%
\pgfpathclose%
\pgfusepath{fill}%
\end{pgfscope}%
\begin{pgfscope}%
\pgfpathrectangle{\pgfqpoint{1.254980in}{0.150000in}}{\pgfqpoint{5.490039in}{5.490039in}}%
\pgfusepath{clip}%
\pgfsetbuttcap%
\pgfsetroundjoin%
\definecolor{currentfill}{rgb}{0.194100,0.399323,0.555565}%
\pgfsetfillcolor{currentfill}%
\pgfsetfillopacity{0.700000}%
\pgfsetlinewidth{0.000000pt}%
\definecolor{currentstroke}{rgb}{0.000000,0.000000,0.000000}%
\pgfsetstrokecolor{currentstroke}%
\pgfsetdash{}{0pt}%
\pgfpathmoveto{\pgfqpoint{5.044027in}{2.934640in}}%
\pgfpathlineto{\pgfqpoint{5.057414in}{2.937548in}}%
\pgfpathlineto{\pgfqpoint{5.070813in}{2.940611in}}%
\pgfpathlineto{\pgfqpoint{5.084225in}{2.943828in}}%
\pgfpathlineto{\pgfqpoint{5.097649in}{2.947200in}}%
\pgfpathlineto{\pgfqpoint{5.104830in}{2.955391in}}%
\pgfpathlineto{\pgfqpoint{5.112009in}{2.963703in}}%
\pgfpathlineto{\pgfqpoint{5.119186in}{2.972143in}}%
\pgfpathlineto{\pgfqpoint{5.126360in}{2.980715in}}%
\pgfpathlineto{\pgfqpoint{5.112955in}{2.977851in}}%
\pgfpathlineto{\pgfqpoint{5.099561in}{2.975140in}}%
\pgfpathlineto{\pgfqpoint{5.086181in}{2.972583in}}%
\pgfpathlineto{\pgfqpoint{5.072812in}{2.970180in}}%
\pgfpathlineto{\pgfqpoint{5.065619in}{2.961092in}}%
\pgfpathlineto{\pgfqpoint{5.058424in}{2.952143in}}%
\pgfpathlineto{\pgfqpoint{5.051227in}{2.943327in}}%
\pgfpathlineto{\pgfqpoint{5.044027in}{2.934640in}}%
\pgfpathclose%
\pgfusepath{fill}%
\end{pgfscope}%
\begin{pgfscope}%
\pgfpathrectangle{\pgfqpoint{1.254980in}{0.150000in}}{\pgfqpoint{5.490039in}{5.490039in}}%
\pgfusepath{clip}%
\pgfsetbuttcap%
\pgfsetroundjoin%
\definecolor{currentfill}{rgb}{0.260571,0.246922,0.522828}%
\pgfsetfillcolor{currentfill}%
\pgfsetfillopacity{0.700000}%
\pgfsetlinewidth{0.000000pt}%
\definecolor{currentstroke}{rgb}{0.000000,0.000000,0.000000}%
\pgfsetstrokecolor{currentstroke}%
\pgfsetdash{}{0pt}%
\pgfpathmoveto{\pgfqpoint{4.385718in}{2.586820in}}%
\pgfpathlineto{\pgfqpoint{4.398863in}{2.587642in}}%
\pgfpathlineto{\pgfqpoint{4.412018in}{2.588631in}}%
\pgfpathlineto{\pgfqpoint{4.425182in}{2.589785in}}%
\pgfpathlineto{\pgfqpoint{4.438355in}{2.591106in}}%
\pgfpathlineto{\pgfqpoint{4.445772in}{2.600195in}}%
\pgfpathlineto{\pgfqpoint{4.453183in}{2.609317in}}%
\pgfpathlineto{\pgfqpoint{4.460591in}{2.618473in}}%
\pgfpathlineto{\pgfqpoint{4.467993in}{2.627668in}}%
\pgfpathlineto{\pgfqpoint{4.454830in}{2.626629in}}%
\pgfpathlineto{\pgfqpoint{4.441676in}{2.625755in}}%
\pgfpathlineto{\pgfqpoint{4.428532in}{2.625048in}}%
\pgfpathlineto{\pgfqpoint{4.415397in}{2.624506in}}%
\pgfpathlineto{\pgfqpoint{4.407984in}{2.615020in}}%
\pgfpathlineto{\pgfqpoint{4.400566in}{2.605579in}}%
\pgfpathlineto{\pgfqpoint{4.393144in}{2.596180in}}%
\pgfpathlineto{\pgfqpoint{4.385718in}{2.586820in}}%
\pgfpathclose%
\pgfusepath{fill}%
\end{pgfscope}%
\begin{pgfscope}%
\pgfpathrectangle{\pgfqpoint{1.254980in}{0.150000in}}{\pgfqpoint{5.490039in}{5.490039in}}%
\pgfusepath{clip}%
\pgfsetbuttcap%
\pgfsetroundjoin%
\definecolor{currentfill}{rgb}{0.185556,0.418570,0.556753}%
\pgfsetfillcolor{currentfill}%
\pgfsetfillopacity{0.700000}%
\pgfsetlinewidth{0.000000pt}%
\definecolor{currentstroke}{rgb}{0.000000,0.000000,0.000000}%
\pgfsetstrokecolor{currentstroke}%
\pgfsetdash{}{0pt}%
\pgfpathmoveto{\pgfqpoint{5.126360in}{2.980715in}}%
\pgfpathlineto{\pgfqpoint{5.139778in}{2.983733in}}%
\pgfpathlineto{\pgfqpoint{5.153209in}{2.986904in}}%
\pgfpathlineto{\pgfqpoint{5.166652in}{2.990229in}}%
\pgfpathlineto{\pgfqpoint{5.180108in}{2.993706in}}%
\pgfpathlineto{\pgfqpoint{5.187262in}{3.001893in}}%
\pgfpathlineto{\pgfqpoint{5.194413in}{3.010218in}}%
\pgfpathlineto{\pgfqpoint{5.201563in}{3.018687in}}%
\pgfpathlineto{\pgfqpoint{5.208711in}{3.027306in}}%
\pgfpathlineto{\pgfqpoint{5.195274in}{3.024364in}}%
\pgfpathlineto{\pgfqpoint{5.181851in}{3.021574in}}%
\pgfpathlineto{\pgfqpoint{5.168440in}{3.018937in}}%
\pgfpathlineto{\pgfqpoint{5.155041in}{3.016453in}}%
\pgfpathlineto{\pgfqpoint{5.147873in}{3.007289in}}%
\pgfpathlineto{\pgfqpoint{5.140704in}{2.998283in}}%
\pgfpathlineto{\pgfqpoint{5.133533in}{2.989427in}}%
\pgfpathlineto{\pgfqpoint{5.126360in}{2.980715in}}%
\pgfpathclose%
\pgfusepath{fill}%
\end{pgfscope}%
\begin{pgfscope}%
\pgfpathrectangle{\pgfqpoint{1.254980in}{0.150000in}}{\pgfqpoint{5.490039in}{5.490039in}}%
\pgfusepath{clip}%
\pgfsetbuttcap%
\pgfsetroundjoin%
\definecolor{currentfill}{rgb}{0.177423,0.437527,0.557565}%
\pgfsetfillcolor{currentfill}%
\pgfsetfillopacity{0.700000}%
\pgfsetlinewidth{0.000000pt}%
\definecolor{currentstroke}{rgb}{0.000000,0.000000,0.000000}%
\pgfsetstrokecolor{currentstroke}%
\pgfsetdash{}{0pt}%
\pgfpathmoveto{\pgfqpoint{5.208711in}{3.027306in}}%
\pgfpathlineto{\pgfqpoint{5.222160in}{3.030400in}}%
\pgfpathlineto{\pgfqpoint{5.235622in}{3.033647in}}%
\pgfpathlineto{\pgfqpoint{5.249096in}{3.037046in}}%
\pgfpathlineto{\pgfqpoint{5.262584in}{3.040597in}}%
\pgfpathlineto{\pgfqpoint{5.269710in}{3.048819in}}%
\pgfpathlineto{\pgfqpoint{5.276835in}{3.057198in}}%
\pgfpathlineto{\pgfqpoint{5.283959in}{3.065739in}}%
\pgfpathlineto{\pgfqpoint{5.291082in}{3.074449in}}%
\pgfpathlineto{\pgfqpoint{5.277616in}{3.071461in}}%
\pgfpathlineto{\pgfqpoint{5.264162in}{3.068625in}}%
\pgfpathlineto{\pgfqpoint{5.250721in}{3.065941in}}%
\pgfpathlineto{\pgfqpoint{5.237293in}{3.063408in}}%
\pgfpathlineto{\pgfqpoint{5.230148in}{3.054126in}}%
\pgfpathlineto{\pgfqpoint{5.223003in}{3.045019in}}%
\pgfpathlineto{\pgfqpoint{5.215858in}{3.036081in}}%
\pgfpathlineto{\pgfqpoint{5.208711in}{3.027306in}}%
\pgfpathclose%
\pgfusepath{fill}%
\end{pgfscope}%
\begin{pgfscope}%
\pgfpathrectangle{\pgfqpoint{1.254980in}{0.150000in}}{\pgfqpoint{5.490039in}{5.490039in}}%
\pgfusepath{clip}%
\pgfsetbuttcap%
\pgfsetroundjoin%
\definecolor{currentfill}{rgb}{0.257322,0.256130,0.526563}%
\pgfsetfillcolor{currentfill}%
\pgfsetfillopacity{0.700000}%
\pgfsetlinewidth{0.000000pt}%
\definecolor{currentstroke}{rgb}{0.000000,0.000000,0.000000}%
\pgfsetstrokecolor{currentstroke}%
\pgfsetdash{}{0pt}%
\pgfpathmoveto{\pgfqpoint{2.927730in}{2.635185in}}%
\pgfpathlineto{\pgfqpoint{2.940782in}{2.620154in}}%
\pgfpathlineto{\pgfqpoint{2.953828in}{2.605385in}}%
\pgfpathlineto{\pgfqpoint{2.966869in}{2.590877in}}%
\pgfpathlineto{\pgfqpoint{2.979905in}{2.576628in}}%
\pgfpathlineto{\pgfqpoint{2.987811in}{2.585378in}}%
\pgfpathlineto{\pgfqpoint{2.995710in}{2.594230in}}%
\pgfpathlineto{\pgfqpoint{3.003601in}{2.603184in}}%
\pgfpathlineto{\pgfqpoint{3.011484in}{2.612238in}}%
\pgfpathlineto{\pgfqpoint{2.998466in}{2.626430in}}%
\pgfpathlineto{\pgfqpoint{2.985443in}{2.640881in}}%
\pgfpathlineto{\pgfqpoint{2.972415in}{2.655591in}}%
\pgfpathlineto{\pgfqpoint{2.959381in}{2.670564in}}%
\pgfpathlineto{\pgfqpoint{2.951480in}{2.661557in}}%
\pgfpathlineto{\pgfqpoint{2.943571in}{2.652657in}}%
\pgfpathlineto{\pgfqpoint{2.935654in}{2.643867in}}%
\pgfpathlineto{\pgfqpoint{2.927730in}{2.635185in}}%
\pgfpathclose%
\pgfusepath{fill}%
\end{pgfscope}%
\begin{pgfscope}%
\pgfpathrectangle{\pgfqpoint{1.254980in}{0.150000in}}{\pgfqpoint{5.490039in}{5.490039in}}%
\pgfusepath{clip}%
\pgfsetbuttcap%
\pgfsetroundjoin%
\definecolor{currentfill}{rgb}{0.246811,0.283237,0.535941}%
\pgfsetfillcolor{currentfill}%
\pgfsetfillopacity{0.700000}%
\pgfsetlinewidth{0.000000pt}%
\definecolor{currentstroke}{rgb}{0.000000,0.000000,0.000000}%
\pgfsetstrokecolor{currentstroke}%
\pgfsetdash{}{0pt}%
\pgfpathmoveto{\pgfqpoint{2.875458in}{2.697981in}}%
\pgfpathlineto{\pgfqpoint{2.888536in}{2.681877in}}%
\pgfpathlineto{\pgfqpoint{2.901607in}{2.666044in}}%
\pgfpathlineto{\pgfqpoint{2.914671in}{2.650481in}}%
\pgfpathlineto{\pgfqpoint{2.927730in}{2.635185in}}%
\pgfpathlineto{\pgfqpoint{2.935654in}{2.643867in}}%
\pgfpathlineto{\pgfqpoint{2.943571in}{2.652657in}}%
\pgfpathlineto{\pgfqpoint{2.951480in}{2.661557in}}%
\pgfpathlineto{\pgfqpoint{2.959381in}{2.670564in}}%
\pgfpathlineto{\pgfqpoint{2.946341in}{2.685802in}}%
\pgfpathlineto{\pgfqpoint{2.933296in}{2.701306in}}%
\pgfpathlineto{\pgfqpoint{2.920244in}{2.717081in}}%
\pgfpathlineto{\pgfqpoint{2.907185in}{2.733127in}}%
\pgfpathlineto{\pgfqpoint{2.899266in}{2.724167in}}%
\pgfpathlineto{\pgfqpoint{2.891338in}{2.715323in}}%
\pgfpathlineto{\pgfqpoint{2.883402in}{2.706595in}}%
\pgfpathlineto{\pgfqpoint{2.875458in}{2.697981in}}%
\pgfpathclose%
\pgfusepath{fill}%
\end{pgfscope}%
\begin{pgfscope}%
\pgfpathrectangle{\pgfqpoint{1.254980in}{0.150000in}}{\pgfqpoint{5.490039in}{5.490039in}}%
\pgfusepath{clip}%
\pgfsetbuttcap%
\pgfsetroundjoin%
\definecolor{currentfill}{rgb}{0.266580,0.228262,0.514349}%
\pgfsetfillcolor{currentfill}%
\pgfsetfillopacity{0.700000}%
\pgfsetlinewidth{0.000000pt}%
\definecolor{currentstroke}{rgb}{0.000000,0.000000,0.000000}%
\pgfsetstrokecolor{currentstroke}%
\pgfsetdash{}{0pt}%
\pgfpathmoveto{\pgfqpoint{4.303437in}{2.547072in}}%
\pgfpathlineto{\pgfqpoint{4.316556in}{2.547477in}}%
\pgfpathlineto{\pgfqpoint{4.329683in}{2.548050in}}%
\pgfpathlineto{\pgfqpoint{4.342820in}{2.548792in}}%
\pgfpathlineto{\pgfqpoint{4.355966in}{2.549701in}}%
\pgfpathlineto{\pgfqpoint{4.363411in}{2.558938in}}%
\pgfpathlineto{\pgfqpoint{4.370851in}{2.568202in}}%
\pgfpathlineto{\pgfqpoint{4.378287in}{2.577495in}}%
\pgfpathlineto{\pgfqpoint{4.385718in}{2.586820in}}%
\pgfpathlineto{\pgfqpoint{4.372582in}{2.586164in}}%
\pgfpathlineto{\pgfqpoint{4.359455in}{2.585676in}}%
\pgfpathlineto{\pgfqpoint{4.346337in}{2.585355in}}%
\pgfpathlineto{\pgfqpoint{4.333227in}{2.585203in}}%
\pgfpathlineto{\pgfqpoint{4.325787in}{2.575615in}}%
\pgfpathlineto{\pgfqpoint{4.318342in}{2.566066in}}%
\pgfpathlineto{\pgfqpoint{4.310892in}{2.556552in}}%
\pgfpathlineto{\pgfqpoint{4.303437in}{2.547072in}}%
\pgfpathclose%
\pgfusepath{fill}%
\end{pgfscope}%
\begin{pgfscope}%
\pgfpathrectangle{\pgfqpoint{1.254980in}{0.150000in}}{\pgfqpoint{5.490039in}{5.490039in}}%
\pgfusepath{clip}%
\pgfsetbuttcap%
\pgfsetroundjoin%
\definecolor{currentfill}{rgb}{0.169646,0.456262,0.558030}%
\pgfsetfillcolor{currentfill}%
\pgfsetfillopacity{0.700000}%
\pgfsetlinewidth{0.000000pt}%
\definecolor{currentstroke}{rgb}{0.000000,0.000000,0.000000}%
\pgfsetstrokecolor{currentstroke}%
\pgfsetdash{}{0pt}%
\pgfpathmoveto{\pgfqpoint{5.291082in}{3.074449in}}%
\pgfpathlineto{\pgfqpoint{5.304562in}{3.077588in}}%
\pgfpathlineto{\pgfqpoint{5.318054in}{3.080878in}}%
\pgfpathlineto{\pgfqpoint{5.331560in}{3.084319in}}%
\pgfpathlineto{\pgfqpoint{5.345079in}{3.087911in}}%
\pgfpathlineto{\pgfqpoint{5.352180in}{3.096215in}}%
\pgfpathlineto{\pgfqpoint{5.359279in}{3.104694in}}%
\pgfpathlineto{\pgfqpoint{5.366379in}{3.113355in}}%
\pgfpathlineto{\pgfqpoint{5.373478in}{3.122204in}}%
\pgfpathlineto{\pgfqpoint{5.359982in}{3.119204in}}%
\pgfpathlineto{\pgfqpoint{5.346499in}{3.116353in}}%
\pgfpathlineto{\pgfqpoint{5.333029in}{3.113654in}}%
\pgfpathlineto{\pgfqpoint{5.319572in}{3.111105in}}%
\pgfpathlineto{\pgfqpoint{5.312449in}{3.101655in}}%
\pgfpathlineto{\pgfqpoint{5.305327in}{3.092400in}}%
\pgfpathlineto{\pgfqpoint{5.298205in}{3.083334in}}%
\pgfpathlineto{\pgfqpoint{5.291082in}{3.074449in}}%
\pgfpathclose%
\pgfusepath{fill}%
\end{pgfscope}%
\begin{pgfscope}%
\pgfpathrectangle{\pgfqpoint{1.254980in}{0.150000in}}{\pgfqpoint{5.490039in}{5.490039in}}%
\pgfusepath{clip}%
\pgfsetbuttcap%
\pgfsetroundjoin%
\definecolor{currentfill}{rgb}{0.266580,0.228262,0.514349}%
\pgfsetfillcolor{currentfill}%
\pgfsetfillopacity{0.700000}%
\pgfsetlinewidth{0.000000pt}%
\definecolor{currentstroke}{rgb}{0.000000,0.000000,0.000000}%
\pgfsetstrokecolor{currentstroke}%
\pgfsetdash{}{0pt}%
\pgfpathmoveto{\pgfqpoint{2.979905in}{2.576628in}}%
\pgfpathlineto{\pgfqpoint{2.992936in}{2.562634in}}%
\pgfpathlineto{\pgfqpoint{3.005962in}{2.548895in}}%
\pgfpathlineto{\pgfqpoint{3.018983in}{2.535407in}}%
\pgfpathlineto{\pgfqpoint{3.032000in}{2.522169in}}%
\pgfpathlineto{\pgfqpoint{3.039888in}{2.530987in}}%
\pgfpathlineto{\pgfqpoint{3.047769in}{2.539900in}}%
\pgfpathlineto{\pgfqpoint{3.055643in}{2.548906in}}%
\pgfpathlineto{\pgfqpoint{3.063509in}{2.558008in}}%
\pgfpathlineto{\pgfqpoint{3.050509in}{2.571189in}}%
\pgfpathlineto{\pgfqpoint{3.037505in}{2.584619in}}%
\pgfpathlineto{\pgfqpoint{3.024497in}{2.598302in}}%
\pgfpathlineto{\pgfqpoint{3.011484in}{2.612238in}}%
\pgfpathlineto{\pgfqpoint{3.003601in}{2.603184in}}%
\pgfpathlineto{\pgfqpoint{2.995710in}{2.594230in}}%
\pgfpathlineto{\pgfqpoint{2.987811in}{2.585378in}}%
\pgfpathlineto{\pgfqpoint{2.979905in}{2.576628in}}%
\pgfpathclose%
\pgfusepath{fill}%
\end{pgfscope}%
\begin{pgfscope}%
\pgfpathrectangle{\pgfqpoint{1.254980in}{0.150000in}}{\pgfqpoint{5.490039in}{5.490039in}}%
\pgfusepath{clip}%
\pgfsetbuttcap%
\pgfsetroundjoin%
\definecolor{currentfill}{rgb}{0.283229,0.120777,0.440584}%
\pgfsetfillcolor{currentfill}%
\pgfsetfillopacity{0.700000}%
\pgfsetlinewidth{0.000000pt}%
\definecolor{currentstroke}{rgb}{0.000000,0.000000,0.000000}%
\pgfsetstrokecolor{currentstroke}%
\pgfsetdash{}{0pt}%
\pgfpathmoveto{\pgfqpoint{3.405877in}{2.349859in}}%
\pgfpathlineto{\pgfqpoint{3.418829in}{2.342380in}}%
\pgfpathlineto{\pgfqpoint{3.431782in}{2.335109in}}%
\pgfpathlineto{\pgfqpoint{3.444737in}{2.328044in}}%
\pgfpathlineto{\pgfqpoint{3.457693in}{2.321185in}}%
\pgfpathlineto{\pgfqpoint{3.465430in}{2.330827in}}%
\pgfpathlineto{\pgfqpoint{3.473161in}{2.340516in}}%
\pgfpathlineto{\pgfqpoint{3.480886in}{2.350253in}}%
\pgfpathlineto{\pgfqpoint{3.488606in}{2.360037in}}%
\pgfpathlineto{\pgfqpoint{3.475661in}{2.366898in}}%
\pgfpathlineto{\pgfqpoint{3.462718in}{2.373964in}}%
\pgfpathlineto{\pgfqpoint{3.449777in}{2.381237in}}%
\pgfpathlineto{\pgfqpoint{3.436836in}{2.388718in}}%
\pgfpathlineto{\pgfqpoint{3.429105in}{2.378921in}}%
\pgfpathlineto{\pgfqpoint{3.421368in}{2.369180in}}%
\pgfpathlineto{\pgfqpoint{3.413625in}{2.359492in}}%
\pgfpathlineto{\pgfqpoint{3.405877in}{2.349859in}}%
\pgfpathclose%
\pgfusepath{fill}%
\end{pgfscope}%
\begin{pgfscope}%
\pgfpathrectangle{\pgfqpoint{1.254980in}{0.150000in}}{\pgfqpoint{5.490039in}{5.490039in}}%
\pgfusepath{clip}%
\pgfsetbuttcap%
\pgfsetroundjoin%
\definecolor{currentfill}{rgb}{0.233603,0.313828,0.543914}%
\pgfsetfillcolor{currentfill}%
\pgfsetfillopacity{0.700000}%
\pgfsetlinewidth{0.000000pt}%
\definecolor{currentstroke}{rgb}{0.000000,0.000000,0.000000}%
\pgfsetstrokecolor{currentstroke}%
\pgfsetdash{}{0pt}%
\pgfpathmoveto{\pgfqpoint{2.823075in}{2.765171in}}%
\pgfpathlineto{\pgfqpoint{2.836182in}{2.747953in}}%
\pgfpathlineto{\pgfqpoint{2.849282in}{2.731017in}}%
\pgfpathlineto{\pgfqpoint{2.862374in}{2.714361in}}%
\pgfpathlineto{\pgfqpoint{2.875458in}{2.697981in}}%
\pgfpathlineto{\pgfqpoint{2.883402in}{2.706595in}}%
\pgfpathlineto{\pgfqpoint{2.891338in}{2.715323in}}%
\pgfpathlineto{\pgfqpoint{2.899266in}{2.724167in}}%
\pgfpathlineto{\pgfqpoint{2.907185in}{2.733127in}}%
\pgfpathlineto{\pgfqpoint{2.894120in}{2.749447in}}%
\pgfpathlineto{\pgfqpoint{2.881047in}{2.766044in}}%
\pgfpathlineto{\pgfqpoint{2.867967in}{2.782921in}}%
\pgfpathlineto{\pgfqpoint{2.854880in}{2.800079in}}%
\pgfpathlineto{\pgfqpoint{2.846941in}{2.791168in}}%
\pgfpathlineto{\pgfqpoint{2.838994in}{2.782380in}}%
\pgfpathlineto{\pgfqpoint{2.831039in}{2.773714in}}%
\pgfpathlineto{\pgfqpoint{2.823075in}{2.765171in}}%
\pgfpathclose%
\pgfusepath{fill}%
\end{pgfscope}%
\begin{pgfscope}%
\pgfpathrectangle{\pgfqpoint{1.254980in}{0.150000in}}{\pgfqpoint{5.490039in}{5.490039in}}%
\pgfusepath{clip}%
\pgfsetbuttcap%
\pgfsetroundjoin%
\definecolor{currentfill}{rgb}{0.271828,0.209303,0.504434}%
\pgfsetfillcolor{currentfill}%
\pgfsetfillopacity{0.700000}%
\pgfsetlinewidth{0.000000pt}%
\definecolor{currentstroke}{rgb}{0.000000,0.000000,0.000000}%
\pgfsetstrokecolor{currentstroke}%
\pgfsetdash{}{0pt}%
\pgfpathmoveto{\pgfqpoint{4.221146in}{2.508593in}}%
\pgfpathlineto{\pgfqpoint{4.234240in}{2.508545in}}%
\pgfpathlineto{\pgfqpoint{4.247342in}{2.508667in}}%
\pgfpathlineto{\pgfqpoint{4.260452in}{2.508959in}}%
\pgfpathlineto{\pgfqpoint{4.273572in}{2.509420in}}%
\pgfpathlineto{\pgfqpoint{4.281045in}{2.518798in}}%
\pgfpathlineto{\pgfqpoint{4.288514in}{2.528198in}}%
\pgfpathlineto{\pgfqpoint{4.295978in}{2.537621in}}%
\pgfpathlineto{\pgfqpoint{4.303437in}{2.547072in}}%
\pgfpathlineto{\pgfqpoint{4.290327in}{2.546836in}}%
\pgfpathlineto{\pgfqpoint{4.277225in}{2.546769in}}%
\pgfpathlineto{\pgfqpoint{4.264132in}{2.546872in}}%
\pgfpathlineto{\pgfqpoint{4.251047in}{2.547145in}}%
\pgfpathlineto{\pgfqpoint{4.243579in}{2.537459in}}%
\pgfpathlineto{\pgfqpoint{4.236106in}{2.527807in}}%
\pgfpathlineto{\pgfqpoint{4.228628in}{2.518186in}}%
\pgfpathlineto{\pgfqpoint{4.221146in}{2.508593in}}%
\pgfpathclose%
\pgfusepath{fill}%
\end{pgfscope}%
\begin{pgfscope}%
\pgfpathrectangle{\pgfqpoint{1.254980in}{0.150000in}}{\pgfqpoint{5.490039in}{5.490039in}}%
\pgfusepath{clip}%
\pgfsetbuttcap%
\pgfsetroundjoin%
\definecolor{currentfill}{rgb}{0.162142,0.474838,0.558140}%
\pgfsetfillcolor{currentfill}%
\pgfsetfillopacity{0.700000}%
\pgfsetlinewidth{0.000000pt}%
\definecolor{currentstroke}{rgb}{0.000000,0.000000,0.000000}%
\pgfsetstrokecolor{currentstroke}%
\pgfsetdash{}{0pt}%
\pgfpathmoveto{\pgfqpoint{5.373478in}{3.122204in}}%
\pgfpathlineto{\pgfqpoint{5.386988in}{3.125355in}}%
\pgfpathlineto{\pgfqpoint{5.400511in}{3.128656in}}%
\pgfpathlineto{\pgfqpoint{5.414047in}{3.132107in}}%
\pgfpathlineto{\pgfqpoint{5.427597in}{3.135708in}}%
\pgfpathlineto{\pgfqpoint{5.434673in}{3.144144in}}%
\pgfpathlineto{\pgfqpoint{5.441749in}{3.152775in}}%
\pgfpathlineto{\pgfqpoint{5.448827in}{3.161608in}}%
\pgfpathlineto{\pgfqpoint{5.455905in}{3.170651in}}%
\pgfpathlineto{\pgfqpoint{5.442380in}{3.167670in}}%
\pgfpathlineto{\pgfqpoint{5.428868in}{3.164838in}}%
\pgfpathlineto{\pgfqpoint{5.415369in}{3.162155in}}%
\pgfpathlineto{\pgfqpoint{5.401883in}{3.159622in}}%
\pgfpathlineto{\pgfqpoint{5.394781in}{3.149951in}}%
\pgfpathlineto{\pgfqpoint{5.387679in}{3.140495in}}%
\pgfpathlineto{\pgfqpoint{5.380579in}{3.131249in}}%
\pgfpathlineto{\pgfqpoint{5.373478in}{3.122204in}}%
\pgfpathclose%
\pgfusepath{fill}%
\end{pgfscope}%
\begin{pgfscope}%
\pgfpathrectangle{\pgfqpoint{1.254980in}{0.150000in}}{\pgfqpoint{5.490039in}{5.490039in}}%
\pgfusepath{clip}%
\pgfsetbuttcap%
\pgfsetroundjoin%
\definecolor{currentfill}{rgb}{0.283072,0.130895,0.449241}%
\pgfsetfillcolor{currentfill}%
\pgfsetfillopacity{0.700000}%
\pgfsetlinewidth{0.000000pt}%
\definecolor{currentstroke}{rgb}{0.000000,0.000000,0.000000}%
\pgfsetstrokecolor{currentstroke}%
\pgfsetdash{}{0pt}%
\pgfpathmoveto{\pgfqpoint{3.757328in}{2.355962in}}%
\pgfpathlineto{\pgfqpoint{3.770308in}{2.352337in}}%
\pgfpathlineto{\pgfqpoint{3.783293in}{2.348900in}}%
\pgfpathlineto{\pgfqpoint{3.796283in}{2.345649in}}%
\pgfpathlineto{\pgfqpoint{3.809278in}{2.342583in}}%
\pgfpathlineto{\pgfqpoint{3.816901in}{2.352460in}}%
\pgfpathlineto{\pgfqpoint{3.824519in}{2.362359in}}%
\pgfpathlineto{\pgfqpoint{3.832132in}{2.372283in}}%
\pgfpathlineto{\pgfqpoint{3.839740in}{2.382231in}}%
\pgfpathlineto{\pgfqpoint{3.826753in}{2.385383in}}%
\pgfpathlineto{\pgfqpoint{3.813773in}{2.388719in}}%
\pgfpathlineto{\pgfqpoint{3.800797in}{2.392242in}}%
\pgfpathlineto{\pgfqpoint{3.787825in}{2.395952in}}%
\pgfpathlineto{\pgfqpoint{3.780209in}{2.385907in}}%
\pgfpathlineto{\pgfqpoint{3.772587in}{2.375895in}}%
\pgfpathlineto{\pgfqpoint{3.764960in}{2.365913in}}%
\pgfpathlineto{\pgfqpoint{3.757328in}{2.355962in}}%
\pgfpathclose%
\pgfusepath{fill}%
\end{pgfscope}%
\begin{pgfscope}%
\pgfpathrectangle{\pgfqpoint{1.254980in}{0.150000in}}{\pgfqpoint{5.490039in}{5.490039in}}%
\pgfusepath{clip}%
\pgfsetbuttcap%
\pgfsetroundjoin%
\definecolor{currentfill}{rgb}{0.283197,0.115680,0.436115}%
\pgfsetfillcolor{currentfill}%
\pgfsetfillopacity{0.700000}%
\pgfsetlinewidth{0.000000pt}%
\definecolor{currentstroke}{rgb}{0.000000,0.000000,0.000000}%
\pgfsetstrokecolor{currentstroke}%
\pgfsetdash{}{0pt}%
\pgfpathmoveto{\pgfqpoint{3.540406in}{2.334622in}}%
\pgfpathlineto{\pgfqpoint{3.553361in}{2.328770in}}%
\pgfpathlineto{\pgfqpoint{3.566320in}{2.323117in}}%
\pgfpathlineto{\pgfqpoint{3.579281in}{2.317660in}}%
\pgfpathlineto{\pgfqpoint{3.592246in}{2.312401in}}%
\pgfpathlineto{\pgfqpoint{3.599938in}{2.322195in}}%
\pgfpathlineto{\pgfqpoint{3.607626in}{2.332024in}}%
\pgfpathlineto{\pgfqpoint{3.615308in}{2.341890in}}%
\pgfpathlineto{\pgfqpoint{3.622985in}{2.351792in}}%
\pgfpathlineto{\pgfqpoint{3.610031in}{2.357082in}}%
\pgfpathlineto{\pgfqpoint{3.597080in}{2.362568in}}%
\pgfpathlineto{\pgfqpoint{3.584132in}{2.368251in}}%
\pgfpathlineto{\pgfqpoint{3.571187in}{2.374133in}}%
\pgfpathlineto{\pgfqpoint{3.563500in}{2.364190in}}%
\pgfpathlineto{\pgfqpoint{3.555807in}{2.354292in}}%
\pgfpathlineto{\pgfqpoint{3.548109in}{2.344436in}}%
\pgfpathlineto{\pgfqpoint{3.540406in}{2.334622in}}%
\pgfpathclose%
\pgfusepath{fill}%
\end{pgfscope}%
\begin{pgfscope}%
\pgfpathrectangle{\pgfqpoint{1.254980in}{0.150000in}}{\pgfqpoint{5.490039in}{5.490039in}}%
\pgfusepath{clip}%
\pgfsetbuttcap%
\pgfsetroundjoin%
\definecolor{currentfill}{rgb}{0.154815,0.493313,0.557840}%
\pgfsetfillcolor{currentfill}%
\pgfsetfillopacity{0.700000}%
\pgfsetlinewidth{0.000000pt}%
\definecolor{currentstroke}{rgb}{0.000000,0.000000,0.000000}%
\pgfsetstrokecolor{currentstroke}%
\pgfsetdash{}{0pt}%
\pgfpathmoveto{\pgfqpoint{5.455905in}{3.170651in}}%
\pgfpathlineto{\pgfqpoint{5.469444in}{3.173782in}}%
\pgfpathlineto{\pgfqpoint{5.482996in}{3.177062in}}%
\pgfpathlineto{\pgfqpoint{5.496562in}{3.180492in}}%
\pgfpathlineto{\pgfqpoint{5.510142in}{3.184070in}}%
\pgfpathlineto{\pgfqpoint{5.517196in}{3.192692in}}%
\pgfpathlineto{\pgfqpoint{5.524251in}{3.201532in}}%
\pgfpathlineto{\pgfqpoint{5.531308in}{3.210596in}}%
\pgfpathlineto{\pgfqpoint{5.538367in}{3.219892in}}%
\pgfpathlineto{\pgfqpoint{5.524814in}{3.216961in}}%
\pgfpathlineto{\pgfqpoint{5.511274in}{3.214179in}}%
\pgfpathlineto{\pgfqpoint{5.497747in}{3.211545in}}%
\pgfpathlineto{\pgfqpoint{5.484234in}{3.209060in}}%
\pgfpathlineto{\pgfqpoint{5.477149in}{3.199108in}}%
\pgfpathlineto{\pgfqpoint{5.470066in}{3.189394in}}%
\pgfpathlineto{\pgfqpoint{5.462984in}{3.179911in}}%
\pgfpathlineto{\pgfqpoint{5.455905in}{3.170651in}}%
\pgfpathclose%
\pgfusepath{fill}%
\end{pgfscope}%
\begin{pgfscope}%
\pgfpathrectangle{\pgfqpoint{1.254980in}{0.150000in}}{\pgfqpoint{5.490039in}{5.490039in}}%
\pgfusepath{clip}%
\pgfsetbuttcap%
\pgfsetroundjoin%
\definecolor{currentfill}{rgb}{0.273006,0.204520,0.501721}%
\pgfsetfillcolor{currentfill}%
\pgfsetfillopacity{0.700000}%
\pgfsetlinewidth{0.000000pt}%
\definecolor{currentstroke}{rgb}{0.000000,0.000000,0.000000}%
\pgfsetstrokecolor{currentstroke}%
\pgfsetdash{}{0pt}%
\pgfpathmoveto{\pgfqpoint{3.032000in}{2.522169in}}%
\pgfpathlineto{\pgfqpoint{3.045013in}{2.509179in}}%
\pgfpathlineto{\pgfqpoint{3.058022in}{2.496435in}}%
\pgfpathlineto{\pgfqpoint{3.071028in}{2.483935in}}%
\pgfpathlineto{\pgfqpoint{3.084030in}{2.471677in}}%
\pgfpathlineto{\pgfqpoint{3.091901in}{2.480561in}}%
\pgfpathlineto{\pgfqpoint{3.099765in}{2.489534in}}%
\pgfpathlineto{\pgfqpoint{3.107622in}{2.498593in}}%
\pgfpathlineto{\pgfqpoint{3.115472in}{2.507740in}}%
\pgfpathlineto{\pgfqpoint{3.102487in}{2.519942in}}%
\pgfpathlineto{\pgfqpoint{3.089498in}{2.532386in}}%
\pgfpathlineto{\pgfqpoint{3.076505in}{2.545074in}}%
\pgfpathlineto{\pgfqpoint{3.063509in}{2.558008in}}%
\pgfpathlineto{\pgfqpoint{3.055643in}{2.548906in}}%
\pgfpathlineto{\pgfqpoint{3.047769in}{2.539900in}}%
\pgfpathlineto{\pgfqpoint{3.039888in}{2.530987in}}%
\pgfpathlineto{\pgfqpoint{3.032000in}{2.522169in}}%
\pgfpathclose%
\pgfusepath{fill}%
\end{pgfscope}%
\begin{pgfscope}%
\pgfpathrectangle{\pgfqpoint{1.254980in}{0.150000in}}{\pgfqpoint{5.490039in}{5.490039in}}%
\pgfusepath{clip}%
\pgfsetbuttcap%
\pgfsetroundjoin%
\definecolor{currentfill}{rgb}{0.276194,0.190074,0.493001}%
\pgfsetfillcolor{currentfill}%
\pgfsetfillopacity{0.700000}%
\pgfsetlinewidth{0.000000pt}%
\definecolor{currentstroke}{rgb}{0.000000,0.000000,0.000000}%
\pgfsetstrokecolor{currentstroke}%
\pgfsetdash{}{0pt}%
\pgfpathmoveto{\pgfqpoint{4.138838in}{2.471573in}}%
\pgfpathlineto{\pgfqpoint{4.151909in}{2.471034in}}%
\pgfpathlineto{\pgfqpoint{4.164987in}{2.470668in}}%
\pgfpathlineto{\pgfqpoint{4.178074in}{2.470474in}}%
\pgfpathlineto{\pgfqpoint{4.191168in}{2.470452in}}%
\pgfpathlineto{\pgfqpoint{4.198670in}{2.479958in}}%
\pgfpathlineto{\pgfqpoint{4.206167in}{2.489481in}}%
\pgfpathlineto{\pgfqpoint{4.213659in}{2.499026in}}%
\pgfpathlineto{\pgfqpoint{4.221146in}{2.508593in}}%
\pgfpathlineto{\pgfqpoint{4.208060in}{2.508813in}}%
\pgfpathlineto{\pgfqpoint{4.194982in}{2.509204in}}%
\pgfpathlineto{\pgfqpoint{4.181913in}{2.509767in}}%
\pgfpathlineto{\pgfqpoint{4.168851in}{2.510503in}}%
\pgfpathlineto{\pgfqpoint{4.161355in}{2.500728in}}%
\pgfpathlineto{\pgfqpoint{4.153854in}{2.490983in}}%
\pgfpathlineto{\pgfqpoint{4.146348in}{2.481265in}}%
\pgfpathlineto{\pgfqpoint{4.138838in}{2.471573in}}%
\pgfpathclose%
\pgfusepath{fill}%
\end{pgfscope}%
\begin{pgfscope}%
\pgfpathrectangle{\pgfqpoint{1.254980in}{0.150000in}}{\pgfqpoint{5.490039in}{5.490039in}}%
\pgfusepath{clip}%
\pgfsetbuttcap%
\pgfsetroundjoin%
\definecolor{currentfill}{rgb}{0.220057,0.343307,0.549413}%
\pgfsetfillcolor{currentfill}%
\pgfsetfillopacity{0.700000}%
\pgfsetlinewidth{0.000000pt}%
\definecolor{currentstroke}{rgb}{0.000000,0.000000,0.000000}%
\pgfsetstrokecolor{currentstroke}%
\pgfsetdash{}{0pt}%
\pgfpathmoveto{\pgfqpoint{2.770561in}{2.836918in}}%
\pgfpathlineto{\pgfqpoint{2.783703in}{2.818545in}}%
\pgfpathlineto{\pgfqpoint{2.796835in}{2.800464in}}%
\pgfpathlineto{\pgfqpoint{2.809959in}{2.782674in}}%
\pgfpathlineto{\pgfqpoint{2.823075in}{2.765171in}}%
\pgfpathlineto{\pgfqpoint{2.831039in}{2.773714in}}%
\pgfpathlineto{\pgfqpoint{2.838994in}{2.782380in}}%
\pgfpathlineto{\pgfqpoint{2.846941in}{2.791168in}}%
\pgfpathlineto{\pgfqpoint{2.854880in}{2.800079in}}%
\pgfpathlineto{\pgfqpoint{2.841785in}{2.817522in}}%
\pgfpathlineto{\pgfqpoint{2.828681in}{2.835252in}}%
\pgfpathlineto{\pgfqpoint{2.815569in}{2.853272in}}%
\pgfpathlineto{\pgfqpoint{2.802449in}{2.871585in}}%
\pgfpathlineto{\pgfqpoint{2.794490in}{2.862724in}}%
\pgfpathlineto{\pgfqpoint{2.786523in}{2.853992in}}%
\pgfpathlineto{\pgfqpoint{2.778546in}{2.845390in}}%
\pgfpathlineto{\pgfqpoint{2.770561in}{2.836918in}}%
\pgfpathclose%
\pgfusepath{fill}%
\end{pgfscope}%
\begin{pgfscope}%
\pgfpathrectangle{\pgfqpoint{1.254980in}{0.150000in}}{\pgfqpoint{5.490039in}{5.490039in}}%
\pgfusepath{clip}%
\pgfsetbuttcap%
\pgfsetroundjoin%
\definecolor{currentfill}{rgb}{0.282884,0.135920,0.453427}%
\pgfsetfillcolor{currentfill}%
\pgfsetfillopacity{0.700000}%
\pgfsetlinewidth{0.000000pt}%
\definecolor{currentstroke}{rgb}{0.000000,0.000000,0.000000}%
\pgfsetstrokecolor{currentstroke}%
\pgfsetdash{}{0pt}%
\pgfpathmoveto{\pgfqpoint{3.271124in}{2.379536in}}%
\pgfpathlineto{\pgfqpoint{3.284086in}{2.370321in}}%
\pgfpathlineto{\pgfqpoint{3.297049in}{2.361324in}}%
\pgfpathlineto{\pgfqpoint{3.310010in}{2.352544in}}%
\pgfpathlineto{\pgfqpoint{3.322972in}{2.343981in}}%
\pgfpathlineto{\pgfqpoint{3.330757in}{2.353365in}}%
\pgfpathlineto{\pgfqpoint{3.338537in}{2.362809in}}%
\pgfpathlineto{\pgfqpoint{3.346310in}{2.372314in}}%
\pgfpathlineto{\pgfqpoint{3.354077in}{2.381878in}}%
\pgfpathlineto{\pgfqpoint{3.341128in}{2.390415in}}%
\pgfpathlineto{\pgfqpoint{3.328179in}{2.399169in}}%
\pgfpathlineto{\pgfqpoint{3.315230in}{2.408139in}}%
\pgfpathlineto{\pgfqpoint{3.302282in}{2.417328in}}%
\pgfpathlineto{\pgfqpoint{3.294501in}{2.407779in}}%
\pgfpathlineto{\pgfqpoint{3.286715in}{2.398298in}}%
\pgfpathlineto{\pgfqpoint{3.278923in}{2.388884in}}%
\pgfpathlineto{\pgfqpoint{3.271124in}{2.379536in}}%
\pgfpathclose%
\pgfusepath{fill}%
\end{pgfscope}%
\begin{pgfscope}%
\pgfpathrectangle{\pgfqpoint{1.254980in}{0.150000in}}{\pgfqpoint{5.490039in}{5.490039in}}%
\pgfusepath{clip}%
\pgfsetbuttcap%
\pgfsetroundjoin%
\definecolor{currentfill}{rgb}{0.278826,0.175490,0.483397}%
\pgfsetfillcolor{currentfill}%
\pgfsetfillopacity{0.700000}%
\pgfsetlinewidth{0.000000pt}%
\definecolor{currentstroke}{rgb}{0.000000,0.000000,0.000000}%
\pgfsetstrokecolor{currentstroke}%
\pgfsetdash{}{0pt}%
\pgfpathmoveto{\pgfqpoint{4.056507in}{2.436220in}}%
\pgfpathlineto{\pgfqpoint{4.069556in}{2.435153in}}%
\pgfpathlineto{\pgfqpoint{4.082613in}{2.434262in}}%
\pgfpathlineto{\pgfqpoint{4.095677in}{2.433545in}}%
\pgfpathlineto{\pgfqpoint{4.108749in}{2.433002in}}%
\pgfpathlineto{\pgfqpoint{4.116278in}{2.442619in}}%
\pgfpathlineto{\pgfqpoint{4.123803in}{2.452252in}}%
\pgfpathlineto{\pgfqpoint{4.131323in}{2.461902in}}%
\pgfpathlineto{\pgfqpoint{4.138838in}{2.471573in}}%
\pgfpathlineto{\pgfqpoint{4.125775in}{2.472285in}}%
\pgfpathlineto{\pgfqpoint{4.112720in}{2.473171in}}%
\pgfpathlineto{\pgfqpoint{4.099671in}{2.474231in}}%
\pgfpathlineto{\pgfqpoint{4.086630in}{2.475467in}}%
\pgfpathlineto{\pgfqpoint{4.079107in}{2.465617in}}%
\pgfpathlineto{\pgfqpoint{4.071578in}{2.455794in}}%
\pgfpathlineto{\pgfqpoint{4.064045in}{2.445996in}}%
\pgfpathlineto{\pgfqpoint{4.056507in}{2.436220in}}%
\pgfpathclose%
\pgfusepath{fill}%
\end{pgfscope}%
\begin{pgfscope}%
\pgfpathrectangle{\pgfqpoint{1.254980in}{0.150000in}}{\pgfqpoint{5.490039in}{5.490039in}}%
\pgfusepath{clip}%
\pgfsetbuttcap%
\pgfsetroundjoin%
\definecolor{currentfill}{rgb}{0.278012,0.180367,0.486697}%
\pgfsetfillcolor{currentfill}%
\pgfsetfillopacity{0.700000}%
\pgfsetlinewidth{0.000000pt}%
\definecolor{currentstroke}{rgb}{0.000000,0.000000,0.000000}%
\pgfsetstrokecolor{currentstroke}%
\pgfsetdash{}{0pt}%
\pgfpathmoveto{\pgfqpoint{3.084030in}{2.471677in}}%
\pgfpathlineto{\pgfqpoint{3.097029in}{2.459659in}}%
\pgfpathlineto{\pgfqpoint{3.110025in}{2.447879in}}%
\pgfpathlineto{\pgfqpoint{3.123018in}{2.436336in}}%
\pgfpathlineto{\pgfqpoint{3.136008in}{2.425027in}}%
\pgfpathlineto{\pgfqpoint{3.143863in}{2.433978in}}%
\pgfpathlineto{\pgfqpoint{3.151711in}{2.443010in}}%
\pgfpathlineto{\pgfqpoint{3.159552in}{2.452122in}}%
\pgfpathlineto{\pgfqpoint{3.167387in}{2.461314in}}%
\pgfpathlineto{\pgfqpoint{3.154412in}{2.472567in}}%
\pgfpathlineto{\pgfqpoint{3.141435in}{2.484054in}}%
\pgfpathlineto{\pgfqpoint{3.128455in}{2.495778in}}%
\pgfpathlineto{\pgfqpoint{3.115472in}{2.507740in}}%
\pgfpathlineto{\pgfqpoint{3.107622in}{2.498593in}}%
\pgfpathlineto{\pgfqpoint{3.099765in}{2.489534in}}%
\pgfpathlineto{\pgfqpoint{3.091901in}{2.480561in}}%
\pgfpathlineto{\pgfqpoint{3.084030in}{2.471677in}}%
\pgfpathclose%
\pgfusepath{fill}%
\end{pgfscope}%
\begin{pgfscope}%
\pgfpathrectangle{\pgfqpoint{1.254980in}{0.150000in}}{\pgfqpoint{5.490039in}{5.490039in}}%
\pgfusepath{clip}%
\pgfsetbuttcap%
\pgfsetroundjoin%
\definecolor{currentfill}{rgb}{0.147607,0.511733,0.557049}%
\pgfsetfillcolor{currentfill}%
\pgfsetfillopacity{0.700000}%
\pgfsetlinewidth{0.000000pt}%
\definecolor{currentstroke}{rgb}{0.000000,0.000000,0.000000}%
\pgfsetstrokecolor{currentstroke}%
\pgfsetdash{}{0pt}%
\pgfpathmoveto{\pgfqpoint{5.538367in}{3.219892in}}%
\pgfpathlineto{\pgfqpoint{5.551935in}{3.222971in}}%
\pgfpathlineto{\pgfqpoint{5.565516in}{3.226199in}}%
\pgfpathlineto{\pgfqpoint{5.579110in}{3.229574in}}%
\pgfpathlineto{\pgfqpoint{5.592719in}{3.233098in}}%
\pgfpathlineto{\pgfqpoint{5.599754in}{3.241968in}}%
\pgfpathlineto{\pgfqpoint{5.606790in}{3.251078in}}%
\pgfpathlineto{\pgfqpoint{5.613830in}{3.260436in}}%
\pgfpathlineto{\pgfqpoint{5.600242in}{3.257416in}}%
\pgfpathlineto{\pgfqpoint{5.586668in}{3.254544in}}%
\pgfpathlineto{\pgfqpoint{5.573108in}{3.251820in}}%
\pgfpathlineto{\pgfqpoint{5.559561in}{3.249244in}}%
\pgfpathlineto{\pgfqpoint{5.552493in}{3.239209in}}%
\pgfpathlineto{\pgfqpoint{5.545429in}{3.229427in}}%
\pgfpathlineto{\pgfqpoint{5.538367in}{3.219892in}}%
\pgfpathclose%
\pgfusepath{fill}%
\end{pgfscope}%
\begin{pgfscope}%
\pgfpathrectangle{\pgfqpoint{1.254980in}{0.150000in}}{\pgfqpoint{5.490039in}{5.490039in}}%
\pgfusepath{clip}%
\pgfsetbuttcap%
\pgfsetroundjoin%
\definecolor{currentfill}{rgb}{0.283229,0.120777,0.440584}%
\pgfsetfillcolor{currentfill}%
\pgfsetfillopacity{0.700000}%
\pgfsetlinewidth{0.000000pt}%
\definecolor{currentstroke}{rgb}{0.000000,0.000000,0.000000}%
\pgfsetstrokecolor{currentstroke}%
\pgfsetdash{}{0pt}%
\pgfpathmoveto{\pgfqpoint{3.674836in}{2.332577in}}%
\pgfpathlineto{\pgfqpoint{3.687808in}{2.328254in}}%
\pgfpathlineto{\pgfqpoint{3.700784in}{2.324122in}}%
\pgfpathlineto{\pgfqpoint{3.713765in}{2.320179in}}%
\pgfpathlineto{\pgfqpoint{3.726750in}{2.316425in}}%
\pgfpathlineto{\pgfqpoint{3.734402in}{2.326271in}}%
\pgfpathlineto{\pgfqpoint{3.742049in}{2.336141in}}%
\pgfpathlineto{\pgfqpoint{3.749691in}{2.346038in}}%
\pgfpathlineto{\pgfqpoint{3.757328in}{2.355962in}}%
\pgfpathlineto{\pgfqpoint{3.744353in}{2.359773in}}%
\pgfpathlineto{\pgfqpoint{3.731381in}{2.363774in}}%
\pgfpathlineto{\pgfqpoint{3.718415in}{2.367964in}}%
\pgfpathlineto{\pgfqpoint{3.705452in}{2.372345in}}%
\pgfpathlineto{\pgfqpoint{3.697806in}{2.362353in}}%
\pgfpathlineto{\pgfqpoint{3.690154in}{2.352395in}}%
\pgfpathlineto{\pgfqpoint{3.682498in}{2.342470in}}%
\pgfpathlineto{\pgfqpoint{3.674836in}{2.332577in}}%
\pgfpathclose%
\pgfusepath{fill}%
\end{pgfscope}%
\begin{pgfscope}%
\pgfpathrectangle{\pgfqpoint{1.254980in}{0.150000in}}{\pgfqpoint{5.490039in}{5.490039in}}%
\pgfusepath{clip}%
\pgfsetbuttcap%
\pgfsetroundjoin%
\definecolor{currentfill}{rgb}{0.204903,0.375746,0.553533}%
\pgfsetfillcolor{currentfill}%
\pgfsetfillopacity{0.700000}%
\pgfsetlinewidth{0.000000pt}%
\definecolor{currentstroke}{rgb}{0.000000,0.000000,0.000000}%
\pgfsetstrokecolor{currentstroke}%
\pgfsetdash{}{0pt}%
\pgfpathmoveto{\pgfqpoint{2.717900in}{2.913399in}}%
\pgfpathlineto{\pgfqpoint{2.731081in}{2.893825in}}%
\pgfpathlineto{\pgfqpoint{2.744251in}{2.874555in}}%
\pgfpathlineto{\pgfqpoint{2.757411in}{2.855587in}}%
\pgfpathlineto{\pgfqpoint{2.770561in}{2.836918in}}%
\pgfpathlineto{\pgfqpoint{2.778546in}{2.845390in}}%
\pgfpathlineto{\pgfqpoint{2.786523in}{2.853992in}}%
\pgfpathlineto{\pgfqpoint{2.794490in}{2.862724in}}%
\pgfpathlineto{\pgfqpoint{2.802449in}{2.871585in}}%
\pgfpathlineto{\pgfqpoint{2.789319in}{2.890193in}}%
\pgfpathlineto{\pgfqpoint{2.776180in}{2.909100in}}%
\pgfpathlineto{\pgfqpoint{2.763032in}{2.928308in}}%
\pgfpathlineto{\pgfqpoint{2.749873in}{2.947820in}}%
\pgfpathlineto{\pgfqpoint{2.741894in}{2.939010in}}%
\pgfpathlineto{\pgfqpoint{2.733905in}{2.930336in}}%
\pgfpathlineto{\pgfqpoint{2.725907in}{2.921799in}}%
\pgfpathlineto{\pgfqpoint{2.717900in}{2.913399in}}%
\pgfpathclose%
\pgfusepath{fill}%
\end{pgfscope}%
\begin{pgfscope}%
\pgfpathrectangle{\pgfqpoint{1.254980in}{0.150000in}}{\pgfqpoint{5.490039in}{5.490039in}}%
\pgfusepath{clip}%
\pgfsetbuttcap%
\pgfsetroundjoin%
\definecolor{currentfill}{rgb}{0.280868,0.160771,0.472899}%
\pgfsetfillcolor{currentfill}%
\pgfsetfillopacity{0.700000}%
\pgfsetlinewidth{0.000000pt}%
\definecolor{currentstroke}{rgb}{0.000000,0.000000,0.000000}%
\pgfsetstrokecolor{currentstroke}%
\pgfsetdash{}{0pt}%
\pgfpathmoveto{\pgfqpoint{3.974144in}{2.402765in}}%
\pgfpathlineto{\pgfqpoint{3.987174in}{2.401133in}}%
\pgfpathlineto{\pgfqpoint{4.000211in}{2.399678in}}%
\pgfpathlineto{\pgfqpoint{4.013255in}{2.398401in}}%
\pgfpathlineto{\pgfqpoint{4.026306in}{2.397300in}}%
\pgfpathlineto{\pgfqpoint{4.033864in}{2.407006in}}%
\pgfpathlineto{\pgfqpoint{4.041416in}{2.416727in}}%
\pgfpathlineto{\pgfqpoint{4.048964in}{2.426464in}}%
\pgfpathlineto{\pgfqpoint{4.056507in}{2.436220in}}%
\pgfpathlineto{\pgfqpoint{4.043465in}{2.437462in}}%
\pgfpathlineto{\pgfqpoint{4.030429in}{2.438881in}}%
\pgfpathlineto{\pgfqpoint{4.017401in}{2.440477in}}%
\pgfpathlineto{\pgfqpoint{4.004379in}{2.442250in}}%
\pgfpathlineto{\pgfqpoint{3.996827in}{2.432343in}}%
\pgfpathlineto{\pgfqpoint{3.989271in}{2.422461in}}%
\pgfpathlineto{\pgfqpoint{3.981710in}{2.412603in}}%
\pgfpathlineto{\pgfqpoint{3.974144in}{2.402765in}}%
\pgfpathclose%
\pgfusepath{fill}%
\end{pgfscope}%
\begin{pgfscope}%
\pgfpathrectangle{\pgfqpoint{1.254980in}{0.150000in}}{\pgfqpoint{5.490039in}{5.490039in}}%
\pgfusepath{clip}%
\pgfsetbuttcap%
\pgfsetroundjoin%
\definecolor{currentfill}{rgb}{0.283091,0.110553,0.431554}%
\pgfsetfillcolor{currentfill}%
\pgfsetfillopacity{0.700000}%
\pgfsetlinewidth{0.000000pt}%
\definecolor{currentstroke}{rgb}{0.000000,0.000000,0.000000}%
\pgfsetstrokecolor{currentstroke}%
\pgfsetdash{}{0pt}%
\pgfpathmoveto{\pgfqpoint{3.457693in}{2.321185in}}%
\pgfpathlineto{\pgfqpoint{3.470651in}{2.314530in}}%
\pgfpathlineto{\pgfqpoint{3.483611in}{2.308078in}}%
\pgfpathlineto{\pgfqpoint{3.496573in}{2.301828in}}%
\pgfpathlineto{\pgfqpoint{3.509537in}{2.295779in}}%
\pgfpathlineto{\pgfqpoint{3.517263in}{2.305429in}}%
\pgfpathlineto{\pgfqpoint{3.524983in}{2.315119in}}%
\pgfpathlineto{\pgfqpoint{3.532697in}{2.324850in}}%
\pgfpathlineto{\pgfqpoint{3.540406in}{2.334622in}}%
\pgfpathlineto{\pgfqpoint{3.527452in}{2.340674in}}%
\pgfpathlineto{\pgfqpoint{3.514502in}{2.346926in}}%
\pgfpathlineto{\pgfqpoint{3.501553in}{2.353380in}}%
\pgfpathlineto{\pgfqpoint{3.488606in}{2.360037in}}%
\pgfpathlineto{\pgfqpoint{3.480886in}{2.350253in}}%
\pgfpathlineto{\pgfqpoint{3.473161in}{2.340516in}}%
\pgfpathlineto{\pgfqpoint{3.465430in}{2.330827in}}%
\pgfpathlineto{\pgfqpoint{3.457693in}{2.321185in}}%
\pgfpathclose%
\pgfusepath{fill}%
\end{pgfscope}%
\begin{pgfscope}%
\pgfpathrectangle{\pgfqpoint{1.254980in}{0.150000in}}{\pgfqpoint{5.490039in}{5.490039in}}%
\pgfusepath{clip}%
\pgfsetbuttcap%
\pgfsetroundjoin%
\definecolor{currentfill}{rgb}{0.280868,0.160771,0.472899}%
\pgfsetfillcolor{currentfill}%
\pgfsetfillopacity{0.700000}%
\pgfsetlinewidth{0.000000pt}%
\definecolor{currentstroke}{rgb}{0.000000,0.000000,0.000000}%
\pgfsetstrokecolor{currentstroke}%
\pgfsetdash{}{0pt}%
\pgfpathmoveto{\pgfqpoint{3.136008in}{2.425027in}}%
\pgfpathlineto{\pgfqpoint{3.148996in}{2.413952in}}%
\pgfpathlineto{\pgfqpoint{3.161982in}{2.403108in}}%
\pgfpathlineto{\pgfqpoint{3.174967in}{2.392494in}}%
\pgfpathlineto{\pgfqpoint{3.187949in}{2.382108in}}%
\pgfpathlineto{\pgfqpoint{3.195788in}{2.391124in}}%
\pgfpathlineto{\pgfqpoint{3.203621in}{2.400214in}}%
\pgfpathlineto{\pgfqpoint{3.211447in}{2.409377in}}%
\pgfpathlineto{\pgfqpoint{3.219266in}{2.418614in}}%
\pgfpathlineto{\pgfqpoint{3.206299in}{2.428946in}}%
\pgfpathlineto{\pgfqpoint{3.193330in}{2.439505in}}%
\pgfpathlineto{\pgfqpoint{3.180359in}{2.450294in}}%
\pgfpathlineto{\pgfqpoint{3.167387in}{2.461314in}}%
\pgfpathlineto{\pgfqpoint{3.159552in}{2.452122in}}%
\pgfpathlineto{\pgfqpoint{3.151711in}{2.443010in}}%
\pgfpathlineto{\pgfqpoint{3.143863in}{2.433978in}}%
\pgfpathlineto{\pgfqpoint{3.136008in}{2.425027in}}%
\pgfpathclose%
\pgfusepath{fill}%
\end{pgfscope}%
\begin{pgfscope}%
\pgfpathrectangle{\pgfqpoint{1.254980in}{0.150000in}}{\pgfqpoint{5.490039in}{5.490039in}}%
\pgfusepath{clip}%
\pgfsetbuttcap%
\pgfsetroundjoin%
\definecolor{currentfill}{rgb}{0.283229,0.120777,0.440584}%
\pgfsetfillcolor{currentfill}%
\pgfsetfillopacity{0.700000}%
\pgfsetlinewidth{0.000000pt}%
\definecolor{currentstroke}{rgb}{0.000000,0.000000,0.000000}%
\pgfsetstrokecolor{currentstroke}%
\pgfsetdash{}{0pt}%
\pgfpathmoveto{\pgfqpoint{3.322972in}{2.343981in}}%
\pgfpathlineto{\pgfqpoint{3.335934in}{2.335631in}}%
\pgfpathlineto{\pgfqpoint{3.348897in}{2.327495in}}%
\pgfpathlineto{\pgfqpoint{3.361860in}{2.319571in}}%
\pgfpathlineto{\pgfqpoint{3.374823in}{2.311857in}}%
\pgfpathlineto{\pgfqpoint{3.382596in}{2.321278in}}%
\pgfpathlineto{\pgfqpoint{3.390362in}{2.330752in}}%
\pgfpathlineto{\pgfqpoint{3.398122in}{2.340279in}}%
\pgfpathlineto{\pgfqpoint{3.405877in}{2.349859in}}%
\pgfpathlineto{\pgfqpoint{3.392925in}{2.357547in}}%
\pgfpathlineto{\pgfqpoint{3.379975in}{2.365445in}}%
\pgfpathlineto{\pgfqpoint{3.367026in}{2.373555in}}%
\pgfpathlineto{\pgfqpoint{3.354077in}{2.381878in}}%
\pgfpathlineto{\pgfqpoint{3.346310in}{2.372314in}}%
\pgfpathlineto{\pgfqpoint{3.338537in}{2.362809in}}%
\pgfpathlineto{\pgfqpoint{3.330757in}{2.353365in}}%
\pgfpathlineto{\pgfqpoint{3.322972in}{2.343981in}}%
\pgfpathclose%
\pgfusepath{fill}%
\end{pgfscope}%
\begin{pgfscope}%
\pgfpathrectangle{\pgfqpoint{1.254980in}{0.150000in}}{\pgfqpoint{5.490039in}{5.490039in}}%
\pgfusepath{clip}%
\pgfsetbuttcap%
\pgfsetroundjoin%
\definecolor{currentfill}{rgb}{0.282290,0.145912,0.461510}%
\pgfsetfillcolor{currentfill}%
\pgfsetfillopacity{0.700000}%
\pgfsetlinewidth{0.000000pt}%
\definecolor{currentstroke}{rgb}{0.000000,0.000000,0.000000}%
\pgfsetstrokecolor{currentstroke}%
\pgfsetdash{}{0pt}%
\pgfpathmoveto{\pgfqpoint{3.891738in}{2.371462in}}%
\pgfpathlineto{\pgfqpoint{3.904752in}{2.369225in}}%
\pgfpathlineto{\pgfqpoint{3.917772in}{2.367169in}}%
\pgfpathlineto{\pgfqpoint{3.930798in}{2.365292in}}%
\pgfpathlineto{\pgfqpoint{3.943831in}{2.363595in}}%
\pgfpathlineto{\pgfqpoint{3.951416in}{2.373364in}}%
\pgfpathlineto{\pgfqpoint{3.958997in}{2.383148in}}%
\pgfpathlineto{\pgfqpoint{3.966573in}{2.392948in}}%
\pgfpathlineto{\pgfqpoint{3.974144in}{2.402765in}}%
\pgfpathlineto{\pgfqpoint{3.961120in}{2.404576in}}%
\pgfpathlineto{\pgfqpoint{3.948102in}{2.406566in}}%
\pgfpathlineto{\pgfqpoint{3.935091in}{2.408736in}}%
\pgfpathlineto{\pgfqpoint{3.922085in}{2.411087in}}%
\pgfpathlineto{\pgfqpoint{3.914506in}{2.401145in}}%
\pgfpathlineto{\pgfqpoint{3.906922in}{2.391228in}}%
\pgfpathlineto{\pgfqpoint{3.899332in}{2.381334in}}%
\pgfpathlineto{\pgfqpoint{3.891738in}{2.371462in}}%
\pgfpathclose%
\pgfusepath{fill}%
\end{pgfscope}%
\begin{pgfscope}%
\pgfpathrectangle{\pgfqpoint{1.254980in}{0.150000in}}{\pgfqpoint{5.490039in}{5.490039in}}%
\pgfusepath{clip}%
\pgfsetbuttcap%
\pgfsetroundjoin%
\definecolor{currentfill}{rgb}{0.283091,0.110553,0.431554}%
\pgfsetfillcolor{currentfill}%
\pgfsetfillopacity{0.700000}%
\pgfsetlinewidth{0.000000pt}%
\definecolor{currentstroke}{rgb}{0.000000,0.000000,0.000000}%
\pgfsetstrokecolor{currentstroke}%
\pgfsetdash{}{0pt}%
\pgfpathmoveto{\pgfqpoint{3.592246in}{2.312401in}}%
\pgfpathlineto{\pgfqpoint{3.605213in}{2.307337in}}%
\pgfpathlineto{\pgfqpoint{3.618184in}{2.302467in}}%
\pgfpathlineto{\pgfqpoint{3.631158in}{2.297790in}}%
\pgfpathlineto{\pgfqpoint{3.644137in}{2.293307in}}%
\pgfpathlineto{\pgfqpoint{3.651819in}{2.303081in}}%
\pgfpathlineto{\pgfqpoint{3.659497in}{2.312883in}}%
\pgfpathlineto{\pgfqpoint{3.667169in}{2.322715in}}%
\pgfpathlineto{\pgfqpoint{3.674836in}{2.332577in}}%
\pgfpathlineto{\pgfqpoint{3.661868in}{2.337091in}}%
\pgfpathlineto{\pgfqpoint{3.648903in}{2.341798in}}%
\pgfpathlineto{\pgfqpoint{3.635942in}{2.346698in}}%
\pgfpathlineto{\pgfqpoint{3.622985in}{2.351792in}}%
\pgfpathlineto{\pgfqpoint{3.615308in}{2.341890in}}%
\pgfpathlineto{\pgfqpoint{3.607626in}{2.332024in}}%
\pgfpathlineto{\pgfqpoint{3.599938in}{2.322195in}}%
\pgfpathlineto{\pgfqpoint{3.592246in}{2.312401in}}%
\pgfpathclose%
\pgfusepath{fill}%
\end{pgfscope}%
\begin{pgfscope}%
\pgfpathrectangle{\pgfqpoint{1.254980in}{0.150000in}}{\pgfqpoint{5.490039in}{5.490039in}}%
\pgfusepath{clip}%
\pgfsetbuttcap%
\pgfsetroundjoin%
\definecolor{currentfill}{rgb}{0.283072,0.130895,0.449241}%
\pgfsetfillcolor{currentfill}%
\pgfsetfillopacity{0.700000}%
\pgfsetlinewidth{0.000000pt}%
\definecolor{currentstroke}{rgb}{0.000000,0.000000,0.000000}%
\pgfsetstrokecolor{currentstroke}%
\pgfsetdash{}{0pt}%
\pgfpathmoveto{\pgfqpoint{3.809278in}{2.342583in}}%
\pgfpathlineto{\pgfqpoint{3.822278in}{2.339702in}}%
\pgfpathlineto{\pgfqpoint{3.835284in}{2.337005in}}%
\pgfpathlineto{\pgfqpoint{3.848295in}{2.334490in}}%
\pgfpathlineto{\pgfqpoint{3.861312in}{2.332158in}}%
\pgfpathlineto{\pgfqpoint{3.868926in}{2.341959in}}%
\pgfpathlineto{\pgfqpoint{3.876535in}{2.351776in}}%
\pgfpathlineto{\pgfqpoint{3.884139in}{2.361610in}}%
\pgfpathlineto{\pgfqpoint{3.891738in}{2.371462in}}%
\pgfpathlineto{\pgfqpoint{3.878730in}{2.373880in}}%
\pgfpathlineto{\pgfqpoint{3.865728in}{2.376481in}}%
\pgfpathlineto{\pgfqpoint{3.852731in}{2.379264in}}%
\pgfpathlineto{\pgfqpoint{3.839740in}{2.382231in}}%
\pgfpathlineto{\pgfqpoint{3.832132in}{2.372283in}}%
\pgfpathlineto{\pgfqpoint{3.824519in}{2.362359in}}%
\pgfpathlineto{\pgfqpoint{3.816901in}{2.352460in}}%
\pgfpathlineto{\pgfqpoint{3.809278in}{2.342583in}}%
\pgfpathclose%
\pgfusepath{fill}%
\end{pgfscope}%
\begin{pgfscope}%
\pgfpathrectangle{\pgfqpoint{1.254980in}{0.150000in}}{\pgfqpoint{5.490039in}{5.490039in}}%
\pgfusepath{clip}%
\pgfsetbuttcap%
\pgfsetroundjoin%
\definecolor{currentfill}{rgb}{0.282290,0.145912,0.461510}%
\pgfsetfillcolor{currentfill}%
\pgfsetfillopacity{0.700000}%
\pgfsetlinewidth{0.000000pt}%
\definecolor{currentstroke}{rgb}{0.000000,0.000000,0.000000}%
\pgfsetstrokecolor{currentstroke}%
\pgfsetdash{}{0pt}%
\pgfpathmoveto{\pgfqpoint{3.187949in}{2.382108in}}%
\pgfpathlineto{\pgfqpoint{3.200930in}{2.371948in}}%
\pgfpathlineto{\pgfqpoint{3.213910in}{2.362013in}}%
\pgfpathlineto{\pgfqpoint{3.226888in}{2.352301in}}%
\pgfpathlineto{\pgfqpoint{3.239865in}{2.342811in}}%
\pgfpathlineto{\pgfqpoint{3.247690in}{2.351892in}}%
\pgfpathlineto{\pgfqpoint{3.255508in}{2.361040in}}%
\pgfpathlineto{\pgfqpoint{3.263319in}{2.370255in}}%
\pgfpathlineto{\pgfqpoint{3.271124in}{2.379536in}}%
\pgfpathlineto{\pgfqpoint{3.258161in}{2.388972in}}%
\pgfpathlineto{\pgfqpoint{3.245197in}{2.398629in}}%
\pgfpathlineto{\pgfqpoint{3.232232in}{2.408509in}}%
\pgfpathlineto{\pgfqpoint{3.219266in}{2.418614in}}%
\pgfpathlineto{\pgfqpoint{3.211447in}{2.409377in}}%
\pgfpathlineto{\pgfqpoint{3.203621in}{2.400214in}}%
\pgfpathlineto{\pgfqpoint{3.195788in}{2.391124in}}%
\pgfpathlineto{\pgfqpoint{3.187949in}{2.382108in}}%
\pgfpathclose%
\pgfusepath{fill}%
\end{pgfscope}%
\begin{pgfscope}%
\pgfpathrectangle{\pgfqpoint{1.254980in}{0.150000in}}{\pgfqpoint{5.490039in}{5.490039in}}%
\pgfusepath{clip}%
\pgfsetbuttcap%
\pgfsetroundjoin%
\definecolor{currentfill}{rgb}{0.239346,0.300855,0.540844}%
\pgfsetfillcolor{currentfill}%
\pgfsetfillopacity{0.700000}%
\pgfsetlinewidth{0.000000pt}%
\definecolor{currentstroke}{rgb}{0.000000,0.000000,0.000000}%
\pgfsetstrokecolor{currentstroke}%
\pgfsetdash{}{0pt}%
\pgfpathmoveto{\pgfqpoint{4.603136in}{2.676647in}}%
\pgfpathlineto{\pgfqpoint{4.616379in}{2.678847in}}%
\pgfpathlineto{\pgfqpoint{4.629633in}{2.681208in}}%
\pgfpathlineto{\pgfqpoint{4.642898in}{2.683731in}}%
\pgfpathlineto{\pgfqpoint{4.656173in}{2.686415in}}%
\pgfpathlineto{\pgfqpoint{4.663520in}{2.694875in}}%
\pgfpathlineto{\pgfqpoint{4.670862in}{2.703375in}}%
\pgfpathlineto{\pgfqpoint{4.678199in}{2.711919in}}%
\pgfpathlineto{\pgfqpoint{4.685532in}{2.720512in}}%
\pgfpathlineto{\pgfqpoint{4.672269in}{2.718167in}}%
\pgfpathlineto{\pgfqpoint{4.659016in}{2.715983in}}%
\pgfpathlineto{\pgfqpoint{4.645774in}{2.713960in}}%
\pgfpathlineto{\pgfqpoint{4.632543in}{2.712099in}}%
\pgfpathlineto{\pgfqpoint{4.625197in}{2.703157in}}%
\pgfpathlineto{\pgfqpoint{4.617848in}{2.694271in}}%
\pgfpathlineto{\pgfqpoint{4.610494in}{2.685436in}}%
\pgfpathlineto{\pgfqpoint{4.603136in}{2.676647in}}%
\pgfpathclose%
\pgfusepath{fill}%
\end{pgfscope}%
\begin{pgfscope}%
\pgfpathrectangle{\pgfqpoint{1.254980in}{0.150000in}}{\pgfqpoint{5.490039in}{5.490039in}}%
\pgfusepath{clip}%
\pgfsetbuttcap%
\pgfsetroundjoin%
\definecolor{currentfill}{rgb}{0.231674,0.318106,0.544834}%
\pgfsetfillcolor{currentfill}%
\pgfsetfillopacity{0.700000}%
\pgfsetlinewidth{0.000000pt}%
\definecolor{currentstroke}{rgb}{0.000000,0.000000,0.000000}%
\pgfsetstrokecolor{currentstroke}%
\pgfsetdash{}{0pt}%
\pgfpathmoveto{\pgfqpoint{4.685532in}{2.720512in}}%
\pgfpathlineto{\pgfqpoint{4.698807in}{2.723017in}}%
\pgfpathlineto{\pgfqpoint{4.712093in}{2.725682in}}%
\pgfpathlineto{\pgfqpoint{4.725390in}{2.728507in}}%
\pgfpathlineto{\pgfqpoint{4.738698in}{2.731492in}}%
\pgfpathlineto{\pgfqpoint{4.746014in}{2.739780in}}%
\pgfpathlineto{\pgfqpoint{4.753326in}{2.748118in}}%
\pgfpathlineto{\pgfqpoint{4.760633in}{2.756510in}}%
\pgfpathlineto{\pgfqpoint{4.767936in}{2.764960in}}%
\pgfpathlineto{\pgfqpoint{4.754641in}{2.762343in}}%
\pgfpathlineto{\pgfqpoint{4.741357in}{2.759885in}}%
\pgfpathlineto{\pgfqpoint{4.728084in}{2.757586in}}%
\pgfpathlineto{\pgfqpoint{4.714823in}{2.755448in}}%
\pgfpathlineto{\pgfqpoint{4.707506in}{2.746621in}}%
\pgfpathlineto{\pgfqpoint{4.700186in}{2.737859in}}%
\pgfpathlineto{\pgfqpoint{4.692861in}{2.729157in}}%
\pgfpathlineto{\pgfqpoint{4.685532in}{2.720512in}}%
\pgfpathclose%
\pgfusepath{fill}%
\end{pgfscope}%
\begin{pgfscope}%
\pgfpathrectangle{\pgfqpoint{1.254980in}{0.150000in}}{\pgfqpoint{5.490039in}{5.490039in}}%
\pgfusepath{clip}%
\pgfsetbuttcap%
\pgfsetroundjoin%
\definecolor{currentfill}{rgb}{0.221989,0.339161,0.548752}%
\pgfsetfillcolor{currentfill}%
\pgfsetfillopacity{0.700000}%
\pgfsetlinewidth{0.000000pt}%
\definecolor{currentstroke}{rgb}{0.000000,0.000000,0.000000}%
\pgfsetstrokecolor{currentstroke}%
\pgfsetdash{}{0pt}%
\pgfpathmoveto{\pgfqpoint{4.767936in}{2.764960in}}%
\pgfpathlineto{\pgfqpoint{4.781243in}{2.767736in}}%
\pgfpathlineto{\pgfqpoint{4.794561in}{2.770671in}}%
\pgfpathlineto{\pgfqpoint{4.807890in}{2.773764in}}%
\pgfpathlineto{\pgfqpoint{4.821232in}{2.777016in}}%
\pgfpathlineto{\pgfqpoint{4.828517in}{2.785143in}}%
\pgfpathlineto{\pgfqpoint{4.835798in}{2.793331in}}%
\pgfpathlineto{\pgfqpoint{4.843075in}{2.801584in}}%
\pgfpathlineto{\pgfqpoint{4.850348in}{2.809906in}}%
\pgfpathlineto{\pgfqpoint{4.837021in}{2.807050in}}%
\pgfpathlineto{\pgfqpoint{4.823706in}{2.804353in}}%
\pgfpathlineto{\pgfqpoint{4.810402in}{2.801813in}}%
\pgfpathlineto{\pgfqpoint{4.797109in}{2.799431in}}%
\pgfpathlineto{\pgfqpoint{4.789822in}{2.790704in}}%
\pgfpathlineto{\pgfqpoint{4.782530in}{2.782052in}}%
\pgfpathlineto{\pgfqpoint{4.775235in}{2.773473in}}%
\pgfpathlineto{\pgfqpoint{4.767936in}{2.764960in}}%
\pgfpathclose%
\pgfusepath{fill}%
\end{pgfscope}%
\begin{pgfscope}%
\pgfpathrectangle{\pgfqpoint{1.254980in}{0.150000in}}{\pgfqpoint{5.490039in}{5.490039in}}%
\pgfusepath{clip}%
\pgfsetbuttcap%
\pgfsetroundjoin%
\definecolor{currentfill}{rgb}{0.248629,0.278775,0.534556}%
\pgfsetfillcolor{currentfill}%
\pgfsetfillopacity{0.700000}%
\pgfsetlinewidth{0.000000pt}%
\definecolor{currentstroke}{rgb}{0.000000,0.000000,0.000000}%
\pgfsetstrokecolor{currentstroke}%
\pgfsetdash{}{0pt}%
\pgfpathmoveto{\pgfqpoint{4.520744in}{2.633470in}}%
\pgfpathlineto{\pgfqpoint{4.533957in}{2.635330in}}%
\pgfpathlineto{\pgfqpoint{4.547180in}{2.637354in}}%
\pgfpathlineto{\pgfqpoint{4.560413in}{2.639540in}}%
\pgfpathlineto{\pgfqpoint{4.573657in}{2.641889in}}%
\pgfpathlineto{\pgfqpoint{4.581034in}{2.650526in}}%
\pgfpathlineto{\pgfqpoint{4.588406in}{2.659196in}}%
\pgfpathlineto{\pgfqpoint{4.595773in}{2.667902in}}%
\pgfpathlineto{\pgfqpoint{4.603136in}{2.676647in}}%
\pgfpathlineto{\pgfqpoint{4.589903in}{2.674609in}}%
\pgfpathlineto{\pgfqpoint{4.576681in}{2.672734in}}%
\pgfpathlineto{\pgfqpoint{4.563469in}{2.671020in}}%
\pgfpathlineto{\pgfqpoint{4.550267in}{2.669470in}}%
\pgfpathlineto{\pgfqpoint{4.542893in}{2.660404in}}%
\pgfpathlineto{\pgfqpoint{4.535515in}{2.651385in}}%
\pgfpathlineto{\pgfqpoint{4.528132in}{2.642408in}}%
\pgfpathlineto{\pgfqpoint{4.520744in}{2.633470in}}%
\pgfpathclose%
\pgfusepath{fill}%
\end{pgfscope}%
\begin{pgfscope}%
\pgfpathrectangle{\pgfqpoint{1.254980in}{0.150000in}}{\pgfqpoint{5.490039in}{5.490039in}}%
\pgfusepath{clip}%
\pgfsetbuttcap%
\pgfsetroundjoin%
\definecolor{currentfill}{rgb}{0.214298,0.355619,0.551184}%
\pgfsetfillcolor{currentfill}%
\pgfsetfillopacity{0.700000}%
\pgfsetlinewidth{0.000000pt}%
\definecolor{currentstroke}{rgb}{0.000000,0.000000,0.000000}%
\pgfsetstrokecolor{currentstroke}%
\pgfsetdash{}{0pt}%
\pgfpathmoveto{\pgfqpoint{4.850348in}{2.809906in}}%
\pgfpathlineto{\pgfqpoint{4.863687in}{2.812920in}}%
\pgfpathlineto{\pgfqpoint{4.877038in}{2.816090in}}%
\pgfpathlineto{\pgfqpoint{4.890401in}{2.819419in}}%
\pgfpathlineto{\pgfqpoint{4.903776in}{2.822904in}}%
\pgfpathlineto{\pgfqpoint{4.911030in}{2.830886in}}%
\pgfpathlineto{\pgfqpoint{4.918280in}{2.838941in}}%
\pgfpathlineto{\pgfqpoint{4.925527in}{2.847073in}}%
\pgfpathlineto{\pgfqpoint{4.932770in}{2.855287in}}%
\pgfpathlineto{\pgfqpoint{4.919411in}{2.852226in}}%
\pgfpathlineto{\pgfqpoint{4.906063in}{2.849322in}}%
\pgfpathlineto{\pgfqpoint{4.892728in}{2.846575in}}%
\pgfpathlineto{\pgfqpoint{4.879404in}{2.843984in}}%
\pgfpathlineto{\pgfqpoint{4.872145in}{2.835337in}}%
\pgfpathlineto{\pgfqpoint{4.864883in}{2.826778in}}%
\pgfpathlineto{\pgfqpoint{4.857618in}{2.818303in}}%
\pgfpathlineto{\pgfqpoint{4.850348in}{2.809906in}}%
\pgfpathclose%
\pgfusepath{fill}%
\end{pgfscope}%
\begin{pgfscope}%
\pgfpathrectangle{\pgfqpoint{1.254980in}{0.150000in}}{\pgfqpoint{5.490039in}{5.490039in}}%
\pgfusepath{clip}%
\pgfsetbuttcap%
\pgfsetroundjoin%
\definecolor{currentfill}{rgb}{0.255645,0.260703,0.528312}%
\pgfsetfillcolor{currentfill}%
\pgfsetfillopacity{0.700000}%
\pgfsetlinewidth{0.000000pt}%
\definecolor{currentstroke}{rgb}{0.000000,0.000000,0.000000}%
\pgfsetstrokecolor{currentstroke}%
\pgfsetdash{}{0pt}%
\pgfpathmoveto{\pgfqpoint{4.438355in}{2.591106in}}%
\pgfpathlineto{\pgfqpoint{4.451538in}{2.592592in}}%
\pgfpathlineto{\pgfqpoint{4.464731in}{2.594242in}}%
\pgfpathlineto{\pgfqpoint{4.477934in}{2.596057in}}%
\pgfpathlineto{\pgfqpoint{4.491147in}{2.598037in}}%
\pgfpathlineto{\pgfqpoint{4.498553in}{2.606854in}}%
\pgfpathlineto{\pgfqpoint{4.505955in}{2.615696in}}%
\pgfpathlineto{\pgfqpoint{4.513352in}{2.624567in}}%
\pgfpathlineto{\pgfqpoint{4.520744in}{2.633470in}}%
\pgfpathlineto{\pgfqpoint{4.507542in}{2.631773in}}%
\pgfpathlineto{\pgfqpoint{4.494349in}{2.630241in}}%
\pgfpathlineto{\pgfqpoint{4.481166in}{2.628872in}}%
\pgfpathlineto{\pgfqpoint{4.467993in}{2.627668in}}%
\pgfpathlineto{\pgfqpoint{4.460591in}{2.618473in}}%
\pgfpathlineto{\pgfqpoint{4.453183in}{2.609317in}}%
\pgfpathlineto{\pgfqpoint{4.445772in}{2.600195in}}%
\pgfpathlineto{\pgfqpoint{4.438355in}{2.591106in}}%
\pgfpathclose%
\pgfusepath{fill}%
\end{pgfscope}%
\begin{pgfscope}%
\pgfpathrectangle{\pgfqpoint{1.254980in}{0.150000in}}{\pgfqpoint{5.490039in}{5.490039in}}%
\pgfusepath{clip}%
\pgfsetbuttcap%
\pgfsetroundjoin%
\definecolor{currentfill}{rgb}{0.204903,0.375746,0.553533}%
\pgfsetfillcolor{currentfill}%
\pgfsetfillopacity{0.700000}%
\pgfsetlinewidth{0.000000pt}%
\definecolor{currentstroke}{rgb}{0.000000,0.000000,0.000000}%
\pgfsetstrokecolor{currentstroke}%
\pgfsetdash{}{0pt}%
\pgfpathmoveto{\pgfqpoint{4.932770in}{2.855287in}}%
\pgfpathlineto{\pgfqpoint{4.946142in}{2.858504in}}%
\pgfpathlineto{\pgfqpoint{4.959526in}{2.861878in}}%
\pgfpathlineto{\pgfqpoint{4.972922in}{2.865407in}}%
\pgfpathlineto{\pgfqpoint{4.986330in}{2.869092in}}%
\pgfpathlineto{\pgfqpoint{4.993554in}{2.876951in}}%
\pgfpathlineto{\pgfqpoint{5.000773in}{2.884895in}}%
\pgfpathlineto{\pgfqpoint{5.007990in}{2.892930in}}%
\pgfpathlineto{\pgfqpoint{5.015203in}{2.901059in}}%
\pgfpathlineto{\pgfqpoint{5.001812in}{2.897827in}}%
\pgfpathlineto{\pgfqpoint{4.988432in}{2.894749in}}%
\pgfpathlineto{\pgfqpoint{4.975065in}{2.891828in}}%
\pgfpathlineto{\pgfqpoint{4.961709in}{2.889062in}}%
\pgfpathlineto{\pgfqpoint{4.954479in}{2.880470in}}%
\pgfpathlineto{\pgfqpoint{4.947246in}{2.871981in}}%
\pgfpathlineto{\pgfqpoint{4.940010in}{2.863588in}}%
\pgfpathlineto{\pgfqpoint{4.932770in}{2.855287in}}%
\pgfpathclose%
\pgfusepath{fill}%
\end{pgfscope}%
\begin{pgfscope}%
\pgfpathrectangle{\pgfqpoint{1.254980in}{0.150000in}}{\pgfqpoint{5.490039in}{5.490039in}}%
\pgfusepath{clip}%
\pgfsetbuttcap%
\pgfsetroundjoin%
\definecolor{currentfill}{rgb}{0.263663,0.237631,0.518762}%
\pgfsetfillcolor{currentfill}%
\pgfsetfillopacity{0.700000}%
\pgfsetlinewidth{0.000000pt}%
\definecolor{currentstroke}{rgb}{0.000000,0.000000,0.000000}%
\pgfsetstrokecolor{currentstroke}%
\pgfsetdash{}{0pt}%
\pgfpathmoveto{\pgfqpoint{4.355966in}{2.549701in}}%
\pgfpathlineto{\pgfqpoint{4.369120in}{2.550777in}}%
\pgfpathlineto{\pgfqpoint{4.382285in}{2.552019in}}%
\pgfpathlineto{\pgfqpoint{4.395458in}{2.553428in}}%
\pgfpathlineto{\pgfqpoint{4.408641in}{2.555003in}}%
\pgfpathlineto{\pgfqpoint{4.416077in}{2.563997in}}%
\pgfpathlineto{\pgfqpoint{4.423508in}{2.573010in}}%
\pgfpathlineto{\pgfqpoint{4.430934in}{2.582045in}}%
\pgfpathlineto{\pgfqpoint{4.438355in}{2.591106in}}%
\pgfpathlineto{\pgfqpoint{4.425182in}{2.589785in}}%
\pgfpathlineto{\pgfqpoint{4.412018in}{2.588631in}}%
\pgfpathlineto{\pgfqpoint{4.398863in}{2.587642in}}%
\pgfpathlineto{\pgfqpoint{4.385718in}{2.586820in}}%
\pgfpathlineto{\pgfqpoint{4.378287in}{2.577495in}}%
\pgfpathlineto{\pgfqpoint{4.370851in}{2.568202in}}%
\pgfpathlineto{\pgfqpoint{4.363411in}{2.558938in}}%
\pgfpathlineto{\pgfqpoint{4.355966in}{2.549701in}}%
\pgfpathclose%
\pgfusepath{fill}%
\end{pgfscope}%
\begin{pgfscope}%
\pgfpathrectangle{\pgfqpoint{1.254980in}{0.150000in}}{\pgfqpoint{5.490039in}{5.490039in}}%
\pgfusepath{clip}%
\pgfsetbuttcap%
\pgfsetroundjoin%
\definecolor{currentfill}{rgb}{0.195860,0.395433,0.555276}%
\pgfsetfillcolor{currentfill}%
\pgfsetfillopacity{0.700000}%
\pgfsetlinewidth{0.000000pt}%
\definecolor{currentstroke}{rgb}{0.000000,0.000000,0.000000}%
\pgfsetstrokecolor{currentstroke}%
\pgfsetdash{}{0pt}%
\pgfpathmoveto{\pgfqpoint{5.015203in}{2.901059in}}%
\pgfpathlineto{\pgfqpoint{5.028608in}{2.904447in}}%
\pgfpathlineto{\pgfqpoint{5.042024in}{2.907990in}}%
\pgfpathlineto{\pgfqpoint{5.055454in}{2.911688in}}%
\pgfpathlineto{\pgfqpoint{5.068896in}{2.915540in}}%
\pgfpathlineto{\pgfqpoint{5.076089in}{2.923300in}}%
\pgfpathlineto{\pgfqpoint{5.083278in}{2.931160in}}%
\pgfpathlineto{\pgfqpoint{5.090465in}{2.939125in}}%
\pgfpathlineto{\pgfqpoint{5.097649in}{2.947200in}}%
\pgfpathlineto{\pgfqpoint{5.084225in}{2.943828in}}%
\pgfpathlineto{\pgfqpoint{5.070813in}{2.940611in}}%
\pgfpathlineto{\pgfqpoint{5.057414in}{2.937548in}}%
\pgfpathlineto{\pgfqpoint{5.044027in}{2.934640in}}%
\pgfpathlineto{\pgfqpoint{5.036825in}{2.926075in}}%
\pgfpathlineto{\pgfqpoint{5.029621in}{2.917627in}}%
\pgfpathlineto{\pgfqpoint{5.022414in}{2.909290in}}%
\pgfpathlineto{\pgfqpoint{5.015203in}{2.901059in}}%
\pgfpathclose%
\pgfusepath{fill}%
\end{pgfscope}%
\begin{pgfscope}%
\pgfpathrectangle{\pgfqpoint{1.254980in}{0.150000in}}{\pgfqpoint{5.490039in}{5.490039in}}%
\pgfusepath{clip}%
\pgfsetbuttcap%
\pgfsetroundjoin%
\definecolor{currentfill}{rgb}{0.187231,0.414746,0.556547}%
\pgfsetfillcolor{currentfill}%
\pgfsetfillopacity{0.700000}%
\pgfsetlinewidth{0.000000pt}%
\definecolor{currentstroke}{rgb}{0.000000,0.000000,0.000000}%
\pgfsetstrokecolor{currentstroke}%
\pgfsetdash{}{0pt}%
\pgfpathmoveto{\pgfqpoint{5.097649in}{2.947200in}}%
\pgfpathlineto{\pgfqpoint{5.111086in}{2.950725in}}%
\pgfpathlineto{\pgfqpoint{5.124535in}{2.954404in}}%
\pgfpathlineto{\pgfqpoint{5.137998in}{2.958237in}}%
\pgfpathlineto{\pgfqpoint{5.151474in}{2.962224in}}%
\pgfpathlineto{\pgfqpoint{5.158636in}{2.969916in}}%
\pgfpathlineto{\pgfqpoint{5.165796in}{2.977724in}}%
\pgfpathlineto{\pgfqpoint{5.172953in}{2.985652in}}%
\pgfpathlineto{\pgfqpoint{5.180108in}{2.993706in}}%
\pgfpathlineto{\pgfqpoint{5.166652in}{2.990229in}}%
\pgfpathlineto{\pgfqpoint{5.153209in}{2.986904in}}%
\pgfpathlineto{\pgfqpoint{5.139778in}{2.983733in}}%
\pgfpathlineto{\pgfqpoint{5.126360in}{2.980715in}}%
\pgfpathlineto{\pgfqpoint{5.119186in}{2.972143in}}%
\pgfpathlineto{\pgfqpoint{5.112009in}{2.963703in}}%
\pgfpathlineto{\pgfqpoint{5.104830in}{2.955391in}}%
\pgfpathlineto{\pgfqpoint{5.097649in}{2.947200in}}%
\pgfpathclose%
\pgfusepath{fill}%
\end{pgfscope}%
\begin{pgfscope}%
\pgfpathrectangle{\pgfqpoint{1.254980in}{0.150000in}}{\pgfqpoint{5.490039in}{5.490039in}}%
\pgfusepath{clip}%
\pgfsetbuttcap%
\pgfsetroundjoin%
\definecolor{currentfill}{rgb}{0.269308,0.218818,0.509577}%
\pgfsetfillcolor{currentfill}%
\pgfsetfillopacity{0.700000}%
\pgfsetlinewidth{0.000000pt}%
\definecolor{currentstroke}{rgb}{0.000000,0.000000,0.000000}%
\pgfsetstrokecolor{currentstroke}%
\pgfsetdash{}{0pt}%
\pgfpathmoveto{\pgfqpoint{4.273572in}{2.509420in}}%
\pgfpathlineto{\pgfqpoint{4.286699in}{2.510051in}}%
\pgfpathlineto{\pgfqpoint{4.299836in}{2.510850in}}%
\pgfpathlineto{\pgfqpoint{4.312982in}{2.511818in}}%
\pgfpathlineto{\pgfqpoint{4.326137in}{2.512953in}}%
\pgfpathlineto{\pgfqpoint{4.333601in}{2.522115in}}%
\pgfpathlineto{\pgfqpoint{4.341061in}{2.531292in}}%
\pgfpathlineto{\pgfqpoint{4.348516in}{2.540486in}}%
\pgfpathlineto{\pgfqpoint{4.355966in}{2.549701in}}%
\pgfpathlineto{\pgfqpoint{4.342820in}{2.548792in}}%
\pgfpathlineto{\pgfqpoint{4.329683in}{2.548050in}}%
\pgfpathlineto{\pgfqpoint{4.316556in}{2.547477in}}%
\pgfpathlineto{\pgfqpoint{4.303437in}{2.547072in}}%
\pgfpathlineto{\pgfqpoint{4.295978in}{2.537621in}}%
\pgfpathlineto{\pgfqpoint{4.288514in}{2.528198in}}%
\pgfpathlineto{\pgfqpoint{4.281045in}{2.518798in}}%
\pgfpathlineto{\pgfqpoint{4.273572in}{2.509420in}}%
\pgfpathclose%
\pgfusepath{fill}%
\end{pgfscope}%
\begin{pgfscope}%
\pgfpathrectangle{\pgfqpoint{1.254980in}{0.150000in}}{\pgfqpoint{5.490039in}{5.490039in}}%
\pgfusepath{clip}%
\pgfsetbuttcap%
\pgfsetroundjoin%
\definecolor{currentfill}{rgb}{0.258965,0.251537,0.524736}%
\pgfsetfillcolor{currentfill}%
\pgfsetfillopacity{0.700000}%
\pgfsetlinewidth{0.000000pt}%
\definecolor{currentstroke}{rgb}{0.000000,0.000000,0.000000}%
\pgfsetstrokecolor{currentstroke}%
\pgfsetdash{}{0pt}%
\pgfpathmoveto{\pgfqpoint{2.895950in}{2.601548in}}%
\pgfpathlineto{\pgfqpoint{2.909022in}{2.586431in}}%
\pgfpathlineto{\pgfqpoint{2.922088in}{2.571577in}}%
\pgfpathlineto{\pgfqpoint{2.935148in}{2.556983in}}%
\pgfpathlineto{\pgfqpoint{2.948203in}{2.542648in}}%
\pgfpathlineto{\pgfqpoint{2.956140in}{2.550989in}}%
\pgfpathlineto{\pgfqpoint{2.964070in}{2.559433in}}%
\pgfpathlineto{\pgfqpoint{2.971991in}{2.567979in}}%
\pgfpathlineto{\pgfqpoint{2.979905in}{2.576628in}}%
\pgfpathlineto{\pgfqpoint{2.966869in}{2.590877in}}%
\pgfpathlineto{\pgfqpoint{2.953828in}{2.605385in}}%
\pgfpathlineto{\pgfqpoint{2.940782in}{2.620154in}}%
\pgfpathlineto{\pgfqpoint{2.927730in}{2.635185in}}%
\pgfpathlineto{\pgfqpoint{2.919797in}{2.626611in}}%
\pgfpathlineto{\pgfqpoint{2.911856in}{2.618147in}}%
\pgfpathlineto{\pgfqpoint{2.903907in}{2.609793in}}%
\pgfpathlineto{\pgfqpoint{2.895950in}{2.601548in}}%
\pgfpathclose%
\pgfusepath{fill}%
\end{pgfscope}%
\begin{pgfscope}%
\pgfpathrectangle{\pgfqpoint{1.254980in}{0.150000in}}{\pgfqpoint{5.490039in}{5.490039in}}%
\pgfusepath{clip}%
\pgfsetbuttcap%
\pgfsetroundjoin%
\definecolor{currentfill}{rgb}{0.179019,0.433756,0.557430}%
\pgfsetfillcolor{currentfill}%
\pgfsetfillopacity{0.700000}%
\pgfsetlinewidth{0.000000pt}%
\definecolor{currentstroke}{rgb}{0.000000,0.000000,0.000000}%
\pgfsetstrokecolor{currentstroke}%
\pgfsetdash{}{0pt}%
\pgfpathmoveto{\pgfqpoint{5.180108in}{2.993706in}}%
\pgfpathlineto{\pgfqpoint{5.193578in}{2.997336in}}%
\pgfpathlineto{\pgfqpoint{5.207060in}{3.001119in}}%
\pgfpathlineto{\pgfqpoint{5.220556in}{3.005055in}}%
\pgfpathlineto{\pgfqpoint{5.234065in}{3.009144in}}%
\pgfpathlineto{\pgfqpoint{5.241197in}{3.016803in}}%
\pgfpathlineto{\pgfqpoint{5.248328in}{3.024595in}}%
\pgfpathlineto{\pgfqpoint{5.255457in}{3.032524in}}%
\pgfpathlineto{\pgfqpoint{5.262584in}{3.040597in}}%
\pgfpathlineto{\pgfqpoint{5.249096in}{3.037046in}}%
\pgfpathlineto{\pgfqpoint{5.235622in}{3.033647in}}%
\pgfpathlineto{\pgfqpoint{5.222160in}{3.030400in}}%
\pgfpathlineto{\pgfqpoint{5.208711in}{3.027306in}}%
\pgfpathlineto{\pgfqpoint{5.201563in}{3.018687in}}%
\pgfpathlineto{\pgfqpoint{5.194413in}{3.010218in}}%
\pgfpathlineto{\pgfqpoint{5.187262in}{3.001893in}}%
\pgfpathlineto{\pgfqpoint{5.180108in}{2.993706in}}%
\pgfpathclose%
\pgfusepath{fill}%
\end{pgfscope}%
\begin{pgfscope}%
\pgfpathrectangle{\pgfqpoint{1.254980in}{0.150000in}}{\pgfqpoint{5.490039in}{5.490039in}}%
\pgfusepath{clip}%
\pgfsetbuttcap%
\pgfsetroundjoin%
\definecolor{currentfill}{rgb}{0.248629,0.278775,0.534556}%
\pgfsetfillcolor{currentfill}%
\pgfsetfillopacity{0.700000}%
\pgfsetlinewidth{0.000000pt}%
\definecolor{currentstroke}{rgb}{0.000000,0.000000,0.000000}%
\pgfsetstrokecolor{currentstroke}%
\pgfsetdash{}{0pt}%
\pgfpathmoveto{\pgfqpoint{2.843598in}{2.664692in}}%
\pgfpathlineto{\pgfqpoint{2.856697in}{2.648500in}}%
\pgfpathlineto{\pgfqpoint{2.869788in}{2.632580in}}%
\pgfpathlineto{\pgfqpoint{2.882872in}{2.616930in}}%
\pgfpathlineto{\pgfqpoint{2.895950in}{2.601548in}}%
\pgfpathlineto{\pgfqpoint{2.903907in}{2.609793in}}%
\pgfpathlineto{\pgfqpoint{2.911856in}{2.618147in}}%
\pgfpathlineto{\pgfqpoint{2.919797in}{2.626611in}}%
\pgfpathlineto{\pgfqpoint{2.927730in}{2.635185in}}%
\pgfpathlineto{\pgfqpoint{2.914671in}{2.650481in}}%
\pgfpathlineto{\pgfqpoint{2.901607in}{2.666044in}}%
\pgfpathlineto{\pgfqpoint{2.888536in}{2.681877in}}%
\pgfpathlineto{\pgfqpoint{2.875458in}{2.697981in}}%
\pgfpathlineto{\pgfqpoint{2.867506in}{2.689484in}}%
\pgfpathlineto{\pgfqpoint{2.859545in}{2.681103in}}%
\pgfpathlineto{\pgfqpoint{2.851576in}{2.672839in}}%
\pgfpathlineto{\pgfqpoint{2.843598in}{2.664692in}}%
\pgfpathclose%
\pgfusepath{fill}%
\end{pgfscope}%
\begin{pgfscope}%
\pgfpathrectangle{\pgfqpoint{1.254980in}{0.150000in}}{\pgfqpoint{5.490039in}{5.490039in}}%
\pgfusepath{clip}%
\pgfsetbuttcap%
\pgfsetroundjoin%
\definecolor{currentfill}{rgb}{0.283091,0.110553,0.431554}%
\pgfsetfillcolor{currentfill}%
\pgfsetfillopacity{0.700000}%
\pgfsetlinewidth{0.000000pt}%
\definecolor{currentstroke}{rgb}{0.000000,0.000000,0.000000}%
\pgfsetstrokecolor{currentstroke}%
\pgfsetdash{}{0pt}%
\pgfpathmoveto{\pgfqpoint{3.374823in}{2.311857in}}%
\pgfpathlineto{\pgfqpoint{3.387788in}{2.304352in}}%
\pgfpathlineto{\pgfqpoint{3.400754in}{2.297056in}}%
\pgfpathlineto{\pgfqpoint{3.413721in}{2.289966in}}%
\pgfpathlineto{\pgfqpoint{3.426689in}{2.283081in}}%
\pgfpathlineto{\pgfqpoint{3.434449in}{2.292538in}}%
\pgfpathlineto{\pgfqpoint{3.442203in}{2.302041in}}%
\pgfpathlineto{\pgfqpoint{3.449951in}{2.311590in}}%
\pgfpathlineto{\pgfqpoint{3.457693in}{2.321185in}}%
\pgfpathlineto{\pgfqpoint{3.444737in}{2.328044in}}%
\pgfpathlineto{\pgfqpoint{3.431782in}{2.335109in}}%
\pgfpathlineto{\pgfqpoint{3.418829in}{2.342380in}}%
\pgfpathlineto{\pgfqpoint{3.405877in}{2.349859in}}%
\pgfpathlineto{\pgfqpoint{3.398122in}{2.340279in}}%
\pgfpathlineto{\pgfqpoint{3.390362in}{2.330752in}}%
\pgfpathlineto{\pgfqpoint{3.382596in}{2.321278in}}%
\pgfpathlineto{\pgfqpoint{3.374823in}{2.311857in}}%
\pgfpathclose%
\pgfusepath{fill}%
\end{pgfscope}%
\begin{pgfscope}%
\pgfpathrectangle{\pgfqpoint{1.254980in}{0.150000in}}{\pgfqpoint{5.490039in}{5.490039in}}%
\pgfusepath{clip}%
\pgfsetbuttcap%
\pgfsetroundjoin%
\definecolor{currentfill}{rgb}{0.283229,0.120777,0.440584}%
\pgfsetfillcolor{currentfill}%
\pgfsetfillopacity{0.700000}%
\pgfsetlinewidth{0.000000pt}%
\definecolor{currentstroke}{rgb}{0.000000,0.000000,0.000000}%
\pgfsetstrokecolor{currentstroke}%
\pgfsetdash{}{0pt}%
\pgfpathmoveto{\pgfqpoint{3.726750in}{2.316425in}}%
\pgfpathlineto{\pgfqpoint{3.739739in}{2.312860in}}%
\pgfpathlineto{\pgfqpoint{3.752733in}{2.309481in}}%
\pgfpathlineto{\pgfqpoint{3.765732in}{2.306288in}}%
\pgfpathlineto{\pgfqpoint{3.778736in}{2.303281in}}%
\pgfpathlineto{\pgfqpoint{3.786379in}{2.313078in}}%
\pgfpathlineto{\pgfqpoint{3.794017in}{2.322893in}}%
\pgfpathlineto{\pgfqpoint{3.801650in}{2.332728in}}%
\pgfpathlineto{\pgfqpoint{3.809278in}{2.342583in}}%
\pgfpathlineto{\pgfqpoint{3.796283in}{2.345649in}}%
\pgfpathlineto{\pgfqpoint{3.783293in}{2.348900in}}%
\pgfpathlineto{\pgfqpoint{3.770308in}{2.352337in}}%
\pgfpathlineto{\pgfqpoint{3.757328in}{2.355962in}}%
\pgfpathlineto{\pgfqpoint{3.749691in}{2.346038in}}%
\pgfpathlineto{\pgfqpoint{3.742049in}{2.336141in}}%
\pgfpathlineto{\pgfqpoint{3.734402in}{2.326271in}}%
\pgfpathlineto{\pgfqpoint{3.726750in}{2.316425in}}%
\pgfpathclose%
\pgfusepath{fill}%
\end{pgfscope}%
\begin{pgfscope}%
\pgfpathrectangle{\pgfqpoint{1.254980in}{0.150000in}}{\pgfqpoint{5.490039in}{5.490039in}}%
\pgfusepath{clip}%
\pgfsetbuttcap%
\pgfsetroundjoin%
\definecolor{currentfill}{rgb}{0.274128,0.199721,0.498911}%
\pgfsetfillcolor{currentfill}%
\pgfsetfillopacity{0.700000}%
\pgfsetlinewidth{0.000000pt}%
\definecolor{currentstroke}{rgb}{0.000000,0.000000,0.000000}%
\pgfsetstrokecolor{currentstroke}%
\pgfsetdash{}{0pt}%
\pgfpathmoveto{\pgfqpoint{4.191168in}{2.470452in}}%
\pgfpathlineto{\pgfqpoint{4.204271in}{2.470601in}}%
\pgfpathlineto{\pgfqpoint{4.217382in}{2.470921in}}%
\pgfpathlineto{\pgfqpoint{4.230501in}{2.471411in}}%
\pgfpathlineto{\pgfqpoint{4.243629in}{2.472071in}}%
\pgfpathlineto{\pgfqpoint{4.251122in}{2.481389in}}%
\pgfpathlineto{\pgfqpoint{4.258610in}{2.490718in}}%
\pgfpathlineto{\pgfqpoint{4.266093in}{2.500061in}}%
\pgfpathlineto{\pgfqpoint{4.273572in}{2.509420in}}%
\pgfpathlineto{\pgfqpoint{4.260452in}{2.508959in}}%
\pgfpathlineto{\pgfqpoint{4.247342in}{2.508667in}}%
\pgfpathlineto{\pgfqpoint{4.234240in}{2.508545in}}%
\pgfpathlineto{\pgfqpoint{4.221146in}{2.508593in}}%
\pgfpathlineto{\pgfqpoint{4.213659in}{2.499026in}}%
\pgfpathlineto{\pgfqpoint{4.206167in}{2.489481in}}%
\pgfpathlineto{\pgfqpoint{4.198670in}{2.479958in}}%
\pgfpathlineto{\pgfqpoint{4.191168in}{2.470452in}}%
\pgfpathclose%
\pgfusepath{fill}%
\end{pgfscope}%
\begin{pgfscope}%
\pgfpathrectangle{\pgfqpoint{1.254980in}{0.150000in}}{\pgfqpoint{5.490039in}{5.490039in}}%
\pgfusepath{clip}%
\pgfsetbuttcap%
\pgfsetroundjoin%
\definecolor{currentfill}{rgb}{0.171176,0.452530,0.557965}%
\pgfsetfillcolor{currentfill}%
\pgfsetfillopacity{0.700000}%
\pgfsetlinewidth{0.000000pt}%
\definecolor{currentstroke}{rgb}{0.000000,0.000000,0.000000}%
\pgfsetstrokecolor{currentstroke}%
\pgfsetdash{}{0pt}%
\pgfpathmoveto{\pgfqpoint{5.262584in}{3.040597in}}%
\pgfpathlineto{\pgfqpoint{5.276086in}{3.044299in}}%
\pgfpathlineto{\pgfqpoint{5.289600in}{3.048154in}}%
\pgfpathlineto{\pgfqpoint{5.303128in}{3.052160in}}%
\pgfpathlineto{\pgfqpoint{5.316670in}{3.056318in}}%
\pgfpathlineto{\pgfqpoint{5.323774in}{3.063985in}}%
\pgfpathlineto{\pgfqpoint{5.330877in}{3.071802in}}%
\pgfpathlineto{\pgfqpoint{5.337979in}{3.079775in}}%
\pgfpathlineto{\pgfqpoint{5.345079in}{3.087911in}}%
\pgfpathlineto{\pgfqpoint{5.331560in}{3.084319in}}%
\pgfpathlineto{\pgfqpoint{5.318054in}{3.080878in}}%
\pgfpathlineto{\pgfqpoint{5.304562in}{3.077588in}}%
\pgfpathlineto{\pgfqpoint{5.291082in}{3.074449in}}%
\pgfpathlineto{\pgfqpoint{5.283959in}{3.065739in}}%
\pgfpathlineto{\pgfqpoint{5.276835in}{3.057198in}}%
\pgfpathlineto{\pgfqpoint{5.269710in}{3.048819in}}%
\pgfpathlineto{\pgfqpoint{5.262584in}{3.040597in}}%
\pgfpathclose%
\pgfusepath{fill}%
\end{pgfscope}%
\begin{pgfscope}%
\pgfpathrectangle{\pgfqpoint{1.254980in}{0.150000in}}{\pgfqpoint{5.490039in}{5.490039in}}%
\pgfusepath{clip}%
\pgfsetbuttcap%
\pgfsetroundjoin%
\definecolor{currentfill}{rgb}{0.267968,0.223549,0.512008}%
\pgfsetfillcolor{currentfill}%
\pgfsetfillopacity{0.700000}%
\pgfsetlinewidth{0.000000pt}%
\definecolor{currentstroke}{rgb}{0.000000,0.000000,0.000000}%
\pgfsetstrokecolor{currentstroke}%
\pgfsetdash{}{0pt}%
\pgfpathmoveto{\pgfqpoint{2.948203in}{2.542648in}}%
\pgfpathlineto{\pgfqpoint{2.961252in}{2.528569in}}%
\pgfpathlineto{\pgfqpoint{2.974297in}{2.514745in}}%
\pgfpathlineto{\pgfqpoint{2.987337in}{2.501172in}}%
\pgfpathlineto{\pgfqpoint{3.000372in}{2.487850in}}%
\pgfpathlineto{\pgfqpoint{3.008290in}{2.496286in}}%
\pgfpathlineto{\pgfqpoint{3.016201in}{2.504818in}}%
\pgfpathlineto{\pgfqpoint{3.024104in}{2.513446in}}%
\pgfpathlineto{\pgfqpoint{3.032000in}{2.522169in}}%
\pgfpathlineto{\pgfqpoint{3.018983in}{2.535407in}}%
\pgfpathlineto{\pgfqpoint{3.005962in}{2.548895in}}%
\pgfpathlineto{\pgfqpoint{2.992936in}{2.562634in}}%
\pgfpathlineto{\pgfqpoint{2.979905in}{2.576628in}}%
\pgfpathlineto{\pgfqpoint{2.971991in}{2.567979in}}%
\pgfpathlineto{\pgfqpoint{2.964070in}{2.559433in}}%
\pgfpathlineto{\pgfqpoint{2.956140in}{2.550989in}}%
\pgfpathlineto{\pgfqpoint{2.948203in}{2.542648in}}%
\pgfpathclose%
\pgfusepath{fill}%
\end{pgfscope}%
\begin{pgfscope}%
\pgfpathrectangle{\pgfqpoint{1.254980in}{0.150000in}}{\pgfqpoint{5.490039in}{5.490039in}}%
\pgfusepath{clip}%
\pgfsetbuttcap%
\pgfsetroundjoin%
\definecolor{currentfill}{rgb}{0.282910,0.105393,0.426902}%
\pgfsetfillcolor{currentfill}%
\pgfsetfillopacity{0.700000}%
\pgfsetlinewidth{0.000000pt}%
\definecolor{currentstroke}{rgb}{0.000000,0.000000,0.000000}%
\pgfsetstrokecolor{currentstroke}%
\pgfsetdash{}{0pt}%
\pgfpathmoveto{\pgfqpoint{3.509537in}{2.295779in}}%
\pgfpathlineto{\pgfqpoint{3.522504in}{2.289929in}}%
\pgfpathlineto{\pgfqpoint{3.535474in}{2.284278in}}%
\pgfpathlineto{\pgfqpoint{3.548446in}{2.278825in}}%
\pgfpathlineto{\pgfqpoint{3.561421in}{2.273568in}}%
\pgfpathlineto{\pgfqpoint{3.569135in}{2.283226in}}%
\pgfpathlineto{\pgfqpoint{3.576844in}{2.292917in}}%
\pgfpathlineto{\pgfqpoint{3.584547in}{2.302642in}}%
\pgfpathlineto{\pgfqpoint{3.592246in}{2.312401in}}%
\pgfpathlineto{\pgfqpoint{3.579281in}{2.317660in}}%
\pgfpathlineto{\pgfqpoint{3.566320in}{2.323117in}}%
\pgfpathlineto{\pgfqpoint{3.553361in}{2.328770in}}%
\pgfpathlineto{\pgfqpoint{3.540406in}{2.334622in}}%
\pgfpathlineto{\pgfqpoint{3.532697in}{2.324850in}}%
\pgfpathlineto{\pgfqpoint{3.524983in}{2.315119in}}%
\pgfpathlineto{\pgfqpoint{3.517263in}{2.305429in}}%
\pgfpathlineto{\pgfqpoint{3.509537in}{2.295779in}}%
\pgfpathclose%
\pgfusepath{fill}%
\end{pgfscope}%
\begin{pgfscope}%
\pgfpathrectangle{\pgfqpoint{1.254980in}{0.150000in}}{\pgfqpoint{5.490039in}{5.490039in}}%
\pgfusepath{clip}%
\pgfsetbuttcap%
\pgfsetroundjoin%
\definecolor{currentfill}{rgb}{0.235526,0.309527,0.542944}%
\pgfsetfillcolor{currentfill}%
\pgfsetfillopacity{0.700000}%
\pgfsetlinewidth{0.000000pt}%
\definecolor{currentstroke}{rgb}{0.000000,0.000000,0.000000}%
\pgfsetstrokecolor{currentstroke}%
\pgfsetdash{}{0pt}%
\pgfpathmoveto{\pgfqpoint{2.791131in}{2.732233in}}%
\pgfpathlineto{\pgfqpoint{2.804260in}{2.714927in}}%
\pgfpathlineto{\pgfqpoint{2.817380in}{2.697903in}}%
\pgfpathlineto{\pgfqpoint{2.830493in}{2.681159in}}%
\pgfpathlineto{\pgfqpoint{2.843598in}{2.664692in}}%
\pgfpathlineto{\pgfqpoint{2.851576in}{2.672839in}}%
\pgfpathlineto{\pgfqpoint{2.859545in}{2.681103in}}%
\pgfpathlineto{\pgfqpoint{2.867506in}{2.689484in}}%
\pgfpathlineto{\pgfqpoint{2.875458in}{2.697981in}}%
\pgfpathlineto{\pgfqpoint{2.862374in}{2.714361in}}%
\pgfpathlineto{\pgfqpoint{2.849282in}{2.731017in}}%
\pgfpathlineto{\pgfqpoint{2.836182in}{2.747953in}}%
\pgfpathlineto{\pgfqpoint{2.823075in}{2.765171in}}%
\pgfpathlineto{\pgfqpoint{2.815102in}{2.756751in}}%
\pgfpathlineto{\pgfqpoint{2.807120in}{2.748454in}}%
\pgfpathlineto{\pgfqpoint{2.799130in}{2.740282in}}%
\pgfpathlineto{\pgfqpoint{2.791131in}{2.732233in}}%
\pgfpathclose%
\pgfusepath{fill}%
\end{pgfscope}%
\begin{pgfscope}%
\pgfpathrectangle{\pgfqpoint{1.254980in}{0.150000in}}{\pgfqpoint{5.490039in}{5.490039in}}%
\pgfusepath{clip}%
\pgfsetbuttcap%
\pgfsetroundjoin%
\definecolor{currentfill}{rgb}{0.163625,0.471133,0.558148}%
\pgfsetfillcolor{currentfill}%
\pgfsetfillopacity{0.700000}%
\pgfsetlinewidth{0.000000pt}%
\definecolor{currentstroke}{rgb}{0.000000,0.000000,0.000000}%
\pgfsetstrokecolor{currentstroke}%
\pgfsetdash{}{0pt}%
\pgfpathmoveto{\pgfqpoint{5.345079in}{3.087911in}}%
\pgfpathlineto{\pgfqpoint{5.358612in}{3.091654in}}%
\pgfpathlineto{\pgfqpoint{5.372159in}{3.095547in}}%
\pgfpathlineto{\pgfqpoint{5.385719in}{3.099592in}}%
\pgfpathlineto{\pgfqpoint{5.399293in}{3.103787in}}%
\pgfpathlineto{\pgfqpoint{5.406369in}{3.111507in}}%
\pgfpathlineto{\pgfqpoint{5.413445in}{3.119397in}}%
\pgfpathlineto{\pgfqpoint{5.420521in}{3.127461in}}%
\pgfpathlineto{\pgfqpoint{5.427597in}{3.135708in}}%
\pgfpathlineto{\pgfqpoint{5.414047in}{3.132107in}}%
\pgfpathlineto{\pgfqpoint{5.400511in}{3.128656in}}%
\pgfpathlineto{\pgfqpoint{5.386988in}{3.125355in}}%
\pgfpathlineto{\pgfqpoint{5.373478in}{3.122204in}}%
\pgfpathlineto{\pgfqpoint{5.366379in}{3.113355in}}%
\pgfpathlineto{\pgfqpoint{5.359279in}{3.104694in}}%
\pgfpathlineto{\pgfqpoint{5.352180in}{3.096215in}}%
\pgfpathlineto{\pgfqpoint{5.345079in}{3.087911in}}%
\pgfpathclose%
\pgfusepath{fill}%
\end{pgfscope}%
\begin{pgfscope}%
\pgfpathrectangle{\pgfqpoint{1.254980in}{0.150000in}}{\pgfqpoint{5.490039in}{5.490039in}}%
\pgfusepath{clip}%
\pgfsetbuttcap%
\pgfsetroundjoin%
\definecolor{currentfill}{rgb}{0.277134,0.185228,0.489898}%
\pgfsetfillcolor{currentfill}%
\pgfsetfillopacity{0.700000}%
\pgfsetlinewidth{0.000000pt}%
\definecolor{currentstroke}{rgb}{0.000000,0.000000,0.000000}%
\pgfsetstrokecolor{currentstroke}%
\pgfsetdash{}{0pt}%
\pgfpathmoveto{\pgfqpoint{4.108749in}{2.433002in}}%
\pgfpathlineto{\pgfqpoint{4.121828in}{2.432633in}}%
\pgfpathlineto{\pgfqpoint{4.134915in}{2.432437in}}%
\pgfpathlineto{\pgfqpoint{4.148010in}{2.432413in}}%
\pgfpathlineto{\pgfqpoint{4.161113in}{2.432561in}}%
\pgfpathlineto{\pgfqpoint{4.168634in}{2.442018in}}%
\pgfpathlineto{\pgfqpoint{4.176150in}{2.451484in}}%
\pgfpathlineto{\pgfqpoint{4.183662in}{2.460961in}}%
\pgfpathlineto{\pgfqpoint{4.191168in}{2.470452in}}%
\pgfpathlineto{\pgfqpoint{4.178074in}{2.470474in}}%
\pgfpathlineto{\pgfqpoint{4.164987in}{2.470668in}}%
\pgfpathlineto{\pgfqpoint{4.151909in}{2.471034in}}%
\pgfpathlineto{\pgfqpoint{4.138838in}{2.471573in}}%
\pgfpathlineto{\pgfqpoint{4.131323in}{2.461902in}}%
\pgfpathlineto{\pgfqpoint{4.123803in}{2.452252in}}%
\pgfpathlineto{\pgfqpoint{4.116278in}{2.442619in}}%
\pgfpathlineto{\pgfqpoint{4.108749in}{2.433002in}}%
\pgfpathclose%
\pgfusepath{fill}%
\end{pgfscope}%
\begin{pgfscope}%
\pgfpathrectangle{\pgfqpoint{1.254980in}{0.150000in}}{\pgfqpoint{5.490039in}{5.490039in}}%
\pgfusepath{clip}%
\pgfsetbuttcap%
\pgfsetroundjoin%
\definecolor{currentfill}{rgb}{0.156270,0.489624,0.557936}%
\pgfsetfillcolor{currentfill}%
\pgfsetfillopacity{0.700000}%
\pgfsetlinewidth{0.000000pt}%
\definecolor{currentstroke}{rgb}{0.000000,0.000000,0.000000}%
\pgfsetstrokecolor{currentstroke}%
\pgfsetdash{}{0pt}%
\pgfpathmoveto{\pgfqpoint{5.427597in}{3.135708in}}%
\pgfpathlineto{\pgfqpoint{5.441161in}{3.139459in}}%
\pgfpathlineto{\pgfqpoint{5.454738in}{3.143360in}}%
\pgfpathlineto{\pgfqpoint{5.468330in}{3.147411in}}%
\pgfpathlineto{\pgfqpoint{5.481935in}{3.151611in}}%
\pgfpathlineto{\pgfqpoint{5.488986in}{3.159435in}}%
\pgfpathlineto{\pgfqpoint{5.496037in}{3.167448in}}%
\pgfpathlineto{\pgfqpoint{5.503089in}{3.175658in}}%
\pgfpathlineto{\pgfqpoint{5.510142in}{3.184070in}}%
\pgfpathlineto{\pgfqpoint{5.496562in}{3.180492in}}%
\pgfpathlineto{\pgfqpoint{5.482996in}{3.177062in}}%
\pgfpathlineto{\pgfqpoint{5.469444in}{3.173782in}}%
\pgfpathlineto{\pgfqpoint{5.455905in}{3.170651in}}%
\pgfpathlineto{\pgfqpoint{5.448827in}{3.161608in}}%
\pgfpathlineto{\pgfqpoint{5.441749in}{3.152775in}}%
\pgfpathlineto{\pgfqpoint{5.434673in}{3.144144in}}%
\pgfpathlineto{\pgfqpoint{5.427597in}{3.135708in}}%
\pgfpathclose%
\pgfusepath{fill}%
\end{pgfscope}%
\begin{pgfscope}%
\pgfpathrectangle{\pgfqpoint{1.254980in}{0.150000in}}{\pgfqpoint{5.490039in}{5.490039in}}%
\pgfusepath{clip}%
\pgfsetbuttcap%
\pgfsetroundjoin%
\definecolor{currentfill}{rgb}{0.274128,0.199721,0.498911}%
\pgfsetfillcolor{currentfill}%
\pgfsetfillopacity{0.700000}%
\pgfsetlinewidth{0.000000pt}%
\definecolor{currentstroke}{rgb}{0.000000,0.000000,0.000000}%
\pgfsetstrokecolor{currentstroke}%
\pgfsetdash{}{0pt}%
\pgfpathmoveto{\pgfqpoint{3.000372in}{2.487850in}}%
\pgfpathlineto{\pgfqpoint{3.013403in}{2.474775in}}%
\pgfpathlineto{\pgfqpoint{3.026430in}{2.461947in}}%
\pgfpathlineto{\pgfqpoint{3.039453in}{2.449362in}}%
\pgfpathlineto{\pgfqpoint{3.052472in}{2.437020in}}%
\pgfpathlineto{\pgfqpoint{3.060373in}{2.445551in}}%
\pgfpathlineto{\pgfqpoint{3.068266in}{2.454171in}}%
\pgfpathlineto{\pgfqpoint{3.076151in}{2.462880in}}%
\pgfpathlineto{\pgfqpoint{3.084030in}{2.471677in}}%
\pgfpathlineto{\pgfqpoint{3.071028in}{2.483935in}}%
\pgfpathlineto{\pgfqpoint{3.058022in}{2.496435in}}%
\pgfpathlineto{\pgfqpoint{3.045013in}{2.509179in}}%
\pgfpathlineto{\pgfqpoint{3.032000in}{2.522169in}}%
\pgfpathlineto{\pgfqpoint{3.024104in}{2.513446in}}%
\pgfpathlineto{\pgfqpoint{3.016201in}{2.504818in}}%
\pgfpathlineto{\pgfqpoint{3.008290in}{2.496286in}}%
\pgfpathlineto{\pgfqpoint{3.000372in}{2.487850in}}%
\pgfpathclose%
\pgfusepath{fill}%
\end{pgfscope}%
\begin{pgfscope}%
\pgfpathrectangle{\pgfqpoint{1.254980in}{0.150000in}}{\pgfqpoint{5.490039in}{5.490039in}}%
\pgfusepath{clip}%
\pgfsetbuttcap%
\pgfsetroundjoin%
\definecolor{currentfill}{rgb}{0.283072,0.130895,0.449241}%
\pgfsetfillcolor{currentfill}%
\pgfsetfillopacity{0.700000}%
\pgfsetlinewidth{0.000000pt}%
\definecolor{currentstroke}{rgb}{0.000000,0.000000,0.000000}%
\pgfsetstrokecolor{currentstroke}%
\pgfsetdash{}{0pt}%
\pgfpathmoveto{\pgfqpoint{3.239865in}{2.342811in}}%
\pgfpathlineto{\pgfqpoint{3.252842in}{2.333542in}}%
\pgfpathlineto{\pgfqpoint{3.265819in}{2.324491in}}%
\pgfpathlineto{\pgfqpoint{3.278794in}{2.315658in}}%
\pgfpathlineto{\pgfqpoint{3.291770in}{2.307040in}}%
\pgfpathlineto{\pgfqpoint{3.299580in}{2.316185in}}%
\pgfpathlineto{\pgfqpoint{3.307384in}{2.325391in}}%
\pgfpathlineto{\pgfqpoint{3.315181in}{2.334656in}}%
\pgfpathlineto{\pgfqpoint{3.322972in}{2.343981in}}%
\pgfpathlineto{\pgfqpoint{3.310010in}{2.352544in}}%
\pgfpathlineto{\pgfqpoint{3.297049in}{2.361324in}}%
\pgfpathlineto{\pgfqpoint{3.284086in}{2.370321in}}%
\pgfpathlineto{\pgfqpoint{3.271124in}{2.379536in}}%
\pgfpathlineto{\pgfqpoint{3.263319in}{2.370255in}}%
\pgfpathlineto{\pgfqpoint{3.255508in}{2.361040in}}%
\pgfpathlineto{\pgfqpoint{3.247690in}{2.351892in}}%
\pgfpathlineto{\pgfqpoint{3.239865in}{2.342811in}}%
\pgfpathclose%
\pgfusepath{fill}%
\end{pgfscope}%
\begin{pgfscope}%
\pgfpathrectangle{\pgfqpoint{1.254980in}{0.150000in}}{\pgfqpoint{5.490039in}{5.490039in}}%
\pgfusepath{clip}%
\pgfsetbuttcap%
\pgfsetroundjoin%
\definecolor{currentfill}{rgb}{0.221989,0.339161,0.548752}%
\pgfsetfillcolor{currentfill}%
\pgfsetfillopacity{0.700000}%
\pgfsetlinewidth{0.000000pt}%
\definecolor{currentstroke}{rgb}{0.000000,0.000000,0.000000}%
\pgfsetstrokecolor{currentstroke}%
\pgfsetdash{}{0pt}%
\pgfpathmoveto{\pgfqpoint{2.738530in}{2.804335in}}%
\pgfpathlineto{\pgfqpoint{2.751694in}{2.785872in}}%
\pgfpathlineto{\pgfqpoint{2.764848in}{2.767703in}}%
\pgfpathlineto{\pgfqpoint{2.777994in}{2.749824in}}%
\pgfpathlineto{\pgfqpoint{2.791131in}{2.732233in}}%
\pgfpathlineto{\pgfqpoint{2.799130in}{2.740282in}}%
\pgfpathlineto{\pgfqpoint{2.807120in}{2.748454in}}%
\pgfpathlineto{\pgfqpoint{2.815102in}{2.756751in}}%
\pgfpathlineto{\pgfqpoint{2.823075in}{2.765171in}}%
\pgfpathlineto{\pgfqpoint{2.809959in}{2.782674in}}%
\pgfpathlineto{\pgfqpoint{2.796835in}{2.800464in}}%
\pgfpathlineto{\pgfqpoint{2.783703in}{2.818545in}}%
\pgfpathlineto{\pgfqpoint{2.770561in}{2.836918in}}%
\pgfpathlineto{\pgfqpoint{2.762567in}{2.828576in}}%
\pgfpathlineto{\pgfqpoint{2.754564in}{2.820365in}}%
\pgfpathlineto{\pgfqpoint{2.746552in}{2.812284in}}%
\pgfpathlineto{\pgfqpoint{2.738530in}{2.804335in}}%
\pgfpathclose%
\pgfusepath{fill}%
\end{pgfscope}%
\begin{pgfscope}%
\pgfpathrectangle{\pgfqpoint{1.254980in}{0.150000in}}{\pgfqpoint{5.490039in}{5.490039in}}%
\pgfusepath{clip}%
\pgfsetbuttcap%
\pgfsetroundjoin%
\definecolor{currentfill}{rgb}{0.149039,0.508051,0.557250}%
\pgfsetfillcolor{currentfill}%
\pgfsetfillopacity{0.700000}%
\pgfsetlinewidth{0.000000pt}%
\definecolor{currentstroke}{rgb}{0.000000,0.000000,0.000000}%
\pgfsetstrokecolor{currentstroke}%
\pgfsetdash{}{0pt}%
\pgfpathmoveto{\pgfqpoint{5.510142in}{3.184070in}}%
\pgfpathlineto{\pgfqpoint{5.523735in}{3.187797in}}%
\pgfpathlineto{\pgfqpoint{5.537343in}{3.191674in}}%
\pgfpathlineto{\pgfqpoint{5.550965in}{3.195699in}}%
\pgfpathlineto{\pgfqpoint{5.564602in}{3.199873in}}%
\pgfpathlineto{\pgfqpoint{5.571629in}{3.207856in}}%
\pgfpathlineto{\pgfqpoint{5.578657in}{3.216049in}}%
\pgfpathlineto{\pgfqpoint{5.585687in}{3.224461in}}%
\pgfpathlineto{\pgfqpoint{5.592719in}{3.233098in}}%
\pgfpathlineto{\pgfqpoint{5.579110in}{3.229574in}}%
\pgfpathlineto{\pgfqpoint{5.565516in}{3.226199in}}%
\pgfpathlineto{\pgfqpoint{5.551935in}{3.222971in}}%
\pgfpathlineto{\pgfqpoint{5.538367in}{3.219892in}}%
\pgfpathlineto{\pgfqpoint{5.531308in}{3.210596in}}%
\pgfpathlineto{\pgfqpoint{5.524251in}{3.201532in}}%
\pgfpathlineto{\pgfqpoint{5.517196in}{3.192692in}}%
\pgfpathlineto{\pgfqpoint{5.510142in}{3.184070in}}%
\pgfpathclose%
\pgfusepath{fill}%
\end{pgfscope}%
\begin{pgfscope}%
\pgfpathrectangle{\pgfqpoint{1.254980in}{0.150000in}}{\pgfqpoint{5.490039in}{5.490039in}}%
\pgfusepath{clip}%
\pgfsetbuttcap%
\pgfsetroundjoin%
\definecolor{currentfill}{rgb}{0.280255,0.165693,0.476498}%
\pgfsetfillcolor{currentfill}%
\pgfsetfillopacity{0.700000}%
\pgfsetlinewidth{0.000000pt}%
\definecolor{currentstroke}{rgb}{0.000000,0.000000,0.000000}%
\pgfsetstrokecolor{currentstroke}%
\pgfsetdash{}{0pt}%
\pgfpathmoveto{\pgfqpoint{4.026306in}{2.397300in}}%
\pgfpathlineto{\pgfqpoint{4.039364in}{2.396376in}}%
\pgfpathlineto{\pgfqpoint{4.052429in}{2.395626in}}%
\pgfpathlineto{\pgfqpoint{4.065501in}{2.395052in}}%
\pgfpathlineto{\pgfqpoint{4.078581in}{2.394652in}}%
\pgfpathlineto{\pgfqpoint{4.086130in}{2.404226in}}%
\pgfpathlineto{\pgfqpoint{4.093675in}{2.413807in}}%
\pgfpathlineto{\pgfqpoint{4.101214in}{2.423399in}}%
\pgfpathlineto{\pgfqpoint{4.108749in}{2.433002in}}%
\pgfpathlineto{\pgfqpoint{4.095677in}{2.433545in}}%
\pgfpathlineto{\pgfqpoint{4.082613in}{2.434262in}}%
\pgfpathlineto{\pgfqpoint{4.069556in}{2.435153in}}%
\pgfpathlineto{\pgfqpoint{4.056507in}{2.436220in}}%
\pgfpathlineto{\pgfqpoint{4.048964in}{2.426464in}}%
\pgfpathlineto{\pgfqpoint{4.041416in}{2.416727in}}%
\pgfpathlineto{\pgfqpoint{4.033864in}{2.407006in}}%
\pgfpathlineto{\pgfqpoint{4.026306in}{2.397300in}}%
\pgfpathclose%
\pgfusepath{fill}%
\end{pgfscope}%
\begin{pgfscope}%
\pgfpathrectangle{\pgfqpoint{1.254980in}{0.150000in}}{\pgfqpoint{5.490039in}{5.490039in}}%
\pgfusepath{clip}%
\pgfsetbuttcap%
\pgfsetroundjoin%
\definecolor{currentfill}{rgb}{0.283091,0.110553,0.431554}%
\pgfsetfillcolor{currentfill}%
\pgfsetfillopacity{0.700000}%
\pgfsetlinewidth{0.000000pt}%
\definecolor{currentstroke}{rgb}{0.000000,0.000000,0.000000}%
\pgfsetstrokecolor{currentstroke}%
\pgfsetdash{}{0pt}%
\pgfpathmoveto{\pgfqpoint{3.644137in}{2.293307in}}%
\pgfpathlineto{\pgfqpoint{3.657119in}{2.289014in}}%
\pgfpathlineto{\pgfqpoint{3.670104in}{2.284913in}}%
\pgfpathlineto{\pgfqpoint{3.683095in}{2.281001in}}%
\pgfpathlineto{\pgfqpoint{3.696089in}{2.277278in}}%
\pgfpathlineto{\pgfqpoint{3.703762in}{2.287031in}}%
\pgfpathlineto{\pgfqpoint{3.711430in}{2.296806in}}%
\pgfpathlineto{\pgfqpoint{3.719092in}{2.306604in}}%
\pgfpathlineto{\pgfqpoint{3.726750in}{2.316425in}}%
\pgfpathlineto{\pgfqpoint{3.713765in}{2.320179in}}%
\pgfpathlineto{\pgfqpoint{3.700784in}{2.324122in}}%
\pgfpathlineto{\pgfqpoint{3.687808in}{2.328254in}}%
\pgfpathlineto{\pgfqpoint{3.674836in}{2.332577in}}%
\pgfpathlineto{\pgfqpoint{3.667169in}{2.322715in}}%
\pgfpathlineto{\pgfqpoint{3.659497in}{2.312883in}}%
\pgfpathlineto{\pgfqpoint{3.651819in}{2.303081in}}%
\pgfpathlineto{\pgfqpoint{3.644137in}{2.293307in}}%
\pgfpathclose%
\pgfusepath{fill}%
\end{pgfscope}%
\begin{pgfscope}%
\pgfpathrectangle{\pgfqpoint{1.254980in}{0.150000in}}{\pgfqpoint{5.490039in}{5.490039in}}%
\pgfusepath{clip}%
\pgfsetbuttcap%
\pgfsetroundjoin%
\definecolor{currentfill}{rgb}{0.278826,0.175490,0.483397}%
\pgfsetfillcolor{currentfill}%
\pgfsetfillopacity{0.700000}%
\pgfsetlinewidth{0.000000pt}%
\definecolor{currentstroke}{rgb}{0.000000,0.000000,0.000000}%
\pgfsetstrokecolor{currentstroke}%
\pgfsetdash{}{0pt}%
\pgfpathmoveto{\pgfqpoint{3.052472in}{2.437020in}}%
\pgfpathlineto{\pgfqpoint{3.065489in}{2.424919in}}%
\pgfpathlineto{\pgfqpoint{3.078502in}{2.413056in}}%
\pgfpathlineto{\pgfqpoint{3.091512in}{2.401429in}}%
\pgfpathlineto{\pgfqpoint{3.104519in}{2.390038in}}%
\pgfpathlineto{\pgfqpoint{3.112402in}{2.398662in}}%
\pgfpathlineto{\pgfqpoint{3.120278in}{2.407369in}}%
\pgfpathlineto{\pgfqpoint{3.128146in}{2.416158in}}%
\pgfpathlineto{\pgfqpoint{3.136008in}{2.425027in}}%
\pgfpathlineto{\pgfqpoint{3.123018in}{2.436336in}}%
\pgfpathlineto{\pgfqpoint{3.110025in}{2.447879in}}%
\pgfpathlineto{\pgfqpoint{3.097029in}{2.459659in}}%
\pgfpathlineto{\pgfqpoint{3.084030in}{2.471677in}}%
\pgfpathlineto{\pgfqpoint{3.076151in}{2.462880in}}%
\pgfpathlineto{\pgfqpoint{3.068266in}{2.454171in}}%
\pgfpathlineto{\pgfqpoint{3.060373in}{2.445551in}}%
\pgfpathlineto{\pgfqpoint{3.052472in}{2.437020in}}%
\pgfpathclose%
\pgfusepath{fill}%
\end{pgfscope}%
\begin{pgfscope}%
\pgfpathrectangle{\pgfqpoint{1.254980in}{0.150000in}}{\pgfqpoint{5.490039in}{5.490039in}}%
\pgfusepath{clip}%
\pgfsetbuttcap%
\pgfsetroundjoin%
\definecolor{currentfill}{rgb}{0.281887,0.150881,0.465405}%
\pgfsetfillcolor{currentfill}%
\pgfsetfillopacity{0.700000}%
\pgfsetlinewidth{0.000000pt}%
\definecolor{currentstroke}{rgb}{0.000000,0.000000,0.000000}%
\pgfsetstrokecolor{currentstroke}%
\pgfsetdash{}{0pt}%
\pgfpathmoveto{\pgfqpoint{3.943831in}{2.363595in}}%
\pgfpathlineto{\pgfqpoint{3.956869in}{2.362077in}}%
\pgfpathlineto{\pgfqpoint{3.969915in}{2.360737in}}%
\pgfpathlineto{\pgfqpoint{3.982967in}{2.359574in}}%
\pgfpathlineto{\pgfqpoint{3.996026in}{2.358588in}}%
\pgfpathlineto{\pgfqpoint{4.003604in}{2.368252in}}%
\pgfpathlineto{\pgfqpoint{4.011176in}{2.377925in}}%
\pgfpathlineto{\pgfqpoint{4.018743in}{2.387607in}}%
\pgfpathlineto{\pgfqpoint{4.026306in}{2.397300in}}%
\pgfpathlineto{\pgfqpoint{4.013255in}{2.398401in}}%
\pgfpathlineto{\pgfqpoint{4.000211in}{2.399678in}}%
\pgfpathlineto{\pgfqpoint{3.987174in}{2.401133in}}%
\pgfpathlineto{\pgfqpoint{3.974144in}{2.402765in}}%
\pgfpathlineto{\pgfqpoint{3.966573in}{2.392948in}}%
\pgfpathlineto{\pgfqpoint{3.958997in}{2.383148in}}%
\pgfpathlineto{\pgfqpoint{3.951416in}{2.373364in}}%
\pgfpathlineto{\pgfqpoint{3.943831in}{2.363595in}}%
\pgfpathclose%
\pgfusepath{fill}%
\end{pgfscope}%
\begin{pgfscope}%
\pgfpathrectangle{\pgfqpoint{1.254980in}{0.150000in}}{\pgfqpoint{5.490039in}{5.490039in}}%
\pgfusepath{clip}%
\pgfsetbuttcap%
\pgfsetroundjoin%
\definecolor{currentfill}{rgb}{0.141935,0.526453,0.555991}%
\pgfsetfillcolor{currentfill}%
\pgfsetfillopacity{0.700000}%
\pgfsetlinewidth{0.000000pt}%
\definecolor{currentstroke}{rgb}{0.000000,0.000000,0.000000}%
\pgfsetstrokecolor{currentstroke}%
\pgfsetdash{}{0pt}%
\pgfpathmoveto{\pgfqpoint{5.592719in}{3.233098in}}%
\pgfpathlineto{\pgfqpoint{5.606342in}{3.236770in}}%
\pgfpathlineto{\pgfqpoint{5.619979in}{3.240590in}}%
\pgfpathlineto{\pgfqpoint{5.633631in}{3.244559in}}%
\pgfpathlineto{\pgfqpoint{5.647297in}{3.248675in}}%
\pgfpathlineto{\pgfqpoint{5.654303in}{3.256877in}}%
\pgfpathlineto{\pgfqpoint{5.661311in}{3.265313in}}%
\pgfpathlineto{\pgfqpoint{5.668322in}{3.273990in}}%
\pgfpathlineto{\pgfqpoint{5.654678in}{3.270380in}}%
\pgfpathlineto{\pgfqpoint{5.641048in}{3.266918in}}%
\pgfpathlineto{\pgfqpoint{5.627432in}{3.263603in}}%
\pgfpathlineto{\pgfqpoint{5.613830in}{3.260436in}}%
\pgfpathlineto{\pgfqpoint{5.606790in}{3.251078in}}%
\pgfpathlineto{\pgfqpoint{5.599754in}{3.241968in}}%
\pgfpathlineto{\pgfqpoint{5.592719in}{3.233098in}}%
\pgfpathclose%
\pgfusepath{fill}%
\end{pgfscope}%
\begin{pgfscope}%
\pgfpathrectangle{\pgfqpoint{1.254980in}{0.150000in}}{\pgfqpoint{5.490039in}{5.490039in}}%
\pgfusepath{clip}%
\pgfsetbuttcap%
\pgfsetroundjoin%
\definecolor{currentfill}{rgb}{0.206756,0.371758,0.553117}%
\pgfsetfillcolor{currentfill}%
\pgfsetfillopacity{0.700000}%
\pgfsetlinewidth{0.000000pt}%
\definecolor{currentstroke}{rgb}{0.000000,0.000000,0.000000}%
\pgfsetstrokecolor{currentstroke}%
\pgfsetdash{}{0pt}%
\pgfpathmoveto{\pgfqpoint{2.685778in}{2.881176in}}%
\pgfpathlineto{\pgfqpoint{2.698981in}{2.861511in}}%
\pgfpathlineto{\pgfqpoint{2.712174in}{2.842152in}}%
\pgfpathlineto{\pgfqpoint{2.725357in}{2.823094in}}%
\pgfpathlineto{\pgfqpoint{2.738530in}{2.804335in}}%
\pgfpathlineto{\pgfqpoint{2.746552in}{2.812284in}}%
\pgfpathlineto{\pgfqpoint{2.754564in}{2.820365in}}%
\pgfpathlineto{\pgfqpoint{2.762567in}{2.828576in}}%
\pgfpathlineto{\pgfqpoint{2.770561in}{2.836918in}}%
\pgfpathlineto{\pgfqpoint{2.757411in}{2.855587in}}%
\pgfpathlineto{\pgfqpoint{2.744251in}{2.874555in}}%
\pgfpathlineto{\pgfqpoint{2.731081in}{2.893825in}}%
\pgfpathlineto{\pgfqpoint{2.717900in}{2.913399in}}%
\pgfpathlineto{\pgfqpoint{2.709884in}{2.905136in}}%
\pgfpathlineto{\pgfqpoint{2.701858in}{2.897011in}}%
\pgfpathlineto{\pgfqpoint{2.693823in}{2.889024in}}%
\pgfpathlineto{\pgfqpoint{2.685778in}{2.881176in}}%
\pgfpathclose%
\pgfusepath{fill}%
\end{pgfscope}%
\begin{pgfscope}%
\pgfpathrectangle{\pgfqpoint{1.254980in}{0.150000in}}{\pgfqpoint{5.490039in}{5.490039in}}%
\pgfusepath{clip}%
\pgfsetbuttcap%
\pgfsetroundjoin%
\definecolor{currentfill}{rgb}{0.282910,0.105393,0.426902}%
\pgfsetfillcolor{currentfill}%
\pgfsetfillopacity{0.700000}%
\pgfsetlinewidth{0.000000pt}%
\definecolor{currentstroke}{rgb}{0.000000,0.000000,0.000000}%
\pgfsetstrokecolor{currentstroke}%
\pgfsetdash{}{0pt}%
\pgfpathmoveto{\pgfqpoint{3.426689in}{2.283081in}}%
\pgfpathlineto{\pgfqpoint{3.439659in}{2.276401in}}%
\pgfpathlineto{\pgfqpoint{3.452631in}{2.269924in}}%
\pgfpathlineto{\pgfqpoint{3.465605in}{2.263649in}}%
\pgfpathlineto{\pgfqpoint{3.478581in}{2.257574in}}%
\pgfpathlineto{\pgfqpoint{3.486328in}{2.267066in}}%
\pgfpathlineto{\pgfqpoint{3.494070in}{2.276598in}}%
\pgfpathlineto{\pgfqpoint{3.501807in}{2.286168in}}%
\pgfpathlineto{\pgfqpoint{3.509537in}{2.295779in}}%
\pgfpathlineto{\pgfqpoint{3.496573in}{2.301828in}}%
\pgfpathlineto{\pgfqpoint{3.483611in}{2.308078in}}%
\pgfpathlineto{\pgfqpoint{3.470651in}{2.314530in}}%
\pgfpathlineto{\pgfqpoint{3.457693in}{2.321185in}}%
\pgfpathlineto{\pgfqpoint{3.449951in}{2.311590in}}%
\pgfpathlineto{\pgfqpoint{3.442203in}{2.302041in}}%
\pgfpathlineto{\pgfqpoint{3.434449in}{2.292538in}}%
\pgfpathlineto{\pgfqpoint{3.426689in}{2.283081in}}%
\pgfpathclose%
\pgfusepath{fill}%
\end{pgfscope}%
\begin{pgfscope}%
\pgfpathrectangle{\pgfqpoint{1.254980in}{0.150000in}}{\pgfqpoint{5.490039in}{5.490039in}}%
\pgfusepath{clip}%
\pgfsetbuttcap%
\pgfsetroundjoin%
\definecolor{currentfill}{rgb}{0.282884,0.135920,0.453427}%
\pgfsetfillcolor{currentfill}%
\pgfsetfillopacity{0.700000}%
\pgfsetlinewidth{0.000000pt}%
\definecolor{currentstroke}{rgb}{0.000000,0.000000,0.000000}%
\pgfsetstrokecolor{currentstroke}%
\pgfsetdash{}{0pt}%
\pgfpathmoveto{\pgfqpoint{3.861312in}{2.332158in}}%
\pgfpathlineto{\pgfqpoint{3.874334in}{2.330007in}}%
\pgfpathlineto{\pgfqpoint{3.887363in}{2.328038in}}%
\pgfpathlineto{\pgfqpoint{3.900397in}{2.326248in}}%
\pgfpathlineto{\pgfqpoint{3.913438in}{2.324638in}}%
\pgfpathlineto{\pgfqpoint{3.921044in}{2.334362in}}%
\pgfpathlineto{\pgfqpoint{3.928644in}{2.344096in}}%
\pgfpathlineto{\pgfqpoint{3.936240in}{2.353840in}}%
\pgfpathlineto{\pgfqpoint{3.943831in}{2.363595in}}%
\pgfpathlineto{\pgfqpoint{3.930798in}{2.365292in}}%
\pgfpathlineto{\pgfqpoint{3.917772in}{2.367169in}}%
\pgfpathlineto{\pgfqpoint{3.904752in}{2.369225in}}%
\pgfpathlineto{\pgfqpoint{3.891738in}{2.371462in}}%
\pgfpathlineto{\pgfqpoint{3.884139in}{2.361610in}}%
\pgfpathlineto{\pgfqpoint{3.876535in}{2.351776in}}%
\pgfpathlineto{\pgfqpoint{3.868926in}{2.341959in}}%
\pgfpathlineto{\pgfqpoint{3.861312in}{2.332158in}}%
\pgfpathclose%
\pgfusepath{fill}%
\end{pgfscope}%
\begin{pgfscope}%
\pgfpathrectangle{\pgfqpoint{1.254980in}{0.150000in}}{\pgfqpoint{5.490039in}{5.490039in}}%
\pgfusepath{clip}%
\pgfsetbuttcap%
\pgfsetroundjoin%
\definecolor{currentfill}{rgb}{0.283197,0.115680,0.436115}%
\pgfsetfillcolor{currentfill}%
\pgfsetfillopacity{0.700000}%
\pgfsetlinewidth{0.000000pt}%
\definecolor{currentstroke}{rgb}{0.000000,0.000000,0.000000}%
\pgfsetstrokecolor{currentstroke}%
\pgfsetdash{}{0pt}%
\pgfpathmoveto{\pgfqpoint{3.291770in}{2.307040in}}%
\pgfpathlineto{\pgfqpoint{3.304746in}{2.298637in}}%
\pgfpathlineto{\pgfqpoint{3.317722in}{2.290448in}}%
\pgfpathlineto{\pgfqpoint{3.330698in}{2.282470in}}%
\pgfpathlineto{\pgfqpoint{3.343675in}{2.274703in}}%
\pgfpathlineto{\pgfqpoint{3.351471in}{2.283912in}}%
\pgfpathlineto{\pgfqpoint{3.359261in}{2.293174in}}%
\pgfpathlineto{\pgfqpoint{3.367045in}{2.302489in}}%
\pgfpathlineto{\pgfqpoint{3.374823in}{2.311857in}}%
\pgfpathlineto{\pgfqpoint{3.361860in}{2.319571in}}%
\pgfpathlineto{\pgfqpoint{3.348897in}{2.327495in}}%
\pgfpathlineto{\pgfqpoint{3.335934in}{2.335631in}}%
\pgfpathlineto{\pgfqpoint{3.322972in}{2.343981in}}%
\pgfpathlineto{\pgfqpoint{3.315181in}{2.334656in}}%
\pgfpathlineto{\pgfqpoint{3.307384in}{2.325391in}}%
\pgfpathlineto{\pgfqpoint{3.299580in}{2.316185in}}%
\pgfpathlineto{\pgfqpoint{3.291770in}{2.307040in}}%
\pgfpathclose%
\pgfusepath{fill}%
\end{pgfscope}%
\begin{pgfscope}%
\pgfpathrectangle{\pgfqpoint{1.254980in}{0.150000in}}{\pgfqpoint{5.490039in}{5.490039in}}%
\pgfusepath{clip}%
\pgfsetbuttcap%
\pgfsetroundjoin%
\definecolor{currentfill}{rgb}{0.281412,0.155834,0.469201}%
\pgfsetfillcolor{currentfill}%
\pgfsetfillopacity{0.700000}%
\pgfsetlinewidth{0.000000pt}%
\definecolor{currentstroke}{rgb}{0.000000,0.000000,0.000000}%
\pgfsetstrokecolor{currentstroke}%
\pgfsetdash{}{0pt}%
\pgfpathmoveto{\pgfqpoint{3.104519in}{2.390038in}}%
\pgfpathlineto{\pgfqpoint{3.117524in}{2.378879in}}%
\pgfpathlineto{\pgfqpoint{3.130526in}{2.367953in}}%
\pgfpathlineto{\pgfqpoint{3.143526in}{2.357256in}}%
\pgfpathlineto{\pgfqpoint{3.156525in}{2.346787in}}%
\pgfpathlineto{\pgfqpoint{3.164391in}{2.355505in}}%
\pgfpathlineto{\pgfqpoint{3.172250in}{2.364298in}}%
\pgfpathlineto{\pgfqpoint{3.180103in}{2.373166in}}%
\pgfpathlineto{\pgfqpoint{3.187949in}{2.382108in}}%
\pgfpathlineto{\pgfqpoint{3.174967in}{2.392494in}}%
\pgfpathlineto{\pgfqpoint{3.161982in}{2.403108in}}%
\pgfpathlineto{\pgfqpoint{3.148996in}{2.413952in}}%
\pgfpathlineto{\pgfqpoint{3.136008in}{2.425027in}}%
\pgfpathlineto{\pgfqpoint{3.128146in}{2.416158in}}%
\pgfpathlineto{\pgfqpoint{3.120278in}{2.407369in}}%
\pgfpathlineto{\pgfqpoint{3.112402in}{2.398662in}}%
\pgfpathlineto{\pgfqpoint{3.104519in}{2.390038in}}%
\pgfpathclose%
\pgfusepath{fill}%
\end{pgfscope}%
\begin{pgfscope}%
\pgfpathrectangle{\pgfqpoint{1.254980in}{0.150000in}}{\pgfqpoint{5.490039in}{5.490039in}}%
\pgfusepath{clip}%
\pgfsetbuttcap%
\pgfsetroundjoin%
\definecolor{currentfill}{rgb}{0.282910,0.105393,0.426902}%
\pgfsetfillcolor{currentfill}%
\pgfsetfillopacity{0.700000}%
\pgfsetlinewidth{0.000000pt}%
\definecolor{currentstroke}{rgb}{0.000000,0.000000,0.000000}%
\pgfsetstrokecolor{currentstroke}%
\pgfsetdash{}{0pt}%
\pgfpathmoveto{\pgfqpoint{3.561421in}{2.273568in}}%
\pgfpathlineto{\pgfqpoint{3.574399in}{2.268506in}}%
\pgfpathlineto{\pgfqpoint{3.587380in}{2.263640in}}%
\pgfpathlineto{\pgfqpoint{3.600365in}{2.258966in}}%
\pgfpathlineto{\pgfqpoint{3.613353in}{2.254485in}}%
\pgfpathlineto{\pgfqpoint{3.621057in}{2.264151in}}%
\pgfpathlineto{\pgfqpoint{3.628755in}{2.273842in}}%
\pgfpathlineto{\pgfqpoint{3.636449in}{2.283561in}}%
\pgfpathlineto{\pgfqpoint{3.644137in}{2.293307in}}%
\pgfpathlineto{\pgfqpoint{3.631158in}{2.297790in}}%
\pgfpathlineto{\pgfqpoint{3.618184in}{2.302467in}}%
\pgfpathlineto{\pgfqpoint{3.605213in}{2.307337in}}%
\pgfpathlineto{\pgfqpoint{3.592246in}{2.312401in}}%
\pgfpathlineto{\pgfqpoint{3.584547in}{2.302642in}}%
\pgfpathlineto{\pgfqpoint{3.576844in}{2.292917in}}%
\pgfpathlineto{\pgfqpoint{3.569135in}{2.283226in}}%
\pgfpathlineto{\pgfqpoint{3.561421in}{2.273568in}}%
\pgfpathclose%
\pgfusepath{fill}%
\end{pgfscope}%
\begin{pgfscope}%
\pgfpathrectangle{\pgfqpoint{1.254980in}{0.150000in}}{\pgfqpoint{5.490039in}{5.490039in}}%
\pgfusepath{clip}%
\pgfsetbuttcap%
\pgfsetroundjoin%
\definecolor{currentfill}{rgb}{0.283229,0.120777,0.440584}%
\pgfsetfillcolor{currentfill}%
\pgfsetfillopacity{0.700000}%
\pgfsetlinewidth{0.000000pt}%
\definecolor{currentstroke}{rgb}{0.000000,0.000000,0.000000}%
\pgfsetstrokecolor{currentstroke}%
\pgfsetdash{}{0pt}%
\pgfpathmoveto{\pgfqpoint{3.778736in}{2.303281in}}%
\pgfpathlineto{\pgfqpoint{3.791745in}{2.300458in}}%
\pgfpathlineto{\pgfqpoint{3.804760in}{2.297819in}}%
\pgfpathlineto{\pgfqpoint{3.817780in}{2.295364in}}%
\pgfpathlineto{\pgfqpoint{3.830805in}{2.293091in}}%
\pgfpathlineto{\pgfqpoint{3.838439in}{2.302839in}}%
\pgfpathlineto{\pgfqpoint{3.846068in}{2.312599in}}%
\pgfpathlineto{\pgfqpoint{3.853692in}{2.322372in}}%
\pgfpathlineto{\pgfqpoint{3.861312in}{2.332158in}}%
\pgfpathlineto{\pgfqpoint{3.848295in}{2.334490in}}%
\pgfpathlineto{\pgfqpoint{3.835284in}{2.337005in}}%
\pgfpathlineto{\pgfqpoint{3.822278in}{2.339702in}}%
\pgfpathlineto{\pgfqpoint{3.809278in}{2.342583in}}%
\pgfpathlineto{\pgfqpoint{3.801650in}{2.332728in}}%
\pgfpathlineto{\pgfqpoint{3.794017in}{2.322893in}}%
\pgfpathlineto{\pgfqpoint{3.786379in}{2.313078in}}%
\pgfpathlineto{\pgfqpoint{3.778736in}{2.303281in}}%
\pgfpathclose%
\pgfusepath{fill}%
\end{pgfscope}%
\begin{pgfscope}%
\pgfpathrectangle{\pgfqpoint{1.254980in}{0.150000in}}{\pgfqpoint{5.490039in}{5.490039in}}%
\pgfusepath{clip}%
\pgfsetbuttcap%
\pgfsetroundjoin%
\definecolor{currentfill}{rgb}{0.282884,0.135920,0.453427}%
\pgfsetfillcolor{currentfill}%
\pgfsetfillopacity{0.700000}%
\pgfsetlinewidth{0.000000pt}%
\definecolor{currentstroke}{rgb}{0.000000,0.000000,0.000000}%
\pgfsetstrokecolor{currentstroke}%
\pgfsetdash{}{0pt}%
\pgfpathmoveto{\pgfqpoint{3.156525in}{2.346787in}}%
\pgfpathlineto{\pgfqpoint{3.169521in}{2.336545in}}%
\pgfpathlineto{\pgfqpoint{3.182517in}{2.326528in}}%
\pgfpathlineto{\pgfqpoint{3.195511in}{2.316734in}}%
\pgfpathlineto{\pgfqpoint{3.208503in}{2.307162in}}%
\pgfpathlineto{\pgfqpoint{3.216354in}{2.315973in}}%
\pgfpathlineto{\pgfqpoint{3.224198in}{2.324851in}}%
\pgfpathlineto{\pgfqpoint{3.232035in}{2.333797in}}%
\pgfpathlineto{\pgfqpoint{3.239865in}{2.342811in}}%
\pgfpathlineto{\pgfqpoint{3.226888in}{2.352301in}}%
\pgfpathlineto{\pgfqpoint{3.213910in}{2.362013in}}%
\pgfpathlineto{\pgfqpoint{3.200930in}{2.371948in}}%
\pgfpathlineto{\pgfqpoint{3.187949in}{2.382108in}}%
\pgfpathlineto{\pgfqpoint{3.180103in}{2.373166in}}%
\pgfpathlineto{\pgfqpoint{3.172250in}{2.364298in}}%
\pgfpathlineto{\pgfqpoint{3.164391in}{2.355505in}}%
\pgfpathlineto{\pgfqpoint{3.156525in}{2.346787in}}%
\pgfpathclose%
\pgfusepath{fill}%
\end{pgfscope}%
\begin{pgfscope}%
\pgfpathrectangle{\pgfqpoint{1.254980in}{0.150000in}}{\pgfqpoint{5.490039in}{5.490039in}}%
\pgfusepath{clip}%
\pgfsetbuttcap%
\pgfsetroundjoin%
\definecolor{currentfill}{rgb}{0.243113,0.292092,0.538516}%
\pgfsetfillcolor{currentfill}%
\pgfsetfillopacity{0.700000}%
\pgfsetlinewidth{0.000000pt}%
\definecolor{currentstroke}{rgb}{0.000000,0.000000,0.000000}%
\pgfsetstrokecolor{currentstroke}%
\pgfsetdash{}{0pt}%
\pgfpathmoveto{\pgfqpoint{4.573657in}{2.641889in}}%
\pgfpathlineto{\pgfqpoint{4.586912in}{2.644400in}}%
\pgfpathlineto{\pgfqpoint{4.600177in}{2.647073in}}%
\pgfpathlineto{\pgfqpoint{4.613453in}{2.649908in}}%
\pgfpathlineto{\pgfqpoint{4.626741in}{2.652904in}}%
\pgfpathlineto{\pgfqpoint{4.634106in}{2.661240in}}%
\pgfpathlineto{\pgfqpoint{4.641466in}{2.669601in}}%
\pgfpathlineto{\pgfqpoint{4.648822in}{2.677992in}}%
\pgfpathlineto{\pgfqpoint{4.656173in}{2.686415in}}%
\pgfpathlineto{\pgfqpoint{4.642898in}{2.683731in}}%
\pgfpathlineto{\pgfqpoint{4.629633in}{2.681208in}}%
\pgfpathlineto{\pgfqpoint{4.616379in}{2.678847in}}%
\pgfpathlineto{\pgfqpoint{4.603136in}{2.676647in}}%
\pgfpathlineto{\pgfqpoint{4.595773in}{2.667902in}}%
\pgfpathlineto{\pgfqpoint{4.588406in}{2.659196in}}%
\pgfpathlineto{\pgfqpoint{4.581034in}{2.650526in}}%
\pgfpathlineto{\pgfqpoint{4.573657in}{2.641889in}}%
\pgfpathclose%
\pgfusepath{fill}%
\end{pgfscope}%
\begin{pgfscope}%
\pgfpathrectangle{\pgfqpoint{1.254980in}{0.150000in}}{\pgfqpoint{5.490039in}{5.490039in}}%
\pgfusepath{clip}%
\pgfsetbuttcap%
\pgfsetroundjoin%
\definecolor{currentfill}{rgb}{0.250425,0.274290,0.533103}%
\pgfsetfillcolor{currentfill}%
\pgfsetfillopacity{0.700000}%
\pgfsetlinewidth{0.000000pt}%
\definecolor{currentstroke}{rgb}{0.000000,0.000000,0.000000}%
\pgfsetstrokecolor{currentstroke}%
\pgfsetdash{}{0pt}%
\pgfpathmoveto{\pgfqpoint{4.491147in}{2.598037in}}%
\pgfpathlineto{\pgfqpoint{4.504370in}{2.600180in}}%
\pgfpathlineto{\pgfqpoint{4.517604in}{2.602487in}}%
\pgfpathlineto{\pgfqpoint{4.530848in}{2.604957in}}%
\pgfpathlineto{\pgfqpoint{4.544103in}{2.607590in}}%
\pgfpathlineto{\pgfqpoint{4.551499in}{2.616133in}}%
\pgfpathlineto{\pgfqpoint{4.558889in}{2.624696in}}%
\pgfpathlineto{\pgfqpoint{4.566276in}{2.633280in}}%
\pgfpathlineto{\pgfqpoint{4.573657in}{2.641889in}}%
\pgfpathlineto{\pgfqpoint{4.560413in}{2.639540in}}%
\pgfpathlineto{\pgfqpoint{4.547180in}{2.637354in}}%
\pgfpathlineto{\pgfqpoint{4.533957in}{2.635330in}}%
\pgfpathlineto{\pgfqpoint{4.520744in}{2.633470in}}%
\pgfpathlineto{\pgfqpoint{4.513352in}{2.624567in}}%
\pgfpathlineto{\pgfqpoint{4.505955in}{2.615696in}}%
\pgfpathlineto{\pgfqpoint{4.498553in}{2.606854in}}%
\pgfpathlineto{\pgfqpoint{4.491147in}{2.598037in}}%
\pgfpathclose%
\pgfusepath{fill}%
\end{pgfscope}%
\begin{pgfscope}%
\pgfpathrectangle{\pgfqpoint{1.254980in}{0.150000in}}{\pgfqpoint{5.490039in}{5.490039in}}%
\pgfusepath{clip}%
\pgfsetbuttcap%
\pgfsetroundjoin%
\definecolor{currentfill}{rgb}{0.233603,0.313828,0.543914}%
\pgfsetfillcolor{currentfill}%
\pgfsetfillopacity{0.700000}%
\pgfsetlinewidth{0.000000pt}%
\definecolor{currentstroke}{rgb}{0.000000,0.000000,0.000000}%
\pgfsetstrokecolor{currentstroke}%
\pgfsetdash{}{0pt}%
\pgfpathmoveto{\pgfqpoint{4.656173in}{2.686415in}}%
\pgfpathlineto{\pgfqpoint{4.669460in}{2.689260in}}%
\pgfpathlineto{\pgfqpoint{4.682758in}{2.692265in}}%
\pgfpathlineto{\pgfqpoint{4.696068in}{2.695430in}}%
\pgfpathlineto{\pgfqpoint{4.709389in}{2.698756in}}%
\pgfpathlineto{\pgfqpoint{4.716723in}{2.706885in}}%
\pgfpathlineto{\pgfqpoint{4.724052in}{2.715049in}}%
\pgfpathlineto{\pgfqpoint{4.731377in}{2.723250in}}%
\pgfpathlineto{\pgfqpoint{4.738698in}{2.731492in}}%
\pgfpathlineto{\pgfqpoint{4.725390in}{2.728507in}}%
\pgfpathlineto{\pgfqpoint{4.712093in}{2.725682in}}%
\pgfpathlineto{\pgfqpoint{4.698807in}{2.723017in}}%
\pgfpathlineto{\pgfqpoint{4.685532in}{2.720512in}}%
\pgfpathlineto{\pgfqpoint{4.678199in}{2.711919in}}%
\pgfpathlineto{\pgfqpoint{4.670862in}{2.703375in}}%
\pgfpathlineto{\pgfqpoint{4.663520in}{2.694875in}}%
\pgfpathlineto{\pgfqpoint{4.656173in}{2.686415in}}%
\pgfpathclose%
\pgfusepath{fill}%
\end{pgfscope}%
\begin{pgfscope}%
\pgfpathrectangle{\pgfqpoint{1.254980in}{0.150000in}}{\pgfqpoint{5.490039in}{5.490039in}}%
\pgfusepath{clip}%
\pgfsetbuttcap%
\pgfsetroundjoin%
\definecolor{currentfill}{rgb}{0.258965,0.251537,0.524736}%
\pgfsetfillcolor{currentfill}%
\pgfsetfillopacity{0.700000}%
\pgfsetlinewidth{0.000000pt}%
\definecolor{currentstroke}{rgb}{0.000000,0.000000,0.000000}%
\pgfsetstrokecolor{currentstroke}%
\pgfsetdash{}{0pt}%
\pgfpathmoveto{\pgfqpoint{4.408641in}{2.555003in}}%
\pgfpathlineto{\pgfqpoint{4.421834in}{2.556744in}}%
\pgfpathlineto{\pgfqpoint{4.435037in}{2.558650in}}%
\pgfpathlineto{\pgfqpoint{4.448250in}{2.560720in}}%
\pgfpathlineto{\pgfqpoint{4.461473in}{2.562955in}}%
\pgfpathlineto{\pgfqpoint{4.468899in}{2.571704in}}%
\pgfpathlineto{\pgfqpoint{4.476320in}{2.580465in}}%
\pgfpathlineto{\pgfqpoint{4.483736in}{2.589241in}}%
\pgfpathlineto{\pgfqpoint{4.491147in}{2.598037in}}%
\pgfpathlineto{\pgfqpoint{4.477934in}{2.596057in}}%
\pgfpathlineto{\pgfqpoint{4.464731in}{2.594242in}}%
\pgfpathlineto{\pgfqpoint{4.451538in}{2.592592in}}%
\pgfpathlineto{\pgfqpoint{4.438355in}{2.591106in}}%
\pgfpathlineto{\pgfqpoint{4.430934in}{2.582045in}}%
\pgfpathlineto{\pgfqpoint{4.423508in}{2.573010in}}%
\pgfpathlineto{\pgfqpoint{4.416077in}{2.563997in}}%
\pgfpathlineto{\pgfqpoint{4.408641in}{2.555003in}}%
\pgfpathclose%
\pgfusepath{fill}%
\end{pgfscope}%
\begin{pgfscope}%
\pgfpathrectangle{\pgfqpoint{1.254980in}{0.150000in}}{\pgfqpoint{5.490039in}{5.490039in}}%
\pgfusepath{clip}%
\pgfsetbuttcap%
\pgfsetroundjoin%
\definecolor{currentfill}{rgb}{0.223925,0.334994,0.548053}%
\pgfsetfillcolor{currentfill}%
\pgfsetfillopacity{0.700000}%
\pgfsetlinewidth{0.000000pt}%
\definecolor{currentstroke}{rgb}{0.000000,0.000000,0.000000}%
\pgfsetstrokecolor{currentstroke}%
\pgfsetdash{}{0pt}%
\pgfpathmoveto{\pgfqpoint{4.738698in}{2.731492in}}%
\pgfpathlineto{\pgfqpoint{4.752018in}{2.734636in}}%
\pgfpathlineto{\pgfqpoint{4.765349in}{2.737939in}}%
\pgfpathlineto{\pgfqpoint{4.778692in}{2.741401in}}%
\pgfpathlineto{\pgfqpoint{4.792047in}{2.745022in}}%
\pgfpathlineto{\pgfqpoint{4.799350in}{2.752952in}}%
\pgfpathlineto{\pgfqpoint{4.806648in}{2.760924in}}%
\pgfpathlineto{\pgfqpoint{4.813942in}{2.768944in}}%
\pgfpathlineto{\pgfqpoint{4.821232in}{2.777016in}}%
\pgfpathlineto{\pgfqpoint{4.807890in}{2.773764in}}%
\pgfpathlineto{\pgfqpoint{4.794561in}{2.770671in}}%
\pgfpathlineto{\pgfqpoint{4.781243in}{2.767736in}}%
\pgfpathlineto{\pgfqpoint{4.767936in}{2.764960in}}%
\pgfpathlineto{\pgfqpoint{4.760633in}{2.756510in}}%
\pgfpathlineto{\pgfqpoint{4.753326in}{2.748118in}}%
\pgfpathlineto{\pgfqpoint{4.746014in}{2.739780in}}%
\pgfpathlineto{\pgfqpoint{4.738698in}{2.731492in}}%
\pgfpathclose%
\pgfusepath{fill}%
\end{pgfscope}%
\begin{pgfscope}%
\pgfpathrectangle{\pgfqpoint{1.254980in}{0.150000in}}{\pgfqpoint{5.490039in}{5.490039in}}%
\pgfusepath{clip}%
\pgfsetbuttcap%
\pgfsetroundjoin%
\definecolor{currentfill}{rgb}{0.265145,0.232956,0.516599}%
\pgfsetfillcolor{currentfill}%
\pgfsetfillopacity{0.700000}%
\pgfsetlinewidth{0.000000pt}%
\definecolor{currentstroke}{rgb}{0.000000,0.000000,0.000000}%
\pgfsetstrokecolor{currentstroke}%
\pgfsetdash{}{0pt}%
\pgfpathmoveto{\pgfqpoint{4.326137in}{2.512953in}}%
\pgfpathlineto{\pgfqpoint{4.339301in}{2.514256in}}%
\pgfpathlineto{\pgfqpoint{4.352474in}{2.515725in}}%
\pgfpathlineto{\pgfqpoint{4.365657in}{2.517362in}}%
\pgfpathlineto{\pgfqpoint{4.378850in}{2.519164in}}%
\pgfpathlineto{\pgfqpoint{4.386305in}{2.528109in}}%
\pgfpathlineto{\pgfqpoint{4.393756in}{2.537062in}}%
\pgfpathlineto{\pgfqpoint{4.401201in}{2.546026in}}%
\pgfpathlineto{\pgfqpoint{4.408641in}{2.555003in}}%
\pgfpathlineto{\pgfqpoint{4.395458in}{2.553428in}}%
\pgfpathlineto{\pgfqpoint{4.382285in}{2.552019in}}%
\pgfpathlineto{\pgfqpoint{4.369120in}{2.550777in}}%
\pgfpathlineto{\pgfqpoint{4.355966in}{2.549701in}}%
\pgfpathlineto{\pgfqpoint{4.348516in}{2.540486in}}%
\pgfpathlineto{\pgfqpoint{4.341061in}{2.531292in}}%
\pgfpathlineto{\pgfqpoint{4.333601in}{2.522115in}}%
\pgfpathlineto{\pgfqpoint{4.326137in}{2.512953in}}%
\pgfpathclose%
\pgfusepath{fill}%
\end{pgfscope}%
\begin{pgfscope}%
\pgfpathrectangle{\pgfqpoint{1.254980in}{0.150000in}}{\pgfqpoint{5.490039in}{5.490039in}}%
\pgfusepath{clip}%
\pgfsetbuttcap%
\pgfsetroundjoin%
\definecolor{currentfill}{rgb}{0.214298,0.355619,0.551184}%
\pgfsetfillcolor{currentfill}%
\pgfsetfillopacity{0.700000}%
\pgfsetlinewidth{0.000000pt}%
\definecolor{currentstroke}{rgb}{0.000000,0.000000,0.000000}%
\pgfsetstrokecolor{currentstroke}%
\pgfsetdash{}{0pt}%
\pgfpathmoveto{\pgfqpoint{4.821232in}{2.777016in}}%
\pgfpathlineto{\pgfqpoint{4.834585in}{2.780426in}}%
\pgfpathlineto{\pgfqpoint{4.847950in}{2.783993in}}%
\pgfpathlineto{\pgfqpoint{4.861328in}{2.787719in}}%
\pgfpathlineto{\pgfqpoint{4.874717in}{2.791601in}}%
\pgfpathlineto{\pgfqpoint{4.881988in}{2.799342in}}%
\pgfpathlineto{\pgfqpoint{4.889255in}{2.807136in}}%
\pgfpathlineto{\pgfqpoint{4.896517in}{2.814988in}}%
\pgfpathlineto{\pgfqpoint{4.903776in}{2.822904in}}%
\pgfpathlineto{\pgfqpoint{4.890401in}{2.819419in}}%
\pgfpathlineto{\pgfqpoint{4.877038in}{2.816090in}}%
\pgfpathlineto{\pgfqpoint{4.863687in}{2.812920in}}%
\pgfpathlineto{\pgfqpoint{4.850348in}{2.809906in}}%
\pgfpathlineto{\pgfqpoint{4.843075in}{2.801584in}}%
\pgfpathlineto{\pgfqpoint{4.835798in}{2.793331in}}%
\pgfpathlineto{\pgfqpoint{4.828517in}{2.785143in}}%
\pgfpathlineto{\pgfqpoint{4.821232in}{2.777016in}}%
\pgfpathclose%
\pgfusepath{fill}%
\end{pgfscope}%
\begin{pgfscope}%
\pgfpathrectangle{\pgfqpoint{1.254980in}{0.150000in}}{\pgfqpoint{5.490039in}{5.490039in}}%
\pgfusepath{clip}%
\pgfsetbuttcap%
\pgfsetroundjoin%
\definecolor{currentfill}{rgb}{0.206756,0.371758,0.553117}%
\pgfsetfillcolor{currentfill}%
\pgfsetfillopacity{0.700000}%
\pgfsetlinewidth{0.000000pt}%
\definecolor{currentstroke}{rgb}{0.000000,0.000000,0.000000}%
\pgfsetstrokecolor{currentstroke}%
\pgfsetdash{}{0pt}%
\pgfpathmoveto{\pgfqpoint{4.903776in}{2.822904in}}%
\pgfpathlineto{\pgfqpoint{4.917163in}{2.826546in}}%
\pgfpathlineto{\pgfqpoint{4.930562in}{2.830344in}}%
\pgfpathlineto{\pgfqpoint{4.943974in}{2.834299in}}%
\pgfpathlineto{\pgfqpoint{4.957399in}{2.838411in}}%
\pgfpathlineto{\pgfqpoint{4.964637in}{2.845978in}}%
\pgfpathlineto{\pgfqpoint{4.971872in}{2.853611in}}%
\pgfpathlineto{\pgfqpoint{4.979103in}{2.861314in}}%
\pgfpathlineto{\pgfqpoint{4.986330in}{2.869092in}}%
\pgfpathlineto{\pgfqpoint{4.972922in}{2.865407in}}%
\pgfpathlineto{\pgfqpoint{4.959526in}{2.861878in}}%
\pgfpathlineto{\pgfqpoint{4.946142in}{2.858504in}}%
\pgfpathlineto{\pgfqpoint{4.932770in}{2.855287in}}%
\pgfpathlineto{\pgfqpoint{4.925527in}{2.847073in}}%
\pgfpathlineto{\pgfqpoint{4.918280in}{2.838941in}}%
\pgfpathlineto{\pgfqpoint{4.911030in}{2.830886in}}%
\pgfpathlineto{\pgfqpoint{4.903776in}{2.822904in}}%
\pgfpathclose%
\pgfusepath{fill}%
\end{pgfscope}%
\begin{pgfscope}%
\pgfpathrectangle{\pgfqpoint{1.254980in}{0.150000in}}{\pgfqpoint{5.490039in}{5.490039in}}%
\pgfusepath{clip}%
\pgfsetbuttcap%
\pgfsetroundjoin%
\definecolor{currentfill}{rgb}{0.270595,0.214069,0.507052}%
\pgfsetfillcolor{currentfill}%
\pgfsetfillopacity{0.700000}%
\pgfsetlinewidth{0.000000pt}%
\definecolor{currentstroke}{rgb}{0.000000,0.000000,0.000000}%
\pgfsetstrokecolor{currentstroke}%
\pgfsetdash{}{0pt}%
\pgfpathmoveto{\pgfqpoint{4.243629in}{2.472071in}}%
\pgfpathlineto{\pgfqpoint{4.256766in}{2.472900in}}%
\pgfpathlineto{\pgfqpoint{4.269911in}{2.473898in}}%
\pgfpathlineto{\pgfqpoint{4.283066in}{2.475064in}}%
\pgfpathlineto{\pgfqpoint{4.296230in}{2.476399in}}%
\pgfpathlineto{\pgfqpoint{4.303714in}{2.485528in}}%
\pgfpathlineto{\pgfqpoint{4.311193in}{2.494662in}}%
\pgfpathlineto{\pgfqpoint{4.318667in}{2.503803in}}%
\pgfpathlineto{\pgfqpoint{4.326137in}{2.512953in}}%
\pgfpathlineto{\pgfqpoint{4.312982in}{2.511818in}}%
\pgfpathlineto{\pgfqpoint{4.299836in}{2.510850in}}%
\pgfpathlineto{\pgfqpoint{4.286699in}{2.510051in}}%
\pgfpathlineto{\pgfqpoint{4.273572in}{2.509420in}}%
\pgfpathlineto{\pgfqpoint{4.266093in}{2.500061in}}%
\pgfpathlineto{\pgfqpoint{4.258610in}{2.490718in}}%
\pgfpathlineto{\pgfqpoint{4.251122in}{2.481389in}}%
\pgfpathlineto{\pgfqpoint{4.243629in}{2.472071in}}%
\pgfpathclose%
\pgfusepath{fill}%
\end{pgfscope}%
\begin{pgfscope}%
\pgfpathrectangle{\pgfqpoint{1.254980in}{0.150000in}}{\pgfqpoint{5.490039in}{5.490039in}}%
\pgfusepath{clip}%
\pgfsetbuttcap%
\pgfsetroundjoin%
\definecolor{currentfill}{rgb}{0.197636,0.391528,0.554969}%
\pgfsetfillcolor{currentfill}%
\pgfsetfillopacity{0.700000}%
\pgfsetlinewidth{0.000000pt}%
\definecolor{currentstroke}{rgb}{0.000000,0.000000,0.000000}%
\pgfsetstrokecolor{currentstroke}%
\pgfsetdash{}{0pt}%
\pgfpathmoveto{\pgfqpoint{4.986330in}{2.869092in}}%
\pgfpathlineto{\pgfqpoint{4.999751in}{2.872933in}}%
\pgfpathlineto{\pgfqpoint{5.013185in}{2.876930in}}%
\pgfpathlineto{\pgfqpoint{5.026632in}{2.881081in}}%
\pgfpathlineto{\pgfqpoint{5.040091in}{2.885389in}}%
\pgfpathlineto{\pgfqpoint{5.047298in}{2.892803in}}%
\pgfpathlineto{\pgfqpoint{5.054501in}{2.900297in}}%
\pgfpathlineto{\pgfqpoint{5.061700in}{2.907874in}}%
\pgfpathlineto{\pgfqpoint{5.068896in}{2.915540in}}%
\pgfpathlineto{\pgfqpoint{5.055454in}{2.911688in}}%
\pgfpathlineto{\pgfqpoint{5.042024in}{2.907990in}}%
\pgfpathlineto{\pgfqpoint{5.028608in}{2.904447in}}%
\pgfpathlineto{\pgfqpoint{5.015203in}{2.901059in}}%
\pgfpathlineto{\pgfqpoint{5.007990in}{2.892930in}}%
\pgfpathlineto{\pgfqpoint{5.000773in}{2.884895in}}%
\pgfpathlineto{\pgfqpoint{4.993554in}{2.876951in}}%
\pgfpathlineto{\pgfqpoint{4.986330in}{2.869092in}}%
\pgfpathclose%
\pgfusepath{fill}%
\end{pgfscope}%
\begin{pgfscope}%
\pgfpathrectangle{\pgfqpoint{1.254980in}{0.150000in}}{\pgfqpoint{5.490039in}{5.490039in}}%
\pgfusepath{clip}%
\pgfsetbuttcap%
\pgfsetroundjoin%
\definecolor{currentfill}{rgb}{0.283091,0.110553,0.431554}%
\pgfsetfillcolor{currentfill}%
\pgfsetfillopacity{0.700000}%
\pgfsetlinewidth{0.000000pt}%
\definecolor{currentstroke}{rgb}{0.000000,0.000000,0.000000}%
\pgfsetstrokecolor{currentstroke}%
\pgfsetdash{}{0pt}%
\pgfpathmoveto{\pgfqpoint{3.696089in}{2.277278in}}%
\pgfpathlineto{\pgfqpoint{3.709088in}{2.273742in}}%
\pgfpathlineto{\pgfqpoint{3.722091in}{2.270395in}}%
\pgfpathlineto{\pgfqpoint{3.735100in}{2.267233in}}%
\pgfpathlineto{\pgfqpoint{3.748113in}{2.264257in}}%
\pgfpathlineto{\pgfqpoint{3.755776in}{2.273990in}}%
\pgfpathlineto{\pgfqpoint{3.763435in}{2.283737in}}%
\pgfpathlineto{\pgfqpoint{3.771088in}{2.293501in}}%
\pgfpathlineto{\pgfqpoint{3.778736in}{2.303281in}}%
\pgfpathlineto{\pgfqpoint{3.765732in}{2.306288in}}%
\pgfpathlineto{\pgfqpoint{3.752733in}{2.309481in}}%
\pgfpathlineto{\pgfqpoint{3.739739in}{2.312860in}}%
\pgfpathlineto{\pgfqpoint{3.726750in}{2.316425in}}%
\pgfpathlineto{\pgfqpoint{3.719092in}{2.306604in}}%
\pgfpathlineto{\pgfqpoint{3.711430in}{2.296806in}}%
\pgfpathlineto{\pgfqpoint{3.703762in}{2.287031in}}%
\pgfpathlineto{\pgfqpoint{3.696089in}{2.277278in}}%
\pgfpathclose%
\pgfusepath{fill}%
\end{pgfscope}%
\begin{pgfscope}%
\pgfpathrectangle{\pgfqpoint{1.254980in}{0.150000in}}{\pgfqpoint{5.490039in}{5.490039in}}%
\pgfusepath{clip}%
\pgfsetbuttcap%
\pgfsetroundjoin%
\definecolor{currentfill}{rgb}{0.282910,0.105393,0.426902}%
\pgfsetfillcolor{currentfill}%
\pgfsetfillopacity{0.700000}%
\pgfsetlinewidth{0.000000pt}%
\definecolor{currentstroke}{rgb}{0.000000,0.000000,0.000000}%
\pgfsetstrokecolor{currentstroke}%
\pgfsetdash{}{0pt}%
\pgfpathmoveto{\pgfqpoint{3.343675in}{2.274703in}}%
\pgfpathlineto{\pgfqpoint{3.356653in}{2.267145in}}%
\pgfpathlineto{\pgfqpoint{3.369632in}{2.259795in}}%
\pgfpathlineto{\pgfqpoint{3.382611in}{2.252652in}}%
\pgfpathlineto{\pgfqpoint{3.395593in}{2.245715in}}%
\pgfpathlineto{\pgfqpoint{3.403375in}{2.254987in}}%
\pgfpathlineto{\pgfqpoint{3.411152in}{2.264306in}}%
\pgfpathlineto{\pgfqpoint{3.418924in}{2.273670in}}%
\pgfpathlineto{\pgfqpoint{3.426689in}{2.283081in}}%
\pgfpathlineto{\pgfqpoint{3.413721in}{2.289966in}}%
\pgfpathlineto{\pgfqpoint{3.400754in}{2.297056in}}%
\pgfpathlineto{\pgfqpoint{3.387788in}{2.304352in}}%
\pgfpathlineto{\pgfqpoint{3.374823in}{2.311857in}}%
\pgfpathlineto{\pgfqpoint{3.367045in}{2.302489in}}%
\pgfpathlineto{\pgfqpoint{3.359261in}{2.293174in}}%
\pgfpathlineto{\pgfqpoint{3.351471in}{2.283912in}}%
\pgfpathlineto{\pgfqpoint{3.343675in}{2.274703in}}%
\pgfpathclose%
\pgfusepath{fill}%
\end{pgfscope}%
\begin{pgfscope}%
\pgfpathrectangle{\pgfqpoint{1.254980in}{0.150000in}}{\pgfqpoint{5.490039in}{5.490039in}}%
\pgfusepath{clip}%
\pgfsetbuttcap%
\pgfsetroundjoin%
\definecolor{currentfill}{rgb}{0.275191,0.194905,0.496005}%
\pgfsetfillcolor{currentfill}%
\pgfsetfillopacity{0.700000}%
\pgfsetlinewidth{0.000000pt}%
\definecolor{currentstroke}{rgb}{0.000000,0.000000,0.000000}%
\pgfsetstrokecolor{currentstroke}%
\pgfsetdash{}{0pt}%
\pgfpathmoveto{\pgfqpoint{4.161113in}{2.432561in}}%
\pgfpathlineto{\pgfqpoint{4.174224in}{2.432881in}}%
\pgfpathlineto{\pgfqpoint{4.187343in}{2.433372in}}%
\pgfpathlineto{\pgfqpoint{4.200471in}{2.434033in}}%
\pgfpathlineto{\pgfqpoint{4.213608in}{2.434864in}}%
\pgfpathlineto{\pgfqpoint{4.221120in}{2.444160in}}%
\pgfpathlineto{\pgfqpoint{4.228628in}{2.453459in}}%
\pgfpathlineto{\pgfqpoint{4.236131in}{2.462761in}}%
\pgfpathlineto{\pgfqpoint{4.243629in}{2.472071in}}%
\pgfpathlineto{\pgfqpoint{4.230501in}{2.471411in}}%
\pgfpathlineto{\pgfqpoint{4.217382in}{2.470921in}}%
\pgfpathlineto{\pgfqpoint{4.204271in}{2.470601in}}%
\pgfpathlineto{\pgfqpoint{4.191168in}{2.470452in}}%
\pgfpathlineto{\pgfqpoint{4.183662in}{2.460961in}}%
\pgfpathlineto{\pgfqpoint{4.176150in}{2.451484in}}%
\pgfpathlineto{\pgfqpoint{4.168634in}{2.442018in}}%
\pgfpathlineto{\pgfqpoint{4.161113in}{2.432561in}}%
\pgfpathclose%
\pgfusepath{fill}%
\end{pgfscope}%
\begin{pgfscope}%
\pgfpathrectangle{\pgfqpoint{1.254980in}{0.150000in}}{\pgfqpoint{5.490039in}{5.490039in}}%
\pgfusepath{clip}%
\pgfsetbuttcap%
\pgfsetroundjoin%
\definecolor{currentfill}{rgb}{0.188923,0.410910,0.556326}%
\pgfsetfillcolor{currentfill}%
\pgfsetfillopacity{0.700000}%
\pgfsetlinewidth{0.000000pt}%
\definecolor{currentstroke}{rgb}{0.000000,0.000000,0.000000}%
\pgfsetstrokecolor{currentstroke}%
\pgfsetdash{}{0pt}%
\pgfpathmoveto{\pgfqpoint{5.068896in}{2.915540in}}%
\pgfpathlineto{\pgfqpoint{5.082351in}{2.919547in}}%
\pgfpathlineto{\pgfqpoint{5.095819in}{2.923708in}}%
\pgfpathlineto{\pgfqpoint{5.109300in}{2.928024in}}%
\pgfpathlineto{\pgfqpoint{5.122795in}{2.932494in}}%
\pgfpathlineto{\pgfqpoint{5.129969in}{2.939781in}}%
\pgfpathlineto{\pgfqpoint{5.137141in}{2.947162in}}%
\pgfpathlineto{\pgfqpoint{5.144309in}{2.954641in}}%
\pgfpathlineto{\pgfqpoint{5.151474in}{2.962224in}}%
\pgfpathlineto{\pgfqpoint{5.137998in}{2.958237in}}%
\pgfpathlineto{\pgfqpoint{5.124535in}{2.954404in}}%
\pgfpathlineto{\pgfqpoint{5.111086in}{2.950725in}}%
\pgfpathlineto{\pgfqpoint{5.097649in}{2.947200in}}%
\pgfpathlineto{\pgfqpoint{5.090465in}{2.939125in}}%
\pgfpathlineto{\pgfqpoint{5.083278in}{2.931160in}}%
\pgfpathlineto{\pgfqpoint{5.076089in}{2.923300in}}%
\pgfpathlineto{\pgfqpoint{5.068896in}{2.915540in}}%
\pgfpathclose%
\pgfusepath{fill}%
\end{pgfscope}%
\begin{pgfscope}%
\pgfpathrectangle{\pgfqpoint{1.254980in}{0.150000in}}{\pgfqpoint{5.490039in}{5.490039in}}%
\pgfusepath{clip}%
\pgfsetbuttcap%
\pgfsetroundjoin%
\definecolor{currentfill}{rgb}{0.282656,0.100196,0.422160}%
\pgfsetfillcolor{currentfill}%
\pgfsetfillopacity{0.700000}%
\pgfsetlinewidth{0.000000pt}%
\definecolor{currentstroke}{rgb}{0.000000,0.000000,0.000000}%
\pgfsetstrokecolor{currentstroke}%
\pgfsetdash{}{0pt}%
\pgfpathmoveto{\pgfqpoint{3.478581in}{2.257574in}}%
\pgfpathlineto{\pgfqpoint{3.491559in}{2.251700in}}%
\pgfpathlineto{\pgfqpoint{3.504539in}{2.246024in}}%
\pgfpathlineto{\pgfqpoint{3.517523in}{2.240546in}}%
\pgfpathlineto{\pgfqpoint{3.530509in}{2.235264in}}%
\pgfpathlineto{\pgfqpoint{3.538245in}{2.244792in}}%
\pgfpathlineto{\pgfqpoint{3.545976in}{2.254351in}}%
\pgfpathlineto{\pgfqpoint{3.553701in}{2.263943in}}%
\pgfpathlineto{\pgfqpoint{3.561421in}{2.273568in}}%
\pgfpathlineto{\pgfqpoint{3.548446in}{2.278825in}}%
\pgfpathlineto{\pgfqpoint{3.535474in}{2.284278in}}%
\pgfpathlineto{\pgfqpoint{3.522504in}{2.289929in}}%
\pgfpathlineto{\pgfqpoint{3.509537in}{2.295779in}}%
\pgfpathlineto{\pgfqpoint{3.501807in}{2.286168in}}%
\pgfpathlineto{\pgfqpoint{3.494070in}{2.276598in}}%
\pgfpathlineto{\pgfqpoint{3.486328in}{2.267066in}}%
\pgfpathlineto{\pgfqpoint{3.478581in}{2.257574in}}%
\pgfpathclose%
\pgfusepath{fill}%
\end{pgfscope}%
\begin{pgfscope}%
\pgfpathrectangle{\pgfqpoint{1.254980in}{0.150000in}}{\pgfqpoint{5.490039in}{5.490039in}}%
\pgfusepath{clip}%
\pgfsetbuttcap%
\pgfsetroundjoin%
\definecolor{currentfill}{rgb}{0.260571,0.246922,0.522828}%
\pgfsetfillcolor{currentfill}%
\pgfsetfillopacity{0.700000}%
\pgfsetlinewidth{0.000000pt}%
\definecolor{currentstroke}{rgb}{0.000000,0.000000,0.000000}%
\pgfsetstrokecolor{currentstroke}%
\pgfsetdash{}{0pt}%
\pgfpathmoveto{\pgfqpoint{2.864038in}{2.569684in}}%
\pgfpathlineto{\pgfqpoint{2.877131in}{2.554452in}}%
\pgfpathlineto{\pgfqpoint{2.890218in}{2.539484in}}%
\pgfpathlineto{\pgfqpoint{2.903298in}{2.524776in}}%
\pgfpathlineto{\pgfqpoint{2.916373in}{2.510328in}}%
\pgfpathlineto{\pgfqpoint{2.924343in}{2.518250in}}%
\pgfpathlineto{\pgfqpoint{2.932304in}{2.526278in}}%
\pgfpathlineto{\pgfqpoint{2.940258in}{2.534411in}}%
\pgfpathlineto{\pgfqpoint{2.948203in}{2.542648in}}%
\pgfpathlineto{\pgfqpoint{2.935148in}{2.556983in}}%
\pgfpathlineto{\pgfqpoint{2.922088in}{2.571577in}}%
\pgfpathlineto{\pgfqpoint{2.909022in}{2.586431in}}%
\pgfpathlineto{\pgfqpoint{2.895950in}{2.601548in}}%
\pgfpathlineto{\pgfqpoint{2.887985in}{2.593414in}}%
\pgfpathlineto{\pgfqpoint{2.880011in}{2.585392in}}%
\pgfpathlineto{\pgfqpoint{2.872029in}{2.577481in}}%
\pgfpathlineto{\pgfqpoint{2.864038in}{2.569684in}}%
\pgfpathclose%
\pgfusepath{fill}%
\end{pgfscope}%
\begin{pgfscope}%
\pgfpathrectangle{\pgfqpoint{1.254980in}{0.150000in}}{\pgfqpoint{5.490039in}{5.490039in}}%
\pgfusepath{clip}%
\pgfsetbuttcap%
\pgfsetroundjoin%
\definecolor{currentfill}{rgb}{0.250425,0.274290,0.533103}%
\pgfsetfillcolor{currentfill}%
\pgfsetfillopacity{0.700000}%
\pgfsetlinewidth{0.000000pt}%
\definecolor{currentstroke}{rgb}{0.000000,0.000000,0.000000}%
\pgfsetstrokecolor{currentstroke}%
\pgfsetdash{}{0pt}%
\pgfpathmoveto{\pgfqpoint{2.811601in}{2.633288in}}%
\pgfpathlineto{\pgfqpoint{2.824721in}{2.616980in}}%
\pgfpathlineto{\pgfqpoint{2.837833in}{2.600945in}}%
\pgfpathlineto{\pgfqpoint{2.850939in}{2.585181in}}%
\pgfpathlineto{\pgfqpoint{2.864038in}{2.569684in}}%
\pgfpathlineto{\pgfqpoint{2.872029in}{2.577481in}}%
\pgfpathlineto{\pgfqpoint{2.880011in}{2.585392in}}%
\pgfpathlineto{\pgfqpoint{2.887985in}{2.593414in}}%
\pgfpathlineto{\pgfqpoint{2.895950in}{2.601548in}}%
\pgfpathlineto{\pgfqpoint{2.882872in}{2.616930in}}%
\pgfpathlineto{\pgfqpoint{2.869788in}{2.632580in}}%
\pgfpathlineto{\pgfqpoint{2.856697in}{2.648500in}}%
\pgfpathlineto{\pgfqpoint{2.843598in}{2.664692in}}%
\pgfpathlineto{\pgfqpoint{2.835612in}{2.656663in}}%
\pgfpathlineto{\pgfqpoint{2.827617in}{2.648752in}}%
\pgfpathlineto{\pgfqpoint{2.819613in}{2.640960in}}%
\pgfpathlineto{\pgfqpoint{2.811601in}{2.633288in}}%
\pgfpathclose%
\pgfusepath{fill}%
\end{pgfscope}%
\begin{pgfscope}%
\pgfpathrectangle{\pgfqpoint{1.254980in}{0.150000in}}{\pgfqpoint{5.490039in}{5.490039in}}%
\pgfusepath{clip}%
\pgfsetbuttcap%
\pgfsetroundjoin%
\definecolor{currentfill}{rgb}{0.180629,0.429975,0.557282}%
\pgfsetfillcolor{currentfill}%
\pgfsetfillopacity{0.700000}%
\pgfsetlinewidth{0.000000pt}%
\definecolor{currentstroke}{rgb}{0.000000,0.000000,0.000000}%
\pgfsetstrokecolor{currentstroke}%
\pgfsetdash{}{0pt}%
\pgfpathmoveto{\pgfqpoint{5.151474in}{2.962224in}}%
\pgfpathlineto{\pgfqpoint{5.164963in}{2.966364in}}%
\pgfpathlineto{\pgfqpoint{5.178465in}{2.970658in}}%
\pgfpathlineto{\pgfqpoint{5.191981in}{2.975105in}}%
\pgfpathlineto{\pgfqpoint{5.205510in}{2.979705in}}%
\pgfpathlineto{\pgfqpoint{5.212653in}{2.986896in}}%
\pgfpathlineto{\pgfqpoint{5.219792in}{2.994196in}}%
\pgfpathlineto{\pgfqpoint{5.226930in}{3.001610in}}%
\pgfpathlineto{\pgfqpoint{5.234065in}{3.009144in}}%
\pgfpathlineto{\pgfqpoint{5.220556in}{3.005055in}}%
\pgfpathlineto{\pgfqpoint{5.207060in}{3.001119in}}%
\pgfpathlineto{\pgfqpoint{5.193578in}{2.997336in}}%
\pgfpathlineto{\pgfqpoint{5.180108in}{2.993706in}}%
\pgfpathlineto{\pgfqpoint{5.172953in}{2.985652in}}%
\pgfpathlineto{\pgfqpoint{5.165796in}{2.977724in}}%
\pgfpathlineto{\pgfqpoint{5.158636in}{2.969916in}}%
\pgfpathlineto{\pgfqpoint{5.151474in}{2.962224in}}%
\pgfpathclose%
\pgfusepath{fill}%
\end{pgfscope}%
\begin{pgfscope}%
\pgfpathrectangle{\pgfqpoint{1.254980in}{0.150000in}}{\pgfqpoint{5.490039in}{5.490039in}}%
\pgfusepath{clip}%
\pgfsetbuttcap%
\pgfsetroundjoin%
\definecolor{currentfill}{rgb}{0.269308,0.218818,0.509577}%
\pgfsetfillcolor{currentfill}%
\pgfsetfillopacity{0.700000}%
\pgfsetlinewidth{0.000000pt}%
\definecolor{currentstroke}{rgb}{0.000000,0.000000,0.000000}%
\pgfsetstrokecolor{currentstroke}%
\pgfsetdash{}{0pt}%
\pgfpathmoveto{\pgfqpoint{2.916373in}{2.510328in}}%
\pgfpathlineto{\pgfqpoint{2.929442in}{2.496136in}}%
\pgfpathlineto{\pgfqpoint{2.942507in}{2.482198in}}%
\pgfpathlineto{\pgfqpoint{2.955566in}{2.468512in}}%
\pgfpathlineto{\pgfqpoint{2.968621in}{2.455077in}}%
\pgfpathlineto{\pgfqpoint{2.976570in}{2.463123in}}%
\pgfpathlineto{\pgfqpoint{2.984512in}{2.471268in}}%
\pgfpathlineto{\pgfqpoint{2.992446in}{2.479510in}}%
\pgfpathlineto{\pgfqpoint{3.000372in}{2.487850in}}%
\pgfpathlineto{\pgfqpoint{2.987337in}{2.501172in}}%
\pgfpathlineto{\pgfqpoint{2.974297in}{2.514745in}}%
\pgfpathlineto{\pgfqpoint{2.961252in}{2.528569in}}%
\pgfpathlineto{\pgfqpoint{2.948203in}{2.542648in}}%
\pgfpathlineto{\pgfqpoint{2.940258in}{2.534411in}}%
\pgfpathlineto{\pgfqpoint{2.932304in}{2.526278in}}%
\pgfpathlineto{\pgfqpoint{2.924343in}{2.518250in}}%
\pgfpathlineto{\pgfqpoint{2.916373in}{2.510328in}}%
\pgfpathclose%
\pgfusepath{fill}%
\end{pgfscope}%
\begin{pgfscope}%
\pgfpathrectangle{\pgfqpoint{1.254980in}{0.150000in}}{\pgfqpoint{5.490039in}{5.490039in}}%
\pgfusepath{clip}%
\pgfsetbuttcap%
\pgfsetroundjoin%
\definecolor{currentfill}{rgb}{0.278826,0.175490,0.483397}%
\pgfsetfillcolor{currentfill}%
\pgfsetfillopacity{0.700000}%
\pgfsetlinewidth{0.000000pt}%
\definecolor{currentstroke}{rgb}{0.000000,0.000000,0.000000}%
\pgfsetstrokecolor{currentstroke}%
\pgfsetdash{}{0pt}%
\pgfpathmoveto{\pgfqpoint{4.078581in}{2.394652in}}%
\pgfpathlineto{\pgfqpoint{4.091669in}{2.394425in}}%
\pgfpathlineto{\pgfqpoint{4.104764in}{2.394372in}}%
\pgfpathlineto{\pgfqpoint{4.117867in}{2.394491in}}%
\pgfpathlineto{\pgfqpoint{4.130979in}{2.394782in}}%
\pgfpathlineto{\pgfqpoint{4.138520in}{2.404223in}}%
\pgfpathlineto{\pgfqpoint{4.146056in}{2.413666in}}%
\pgfpathlineto{\pgfqpoint{4.153587in}{2.423111in}}%
\pgfpathlineto{\pgfqpoint{4.161113in}{2.432561in}}%
\pgfpathlineto{\pgfqpoint{4.148010in}{2.432413in}}%
\pgfpathlineto{\pgfqpoint{4.134915in}{2.432437in}}%
\pgfpathlineto{\pgfqpoint{4.121828in}{2.432633in}}%
\pgfpathlineto{\pgfqpoint{4.108749in}{2.433002in}}%
\pgfpathlineto{\pgfqpoint{4.101214in}{2.423399in}}%
\pgfpathlineto{\pgfqpoint{4.093675in}{2.413807in}}%
\pgfpathlineto{\pgfqpoint{4.086130in}{2.404226in}}%
\pgfpathlineto{\pgfqpoint{4.078581in}{2.394652in}}%
\pgfpathclose%
\pgfusepath{fill}%
\end{pgfscope}%
\begin{pgfscope}%
\pgfpathrectangle{\pgfqpoint{1.254980in}{0.150000in}}{\pgfqpoint{5.490039in}{5.490039in}}%
\pgfusepath{clip}%
\pgfsetbuttcap%
\pgfsetroundjoin%
\definecolor{currentfill}{rgb}{0.172719,0.448791,0.557885}%
\pgfsetfillcolor{currentfill}%
\pgfsetfillopacity{0.700000}%
\pgfsetlinewidth{0.000000pt}%
\definecolor{currentstroke}{rgb}{0.000000,0.000000,0.000000}%
\pgfsetstrokecolor{currentstroke}%
\pgfsetdash{}{0pt}%
\pgfpathmoveto{\pgfqpoint{5.234065in}{3.009144in}}%
\pgfpathlineto{\pgfqpoint{5.247587in}{3.013385in}}%
\pgfpathlineto{\pgfqpoint{5.261123in}{3.017778in}}%
\pgfpathlineto{\pgfqpoint{5.274673in}{3.022323in}}%
\pgfpathlineto{\pgfqpoint{5.288237in}{3.027021in}}%
\pgfpathlineto{\pgfqpoint{5.295348in}{3.034152in}}%
\pgfpathlineto{\pgfqpoint{5.302457in}{3.041407in}}%
\pgfpathlineto{\pgfqpoint{5.309565in}{3.048794in}}%
\pgfpathlineto{\pgfqpoint{5.316670in}{3.056318in}}%
\pgfpathlineto{\pgfqpoint{5.303128in}{3.052160in}}%
\pgfpathlineto{\pgfqpoint{5.289600in}{3.048154in}}%
\pgfpathlineto{\pgfqpoint{5.276086in}{3.044299in}}%
\pgfpathlineto{\pgfqpoint{5.262584in}{3.040597in}}%
\pgfpathlineto{\pgfqpoint{5.255457in}{3.032524in}}%
\pgfpathlineto{\pgfqpoint{5.248328in}{3.024595in}}%
\pgfpathlineto{\pgfqpoint{5.241197in}{3.016803in}}%
\pgfpathlineto{\pgfqpoint{5.234065in}{3.009144in}}%
\pgfpathclose%
\pgfusepath{fill}%
\end{pgfscope}%
\begin{pgfscope}%
\pgfpathrectangle{\pgfqpoint{1.254980in}{0.150000in}}{\pgfqpoint{5.490039in}{5.490039in}}%
\pgfusepath{clip}%
\pgfsetbuttcap%
\pgfsetroundjoin%
\definecolor{currentfill}{rgb}{0.237441,0.305202,0.541921}%
\pgfsetfillcolor{currentfill}%
\pgfsetfillopacity{0.700000}%
\pgfsetlinewidth{0.000000pt}%
\definecolor{currentstroke}{rgb}{0.000000,0.000000,0.000000}%
\pgfsetstrokecolor{currentstroke}%
\pgfsetdash{}{0pt}%
\pgfpathmoveto{\pgfqpoint{2.759044in}{2.701294in}}%
\pgfpathlineto{\pgfqpoint{2.772195in}{2.683871in}}%
\pgfpathlineto{\pgfqpoint{2.785338in}{2.666731in}}%
\pgfpathlineto{\pgfqpoint{2.798473in}{2.649870in}}%
\pgfpathlineto{\pgfqpoint{2.811601in}{2.633288in}}%
\pgfpathlineto{\pgfqpoint{2.819613in}{2.640960in}}%
\pgfpathlineto{\pgfqpoint{2.827617in}{2.648752in}}%
\pgfpathlineto{\pgfqpoint{2.835612in}{2.656663in}}%
\pgfpathlineto{\pgfqpoint{2.843598in}{2.664692in}}%
\pgfpathlineto{\pgfqpoint{2.830493in}{2.681159in}}%
\pgfpathlineto{\pgfqpoint{2.817380in}{2.697903in}}%
\pgfpathlineto{\pgfqpoint{2.804260in}{2.714927in}}%
\pgfpathlineto{\pgfqpoint{2.791131in}{2.732233in}}%
\pgfpathlineto{\pgfqpoint{2.783123in}{2.724309in}}%
\pgfpathlineto{\pgfqpoint{2.775106in}{2.716511in}}%
\pgfpathlineto{\pgfqpoint{2.767079in}{2.708839in}}%
\pgfpathlineto{\pgfqpoint{2.759044in}{2.701294in}}%
\pgfpathclose%
\pgfusepath{fill}%
\end{pgfscope}%
\begin{pgfscope}%
\pgfpathrectangle{\pgfqpoint{1.254980in}{0.150000in}}{\pgfqpoint{5.490039in}{5.490039in}}%
\pgfusepath{clip}%
\pgfsetbuttcap%
\pgfsetroundjoin%
\definecolor{currentfill}{rgb}{0.165117,0.467423,0.558141}%
\pgfsetfillcolor{currentfill}%
\pgfsetfillopacity{0.700000}%
\pgfsetlinewidth{0.000000pt}%
\definecolor{currentstroke}{rgb}{0.000000,0.000000,0.000000}%
\pgfsetstrokecolor{currentstroke}%
\pgfsetdash{}{0pt}%
\pgfpathmoveto{\pgfqpoint{5.316670in}{3.056318in}}%
\pgfpathlineto{\pgfqpoint{5.330226in}{3.060627in}}%
\pgfpathlineto{\pgfqpoint{5.343795in}{3.065088in}}%
\pgfpathlineto{\pgfqpoint{5.357379in}{3.069700in}}%
\pgfpathlineto{\pgfqpoint{5.370976in}{3.074464in}}%
\pgfpathlineto{\pgfqpoint{5.378057in}{3.081573in}}%
\pgfpathlineto{\pgfqpoint{5.385137in}{3.088826in}}%
\pgfpathlineto{\pgfqpoint{5.392215in}{3.096229in}}%
\pgfpathlineto{\pgfqpoint{5.399293in}{3.103787in}}%
\pgfpathlineto{\pgfqpoint{5.385719in}{3.099592in}}%
\pgfpathlineto{\pgfqpoint{5.372159in}{3.095547in}}%
\pgfpathlineto{\pgfqpoint{5.358612in}{3.091654in}}%
\pgfpathlineto{\pgfqpoint{5.345079in}{3.087911in}}%
\pgfpathlineto{\pgfqpoint{5.337979in}{3.079775in}}%
\pgfpathlineto{\pgfqpoint{5.330877in}{3.071802in}}%
\pgfpathlineto{\pgfqpoint{5.323774in}{3.063985in}}%
\pgfpathlineto{\pgfqpoint{5.316670in}{3.056318in}}%
\pgfpathclose%
\pgfusepath{fill}%
\end{pgfscope}%
\begin{pgfscope}%
\pgfpathrectangle{\pgfqpoint{1.254980in}{0.150000in}}{\pgfqpoint{5.490039in}{5.490039in}}%
\pgfusepath{clip}%
\pgfsetbuttcap%
\pgfsetroundjoin%
\definecolor{currentfill}{rgb}{0.283229,0.120777,0.440584}%
\pgfsetfillcolor{currentfill}%
\pgfsetfillopacity{0.700000}%
\pgfsetlinewidth{0.000000pt}%
\definecolor{currentstroke}{rgb}{0.000000,0.000000,0.000000}%
\pgfsetstrokecolor{currentstroke}%
\pgfsetdash{}{0pt}%
\pgfpathmoveto{\pgfqpoint{3.208503in}{2.307162in}}%
\pgfpathlineto{\pgfqpoint{3.221495in}{2.297811in}}%
\pgfpathlineto{\pgfqpoint{3.234487in}{2.288679in}}%
\pgfpathlineto{\pgfqpoint{3.247477in}{2.279764in}}%
\pgfpathlineto{\pgfqpoint{3.260468in}{2.271065in}}%
\pgfpathlineto{\pgfqpoint{3.268303in}{2.279967in}}%
\pgfpathlineto{\pgfqpoint{3.276132in}{2.288931in}}%
\pgfpathlineto{\pgfqpoint{3.283954in}{2.297955in}}%
\pgfpathlineto{\pgfqpoint{3.291770in}{2.307040in}}%
\pgfpathlineto{\pgfqpoint{3.278794in}{2.315658in}}%
\pgfpathlineto{\pgfqpoint{3.265819in}{2.324491in}}%
\pgfpathlineto{\pgfqpoint{3.252842in}{2.333542in}}%
\pgfpathlineto{\pgfqpoint{3.239865in}{2.342811in}}%
\pgfpathlineto{\pgfqpoint{3.232035in}{2.333797in}}%
\pgfpathlineto{\pgfqpoint{3.224198in}{2.324851in}}%
\pgfpathlineto{\pgfqpoint{3.216354in}{2.315973in}}%
\pgfpathlineto{\pgfqpoint{3.208503in}{2.307162in}}%
\pgfpathclose%
\pgfusepath{fill}%
\end{pgfscope}%
\begin{pgfscope}%
\pgfpathrectangle{\pgfqpoint{1.254980in}{0.150000in}}{\pgfqpoint{5.490039in}{5.490039in}}%
\pgfusepath{clip}%
\pgfsetbuttcap%
\pgfsetroundjoin%
\definecolor{currentfill}{rgb}{0.157729,0.485932,0.558013}%
\pgfsetfillcolor{currentfill}%
\pgfsetfillopacity{0.700000}%
\pgfsetlinewidth{0.000000pt}%
\definecolor{currentstroke}{rgb}{0.000000,0.000000,0.000000}%
\pgfsetstrokecolor{currentstroke}%
\pgfsetdash{}{0pt}%
\pgfpathmoveto{\pgfqpoint{5.399293in}{3.103787in}}%
\pgfpathlineto{\pgfqpoint{5.412881in}{3.108133in}}%
\pgfpathlineto{\pgfqpoint{5.426483in}{3.112629in}}%
\pgfpathlineto{\pgfqpoint{5.440099in}{3.117276in}}%
\pgfpathlineto{\pgfqpoint{5.453730in}{3.122073in}}%
\pgfpathlineto{\pgfqpoint{5.460782in}{3.129207in}}%
\pgfpathlineto{\pgfqpoint{5.467834in}{3.136504in}}%
\pgfpathlineto{\pgfqpoint{5.474884in}{3.143970in}}%
\pgfpathlineto{\pgfqpoint{5.481935in}{3.151611in}}%
\pgfpathlineto{\pgfqpoint{5.468330in}{3.147411in}}%
\pgfpathlineto{\pgfqpoint{5.454738in}{3.143360in}}%
\pgfpathlineto{\pgfqpoint{5.441161in}{3.139459in}}%
\pgfpathlineto{\pgfqpoint{5.427597in}{3.135708in}}%
\pgfpathlineto{\pgfqpoint{5.420521in}{3.127461in}}%
\pgfpathlineto{\pgfqpoint{5.413445in}{3.119397in}}%
\pgfpathlineto{\pgfqpoint{5.406369in}{3.111507in}}%
\pgfpathlineto{\pgfqpoint{5.399293in}{3.103787in}}%
\pgfpathclose%
\pgfusepath{fill}%
\end{pgfscope}%
\begin{pgfscope}%
\pgfpathrectangle{\pgfqpoint{1.254980in}{0.150000in}}{\pgfqpoint{5.490039in}{5.490039in}}%
\pgfusepath{clip}%
\pgfsetbuttcap%
\pgfsetroundjoin%
\definecolor{currentfill}{rgb}{0.275191,0.194905,0.496005}%
\pgfsetfillcolor{currentfill}%
\pgfsetfillopacity{0.700000}%
\pgfsetlinewidth{0.000000pt}%
\definecolor{currentstroke}{rgb}{0.000000,0.000000,0.000000}%
\pgfsetstrokecolor{currentstroke}%
\pgfsetdash{}{0pt}%
\pgfpathmoveto{\pgfqpoint{2.968621in}{2.455077in}}%
\pgfpathlineto{\pgfqpoint{2.981671in}{2.441890in}}%
\pgfpathlineto{\pgfqpoint{2.994717in}{2.428950in}}%
\pgfpathlineto{\pgfqpoint{3.007759in}{2.416254in}}%
\pgfpathlineto{\pgfqpoint{3.020797in}{2.403800in}}%
\pgfpathlineto{\pgfqpoint{3.028727in}{2.411968in}}%
\pgfpathlineto{\pgfqpoint{3.036650in}{2.420228in}}%
\pgfpathlineto{\pgfqpoint{3.044565in}{2.428579in}}%
\pgfpathlineto{\pgfqpoint{3.052472in}{2.437020in}}%
\pgfpathlineto{\pgfqpoint{3.039453in}{2.449362in}}%
\pgfpathlineto{\pgfqpoint{3.026430in}{2.461947in}}%
\pgfpathlineto{\pgfqpoint{3.013403in}{2.474775in}}%
\pgfpathlineto{\pgfqpoint{3.000372in}{2.487850in}}%
\pgfpathlineto{\pgfqpoint{2.992446in}{2.479510in}}%
\pgfpathlineto{\pgfqpoint{2.984512in}{2.471268in}}%
\pgfpathlineto{\pgfqpoint{2.976570in}{2.463123in}}%
\pgfpathlineto{\pgfqpoint{2.968621in}{2.455077in}}%
\pgfpathclose%
\pgfusepath{fill}%
\end{pgfscope}%
\begin{pgfscope}%
\pgfpathrectangle{\pgfqpoint{1.254980in}{0.150000in}}{\pgfqpoint{5.490039in}{5.490039in}}%
\pgfusepath{clip}%
\pgfsetbuttcap%
\pgfsetroundjoin%
\definecolor{currentfill}{rgb}{0.280868,0.160771,0.472899}%
\pgfsetfillcolor{currentfill}%
\pgfsetfillopacity{0.700000}%
\pgfsetlinewidth{0.000000pt}%
\definecolor{currentstroke}{rgb}{0.000000,0.000000,0.000000}%
\pgfsetstrokecolor{currentstroke}%
\pgfsetdash{}{0pt}%
\pgfpathmoveto{\pgfqpoint{3.996026in}{2.358588in}}%
\pgfpathlineto{\pgfqpoint{4.009093in}{2.357778in}}%
\pgfpathlineto{\pgfqpoint{4.022166in}{2.357143in}}%
\pgfpathlineto{\pgfqpoint{4.035247in}{2.356684in}}%
\pgfpathlineto{\pgfqpoint{4.048335in}{2.356398in}}%
\pgfpathlineto{\pgfqpoint{4.055904in}{2.365959in}}%
\pgfpathlineto{\pgfqpoint{4.063468in}{2.375520in}}%
\pgfpathlineto{\pgfqpoint{4.071027in}{2.385083in}}%
\pgfpathlineto{\pgfqpoint{4.078581in}{2.394652in}}%
\pgfpathlineto{\pgfqpoint{4.065501in}{2.395052in}}%
\pgfpathlineto{\pgfqpoint{4.052429in}{2.395626in}}%
\pgfpathlineto{\pgfqpoint{4.039364in}{2.396376in}}%
\pgfpathlineto{\pgfqpoint{4.026306in}{2.397300in}}%
\pgfpathlineto{\pgfqpoint{4.018743in}{2.387607in}}%
\pgfpathlineto{\pgfqpoint{4.011176in}{2.377925in}}%
\pgfpathlineto{\pgfqpoint{4.003604in}{2.368252in}}%
\pgfpathlineto{\pgfqpoint{3.996026in}{2.358588in}}%
\pgfpathclose%
\pgfusepath{fill}%
\end{pgfscope}%
\begin{pgfscope}%
\pgfpathrectangle{\pgfqpoint{1.254980in}{0.150000in}}{\pgfqpoint{5.490039in}{5.490039in}}%
\pgfusepath{clip}%
\pgfsetbuttcap%
\pgfsetroundjoin%
\definecolor{currentfill}{rgb}{0.150476,0.504369,0.557430}%
\pgfsetfillcolor{currentfill}%
\pgfsetfillopacity{0.700000}%
\pgfsetlinewidth{0.000000pt}%
\definecolor{currentstroke}{rgb}{0.000000,0.000000,0.000000}%
\pgfsetstrokecolor{currentstroke}%
\pgfsetdash{}{0pt}%
\pgfpathmoveto{\pgfqpoint{5.481935in}{3.151611in}}%
\pgfpathlineto{\pgfqpoint{5.495555in}{3.155961in}}%
\pgfpathlineto{\pgfqpoint{5.509189in}{3.160461in}}%
\pgfpathlineto{\pgfqpoint{5.522838in}{3.165111in}}%
\pgfpathlineto{\pgfqpoint{5.536501in}{3.169910in}}%
\pgfpathlineto{\pgfqpoint{5.543526in}{3.177119in}}%
\pgfpathlineto{\pgfqpoint{5.550550in}{3.184512in}}%
\pgfpathlineto{\pgfqpoint{5.557576in}{3.192094in}}%
\pgfpathlineto{\pgfqpoint{5.564602in}{3.199873in}}%
\pgfpathlineto{\pgfqpoint{5.550965in}{3.195699in}}%
\pgfpathlineto{\pgfqpoint{5.537343in}{3.191674in}}%
\pgfpathlineto{\pgfqpoint{5.523735in}{3.187797in}}%
\pgfpathlineto{\pgfqpoint{5.510142in}{3.184070in}}%
\pgfpathlineto{\pgfqpoint{5.503089in}{3.175658in}}%
\pgfpathlineto{\pgfqpoint{5.496037in}{3.167448in}}%
\pgfpathlineto{\pgfqpoint{5.488986in}{3.159435in}}%
\pgfpathlineto{\pgfqpoint{5.481935in}{3.151611in}}%
\pgfpathclose%
\pgfusepath{fill}%
\end{pgfscope}%
\begin{pgfscope}%
\pgfpathrectangle{\pgfqpoint{1.254980in}{0.150000in}}{\pgfqpoint{5.490039in}{5.490039in}}%
\pgfusepath{clip}%
\pgfsetbuttcap%
\pgfsetroundjoin%
\definecolor{currentfill}{rgb}{0.221989,0.339161,0.548752}%
\pgfsetfillcolor{currentfill}%
\pgfsetfillopacity{0.700000}%
\pgfsetlinewidth{0.000000pt}%
\definecolor{currentstroke}{rgb}{0.000000,0.000000,0.000000}%
\pgfsetstrokecolor{currentstroke}%
\pgfsetdash{}{0pt}%
\pgfpathmoveto{\pgfqpoint{2.706350in}{2.773866in}}%
\pgfpathlineto{\pgfqpoint{2.719537in}{2.755285in}}%
\pgfpathlineto{\pgfqpoint{2.732715in}{2.736998in}}%
\pgfpathlineto{\pgfqpoint{2.745884in}{2.719002in}}%
\pgfpathlineto{\pgfqpoint{2.759044in}{2.701294in}}%
\pgfpathlineto{\pgfqpoint{2.767079in}{2.708839in}}%
\pgfpathlineto{\pgfqpoint{2.775106in}{2.716511in}}%
\pgfpathlineto{\pgfqpoint{2.783123in}{2.724309in}}%
\pgfpathlineto{\pgfqpoint{2.791131in}{2.732233in}}%
\pgfpathlineto{\pgfqpoint{2.777994in}{2.749824in}}%
\pgfpathlineto{\pgfqpoint{2.764848in}{2.767703in}}%
\pgfpathlineto{\pgfqpoint{2.751694in}{2.785872in}}%
\pgfpathlineto{\pgfqpoint{2.738530in}{2.804335in}}%
\pgfpathlineto{\pgfqpoint{2.730499in}{2.796518in}}%
\pgfpathlineto{\pgfqpoint{2.722459in}{2.788834in}}%
\pgfpathlineto{\pgfqpoint{2.714409in}{2.781283in}}%
\pgfpathlineto{\pgfqpoint{2.706350in}{2.773866in}}%
\pgfpathclose%
\pgfusepath{fill}%
\end{pgfscope}%
\begin{pgfscope}%
\pgfpathrectangle{\pgfqpoint{1.254980in}{0.150000in}}{\pgfqpoint{5.490039in}{5.490039in}}%
\pgfusepath{clip}%
\pgfsetbuttcap%
\pgfsetroundjoin%
\definecolor{currentfill}{rgb}{0.282656,0.100196,0.422160}%
\pgfsetfillcolor{currentfill}%
\pgfsetfillopacity{0.700000}%
\pgfsetlinewidth{0.000000pt}%
\definecolor{currentstroke}{rgb}{0.000000,0.000000,0.000000}%
\pgfsetstrokecolor{currentstroke}%
\pgfsetdash{}{0pt}%
\pgfpathmoveto{\pgfqpoint{3.613353in}{2.254485in}}%
\pgfpathlineto{\pgfqpoint{3.626345in}{2.250196in}}%
\pgfpathlineto{\pgfqpoint{3.639341in}{2.246098in}}%
\pgfpathlineto{\pgfqpoint{3.652341in}{2.242190in}}%
\pgfpathlineto{\pgfqpoint{3.665346in}{2.238470in}}%
\pgfpathlineto{\pgfqpoint{3.673039in}{2.248142in}}%
\pgfpathlineto{\pgfqpoint{3.680728in}{2.257833in}}%
\pgfpathlineto{\pgfqpoint{3.688411in}{2.267545in}}%
\pgfpathlineto{\pgfqpoint{3.696089in}{2.277278in}}%
\pgfpathlineto{\pgfqpoint{3.683095in}{2.281001in}}%
\pgfpathlineto{\pgfqpoint{3.670104in}{2.284913in}}%
\pgfpathlineto{\pgfqpoint{3.657119in}{2.289014in}}%
\pgfpathlineto{\pgfqpoint{3.644137in}{2.293307in}}%
\pgfpathlineto{\pgfqpoint{3.636449in}{2.283561in}}%
\pgfpathlineto{\pgfqpoint{3.628755in}{2.273842in}}%
\pgfpathlineto{\pgfqpoint{3.621057in}{2.264151in}}%
\pgfpathlineto{\pgfqpoint{3.613353in}{2.254485in}}%
\pgfpathclose%
\pgfusepath{fill}%
\end{pgfscope}%
\begin{pgfscope}%
\pgfpathrectangle{\pgfqpoint{1.254980in}{0.150000in}}{\pgfqpoint{5.490039in}{5.490039in}}%
\pgfusepath{clip}%
\pgfsetbuttcap%
\pgfsetroundjoin%
\definecolor{currentfill}{rgb}{0.143343,0.522773,0.556295}%
\pgfsetfillcolor{currentfill}%
\pgfsetfillopacity{0.700000}%
\pgfsetlinewidth{0.000000pt}%
\definecolor{currentstroke}{rgb}{0.000000,0.000000,0.000000}%
\pgfsetstrokecolor{currentstroke}%
\pgfsetdash{}{0pt}%
\pgfpathmoveto{\pgfqpoint{5.564602in}{3.199873in}}%
\pgfpathlineto{\pgfqpoint{5.578252in}{3.204196in}}%
\pgfpathlineto{\pgfqpoint{5.591918in}{3.208668in}}%
\pgfpathlineto{\pgfqpoint{5.605598in}{3.213288in}}%
\pgfpathlineto{\pgfqpoint{5.619292in}{3.218058in}}%
\pgfpathlineto{\pgfqpoint{5.626291in}{3.225398in}}%
\pgfpathlineto{\pgfqpoint{5.633292in}{3.232942in}}%
\pgfpathlineto{\pgfqpoint{5.640293in}{3.240699in}}%
\pgfpathlineto{\pgfqpoint{5.647297in}{3.248675in}}%
\pgfpathlineto{\pgfqpoint{5.633631in}{3.244559in}}%
\pgfpathlineto{\pgfqpoint{5.619979in}{3.240590in}}%
\pgfpathlineto{\pgfqpoint{5.606342in}{3.236770in}}%
\pgfpathlineto{\pgfqpoint{5.592719in}{3.233098in}}%
\pgfpathlineto{\pgfqpoint{5.585687in}{3.224461in}}%
\pgfpathlineto{\pgfqpoint{5.578657in}{3.216049in}}%
\pgfpathlineto{\pgfqpoint{5.571629in}{3.207856in}}%
\pgfpathlineto{\pgfqpoint{5.564602in}{3.199873in}}%
\pgfpathclose%
\pgfusepath{fill}%
\end{pgfscope}%
\begin{pgfscope}%
\pgfpathrectangle{\pgfqpoint{1.254980in}{0.150000in}}{\pgfqpoint{5.490039in}{5.490039in}}%
\pgfusepath{clip}%
\pgfsetbuttcap%
\pgfsetroundjoin%
\definecolor{currentfill}{rgb}{0.282623,0.140926,0.457517}%
\pgfsetfillcolor{currentfill}%
\pgfsetfillopacity{0.700000}%
\pgfsetlinewidth{0.000000pt}%
\definecolor{currentstroke}{rgb}{0.000000,0.000000,0.000000}%
\pgfsetstrokecolor{currentstroke}%
\pgfsetdash{}{0pt}%
\pgfpathmoveto{\pgfqpoint{3.913438in}{2.324638in}}%
\pgfpathlineto{\pgfqpoint{3.926486in}{2.323206in}}%
\pgfpathlineto{\pgfqpoint{3.939540in}{2.321953in}}%
\pgfpathlineto{\pgfqpoint{3.952600in}{2.320877in}}%
\pgfpathlineto{\pgfqpoint{3.965668in}{2.319978in}}%
\pgfpathlineto{\pgfqpoint{3.973265in}{2.329626in}}%
\pgfpathlineto{\pgfqpoint{3.980857in}{2.339276in}}%
\pgfpathlineto{\pgfqpoint{3.988444in}{2.348929in}}%
\pgfpathlineto{\pgfqpoint{3.996026in}{2.358588in}}%
\pgfpathlineto{\pgfqpoint{3.982967in}{2.359574in}}%
\pgfpathlineto{\pgfqpoint{3.969915in}{2.360737in}}%
\pgfpathlineto{\pgfqpoint{3.956869in}{2.362077in}}%
\pgfpathlineto{\pgfqpoint{3.943831in}{2.363595in}}%
\pgfpathlineto{\pgfqpoint{3.936240in}{2.353840in}}%
\pgfpathlineto{\pgfqpoint{3.928644in}{2.344096in}}%
\pgfpathlineto{\pgfqpoint{3.921044in}{2.334362in}}%
\pgfpathlineto{\pgfqpoint{3.913438in}{2.324638in}}%
\pgfpathclose%
\pgfusepath{fill}%
\end{pgfscope}%
\begin{pgfscope}%
\pgfpathrectangle{\pgfqpoint{1.254980in}{0.150000in}}{\pgfqpoint{5.490039in}{5.490039in}}%
\pgfusepath{clip}%
\pgfsetbuttcap%
\pgfsetroundjoin%
\definecolor{currentfill}{rgb}{0.279574,0.170599,0.479997}%
\pgfsetfillcolor{currentfill}%
\pgfsetfillopacity{0.700000}%
\pgfsetlinewidth{0.000000pt}%
\definecolor{currentstroke}{rgb}{0.000000,0.000000,0.000000}%
\pgfsetstrokecolor{currentstroke}%
\pgfsetdash{}{0pt}%
\pgfpathmoveto{\pgfqpoint{3.020797in}{2.403800in}}%
\pgfpathlineto{\pgfqpoint{3.033831in}{2.391587in}}%
\pgfpathlineto{\pgfqpoint{3.046862in}{2.379612in}}%
\pgfpathlineto{\pgfqpoint{3.059890in}{2.367875in}}%
\pgfpathlineto{\pgfqpoint{3.072915in}{2.356372in}}%
\pgfpathlineto{\pgfqpoint{3.080827in}{2.364662in}}%
\pgfpathlineto{\pgfqpoint{3.088731in}{2.373037in}}%
\pgfpathlineto{\pgfqpoint{3.096629in}{2.381496in}}%
\pgfpathlineto{\pgfqpoint{3.104519in}{2.390038in}}%
\pgfpathlineto{\pgfqpoint{3.091512in}{2.401429in}}%
\pgfpathlineto{\pgfqpoint{3.078502in}{2.413056in}}%
\pgfpathlineto{\pgfqpoint{3.065489in}{2.424919in}}%
\pgfpathlineto{\pgfqpoint{3.052472in}{2.437020in}}%
\pgfpathlineto{\pgfqpoint{3.044565in}{2.428579in}}%
\pgfpathlineto{\pgfqpoint{3.036650in}{2.420228in}}%
\pgfpathlineto{\pgfqpoint{3.028727in}{2.411968in}}%
\pgfpathlineto{\pgfqpoint{3.020797in}{2.403800in}}%
\pgfpathclose%
\pgfusepath{fill}%
\end{pgfscope}%
\begin{pgfscope}%
\pgfpathrectangle{\pgfqpoint{1.254980in}{0.150000in}}{\pgfqpoint{5.490039in}{5.490039in}}%
\pgfusepath{clip}%
\pgfsetbuttcap%
\pgfsetroundjoin%
\definecolor{currentfill}{rgb}{0.206756,0.371758,0.553117}%
\pgfsetfillcolor{currentfill}%
\pgfsetfillopacity{0.700000}%
\pgfsetlinewidth{0.000000pt}%
\definecolor{currentstroke}{rgb}{0.000000,0.000000,0.000000}%
\pgfsetstrokecolor{currentstroke}%
\pgfsetdash{}{0pt}%
\pgfpathmoveto{\pgfqpoint{2.653501in}{2.851182in}}%
\pgfpathlineto{\pgfqpoint{2.666728in}{2.831398in}}%
\pgfpathlineto{\pgfqpoint{2.679946in}{2.811919in}}%
\pgfpathlineto{\pgfqpoint{2.693153in}{2.792743in}}%
\pgfpathlineto{\pgfqpoint{2.706350in}{2.773866in}}%
\pgfpathlineto{\pgfqpoint{2.714409in}{2.781283in}}%
\pgfpathlineto{\pgfqpoint{2.722459in}{2.788834in}}%
\pgfpathlineto{\pgfqpoint{2.730499in}{2.796518in}}%
\pgfpathlineto{\pgfqpoint{2.738530in}{2.804335in}}%
\pgfpathlineto{\pgfqpoint{2.725357in}{2.823094in}}%
\pgfpathlineto{\pgfqpoint{2.712174in}{2.842152in}}%
\pgfpathlineto{\pgfqpoint{2.698981in}{2.861511in}}%
\pgfpathlineto{\pgfqpoint{2.685778in}{2.881176in}}%
\pgfpathlineto{\pgfqpoint{2.677723in}{2.873467in}}%
\pgfpathlineto{\pgfqpoint{2.669659in}{2.865897in}}%
\pgfpathlineto{\pgfqpoint{2.661585in}{2.858469in}}%
\pgfpathlineto{\pgfqpoint{2.653501in}{2.851182in}}%
\pgfpathclose%
\pgfusepath{fill}%
\end{pgfscope}%
\begin{pgfscope}%
\pgfpathrectangle{\pgfqpoint{1.254980in}{0.150000in}}{\pgfqpoint{5.490039in}{5.490039in}}%
\pgfusepath{clip}%
\pgfsetbuttcap%
\pgfsetroundjoin%
\definecolor{currentfill}{rgb}{0.282327,0.094955,0.417331}%
\pgfsetfillcolor{currentfill}%
\pgfsetfillopacity{0.700000}%
\pgfsetlinewidth{0.000000pt}%
\definecolor{currentstroke}{rgb}{0.000000,0.000000,0.000000}%
\pgfsetstrokecolor{currentstroke}%
\pgfsetdash{}{0pt}%
\pgfpathmoveto{\pgfqpoint{3.395593in}{2.245715in}}%
\pgfpathlineto{\pgfqpoint{3.408575in}{2.238982in}}%
\pgfpathlineto{\pgfqpoint{3.421559in}{2.232452in}}%
\pgfpathlineto{\pgfqpoint{3.434545in}{2.226125in}}%
\pgfpathlineto{\pgfqpoint{3.447533in}{2.219998in}}%
\pgfpathlineto{\pgfqpoint{3.455304in}{2.229333in}}%
\pgfpathlineto{\pgfqpoint{3.463068in}{2.238708in}}%
\pgfpathlineto{\pgfqpoint{3.470827in}{2.248121in}}%
\pgfpathlineto{\pgfqpoint{3.478581in}{2.257574in}}%
\pgfpathlineto{\pgfqpoint{3.465605in}{2.263649in}}%
\pgfpathlineto{\pgfqpoint{3.452631in}{2.269924in}}%
\pgfpathlineto{\pgfqpoint{3.439659in}{2.276401in}}%
\pgfpathlineto{\pgfqpoint{3.426689in}{2.283081in}}%
\pgfpathlineto{\pgfqpoint{3.418924in}{2.273670in}}%
\pgfpathlineto{\pgfqpoint{3.411152in}{2.264306in}}%
\pgfpathlineto{\pgfqpoint{3.403375in}{2.254987in}}%
\pgfpathlineto{\pgfqpoint{3.395593in}{2.245715in}}%
\pgfpathclose%
\pgfusepath{fill}%
\end{pgfscope}%
\begin{pgfscope}%
\pgfpathrectangle{\pgfqpoint{1.254980in}{0.150000in}}{\pgfqpoint{5.490039in}{5.490039in}}%
\pgfusepath{clip}%
\pgfsetbuttcap%
\pgfsetroundjoin%
\definecolor{currentfill}{rgb}{0.283187,0.125848,0.444960}%
\pgfsetfillcolor{currentfill}%
\pgfsetfillopacity{0.700000}%
\pgfsetlinewidth{0.000000pt}%
\definecolor{currentstroke}{rgb}{0.000000,0.000000,0.000000}%
\pgfsetstrokecolor{currentstroke}%
\pgfsetdash{}{0pt}%
\pgfpathmoveto{\pgfqpoint{3.830805in}{2.293091in}}%
\pgfpathlineto{\pgfqpoint{3.843836in}{2.290999in}}%
\pgfpathlineto{\pgfqpoint{3.856874in}{2.289088in}}%
\pgfpathlineto{\pgfqpoint{3.869917in}{2.287358in}}%
\pgfpathlineto{\pgfqpoint{3.882966in}{2.285807in}}%
\pgfpathlineto{\pgfqpoint{3.890592in}{2.295507in}}%
\pgfpathlineto{\pgfqpoint{3.898212in}{2.305211in}}%
\pgfpathlineto{\pgfqpoint{3.905828in}{2.314921in}}%
\pgfpathlineto{\pgfqpoint{3.913438in}{2.324638in}}%
\pgfpathlineto{\pgfqpoint{3.900397in}{2.326248in}}%
\pgfpathlineto{\pgfqpoint{3.887363in}{2.328038in}}%
\pgfpathlineto{\pgfqpoint{3.874334in}{2.330007in}}%
\pgfpathlineto{\pgfqpoint{3.861312in}{2.332158in}}%
\pgfpathlineto{\pgfqpoint{3.853692in}{2.322372in}}%
\pgfpathlineto{\pgfqpoint{3.846068in}{2.312599in}}%
\pgfpathlineto{\pgfqpoint{3.838439in}{2.302839in}}%
\pgfpathlineto{\pgfqpoint{3.830805in}{2.293091in}}%
\pgfpathclose%
\pgfusepath{fill}%
\end{pgfscope}%
\begin{pgfscope}%
\pgfpathrectangle{\pgfqpoint{1.254980in}{0.150000in}}{\pgfqpoint{5.490039in}{5.490039in}}%
\pgfusepath{clip}%
\pgfsetbuttcap%
\pgfsetroundjoin%
\definecolor{currentfill}{rgb}{0.136408,0.541173,0.554483}%
\pgfsetfillcolor{currentfill}%
\pgfsetfillopacity{0.700000}%
\pgfsetlinewidth{0.000000pt}%
\definecolor{currentstroke}{rgb}{0.000000,0.000000,0.000000}%
\pgfsetstrokecolor{currentstroke}%
\pgfsetdash{}{0pt}%
\pgfpathmoveto{\pgfqpoint{5.647297in}{3.248675in}}%
\pgfpathlineto{\pgfqpoint{5.660978in}{3.252939in}}%
\pgfpathlineto{\pgfqpoint{5.674673in}{3.257351in}}%
\pgfpathlineto{\pgfqpoint{5.688383in}{3.261911in}}%
\pgfpathlineto{\pgfqpoint{5.702109in}{3.266619in}}%
\pgfpathlineto{\pgfqpoint{5.709085in}{3.274151in}}%
\pgfpathlineto{\pgfqpoint{5.716063in}{3.281910in}}%
\pgfpathlineto{\pgfqpoint{5.723044in}{3.289904in}}%
\pgfpathlineto{\pgfqpoint{5.709341in}{3.285704in}}%
\pgfpathlineto{\pgfqpoint{5.695654in}{3.281652in}}%
\pgfpathlineto{\pgfqpoint{5.681981in}{3.277747in}}%
\pgfpathlineto{\pgfqpoint{5.668322in}{3.273990in}}%
\pgfpathlineto{\pgfqpoint{5.661311in}{3.265313in}}%
\pgfpathlineto{\pgfqpoint{5.654303in}{3.256877in}}%
\pgfpathlineto{\pgfqpoint{5.647297in}{3.248675in}}%
\pgfpathclose%
\pgfusepath{fill}%
\end{pgfscope}%
\begin{pgfscope}%
\pgfpathrectangle{\pgfqpoint{1.254980in}{0.150000in}}{\pgfqpoint{5.490039in}{5.490039in}}%
\pgfusepath{clip}%
\pgfsetbuttcap%
\pgfsetroundjoin%
\definecolor{currentfill}{rgb}{0.283091,0.110553,0.431554}%
\pgfsetfillcolor{currentfill}%
\pgfsetfillopacity{0.700000}%
\pgfsetlinewidth{0.000000pt}%
\definecolor{currentstroke}{rgb}{0.000000,0.000000,0.000000}%
\pgfsetstrokecolor{currentstroke}%
\pgfsetdash{}{0pt}%
\pgfpathmoveto{\pgfqpoint{3.260468in}{2.271065in}}%
\pgfpathlineto{\pgfqpoint{3.273458in}{2.262581in}}%
\pgfpathlineto{\pgfqpoint{3.286448in}{2.254311in}}%
\pgfpathlineto{\pgfqpoint{3.299439in}{2.246252in}}%
\pgfpathlineto{\pgfqpoint{3.312430in}{2.238404in}}%
\pgfpathlineto{\pgfqpoint{3.320250in}{2.247398in}}%
\pgfpathlineto{\pgfqpoint{3.328065in}{2.256445in}}%
\pgfpathlineto{\pgfqpoint{3.335873in}{2.265547in}}%
\pgfpathlineto{\pgfqpoint{3.343675in}{2.274703in}}%
\pgfpathlineto{\pgfqpoint{3.330698in}{2.282470in}}%
\pgfpathlineto{\pgfqpoint{3.317722in}{2.290448in}}%
\pgfpathlineto{\pgfqpoint{3.304746in}{2.298637in}}%
\pgfpathlineto{\pgfqpoint{3.291770in}{2.307040in}}%
\pgfpathlineto{\pgfqpoint{3.283954in}{2.297955in}}%
\pgfpathlineto{\pgfqpoint{3.276132in}{2.288931in}}%
\pgfpathlineto{\pgfqpoint{3.268303in}{2.279967in}}%
\pgfpathlineto{\pgfqpoint{3.260468in}{2.271065in}}%
\pgfpathclose%
\pgfusepath{fill}%
\end{pgfscope}%
\begin{pgfscope}%
\pgfpathrectangle{\pgfqpoint{1.254980in}{0.150000in}}{\pgfqpoint{5.490039in}{5.490039in}}%
\pgfusepath{clip}%
\pgfsetbuttcap%
\pgfsetroundjoin%
\definecolor{currentfill}{rgb}{0.281887,0.150881,0.465405}%
\pgfsetfillcolor{currentfill}%
\pgfsetfillopacity{0.700000}%
\pgfsetlinewidth{0.000000pt}%
\definecolor{currentstroke}{rgb}{0.000000,0.000000,0.000000}%
\pgfsetstrokecolor{currentstroke}%
\pgfsetdash{}{0pt}%
\pgfpathmoveto{\pgfqpoint{3.072915in}{2.356372in}}%
\pgfpathlineto{\pgfqpoint{3.085937in}{2.345103in}}%
\pgfpathlineto{\pgfqpoint{3.098957in}{2.334066in}}%
\pgfpathlineto{\pgfqpoint{3.111975in}{2.323259in}}%
\pgfpathlineto{\pgfqpoint{3.124990in}{2.312680in}}%
\pgfpathlineto{\pgfqpoint{3.132884in}{2.321091in}}%
\pgfpathlineto{\pgfqpoint{3.140771in}{2.329579in}}%
\pgfpathlineto{\pgfqpoint{3.148651in}{2.338145in}}%
\pgfpathlineto{\pgfqpoint{3.156525in}{2.346787in}}%
\pgfpathlineto{\pgfqpoint{3.143526in}{2.357256in}}%
\pgfpathlineto{\pgfqpoint{3.130526in}{2.367953in}}%
\pgfpathlineto{\pgfqpoint{3.117524in}{2.378879in}}%
\pgfpathlineto{\pgfqpoint{3.104519in}{2.390038in}}%
\pgfpathlineto{\pgfqpoint{3.096629in}{2.381496in}}%
\pgfpathlineto{\pgfqpoint{3.088731in}{2.373037in}}%
\pgfpathlineto{\pgfqpoint{3.080827in}{2.364662in}}%
\pgfpathlineto{\pgfqpoint{3.072915in}{2.356372in}}%
\pgfpathclose%
\pgfusepath{fill}%
\end{pgfscope}%
\begin{pgfscope}%
\pgfpathrectangle{\pgfqpoint{1.254980in}{0.150000in}}{\pgfqpoint{5.490039in}{5.490039in}}%
\pgfusepath{clip}%
\pgfsetbuttcap%
\pgfsetroundjoin%
\definecolor{currentfill}{rgb}{0.282327,0.094955,0.417331}%
\pgfsetfillcolor{currentfill}%
\pgfsetfillopacity{0.700000}%
\pgfsetlinewidth{0.000000pt}%
\definecolor{currentstroke}{rgb}{0.000000,0.000000,0.000000}%
\pgfsetstrokecolor{currentstroke}%
\pgfsetdash{}{0pt}%
\pgfpathmoveto{\pgfqpoint{3.530509in}{2.235264in}}%
\pgfpathlineto{\pgfqpoint{3.543498in}{2.230179in}}%
\pgfpathlineto{\pgfqpoint{3.556490in}{2.225287in}}%
\pgfpathlineto{\pgfqpoint{3.569486in}{2.220590in}}%
\pgfpathlineto{\pgfqpoint{3.582485in}{2.216085in}}%
\pgfpathlineto{\pgfqpoint{3.590210in}{2.225646in}}%
\pgfpathlineto{\pgfqpoint{3.597930in}{2.235234in}}%
\pgfpathlineto{\pgfqpoint{3.605644in}{2.244847in}}%
\pgfpathlineto{\pgfqpoint{3.613353in}{2.254485in}}%
\pgfpathlineto{\pgfqpoint{3.600365in}{2.258966in}}%
\pgfpathlineto{\pgfqpoint{3.587380in}{2.263640in}}%
\pgfpathlineto{\pgfqpoint{3.574399in}{2.268506in}}%
\pgfpathlineto{\pgfqpoint{3.561421in}{2.273568in}}%
\pgfpathlineto{\pgfqpoint{3.553701in}{2.263943in}}%
\pgfpathlineto{\pgfqpoint{3.545976in}{2.254351in}}%
\pgfpathlineto{\pgfqpoint{3.538245in}{2.244792in}}%
\pgfpathlineto{\pgfqpoint{3.530509in}{2.235264in}}%
\pgfpathclose%
\pgfusepath{fill}%
\end{pgfscope}%
\begin{pgfscope}%
\pgfpathrectangle{\pgfqpoint{1.254980in}{0.150000in}}{\pgfqpoint{5.490039in}{5.490039in}}%
\pgfusepath{clip}%
\pgfsetbuttcap%
\pgfsetroundjoin%
\definecolor{currentfill}{rgb}{0.283197,0.115680,0.436115}%
\pgfsetfillcolor{currentfill}%
\pgfsetfillopacity{0.700000}%
\pgfsetlinewidth{0.000000pt}%
\definecolor{currentstroke}{rgb}{0.000000,0.000000,0.000000}%
\pgfsetstrokecolor{currentstroke}%
\pgfsetdash{}{0pt}%
\pgfpathmoveto{\pgfqpoint{3.748113in}{2.264257in}}%
\pgfpathlineto{\pgfqpoint{3.761131in}{2.261466in}}%
\pgfpathlineto{\pgfqpoint{3.774155in}{2.258859in}}%
\pgfpathlineto{\pgfqpoint{3.787184in}{2.256434in}}%
\pgfpathlineto{\pgfqpoint{3.800218in}{2.254193in}}%
\pgfpathlineto{\pgfqpoint{3.807872in}{2.263904in}}%
\pgfpathlineto{\pgfqpoint{3.815522in}{2.273624in}}%
\pgfpathlineto{\pgfqpoint{3.823166in}{2.283352in}}%
\pgfpathlineto{\pgfqpoint{3.830805in}{2.293091in}}%
\pgfpathlineto{\pgfqpoint{3.817780in}{2.295364in}}%
\pgfpathlineto{\pgfqpoint{3.804760in}{2.297819in}}%
\pgfpathlineto{\pgfqpoint{3.791745in}{2.300458in}}%
\pgfpathlineto{\pgfqpoint{3.778736in}{2.303281in}}%
\pgfpathlineto{\pgfqpoint{3.771088in}{2.293501in}}%
\pgfpathlineto{\pgfqpoint{3.763435in}{2.283737in}}%
\pgfpathlineto{\pgfqpoint{3.755776in}{2.273990in}}%
\pgfpathlineto{\pgfqpoint{3.748113in}{2.264257in}}%
\pgfpathclose%
\pgfusepath{fill}%
\end{pgfscope}%
\begin{pgfscope}%
\pgfpathrectangle{\pgfqpoint{1.254980in}{0.150000in}}{\pgfqpoint{5.490039in}{5.490039in}}%
\pgfusepath{clip}%
\pgfsetbuttcap%
\pgfsetroundjoin%
\definecolor{currentfill}{rgb}{0.283072,0.130895,0.449241}%
\pgfsetfillcolor{currentfill}%
\pgfsetfillopacity{0.700000}%
\pgfsetlinewidth{0.000000pt}%
\definecolor{currentstroke}{rgb}{0.000000,0.000000,0.000000}%
\pgfsetstrokecolor{currentstroke}%
\pgfsetdash{}{0pt}%
\pgfpathmoveto{\pgfqpoint{3.124990in}{2.312680in}}%
\pgfpathlineto{\pgfqpoint{3.138003in}{2.302328in}}%
\pgfpathlineto{\pgfqpoint{3.151015in}{2.292201in}}%
\pgfpathlineto{\pgfqpoint{3.164026in}{2.282297in}}%
\pgfpathlineto{\pgfqpoint{3.177035in}{2.272616in}}%
\pgfpathlineto{\pgfqpoint{3.184912in}{2.281147in}}%
\pgfpathlineto{\pgfqpoint{3.192783in}{2.289749in}}%
\pgfpathlineto{\pgfqpoint{3.200646in}{2.298421in}}%
\pgfpathlineto{\pgfqpoint{3.208503in}{2.307162in}}%
\pgfpathlineto{\pgfqpoint{3.195511in}{2.316734in}}%
\pgfpathlineto{\pgfqpoint{3.182517in}{2.326528in}}%
\pgfpathlineto{\pgfqpoint{3.169521in}{2.336545in}}%
\pgfpathlineto{\pgfqpoint{3.156525in}{2.346787in}}%
\pgfpathlineto{\pgfqpoint{3.148651in}{2.338145in}}%
\pgfpathlineto{\pgfqpoint{3.140771in}{2.329579in}}%
\pgfpathlineto{\pgfqpoint{3.132884in}{2.321091in}}%
\pgfpathlineto{\pgfqpoint{3.124990in}{2.312680in}}%
\pgfpathclose%
\pgfusepath{fill}%
\end{pgfscope}%
\begin{pgfscope}%
\pgfpathrectangle{\pgfqpoint{1.254980in}{0.150000in}}{\pgfqpoint{5.490039in}{5.490039in}}%
\pgfusepath{clip}%
\pgfsetbuttcap%
\pgfsetroundjoin%
\definecolor{currentfill}{rgb}{0.252194,0.269783,0.531579}%
\pgfsetfillcolor{currentfill}%
\pgfsetfillopacity{0.700000}%
\pgfsetlinewidth{0.000000pt}%
\definecolor{currentstroke}{rgb}{0.000000,0.000000,0.000000}%
\pgfsetstrokecolor{currentstroke}%
\pgfsetdash{}{0pt}%
\pgfpathmoveto{\pgfqpoint{4.461473in}{2.562955in}}%
\pgfpathlineto{\pgfqpoint{4.474707in}{2.565355in}}%
\pgfpathlineto{\pgfqpoint{4.487950in}{2.567918in}}%
\pgfpathlineto{\pgfqpoint{4.501205in}{2.570644in}}%
\pgfpathlineto{\pgfqpoint{4.514470in}{2.573534in}}%
\pgfpathlineto{\pgfqpoint{4.521885in}{2.582036in}}%
\pgfpathlineto{\pgfqpoint{4.529296in}{2.590544in}}%
\pgfpathlineto{\pgfqpoint{4.536702in}{2.599061in}}%
\pgfpathlineto{\pgfqpoint{4.544103in}{2.607590in}}%
\pgfpathlineto{\pgfqpoint{4.530848in}{2.604957in}}%
\pgfpathlineto{\pgfqpoint{4.517604in}{2.602487in}}%
\pgfpathlineto{\pgfqpoint{4.504370in}{2.600180in}}%
\pgfpathlineto{\pgfqpoint{4.491147in}{2.598037in}}%
\pgfpathlineto{\pgfqpoint{4.483736in}{2.589241in}}%
\pgfpathlineto{\pgfqpoint{4.476320in}{2.580465in}}%
\pgfpathlineto{\pgfqpoint{4.468899in}{2.571704in}}%
\pgfpathlineto{\pgfqpoint{4.461473in}{2.562955in}}%
\pgfpathclose%
\pgfusepath{fill}%
\end{pgfscope}%
\begin{pgfscope}%
\pgfpathrectangle{\pgfqpoint{1.254980in}{0.150000in}}{\pgfqpoint{5.490039in}{5.490039in}}%
\pgfusepath{clip}%
\pgfsetbuttcap%
\pgfsetroundjoin%
\definecolor{currentfill}{rgb}{0.260571,0.246922,0.522828}%
\pgfsetfillcolor{currentfill}%
\pgfsetfillopacity{0.700000}%
\pgfsetlinewidth{0.000000pt}%
\definecolor{currentstroke}{rgb}{0.000000,0.000000,0.000000}%
\pgfsetstrokecolor{currentstroke}%
\pgfsetdash{}{0pt}%
\pgfpathmoveto{\pgfqpoint{4.378850in}{2.519164in}}%
\pgfpathlineto{\pgfqpoint{4.392052in}{2.521132in}}%
\pgfpathlineto{\pgfqpoint{4.405265in}{2.523266in}}%
\pgfpathlineto{\pgfqpoint{4.418487in}{2.525565in}}%
\pgfpathlineto{\pgfqpoint{4.431720in}{2.528028in}}%
\pgfpathlineto{\pgfqpoint{4.439166in}{2.536756in}}%
\pgfpathlineto{\pgfqpoint{4.446607in}{2.545484in}}%
\pgfpathlineto{\pgfqpoint{4.454042in}{2.554216in}}%
\pgfpathlineto{\pgfqpoint{4.461473in}{2.562955in}}%
\pgfpathlineto{\pgfqpoint{4.448250in}{2.560720in}}%
\pgfpathlineto{\pgfqpoint{4.435037in}{2.558650in}}%
\pgfpathlineto{\pgfqpoint{4.421834in}{2.556744in}}%
\pgfpathlineto{\pgfqpoint{4.408641in}{2.555003in}}%
\pgfpathlineto{\pgfqpoint{4.401201in}{2.546026in}}%
\pgfpathlineto{\pgfqpoint{4.393756in}{2.537062in}}%
\pgfpathlineto{\pgfqpoint{4.386305in}{2.528109in}}%
\pgfpathlineto{\pgfqpoint{4.378850in}{2.519164in}}%
\pgfpathclose%
\pgfusepath{fill}%
\end{pgfscope}%
\begin{pgfscope}%
\pgfpathrectangle{\pgfqpoint{1.254980in}{0.150000in}}{\pgfqpoint{5.490039in}{5.490039in}}%
\pgfusepath{clip}%
\pgfsetbuttcap%
\pgfsetroundjoin%
\definecolor{currentfill}{rgb}{0.244972,0.287675,0.537260}%
\pgfsetfillcolor{currentfill}%
\pgfsetfillopacity{0.700000}%
\pgfsetlinewidth{0.000000pt}%
\definecolor{currentstroke}{rgb}{0.000000,0.000000,0.000000}%
\pgfsetstrokecolor{currentstroke}%
\pgfsetdash{}{0pt}%
\pgfpathmoveto{\pgfqpoint{4.544103in}{2.607590in}}%
\pgfpathlineto{\pgfqpoint{4.557368in}{2.610385in}}%
\pgfpathlineto{\pgfqpoint{4.570644in}{2.613343in}}%
\pgfpathlineto{\pgfqpoint{4.583932in}{2.616463in}}%
\pgfpathlineto{\pgfqpoint{4.597230in}{2.619744in}}%
\pgfpathlineto{\pgfqpoint{4.604615in}{2.628014in}}%
\pgfpathlineto{\pgfqpoint{4.611995in}{2.636294in}}%
\pgfpathlineto{\pgfqpoint{4.619370in}{2.644590in}}%
\pgfpathlineto{\pgfqpoint{4.626741in}{2.652904in}}%
\pgfpathlineto{\pgfqpoint{4.613453in}{2.649908in}}%
\pgfpathlineto{\pgfqpoint{4.600177in}{2.647073in}}%
\pgfpathlineto{\pgfqpoint{4.586912in}{2.644400in}}%
\pgfpathlineto{\pgfqpoint{4.573657in}{2.641889in}}%
\pgfpathlineto{\pgfqpoint{4.566276in}{2.633280in}}%
\pgfpathlineto{\pgfqpoint{4.558889in}{2.624696in}}%
\pgfpathlineto{\pgfqpoint{4.551499in}{2.616133in}}%
\pgfpathlineto{\pgfqpoint{4.544103in}{2.607590in}}%
\pgfpathclose%
\pgfusepath{fill}%
\end{pgfscope}%
\begin{pgfscope}%
\pgfpathrectangle{\pgfqpoint{1.254980in}{0.150000in}}{\pgfqpoint{5.490039in}{5.490039in}}%
\pgfusepath{clip}%
\pgfsetbuttcap%
\pgfsetroundjoin%
\definecolor{currentfill}{rgb}{0.266580,0.228262,0.514349}%
\pgfsetfillcolor{currentfill}%
\pgfsetfillopacity{0.700000}%
\pgfsetlinewidth{0.000000pt}%
\definecolor{currentstroke}{rgb}{0.000000,0.000000,0.000000}%
\pgfsetstrokecolor{currentstroke}%
\pgfsetdash{}{0pt}%
\pgfpathmoveto{\pgfqpoint{4.296230in}{2.476399in}}%
\pgfpathlineto{\pgfqpoint{4.309403in}{2.477901in}}%
\pgfpathlineto{\pgfqpoint{4.322585in}{2.479571in}}%
\pgfpathlineto{\pgfqpoint{4.335777in}{2.481407in}}%
\pgfpathlineto{\pgfqpoint{4.348979in}{2.483409in}}%
\pgfpathlineto{\pgfqpoint{4.356454in}{2.492349in}}%
\pgfpathlineto{\pgfqpoint{4.363924in}{2.501287in}}%
\pgfpathlineto{\pgfqpoint{4.371390in}{2.510224in}}%
\pgfpathlineto{\pgfqpoint{4.378850in}{2.519164in}}%
\pgfpathlineto{\pgfqpoint{4.365657in}{2.517362in}}%
\pgfpathlineto{\pgfqpoint{4.352474in}{2.515725in}}%
\pgfpathlineto{\pgfqpoint{4.339301in}{2.514256in}}%
\pgfpathlineto{\pgfqpoint{4.326137in}{2.512953in}}%
\pgfpathlineto{\pgfqpoint{4.318667in}{2.503803in}}%
\pgfpathlineto{\pgfqpoint{4.311193in}{2.494662in}}%
\pgfpathlineto{\pgfqpoint{4.303714in}{2.485528in}}%
\pgfpathlineto{\pgfqpoint{4.296230in}{2.476399in}}%
\pgfpathclose%
\pgfusepath{fill}%
\end{pgfscope}%
\begin{pgfscope}%
\pgfpathrectangle{\pgfqpoint{1.254980in}{0.150000in}}{\pgfqpoint{5.490039in}{5.490039in}}%
\pgfusepath{clip}%
\pgfsetbuttcap%
\pgfsetroundjoin%
\definecolor{currentfill}{rgb}{0.235526,0.309527,0.542944}%
\pgfsetfillcolor{currentfill}%
\pgfsetfillopacity{0.700000}%
\pgfsetlinewidth{0.000000pt}%
\definecolor{currentstroke}{rgb}{0.000000,0.000000,0.000000}%
\pgfsetstrokecolor{currentstroke}%
\pgfsetdash{}{0pt}%
\pgfpathmoveto{\pgfqpoint{4.626741in}{2.652904in}}%
\pgfpathlineto{\pgfqpoint{4.640039in}{2.656061in}}%
\pgfpathlineto{\pgfqpoint{4.653349in}{2.659380in}}%
\pgfpathlineto{\pgfqpoint{4.666670in}{2.662859in}}%
\pgfpathlineto{\pgfqpoint{4.680003in}{2.666498in}}%
\pgfpathlineto{\pgfqpoint{4.687357in}{2.674531in}}%
\pgfpathlineto{\pgfqpoint{4.694706in}{2.682582in}}%
\pgfpathlineto{\pgfqpoint{4.702049in}{2.690656in}}%
\pgfpathlineto{\pgfqpoint{4.709389in}{2.698756in}}%
\pgfpathlineto{\pgfqpoint{4.696068in}{2.695430in}}%
\pgfpathlineto{\pgfqpoint{4.682758in}{2.692265in}}%
\pgfpathlineto{\pgfqpoint{4.669460in}{2.689260in}}%
\pgfpathlineto{\pgfqpoint{4.656173in}{2.686415in}}%
\pgfpathlineto{\pgfqpoint{4.648822in}{2.677992in}}%
\pgfpathlineto{\pgfqpoint{4.641466in}{2.669601in}}%
\pgfpathlineto{\pgfqpoint{4.634106in}{2.661240in}}%
\pgfpathlineto{\pgfqpoint{4.626741in}{2.652904in}}%
\pgfpathclose%
\pgfusepath{fill}%
\end{pgfscope}%
\begin{pgfscope}%
\pgfpathrectangle{\pgfqpoint{1.254980in}{0.150000in}}{\pgfqpoint{5.490039in}{5.490039in}}%
\pgfusepath{clip}%
\pgfsetbuttcap%
\pgfsetroundjoin%
\definecolor{currentfill}{rgb}{0.271828,0.209303,0.504434}%
\pgfsetfillcolor{currentfill}%
\pgfsetfillopacity{0.700000}%
\pgfsetlinewidth{0.000000pt}%
\definecolor{currentstroke}{rgb}{0.000000,0.000000,0.000000}%
\pgfsetstrokecolor{currentstroke}%
\pgfsetdash{}{0pt}%
\pgfpathmoveto{\pgfqpoint{4.213608in}{2.434864in}}%
\pgfpathlineto{\pgfqpoint{4.226753in}{2.435864in}}%
\pgfpathlineto{\pgfqpoint{4.239907in}{2.437034in}}%
\pgfpathlineto{\pgfqpoint{4.253070in}{2.438372in}}%
\pgfpathlineto{\pgfqpoint{4.266243in}{2.439879in}}%
\pgfpathlineto{\pgfqpoint{4.273747in}{2.449014in}}%
\pgfpathlineto{\pgfqpoint{4.281246in}{2.458144in}}%
\pgfpathlineto{\pgfqpoint{4.288740in}{2.467272in}}%
\pgfpathlineto{\pgfqpoint{4.296230in}{2.476399in}}%
\pgfpathlineto{\pgfqpoint{4.283066in}{2.475064in}}%
\pgfpathlineto{\pgfqpoint{4.269911in}{2.473898in}}%
\pgfpathlineto{\pgfqpoint{4.256766in}{2.472900in}}%
\pgfpathlineto{\pgfqpoint{4.243629in}{2.472071in}}%
\pgfpathlineto{\pgfqpoint{4.236131in}{2.462761in}}%
\pgfpathlineto{\pgfqpoint{4.228628in}{2.453459in}}%
\pgfpathlineto{\pgfqpoint{4.221120in}{2.444160in}}%
\pgfpathlineto{\pgfqpoint{4.213608in}{2.434864in}}%
\pgfpathclose%
\pgfusepath{fill}%
\end{pgfscope}%
\begin{pgfscope}%
\pgfpathrectangle{\pgfqpoint{1.254980in}{0.150000in}}{\pgfqpoint{5.490039in}{5.490039in}}%
\pgfusepath{clip}%
\pgfsetbuttcap%
\pgfsetroundjoin%
\definecolor{currentfill}{rgb}{0.225863,0.330805,0.547314}%
\pgfsetfillcolor{currentfill}%
\pgfsetfillopacity{0.700000}%
\pgfsetlinewidth{0.000000pt}%
\definecolor{currentstroke}{rgb}{0.000000,0.000000,0.000000}%
\pgfsetstrokecolor{currentstroke}%
\pgfsetdash{}{0pt}%
\pgfpathmoveto{\pgfqpoint{4.709389in}{2.698756in}}%
\pgfpathlineto{\pgfqpoint{4.722721in}{2.702241in}}%
\pgfpathlineto{\pgfqpoint{4.736065in}{2.705886in}}%
\pgfpathlineto{\pgfqpoint{4.749421in}{2.709690in}}%
\pgfpathlineto{\pgfqpoint{4.762789in}{2.713654in}}%
\pgfpathlineto{\pgfqpoint{4.770111in}{2.721452in}}%
\pgfpathlineto{\pgfqpoint{4.777428in}{2.729276in}}%
\pgfpathlineto{\pgfqpoint{4.784740in}{2.737132in}}%
\pgfpathlineto{\pgfqpoint{4.792047in}{2.745022in}}%
\pgfpathlineto{\pgfqpoint{4.778692in}{2.741401in}}%
\pgfpathlineto{\pgfqpoint{4.765349in}{2.737939in}}%
\pgfpathlineto{\pgfqpoint{4.752018in}{2.734636in}}%
\pgfpathlineto{\pgfqpoint{4.738698in}{2.731492in}}%
\pgfpathlineto{\pgfqpoint{4.731377in}{2.723250in}}%
\pgfpathlineto{\pgfqpoint{4.724052in}{2.715049in}}%
\pgfpathlineto{\pgfqpoint{4.716723in}{2.706885in}}%
\pgfpathlineto{\pgfqpoint{4.709389in}{2.698756in}}%
\pgfpathclose%
\pgfusepath{fill}%
\end{pgfscope}%
\begin{pgfscope}%
\pgfpathrectangle{\pgfqpoint{1.254980in}{0.150000in}}{\pgfqpoint{5.490039in}{5.490039in}}%
\pgfusepath{clip}%
\pgfsetbuttcap%
\pgfsetroundjoin%
\definecolor{currentfill}{rgb}{0.282656,0.100196,0.422160}%
\pgfsetfillcolor{currentfill}%
\pgfsetfillopacity{0.700000}%
\pgfsetlinewidth{0.000000pt}%
\definecolor{currentstroke}{rgb}{0.000000,0.000000,0.000000}%
\pgfsetstrokecolor{currentstroke}%
\pgfsetdash{}{0pt}%
\pgfpathmoveto{\pgfqpoint{3.665346in}{2.238470in}}%
\pgfpathlineto{\pgfqpoint{3.678354in}{2.234939in}}%
\pgfpathlineto{\pgfqpoint{3.691367in}{2.231594in}}%
\pgfpathlineto{\pgfqpoint{3.704385in}{2.228437in}}%
\pgfpathlineto{\pgfqpoint{3.717408in}{2.225465in}}%
\pgfpathlineto{\pgfqpoint{3.725092in}{2.235143in}}%
\pgfpathlineto{\pgfqpoint{3.732771in}{2.244834in}}%
\pgfpathlineto{\pgfqpoint{3.740444in}{2.254539in}}%
\pgfpathlineto{\pgfqpoint{3.748113in}{2.264257in}}%
\pgfpathlineto{\pgfqpoint{3.735100in}{2.267233in}}%
\pgfpathlineto{\pgfqpoint{3.722091in}{2.270395in}}%
\pgfpathlineto{\pgfqpoint{3.709088in}{2.273742in}}%
\pgfpathlineto{\pgfqpoint{3.696089in}{2.277278in}}%
\pgfpathlineto{\pgfqpoint{3.688411in}{2.267545in}}%
\pgfpathlineto{\pgfqpoint{3.680728in}{2.257833in}}%
\pgfpathlineto{\pgfqpoint{3.673039in}{2.248142in}}%
\pgfpathlineto{\pgfqpoint{3.665346in}{2.238470in}}%
\pgfpathclose%
\pgfusepath{fill}%
\end{pgfscope}%
\begin{pgfscope}%
\pgfpathrectangle{\pgfqpoint{1.254980in}{0.150000in}}{\pgfqpoint{5.490039in}{5.490039in}}%
\pgfusepath{clip}%
\pgfsetbuttcap%
\pgfsetroundjoin%
\definecolor{currentfill}{rgb}{0.216210,0.351535,0.550627}%
\pgfsetfillcolor{currentfill}%
\pgfsetfillopacity{0.700000}%
\pgfsetlinewidth{0.000000pt}%
\definecolor{currentstroke}{rgb}{0.000000,0.000000,0.000000}%
\pgfsetstrokecolor{currentstroke}%
\pgfsetdash{}{0pt}%
\pgfpathmoveto{\pgfqpoint{4.792047in}{2.745022in}}%
\pgfpathlineto{\pgfqpoint{4.805414in}{2.748802in}}%
\pgfpathlineto{\pgfqpoint{4.818794in}{2.752740in}}%
\pgfpathlineto{\pgfqpoint{4.832185in}{2.756836in}}%
\pgfpathlineto{\pgfqpoint{4.845589in}{2.761090in}}%
\pgfpathlineto{\pgfqpoint{4.852878in}{2.768659in}}%
\pgfpathlineto{\pgfqpoint{4.860162in}{2.776264in}}%
\pgfpathlineto{\pgfqpoint{4.867442in}{2.783910in}}%
\pgfpathlineto{\pgfqpoint{4.874717in}{2.791601in}}%
\pgfpathlineto{\pgfqpoint{4.861328in}{2.787719in}}%
\pgfpathlineto{\pgfqpoint{4.847950in}{2.783993in}}%
\pgfpathlineto{\pgfqpoint{4.834585in}{2.780426in}}%
\pgfpathlineto{\pgfqpoint{4.821232in}{2.777016in}}%
\pgfpathlineto{\pgfqpoint{4.813942in}{2.768944in}}%
\pgfpathlineto{\pgfqpoint{4.806648in}{2.760924in}}%
\pgfpathlineto{\pgfqpoint{4.799350in}{2.752952in}}%
\pgfpathlineto{\pgfqpoint{4.792047in}{2.745022in}}%
\pgfpathclose%
\pgfusepath{fill}%
\end{pgfscope}%
\begin{pgfscope}%
\pgfpathrectangle{\pgfqpoint{1.254980in}{0.150000in}}{\pgfqpoint{5.490039in}{5.490039in}}%
\pgfusepath{clip}%
\pgfsetbuttcap%
\pgfsetroundjoin%
\definecolor{currentfill}{rgb}{0.282656,0.100196,0.422160}%
\pgfsetfillcolor{currentfill}%
\pgfsetfillopacity{0.700000}%
\pgfsetlinewidth{0.000000pt}%
\definecolor{currentstroke}{rgb}{0.000000,0.000000,0.000000}%
\pgfsetstrokecolor{currentstroke}%
\pgfsetdash{}{0pt}%
\pgfpathmoveto{\pgfqpoint{3.312430in}{2.238404in}}%
\pgfpathlineto{\pgfqpoint{3.325421in}{2.230766in}}%
\pgfpathlineto{\pgfqpoint{3.338414in}{2.223336in}}%
\pgfpathlineto{\pgfqpoint{3.351407in}{2.216112in}}%
\pgfpathlineto{\pgfqpoint{3.364401in}{2.209095in}}%
\pgfpathlineto{\pgfqpoint{3.372208in}{2.218179in}}%
\pgfpathlineto{\pgfqpoint{3.380009in}{2.227310in}}%
\pgfpathlineto{\pgfqpoint{3.387804in}{2.236489in}}%
\pgfpathlineto{\pgfqpoint{3.395593in}{2.245715in}}%
\pgfpathlineto{\pgfqpoint{3.382611in}{2.252652in}}%
\pgfpathlineto{\pgfqpoint{3.369632in}{2.259795in}}%
\pgfpathlineto{\pgfqpoint{3.356653in}{2.267145in}}%
\pgfpathlineto{\pgfqpoint{3.343675in}{2.274703in}}%
\pgfpathlineto{\pgfqpoint{3.335873in}{2.265547in}}%
\pgfpathlineto{\pgfqpoint{3.328065in}{2.256445in}}%
\pgfpathlineto{\pgfqpoint{3.320250in}{2.247398in}}%
\pgfpathlineto{\pgfqpoint{3.312430in}{2.238404in}}%
\pgfpathclose%
\pgfusepath{fill}%
\end{pgfscope}%
\begin{pgfscope}%
\pgfpathrectangle{\pgfqpoint{1.254980in}{0.150000in}}{\pgfqpoint{5.490039in}{5.490039in}}%
\pgfusepath{clip}%
\pgfsetbuttcap%
\pgfsetroundjoin%
\definecolor{currentfill}{rgb}{0.277134,0.185228,0.489898}%
\pgfsetfillcolor{currentfill}%
\pgfsetfillopacity{0.700000}%
\pgfsetlinewidth{0.000000pt}%
\definecolor{currentstroke}{rgb}{0.000000,0.000000,0.000000}%
\pgfsetstrokecolor{currentstroke}%
\pgfsetdash{}{0pt}%
\pgfpathmoveto{\pgfqpoint{4.130979in}{2.394782in}}%
\pgfpathlineto{\pgfqpoint{4.144098in}{2.395245in}}%
\pgfpathlineto{\pgfqpoint{4.157226in}{2.395879in}}%
\pgfpathlineto{\pgfqpoint{4.170362in}{2.396684in}}%
\pgfpathlineto{\pgfqpoint{4.183507in}{2.397659in}}%
\pgfpathlineto{\pgfqpoint{4.191040in}{2.406967in}}%
\pgfpathlineto{\pgfqpoint{4.198567in}{2.416269in}}%
\pgfpathlineto{\pgfqpoint{4.206090in}{2.425567in}}%
\pgfpathlineto{\pgfqpoint{4.213608in}{2.434864in}}%
\pgfpathlineto{\pgfqpoint{4.200471in}{2.434033in}}%
\pgfpathlineto{\pgfqpoint{4.187343in}{2.433372in}}%
\pgfpathlineto{\pgfqpoint{4.174224in}{2.432881in}}%
\pgfpathlineto{\pgfqpoint{4.161113in}{2.432561in}}%
\pgfpathlineto{\pgfqpoint{4.153587in}{2.423111in}}%
\pgfpathlineto{\pgfqpoint{4.146056in}{2.413666in}}%
\pgfpathlineto{\pgfqpoint{4.138520in}{2.404223in}}%
\pgfpathlineto{\pgfqpoint{4.130979in}{2.394782in}}%
\pgfpathclose%
\pgfusepath{fill}%
\end{pgfscope}%
\begin{pgfscope}%
\pgfpathrectangle{\pgfqpoint{1.254980in}{0.150000in}}{\pgfqpoint{5.490039in}{5.490039in}}%
\pgfusepath{clip}%
\pgfsetbuttcap%
\pgfsetroundjoin%
\definecolor{currentfill}{rgb}{0.281924,0.089666,0.412415}%
\pgfsetfillcolor{currentfill}%
\pgfsetfillopacity{0.700000}%
\pgfsetlinewidth{0.000000pt}%
\definecolor{currentstroke}{rgb}{0.000000,0.000000,0.000000}%
\pgfsetstrokecolor{currentstroke}%
\pgfsetdash{}{0pt}%
\pgfpathmoveto{\pgfqpoint{3.447533in}{2.219998in}}%
\pgfpathlineto{\pgfqpoint{3.460524in}{2.214071in}}%
\pgfpathlineto{\pgfqpoint{3.473516in}{2.208344in}}%
\pgfpathlineto{\pgfqpoint{3.486511in}{2.202813in}}%
\pgfpathlineto{\pgfqpoint{3.499509in}{2.197480in}}%
\pgfpathlineto{\pgfqpoint{3.507267in}{2.206877in}}%
\pgfpathlineto{\pgfqpoint{3.515020in}{2.216307in}}%
\pgfpathlineto{\pgfqpoint{3.522767in}{2.225770in}}%
\pgfpathlineto{\pgfqpoint{3.530509in}{2.235264in}}%
\pgfpathlineto{\pgfqpoint{3.517523in}{2.240546in}}%
\pgfpathlineto{\pgfqpoint{3.504539in}{2.246024in}}%
\pgfpathlineto{\pgfqpoint{3.491559in}{2.251700in}}%
\pgfpathlineto{\pgfqpoint{3.478581in}{2.257574in}}%
\pgfpathlineto{\pgfqpoint{3.470827in}{2.248121in}}%
\pgfpathlineto{\pgfqpoint{3.463068in}{2.238708in}}%
\pgfpathlineto{\pgfqpoint{3.455304in}{2.229333in}}%
\pgfpathlineto{\pgfqpoint{3.447533in}{2.219998in}}%
\pgfpathclose%
\pgfusepath{fill}%
\end{pgfscope}%
\begin{pgfscope}%
\pgfpathrectangle{\pgfqpoint{1.254980in}{0.150000in}}{\pgfqpoint{5.490039in}{5.490039in}}%
\pgfusepath{clip}%
\pgfsetbuttcap%
\pgfsetroundjoin%
\definecolor{currentfill}{rgb}{0.206756,0.371758,0.553117}%
\pgfsetfillcolor{currentfill}%
\pgfsetfillopacity{0.700000}%
\pgfsetlinewidth{0.000000pt}%
\definecolor{currentstroke}{rgb}{0.000000,0.000000,0.000000}%
\pgfsetstrokecolor{currentstroke}%
\pgfsetdash{}{0pt}%
\pgfpathmoveto{\pgfqpoint{4.874717in}{2.791601in}}%
\pgfpathlineto{\pgfqpoint{4.888119in}{2.795642in}}%
\pgfpathlineto{\pgfqpoint{4.901534in}{2.799839in}}%
\pgfpathlineto{\pgfqpoint{4.914961in}{2.804193in}}%
\pgfpathlineto{\pgfqpoint{4.928401in}{2.808704in}}%
\pgfpathlineto{\pgfqpoint{4.935657in}{2.816056in}}%
\pgfpathlineto{\pgfqpoint{4.942909in}{2.823454in}}%
\pgfpathlineto{\pgfqpoint{4.950156in}{2.830904in}}%
\pgfpathlineto{\pgfqpoint{4.957399in}{2.838411in}}%
\pgfpathlineto{\pgfqpoint{4.943974in}{2.834299in}}%
\pgfpathlineto{\pgfqpoint{4.930562in}{2.830344in}}%
\pgfpathlineto{\pgfqpoint{4.917163in}{2.826546in}}%
\pgfpathlineto{\pgfqpoint{4.903776in}{2.822904in}}%
\pgfpathlineto{\pgfqpoint{4.896517in}{2.814988in}}%
\pgfpathlineto{\pgfqpoint{4.889255in}{2.807136in}}%
\pgfpathlineto{\pgfqpoint{4.881988in}{2.799342in}}%
\pgfpathlineto{\pgfqpoint{4.874717in}{2.791601in}}%
\pgfpathclose%
\pgfusepath{fill}%
\end{pgfscope}%
\begin{pgfscope}%
\pgfpathrectangle{\pgfqpoint{1.254980in}{0.150000in}}{\pgfqpoint{5.490039in}{5.490039in}}%
\pgfusepath{clip}%
\pgfsetbuttcap%
\pgfsetroundjoin%
\definecolor{currentfill}{rgb}{0.197636,0.391528,0.554969}%
\pgfsetfillcolor{currentfill}%
\pgfsetfillopacity{0.700000}%
\pgfsetlinewidth{0.000000pt}%
\definecolor{currentstroke}{rgb}{0.000000,0.000000,0.000000}%
\pgfsetstrokecolor{currentstroke}%
\pgfsetdash{}{0pt}%
\pgfpathmoveto{\pgfqpoint{4.957399in}{2.838411in}}%
\pgfpathlineto{\pgfqpoint{4.970836in}{2.842678in}}%
\pgfpathlineto{\pgfqpoint{4.984286in}{2.847102in}}%
\pgfpathlineto{\pgfqpoint{4.997749in}{2.851682in}}%
\pgfpathlineto{\pgfqpoint{5.011226in}{2.856417in}}%
\pgfpathlineto{\pgfqpoint{5.018448in}{2.863567in}}%
\pgfpathlineto{\pgfqpoint{5.025667in}{2.870775in}}%
\pgfpathlineto{\pgfqpoint{5.032881in}{2.878047in}}%
\pgfpathlineto{\pgfqpoint{5.040091in}{2.885389in}}%
\pgfpathlineto{\pgfqpoint{5.026632in}{2.881081in}}%
\pgfpathlineto{\pgfqpoint{5.013185in}{2.876930in}}%
\pgfpathlineto{\pgfqpoint{4.999751in}{2.872933in}}%
\pgfpathlineto{\pgfqpoint{4.986330in}{2.869092in}}%
\pgfpathlineto{\pgfqpoint{4.979103in}{2.861314in}}%
\pgfpathlineto{\pgfqpoint{4.971872in}{2.853611in}}%
\pgfpathlineto{\pgfqpoint{4.964637in}{2.845978in}}%
\pgfpathlineto{\pgfqpoint{4.957399in}{2.838411in}}%
\pgfpathclose%
\pgfusepath{fill}%
\end{pgfscope}%
\begin{pgfscope}%
\pgfpathrectangle{\pgfqpoint{1.254980in}{0.150000in}}{\pgfqpoint{5.490039in}{5.490039in}}%
\pgfusepath{clip}%
\pgfsetbuttcap%
\pgfsetroundjoin%
\definecolor{currentfill}{rgb}{0.279574,0.170599,0.479997}%
\pgfsetfillcolor{currentfill}%
\pgfsetfillopacity{0.700000}%
\pgfsetlinewidth{0.000000pt}%
\definecolor{currentstroke}{rgb}{0.000000,0.000000,0.000000}%
\pgfsetstrokecolor{currentstroke}%
\pgfsetdash{}{0pt}%
\pgfpathmoveto{\pgfqpoint{4.048335in}{2.356398in}}%
\pgfpathlineto{\pgfqpoint{4.061431in}{2.356287in}}%
\pgfpathlineto{\pgfqpoint{4.074534in}{2.356349in}}%
\pgfpathlineto{\pgfqpoint{4.087646in}{2.356584in}}%
\pgfpathlineto{\pgfqpoint{4.100765in}{2.356991in}}%
\pgfpathlineto{\pgfqpoint{4.108326in}{2.366446in}}%
\pgfpathlineto{\pgfqpoint{4.115882in}{2.375895in}}%
\pgfpathlineto{\pgfqpoint{4.123433in}{2.385340in}}%
\pgfpathlineto{\pgfqpoint{4.130979in}{2.394782in}}%
\pgfpathlineto{\pgfqpoint{4.117867in}{2.394491in}}%
\pgfpathlineto{\pgfqpoint{4.104764in}{2.394372in}}%
\pgfpathlineto{\pgfqpoint{4.091669in}{2.394425in}}%
\pgfpathlineto{\pgfqpoint{4.078581in}{2.394652in}}%
\pgfpathlineto{\pgfqpoint{4.071027in}{2.385083in}}%
\pgfpathlineto{\pgfqpoint{4.063468in}{2.375520in}}%
\pgfpathlineto{\pgfqpoint{4.055904in}{2.365959in}}%
\pgfpathlineto{\pgfqpoint{4.048335in}{2.356398in}}%
\pgfpathclose%
\pgfusepath{fill}%
\end{pgfscope}%
\begin{pgfscope}%
\pgfpathrectangle{\pgfqpoint{1.254980in}{0.150000in}}{\pgfqpoint{5.490039in}{5.490039in}}%
\pgfusepath{clip}%
\pgfsetbuttcap%
\pgfsetroundjoin%
\definecolor{currentfill}{rgb}{0.188923,0.410910,0.556326}%
\pgfsetfillcolor{currentfill}%
\pgfsetfillopacity{0.700000}%
\pgfsetlinewidth{0.000000pt}%
\definecolor{currentstroke}{rgb}{0.000000,0.000000,0.000000}%
\pgfsetstrokecolor{currentstroke}%
\pgfsetdash{}{0pt}%
\pgfpathmoveto{\pgfqpoint{5.040091in}{2.885389in}}%
\pgfpathlineto{\pgfqpoint{5.053564in}{2.889851in}}%
\pgfpathlineto{\pgfqpoint{5.067050in}{2.894468in}}%
\pgfpathlineto{\pgfqpoint{5.080549in}{2.899240in}}%
\pgfpathlineto{\pgfqpoint{5.094062in}{2.904167in}}%
\pgfpathlineto{\pgfqpoint{5.101251in}{2.911135in}}%
\pgfpathlineto{\pgfqpoint{5.108436in}{2.918176in}}%
\pgfpathlineto{\pgfqpoint{5.115617in}{2.925293in}}%
\pgfpathlineto{\pgfqpoint{5.122795in}{2.932494in}}%
\pgfpathlineto{\pgfqpoint{5.109300in}{2.928024in}}%
\pgfpathlineto{\pgfqpoint{5.095819in}{2.923708in}}%
\pgfpathlineto{\pgfqpoint{5.082351in}{2.919547in}}%
\pgfpathlineto{\pgfqpoint{5.068896in}{2.915540in}}%
\pgfpathlineto{\pgfqpoint{5.061700in}{2.907874in}}%
\pgfpathlineto{\pgfqpoint{5.054501in}{2.900297in}}%
\pgfpathlineto{\pgfqpoint{5.047298in}{2.892803in}}%
\pgfpathlineto{\pgfqpoint{5.040091in}{2.885389in}}%
\pgfpathclose%
\pgfusepath{fill}%
\end{pgfscope}%
\begin{pgfscope}%
\pgfpathrectangle{\pgfqpoint{1.254980in}{0.150000in}}{\pgfqpoint{5.490039in}{5.490039in}}%
\pgfusepath{clip}%
\pgfsetbuttcap%
\pgfsetroundjoin%
\definecolor{currentfill}{rgb}{0.262138,0.242286,0.520837}%
\pgfsetfillcolor{currentfill}%
\pgfsetfillopacity{0.700000}%
\pgfsetlinewidth{0.000000pt}%
\definecolor{currentstroke}{rgb}{0.000000,0.000000,0.000000}%
\pgfsetstrokecolor{currentstroke}%
\pgfsetdash{}{0pt}%
\pgfpathmoveto{\pgfqpoint{2.831988in}{2.539639in}}%
\pgfpathlineto{\pgfqpoint{2.845103in}{2.524265in}}%
\pgfpathlineto{\pgfqpoint{2.858212in}{2.509154in}}%
\pgfpathlineto{\pgfqpoint{2.871314in}{2.494305in}}%
\pgfpathlineto{\pgfqpoint{2.884411in}{2.479714in}}%
\pgfpathlineto{\pgfqpoint{2.892414in}{2.487204in}}%
\pgfpathlineto{\pgfqpoint{2.900409in}{2.494804in}}%
\pgfpathlineto{\pgfqpoint{2.908395in}{2.502512in}}%
\pgfpathlineto{\pgfqpoint{2.916373in}{2.510328in}}%
\pgfpathlineto{\pgfqpoint{2.903298in}{2.524776in}}%
\pgfpathlineto{\pgfqpoint{2.890218in}{2.539484in}}%
\pgfpathlineto{\pgfqpoint{2.877131in}{2.554452in}}%
\pgfpathlineto{\pgfqpoint{2.864038in}{2.569684in}}%
\pgfpathlineto{\pgfqpoint{2.856039in}{2.562000in}}%
\pgfpathlineto{\pgfqpoint{2.848031in}{2.554430in}}%
\pgfpathlineto{\pgfqpoint{2.840014in}{2.546976in}}%
\pgfpathlineto{\pgfqpoint{2.831988in}{2.539639in}}%
\pgfpathclose%
\pgfusepath{fill}%
\end{pgfscope}%
\begin{pgfscope}%
\pgfpathrectangle{\pgfqpoint{1.254980in}{0.150000in}}{\pgfqpoint{5.490039in}{5.490039in}}%
\pgfusepath{clip}%
\pgfsetbuttcap%
\pgfsetroundjoin%
\definecolor{currentfill}{rgb}{0.252194,0.269783,0.531579}%
\pgfsetfillcolor{currentfill}%
\pgfsetfillopacity{0.700000}%
\pgfsetlinewidth{0.000000pt}%
\definecolor{currentstroke}{rgb}{0.000000,0.000000,0.000000}%
\pgfsetstrokecolor{currentstroke}%
\pgfsetdash{}{0pt}%
\pgfpathmoveto{\pgfqpoint{2.779460in}{2.603817in}}%
\pgfpathlineto{\pgfqpoint{2.792603in}{2.587366in}}%
\pgfpathlineto{\pgfqpoint{2.805739in}{2.571187in}}%
\pgfpathlineto{\pgfqpoint{2.818867in}{2.555279in}}%
\pgfpathlineto{\pgfqpoint{2.831988in}{2.539639in}}%
\pgfpathlineto{\pgfqpoint{2.840014in}{2.546976in}}%
\pgfpathlineto{\pgfqpoint{2.848031in}{2.554430in}}%
\pgfpathlineto{\pgfqpoint{2.856039in}{2.562000in}}%
\pgfpathlineto{\pgfqpoint{2.864038in}{2.569684in}}%
\pgfpathlineto{\pgfqpoint{2.850939in}{2.585181in}}%
\pgfpathlineto{\pgfqpoint{2.837833in}{2.600945in}}%
\pgfpathlineto{\pgfqpoint{2.824721in}{2.616980in}}%
\pgfpathlineto{\pgfqpoint{2.811601in}{2.633288in}}%
\pgfpathlineto{\pgfqpoint{2.803579in}{2.625737in}}%
\pgfpathlineto{\pgfqpoint{2.795549in}{2.618307in}}%
\pgfpathlineto{\pgfqpoint{2.787509in}{2.611000in}}%
\pgfpathlineto{\pgfqpoint{2.779460in}{2.603817in}}%
\pgfpathclose%
\pgfusepath{fill}%
\end{pgfscope}%
\begin{pgfscope}%
\pgfpathrectangle{\pgfqpoint{1.254980in}{0.150000in}}{\pgfqpoint{5.490039in}{5.490039in}}%
\pgfusepath{clip}%
\pgfsetbuttcap%
\pgfsetroundjoin%
\definecolor{currentfill}{rgb}{0.180629,0.429975,0.557282}%
\pgfsetfillcolor{currentfill}%
\pgfsetfillopacity{0.700000}%
\pgfsetlinewidth{0.000000pt}%
\definecolor{currentstroke}{rgb}{0.000000,0.000000,0.000000}%
\pgfsetstrokecolor{currentstroke}%
\pgfsetdash{}{0pt}%
\pgfpathmoveto{\pgfqpoint{5.122795in}{2.932494in}}%
\pgfpathlineto{\pgfqpoint{5.136303in}{2.937117in}}%
\pgfpathlineto{\pgfqpoint{5.149824in}{2.941895in}}%
\pgfpathlineto{\pgfqpoint{5.163360in}{2.946827in}}%
\pgfpathlineto{\pgfqpoint{5.176909in}{2.951913in}}%
\pgfpathlineto{\pgfqpoint{5.184064in}{2.958726in}}%
\pgfpathlineto{\pgfqpoint{5.191216in}{2.965625in}}%
\pgfpathlineto{\pgfqpoint{5.198364in}{2.972616in}}%
\pgfpathlineto{\pgfqpoint{5.205510in}{2.979705in}}%
\pgfpathlineto{\pgfqpoint{5.191981in}{2.975105in}}%
\pgfpathlineto{\pgfqpoint{5.178465in}{2.970658in}}%
\pgfpathlineto{\pgfqpoint{5.164963in}{2.966364in}}%
\pgfpathlineto{\pgfqpoint{5.151474in}{2.962224in}}%
\pgfpathlineto{\pgfqpoint{5.144309in}{2.954641in}}%
\pgfpathlineto{\pgfqpoint{5.137141in}{2.947162in}}%
\pgfpathlineto{\pgfqpoint{5.129969in}{2.939781in}}%
\pgfpathlineto{\pgfqpoint{5.122795in}{2.932494in}}%
\pgfpathclose%
\pgfusepath{fill}%
\end{pgfscope}%
\begin{pgfscope}%
\pgfpathrectangle{\pgfqpoint{1.254980in}{0.150000in}}{\pgfqpoint{5.490039in}{5.490039in}}%
\pgfusepath{clip}%
\pgfsetbuttcap%
\pgfsetroundjoin%
\definecolor{currentfill}{rgb}{0.270595,0.214069,0.507052}%
\pgfsetfillcolor{currentfill}%
\pgfsetfillopacity{0.700000}%
\pgfsetlinewidth{0.000000pt}%
\definecolor{currentstroke}{rgb}{0.000000,0.000000,0.000000}%
\pgfsetstrokecolor{currentstroke}%
\pgfsetdash{}{0pt}%
\pgfpathmoveto{\pgfqpoint{2.884411in}{2.479714in}}%
\pgfpathlineto{\pgfqpoint{2.897501in}{2.465380in}}%
\pgfpathlineto{\pgfqpoint{2.910587in}{2.451301in}}%
\pgfpathlineto{\pgfqpoint{2.923667in}{2.437475in}}%
\pgfpathlineto{\pgfqpoint{2.936742in}{2.423899in}}%
\pgfpathlineto{\pgfqpoint{2.944724in}{2.431541in}}%
\pgfpathlineto{\pgfqpoint{2.952698in}{2.439285in}}%
\pgfpathlineto{\pgfqpoint{2.960663in}{2.447131in}}%
\pgfpathlineto{\pgfqpoint{2.968621in}{2.455077in}}%
\pgfpathlineto{\pgfqpoint{2.955566in}{2.468512in}}%
\pgfpathlineto{\pgfqpoint{2.942507in}{2.482198in}}%
\pgfpathlineto{\pgfqpoint{2.929442in}{2.496136in}}%
\pgfpathlineto{\pgfqpoint{2.916373in}{2.510328in}}%
\pgfpathlineto{\pgfqpoint{2.908395in}{2.502512in}}%
\pgfpathlineto{\pgfqpoint{2.900409in}{2.494804in}}%
\pgfpathlineto{\pgfqpoint{2.892414in}{2.487204in}}%
\pgfpathlineto{\pgfqpoint{2.884411in}{2.479714in}}%
\pgfpathclose%
\pgfusepath{fill}%
\end{pgfscope}%
\begin{pgfscope}%
\pgfpathrectangle{\pgfqpoint{1.254980in}{0.150000in}}{\pgfqpoint{5.490039in}{5.490039in}}%
\pgfusepath{clip}%
\pgfsetbuttcap%
\pgfsetroundjoin%
\definecolor{currentfill}{rgb}{0.281887,0.150881,0.465405}%
\pgfsetfillcolor{currentfill}%
\pgfsetfillopacity{0.700000}%
\pgfsetlinewidth{0.000000pt}%
\definecolor{currentstroke}{rgb}{0.000000,0.000000,0.000000}%
\pgfsetstrokecolor{currentstroke}%
\pgfsetdash{}{0pt}%
\pgfpathmoveto{\pgfqpoint{3.965668in}{2.319978in}}%
\pgfpathlineto{\pgfqpoint{3.978742in}{2.319255in}}%
\pgfpathlineto{\pgfqpoint{3.991824in}{2.318708in}}%
\pgfpathlineto{\pgfqpoint{4.004913in}{2.318337in}}%
\pgfpathlineto{\pgfqpoint{4.018009in}{2.318139in}}%
\pgfpathlineto{\pgfqpoint{4.025598in}{2.327710in}}%
\pgfpathlineto{\pgfqpoint{4.033182in}{2.337275in}}%
\pgfpathlineto{\pgfqpoint{4.040761in}{2.346838in}}%
\pgfpathlineto{\pgfqpoint{4.048335in}{2.356398in}}%
\pgfpathlineto{\pgfqpoint{4.035247in}{2.356684in}}%
\pgfpathlineto{\pgfqpoint{4.022166in}{2.357143in}}%
\pgfpathlineto{\pgfqpoint{4.009093in}{2.357778in}}%
\pgfpathlineto{\pgfqpoint{3.996026in}{2.358588in}}%
\pgfpathlineto{\pgfqpoint{3.988444in}{2.348929in}}%
\pgfpathlineto{\pgfqpoint{3.980857in}{2.339276in}}%
\pgfpathlineto{\pgfqpoint{3.973265in}{2.329626in}}%
\pgfpathlineto{\pgfqpoint{3.965668in}{2.319978in}}%
\pgfpathclose%
\pgfusepath{fill}%
\end{pgfscope}%
\begin{pgfscope}%
\pgfpathrectangle{\pgfqpoint{1.254980in}{0.150000in}}{\pgfqpoint{5.490039in}{5.490039in}}%
\pgfusepath{clip}%
\pgfsetbuttcap%
\pgfsetroundjoin%
\definecolor{currentfill}{rgb}{0.172719,0.448791,0.557885}%
\pgfsetfillcolor{currentfill}%
\pgfsetfillopacity{0.700000}%
\pgfsetlinewidth{0.000000pt}%
\definecolor{currentstroke}{rgb}{0.000000,0.000000,0.000000}%
\pgfsetstrokecolor{currentstroke}%
\pgfsetdash{}{0pt}%
\pgfpathmoveto{\pgfqpoint{5.205510in}{2.979705in}}%
\pgfpathlineto{\pgfqpoint{5.219053in}{2.984458in}}%
\pgfpathlineto{\pgfqpoint{5.232610in}{2.989364in}}%
\pgfpathlineto{\pgfqpoint{5.246181in}{2.994423in}}%
\pgfpathlineto{\pgfqpoint{5.259766in}{2.999635in}}%
\pgfpathlineto{\pgfqpoint{5.266888in}{3.006323in}}%
\pgfpathlineto{\pgfqpoint{5.274007in}{3.013113in}}%
\pgfpathlineto{\pgfqpoint{5.281123in}{3.020010in}}%
\pgfpathlineto{\pgfqpoint{5.288237in}{3.027021in}}%
\pgfpathlineto{\pgfqpoint{5.274673in}{3.022323in}}%
\pgfpathlineto{\pgfqpoint{5.261123in}{3.017778in}}%
\pgfpathlineto{\pgfqpoint{5.247587in}{3.013385in}}%
\pgfpathlineto{\pgfqpoint{5.234065in}{3.009144in}}%
\pgfpathlineto{\pgfqpoint{5.226930in}{3.001610in}}%
\pgfpathlineto{\pgfqpoint{5.219792in}{2.994196in}}%
\pgfpathlineto{\pgfqpoint{5.212653in}{2.986896in}}%
\pgfpathlineto{\pgfqpoint{5.205510in}{2.979705in}}%
\pgfpathclose%
\pgfusepath{fill}%
\end{pgfscope}%
\begin{pgfscope}%
\pgfpathrectangle{\pgfqpoint{1.254980in}{0.150000in}}{\pgfqpoint{5.490039in}{5.490039in}}%
\pgfusepath{clip}%
\pgfsetbuttcap%
\pgfsetroundjoin%
\definecolor{currentfill}{rgb}{0.283197,0.115680,0.436115}%
\pgfsetfillcolor{currentfill}%
\pgfsetfillopacity{0.700000}%
\pgfsetlinewidth{0.000000pt}%
\definecolor{currentstroke}{rgb}{0.000000,0.000000,0.000000}%
\pgfsetstrokecolor{currentstroke}%
\pgfsetdash{}{0pt}%
\pgfpathmoveto{\pgfqpoint{3.177035in}{2.272616in}}%
\pgfpathlineto{\pgfqpoint{3.190043in}{2.263156in}}%
\pgfpathlineto{\pgfqpoint{3.203050in}{2.253914in}}%
\pgfpathlineto{\pgfqpoint{3.216056in}{2.244890in}}%
\pgfpathlineto{\pgfqpoint{3.229062in}{2.236083in}}%
\pgfpathlineto{\pgfqpoint{3.236923in}{2.244733in}}%
\pgfpathlineto{\pgfqpoint{3.244778in}{2.253448in}}%
\pgfpathlineto{\pgfqpoint{3.252626in}{2.262225in}}%
\pgfpathlineto{\pgfqpoint{3.260468in}{2.271065in}}%
\pgfpathlineto{\pgfqpoint{3.247477in}{2.279764in}}%
\pgfpathlineto{\pgfqpoint{3.234487in}{2.288679in}}%
\pgfpathlineto{\pgfqpoint{3.221495in}{2.297811in}}%
\pgfpathlineto{\pgfqpoint{3.208503in}{2.307162in}}%
\pgfpathlineto{\pgfqpoint{3.200646in}{2.298421in}}%
\pgfpathlineto{\pgfqpoint{3.192783in}{2.289749in}}%
\pgfpathlineto{\pgfqpoint{3.184912in}{2.281147in}}%
\pgfpathlineto{\pgfqpoint{3.177035in}{2.272616in}}%
\pgfpathclose%
\pgfusepath{fill}%
\end{pgfscope}%
\begin{pgfscope}%
\pgfpathrectangle{\pgfqpoint{1.254980in}{0.150000in}}{\pgfqpoint{5.490039in}{5.490039in}}%
\pgfusepath{clip}%
\pgfsetbuttcap%
\pgfsetroundjoin%
\definecolor{currentfill}{rgb}{0.239346,0.300855,0.540844}%
\pgfsetfillcolor{currentfill}%
\pgfsetfillopacity{0.700000}%
\pgfsetlinewidth{0.000000pt}%
\definecolor{currentstroke}{rgb}{0.000000,0.000000,0.000000}%
\pgfsetstrokecolor{currentstroke}%
\pgfsetdash{}{0pt}%
\pgfpathmoveto{\pgfqpoint{2.726808in}{2.672402in}}%
\pgfpathlineto{\pgfqpoint{2.739983in}{2.654834in}}%
\pgfpathlineto{\pgfqpoint{2.753150in}{2.637549in}}%
\pgfpathlineto{\pgfqpoint{2.766309in}{2.620544in}}%
\pgfpathlineto{\pgfqpoint{2.779460in}{2.603817in}}%
\pgfpathlineto{\pgfqpoint{2.787509in}{2.611000in}}%
\pgfpathlineto{\pgfqpoint{2.795549in}{2.618307in}}%
\pgfpathlineto{\pgfqpoint{2.803579in}{2.625737in}}%
\pgfpathlineto{\pgfqpoint{2.811601in}{2.633288in}}%
\pgfpathlineto{\pgfqpoint{2.798473in}{2.649870in}}%
\pgfpathlineto{\pgfqpoint{2.785338in}{2.666731in}}%
\pgfpathlineto{\pgfqpoint{2.772195in}{2.683871in}}%
\pgfpathlineto{\pgfqpoint{2.759044in}{2.701294in}}%
\pgfpathlineto{\pgfqpoint{2.750999in}{2.693877in}}%
\pgfpathlineto{\pgfqpoint{2.742945in}{2.686588in}}%
\pgfpathlineto{\pgfqpoint{2.734881in}{2.679430in}}%
\pgfpathlineto{\pgfqpoint{2.726808in}{2.672402in}}%
\pgfpathclose%
\pgfusepath{fill}%
\end{pgfscope}%
\begin{pgfscope}%
\pgfpathrectangle{\pgfqpoint{1.254980in}{0.150000in}}{\pgfqpoint{5.490039in}{5.490039in}}%
\pgfusepath{clip}%
\pgfsetbuttcap%
\pgfsetroundjoin%
\definecolor{currentfill}{rgb}{0.165117,0.467423,0.558141}%
\pgfsetfillcolor{currentfill}%
\pgfsetfillopacity{0.700000}%
\pgfsetlinewidth{0.000000pt}%
\definecolor{currentstroke}{rgb}{0.000000,0.000000,0.000000}%
\pgfsetstrokecolor{currentstroke}%
\pgfsetdash{}{0pt}%
\pgfpathmoveto{\pgfqpoint{5.288237in}{3.027021in}}%
\pgfpathlineto{\pgfqpoint{5.301814in}{3.031871in}}%
\pgfpathlineto{\pgfqpoint{5.315406in}{3.036873in}}%
\pgfpathlineto{\pgfqpoint{5.329012in}{3.042028in}}%
\pgfpathlineto{\pgfqpoint{5.342633in}{3.047334in}}%
\pgfpathlineto{\pgfqpoint{5.349722in}{3.053932in}}%
\pgfpathlineto{\pgfqpoint{5.356809in}{3.060649in}}%
\pgfpathlineto{\pgfqpoint{5.363893in}{3.067491in}}%
\pgfpathlineto{\pgfqpoint{5.370976in}{3.074464in}}%
\pgfpathlineto{\pgfqpoint{5.357379in}{3.069700in}}%
\pgfpathlineto{\pgfqpoint{5.343795in}{3.065088in}}%
\pgfpathlineto{\pgfqpoint{5.330226in}{3.060627in}}%
\pgfpathlineto{\pgfqpoint{5.316670in}{3.056318in}}%
\pgfpathlineto{\pgfqpoint{5.309565in}{3.048794in}}%
\pgfpathlineto{\pgfqpoint{5.302457in}{3.041407in}}%
\pgfpathlineto{\pgfqpoint{5.295348in}{3.034152in}}%
\pgfpathlineto{\pgfqpoint{5.288237in}{3.027021in}}%
\pgfpathclose%
\pgfusepath{fill}%
\end{pgfscope}%
\begin{pgfscope}%
\pgfpathrectangle{\pgfqpoint{1.254980in}{0.150000in}}{\pgfqpoint{5.490039in}{5.490039in}}%
\pgfusepath{clip}%
\pgfsetbuttcap%
\pgfsetroundjoin%
\definecolor{currentfill}{rgb}{0.282327,0.094955,0.417331}%
\pgfsetfillcolor{currentfill}%
\pgfsetfillopacity{0.700000}%
\pgfsetlinewidth{0.000000pt}%
\definecolor{currentstroke}{rgb}{0.000000,0.000000,0.000000}%
\pgfsetstrokecolor{currentstroke}%
\pgfsetdash{}{0pt}%
\pgfpathmoveto{\pgfqpoint{3.582485in}{2.216085in}}%
\pgfpathlineto{\pgfqpoint{3.595487in}{2.211772in}}%
\pgfpathlineto{\pgfqpoint{3.608494in}{2.207649in}}%
\pgfpathlineto{\pgfqpoint{3.621504in}{2.203717in}}%
\pgfpathlineto{\pgfqpoint{3.634519in}{2.199973in}}%
\pgfpathlineto{\pgfqpoint{3.642233in}{2.209569in}}%
\pgfpathlineto{\pgfqpoint{3.649943in}{2.219184in}}%
\pgfpathlineto{\pgfqpoint{3.657647in}{2.228817in}}%
\pgfpathlineto{\pgfqpoint{3.665346in}{2.238470in}}%
\pgfpathlineto{\pgfqpoint{3.652341in}{2.242190in}}%
\pgfpathlineto{\pgfqpoint{3.639341in}{2.246098in}}%
\pgfpathlineto{\pgfqpoint{3.626345in}{2.250196in}}%
\pgfpathlineto{\pgfqpoint{3.613353in}{2.254485in}}%
\pgfpathlineto{\pgfqpoint{3.605644in}{2.244847in}}%
\pgfpathlineto{\pgfqpoint{3.597930in}{2.235234in}}%
\pgfpathlineto{\pgfqpoint{3.590210in}{2.225646in}}%
\pgfpathlineto{\pgfqpoint{3.582485in}{2.216085in}}%
\pgfpathclose%
\pgfusepath{fill}%
\end{pgfscope}%
\begin{pgfscope}%
\pgfpathrectangle{\pgfqpoint{1.254980in}{0.150000in}}{\pgfqpoint{5.490039in}{5.490039in}}%
\pgfusepath{clip}%
\pgfsetbuttcap%
\pgfsetroundjoin%
\definecolor{currentfill}{rgb}{0.276194,0.190074,0.493001}%
\pgfsetfillcolor{currentfill}%
\pgfsetfillopacity{0.700000}%
\pgfsetlinewidth{0.000000pt}%
\definecolor{currentstroke}{rgb}{0.000000,0.000000,0.000000}%
\pgfsetstrokecolor{currentstroke}%
\pgfsetdash{}{0pt}%
\pgfpathmoveto{\pgfqpoint{2.936742in}{2.423899in}}%
\pgfpathlineto{\pgfqpoint{2.949812in}{2.410572in}}%
\pgfpathlineto{\pgfqpoint{2.962878in}{2.397491in}}%
\pgfpathlineto{\pgfqpoint{2.975940in}{2.384655in}}%
\pgfpathlineto{\pgfqpoint{2.988998in}{2.372061in}}%
\pgfpathlineto{\pgfqpoint{2.996959in}{2.379854in}}%
\pgfpathlineto{\pgfqpoint{3.004913in}{2.387742in}}%
\pgfpathlineto{\pgfqpoint{3.012859in}{2.395724in}}%
\pgfpathlineto{\pgfqpoint{3.020797in}{2.403800in}}%
\pgfpathlineto{\pgfqpoint{3.007759in}{2.416254in}}%
\pgfpathlineto{\pgfqpoint{2.994717in}{2.428950in}}%
\pgfpathlineto{\pgfqpoint{2.981671in}{2.441890in}}%
\pgfpathlineto{\pgfqpoint{2.968621in}{2.455077in}}%
\pgfpathlineto{\pgfqpoint{2.960663in}{2.447131in}}%
\pgfpathlineto{\pgfqpoint{2.952698in}{2.439285in}}%
\pgfpathlineto{\pgfqpoint{2.944724in}{2.431541in}}%
\pgfpathlineto{\pgfqpoint{2.936742in}{2.423899in}}%
\pgfpathclose%
\pgfusepath{fill}%
\end{pgfscope}%
\begin{pgfscope}%
\pgfpathrectangle{\pgfqpoint{1.254980in}{0.150000in}}{\pgfqpoint{5.490039in}{5.490039in}}%
\pgfusepath{clip}%
\pgfsetbuttcap%
\pgfsetroundjoin%
\definecolor{currentfill}{rgb}{0.157729,0.485932,0.558013}%
\pgfsetfillcolor{currentfill}%
\pgfsetfillopacity{0.700000}%
\pgfsetlinewidth{0.000000pt}%
\definecolor{currentstroke}{rgb}{0.000000,0.000000,0.000000}%
\pgfsetstrokecolor{currentstroke}%
\pgfsetdash{}{0pt}%
\pgfpathmoveto{\pgfqpoint{5.370976in}{3.074464in}}%
\pgfpathlineto{\pgfqpoint{5.384588in}{3.079379in}}%
\pgfpathlineto{\pgfqpoint{5.398214in}{3.084444in}}%
\pgfpathlineto{\pgfqpoint{5.411855in}{3.089662in}}%
\pgfpathlineto{\pgfqpoint{5.425510in}{3.095030in}}%
\pgfpathlineto{\pgfqpoint{5.432568in}{3.101579in}}%
\pgfpathlineto{\pgfqpoint{5.439623in}{3.108265in}}%
\pgfpathlineto{\pgfqpoint{5.446677in}{3.115094in}}%
\pgfpathlineto{\pgfqpoint{5.453730in}{3.122073in}}%
\pgfpathlineto{\pgfqpoint{5.440099in}{3.117276in}}%
\pgfpathlineto{\pgfqpoint{5.426483in}{3.112629in}}%
\pgfpathlineto{\pgfqpoint{5.412881in}{3.108133in}}%
\pgfpathlineto{\pgfqpoint{5.399293in}{3.103787in}}%
\pgfpathlineto{\pgfqpoint{5.392215in}{3.096229in}}%
\pgfpathlineto{\pgfqpoint{5.385137in}{3.088826in}}%
\pgfpathlineto{\pgfqpoint{5.378057in}{3.081573in}}%
\pgfpathlineto{\pgfqpoint{5.370976in}{3.074464in}}%
\pgfpathclose%
\pgfusepath{fill}%
\end{pgfscope}%
\begin{pgfscope}%
\pgfpathrectangle{\pgfqpoint{1.254980in}{0.150000in}}{\pgfqpoint{5.490039in}{5.490039in}}%
\pgfusepath{clip}%
\pgfsetbuttcap%
\pgfsetroundjoin%
\definecolor{currentfill}{rgb}{0.282884,0.135920,0.453427}%
\pgfsetfillcolor{currentfill}%
\pgfsetfillopacity{0.700000}%
\pgfsetlinewidth{0.000000pt}%
\definecolor{currentstroke}{rgb}{0.000000,0.000000,0.000000}%
\pgfsetstrokecolor{currentstroke}%
\pgfsetdash{}{0pt}%
\pgfpathmoveto{\pgfqpoint{3.882966in}{2.285807in}}%
\pgfpathlineto{\pgfqpoint{3.896022in}{2.284435in}}%
\pgfpathlineto{\pgfqpoint{3.909085in}{2.283242in}}%
\pgfpathlineto{\pgfqpoint{3.922154in}{2.282226in}}%
\pgfpathlineto{\pgfqpoint{3.935229in}{2.281387in}}%
\pgfpathlineto{\pgfqpoint{3.942847in}{2.291037in}}%
\pgfpathlineto{\pgfqpoint{3.950459in}{2.300685in}}%
\pgfpathlineto{\pgfqpoint{3.958066in}{2.310331in}}%
\pgfpathlineto{\pgfqpoint{3.965668in}{2.319978in}}%
\pgfpathlineto{\pgfqpoint{3.952600in}{2.320877in}}%
\pgfpathlineto{\pgfqpoint{3.939540in}{2.321953in}}%
\pgfpathlineto{\pgfqpoint{3.926486in}{2.323206in}}%
\pgfpathlineto{\pgfqpoint{3.913438in}{2.324638in}}%
\pgfpathlineto{\pgfqpoint{3.905828in}{2.314921in}}%
\pgfpathlineto{\pgfqpoint{3.898212in}{2.305211in}}%
\pgfpathlineto{\pgfqpoint{3.890592in}{2.295507in}}%
\pgfpathlineto{\pgfqpoint{3.882966in}{2.285807in}}%
\pgfpathclose%
\pgfusepath{fill}%
\end{pgfscope}%
\begin{pgfscope}%
\pgfpathrectangle{\pgfqpoint{1.254980in}{0.150000in}}{\pgfqpoint{5.490039in}{5.490039in}}%
\pgfusepath{clip}%
\pgfsetbuttcap%
\pgfsetroundjoin%
\definecolor{currentfill}{rgb}{0.150476,0.504369,0.557430}%
\pgfsetfillcolor{currentfill}%
\pgfsetfillopacity{0.700000}%
\pgfsetlinewidth{0.000000pt}%
\definecolor{currentstroke}{rgb}{0.000000,0.000000,0.000000}%
\pgfsetstrokecolor{currentstroke}%
\pgfsetdash{}{0pt}%
\pgfpathmoveto{\pgfqpoint{5.453730in}{3.122073in}}%
\pgfpathlineto{\pgfqpoint{5.467375in}{3.127020in}}%
\pgfpathlineto{\pgfqpoint{5.481035in}{3.132118in}}%
\pgfpathlineto{\pgfqpoint{5.494710in}{3.137366in}}%
\pgfpathlineto{\pgfqpoint{5.508400in}{3.142765in}}%
\pgfpathlineto{\pgfqpoint{5.515426in}{3.149311in}}%
\pgfpathlineto{\pgfqpoint{5.522452in}{3.156012in}}%
\pgfpathlineto{\pgfqpoint{5.529476in}{3.162876in}}%
\pgfpathlineto{\pgfqpoint{5.536501in}{3.169910in}}%
\pgfpathlineto{\pgfqpoint{5.522838in}{3.165111in}}%
\pgfpathlineto{\pgfqpoint{5.509189in}{3.160461in}}%
\pgfpathlineto{\pgfqpoint{5.495555in}{3.155961in}}%
\pgfpathlineto{\pgfqpoint{5.481935in}{3.151611in}}%
\pgfpathlineto{\pgfqpoint{5.474884in}{3.143970in}}%
\pgfpathlineto{\pgfqpoint{5.467834in}{3.136504in}}%
\pgfpathlineto{\pgfqpoint{5.460782in}{3.129207in}}%
\pgfpathlineto{\pgfqpoint{5.453730in}{3.122073in}}%
\pgfpathclose%
\pgfusepath{fill}%
\end{pgfscope}%
\begin{pgfscope}%
\pgfpathrectangle{\pgfqpoint{1.254980in}{0.150000in}}{\pgfqpoint{5.490039in}{5.490039in}}%
\pgfusepath{clip}%
\pgfsetbuttcap%
\pgfsetroundjoin%
\definecolor{currentfill}{rgb}{0.223925,0.334994,0.548053}%
\pgfsetfillcolor{currentfill}%
\pgfsetfillopacity{0.700000}%
\pgfsetlinewidth{0.000000pt}%
\definecolor{currentstroke}{rgb}{0.000000,0.000000,0.000000}%
\pgfsetstrokecolor{currentstroke}%
\pgfsetdash{}{0pt}%
\pgfpathmoveto{\pgfqpoint{2.674015in}{2.745559in}}%
\pgfpathlineto{\pgfqpoint{2.687227in}{2.726832in}}%
\pgfpathlineto{\pgfqpoint{2.700430in}{2.708399in}}%
\pgfpathlineto{\pgfqpoint{2.713623in}{2.690256in}}%
\pgfpathlineto{\pgfqpoint{2.726808in}{2.672402in}}%
\pgfpathlineto{\pgfqpoint{2.734881in}{2.679430in}}%
\pgfpathlineto{\pgfqpoint{2.742945in}{2.686588in}}%
\pgfpathlineto{\pgfqpoint{2.750999in}{2.693877in}}%
\pgfpathlineto{\pgfqpoint{2.759044in}{2.701294in}}%
\pgfpathlineto{\pgfqpoint{2.745884in}{2.719002in}}%
\pgfpathlineto{\pgfqpoint{2.732715in}{2.736998in}}%
\pgfpathlineto{\pgfqpoint{2.719537in}{2.755285in}}%
\pgfpathlineto{\pgfqpoint{2.706350in}{2.773866in}}%
\pgfpathlineto{\pgfqpoint{2.698281in}{2.766584in}}%
\pgfpathlineto{\pgfqpoint{2.690202in}{2.759438in}}%
\pgfpathlineto{\pgfqpoint{2.682113in}{2.752430in}}%
\pgfpathlineto{\pgfqpoint{2.674015in}{2.745559in}}%
\pgfpathclose%
\pgfusepath{fill}%
\end{pgfscope}%
\begin{pgfscope}%
\pgfpathrectangle{\pgfqpoint{1.254980in}{0.150000in}}{\pgfqpoint{5.490039in}{5.490039in}}%
\pgfusepath{clip}%
\pgfsetbuttcap%
\pgfsetroundjoin%
\definecolor{currentfill}{rgb}{0.143343,0.522773,0.556295}%
\pgfsetfillcolor{currentfill}%
\pgfsetfillopacity{0.700000}%
\pgfsetlinewidth{0.000000pt}%
\definecolor{currentstroke}{rgb}{0.000000,0.000000,0.000000}%
\pgfsetstrokecolor{currentstroke}%
\pgfsetdash{}{0pt}%
\pgfpathmoveto{\pgfqpoint{5.536501in}{3.169910in}}%
\pgfpathlineto{\pgfqpoint{5.550179in}{3.174858in}}%
\pgfpathlineto{\pgfqpoint{5.563872in}{3.179956in}}%
\pgfpathlineto{\pgfqpoint{5.577580in}{3.185204in}}%
\pgfpathlineto{\pgfqpoint{5.591303in}{3.190601in}}%
\pgfpathlineto{\pgfqpoint{5.598300in}{3.197194in}}%
\pgfpathlineto{\pgfqpoint{5.605297in}{3.203962in}}%
\pgfpathlineto{\pgfqpoint{5.612294in}{3.210915in}}%
\pgfpathlineto{\pgfqpoint{5.619292in}{3.218058in}}%
\pgfpathlineto{\pgfqpoint{5.605598in}{3.213288in}}%
\pgfpathlineto{\pgfqpoint{5.591918in}{3.208668in}}%
\pgfpathlineto{\pgfqpoint{5.578252in}{3.204196in}}%
\pgfpathlineto{\pgfqpoint{5.564602in}{3.199873in}}%
\pgfpathlineto{\pgfqpoint{5.557576in}{3.192094in}}%
\pgfpathlineto{\pgfqpoint{5.550550in}{3.184512in}}%
\pgfpathlineto{\pgfqpoint{5.543526in}{3.177119in}}%
\pgfpathlineto{\pgfqpoint{5.536501in}{3.169910in}}%
\pgfpathclose%
\pgfusepath{fill}%
\end{pgfscope}%
\begin{pgfscope}%
\pgfpathrectangle{\pgfqpoint{1.254980in}{0.150000in}}{\pgfqpoint{5.490039in}{5.490039in}}%
\pgfusepath{clip}%
\pgfsetbuttcap%
\pgfsetroundjoin%
\definecolor{currentfill}{rgb}{0.280255,0.165693,0.476498}%
\pgfsetfillcolor{currentfill}%
\pgfsetfillopacity{0.700000}%
\pgfsetlinewidth{0.000000pt}%
\definecolor{currentstroke}{rgb}{0.000000,0.000000,0.000000}%
\pgfsetstrokecolor{currentstroke}%
\pgfsetdash{}{0pt}%
\pgfpathmoveto{\pgfqpoint{2.988998in}{2.372061in}}%
\pgfpathlineto{\pgfqpoint{3.002052in}{2.359709in}}%
\pgfpathlineto{\pgfqpoint{3.015102in}{2.347595in}}%
\pgfpathlineto{\pgfqpoint{3.028149in}{2.335718in}}%
\pgfpathlineto{\pgfqpoint{3.041193in}{2.324077in}}%
\pgfpathlineto{\pgfqpoint{3.049135in}{2.332019in}}%
\pgfpathlineto{\pgfqpoint{3.057069in}{2.340050in}}%
\pgfpathlineto{\pgfqpoint{3.064996in}{2.348168in}}%
\pgfpathlineto{\pgfqpoint{3.072915in}{2.356372in}}%
\pgfpathlineto{\pgfqpoint{3.059890in}{2.367875in}}%
\pgfpathlineto{\pgfqpoint{3.046862in}{2.379612in}}%
\pgfpathlineto{\pgfqpoint{3.033831in}{2.391587in}}%
\pgfpathlineto{\pgfqpoint{3.020797in}{2.403800in}}%
\pgfpathlineto{\pgfqpoint{3.012859in}{2.395724in}}%
\pgfpathlineto{\pgfqpoint{3.004913in}{2.387742in}}%
\pgfpathlineto{\pgfqpoint{2.996959in}{2.379854in}}%
\pgfpathlineto{\pgfqpoint{2.988998in}{2.372061in}}%
\pgfpathclose%
\pgfusepath{fill}%
\end{pgfscope}%
\begin{pgfscope}%
\pgfpathrectangle{\pgfqpoint{1.254980in}{0.150000in}}{\pgfqpoint{5.490039in}{5.490039in}}%
\pgfusepath{clip}%
\pgfsetbuttcap%
\pgfsetroundjoin%
\definecolor{currentfill}{rgb}{0.136408,0.541173,0.554483}%
\pgfsetfillcolor{currentfill}%
\pgfsetfillopacity{0.700000}%
\pgfsetlinewidth{0.000000pt}%
\definecolor{currentstroke}{rgb}{0.000000,0.000000,0.000000}%
\pgfsetstrokecolor{currentstroke}%
\pgfsetdash{}{0pt}%
\pgfpathmoveto{\pgfqpoint{5.619292in}{3.218058in}}%
\pgfpathlineto{\pgfqpoint{5.633002in}{3.222976in}}%
\pgfpathlineto{\pgfqpoint{5.646727in}{3.228042in}}%
\pgfpathlineto{\pgfqpoint{5.660467in}{3.233258in}}%
\pgfpathlineto{\pgfqpoint{5.674222in}{3.238622in}}%
\pgfpathlineto{\pgfqpoint{5.681192in}{3.245317in}}%
\pgfpathlineto{\pgfqpoint{5.688163in}{3.252210in}}%
\pgfpathlineto{\pgfqpoint{5.695135in}{3.259308in}}%
\pgfpathlineto{\pgfqpoint{5.702109in}{3.266619in}}%
\pgfpathlineto{\pgfqpoint{5.688383in}{3.261911in}}%
\pgfpathlineto{\pgfqpoint{5.674673in}{3.257351in}}%
\pgfpathlineto{\pgfqpoint{5.660978in}{3.252939in}}%
\pgfpathlineto{\pgfqpoint{5.647297in}{3.248675in}}%
\pgfpathlineto{\pgfqpoint{5.640293in}{3.240699in}}%
\pgfpathlineto{\pgfqpoint{5.633292in}{3.232942in}}%
\pgfpathlineto{\pgfqpoint{5.626291in}{3.225398in}}%
\pgfpathlineto{\pgfqpoint{5.619292in}{3.218058in}}%
\pgfpathclose%
\pgfusepath{fill}%
\end{pgfscope}%
\begin{pgfscope}%
\pgfpathrectangle{\pgfqpoint{1.254980in}{0.150000in}}{\pgfqpoint{5.490039in}{5.490039in}}%
\pgfusepath{clip}%
\pgfsetbuttcap%
\pgfsetroundjoin%
\definecolor{currentfill}{rgb}{0.283229,0.120777,0.440584}%
\pgfsetfillcolor{currentfill}%
\pgfsetfillopacity{0.700000}%
\pgfsetlinewidth{0.000000pt}%
\definecolor{currentstroke}{rgb}{0.000000,0.000000,0.000000}%
\pgfsetstrokecolor{currentstroke}%
\pgfsetdash{}{0pt}%
\pgfpathmoveto{\pgfqpoint{3.800218in}{2.254193in}}%
\pgfpathlineto{\pgfqpoint{3.813258in}{2.252133in}}%
\pgfpathlineto{\pgfqpoint{3.826304in}{2.250254in}}%
\pgfpathlineto{\pgfqpoint{3.839356in}{2.248556in}}%
\pgfpathlineto{\pgfqpoint{3.852414in}{2.247037in}}%
\pgfpathlineto{\pgfqpoint{3.860060in}{2.256727in}}%
\pgfpathlineto{\pgfqpoint{3.867700in}{2.266418in}}%
\pgfpathlineto{\pgfqpoint{3.875336in}{2.276111in}}%
\pgfpathlineto{\pgfqpoint{3.882966in}{2.285807in}}%
\pgfpathlineto{\pgfqpoint{3.869917in}{2.287358in}}%
\pgfpathlineto{\pgfqpoint{3.856874in}{2.289088in}}%
\pgfpathlineto{\pgfqpoint{3.843836in}{2.290999in}}%
\pgfpathlineto{\pgfqpoint{3.830805in}{2.293091in}}%
\pgfpathlineto{\pgfqpoint{3.823166in}{2.283352in}}%
\pgfpathlineto{\pgfqpoint{3.815522in}{2.273624in}}%
\pgfpathlineto{\pgfqpoint{3.807872in}{2.263904in}}%
\pgfpathlineto{\pgfqpoint{3.800218in}{2.254193in}}%
\pgfpathclose%
\pgfusepath{fill}%
\end{pgfscope}%
\begin{pgfscope}%
\pgfpathrectangle{\pgfqpoint{1.254980in}{0.150000in}}{\pgfqpoint{5.490039in}{5.490039in}}%
\pgfusepath{clip}%
\pgfsetbuttcap%
\pgfsetroundjoin%
\definecolor{currentfill}{rgb}{0.281924,0.089666,0.412415}%
\pgfsetfillcolor{currentfill}%
\pgfsetfillopacity{0.700000}%
\pgfsetlinewidth{0.000000pt}%
\definecolor{currentstroke}{rgb}{0.000000,0.000000,0.000000}%
\pgfsetstrokecolor{currentstroke}%
\pgfsetdash{}{0pt}%
\pgfpathmoveto{\pgfqpoint{3.364401in}{2.209095in}}%
\pgfpathlineto{\pgfqpoint{3.377397in}{2.202282in}}%
\pgfpathlineto{\pgfqpoint{3.390394in}{2.195672in}}%
\pgfpathlineto{\pgfqpoint{3.403393in}{2.189265in}}%
\pgfpathlineto{\pgfqpoint{3.416394in}{2.183058in}}%
\pgfpathlineto{\pgfqpoint{3.424188in}{2.192232in}}%
\pgfpathlineto{\pgfqpoint{3.431975in}{2.201447in}}%
\pgfpathlineto{\pgfqpoint{3.439757in}{2.210703in}}%
\pgfpathlineto{\pgfqpoint{3.447533in}{2.219998in}}%
\pgfpathlineto{\pgfqpoint{3.434545in}{2.226125in}}%
\pgfpathlineto{\pgfqpoint{3.421559in}{2.232452in}}%
\pgfpathlineto{\pgfqpoint{3.408575in}{2.238982in}}%
\pgfpathlineto{\pgfqpoint{3.395593in}{2.245715in}}%
\pgfpathlineto{\pgfqpoint{3.387804in}{2.236489in}}%
\pgfpathlineto{\pgfqpoint{3.380009in}{2.227310in}}%
\pgfpathlineto{\pgfqpoint{3.372208in}{2.218179in}}%
\pgfpathlineto{\pgfqpoint{3.364401in}{2.209095in}}%
\pgfpathclose%
\pgfusepath{fill}%
\end{pgfscope}%
\begin{pgfscope}%
\pgfpathrectangle{\pgfqpoint{1.254980in}{0.150000in}}{\pgfqpoint{5.490039in}{5.490039in}}%
\pgfusepath{clip}%
\pgfsetbuttcap%
\pgfsetroundjoin%
\definecolor{currentfill}{rgb}{0.282910,0.105393,0.426902}%
\pgfsetfillcolor{currentfill}%
\pgfsetfillopacity{0.700000}%
\pgfsetlinewidth{0.000000pt}%
\definecolor{currentstroke}{rgb}{0.000000,0.000000,0.000000}%
\pgfsetstrokecolor{currentstroke}%
\pgfsetdash{}{0pt}%
\pgfpathmoveto{\pgfqpoint{3.229062in}{2.236083in}}%
\pgfpathlineto{\pgfqpoint{3.242068in}{2.227490in}}%
\pgfpathlineto{\pgfqpoint{3.255073in}{2.219111in}}%
\pgfpathlineto{\pgfqpoint{3.268078in}{2.210944in}}%
\pgfpathlineto{\pgfqpoint{3.281084in}{2.202988in}}%
\pgfpathlineto{\pgfqpoint{3.288930in}{2.211757in}}%
\pgfpathlineto{\pgfqpoint{3.296770in}{2.220583in}}%
\pgfpathlineto{\pgfqpoint{3.304603in}{2.229466in}}%
\pgfpathlineto{\pgfqpoint{3.312430in}{2.238404in}}%
\pgfpathlineto{\pgfqpoint{3.299439in}{2.246252in}}%
\pgfpathlineto{\pgfqpoint{3.286448in}{2.254311in}}%
\pgfpathlineto{\pgfqpoint{3.273458in}{2.262581in}}%
\pgfpathlineto{\pgfqpoint{3.260468in}{2.271065in}}%
\pgfpathlineto{\pgfqpoint{3.252626in}{2.262225in}}%
\pgfpathlineto{\pgfqpoint{3.244778in}{2.253448in}}%
\pgfpathlineto{\pgfqpoint{3.236923in}{2.244733in}}%
\pgfpathlineto{\pgfqpoint{3.229062in}{2.236083in}}%
\pgfpathclose%
\pgfusepath{fill}%
\end{pgfscope}%
\begin{pgfscope}%
\pgfpathrectangle{\pgfqpoint{1.254980in}{0.150000in}}{\pgfqpoint{5.490039in}{5.490039in}}%
\pgfusepath{clip}%
\pgfsetbuttcap%
\pgfsetroundjoin%
\definecolor{currentfill}{rgb}{0.208623,0.367752,0.552675}%
\pgfsetfillcolor{currentfill}%
\pgfsetfillopacity{0.700000}%
\pgfsetlinewidth{0.000000pt}%
\definecolor{currentstroke}{rgb}{0.000000,0.000000,0.000000}%
\pgfsetstrokecolor{currentstroke}%
\pgfsetdash{}{0pt}%
\pgfpathmoveto{\pgfqpoint{2.621062in}{2.823467in}}%
\pgfpathlineto{\pgfqpoint{2.634316in}{2.803534in}}%
\pgfpathlineto{\pgfqpoint{2.647559in}{2.783908in}}%
\pgfpathlineto{\pgfqpoint{2.660792in}{2.764584in}}%
\pgfpathlineto{\pgfqpoint{2.674015in}{2.745559in}}%
\pgfpathlineto{\pgfqpoint{2.682113in}{2.752430in}}%
\pgfpathlineto{\pgfqpoint{2.690202in}{2.759438in}}%
\pgfpathlineto{\pgfqpoint{2.698281in}{2.766584in}}%
\pgfpathlineto{\pgfqpoint{2.706350in}{2.773866in}}%
\pgfpathlineto{\pgfqpoint{2.693153in}{2.792743in}}%
\pgfpathlineto{\pgfqpoint{2.679946in}{2.811919in}}%
\pgfpathlineto{\pgfqpoint{2.666728in}{2.831398in}}%
\pgfpathlineto{\pgfqpoint{2.653501in}{2.851182in}}%
\pgfpathlineto{\pgfqpoint{2.645406in}{2.844037in}}%
\pgfpathlineto{\pgfqpoint{2.637302in}{2.837035in}}%
\pgfpathlineto{\pgfqpoint{2.629187in}{2.830178in}}%
\pgfpathlineto{\pgfqpoint{2.621062in}{2.823467in}}%
\pgfpathclose%
\pgfusepath{fill}%
\end{pgfscope}%
\begin{pgfscope}%
\pgfpathrectangle{\pgfqpoint{1.254980in}{0.150000in}}{\pgfqpoint{5.490039in}{5.490039in}}%
\pgfusepath{clip}%
\pgfsetbuttcap%
\pgfsetroundjoin%
\definecolor{currentfill}{rgb}{0.281446,0.084320,0.407414}%
\pgfsetfillcolor{currentfill}%
\pgfsetfillopacity{0.700000}%
\pgfsetlinewidth{0.000000pt}%
\definecolor{currentstroke}{rgb}{0.000000,0.000000,0.000000}%
\pgfsetstrokecolor{currentstroke}%
\pgfsetdash{}{0pt}%
\pgfpathmoveto{\pgfqpoint{3.499509in}{2.197480in}}%
\pgfpathlineto{\pgfqpoint{3.512509in}{2.192342in}}%
\pgfpathlineto{\pgfqpoint{3.525513in}{2.187399in}}%
\pgfpathlineto{\pgfqpoint{3.538520in}{2.182650in}}%
\pgfpathlineto{\pgfqpoint{3.551530in}{2.178093in}}%
\pgfpathlineto{\pgfqpoint{3.559277in}{2.187553in}}%
\pgfpathlineto{\pgfqpoint{3.567018in}{2.197038in}}%
\pgfpathlineto{\pgfqpoint{3.574754in}{2.206549in}}%
\pgfpathlineto{\pgfqpoint{3.582485in}{2.216085in}}%
\pgfpathlineto{\pgfqpoint{3.569486in}{2.220590in}}%
\pgfpathlineto{\pgfqpoint{3.556490in}{2.225287in}}%
\pgfpathlineto{\pgfqpoint{3.543498in}{2.230179in}}%
\pgfpathlineto{\pgfqpoint{3.530509in}{2.235264in}}%
\pgfpathlineto{\pgfqpoint{3.522767in}{2.225770in}}%
\pgfpathlineto{\pgfqpoint{3.515020in}{2.216307in}}%
\pgfpathlineto{\pgfqpoint{3.507267in}{2.206877in}}%
\pgfpathlineto{\pgfqpoint{3.499509in}{2.197480in}}%
\pgfpathclose%
\pgfusepath{fill}%
\end{pgfscope}%
\begin{pgfscope}%
\pgfpathrectangle{\pgfqpoint{1.254980in}{0.150000in}}{\pgfqpoint{5.490039in}{5.490039in}}%
\pgfusepath{clip}%
\pgfsetbuttcap%
\pgfsetroundjoin%
\definecolor{currentfill}{rgb}{0.282290,0.145912,0.461510}%
\pgfsetfillcolor{currentfill}%
\pgfsetfillopacity{0.700000}%
\pgfsetlinewidth{0.000000pt}%
\definecolor{currentstroke}{rgb}{0.000000,0.000000,0.000000}%
\pgfsetstrokecolor{currentstroke}%
\pgfsetdash{}{0pt}%
\pgfpathmoveto{\pgfqpoint{3.041193in}{2.324077in}}%
\pgfpathlineto{\pgfqpoint{3.054234in}{2.312670in}}%
\pgfpathlineto{\pgfqpoint{3.067272in}{2.301494in}}%
\pgfpathlineto{\pgfqpoint{3.080308in}{2.290549in}}%
\pgfpathlineto{\pgfqpoint{3.093341in}{2.279832in}}%
\pgfpathlineto{\pgfqpoint{3.101264in}{2.287923in}}%
\pgfpathlineto{\pgfqpoint{3.109180in}{2.296095in}}%
\pgfpathlineto{\pgfqpoint{3.117089in}{2.304347in}}%
\pgfpathlineto{\pgfqpoint{3.124990in}{2.312680in}}%
\pgfpathlineto{\pgfqpoint{3.111975in}{2.323259in}}%
\pgfpathlineto{\pgfqpoint{3.098957in}{2.334066in}}%
\pgfpathlineto{\pgfqpoint{3.085937in}{2.345103in}}%
\pgfpathlineto{\pgfqpoint{3.072915in}{2.356372in}}%
\pgfpathlineto{\pgfqpoint{3.064996in}{2.348168in}}%
\pgfpathlineto{\pgfqpoint{3.057069in}{2.340050in}}%
\pgfpathlineto{\pgfqpoint{3.049135in}{2.332019in}}%
\pgfpathlineto{\pgfqpoint{3.041193in}{2.324077in}}%
\pgfpathclose%
\pgfusepath{fill}%
\end{pgfscope}%
\begin{pgfscope}%
\pgfpathrectangle{\pgfqpoint{1.254980in}{0.150000in}}{\pgfqpoint{5.490039in}{5.490039in}}%
\pgfusepath{clip}%
\pgfsetbuttcap%
\pgfsetroundjoin%
\definecolor{currentfill}{rgb}{0.129933,0.559582,0.551864}%
\pgfsetfillcolor{currentfill}%
\pgfsetfillopacity{0.700000}%
\pgfsetlinewidth{0.000000pt}%
\definecolor{currentstroke}{rgb}{0.000000,0.000000,0.000000}%
\pgfsetstrokecolor{currentstroke}%
\pgfsetdash{}{0pt}%
\pgfpathmoveto{\pgfqpoint{5.702109in}{3.266619in}}%
\pgfpathlineto{\pgfqpoint{5.715849in}{3.271475in}}%
\pgfpathlineto{\pgfqpoint{5.729605in}{3.276480in}}%
\pgfpathlineto{\pgfqpoint{5.743376in}{3.281632in}}%
\pgfpathlineto{\pgfqpoint{5.757162in}{3.286932in}}%
\pgfpathlineto{\pgfqpoint{5.764107in}{3.293789in}}%
\pgfpathlineto{\pgfqpoint{5.771054in}{3.300868in}}%
\pgfpathlineto{\pgfqpoint{5.778004in}{3.308176in}}%
\pgfpathlineto{\pgfqpoint{5.764241in}{3.303387in}}%
\pgfpathlineto{\pgfqpoint{5.750494in}{3.298745in}}%
\pgfpathlineto{\pgfqpoint{5.736761in}{3.294251in}}%
\pgfpathlineto{\pgfqpoint{5.723044in}{3.289904in}}%
\pgfpathlineto{\pgfqpoint{5.716063in}{3.281910in}}%
\pgfpathlineto{\pgfqpoint{5.709085in}{3.274151in}}%
\pgfpathlineto{\pgfqpoint{5.702109in}{3.266619in}}%
\pgfpathclose%
\pgfusepath{fill}%
\end{pgfscope}%
\begin{pgfscope}%
\pgfpathrectangle{\pgfqpoint{1.254980in}{0.150000in}}{\pgfqpoint{5.490039in}{5.490039in}}%
\pgfusepath{clip}%
\pgfsetbuttcap%
\pgfsetroundjoin%
\definecolor{currentfill}{rgb}{0.282910,0.105393,0.426902}%
\pgfsetfillcolor{currentfill}%
\pgfsetfillopacity{0.700000}%
\pgfsetlinewidth{0.000000pt}%
\definecolor{currentstroke}{rgb}{0.000000,0.000000,0.000000}%
\pgfsetstrokecolor{currentstroke}%
\pgfsetdash{}{0pt}%
\pgfpathmoveto{\pgfqpoint{3.717408in}{2.225465in}}%
\pgfpathlineto{\pgfqpoint{3.730436in}{2.222677in}}%
\pgfpathlineto{\pgfqpoint{3.743469in}{2.220074in}}%
\pgfpathlineto{\pgfqpoint{3.756507in}{2.217654in}}%
\pgfpathlineto{\pgfqpoint{3.769550in}{2.215417in}}%
\pgfpathlineto{\pgfqpoint{3.777225in}{2.225102in}}%
\pgfpathlineto{\pgfqpoint{3.784894in}{2.234792in}}%
\pgfpathlineto{\pgfqpoint{3.792559in}{2.244489in}}%
\pgfpathlineto{\pgfqpoint{3.800218in}{2.254193in}}%
\pgfpathlineto{\pgfqpoint{3.787184in}{2.256434in}}%
\pgfpathlineto{\pgfqpoint{3.774155in}{2.258859in}}%
\pgfpathlineto{\pgfqpoint{3.761131in}{2.261466in}}%
\pgfpathlineto{\pgfqpoint{3.748113in}{2.264257in}}%
\pgfpathlineto{\pgfqpoint{3.740444in}{2.254539in}}%
\pgfpathlineto{\pgfqpoint{3.732771in}{2.244834in}}%
\pgfpathlineto{\pgfqpoint{3.725092in}{2.235143in}}%
\pgfpathlineto{\pgfqpoint{3.717408in}{2.225465in}}%
\pgfpathclose%
\pgfusepath{fill}%
\end{pgfscope}%
\begin{pgfscope}%
\pgfpathrectangle{\pgfqpoint{1.254980in}{0.150000in}}{\pgfqpoint{5.490039in}{5.490039in}}%
\pgfusepath{clip}%
\pgfsetbuttcap%
\pgfsetroundjoin%
\definecolor{currentfill}{rgb}{0.283187,0.125848,0.444960}%
\pgfsetfillcolor{currentfill}%
\pgfsetfillopacity{0.700000}%
\pgfsetlinewidth{0.000000pt}%
\definecolor{currentstroke}{rgb}{0.000000,0.000000,0.000000}%
\pgfsetstrokecolor{currentstroke}%
\pgfsetdash{}{0pt}%
\pgfpathmoveto{\pgfqpoint{3.093341in}{2.279832in}}%
\pgfpathlineto{\pgfqpoint{3.106372in}{2.269342in}}%
\pgfpathlineto{\pgfqpoint{3.119402in}{2.259078in}}%
\pgfpathlineto{\pgfqpoint{3.132429in}{2.249037in}}%
\pgfpathlineto{\pgfqpoint{3.145456in}{2.239219in}}%
\pgfpathlineto{\pgfqpoint{3.153361in}{2.247457in}}%
\pgfpathlineto{\pgfqpoint{3.161259in}{2.255770in}}%
\pgfpathlineto{\pgfqpoint{3.169150in}{2.264157in}}%
\pgfpathlineto{\pgfqpoint{3.177035in}{2.272616in}}%
\pgfpathlineto{\pgfqpoint{3.164026in}{2.282297in}}%
\pgfpathlineto{\pgfqpoint{3.151015in}{2.292201in}}%
\pgfpathlineto{\pgfqpoint{3.138003in}{2.302328in}}%
\pgfpathlineto{\pgfqpoint{3.124990in}{2.312680in}}%
\pgfpathlineto{\pgfqpoint{3.117089in}{2.304347in}}%
\pgfpathlineto{\pgfqpoint{3.109180in}{2.296095in}}%
\pgfpathlineto{\pgfqpoint{3.101264in}{2.287923in}}%
\pgfpathlineto{\pgfqpoint{3.093341in}{2.279832in}}%
\pgfpathclose%
\pgfusepath{fill}%
\end{pgfscope}%
\begin{pgfscope}%
\pgfpathrectangle{\pgfqpoint{1.254980in}{0.150000in}}{\pgfqpoint{5.490039in}{5.490039in}}%
\pgfusepath{clip}%
\pgfsetbuttcap%
\pgfsetroundjoin%
\definecolor{currentfill}{rgb}{0.262138,0.242286,0.520837}%
\pgfsetfillcolor{currentfill}%
\pgfsetfillopacity{0.700000}%
\pgfsetlinewidth{0.000000pt}%
\definecolor{currentstroke}{rgb}{0.000000,0.000000,0.000000}%
\pgfsetstrokecolor{currentstroke}%
\pgfsetdash{}{0pt}%
\pgfpathmoveto{\pgfqpoint{4.348979in}{2.483409in}}%
\pgfpathlineto{\pgfqpoint{4.362190in}{2.485578in}}%
\pgfpathlineto{\pgfqpoint{4.375412in}{2.487913in}}%
\pgfpathlineto{\pgfqpoint{4.388644in}{2.490413in}}%
\pgfpathlineto{\pgfqpoint{4.401885in}{2.493077in}}%
\pgfpathlineto{\pgfqpoint{4.409352in}{2.501827in}}%
\pgfpathlineto{\pgfqpoint{4.416813in}{2.510566in}}%
\pgfpathlineto{\pgfqpoint{4.424269in}{2.519300in}}%
\pgfpathlineto{\pgfqpoint{4.431720in}{2.528028in}}%
\pgfpathlineto{\pgfqpoint{4.418487in}{2.525565in}}%
\pgfpathlineto{\pgfqpoint{4.405265in}{2.523266in}}%
\pgfpathlineto{\pgfqpoint{4.392052in}{2.521132in}}%
\pgfpathlineto{\pgfqpoint{4.378850in}{2.519164in}}%
\pgfpathlineto{\pgfqpoint{4.371390in}{2.510224in}}%
\pgfpathlineto{\pgfqpoint{4.363924in}{2.501287in}}%
\pgfpathlineto{\pgfqpoint{4.356454in}{2.492349in}}%
\pgfpathlineto{\pgfqpoint{4.348979in}{2.483409in}}%
\pgfpathclose%
\pgfusepath{fill}%
\end{pgfscope}%
\begin{pgfscope}%
\pgfpathrectangle{\pgfqpoint{1.254980in}{0.150000in}}{\pgfqpoint{5.490039in}{5.490039in}}%
\pgfusepath{clip}%
\pgfsetbuttcap%
\pgfsetroundjoin%
\definecolor{currentfill}{rgb}{0.267968,0.223549,0.512008}%
\pgfsetfillcolor{currentfill}%
\pgfsetfillopacity{0.700000}%
\pgfsetlinewidth{0.000000pt}%
\definecolor{currentstroke}{rgb}{0.000000,0.000000,0.000000}%
\pgfsetstrokecolor{currentstroke}%
\pgfsetdash{}{0pt}%
\pgfpathmoveto{\pgfqpoint{4.266243in}{2.439879in}}%
\pgfpathlineto{\pgfqpoint{4.279425in}{2.441553in}}%
\pgfpathlineto{\pgfqpoint{4.292616in}{2.443395in}}%
\pgfpathlineto{\pgfqpoint{4.305816in}{2.445404in}}%
\pgfpathlineto{\pgfqpoint{4.319027in}{2.447580in}}%
\pgfpathlineto{\pgfqpoint{4.326522in}{2.456552in}}%
\pgfpathlineto{\pgfqpoint{4.334013in}{2.465513in}}%
\pgfpathlineto{\pgfqpoint{4.341498in}{2.474465in}}%
\pgfpathlineto{\pgfqpoint{4.348979in}{2.483409in}}%
\pgfpathlineto{\pgfqpoint{4.335777in}{2.481407in}}%
\pgfpathlineto{\pgfqpoint{4.322585in}{2.479571in}}%
\pgfpathlineto{\pgfqpoint{4.309403in}{2.477901in}}%
\pgfpathlineto{\pgfqpoint{4.296230in}{2.476399in}}%
\pgfpathlineto{\pgfqpoint{4.288740in}{2.467272in}}%
\pgfpathlineto{\pgfqpoint{4.281246in}{2.458144in}}%
\pgfpathlineto{\pgfqpoint{4.273747in}{2.449014in}}%
\pgfpathlineto{\pgfqpoint{4.266243in}{2.439879in}}%
\pgfpathclose%
\pgfusepath{fill}%
\end{pgfscope}%
\begin{pgfscope}%
\pgfpathrectangle{\pgfqpoint{1.254980in}{0.150000in}}{\pgfqpoint{5.490039in}{5.490039in}}%
\pgfusepath{clip}%
\pgfsetbuttcap%
\pgfsetroundjoin%
\definecolor{currentfill}{rgb}{0.253935,0.265254,0.529983}%
\pgfsetfillcolor{currentfill}%
\pgfsetfillopacity{0.700000}%
\pgfsetlinewidth{0.000000pt}%
\definecolor{currentstroke}{rgb}{0.000000,0.000000,0.000000}%
\pgfsetstrokecolor{currentstroke}%
\pgfsetdash{}{0pt}%
\pgfpathmoveto{\pgfqpoint{4.431720in}{2.528028in}}%
\pgfpathlineto{\pgfqpoint{4.444963in}{2.530657in}}%
\pgfpathlineto{\pgfqpoint{4.458216in}{2.533449in}}%
\pgfpathlineto{\pgfqpoint{4.471480in}{2.536405in}}%
\pgfpathlineto{\pgfqpoint{4.484755in}{2.539524in}}%
\pgfpathlineto{\pgfqpoint{4.492192in}{2.548033in}}%
\pgfpathlineto{\pgfqpoint{4.499623in}{2.556535in}}%
\pgfpathlineto{\pgfqpoint{4.507049in}{2.565035in}}%
\pgfpathlineto{\pgfqpoint{4.514470in}{2.573534in}}%
\pgfpathlineto{\pgfqpoint{4.501205in}{2.570644in}}%
\pgfpathlineto{\pgfqpoint{4.487950in}{2.567918in}}%
\pgfpathlineto{\pgfqpoint{4.474707in}{2.565355in}}%
\pgfpathlineto{\pgfqpoint{4.461473in}{2.562955in}}%
\pgfpathlineto{\pgfqpoint{4.454042in}{2.554216in}}%
\pgfpathlineto{\pgfqpoint{4.446607in}{2.545484in}}%
\pgfpathlineto{\pgfqpoint{4.439166in}{2.536756in}}%
\pgfpathlineto{\pgfqpoint{4.431720in}{2.528028in}}%
\pgfpathclose%
\pgfusepath{fill}%
\end{pgfscope}%
\begin{pgfscope}%
\pgfpathrectangle{\pgfqpoint{1.254980in}{0.150000in}}{\pgfqpoint{5.490039in}{5.490039in}}%
\pgfusepath{clip}%
\pgfsetbuttcap%
\pgfsetroundjoin%
\definecolor{currentfill}{rgb}{0.282327,0.094955,0.417331}%
\pgfsetfillcolor{currentfill}%
\pgfsetfillopacity{0.700000}%
\pgfsetlinewidth{0.000000pt}%
\definecolor{currentstroke}{rgb}{0.000000,0.000000,0.000000}%
\pgfsetstrokecolor{currentstroke}%
\pgfsetdash{}{0pt}%
\pgfpathmoveto{\pgfqpoint{3.634519in}{2.199973in}}%
\pgfpathlineto{\pgfqpoint{3.647537in}{2.196418in}}%
\pgfpathlineto{\pgfqpoint{3.660560in}{2.193051in}}%
\pgfpathlineto{\pgfqpoint{3.673588in}{2.189869in}}%
\pgfpathlineto{\pgfqpoint{3.686621in}{2.186874in}}%
\pgfpathlineto{\pgfqpoint{3.694325in}{2.196503in}}%
\pgfpathlineto{\pgfqpoint{3.702025in}{2.206145in}}%
\pgfpathlineto{\pgfqpoint{3.709719in}{2.215799in}}%
\pgfpathlineto{\pgfqpoint{3.717408in}{2.225465in}}%
\pgfpathlineto{\pgfqpoint{3.704385in}{2.228437in}}%
\pgfpathlineto{\pgfqpoint{3.691367in}{2.231594in}}%
\pgfpathlineto{\pgfqpoint{3.678354in}{2.234939in}}%
\pgfpathlineto{\pgfqpoint{3.665346in}{2.238470in}}%
\pgfpathlineto{\pgfqpoint{3.657647in}{2.228817in}}%
\pgfpathlineto{\pgfqpoint{3.649943in}{2.219184in}}%
\pgfpathlineto{\pgfqpoint{3.642233in}{2.209569in}}%
\pgfpathlineto{\pgfqpoint{3.634519in}{2.199973in}}%
\pgfpathclose%
\pgfusepath{fill}%
\end{pgfscope}%
\begin{pgfscope}%
\pgfpathrectangle{\pgfqpoint{1.254980in}{0.150000in}}{\pgfqpoint{5.490039in}{5.490039in}}%
\pgfusepath{clip}%
\pgfsetbuttcap%
\pgfsetroundjoin%
\definecolor{currentfill}{rgb}{0.274128,0.199721,0.498911}%
\pgfsetfillcolor{currentfill}%
\pgfsetfillopacity{0.700000}%
\pgfsetlinewidth{0.000000pt}%
\definecolor{currentstroke}{rgb}{0.000000,0.000000,0.000000}%
\pgfsetstrokecolor{currentstroke}%
\pgfsetdash{}{0pt}%
\pgfpathmoveto{\pgfqpoint{4.183507in}{2.397659in}}%
\pgfpathlineto{\pgfqpoint{4.196661in}{2.398803in}}%
\pgfpathlineto{\pgfqpoint{4.209823in}{2.400117in}}%
\pgfpathlineto{\pgfqpoint{4.222995in}{2.401600in}}%
\pgfpathlineto{\pgfqpoint{4.236175in}{2.403251in}}%
\pgfpathlineto{\pgfqpoint{4.243700in}{2.412425in}}%
\pgfpathlineto{\pgfqpoint{4.251219in}{2.421587in}}%
\pgfpathlineto{\pgfqpoint{4.258734in}{2.430737in}}%
\pgfpathlineto{\pgfqpoint{4.266243in}{2.439879in}}%
\pgfpathlineto{\pgfqpoint{4.253070in}{2.438372in}}%
\pgfpathlineto{\pgfqpoint{4.239907in}{2.437034in}}%
\pgfpathlineto{\pgfqpoint{4.226753in}{2.435864in}}%
\pgfpathlineto{\pgfqpoint{4.213608in}{2.434864in}}%
\pgfpathlineto{\pgfqpoint{4.206090in}{2.425567in}}%
\pgfpathlineto{\pgfqpoint{4.198567in}{2.416269in}}%
\pgfpathlineto{\pgfqpoint{4.191040in}{2.406967in}}%
\pgfpathlineto{\pgfqpoint{4.183507in}{2.397659in}}%
\pgfpathclose%
\pgfusepath{fill}%
\end{pgfscope}%
\begin{pgfscope}%
\pgfpathrectangle{\pgfqpoint{1.254980in}{0.150000in}}{\pgfqpoint{5.490039in}{5.490039in}}%
\pgfusepath{clip}%
\pgfsetbuttcap%
\pgfsetroundjoin%
\definecolor{currentfill}{rgb}{0.244972,0.287675,0.537260}%
\pgfsetfillcolor{currentfill}%
\pgfsetfillopacity{0.700000}%
\pgfsetlinewidth{0.000000pt}%
\definecolor{currentstroke}{rgb}{0.000000,0.000000,0.000000}%
\pgfsetstrokecolor{currentstroke}%
\pgfsetdash{}{0pt}%
\pgfpathmoveto{\pgfqpoint{4.514470in}{2.573534in}}%
\pgfpathlineto{\pgfqpoint{4.527745in}{2.576587in}}%
\pgfpathlineto{\pgfqpoint{4.541032in}{2.579802in}}%
\pgfpathlineto{\pgfqpoint{4.554330in}{2.583180in}}%
\pgfpathlineto{\pgfqpoint{4.567639in}{2.586720in}}%
\pgfpathlineto{\pgfqpoint{4.575044in}{2.594975in}}%
\pgfpathlineto{\pgfqpoint{4.582445in}{2.603228in}}%
\pgfpathlineto{\pgfqpoint{4.589840in}{2.611484in}}%
\pgfpathlineto{\pgfqpoint{4.597230in}{2.619744in}}%
\pgfpathlineto{\pgfqpoint{4.583932in}{2.616463in}}%
\pgfpathlineto{\pgfqpoint{4.570644in}{2.613343in}}%
\pgfpathlineto{\pgfqpoint{4.557368in}{2.610385in}}%
\pgfpathlineto{\pgfqpoint{4.544103in}{2.607590in}}%
\pgfpathlineto{\pgfqpoint{4.536702in}{2.599061in}}%
\pgfpathlineto{\pgfqpoint{4.529296in}{2.590544in}}%
\pgfpathlineto{\pgfqpoint{4.521885in}{2.582036in}}%
\pgfpathlineto{\pgfqpoint{4.514470in}{2.573534in}}%
\pgfpathclose%
\pgfusepath{fill}%
\end{pgfscope}%
\begin{pgfscope}%
\pgfpathrectangle{\pgfqpoint{1.254980in}{0.150000in}}{\pgfqpoint{5.490039in}{5.490039in}}%
\pgfusepath{clip}%
\pgfsetbuttcap%
\pgfsetroundjoin%
\definecolor{currentfill}{rgb}{0.237441,0.305202,0.541921}%
\pgfsetfillcolor{currentfill}%
\pgfsetfillopacity{0.700000}%
\pgfsetlinewidth{0.000000pt}%
\definecolor{currentstroke}{rgb}{0.000000,0.000000,0.000000}%
\pgfsetstrokecolor{currentstroke}%
\pgfsetdash{}{0pt}%
\pgfpathmoveto{\pgfqpoint{4.597230in}{2.619744in}}%
\pgfpathlineto{\pgfqpoint{4.610540in}{2.623188in}}%
\pgfpathlineto{\pgfqpoint{4.623861in}{2.626792in}}%
\pgfpathlineto{\pgfqpoint{4.637194in}{2.630558in}}%
\pgfpathlineto{\pgfqpoint{4.650538in}{2.634484in}}%
\pgfpathlineto{\pgfqpoint{4.657912in}{2.642477in}}%
\pgfpathlineto{\pgfqpoint{4.665281in}{2.650475in}}%
\pgfpathlineto{\pgfqpoint{4.672645in}{2.658481in}}%
\pgfpathlineto{\pgfqpoint{4.680003in}{2.666498in}}%
\pgfpathlineto{\pgfqpoint{4.666670in}{2.662859in}}%
\pgfpathlineto{\pgfqpoint{4.653349in}{2.659380in}}%
\pgfpathlineto{\pgfqpoint{4.640039in}{2.656061in}}%
\pgfpathlineto{\pgfqpoint{4.626741in}{2.652904in}}%
\pgfpathlineto{\pgfqpoint{4.619370in}{2.644590in}}%
\pgfpathlineto{\pgfqpoint{4.611995in}{2.636294in}}%
\pgfpathlineto{\pgfqpoint{4.604615in}{2.628014in}}%
\pgfpathlineto{\pgfqpoint{4.597230in}{2.619744in}}%
\pgfpathclose%
\pgfusepath{fill}%
\end{pgfscope}%
\begin{pgfscope}%
\pgfpathrectangle{\pgfqpoint{1.254980in}{0.150000in}}{\pgfqpoint{5.490039in}{5.490039in}}%
\pgfusepath{clip}%
\pgfsetbuttcap%
\pgfsetroundjoin%
\definecolor{currentfill}{rgb}{0.278012,0.180367,0.486697}%
\pgfsetfillcolor{currentfill}%
\pgfsetfillopacity{0.700000}%
\pgfsetlinewidth{0.000000pt}%
\definecolor{currentstroke}{rgb}{0.000000,0.000000,0.000000}%
\pgfsetstrokecolor{currentstroke}%
\pgfsetdash{}{0pt}%
\pgfpathmoveto{\pgfqpoint{4.100765in}{2.356991in}}%
\pgfpathlineto{\pgfqpoint{4.113893in}{2.357570in}}%
\pgfpathlineto{\pgfqpoint{4.127029in}{2.358320in}}%
\pgfpathlineto{\pgfqpoint{4.140173in}{2.359241in}}%
\pgfpathlineto{\pgfqpoint{4.153326in}{2.360333in}}%
\pgfpathlineto{\pgfqpoint{4.160879in}{2.369682in}}%
\pgfpathlineto{\pgfqpoint{4.168427in}{2.379018in}}%
\pgfpathlineto{\pgfqpoint{4.175969in}{2.388343in}}%
\pgfpathlineto{\pgfqpoint{4.183507in}{2.397659in}}%
\pgfpathlineto{\pgfqpoint{4.170362in}{2.396684in}}%
\pgfpathlineto{\pgfqpoint{4.157226in}{2.395879in}}%
\pgfpathlineto{\pgfqpoint{4.144098in}{2.395245in}}%
\pgfpathlineto{\pgfqpoint{4.130979in}{2.394782in}}%
\pgfpathlineto{\pgfqpoint{4.123433in}{2.385340in}}%
\pgfpathlineto{\pgfqpoint{4.115882in}{2.375895in}}%
\pgfpathlineto{\pgfqpoint{4.108326in}{2.366446in}}%
\pgfpathlineto{\pgfqpoint{4.100765in}{2.356991in}}%
\pgfpathclose%
\pgfusepath{fill}%
\end{pgfscope}%
\begin{pgfscope}%
\pgfpathrectangle{\pgfqpoint{1.254980in}{0.150000in}}{\pgfqpoint{5.490039in}{5.490039in}}%
\pgfusepath{clip}%
\pgfsetbuttcap%
\pgfsetroundjoin%
\definecolor{currentfill}{rgb}{0.281924,0.089666,0.412415}%
\pgfsetfillcolor{currentfill}%
\pgfsetfillopacity{0.700000}%
\pgfsetlinewidth{0.000000pt}%
\definecolor{currentstroke}{rgb}{0.000000,0.000000,0.000000}%
\pgfsetstrokecolor{currentstroke}%
\pgfsetdash{}{0pt}%
\pgfpathmoveto{\pgfqpoint{3.281084in}{2.202988in}}%
\pgfpathlineto{\pgfqpoint{3.294090in}{2.195241in}}%
\pgfpathlineto{\pgfqpoint{3.307097in}{2.187703in}}%
\pgfpathlineto{\pgfqpoint{3.320105in}{2.180372in}}%
\pgfpathlineto{\pgfqpoint{3.333113in}{2.173247in}}%
\pgfpathlineto{\pgfqpoint{3.340945in}{2.182134in}}%
\pgfpathlineto{\pgfqpoint{3.348770in}{2.191072in}}%
\pgfpathlineto{\pgfqpoint{3.356589in}{2.200059in}}%
\pgfpathlineto{\pgfqpoint{3.364401in}{2.209095in}}%
\pgfpathlineto{\pgfqpoint{3.351407in}{2.216112in}}%
\pgfpathlineto{\pgfqpoint{3.338414in}{2.223336in}}%
\pgfpathlineto{\pgfqpoint{3.325421in}{2.230766in}}%
\pgfpathlineto{\pgfqpoint{3.312430in}{2.238404in}}%
\pgfpathlineto{\pgfqpoint{3.304603in}{2.229466in}}%
\pgfpathlineto{\pgfqpoint{3.296770in}{2.220583in}}%
\pgfpathlineto{\pgfqpoint{3.288930in}{2.211757in}}%
\pgfpathlineto{\pgfqpoint{3.281084in}{2.202988in}}%
\pgfpathclose%
\pgfusepath{fill}%
\end{pgfscope}%
\begin{pgfscope}%
\pgfpathrectangle{\pgfqpoint{1.254980in}{0.150000in}}{\pgfqpoint{5.490039in}{5.490039in}}%
\pgfusepath{clip}%
\pgfsetbuttcap%
\pgfsetroundjoin%
\definecolor{currentfill}{rgb}{0.281446,0.084320,0.407414}%
\pgfsetfillcolor{currentfill}%
\pgfsetfillopacity{0.700000}%
\pgfsetlinewidth{0.000000pt}%
\definecolor{currentstroke}{rgb}{0.000000,0.000000,0.000000}%
\pgfsetstrokecolor{currentstroke}%
\pgfsetdash{}{0pt}%
\pgfpathmoveto{\pgfqpoint{3.416394in}{2.183058in}}%
\pgfpathlineto{\pgfqpoint{3.429397in}{2.177052in}}%
\pgfpathlineto{\pgfqpoint{3.442402in}{2.171244in}}%
\pgfpathlineto{\pgfqpoint{3.455409in}{2.165635in}}%
\pgfpathlineto{\pgfqpoint{3.468419in}{2.160222in}}%
\pgfpathlineto{\pgfqpoint{3.476200in}{2.169486in}}%
\pgfpathlineto{\pgfqpoint{3.483975in}{2.178784in}}%
\pgfpathlineto{\pgfqpoint{3.491745in}{2.188115in}}%
\pgfpathlineto{\pgfqpoint{3.499509in}{2.197480in}}%
\pgfpathlineto{\pgfqpoint{3.486511in}{2.202813in}}%
\pgfpathlineto{\pgfqpoint{3.473516in}{2.208344in}}%
\pgfpathlineto{\pgfqpoint{3.460524in}{2.214071in}}%
\pgfpathlineto{\pgfqpoint{3.447533in}{2.219998in}}%
\pgfpathlineto{\pgfqpoint{3.439757in}{2.210703in}}%
\pgfpathlineto{\pgfqpoint{3.431975in}{2.201447in}}%
\pgfpathlineto{\pgfqpoint{3.424188in}{2.192232in}}%
\pgfpathlineto{\pgfqpoint{3.416394in}{2.183058in}}%
\pgfpathclose%
\pgfusepath{fill}%
\end{pgfscope}%
\begin{pgfscope}%
\pgfpathrectangle{\pgfqpoint{1.254980in}{0.150000in}}{\pgfqpoint{5.490039in}{5.490039in}}%
\pgfusepath{clip}%
\pgfsetbuttcap%
\pgfsetroundjoin%
\definecolor{currentfill}{rgb}{0.227802,0.326594,0.546532}%
\pgfsetfillcolor{currentfill}%
\pgfsetfillopacity{0.700000}%
\pgfsetlinewidth{0.000000pt}%
\definecolor{currentstroke}{rgb}{0.000000,0.000000,0.000000}%
\pgfsetstrokecolor{currentstroke}%
\pgfsetdash{}{0pt}%
\pgfpathmoveto{\pgfqpoint{4.680003in}{2.666498in}}%
\pgfpathlineto{\pgfqpoint{4.693348in}{2.670298in}}%
\pgfpathlineto{\pgfqpoint{4.706704in}{2.674258in}}%
\pgfpathlineto{\pgfqpoint{4.720073in}{2.678377in}}%
\pgfpathlineto{\pgfqpoint{4.733453in}{2.682656in}}%
\pgfpathlineto{\pgfqpoint{4.740795in}{2.690384in}}%
\pgfpathlineto{\pgfqpoint{4.748131in}{2.698124in}}%
\pgfpathlineto{\pgfqpoint{4.755463in}{2.705879in}}%
\pgfpathlineto{\pgfqpoint{4.762789in}{2.713654in}}%
\pgfpathlineto{\pgfqpoint{4.749421in}{2.709690in}}%
\pgfpathlineto{\pgfqpoint{4.736065in}{2.705886in}}%
\pgfpathlineto{\pgfqpoint{4.722721in}{2.702241in}}%
\pgfpathlineto{\pgfqpoint{4.709389in}{2.698756in}}%
\pgfpathlineto{\pgfqpoint{4.702049in}{2.690656in}}%
\pgfpathlineto{\pgfqpoint{4.694706in}{2.682582in}}%
\pgfpathlineto{\pgfqpoint{4.687357in}{2.674531in}}%
\pgfpathlineto{\pgfqpoint{4.680003in}{2.666498in}}%
\pgfpathclose%
\pgfusepath{fill}%
\end{pgfscope}%
\begin{pgfscope}%
\pgfpathrectangle{\pgfqpoint{1.254980in}{0.150000in}}{\pgfqpoint{5.490039in}{5.490039in}}%
\pgfusepath{clip}%
\pgfsetbuttcap%
\pgfsetroundjoin%
\definecolor{currentfill}{rgb}{0.280868,0.160771,0.472899}%
\pgfsetfillcolor{currentfill}%
\pgfsetfillopacity{0.700000}%
\pgfsetlinewidth{0.000000pt}%
\definecolor{currentstroke}{rgb}{0.000000,0.000000,0.000000}%
\pgfsetstrokecolor{currentstroke}%
\pgfsetdash{}{0pt}%
\pgfpathmoveto{\pgfqpoint{4.018009in}{2.318139in}}%
\pgfpathlineto{\pgfqpoint{4.031113in}{2.318116in}}%
\pgfpathlineto{\pgfqpoint{4.044225in}{2.318266in}}%
\pgfpathlineto{\pgfqpoint{4.057344in}{2.318589in}}%
\pgfpathlineto{\pgfqpoint{4.070472in}{2.319085in}}%
\pgfpathlineto{\pgfqpoint{4.078052in}{2.328577in}}%
\pgfpathlineto{\pgfqpoint{4.085628in}{2.338058in}}%
\pgfpathlineto{\pgfqpoint{4.093199in}{2.347529in}}%
\pgfpathlineto{\pgfqpoint{4.100765in}{2.356991in}}%
\pgfpathlineto{\pgfqpoint{4.087646in}{2.356584in}}%
\pgfpathlineto{\pgfqpoint{4.074534in}{2.356349in}}%
\pgfpathlineto{\pgfqpoint{4.061431in}{2.356287in}}%
\pgfpathlineto{\pgfqpoint{4.048335in}{2.356398in}}%
\pgfpathlineto{\pgfqpoint{4.040761in}{2.346838in}}%
\pgfpathlineto{\pgfqpoint{4.033182in}{2.337275in}}%
\pgfpathlineto{\pgfqpoint{4.025598in}{2.327710in}}%
\pgfpathlineto{\pgfqpoint{4.018009in}{2.318139in}}%
\pgfpathclose%
\pgfusepath{fill}%
\end{pgfscope}%
\begin{pgfscope}%
\pgfpathrectangle{\pgfqpoint{1.254980in}{0.150000in}}{\pgfqpoint{5.490039in}{5.490039in}}%
\pgfusepath{clip}%
\pgfsetbuttcap%
\pgfsetroundjoin%
\definecolor{currentfill}{rgb}{0.218130,0.347432,0.550038}%
\pgfsetfillcolor{currentfill}%
\pgfsetfillopacity{0.700000}%
\pgfsetlinewidth{0.000000pt}%
\definecolor{currentstroke}{rgb}{0.000000,0.000000,0.000000}%
\pgfsetstrokecolor{currentstroke}%
\pgfsetdash{}{0pt}%
\pgfpathmoveto{\pgfqpoint{4.762789in}{2.713654in}}%
\pgfpathlineto{\pgfqpoint{4.776170in}{2.717776in}}%
\pgfpathlineto{\pgfqpoint{4.789562in}{2.722057in}}%
\pgfpathlineto{\pgfqpoint{4.802967in}{2.726497in}}%
\pgfpathlineto{\pgfqpoint{4.816385in}{2.731095in}}%
\pgfpathlineto{\pgfqpoint{4.823693in}{2.738560in}}%
\pgfpathlineto{\pgfqpoint{4.830997in}{2.746044in}}%
\pgfpathlineto{\pgfqpoint{4.838295in}{2.753553in}}%
\pgfpathlineto{\pgfqpoint{4.845589in}{2.761090in}}%
\pgfpathlineto{\pgfqpoint{4.832185in}{2.756836in}}%
\pgfpathlineto{\pgfqpoint{4.818794in}{2.752740in}}%
\pgfpathlineto{\pgfqpoint{4.805414in}{2.748802in}}%
\pgfpathlineto{\pgfqpoint{4.792047in}{2.745022in}}%
\pgfpathlineto{\pgfqpoint{4.784740in}{2.737132in}}%
\pgfpathlineto{\pgfqpoint{4.777428in}{2.729276in}}%
\pgfpathlineto{\pgfqpoint{4.770111in}{2.721452in}}%
\pgfpathlineto{\pgfqpoint{4.762789in}{2.713654in}}%
\pgfpathclose%
\pgfusepath{fill}%
\end{pgfscope}%
\begin{pgfscope}%
\pgfpathrectangle{\pgfqpoint{1.254980in}{0.150000in}}{\pgfqpoint{5.490039in}{5.490039in}}%
\pgfusepath{clip}%
\pgfsetbuttcap%
\pgfsetroundjoin%
\definecolor{currentfill}{rgb}{0.208623,0.367752,0.552675}%
\pgfsetfillcolor{currentfill}%
\pgfsetfillopacity{0.700000}%
\pgfsetlinewidth{0.000000pt}%
\definecolor{currentstroke}{rgb}{0.000000,0.000000,0.000000}%
\pgfsetstrokecolor{currentstroke}%
\pgfsetdash{}{0pt}%
\pgfpathmoveto{\pgfqpoint{4.845589in}{2.761090in}}%
\pgfpathlineto{\pgfqpoint{4.859005in}{2.765501in}}%
\pgfpathlineto{\pgfqpoint{4.872434in}{2.770071in}}%
\pgfpathlineto{\pgfqpoint{4.885876in}{2.774798in}}%
\pgfpathlineto{\pgfqpoint{4.899331in}{2.779682in}}%
\pgfpathlineto{\pgfqpoint{4.906606in}{2.786889in}}%
\pgfpathlineto{\pgfqpoint{4.913876in}{2.794125in}}%
\pgfpathlineto{\pgfqpoint{4.921141in}{2.801396in}}%
\pgfpathlineto{\pgfqpoint{4.928401in}{2.808704in}}%
\pgfpathlineto{\pgfqpoint{4.914961in}{2.804193in}}%
\pgfpathlineto{\pgfqpoint{4.901534in}{2.799839in}}%
\pgfpathlineto{\pgfqpoint{4.888119in}{2.795642in}}%
\pgfpathlineto{\pgfqpoint{4.874717in}{2.791601in}}%
\pgfpathlineto{\pgfqpoint{4.867442in}{2.783910in}}%
\pgfpathlineto{\pgfqpoint{4.860162in}{2.776264in}}%
\pgfpathlineto{\pgfqpoint{4.852878in}{2.768659in}}%
\pgfpathlineto{\pgfqpoint{4.845589in}{2.761090in}}%
\pgfpathclose%
\pgfusepath{fill}%
\end{pgfscope}%
\begin{pgfscope}%
\pgfpathrectangle{\pgfqpoint{1.254980in}{0.150000in}}{\pgfqpoint{5.490039in}{5.490039in}}%
\pgfusepath{clip}%
\pgfsetbuttcap%
\pgfsetroundjoin%
\definecolor{currentfill}{rgb}{0.282623,0.140926,0.457517}%
\pgfsetfillcolor{currentfill}%
\pgfsetfillopacity{0.700000}%
\pgfsetlinewidth{0.000000pt}%
\definecolor{currentstroke}{rgb}{0.000000,0.000000,0.000000}%
\pgfsetstrokecolor{currentstroke}%
\pgfsetdash{}{0pt}%
\pgfpathmoveto{\pgfqpoint{3.935229in}{2.281387in}}%
\pgfpathlineto{\pgfqpoint{3.948312in}{2.280724in}}%
\pgfpathlineto{\pgfqpoint{3.961402in}{2.280237in}}%
\pgfpathlineto{\pgfqpoint{3.974499in}{2.279926in}}%
\pgfpathlineto{\pgfqpoint{3.987604in}{2.279789in}}%
\pgfpathlineto{\pgfqpoint{3.995212in}{2.289389in}}%
\pgfpathlineto{\pgfqpoint{4.002816in}{2.298980in}}%
\pgfpathlineto{\pgfqpoint{4.010415in}{2.308563in}}%
\pgfpathlineto{\pgfqpoint{4.018009in}{2.318139in}}%
\pgfpathlineto{\pgfqpoint{4.004913in}{2.318337in}}%
\pgfpathlineto{\pgfqpoint{3.991824in}{2.318708in}}%
\pgfpathlineto{\pgfqpoint{3.978742in}{2.319255in}}%
\pgfpathlineto{\pgfqpoint{3.965668in}{2.319978in}}%
\pgfpathlineto{\pgfqpoint{3.958066in}{2.310331in}}%
\pgfpathlineto{\pgfqpoint{3.950459in}{2.300685in}}%
\pgfpathlineto{\pgfqpoint{3.942847in}{2.291037in}}%
\pgfpathlineto{\pgfqpoint{3.935229in}{2.281387in}}%
\pgfpathclose%
\pgfusepath{fill}%
\end{pgfscope}%
\begin{pgfscope}%
\pgfpathrectangle{\pgfqpoint{1.254980in}{0.150000in}}{\pgfqpoint{5.490039in}{5.490039in}}%
\pgfusepath{clip}%
\pgfsetbuttcap%
\pgfsetroundjoin%
\definecolor{currentfill}{rgb}{0.199430,0.387607,0.554642}%
\pgfsetfillcolor{currentfill}%
\pgfsetfillopacity{0.700000}%
\pgfsetlinewidth{0.000000pt}%
\definecolor{currentstroke}{rgb}{0.000000,0.000000,0.000000}%
\pgfsetstrokecolor{currentstroke}%
\pgfsetdash{}{0pt}%
\pgfpathmoveto{\pgfqpoint{4.928401in}{2.808704in}}%
\pgfpathlineto{\pgfqpoint{4.941854in}{2.813372in}}%
\pgfpathlineto{\pgfqpoint{4.955320in}{2.818197in}}%
\pgfpathlineto{\pgfqpoint{4.968799in}{2.823178in}}%
\pgfpathlineto{\pgfqpoint{4.982292in}{2.828315in}}%
\pgfpathlineto{\pgfqpoint{4.989532in}{2.835275in}}%
\pgfpathlineto{\pgfqpoint{4.996768in}{2.842276in}}%
\pgfpathlineto{\pgfqpoint{5.003999in}{2.849322in}}%
\pgfpathlineto{\pgfqpoint{5.011226in}{2.856417in}}%
\pgfpathlineto{\pgfqpoint{4.997749in}{2.851682in}}%
\pgfpathlineto{\pgfqpoint{4.984286in}{2.847102in}}%
\pgfpathlineto{\pgfqpoint{4.970836in}{2.842678in}}%
\pgfpathlineto{\pgfqpoint{4.957399in}{2.838411in}}%
\pgfpathlineto{\pgfqpoint{4.950156in}{2.830904in}}%
\pgfpathlineto{\pgfqpoint{4.942909in}{2.823454in}}%
\pgfpathlineto{\pgfqpoint{4.935657in}{2.816056in}}%
\pgfpathlineto{\pgfqpoint{4.928401in}{2.808704in}}%
\pgfpathclose%
\pgfusepath{fill}%
\end{pgfscope}%
\begin{pgfscope}%
\pgfpathrectangle{\pgfqpoint{1.254980in}{0.150000in}}{\pgfqpoint{5.490039in}{5.490039in}}%
\pgfusepath{clip}%
\pgfsetbuttcap%
\pgfsetroundjoin%
\definecolor{currentfill}{rgb}{0.190631,0.407061,0.556089}%
\pgfsetfillcolor{currentfill}%
\pgfsetfillopacity{0.700000}%
\pgfsetlinewidth{0.000000pt}%
\definecolor{currentstroke}{rgb}{0.000000,0.000000,0.000000}%
\pgfsetstrokecolor{currentstroke}%
\pgfsetdash{}{0pt}%
\pgfpathmoveto{\pgfqpoint{5.011226in}{2.856417in}}%
\pgfpathlineto{\pgfqpoint{5.024715in}{2.861308in}}%
\pgfpathlineto{\pgfqpoint{5.038219in}{2.866355in}}%
\pgfpathlineto{\pgfqpoint{5.051735in}{2.871557in}}%
\pgfpathlineto{\pgfqpoint{5.065265in}{2.876914in}}%
\pgfpathlineto{\pgfqpoint{5.072471in}{2.883644in}}%
\pgfpathlineto{\pgfqpoint{5.079672in}{2.890426in}}%
\pgfpathlineto{\pgfqpoint{5.086869in}{2.897265in}}%
\pgfpathlineto{\pgfqpoint{5.094062in}{2.904167in}}%
\pgfpathlineto{\pgfqpoint{5.080549in}{2.899240in}}%
\pgfpathlineto{\pgfqpoint{5.067050in}{2.894468in}}%
\pgfpathlineto{\pgfqpoint{5.053564in}{2.889851in}}%
\pgfpathlineto{\pgfqpoint{5.040091in}{2.885389in}}%
\pgfpathlineto{\pgfqpoint{5.032881in}{2.878047in}}%
\pgfpathlineto{\pgfqpoint{5.025667in}{2.870775in}}%
\pgfpathlineto{\pgfqpoint{5.018448in}{2.863567in}}%
\pgfpathlineto{\pgfqpoint{5.011226in}{2.856417in}}%
\pgfpathclose%
\pgfusepath{fill}%
\end{pgfscope}%
\begin{pgfscope}%
\pgfpathrectangle{\pgfqpoint{1.254980in}{0.150000in}}{\pgfqpoint{5.490039in}{5.490039in}}%
\pgfusepath{clip}%
\pgfsetbuttcap%
\pgfsetroundjoin%
\definecolor{currentfill}{rgb}{0.262138,0.242286,0.520837}%
\pgfsetfillcolor{currentfill}%
\pgfsetfillopacity{0.700000}%
\pgfsetlinewidth{0.000000pt}%
\definecolor{currentstroke}{rgb}{0.000000,0.000000,0.000000}%
\pgfsetstrokecolor{currentstroke}%
\pgfsetdash{}{0pt}%
\pgfpathmoveto{\pgfqpoint{2.799795in}{2.511481in}}%
\pgfpathlineto{\pgfqpoint{2.812934in}{2.495936in}}%
\pgfpathlineto{\pgfqpoint{2.826065in}{2.480654in}}%
\pgfpathlineto{\pgfqpoint{2.839191in}{2.465634in}}%
\pgfpathlineto{\pgfqpoint{2.852310in}{2.450874in}}%
\pgfpathlineto{\pgfqpoint{2.860348in}{2.457913in}}%
\pgfpathlineto{\pgfqpoint{2.868378in}{2.465067in}}%
\pgfpathlineto{\pgfqpoint{2.876399in}{2.472335in}}%
\pgfpathlineto{\pgfqpoint{2.884411in}{2.479714in}}%
\pgfpathlineto{\pgfqpoint{2.871314in}{2.494305in}}%
\pgfpathlineto{\pgfqpoint{2.858212in}{2.509154in}}%
\pgfpathlineto{\pgfqpoint{2.845103in}{2.524265in}}%
\pgfpathlineto{\pgfqpoint{2.831988in}{2.539639in}}%
\pgfpathlineto{\pgfqpoint{2.823954in}{2.532420in}}%
\pgfpathlineto{\pgfqpoint{2.815910in}{2.525319in}}%
\pgfpathlineto{\pgfqpoint{2.807857in}{2.518339in}}%
\pgfpathlineto{\pgfqpoint{2.799795in}{2.511481in}}%
\pgfpathclose%
\pgfusepath{fill}%
\end{pgfscope}%
\begin{pgfscope}%
\pgfpathrectangle{\pgfqpoint{1.254980in}{0.150000in}}{\pgfqpoint{5.490039in}{5.490039in}}%
\pgfusepath{clip}%
\pgfsetbuttcap%
\pgfsetroundjoin%
\definecolor{currentfill}{rgb}{0.283091,0.110553,0.431554}%
\pgfsetfillcolor{currentfill}%
\pgfsetfillopacity{0.700000}%
\pgfsetlinewidth{0.000000pt}%
\definecolor{currentstroke}{rgb}{0.000000,0.000000,0.000000}%
\pgfsetstrokecolor{currentstroke}%
\pgfsetdash{}{0pt}%
\pgfpathmoveto{\pgfqpoint{3.145456in}{2.239219in}}%
\pgfpathlineto{\pgfqpoint{3.158481in}{2.229621in}}%
\pgfpathlineto{\pgfqpoint{3.171505in}{2.220243in}}%
\pgfpathlineto{\pgfqpoint{3.184528in}{2.211082in}}%
\pgfpathlineto{\pgfqpoint{3.197550in}{2.202138in}}%
\pgfpathlineto{\pgfqpoint{3.205438in}{2.210524in}}%
\pgfpathlineto{\pgfqpoint{3.213319in}{2.218977in}}%
\pgfpathlineto{\pgfqpoint{3.221194in}{2.227497in}}%
\pgfpathlineto{\pgfqpoint{3.229062in}{2.236083in}}%
\pgfpathlineto{\pgfqpoint{3.216056in}{2.244890in}}%
\pgfpathlineto{\pgfqpoint{3.203050in}{2.253914in}}%
\pgfpathlineto{\pgfqpoint{3.190043in}{2.263156in}}%
\pgfpathlineto{\pgfqpoint{3.177035in}{2.272616in}}%
\pgfpathlineto{\pgfqpoint{3.169150in}{2.264157in}}%
\pgfpathlineto{\pgfqpoint{3.161259in}{2.255770in}}%
\pgfpathlineto{\pgfqpoint{3.153361in}{2.247457in}}%
\pgfpathlineto{\pgfqpoint{3.145456in}{2.239219in}}%
\pgfpathclose%
\pgfusepath{fill}%
\end{pgfscope}%
\begin{pgfscope}%
\pgfpathrectangle{\pgfqpoint{1.254980in}{0.150000in}}{\pgfqpoint{5.490039in}{5.490039in}}%
\pgfusepath{clip}%
\pgfsetbuttcap%
\pgfsetroundjoin%
\definecolor{currentfill}{rgb}{0.281446,0.084320,0.407414}%
\pgfsetfillcolor{currentfill}%
\pgfsetfillopacity{0.700000}%
\pgfsetlinewidth{0.000000pt}%
\definecolor{currentstroke}{rgb}{0.000000,0.000000,0.000000}%
\pgfsetstrokecolor{currentstroke}%
\pgfsetdash{}{0pt}%
\pgfpathmoveto{\pgfqpoint{3.551530in}{2.178093in}}%
\pgfpathlineto{\pgfqpoint{3.564543in}{2.173729in}}%
\pgfpathlineto{\pgfqpoint{3.577561in}{2.169555in}}%
\pgfpathlineto{\pgfqpoint{3.590582in}{2.165571in}}%
\pgfpathlineto{\pgfqpoint{3.603607in}{2.161777in}}%
\pgfpathlineto{\pgfqpoint{3.611343in}{2.171298in}}%
\pgfpathlineto{\pgfqpoint{3.619073in}{2.180838in}}%
\pgfpathlineto{\pgfqpoint{3.626799in}{2.190396in}}%
\pgfpathlineto{\pgfqpoint{3.634519in}{2.199973in}}%
\pgfpathlineto{\pgfqpoint{3.621504in}{2.203717in}}%
\pgfpathlineto{\pgfqpoint{3.608494in}{2.207649in}}%
\pgfpathlineto{\pgfqpoint{3.595487in}{2.211772in}}%
\pgfpathlineto{\pgfqpoint{3.582485in}{2.216085in}}%
\pgfpathlineto{\pgfqpoint{3.574754in}{2.206549in}}%
\pgfpathlineto{\pgfqpoint{3.567018in}{2.197038in}}%
\pgfpathlineto{\pgfqpoint{3.559277in}{2.187553in}}%
\pgfpathlineto{\pgfqpoint{3.551530in}{2.178093in}}%
\pgfpathclose%
\pgfusepath{fill}%
\end{pgfscope}%
\begin{pgfscope}%
\pgfpathrectangle{\pgfqpoint{1.254980in}{0.150000in}}{\pgfqpoint{5.490039in}{5.490039in}}%
\pgfusepath{clip}%
\pgfsetbuttcap%
\pgfsetroundjoin%
\definecolor{currentfill}{rgb}{0.252194,0.269783,0.531579}%
\pgfsetfillcolor{currentfill}%
\pgfsetfillopacity{0.700000}%
\pgfsetlinewidth{0.000000pt}%
\definecolor{currentstroke}{rgb}{0.000000,0.000000,0.000000}%
\pgfsetstrokecolor{currentstroke}%
\pgfsetdash{}{0pt}%
\pgfpathmoveto{\pgfqpoint{2.747169in}{2.576347in}}%
\pgfpathlineto{\pgfqpoint{2.760337in}{2.559723in}}%
\pgfpathlineto{\pgfqpoint{2.773497in}{2.543372in}}%
\pgfpathlineto{\pgfqpoint{2.786650in}{2.527293in}}%
\pgfpathlineto{\pgfqpoint{2.799795in}{2.511481in}}%
\pgfpathlineto{\pgfqpoint{2.807857in}{2.518339in}}%
\pgfpathlineto{\pgfqpoint{2.815910in}{2.525319in}}%
\pgfpathlineto{\pgfqpoint{2.823954in}{2.532420in}}%
\pgfpathlineto{\pgfqpoint{2.831988in}{2.539639in}}%
\pgfpathlineto{\pgfqpoint{2.818867in}{2.555279in}}%
\pgfpathlineto{\pgfqpoint{2.805739in}{2.571187in}}%
\pgfpathlineto{\pgfqpoint{2.792603in}{2.587366in}}%
\pgfpathlineto{\pgfqpoint{2.779460in}{2.603817in}}%
\pgfpathlineto{\pgfqpoint{2.771401in}{2.596759in}}%
\pgfpathlineto{\pgfqpoint{2.763334in}{2.589827in}}%
\pgfpathlineto{\pgfqpoint{2.755256in}{2.583023in}}%
\pgfpathlineto{\pgfqpoint{2.747169in}{2.576347in}}%
\pgfpathclose%
\pgfusepath{fill}%
\end{pgfscope}%
\begin{pgfscope}%
\pgfpathrectangle{\pgfqpoint{1.254980in}{0.150000in}}{\pgfqpoint{5.490039in}{5.490039in}}%
\pgfusepath{clip}%
\pgfsetbuttcap%
\pgfsetroundjoin%
\definecolor{currentfill}{rgb}{0.270595,0.214069,0.507052}%
\pgfsetfillcolor{currentfill}%
\pgfsetfillopacity{0.700000}%
\pgfsetlinewidth{0.000000pt}%
\definecolor{currentstroke}{rgb}{0.000000,0.000000,0.000000}%
\pgfsetstrokecolor{currentstroke}%
\pgfsetdash{}{0pt}%
\pgfpathmoveto{\pgfqpoint{2.852310in}{2.450874in}}%
\pgfpathlineto{\pgfqpoint{2.865423in}{2.436370in}}%
\pgfpathlineto{\pgfqpoint{2.878531in}{2.422121in}}%
\pgfpathlineto{\pgfqpoint{2.891633in}{2.408126in}}%
\pgfpathlineto{\pgfqpoint{2.904730in}{2.394381in}}%
\pgfpathlineto{\pgfqpoint{2.912746in}{2.401601in}}%
\pgfpathlineto{\pgfqpoint{2.920753in}{2.408928in}}%
\pgfpathlineto{\pgfqpoint{2.928752in}{2.416361in}}%
\pgfpathlineto{\pgfqpoint{2.936742in}{2.423899in}}%
\pgfpathlineto{\pgfqpoint{2.923667in}{2.437475in}}%
\pgfpathlineto{\pgfqpoint{2.910587in}{2.451301in}}%
\pgfpathlineto{\pgfqpoint{2.897501in}{2.465380in}}%
\pgfpathlineto{\pgfqpoint{2.884411in}{2.479714in}}%
\pgfpathlineto{\pgfqpoint{2.876399in}{2.472335in}}%
\pgfpathlineto{\pgfqpoint{2.868378in}{2.465067in}}%
\pgfpathlineto{\pgfqpoint{2.860348in}{2.457913in}}%
\pgfpathlineto{\pgfqpoint{2.852310in}{2.450874in}}%
\pgfpathclose%
\pgfusepath{fill}%
\end{pgfscope}%
\begin{pgfscope}%
\pgfpathrectangle{\pgfqpoint{1.254980in}{0.150000in}}{\pgfqpoint{5.490039in}{5.490039in}}%
\pgfusepath{clip}%
\pgfsetbuttcap%
\pgfsetroundjoin%
\definecolor{currentfill}{rgb}{0.182256,0.426184,0.557120}%
\pgfsetfillcolor{currentfill}%
\pgfsetfillopacity{0.700000}%
\pgfsetlinewidth{0.000000pt}%
\definecolor{currentstroke}{rgb}{0.000000,0.000000,0.000000}%
\pgfsetstrokecolor{currentstroke}%
\pgfsetdash{}{0pt}%
\pgfpathmoveto{\pgfqpoint{5.094062in}{2.904167in}}%
\pgfpathlineto{\pgfqpoint{5.107588in}{2.909248in}}%
\pgfpathlineto{\pgfqpoint{5.121128in}{2.914484in}}%
\pgfpathlineto{\pgfqpoint{5.134682in}{2.919875in}}%
\pgfpathlineto{\pgfqpoint{5.148250in}{2.925420in}}%
\pgfpathlineto{\pgfqpoint{5.155421in}{2.931940in}}%
\pgfpathlineto{\pgfqpoint{5.162587in}{2.938525in}}%
\pgfpathlineto{\pgfqpoint{5.169750in}{2.945181in}}%
\pgfpathlineto{\pgfqpoint{5.176909in}{2.951913in}}%
\pgfpathlineto{\pgfqpoint{5.163360in}{2.946827in}}%
\pgfpathlineto{\pgfqpoint{5.149824in}{2.941895in}}%
\pgfpathlineto{\pgfqpoint{5.136303in}{2.937117in}}%
\pgfpathlineto{\pgfqpoint{5.122795in}{2.932494in}}%
\pgfpathlineto{\pgfqpoint{5.115617in}{2.925293in}}%
\pgfpathlineto{\pgfqpoint{5.108436in}{2.918176in}}%
\pgfpathlineto{\pgfqpoint{5.101251in}{2.911135in}}%
\pgfpathlineto{\pgfqpoint{5.094062in}{2.904167in}}%
\pgfpathclose%
\pgfusepath{fill}%
\end{pgfscope}%
\begin{pgfscope}%
\pgfpathrectangle{\pgfqpoint{1.254980in}{0.150000in}}{\pgfqpoint{5.490039in}{5.490039in}}%
\pgfusepath{clip}%
\pgfsetbuttcap%
\pgfsetroundjoin%
\definecolor{currentfill}{rgb}{0.283187,0.125848,0.444960}%
\pgfsetfillcolor{currentfill}%
\pgfsetfillopacity{0.700000}%
\pgfsetlinewidth{0.000000pt}%
\definecolor{currentstroke}{rgb}{0.000000,0.000000,0.000000}%
\pgfsetstrokecolor{currentstroke}%
\pgfsetdash{}{0pt}%
\pgfpathmoveto{\pgfqpoint{3.852414in}{2.247037in}}%
\pgfpathlineto{\pgfqpoint{3.865479in}{2.245698in}}%
\pgfpathlineto{\pgfqpoint{3.878550in}{2.244536in}}%
\pgfpathlineto{\pgfqpoint{3.891627in}{2.243553in}}%
\pgfpathlineto{\pgfqpoint{3.904711in}{2.242746in}}%
\pgfpathlineto{\pgfqpoint{3.912348in}{2.252414in}}%
\pgfpathlineto{\pgfqpoint{3.919980in}{2.262076in}}%
\pgfpathlineto{\pgfqpoint{3.927607in}{2.271733in}}%
\pgfpathlineto{\pgfqpoint{3.935229in}{2.281387in}}%
\pgfpathlineto{\pgfqpoint{3.922154in}{2.282226in}}%
\pgfpathlineto{\pgfqpoint{3.909085in}{2.283242in}}%
\pgfpathlineto{\pgfqpoint{3.896022in}{2.284435in}}%
\pgfpathlineto{\pgfqpoint{3.882966in}{2.285807in}}%
\pgfpathlineto{\pgfqpoint{3.875336in}{2.276111in}}%
\pgfpathlineto{\pgfqpoint{3.867700in}{2.266418in}}%
\pgfpathlineto{\pgfqpoint{3.860060in}{2.256727in}}%
\pgfpathlineto{\pgfqpoint{3.852414in}{2.247037in}}%
\pgfpathclose%
\pgfusepath{fill}%
\end{pgfscope}%
\begin{pgfscope}%
\pgfpathrectangle{\pgfqpoint{1.254980in}{0.150000in}}{\pgfqpoint{5.490039in}{5.490039in}}%
\pgfusepath{clip}%
\pgfsetbuttcap%
\pgfsetroundjoin%
\definecolor{currentfill}{rgb}{0.174274,0.445044,0.557792}%
\pgfsetfillcolor{currentfill}%
\pgfsetfillopacity{0.700000}%
\pgfsetlinewidth{0.000000pt}%
\definecolor{currentstroke}{rgb}{0.000000,0.000000,0.000000}%
\pgfsetstrokecolor{currentstroke}%
\pgfsetdash{}{0pt}%
\pgfpathmoveto{\pgfqpoint{5.176909in}{2.951913in}}%
\pgfpathlineto{\pgfqpoint{5.190472in}{2.957152in}}%
\pgfpathlineto{\pgfqpoint{5.204049in}{2.962545in}}%
\pgfpathlineto{\pgfqpoint{5.217640in}{2.968092in}}%
\pgfpathlineto{\pgfqpoint{5.231245in}{2.973792in}}%
\pgfpathlineto{\pgfqpoint{5.238381in}{2.980127in}}%
\pgfpathlineto{\pgfqpoint{5.245512in}{2.986543in}}%
\pgfpathlineto{\pgfqpoint{5.252641in}{2.993043in}}%
\pgfpathlineto{\pgfqpoint{5.259766in}{2.999635in}}%
\pgfpathlineto{\pgfqpoint{5.246181in}{2.994423in}}%
\pgfpathlineto{\pgfqpoint{5.232610in}{2.989364in}}%
\pgfpathlineto{\pgfqpoint{5.219053in}{2.984458in}}%
\pgfpathlineto{\pgfqpoint{5.205510in}{2.979705in}}%
\pgfpathlineto{\pgfqpoint{5.198364in}{2.972616in}}%
\pgfpathlineto{\pgfqpoint{5.191216in}{2.965625in}}%
\pgfpathlineto{\pgfqpoint{5.184064in}{2.958726in}}%
\pgfpathlineto{\pgfqpoint{5.176909in}{2.951913in}}%
\pgfpathclose%
\pgfusepath{fill}%
\end{pgfscope}%
\begin{pgfscope}%
\pgfpathrectangle{\pgfqpoint{1.254980in}{0.150000in}}{\pgfqpoint{5.490039in}{5.490039in}}%
\pgfusepath{clip}%
\pgfsetbuttcap%
\pgfsetroundjoin%
\definecolor{currentfill}{rgb}{0.239346,0.300855,0.540844}%
\pgfsetfillcolor{currentfill}%
\pgfsetfillopacity{0.700000}%
\pgfsetlinewidth{0.000000pt}%
\definecolor{currentstroke}{rgb}{0.000000,0.000000,0.000000}%
\pgfsetstrokecolor{currentstroke}%
\pgfsetdash{}{0pt}%
\pgfpathmoveto{\pgfqpoint{2.694416in}{2.645626in}}%
\pgfpathlineto{\pgfqpoint{2.707617in}{2.627884in}}%
\pgfpathlineto{\pgfqpoint{2.720810in}{2.610425in}}%
\pgfpathlineto{\pgfqpoint{2.733994in}{2.593247in}}%
\pgfpathlineto{\pgfqpoint{2.747169in}{2.576347in}}%
\pgfpathlineto{\pgfqpoint{2.755256in}{2.583023in}}%
\pgfpathlineto{\pgfqpoint{2.763334in}{2.589827in}}%
\pgfpathlineto{\pgfqpoint{2.771401in}{2.596759in}}%
\pgfpathlineto{\pgfqpoint{2.779460in}{2.603817in}}%
\pgfpathlineto{\pgfqpoint{2.766309in}{2.620544in}}%
\pgfpathlineto{\pgfqpoint{2.753150in}{2.637549in}}%
\pgfpathlineto{\pgfqpoint{2.739983in}{2.654834in}}%
\pgfpathlineto{\pgfqpoint{2.726808in}{2.672402in}}%
\pgfpathlineto{\pgfqpoint{2.718725in}{2.665507in}}%
\pgfpathlineto{\pgfqpoint{2.710632in}{2.658745in}}%
\pgfpathlineto{\pgfqpoint{2.702529in}{2.652118in}}%
\pgfpathlineto{\pgfqpoint{2.694416in}{2.645626in}}%
\pgfpathclose%
\pgfusepath{fill}%
\end{pgfscope}%
\begin{pgfscope}%
\pgfpathrectangle{\pgfqpoint{1.254980in}{0.150000in}}{\pgfqpoint{5.490039in}{5.490039in}}%
\pgfusepath{clip}%
\pgfsetbuttcap%
\pgfsetroundjoin%
\definecolor{currentfill}{rgb}{0.277134,0.185228,0.489898}%
\pgfsetfillcolor{currentfill}%
\pgfsetfillopacity{0.700000}%
\pgfsetlinewidth{0.000000pt}%
\definecolor{currentstroke}{rgb}{0.000000,0.000000,0.000000}%
\pgfsetstrokecolor{currentstroke}%
\pgfsetdash{}{0pt}%
\pgfpathmoveto{\pgfqpoint{2.904730in}{2.394381in}}%
\pgfpathlineto{\pgfqpoint{2.917822in}{2.380885in}}%
\pgfpathlineto{\pgfqpoint{2.930910in}{2.367636in}}%
\pgfpathlineto{\pgfqpoint{2.943993in}{2.354632in}}%
\pgfpathlineto{\pgfqpoint{2.957071in}{2.341871in}}%
\pgfpathlineto{\pgfqpoint{2.965065in}{2.349269in}}%
\pgfpathlineto{\pgfqpoint{2.973051in}{2.356768in}}%
\pgfpathlineto{\pgfqpoint{2.981028in}{2.364366in}}%
\pgfpathlineto{\pgfqpoint{2.988998in}{2.372061in}}%
\pgfpathlineto{\pgfqpoint{2.975940in}{2.384655in}}%
\pgfpathlineto{\pgfqpoint{2.962878in}{2.397491in}}%
\pgfpathlineto{\pgfqpoint{2.949812in}{2.410572in}}%
\pgfpathlineto{\pgfqpoint{2.936742in}{2.423899in}}%
\pgfpathlineto{\pgfqpoint{2.928752in}{2.416361in}}%
\pgfpathlineto{\pgfqpoint{2.920753in}{2.408928in}}%
\pgfpathlineto{\pgfqpoint{2.912746in}{2.401601in}}%
\pgfpathlineto{\pgfqpoint{2.904730in}{2.394381in}}%
\pgfpathclose%
\pgfusepath{fill}%
\end{pgfscope}%
\begin{pgfscope}%
\pgfpathrectangle{\pgfqpoint{1.254980in}{0.150000in}}{\pgfqpoint{5.490039in}{5.490039in}}%
\pgfusepath{clip}%
\pgfsetbuttcap%
\pgfsetroundjoin%
\definecolor{currentfill}{rgb}{0.166617,0.463708,0.558119}%
\pgfsetfillcolor{currentfill}%
\pgfsetfillopacity{0.700000}%
\pgfsetlinewidth{0.000000pt}%
\definecolor{currentstroke}{rgb}{0.000000,0.000000,0.000000}%
\pgfsetstrokecolor{currentstroke}%
\pgfsetdash{}{0pt}%
\pgfpathmoveto{\pgfqpoint{5.259766in}{2.999635in}}%
\pgfpathlineto{\pgfqpoint{5.273365in}{3.005000in}}%
\pgfpathlineto{\pgfqpoint{5.286978in}{3.010517in}}%
\pgfpathlineto{\pgfqpoint{5.300606in}{3.016188in}}%
\pgfpathlineto{\pgfqpoint{5.314249in}{3.022011in}}%
\pgfpathlineto{\pgfqpoint{5.321350in}{3.028192in}}%
\pgfpathlineto{\pgfqpoint{5.328447in}{3.034470in}}%
\pgfpathlineto{\pgfqpoint{5.335541in}{3.040848in}}%
\pgfpathlineto{\pgfqpoint{5.342633in}{3.047334in}}%
\pgfpathlineto{\pgfqpoint{5.329012in}{3.042028in}}%
\pgfpathlineto{\pgfqpoint{5.315406in}{3.036873in}}%
\pgfpathlineto{\pgfqpoint{5.301814in}{3.031871in}}%
\pgfpathlineto{\pgfqpoint{5.288237in}{3.027021in}}%
\pgfpathlineto{\pgfqpoint{5.281123in}{3.020010in}}%
\pgfpathlineto{\pgfqpoint{5.274007in}{3.013113in}}%
\pgfpathlineto{\pgfqpoint{5.266888in}{3.006323in}}%
\pgfpathlineto{\pgfqpoint{5.259766in}{2.999635in}}%
\pgfpathclose%
\pgfusepath{fill}%
\end{pgfscope}%
\begin{pgfscope}%
\pgfpathrectangle{\pgfqpoint{1.254980in}{0.150000in}}{\pgfqpoint{5.490039in}{5.490039in}}%
\pgfusepath{clip}%
\pgfsetbuttcap%
\pgfsetroundjoin%
\definecolor{currentfill}{rgb}{0.159194,0.482237,0.558073}%
\pgfsetfillcolor{currentfill}%
\pgfsetfillopacity{0.700000}%
\pgfsetlinewidth{0.000000pt}%
\definecolor{currentstroke}{rgb}{0.000000,0.000000,0.000000}%
\pgfsetstrokecolor{currentstroke}%
\pgfsetdash{}{0pt}%
\pgfpathmoveto{\pgfqpoint{5.342633in}{3.047334in}}%
\pgfpathlineto{\pgfqpoint{5.356268in}{3.052792in}}%
\pgfpathlineto{\pgfqpoint{5.369917in}{3.058402in}}%
\pgfpathlineto{\pgfqpoint{5.383582in}{3.064164in}}%
\pgfpathlineto{\pgfqpoint{5.397261in}{3.070078in}}%
\pgfpathlineto{\pgfqpoint{5.404327in}{3.076141in}}%
\pgfpathlineto{\pgfqpoint{5.411390in}{3.082317in}}%
\pgfpathlineto{\pgfqpoint{5.418451in}{3.088611in}}%
\pgfpathlineto{\pgfqpoint{5.425510in}{3.095030in}}%
\pgfpathlineto{\pgfqpoint{5.411855in}{3.089662in}}%
\pgfpathlineto{\pgfqpoint{5.398214in}{3.084444in}}%
\pgfpathlineto{\pgfqpoint{5.384588in}{3.079379in}}%
\pgfpathlineto{\pgfqpoint{5.370976in}{3.074464in}}%
\pgfpathlineto{\pgfqpoint{5.363893in}{3.067491in}}%
\pgfpathlineto{\pgfqpoint{5.356809in}{3.060649in}}%
\pgfpathlineto{\pgfqpoint{5.349722in}{3.053932in}}%
\pgfpathlineto{\pgfqpoint{5.342633in}{3.047334in}}%
\pgfpathclose%
\pgfusepath{fill}%
\end{pgfscope}%
\begin{pgfscope}%
\pgfpathrectangle{\pgfqpoint{1.254980in}{0.150000in}}{\pgfqpoint{5.490039in}{5.490039in}}%
\pgfusepath{clip}%
\pgfsetbuttcap%
\pgfsetroundjoin%
\definecolor{currentfill}{rgb}{0.283091,0.110553,0.431554}%
\pgfsetfillcolor{currentfill}%
\pgfsetfillopacity{0.700000}%
\pgfsetlinewidth{0.000000pt}%
\definecolor{currentstroke}{rgb}{0.000000,0.000000,0.000000}%
\pgfsetstrokecolor{currentstroke}%
\pgfsetdash{}{0pt}%
\pgfpathmoveto{\pgfqpoint{3.769550in}{2.215417in}}%
\pgfpathlineto{\pgfqpoint{3.782599in}{2.213362in}}%
\pgfpathlineto{\pgfqpoint{3.795654in}{2.211487in}}%
\pgfpathlineto{\pgfqpoint{3.808715in}{2.209794in}}%
\pgfpathlineto{\pgfqpoint{3.821782in}{2.208280in}}%
\pgfpathlineto{\pgfqpoint{3.829448in}{2.217970in}}%
\pgfpathlineto{\pgfqpoint{3.837108in}{2.227659in}}%
\pgfpathlineto{\pgfqpoint{3.844764in}{2.237348in}}%
\pgfpathlineto{\pgfqpoint{3.852414in}{2.247037in}}%
\pgfpathlineto{\pgfqpoint{3.839356in}{2.248556in}}%
\pgfpathlineto{\pgfqpoint{3.826304in}{2.250254in}}%
\pgfpathlineto{\pgfqpoint{3.813258in}{2.252133in}}%
\pgfpathlineto{\pgfqpoint{3.800218in}{2.254193in}}%
\pgfpathlineto{\pgfqpoint{3.792559in}{2.244489in}}%
\pgfpathlineto{\pgfqpoint{3.784894in}{2.234792in}}%
\pgfpathlineto{\pgfqpoint{3.777225in}{2.225102in}}%
\pgfpathlineto{\pgfqpoint{3.769550in}{2.215417in}}%
\pgfpathclose%
\pgfusepath{fill}%
\end{pgfscope}%
\begin{pgfscope}%
\pgfpathrectangle{\pgfqpoint{1.254980in}{0.150000in}}{\pgfqpoint{5.490039in}{5.490039in}}%
\pgfusepath{clip}%
\pgfsetbuttcap%
\pgfsetroundjoin%
\definecolor{currentfill}{rgb}{0.150476,0.504369,0.557430}%
\pgfsetfillcolor{currentfill}%
\pgfsetfillopacity{0.700000}%
\pgfsetlinewidth{0.000000pt}%
\definecolor{currentstroke}{rgb}{0.000000,0.000000,0.000000}%
\pgfsetstrokecolor{currentstroke}%
\pgfsetdash{}{0pt}%
\pgfpathmoveto{\pgfqpoint{5.425510in}{3.095030in}}%
\pgfpathlineto{\pgfqpoint{5.439181in}{3.100549in}}%
\pgfpathlineto{\pgfqpoint{5.452866in}{3.106220in}}%
\pgfpathlineto{\pgfqpoint{5.466566in}{3.112041in}}%
\pgfpathlineto{\pgfqpoint{5.480281in}{3.118014in}}%
\pgfpathlineto{\pgfqpoint{5.487313in}{3.124000in}}%
\pgfpathlineto{\pgfqpoint{5.494344in}{3.130116in}}%
\pgfpathlineto{\pgfqpoint{5.501372in}{3.136369in}}%
\pgfpathlineto{\pgfqpoint{5.508400in}{3.142765in}}%
\pgfpathlineto{\pgfqpoint{5.494710in}{3.137366in}}%
\pgfpathlineto{\pgfqpoint{5.481035in}{3.132118in}}%
\pgfpathlineto{\pgfqpoint{5.467375in}{3.127020in}}%
\pgfpathlineto{\pgfqpoint{5.453730in}{3.122073in}}%
\pgfpathlineto{\pgfqpoint{5.446677in}{3.115094in}}%
\pgfpathlineto{\pgfqpoint{5.439623in}{3.108265in}}%
\pgfpathlineto{\pgfqpoint{5.432568in}{3.101579in}}%
\pgfpathlineto{\pgfqpoint{5.425510in}{3.095030in}}%
\pgfpathclose%
\pgfusepath{fill}%
\end{pgfscope}%
\begin{pgfscope}%
\pgfpathrectangle{\pgfqpoint{1.254980in}{0.150000in}}{\pgfqpoint{5.490039in}{5.490039in}}%
\pgfusepath{clip}%
\pgfsetbuttcap%
\pgfsetroundjoin%
\definecolor{currentfill}{rgb}{0.281446,0.084320,0.407414}%
\pgfsetfillcolor{currentfill}%
\pgfsetfillopacity{0.700000}%
\pgfsetlinewidth{0.000000pt}%
\definecolor{currentstroke}{rgb}{0.000000,0.000000,0.000000}%
\pgfsetstrokecolor{currentstroke}%
\pgfsetdash{}{0pt}%
\pgfpathmoveto{\pgfqpoint{3.333113in}{2.173247in}}%
\pgfpathlineto{\pgfqpoint{3.346123in}{2.166326in}}%
\pgfpathlineto{\pgfqpoint{3.359134in}{2.159609in}}%
\pgfpathlineto{\pgfqpoint{3.372147in}{2.153094in}}%
\pgfpathlineto{\pgfqpoint{3.385161in}{2.146780in}}%
\pgfpathlineto{\pgfqpoint{3.392978in}{2.155786in}}%
\pgfpathlineto{\pgfqpoint{3.400789in}{2.164834in}}%
\pgfpathlineto{\pgfqpoint{3.408595in}{2.173925in}}%
\pgfpathlineto{\pgfqpoint{3.416394in}{2.183058in}}%
\pgfpathlineto{\pgfqpoint{3.403393in}{2.189265in}}%
\pgfpathlineto{\pgfqpoint{3.390394in}{2.195672in}}%
\pgfpathlineto{\pgfqpoint{3.377397in}{2.202282in}}%
\pgfpathlineto{\pgfqpoint{3.364401in}{2.209095in}}%
\pgfpathlineto{\pgfqpoint{3.356589in}{2.200059in}}%
\pgfpathlineto{\pgfqpoint{3.348770in}{2.191072in}}%
\pgfpathlineto{\pgfqpoint{3.340945in}{2.182134in}}%
\pgfpathlineto{\pgfqpoint{3.333113in}{2.173247in}}%
\pgfpathclose%
\pgfusepath{fill}%
\end{pgfscope}%
\begin{pgfscope}%
\pgfpathrectangle{\pgfqpoint{1.254980in}{0.150000in}}{\pgfqpoint{5.490039in}{5.490039in}}%
\pgfusepath{clip}%
\pgfsetbuttcap%
\pgfsetroundjoin%
\definecolor{currentfill}{rgb}{0.223925,0.334994,0.548053}%
\pgfsetfillcolor{currentfill}%
\pgfsetfillopacity{0.700000}%
\pgfsetlinewidth{0.000000pt}%
\definecolor{currentstroke}{rgb}{0.000000,0.000000,0.000000}%
\pgfsetstrokecolor{currentstroke}%
\pgfsetdash{}{0pt}%
\pgfpathmoveto{\pgfqpoint{2.641517in}{2.719484in}}%
\pgfpathlineto{\pgfqpoint{2.654757in}{2.700581in}}%
\pgfpathlineto{\pgfqpoint{2.667986in}{2.681972in}}%
\pgfpathlineto{\pgfqpoint{2.681206in}{2.663655in}}%
\pgfpathlineto{\pgfqpoint{2.694416in}{2.645626in}}%
\pgfpathlineto{\pgfqpoint{2.702529in}{2.652118in}}%
\pgfpathlineto{\pgfqpoint{2.710632in}{2.658745in}}%
\pgfpathlineto{\pgfqpoint{2.718725in}{2.665507in}}%
\pgfpathlineto{\pgfqpoint{2.726808in}{2.672402in}}%
\pgfpathlineto{\pgfqpoint{2.713623in}{2.690256in}}%
\pgfpathlineto{\pgfqpoint{2.700430in}{2.708399in}}%
\pgfpathlineto{\pgfqpoint{2.687227in}{2.726832in}}%
\pgfpathlineto{\pgfqpoint{2.674015in}{2.745559in}}%
\pgfpathlineto{\pgfqpoint{2.665906in}{2.738828in}}%
\pgfpathlineto{\pgfqpoint{2.657787in}{2.732238in}}%
\pgfpathlineto{\pgfqpoint{2.649657in}{2.725790in}}%
\pgfpathlineto{\pgfqpoint{2.641517in}{2.719484in}}%
\pgfpathclose%
\pgfusepath{fill}%
\end{pgfscope}%
\begin{pgfscope}%
\pgfpathrectangle{\pgfqpoint{1.254980in}{0.150000in}}{\pgfqpoint{5.490039in}{5.490039in}}%
\pgfusepath{clip}%
\pgfsetbuttcap%
\pgfsetroundjoin%
\definecolor{currentfill}{rgb}{0.280255,0.165693,0.476498}%
\pgfsetfillcolor{currentfill}%
\pgfsetfillopacity{0.700000}%
\pgfsetlinewidth{0.000000pt}%
\definecolor{currentstroke}{rgb}{0.000000,0.000000,0.000000}%
\pgfsetstrokecolor{currentstroke}%
\pgfsetdash{}{0pt}%
\pgfpathmoveto{\pgfqpoint{2.957071in}{2.341871in}}%
\pgfpathlineto{\pgfqpoint{2.970146in}{2.329350in}}%
\pgfpathlineto{\pgfqpoint{2.983217in}{2.317069in}}%
\pgfpathlineto{\pgfqpoint{2.996284in}{2.305026in}}%
\pgfpathlineto{\pgfqpoint{3.009348in}{2.293218in}}%
\pgfpathlineto{\pgfqpoint{3.017321in}{2.300794in}}%
\pgfpathlineto{\pgfqpoint{3.025286in}{2.308463in}}%
\pgfpathlineto{\pgfqpoint{3.033243in}{2.316225in}}%
\pgfpathlineto{\pgfqpoint{3.041193in}{2.324077in}}%
\pgfpathlineto{\pgfqpoint{3.028149in}{2.335718in}}%
\pgfpathlineto{\pgfqpoint{3.015102in}{2.347595in}}%
\pgfpathlineto{\pgfqpoint{3.002052in}{2.359709in}}%
\pgfpathlineto{\pgfqpoint{2.988998in}{2.372061in}}%
\pgfpathlineto{\pgfqpoint{2.981028in}{2.364366in}}%
\pgfpathlineto{\pgfqpoint{2.973051in}{2.356768in}}%
\pgfpathlineto{\pgfqpoint{2.965065in}{2.349269in}}%
\pgfpathlineto{\pgfqpoint{2.957071in}{2.341871in}}%
\pgfpathclose%
\pgfusepath{fill}%
\end{pgfscope}%
\begin{pgfscope}%
\pgfpathrectangle{\pgfqpoint{1.254980in}{0.150000in}}{\pgfqpoint{5.490039in}{5.490039in}}%
\pgfusepath{clip}%
\pgfsetbuttcap%
\pgfsetroundjoin%
\definecolor{currentfill}{rgb}{0.143343,0.522773,0.556295}%
\pgfsetfillcolor{currentfill}%
\pgfsetfillopacity{0.700000}%
\pgfsetlinewidth{0.000000pt}%
\definecolor{currentstroke}{rgb}{0.000000,0.000000,0.000000}%
\pgfsetstrokecolor{currentstroke}%
\pgfsetdash{}{0pt}%
\pgfpathmoveto{\pgfqpoint{5.508400in}{3.142765in}}%
\pgfpathlineto{\pgfqpoint{5.522104in}{3.148314in}}%
\pgfpathlineto{\pgfqpoint{5.535824in}{3.154013in}}%
\pgfpathlineto{\pgfqpoint{5.549559in}{3.159862in}}%
\pgfpathlineto{\pgfqpoint{5.563309in}{3.165862in}}%
\pgfpathlineto{\pgfqpoint{5.570309in}{3.171816in}}%
\pgfpathlineto{\pgfqpoint{5.577308in}{3.177919in}}%
\pgfpathlineto{\pgfqpoint{5.584305in}{3.184179in}}%
\pgfpathlineto{\pgfqpoint{5.591303in}{3.190601in}}%
\pgfpathlineto{\pgfqpoint{5.577580in}{3.185204in}}%
\pgfpathlineto{\pgfqpoint{5.563872in}{3.179956in}}%
\pgfpathlineto{\pgfqpoint{5.550179in}{3.174858in}}%
\pgfpathlineto{\pgfqpoint{5.536501in}{3.169910in}}%
\pgfpathlineto{\pgfqpoint{5.529476in}{3.162876in}}%
\pgfpathlineto{\pgfqpoint{5.522452in}{3.156012in}}%
\pgfpathlineto{\pgfqpoint{5.515426in}{3.149311in}}%
\pgfpathlineto{\pgfqpoint{5.508400in}{3.142765in}}%
\pgfpathclose%
\pgfusepath{fill}%
\end{pgfscope}%
\begin{pgfscope}%
\pgfpathrectangle{\pgfqpoint{1.254980in}{0.150000in}}{\pgfqpoint{5.490039in}{5.490039in}}%
\pgfusepath{clip}%
\pgfsetbuttcap%
\pgfsetroundjoin%
\definecolor{currentfill}{rgb}{0.136408,0.541173,0.554483}%
\pgfsetfillcolor{currentfill}%
\pgfsetfillopacity{0.700000}%
\pgfsetlinewidth{0.000000pt}%
\definecolor{currentstroke}{rgb}{0.000000,0.000000,0.000000}%
\pgfsetstrokecolor{currentstroke}%
\pgfsetdash{}{0pt}%
\pgfpathmoveto{\pgfqpoint{5.591303in}{3.190601in}}%
\pgfpathlineto{\pgfqpoint{5.605041in}{3.196148in}}%
\pgfpathlineto{\pgfqpoint{5.618794in}{3.201844in}}%
\pgfpathlineto{\pgfqpoint{5.632563in}{3.207690in}}%
\pgfpathlineto{\pgfqpoint{5.646348in}{3.213685in}}%
\pgfpathlineto{\pgfqpoint{5.653316in}{3.219657in}}%
\pgfpathlineto{\pgfqpoint{5.660285in}{3.225799in}}%
\pgfpathlineto{\pgfqpoint{5.667253in}{3.232119in}}%
\pgfpathlineto{\pgfqpoint{5.674222in}{3.238622in}}%
\pgfpathlineto{\pgfqpoint{5.660467in}{3.233258in}}%
\pgfpathlineto{\pgfqpoint{5.646727in}{3.228042in}}%
\pgfpathlineto{\pgfqpoint{5.633002in}{3.222976in}}%
\pgfpathlineto{\pgfqpoint{5.619292in}{3.218058in}}%
\pgfpathlineto{\pgfqpoint{5.612294in}{3.210915in}}%
\pgfpathlineto{\pgfqpoint{5.605297in}{3.203962in}}%
\pgfpathlineto{\pgfqpoint{5.598300in}{3.197194in}}%
\pgfpathlineto{\pgfqpoint{5.591303in}{3.190601in}}%
\pgfpathclose%
\pgfusepath{fill}%
\end{pgfscope}%
\begin{pgfscope}%
\pgfpathrectangle{\pgfqpoint{1.254980in}{0.150000in}}{\pgfqpoint{5.490039in}{5.490039in}}%
\pgfusepath{clip}%
\pgfsetbuttcap%
\pgfsetroundjoin%
\definecolor{currentfill}{rgb}{0.280894,0.078907,0.402329}%
\pgfsetfillcolor{currentfill}%
\pgfsetfillopacity{0.700000}%
\pgfsetlinewidth{0.000000pt}%
\definecolor{currentstroke}{rgb}{0.000000,0.000000,0.000000}%
\pgfsetstrokecolor{currentstroke}%
\pgfsetdash{}{0pt}%
\pgfpathmoveto{\pgfqpoint{3.468419in}{2.160222in}}%
\pgfpathlineto{\pgfqpoint{3.481432in}{2.155005in}}%
\pgfpathlineto{\pgfqpoint{3.494447in}{2.149983in}}%
\pgfpathlineto{\pgfqpoint{3.507466in}{2.145154in}}%
\pgfpathlineto{\pgfqpoint{3.520487in}{2.140519in}}%
\pgfpathlineto{\pgfqpoint{3.528256in}{2.149872in}}%
\pgfpathlineto{\pgfqpoint{3.536020in}{2.159253in}}%
\pgfpathlineto{\pgfqpoint{3.543778in}{2.168660in}}%
\pgfpathlineto{\pgfqpoint{3.551530in}{2.178093in}}%
\pgfpathlineto{\pgfqpoint{3.538520in}{2.182650in}}%
\pgfpathlineto{\pgfqpoint{3.525513in}{2.187399in}}%
\pgfpathlineto{\pgfqpoint{3.512509in}{2.192342in}}%
\pgfpathlineto{\pgfqpoint{3.499509in}{2.197480in}}%
\pgfpathlineto{\pgfqpoint{3.491745in}{2.188115in}}%
\pgfpathlineto{\pgfqpoint{3.483975in}{2.178784in}}%
\pgfpathlineto{\pgfqpoint{3.476200in}{2.169486in}}%
\pgfpathlineto{\pgfqpoint{3.468419in}{2.160222in}}%
\pgfpathclose%
\pgfusepath{fill}%
\end{pgfscope}%
\begin{pgfscope}%
\pgfpathrectangle{\pgfqpoint{1.254980in}{0.150000in}}{\pgfqpoint{5.490039in}{5.490039in}}%
\pgfusepath{clip}%
\pgfsetbuttcap%
\pgfsetroundjoin%
\definecolor{currentfill}{rgb}{0.282656,0.100196,0.422160}%
\pgfsetfillcolor{currentfill}%
\pgfsetfillopacity{0.700000}%
\pgfsetlinewidth{0.000000pt}%
\definecolor{currentstroke}{rgb}{0.000000,0.000000,0.000000}%
\pgfsetstrokecolor{currentstroke}%
\pgfsetdash{}{0pt}%
\pgfpathmoveto{\pgfqpoint{3.197550in}{2.202138in}}%
\pgfpathlineto{\pgfqpoint{3.210572in}{2.193409in}}%
\pgfpathlineto{\pgfqpoint{3.223593in}{2.184894in}}%
\pgfpathlineto{\pgfqpoint{3.236614in}{2.176591in}}%
\pgfpathlineto{\pgfqpoint{3.249636in}{2.168499in}}%
\pgfpathlineto{\pgfqpoint{3.257508in}{2.177031in}}%
\pgfpathlineto{\pgfqpoint{3.265373in}{2.185624in}}%
\pgfpathlineto{\pgfqpoint{3.273232in}{2.194277in}}%
\pgfpathlineto{\pgfqpoint{3.281084in}{2.202988in}}%
\pgfpathlineto{\pgfqpoint{3.268078in}{2.210944in}}%
\pgfpathlineto{\pgfqpoint{3.255073in}{2.219111in}}%
\pgfpathlineto{\pgfqpoint{3.242068in}{2.227490in}}%
\pgfpathlineto{\pgfqpoint{3.229062in}{2.236083in}}%
\pgfpathlineto{\pgfqpoint{3.221194in}{2.227497in}}%
\pgfpathlineto{\pgfqpoint{3.213319in}{2.218977in}}%
\pgfpathlineto{\pgfqpoint{3.205438in}{2.210524in}}%
\pgfpathlineto{\pgfqpoint{3.197550in}{2.202138in}}%
\pgfpathclose%
\pgfusepath{fill}%
\end{pgfscope}%
\begin{pgfscope}%
\pgfpathrectangle{\pgfqpoint{1.254980in}{0.150000in}}{\pgfqpoint{5.490039in}{5.490039in}}%
\pgfusepath{clip}%
\pgfsetbuttcap%
\pgfsetroundjoin%
\definecolor{currentfill}{rgb}{0.129933,0.559582,0.551864}%
\pgfsetfillcolor{currentfill}%
\pgfsetfillopacity{0.700000}%
\pgfsetlinewidth{0.000000pt}%
\definecolor{currentstroke}{rgb}{0.000000,0.000000,0.000000}%
\pgfsetstrokecolor{currentstroke}%
\pgfsetdash{}{0pt}%
\pgfpathmoveto{\pgfqpoint{5.674222in}{3.238622in}}%
\pgfpathlineto{\pgfqpoint{5.687993in}{3.244135in}}%
\pgfpathlineto{\pgfqpoint{5.701779in}{3.249797in}}%
\pgfpathlineto{\pgfqpoint{5.715581in}{3.255607in}}%
\pgfpathlineto{\pgfqpoint{5.729399in}{3.261567in}}%
\pgfpathlineto{\pgfqpoint{5.736338in}{3.267613in}}%
\pgfpathlineto{\pgfqpoint{5.743278in}{3.273851in}}%
\pgfpathlineto{\pgfqpoint{5.750220in}{3.280288in}}%
\pgfpathlineto{\pgfqpoint{5.757162in}{3.286932in}}%
\pgfpathlineto{\pgfqpoint{5.743376in}{3.281632in}}%
\pgfpathlineto{\pgfqpoint{5.729605in}{3.276480in}}%
\pgfpathlineto{\pgfqpoint{5.715849in}{3.271475in}}%
\pgfpathlineto{\pgfqpoint{5.702109in}{3.266619in}}%
\pgfpathlineto{\pgfqpoint{5.695135in}{3.259308in}}%
\pgfpathlineto{\pgfqpoint{5.688163in}{3.252210in}}%
\pgfpathlineto{\pgfqpoint{5.681192in}{3.245317in}}%
\pgfpathlineto{\pgfqpoint{5.674222in}{3.238622in}}%
\pgfpathclose%
\pgfusepath{fill}%
\end{pgfscope}%
\begin{pgfscope}%
\pgfpathrectangle{\pgfqpoint{1.254980in}{0.150000in}}{\pgfqpoint{5.490039in}{5.490039in}}%
\pgfusepath{clip}%
\pgfsetbuttcap%
\pgfsetroundjoin%
\definecolor{currentfill}{rgb}{0.282327,0.094955,0.417331}%
\pgfsetfillcolor{currentfill}%
\pgfsetfillopacity{0.700000}%
\pgfsetlinewidth{0.000000pt}%
\definecolor{currentstroke}{rgb}{0.000000,0.000000,0.000000}%
\pgfsetstrokecolor{currentstroke}%
\pgfsetdash{}{0pt}%
\pgfpathmoveto{\pgfqpoint{3.686621in}{2.186874in}}%
\pgfpathlineto{\pgfqpoint{3.699658in}{2.184063in}}%
\pgfpathlineto{\pgfqpoint{3.712700in}{2.181437in}}%
\pgfpathlineto{\pgfqpoint{3.725748in}{2.178994in}}%
\pgfpathlineto{\pgfqpoint{3.738801in}{2.176733in}}%
\pgfpathlineto{\pgfqpoint{3.746496in}{2.186397in}}%
\pgfpathlineto{\pgfqpoint{3.754186in}{2.196065in}}%
\pgfpathlineto{\pgfqpoint{3.761871in}{2.205738in}}%
\pgfpathlineto{\pgfqpoint{3.769550in}{2.215417in}}%
\pgfpathlineto{\pgfqpoint{3.756507in}{2.217654in}}%
\pgfpathlineto{\pgfqpoint{3.743469in}{2.220074in}}%
\pgfpathlineto{\pgfqpoint{3.730436in}{2.222677in}}%
\pgfpathlineto{\pgfqpoint{3.717408in}{2.225465in}}%
\pgfpathlineto{\pgfqpoint{3.709719in}{2.215799in}}%
\pgfpathlineto{\pgfqpoint{3.702025in}{2.206145in}}%
\pgfpathlineto{\pgfqpoint{3.694325in}{2.196503in}}%
\pgfpathlineto{\pgfqpoint{3.686621in}{2.186874in}}%
\pgfpathclose%
\pgfusepath{fill}%
\end{pgfscope}%
\begin{pgfscope}%
\pgfpathrectangle{\pgfqpoint{1.254980in}{0.150000in}}{\pgfqpoint{5.490039in}{5.490039in}}%
\pgfusepath{clip}%
\pgfsetbuttcap%
\pgfsetroundjoin%
\definecolor{currentfill}{rgb}{0.282623,0.140926,0.457517}%
\pgfsetfillcolor{currentfill}%
\pgfsetfillopacity{0.700000}%
\pgfsetlinewidth{0.000000pt}%
\definecolor{currentstroke}{rgb}{0.000000,0.000000,0.000000}%
\pgfsetstrokecolor{currentstroke}%
\pgfsetdash{}{0pt}%
\pgfpathmoveto{\pgfqpoint{3.009348in}{2.293218in}}%
\pgfpathlineto{\pgfqpoint{3.022408in}{2.281644in}}%
\pgfpathlineto{\pgfqpoint{3.035466in}{2.270302in}}%
\pgfpathlineto{\pgfqpoint{3.048521in}{2.259191in}}%
\pgfpathlineto{\pgfqpoint{3.061574in}{2.248308in}}%
\pgfpathlineto{\pgfqpoint{3.069527in}{2.256061in}}%
\pgfpathlineto{\pgfqpoint{3.077472in}{2.263900in}}%
\pgfpathlineto{\pgfqpoint{3.085410in}{2.271824in}}%
\pgfpathlineto{\pgfqpoint{3.093341in}{2.279832in}}%
\pgfpathlineto{\pgfqpoint{3.080308in}{2.290549in}}%
\pgfpathlineto{\pgfqpoint{3.067272in}{2.301494in}}%
\pgfpathlineto{\pgfqpoint{3.054234in}{2.312670in}}%
\pgfpathlineto{\pgfqpoint{3.041193in}{2.324077in}}%
\pgfpathlineto{\pgfqpoint{3.033243in}{2.316225in}}%
\pgfpathlineto{\pgfqpoint{3.025286in}{2.308463in}}%
\pgfpathlineto{\pgfqpoint{3.017321in}{2.300794in}}%
\pgfpathlineto{\pgfqpoint{3.009348in}{2.293218in}}%
\pgfpathclose%
\pgfusepath{fill}%
\end{pgfscope}%
\begin{pgfscope}%
\pgfpathrectangle{\pgfqpoint{1.254980in}{0.150000in}}{\pgfqpoint{5.490039in}{5.490039in}}%
\pgfusepath{clip}%
\pgfsetbuttcap%
\pgfsetroundjoin%
\definecolor{currentfill}{rgb}{0.208623,0.367752,0.552675}%
\pgfsetfillcolor{currentfill}%
\pgfsetfillopacity{0.700000}%
\pgfsetlinewidth{0.000000pt}%
\definecolor{currentstroke}{rgb}{0.000000,0.000000,0.000000}%
\pgfsetstrokecolor{currentstroke}%
\pgfsetdash{}{0pt}%
\pgfpathmoveto{\pgfqpoint{2.588455in}{2.798099in}}%
\pgfpathlineto{\pgfqpoint{2.601737in}{2.777989in}}%
\pgfpathlineto{\pgfqpoint{2.615008in}{2.758186in}}%
\pgfpathlineto{\pgfqpoint{2.628268in}{2.738685in}}%
\pgfpathlineto{\pgfqpoint{2.641517in}{2.719484in}}%
\pgfpathlineto{\pgfqpoint{2.649657in}{2.725790in}}%
\pgfpathlineto{\pgfqpoint{2.657787in}{2.732238in}}%
\pgfpathlineto{\pgfqpoint{2.665906in}{2.738828in}}%
\pgfpathlineto{\pgfqpoint{2.674015in}{2.745559in}}%
\pgfpathlineto{\pgfqpoint{2.660792in}{2.764584in}}%
\pgfpathlineto{\pgfqpoint{2.647559in}{2.783908in}}%
\pgfpathlineto{\pgfqpoint{2.634316in}{2.803534in}}%
\pgfpathlineto{\pgfqpoint{2.621062in}{2.823467in}}%
\pgfpathlineto{\pgfqpoint{2.612926in}{2.816902in}}%
\pgfpathlineto{\pgfqpoint{2.604780in}{2.810484in}}%
\pgfpathlineto{\pgfqpoint{2.596623in}{2.804217in}}%
\pgfpathlineto{\pgfqpoint{2.588455in}{2.798099in}}%
\pgfpathclose%
\pgfusepath{fill}%
\end{pgfscope}%
\begin{pgfscope}%
\pgfpathrectangle{\pgfqpoint{1.254980in}{0.150000in}}{\pgfqpoint{5.490039in}{5.490039in}}%
\pgfusepath{clip}%
\pgfsetbuttcap%
\pgfsetroundjoin%
\definecolor{currentfill}{rgb}{0.124395,0.578002,0.548287}%
\pgfsetfillcolor{currentfill}%
\pgfsetfillopacity{0.700000}%
\pgfsetlinewidth{0.000000pt}%
\definecolor{currentstroke}{rgb}{0.000000,0.000000,0.000000}%
\pgfsetstrokecolor{currentstroke}%
\pgfsetdash{}{0pt}%
\pgfpathmoveto{\pgfqpoint{5.757162in}{3.286932in}}%
\pgfpathlineto{\pgfqpoint{5.770965in}{3.292379in}}%
\pgfpathlineto{\pgfqpoint{5.784783in}{3.297975in}}%
\pgfpathlineto{\pgfqpoint{5.798616in}{3.303720in}}%
\pgfpathlineto{\pgfqpoint{5.812466in}{3.309612in}}%
\pgfpathlineto{\pgfqpoint{5.819379in}{3.315792in}}%
\pgfpathlineto{\pgfqpoint{5.826293in}{3.322188in}}%
\pgfpathlineto{\pgfqpoint{5.833210in}{3.328806in}}%
\pgfpathlineto{\pgfqpoint{5.819385in}{3.323428in}}%
\pgfpathlineto{\pgfqpoint{5.805575in}{3.318196in}}%
\pgfpathlineto{\pgfqpoint{5.791782in}{3.313112in}}%
\pgfpathlineto{\pgfqpoint{5.778004in}{3.308176in}}%
\pgfpathlineto{\pgfqpoint{5.771054in}{3.300868in}}%
\pgfpathlineto{\pgfqpoint{5.764107in}{3.293789in}}%
\pgfpathlineto{\pgfqpoint{5.757162in}{3.286932in}}%
\pgfpathclose%
\pgfusepath{fill}%
\end{pgfscope}%
\begin{pgfscope}%
\pgfpathrectangle{\pgfqpoint{1.254980in}{0.150000in}}{\pgfqpoint{5.490039in}{5.490039in}}%
\pgfusepath{clip}%
\pgfsetbuttcap%
\pgfsetroundjoin%
\definecolor{currentfill}{rgb}{0.281446,0.084320,0.407414}%
\pgfsetfillcolor{currentfill}%
\pgfsetfillopacity{0.700000}%
\pgfsetlinewidth{0.000000pt}%
\definecolor{currentstroke}{rgb}{0.000000,0.000000,0.000000}%
\pgfsetstrokecolor{currentstroke}%
\pgfsetdash{}{0pt}%
\pgfpathmoveto{\pgfqpoint{3.603607in}{2.161777in}}%
\pgfpathlineto{\pgfqpoint{3.616636in}{2.158171in}}%
\pgfpathlineto{\pgfqpoint{3.629669in}{2.154752in}}%
\pgfpathlineto{\pgfqpoint{3.642707in}{2.151520in}}%
\pgfpathlineto{\pgfqpoint{3.655750in}{2.148473in}}%
\pgfpathlineto{\pgfqpoint{3.663475in}{2.158055in}}%
\pgfpathlineto{\pgfqpoint{3.671196in}{2.167650in}}%
\pgfpathlineto{\pgfqpoint{3.678911in}{2.177256in}}%
\pgfpathlineto{\pgfqpoint{3.686621in}{2.186874in}}%
\pgfpathlineto{\pgfqpoint{3.673588in}{2.189869in}}%
\pgfpathlineto{\pgfqpoint{3.660560in}{2.193051in}}%
\pgfpathlineto{\pgfqpoint{3.647537in}{2.196418in}}%
\pgfpathlineto{\pgfqpoint{3.634519in}{2.199973in}}%
\pgfpathlineto{\pgfqpoint{3.626799in}{2.190396in}}%
\pgfpathlineto{\pgfqpoint{3.619073in}{2.180838in}}%
\pgfpathlineto{\pgfqpoint{3.611343in}{2.171298in}}%
\pgfpathlineto{\pgfqpoint{3.603607in}{2.161777in}}%
\pgfpathclose%
\pgfusepath{fill}%
\end{pgfscope}%
\begin{pgfscope}%
\pgfpathrectangle{\pgfqpoint{1.254980in}{0.150000in}}{\pgfqpoint{5.490039in}{5.490039in}}%
\pgfusepath{clip}%
\pgfsetbuttcap%
\pgfsetroundjoin%
\definecolor{currentfill}{rgb}{0.270595,0.214069,0.507052}%
\pgfsetfillcolor{currentfill}%
\pgfsetfillopacity{0.700000}%
\pgfsetlinewidth{0.000000pt}%
\definecolor{currentstroke}{rgb}{0.000000,0.000000,0.000000}%
\pgfsetstrokecolor{currentstroke}%
\pgfsetdash{}{0pt}%
\pgfpathmoveto{\pgfqpoint{4.236175in}{2.403251in}}%
\pgfpathlineto{\pgfqpoint{4.249365in}{2.405071in}}%
\pgfpathlineto{\pgfqpoint{4.262565in}{2.407058in}}%
\pgfpathlineto{\pgfqpoint{4.275774in}{2.409212in}}%
\pgfpathlineto{\pgfqpoint{4.288992in}{2.411533in}}%
\pgfpathlineto{\pgfqpoint{4.296509in}{2.420572in}}%
\pgfpathlineto{\pgfqpoint{4.304020in}{2.429592in}}%
\pgfpathlineto{\pgfqpoint{4.311526in}{2.438593in}}%
\pgfpathlineto{\pgfqpoint{4.319027in}{2.447580in}}%
\pgfpathlineto{\pgfqpoint{4.305816in}{2.445404in}}%
\pgfpathlineto{\pgfqpoint{4.292616in}{2.443395in}}%
\pgfpathlineto{\pgfqpoint{4.279425in}{2.441553in}}%
\pgfpathlineto{\pgfqpoint{4.266243in}{2.439879in}}%
\pgfpathlineto{\pgfqpoint{4.258734in}{2.430737in}}%
\pgfpathlineto{\pgfqpoint{4.251219in}{2.421587in}}%
\pgfpathlineto{\pgfqpoint{4.243700in}{2.412425in}}%
\pgfpathlineto{\pgfqpoint{4.236175in}{2.403251in}}%
\pgfpathclose%
\pgfusepath{fill}%
\end{pgfscope}%
\begin{pgfscope}%
\pgfpathrectangle{\pgfqpoint{1.254980in}{0.150000in}}{\pgfqpoint{5.490039in}{5.490039in}}%
\pgfusepath{clip}%
\pgfsetbuttcap%
\pgfsetroundjoin%
\definecolor{currentfill}{rgb}{0.263663,0.237631,0.518762}%
\pgfsetfillcolor{currentfill}%
\pgfsetfillopacity{0.700000}%
\pgfsetlinewidth{0.000000pt}%
\definecolor{currentstroke}{rgb}{0.000000,0.000000,0.000000}%
\pgfsetstrokecolor{currentstroke}%
\pgfsetdash{}{0pt}%
\pgfpathmoveto{\pgfqpoint{4.319027in}{2.447580in}}%
\pgfpathlineto{\pgfqpoint{4.332247in}{2.449922in}}%
\pgfpathlineto{\pgfqpoint{4.345477in}{2.452430in}}%
\pgfpathlineto{\pgfqpoint{4.358717in}{2.455103in}}%
\pgfpathlineto{\pgfqpoint{4.371968in}{2.457942in}}%
\pgfpathlineto{\pgfqpoint{4.379455in}{2.466751in}}%
\pgfpathlineto{\pgfqpoint{4.386937in}{2.475542in}}%
\pgfpathlineto{\pgfqpoint{4.394414in}{2.484317in}}%
\pgfpathlineto{\pgfqpoint{4.401885in}{2.493077in}}%
\pgfpathlineto{\pgfqpoint{4.388644in}{2.490413in}}%
\pgfpathlineto{\pgfqpoint{4.375412in}{2.487913in}}%
\pgfpathlineto{\pgfqpoint{4.362190in}{2.485578in}}%
\pgfpathlineto{\pgfqpoint{4.348979in}{2.483409in}}%
\pgfpathlineto{\pgfqpoint{4.341498in}{2.474465in}}%
\pgfpathlineto{\pgfqpoint{4.334013in}{2.465513in}}%
\pgfpathlineto{\pgfqpoint{4.326522in}{2.456552in}}%
\pgfpathlineto{\pgfqpoint{4.319027in}{2.447580in}}%
\pgfpathclose%
\pgfusepath{fill}%
\end{pgfscope}%
\begin{pgfscope}%
\pgfpathrectangle{\pgfqpoint{1.254980in}{0.150000in}}{\pgfqpoint{5.490039in}{5.490039in}}%
\pgfusepath{clip}%
\pgfsetbuttcap%
\pgfsetroundjoin%
\definecolor{currentfill}{rgb}{0.275191,0.194905,0.496005}%
\pgfsetfillcolor{currentfill}%
\pgfsetfillopacity{0.700000}%
\pgfsetlinewidth{0.000000pt}%
\definecolor{currentstroke}{rgb}{0.000000,0.000000,0.000000}%
\pgfsetstrokecolor{currentstroke}%
\pgfsetdash{}{0pt}%
\pgfpathmoveto{\pgfqpoint{4.153326in}{2.360333in}}%
\pgfpathlineto{\pgfqpoint{4.166488in}{2.361594in}}%
\pgfpathlineto{\pgfqpoint{4.179658in}{2.363025in}}%
\pgfpathlineto{\pgfqpoint{4.192838in}{2.364625in}}%
\pgfpathlineto{\pgfqpoint{4.206026in}{2.366393in}}%
\pgfpathlineto{\pgfqpoint{4.213571in}{2.375635in}}%
\pgfpathlineto{\pgfqpoint{4.221111in}{2.384858in}}%
\pgfpathlineto{\pgfqpoint{4.228646in}{2.394063in}}%
\pgfpathlineto{\pgfqpoint{4.236175in}{2.403251in}}%
\pgfpathlineto{\pgfqpoint{4.222995in}{2.401600in}}%
\pgfpathlineto{\pgfqpoint{4.209823in}{2.400117in}}%
\pgfpathlineto{\pgfqpoint{4.196661in}{2.398803in}}%
\pgfpathlineto{\pgfqpoint{4.183507in}{2.397659in}}%
\pgfpathlineto{\pgfqpoint{4.175969in}{2.388343in}}%
\pgfpathlineto{\pgfqpoint{4.168427in}{2.379018in}}%
\pgfpathlineto{\pgfqpoint{4.160879in}{2.369682in}}%
\pgfpathlineto{\pgfqpoint{4.153326in}{2.360333in}}%
\pgfpathclose%
\pgfusepath{fill}%
\end{pgfscope}%
\begin{pgfscope}%
\pgfpathrectangle{\pgfqpoint{1.254980in}{0.150000in}}{\pgfqpoint{5.490039in}{5.490039in}}%
\pgfusepath{clip}%
\pgfsetbuttcap%
\pgfsetroundjoin%
\definecolor{currentfill}{rgb}{0.255645,0.260703,0.528312}%
\pgfsetfillcolor{currentfill}%
\pgfsetfillopacity{0.700000}%
\pgfsetlinewidth{0.000000pt}%
\definecolor{currentstroke}{rgb}{0.000000,0.000000,0.000000}%
\pgfsetstrokecolor{currentstroke}%
\pgfsetdash{}{0pt}%
\pgfpathmoveto{\pgfqpoint{4.401885in}{2.493077in}}%
\pgfpathlineto{\pgfqpoint{4.415138in}{2.495907in}}%
\pgfpathlineto{\pgfqpoint{4.428400in}{2.498901in}}%
\pgfpathlineto{\pgfqpoint{4.441674in}{2.502060in}}%
\pgfpathlineto{\pgfqpoint{4.454958in}{2.505382in}}%
\pgfpathlineto{\pgfqpoint{4.462415in}{2.513939in}}%
\pgfpathlineto{\pgfqpoint{4.469867in}{2.522480in}}%
\pgfpathlineto{\pgfqpoint{4.477314in}{2.531008in}}%
\pgfpathlineto{\pgfqpoint{4.484755in}{2.539524in}}%
\pgfpathlineto{\pgfqpoint{4.471480in}{2.536405in}}%
\pgfpathlineto{\pgfqpoint{4.458216in}{2.533449in}}%
\pgfpathlineto{\pgfqpoint{4.444963in}{2.530657in}}%
\pgfpathlineto{\pgfqpoint{4.431720in}{2.528028in}}%
\pgfpathlineto{\pgfqpoint{4.424269in}{2.519300in}}%
\pgfpathlineto{\pgfqpoint{4.416813in}{2.510566in}}%
\pgfpathlineto{\pgfqpoint{4.409352in}{2.501827in}}%
\pgfpathlineto{\pgfqpoint{4.401885in}{2.493077in}}%
\pgfpathclose%
\pgfusepath{fill}%
\end{pgfscope}%
\begin{pgfscope}%
\pgfpathrectangle{\pgfqpoint{1.254980in}{0.150000in}}{\pgfqpoint{5.490039in}{5.490039in}}%
\pgfusepath{clip}%
\pgfsetbuttcap%
\pgfsetroundjoin%
\definecolor{currentfill}{rgb}{0.278826,0.175490,0.483397}%
\pgfsetfillcolor{currentfill}%
\pgfsetfillopacity{0.700000}%
\pgfsetlinewidth{0.000000pt}%
\definecolor{currentstroke}{rgb}{0.000000,0.000000,0.000000}%
\pgfsetstrokecolor{currentstroke}%
\pgfsetdash{}{0pt}%
\pgfpathmoveto{\pgfqpoint{4.070472in}{2.319085in}}%
\pgfpathlineto{\pgfqpoint{4.083607in}{2.319752in}}%
\pgfpathlineto{\pgfqpoint{4.096751in}{2.320591in}}%
\pgfpathlineto{\pgfqpoint{4.109903in}{2.321601in}}%
\pgfpathlineto{\pgfqpoint{4.123064in}{2.322782in}}%
\pgfpathlineto{\pgfqpoint{4.130637in}{2.332196in}}%
\pgfpathlineto{\pgfqpoint{4.138205in}{2.341591in}}%
\pgfpathlineto{\pgfqpoint{4.145768in}{2.350970in}}%
\pgfpathlineto{\pgfqpoint{4.153326in}{2.360333in}}%
\pgfpathlineto{\pgfqpoint{4.140173in}{2.359241in}}%
\pgfpathlineto{\pgfqpoint{4.127029in}{2.358320in}}%
\pgfpathlineto{\pgfqpoint{4.113893in}{2.357570in}}%
\pgfpathlineto{\pgfqpoint{4.100765in}{2.356991in}}%
\pgfpathlineto{\pgfqpoint{4.093199in}{2.347529in}}%
\pgfpathlineto{\pgfqpoint{4.085628in}{2.338058in}}%
\pgfpathlineto{\pgfqpoint{4.078052in}{2.328577in}}%
\pgfpathlineto{\pgfqpoint{4.070472in}{2.319085in}}%
\pgfpathclose%
\pgfusepath{fill}%
\end{pgfscope}%
\begin{pgfscope}%
\pgfpathrectangle{\pgfqpoint{1.254980in}{0.150000in}}{\pgfqpoint{5.490039in}{5.490039in}}%
\pgfusepath{clip}%
\pgfsetbuttcap%
\pgfsetroundjoin%
\definecolor{currentfill}{rgb}{0.283187,0.125848,0.444960}%
\pgfsetfillcolor{currentfill}%
\pgfsetfillopacity{0.700000}%
\pgfsetlinewidth{0.000000pt}%
\definecolor{currentstroke}{rgb}{0.000000,0.000000,0.000000}%
\pgfsetstrokecolor{currentstroke}%
\pgfsetdash{}{0pt}%
\pgfpathmoveto{\pgfqpoint{3.061574in}{2.248308in}}%
\pgfpathlineto{\pgfqpoint{3.074624in}{2.237653in}}%
\pgfpathlineto{\pgfqpoint{3.087672in}{2.227223in}}%
\pgfpathlineto{\pgfqpoint{3.100718in}{2.217017in}}%
\pgfpathlineto{\pgfqpoint{3.113763in}{2.207034in}}%
\pgfpathlineto{\pgfqpoint{3.121697in}{2.214962in}}%
\pgfpathlineto{\pgfqpoint{3.129624in}{2.222970in}}%
\pgfpathlineto{\pgfqpoint{3.137543in}{2.231056in}}%
\pgfpathlineto{\pgfqpoint{3.145456in}{2.239219in}}%
\pgfpathlineto{\pgfqpoint{3.132429in}{2.249037in}}%
\pgfpathlineto{\pgfqpoint{3.119402in}{2.259078in}}%
\pgfpathlineto{\pgfqpoint{3.106372in}{2.269342in}}%
\pgfpathlineto{\pgfqpoint{3.093341in}{2.279832in}}%
\pgfpathlineto{\pgfqpoint{3.085410in}{2.271824in}}%
\pgfpathlineto{\pgfqpoint{3.077472in}{2.263900in}}%
\pgfpathlineto{\pgfqpoint{3.069527in}{2.256061in}}%
\pgfpathlineto{\pgfqpoint{3.061574in}{2.248308in}}%
\pgfpathclose%
\pgfusepath{fill}%
\end{pgfscope}%
\begin{pgfscope}%
\pgfpathrectangle{\pgfqpoint{1.254980in}{0.150000in}}{\pgfqpoint{5.490039in}{5.490039in}}%
\pgfusepath{clip}%
\pgfsetbuttcap%
\pgfsetroundjoin%
\definecolor{currentfill}{rgb}{0.246811,0.283237,0.535941}%
\pgfsetfillcolor{currentfill}%
\pgfsetfillopacity{0.700000}%
\pgfsetlinewidth{0.000000pt}%
\definecolor{currentstroke}{rgb}{0.000000,0.000000,0.000000}%
\pgfsetstrokecolor{currentstroke}%
\pgfsetdash{}{0pt}%
\pgfpathmoveto{\pgfqpoint{4.484755in}{2.539524in}}%
\pgfpathlineto{\pgfqpoint{4.498041in}{2.542807in}}%
\pgfpathlineto{\pgfqpoint{4.511337in}{2.546253in}}%
\pgfpathlineto{\pgfqpoint{4.524645in}{2.549862in}}%
\pgfpathlineto{\pgfqpoint{4.537964in}{2.553633in}}%
\pgfpathlineto{\pgfqpoint{4.545391in}{2.561921in}}%
\pgfpathlineto{\pgfqpoint{4.552812in}{2.570196in}}%
\pgfpathlineto{\pgfqpoint{4.560228in}{2.578461in}}%
\pgfpathlineto{\pgfqpoint{4.567639in}{2.586720in}}%
\pgfpathlineto{\pgfqpoint{4.554330in}{2.583180in}}%
\pgfpathlineto{\pgfqpoint{4.541032in}{2.579802in}}%
\pgfpathlineto{\pgfqpoint{4.527745in}{2.576587in}}%
\pgfpathlineto{\pgfqpoint{4.514470in}{2.573534in}}%
\pgfpathlineto{\pgfqpoint{4.507049in}{2.565035in}}%
\pgfpathlineto{\pgfqpoint{4.499623in}{2.556535in}}%
\pgfpathlineto{\pgfqpoint{4.492192in}{2.548033in}}%
\pgfpathlineto{\pgfqpoint{4.484755in}{2.539524in}}%
\pgfpathclose%
\pgfusepath{fill}%
\end{pgfscope}%
\begin{pgfscope}%
\pgfpathrectangle{\pgfqpoint{1.254980in}{0.150000in}}{\pgfqpoint{5.490039in}{5.490039in}}%
\pgfusepath{clip}%
\pgfsetbuttcap%
\pgfsetroundjoin%
\definecolor{currentfill}{rgb}{0.280267,0.073417,0.397163}%
\pgfsetfillcolor{currentfill}%
\pgfsetfillopacity{0.700000}%
\pgfsetlinewidth{0.000000pt}%
\definecolor{currentstroke}{rgb}{0.000000,0.000000,0.000000}%
\pgfsetstrokecolor{currentstroke}%
\pgfsetdash{}{0pt}%
\pgfpathmoveto{\pgfqpoint{3.385161in}{2.146780in}}%
\pgfpathlineto{\pgfqpoint{3.398177in}{2.140667in}}%
\pgfpathlineto{\pgfqpoint{3.411195in}{2.134752in}}%
\pgfpathlineto{\pgfqpoint{3.424215in}{2.129036in}}%
\pgfpathlineto{\pgfqpoint{3.437238in}{2.123516in}}%
\pgfpathlineto{\pgfqpoint{3.445042in}{2.132639in}}%
\pgfpathlineto{\pgfqpoint{3.452840in}{2.141798in}}%
\pgfpathlineto{\pgfqpoint{3.460633in}{2.150992in}}%
\pgfpathlineto{\pgfqpoint{3.468419in}{2.160222in}}%
\pgfpathlineto{\pgfqpoint{3.455409in}{2.165635in}}%
\pgfpathlineto{\pgfqpoint{3.442402in}{2.171244in}}%
\pgfpathlineto{\pgfqpoint{3.429397in}{2.177052in}}%
\pgfpathlineto{\pgfqpoint{3.416394in}{2.183058in}}%
\pgfpathlineto{\pgfqpoint{3.408595in}{2.173925in}}%
\pgfpathlineto{\pgfqpoint{3.400789in}{2.164834in}}%
\pgfpathlineto{\pgfqpoint{3.392978in}{2.155786in}}%
\pgfpathlineto{\pgfqpoint{3.385161in}{2.146780in}}%
\pgfpathclose%
\pgfusepath{fill}%
\end{pgfscope}%
\begin{pgfscope}%
\pgfpathrectangle{\pgfqpoint{1.254980in}{0.150000in}}{\pgfqpoint{5.490039in}{5.490039in}}%
\pgfusepath{clip}%
\pgfsetbuttcap%
\pgfsetroundjoin%
\definecolor{currentfill}{rgb}{0.281412,0.155834,0.469201}%
\pgfsetfillcolor{currentfill}%
\pgfsetfillopacity{0.700000}%
\pgfsetlinewidth{0.000000pt}%
\definecolor{currentstroke}{rgb}{0.000000,0.000000,0.000000}%
\pgfsetstrokecolor{currentstroke}%
\pgfsetdash{}{0pt}%
\pgfpathmoveto{\pgfqpoint{3.987604in}{2.279789in}}%
\pgfpathlineto{\pgfqpoint{4.000716in}{2.279826in}}%
\pgfpathlineto{\pgfqpoint{4.013835in}{2.280037in}}%
\pgfpathlineto{\pgfqpoint{4.026963in}{2.280421in}}%
\pgfpathlineto{\pgfqpoint{4.040098in}{2.280978in}}%
\pgfpathlineto{\pgfqpoint{4.047699in}{2.290527in}}%
\pgfpathlineto{\pgfqpoint{4.055295in}{2.300061in}}%
\pgfpathlineto{\pgfqpoint{4.062886in}{2.309580in}}%
\pgfpathlineto{\pgfqpoint{4.070472in}{2.319085in}}%
\pgfpathlineto{\pgfqpoint{4.057344in}{2.318589in}}%
\pgfpathlineto{\pgfqpoint{4.044225in}{2.318266in}}%
\pgfpathlineto{\pgfqpoint{4.031113in}{2.318116in}}%
\pgfpathlineto{\pgfqpoint{4.018009in}{2.318139in}}%
\pgfpathlineto{\pgfqpoint{4.010415in}{2.308563in}}%
\pgfpathlineto{\pgfqpoint{4.002816in}{2.298980in}}%
\pgfpathlineto{\pgfqpoint{3.995212in}{2.289389in}}%
\pgfpathlineto{\pgfqpoint{3.987604in}{2.279789in}}%
\pgfpathclose%
\pgfusepath{fill}%
\end{pgfscope}%
\begin{pgfscope}%
\pgfpathrectangle{\pgfqpoint{1.254980in}{0.150000in}}{\pgfqpoint{5.490039in}{5.490039in}}%
\pgfusepath{clip}%
\pgfsetbuttcap%
\pgfsetroundjoin%
\definecolor{currentfill}{rgb}{0.281446,0.084320,0.407414}%
\pgfsetfillcolor{currentfill}%
\pgfsetfillopacity{0.700000}%
\pgfsetlinewidth{0.000000pt}%
\definecolor{currentstroke}{rgb}{0.000000,0.000000,0.000000}%
\pgfsetstrokecolor{currentstroke}%
\pgfsetdash{}{0pt}%
\pgfpathmoveto{\pgfqpoint{3.249636in}{2.168499in}}%
\pgfpathlineto{\pgfqpoint{3.262657in}{2.160617in}}%
\pgfpathlineto{\pgfqpoint{3.275680in}{2.152943in}}%
\pgfpathlineto{\pgfqpoint{3.288702in}{2.145477in}}%
\pgfpathlineto{\pgfqpoint{3.301726in}{2.138216in}}%
\pgfpathlineto{\pgfqpoint{3.309582in}{2.146894in}}%
\pgfpathlineto{\pgfqpoint{3.317432in}{2.155626in}}%
\pgfpathlineto{\pgfqpoint{3.325276in}{2.164410in}}%
\pgfpathlineto{\pgfqpoint{3.333113in}{2.173247in}}%
\pgfpathlineto{\pgfqpoint{3.320105in}{2.180372in}}%
\pgfpathlineto{\pgfqpoint{3.307097in}{2.187703in}}%
\pgfpathlineto{\pgfqpoint{3.294090in}{2.195241in}}%
\pgfpathlineto{\pgfqpoint{3.281084in}{2.202988in}}%
\pgfpathlineto{\pgfqpoint{3.273232in}{2.194277in}}%
\pgfpathlineto{\pgfqpoint{3.265373in}{2.185624in}}%
\pgfpathlineto{\pgfqpoint{3.257508in}{2.177031in}}%
\pgfpathlineto{\pgfqpoint{3.249636in}{2.168499in}}%
\pgfpathclose%
\pgfusepath{fill}%
\end{pgfscope}%
\begin{pgfscope}%
\pgfpathrectangle{\pgfqpoint{1.254980in}{0.150000in}}{\pgfqpoint{5.490039in}{5.490039in}}%
\pgfusepath{clip}%
\pgfsetbuttcap%
\pgfsetroundjoin%
\definecolor{currentfill}{rgb}{0.237441,0.305202,0.541921}%
\pgfsetfillcolor{currentfill}%
\pgfsetfillopacity{0.700000}%
\pgfsetlinewidth{0.000000pt}%
\definecolor{currentstroke}{rgb}{0.000000,0.000000,0.000000}%
\pgfsetstrokecolor{currentstroke}%
\pgfsetdash{}{0pt}%
\pgfpathmoveto{\pgfqpoint{4.567639in}{2.586720in}}%
\pgfpathlineto{\pgfqpoint{4.580959in}{2.590422in}}%
\pgfpathlineto{\pgfqpoint{4.594291in}{2.594285in}}%
\pgfpathlineto{\pgfqpoint{4.607635in}{2.598310in}}%
\pgfpathlineto{\pgfqpoint{4.620990in}{2.602496in}}%
\pgfpathlineto{\pgfqpoint{4.628385in}{2.610502in}}%
\pgfpathlineto{\pgfqpoint{4.635775in}{2.618500in}}%
\pgfpathlineto{\pgfqpoint{4.643159in}{2.626493in}}%
\pgfpathlineto{\pgfqpoint{4.650538in}{2.634484in}}%
\pgfpathlineto{\pgfqpoint{4.637194in}{2.630558in}}%
\pgfpathlineto{\pgfqpoint{4.623861in}{2.626792in}}%
\pgfpathlineto{\pgfqpoint{4.610540in}{2.623188in}}%
\pgfpathlineto{\pgfqpoint{4.597230in}{2.619744in}}%
\pgfpathlineto{\pgfqpoint{4.589840in}{2.611484in}}%
\pgfpathlineto{\pgfqpoint{4.582445in}{2.603228in}}%
\pgfpathlineto{\pgfqpoint{4.575044in}{2.594975in}}%
\pgfpathlineto{\pgfqpoint{4.567639in}{2.586720in}}%
\pgfpathclose%
\pgfusepath{fill}%
\end{pgfscope}%
\begin{pgfscope}%
\pgfpathrectangle{\pgfqpoint{1.254980in}{0.150000in}}{\pgfqpoint{5.490039in}{5.490039in}}%
\pgfusepath{clip}%
\pgfsetbuttcap%
\pgfsetroundjoin%
\definecolor{currentfill}{rgb}{0.227802,0.326594,0.546532}%
\pgfsetfillcolor{currentfill}%
\pgfsetfillopacity{0.700000}%
\pgfsetlinewidth{0.000000pt}%
\definecolor{currentstroke}{rgb}{0.000000,0.000000,0.000000}%
\pgfsetstrokecolor{currentstroke}%
\pgfsetdash{}{0pt}%
\pgfpathmoveto{\pgfqpoint{4.650538in}{2.634484in}}%
\pgfpathlineto{\pgfqpoint{4.663894in}{2.638571in}}%
\pgfpathlineto{\pgfqpoint{4.677262in}{2.642818in}}%
\pgfpathlineto{\pgfqpoint{4.690642in}{2.647226in}}%
\pgfpathlineto{\pgfqpoint{4.704035in}{2.651793in}}%
\pgfpathlineto{\pgfqpoint{4.711397in}{2.659509in}}%
\pgfpathlineto{\pgfqpoint{4.718755in}{2.667222in}}%
\pgfpathlineto{\pgfqpoint{4.726107in}{2.674936in}}%
\pgfpathlineto{\pgfqpoint{4.733453in}{2.682656in}}%
\pgfpathlineto{\pgfqpoint{4.720073in}{2.678377in}}%
\pgfpathlineto{\pgfqpoint{4.706704in}{2.674258in}}%
\pgfpathlineto{\pgfqpoint{4.693348in}{2.670298in}}%
\pgfpathlineto{\pgfqpoint{4.680003in}{2.666498in}}%
\pgfpathlineto{\pgfqpoint{4.672645in}{2.658481in}}%
\pgfpathlineto{\pgfqpoint{4.665281in}{2.650475in}}%
\pgfpathlineto{\pgfqpoint{4.657912in}{2.642477in}}%
\pgfpathlineto{\pgfqpoint{4.650538in}{2.634484in}}%
\pgfpathclose%
\pgfusepath{fill}%
\end{pgfscope}%
\begin{pgfscope}%
\pgfpathrectangle{\pgfqpoint{1.254980in}{0.150000in}}{\pgfqpoint{5.490039in}{5.490039in}}%
\pgfusepath{clip}%
\pgfsetbuttcap%
\pgfsetroundjoin%
\definecolor{currentfill}{rgb}{0.282884,0.135920,0.453427}%
\pgfsetfillcolor{currentfill}%
\pgfsetfillopacity{0.700000}%
\pgfsetlinewidth{0.000000pt}%
\definecolor{currentstroke}{rgb}{0.000000,0.000000,0.000000}%
\pgfsetstrokecolor{currentstroke}%
\pgfsetdash{}{0pt}%
\pgfpathmoveto{\pgfqpoint{3.904711in}{2.242746in}}%
\pgfpathlineto{\pgfqpoint{3.917802in}{2.242116in}}%
\pgfpathlineto{\pgfqpoint{3.930900in}{2.241662in}}%
\pgfpathlineto{\pgfqpoint{3.944006in}{2.241384in}}%
\pgfpathlineto{\pgfqpoint{3.957118in}{2.241280in}}%
\pgfpathlineto{\pgfqpoint{3.964747in}{2.250925in}}%
\pgfpathlineto{\pgfqpoint{3.972371in}{2.260558in}}%
\pgfpathlineto{\pgfqpoint{3.979990in}{2.270179in}}%
\pgfpathlineto{\pgfqpoint{3.987604in}{2.279789in}}%
\pgfpathlineto{\pgfqpoint{3.974499in}{2.279926in}}%
\pgfpathlineto{\pgfqpoint{3.961402in}{2.280237in}}%
\pgfpathlineto{\pgfqpoint{3.948312in}{2.280724in}}%
\pgfpathlineto{\pgfqpoint{3.935229in}{2.281387in}}%
\pgfpathlineto{\pgfqpoint{3.927607in}{2.271733in}}%
\pgfpathlineto{\pgfqpoint{3.919980in}{2.262076in}}%
\pgfpathlineto{\pgfqpoint{3.912348in}{2.252414in}}%
\pgfpathlineto{\pgfqpoint{3.904711in}{2.242746in}}%
\pgfpathclose%
\pgfusepath{fill}%
\end{pgfscope}%
\begin{pgfscope}%
\pgfpathrectangle{\pgfqpoint{1.254980in}{0.150000in}}{\pgfqpoint{5.490039in}{5.490039in}}%
\pgfusepath{clip}%
\pgfsetbuttcap%
\pgfsetroundjoin%
\definecolor{currentfill}{rgb}{0.218130,0.347432,0.550038}%
\pgfsetfillcolor{currentfill}%
\pgfsetfillopacity{0.700000}%
\pgfsetlinewidth{0.000000pt}%
\definecolor{currentstroke}{rgb}{0.000000,0.000000,0.000000}%
\pgfsetstrokecolor{currentstroke}%
\pgfsetdash{}{0pt}%
\pgfpathmoveto{\pgfqpoint{4.733453in}{2.682656in}}%
\pgfpathlineto{\pgfqpoint{4.746846in}{2.687094in}}%
\pgfpathlineto{\pgfqpoint{4.760251in}{2.691692in}}%
\pgfpathlineto{\pgfqpoint{4.773669in}{2.696449in}}%
\pgfpathlineto{\pgfqpoint{4.787099in}{2.701364in}}%
\pgfpathlineto{\pgfqpoint{4.794428in}{2.708786in}}%
\pgfpathlineto{\pgfqpoint{4.801752in}{2.716212in}}%
\pgfpathlineto{\pgfqpoint{4.809071in}{2.723647in}}%
\pgfpathlineto{\pgfqpoint{4.816385in}{2.731095in}}%
\pgfpathlineto{\pgfqpoint{4.802967in}{2.726497in}}%
\pgfpathlineto{\pgfqpoint{4.789562in}{2.722057in}}%
\pgfpathlineto{\pgfqpoint{4.776170in}{2.717776in}}%
\pgfpathlineto{\pgfqpoint{4.762789in}{2.713654in}}%
\pgfpathlineto{\pgfqpoint{4.755463in}{2.705879in}}%
\pgfpathlineto{\pgfqpoint{4.748131in}{2.698124in}}%
\pgfpathlineto{\pgfqpoint{4.740795in}{2.690384in}}%
\pgfpathlineto{\pgfqpoint{4.733453in}{2.682656in}}%
\pgfpathclose%
\pgfusepath{fill}%
\end{pgfscope}%
\begin{pgfscope}%
\pgfpathrectangle{\pgfqpoint{1.254980in}{0.150000in}}{\pgfqpoint{5.490039in}{5.490039in}}%
\pgfusepath{clip}%
\pgfsetbuttcap%
\pgfsetroundjoin%
\definecolor{currentfill}{rgb}{0.208623,0.367752,0.552675}%
\pgfsetfillcolor{currentfill}%
\pgfsetfillopacity{0.700000}%
\pgfsetlinewidth{0.000000pt}%
\definecolor{currentstroke}{rgb}{0.000000,0.000000,0.000000}%
\pgfsetstrokecolor{currentstroke}%
\pgfsetdash{}{0pt}%
\pgfpathmoveto{\pgfqpoint{4.816385in}{2.731095in}}%
\pgfpathlineto{\pgfqpoint{4.829815in}{2.735852in}}%
\pgfpathlineto{\pgfqpoint{4.843257in}{2.740766in}}%
\pgfpathlineto{\pgfqpoint{4.856713in}{2.745839in}}%
\pgfpathlineto{\pgfqpoint{4.870182in}{2.751070in}}%
\pgfpathlineto{\pgfqpoint{4.877477in}{2.758198in}}%
\pgfpathlineto{\pgfqpoint{4.884767in}{2.765341in}}%
\pgfpathlineto{\pgfqpoint{4.892051in}{2.772501in}}%
\pgfpathlineto{\pgfqpoint{4.899331in}{2.779682in}}%
\pgfpathlineto{\pgfqpoint{4.885876in}{2.774798in}}%
\pgfpathlineto{\pgfqpoint{4.872434in}{2.770071in}}%
\pgfpathlineto{\pgfqpoint{4.859005in}{2.765501in}}%
\pgfpathlineto{\pgfqpoint{4.845589in}{2.761090in}}%
\pgfpathlineto{\pgfqpoint{4.838295in}{2.753553in}}%
\pgfpathlineto{\pgfqpoint{4.830997in}{2.746044in}}%
\pgfpathlineto{\pgfqpoint{4.823693in}{2.738560in}}%
\pgfpathlineto{\pgfqpoint{4.816385in}{2.731095in}}%
\pgfpathclose%
\pgfusepath{fill}%
\end{pgfscope}%
\begin{pgfscope}%
\pgfpathrectangle{\pgfqpoint{1.254980in}{0.150000in}}{\pgfqpoint{5.490039in}{5.490039in}}%
\pgfusepath{clip}%
\pgfsetbuttcap%
\pgfsetroundjoin%
\definecolor{currentfill}{rgb}{0.280894,0.078907,0.402329}%
\pgfsetfillcolor{currentfill}%
\pgfsetfillopacity{0.700000}%
\pgfsetlinewidth{0.000000pt}%
\definecolor{currentstroke}{rgb}{0.000000,0.000000,0.000000}%
\pgfsetstrokecolor{currentstroke}%
\pgfsetdash{}{0pt}%
\pgfpathmoveto{\pgfqpoint{3.520487in}{2.140519in}}%
\pgfpathlineto{\pgfqpoint{3.533512in}{2.136075in}}%
\pgfpathlineto{\pgfqpoint{3.546541in}{2.131823in}}%
\pgfpathlineto{\pgfqpoint{3.559573in}{2.127760in}}%
\pgfpathlineto{\pgfqpoint{3.572609in}{2.123887in}}%
\pgfpathlineto{\pgfqpoint{3.580367in}{2.133330in}}%
\pgfpathlineto{\pgfqpoint{3.588119in}{2.142792in}}%
\pgfpathlineto{\pgfqpoint{3.595866in}{2.152275in}}%
\pgfpathlineto{\pgfqpoint{3.603607in}{2.161777in}}%
\pgfpathlineto{\pgfqpoint{3.590582in}{2.165571in}}%
\pgfpathlineto{\pgfqpoint{3.577561in}{2.169555in}}%
\pgfpathlineto{\pgfqpoint{3.564543in}{2.173729in}}%
\pgfpathlineto{\pgfqpoint{3.551530in}{2.178093in}}%
\pgfpathlineto{\pgfqpoint{3.543778in}{2.168660in}}%
\pgfpathlineto{\pgfqpoint{3.536020in}{2.159253in}}%
\pgfpathlineto{\pgfqpoint{3.528256in}{2.149872in}}%
\pgfpathlineto{\pgfqpoint{3.520487in}{2.140519in}}%
\pgfpathclose%
\pgfusepath{fill}%
\end{pgfscope}%
\begin{pgfscope}%
\pgfpathrectangle{\pgfqpoint{1.254980in}{0.150000in}}{\pgfqpoint{5.490039in}{5.490039in}}%
\pgfusepath{clip}%
\pgfsetbuttcap%
\pgfsetroundjoin%
\definecolor{currentfill}{rgb}{0.283197,0.115680,0.436115}%
\pgfsetfillcolor{currentfill}%
\pgfsetfillopacity{0.700000}%
\pgfsetlinewidth{0.000000pt}%
\definecolor{currentstroke}{rgb}{0.000000,0.000000,0.000000}%
\pgfsetstrokecolor{currentstroke}%
\pgfsetdash{}{0pt}%
\pgfpathmoveto{\pgfqpoint{3.821782in}{2.208280in}}%
\pgfpathlineto{\pgfqpoint{3.834855in}{2.206945in}}%
\pgfpathlineto{\pgfqpoint{3.847935in}{2.205788in}}%
\pgfpathlineto{\pgfqpoint{3.861021in}{2.204810in}}%
\pgfpathlineto{\pgfqpoint{3.874113in}{2.204008in}}%
\pgfpathlineto{\pgfqpoint{3.881770in}{2.213704in}}%
\pgfpathlineto{\pgfqpoint{3.889422in}{2.223392in}}%
\pgfpathlineto{\pgfqpoint{3.897069in}{2.233072in}}%
\pgfpathlineto{\pgfqpoint{3.904711in}{2.242746in}}%
\pgfpathlineto{\pgfqpoint{3.891627in}{2.243553in}}%
\pgfpathlineto{\pgfqpoint{3.878550in}{2.244536in}}%
\pgfpathlineto{\pgfqpoint{3.865479in}{2.245698in}}%
\pgfpathlineto{\pgfqpoint{3.852414in}{2.247037in}}%
\pgfpathlineto{\pgfqpoint{3.844764in}{2.237348in}}%
\pgfpathlineto{\pgfqpoint{3.837108in}{2.227659in}}%
\pgfpathlineto{\pgfqpoint{3.829448in}{2.217970in}}%
\pgfpathlineto{\pgfqpoint{3.821782in}{2.208280in}}%
\pgfpathclose%
\pgfusepath{fill}%
\end{pgfscope}%
\begin{pgfscope}%
\pgfpathrectangle{\pgfqpoint{1.254980in}{0.150000in}}{\pgfqpoint{5.490039in}{5.490039in}}%
\pgfusepath{clip}%
\pgfsetbuttcap%
\pgfsetroundjoin%
\definecolor{currentfill}{rgb}{0.199430,0.387607,0.554642}%
\pgfsetfillcolor{currentfill}%
\pgfsetfillopacity{0.700000}%
\pgfsetlinewidth{0.000000pt}%
\definecolor{currentstroke}{rgb}{0.000000,0.000000,0.000000}%
\pgfsetstrokecolor{currentstroke}%
\pgfsetdash{}{0pt}%
\pgfpathmoveto{\pgfqpoint{4.899331in}{2.779682in}}%
\pgfpathlineto{\pgfqpoint{4.912799in}{2.784723in}}%
\pgfpathlineto{\pgfqpoint{4.926280in}{2.789922in}}%
\pgfpathlineto{\pgfqpoint{4.939774in}{2.795277in}}%
\pgfpathlineto{\pgfqpoint{4.953282in}{2.800790in}}%
\pgfpathlineto{\pgfqpoint{4.960542in}{2.807632in}}%
\pgfpathlineto{\pgfqpoint{4.967797in}{2.814498in}}%
\pgfpathlineto{\pgfqpoint{4.975047in}{2.821390in}}%
\pgfpathlineto{\pgfqpoint{4.982292in}{2.828315in}}%
\pgfpathlineto{\pgfqpoint{4.968799in}{2.823178in}}%
\pgfpathlineto{\pgfqpoint{4.955320in}{2.818197in}}%
\pgfpathlineto{\pgfqpoint{4.941854in}{2.813372in}}%
\pgfpathlineto{\pgfqpoint{4.928401in}{2.808704in}}%
\pgfpathlineto{\pgfqpoint{4.921141in}{2.801396in}}%
\pgfpathlineto{\pgfqpoint{4.913876in}{2.794125in}}%
\pgfpathlineto{\pgfqpoint{4.906606in}{2.786889in}}%
\pgfpathlineto{\pgfqpoint{4.899331in}{2.779682in}}%
\pgfpathclose%
\pgfusepath{fill}%
\end{pgfscope}%
\begin{pgfscope}%
\pgfpathrectangle{\pgfqpoint{1.254980in}{0.150000in}}{\pgfqpoint{5.490039in}{5.490039in}}%
\pgfusepath{clip}%
\pgfsetbuttcap%
\pgfsetroundjoin%
\definecolor{currentfill}{rgb}{0.282910,0.105393,0.426902}%
\pgfsetfillcolor{currentfill}%
\pgfsetfillopacity{0.700000}%
\pgfsetlinewidth{0.000000pt}%
\definecolor{currentstroke}{rgb}{0.000000,0.000000,0.000000}%
\pgfsetstrokecolor{currentstroke}%
\pgfsetdash{}{0pt}%
\pgfpathmoveto{\pgfqpoint{3.113763in}{2.207034in}}%
\pgfpathlineto{\pgfqpoint{3.126806in}{2.197271in}}%
\pgfpathlineto{\pgfqpoint{3.139847in}{2.187728in}}%
\pgfpathlineto{\pgfqpoint{3.152888in}{2.178404in}}%
\pgfpathlineto{\pgfqpoint{3.165928in}{2.169295in}}%
\pgfpathlineto{\pgfqpoint{3.173844in}{2.177399in}}%
\pgfpathlineto{\pgfqpoint{3.181753in}{2.185574in}}%
\pgfpathlineto{\pgfqpoint{3.189655in}{2.193821in}}%
\pgfpathlineto{\pgfqpoint{3.197550in}{2.202138in}}%
\pgfpathlineto{\pgfqpoint{3.184528in}{2.211082in}}%
\pgfpathlineto{\pgfqpoint{3.171505in}{2.220243in}}%
\pgfpathlineto{\pgfqpoint{3.158481in}{2.229621in}}%
\pgfpathlineto{\pgfqpoint{3.145456in}{2.239219in}}%
\pgfpathlineto{\pgfqpoint{3.137543in}{2.231056in}}%
\pgfpathlineto{\pgfqpoint{3.129624in}{2.222970in}}%
\pgfpathlineto{\pgfqpoint{3.121697in}{2.214962in}}%
\pgfpathlineto{\pgfqpoint{3.113763in}{2.207034in}}%
\pgfpathclose%
\pgfusepath{fill}%
\end{pgfscope}%
\begin{pgfscope}%
\pgfpathrectangle{\pgfqpoint{1.254980in}{0.150000in}}{\pgfqpoint{5.490039in}{5.490039in}}%
\pgfusepath{clip}%
\pgfsetbuttcap%
\pgfsetroundjoin%
\definecolor{currentfill}{rgb}{0.190631,0.407061,0.556089}%
\pgfsetfillcolor{currentfill}%
\pgfsetfillopacity{0.700000}%
\pgfsetlinewidth{0.000000pt}%
\definecolor{currentstroke}{rgb}{0.000000,0.000000,0.000000}%
\pgfsetstrokecolor{currentstroke}%
\pgfsetdash{}{0pt}%
\pgfpathmoveto{\pgfqpoint{4.982292in}{2.828315in}}%
\pgfpathlineto{\pgfqpoint{4.995798in}{2.833608in}}%
\pgfpathlineto{\pgfqpoint{5.009317in}{2.839058in}}%
\pgfpathlineto{\pgfqpoint{5.022850in}{2.844663in}}%
\pgfpathlineto{\pgfqpoint{5.036397in}{2.850425in}}%
\pgfpathlineto{\pgfqpoint{5.043621in}{2.856992in}}%
\pgfpathlineto{\pgfqpoint{5.050841in}{2.863593in}}%
\pgfpathlineto{\pgfqpoint{5.058055in}{2.870232in}}%
\pgfpathlineto{\pgfqpoint{5.065265in}{2.876914in}}%
\pgfpathlineto{\pgfqpoint{5.051735in}{2.871557in}}%
\pgfpathlineto{\pgfqpoint{5.038219in}{2.866355in}}%
\pgfpathlineto{\pgfqpoint{5.024715in}{2.861308in}}%
\pgfpathlineto{\pgfqpoint{5.011226in}{2.856417in}}%
\pgfpathlineto{\pgfqpoint{5.003999in}{2.849322in}}%
\pgfpathlineto{\pgfqpoint{4.996768in}{2.842276in}}%
\pgfpathlineto{\pgfqpoint{4.989532in}{2.835275in}}%
\pgfpathlineto{\pgfqpoint{4.982292in}{2.828315in}}%
\pgfpathclose%
\pgfusepath{fill}%
\end{pgfscope}%
\begin{pgfscope}%
\pgfpathrectangle{\pgfqpoint{1.254980in}{0.150000in}}{\pgfqpoint{5.490039in}{5.490039in}}%
\pgfusepath{clip}%
\pgfsetbuttcap%
\pgfsetroundjoin%
\definecolor{currentfill}{rgb}{0.262138,0.242286,0.520837}%
\pgfsetfillcolor{currentfill}%
\pgfsetfillopacity{0.700000}%
\pgfsetlinewidth{0.000000pt}%
\definecolor{currentstroke}{rgb}{0.000000,0.000000,0.000000}%
\pgfsetstrokecolor{currentstroke}%
\pgfsetdash{}{0pt}%
\pgfpathmoveto{\pgfqpoint{2.767452in}{2.485295in}}%
\pgfpathlineto{\pgfqpoint{2.780615in}{2.469550in}}%
\pgfpathlineto{\pgfqpoint{2.793772in}{2.454069in}}%
\pgfpathlineto{\pgfqpoint{2.806922in}{2.438851in}}%
\pgfpathlineto{\pgfqpoint{2.820065in}{2.423891in}}%
\pgfpathlineto{\pgfqpoint{2.828140in}{2.430457in}}%
\pgfpathlineto{\pgfqpoint{2.836206in}{2.437144in}}%
\pgfpathlineto{\pgfqpoint{2.844263in}{2.443950in}}%
\pgfpathlineto{\pgfqpoint{2.852310in}{2.450874in}}%
\pgfpathlineto{\pgfqpoint{2.839191in}{2.465634in}}%
\pgfpathlineto{\pgfqpoint{2.826065in}{2.480654in}}%
\pgfpathlineto{\pgfqpoint{2.812934in}{2.495936in}}%
\pgfpathlineto{\pgfqpoint{2.799795in}{2.511481in}}%
\pgfpathlineto{\pgfqpoint{2.791724in}{2.504746in}}%
\pgfpathlineto{\pgfqpoint{2.783643in}{2.498136in}}%
\pgfpathlineto{\pgfqpoint{2.775552in}{2.491652in}}%
\pgfpathlineto{\pgfqpoint{2.767452in}{2.485295in}}%
\pgfpathclose%
\pgfusepath{fill}%
\end{pgfscope}%
\begin{pgfscope}%
\pgfpathrectangle{\pgfqpoint{1.254980in}{0.150000in}}{\pgfqpoint{5.490039in}{5.490039in}}%
\pgfusepath{clip}%
\pgfsetbuttcap%
\pgfsetroundjoin%
\definecolor{currentfill}{rgb}{0.270595,0.214069,0.507052}%
\pgfsetfillcolor{currentfill}%
\pgfsetfillopacity{0.700000}%
\pgfsetlinewidth{0.000000pt}%
\definecolor{currentstroke}{rgb}{0.000000,0.000000,0.000000}%
\pgfsetstrokecolor{currentstroke}%
\pgfsetdash{}{0pt}%
\pgfpathmoveto{\pgfqpoint{2.820065in}{2.423891in}}%
\pgfpathlineto{\pgfqpoint{2.833203in}{2.409190in}}%
\pgfpathlineto{\pgfqpoint{2.846334in}{2.394743in}}%
\pgfpathlineto{\pgfqpoint{2.859460in}{2.380550in}}%
\pgfpathlineto{\pgfqpoint{2.872580in}{2.366608in}}%
\pgfpathlineto{\pgfqpoint{2.880631in}{2.373382in}}%
\pgfpathlineto{\pgfqpoint{2.888673in}{2.380270in}}%
\pgfpathlineto{\pgfqpoint{2.896706in}{2.387271in}}%
\pgfpathlineto{\pgfqpoint{2.904730in}{2.394381in}}%
\pgfpathlineto{\pgfqpoint{2.891633in}{2.408126in}}%
\pgfpathlineto{\pgfqpoint{2.878531in}{2.422121in}}%
\pgfpathlineto{\pgfqpoint{2.865423in}{2.436370in}}%
\pgfpathlineto{\pgfqpoint{2.852310in}{2.450874in}}%
\pgfpathlineto{\pgfqpoint{2.844263in}{2.443950in}}%
\pgfpathlineto{\pgfqpoint{2.836206in}{2.437144in}}%
\pgfpathlineto{\pgfqpoint{2.828140in}{2.430457in}}%
\pgfpathlineto{\pgfqpoint{2.820065in}{2.423891in}}%
\pgfpathclose%
\pgfusepath{fill}%
\end{pgfscope}%
\begin{pgfscope}%
\pgfpathrectangle{\pgfqpoint{1.254980in}{0.150000in}}{\pgfqpoint{5.490039in}{5.490039in}}%
\pgfusepath{clip}%
\pgfsetbuttcap%
\pgfsetroundjoin%
\definecolor{currentfill}{rgb}{0.252194,0.269783,0.531579}%
\pgfsetfillcolor{currentfill}%
\pgfsetfillopacity{0.700000}%
\pgfsetlinewidth{0.000000pt}%
\definecolor{currentstroke}{rgb}{0.000000,0.000000,0.000000}%
\pgfsetstrokecolor{currentstroke}%
\pgfsetdash{}{0pt}%
\pgfpathmoveto{\pgfqpoint{2.714723in}{2.550965in}}%
\pgfpathlineto{\pgfqpoint{2.727917in}{2.534139in}}%
\pgfpathlineto{\pgfqpoint{2.741103in}{2.517588in}}%
\pgfpathlineto{\pgfqpoint{2.754281in}{2.501307in}}%
\pgfpathlineto{\pgfqpoint{2.767452in}{2.485295in}}%
\pgfpathlineto{\pgfqpoint{2.775552in}{2.491652in}}%
\pgfpathlineto{\pgfqpoint{2.783643in}{2.498136in}}%
\pgfpathlineto{\pgfqpoint{2.791724in}{2.504746in}}%
\pgfpathlineto{\pgfqpoint{2.799795in}{2.511481in}}%
\pgfpathlineto{\pgfqpoint{2.786650in}{2.527293in}}%
\pgfpathlineto{\pgfqpoint{2.773497in}{2.543372in}}%
\pgfpathlineto{\pgfqpoint{2.760337in}{2.559723in}}%
\pgfpathlineto{\pgfqpoint{2.747169in}{2.576347in}}%
\pgfpathlineto{\pgfqpoint{2.739073in}{2.569802in}}%
\pgfpathlineto{\pgfqpoint{2.730966in}{2.563389in}}%
\pgfpathlineto{\pgfqpoint{2.722850in}{2.557109in}}%
\pgfpathlineto{\pgfqpoint{2.714723in}{2.550965in}}%
\pgfpathclose%
\pgfusepath{fill}%
\end{pgfscope}%
\begin{pgfscope}%
\pgfpathrectangle{\pgfqpoint{1.254980in}{0.150000in}}{\pgfqpoint{5.490039in}{5.490039in}}%
\pgfusepath{clip}%
\pgfsetbuttcap%
\pgfsetroundjoin%
\definecolor{currentfill}{rgb}{0.182256,0.426184,0.557120}%
\pgfsetfillcolor{currentfill}%
\pgfsetfillopacity{0.700000}%
\pgfsetlinewidth{0.000000pt}%
\definecolor{currentstroke}{rgb}{0.000000,0.000000,0.000000}%
\pgfsetstrokecolor{currentstroke}%
\pgfsetdash{}{0pt}%
\pgfpathmoveto{\pgfqpoint{5.065265in}{2.876914in}}%
\pgfpathlineto{\pgfqpoint{5.078809in}{2.882427in}}%
\pgfpathlineto{\pgfqpoint{5.092367in}{2.888095in}}%
\pgfpathlineto{\pgfqpoint{5.105939in}{2.893918in}}%
\pgfpathlineto{\pgfqpoint{5.119526in}{2.899896in}}%
\pgfpathlineto{\pgfqpoint{5.126713in}{2.906203in}}%
\pgfpathlineto{\pgfqpoint{5.133897in}{2.912556in}}%
\pgfpathlineto{\pgfqpoint{5.141076in}{2.918960in}}%
\pgfpathlineto{\pgfqpoint{5.148250in}{2.925420in}}%
\pgfpathlineto{\pgfqpoint{5.134682in}{2.919875in}}%
\pgfpathlineto{\pgfqpoint{5.121128in}{2.914484in}}%
\pgfpathlineto{\pgfqpoint{5.107588in}{2.909248in}}%
\pgfpathlineto{\pgfqpoint{5.094062in}{2.904167in}}%
\pgfpathlineto{\pgfqpoint{5.086869in}{2.897265in}}%
\pgfpathlineto{\pgfqpoint{5.079672in}{2.890426in}}%
\pgfpathlineto{\pgfqpoint{5.072471in}{2.883644in}}%
\pgfpathlineto{\pgfqpoint{5.065265in}{2.876914in}}%
\pgfpathclose%
\pgfusepath{fill}%
\end{pgfscope}%
\begin{pgfscope}%
\pgfpathrectangle{\pgfqpoint{1.254980in}{0.150000in}}{\pgfqpoint{5.490039in}{5.490039in}}%
\pgfusepath{clip}%
\pgfsetbuttcap%
\pgfsetroundjoin%
\definecolor{currentfill}{rgb}{0.282656,0.100196,0.422160}%
\pgfsetfillcolor{currentfill}%
\pgfsetfillopacity{0.700000}%
\pgfsetlinewidth{0.000000pt}%
\definecolor{currentstroke}{rgb}{0.000000,0.000000,0.000000}%
\pgfsetstrokecolor{currentstroke}%
\pgfsetdash{}{0pt}%
\pgfpathmoveto{\pgfqpoint{3.738801in}{2.176733in}}%
\pgfpathlineto{\pgfqpoint{3.751859in}{2.174655in}}%
\pgfpathlineto{\pgfqpoint{3.764923in}{2.172758in}}%
\pgfpathlineto{\pgfqpoint{3.777993in}{2.171041in}}%
\pgfpathlineto{\pgfqpoint{3.791069in}{2.169504in}}%
\pgfpathlineto{\pgfqpoint{3.798755in}{2.179201in}}%
\pgfpathlineto{\pgfqpoint{3.806436in}{2.188895in}}%
\pgfpathlineto{\pgfqpoint{3.814111in}{2.198588in}}%
\pgfpathlineto{\pgfqpoint{3.821782in}{2.208280in}}%
\pgfpathlineto{\pgfqpoint{3.808715in}{2.209794in}}%
\pgfpathlineto{\pgfqpoint{3.795654in}{2.211487in}}%
\pgfpathlineto{\pgfqpoint{3.782599in}{2.213362in}}%
\pgfpathlineto{\pgfqpoint{3.769550in}{2.215417in}}%
\pgfpathlineto{\pgfqpoint{3.761871in}{2.205738in}}%
\pgfpathlineto{\pgfqpoint{3.754186in}{2.196065in}}%
\pgfpathlineto{\pgfqpoint{3.746496in}{2.186397in}}%
\pgfpathlineto{\pgfqpoint{3.738801in}{2.176733in}}%
\pgfpathclose%
\pgfusepath{fill}%
\end{pgfscope}%
\begin{pgfscope}%
\pgfpathrectangle{\pgfqpoint{1.254980in}{0.150000in}}{\pgfqpoint{5.490039in}{5.490039in}}%
\pgfusepath{clip}%
\pgfsetbuttcap%
\pgfsetroundjoin%
\definecolor{currentfill}{rgb}{0.174274,0.445044,0.557792}%
\pgfsetfillcolor{currentfill}%
\pgfsetfillopacity{0.700000}%
\pgfsetlinewidth{0.000000pt}%
\definecolor{currentstroke}{rgb}{0.000000,0.000000,0.000000}%
\pgfsetstrokecolor{currentstroke}%
\pgfsetdash{}{0pt}%
\pgfpathmoveto{\pgfqpoint{5.148250in}{2.925420in}}%
\pgfpathlineto{\pgfqpoint{5.161832in}{2.931119in}}%
\pgfpathlineto{\pgfqpoint{5.175429in}{2.936973in}}%
\pgfpathlineto{\pgfqpoint{5.189040in}{2.942981in}}%
\pgfpathlineto{\pgfqpoint{5.202665in}{2.949143in}}%
\pgfpathlineto{\pgfqpoint{5.209816in}{2.955211in}}%
\pgfpathlineto{\pgfqpoint{5.216963in}{2.961339in}}%
\pgfpathlineto{\pgfqpoint{5.224106in}{2.967531in}}%
\pgfpathlineto{\pgfqpoint{5.231245in}{2.973792in}}%
\pgfpathlineto{\pgfqpoint{5.217640in}{2.968092in}}%
\pgfpathlineto{\pgfqpoint{5.204049in}{2.962545in}}%
\pgfpathlineto{\pgfqpoint{5.190472in}{2.957152in}}%
\pgfpathlineto{\pgfqpoint{5.176909in}{2.951913in}}%
\pgfpathlineto{\pgfqpoint{5.169750in}{2.945181in}}%
\pgfpathlineto{\pgfqpoint{5.162587in}{2.938525in}}%
\pgfpathlineto{\pgfqpoint{5.155421in}{2.931940in}}%
\pgfpathlineto{\pgfqpoint{5.148250in}{2.925420in}}%
\pgfpathclose%
\pgfusepath{fill}%
\end{pgfscope}%
\begin{pgfscope}%
\pgfpathrectangle{\pgfqpoint{1.254980in}{0.150000in}}{\pgfqpoint{5.490039in}{5.490039in}}%
\pgfusepath{clip}%
\pgfsetbuttcap%
\pgfsetroundjoin%
\definecolor{currentfill}{rgb}{0.277134,0.185228,0.489898}%
\pgfsetfillcolor{currentfill}%
\pgfsetfillopacity{0.700000}%
\pgfsetlinewidth{0.000000pt}%
\definecolor{currentstroke}{rgb}{0.000000,0.000000,0.000000}%
\pgfsetstrokecolor{currentstroke}%
\pgfsetdash{}{0pt}%
\pgfpathmoveto{\pgfqpoint{2.872580in}{2.366608in}}%
\pgfpathlineto{\pgfqpoint{2.885695in}{2.352915in}}%
\pgfpathlineto{\pgfqpoint{2.898805in}{2.339469in}}%
\pgfpathlineto{\pgfqpoint{2.911911in}{2.326269in}}%
\pgfpathlineto{\pgfqpoint{2.925012in}{2.313312in}}%
\pgfpathlineto{\pgfqpoint{2.933039in}{2.320293in}}%
\pgfpathlineto{\pgfqpoint{2.941058in}{2.327381in}}%
\pgfpathlineto{\pgfqpoint{2.949069in}{2.334574in}}%
\pgfpathlineto{\pgfqpoint{2.957071in}{2.341871in}}%
\pgfpathlineto{\pgfqpoint{2.943993in}{2.354632in}}%
\pgfpathlineto{\pgfqpoint{2.930910in}{2.367636in}}%
\pgfpathlineto{\pgfqpoint{2.917822in}{2.380885in}}%
\pgfpathlineto{\pgfqpoint{2.904730in}{2.394381in}}%
\pgfpathlineto{\pgfqpoint{2.896706in}{2.387271in}}%
\pgfpathlineto{\pgfqpoint{2.888673in}{2.380270in}}%
\pgfpathlineto{\pgfqpoint{2.880631in}{2.373382in}}%
\pgfpathlineto{\pgfqpoint{2.872580in}{2.366608in}}%
\pgfpathclose%
\pgfusepath{fill}%
\end{pgfscope}%
\begin{pgfscope}%
\pgfpathrectangle{\pgfqpoint{1.254980in}{0.150000in}}{\pgfqpoint{5.490039in}{5.490039in}}%
\pgfusepath{clip}%
\pgfsetbuttcap%
\pgfsetroundjoin%
\definecolor{currentfill}{rgb}{0.239346,0.300855,0.540844}%
\pgfsetfillcolor{currentfill}%
\pgfsetfillopacity{0.700000}%
\pgfsetlinewidth{0.000000pt}%
\definecolor{currentstroke}{rgb}{0.000000,0.000000,0.000000}%
\pgfsetstrokecolor{currentstroke}%
\pgfsetdash{}{0pt}%
\pgfpathmoveto{\pgfqpoint{2.661862in}{2.621053in}}%
\pgfpathlineto{\pgfqpoint{2.675091in}{2.603108in}}%
\pgfpathlineto{\pgfqpoint{2.688310in}{2.585446in}}%
\pgfpathlineto{\pgfqpoint{2.701521in}{2.568066in}}%
\pgfpathlineto{\pgfqpoint{2.714723in}{2.550965in}}%
\pgfpathlineto{\pgfqpoint{2.722850in}{2.557109in}}%
\pgfpathlineto{\pgfqpoint{2.730966in}{2.563389in}}%
\pgfpathlineto{\pgfqpoint{2.739073in}{2.569802in}}%
\pgfpathlineto{\pgfqpoint{2.747169in}{2.576347in}}%
\pgfpathlineto{\pgfqpoint{2.733994in}{2.593247in}}%
\pgfpathlineto{\pgfqpoint{2.720810in}{2.610425in}}%
\pgfpathlineto{\pgfqpoint{2.707617in}{2.627884in}}%
\pgfpathlineto{\pgfqpoint{2.694416in}{2.645626in}}%
\pgfpathlineto{\pgfqpoint{2.686293in}{2.639273in}}%
\pgfpathlineto{\pgfqpoint{2.678160in}{2.633058in}}%
\pgfpathlineto{\pgfqpoint{2.670016in}{2.626985in}}%
\pgfpathlineto{\pgfqpoint{2.661862in}{2.621053in}}%
\pgfpathclose%
\pgfusepath{fill}%
\end{pgfscope}%
\begin{pgfscope}%
\pgfpathrectangle{\pgfqpoint{1.254980in}{0.150000in}}{\pgfqpoint{5.490039in}{5.490039in}}%
\pgfusepath{clip}%
\pgfsetbuttcap%
\pgfsetroundjoin%
\definecolor{currentfill}{rgb}{0.165117,0.467423,0.558141}%
\pgfsetfillcolor{currentfill}%
\pgfsetfillopacity{0.700000}%
\pgfsetlinewidth{0.000000pt}%
\definecolor{currentstroke}{rgb}{0.000000,0.000000,0.000000}%
\pgfsetstrokecolor{currentstroke}%
\pgfsetdash{}{0pt}%
\pgfpathmoveto{\pgfqpoint{5.231245in}{2.973792in}}%
\pgfpathlineto{\pgfqpoint{5.244865in}{2.979645in}}%
\pgfpathlineto{\pgfqpoint{5.258500in}{2.985652in}}%
\pgfpathlineto{\pgfqpoint{5.272149in}{2.991813in}}%
\pgfpathlineto{\pgfqpoint{5.285813in}{2.998127in}}%
\pgfpathlineto{\pgfqpoint{5.292928in}{3.003982in}}%
\pgfpathlineto{\pgfqpoint{5.300038in}{3.009911in}}%
\pgfpathlineto{\pgfqpoint{5.307145in}{3.015919in}}%
\pgfpathlineto{\pgfqpoint{5.314249in}{3.022011in}}%
\pgfpathlineto{\pgfqpoint{5.300606in}{3.016188in}}%
\pgfpathlineto{\pgfqpoint{5.286978in}{3.010517in}}%
\pgfpathlineto{\pgfqpoint{5.273365in}{3.005000in}}%
\pgfpathlineto{\pgfqpoint{5.259766in}{2.999635in}}%
\pgfpathlineto{\pgfqpoint{5.252641in}{2.993043in}}%
\pgfpathlineto{\pgfqpoint{5.245512in}{2.986543in}}%
\pgfpathlineto{\pgfqpoint{5.238381in}{2.980127in}}%
\pgfpathlineto{\pgfqpoint{5.231245in}{2.973792in}}%
\pgfpathclose%
\pgfusepath{fill}%
\end{pgfscope}%
\begin{pgfscope}%
\pgfpathrectangle{\pgfqpoint{1.254980in}{0.150000in}}{\pgfqpoint{5.490039in}{5.490039in}}%
\pgfusepath{clip}%
\pgfsetbuttcap%
\pgfsetroundjoin%
\definecolor{currentfill}{rgb}{0.280894,0.078907,0.402329}%
\pgfsetfillcolor{currentfill}%
\pgfsetfillopacity{0.700000}%
\pgfsetlinewidth{0.000000pt}%
\definecolor{currentstroke}{rgb}{0.000000,0.000000,0.000000}%
\pgfsetstrokecolor{currentstroke}%
\pgfsetdash{}{0pt}%
\pgfpathmoveto{\pgfqpoint{3.301726in}{2.138216in}}%
\pgfpathlineto{\pgfqpoint{3.314750in}{2.131160in}}%
\pgfpathlineto{\pgfqpoint{3.327776in}{2.124308in}}%
\pgfpathlineto{\pgfqpoint{3.340803in}{2.117658in}}%
\pgfpathlineto{\pgfqpoint{3.353831in}{2.111210in}}%
\pgfpathlineto{\pgfqpoint{3.361673in}{2.120033in}}%
\pgfpathlineto{\pgfqpoint{3.369508in}{2.128903in}}%
\pgfpathlineto{\pgfqpoint{3.377338in}{2.137819in}}%
\pgfpathlineto{\pgfqpoint{3.385161in}{2.146780in}}%
\pgfpathlineto{\pgfqpoint{3.372147in}{2.153094in}}%
\pgfpathlineto{\pgfqpoint{3.359134in}{2.159609in}}%
\pgfpathlineto{\pgfqpoint{3.346123in}{2.166326in}}%
\pgfpathlineto{\pgfqpoint{3.333113in}{2.173247in}}%
\pgfpathlineto{\pgfqpoint{3.325276in}{2.164410in}}%
\pgfpathlineto{\pgfqpoint{3.317432in}{2.155626in}}%
\pgfpathlineto{\pgfqpoint{3.309582in}{2.146894in}}%
\pgfpathlineto{\pgfqpoint{3.301726in}{2.138216in}}%
\pgfpathclose%
\pgfusepath{fill}%
\end{pgfscope}%
\begin{pgfscope}%
\pgfpathrectangle{\pgfqpoint{1.254980in}{0.150000in}}{\pgfqpoint{5.490039in}{5.490039in}}%
\pgfusepath{clip}%
\pgfsetbuttcap%
\pgfsetroundjoin%
\definecolor{currentfill}{rgb}{0.157729,0.485932,0.558013}%
\pgfsetfillcolor{currentfill}%
\pgfsetfillopacity{0.700000}%
\pgfsetlinewidth{0.000000pt}%
\definecolor{currentstroke}{rgb}{0.000000,0.000000,0.000000}%
\pgfsetstrokecolor{currentstroke}%
\pgfsetdash{}{0pt}%
\pgfpathmoveto{\pgfqpoint{5.314249in}{3.022011in}}%
\pgfpathlineto{\pgfqpoint{5.327907in}{3.027986in}}%
\pgfpathlineto{\pgfqpoint{5.341579in}{3.034115in}}%
\pgfpathlineto{\pgfqpoint{5.355266in}{3.040395in}}%
\pgfpathlineto{\pgfqpoint{5.368968in}{3.046829in}}%
\pgfpathlineto{\pgfqpoint{5.376046in}{3.052502in}}%
\pgfpathlineto{\pgfqpoint{5.383121in}{3.058264in}}%
\pgfpathlineto{\pgfqpoint{5.390192in}{3.064120in}}%
\pgfpathlineto{\pgfqpoint{5.397261in}{3.070078in}}%
\pgfpathlineto{\pgfqpoint{5.383582in}{3.064164in}}%
\pgfpathlineto{\pgfqpoint{5.369917in}{3.058402in}}%
\pgfpathlineto{\pgfqpoint{5.356268in}{3.052792in}}%
\pgfpathlineto{\pgfqpoint{5.342633in}{3.047334in}}%
\pgfpathlineto{\pgfqpoint{5.335541in}{3.040848in}}%
\pgfpathlineto{\pgfqpoint{5.328447in}{3.034470in}}%
\pgfpathlineto{\pgfqpoint{5.321350in}{3.028192in}}%
\pgfpathlineto{\pgfqpoint{5.314249in}{3.022011in}}%
\pgfpathclose%
\pgfusepath{fill}%
\end{pgfscope}%
\begin{pgfscope}%
\pgfpathrectangle{\pgfqpoint{1.254980in}{0.150000in}}{\pgfqpoint{5.490039in}{5.490039in}}%
\pgfusepath{clip}%
\pgfsetbuttcap%
\pgfsetroundjoin%
\definecolor{currentfill}{rgb}{0.280267,0.073417,0.397163}%
\pgfsetfillcolor{currentfill}%
\pgfsetfillopacity{0.700000}%
\pgfsetlinewidth{0.000000pt}%
\definecolor{currentstroke}{rgb}{0.000000,0.000000,0.000000}%
\pgfsetstrokecolor{currentstroke}%
\pgfsetdash{}{0pt}%
\pgfpathmoveto{\pgfqpoint{3.437238in}{2.123516in}}%
\pgfpathlineto{\pgfqpoint{3.450263in}{2.118193in}}%
\pgfpathlineto{\pgfqpoint{3.463291in}{2.113064in}}%
\pgfpathlineto{\pgfqpoint{3.476322in}{2.108129in}}%
\pgfpathlineto{\pgfqpoint{3.489356in}{2.103387in}}%
\pgfpathlineto{\pgfqpoint{3.497147in}{2.112626in}}%
\pgfpathlineto{\pgfqpoint{3.504933in}{2.121895in}}%
\pgfpathlineto{\pgfqpoint{3.512713in}{2.131193in}}%
\pgfpathlineto{\pgfqpoint{3.520487in}{2.140519in}}%
\pgfpathlineto{\pgfqpoint{3.507466in}{2.145154in}}%
\pgfpathlineto{\pgfqpoint{3.494447in}{2.149983in}}%
\pgfpathlineto{\pgfqpoint{3.481432in}{2.155005in}}%
\pgfpathlineto{\pgfqpoint{3.468419in}{2.160222in}}%
\pgfpathlineto{\pgfqpoint{3.460633in}{2.150992in}}%
\pgfpathlineto{\pgfqpoint{3.452840in}{2.141798in}}%
\pgfpathlineto{\pgfqpoint{3.445042in}{2.132639in}}%
\pgfpathlineto{\pgfqpoint{3.437238in}{2.123516in}}%
\pgfpathclose%
\pgfusepath{fill}%
\end{pgfscope}%
\begin{pgfscope}%
\pgfpathrectangle{\pgfqpoint{1.254980in}{0.150000in}}{\pgfqpoint{5.490039in}{5.490039in}}%
\pgfusepath{clip}%
\pgfsetbuttcap%
\pgfsetroundjoin%
\definecolor{currentfill}{rgb}{0.280868,0.160771,0.472899}%
\pgfsetfillcolor{currentfill}%
\pgfsetfillopacity{0.700000}%
\pgfsetlinewidth{0.000000pt}%
\definecolor{currentstroke}{rgb}{0.000000,0.000000,0.000000}%
\pgfsetstrokecolor{currentstroke}%
\pgfsetdash{}{0pt}%
\pgfpathmoveto{\pgfqpoint{2.925012in}{2.313312in}}%
\pgfpathlineto{\pgfqpoint{2.938108in}{2.300596in}}%
\pgfpathlineto{\pgfqpoint{2.951201in}{2.288120in}}%
\pgfpathlineto{\pgfqpoint{2.964289in}{2.275881in}}%
\pgfpathlineto{\pgfqpoint{2.977375in}{2.263878in}}%
\pgfpathlineto{\pgfqpoint{2.985380in}{2.271065in}}%
\pgfpathlineto{\pgfqpoint{2.993378in}{2.278352in}}%
\pgfpathlineto{\pgfqpoint{3.001367in}{2.285737in}}%
\pgfpathlineto{\pgfqpoint{3.009348in}{2.293218in}}%
\pgfpathlineto{\pgfqpoint{2.996284in}{2.305026in}}%
\pgfpathlineto{\pgfqpoint{2.983217in}{2.317069in}}%
\pgfpathlineto{\pgfqpoint{2.970146in}{2.329350in}}%
\pgfpathlineto{\pgfqpoint{2.957071in}{2.341871in}}%
\pgfpathlineto{\pgfqpoint{2.949069in}{2.334574in}}%
\pgfpathlineto{\pgfqpoint{2.941058in}{2.327381in}}%
\pgfpathlineto{\pgfqpoint{2.933039in}{2.320293in}}%
\pgfpathlineto{\pgfqpoint{2.925012in}{2.313312in}}%
\pgfpathclose%
\pgfusepath{fill}%
\end{pgfscope}%
\begin{pgfscope}%
\pgfpathrectangle{\pgfqpoint{1.254980in}{0.150000in}}{\pgfqpoint{5.490039in}{5.490039in}}%
\pgfusepath{clip}%
\pgfsetbuttcap%
\pgfsetroundjoin%
\definecolor{currentfill}{rgb}{0.150476,0.504369,0.557430}%
\pgfsetfillcolor{currentfill}%
\pgfsetfillopacity{0.700000}%
\pgfsetlinewidth{0.000000pt}%
\definecolor{currentstroke}{rgb}{0.000000,0.000000,0.000000}%
\pgfsetstrokecolor{currentstroke}%
\pgfsetdash{}{0pt}%
\pgfpathmoveto{\pgfqpoint{5.397261in}{3.070078in}}%
\pgfpathlineto{\pgfqpoint{5.410955in}{3.076143in}}%
\pgfpathlineto{\pgfqpoint{5.424665in}{3.082361in}}%
\pgfpathlineto{\pgfqpoint{5.438390in}{3.088730in}}%
\pgfpathlineto{\pgfqpoint{5.452130in}{3.095251in}}%
\pgfpathlineto{\pgfqpoint{5.459171in}{3.100777in}}%
\pgfpathlineto{\pgfqpoint{5.466210in}{3.106409in}}%
\pgfpathlineto{\pgfqpoint{5.473247in}{3.112152in}}%
\pgfpathlineto{\pgfqpoint{5.480281in}{3.118014in}}%
\pgfpathlineto{\pgfqpoint{5.466566in}{3.112041in}}%
\pgfpathlineto{\pgfqpoint{5.452866in}{3.106220in}}%
\pgfpathlineto{\pgfqpoint{5.439181in}{3.100549in}}%
\pgfpathlineto{\pgfqpoint{5.425510in}{3.095030in}}%
\pgfpathlineto{\pgfqpoint{5.418451in}{3.088611in}}%
\pgfpathlineto{\pgfqpoint{5.411390in}{3.082317in}}%
\pgfpathlineto{\pgfqpoint{5.404327in}{3.076141in}}%
\pgfpathlineto{\pgfqpoint{5.397261in}{3.070078in}}%
\pgfpathclose%
\pgfusepath{fill}%
\end{pgfscope}%
\begin{pgfscope}%
\pgfpathrectangle{\pgfqpoint{1.254980in}{0.150000in}}{\pgfqpoint{5.490039in}{5.490039in}}%
\pgfusepath{clip}%
\pgfsetbuttcap%
\pgfsetroundjoin%
\definecolor{currentfill}{rgb}{0.281924,0.089666,0.412415}%
\pgfsetfillcolor{currentfill}%
\pgfsetfillopacity{0.700000}%
\pgfsetlinewidth{0.000000pt}%
\definecolor{currentstroke}{rgb}{0.000000,0.000000,0.000000}%
\pgfsetstrokecolor{currentstroke}%
\pgfsetdash{}{0pt}%
\pgfpathmoveto{\pgfqpoint{3.655750in}{2.148473in}}%
\pgfpathlineto{\pgfqpoint{3.668797in}{2.145612in}}%
\pgfpathlineto{\pgfqpoint{3.681849in}{2.142935in}}%
\pgfpathlineto{\pgfqpoint{3.694907in}{2.140441in}}%
\pgfpathlineto{\pgfqpoint{3.707969in}{2.138130in}}%
\pgfpathlineto{\pgfqpoint{3.715685in}{2.147773in}}%
\pgfpathlineto{\pgfqpoint{3.723395in}{2.157422in}}%
\pgfpathlineto{\pgfqpoint{3.731101in}{2.167075in}}%
\pgfpathlineto{\pgfqpoint{3.738801in}{2.176733in}}%
\pgfpathlineto{\pgfqpoint{3.725748in}{2.178994in}}%
\pgfpathlineto{\pgfqpoint{3.712700in}{2.181437in}}%
\pgfpathlineto{\pgfqpoint{3.699658in}{2.184063in}}%
\pgfpathlineto{\pgfqpoint{3.686621in}{2.186874in}}%
\pgfpathlineto{\pgfqpoint{3.678911in}{2.177256in}}%
\pgfpathlineto{\pgfqpoint{3.671196in}{2.167650in}}%
\pgfpathlineto{\pgfqpoint{3.663475in}{2.158055in}}%
\pgfpathlineto{\pgfqpoint{3.655750in}{2.148473in}}%
\pgfpathclose%
\pgfusepath{fill}%
\end{pgfscope}%
\begin{pgfscope}%
\pgfpathrectangle{\pgfqpoint{1.254980in}{0.150000in}}{\pgfqpoint{5.490039in}{5.490039in}}%
\pgfusepath{clip}%
\pgfsetbuttcap%
\pgfsetroundjoin%
\definecolor{currentfill}{rgb}{0.223925,0.334994,0.548053}%
\pgfsetfillcolor{currentfill}%
\pgfsetfillopacity{0.700000}%
\pgfsetlinewidth{0.000000pt}%
\definecolor{currentstroke}{rgb}{0.000000,0.000000,0.000000}%
\pgfsetstrokecolor{currentstroke}%
\pgfsetdash{}{0pt}%
\pgfpathmoveto{\pgfqpoint{2.608851in}{2.695728in}}%
\pgfpathlineto{\pgfqpoint{2.622119in}{2.676620in}}%
\pgfpathlineto{\pgfqpoint{2.635376in}{2.657807in}}%
\pgfpathlineto{\pgfqpoint{2.648624in}{2.639285in}}%
\pgfpathlineto{\pgfqpoint{2.661862in}{2.621053in}}%
\pgfpathlineto{\pgfqpoint{2.670016in}{2.626985in}}%
\pgfpathlineto{\pgfqpoint{2.678160in}{2.633058in}}%
\pgfpathlineto{\pgfqpoint{2.686293in}{2.639273in}}%
\pgfpathlineto{\pgfqpoint{2.694416in}{2.645626in}}%
\pgfpathlineto{\pgfqpoint{2.681206in}{2.663655in}}%
\pgfpathlineto{\pgfqpoint{2.667986in}{2.681972in}}%
\pgfpathlineto{\pgfqpoint{2.654757in}{2.700581in}}%
\pgfpathlineto{\pgfqpoint{2.641517in}{2.719484in}}%
\pgfpathlineto{\pgfqpoint{2.633367in}{2.713324in}}%
\pgfpathlineto{\pgfqpoint{2.625206in}{2.707310in}}%
\pgfpathlineto{\pgfqpoint{2.617034in}{2.701444in}}%
\pgfpathlineto{\pgfqpoint{2.608851in}{2.695728in}}%
\pgfpathclose%
\pgfusepath{fill}%
\end{pgfscope}%
\begin{pgfscope}%
\pgfpathrectangle{\pgfqpoint{1.254980in}{0.150000in}}{\pgfqpoint{5.490039in}{5.490039in}}%
\pgfusepath{clip}%
\pgfsetbuttcap%
\pgfsetroundjoin%
\definecolor{currentfill}{rgb}{0.282327,0.094955,0.417331}%
\pgfsetfillcolor{currentfill}%
\pgfsetfillopacity{0.700000}%
\pgfsetlinewidth{0.000000pt}%
\definecolor{currentstroke}{rgb}{0.000000,0.000000,0.000000}%
\pgfsetstrokecolor{currentstroke}%
\pgfsetdash{}{0pt}%
\pgfpathmoveto{\pgfqpoint{3.165928in}{2.169295in}}%
\pgfpathlineto{\pgfqpoint{3.178967in}{2.160402in}}%
\pgfpathlineto{\pgfqpoint{3.192005in}{2.151723in}}%
\pgfpathlineto{\pgfqpoint{3.205043in}{2.143257in}}%
\pgfpathlineto{\pgfqpoint{3.218081in}{2.135001in}}%
\pgfpathlineto{\pgfqpoint{3.225980in}{2.143279in}}%
\pgfpathlineto{\pgfqpoint{3.233872in}{2.151622in}}%
\pgfpathlineto{\pgfqpoint{3.241757in}{2.160029in}}%
\pgfpathlineto{\pgfqpoint{3.249636in}{2.168499in}}%
\pgfpathlineto{\pgfqpoint{3.236614in}{2.176591in}}%
\pgfpathlineto{\pgfqpoint{3.223593in}{2.184894in}}%
\pgfpathlineto{\pgfqpoint{3.210572in}{2.193409in}}%
\pgfpathlineto{\pgfqpoint{3.197550in}{2.202138in}}%
\pgfpathlineto{\pgfqpoint{3.189655in}{2.193821in}}%
\pgfpathlineto{\pgfqpoint{3.181753in}{2.185574in}}%
\pgfpathlineto{\pgfqpoint{3.173844in}{2.177399in}}%
\pgfpathlineto{\pgfqpoint{3.165928in}{2.169295in}}%
\pgfpathclose%
\pgfusepath{fill}%
\end{pgfscope}%
\begin{pgfscope}%
\pgfpathrectangle{\pgfqpoint{1.254980in}{0.150000in}}{\pgfqpoint{5.490039in}{5.490039in}}%
\pgfusepath{clip}%
\pgfsetbuttcap%
\pgfsetroundjoin%
\definecolor{currentfill}{rgb}{0.143343,0.522773,0.556295}%
\pgfsetfillcolor{currentfill}%
\pgfsetfillopacity{0.700000}%
\pgfsetlinewidth{0.000000pt}%
\definecolor{currentstroke}{rgb}{0.000000,0.000000,0.000000}%
\pgfsetstrokecolor{currentstroke}%
\pgfsetdash{}{0pt}%
\pgfpathmoveto{\pgfqpoint{5.480281in}{3.118014in}}%
\pgfpathlineto{\pgfqpoint{5.494012in}{3.124138in}}%
\pgfpathlineto{\pgfqpoint{5.507757in}{3.130412in}}%
\pgfpathlineto{\pgfqpoint{5.521519in}{3.136838in}}%
\pgfpathlineto{\pgfqpoint{5.535296in}{3.143415in}}%
\pgfpathlineto{\pgfqpoint{5.542302in}{3.148835in}}%
\pgfpathlineto{\pgfqpoint{5.549306in}{3.154378in}}%
\pgfpathlineto{\pgfqpoint{5.556309in}{3.160052in}}%
\pgfpathlineto{\pgfqpoint{5.563309in}{3.165862in}}%
\pgfpathlineto{\pgfqpoint{5.549559in}{3.159862in}}%
\pgfpathlineto{\pgfqpoint{5.535824in}{3.154013in}}%
\pgfpathlineto{\pgfqpoint{5.522104in}{3.148314in}}%
\pgfpathlineto{\pgfqpoint{5.508400in}{3.142765in}}%
\pgfpathlineto{\pgfqpoint{5.501372in}{3.136369in}}%
\pgfpathlineto{\pgfqpoint{5.494344in}{3.130116in}}%
\pgfpathlineto{\pgfqpoint{5.487313in}{3.124000in}}%
\pgfpathlineto{\pgfqpoint{5.480281in}{3.118014in}}%
\pgfpathclose%
\pgfusepath{fill}%
\end{pgfscope}%
\begin{pgfscope}%
\pgfpathrectangle{\pgfqpoint{1.254980in}{0.150000in}}{\pgfqpoint{5.490039in}{5.490039in}}%
\pgfusepath{clip}%
\pgfsetbuttcap%
\pgfsetroundjoin%
\definecolor{currentfill}{rgb}{0.136408,0.541173,0.554483}%
\pgfsetfillcolor{currentfill}%
\pgfsetfillopacity{0.700000}%
\pgfsetlinewidth{0.000000pt}%
\definecolor{currentstroke}{rgb}{0.000000,0.000000,0.000000}%
\pgfsetstrokecolor{currentstroke}%
\pgfsetdash{}{0pt}%
\pgfpathmoveto{\pgfqpoint{5.563309in}{3.165862in}}%
\pgfpathlineto{\pgfqpoint{5.577075in}{3.172012in}}%
\pgfpathlineto{\pgfqpoint{5.590857in}{3.178313in}}%
\pgfpathlineto{\pgfqpoint{5.604654in}{3.184763in}}%
\pgfpathlineto{\pgfqpoint{5.618468in}{3.191364in}}%
\pgfpathlineto{\pgfqpoint{5.625439in}{3.196723in}}%
\pgfpathlineto{\pgfqpoint{5.632410in}{3.202225in}}%
\pgfpathlineto{\pgfqpoint{5.639379in}{3.207877in}}%
\pgfpathlineto{\pgfqpoint{5.646348in}{3.213685in}}%
\pgfpathlineto{\pgfqpoint{5.632563in}{3.207690in}}%
\pgfpathlineto{\pgfqpoint{5.618794in}{3.201844in}}%
\pgfpathlineto{\pgfqpoint{5.605041in}{3.196148in}}%
\pgfpathlineto{\pgfqpoint{5.591303in}{3.190601in}}%
\pgfpathlineto{\pgfqpoint{5.584305in}{3.184179in}}%
\pgfpathlineto{\pgfqpoint{5.577308in}{3.177919in}}%
\pgfpathlineto{\pgfqpoint{5.570309in}{3.171816in}}%
\pgfpathlineto{\pgfqpoint{5.563309in}{3.165862in}}%
\pgfpathclose%
\pgfusepath{fill}%
\end{pgfscope}%
\begin{pgfscope}%
\pgfpathrectangle{\pgfqpoint{1.254980in}{0.150000in}}{\pgfqpoint{5.490039in}{5.490039in}}%
\pgfusepath{clip}%
\pgfsetbuttcap%
\pgfsetroundjoin%
\definecolor{currentfill}{rgb}{0.129933,0.559582,0.551864}%
\pgfsetfillcolor{currentfill}%
\pgfsetfillopacity{0.700000}%
\pgfsetlinewidth{0.000000pt}%
\definecolor{currentstroke}{rgb}{0.000000,0.000000,0.000000}%
\pgfsetstrokecolor{currentstroke}%
\pgfsetdash{}{0pt}%
\pgfpathmoveto{\pgfqpoint{5.646348in}{3.213685in}}%
\pgfpathlineto{\pgfqpoint{5.660148in}{3.219830in}}%
\pgfpathlineto{\pgfqpoint{5.673965in}{3.226124in}}%
\pgfpathlineto{\pgfqpoint{5.687797in}{3.232568in}}%
\pgfpathlineto{\pgfqpoint{5.701645in}{3.239162in}}%
\pgfpathlineto{\pgfqpoint{5.708584in}{3.244510in}}%
\pgfpathlineto{\pgfqpoint{5.715522in}{3.250022in}}%
\pgfpathlineto{\pgfqpoint{5.722460in}{3.255706in}}%
\pgfpathlineto{\pgfqpoint{5.729399in}{3.261567in}}%
\pgfpathlineto{\pgfqpoint{5.715581in}{3.255607in}}%
\pgfpathlineto{\pgfqpoint{5.701779in}{3.249797in}}%
\pgfpathlineto{\pgfqpoint{5.687993in}{3.244135in}}%
\pgfpathlineto{\pgfqpoint{5.674222in}{3.238622in}}%
\pgfpathlineto{\pgfqpoint{5.667253in}{3.232119in}}%
\pgfpathlineto{\pgfqpoint{5.660285in}{3.225799in}}%
\pgfpathlineto{\pgfqpoint{5.653316in}{3.219657in}}%
\pgfpathlineto{\pgfqpoint{5.646348in}{3.213685in}}%
\pgfpathclose%
\pgfusepath{fill}%
\end{pgfscope}%
\begin{pgfscope}%
\pgfpathrectangle{\pgfqpoint{1.254980in}{0.150000in}}{\pgfqpoint{5.490039in}{5.490039in}}%
\pgfusepath{clip}%
\pgfsetbuttcap%
\pgfsetroundjoin%
\definecolor{currentfill}{rgb}{0.282623,0.140926,0.457517}%
\pgfsetfillcolor{currentfill}%
\pgfsetfillopacity{0.700000}%
\pgfsetlinewidth{0.000000pt}%
\definecolor{currentstroke}{rgb}{0.000000,0.000000,0.000000}%
\pgfsetstrokecolor{currentstroke}%
\pgfsetdash{}{0pt}%
\pgfpathmoveto{\pgfqpoint{2.977375in}{2.263878in}}%
\pgfpathlineto{\pgfqpoint{2.990456in}{2.252110in}}%
\pgfpathlineto{\pgfqpoint{3.003535in}{2.240573in}}%
\pgfpathlineto{\pgfqpoint{3.016610in}{2.229268in}}%
\pgfpathlineto{\pgfqpoint{3.029683in}{2.218191in}}%
\pgfpathlineto{\pgfqpoint{3.037668in}{2.225583in}}%
\pgfpathlineto{\pgfqpoint{3.045644in}{2.233068in}}%
\pgfpathlineto{\pgfqpoint{3.053613in}{2.240643in}}%
\pgfpathlineto{\pgfqpoint{3.061574in}{2.248308in}}%
\pgfpathlineto{\pgfqpoint{3.048521in}{2.259191in}}%
\pgfpathlineto{\pgfqpoint{3.035466in}{2.270302in}}%
\pgfpathlineto{\pgfqpoint{3.022408in}{2.281644in}}%
\pgfpathlineto{\pgfqpoint{3.009348in}{2.293218in}}%
\pgfpathlineto{\pgfqpoint{3.001367in}{2.285737in}}%
\pgfpathlineto{\pgfqpoint{2.993378in}{2.278352in}}%
\pgfpathlineto{\pgfqpoint{2.985380in}{2.271065in}}%
\pgfpathlineto{\pgfqpoint{2.977375in}{2.263878in}}%
\pgfpathclose%
\pgfusepath{fill}%
\end{pgfscope}%
\begin{pgfscope}%
\pgfpathrectangle{\pgfqpoint{1.254980in}{0.150000in}}{\pgfqpoint{5.490039in}{5.490039in}}%
\pgfusepath{clip}%
\pgfsetbuttcap%
\pgfsetroundjoin%
\definecolor{currentfill}{rgb}{0.124395,0.578002,0.548287}%
\pgfsetfillcolor{currentfill}%
\pgfsetfillopacity{0.700000}%
\pgfsetlinewidth{0.000000pt}%
\definecolor{currentstroke}{rgb}{0.000000,0.000000,0.000000}%
\pgfsetstrokecolor{currentstroke}%
\pgfsetdash{}{0pt}%
\pgfpathmoveto{\pgfqpoint{5.729399in}{3.261567in}}%
\pgfpathlineto{\pgfqpoint{5.743233in}{3.267675in}}%
\pgfpathlineto{\pgfqpoint{5.757083in}{3.273932in}}%
\pgfpathlineto{\pgfqpoint{5.770949in}{3.280338in}}%
\pgfpathlineto{\pgfqpoint{5.784831in}{3.286893in}}%
\pgfpathlineto{\pgfqpoint{5.791738in}{3.292286in}}%
\pgfpathlineto{\pgfqpoint{5.798646in}{3.297866in}}%
\pgfpathlineto{\pgfqpoint{5.805556in}{3.303639in}}%
\pgfpathlineto{\pgfqpoint{5.812466in}{3.309612in}}%
\pgfpathlineto{\pgfqpoint{5.798616in}{3.303720in}}%
\pgfpathlineto{\pgfqpoint{5.784783in}{3.297975in}}%
\pgfpathlineto{\pgfqpoint{5.770965in}{3.292379in}}%
\pgfpathlineto{\pgfqpoint{5.757162in}{3.286932in}}%
\pgfpathlineto{\pgfqpoint{5.750220in}{3.280288in}}%
\pgfpathlineto{\pgfqpoint{5.743278in}{3.273851in}}%
\pgfpathlineto{\pgfqpoint{5.736338in}{3.267613in}}%
\pgfpathlineto{\pgfqpoint{5.729399in}{3.261567in}}%
\pgfpathclose%
\pgfusepath{fill}%
\end{pgfscope}%
\begin{pgfscope}%
\pgfpathrectangle{\pgfqpoint{1.254980in}{0.150000in}}{\pgfqpoint{5.490039in}{5.490039in}}%
\pgfusepath{clip}%
\pgfsetbuttcap%
\pgfsetroundjoin%
\definecolor{currentfill}{rgb}{0.206756,0.371758,0.553117}%
\pgfsetfillcolor{currentfill}%
\pgfsetfillopacity{0.700000}%
\pgfsetlinewidth{0.000000pt}%
\definecolor{currentstroke}{rgb}{0.000000,0.000000,0.000000}%
\pgfsetstrokecolor{currentstroke}%
\pgfsetdash{}{0pt}%
\pgfpathmoveto{\pgfqpoint{2.555671in}{2.775168in}}%
\pgfpathlineto{\pgfqpoint{2.568983in}{2.754851in}}%
\pgfpathlineto{\pgfqpoint{2.582283in}{2.734841in}}%
\pgfpathlineto{\pgfqpoint{2.595573in}{2.715134in}}%
\pgfpathlineto{\pgfqpoint{2.608851in}{2.695728in}}%
\pgfpathlineto{\pgfqpoint{2.617034in}{2.701444in}}%
\pgfpathlineto{\pgfqpoint{2.625206in}{2.707310in}}%
\pgfpathlineto{\pgfqpoint{2.633367in}{2.713324in}}%
\pgfpathlineto{\pgfqpoint{2.641517in}{2.719484in}}%
\pgfpathlineto{\pgfqpoint{2.628268in}{2.738685in}}%
\pgfpathlineto{\pgfqpoint{2.615008in}{2.758186in}}%
\pgfpathlineto{\pgfqpoint{2.601737in}{2.777989in}}%
\pgfpathlineto{\pgfqpoint{2.588455in}{2.798099in}}%
\pgfpathlineto{\pgfqpoint{2.580276in}{2.792134in}}%
\pgfpathlineto{\pgfqpoint{2.572085in}{2.786323in}}%
\pgfpathlineto{\pgfqpoint{2.563884in}{2.780667in}}%
\pgfpathlineto{\pgfqpoint{2.555671in}{2.775168in}}%
\pgfpathclose%
\pgfusepath{fill}%
\end{pgfscope}%
\begin{pgfscope}%
\pgfpathrectangle{\pgfqpoint{1.254980in}{0.150000in}}{\pgfqpoint{5.490039in}{5.490039in}}%
\pgfusepath{clip}%
\pgfsetbuttcap%
\pgfsetroundjoin%
\definecolor{currentfill}{rgb}{0.280894,0.078907,0.402329}%
\pgfsetfillcolor{currentfill}%
\pgfsetfillopacity{0.700000}%
\pgfsetlinewidth{0.000000pt}%
\definecolor{currentstroke}{rgb}{0.000000,0.000000,0.000000}%
\pgfsetstrokecolor{currentstroke}%
\pgfsetdash{}{0pt}%
\pgfpathmoveto{\pgfqpoint{3.572609in}{2.123887in}}%
\pgfpathlineto{\pgfqpoint{3.585649in}{2.120202in}}%
\pgfpathlineto{\pgfqpoint{3.598694in}{2.116705in}}%
\pgfpathlineto{\pgfqpoint{3.611742in}{2.113395in}}%
\pgfpathlineto{\pgfqpoint{3.624795in}{2.110270in}}%
\pgfpathlineto{\pgfqpoint{3.632542in}{2.119801in}}%
\pgfpathlineto{\pgfqpoint{3.640283in}{2.129346in}}%
\pgfpathlineto{\pgfqpoint{3.648019in}{2.138903in}}%
\pgfpathlineto{\pgfqpoint{3.655750in}{2.148473in}}%
\pgfpathlineto{\pgfqpoint{3.642707in}{2.151520in}}%
\pgfpathlineto{\pgfqpoint{3.629669in}{2.154752in}}%
\pgfpathlineto{\pgfqpoint{3.616636in}{2.158171in}}%
\pgfpathlineto{\pgfqpoint{3.603607in}{2.161777in}}%
\pgfpathlineto{\pgfqpoint{3.595866in}{2.152275in}}%
\pgfpathlineto{\pgfqpoint{3.588119in}{2.142792in}}%
\pgfpathlineto{\pgfqpoint{3.580367in}{2.133330in}}%
\pgfpathlineto{\pgfqpoint{3.572609in}{2.123887in}}%
\pgfpathclose%
\pgfusepath{fill}%
\end{pgfscope}%
\begin{pgfscope}%
\pgfpathrectangle{\pgfqpoint{1.254980in}{0.150000in}}{\pgfqpoint{5.490039in}{5.490039in}}%
\pgfusepath{clip}%
\pgfsetbuttcap%
\pgfsetroundjoin%
\definecolor{currentfill}{rgb}{0.271828,0.209303,0.504434}%
\pgfsetfillcolor{currentfill}%
\pgfsetfillopacity{0.700000}%
\pgfsetlinewidth{0.000000pt}%
\definecolor{currentstroke}{rgb}{0.000000,0.000000,0.000000}%
\pgfsetstrokecolor{currentstroke}%
\pgfsetdash{}{0pt}%
\pgfpathmoveto{\pgfqpoint{4.206026in}{2.366393in}}%
\pgfpathlineto{\pgfqpoint{4.219224in}{2.368330in}}%
\pgfpathlineto{\pgfqpoint{4.232432in}{2.370435in}}%
\pgfpathlineto{\pgfqpoint{4.245649in}{2.372707in}}%
\pgfpathlineto{\pgfqpoint{4.258876in}{2.375146in}}%
\pgfpathlineto{\pgfqpoint{4.266413in}{2.384281in}}%
\pgfpathlineto{\pgfqpoint{4.273944in}{2.393389in}}%
\pgfpathlineto{\pgfqpoint{4.281471in}{2.402473in}}%
\pgfpathlineto{\pgfqpoint{4.288992in}{2.411533in}}%
\pgfpathlineto{\pgfqpoint{4.275774in}{2.409212in}}%
\pgfpathlineto{\pgfqpoint{4.262565in}{2.407058in}}%
\pgfpathlineto{\pgfqpoint{4.249365in}{2.405071in}}%
\pgfpathlineto{\pgfqpoint{4.236175in}{2.403251in}}%
\pgfpathlineto{\pgfqpoint{4.228646in}{2.394063in}}%
\pgfpathlineto{\pgfqpoint{4.221111in}{2.384858in}}%
\pgfpathlineto{\pgfqpoint{4.213571in}{2.375635in}}%
\pgfpathlineto{\pgfqpoint{4.206026in}{2.366393in}}%
\pgfpathclose%
\pgfusepath{fill}%
\end{pgfscope}%
\begin{pgfscope}%
\pgfpathrectangle{\pgfqpoint{1.254980in}{0.150000in}}{\pgfqpoint{5.490039in}{5.490039in}}%
\pgfusepath{clip}%
\pgfsetbuttcap%
\pgfsetroundjoin%
\definecolor{currentfill}{rgb}{0.277134,0.185228,0.489898}%
\pgfsetfillcolor{currentfill}%
\pgfsetfillopacity{0.700000}%
\pgfsetlinewidth{0.000000pt}%
\definecolor{currentstroke}{rgb}{0.000000,0.000000,0.000000}%
\pgfsetstrokecolor{currentstroke}%
\pgfsetdash{}{0pt}%
\pgfpathmoveto{\pgfqpoint{4.123064in}{2.322782in}}%
\pgfpathlineto{\pgfqpoint{4.136234in}{2.324132in}}%
\pgfpathlineto{\pgfqpoint{4.149412in}{2.325653in}}%
\pgfpathlineto{\pgfqpoint{4.162600in}{2.327342in}}%
\pgfpathlineto{\pgfqpoint{4.175796in}{2.329200in}}%
\pgfpathlineto{\pgfqpoint{4.183361in}{2.338535in}}%
\pgfpathlineto{\pgfqpoint{4.190921in}{2.347845in}}%
\pgfpathlineto{\pgfqpoint{4.198477in}{2.357130in}}%
\pgfpathlineto{\pgfqpoint{4.206026in}{2.366393in}}%
\pgfpathlineto{\pgfqpoint{4.192838in}{2.364625in}}%
\pgfpathlineto{\pgfqpoint{4.179658in}{2.363025in}}%
\pgfpathlineto{\pgfqpoint{4.166488in}{2.361594in}}%
\pgfpathlineto{\pgfqpoint{4.153326in}{2.360333in}}%
\pgfpathlineto{\pgfqpoint{4.145768in}{2.350970in}}%
\pgfpathlineto{\pgfqpoint{4.138205in}{2.341591in}}%
\pgfpathlineto{\pgfqpoint{4.130637in}{2.332196in}}%
\pgfpathlineto{\pgfqpoint{4.123064in}{2.322782in}}%
\pgfpathclose%
\pgfusepath{fill}%
\end{pgfscope}%
\begin{pgfscope}%
\pgfpathrectangle{\pgfqpoint{1.254980in}{0.150000in}}{\pgfqpoint{5.490039in}{5.490039in}}%
\pgfusepath{clip}%
\pgfsetbuttcap%
\pgfsetroundjoin%
\definecolor{currentfill}{rgb}{0.265145,0.232956,0.516599}%
\pgfsetfillcolor{currentfill}%
\pgfsetfillopacity{0.700000}%
\pgfsetlinewidth{0.000000pt}%
\definecolor{currentstroke}{rgb}{0.000000,0.000000,0.000000}%
\pgfsetstrokecolor{currentstroke}%
\pgfsetdash{}{0pt}%
\pgfpathmoveto{\pgfqpoint{4.288992in}{2.411533in}}%
\pgfpathlineto{\pgfqpoint{4.302221in}{2.414021in}}%
\pgfpathlineto{\pgfqpoint{4.315460in}{2.416675in}}%
\pgfpathlineto{\pgfqpoint{4.328708in}{2.419495in}}%
\pgfpathlineto{\pgfqpoint{4.341967in}{2.422480in}}%
\pgfpathlineto{\pgfqpoint{4.349475in}{2.431383in}}%
\pgfpathlineto{\pgfqpoint{4.356978in}{2.440260in}}%
\pgfpathlineto{\pgfqpoint{4.364476in}{2.449112in}}%
\pgfpathlineto{\pgfqpoint{4.371968in}{2.457942in}}%
\pgfpathlineto{\pgfqpoint{4.358717in}{2.455103in}}%
\pgfpathlineto{\pgfqpoint{4.345477in}{2.452430in}}%
\pgfpathlineto{\pgfqpoint{4.332247in}{2.449922in}}%
\pgfpathlineto{\pgfqpoint{4.319027in}{2.447580in}}%
\pgfpathlineto{\pgfqpoint{4.311526in}{2.438593in}}%
\pgfpathlineto{\pgfqpoint{4.304020in}{2.429592in}}%
\pgfpathlineto{\pgfqpoint{4.296509in}{2.420572in}}%
\pgfpathlineto{\pgfqpoint{4.288992in}{2.411533in}}%
\pgfpathclose%
\pgfusepath{fill}%
\end{pgfscope}%
\begin{pgfscope}%
\pgfpathrectangle{\pgfqpoint{1.254980in}{0.150000in}}{\pgfqpoint{5.490039in}{5.490039in}}%
\pgfusepath{clip}%
\pgfsetbuttcap%
\pgfsetroundjoin%
\definecolor{currentfill}{rgb}{0.280255,0.165693,0.476498}%
\pgfsetfillcolor{currentfill}%
\pgfsetfillopacity{0.700000}%
\pgfsetlinewidth{0.000000pt}%
\definecolor{currentstroke}{rgb}{0.000000,0.000000,0.000000}%
\pgfsetstrokecolor{currentstroke}%
\pgfsetdash{}{0pt}%
\pgfpathmoveto{\pgfqpoint{4.040098in}{2.280978in}}%
\pgfpathlineto{\pgfqpoint{4.053241in}{2.281706in}}%
\pgfpathlineto{\pgfqpoint{4.066393in}{2.282606in}}%
\pgfpathlineto{\pgfqpoint{4.079553in}{2.283678in}}%
\pgfpathlineto{\pgfqpoint{4.092722in}{2.284920in}}%
\pgfpathlineto{\pgfqpoint{4.100315in}{2.294419in}}%
\pgfpathlineto{\pgfqpoint{4.107903in}{2.303894in}}%
\pgfpathlineto{\pgfqpoint{4.115486in}{2.313348in}}%
\pgfpathlineto{\pgfqpoint{4.123064in}{2.322782in}}%
\pgfpathlineto{\pgfqpoint{4.109903in}{2.321601in}}%
\pgfpathlineto{\pgfqpoint{4.096751in}{2.320591in}}%
\pgfpathlineto{\pgfqpoint{4.083607in}{2.319752in}}%
\pgfpathlineto{\pgfqpoint{4.070472in}{2.319085in}}%
\pgfpathlineto{\pgfqpoint{4.062886in}{2.309580in}}%
\pgfpathlineto{\pgfqpoint{4.055295in}{2.300061in}}%
\pgfpathlineto{\pgfqpoint{4.047699in}{2.290527in}}%
\pgfpathlineto{\pgfqpoint{4.040098in}{2.280978in}}%
\pgfpathclose%
\pgfusepath{fill}%
\end{pgfscope}%
\begin{pgfscope}%
\pgfpathrectangle{\pgfqpoint{1.254980in}{0.150000in}}{\pgfqpoint{5.490039in}{5.490039in}}%
\pgfusepath{clip}%
\pgfsetbuttcap%
\pgfsetroundjoin%
\definecolor{currentfill}{rgb}{0.257322,0.256130,0.526563}%
\pgfsetfillcolor{currentfill}%
\pgfsetfillopacity{0.700000}%
\pgfsetlinewidth{0.000000pt}%
\definecolor{currentstroke}{rgb}{0.000000,0.000000,0.000000}%
\pgfsetstrokecolor{currentstroke}%
\pgfsetdash{}{0pt}%
\pgfpathmoveto{\pgfqpoint{4.371968in}{2.457942in}}%
\pgfpathlineto{\pgfqpoint{4.385229in}{2.460946in}}%
\pgfpathlineto{\pgfqpoint{4.398500in}{2.464114in}}%
\pgfpathlineto{\pgfqpoint{4.411783in}{2.467447in}}%
\pgfpathlineto{\pgfqpoint{4.425076in}{2.470945in}}%
\pgfpathlineto{\pgfqpoint{4.432554in}{2.479589in}}%
\pgfpathlineto{\pgfqpoint{4.440027in}{2.488209in}}%
\pgfpathlineto{\pgfqpoint{4.447495in}{2.496806in}}%
\pgfpathlineto{\pgfqpoint{4.454958in}{2.505382in}}%
\pgfpathlineto{\pgfqpoint{4.441674in}{2.502060in}}%
\pgfpathlineto{\pgfqpoint{4.428400in}{2.498901in}}%
\pgfpathlineto{\pgfqpoint{4.415138in}{2.495907in}}%
\pgfpathlineto{\pgfqpoint{4.401885in}{2.493077in}}%
\pgfpathlineto{\pgfqpoint{4.394414in}{2.484317in}}%
\pgfpathlineto{\pgfqpoint{4.386937in}{2.475542in}}%
\pgfpathlineto{\pgfqpoint{4.379455in}{2.466751in}}%
\pgfpathlineto{\pgfqpoint{4.371968in}{2.457942in}}%
\pgfpathclose%
\pgfusepath{fill}%
\end{pgfscope}%
\begin{pgfscope}%
\pgfpathrectangle{\pgfqpoint{1.254980in}{0.150000in}}{\pgfqpoint{5.490039in}{5.490039in}}%
\pgfusepath{clip}%
\pgfsetbuttcap%
\pgfsetroundjoin%
\definecolor{currentfill}{rgb}{0.120565,0.596422,0.543611}%
\pgfsetfillcolor{currentfill}%
\pgfsetfillopacity{0.700000}%
\pgfsetlinewidth{0.000000pt}%
\definecolor{currentstroke}{rgb}{0.000000,0.000000,0.000000}%
\pgfsetstrokecolor{currentstroke}%
\pgfsetdash{}{0pt}%
\pgfpathmoveto{\pgfqpoint{5.812466in}{3.309612in}}%
\pgfpathlineto{\pgfqpoint{5.826332in}{3.315652in}}%
\pgfpathlineto{\pgfqpoint{5.840214in}{3.321840in}}%
\pgfpathlineto{\pgfqpoint{5.854113in}{3.328177in}}%
\pgfpathlineto{\pgfqpoint{5.868028in}{3.334662in}}%
\pgfpathlineto{\pgfqpoint{5.874907in}{3.340162in}}%
\pgfpathlineto{\pgfqpoint{5.881787in}{3.345871in}}%
\pgfpathlineto{\pgfqpoint{5.888670in}{3.351797in}}%
\pgfpathlineto{\pgfqpoint{5.874780in}{3.345828in}}%
\pgfpathlineto{\pgfqpoint{5.860907in}{3.340006in}}%
\pgfpathlineto{\pgfqpoint{5.847051in}{3.334333in}}%
\pgfpathlineto{\pgfqpoint{5.833210in}{3.328806in}}%
\pgfpathlineto{\pgfqpoint{5.826293in}{3.322188in}}%
\pgfpathlineto{\pgfqpoint{5.819379in}{3.315792in}}%
\pgfpathlineto{\pgfqpoint{5.812466in}{3.309612in}}%
\pgfpathclose%
\pgfusepath{fill}%
\end{pgfscope}%
\begin{pgfscope}%
\pgfpathrectangle{\pgfqpoint{1.254980in}{0.150000in}}{\pgfqpoint{5.490039in}{5.490039in}}%
\pgfusepath{clip}%
\pgfsetbuttcap%
\pgfsetroundjoin%
\definecolor{currentfill}{rgb}{0.282290,0.145912,0.461510}%
\pgfsetfillcolor{currentfill}%
\pgfsetfillopacity{0.700000}%
\pgfsetlinewidth{0.000000pt}%
\definecolor{currentstroke}{rgb}{0.000000,0.000000,0.000000}%
\pgfsetstrokecolor{currentstroke}%
\pgfsetdash{}{0pt}%
\pgfpathmoveto{\pgfqpoint{3.957118in}{2.241280in}}%
\pgfpathlineto{\pgfqpoint{3.970238in}{2.241350in}}%
\pgfpathlineto{\pgfqpoint{3.983366in}{2.241594in}}%
\pgfpathlineto{\pgfqpoint{3.996501in}{2.242012in}}%
\pgfpathlineto{\pgfqpoint{4.009644in}{2.242602in}}%
\pgfpathlineto{\pgfqpoint{4.017265in}{2.252224in}}%
\pgfpathlineto{\pgfqpoint{4.024881in}{2.261827in}}%
\pgfpathlineto{\pgfqpoint{4.032492in}{2.271411in}}%
\pgfpathlineto{\pgfqpoint{4.040098in}{2.280978in}}%
\pgfpathlineto{\pgfqpoint{4.026963in}{2.280421in}}%
\pgfpathlineto{\pgfqpoint{4.013835in}{2.280037in}}%
\pgfpathlineto{\pgfqpoint{4.000716in}{2.279826in}}%
\pgfpathlineto{\pgfqpoint{3.987604in}{2.279789in}}%
\pgfpathlineto{\pgfqpoint{3.979990in}{2.270179in}}%
\pgfpathlineto{\pgfqpoint{3.972371in}{2.260558in}}%
\pgfpathlineto{\pgfqpoint{3.964747in}{2.250925in}}%
\pgfpathlineto{\pgfqpoint{3.957118in}{2.241280in}}%
\pgfpathclose%
\pgfusepath{fill}%
\end{pgfscope}%
\begin{pgfscope}%
\pgfpathrectangle{\pgfqpoint{1.254980in}{0.150000in}}{\pgfqpoint{5.490039in}{5.490039in}}%
\pgfusepath{clip}%
\pgfsetbuttcap%
\pgfsetroundjoin%
\definecolor{currentfill}{rgb}{0.279566,0.067836,0.391917}%
\pgfsetfillcolor{currentfill}%
\pgfsetfillopacity{0.700000}%
\pgfsetlinewidth{0.000000pt}%
\definecolor{currentstroke}{rgb}{0.000000,0.000000,0.000000}%
\pgfsetstrokecolor{currentstroke}%
\pgfsetdash{}{0pt}%
\pgfpathmoveto{\pgfqpoint{3.353831in}{2.111210in}}%
\pgfpathlineto{\pgfqpoint{3.366861in}{2.104962in}}%
\pgfpathlineto{\pgfqpoint{3.379893in}{2.098913in}}%
\pgfpathlineto{\pgfqpoint{3.392927in}{2.093062in}}%
\pgfpathlineto{\pgfqpoint{3.405963in}{2.087408in}}%
\pgfpathlineto{\pgfqpoint{3.413791in}{2.096376in}}%
\pgfpathlineto{\pgfqpoint{3.421612in}{2.105384in}}%
\pgfpathlineto{\pgfqpoint{3.429428in}{2.114431in}}%
\pgfpathlineto{\pgfqpoint{3.437238in}{2.123516in}}%
\pgfpathlineto{\pgfqpoint{3.424215in}{2.129036in}}%
\pgfpathlineto{\pgfqpoint{3.411195in}{2.134752in}}%
\pgfpathlineto{\pgfqpoint{3.398177in}{2.140667in}}%
\pgfpathlineto{\pgfqpoint{3.385161in}{2.146780in}}%
\pgfpathlineto{\pgfqpoint{3.377338in}{2.137819in}}%
\pgfpathlineto{\pgfqpoint{3.369508in}{2.128903in}}%
\pgfpathlineto{\pgfqpoint{3.361673in}{2.120033in}}%
\pgfpathlineto{\pgfqpoint{3.353831in}{2.111210in}}%
\pgfpathclose%
\pgfusepath{fill}%
\end{pgfscope}%
\begin{pgfscope}%
\pgfpathrectangle{\pgfqpoint{1.254980in}{0.150000in}}{\pgfqpoint{5.490039in}{5.490039in}}%
\pgfusepath{clip}%
\pgfsetbuttcap%
\pgfsetroundjoin%
\definecolor{currentfill}{rgb}{0.248629,0.278775,0.534556}%
\pgfsetfillcolor{currentfill}%
\pgfsetfillopacity{0.700000}%
\pgfsetlinewidth{0.000000pt}%
\definecolor{currentstroke}{rgb}{0.000000,0.000000,0.000000}%
\pgfsetstrokecolor{currentstroke}%
\pgfsetdash{}{0pt}%
\pgfpathmoveto{\pgfqpoint{4.454958in}{2.505382in}}%
\pgfpathlineto{\pgfqpoint{4.468253in}{2.508867in}}%
\pgfpathlineto{\pgfqpoint{4.481559in}{2.512516in}}%
\pgfpathlineto{\pgfqpoint{4.494876in}{2.516328in}}%
\pgfpathlineto{\pgfqpoint{4.508204in}{2.520303in}}%
\pgfpathlineto{\pgfqpoint{4.515652in}{2.528667in}}%
\pgfpathlineto{\pgfqpoint{4.523095in}{2.537009in}}%
\pgfpathlineto{\pgfqpoint{4.530532in}{2.545330in}}%
\pgfpathlineto{\pgfqpoint{4.537964in}{2.553633in}}%
\pgfpathlineto{\pgfqpoint{4.524645in}{2.549862in}}%
\pgfpathlineto{\pgfqpoint{4.511337in}{2.546253in}}%
\pgfpathlineto{\pgfqpoint{4.498041in}{2.542807in}}%
\pgfpathlineto{\pgfqpoint{4.484755in}{2.539524in}}%
\pgfpathlineto{\pgfqpoint{4.477314in}{2.531008in}}%
\pgfpathlineto{\pgfqpoint{4.469867in}{2.522480in}}%
\pgfpathlineto{\pgfqpoint{4.462415in}{2.513939in}}%
\pgfpathlineto{\pgfqpoint{4.454958in}{2.505382in}}%
\pgfpathclose%
\pgfusepath{fill}%
\end{pgfscope}%
\begin{pgfscope}%
\pgfpathrectangle{\pgfqpoint{1.254980in}{0.150000in}}{\pgfqpoint{5.490039in}{5.490039in}}%
\pgfusepath{clip}%
\pgfsetbuttcap%
\pgfsetroundjoin%
\definecolor{currentfill}{rgb}{0.283229,0.120777,0.440584}%
\pgfsetfillcolor{currentfill}%
\pgfsetfillopacity{0.700000}%
\pgfsetlinewidth{0.000000pt}%
\definecolor{currentstroke}{rgb}{0.000000,0.000000,0.000000}%
\pgfsetstrokecolor{currentstroke}%
\pgfsetdash{}{0pt}%
\pgfpathmoveto{\pgfqpoint{3.029683in}{2.218191in}}%
\pgfpathlineto{\pgfqpoint{3.042753in}{2.207343in}}%
\pgfpathlineto{\pgfqpoint{3.055821in}{2.196719in}}%
\pgfpathlineto{\pgfqpoint{3.068887in}{2.186320in}}%
\pgfpathlineto{\pgfqpoint{3.081951in}{2.176144in}}%
\pgfpathlineto{\pgfqpoint{3.089915in}{2.183740in}}%
\pgfpathlineto{\pgfqpoint{3.097872in}{2.191421in}}%
\pgfpathlineto{\pgfqpoint{3.105821in}{2.199186in}}%
\pgfpathlineto{\pgfqpoint{3.113763in}{2.207034in}}%
\pgfpathlineto{\pgfqpoint{3.100718in}{2.217017in}}%
\pgfpathlineto{\pgfqpoint{3.087672in}{2.227223in}}%
\pgfpathlineto{\pgfqpoint{3.074624in}{2.237653in}}%
\pgfpathlineto{\pgfqpoint{3.061574in}{2.248308in}}%
\pgfpathlineto{\pgfqpoint{3.053613in}{2.240643in}}%
\pgfpathlineto{\pgfqpoint{3.045644in}{2.233068in}}%
\pgfpathlineto{\pgfqpoint{3.037668in}{2.225583in}}%
\pgfpathlineto{\pgfqpoint{3.029683in}{2.218191in}}%
\pgfpathclose%
\pgfusepath{fill}%
\end{pgfscope}%
\begin{pgfscope}%
\pgfpathrectangle{\pgfqpoint{1.254980in}{0.150000in}}{\pgfqpoint{5.490039in}{5.490039in}}%
\pgfusepath{clip}%
\pgfsetbuttcap%
\pgfsetroundjoin%
\definecolor{currentfill}{rgb}{0.280894,0.078907,0.402329}%
\pgfsetfillcolor{currentfill}%
\pgfsetfillopacity{0.700000}%
\pgfsetlinewidth{0.000000pt}%
\definecolor{currentstroke}{rgb}{0.000000,0.000000,0.000000}%
\pgfsetstrokecolor{currentstroke}%
\pgfsetdash{}{0pt}%
\pgfpathmoveto{\pgfqpoint{3.218081in}{2.135001in}}%
\pgfpathlineto{\pgfqpoint{3.231119in}{2.126956in}}%
\pgfpathlineto{\pgfqpoint{3.244157in}{2.119119in}}%
\pgfpathlineto{\pgfqpoint{3.257196in}{2.111489in}}%
\pgfpathlineto{\pgfqpoint{3.270235in}{2.104065in}}%
\pgfpathlineto{\pgfqpoint{3.278118in}{2.112516in}}%
\pgfpathlineto{\pgfqpoint{3.285994in}{2.121026in}}%
\pgfpathlineto{\pgfqpoint{3.293863in}{2.129593in}}%
\pgfpathlineto{\pgfqpoint{3.301726in}{2.138216in}}%
\pgfpathlineto{\pgfqpoint{3.288702in}{2.145477in}}%
\pgfpathlineto{\pgfqpoint{3.275680in}{2.152943in}}%
\pgfpathlineto{\pgfqpoint{3.262657in}{2.160617in}}%
\pgfpathlineto{\pgfqpoint{3.249636in}{2.168499in}}%
\pgfpathlineto{\pgfqpoint{3.241757in}{2.160029in}}%
\pgfpathlineto{\pgfqpoint{3.233872in}{2.151622in}}%
\pgfpathlineto{\pgfqpoint{3.225980in}{2.143279in}}%
\pgfpathlineto{\pgfqpoint{3.218081in}{2.135001in}}%
\pgfpathclose%
\pgfusepath{fill}%
\end{pgfscope}%
\begin{pgfscope}%
\pgfpathrectangle{\pgfqpoint{1.254980in}{0.150000in}}{\pgfqpoint{5.490039in}{5.490039in}}%
\pgfusepath{clip}%
\pgfsetbuttcap%
\pgfsetroundjoin%
\definecolor{currentfill}{rgb}{0.239346,0.300855,0.540844}%
\pgfsetfillcolor{currentfill}%
\pgfsetfillopacity{0.700000}%
\pgfsetlinewidth{0.000000pt}%
\definecolor{currentstroke}{rgb}{0.000000,0.000000,0.000000}%
\pgfsetstrokecolor{currentstroke}%
\pgfsetdash{}{0pt}%
\pgfpathmoveto{\pgfqpoint{4.537964in}{2.553633in}}%
\pgfpathlineto{\pgfqpoint{4.551295in}{2.557566in}}%
\pgfpathlineto{\pgfqpoint{4.564637in}{2.561661in}}%
\pgfpathlineto{\pgfqpoint{4.577990in}{2.565918in}}%
\pgfpathlineto{\pgfqpoint{4.591356in}{2.570337in}}%
\pgfpathlineto{\pgfqpoint{4.598772in}{2.578403in}}%
\pgfpathlineto{\pgfqpoint{4.606184in}{2.586450in}}%
\pgfpathlineto{\pgfqpoint{4.613589in}{2.594480in}}%
\pgfpathlineto{\pgfqpoint{4.620990in}{2.602496in}}%
\pgfpathlineto{\pgfqpoint{4.607635in}{2.598310in}}%
\pgfpathlineto{\pgfqpoint{4.594291in}{2.594285in}}%
\pgfpathlineto{\pgfqpoint{4.580959in}{2.590422in}}%
\pgfpathlineto{\pgfqpoint{4.567639in}{2.586720in}}%
\pgfpathlineto{\pgfqpoint{4.560228in}{2.578461in}}%
\pgfpathlineto{\pgfqpoint{4.552812in}{2.570196in}}%
\pgfpathlineto{\pgfqpoint{4.545391in}{2.561921in}}%
\pgfpathlineto{\pgfqpoint{4.537964in}{2.553633in}}%
\pgfpathclose%
\pgfusepath{fill}%
\end{pgfscope}%
\begin{pgfscope}%
\pgfpathrectangle{\pgfqpoint{1.254980in}{0.150000in}}{\pgfqpoint{5.490039in}{5.490039in}}%
\pgfusepath{clip}%
\pgfsetbuttcap%
\pgfsetroundjoin%
\definecolor{currentfill}{rgb}{0.283187,0.125848,0.444960}%
\pgfsetfillcolor{currentfill}%
\pgfsetfillopacity{0.700000}%
\pgfsetlinewidth{0.000000pt}%
\definecolor{currentstroke}{rgb}{0.000000,0.000000,0.000000}%
\pgfsetstrokecolor{currentstroke}%
\pgfsetdash{}{0pt}%
\pgfpathmoveto{\pgfqpoint{3.874113in}{2.204008in}}%
\pgfpathlineto{\pgfqpoint{3.887212in}{2.203384in}}%
\pgfpathlineto{\pgfqpoint{3.900319in}{2.202935in}}%
\pgfpathlineto{\pgfqpoint{3.913432in}{2.202662in}}%
\pgfpathlineto{\pgfqpoint{3.926553in}{2.202563in}}%
\pgfpathlineto{\pgfqpoint{3.934202in}{2.212264in}}%
\pgfpathlineto{\pgfqpoint{3.941845in}{2.221950in}}%
\pgfpathlineto{\pgfqpoint{3.949484in}{2.231622in}}%
\pgfpathlineto{\pgfqpoint{3.957118in}{2.241280in}}%
\pgfpathlineto{\pgfqpoint{3.944006in}{2.241384in}}%
\pgfpathlineto{\pgfqpoint{3.930900in}{2.241662in}}%
\pgfpathlineto{\pgfqpoint{3.917802in}{2.242116in}}%
\pgfpathlineto{\pgfqpoint{3.904711in}{2.242746in}}%
\pgfpathlineto{\pgfqpoint{3.897069in}{2.233072in}}%
\pgfpathlineto{\pgfqpoint{3.889422in}{2.223392in}}%
\pgfpathlineto{\pgfqpoint{3.881770in}{2.213704in}}%
\pgfpathlineto{\pgfqpoint{3.874113in}{2.204008in}}%
\pgfpathclose%
\pgfusepath{fill}%
\end{pgfscope}%
\begin{pgfscope}%
\pgfpathrectangle{\pgfqpoint{1.254980in}{0.150000in}}{\pgfqpoint{5.490039in}{5.490039in}}%
\pgfusepath{clip}%
\pgfsetbuttcap%
\pgfsetroundjoin%
\definecolor{currentfill}{rgb}{0.229739,0.322361,0.545706}%
\pgfsetfillcolor{currentfill}%
\pgfsetfillopacity{0.700000}%
\pgfsetlinewidth{0.000000pt}%
\definecolor{currentstroke}{rgb}{0.000000,0.000000,0.000000}%
\pgfsetstrokecolor{currentstroke}%
\pgfsetdash{}{0pt}%
\pgfpathmoveto{\pgfqpoint{4.620990in}{2.602496in}}%
\pgfpathlineto{\pgfqpoint{4.634357in}{2.606843in}}%
\pgfpathlineto{\pgfqpoint{4.647736in}{2.611351in}}%
\pgfpathlineto{\pgfqpoint{4.661127in}{2.616020in}}%
\pgfpathlineto{\pgfqpoint{4.674531in}{2.620848in}}%
\pgfpathlineto{\pgfqpoint{4.681915in}{2.628604in}}%
\pgfpathlineto{\pgfqpoint{4.689294in}{2.636344in}}%
\pgfpathlineto{\pgfqpoint{4.696667in}{2.644073in}}%
\pgfpathlineto{\pgfqpoint{4.704035in}{2.651793in}}%
\pgfpathlineto{\pgfqpoint{4.690642in}{2.647226in}}%
\pgfpathlineto{\pgfqpoint{4.677262in}{2.642818in}}%
\pgfpathlineto{\pgfqpoint{4.663894in}{2.638571in}}%
\pgfpathlineto{\pgfqpoint{4.650538in}{2.634484in}}%
\pgfpathlineto{\pgfqpoint{4.643159in}{2.626493in}}%
\pgfpathlineto{\pgfqpoint{4.635775in}{2.618500in}}%
\pgfpathlineto{\pgfqpoint{4.628385in}{2.610502in}}%
\pgfpathlineto{\pgfqpoint{4.620990in}{2.602496in}}%
\pgfpathclose%
\pgfusepath{fill}%
\end{pgfscope}%
\begin{pgfscope}%
\pgfpathrectangle{\pgfqpoint{1.254980in}{0.150000in}}{\pgfqpoint{5.490039in}{5.490039in}}%
\pgfusepath{clip}%
\pgfsetbuttcap%
\pgfsetroundjoin%
\definecolor{currentfill}{rgb}{0.279566,0.067836,0.391917}%
\pgfsetfillcolor{currentfill}%
\pgfsetfillopacity{0.700000}%
\pgfsetlinewidth{0.000000pt}%
\definecolor{currentstroke}{rgb}{0.000000,0.000000,0.000000}%
\pgfsetstrokecolor{currentstroke}%
\pgfsetdash{}{0pt}%
\pgfpathmoveto{\pgfqpoint{3.489356in}{2.103387in}}%
\pgfpathlineto{\pgfqpoint{3.502393in}{2.098837in}}%
\pgfpathlineto{\pgfqpoint{3.515433in}{2.094478in}}%
\pgfpathlineto{\pgfqpoint{3.528477in}{2.090310in}}%
\pgfpathlineto{\pgfqpoint{3.541525in}{2.086330in}}%
\pgfpathlineto{\pgfqpoint{3.549304in}{2.095686in}}%
\pgfpathlineto{\pgfqpoint{3.557078in}{2.105065in}}%
\pgfpathlineto{\pgfqpoint{3.564846in}{2.114465in}}%
\pgfpathlineto{\pgfqpoint{3.572609in}{2.123887in}}%
\pgfpathlineto{\pgfqpoint{3.559573in}{2.127760in}}%
\pgfpathlineto{\pgfqpoint{3.546541in}{2.131823in}}%
\pgfpathlineto{\pgfqpoint{3.533512in}{2.136075in}}%
\pgfpathlineto{\pgfqpoint{3.520487in}{2.140519in}}%
\pgfpathlineto{\pgfqpoint{3.512713in}{2.131193in}}%
\pgfpathlineto{\pgfqpoint{3.504933in}{2.121895in}}%
\pgfpathlineto{\pgfqpoint{3.497147in}{2.112626in}}%
\pgfpathlineto{\pgfqpoint{3.489356in}{2.103387in}}%
\pgfpathclose%
\pgfusepath{fill}%
\end{pgfscope}%
\begin{pgfscope}%
\pgfpathrectangle{\pgfqpoint{1.254980in}{0.150000in}}{\pgfqpoint{5.490039in}{5.490039in}}%
\pgfusepath{clip}%
\pgfsetbuttcap%
\pgfsetroundjoin%
\definecolor{currentfill}{rgb}{0.283091,0.110553,0.431554}%
\pgfsetfillcolor{currentfill}%
\pgfsetfillopacity{0.700000}%
\pgfsetlinewidth{0.000000pt}%
\definecolor{currentstroke}{rgb}{0.000000,0.000000,0.000000}%
\pgfsetstrokecolor{currentstroke}%
\pgfsetdash{}{0pt}%
\pgfpathmoveto{\pgfqpoint{3.791069in}{2.169504in}}%
\pgfpathlineto{\pgfqpoint{3.804151in}{2.168147in}}%
\pgfpathlineto{\pgfqpoint{3.817239in}{2.166968in}}%
\pgfpathlineto{\pgfqpoint{3.830333in}{2.165967in}}%
\pgfpathlineto{\pgfqpoint{3.843435in}{2.165143in}}%
\pgfpathlineto{\pgfqpoint{3.851112in}{2.174872in}}%
\pgfpathlineto{\pgfqpoint{3.858784in}{2.184593in}}%
\pgfpathlineto{\pgfqpoint{3.866451in}{2.194305in}}%
\pgfpathlineto{\pgfqpoint{3.874113in}{2.204008in}}%
\pgfpathlineto{\pgfqpoint{3.861021in}{2.204810in}}%
\pgfpathlineto{\pgfqpoint{3.847935in}{2.205788in}}%
\pgfpathlineto{\pgfqpoint{3.834855in}{2.206945in}}%
\pgfpathlineto{\pgfqpoint{3.821782in}{2.208280in}}%
\pgfpathlineto{\pgfqpoint{3.814111in}{2.198588in}}%
\pgfpathlineto{\pgfqpoint{3.806436in}{2.188895in}}%
\pgfpathlineto{\pgfqpoint{3.798755in}{2.179201in}}%
\pgfpathlineto{\pgfqpoint{3.791069in}{2.169504in}}%
\pgfpathclose%
\pgfusepath{fill}%
\end{pgfscope}%
\begin{pgfscope}%
\pgfpathrectangle{\pgfqpoint{1.254980in}{0.150000in}}{\pgfqpoint{5.490039in}{5.490039in}}%
\pgfusepath{clip}%
\pgfsetbuttcap%
\pgfsetroundjoin%
\definecolor{currentfill}{rgb}{0.220057,0.343307,0.549413}%
\pgfsetfillcolor{currentfill}%
\pgfsetfillopacity{0.700000}%
\pgfsetlinewidth{0.000000pt}%
\definecolor{currentstroke}{rgb}{0.000000,0.000000,0.000000}%
\pgfsetstrokecolor{currentstroke}%
\pgfsetdash{}{0pt}%
\pgfpathmoveto{\pgfqpoint{4.704035in}{2.651793in}}%
\pgfpathlineto{\pgfqpoint{4.717440in}{2.656521in}}%
\pgfpathlineto{\pgfqpoint{4.730857in}{2.661408in}}%
\pgfpathlineto{\pgfqpoint{4.744286in}{2.666454in}}%
\pgfpathlineto{\pgfqpoint{4.757729in}{2.671660in}}%
\pgfpathlineto{\pgfqpoint{4.765080in}{2.679096in}}%
\pgfpathlineto{\pgfqpoint{4.772425in}{2.686523in}}%
\pgfpathlineto{\pgfqpoint{4.779765in}{2.693945in}}%
\pgfpathlineto{\pgfqpoint{4.787099in}{2.701364in}}%
\pgfpathlineto{\pgfqpoint{4.773669in}{2.696449in}}%
\pgfpathlineto{\pgfqpoint{4.760251in}{2.691692in}}%
\pgfpathlineto{\pgfqpoint{4.746846in}{2.687094in}}%
\pgfpathlineto{\pgfqpoint{4.733453in}{2.682656in}}%
\pgfpathlineto{\pgfqpoint{4.726107in}{2.674936in}}%
\pgfpathlineto{\pgfqpoint{4.718755in}{2.667222in}}%
\pgfpathlineto{\pgfqpoint{4.711397in}{2.659509in}}%
\pgfpathlineto{\pgfqpoint{4.704035in}{2.651793in}}%
\pgfpathclose%
\pgfusepath{fill}%
\end{pgfscope}%
\begin{pgfscope}%
\pgfpathrectangle{\pgfqpoint{1.254980in}{0.150000in}}{\pgfqpoint{5.490039in}{5.490039in}}%
\pgfusepath{clip}%
\pgfsetbuttcap%
\pgfsetroundjoin%
\definecolor{currentfill}{rgb}{0.208623,0.367752,0.552675}%
\pgfsetfillcolor{currentfill}%
\pgfsetfillopacity{0.700000}%
\pgfsetlinewidth{0.000000pt}%
\definecolor{currentstroke}{rgb}{0.000000,0.000000,0.000000}%
\pgfsetstrokecolor{currentstroke}%
\pgfsetdash{}{0pt}%
\pgfpathmoveto{\pgfqpoint{4.787099in}{2.701364in}}%
\pgfpathlineto{\pgfqpoint{4.800542in}{2.706439in}}%
\pgfpathlineto{\pgfqpoint{4.813998in}{2.711671in}}%
\pgfpathlineto{\pgfqpoint{4.827467in}{2.717063in}}%
\pgfpathlineto{\pgfqpoint{4.840949in}{2.722612in}}%
\pgfpathlineto{\pgfqpoint{4.848266in}{2.729725in}}%
\pgfpathlineto{\pgfqpoint{4.855577in}{2.736837in}}%
\pgfpathlineto{\pgfqpoint{4.862882in}{2.743950in}}%
\pgfpathlineto{\pgfqpoint{4.870182in}{2.751070in}}%
\pgfpathlineto{\pgfqpoint{4.856713in}{2.745839in}}%
\pgfpathlineto{\pgfqpoint{4.843257in}{2.740766in}}%
\pgfpathlineto{\pgfqpoint{4.829815in}{2.735852in}}%
\pgfpathlineto{\pgfqpoint{4.816385in}{2.731095in}}%
\pgfpathlineto{\pgfqpoint{4.809071in}{2.723647in}}%
\pgfpathlineto{\pgfqpoint{4.801752in}{2.716212in}}%
\pgfpathlineto{\pgfqpoint{4.794428in}{2.708786in}}%
\pgfpathlineto{\pgfqpoint{4.787099in}{2.701364in}}%
\pgfpathclose%
\pgfusepath{fill}%
\end{pgfscope}%
\begin{pgfscope}%
\pgfpathrectangle{\pgfqpoint{1.254980in}{0.150000in}}{\pgfqpoint{5.490039in}{5.490039in}}%
\pgfusepath{clip}%
\pgfsetbuttcap%
\pgfsetroundjoin%
\definecolor{currentfill}{rgb}{0.282910,0.105393,0.426902}%
\pgfsetfillcolor{currentfill}%
\pgfsetfillopacity{0.700000}%
\pgfsetlinewidth{0.000000pt}%
\definecolor{currentstroke}{rgb}{0.000000,0.000000,0.000000}%
\pgfsetstrokecolor{currentstroke}%
\pgfsetdash{}{0pt}%
\pgfpathmoveto{\pgfqpoint{3.081951in}{2.176144in}}%
\pgfpathlineto{\pgfqpoint{3.095013in}{2.166189in}}%
\pgfpathlineto{\pgfqpoint{3.108074in}{2.156454in}}%
\pgfpathlineto{\pgfqpoint{3.121133in}{2.146937in}}%
\pgfpathlineto{\pgfqpoint{3.134191in}{2.137637in}}%
\pgfpathlineto{\pgfqpoint{3.142136in}{2.145435in}}%
\pgfpathlineto{\pgfqpoint{3.150074in}{2.153312in}}%
\pgfpathlineto{\pgfqpoint{3.158005in}{2.161266in}}%
\pgfpathlineto{\pgfqpoint{3.165928in}{2.169295in}}%
\pgfpathlineto{\pgfqpoint{3.152888in}{2.178404in}}%
\pgfpathlineto{\pgfqpoint{3.139847in}{2.187728in}}%
\pgfpathlineto{\pgfqpoint{3.126806in}{2.197271in}}%
\pgfpathlineto{\pgfqpoint{3.113763in}{2.207034in}}%
\pgfpathlineto{\pgfqpoint{3.105821in}{2.199186in}}%
\pgfpathlineto{\pgfqpoint{3.097872in}{2.191421in}}%
\pgfpathlineto{\pgfqpoint{3.089915in}{2.183740in}}%
\pgfpathlineto{\pgfqpoint{3.081951in}{2.176144in}}%
\pgfpathclose%
\pgfusepath{fill}%
\end{pgfscope}%
\begin{pgfscope}%
\pgfpathrectangle{\pgfqpoint{1.254980in}{0.150000in}}{\pgfqpoint{5.490039in}{5.490039in}}%
\pgfusepath{clip}%
\pgfsetbuttcap%
\pgfsetroundjoin%
\definecolor{currentfill}{rgb}{0.199430,0.387607,0.554642}%
\pgfsetfillcolor{currentfill}%
\pgfsetfillopacity{0.700000}%
\pgfsetlinewidth{0.000000pt}%
\definecolor{currentstroke}{rgb}{0.000000,0.000000,0.000000}%
\pgfsetstrokecolor{currentstroke}%
\pgfsetdash{}{0pt}%
\pgfpathmoveto{\pgfqpoint{4.870182in}{2.751070in}}%
\pgfpathlineto{\pgfqpoint{4.883664in}{2.756458in}}%
\pgfpathlineto{\pgfqpoint{4.897160in}{2.762003in}}%
\pgfpathlineto{\pgfqpoint{4.910668in}{2.767707in}}%
\pgfpathlineto{\pgfqpoint{4.924191in}{2.773567in}}%
\pgfpathlineto{\pgfqpoint{4.931471in}{2.780359in}}%
\pgfpathlineto{\pgfqpoint{4.938747in}{2.787157in}}%
\pgfpathlineto{\pgfqpoint{4.946017in}{2.793966in}}%
\pgfpathlineto{\pgfqpoint{4.953282in}{2.800790in}}%
\pgfpathlineto{\pgfqpoint{4.939774in}{2.795277in}}%
\pgfpathlineto{\pgfqpoint{4.926280in}{2.789922in}}%
\pgfpathlineto{\pgfqpoint{4.912799in}{2.784723in}}%
\pgfpathlineto{\pgfqpoint{4.899331in}{2.779682in}}%
\pgfpathlineto{\pgfqpoint{4.892051in}{2.772501in}}%
\pgfpathlineto{\pgfqpoint{4.884767in}{2.765341in}}%
\pgfpathlineto{\pgfqpoint{4.877477in}{2.758198in}}%
\pgfpathlineto{\pgfqpoint{4.870182in}{2.751070in}}%
\pgfpathclose%
\pgfusepath{fill}%
\end{pgfscope}%
\begin{pgfscope}%
\pgfpathrectangle{\pgfqpoint{1.254980in}{0.150000in}}{\pgfqpoint{5.490039in}{5.490039in}}%
\pgfusepath{clip}%
\pgfsetbuttcap%
\pgfsetroundjoin%
\definecolor{currentfill}{rgb}{0.282327,0.094955,0.417331}%
\pgfsetfillcolor{currentfill}%
\pgfsetfillopacity{0.700000}%
\pgfsetlinewidth{0.000000pt}%
\definecolor{currentstroke}{rgb}{0.000000,0.000000,0.000000}%
\pgfsetstrokecolor{currentstroke}%
\pgfsetdash{}{0pt}%
\pgfpathmoveto{\pgfqpoint{3.707969in}{2.138130in}}%
\pgfpathlineto{\pgfqpoint{3.721037in}{2.136001in}}%
\pgfpathlineto{\pgfqpoint{3.734111in}{2.134054in}}%
\pgfpathlineto{\pgfqpoint{3.747190in}{2.132287in}}%
\pgfpathlineto{\pgfqpoint{3.760275in}{2.130700in}}%
\pgfpathlineto{\pgfqpoint{3.767981in}{2.140404in}}%
\pgfpathlineto{\pgfqpoint{3.775682in}{2.150106in}}%
\pgfpathlineto{\pgfqpoint{3.783378in}{2.159806in}}%
\pgfpathlineto{\pgfqpoint{3.791069in}{2.169504in}}%
\pgfpathlineto{\pgfqpoint{3.777993in}{2.171041in}}%
\pgfpathlineto{\pgfqpoint{3.764923in}{2.172758in}}%
\pgfpathlineto{\pgfqpoint{3.751859in}{2.174655in}}%
\pgfpathlineto{\pgfqpoint{3.738801in}{2.176733in}}%
\pgfpathlineto{\pgfqpoint{3.731101in}{2.167075in}}%
\pgfpathlineto{\pgfqpoint{3.723395in}{2.157422in}}%
\pgfpathlineto{\pgfqpoint{3.715685in}{2.147773in}}%
\pgfpathlineto{\pgfqpoint{3.707969in}{2.138130in}}%
\pgfpathclose%
\pgfusepath{fill}%
\end{pgfscope}%
\begin{pgfscope}%
\pgfpathrectangle{\pgfqpoint{1.254980in}{0.150000in}}{\pgfqpoint{5.490039in}{5.490039in}}%
\pgfusepath{clip}%
\pgfsetbuttcap%
\pgfsetroundjoin%
\definecolor{currentfill}{rgb}{0.190631,0.407061,0.556089}%
\pgfsetfillcolor{currentfill}%
\pgfsetfillopacity{0.700000}%
\pgfsetlinewidth{0.000000pt}%
\definecolor{currentstroke}{rgb}{0.000000,0.000000,0.000000}%
\pgfsetstrokecolor{currentstroke}%
\pgfsetdash{}{0pt}%
\pgfpathmoveto{\pgfqpoint{4.953282in}{2.800790in}}%
\pgfpathlineto{\pgfqpoint{4.966803in}{2.806459in}}%
\pgfpathlineto{\pgfqpoint{4.980339in}{2.812284in}}%
\pgfpathlineto{\pgfqpoint{4.993887in}{2.818267in}}%
\pgfpathlineto{\pgfqpoint{5.007450in}{2.824406in}}%
\pgfpathlineto{\pgfqpoint{5.014695in}{2.830882in}}%
\pgfpathlineto{\pgfqpoint{5.021934in}{2.837374in}}%
\pgfpathlineto{\pgfqpoint{5.029168in}{2.843887in}}%
\pgfpathlineto{\pgfqpoint{5.036397in}{2.850425in}}%
\pgfpathlineto{\pgfqpoint{5.022850in}{2.844663in}}%
\pgfpathlineto{\pgfqpoint{5.009317in}{2.839058in}}%
\pgfpathlineto{\pgfqpoint{4.995798in}{2.833608in}}%
\pgfpathlineto{\pgfqpoint{4.982292in}{2.828315in}}%
\pgfpathlineto{\pgfqpoint{4.975047in}{2.821390in}}%
\pgfpathlineto{\pgfqpoint{4.967797in}{2.814498in}}%
\pgfpathlineto{\pgfqpoint{4.960542in}{2.807632in}}%
\pgfpathlineto{\pgfqpoint{4.953282in}{2.800790in}}%
\pgfpathclose%
\pgfusepath{fill}%
\end{pgfscope}%
\begin{pgfscope}%
\pgfpathrectangle{\pgfqpoint{1.254980in}{0.150000in}}{\pgfqpoint{5.490039in}{5.490039in}}%
\pgfusepath{clip}%
\pgfsetbuttcap%
\pgfsetroundjoin%
\definecolor{currentfill}{rgb}{0.262138,0.242286,0.520837}%
\pgfsetfillcolor{currentfill}%
\pgfsetfillopacity{0.700000}%
\pgfsetlinewidth{0.000000pt}%
\definecolor{currentstroke}{rgb}{0.000000,0.000000,0.000000}%
\pgfsetstrokecolor{currentstroke}%
\pgfsetdash{}{0pt}%
\pgfpathmoveto{\pgfqpoint{2.734951in}{2.461187in}}%
\pgfpathlineto{\pgfqpoint{2.748141in}{2.445214in}}%
\pgfpathlineto{\pgfqpoint{2.761324in}{2.429505in}}%
\pgfpathlineto{\pgfqpoint{2.774500in}{2.414058in}}%
\pgfpathlineto{\pgfqpoint{2.787669in}{2.398872in}}%
\pgfpathlineto{\pgfqpoint{2.795783in}{2.404936in}}%
\pgfpathlineto{\pgfqpoint{2.803887in}{2.411129in}}%
\pgfpathlineto{\pgfqpoint{2.811981in}{2.417448in}}%
\pgfpathlineto{\pgfqpoint{2.820065in}{2.423891in}}%
\pgfpathlineto{\pgfqpoint{2.806922in}{2.438851in}}%
\pgfpathlineto{\pgfqpoint{2.793772in}{2.454069in}}%
\pgfpathlineto{\pgfqpoint{2.780615in}{2.469550in}}%
\pgfpathlineto{\pgfqpoint{2.767452in}{2.485295in}}%
\pgfpathlineto{\pgfqpoint{2.759341in}{2.479069in}}%
\pgfpathlineto{\pgfqpoint{2.751221in}{2.472974in}}%
\pgfpathlineto{\pgfqpoint{2.743091in}{2.467013in}}%
\pgfpathlineto{\pgfqpoint{2.734951in}{2.461187in}}%
\pgfpathclose%
\pgfusepath{fill}%
\end{pgfscope}%
\begin{pgfscope}%
\pgfpathrectangle{\pgfqpoint{1.254980in}{0.150000in}}{\pgfqpoint{5.490039in}{5.490039in}}%
\pgfusepath{clip}%
\pgfsetbuttcap%
\pgfsetroundjoin%
\definecolor{currentfill}{rgb}{0.270595,0.214069,0.507052}%
\pgfsetfillcolor{currentfill}%
\pgfsetfillopacity{0.700000}%
\pgfsetlinewidth{0.000000pt}%
\definecolor{currentstroke}{rgb}{0.000000,0.000000,0.000000}%
\pgfsetstrokecolor{currentstroke}%
\pgfsetdash{}{0pt}%
\pgfpathmoveto{\pgfqpoint{2.787669in}{2.398872in}}%
\pgfpathlineto{\pgfqpoint{2.800832in}{2.383943in}}%
\pgfpathlineto{\pgfqpoint{2.813989in}{2.369270in}}%
\pgfpathlineto{\pgfqpoint{2.827140in}{2.354851in}}%
\pgfpathlineto{\pgfqpoint{2.840285in}{2.340683in}}%
\pgfpathlineto{\pgfqpoint{2.848373in}{2.346985in}}%
\pgfpathlineto{\pgfqpoint{2.856451in}{2.353407in}}%
\pgfpathlineto{\pgfqpoint{2.864520in}{2.359949in}}%
\pgfpathlineto{\pgfqpoint{2.872580in}{2.366608in}}%
\pgfpathlineto{\pgfqpoint{2.859460in}{2.380550in}}%
\pgfpathlineto{\pgfqpoint{2.846334in}{2.394743in}}%
\pgfpathlineto{\pgfqpoint{2.833203in}{2.409190in}}%
\pgfpathlineto{\pgfqpoint{2.820065in}{2.423891in}}%
\pgfpathlineto{\pgfqpoint{2.811981in}{2.417448in}}%
\pgfpathlineto{\pgfqpoint{2.803887in}{2.411129in}}%
\pgfpathlineto{\pgfqpoint{2.795783in}{2.404936in}}%
\pgfpathlineto{\pgfqpoint{2.787669in}{2.398872in}}%
\pgfpathclose%
\pgfusepath{fill}%
\end{pgfscope}%
\begin{pgfscope}%
\pgfpathrectangle{\pgfqpoint{1.254980in}{0.150000in}}{\pgfqpoint{5.490039in}{5.490039in}}%
\pgfusepath{clip}%
\pgfsetbuttcap%
\pgfsetroundjoin%
\definecolor{currentfill}{rgb}{0.182256,0.426184,0.557120}%
\pgfsetfillcolor{currentfill}%
\pgfsetfillopacity{0.700000}%
\pgfsetlinewidth{0.000000pt}%
\definecolor{currentstroke}{rgb}{0.000000,0.000000,0.000000}%
\pgfsetstrokecolor{currentstroke}%
\pgfsetdash{}{0pt}%
\pgfpathmoveto{\pgfqpoint{5.036397in}{2.850425in}}%
\pgfpathlineto{\pgfqpoint{5.049958in}{2.856342in}}%
\pgfpathlineto{\pgfqpoint{5.063533in}{2.862415in}}%
\pgfpathlineto{\pgfqpoint{5.077122in}{2.868644in}}%
\pgfpathlineto{\pgfqpoint{5.090726in}{2.875028in}}%
\pgfpathlineto{\pgfqpoint{5.097933in}{2.881200in}}%
\pgfpathlineto{\pgfqpoint{5.105136in}{2.887399in}}%
\pgfpathlineto{\pgfqpoint{5.112333in}{2.893629in}}%
\pgfpathlineto{\pgfqpoint{5.119526in}{2.899896in}}%
\pgfpathlineto{\pgfqpoint{5.105939in}{2.893918in}}%
\pgfpathlineto{\pgfqpoint{5.092367in}{2.888095in}}%
\pgfpathlineto{\pgfqpoint{5.078809in}{2.882427in}}%
\pgfpathlineto{\pgfqpoint{5.065265in}{2.876914in}}%
\pgfpathlineto{\pgfqpoint{5.058055in}{2.870232in}}%
\pgfpathlineto{\pgfqpoint{5.050841in}{2.863593in}}%
\pgfpathlineto{\pgfqpoint{5.043621in}{2.856992in}}%
\pgfpathlineto{\pgfqpoint{5.036397in}{2.850425in}}%
\pgfpathclose%
\pgfusepath{fill}%
\end{pgfscope}%
\begin{pgfscope}%
\pgfpathrectangle{\pgfqpoint{1.254980in}{0.150000in}}{\pgfqpoint{5.490039in}{5.490039in}}%
\pgfusepath{clip}%
\pgfsetbuttcap%
\pgfsetroundjoin%
\definecolor{currentfill}{rgb}{0.280267,0.073417,0.397163}%
\pgfsetfillcolor{currentfill}%
\pgfsetfillopacity{0.700000}%
\pgfsetlinewidth{0.000000pt}%
\definecolor{currentstroke}{rgb}{0.000000,0.000000,0.000000}%
\pgfsetstrokecolor{currentstroke}%
\pgfsetdash{}{0pt}%
\pgfpathmoveto{\pgfqpoint{3.270235in}{2.104065in}}%
\pgfpathlineto{\pgfqpoint{3.283275in}{2.096847in}}%
\pgfpathlineto{\pgfqpoint{3.296316in}{2.089832in}}%
\pgfpathlineto{\pgfqpoint{3.309358in}{2.083020in}}%
\pgfpathlineto{\pgfqpoint{3.322402in}{2.076409in}}%
\pgfpathlineto{\pgfqpoint{3.330269in}{2.085033in}}%
\pgfpathlineto{\pgfqpoint{3.338129in}{2.093708in}}%
\pgfpathlineto{\pgfqpoint{3.345983in}{2.102435in}}%
\pgfpathlineto{\pgfqpoint{3.353831in}{2.111210in}}%
\pgfpathlineto{\pgfqpoint{3.340803in}{2.117658in}}%
\pgfpathlineto{\pgfqpoint{3.327776in}{2.124308in}}%
\pgfpathlineto{\pgfqpoint{3.314750in}{2.131160in}}%
\pgfpathlineto{\pgfqpoint{3.301726in}{2.138216in}}%
\pgfpathlineto{\pgfqpoint{3.293863in}{2.129593in}}%
\pgfpathlineto{\pgfqpoint{3.285994in}{2.121026in}}%
\pgfpathlineto{\pgfqpoint{3.278118in}{2.112516in}}%
\pgfpathlineto{\pgfqpoint{3.270235in}{2.104065in}}%
\pgfpathclose%
\pgfusepath{fill}%
\end{pgfscope}%
\begin{pgfscope}%
\pgfpathrectangle{\pgfqpoint{1.254980in}{0.150000in}}{\pgfqpoint{5.490039in}{5.490039in}}%
\pgfusepath{clip}%
\pgfsetbuttcap%
\pgfsetroundjoin%
\definecolor{currentfill}{rgb}{0.250425,0.274290,0.533103}%
\pgfsetfillcolor{currentfill}%
\pgfsetfillopacity{0.700000}%
\pgfsetlinewidth{0.000000pt}%
\definecolor{currentstroke}{rgb}{0.000000,0.000000,0.000000}%
\pgfsetstrokecolor{currentstroke}%
\pgfsetdash{}{0pt}%
\pgfpathmoveto{\pgfqpoint{2.682112in}{2.527775in}}%
\pgfpathlineto{\pgfqpoint{2.695334in}{2.510719in}}%
\pgfpathlineto{\pgfqpoint{2.708548in}{2.493938in}}%
\pgfpathlineto{\pgfqpoint{2.721753in}{2.477428in}}%
\pgfpathlineto{\pgfqpoint{2.734951in}{2.461187in}}%
\pgfpathlineto{\pgfqpoint{2.743091in}{2.467013in}}%
\pgfpathlineto{\pgfqpoint{2.751221in}{2.472974in}}%
\pgfpathlineto{\pgfqpoint{2.759341in}{2.479069in}}%
\pgfpathlineto{\pgfqpoint{2.767452in}{2.485295in}}%
\pgfpathlineto{\pgfqpoint{2.754281in}{2.501307in}}%
\pgfpathlineto{\pgfqpoint{2.741103in}{2.517588in}}%
\pgfpathlineto{\pgfqpoint{2.727917in}{2.534139in}}%
\pgfpathlineto{\pgfqpoint{2.714723in}{2.550965in}}%
\pgfpathlineto{\pgfqpoint{2.706586in}{2.544957in}}%
\pgfpathlineto{\pgfqpoint{2.698439in}{2.539088in}}%
\pgfpathlineto{\pgfqpoint{2.690281in}{2.533360in}}%
\pgfpathlineto{\pgfqpoint{2.682112in}{2.527775in}}%
\pgfpathclose%
\pgfusepath{fill}%
\end{pgfscope}%
\begin{pgfscope}%
\pgfpathrectangle{\pgfqpoint{1.254980in}{0.150000in}}{\pgfqpoint{5.490039in}{5.490039in}}%
\pgfusepath{clip}%
\pgfsetbuttcap%
\pgfsetroundjoin%
\definecolor{currentfill}{rgb}{0.278791,0.062145,0.386592}%
\pgfsetfillcolor{currentfill}%
\pgfsetfillopacity{0.700000}%
\pgfsetlinewidth{0.000000pt}%
\definecolor{currentstroke}{rgb}{0.000000,0.000000,0.000000}%
\pgfsetstrokecolor{currentstroke}%
\pgfsetdash{}{0pt}%
\pgfpathmoveto{\pgfqpoint{3.405963in}{2.087408in}}%
\pgfpathlineto{\pgfqpoint{3.419002in}{2.081950in}}%
\pgfpathlineto{\pgfqpoint{3.432043in}{2.076687in}}%
\pgfpathlineto{\pgfqpoint{3.445086in}{2.071618in}}%
\pgfpathlineto{\pgfqpoint{3.458133in}{2.066742in}}%
\pgfpathlineto{\pgfqpoint{3.465947in}{2.075854in}}%
\pgfpathlineto{\pgfqpoint{3.473756in}{2.085000in}}%
\pgfpathlineto{\pgfqpoint{3.481559in}{2.094178in}}%
\pgfpathlineto{\pgfqpoint{3.489356in}{2.103387in}}%
\pgfpathlineto{\pgfqpoint{3.476322in}{2.108129in}}%
\pgfpathlineto{\pgfqpoint{3.463291in}{2.113064in}}%
\pgfpathlineto{\pgfqpoint{3.450263in}{2.118193in}}%
\pgfpathlineto{\pgfqpoint{3.437238in}{2.123516in}}%
\pgfpathlineto{\pgfqpoint{3.429428in}{2.114431in}}%
\pgfpathlineto{\pgfqpoint{3.421612in}{2.105384in}}%
\pgfpathlineto{\pgfqpoint{3.413791in}{2.096376in}}%
\pgfpathlineto{\pgfqpoint{3.405963in}{2.087408in}}%
\pgfpathclose%
\pgfusepath{fill}%
\end{pgfscope}%
\begin{pgfscope}%
\pgfpathrectangle{\pgfqpoint{1.254980in}{0.150000in}}{\pgfqpoint{5.490039in}{5.490039in}}%
\pgfusepath{clip}%
\pgfsetbuttcap%
\pgfsetroundjoin%
\definecolor{currentfill}{rgb}{0.277134,0.185228,0.489898}%
\pgfsetfillcolor{currentfill}%
\pgfsetfillopacity{0.700000}%
\pgfsetlinewidth{0.000000pt}%
\definecolor{currentstroke}{rgb}{0.000000,0.000000,0.000000}%
\pgfsetstrokecolor{currentstroke}%
\pgfsetdash{}{0pt}%
\pgfpathmoveto{\pgfqpoint{2.840285in}{2.340683in}}%
\pgfpathlineto{\pgfqpoint{2.853424in}{2.326765in}}%
\pgfpathlineto{\pgfqpoint{2.866559in}{2.313095in}}%
\pgfpathlineto{\pgfqpoint{2.879688in}{2.299669in}}%
\pgfpathlineto{\pgfqpoint{2.892813in}{2.286488in}}%
\pgfpathlineto{\pgfqpoint{2.900876in}{2.293025in}}%
\pgfpathlineto{\pgfqpoint{2.908930in}{2.299676in}}%
\pgfpathlineto{\pgfqpoint{2.916975in}{2.306439in}}%
\pgfpathlineto{\pgfqpoint{2.925012in}{2.313312in}}%
\pgfpathlineto{\pgfqpoint{2.911911in}{2.326269in}}%
\pgfpathlineto{\pgfqpoint{2.898805in}{2.339469in}}%
\pgfpathlineto{\pgfqpoint{2.885695in}{2.352915in}}%
\pgfpathlineto{\pgfqpoint{2.872580in}{2.366608in}}%
\pgfpathlineto{\pgfqpoint{2.864520in}{2.359949in}}%
\pgfpathlineto{\pgfqpoint{2.856451in}{2.353407in}}%
\pgfpathlineto{\pgfqpoint{2.848373in}{2.346985in}}%
\pgfpathlineto{\pgfqpoint{2.840285in}{2.340683in}}%
\pgfpathclose%
\pgfusepath{fill}%
\end{pgfscope}%
\begin{pgfscope}%
\pgfpathrectangle{\pgfqpoint{1.254980in}{0.150000in}}{\pgfqpoint{5.490039in}{5.490039in}}%
\pgfusepath{clip}%
\pgfsetbuttcap%
\pgfsetroundjoin%
\definecolor{currentfill}{rgb}{0.172719,0.448791,0.557885}%
\pgfsetfillcolor{currentfill}%
\pgfsetfillopacity{0.700000}%
\pgfsetlinewidth{0.000000pt}%
\definecolor{currentstroke}{rgb}{0.000000,0.000000,0.000000}%
\pgfsetstrokecolor{currentstroke}%
\pgfsetdash{}{0pt}%
\pgfpathmoveto{\pgfqpoint{5.119526in}{2.899896in}}%
\pgfpathlineto{\pgfqpoint{5.133126in}{2.906029in}}%
\pgfpathlineto{\pgfqpoint{5.146741in}{2.912316in}}%
\pgfpathlineto{\pgfqpoint{5.160370in}{2.918759in}}%
\pgfpathlineto{\pgfqpoint{5.174015in}{2.925357in}}%
\pgfpathlineto{\pgfqpoint{5.181184in}{2.931240in}}%
\pgfpathlineto{\pgfqpoint{5.188349in}{2.937162in}}%
\pgfpathlineto{\pgfqpoint{5.195509in}{2.943128in}}%
\pgfpathlineto{\pgfqpoint{5.202665in}{2.949143in}}%
\pgfpathlineto{\pgfqpoint{5.189040in}{2.942981in}}%
\pgfpathlineto{\pgfqpoint{5.175429in}{2.936973in}}%
\pgfpathlineto{\pgfqpoint{5.161832in}{2.931119in}}%
\pgfpathlineto{\pgfqpoint{5.148250in}{2.925420in}}%
\pgfpathlineto{\pgfqpoint{5.141076in}{2.918960in}}%
\pgfpathlineto{\pgfqpoint{5.133897in}{2.912556in}}%
\pgfpathlineto{\pgfqpoint{5.126713in}{2.906203in}}%
\pgfpathlineto{\pgfqpoint{5.119526in}{2.899896in}}%
\pgfpathclose%
\pgfusepath{fill}%
\end{pgfscope}%
\begin{pgfscope}%
\pgfpathrectangle{\pgfqpoint{1.254980in}{0.150000in}}{\pgfqpoint{5.490039in}{5.490039in}}%
\pgfusepath{clip}%
\pgfsetbuttcap%
\pgfsetroundjoin%
\definecolor{currentfill}{rgb}{0.280894,0.078907,0.402329}%
\pgfsetfillcolor{currentfill}%
\pgfsetfillopacity{0.700000}%
\pgfsetlinewidth{0.000000pt}%
\definecolor{currentstroke}{rgb}{0.000000,0.000000,0.000000}%
\pgfsetstrokecolor{currentstroke}%
\pgfsetdash{}{0pt}%
\pgfpathmoveto{\pgfqpoint{3.624795in}{2.110270in}}%
\pgfpathlineto{\pgfqpoint{3.637853in}{2.107330in}}%
\pgfpathlineto{\pgfqpoint{3.650915in}{2.104575in}}%
\pgfpathlineto{\pgfqpoint{3.663982in}{2.102003in}}%
\pgfpathlineto{\pgfqpoint{3.677055in}{2.099614in}}%
\pgfpathlineto{\pgfqpoint{3.684791in}{2.109234in}}%
\pgfpathlineto{\pgfqpoint{3.692522in}{2.118860in}}%
\pgfpathlineto{\pgfqpoint{3.700248in}{2.128492in}}%
\pgfpathlineto{\pgfqpoint{3.707969in}{2.138130in}}%
\pgfpathlineto{\pgfqpoint{3.694907in}{2.140441in}}%
\pgfpathlineto{\pgfqpoint{3.681849in}{2.142935in}}%
\pgfpathlineto{\pgfqpoint{3.668797in}{2.145612in}}%
\pgfpathlineto{\pgfqpoint{3.655750in}{2.148473in}}%
\pgfpathlineto{\pgfqpoint{3.648019in}{2.138903in}}%
\pgfpathlineto{\pgfqpoint{3.640283in}{2.129346in}}%
\pgfpathlineto{\pgfqpoint{3.632542in}{2.119801in}}%
\pgfpathlineto{\pgfqpoint{3.624795in}{2.110270in}}%
\pgfpathclose%
\pgfusepath{fill}%
\end{pgfscope}%
\begin{pgfscope}%
\pgfpathrectangle{\pgfqpoint{1.254980in}{0.150000in}}{\pgfqpoint{5.490039in}{5.490039in}}%
\pgfusepath{clip}%
\pgfsetbuttcap%
\pgfsetroundjoin%
\definecolor{currentfill}{rgb}{0.237441,0.305202,0.541921}%
\pgfsetfillcolor{currentfill}%
\pgfsetfillopacity{0.700000}%
\pgfsetlinewidth{0.000000pt}%
\definecolor{currentstroke}{rgb}{0.000000,0.000000,0.000000}%
\pgfsetstrokecolor{currentstroke}%
\pgfsetdash{}{0pt}%
\pgfpathmoveto{\pgfqpoint{2.629137in}{2.598790in}}%
\pgfpathlineto{\pgfqpoint{2.642395in}{2.580612in}}%
\pgfpathlineto{\pgfqpoint{2.655643in}{2.562719in}}%
\pgfpathlineto{\pgfqpoint{2.668882in}{2.545107in}}%
\pgfpathlineto{\pgfqpoint{2.682112in}{2.527775in}}%
\pgfpathlineto{\pgfqpoint{2.690281in}{2.533360in}}%
\pgfpathlineto{\pgfqpoint{2.698439in}{2.539088in}}%
\pgfpathlineto{\pgfqpoint{2.706586in}{2.544957in}}%
\pgfpathlineto{\pgfqpoint{2.714723in}{2.550965in}}%
\pgfpathlineto{\pgfqpoint{2.701521in}{2.568066in}}%
\pgfpathlineto{\pgfqpoint{2.688310in}{2.585446in}}%
\pgfpathlineto{\pgfqpoint{2.675091in}{2.603108in}}%
\pgfpathlineto{\pgfqpoint{2.661862in}{2.621053in}}%
\pgfpathlineto{\pgfqpoint{2.653697in}{2.615266in}}%
\pgfpathlineto{\pgfqpoint{2.645521in}{2.609625in}}%
\pgfpathlineto{\pgfqpoint{2.637335in}{2.604132in}}%
\pgfpathlineto{\pgfqpoint{2.629137in}{2.598790in}}%
\pgfpathclose%
\pgfusepath{fill}%
\end{pgfscope}%
\begin{pgfscope}%
\pgfpathrectangle{\pgfqpoint{1.254980in}{0.150000in}}{\pgfqpoint{5.490039in}{5.490039in}}%
\pgfusepath{clip}%
\pgfsetbuttcap%
\pgfsetroundjoin%
\definecolor{currentfill}{rgb}{0.165117,0.467423,0.558141}%
\pgfsetfillcolor{currentfill}%
\pgfsetfillopacity{0.700000}%
\pgfsetlinewidth{0.000000pt}%
\definecolor{currentstroke}{rgb}{0.000000,0.000000,0.000000}%
\pgfsetstrokecolor{currentstroke}%
\pgfsetdash{}{0pt}%
\pgfpathmoveto{\pgfqpoint{5.202665in}{2.949143in}}%
\pgfpathlineto{\pgfqpoint{5.216305in}{2.955459in}}%
\pgfpathlineto{\pgfqpoint{5.229959in}{2.961929in}}%
\pgfpathlineto{\pgfqpoint{5.243629in}{2.968553in}}%
\pgfpathlineto{\pgfqpoint{5.257314in}{2.975332in}}%
\pgfpathlineto{\pgfqpoint{5.264445in}{2.980947in}}%
\pgfpathlineto{\pgfqpoint{5.271572in}{2.986614in}}%
\pgfpathlineto{\pgfqpoint{5.278694in}{2.992339in}}%
\pgfpathlineto{\pgfqpoint{5.285813in}{2.998127in}}%
\pgfpathlineto{\pgfqpoint{5.272149in}{2.991813in}}%
\pgfpathlineto{\pgfqpoint{5.258500in}{2.985652in}}%
\pgfpathlineto{\pgfqpoint{5.244865in}{2.979645in}}%
\pgfpathlineto{\pgfqpoint{5.231245in}{2.973792in}}%
\pgfpathlineto{\pgfqpoint{5.224106in}{2.967531in}}%
\pgfpathlineto{\pgfqpoint{5.216963in}{2.961339in}}%
\pgfpathlineto{\pgfqpoint{5.209816in}{2.955211in}}%
\pgfpathlineto{\pgfqpoint{5.202665in}{2.949143in}}%
\pgfpathclose%
\pgfusepath{fill}%
\end{pgfscope}%
\begin{pgfscope}%
\pgfpathrectangle{\pgfqpoint{1.254980in}{0.150000in}}{\pgfqpoint{5.490039in}{5.490039in}}%
\pgfusepath{clip}%
\pgfsetbuttcap%
\pgfsetroundjoin%
\definecolor{currentfill}{rgb}{0.281924,0.089666,0.412415}%
\pgfsetfillcolor{currentfill}%
\pgfsetfillopacity{0.700000}%
\pgfsetlinewidth{0.000000pt}%
\definecolor{currentstroke}{rgb}{0.000000,0.000000,0.000000}%
\pgfsetstrokecolor{currentstroke}%
\pgfsetdash{}{0pt}%
\pgfpathmoveto{\pgfqpoint{3.134191in}{2.137637in}}%
\pgfpathlineto{\pgfqpoint{3.147249in}{2.128552in}}%
\pgfpathlineto{\pgfqpoint{3.160305in}{2.119681in}}%
\pgfpathlineto{\pgfqpoint{3.173361in}{2.111023in}}%
\pgfpathlineto{\pgfqpoint{3.186416in}{2.102576in}}%
\pgfpathlineto{\pgfqpoint{3.194343in}{2.110577in}}%
\pgfpathlineto{\pgfqpoint{3.202263in}{2.118649in}}%
\pgfpathlineto{\pgfqpoint{3.210176in}{2.126791in}}%
\pgfpathlineto{\pgfqpoint{3.218081in}{2.135001in}}%
\pgfpathlineto{\pgfqpoint{3.205043in}{2.143257in}}%
\pgfpathlineto{\pgfqpoint{3.192005in}{2.151723in}}%
\pgfpathlineto{\pgfqpoint{3.178967in}{2.160402in}}%
\pgfpathlineto{\pgfqpoint{3.165928in}{2.169295in}}%
\pgfpathlineto{\pgfqpoint{3.158005in}{2.161266in}}%
\pgfpathlineto{\pgfqpoint{3.150074in}{2.153312in}}%
\pgfpathlineto{\pgfqpoint{3.142136in}{2.145435in}}%
\pgfpathlineto{\pgfqpoint{3.134191in}{2.137637in}}%
\pgfpathclose%
\pgfusepath{fill}%
\end{pgfscope}%
\begin{pgfscope}%
\pgfpathrectangle{\pgfqpoint{1.254980in}{0.150000in}}{\pgfqpoint{5.490039in}{5.490039in}}%
\pgfusepath{clip}%
\pgfsetbuttcap%
\pgfsetroundjoin%
\definecolor{currentfill}{rgb}{0.280868,0.160771,0.472899}%
\pgfsetfillcolor{currentfill}%
\pgfsetfillopacity{0.700000}%
\pgfsetlinewidth{0.000000pt}%
\definecolor{currentstroke}{rgb}{0.000000,0.000000,0.000000}%
\pgfsetstrokecolor{currentstroke}%
\pgfsetdash{}{0pt}%
\pgfpathmoveto{\pgfqpoint{2.892813in}{2.286488in}}%
\pgfpathlineto{\pgfqpoint{2.905932in}{2.273548in}}%
\pgfpathlineto{\pgfqpoint{2.919048in}{2.260848in}}%
\pgfpathlineto{\pgfqpoint{2.932160in}{2.248386in}}%
\pgfpathlineto{\pgfqpoint{2.945267in}{2.236160in}}%
\pgfpathlineto{\pgfqpoint{2.953307in}{2.242931in}}%
\pgfpathlineto{\pgfqpoint{2.961338in}{2.249809in}}%
\pgfpathlineto{\pgfqpoint{2.969361in}{2.256792in}}%
\pgfpathlineto{\pgfqpoint{2.977375in}{2.263878in}}%
\pgfpathlineto{\pgfqpoint{2.964289in}{2.275881in}}%
\pgfpathlineto{\pgfqpoint{2.951201in}{2.288120in}}%
\pgfpathlineto{\pgfqpoint{2.938108in}{2.300596in}}%
\pgfpathlineto{\pgfqpoint{2.925012in}{2.313312in}}%
\pgfpathlineto{\pgfqpoint{2.916975in}{2.306439in}}%
\pgfpathlineto{\pgfqpoint{2.908930in}{2.299676in}}%
\pgfpathlineto{\pgfqpoint{2.900876in}{2.293025in}}%
\pgfpathlineto{\pgfqpoint{2.892813in}{2.286488in}}%
\pgfpathclose%
\pgfusepath{fill}%
\end{pgfscope}%
\begin{pgfscope}%
\pgfpathrectangle{\pgfqpoint{1.254980in}{0.150000in}}{\pgfqpoint{5.490039in}{5.490039in}}%
\pgfusepath{clip}%
\pgfsetbuttcap%
\pgfsetroundjoin%
\definecolor{currentfill}{rgb}{0.157729,0.485932,0.558013}%
\pgfsetfillcolor{currentfill}%
\pgfsetfillopacity{0.700000}%
\pgfsetlinewidth{0.000000pt}%
\definecolor{currentstroke}{rgb}{0.000000,0.000000,0.000000}%
\pgfsetstrokecolor{currentstroke}%
\pgfsetdash{}{0pt}%
\pgfpathmoveto{\pgfqpoint{5.285813in}{2.998127in}}%
\pgfpathlineto{\pgfqpoint{5.299492in}{3.004594in}}%
\pgfpathlineto{\pgfqpoint{5.313186in}{3.011214in}}%
\pgfpathlineto{\pgfqpoint{5.326895in}{3.017988in}}%
\pgfpathlineto{\pgfqpoint{5.340620in}{3.024915in}}%
\pgfpathlineto{\pgfqpoint{5.347713in}{3.030288in}}%
\pgfpathlineto{\pgfqpoint{5.354802in}{3.035727in}}%
\pgfpathlineto{\pgfqpoint{5.361887in}{3.041239in}}%
\pgfpathlineto{\pgfqpoint{5.368968in}{3.046829in}}%
\pgfpathlineto{\pgfqpoint{5.355266in}{3.040395in}}%
\pgfpathlineto{\pgfqpoint{5.341579in}{3.034115in}}%
\pgfpathlineto{\pgfqpoint{5.327907in}{3.027986in}}%
\pgfpathlineto{\pgfqpoint{5.314249in}{3.022011in}}%
\pgfpathlineto{\pgfqpoint{5.307145in}{3.015919in}}%
\pgfpathlineto{\pgfqpoint{5.300038in}{3.009911in}}%
\pgfpathlineto{\pgfqpoint{5.292928in}{3.003982in}}%
\pgfpathlineto{\pgfqpoint{5.285813in}{2.998127in}}%
\pgfpathclose%
\pgfusepath{fill}%
\end{pgfscope}%
\begin{pgfscope}%
\pgfpathrectangle{\pgfqpoint{1.254980in}{0.150000in}}{\pgfqpoint{5.490039in}{5.490039in}}%
\pgfusepath{clip}%
\pgfsetbuttcap%
\pgfsetroundjoin%
\definecolor{currentfill}{rgb}{0.150476,0.504369,0.557430}%
\pgfsetfillcolor{currentfill}%
\pgfsetfillopacity{0.700000}%
\pgfsetlinewidth{0.000000pt}%
\definecolor{currentstroke}{rgb}{0.000000,0.000000,0.000000}%
\pgfsetstrokecolor{currentstroke}%
\pgfsetdash{}{0pt}%
\pgfpathmoveto{\pgfqpoint{5.368968in}{3.046829in}}%
\pgfpathlineto{\pgfqpoint{5.382686in}{3.053415in}}%
\pgfpathlineto{\pgfqpoint{5.396419in}{3.060153in}}%
\pgfpathlineto{\pgfqpoint{5.410168in}{3.067045in}}%
\pgfpathlineto{\pgfqpoint{5.423932in}{3.074088in}}%
\pgfpathlineto{\pgfqpoint{5.430986in}{3.079249in}}%
\pgfpathlineto{\pgfqpoint{5.438037in}{3.084493in}}%
\pgfpathlineto{\pgfqpoint{5.445085in}{3.089825in}}%
\pgfpathlineto{\pgfqpoint{5.452130in}{3.095251in}}%
\pgfpathlineto{\pgfqpoint{5.438390in}{3.088730in}}%
\pgfpathlineto{\pgfqpoint{5.424665in}{3.082361in}}%
\pgfpathlineto{\pgfqpoint{5.410955in}{3.076143in}}%
\pgfpathlineto{\pgfqpoint{5.397261in}{3.070078in}}%
\pgfpathlineto{\pgfqpoint{5.390192in}{3.064120in}}%
\pgfpathlineto{\pgfqpoint{5.383121in}{3.058264in}}%
\pgfpathlineto{\pgfqpoint{5.376046in}{3.052502in}}%
\pgfpathlineto{\pgfqpoint{5.368968in}{3.046829in}}%
\pgfpathclose%
\pgfusepath{fill}%
\end{pgfscope}%
\begin{pgfscope}%
\pgfpathrectangle{\pgfqpoint{1.254980in}{0.150000in}}{\pgfqpoint{5.490039in}{5.490039in}}%
\pgfusepath{clip}%
\pgfsetbuttcap%
\pgfsetroundjoin%
\definecolor{currentfill}{rgb}{0.221989,0.339161,0.548752}%
\pgfsetfillcolor{currentfill}%
\pgfsetfillopacity{0.700000}%
\pgfsetlinewidth{0.000000pt}%
\definecolor{currentstroke}{rgb}{0.000000,0.000000,0.000000}%
\pgfsetstrokecolor{currentstroke}%
\pgfsetdash{}{0pt}%
\pgfpathmoveto{\pgfqpoint{2.576007in}{2.674399in}}%
\pgfpathlineto{\pgfqpoint{2.589305in}{2.655056in}}%
\pgfpathlineto{\pgfqpoint{2.602593in}{2.636009in}}%
\pgfpathlineto{\pgfqpoint{2.615870in}{2.617254in}}%
\pgfpathlineto{\pgfqpoint{2.629137in}{2.598790in}}%
\pgfpathlineto{\pgfqpoint{2.637335in}{2.604132in}}%
\pgfpathlineto{\pgfqpoint{2.645521in}{2.609625in}}%
\pgfpathlineto{\pgfqpoint{2.653697in}{2.615266in}}%
\pgfpathlineto{\pgfqpoint{2.661862in}{2.621053in}}%
\pgfpathlineto{\pgfqpoint{2.648624in}{2.639285in}}%
\pgfpathlineto{\pgfqpoint{2.635376in}{2.657807in}}%
\pgfpathlineto{\pgfqpoint{2.622119in}{2.676620in}}%
\pgfpathlineto{\pgfqpoint{2.608851in}{2.695728in}}%
\pgfpathlineto{\pgfqpoint{2.600657in}{2.690164in}}%
\pgfpathlineto{\pgfqpoint{2.592452in}{2.684753in}}%
\pgfpathlineto{\pgfqpoint{2.584235in}{2.679497in}}%
\pgfpathlineto{\pgfqpoint{2.576007in}{2.674399in}}%
\pgfpathclose%
\pgfusepath{fill}%
\end{pgfscope}%
\begin{pgfscope}%
\pgfpathrectangle{\pgfqpoint{1.254980in}{0.150000in}}{\pgfqpoint{5.490039in}{5.490039in}}%
\pgfusepath{clip}%
\pgfsetbuttcap%
\pgfsetroundjoin%
\definecolor{currentfill}{rgb}{0.143343,0.522773,0.556295}%
\pgfsetfillcolor{currentfill}%
\pgfsetfillopacity{0.700000}%
\pgfsetlinewidth{0.000000pt}%
\definecolor{currentstroke}{rgb}{0.000000,0.000000,0.000000}%
\pgfsetstrokecolor{currentstroke}%
\pgfsetdash{}{0pt}%
\pgfpathmoveto{\pgfqpoint{5.452130in}{3.095251in}}%
\pgfpathlineto{\pgfqpoint{5.465885in}{3.101924in}}%
\pgfpathlineto{\pgfqpoint{5.479657in}{3.108749in}}%
\pgfpathlineto{\pgfqpoint{5.493444in}{3.115725in}}%
\pgfpathlineto{\pgfqpoint{5.507247in}{3.122854in}}%
\pgfpathlineto{\pgfqpoint{5.514263in}{3.127839in}}%
\pgfpathlineto{\pgfqpoint{5.521277in}{3.132923in}}%
\pgfpathlineto{\pgfqpoint{5.528288in}{3.138113in}}%
\pgfpathlineto{\pgfqpoint{5.535296in}{3.143415in}}%
\pgfpathlineto{\pgfqpoint{5.521519in}{3.136838in}}%
\pgfpathlineto{\pgfqpoint{5.507757in}{3.130412in}}%
\pgfpathlineto{\pgfqpoint{5.494012in}{3.124138in}}%
\pgfpathlineto{\pgfqpoint{5.480281in}{3.118014in}}%
\pgfpathlineto{\pgfqpoint{5.473247in}{3.112152in}}%
\pgfpathlineto{\pgfqpoint{5.466210in}{3.106409in}}%
\pgfpathlineto{\pgfqpoint{5.459171in}{3.100777in}}%
\pgfpathlineto{\pgfqpoint{5.452130in}{3.095251in}}%
\pgfpathclose%
\pgfusepath{fill}%
\end{pgfscope}%
\begin{pgfscope}%
\pgfpathrectangle{\pgfqpoint{1.254980in}{0.150000in}}{\pgfqpoint{5.490039in}{5.490039in}}%
\pgfusepath{clip}%
\pgfsetbuttcap%
\pgfsetroundjoin%
\definecolor{currentfill}{rgb}{0.279566,0.067836,0.391917}%
\pgfsetfillcolor{currentfill}%
\pgfsetfillopacity{0.700000}%
\pgfsetlinewidth{0.000000pt}%
\definecolor{currentstroke}{rgb}{0.000000,0.000000,0.000000}%
\pgfsetstrokecolor{currentstroke}%
\pgfsetdash{}{0pt}%
\pgfpathmoveto{\pgfqpoint{3.541525in}{2.086330in}}%
\pgfpathlineto{\pgfqpoint{3.554576in}{2.082540in}}%
\pgfpathlineto{\pgfqpoint{3.567631in}{2.078936in}}%
\pgfpathlineto{\pgfqpoint{3.580691in}{2.075520in}}%
\pgfpathlineto{\pgfqpoint{3.593755in}{2.072289in}}%
\pgfpathlineto{\pgfqpoint{3.601523in}{2.081762in}}%
\pgfpathlineto{\pgfqpoint{3.609286in}{2.091250in}}%
\pgfpathlineto{\pgfqpoint{3.617043in}{2.100753in}}%
\pgfpathlineto{\pgfqpoint{3.624795in}{2.110270in}}%
\pgfpathlineto{\pgfqpoint{3.611742in}{2.113395in}}%
\pgfpathlineto{\pgfqpoint{3.598694in}{2.116705in}}%
\pgfpathlineto{\pgfqpoint{3.585649in}{2.120202in}}%
\pgfpathlineto{\pgfqpoint{3.572609in}{2.123887in}}%
\pgfpathlineto{\pgfqpoint{3.564846in}{2.114465in}}%
\pgfpathlineto{\pgfqpoint{3.557078in}{2.105065in}}%
\pgfpathlineto{\pgfqpoint{3.549304in}{2.095686in}}%
\pgfpathlineto{\pgfqpoint{3.541525in}{2.086330in}}%
\pgfpathclose%
\pgfusepath{fill}%
\end{pgfscope}%
\begin{pgfscope}%
\pgfpathrectangle{\pgfqpoint{1.254980in}{0.150000in}}{\pgfqpoint{5.490039in}{5.490039in}}%
\pgfusepath{clip}%
\pgfsetbuttcap%
\pgfsetroundjoin%
\definecolor{currentfill}{rgb}{0.136408,0.541173,0.554483}%
\pgfsetfillcolor{currentfill}%
\pgfsetfillopacity{0.700000}%
\pgfsetlinewidth{0.000000pt}%
\definecolor{currentstroke}{rgb}{0.000000,0.000000,0.000000}%
\pgfsetstrokecolor{currentstroke}%
\pgfsetdash{}{0pt}%
\pgfpathmoveto{\pgfqpoint{5.535296in}{3.143415in}}%
\pgfpathlineto{\pgfqpoint{5.549089in}{3.150143in}}%
\pgfpathlineto{\pgfqpoint{5.562898in}{3.157022in}}%
\pgfpathlineto{\pgfqpoint{5.576723in}{3.164052in}}%
\pgfpathlineto{\pgfqpoint{5.590564in}{3.171234in}}%
\pgfpathlineto{\pgfqpoint{5.597543in}{3.176084in}}%
\pgfpathlineto{\pgfqpoint{5.604520in}{3.181051in}}%
\pgfpathlineto{\pgfqpoint{5.611495in}{3.186142in}}%
\pgfpathlineto{\pgfqpoint{5.618468in}{3.191364in}}%
\pgfpathlineto{\pgfqpoint{5.604654in}{3.184763in}}%
\pgfpathlineto{\pgfqpoint{5.590857in}{3.178313in}}%
\pgfpathlineto{\pgfqpoint{5.577075in}{3.172012in}}%
\pgfpathlineto{\pgfqpoint{5.563309in}{3.165862in}}%
\pgfpathlineto{\pgfqpoint{5.556309in}{3.160052in}}%
\pgfpathlineto{\pgfqpoint{5.549306in}{3.154378in}}%
\pgfpathlineto{\pgfqpoint{5.542302in}{3.148835in}}%
\pgfpathlineto{\pgfqpoint{5.535296in}{3.143415in}}%
\pgfpathclose%
\pgfusepath{fill}%
\end{pgfscope}%
\begin{pgfscope}%
\pgfpathrectangle{\pgfqpoint{1.254980in}{0.150000in}}{\pgfqpoint{5.490039in}{5.490039in}}%
\pgfusepath{clip}%
\pgfsetbuttcap%
\pgfsetroundjoin%
\definecolor{currentfill}{rgb}{0.278012,0.180367,0.486697}%
\pgfsetfillcolor{currentfill}%
\pgfsetfillopacity{0.700000}%
\pgfsetlinewidth{0.000000pt}%
\definecolor{currentstroke}{rgb}{0.000000,0.000000,0.000000}%
\pgfsetstrokecolor{currentstroke}%
\pgfsetdash{}{0pt}%
\pgfpathmoveto{\pgfqpoint{4.092722in}{2.284920in}}%
\pgfpathlineto{\pgfqpoint{4.105899in}{2.286332in}}%
\pgfpathlineto{\pgfqpoint{4.119085in}{2.287914in}}%
\pgfpathlineto{\pgfqpoint{4.132280in}{2.289666in}}%
\pgfpathlineto{\pgfqpoint{4.145484in}{2.291586in}}%
\pgfpathlineto{\pgfqpoint{4.153070in}{2.301033in}}%
\pgfpathlineto{\pgfqpoint{4.160650in}{2.310451in}}%
\pgfpathlineto{\pgfqpoint{4.168226in}{2.319839in}}%
\pgfpathlineto{\pgfqpoint{4.175796in}{2.329200in}}%
\pgfpathlineto{\pgfqpoint{4.162600in}{2.327342in}}%
\pgfpathlineto{\pgfqpoint{4.149412in}{2.325653in}}%
\pgfpathlineto{\pgfqpoint{4.136234in}{2.324132in}}%
\pgfpathlineto{\pgfqpoint{4.123064in}{2.322782in}}%
\pgfpathlineto{\pgfqpoint{4.115486in}{2.313348in}}%
\pgfpathlineto{\pgfqpoint{4.107903in}{2.303894in}}%
\pgfpathlineto{\pgfqpoint{4.100315in}{2.294419in}}%
\pgfpathlineto{\pgfqpoint{4.092722in}{2.284920in}}%
\pgfpathclose%
\pgfusepath{fill}%
\end{pgfscope}%
\begin{pgfscope}%
\pgfpathrectangle{\pgfqpoint{1.254980in}{0.150000in}}{\pgfqpoint{5.490039in}{5.490039in}}%
\pgfusepath{clip}%
\pgfsetbuttcap%
\pgfsetroundjoin%
\definecolor{currentfill}{rgb}{0.282623,0.140926,0.457517}%
\pgfsetfillcolor{currentfill}%
\pgfsetfillopacity{0.700000}%
\pgfsetlinewidth{0.000000pt}%
\definecolor{currentstroke}{rgb}{0.000000,0.000000,0.000000}%
\pgfsetstrokecolor{currentstroke}%
\pgfsetdash{}{0pt}%
\pgfpathmoveto{\pgfqpoint{2.945267in}{2.236160in}}%
\pgfpathlineto{\pgfqpoint{2.958371in}{2.224169in}}%
\pgfpathlineto{\pgfqpoint{2.971472in}{2.212410in}}%
\pgfpathlineto{\pgfqpoint{2.984570in}{2.200883in}}%
\pgfpathlineto{\pgfqpoint{2.997664in}{2.189584in}}%
\pgfpathlineto{\pgfqpoint{3.005681in}{2.196588in}}%
\pgfpathlineto{\pgfqpoint{3.013690in}{2.203692in}}%
\pgfpathlineto{\pgfqpoint{3.021691in}{2.210894in}}%
\pgfpathlineto{\pgfqpoint{3.029683in}{2.218191in}}%
\pgfpathlineto{\pgfqpoint{3.016610in}{2.229268in}}%
\pgfpathlineto{\pgfqpoint{3.003535in}{2.240573in}}%
\pgfpathlineto{\pgfqpoint{2.990456in}{2.252110in}}%
\pgfpathlineto{\pgfqpoint{2.977375in}{2.263878in}}%
\pgfpathlineto{\pgfqpoint{2.969361in}{2.256792in}}%
\pgfpathlineto{\pgfqpoint{2.961338in}{2.249809in}}%
\pgfpathlineto{\pgfqpoint{2.953307in}{2.242931in}}%
\pgfpathlineto{\pgfqpoint{2.945267in}{2.236160in}}%
\pgfpathclose%
\pgfusepath{fill}%
\end{pgfscope}%
\begin{pgfscope}%
\pgfpathrectangle{\pgfqpoint{1.254980in}{0.150000in}}{\pgfqpoint{5.490039in}{5.490039in}}%
\pgfusepath{clip}%
\pgfsetbuttcap%
\pgfsetroundjoin%
\definecolor{currentfill}{rgb}{0.273006,0.204520,0.501721}%
\pgfsetfillcolor{currentfill}%
\pgfsetfillopacity{0.700000}%
\pgfsetlinewidth{0.000000pt}%
\definecolor{currentstroke}{rgb}{0.000000,0.000000,0.000000}%
\pgfsetstrokecolor{currentstroke}%
\pgfsetdash{}{0pt}%
\pgfpathmoveto{\pgfqpoint{4.175796in}{2.329200in}}%
\pgfpathlineto{\pgfqpoint{4.189002in}{2.331227in}}%
\pgfpathlineto{\pgfqpoint{4.202217in}{2.333422in}}%
\pgfpathlineto{\pgfqpoint{4.215442in}{2.335784in}}%
\pgfpathlineto{\pgfqpoint{4.228676in}{2.338314in}}%
\pgfpathlineto{\pgfqpoint{4.236234in}{2.347569in}}%
\pgfpathlineto{\pgfqpoint{4.243786in}{2.356791in}}%
\pgfpathlineto{\pgfqpoint{4.251333in}{2.365983in}}%
\pgfpathlineto{\pgfqpoint{4.258876in}{2.375146in}}%
\pgfpathlineto{\pgfqpoint{4.245649in}{2.372707in}}%
\pgfpathlineto{\pgfqpoint{4.232432in}{2.370435in}}%
\pgfpathlineto{\pgfqpoint{4.219224in}{2.368330in}}%
\pgfpathlineto{\pgfqpoint{4.206026in}{2.366393in}}%
\pgfpathlineto{\pgfqpoint{4.198477in}{2.357130in}}%
\pgfpathlineto{\pgfqpoint{4.190921in}{2.347845in}}%
\pgfpathlineto{\pgfqpoint{4.183361in}{2.338535in}}%
\pgfpathlineto{\pgfqpoint{4.175796in}{2.329200in}}%
\pgfpathclose%
\pgfusepath{fill}%
\end{pgfscope}%
\begin{pgfscope}%
\pgfpathrectangle{\pgfqpoint{1.254980in}{0.150000in}}{\pgfqpoint{5.490039in}{5.490039in}}%
\pgfusepath{clip}%
\pgfsetbuttcap%
\pgfsetroundjoin%
\definecolor{currentfill}{rgb}{0.280868,0.160771,0.472899}%
\pgfsetfillcolor{currentfill}%
\pgfsetfillopacity{0.700000}%
\pgfsetlinewidth{0.000000pt}%
\definecolor{currentstroke}{rgb}{0.000000,0.000000,0.000000}%
\pgfsetstrokecolor{currentstroke}%
\pgfsetdash{}{0pt}%
\pgfpathmoveto{\pgfqpoint{4.009644in}{2.242602in}}%
\pgfpathlineto{\pgfqpoint{4.022795in}{2.243364in}}%
\pgfpathlineto{\pgfqpoint{4.035955in}{2.244298in}}%
\pgfpathlineto{\pgfqpoint{4.049122in}{2.245403in}}%
\pgfpathlineto{\pgfqpoint{4.062299in}{2.246679in}}%
\pgfpathlineto{\pgfqpoint{4.069912in}{2.256278in}}%
\pgfpathlineto{\pgfqpoint{4.077520in}{2.265850in}}%
\pgfpathlineto{\pgfqpoint{4.085123in}{2.275397in}}%
\pgfpathlineto{\pgfqpoint{4.092722in}{2.284920in}}%
\pgfpathlineto{\pgfqpoint{4.079553in}{2.283678in}}%
\pgfpathlineto{\pgfqpoint{4.066393in}{2.282606in}}%
\pgfpathlineto{\pgfqpoint{4.053241in}{2.281706in}}%
\pgfpathlineto{\pgfqpoint{4.040098in}{2.280978in}}%
\pgfpathlineto{\pgfqpoint{4.032492in}{2.271411in}}%
\pgfpathlineto{\pgfqpoint{4.024881in}{2.261827in}}%
\pgfpathlineto{\pgfqpoint{4.017265in}{2.252224in}}%
\pgfpathlineto{\pgfqpoint{4.009644in}{2.242602in}}%
\pgfpathclose%
\pgfusepath{fill}%
\end{pgfscope}%
\begin{pgfscope}%
\pgfpathrectangle{\pgfqpoint{1.254980in}{0.150000in}}{\pgfqpoint{5.490039in}{5.490039in}}%
\pgfusepath{clip}%
\pgfsetbuttcap%
\pgfsetroundjoin%
\definecolor{currentfill}{rgb}{0.129933,0.559582,0.551864}%
\pgfsetfillcolor{currentfill}%
\pgfsetfillopacity{0.700000}%
\pgfsetlinewidth{0.000000pt}%
\definecolor{currentstroke}{rgb}{0.000000,0.000000,0.000000}%
\pgfsetstrokecolor{currentstroke}%
\pgfsetdash{}{0pt}%
\pgfpathmoveto{\pgfqpoint{5.618468in}{3.191364in}}%
\pgfpathlineto{\pgfqpoint{5.632297in}{3.198116in}}%
\pgfpathlineto{\pgfqpoint{5.646143in}{3.205017in}}%
\pgfpathlineto{\pgfqpoint{5.660004in}{3.212070in}}%
\pgfpathlineto{\pgfqpoint{5.673883in}{3.219273in}}%
\pgfpathlineto{\pgfqpoint{5.680825in}{3.224033in}}%
\pgfpathlineto{\pgfqpoint{5.687766in}{3.228930in}}%
\pgfpathlineto{\pgfqpoint{5.694706in}{3.233971in}}%
\pgfpathlineto{\pgfqpoint{5.701645in}{3.239162in}}%
\pgfpathlineto{\pgfqpoint{5.687797in}{3.232568in}}%
\pgfpathlineto{\pgfqpoint{5.673965in}{3.226124in}}%
\pgfpathlineto{\pgfqpoint{5.660148in}{3.219830in}}%
\pgfpathlineto{\pgfqpoint{5.646348in}{3.213685in}}%
\pgfpathlineto{\pgfqpoint{5.639379in}{3.207877in}}%
\pgfpathlineto{\pgfqpoint{5.632410in}{3.202225in}}%
\pgfpathlineto{\pgfqpoint{5.625439in}{3.196723in}}%
\pgfpathlineto{\pgfqpoint{5.618468in}{3.191364in}}%
\pgfpathclose%
\pgfusepath{fill}%
\end{pgfscope}%
\begin{pgfscope}%
\pgfpathrectangle{\pgfqpoint{1.254980in}{0.150000in}}{\pgfqpoint{5.490039in}{5.490039in}}%
\pgfusepath{clip}%
\pgfsetbuttcap%
\pgfsetroundjoin%
\definecolor{currentfill}{rgb}{0.266580,0.228262,0.514349}%
\pgfsetfillcolor{currentfill}%
\pgfsetfillopacity{0.700000}%
\pgfsetlinewidth{0.000000pt}%
\definecolor{currentstroke}{rgb}{0.000000,0.000000,0.000000}%
\pgfsetstrokecolor{currentstroke}%
\pgfsetdash{}{0pt}%
\pgfpathmoveto{\pgfqpoint{4.258876in}{2.375146in}}%
\pgfpathlineto{\pgfqpoint{4.272112in}{2.377752in}}%
\pgfpathlineto{\pgfqpoint{4.285359in}{2.380524in}}%
\pgfpathlineto{\pgfqpoint{4.298615in}{2.383463in}}%
\pgfpathlineto{\pgfqpoint{4.311882in}{2.386567in}}%
\pgfpathlineto{\pgfqpoint{4.319411in}{2.395594in}}%
\pgfpathlineto{\pgfqpoint{4.326935in}{2.404587in}}%
\pgfpathlineto{\pgfqpoint{4.334454in}{2.413548in}}%
\pgfpathlineto{\pgfqpoint{4.341967in}{2.422480in}}%
\pgfpathlineto{\pgfqpoint{4.328708in}{2.419495in}}%
\pgfpathlineto{\pgfqpoint{4.315460in}{2.416675in}}%
\pgfpathlineto{\pgfqpoint{4.302221in}{2.414021in}}%
\pgfpathlineto{\pgfqpoint{4.288992in}{2.411533in}}%
\pgfpathlineto{\pgfqpoint{4.281471in}{2.402473in}}%
\pgfpathlineto{\pgfqpoint{4.273944in}{2.393389in}}%
\pgfpathlineto{\pgfqpoint{4.266413in}{2.384281in}}%
\pgfpathlineto{\pgfqpoint{4.258876in}{2.375146in}}%
\pgfpathclose%
\pgfusepath{fill}%
\end{pgfscope}%
\begin{pgfscope}%
\pgfpathrectangle{\pgfqpoint{1.254980in}{0.150000in}}{\pgfqpoint{5.490039in}{5.490039in}}%
\pgfusepath{clip}%
\pgfsetbuttcap%
\pgfsetroundjoin%
\definecolor{currentfill}{rgb}{0.282884,0.135920,0.453427}%
\pgfsetfillcolor{currentfill}%
\pgfsetfillopacity{0.700000}%
\pgfsetlinewidth{0.000000pt}%
\definecolor{currentstroke}{rgb}{0.000000,0.000000,0.000000}%
\pgfsetstrokecolor{currentstroke}%
\pgfsetdash{}{0pt}%
\pgfpathmoveto{\pgfqpoint{3.926553in}{2.202563in}}%
\pgfpathlineto{\pgfqpoint{3.939681in}{2.202639in}}%
\pgfpathlineto{\pgfqpoint{3.952816in}{2.202889in}}%
\pgfpathlineto{\pgfqpoint{3.965959in}{2.203312in}}%
\pgfpathlineto{\pgfqpoint{3.979110in}{2.203908in}}%
\pgfpathlineto{\pgfqpoint{3.986751in}{2.213613in}}%
\pgfpathlineto{\pgfqpoint{3.994387in}{2.223297in}}%
\pgfpathlineto{\pgfqpoint{4.002018in}{2.232960in}}%
\pgfpathlineto{\pgfqpoint{4.009644in}{2.242602in}}%
\pgfpathlineto{\pgfqpoint{3.996501in}{2.242012in}}%
\pgfpathlineto{\pgfqpoint{3.983366in}{2.241594in}}%
\pgfpathlineto{\pgfqpoint{3.970238in}{2.241350in}}%
\pgfpathlineto{\pgfqpoint{3.957118in}{2.241280in}}%
\pgfpathlineto{\pgfqpoint{3.949484in}{2.231622in}}%
\pgfpathlineto{\pgfqpoint{3.941845in}{2.221950in}}%
\pgfpathlineto{\pgfqpoint{3.934202in}{2.212264in}}%
\pgfpathlineto{\pgfqpoint{3.926553in}{2.202563in}}%
\pgfpathclose%
\pgfusepath{fill}%
\end{pgfscope}%
\begin{pgfscope}%
\pgfpathrectangle{\pgfqpoint{1.254980in}{0.150000in}}{\pgfqpoint{5.490039in}{5.490039in}}%
\pgfusepath{clip}%
\pgfsetbuttcap%
\pgfsetroundjoin%
\definecolor{currentfill}{rgb}{0.124395,0.578002,0.548287}%
\pgfsetfillcolor{currentfill}%
\pgfsetfillopacity{0.700000}%
\pgfsetlinewidth{0.000000pt}%
\definecolor{currentstroke}{rgb}{0.000000,0.000000,0.000000}%
\pgfsetstrokecolor{currentstroke}%
\pgfsetdash{}{0pt}%
\pgfpathmoveto{\pgfqpoint{5.701645in}{3.239162in}}%
\pgfpathlineto{\pgfqpoint{5.715510in}{3.245905in}}%
\pgfpathlineto{\pgfqpoint{5.729391in}{3.252798in}}%
\pgfpathlineto{\pgfqpoint{5.743289in}{3.259841in}}%
\pgfpathlineto{\pgfqpoint{5.757203in}{3.267034in}}%
\pgfpathlineto{\pgfqpoint{5.764110in}{3.271755in}}%
\pgfpathlineto{\pgfqpoint{5.771017in}{3.276634in}}%
\pgfpathlineto{\pgfqpoint{5.777924in}{3.281677in}}%
\pgfpathlineto{\pgfqpoint{5.784831in}{3.286893in}}%
\pgfpathlineto{\pgfqpoint{5.770949in}{3.280338in}}%
\pgfpathlineto{\pgfqpoint{5.757083in}{3.273932in}}%
\pgfpathlineto{\pgfqpoint{5.743233in}{3.267675in}}%
\pgfpathlineto{\pgfqpoint{5.729399in}{3.261567in}}%
\pgfpathlineto{\pgfqpoint{5.722460in}{3.255706in}}%
\pgfpathlineto{\pgfqpoint{5.715522in}{3.250022in}}%
\pgfpathlineto{\pgfqpoint{5.708584in}{3.244510in}}%
\pgfpathlineto{\pgfqpoint{5.701645in}{3.239162in}}%
\pgfpathclose%
\pgfusepath{fill}%
\end{pgfscope}%
\begin{pgfscope}%
\pgfpathrectangle{\pgfqpoint{1.254980in}{0.150000in}}{\pgfqpoint{5.490039in}{5.490039in}}%
\pgfusepath{clip}%
\pgfsetbuttcap%
\pgfsetroundjoin%
\definecolor{currentfill}{rgb}{0.258965,0.251537,0.524736}%
\pgfsetfillcolor{currentfill}%
\pgfsetfillopacity{0.700000}%
\pgfsetlinewidth{0.000000pt}%
\definecolor{currentstroke}{rgb}{0.000000,0.000000,0.000000}%
\pgfsetstrokecolor{currentstroke}%
\pgfsetdash{}{0pt}%
\pgfpathmoveto{\pgfqpoint{4.341967in}{2.422480in}}%
\pgfpathlineto{\pgfqpoint{4.355236in}{2.425630in}}%
\pgfpathlineto{\pgfqpoint{4.368516in}{2.428946in}}%
\pgfpathlineto{\pgfqpoint{4.381807in}{2.432426in}}%
\pgfpathlineto{\pgfqpoint{4.395108in}{2.436071in}}%
\pgfpathlineto{\pgfqpoint{4.402608in}{2.444837in}}%
\pgfpathlineto{\pgfqpoint{4.410102in}{2.453570in}}%
\pgfpathlineto{\pgfqpoint{4.417592in}{2.462272in}}%
\pgfpathlineto{\pgfqpoint{4.425076in}{2.470945in}}%
\pgfpathlineto{\pgfqpoint{4.411783in}{2.467447in}}%
\pgfpathlineto{\pgfqpoint{4.398500in}{2.464114in}}%
\pgfpathlineto{\pgfqpoint{4.385229in}{2.460946in}}%
\pgfpathlineto{\pgfqpoint{4.371968in}{2.457942in}}%
\pgfpathlineto{\pgfqpoint{4.364476in}{2.449112in}}%
\pgfpathlineto{\pgfqpoint{4.356978in}{2.440260in}}%
\pgfpathlineto{\pgfqpoint{4.349475in}{2.431383in}}%
\pgfpathlineto{\pgfqpoint{4.341967in}{2.422480in}}%
\pgfpathclose%
\pgfusepath{fill}%
\end{pgfscope}%
\begin{pgfscope}%
\pgfpathrectangle{\pgfqpoint{1.254980in}{0.150000in}}{\pgfqpoint{5.490039in}{5.490039in}}%
\pgfusepath{clip}%
\pgfsetbuttcap%
\pgfsetroundjoin%
\definecolor{currentfill}{rgb}{0.120565,0.596422,0.543611}%
\pgfsetfillcolor{currentfill}%
\pgfsetfillopacity{0.700000}%
\pgfsetlinewidth{0.000000pt}%
\definecolor{currentstroke}{rgb}{0.000000,0.000000,0.000000}%
\pgfsetstrokecolor{currentstroke}%
\pgfsetdash{}{0pt}%
\pgfpathmoveto{\pgfqpoint{5.784831in}{3.286893in}}%
\pgfpathlineto{\pgfqpoint{5.798730in}{3.293596in}}%
\pgfpathlineto{\pgfqpoint{5.812645in}{3.300449in}}%
\pgfpathlineto{\pgfqpoint{5.826578in}{3.307451in}}%
\pgfpathlineto{\pgfqpoint{5.840527in}{3.314603in}}%
\pgfpathlineto{\pgfqpoint{5.847401in}{3.319341in}}%
\pgfpathlineto{\pgfqpoint{5.854276in}{3.324259in}}%
\pgfpathlineto{\pgfqpoint{5.861151in}{3.329363in}}%
\pgfpathlineto{\pgfqpoint{5.868028in}{3.334662in}}%
\pgfpathlineto{\pgfqpoint{5.854113in}{3.328177in}}%
\pgfpathlineto{\pgfqpoint{5.840214in}{3.321840in}}%
\pgfpathlineto{\pgfqpoint{5.826332in}{3.315652in}}%
\pgfpathlineto{\pgfqpoint{5.812466in}{3.309612in}}%
\pgfpathlineto{\pgfqpoint{5.805556in}{3.303639in}}%
\pgfpathlineto{\pgfqpoint{5.798646in}{3.297866in}}%
\pgfpathlineto{\pgfqpoint{5.791738in}{3.292286in}}%
\pgfpathlineto{\pgfqpoint{5.784831in}{3.286893in}}%
\pgfpathclose%
\pgfusepath{fill}%
\end{pgfscope}%
\begin{pgfscope}%
\pgfpathrectangle{\pgfqpoint{1.254980in}{0.150000in}}{\pgfqpoint{5.490039in}{5.490039in}}%
\pgfusepath{clip}%
\pgfsetbuttcap%
\pgfsetroundjoin%
\definecolor{currentfill}{rgb}{0.278791,0.062145,0.386592}%
\pgfsetfillcolor{currentfill}%
\pgfsetfillopacity{0.700000}%
\pgfsetlinewidth{0.000000pt}%
\definecolor{currentstroke}{rgb}{0.000000,0.000000,0.000000}%
\pgfsetstrokecolor{currentstroke}%
\pgfsetdash{}{0pt}%
\pgfpathmoveto{\pgfqpoint{3.322402in}{2.076409in}}%
\pgfpathlineto{\pgfqpoint{3.335447in}{2.069999in}}%
\pgfpathlineto{\pgfqpoint{3.348493in}{2.063787in}}%
\pgfpathlineto{\pgfqpoint{3.361542in}{2.057774in}}%
\pgfpathlineto{\pgfqpoint{3.374592in}{2.051958in}}%
\pgfpathlineto{\pgfqpoint{3.382444in}{2.060755in}}%
\pgfpathlineto{\pgfqpoint{3.390290in}{2.069596in}}%
\pgfpathlineto{\pgfqpoint{3.398129in}{2.078481in}}%
\pgfpathlineto{\pgfqpoint{3.405963in}{2.087408in}}%
\pgfpathlineto{\pgfqpoint{3.392927in}{2.093062in}}%
\pgfpathlineto{\pgfqpoint{3.379893in}{2.098913in}}%
\pgfpathlineto{\pgfqpoint{3.366861in}{2.104962in}}%
\pgfpathlineto{\pgfqpoint{3.353831in}{2.111210in}}%
\pgfpathlineto{\pgfqpoint{3.345983in}{2.102435in}}%
\pgfpathlineto{\pgfqpoint{3.338129in}{2.093708in}}%
\pgfpathlineto{\pgfqpoint{3.330269in}{2.085033in}}%
\pgfpathlineto{\pgfqpoint{3.322402in}{2.076409in}}%
\pgfpathclose%
\pgfusepath{fill}%
\end{pgfscope}%
\begin{pgfscope}%
\pgfpathrectangle{\pgfqpoint{1.254980in}{0.150000in}}{\pgfqpoint{5.490039in}{5.490039in}}%
\pgfusepath{clip}%
\pgfsetbuttcap%
\pgfsetroundjoin%
\definecolor{currentfill}{rgb}{0.283229,0.120777,0.440584}%
\pgfsetfillcolor{currentfill}%
\pgfsetfillopacity{0.700000}%
\pgfsetlinewidth{0.000000pt}%
\definecolor{currentstroke}{rgb}{0.000000,0.000000,0.000000}%
\pgfsetstrokecolor{currentstroke}%
\pgfsetdash{}{0pt}%
\pgfpathmoveto{\pgfqpoint{3.843435in}{2.165143in}}%
\pgfpathlineto{\pgfqpoint{3.856542in}{2.164496in}}%
\pgfpathlineto{\pgfqpoint{3.869657in}{2.164025in}}%
\pgfpathlineto{\pgfqpoint{3.882779in}{2.163730in}}%
\pgfpathlineto{\pgfqpoint{3.895907in}{2.163609in}}%
\pgfpathlineto{\pgfqpoint{3.903576in}{2.173371in}}%
\pgfpathlineto{\pgfqpoint{3.911240in}{2.183117in}}%
\pgfpathlineto{\pgfqpoint{3.918899in}{2.192848in}}%
\pgfpathlineto{\pgfqpoint{3.926553in}{2.202563in}}%
\pgfpathlineto{\pgfqpoint{3.913432in}{2.202662in}}%
\pgfpathlineto{\pgfqpoint{3.900319in}{2.202935in}}%
\pgfpathlineto{\pgfqpoint{3.887212in}{2.203384in}}%
\pgfpathlineto{\pgfqpoint{3.874113in}{2.204008in}}%
\pgfpathlineto{\pgfqpoint{3.866451in}{2.194305in}}%
\pgfpathlineto{\pgfqpoint{3.858784in}{2.184593in}}%
\pgfpathlineto{\pgfqpoint{3.851112in}{2.174872in}}%
\pgfpathlineto{\pgfqpoint{3.843435in}{2.165143in}}%
\pgfpathclose%
\pgfusepath{fill}%
\end{pgfscope}%
\begin{pgfscope}%
\pgfpathrectangle{\pgfqpoint{1.254980in}{0.150000in}}{\pgfqpoint{5.490039in}{5.490039in}}%
\pgfusepath{clip}%
\pgfsetbuttcap%
\pgfsetroundjoin%
\definecolor{currentfill}{rgb}{0.250425,0.274290,0.533103}%
\pgfsetfillcolor{currentfill}%
\pgfsetfillopacity{0.700000}%
\pgfsetlinewidth{0.000000pt}%
\definecolor{currentstroke}{rgb}{0.000000,0.000000,0.000000}%
\pgfsetstrokecolor{currentstroke}%
\pgfsetdash{}{0pt}%
\pgfpathmoveto{\pgfqpoint{4.425076in}{2.470945in}}%
\pgfpathlineto{\pgfqpoint{4.438379in}{2.474605in}}%
\pgfpathlineto{\pgfqpoint{4.451694in}{2.478430in}}%
\pgfpathlineto{\pgfqpoint{4.465020in}{2.482418in}}%
\pgfpathlineto{\pgfqpoint{4.478358in}{2.486568in}}%
\pgfpathlineto{\pgfqpoint{4.485828in}{2.495048in}}%
\pgfpathlineto{\pgfqpoint{4.493292in}{2.503495in}}%
\pgfpathlineto{\pgfqpoint{4.500751in}{2.511913in}}%
\pgfpathlineto{\pgfqpoint{4.508204in}{2.520303in}}%
\pgfpathlineto{\pgfqpoint{4.494876in}{2.516328in}}%
\pgfpathlineto{\pgfqpoint{4.481559in}{2.512516in}}%
\pgfpathlineto{\pgfqpoint{4.468253in}{2.508867in}}%
\pgfpathlineto{\pgfqpoint{4.454958in}{2.505382in}}%
\pgfpathlineto{\pgfqpoint{4.447495in}{2.496806in}}%
\pgfpathlineto{\pgfqpoint{4.440027in}{2.488209in}}%
\pgfpathlineto{\pgfqpoint{4.432554in}{2.479589in}}%
\pgfpathlineto{\pgfqpoint{4.425076in}{2.470945in}}%
\pgfpathclose%
\pgfusepath{fill}%
\end{pgfscope}%
\begin{pgfscope}%
\pgfpathrectangle{\pgfqpoint{1.254980in}{0.150000in}}{\pgfqpoint{5.490039in}{5.490039in}}%
\pgfusepath{clip}%
\pgfsetbuttcap%
\pgfsetroundjoin%
\definecolor{currentfill}{rgb}{0.206756,0.371758,0.553117}%
\pgfsetfillcolor{currentfill}%
\pgfsetfillopacity{0.700000}%
\pgfsetlinewidth{0.000000pt}%
\definecolor{currentstroke}{rgb}{0.000000,0.000000,0.000000}%
\pgfsetstrokecolor{currentstroke}%
\pgfsetdash{}{0pt}%
\pgfpathmoveto{\pgfqpoint{2.522703in}{2.754782in}}%
\pgfpathlineto{\pgfqpoint{2.536046in}{2.734228in}}%
\pgfpathlineto{\pgfqpoint{2.549378in}{2.713982in}}%
\pgfpathlineto{\pgfqpoint{2.562698in}{2.694040in}}%
\pgfpathlineto{\pgfqpoint{2.576007in}{2.674399in}}%
\pgfpathlineto{\pgfqpoint{2.584235in}{2.679497in}}%
\pgfpathlineto{\pgfqpoint{2.592452in}{2.684753in}}%
\pgfpathlineto{\pgfqpoint{2.600657in}{2.690164in}}%
\pgfpathlineto{\pgfqpoint{2.608851in}{2.695728in}}%
\pgfpathlineto{\pgfqpoint{2.595573in}{2.715134in}}%
\pgfpathlineto{\pgfqpoint{2.582283in}{2.734841in}}%
\pgfpathlineto{\pgfqpoint{2.568983in}{2.754851in}}%
\pgfpathlineto{\pgfqpoint{2.555671in}{2.775168in}}%
\pgfpathlineto{\pgfqpoint{2.547447in}{2.769829in}}%
\pgfpathlineto{\pgfqpoint{2.539211in}{2.764650in}}%
\pgfpathlineto{\pgfqpoint{2.530963in}{2.759633in}}%
\pgfpathlineto{\pgfqpoint{2.522703in}{2.754782in}}%
\pgfpathclose%
\pgfusepath{fill}%
\end{pgfscope}%
\begin{pgfscope}%
\pgfpathrectangle{\pgfqpoint{1.254980in}{0.150000in}}{\pgfqpoint{5.490039in}{5.490039in}}%
\pgfusepath{clip}%
\pgfsetbuttcap%
\pgfsetroundjoin%
\definecolor{currentfill}{rgb}{0.280894,0.078907,0.402329}%
\pgfsetfillcolor{currentfill}%
\pgfsetfillopacity{0.700000}%
\pgfsetlinewidth{0.000000pt}%
\definecolor{currentstroke}{rgb}{0.000000,0.000000,0.000000}%
\pgfsetstrokecolor{currentstroke}%
\pgfsetdash{}{0pt}%
\pgfpathmoveto{\pgfqpoint{3.186416in}{2.102576in}}%
\pgfpathlineto{\pgfqpoint{3.199472in}{2.094340in}}%
\pgfpathlineto{\pgfqpoint{3.212527in}{2.086312in}}%
\pgfpathlineto{\pgfqpoint{3.225583in}{2.078491in}}%
\pgfpathlineto{\pgfqpoint{3.238639in}{2.070877in}}%
\pgfpathlineto{\pgfqpoint{3.246548in}{2.079079in}}%
\pgfpathlineto{\pgfqpoint{3.254451in}{2.087345in}}%
\pgfpathlineto{\pgfqpoint{3.262346in}{2.095675in}}%
\pgfpathlineto{\pgfqpoint{3.270235in}{2.104065in}}%
\pgfpathlineto{\pgfqpoint{3.257196in}{2.111489in}}%
\pgfpathlineto{\pgfqpoint{3.244157in}{2.119119in}}%
\pgfpathlineto{\pgfqpoint{3.231119in}{2.126956in}}%
\pgfpathlineto{\pgfqpoint{3.218081in}{2.135001in}}%
\pgfpathlineto{\pgfqpoint{3.210176in}{2.126791in}}%
\pgfpathlineto{\pgfqpoint{3.202263in}{2.118649in}}%
\pgfpathlineto{\pgfqpoint{3.194343in}{2.110577in}}%
\pgfpathlineto{\pgfqpoint{3.186416in}{2.102576in}}%
\pgfpathclose%
\pgfusepath{fill}%
\end{pgfscope}%
\begin{pgfscope}%
\pgfpathrectangle{\pgfqpoint{1.254980in}{0.150000in}}{\pgfqpoint{5.490039in}{5.490039in}}%
\pgfusepath{clip}%
\pgfsetbuttcap%
\pgfsetroundjoin%
\definecolor{currentfill}{rgb}{0.241237,0.296485,0.539709}%
\pgfsetfillcolor{currentfill}%
\pgfsetfillopacity{0.700000}%
\pgfsetlinewidth{0.000000pt}%
\definecolor{currentstroke}{rgb}{0.000000,0.000000,0.000000}%
\pgfsetstrokecolor{currentstroke}%
\pgfsetdash{}{0pt}%
\pgfpathmoveto{\pgfqpoint{4.508204in}{2.520303in}}%
\pgfpathlineto{\pgfqpoint{4.521544in}{2.524440in}}%
\pgfpathlineto{\pgfqpoint{4.534896in}{2.528739in}}%
\pgfpathlineto{\pgfqpoint{4.548259in}{2.533201in}}%
\pgfpathlineto{\pgfqpoint{4.561634in}{2.537824in}}%
\pgfpathlineto{\pgfqpoint{4.569073in}{2.545995in}}%
\pgfpathlineto{\pgfqpoint{4.576506in}{2.554135in}}%
\pgfpathlineto{\pgfqpoint{4.583934in}{2.562248in}}%
\pgfpathlineto{\pgfqpoint{4.591356in}{2.570337in}}%
\pgfpathlineto{\pgfqpoint{4.577990in}{2.565918in}}%
\pgfpathlineto{\pgfqpoint{4.564637in}{2.561661in}}%
\pgfpathlineto{\pgfqpoint{4.551295in}{2.557566in}}%
\pgfpathlineto{\pgfqpoint{4.537964in}{2.553633in}}%
\pgfpathlineto{\pgfqpoint{4.530532in}{2.545330in}}%
\pgfpathlineto{\pgfqpoint{4.523095in}{2.537009in}}%
\pgfpathlineto{\pgfqpoint{4.515652in}{2.528667in}}%
\pgfpathlineto{\pgfqpoint{4.508204in}{2.520303in}}%
\pgfpathclose%
\pgfusepath{fill}%
\end{pgfscope}%
\begin{pgfscope}%
\pgfpathrectangle{\pgfqpoint{1.254980in}{0.150000in}}{\pgfqpoint{5.490039in}{5.490039in}}%
\pgfusepath{clip}%
\pgfsetbuttcap%
\pgfsetroundjoin%
\definecolor{currentfill}{rgb}{0.283229,0.120777,0.440584}%
\pgfsetfillcolor{currentfill}%
\pgfsetfillopacity{0.700000}%
\pgfsetlinewidth{0.000000pt}%
\definecolor{currentstroke}{rgb}{0.000000,0.000000,0.000000}%
\pgfsetstrokecolor{currentstroke}%
\pgfsetdash{}{0pt}%
\pgfpathmoveto{\pgfqpoint{2.997664in}{2.189584in}}%
\pgfpathlineto{\pgfqpoint{3.010756in}{2.178514in}}%
\pgfpathlineto{\pgfqpoint{3.023845in}{2.167669in}}%
\pgfpathlineto{\pgfqpoint{3.036931in}{2.157049in}}%
\pgfpathlineto{\pgfqpoint{3.050016in}{2.146652in}}%
\pgfpathlineto{\pgfqpoint{3.058011in}{2.153888in}}%
\pgfpathlineto{\pgfqpoint{3.065999in}{2.161217in}}%
\pgfpathlineto{\pgfqpoint{3.073979in}{2.168636in}}%
\pgfpathlineto{\pgfqpoint{3.081951in}{2.176144in}}%
\pgfpathlineto{\pgfqpoint{3.068887in}{2.186320in}}%
\pgfpathlineto{\pgfqpoint{3.055821in}{2.196719in}}%
\pgfpathlineto{\pgfqpoint{3.042753in}{2.207343in}}%
\pgfpathlineto{\pgfqpoint{3.029683in}{2.218191in}}%
\pgfpathlineto{\pgfqpoint{3.021691in}{2.210894in}}%
\pgfpathlineto{\pgfqpoint{3.013690in}{2.203692in}}%
\pgfpathlineto{\pgfqpoint{3.005681in}{2.196588in}}%
\pgfpathlineto{\pgfqpoint{2.997664in}{2.189584in}}%
\pgfpathclose%
\pgfusepath{fill}%
\end{pgfscope}%
\begin{pgfscope}%
\pgfpathrectangle{\pgfqpoint{1.254980in}{0.150000in}}{\pgfqpoint{5.490039in}{5.490039in}}%
\pgfusepath{clip}%
\pgfsetbuttcap%
\pgfsetroundjoin%
\definecolor{currentfill}{rgb}{0.282656,0.100196,0.422160}%
\pgfsetfillcolor{currentfill}%
\pgfsetfillopacity{0.700000}%
\pgfsetlinewidth{0.000000pt}%
\definecolor{currentstroke}{rgb}{0.000000,0.000000,0.000000}%
\pgfsetstrokecolor{currentstroke}%
\pgfsetdash{}{0pt}%
\pgfpathmoveto{\pgfqpoint{3.760275in}{2.130700in}}%
\pgfpathlineto{\pgfqpoint{3.773365in}{2.129292in}}%
\pgfpathlineto{\pgfqpoint{3.786462in}{2.128063in}}%
\pgfpathlineto{\pgfqpoint{3.799566in}{2.127012in}}%
\pgfpathlineto{\pgfqpoint{3.812676in}{2.126138in}}%
\pgfpathlineto{\pgfqpoint{3.820373in}{2.135903in}}%
\pgfpathlineto{\pgfqpoint{3.828065in}{2.145658in}}%
\pgfpathlineto{\pgfqpoint{3.835752in}{2.155405in}}%
\pgfpathlineto{\pgfqpoint{3.843435in}{2.165143in}}%
\pgfpathlineto{\pgfqpoint{3.830333in}{2.165967in}}%
\pgfpathlineto{\pgfqpoint{3.817239in}{2.166968in}}%
\pgfpathlineto{\pgfqpoint{3.804151in}{2.168147in}}%
\pgfpathlineto{\pgfqpoint{3.791069in}{2.169504in}}%
\pgfpathlineto{\pgfqpoint{3.783378in}{2.159806in}}%
\pgfpathlineto{\pgfqpoint{3.775682in}{2.150106in}}%
\pgfpathlineto{\pgfqpoint{3.767981in}{2.140404in}}%
\pgfpathlineto{\pgfqpoint{3.760275in}{2.130700in}}%
\pgfpathclose%
\pgfusepath{fill}%
\end{pgfscope}%
\begin{pgfscope}%
\pgfpathrectangle{\pgfqpoint{1.254980in}{0.150000in}}{\pgfqpoint{5.490039in}{5.490039in}}%
\pgfusepath{clip}%
\pgfsetbuttcap%
\pgfsetroundjoin%
\definecolor{currentfill}{rgb}{0.278791,0.062145,0.386592}%
\pgfsetfillcolor{currentfill}%
\pgfsetfillopacity{0.700000}%
\pgfsetlinewidth{0.000000pt}%
\definecolor{currentstroke}{rgb}{0.000000,0.000000,0.000000}%
\pgfsetstrokecolor{currentstroke}%
\pgfsetdash{}{0pt}%
\pgfpathmoveto{\pgfqpoint{3.458133in}{2.066742in}}%
\pgfpathlineto{\pgfqpoint{3.471182in}{2.062058in}}%
\pgfpathlineto{\pgfqpoint{3.484235in}{2.057565in}}%
\pgfpathlineto{\pgfqpoint{3.497291in}{2.053263in}}%
\pgfpathlineto{\pgfqpoint{3.510351in}{2.049150in}}%
\pgfpathlineto{\pgfqpoint{3.518153in}{2.058407in}}%
\pgfpathlineto{\pgfqpoint{3.525949in}{2.067690in}}%
\pgfpathlineto{\pgfqpoint{3.533740in}{2.076998in}}%
\pgfpathlineto{\pgfqpoint{3.541525in}{2.086330in}}%
\pgfpathlineto{\pgfqpoint{3.528477in}{2.090310in}}%
\pgfpathlineto{\pgfqpoint{3.515433in}{2.094478in}}%
\pgfpathlineto{\pgfqpoint{3.502393in}{2.098837in}}%
\pgfpathlineto{\pgfqpoint{3.489356in}{2.103387in}}%
\pgfpathlineto{\pgfqpoint{3.481559in}{2.094178in}}%
\pgfpathlineto{\pgfqpoint{3.473756in}{2.085000in}}%
\pgfpathlineto{\pgfqpoint{3.465947in}{2.075854in}}%
\pgfpathlineto{\pgfqpoint{3.458133in}{2.066742in}}%
\pgfpathclose%
\pgfusepath{fill}%
\end{pgfscope}%
\begin{pgfscope}%
\pgfpathrectangle{\pgfqpoint{1.254980in}{0.150000in}}{\pgfqpoint{5.490039in}{5.490039in}}%
\pgfusepath{clip}%
\pgfsetbuttcap%
\pgfsetroundjoin%
\definecolor{currentfill}{rgb}{0.231674,0.318106,0.544834}%
\pgfsetfillcolor{currentfill}%
\pgfsetfillopacity{0.700000}%
\pgfsetlinewidth{0.000000pt}%
\definecolor{currentstroke}{rgb}{0.000000,0.000000,0.000000}%
\pgfsetstrokecolor{currentstroke}%
\pgfsetdash{}{0pt}%
\pgfpathmoveto{\pgfqpoint{4.591356in}{2.570337in}}%
\pgfpathlineto{\pgfqpoint{4.604733in}{2.574916in}}%
\pgfpathlineto{\pgfqpoint{4.618123in}{2.579657in}}%
\pgfpathlineto{\pgfqpoint{4.631524in}{2.584559in}}%
\pgfpathlineto{\pgfqpoint{4.644938in}{2.589622in}}%
\pgfpathlineto{\pgfqpoint{4.652345in}{2.597465in}}%
\pgfpathlineto{\pgfqpoint{4.659746in}{2.605282in}}%
\pgfpathlineto{\pgfqpoint{4.667141in}{2.613076in}}%
\pgfpathlineto{\pgfqpoint{4.674531in}{2.620848in}}%
\pgfpathlineto{\pgfqpoint{4.661127in}{2.616020in}}%
\pgfpathlineto{\pgfqpoint{4.647736in}{2.611351in}}%
\pgfpathlineto{\pgfqpoint{4.634357in}{2.606843in}}%
\pgfpathlineto{\pgfqpoint{4.620990in}{2.602496in}}%
\pgfpathlineto{\pgfqpoint{4.613589in}{2.594480in}}%
\pgfpathlineto{\pgfqpoint{4.606184in}{2.586450in}}%
\pgfpathlineto{\pgfqpoint{4.598772in}{2.578403in}}%
\pgfpathlineto{\pgfqpoint{4.591356in}{2.570337in}}%
\pgfpathclose%
\pgfusepath{fill}%
\end{pgfscope}%
\begin{pgfscope}%
\pgfpathrectangle{\pgfqpoint{1.254980in}{0.150000in}}{\pgfqpoint{5.490039in}{5.490039in}}%
\pgfusepath{clip}%
\pgfsetbuttcap%
\pgfsetroundjoin%
\definecolor{currentfill}{rgb}{0.119483,0.614817,0.537692}%
\pgfsetfillcolor{currentfill}%
\pgfsetfillopacity{0.700000}%
\pgfsetlinewidth{0.000000pt}%
\definecolor{currentstroke}{rgb}{0.000000,0.000000,0.000000}%
\pgfsetstrokecolor{currentstroke}%
\pgfsetdash{}{0pt}%
\pgfpathmoveto{\pgfqpoint{5.868028in}{3.334662in}}%
\pgfpathlineto{\pgfqpoint{5.881960in}{3.341295in}}%
\pgfpathlineto{\pgfqpoint{5.895909in}{3.348077in}}%
\pgfpathlineto{\pgfqpoint{5.909874in}{3.355007in}}%
\pgfpathlineto{\pgfqpoint{5.923857in}{3.362086in}}%
\pgfpathlineto{\pgfqpoint{5.930700in}{3.366902in}}%
\pgfpathlineto{\pgfqpoint{5.937545in}{3.371920in}}%
\pgfpathlineto{\pgfqpoint{5.944392in}{3.377149in}}%
\pgfpathlineto{\pgfqpoint{5.930436in}{3.370590in}}%
\pgfpathlineto{\pgfqpoint{5.916498in}{3.364178in}}%
\pgfpathlineto{\pgfqpoint{5.902575in}{3.357913in}}%
\pgfpathlineto{\pgfqpoint{5.888670in}{3.351797in}}%
\pgfpathlineto{\pgfqpoint{5.881787in}{3.345871in}}%
\pgfpathlineto{\pgfqpoint{5.874907in}{3.340162in}}%
\pgfpathlineto{\pgfqpoint{5.868028in}{3.334662in}}%
\pgfpathclose%
\pgfusepath{fill}%
\end{pgfscope}%
\begin{pgfscope}%
\pgfpathrectangle{\pgfqpoint{1.254980in}{0.150000in}}{\pgfqpoint{5.490039in}{5.490039in}}%
\pgfusepath{clip}%
\pgfsetbuttcap%
\pgfsetroundjoin%
\definecolor{currentfill}{rgb}{0.220057,0.343307,0.549413}%
\pgfsetfillcolor{currentfill}%
\pgfsetfillopacity{0.700000}%
\pgfsetlinewidth{0.000000pt}%
\definecolor{currentstroke}{rgb}{0.000000,0.000000,0.000000}%
\pgfsetstrokecolor{currentstroke}%
\pgfsetdash{}{0pt}%
\pgfpathmoveto{\pgfqpoint{4.674531in}{2.620848in}}%
\pgfpathlineto{\pgfqpoint{4.687947in}{2.625837in}}%
\pgfpathlineto{\pgfqpoint{4.701375in}{2.630986in}}%
\pgfpathlineto{\pgfqpoint{4.714816in}{2.636295in}}%
\pgfpathlineto{\pgfqpoint{4.728270in}{2.641764in}}%
\pgfpathlineto{\pgfqpoint{4.735643in}{2.649267in}}%
\pgfpathlineto{\pgfqpoint{4.743011in}{2.656749in}}%
\pgfpathlineto{\pgfqpoint{4.750373in}{2.664212in}}%
\pgfpathlineto{\pgfqpoint{4.757729in}{2.671660in}}%
\pgfpathlineto{\pgfqpoint{4.744286in}{2.666454in}}%
\pgfpathlineto{\pgfqpoint{4.730857in}{2.661408in}}%
\pgfpathlineto{\pgfqpoint{4.717440in}{2.656521in}}%
\pgfpathlineto{\pgfqpoint{4.704035in}{2.651793in}}%
\pgfpathlineto{\pgfqpoint{4.696667in}{2.644073in}}%
\pgfpathlineto{\pgfqpoint{4.689294in}{2.636344in}}%
\pgfpathlineto{\pgfqpoint{4.681915in}{2.628604in}}%
\pgfpathlineto{\pgfqpoint{4.674531in}{2.620848in}}%
\pgfpathclose%
\pgfusepath{fill}%
\end{pgfscope}%
\begin{pgfscope}%
\pgfpathrectangle{\pgfqpoint{1.254980in}{0.150000in}}{\pgfqpoint{5.490039in}{5.490039in}}%
\pgfusepath{clip}%
\pgfsetbuttcap%
\pgfsetroundjoin%
\definecolor{currentfill}{rgb}{0.281446,0.084320,0.407414}%
\pgfsetfillcolor{currentfill}%
\pgfsetfillopacity{0.700000}%
\pgfsetlinewidth{0.000000pt}%
\definecolor{currentstroke}{rgb}{0.000000,0.000000,0.000000}%
\pgfsetstrokecolor{currentstroke}%
\pgfsetdash{}{0pt}%
\pgfpathmoveto{\pgfqpoint{3.677055in}{2.099614in}}%
\pgfpathlineto{\pgfqpoint{3.690132in}{2.097408in}}%
\pgfpathlineto{\pgfqpoint{3.703216in}{2.095382in}}%
\pgfpathlineto{\pgfqpoint{3.716304in}{2.093537in}}%
\pgfpathlineto{\pgfqpoint{3.729398in}{2.091873in}}%
\pgfpathlineto{\pgfqpoint{3.737125in}{2.101581in}}%
\pgfpathlineto{\pgfqpoint{3.744847in}{2.111288in}}%
\pgfpathlineto{\pgfqpoint{3.752563in}{2.120995in}}%
\pgfpathlineto{\pgfqpoint{3.760275in}{2.130700in}}%
\pgfpathlineto{\pgfqpoint{3.747190in}{2.132287in}}%
\pgfpathlineto{\pgfqpoint{3.734111in}{2.134054in}}%
\pgfpathlineto{\pgfqpoint{3.721037in}{2.136001in}}%
\pgfpathlineto{\pgfqpoint{3.707969in}{2.138130in}}%
\pgfpathlineto{\pgfqpoint{3.700248in}{2.128492in}}%
\pgfpathlineto{\pgfqpoint{3.692522in}{2.118860in}}%
\pgfpathlineto{\pgfqpoint{3.684791in}{2.109234in}}%
\pgfpathlineto{\pgfqpoint{3.677055in}{2.099614in}}%
\pgfpathclose%
\pgfusepath{fill}%
\end{pgfscope}%
\begin{pgfscope}%
\pgfpathrectangle{\pgfqpoint{1.254980in}{0.150000in}}{\pgfqpoint{5.490039in}{5.490039in}}%
\pgfusepath{clip}%
\pgfsetbuttcap%
\pgfsetroundjoin%
\definecolor{currentfill}{rgb}{0.210503,0.363727,0.552206}%
\pgfsetfillcolor{currentfill}%
\pgfsetfillopacity{0.700000}%
\pgfsetlinewidth{0.000000pt}%
\definecolor{currentstroke}{rgb}{0.000000,0.000000,0.000000}%
\pgfsetstrokecolor{currentstroke}%
\pgfsetdash{}{0pt}%
\pgfpathmoveto{\pgfqpoint{4.757729in}{2.671660in}}%
\pgfpathlineto{\pgfqpoint{4.771184in}{2.677025in}}%
\pgfpathlineto{\pgfqpoint{4.784653in}{2.682549in}}%
\pgfpathlineto{\pgfqpoint{4.798134in}{2.688232in}}%
\pgfpathlineto{\pgfqpoint{4.811629in}{2.694073in}}%
\pgfpathlineto{\pgfqpoint{4.818967in}{2.701228in}}%
\pgfpathlineto{\pgfqpoint{4.826300in}{2.708367in}}%
\pgfpathlineto{\pgfqpoint{4.833628in}{2.715494in}}%
\pgfpathlineto{\pgfqpoint{4.840949in}{2.722612in}}%
\pgfpathlineto{\pgfqpoint{4.827467in}{2.717063in}}%
\pgfpathlineto{\pgfqpoint{4.813998in}{2.711671in}}%
\pgfpathlineto{\pgfqpoint{4.800542in}{2.706439in}}%
\pgfpathlineto{\pgfqpoint{4.787099in}{2.701364in}}%
\pgfpathlineto{\pgfqpoint{4.779765in}{2.693945in}}%
\pgfpathlineto{\pgfqpoint{4.772425in}{2.686523in}}%
\pgfpathlineto{\pgfqpoint{4.765080in}{2.679096in}}%
\pgfpathlineto{\pgfqpoint{4.757729in}{2.671660in}}%
\pgfpathclose%
\pgfusepath{fill}%
\end{pgfscope}%
\begin{pgfscope}%
\pgfpathrectangle{\pgfqpoint{1.254980in}{0.150000in}}{\pgfqpoint{5.490039in}{5.490039in}}%
\pgfusepath{clip}%
\pgfsetbuttcap%
\pgfsetroundjoin%
\definecolor{currentfill}{rgb}{0.282656,0.100196,0.422160}%
\pgfsetfillcolor{currentfill}%
\pgfsetfillopacity{0.700000}%
\pgfsetlinewidth{0.000000pt}%
\definecolor{currentstroke}{rgb}{0.000000,0.000000,0.000000}%
\pgfsetstrokecolor{currentstroke}%
\pgfsetdash{}{0pt}%
\pgfpathmoveto{\pgfqpoint{3.050016in}{2.146652in}}%
\pgfpathlineto{\pgfqpoint{3.063098in}{2.136477in}}%
\pgfpathlineto{\pgfqpoint{3.076179in}{2.126521in}}%
\pgfpathlineto{\pgfqpoint{3.089258in}{2.116784in}}%
\pgfpathlineto{\pgfqpoint{3.102336in}{2.107264in}}%
\pgfpathlineto{\pgfqpoint{3.110311in}{2.114730in}}%
\pgfpathlineto{\pgfqpoint{3.118279in}{2.122283in}}%
\pgfpathlineto{\pgfqpoint{3.126239in}{2.129919in}}%
\pgfpathlineto{\pgfqpoint{3.134191in}{2.137637in}}%
\pgfpathlineto{\pgfqpoint{3.121133in}{2.146937in}}%
\pgfpathlineto{\pgfqpoint{3.108074in}{2.156454in}}%
\pgfpathlineto{\pgfqpoint{3.095013in}{2.166189in}}%
\pgfpathlineto{\pgfqpoint{3.081951in}{2.176144in}}%
\pgfpathlineto{\pgfqpoint{3.073979in}{2.168636in}}%
\pgfpathlineto{\pgfqpoint{3.065999in}{2.161217in}}%
\pgfpathlineto{\pgfqpoint{3.058011in}{2.153888in}}%
\pgfpathlineto{\pgfqpoint{3.050016in}{2.146652in}}%
\pgfpathclose%
\pgfusepath{fill}%
\end{pgfscope}%
\begin{pgfscope}%
\pgfpathrectangle{\pgfqpoint{1.254980in}{0.150000in}}{\pgfqpoint{5.490039in}{5.490039in}}%
\pgfusepath{clip}%
\pgfsetbuttcap%
\pgfsetroundjoin%
\definecolor{currentfill}{rgb}{0.199430,0.387607,0.554642}%
\pgfsetfillcolor{currentfill}%
\pgfsetfillopacity{0.700000}%
\pgfsetlinewidth{0.000000pt}%
\definecolor{currentstroke}{rgb}{0.000000,0.000000,0.000000}%
\pgfsetstrokecolor{currentstroke}%
\pgfsetdash{}{0pt}%
\pgfpathmoveto{\pgfqpoint{4.840949in}{2.722612in}}%
\pgfpathlineto{\pgfqpoint{4.854445in}{2.728320in}}%
\pgfpathlineto{\pgfqpoint{4.867954in}{2.734186in}}%
\pgfpathlineto{\pgfqpoint{4.881476in}{2.740210in}}%
\pgfpathlineto{\pgfqpoint{4.895012in}{2.746392in}}%
\pgfpathlineto{\pgfqpoint{4.902315in}{2.753195in}}%
\pgfpathlineto{\pgfqpoint{4.909613in}{2.759989in}}%
\pgfpathlineto{\pgfqpoint{4.916904in}{2.766779in}}%
\pgfpathlineto{\pgfqpoint{4.924191in}{2.773567in}}%
\pgfpathlineto{\pgfqpoint{4.910668in}{2.767707in}}%
\pgfpathlineto{\pgfqpoint{4.897160in}{2.762003in}}%
\pgfpathlineto{\pgfqpoint{4.883664in}{2.756458in}}%
\pgfpathlineto{\pgfqpoint{4.870182in}{2.751070in}}%
\pgfpathlineto{\pgfqpoint{4.862882in}{2.743950in}}%
\pgfpathlineto{\pgfqpoint{4.855577in}{2.736837in}}%
\pgfpathlineto{\pgfqpoint{4.848266in}{2.729725in}}%
\pgfpathlineto{\pgfqpoint{4.840949in}{2.722612in}}%
\pgfpathclose%
\pgfusepath{fill}%
\end{pgfscope}%
\begin{pgfscope}%
\pgfpathrectangle{\pgfqpoint{1.254980in}{0.150000in}}{\pgfqpoint{5.490039in}{5.490039in}}%
\pgfusepath{clip}%
\pgfsetbuttcap%
\pgfsetroundjoin%
\definecolor{currentfill}{rgb}{0.279566,0.067836,0.391917}%
\pgfsetfillcolor{currentfill}%
\pgfsetfillopacity{0.700000}%
\pgfsetlinewidth{0.000000pt}%
\definecolor{currentstroke}{rgb}{0.000000,0.000000,0.000000}%
\pgfsetstrokecolor{currentstroke}%
\pgfsetdash{}{0pt}%
\pgfpathmoveto{\pgfqpoint{3.238639in}{2.070877in}}%
\pgfpathlineto{\pgfqpoint{3.251695in}{2.063468in}}%
\pgfpathlineto{\pgfqpoint{3.264753in}{2.056263in}}%
\pgfpathlineto{\pgfqpoint{3.277811in}{2.049261in}}%
\pgfpathlineto{\pgfqpoint{3.290870in}{2.042460in}}%
\pgfpathlineto{\pgfqpoint{3.298763in}{2.050862in}}%
\pgfpathlineto{\pgfqpoint{3.306649in}{2.059322in}}%
\pgfpathlineto{\pgfqpoint{3.314529in}{2.067838in}}%
\pgfpathlineto{\pgfqpoint{3.322402in}{2.076409in}}%
\pgfpathlineto{\pgfqpoint{3.309358in}{2.083020in}}%
\pgfpathlineto{\pgfqpoint{3.296316in}{2.089832in}}%
\pgfpathlineto{\pgfqpoint{3.283275in}{2.096847in}}%
\pgfpathlineto{\pgfqpoint{3.270235in}{2.104065in}}%
\pgfpathlineto{\pgfqpoint{3.262346in}{2.095675in}}%
\pgfpathlineto{\pgfqpoint{3.254451in}{2.087345in}}%
\pgfpathlineto{\pgfqpoint{3.246548in}{2.079079in}}%
\pgfpathlineto{\pgfqpoint{3.238639in}{2.070877in}}%
\pgfpathclose%
\pgfusepath{fill}%
\end{pgfscope}%
\begin{pgfscope}%
\pgfpathrectangle{\pgfqpoint{1.254980in}{0.150000in}}{\pgfqpoint{5.490039in}{5.490039in}}%
\pgfusepath{clip}%
\pgfsetbuttcap%
\pgfsetroundjoin%
\definecolor{currentfill}{rgb}{0.280267,0.073417,0.397163}%
\pgfsetfillcolor{currentfill}%
\pgfsetfillopacity{0.700000}%
\pgfsetlinewidth{0.000000pt}%
\definecolor{currentstroke}{rgb}{0.000000,0.000000,0.000000}%
\pgfsetstrokecolor{currentstroke}%
\pgfsetdash{}{0pt}%
\pgfpathmoveto{\pgfqpoint{3.593755in}{2.072289in}}%
\pgfpathlineto{\pgfqpoint{3.606823in}{2.069244in}}%
\pgfpathlineto{\pgfqpoint{3.619896in}{2.066383in}}%
\pgfpathlineto{\pgfqpoint{3.632974in}{2.063706in}}%
\pgfpathlineto{\pgfqpoint{3.646057in}{2.061211in}}%
\pgfpathlineto{\pgfqpoint{3.653814in}{2.070800in}}%
\pgfpathlineto{\pgfqpoint{3.661566in}{2.080397in}}%
\pgfpathlineto{\pgfqpoint{3.669313in}{2.090002in}}%
\pgfpathlineto{\pgfqpoint{3.677055in}{2.099614in}}%
\pgfpathlineto{\pgfqpoint{3.663982in}{2.102003in}}%
\pgfpathlineto{\pgfqpoint{3.650915in}{2.104575in}}%
\pgfpathlineto{\pgfqpoint{3.637853in}{2.107330in}}%
\pgfpathlineto{\pgfqpoint{3.624795in}{2.110270in}}%
\pgfpathlineto{\pgfqpoint{3.617043in}{2.100753in}}%
\pgfpathlineto{\pgfqpoint{3.609286in}{2.091250in}}%
\pgfpathlineto{\pgfqpoint{3.601523in}{2.081762in}}%
\pgfpathlineto{\pgfqpoint{3.593755in}{2.072289in}}%
\pgfpathclose%
\pgfusepath{fill}%
\end{pgfscope}%
\begin{pgfscope}%
\pgfpathrectangle{\pgfqpoint{1.254980in}{0.150000in}}{\pgfqpoint{5.490039in}{5.490039in}}%
\pgfusepath{clip}%
\pgfsetbuttcap%
\pgfsetroundjoin%
\definecolor{currentfill}{rgb}{0.190631,0.407061,0.556089}%
\pgfsetfillcolor{currentfill}%
\pgfsetfillopacity{0.700000}%
\pgfsetlinewidth{0.000000pt}%
\definecolor{currentstroke}{rgb}{0.000000,0.000000,0.000000}%
\pgfsetstrokecolor{currentstroke}%
\pgfsetdash{}{0pt}%
\pgfpathmoveto{\pgfqpoint{4.924191in}{2.773567in}}%
\pgfpathlineto{\pgfqpoint{4.937727in}{2.779585in}}%
\pgfpathlineto{\pgfqpoint{4.951277in}{2.785760in}}%
\pgfpathlineto{\pgfqpoint{4.964841in}{2.792092in}}%
\pgfpathlineto{\pgfqpoint{4.978419in}{2.798582in}}%
\pgfpathlineto{\pgfqpoint{4.985685in}{2.805034in}}%
\pgfpathlineto{\pgfqpoint{4.992945in}{2.811486in}}%
\pgfpathlineto{\pgfqpoint{5.000201in}{2.817942in}}%
\pgfpathlineto{\pgfqpoint{5.007450in}{2.824406in}}%
\pgfpathlineto{\pgfqpoint{4.993887in}{2.818267in}}%
\pgfpathlineto{\pgfqpoint{4.980339in}{2.812284in}}%
\pgfpathlineto{\pgfqpoint{4.966803in}{2.806459in}}%
\pgfpathlineto{\pgfqpoint{4.953282in}{2.800790in}}%
\pgfpathlineto{\pgfqpoint{4.946017in}{2.793966in}}%
\pgfpathlineto{\pgfqpoint{4.938747in}{2.787157in}}%
\pgfpathlineto{\pgfqpoint{4.931471in}{2.780359in}}%
\pgfpathlineto{\pgfqpoint{4.924191in}{2.773567in}}%
\pgfpathclose%
\pgfusepath{fill}%
\end{pgfscope}%
\begin{pgfscope}%
\pgfpathrectangle{\pgfqpoint{1.254980in}{0.150000in}}{\pgfqpoint{5.490039in}{5.490039in}}%
\pgfusepath{clip}%
\pgfsetbuttcap%
\pgfsetroundjoin%
\definecolor{currentfill}{rgb}{0.277941,0.056324,0.381191}%
\pgfsetfillcolor{currentfill}%
\pgfsetfillopacity{0.700000}%
\pgfsetlinewidth{0.000000pt}%
\definecolor{currentstroke}{rgb}{0.000000,0.000000,0.000000}%
\pgfsetstrokecolor{currentstroke}%
\pgfsetdash{}{0pt}%
\pgfpathmoveto{\pgfqpoint{3.374592in}{2.051958in}}%
\pgfpathlineto{\pgfqpoint{3.387644in}{2.046339in}}%
\pgfpathlineto{\pgfqpoint{3.400699in}{2.040914in}}%
\pgfpathlineto{\pgfqpoint{3.413757in}{2.035683in}}%
\pgfpathlineto{\pgfqpoint{3.426817in}{2.030646in}}%
\pgfpathlineto{\pgfqpoint{3.434655in}{2.039614in}}%
\pgfpathlineto{\pgfqpoint{3.442487in}{2.048621in}}%
\pgfpathlineto{\pgfqpoint{3.450313in}{2.057664in}}%
\pgfpathlineto{\pgfqpoint{3.458133in}{2.066742in}}%
\pgfpathlineto{\pgfqpoint{3.445086in}{2.071618in}}%
\pgfpathlineto{\pgfqpoint{3.432043in}{2.076687in}}%
\pgfpathlineto{\pgfqpoint{3.419002in}{2.081950in}}%
\pgfpathlineto{\pgfqpoint{3.405963in}{2.087408in}}%
\pgfpathlineto{\pgfqpoint{3.398129in}{2.078481in}}%
\pgfpathlineto{\pgfqpoint{3.390290in}{2.069596in}}%
\pgfpathlineto{\pgfqpoint{3.382444in}{2.060755in}}%
\pgfpathlineto{\pgfqpoint{3.374592in}{2.051958in}}%
\pgfpathclose%
\pgfusepath{fill}%
\end{pgfscope}%
\begin{pgfscope}%
\pgfpathrectangle{\pgfqpoint{1.254980in}{0.150000in}}{\pgfqpoint{5.490039in}{5.490039in}}%
\pgfusepath{clip}%
\pgfsetbuttcap%
\pgfsetroundjoin%
\definecolor{currentfill}{rgb}{0.180629,0.429975,0.557282}%
\pgfsetfillcolor{currentfill}%
\pgfsetfillopacity{0.700000}%
\pgfsetlinewidth{0.000000pt}%
\definecolor{currentstroke}{rgb}{0.000000,0.000000,0.000000}%
\pgfsetstrokecolor{currentstroke}%
\pgfsetdash{}{0pt}%
\pgfpathmoveto{\pgfqpoint{5.007450in}{2.824406in}}%
\pgfpathlineto{\pgfqpoint{5.021027in}{2.830701in}}%
\pgfpathlineto{\pgfqpoint{5.034619in}{2.837152in}}%
\pgfpathlineto{\pgfqpoint{5.048224in}{2.843760in}}%
\pgfpathlineto{\pgfqpoint{5.061844in}{2.850524in}}%
\pgfpathlineto{\pgfqpoint{5.069073in}{2.856632in}}%
\pgfpathlineto{\pgfqpoint{5.076296in}{2.862749in}}%
\pgfpathlineto{\pgfqpoint{5.083513in}{2.868879in}}%
\pgfpathlineto{\pgfqpoint{5.090726in}{2.875028in}}%
\pgfpathlineto{\pgfqpoint{5.077122in}{2.868644in}}%
\pgfpathlineto{\pgfqpoint{5.063533in}{2.862415in}}%
\pgfpathlineto{\pgfqpoint{5.049958in}{2.856342in}}%
\pgfpathlineto{\pgfqpoint{5.036397in}{2.850425in}}%
\pgfpathlineto{\pgfqpoint{5.029168in}{2.843887in}}%
\pgfpathlineto{\pgfqpoint{5.021934in}{2.837374in}}%
\pgfpathlineto{\pgfqpoint{5.014695in}{2.830882in}}%
\pgfpathlineto{\pgfqpoint{5.007450in}{2.824406in}}%
\pgfpathclose%
\pgfusepath{fill}%
\end{pgfscope}%
\begin{pgfscope}%
\pgfpathrectangle{\pgfqpoint{1.254980in}{0.150000in}}{\pgfqpoint{5.490039in}{5.490039in}}%
\pgfusepath{clip}%
\pgfsetbuttcap%
\pgfsetroundjoin%
\definecolor{currentfill}{rgb}{0.262138,0.242286,0.520837}%
\pgfsetfillcolor{currentfill}%
\pgfsetfillopacity{0.700000}%
\pgfsetlinewidth{0.000000pt}%
\definecolor{currentstroke}{rgb}{0.000000,0.000000,0.000000}%
\pgfsetstrokecolor{currentstroke}%
\pgfsetdash{}{0pt}%
\pgfpathmoveto{\pgfqpoint{2.702285in}{2.439281in}}%
\pgfpathlineto{\pgfqpoint{2.715503in}{2.423050in}}%
\pgfpathlineto{\pgfqpoint{2.728714in}{2.407084in}}%
\pgfpathlineto{\pgfqpoint{2.741918in}{2.391381in}}%
\pgfpathlineto{\pgfqpoint{2.755115in}{2.375938in}}%
\pgfpathlineto{\pgfqpoint{2.763269in}{2.381469in}}%
\pgfpathlineto{\pgfqpoint{2.771413in}{2.387136in}}%
\pgfpathlineto{\pgfqpoint{2.779546in}{2.392938in}}%
\pgfpathlineto{\pgfqpoint{2.787669in}{2.398872in}}%
\pgfpathlineto{\pgfqpoint{2.774500in}{2.414058in}}%
\pgfpathlineto{\pgfqpoint{2.761324in}{2.429505in}}%
\pgfpathlineto{\pgfqpoint{2.748141in}{2.445214in}}%
\pgfpathlineto{\pgfqpoint{2.734951in}{2.461187in}}%
\pgfpathlineto{\pgfqpoint{2.726800in}{2.455499in}}%
\pgfpathlineto{\pgfqpoint{2.718639in}{2.449950in}}%
\pgfpathlineto{\pgfqpoint{2.710467in}{2.444544in}}%
\pgfpathlineto{\pgfqpoint{2.702285in}{2.439281in}}%
\pgfpathclose%
\pgfusepath{fill}%
\end{pgfscope}%
\begin{pgfscope}%
\pgfpathrectangle{\pgfqpoint{1.254980in}{0.150000in}}{\pgfqpoint{5.490039in}{5.490039in}}%
\pgfusepath{clip}%
\pgfsetbuttcap%
\pgfsetroundjoin%
\definecolor{currentfill}{rgb}{0.270595,0.214069,0.507052}%
\pgfsetfillcolor{currentfill}%
\pgfsetfillopacity{0.700000}%
\pgfsetlinewidth{0.000000pt}%
\definecolor{currentstroke}{rgb}{0.000000,0.000000,0.000000}%
\pgfsetstrokecolor{currentstroke}%
\pgfsetdash{}{0pt}%
\pgfpathmoveto{\pgfqpoint{2.755115in}{2.375938in}}%
\pgfpathlineto{\pgfqpoint{2.768305in}{2.360754in}}%
\pgfpathlineto{\pgfqpoint{2.781489in}{2.345826in}}%
\pgfpathlineto{\pgfqpoint{2.794666in}{2.331152in}}%
\pgfpathlineto{\pgfqpoint{2.807837in}{2.316730in}}%
\pgfpathlineto{\pgfqpoint{2.815964in}{2.322526in}}%
\pgfpathlineto{\pgfqpoint{2.824081in}{2.328452in}}%
\pgfpathlineto{\pgfqpoint{2.832187in}{2.334505in}}%
\pgfpathlineto{\pgfqpoint{2.840285in}{2.340683in}}%
\pgfpathlineto{\pgfqpoint{2.827140in}{2.354851in}}%
\pgfpathlineto{\pgfqpoint{2.813989in}{2.369270in}}%
\pgfpathlineto{\pgfqpoint{2.800832in}{2.383943in}}%
\pgfpathlineto{\pgfqpoint{2.787669in}{2.398872in}}%
\pgfpathlineto{\pgfqpoint{2.779546in}{2.392938in}}%
\pgfpathlineto{\pgfqpoint{2.771413in}{2.387136in}}%
\pgfpathlineto{\pgfqpoint{2.763269in}{2.381469in}}%
\pgfpathlineto{\pgfqpoint{2.755115in}{2.375938in}}%
\pgfpathclose%
\pgfusepath{fill}%
\end{pgfscope}%
\begin{pgfscope}%
\pgfpathrectangle{\pgfqpoint{1.254980in}{0.150000in}}{\pgfqpoint{5.490039in}{5.490039in}}%
\pgfusepath{clip}%
\pgfsetbuttcap%
\pgfsetroundjoin%
\definecolor{currentfill}{rgb}{0.250425,0.274290,0.533103}%
\pgfsetfillcolor{currentfill}%
\pgfsetfillopacity{0.700000}%
\pgfsetlinewidth{0.000000pt}%
\definecolor{currentstroke}{rgb}{0.000000,0.000000,0.000000}%
\pgfsetstrokecolor{currentstroke}%
\pgfsetdash{}{0pt}%
\pgfpathmoveto{\pgfqpoint{2.649330in}{2.506903in}}%
\pgfpathlineto{\pgfqpoint{2.662581in}{2.489588in}}%
\pgfpathlineto{\pgfqpoint{2.675824in}{2.472547in}}%
\pgfpathlineto{\pgfqpoint{2.689058in}{2.455779in}}%
\pgfpathlineto{\pgfqpoint{2.702285in}{2.439281in}}%
\pgfpathlineto{\pgfqpoint{2.710467in}{2.444544in}}%
\pgfpathlineto{\pgfqpoint{2.718639in}{2.449950in}}%
\pgfpathlineto{\pgfqpoint{2.726800in}{2.455499in}}%
\pgfpathlineto{\pgfqpoint{2.734951in}{2.461187in}}%
\pgfpathlineto{\pgfqpoint{2.721753in}{2.477428in}}%
\pgfpathlineto{\pgfqpoint{2.708548in}{2.493938in}}%
\pgfpathlineto{\pgfqpoint{2.695334in}{2.510719in}}%
\pgfpathlineto{\pgfqpoint{2.682112in}{2.527775in}}%
\pgfpathlineto{\pgfqpoint{2.673933in}{2.522334in}}%
\pgfpathlineto{\pgfqpoint{2.665743in}{2.517041in}}%
\pgfpathlineto{\pgfqpoint{2.657542in}{2.511896in}}%
\pgfpathlineto{\pgfqpoint{2.649330in}{2.506903in}}%
\pgfpathclose%
\pgfusepath{fill}%
\end{pgfscope}%
\begin{pgfscope}%
\pgfpathrectangle{\pgfqpoint{1.254980in}{0.150000in}}{\pgfqpoint{5.490039in}{5.490039in}}%
\pgfusepath{clip}%
\pgfsetbuttcap%
\pgfsetroundjoin%
\definecolor{currentfill}{rgb}{0.277134,0.185228,0.489898}%
\pgfsetfillcolor{currentfill}%
\pgfsetfillopacity{0.700000}%
\pgfsetlinewidth{0.000000pt}%
\definecolor{currentstroke}{rgb}{0.000000,0.000000,0.000000}%
\pgfsetstrokecolor{currentstroke}%
\pgfsetdash{}{0pt}%
\pgfpathmoveto{\pgfqpoint{2.807837in}{2.316730in}}%
\pgfpathlineto{\pgfqpoint{2.821003in}{2.302558in}}%
\pgfpathlineto{\pgfqpoint{2.834163in}{2.288634in}}%
\pgfpathlineto{\pgfqpoint{2.847317in}{2.274956in}}%
\pgfpathlineto{\pgfqpoint{2.860467in}{2.261521in}}%
\pgfpathlineto{\pgfqpoint{2.868568in}{2.267581in}}%
\pgfpathlineto{\pgfqpoint{2.876659in}{2.273764in}}%
\pgfpathlineto{\pgfqpoint{2.884740in}{2.280067in}}%
\pgfpathlineto{\pgfqpoint{2.892813in}{2.286488in}}%
\pgfpathlineto{\pgfqpoint{2.879688in}{2.299669in}}%
\pgfpathlineto{\pgfqpoint{2.866559in}{2.313095in}}%
\pgfpathlineto{\pgfqpoint{2.853424in}{2.326765in}}%
\pgfpathlineto{\pgfqpoint{2.840285in}{2.340683in}}%
\pgfpathlineto{\pgfqpoint{2.832187in}{2.334505in}}%
\pgfpathlineto{\pgfqpoint{2.824081in}{2.328452in}}%
\pgfpathlineto{\pgfqpoint{2.815964in}{2.322526in}}%
\pgfpathlineto{\pgfqpoint{2.807837in}{2.316730in}}%
\pgfpathclose%
\pgfusepath{fill}%
\end{pgfscope}%
\begin{pgfscope}%
\pgfpathrectangle{\pgfqpoint{1.254980in}{0.150000in}}{\pgfqpoint{5.490039in}{5.490039in}}%
\pgfusepath{clip}%
\pgfsetbuttcap%
\pgfsetroundjoin%
\definecolor{currentfill}{rgb}{0.172719,0.448791,0.557885}%
\pgfsetfillcolor{currentfill}%
\pgfsetfillopacity{0.700000}%
\pgfsetlinewidth{0.000000pt}%
\definecolor{currentstroke}{rgb}{0.000000,0.000000,0.000000}%
\pgfsetstrokecolor{currentstroke}%
\pgfsetdash{}{0pt}%
\pgfpathmoveto{\pgfqpoint{5.090726in}{2.875028in}}%
\pgfpathlineto{\pgfqpoint{5.104344in}{2.881568in}}%
\pgfpathlineto{\pgfqpoint{5.117977in}{2.888264in}}%
\pgfpathlineto{\pgfqpoint{5.131624in}{2.895115in}}%
\pgfpathlineto{\pgfqpoint{5.145286in}{2.902121in}}%
\pgfpathlineto{\pgfqpoint{5.152476in}{2.907895in}}%
\pgfpathlineto{\pgfqpoint{5.159661in}{2.913689in}}%
\pgfpathlineto{\pgfqpoint{5.166840in}{2.919508in}}%
\pgfpathlineto{\pgfqpoint{5.174015in}{2.925357in}}%
\pgfpathlineto{\pgfqpoint{5.160370in}{2.918759in}}%
\pgfpathlineto{\pgfqpoint{5.146741in}{2.912316in}}%
\pgfpathlineto{\pgfqpoint{5.133126in}{2.906029in}}%
\pgfpathlineto{\pgfqpoint{5.119526in}{2.899896in}}%
\pgfpathlineto{\pgfqpoint{5.112333in}{2.893629in}}%
\pgfpathlineto{\pgfqpoint{5.105136in}{2.887399in}}%
\pgfpathlineto{\pgfqpoint{5.097933in}{2.881200in}}%
\pgfpathlineto{\pgfqpoint{5.090726in}{2.875028in}}%
\pgfpathclose%
\pgfusepath{fill}%
\end{pgfscope}%
\begin{pgfscope}%
\pgfpathrectangle{\pgfqpoint{1.254980in}{0.150000in}}{\pgfqpoint{5.490039in}{5.490039in}}%
\pgfusepath{clip}%
\pgfsetbuttcap%
\pgfsetroundjoin%
\definecolor{currentfill}{rgb}{0.281446,0.084320,0.407414}%
\pgfsetfillcolor{currentfill}%
\pgfsetfillopacity{0.700000}%
\pgfsetlinewidth{0.000000pt}%
\definecolor{currentstroke}{rgb}{0.000000,0.000000,0.000000}%
\pgfsetstrokecolor{currentstroke}%
\pgfsetdash{}{0pt}%
\pgfpathmoveto{\pgfqpoint{3.102336in}{2.107264in}}%
\pgfpathlineto{\pgfqpoint{3.115412in}{2.097959in}}%
\pgfpathlineto{\pgfqpoint{3.128488in}{2.088869in}}%
\pgfpathlineto{\pgfqpoint{3.141563in}{2.079992in}}%
\pgfpathlineto{\pgfqpoint{3.154637in}{2.071326in}}%
\pgfpathlineto{\pgfqpoint{3.162593in}{2.079022in}}%
\pgfpathlineto{\pgfqpoint{3.170541in}{2.086797in}}%
\pgfpathlineto{\pgfqpoint{3.178483in}{2.094649in}}%
\pgfpathlineto{\pgfqpoint{3.186416in}{2.102576in}}%
\pgfpathlineto{\pgfqpoint{3.173361in}{2.111023in}}%
\pgfpathlineto{\pgfqpoint{3.160305in}{2.119681in}}%
\pgfpathlineto{\pgfqpoint{3.147249in}{2.128552in}}%
\pgfpathlineto{\pgfqpoint{3.134191in}{2.137637in}}%
\pgfpathlineto{\pgfqpoint{3.126239in}{2.129919in}}%
\pgfpathlineto{\pgfqpoint{3.118279in}{2.122283in}}%
\pgfpathlineto{\pgfqpoint{3.110311in}{2.114730in}}%
\pgfpathlineto{\pgfqpoint{3.102336in}{2.107264in}}%
\pgfpathclose%
\pgfusepath{fill}%
\end{pgfscope}%
\begin{pgfscope}%
\pgfpathrectangle{\pgfqpoint{1.254980in}{0.150000in}}{\pgfqpoint{5.490039in}{5.490039in}}%
\pgfusepath{clip}%
\pgfsetbuttcap%
\pgfsetroundjoin%
\definecolor{currentfill}{rgb}{0.165117,0.467423,0.558141}%
\pgfsetfillcolor{currentfill}%
\pgfsetfillopacity{0.700000}%
\pgfsetlinewidth{0.000000pt}%
\definecolor{currentstroke}{rgb}{0.000000,0.000000,0.000000}%
\pgfsetstrokecolor{currentstroke}%
\pgfsetdash{}{0pt}%
\pgfpathmoveto{\pgfqpoint{5.174015in}{2.925357in}}%
\pgfpathlineto{\pgfqpoint{5.187674in}{2.932109in}}%
\pgfpathlineto{\pgfqpoint{5.201348in}{2.939016in}}%
\pgfpathlineto{\pgfqpoint{5.215037in}{2.946078in}}%
\pgfpathlineto{\pgfqpoint{5.228741in}{2.953295in}}%
\pgfpathlineto{\pgfqpoint{5.235892in}{2.958750in}}%
\pgfpathlineto{\pgfqpoint{5.243037in}{2.964239in}}%
\pgfpathlineto{\pgfqpoint{5.250178in}{2.969764in}}%
\pgfpathlineto{\pgfqpoint{5.257314in}{2.975332in}}%
\pgfpathlineto{\pgfqpoint{5.243629in}{2.968553in}}%
\pgfpathlineto{\pgfqpoint{5.229959in}{2.961929in}}%
\pgfpathlineto{\pgfqpoint{5.216305in}{2.955459in}}%
\pgfpathlineto{\pgfqpoint{5.202665in}{2.949143in}}%
\pgfpathlineto{\pgfqpoint{5.195509in}{2.943128in}}%
\pgfpathlineto{\pgfqpoint{5.188349in}{2.937162in}}%
\pgfpathlineto{\pgfqpoint{5.181184in}{2.931240in}}%
\pgfpathlineto{\pgfqpoint{5.174015in}{2.925357in}}%
\pgfpathclose%
\pgfusepath{fill}%
\end{pgfscope}%
\begin{pgfscope}%
\pgfpathrectangle{\pgfqpoint{1.254980in}{0.150000in}}{\pgfqpoint{5.490039in}{5.490039in}}%
\pgfusepath{clip}%
\pgfsetbuttcap%
\pgfsetroundjoin%
\definecolor{currentfill}{rgb}{0.278791,0.062145,0.386592}%
\pgfsetfillcolor{currentfill}%
\pgfsetfillopacity{0.700000}%
\pgfsetlinewidth{0.000000pt}%
\definecolor{currentstroke}{rgb}{0.000000,0.000000,0.000000}%
\pgfsetstrokecolor{currentstroke}%
\pgfsetdash{}{0pt}%
\pgfpathmoveto{\pgfqpoint{3.510351in}{2.049150in}}%
\pgfpathlineto{\pgfqpoint{3.523414in}{2.045226in}}%
\pgfpathlineto{\pgfqpoint{3.536481in}{2.041489in}}%
\pgfpathlineto{\pgfqpoint{3.549553in}{2.037939in}}%
\pgfpathlineto{\pgfqpoint{3.562628in}{2.034575in}}%
\pgfpathlineto{\pgfqpoint{3.570418in}{2.043976in}}%
\pgfpathlineto{\pgfqpoint{3.578202in}{2.053396in}}%
\pgfpathlineto{\pgfqpoint{3.585981in}{2.062834in}}%
\pgfpathlineto{\pgfqpoint{3.593755in}{2.072289in}}%
\pgfpathlineto{\pgfqpoint{3.580691in}{2.075520in}}%
\pgfpathlineto{\pgfqpoint{3.567631in}{2.078936in}}%
\pgfpathlineto{\pgfqpoint{3.554576in}{2.082540in}}%
\pgfpathlineto{\pgfqpoint{3.541525in}{2.086330in}}%
\pgfpathlineto{\pgfqpoint{3.533740in}{2.076998in}}%
\pgfpathlineto{\pgfqpoint{3.525949in}{2.067690in}}%
\pgfpathlineto{\pgfqpoint{3.518153in}{2.058407in}}%
\pgfpathlineto{\pgfqpoint{3.510351in}{2.049150in}}%
\pgfpathclose%
\pgfusepath{fill}%
\end{pgfscope}%
\begin{pgfscope}%
\pgfpathrectangle{\pgfqpoint{1.254980in}{0.150000in}}{\pgfqpoint{5.490039in}{5.490039in}}%
\pgfusepath{clip}%
\pgfsetbuttcap%
\pgfsetroundjoin%
\definecolor{currentfill}{rgb}{0.235526,0.309527,0.542944}%
\pgfsetfillcolor{currentfill}%
\pgfsetfillopacity{0.700000}%
\pgfsetlinewidth{0.000000pt}%
\definecolor{currentstroke}{rgb}{0.000000,0.000000,0.000000}%
\pgfsetstrokecolor{currentstroke}%
\pgfsetdash{}{0pt}%
\pgfpathmoveto{\pgfqpoint{2.596233in}{2.578961in}}%
\pgfpathlineto{\pgfqpoint{2.609522in}{2.560521in}}%
\pgfpathlineto{\pgfqpoint{2.622800in}{2.542367in}}%
\pgfpathlineto{\pgfqpoint{2.636070in}{2.524495in}}%
\pgfpathlineto{\pgfqpoint{2.649330in}{2.506903in}}%
\pgfpathlineto{\pgfqpoint{2.657542in}{2.511896in}}%
\pgfpathlineto{\pgfqpoint{2.665743in}{2.517041in}}%
\pgfpathlineto{\pgfqpoint{2.673933in}{2.522334in}}%
\pgfpathlineto{\pgfqpoint{2.682112in}{2.527775in}}%
\pgfpathlineto{\pgfqpoint{2.668882in}{2.545107in}}%
\pgfpathlineto{\pgfqpoint{2.655643in}{2.562719in}}%
\pgfpathlineto{\pgfqpoint{2.642395in}{2.580612in}}%
\pgfpathlineto{\pgfqpoint{2.629137in}{2.598790in}}%
\pgfpathlineto{\pgfqpoint{2.620928in}{2.593599in}}%
\pgfpathlineto{\pgfqpoint{2.612708in}{2.588562in}}%
\pgfpathlineto{\pgfqpoint{2.604477in}{2.583682in}}%
\pgfpathlineto{\pgfqpoint{2.596233in}{2.578961in}}%
\pgfpathclose%
\pgfusepath{fill}%
\end{pgfscope}%
\begin{pgfscope}%
\pgfpathrectangle{\pgfqpoint{1.254980in}{0.150000in}}{\pgfqpoint{5.490039in}{5.490039in}}%
\pgfusepath{clip}%
\pgfsetbuttcap%
\pgfsetroundjoin%
\definecolor{currentfill}{rgb}{0.280868,0.160771,0.472899}%
\pgfsetfillcolor{currentfill}%
\pgfsetfillopacity{0.700000}%
\pgfsetlinewidth{0.000000pt}%
\definecolor{currentstroke}{rgb}{0.000000,0.000000,0.000000}%
\pgfsetstrokecolor{currentstroke}%
\pgfsetdash{}{0pt}%
\pgfpathmoveto{\pgfqpoint{2.860467in}{2.261521in}}%
\pgfpathlineto{\pgfqpoint{2.873612in}{2.248329in}}%
\pgfpathlineto{\pgfqpoint{2.886752in}{2.235377in}}%
\pgfpathlineto{\pgfqpoint{2.899888in}{2.222663in}}%
\pgfpathlineto{\pgfqpoint{2.913020in}{2.210186in}}%
\pgfpathlineto{\pgfqpoint{2.921095in}{2.216509in}}%
\pgfpathlineto{\pgfqpoint{2.929162in}{2.222947in}}%
\pgfpathlineto{\pgfqpoint{2.937219in}{2.229498in}}%
\pgfpathlineto{\pgfqpoint{2.945267in}{2.236160in}}%
\pgfpathlineto{\pgfqpoint{2.932160in}{2.248386in}}%
\pgfpathlineto{\pgfqpoint{2.919048in}{2.260848in}}%
\pgfpathlineto{\pgfqpoint{2.905932in}{2.273548in}}%
\pgfpathlineto{\pgfqpoint{2.892813in}{2.286488in}}%
\pgfpathlineto{\pgfqpoint{2.884740in}{2.280067in}}%
\pgfpathlineto{\pgfqpoint{2.876659in}{2.273764in}}%
\pgfpathlineto{\pgfqpoint{2.868568in}{2.267581in}}%
\pgfpathlineto{\pgfqpoint{2.860467in}{2.261521in}}%
\pgfpathclose%
\pgfusepath{fill}%
\end{pgfscope}%
\begin{pgfscope}%
\pgfpathrectangle{\pgfqpoint{1.254980in}{0.150000in}}{\pgfqpoint{5.490039in}{5.490039in}}%
\pgfusepath{clip}%
\pgfsetbuttcap%
\pgfsetroundjoin%
\definecolor{currentfill}{rgb}{0.278826,0.175490,0.483397}%
\pgfsetfillcolor{currentfill}%
\pgfsetfillopacity{0.700000}%
\pgfsetlinewidth{0.000000pt}%
\definecolor{currentstroke}{rgb}{0.000000,0.000000,0.000000}%
\pgfsetstrokecolor{currentstroke}%
\pgfsetdash{}{0pt}%
\pgfpathmoveto{\pgfqpoint{4.062299in}{2.246679in}}%
\pgfpathlineto{\pgfqpoint{4.075483in}{2.248125in}}%
\pgfpathlineto{\pgfqpoint{4.088677in}{2.249742in}}%
\pgfpathlineto{\pgfqpoint{4.101879in}{2.251527in}}%
\pgfpathlineto{\pgfqpoint{4.115090in}{2.253482in}}%
\pgfpathlineto{\pgfqpoint{4.122696in}{2.263057in}}%
\pgfpathlineto{\pgfqpoint{4.130297in}{2.272599in}}%
\pgfpathlineto{\pgfqpoint{4.137893in}{2.282108in}}%
\pgfpathlineto{\pgfqpoint{4.145484in}{2.291586in}}%
\pgfpathlineto{\pgfqpoint{4.132280in}{2.289666in}}%
\pgfpathlineto{\pgfqpoint{4.119085in}{2.287914in}}%
\pgfpathlineto{\pgfqpoint{4.105899in}{2.286332in}}%
\pgfpathlineto{\pgfqpoint{4.092722in}{2.284920in}}%
\pgfpathlineto{\pgfqpoint{4.085123in}{2.275397in}}%
\pgfpathlineto{\pgfqpoint{4.077520in}{2.265850in}}%
\pgfpathlineto{\pgfqpoint{4.069912in}{2.256278in}}%
\pgfpathlineto{\pgfqpoint{4.062299in}{2.246679in}}%
\pgfpathclose%
\pgfusepath{fill}%
\end{pgfscope}%
\begin{pgfscope}%
\pgfpathrectangle{\pgfqpoint{1.254980in}{0.150000in}}{\pgfqpoint{5.490039in}{5.490039in}}%
\pgfusepath{clip}%
\pgfsetbuttcap%
\pgfsetroundjoin%
\definecolor{currentfill}{rgb}{0.281887,0.150881,0.465405}%
\pgfsetfillcolor{currentfill}%
\pgfsetfillopacity{0.700000}%
\pgfsetlinewidth{0.000000pt}%
\definecolor{currentstroke}{rgb}{0.000000,0.000000,0.000000}%
\pgfsetstrokecolor{currentstroke}%
\pgfsetdash{}{0pt}%
\pgfpathmoveto{\pgfqpoint{3.979110in}{2.203908in}}%
\pgfpathlineto{\pgfqpoint{3.992269in}{2.204677in}}%
\pgfpathlineto{\pgfqpoint{4.005436in}{2.205617in}}%
\pgfpathlineto{\pgfqpoint{4.018611in}{2.206728in}}%
\pgfpathlineto{\pgfqpoint{4.031795in}{2.208010in}}%
\pgfpathlineto{\pgfqpoint{4.039428in}{2.217720in}}%
\pgfpathlineto{\pgfqpoint{4.047057in}{2.227401in}}%
\pgfpathlineto{\pgfqpoint{4.054680in}{2.237054in}}%
\pgfpathlineto{\pgfqpoint{4.062299in}{2.246679in}}%
\pgfpathlineto{\pgfqpoint{4.049122in}{2.245403in}}%
\pgfpathlineto{\pgfqpoint{4.035955in}{2.244298in}}%
\pgfpathlineto{\pgfqpoint{4.022795in}{2.243364in}}%
\pgfpathlineto{\pgfqpoint{4.009644in}{2.242602in}}%
\pgfpathlineto{\pgfqpoint{4.002018in}{2.232960in}}%
\pgfpathlineto{\pgfqpoint{3.994387in}{2.223297in}}%
\pgfpathlineto{\pgfqpoint{3.986751in}{2.213613in}}%
\pgfpathlineto{\pgfqpoint{3.979110in}{2.203908in}}%
\pgfpathclose%
\pgfusepath{fill}%
\end{pgfscope}%
\begin{pgfscope}%
\pgfpathrectangle{\pgfqpoint{1.254980in}{0.150000in}}{\pgfqpoint{5.490039in}{5.490039in}}%
\pgfusepath{clip}%
\pgfsetbuttcap%
\pgfsetroundjoin%
\definecolor{currentfill}{rgb}{0.275191,0.194905,0.496005}%
\pgfsetfillcolor{currentfill}%
\pgfsetfillopacity{0.700000}%
\pgfsetlinewidth{0.000000pt}%
\definecolor{currentstroke}{rgb}{0.000000,0.000000,0.000000}%
\pgfsetstrokecolor{currentstroke}%
\pgfsetdash{}{0pt}%
\pgfpathmoveto{\pgfqpoint{4.145484in}{2.291586in}}%
\pgfpathlineto{\pgfqpoint{4.158697in}{2.293675in}}%
\pgfpathlineto{\pgfqpoint{4.171920in}{2.295932in}}%
\pgfpathlineto{\pgfqpoint{4.185152in}{2.298357in}}%
\pgfpathlineto{\pgfqpoint{4.198393in}{2.300949in}}%
\pgfpathlineto{\pgfqpoint{4.205972in}{2.310344in}}%
\pgfpathlineto{\pgfqpoint{4.213545in}{2.319703in}}%
\pgfpathlineto{\pgfqpoint{4.221113in}{2.329026in}}%
\pgfpathlineto{\pgfqpoint{4.228676in}{2.338314in}}%
\pgfpathlineto{\pgfqpoint{4.215442in}{2.335784in}}%
\pgfpathlineto{\pgfqpoint{4.202217in}{2.333422in}}%
\pgfpathlineto{\pgfqpoint{4.189002in}{2.331227in}}%
\pgfpathlineto{\pgfqpoint{4.175796in}{2.329200in}}%
\pgfpathlineto{\pgfqpoint{4.168226in}{2.319839in}}%
\pgfpathlineto{\pgfqpoint{4.160650in}{2.310451in}}%
\pgfpathlineto{\pgfqpoint{4.153070in}{2.301033in}}%
\pgfpathlineto{\pgfqpoint{4.145484in}{2.291586in}}%
\pgfpathclose%
\pgfusepath{fill}%
\end{pgfscope}%
\begin{pgfscope}%
\pgfpathrectangle{\pgfqpoint{1.254980in}{0.150000in}}{\pgfqpoint{5.490039in}{5.490039in}}%
\pgfusepath{clip}%
\pgfsetbuttcap%
\pgfsetroundjoin%
\definecolor{currentfill}{rgb}{0.156270,0.489624,0.557936}%
\pgfsetfillcolor{currentfill}%
\pgfsetfillopacity{0.700000}%
\pgfsetlinewidth{0.000000pt}%
\definecolor{currentstroke}{rgb}{0.000000,0.000000,0.000000}%
\pgfsetstrokecolor{currentstroke}%
\pgfsetdash{}{0pt}%
\pgfpathmoveto{\pgfqpoint{5.257314in}{2.975332in}}%
\pgfpathlineto{\pgfqpoint{5.271013in}{2.982264in}}%
\pgfpathlineto{\pgfqpoint{5.284728in}{2.989351in}}%
\pgfpathlineto{\pgfqpoint{5.298459in}{2.996591in}}%
\pgfpathlineto{\pgfqpoint{5.312205in}{3.003986in}}%
\pgfpathlineto{\pgfqpoint{5.319316in}{3.009144in}}%
\pgfpathlineto{\pgfqpoint{5.326422in}{3.014349in}}%
\pgfpathlineto{\pgfqpoint{5.333523in}{3.019604in}}%
\pgfpathlineto{\pgfqpoint{5.340620in}{3.024915in}}%
\pgfpathlineto{\pgfqpoint{5.326895in}{3.017988in}}%
\pgfpathlineto{\pgfqpoint{5.313186in}{3.011214in}}%
\pgfpathlineto{\pgfqpoint{5.299492in}{3.004594in}}%
\pgfpathlineto{\pgfqpoint{5.285813in}{2.998127in}}%
\pgfpathlineto{\pgfqpoint{5.278694in}{2.992339in}}%
\pgfpathlineto{\pgfqpoint{5.271572in}{2.986614in}}%
\pgfpathlineto{\pgfqpoint{5.264445in}{2.980947in}}%
\pgfpathlineto{\pgfqpoint{5.257314in}{2.975332in}}%
\pgfpathclose%
\pgfusepath{fill}%
\end{pgfscope}%
\begin{pgfscope}%
\pgfpathrectangle{\pgfqpoint{1.254980in}{0.150000in}}{\pgfqpoint{5.490039in}{5.490039in}}%
\pgfusepath{clip}%
\pgfsetbuttcap%
\pgfsetroundjoin%
\definecolor{currentfill}{rgb}{0.283072,0.130895,0.449241}%
\pgfsetfillcolor{currentfill}%
\pgfsetfillopacity{0.700000}%
\pgfsetlinewidth{0.000000pt}%
\definecolor{currentstroke}{rgb}{0.000000,0.000000,0.000000}%
\pgfsetstrokecolor{currentstroke}%
\pgfsetdash{}{0pt}%
\pgfpathmoveto{\pgfqpoint{3.895907in}{2.163609in}}%
\pgfpathlineto{\pgfqpoint{3.909043in}{2.163663in}}%
\pgfpathlineto{\pgfqpoint{3.922187in}{2.163891in}}%
\pgfpathlineto{\pgfqpoint{3.935338in}{2.164293in}}%
\pgfpathlineto{\pgfqpoint{3.948496in}{2.164867in}}%
\pgfpathlineto{\pgfqpoint{3.956157in}{2.174661in}}%
\pgfpathlineto{\pgfqpoint{3.963813in}{2.184432in}}%
\pgfpathlineto{\pgfqpoint{3.971464in}{2.194181in}}%
\pgfpathlineto{\pgfqpoint{3.979110in}{2.203908in}}%
\pgfpathlineto{\pgfqpoint{3.965959in}{2.203312in}}%
\pgfpathlineto{\pgfqpoint{3.952816in}{2.202889in}}%
\pgfpathlineto{\pgfqpoint{3.939681in}{2.202639in}}%
\pgfpathlineto{\pgfqpoint{3.926553in}{2.202563in}}%
\pgfpathlineto{\pgfqpoint{3.918899in}{2.192848in}}%
\pgfpathlineto{\pgfqpoint{3.911240in}{2.183117in}}%
\pgfpathlineto{\pgfqpoint{3.903576in}{2.173371in}}%
\pgfpathlineto{\pgfqpoint{3.895907in}{2.163609in}}%
\pgfpathclose%
\pgfusepath{fill}%
\end{pgfscope}%
\begin{pgfscope}%
\pgfpathrectangle{\pgfqpoint{1.254980in}{0.150000in}}{\pgfqpoint{5.490039in}{5.490039in}}%
\pgfusepath{clip}%
\pgfsetbuttcap%
\pgfsetroundjoin%
\definecolor{currentfill}{rgb}{0.269308,0.218818,0.509577}%
\pgfsetfillcolor{currentfill}%
\pgfsetfillopacity{0.700000}%
\pgfsetlinewidth{0.000000pt}%
\definecolor{currentstroke}{rgb}{0.000000,0.000000,0.000000}%
\pgfsetstrokecolor{currentstroke}%
\pgfsetdash{}{0pt}%
\pgfpathmoveto{\pgfqpoint{4.228676in}{2.338314in}}%
\pgfpathlineto{\pgfqpoint{4.241920in}{2.341010in}}%
\pgfpathlineto{\pgfqpoint{4.255174in}{2.343874in}}%
\pgfpathlineto{\pgfqpoint{4.268438in}{2.346903in}}%
\pgfpathlineto{\pgfqpoint{4.281713in}{2.350098in}}%
\pgfpathlineto{\pgfqpoint{4.289263in}{2.359273in}}%
\pgfpathlineto{\pgfqpoint{4.296808in}{2.368408in}}%
\pgfpathlineto{\pgfqpoint{4.304348in}{2.377506in}}%
\pgfpathlineto{\pgfqpoint{4.311882in}{2.386567in}}%
\pgfpathlineto{\pgfqpoint{4.298615in}{2.383463in}}%
\pgfpathlineto{\pgfqpoint{4.285359in}{2.380524in}}%
\pgfpathlineto{\pgfqpoint{4.272112in}{2.377752in}}%
\pgfpathlineto{\pgfqpoint{4.258876in}{2.375146in}}%
\pgfpathlineto{\pgfqpoint{4.251333in}{2.365983in}}%
\pgfpathlineto{\pgfqpoint{4.243786in}{2.356791in}}%
\pgfpathlineto{\pgfqpoint{4.236234in}{2.347569in}}%
\pgfpathlineto{\pgfqpoint{4.228676in}{2.338314in}}%
\pgfpathclose%
\pgfusepath{fill}%
\end{pgfscope}%
\begin{pgfscope}%
\pgfpathrectangle{\pgfqpoint{1.254980in}{0.150000in}}{\pgfqpoint{5.490039in}{5.490039in}}%
\pgfusepath{clip}%
\pgfsetbuttcap%
\pgfsetroundjoin%
\definecolor{currentfill}{rgb}{0.149039,0.508051,0.557250}%
\pgfsetfillcolor{currentfill}%
\pgfsetfillopacity{0.700000}%
\pgfsetlinewidth{0.000000pt}%
\definecolor{currentstroke}{rgb}{0.000000,0.000000,0.000000}%
\pgfsetstrokecolor{currentstroke}%
\pgfsetdash{}{0pt}%
\pgfpathmoveto{\pgfqpoint{5.340620in}{3.024915in}}%
\pgfpathlineto{\pgfqpoint{5.354360in}{3.031995in}}%
\pgfpathlineto{\pgfqpoint{5.368116in}{3.039229in}}%
\pgfpathlineto{\pgfqpoint{5.381888in}{3.046616in}}%
\pgfpathlineto{\pgfqpoint{5.395675in}{3.054157in}}%
\pgfpathlineto{\pgfqpoint{5.402746in}{3.059044in}}%
\pgfpathlineto{\pgfqpoint{5.409812in}{3.063991in}}%
\pgfpathlineto{\pgfqpoint{5.416874in}{3.069004in}}%
\pgfpathlineto{\pgfqpoint{5.423932in}{3.074088in}}%
\pgfpathlineto{\pgfqpoint{5.410168in}{3.067045in}}%
\pgfpathlineto{\pgfqpoint{5.396419in}{3.060153in}}%
\pgfpathlineto{\pgfqpoint{5.382686in}{3.053415in}}%
\pgfpathlineto{\pgfqpoint{5.368968in}{3.046829in}}%
\pgfpathlineto{\pgfqpoint{5.361887in}{3.041239in}}%
\pgfpathlineto{\pgfqpoint{5.354802in}{3.035727in}}%
\pgfpathlineto{\pgfqpoint{5.347713in}{3.030288in}}%
\pgfpathlineto{\pgfqpoint{5.340620in}{3.024915in}}%
\pgfpathclose%
\pgfusepath{fill}%
\end{pgfscope}%
\begin{pgfscope}%
\pgfpathrectangle{\pgfqpoint{1.254980in}{0.150000in}}{\pgfqpoint{5.490039in}{5.490039in}}%
\pgfusepath{clip}%
\pgfsetbuttcap%
\pgfsetroundjoin%
\definecolor{currentfill}{rgb}{0.262138,0.242286,0.520837}%
\pgfsetfillcolor{currentfill}%
\pgfsetfillopacity{0.700000}%
\pgfsetlinewidth{0.000000pt}%
\definecolor{currentstroke}{rgb}{0.000000,0.000000,0.000000}%
\pgfsetstrokecolor{currentstroke}%
\pgfsetdash{}{0pt}%
\pgfpathmoveto{\pgfqpoint{4.311882in}{2.386567in}}%
\pgfpathlineto{\pgfqpoint{4.325159in}{2.389837in}}%
\pgfpathlineto{\pgfqpoint{4.338447in}{2.393272in}}%
\pgfpathlineto{\pgfqpoint{4.351745in}{2.396871in}}%
\pgfpathlineto{\pgfqpoint{4.365054in}{2.400636in}}%
\pgfpathlineto{\pgfqpoint{4.372576in}{2.409553in}}%
\pgfpathlineto{\pgfqpoint{4.380092in}{2.418430in}}%
\pgfpathlineto{\pgfqpoint{4.387603in}{2.427269in}}%
\pgfpathlineto{\pgfqpoint{4.395108in}{2.436071in}}%
\pgfpathlineto{\pgfqpoint{4.381807in}{2.432426in}}%
\pgfpathlineto{\pgfqpoint{4.368516in}{2.428946in}}%
\pgfpathlineto{\pgfqpoint{4.355236in}{2.425630in}}%
\pgfpathlineto{\pgfqpoint{4.341967in}{2.422480in}}%
\pgfpathlineto{\pgfqpoint{4.334454in}{2.413548in}}%
\pgfpathlineto{\pgfqpoint{4.326935in}{2.404587in}}%
\pgfpathlineto{\pgfqpoint{4.319411in}{2.395594in}}%
\pgfpathlineto{\pgfqpoint{4.311882in}{2.386567in}}%
\pgfpathclose%
\pgfusepath{fill}%
\end{pgfscope}%
\begin{pgfscope}%
\pgfpathrectangle{\pgfqpoint{1.254980in}{0.150000in}}{\pgfqpoint{5.490039in}{5.490039in}}%
\pgfusepath{clip}%
\pgfsetbuttcap%
\pgfsetroundjoin%
\definecolor{currentfill}{rgb}{0.283091,0.110553,0.431554}%
\pgfsetfillcolor{currentfill}%
\pgfsetfillopacity{0.700000}%
\pgfsetlinewidth{0.000000pt}%
\definecolor{currentstroke}{rgb}{0.000000,0.000000,0.000000}%
\pgfsetstrokecolor{currentstroke}%
\pgfsetdash{}{0pt}%
\pgfpathmoveto{\pgfqpoint{3.812676in}{2.126138in}}%
\pgfpathlineto{\pgfqpoint{3.825792in}{2.125441in}}%
\pgfpathlineto{\pgfqpoint{3.838915in}{2.124921in}}%
\pgfpathlineto{\pgfqpoint{3.852045in}{2.124576in}}%
\pgfpathlineto{\pgfqpoint{3.865182in}{2.124405in}}%
\pgfpathlineto{\pgfqpoint{3.872871in}{2.134230in}}%
\pgfpathlineto{\pgfqpoint{3.880555in}{2.144039in}}%
\pgfpathlineto{\pgfqpoint{3.888233in}{2.153832in}}%
\pgfpathlineto{\pgfqpoint{3.895907in}{2.163609in}}%
\pgfpathlineto{\pgfqpoint{3.882779in}{2.163730in}}%
\pgfpathlineto{\pgfqpoint{3.869657in}{2.164025in}}%
\pgfpathlineto{\pgfqpoint{3.856542in}{2.164496in}}%
\pgfpathlineto{\pgfqpoint{3.843435in}{2.165143in}}%
\pgfpathlineto{\pgfqpoint{3.835752in}{2.155405in}}%
\pgfpathlineto{\pgfqpoint{3.828065in}{2.145658in}}%
\pgfpathlineto{\pgfqpoint{3.820373in}{2.135903in}}%
\pgfpathlineto{\pgfqpoint{3.812676in}{2.126138in}}%
\pgfpathclose%
\pgfusepath{fill}%
\end{pgfscope}%
\begin{pgfscope}%
\pgfpathrectangle{\pgfqpoint{1.254980in}{0.150000in}}{\pgfqpoint{5.490039in}{5.490039in}}%
\pgfusepath{clip}%
\pgfsetbuttcap%
\pgfsetroundjoin%
\definecolor{currentfill}{rgb}{0.282623,0.140926,0.457517}%
\pgfsetfillcolor{currentfill}%
\pgfsetfillopacity{0.700000}%
\pgfsetlinewidth{0.000000pt}%
\definecolor{currentstroke}{rgb}{0.000000,0.000000,0.000000}%
\pgfsetstrokecolor{currentstroke}%
\pgfsetdash{}{0pt}%
\pgfpathmoveto{\pgfqpoint{2.913020in}{2.210186in}}%
\pgfpathlineto{\pgfqpoint{2.926148in}{2.197944in}}%
\pgfpathlineto{\pgfqpoint{2.939272in}{2.185934in}}%
\pgfpathlineto{\pgfqpoint{2.952392in}{2.174156in}}%
\pgfpathlineto{\pgfqpoint{2.965510in}{2.162608in}}%
\pgfpathlineto{\pgfqpoint{2.973561in}{2.169192in}}%
\pgfpathlineto{\pgfqpoint{2.981604in}{2.175884in}}%
\pgfpathlineto{\pgfqpoint{2.989638in}{2.182682in}}%
\pgfpathlineto{\pgfqpoint{2.997664in}{2.189584in}}%
\pgfpathlineto{\pgfqpoint{2.984570in}{2.200883in}}%
\pgfpathlineto{\pgfqpoint{2.971472in}{2.212410in}}%
\pgfpathlineto{\pgfqpoint{2.958371in}{2.224169in}}%
\pgfpathlineto{\pgfqpoint{2.945267in}{2.236160in}}%
\pgfpathlineto{\pgfqpoint{2.937219in}{2.229498in}}%
\pgfpathlineto{\pgfqpoint{2.929162in}{2.222947in}}%
\pgfpathlineto{\pgfqpoint{2.921095in}{2.216509in}}%
\pgfpathlineto{\pgfqpoint{2.913020in}{2.210186in}}%
\pgfpathclose%
\pgfusepath{fill}%
\end{pgfscope}%
\begin{pgfscope}%
\pgfpathrectangle{\pgfqpoint{1.254980in}{0.150000in}}{\pgfqpoint{5.490039in}{5.490039in}}%
\pgfusepath{clip}%
\pgfsetbuttcap%
\pgfsetroundjoin%
\definecolor{currentfill}{rgb}{0.141935,0.526453,0.555991}%
\pgfsetfillcolor{currentfill}%
\pgfsetfillopacity{0.700000}%
\pgfsetlinewidth{0.000000pt}%
\definecolor{currentstroke}{rgb}{0.000000,0.000000,0.000000}%
\pgfsetstrokecolor{currentstroke}%
\pgfsetdash{}{0pt}%
\pgfpathmoveto{\pgfqpoint{5.423932in}{3.074088in}}%
\pgfpathlineto{\pgfqpoint{5.437712in}{3.081285in}}%
\pgfpathlineto{\pgfqpoint{5.451508in}{3.088634in}}%
\pgfpathlineto{\pgfqpoint{5.465320in}{3.096135in}}%
\pgfpathlineto{\pgfqpoint{5.479148in}{3.103790in}}%
\pgfpathlineto{\pgfqpoint{5.486178in}{3.108436in}}%
\pgfpathlineto{\pgfqpoint{5.493205in}{3.113158in}}%
\pgfpathlineto{\pgfqpoint{5.500228in}{3.117962in}}%
\pgfpathlineto{\pgfqpoint{5.507247in}{3.122854in}}%
\pgfpathlineto{\pgfqpoint{5.493444in}{3.115725in}}%
\pgfpathlineto{\pgfqpoint{5.479657in}{3.108749in}}%
\pgfpathlineto{\pgfqpoint{5.465885in}{3.101924in}}%
\pgfpathlineto{\pgfqpoint{5.452130in}{3.095251in}}%
\pgfpathlineto{\pgfqpoint{5.445085in}{3.089825in}}%
\pgfpathlineto{\pgfqpoint{5.438037in}{3.084493in}}%
\pgfpathlineto{\pgfqpoint{5.430986in}{3.079249in}}%
\pgfpathlineto{\pgfqpoint{5.423932in}{3.074088in}}%
\pgfpathclose%
\pgfusepath{fill}%
\end{pgfscope}%
\begin{pgfscope}%
\pgfpathrectangle{\pgfqpoint{1.254980in}{0.150000in}}{\pgfqpoint{5.490039in}{5.490039in}}%
\pgfusepath{clip}%
\pgfsetbuttcap%
\pgfsetroundjoin%
\definecolor{currentfill}{rgb}{0.220057,0.343307,0.549413}%
\pgfsetfillcolor{currentfill}%
\pgfsetfillopacity{0.700000}%
\pgfsetlinewidth{0.000000pt}%
\definecolor{currentstroke}{rgb}{0.000000,0.000000,0.000000}%
\pgfsetstrokecolor{currentstroke}%
\pgfsetdash{}{0pt}%
\pgfpathmoveto{\pgfqpoint{2.542976in}{2.655622in}}%
\pgfpathlineto{\pgfqpoint{2.556307in}{2.636016in}}%
\pgfpathlineto{\pgfqpoint{2.569626in}{2.616705in}}%
\pgfpathlineto{\pgfqpoint{2.582935in}{2.597688in}}%
\pgfpathlineto{\pgfqpoint{2.596233in}{2.578961in}}%
\pgfpathlineto{\pgfqpoint{2.604477in}{2.583682in}}%
\pgfpathlineto{\pgfqpoint{2.612708in}{2.588562in}}%
\pgfpathlineto{\pgfqpoint{2.620928in}{2.593599in}}%
\pgfpathlineto{\pgfqpoint{2.629137in}{2.598790in}}%
\pgfpathlineto{\pgfqpoint{2.615870in}{2.617254in}}%
\pgfpathlineto{\pgfqpoint{2.602593in}{2.636009in}}%
\pgfpathlineto{\pgfqpoint{2.589305in}{2.655056in}}%
\pgfpathlineto{\pgfqpoint{2.576007in}{2.674399in}}%
\pgfpathlineto{\pgfqpoint{2.567767in}{2.669460in}}%
\pgfpathlineto{\pgfqpoint{2.559516in}{2.664683in}}%
\pgfpathlineto{\pgfqpoint{2.551252in}{2.660069in}}%
\pgfpathlineto{\pgfqpoint{2.542976in}{2.655622in}}%
\pgfpathclose%
\pgfusepath{fill}%
\end{pgfscope}%
\begin{pgfscope}%
\pgfpathrectangle{\pgfqpoint{1.254980in}{0.150000in}}{\pgfqpoint{5.490039in}{5.490039in}}%
\pgfusepath{clip}%
\pgfsetbuttcap%
\pgfsetroundjoin%
\definecolor{currentfill}{rgb}{0.277941,0.056324,0.381191}%
\pgfsetfillcolor{currentfill}%
\pgfsetfillopacity{0.700000}%
\pgfsetlinewidth{0.000000pt}%
\definecolor{currentstroke}{rgb}{0.000000,0.000000,0.000000}%
\pgfsetstrokecolor{currentstroke}%
\pgfsetdash{}{0pt}%
\pgfpathmoveto{\pgfqpoint{3.290870in}{2.042460in}}%
\pgfpathlineto{\pgfqpoint{3.303930in}{2.035859in}}%
\pgfpathlineto{\pgfqpoint{3.316992in}{2.029458in}}%
\pgfpathlineto{\pgfqpoint{3.330056in}{2.023256in}}%
\pgfpathlineto{\pgfqpoint{3.343121in}{2.017250in}}%
\pgfpathlineto{\pgfqpoint{3.350998in}{2.025853in}}%
\pgfpathlineto{\pgfqpoint{3.358869in}{2.034506in}}%
\pgfpathlineto{\pgfqpoint{3.366734in}{2.043208in}}%
\pgfpathlineto{\pgfqpoint{3.374592in}{2.051958in}}%
\pgfpathlineto{\pgfqpoint{3.361542in}{2.057774in}}%
\pgfpathlineto{\pgfqpoint{3.348493in}{2.063787in}}%
\pgfpathlineto{\pgfqpoint{3.335447in}{2.069999in}}%
\pgfpathlineto{\pgfqpoint{3.322402in}{2.076409in}}%
\pgfpathlineto{\pgfqpoint{3.314529in}{2.067838in}}%
\pgfpathlineto{\pgfqpoint{3.306649in}{2.059322in}}%
\pgfpathlineto{\pgfqpoint{3.298763in}{2.050862in}}%
\pgfpathlineto{\pgfqpoint{3.290870in}{2.042460in}}%
\pgfpathclose%
\pgfusepath{fill}%
\end{pgfscope}%
\begin{pgfscope}%
\pgfpathrectangle{\pgfqpoint{1.254980in}{0.150000in}}{\pgfqpoint{5.490039in}{5.490039in}}%
\pgfusepath{clip}%
\pgfsetbuttcap%
\pgfsetroundjoin%
\definecolor{currentfill}{rgb}{0.252194,0.269783,0.531579}%
\pgfsetfillcolor{currentfill}%
\pgfsetfillopacity{0.700000}%
\pgfsetlinewidth{0.000000pt}%
\definecolor{currentstroke}{rgb}{0.000000,0.000000,0.000000}%
\pgfsetstrokecolor{currentstroke}%
\pgfsetdash{}{0pt}%
\pgfpathmoveto{\pgfqpoint{4.395108in}{2.436071in}}%
\pgfpathlineto{\pgfqpoint{4.408420in}{2.439879in}}%
\pgfpathlineto{\pgfqpoint{4.421743in}{2.443852in}}%
\pgfpathlineto{\pgfqpoint{4.435078in}{2.447987in}}%
\pgfpathlineto{\pgfqpoint{4.448424in}{2.452287in}}%
\pgfpathlineto{\pgfqpoint{4.455916in}{2.460916in}}%
\pgfpathlineto{\pgfqpoint{4.463402in}{2.469504in}}%
\pgfpathlineto{\pgfqpoint{4.470883in}{2.478054in}}%
\pgfpathlineto{\pgfqpoint{4.478358in}{2.486568in}}%
\pgfpathlineto{\pgfqpoint{4.465020in}{2.482418in}}%
\pgfpathlineto{\pgfqpoint{4.451694in}{2.478430in}}%
\pgfpathlineto{\pgfqpoint{4.438379in}{2.474605in}}%
\pgfpathlineto{\pgfqpoint{4.425076in}{2.470945in}}%
\pgfpathlineto{\pgfqpoint{4.417592in}{2.462272in}}%
\pgfpathlineto{\pgfqpoint{4.410102in}{2.453570in}}%
\pgfpathlineto{\pgfqpoint{4.402608in}{2.444837in}}%
\pgfpathlineto{\pgfqpoint{4.395108in}{2.436071in}}%
\pgfpathclose%
\pgfusepath{fill}%
\end{pgfscope}%
\begin{pgfscope}%
\pgfpathrectangle{\pgfqpoint{1.254980in}{0.150000in}}{\pgfqpoint{5.490039in}{5.490039in}}%
\pgfusepath{clip}%
\pgfsetbuttcap%
\pgfsetroundjoin%
\definecolor{currentfill}{rgb}{0.135066,0.544853,0.554029}%
\pgfsetfillcolor{currentfill}%
\pgfsetfillopacity{0.700000}%
\pgfsetlinewidth{0.000000pt}%
\definecolor{currentstroke}{rgb}{0.000000,0.000000,0.000000}%
\pgfsetstrokecolor{currentstroke}%
\pgfsetdash{}{0pt}%
\pgfpathmoveto{\pgfqpoint{5.507247in}{3.122854in}}%
\pgfpathlineto{\pgfqpoint{5.521066in}{3.130134in}}%
\pgfpathlineto{\pgfqpoint{5.534902in}{3.137566in}}%
\pgfpathlineto{\pgfqpoint{5.548753in}{3.145151in}}%
\pgfpathlineto{\pgfqpoint{5.562622in}{3.152887in}}%
\pgfpathlineto{\pgfqpoint{5.569612in}{3.157328in}}%
\pgfpathlineto{\pgfqpoint{5.576599in}{3.161862in}}%
\pgfpathlineto{\pgfqpoint{5.583583in}{3.166496in}}%
\pgfpathlineto{\pgfqpoint{5.590564in}{3.171234in}}%
\pgfpathlineto{\pgfqpoint{5.576723in}{3.164052in}}%
\pgfpathlineto{\pgfqpoint{5.562898in}{3.157022in}}%
\pgfpathlineto{\pgfqpoint{5.549089in}{3.150143in}}%
\pgfpathlineto{\pgfqpoint{5.535296in}{3.143415in}}%
\pgfpathlineto{\pgfqpoint{5.528288in}{3.138113in}}%
\pgfpathlineto{\pgfqpoint{5.521277in}{3.132923in}}%
\pgfpathlineto{\pgfqpoint{5.514263in}{3.127839in}}%
\pgfpathlineto{\pgfqpoint{5.507247in}{3.122854in}}%
\pgfpathclose%
\pgfusepath{fill}%
\end{pgfscope}%
\begin{pgfscope}%
\pgfpathrectangle{\pgfqpoint{1.254980in}{0.150000in}}{\pgfqpoint{5.490039in}{5.490039in}}%
\pgfusepath{clip}%
\pgfsetbuttcap%
\pgfsetroundjoin%
\definecolor{currentfill}{rgb}{0.282327,0.094955,0.417331}%
\pgfsetfillcolor{currentfill}%
\pgfsetfillopacity{0.700000}%
\pgfsetlinewidth{0.000000pt}%
\definecolor{currentstroke}{rgb}{0.000000,0.000000,0.000000}%
\pgfsetstrokecolor{currentstroke}%
\pgfsetdash{}{0pt}%
\pgfpathmoveto{\pgfqpoint{3.729398in}{2.091873in}}%
\pgfpathlineto{\pgfqpoint{3.742499in}{2.090387in}}%
\pgfpathlineto{\pgfqpoint{3.755605in}{2.089081in}}%
\pgfpathlineto{\pgfqpoint{3.768717in}{2.087952in}}%
\pgfpathlineto{\pgfqpoint{3.781836in}{2.087001in}}%
\pgfpathlineto{\pgfqpoint{3.789553in}{2.096796in}}%
\pgfpathlineto{\pgfqpoint{3.797266in}{2.106585in}}%
\pgfpathlineto{\pgfqpoint{3.804973in}{2.116366in}}%
\pgfpathlineto{\pgfqpoint{3.812676in}{2.126138in}}%
\pgfpathlineto{\pgfqpoint{3.799566in}{2.127012in}}%
\pgfpathlineto{\pgfqpoint{3.786462in}{2.128063in}}%
\pgfpathlineto{\pgfqpoint{3.773365in}{2.129292in}}%
\pgfpathlineto{\pgfqpoint{3.760275in}{2.130700in}}%
\pgfpathlineto{\pgfqpoint{3.752563in}{2.120995in}}%
\pgfpathlineto{\pgfqpoint{3.744847in}{2.111288in}}%
\pgfpathlineto{\pgfqpoint{3.737125in}{2.101581in}}%
\pgfpathlineto{\pgfqpoint{3.729398in}{2.091873in}}%
\pgfpathclose%
\pgfusepath{fill}%
\end{pgfscope}%
\begin{pgfscope}%
\pgfpathrectangle{\pgfqpoint{1.254980in}{0.150000in}}{\pgfqpoint{5.490039in}{5.490039in}}%
\pgfusepath{clip}%
\pgfsetbuttcap%
\pgfsetroundjoin%
\definecolor{currentfill}{rgb}{0.128729,0.563265,0.551229}%
\pgfsetfillcolor{currentfill}%
\pgfsetfillopacity{0.700000}%
\pgfsetlinewidth{0.000000pt}%
\definecolor{currentstroke}{rgb}{0.000000,0.000000,0.000000}%
\pgfsetstrokecolor{currentstroke}%
\pgfsetdash{}{0pt}%
\pgfpathmoveto{\pgfqpoint{5.590564in}{3.171234in}}%
\pgfpathlineto{\pgfqpoint{5.604422in}{3.178567in}}%
\pgfpathlineto{\pgfqpoint{5.618296in}{3.186051in}}%
\pgfpathlineto{\pgfqpoint{5.632186in}{3.193686in}}%
\pgfpathlineto{\pgfqpoint{5.646093in}{3.201473in}}%
\pgfpathlineto{\pgfqpoint{5.653044in}{3.205749in}}%
\pgfpathlineto{\pgfqpoint{5.659992in}{3.210137in}}%
\pgfpathlineto{\pgfqpoint{5.666938in}{3.214643in}}%
\pgfpathlineto{\pgfqpoint{5.673883in}{3.219273in}}%
\pgfpathlineto{\pgfqpoint{5.660004in}{3.212070in}}%
\pgfpathlineto{\pgfqpoint{5.646143in}{3.205017in}}%
\pgfpathlineto{\pgfqpoint{5.632297in}{3.198116in}}%
\pgfpathlineto{\pgfqpoint{5.618468in}{3.191364in}}%
\pgfpathlineto{\pgfqpoint{5.611495in}{3.186142in}}%
\pgfpathlineto{\pgfqpoint{5.604520in}{3.181051in}}%
\pgfpathlineto{\pgfqpoint{5.597543in}{3.176084in}}%
\pgfpathlineto{\pgfqpoint{5.590564in}{3.171234in}}%
\pgfpathclose%
\pgfusepath{fill}%
\end{pgfscope}%
\begin{pgfscope}%
\pgfpathrectangle{\pgfqpoint{1.254980in}{0.150000in}}{\pgfqpoint{5.490039in}{5.490039in}}%
\pgfusepath{clip}%
\pgfsetbuttcap%
\pgfsetroundjoin%
\definecolor{currentfill}{rgb}{0.280267,0.073417,0.397163}%
\pgfsetfillcolor{currentfill}%
\pgfsetfillopacity{0.700000}%
\pgfsetlinewidth{0.000000pt}%
\definecolor{currentstroke}{rgb}{0.000000,0.000000,0.000000}%
\pgfsetstrokecolor{currentstroke}%
\pgfsetdash{}{0pt}%
\pgfpathmoveto{\pgfqpoint{3.154637in}{2.071326in}}%
\pgfpathlineto{\pgfqpoint{3.167711in}{2.062870in}}%
\pgfpathlineto{\pgfqpoint{3.180785in}{2.054623in}}%
\pgfpathlineto{\pgfqpoint{3.193858in}{2.046584in}}%
\pgfpathlineto{\pgfqpoint{3.206932in}{2.038752in}}%
\pgfpathlineto{\pgfqpoint{3.214869in}{2.046677in}}%
\pgfpathlineto{\pgfqpoint{3.222799in}{2.054675in}}%
\pgfpathlineto{\pgfqpoint{3.230723in}{2.062742in}}%
\pgfpathlineto{\pgfqpoint{3.238639in}{2.070877in}}%
\pgfpathlineto{\pgfqpoint{3.225583in}{2.078491in}}%
\pgfpathlineto{\pgfqpoint{3.212527in}{2.086312in}}%
\pgfpathlineto{\pgfqpoint{3.199472in}{2.094340in}}%
\pgfpathlineto{\pgfqpoint{3.186416in}{2.102576in}}%
\pgfpathlineto{\pgfqpoint{3.178483in}{2.094649in}}%
\pgfpathlineto{\pgfqpoint{3.170541in}{2.086797in}}%
\pgfpathlineto{\pgfqpoint{3.162593in}{2.079022in}}%
\pgfpathlineto{\pgfqpoint{3.154637in}{2.071326in}}%
\pgfpathclose%
\pgfusepath{fill}%
\end{pgfscope}%
\begin{pgfscope}%
\pgfpathrectangle{\pgfqpoint{1.254980in}{0.150000in}}{\pgfqpoint{5.490039in}{5.490039in}}%
\pgfusepath{clip}%
\pgfsetbuttcap%
\pgfsetroundjoin%
\definecolor{currentfill}{rgb}{0.243113,0.292092,0.538516}%
\pgfsetfillcolor{currentfill}%
\pgfsetfillopacity{0.700000}%
\pgfsetlinewidth{0.000000pt}%
\definecolor{currentstroke}{rgb}{0.000000,0.000000,0.000000}%
\pgfsetstrokecolor{currentstroke}%
\pgfsetdash{}{0pt}%
\pgfpathmoveto{\pgfqpoint{4.478358in}{2.486568in}}%
\pgfpathlineto{\pgfqpoint{4.491707in}{2.490882in}}%
\pgfpathlineto{\pgfqpoint{4.505067in}{2.495358in}}%
\pgfpathlineto{\pgfqpoint{4.518440in}{2.499997in}}%
\pgfpathlineto{\pgfqpoint{4.531824in}{2.504797in}}%
\pgfpathlineto{\pgfqpoint{4.539285in}{2.513111in}}%
\pgfpathlineto{\pgfqpoint{4.546740in}{2.521385in}}%
\pgfpathlineto{\pgfqpoint{4.554190in}{2.529622in}}%
\pgfpathlineto{\pgfqpoint{4.561634in}{2.537824in}}%
\pgfpathlineto{\pgfqpoint{4.548259in}{2.533201in}}%
\pgfpathlineto{\pgfqpoint{4.534896in}{2.528739in}}%
\pgfpathlineto{\pgfqpoint{4.521544in}{2.524440in}}%
\pgfpathlineto{\pgfqpoint{4.508204in}{2.520303in}}%
\pgfpathlineto{\pgfqpoint{4.500751in}{2.511913in}}%
\pgfpathlineto{\pgfqpoint{4.493292in}{2.503495in}}%
\pgfpathlineto{\pgfqpoint{4.485828in}{2.495048in}}%
\pgfpathlineto{\pgfqpoint{4.478358in}{2.486568in}}%
\pgfpathclose%
\pgfusepath{fill}%
\end{pgfscope}%
\begin{pgfscope}%
\pgfpathrectangle{\pgfqpoint{1.254980in}{0.150000in}}{\pgfqpoint{5.490039in}{5.490039in}}%
\pgfusepath{clip}%
\pgfsetbuttcap%
\pgfsetroundjoin%
\definecolor{currentfill}{rgb}{0.123463,0.581687,0.547445}%
\pgfsetfillcolor{currentfill}%
\pgfsetfillopacity{0.700000}%
\pgfsetlinewidth{0.000000pt}%
\definecolor{currentstroke}{rgb}{0.000000,0.000000,0.000000}%
\pgfsetstrokecolor{currentstroke}%
\pgfsetdash{}{0pt}%
\pgfpathmoveto{\pgfqpoint{5.673883in}{3.219273in}}%
\pgfpathlineto{\pgfqpoint{5.687777in}{3.226626in}}%
\pgfpathlineto{\pgfqpoint{5.701689in}{3.234130in}}%
\pgfpathlineto{\pgfqpoint{5.715617in}{3.241785in}}%
\pgfpathlineto{\pgfqpoint{5.729563in}{3.249590in}}%
\pgfpathlineto{\pgfqpoint{5.736475in}{3.253748in}}%
\pgfpathlineto{\pgfqpoint{5.743385in}{3.258037in}}%
\pgfpathlineto{\pgfqpoint{5.750295in}{3.262464in}}%
\pgfpathlineto{\pgfqpoint{5.757203in}{3.267034in}}%
\pgfpathlineto{\pgfqpoint{5.743289in}{3.259841in}}%
\pgfpathlineto{\pgfqpoint{5.729391in}{3.252798in}}%
\pgfpathlineto{\pgfqpoint{5.715510in}{3.245905in}}%
\pgfpathlineto{\pgfqpoint{5.701645in}{3.239162in}}%
\pgfpathlineto{\pgfqpoint{5.694706in}{3.233971in}}%
\pgfpathlineto{\pgfqpoint{5.687766in}{3.228930in}}%
\pgfpathlineto{\pgfqpoint{5.680825in}{3.224033in}}%
\pgfpathlineto{\pgfqpoint{5.673883in}{3.219273in}}%
\pgfpathclose%
\pgfusepath{fill}%
\end{pgfscope}%
\begin{pgfscope}%
\pgfpathrectangle{\pgfqpoint{1.254980in}{0.150000in}}{\pgfqpoint{5.490039in}{5.490039in}}%
\pgfusepath{clip}%
\pgfsetbuttcap%
\pgfsetroundjoin%
\definecolor{currentfill}{rgb}{0.277941,0.056324,0.381191}%
\pgfsetfillcolor{currentfill}%
\pgfsetfillopacity{0.700000}%
\pgfsetlinewidth{0.000000pt}%
\definecolor{currentstroke}{rgb}{0.000000,0.000000,0.000000}%
\pgfsetstrokecolor{currentstroke}%
\pgfsetdash{}{0pt}%
\pgfpathmoveto{\pgfqpoint{3.426817in}{2.030646in}}%
\pgfpathlineto{\pgfqpoint{3.439879in}{2.025801in}}%
\pgfpathlineto{\pgfqpoint{3.452945in}{2.021147in}}%
\pgfpathlineto{\pgfqpoint{3.466014in}{2.016683in}}%
\pgfpathlineto{\pgfqpoint{3.479087in}{2.012409in}}%
\pgfpathlineto{\pgfqpoint{3.486911in}{2.021549in}}%
\pgfpathlineto{\pgfqpoint{3.494730in}{2.030720in}}%
\pgfpathlineto{\pgfqpoint{3.502543in}{2.039921in}}%
\pgfpathlineto{\pgfqpoint{3.510351in}{2.049150in}}%
\pgfpathlineto{\pgfqpoint{3.497291in}{2.053263in}}%
\pgfpathlineto{\pgfqpoint{3.484235in}{2.057565in}}%
\pgfpathlineto{\pgfqpoint{3.471182in}{2.062058in}}%
\pgfpathlineto{\pgfqpoint{3.458133in}{2.066742in}}%
\pgfpathlineto{\pgfqpoint{3.450313in}{2.057664in}}%
\pgfpathlineto{\pgfqpoint{3.442487in}{2.048621in}}%
\pgfpathlineto{\pgfqpoint{3.434655in}{2.039614in}}%
\pgfpathlineto{\pgfqpoint{3.426817in}{2.030646in}}%
\pgfpathclose%
\pgfusepath{fill}%
\end{pgfscope}%
\begin{pgfscope}%
\pgfpathrectangle{\pgfqpoint{1.254980in}{0.150000in}}{\pgfqpoint{5.490039in}{5.490039in}}%
\pgfusepath{clip}%
\pgfsetbuttcap%
\pgfsetroundjoin%
\definecolor{currentfill}{rgb}{0.120092,0.600104,0.542530}%
\pgfsetfillcolor{currentfill}%
\pgfsetfillopacity{0.700000}%
\pgfsetlinewidth{0.000000pt}%
\definecolor{currentstroke}{rgb}{0.000000,0.000000,0.000000}%
\pgfsetstrokecolor{currentstroke}%
\pgfsetdash{}{0pt}%
\pgfpathmoveto{\pgfqpoint{5.757203in}{3.267034in}}%
\pgfpathlineto{\pgfqpoint{5.771134in}{3.274377in}}%
\pgfpathlineto{\pgfqpoint{5.785082in}{3.281869in}}%
\pgfpathlineto{\pgfqpoint{5.799047in}{3.289512in}}%
\pgfpathlineto{\pgfqpoint{5.813029in}{3.297305in}}%
\pgfpathlineto{\pgfqpoint{5.819904in}{3.301395in}}%
\pgfpathlineto{\pgfqpoint{5.826779in}{3.305637in}}%
\pgfpathlineto{\pgfqpoint{5.833653in}{3.310037in}}%
\pgfpathlineto{\pgfqpoint{5.840527in}{3.314603in}}%
\pgfpathlineto{\pgfqpoint{5.826578in}{3.307451in}}%
\pgfpathlineto{\pgfqpoint{5.812645in}{3.300449in}}%
\pgfpathlineto{\pgfqpoint{5.798730in}{3.293596in}}%
\pgfpathlineto{\pgfqpoint{5.784831in}{3.286893in}}%
\pgfpathlineto{\pgfqpoint{5.777924in}{3.281677in}}%
\pgfpathlineto{\pgfqpoint{5.771017in}{3.276634in}}%
\pgfpathlineto{\pgfqpoint{5.764110in}{3.271755in}}%
\pgfpathlineto{\pgfqpoint{5.757203in}{3.267034in}}%
\pgfpathclose%
\pgfusepath{fill}%
\end{pgfscope}%
\begin{pgfscope}%
\pgfpathrectangle{\pgfqpoint{1.254980in}{0.150000in}}{\pgfqpoint{5.490039in}{5.490039in}}%
\pgfusepath{clip}%
\pgfsetbuttcap%
\pgfsetroundjoin%
\definecolor{currentfill}{rgb}{0.283229,0.120777,0.440584}%
\pgfsetfillcolor{currentfill}%
\pgfsetfillopacity{0.700000}%
\pgfsetlinewidth{0.000000pt}%
\definecolor{currentstroke}{rgb}{0.000000,0.000000,0.000000}%
\pgfsetstrokecolor{currentstroke}%
\pgfsetdash{}{0pt}%
\pgfpathmoveto{\pgfqpoint{2.965510in}{2.162608in}}%
\pgfpathlineto{\pgfqpoint{2.978624in}{2.151287in}}%
\pgfpathlineto{\pgfqpoint{2.991735in}{2.140193in}}%
\pgfpathlineto{\pgfqpoint{3.004844in}{2.129324in}}%
\pgfpathlineto{\pgfqpoint{3.017951in}{2.118678in}}%
\pgfpathlineto{\pgfqpoint{3.025979in}{2.125522in}}%
\pgfpathlineto{\pgfqpoint{3.034000in}{2.132467in}}%
\pgfpathlineto{\pgfqpoint{3.042012in}{2.139511in}}%
\pgfpathlineto{\pgfqpoint{3.050016in}{2.146652in}}%
\pgfpathlineto{\pgfqpoint{3.036931in}{2.157049in}}%
\pgfpathlineto{\pgfqpoint{3.023845in}{2.167669in}}%
\pgfpathlineto{\pgfqpoint{3.010756in}{2.178514in}}%
\pgfpathlineto{\pgfqpoint{2.997664in}{2.189584in}}%
\pgfpathlineto{\pgfqpoint{2.989638in}{2.182682in}}%
\pgfpathlineto{\pgfqpoint{2.981604in}{2.175884in}}%
\pgfpathlineto{\pgfqpoint{2.973561in}{2.169192in}}%
\pgfpathlineto{\pgfqpoint{2.965510in}{2.162608in}}%
\pgfpathclose%
\pgfusepath{fill}%
\end{pgfscope}%
\begin{pgfscope}%
\pgfpathrectangle{\pgfqpoint{1.254980in}{0.150000in}}{\pgfqpoint{5.490039in}{5.490039in}}%
\pgfusepath{clip}%
\pgfsetbuttcap%
\pgfsetroundjoin%
\definecolor{currentfill}{rgb}{0.231674,0.318106,0.544834}%
\pgfsetfillcolor{currentfill}%
\pgfsetfillopacity{0.700000}%
\pgfsetlinewidth{0.000000pt}%
\definecolor{currentstroke}{rgb}{0.000000,0.000000,0.000000}%
\pgfsetstrokecolor{currentstroke}%
\pgfsetdash{}{0pt}%
\pgfpathmoveto{\pgfqpoint{4.561634in}{2.537824in}}%
\pgfpathlineto{\pgfqpoint{4.575021in}{2.542609in}}%
\pgfpathlineto{\pgfqpoint{4.588421in}{2.547556in}}%
\pgfpathlineto{\pgfqpoint{4.601832in}{2.552664in}}%
\pgfpathlineto{\pgfqpoint{4.615256in}{2.557933in}}%
\pgfpathlineto{\pgfqpoint{4.622685in}{2.565908in}}%
\pgfpathlineto{\pgfqpoint{4.630108in}{2.573846in}}%
\pgfpathlineto{\pgfqpoint{4.637526in}{2.581750in}}%
\pgfpathlineto{\pgfqpoint{4.644938in}{2.589622in}}%
\pgfpathlineto{\pgfqpoint{4.631524in}{2.584559in}}%
\pgfpathlineto{\pgfqpoint{4.618123in}{2.579657in}}%
\pgfpathlineto{\pgfqpoint{4.604733in}{2.574916in}}%
\pgfpathlineto{\pgfqpoint{4.591356in}{2.570337in}}%
\pgfpathlineto{\pgfqpoint{4.583934in}{2.562248in}}%
\pgfpathlineto{\pgfqpoint{4.576506in}{2.554135in}}%
\pgfpathlineto{\pgfqpoint{4.569073in}{2.545995in}}%
\pgfpathlineto{\pgfqpoint{4.561634in}{2.537824in}}%
\pgfpathclose%
\pgfusepath{fill}%
\end{pgfscope}%
\begin{pgfscope}%
\pgfpathrectangle{\pgfqpoint{1.254980in}{0.150000in}}{\pgfqpoint{5.490039in}{5.490039in}}%
\pgfusepath{clip}%
\pgfsetbuttcap%
\pgfsetroundjoin%
\definecolor{currentfill}{rgb}{0.119699,0.618490,0.536347}%
\pgfsetfillcolor{currentfill}%
\pgfsetfillopacity{0.700000}%
\pgfsetlinewidth{0.000000pt}%
\definecolor{currentstroke}{rgb}{0.000000,0.000000,0.000000}%
\pgfsetstrokecolor{currentstroke}%
\pgfsetdash{}{0pt}%
\pgfpathmoveto{\pgfqpoint{5.840527in}{3.314603in}}%
\pgfpathlineto{\pgfqpoint{5.854493in}{3.321903in}}%
\pgfpathlineto{\pgfqpoint{5.868476in}{3.329353in}}%
\pgfpathlineto{\pgfqpoint{5.882477in}{3.336953in}}%
\pgfpathlineto{\pgfqpoint{5.896494in}{3.344701in}}%
\pgfpathlineto{\pgfqpoint{5.903334in}{3.348780in}}%
\pgfpathlineto{\pgfqpoint{5.910174in}{3.353032in}}%
\pgfpathlineto{\pgfqpoint{5.917015in}{3.357465in}}%
\pgfpathlineto{\pgfqpoint{5.923857in}{3.362086in}}%
\pgfpathlineto{\pgfqpoint{5.909874in}{3.355007in}}%
\pgfpathlineto{\pgfqpoint{5.895909in}{3.348077in}}%
\pgfpathlineto{\pgfqpoint{5.881960in}{3.341295in}}%
\pgfpathlineto{\pgfqpoint{5.868028in}{3.334662in}}%
\pgfpathlineto{\pgfqpoint{5.861151in}{3.329363in}}%
\pgfpathlineto{\pgfqpoint{5.854276in}{3.324259in}}%
\pgfpathlineto{\pgfqpoint{5.847401in}{3.319341in}}%
\pgfpathlineto{\pgfqpoint{5.840527in}{3.314603in}}%
\pgfpathclose%
\pgfusepath{fill}%
\end{pgfscope}%
\begin{pgfscope}%
\pgfpathrectangle{\pgfqpoint{1.254980in}{0.150000in}}{\pgfqpoint{5.490039in}{5.490039in}}%
\pgfusepath{clip}%
\pgfsetbuttcap%
\pgfsetroundjoin%
\definecolor{currentfill}{rgb}{0.204903,0.375746,0.553533}%
\pgfsetfillcolor{currentfill}%
\pgfsetfillopacity{0.700000}%
\pgfsetlinewidth{0.000000pt}%
\definecolor{currentstroke}{rgb}{0.000000,0.000000,0.000000}%
\pgfsetstrokecolor{currentstroke}%
\pgfsetdash{}{0pt}%
\pgfpathmoveto{\pgfqpoint{2.489540in}{2.737068in}}%
\pgfpathlineto{\pgfqpoint{2.502917in}{2.716247in}}%
\pgfpathlineto{\pgfqpoint{2.516282in}{2.695735in}}%
\pgfpathlineto{\pgfqpoint{2.529635in}{2.675528in}}%
\pgfpathlineto{\pgfqpoint{2.542976in}{2.655622in}}%
\pgfpathlineto{\pgfqpoint{2.551252in}{2.660069in}}%
\pgfpathlineto{\pgfqpoint{2.559516in}{2.664683in}}%
\pgfpathlineto{\pgfqpoint{2.567767in}{2.669460in}}%
\pgfpathlineto{\pgfqpoint{2.576007in}{2.674399in}}%
\pgfpathlineto{\pgfqpoint{2.562698in}{2.694040in}}%
\pgfpathlineto{\pgfqpoint{2.549378in}{2.713982in}}%
\pgfpathlineto{\pgfqpoint{2.536046in}{2.734228in}}%
\pgfpathlineto{\pgfqpoint{2.522703in}{2.754782in}}%
\pgfpathlineto{\pgfqpoint{2.514431in}{2.750097in}}%
\pgfpathlineto{\pgfqpoint{2.506146in}{2.745582in}}%
\pgfpathlineto{\pgfqpoint{2.497849in}{2.741238in}}%
\pgfpathlineto{\pgfqpoint{2.489540in}{2.737068in}}%
\pgfpathclose%
\pgfusepath{fill}%
\end{pgfscope}%
\begin{pgfscope}%
\pgfpathrectangle{\pgfqpoint{1.254980in}{0.150000in}}{\pgfqpoint{5.490039in}{5.490039in}}%
\pgfusepath{clip}%
\pgfsetbuttcap%
\pgfsetroundjoin%
\definecolor{currentfill}{rgb}{0.280894,0.078907,0.402329}%
\pgfsetfillcolor{currentfill}%
\pgfsetfillopacity{0.700000}%
\pgfsetlinewidth{0.000000pt}%
\definecolor{currentstroke}{rgb}{0.000000,0.000000,0.000000}%
\pgfsetstrokecolor{currentstroke}%
\pgfsetdash{}{0pt}%
\pgfpathmoveto{\pgfqpoint{3.646057in}{2.061211in}}%
\pgfpathlineto{\pgfqpoint{3.659145in}{2.058899in}}%
\pgfpathlineto{\pgfqpoint{3.672238in}{2.056768in}}%
\pgfpathlineto{\pgfqpoint{3.685336in}{2.054818in}}%
\pgfpathlineto{\pgfqpoint{3.698440in}{2.053048in}}%
\pgfpathlineto{\pgfqpoint{3.706187in}{2.062752in}}%
\pgfpathlineto{\pgfqpoint{3.713930in}{2.072458in}}%
\pgfpathlineto{\pgfqpoint{3.721667in}{2.082165in}}%
\pgfpathlineto{\pgfqpoint{3.729398in}{2.091873in}}%
\pgfpathlineto{\pgfqpoint{3.716304in}{2.093537in}}%
\pgfpathlineto{\pgfqpoint{3.703216in}{2.095382in}}%
\pgfpathlineto{\pgfqpoint{3.690132in}{2.097408in}}%
\pgfpathlineto{\pgfqpoint{3.677055in}{2.099614in}}%
\pgfpathlineto{\pgfqpoint{3.669313in}{2.090002in}}%
\pgfpathlineto{\pgfqpoint{3.661566in}{2.080397in}}%
\pgfpathlineto{\pgfqpoint{3.653814in}{2.070800in}}%
\pgfpathlineto{\pgfqpoint{3.646057in}{2.061211in}}%
\pgfpathclose%
\pgfusepath{fill}%
\end{pgfscope}%
\begin{pgfscope}%
\pgfpathrectangle{\pgfqpoint{1.254980in}{0.150000in}}{\pgfqpoint{5.490039in}{5.490039in}}%
\pgfusepath{clip}%
\pgfsetbuttcap%
\pgfsetroundjoin%
\definecolor{currentfill}{rgb}{0.221989,0.339161,0.548752}%
\pgfsetfillcolor{currentfill}%
\pgfsetfillopacity{0.700000}%
\pgfsetlinewidth{0.000000pt}%
\definecolor{currentstroke}{rgb}{0.000000,0.000000,0.000000}%
\pgfsetstrokecolor{currentstroke}%
\pgfsetdash{}{0pt}%
\pgfpathmoveto{\pgfqpoint{4.644938in}{2.589622in}}%
\pgfpathlineto{\pgfqpoint{4.658365in}{2.594845in}}%
\pgfpathlineto{\pgfqpoint{4.671804in}{2.600229in}}%
\pgfpathlineto{\pgfqpoint{4.685256in}{2.605773in}}%
\pgfpathlineto{\pgfqpoint{4.698720in}{2.611477in}}%
\pgfpathlineto{\pgfqpoint{4.706116in}{2.619095in}}%
\pgfpathlineto{\pgfqpoint{4.713507in}{2.626681in}}%
\pgfpathlineto{\pgfqpoint{4.720891in}{2.634236in}}%
\pgfpathlineto{\pgfqpoint{4.728270in}{2.641764in}}%
\pgfpathlineto{\pgfqpoint{4.714816in}{2.636295in}}%
\pgfpathlineto{\pgfqpoint{4.701375in}{2.630986in}}%
\pgfpathlineto{\pgfqpoint{4.687947in}{2.625837in}}%
\pgfpathlineto{\pgfqpoint{4.674531in}{2.620848in}}%
\pgfpathlineto{\pgfqpoint{4.667141in}{2.613076in}}%
\pgfpathlineto{\pgfqpoint{4.659746in}{2.605282in}}%
\pgfpathlineto{\pgfqpoint{4.652345in}{2.597465in}}%
\pgfpathlineto{\pgfqpoint{4.644938in}{2.589622in}}%
\pgfpathclose%
\pgfusepath{fill}%
\end{pgfscope}%
\begin{pgfscope}%
\pgfpathrectangle{\pgfqpoint{1.254980in}{0.150000in}}{\pgfqpoint{5.490039in}{5.490039in}}%
\pgfusepath{clip}%
\pgfsetbuttcap%
\pgfsetroundjoin%
\definecolor{currentfill}{rgb}{0.123444,0.636809,0.528763}%
\pgfsetfillcolor{currentfill}%
\pgfsetfillopacity{0.700000}%
\pgfsetlinewidth{0.000000pt}%
\definecolor{currentstroke}{rgb}{0.000000,0.000000,0.000000}%
\pgfsetstrokecolor{currentstroke}%
\pgfsetdash{}{0pt}%
\pgfpathmoveto{\pgfqpoint{5.923857in}{3.362086in}}%
\pgfpathlineto{\pgfqpoint{5.937857in}{3.369313in}}%
\pgfpathlineto{\pgfqpoint{5.951874in}{3.376689in}}%
\pgfpathlineto{\pgfqpoint{5.965908in}{3.384214in}}%
\pgfpathlineto{\pgfqpoint{5.979960in}{3.391887in}}%
\pgfpathlineto{\pgfqpoint{5.986767in}{3.396015in}}%
\pgfpathlineto{\pgfqpoint{5.993575in}{3.400340in}}%
\pgfpathlineto{\pgfqpoint{6.000385in}{3.404869in}}%
\pgfpathlineto{\pgfqpoint{5.986361in}{3.397717in}}%
\pgfpathlineto{\pgfqpoint{5.972354in}{3.390713in}}%
\pgfpathlineto{\pgfqpoint{5.958365in}{3.383857in}}%
\pgfpathlineto{\pgfqpoint{5.944392in}{3.377149in}}%
\pgfpathlineto{\pgfqpoint{5.937545in}{3.371920in}}%
\pgfpathlineto{\pgfqpoint{5.930700in}{3.366902in}}%
\pgfpathlineto{\pgfqpoint{5.923857in}{3.362086in}}%
\pgfpathclose%
\pgfusepath{fill}%
\end{pgfscope}%
\begin{pgfscope}%
\pgfpathrectangle{\pgfqpoint{1.254980in}{0.150000in}}{\pgfqpoint{5.490039in}{5.490039in}}%
\pgfusepath{clip}%
\pgfsetbuttcap%
\pgfsetroundjoin%
\definecolor{currentfill}{rgb}{0.210503,0.363727,0.552206}%
\pgfsetfillcolor{currentfill}%
\pgfsetfillopacity{0.700000}%
\pgfsetlinewidth{0.000000pt}%
\definecolor{currentstroke}{rgb}{0.000000,0.000000,0.000000}%
\pgfsetstrokecolor{currentstroke}%
\pgfsetdash{}{0pt}%
\pgfpathmoveto{\pgfqpoint{4.728270in}{2.641764in}}%
\pgfpathlineto{\pgfqpoint{4.741737in}{2.647392in}}%
\pgfpathlineto{\pgfqpoint{4.755217in}{2.653180in}}%
\pgfpathlineto{\pgfqpoint{4.768710in}{2.659127in}}%
\pgfpathlineto{\pgfqpoint{4.782217in}{2.665233in}}%
\pgfpathlineto{\pgfqpoint{4.789578in}{2.672483in}}%
\pgfpathlineto{\pgfqpoint{4.796934in}{2.679704in}}%
\pgfpathlineto{\pgfqpoint{4.804284in}{2.686900in}}%
\pgfpathlineto{\pgfqpoint{4.811629in}{2.694073in}}%
\pgfpathlineto{\pgfqpoint{4.798134in}{2.688232in}}%
\pgfpathlineto{\pgfqpoint{4.784653in}{2.682549in}}%
\pgfpathlineto{\pgfqpoint{4.771184in}{2.677025in}}%
\pgfpathlineto{\pgfqpoint{4.757729in}{2.671660in}}%
\pgfpathlineto{\pgfqpoint{4.750373in}{2.664212in}}%
\pgfpathlineto{\pgfqpoint{4.743011in}{2.656749in}}%
\pgfpathlineto{\pgfqpoint{4.735643in}{2.649267in}}%
\pgfpathlineto{\pgfqpoint{4.728270in}{2.641764in}}%
\pgfpathclose%
\pgfusepath{fill}%
\end{pgfscope}%
\begin{pgfscope}%
\pgfpathrectangle{\pgfqpoint{1.254980in}{0.150000in}}{\pgfqpoint{5.490039in}{5.490039in}}%
\pgfusepath{clip}%
\pgfsetbuttcap%
\pgfsetroundjoin%
\definecolor{currentfill}{rgb}{0.282656,0.100196,0.422160}%
\pgfsetfillcolor{currentfill}%
\pgfsetfillopacity{0.700000}%
\pgfsetlinewidth{0.000000pt}%
\definecolor{currentstroke}{rgb}{0.000000,0.000000,0.000000}%
\pgfsetstrokecolor{currentstroke}%
\pgfsetdash{}{0pt}%
\pgfpathmoveto{\pgfqpoint{3.017951in}{2.118678in}}%
\pgfpathlineto{\pgfqpoint{3.031055in}{2.108254in}}%
\pgfpathlineto{\pgfqpoint{3.044157in}{2.098050in}}%
\pgfpathlineto{\pgfqpoint{3.057257in}{2.088064in}}%
\pgfpathlineto{\pgfqpoint{3.070356in}{2.078296in}}%
\pgfpathlineto{\pgfqpoint{3.078363in}{2.085399in}}%
\pgfpathlineto{\pgfqpoint{3.086362in}{2.092596in}}%
\pgfpathlineto{\pgfqpoint{3.094353in}{2.099885in}}%
\pgfpathlineto{\pgfqpoint{3.102336in}{2.107264in}}%
\pgfpathlineto{\pgfqpoint{3.089258in}{2.116784in}}%
\pgfpathlineto{\pgfqpoint{3.076179in}{2.126521in}}%
\pgfpathlineto{\pgfqpoint{3.063098in}{2.136477in}}%
\pgfpathlineto{\pgfqpoint{3.050016in}{2.146652in}}%
\pgfpathlineto{\pgfqpoint{3.042012in}{2.139511in}}%
\pgfpathlineto{\pgfqpoint{3.034000in}{2.132467in}}%
\pgfpathlineto{\pgfqpoint{3.025979in}{2.125522in}}%
\pgfpathlineto{\pgfqpoint{3.017951in}{2.118678in}}%
\pgfpathclose%
\pgfusepath{fill}%
\end{pgfscope}%
\begin{pgfscope}%
\pgfpathrectangle{\pgfqpoint{1.254980in}{0.150000in}}{\pgfqpoint{5.490039in}{5.490039in}}%
\pgfusepath{clip}%
\pgfsetbuttcap%
\pgfsetroundjoin%
\definecolor{currentfill}{rgb}{0.279566,0.067836,0.391917}%
\pgfsetfillcolor{currentfill}%
\pgfsetfillopacity{0.700000}%
\pgfsetlinewidth{0.000000pt}%
\definecolor{currentstroke}{rgb}{0.000000,0.000000,0.000000}%
\pgfsetstrokecolor{currentstroke}%
\pgfsetdash{}{0pt}%
\pgfpathmoveto{\pgfqpoint{3.562628in}{2.034575in}}%
\pgfpathlineto{\pgfqpoint{3.575707in}{2.031397in}}%
\pgfpathlineto{\pgfqpoint{3.588792in}{2.028403in}}%
\pgfpathlineto{\pgfqpoint{3.601880in}{2.025592in}}%
\pgfpathlineto{\pgfqpoint{3.614974in}{2.022964in}}%
\pgfpathlineto{\pgfqpoint{3.622752in}{2.032508in}}%
\pgfpathlineto{\pgfqpoint{3.630526in}{2.042065in}}%
\pgfpathlineto{\pgfqpoint{3.638294in}{2.051633in}}%
\pgfpathlineto{\pgfqpoint{3.646057in}{2.061211in}}%
\pgfpathlineto{\pgfqpoint{3.632974in}{2.063706in}}%
\pgfpathlineto{\pgfqpoint{3.619896in}{2.066383in}}%
\pgfpathlineto{\pgfqpoint{3.606823in}{2.069244in}}%
\pgfpathlineto{\pgfqpoint{3.593755in}{2.072289in}}%
\pgfpathlineto{\pgfqpoint{3.585981in}{2.062834in}}%
\pgfpathlineto{\pgfqpoint{3.578202in}{2.053396in}}%
\pgfpathlineto{\pgfqpoint{3.570418in}{2.043976in}}%
\pgfpathlineto{\pgfqpoint{3.562628in}{2.034575in}}%
\pgfpathclose%
\pgfusepath{fill}%
\end{pgfscope}%
\begin{pgfscope}%
\pgfpathrectangle{\pgfqpoint{1.254980in}{0.150000in}}{\pgfqpoint{5.490039in}{5.490039in}}%
\pgfusepath{clip}%
\pgfsetbuttcap%
\pgfsetroundjoin%
\definecolor{currentfill}{rgb}{0.199430,0.387607,0.554642}%
\pgfsetfillcolor{currentfill}%
\pgfsetfillopacity{0.700000}%
\pgfsetlinewidth{0.000000pt}%
\definecolor{currentstroke}{rgb}{0.000000,0.000000,0.000000}%
\pgfsetstrokecolor{currentstroke}%
\pgfsetdash{}{0pt}%
\pgfpathmoveto{\pgfqpoint{4.811629in}{2.694073in}}%
\pgfpathlineto{\pgfqpoint{4.825137in}{2.700073in}}%
\pgfpathlineto{\pgfqpoint{4.838659in}{2.706232in}}%
\pgfpathlineto{\pgfqpoint{4.852194in}{2.712550in}}%
\pgfpathlineto{\pgfqpoint{4.865743in}{2.719025in}}%
\pgfpathlineto{\pgfqpoint{4.873069in}{2.725897in}}%
\pgfpathlineto{\pgfqpoint{4.880389in}{2.732746in}}%
\pgfpathlineto{\pgfqpoint{4.887704in}{2.739577in}}%
\pgfpathlineto{\pgfqpoint{4.895012in}{2.746392in}}%
\pgfpathlineto{\pgfqpoint{4.881476in}{2.740210in}}%
\pgfpathlineto{\pgfqpoint{4.867954in}{2.734186in}}%
\pgfpathlineto{\pgfqpoint{4.854445in}{2.728320in}}%
\pgfpathlineto{\pgfqpoint{4.840949in}{2.722612in}}%
\pgfpathlineto{\pgfqpoint{4.833628in}{2.715494in}}%
\pgfpathlineto{\pgfqpoint{4.826300in}{2.708367in}}%
\pgfpathlineto{\pgfqpoint{4.818967in}{2.701228in}}%
\pgfpathlineto{\pgfqpoint{4.811629in}{2.694073in}}%
\pgfpathclose%
\pgfusepath{fill}%
\end{pgfscope}%
\begin{pgfscope}%
\pgfpathrectangle{\pgfqpoint{1.254980in}{0.150000in}}{\pgfqpoint{5.490039in}{5.490039in}}%
\pgfusepath{clip}%
\pgfsetbuttcap%
\pgfsetroundjoin%
\definecolor{currentfill}{rgb}{0.277018,0.050344,0.375715}%
\pgfsetfillcolor{currentfill}%
\pgfsetfillopacity{0.700000}%
\pgfsetlinewidth{0.000000pt}%
\definecolor{currentstroke}{rgb}{0.000000,0.000000,0.000000}%
\pgfsetstrokecolor{currentstroke}%
\pgfsetdash{}{0pt}%
\pgfpathmoveto{\pgfqpoint{3.343121in}{2.017250in}}%
\pgfpathlineto{\pgfqpoint{3.356189in}{2.011441in}}%
\pgfpathlineto{\pgfqpoint{3.369258in}{2.005827in}}%
\pgfpathlineto{\pgfqpoint{3.382330in}{2.000407in}}%
\pgfpathlineto{\pgfqpoint{3.395404in}{1.995180in}}%
\pgfpathlineto{\pgfqpoint{3.403266in}{2.003983in}}%
\pgfpathlineto{\pgfqpoint{3.411123in}{2.012829in}}%
\pgfpathlineto{\pgfqpoint{3.418973in}{2.021717in}}%
\pgfpathlineto{\pgfqpoint{3.426817in}{2.030646in}}%
\pgfpathlineto{\pgfqpoint{3.413757in}{2.035683in}}%
\pgfpathlineto{\pgfqpoint{3.400699in}{2.040914in}}%
\pgfpathlineto{\pgfqpoint{3.387644in}{2.046339in}}%
\pgfpathlineto{\pgfqpoint{3.374592in}{2.051958in}}%
\pgfpathlineto{\pgfqpoint{3.366734in}{2.043208in}}%
\pgfpathlineto{\pgfqpoint{3.358869in}{2.034506in}}%
\pgfpathlineto{\pgfqpoint{3.350998in}{2.025853in}}%
\pgfpathlineto{\pgfqpoint{3.343121in}{2.017250in}}%
\pgfpathclose%
\pgfusepath{fill}%
\end{pgfscope}%
\begin{pgfscope}%
\pgfpathrectangle{\pgfqpoint{1.254980in}{0.150000in}}{\pgfqpoint{5.490039in}{5.490039in}}%
\pgfusepath{clip}%
\pgfsetbuttcap%
\pgfsetroundjoin%
\definecolor{currentfill}{rgb}{0.278791,0.062145,0.386592}%
\pgfsetfillcolor{currentfill}%
\pgfsetfillopacity{0.700000}%
\pgfsetlinewidth{0.000000pt}%
\definecolor{currentstroke}{rgb}{0.000000,0.000000,0.000000}%
\pgfsetstrokecolor{currentstroke}%
\pgfsetdash{}{0pt}%
\pgfpathmoveto{\pgfqpoint{3.206932in}{2.038752in}}%
\pgfpathlineto{\pgfqpoint{3.220006in}{2.031124in}}%
\pgfpathlineto{\pgfqpoint{3.233080in}{2.023701in}}%
\pgfpathlineto{\pgfqpoint{3.246156in}{2.016481in}}%
\pgfpathlineto{\pgfqpoint{3.259232in}{2.009462in}}%
\pgfpathlineto{\pgfqpoint{3.267151in}{2.017616in}}%
\pgfpathlineto{\pgfqpoint{3.275064in}{2.025835in}}%
\pgfpathlineto{\pgfqpoint{3.282971in}{2.034117in}}%
\pgfpathlineto{\pgfqpoint{3.290870in}{2.042460in}}%
\pgfpathlineto{\pgfqpoint{3.277811in}{2.049261in}}%
\pgfpathlineto{\pgfqpoint{3.264753in}{2.056263in}}%
\pgfpathlineto{\pgfqpoint{3.251695in}{2.063468in}}%
\pgfpathlineto{\pgfqpoint{3.238639in}{2.070877in}}%
\pgfpathlineto{\pgfqpoint{3.230723in}{2.062742in}}%
\pgfpathlineto{\pgfqpoint{3.222799in}{2.054675in}}%
\pgfpathlineto{\pgfqpoint{3.214869in}{2.046677in}}%
\pgfpathlineto{\pgfqpoint{3.206932in}{2.038752in}}%
\pgfpathclose%
\pgfusepath{fill}%
\end{pgfscope}%
\begin{pgfscope}%
\pgfpathrectangle{\pgfqpoint{1.254980in}{0.150000in}}{\pgfqpoint{5.490039in}{5.490039in}}%
\pgfusepath{clip}%
\pgfsetbuttcap%
\pgfsetroundjoin%
\definecolor{currentfill}{rgb}{0.190631,0.407061,0.556089}%
\pgfsetfillcolor{currentfill}%
\pgfsetfillopacity{0.700000}%
\pgfsetlinewidth{0.000000pt}%
\definecolor{currentstroke}{rgb}{0.000000,0.000000,0.000000}%
\pgfsetstrokecolor{currentstroke}%
\pgfsetdash{}{0pt}%
\pgfpathmoveto{\pgfqpoint{4.895012in}{2.746392in}}%
\pgfpathlineto{\pgfqpoint{4.908562in}{2.752731in}}%
\pgfpathlineto{\pgfqpoint{4.922126in}{2.759228in}}%
\pgfpathlineto{\pgfqpoint{4.935704in}{2.765883in}}%
\pgfpathlineto{\pgfqpoint{4.949297in}{2.772696in}}%
\pgfpathlineto{\pgfqpoint{4.956586in}{2.779186in}}%
\pgfpathlineto{\pgfqpoint{4.963869in}{2.785662in}}%
\pgfpathlineto{\pgfqpoint{4.971147in}{2.792125in}}%
\pgfpathlineto{\pgfqpoint{4.978419in}{2.798582in}}%
\pgfpathlineto{\pgfqpoint{4.964841in}{2.792092in}}%
\pgfpathlineto{\pgfqpoint{4.951277in}{2.785760in}}%
\pgfpathlineto{\pgfqpoint{4.937727in}{2.779585in}}%
\pgfpathlineto{\pgfqpoint{4.924191in}{2.773567in}}%
\pgfpathlineto{\pgfqpoint{4.916904in}{2.766779in}}%
\pgfpathlineto{\pgfqpoint{4.909613in}{2.759989in}}%
\pgfpathlineto{\pgfqpoint{4.902315in}{2.753195in}}%
\pgfpathlineto{\pgfqpoint{4.895012in}{2.746392in}}%
\pgfpathclose%
\pgfusepath{fill}%
\end{pgfscope}%
\begin{pgfscope}%
\pgfpathrectangle{\pgfqpoint{1.254980in}{0.150000in}}{\pgfqpoint{5.490039in}{5.490039in}}%
\pgfusepath{clip}%
\pgfsetbuttcap%
\pgfsetroundjoin%
\definecolor{currentfill}{rgb}{0.180629,0.429975,0.557282}%
\pgfsetfillcolor{currentfill}%
\pgfsetfillopacity{0.700000}%
\pgfsetlinewidth{0.000000pt}%
\definecolor{currentstroke}{rgb}{0.000000,0.000000,0.000000}%
\pgfsetstrokecolor{currentstroke}%
\pgfsetdash{}{0pt}%
\pgfpathmoveto{\pgfqpoint{4.978419in}{2.798582in}}%
\pgfpathlineto{\pgfqpoint{4.992011in}{2.805228in}}%
\pgfpathlineto{\pgfqpoint{5.005617in}{2.812031in}}%
\pgfpathlineto{\pgfqpoint{5.019238in}{2.818991in}}%
\pgfpathlineto{\pgfqpoint{5.032874in}{2.826107in}}%
\pgfpathlineto{\pgfqpoint{5.040125in}{2.832218in}}%
\pgfpathlineto{\pgfqpoint{5.047371in}{2.838321in}}%
\pgfpathlineto{\pgfqpoint{5.054610in}{2.844422in}}%
\pgfpathlineto{\pgfqpoint{5.061844in}{2.850524in}}%
\pgfpathlineto{\pgfqpoint{5.048224in}{2.843760in}}%
\pgfpathlineto{\pgfqpoint{5.034619in}{2.837152in}}%
\pgfpathlineto{\pgfqpoint{5.021027in}{2.830701in}}%
\pgfpathlineto{\pgfqpoint{5.007450in}{2.824406in}}%
\pgfpathlineto{\pgfqpoint{5.000201in}{2.817942in}}%
\pgfpathlineto{\pgfqpoint{4.992945in}{2.811486in}}%
\pgfpathlineto{\pgfqpoint{4.985685in}{2.805034in}}%
\pgfpathlineto{\pgfqpoint{4.978419in}{2.798582in}}%
\pgfpathclose%
\pgfusepath{fill}%
\end{pgfscope}%
\begin{pgfscope}%
\pgfpathrectangle{\pgfqpoint{1.254980in}{0.150000in}}{\pgfqpoint{5.490039in}{5.490039in}}%
\pgfusepath{clip}%
\pgfsetbuttcap%
\pgfsetroundjoin%
\definecolor{currentfill}{rgb}{0.277941,0.056324,0.381191}%
\pgfsetfillcolor{currentfill}%
\pgfsetfillopacity{0.700000}%
\pgfsetlinewidth{0.000000pt}%
\definecolor{currentstroke}{rgb}{0.000000,0.000000,0.000000}%
\pgfsetstrokecolor{currentstroke}%
\pgfsetdash{}{0pt}%
\pgfpathmoveto{\pgfqpoint{3.479087in}{2.012409in}}%
\pgfpathlineto{\pgfqpoint{3.492162in}{2.008324in}}%
\pgfpathlineto{\pgfqpoint{3.505242in}{2.004426in}}%
\pgfpathlineto{\pgfqpoint{3.518325in}{2.000715in}}%
\pgfpathlineto{\pgfqpoint{3.531413in}{1.997190in}}%
\pgfpathlineto{\pgfqpoint{3.539225in}{2.006502in}}%
\pgfpathlineto{\pgfqpoint{3.547031in}{2.015837in}}%
\pgfpathlineto{\pgfqpoint{3.554832in}{2.025195in}}%
\pgfpathlineto{\pgfqpoint{3.562628in}{2.034575in}}%
\pgfpathlineto{\pgfqpoint{3.549553in}{2.037939in}}%
\pgfpathlineto{\pgfqpoint{3.536481in}{2.041489in}}%
\pgfpathlineto{\pgfqpoint{3.523414in}{2.045226in}}%
\pgfpathlineto{\pgfqpoint{3.510351in}{2.049150in}}%
\pgfpathlineto{\pgfqpoint{3.502543in}{2.039921in}}%
\pgfpathlineto{\pgfqpoint{3.494730in}{2.030720in}}%
\pgfpathlineto{\pgfqpoint{3.486911in}{2.021549in}}%
\pgfpathlineto{\pgfqpoint{3.479087in}{2.012409in}}%
\pgfpathclose%
\pgfusepath{fill}%
\end{pgfscope}%
\begin{pgfscope}%
\pgfpathrectangle{\pgfqpoint{1.254980in}{0.150000in}}{\pgfqpoint{5.490039in}{5.490039in}}%
\pgfusepath{clip}%
\pgfsetbuttcap%
\pgfsetroundjoin%
\definecolor{currentfill}{rgb}{0.280255,0.165693,0.476498}%
\pgfsetfillcolor{currentfill}%
\pgfsetfillopacity{0.700000}%
\pgfsetlinewidth{0.000000pt}%
\definecolor{currentstroke}{rgb}{0.000000,0.000000,0.000000}%
\pgfsetstrokecolor{currentstroke}%
\pgfsetdash{}{0pt}%
\pgfpathmoveto{\pgfqpoint{4.031795in}{2.208010in}}%
\pgfpathlineto{\pgfqpoint{4.044987in}{2.209463in}}%
\pgfpathlineto{\pgfqpoint{4.058188in}{2.211086in}}%
\pgfpathlineto{\pgfqpoint{4.071397in}{2.212878in}}%
\pgfpathlineto{\pgfqpoint{4.084616in}{2.214840in}}%
\pgfpathlineto{\pgfqpoint{4.092242in}{2.224553in}}%
\pgfpathlineto{\pgfqpoint{4.099863in}{2.234231in}}%
\pgfpathlineto{\pgfqpoint{4.107479in}{2.243874in}}%
\pgfpathlineto{\pgfqpoint{4.115090in}{2.253482in}}%
\pgfpathlineto{\pgfqpoint{4.101879in}{2.251527in}}%
\pgfpathlineto{\pgfqpoint{4.088677in}{2.249742in}}%
\pgfpathlineto{\pgfqpoint{4.075483in}{2.248125in}}%
\pgfpathlineto{\pgfqpoint{4.062299in}{2.246679in}}%
\pgfpathlineto{\pgfqpoint{4.054680in}{2.237054in}}%
\pgfpathlineto{\pgfqpoint{4.047057in}{2.227401in}}%
\pgfpathlineto{\pgfqpoint{4.039428in}{2.217720in}}%
\pgfpathlineto{\pgfqpoint{4.031795in}{2.208010in}}%
\pgfpathclose%
\pgfusepath{fill}%
\end{pgfscope}%
\begin{pgfscope}%
\pgfpathrectangle{\pgfqpoint{1.254980in}{0.150000in}}{\pgfqpoint{5.490039in}{5.490039in}}%
\pgfusepath{clip}%
\pgfsetbuttcap%
\pgfsetroundjoin%
\definecolor{currentfill}{rgb}{0.282623,0.140926,0.457517}%
\pgfsetfillcolor{currentfill}%
\pgfsetfillopacity{0.700000}%
\pgfsetlinewidth{0.000000pt}%
\definecolor{currentstroke}{rgb}{0.000000,0.000000,0.000000}%
\pgfsetstrokecolor{currentstroke}%
\pgfsetdash{}{0pt}%
\pgfpathmoveto{\pgfqpoint{3.948496in}{2.164867in}}%
\pgfpathlineto{\pgfqpoint{3.961663in}{2.165613in}}%
\pgfpathlineto{\pgfqpoint{3.974837in}{2.166532in}}%
\pgfpathlineto{\pgfqpoint{3.988020in}{2.167622in}}%
\pgfpathlineto{\pgfqpoint{4.001211in}{2.168882in}}%
\pgfpathlineto{\pgfqpoint{4.008864in}{2.178708in}}%
\pgfpathlineto{\pgfqpoint{4.016513in}{2.188505in}}%
\pgfpathlineto{\pgfqpoint{4.024156in}{2.198272in}}%
\pgfpathlineto{\pgfqpoint{4.031795in}{2.208010in}}%
\pgfpathlineto{\pgfqpoint{4.018611in}{2.206728in}}%
\pgfpathlineto{\pgfqpoint{4.005436in}{2.205617in}}%
\pgfpathlineto{\pgfqpoint{3.992269in}{2.204677in}}%
\pgfpathlineto{\pgfqpoint{3.979110in}{2.203908in}}%
\pgfpathlineto{\pgfqpoint{3.971464in}{2.194181in}}%
\pgfpathlineto{\pgfqpoint{3.963813in}{2.184432in}}%
\pgfpathlineto{\pgfqpoint{3.956157in}{2.174661in}}%
\pgfpathlineto{\pgfqpoint{3.948496in}{2.164867in}}%
\pgfpathclose%
\pgfusepath{fill}%
\end{pgfscope}%
\begin{pgfscope}%
\pgfpathrectangle{\pgfqpoint{1.254980in}{0.150000in}}{\pgfqpoint{5.490039in}{5.490039in}}%
\pgfusepath{clip}%
\pgfsetbuttcap%
\pgfsetroundjoin%
\definecolor{currentfill}{rgb}{0.281446,0.084320,0.407414}%
\pgfsetfillcolor{currentfill}%
\pgfsetfillopacity{0.700000}%
\pgfsetlinewidth{0.000000pt}%
\definecolor{currentstroke}{rgb}{0.000000,0.000000,0.000000}%
\pgfsetstrokecolor{currentstroke}%
\pgfsetdash{}{0pt}%
\pgfpathmoveto{\pgfqpoint{3.070356in}{2.078296in}}%
\pgfpathlineto{\pgfqpoint{3.083453in}{2.068744in}}%
\pgfpathlineto{\pgfqpoint{3.096549in}{2.059406in}}%
\pgfpathlineto{\pgfqpoint{3.109644in}{2.050281in}}%
\pgfpathlineto{\pgfqpoint{3.122738in}{2.041368in}}%
\pgfpathlineto{\pgfqpoint{3.130724in}{2.048729in}}%
\pgfpathlineto{\pgfqpoint{3.138703in}{2.056177in}}%
\pgfpathlineto{\pgfqpoint{3.146674in}{2.063710in}}%
\pgfpathlineto{\pgfqpoint{3.154637in}{2.071326in}}%
\pgfpathlineto{\pgfqpoint{3.141563in}{2.079992in}}%
\pgfpathlineto{\pgfqpoint{3.128488in}{2.088869in}}%
\pgfpathlineto{\pgfqpoint{3.115412in}{2.097959in}}%
\pgfpathlineto{\pgfqpoint{3.102336in}{2.107264in}}%
\pgfpathlineto{\pgfqpoint{3.094353in}{2.099885in}}%
\pgfpathlineto{\pgfqpoint{3.086362in}{2.092596in}}%
\pgfpathlineto{\pgfqpoint{3.078363in}{2.085399in}}%
\pgfpathlineto{\pgfqpoint{3.070356in}{2.078296in}}%
\pgfpathclose%
\pgfusepath{fill}%
\end{pgfscope}%
\begin{pgfscope}%
\pgfpathrectangle{\pgfqpoint{1.254980in}{0.150000in}}{\pgfqpoint{5.490039in}{5.490039in}}%
\pgfusepath{clip}%
\pgfsetbuttcap%
\pgfsetroundjoin%
\definecolor{currentfill}{rgb}{0.276194,0.190074,0.493001}%
\pgfsetfillcolor{currentfill}%
\pgfsetfillopacity{0.700000}%
\pgfsetlinewidth{0.000000pt}%
\definecolor{currentstroke}{rgb}{0.000000,0.000000,0.000000}%
\pgfsetstrokecolor{currentstroke}%
\pgfsetdash{}{0pt}%
\pgfpathmoveto{\pgfqpoint{4.115090in}{2.253482in}}%
\pgfpathlineto{\pgfqpoint{4.128311in}{2.255606in}}%
\pgfpathlineto{\pgfqpoint{4.141540in}{2.257898in}}%
\pgfpathlineto{\pgfqpoint{4.154780in}{2.260357in}}%
\pgfpathlineto{\pgfqpoint{4.168028in}{2.262985in}}%
\pgfpathlineto{\pgfqpoint{4.175627in}{2.272535in}}%
\pgfpathlineto{\pgfqpoint{4.183221in}{2.282046in}}%
\pgfpathlineto{\pgfqpoint{4.190810in}{2.291517in}}%
\pgfpathlineto{\pgfqpoint{4.198393in}{2.300949in}}%
\pgfpathlineto{\pgfqpoint{4.185152in}{2.298357in}}%
\pgfpathlineto{\pgfqpoint{4.171920in}{2.295932in}}%
\pgfpathlineto{\pgfqpoint{4.158697in}{2.293675in}}%
\pgfpathlineto{\pgfqpoint{4.145484in}{2.291586in}}%
\pgfpathlineto{\pgfqpoint{4.137893in}{2.282108in}}%
\pgfpathlineto{\pgfqpoint{4.130297in}{2.272599in}}%
\pgfpathlineto{\pgfqpoint{4.122696in}{2.263057in}}%
\pgfpathlineto{\pgfqpoint{4.115090in}{2.253482in}}%
\pgfpathclose%
\pgfusepath{fill}%
\end{pgfscope}%
\begin{pgfscope}%
\pgfpathrectangle{\pgfqpoint{1.254980in}{0.150000in}}{\pgfqpoint{5.490039in}{5.490039in}}%
\pgfusepath{clip}%
\pgfsetbuttcap%
\pgfsetroundjoin%
\definecolor{currentfill}{rgb}{0.269308,0.218818,0.509577}%
\pgfsetfillcolor{currentfill}%
\pgfsetfillopacity{0.700000}%
\pgfsetlinewidth{0.000000pt}%
\definecolor{currentstroke}{rgb}{0.000000,0.000000,0.000000}%
\pgfsetstrokecolor{currentstroke}%
\pgfsetdash{}{0pt}%
\pgfpathmoveto{\pgfqpoint{2.722393in}{2.355234in}}%
\pgfpathlineto{\pgfqpoint{2.735612in}{2.339766in}}%
\pgfpathlineto{\pgfqpoint{2.748824in}{2.324553in}}%
\pgfpathlineto{\pgfqpoint{2.762030in}{2.309596in}}%
\pgfpathlineto{\pgfqpoint{2.775229in}{2.294891in}}%
\pgfpathlineto{\pgfqpoint{2.783397in}{2.300144in}}%
\pgfpathlineto{\pgfqpoint{2.791554in}{2.305537in}}%
\pgfpathlineto{\pgfqpoint{2.799701in}{2.311066in}}%
\pgfpathlineto{\pgfqpoint{2.807837in}{2.316730in}}%
\pgfpathlineto{\pgfqpoint{2.794666in}{2.331152in}}%
\pgfpathlineto{\pgfqpoint{2.781489in}{2.345826in}}%
\pgfpathlineto{\pgfqpoint{2.768305in}{2.360754in}}%
\pgfpathlineto{\pgfqpoint{2.755115in}{2.375938in}}%
\pgfpathlineto{\pgfqpoint{2.746951in}{2.370547in}}%
\pgfpathlineto{\pgfqpoint{2.738776in}{2.365298in}}%
\pgfpathlineto{\pgfqpoint{2.730590in}{2.360193in}}%
\pgfpathlineto{\pgfqpoint{2.722393in}{2.355234in}}%
\pgfpathclose%
\pgfusepath{fill}%
\end{pgfscope}%
\begin{pgfscope}%
\pgfpathrectangle{\pgfqpoint{1.254980in}{0.150000in}}{\pgfqpoint{5.490039in}{5.490039in}}%
\pgfusepath{clip}%
\pgfsetbuttcap%
\pgfsetroundjoin%
\definecolor{currentfill}{rgb}{0.283229,0.120777,0.440584}%
\pgfsetfillcolor{currentfill}%
\pgfsetfillopacity{0.700000}%
\pgfsetlinewidth{0.000000pt}%
\definecolor{currentstroke}{rgb}{0.000000,0.000000,0.000000}%
\pgfsetstrokecolor{currentstroke}%
\pgfsetdash{}{0pt}%
\pgfpathmoveto{\pgfqpoint{3.865182in}{2.124405in}}%
\pgfpathlineto{\pgfqpoint{3.878326in}{2.124410in}}%
\pgfpathlineto{\pgfqpoint{3.891477in}{2.124588in}}%
\pgfpathlineto{\pgfqpoint{3.904636in}{2.124940in}}%
\pgfpathlineto{\pgfqpoint{3.917803in}{2.125465in}}%
\pgfpathlineto{\pgfqpoint{3.925483in}{2.135349in}}%
\pgfpathlineto{\pgfqpoint{3.933159in}{2.145211in}}%
\pgfpathlineto{\pgfqpoint{3.940830in}{2.155050in}}%
\pgfpathlineto{\pgfqpoint{3.948496in}{2.164867in}}%
\pgfpathlineto{\pgfqpoint{3.935338in}{2.164293in}}%
\pgfpathlineto{\pgfqpoint{3.922187in}{2.163891in}}%
\pgfpathlineto{\pgfqpoint{3.909043in}{2.163663in}}%
\pgfpathlineto{\pgfqpoint{3.895907in}{2.163609in}}%
\pgfpathlineto{\pgfqpoint{3.888233in}{2.153832in}}%
\pgfpathlineto{\pgfqpoint{3.880555in}{2.144039in}}%
\pgfpathlineto{\pgfqpoint{3.872871in}{2.134230in}}%
\pgfpathlineto{\pgfqpoint{3.865182in}{2.124405in}}%
\pgfpathclose%
\pgfusepath{fill}%
\end{pgfscope}%
\begin{pgfscope}%
\pgfpathrectangle{\pgfqpoint{1.254980in}{0.150000in}}{\pgfqpoint{5.490039in}{5.490039in}}%
\pgfusepath{clip}%
\pgfsetbuttcap%
\pgfsetroundjoin%
\definecolor{currentfill}{rgb}{0.260571,0.246922,0.522828}%
\pgfsetfillcolor{currentfill}%
\pgfsetfillopacity{0.700000}%
\pgfsetlinewidth{0.000000pt}%
\definecolor{currentstroke}{rgb}{0.000000,0.000000,0.000000}%
\pgfsetstrokecolor{currentstroke}%
\pgfsetdash{}{0pt}%
\pgfpathmoveto{\pgfqpoint{2.669444in}{2.419720in}}%
\pgfpathlineto{\pgfqpoint{2.682693in}{2.403203in}}%
\pgfpathlineto{\pgfqpoint{2.695934in}{2.386951in}}%
\pgfpathlineto{\pgfqpoint{2.709167in}{2.370962in}}%
\pgfpathlineto{\pgfqpoint{2.722393in}{2.355234in}}%
\pgfpathlineto{\pgfqpoint{2.730590in}{2.360193in}}%
\pgfpathlineto{\pgfqpoint{2.738776in}{2.365298in}}%
\pgfpathlineto{\pgfqpoint{2.746951in}{2.370547in}}%
\pgfpathlineto{\pgfqpoint{2.755115in}{2.375938in}}%
\pgfpathlineto{\pgfqpoint{2.741918in}{2.391381in}}%
\pgfpathlineto{\pgfqpoint{2.728714in}{2.407084in}}%
\pgfpathlineto{\pgfqpoint{2.715503in}{2.423050in}}%
\pgfpathlineto{\pgfqpoint{2.702285in}{2.439281in}}%
\pgfpathlineto{\pgfqpoint{2.694091in}{2.434165in}}%
\pgfpathlineto{\pgfqpoint{2.685887in}{2.429198in}}%
\pgfpathlineto{\pgfqpoint{2.677671in}{2.424382in}}%
\pgfpathlineto{\pgfqpoint{2.669444in}{2.419720in}}%
\pgfpathclose%
\pgfusepath{fill}%
\end{pgfscope}%
\begin{pgfscope}%
\pgfpathrectangle{\pgfqpoint{1.254980in}{0.150000in}}{\pgfqpoint{5.490039in}{5.490039in}}%
\pgfusepath{clip}%
\pgfsetbuttcap%
\pgfsetroundjoin%
\definecolor{currentfill}{rgb}{0.172719,0.448791,0.557885}%
\pgfsetfillcolor{currentfill}%
\pgfsetfillopacity{0.700000}%
\pgfsetlinewidth{0.000000pt}%
\definecolor{currentstroke}{rgb}{0.000000,0.000000,0.000000}%
\pgfsetstrokecolor{currentstroke}%
\pgfsetdash{}{0pt}%
\pgfpathmoveto{\pgfqpoint{5.061844in}{2.850524in}}%
\pgfpathlineto{\pgfqpoint{5.075479in}{2.857444in}}%
\pgfpathlineto{\pgfqpoint{5.089128in}{2.864521in}}%
\pgfpathlineto{\pgfqpoint{5.102793in}{2.871753in}}%
\pgfpathlineto{\pgfqpoint{5.116472in}{2.879142in}}%
\pgfpathlineto{\pgfqpoint{5.123684in}{2.884878in}}%
\pgfpathlineto{\pgfqpoint{5.130890in}{2.890617in}}%
\pgfpathlineto{\pgfqpoint{5.138091in}{2.896364in}}%
\pgfpathlineto{\pgfqpoint{5.145286in}{2.902121in}}%
\pgfpathlineto{\pgfqpoint{5.131624in}{2.895115in}}%
\pgfpathlineto{\pgfqpoint{5.117977in}{2.888264in}}%
\pgfpathlineto{\pgfqpoint{5.104344in}{2.881568in}}%
\pgfpathlineto{\pgfqpoint{5.090726in}{2.875028in}}%
\pgfpathlineto{\pgfqpoint{5.083513in}{2.868879in}}%
\pgfpathlineto{\pgfqpoint{5.076296in}{2.862749in}}%
\pgfpathlineto{\pgfqpoint{5.069073in}{2.856632in}}%
\pgfpathlineto{\pgfqpoint{5.061844in}{2.850524in}}%
\pgfpathclose%
\pgfusepath{fill}%
\end{pgfscope}%
\begin{pgfscope}%
\pgfpathrectangle{\pgfqpoint{1.254980in}{0.150000in}}{\pgfqpoint{5.490039in}{5.490039in}}%
\pgfusepath{clip}%
\pgfsetbuttcap%
\pgfsetroundjoin%
\definecolor{currentfill}{rgb}{0.270595,0.214069,0.507052}%
\pgfsetfillcolor{currentfill}%
\pgfsetfillopacity{0.700000}%
\pgfsetlinewidth{0.000000pt}%
\definecolor{currentstroke}{rgb}{0.000000,0.000000,0.000000}%
\pgfsetstrokecolor{currentstroke}%
\pgfsetdash{}{0pt}%
\pgfpathmoveto{\pgfqpoint{4.198393in}{2.300949in}}%
\pgfpathlineto{\pgfqpoint{4.211645in}{2.303709in}}%
\pgfpathlineto{\pgfqpoint{4.224906in}{2.306635in}}%
\pgfpathlineto{\pgfqpoint{4.238177in}{2.309728in}}%
\pgfpathlineto{\pgfqpoint{4.251459in}{2.312987in}}%
\pgfpathlineto{\pgfqpoint{4.259030in}{2.322329in}}%
\pgfpathlineto{\pgfqpoint{4.266596in}{2.331628in}}%
\pgfpathlineto{\pgfqpoint{4.274157in}{2.340884in}}%
\pgfpathlineto{\pgfqpoint{4.281713in}{2.350098in}}%
\pgfpathlineto{\pgfqpoint{4.268438in}{2.346903in}}%
\pgfpathlineto{\pgfqpoint{4.255174in}{2.343874in}}%
\pgfpathlineto{\pgfqpoint{4.241920in}{2.341010in}}%
\pgfpathlineto{\pgfqpoint{4.228676in}{2.338314in}}%
\pgfpathlineto{\pgfqpoint{4.221113in}{2.329026in}}%
\pgfpathlineto{\pgfqpoint{4.213545in}{2.319703in}}%
\pgfpathlineto{\pgfqpoint{4.205972in}{2.310344in}}%
\pgfpathlineto{\pgfqpoint{4.198393in}{2.300949in}}%
\pgfpathclose%
\pgfusepath{fill}%
\end{pgfscope}%
\begin{pgfscope}%
\pgfpathrectangle{\pgfqpoint{1.254980in}{0.150000in}}{\pgfqpoint{5.490039in}{5.490039in}}%
\pgfusepath{clip}%
\pgfsetbuttcap%
\pgfsetroundjoin%
\definecolor{currentfill}{rgb}{0.276194,0.190074,0.493001}%
\pgfsetfillcolor{currentfill}%
\pgfsetfillopacity{0.700000}%
\pgfsetlinewidth{0.000000pt}%
\definecolor{currentstroke}{rgb}{0.000000,0.000000,0.000000}%
\pgfsetstrokecolor{currentstroke}%
\pgfsetdash{}{0pt}%
\pgfpathmoveto{\pgfqpoint{2.775229in}{2.294891in}}%
\pgfpathlineto{\pgfqpoint{2.788422in}{2.280436in}}%
\pgfpathlineto{\pgfqpoint{2.801609in}{2.266229in}}%
\pgfpathlineto{\pgfqpoint{2.814791in}{2.252269in}}%
\pgfpathlineto{\pgfqpoint{2.827968in}{2.238553in}}%
\pgfpathlineto{\pgfqpoint{2.836107in}{2.244100in}}%
\pgfpathlineto{\pgfqpoint{2.844237in}{2.249778in}}%
\pgfpathlineto{\pgfqpoint{2.852357in}{2.255586in}}%
\pgfpathlineto{\pgfqpoint{2.860467in}{2.261521in}}%
\pgfpathlineto{\pgfqpoint{2.847317in}{2.274956in}}%
\pgfpathlineto{\pgfqpoint{2.834163in}{2.288634in}}%
\pgfpathlineto{\pgfqpoint{2.821003in}{2.302558in}}%
\pgfpathlineto{\pgfqpoint{2.807837in}{2.316730in}}%
\pgfpathlineto{\pgfqpoint{2.799701in}{2.311066in}}%
\pgfpathlineto{\pgfqpoint{2.791554in}{2.305537in}}%
\pgfpathlineto{\pgfqpoint{2.783397in}{2.300144in}}%
\pgfpathlineto{\pgfqpoint{2.775229in}{2.294891in}}%
\pgfpathclose%
\pgfusepath{fill}%
\end{pgfscope}%
\begin{pgfscope}%
\pgfpathrectangle{\pgfqpoint{1.254980in}{0.150000in}}{\pgfqpoint{5.490039in}{5.490039in}}%
\pgfusepath{clip}%
\pgfsetbuttcap%
\pgfsetroundjoin%
\definecolor{currentfill}{rgb}{0.248629,0.278775,0.534556}%
\pgfsetfillcolor{currentfill}%
\pgfsetfillopacity{0.700000}%
\pgfsetlinewidth{0.000000pt}%
\definecolor{currentstroke}{rgb}{0.000000,0.000000,0.000000}%
\pgfsetstrokecolor{currentstroke}%
\pgfsetdash{}{0pt}%
\pgfpathmoveto{\pgfqpoint{2.616366in}{2.488494in}}%
\pgfpathlineto{\pgfqpoint{2.629648in}{2.470890in}}%
\pgfpathlineto{\pgfqpoint{2.642922in}{2.453562in}}%
\pgfpathlineto{\pgfqpoint{2.656187in}{2.436506in}}%
\pgfpathlineto{\pgfqpoint{2.669444in}{2.419720in}}%
\pgfpathlineto{\pgfqpoint{2.677671in}{2.424382in}}%
\pgfpathlineto{\pgfqpoint{2.685887in}{2.429198in}}%
\pgfpathlineto{\pgfqpoint{2.694091in}{2.434165in}}%
\pgfpathlineto{\pgfqpoint{2.702285in}{2.439281in}}%
\pgfpathlineto{\pgfqpoint{2.689058in}{2.455779in}}%
\pgfpathlineto{\pgfqpoint{2.675824in}{2.472547in}}%
\pgfpathlineto{\pgfqpoint{2.662581in}{2.489588in}}%
\pgfpathlineto{\pgfqpoint{2.649330in}{2.506903in}}%
\pgfpathlineto{\pgfqpoint{2.641106in}{2.502064in}}%
\pgfpathlineto{\pgfqpoint{2.632871in}{2.497381in}}%
\pgfpathlineto{\pgfqpoint{2.624624in}{2.492857in}}%
\pgfpathlineto{\pgfqpoint{2.616366in}{2.488494in}}%
\pgfpathclose%
\pgfusepath{fill}%
\end{pgfscope}%
\begin{pgfscope}%
\pgfpathrectangle{\pgfqpoint{1.254980in}{0.150000in}}{\pgfqpoint{5.490039in}{5.490039in}}%
\pgfusepath{clip}%
\pgfsetbuttcap%
\pgfsetroundjoin%
\definecolor{currentfill}{rgb}{0.282910,0.105393,0.426902}%
\pgfsetfillcolor{currentfill}%
\pgfsetfillopacity{0.700000}%
\pgfsetlinewidth{0.000000pt}%
\definecolor{currentstroke}{rgb}{0.000000,0.000000,0.000000}%
\pgfsetstrokecolor{currentstroke}%
\pgfsetdash{}{0pt}%
\pgfpathmoveto{\pgfqpoint{3.781836in}{2.087001in}}%
\pgfpathlineto{\pgfqpoint{3.794961in}{2.086226in}}%
\pgfpathlineto{\pgfqpoint{3.808092in}{2.085628in}}%
\pgfpathlineto{\pgfqpoint{3.821231in}{2.085206in}}%
\pgfpathlineto{\pgfqpoint{3.834376in}{2.084958in}}%
\pgfpathlineto{\pgfqpoint{3.842085in}{2.094842in}}%
\pgfpathlineto{\pgfqpoint{3.849789in}{2.104711in}}%
\pgfpathlineto{\pgfqpoint{3.857488in}{2.114566in}}%
\pgfpathlineto{\pgfqpoint{3.865182in}{2.124405in}}%
\pgfpathlineto{\pgfqpoint{3.852045in}{2.124576in}}%
\pgfpathlineto{\pgfqpoint{3.838915in}{2.124921in}}%
\pgfpathlineto{\pgfqpoint{3.825792in}{2.125441in}}%
\pgfpathlineto{\pgfqpoint{3.812676in}{2.126138in}}%
\pgfpathlineto{\pgfqpoint{3.804973in}{2.116366in}}%
\pgfpathlineto{\pgfqpoint{3.797266in}{2.106585in}}%
\pgfpathlineto{\pgfqpoint{3.789553in}{2.096796in}}%
\pgfpathlineto{\pgfqpoint{3.781836in}{2.087001in}}%
\pgfpathclose%
\pgfusepath{fill}%
\end{pgfscope}%
\begin{pgfscope}%
\pgfpathrectangle{\pgfqpoint{1.254980in}{0.150000in}}{\pgfqpoint{5.490039in}{5.490039in}}%
\pgfusepath{clip}%
\pgfsetbuttcap%
\pgfsetroundjoin%
\definecolor{currentfill}{rgb}{0.263663,0.237631,0.518762}%
\pgfsetfillcolor{currentfill}%
\pgfsetfillopacity{0.700000}%
\pgfsetlinewidth{0.000000pt}%
\definecolor{currentstroke}{rgb}{0.000000,0.000000,0.000000}%
\pgfsetstrokecolor{currentstroke}%
\pgfsetdash{}{0pt}%
\pgfpathmoveto{\pgfqpoint{4.281713in}{2.350098in}}%
\pgfpathlineto{\pgfqpoint{4.294997in}{2.353459in}}%
\pgfpathlineto{\pgfqpoint{4.308292in}{2.356986in}}%
\pgfpathlineto{\pgfqpoint{4.321598in}{2.360677in}}%
\pgfpathlineto{\pgfqpoint{4.334915in}{2.364533in}}%
\pgfpathlineto{\pgfqpoint{4.342458in}{2.373627in}}%
\pgfpathlineto{\pgfqpoint{4.349995in}{2.382674in}}%
\pgfpathlineto{\pgfqpoint{4.357528in}{2.391676in}}%
\pgfpathlineto{\pgfqpoint{4.365054in}{2.400636in}}%
\pgfpathlineto{\pgfqpoint{4.351745in}{2.396871in}}%
\pgfpathlineto{\pgfqpoint{4.338447in}{2.393272in}}%
\pgfpathlineto{\pgfqpoint{4.325159in}{2.389837in}}%
\pgfpathlineto{\pgfqpoint{4.311882in}{2.386567in}}%
\pgfpathlineto{\pgfqpoint{4.304348in}{2.377506in}}%
\pgfpathlineto{\pgfqpoint{4.296808in}{2.368408in}}%
\pgfpathlineto{\pgfqpoint{4.289263in}{2.359273in}}%
\pgfpathlineto{\pgfqpoint{4.281713in}{2.350098in}}%
\pgfpathclose%
\pgfusepath{fill}%
\end{pgfscope}%
\begin{pgfscope}%
\pgfpathrectangle{\pgfqpoint{1.254980in}{0.150000in}}{\pgfqpoint{5.490039in}{5.490039in}}%
\pgfusepath{clip}%
\pgfsetbuttcap%
\pgfsetroundjoin%
\definecolor{currentfill}{rgb}{0.163625,0.471133,0.558148}%
\pgfsetfillcolor{currentfill}%
\pgfsetfillopacity{0.700000}%
\pgfsetlinewidth{0.000000pt}%
\definecolor{currentstroke}{rgb}{0.000000,0.000000,0.000000}%
\pgfsetstrokecolor{currentstroke}%
\pgfsetdash{}{0pt}%
\pgfpathmoveto{\pgfqpoint{5.145286in}{2.902121in}}%
\pgfpathlineto{\pgfqpoint{5.158964in}{2.909283in}}%
\pgfpathlineto{\pgfqpoint{5.172656in}{2.916601in}}%
\pgfpathlineto{\pgfqpoint{5.186364in}{2.924073in}}%
\pgfpathlineto{\pgfqpoint{5.200087in}{2.931702in}}%
\pgfpathlineto{\pgfqpoint{5.207258in}{2.937075in}}%
\pgfpathlineto{\pgfqpoint{5.214425in}{2.942462in}}%
\pgfpathlineto{\pgfqpoint{5.221585in}{2.947867in}}%
\pgfpathlineto{\pgfqpoint{5.228741in}{2.953295in}}%
\pgfpathlineto{\pgfqpoint{5.215037in}{2.946078in}}%
\pgfpathlineto{\pgfqpoint{5.201348in}{2.939016in}}%
\pgfpathlineto{\pgfqpoint{5.187674in}{2.932109in}}%
\pgfpathlineto{\pgfqpoint{5.174015in}{2.925357in}}%
\pgfpathlineto{\pgfqpoint{5.166840in}{2.919508in}}%
\pgfpathlineto{\pgfqpoint{5.159661in}{2.913689in}}%
\pgfpathlineto{\pgfqpoint{5.152476in}{2.907895in}}%
\pgfpathlineto{\pgfqpoint{5.145286in}{2.902121in}}%
\pgfpathclose%
\pgfusepath{fill}%
\end{pgfscope}%
\begin{pgfscope}%
\pgfpathrectangle{\pgfqpoint{1.254980in}{0.150000in}}{\pgfqpoint{5.490039in}{5.490039in}}%
\pgfusepath{clip}%
\pgfsetbuttcap%
\pgfsetroundjoin%
\definecolor{currentfill}{rgb}{0.280255,0.165693,0.476498}%
\pgfsetfillcolor{currentfill}%
\pgfsetfillopacity{0.700000}%
\pgfsetlinewidth{0.000000pt}%
\definecolor{currentstroke}{rgb}{0.000000,0.000000,0.000000}%
\pgfsetstrokecolor{currentstroke}%
\pgfsetdash{}{0pt}%
\pgfpathmoveto{\pgfqpoint{2.827968in}{2.238553in}}%
\pgfpathlineto{\pgfqpoint{2.841139in}{2.225080in}}%
\pgfpathlineto{\pgfqpoint{2.854305in}{2.211847in}}%
\pgfpathlineto{\pgfqpoint{2.867467in}{2.198853in}}%
\pgfpathlineto{\pgfqpoint{2.880624in}{2.186096in}}%
\pgfpathlineto{\pgfqpoint{2.888738in}{2.191934in}}%
\pgfpathlineto{\pgfqpoint{2.896841in}{2.197896in}}%
\pgfpathlineto{\pgfqpoint{2.904935in}{2.203981in}}%
\pgfpathlineto{\pgfqpoint{2.913020in}{2.210186in}}%
\pgfpathlineto{\pgfqpoint{2.899888in}{2.222663in}}%
\pgfpathlineto{\pgfqpoint{2.886752in}{2.235377in}}%
\pgfpathlineto{\pgfqpoint{2.873612in}{2.248329in}}%
\pgfpathlineto{\pgfqpoint{2.860467in}{2.261521in}}%
\pgfpathlineto{\pgfqpoint{2.852357in}{2.255586in}}%
\pgfpathlineto{\pgfqpoint{2.844237in}{2.249778in}}%
\pgfpathlineto{\pgfqpoint{2.836107in}{2.244100in}}%
\pgfpathlineto{\pgfqpoint{2.827968in}{2.238553in}}%
\pgfpathclose%
\pgfusepath{fill}%
\end{pgfscope}%
\begin{pgfscope}%
\pgfpathrectangle{\pgfqpoint{1.254980in}{0.150000in}}{\pgfqpoint{5.490039in}{5.490039in}}%
\pgfusepath{clip}%
\pgfsetbuttcap%
\pgfsetroundjoin%
\definecolor{currentfill}{rgb}{0.253935,0.265254,0.529983}%
\pgfsetfillcolor{currentfill}%
\pgfsetfillopacity{0.700000}%
\pgfsetlinewidth{0.000000pt}%
\definecolor{currentstroke}{rgb}{0.000000,0.000000,0.000000}%
\pgfsetstrokecolor{currentstroke}%
\pgfsetdash{}{0pt}%
\pgfpathmoveto{\pgfqpoint{4.365054in}{2.400636in}}%
\pgfpathlineto{\pgfqpoint{4.378375in}{2.404564in}}%
\pgfpathlineto{\pgfqpoint{4.391706in}{2.408657in}}%
\pgfpathlineto{\pgfqpoint{4.405048in}{2.412913in}}%
\pgfpathlineto{\pgfqpoint{4.418402in}{2.417333in}}%
\pgfpathlineto{\pgfqpoint{4.425916in}{2.426140in}}%
\pgfpathlineto{\pgfqpoint{4.433424in}{2.434901in}}%
\pgfpathlineto{\pgfqpoint{4.440927in}{2.443616in}}%
\pgfpathlineto{\pgfqpoint{4.448424in}{2.452287in}}%
\pgfpathlineto{\pgfqpoint{4.435078in}{2.447987in}}%
\pgfpathlineto{\pgfqpoint{4.421743in}{2.443852in}}%
\pgfpathlineto{\pgfqpoint{4.408420in}{2.439879in}}%
\pgfpathlineto{\pgfqpoint{4.395108in}{2.436071in}}%
\pgfpathlineto{\pgfqpoint{4.387603in}{2.427269in}}%
\pgfpathlineto{\pgfqpoint{4.380092in}{2.418430in}}%
\pgfpathlineto{\pgfqpoint{4.372576in}{2.409553in}}%
\pgfpathlineto{\pgfqpoint{4.365054in}{2.400636in}}%
\pgfpathclose%
\pgfusepath{fill}%
\end{pgfscope}%
\begin{pgfscope}%
\pgfpathrectangle{\pgfqpoint{1.254980in}{0.150000in}}{\pgfqpoint{5.490039in}{5.490039in}}%
\pgfusepath{clip}%
\pgfsetbuttcap%
\pgfsetroundjoin%
\definecolor{currentfill}{rgb}{0.156270,0.489624,0.557936}%
\pgfsetfillcolor{currentfill}%
\pgfsetfillopacity{0.700000}%
\pgfsetlinewidth{0.000000pt}%
\definecolor{currentstroke}{rgb}{0.000000,0.000000,0.000000}%
\pgfsetstrokecolor{currentstroke}%
\pgfsetdash{}{0pt}%
\pgfpathmoveto{\pgfqpoint{5.228741in}{2.953295in}}%
\pgfpathlineto{\pgfqpoint{5.242461in}{2.960666in}}%
\pgfpathlineto{\pgfqpoint{5.256196in}{2.968192in}}%
\pgfpathlineto{\pgfqpoint{5.269947in}{2.975873in}}%
\pgfpathlineto{\pgfqpoint{5.283713in}{2.983708in}}%
\pgfpathlineto{\pgfqpoint{5.290844in}{2.988734in}}%
\pgfpathlineto{\pgfqpoint{5.297969in}{2.993786in}}%
\pgfpathlineto{\pgfqpoint{5.305090in}{2.998868in}}%
\pgfpathlineto{\pgfqpoint{5.312205in}{3.003986in}}%
\pgfpathlineto{\pgfqpoint{5.298459in}{2.996591in}}%
\pgfpathlineto{\pgfqpoint{5.284728in}{2.989351in}}%
\pgfpathlineto{\pgfqpoint{5.271013in}{2.982264in}}%
\pgfpathlineto{\pgfqpoint{5.257314in}{2.975332in}}%
\pgfpathlineto{\pgfqpoint{5.250178in}{2.969764in}}%
\pgfpathlineto{\pgfqpoint{5.243037in}{2.964239in}}%
\pgfpathlineto{\pgfqpoint{5.235892in}{2.958750in}}%
\pgfpathlineto{\pgfqpoint{5.228741in}{2.953295in}}%
\pgfpathclose%
\pgfusepath{fill}%
\end{pgfscope}%
\begin{pgfscope}%
\pgfpathrectangle{\pgfqpoint{1.254980in}{0.150000in}}{\pgfqpoint{5.490039in}{5.490039in}}%
\pgfusepath{clip}%
\pgfsetbuttcap%
\pgfsetroundjoin%
\definecolor{currentfill}{rgb}{0.277018,0.050344,0.375715}%
\pgfsetfillcolor{currentfill}%
\pgfsetfillopacity{0.700000}%
\pgfsetlinewidth{0.000000pt}%
\definecolor{currentstroke}{rgb}{0.000000,0.000000,0.000000}%
\pgfsetstrokecolor{currentstroke}%
\pgfsetdash{}{0pt}%
\pgfpathmoveto{\pgfqpoint{3.259232in}{2.009462in}}%
\pgfpathlineto{\pgfqpoint{3.272309in}{2.002644in}}%
\pgfpathlineto{\pgfqpoint{3.285387in}{1.996025in}}%
\pgfpathlineto{\pgfqpoint{3.298467in}{1.989605in}}%
\pgfpathlineto{\pgfqpoint{3.311548in}{1.983383in}}%
\pgfpathlineto{\pgfqpoint{3.319451in}{1.991764in}}%
\pgfpathlineto{\pgfqpoint{3.327348in}{2.000204in}}%
\pgfpathlineto{\pgfqpoint{3.335238in}{2.008700in}}%
\pgfpathlineto{\pgfqpoint{3.343121in}{2.017250in}}%
\pgfpathlineto{\pgfqpoint{3.330056in}{2.023256in}}%
\pgfpathlineto{\pgfqpoint{3.316992in}{2.029458in}}%
\pgfpathlineto{\pgfqpoint{3.303930in}{2.035859in}}%
\pgfpathlineto{\pgfqpoint{3.290870in}{2.042460in}}%
\pgfpathlineto{\pgfqpoint{3.282971in}{2.034117in}}%
\pgfpathlineto{\pgfqpoint{3.275064in}{2.025835in}}%
\pgfpathlineto{\pgfqpoint{3.267151in}{2.017616in}}%
\pgfpathlineto{\pgfqpoint{3.259232in}{2.009462in}}%
\pgfpathclose%
\pgfusepath{fill}%
\end{pgfscope}%
\begin{pgfscope}%
\pgfpathrectangle{\pgfqpoint{1.254980in}{0.150000in}}{\pgfqpoint{5.490039in}{5.490039in}}%
\pgfusepath{clip}%
\pgfsetbuttcap%
\pgfsetroundjoin%
\definecolor{currentfill}{rgb}{0.233603,0.313828,0.543914}%
\pgfsetfillcolor{currentfill}%
\pgfsetfillopacity{0.700000}%
\pgfsetlinewidth{0.000000pt}%
\definecolor{currentstroke}{rgb}{0.000000,0.000000,0.000000}%
\pgfsetstrokecolor{currentstroke}%
\pgfsetdash{}{0pt}%
\pgfpathmoveto{\pgfqpoint{2.563140in}{2.561713in}}%
\pgfpathlineto{\pgfqpoint{2.576461in}{2.542982in}}%
\pgfpathlineto{\pgfqpoint{2.589772in}{2.524538in}}%
\pgfpathlineto{\pgfqpoint{2.603074in}{2.506376in}}%
\pgfpathlineto{\pgfqpoint{2.616366in}{2.488494in}}%
\pgfpathlineto{\pgfqpoint{2.624624in}{2.492857in}}%
\pgfpathlineto{\pgfqpoint{2.632871in}{2.497381in}}%
\pgfpathlineto{\pgfqpoint{2.641106in}{2.502064in}}%
\pgfpathlineto{\pgfqpoint{2.649330in}{2.506903in}}%
\pgfpathlineto{\pgfqpoint{2.636070in}{2.524495in}}%
\pgfpathlineto{\pgfqpoint{2.622800in}{2.542367in}}%
\pgfpathlineto{\pgfqpoint{2.609522in}{2.560521in}}%
\pgfpathlineto{\pgfqpoint{2.596233in}{2.578961in}}%
\pgfpathlineto{\pgfqpoint{2.587978in}{2.574401in}}%
\pgfpathlineto{\pgfqpoint{2.579711in}{2.570004in}}%
\pgfpathlineto{\pgfqpoint{2.571431in}{2.565774in}}%
\pgfpathlineto{\pgfqpoint{2.563140in}{2.561713in}}%
\pgfpathclose%
\pgfusepath{fill}%
\end{pgfscope}%
\begin{pgfscope}%
\pgfpathrectangle{\pgfqpoint{1.254980in}{0.150000in}}{\pgfqpoint{5.490039in}{5.490039in}}%
\pgfusepath{clip}%
\pgfsetbuttcap%
\pgfsetroundjoin%
\definecolor{currentfill}{rgb}{0.281446,0.084320,0.407414}%
\pgfsetfillcolor{currentfill}%
\pgfsetfillopacity{0.700000}%
\pgfsetlinewidth{0.000000pt}%
\definecolor{currentstroke}{rgb}{0.000000,0.000000,0.000000}%
\pgfsetstrokecolor{currentstroke}%
\pgfsetdash{}{0pt}%
\pgfpathmoveto{\pgfqpoint{3.698440in}{2.053048in}}%
\pgfpathlineto{\pgfqpoint{3.711550in}{2.051457in}}%
\pgfpathlineto{\pgfqpoint{3.724665in}{2.050045in}}%
\pgfpathlineto{\pgfqpoint{3.737787in}{2.048811in}}%
\pgfpathlineto{\pgfqpoint{3.750915in}{2.047755in}}%
\pgfpathlineto{\pgfqpoint{3.758653in}{2.057574in}}%
\pgfpathlineto{\pgfqpoint{3.766385in}{2.067389in}}%
\pgfpathlineto{\pgfqpoint{3.774113in}{2.077198in}}%
\pgfpathlineto{\pgfqpoint{3.781836in}{2.087001in}}%
\pgfpathlineto{\pgfqpoint{3.768717in}{2.087952in}}%
\pgfpathlineto{\pgfqpoint{3.755605in}{2.089081in}}%
\pgfpathlineto{\pgfqpoint{3.742499in}{2.090387in}}%
\pgfpathlineto{\pgfqpoint{3.729398in}{2.091873in}}%
\pgfpathlineto{\pgfqpoint{3.721667in}{2.082165in}}%
\pgfpathlineto{\pgfqpoint{3.713930in}{2.072458in}}%
\pgfpathlineto{\pgfqpoint{3.706187in}{2.062752in}}%
\pgfpathlineto{\pgfqpoint{3.698440in}{2.053048in}}%
\pgfpathclose%
\pgfusepath{fill}%
\end{pgfscope}%
\begin{pgfscope}%
\pgfpathrectangle{\pgfqpoint{1.254980in}{0.150000in}}{\pgfqpoint{5.490039in}{5.490039in}}%
\pgfusepath{clip}%
\pgfsetbuttcap%
\pgfsetroundjoin%
\definecolor{currentfill}{rgb}{0.147607,0.511733,0.557049}%
\pgfsetfillcolor{currentfill}%
\pgfsetfillopacity{0.700000}%
\pgfsetlinewidth{0.000000pt}%
\definecolor{currentstroke}{rgb}{0.000000,0.000000,0.000000}%
\pgfsetstrokecolor{currentstroke}%
\pgfsetdash{}{0pt}%
\pgfpathmoveto{\pgfqpoint{5.312205in}{3.003986in}}%
\pgfpathlineto{\pgfqpoint{5.325967in}{3.011534in}}%
\pgfpathlineto{\pgfqpoint{5.339745in}{3.019237in}}%
\pgfpathlineto{\pgfqpoint{5.353538in}{3.027094in}}%
\pgfpathlineto{\pgfqpoint{5.367348in}{3.035105in}}%
\pgfpathlineto{\pgfqpoint{5.374437in}{3.039804in}}%
\pgfpathlineto{\pgfqpoint{5.381521in}{3.044542in}}%
\pgfpathlineto{\pgfqpoint{5.388600in}{3.049325in}}%
\pgfpathlineto{\pgfqpoint{5.395675in}{3.054157in}}%
\pgfpathlineto{\pgfqpoint{5.381888in}{3.046616in}}%
\pgfpathlineto{\pgfqpoint{5.368116in}{3.039229in}}%
\pgfpathlineto{\pgfqpoint{5.354360in}{3.031995in}}%
\pgfpathlineto{\pgfqpoint{5.340620in}{3.024915in}}%
\pgfpathlineto{\pgfqpoint{5.333523in}{3.019604in}}%
\pgfpathlineto{\pgfqpoint{5.326422in}{3.014349in}}%
\pgfpathlineto{\pgfqpoint{5.319316in}{3.009144in}}%
\pgfpathlineto{\pgfqpoint{5.312205in}{3.003986in}}%
\pgfpathclose%
\pgfusepath{fill}%
\end{pgfscope}%
\begin{pgfscope}%
\pgfpathrectangle{\pgfqpoint{1.254980in}{0.150000in}}{\pgfqpoint{5.490039in}{5.490039in}}%
\pgfusepath{clip}%
\pgfsetbuttcap%
\pgfsetroundjoin%
\definecolor{currentfill}{rgb}{0.282623,0.140926,0.457517}%
\pgfsetfillcolor{currentfill}%
\pgfsetfillopacity{0.700000}%
\pgfsetlinewidth{0.000000pt}%
\definecolor{currentstroke}{rgb}{0.000000,0.000000,0.000000}%
\pgfsetstrokecolor{currentstroke}%
\pgfsetdash{}{0pt}%
\pgfpathmoveto{\pgfqpoint{2.880624in}{2.186096in}}%
\pgfpathlineto{\pgfqpoint{2.893777in}{2.173574in}}%
\pgfpathlineto{\pgfqpoint{2.906927in}{2.161286in}}%
\pgfpathlineto{\pgfqpoint{2.920072in}{2.149229in}}%
\pgfpathlineto{\pgfqpoint{2.933214in}{2.137402in}}%
\pgfpathlineto{\pgfqpoint{2.941301in}{2.143529in}}%
\pgfpathlineto{\pgfqpoint{2.949380in}{2.149774in}}%
\pgfpathlineto{\pgfqpoint{2.957449in}{2.156135in}}%
\pgfpathlineto{\pgfqpoint{2.965510in}{2.162608in}}%
\pgfpathlineto{\pgfqpoint{2.952392in}{2.174156in}}%
\pgfpathlineto{\pgfqpoint{2.939272in}{2.185934in}}%
\pgfpathlineto{\pgfqpoint{2.926148in}{2.197944in}}%
\pgfpathlineto{\pgfqpoint{2.913020in}{2.210186in}}%
\pgfpathlineto{\pgfqpoint{2.904935in}{2.203981in}}%
\pgfpathlineto{\pgfqpoint{2.896841in}{2.197896in}}%
\pgfpathlineto{\pgfqpoint{2.888738in}{2.191934in}}%
\pgfpathlineto{\pgfqpoint{2.880624in}{2.186096in}}%
\pgfpathclose%
\pgfusepath{fill}%
\end{pgfscope}%
\begin{pgfscope}%
\pgfpathrectangle{\pgfqpoint{1.254980in}{0.150000in}}{\pgfqpoint{5.490039in}{5.490039in}}%
\pgfusepath{clip}%
\pgfsetbuttcap%
\pgfsetroundjoin%
\definecolor{currentfill}{rgb}{0.277018,0.050344,0.375715}%
\pgfsetfillcolor{currentfill}%
\pgfsetfillopacity{0.700000}%
\pgfsetlinewidth{0.000000pt}%
\definecolor{currentstroke}{rgb}{0.000000,0.000000,0.000000}%
\pgfsetstrokecolor{currentstroke}%
\pgfsetdash{}{0pt}%
\pgfpathmoveto{\pgfqpoint{3.395404in}{1.995180in}}%
\pgfpathlineto{\pgfqpoint{3.408481in}{1.990146in}}%
\pgfpathlineto{\pgfqpoint{3.421561in}{1.985303in}}%
\pgfpathlineto{\pgfqpoint{3.434643in}{1.980651in}}%
\pgfpathlineto{\pgfqpoint{3.447729in}{1.976188in}}%
\pgfpathlineto{\pgfqpoint{3.455577in}{1.985189in}}%
\pgfpathlineto{\pgfqpoint{3.463420in}{1.994228in}}%
\pgfpathlineto{\pgfqpoint{3.471256in}{2.003301in}}%
\pgfpathlineto{\pgfqpoint{3.479087in}{2.012409in}}%
\pgfpathlineto{\pgfqpoint{3.466014in}{2.016683in}}%
\pgfpathlineto{\pgfqpoint{3.452945in}{2.021147in}}%
\pgfpathlineto{\pgfqpoint{3.439879in}{2.025801in}}%
\pgfpathlineto{\pgfqpoint{3.426817in}{2.030646in}}%
\pgfpathlineto{\pgfqpoint{3.418973in}{2.021717in}}%
\pgfpathlineto{\pgfqpoint{3.411123in}{2.012829in}}%
\pgfpathlineto{\pgfqpoint{3.403266in}{2.003983in}}%
\pgfpathlineto{\pgfqpoint{3.395404in}{1.995180in}}%
\pgfpathclose%
\pgfusepath{fill}%
\end{pgfscope}%
\begin{pgfscope}%
\pgfpathrectangle{\pgfqpoint{1.254980in}{0.150000in}}{\pgfqpoint{5.490039in}{5.490039in}}%
\pgfusepath{clip}%
\pgfsetbuttcap%
\pgfsetroundjoin%
\definecolor{currentfill}{rgb}{0.244972,0.287675,0.537260}%
\pgfsetfillcolor{currentfill}%
\pgfsetfillopacity{0.700000}%
\pgfsetlinewidth{0.000000pt}%
\definecolor{currentstroke}{rgb}{0.000000,0.000000,0.000000}%
\pgfsetstrokecolor{currentstroke}%
\pgfsetdash{}{0pt}%
\pgfpathmoveto{\pgfqpoint{4.448424in}{2.452287in}}%
\pgfpathlineto{\pgfqpoint{4.461781in}{2.456749in}}%
\pgfpathlineto{\pgfqpoint{4.475150in}{2.461374in}}%
\pgfpathlineto{\pgfqpoint{4.488531in}{2.466162in}}%
\pgfpathlineto{\pgfqpoint{4.501923in}{2.471112in}}%
\pgfpathlineto{\pgfqpoint{4.509407in}{2.479602in}}%
\pgfpathlineto{\pgfqpoint{4.516885in}{2.488045in}}%
\pgfpathlineto{\pgfqpoint{4.524357in}{2.496443in}}%
\pgfpathlineto{\pgfqpoint{4.531824in}{2.504797in}}%
\pgfpathlineto{\pgfqpoint{4.518440in}{2.499997in}}%
\pgfpathlineto{\pgfqpoint{4.505067in}{2.495358in}}%
\pgfpathlineto{\pgfqpoint{4.491707in}{2.490882in}}%
\pgfpathlineto{\pgfqpoint{4.478358in}{2.486568in}}%
\pgfpathlineto{\pgfqpoint{4.470883in}{2.478054in}}%
\pgfpathlineto{\pgfqpoint{4.463402in}{2.469504in}}%
\pgfpathlineto{\pgfqpoint{4.455916in}{2.460916in}}%
\pgfpathlineto{\pgfqpoint{4.448424in}{2.452287in}}%
\pgfpathclose%
\pgfusepath{fill}%
\end{pgfscope}%
\begin{pgfscope}%
\pgfpathrectangle{\pgfqpoint{1.254980in}{0.150000in}}{\pgfqpoint{5.490039in}{5.490039in}}%
\pgfusepath{clip}%
\pgfsetbuttcap%
\pgfsetroundjoin%
\definecolor{currentfill}{rgb}{0.279566,0.067836,0.391917}%
\pgfsetfillcolor{currentfill}%
\pgfsetfillopacity{0.700000}%
\pgfsetlinewidth{0.000000pt}%
\definecolor{currentstroke}{rgb}{0.000000,0.000000,0.000000}%
\pgfsetstrokecolor{currentstroke}%
\pgfsetdash{}{0pt}%
\pgfpathmoveto{\pgfqpoint{3.122738in}{2.041368in}}%
\pgfpathlineto{\pgfqpoint{3.135831in}{2.032666in}}%
\pgfpathlineto{\pgfqpoint{3.148924in}{2.024173in}}%
\pgfpathlineto{\pgfqpoint{3.162017in}{2.015887in}}%
\pgfpathlineto{\pgfqpoint{3.175110in}{2.007808in}}%
\pgfpathlineto{\pgfqpoint{3.183076in}{2.015426in}}%
\pgfpathlineto{\pgfqpoint{3.191035in}{2.023124in}}%
\pgfpathlineto{\pgfqpoint{3.198987in}{2.030900in}}%
\pgfpathlineto{\pgfqpoint{3.206932in}{2.038752in}}%
\pgfpathlineto{\pgfqpoint{3.193858in}{2.046584in}}%
\pgfpathlineto{\pgfqpoint{3.180785in}{2.054623in}}%
\pgfpathlineto{\pgfqpoint{3.167711in}{2.062870in}}%
\pgfpathlineto{\pgfqpoint{3.154637in}{2.071326in}}%
\pgfpathlineto{\pgfqpoint{3.146674in}{2.063710in}}%
\pgfpathlineto{\pgfqpoint{3.138703in}{2.056177in}}%
\pgfpathlineto{\pgfqpoint{3.130724in}{2.048729in}}%
\pgfpathlineto{\pgfqpoint{3.122738in}{2.041368in}}%
\pgfpathclose%
\pgfusepath{fill}%
\end{pgfscope}%
\begin{pgfscope}%
\pgfpathrectangle{\pgfqpoint{1.254980in}{0.150000in}}{\pgfqpoint{5.490039in}{5.490039in}}%
\pgfusepath{clip}%
\pgfsetbuttcap%
\pgfsetroundjoin%
\definecolor{currentfill}{rgb}{0.140536,0.530132,0.555659}%
\pgfsetfillcolor{currentfill}%
\pgfsetfillopacity{0.700000}%
\pgfsetlinewidth{0.000000pt}%
\definecolor{currentstroke}{rgb}{0.000000,0.000000,0.000000}%
\pgfsetstrokecolor{currentstroke}%
\pgfsetdash{}{0pt}%
\pgfpathmoveto{\pgfqpoint{5.395675in}{3.054157in}}%
\pgfpathlineto{\pgfqpoint{5.409479in}{3.061850in}}%
\pgfpathlineto{\pgfqpoint{5.423298in}{3.069698in}}%
\pgfpathlineto{\pgfqpoint{5.437134in}{3.077698in}}%
\pgfpathlineto{\pgfqpoint{5.450986in}{3.085852in}}%
\pgfpathlineto{\pgfqpoint{5.458033in}{3.090251in}}%
\pgfpathlineto{\pgfqpoint{5.465076in}{3.094703in}}%
\pgfpathlineto{\pgfqpoint{5.472114in}{3.099214in}}%
\pgfpathlineto{\pgfqpoint{5.479148in}{3.103790in}}%
\pgfpathlineto{\pgfqpoint{5.465320in}{3.096135in}}%
\pgfpathlineto{\pgfqpoint{5.451508in}{3.088634in}}%
\pgfpathlineto{\pgfqpoint{5.437712in}{3.081285in}}%
\pgfpathlineto{\pgfqpoint{5.423932in}{3.074088in}}%
\pgfpathlineto{\pgfqpoint{5.416874in}{3.069004in}}%
\pgfpathlineto{\pgfqpoint{5.409812in}{3.063991in}}%
\pgfpathlineto{\pgfqpoint{5.402746in}{3.059044in}}%
\pgfpathlineto{\pgfqpoint{5.395675in}{3.054157in}}%
\pgfpathclose%
\pgfusepath{fill}%
\end{pgfscope}%
\begin{pgfscope}%
\pgfpathrectangle{\pgfqpoint{1.254980in}{0.150000in}}{\pgfqpoint{5.490039in}{5.490039in}}%
\pgfusepath{clip}%
\pgfsetbuttcap%
\pgfsetroundjoin%
\definecolor{currentfill}{rgb}{0.279566,0.067836,0.391917}%
\pgfsetfillcolor{currentfill}%
\pgfsetfillopacity{0.700000}%
\pgfsetlinewidth{0.000000pt}%
\definecolor{currentstroke}{rgb}{0.000000,0.000000,0.000000}%
\pgfsetstrokecolor{currentstroke}%
\pgfsetdash{}{0pt}%
\pgfpathmoveto{\pgfqpoint{3.614974in}{2.022964in}}%
\pgfpathlineto{\pgfqpoint{3.628072in}{2.020519in}}%
\pgfpathlineto{\pgfqpoint{3.641176in}{2.018255in}}%
\pgfpathlineto{\pgfqpoint{3.654284in}{2.016172in}}%
\pgfpathlineto{\pgfqpoint{3.667398in}{2.014269in}}%
\pgfpathlineto{\pgfqpoint{3.675167in}{2.023956in}}%
\pgfpathlineto{\pgfqpoint{3.682930in}{2.033649in}}%
\pgfpathlineto{\pgfqpoint{3.690688in}{2.043347in}}%
\pgfpathlineto{\pgfqpoint{3.698440in}{2.053048in}}%
\pgfpathlineto{\pgfqpoint{3.685336in}{2.054818in}}%
\pgfpathlineto{\pgfqpoint{3.672238in}{2.056768in}}%
\pgfpathlineto{\pgfqpoint{3.659145in}{2.058899in}}%
\pgfpathlineto{\pgfqpoint{3.646057in}{2.061211in}}%
\pgfpathlineto{\pgfqpoint{3.638294in}{2.051633in}}%
\pgfpathlineto{\pgfqpoint{3.630526in}{2.042065in}}%
\pgfpathlineto{\pgfqpoint{3.622752in}{2.032508in}}%
\pgfpathlineto{\pgfqpoint{3.614974in}{2.022964in}}%
\pgfpathclose%
\pgfusepath{fill}%
\end{pgfscope}%
\begin{pgfscope}%
\pgfpathrectangle{\pgfqpoint{1.254980in}{0.150000in}}{\pgfqpoint{5.490039in}{5.490039in}}%
\pgfusepath{clip}%
\pgfsetbuttcap%
\pgfsetroundjoin%
\definecolor{currentfill}{rgb}{0.218130,0.347432,0.550038}%
\pgfsetfillcolor{currentfill}%
\pgfsetfillopacity{0.700000}%
\pgfsetlinewidth{0.000000pt}%
\definecolor{currentstroke}{rgb}{0.000000,0.000000,0.000000}%
\pgfsetstrokecolor{currentstroke}%
\pgfsetdash{}{0pt}%
\pgfpathmoveto{\pgfqpoint{2.509748in}{2.639546in}}%
\pgfpathlineto{\pgfqpoint{2.523113in}{2.619645in}}%
\pgfpathlineto{\pgfqpoint{2.536466in}{2.600041in}}%
\pgfpathlineto{\pgfqpoint{2.549808in}{2.580732in}}%
\pgfpathlineto{\pgfqpoint{2.563140in}{2.561713in}}%
\pgfpathlineto{\pgfqpoint{2.571431in}{2.565774in}}%
\pgfpathlineto{\pgfqpoint{2.579711in}{2.570004in}}%
\pgfpathlineto{\pgfqpoint{2.587978in}{2.574401in}}%
\pgfpathlineto{\pgfqpoint{2.596233in}{2.578961in}}%
\pgfpathlineto{\pgfqpoint{2.582935in}{2.597688in}}%
\pgfpathlineto{\pgfqpoint{2.569626in}{2.616705in}}%
\pgfpathlineto{\pgfqpoint{2.556307in}{2.636016in}}%
\pgfpathlineto{\pgfqpoint{2.542976in}{2.655622in}}%
\pgfpathlineto{\pgfqpoint{2.534688in}{2.651344in}}%
\pgfpathlineto{\pgfqpoint{2.526388in}{2.647236in}}%
\pgfpathlineto{\pgfqpoint{2.518074in}{2.643303in}}%
\pgfpathlineto{\pgfqpoint{2.509748in}{2.639546in}}%
\pgfpathclose%
\pgfusepath{fill}%
\end{pgfscope}%
\begin{pgfscope}%
\pgfpathrectangle{\pgfqpoint{1.254980in}{0.150000in}}{\pgfqpoint{5.490039in}{5.490039in}}%
\pgfusepath{clip}%
\pgfsetbuttcap%
\pgfsetroundjoin%
\definecolor{currentfill}{rgb}{0.233603,0.313828,0.543914}%
\pgfsetfillcolor{currentfill}%
\pgfsetfillopacity{0.700000}%
\pgfsetlinewidth{0.000000pt}%
\definecolor{currentstroke}{rgb}{0.000000,0.000000,0.000000}%
\pgfsetstrokecolor{currentstroke}%
\pgfsetdash{}{0pt}%
\pgfpathmoveto{\pgfqpoint{4.531824in}{2.504797in}}%
\pgfpathlineto{\pgfqpoint{4.545220in}{2.509760in}}%
\pgfpathlineto{\pgfqpoint{4.558628in}{2.514885in}}%
\pgfpathlineto{\pgfqpoint{4.572049in}{2.520171in}}%
\pgfpathlineto{\pgfqpoint{4.585482in}{2.525618in}}%
\pgfpathlineto{\pgfqpoint{4.592934in}{2.533763in}}%
\pgfpathlineto{\pgfqpoint{4.600380in}{2.541863in}}%
\pgfpathlineto{\pgfqpoint{4.607821in}{2.549919in}}%
\pgfpathlineto{\pgfqpoint{4.615256in}{2.557933in}}%
\pgfpathlineto{\pgfqpoint{4.601832in}{2.552664in}}%
\pgfpathlineto{\pgfqpoint{4.588421in}{2.547556in}}%
\pgfpathlineto{\pgfqpoint{4.575021in}{2.542609in}}%
\pgfpathlineto{\pgfqpoint{4.561634in}{2.537824in}}%
\pgfpathlineto{\pgfqpoint{4.554190in}{2.529622in}}%
\pgfpathlineto{\pgfqpoint{4.546740in}{2.521385in}}%
\pgfpathlineto{\pgfqpoint{4.539285in}{2.513111in}}%
\pgfpathlineto{\pgfqpoint{4.531824in}{2.504797in}}%
\pgfpathclose%
\pgfusepath{fill}%
\end{pgfscope}%
\begin{pgfscope}%
\pgfpathrectangle{\pgfqpoint{1.254980in}{0.150000in}}{\pgfqpoint{5.490039in}{5.490039in}}%
\pgfusepath{clip}%
\pgfsetbuttcap%
\pgfsetroundjoin%
\definecolor{currentfill}{rgb}{0.133743,0.548535,0.553541}%
\pgfsetfillcolor{currentfill}%
\pgfsetfillopacity{0.700000}%
\pgfsetlinewidth{0.000000pt}%
\definecolor{currentstroke}{rgb}{0.000000,0.000000,0.000000}%
\pgfsetstrokecolor{currentstroke}%
\pgfsetdash{}{0pt}%
\pgfpathmoveto{\pgfqpoint{5.479148in}{3.103790in}}%
\pgfpathlineto{\pgfqpoint{5.492993in}{3.111597in}}%
\pgfpathlineto{\pgfqpoint{5.506854in}{3.119556in}}%
\pgfpathlineto{\pgfqpoint{5.520731in}{3.127669in}}%
\pgfpathlineto{\pgfqpoint{5.534625in}{3.135935in}}%
\pgfpathlineto{\pgfqpoint{5.541630in}{3.140063in}}%
\pgfpathlineto{\pgfqpoint{5.548631in}{3.144260in}}%
\pgfpathlineto{\pgfqpoint{5.555628in}{3.148533in}}%
\pgfpathlineto{\pgfqpoint{5.562622in}{3.152887in}}%
\pgfpathlineto{\pgfqpoint{5.548753in}{3.145151in}}%
\pgfpathlineto{\pgfqpoint{5.534902in}{3.137566in}}%
\pgfpathlineto{\pgfqpoint{5.521066in}{3.130134in}}%
\pgfpathlineto{\pgfqpoint{5.507247in}{3.122854in}}%
\pgfpathlineto{\pgfqpoint{5.500228in}{3.117962in}}%
\pgfpathlineto{\pgfqpoint{5.493205in}{3.113158in}}%
\pgfpathlineto{\pgfqpoint{5.486178in}{3.108436in}}%
\pgfpathlineto{\pgfqpoint{5.479148in}{3.103790in}}%
\pgfpathclose%
\pgfusepath{fill}%
\end{pgfscope}%
\begin{pgfscope}%
\pgfpathrectangle{\pgfqpoint{1.254980in}{0.150000in}}{\pgfqpoint{5.490039in}{5.490039in}}%
\pgfusepath{clip}%
\pgfsetbuttcap%
\pgfsetroundjoin%
\definecolor{currentfill}{rgb}{0.127568,0.566949,0.550556}%
\pgfsetfillcolor{currentfill}%
\pgfsetfillopacity{0.700000}%
\pgfsetlinewidth{0.000000pt}%
\definecolor{currentstroke}{rgb}{0.000000,0.000000,0.000000}%
\pgfsetstrokecolor{currentstroke}%
\pgfsetdash{}{0pt}%
\pgfpathmoveto{\pgfqpoint{5.562622in}{3.152887in}}%
\pgfpathlineto{\pgfqpoint{5.576506in}{3.160776in}}%
\pgfpathlineto{\pgfqpoint{5.590408in}{3.168816in}}%
\pgfpathlineto{\pgfqpoint{5.604326in}{3.177009in}}%
\pgfpathlineto{\pgfqpoint{5.618262in}{3.185354in}}%
\pgfpathlineto{\pgfqpoint{5.625224in}{3.189248in}}%
\pgfpathlineto{\pgfqpoint{5.632184in}{3.193228in}}%
\pgfpathlineto{\pgfqpoint{5.639140in}{3.197301in}}%
\pgfpathlineto{\pgfqpoint{5.646093in}{3.201473in}}%
\pgfpathlineto{\pgfqpoint{5.632186in}{3.193686in}}%
\pgfpathlineto{\pgfqpoint{5.618296in}{3.186051in}}%
\pgfpathlineto{\pgfqpoint{5.604422in}{3.178567in}}%
\pgfpathlineto{\pgfqpoint{5.590564in}{3.171234in}}%
\pgfpathlineto{\pgfqpoint{5.583583in}{3.166496in}}%
\pgfpathlineto{\pgfqpoint{5.576599in}{3.161862in}}%
\pgfpathlineto{\pgfqpoint{5.569612in}{3.157328in}}%
\pgfpathlineto{\pgfqpoint{5.562622in}{3.152887in}}%
\pgfpathclose%
\pgfusepath{fill}%
\end{pgfscope}%
\begin{pgfscope}%
\pgfpathrectangle{\pgfqpoint{1.254980in}{0.150000in}}{\pgfqpoint{5.490039in}{5.490039in}}%
\pgfusepath{clip}%
\pgfsetbuttcap%
\pgfsetroundjoin%
\definecolor{currentfill}{rgb}{0.283229,0.120777,0.440584}%
\pgfsetfillcolor{currentfill}%
\pgfsetfillopacity{0.700000}%
\pgfsetlinewidth{0.000000pt}%
\definecolor{currentstroke}{rgb}{0.000000,0.000000,0.000000}%
\pgfsetstrokecolor{currentstroke}%
\pgfsetdash{}{0pt}%
\pgfpathmoveto{\pgfqpoint{2.933214in}{2.137402in}}%
\pgfpathlineto{\pgfqpoint{2.946352in}{2.125803in}}%
\pgfpathlineto{\pgfqpoint{2.959487in}{2.114431in}}%
\pgfpathlineto{\pgfqpoint{2.972620in}{2.103285in}}%
\pgfpathlineto{\pgfqpoint{2.985749in}{2.092361in}}%
\pgfpathlineto{\pgfqpoint{2.993813in}{2.098777in}}%
\pgfpathlineto{\pgfqpoint{3.001867in}{2.105303in}}%
\pgfpathlineto{\pgfqpoint{3.009913in}{2.111938in}}%
\pgfpathlineto{\pgfqpoint{3.017951in}{2.118678in}}%
\pgfpathlineto{\pgfqpoint{3.004844in}{2.129324in}}%
\pgfpathlineto{\pgfqpoint{2.991735in}{2.140193in}}%
\pgfpathlineto{\pgfqpoint{2.978624in}{2.151287in}}%
\pgfpathlineto{\pgfqpoint{2.965510in}{2.162608in}}%
\pgfpathlineto{\pgfqpoint{2.957449in}{2.156135in}}%
\pgfpathlineto{\pgfqpoint{2.949380in}{2.149774in}}%
\pgfpathlineto{\pgfqpoint{2.941301in}{2.143529in}}%
\pgfpathlineto{\pgfqpoint{2.933214in}{2.137402in}}%
\pgfpathclose%
\pgfusepath{fill}%
\end{pgfscope}%
\begin{pgfscope}%
\pgfpathrectangle{\pgfqpoint{1.254980in}{0.150000in}}{\pgfqpoint{5.490039in}{5.490039in}}%
\pgfusepath{clip}%
\pgfsetbuttcap%
\pgfsetroundjoin%
\definecolor{currentfill}{rgb}{0.122606,0.585371,0.546557}%
\pgfsetfillcolor{currentfill}%
\pgfsetfillopacity{0.700000}%
\pgfsetlinewidth{0.000000pt}%
\definecolor{currentstroke}{rgb}{0.000000,0.000000,0.000000}%
\pgfsetstrokecolor{currentstroke}%
\pgfsetdash{}{0pt}%
\pgfpathmoveto{\pgfqpoint{5.646093in}{3.201473in}}%
\pgfpathlineto{\pgfqpoint{5.660017in}{3.209411in}}%
\pgfpathlineto{\pgfqpoint{5.673959in}{3.217501in}}%
\pgfpathlineto{\pgfqpoint{5.687917in}{3.225742in}}%
\pgfpathlineto{\pgfqpoint{5.701892in}{3.234135in}}%
\pgfpathlineto{\pgfqpoint{5.708814in}{3.237835in}}%
\pgfpathlineto{\pgfqpoint{5.715732in}{3.241640in}}%
\pgfpathlineto{\pgfqpoint{5.722649in}{3.245556in}}%
\pgfpathlineto{\pgfqpoint{5.729563in}{3.249590in}}%
\pgfpathlineto{\pgfqpoint{5.715617in}{3.241785in}}%
\pgfpathlineto{\pgfqpoint{5.701689in}{3.234130in}}%
\pgfpathlineto{\pgfqpoint{5.687777in}{3.226626in}}%
\pgfpathlineto{\pgfqpoint{5.673883in}{3.219273in}}%
\pgfpathlineto{\pgfqpoint{5.666938in}{3.214643in}}%
\pgfpathlineto{\pgfqpoint{5.659992in}{3.210137in}}%
\pgfpathlineto{\pgfqpoint{5.653044in}{3.205749in}}%
\pgfpathlineto{\pgfqpoint{5.646093in}{3.201473in}}%
\pgfpathclose%
\pgfusepath{fill}%
\end{pgfscope}%
\begin{pgfscope}%
\pgfpathrectangle{\pgfqpoint{1.254980in}{0.150000in}}{\pgfqpoint{5.490039in}{5.490039in}}%
\pgfusepath{clip}%
\pgfsetbuttcap%
\pgfsetroundjoin%
\definecolor{currentfill}{rgb}{0.221989,0.339161,0.548752}%
\pgfsetfillcolor{currentfill}%
\pgfsetfillopacity{0.700000}%
\pgfsetlinewidth{0.000000pt}%
\definecolor{currentstroke}{rgb}{0.000000,0.000000,0.000000}%
\pgfsetstrokecolor{currentstroke}%
\pgfsetdash{}{0pt}%
\pgfpathmoveto{\pgfqpoint{4.615256in}{2.557933in}}%
\pgfpathlineto{\pgfqpoint{4.628692in}{2.563362in}}%
\pgfpathlineto{\pgfqpoint{4.642141in}{2.568953in}}%
\pgfpathlineto{\pgfqpoint{4.655603in}{2.574704in}}%
\pgfpathlineto{\pgfqpoint{4.669078in}{2.580616in}}%
\pgfpathlineto{\pgfqpoint{4.676497in}{2.588394in}}%
\pgfpathlineto{\pgfqpoint{4.683911in}{2.596129in}}%
\pgfpathlineto{\pgfqpoint{4.691318in}{2.603822in}}%
\pgfpathlineto{\pgfqpoint{4.698720in}{2.611477in}}%
\pgfpathlineto{\pgfqpoint{4.685256in}{2.605773in}}%
\pgfpathlineto{\pgfqpoint{4.671804in}{2.600229in}}%
\pgfpathlineto{\pgfqpoint{4.658365in}{2.594845in}}%
\pgfpathlineto{\pgfqpoint{4.644938in}{2.589622in}}%
\pgfpathlineto{\pgfqpoint{4.637526in}{2.581750in}}%
\pgfpathlineto{\pgfqpoint{4.630108in}{2.573846in}}%
\pgfpathlineto{\pgfqpoint{4.622685in}{2.565908in}}%
\pgfpathlineto{\pgfqpoint{4.615256in}{2.557933in}}%
\pgfpathclose%
\pgfusepath{fill}%
\end{pgfscope}%
\begin{pgfscope}%
\pgfpathrectangle{\pgfqpoint{1.254980in}{0.150000in}}{\pgfqpoint{5.490039in}{5.490039in}}%
\pgfusepath{clip}%
\pgfsetbuttcap%
\pgfsetroundjoin%
\definecolor{currentfill}{rgb}{0.119738,0.603785,0.541400}%
\pgfsetfillcolor{currentfill}%
\pgfsetfillopacity{0.700000}%
\pgfsetlinewidth{0.000000pt}%
\definecolor{currentstroke}{rgb}{0.000000,0.000000,0.000000}%
\pgfsetstrokecolor{currentstroke}%
\pgfsetdash{}{0pt}%
\pgfpathmoveto{\pgfqpoint{5.729563in}{3.249590in}}%
\pgfpathlineto{\pgfqpoint{5.743525in}{3.257547in}}%
\pgfpathlineto{\pgfqpoint{5.757505in}{3.265654in}}%
\pgfpathlineto{\pgfqpoint{5.771502in}{3.273912in}}%
\pgfpathlineto{\pgfqpoint{5.785516in}{3.282322in}}%
\pgfpathlineto{\pgfqpoint{5.792397in}{3.285874in}}%
\pgfpathlineto{\pgfqpoint{5.799276in}{3.289551in}}%
\pgfpathlineto{\pgfqpoint{5.806153in}{3.293359in}}%
\pgfpathlineto{\pgfqpoint{5.813029in}{3.297305in}}%
\pgfpathlineto{\pgfqpoint{5.799047in}{3.289512in}}%
\pgfpathlineto{\pgfqpoint{5.785082in}{3.281869in}}%
\pgfpathlineto{\pgfqpoint{5.771134in}{3.274377in}}%
\pgfpathlineto{\pgfqpoint{5.757203in}{3.267034in}}%
\pgfpathlineto{\pgfqpoint{5.750295in}{3.262464in}}%
\pgfpathlineto{\pgfqpoint{5.743385in}{3.258037in}}%
\pgfpathlineto{\pgfqpoint{5.736475in}{3.253748in}}%
\pgfpathlineto{\pgfqpoint{5.729563in}{3.249590in}}%
\pgfpathclose%
\pgfusepath{fill}%
\end{pgfscope}%
\begin{pgfscope}%
\pgfpathrectangle{\pgfqpoint{1.254980in}{0.150000in}}{\pgfqpoint{5.490039in}{5.490039in}}%
\pgfusepath{clip}%
\pgfsetbuttcap%
\pgfsetroundjoin%
\definecolor{currentfill}{rgb}{0.120081,0.622161,0.534946}%
\pgfsetfillcolor{currentfill}%
\pgfsetfillopacity{0.700000}%
\pgfsetlinewidth{0.000000pt}%
\definecolor{currentstroke}{rgb}{0.000000,0.000000,0.000000}%
\pgfsetstrokecolor{currentstroke}%
\pgfsetdash{}{0pt}%
\pgfpathmoveto{\pgfqpoint{5.813029in}{3.297305in}}%
\pgfpathlineto{\pgfqpoint{5.827029in}{3.305248in}}%
\pgfpathlineto{\pgfqpoint{5.841046in}{3.313341in}}%
\pgfpathlineto{\pgfqpoint{5.855081in}{3.321585in}}%
\pgfpathlineto{\pgfqpoint{5.869133in}{3.329979in}}%
\pgfpathlineto{\pgfqpoint{5.875974in}{3.333435in}}%
\pgfpathlineto{\pgfqpoint{5.882815in}{3.337036in}}%
\pgfpathlineto{\pgfqpoint{5.889655in}{3.340789in}}%
\pgfpathlineto{\pgfqpoint{5.896494in}{3.344701in}}%
\pgfpathlineto{\pgfqpoint{5.882477in}{3.336953in}}%
\pgfpathlineto{\pgfqpoint{5.868476in}{3.329353in}}%
\pgfpathlineto{\pgfqpoint{5.854493in}{3.321903in}}%
\pgfpathlineto{\pgfqpoint{5.840527in}{3.314603in}}%
\pgfpathlineto{\pgfqpoint{5.833653in}{3.310037in}}%
\pgfpathlineto{\pgfqpoint{5.826779in}{3.305637in}}%
\pgfpathlineto{\pgfqpoint{5.819904in}{3.301395in}}%
\pgfpathlineto{\pgfqpoint{5.813029in}{3.297305in}}%
\pgfpathclose%
\pgfusepath{fill}%
\end{pgfscope}%
\begin{pgfscope}%
\pgfpathrectangle{\pgfqpoint{1.254980in}{0.150000in}}{\pgfqpoint{5.490039in}{5.490039in}}%
\pgfusepath{clip}%
\pgfsetbuttcap%
\pgfsetroundjoin%
\definecolor{currentfill}{rgb}{0.277941,0.056324,0.381191}%
\pgfsetfillcolor{currentfill}%
\pgfsetfillopacity{0.700000}%
\pgfsetlinewidth{0.000000pt}%
\definecolor{currentstroke}{rgb}{0.000000,0.000000,0.000000}%
\pgfsetstrokecolor{currentstroke}%
\pgfsetdash{}{0pt}%
\pgfpathmoveto{\pgfqpoint{3.531413in}{1.997190in}}%
\pgfpathlineto{\pgfqpoint{3.544504in}{1.993851in}}%
\pgfpathlineto{\pgfqpoint{3.557600in}{1.990696in}}%
\pgfpathlineto{\pgfqpoint{3.570700in}{1.987724in}}%
\pgfpathlineto{\pgfqpoint{3.583804in}{1.984936in}}%
\pgfpathlineto{\pgfqpoint{3.591605in}{1.994418in}}%
\pgfpathlineto{\pgfqpoint{3.599400in}{2.003918in}}%
\pgfpathlineto{\pgfqpoint{3.607190in}{2.013434in}}%
\pgfpathlineto{\pgfqpoint{3.614974in}{2.022964in}}%
\pgfpathlineto{\pgfqpoint{3.601880in}{2.025592in}}%
\pgfpathlineto{\pgfqpoint{3.588792in}{2.028403in}}%
\pgfpathlineto{\pgfqpoint{3.575707in}{2.031397in}}%
\pgfpathlineto{\pgfqpoint{3.562628in}{2.034575in}}%
\pgfpathlineto{\pgfqpoint{3.554832in}{2.025195in}}%
\pgfpathlineto{\pgfqpoint{3.547031in}{2.015837in}}%
\pgfpathlineto{\pgfqpoint{3.539225in}{2.006502in}}%
\pgfpathlineto{\pgfqpoint{3.531413in}{1.997190in}}%
\pgfpathclose%
\pgfusepath{fill}%
\end{pgfscope}%
\begin{pgfscope}%
\pgfpathrectangle{\pgfqpoint{1.254980in}{0.150000in}}{\pgfqpoint{5.490039in}{5.490039in}}%
\pgfusepath{clip}%
\pgfsetbuttcap%
\pgfsetroundjoin%
\definecolor{currentfill}{rgb}{0.124780,0.640461,0.527068}%
\pgfsetfillcolor{currentfill}%
\pgfsetfillopacity{0.700000}%
\pgfsetlinewidth{0.000000pt}%
\definecolor{currentstroke}{rgb}{0.000000,0.000000,0.000000}%
\pgfsetstrokecolor{currentstroke}%
\pgfsetdash{}{0pt}%
\pgfpathmoveto{\pgfqpoint{5.896494in}{3.344701in}}%
\pgfpathlineto{\pgfqpoint{5.910530in}{3.352600in}}%
\pgfpathlineto{\pgfqpoint{5.924583in}{3.360648in}}%
\pgfpathlineto{\pgfqpoint{5.938654in}{3.368846in}}%
\pgfpathlineto{\pgfqpoint{5.952743in}{3.377194in}}%
\pgfpathlineto{\pgfqpoint{5.959546in}{3.380609in}}%
\pgfpathlineto{\pgfqpoint{5.966350in}{3.384192in}}%
\pgfpathlineto{\pgfqpoint{5.973155in}{3.387949in}}%
\pgfpathlineto{\pgfqpoint{5.979960in}{3.391887in}}%
\pgfpathlineto{\pgfqpoint{5.965908in}{3.384214in}}%
\pgfpathlineto{\pgfqpoint{5.951874in}{3.376689in}}%
\pgfpathlineto{\pgfqpoint{5.937857in}{3.369313in}}%
\pgfpathlineto{\pgfqpoint{5.923857in}{3.362086in}}%
\pgfpathlineto{\pgfqpoint{5.917015in}{3.357465in}}%
\pgfpathlineto{\pgfqpoint{5.910174in}{3.353032in}}%
\pgfpathlineto{\pgfqpoint{5.903334in}{3.348780in}}%
\pgfpathlineto{\pgfqpoint{5.896494in}{3.344701in}}%
\pgfpathclose%
\pgfusepath{fill}%
\end{pgfscope}%
\begin{pgfscope}%
\pgfpathrectangle{\pgfqpoint{1.254980in}{0.150000in}}{\pgfqpoint{5.490039in}{5.490039in}}%
\pgfusepath{clip}%
\pgfsetbuttcap%
\pgfsetroundjoin%
\definecolor{currentfill}{rgb}{0.201239,0.383670,0.554294}%
\pgfsetfillcolor{currentfill}%
\pgfsetfillopacity{0.700000}%
\pgfsetlinewidth{0.000000pt}%
\definecolor{currentstroke}{rgb}{0.000000,0.000000,0.000000}%
\pgfsetstrokecolor{currentstroke}%
\pgfsetdash{}{0pt}%
\pgfpathmoveto{\pgfqpoint{2.456172in}{2.722173in}}%
\pgfpathlineto{\pgfqpoint{2.469584in}{2.701056in}}%
\pgfpathlineto{\pgfqpoint{2.482984in}{2.680248in}}%
\pgfpathlineto{\pgfqpoint{2.496372in}{2.659746in}}%
\pgfpathlineto{\pgfqpoint{2.509748in}{2.639546in}}%
\pgfpathlineto{\pgfqpoint{2.518074in}{2.643303in}}%
\pgfpathlineto{\pgfqpoint{2.526388in}{2.647236in}}%
\pgfpathlineto{\pgfqpoint{2.534688in}{2.651344in}}%
\pgfpathlineto{\pgfqpoint{2.542976in}{2.655622in}}%
\pgfpathlineto{\pgfqpoint{2.529635in}{2.675528in}}%
\pgfpathlineto{\pgfqpoint{2.516282in}{2.695735in}}%
\pgfpathlineto{\pgfqpoint{2.502917in}{2.716247in}}%
\pgfpathlineto{\pgfqpoint{2.489540in}{2.737068in}}%
\pgfpathlineto{\pgfqpoint{2.481217in}{2.733073in}}%
\pgfpathlineto{\pgfqpoint{2.472882in}{2.729258in}}%
\pgfpathlineto{\pgfqpoint{2.464534in}{2.725624in}}%
\pgfpathlineto{\pgfqpoint{2.456172in}{2.722173in}}%
\pgfpathclose%
\pgfusepath{fill}%
\end{pgfscope}%
\begin{pgfscope}%
\pgfpathrectangle{\pgfqpoint{1.254980in}{0.150000in}}{\pgfqpoint{5.490039in}{5.490039in}}%
\pgfusepath{clip}%
\pgfsetbuttcap%
\pgfsetroundjoin%
\definecolor{currentfill}{rgb}{0.210503,0.363727,0.552206}%
\pgfsetfillcolor{currentfill}%
\pgfsetfillopacity{0.700000}%
\pgfsetlinewidth{0.000000pt}%
\definecolor{currentstroke}{rgb}{0.000000,0.000000,0.000000}%
\pgfsetstrokecolor{currentstroke}%
\pgfsetdash{}{0pt}%
\pgfpathmoveto{\pgfqpoint{4.698720in}{2.611477in}}%
\pgfpathlineto{\pgfqpoint{4.712198in}{2.617341in}}%
\pgfpathlineto{\pgfqpoint{4.725689in}{2.623364in}}%
\pgfpathlineto{\pgfqpoint{4.739193in}{2.629548in}}%
\pgfpathlineto{\pgfqpoint{4.752710in}{2.635891in}}%
\pgfpathlineto{\pgfqpoint{4.760096in}{2.643284in}}%
\pgfpathlineto{\pgfqpoint{4.767475in}{2.650636in}}%
\pgfpathlineto{\pgfqpoint{4.774849in}{2.657952in}}%
\pgfpathlineto{\pgfqpoint{4.782217in}{2.665233in}}%
\pgfpathlineto{\pgfqpoint{4.768710in}{2.659127in}}%
\pgfpathlineto{\pgfqpoint{4.755217in}{2.653180in}}%
\pgfpathlineto{\pgfqpoint{4.741737in}{2.647392in}}%
\pgfpathlineto{\pgfqpoint{4.728270in}{2.641764in}}%
\pgfpathlineto{\pgfqpoint{4.720891in}{2.634236in}}%
\pgfpathlineto{\pgfqpoint{4.713507in}{2.626681in}}%
\pgfpathlineto{\pgfqpoint{4.706116in}{2.619095in}}%
\pgfpathlineto{\pgfqpoint{4.698720in}{2.611477in}}%
\pgfpathclose%
\pgfusepath{fill}%
\end{pgfscope}%
\begin{pgfscope}%
\pgfpathrectangle{\pgfqpoint{1.254980in}{0.150000in}}{\pgfqpoint{5.490039in}{5.490039in}}%
\pgfusepath{clip}%
\pgfsetbuttcap%
\pgfsetroundjoin%
\definecolor{currentfill}{rgb}{0.276022,0.044167,0.370164}%
\pgfsetfillcolor{currentfill}%
\pgfsetfillopacity{0.700000}%
\pgfsetlinewidth{0.000000pt}%
\definecolor{currentstroke}{rgb}{0.000000,0.000000,0.000000}%
\pgfsetstrokecolor{currentstroke}%
\pgfsetdash{}{0pt}%
\pgfpathmoveto{\pgfqpoint{3.311548in}{1.983383in}}%
\pgfpathlineto{\pgfqpoint{3.324631in}{1.977356in}}%
\pgfpathlineto{\pgfqpoint{3.337716in}{1.971525in}}%
\pgfpathlineto{\pgfqpoint{3.350803in}{1.965888in}}%
\pgfpathlineto{\pgfqpoint{3.363893in}{1.960445in}}%
\pgfpathlineto{\pgfqpoint{3.371780in}{1.969054in}}%
\pgfpathlineto{\pgfqpoint{3.379661in}{1.977714in}}%
\pgfpathlineto{\pgfqpoint{3.387536in}{1.986424in}}%
\pgfpathlineto{\pgfqpoint{3.395404in}{1.995180in}}%
\pgfpathlineto{\pgfqpoint{3.382330in}{2.000407in}}%
\pgfpathlineto{\pgfqpoint{3.369258in}{2.005827in}}%
\pgfpathlineto{\pgfqpoint{3.356189in}{2.011441in}}%
\pgfpathlineto{\pgfqpoint{3.343121in}{2.017250in}}%
\pgfpathlineto{\pgfqpoint{3.335238in}{2.008700in}}%
\pgfpathlineto{\pgfqpoint{3.327348in}{2.000204in}}%
\pgfpathlineto{\pgfqpoint{3.319451in}{1.991764in}}%
\pgfpathlineto{\pgfqpoint{3.311548in}{1.983383in}}%
\pgfpathclose%
\pgfusepath{fill}%
\end{pgfscope}%
\begin{pgfscope}%
\pgfpathrectangle{\pgfqpoint{1.254980in}{0.150000in}}{\pgfqpoint{5.490039in}{5.490039in}}%
\pgfusepath{clip}%
\pgfsetbuttcap%
\pgfsetroundjoin%
\definecolor{currentfill}{rgb}{0.282656,0.100196,0.422160}%
\pgfsetfillcolor{currentfill}%
\pgfsetfillopacity{0.700000}%
\pgfsetlinewidth{0.000000pt}%
\definecolor{currentstroke}{rgb}{0.000000,0.000000,0.000000}%
\pgfsetstrokecolor{currentstroke}%
\pgfsetdash{}{0pt}%
\pgfpathmoveto{\pgfqpoint{2.985749in}{2.092361in}}%
\pgfpathlineto{\pgfqpoint{2.998877in}{2.081660in}}%
\pgfpathlineto{\pgfqpoint{3.012001in}{2.071179in}}%
\pgfpathlineto{\pgfqpoint{3.025124in}{2.060917in}}%
\pgfpathlineto{\pgfqpoint{3.038245in}{2.050873in}}%
\pgfpathlineto{\pgfqpoint{3.046285in}{2.057576in}}%
\pgfpathlineto{\pgfqpoint{3.054317in}{2.064382in}}%
\pgfpathlineto{\pgfqpoint{3.062341in}{2.071290in}}%
\pgfpathlineto{\pgfqpoint{3.070356in}{2.078296in}}%
\pgfpathlineto{\pgfqpoint{3.057257in}{2.088064in}}%
\pgfpathlineto{\pgfqpoint{3.044157in}{2.098050in}}%
\pgfpathlineto{\pgfqpoint{3.031055in}{2.108254in}}%
\pgfpathlineto{\pgfqpoint{3.017951in}{2.118678in}}%
\pgfpathlineto{\pgfqpoint{3.009913in}{2.111938in}}%
\pgfpathlineto{\pgfqpoint{3.001867in}{2.105303in}}%
\pgfpathlineto{\pgfqpoint{2.993813in}{2.098777in}}%
\pgfpathlineto{\pgfqpoint{2.985749in}{2.092361in}}%
\pgfpathclose%
\pgfusepath{fill}%
\end{pgfscope}%
\begin{pgfscope}%
\pgfpathrectangle{\pgfqpoint{1.254980in}{0.150000in}}{\pgfqpoint{5.490039in}{5.490039in}}%
\pgfusepath{clip}%
\pgfsetbuttcap%
\pgfsetroundjoin%
\definecolor{currentfill}{rgb}{0.277941,0.056324,0.381191}%
\pgfsetfillcolor{currentfill}%
\pgfsetfillopacity{0.700000}%
\pgfsetlinewidth{0.000000pt}%
\definecolor{currentstroke}{rgb}{0.000000,0.000000,0.000000}%
\pgfsetstrokecolor{currentstroke}%
\pgfsetdash{}{0pt}%
\pgfpathmoveto{\pgfqpoint{3.175110in}{2.007808in}}%
\pgfpathlineto{\pgfqpoint{3.188202in}{1.999935in}}%
\pgfpathlineto{\pgfqpoint{3.201295in}{1.992265in}}%
\pgfpathlineto{\pgfqpoint{3.214389in}{1.984799in}}%
\pgfpathlineto{\pgfqpoint{3.227483in}{1.977535in}}%
\pgfpathlineto{\pgfqpoint{3.235430in}{1.985409in}}%
\pgfpathlineto{\pgfqpoint{3.243371in}{1.993357in}}%
\pgfpathlineto{\pgfqpoint{3.251305in}{2.001375in}}%
\pgfpathlineto{\pgfqpoint{3.259232in}{2.009462in}}%
\pgfpathlineto{\pgfqpoint{3.246156in}{2.016481in}}%
\pgfpathlineto{\pgfqpoint{3.233080in}{2.023701in}}%
\pgfpathlineto{\pgfqpoint{3.220006in}{2.031124in}}%
\pgfpathlineto{\pgfqpoint{3.206932in}{2.038752in}}%
\pgfpathlineto{\pgfqpoint{3.198987in}{2.030900in}}%
\pgfpathlineto{\pgfqpoint{3.191035in}{2.023124in}}%
\pgfpathlineto{\pgfqpoint{3.183076in}{2.015426in}}%
\pgfpathlineto{\pgfqpoint{3.175110in}{2.007808in}}%
\pgfpathclose%
\pgfusepath{fill}%
\end{pgfscope}%
\begin{pgfscope}%
\pgfpathrectangle{\pgfqpoint{1.254980in}{0.150000in}}{\pgfqpoint{5.490039in}{5.490039in}}%
\pgfusepath{clip}%
\pgfsetbuttcap%
\pgfsetroundjoin%
\definecolor{currentfill}{rgb}{0.201239,0.383670,0.554294}%
\pgfsetfillcolor{currentfill}%
\pgfsetfillopacity{0.700000}%
\pgfsetlinewidth{0.000000pt}%
\definecolor{currentstroke}{rgb}{0.000000,0.000000,0.000000}%
\pgfsetstrokecolor{currentstroke}%
\pgfsetdash{}{0pt}%
\pgfpathmoveto{\pgfqpoint{4.782217in}{2.665233in}}%
\pgfpathlineto{\pgfqpoint{4.795737in}{2.671498in}}%
\pgfpathlineto{\pgfqpoint{4.809270in}{2.677922in}}%
\pgfpathlineto{\pgfqpoint{4.822818in}{2.684506in}}%
\pgfpathlineto{\pgfqpoint{4.836379in}{2.691247in}}%
\pgfpathlineto{\pgfqpoint{4.843729in}{2.698242in}}%
\pgfpathlineto{\pgfqpoint{4.851073in}{2.705201in}}%
\pgfpathlineto{\pgfqpoint{4.858411in}{2.712127in}}%
\pgfpathlineto{\pgfqpoint{4.865743in}{2.719025in}}%
\pgfpathlineto{\pgfqpoint{4.852194in}{2.712550in}}%
\pgfpathlineto{\pgfqpoint{4.838659in}{2.706232in}}%
\pgfpathlineto{\pgfqpoint{4.825137in}{2.700073in}}%
\pgfpathlineto{\pgfqpoint{4.811629in}{2.694073in}}%
\pgfpathlineto{\pgfqpoint{4.804284in}{2.686900in}}%
\pgfpathlineto{\pgfqpoint{4.796934in}{2.679704in}}%
\pgfpathlineto{\pgfqpoint{4.789578in}{2.672483in}}%
\pgfpathlineto{\pgfqpoint{4.782217in}{2.665233in}}%
\pgfpathclose%
\pgfusepath{fill}%
\end{pgfscope}%
\begin{pgfscope}%
\pgfpathrectangle{\pgfqpoint{1.254980in}{0.150000in}}{\pgfqpoint{5.490039in}{5.490039in}}%
\pgfusepath{clip}%
\pgfsetbuttcap%
\pgfsetroundjoin%
\definecolor{currentfill}{rgb}{0.134692,0.658636,0.517649}%
\pgfsetfillcolor{currentfill}%
\pgfsetfillopacity{0.700000}%
\pgfsetlinewidth{0.000000pt}%
\definecolor{currentstroke}{rgb}{0.000000,0.000000,0.000000}%
\pgfsetstrokecolor{currentstroke}%
\pgfsetdash{}{0pt}%
\pgfpathmoveto{\pgfqpoint{5.979960in}{3.391887in}}%
\pgfpathlineto{\pgfqpoint{5.994030in}{3.399710in}}%
\pgfpathlineto{\pgfqpoint{6.008118in}{3.407682in}}%
\pgfpathlineto{\pgfqpoint{6.022223in}{3.415803in}}%
\pgfpathlineto{\pgfqpoint{6.036347in}{3.424073in}}%
\pgfpathlineto{\pgfqpoint{6.043116in}{3.427509in}}%
\pgfpathlineto{\pgfqpoint{6.049886in}{3.431136in}}%
\pgfpathlineto{\pgfqpoint{6.056657in}{3.434960in}}%
\pgfpathlineto{\pgfqpoint{6.042563in}{3.427215in}}%
\pgfpathlineto{\pgfqpoint{6.028486in}{3.419618in}}%
\pgfpathlineto{\pgfqpoint{6.014427in}{3.412169in}}%
\pgfpathlineto{\pgfqpoint{6.000385in}{3.404869in}}%
\pgfpathlineto{\pgfqpoint{5.993575in}{3.400340in}}%
\pgfpathlineto{\pgfqpoint{5.986767in}{3.396015in}}%
\pgfpathlineto{\pgfqpoint{5.979960in}{3.391887in}}%
\pgfpathclose%
\pgfusepath{fill}%
\end{pgfscope}%
\begin{pgfscope}%
\pgfpathrectangle{\pgfqpoint{1.254980in}{0.150000in}}{\pgfqpoint{5.490039in}{5.490039in}}%
\pgfusepath{clip}%
\pgfsetbuttcap%
\pgfsetroundjoin%
\definecolor{currentfill}{rgb}{0.282884,0.135920,0.453427}%
\pgfsetfillcolor{currentfill}%
\pgfsetfillopacity{0.700000}%
\pgfsetlinewidth{0.000000pt}%
\definecolor{currentstroke}{rgb}{0.000000,0.000000,0.000000}%
\pgfsetstrokecolor{currentstroke}%
\pgfsetdash{}{0pt}%
\pgfpathmoveto{\pgfqpoint{3.917803in}{2.125465in}}%
\pgfpathlineto{\pgfqpoint{3.930977in}{2.126162in}}%
\pgfpathlineto{\pgfqpoint{3.944159in}{2.127031in}}%
\pgfpathlineto{\pgfqpoint{3.957349in}{2.128072in}}%
\pgfpathlineto{\pgfqpoint{3.970547in}{2.129283in}}%
\pgfpathlineto{\pgfqpoint{3.978221in}{2.139227in}}%
\pgfpathlineto{\pgfqpoint{3.985889in}{2.149142in}}%
\pgfpathlineto{\pgfqpoint{3.993552in}{2.159027in}}%
\pgfpathlineto{\pgfqpoint{4.001211in}{2.168882in}}%
\pgfpathlineto{\pgfqpoint{3.988020in}{2.167622in}}%
\pgfpathlineto{\pgfqpoint{3.974837in}{2.166532in}}%
\pgfpathlineto{\pgfqpoint{3.961663in}{2.165613in}}%
\pgfpathlineto{\pgfqpoint{3.948496in}{2.164867in}}%
\pgfpathlineto{\pgfqpoint{3.940830in}{2.155050in}}%
\pgfpathlineto{\pgfqpoint{3.933159in}{2.145211in}}%
\pgfpathlineto{\pgfqpoint{3.925483in}{2.135349in}}%
\pgfpathlineto{\pgfqpoint{3.917803in}{2.125465in}}%
\pgfpathclose%
\pgfusepath{fill}%
\end{pgfscope}%
\begin{pgfscope}%
\pgfpathrectangle{\pgfqpoint{1.254980in}{0.150000in}}{\pgfqpoint{5.490039in}{5.490039in}}%
\pgfusepath{clip}%
\pgfsetbuttcap%
\pgfsetroundjoin%
\definecolor{currentfill}{rgb}{0.281412,0.155834,0.469201}%
\pgfsetfillcolor{currentfill}%
\pgfsetfillopacity{0.700000}%
\pgfsetlinewidth{0.000000pt}%
\definecolor{currentstroke}{rgb}{0.000000,0.000000,0.000000}%
\pgfsetstrokecolor{currentstroke}%
\pgfsetdash{}{0pt}%
\pgfpathmoveto{\pgfqpoint{4.001211in}{2.168882in}}%
\pgfpathlineto{\pgfqpoint{4.014410in}{2.170314in}}%
\pgfpathlineto{\pgfqpoint{4.027618in}{2.171915in}}%
\pgfpathlineto{\pgfqpoint{4.040835in}{2.173687in}}%
\pgfpathlineto{\pgfqpoint{4.054060in}{2.175627in}}%
\pgfpathlineto{\pgfqpoint{4.061707in}{2.185485in}}%
\pgfpathlineto{\pgfqpoint{4.069348in}{2.195306in}}%
\pgfpathlineto{\pgfqpoint{4.076984in}{2.205091in}}%
\pgfpathlineto{\pgfqpoint{4.084616in}{2.214840in}}%
\pgfpathlineto{\pgfqpoint{4.071397in}{2.212878in}}%
\pgfpathlineto{\pgfqpoint{4.058188in}{2.211086in}}%
\pgfpathlineto{\pgfqpoint{4.044987in}{2.209463in}}%
\pgfpathlineto{\pgfqpoint{4.031795in}{2.208010in}}%
\pgfpathlineto{\pgfqpoint{4.024156in}{2.198272in}}%
\pgfpathlineto{\pgfqpoint{4.016513in}{2.188505in}}%
\pgfpathlineto{\pgfqpoint{4.008864in}{2.178708in}}%
\pgfpathlineto{\pgfqpoint{4.001211in}{2.168882in}}%
\pgfpathclose%
\pgfusepath{fill}%
\end{pgfscope}%
\begin{pgfscope}%
\pgfpathrectangle{\pgfqpoint{1.254980in}{0.150000in}}{\pgfqpoint{5.490039in}{5.490039in}}%
\pgfusepath{clip}%
\pgfsetbuttcap%
\pgfsetroundjoin%
\definecolor{currentfill}{rgb}{0.190631,0.407061,0.556089}%
\pgfsetfillcolor{currentfill}%
\pgfsetfillopacity{0.700000}%
\pgfsetlinewidth{0.000000pt}%
\definecolor{currentstroke}{rgb}{0.000000,0.000000,0.000000}%
\pgfsetstrokecolor{currentstroke}%
\pgfsetdash{}{0pt}%
\pgfpathmoveto{\pgfqpoint{4.865743in}{2.719025in}}%
\pgfpathlineto{\pgfqpoint{4.879306in}{2.725659in}}%
\pgfpathlineto{\pgfqpoint{4.892883in}{2.732451in}}%
\pgfpathlineto{\pgfqpoint{4.906474in}{2.739401in}}%
\pgfpathlineto{\pgfqpoint{4.920080in}{2.746509in}}%
\pgfpathlineto{\pgfqpoint{4.927393in}{2.753096in}}%
\pgfpathlineto{\pgfqpoint{4.934700in}{2.759654in}}%
\pgfpathlineto{\pgfqpoint{4.942001in}{2.766186in}}%
\pgfpathlineto{\pgfqpoint{4.949297in}{2.772696in}}%
\pgfpathlineto{\pgfqpoint{4.935704in}{2.765883in}}%
\pgfpathlineto{\pgfqpoint{4.922126in}{2.759228in}}%
\pgfpathlineto{\pgfqpoint{4.908562in}{2.752731in}}%
\pgfpathlineto{\pgfqpoint{4.895012in}{2.746392in}}%
\pgfpathlineto{\pgfqpoint{4.887704in}{2.739577in}}%
\pgfpathlineto{\pgfqpoint{4.880389in}{2.732746in}}%
\pgfpathlineto{\pgfqpoint{4.873069in}{2.725897in}}%
\pgfpathlineto{\pgfqpoint{4.865743in}{2.719025in}}%
\pgfpathclose%
\pgfusepath{fill}%
\end{pgfscope}%
\begin{pgfscope}%
\pgfpathrectangle{\pgfqpoint{1.254980in}{0.150000in}}{\pgfqpoint{5.490039in}{5.490039in}}%
\pgfusepath{clip}%
\pgfsetbuttcap%
\pgfsetroundjoin%
\definecolor{currentfill}{rgb}{0.277018,0.050344,0.375715}%
\pgfsetfillcolor{currentfill}%
\pgfsetfillopacity{0.700000}%
\pgfsetlinewidth{0.000000pt}%
\definecolor{currentstroke}{rgb}{0.000000,0.000000,0.000000}%
\pgfsetstrokecolor{currentstroke}%
\pgfsetdash{}{0pt}%
\pgfpathmoveto{\pgfqpoint{3.447729in}{1.976188in}}%
\pgfpathlineto{\pgfqpoint{3.460818in}{1.971913in}}%
\pgfpathlineto{\pgfqpoint{3.473911in}{1.967827in}}%
\pgfpathlineto{\pgfqpoint{3.487007in}{1.963927in}}%
\pgfpathlineto{\pgfqpoint{3.500107in}{1.960214in}}%
\pgfpathlineto{\pgfqpoint{3.507942in}{1.969415in}}%
\pgfpathlineto{\pgfqpoint{3.515771in}{1.978645in}}%
\pgfpathlineto{\pgfqpoint{3.523595in}{1.987904in}}%
\pgfpathlineto{\pgfqpoint{3.531413in}{1.997190in}}%
\pgfpathlineto{\pgfqpoint{3.518325in}{2.000715in}}%
\pgfpathlineto{\pgfqpoint{3.505242in}{2.004426in}}%
\pgfpathlineto{\pgfqpoint{3.492162in}{2.008324in}}%
\pgfpathlineto{\pgfqpoint{3.479087in}{2.012409in}}%
\pgfpathlineto{\pgfqpoint{3.471256in}{2.003301in}}%
\pgfpathlineto{\pgfqpoint{3.463420in}{1.994228in}}%
\pgfpathlineto{\pgfqpoint{3.455577in}{1.985189in}}%
\pgfpathlineto{\pgfqpoint{3.447729in}{1.976188in}}%
\pgfpathclose%
\pgfusepath{fill}%
\end{pgfscope}%
\begin{pgfscope}%
\pgfpathrectangle{\pgfqpoint{1.254980in}{0.150000in}}{\pgfqpoint{5.490039in}{5.490039in}}%
\pgfusepath{clip}%
\pgfsetbuttcap%
\pgfsetroundjoin%
\definecolor{currentfill}{rgb}{0.278012,0.180367,0.486697}%
\pgfsetfillcolor{currentfill}%
\pgfsetfillopacity{0.700000}%
\pgfsetlinewidth{0.000000pt}%
\definecolor{currentstroke}{rgb}{0.000000,0.000000,0.000000}%
\pgfsetstrokecolor{currentstroke}%
\pgfsetdash{}{0pt}%
\pgfpathmoveto{\pgfqpoint{4.084616in}{2.214840in}}%
\pgfpathlineto{\pgfqpoint{4.097843in}{2.216970in}}%
\pgfpathlineto{\pgfqpoint{4.111080in}{2.219269in}}%
\pgfpathlineto{\pgfqpoint{4.124326in}{2.221735in}}%
\pgfpathlineto{\pgfqpoint{4.137581in}{2.224370in}}%
\pgfpathlineto{\pgfqpoint{4.145201in}{2.234087in}}%
\pgfpathlineto{\pgfqpoint{4.152815in}{2.243761in}}%
\pgfpathlineto{\pgfqpoint{4.160424in}{2.253394in}}%
\pgfpathlineto{\pgfqpoint{4.168028in}{2.262985in}}%
\pgfpathlineto{\pgfqpoint{4.154780in}{2.260357in}}%
\pgfpathlineto{\pgfqpoint{4.141540in}{2.257898in}}%
\pgfpathlineto{\pgfqpoint{4.128311in}{2.255606in}}%
\pgfpathlineto{\pgfqpoint{4.115090in}{2.253482in}}%
\pgfpathlineto{\pgfqpoint{4.107479in}{2.243874in}}%
\pgfpathlineto{\pgfqpoint{4.099863in}{2.234231in}}%
\pgfpathlineto{\pgfqpoint{4.092242in}{2.224553in}}%
\pgfpathlineto{\pgfqpoint{4.084616in}{2.214840in}}%
\pgfpathclose%
\pgfusepath{fill}%
\end{pgfscope}%
\begin{pgfscope}%
\pgfpathrectangle{\pgfqpoint{1.254980in}{0.150000in}}{\pgfqpoint{5.490039in}{5.490039in}}%
\pgfusepath{clip}%
\pgfsetbuttcap%
\pgfsetroundjoin%
\definecolor{currentfill}{rgb}{0.283197,0.115680,0.436115}%
\pgfsetfillcolor{currentfill}%
\pgfsetfillopacity{0.700000}%
\pgfsetlinewidth{0.000000pt}%
\definecolor{currentstroke}{rgb}{0.000000,0.000000,0.000000}%
\pgfsetstrokecolor{currentstroke}%
\pgfsetdash{}{0pt}%
\pgfpathmoveto{\pgfqpoint{3.834376in}{2.084958in}}%
\pgfpathlineto{\pgfqpoint{3.847529in}{2.084885in}}%
\pgfpathlineto{\pgfqpoint{3.860688in}{2.084986in}}%
\pgfpathlineto{\pgfqpoint{3.873855in}{2.085261in}}%
\pgfpathlineto{\pgfqpoint{3.887029in}{2.085709in}}%
\pgfpathlineto{\pgfqpoint{3.894730in}{2.095680in}}%
\pgfpathlineto{\pgfqpoint{3.902426in}{2.105630in}}%
\pgfpathlineto{\pgfqpoint{3.910117in}{2.115558in}}%
\pgfpathlineto{\pgfqpoint{3.917803in}{2.125465in}}%
\pgfpathlineto{\pgfqpoint{3.904636in}{2.124940in}}%
\pgfpathlineto{\pgfqpoint{3.891477in}{2.124588in}}%
\pgfpathlineto{\pgfqpoint{3.878326in}{2.124410in}}%
\pgfpathlineto{\pgfqpoint{3.865182in}{2.124405in}}%
\pgfpathlineto{\pgfqpoint{3.857488in}{2.114566in}}%
\pgfpathlineto{\pgfqpoint{3.849789in}{2.104711in}}%
\pgfpathlineto{\pgfqpoint{3.842085in}{2.094842in}}%
\pgfpathlineto{\pgfqpoint{3.834376in}{2.084958in}}%
\pgfpathclose%
\pgfusepath{fill}%
\end{pgfscope}%
\begin{pgfscope}%
\pgfpathrectangle{\pgfqpoint{1.254980in}{0.150000in}}{\pgfqpoint{5.490039in}{5.490039in}}%
\pgfusepath{clip}%
\pgfsetbuttcap%
\pgfsetroundjoin%
\definecolor{currentfill}{rgb}{0.271828,0.209303,0.504434}%
\pgfsetfillcolor{currentfill}%
\pgfsetfillopacity{0.700000}%
\pgfsetlinewidth{0.000000pt}%
\definecolor{currentstroke}{rgb}{0.000000,0.000000,0.000000}%
\pgfsetstrokecolor{currentstroke}%
\pgfsetdash{}{0pt}%
\pgfpathmoveto{\pgfqpoint{4.168028in}{2.262985in}}%
\pgfpathlineto{\pgfqpoint{4.181287in}{2.265779in}}%
\pgfpathlineto{\pgfqpoint{4.194555in}{2.268741in}}%
\pgfpathlineto{\pgfqpoint{4.207833in}{2.271869in}}%
\pgfpathlineto{\pgfqpoint{4.221122in}{2.275163in}}%
\pgfpathlineto{\pgfqpoint{4.228714in}{2.284689in}}%
\pgfpathlineto{\pgfqpoint{4.236301in}{2.294167in}}%
\pgfpathlineto{\pgfqpoint{4.243883in}{2.303600in}}%
\pgfpathlineto{\pgfqpoint{4.251459in}{2.312987in}}%
\pgfpathlineto{\pgfqpoint{4.238177in}{2.309728in}}%
\pgfpathlineto{\pgfqpoint{4.224906in}{2.306635in}}%
\pgfpathlineto{\pgfqpoint{4.211645in}{2.303709in}}%
\pgfpathlineto{\pgfqpoint{4.198393in}{2.300949in}}%
\pgfpathlineto{\pgfqpoint{4.190810in}{2.291517in}}%
\pgfpathlineto{\pgfqpoint{4.183221in}{2.282046in}}%
\pgfpathlineto{\pgfqpoint{4.175627in}{2.272535in}}%
\pgfpathlineto{\pgfqpoint{4.168028in}{2.262985in}}%
\pgfpathclose%
\pgfusepath{fill}%
\end{pgfscope}%
\begin{pgfscope}%
\pgfpathrectangle{\pgfqpoint{1.254980in}{0.150000in}}{\pgfqpoint{5.490039in}{5.490039in}}%
\pgfusepath{clip}%
\pgfsetbuttcap%
\pgfsetroundjoin%
\definecolor{currentfill}{rgb}{0.282327,0.094955,0.417331}%
\pgfsetfillcolor{currentfill}%
\pgfsetfillopacity{0.700000}%
\pgfsetlinewidth{0.000000pt}%
\definecolor{currentstroke}{rgb}{0.000000,0.000000,0.000000}%
\pgfsetstrokecolor{currentstroke}%
\pgfsetdash{}{0pt}%
\pgfpathmoveto{\pgfqpoint{3.750915in}{2.047755in}}%
\pgfpathlineto{\pgfqpoint{3.764049in}{2.046875in}}%
\pgfpathlineto{\pgfqpoint{3.777189in}{2.046172in}}%
\pgfpathlineto{\pgfqpoint{3.790336in}{2.045645in}}%
\pgfpathlineto{\pgfqpoint{3.803490in}{2.045292in}}%
\pgfpathlineto{\pgfqpoint{3.811219in}{2.055227in}}%
\pgfpathlineto{\pgfqpoint{3.818943in}{2.065150in}}%
\pgfpathlineto{\pgfqpoint{3.826662in}{2.075061in}}%
\pgfpathlineto{\pgfqpoint{3.834376in}{2.084958in}}%
\pgfpathlineto{\pgfqpoint{3.821231in}{2.085206in}}%
\pgfpathlineto{\pgfqpoint{3.808092in}{2.085628in}}%
\pgfpathlineto{\pgfqpoint{3.794961in}{2.086226in}}%
\pgfpathlineto{\pgfqpoint{3.781836in}{2.087001in}}%
\pgfpathlineto{\pgfqpoint{3.774113in}{2.077198in}}%
\pgfpathlineto{\pgfqpoint{3.766385in}{2.067389in}}%
\pgfpathlineto{\pgfqpoint{3.758653in}{2.057574in}}%
\pgfpathlineto{\pgfqpoint{3.750915in}{2.047755in}}%
\pgfpathclose%
\pgfusepath{fill}%
\end{pgfscope}%
\begin{pgfscope}%
\pgfpathrectangle{\pgfqpoint{1.254980in}{0.150000in}}{\pgfqpoint{5.490039in}{5.490039in}}%
\pgfusepath{clip}%
\pgfsetbuttcap%
\pgfsetroundjoin%
\definecolor{currentfill}{rgb}{0.180629,0.429975,0.557282}%
\pgfsetfillcolor{currentfill}%
\pgfsetfillopacity{0.700000}%
\pgfsetlinewidth{0.000000pt}%
\definecolor{currentstroke}{rgb}{0.000000,0.000000,0.000000}%
\pgfsetstrokecolor{currentstroke}%
\pgfsetdash{}{0pt}%
\pgfpathmoveto{\pgfqpoint{4.949297in}{2.772696in}}%
\pgfpathlineto{\pgfqpoint{4.962903in}{2.779666in}}%
\pgfpathlineto{\pgfqpoint{4.976524in}{2.786793in}}%
\pgfpathlineto{\pgfqpoint{4.990160in}{2.794078in}}%
\pgfpathlineto{\pgfqpoint{5.003810in}{2.801520in}}%
\pgfpathlineto{\pgfqpoint{5.011085in}{2.807696in}}%
\pgfpathlineto{\pgfqpoint{5.018354in}{2.813850in}}%
\pgfpathlineto{\pgfqpoint{5.025617in}{2.819986in}}%
\pgfpathlineto{\pgfqpoint{5.032874in}{2.826107in}}%
\pgfpathlineto{\pgfqpoint{5.019238in}{2.818991in}}%
\pgfpathlineto{\pgfqpoint{5.005617in}{2.812031in}}%
\pgfpathlineto{\pgfqpoint{4.992011in}{2.805228in}}%
\pgfpathlineto{\pgfqpoint{4.978419in}{2.798582in}}%
\pgfpathlineto{\pgfqpoint{4.971147in}{2.792125in}}%
\pgfpathlineto{\pgfqpoint{4.963869in}{2.785662in}}%
\pgfpathlineto{\pgfqpoint{4.956586in}{2.779186in}}%
\pgfpathlineto{\pgfqpoint{4.949297in}{2.772696in}}%
\pgfpathclose%
\pgfusepath{fill}%
\end{pgfscope}%
\begin{pgfscope}%
\pgfpathrectangle{\pgfqpoint{1.254980in}{0.150000in}}{\pgfqpoint{5.490039in}{5.490039in}}%
\pgfusepath{clip}%
\pgfsetbuttcap%
\pgfsetroundjoin%
\definecolor{currentfill}{rgb}{0.281446,0.084320,0.407414}%
\pgfsetfillcolor{currentfill}%
\pgfsetfillopacity{0.700000}%
\pgfsetlinewidth{0.000000pt}%
\definecolor{currentstroke}{rgb}{0.000000,0.000000,0.000000}%
\pgfsetstrokecolor{currentstroke}%
\pgfsetdash{}{0pt}%
\pgfpathmoveto{\pgfqpoint{3.038245in}{2.050873in}}%
\pgfpathlineto{\pgfqpoint{3.051364in}{2.041045in}}%
\pgfpathlineto{\pgfqpoint{3.064482in}{2.031431in}}%
\pgfpathlineto{\pgfqpoint{3.077598in}{2.022031in}}%
\pgfpathlineto{\pgfqpoint{3.090713in}{2.012843in}}%
\pgfpathlineto{\pgfqpoint{3.098732in}{2.019832in}}%
\pgfpathlineto{\pgfqpoint{3.106742in}{2.026917in}}%
\pgfpathlineto{\pgfqpoint{3.114744in}{2.034097in}}%
\pgfpathlineto{\pgfqpoint{3.122738in}{2.041368in}}%
\pgfpathlineto{\pgfqpoint{3.109644in}{2.050281in}}%
\pgfpathlineto{\pgfqpoint{3.096549in}{2.059406in}}%
\pgfpathlineto{\pgfqpoint{3.083453in}{2.068744in}}%
\pgfpathlineto{\pgfqpoint{3.070356in}{2.078296in}}%
\pgfpathlineto{\pgfqpoint{3.062341in}{2.071290in}}%
\pgfpathlineto{\pgfqpoint{3.054317in}{2.064382in}}%
\pgfpathlineto{\pgfqpoint{3.046285in}{2.057576in}}%
\pgfpathlineto{\pgfqpoint{3.038245in}{2.050873in}}%
\pgfpathclose%
\pgfusepath{fill}%
\end{pgfscope}%
\begin{pgfscope}%
\pgfpathrectangle{\pgfqpoint{1.254980in}{0.150000in}}{\pgfqpoint{5.490039in}{5.490039in}}%
\pgfusepath{clip}%
\pgfsetbuttcap%
\pgfsetroundjoin%
\definecolor{currentfill}{rgb}{0.265145,0.232956,0.516599}%
\pgfsetfillcolor{currentfill}%
\pgfsetfillopacity{0.700000}%
\pgfsetlinewidth{0.000000pt}%
\definecolor{currentstroke}{rgb}{0.000000,0.000000,0.000000}%
\pgfsetstrokecolor{currentstroke}%
\pgfsetdash{}{0pt}%
\pgfpathmoveto{\pgfqpoint{4.251459in}{2.312987in}}%
\pgfpathlineto{\pgfqpoint{4.264751in}{2.316411in}}%
\pgfpathlineto{\pgfqpoint{4.278053in}{2.320001in}}%
\pgfpathlineto{\pgfqpoint{4.291366in}{2.323756in}}%
\pgfpathlineto{\pgfqpoint{4.304690in}{2.327677in}}%
\pgfpathlineto{\pgfqpoint{4.312254in}{2.336966in}}%
\pgfpathlineto{\pgfqpoint{4.319813in}{2.346204in}}%
\pgfpathlineto{\pgfqpoint{4.327367in}{2.355393in}}%
\pgfpathlineto{\pgfqpoint{4.334915in}{2.364533in}}%
\pgfpathlineto{\pgfqpoint{4.321598in}{2.360677in}}%
\pgfpathlineto{\pgfqpoint{4.308292in}{2.356986in}}%
\pgfpathlineto{\pgfqpoint{4.294997in}{2.353459in}}%
\pgfpathlineto{\pgfqpoint{4.281713in}{2.350098in}}%
\pgfpathlineto{\pgfqpoint{4.274157in}{2.340884in}}%
\pgfpathlineto{\pgfqpoint{4.266596in}{2.331628in}}%
\pgfpathlineto{\pgfqpoint{4.259030in}{2.322329in}}%
\pgfpathlineto{\pgfqpoint{4.251459in}{2.312987in}}%
\pgfpathclose%
\pgfusepath{fill}%
\end{pgfscope}%
\begin{pgfscope}%
\pgfpathrectangle{\pgfqpoint{1.254980in}{0.150000in}}{\pgfqpoint{5.490039in}{5.490039in}}%
\pgfusepath{clip}%
\pgfsetbuttcap%
\pgfsetroundjoin%
\definecolor{currentfill}{rgb}{0.280894,0.078907,0.402329}%
\pgfsetfillcolor{currentfill}%
\pgfsetfillopacity{0.700000}%
\pgfsetlinewidth{0.000000pt}%
\definecolor{currentstroke}{rgb}{0.000000,0.000000,0.000000}%
\pgfsetstrokecolor{currentstroke}%
\pgfsetdash{}{0pt}%
\pgfpathmoveto{\pgfqpoint{3.667398in}{2.014269in}}%
\pgfpathlineto{\pgfqpoint{3.680518in}{2.012545in}}%
\pgfpathlineto{\pgfqpoint{3.693644in}{2.011000in}}%
\pgfpathlineto{\pgfqpoint{3.706775in}{2.009634in}}%
\pgfpathlineto{\pgfqpoint{3.719912in}{2.008444in}}%
\pgfpathlineto{\pgfqpoint{3.727670in}{2.018275in}}%
\pgfpathlineto{\pgfqpoint{3.735424in}{2.028104in}}%
\pgfpathlineto{\pgfqpoint{3.743172in}{2.037931in}}%
\pgfpathlineto{\pgfqpoint{3.750915in}{2.047755in}}%
\pgfpathlineto{\pgfqpoint{3.737787in}{2.048811in}}%
\pgfpathlineto{\pgfqpoint{3.724665in}{2.050045in}}%
\pgfpathlineto{\pgfqpoint{3.711550in}{2.051457in}}%
\pgfpathlineto{\pgfqpoint{3.698440in}{2.053048in}}%
\pgfpathlineto{\pgfqpoint{3.690688in}{2.043347in}}%
\pgfpathlineto{\pgfqpoint{3.682930in}{2.033649in}}%
\pgfpathlineto{\pgfqpoint{3.675167in}{2.023956in}}%
\pgfpathlineto{\pgfqpoint{3.667398in}{2.014269in}}%
\pgfpathclose%
\pgfusepath{fill}%
\end{pgfscope}%
\begin{pgfscope}%
\pgfpathrectangle{\pgfqpoint{1.254980in}{0.150000in}}{\pgfqpoint{5.490039in}{5.490039in}}%
\pgfusepath{clip}%
\pgfsetbuttcap%
\pgfsetroundjoin%
\definecolor{currentfill}{rgb}{0.171176,0.452530,0.557965}%
\pgfsetfillcolor{currentfill}%
\pgfsetfillopacity{0.700000}%
\pgfsetlinewidth{0.000000pt}%
\definecolor{currentstroke}{rgb}{0.000000,0.000000,0.000000}%
\pgfsetstrokecolor{currentstroke}%
\pgfsetdash{}{0pt}%
\pgfpathmoveto{\pgfqpoint{5.032874in}{2.826107in}}%
\pgfpathlineto{\pgfqpoint{5.046525in}{2.833381in}}%
\pgfpathlineto{\pgfqpoint{5.060190in}{2.840811in}}%
\pgfpathlineto{\pgfqpoint{5.073871in}{2.848398in}}%
\pgfpathlineto{\pgfqpoint{5.087566in}{2.856141in}}%
\pgfpathlineto{\pgfqpoint{5.094802in}{2.861908in}}%
\pgfpathlineto{\pgfqpoint{5.102031in}{2.867661in}}%
\pgfpathlineto{\pgfqpoint{5.109255in}{2.873404in}}%
\pgfpathlineto{\pgfqpoint{5.116472in}{2.879142in}}%
\pgfpathlineto{\pgfqpoint{5.102793in}{2.871753in}}%
\pgfpathlineto{\pgfqpoint{5.089128in}{2.864521in}}%
\pgfpathlineto{\pgfqpoint{5.075479in}{2.857444in}}%
\pgfpathlineto{\pgfqpoint{5.061844in}{2.850524in}}%
\pgfpathlineto{\pgfqpoint{5.054610in}{2.844422in}}%
\pgfpathlineto{\pgfqpoint{5.047371in}{2.838321in}}%
\pgfpathlineto{\pgfqpoint{5.040125in}{2.832218in}}%
\pgfpathlineto{\pgfqpoint{5.032874in}{2.826107in}}%
\pgfpathclose%
\pgfusepath{fill}%
\end{pgfscope}%
\begin{pgfscope}%
\pgfpathrectangle{\pgfqpoint{1.254980in}{0.150000in}}{\pgfqpoint{5.490039in}{5.490039in}}%
\pgfusepath{clip}%
\pgfsetbuttcap%
\pgfsetroundjoin%
\definecolor{currentfill}{rgb}{0.255645,0.260703,0.528312}%
\pgfsetfillcolor{currentfill}%
\pgfsetfillopacity{0.700000}%
\pgfsetlinewidth{0.000000pt}%
\definecolor{currentstroke}{rgb}{0.000000,0.000000,0.000000}%
\pgfsetstrokecolor{currentstroke}%
\pgfsetdash{}{0pt}%
\pgfpathmoveto{\pgfqpoint{4.334915in}{2.364533in}}%
\pgfpathlineto{\pgfqpoint{4.348242in}{2.368554in}}%
\pgfpathlineto{\pgfqpoint{4.361581in}{2.372739in}}%
\pgfpathlineto{\pgfqpoint{4.374931in}{2.377088in}}%
\pgfpathlineto{\pgfqpoint{4.388292in}{2.381600in}}%
\pgfpathlineto{\pgfqpoint{4.395828in}{2.390612in}}%
\pgfpathlineto{\pgfqpoint{4.403358in}{2.399570in}}%
\pgfpathlineto{\pgfqpoint{4.410883in}{2.408476in}}%
\pgfpathlineto{\pgfqpoint{4.418402in}{2.417333in}}%
\pgfpathlineto{\pgfqpoint{4.405048in}{2.412913in}}%
\pgfpathlineto{\pgfqpoint{4.391706in}{2.408657in}}%
\pgfpathlineto{\pgfqpoint{4.378375in}{2.404564in}}%
\pgfpathlineto{\pgfqpoint{4.365054in}{2.400636in}}%
\pgfpathlineto{\pgfqpoint{4.357528in}{2.391676in}}%
\pgfpathlineto{\pgfqpoint{4.349995in}{2.382674in}}%
\pgfpathlineto{\pgfqpoint{4.342458in}{2.373627in}}%
\pgfpathlineto{\pgfqpoint{4.334915in}{2.364533in}}%
\pgfpathclose%
\pgfusepath{fill}%
\end{pgfscope}%
\begin{pgfscope}%
\pgfpathrectangle{\pgfqpoint{1.254980in}{0.150000in}}{\pgfqpoint{5.490039in}{5.490039in}}%
\pgfusepath{clip}%
\pgfsetbuttcap%
\pgfsetroundjoin%
\definecolor{currentfill}{rgb}{0.267968,0.223549,0.512008}%
\pgfsetfillcolor{currentfill}%
\pgfsetfillopacity{0.700000}%
\pgfsetlinewidth{0.000000pt}%
\definecolor{currentstroke}{rgb}{0.000000,0.000000,0.000000}%
\pgfsetstrokecolor{currentstroke}%
\pgfsetdash{}{0pt}%
\pgfpathmoveto{\pgfqpoint{2.689494in}{2.336922in}}%
\pgfpathlineto{\pgfqpoint{2.702744in}{2.321140in}}%
\pgfpathlineto{\pgfqpoint{2.715987in}{2.305614in}}%
\pgfpathlineto{\pgfqpoint{2.729222in}{2.290344in}}%
\pgfpathlineto{\pgfqpoint{2.742451in}{2.275327in}}%
\pgfpathlineto{\pgfqpoint{2.750662in}{2.279995in}}%
\pgfpathlineto{\pgfqpoint{2.758862in}{2.284813in}}%
\pgfpathlineto{\pgfqpoint{2.767051in}{2.289780in}}%
\pgfpathlineto{\pgfqpoint{2.775229in}{2.294891in}}%
\pgfpathlineto{\pgfqpoint{2.762030in}{2.309596in}}%
\pgfpathlineto{\pgfqpoint{2.748824in}{2.324553in}}%
\pgfpathlineto{\pgfqpoint{2.735612in}{2.339766in}}%
\pgfpathlineto{\pgfqpoint{2.722393in}{2.355234in}}%
\pgfpathlineto{\pgfqpoint{2.714185in}{2.350425in}}%
\pgfpathlineto{\pgfqpoint{2.705967in}{2.345768in}}%
\pgfpathlineto{\pgfqpoint{2.697736in}{2.341266in}}%
\pgfpathlineto{\pgfqpoint{2.689494in}{2.336922in}}%
\pgfpathclose%
\pgfusepath{fill}%
\end{pgfscope}%
\begin{pgfscope}%
\pgfpathrectangle{\pgfqpoint{1.254980in}{0.150000in}}{\pgfqpoint{5.490039in}{5.490039in}}%
\pgfusepath{clip}%
\pgfsetbuttcap%
\pgfsetroundjoin%
\definecolor{currentfill}{rgb}{0.258965,0.251537,0.524736}%
\pgfsetfillcolor{currentfill}%
\pgfsetfillopacity{0.700000}%
\pgfsetlinewidth{0.000000pt}%
\definecolor{currentstroke}{rgb}{0.000000,0.000000,0.000000}%
\pgfsetstrokecolor{currentstroke}%
\pgfsetdash{}{0pt}%
\pgfpathmoveto{\pgfqpoint{2.636420in}{2.402669in}}%
\pgfpathlineto{\pgfqpoint{2.649700in}{2.385835in}}%
\pgfpathlineto{\pgfqpoint{2.662973in}{2.369268in}}%
\pgfpathlineto{\pgfqpoint{2.676237in}{2.352964in}}%
\pgfpathlineto{\pgfqpoint{2.689494in}{2.336922in}}%
\pgfpathlineto{\pgfqpoint{2.697736in}{2.341266in}}%
\pgfpathlineto{\pgfqpoint{2.705967in}{2.345768in}}%
\pgfpathlineto{\pgfqpoint{2.714185in}{2.350425in}}%
\pgfpathlineto{\pgfqpoint{2.722393in}{2.355234in}}%
\pgfpathlineto{\pgfqpoint{2.709167in}{2.370962in}}%
\pgfpathlineto{\pgfqpoint{2.695934in}{2.386951in}}%
\pgfpathlineto{\pgfqpoint{2.682693in}{2.403203in}}%
\pgfpathlineto{\pgfqpoint{2.669444in}{2.419720in}}%
\pgfpathlineto{\pgfqpoint{2.661206in}{2.415215in}}%
\pgfpathlineto{\pgfqpoint{2.652956in}{2.410870in}}%
\pgfpathlineto{\pgfqpoint{2.644694in}{2.406687in}}%
\pgfpathlineto{\pgfqpoint{2.636420in}{2.402669in}}%
\pgfpathclose%
\pgfusepath{fill}%
\end{pgfscope}%
\begin{pgfscope}%
\pgfpathrectangle{\pgfqpoint{1.254980in}{0.150000in}}{\pgfqpoint{5.490039in}{5.490039in}}%
\pgfusepath{clip}%
\pgfsetbuttcap%
\pgfsetroundjoin%
\definecolor{currentfill}{rgb}{0.275191,0.194905,0.496005}%
\pgfsetfillcolor{currentfill}%
\pgfsetfillopacity{0.700000}%
\pgfsetlinewidth{0.000000pt}%
\definecolor{currentstroke}{rgb}{0.000000,0.000000,0.000000}%
\pgfsetstrokecolor{currentstroke}%
\pgfsetdash{}{0pt}%
\pgfpathmoveto{\pgfqpoint{2.742451in}{2.275327in}}%
\pgfpathlineto{\pgfqpoint{2.755674in}{2.260560in}}%
\pgfpathlineto{\pgfqpoint{2.768890in}{2.246042in}}%
\pgfpathlineto{\pgfqpoint{2.782101in}{2.231771in}}%
\pgfpathlineto{\pgfqpoint{2.795306in}{2.217745in}}%
\pgfpathlineto{\pgfqpoint{2.803487in}{2.222735in}}%
\pgfpathlineto{\pgfqpoint{2.811657in}{2.227869in}}%
\pgfpathlineto{\pgfqpoint{2.819818in}{2.233142in}}%
\pgfpathlineto{\pgfqpoint{2.827968in}{2.238553in}}%
\pgfpathlineto{\pgfqpoint{2.814791in}{2.252269in}}%
\pgfpathlineto{\pgfqpoint{2.801609in}{2.266229in}}%
\pgfpathlineto{\pgfqpoint{2.788422in}{2.280436in}}%
\pgfpathlineto{\pgfqpoint{2.775229in}{2.294891in}}%
\pgfpathlineto{\pgfqpoint{2.767051in}{2.289780in}}%
\pgfpathlineto{\pgfqpoint{2.758862in}{2.284813in}}%
\pgfpathlineto{\pgfqpoint{2.750662in}{2.279995in}}%
\pgfpathlineto{\pgfqpoint{2.742451in}{2.275327in}}%
\pgfpathclose%
\pgfusepath{fill}%
\end{pgfscope}%
\begin{pgfscope}%
\pgfpathrectangle{\pgfqpoint{1.254980in}{0.150000in}}{\pgfqpoint{5.490039in}{5.490039in}}%
\pgfusepath{clip}%
\pgfsetbuttcap%
\pgfsetroundjoin%
\definecolor{currentfill}{rgb}{0.277018,0.050344,0.375715}%
\pgfsetfillcolor{currentfill}%
\pgfsetfillopacity{0.700000}%
\pgfsetlinewidth{0.000000pt}%
\definecolor{currentstroke}{rgb}{0.000000,0.000000,0.000000}%
\pgfsetstrokecolor{currentstroke}%
\pgfsetdash{}{0pt}%
\pgfpathmoveto{\pgfqpoint{3.227483in}{1.977535in}}%
\pgfpathlineto{\pgfqpoint{3.240577in}{1.970471in}}%
\pgfpathlineto{\pgfqpoint{3.253673in}{1.963607in}}%
\pgfpathlineto{\pgfqpoint{3.266770in}{1.956942in}}%
\pgfpathlineto{\pgfqpoint{3.279868in}{1.950474in}}%
\pgfpathlineto{\pgfqpoint{3.287798in}{1.958604in}}%
\pgfpathlineto{\pgfqpoint{3.295722in}{1.966800in}}%
\pgfpathlineto{\pgfqpoint{3.303638in}{1.975061in}}%
\pgfpathlineto{\pgfqpoint{3.311548in}{1.983383in}}%
\pgfpathlineto{\pgfqpoint{3.298467in}{1.989605in}}%
\pgfpathlineto{\pgfqpoint{3.285387in}{1.996025in}}%
\pgfpathlineto{\pgfqpoint{3.272309in}{2.002644in}}%
\pgfpathlineto{\pgfqpoint{3.259232in}{2.009462in}}%
\pgfpathlineto{\pgfqpoint{3.251305in}{2.001375in}}%
\pgfpathlineto{\pgfqpoint{3.243371in}{1.993357in}}%
\pgfpathlineto{\pgfqpoint{3.235430in}{1.985409in}}%
\pgfpathlineto{\pgfqpoint{3.227483in}{1.977535in}}%
\pgfpathclose%
\pgfusepath{fill}%
\end{pgfscope}%
\begin{pgfscope}%
\pgfpathrectangle{\pgfqpoint{1.254980in}{0.150000in}}{\pgfqpoint{5.490039in}{5.490039in}}%
\pgfusepath{clip}%
\pgfsetbuttcap%
\pgfsetroundjoin%
\definecolor{currentfill}{rgb}{0.163625,0.471133,0.558148}%
\pgfsetfillcolor{currentfill}%
\pgfsetfillopacity{0.700000}%
\pgfsetlinewidth{0.000000pt}%
\definecolor{currentstroke}{rgb}{0.000000,0.000000,0.000000}%
\pgfsetstrokecolor{currentstroke}%
\pgfsetdash{}{0pt}%
\pgfpathmoveto{\pgfqpoint{5.116472in}{2.879142in}}%
\pgfpathlineto{\pgfqpoint{5.130167in}{2.886686in}}%
\pgfpathlineto{\pgfqpoint{5.143877in}{2.894387in}}%
\pgfpathlineto{\pgfqpoint{5.157602in}{2.902244in}}%
\pgfpathlineto{\pgfqpoint{5.171343in}{2.910257in}}%
\pgfpathlineto{\pgfqpoint{5.178538in}{2.915619in}}%
\pgfpathlineto{\pgfqpoint{5.185726in}{2.920978in}}%
\pgfpathlineto{\pgfqpoint{5.192909in}{2.926338in}}%
\pgfpathlineto{\pgfqpoint{5.200087in}{2.931702in}}%
\pgfpathlineto{\pgfqpoint{5.186364in}{2.924073in}}%
\pgfpathlineto{\pgfqpoint{5.172656in}{2.916601in}}%
\pgfpathlineto{\pgfqpoint{5.158964in}{2.909283in}}%
\pgfpathlineto{\pgfqpoint{5.145286in}{2.902121in}}%
\pgfpathlineto{\pgfqpoint{5.138091in}{2.896364in}}%
\pgfpathlineto{\pgfqpoint{5.130890in}{2.890617in}}%
\pgfpathlineto{\pgfqpoint{5.123684in}{2.884878in}}%
\pgfpathlineto{\pgfqpoint{5.116472in}{2.879142in}}%
\pgfpathclose%
\pgfusepath{fill}%
\end{pgfscope}%
\begin{pgfscope}%
\pgfpathrectangle{\pgfqpoint{1.254980in}{0.150000in}}{\pgfqpoint{5.490039in}{5.490039in}}%
\pgfusepath{clip}%
\pgfsetbuttcap%
\pgfsetroundjoin%
\definecolor{currentfill}{rgb}{0.276022,0.044167,0.370164}%
\pgfsetfillcolor{currentfill}%
\pgfsetfillopacity{0.700000}%
\pgfsetlinewidth{0.000000pt}%
\definecolor{currentstroke}{rgb}{0.000000,0.000000,0.000000}%
\pgfsetstrokecolor{currentstroke}%
\pgfsetdash{}{0pt}%
\pgfpathmoveto{\pgfqpoint{3.363893in}{1.960445in}}%
\pgfpathlineto{\pgfqpoint{3.376984in}{1.955194in}}%
\pgfpathlineto{\pgfqpoint{3.390079in}{1.950134in}}%
\pgfpathlineto{\pgfqpoint{3.403176in}{1.945265in}}%
\pgfpathlineto{\pgfqpoint{3.416276in}{1.940585in}}%
\pgfpathlineto{\pgfqpoint{3.424148in}{1.949422in}}%
\pgfpathlineto{\pgfqpoint{3.432015in}{1.958302in}}%
\pgfpathlineto{\pgfqpoint{3.439875in}{1.967225in}}%
\pgfpathlineto{\pgfqpoint{3.447729in}{1.976188in}}%
\pgfpathlineto{\pgfqpoint{3.434643in}{1.980651in}}%
\pgfpathlineto{\pgfqpoint{3.421561in}{1.985303in}}%
\pgfpathlineto{\pgfqpoint{3.408481in}{1.990146in}}%
\pgfpathlineto{\pgfqpoint{3.395404in}{1.995180in}}%
\pgfpathlineto{\pgfqpoint{3.387536in}{1.986424in}}%
\pgfpathlineto{\pgfqpoint{3.379661in}{1.977714in}}%
\pgfpathlineto{\pgfqpoint{3.371780in}{1.969054in}}%
\pgfpathlineto{\pgfqpoint{3.363893in}{1.960445in}}%
\pgfpathclose%
\pgfusepath{fill}%
\end{pgfscope}%
\begin{pgfscope}%
\pgfpathrectangle{\pgfqpoint{1.254980in}{0.150000in}}{\pgfqpoint{5.490039in}{5.490039in}}%
\pgfusepath{clip}%
\pgfsetbuttcap%
\pgfsetroundjoin%
\definecolor{currentfill}{rgb}{0.246811,0.283237,0.535941}%
\pgfsetfillcolor{currentfill}%
\pgfsetfillopacity{0.700000}%
\pgfsetlinewidth{0.000000pt}%
\definecolor{currentstroke}{rgb}{0.000000,0.000000,0.000000}%
\pgfsetstrokecolor{currentstroke}%
\pgfsetdash{}{0pt}%
\pgfpathmoveto{\pgfqpoint{4.418402in}{2.417333in}}%
\pgfpathlineto{\pgfqpoint{4.431767in}{2.421916in}}%
\pgfpathlineto{\pgfqpoint{4.445144in}{2.426662in}}%
\pgfpathlineto{\pgfqpoint{4.458532in}{2.431571in}}%
\pgfpathlineto{\pgfqpoint{4.471933in}{2.436643in}}%
\pgfpathlineto{\pgfqpoint{4.479439in}{2.445340in}}%
\pgfpathlineto{\pgfqpoint{4.486939in}{2.453983in}}%
\pgfpathlineto{\pgfqpoint{4.494434in}{2.462573in}}%
\pgfpathlineto{\pgfqpoint{4.501923in}{2.471112in}}%
\pgfpathlineto{\pgfqpoint{4.488531in}{2.466162in}}%
\pgfpathlineto{\pgfqpoint{4.475150in}{2.461374in}}%
\pgfpathlineto{\pgfqpoint{4.461781in}{2.456749in}}%
\pgfpathlineto{\pgfqpoint{4.448424in}{2.452287in}}%
\pgfpathlineto{\pgfqpoint{4.440927in}{2.443616in}}%
\pgfpathlineto{\pgfqpoint{4.433424in}{2.434901in}}%
\pgfpathlineto{\pgfqpoint{4.425916in}{2.426140in}}%
\pgfpathlineto{\pgfqpoint{4.418402in}{2.417333in}}%
\pgfpathclose%
\pgfusepath{fill}%
\end{pgfscope}%
\begin{pgfscope}%
\pgfpathrectangle{\pgfqpoint{1.254980in}{0.150000in}}{\pgfqpoint{5.490039in}{5.490039in}}%
\pgfusepath{clip}%
\pgfsetbuttcap%
\pgfsetroundjoin%
\definecolor{currentfill}{rgb}{0.246811,0.283237,0.535941}%
\pgfsetfillcolor{currentfill}%
\pgfsetfillopacity{0.700000}%
\pgfsetlinewidth{0.000000pt}%
\definecolor{currentstroke}{rgb}{0.000000,0.000000,0.000000}%
\pgfsetstrokecolor{currentstroke}%
\pgfsetdash{}{0pt}%
\pgfpathmoveto{\pgfqpoint{2.583209in}{2.472713in}}%
\pgfpathlineto{\pgfqpoint{2.596526in}{2.454790in}}%
\pgfpathlineto{\pgfqpoint{2.609832in}{2.437144in}}%
\pgfpathlineto{\pgfqpoint{2.623130in}{2.419771in}}%
\pgfpathlineto{\pgfqpoint{2.636420in}{2.402669in}}%
\pgfpathlineto{\pgfqpoint{2.644694in}{2.406687in}}%
\pgfpathlineto{\pgfqpoint{2.652956in}{2.410870in}}%
\pgfpathlineto{\pgfqpoint{2.661206in}{2.415215in}}%
\pgfpathlineto{\pgfqpoint{2.669444in}{2.419720in}}%
\pgfpathlineto{\pgfqpoint{2.656187in}{2.436506in}}%
\pgfpathlineto{\pgfqpoint{2.642922in}{2.453562in}}%
\pgfpathlineto{\pgfqpoint{2.629648in}{2.470890in}}%
\pgfpathlineto{\pgfqpoint{2.616366in}{2.488494in}}%
\pgfpathlineto{\pgfqpoint{2.608095in}{2.484295in}}%
\pgfpathlineto{\pgfqpoint{2.599812in}{2.480264in}}%
\pgfpathlineto{\pgfqpoint{2.591517in}{2.476402in}}%
\pgfpathlineto{\pgfqpoint{2.583209in}{2.472713in}}%
\pgfpathclose%
\pgfusepath{fill}%
\end{pgfscope}%
\begin{pgfscope}%
\pgfpathrectangle{\pgfqpoint{1.254980in}{0.150000in}}{\pgfqpoint{5.490039in}{5.490039in}}%
\pgfusepath{clip}%
\pgfsetbuttcap%
\pgfsetroundjoin%
\definecolor{currentfill}{rgb}{0.280255,0.165693,0.476498}%
\pgfsetfillcolor{currentfill}%
\pgfsetfillopacity{0.700000}%
\pgfsetlinewidth{0.000000pt}%
\definecolor{currentstroke}{rgb}{0.000000,0.000000,0.000000}%
\pgfsetstrokecolor{currentstroke}%
\pgfsetdash{}{0pt}%
\pgfpathmoveto{\pgfqpoint{2.795306in}{2.217745in}}%
\pgfpathlineto{\pgfqpoint{2.808505in}{2.203962in}}%
\pgfpathlineto{\pgfqpoint{2.821699in}{2.190420in}}%
\pgfpathlineto{\pgfqpoint{2.834888in}{2.177117in}}%
\pgfpathlineto{\pgfqpoint{2.848073in}{2.164051in}}%
\pgfpathlineto{\pgfqpoint{2.856226in}{2.169361in}}%
\pgfpathlineto{\pgfqpoint{2.864369in}{2.174807in}}%
\pgfpathlineto{\pgfqpoint{2.872501in}{2.180387in}}%
\pgfpathlineto{\pgfqpoint{2.880624in}{2.186096in}}%
\pgfpathlineto{\pgfqpoint{2.867467in}{2.198853in}}%
\pgfpathlineto{\pgfqpoint{2.854305in}{2.211847in}}%
\pgfpathlineto{\pgfqpoint{2.841139in}{2.225080in}}%
\pgfpathlineto{\pgfqpoint{2.827968in}{2.238553in}}%
\pgfpathlineto{\pgfqpoint{2.819818in}{2.233142in}}%
\pgfpathlineto{\pgfqpoint{2.811657in}{2.227869in}}%
\pgfpathlineto{\pgfqpoint{2.803487in}{2.222735in}}%
\pgfpathlineto{\pgfqpoint{2.795306in}{2.217745in}}%
\pgfpathclose%
\pgfusepath{fill}%
\end{pgfscope}%
\begin{pgfscope}%
\pgfpathrectangle{\pgfqpoint{1.254980in}{0.150000in}}{\pgfqpoint{5.490039in}{5.490039in}}%
\pgfusepath{clip}%
\pgfsetbuttcap%
\pgfsetroundjoin%
\definecolor{currentfill}{rgb}{0.278791,0.062145,0.386592}%
\pgfsetfillcolor{currentfill}%
\pgfsetfillopacity{0.700000}%
\pgfsetlinewidth{0.000000pt}%
\definecolor{currentstroke}{rgb}{0.000000,0.000000,0.000000}%
\pgfsetstrokecolor{currentstroke}%
\pgfsetdash{}{0pt}%
\pgfpathmoveto{\pgfqpoint{3.583804in}{1.984936in}}%
\pgfpathlineto{\pgfqpoint{3.596914in}{1.982330in}}%
\pgfpathlineto{\pgfqpoint{3.610028in}{1.979905in}}%
\pgfpathlineto{\pgfqpoint{3.623148in}{1.977661in}}%
\pgfpathlineto{\pgfqpoint{3.636273in}{1.975598in}}%
\pgfpathlineto{\pgfqpoint{3.644062in}{1.985251in}}%
\pgfpathlineto{\pgfqpoint{3.651846in}{1.994915in}}%
\pgfpathlineto{\pgfqpoint{3.659625in}{2.004588in}}%
\pgfpathlineto{\pgfqpoint{3.667398in}{2.014269in}}%
\pgfpathlineto{\pgfqpoint{3.654284in}{2.016172in}}%
\pgfpathlineto{\pgfqpoint{3.641176in}{2.018255in}}%
\pgfpathlineto{\pgfqpoint{3.628072in}{2.020519in}}%
\pgfpathlineto{\pgfqpoint{3.614974in}{2.022964in}}%
\pgfpathlineto{\pgfqpoint{3.607190in}{2.013434in}}%
\pgfpathlineto{\pgfqpoint{3.599400in}{2.003918in}}%
\pgfpathlineto{\pgfqpoint{3.591605in}{1.994418in}}%
\pgfpathlineto{\pgfqpoint{3.583804in}{1.984936in}}%
\pgfpathclose%
\pgfusepath{fill}%
\end{pgfscope}%
\begin{pgfscope}%
\pgfpathrectangle{\pgfqpoint{1.254980in}{0.150000in}}{\pgfqpoint{5.490039in}{5.490039in}}%
\pgfusepath{clip}%
\pgfsetbuttcap%
\pgfsetroundjoin%
\definecolor{currentfill}{rgb}{0.154815,0.493313,0.557840}%
\pgfsetfillcolor{currentfill}%
\pgfsetfillopacity{0.700000}%
\pgfsetlinewidth{0.000000pt}%
\definecolor{currentstroke}{rgb}{0.000000,0.000000,0.000000}%
\pgfsetstrokecolor{currentstroke}%
\pgfsetdash{}{0pt}%
\pgfpathmoveto{\pgfqpoint{5.200087in}{2.931702in}}%
\pgfpathlineto{\pgfqpoint{5.213825in}{2.939485in}}%
\pgfpathlineto{\pgfqpoint{5.227580in}{2.947424in}}%
\pgfpathlineto{\pgfqpoint{5.241350in}{2.955519in}}%
\pgfpathlineto{\pgfqpoint{5.255136in}{2.963768in}}%
\pgfpathlineto{\pgfqpoint{5.262289in}{2.968738in}}%
\pgfpathlineto{\pgfqpoint{5.269436in}{2.973715in}}%
\pgfpathlineto{\pgfqpoint{5.276577in}{2.978704in}}%
\pgfpathlineto{\pgfqpoint{5.283713in}{2.983708in}}%
\pgfpathlineto{\pgfqpoint{5.269947in}{2.975873in}}%
\pgfpathlineto{\pgfqpoint{5.256196in}{2.968192in}}%
\pgfpathlineto{\pgfqpoint{5.242461in}{2.960666in}}%
\pgfpathlineto{\pgfqpoint{5.228741in}{2.953295in}}%
\pgfpathlineto{\pgfqpoint{5.221585in}{2.947867in}}%
\pgfpathlineto{\pgfqpoint{5.214425in}{2.942462in}}%
\pgfpathlineto{\pgfqpoint{5.207258in}{2.937075in}}%
\pgfpathlineto{\pgfqpoint{5.200087in}{2.931702in}}%
\pgfpathclose%
\pgfusepath{fill}%
\end{pgfscope}%
\begin{pgfscope}%
\pgfpathrectangle{\pgfqpoint{1.254980in}{0.150000in}}{\pgfqpoint{5.490039in}{5.490039in}}%
\pgfusepath{clip}%
\pgfsetbuttcap%
\pgfsetroundjoin%
\definecolor{currentfill}{rgb}{0.279566,0.067836,0.391917}%
\pgfsetfillcolor{currentfill}%
\pgfsetfillopacity{0.700000}%
\pgfsetlinewidth{0.000000pt}%
\definecolor{currentstroke}{rgb}{0.000000,0.000000,0.000000}%
\pgfsetstrokecolor{currentstroke}%
\pgfsetdash{}{0pt}%
\pgfpathmoveto{\pgfqpoint{3.090713in}{2.012843in}}%
\pgfpathlineto{\pgfqpoint{3.103828in}{2.003866in}}%
\pgfpathlineto{\pgfqpoint{3.116941in}{1.995098in}}%
\pgfpathlineto{\pgfqpoint{3.130054in}{1.986538in}}%
\pgfpathlineto{\pgfqpoint{3.143167in}{1.978184in}}%
\pgfpathlineto{\pgfqpoint{3.151164in}{1.985458in}}%
\pgfpathlineto{\pgfqpoint{3.159154in}{1.992822in}}%
\pgfpathlineto{\pgfqpoint{3.167135in}{2.000273in}}%
\pgfpathlineto{\pgfqpoint{3.175110in}{2.007808in}}%
\pgfpathlineto{\pgfqpoint{3.162017in}{2.015887in}}%
\pgfpathlineto{\pgfqpoint{3.148924in}{2.024173in}}%
\pgfpathlineto{\pgfqpoint{3.135831in}{2.032666in}}%
\pgfpathlineto{\pgfqpoint{3.122738in}{2.041368in}}%
\pgfpathlineto{\pgfqpoint{3.114744in}{2.034097in}}%
\pgfpathlineto{\pgfqpoint{3.106742in}{2.026917in}}%
\pgfpathlineto{\pgfqpoint{3.098732in}{2.019832in}}%
\pgfpathlineto{\pgfqpoint{3.090713in}{2.012843in}}%
\pgfpathclose%
\pgfusepath{fill}%
\end{pgfscope}%
\begin{pgfscope}%
\pgfpathrectangle{\pgfqpoint{1.254980in}{0.150000in}}{\pgfqpoint{5.490039in}{5.490039in}}%
\pgfusepath{clip}%
\pgfsetbuttcap%
\pgfsetroundjoin%
\definecolor{currentfill}{rgb}{0.235526,0.309527,0.542944}%
\pgfsetfillcolor{currentfill}%
\pgfsetfillopacity{0.700000}%
\pgfsetlinewidth{0.000000pt}%
\definecolor{currentstroke}{rgb}{0.000000,0.000000,0.000000}%
\pgfsetstrokecolor{currentstroke}%
\pgfsetdash{}{0pt}%
\pgfpathmoveto{\pgfqpoint{4.501923in}{2.471112in}}%
\pgfpathlineto{\pgfqpoint{4.515328in}{2.476225in}}%
\pgfpathlineto{\pgfqpoint{4.528745in}{2.481499in}}%
\pgfpathlineto{\pgfqpoint{4.542174in}{2.486935in}}%
\pgfpathlineto{\pgfqpoint{4.555615in}{2.492533in}}%
\pgfpathlineto{\pgfqpoint{4.563090in}{2.500884in}}%
\pgfpathlineto{\pgfqpoint{4.570560in}{2.509180in}}%
\pgfpathlineto{\pgfqpoint{4.578024in}{2.517424in}}%
\pgfpathlineto{\pgfqpoint{4.585482in}{2.525618in}}%
\pgfpathlineto{\pgfqpoint{4.572049in}{2.520171in}}%
\pgfpathlineto{\pgfqpoint{4.558628in}{2.514885in}}%
\pgfpathlineto{\pgfqpoint{4.545220in}{2.509760in}}%
\pgfpathlineto{\pgfqpoint{4.531824in}{2.504797in}}%
\pgfpathlineto{\pgfqpoint{4.524357in}{2.496443in}}%
\pgfpathlineto{\pgfqpoint{4.516885in}{2.488045in}}%
\pgfpathlineto{\pgfqpoint{4.509407in}{2.479602in}}%
\pgfpathlineto{\pgfqpoint{4.501923in}{2.471112in}}%
\pgfpathclose%
\pgfusepath{fill}%
\end{pgfscope}%
\begin{pgfscope}%
\pgfpathrectangle{\pgfqpoint{1.254980in}{0.150000in}}{\pgfqpoint{5.490039in}{5.490039in}}%
\pgfusepath{clip}%
\pgfsetbuttcap%
\pgfsetroundjoin%
\definecolor{currentfill}{rgb}{0.282290,0.145912,0.461510}%
\pgfsetfillcolor{currentfill}%
\pgfsetfillopacity{0.700000}%
\pgfsetlinewidth{0.000000pt}%
\definecolor{currentstroke}{rgb}{0.000000,0.000000,0.000000}%
\pgfsetstrokecolor{currentstroke}%
\pgfsetdash{}{0pt}%
\pgfpathmoveto{\pgfqpoint{2.848073in}{2.164051in}}%
\pgfpathlineto{\pgfqpoint{2.861253in}{2.151221in}}%
\pgfpathlineto{\pgfqpoint{2.874429in}{2.138624in}}%
\pgfpathlineto{\pgfqpoint{2.887600in}{2.126260in}}%
\pgfpathlineto{\pgfqpoint{2.900768in}{2.114126in}}%
\pgfpathlineto{\pgfqpoint{2.908894in}{2.119755in}}%
\pgfpathlineto{\pgfqpoint{2.917010in}{2.125512in}}%
\pgfpathlineto{\pgfqpoint{2.925117in}{2.131395in}}%
\pgfpathlineto{\pgfqpoint{2.933214in}{2.137402in}}%
\pgfpathlineto{\pgfqpoint{2.920072in}{2.149229in}}%
\pgfpathlineto{\pgfqpoint{2.906927in}{2.161286in}}%
\pgfpathlineto{\pgfqpoint{2.893777in}{2.173574in}}%
\pgfpathlineto{\pgfqpoint{2.880624in}{2.186096in}}%
\pgfpathlineto{\pgfqpoint{2.872501in}{2.180387in}}%
\pgfpathlineto{\pgfqpoint{2.864369in}{2.174807in}}%
\pgfpathlineto{\pgfqpoint{2.856226in}{2.169361in}}%
\pgfpathlineto{\pgfqpoint{2.848073in}{2.164051in}}%
\pgfpathclose%
\pgfusepath{fill}%
\end{pgfscope}%
\begin{pgfscope}%
\pgfpathrectangle{\pgfqpoint{1.254980in}{0.150000in}}{\pgfqpoint{5.490039in}{5.490039in}}%
\pgfusepath{clip}%
\pgfsetbuttcap%
\pgfsetroundjoin%
\definecolor{currentfill}{rgb}{0.231674,0.318106,0.544834}%
\pgfsetfillcolor{currentfill}%
\pgfsetfillopacity{0.700000}%
\pgfsetlinewidth{0.000000pt}%
\definecolor{currentstroke}{rgb}{0.000000,0.000000,0.000000}%
\pgfsetstrokecolor{currentstroke}%
\pgfsetdash{}{0pt}%
\pgfpathmoveto{\pgfqpoint{2.529846in}{2.547212in}}%
\pgfpathlineto{\pgfqpoint{2.543202in}{2.528160in}}%
\pgfpathlineto{\pgfqpoint{2.556548in}{2.509395in}}%
\pgfpathlineto{\pgfqpoint{2.569884in}{2.490913in}}%
\pgfpathlineto{\pgfqpoint{2.583209in}{2.472713in}}%
\pgfpathlineto{\pgfqpoint{2.591517in}{2.476402in}}%
\pgfpathlineto{\pgfqpoint{2.599812in}{2.480264in}}%
\pgfpathlineto{\pgfqpoint{2.608095in}{2.484295in}}%
\pgfpathlineto{\pgfqpoint{2.616366in}{2.488494in}}%
\pgfpathlineto{\pgfqpoint{2.603074in}{2.506376in}}%
\pgfpathlineto{\pgfqpoint{2.589772in}{2.524538in}}%
\pgfpathlineto{\pgfqpoint{2.576461in}{2.542982in}}%
\pgfpathlineto{\pgfqpoint{2.563140in}{2.561713in}}%
\pgfpathlineto{\pgfqpoint{2.554836in}{2.557823in}}%
\pgfpathlineto{\pgfqpoint{2.546519in}{2.554108in}}%
\pgfpathlineto{\pgfqpoint{2.538189in}{2.550570in}}%
\pgfpathlineto{\pgfqpoint{2.529846in}{2.547212in}}%
\pgfpathclose%
\pgfusepath{fill}%
\end{pgfscope}%
\begin{pgfscope}%
\pgfpathrectangle{\pgfqpoint{1.254980in}{0.150000in}}{\pgfqpoint{5.490039in}{5.490039in}}%
\pgfusepath{clip}%
\pgfsetbuttcap%
\pgfsetroundjoin%
\definecolor{currentfill}{rgb}{0.147607,0.511733,0.557049}%
\pgfsetfillcolor{currentfill}%
\pgfsetfillopacity{0.700000}%
\pgfsetlinewidth{0.000000pt}%
\definecolor{currentstroke}{rgb}{0.000000,0.000000,0.000000}%
\pgfsetstrokecolor{currentstroke}%
\pgfsetdash{}{0pt}%
\pgfpathmoveto{\pgfqpoint{5.283713in}{2.983708in}}%
\pgfpathlineto{\pgfqpoint{5.297496in}{2.991699in}}%
\pgfpathlineto{\pgfqpoint{5.311294in}{2.999844in}}%
\pgfpathlineto{\pgfqpoint{5.325109in}{3.008144in}}%
\pgfpathlineto{\pgfqpoint{5.338940in}{3.016599in}}%
\pgfpathlineto{\pgfqpoint{5.346050in}{3.021192in}}%
\pgfpathlineto{\pgfqpoint{5.353155in}{3.025804in}}%
\pgfpathlineto{\pgfqpoint{5.360254in}{3.030440in}}%
\pgfpathlineto{\pgfqpoint{5.367348in}{3.035105in}}%
\pgfpathlineto{\pgfqpoint{5.353538in}{3.027094in}}%
\pgfpathlineto{\pgfqpoint{5.339745in}{3.019237in}}%
\pgfpathlineto{\pgfqpoint{5.325967in}{3.011534in}}%
\pgfpathlineto{\pgfqpoint{5.312205in}{3.003986in}}%
\pgfpathlineto{\pgfqpoint{5.305090in}{2.998868in}}%
\pgfpathlineto{\pgfqpoint{5.297969in}{2.993786in}}%
\pgfpathlineto{\pgfqpoint{5.290844in}{2.988734in}}%
\pgfpathlineto{\pgfqpoint{5.283713in}{2.983708in}}%
\pgfpathclose%
\pgfusepath{fill}%
\end{pgfscope}%
\begin{pgfscope}%
\pgfpathrectangle{\pgfqpoint{1.254980in}{0.150000in}}{\pgfqpoint{5.490039in}{5.490039in}}%
\pgfusepath{clip}%
\pgfsetbuttcap%
\pgfsetroundjoin%
\definecolor{currentfill}{rgb}{0.139147,0.533812,0.555298}%
\pgfsetfillcolor{currentfill}%
\pgfsetfillopacity{0.700000}%
\pgfsetlinewidth{0.000000pt}%
\definecolor{currentstroke}{rgb}{0.000000,0.000000,0.000000}%
\pgfsetstrokecolor{currentstroke}%
\pgfsetdash{}{0pt}%
\pgfpathmoveto{\pgfqpoint{5.367348in}{3.035105in}}%
\pgfpathlineto{\pgfqpoint{5.381174in}{3.043270in}}%
\pgfpathlineto{\pgfqpoint{5.395016in}{3.051589in}}%
\pgfpathlineto{\pgfqpoint{5.408875in}{3.060062in}}%
\pgfpathlineto{\pgfqpoint{5.422750in}{3.068690in}}%
\pgfpathlineto{\pgfqpoint{5.429817in}{3.072927in}}%
\pgfpathlineto{\pgfqpoint{5.436878in}{3.077196in}}%
\pgfpathlineto{\pgfqpoint{5.443935in}{3.081503in}}%
\pgfpathlineto{\pgfqpoint{5.450986in}{3.085852in}}%
\pgfpathlineto{\pgfqpoint{5.437134in}{3.077698in}}%
\pgfpathlineto{\pgfqpoint{5.423298in}{3.069698in}}%
\pgfpathlineto{\pgfqpoint{5.409479in}{3.061850in}}%
\pgfpathlineto{\pgfqpoint{5.395675in}{3.054157in}}%
\pgfpathlineto{\pgfqpoint{5.388600in}{3.049325in}}%
\pgfpathlineto{\pgfqpoint{5.381521in}{3.044542in}}%
\pgfpathlineto{\pgfqpoint{5.374437in}{3.039804in}}%
\pgfpathlineto{\pgfqpoint{5.367348in}{3.035105in}}%
\pgfpathclose%
\pgfusepath{fill}%
\end{pgfscope}%
\begin{pgfscope}%
\pgfpathrectangle{\pgfqpoint{1.254980in}{0.150000in}}{\pgfqpoint{5.490039in}{5.490039in}}%
\pgfusepath{clip}%
\pgfsetbuttcap%
\pgfsetroundjoin%
\definecolor{currentfill}{rgb}{0.223925,0.334994,0.548053}%
\pgfsetfillcolor{currentfill}%
\pgfsetfillopacity{0.700000}%
\pgfsetlinewidth{0.000000pt}%
\definecolor{currentstroke}{rgb}{0.000000,0.000000,0.000000}%
\pgfsetstrokecolor{currentstroke}%
\pgfsetdash{}{0pt}%
\pgfpathmoveto{\pgfqpoint{4.585482in}{2.525618in}}%
\pgfpathlineto{\pgfqpoint{4.598927in}{2.531227in}}%
\pgfpathlineto{\pgfqpoint{4.612385in}{2.536996in}}%
\pgfpathlineto{\pgfqpoint{4.625856in}{2.542927in}}%
\pgfpathlineto{\pgfqpoint{4.639340in}{2.549018in}}%
\pgfpathlineto{\pgfqpoint{4.646783in}{2.556995in}}%
\pgfpathlineto{\pgfqpoint{4.654221in}{2.564919in}}%
\pgfpathlineto{\pgfqpoint{4.661652in}{2.572792in}}%
\pgfpathlineto{\pgfqpoint{4.669078in}{2.580616in}}%
\pgfpathlineto{\pgfqpoint{4.655603in}{2.574704in}}%
\pgfpathlineto{\pgfqpoint{4.642141in}{2.568953in}}%
\pgfpathlineto{\pgfqpoint{4.628692in}{2.563362in}}%
\pgfpathlineto{\pgfqpoint{4.615256in}{2.557933in}}%
\pgfpathlineto{\pgfqpoint{4.607821in}{2.549919in}}%
\pgfpathlineto{\pgfqpoint{4.600380in}{2.541863in}}%
\pgfpathlineto{\pgfqpoint{4.592934in}{2.533763in}}%
\pgfpathlineto{\pgfqpoint{4.585482in}{2.525618in}}%
\pgfpathclose%
\pgfusepath{fill}%
\end{pgfscope}%
\begin{pgfscope}%
\pgfpathrectangle{\pgfqpoint{1.254980in}{0.150000in}}{\pgfqpoint{5.490039in}{5.490039in}}%
\pgfusepath{clip}%
\pgfsetbuttcap%
\pgfsetroundjoin%
\definecolor{currentfill}{rgb}{0.277018,0.050344,0.375715}%
\pgfsetfillcolor{currentfill}%
\pgfsetfillopacity{0.700000}%
\pgfsetlinewidth{0.000000pt}%
\definecolor{currentstroke}{rgb}{0.000000,0.000000,0.000000}%
\pgfsetstrokecolor{currentstroke}%
\pgfsetdash{}{0pt}%
\pgfpathmoveto{\pgfqpoint{3.500107in}{1.960214in}}%
\pgfpathlineto{\pgfqpoint{3.513211in}{1.956686in}}%
\pgfpathlineto{\pgfqpoint{3.526318in}{1.953342in}}%
\pgfpathlineto{\pgfqpoint{3.539431in}{1.950183in}}%
\pgfpathlineto{\pgfqpoint{3.552547in}{1.947206in}}%
\pgfpathlineto{\pgfqpoint{3.560370in}{1.956605in}}%
\pgfpathlineto{\pgfqpoint{3.568187in}{1.966028in}}%
\pgfpathlineto{\pgfqpoint{3.575999in}{1.975472in}}%
\pgfpathlineto{\pgfqpoint{3.583804in}{1.984936in}}%
\pgfpathlineto{\pgfqpoint{3.570700in}{1.987724in}}%
\pgfpathlineto{\pgfqpoint{3.557600in}{1.990696in}}%
\pgfpathlineto{\pgfqpoint{3.544504in}{1.993851in}}%
\pgfpathlineto{\pgfqpoint{3.531413in}{1.997190in}}%
\pgfpathlineto{\pgfqpoint{3.523595in}{1.987904in}}%
\pgfpathlineto{\pgfqpoint{3.515771in}{1.978645in}}%
\pgfpathlineto{\pgfqpoint{3.507942in}{1.969415in}}%
\pgfpathlineto{\pgfqpoint{3.500107in}{1.960214in}}%
\pgfpathclose%
\pgfusepath{fill}%
\end{pgfscope}%
\begin{pgfscope}%
\pgfpathrectangle{\pgfqpoint{1.254980in}{0.150000in}}{\pgfqpoint{5.490039in}{5.490039in}}%
\pgfusepath{clip}%
\pgfsetbuttcap%
\pgfsetroundjoin%
\definecolor{currentfill}{rgb}{0.283229,0.120777,0.440584}%
\pgfsetfillcolor{currentfill}%
\pgfsetfillopacity{0.700000}%
\pgfsetlinewidth{0.000000pt}%
\definecolor{currentstroke}{rgb}{0.000000,0.000000,0.000000}%
\pgfsetstrokecolor{currentstroke}%
\pgfsetdash{}{0pt}%
\pgfpathmoveto{\pgfqpoint{2.900768in}{2.114126in}}%
\pgfpathlineto{\pgfqpoint{2.913932in}{2.102221in}}%
\pgfpathlineto{\pgfqpoint{2.927093in}{2.090542in}}%
\pgfpathlineto{\pgfqpoint{2.940250in}{2.079090in}}%
\pgfpathlineto{\pgfqpoint{2.953405in}{2.067861in}}%
\pgfpathlineto{\pgfqpoint{2.961505in}{2.073806in}}%
\pgfpathlineto{\pgfqpoint{2.969595in}{2.079873in}}%
\pgfpathlineto{\pgfqpoint{2.977677in}{2.086059in}}%
\pgfpathlineto{\pgfqpoint{2.985749in}{2.092361in}}%
\pgfpathlineto{\pgfqpoint{2.972620in}{2.103285in}}%
\pgfpathlineto{\pgfqpoint{2.959487in}{2.114431in}}%
\pgfpathlineto{\pgfqpoint{2.946352in}{2.125803in}}%
\pgfpathlineto{\pgfqpoint{2.933214in}{2.137402in}}%
\pgfpathlineto{\pgfqpoint{2.925117in}{2.131395in}}%
\pgfpathlineto{\pgfqpoint{2.917010in}{2.125512in}}%
\pgfpathlineto{\pgfqpoint{2.908894in}{2.119755in}}%
\pgfpathlineto{\pgfqpoint{2.900768in}{2.114126in}}%
\pgfpathclose%
\pgfusepath{fill}%
\end{pgfscope}%
\begin{pgfscope}%
\pgfpathrectangle{\pgfqpoint{1.254980in}{0.150000in}}{\pgfqpoint{5.490039in}{5.490039in}}%
\pgfusepath{clip}%
\pgfsetbuttcap%
\pgfsetroundjoin%
\definecolor{currentfill}{rgb}{0.132444,0.552216,0.553018}%
\pgfsetfillcolor{currentfill}%
\pgfsetfillopacity{0.700000}%
\pgfsetlinewidth{0.000000pt}%
\definecolor{currentstroke}{rgb}{0.000000,0.000000,0.000000}%
\pgfsetstrokecolor{currentstroke}%
\pgfsetdash{}{0pt}%
\pgfpathmoveto{\pgfqpoint{5.450986in}{3.085852in}}%
\pgfpathlineto{\pgfqpoint{5.464855in}{3.094160in}}%
\pgfpathlineto{\pgfqpoint{5.478741in}{3.102621in}}%
\pgfpathlineto{\pgfqpoint{5.492643in}{3.111236in}}%
\pgfpathlineto{\pgfqpoint{5.506563in}{3.120005in}}%
\pgfpathlineto{\pgfqpoint{5.513585in}{3.123911in}}%
\pgfpathlineto{\pgfqpoint{5.520603in}{3.127865in}}%
\pgfpathlineto{\pgfqpoint{5.527616in}{3.131870in}}%
\pgfpathlineto{\pgfqpoint{5.534625in}{3.135935in}}%
\pgfpathlineto{\pgfqpoint{5.520731in}{3.127669in}}%
\pgfpathlineto{\pgfqpoint{5.506854in}{3.119556in}}%
\pgfpathlineto{\pgfqpoint{5.492993in}{3.111597in}}%
\pgfpathlineto{\pgfqpoint{5.479148in}{3.103790in}}%
\pgfpathlineto{\pgfqpoint{5.472114in}{3.099214in}}%
\pgfpathlineto{\pgfqpoint{5.465076in}{3.094703in}}%
\pgfpathlineto{\pgfqpoint{5.458033in}{3.090251in}}%
\pgfpathlineto{\pgfqpoint{5.450986in}{3.085852in}}%
\pgfpathclose%
\pgfusepath{fill}%
\end{pgfscope}%
\begin{pgfscope}%
\pgfpathrectangle{\pgfqpoint{1.254980in}{0.150000in}}{\pgfqpoint{5.490039in}{5.490039in}}%
\pgfusepath{clip}%
\pgfsetbuttcap%
\pgfsetroundjoin%
\definecolor{currentfill}{rgb}{0.216210,0.351535,0.550627}%
\pgfsetfillcolor{currentfill}%
\pgfsetfillopacity{0.700000}%
\pgfsetlinewidth{0.000000pt}%
\definecolor{currentstroke}{rgb}{0.000000,0.000000,0.000000}%
\pgfsetstrokecolor{currentstroke}%
\pgfsetdash{}{0pt}%
\pgfpathmoveto{\pgfqpoint{2.476312in}{2.626336in}}%
\pgfpathlineto{\pgfqpoint{2.489713in}{2.606112in}}%
\pgfpathlineto{\pgfqpoint{2.503102in}{2.586185in}}%
\pgfpathlineto{\pgfqpoint{2.516480in}{2.566552in}}%
\pgfpathlineto{\pgfqpoint{2.529846in}{2.547212in}}%
\pgfpathlineto{\pgfqpoint{2.538189in}{2.550570in}}%
\pgfpathlineto{\pgfqpoint{2.546519in}{2.554108in}}%
\pgfpathlineto{\pgfqpoint{2.554836in}{2.557823in}}%
\pgfpathlineto{\pgfqpoint{2.563140in}{2.561713in}}%
\pgfpathlineto{\pgfqpoint{2.549808in}{2.580732in}}%
\pgfpathlineto{\pgfqpoint{2.536466in}{2.600041in}}%
\pgfpathlineto{\pgfqpoint{2.523113in}{2.619645in}}%
\pgfpathlineto{\pgfqpoint{2.509748in}{2.639546in}}%
\pgfpathlineto{\pgfqpoint{2.501409in}{2.635967in}}%
\pgfpathlineto{\pgfqpoint{2.493057in}{2.632571in}}%
\pgfpathlineto{\pgfqpoint{2.484691in}{2.629360in}}%
\pgfpathlineto{\pgfqpoint{2.476312in}{2.626336in}}%
\pgfpathclose%
\pgfusepath{fill}%
\end{pgfscope}%
\begin{pgfscope}%
\pgfpathrectangle{\pgfqpoint{1.254980in}{0.150000in}}{\pgfqpoint{5.490039in}{5.490039in}}%
\pgfusepath{clip}%
\pgfsetbuttcap%
\pgfsetroundjoin%
\definecolor{currentfill}{rgb}{0.126453,0.570633,0.549841}%
\pgfsetfillcolor{currentfill}%
\pgfsetfillopacity{0.700000}%
\pgfsetlinewidth{0.000000pt}%
\definecolor{currentstroke}{rgb}{0.000000,0.000000,0.000000}%
\pgfsetstrokecolor{currentstroke}%
\pgfsetdash{}{0pt}%
\pgfpathmoveto{\pgfqpoint{5.534625in}{3.135935in}}%
\pgfpathlineto{\pgfqpoint{5.548537in}{3.144353in}}%
\pgfpathlineto{\pgfqpoint{5.562465in}{3.152925in}}%
\pgfpathlineto{\pgfqpoint{5.576410in}{3.161649in}}%
\pgfpathlineto{\pgfqpoint{5.590372in}{3.170527in}}%
\pgfpathlineto{\pgfqpoint{5.597351in}{3.174133in}}%
\pgfpathlineto{\pgfqpoint{5.604325in}{3.177803in}}%
\pgfpathlineto{\pgfqpoint{5.611295in}{3.181541in}}%
\pgfpathlineto{\pgfqpoint{5.618262in}{3.185354in}}%
\pgfpathlineto{\pgfqpoint{5.604326in}{3.177009in}}%
\pgfpathlineto{\pgfqpoint{5.590408in}{3.168816in}}%
\pgfpathlineto{\pgfqpoint{5.576506in}{3.160776in}}%
\pgfpathlineto{\pgfqpoint{5.562622in}{3.152887in}}%
\pgfpathlineto{\pgfqpoint{5.555628in}{3.148533in}}%
\pgfpathlineto{\pgfqpoint{5.548631in}{3.144260in}}%
\pgfpathlineto{\pgfqpoint{5.541630in}{3.140063in}}%
\pgfpathlineto{\pgfqpoint{5.534625in}{3.135935in}}%
\pgfpathclose%
\pgfusepath{fill}%
\end{pgfscope}%
\begin{pgfscope}%
\pgfpathrectangle{\pgfqpoint{1.254980in}{0.150000in}}{\pgfqpoint{5.490039in}{5.490039in}}%
\pgfusepath{clip}%
\pgfsetbuttcap%
\pgfsetroundjoin%
\definecolor{currentfill}{rgb}{0.212395,0.359683,0.551710}%
\pgfsetfillcolor{currentfill}%
\pgfsetfillopacity{0.700000}%
\pgfsetlinewidth{0.000000pt}%
\definecolor{currentstroke}{rgb}{0.000000,0.000000,0.000000}%
\pgfsetstrokecolor{currentstroke}%
\pgfsetdash{}{0pt}%
\pgfpathmoveto{\pgfqpoint{4.669078in}{2.580616in}}%
\pgfpathlineto{\pgfqpoint{4.682565in}{2.586688in}}%
\pgfpathlineto{\pgfqpoint{4.696066in}{2.592920in}}%
\pgfpathlineto{\pgfqpoint{4.709580in}{2.599312in}}%
\pgfpathlineto{\pgfqpoint{4.723108in}{2.605865in}}%
\pgfpathlineto{\pgfqpoint{4.730518in}{2.613445in}}%
\pgfpathlineto{\pgfqpoint{4.737921in}{2.620974in}}%
\pgfpathlineto{\pgfqpoint{4.745319in}{2.628455in}}%
\pgfpathlineto{\pgfqpoint{4.752710in}{2.635891in}}%
\pgfpathlineto{\pgfqpoint{4.739193in}{2.629548in}}%
\pgfpathlineto{\pgfqpoint{4.725689in}{2.623364in}}%
\pgfpathlineto{\pgfqpoint{4.712198in}{2.617341in}}%
\pgfpathlineto{\pgfqpoint{4.698720in}{2.611477in}}%
\pgfpathlineto{\pgfqpoint{4.691318in}{2.603822in}}%
\pgfpathlineto{\pgfqpoint{4.683911in}{2.596129in}}%
\pgfpathlineto{\pgfqpoint{4.676497in}{2.588394in}}%
\pgfpathlineto{\pgfqpoint{4.669078in}{2.580616in}}%
\pgfpathclose%
\pgfusepath{fill}%
\end{pgfscope}%
\begin{pgfscope}%
\pgfpathrectangle{\pgfqpoint{1.254980in}{0.150000in}}{\pgfqpoint{5.490039in}{5.490039in}}%
\pgfusepath{clip}%
\pgfsetbuttcap%
\pgfsetroundjoin%
\definecolor{currentfill}{rgb}{0.274952,0.037752,0.364543}%
\pgfsetfillcolor{currentfill}%
\pgfsetfillopacity{0.700000}%
\pgfsetlinewidth{0.000000pt}%
\definecolor{currentstroke}{rgb}{0.000000,0.000000,0.000000}%
\pgfsetstrokecolor{currentstroke}%
\pgfsetdash{}{0pt}%
\pgfpathmoveto{\pgfqpoint{3.279868in}{1.950474in}}%
\pgfpathlineto{\pgfqpoint{3.292968in}{1.944203in}}%
\pgfpathlineto{\pgfqpoint{3.306069in}{1.938127in}}%
\pgfpathlineto{\pgfqpoint{3.319173in}{1.932245in}}%
\pgfpathlineto{\pgfqpoint{3.332278in}{1.926557in}}%
\pgfpathlineto{\pgfqpoint{3.340192in}{1.934942in}}%
\pgfpathlineto{\pgfqpoint{3.348098in}{1.943387in}}%
\pgfpathlineto{\pgfqpoint{3.355999in}{1.951888in}}%
\pgfpathlineto{\pgfqpoint{3.363893in}{1.960445in}}%
\pgfpathlineto{\pgfqpoint{3.350803in}{1.965888in}}%
\pgfpathlineto{\pgfqpoint{3.337716in}{1.971525in}}%
\pgfpathlineto{\pgfqpoint{3.324631in}{1.977356in}}%
\pgfpathlineto{\pgfqpoint{3.311548in}{1.983383in}}%
\pgfpathlineto{\pgfqpoint{3.303638in}{1.975061in}}%
\pgfpathlineto{\pgfqpoint{3.295722in}{1.966800in}}%
\pgfpathlineto{\pgfqpoint{3.287798in}{1.958604in}}%
\pgfpathlineto{\pgfqpoint{3.279868in}{1.950474in}}%
\pgfpathclose%
\pgfusepath{fill}%
\end{pgfscope}%
\begin{pgfscope}%
\pgfpathrectangle{\pgfqpoint{1.254980in}{0.150000in}}{\pgfqpoint{5.490039in}{5.490039in}}%
\pgfusepath{clip}%
\pgfsetbuttcap%
\pgfsetroundjoin%
\definecolor{currentfill}{rgb}{0.121148,0.592739,0.544641}%
\pgfsetfillcolor{currentfill}%
\pgfsetfillopacity{0.700000}%
\pgfsetlinewidth{0.000000pt}%
\definecolor{currentstroke}{rgb}{0.000000,0.000000,0.000000}%
\pgfsetstrokecolor{currentstroke}%
\pgfsetdash{}{0pt}%
\pgfpathmoveto{\pgfqpoint{5.618262in}{3.185354in}}%
\pgfpathlineto{\pgfqpoint{5.632214in}{3.193852in}}%
\pgfpathlineto{\pgfqpoint{5.646184in}{3.202502in}}%
\pgfpathlineto{\pgfqpoint{5.660171in}{3.211304in}}%
\pgfpathlineto{\pgfqpoint{5.674176in}{3.220259in}}%
\pgfpathlineto{\pgfqpoint{5.681110in}{3.223602in}}%
\pgfpathlineto{\pgfqpoint{5.688041in}{3.227025in}}%
\pgfpathlineto{\pgfqpoint{5.694968in}{3.230534in}}%
\pgfpathlineto{\pgfqpoint{5.701892in}{3.234135in}}%
\pgfpathlineto{\pgfqpoint{5.687917in}{3.225742in}}%
\pgfpathlineto{\pgfqpoint{5.673959in}{3.217501in}}%
\pgfpathlineto{\pgfqpoint{5.660017in}{3.209411in}}%
\pgfpathlineto{\pgfqpoint{5.646093in}{3.201473in}}%
\pgfpathlineto{\pgfqpoint{5.639140in}{3.197301in}}%
\pgfpathlineto{\pgfqpoint{5.632184in}{3.193228in}}%
\pgfpathlineto{\pgfqpoint{5.625224in}{3.189248in}}%
\pgfpathlineto{\pgfqpoint{5.618262in}{3.185354in}}%
\pgfpathclose%
\pgfusepath{fill}%
\end{pgfscope}%
\begin{pgfscope}%
\pgfpathrectangle{\pgfqpoint{1.254980in}{0.150000in}}{\pgfqpoint{5.490039in}{5.490039in}}%
\pgfusepath{clip}%
\pgfsetbuttcap%
\pgfsetroundjoin%
\definecolor{currentfill}{rgb}{0.119423,0.611141,0.538982}%
\pgfsetfillcolor{currentfill}%
\pgfsetfillopacity{0.700000}%
\pgfsetlinewidth{0.000000pt}%
\definecolor{currentstroke}{rgb}{0.000000,0.000000,0.000000}%
\pgfsetstrokecolor{currentstroke}%
\pgfsetdash{}{0pt}%
\pgfpathmoveto{\pgfqpoint{5.701892in}{3.234135in}}%
\pgfpathlineto{\pgfqpoint{5.715885in}{3.242680in}}%
\pgfpathlineto{\pgfqpoint{5.729896in}{3.251377in}}%
\pgfpathlineto{\pgfqpoint{5.743924in}{3.260225in}}%
\pgfpathlineto{\pgfqpoint{5.757970in}{3.269226in}}%
\pgfpathlineto{\pgfqpoint{5.764860in}{3.272346in}}%
\pgfpathlineto{\pgfqpoint{5.771748in}{3.275564in}}%
\pgfpathlineto{\pgfqpoint{5.778634in}{3.278887in}}%
\pgfpathlineto{\pgfqpoint{5.785516in}{3.282322in}}%
\pgfpathlineto{\pgfqpoint{5.771502in}{3.273912in}}%
\pgfpathlineto{\pgfqpoint{5.757505in}{3.265654in}}%
\pgfpathlineto{\pgfqpoint{5.743525in}{3.257547in}}%
\pgfpathlineto{\pgfqpoint{5.729563in}{3.249590in}}%
\pgfpathlineto{\pgfqpoint{5.722649in}{3.245556in}}%
\pgfpathlineto{\pgfqpoint{5.715732in}{3.241640in}}%
\pgfpathlineto{\pgfqpoint{5.708814in}{3.237835in}}%
\pgfpathlineto{\pgfqpoint{5.701892in}{3.234135in}}%
\pgfpathclose%
\pgfusepath{fill}%
\end{pgfscope}%
\begin{pgfscope}%
\pgfpathrectangle{\pgfqpoint{1.254980in}{0.150000in}}{\pgfqpoint{5.490039in}{5.490039in}}%
\pgfusepath{clip}%
\pgfsetbuttcap%
\pgfsetroundjoin%
\definecolor{currentfill}{rgb}{0.277941,0.056324,0.381191}%
\pgfsetfillcolor{currentfill}%
\pgfsetfillopacity{0.700000}%
\pgfsetlinewidth{0.000000pt}%
\definecolor{currentstroke}{rgb}{0.000000,0.000000,0.000000}%
\pgfsetstrokecolor{currentstroke}%
\pgfsetdash{}{0pt}%
\pgfpathmoveto{\pgfqpoint{3.143167in}{1.978184in}}%
\pgfpathlineto{\pgfqpoint{3.156279in}{1.970037in}}%
\pgfpathlineto{\pgfqpoint{3.169392in}{1.962094in}}%
\pgfpathlineto{\pgfqpoint{3.182504in}{1.954354in}}%
\pgfpathlineto{\pgfqpoint{3.195617in}{1.946816in}}%
\pgfpathlineto{\pgfqpoint{3.203595in}{1.954374in}}%
\pgfpathlineto{\pgfqpoint{3.211565in}{1.962015in}}%
\pgfpathlineto{\pgfqpoint{3.219527in}{1.969736in}}%
\pgfpathlineto{\pgfqpoint{3.227483in}{1.977535in}}%
\pgfpathlineto{\pgfqpoint{3.214389in}{1.984799in}}%
\pgfpathlineto{\pgfqpoint{3.201295in}{1.992265in}}%
\pgfpathlineto{\pgfqpoint{3.188202in}{1.999935in}}%
\pgfpathlineto{\pgfqpoint{3.175110in}{2.007808in}}%
\pgfpathlineto{\pgfqpoint{3.167135in}{2.000273in}}%
\pgfpathlineto{\pgfqpoint{3.159154in}{1.992822in}}%
\pgfpathlineto{\pgfqpoint{3.151164in}{1.985458in}}%
\pgfpathlineto{\pgfqpoint{3.143167in}{1.978184in}}%
\pgfpathclose%
\pgfusepath{fill}%
\end{pgfscope}%
\begin{pgfscope}%
\pgfpathrectangle{\pgfqpoint{1.254980in}{0.150000in}}{\pgfqpoint{5.490039in}{5.490039in}}%
\pgfusepath{clip}%
\pgfsetbuttcap%
\pgfsetroundjoin%
\definecolor{currentfill}{rgb}{0.283187,0.125848,0.444960}%
\pgfsetfillcolor{currentfill}%
\pgfsetfillopacity{0.700000}%
\pgfsetlinewidth{0.000000pt}%
\definecolor{currentstroke}{rgb}{0.000000,0.000000,0.000000}%
\pgfsetstrokecolor{currentstroke}%
\pgfsetdash{}{0pt}%
\pgfpathmoveto{\pgfqpoint{3.887029in}{2.085709in}}%
\pgfpathlineto{\pgfqpoint{3.900211in}{2.086329in}}%
\pgfpathlineto{\pgfqpoint{3.913401in}{2.087121in}}%
\pgfpathlineto{\pgfqpoint{3.926598in}{2.088084in}}%
\pgfpathlineto{\pgfqpoint{3.939804in}{2.089219in}}%
\pgfpathlineto{\pgfqpoint{3.947497in}{2.099278in}}%
\pgfpathlineto{\pgfqpoint{3.955185in}{2.109308in}}%
\pgfpathlineto{\pgfqpoint{3.962869in}{2.119310in}}%
\pgfpathlineto{\pgfqpoint{3.970547in}{2.129283in}}%
\pgfpathlineto{\pgfqpoint{3.957349in}{2.128072in}}%
\pgfpathlineto{\pgfqpoint{3.944159in}{2.127031in}}%
\pgfpathlineto{\pgfqpoint{3.930977in}{2.126162in}}%
\pgfpathlineto{\pgfqpoint{3.917803in}{2.125465in}}%
\pgfpathlineto{\pgfqpoint{3.910117in}{2.115558in}}%
\pgfpathlineto{\pgfqpoint{3.902426in}{2.105630in}}%
\pgfpathlineto{\pgfqpoint{3.894730in}{2.095680in}}%
\pgfpathlineto{\pgfqpoint{3.887029in}{2.085709in}}%
\pgfpathclose%
\pgfusepath{fill}%
\end{pgfscope}%
\begin{pgfscope}%
\pgfpathrectangle{\pgfqpoint{1.254980in}{0.150000in}}{\pgfqpoint{5.490039in}{5.490039in}}%
\pgfusepath{clip}%
\pgfsetbuttcap%
\pgfsetroundjoin%
\definecolor{currentfill}{rgb}{0.281887,0.150881,0.465405}%
\pgfsetfillcolor{currentfill}%
\pgfsetfillopacity{0.700000}%
\pgfsetlinewidth{0.000000pt}%
\definecolor{currentstroke}{rgb}{0.000000,0.000000,0.000000}%
\pgfsetstrokecolor{currentstroke}%
\pgfsetdash{}{0pt}%
\pgfpathmoveto{\pgfqpoint{3.970547in}{2.129283in}}%
\pgfpathlineto{\pgfqpoint{3.983754in}{2.130666in}}%
\pgfpathlineto{\pgfqpoint{3.996969in}{2.132218in}}%
\pgfpathlineto{\pgfqpoint{4.010192in}{2.133940in}}%
\pgfpathlineto{\pgfqpoint{4.023425in}{2.135832in}}%
\pgfpathlineto{\pgfqpoint{4.031091in}{2.145835in}}%
\pgfpathlineto{\pgfqpoint{4.038752in}{2.155802in}}%
\pgfpathlineto{\pgfqpoint{4.046409in}{2.165733in}}%
\pgfpathlineto{\pgfqpoint{4.054060in}{2.175627in}}%
\pgfpathlineto{\pgfqpoint{4.040835in}{2.173687in}}%
\pgfpathlineto{\pgfqpoint{4.027618in}{2.171915in}}%
\pgfpathlineto{\pgfqpoint{4.014410in}{2.170314in}}%
\pgfpathlineto{\pgfqpoint{4.001211in}{2.168882in}}%
\pgfpathlineto{\pgfqpoint{3.993552in}{2.159027in}}%
\pgfpathlineto{\pgfqpoint{3.985889in}{2.149142in}}%
\pgfpathlineto{\pgfqpoint{3.978221in}{2.139227in}}%
\pgfpathlineto{\pgfqpoint{3.970547in}{2.129283in}}%
\pgfpathclose%
\pgfusepath{fill}%
\end{pgfscope}%
\begin{pgfscope}%
\pgfpathrectangle{\pgfqpoint{1.254980in}{0.150000in}}{\pgfqpoint{5.490039in}{5.490039in}}%
\pgfusepath{clip}%
\pgfsetbuttcap%
\pgfsetroundjoin%
\definecolor{currentfill}{rgb}{0.282910,0.105393,0.426902}%
\pgfsetfillcolor{currentfill}%
\pgfsetfillopacity{0.700000}%
\pgfsetlinewidth{0.000000pt}%
\definecolor{currentstroke}{rgb}{0.000000,0.000000,0.000000}%
\pgfsetstrokecolor{currentstroke}%
\pgfsetdash{}{0pt}%
\pgfpathmoveto{\pgfqpoint{3.803490in}{2.045292in}}%
\pgfpathlineto{\pgfqpoint{3.816651in}{2.045114in}}%
\pgfpathlineto{\pgfqpoint{3.829819in}{2.045110in}}%
\pgfpathlineto{\pgfqpoint{3.842994in}{2.045280in}}%
\pgfpathlineto{\pgfqpoint{3.856176in}{2.045623in}}%
\pgfpathlineto{\pgfqpoint{3.863897in}{2.055673in}}%
\pgfpathlineto{\pgfqpoint{3.871613in}{2.065705in}}%
\pgfpathlineto{\pgfqpoint{3.879323in}{2.075717in}}%
\pgfpathlineto{\pgfqpoint{3.887029in}{2.085709in}}%
\pgfpathlineto{\pgfqpoint{3.873855in}{2.085261in}}%
\pgfpathlineto{\pgfqpoint{3.860688in}{2.084986in}}%
\pgfpathlineto{\pgfqpoint{3.847529in}{2.084885in}}%
\pgfpathlineto{\pgfqpoint{3.834376in}{2.084958in}}%
\pgfpathlineto{\pgfqpoint{3.826662in}{2.075061in}}%
\pgfpathlineto{\pgfqpoint{3.818943in}{2.065150in}}%
\pgfpathlineto{\pgfqpoint{3.811219in}{2.055227in}}%
\pgfpathlineto{\pgfqpoint{3.803490in}{2.045292in}}%
\pgfpathclose%
\pgfusepath{fill}%
\end{pgfscope}%
\begin{pgfscope}%
\pgfpathrectangle{\pgfqpoint{1.254980in}{0.150000in}}{\pgfqpoint{5.490039in}{5.490039in}}%
\pgfusepath{clip}%
\pgfsetbuttcap%
\pgfsetroundjoin%
\definecolor{currentfill}{rgb}{0.121380,0.629492,0.531973}%
\pgfsetfillcolor{currentfill}%
\pgfsetfillopacity{0.700000}%
\pgfsetlinewidth{0.000000pt}%
\definecolor{currentstroke}{rgb}{0.000000,0.000000,0.000000}%
\pgfsetstrokecolor{currentstroke}%
\pgfsetdash{}{0pt}%
\pgfpathmoveto{\pgfqpoint{5.785516in}{3.282322in}}%
\pgfpathlineto{\pgfqpoint{5.799549in}{3.290882in}}%
\pgfpathlineto{\pgfqpoint{5.813598in}{3.299594in}}%
\pgfpathlineto{\pgfqpoint{5.827666in}{3.308458in}}%
\pgfpathlineto{\pgfqpoint{5.841752in}{3.317473in}}%
\pgfpathlineto{\pgfqpoint{5.848600in}{3.320415in}}%
\pgfpathlineto{\pgfqpoint{5.855446in}{3.323476in}}%
\pgfpathlineto{\pgfqpoint{5.862290in}{3.326662in}}%
\pgfpathlineto{\pgfqpoint{5.869133in}{3.329979in}}%
\pgfpathlineto{\pgfqpoint{5.855081in}{3.321585in}}%
\pgfpathlineto{\pgfqpoint{5.841046in}{3.313341in}}%
\pgfpathlineto{\pgfqpoint{5.827029in}{3.305248in}}%
\pgfpathlineto{\pgfqpoint{5.813029in}{3.297305in}}%
\pgfpathlineto{\pgfqpoint{5.806153in}{3.293359in}}%
\pgfpathlineto{\pgfqpoint{5.799276in}{3.289551in}}%
\pgfpathlineto{\pgfqpoint{5.792397in}{3.285874in}}%
\pgfpathlineto{\pgfqpoint{5.785516in}{3.282322in}}%
\pgfpathclose%
\pgfusepath{fill}%
\end{pgfscope}%
\begin{pgfscope}%
\pgfpathrectangle{\pgfqpoint{1.254980in}{0.150000in}}{\pgfqpoint{5.490039in}{5.490039in}}%
\pgfusepath{clip}%
\pgfsetbuttcap%
\pgfsetroundjoin%
\definecolor{currentfill}{rgb}{0.201239,0.383670,0.554294}%
\pgfsetfillcolor{currentfill}%
\pgfsetfillopacity{0.700000}%
\pgfsetlinewidth{0.000000pt}%
\definecolor{currentstroke}{rgb}{0.000000,0.000000,0.000000}%
\pgfsetstrokecolor{currentstroke}%
\pgfsetdash{}{0pt}%
\pgfpathmoveto{\pgfqpoint{4.752710in}{2.635891in}}%
\pgfpathlineto{\pgfqpoint{4.766242in}{2.642393in}}%
\pgfpathlineto{\pgfqpoint{4.779786in}{2.649055in}}%
\pgfpathlineto{\pgfqpoint{4.793345in}{2.655877in}}%
\pgfpathlineto{\pgfqpoint{4.806917in}{2.662857in}}%
\pgfpathlineto{\pgfqpoint{4.814292in}{2.670023in}}%
\pgfpathlineto{\pgfqpoint{4.821660in}{2.677141in}}%
\pgfpathlineto{\pgfqpoint{4.829023in}{2.684215in}}%
\pgfpathlineto{\pgfqpoint{4.836379in}{2.691247in}}%
\pgfpathlineto{\pgfqpoint{4.822818in}{2.684506in}}%
\pgfpathlineto{\pgfqpoint{4.809270in}{2.677922in}}%
\pgfpathlineto{\pgfqpoint{4.795737in}{2.671498in}}%
\pgfpathlineto{\pgfqpoint{4.782217in}{2.665233in}}%
\pgfpathlineto{\pgfqpoint{4.774849in}{2.657952in}}%
\pgfpathlineto{\pgfqpoint{4.767475in}{2.650636in}}%
\pgfpathlineto{\pgfqpoint{4.760096in}{2.643284in}}%
\pgfpathlineto{\pgfqpoint{4.752710in}{2.635891in}}%
\pgfpathclose%
\pgfusepath{fill}%
\end{pgfscope}%
\begin{pgfscope}%
\pgfpathrectangle{\pgfqpoint{1.254980in}{0.150000in}}{\pgfqpoint{5.490039in}{5.490039in}}%
\pgfusepath{clip}%
\pgfsetbuttcap%
\pgfsetroundjoin%
\definecolor{currentfill}{rgb}{0.278826,0.175490,0.483397}%
\pgfsetfillcolor{currentfill}%
\pgfsetfillopacity{0.700000}%
\pgfsetlinewidth{0.000000pt}%
\definecolor{currentstroke}{rgb}{0.000000,0.000000,0.000000}%
\pgfsetstrokecolor{currentstroke}%
\pgfsetdash{}{0pt}%
\pgfpathmoveto{\pgfqpoint{4.054060in}{2.175627in}}%
\pgfpathlineto{\pgfqpoint{4.067295in}{2.177736in}}%
\pgfpathlineto{\pgfqpoint{4.080538in}{2.180014in}}%
\pgfpathlineto{\pgfqpoint{4.093791in}{2.182460in}}%
\pgfpathlineto{\pgfqpoint{4.107053in}{2.185074in}}%
\pgfpathlineto{\pgfqpoint{4.114693in}{2.194963in}}%
\pgfpathlineto{\pgfqpoint{4.122327in}{2.204808in}}%
\pgfpathlineto{\pgfqpoint{4.129957in}{2.214611in}}%
\pgfpathlineto{\pgfqpoint{4.137581in}{2.224370in}}%
\pgfpathlineto{\pgfqpoint{4.124326in}{2.221735in}}%
\pgfpathlineto{\pgfqpoint{4.111080in}{2.219269in}}%
\pgfpathlineto{\pgfqpoint{4.097843in}{2.216970in}}%
\pgfpathlineto{\pgfqpoint{4.084616in}{2.214840in}}%
\pgfpathlineto{\pgfqpoint{4.076984in}{2.205091in}}%
\pgfpathlineto{\pgfqpoint{4.069348in}{2.195306in}}%
\pgfpathlineto{\pgfqpoint{4.061707in}{2.185485in}}%
\pgfpathlineto{\pgfqpoint{4.054060in}{2.175627in}}%
\pgfpathclose%
\pgfusepath{fill}%
\end{pgfscope}%
\begin{pgfscope}%
\pgfpathrectangle{\pgfqpoint{1.254980in}{0.150000in}}{\pgfqpoint{5.490039in}{5.490039in}}%
\pgfusepath{clip}%
\pgfsetbuttcap%
\pgfsetroundjoin%
\definecolor{currentfill}{rgb}{0.282656,0.100196,0.422160}%
\pgfsetfillcolor{currentfill}%
\pgfsetfillopacity{0.700000}%
\pgfsetlinewidth{0.000000pt}%
\definecolor{currentstroke}{rgb}{0.000000,0.000000,0.000000}%
\pgfsetstrokecolor{currentstroke}%
\pgfsetdash{}{0pt}%
\pgfpathmoveto{\pgfqpoint{2.953405in}{2.067861in}}%
\pgfpathlineto{\pgfqpoint{2.966556in}{2.056854in}}%
\pgfpathlineto{\pgfqpoint{2.979705in}{2.046068in}}%
\pgfpathlineto{\pgfqpoint{2.992852in}{2.035501in}}%
\pgfpathlineto{\pgfqpoint{3.005997in}{2.025153in}}%
\pgfpathlineto{\pgfqpoint{3.014072in}{2.031414in}}%
\pgfpathlineto{\pgfqpoint{3.022138in}{2.037789in}}%
\pgfpathlineto{\pgfqpoint{3.030196in}{2.044277in}}%
\pgfpathlineto{\pgfqpoint{3.038245in}{2.050873in}}%
\pgfpathlineto{\pgfqpoint{3.025124in}{2.060917in}}%
\pgfpathlineto{\pgfqpoint{3.012001in}{2.071179in}}%
\pgfpathlineto{\pgfqpoint{2.998877in}{2.081660in}}%
\pgfpathlineto{\pgfqpoint{2.985749in}{2.092361in}}%
\pgfpathlineto{\pgfqpoint{2.977677in}{2.086059in}}%
\pgfpathlineto{\pgfqpoint{2.969595in}{2.079873in}}%
\pgfpathlineto{\pgfqpoint{2.961505in}{2.073806in}}%
\pgfpathlineto{\pgfqpoint{2.953405in}{2.067861in}}%
\pgfpathclose%
\pgfusepath{fill}%
\end{pgfscope}%
\begin{pgfscope}%
\pgfpathrectangle{\pgfqpoint{1.254980in}{0.150000in}}{\pgfqpoint{5.490039in}{5.490039in}}%
\pgfusepath{clip}%
\pgfsetbuttcap%
\pgfsetroundjoin%
\definecolor{currentfill}{rgb}{0.274952,0.037752,0.364543}%
\pgfsetfillcolor{currentfill}%
\pgfsetfillopacity{0.700000}%
\pgfsetlinewidth{0.000000pt}%
\definecolor{currentstroke}{rgb}{0.000000,0.000000,0.000000}%
\pgfsetstrokecolor{currentstroke}%
\pgfsetdash{}{0pt}%
\pgfpathmoveto{\pgfqpoint{3.416276in}{1.940585in}}%
\pgfpathlineto{\pgfqpoint{3.429379in}{1.936095in}}%
\pgfpathlineto{\pgfqpoint{3.442485in}{1.931792in}}%
\pgfpathlineto{\pgfqpoint{3.455595in}{1.927676in}}%
\pgfpathlineto{\pgfqpoint{3.468708in}{1.923746in}}%
\pgfpathlineto{\pgfqpoint{3.476566in}{1.932809in}}%
\pgfpathlineto{\pgfqpoint{3.484419in}{1.941910in}}%
\pgfpathlineto{\pgfqpoint{3.492266in}{1.951045in}}%
\pgfpathlineto{\pgfqpoint{3.500107in}{1.960214in}}%
\pgfpathlineto{\pgfqpoint{3.487007in}{1.963927in}}%
\pgfpathlineto{\pgfqpoint{3.473911in}{1.967827in}}%
\pgfpathlineto{\pgfqpoint{3.460818in}{1.971913in}}%
\pgfpathlineto{\pgfqpoint{3.447729in}{1.976188in}}%
\pgfpathlineto{\pgfqpoint{3.439875in}{1.967225in}}%
\pgfpathlineto{\pgfqpoint{3.432015in}{1.958302in}}%
\pgfpathlineto{\pgfqpoint{3.424148in}{1.949422in}}%
\pgfpathlineto{\pgfqpoint{3.416276in}{1.940585in}}%
\pgfpathclose%
\pgfusepath{fill}%
\end{pgfscope}%
\begin{pgfscope}%
\pgfpathrectangle{\pgfqpoint{1.254980in}{0.150000in}}{\pgfqpoint{5.490039in}{5.490039in}}%
\pgfusepath{clip}%
\pgfsetbuttcap%
\pgfsetroundjoin%
\definecolor{currentfill}{rgb}{0.128087,0.647749,0.523491}%
\pgfsetfillcolor{currentfill}%
\pgfsetfillopacity{0.700000}%
\pgfsetlinewidth{0.000000pt}%
\definecolor{currentstroke}{rgb}{0.000000,0.000000,0.000000}%
\pgfsetstrokecolor{currentstroke}%
\pgfsetdash{}{0pt}%
\pgfpathmoveto{\pgfqpoint{5.869133in}{3.329979in}}%
\pgfpathlineto{\pgfqpoint{5.883203in}{3.338524in}}%
\pgfpathlineto{\pgfqpoint{5.897291in}{3.347220in}}%
\pgfpathlineto{\pgfqpoint{5.911398in}{3.356067in}}%
\pgfpathlineto{\pgfqpoint{5.925522in}{3.365064in}}%
\pgfpathlineto{\pgfqpoint{5.932329in}{3.367881in}}%
\pgfpathlineto{\pgfqpoint{5.939134in}{3.370837in}}%
\pgfpathlineto{\pgfqpoint{5.945939in}{3.373939in}}%
\pgfpathlineto{\pgfqpoint{5.952743in}{3.377194in}}%
\pgfpathlineto{\pgfqpoint{5.938654in}{3.368846in}}%
\pgfpathlineto{\pgfqpoint{5.924583in}{3.360648in}}%
\pgfpathlineto{\pgfqpoint{5.910530in}{3.352600in}}%
\pgfpathlineto{\pgfqpoint{5.896494in}{3.344701in}}%
\pgfpathlineto{\pgfqpoint{5.889655in}{3.340789in}}%
\pgfpathlineto{\pgfqpoint{5.882815in}{3.337036in}}%
\pgfpathlineto{\pgfqpoint{5.875974in}{3.333435in}}%
\pgfpathlineto{\pgfqpoint{5.869133in}{3.329979in}}%
\pgfpathclose%
\pgfusepath{fill}%
\end{pgfscope}%
\begin{pgfscope}%
\pgfpathrectangle{\pgfqpoint{1.254980in}{0.150000in}}{\pgfqpoint{5.490039in}{5.490039in}}%
\pgfusepath{clip}%
\pgfsetbuttcap%
\pgfsetroundjoin%
\definecolor{currentfill}{rgb}{0.140210,0.665859,0.513427}%
\pgfsetfillcolor{currentfill}%
\pgfsetfillopacity{0.700000}%
\pgfsetlinewidth{0.000000pt}%
\definecolor{currentstroke}{rgb}{0.000000,0.000000,0.000000}%
\pgfsetstrokecolor{currentstroke}%
\pgfsetdash{}{0pt}%
\pgfpathmoveto{\pgfqpoint{5.952743in}{3.377194in}}%
\pgfpathlineto{\pgfqpoint{5.966849in}{3.385692in}}%
\pgfpathlineto{\pgfqpoint{5.980974in}{3.394340in}}%
\pgfpathlineto{\pgfqpoint{5.995118in}{3.403139in}}%
\pgfpathlineto{\pgfqpoint{6.009280in}{3.412088in}}%
\pgfpathlineto{\pgfqpoint{6.016046in}{3.414835in}}%
\pgfpathlineto{\pgfqpoint{6.022813in}{3.417744in}}%
\pgfpathlineto{\pgfqpoint{6.029580in}{3.420821in}}%
\pgfpathlineto{\pgfqpoint{6.036347in}{3.424073in}}%
\pgfpathlineto{\pgfqpoint{6.022223in}{3.415803in}}%
\pgfpathlineto{\pgfqpoint{6.008118in}{3.407682in}}%
\pgfpathlineto{\pgfqpoint{5.994030in}{3.399710in}}%
\pgfpathlineto{\pgfqpoint{5.979960in}{3.391887in}}%
\pgfpathlineto{\pgfqpoint{5.973155in}{3.387949in}}%
\pgfpathlineto{\pgfqpoint{5.966350in}{3.384192in}}%
\pgfpathlineto{\pgfqpoint{5.959546in}{3.380609in}}%
\pgfpathlineto{\pgfqpoint{5.952743in}{3.377194in}}%
\pgfpathclose%
\pgfusepath{fill}%
\end{pgfscope}%
\begin{pgfscope}%
\pgfpathrectangle{\pgfqpoint{1.254980in}{0.150000in}}{\pgfqpoint{5.490039in}{5.490039in}}%
\pgfusepath{clip}%
\pgfsetbuttcap%
\pgfsetroundjoin%
\definecolor{currentfill}{rgb}{0.199430,0.387607,0.554642}%
\pgfsetfillcolor{currentfill}%
\pgfsetfillopacity{0.700000}%
\pgfsetlinewidth{0.000000pt}%
\definecolor{currentstroke}{rgb}{0.000000,0.000000,0.000000}%
\pgfsetstrokecolor{currentstroke}%
\pgfsetdash{}{0pt}%
\pgfpathmoveto{\pgfqpoint{2.422586in}{2.710267in}}%
\pgfpathlineto{\pgfqpoint{2.436037in}{2.688823in}}%
\pgfpathlineto{\pgfqpoint{2.449474in}{2.667689in}}%
\pgfpathlineto{\pgfqpoint{2.462899in}{2.646861in}}%
\pgfpathlineto{\pgfqpoint{2.476312in}{2.626336in}}%
\pgfpathlineto{\pgfqpoint{2.484691in}{2.629360in}}%
\pgfpathlineto{\pgfqpoint{2.493057in}{2.632571in}}%
\pgfpathlineto{\pgfqpoint{2.501409in}{2.635967in}}%
\pgfpathlineto{\pgfqpoint{2.509748in}{2.639546in}}%
\pgfpathlineto{\pgfqpoint{2.496372in}{2.659746in}}%
\pgfpathlineto{\pgfqpoint{2.482984in}{2.680248in}}%
\pgfpathlineto{\pgfqpoint{2.469584in}{2.701056in}}%
\pgfpathlineto{\pgfqpoint{2.456172in}{2.722173in}}%
\pgfpathlineto{\pgfqpoint{2.447796in}{2.718910in}}%
\pgfpathlineto{\pgfqpoint{2.439407in}{2.715835in}}%
\pgfpathlineto{\pgfqpoint{2.431004in}{2.712954in}}%
\pgfpathlineto{\pgfqpoint{2.422586in}{2.710267in}}%
\pgfpathclose%
\pgfusepath{fill}%
\end{pgfscope}%
\begin{pgfscope}%
\pgfpathrectangle{\pgfqpoint{1.254980in}{0.150000in}}{\pgfqpoint{5.490039in}{5.490039in}}%
\pgfusepath{clip}%
\pgfsetbuttcap%
\pgfsetroundjoin%
\definecolor{currentfill}{rgb}{0.274128,0.199721,0.498911}%
\pgfsetfillcolor{currentfill}%
\pgfsetfillopacity{0.700000}%
\pgfsetlinewidth{0.000000pt}%
\definecolor{currentstroke}{rgb}{0.000000,0.000000,0.000000}%
\pgfsetstrokecolor{currentstroke}%
\pgfsetdash{}{0pt}%
\pgfpathmoveto{\pgfqpoint{4.137581in}{2.224370in}}%
\pgfpathlineto{\pgfqpoint{4.150847in}{2.227172in}}%
\pgfpathlineto{\pgfqpoint{4.164122in}{2.230140in}}%
\pgfpathlineto{\pgfqpoint{4.177407in}{2.233276in}}%
\pgfpathlineto{\pgfqpoint{4.190701in}{2.236578in}}%
\pgfpathlineto{\pgfqpoint{4.198314in}{2.246297in}}%
\pgfpathlineto{\pgfqpoint{4.205922in}{2.255968in}}%
\pgfpathlineto{\pgfqpoint{4.213524in}{2.265590in}}%
\pgfpathlineto{\pgfqpoint{4.221122in}{2.275163in}}%
\pgfpathlineto{\pgfqpoint{4.207833in}{2.271869in}}%
\pgfpathlineto{\pgfqpoint{4.194555in}{2.268741in}}%
\pgfpathlineto{\pgfqpoint{4.181287in}{2.265779in}}%
\pgfpathlineto{\pgfqpoint{4.168028in}{2.262985in}}%
\pgfpathlineto{\pgfqpoint{4.160424in}{2.253394in}}%
\pgfpathlineto{\pgfqpoint{4.152815in}{2.243761in}}%
\pgfpathlineto{\pgfqpoint{4.145201in}{2.234087in}}%
\pgfpathlineto{\pgfqpoint{4.137581in}{2.224370in}}%
\pgfpathclose%
\pgfusepath{fill}%
\end{pgfscope}%
\begin{pgfscope}%
\pgfpathrectangle{\pgfqpoint{1.254980in}{0.150000in}}{\pgfqpoint{5.490039in}{5.490039in}}%
\pgfusepath{clip}%
\pgfsetbuttcap%
\pgfsetroundjoin%
\definecolor{currentfill}{rgb}{0.281446,0.084320,0.407414}%
\pgfsetfillcolor{currentfill}%
\pgfsetfillopacity{0.700000}%
\pgfsetlinewidth{0.000000pt}%
\definecolor{currentstroke}{rgb}{0.000000,0.000000,0.000000}%
\pgfsetstrokecolor{currentstroke}%
\pgfsetdash{}{0pt}%
\pgfpathmoveto{\pgfqpoint{3.719912in}{2.008444in}}%
\pgfpathlineto{\pgfqpoint{3.733055in}{2.007432in}}%
\pgfpathlineto{\pgfqpoint{3.746205in}{2.006596in}}%
\pgfpathlineto{\pgfqpoint{3.759361in}{2.005936in}}%
\pgfpathlineto{\pgfqpoint{3.772524in}{2.005451in}}%
\pgfpathlineto{\pgfqpoint{3.780273in}{2.015424in}}%
\pgfpathlineto{\pgfqpoint{3.788017in}{2.025390in}}%
\pgfpathlineto{\pgfqpoint{3.795756in}{2.035346in}}%
\pgfpathlineto{\pgfqpoint{3.803490in}{2.045292in}}%
\pgfpathlineto{\pgfqpoint{3.790336in}{2.045645in}}%
\pgfpathlineto{\pgfqpoint{3.777189in}{2.046172in}}%
\pgfpathlineto{\pgfqpoint{3.764049in}{2.046875in}}%
\pgfpathlineto{\pgfqpoint{3.750915in}{2.047755in}}%
\pgfpathlineto{\pgfqpoint{3.743172in}{2.037931in}}%
\pgfpathlineto{\pgfqpoint{3.735424in}{2.028104in}}%
\pgfpathlineto{\pgfqpoint{3.727670in}{2.018275in}}%
\pgfpathlineto{\pgfqpoint{3.719912in}{2.008444in}}%
\pgfpathclose%
\pgfusepath{fill}%
\end{pgfscope}%
\begin{pgfscope}%
\pgfpathrectangle{\pgfqpoint{1.254980in}{0.150000in}}{\pgfqpoint{5.490039in}{5.490039in}}%
\pgfusepath{clip}%
\pgfsetbuttcap%
\pgfsetroundjoin%
\definecolor{currentfill}{rgb}{0.190631,0.407061,0.556089}%
\pgfsetfillcolor{currentfill}%
\pgfsetfillopacity{0.700000}%
\pgfsetlinewidth{0.000000pt}%
\definecolor{currentstroke}{rgb}{0.000000,0.000000,0.000000}%
\pgfsetstrokecolor{currentstroke}%
\pgfsetdash{}{0pt}%
\pgfpathmoveto{\pgfqpoint{4.836379in}{2.691247in}}%
\pgfpathlineto{\pgfqpoint{4.849954in}{2.698148in}}%
\pgfpathlineto{\pgfqpoint{4.863543in}{2.705207in}}%
\pgfpathlineto{\pgfqpoint{4.877147in}{2.712425in}}%
\pgfpathlineto{\pgfqpoint{4.890765in}{2.719802in}}%
\pgfpathlineto{\pgfqpoint{4.898103in}{2.726539in}}%
\pgfpathlineto{\pgfqpoint{4.905435in}{2.733234in}}%
\pgfpathlineto{\pgfqpoint{4.912761in}{2.739889in}}%
\pgfpathlineto{\pgfqpoint{4.920080in}{2.746509in}}%
\pgfpathlineto{\pgfqpoint{4.906474in}{2.739401in}}%
\pgfpathlineto{\pgfqpoint{4.892883in}{2.732451in}}%
\pgfpathlineto{\pgfqpoint{4.879306in}{2.725659in}}%
\pgfpathlineto{\pgfqpoint{4.865743in}{2.719025in}}%
\pgfpathlineto{\pgfqpoint{4.858411in}{2.712127in}}%
\pgfpathlineto{\pgfqpoint{4.851073in}{2.705201in}}%
\pgfpathlineto{\pgfqpoint{4.843729in}{2.698242in}}%
\pgfpathlineto{\pgfqpoint{4.836379in}{2.691247in}}%
\pgfpathclose%
\pgfusepath{fill}%
\end{pgfscope}%
\begin{pgfscope}%
\pgfpathrectangle{\pgfqpoint{1.254980in}{0.150000in}}{\pgfqpoint{5.490039in}{5.490039in}}%
\pgfusepath{clip}%
\pgfsetbuttcap%
\pgfsetroundjoin%
\definecolor{currentfill}{rgb}{0.266580,0.228262,0.514349}%
\pgfsetfillcolor{currentfill}%
\pgfsetfillopacity{0.700000}%
\pgfsetlinewidth{0.000000pt}%
\definecolor{currentstroke}{rgb}{0.000000,0.000000,0.000000}%
\pgfsetstrokecolor{currentstroke}%
\pgfsetdash{}{0pt}%
\pgfpathmoveto{\pgfqpoint{4.221122in}{2.275163in}}%
\pgfpathlineto{\pgfqpoint{4.234420in}{2.278623in}}%
\pgfpathlineto{\pgfqpoint{4.247729in}{2.282249in}}%
\pgfpathlineto{\pgfqpoint{4.261049in}{2.286040in}}%
\pgfpathlineto{\pgfqpoint{4.274379in}{2.289996in}}%
\pgfpathlineto{\pgfqpoint{4.281965in}{2.299497in}}%
\pgfpathlineto{\pgfqpoint{4.289545in}{2.308943in}}%
\pgfpathlineto{\pgfqpoint{4.297120in}{2.318336in}}%
\pgfpathlineto{\pgfqpoint{4.304690in}{2.327677in}}%
\pgfpathlineto{\pgfqpoint{4.291366in}{2.323756in}}%
\pgfpathlineto{\pgfqpoint{4.278053in}{2.320001in}}%
\pgfpathlineto{\pgfqpoint{4.264751in}{2.316411in}}%
\pgfpathlineto{\pgfqpoint{4.251459in}{2.312987in}}%
\pgfpathlineto{\pgfqpoint{4.243883in}{2.303600in}}%
\pgfpathlineto{\pgfqpoint{4.236301in}{2.294167in}}%
\pgfpathlineto{\pgfqpoint{4.228714in}{2.284689in}}%
\pgfpathlineto{\pgfqpoint{4.221122in}{2.275163in}}%
\pgfpathclose%
\pgfusepath{fill}%
\end{pgfscope}%
\begin{pgfscope}%
\pgfpathrectangle{\pgfqpoint{1.254980in}{0.150000in}}{\pgfqpoint{5.490039in}{5.490039in}}%
\pgfusepath{clip}%
\pgfsetbuttcap%
\pgfsetroundjoin%
\definecolor{currentfill}{rgb}{0.279566,0.067836,0.391917}%
\pgfsetfillcolor{currentfill}%
\pgfsetfillopacity{0.700000}%
\pgfsetlinewidth{0.000000pt}%
\definecolor{currentstroke}{rgb}{0.000000,0.000000,0.000000}%
\pgfsetstrokecolor{currentstroke}%
\pgfsetdash{}{0pt}%
\pgfpathmoveto{\pgfqpoint{3.636273in}{1.975598in}}%
\pgfpathlineto{\pgfqpoint{3.649403in}{1.973714in}}%
\pgfpathlineto{\pgfqpoint{3.662538in}{1.972008in}}%
\pgfpathlineto{\pgfqpoint{3.675679in}{1.970481in}}%
\pgfpathlineto{\pgfqpoint{3.688826in}{1.969131in}}%
\pgfpathlineto{\pgfqpoint{3.696606in}{1.978956in}}%
\pgfpathlineto{\pgfqpoint{3.704380in}{1.988784in}}%
\pgfpathlineto{\pgfqpoint{3.712148in}{1.998614in}}%
\pgfpathlineto{\pgfqpoint{3.719912in}{2.008444in}}%
\pgfpathlineto{\pgfqpoint{3.706775in}{2.009634in}}%
\pgfpathlineto{\pgfqpoint{3.693644in}{2.011000in}}%
\pgfpathlineto{\pgfqpoint{3.680518in}{2.012545in}}%
\pgfpathlineto{\pgfqpoint{3.667398in}{2.014269in}}%
\pgfpathlineto{\pgfqpoint{3.659625in}{2.004588in}}%
\pgfpathlineto{\pgfqpoint{3.651846in}{1.994915in}}%
\pgfpathlineto{\pgfqpoint{3.644062in}{1.985251in}}%
\pgfpathlineto{\pgfqpoint{3.636273in}{1.975598in}}%
\pgfpathclose%
\pgfusepath{fill}%
\end{pgfscope}%
\begin{pgfscope}%
\pgfpathrectangle{\pgfqpoint{1.254980in}{0.150000in}}{\pgfqpoint{5.490039in}{5.490039in}}%
\pgfusepath{clip}%
\pgfsetbuttcap%
\pgfsetroundjoin%
\definecolor{currentfill}{rgb}{0.157851,0.683765,0.501686}%
\pgfsetfillcolor{currentfill}%
\pgfsetfillopacity{0.700000}%
\pgfsetlinewidth{0.000000pt}%
\definecolor{currentstroke}{rgb}{0.000000,0.000000,0.000000}%
\pgfsetstrokecolor{currentstroke}%
\pgfsetdash{}{0pt}%
\pgfpathmoveto{\pgfqpoint{6.036347in}{3.424073in}}%
\pgfpathlineto{\pgfqpoint{6.050489in}{3.432493in}}%
\pgfpathlineto{\pgfqpoint{6.064650in}{3.441063in}}%
\pgfpathlineto{\pgfqpoint{6.078829in}{3.449782in}}%
\pgfpathlineto{\pgfqpoint{6.093026in}{3.458651in}}%
\pgfpathlineto{\pgfqpoint{6.099755in}{3.461390in}}%
\pgfpathlineto{\pgfqpoint{6.106485in}{3.464314in}}%
\pgfpathlineto{\pgfqpoint{6.113217in}{3.467430in}}%
\pgfpathlineto{\pgfqpoint{6.099050in}{3.459089in}}%
\pgfpathlineto{\pgfqpoint{6.084901in}{3.450897in}}%
\pgfpathlineto{\pgfqpoint{6.070770in}{3.442854in}}%
\pgfpathlineto{\pgfqpoint{6.056657in}{3.434960in}}%
\pgfpathlineto{\pgfqpoint{6.049886in}{3.431136in}}%
\pgfpathlineto{\pgfqpoint{6.043116in}{3.427509in}}%
\pgfpathlineto{\pgfqpoint{6.036347in}{3.424073in}}%
\pgfpathclose%
\pgfusepath{fill}%
\end{pgfscope}%
\begin{pgfscope}%
\pgfpathrectangle{\pgfqpoint{1.254980in}{0.150000in}}{\pgfqpoint{5.490039in}{5.490039in}}%
\pgfusepath{clip}%
\pgfsetbuttcap%
\pgfsetroundjoin%
\definecolor{currentfill}{rgb}{0.258965,0.251537,0.524736}%
\pgfsetfillcolor{currentfill}%
\pgfsetfillopacity{0.700000}%
\pgfsetlinewidth{0.000000pt}%
\definecolor{currentstroke}{rgb}{0.000000,0.000000,0.000000}%
\pgfsetstrokecolor{currentstroke}%
\pgfsetdash{}{0pt}%
\pgfpathmoveto{\pgfqpoint{4.304690in}{2.327677in}}%
\pgfpathlineto{\pgfqpoint{4.318024in}{2.331761in}}%
\pgfpathlineto{\pgfqpoint{4.331370in}{2.336011in}}%
\pgfpathlineto{\pgfqpoint{4.344726in}{2.340424in}}%
\pgfpathlineto{\pgfqpoint{4.358094in}{2.345001in}}%
\pgfpathlineto{\pgfqpoint{4.365652in}{2.354236in}}%
\pgfpathlineto{\pgfqpoint{4.373204in}{2.363414in}}%
\pgfpathlineto{\pgfqpoint{4.380751in}{2.372535in}}%
\pgfpathlineto{\pgfqpoint{4.388292in}{2.381600in}}%
\pgfpathlineto{\pgfqpoint{4.374931in}{2.377088in}}%
\pgfpathlineto{\pgfqpoint{4.361581in}{2.372739in}}%
\pgfpathlineto{\pgfqpoint{4.348242in}{2.368554in}}%
\pgfpathlineto{\pgfqpoint{4.334915in}{2.364533in}}%
\pgfpathlineto{\pgfqpoint{4.327367in}{2.355393in}}%
\pgfpathlineto{\pgfqpoint{4.319813in}{2.346204in}}%
\pgfpathlineto{\pgfqpoint{4.312254in}{2.336966in}}%
\pgfpathlineto{\pgfqpoint{4.304690in}{2.327677in}}%
\pgfpathclose%
\pgfusepath{fill}%
\end{pgfscope}%
\begin{pgfscope}%
\pgfpathrectangle{\pgfqpoint{1.254980in}{0.150000in}}{\pgfqpoint{5.490039in}{5.490039in}}%
\pgfusepath{clip}%
\pgfsetbuttcap%
\pgfsetroundjoin%
\definecolor{currentfill}{rgb}{0.180629,0.429975,0.557282}%
\pgfsetfillcolor{currentfill}%
\pgfsetfillopacity{0.700000}%
\pgfsetlinewidth{0.000000pt}%
\definecolor{currentstroke}{rgb}{0.000000,0.000000,0.000000}%
\pgfsetstrokecolor{currentstroke}%
\pgfsetdash{}{0pt}%
\pgfpathmoveto{\pgfqpoint{4.920080in}{2.746509in}}%
\pgfpathlineto{\pgfqpoint{4.933700in}{2.753775in}}%
\pgfpathlineto{\pgfqpoint{4.947335in}{2.761200in}}%
\pgfpathlineto{\pgfqpoint{4.960984in}{2.768782in}}%
\pgfpathlineto{\pgfqpoint{4.974648in}{2.776521in}}%
\pgfpathlineto{\pgfqpoint{4.981948in}{2.782822in}}%
\pgfpathlineto{\pgfqpoint{4.989242in}{2.789086in}}%
\pgfpathlineto{\pgfqpoint{4.996529in}{2.795317in}}%
\pgfpathlineto{\pgfqpoint{5.003810in}{2.801520in}}%
\pgfpathlineto{\pgfqpoint{4.990160in}{2.794078in}}%
\pgfpathlineto{\pgfqpoint{4.976524in}{2.786793in}}%
\pgfpathlineto{\pgfqpoint{4.962903in}{2.779666in}}%
\pgfpathlineto{\pgfqpoint{4.949297in}{2.772696in}}%
\pgfpathlineto{\pgfqpoint{4.942001in}{2.766186in}}%
\pgfpathlineto{\pgfqpoint{4.934700in}{2.759654in}}%
\pgfpathlineto{\pgfqpoint{4.927393in}{2.753096in}}%
\pgfpathlineto{\pgfqpoint{4.920080in}{2.746509in}}%
\pgfpathclose%
\pgfusepath{fill}%
\end{pgfscope}%
\begin{pgfscope}%
\pgfpathrectangle{\pgfqpoint{1.254980in}{0.150000in}}{\pgfqpoint{5.490039in}{5.490039in}}%
\pgfusepath{clip}%
\pgfsetbuttcap%
\pgfsetroundjoin%
\definecolor{currentfill}{rgb}{0.281446,0.084320,0.407414}%
\pgfsetfillcolor{currentfill}%
\pgfsetfillopacity{0.700000}%
\pgfsetlinewidth{0.000000pt}%
\definecolor{currentstroke}{rgb}{0.000000,0.000000,0.000000}%
\pgfsetstrokecolor{currentstroke}%
\pgfsetdash{}{0pt}%
\pgfpathmoveto{\pgfqpoint{3.005997in}{2.025153in}}%
\pgfpathlineto{\pgfqpoint{3.019139in}{2.015020in}}%
\pgfpathlineto{\pgfqpoint{3.032280in}{2.005103in}}%
\pgfpathlineto{\pgfqpoint{3.045419in}{1.995399in}}%
\pgfpathlineto{\pgfqpoint{3.058557in}{1.985908in}}%
\pgfpathlineto{\pgfqpoint{3.066609in}{1.992483in}}%
\pgfpathlineto{\pgfqpoint{3.074652in}{1.999166in}}%
\pgfpathlineto{\pgfqpoint{3.082687in}{2.005954in}}%
\pgfpathlineto{\pgfqpoint{3.090713in}{2.012843in}}%
\pgfpathlineto{\pgfqpoint{3.077598in}{2.022031in}}%
\pgfpathlineto{\pgfqpoint{3.064482in}{2.031431in}}%
\pgfpathlineto{\pgfqpoint{3.051364in}{2.041045in}}%
\pgfpathlineto{\pgfqpoint{3.038245in}{2.050873in}}%
\pgfpathlineto{\pgfqpoint{3.030196in}{2.044277in}}%
\pgfpathlineto{\pgfqpoint{3.022138in}{2.037789in}}%
\pgfpathlineto{\pgfqpoint{3.014072in}{2.031414in}}%
\pgfpathlineto{\pgfqpoint{3.005997in}{2.025153in}}%
\pgfpathclose%
\pgfusepath{fill}%
\end{pgfscope}%
\begin{pgfscope}%
\pgfpathrectangle{\pgfqpoint{1.254980in}{0.150000in}}{\pgfqpoint{5.490039in}{5.490039in}}%
\pgfusepath{clip}%
\pgfsetbuttcap%
\pgfsetroundjoin%
\definecolor{currentfill}{rgb}{0.276022,0.044167,0.370164}%
\pgfsetfillcolor{currentfill}%
\pgfsetfillopacity{0.700000}%
\pgfsetlinewidth{0.000000pt}%
\definecolor{currentstroke}{rgb}{0.000000,0.000000,0.000000}%
\pgfsetstrokecolor{currentstroke}%
\pgfsetdash{}{0pt}%
\pgfpathmoveto{\pgfqpoint{3.195617in}{1.946816in}}%
\pgfpathlineto{\pgfqpoint{3.208731in}{1.939479in}}%
\pgfpathlineto{\pgfqpoint{3.221845in}{1.932342in}}%
\pgfpathlineto{\pgfqpoint{3.234961in}{1.925403in}}%
\pgfpathlineto{\pgfqpoint{3.248077in}{1.918662in}}%
\pgfpathlineto{\pgfqpoint{3.256035in}{1.926504in}}%
\pgfpathlineto{\pgfqpoint{3.263987in}{1.934422in}}%
\pgfpathlineto{\pgfqpoint{3.271931in}{1.942413in}}%
\pgfpathlineto{\pgfqpoint{3.279868in}{1.950474in}}%
\pgfpathlineto{\pgfqpoint{3.266770in}{1.956942in}}%
\pgfpathlineto{\pgfqpoint{3.253673in}{1.963607in}}%
\pgfpathlineto{\pgfqpoint{3.240577in}{1.970471in}}%
\pgfpathlineto{\pgfqpoint{3.227483in}{1.977535in}}%
\pgfpathlineto{\pgfqpoint{3.219527in}{1.969736in}}%
\pgfpathlineto{\pgfqpoint{3.211565in}{1.962015in}}%
\pgfpathlineto{\pgfqpoint{3.203595in}{1.954374in}}%
\pgfpathlineto{\pgfqpoint{3.195617in}{1.946816in}}%
\pgfpathclose%
\pgfusepath{fill}%
\end{pgfscope}%
\begin{pgfscope}%
\pgfpathrectangle{\pgfqpoint{1.254980in}{0.150000in}}{\pgfqpoint{5.490039in}{5.490039in}}%
\pgfusepath{clip}%
\pgfsetbuttcap%
\pgfsetroundjoin%
\definecolor{currentfill}{rgb}{0.277941,0.056324,0.381191}%
\pgfsetfillcolor{currentfill}%
\pgfsetfillopacity{0.700000}%
\pgfsetlinewidth{0.000000pt}%
\definecolor{currentstroke}{rgb}{0.000000,0.000000,0.000000}%
\pgfsetstrokecolor{currentstroke}%
\pgfsetdash{}{0pt}%
\pgfpathmoveto{\pgfqpoint{3.552547in}{1.947206in}}%
\pgfpathlineto{\pgfqpoint{3.565668in}{1.944411in}}%
\pgfpathlineto{\pgfqpoint{3.578794in}{1.941799in}}%
\pgfpathlineto{\pgfqpoint{3.591925in}{1.939366in}}%
\pgfpathlineto{\pgfqpoint{3.605061in}{1.937114in}}%
\pgfpathlineto{\pgfqpoint{3.612872in}{1.946713in}}%
\pgfpathlineto{\pgfqpoint{3.620677in}{1.956327in}}%
\pgfpathlineto{\pgfqpoint{3.628478in}{1.965956in}}%
\pgfpathlineto{\pgfqpoint{3.636273in}{1.975598in}}%
\pgfpathlineto{\pgfqpoint{3.623148in}{1.977661in}}%
\pgfpathlineto{\pgfqpoint{3.610028in}{1.979905in}}%
\pgfpathlineto{\pgfqpoint{3.596914in}{1.982330in}}%
\pgfpathlineto{\pgfqpoint{3.583804in}{1.984936in}}%
\pgfpathlineto{\pgfqpoint{3.575999in}{1.975472in}}%
\pgfpathlineto{\pgfqpoint{3.568187in}{1.966028in}}%
\pgfpathlineto{\pgfqpoint{3.560370in}{1.956605in}}%
\pgfpathlineto{\pgfqpoint{3.552547in}{1.947206in}}%
\pgfpathclose%
\pgfusepath{fill}%
\end{pgfscope}%
\begin{pgfscope}%
\pgfpathrectangle{\pgfqpoint{1.254980in}{0.150000in}}{\pgfqpoint{5.490039in}{5.490039in}}%
\pgfusepath{clip}%
\pgfsetbuttcap%
\pgfsetroundjoin%
\definecolor{currentfill}{rgb}{0.248629,0.278775,0.534556}%
\pgfsetfillcolor{currentfill}%
\pgfsetfillopacity{0.700000}%
\pgfsetlinewidth{0.000000pt}%
\definecolor{currentstroke}{rgb}{0.000000,0.000000,0.000000}%
\pgfsetstrokecolor{currentstroke}%
\pgfsetdash{}{0pt}%
\pgfpathmoveto{\pgfqpoint{4.388292in}{2.381600in}}%
\pgfpathlineto{\pgfqpoint{4.401664in}{2.386276in}}%
\pgfpathlineto{\pgfqpoint{4.415048in}{2.391116in}}%
\pgfpathlineto{\pgfqpoint{4.428444in}{2.396118in}}%
\pgfpathlineto{\pgfqpoint{4.441852in}{2.401284in}}%
\pgfpathlineto{\pgfqpoint{4.449381in}{2.410212in}}%
\pgfpathlineto{\pgfqpoint{4.456904in}{2.419081in}}%
\pgfpathlineto{\pgfqpoint{4.464421in}{2.427891in}}%
\pgfpathlineto{\pgfqpoint{4.471933in}{2.436643in}}%
\pgfpathlineto{\pgfqpoint{4.458532in}{2.431571in}}%
\pgfpathlineto{\pgfqpoint{4.445144in}{2.426662in}}%
\pgfpathlineto{\pgfqpoint{4.431767in}{2.421916in}}%
\pgfpathlineto{\pgfqpoint{4.418402in}{2.417333in}}%
\pgfpathlineto{\pgfqpoint{4.410883in}{2.408476in}}%
\pgfpathlineto{\pgfqpoint{4.403358in}{2.399570in}}%
\pgfpathlineto{\pgfqpoint{4.395828in}{2.390612in}}%
\pgfpathlineto{\pgfqpoint{4.388292in}{2.381600in}}%
\pgfpathclose%
\pgfusepath{fill}%
\end{pgfscope}%
\begin{pgfscope}%
\pgfpathrectangle{\pgfqpoint{1.254980in}{0.150000in}}{\pgfqpoint{5.490039in}{5.490039in}}%
\pgfusepath{clip}%
\pgfsetbuttcap%
\pgfsetroundjoin%
\definecolor{currentfill}{rgb}{0.274952,0.037752,0.364543}%
\pgfsetfillcolor{currentfill}%
\pgfsetfillopacity{0.700000}%
\pgfsetlinewidth{0.000000pt}%
\definecolor{currentstroke}{rgb}{0.000000,0.000000,0.000000}%
\pgfsetstrokecolor{currentstroke}%
\pgfsetdash{}{0pt}%
\pgfpathmoveto{\pgfqpoint{3.332278in}{1.926557in}}%
\pgfpathlineto{\pgfqpoint{3.345386in}{1.921062in}}%
\pgfpathlineto{\pgfqpoint{3.358495in}{1.915758in}}%
\pgfpathlineto{\pgfqpoint{3.371608in}{1.910644in}}%
\pgfpathlineto{\pgfqpoint{3.384723in}{1.905720in}}%
\pgfpathlineto{\pgfqpoint{3.392620in}{1.914360in}}%
\pgfpathlineto{\pgfqpoint{3.400512in}{1.923053in}}%
\pgfpathlineto{\pgfqpoint{3.408397in}{1.931795in}}%
\pgfpathlineto{\pgfqpoint{3.416276in}{1.940585in}}%
\pgfpathlineto{\pgfqpoint{3.403176in}{1.945265in}}%
\pgfpathlineto{\pgfqpoint{3.390079in}{1.950134in}}%
\pgfpathlineto{\pgfqpoint{3.376984in}{1.955194in}}%
\pgfpathlineto{\pgfqpoint{3.363893in}{1.960445in}}%
\pgfpathlineto{\pgfqpoint{3.355999in}{1.951888in}}%
\pgfpathlineto{\pgfqpoint{3.348098in}{1.943387in}}%
\pgfpathlineto{\pgfqpoint{3.340192in}{1.934942in}}%
\pgfpathlineto{\pgfqpoint{3.332278in}{1.926557in}}%
\pgfpathclose%
\pgfusepath{fill}%
\end{pgfscope}%
\begin{pgfscope}%
\pgfpathrectangle{\pgfqpoint{1.254980in}{0.150000in}}{\pgfqpoint{5.490039in}{5.490039in}}%
\pgfusepath{clip}%
\pgfsetbuttcap%
\pgfsetroundjoin%
\definecolor{currentfill}{rgb}{0.171176,0.452530,0.557965}%
\pgfsetfillcolor{currentfill}%
\pgfsetfillopacity{0.700000}%
\pgfsetlinewidth{0.000000pt}%
\definecolor{currentstroke}{rgb}{0.000000,0.000000,0.000000}%
\pgfsetstrokecolor{currentstroke}%
\pgfsetdash{}{0pt}%
\pgfpathmoveto{\pgfqpoint{5.003810in}{2.801520in}}%
\pgfpathlineto{\pgfqpoint{5.017476in}{2.809119in}}%
\pgfpathlineto{\pgfqpoint{5.031156in}{2.816876in}}%
\pgfpathlineto{\pgfqpoint{5.044852in}{2.824789in}}%
\pgfpathlineto{\pgfqpoint{5.058563in}{2.832861in}}%
\pgfpathlineto{\pgfqpoint{5.065823in}{2.838720in}}%
\pgfpathlineto{\pgfqpoint{5.073077in}{2.844551in}}%
\pgfpathlineto{\pgfqpoint{5.080325in}{2.850357in}}%
\pgfpathlineto{\pgfqpoint{5.087566in}{2.856141in}}%
\pgfpathlineto{\pgfqpoint{5.073871in}{2.848398in}}%
\pgfpathlineto{\pgfqpoint{5.060190in}{2.840811in}}%
\pgfpathlineto{\pgfqpoint{5.046525in}{2.833381in}}%
\pgfpathlineto{\pgfqpoint{5.032874in}{2.826107in}}%
\pgfpathlineto{\pgfqpoint{5.025617in}{2.819986in}}%
\pgfpathlineto{\pgfqpoint{5.018354in}{2.813850in}}%
\pgfpathlineto{\pgfqpoint{5.011085in}{2.807696in}}%
\pgfpathlineto{\pgfqpoint{5.003810in}{2.801520in}}%
\pgfpathclose%
\pgfusepath{fill}%
\end{pgfscope}%
\begin{pgfscope}%
\pgfpathrectangle{\pgfqpoint{1.254980in}{0.150000in}}{\pgfqpoint{5.490039in}{5.490039in}}%
\pgfusepath{clip}%
\pgfsetbuttcap%
\pgfsetroundjoin%
\definecolor{currentfill}{rgb}{0.266580,0.228262,0.514349}%
\pgfsetfillcolor{currentfill}%
\pgfsetfillopacity{0.700000}%
\pgfsetlinewidth{0.000000pt}%
\definecolor{currentstroke}{rgb}{0.000000,0.000000,0.000000}%
\pgfsetstrokecolor{currentstroke}%
\pgfsetdash{}{0pt}%
\pgfpathmoveto{\pgfqpoint{2.656409in}{2.321185in}}%
\pgfpathlineto{\pgfqpoint{2.669691in}{2.305059in}}%
\pgfpathlineto{\pgfqpoint{2.682966in}{2.289191in}}%
\pgfpathlineto{\pgfqpoint{2.696233in}{2.273579in}}%
\pgfpathlineto{\pgfqpoint{2.709494in}{2.258220in}}%
\pgfpathlineto{\pgfqpoint{2.717750in}{2.262255in}}%
\pgfpathlineto{\pgfqpoint{2.725996in}{2.266454in}}%
\pgfpathlineto{\pgfqpoint{2.734229in}{2.270812in}}%
\pgfpathlineto{\pgfqpoint{2.742451in}{2.275327in}}%
\pgfpathlineto{\pgfqpoint{2.729222in}{2.290344in}}%
\pgfpathlineto{\pgfqpoint{2.715987in}{2.305614in}}%
\pgfpathlineto{\pgfqpoint{2.702744in}{2.321140in}}%
\pgfpathlineto{\pgfqpoint{2.689494in}{2.336922in}}%
\pgfpathlineto{\pgfqpoint{2.681241in}{2.332739in}}%
\pgfpathlineto{\pgfqpoint{2.672976in}{2.328720in}}%
\pgfpathlineto{\pgfqpoint{2.664698in}{2.324867in}}%
\pgfpathlineto{\pgfqpoint{2.656409in}{2.321185in}}%
\pgfpathclose%
\pgfusepath{fill}%
\end{pgfscope}%
\begin{pgfscope}%
\pgfpathrectangle{\pgfqpoint{1.254980in}{0.150000in}}{\pgfqpoint{5.490039in}{5.490039in}}%
\pgfusepath{clip}%
\pgfsetbuttcap%
\pgfsetroundjoin%
\definecolor{currentfill}{rgb}{0.162142,0.474838,0.558140}%
\pgfsetfillcolor{currentfill}%
\pgfsetfillopacity{0.700000}%
\pgfsetlinewidth{0.000000pt}%
\definecolor{currentstroke}{rgb}{0.000000,0.000000,0.000000}%
\pgfsetstrokecolor{currentstroke}%
\pgfsetdash{}{0pt}%
\pgfpathmoveto{\pgfqpoint{5.087566in}{2.856141in}}%
\pgfpathlineto{\pgfqpoint{5.101277in}{2.864041in}}%
\pgfpathlineto{\pgfqpoint{5.115004in}{2.872098in}}%
\pgfpathlineto{\pgfqpoint{5.128746in}{2.880312in}}%
\pgfpathlineto{\pgfqpoint{5.142503in}{2.888682in}}%
\pgfpathlineto{\pgfqpoint{5.149723in}{2.894102in}}%
\pgfpathlineto{\pgfqpoint{5.156936in}{2.899502in}}%
\pgfpathlineto{\pgfqpoint{5.164142in}{2.904886in}}%
\pgfpathlineto{\pgfqpoint{5.171343in}{2.910257in}}%
\pgfpathlineto{\pgfqpoint{5.157602in}{2.902244in}}%
\pgfpathlineto{\pgfqpoint{5.143877in}{2.894387in}}%
\pgfpathlineto{\pgfqpoint{5.130167in}{2.886686in}}%
\pgfpathlineto{\pgfqpoint{5.116472in}{2.879142in}}%
\pgfpathlineto{\pgfqpoint{5.109255in}{2.873404in}}%
\pgfpathlineto{\pgfqpoint{5.102031in}{2.867661in}}%
\pgfpathlineto{\pgfqpoint{5.094802in}{2.861908in}}%
\pgfpathlineto{\pgfqpoint{5.087566in}{2.856141in}}%
\pgfpathclose%
\pgfusepath{fill}%
\end{pgfscope}%
\begin{pgfscope}%
\pgfpathrectangle{\pgfqpoint{1.254980in}{0.150000in}}{\pgfqpoint{5.490039in}{5.490039in}}%
\pgfusepath{clip}%
\pgfsetbuttcap%
\pgfsetroundjoin%
\definecolor{currentfill}{rgb}{0.274128,0.199721,0.498911}%
\pgfsetfillcolor{currentfill}%
\pgfsetfillopacity{0.700000}%
\pgfsetlinewidth{0.000000pt}%
\definecolor{currentstroke}{rgb}{0.000000,0.000000,0.000000}%
\pgfsetstrokecolor{currentstroke}%
\pgfsetdash{}{0pt}%
\pgfpathmoveto{\pgfqpoint{2.709494in}{2.258220in}}%
\pgfpathlineto{\pgfqpoint{2.722747in}{2.243112in}}%
\pgfpathlineto{\pgfqpoint{2.735995in}{2.228254in}}%
\pgfpathlineto{\pgfqpoint{2.749236in}{2.213643in}}%
\pgfpathlineto{\pgfqpoint{2.762471in}{2.199277in}}%
\pgfpathlineto{\pgfqpoint{2.770696in}{2.203664in}}%
\pgfpathlineto{\pgfqpoint{2.778910in}{2.208207in}}%
\pgfpathlineto{\pgfqpoint{2.787113in}{2.212901in}}%
\pgfpathlineto{\pgfqpoint{2.795306in}{2.217745in}}%
\pgfpathlineto{\pgfqpoint{2.782101in}{2.231771in}}%
\pgfpathlineto{\pgfqpoint{2.768890in}{2.246042in}}%
\pgfpathlineto{\pgfqpoint{2.755674in}{2.260560in}}%
\pgfpathlineto{\pgfqpoint{2.742451in}{2.275327in}}%
\pgfpathlineto{\pgfqpoint{2.734229in}{2.270812in}}%
\pgfpathlineto{\pgfqpoint{2.725996in}{2.266454in}}%
\pgfpathlineto{\pgfqpoint{2.717750in}{2.262255in}}%
\pgfpathlineto{\pgfqpoint{2.709494in}{2.258220in}}%
\pgfpathclose%
\pgfusepath{fill}%
\end{pgfscope}%
\begin{pgfscope}%
\pgfpathrectangle{\pgfqpoint{1.254980in}{0.150000in}}{\pgfqpoint{5.490039in}{5.490039in}}%
\pgfusepath{clip}%
\pgfsetbuttcap%
\pgfsetroundjoin%
\definecolor{currentfill}{rgb}{0.237441,0.305202,0.541921}%
\pgfsetfillcolor{currentfill}%
\pgfsetfillopacity{0.700000}%
\pgfsetlinewidth{0.000000pt}%
\definecolor{currentstroke}{rgb}{0.000000,0.000000,0.000000}%
\pgfsetstrokecolor{currentstroke}%
\pgfsetdash{}{0pt}%
\pgfpathmoveto{\pgfqpoint{4.471933in}{2.436643in}}%
\pgfpathlineto{\pgfqpoint{4.485345in}{2.441877in}}%
\pgfpathlineto{\pgfqpoint{4.498770in}{2.447274in}}%
\pgfpathlineto{\pgfqpoint{4.512206in}{2.452833in}}%
\pgfpathlineto{\pgfqpoint{4.525656in}{2.458553in}}%
\pgfpathlineto{\pgfqpoint{4.533154in}{2.467139in}}%
\pgfpathlineto{\pgfqpoint{4.540647in}{2.475663in}}%
\pgfpathlineto{\pgfqpoint{4.548134in}{2.484127in}}%
\pgfpathlineto{\pgfqpoint{4.555615in}{2.492533in}}%
\pgfpathlineto{\pgfqpoint{4.542174in}{2.486935in}}%
\pgfpathlineto{\pgfqpoint{4.528745in}{2.481499in}}%
\pgfpathlineto{\pgfqpoint{4.515328in}{2.476225in}}%
\pgfpathlineto{\pgfqpoint{4.501923in}{2.471112in}}%
\pgfpathlineto{\pgfqpoint{4.494434in}{2.462573in}}%
\pgfpathlineto{\pgfqpoint{4.486939in}{2.453983in}}%
\pgfpathlineto{\pgfqpoint{4.479439in}{2.445340in}}%
\pgfpathlineto{\pgfqpoint{4.471933in}{2.436643in}}%
\pgfpathclose%
\pgfusepath{fill}%
\end{pgfscope}%
\begin{pgfscope}%
\pgfpathrectangle{\pgfqpoint{1.254980in}{0.150000in}}{\pgfqpoint{5.490039in}{5.490039in}}%
\pgfusepath{clip}%
\pgfsetbuttcap%
\pgfsetroundjoin%
\definecolor{currentfill}{rgb}{0.255645,0.260703,0.528312}%
\pgfsetfillcolor{currentfill}%
\pgfsetfillopacity{0.700000}%
\pgfsetlinewidth{0.000000pt}%
\definecolor{currentstroke}{rgb}{0.000000,0.000000,0.000000}%
\pgfsetstrokecolor{currentstroke}%
\pgfsetdash{}{0pt}%
\pgfpathmoveto{\pgfqpoint{2.603199in}{2.388310in}}%
\pgfpathlineto{\pgfqpoint{2.616514in}{2.371131in}}%
\pgfpathlineto{\pgfqpoint{2.629821in}{2.354219in}}%
\pgfpathlineto{\pgfqpoint{2.643119in}{2.337571in}}%
\pgfpathlineto{\pgfqpoint{2.656409in}{2.321185in}}%
\pgfpathlineto{\pgfqpoint{2.664698in}{2.324867in}}%
\pgfpathlineto{\pgfqpoint{2.672976in}{2.328720in}}%
\pgfpathlineto{\pgfqpoint{2.681241in}{2.332739in}}%
\pgfpathlineto{\pgfqpoint{2.689494in}{2.336922in}}%
\pgfpathlineto{\pgfqpoint{2.676237in}{2.352964in}}%
\pgfpathlineto{\pgfqpoint{2.662973in}{2.369268in}}%
\pgfpathlineto{\pgfqpoint{2.649700in}{2.385835in}}%
\pgfpathlineto{\pgfqpoint{2.636420in}{2.402669in}}%
\pgfpathlineto{\pgfqpoint{2.628133in}{2.398819in}}%
\pgfpathlineto{\pgfqpoint{2.619835in}{2.395141in}}%
\pgfpathlineto{\pgfqpoint{2.611523in}{2.391637in}}%
\pgfpathlineto{\pgfqpoint{2.603199in}{2.388310in}}%
\pgfpathclose%
\pgfusepath{fill}%
\end{pgfscope}%
\begin{pgfscope}%
\pgfpathrectangle{\pgfqpoint{1.254980in}{0.150000in}}{\pgfqpoint{5.490039in}{5.490039in}}%
\pgfusepath{clip}%
\pgfsetbuttcap%
\pgfsetroundjoin%
\definecolor{currentfill}{rgb}{0.278826,0.175490,0.483397}%
\pgfsetfillcolor{currentfill}%
\pgfsetfillopacity{0.700000}%
\pgfsetlinewidth{0.000000pt}%
\definecolor{currentstroke}{rgb}{0.000000,0.000000,0.000000}%
\pgfsetstrokecolor{currentstroke}%
\pgfsetdash{}{0pt}%
\pgfpathmoveto{\pgfqpoint{2.762471in}{2.199277in}}%
\pgfpathlineto{\pgfqpoint{2.775700in}{2.185155in}}%
\pgfpathlineto{\pgfqpoint{2.788924in}{2.171274in}}%
\pgfpathlineto{\pgfqpoint{2.802143in}{2.157633in}}%
\pgfpathlineto{\pgfqpoint{2.815356in}{2.144230in}}%
\pgfpathlineto{\pgfqpoint{2.823551in}{2.148966in}}%
\pgfpathlineto{\pgfqpoint{2.831736in}{2.153851in}}%
\pgfpathlineto{\pgfqpoint{2.839910in}{2.158880in}}%
\pgfpathlineto{\pgfqpoint{2.848073in}{2.164051in}}%
\pgfpathlineto{\pgfqpoint{2.834888in}{2.177117in}}%
\pgfpathlineto{\pgfqpoint{2.821699in}{2.190420in}}%
\pgfpathlineto{\pgfqpoint{2.808505in}{2.203962in}}%
\pgfpathlineto{\pgfqpoint{2.795306in}{2.217745in}}%
\pgfpathlineto{\pgfqpoint{2.787113in}{2.212901in}}%
\pgfpathlineto{\pgfqpoint{2.778910in}{2.208207in}}%
\pgfpathlineto{\pgfqpoint{2.770696in}{2.203664in}}%
\pgfpathlineto{\pgfqpoint{2.762471in}{2.199277in}}%
\pgfpathclose%
\pgfusepath{fill}%
\end{pgfscope}%
\begin{pgfscope}%
\pgfpathrectangle{\pgfqpoint{1.254980in}{0.150000in}}{\pgfqpoint{5.490039in}{5.490039in}}%
\pgfusepath{clip}%
\pgfsetbuttcap%
\pgfsetroundjoin%
\definecolor{currentfill}{rgb}{0.279566,0.067836,0.391917}%
\pgfsetfillcolor{currentfill}%
\pgfsetfillopacity{0.700000}%
\pgfsetlinewidth{0.000000pt}%
\definecolor{currentstroke}{rgb}{0.000000,0.000000,0.000000}%
\pgfsetstrokecolor{currentstroke}%
\pgfsetdash{}{0pt}%
\pgfpathmoveto{\pgfqpoint{3.058557in}{1.985908in}}%
\pgfpathlineto{\pgfqpoint{3.071693in}{1.976627in}}%
\pgfpathlineto{\pgfqpoint{3.084829in}{1.967556in}}%
\pgfpathlineto{\pgfqpoint{3.097963in}{1.958693in}}%
\pgfpathlineto{\pgfqpoint{3.111097in}{1.950037in}}%
\pgfpathlineto{\pgfqpoint{3.119127in}{1.956926in}}%
\pgfpathlineto{\pgfqpoint{3.127148in}{1.963916in}}%
\pgfpathlineto{\pgfqpoint{3.135162in}{1.971003in}}%
\pgfpathlineto{\pgfqpoint{3.143167in}{1.978184in}}%
\pgfpathlineto{\pgfqpoint{3.130054in}{1.986538in}}%
\pgfpathlineto{\pgfqpoint{3.116941in}{1.995098in}}%
\pgfpathlineto{\pgfqpoint{3.103828in}{2.003866in}}%
\pgfpathlineto{\pgfqpoint{3.090713in}{2.012843in}}%
\pgfpathlineto{\pgfqpoint{3.082687in}{2.005954in}}%
\pgfpathlineto{\pgfqpoint{3.074652in}{1.999166in}}%
\pgfpathlineto{\pgfqpoint{3.066609in}{1.992483in}}%
\pgfpathlineto{\pgfqpoint{3.058557in}{1.985908in}}%
\pgfpathclose%
\pgfusepath{fill}%
\end{pgfscope}%
\begin{pgfscope}%
\pgfpathrectangle{\pgfqpoint{1.254980in}{0.150000in}}{\pgfqpoint{5.490039in}{5.490039in}}%
\pgfusepath{clip}%
\pgfsetbuttcap%
\pgfsetroundjoin%
\definecolor{currentfill}{rgb}{0.276022,0.044167,0.370164}%
\pgfsetfillcolor{currentfill}%
\pgfsetfillopacity{0.700000}%
\pgfsetlinewidth{0.000000pt}%
\definecolor{currentstroke}{rgb}{0.000000,0.000000,0.000000}%
\pgfsetstrokecolor{currentstroke}%
\pgfsetdash{}{0pt}%
\pgfpathmoveto{\pgfqpoint{3.468708in}{1.923746in}}%
\pgfpathlineto{\pgfqpoint{3.481825in}{1.920002in}}%
\pgfpathlineto{\pgfqpoint{3.494946in}{1.916442in}}%
\pgfpathlineto{\pgfqpoint{3.508071in}{1.913066in}}%
\pgfpathlineto{\pgfqpoint{3.521200in}{1.909874in}}%
\pgfpathlineto{\pgfqpoint{3.529045in}{1.919163in}}%
\pgfpathlineto{\pgfqpoint{3.536885in}{1.928483in}}%
\pgfpathlineto{\pgfqpoint{3.544719in}{1.937831in}}%
\pgfpathlineto{\pgfqpoint{3.552547in}{1.947206in}}%
\pgfpathlineto{\pgfqpoint{3.539431in}{1.950183in}}%
\pgfpathlineto{\pgfqpoint{3.526318in}{1.953342in}}%
\pgfpathlineto{\pgfqpoint{3.513211in}{1.956686in}}%
\pgfpathlineto{\pgfqpoint{3.500107in}{1.960214in}}%
\pgfpathlineto{\pgfqpoint{3.492266in}{1.951045in}}%
\pgfpathlineto{\pgfqpoint{3.484419in}{1.941910in}}%
\pgfpathlineto{\pgfqpoint{3.476566in}{1.932809in}}%
\pgfpathlineto{\pgfqpoint{3.468708in}{1.923746in}}%
\pgfpathclose%
\pgfusepath{fill}%
\end{pgfscope}%
\begin{pgfscope}%
\pgfpathrectangle{\pgfqpoint{1.254980in}{0.150000in}}{\pgfqpoint{5.490039in}{5.490039in}}%
\pgfusepath{clip}%
\pgfsetbuttcap%
\pgfsetroundjoin%
\definecolor{currentfill}{rgb}{0.153364,0.497000,0.557724}%
\pgfsetfillcolor{currentfill}%
\pgfsetfillopacity{0.700000}%
\pgfsetlinewidth{0.000000pt}%
\definecolor{currentstroke}{rgb}{0.000000,0.000000,0.000000}%
\pgfsetstrokecolor{currentstroke}%
\pgfsetdash{}{0pt}%
\pgfpathmoveto{\pgfqpoint{5.171343in}{2.910257in}}%
\pgfpathlineto{\pgfqpoint{5.185100in}{2.918425in}}%
\pgfpathlineto{\pgfqpoint{5.198872in}{2.926750in}}%
\pgfpathlineto{\pgfqpoint{5.212660in}{2.935231in}}%
\pgfpathlineto{\pgfqpoint{5.226465in}{2.943869in}}%
\pgfpathlineto{\pgfqpoint{5.233642in}{2.948855in}}%
\pgfpathlineto{\pgfqpoint{5.240813in}{2.953831in}}%
\pgfpathlineto{\pgfqpoint{5.247977in}{2.958801in}}%
\pgfpathlineto{\pgfqpoint{5.255136in}{2.963768in}}%
\pgfpathlineto{\pgfqpoint{5.241350in}{2.955519in}}%
\pgfpathlineto{\pgfqpoint{5.227580in}{2.947424in}}%
\pgfpathlineto{\pgfqpoint{5.213825in}{2.939485in}}%
\pgfpathlineto{\pgfqpoint{5.200087in}{2.931702in}}%
\pgfpathlineto{\pgfqpoint{5.192909in}{2.926338in}}%
\pgfpathlineto{\pgfqpoint{5.185726in}{2.920978in}}%
\pgfpathlineto{\pgfqpoint{5.178538in}{2.915619in}}%
\pgfpathlineto{\pgfqpoint{5.171343in}{2.910257in}}%
\pgfpathclose%
\pgfusepath{fill}%
\end{pgfscope}%
\begin{pgfscope}%
\pgfpathrectangle{\pgfqpoint{1.254980in}{0.150000in}}{\pgfqpoint{5.490039in}{5.490039in}}%
\pgfusepath{clip}%
\pgfsetbuttcap%
\pgfsetroundjoin%
\definecolor{currentfill}{rgb}{0.243113,0.292092,0.538516}%
\pgfsetfillcolor{currentfill}%
\pgfsetfillopacity{0.700000}%
\pgfsetlinewidth{0.000000pt}%
\definecolor{currentstroke}{rgb}{0.000000,0.000000,0.000000}%
\pgfsetstrokecolor{currentstroke}%
\pgfsetdash{}{0pt}%
\pgfpathmoveto{\pgfqpoint{2.549850in}{2.459744in}}%
\pgfpathlineto{\pgfqpoint{2.563201in}{2.441473in}}%
\pgfpathlineto{\pgfqpoint{2.576543in}{2.423479in}}%
\pgfpathlineto{\pgfqpoint{2.589876in}{2.405759in}}%
\pgfpathlineto{\pgfqpoint{2.603199in}{2.388310in}}%
\pgfpathlineto{\pgfqpoint{2.611523in}{2.391637in}}%
\pgfpathlineto{\pgfqpoint{2.619835in}{2.395141in}}%
\pgfpathlineto{\pgfqpoint{2.628133in}{2.398819in}}%
\pgfpathlineto{\pgfqpoint{2.636420in}{2.402669in}}%
\pgfpathlineto{\pgfqpoint{2.623130in}{2.419771in}}%
\pgfpathlineto{\pgfqpoint{2.609832in}{2.437144in}}%
\pgfpathlineto{\pgfqpoint{2.596526in}{2.454790in}}%
\pgfpathlineto{\pgfqpoint{2.583209in}{2.472713in}}%
\pgfpathlineto{\pgfqpoint{2.574889in}{2.469199in}}%
\pgfpathlineto{\pgfqpoint{2.566556in}{2.465864in}}%
\pgfpathlineto{\pgfqpoint{2.558209in}{2.462711in}}%
\pgfpathlineto{\pgfqpoint{2.549850in}{2.459744in}}%
\pgfpathclose%
\pgfusepath{fill}%
\end{pgfscope}%
\begin{pgfscope}%
\pgfpathrectangle{\pgfqpoint{1.254980in}{0.150000in}}{\pgfqpoint{5.490039in}{5.490039in}}%
\pgfusepath{clip}%
\pgfsetbuttcap%
\pgfsetroundjoin%
\definecolor{currentfill}{rgb}{0.225863,0.330805,0.547314}%
\pgfsetfillcolor{currentfill}%
\pgfsetfillopacity{0.700000}%
\pgfsetlinewidth{0.000000pt}%
\definecolor{currentstroke}{rgb}{0.000000,0.000000,0.000000}%
\pgfsetstrokecolor{currentstroke}%
\pgfsetdash{}{0pt}%
\pgfpathmoveto{\pgfqpoint{4.555615in}{2.492533in}}%
\pgfpathlineto{\pgfqpoint{4.569069in}{2.498293in}}%
\pgfpathlineto{\pgfqpoint{4.582536in}{2.504214in}}%
\pgfpathlineto{\pgfqpoint{4.596015in}{2.510296in}}%
\pgfpathlineto{\pgfqpoint{4.609508in}{2.516539in}}%
\pgfpathlineto{\pgfqpoint{4.616975in}{2.524749in}}%
\pgfpathlineto{\pgfqpoint{4.624436in}{2.532897in}}%
\pgfpathlineto{\pgfqpoint{4.631891in}{2.540987in}}%
\pgfpathlineto{\pgfqpoint{4.639340in}{2.549018in}}%
\pgfpathlineto{\pgfqpoint{4.625856in}{2.542927in}}%
\pgfpathlineto{\pgfqpoint{4.612385in}{2.536996in}}%
\pgfpathlineto{\pgfqpoint{4.598927in}{2.531227in}}%
\pgfpathlineto{\pgfqpoint{4.585482in}{2.525618in}}%
\pgfpathlineto{\pgfqpoint{4.578024in}{2.517424in}}%
\pgfpathlineto{\pgfqpoint{4.570560in}{2.509180in}}%
\pgfpathlineto{\pgfqpoint{4.563090in}{2.500884in}}%
\pgfpathlineto{\pgfqpoint{4.555615in}{2.492533in}}%
\pgfpathclose%
\pgfusepath{fill}%
\end{pgfscope}%
\begin{pgfscope}%
\pgfpathrectangle{\pgfqpoint{1.254980in}{0.150000in}}{\pgfqpoint{5.490039in}{5.490039in}}%
\pgfusepath{clip}%
\pgfsetbuttcap%
\pgfsetroundjoin%
\definecolor{currentfill}{rgb}{0.282290,0.145912,0.461510}%
\pgfsetfillcolor{currentfill}%
\pgfsetfillopacity{0.700000}%
\pgfsetlinewidth{0.000000pt}%
\definecolor{currentstroke}{rgb}{0.000000,0.000000,0.000000}%
\pgfsetstrokecolor{currentstroke}%
\pgfsetdash{}{0pt}%
\pgfpathmoveto{\pgfqpoint{2.815356in}{2.144230in}}%
\pgfpathlineto{\pgfqpoint{2.828565in}{2.131063in}}%
\pgfpathlineto{\pgfqpoint{2.841769in}{2.118130in}}%
\pgfpathlineto{\pgfqpoint{2.854968in}{2.105429in}}%
\pgfpathlineto{\pgfqpoint{2.868164in}{2.092959in}}%
\pgfpathlineto{\pgfqpoint{2.876330in}{2.098043in}}%
\pgfpathlineto{\pgfqpoint{2.884486in}{2.103267in}}%
\pgfpathlineto{\pgfqpoint{2.892632in}{2.108629in}}%
\pgfpathlineto{\pgfqpoint{2.900768in}{2.114126in}}%
\pgfpathlineto{\pgfqpoint{2.887600in}{2.126260in}}%
\pgfpathlineto{\pgfqpoint{2.874429in}{2.138624in}}%
\pgfpathlineto{\pgfqpoint{2.861253in}{2.151221in}}%
\pgfpathlineto{\pgfqpoint{2.848073in}{2.164051in}}%
\pgfpathlineto{\pgfqpoint{2.839910in}{2.158880in}}%
\pgfpathlineto{\pgfqpoint{2.831736in}{2.153851in}}%
\pgfpathlineto{\pgfqpoint{2.823551in}{2.148966in}}%
\pgfpathlineto{\pgfqpoint{2.815356in}{2.144230in}}%
\pgfpathclose%
\pgfusepath{fill}%
\end{pgfscope}%
\begin{pgfscope}%
\pgfpathrectangle{\pgfqpoint{1.254980in}{0.150000in}}{\pgfqpoint{5.490039in}{5.490039in}}%
\pgfusepath{clip}%
\pgfsetbuttcap%
\pgfsetroundjoin%
\definecolor{currentfill}{rgb}{0.146180,0.515413,0.556823}%
\pgfsetfillcolor{currentfill}%
\pgfsetfillopacity{0.700000}%
\pgfsetlinewidth{0.000000pt}%
\definecolor{currentstroke}{rgb}{0.000000,0.000000,0.000000}%
\pgfsetstrokecolor{currentstroke}%
\pgfsetdash{}{0pt}%
\pgfpathmoveto{\pgfqpoint{5.255136in}{2.963768in}}%
\pgfpathlineto{\pgfqpoint{5.268938in}{2.972174in}}%
\pgfpathlineto{\pgfqpoint{5.282756in}{2.980735in}}%
\pgfpathlineto{\pgfqpoint{5.296591in}{2.989451in}}%
\pgfpathlineto{\pgfqpoint{5.310442in}{2.998323in}}%
\pgfpathlineto{\pgfqpoint{5.317576in}{3.002887in}}%
\pgfpathlineto{\pgfqpoint{5.324703in}{3.007451in}}%
\pgfpathlineto{\pgfqpoint{5.331824in}{3.012020in}}%
\pgfpathlineto{\pgfqpoint{5.338940in}{3.016599in}}%
\pgfpathlineto{\pgfqpoint{5.325109in}{3.008144in}}%
\pgfpathlineto{\pgfqpoint{5.311294in}{2.999844in}}%
\pgfpathlineto{\pgfqpoint{5.297496in}{2.991699in}}%
\pgfpathlineto{\pgfqpoint{5.283713in}{2.983708in}}%
\pgfpathlineto{\pgfqpoint{5.276577in}{2.978704in}}%
\pgfpathlineto{\pgfqpoint{5.269436in}{2.973715in}}%
\pgfpathlineto{\pgfqpoint{5.262289in}{2.968738in}}%
\pgfpathlineto{\pgfqpoint{5.255136in}{2.963768in}}%
\pgfpathclose%
\pgfusepath{fill}%
\end{pgfscope}%
\begin{pgfscope}%
\pgfpathrectangle{\pgfqpoint{1.254980in}{0.150000in}}{\pgfqpoint{5.490039in}{5.490039in}}%
\pgfusepath{clip}%
\pgfsetbuttcap%
\pgfsetroundjoin%
\definecolor{currentfill}{rgb}{0.227802,0.326594,0.546532}%
\pgfsetfillcolor{currentfill}%
\pgfsetfillopacity{0.700000}%
\pgfsetlinewidth{0.000000pt}%
\definecolor{currentstroke}{rgb}{0.000000,0.000000,0.000000}%
\pgfsetstrokecolor{currentstroke}%
\pgfsetdash{}{0pt}%
\pgfpathmoveto{\pgfqpoint{2.496341in}{2.535643in}}%
\pgfpathlineto{\pgfqpoint{2.509734in}{2.516241in}}%
\pgfpathlineto{\pgfqpoint{2.523116in}{2.497125in}}%
\pgfpathlineto{\pgfqpoint{2.536488in}{2.478293in}}%
\pgfpathlineto{\pgfqpoint{2.549850in}{2.459744in}}%
\pgfpathlineto{\pgfqpoint{2.558209in}{2.462711in}}%
\pgfpathlineto{\pgfqpoint{2.566556in}{2.465864in}}%
\pgfpathlineto{\pgfqpoint{2.574889in}{2.469199in}}%
\pgfpathlineto{\pgfqpoint{2.583209in}{2.472713in}}%
\pgfpathlineto{\pgfqpoint{2.569884in}{2.490913in}}%
\pgfpathlineto{\pgfqpoint{2.556548in}{2.509395in}}%
\pgfpathlineto{\pgfqpoint{2.543202in}{2.528160in}}%
\pgfpathlineto{\pgfqpoint{2.529846in}{2.547212in}}%
\pgfpathlineto{\pgfqpoint{2.521490in}{2.544037in}}%
\pgfpathlineto{\pgfqpoint{2.513121in}{2.541049in}}%
\pgfpathlineto{\pgfqpoint{2.504738in}{2.538250in}}%
\pgfpathlineto{\pgfqpoint{2.496341in}{2.535643in}}%
\pgfpathclose%
\pgfusepath{fill}%
\end{pgfscope}%
\begin{pgfscope}%
\pgfpathrectangle{\pgfqpoint{1.254980in}{0.150000in}}{\pgfqpoint{5.490039in}{5.490039in}}%
\pgfusepath{clip}%
\pgfsetbuttcap%
\pgfsetroundjoin%
\definecolor{currentfill}{rgb}{0.283229,0.120777,0.440584}%
\pgfsetfillcolor{currentfill}%
\pgfsetfillopacity{0.700000}%
\pgfsetlinewidth{0.000000pt}%
\definecolor{currentstroke}{rgb}{0.000000,0.000000,0.000000}%
\pgfsetstrokecolor{currentstroke}%
\pgfsetdash{}{0pt}%
\pgfpathmoveto{\pgfqpoint{3.856176in}{2.045623in}}%
\pgfpathlineto{\pgfqpoint{3.869366in}{2.046138in}}%
\pgfpathlineto{\pgfqpoint{3.882563in}{2.046825in}}%
\pgfpathlineto{\pgfqpoint{3.895768in}{2.047684in}}%
\pgfpathlineto{\pgfqpoint{3.908982in}{2.048714in}}%
\pgfpathlineto{\pgfqpoint{3.916695in}{2.058879in}}%
\pgfpathlineto{\pgfqpoint{3.924403in}{2.069019in}}%
\pgfpathlineto{\pgfqpoint{3.932106in}{2.079133in}}%
\pgfpathlineto{\pgfqpoint{3.939804in}{2.089219in}}%
\pgfpathlineto{\pgfqpoint{3.926598in}{2.088084in}}%
\pgfpathlineto{\pgfqpoint{3.913401in}{2.087121in}}%
\pgfpathlineto{\pgfqpoint{3.900211in}{2.086329in}}%
\pgfpathlineto{\pgfqpoint{3.887029in}{2.085709in}}%
\pgfpathlineto{\pgfqpoint{3.879323in}{2.075717in}}%
\pgfpathlineto{\pgfqpoint{3.871613in}{2.065705in}}%
\pgfpathlineto{\pgfqpoint{3.863897in}{2.055673in}}%
\pgfpathlineto{\pgfqpoint{3.856176in}{2.045623in}}%
\pgfpathclose%
\pgfusepath{fill}%
\end{pgfscope}%
\begin{pgfscope}%
\pgfpathrectangle{\pgfqpoint{1.254980in}{0.150000in}}{\pgfqpoint{5.490039in}{5.490039in}}%
\pgfusepath{clip}%
\pgfsetbuttcap%
\pgfsetroundjoin%
\definecolor{currentfill}{rgb}{0.282623,0.140926,0.457517}%
\pgfsetfillcolor{currentfill}%
\pgfsetfillopacity{0.700000}%
\pgfsetlinewidth{0.000000pt}%
\definecolor{currentstroke}{rgb}{0.000000,0.000000,0.000000}%
\pgfsetstrokecolor{currentstroke}%
\pgfsetdash{}{0pt}%
\pgfpathmoveto{\pgfqpoint{3.939804in}{2.089219in}}%
\pgfpathlineto{\pgfqpoint{3.953018in}{2.090524in}}%
\pgfpathlineto{\pgfqpoint{3.966240in}{2.092000in}}%
\pgfpathlineto{\pgfqpoint{3.979471in}{2.093645in}}%
\pgfpathlineto{\pgfqpoint{3.992710in}{2.095459in}}%
\pgfpathlineto{\pgfqpoint{4.000396in}{2.105606in}}%
\pgfpathlineto{\pgfqpoint{4.008077in}{2.115717in}}%
\pgfpathlineto{\pgfqpoint{4.015753in}{2.125792in}}%
\pgfpathlineto{\pgfqpoint{4.023425in}{2.135832in}}%
\pgfpathlineto{\pgfqpoint{4.010192in}{2.133940in}}%
\pgfpathlineto{\pgfqpoint{3.996969in}{2.132218in}}%
\pgfpathlineto{\pgfqpoint{3.983754in}{2.130666in}}%
\pgfpathlineto{\pgfqpoint{3.970547in}{2.129283in}}%
\pgfpathlineto{\pgfqpoint{3.962869in}{2.119310in}}%
\pgfpathlineto{\pgfqpoint{3.955185in}{2.109308in}}%
\pgfpathlineto{\pgfqpoint{3.947497in}{2.099278in}}%
\pgfpathlineto{\pgfqpoint{3.939804in}{2.089219in}}%
\pgfpathclose%
\pgfusepath{fill}%
\end{pgfscope}%
\begin{pgfscope}%
\pgfpathrectangle{\pgfqpoint{1.254980in}{0.150000in}}{\pgfqpoint{5.490039in}{5.490039in}}%
\pgfusepath{clip}%
\pgfsetbuttcap%
\pgfsetroundjoin%
\definecolor{currentfill}{rgb}{0.212395,0.359683,0.551710}%
\pgfsetfillcolor{currentfill}%
\pgfsetfillopacity{0.700000}%
\pgfsetlinewidth{0.000000pt}%
\definecolor{currentstroke}{rgb}{0.000000,0.000000,0.000000}%
\pgfsetstrokecolor{currentstroke}%
\pgfsetdash{}{0pt}%
\pgfpathmoveto{\pgfqpoint{4.639340in}{2.549018in}}%
\pgfpathlineto{\pgfqpoint{4.652837in}{2.555271in}}%
\pgfpathlineto{\pgfqpoint{4.666348in}{2.561683in}}%
\pgfpathlineto{\pgfqpoint{4.679871in}{2.568256in}}%
\pgfpathlineto{\pgfqpoint{4.693408in}{2.574990in}}%
\pgfpathlineto{\pgfqpoint{4.700842in}{2.582796in}}%
\pgfpathlineto{\pgfqpoint{4.708270in}{2.590543in}}%
\pgfpathlineto{\pgfqpoint{4.715692in}{2.598231in}}%
\pgfpathlineto{\pgfqpoint{4.723108in}{2.605865in}}%
\pgfpathlineto{\pgfqpoint{4.709580in}{2.599312in}}%
\pgfpathlineto{\pgfqpoint{4.696066in}{2.592920in}}%
\pgfpathlineto{\pgfqpoint{4.682565in}{2.586688in}}%
\pgfpathlineto{\pgfqpoint{4.669078in}{2.580616in}}%
\pgfpathlineto{\pgfqpoint{4.661652in}{2.572792in}}%
\pgfpathlineto{\pgfqpoint{4.654221in}{2.564919in}}%
\pgfpathlineto{\pgfqpoint{4.646783in}{2.556995in}}%
\pgfpathlineto{\pgfqpoint{4.639340in}{2.549018in}}%
\pgfpathclose%
\pgfusepath{fill}%
\end{pgfscope}%
\begin{pgfscope}%
\pgfpathrectangle{\pgfqpoint{1.254980in}{0.150000in}}{\pgfqpoint{5.490039in}{5.490039in}}%
\pgfusepath{clip}%
\pgfsetbuttcap%
\pgfsetroundjoin%
\definecolor{currentfill}{rgb}{0.274952,0.037752,0.364543}%
\pgfsetfillcolor{currentfill}%
\pgfsetfillopacity{0.700000}%
\pgfsetlinewidth{0.000000pt}%
\definecolor{currentstroke}{rgb}{0.000000,0.000000,0.000000}%
\pgfsetstrokecolor{currentstroke}%
\pgfsetdash{}{0pt}%
\pgfpathmoveto{\pgfqpoint{3.248077in}{1.918662in}}%
\pgfpathlineto{\pgfqpoint{3.261194in}{1.912118in}}%
\pgfpathlineto{\pgfqpoint{3.274313in}{1.905769in}}%
\pgfpathlineto{\pgfqpoint{3.287434in}{1.899615in}}%
\pgfpathlineto{\pgfqpoint{3.300556in}{1.893655in}}%
\pgfpathlineto{\pgfqpoint{3.308497in}{1.901780in}}%
\pgfpathlineto{\pgfqpoint{3.316431in}{1.909974in}}%
\pgfpathlineto{\pgfqpoint{3.324358in}{1.918233in}}%
\pgfpathlineto{\pgfqpoint{3.332278in}{1.926557in}}%
\pgfpathlineto{\pgfqpoint{3.319173in}{1.932245in}}%
\pgfpathlineto{\pgfqpoint{3.306069in}{1.938127in}}%
\pgfpathlineto{\pgfqpoint{3.292968in}{1.944203in}}%
\pgfpathlineto{\pgfqpoint{3.279868in}{1.950474in}}%
\pgfpathlineto{\pgfqpoint{3.271931in}{1.942413in}}%
\pgfpathlineto{\pgfqpoint{3.263987in}{1.934422in}}%
\pgfpathlineto{\pgfqpoint{3.256035in}{1.926504in}}%
\pgfpathlineto{\pgfqpoint{3.248077in}{1.918662in}}%
\pgfpathclose%
\pgfusepath{fill}%
\end{pgfscope}%
\begin{pgfscope}%
\pgfpathrectangle{\pgfqpoint{1.254980in}{0.150000in}}{\pgfqpoint{5.490039in}{5.490039in}}%
\pgfusepath{clip}%
\pgfsetbuttcap%
\pgfsetroundjoin%
\definecolor{currentfill}{rgb}{0.137770,0.537492,0.554906}%
\pgfsetfillcolor{currentfill}%
\pgfsetfillopacity{0.700000}%
\pgfsetlinewidth{0.000000pt}%
\definecolor{currentstroke}{rgb}{0.000000,0.000000,0.000000}%
\pgfsetstrokecolor{currentstroke}%
\pgfsetdash{}{0pt}%
\pgfpathmoveto{\pgfqpoint{5.338940in}{3.016599in}}%
\pgfpathlineto{\pgfqpoint{5.352787in}{3.025208in}}%
\pgfpathlineto{\pgfqpoint{5.366651in}{3.033973in}}%
\pgfpathlineto{\pgfqpoint{5.380532in}{3.042893in}}%
\pgfpathlineto{\pgfqpoint{5.394430in}{3.051968in}}%
\pgfpathlineto{\pgfqpoint{5.401518in}{3.056125in}}%
\pgfpathlineto{\pgfqpoint{5.408601in}{3.060294in}}%
\pgfpathlineto{\pgfqpoint{5.415679in}{3.064481in}}%
\pgfpathlineto{\pgfqpoint{5.422750in}{3.068690in}}%
\pgfpathlineto{\pgfqpoint{5.408875in}{3.060062in}}%
\pgfpathlineto{\pgfqpoint{5.395016in}{3.051589in}}%
\pgfpathlineto{\pgfqpoint{5.381174in}{3.043270in}}%
\pgfpathlineto{\pgfqpoint{5.367348in}{3.035105in}}%
\pgfpathlineto{\pgfqpoint{5.360254in}{3.030440in}}%
\pgfpathlineto{\pgfqpoint{5.353155in}{3.025804in}}%
\pgfpathlineto{\pgfqpoint{5.346050in}{3.021192in}}%
\pgfpathlineto{\pgfqpoint{5.338940in}{3.016599in}}%
\pgfpathclose%
\pgfusepath{fill}%
\end{pgfscope}%
\begin{pgfscope}%
\pgfpathrectangle{\pgfqpoint{1.254980in}{0.150000in}}{\pgfqpoint{5.490039in}{5.490039in}}%
\pgfusepath{clip}%
\pgfsetbuttcap%
\pgfsetroundjoin%
\definecolor{currentfill}{rgb}{0.282327,0.094955,0.417331}%
\pgfsetfillcolor{currentfill}%
\pgfsetfillopacity{0.700000}%
\pgfsetlinewidth{0.000000pt}%
\definecolor{currentstroke}{rgb}{0.000000,0.000000,0.000000}%
\pgfsetstrokecolor{currentstroke}%
\pgfsetdash{}{0pt}%
\pgfpathmoveto{\pgfqpoint{3.772524in}{2.005451in}}%
\pgfpathlineto{\pgfqpoint{3.785693in}{2.005140in}}%
\pgfpathlineto{\pgfqpoint{3.798869in}{2.005003in}}%
\pgfpathlineto{\pgfqpoint{3.812053in}{2.005040in}}%
\pgfpathlineto{\pgfqpoint{3.825243in}{2.005250in}}%
\pgfpathlineto{\pgfqpoint{3.832984in}{2.015367in}}%
\pgfpathlineto{\pgfqpoint{3.840720in}{2.025469in}}%
\pgfpathlineto{\pgfqpoint{3.848450in}{2.035554in}}%
\pgfpathlineto{\pgfqpoint{3.856176in}{2.045623in}}%
\pgfpathlineto{\pgfqpoint{3.842994in}{2.045280in}}%
\pgfpathlineto{\pgfqpoint{3.829819in}{2.045110in}}%
\pgfpathlineto{\pgfqpoint{3.816651in}{2.045114in}}%
\pgfpathlineto{\pgfqpoint{3.803490in}{2.045292in}}%
\pgfpathlineto{\pgfqpoint{3.795756in}{2.035346in}}%
\pgfpathlineto{\pgfqpoint{3.788017in}{2.025390in}}%
\pgfpathlineto{\pgfqpoint{3.780273in}{2.015424in}}%
\pgfpathlineto{\pgfqpoint{3.772524in}{2.005451in}}%
\pgfpathclose%
\pgfusepath{fill}%
\end{pgfscope}%
\begin{pgfscope}%
\pgfpathrectangle{\pgfqpoint{1.254980in}{0.150000in}}{\pgfqpoint{5.490039in}{5.490039in}}%
\pgfusepath{clip}%
\pgfsetbuttcap%
\pgfsetroundjoin%
\definecolor{currentfill}{rgb}{0.280255,0.165693,0.476498}%
\pgfsetfillcolor{currentfill}%
\pgfsetfillopacity{0.700000}%
\pgfsetlinewidth{0.000000pt}%
\definecolor{currentstroke}{rgb}{0.000000,0.000000,0.000000}%
\pgfsetstrokecolor{currentstroke}%
\pgfsetdash{}{0pt}%
\pgfpathmoveto{\pgfqpoint{4.023425in}{2.135832in}}%
\pgfpathlineto{\pgfqpoint{4.036666in}{2.137892in}}%
\pgfpathlineto{\pgfqpoint{4.049916in}{2.140121in}}%
\pgfpathlineto{\pgfqpoint{4.063176in}{2.142518in}}%
\pgfpathlineto{\pgfqpoint{4.076444in}{2.145083in}}%
\pgfpathlineto{\pgfqpoint{4.084104in}{2.155146in}}%
\pgfpathlineto{\pgfqpoint{4.091759in}{2.165165in}}%
\pgfpathlineto{\pgfqpoint{4.099409in}{2.175141in}}%
\pgfpathlineto{\pgfqpoint{4.107053in}{2.185074in}}%
\pgfpathlineto{\pgfqpoint{4.093791in}{2.182460in}}%
\pgfpathlineto{\pgfqpoint{4.080538in}{2.180014in}}%
\pgfpathlineto{\pgfqpoint{4.067295in}{2.177736in}}%
\pgfpathlineto{\pgfqpoint{4.054060in}{2.175627in}}%
\pgfpathlineto{\pgfqpoint{4.046409in}{2.165733in}}%
\pgfpathlineto{\pgfqpoint{4.038752in}{2.155802in}}%
\pgfpathlineto{\pgfqpoint{4.031091in}{2.145835in}}%
\pgfpathlineto{\pgfqpoint{4.023425in}{2.135832in}}%
\pgfpathclose%
\pgfusepath{fill}%
\end{pgfscope}%
\begin{pgfscope}%
\pgfpathrectangle{\pgfqpoint{1.254980in}{0.150000in}}{\pgfqpoint{5.490039in}{5.490039in}}%
\pgfusepath{clip}%
\pgfsetbuttcap%
\pgfsetroundjoin%
\definecolor{currentfill}{rgb}{0.283187,0.125848,0.444960}%
\pgfsetfillcolor{currentfill}%
\pgfsetfillopacity{0.700000}%
\pgfsetlinewidth{0.000000pt}%
\definecolor{currentstroke}{rgb}{0.000000,0.000000,0.000000}%
\pgfsetstrokecolor{currentstroke}%
\pgfsetdash{}{0pt}%
\pgfpathmoveto{\pgfqpoint{2.868164in}{2.092959in}}%
\pgfpathlineto{\pgfqpoint{2.881355in}{2.080718in}}%
\pgfpathlineto{\pgfqpoint{2.894543in}{2.068705in}}%
\pgfpathlineto{\pgfqpoint{2.907727in}{2.056917in}}%
\pgfpathlineto{\pgfqpoint{2.920908in}{2.045354in}}%
\pgfpathlineto{\pgfqpoint{2.929047in}{2.050783in}}%
\pgfpathlineto{\pgfqpoint{2.937176in}{2.056346in}}%
\pgfpathlineto{\pgfqpoint{2.945295in}{2.062040in}}%
\pgfpathlineto{\pgfqpoint{2.953405in}{2.067861in}}%
\pgfpathlineto{\pgfqpoint{2.940250in}{2.079090in}}%
\pgfpathlineto{\pgfqpoint{2.927093in}{2.090542in}}%
\pgfpathlineto{\pgfqpoint{2.913932in}{2.102221in}}%
\pgfpathlineto{\pgfqpoint{2.900768in}{2.114126in}}%
\pgfpathlineto{\pgfqpoint{2.892632in}{2.108629in}}%
\pgfpathlineto{\pgfqpoint{2.884486in}{2.103267in}}%
\pgfpathlineto{\pgfqpoint{2.876330in}{2.098043in}}%
\pgfpathlineto{\pgfqpoint{2.868164in}{2.092959in}}%
\pgfpathclose%
\pgfusepath{fill}%
\end{pgfscope}%
\begin{pgfscope}%
\pgfpathrectangle{\pgfqpoint{1.254980in}{0.150000in}}{\pgfqpoint{5.490039in}{5.490039in}}%
\pgfusepath{clip}%
\pgfsetbuttcap%
\pgfsetroundjoin%
\definecolor{currentfill}{rgb}{0.129933,0.559582,0.551864}%
\pgfsetfillcolor{currentfill}%
\pgfsetfillopacity{0.700000}%
\pgfsetlinewidth{0.000000pt}%
\definecolor{currentstroke}{rgb}{0.000000,0.000000,0.000000}%
\pgfsetstrokecolor{currentstroke}%
\pgfsetdash{}{0pt}%
\pgfpathmoveto{\pgfqpoint{5.422750in}{3.068690in}}%
\pgfpathlineto{\pgfqpoint{5.436643in}{3.077472in}}%
\pgfpathlineto{\pgfqpoint{5.450552in}{3.086409in}}%
\pgfpathlineto{\pgfqpoint{5.464478in}{3.095500in}}%
\pgfpathlineto{\pgfqpoint{5.478421in}{3.104745in}}%
\pgfpathlineto{\pgfqpoint{5.485465in}{3.108516in}}%
\pgfpathlineto{\pgfqpoint{5.492503in}{3.112313in}}%
\pgfpathlineto{\pgfqpoint{5.499535in}{3.116141in}}%
\pgfpathlineto{\pgfqpoint{5.506563in}{3.120005in}}%
\pgfpathlineto{\pgfqpoint{5.492643in}{3.111236in}}%
\pgfpathlineto{\pgfqpoint{5.478741in}{3.102621in}}%
\pgfpathlineto{\pgfqpoint{5.464855in}{3.094160in}}%
\pgfpathlineto{\pgfqpoint{5.450986in}{3.085852in}}%
\pgfpathlineto{\pgfqpoint{5.443935in}{3.081503in}}%
\pgfpathlineto{\pgfqpoint{5.436878in}{3.077196in}}%
\pgfpathlineto{\pgfqpoint{5.429817in}{3.072927in}}%
\pgfpathlineto{\pgfqpoint{5.422750in}{3.068690in}}%
\pgfpathclose%
\pgfusepath{fill}%
\end{pgfscope}%
\begin{pgfscope}%
\pgfpathrectangle{\pgfqpoint{1.254980in}{0.150000in}}{\pgfqpoint{5.490039in}{5.490039in}}%
\pgfusepath{clip}%
\pgfsetbuttcap%
\pgfsetroundjoin%
\definecolor{currentfill}{rgb}{0.276194,0.190074,0.493001}%
\pgfsetfillcolor{currentfill}%
\pgfsetfillopacity{0.700000}%
\pgfsetlinewidth{0.000000pt}%
\definecolor{currentstroke}{rgb}{0.000000,0.000000,0.000000}%
\pgfsetstrokecolor{currentstroke}%
\pgfsetdash{}{0pt}%
\pgfpathmoveto{\pgfqpoint{4.107053in}{2.185074in}}%
\pgfpathlineto{\pgfqpoint{4.120325in}{2.187855in}}%
\pgfpathlineto{\pgfqpoint{4.133606in}{2.190803in}}%
\pgfpathlineto{\pgfqpoint{4.146898in}{2.193918in}}%
\pgfpathlineto{\pgfqpoint{4.160199in}{2.197199in}}%
\pgfpathlineto{\pgfqpoint{4.167832in}{2.207119in}}%
\pgfpathlineto{\pgfqpoint{4.175460in}{2.216989in}}%
\pgfpathlineto{\pgfqpoint{4.183084in}{2.226808in}}%
\pgfpathlineto{\pgfqpoint{4.190701in}{2.236578in}}%
\pgfpathlineto{\pgfqpoint{4.177407in}{2.233276in}}%
\pgfpathlineto{\pgfqpoint{4.164122in}{2.230140in}}%
\pgfpathlineto{\pgfqpoint{4.150847in}{2.227172in}}%
\pgfpathlineto{\pgfqpoint{4.137581in}{2.224370in}}%
\pgfpathlineto{\pgfqpoint{4.129957in}{2.214611in}}%
\pgfpathlineto{\pgfqpoint{4.122327in}{2.204808in}}%
\pgfpathlineto{\pgfqpoint{4.114693in}{2.194963in}}%
\pgfpathlineto{\pgfqpoint{4.107053in}{2.185074in}}%
\pgfpathclose%
\pgfusepath{fill}%
\end{pgfscope}%
\begin{pgfscope}%
\pgfpathrectangle{\pgfqpoint{1.254980in}{0.150000in}}{\pgfqpoint{5.490039in}{5.490039in}}%
\pgfusepath{clip}%
\pgfsetbuttcap%
\pgfsetroundjoin%
\definecolor{currentfill}{rgb}{0.280894,0.078907,0.402329}%
\pgfsetfillcolor{currentfill}%
\pgfsetfillopacity{0.700000}%
\pgfsetlinewidth{0.000000pt}%
\definecolor{currentstroke}{rgb}{0.000000,0.000000,0.000000}%
\pgfsetstrokecolor{currentstroke}%
\pgfsetdash{}{0pt}%
\pgfpathmoveto{\pgfqpoint{3.688826in}{1.969131in}}%
\pgfpathlineto{\pgfqpoint{3.701979in}{1.967958in}}%
\pgfpathlineto{\pgfqpoint{3.715139in}{1.966962in}}%
\pgfpathlineto{\pgfqpoint{3.728304in}{1.966141in}}%
\pgfpathlineto{\pgfqpoint{3.741476in}{1.965495in}}%
\pgfpathlineto{\pgfqpoint{3.749245in}{1.975491in}}%
\pgfpathlineto{\pgfqpoint{3.757010in}{1.985483in}}%
\pgfpathlineto{\pgfqpoint{3.764769in}{1.995470in}}%
\pgfpathlineto{\pgfqpoint{3.772524in}{2.005451in}}%
\pgfpathlineto{\pgfqpoint{3.759361in}{2.005936in}}%
\pgfpathlineto{\pgfqpoint{3.746205in}{2.006596in}}%
\pgfpathlineto{\pgfqpoint{3.733055in}{2.007432in}}%
\pgfpathlineto{\pgfqpoint{3.719912in}{2.008444in}}%
\pgfpathlineto{\pgfqpoint{3.712148in}{1.998614in}}%
\pgfpathlineto{\pgfqpoint{3.704380in}{1.988784in}}%
\pgfpathlineto{\pgfqpoint{3.696606in}{1.978956in}}%
\pgfpathlineto{\pgfqpoint{3.688826in}{1.969131in}}%
\pgfpathclose%
\pgfusepath{fill}%
\end{pgfscope}%
\begin{pgfscope}%
\pgfpathrectangle{\pgfqpoint{1.254980in}{0.150000in}}{\pgfqpoint{5.490039in}{5.490039in}}%
\pgfusepath{clip}%
\pgfsetbuttcap%
\pgfsetroundjoin%
\definecolor{currentfill}{rgb}{0.273809,0.031497,0.358853}%
\pgfsetfillcolor{currentfill}%
\pgfsetfillopacity{0.700000}%
\pgfsetlinewidth{0.000000pt}%
\definecolor{currentstroke}{rgb}{0.000000,0.000000,0.000000}%
\pgfsetstrokecolor{currentstroke}%
\pgfsetdash{}{0pt}%
\pgfpathmoveto{\pgfqpoint{3.384723in}{1.905720in}}%
\pgfpathlineto{\pgfqpoint{3.397841in}{1.900985in}}%
\pgfpathlineto{\pgfqpoint{3.410962in}{1.896438in}}%
\pgfpathlineto{\pgfqpoint{3.424086in}{1.892078in}}%
\pgfpathlineto{\pgfqpoint{3.437213in}{1.887905in}}%
\pgfpathlineto{\pgfqpoint{3.445096in}{1.896799in}}%
\pgfpathlineto{\pgfqpoint{3.452973in}{1.905739in}}%
\pgfpathlineto{\pgfqpoint{3.460843in}{1.914722in}}%
\pgfpathlineto{\pgfqpoint{3.468708in}{1.923746in}}%
\pgfpathlineto{\pgfqpoint{3.455595in}{1.927676in}}%
\pgfpathlineto{\pgfqpoint{3.442485in}{1.931792in}}%
\pgfpathlineto{\pgfqpoint{3.429379in}{1.936095in}}%
\pgfpathlineto{\pgfqpoint{3.416276in}{1.940585in}}%
\pgfpathlineto{\pgfqpoint{3.408397in}{1.931795in}}%
\pgfpathlineto{\pgfqpoint{3.400512in}{1.923053in}}%
\pgfpathlineto{\pgfqpoint{3.392620in}{1.914360in}}%
\pgfpathlineto{\pgfqpoint{3.384723in}{1.905720in}}%
\pgfpathclose%
\pgfusepath{fill}%
\end{pgfscope}%
\begin{pgfscope}%
\pgfpathrectangle{\pgfqpoint{1.254980in}{0.150000in}}{\pgfqpoint{5.490039in}{5.490039in}}%
\pgfusepath{clip}%
\pgfsetbuttcap%
\pgfsetroundjoin%
\definecolor{currentfill}{rgb}{0.277941,0.056324,0.381191}%
\pgfsetfillcolor{currentfill}%
\pgfsetfillopacity{0.700000}%
\pgfsetlinewidth{0.000000pt}%
\definecolor{currentstroke}{rgb}{0.000000,0.000000,0.000000}%
\pgfsetstrokecolor{currentstroke}%
\pgfsetdash{}{0pt}%
\pgfpathmoveto{\pgfqpoint{3.111097in}{1.950037in}}%
\pgfpathlineto{\pgfqpoint{3.124231in}{1.941588in}}%
\pgfpathlineto{\pgfqpoint{3.137364in}{1.933342in}}%
\pgfpathlineto{\pgfqpoint{3.150497in}{1.925301in}}%
\pgfpathlineto{\pgfqpoint{3.163631in}{1.917461in}}%
\pgfpathlineto{\pgfqpoint{3.171639in}{1.924662in}}%
\pgfpathlineto{\pgfqpoint{3.179640in}{1.931957in}}%
\pgfpathlineto{\pgfqpoint{3.187632in}{1.939343in}}%
\pgfpathlineto{\pgfqpoint{3.195617in}{1.946816in}}%
\pgfpathlineto{\pgfqpoint{3.182504in}{1.954354in}}%
\pgfpathlineto{\pgfqpoint{3.169392in}{1.962094in}}%
\pgfpathlineto{\pgfqpoint{3.156279in}{1.970037in}}%
\pgfpathlineto{\pgfqpoint{3.143167in}{1.978184in}}%
\pgfpathlineto{\pgfqpoint{3.135162in}{1.971003in}}%
\pgfpathlineto{\pgfqpoint{3.127148in}{1.963916in}}%
\pgfpathlineto{\pgfqpoint{3.119127in}{1.956926in}}%
\pgfpathlineto{\pgfqpoint{3.111097in}{1.950037in}}%
\pgfpathclose%
\pgfusepath{fill}%
\end{pgfscope}%
\begin{pgfscope}%
\pgfpathrectangle{\pgfqpoint{1.254980in}{0.150000in}}{\pgfqpoint{5.490039in}{5.490039in}}%
\pgfusepath{clip}%
\pgfsetbuttcap%
\pgfsetroundjoin%
\definecolor{currentfill}{rgb}{0.201239,0.383670,0.554294}%
\pgfsetfillcolor{currentfill}%
\pgfsetfillopacity{0.700000}%
\pgfsetlinewidth{0.000000pt}%
\definecolor{currentstroke}{rgb}{0.000000,0.000000,0.000000}%
\pgfsetstrokecolor{currentstroke}%
\pgfsetdash{}{0pt}%
\pgfpathmoveto{\pgfqpoint{4.723108in}{2.605865in}}%
\pgfpathlineto{\pgfqpoint{4.736649in}{2.612577in}}%
\pgfpathlineto{\pgfqpoint{4.750204in}{2.619449in}}%
\pgfpathlineto{\pgfqpoint{4.763773in}{2.626480in}}%
\pgfpathlineto{\pgfqpoint{4.777356in}{2.633671in}}%
\pgfpathlineto{\pgfqpoint{4.784756in}{2.641052in}}%
\pgfpathlineto{\pgfqpoint{4.792149in}{2.648375in}}%
\pgfpathlineto{\pgfqpoint{4.799536in}{2.655642in}}%
\pgfpathlineto{\pgfqpoint{4.806917in}{2.662857in}}%
\pgfpathlineto{\pgfqpoint{4.793345in}{2.655877in}}%
\pgfpathlineto{\pgfqpoint{4.779786in}{2.649055in}}%
\pgfpathlineto{\pgfqpoint{4.766242in}{2.642393in}}%
\pgfpathlineto{\pgfqpoint{4.752710in}{2.635891in}}%
\pgfpathlineto{\pgfqpoint{4.745319in}{2.628455in}}%
\pgfpathlineto{\pgfqpoint{4.737921in}{2.620974in}}%
\pgfpathlineto{\pgfqpoint{4.730518in}{2.613445in}}%
\pgfpathlineto{\pgfqpoint{4.723108in}{2.605865in}}%
\pgfpathclose%
\pgfusepath{fill}%
\end{pgfscope}%
\begin{pgfscope}%
\pgfpathrectangle{\pgfqpoint{1.254980in}{0.150000in}}{\pgfqpoint{5.490039in}{5.490039in}}%
\pgfusepath{clip}%
\pgfsetbuttcap%
\pgfsetroundjoin%
\definecolor{currentfill}{rgb}{0.124395,0.578002,0.548287}%
\pgfsetfillcolor{currentfill}%
\pgfsetfillopacity{0.700000}%
\pgfsetlinewidth{0.000000pt}%
\definecolor{currentstroke}{rgb}{0.000000,0.000000,0.000000}%
\pgfsetstrokecolor{currentstroke}%
\pgfsetdash{}{0pt}%
\pgfpathmoveto{\pgfqpoint{5.506563in}{3.120005in}}%
\pgfpathlineto{\pgfqpoint{5.520499in}{3.128928in}}%
\pgfpathlineto{\pgfqpoint{5.534453in}{3.138004in}}%
\pgfpathlineto{\pgfqpoint{5.548424in}{3.147234in}}%
\pgfpathlineto{\pgfqpoint{5.562412in}{3.156619in}}%
\pgfpathlineto{\pgfqpoint{5.569410in}{3.160029in}}%
\pgfpathlineto{\pgfqpoint{5.576402in}{3.163481in}}%
\pgfpathlineto{\pgfqpoint{5.583389in}{3.166978in}}%
\pgfpathlineto{\pgfqpoint{5.590372in}{3.170527in}}%
\pgfpathlineto{\pgfqpoint{5.576410in}{3.161649in}}%
\pgfpathlineto{\pgfqpoint{5.562465in}{3.152925in}}%
\pgfpathlineto{\pgfqpoint{5.548537in}{3.144353in}}%
\pgfpathlineto{\pgfqpoint{5.534625in}{3.135935in}}%
\pgfpathlineto{\pgfqpoint{5.527616in}{3.131870in}}%
\pgfpathlineto{\pgfqpoint{5.520603in}{3.127865in}}%
\pgfpathlineto{\pgfqpoint{5.513585in}{3.123911in}}%
\pgfpathlineto{\pgfqpoint{5.506563in}{3.120005in}}%
\pgfpathclose%
\pgfusepath{fill}%
\end{pgfscope}%
\begin{pgfscope}%
\pgfpathrectangle{\pgfqpoint{1.254980in}{0.150000in}}{\pgfqpoint{5.490039in}{5.490039in}}%
\pgfusepath{clip}%
\pgfsetbuttcap%
\pgfsetroundjoin%
\definecolor{currentfill}{rgb}{0.212395,0.359683,0.551710}%
\pgfsetfillcolor{currentfill}%
\pgfsetfillopacity{0.700000}%
\pgfsetlinewidth{0.000000pt}%
\definecolor{currentstroke}{rgb}{0.000000,0.000000,0.000000}%
\pgfsetstrokecolor{currentstroke}%
\pgfsetdash{}{0pt}%
\pgfpathmoveto{\pgfqpoint{2.442655in}{2.616181in}}%
\pgfpathlineto{\pgfqpoint{2.456094in}{2.595602in}}%
\pgfpathlineto{\pgfqpoint{2.469521in}{2.575321in}}%
\pgfpathlineto{\pgfqpoint{2.482937in}{2.555336in}}%
\pgfpathlineto{\pgfqpoint{2.496341in}{2.535643in}}%
\pgfpathlineto{\pgfqpoint{2.504738in}{2.538250in}}%
\pgfpathlineto{\pgfqpoint{2.513121in}{2.541049in}}%
\pgfpathlineto{\pgfqpoint{2.521490in}{2.544037in}}%
\pgfpathlineto{\pgfqpoint{2.529846in}{2.547212in}}%
\pgfpathlineto{\pgfqpoint{2.516480in}{2.566552in}}%
\pgfpathlineto{\pgfqpoint{2.503102in}{2.586185in}}%
\pgfpathlineto{\pgfqpoint{2.489713in}{2.606112in}}%
\pgfpathlineto{\pgfqpoint{2.476312in}{2.626336in}}%
\pgfpathlineto{\pgfqpoint{2.467919in}{2.623503in}}%
\pgfpathlineto{\pgfqpoint{2.459512in}{2.620864in}}%
\pgfpathlineto{\pgfqpoint{2.451090in}{2.618422in}}%
\pgfpathlineto{\pgfqpoint{2.442655in}{2.616181in}}%
\pgfpathclose%
\pgfusepath{fill}%
\end{pgfscope}%
\begin{pgfscope}%
\pgfpathrectangle{\pgfqpoint{1.254980in}{0.150000in}}{\pgfqpoint{5.490039in}{5.490039in}}%
\pgfusepath{clip}%
\pgfsetbuttcap%
\pgfsetroundjoin%
\definecolor{currentfill}{rgb}{0.269308,0.218818,0.509577}%
\pgfsetfillcolor{currentfill}%
\pgfsetfillopacity{0.700000}%
\pgfsetlinewidth{0.000000pt}%
\definecolor{currentstroke}{rgb}{0.000000,0.000000,0.000000}%
\pgfsetstrokecolor{currentstroke}%
\pgfsetdash{}{0pt}%
\pgfpathmoveto{\pgfqpoint{4.190701in}{2.236578in}}%
\pgfpathlineto{\pgfqpoint{4.204007in}{2.240045in}}%
\pgfpathlineto{\pgfqpoint{4.217322in}{2.243679in}}%
\pgfpathlineto{\pgfqpoint{4.230648in}{2.247478in}}%
\pgfpathlineto{\pgfqpoint{4.243985in}{2.251442in}}%
\pgfpathlineto{\pgfqpoint{4.251591in}{2.261165in}}%
\pgfpathlineto{\pgfqpoint{4.259192in}{2.270831in}}%
\pgfpathlineto{\pgfqpoint{4.266788in}{2.280441in}}%
\pgfpathlineto{\pgfqpoint{4.274379in}{2.289996in}}%
\pgfpathlineto{\pgfqpoint{4.261049in}{2.286040in}}%
\pgfpathlineto{\pgfqpoint{4.247729in}{2.282249in}}%
\pgfpathlineto{\pgfqpoint{4.234420in}{2.278623in}}%
\pgfpathlineto{\pgfqpoint{4.221122in}{2.275163in}}%
\pgfpathlineto{\pgfqpoint{4.213524in}{2.265590in}}%
\pgfpathlineto{\pgfqpoint{4.205922in}{2.255968in}}%
\pgfpathlineto{\pgfqpoint{4.198314in}{2.246297in}}%
\pgfpathlineto{\pgfqpoint{4.190701in}{2.236578in}}%
\pgfpathclose%
\pgfusepath{fill}%
\end{pgfscope}%
\begin{pgfscope}%
\pgfpathrectangle{\pgfqpoint{1.254980in}{0.150000in}}{\pgfqpoint{5.490039in}{5.490039in}}%
\pgfusepath{clip}%
\pgfsetbuttcap%
\pgfsetroundjoin%
\definecolor{currentfill}{rgb}{0.120565,0.596422,0.543611}%
\pgfsetfillcolor{currentfill}%
\pgfsetfillopacity{0.700000}%
\pgfsetlinewidth{0.000000pt}%
\definecolor{currentstroke}{rgb}{0.000000,0.000000,0.000000}%
\pgfsetstrokecolor{currentstroke}%
\pgfsetdash{}{0pt}%
\pgfpathmoveto{\pgfqpoint{5.590372in}{3.170527in}}%
\pgfpathlineto{\pgfqpoint{5.604352in}{3.179558in}}%
\pgfpathlineto{\pgfqpoint{5.618350in}{3.188743in}}%
\pgfpathlineto{\pgfqpoint{5.632365in}{3.198080in}}%
\pgfpathlineto{\pgfqpoint{5.646398in}{3.207572in}}%
\pgfpathlineto{\pgfqpoint{5.653349in}{3.210653in}}%
\pgfpathlineto{\pgfqpoint{5.660295in}{3.213791in}}%
\pgfpathlineto{\pgfqpoint{5.667237in}{3.216991in}}%
\pgfpathlineto{\pgfqpoint{5.674176in}{3.220259in}}%
\pgfpathlineto{\pgfqpoint{5.660171in}{3.211304in}}%
\pgfpathlineto{\pgfqpoint{5.646184in}{3.202502in}}%
\pgfpathlineto{\pgfqpoint{5.632214in}{3.193852in}}%
\pgfpathlineto{\pgfqpoint{5.618262in}{3.185354in}}%
\pgfpathlineto{\pgfqpoint{5.611295in}{3.181541in}}%
\pgfpathlineto{\pgfqpoint{5.604325in}{3.177803in}}%
\pgfpathlineto{\pgfqpoint{5.597351in}{3.174133in}}%
\pgfpathlineto{\pgfqpoint{5.590372in}{3.170527in}}%
\pgfpathclose%
\pgfusepath{fill}%
\end{pgfscope}%
\begin{pgfscope}%
\pgfpathrectangle{\pgfqpoint{1.254980in}{0.150000in}}{\pgfqpoint{5.490039in}{5.490039in}}%
\pgfusepath{clip}%
\pgfsetbuttcap%
\pgfsetroundjoin%
\definecolor{currentfill}{rgb}{0.282910,0.105393,0.426902}%
\pgfsetfillcolor{currentfill}%
\pgfsetfillopacity{0.700000}%
\pgfsetlinewidth{0.000000pt}%
\definecolor{currentstroke}{rgb}{0.000000,0.000000,0.000000}%
\pgfsetstrokecolor{currentstroke}%
\pgfsetdash{}{0pt}%
\pgfpathmoveto{\pgfqpoint{2.920908in}{2.045354in}}%
\pgfpathlineto{\pgfqpoint{2.934086in}{2.034013in}}%
\pgfpathlineto{\pgfqpoint{2.947261in}{2.022894in}}%
\pgfpathlineto{\pgfqpoint{2.960433in}{2.011994in}}%
\pgfpathlineto{\pgfqpoint{2.973603in}{2.001312in}}%
\pgfpathlineto{\pgfqpoint{2.981715in}{2.007085in}}%
\pgfpathlineto{\pgfqpoint{2.989818in}{2.012985in}}%
\pgfpathlineto{\pgfqpoint{2.997912in}{2.019009in}}%
\pgfpathlineto{\pgfqpoint{3.005997in}{2.025153in}}%
\pgfpathlineto{\pgfqpoint{2.992852in}{2.035501in}}%
\pgfpathlineto{\pgfqpoint{2.979705in}{2.046068in}}%
\pgfpathlineto{\pgfqpoint{2.966556in}{2.056854in}}%
\pgfpathlineto{\pgfqpoint{2.953405in}{2.067861in}}%
\pgfpathlineto{\pgfqpoint{2.945295in}{2.062040in}}%
\pgfpathlineto{\pgfqpoint{2.937176in}{2.056346in}}%
\pgfpathlineto{\pgfqpoint{2.929047in}{2.050783in}}%
\pgfpathlineto{\pgfqpoint{2.920908in}{2.045354in}}%
\pgfpathclose%
\pgfusepath{fill}%
\end{pgfscope}%
\begin{pgfscope}%
\pgfpathrectangle{\pgfqpoint{1.254980in}{0.150000in}}{\pgfqpoint{5.490039in}{5.490039in}}%
\pgfusepath{clip}%
\pgfsetbuttcap%
\pgfsetroundjoin%
\definecolor{currentfill}{rgb}{0.278791,0.062145,0.386592}%
\pgfsetfillcolor{currentfill}%
\pgfsetfillopacity{0.700000}%
\pgfsetlinewidth{0.000000pt}%
\definecolor{currentstroke}{rgb}{0.000000,0.000000,0.000000}%
\pgfsetstrokecolor{currentstroke}%
\pgfsetdash{}{0pt}%
\pgfpathmoveto{\pgfqpoint{3.605061in}{1.937114in}}%
\pgfpathlineto{\pgfqpoint{3.618202in}{1.935042in}}%
\pgfpathlineto{\pgfqpoint{3.631348in}{1.933148in}}%
\pgfpathlineto{\pgfqpoint{3.644500in}{1.931433in}}%
\pgfpathlineto{\pgfqpoint{3.657657in}{1.929895in}}%
\pgfpathlineto{\pgfqpoint{3.665457in}{1.939692in}}%
\pgfpathlineto{\pgfqpoint{3.673252in}{1.949498in}}%
\pgfpathlineto{\pgfqpoint{3.681042in}{1.959311in}}%
\pgfpathlineto{\pgfqpoint{3.688826in}{1.969131in}}%
\pgfpathlineto{\pgfqpoint{3.675679in}{1.970481in}}%
\pgfpathlineto{\pgfqpoint{3.662538in}{1.972008in}}%
\pgfpathlineto{\pgfqpoint{3.649403in}{1.973714in}}%
\pgfpathlineto{\pgfqpoint{3.636273in}{1.975598in}}%
\pgfpathlineto{\pgfqpoint{3.628478in}{1.965956in}}%
\pgfpathlineto{\pgfqpoint{3.620677in}{1.956327in}}%
\pgfpathlineto{\pgfqpoint{3.612872in}{1.946713in}}%
\pgfpathlineto{\pgfqpoint{3.605061in}{1.937114in}}%
\pgfpathclose%
\pgfusepath{fill}%
\end{pgfscope}%
\begin{pgfscope}%
\pgfpathrectangle{\pgfqpoint{1.254980in}{0.150000in}}{\pgfqpoint{5.490039in}{5.490039in}}%
\pgfusepath{clip}%
\pgfsetbuttcap%
\pgfsetroundjoin%
\definecolor{currentfill}{rgb}{0.119483,0.614817,0.537692}%
\pgfsetfillcolor{currentfill}%
\pgfsetfillopacity{0.700000}%
\pgfsetlinewidth{0.000000pt}%
\definecolor{currentstroke}{rgb}{0.000000,0.000000,0.000000}%
\pgfsetstrokecolor{currentstroke}%
\pgfsetdash{}{0pt}%
\pgfpathmoveto{\pgfqpoint{5.674176in}{3.220259in}}%
\pgfpathlineto{\pgfqpoint{5.688198in}{3.229367in}}%
\pgfpathlineto{\pgfqpoint{5.702238in}{3.238628in}}%
\pgfpathlineto{\pgfqpoint{5.716296in}{3.248042in}}%
\pgfpathlineto{\pgfqpoint{5.730373in}{3.257608in}}%
\pgfpathlineto{\pgfqpoint{5.737277in}{3.260396in}}%
\pgfpathlineto{\pgfqpoint{5.744178in}{3.263258in}}%
\pgfpathlineto{\pgfqpoint{5.751076in}{3.266199in}}%
\pgfpathlineto{\pgfqpoint{5.757970in}{3.269226in}}%
\pgfpathlineto{\pgfqpoint{5.743924in}{3.260225in}}%
\pgfpathlineto{\pgfqpoint{5.729896in}{3.251377in}}%
\pgfpathlineto{\pgfqpoint{5.715885in}{3.242680in}}%
\pgfpathlineto{\pgfqpoint{5.701892in}{3.234135in}}%
\pgfpathlineto{\pgfqpoint{5.694968in}{3.230534in}}%
\pgfpathlineto{\pgfqpoint{5.688041in}{3.227025in}}%
\pgfpathlineto{\pgfqpoint{5.681110in}{3.223602in}}%
\pgfpathlineto{\pgfqpoint{5.674176in}{3.220259in}}%
\pgfpathclose%
\pgfusepath{fill}%
\end{pgfscope}%
\begin{pgfscope}%
\pgfpathrectangle{\pgfqpoint{1.254980in}{0.150000in}}{\pgfqpoint{5.490039in}{5.490039in}}%
\pgfusepath{clip}%
\pgfsetbuttcap%
\pgfsetroundjoin%
\definecolor{currentfill}{rgb}{0.190631,0.407061,0.556089}%
\pgfsetfillcolor{currentfill}%
\pgfsetfillopacity{0.700000}%
\pgfsetlinewidth{0.000000pt}%
\definecolor{currentstroke}{rgb}{0.000000,0.000000,0.000000}%
\pgfsetstrokecolor{currentstroke}%
\pgfsetdash{}{0pt}%
\pgfpathmoveto{\pgfqpoint{4.806917in}{2.662857in}}%
\pgfpathlineto{\pgfqpoint{4.820504in}{2.669997in}}%
\pgfpathlineto{\pgfqpoint{4.834105in}{2.677296in}}%
\pgfpathlineto{\pgfqpoint{4.847720in}{2.684753in}}%
\pgfpathlineto{\pgfqpoint{4.861349in}{2.692370in}}%
\pgfpathlineto{\pgfqpoint{4.868713in}{2.699306in}}%
\pgfpathlineto{\pgfqpoint{4.876070in}{2.706188in}}%
\pgfpathlineto{\pgfqpoint{4.883421in}{2.713019in}}%
\pgfpathlineto{\pgfqpoint{4.890765in}{2.719802in}}%
\pgfpathlineto{\pgfqpoint{4.877147in}{2.712425in}}%
\pgfpathlineto{\pgfqpoint{4.863543in}{2.705207in}}%
\pgfpathlineto{\pgfqpoint{4.849954in}{2.698148in}}%
\pgfpathlineto{\pgfqpoint{4.836379in}{2.691247in}}%
\pgfpathlineto{\pgfqpoint{4.829023in}{2.684215in}}%
\pgfpathlineto{\pgfqpoint{4.821660in}{2.677141in}}%
\pgfpathlineto{\pgfqpoint{4.814292in}{2.670023in}}%
\pgfpathlineto{\pgfqpoint{4.806917in}{2.662857in}}%
\pgfpathclose%
\pgfusepath{fill}%
\end{pgfscope}%
\begin{pgfscope}%
\pgfpathrectangle{\pgfqpoint{1.254980in}{0.150000in}}{\pgfqpoint{5.490039in}{5.490039in}}%
\pgfusepath{clip}%
\pgfsetbuttcap%
\pgfsetroundjoin%
\definecolor{currentfill}{rgb}{0.260571,0.246922,0.522828}%
\pgfsetfillcolor{currentfill}%
\pgfsetfillopacity{0.700000}%
\pgfsetlinewidth{0.000000pt}%
\definecolor{currentstroke}{rgb}{0.000000,0.000000,0.000000}%
\pgfsetstrokecolor{currentstroke}%
\pgfsetdash{}{0pt}%
\pgfpathmoveto{\pgfqpoint{4.274379in}{2.289996in}}%
\pgfpathlineto{\pgfqpoint{4.287720in}{2.294117in}}%
\pgfpathlineto{\pgfqpoint{4.301072in}{2.298403in}}%
\pgfpathlineto{\pgfqpoint{4.314436in}{2.302853in}}%
\pgfpathlineto{\pgfqpoint{4.327810in}{2.307466in}}%
\pgfpathlineto{\pgfqpoint{4.335389in}{2.316941in}}%
\pgfpathlineto{\pgfqpoint{4.342963in}{2.326354in}}%
\pgfpathlineto{\pgfqpoint{4.350531in}{2.335708in}}%
\pgfpathlineto{\pgfqpoint{4.358094in}{2.345001in}}%
\pgfpathlineto{\pgfqpoint{4.344726in}{2.340424in}}%
\pgfpathlineto{\pgfqpoint{4.331370in}{2.336011in}}%
\pgfpathlineto{\pgfqpoint{4.318024in}{2.331761in}}%
\pgfpathlineto{\pgfqpoint{4.304690in}{2.327677in}}%
\pgfpathlineto{\pgfqpoint{4.297120in}{2.318336in}}%
\pgfpathlineto{\pgfqpoint{4.289545in}{2.308943in}}%
\pgfpathlineto{\pgfqpoint{4.281965in}{2.299497in}}%
\pgfpathlineto{\pgfqpoint{4.274379in}{2.289996in}}%
\pgfpathclose%
\pgfusepath{fill}%
\end{pgfscope}%
\begin{pgfscope}%
\pgfpathrectangle{\pgfqpoint{1.254980in}{0.150000in}}{\pgfqpoint{5.490039in}{5.490039in}}%
\pgfusepath{clip}%
\pgfsetbuttcap%
\pgfsetroundjoin%
\definecolor{currentfill}{rgb}{0.122312,0.633153,0.530398}%
\pgfsetfillcolor{currentfill}%
\pgfsetfillopacity{0.700000}%
\pgfsetlinewidth{0.000000pt}%
\definecolor{currentstroke}{rgb}{0.000000,0.000000,0.000000}%
\pgfsetstrokecolor{currentstroke}%
\pgfsetdash{}{0pt}%
\pgfpathmoveto{\pgfqpoint{5.757970in}{3.269226in}}%
\pgfpathlineto{\pgfqpoint{5.772034in}{3.278379in}}%
\pgfpathlineto{\pgfqpoint{5.786115in}{3.287685in}}%
\pgfpathlineto{\pgfqpoint{5.800216in}{3.297142in}}%
\pgfpathlineto{\pgfqpoint{5.814334in}{3.306753in}}%
\pgfpathlineto{\pgfqpoint{5.821193in}{3.309288in}}%
\pgfpathlineto{\pgfqpoint{5.828049in}{3.311915in}}%
\pgfpathlineto{\pgfqpoint{5.834902in}{3.314641in}}%
\pgfpathlineto{\pgfqpoint{5.841752in}{3.317473in}}%
\pgfpathlineto{\pgfqpoint{5.827666in}{3.308458in}}%
\pgfpathlineto{\pgfqpoint{5.813598in}{3.299594in}}%
\pgfpathlineto{\pgfqpoint{5.799549in}{3.290882in}}%
\pgfpathlineto{\pgfqpoint{5.785516in}{3.282322in}}%
\pgfpathlineto{\pgfqpoint{5.778634in}{3.278887in}}%
\pgfpathlineto{\pgfqpoint{5.771748in}{3.275564in}}%
\pgfpathlineto{\pgfqpoint{5.764860in}{3.272346in}}%
\pgfpathlineto{\pgfqpoint{5.757970in}{3.269226in}}%
\pgfpathclose%
\pgfusepath{fill}%
\end{pgfscope}%
\begin{pgfscope}%
\pgfpathrectangle{\pgfqpoint{1.254980in}{0.150000in}}{\pgfqpoint{5.490039in}{5.490039in}}%
\pgfusepath{clip}%
\pgfsetbuttcap%
\pgfsetroundjoin%
\definecolor{currentfill}{rgb}{0.130067,0.651384,0.521608}%
\pgfsetfillcolor{currentfill}%
\pgfsetfillopacity{0.700000}%
\pgfsetlinewidth{0.000000pt}%
\definecolor{currentstroke}{rgb}{0.000000,0.000000,0.000000}%
\pgfsetstrokecolor{currentstroke}%
\pgfsetdash{}{0pt}%
\pgfpathmoveto{\pgfqpoint{5.841752in}{3.317473in}}%
\pgfpathlineto{\pgfqpoint{5.855856in}{3.326639in}}%
\pgfpathlineto{\pgfqpoint{5.869979in}{3.335957in}}%
\pgfpathlineto{\pgfqpoint{5.884120in}{3.345428in}}%
\pgfpathlineto{\pgfqpoint{5.898279in}{3.355050in}}%
\pgfpathlineto{\pgfqpoint{5.905093in}{3.357379in}}%
\pgfpathlineto{\pgfqpoint{5.911905in}{3.359820in}}%
\pgfpathlineto{\pgfqpoint{5.918714in}{3.362379in}}%
\pgfpathlineto{\pgfqpoint{5.925522in}{3.365064in}}%
\pgfpathlineto{\pgfqpoint{5.911398in}{3.356067in}}%
\pgfpathlineto{\pgfqpoint{5.897291in}{3.347220in}}%
\pgfpathlineto{\pgfqpoint{5.883203in}{3.338524in}}%
\pgfpathlineto{\pgfqpoint{5.869133in}{3.329979in}}%
\pgfpathlineto{\pgfqpoint{5.862290in}{3.326662in}}%
\pgfpathlineto{\pgfqpoint{5.855446in}{3.323476in}}%
\pgfpathlineto{\pgfqpoint{5.848600in}{3.320415in}}%
\pgfpathlineto{\pgfqpoint{5.841752in}{3.317473in}}%
\pgfpathclose%
\pgfusepath{fill}%
\end{pgfscope}%
\begin{pgfscope}%
\pgfpathrectangle{\pgfqpoint{1.254980in}{0.150000in}}{\pgfqpoint{5.490039in}{5.490039in}}%
\pgfusepath{clip}%
\pgfsetbuttcap%
\pgfsetroundjoin%
\definecolor{currentfill}{rgb}{0.143303,0.669459,0.511215}%
\pgfsetfillcolor{currentfill}%
\pgfsetfillopacity{0.700000}%
\pgfsetlinewidth{0.000000pt}%
\definecolor{currentstroke}{rgb}{0.000000,0.000000,0.000000}%
\pgfsetstrokecolor{currentstroke}%
\pgfsetdash{}{0pt}%
\pgfpathmoveto{\pgfqpoint{5.925522in}{3.365064in}}%
\pgfpathlineto{\pgfqpoint{5.939665in}{3.374213in}}%
\pgfpathlineto{\pgfqpoint{5.953827in}{3.383513in}}%
\pgfpathlineto{\pgfqpoint{5.968007in}{3.392964in}}%
\pgfpathlineto{\pgfqpoint{5.982206in}{3.402567in}}%
\pgfpathlineto{\pgfqpoint{5.988976in}{3.404741in}}%
\pgfpathlineto{\pgfqpoint{5.995745in}{3.407047in}}%
\pgfpathlineto{\pgfqpoint{6.002513in}{3.409494in}}%
\pgfpathlineto{\pgfqpoint{6.009280in}{3.412088in}}%
\pgfpathlineto{\pgfqpoint{5.995118in}{3.403139in}}%
\pgfpathlineto{\pgfqpoint{5.980974in}{3.394340in}}%
\pgfpathlineto{\pgfqpoint{5.966849in}{3.385692in}}%
\pgfpathlineto{\pgfqpoint{5.952743in}{3.377194in}}%
\pgfpathlineto{\pgfqpoint{5.945939in}{3.373939in}}%
\pgfpathlineto{\pgfqpoint{5.939134in}{3.370837in}}%
\pgfpathlineto{\pgfqpoint{5.932329in}{3.367881in}}%
\pgfpathlineto{\pgfqpoint{5.925522in}{3.365064in}}%
\pgfpathclose%
\pgfusepath{fill}%
\end{pgfscope}%
\begin{pgfscope}%
\pgfpathrectangle{\pgfqpoint{1.254980in}{0.150000in}}{\pgfqpoint{5.490039in}{5.490039in}}%
\pgfusepath{clip}%
\pgfsetbuttcap%
\pgfsetroundjoin%
\definecolor{currentfill}{rgb}{0.276022,0.044167,0.370164}%
\pgfsetfillcolor{currentfill}%
\pgfsetfillopacity{0.700000}%
\pgfsetlinewidth{0.000000pt}%
\definecolor{currentstroke}{rgb}{0.000000,0.000000,0.000000}%
\pgfsetstrokecolor{currentstroke}%
\pgfsetdash{}{0pt}%
\pgfpathmoveto{\pgfqpoint{3.521200in}{1.909874in}}%
\pgfpathlineto{\pgfqpoint{3.534333in}{1.906863in}}%
\pgfpathlineto{\pgfqpoint{3.547471in}{1.904034in}}%
\pgfpathlineto{\pgfqpoint{3.560614in}{1.901386in}}%
\pgfpathlineto{\pgfqpoint{3.573761in}{1.898918in}}%
\pgfpathlineto{\pgfqpoint{3.581594in}{1.908434in}}%
\pgfpathlineto{\pgfqpoint{3.589422in}{1.917973in}}%
\pgfpathlineto{\pgfqpoint{3.597244in}{1.927534in}}%
\pgfpathlineto{\pgfqpoint{3.605061in}{1.937114in}}%
\pgfpathlineto{\pgfqpoint{3.591925in}{1.939366in}}%
\pgfpathlineto{\pgfqpoint{3.578794in}{1.941799in}}%
\pgfpathlineto{\pgfqpoint{3.565668in}{1.944411in}}%
\pgfpathlineto{\pgfqpoint{3.552547in}{1.947206in}}%
\pgfpathlineto{\pgfqpoint{3.544719in}{1.937831in}}%
\pgfpathlineto{\pgfqpoint{3.536885in}{1.928483in}}%
\pgfpathlineto{\pgfqpoint{3.529045in}{1.919163in}}%
\pgfpathlineto{\pgfqpoint{3.521200in}{1.909874in}}%
\pgfpathclose%
\pgfusepath{fill}%
\end{pgfscope}%
\begin{pgfscope}%
\pgfpathrectangle{\pgfqpoint{1.254980in}{0.150000in}}{\pgfqpoint{5.490039in}{5.490039in}}%
\pgfusepath{clip}%
\pgfsetbuttcap%
\pgfsetroundjoin%
\definecolor{currentfill}{rgb}{0.162016,0.687316,0.499129}%
\pgfsetfillcolor{currentfill}%
\pgfsetfillopacity{0.700000}%
\pgfsetlinewidth{0.000000pt}%
\definecolor{currentstroke}{rgb}{0.000000,0.000000,0.000000}%
\pgfsetstrokecolor{currentstroke}%
\pgfsetdash{}{0pt}%
\pgfpathmoveto{\pgfqpoint{6.009280in}{3.412088in}}%
\pgfpathlineto{\pgfqpoint{6.023460in}{3.421187in}}%
\pgfpathlineto{\pgfqpoint{6.037659in}{3.430437in}}%
\pgfpathlineto{\pgfqpoint{6.051877in}{3.439838in}}%
\pgfpathlineto{\pgfqpoint{6.066115in}{3.449390in}}%
\pgfpathlineto{\pgfqpoint{6.072843in}{3.451465in}}%
\pgfpathlineto{\pgfqpoint{6.079570in}{3.453696in}}%
\pgfpathlineto{\pgfqpoint{6.086298in}{3.456089in}}%
\pgfpathlineto{\pgfqpoint{6.093026in}{3.458651in}}%
\pgfpathlineto{\pgfqpoint{6.078829in}{3.449782in}}%
\pgfpathlineto{\pgfqpoint{6.064650in}{3.441063in}}%
\pgfpathlineto{\pgfqpoint{6.050489in}{3.432493in}}%
\pgfpathlineto{\pgfqpoint{6.036347in}{3.424073in}}%
\pgfpathlineto{\pgfqpoint{6.029580in}{3.420821in}}%
\pgfpathlineto{\pgfqpoint{6.022813in}{3.417744in}}%
\pgfpathlineto{\pgfqpoint{6.016046in}{3.414835in}}%
\pgfpathlineto{\pgfqpoint{6.009280in}{3.412088in}}%
\pgfpathclose%
\pgfusepath{fill}%
\end{pgfscope}%
\begin{pgfscope}%
\pgfpathrectangle{\pgfqpoint{1.254980in}{0.150000in}}{\pgfqpoint{5.490039in}{5.490039in}}%
\pgfusepath{clip}%
\pgfsetbuttcap%
\pgfsetroundjoin%
\definecolor{currentfill}{rgb}{0.194100,0.399323,0.555565}%
\pgfsetfillcolor{currentfill}%
\pgfsetfillopacity{0.700000}%
\pgfsetlinewidth{0.000000pt}%
\definecolor{currentstroke}{rgb}{0.000000,0.000000,0.000000}%
\pgfsetstrokecolor{currentstroke}%
\pgfsetdash{}{0pt}%
\pgfpathmoveto{\pgfqpoint{2.388771in}{2.701538in}}%
\pgfpathlineto{\pgfqpoint{2.402262in}{2.679736in}}%
\pgfpathlineto{\pgfqpoint{2.415739in}{2.658245in}}%
\pgfpathlineto{\pgfqpoint{2.429203in}{2.637061in}}%
\pgfpathlineto{\pgfqpoint{2.442655in}{2.616181in}}%
\pgfpathlineto{\pgfqpoint{2.451090in}{2.618422in}}%
\pgfpathlineto{\pgfqpoint{2.459512in}{2.620864in}}%
\pgfpathlineto{\pgfqpoint{2.467919in}{2.623503in}}%
\pgfpathlineto{\pgfqpoint{2.476312in}{2.626336in}}%
\pgfpathlineto{\pgfqpoint{2.462899in}{2.646861in}}%
\pgfpathlineto{\pgfqpoint{2.449474in}{2.667689in}}%
\pgfpathlineto{\pgfqpoint{2.436037in}{2.688823in}}%
\pgfpathlineto{\pgfqpoint{2.422586in}{2.710267in}}%
\pgfpathlineto{\pgfqpoint{2.414155in}{2.707779in}}%
\pgfpathlineto{\pgfqpoint{2.405708in}{2.705493in}}%
\pgfpathlineto{\pgfqpoint{2.397247in}{2.703411in}}%
\pgfpathlineto{\pgfqpoint{2.388771in}{2.701538in}}%
\pgfpathclose%
\pgfusepath{fill}%
\end{pgfscope}%
\begin{pgfscope}%
\pgfpathrectangle{\pgfqpoint{1.254980in}{0.150000in}}{\pgfqpoint{5.490039in}{5.490039in}}%
\pgfusepath{clip}%
\pgfsetbuttcap%
\pgfsetroundjoin%
\definecolor{currentfill}{rgb}{0.180653,0.701402,0.488189}%
\pgfsetfillcolor{currentfill}%
\pgfsetfillopacity{0.700000}%
\pgfsetlinewidth{0.000000pt}%
\definecolor{currentstroke}{rgb}{0.000000,0.000000,0.000000}%
\pgfsetstrokecolor{currentstroke}%
\pgfsetdash{}{0pt}%
\pgfpathmoveto{\pgfqpoint{6.093026in}{3.458651in}}%
\pgfpathlineto{\pgfqpoint{6.107243in}{3.467670in}}%
\pgfpathlineto{\pgfqpoint{6.121478in}{3.476839in}}%
\pgfpathlineto{\pgfqpoint{6.135732in}{3.486158in}}%
\pgfpathlineto{\pgfqpoint{6.142430in}{3.488373in}}%
\pgfpathlineto{\pgfqpoint{6.149130in}{3.490767in}}%
\pgfpathlineto{\pgfqpoint{6.155831in}{3.493348in}}%
\pgfpathlineto{\pgfqpoint{6.141607in}{3.484559in}}%
\pgfpathlineto{\pgfqpoint{6.127403in}{3.475920in}}%
\pgfpathlineto{\pgfqpoint{6.113217in}{3.467430in}}%
\pgfpathlineto{\pgfqpoint{6.106485in}{3.464314in}}%
\pgfpathlineto{\pgfqpoint{6.099755in}{3.461390in}}%
\pgfpathlineto{\pgfqpoint{6.093026in}{3.458651in}}%
\pgfpathclose%
\pgfusepath{fill}%
\end{pgfscope}%
\begin{pgfscope}%
\pgfpathrectangle{\pgfqpoint{1.254980in}{0.150000in}}{\pgfqpoint{5.490039in}{5.490039in}}%
\pgfusepath{clip}%
\pgfsetbuttcap%
\pgfsetroundjoin%
\definecolor{currentfill}{rgb}{0.250425,0.274290,0.533103}%
\pgfsetfillcolor{currentfill}%
\pgfsetfillopacity{0.700000}%
\pgfsetlinewidth{0.000000pt}%
\definecolor{currentstroke}{rgb}{0.000000,0.000000,0.000000}%
\pgfsetstrokecolor{currentstroke}%
\pgfsetdash{}{0pt}%
\pgfpathmoveto{\pgfqpoint{4.358094in}{2.345001in}}%
\pgfpathlineto{\pgfqpoint{4.371474in}{2.349742in}}%
\pgfpathlineto{\pgfqpoint{4.384865in}{2.354647in}}%
\pgfpathlineto{\pgfqpoint{4.398267in}{2.359714in}}%
\pgfpathlineto{\pgfqpoint{4.411682in}{2.364945in}}%
\pgfpathlineto{\pgfqpoint{4.419232in}{2.374126in}}%
\pgfpathlineto{\pgfqpoint{4.426778in}{2.383242in}}%
\pgfpathlineto{\pgfqpoint{4.434318in}{2.392294in}}%
\pgfpathlineto{\pgfqpoint{4.441852in}{2.401284in}}%
\pgfpathlineto{\pgfqpoint{4.428444in}{2.396118in}}%
\pgfpathlineto{\pgfqpoint{4.415048in}{2.391116in}}%
\pgfpathlineto{\pgfqpoint{4.401664in}{2.386276in}}%
\pgfpathlineto{\pgfqpoint{4.388292in}{2.381600in}}%
\pgfpathlineto{\pgfqpoint{4.380751in}{2.372535in}}%
\pgfpathlineto{\pgfqpoint{4.373204in}{2.363414in}}%
\pgfpathlineto{\pgfqpoint{4.365652in}{2.354236in}}%
\pgfpathlineto{\pgfqpoint{4.358094in}{2.345001in}}%
\pgfpathclose%
\pgfusepath{fill}%
\end{pgfscope}%
\begin{pgfscope}%
\pgfpathrectangle{\pgfqpoint{1.254980in}{0.150000in}}{\pgfqpoint{5.490039in}{5.490039in}}%
\pgfusepath{clip}%
\pgfsetbuttcap%
\pgfsetroundjoin%
\definecolor{currentfill}{rgb}{0.180629,0.429975,0.557282}%
\pgfsetfillcolor{currentfill}%
\pgfsetfillopacity{0.700000}%
\pgfsetlinewidth{0.000000pt}%
\definecolor{currentstroke}{rgb}{0.000000,0.000000,0.000000}%
\pgfsetstrokecolor{currentstroke}%
\pgfsetdash{}{0pt}%
\pgfpathmoveto{\pgfqpoint{4.890765in}{2.719802in}}%
\pgfpathlineto{\pgfqpoint{4.904398in}{2.727336in}}%
\pgfpathlineto{\pgfqpoint{4.918045in}{2.735030in}}%
\pgfpathlineto{\pgfqpoint{4.931707in}{2.742881in}}%
\pgfpathlineto{\pgfqpoint{4.945384in}{2.750891in}}%
\pgfpathlineto{\pgfqpoint{4.952710in}{2.757370in}}%
\pgfpathlineto{\pgfqpoint{4.960029in}{2.763799in}}%
\pgfpathlineto{\pgfqpoint{4.967342in}{2.770181in}}%
\pgfpathlineto{\pgfqpoint{4.974648in}{2.776521in}}%
\pgfpathlineto{\pgfqpoint{4.960984in}{2.768782in}}%
\pgfpathlineto{\pgfqpoint{4.947335in}{2.761200in}}%
\pgfpathlineto{\pgfqpoint{4.933700in}{2.753775in}}%
\pgfpathlineto{\pgfqpoint{4.920080in}{2.746509in}}%
\pgfpathlineto{\pgfqpoint{4.912761in}{2.739889in}}%
\pgfpathlineto{\pgfqpoint{4.905435in}{2.733234in}}%
\pgfpathlineto{\pgfqpoint{4.898103in}{2.726539in}}%
\pgfpathlineto{\pgfqpoint{4.890765in}{2.719802in}}%
\pgfpathclose%
\pgfusepath{fill}%
\end{pgfscope}%
\begin{pgfscope}%
\pgfpathrectangle{\pgfqpoint{1.254980in}{0.150000in}}{\pgfqpoint{5.490039in}{5.490039in}}%
\pgfusepath{clip}%
\pgfsetbuttcap%
\pgfsetroundjoin%
\definecolor{currentfill}{rgb}{0.281446,0.084320,0.407414}%
\pgfsetfillcolor{currentfill}%
\pgfsetfillopacity{0.700000}%
\pgfsetlinewidth{0.000000pt}%
\definecolor{currentstroke}{rgb}{0.000000,0.000000,0.000000}%
\pgfsetstrokecolor{currentstroke}%
\pgfsetdash{}{0pt}%
\pgfpathmoveto{\pgfqpoint{2.973603in}{2.001312in}}%
\pgfpathlineto{\pgfqpoint{2.986770in}{1.990847in}}%
\pgfpathlineto{\pgfqpoint{2.999935in}{1.980597in}}%
\pgfpathlineto{\pgfqpoint{3.013099in}{1.970561in}}%
\pgfpathlineto{\pgfqpoint{3.026260in}{1.960738in}}%
\pgfpathlineto{\pgfqpoint{3.034348in}{1.966854in}}%
\pgfpathlineto{\pgfqpoint{3.042426in}{1.973090in}}%
\pgfpathlineto{\pgfqpoint{3.050496in}{1.979442in}}%
\pgfpathlineto{\pgfqpoint{3.058557in}{1.985908in}}%
\pgfpathlineto{\pgfqpoint{3.045419in}{1.995399in}}%
\pgfpathlineto{\pgfqpoint{3.032280in}{2.005103in}}%
\pgfpathlineto{\pgfqpoint{3.019139in}{2.015020in}}%
\pgfpathlineto{\pgfqpoint{3.005997in}{2.025153in}}%
\pgfpathlineto{\pgfqpoint{2.997912in}{2.019009in}}%
\pgfpathlineto{\pgfqpoint{2.989818in}{2.012985in}}%
\pgfpathlineto{\pgfqpoint{2.981715in}{2.007085in}}%
\pgfpathlineto{\pgfqpoint{2.973603in}{2.001312in}}%
\pgfpathclose%
\pgfusepath{fill}%
\end{pgfscope}%
\begin{pgfscope}%
\pgfpathrectangle{\pgfqpoint{1.254980in}{0.150000in}}{\pgfqpoint{5.490039in}{5.490039in}}%
\pgfusepath{clip}%
\pgfsetbuttcap%
\pgfsetroundjoin%
\definecolor{currentfill}{rgb}{0.273809,0.031497,0.358853}%
\pgfsetfillcolor{currentfill}%
\pgfsetfillopacity{0.700000}%
\pgfsetlinewidth{0.000000pt}%
\definecolor{currentstroke}{rgb}{0.000000,0.000000,0.000000}%
\pgfsetstrokecolor{currentstroke}%
\pgfsetdash{}{0pt}%
\pgfpathmoveto{\pgfqpoint{3.300556in}{1.893655in}}%
\pgfpathlineto{\pgfqpoint{3.313681in}{1.887887in}}%
\pgfpathlineto{\pgfqpoint{3.326807in}{1.882310in}}%
\pgfpathlineto{\pgfqpoint{3.339936in}{1.876925in}}%
\pgfpathlineto{\pgfqpoint{3.353067in}{1.871729in}}%
\pgfpathlineto{\pgfqpoint{3.360991in}{1.880137in}}%
\pgfpathlineto{\pgfqpoint{3.368908in}{1.888606in}}%
\pgfpathlineto{\pgfqpoint{3.376819in}{1.897135in}}%
\pgfpathlineto{\pgfqpoint{3.384723in}{1.905720in}}%
\pgfpathlineto{\pgfqpoint{3.371608in}{1.910644in}}%
\pgfpathlineto{\pgfqpoint{3.358495in}{1.915758in}}%
\pgfpathlineto{\pgfqpoint{3.345386in}{1.921062in}}%
\pgfpathlineto{\pgfqpoint{3.332278in}{1.926557in}}%
\pgfpathlineto{\pgfqpoint{3.324358in}{1.918233in}}%
\pgfpathlineto{\pgfqpoint{3.316431in}{1.909974in}}%
\pgfpathlineto{\pgfqpoint{3.308497in}{1.901780in}}%
\pgfpathlineto{\pgfqpoint{3.300556in}{1.893655in}}%
\pgfpathclose%
\pgfusepath{fill}%
\end{pgfscope}%
\begin{pgfscope}%
\pgfpathrectangle{\pgfqpoint{1.254980in}{0.150000in}}{\pgfqpoint{5.490039in}{5.490039in}}%
\pgfusepath{clip}%
\pgfsetbuttcap%
\pgfsetroundjoin%
\definecolor{currentfill}{rgb}{0.276022,0.044167,0.370164}%
\pgfsetfillcolor{currentfill}%
\pgfsetfillopacity{0.700000}%
\pgfsetlinewidth{0.000000pt}%
\definecolor{currentstroke}{rgb}{0.000000,0.000000,0.000000}%
\pgfsetstrokecolor{currentstroke}%
\pgfsetdash{}{0pt}%
\pgfpathmoveto{\pgfqpoint{3.163631in}{1.917461in}}%
\pgfpathlineto{\pgfqpoint{3.176764in}{1.909823in}}%
\pgfpathlineto{\pgfqpoint{3.189898in}{1.902384in}}%
\pgfpathlineto{\pgfqpoint{3.203033in}{1.895144in}}%
\pgfpathlineto{\pgfqpoint{3.216169in}{1.888103in}}%
\pgfpathlineto{\pgfqpoint{3.224157in}{1.895616in}}%
\pgfpathlineto{\pgfqpoint{3.232138in}{1.903215in}}%
\pgfpathlineto{\pgfqpoint{3.240111in}{1.910898in}}%
\pgfpathlineto{\pgfqpoint{3.248077in}{1.918662in}}%
\pgfpathlineto{\pgfqpoint{3.234961in}{1.925403in}}%
\pgfpathlineto{\pgfqpoint{3.221845in}{1.932342in}}%
\pgfpathlineto{\pgfqpoint{3.208731in}{1.939479in}}%
\pgfpathlineto{\pgfqpoint{3.195617in}{1.946816in}}%
\pgfpathlineto{\pgfqpoint{3.187632in}{1.939343in}}%
\pgfpathlineto{\pgfqpoint{3.179640in}{1.931957in}}%
\pgfpathlineto{\pgfqpoint{3.171639in}{1.924662in}}%
\pgfpathlineto{\pgfqpoint{3.163631in}{1.917461in}}%
\pgfpathclose%
\pgfusepath{fill}%
\end{pgfscope}%
\begin{pgfscope}%
\pgfpathrectangle{\pgfqpoint{1.254980in}{0.150000in}}{\pgfqpoint{5.490039in}{5.490039in}}%
\pgfusepath{clip}%
\pgfsetbuttcap%
\pgfsetroundjoin%
\definecolor{currentfill}{rgb}{0.171176,0.452530,0.557965}%
\pgfsetfillcolor{currentfill}%
\pgfsetfillopacity{0.700000}%
\pgfsetlinewidth{0.000000pt}%
\definecolor{currentstroke}{rgb}{0.000000,0.000000,0.000000}%
\pgfsetstrokecolor{currentstroke}%
\pgfsetdash{}{0pt}%
\pgfpathmoveto{\pgfqpoint{4.974648in}{2.776521in}}%
\pgfpathlineto{\pgfqpoint{4.988328in}{2.784419in}}%
\pgfpathlineto{\pgfqpoint{5.002022in}{2.792475in}}%
\pgfpathlineto{\pgfqpoint{5.015732in}{2.800688in}}%
\pgfpathlineto{\pgfqpoint{5.029457in}{2.809059in}}%
\pgfpathlineto{\pgfqpoint{5.036743in}{2.815071in}}%
\pgfpathlineto{\pgfqpoint{5.044023in}{2.821039in}}%
\pgfpathlineto{\pgfqpoint{5.051296in}{2.826968in}}%
\pgfpathlineto{\pgfqpoint{5.058563in}{2.832861in}}%
\pgfpathlineto{\pgfqpoint{5.044852in}{2.824789in}}%
\pgfpathlineto{\pgfqpoint{5.031156in}{2.816876in}}%
\pgfpathlineto{\pgfqpoint{5.017476in}{2.809119in}}%
\pgfpathlineto{\pgfqpoint{5.003810in}{2.801520in}}%
\pgfpathlineto{\pgfqpoint{4.996529in}{2.795317in}}%
\pgfpathlineto{\pgfqpoint{4.989242in}{2.789086in}}%
\pgfpathlineto{\pgfqpoint{4.981948in}{2.782822in}}%
\pgfpathlineto{\pgfqpoint{4.974648in}{2.776521in}}%
\pgfpathclose%
\pgfusepath{fill}%
\end{pgfscope}%
\begin{pgfscope}%
\pgfpathrectangle{\pgfqpoint{1.254980in}{0.150000in}}{\pgfqpoint{5.490039in}{5.490039in}}%
\pgfusepath{clip}%
\pgfsetbuttcap%
\pgfsetroundjoin%
\definecolor{currentfill}{rgb}{0.239346,0.300855,0.540844}%
\pgfsetfillcolor{currentfill}%
\pgfsetfillopacity{0.700000}%
\pgfsetlinewidth{0.000000pt}%
\definecolor{currentstroke}{rgb}{0.000000,0.000000,0.000000}%
\pgfsetstrokecolor{currentstroke}%
\pgfsetdash{}{0pt}%
\pgfpathmoveto{\pgfqpoint{4.441852in}{2.401284in}}%
\pgfpathlineto{\pgfqpoint{4.455272in}{2.406612in}}%
\pgfpathlineto{\pgfqpoint{4.468703in}{2.412102in}}%
\pgfpathlineto{\pgfqpoint{4.482147in}{2.417755in}}%
\pgfpathlineto{\pgfqpoint{4.495604in}{2.423570in}}%
\pgfpathlineto{\pgfqpoint{4.503126in}{2.432415in}}%
\pgfpathlineto{\pgfqpoint{4.510641in}{2.441193in}}%
\pgfpathlineto{\pgfqpoint{4.518151in}{2.449905in}}%
\pgfpathlineto{\pgfqpoint{4.525656in}{2.458553in}}%
\pgfpathlineto{\pgfqpoint{4.512206in}{2.452833in}}%
\pgfpathlineto{\pgfqpoint{4.498770in}{2.447274in}}%
\pgfpathlineto{\pgfqpoint{4.485345in}{2.441877in}}%
\pgfpathlineto{\pgfqpoint{4.471933in}{2.436643in}}%
\pgfpathlineto{\pgfqpoint{4.464421in}{2.427891in}}%
\pgfpathlineto{\pgfqpoint{4.456904in}{2.419081in}}%
\pgfpathlineto{\pgfqpoint{4.449381in}{2.410212in}}%
\pgfpathlineto{\pgfqpoint{4.441852in}{2.401284in}}%
\pgfpathclose%
\pgfusepath{fill}%
\end{pgfscope}%
\begin{pgfscope}%
\pgfpathrectangle{\pgfqpoint{1.254980in}{0.150000in}}{\pgfqpoint{5.490039in}{5.490039in}}%
\pgfusepath{clip}%
\pgfsetbuttcap%
\pgfsetroundjoin%
\definecolor{currentfill}{rgb}{0.273809,0.031497,0.358853}%
\pgfsetfillcolor{currentfill}%
\pgfsetfillopacity{0.700000}%
\pgfsetlinewidth{0.000000pt}%
\definecolor{currentstroke}{rgb}{0.000000,0.000000,0.000000}%
\pgfsetstrokecolor{currentstroke}%
\pgfsetdash{}{0pt}%
\pgfpathmoveto{\pgfqpoint{3.437213in}{1.887905in}}%
\pgfpathlineto{\pgfqpoint{3.450344in}{1.883916in}}%
\pgfpathlineto{\pgfqpoint{3.463479in}{1.880113in}}%
\pgfpathlineto{\pgfqpoint{3.476617in}{1.876493in}}%
\pgfpathlineto{\pgfqpoint{3.489759in}{1.873056in}}%
\pgfpathlineto{\pgfqpoint{3.497628in}{1.882205in}}%
\pgfpathlineto{\pgfqpoint{3.505491in}{1.891393in}}%
\pgfpathlineto{\pgfqpoint{3.513348in}{1.900616in}}%
\pgfpathlineto{\pgfqpoint{3.521200in}{1.909874in}}%
\pgfpathlineto{\pgfqpoint{3.508071in}{1.913066in}}%
\pgfpathlineto{\pgfqpoint{3.494946in}{1.916442in}}%
\pgfpathlineto{\pgfqpoint{3.481825in}{1.920002in}}%
\pgfpathlineto{\pgfqpoint{3.468708in}{1.923746in}}%
\pgfpathlineto{\pgfqpoint{3.460843in}{1.914722in}}%
\pgfpathlineto{\pgfqpoint{3.452973in}{1.905739in}}%
\pgfpathlineto{\pgfqpoint{3.445096in}{1.896799in}}%
\pgfpathlineto{\pgfqpoint{3.437213in}{1.887905in}}%
\pgfpathclose%
\pgfusepath{fill}%
\end{pgfscope}%
\begin{pgfscope}%
\pgfpathrectangle{\pgfqpoint{1.254980in}{0.150000in}}{\pgfqpoint{5.490039in}{5.490039in}}%
\pgfusepath{clip}%
\pgfsetbuttcap%
\pgfsetroundjoin%
\definecolor{currentfill}{rgb}{0.160665,0.478540,0.558115}%
\pgfsetfillcolor{currentfill}%
\pgfsetfillopacity{0.700000}%
\pgfsetlinewidth{0.000000pt}%
\definecolor{currentstroke}{rgb}{0.000000,0.000000,0.000000}%
\pgfsetstrokecolor{currentstroke}%
\pgfsetdash{}{0pt}%
\pgfpathmoveto{\pgfqpoint{5.058563in}{2.832861in}}%
\pgfpathlineto{\pgfqpoint{5.072289in}{2.841089in}}%
\pgfpathlineto{\pgfqpoint{5.086031in}{2.849475in}}%
\pgfpathlineto{\pgfqpoint{5.099788in}{2.858018in}}%
\pgfpathlineto{\pgfqpoint{5.113562in}{2.866718in}}%
\pgfpathlineto{\pgfqpoint{5.120807in}{2.872259in}}%
\pgfpathlineto{\pgfqpoint{5.128046in}{2.877764in}}%
\pgfpathlineto{\pgfqpoint{5.135278in}{2.883237in}}%
\pgfpathlineto{\pgfqpoint{5.142503in}{2.888682in}}%
\pgfpathlineto{\pgfqpoint{5.128746in}{2.880312in}}%
\pgfpathlineto{\pgfqpoint{5.115004in}{2.872098in}}%
\pgfpathlineto{\pgfqpoint{5.101277in}{2.864041in}}%
\pgfpathlineto{\pgfqpoint{5.087566in}{2.856141in}}%
\pgfpathlineto{\pgfqpoint{5.080325in}{2.850357in}}%
\pgfpathlineto{\pgfqpoint{5.073077in}{2.844551in}}%
\pgfpathlineto{\pgfqpoint{5.065823in}{2.838720in}}%
\pgfpathlineto{\pgfqpoint{5.058563in}{2.832861in}}%
\pgfpathclose%
\pgfusepath{fill}%
\end{pgfscope}%
\begin{pgfscope}%
\pgfpathrectangle{\pgfqpoint{1.254980in}{0.150000in}}{\pgfqpoint{5.490039in}{5.490039in}}%
\pgfusepath{clip}%
\pgfsetbuttcap%
\pgfsetroundjoin%
\definecolor{currentfill}{rgb}{0.225863,0.330805,0.547314}%
\pgfsetfillcolor{currentfill}%
\pgfsetfillopacity{0.700000}%
\pgfsetlinewidth{0.000000pt}%
\definecolor{currentstroke}{rgb}{0.000000,0.000000,0.000000}%
\pgfsetstrokecolor{currentstroke}%
\pgfsetdash{}{0pt}%
\pgfpathmoveto{\pgfqpoint{4.525656in}{2.458553in}}%
\pgfpathlineto{\pgfqpoint{4.539118in}{2.464436in}}%
\pgfpathlineto{\pgfqpoint{4.552592in}{2.470480in}}%
\pgfpathlineto{\pgfqpoint{4.566080in}{2.476685in}}%
\pgfpathlineto{\pgfqpoint{4.579580in}{2.483052in}}%
\pgfpathlineto{\pgfqpoint{4.587071in}{2.491525in}}%
\pgfpathlineto{\pgfqpoint{4.594556in}{2.499929in}}%
\pgfpathlineto{\pgfqpoint{4.602035in}{2.508267in}}%
\pgfpathlineto{\pgfqpoint{4.609508in}{2.516539in}}%
\pgfpathlineto{\pgfqpoint{4.596015in}{2.510296in}}%
\pgfpathlineto{\pgfqpoint{4.582536in}{2.504214in}}%
\pgfpathlineto{\pgfqpoint{4.569069in}{2.498293in}}%
\pgfpathlineto{\pgfqpoint{4.555615in}{2.492533in}}%
\pgfpathlineto{\pgfqpoint{4.548134in}{2.484127in}}%
\pgfpathlineto{\pgfqpoint{4.540647in}{2.475663in}}%
\pgfpathlineto{\pgfqpoint{4.533154in}{2.467139in}}%
\pgfpathlineto{\pgfqpoint{4.525656in}{2.458553in}}%
\pgfpathclose%
\pgfusepath{fill}%
\end{pgfscope}%
\begin{pgfscope}%
\pgfpathrectangle{\pgfqpoint{1.254980in}{0.150000in}}{\pgfqpoint{5.490039in}{5.490039in}}%
\pgfusepath{clip}%
\pgfsetbuttcap%
\pgfsetroundjoin%
\definecolor{currentfill}{rgb}{0.279566,0.067836,0.391917}%
\pgfsetfillcolor{currentfill}%
\pgfsetfillopacity{0.700000}%
\pgfsetlinewidth{0.000000pt}%
\definecolor{currentstroke}{rgb}{0.000000,0.000000,0.000000}%
\pgfsetstrokecolor{currentstroke}%
\pgfsetdash{}{0pt}%
\pgfpathmoveto{\pgfqpoint{3.026260in}{1.960738in}}%
\pgfpathlineto{\pgfqpoint{3.039421in}{1.951126in}}%
\pgfpathlineto{\pgfqpoint{3.052579in}{1.941724in}}%
\pgfpathlineto{\pgfqpoint{3.065737in}{1.932530in}}%
\pgfpathlineto{\pgfqpoint{3.078894in}{1.923543in}}%
\pgfpathlineto{\pgfqpoint{3.086958in}{1.930002in}}%
\pgfpathlineto{\pgfqpoint{3.095013in}{1.936572in}}%
\pgfpathlineto{\pgfqpoint{3.103059in}{1.943252in}}%
\pgfpathlineto{\pgfqpoint{3.111097in}{1.950037in}}%
\pgfpathlineto{\pgfqpoint{3.097963in}{1.958693in}}%
\pgfpathlineto{\pgfqpoint{3.084829in}{1.967556in}}%
\pgfpathlineto{\pgfqpoint{3.071693in}{1.976627in}}%
\pgfpathlineto{\pgfqpoint{3.058557in}{1.985908in}}%
\pgfpathlineto{\pgfqpoint{3.050496in}{1.979442in}}%
\pgfpathlineto{\pgfqpoint{3.042426in}{1.973090in}}%
\pgfpathlineto{\pgfqpoint{3.034348in}{1.966854in}}%
\pgfpathlineto{\pgfqpoint{3.026260in}{1.960738in}}%
\pgfpathclose%
\pgfusepath{fill}%
\end{pgfscope}%
\begin{pgfscope}%
\pgfpathrectangle{\pgfqpoint{1.254980in}{0.150000in}}{\pgfqpoint{5.490039in}{5.490039in}}%
\pgfusepath{clip}%
\pgfsetbuttcap%
\pgfsetroundjoin%
\definecolor{currentfill}{rgb}{0.283091,0.110553,0.431554}%
\pgfsetfillcolor{currentfill}%
\pgfsetfillopacity{0.700000}%
\pgfsetlinewidth{0.000000pt}%
\definecolor{currentstroke}{rgb}{0.000000,0.000000,0.000000}%
\pgfsetstrokecolor{currentstroke}%
\pgfsetdash{}{0pt}%
\pgfpathmoveto{\pgfqpoint{3.825243in}{2.005250in}}%
\pgfpathlineto{\pgfqpoint{3.838441in}{2.005633in}}%
\pgfpathlineto{\pgfqpoint{3.851646in}{2.006187in}}%
\pgfpathlineto{\pgfqpoint{3.864859in}{2.006913in}}%
\pgfpathlineto{\pgfqpoint{3.878080in}{2.007810in}}%
\pgfpathlineto{\pgfqpoint{3.885813in}{2.018070in}}%
\pgfpathlineto{\pgfqpoint{3.893541in}{2.028308in}}%
\pgfpathlineto{\pgfqpoint{3.901264in}{2.038523in}}%
\pgfpathlineto{\pgfqpoint{3.908982in}{2.048714in}}%
\pgfpathlineto{\pgfqpoint{3.895768in}{2.047684in}}%
\pgfpathlineto{\pgfqpoint{3.882563in}{2.046825in}}%
\pgfpathlineto{\pgfqpoint{3.869366in}{2.046138in}}%
\pgfpathlineto{\pgfqpoint{3.856176in}{2.045623in}}%
\pgfpathlineto{\pgfqpoint{3.848450in}{2.035554in}}%
\pgfpathlineto{\pgfqpoint{3.840720in}{2.025469in}}%
\pgfpathlineto{\pgfqpoint{3.832984in}{2.015367in}}%
\pgfpathlineto{\pgfqpoint{3.825243in}{2.005250in}}%
\pgfpathclose%
\pgfusepath{fill}%
\end{pgfscope}%
\begin{pgfscope}%
\pgfpathrectangle{\pgfqpoint{1.254980in}{0.150000in}}{\pgfqpoint{5.490039in}{5.490039in}}%
\pgfusepath{clip}%
\pgfsetbuttcap%
\pgfsetroundjoin%
\definecolor{currentfill}{rgb}{0.283072,0.130895,0.449241}%
\pgfsetfillcolor{currentfill}%
\pgfsetfillopacity{0.700000}%
\pgfsetlinewidth{0.000000pt}%
\definecolor{currentstroke}{rgb}{0.000000,0.000000,0.000000}%
\pgfsetstrokecolor{currentstroke}%
\pgfsetdash{}{0pt}%
\pgfpathmoveto{\pgfqpoint{3.908982in}{2.048714in}}%
\pgfpathlineto{\pgfqpoint{3.922203in}{2.049914in}}%
\pgfpathlineto{\pgfqpoint{3.935432in}{2.051285in}}%
\pgfpathlineto{\pgfqpoint{3.948670in}{2.052825in}}%
\pgfpathlineto{\pgfqpoint{3.961916in}{2.054535in}}%
\pgfpathlineto{\pgfqpoint{3.969622in}{2.064816in}}%
\pgfpathlineto{\pgfqpoint{3.977323in}{2.075064in}}%
\pgfpathlineto{\pgfqpoint{3.985019in}{2.085279in}}%
\pgfpathlineto{\pgfqpoint{3.992710in}{2.095459in}}%
\pgfpathlineto{\pgfqpoint{3.979471in}{2.093645in}}%
\pgfpathlineto{\pgfqpoint{3.966240in}{2.092000in}}%
\pgfpathlineto{\pgfqpoint{3.953018in}{2.090524in}}%
\pgfpathlineto{\pgfqpoint{3.939804in}{2.089219in}}%
\pgfpathlineto{\pgfqpoint{3.932106in}{2.079133in}}%
\pgfpathlineto{\pgfqpoint{3.924403in}{2.069019in}}%
\pgfpathlineto{\pgfqpoint{3.916695in}{2.058879in}}%
\pgfpathlineto{\pgfqpoint{3.908982in}{2.048714in}}%
\pgfpathclose%
\pgfusepath{fill}%
\end{pgfscope}%
\begin{pgfscope}%
\pgfpathrectangle{\pgfqpoint{1.254980in}{0.150000in}}{\pgfqpoint{5.490039in}{5.490039in}}%
\pgfusepath{clip}%
\pgfsetbuttcap%
\pgfsetroundjoin%
\definecolor{currentfill}{rgb}{0.271828,0.209303,0.504434}%
\pgfsetfillcolor{currentfill}%
\pgfsetfillopacity{0.700000}%
\pgfsetlinewidth{0.000000pt}%
\definecolor{currentstroke}{rgb}{0.000000,0.000000,0.000000}%
\pgfsetstrokecolor{currentstroke}%
\pgfsetdash{}{0pt}%
\pgfpathmoveto{\pgfqpoint{2.676345in}{2.243771in}}%
\pgfpathlineto{\pgfqpoint{2.689632in}{2.228293in}}%
\pgfpathlineto{\pgfqpoint{2.702913in}{2.213065in}}%
\pgfpathlineto{\pgfqpoint{2.716186in}{2.198085in}}%
\pgfpathlineto{\pgfqpoint{2.729454in}{2.183350in}}%
\pgfpathlineto{\pgfqpoint{2.737726in}{2.187082in}}%
\pgfpathlineto{\pgfqpoint{2.745986in}{2.190983in}}%
\pgfpathlineto{\pgfqpoint{2.754234in}{2.195049in}}%
\pgfpathlineto{\pgfqpoint{2.762471in}{2.199277in}}%
\pgfpathlineto{\pgfqpoint{2.749236in}{2.213643in}}%
\pgfpathlineto{\pgfqpoint{2.735995in}{2.228254in}}%
\pgfpathlineto{\pgfqpoint{2.722747in}{2.243112in}}%
\pgfpathlineto{\pgfqpoint{2.709494in}{2.258220in}}%
\pgfpathlineto{\pgfqpoint{2.701225in}{2.254350in}}%
\pgfpathlineto{\pgfqpoint{2.692944in}{2.250650in}}%
\pgfpathlineto{\pgfqpoint{2.684651in}{2.247122in}}%
\pgfpathlineto{\pgfqpoint{2.676345in}{2.243771in}}%
\pgfpathclose%
\pgfusepath{fill}%
\end{pgfscope}%
\begin{pgfscope}%
\pgfpathrectangle{\pgfqpoint{1.254980in}{0.150000in}}{\pgfqpoint{5.490039in}{5.490039in}}%
\pgfusepath{clip}%
\pgfsetbuttcap%
\pgfsetroundjoin%
\definecolor{currentfill}{rgb}{0.263663,0.237631,0.518762}%
\pgfsetfillcolor{currentfill}%
\pgfsetfillopacity{0.700000}%
\pgfsetlinewidth{0.000000pt}%
\definecolor{currentstroke}{rgb}{0.000000,0.000000,0.000000}%
\pgfsetstrokecolor{currentstroke}%
\pgfsetdash{}{0pt}%
\pgfpathmoveto{\pgfqpoint{2.623124in}{2.308225in}}%
\pgfpathlineto{\pgfqpoint{2.636440in}{2.291726in}}%
\pgfpathlineto{\pgfqpoint{2.649750in}{2.275486in}}%
\pgfpathlineto{\pgfqpoint{2.663051in}{2.259501in}}%
\pgfpathlineto{\pgfqpoint{2.676345in}{2.243771in}}%
\pgfpathlineto{\pgfqpoint{2.684651in}{2.247122in}}%
\pgfpathlineto{\pgfqpoint{2.692944in}{2.250650in}}%
\pgfpathlineto{\pgfqpoint{2.701225in}{2.254350in}}%
\pgfpathlineto{\pgfqpoint{2.709494in}{2.258220in}}%
\pgfpathlineto{\pgfqpoint{2.696233in}{2.273579in}}%
\pgfpathlineto{\pgfqpoint{2.682966in}{2.289191in}}%
\pgfpathlineto{\pgfqpoint{2.669691in}{2.305059in}}%
\pgfpathlineto{\pgfqpoint{2.656409in}{2.321185in}}%
\pgfpathlineto{\pgfqpoint{2.648106in}{2.317676in}}%
\pgfpathlineto{\pgfqpoint{2.639792in}{2.314344in}}%
\pgfpathlineto{\pgfqpoint{2.631464in}{2.311193in}}%
\pgfpathlineto{\pgfqpoint{2.623124in}{2.308225in}}%
\pgfpathclose%
\pgfusepath{fill}%
\end{pgfscope}%
\begin{pgfscope}%
\pgfpathrectangle{\pgfqpoint{1.254980in}{0.150000in}}{\pgfqpoint{5.490039in}{5.490039in}}%
\pgfusepath{clip}%
\pgfsetbuttcap%
\pgfsetroundjoin%
\definecolor{currentfill}{rgb}{0.281924,0.089666,0.412415}%
\pgfsetfillcolor{currentfill}%
\pgfsetfillopacity{0.700000}%
\pgfsetlinewidth{0.000000pt}%
\definecolor{currentstroke}{rgb}{0.000000,0.000000,0.000000}%
\pgfsetstrokecolor{currentstroke}%
\pgfsetdash{}{0pt}%
\pgfpathmoveto{\pgfqpoint{3.741476in}{1.965495in}}%
\pgfpathlineto{\pgfqpoint{3.754654in}{1.965024in}}%
\pgfpathlineto{\pgfqpoint{3.767839in}{1.964727in}}%
\pgfpathlineto{\pgfqpoint{3.781031in}{1.964603in}}%
\pgfpathlineto{\pgfqpoint{3.794230in}{1.964653in}}%
\pgfpathlineto{\pgfqpoint{3.801991in}{1.974819in}}%
\pgfpathlineto{\pgfqpoint{3.809747in}{1.984975in}}%
\pgfpathlineto{\pgfqpoint{3.817498in}{1.995119in}}%
\pgfpathlineto{\pgfqpoint{3.825243in}{2.005250in}}%
\pgfpathlineto{\pgfqpoint{3.812053in}{2.005040in}}%
\pgfpathlineto{\pgfqpoint{3.798869in}{2.005003in}}%
\pgfpathlineto{\pgfqpoint{3.785693in}{2.005140in}}%
\pgfpathlineto{\pgfqpoint{3.772524in}{2.005451in}}%
\pgfpathlineto{\pgfqpoint{3.764769in}{1.995470in}}%
\pgfpathlineto{\pgfqpoint{3.757010in}{1.985483in}}%
\pgfpathlineto{\pgfqpoint{3.749245in}{1.975491in}}%
\pgfpathlineto{\pgfqpoint{3.741476in}{1.965495in}}%
\pgfpathclose%
\pgfusepath{fill}%
\end{pgfscope}%
\begin{pgfscope}%
\pgfpathrectangle{\pgfqpoint{1.254980in}{0.150000in}}{\pgfqpoint{5.490039in}{5.490039in}}%
\pgfusepath{clip}%
\pgfsetbuttcap%
\pgfsetroundjoin%
\definecolor{currentfill}{rgb}{0.281412,0.155834,0.469201}%
\pgfsetfillcolor{currentfill}%
\pgfsetfillopacity{0.700000}%
\pgfsetlinewidth{0.000000pt}%
\definecolor{currentstroke}{rgb}{0.000000,0.000000,0.000000}%
\pgfsetstrokecolor{currentstroke}%
\pgfsetdash{}{0pt}%
\pgfpathmoveto{\pgfqpoint{3.992710in}{2.095459in}}%
\pgfpathlineto{\pgfqpoint{4.005958in}{2.097443in}}%
\pgfpathlineto{\pgfqpoint{4.019215in}{2.099595in}}%
\pgfpathlineto{\pgfqpoint{4.032481in}{2.101915in}}%
\pgfpathlineto{\pgfqpoint{4.045756in}{2.104403in}}%
\pgfpathlineto{\pgfqpoint{4.053435in}{2.114637in}}%
\pgfpathlineto{\pgfqpoint{4.061110in}{2.124828in}}%
\pgfpathlineto{\pgfqpoint{4.068780in}{2.134977in}}%
\pgfpathlineto{\pgfqpoint{4.076444in}{2.145083in}}%
\pgfpathlineto{\pgfqpoint{4.063176in}{2.142518in}}%
\pgfpathlineto{\pgfqpoint{4.049916in}{2.140121in}}%
\pgfpathlineto{\pgfqpoint{4.036666in}{2.137892in}}%
\pgfpathlineto{\pgfqpoint{4.023425in}{2.135832in}}%
\pgfpathlineto{\pgfqpoint{4.015753in}{2.125792in}}%
\pgfpathlineto{\pgfqpoint{4.008077in}{2.115717in}}%
\pgfpathlineto{\pgfqpoint{4.000396in}{2.105606in}}%
\pgfpathlineto{\pgfqpoint{3.992710in}{2.095459in}}%
\pgfpathclose%
\pgfusepath{fill}%
\end{pgfscope}%
\begin{pgfscope}%
\pgfpathrectangle{\pgfqpoint{1.254980in}{0.150000in}}{\pgfqpoint{5.490039in}{5.490039in}}%
\pgfusepath{clip}%
\pgfsetbuttcap%
\pgfsetroundjoin%
\definecolor{currentfill}{rgb}{0.151918,0.500685,0.557587}%
\pgfsetfillcolor{currentfill}%
\pgfsetfillopacity{0.700000}%
\pgfsetlinewidth{0.000000pt}%
\definecolor{currentstroke}{rgb}{0.000000,0.000000,0.000000}%
\pgfsetstrokecolor{currentstroke}%
\pgfsetdash{}{0pt}%
\pgfpathmoveto{\pgfqpoint{5.142503in}{2.888682in}}%
\pgfpathlineto{\pgfqpoint{5.156277in}{2.897209in}}%
\pgfpathlineto{\pgfqpoint{5.170066in}{2.905893in}}%
\pgfpathlineto{\pgfqpoint{5.183872in}{2.914733in}}%
\pgfpathlineto{\pgfqpoint{5.197694in}{2.923730in}}%
\pgfpathlineto{\pgfqpoint{5.204896in}{2.928802in}}%
\pgfpathlineto{\pgfqpoint{5.212092in}{2.933846in}}%
\pgfpathlineto{\pgfqpoint{5.219282in}{2.938867in}}%
\pgfpathlineto{\pgfqpoint{5.226465in}{2.943869in}}%
\pgfpathlineto{\pgfqpoint{5.212660in}{2.935231in}}%
\pgfpathlineto{\pgfqpoint{5.198872in}{2.926750in}}%
\pgfpathlineto{\pgfqpoint{5.185100in}{2.918425in}}%
\pgfpathlineto{\pgfqpoint{5.171343in}{2.910257in}}%
\pgfpathlineto{\pgfqpoint{5.164142in}{2.904886in}}%
\pgfpathlineto{\pgfqpoint{5.156936in}{2.899502in}}%
\pgfpathlineto{\pgfqpoint{5.149723in}{2.894102in}}%
\pgfpathlineto{\pgfqpoint{5.142503in}{2.888682in}}%
\pgfpathclose%
\pgfusepath{fill}%
\end{pgfscope}%
\begin{pgfscope}%
\pgfpathrectangle{\pgfqpoint{1.254980in}{0.150000in}}{\pgfqpoint{5.490039in}{5.490039in}}%
\pgfusepath{clip}%
\pgfsetbuttcap%
\pgfsetroundjoin%
\definecolor{currentfill}{rgb}{0.278012,0.180367,0.486697}%
\pgfsetfillcolor{currentfill}%
\pgfsetfillopacity{0.700000}%
\pgfsetlinewidth{0.000000pt}%
\definecolor{currentstroke}{rgb}{0.000000,0.000000,0.000000}%
\pgfsetstrokecolor{currentstroke}%
\pgfsetdash{}{0pt}%
\pgfpathmoveto{\pgfqpoint{2.729454in}{2.183350in}}%
\pgfpathlineto{\pgfqpoint{2.742715in}{2.168860in}}%
\pgfpathlineto{\pgfqpoint{2.755970in}{2.154611in}}%
\pgfpathlineto{\pgfqpoint{2.769220in}{2.140603in}}%
\pgfpathlineto{\pgfqpoint{2.782464in}{2.126833in}}%
\pgfpathlineto{\pgfqpoint{2.790704in}{2.130943in}}%
\pgfpathlineto{\pgfqpoint{2.798933in}{2.135215in}}%
\pgfpathlineto{\pgfqpoint{2.807150in}{2.139645in}}%
\pgfpathlineto{\pgfqpoint{2.815356in}{2.144230in}}%
\pgfpathlineto{\pgfqpoint{2.802143in}{2.157633in}}%
\pgfpathlineto{\pgfqpoint{2.788924in}{2.171274in}}%
\pgfpathlineto{\pgfqpoint{2.775700in}{2.185155in}}%
\pgfpathlineto{\pgfqpoint{2.762471in}{2.199277in}}%
\pgfpathlineto{\pgfqpoint{2.754234in}{2.195049in}}%
\pgfpathlineto{\pgfqpoint{2.745986in}{2.190983in}}%
\pgfpathlineto{\pgfqpoint{2.737726in}{2.187082in}}%
\pgfpathlineto{\pgfqpoint{2.729454in}{2.183350in}}%
\pgfpathclose%
\pgfusepath{fill}%
\end{pgfscope}%
\begin{pgfscope}%
\pgfpathrectangle{\pgfqpoint{1.254980in}{0.150000in}}{\pgfqpoint{5.490039in}{5.490039in}}%
\pgfusepath{clip}%
\pgfsetbuttcap%
\pgfsetroundjoin%
\definecolor{currentfill}{rgb}{0.252194,0.269783,0.531579}%
\pgfsetfillcolor{currentfill}%
\pgfsetfillopacity{0.700000}%
\pgfsetlinewidth{0.000000pt}%
\definecolor{currentstroke}{rgb}{0.000000,0.000000,0.000000}%
\pgfsetstrokecolor{currentstroke}%
\pgfsetdash{}{0pt}%
\pgfpathmoveto{\pgfqpoint{2.569772in}{2.376849in}}%
\pgfpathlineto{\pgfqpoint{2.583123in}{2.359294in}}%
\pgfpathlineto{\pgfqpoint{2.596465in}{2.342007in}}%
\pgfpathlineto{\pgfqpoint{2.609798in}{2.324984in}}%
\pgfpathlineto{\pgfqpoint{2.623124in}{2.308225in}}%
\pgfpathlineto{\pgfqpoint{2.631464in}{2.311193in}}%
\pgfpathlineto{\pgfqpoint{2.639792in}{2.314344in}}%
\pgfpathlineto{\pgfqpoint{2.648106in}{2.317676in}}%
\pgfpathlineto{\pgfqpoint{2.656409in}{2.321185in}}%
\pgfpathlineto{\pgfqpoint{2.643119in}{2.337571in}}%
\pgfpathlineto{\pgfqpoint{2.629821in}{2.354219in}}%
\pgfpathlineto{\pgfqpoint{2.616514in}{2.371131in}}%
\pgfpathlineto{\pgfqpoint{2.603199in}{2.388310in}}%
\pgfpathlineto{\pgfqpoint{2.594863in}{2.385165in}}%
\pgfpathlineto{\pgfqpoint{2.586513in}{2.382204in}}%
\pgfpathlineto{\pgfqpoint{2.578149in}{2.379431in}}%
\pgfpathlineto{\pgfqpoint{2.569772in}{2.376849in}}%
\pgfpathclose%
\pgfusepath{fill}%
\end{pgfscope}%
\begin{pgfscope}%
\pgfpathrectangle{\pgfqpoint{1.254980in}{0.150000in}}{\pgfqpoint{5.490039in}{5.490039in}}%
\pgfusepath{clip}%
\pgfsetbuttcap%
\pgfsetroundjoin%
\definecolor{currentfill}{rgb}{0.214298,0.355619,0.551184}%
\pgfsetfillcolor{currentfill}%
\pgfsetfillopacity{0.700000}%
\pgfsetlinewidth{0.000000pt}%
\definecolor{currentstroke}{rgb}{0.000000,0.000000,0.000000}%
\pgfsetstrokecolor{currentstroke}%
\pgfsetdash{}{0pt}%
\pgfpathmoveto{\pgfqpoint{4.609508in}{2.516539in}}%
\pgfpathlineto{\pgfqpoint{4.623013in}{2.522944in}}%
\pgfpathlineto{\pgfqpoint{4.636532in}{2.529509in}}%
\pgfpathlineto{\pgfqpoint{4.650065in}{2.536234in}}%
\pgfpathlineto{\pgfqpoint{4.663610in}{2.543121in}}%
\pgfpathlineto{\pgfqpoint{4.671069in}{2.551189in}}%
\pgfpathlineto{\pgfqpoint{4.678522in}{2.559188in}}%
\pgfpathlineto{\pgfqpoint{4.685968in}{2.567121in}}%
\pgfpathlineto{\pgfqpoint{4.693408in}{2.574990in}}%
\pgfpathlineto{\pgfqpoint{4.679871in}{2.568256in}}%
\pgfpathlineto{\pgfqpoint{4.666348in}{2.561683in}}%
\pgfpathlineto{\pgfqpoint{4.652837in}{2.555271in}}%
\pgfpathlineto{\pgfqpoint{4.639340in}{2.549018in}}%
\pgfpathlineto{\pgfqpoint{4.631891in}{2.540987in}}%
\pgfpathlineto{\pgfqpoint{4.624436in}{2.532897in}}%
\pgfpathlineto{\pgfqpoint{4.616975in}{2.524749in}}%
\pgfpathlineto{\pgfqpoint{4.609508in}{2.516539in}}%
\pgfpathclose%
\pgfusepath{fill}%
\end{pgfscope}%
\begin{pgfscope}%
\pgfpathrectangle{\pgfqpoint{1.254980in}{0.150000in}}{\pgfqpoint{5.490039in}{5.490039in}}%
\pgfusepath{clip}%
\pgfsetbuttcap%
\pgfsetroundjoin%
\definecolor{currentfill}{rgb}{0.277134,0.185228,0.489898}%
\pgfsetfillcolor{currentfill}%
\pgfsetfillopacity{0.700000}%
\pgfsetlinewidth{0.000000pt}%
\definecolor{currentstroke}{rgb}{0.000000,0.000000,0.000000}%
\pgfsetstrokecolor{currentstroke}%
\pgfsetdash{}{0pt}%
\pgfpathmoveto{\pgfqpoint{4.076444in}{2.145083in}}%
\pgfpathlineto{\pgfqpoint{4.089723in}{2.147815in}}%
\pgfpathlineto{\pgfqpoint{4.103010in}{2.150715in}}%
\pgfpathlineto{\pgfqpoint{4.116308in}{2.153781in}}%
\pgfpathlineto{\pgfqpoint{4.129615in}{2.157013in}}%
\pgfpathlineto{\pgfqpoint{4.137269in}{2.167135in}}%
\pgfpathlineto{\pgfqpoint{4.144917in}{2.177207in}}%
\pgfpathlineto{\pgfqpoint{4.152561in}{2.187228in}}%
\pgfpathlineto{\pgfqpoint{4.160199in}{2.197199in}}%
\pgfpathlineto{\pgfqpoint{4.146898in}{2.193918in}}%
\pgfpathlineto{\pgfqpoint{4.133606in}{2.190803in}}%
\pgfpathlineto{\pgfqpoint{4.120325in}{2.187855in}}%
\pgfpathlineto{\pgfqpoint{4.107053in}{2.185074in}}%
\pgfpathlineto{\pgfqpoint{4.099409in}{2.175141in}}%
\pgfpathlineto{\pgfqpoint{4.091759in}{2.165165in}}%
\pgfpathlineto{\pgfqpoint{4.084104in}{2.155146in}}%
\pgfpathlineto{\pgfqpoint{4.076444in}{2.145083in}}%
\pgfpathclose%
\pgfusepath{fill}%
\end{pgfscope}%
\begin{pgfscope}%
\pgfpathrectangle{\pgfqpoint{1.254980in}{0.150000in}}{\pgfqpoint{5.490039in}{5.490039in}}%
\pgfusepath{clip}%
\pgfsetbuttcap%
\pgfsetroundjoin%
\definecolor{currentfill}{rgb}{0.273809,0.031497,0.358853}%
\pgfsetfillcolor{currentfill}%
\pgfsetfillopacity{0.700000}%
\pgfsetlinewidth{0.000000pt}%
\definecolor{currentstroke}{rgb}{0.000000,0.000000,0.000000}%
\pgfsetstrokecolor{currentstroke}%
\pgfsetdash{}{0pt}%
\pgfpathmoveto{\pgfqpoint{3.216169in}{1.888103in}}%
\pgfpathlineto{\pgfqpoint{3.229305in}{1.881258in}}%
\pgfpathlineto{\pgfqpoint{3.242443in}{1.874608in}}%
\pgfpathlineto{\pgfqpoint{3.255582in}{1.868153in}}%
\pgfpathlineto{\pgfqpoint{3.268722in}{1.861892in}}%
\pgfpathlineto{\pgfqpoint{3.276692in}{1.869717in}}%
\pgfpathlineto{\pgfqpoint{3.284654in}{1.877620in}}%
\pgfpathlineto{\pgfqpoint{3.292608in}{1.885601in}}%
\pgfpathlineto{\pgfqpoint{3.300556in}{1.893655in}}%
\pgfpathlineto{\pgfqpoint{3.287434in}{1.899615in}}%
\pgfpathlineto{\pgfqpoint{3.274313in}{1.905769in}}%
\pgfpathlineto{\pgfqpoint{3.261194in}{1.912118in}}%
\pgfpathlineto{\pgfqpoint{3.248077in}{1.918662in}}%
\pgfpathlineto{\pgfqpoint{3.240111in}{1.910898in}}%
\pgfpathlineto{\pgfqpoint{3.232138in}{1.903215in}}%
\pgfpathlineto{\pgfqpoint{3.224157in}{1.895616in}}%
\pgfpathlineto{\pgfqpoint{3.216169in}{1.888103in}}%
\pgfpathclose%
\pgfusepath{fill}%
\end{pgfscope}%
\begin{pgfscope}%
\pgfpathrectangle{\pgfqpoint{1.254980in}{0.150000in}}{\pgfqpoint{5.490039in}{5.490039in}}%
\pgfusepath{clip}%
\pgfsetbuttcap%
\pgfsetroundjoin%
\definecolor{currentfill}{rgb}{0.279566,0.067836,0.391917}%
\pgfsetfillcolor{currentfill}%
\pgfsetfillopacity{0.700000}%
\pgfsetlinewidth{0.000000pt}%
\definecolor{currentstroke}{rgb}{0.000000,0.000000,0.000000}%
\pgfsetstrokecolor{currentstroke}%
\pgfsetdash{}{0pt}%
\pgfpathmoveto{\pgfqpoint{3.657657in}{1.929895in}}%
\pgfpathlineto{\pgfqpoint{3.670820in}{1.928534in}}%
\pgfpathlineto{\pgfqpoint{3.683989in}{1.927349in}}%
\pgfpathlineto{\pgfqpoint{3.697164in}{1.926340in}}%
\pgfpathlineto{\pgfqpoint{3.710346in}{1.925506in}}%
\pgfpathlineto{\pgfqpoint{3.718136in}{1.935501in}}%
\pgfpathlineto{\pgfqpoint{3.725921in}{1.945499in}}%
\pgfpathlineto{\pgfqpoint{3.733701in}{1.955497in}}%
\pgfpathlineto{\pgfqpoint{3.741476in}{1.965495in}}%
\pgfpathlineto{\pgfqpoint{3.728304in}{1.966141in}}%
\pgfpathlineto{\pgfqpoint{3.715139in}{1.966962in}}%
\pgfpathlineto{\pgfqpoint{3.701979in}{1.967958in}}%
\pgfpathlineto{\pgfqpoint{3.688826in}{1.969131in}}%
\pgfpathlineto{\pgfqpoint{3.681042in}{1.959311in}}%
\pgfpathlineto{\pgfqpoint{3.673252in}{1.949498in}}%
\pgfpathlineto{\pgfqpoint{3.665457in}{1.939692in}}%
\pgfpathlineto{\pgfqpoint{3.657657in}{1.929895in}}%
\pgfpathclose%
\pgfusepath{fill}%
\end{pgfscope}%
\begin{pgfscope}%
\pgfpathrectangle{\pgfqpoint{1.254980in}{0.150000in}}{\pgfqpoint{5.490039in}{5.490039in}}%
\pgfusepath{clip}%
\pgfsetbuttcap%
\pgfsetroundjoin%
\definecolor{currentfill}{rgb}{0.281412,0.155834,0.469201}%
\pgfsetfillcolor{currentfill}%
\pgfsetfillopacity{0.700000}%
\pgfsetlinewidth{0.000000pt}%
\definecolor{currentstroke}{rgb}{0.000000,0.000000,0.000000}%
\pgfsetstrokecolor{currentstroke}%
\pgfsetdash{}{0pt}%
\pgfpathmoveto{\pgfqpoint{2.782464in}{2.126833in}}%
\pgfpathlineto{\pgfqpoint{2.795703in}{2.113299in}}%
\pgfpathlineto{\pgfqpoint{2.808937in}{2.100000in}}%
\pgfpathlineto{\pgfqpoint{2.822167in}{2.086934in}}%
\pgfpathlineto{\pgfqpoint{2.835392in}{2.074099in}}%
\pgfpathlineto{\pgfqpoint{2.843601in}{2.078586in}}%
\pgfpathlineto{\pgfqpoint{2.851799in}{2.083228in}}%
\pgfpathlineto{\pgfqpoint{2.859987in}{2.088020in}}%
\pgfpathlineto{\pgfqpoint{2.868164in}{2.092959in}}%
\pgfpathlineto{\pgfqpoint{2.854968in}{2.105429in}}%
\pgfpathlineto{\pgfqpoint{2.841769in}{2.118130in}}%
\pgfpathlineto{\pgfqpoint{2.828565in}{2.131063in}}%
\pgfpathlineto{\pgfqpoint{2.815356in}{2.144230in}}%
\pgfpathlineto{\pgfqpoint{2.807150in}{2.139645in}}%
\pgfpathlineto{\pgfqpoint{2.798933in}{2.135215in}}%
\pgfpathlineto{\pgfqpoint{2.790704in}{2.130943in}}%
\pgfpathlineto{\pgfqpoint{2.782464in}{2.126833in}}%
\pgfpathclose%
\pgfusepath{fill}%
\end{pgfscope}%
\begin{pgfscope}%
\pgfpathrectangle{\pgfqpoint{1.254980in}{0.150000in}}{\pgfqpoint{5.490039in}{5.490039in}}%
\pgfusepath{clip}%
\pgfsetbuttcap%
\pgfsetroundjoin%
\definecolor{currentfill}{rgb}{0.144759,0.519093,0.556572}%
\pgfsetfillcolor{currentfill}%
\pgfsetfillopacity{0.700000}%
\pgfsetlinewidth{0.000000pt}%
\definecolor{currentstroke}{rgb}{0.000000,0.000000,0.000000}%
\pgfsetstrokecolor{currentstroke}%
\pgfsetdash{}{0pt}%
\pgfpathmoveto{\pgfqpoint{5.226465in}{2.943869in}}%
\pgfpathlineto{\pgfqpoint{5.240286in}{2.952662in}}%
\pgfpathlineto{\pgfqpoint{5.254123in}{2.961611in}}%
\pgfpathlineto{\pgfqpoint{5.267977in}{2.970717in}}%
\pgfpathlineto{\pgfqpoint{5.281847in}{2.979979in}}%
\pgfpathlineto{\pgfqpoint{5.289005in}{2.984587in}}%
\pgfpathlineto{\pgfqpoint{5.296157in}{2.989178in}}%
\pgfpathlineto{\pgfqpoint{5.303303in}{2.993755in}}%
\pgfpathlineto{\pgfqpoint{5.310442in}{2.998323in}}%
\pgfpathlineto{\pgfqpoint{5.296591in}{2.989451in}}%
\pgfpathlineto{\pgfqpoint{5.282756in}{2.980735in}}%
\pgfpathlineto{\pgfqpoint{5.268938in}{2.972174in}}%
\pgfpathlineto{\pgfqpoint{5.255136in}{2.963768in}}%
\pgfpathlineto{\pgfqpoint{5.247977in}{2.958801in}}%
\pgfpathlineto{\pgfqpoint{5.240813in}{2.953831in}}%
\pgfpathlineto{\pgfqpoint{5.233642in}{2.948855in}}%
\pgfpathlineto{\pgfqpoint{5.226465in}{2.943869in}}%
\pgfpathclose%
\pgfusepath{fill}%
\end{pgfscope}%
\begin{pgfscope}%
\pgfpathrectangle{\pgfqpoint{1.254980in}{0.150000in}}{\pgfqpoint{5.490039in}{5.490039in}}%
\pgfusepath{clip}%
\pgfsetbuttcap%
\pgfsetroundjoin%
\definecolor{currentfill}{rgb}{0.239346,0.300855,0.540844}%
\pgfsetfillcolor{currentfill}%
\pgfsetfillopacity{0.700000}%
\pgfsetlinewidth{0.000000pt}%
\definecolor{currentstroke}{rgb}{0.000000,0.000000,0.000000}%
\pgfsetstrokecolor{currentstroke}%
\pgfsetdash{}{0pt}%
\pgfpathmoveto{\pgfqpoint{2.516274in}{2.449792in}}%
\pgfpathlineto{\pgfqpoint{2.529663in}{2.431143in}}%
\pgfpathlineto{\pgfqpoint{2.543042in}{2.412771in}}%
\pgfpathlineto{\pgfqpoint{2.556412in}{2.394674in}}%
\pgfpathlineto{\pgfqpoint{2.569772in}{2.376849in}}%
\pgfpathlineto{\pgfqpoint{2.578149in}{2.379431in}}%
\pgfpathlineto{\pgfqpoint{2.586513in}{2.382204in}}%
\pgfpathlineto{\pgfqpoint{2.594863in}{2.385165in}}%
\pgfpathlineto{\pgfqpoint{2.603199in}{2.388310in}}%
\pgfpathlineto{\pgfqpoint{2.589876in}{2.405759in}}%
\pgfpathlineto{\pgfqpoint{2.576543in}{2.423479in}}%
\pgfpathlineto{\pgfqpoint{2.563201in}{2.441473in}}%
\pgfpathlineto{\pgfqpoint{2.549850in}{2.459744in}}%
\pgfpathlineto{\pgfqpoint{2.541476in}{2.456964in}}%
\pgfpathlineto{\pgfqpoint{2.533089in}{2.454377in}}%
\pgfpathlineto{\pgfqpoint{2.524689in}{2.451985in}}%
\pgfpathlineto{\pgfqpoint{2.516274in}{2.449792in}}%
\pgfpathclose%
\pgfusepath{fill}%
\end{pgfscope}%
\begin{pgfscope}%
\pgfpathrectangle{\pgfqpoint{1.254980in}{0.150000in}}{\pgfqpoint{5.490039in}{5.490039in}}%
\pgfusepath{clip}%
\pgfsetbuttcap%
\pgfsetroundjoin%
\definecolor{currentfill}{rgb}{0.272594,0.025563,0.353093}%
\pgfsetfillcolor{currentfill}%
\pgfsetfillopacity{0.700000}%
\pgfsetlinewidth{0.000000pt}%
\definecolor{currentstroke}{rgb}{0.000000,0.000000,0.000000}%
\pgfsetstrokecolor{currentstroke}%
\pgfsetdash{}{0pt}%
\pgfpathmoveto{\pgfqpoint{3.353067in}{1.871729in}}%
\pgfpathlineto{\pgfqpoint{3.366200in}{1.866722in}}%
\pgfpathlineto{\pgfqpoint{3.379337in}{1.861903in}}%
\pgfpathlineto{\pgfqpoint{3.392476in}{1.857271in}}%
\pgfpathlineto{\pgfqpoint{3.405619in}{1.852825in}}%
\pgfpathlineto{\pgfqpoint{3.413527in}{1.861516in}}%
\pgfpathlineto{\pgfqpoint{3.421429in}{1.870261in}}%
\pgfpathlineto{\pgfqpoint{3.429324in}{1.879058in}}%
\pgfpathlineto{\pgfqpoint{3.437213in}{1.887905in}}%
\pgfpathlineto{\pgfqpoint{3.424086in}{1.892078in}}%
\pgfpathlineto{\pgfqpoint{3.410962in}{1.896438in}}%
\pgfpathlineto{\pgfqpoint{3.397841in}{1.900985in}}%
\pgfpathlineto{\pgfqpoint{3.384723in}{1.905720in}}%
\pgfpathlineto{\pgfqpoint{3.376819in}{1.897135in}}%
\pgfpathlineto{\pgfqpoint{3.368908in}{1.888606in}}%
\pgfpathlineto{\pgfqpoint{3.360991in}{1.880137in}}%
\pgfpathlineto{\pgfqpoint{3.353067in}{1.871729in}}%
\pgfpathclose%
\pgfusepath{fill}%
\end{pgfscope}%
\begin{pgfscope}%
\pgfpathrectangle{\pgfqpoint{1.254980in}{0.150000in}}{\pgfqpoint{5.490039in}{5.490039in}}%
\pgfusepath{clip}%
\pgfsetbuttcap%
\pgfsetroundjoin%
\definecolor{currentfill}{rgb}{0.270595,0.214069,0.507052}%
\pgfsetfillcolor{currentfill}%
\pgfsetfillopacity{0.700000}%
\pgfsetlinewidth{0.000000pt}%
\definecolor{currentstroke}{rgb}{0.000000,0.000000,0.000000}%
\pgfsetstrokecolor{currentstroke}%
\pgfsetdash{}{0pt}%
\pgfpathmoveto{\pgfqpoint{4.160199in}{2.197199in}}%
\pgfpathlineto{\pgfqpoint{4.173510in}{2.200646in}}%
\pgfpathlineto{\pgfqpoint{4.186832in}{2.204259in}}%
\pgfpathlineto{\pgfqpoint{4.200164in}{2.208038in}}%
\pgfpathlineto{\pgfqpoint{4.213506in}{2.211982in}}%
\pgfpathlineto{\pgfqpoint{4.221134in}{2.221933in}}%
\pgfpathlineto{\pgfqpoint{4.228756in}{2.231826in}}%
\pgfpathlineto{\pgfqpoint{4.236373in}{2.241663in}}%
\pgfpathlineto{\pgfqpoint{4.243985in}{2.251442in}}%
\pgfpathlineto{\pgfqpoint{4.230648in}{2.247478in}}%
\pgfpathlineto{\pgfqpoint{4.217322in}{2.243679in}}%
\pgfpathlineto{\pgfqpoint{4.204007in}{2.240045in}}%
\pgfpathlineto{\pgfqpoint{4.190701in}{2.236578in}}%
\pgfpathlineto{\pgfqpoint{4.183084in}{2.226808in}}%
\pgfpathlineto{\pgfqpoint{4.175460in}{2.216989in}}%
\pgfpathlineto{\pgfqpoint{4.167832in}{2.207119in}}%
\pgfpathlineto{\pgfqpoint{4.160199in}{2.197199in}}%
\pgfpathclose%
\pgfusepath{fill}%
\end{pgfscope}%
\begin{pgfscope}%
\pgfpathrectangle{\pgfqpoint{1.254980in}{0.150000in}}{\pgfqpoint{5.490039in}{5.490039in}}%
\pgfusepath{clip}%
\pgfsetbuttcap%
\pgfsetroundjoin%
\definecolor{currentfill}{rgb}{0.277018,0.050344,0.375715}%
\pgfsetfillcolor{currentfill}%
\pgfsetfillopacity{0.700000}%
\pgfsetlinewidth{0.000000pt}%
\definecolor{currentstroke}{rgb}{0.000000,0.000000,0.000000}%
\pgfsetstrokecolor{currentstroke}%
\pgfsetdash{}{0pt}%
\pgfpathmoveto{\pgfqpoint{3.573761in}{1.898918in}}%
\pgfpathlineto{\pgfqpoint{3.586914in}{1.896630in}}%
\pgfpathlineto{\pgfqpoint{3.600071in}{1.894520in}}%
\pgfpathlineto{\pgfqpoint{3.613234in}{1.892588in}}%
\pgfpathlineto{\pgfqpoint{3.626402in}{1.890834in}}%
\pgfpathlineto{\pgfqpoint{3.634224in}{1.900577in}}%
\pgfpathlineto{\pgfqpoint{3.642040in}{1.910336in}}%
\pgfpathlineto{\pgfqpoint{3.649851in}{1.920109in}}%
\pgfpathlineto{\pgfqpoint{3.657657in}{1.929895in}}%
\pgfpathlineto{\pgfqpoint{3.644500in}{1.931433in}}%
\pgfpathlineto{\pgfqpoint{3.631348in}{1.933148in}}%
\pgfpathlineto{\pgfqpoint{3.618202in}{1.935042in}}%
\pgfpathlineto{\pgfqpoint{3.605061in}{1.937114in}}%
\pgfpathlineto{\pgfqpoint{3.597244in}{1.927534in}}%
\pgfpathlineto{\pgfqpoint{3.589422in}{1.917973in}}%
\pgfpathlineto{\pgfqpoint{3.581594in}{1.908434in}}%
\pgfpathlineto{\pgfqpoint{3.573761in}{1.898918in}}%
\pgfpathclose%
\pgfusepath{fill}%
\end{pgfscope}%
\begin{pgfscope}%
\pgfpathrectangle{\pgfqpoint{1.254980in}{0.150000in}}{\pgfqpoint{5.490039in}{5.490039in}}%
\pgfusepath{clip}%
\pgfsetbuttcap%
\pgfsetroundjoin%
\definecolor{currentfill}{rgb}{0.203063,0.379716,0.553925}%
\pgfsetfillcolor{currentfill}%
\pgfsetfillopacity{0.700000}%
\pgfsetlinewidth{0.000000pt}%
\definecolor{currentstroke}{rgb}{0.000000,0.000000,0.000000}%
\pgfsetstrokecolor{currentstroke}%
\pgfsetdash{}{0pt}%
\pgfpathmoveto{\pgfqpoint{4.693408in}{2.574990in}}%
\pgfpathlineto{\pgfqpoint{4.706959in}{2.581883in}}%
\pgfpathlineto{\pgfqpoint{4.720524in}{2.588937in}}%
\pgfpathlineto{\pgfqpoint{4.734102in}{2.596151in}}%
\pgfpathlineto{\pgfqpoint{4.747695in}{2.603525in}}%
\pgfpathlineto{\pgfqpoint{4.755119in}{2.611160in}}%
\pgfpathlineto{\pgfqpoint{4.762538in}{2.618728in}}%
\pgfpathlineto{\pgfqpoint{4.769950in}{2.626231in}}%
\pgfpathlineto{\pgfqpoint{4.777356in}{2.633671in}}%
\pgfpathlineto{\pgfqpoint{4.763773in}{2.626480in}}%
\pgfpathlineto{\pgfqpoint{4.750204in}{2.619449in}}%
\pgfpathlineto{\pgfqpoint{4.736649in}{2.612577in}}%
\pgfpathlineto{\pgfqpoint{4.723108in}{2.605865in}}%
\pgfpathlineto{\pgfqpoint{4.715692in}{2.598231in}}%
\pgfpathlineto{\pgfqpoint{4.708270in}{2.590543in}}%
\pgfpathlineto{\pgfqpoint{4.700842in}{2.582796in}}%
\pgfpathlineto{\pgfqpoint{4.693408in}{2.574990in}}%
\pgfpathclose%
\pgfusepath{fill}%
\end{pgfscope}%
\begin{pgfscope}%
\pgfpathrectangle{\pgfqpoint{1.254980in}{0.150000in}}{\pgfqpoint{5.490039in}{5.490039in}}%
\pgfusepath{clip}%
\pgfsetbuttcap%
\pgfsetroundjoin%
\definecolor{currentfill}{rgb}{0.136408,0.541173,0.554483}%
\pgfsetfillcolor{currentfill}%
\pgfsetfillopacity{0.700000}%
\pgfsetlinewidth{0.000000pt}%
\definecolor{currentstroke}{rgb}{0.000000,0.000000,0.000000}%
\pgfsetstrokecolor{currentstroke}%
\pgfsetdash{}{0pt}%
\pgfpathmoveto{\pgfqpoint{5.310442in}{2.998323in}}%
\pgfpathlineto{\pgfqpoint{5.324310in}{3.007351in}}%
\pgfpathlineto{\pgfqpoint{5.338195in}{3.016534in}}%
\pgfpathlineto{\pgfqpoint{5.352096in}{3.025873in}}%
\pgfpathlineto{\pgfqpoint{5.366015in}{3.035368in}}%
\pgfpathlineto{\pgfqpoint{5.373128in}{3.039523in}}%
\pgfpathlineto{\pgfqpoint{5.380234in}{3.043671in}}%
\pgfpathlineto{\pgfqpoint{5.387335in}{3.047818in}}%
\pgfpathlineto{\pgfqpoint{5.394430in}{3.051968in}}%
\pgfpathlineto{\pgfqpoint{5.380532in}{3.042893in}}%
\pgfpathlineto{\pgfqpoint{5.366651in}{3.033973in}}%
\pgfpathlineto{\pgfqpoint{5.352787in}{3.025208in}}%
\pgfpathlineto{\pgfqpoint{5.338940in}{3.016599in}}%
\pgfpathlineto{\pgfqpoint{5.331824in}{3.012020in}}%
\pgfpathlineto{\pgfqpoint{5.324703in}{3.007451in}}%
\pgfpathlineto{\pgfqpoint{5.317576in}{3.002887in}}%
\pgfpathlineto{\pgfqpoint{5.310442in}{2.998323in}}%
\pgfpathclose%
\pgfusepath{fill}%
\end{pgfscope}%
\begin{pgfscope}%
\pgfpathrectangle{\pgfqpoint{1.254980in}{0.150000in}}{\pgfqpoint{5.490039in}{5.490039in}}%
\pgfusepath{clip}%
\pgfsetbuttcap%
\pgfsetroundjoin%
\definecolor{currentfill}{rgb}{0.277941,0.056324,0.381191}%
\pgfsetfillcolor{currentfill}%
\pgfsetfillopacity{0.700000}%
\pgfsetlinewidth{0.000000pt}%
\definecolor{currentstroke}{rgb}{0.000000,0.000000,0.000000}%
\pgfsetstrokecolor{currentstroke}%
\pgfsetdash{}{0pt}%
\pgfpathmoveto{\pgfqpoint{3.078894in}{1.923543in}}%
\pgfpathlineto{\pgfqpoint{3.092050in}{1.914763in}}%
\pgfpathlineto{\pgfqpoint{3.105206in}{1.906188in}}%
\pgfpathlineto{\pgfqpoint{3.118361in}{1.897816in}}%
\pgfpathlineto{\pgfqpoint{3.131516in}{1.889646in}}%
\pgfpathlineto{\pgfqpoint{3.139557in}{1.896445in}}%
\pgfpathlineto{\pgfqpoint{3.147590in}{1.903349in}}%
\pgfpathlineto{\pgfqpoint{3.155614in}{1.910356in}}%
\pgfpathlineto{\pgfqpoint{3.163631in}{1.917461in}}%
\pgfpathlineto{\pgfqpoint{3.150497in}{1.925301in}}%
\pgfpathlineto{\pgfqpoint{3.137364in}{1.933342in}}%
\pgfpathlineto{\pgfqpoint{3.124231in}{1.941588in}}%
\pgfpathlineto{\pgfqpoint{3.111097in}{1.950037in}}%
\pgfpathlineto{\pgfqpoint{3.103059in}{1.943252in}}%
\pgfpathlineto{\pgfqpoint{3.095013in}{1.936572in}}%
\pgfpathlineto{\pgfqpoint{3.086958in}{1.930002in}}%
\pgfpathlineto{\pgfqpoint{3.078894in}{1.923543in}}%
\pgfpathclose%
\pgfusepath{fill}%
\end{pgfscope}%
\begin{pgfscope}%
\pgfpathrectangle{\pgfqpoint{1.254980in}{0.150000in}}{\pgfqpoint{5.490039in}{5.490039in}}%
\pgfusepath{clip}%
\pgfsetbuttcap%
\pgfsetroundjoin%
\definecolor{currentfill}{rgb}{0.283072,0.130895,0.449241}%
\pgfsetfillcolor{currentfill}%
\pgfsetfillopacity{0.700000}%
\pgfsetlinewidth{0.000000pt}%
\definecolor{currentstroke}{rgb}{0.000000,0.000000,0.000000}%
\pgfsetstrokecolor{currentstroke}%
\pgfsetdash{}{0pt}%
\pgfpathmoveto{\pgfqpoint{2.835392in}{2.074099in}}%
\pgfpathlineto{\pgfqpoint{2.848612in}{2.061494in}}%
\pgfpathlineto{\pgfqpoint{2.861829in}{2.049117in}}%
\pgfpathlineto{\pgfqpoint{2.875041in}{2.036966in}}%
\pgfpathlineto{\pgfqpoint{2.888250in}{2.025039in}}%
\pgfpathlineto{\pgfqpoint{2.896430in}{2.029901in}}%
\pgfpathlineto{\pgfqpoint{2.904600in}{2.034909in}}%
\pgfpathlineto{\pgfqpoint{2.912759in}{2.040062in}}%
\pgfpathlineto{\pgfqpoint{2.920908in}{2.045354in}}%
\pgfpathlineto{\pgfqpoint{2.907727in}{2.056917in}}%
\pgfpathlineto{\pgfqpoint{2.894543in}{2.068705in}}%
\pgfpathlineto{\pgfqpoint{2.881355in}{2.080718in}}%
\pgfpathlineto{\pgfqpoint{2.868164in}{2.092959in}}%
\pgfpathlineto{\pgfqpoint{2.859987in}{2.088020in}}%
\pgfpathlineto{\pgfqpoint{2.851799in}{2.083228in}}%
\pgfpathlineto{\pgfqpoint{2.843601in}{2.078586in}}%
\pgfpathlineto{\pgfqpoint{2.835392in}{2.074099in}}%
\pgfpathclose%
\pgfusepath{fill}%
\end{pgfscope}%
\begin{pgfscope}%
\pgfpathrectangle{\pgfqpoint{1.254980in}{0.150000in}}{\pgfqpoint{5.490039in}{5.490039in}}%
\pgfusepath{clip}%
\pgfsetbuttcap%
\pgfsetroundjoin%
\definecolor{currentfill}{rgb}{0.223925,0.334994,0.548053}%
\pgfsetfillcolor{currentfill}%
\pgfsetfillopacity{0.700000}%
\pgfsetlinewidth{0.000000pt}%
\definecolor{currentstroke}{rgb}{0.000000,0.000000,0.000000}%
\pgfsetstrokecolor{currentstroke}%
\pgfsetdash{}{0pt}%
\pgfpathmoveto{\pgfqpoint{2.462610in}{2.527215in}}%
\pgfpathlineto{\pgfqpoint{2.476043in}{2.507430in}}%
\pgfpathlineto{\pgfqpoint{2.489464in}{2.487933in}}%
\pgfpathlineto{\pgfqpoint{2.502874in}{2.468722in}}%
\pgfpathlineto{\pgfqpoint{2.516274in}{2.449792in}}%
\pgfpathlineto{\pgfqpoint{2.524689in}{2.451985in}}%
\pgfpathlineto{\pgfqpoint{2.533089in}{2.454377in}}%
\pgfpathlineto{\pgfqpoint{2.541476in}{2.456964in}}%
\pgfpathlineto{\pgfqpoint{2.549850in}{2.459744in}}%
\pgfpathlineto{\pgfqpoint{2.536488in}{2.478293in}}%
\pgfpathlineto{\pgfqpoint{2.523116in}{2.497125in}}%
\pgfpathlineto{\pgfqpoint{2.509734in}{2.516241in}}%
\pgfpathlineto{\pgfqpoint{2.496341in}{2.535643in}}%
\pgfpathlineto{\pgfqpoint{2.487930in}{2.533233in}}%
\pgfpathlineto{\pgfqpoint{2.479505in}{2.531023in}}%
\pgfpathlineto{\pgfqpoint{2.471065in}{2.529015in}}%
\pgfpathlineto{\pgfqpoint{2.462610in}{2.527215in}}%
\pgfpathclose%
\pgfusepath{fill}%
\end{pgfscope}%
\begin{pgfscope}%
\pgfpathrectangle{\pgfqpoint{1.254980in}{0.150000in}}{\pgfqpoint{5.490039in}{5.490039in}}%
\pgfusepath{clip}%
\pgfsetbuttcap%
\pgfsetroundjoin%
\definecolor{currentfill}{rgb}{0.262138,0.242286,0.520837}%
\pgfsetfillcolor{currentfill}%
\pgfsetfillopacity{0.700000}%
\pgfsetlinewidth{0.000000pt}%
\definecolor{currentstroke}{rgb}{0.000000,0.000000,0.000000}%
\pgfsetstrokecolor{currentstroke}%
\pgfsetdash{}{0pt}%
\pgfpathmoveto{\pgfqpoint{4.243985in}{2.251442in}}%
\pgfpathlineto{\pgfqpoint{4.257332in}{2.255571in}}%
\pgfpathlineto{\pgfqpoint{4.270690in}{2.259865in}}%
\pgfpathlineto{\pgfqpoint{4.284059in}{2.264323in}}%
\pgfpathlineto{\pgfqpoint{4.297439in}{2.268945in}}%
\pgfpathlineto{\pgfqpoint{4.305040in}{2.278670in}}%
\pgfpathlineto{\pgfqpoint{4.312635in}{2.288331in}}%
\pgfpathlineto{\pgfqpoint{4.320225in}{2.297930in}}%
\pgfpathlineto{\pgfqpoint{4.327810in}{2.307466in}}%
\pgfpathlineto{\pgfqpoint{4.314436in}{2.302853in}}%
\pgfpathlineto{\pgfqpoint{4.301072in}{2.298403in}}%
\pgfpathlineto{\pgfqpoint{4.287720in}{2.294117in}}%
\pgfpathlineto{\pgfqpoint{4.274379in}{2.289996in}}%
\pgfpathlineto{\pgfqpoint{4.266788in}{2.280441in}}%
\pgfpathlineto{\pgfqpoint{4.259192in}{2.270831in}}%
\pgfpathlineto{\pgfqpoint{4.251591in}{2.261165in}}%
\pgfpathlineto{\pgfqpoint{4.243985in}{2.251442in}}%
\pgfpathclose%
\pgfusepath{fill}%
\end{pgfscope}%
\begin{pgfscope}%
\pgfpathrectangle{\pgfqpoint{1.254980in}{0.150000in}}{\pgfqpoint{5.490039in}{5.490039in}}%
\pgfusepath{clip}%
\pgfsetbuttcap%
\pgfsetroundjoin%
\definecolor{currentfill}{rgb}{0.128729,0.563265,0.551229}%
\pgfsetfillcolor{currentfill}%
\pgfsetfillopacity{0.700000}%
\pgfsetlinewidth{0.000000pt}%
\definecolor{currentstroke}{rgb}{0.000000,0.000000,0.000000}%
\pgfsetstrokecolor{currentstroke}%
\pgfsetdash{}{0pt}%
\pgfpathmoveto{\pgfqpoint{5.394430in}{3.051968in}}%
\pgfpathlineto{\pgfqpoint{5.408344in}{3.061197in}}%
\pgfpathlineto{\pgfqpoint{5.422276in}{3.070583in}}%
\pgfpathlineto{\pgfqpoint{5.436225in}{3.080123in}}%
\pgfpathlineto{\pgfqpoint{5.450191in}{3.089819in}}%
\pgfpathlineto{\pgfqpoint{5.457258in}{3.093537in}}%
\pgfpathlineto{\pgfqpoint{5.464318in}{3.097261in}}%
\pgfpathlineto{\pgfqpoint{5.471373in}{3.100995in}}%
\pgfpathlineto{\pgfqpoint{5.478421in}{3.104745in}}%
\pgfpathlineto{\pgfqpoint{5.464478in}{3.095500in}}%
\pgfpathlineto{\pgfqpoint{5.450552in}{3.086409in}}%
\pgfpathlineto{\pgfqpoint{5.436643in}{3.077472in}}%
\pgfpathlineto{\pgfqpoint{5.422750in}{3.068690in}}%
\pgfpathlineto{\pgfqpoint{5.415679in}{3.064481in}}%
\pgfpathlineto{\pgfqpoint{5.408601in}{3.060294in}}%
\pgfpathlineto{\pgfqpoint{5.401518in}{3.056125in}}%
\pgfpathlineto{\pgfqpoint{5.394430in}{3.051968in}}%
\pgfpathclose%
\pgfusepath{fill}%
\end{pgfscope}%
\begin{pgfscope}%
\pgfpathrectangle{\pgfqpoint{1.254980in}{0.150000in}}{\pgfqpoint{5.490039in}{5.490039in}}%
\pgfusepath{clip}%
\pgfsetbuttcap%
\pgfsetroundjoin%
\definecolor{currentfill}{rgb}{0.190631,0.407061,0.556089}%
\pgfsetfillcolor{currentfill}%
\pgfsetfillopacity{0.700000}%
\pgfsetlinewidth{0.000000pt}%
\definecolor{currentstroke}{rgb}{0.000000,0.000000,0.000000}%
\pgfsetstrokecolor{currentstroke}%
\pgfsetdash{}{0pt}%
\pgfpathmoveto{\pgfqpoint{4.777356in}{2.633671in}}%
\pgfpathlineto{\pgfqpoint{4.790953in}{2.641022in}}%
\pgfpathlineto{\pgfqpoint{4.804565in}{2.648532in}}%
\pgfpathlineto{\pgfqpoint{4.818190in}{2.656202in}}%
\pgfpathlineto{\pgfqpoint{4.831831in}{2.664031in}}%
\pgfpathlineto{\pgfqpoint{4.839220in}{2.671210in}}%
\pgfpathlineto{\pgfqpoint{4.846603in}{2.678325in}}%
\pgfpathlineto{\pgfqpoint{4.853979in}{2.685377in}}%
\pgfpathlineto{\pgfqpoint{4.861349in}{2.692370in}}%
\pgfpathlineto{\pgfqpoint{4.847720in}{2.684753in}}%
\pgfpathlineto{\pgfqpoint{4.834105in}{2.677296in}}%
\pgfpathlineto{\pgfqpoint{4.820504in}{2.669997in}}%
\pgfpathlineto{\pgfqpoint{4.806917in}{2.662857in}}%
\pgfpathlineto{\pgfqpoint{4.799536in}{2.655642in}}%
\pgfpathlineto{\pgfqpoint{4.792149in}{2.648375in}}%
\pgfpathlineto{\pgfqpoint{4.784756in}{2.641052in}}%
\pgfpathlineto{\pgfqpoint{4.777356in}{2.633671in}}%
\pgfpathclose%
\pgfusepath{fill}%
\end{pgfscope}%
\begin{pgfscope}%
\pgfpathrectangle{\pgfqpoint{1.254980in}{0.150000in}}{\pgfqpoint{5.490039in}{5.490039in}}%
\pgfusepath{clip}%
\pgfsetbuttcap%
\pgfsetroundjoin%
\definecolor{currentfill}{rgb}{0.283091,0.110553,0.431554}%
\pgfsetfillcolor{currentfill}%
\pgfsetfillopacity{0.700000}%
\pgfsetlinewidth{0.000000pt}%
\definecolor{currentstroke}{rgb}{0.000000,0.000000,0.000000}%
\pgfsetstrokecolor{currentstroke}%
\pgfsetdash{}{0pt}%
\pgfpathmoveto{\pgfqpoint{2.888250in}{2.025039in}}%
\pgfpathlineto{\pgfqpoint{2.901456in}{2.013335in}}%
\pgfpathlineto{\pgfqpoint{2.914658in}{2.001854in}}%
\pgfpathlineto{\pgfqpoint{2.927858in}{1.990592in}}%
\pgfpathlineto{\pgfqpoint{2.941054in}{1.979548in}}%
\pgfpathlineto{\pgfqpoint{2.949206in}{1.984783in}}%
\pgfpathlineto{\pgfqpoint{2.957348in}{1.990157in}}%
\pgfpathlineto{\pgfqpoint{2.965480in}{1.995668in}}%
\pgfpathlineto{\pgfqpoint{2.973603in}{2.001312in}}%
\pgfpathlineto{\pgfqpoint{2.960433in}{2.011994in}}%
\pgfpathlineto{\pgfqpoint{2.947261in}{2.022894in}}%
\pgfpathlineto{\pgfqpoint{2.934086in}{2.034013in}}%
\pgfpathlineto{\pgfqpoint{2.920908in}{2.045354in}}%
\pgfpathlineto{\pgfqpoint{2.912759in}{2.040062in}}%
\pgfpathlineto{\pgfqpoint{2.904600in}{2.034909in}}%
\pgfpathlineto{\pgfqpoint{2.896430in}{2.029901in}}%
\pgfpathlineto{\pgfqpoint{2.888250in}{2.025039in}}%
\pgfpathclose%
\pgfusepath{fill}%
\end{pgfscope}%
\begin{pgfscope}%
\pgfpathrectangle{\pgfqpoint{1.254980in}{0.150000in}}{\pgfqpoint{5.490039in}{5.490039in}}%
\pgfusepath{clip}%
\pgfsetbuttcap%
\pgfsetroundjoin%
\definecolor{currentfill}{rgb}{0.274952,0.037752,0.364543}%
\pgfsetfillcolor{currentfill}%
\pgfsetfillopacity{0.700000}%
\pgfsetlinewidth{0.000000pt}%
\definecolor{currentstroke}{rgb}{0.000000,0.000000,0.000000}%
\pgfsetstrokecolor{currentstroke}%
\pgfsetdash{}{0pt}%
\pgfpathmoveto{\pgfqpoint{3.489759in}{1.873056in}}%
\pgfpathlineto{\pgfqpoint{3.502906in}{1.869802in}}%
\pgfpathlineto{\pgfqpoint{3.516057in}{1.866729in}}%
\pgfpathlineto{\pgfqpoint{3.529212in}{1.863837in}}%
\pgfpathlineto{\pgfqpoint{3.542372in}{1.861126in}}%
\pgfpathlineto{\pgfqpoint{3.550228in}{1.870529in}}%
\pgfpathlineto{\pgfqpoint{3.558078in}{1.879963in}}%
\pgfpathlineto{\pgfqpoint{3.565922in}{1.889427in}}%
\pgfpathlineto{\pgfqpoint{3.573761in}{1.898918in}}%
\pgfpathlineto{\pgfqpoint{3.560614in}{1.901386in}}%
\pgfpathlineto{\pgfqpoint{3.547471in}{1.904034in}}%
\pgfpathlineto{\pgfqpoint{3.534333in}{1.906863in}}%
\pgfpathlineto{\pgfqpoint{3.521200in}{1.909874in}}%
\pgfpathlineto{\pgfqpoint{3.513348in}{1.900616in}}%
\pgfpathlineto{\pgfqpoint{3.505491in}{1.891393in}}%
\pgfpathlineto{\pgfqpoint{3.497628in}{1.882205in}}%
\pgfpathlineto{\pgfqpoint{3.489759in}{1.873056in}}%
\pgfpathclose%
\pgfusepath{fill}%
\end{pgfscope}%
\begin{pgfscope}%
\pgfpathrectangle{\pgfqpoint{1.254980in}{0.150000in}}{\pgfqpoint{5.490039in}{5.490039in}}%
\pgfusepath{clip}%
\pgfsetbuttcap%
\pgfsetroundjoin%
\definecolor{currentfill}{rgb}{0.122606,0.585371,0.546557}%
\pgfsetfillcolor{currentfill}%
\pgfsetfillopacity{0.700000}%
\pgfsetlinewidth{0.000000pt}%
\definecolor{currentstroke}{rgb}{0.000000,0.000000,0.000000}%
\pgfsetstrokecolor{currentstroke}%
\pgfsetdash{}{0pt}%
\pgfpathmoveto{\pgfqpoint{5.478421in}{3.104745in}}%
\pgfpathlineto{\pgfqpoint{5.492382in}{3.114145in}}%
\pgfpathlineto{\pgfqpoint{5.506360in}{3.123700in}}%
\pgfpathlineto{\pgfqpoint{5.520356in}{3.133410in}}%
\pgfpathlineto{\pgfqpoint{5.534370in}{3.143275in}}%
\pgfpathlineto{\pgfqpoint{5.541389in}{3.146577in}}%
\pgfpathlineto{\pgfqpoint{5.548402in}{3.149898in}}%
\pgfpathlineto{\pgfqpoint{5.555410in}{3.153243in}}%
\pgfpathlineto{\pgfqpoint{5.562412in}{3.156619in}}%
\pgfpathlineto{\pgfqpoint{5.548424in}{3.147234in}}%
\pgfpathlineto{\pgfqpoint{5.534453in}{3.138004in}}%
\pgfpathlineto{\pgfqpoint{5.520499in}{3.128928in}}%
\pgfpathlineto{\pgfqpoint{5.506563in}{3.120005in}}%
\pgfpathlineto{\pgfqpoint{5.499535in}{3.116141in}}%
\pgfpathlineto{\pgfqpoint{5.492503in}{3.112313in}}%
\pgfpathlineto{\pgfqpoint{5.485465in}{3.108516in}}%
\pgfpathlineto{\pgfqpoint{5.478421in}{3.104745in}}%
\pgfpathclose%
\pgfusepath{fill}%
\end{pgfscope}%
\begin{pgfscope}%
\pgfpathrectangle{\pgfqpoint{1.254980in}{0.150000in}}{\pgfqpoint{5.490039in}{5.490039in}}%
\pgfusepath{clip}%
\pgfsetbuttcap%
\pgfsetroundjoin%
\definecolor{currentfill}{rgb}{0.252194,0.269783,0.531579}%
\pgfsetfillcolor{currentfill}%
\pgfsetfillopacity{0.700000}%
\pgfsetlinewidth{0.000000pt}%
\definecolor{currentstroke}{rgb}{0.000000,0.000000,0.000000}%
\pgfsetstrokecolor{currentstroke}%
\pgfsetdash{}{0pt}%
\pgfpathmoveto{\pgfqpoint{4.327810in}{2.307466in}}%
\pgfpathlineto{\pgfqpoint{4.341196in}{2.312244in}}%
\pgfpathlineto{\pgfqpoint{4.354593in}{2.317185in}}%
\pgfpathlineto{\pgfqpoint{4.368002in}{2.322290in}}%
\pgfpathlineto{\pgfqpoint{4.381422in}{2.327558in}}%
\pgfpathlineto{\pgfqpoint{4.388995in}{2.337006in}}%
\pgfpathlineto{\pgfqpoint{4.396563in}{2.346386in}}%
\pgfpathlineto{\pgfqpoint{4.404125in}{2.355699in}}%
\pgfpathlineto{\pgfqpoint{4.411682in}{2.364945in}}%
\pgfpathlineto{\pgfqpoint{4.398267in}{2.359714in}}%
\pgfpathlineto{\pgfqpoint{4.384865in}{2.354647in}}%
\pgfpathlineto{\pgfqpoint{4.371474in}{2.349742in}}%
\pgfpathlineto{\pgfqpoint{4.358094in}{2.345001in}}%
\pgfpathlineto{\pgfqpoint{4.350531in}{2.335708in}}%
\pgfpathlineto{\pgfqpoint{4.342963in}{2.326354in}}%
\pgfpathlineto{\pgfqpoint{4.335389in}{2.316941in}}%
\pgfpathlineto{\pgfqpoint{4.327810in}{2.307466in}}%
\pgfpathclose%
\pgfusepath{fill}%
\end{pgfscope}%
\begin{pgfscope}%
\pgfpathrectangle{\pgfqpoint{1.254980in}{0.150000in}}{\pgfqpoint{5.490039in}{5.490039in}}%
\pgfusepath{clip}%
\pgfsetbuttcap%
\pgfsetroundjoin%
\definecolor{currentfill}{rgb}{0.119738,0.603785,0.541400}%
\pgfsetfillcolor{currentfill}%
\pgfsetfillopacity{0.700000}%
\pgfsetlinewidth{0.000000pt}%
\definecolor{currentstroke}{rgb}{0.000000,0.000000,0.000000}%
\pgfsetstrokecolor{currentstroke}%
\pgfsetdash{}{0pt}%
\pgfpathmoveto{\pgfqpoint{5.562412in}{3.156619in}}%
\pgfpathlineto{\pgfqpoint{5.576419in}{3.166157in}}%
\pgfpathlineto{\pgfqpoint{5.590442in}{3.175850in}}%
\pgfpathlineto{\pgfqpoint{5.604484in}{3.185698in}}%
\pgfpathlineto{\pgfqpoint{5.618544in}{3.195699in}}%
\pgfpathlineto{\pgfqpoint{5.625515in}{3.198611in}}%
\pgfpathlineto{\pgfqpoint{5.632481in}{3.201556in}}%
\pgfpathlineto{\pgfqpoint{5.639442in}{3.204542in}}%
\pgfpathlineto{\pgfqpoint{5.646398in}{3.207572in}}%
\pgfpathlineto{\pgfqpoint{5.632365in}{3.198080in}}%
\pgfpathlineto{\pgfqpoint{5.618350in}{3.188743in}}%
\pgfpathlineto{\pgfqpoint{5.604352in}{3.179558in}}%
\pgfpathlineto{\pgfqpoint{5.590372in}{3.170527in}}%
\pgfpathlineto{\pgfqpoint{5.583389in}{3.166978in}}%
\pgfpathlineto{\pgfqpoint{5.576402in}{3.163481in}}%
\pgfpathlineto{\pgfqpoint{5.569410in}{3.160029in}}%
\pgfpathlineto{\pgfqpoint{5.562412in}{3.156619in}}%
\pgfpathclose%
\pgfusepath{fill}%
\end{pgfscope}%
\begin{pgfscope}%
\pgfpathrectangle{\pgfqpoint{1.254980in}{0.150000in}}{\pgfqpoint{5.490039in}{5.490039in}}%
\pgfusepath{clip}%
\pgfsetbuttcap%
\pgfsetroundjoin%
\definecolor{currentfill}{rgb}{0.206756,0.371758,0.553117}%
\pgfsetfillcolor{currentfill}%
\pgfsetfillopacity{0.700000}%
\pgfsetlinewidth{0.000000pt}%
\definecolor{currentstroke}{rgb}{0.000000,0.000000,0.000000}%
\pgfsetstrokecolor{currentstroke}%
\pgfsetdash{}{0pt}%
\pgfpathmoveto{\pgfqpoint{2.408763in}{2.609288in}}%
\pgfpathlineto{\pgfqpoint{2.422243in}{2.588324in}}%
\pgfpathlineto{\pgfqpoint{2.435711in}{2.567659in}}%
\pgfpathlineto{\pgfqpoint{2.449166in}{2.547290in}}%
\pgfpathlineto{\pgfqpoint{2.462610in}{2.527215in}}%
\pgfpathlineto{\pgfqpoint{2.471065in}{2.529015in}}%
\pgfpathlineto{\pgfqpoint{2.479505in}{2.531023in}}%
\pgfpathlineto{\pgfqpoint{2.487930in}{2.533233in}}%
\pgfpathlineto{\pgfqpoint{2.496341in}{2.535643in}}%
\pgfpathlineto{\pgfqpoint{2.482937in}{2.555336in}}%
\pgfpathlineto{\pgfqpoint{2.469521in}{2.575321in}}%
\pgfpathlineto{\pgfqpoint{2.456094in}{2.595602in}}%
\pgfpathlineto{\pgfqpoint{2.442655in}{2.616181in}}%
\pgfpathlineto{\pgfqpoint{2.434204in}{2.614143in}}%
\pgfpathlineto{\pgfqpoint{2.425739in}{2.612313in}}%
\pgfpathlineto{\pgfqpoint{2.417259in}{2.610693in}}%
\pgfpathlineto{\pgfqpoint{2.408763in}{2.609288in}}%
\pgfpathclose%
\pgfusepath{fill}%
\end{pgfscope}%
\begin{pgfscope}%
\pgfpathrectangle{\pgfqpoint{1.254980in}{0.150000in}}{\pgfqpoint{5.490039in}{5.490039in}}%
\pgfusepath{clip}%
\pgfsetbuttcap%
\pgfsetroundjoin%
\definecolor{currentfill}{rgb}{0.272594,0.025563,0.353093}%
\pgfsetfillcolor{currentfill}%
\pgfsetfillopacity{0.700000}%
\pgfsetlinewidth{0.000000pt}%
\definecolor{currentstroke}{rgb}{0.000000,0.000000,0.000000}%
\pgfsetstrokecolor{currentstroke}%
\pgfsetdash{}{0pt}%
\pgfpathmoveto{\pgfqpoint{3.268722in}{1.861892in}}%
\pgfpathlineto{\pgfqpoint{3.281864in}{1.855824in}}%
\pgfpathlineto{\pgfqpoint{3.295008in}{1.849948in}}%
\pgfpathlineto{\pgfqpoint{3.308154in}{1.844262in}}%
\pgfpathlineto{\pgfqpoint{3.321303in}{1.838766in}}%
\pgfpathlineto{\pgfqpoint{3.329254in}{1.846901in}}%
\pgfpathlineto{\pgfqpoint{3.337198in}{1.855108in}}%
\pgfpathlineto{\pgfqpoint{3.345136in}{1.863385in}}%
\pgfpathlineto{\pgfqpoint{3.353067in}{1.871729in}}%
\pgfpathlineto{\pgfqpoint{3.339936in}{1.876925in}}%
\pgfpathlineto{\pgfqpoint{3.326807in}{1.882310in}}%
\pgfpathlineto{\pgfqpoint{3.313681in}{1.887887in}}%
\pgfpathlineto{\pgfqpoint{3.300556in}{1.893655in}}%
\pgfpathlineto{\pgfqpoint{3.292608in}{1.885601in}}%
\pgfpathlineto{\pgfqpoint{3.284654in}{1.877620in}}%
\pgfpathlineto{\pgfqpoint{3.276692in}{1.869717in}}%
\pgfpathlineto{\pgfqpoint{3.268722in}{1.861892in}}%
\pgfpathclose%
\pgfusepath{fill}%
\end{pgfscope}%
\begin{pgfscope}%
\pgfpathrectangle{\pgfqpoint{1.254980in}{0.150000in}}{\pgfqpoint{5.490039in}{5.490039in}}%
\pgfusepath{clip}%
\pgfsetbuttcap%
\pgfsetroundjoin%
\definecolor{currentfill}{rgb}{0.180629,0.429975,0.557282}%
\pgfsetfillcolor{currentfill}%
\pgfsetfillopacity{0.700000}%
\pgfsetlinewidth{0.000000pt}%
\definecolor{currentstroke}{rgb}{0.000000,0.000000,0.000000}%
\pgfsetstrokecolor{currentstroke}%
\pgfsetdash{}{0pt}%
\pgfpathmoveto{\pgfqpoint{4.861349in}{2.692370in}}%
\pgfpathlineto{\pgfqpoint{4.874994in}{2.700146in}}%
\pgfpathlineto{\pgfqpoint{4.888653in}{2.708080in}}%
\pgfpathlineto{\pgfqpoint{4.902327in}{2.716174in}}%
\pgfpathlineto{\pgfqpoint{4.916016in}{2.724426in}}%
\pgfpathlineto{\pgfqpoint{4.923368in}{2.731131in}}%
\pgfpathlineto{\pgfqpoint{4.930713in}{2.737775in}}%
\pgfpathlineto{\pgfqpoint{4.938052in}{2.744361in}}%
\pgfpathlineto{\pgfqpoint{4.945384in}{2.750891in}}%
\pgfpathlineto{\pgfqpoint{4.931707in}{2.742881in}}%
\pgfpathlineto{\pgfqpoint{4.918045in}{2.735030in}}%
\pgfpathlineto{\pgfqpoint{4.904398in}{2.727336in}}%
\pgfpathlineto{\pgfqpoint{4.890765in}{2.719802in}}%
\pgfpathlineto{\pgfqpoint{4.883421in}{2.713019in}}%
\pgfpathlineto{\pgfqpoint{4.876070in}{2.706188in}}%
\pgfpathlineto{\pgfqpoint{4.868713in}{2.699306in}}%
\pgfpathlineto{\pgfqpoint{4.861349in}{2.692370in}}%
\pgfpathclose%
\pgfusepath{fill}%
\end{pgfscope}%
\begin{pgfscope}%
\pgfpathrectangle{\pgfqpoint{1.254980in}{0.150000in}}{\pgfqpoint{5.490039in}{5.490039in}}%
\pgfusepath{clip}%
\pgfsetbuttcap%
\pgfsetroundjoin%
\definecolor{currentfill}{rgb}{0.120081,0.622161,0.534946}%
\pgfsetfillcolor{currentfill}%
\pgfsetfillopacity{0.700000}%
\pgfsetlinewidth{0.000000pt}%
\definecolor{currentstroke}{rgb}{0.000000,0.000000,0.000000}%
\pgfsetstrokecolor{currentstroke}%
\pgfsetdash{}{0pt}%
\pgfpathmoveto{\pgfqpoint{5.646398in}{3.207572in}}%
\pgfpathlineto{\pgfqpoint{5.660448in}{3.217217in}}%
\pgfpathlineto{\pgfqpoint{5.674517in}{3.227016in}}%
\pgfpathlineto{\pgfqpoint{5.688604in}{3.236969in}}%
\pgfpathlineto{\pgfqpoint{5.702710in}{3.247076in}}%
\pgfpathlineto{\pgfqpoint{5.709632in}{3.249628in}}%
\pgfpathlineto{\pgfqpoint{5.716550in}{3.252230in}}%
\pgfpathlineto{\pgfqpoint{5.723464in}{3.254888in}}%
\pgfpathlineto{\pgfqpoint{5.730373in}{3.257608in}}%
\pgfpathlineto{\pgfqpoint{5.716296in}{3.248042in}}%
\pgfpathlineto{\pgfqpoint{5.702238in}{3.238628in}}%
\pgfpathlineto{\pgfqpoint{5.688198in}{3.229367in}}%
\pgfpathlineto{\pgfqpoint{5.674176in}{3.220259in}}%
\pgfpathlineto{\pgfqpoint{5.667237in}{3.216991in}}%
\pgfpathlineto{\pgfqpoint{5.660295in}{3.213791in}}%
\pgfpathlineto{\pgfqpoint{5.653349in}{3.210653in}}%
\pgfpathlineto{\pgfqpoint{5.646398in}{3.207572in}}%
\pgfpathclose%
\pgfusepath{fill}%
\end{pgfscope}%
\begin{pgfscope}%
\pgfpathrectangle{\pgfqpoint{1.254980in}{0.150000in}}{\pgfqpoint{5.490039in}{5.490039in}}%
\pgfusepath{clip}%
\pgfsetbuttcap%
\pgfsetroundjoin%
\definecolor{currentfill}{rgb}{0.241237,0.296485,0.539709}%
\pgfsetfillcolor{currentfill}%
\pgfsetfillopacity{0.700000}%
\pgfsetlinewidth{0.000000pt}%
\definecolor{currentstroke}{rgb}{0.000000,0.000000,0.000000}%
\pgfsetstrokecolor{currentstroke}%
\pgfsetdash{}{0pt}%
\pgfpathmoveto{\pgfqpoint{4.411682in}{2.364945in}}%
\pgfpathlineto{\pgfqpoint{4.425108in}{2.370339in}}%
\pgfpathlineto{\pgfqpoint{4.438546in}{2.375895in}}%
\pgfpathlineto{\pgfqpoint{4.451997in}{2.381614in}}%
\pgfpathlineto{\pgfqpoint{4.465460in}{2.387495in}}%
\pgfpathlineto{\pgfqpoint{4.473005in}{2.396620in}}%
\pgfpathlineto{\pgfqpoint{4.480544in}{2.405674in}}%
\pgfpathlineto{\pgfqpoint{4.488077in}{2.414657in}}%
\pgfpathlineto{\pgfqpoint{4.495604in}{2.423570in}}%
\pgfpathlineto{\pgfqpoint{4.482147in}{2.417755in}}%
\pgfpathlineto{\pgfqpoint{4.468703in}{2.412102in}}%
\pgfpathlineto{\pgfqpoint{4.455272in}{2.406612in}}%
\pgfpathlineto{\pgfqpoint{4.441852in}{2.401284in}}%
\pgfpathlineto{\pgfqpoint{4.434318in}{2.392294in}}%
\pgfpathlineto{\pgfqpoint{4.426778in}{2.383242in}}%
\pgfpathlineto{\pgfqpoint{4.419232in}{2.374126in}}%
\pgfpathlineto{\pgfqpoint{4.411682in}{2.364945in}}%
\pgfpathclose%
\pgfusepath{fill}%
\end{pgfscope}%
\begin{pgfscope}%
\pgfpathrectangle{\pgfqpoint{1.254980in}{0.150000in}}{\pgfqpoint{5.490039in}{5.490039in}}%
\pgfusepath{clip}%
\pgfsetbuttcap%
\pgfsetroundjoin%
\definecolor{currentfill}{rgb}{0.191090,0.708366,0.482284}%
\pgfsetfillcolor{currentfill}%
\pgfsetfillopacity{0.700000}%
\pgfsetlinewidth{0.000000pt}%
\definecolor{currentstroke}{rgb}{0.000000,0.000000,0.000000}%
\pgfsetstrokecolor{currentstroke}%
\pgfsetdash{}{0pt}%
\pgfpathmoveto{\pgfqpoint{6.066115in}{3.449390in}}%
\pgfpathlineto{\pgfqpoint{6.080371in}{3.459093in}}%
\pgfpathlineto{\pgfqpoint{6.094646in}{3.468947in}}%
\pgfpathlineto{\pgfqpoint{6.108941in}{3.478953in}}%
\pgfpathlineto{\pgfqpoint{6.115639in}{3.480521in}}%
\pgfpathlineto{\pgfqpoint{6.122337in}{3.482240in}}%
\pgfpathlineto{\pgfqpoint{6.129034in}{3.484117in}}%
\pgfpathlineto{\pgfqpoint{6.135732in}{3.486158in}}%
\pgfpathlineto{\pgfqpoint{6.121478in}{3.476839in}}%
\pgfpathlineto{\pgfqpoint{6.107243in}{3.467670in}}%
\pgfpathlineto{\pgfqpoint{6.093026in}{3.458651in}}%
\pgfpathlineto{\pgfqpoint{6.086298in}{3.456089in}}%
\pgfpathlineto{\pgfqpoint{6.079570in}{3.453696in}}%
\pgfpathlineto{\pgfqpoint{6.072843in}{3.451465in}}%
\pgfpathlineto{\pgfqpoint{6.066115in}{3.449390in}}%
\pgfpathclose%
\pgfusepath{fill}%
\end{pgfscope}%
\begin{pgfscope}%
\pgfpathrectangle{\pgfqpoint{1.254980in}{0.150000in}}{\pgfqpoint{5.490039in}{5.490039in}}%
\pgfusepath{clip}%
\pgfsetbuttcap%
\pgfsetroundjoin%
\definecolor{currentfill}{rgb}{0.276022,0.044167,0.370164}%
\pgfsetfillcolor{currentfill}%
\pgfsetfillopacity{0.700000}%
\pgfsetlinewidth{0.000000pt}%
\definecolor{currentstroke}{rgb}{0.000000,0.000000,0.000000}%
\pgfsetstrokecolor{currentstroke}%
\pgfsetdash{}{0pt}%
\pgfpathmoveto{\pgfqpoint{3.131516in}{1.889646in}}%
\pgfpathlineto{\pgfqpoint{3.144671in}{1.881678in}}%
\pgfpathlineto{\pgfqpoint{3.157826in}{1.873910in}}%
\pgfpathlineto{\pgfqpoint{3.170981in}{1.866341in}}%
\pgfpathlineto{\pgfqpoint{3.184137in}{1.858970in}}%
\pgfpathlineto{\pgfqpoint{3.192157in}{1.866109in}}%
\pgfpathlineto{\pgfqpoint{3.200169in}{1.873346in}}%
\pgfpathlineto{\pgfqpoint{3.208173in}{1.880678in}}%
\pgfpathlineto{\pgfqpoint{3.216169in}{1.888103in}}%
\pgfpathlineto{\pgfqpoint{3.203033in}{1.895144in}}%
\pgfpathlineto{\pgfqpoint{3.189898in}{1.902384in}}%
\pgfpathlineto{\pgfqpoint{3.176764in}{1.909823in}}%
\pgfpathlineto{\pgfqpoint{3.163631in}{1.917461in}}%
\pgfpathlineto{\pgfqpoint{3.155614in}{1.910356in}}%
\pgfpathlineto{\pgfqpoint{3.147590in}{1.903349in}}%
\pgfpathlineto{\pgfqpoint{3.139557in}{1.896445in}}%
\pgfpathlineto{\pgfqpoint{3.131516in}{1.889646in}}%
\pgfpathclose%
\pgfusepath{fill}%
\end{pgfscope}%
\begin{pgfscope}%
\pgfpathrectangle{\pgfqpoint{1.254980in}{0.150000in}}{\pgfqpoint{5.490039in}{5.490039in}}%
\pgfusepath{clip}%
\pgfsetbuttcap%
\pgfsetroundjoin%
\definecolor{currentfill}{rgb}{0.124780,0.640461,0.527068}%
\pgfsetfillcolor{currentfill}%
\pgfsetfillopacity{0.700000}%
\pgfsetlinewidth{0.000000pt}%
\definecolor{currentstroke}{rgb}{0.000000,0.000000,0.000000}%
\pgfsetstrokecolor{currentstroke}%
\pgfsetdash{}{0pt}%
\pgfpathmoveto{\pgfqpoint{5.730373in}{3.257608in}}%
\pgfpathlineto{\pgfqpoint{5.744467in}{3.267328in}}%
\pgfpathlineto{\pgfqpoint{5.758580in}{3.277201in}}%
\pgfpathlineto{\pgfqpoint{5.772711in}{3.287228in}}%
\pgfpathlineto{\pgfqpoint{5.786861in}{3.297408in}}%
\pgfpathlineto{\pgfqpoint{5.793735in}{3.299637in}}%
\pgfpathlineto{\pgfqpoint{5.800605in}{3.301933in}}%
\pgfpathlineto{\pgfqpoint{5.807472in}{3.304303in}}%
\pgfpathlineto{\pgfqpoint{5.814334in}{3.306753in}}%
\pgfpathlineto{\pgfqpoint{5.800216in}{3.297142in}}%
\pgfpathlineto{\pgfqpoint{5.786115in}{3.287685in}}%
\pgfpathlineto{\pgfqpoint{5.772034in}{3.278379in}}%
\pgfpathlineto{\pgfqpoint{5.757970in}{3.269226in}}%
\pgfpathlineto{\pgfqpoint{5.751076in}{3.266199in}}%
\pgfpathlineto{\pgfqpoint{5.744178in}{3.263258in}}%
\pgfpathlineto{\pgfqpoint{5.737277in}{3.260396in}}%
\pgfpathlineto{\pgfqpoint{5.730373in}{3.257608in}}%
\pgfpathclose%
\pgfusepath{fill}%
\end{pgfscope}%
\begin{pgfscope}%
\pgfpathrectangle{\pgfqpoint{1.254980in}{0.150000in}}{\pgfqpoint{5.490039in}{5.490039in}}%
\pgfusepath{clip}%
\pgfsetbuttcap%
\pgfsetroundjoin%
\definecolor{currentfill}{rgb}{0.281924,0.089666,0.412415}%
\pgfsetfillcolor{currentfill}%
\pgfsetfillopacity{0.700000}%
\pgfsetlinewidth{0.000000pt}%
\definecolor{currentstroke}{rgb}{0.000000,0.000000,0.000000}%
\pgfsetstrokecolor{currentstroke}%
\pgfsetdash{}{0pt}%
\pgfpathmoveto{\pgfqpoint{2.941054in}{1.979548in}}%
\pgfpathlineto{\pgfqpoint{2.954248in}{1.968722in}}%
\pgfpathlineto{\pgfqpoint{2.967440in}{1.958111in}}%
\pgfpathlineto{\pgfqpoint{2.980629in}{1.947714in}}%
\pgfpathlineto{\pgfqpoint{2.993816in}{1.937531in}}%
\pgfpathlineto{\pgfqpoint{3.001942in}{1.943137in}}%
\pgfpathlineto{\pgfqpoint{3.010057in}{1.948876in}}%
\pgfpathlineto{\pgfqpoint{3.018164in}{1.954744in}}%
\pgfpathlineto{\pgfqpoint{3.026260in}{1.960738in}}%
\pgfpathlineto{\pgfqpoint{3.013099in}{1.970561in}}%
\pgfpathlineto{\pgfqpoint{2.999935in}{1.980597in}}%
\pgfpathlineto{\pgfqpoint{2.986770in}{1.990847in}}%
\pgfpathlineto{\pgfqpoint{2.973603in}{2.001312in}}%
\pgfpathlineto{\pgfqpoint{2.965480in}{1.995668in}}%
\pgfpathlineto{\pgfqpoint{2.957348in}{1.990157in}}%
\pgfpathlineto{\pgfqpoint{2.949206in}{1.984783in}}%
\pgfpathlineto{\pgfqpoint{2.941054in}{1.979548in}}%
\pgfpathclose%
\pgfusepath{fill}%
\end{pgfscope}%
\begin{pgfscope}%
\pgfpathrectangle{\pgfqpoint{1.254980in}{0.150000in}}{\pgfqpoint{5.490039in}{5.490039in}}%
\pgfusepath{clip}%
\pgfsetbuttcap%
\pgfsetroundjoin%
\definecolor{currentfill}{rgb}{0.137339,0.662252,0.515571}%
\pgfsetfillcolor{currentfill}%
\pgfsetfillopacity{0.700000}%
\pgfsetlinewidth{0.000000pt}%
\definecolor{currentstroke}{rgb}{0.000000,0.000000,0.000000}%
\pgfsetstrokecolor{currentstroke}%
\pgfsetdash{}{0pt}%
\pgfpathmoveto{\pgfqpoint{5.814334in}{3.306753in}}%
\pgfpathlineto{\pgfqpoint{5.828471in}{3.316515in}}%
\pgfpathlineto{\pgfqpoint{5.842627in}{3.326431in}}%
\pgfpathlineto{\pgfqpoint{5.856801in}{3.336500in}}%
\pgfpathlineto{\pgfqpoint{5.870994in}{3.346722in}}%
\pgfpathlineto{\pgfqpoint{5.877820in}{3.348669in}}%
\pgfpathlineto{\pgfqpoint{5.884643in}{3.350701in}}%
\pgfpathlineto{\pgfqpoint{5.891462in}{3.352826in}}%
\pgfpathlineto{\pgfqpoint{5.898279in}{3.355050in}}%
\pgfpathlineto{\pgfqpoint{5.884120in}{3.345428in}}%
\pgfpathlineto{\pgfqpoint{5.869979in}{3.335957in}}%
\pgfpathlineto{\pgfqpoint{5.855856in}{3.326639in}}%
\pgfpathlineto{\pgfqpoint{5.841752in}{3.317473in}}%
\pgfpathlineto{\pgfqpoint{5.834902in}{3.314641in}}%
\pgfpathlineto{\pgfqpoint{5.828049in}{3.311915in}}%
\pgfpathlineto{\pgfqpoint{5.821193in}{3.309288in}}%
\pgfpathlineto{\pgfqpoint{5.814334in}{3.306753in}}%
\pgfpathclose%
\pgfusepath{fill}%
\end{pgfscope}%
\begin{pgfscope}%
\pgfpathrectangle{\pgfqpoint{1.254980in}{0.150000in}}{\pgfqpoint{5.490039in}{5.490039in}}%
\pgfusepath{clip}%
\pgfsetbuttcap%
\pgfsetroundjoin%
\definecolor{currentfill}{rgb}{0.272594,0.025563,0.353093}%
\pgfsetfillcolor{currentfill}%
\pgfsetfillopacity{0.700000}%
\pgfsetlinewidth{0.000000pt}%
\definecolor{currentstroke}{rgb}{0.000000,0.000000,0.000000}%
\pgfsetstrokecolor{currentstroke}%
\pgfsetdash{}{0pt}%
\pgfpathmoveto{\pgfqpoint{3.405619in}{1.852825in}}%
\pgfpathlineto{\pgfqpoint{3.418764in}{1.848565in}}%
\pgfpathlineto{\pgfqpoint{3.431914in}{1.844490in}}%
\pgfpathlineto{\pgfqpoint{3.445066in}{1.840599in}}%
\pgfpathlineto{\pgfqpoint{3.458223in}{1.836890in}}%
\pgfpathlineto{\pgfqpoint{3.466116in}{1.845863in}}%
\pgfpathlineto{\pgfqpoint{3.474003in}{1.854883in}}%
\pgfpathlineto{\pgfqpoint{3.481884in}{1.863948in}}%
\pgfpathlineto{\pgfqpoint{3.489759in}{1.873056in}}%
\pgfpathlineto{\pgfqpoint{3.476617in}{1.876493in}}%
\pgfpathlineto{\pgfqpoint{3.463479in}{1.880113in}}%
\pgfpathlineto{\pgfqpoint{3.450344in}{1.883916in}}%
\pgfpathlineto{\pgfqpoint{3.437213in}{1.887905in}}%
\pgfpathlineto{\pgfqpoint{3.429324in}{1.879058in}}%
\pgfpathlineto{\pgfqpoint{3.421429in}{1.870261in}}%
\pgfpathlineto{\pgfqpoint{3.413527in}{1.861516in}}%
\pgfpathlineto{\pgfqpoint{3.405619in}{1.852825in}}%
\pgfpathclose%
\pgfusepath{fill}%
\end{pgfscope}%
\begin{pgfscope}%
\pgfpathrectangle{\pgfqpoint{1.254980in}{0.150000in}}{\pgfqpoint{5.490039in}{5.490039in}}%
\pgfusepath{clip}%
\pgfsetbuttcap%
\pgfsetroundjoin%
\definecolor{currentfill}{rgb}{0.153894,0.680203,0.504172}%
\pgfsetfillcolor{currentfill}%
\pgfsetfillopacity{0.700000}%
\pgfsetlinewidth{0.000000pt}%
\definecolor{currentstroke}{rgb}{0.000000,0.000000,0.000000}%
\pgfsetstrokecolor{currentstroke}%
\pgfsetdash{}{0pt}%
\pgfpathmoveto{\pgfqpoint{5.898279in}{3.355050in}}%
\pgfpathlineto{\pgfqpoint{5.912457in}{3.364824in}}%
\pgfpathlineto{\pgfqpoint{5.926654in}{3.374751in}}%
\pgfpathlineto{\pgfqpoint{5.940870in}{3.384830in}}%
\pgfpathlineto{\pgfqpoint{5.955106in}{3.395062in}}%
\pgfpathlineto{\pgfqpoint{5.961884in}{3.396773in}}%
\pgfpathlineto{\pgfqpoint{5.968660in}{3.398590in}}%
\pgfpathlineto{\pgfqpoint{5.975434in}{3.400519in}}%
\pgfpathlineto{\pgfqpoint{5.982206in}{3.402567in}}%
\pgfpathlineto{\pgfqpoint{5.968007in}{3.392964in}}%
\pgfpathlineto{\pgfqpoint{5.953827in}{3.383513in}}%
\pgfpathlineto{\pgfqpoint{5.939665in}{3.374213in}}%
\pgfpathlineto{\pgfqpoint{5.925522in}{3.365064in}}%
\pgfpathlineto{\pgfqpoint{5.918714in}{3.362379in}}%
\pgfpathlineto{\pgfqpoint{5.911905in}{3.359820in}}%
\pgfpathlineto{\pgfqpoint{5.905093in}{3.357379in}}%
\pgfpathlineto{\pgfqpoint{5.898279in}{3.355050in}}%
\pgfpathclose%
\pgfusepath{fill}%
\end{pgfscope}%
\begin{pgfscope}%
\pgfpathrectangle{\pgfqpoint{1.254980in}{0.150000in}}{\pgfqpoint{5.490039in}{5.490039in}}%
\pgfusepath{clip}%
\pgfsetbuttcap%
\pgfsetroundjoin%
\definecolor{currentfill}{rgb}{0.169646,0.456262,0.558030}%
\pgfsetfillcolor{currentfill}%
\pgfsetfillopacity{0.700000}%
\pgfsetlinewidth{0.000000pt}%
\definecolor{currentstroke}{rgb}{0.000000,0.000000,0.000000}%
\pgfsetstrokecolor{currentstroke}%
\pgfsetdash{}{0pt}%
\pgfpathmoveto{\pgfqpoint{4.945384in}{2.750891in}}%
\pgfpathlineto{\pgfqpoint{4.959077in}{2.759060in}}%
\pgfpathlineto{\pgfqpoint{4.972784in}{2.767387in}}%
\pgfpathlineto{\pgfqpoint{4.986507in}{2.775872in}}%
\pgfpathlineto{\pgfqpoint{5.000245in}{2.784515in}}%
\pgfpathlineto{\pgfqpoint{5.007558in}{2.790732in}}%
\pgfpathlineto{\pgfqpoint{5.014864in}{2.796893in}}%
\pgfpathlineto{\pgfqpoint{5.022164in}{2.803001in}}%
\pgfpathlineto{\pgfqpoint{5.029457in}{2.809059in}}%
\pgfpathlineto{\pgfqpoint{5.015732in}{2.800688in}}%
\pgfpathlineto{\pgfqpoint{5.002022in}{2.792475in}}%
\pgfpathlineto{\pgfqpoint{4.988328in}{2.784419in}}%
\pgfpathlineto{\pgfqpoint{4.974648in}{2.776521in}}%
\pgfpathlineto{\pgfqpoint{4.967342in}{2.770181in}}%
\pgfpathlineto{\pgfqpoint{4.960029in}{2.763799in}}%
\pgfpathlineto{\pgfqpoint{4.952710in}{2.757370in}}%
\pgfpathlineto{\pgfqpoint{4.945384in}{2.750891in}}%
\pgfpathclose%
\pgfusepath{fill}%
\end{pgfscope}%
\begin{pgfscope}%
\pgfpathrectangle{\pgfqpoint{1.254980in}{0.150000in}}{\pgfqpoint{5.490039in}{5.490039in}}%
\pgfusepath{clip}%
\pgfsetbuttcap%
\pgfsetroundjoin%
\definecolor{currentfill}{rgb}{0.170948,0.694384,0.493803}%
\pgfsetfillcolor{currentfill}%
\pgfsetfillopacity{0.700000}%
\pgfsetlinewidth{0.000000pt}%
\definecolor{currentstroke}{rgb}{0.000000,0.000000,0.000000}%
\pgfsetstrokecolor{currentstroke}%
\pgfsetdash{}{0pt}%
\pgfpathmoveto{\pgfqpoint{5.982206in}{3.402567in}}%
\pgfpathlineto{\pgfqpoint{5.996424in}{3.412321in}}%
\pgfpathlineto{\pgfqpoint{6.010661in}{3.422227in}}%
\pgfpathlineto{\pgfqpoint{6.024917in}{3.432285in}}%
\pgfpathlineto{\pgfqpoint{6.039193in}{3.442496in}}%
\pgfpathlineto{\pgfqpoint{6.045926in}{3.444022in}}%
\pgfpathlineto{\pgfqpoint{6.052656in}{3.445676in}}%
\pgfpathlineto{\pgfqpoint{6.059386in}{3.447462in}}%
\pgfpathlineto{\pgfqpoint{6.066115in}{3.449390in}}%
\pgfpathlineto{\pgfqpoint{6.051877in}{3.439838in}}%
\pgfpathlineto{\pgfqpoint{6.037659in}{3.430437in}}%
\pgfpathlineto{\pgfqpoint{6.023460in}{3.421187in}}%
\pgfpathlineto{\pgfqpoint{6.009280in}{3.412088in}}%
\pgfpathlineto{\pgfqpoint{6.002513in}{3.409494in}}%
\pgfpathlineto{\pgfqpoint{5.995745in}{3.407047in}}%
\pgfpathlineto{\pgfqpoint{5.988976in}{3.404741in}}%
\pgfpathlineto{\pgfqpoint{5.982206in}{3.402567in}}%
\pgfpathclose%
\pgfusepath{fill}%
\end{pgfscope}%
\begin{pgfscope}%
\pgfpathrectangle{\pgfqpoint{1.254980in}{0.150000in}}{\pgfqpoint{5.490039in}{5.490039in}}%
\pgfusepath{clip}%
\pgfsetbuttcap%
\pgfsetroundjoin%
\definecolor{currentfill}{rgb}{0.282656,0.100196,0.422160}%
\pgfsetfillcolor{currentfill}%
\pgfsetfillopacity{0.700000}%
\pgfsetlinewidth{0.000000pt}%
\definecolor{currentstroke}{rgb}{0.000000,0.000000,0.000000}%
\pgfsetstrokecolor{currentstroke}%
\pgfsetdash{}{0pt}%
\pgfpathmoveto{\pgfqpoint{3.794230in}{1.964653in}}%
\pgfpathlineto{\pgfqpoint{3.807437in}{1.964875in}}%
\pgfpathlineto{\pgfqpoint{3.820650in}{1.965269in}}%
\pgfpathlineto{\pgfqpoint{3.833871in}{1.965834in}}%
\pgfpathlineto{\pgfqpoint{3.847099in}{1.966571in}}%
\pgfpathlineto{\pgfqpoint{3.854852in}{1.976908in}}%
\pgfpathlineto{\pgfqpoint{3.862600in}{1.987228in}}%
\pgfpathlineto{\pgfqpoint{3.870342in}{1.997529in}}%
\pgfpathlineto{\pgfqpoint{3.878080in}{2.007810in}}%
\pgfpathlineto{\pgfqpoint{3.864859in}{2.006913in}}%
\pgfpathlineto{\pgfqpoint{3.851646in}{2.006187in}}%
\pgfpathlineto{\pgfqpoint{3.838441in}{2.005633in}}%
\pgfpathlineto{\pgfqpoint{3.825243in}{2.005250in}}%
\pgfpathlineto{\pgfqpoint{3.817498in}{1.995119in}}%
\pgfpathlineto{\pgfqpoint{3.809747in}{1.984975in}}%
\pgfpathlineto{\pgfqpoint{3.801991in}{1.974819in}}%
\pgfpathlineto{\pgfqpoint{3.794230in}{1.964653in}}%
\pgfpathclose%
\pgfusepath{fill}%
\end{pgfscope}%
\begin{pgfscope}%
\pgfpathrectangle{\pgfqpoint{1.254980in}{0.150000in}}{\pgfqpoint{5.490039in}{5.490039in}}%
\pgfusepath{clip}%
\pgfsetbuttcap%
\pgfsetroundjoin%
\definecolor{currentfill}{rgb}{0.283187,0.125848,0.444960}%
\pgfsetfillcolor{currentfill}%
\pgfsetfillopacity{0.700000}%
\pgfsetlinewidth{0.000000pt}%
\definecolor{currentstroke}{rgb}{0.000000,0.000000,0.000000}%
\pgfsetstrokecolor{currentstroke}%
\pgfsetdash{}{0pt}%
\pgfpathmoveto{\pgfqpoint{3.878080in}{2.007810in}}%
\pgfpathlineto{\pgfqpoint{3.891309in}{2.008878in}}%
\pgfpathlineto{\pgfqpoint{3.904545in}{2.010116in}}%
\pgfpathlineto{\pgfqpoint{3.917790in}{2.011524in}}%
\pgfpathlineto{\pgfqpoint{3.931043in}{2.013101in}}%
\pgfpathlineto{\pgfqpoint{3.938769in}{2.023504in}}%
\pgfpathlineto{\pgfqpoint{3.946489in}{2.033878in}}%
\pgfpathlineto{\pgfqpoint{3.954205in}{2.044222in}}%
\pgfpathlineto{\pgfqpoint{3.961916in}{2.054535in}}%
\pgfpathlineto{\pgfqpoint{3.948670in}{2.052825in}}%
\pgfpathlineto{\pgfqpoint{3.935432in}{2.051285in}}%
\pgfpathlineto{\pgfqpoint{3.922203in}{2.049914in}}%
\pgfpathlineto{\pgfqpoint{3.908982in}{2.048714in}}%
\pgfpathlineto{\pgfqpoint{3.901264in}{2.038523in}}%
\pgfpathlineto{\pgfqpoint{3.893541in}{2.028308in}}%
\pgfpathlineto{\pgfqpoint{3.885813in}{2.018070in}}%
\pgfpathlineto{\pgfqpoint{3.878080in}{2.007810in}}%
\pgfpathclose%
\pgfusepath{fill}%
\end{pgfscope}%
\begin{pgfscope}%
\pgfpathrectangle{\pgfqpoint{1.254980in}{0.150000in}}{\pgfqpoint{5.490039in}{5.490039in}}%
\pgfusepath{clip}%
\pgfsetbuttcap%
\pgfsetroundjoin%
\definecolor{currentfill}{rgb}{0.227802,0.326594,0.546532}%
\pgfsetfillcolor{currentfill}%
\pgfsetfillopacity{0.700000}%
\pgfsetlinewidth{0.000000pt}%
\definecolor{currentstroke}{rgb}{0.000000,0.000000,0.000000}%
\pgfsetstrokecolor{currentstroke}%
\pgfsetdash{}{0pt}%
\pgfpathmoveto{\pgfqpoint{4.495604in}{2.423570in}}%
\pgfpathlineto{\pgfqpoint{4.509073in}{2.429547in}}%
\pgfpathlineto{\pgfqpoint{4.522555in}{2.435686in}}%
\pgfpathlineto{\pgfqpoint{4.536049in}{2.441987in}}%
\pgfpathlineto{\pgfqpoint{4.549557in}{2.448449in}}%
\pgfpathlineto{\pgfqpoint{4.557071in}{2.457209in}}%
\pgfpathlineto{\pgfqpoint{4.564580in}{2.465896in}}%
\pgfpathlineto{\pgfqpoint{4.572083in}{2.474510in}}%
\pgfpathlineto{\pgfqpoint{4.579580in}{2.483052in}}%
\pgfpathlineto{\pgfqpoint{4.566080in}{2.476685in}}%
\pgfpathlineto{\pgfqpoint{4.552592in}{2.470480in}}%
\pgfpathlineto{\pgfqpoint{4.539118in}{2.464436in}}%
\pgfpathlineto{\pgfqpoint{4.525656in}{2.458553in}}%
\pgfpathlineto{\pgfqpoint{4.518151in}{2.449905in}}%
\pgfpathlineto{\pgfqpoint{4.510641in}{2.441193in}}%
\pgfpathlineto{\pgfqpoint{4.503126in}{2.432415in}}%
\pgfpathlineto{\pgfqpoint{4.495604in}{2.423570in}}%
\pgfpathclose%
\pgfusepath{fill}%
\end{pgfscope}%
\begin{pgfscope}%
\pgfpathrectangle{\pgfqpoint{1.254980in}{0.150000in}}{\pgfqpoint{5.490039in}{5.490039in}}%
\pgfusepath{clip}%
\pgfsetbuttcap%
\pgfsetroundjoin%
\definecolor{currentfill}{rgb}{0.190631,0.407061,0.556089}%
\pgfsetfillcolor{currentfill}%
\pgfsetfillopacity{0.700000}%
\pgfsetlinewidth{0.000000pt}%
\definecolor{currentstroke}{rgb}{0.000000,0.000000,0.000000}%
\pgfsetstrokecolor{currentstroke}%
\pgfsetdash{}{0pt}%
\pgfpathmoveto{\pgfqpoint{2.354712in}{2.696196in}}%
\pgfpathlineto{\pgfqpoint{2.368245in}{2.674005in}}%
\pgfpathlineto{\pgfqpoint{2.381764in}{2.652125in}}%
\pgfpathlineto{\pgfqpoint{2.395270in}{2.630554in}}%
\pgfpathlineto{\pgfqpoint{2.408763in}{2.609288in}}%
\pgfpathlineto{\pgfqpoint{2.417259in}{2.610693in}}%
\pgfpathlineto{\pgfqpoint{2.425739in}{2.612313in}}%
\pgfpathlineto{\pgfqpoint{2.434204in}{2.614143in}}%
\pgfpathlineto{\pgfqpoint{2.442655in}{2.616181in}}%
\pgfpathlineto{\pgfqpoint{2.429203in}{2.637061in}}%
\pgfpathlineto{\pgfqpoint{2.415739in}{2.658245in}}%
\pgfpathlineto{\pgfqpoint{2.402262in}{2.679736in}}%
\pgfpathlineto{\pgfqpoint{2.388771in}{2.701538in}}%
\pgfpathlineto{\pgfqpoint{2.380280in}{2.699876in}}%
\pgfpathlineto{\pgfqpoint{2.371773in}{2.698430in}}%
\pgfpathlineto{\pgfqpoint{2.363250in}{2.697202in}}%
\pgfpathlineto{\pgfqpoint{2.354712in}{2.696196in}}%
\pgfpathclose%
\pgfusepath{fill}%
\end{pgfscope}%
\begin{pgfscope}%
\pgfpathrectangle{\pgfqpoint{1.254980in}{0.150000in}}{\pgfqpoint{5.490039in}{5.490039in}}%
\pgfusepath{clip}%
\pgfsetbuttcap%
\pgfsetroundjoin%
\definecolor{currentfill}{rgb}{0.281887,0.150881,0.465405}%
\pgfsetfillcolor{currentfill}%
\pgfsetfillopacity{0.700000}%
\pgfsetlinewidth{0.000000pt}%
\definecolor{currentstroke}{rgb}{0.000000,0.000000,0.000000}%
\pgfsetstrokecolor{currentstroke}%
\pgfsetdash{}{0pt}%
\pgfpathmoveto{\pgfqpoint{3.961916in}{2.054535in}}%
\pgfpathlineto{\pgfqpoint{3.975171in}{2.056414in}}%
\pgfpathlineto{\pgfqpoint{3.988434in}{2.058461in}}%
\pgfpathlineto{\pgfqpoint{4.001706in}{2.060676in}}%
\pgfpathlineto{\pgfqpoint{4.014988in}{2.063060in}}%
\pgfpathlineto{\pgfqpoint{4.022687in}{2.073456in}}%
\pgfpathlineto{\pgfqpoint{4.030382in}{2.083812in}}%
\pgfpathlineto{\pgfqpoint{4.038071in}{2.094128in}}%
\pgfpathlineto{\pgfqpoint{4.045756in}{2.104403in}}%
\pgfpathlineto{\pgfqpoint{4.032481in}{2.101915in}}%
\pgfpathlineto{\pgfqpoint{4.019215in}{2.099595in}}%
\pgfpathlineto{\pgfqpoint{4.005958in}{2.097443in}}%
\pgfpathlineto{\pgfqpoint{3.992710in}{2.095459in}}%
\pgfpathlineto{\pgfqpoint{3.985019in}{2.085279in}}%
\pgfpathlineto{\pgfqpoint{3.977323in}{2.075064in}}%
\pgfpathlineto{\pgfqpoint{3.969622in}{2.064816in}}%
\pgfpathlineto{\pgfqpoint{3.961916in}{2.054535in}}%
\pgfpathclose%
\pgfusepath{fill}%
\end{pgfscope}%
\begin{pgfscope}%
\pgfpathrectangle{\pgfqpoint{1.254980in}{0.150000in}}{\pgfqpoint{5.490039in}{5.490039in}}%
\pgfusepath{clip}%
\pgfsetbuttcap%
\pgfsetroundjoin%
\definecolor{currentfill}{rgb}{0.280894,0.078907,0.402329}%
\pgfsetfillcolor{currentfill}%
\pgfsetfillopacity{0.700000}%
\pgfsetlinewidth{0.000000pt}%
\definecolor{currentstroke}{rgb}{0.000000,0.000000,0.000000}%
\pgfsetstrokecolor{currentstroke}%
\pgfsetdash{}{0pt}%
\pgfpathmoveto{\pgfqpoint{3.710346in}{1.925506in}}%
\pgfpathlineto{\pgfqpoint{3.723534in}{1.924846in}}%
\pgfpathlineto{\pgfqpoint{3.736728in}{1.924361in}}%
\pgfpathlineto{\pgfqpoint{3.749929in}{1.924049in}}%
\pgfpathlineto{\pgfqpoint{3.763137in}{1.923910in}}%
\pgfpathlineto{\pgfqpoint{3.770918in}{1.934105in}}%
\pgfpathlineto{\pgfqpoint{3.778694in}{1.944294in}}%
\pgfpathlineto{\pgfqpoint{3.786465in}{1.954477in}}%
\pgfpathlineto{\pgfqpoint{3.794230in}{1.964653in}}%
\pgfpathlineto{\pgfqpoint{3.781031in}{1.964603in}}%
\pgfpathlineto{\pgfqpoint{3.767839in}{1.964727in}}%
\pgfpathlineto{\pgfqpoint{3.754654in}{1.965024in}}%
\pgfpathlineto{\pgfqpoint{3.741476in}{1.965495in}}%
\pgfpathlineto{\pgfqpoint{3.733701in}{1.955497in}}%
\pgfpathlineto{\pgfqpoint{3.725921in}{1.945499in}}%
\pgfpathlineto{\pgfqpoint{3.718136in}{1.935501in}}%
\pgfpathlineto{\pgfqpoint{3.710346in}{1.925506in}}%
\pgfpathclose%
\pgfusepath{fill}%
\end{pgfscope}%
\begin{pgfscope}%
\pgfpathrectangle{\pgfqpoint{1.254980in}{0.150000in}}{\pgfqpoint{5.490039in}{5.490039in}}%
\pgfusepath{clip}%
\pgfsetbuttcap%
\pgfsetroundjoin%
\definecolor{currentfill}{rgb}{0.278826,0.175490,0.483397}%
\pgfsetfillcolor{currentfill}%
\pgfsetfillopacity{0.700000}%
\pgfsetlinewidth{0.000000pt}%
\definecolor{currentstroke}{rgb}{0.000000,0.000000,0.000000}%
\pgfsetstrokecolor{currentstroke}%
\pgfsetdash{}{0pt}%
\pgfpathmoveto{\pgfqpoint{4.045756in}{2.104403in}}%
\pgfpathlineto{\pgfqpoint{4.059040in}{2.107059in}}%
\pgfpathlineto{\pgfqpoint{4.072334in}{2.109882in}}%
\pgfpathlineto{\pgfqpoint{4.085638in}{2.112871in}}%
\pgfpathlineto{\pgfqpoint{4.098951in}{2.116027in}}%
\pgfpathlineto{\pgfqpoint{4.106624in}{2.126348in}}%
\pgfpathlineto{\pgfqpoint{4.114293in}{2.136619in}}%
\pgfpathlineto{\pgfqpoint{4.121957in}{2.146841in}}%
\pgfpathlineto{\pgfqpoint{4.129615in}{2.157013in}}%
\pgfpathlineto{\pgfqpoint{4.116308in}{2.153781in}}%
\pgfpathlineto{\pgfqpoint{4.103010in}{2.150715in}}%
\pgfpathlineto{\pgfqpoint{4.089723in}{2.147815in}}%
\pgfpathlineto{\pgfqpoint{4.076444in}{2.145083in}}%
\pgfpathlineto{\pgfqpoint{4.068780in}{2.134977in}}%
\pgfpathlineto{\pgfqpoint{4.061110in}{2.124828in}}%
\pgfpathlineto{\pgfqpoint{4.053435in}{2.114637in}}%
\pgfpathlineto{\pgfqpoint{4.045756in}{2.104403in}}%
\pgfpathclose%
\pgfusepath{fill}%
\end{pgfscope}%
\begin{pgfscope}%
\pgfpathrectangle{\pgfqpoint{1.254980in}{0.150000in}}{\pgfqpoint{5.490039in}{5.490039in}}%
\pgfusepath{clip}%
\pgfsetbuttcap%
\pgfsetroundjoin%
\definecolor{currentfill}{rgb}{0.160665,0.478540,0.558115}%
\pgfsetfillcolor{currentfill}%
\pgfsetfillopacity{0.700000}%
\pgfsetlinewidth{0.000000pt}%
\definecolor{currentstroke}{rgb}{0.000000,0.000000,0.000000}%
\pgfsetstrokecolor{currentstroke}%
\pgfsetdash{}{0pt}%
\pgfpathmoveto{\pgfqpoint{5.029457in}{2.809059in}}%
\pgfpathlineto{\pgfqpoint{5.043198in}{2.817589in}}%
\pgfpathlineto{\pgfqpoint{5.056954in}{2.826275in}}%
\pgfpathlineto{\pgfqpoint{5.070726in}{2.835120in}}%
\pgfpathlineto{\pgfqpoint{5.084514in}{2.844123in}}%
\pgfpathlineto{\pgfqpoint{5.091786in}{2.849843in}}%
\pgfpathlineto{\pgfqpoint{5.099052in}{2.855514in}}%
\pgfpathlineto{\pgfqpoint{5.106310in}{2.861137in}}%
\pgfpathlineto{\pgfqpoint{5.113562in}{2.866718in}}%
\pgfpathlineto{\pgfqpoint{5.099788in}{2.858018in}}%
\pgfpathlineto{\pgfqpoint{5.086031in}{2.849475in}}%
\pgfpathlineto{\pgfqpoint{5.072289in}{2.841089in}}%
\pgfpathlineto{\pgfqpoint{5.058563in}{2.832861in}}%
\pgfpathlineto{\pgfqpoint{5.051296in}{2.826968in}}%
\pgfpathlineto{\pgfqpoint{5.044023in}{2.821039in}}%
\pgfpathlineto{\pgfqpoint{5.036743in}{2.815071in}}%
\pgfpathlineto{\pgfqpoint{5.029457in}{2.809059in}}%
\pgfpathclose%
\pgfusepath{fill}%
\end{pgfscope}%
\begin{pgfscope}%
\pgfpathrectangle{\pgfqpoint{1.254980in}{0.150000in}}{\pgfqpoint{5.490039in}{5.490039in}}%
\pgfusepath{clip}%
\pgfsetbuttcap%
\pgfsetroundjoin%
\definecolor{currentfill}{rgb}{0.278791,0.062145,0.386592}%
\pgfsetfillcolor{currentfill}%
\pgfsetfillopacity{0.700000}%
\pgfsetlinewidth{0.000000pt}%
\definecolor{currentstroke}{rgb}{0.000000,0.000000,0.000000}%
\pgfsetstrokecolor{currentstroke}%
\pgfsetdash{}{0pt}%
\pgfpathmoveto{\pgfqpoint{3.626402in}{1.890834in}}%
\pgfpathlineto{\pgfqpoint{3.639576in}{1.889257in}}%
\pgfpathlineto{\pgfqpoint{3.652756in}{1.887856in}}%
\pgfpathlineto{\pgfqpoint{3.665941in}{1.886631in}}%
\pgfpathlineto{\pgfqpoint{3.679132in}{1.885581in}}%
\pgfpathlineto{\pgfqpoint{3.686944in}{1.895550in}}%
\pgfpathlineto{\pgfqpoint{3.694750in}{1.905529in}}%
\pgfpathlineto{\pgfqpoint{3.702550in}{1.915514in}}%
\pgfpathlineto{\pgfqpoint{3.710346in}{1.925506in}}%
\pgfpathlineto{\pgfqpoint{3.697164in}{1.926340in}}%
\pgfpathlineto{\pgfqpoint{3.683989in}{1.927349in}}%
\pgfpathlineto{\pgfqpoint{3.670820in}{1.928534in}}%
\pgfpathlineto{\pgfqpoint{3.657657in}{1.929895in}}%
\pgfpathlineto{\pgfqpoint{3.649851in}{1.920109in}}%
\pgfpathlineto{\pgfqpoint{3.642040in}{1.910336in}}%
\pgfpathlineto{\pgfqpoint{3.634224in}{1.900577in}}%
\pgfpathlineto{\pgfqpoint{3.626402in}{1.890834in}}%
\pgfpathclose%
\pgfusepath{fill}%
\end{pgfscope}%
\begin{pgfscope}%
\pgfpathrectangle{\pgfqpoint{1.254980in}{0.150000in}}{\pgfqpoint{5.490039in}{5.490039in}}%
\pgfusepath{clip}%
\pgfsetbuttcap%
\pgfsetroundjoin%
\definecolor{currentfill}{rgb}{0.280267,0.073417,0.397163}%
\pgfsetfillcolor{currentfill}%
\pgfsetfillopacity{0.700000}%
\pgfsetlinewidth{0.000000pt}%
\definecolor{currentstroke}{rgb}{0.000000,0.000000,0.000000}%
\pgfsetstrokecolor{currentstroke}%
\pgfsetdash{}{0pt}%
\pgfpathmoveto{\pgfqpoint{2.993816in}{1.937531in}}%
\pgfpathlineto{\pgfqpoint{3.007002in}{1.927558in}}%
\pgfpathlineto{\pgfqpoint{3.020186in}{1.917796in}}%
\pgfpathlineto{\pgfqpoint{3.033368in}{1.908243in}}%
\pgfpathlineto{\pgfqpoint{3.046549in}{1.898898in}}%
\pgfpathlineto{\pgfqpoint{3.054649in}{1.904874in}}%
\pgfpathlineto{\pgfqpoint{3.062740in}{1.910976in}}%
\pgfpathlineto{\pgfqpoint{3.070821in}{1.917201in}}%
\pgfpathlineto{\pgfqpoint{3.078894in}{1.923543in}}%
\pgfpathlineto{\pgfqpoint{3.065737in}{1.932530in}}%
\pgfpathlineto{\pgfqpoint{3.052579in}{1.941724in}}%
\pgfpathlineto{\pgfqpoint{3.039421in}{1.951126in}}%
\pgfpathlineto{\pgfqpoint{3.026260in}{1.960738in}}%
\pgfpathlineto{\pgfqpoint{3.018164in}{1.954744in}}%
\pgfpathlineto{\pgfqpoint{3.010057in}{1.948876in}}%
\pgfpathlineto{\pgfqpoint{3.001942in}{1.943137in}}%
\pgfpathlineto{\pgfqpoint{2.993816in}{1.937531in}}%
\pgfpathclose%
\pgfusepath{fill}%
\end{pgfscope}%
\begin{pgfscope}%
\pgfpathrectangle{\pgfqpoint{1.254980in}{0.150000in}}{\pgfqpoint{5.490039in}{5.490039in}}%
\pgfusepath{clip}%
\pgfsetbuttcap%
\pgfsetroundjoin%
\definecolor{currentfill}{rgb}{0.216210,0.351535,0.550627}%
\pgfsetfillcolor{currentfill}%
\pgfsetfillopacity{0.700000}%
\pgfsetlinewidth{0.000000pt}%
\definecolor{currentstroke}{rgb}{0.000000,0.000000,0.000000}%
\pgfsetstrokecolor{currentstroke}%
\pgfsetdash{}{0pt}%
\pgfpathmoveto{\pgfqpoint{4.579580in}{2.483052in}}%
\pgfpathlineto{\pgfqpoint{4.593093in}{2.489580in}}%
\pgfpathlineto{\pgfqpoint{4.606620in}{2.496269in}}%
\pgfpathlineto{\pgfqpoint{4.620160in}{2.503120in}}%
\pgfpathlineto{\pgfqpoint{4.633714in}{2.510131in}}%
\pgfpathlineto{\pgfqpoint{4.641197in}{2.518490in}}%
\pgfpathlineto{\pgfqpoint{4.648674in}{2.526773in}}%
\pgfpathlineto{\pgfqpoint{4.656145in}{2.534983in}}%
\pgfpathlineto{\pgfqpoint{4.663610in}{2.543121in}}%
\pgfpathlineto{\pgfqpoint{4.650065in}{2.536234in}}%
\pgfpathlineto{\pgfqpoint{4.636532in}{2.529509in}}%
\pgfpathlineto{\pgfqpoint{4.623013in}{2.522944in}}%
\pgfpathlineto{\pgfqpoint{4.609508in}{2.516539in}}%
\pgfpathlineto{\pgfqpoint{4.602035in}{2.508267in}}%
\pgfpathlineto{\pgfqpoint{4.594556in}{2.499929in}}%
\pgfpathlineto{\pgfqpoint{4.587071in}{2.491525in}}%
\pgfpathlineto{\pgfqpoint{4.579580in}{2.483052in}}%
\pgfpathclose%
\pgfusepath{fill}%
\end{pgfscope}%
\begin{pgfscope}%
\pgfpathrectangle{\pgfqpoint{1.254980in}{0.150000in}}{\pgfqpoint{5.490039in}{5.490039in}}%
\pgfusepath{clip}%
\pgfsetbuttcap%
\pgfsetroundjoin%
\definecolor{currentfill}{rgb}{0.273006,0.204520,0.501721}%
\pgfsetfillcolor{currentfill}%
\pgfsetfillopacity{0.700000}%
\pgfsetlinewidth{0.000000pt}%
\definecolor{currentstroke}{rgb}{0.000000,0.000000,0.000000}%
\pgfsetstrokecolor{currentstroke}%
\pgfsetdash{}{0pt}%
\pgfpathmoveto{\pgfqpoint{4.129615in}{2.157013in}}%
\pgfpathlineto{\pgfqpoint{4.142932in}{2.160412in}}%
\pgfpathlineto{\pgfqpoint{4.156260in}{2.163977in}}%
\pgfpathlineto{\pgfqpoint{4.169598in}{2.167707in}}%
\pgfpathlineto{\pgfqpoint{4.182946in}{2.171603in}}%
\pgfpathlineto{\pgfqpoint{4.190594in}{2.181784in}}%
\pgfpathlineto{\pgfqpoint{4.198236in}{2.191907in}}%
\pgfpathlineto{\pgfqpoint{4.205874in}{2.201973in}}%
\pgfpathlineto{\pgfqpoint{4.213506in}{2.211982in}}%
\pgfpathlineto{\pgfqpoint{4.200164in}{2.208038in}}%
\pgfpathlineto{\pgfqpoint{4.186832in}{2.204259in}}%
\pgfpathlineto{\pgfqpoint{4.173510in}{2.200646in}}%
\pgfpathlineto{\pgfqpoint{4.160199in}{2.197199in}}%
\pgfpathlineto{\pgfqpoint{4.152561in}{2.187228in}}%
\pgfpathlineto{\pgfqpoint{4.144917in}{2.177207in}}%
\pgfpathlineto{\pgfqpoint{4.137269in}{2.167135in}}%
\pgfpathlineto{\pgfqpoint{4.129615in}{2.157013in}}%
\pgfpathclose%
\pgfusepath{fill}%
\end{pgfscope}%
\begin{pgfscope}%
\pgfpathrectangle{\pgfqpoint{1.254980in}{0.150000in}}{\pgfqpoint{5.490039in}{5.490039in}}%
\pgfusepath{clip}%
\pgfsetbuttcap%
\pgfsetroundjoin%
\definecolor{currentfill}{rgb}{0.273809,0.031497,0.358853}%
\pgfsetfillcolor{currentfill}%
\pgfsetfillopacity{0.700000}%
\pgfsetlinewidth{0.000000pt}%
\definecolor{currentstroke}{rgb}{0.000000,0.000000,0.000000}%
\pgfsetstrokecolor{currentstroke}%
\pgfsetdash{}{0pt}%
\pgfpathmoveto{\pgfqpoint{3.184137in}{1.858970in}}%
\pgfpathlineto{\pgfqpoint{3.197294in}{1.851796in}}%
\pgfpathlineto{\pgfqpoint{3.210452in}{1.844818in}}%
\pgfpathlineto{\pgfqpoint{3.223610in}{1.838035in}}%
\pgfpathlineto{\pgfqpoint{3.236770in}{1.831445in}}%
\pgfpathlineto{\pgfqpoint{3.244770in}{1.838924in}}%
\pgfpathlineto{\pgfqpoint{3.252761in}{1.846493in}}%
\pgfpathlineto{\pgfqpoint{3.260746in}{1.854150in}}%
\pgfpathlineto{\pgfqpoint{3.268722in}{1.861892in}}%
\pgfpathlineto{\pgfqpoint{3.255582in}{1.868153in}}%
\pgfpathlineto{\pgfqpoint{3.242443in}{1.874608in}}%
\pgfpathlineto{\pgfqpoint{3.229305in}{1.881258in}}%
\pgfpathlineto{\pgfqpoint{3.216169in}{1.888103in}}%
\pgfpathlineto{\pgfqpoint{3.208173in}{1.880678in}}%
\pgfpathlineto{\pgfqpoint{3.200169in}{1.873346in}}%
\pgfpathlineto{\pgfqpoint{3.192157in}{1.866109in}}%
\pgfpathlineto{\pgfqpoint{3.184137in}{1.858970in}}%
\pgfpathclose%
\pgfusepath{fill}%
\end{pgfscope}%
\begin{pgfscope}%
\pgfpathrectangle{\pgfqpoint{1.254980in}{0.150000in}}{\pgfqpoint{5.490039in}{5.490039in}}%
\pgfusepath{clip}%
\pgfsetbuttcap%
\pgfsetroundjoin%
\definecolor{currentfill}{rgb}{0.151918,0.500685,0.557587}%
\pgfsetfillcolor{currentfill}%
\pgfsetfillopacity{0.700000}%
\pgfsetlinewidth{0.000000pt}%
\definecolor{currentstroke}{rgb}{0.000000,0.000000,0.000000}%
\pgfsetstrokecolor{currentstroke}%
\pgfsetdash{}{0pt}%
\pgfpathmoveto{\pgfqpoint{5.113562in}{2.866718in}}%
\pgfpathlineto{\pgfqpoint{5.127351in}{2.875576in}}%
\pgfpathlineto{\pgfqpoint{5.141157in}{2.884590in}}%
\pgfpathlineto{\pgfqpoint{5.154979in}{2.893763in}}%
\pgfpathlineto{\pgfqpoint{5.168817in}{2.903093in}}%
\pgfpathlineto{\pgfqpoint{5.176046in}{2.908312in}}%
\pgfpathlineto{\pgfqpoint{5.183269in}{2.913489in}}%
\pgfpathlineto{\pgfqpoint{5.190485in}{2.918627in}}%
\pgfpathlineto{\pgfqpoint{5.197694in}{2.923730in}}%
\pgfpathlineto{\pgfqpoint{5.183872in}{2.914733in}}%
\pgfpathlineto{\pgfqpoint{5.170066in}{2.905893in}}%
\pgfpathlineto{\pgfqpoint{5.156277in}{2.897209in}}%
\pgfpathlineto{\pgfqpoint{5.142503in}{2.888682in}}%
\pgfpathlineto{\pgfqpoint{5.135278in}{2.883237in}}%
\pgfpathlineto{\pgfqpoint{5.128046in}{2.877764in}}%
\pgfpathlineto{\pgfqpoint{5.120807in}{2.872259in}}%
\pgfpathlineto{\pgfqpoint{5.113562in}{2.866718in}}%
\pgfpathclose%
\pgfusepath{fill}%
\end{pgfscope}%
\begin{pgfscope}%
\pgfpathrectangle{\pgfqpoint{1.254980in}{0.150000in}}{\pgfqpoint{5.490039in}{5.490039in}}%
\pgfusepath{clip}%
\pgfsetbuttcap%
\pgfsetroundjoin%
\definecolor{currentfill}{rgb}{0.272594,0.025563,0.353093}%
\pgfsetfillcolor{currentfill}%
\pgfsetfillopacity{0.700000}%
\pgfsetlinewidth{0.000000pt}%
\definecolor{currentstroke}{rgb}{0.000000,0.000000,0.000000}%
\pgfsetstrokecolor{currentstroke}%
\pgfsetdash{}{0pt}%
\pgfpathmoveto{\pgfqpoint{3.321303in}{1.838766in}}%
\pgfpathlineto{\pgfqpoint{3.334453in}{1.833459in}}%
\pgfpathlineto{\pgfqpoint{3.347606in}{1.828340in}}%
\pgfpathlineto{\pgfqpoint{3.360762in}{1.823409in}}%
\pgfpathlineto{\pgfqpoint{3.373921in}{1.818664in}}%
\pgfpathlineto{\pgfqpoint{3.381855in}{1.827109in}}%
\pgfpathlineto{\pgfqpoint{3.389783in}{1.835619in}}%
\pgfpathlineto{\pgfqpoint{3.397704in}{1.844193in}}%
\pgfpathlineto{\pgfqpoint{3.405619in}{1.852825in}}%
\pgfpathlineto{\pgfqpoint{3.392476in}{1.857271in}}%
\pgfpathlineto{\pgfqpoint{3.379337in}{1.861903in}}%
\pgfpathlineto{\pgfqpoint{3.366200in}{1.866722in}}%
\pgfpathlineto{\pgfqpoint{3.353067in}{1.871729in}}%
\pgfpathlineto{\pgfqpoint{3.345136in}{1.863385in}}%
\pgfpathlineto{\pgfqpoint{3.337198in}{1.855108in}}%
\pgfpathlineto{\pgfqpoint{3.329254in}{1.846901in}}%
\pgfpathlineto{\pgfqpoint{3.321303in}{1.838766in}}%
\pgfpathclose%
\pgfusepath{fill}%
\end{pgfscope}%
\begin{pgfscope}%
\pgfpathrectangle{\pgfqpoint{1.254980in}{0.150000in}}{\pgfqpoint{5.490039in}{5.490039in}}%
\pgfusepath{clip}%
\pgfsetbuttcap%
\pgfsetroundjoin%
\definecolor{currentfill}{rgb}{0.276022,0.044167,0.370164}%
\pgfsetfillcolor{currentfill}%
\pgfsetfillopacity{0.700000}%
\pgfsetlinewidth{0.000000pt}%
\definecolor{currentstroke}{rgb}{0.000000,0.000000,0.000000}%
\pgfsetstrokecolor{currentstroke}%
\pgfsetdash{}{0pt}%
\pgfpathmoveto{\pgfqpoint{3.542372in}{1.861126in}}%
\pgfpathlineto{\pgfqpoint{3.555536in}{1.858593in}}%
\pgfpathlineto{\pgfqpoint{3.568706in}{1.856240in}}%
\pgfpathlineto{\pgfqpoint{3.581880in}{1.854065in}}%
\pgfpathlineto{\pgfqpoint{3.595060in}{1.852067in}}%
\pgfpathlineto{\pgfqpoint{3.602904in}{1.861724in}}%
\pgfpathlineto{\pgfqpoint{3.610742in}{1.871406in}}%
\pgfpathlineto{\pgfqpoint{3.618575in}{1.881110in}}%
\pgfpathlineto{\pgfqpoint{3.626402in}{1.890834in}}%
\pgfpathlineto{\pgfqpoint{3.613234in}{1.892588in}}%
\pgfpathlineto{\pgfqpoint{3.600071in}{1.894520in}}%
\pgfpathlineto{\pgfqpoint{3.586914in}{1.896630in}}%
\pgfpathlineto{\pgfqpoint{3.573761in}{1.898918in}}%
\pgfpathlineto{\pgfqpoint{3.565922in}{1.889427in}}%
\pgfpathlineto{\pgfqpoint{3.558078in}{1.879963in}}%
\pgfpathlineto{\pgfqpoint{3.550228in}{1.870529in}}%
\pgfpathlineto{\pgfqpoint{3.542372in}{1.861126in}}%
\pgfpathclose%
\pgfusepath{fill}%
\end{pgfscope}%
\begin{pgfscope}%
\pgfpathrectangle{\pgfqpoint{1.254980in}{0.150000in}}{\pgfqpoint{5.490039in}{5.490039in}}%
\pgfusepath{clip}%
\pgfsetbuttcap%
\pgfsetroundjoin%
\definecolor{currentfill}{rgb}{0.269308,0.218818,0.509577}%
\pgfsetfillcolor{currentfill}%
\pgfsetfillopacity{0.700000}%
\pgfsetlinewidth{0.000000pt}%
\definecolor{currentstroke}{rgb}{0.000000,0.000000,0.000000}%
\pgfsetstrokecolor{currentstroke}%
\pgfsetdash{}{0pt}%
\pgfpathmoveto{\pgfqpoint{2.642994in}{2.232204in}}%
\pgfpathlineto{\pgfqpoint{2.656316in}{2.216325in}}%
\pgfpathlineto{\pgfqpoint{2.669631in}{2.200697in}}%
\pgfpathlineto{\pgfqpoint{2.682940in}{2.185318in}}%
\pgfpathlineto{\pgfqpoint{2.696241in}{2.170185in}}%
\pgfpathlineto{\pgfqpoint{2.704563in}{2.173204in}}%
\pgfpathlineto{\pgfqpoint{2.712873in}{2.176408in}}%
\pgfpathlineto{\pgfqpoint{2.721169in}{2.179791in}}%
\pgfpathlineto{\pgfqpoint{2.729454in}{2.183350in}}%
\pgfpathlineto{\pgfqpoint{2.716186in}{2.198085in}}%
\pgfpathlineto{\pgfqpoint{2.702913in}{2.213065in}}%
\pgfpathlineto{\pgfqpoint{2.689632in}{2.228293in}}%
\pgfpathlineto{\pgfqpoint{2.676345in}{2.243771in}}%
\pgfpathlineto{\pgfqpoint{2.668027in}{2.240600in}}%
\pgfpathlineto{\pgfqpoint{2.659696in}{2.237613in}}%
\pgfpathlineto{\pgfqpoint{2.651351in}{2.234813in}}%
\pgfpathlineto{\pgfqpoint{2.642994in}{2.232204in}}%
\pgfpathclose%
\pgfusepath{fill}%
\end{pgfscope}%
\begin{pgfscope}%
\pgfpathrectangle{\pgfqpoint{1.254980in}{0.150000in}}{\pgfqpoint{5.490039in}{5.490039in}}%
\pgfusepath{clip}%
\pgfsetbuttcap%
\pgfsetroundjoin%
\definecolor{currentfill}{rgb}{0.265145,0.232956,0.516599}%
\pgfsetfillcolor{currentfill}%
\pgfsetfillopacity{0.700000}%
\pgfsetlinewidth{0.000000pt}%
\definecolor{currentstroke}{rgb}{0.000000,0.000000,0.000000}%
\pgfsetstrokecolor{currentstroke}%
\pgfsetdash{}{0pt}%
\pgfpathmoveto{\pgfqpoint{4.213506in}{2.211982in}}%
\pgfpathlineto{\pgfqpoint{4.226859in}{2.216091in}}%
\pgfpathlineto{\pgfqpoint{4.240223in}{2.220364in}}%
\pgfpathlineto{\pgfqpoint{4.253598in}{2.224802in}}%
\pgfpathlineto{\pgfqpoint{4.266984in}{2.229405in}}%
\pgfpathlineto{\pgfqpoint{4.274606in}{2.239386in}}%
\pgfpathlineto{\pgfqpoint{4.282222in}{2.249303in}}%
\pgfpathlineto{\pgfqpoint{4.289834in}{2.259156in}}%
\pgfpathlineto{\pgfqpoint{4.297439in}{2.268945in}}%
\pgfpathlineto{\pgfqpoint{4.284059in}{2.264323in}}%
\pgfpathlineto{\pgfqpoint{4.270690in}{2.259865in}}%
\pgfpathlineto{\pgfqpoint{4.257332in}{2.255571in}}%
\pgfpathlineto{\pgfqpoint{4.243985in}{2.251442in}}%
\pgfpathlineto{\pgfqpoint{4.236373in}{2.241663in}}%
\pgfpathlineto{\pgfqpoint{4.228756in}{2.231826in}}%
\pgfpathlineto{\pgfqpoint{4.221134in}{2.221933in}}%
\pgfpathlineto{\pgfqpoint{4.213506in}{2.211982in}}%
\pgfpathclose%
\pgfusepath{fill}%
\end{pgfscope}%
\begin{pgfscope}%
\pgfpathrectangle{\pgfqpoint{1.254980in}{0.150000in}}{\pgfqpoint{5.490039in}{5.490039in}}%
\pgfusepath{clip}%
\pgfsetbuttcap%
\pgfsetroundjoin%
\definecolor{currentfill}{rgb}{0.260571,0.246922,0.522828}%
\pgfsetfillcolor{currentfill}%
\pgfsetfillopacity{0.700000}%
\pgfsetlinewidth{0.000000pt}%
\definecolor{currentstroke}{rgb}{0.000000,0.000000,0.000000}%
\pgfsetstrokecolor{currentstroke}%
\pgfsetdash{}{0pt}%
\pgfpathmoveto{\pgfqpoint{2.589627in}{2.298266in}}%
\pgfpathlineto{\pgfqpoint{2.602981in}{2.281364in}}%
\pgfpathlineto{\pgfqpoint{2.616326in}{2.264721in}}%
\pgfpathlineto{\pgfqpoint{2.629664in}{2.248335in}}%
\pgfpathlineto{\pgfqpoint{2.642994in}{2.232204in}}%
\pgfpathlineto{\pgfqpoint{2.651351in}{2.234813in}}%
\pgfpathlineto{\pgfqpoint{2.659696in}{2.237613in}}%
\pgfpathlineto{\pgfqpoint{2.668027in}{2.240600in}}%
\pgfpathlineto{\pgfqpoint{2.676345in}{2.243771in}}%
\pgfpathlineto{\pgfqpoint{2.663051in}{2.259501in}}%
\pgfpathlineto{\pgfqpoint{2.649750in}{2.275486in}}%
\pgfpathlineto{\pgfqpoint{2.636440in}{2.291726in}}%
\pgfpathlineto{\pgfqpoint{2.623124in}{2.308225in}}%
\pgfpathlineto{\pgfqpoint{2.614770in}{2.305445in}}%
\pgfpathlineto{\pgfqpoint{2.606402in}{2.302855in}}%
\pgfpathlineto{\pgfqpoint{2.598022in}{2.300461in}}%
\pgfpathlineto{\pgfqpoint{2.589627in}{2.298266in}}%
\pgfpathclose%
\pgfusepath{fill}%
\end{pgfscope}%
\begin{pgfscope}%
\pgfpathrectangle{\pgfqpoint{1.254980in}{0.150000in}}{\pgfqpoint{5.490039in}{5.490039in}}%
\pgfusepath{clip}%
\pgfsetbuttcap%
\pgfsetroundjoin%
\definecolor{currentfill}{rgb}{0.276194,0.190074,0.493001}%
\pgfsetfillcolor{currentfill}%
\pgfsetfillopacity{0.700000}%
\pgfsetlinewidth{0.000000pt}%
\definecolor{currentstroke}{rgb}{0.000000,0.000000,0.000000}%
\pgfsetstrokecolor{currentstroke}%
\pgfsetdash{}{0pt}%
\pgfpathmoveto{\pgfqpoint{2.696241in}{2.170185in}}%
\pgfpathlineto{\pgfqpoint{2.709536in}{2.155296in}}%
\pgfpathlineto{\pgfqpoint{2.722825in}{2.140650in}}%
\pgfpathlineto{\pgfqpoint{2.736108in}{2.126245in}}%
\pgfpathlineto{\pgfqpoint{2.749385in}{2.112079in}}%
\pgfpathlineto{\pgfqpoint{2.757673in}{2.115507in}}%
\pgfpathlineto{\pgfqpoint{2.765949in}{2.119111in}}%
\pgfpathlineto{\pgfqpoint{2.774212in}{2.122887in}}%
\pgfpathlineto{\pgfqpoint{2.782464in}{2.126833in}}%
\pgfpathlineto{\pgfqpoint{2.769220in}{2.140603in}}%
\pgfpathlineto{\pgfqpoint{2.755970in}{2.154611in}}%
\pgfpathlineto{\pgfqpoint{2.742715in}{2.168860in}}%
\pgfpathlineto{\pgfqpoint{2.729454in}{2.183350in}}%
\pgfpathlineto{\pgfqpoint{2.721169in}{2.179791in}}%
\pgfpathlineto{\pgfqpoint{2.712873in}{2.176408in}}%
\pgfpathlineto{\pgfqpoint{2.704563in}{2.173204in}}%
\pgfpathlineto{\pgfqpoint{2.696241in}{2.170185in}}%
\pgfpathclose%
\pgfusepath{fill}%
\end{pgfscope}%
\begin{pgfscope}%
\pgfpathrectangle{\pgfqpoint{1.254980in}{0.150000in}}{\pgfqpoint{5.490039in}{5.490039in}}%
\pgfusepath{clip}%
\pgfsetbuttcap%
\pgfsetroundjoin%
\definecolor{currentfill}{rgb}{0.203063,0.379716,0.553925}%
\pgfsetfillcolor{currentfill}%
\pgfsetfillopacity{0.700000}%
\pgfsetlinewidth{0.000000pt}%
\definecolor{currentstroke}{rgb}{0.000000,0.000000,0.000000}%
\pgfsetstrokecolor{currentstroke}%
\pgfsetdash{}{0pt}%
\pgfpathmoveto{\pgfqpoint{4.663610in}{2.543121in}}%
\pgfpathlineto{\pgfqpoint{4.677170in}{2.550168in}}%
\pgfpathlineto{\pgfqpoint{4.690743in}{2.557375in}}%
\pgfpathlineto{\pgfqpoint{4.704330in}{2.564743in}}%
\pgfpathlineto{\pgfqpoint{4.717931in}{2.572271in}}%
\pgfpathlineto{\pgfqpoint{4.725382in}{2.580196in}}%
\pgfpathlineto{\pgfqpoint{4.732826in}{2.588045in}}%
\pgfpathlineto{\pgfqpoint{4.740263in}{2.595820in}}%
\pgfpathlineto{\pgfqpoint{4.747695in}{2.603525in}}%
\pgfpathlineto{\pgfqpoint{4.734102in}{2.596151in}}%
\pgfpathlineto{\pgfqpoint{4.720524in}{2.588937in}}%
\pgfpathlineto{\pgfqpoint{4.706959in}{2.581883in}}%
\pgfpathlineto{\pgfqpoint{4.693408in}{2.574990in}}%
\pgfpathlineto{\pgfqpoint{4.685968in}{2.567121in}}%
\pgfpathlineto{\pgfqpoint{4.678522in}{2.559188in}}%
\pgfpathlineto{\pgfqpoint{4.671069in}{2.551189in}}%
\pgfpathlineto{\pgfqpoint{4.663610in}{2.543121in}}%
\pgfpathclose%
\pgfusepath{fill}%
\end{pgfscope}%
\begin{pgfscope}%
\pgfpathrectangle{\pgfqpoint{1.254980in}{0.150000in}}{\pgfqpoint{5.490039in}{5.490039in}}%
\pgfusepath{clip}%
\pgfsetbuttcap%
\pgfsetroundjoin%
\definecolor{currentfill}{rgb}{0.143343,0.522773,0.556295}%
\pgfsetfillcolor{currentfill}%
\pgfsetfillopacity{0.700000}%
\pgfsetlinewidth{0.000000pt}%
\definecolor{currentstroke}{rgb}{0.000000,0.000000,0.000000}%
\pgfsetstrokecolor{currentstroke}%
\pgfsetdash{}{0pt}%
\pgfpathmoveto{\pgfqpoint{5.197694in}{2.923730in}}%
\pgfpathlineto{\pgfqpoint{5.211532in}{2.932884in}}%
\pgfpathlineto{\pgfqpoint{5.225387in}{2.942195in}}%
\pgfpathlineto{\pgfqpoint{5.239259in}{2.951663in}}%
\pgfpathlineto{\pgfqpoint{5.253147in}{2.961288in}}%
\pgfpathlineto{\pgfqpoint{5.260332in}{2.966008in}}%
\pgfpathlineto{\pgfqpoint{5.267510in}{2.970694in}}%
\pgfpathlineto{\pgfqpoint{5.274682in}{2.975350in}}%
\pgfpathlineto{\pgfqpoint{5.281847in}{2.979979in}}%
\pgfpathlineto{\pgfqpoint{5.267977in}{2.970717in}}%
\pgfpathlineto{\pgfqpoint{5.254123in}{2.961611in}}%
\pgfpathlineto{\pgfqpoint{5.240286in}{2.952662in}}%
\pgfpathlineto{\pgfqpoint{5.226465in}{2.943869in}}%
\pgfpathlineto{\pgfqpoint{5.219282in}{2.938867in}}%
\pgfpathlineto{\pgfqpoint{5.212092in}{2.933846in}}%
\pgfpathlineto{\pgfqpoint{5.204896in}{2.928802in}}%
\pgfpathlineto{\pgfqpoint{5.197694in}{2.923730in}}%
\pgfpathclose%
\pgfusepath{fill}%
\end{pgfscope}%
\begin{pgfscope}%
\pgfpathrectangle{\pgfqpoint{1.254980in}{0.150000in}}{\pgfqpoint{5.490039in}{5.490039in}}%
\pgfusepath{clip}%
\pgfsetbuttcap%
\pgfsetroundjoin%
\definecolor{currentfill}{rgb}{0.248629,0.278775,0.534556}%
\pgfsetfillcolor{currentfill}%
\pgfsetfillopacity{0.700000}%
\pgfsetlinewidth{0.000000pt}%
\definecolor{currentstroke}{rgb}{0.000000,0.000000,0.000000}%
\pgfsetstrokecolor{currentstroke}%
\pgfsetdash{}{0pt}%
\pgfpathmoveto{\pgfqpoint{2.536124in}{2.368509in}}%
\pgfpathlineto{\pgfqpoint{2.549513in}{2.350548in}}%
\pgfpathlineto{\pgfqpoint{2.562893in}{2.332856in}}%
\pgfpathlineto{\pgfqpoint{2.576264in}{2.315429in}}%
\pgfpathlineto{\pgfqpoint{2.589627in}{2.298266in}}%
\pgfpathlineto{\pgfqpoint{2.598022in}{2.300461in}}%
\pgfpathlineto{\pgfqpoint{2.606402in}{2.302855in}}%
\pgfpathlineto{\pgfqpoint{2.614770in}{2.305445in}}%
\pgfpathlineto{\pgfqpoint{2.623124in}{2.308225in}}%
\pgfpathlineto{\pgfqpoint{2.609798in}{2.324984in}}%
\pgfpathlineto{\pgfqpoint{2.596465in}{2.342007in}}%
\pgfpathlineto{\pgfqpoint{2.583123in}{2.359294in}}%
\pgfpathlineto{\pgfqpoint{2.569772in}{2.376849in}}%
\pgfpathlineto{\pgfqpoint{2.561381in}{2.374462in}}%
\pgfpathlineto{\pgfqpoint{2.552976in}{2.372274in}}%
\pgfpathlineto{\pgfqpoint{2.544557in}{2.370289in}}%
\pgfpathlineto{\pgfqpoint{2.536124in}{2.368509in}}%
\pgfpathclose%
\pgfusepath{fill}%
\end{pgfscope}%
\begin{pgfscope}%
\pgfpathrectangle{\pgfqpoint{1.254980in}{0.150000in}}{\pgfqpoint{5.490039in}{5.490039in}}%
\pgfusepath{clip}%
\pgfsetbuttcap%
\pgfsetroundjoin%
\definecolor{currentfill}{rgb}{0.280868,0.160771,0.472899}%
\pgfsetfillcolor{currentfill}%
\pgfsetfillopacity{0.700000}%
\pgfsetlinewidth{0.000000pt}%
\definecolor{currentstroke}{rgb}{0.000000,0.000000,0.000000}%
\pgfsetstrokecolor{currentstroke}%
\pgfsetdash{}{0pt}%
\pgfpathmoveto{\pgfqpoint{2.749385in}{2.112079in}}%
\pgfpathlineto{\pgfqpoint{2.762657in}{2.098149in}}%
\pgfpathlineto{\pgfqpoint{2.775923in}{2.084455in}}%
\pgfpathlineto{\pgfqpoint{2.789184in}{2.070995in}}%
\pgfpathlineto{\pgfqpoint{2.802440in}{2.057766in}}%
\pgfpathlineto{\pgfqpoint{2.810696in}{2.061600in}}%
\pgfpathlineto{\pgfqpoint{2.818939in}{2.065602in}}%
\pgfpathlineto{\pgfqpoint{2.827171in}{2.069770in}}%
\pgfpathlineto{\pgfqpoint{2.835392in}{2.074099in}}%
\pgfpathlineto{\pgfqpoint{2.822167in}{2.086934in}}%
\pgfpathlineto{\pgfqpoint{2.808937in}{2.100000in}}%
\pgfpathlineto{\pgfqpoint{2.795703in}{2.113299in}}%
\pgfpathlineto{\pgfqpoint{2.782464in}{2.126833in}}%
\pgfpathlineto{\pgfqpoint{2.774212in}{2.122887in}}%
\pgfpathlineto{\pgfqpoint{2.765949in}{2.119111in}}%
\pgfpathlineto{\pgfqpoint{2.757673in}{2.115507in}}%
\pgfpathlineto{\pgfqpoint{2.749385in}{2.112079in}}%
\pgfpathclose%
\pgfusepath{fill}%
\end{pgfscope}%
\begin{pgfscope}%
\pgfpathrectangle{\pgfqpoint{1.254980in}{0.150000in}}{\pgfqpoint{5.490039in}{5.490039in}}%
\pgfusepath{clip}%
\pgfsetbuttcap%
\pgfsetroundjoin%
\definecolor{currentfill}{rgb}{0.277941,0.056324,0.381191}%
\pgfsetfillcolor{currentfill}%
\pgfsetfillopacity{0.700000}%
\pgfsetlinewidth{0.000000pt}%
\definecolor{currentstroke}{rgb}{0.000000,0.000000,0.000000}%
\pgfsetstrokecolor{currentstroke}%
\pgfsetdash{}{0pt}%
\pgfpathmoveto{\pgfqpoint{3.046549in}{1.898898in}}%
\pgfpathlineto{\pgfqpoint{3.059729in}{1.889758in}}%
\pgfpathlineto{\pgfqpoint{3.072909in}{1.880824in}}%
\pgfpathlineto{\pgfqpoint{3.086087in}{1.872094in}}%
\pgfpathlineto{\pgfqpoint{3.099265in}{1.863566in}}%
\pgfpathlineto{\pgfqpoint{3.107341in}{1.869912in}}%
\pgfpathlineto{\pgfqpoint{3.115408in}{1.876376in}}%
\pgfpathlineto{\pgfqpoint{3.123466in}{1.882956in}}%
\pgfpathlineto{\pgfqpoint{3.131516in}{1.889646in}}%
\pgfpathlineto{\pgfqpoint{3.118361in}{1.897816in}}%
\pgfpathlineto{\pgfqpoint{3.105206in}{1.906188in}}%
\pgfpathlineto{\pgfqpoint{3.092050in}{1.914763in}}%
\pgfpathlineto{\pgfqpoint{3.078894in}{1.923543in}}%
\pgfpathlineto{\pgfqpoint{3.070821in}{1.917201in}}%
\pgfpathlineto{\pgfqpoint{3.062740in}{1.910976in}}%
\pgfpathlineto{\pgfqpoint{3.054649in}{1.904874in}}%
\pgfpathlineto{\pgfqpoint{3.046549in}{1.898898in}}%
\pgfpathclose%
\pgfusepath{fill}%
\end{pgfscope}%
\begin{pgfscope}%
\pgfpathrectangle{\pgfqpoint{1.254980in}{0.150000in}}{\pgfqpoint{5.490039in}{5.490039in}}%
\pgfusepath{clip}%
\pgfsetbuttcap%
\pgfsetroundjoin%
\definecolor{currentfill}{rgb}{0.255645,0.260703,0.528312}%
\pgfsetfillcolor{currentfill}%
\pgfsetfillopacity{0.700000}%
\pgfsetlinewidth{0.000000pt}%
\definecolor{currentstroke}{rgb}{0.000000,0.000000,0.000000}%
\pgfsetstrokecolor{currentstroke}%
\pgfsetdash{}{0pt}%
\pgfpathmoveto{\pgfqpoint{4.297439in}{2.268945in}}%
\pgfpathlineto{\pgfqpoint{4.310831in}{2.273731in}}%
\pgfpathlineto{\pgfqpoint{4.324234in}{2.278681in}}%
\pgfpathlineto{\pgfqpoint{4.337649in}{2.283794in}}%
\pgfpathlineto{\pgfqpoint{4.351075in}{2.289071in}}%
\pgfpathlineto{\pgfqpoint{4.358670in}{2.298798in}}%
\pgfpathlineto{\pgfqpoint{4.366260in}{2.308454in}}%
\pgfpathlineto{\pgfqpoint{4.373844in}{2.318041in}}%
\pgfpathlineto{\pgfqpoint{4.381422in}{2.327558in}}%
\pgfpathlineto{\pgfqpoint{4.368002in}{2.322290in}}%
\pgfpathlineto{\pgfqpoint{4.354593in}{2.317185in}}%
\pgfpathlineto{\pgfqpoint{4.341196in}{2.312244in}}%
\pgfpathlineto{\pgfqpoint{4.327810in}{2.307466in}}%
\pgfpathlineto{\pgfqpoint{4.320225in}{2.297930in}}%
\pgfpathlineto{\pgfqpoint{4.312635in}{2.288331in}}%
\pgfpathlineto{\pgfqpoint{4.305040in}{2.278670in}}%
\pgfpathlineto{\pgfqpoint{4.297439in}{2.268945in}}%
\pgfpathclose%
\pgfusepath{fill}%
\end{pgfscope}%
\begin{pgfscope}%
\pgfpathrectangle{\pgfqpoint{1.254980in}{0.150000in}}{\pgfqpoint{5.490039in}{5.490039in}}%
\pgfusepath{clip}%
\pgfsetbuttcap%
\pgfsetroundjoin%
\definecolor{currentfill}{rgb}{0.273809,0.031497,0.358853}%
\pgfsetfillcolor{currentfill}%
\pgfsetfillopacity{0.700000}%
\pgfsetlinewidth{0.000000pt}%
\definecolor{currentstroke}{rgb}{0.000000,0.000000,0.000000}%
\pgfsetstrokecolor{currentstroke}%
\pgfsetdash{}{0pt}%
\pgfpathmoveto{\pgfqpoint{3.458223in}{1.836890in}}%
\pgfpathlineto{\pgfqpoint{3.471383in}{1.833364in}}%
\pgfpathlineto{\pgfqpoint{3.484548in}{1.830020in}}%
\pgfpathlineto{\pgfqpoint{3.497716in}{1.826857in}}%
\pgfpathlineto{\pgfqpoint{3.510889in}{1.823873in}}%
\pgfpathlineto{\pgfqpoint{3.518769in}{1.833128in}}%
\pgfpathlineto{\pgfqpoint{3.526642in}{1.842423in}}%
\pgfpathlineto{\pgfqpoint{3.534510in}{1.851756in}}%
\pgfpathlineto{\pgfqpoint{3.542372in}{1.861126in}}%
\pgfpathlineto{\pgfqpoint{3.529212in}{1.863837in}}%
\pgfpathlineto{\pgfqpoint{3.516057in}{1.866729in}}%
\pgfpathlineto{\pgfqpoint{3.502906in}{1.869802in}}%
\pgfpathlineto{\pgfqpoint{3.489759in}{1.873056in}}%
\pgfpathlineto{\pgfqpoint{3.481884in}{1.863948in}}%
\pgfpathlineto{\pgfqpoint{3.474003in}{1.854883in}}%
\pgfpathlineto{\pgfqpoint{3.466116in}{1.845863in}}%
\pgfpathlineto{\pgfqpoint{3.458223in}{1.836890in}}%
\pgfpathclose%
\pgfusepath{fill}%
\end{pgfscope}%
\begin{pgfscope}%
\pgfpathrectangle{\pgfqpoint{1.254980in}{0.150000in}}{\pgfqpoint{5.490039in}{5.490039in}}%
\pgfusepath{clip}%
\pgfsetbuttcap%
\pgfsetroundjoin%
\definecolor{currentfill}{rgb}{0.135066,0.544853,0.554029}%
\pgfsetfillcolor{currentfill}%
\pgfsetfillopacity{0.700000}%
\pgfsetlinewidth{0.000000pt}%
\definecolor{currentstroke}{rgb}{0.000000,0.000000,0.000000}%
\pgfsetstrokecolor{currentstroke}%
\pgfsetdash{}{0pt}%
\pgfpathmoveto{\pgfqpoint{5.281847in}{2.979979in}}%
\pgfpathlineto{\pgfqpoint{5.295734in}{2.989398in}}%
\pgfpathlineto{\pgfqpoint{5.309638in}{2.998973in}}%
\pgfpathlineto{\pgfqpoint{5.323559in}{3.008704in}}%
\pgfpathlineto{\pgfqpoint{5.337498in}{3.018592in}}%
\pgfpathlineto{\pgfqpoint{5.344637in}{3.022818in}}%
\pgfpathlineto{\pgfqpoint{5.351769in}{3.027020in}}%
\pgfpathlineto{\pgfqpoint{5.358895in}{3.031202in}}%
\pgfpathlineto{\pgfqpoint{5.366015in}{3.035368in}}%
\pgfpathlineto{\pgfqpoint{5.352096in}{3.025873in}}%
\pgfpathlineto{\pgfqpoint{5.338195in}{3.016534in}}%
\pgfpathlineto{\pgfqpoint{5.324310in}{3.007351in}}%
\pgfpathlineto{\pgfqpoint{5.310442in}{2.998323in}}%
\pgfpathlineto{\pgfqpoint{5.303303in}{2.993755in}}%
\pgfpathlineto{\pgfqpoint{5.296157in}{2.989178in}}%
\pgfpathlineto{\pgfqpoint{5.289005in}{2.984587in}}%
\pgfpathlineto{\pgfqpoint{5.281847in}{2.979979in}}%
\pgfpathclose%
\pgfusepath{fill}%
\end{pgfscope}%
\begin{pgfscope}%
\pgfpathrectangle{\pgfqpoint{1.254980in}{0.150000in}}{\pgfqpoint{5.490039in}{5.490039in}}%
\pgfusepath{clip}%
\pgfsetbuttcap%
\pgfsetroundjoin%
\definecolor{currentfill}{rgb}{0.233603,0.313828,0.543914}%
\pgfsetfillcolor{currentfill}%
\pgfsetfillopacity{0.700000}%
\pgfsetlinewidth{0.000000pt}%
\definecolor{currentstroke}{rgb}{0.000000,0.000000,0.000000}%
\pgfsetstrokecolor{currentstroke}%
\pgfsetdash{}{0pt}%
\pgfpathmoveto{\pgfqpoint{2.482468in}{2.443085in}}%
\pgfpathlineto{\pgfqpoint{2.495897in}{2.424026in}}%
\pgfpathlineto{\pgfqpoint{2.509316in}{2.405246in}}%
\pgfpathlineto{\pgfqpoint{2.522725in}{2.386741in}}%
\pgfpathlineto{\pgfqpoint{2.536124in}{2.368509in}}%
\pgfpathlineto{\pgfqpoint{2.544557in}{2.370289in}}%
\pgfpathlineto{\pgfqpoint{2.552976in}{2.372274in}}%
\pgfpathlineto{\pgfqpoint{2.561381in}{2.374462in}}%
\pgfpathlineto{\pgfqpoint{2.569772in}{2.376849in}}%
\pgfpathlineto{\pgfqpoint{2.556412in}{2.394674in}}%
\pgfpathlineto{\pgfqpoint{2.543042in}{2.412771in}}%
\pgfpathlineto{\pgfqpoint{2.529663in}{2.431143in}}%
\pgfpathlineto{\pgfqpoint{2.516274in}{2.449792in}}%
\pgfpathlineto{\pgfqpoint{2.507844in}{2.447802in}}%
\pgfpathlineto{\pgfqpoint{2.499400in}{2.446018in}}%
\pgfpathlineto{\pgfqpoint{2.490942in}{2.444445in}}%
\pgfpathlineto{\pgfqpoint{2.482468in}{2.443085in}}%
\pgfpathclose%
\pgfusepath{fill}%
\end{pgfscope}%
\begin{pgfscope}%
\pgfpathrectangle{\pgfqpoint{1.254980in}{0.150000in}}{\pgfqpoint{5.490039in}{5.490039in}}%
\pgfusepath{clip}%
\pgfsetbuttcap%
\pgfsetroundjoin%
\definecolor{currentfill}{rgb}{0.282884,0.135920,0.453427}%
\pgfsetfillcolor{currentfill}%
\pgfsetfillopacity{0.700000}%
\pgfsetlinewidth{0.000000pt}%
\definecolor{currentstroke}{rgb}{0.000000,0.000000,0.000000}%
\pgfsetstrokecolor{currentstroke}%
\pgfsetdash{}{0pt}%
\pgfpathmoveto{\pgfqpoint{2.802440in}{2.057766in}}%
\pgfpathlineto{\pgfqpoint{2.815692in}{2.044767in}}%
\pgfpathlineto{\pgfqpoint{2.828939in}{2.031996in}}%
\pgfpathlineto{\pgfqpoint{2.842182in}{2.019452in}}%
\pgfpathlineto{\pgfqpoint{2.855421in}{2.007133in}}%
\pgfpathlineto{\pgfqpoint{2.863645in}{2.011371in}}%
\pgfpathlineto{\pgfqpoint{2.871858in}{2.015771in}}%
\pgfpathlineto{\pgfqpoint{2.880060in}{2.020328in}}%
\pgfpathlineto{\pgfqpoint{2.888250in}{2.025039in}}%
\pgfpathlineto{\pgfqpoint{2.875041in}{2.036966in}}%
\pgfpathlineto{\pgfqpoint{2.861829in}{2.049117in}}%
\pgfpathlineto{\pgfqpoint{2.848612in}{2.061494in}}%
\pgfpathlineto{\pgfqpoint{2.835392in}{2.074099in}}%
\pgfpathlineto{\pgfqpoint{2.827171in}{2.069770in}}%
\pgfpathlineto{\pgfqpoint{2.818939in}{2.065602in}}%
\pgfpathlineto{\pgfqpoint{2.810696in}{2.061600in}}%
\pgfpathlineto{\pgfqpoint{2.802440in}{2.057766in}}%
\pgfpathclose%
\pgfusepath{fill}%
\end{pgfscope}%
\begin{pgfscope}%
\pgfpathrectangle{\pgfqpoint{1.254980in}{0.150000in}}{\pgfqpoint{5.490039in}{5.490039in}}%
\pgfusepath{clip}%
\pgfsetbuttcap%
\pgfsetroundjoin%
\definecolor{currentfill}{rgb}{0.190631,0.407061,0.556089}%
\pgfsetfillcolor{currentfill}%
\pgfsetfillopacity{0.700000}%
\pgfsetlinewidth{0.000000pt}%
\definecolor{currentstroke}{rgb}{0.000000,0.000000,0.000000}%
\pgfsetstrokecolor{currentstroke}%
\pgfsetdash{}{0pt}%
\pgfpathmoveto{\pgfqpoint{4.747695in}{2.603525in}}%
\pgfpathlineto{\pgfqpoint{4.761301in}{2.611058in}}%
\pgfpathlineto{\pgfqpoint{4.774922in}{2.618752in}}%
\pgfpathlineto{\pgfqpoint{4.788558in}{2.626605in}}%
\pgfpathlineto{\pgfqpoint{4.802208in}{2.634618in}}%
\pgfpathlineto{\pgfqpoint{4.809623in}{2.642081in}}%
\pgfpathlineto{\pgfqpoint{4.817032in}{2.649469in}}%
\pgfpathlineto{\pgfqpoint{4.824435in}{2.656785in}}%
\pgfpathlineto{\pgfqpoint{4.831831in}{2.664031in}}%
\pgfpathlineto{\pgfqpoint{4.818190in}{2.656202in}}%
\pgfpathlineto{\pgfqpoint{4.804565in}{2.648532in}}%
\pgfpathlineto{\pgfqpoint{4.790953in}{2.641022in}}%
\pgfpathlineto{\pgfqpoint{4.777356in}{2.633671in}}%
\pgfpathlineto{\pgfqpoint{4.769950in}{2.626231in}}%
\pgfpathlineto{\pgfqpoint{4.762538in}{2.618728in}}%
\pgfpathlineto{\pgfqpoint{4.755119in}{2.611160in}}%
\pgfpathlineto{\pgfqpoint{4.747695in}{2.603525in}}%
\pgfpathclose%
\pgfusepath{fill}%
\end{pgfscope}%
\begin{pgfscope}%
\pgfpathrectangle{\pgfqpoint{1.254980in}{0.150000in}}{\pgfqpoint{5.490039in}{5.490039in}}%
\pgfusepath{clip}%
\pgfsetbuttcap%
\pgfsetroundjoin%
\definecolor{currentfill}{rgb}{0.127568,0.566949,0.550556}%
\pgfsetfillcolor{currentfill}%
\pgfsetfillopacity{0.700000}%
\pgfsetlinewidth{0.000000pt}%
\definecolor{currentstroke}{rgb}{0.000000,0.000000,0.000000}%
\pgfsetstrokecolor{currentstroke}%
\pgfsetdash{}{0pt}%
\pgfpathmoveto{\pgfqpoint{5.366015in}{3.035368in}}%
\pgfpathlineto{\pgfqpoint{5.379950in}{3.045019in}}%
\pgfpathlineto{\pgfqpoint{5.393903in}{3.054826in}}%
\pgfpathlineto{\pgfqpoint{5.407874in}{3.064789in}}%
\pgfpathlineto{\pgfqpoint{5.421862in}{3.074908in}}%
\pgfpathlineto{\pgfqpoint{5.428954in}{3.078651in}}%
\pgfpathlineto{\pgfqpoint{5.436039in}{3.082381in}}%
\pgfpathlineto{\pgfqpoint{5.443118in}{3.086102in}}%
\pgfpathlineto{\pgfqpoint{5.450191in}{3.089819in}}%
\pgfpathlineto{\pgfqpoint{5.436225in}{3.080123in}}%
\pgfpathlineto{\pgfqpoint{5.422276in}{3.070583in}}%
\pgfpathlineto{\pgfqpoint{5.408344in}{3.061197in}}%
\pgfpathlineto{\pgfqpoint{5.394430in}{3.051968in}}%
\pgfpathlineto{\pgfqpoint{5.387335in}{3.047818in}}%
\pgfpathlineto{\pgfqpoint{5.380234in}{3.043671in}}%
\pgfpathlineto{\pgfqpoint{5.373128in}{3.039523in}}%
\pgfpathlineto{\pgfqpoint{5.366015in}{3.035368in}}%
\pgfpathclose%
\pgfusepath{fill}%
\end{pgfscope}%
\begin{pgfscope}%
\pgfpathrectangle{\pgfqpoint{1.254980in}{0.150000in}}{\pgfqpoint{5.490039in}{5.490039in}}%
\pgfusepath{clip}%
\pgfsetbuttcap%
\pgfsetroundjoin%
\definecolor{currentfill}{rgb}{0.243113,0.292092,0.538516}%
\pgfsetfillcolor{currentfill}%
\pgfsetfillopacity{0.700000}%
\pgfsetlinewidth{0.000000pt}%
\definecolor{currentstroke}{rgb}{0.000000,0.000000,0.000000}%
\pgfsetstrokecolor{currentstroke}%
\pgfsetdash{}{0pt}%
\pgfpathmoveto{\pgfqpoint{4.381422in}{2.327558in}}%
\pgfpathlineto{\pgfqpoint{4.394855in}{2.332989in}}%
\pgfpathlineto{\pgfqpoint{4.408299in}{2.338583in}}%
\pgfpathlineto{\pgfqpoint{4.421756in}{2.344339in}}%
\pgfpathlineto{\pgfqpoint{4.435225in}{2.350258in}}%
\pgfpathlineto{\pgfqpoint{4.442792in}{2.359679in}}%
\pgfpathlineto{\pgfqpoint{4.450354in}{2.369025in}}%
\pgfpathlineto{\pgfqpoint{4.457910in}{2.378297in}}%
\pgfpathlineto{\pgfqpoint{4.465460in}{2.387495in}}%
\pgfpathlineto{\pgfqpoint{4.451997in}{2.381614in}}%
\pgfpathlineto{\pgfqpoint{4.438546in}{2.375895in}}%
\pgfpathlineto{\pgfqpoint{4.425108in}{2.370339in}}%
\pgfpathlineto{\pgfqpoint{4.411682in}{2.364945in}}%
\pgfpathlineto{\pgfqpoint{4.404125in}{2.355699in}}%
\pgfpathlineto{\pgfqpoint{4.396563in}{2.346386in}}%
\pgfpathlineto{\pgfqpoint{4.388995in}{2.337006in}}%
\pgfpathlineto{\pgfqpoint{4.381422in}{2.327558in}}%
\pgfpathclose%
\pgfusepath{fill}%
\end{pgfscope}%
\begin{pgfscope}%
\pgfpathrectangle{\pgfqpoint{1.254980in}{0.150000in}}{\pgfqpoint{5.490039in}{5.490039in}}%
\pgfusepath{clip}%
\pgfsetbuttcap%
\pgfsetroundjoin%
\definecolor{currentfill}{rgb}{0.272594,0.025563,0.353093}%
\pgfsetfillcolor{currentfill}%
\pgfsetfillopacity{0.700000}%
\pgfsetlinewidth{0.000000pt}%
\definecolor{currentstroke}{rgb}{0.000000,0.000000,0.000000}%
\pgfsetstrokecolor{currentstroke}%
\pgfsetdash{}{0pt}%
\pgfpathmoveto{\pgfqpoint{3.236770in}{1.831445in}}%
\pgfpathlineto{\pgfqpoint{3.249932in}{1.825049in}}%
\pgfpathlineto{\pgfqpoint{3.263094in}{1.818844in}}%
\pgfpathlineto{\pgfqpoint{3.276259in}{1.812830in}}%
\pgfpathlineto{\pgfqpoint{3.289425in}{1.807006in}}%
\pgfpathlineto{\pgfqpoint{3.297406in}{1.814823in}}%
\pgfpathlineto{\pgfqpoint{3.305379in}{1.822724in}}%
\pgfpathlineto{\pgfqpoint{3.313344in}{1.830706in}}%
\pgfpathlineto{\pgfqpoint{3.321303in}{1.838766in}}%
\pgfpathlineto{\pgfqpoint{3.308154in}{1.844262in}}%
\pgfpathlineto{\pgfqpoint{3.295008in}{1.849948in}}%
\pgfpathlineto{\pgfqpoint{3.281864in}{1.855824in}}%
\pgfpathlineto{\pgfqpoint{3.268722in}{1.861892in}}%
\pgfpathlineto{\pgfqpoint{3.260746in}{1.854150in}}%
\pgfpathlineto{\pgfqpoint{3.252761in}{1.846493in}}%
\pgfpathlineto{\pgfqpoint{3.244770in}{1.838924in}}%
\pgfpathlineto{\pgfqpoint{3.236770in}{1.831445in}}%
\pgfpathclose%
\pgfusepath{fill}%
\end{pgfscope}%
\begin{pgfscope}%
\pgfpathrectangle{\pgfqpoint{1.254980in}{0.150000in}}{\pgfqpoint{5.490039in}{5.490039in}}%
\pgfusepath{clip}%
\pgfsetbuttcap%
\pgfsetroundjoin%
\definecolor{currentfill}{rgb}{0.283197,0.115680,0.436115}%
\pgfsetfillcolor{currentfill}%
\pgfsetfillopacity{0.700000}%
\pgfsetlinewidth{0.000000pt}%
\definecolor{currentstroke}{rgb}{0.000000,0.000000,0.000000}%
\pgfsetstrokecolor{currentstroke}%
\pgfsetdash{}{0pt}%
\pgfpathmoveto{\pgfqpoint{2.855421in}{2.007133in}}%
\pgfpathlineto{\pgfqpoint{2.868657in}{1.995038in}}%
\pgfpathlineto{\pgfqpoint{2.881888in}{1.983165in}}%
\pgfpathlineto{\pgfqpoint{2.895117in}{1.971512in}}%
\pgfpathlineto{\pgfqpoint{2.908342in}{1.960077in}}%
\pgfpathlineto{\pgfqpoint{2.916536in}{1.964718in}}%
\pgfpathlineto{\pgfqpoint{2.924719in}{1.969512in}}%
\pgfpathlineto{\pgfqpoint{2.932892in}{1.974457in}}%
\pgfpathlineto{\pgfqpoint{2.941054in}{1.979548in}}%
\pgfpathlineto{\pgfqpoint{2.927858in}{1.990592in}}%
\pgfpathlineto{\pgfqpoint{2.914658in}{2.001854in}}%
\pgfpathlineto{\pgfqpoint{2.901456in}{2.013335in}}%
\pgfpathlineto{\pgfqpoint{2.888250in}{2.025039in}}%
\pgfpathlineto{\pgfqpoint{2.880060in}{2.020328in}}%
\pgfpathlineto{\pgfqpoint{2.871858in}{2.015771in}}%
\pgfpathlineto{\pgfqpoint{2.863645in}{2.011371in}}%
\pgfpathlineto{\pgfqpoint{2.855421in}{2.007133in}}%
\pgfpathclose%
\pgfusepath{fill}%
\end{pgfscope}%
\begin{pgfscope}%
\pgfpathrectangle{\pgfqpoint{1.254980in}{0.150000in}}{\pgfqpoint{5.490039in}{5.490039in}}%
\pgfusepath{clip}%
\pgfsetbuttcap%
\pgfsetroundjoin%
\definecolor{currentfill}{rgb}{0.218130,0.347432,0.550038}%
\pgfsetfillcolor{currentfill}%
\pgfsetfillopacity{0.700000}%
\pgfsetlinewidth{0.000000pt}%
\definecolor{currentstroke}{rgb}{0.000000,0.000000,0.000000}%
\pgfsetstrokecolor{currentstroke}%
\pgfsetdash{}{0pt}%
\pgfpathmoveto{\pgfqpoint{2.428640in}{2.522153in}}%
\pgfpathlineto{\pgfqpoint{2.442114in}{2.501956in}}%
\pgfpathlineto{\pgfqpoint{2.455576in}{2.482047in}}%
\pgfpathlineto{\pgfqpoint{2.469028in}{2.462424in}}%
\pgfpathlineto{\pgfqpoint{2.482468in}{2.443085in}}%
\pgfpathlineto{\pgfqpoint{2.490942in}{2.444445in}}%
\pgfpathlineto{\pgfqpoint{2.499400in}{2.446018in}}%
\pgfpathlineto{\pgfqpoint{2.507844in}{2.447802in}}%
\pgfpathlineto{\pgfqpoint{2.516274in}{2.449792in}}%
\pgfpathlineto{\pgfqpoint{2.502874in}{2.468722in}}%
\pgfpathlineto{\pgfqpoint{2.489464in}{2.487933in}}%
\pgfpathlineto{\pgfqpoint{2.476043in}{2.507430in}}%
\pgfpathlineto{\pgfqpoint{2.462610in}{2.527215in}}%
\pgfpathlineto{\pgfqpoint{2.454141in}{2.525624in}}%
\pgfpathlineto{\pgfqpoint{2.445656in}{2.524248in}}%
\pgfpathlineto{\pgfqpoint{2.437156in}{2.523090in}}%
\pgfpathlineto{\pgfqpoint{2.428640in}{2.522153in}}%
\pgfpathclose%
\pgfusepath{fill}%
\end{pgfscope}%
\begin{pgfscope}%
\pgfpathrectangle{\pgfqpoint{1.254980in}{0.150000in}}{\pgfqpoint{5.490039in}{5.490039in}}%
\pgfusepath{clip}%
\pgfsetbuttcap%
\pgfsetroundjoin%
\definecolor{currentfill}{rgb}{0.180629,0.429975,0.557282}%
\pgfsetfillcolor{currentfill}%
\pgfsetfillopacity{0.700000}%
\pgfsetlinewidth{0.000000pt}%
\definecolor{currentstroke}{rgb}{0.000000,0.000000,0.000000}%
\pgfsetstrokecolor{currentstroke}%
\pgfsetdash{}{0pt}%
\pgfpathmoveto{\pgfqpoint{4.831831in}{2.664031in}}%
\pgfpathlineto{\pgfqpoint{4.845486in}{2.672019in}}%
\pgfpathlineto{\pgfqpoint{4.859155in}{2.680167in}}%
\pgfpathlineto{\pgfqpoint{4.872840in}{2.688474in}}%
\pgfpathlineto{\pgfqpoint{4.886540in}{2.696940in}}%
\pgfpathlineto{\pgfqpoint{4.893919in}{2.703917in}}%
\pgfpathlineto{\pgfqpoint{4.901291in}{2.710822in}}%
\pgfpathlineto{\pgfqpoint{4.908657in}{2.717657in}}%
\pgfpathlineto{\pgfqpoint{4.916016in}{2.724426in}}%
\pgfpathlineto{\pgfqpoint{4.902327in}{2.716174in}}%
\pgfpathlineto{\pgfqpoint{4.888653in}{2.708080in}}%
\pgfpathlineto{\pgfqpoint{4.874994in}{2.700146in}}%
\pgfpathlineto{\pgfqpoint{4.861349in}{2.692370in}}%
\pgfpathlineto{\pgfqpoint{4.853979in}{2.685377in}}%
\pgfpathlineto{\pgfqpoint{4.846603in}{2.678325in}}%
\pgfpathlineto{\pgfqpoint{4.839220in}{2.671210in}}%
\pgfpathlineto{\pgfqpoint{4.831831in}{2.664031in}}%
\pgfpathclose%
\pgfusepath{fill}%
\end{pgfscope}%
\begin{pgfscope}%
\pgfpathrectangle{\pgfqpoint{1.254980in}{0.150000in}}{\pgfqpoint{5.490039in}{5.490039in}}%
\pgfusepath{clip}%
\pgfsetbuttcap%
\pgfsetroundjoin%
\definecolor{currentfill}{rgb}{0.121831,0.589055,0.545623}%
\pgfsetfillcolor{currentfill}%
\pgfsetfillopacity{0.700000}%
\pgfsetlinewidth{0.000000pt}%
\definecolor{currentstroke}{rgb}{0.000000,0.000000,0.000000}%
\pgfsetstrokecolor{currentstroke}%
\pgfsetdash{}{0pt}%
\pgfpathmoveto{\pgfqpoint{5.450191in}{3.089819in}}%
\pgfpathlineto{\pgfqpoint{5.464175in}{3.099670in}}%
\pgfpathlineto{\pgfqpoint{5.478176in}{3.109677in}}%
\pgfpathlineto{\pgfqpoint{5.492196in}{3.119840in}}%
\pgfpathlineto{\pgfqpoint{5.506233in}{3.130158in}}%
\pgfpathlineto{\pgfqpoint{5.513276in}{3.133434in}}%
\pgfpathlineto{\pgfqpoint{5.520314in}{3.136708in}}%
\pgfpathlineto{\pgfqpoint{5.527345in}{3.139987in}}%
\pgfpathlineto{\pgfqpoint{5.534370in}{3.143275in}}%
\pgfpathlineto{\pgfqpoint{5.520356in}{3.133410in}}%
\pgfpathlineto{\pgfqpoint{5.506360in}{3.123700in}}%
\pgfpathlineto{\pgfqpoint{5.492382in}{3.114145in}}%
\pgfpathlineto{\pgfqpoint{5.478421in}{3.104745in}}%
\pgfpathlineto{\pgfqpoint{5.471373in}{3.100995in}}%
\pgfpathlineto{\pgfqpoint{5.464318in}{3.097261in}}%
\pgfpathlineto{\pgfqpoint{5.457258in}{3.093537in}}%
\pgfpathlineto{\pgfqpoint{5.450191in}{3.089819in}}%
\pgfpathclose%
\pgfusepath{fill}%
\end{pgfscope}%
\begin{pgfscope}%
\pgfpathrectangle{\pgfqpoint{1.254980in}{0.150000in}}{\pgfqpoint{5.490039in}{5.490039in}}%
\pgfusepath{clip}%
\pgfsetbuttcap%
\pgfsetroundjoin%
\definecolor{currentfill}{rgb}{0.283197,0.115680,0.436115}%
\pgfsetfillcolor{currentfill}%
\pgfsetfillopacity{0.700000}%
\pgfsetlinewidth{0.000000pt}%
\definecolor{currentstroke}{rgb}{0.000000,0.000000,0.000000}%
\pgfsetstrokecolor{currentstroke}%
\pgfsetdash{}{0pt}%
\pgfpathmoveto{\pgfqpoint{3.847099in}{1.966571in}}%
\pgfpathlineto{\pgfqpoint{3.860336in}{1.967478in}}%
\pgfpathlineto{\pgfqpoint{3.873580in}{1.968555in}}%
\pgfpathlineto{\pgfqpoint{3.886832in}{1.969802in}}%
\pgfpathlineto{\pgfqpoint{3.900092in}{1.971218in}}%
\pgfpathlineto{\pgfqpoint{3.907837in}{1.981727in}}%
\pgfpathlineto{\pgfqpoint{3.915577in}{1.992211in}}%
\pgfpathlineto{\pgfqpoint{3.923313in}{2.002669in}}%
\pgfpathlineto{\pgfqpoint{3.931043in}{2.013101in}}%
\pgfpathlineto{\pgfqpoint{3.917790in}{2.011524in}}%
\pgfpathlineto{\pgfqpoint{3.904545in}{2.010116in}}%
\pgfpathlineto{\pgfqpoint{3.891309in}{2.008878in}}%
\pgfpathlineto{\pgfqpoint{3.878080in}{2.007810in}}%
\pgfpathlineto{\pgfqpoint{3.870342in}{1.997529in}}%
\pgfpathlineto{\pgfqpoint{3.862600in}{1.987228in}}%
\pgfpathlineto{\pgfqpoint{3.854852in}{1.976908in}}%
\pgfpathlineto{\pgfqpoint{3.847099in}{1.966571in}}%
\pgfpathclose%
\pgfusepath{fill}%
\end{pgfscope}%
\begin{pgfscope}%
\pgfpathrectangle{\pgfqpoint{1.254980in}{0.150000in}}{\pgfqpoint{5.490039in}{5.490039in}}%
\pgfusepath{clip}%
\pgfsetbuttcap%
\pgfsetroundjoin%
\definecolor{currentfill}{rgb}{0.281924,0.089666,0.412415}%
\pgfsetfillcolor{currentfill}%
\pgfsetfillopacity{0.700000}%
\pgfsetlinewidth{0.000000pt}%
\definecolor{currentstroke}{rgb}{0.000000,0.000000,0.000000}%
\pgfsetstrokecolor{currentstroke}%
\pgfsetdash{}{0pt}%
\pgfpathmoveto{\pgfqpoint{3.763137in}{1.923910in}}%
\pgfpathlineto{\pgfqpoint{3.776352in}{1.923944in}}%
\pgfpathlineto{\pgfqpoint{3.789573in}{1.924149in}}%
\pgfpathlineto{\pgfqpoint{3.802803in}{1.924526in}}%
\pgfpathlineto{\pgfqpoint{3.816039in}{1.925075in}}%
\pgfpathlineto{\pgfqpoint{3.823812in}{1.935468in}}%
\pgfpathlineto{\pgfqpoint{3.831579in}{1.945849in}}%
\pgfpathlineto{\pgfqpoint{3.839342in}{1.956217in}}%
\pgfpathlineto{\pgfqpoint{3.847099in}{1.966571in}}%
\pgfpathlineto{\pgfqpoint{3.833871in}{1.965834in}}%
\pgfpathlineto{\pgfqpoint{3.820650in}{1.965269in}}%
\pgfpathlineto{\pgfqpoint{3.807437in}{1.964875in}}%
\pgfpathlineto{\pgfqpoint{3.794230in}{1.964653in}}%
\pgfpathlineto{\pgfqpoint{3.786465in}{1.954477in}}%
\pgfpathlineto{\pgfqpoint{3.778694in}{1.944294in}}%
\pgfpathlineto{\pgfqpoint{3.770918in}{1.934105in}}%
\pgfpathlineto{\pgfqpoint{3.763137in}{1.923910in}}%
\pgfpathclose%
\pgfusepath{fill}%
\end{pgfscope}%
\begin{pgfscope}%
\pgfpathrectangle{\pgfqpoint{1.254980in}{0.150000in}}{\pgfqpoint{5.490039in}{5.490039in}}%
\pgfusepath{clip}%
\pgfsetbuttcap%
\pgfsetroundjoin%
\definecolor{currentfill}{rgb}{0.271305,0.019942,0.347269}%
\pgfsetfillcolor{currentfill}%
\pgfsetfillopacity{0.700000}%
\pgfsetlinewidth{0.000000pt}%
\definecolor{currentstroke}{rgb}{0.000000,0.000000,0.000000}%
\pgfsetstrokecolor{currentstroke}%
\pgfsetdash{}{0pt}%
\pgfpathmoveto{\pgfqpoint{3.373921in}{1.818664in}}%
\pgfpathlineto{\pgfqpoint{3.387082in}{1.814104in}}%
\pgfpathlineto{\pgfqpoint{3.400247in}{1.809729in}}%
\pgfpathlineto{\pgfqpoint{3.413415in}{1.805538in}}%
\pgfpathlineto{\pgfqpoint{3.426586in}{1.801530in}}%
\pgfpathlineto{\pgfqpoint{3.434505in}{1.810286in}}%
\pgfpathlineto{\pgfqpoint{3.442417in}{1.819099in}}%
\pgfpathlineto{\pgfqpoint{3.450323in}{1.827968in}}%
\pgfpathlineto{\pgfqpoint{3.458223in}{1.836890in}}%
\pgfpathlineto{\pgfqpoint{3.445066in}{1.840599in}}%
\pgfpathlineto{\pgfqpoint{3.431914in}{1.844490in}}%
\pgfpathlineto{\pgfqpoint{3.418764in}{1.848565in}}%
\pgfpathlineto{\pgfqpoint{3.405619in}{1.852825in}}%
\pgfpathlineto{\pgfqpoint{3.397704in}{1.844193in}}%
\pgfpathlineto{\pgfqpoint{3.389783in}{1.835619in}}%
\pgfpathlineto{\pgfqpoint{3.381855in}{1.827109in}}%
\pgfpathlineto{\pgfqpoint{3.373921in}{1.818664in}}%
\pgfpathclose%
\pgfusepath{fill}%
\end{pgfscope}%
\begin{pgfscope}%
\pgfpathrectangle{\pgfqpoint{1.254980in}{0.150000in}}{\pgfqpoint{5.490039in}{5.490039in}}%
\pgfusepath{clip}%
\pgfsetbuttcap%
\pgfsetroundjoin%
\definecolor{currentfill}{rgb}{0.276022,0.044167,0.370164}%
\pgfsetfillcolor{currentfill}%
\pgfsetfillopacity{0.700000}%
\pgfsetlinewidth{0.000000pt}%
\definecolor{currentstroke}{rgb}{0.000000,0.000000,0.000000}%
\pgfsetstrokecolor{currentstroke}%
\pgfsetdash{}{0pt}%
\pgfpathmoveto{\pgfqpoint{3.099265in}{1.863566in}}%
\pgfpathlineto{\pgfqpoint{3.112443in}{1.855240in}}%
\pgfpathlineto{\pgfqpoint{3.125621in}{1.847114in}}%
\pgfpathlineto{\pgfqpoint{3.138798in}{1.839188in}}%
\pgfpathlineto{\pgfqpoint{3.151976in}{1.831460in}}%
\pgfpathlineto{\pgfqpoint{3.160029in}{1.838174in}}%
\pgfpathlineto{\pgfqpoint{3.168074in}{1.844999in}}%
\pgfpathlineto{\pgfqpoint{3.176110in}{1.851933in}}%
\pgfpathlineto{\pgfqpoint{3.184137in}{1.858970in}}%
\pgfpathlineto{\pgfqpoint{3.170981in}{1.866341in}}%
\pgfpathlineto{\pgfqpoint{3.157826in}{1.873910in}}%
\pgfpathlineto{\pgfqpoint{3.144671in}{1.881678in}}%
\pgfpathlineto{\pgfqpoint{3.131516in}{1.889646in}}%
\pgfpathlineto{\pgfqpoint{3.123466in}{1.882956in}}%
\pgfpathlineto{\pgfqpoint{3.115408in}{1.876376in}}%
\pgfpathlineto{\pgfqpoint{3.107341in}{1.869912in}}%
\pgfpathlineto{\pgfqpoint{3.099265in}{1.863566in}}%
\pgfpathclose%
\pgfusepath{fill}%
\end{pgfscope}%
\begin{pgfscope}%
\pgfpathrectangle{\pgfqpoint{1.254980in}{0.150000in}}{\pgfqpoint{5.490039in}{5.490039in}}%
\pgfusepath{clip}%
\pgfsetbuttcap%
\pgfsetroundjoin%
\definecolor{currentfill}{rgb}{0.282623,0.140926,0.457517}%
\pgfsetfillcolor{currentfill}%
\pgfsetfillopacity{0.700000}%
\pgfsetlinewidth{0.000000pt}%
\definecolor{currentstroke}{rgb}{0.000000,0.000000,0.000000}%
\pgfsetstrokecolor{currentstroke}%
\pgfsetdash{}{0pt}%
\pgfpathmoveto{\pgfqpoint{3.931043in}{2.013101in}}%
\pgfpathlineto{\pgfqpoint{3.944305in}{2.014847in}}%
\pgfpathlineto{\pgfqpoint{3.957575in}{2.016761in}}%
\pgfpathlineto{\pgfqpoint{3.970854in}{2.018844in}}%
\pgfpathlineto{\pgfqpoint{3.984142in}{2.021094in}}%
\pgfpathlineto{\pgfqpoint{3.991861in}{2.031641in}}%
\pgfpathlineto{\pgfqpoint{3.999575in}{2.042151in}}%
\pgfpathlineto{\pgfqpoint{4.007284in}{2.052624in}}%
\pgfpathlineto{\pgfqpoint{4.014988in}{2.063060in}}%
\pgfpathlineto{\pgfqpoint{4.001706in}{2.060676in}}%
\pgfpathlineto{\pgfqpoint{3.988434in}{2.058461in}}%
\pgfpathlineto{\pgfqpoint{3.975171in}{2.056414in}}%
\pgfpathlineto{\pgfqpoint{3.961916in}{2.054535in}}%
\pgfpathlineto{\pgfqpoint{3.954205in}{2.044222in}}%
\pgfpathlineto{\pgfqpoint{3.946489in}{2.033878in}}%
\pgfpathlineto{\pgfqpoint{3.938769in}{2.023504in}}%
\pgfpathlineto{\pgfqpoint{3.931043in}{2.013101in}}%
\pgfpathclose%
\pgfusepath{fill}%
\end{pgfscope}%
\begin{pgfscope}%
\pgfpathrectangle{\pgfqpoint{1.254980in}{0.150000in}}{\pgfqpoint{5.490039in}{5.490039in}}%
\pgfusepath{clip}%
\pgfsetbuttcap%
\pgfsetroundjoin%
\definecolor{currentfill}{rgb}{0.279566,0.067836,0.391917}%
\pgfsetfillcolor{currentfill}%
\pgfsetfillopacity{0.700000}%
\pgfsetlinewidth{0.000000pt}%
\definecolor{currentstroke}{rgb}{0.000000,0.000000,0.000000}%
\pgfsetstrokecolor{currentstroke}%
\pgfsetdash{}{0pt}%
\pgfpathmoveto{\pgfqpoint{3.679132in}{1.885581in}}%
\pgfpathlineto{\pgfqpoint{3.692330in}{1.884706in}}%
\pgfpathlineto{\pgfqpoint{3.705534in}{1.884005in}}%
\pgfpathlineto{\pgfqpoint{3.718745in}{1.883477in}}%
\pgfpathlineto{\pgfqpoint{3.731962in}{1.883121in}}%
\pgfpathlineto{\pgfqpoint{3.739763in}{1.893317in}}%
\pgfpathlineto{\pgfqpoint{3.747559in}{1.903515in}}%
\pgfpathlineto{\pgfqpoint{3.755351in}{1.913713in}}%
\pgfpathlineto{\pgfqpoint{3.763137in}{1.923910in}}%
\pgfpathlineto{\pgfqpoint{3.749929in}{1.924049in}}%
\pgfpathlineto{\pgfqpoint{3.736728in}{1.924361in}}%
\pgfpathlineto{\pgfqpoint{3.723534in}{1.924846in}}%
\pgfpathlineto{\pgfqpoint{3.710346in}{1.925506in}}%
\pgfpathlineto{\pgfqpoint{3.702550in}{1.915514in}}%
\pgfpathlineto{\pgfqpoint{3.694750in}{1.905529in}}%
\pgfpathlineto{\pgfqpoint{3.686944in}{1.895550in}}%
\pgfpathlineto{\pgfqpoint{3.679132in}{1.885581in}}%
\pgfpathclose%
\pgfusepath{fill}%
\end{pgfscope}%
\begin{pgfscope}%
\pgfpathrectangle{\pgfqpoint{1.254980in}{0.150000in}}{\pgfqpoint{5.490039in}{5.490039in}}%
\pgfusepath{clip}%
\pgfsetbuttcap%
\pgfsetroundjoin%
\definecolor{currentfill}{rgb}{0.119423,0.611141,0.538982}%
\pgfsetfillcolor{currentfill}%
\pgfsetfillopacity{0.700000}%
\pgfsetlinewidth{0.000000pt}%
\definecolor{currentstroke}{rgb}{0.000000,0.000000,0.000000}%
\pgfsetstrokecolor{currentstroke}%
\pgfsetdash{}{0pt}%
\pgfpathmoveto{\pgfqpoint{5.534370in}{3.143275in}}%
\pgfpathlineto{\pgfqpoint{5.548401in}{3.153295in}}%
\pgfpathlineto{\pgfqpoint{5.562450in}{3.163470in}}%
\pgfpathlineto{\pgfqpoint{5.576518in}{3.173800in}}%
\pgfpathlineto{\pgfqpoint{5.590604in}{3.184286in}}%
\pgfpathlineto{\pgfqpoint{5.597598in}{3.187115in}}%
\pgfpathlineto{\pgfqpoint{5.604586in}{3.189957in}}%
\pgfpathlineto{\pgfqpoint{5.611568in}{3.192816in}}%
\pgfpathlineto{\pgfqpoint{5.618544in}{3.195699in}}%
\pgfpathlineto{\pgfqpoint{5.604484in}{3.185698in}}%
\pgfpathlineto{\pgfqpoint{5.590442in}{3.175850in}}%
\pgfpathlineto{\pgfqpoint{5.576419in}{3.166157in}}%
\pgfpathlineto{\pgfqpoint{5.562412in}{3.156619in}}%
\pgfpathlineto{\pgfqpoint{5.555410in}{3.153243in}}%
\pgfpathlineto{\pgfqpoint{5.548402in}{3.149898in}}%
\pgfpathlineto{\pgfqpoint{5.541389in}{3.146577in}}%
\pgfpathlineto{\pgfqpoint{5.534370in}{3.143275in}}%
\pgfpathclose%
\pgfusepath{fill}%
\end{pgfscope}%
\begin{pgfscope}%
\pgfpathrectangle{\pgfqpoint{1.254980in}{0.150000in}}{\pgfqpoint{5.490039in}{5.490039in}}%
\pgfusepath{clip}%
\pgfsetbuttcap%
\pgfsetroundjoin%
\definecolor{currentfill}{rgb}{0.229739,0.322361,0.545706}%
\pgfsetfillcolor{currentfill}%
\pgfsetfillopacity{0.700000}%
\pgfsetlinewidth{0.000000pt}%
\definecolor{currentstroke}{rgb}{0.000000,0.000000,0.000000}%
\pgfsetstrokecolor{currentstroke}%
\pgfsetdash{}{0pt}%
\pgfpathmoveto{\pgfqpoint{4.465460in}{2.387495in}}%
\pgfpathlineto{\pgfqpoint{4.478936in}{2.393538in}}%
\pgfpathlineto{\pgfqpoint{4.492424in}{2.399744in}}%
\pgfpathlineto{\pgfqpoint{4.505925in}{2.406111in}}%
\pgfpathlineto{\pgfqpoint{4.519439in}{2.412640in}}%
\pgfpathlineto{\pgfqpoint{4.526977in}{2.421709in}}%
\pgfpathlineto{\pgfqpoint{4.534510in}{2.430700in}}%
\pgfpathlineto{\pgfqpoint{4.542036in}{2.439613in}}%
\pgfpathlineto{\pgfqpoint{4.549557in}{2.448449in}}%
\pgfpathlineto{\pgfqpoint{4.536049in}{2.441987in}}%
\pgfpathlineto{\pgfqpoint{4.522555in}{2.435686in}}%
\pgfpathlineto{\pgfqpoint{4.509073in}{2.429547in}}%
\pgfpathlineto{\pgfqpoint{4.495604in}{2.423570in}}%
\pgfpathlineto{\pgfqpoint{4.488077in}{2.414657in}}%
\pgfpathlineto{\pgfqpoint{4.480544in}{2.405674in}}%
\pgfpathlineto{\pgfqpoint{4.473005in}{2.396620in}}%
\pgfpathlineto{\pgfqpoint{4.465460in}{2.387495in}}%
\pgfpathclose%
\pgfusepath{fill}%
\end{pgfscope}%
\begin{pgfscope}%
\pgfpathrectangle{\pgfqpoint{1.254980in}{0.150000in}}{\pgfqpoint{5.490039in}{5.490039in}}%
\pgfusepath{clip}%
\pgfsetbuttcap%
\pgfsetroundjoin%
\definecolor{currentfill}{rgb}{0.280255,0.165693,0.476498}%
\pgfsetfillcolor{currentfill}%
\pgfsetfillopacity{0.700000}%
\pgfsetlinewidth{0.000000pt}%
\definecolor{currentstroke}{rgb}{0.000000,0.000000,0.000000}%
\pgfsetstrokecolor{currentstroke}%
\pgfsetdash{}{0pt}%
\pgfpathmoveto{\pgfqpoint{4.014988in}{2.063060in}}%
\pgfpathlineto{\pgfqpoint{4.028279in}{2.065610in}}%
\pgfpathlineto{\pgfqpoint{4.041579in}{2.068328in}}%
\pgfpathlineto{\pgfqpoint{4.054888in}{2.071213in}}%
\pgfpathlineto{\pgfqpoint{4.068207in}{2.074264in}}%
\pgfpathlineto{\pgfqpoint{4.075901in}{2.084776in}}%
\pgfpathlineto{\pgfqpoint{4.083589in}{2.095241in}}%
\pgfpathlineto{\pgfqpoint{4.091272in}{2.105658in}}%
\pgfpathlineto{\pgfqpoint{4.098951in}{2.116027in}}%
\pgfpathlineto{\pgfqpoint{4.085638in}{2.112871in}}%
\pgfpathlineto{\pgfqpoint{4.072334in}{2.109882in}}%
\pgfpathlineto{\pgfqpoint{4.059040in}{2.107059in}}%
\pgfpathlineto{\pgfqpoint{4.045756in}{2.104403in}}%
\pgfpathlineto{\pgfqpoint{4.038071in}{2.094128in}}%
\pgfpathlineto{\pgfqpoint{4.030382in}{2.083812in}}%
\pgfpathlineto{\pgfqpoint{4.022687in}{2.073456in}}%
\pgfpathlineto{\pgfqpoint{4.014988in}{2.063060in}}%
\pgfpathclose%
\pgfusepath{fill}%
\end{pgfscope}%
\begin{pgfscope}%
\pgfpathrectangle{\pgfqpoint{1.254980in}{0.150000in}}{\pgfqpoint{5.490039in}{5.490039in}}%
\pgfusepath{clip}%
\pgfsetbuttcap%
\pgfsetroundjoin%
\definecolor{currentfill}{rgb}{0.282327,0.094955,0.417331}%
\pgfsetfillcolor{currentfill}%
\pgfsetfillopacity{0.700000}%
\pgfsetlinewidth{0.000000pt}%
\definecolor{currentstroke}{rgb}{0.000000,0.000000,0.000000}%
\pgfsetstrokecolor{currentstroke}%
\pgfsetdash{}{0pt}%
\pgfpathmoveto{\pgfqpoint{2.908342in}{1.960077in}}%
\pgfpathlineto{\pgfqpoint{2.921564in}{1.948861in}}%
\pgfpathlineto{\pgfqpoint{2.934784in}{1.937860in}}%
\pgfpathlineto{\pgfqpoint{2.948000in}{1.927074in}}%
\pgfpathlineto{\pgfqpoint{2.961215in}{1.916501in}}%
\pgfpathlineto{\pgfqpoint{2.969381in}{1.921542in}}%
\pgfpathlineto{\pgfqpoint{2.977536in}{1.926729in}}%
\pgfpathlineto{\pgfqpoint{2.985681in}{1.932060in}}%
\pgfpathlineto{\pgfqpoint{2.993816in}{1.937531in}}%
\pgfpathlineto{\pgfqpoint{2.980629in}{1.947714in}}%
\pgfpathlineto{\pgfqpoint{2.967440in}{1.958111in}}%
\pgfpathlineto{\pgfqpoint{2.954248in}{1.968722in}}%
\pgfpathlineto{\pgfqpoint{2.941054in}{1.979548in}}%
\pgfpathlineto{\pgfqpoint{2.932892in}{1.974457in}}%
\pgfpathlineto{\pgfqpoint{2.924719in}{1.969512in}}%
\pgfpathlineto{\pgfqpoint{2.916536in}{1.964718in}}%
\pgfpathlineto{\pgfqpoint{2.908342in}{1.960077in}}%
\pgfpathclose%
\pgfusepath{fill}%
\end{pgfscope}%
\begin{pgfscope}%
\pgfpathrectangle{\pgfqpoint{1.254980in}{0.150000in}}{\pgfqpoint{5.490039in}{5.490039in}}%
\pgfusepath{clip}%
\pgfsetbuttcap%
\pgfsetroundjoin%
\definecolor{currentfill}{rgb}{0.169646,0.456262,0.558030}%
\pgfsetfillcolor{currentfill}%
\pgfsetfillopacity{0.700000}%
\pgfsetlinewidth{0.000000pt}%
\definecolor{currentstroke}{rgb}{0.000000,0.000000,0.000000}%
\pgfsetstrokecolor{currentstroke}%
\pgfsetdash{}{0pt}%
\pgfpathmoveto{\pgfqpoint{4.916016in}{2.724426in}}%
\pgfpathlineto{\pgfqpoint{4.929720in}{2.732837in}}%
\pgfpathlineto{\pgfqpoint{4.943439in}{2.741407in}}%
\pgfpathlineto{\pgfqpoint{4.957174in}{2.750136in}}%
\pgfpathlineto{\pgfqpoint{4.970925in}{2.759023in}}%
\pgfpathlineto{\pgfqpoint{4.978265in}{2.765496in}}%
\pgfpathlineto{\pgfqpoint{4.985599in}{2.771900in}}%
\pgfpathlineto{\pgfqpoint{4.992925in}{2.778239in}}%
\pgfpathlineto{\pgfqpoint{5.000245in}{2.784515in}}%
\pgfpathlineto{\pgfqpoint{4.986507in}{2.775872in}}%
\pgfpathlineto{\pgfqpoint{4.972784in}{2.767387in}}%
\pgfpathlineto{\pgfqpoint{4.959077in}{2.759060in}}%
\pgfpathlineto{\pgfqpoint{4.945384in}{2.750891in}}%
\pgfpathlineto{\pgfqpoint{4.938052in}{2.744361in}}%
\pgfpathlineto{\pgfqpoint{4.930713in}{2.737775in}}%
\pgfpathlineto{\pgfqpoint{4.923368in}{2.731131in}}%
\pgfpathlineto{\pgfqpoint{4.916016in}{2.724426in}}%
\pgfpathclose%
\pgfusepath{fill}%
\end{pgfscope}%
\begin{pgfscope}%
\pgfpathrectangle{\pgfqpoint{1.254980in}{0.150000in}}{\pgfqpoint{5.490039in}{5.490039in}}%
\pgfusepath{clip}%
\pgfsetbuttcap%
\pgfsetroundjoin%
\definecolor{currentfill}{rgb}{0.121380,0.629492,0.531973}%
\pgfsetfillcolor{currentfill}%
\pgfsetfillopacity{0.700000}%
\pgfsetlinewidth{0.000000pt}%
\definecolor{currentstroke}{rgb}{0.000000,0.000000,0.000000}%
\pgfsetstrokecolor{currentstroke}%
\pgfsetdash{}{0pt}%
\pgfpathmoveto{\pgfqpoint{5.618544in}{3.195699in}}%
\pgfpathlineto{\pgfqpoint{5.632622in}{3.205856in}}%
\pgfpathlineto{\pgfqpoint{5.646719in}{3.216167in}}%
\pgfpathlineto{\pgfqpoint{5.660834in}{3.226633in}}%
\pgfpathlineto{\pgfqpoint{5.674967in}{3.237254in}}%
\pgfpathlineto{\pgfqpoint{5.681911in}{3.239662in}}%
\pgfpathlineto{\pgfqpoint{5.688849in}{3.242099in}}%
\pgfpathlineto{\pgfqpoint{5.695782in}{3.244568in}}%
\pgfpathlineto{\pgfqpoint{5.702710in}{3.247076in}}%
\pgfpathlineto{\pgfqpoint{5.688604in}{3.236969in}}%
\pgfpathlineto{\pgfqpoint{5.674517in}{3.227016in}}%
\pgfpathlineto{\pgfqpoint{5.660448in}{3.217217in}}%
\pgfpathlineto{\pgfqpoint{5.646398in}{3.207572in}}%
\pgfpathlineto{\pgfqpoint{5.639442in}{3.204542in}}%
\pgfpathlineto{\pgfqpoint{5.632481in}{3.201556in}}%
\pgfpathlineto{\pgfqpoint{5.625515in}{3.198611in}}%
\pgfpathlineto{\pgfqpoint{5.618544in}{3.195699in}}%
\pgfpathclose%
\pgfusepath{fill}%
\end{pgfscope}%
\begin{pgfscope}%
\pgfpathrectangle{\pgfqpoint{1.254980in}{0.150000in}}{\pgfqpoint{5.490039in}{5.490039in}}%
\pgfusepath{clip}%
\pgfsetbuttcap%
\pgfsetroundjoin%
\definecolor{currentfill}{rgb}{0.277018,0.050344,0.375715}%
\pgfsetfillcolor{currentfill}%
\pgfsetfillopacity{0.700000}%
\pgfsetlinewidth{0.000000pt}%
\definecolor{currentstroke}{rgb}{0.000000,0.000000,0.000000}%
\pgfsetstrokecolor{currentstroke}%
\pgfsetdash{}{0pt}%
\pgfpathmoveto{\pgfqpoint{3.595060in}{1.852067in}}%
\pgfpathlineto{\pgfqpoint{3.608245in}{1.850246in}}%
\pgfpathlineto{\pgfqpoint{3.621436in}{1.848602in}}%
\pgfpathlineto{\pgfqpoint{3.634632in}{1.847133in}}%
\pgfpathlineto{\pgfqpoint{3.647834in}{1.845839in}}%
\pgfpathlineto{\pgfqpoint{3.655667in}{1.855750in}}%
\pgfpathlineto{\pgfqpoint{3.663494in}{1.865679in}}%
\pgfpathlineto{\pgfqpoint{3.671316in}{1.875624in}}%
\pgfpathlineto{\pgfqpoint{3.679132in}{1.885581in}}%
\pgfpathlineto{\pgfqpoint{3.665941in}{1.886631in}}%
\pgfpathlineto{\pgfqpoint{3.652756in}{1.887856in}}%
\pgfpathlineto{\pgfqpoint{3.639576in}{1.889257in}}%
\pgfpathlineto{\pgfqpoint{3.626402in}{1.890834in}}%
\pgfpathlineto{\pgfqpoint{3.618575in}{1.881110in}}%
\pgfpathlineto{\pgfqpoint{3.610742in}{1.871406in}}%
\pgfpathlineto{\pgfqpoint{3.602904in}{1.861724in}}%
\pgfpathlineto{\pgfqpoint{3.595060in}{1.852067in}}%
\pgfpathclose%
\pgfusepath{fill}%
\end{pgfscope}%
\begin{pgfscope}%
\pgfpathrectangle{\pgfqpoint{1.254980in}{0.150000in}}{\pgfqpoint{5.490039in}{5.490039in}}%
\pgfusepath{clip}%
\pgfsetbuttcap%
\pgfsetroundjoin%
\definecolor{currentfill}{rgb}{0.201239,0.383670,0.554294}%
\pgfsetfillcolor{currentfill}%
\pgfsetfillopacity{0.700000}%
\pgfsetlinewidth{0.000000pt}%
\definecolor{currentstroke}{rgb}{0.000000,0.000000,0.000000}%
\pgfsetstrokecolor{currentstroke}%
\pgfsetdash{}{0pt}%
\pgfpathmoveto{\pgfqpoint{2.374622in}{2.605887in}}%
\pgfpathlineto{\pgfqpoint{2.388145in}{2.584506in}}%
\pgfpathlineto{\pgfqpoint{2.401656in}{2.563426in}}%
\pgfpathlineto{\pgfqpoint{2.415154in}{2.542642in}}%
\pgfpathlineto{\pgfqpoint{2.428640in}{2.522153in}}%
\pgfpathlineto{\pgfqpoint{2.437156in}{2.523090in}}%
\pgfpathlineto{\pgfqpoint{2.445656in}{2.524248in}}%
\pgfpathlineto{\pgfqpoint{2.454141in}{2.525624in}}%
\pgfpathlineto{\pgfqpoint{2.462610in}{2.527215in}}%
\pgfpathlineto{\pgfqpoint{2.449166in}{2.547290in}}%
\pgfpathlineto{\pgfqpoint{2.435711in}{2.567659in}}%
\pgfpathlineto{\pgfqpoint{2.422243in}{2.588324in}}%
\pgfpathlineto{\pgfqpoint{2.408763in}{2.609288in}}%
\pgfpathlineto{\pgfqpoint{2.400252in}{2.608101in}}%
\pgfpathlineto{\pgfqpoint{2.391725in}{2.607136in}}%
\pgfpathlineto{\pgfqpoint{2.383181in}{2.606397in}}%
\pgfpathlineto{\pgfqpoint{2.374622in}{2.605887in}}%
\pgfpathclose%
\pgfusepath{fill}%
\end{pgfscope}%
\begin{pgfscope}%
\pgfpathrectangle{\pgfqpoint{1.254980in}{0.150000in}}{\pgfqpoint{5.490039in}{5.490039in}}%
\pgfusepath{clip}%
\pgfsetbuttcap%
\pgfsetroundjoin%
\definecolor{currentfill}{rgb}{0.130067,0.651384,0.521608}%
\pgfsetfillcolor{currentfill}%
\pgfsetfillopacity{0.700000}%
\pgfsetlinewidth{0.000000pt}%
\definecolor{currentstroke}{rgb}{0.000000,0.000000,0.000000}%
\pgfsetstrokecolor{currentstroke}%
\pgfsetdash{}{0pt}%
\pgfpathmoveto{\pgfqpoint{5.702710in}{3.247076in}}%
\pgfpathlineto{\pgfqpoint{5.716834in}{3.257337in}}%
\pgfpathlineto{\pgfqpoint{5.730976in}{3.267752in}}%
\pgfpathlineto{\pgfqpoint{5.745138in}{3.278322in}}%
\pgfpathlineto{\pgfqpoint{5.759318in}{3.289046in}}%
\pgfpathlineto{\pgfqpoint{5.766211in}{3.291065in}}%
\pgfpathlineto{\pgfqpoint{5.773099in}{3.293128in}}%
\pgfpathlineto{\pgfqpoint{5.779983in}{3.295241in}}%
\pgfpathlineto{\pgfqpoint{5.786861in}{3.297408in}}%
\pgfpathlineto{\pgfqpoint{5.772711in}{3.287228in}}%
\pgfpathlineto{\pgfqpoint{5.758580in}{3.277201in}}%
\pgfpathlineto{\pgfqpoint{5.744467in}{3.267328in}}%
\pgfpathlineto{\pgfqpoint{5.730373in}{3.257608in}}%
\pgfpathlineto{\pgfqpoint{5.723464in}{3.254888in}}%
\pgfpathlineto{\pgfqpoint{5.716550in}{3.252230in}}%
\pgfpathlineto{\pgfqpoint{5.709632in}{3.249628in}}%
\pgfpathlineto{\pgfqpoint{5.702710in}{3.247076in}}%
\pgfpathclose%
\pgfusepath{fill}%
\end{pgfscope}%
\begin{pgfscope}%
\pgfpathrectangle{\pgfqpoint{1.254980in}{0.150000in}}{\pgfqpoint{5.490039in}{5.490039in}}%
\pgfusepath{clip}%
\pgfsetbuttcap%
\pgfsetroundjoin%
\definecolor{currentfill}{rgb}{0.275191,0.194905,0.496005}%
\pgfsetfillcolor{currentfill}%
\pgfsetfillopacity{0.700000}%
\pgfsetlinewidth{0.000000pt}%
\definecolor{currentstroke}{rgb}{0.000000,0.000000,0.000000}%
\pgfsetstrokecolor{currentstroke}%
\pgfsetdash{}{0pt}%
\pgfpathmoveto{\pgfqpoint{4.098951in}{2.116027in}}%
\pgfpathlineto{\pgfqpoint{4.112274in}{2.119349in}}%
\pgfpathlineto{\pgfqpoint{4.125607in}{2.122837in}}%
\pgfpathlineto{\pgfqpoint{4.138951in}{2.126491in}}%
\pgfpathlineto{\pgfqpoint{4.152304in}{2.130310in}}%
\pgfpathlineto{\pgfqpoint{4.159972in}{2.140718in}}%
\pgfpathlineto{\pgfqpoint{4.167635in}{2.151069in}}%
\pgfpathlineto{\pgfqpoint{4.175293in}{2.161365in}}%
\pgfpathlineto{\pgfqpoint{4.182946in}{2.171603in}}%
\pgfpathlineto{\pgfqpoint{4.169598in}{2.167707in}}%
\pgfpathlineto{\pgfqpoint{4.156260in}{2.163977in}}%
\pgfpathlineto{\pgfqpoint{4.142932in}{2.160412in}}%
\pgfpathlineto{\pgfqpoint{4.129615in}{2.157013in}}%
\pgfpathlineto{\pgfqpoint{4.121957in}{2.146841in}}%
\pgfpathlineto{\pgfqpoint{4.114293in}{2.136619in}}%
\pgfpathlineto{\pgfqpoint{4.106624in}{2.126348in}}%
\pgfpathlineto{\pgfqpoint{4.098951in}{2.116027in}}%
\pgfpathclose%
\pgfusepath{fill}%
\end{pgfscope}%
\begin{pgfscope}%
\pgfpathrectangle{\pgfqpoint{1.254980in}{0.150000in}}{\pgfqpoint{5.490039in}{5.490039in}}%
\pgfusepath{clip}%
\pgfsetbuttcap%
\pgfsetroundjoin%
\definecolor{currentfill}{rgb}{0.208030,0.718701,0.472873}%
\pgfsetfillcolor{currentfill}%
\pgfsetfillopacity{0.700000}%
\pgfsetlinewidth{0.000000pt}%
\definecolor{currentstroke}{rgb}{0.000000,0.000000,0.000000}%
\pgfsetstrokecolor{currentstroke}%
\pgfsetdash{}{0pt}%
\pgfpathmoveto{\pgfqpoint{6.039193in}{3.442496in}}%
\pgfpathlineto{\pgfqpoint{6.053488in}{3.452858in}}%
\pgfpathlineto{\pgfqpoint{6.067803in}{3.463374in}}%
\pgfpathlineto{\pgfqpoint{6.082138in}{3.474041in}}%
\pgfpathlineto{\pgfqpoint{6.088841in}{3.475079in}}%
\pgfpathlineto{\pgfqpoint{6.095543in}{3.476239in}}%
\pgfpathlineto{\pgfqpoint{6.102243in}{3.477528in}}%
\pgfpathlineto{\pgfqpoint{6.108941in}{3.478953in}}%
\pgfpathlineto{\pgfqpoint{6.094646in}{3.468947in}}%
\pgfpathlineto{\pgfqpoint{6.080371in}{3.459093in}}%
\pgfpathlineto{\pgfqpoint{6.066115in}{3.449390in}}%
\pgfpathlineto{\pgfqpoint{6.059386in}{3.447462in}}%
\pgfpathlineto{\pgfqpoint{6.052656in}{3.445676in}}%
\pgfpathlineto{\pgfqpoint{6.045926in}{3.444022in}}%
\pgfpathlineto{\pgfqpoint{6.039193in}{3.442496in}}%
\pgfpathclose%
\pgfusepath{fill}%
\end{pgfscope}%
\begin{pgfscope}%
\pgfpathrectangle{\pgfqpoint{1.254980in}{0.150000in}}{\pgfqpoint{5.490039in}{5.490039in}}%
\pgfusepath{clip}%
\pgfsetbuttcap%
\pgfsetroundjoin%
\definecolor{currentfill}{rgb}{0.218130,0.347432,0.550038}%
\pgfsetfillcolor{currentfill}%
\pgfsetfillopacity{0.700000}%
\pgfsetlinewidth{0.000000pt}%
\definecolor{currentstroke}{rgb}{0.000000,0.000000,0.000000}%
\pgfsetstrokecolor{currentstroke}%
\pgfsetdash{}{0pt}%
\pgfpathmoveto{\pgfqpoint{4.549557in}{2.448449in}}%
\pgfpathlineto{\pgfqpoint{4.563077in}{2.455072in}}%
\pgfpathlineto{\pgfqpoint{4.576611in}{2.461857in}}%
\pgfpathlineto{\pgfqpoint{4.590158in}{2.468803in}}%
\pgfpathlineto{\pgfqpoint{4.603719in}{2.475911in}}%
\pgfpathlineto{\pgfqpoint{4.611227in}{2.484586in}}%
\pgfpathlineto{\pgfqpoint{4.618729in}{2.493181in}}%
\pgfpathlineto{\pgfqpoint{4.626224in}{2.501695in}}%
\pgfpathlineto{\pgfqpoint{4.633714in}{2.510131in}}%
\pgfpathlineto{\pgfqpoint{4.620160in}{2.503120in}}%
\pgfpathlineto{\pgfqpoint{4.606620in}{2.496269in}}%
\pgfpathlineto{\pgfqpoint{4.593093in}{2.489580in}}%
\pgfpathlineto{\pgfqpoint{4.579580in}{2.483052in}}%
\pgfpathlineto{\pgfqpoint{4.572083in}{2.474510in}}%
\pgfpathlineto{\pgfqpoint{4.564580in}{2.465896in}}%
\pgfpathlineto{\pgfqpoint{4.557071in}{2.457209in}}%
\pgfpathlineto{\pgfqpoint{4.549557in}{2.448449in}}%
\pgfpathclose%
\pgfusepath{fill}%
\end{pgfscope}%
\begin{pgfscope}%
\pgfpathrectangle{\pgfqpoint{1.254980in}{0.150000in}}{\pgfqpoint{5.490039in}{5.490039in}}%
\pgfusepath{clip}%
\pgfsetbuttcap%
\pgfsetroundjoin%
\definecolor{currentfill}{rgb}{0.143303,0.669459,0.511215}%
\pgfsetfillcolor{currentfill}%
\pgfsetfillopacity{0.700000}%
\pgfsetlinewidth{0.000000pt}%
\definecolor{currentstroke}{rgb}{0.000000,0.000000,0.000000}%
\pgfsetstrokecolor{currentstroke}%
\pgfsetdash{}{0pt}%
\pgfpathmoveto{\pgfqpoint{5.786861in}{3.297408in}}%
\pgfpathlineto{\pgfqpoint{5.801030in}{3.307742in}}%
\pgfpathlineto{\pgfqpoint{5.815218in}{3.318230in}}%
\pgfpathlineto{\pgfqpoint{5.829424in}{3.328872in}}%
\pgfpathlineto{\pgfqpoint{5.843650in}{3.339668in}}%
\pgfpathlineto{\pgfqpoint{5.850493in}{3.341334in}}%
\pgfpathlineto{\pgfqpoint{5.857331in}{3.343061in}}%
\pgfpathlineto{\pgfqpoint{5.864164in}{3.344855in}}%
\pgfpathlineto{\pgfqpoint{5.870994in}{3.346722in}}%
\pgfpathlineto{\pgfqpoint{5.856801in}{3.336500in}}%
\pgfpathlineto{\pgfqpoint{5.842627in}{3.326431in}}%
\pgfpathlineto{\pgfqpoint{5.828471in}{3.316515in}}%
\pgfpathlineto{\pgfqpoint{5.814334in}{3.306753in}}%
\pgfpathlineto{\pgfqpoint{5.807472in}{3.304303in}}%
\pgfpathlineto{\pgfqpoint{5.800605in}{3.301933in}}%
\pgfpathlineto{\pgfqpoint{5.793735in}{3.299637in}}%
\pgfpathlineto{\pgfqpoint{5.786861in}{3.297408in}}%
\pgfpathclose%
\pgfusepath{fill}%
\end{pgfscope}%
\begin{pgfscope}%
\pgfpathrectangle{\pgfqpoint{1.254980in}{0.150000in}}{\pgfqpoint{5.490039in}{5.490039in}}%
\pgfusepath{clip}%
\pgfsetbuttcap%
\pgfsetroundjoin%
\definecolor{currentfill}{rgb}{0.160665,0.478540,0.558115}%
\pgfsetfillcolor{currentfill}%
\pgfsetfillopacity{0.700000}%
\pgfsetlinewidth{0.000000pt}%
\definecolor{currentstroke}{rgb}{0.000000,0.000000,0.000000}%
\pgfsetstrokecolor{currentstroke}%
\pgfsetdash{}{0pt}%
\pgfpathmoveto{\pgfqpoint{5.000245in}{2.784515in}}%
\pgfpathlineto{\pgfqpoint{5.013999in}{2.793317in}}%
\pgfpathlineto{\pgfqpoint{5.027769in}{2.802277in}}%
\pgfpathlineto{\pgfqpoint{5.041555in}{2.811395in}}%
\pgfpathlineto{\pgfqpoint{5.055356in}{2.820672in}}%
\pgfpathlineto{\pgfqpoint{5.062656in}{2.826627in}}%
\pgfpathlineto{\pgfqpoint{5.069949in}{2.832518in}}%
\pgfpathlineto{\pgfqpoint{5.077235in}{2.838349in}}%
\pgfpathlineto{\pgfqpoint{5.084514in}{2.844123in}}%
\pgfpathlineto{\pgfqpoint{5.070726in}{2.835120in}}%
\pgfpathlineto{\pgfqpoint{5.056954in}{2.826275in}}%
\pgfpathlineto{\pgfqpoint{5.043198in}{2.817589in}}%
\pgfpathlineto{\pgfqpoint{5.029457in}{2.809059in}}%
\pgfpathlineto{\pgfqpoint{5.022164in}{2.803001in}}%
\pgfpathlineto{\pgfqpoint{5.014864in}{2.796893in}}%
\pgfpathlineto{\pgfqpoint{5.007558in}{2.790732in}}%
\pgfpathlineto{\pgfqpoint{5.000245in}{2.784515in}}%
\pgfpathclose%
\pgfusepath{fill}%
\end{pgfscope}%
\begin{pgfscope}%
\pgfpathrectangle{\pgfqpoint{1.254980in}{0.150000in}}{\pgfqpoint{5.490039in}{5.490039in}}%
\pgfusepath{clip}%
\pgfsetbuttcap%
\pgfsetroundjoin%
\definecolor{currentfill}{rgb}{0.162016,0.687316,0.499129}%
\pgfsetfillcolor{currentfill}%
\pgfsetfillopacity{0.700000}%
\pgfsetlinewidth{0.000000pt}%
\definecolor{currentstroke}{rgb}{0.000000,0.000000,0.000000}%
\pgfsetstrokecolor{currentstroke}%
\pgfsetdash{}{0pt}%
\pgfpathmoveto{\pgfqpoint{5.870994in}{3.346722in}}%
\pgfpathlineto{\pgfqpoint{5.885206in}{3.357097in}}%
\pgfpathlineto{\pgfqpoint{5.899438in}{3.367625in}}%
\pgfpathlineto{\pgfqpoint{5.913689in}{3.378307in}}%
\pgfpathlineto{\pgfqpoint{5.927959in}{3.389143in}}%
\pgfpathlineto{\pgfqpoint{5.934751in}{3.390497in}}%
\pgfpathlineto{\pgfqpoint{5.941539in}{3.391931in}}%
\pgfpathlineto{\pgfqpoint{5.948324in}{3.393450in}}%
\pgfpathlineto{\pgfqpoint{5.955106in}{3.395062in}}%
\pgfpathlineto{\pgfqpoint{5.940870in}{3.384830in}}%
\pgfpathlineto{\pgfqpoint{5.926654in}{3.374751in}}%
\pgfpathlineto{\pgfqpoint{5.912457in}{3.364824in}}%
\pgfpathlineto{\pgfqpoint{5.898279in}{3.355050in}}%
\pgfpathlineto{\pgfqpoint{5.891462in}{3.352826in}}%
\pgfpathlineto{\pgfqpoint{5.884643in}{3.350701in}}%
\pgfpathlineto{\pgfqpoint{5.877820in}{3.348669in}}%
\pgfpathlineto{\pgfqpoint{5.870994in}{3.346722in}}%
\pgfpathclose%
\pgfusepath{fill}%
\end{pgfscope}%
\begin{pgfscope}%
\pgfpathrectangle{\pgfqpoint{1.254980in}{0.150000in}}{\pgfqpoint{5.490039in}{5.490039in}}%
\pgfusepath{clip}%
\pgfsetbuttcap%
\pgfsetroundjoin%
\definecolor{currentfill}{rgb}{0.273809,0.031497,0.358853}%
\pgfsetfillcolor{currentfill}%
\pgfsetfillopacity{0.700000}%
\pgfsetlinewidth{0.000000pt}%
\definecolor{currentstroke}{rgb}{0.000000,0.000000,0.000000}%
\pgfsetstrokecolor{currentstroke}%
\pgfsetdash{}{0pt}%
\pgfpathmoveto{\pgfqpoint{3.510889in}{1.823873in}}%
\pgfpathlineto{\pgfqpoint{3.524067in}{1.821070in}}%
\pgfpathlineto{\pgfqpoint{3.537249in}{1.818445in}}%
\pgfpathlineto{\pgfqpoint{3.550436in}{1.815998in}}%
\pgfpathlineto{\pgfqpoint{3.563628in}{1.813729in}}%
\pgfpathlineto{\pgfqpoint{3.571494in}{1.823265in}}%
\pgfpathlineto{\pgfqpoint{3.579355in}{1.832835in}}%
\pgfpathlineto{\pgfqpoint{3.587210in}{1.842437in}}%
\pgfpathlineto{\pgfqpoint{3.595060in}{1.852067in}}%
\pgfpathlineto{\pgfqpoint{3.581880in}{1.854065in}}%
\pgfpathlineto{\pgfqpoint{3.568706in}{1.856240in}}%
\pgfpathlineto{\pgfqpoint{3.555536in}{1.858593in}}%
\pgfpathlineto{\pgfqpoint{3.542372in}{1.861126in}}%
\pgfpathlineto{\pgfqpoint{3.534510in}{1.851756in}}%
\pgfpathlineto{\pgfqpoint{3.526642in}{1.842423in}}%
\pgfpathlineto{\pgfqpoint{3.518769in}{1.833128in}}%
\pgfpathlineto{\pgfqpoint{3.510889in}{1.823873in}}%
\pgfpathclose%
\pgfusepath{fill}%
\end{pgfscope}%
\begin{pgfscope}%
\pgfpathrectangle{\pgfqpoint{1.254980in}{0.150000in}}{\pgfqpoint{5.490039in}{5.490039in}}%
\pgfusepath{clip}%
\pgfsetbuttcap%
\pgfsetroundjoin%
\definecolor{currentfill}{rgb}{0.185783,0.704891,0.485273}%
\pgfsetfillcolor{currentfill}%
\pgfsetfillopacity{0.700000}%
\pgfsetlinewidth{0.000000pt}%
\definecolor{currentstroke}{rgb}{0.000000,0.000000,0.000000}%
\pgfsetstrokecolor{currentstroke}%
\pgfsetdash{}{0pt}%
\pgfpathmoveto{\pgfqpoint{5.955106in}{3.395062in}}%
\pgfpathlineto{\pgfqpoint{5.969360in}{3.405447in}}%
\pgfpathlineto{\pgfqpoint{5.983634in}{3.415984in}}%
\pgfpathlineto{\pgfqpoint{5.997927in}{3.426675in}}%
\pgfpathlineto{\pgfqpoint{6.012241in}{3.437519in}}%
\pgfpathlineto{\pgfqpoint{6.018983in}{3.438607in}}%
\pgfpathlineto{\pgfqpoint{6.025722in}{3.439795in}}%
\pgfpathlineto{\pgfqpoint{6.032459in}{3.441089in}}%
\pgfpathlineto{\pgfqpoint{6.039193in}{3.442496in}}%
\pgfpathlineto{\pgfqpoint{6.024917in}{3.432285in}}%
\pgfpathlineto{\pgfqpoint{6.010661in}{3.422227in}}%
\pgfpathlineto{\pgfqpoint{5.996424in}{3.412321in}}%
\pgfpathlineto{\pgfqpoint{5.982206in}{3.402567in}}%
\pgfpathlineto{\pgfqpoint{5.975434in}{3.400519in}}%
\pgfpathlineto{\pgfqpoint{5.968660in}{3.398590in}}%
\pgfpathlineto{\pgfqpoint{5.961884in}{3.396773in}}%
\pgfpathlineto{\pgfqpoint{5.955106in}{3.395062in}}%
\pgfpathclose%
\pgfusepath{fill}%
\end{pgfscope}%
\begin{pgfscope}%
\pgfpathrectangle{\pgfqpoint{1.254980in}{0.150000in}}{\pgfqpoint{5.490039in}{5.490039in}}%
\pgfusepath{clip}%
\pgfsetbuttcap%
\pgfsetroundjoin%
\definecolor{currentfill}{rgb}{0.267968,0.223549,0.512008}%
\pgfsetfillcolor{currentfill}%
\pgfsetfillopacity{0.700000}%
\pgfsetlinewidth{0.000000pt}%
\definecolor{currentstroke}{rgb}{0.000000,0.000000,0.000000}%
\pgfsetstrokecolor{currentstroke}%
\pgfsetdash{}{0pt}%
\pgfpathmoveto{\pgfqpoint{4.182946in}{2.171603in}}%
\pgfpathlineto{\pgfqpoint{4.196305in}{2.175663in}}%
\pgfpathlineto{\pgfqpoint{4.209674in}{2.179889in}}%
\pgfpathlineto{\pgfqpoint{4.223054in}{2.184278in}}%
\pgfpathlineto{\pgfqpoint{4.236446in}{2.188832in}}%
\pgfpathlineto{\pgfqpoint{4.244088in}{2.199072in}}%
\pgfpathlineto{\pgfqpoint{4.251725in}{2.209248in}}%
\pgfpathlineto{\pgfqpoint{4.259357in}{2.219359in}}%
\pgfpathlineto{\pgfqpoint{4.266984in}{2.229405in}}%
\pgfpathlineto{\pgfqpoint{4.253598in}{2.224802in}}%
\pgfpathlineto{\pgfqpoint{4.240223in}{2.220364in}}%
\pgfpathlineto{\pgfqpoint{4.226859in}{2.216091in}}%
\pgfpathlineto{\pgfqpoint{4.213506in}{2.211982in}}%
\pgfpathlineto{\pgfqpoint{4.205874in}{2.201973in}}%
\pgfpathlineto{\pgfqpoint{4.198236in}{2.191907in}}%
\pgfpathlineto{\pgfqpoint{4.190594in}{2.181784in}}%
\pgfpathlineto{\pgfqpoint{4.182946in}{2.171603in}}%
\pgfpathclose%
\pgfusepath{fill}%
\end{pgfscope}%
\begin{pgfscope}%
\pgfpathrectangle{\pgfqpoint{1.254980in}{0.150000in}}{\pgfqpoint{5.490039in}{5.490039in}}%
\pgfusepath{clip}%
\pgfsetbuttcap%
\pgfsetroundjoin%
\definecolor{currentfill}{rgb}{0.271305,0.019942,0.347269}%
\pgfsetfillcolor{currentfill}%
\pgfsetfillopacity{0.700000}%
\pgfsetlinewidth{0.000000pt}%
\definecolor{currentstroke}{rgb}{0.000000,0.000000,0.000000}%
\pgfsetstrokecolor{currentstroke}%
\pgfsetdash{}{0pt}%
\pgfpathmoveto{\pgfqpoint{3.289425in}{1.807006in}}%
\pgfpathlineto{\pgfqpoint{3.302594in}{1.801372in}}%
\pgfpathlineto{\pgfqpoint{3.315765in}{1.795925in}}%
\pgfpathlineto{\pgfqpoint{3.328938in}{1.790666in}}%
\pgfpathlineto{\pgfqpoint{3.342114in}{1.785593in}}%
\pgfpathlineto{\pgfqpoint{3.350076in}{1.793748in}}%
\pgfpathlineto{\pgfqpoint{3.358031in}{1.801980in}}%
\pgfpathlineto{\pgfqpoint{3.365979in}{1.810286in}}%
\pgfpathlineto{\pgfqpoint{3.373921in}{1.818664in}}%
\pgfpathlineto{\pgfqpoint{3.360762in}{1.823409in}}%
\pgfpathlineto{\pgfqpoint{3.347606in}{1.828340in}}%
\pgfpathlineto{\pgfqpoint{3.334453in}{1.833459in}}%
\pgfpathlineto{\pgfqpoint{3.321303in}{1.838766in}}%
\pgfpathlineto{\pgfqpoint{3.313344in}{1.830706in}}%
\pgfpathlineto{\pgfqpoint{3.305379in}{1.822724in}}%
\pgfpathlineto{\pgfqpoint{3.297406in}{1.814823in}}%
\pgfpathlineto{\pgfqpoint{3.289425in}{1.807006in}}%
\pgfpathclose%
\pgfusepath{fill}%
\end{pgfscope}%
\begin{pgfscope}%
\pgfpathrectangle{\pgfqpoint{1.254980in}{0.150000in}}{\pgfqpoint{5.490039in}{5.490039in}}%
\pgfusepath{clip}%
\pgfsetbuttcap%
\pgfsetroundjoin%
\definecolor{currentfill}{rgb}{0.280894,0.078907,0.402329}%
\pgfsetfillcolor{currentfill}%
\pgfsetfillopacity{0.700000}%
\pgfsetlinewidth{0.000000pt}%
\definecolor{currentstroke}{rgb}{0.000000,0.000000,0.000000}%
\pgfsetstrokecolor{currentstroke}%
\pgfsetdash{}{0pt}%
\pgfpathmoveto{\pgfqpoint{2.961215in}{1.916501in}}%
\pgfpathlineto{\pgfqpoint{2.974428in}{1.906140in}}%
\pgfpathlineto{\pgfqpoint{2.987638in}{1.895990in}}%
\pgfpathlineto{\pgfqpoint{3.000847in}{1.886048in}}%
\pgfpathlineto{\pgfqpoint{3.014054in}{1.876315in}}%
\pgfpathlineto{\pgfqpoint{3.022192in}{1.881755in}}%
\pgfpathlineto{\pgfqpoint{3.030321in}{1.887334in}}%
\pgfpathlineto{\pgfqpoint{3.038440in}{1.893050in}}%
\pgfpathlineto{\pgfqpoint{3.046549in}{1.898898in}}%
\pgfpathlineto{\pgfqpoint{3.033368in}{1.908243in}}%
\pgfpathlineto{\pgfqpoint{3.020186in}{1.917796in}}%
\pgfpathlineto{\pgfqpoint{3.007002in}{1.927558in}}%
\pgfpathlineto{\pgfqpoint{2.993816in}{1.937531in}}%
\pgfpathlineto{\pgfqpoint{2.985681in}{1.932060in}}%
\pgfpathlineto{\pgfqpoint{2.977536in}{1.926729in}}%
\pgfpathlineto{\pgfqpoint{2.969381in}{1.921542in}}%
\pgfpathlineto{\pgfqpoint{2.961215in}{1.916501in}}%
\pgfpathclose%
\pgfusepath{fill}%
\end{pgfscope}%
\begin{pgfscope}%
\pgfpathrectangle{\pgfqpoint{1.254980in}{0.150000in}}{\pgfqpoint{5.490039in}{5.490039in}}%
\pgfusepath{clip}%
\pgfsetbuttcap%
\pgfsetroundjoin%
\definecolor{currentfill}{rgb}{0.273809,0.031497,0.358853}%
\pgfsetfillcolor{currentfill}%
\pgfsetfillopacity{0.700000}%
\pgfsetlinewidth{0.000000pt}%
\definecolor{currentstroke}{rgb}{0.000000,0.000000,0.000000}%
\pgfsetstrokecolor{currentstroke}%
\pgfsetdash{}{0pt}%
\pgfpathmoveto{\pgfqpoint{3.151976in}{1.831460in}}%
\pgfpathlineto{\pgfqpoint{3.165155in}{1.823929in}}%
\pgfpathlineto{\pgfqpoint{3.178334in}{1.816593in}}%
\pgfpathlineto{\pgfqpoint{3.191513in}{1.809453in}}%
\pgfpathlineto{\pgfqpoint{3.204694in}{1.802507in}}%
\pgfpathlineto{\pgfqpoint{3.212725in}{1.809589in}}%
\pgfpathlineto{\pgfqpoint{3.220748in}{1.816775in}}%
\pgfpathlineto{\pgfqpoint{3.228763in}{1.824061in}}%
\pgfpathlineto{\pgfqpoint{3.236770in}{1.831445in}}%
\pgfpathlineto{\pgfqpoint{3.223610in}{1.838035in}}%
\pgfpathlineto{\pgfqpoint{3.210452in}{1.844818in}}%
\pgfpathlineto{\pgfqpoint{3.197294in}{1.851796in}}%
\pgfpathlineto{\pgfqpoint{3.184137in}{1.858970in}}%
\pgfpathlineto{\pgfqpoint{3.176110in}{1.851933in}}%
\pgfpathlineto{\pgfqpoint{3.168074in}{1.844999in}}%
\pgfpathlineto{\pgfqpoint{3.160029in}{1.838174in}}%
\pgfpathlineto{\pgfqpoint{3.151976in}{1.831460in}}%
\pgfpathclose%
\pgfusepath{fill}%
\end{pgfscope}%
\begin{pgfscope}%
\pgfpathrectangle{\pgfqpoint{1.254980in}{0.150000in}}{\pgfqpoint{5.490039in}{5.490039in}}%
\pgfusepath{clip}%
\pgfsetbuttcap%
\pgfsetroundjoin%
\definecolor{currentfill}{rgb}{0.185556,0.418570,0.556753}%
\pgfsetfillcolor{currentfill}%
\pgfsetfillopacity{0.700000}%
\pgfsetlinewidth{0.000000pt}%
\definecolor{currentstroke}{rgb}{0.000000,0.000000,0.000000}%
\pgfsetstrokecolor{currentstroke}%
\pgfsetdash{}{0pt}%
\pgfpathmoveto{\pgfqpoint{2.320392in}{2.694472in}}%
\pgfpathlineto{\pgfqpoint{2.333971in}{2.671860in}}%
\pgfpathlineto{\pgfqpoint{2.347535in}{2.649561in}}%
\pgfpathlineto{\pgfqpoint{2.361085in}{2.627571in}}%
\pgfpathlineto{\pgfqpoint{2.374622in}{2.605887in}}%
\pgfpathlineto{\pgfqpoint{2.383181in}{2.606397in}}%
\pgfpathlineto{\pgfqpoint{2.391725in}{2.607136in}}%
\pgfpathlineto{\pgfqpoint{2.400252in}{2.608101in}}%
\pgfpathlineto{\pgfqpoint{2.408763in}{2.609288in}}%
\pgfpathlineto{\pgfqpoint{2.395270in}{2.630554in}}%
\pgfpathlineto{\pgfqpoint{2.381764in}{2.652125in}}%
\pgfpathlineto{\pgfqpoint{2.368245in}{2.674005in}}%
\pgfpathlineto{\pgfqpoint{2.354712in}{2.696196in}}%
\pgfpathlineto{\pgfqpoint{2.346157in}{2.695416in}}%
\pgfpathlineto{\pgfqpoint{2.337585in}{2.694867in}}%
\pgfpathlineto{\pgfqpoint{2.328997in}{2.694550in}}%
\pgfpathlineto{\pgfqpoint{2.320392in}{2.694472in}}%
\pgfpathclose%
\pgfusepath{fill}%
\end{pgfscope}%
\begin{pgfscope}%
\pgfpathrectangle{\pgfqpoint{1.254980in}{0.150000in}}{\pgfqpoint{5.490039in}{5.490039in}}%
\pgfusepath{clip}%
\pgfsetbuttcap%
\pgfsetroundjoin%
\definecolor{currentfill}{rgb}{0.204903,0.375746,0.553533}%
\pgfsetfillcolor{currentfill}%
\pgfsetfillopacity{0.700000}%
\pgfsetlinewidth{0.000000pt}%
\definecolor{currentstroke}{rgb}{0.000000,0.000000,0.000000}%
\pgfsetstrokecolor{currentstroke}%
\pgfsetdash{}{0pt}%
\pgfpathmoveto{\pgfqpoint{4.633714in}{2.510131in}}%
\pgfpathlineto{\pgfqpoint{4.647281in}{2.517303in}}%
\pgfpathlineto{\pgfqpoint{4.660862in}{2.524635in}}%
\pgfpathlineto{\pgfqpoint{4.674457in}{2.532129in}}%
\pgfpathlineto{\pgfqpoint{4.688066in}{2.539782in}}%
\pgfpathlineto{\pgfqpoint{4.695542in}{2.548027in}}%
\pgfpathlineto{\pgfqpoint{4.703011in}{2.556188in}}%
\pgfpathlineto{\pgfqpoint{4.710474in}{2.564269in}}%
\pgfpathlineto{\pgfqpoint{4.717931in}{2.572271in}}%
\pgfpathlineto{\pgfqpoint{4.704330in}{2.564743in}}%
\pgfpathlineto{\pgfqpoint{4.690743in}{2.557375in}}%
\pgfpathlineto{\pgfqpoint{4.677170in}{2.550168in}}%
\pgfpathlineto{\pgfqpoint{4.663610in}{2.543121in}}%
\pgfpathlineto{\pgfqpoint{4.656145in}{2.534983in}}%
\pgfpathlineto{\pgfqpoint{4.648674in}{2.526773in}}%
\pgfpathlineto{\pgfqpoint{4.641197in}{2.518490in}}%
\pgfpathlineto{\pgfqpoint{4.633714in}{2.510131in}}%
\pgfpathclose%
\pgfusepath{fill}%
\end{pgfscope}%
\begin{pgfscope}%
\pgfpathrectangle{\pgfqpoint{1.254980in}{0.150000in}}{\pgfqpoint{5.490039in}{5.490039in}}%
\pgfusepath{clip}%
\pgfsetbuttcap%
\pgfsetroundjoin%
\definecolor{currentfill}{rgb}{0.150476,0.504369,0.557430}%
\pgfsetfillcolor{currentfill}%
\pgfsetfillopacity{0.700000}%
\pgfsetlinewidth{0.000000pt}%
\definecolor{currentstroke}{rgb}{0.000000,0.000000,0.000000}%
\pgfsetstrokecolor{currentstroke}%
\pgfsetdash{}{0pt}%
\pgfpathmoveto{\pgfqpoint{5.084514in}{2.844123in}}%
\pgfpathlineto{\pgfqpoint{5.098318in}{2.853283in}}%
\pgfpathlineto{\pgfqpoint{5.112139in}{2.862602in}}%
\pgfpathlineto{\pgfqpoint{5.125976in}{2.872078in}}%
\pgfpathlineto{\pgfqpoint{5.139829in}{2.881712in}}%
\pgfpathlineto{\pgfqpoint{5.147087in}{2.887140in}}%
\pgfpathlineto{\pgfqpoint{5.154337in}{2.892510in}}%
\pgfpathlineto{\pgfqpoint{5.161581in}{2.897826in}}%
\pgfpathlineto{\pgfqpoint{5.168817in}{2.903093in}}%
\pgfpathlineto{\pgfqpoint{5.154979in}{2.893763in}}%
\pgfpathlineto{\pgfqpoint{5.141157in}{2.884590in}}%
\pgfpathlineto{\pgfqpoint{5.127351in}{2.875576in}}%
\pgfpathlineto{\pgfqpoint{5.113562in}{2.866718in}}%
\pgfpathlineto{\pgfqpoint{5.106310in}{2.861137in}}%
\pgfpathlineto{\pgfqpoint{5.099052in}{2.855514in}}%
\pgfpathlineto{\pgfqpoint{5.091786in}{2.849843in}}%
\pgfpathlineto{\pgfqpoint{5.084514in}{2.844123in}}%
\pgfpathclose%
\pgfusepath{fill}%
\end{pgfscope}%
\begin{pgfscope}%
\pgfpathrectangle{\pgfqpoint{1.254980in}{0.150000in}}{\pgfqpoint{5.490039in}{5.490039in}}%
\pgfusepath{clip}%
\pgfsetbuttcap%
\pgfsetroundjoin%
\definecolor{currentfill}{rgb}{0.257322,0.256130,0.526563}%
\pgfsetfillcolor{currentfill}%
\pgfsetfillopacity{0.700000}%
\pgfsetlinewidth{0.000000pt}%
\definecolor{currentstroke}{rgb}{0.000000,0.000000,0.000000}%
\pgfsetstrokecolor{currentstroke}%
\pgfsetdash{}{0pt}%
\pgfpathmoveto{\pgfqpoint{4.266984in}{2.229405in}}%
\pgfpathlineto{\pgfqpoint{4.280381in}{2.234171in}}%
\pgfpathlineto{\pgfqpoint{4.293790in}{2.239101in}}%
\pgfpathlineto{\pgfqpoint{4.307210in}{2.244195in}}%
\pgfpathlineto{\pgfqpoint{4.320642in}{2.249452in}}%
\pgfpathlineto{\pgfqpoint{4.328258in}{2.259464in}}%
\pgfpathlineto{\pgfqpoint{4.335869in}{2.269404in}}%
\pgfpathlineto{\pgfqpoint{4.343475in}{2.279273in}}%
\pgfpathlineto{\pgfqpoint{4.351075in}{2.289071in}}%
\pgfpathlineto{\pgfqpoint{4.337649in}{2.283794in}}%
\pgfpathlineto{\pgfqpoint{4.324234in}{2.278681in}}%
\pgfpathlineto{\pgfqpoint{4.310831in}{2.273731in}}%
\pgfpathlineto{\pgfqpoint{4.297439in}{2.268945in}}%
\pgfpathlineto{\pgfqpoint{4.289834in}{2.259156in}}%
\pgfpathlineto{\pgfqpoint{4.282222in}{2.249303in}}%
\pgfpathlineto{\pgfqpoint{4.274606in}{2.239386in}}%
\pgfpathlineto{\pgfqpoint{4.266984in}{2.229405in}}%
\pgfpathclose%
\pgfusepath{fill}%
\end{pgfscope}%
\begin{pgfscope}%
\pgfpathrectangle{\pgfqpoint{1.254980in}{0.150000in}}{\pgfqpoint{5.490039in}{5.490039in}}%
\pgfusepath{clip}%
\pgfsetbuttcap%
\pgfsetroundjoin%
\definecolor{currentfill}{rgb}{0.272594,0.025563,0.353093}%
\pgfsetfillcolor{currentfill}%
\pgfsetfillopacity{0.700000}%
\pgfsetlinewidth{0.000000pt}%
\definecolor{currentstroke}{rgb}{0.000000,0.000000,0.000000}%
\pgfsetstrokecolor{currentstroke}%
\pgfsetdash{}{0pt}%
\pgfpathmoveto{\pgfqpoint{3.426586in}{1.801530in}}%
\pgfpathlineto{\pgfqpoint{3.439762in}{1.797705in}}%
\pgfpathlineto{\pgfqpoint{3.452940in}{1.794061in}}%
\pgfpathlineto{\pgfqpoint{3.466123in}{1.790598in}}%
\pgfpathlineto{\pgfqpoint{3.479310in}{1.787316in}}%
\pgfpathlineto{\pgfqpoint{3.487214in}{1.796381in}}%
\pgfpathlineto{\pgfqpoint{3.495112in}{1.805497in}}%
\pgfpathlineto{\pgfqpoint{3.503004in}{1.814662in}}%
\pgfpathlineto{\pgfqpoint{3.510889in}{1.823873in}}%
\pgfpathlineto{\pgfqpoint{3.497716in}{1.826857in}}%
\pgfpathlineto{\pgfqpoint{3.484548in}{1.830020in}}%
\pgfpathlineto{\pgfqpoint{3.471383in}{1.833364in}}%
\pgfpathlineto{\pgfqpoint{3.458223in}{1.836890in}}%
\pgfpathlineto{\pgfqpoint{3.450323in}{1.827968in}}%
\pgfpathlineto{\pgfqpoint{3.442417in}{1.819099in}}%
\pgfpathlineto{\pgfqpoint{3.434505in}{1.810286in}}%
\pgfpathlineto{\pgfqpoint{3.426586in}{1.801530in}}%
\pgfpathclose%
\pgfusepath{fill}%
\end{pgfscope}%
\begin{pgfscope}%
\pgfpathrectangle{\pgfqpoint{1.254980in}{0.150000in}}{\pgfqpoint{5.490039in}{5.490039in}}%
\pgfusepath{clip}%
\pgfsetbuttcap%
\pgfsetroundjoin%
\definecolor{currentfill}{rgb}{0.141935,0.526453,0.555991}%
\pgfsetfillcolor{currentfill}%
\pgfsetfillopacity{0.700000}%
\pgfsetlinewidth{0.000000pt}%
\definecolor{currentstroke}{rgb}{0.000000,0.000000,0.000000}%
\pgfsetstrokecolor{currentstroke}%
\pgfsetdash{}{0pt}%
\pgfpathmoveto{\pgfqpoint{5.168817in}{2.903093in}}%
\pgfpathlineto{\pgfqpoint{5.182672in}{2.912580in}}%
\pgfpathlineto{\pgfqpoint{5.196543in}{2.922224in}}%
\pgfpathlineto{\pgfqpoint{5.210431in}{2.932027in}}%
\pgfpathlineto{\pgfqpoint{5.224336in}{2.941987in}}%
\pgfpathlineto{\pgfqpoint{5.231550in}{2.946883in}}%
\pgfpathlineto{\pgfqpoint{5.238756in}{2.951729in}}%
\pgfpathlineto{\pgfqpoint{5.245955in}{2.956530in}}%
\pgfpathlineto{\pgfqpoint{5.253147in}{2.961288in}}%
\pgfpathlineto{\pgfqpoint{5.239259in}{2.951663in}}%
\pgfpathlineto{\pgfqpoint{5.225387in}{2.942195in}}%
\pgfpathlineto{\pgfqpoint{5.211532in}{2.932884in}}%
\pgfpathlineto{\pgfqpoint{5.197694in}{2.923730in}}%
\pgfpathlineto{\pgfqpoint{5.190485in}{2.918627in}}%
\pgfpathlineto{\pgfqpoint{5.183269in}{2.913489in}}%
\pgfpathlineto{\pgfqpoint{5.176046in}{2.908312in}}%
\pgfpathlineto{\pgfqpoint{5.168817in}{2.903093in}}%
\pgfpathclose%
\pgfusepath{fill}%
\end{pgfscope}%
\begin{pgfscope}%
\pgfpathrectangle{\pgfqpoint{1.254980in}{0.150000in}}{\pgfqpoint{5.490039in}{5.490039in}}%
\pgfusepath{clip}%
\pgfsetbuttcap%
\pgfsetroundjoin%
\definecolor{currentfill}{rgb}{0.278791,0.062145,0.386592}%
\pgfsetfillcolor{currentfill}%
\pgfsetfillopacity{0.700000}%
\pgfsetlinewidth{0.000000pt}%
\definecolor{currentstroke}{rgb}{0.000000,0.000000,0.000000}%
\pgfsetstrokecolor{currentstroke}%
\pgfsetdash{}{0pt}%
\pgfpathmoveto{\pgfqpoint{3.014054in}{1.876315in}}%
\pgfpathlineto{\pgfqpoint{3.027260in}{1.866788in}}%
\pgfpathlineto{\pgfqpoint{3.040464in}{1.857467in}}%
\pgfpathlineto{\pgfqpoint{3.053668in}{1.848349in}}%
\pgfpathlineto{\pgfqpoint{3.066871in}{1.839435in}}%
\pgfpathlineto{\pgfqpoint{3.074983in}{1.845273in}}%
\pgfpathlineto{\pgfqpoint{3.083087in}{1.851243in}}%
\pgfpathlineto{\pgfqpoint{3.091181in}{1.857342in}}%
\pgfpathlineto{\pgfqpoint{3.099265in}{1.863566in}}%
\pgfpathlineto{\pgfqpoint{3.086087in}{1.872094in}}%
\pgfpathlineto{\pgfqpoint{3.072909in}{1.880824in}}%
\pgfpathlineto{\pgfqpoint{3.059729in}{1.889758in}}%
\pgfpathlineto{\pgfqpoint{3.046549in}{1.898898in}}%
\pgfpathlineto{\pgfqpoint{3.038440in}{1.893050in}}%
\pgfpathlineto{\pgfqpoint{3.030321in}{1.887334in}}%
\pgfpathlineto{\pgfqpoint{3.022192in}{1.881755in}}%
\pgfpathlineto{\pgfqpoint{3.014054in}{1.876315in}}%
\pgfpathclose%
\pgfusepath{fill}%
\end{pgfscope}%
\begin{pgfscope}%
\pgfpathrectangle{\pgfqpoint{1.254980in}{0.150000in}}{\pgfqpoint{5.490039in}{5.490039in}}%
\pgfusepath{clip}%
\pgfsetbuttcap%
\pgfsetroundjoin%
\definecolor{currentfill}{rgb}{0.192357,0.403199,0.555836}%
\pgfsetfillcolor{currentfill}%
\pgfsetfillopacity{0.700000}%
\pgfsetlinewidth{0.000000pt}%
\definecolor{currentstroke}{rgb}{0.000000,0.000000,0.000000}%
\pgfsetstrokecolor{currentstroke}%
\pgfsetdash{}{0pt}%
\pgfpathmoveto{\pgfqpoint{4.717931in}{2.572271in}}%
\pgfpathlineto{\pgfqpoint{4.731547in}{2.579959in}}%
\pgfpathlineto{\pgfqpoint{4.745176in}{2.587808in}}%
\pgfpathlineto{\pgfqpoint{4.758821in}{2.595816in}}%
\pgfpathlineto{\pgfqpoint{4.772479in}{2.603985in}}%
\pgfpathlineto{\pgfqpoint{4.779921in}{2.611765in}}%
\pgfpathlineto{\pgfqpoint{4.787357in}{2.619463in}}%
\pgfpathlineto{\pgfqpoint{4.794785in}{2.627080in}}%
\pgfpathlineto{\pgfqpoint{4.802208in}{2.634618in}}%
\pgfpathlineto{\pgfqpoint{4.788558in}{2.626605in}}%
\pgfpathlineto{\pgfqpoint{4.774922in}{2.618752in}}%
\pgfpathlineto{\pgfqpoint{4.761301in}{2.611058in}}%
\pgfpathlineto{\pgfqpoint{4.747695in}{2.603525in}}%
\pgfpathlineto{\pgfqpoint{4.740263in}{2.595820in}}%
\pgfpathlineto{\pgfqpoint{4.732826in}{2.588045in}}%
\pgfpathlineto{\pgfqpoint{4.725382in}{2.580196in}}%
\pgfpathlineto{\pgfqpoint{4.717931in}{2.572271in}}%
\pgfpathclose%
\pgfusepath{fill}%
\end{pgfscope}%
\begin{pgfscope}%
\pgfpathrectangle{\pgfqpoint{1.254980in}{0.150000in}}{\pgfqpoint{5.490039in}{5.490039in}}%
\pgfusepath{clip}%
\pgfsetbuttcap%
\pgfsetroundjoin%
\definecolor{currentfill}{rgb}{0.266580,0.228262,0.514349}%
\pgfsetfillcolor{currentfill}%
\pgfsetfillopacity{0.700000}%
\pgfsetlinewidth{0.000000pt}%
\definecolor{currentstroke}{rgb}{0.000000,0.000000,0.000000}%
\pgfsetstrokecolor{currentstroke}%
\pgfsetdash{}{0pt}%
\pgfpathmoveto{\pgfqpoint{2.609425in}{2.223759in}}%
\pgfpathlineto{\pgfqpoint{2.622786in}{2.207450in}}%
\pgfpathlineto{\pgfqpoint{2.636138in}{2.191392in}}%
\pgfpathlineto{\pgfqpoint{2.649483in}{2.175584in}}%
\pgfpathlineto{\pgfqpoint{2.662821in}{2.160022in}}%
\pgfpathlineto{\pgfqpoint{2.671196in}{2.162267in}}%
\pgfpathlineto{\pgfqpoint{2.679558in}{2.164712in}}%
\pgfpathlineto{\pgfqpoint{2.687906in}{2.167353in}}%
\pgfpathlineto{\pgfqpoint{2.696241in}{2.170185in}}%
\pgfpathlineto{\pgfqpoint{2.682940in}{2.185318in}}%
\pgfpathlineto{\pgfqpoint{2.669631in}{2.200697in}}%
\pgfpathlineto{\pgfqpoint{2.656316in}{2.216325in}}%
\pgfpathlineto{\pgfqpoint{2.642994in}{2.232204in}}%
\pgfpathlineto{\pgfqpoint{2.634623in}{2.229790in}}%
\pgfpathlineto{\pgfqpoint{2.626238in}{2.227575in}}%
\pgfpathlineto{\pgfqpoint{2.617839in}{2.225564in}}%
\pgfpathlineto{\pgfqpoint{2.609425in}{2.223759in}}%
\pgfpathclose%
\pgfusepath{fill}%
\end{pgfscope}%
\begin{pgfscope}%
\pgfpathrectangle{\pgfqpoint{1.254980in}{0.150000in}}{\pgfqpoint{5.490039in}{5.490039in}}%
\pgfusepath{clip}%
\pgfsetbuttcap%
\pgfsetroundjoin%
\definecolor{currentfill}{rgb}{0.274128,0.199721,0.498911}%
\pgfsetfillcolor{currentfill}%
\pgfsetfillopacity{0.700000}%
\pgfsetlinewidth{0.000000pt}%
\definecolor{currentstroke}{rgb}{0.000000,0.000000,0.000000}%
\pgfsetstrokecolor{currentstroke}%
\pgfsetdash{}{0pt}%
\pgfpathmoveto{\pgfqpoint{2.662821in}{2.160022in}}%
\pgfpathlineto{\pgfqpoint{2.676152in}{2.144706in}}%
\pgfpathlineto{\pgfqpoint{2.689477in}{2.129632in}}%
\pgfpathlineto{\pgfqpoint{2.702795in}{2.114801in}}%
\pgfpathlineto{\pgfqpoint{2.716107in}{2.100208in}}%
\pgfpathlineto{\pgfqpoint{2.724446in}{2.102892in}}%
\pgfpathlineto{\pgfqpoint{2.732772in}{2.105767in}}%
\pgfpathlineto{\pgfqpoint{2.741085in}{2.108831in}}%
\pgfpathlineto{\pgfqpoint{2.749385in}{2.112079in}}%
\pgfpathlineto{\pgfqpoint{2.736108in}{2.126245in}}%
\pgfpathlineto{\pgfqpoint{2.722825in}{2.140650in}}%
\pgfpathlineto{\pgfqpoint{2.709536in}{2.155296in}}%
\pgfpathlineto{\pgfqpoint{2.696241in}{2.170185in}}%
\pgfpathlineto{\pgfqpoint{2.687906in}{2.167353in}}%
\pgfpathlineto{\pgfqpoint{2.679558in}{2.164712in}}%
\pgfpathlineto{\pgfqpoint{2.671196in}{2.162267in}}%
\pgfpathlineto{\pgfqpoint{2.662821in}{2.160022in}}%
\pgfpathclose%
\pgfusepath{fill}%
\end{pgfscope}%
\begin{pgfscope}%
\pgfpathrectangle{\pgfqpoint{1.254980in}{0.150000in}}{\pgfqpoint{5.490039in}{5.490039in}}%
\pgfusepath{clip}%
\pgfsetbuttcap%
\pgfsetroundjoin%
\definecolor{currentfill}{rgb}{0.246811,0.283237,0.535941}%
\pgfsetfillcolor{currentfill}%
\pgfsetfillopacity{0.700000}%
\pgfsetlinewidth{0.000000pt}%
\definecolor{currentstroke}{rgb}{0.000000,0.000000,0.000000}%
\pgfsetstrokecolor{currentstroke}%
\pgfsetdash{}{0pt}%
\pgfpathmoveto{\pgfqpoint{4.351075in}{2.289071in}}%
\pgfpathlineto{\pgfqpoint{4.364513in}{2.294511in}}%
\pgfpathlineto{\pgfqpoint{4.377964in}{2.300113in}}%
\pgfpathlineto{\pgfqpoint{4.391426in}{2.305879in}}%
\pgfpathlineto{\pgfqpoint{4.404901in}{2.311807in}}%
\pgfpathlineto{\pgfqpoint{4.412490in}{2.321536in}}%
\pgfpathlineto{\pgfqpoint{4.420074in}{2.331187in}}%
\pgfpathlineto{\pgfqpoint{4.427652in}{2.340760in}}%
\pgfpathlineto{\pgfqpoint{4.435225in}{2.350258in}}%
\pgfpathlineto{\pgfqpoint{4.421756in}{2.344339in}}%
\pgfpathlineto{\pgfqpoint{4.408299in}{2.338583in}}%
\pgfpathlineto{\pgfqpoint{4.394855in}{2.332989in}}%
\pgfpathlineto{\pgfqpoint{4.381422in}{2.327558in}}%
\pgfpathlineto{\pgfqpoint{4.373844in}{2.318041in}}%
\pgfpathlineto{\pgfqpoint{4.366260in}{2.308454in}}%
\pgfpathlineto{\pgfqpoint{4.358670in}{2.298798in}}%
\pgfpathlineto{\pgfqpoint{4.351075in}{2.289071in}}%
\pgfpathclose%
\pgfusepath{fill}%
\end{pgfscope}%
\begin{pgfscope}%
\pgfpathrectangle{\pgfqpoint{1.254980in}{0.150000in}}{\pgfqpoint{5.490039in}{5.490039in}}%
\pgfusepath{clip}%
\pgfsetbuttcap%
\pgfsetroundjoin%
\definecolor{currentfill}{rgb}{0.255645,0.260703,0.528312}%
\pgfsetfillcolor{currentfill}%
\pgfsetfillopacity{0.700000}%
\pgfsetlinewidth{0.000000pt}%
\definecolor{currentstroke}{rgb}{0.000000,0.000000,0.000000}%
\pgfsetstrokecolor{currentstroke}%
\pgfsetdash{}{0pt}%
\pgfpathmoveto{\pgfqpoint{2.555904in}{2.291551in}}%
\pgfpathlineto{\pgfqpoint{2.569297in}{2.274216in}}%
\pgfpathlineto{\pgfqpoint{2.582681in}{2.257140in}}%
\pgfpathlineto{\pgfqpoint{2.596058in}{2.240322in}}%
\pgfpathlineto{\pgfqpoint{2.609425in}{2.223759in}}%
\pgfpathlineto{\pgfqpoint{2.617839in}{2.225564in}}%
\pgfpathlineto{\pgfqpoint{2.626238in}{2.227575in}}%
\pgfpathlineto{\pgfqpoint{2.634623in}{2.229790in}}%
\pgfpathlineto{\pgfqpoint{2.642994in}{2.232204in}}%
\pgfpathlineto{\pgfqpoint{2.629664in}{2.248335in}}%
\pgfpathlineto{\pgfqpoint{2.616326in}{2.264721in}}%
\pgfpathlineto{\pgfqpoint{2.602981in}{2.281364in}}%
\pgfpathlineto{\pgfqpoint{2.589627in}{2.298266in}}%
\pgfpathlineto{\pgfqpoint{2.581218in}{2.296273in}}%
\pgfpathlineto{\pgfqpoint{2.572795in}{2.294487in}}%
\pgfpathlineto{\pgfqpoint{2.564357in}{2.292911in}}%
\pgfpathlineto{\pgfqpoint{2.555904in}{2.291551in}}%
\pgfpathclose%
\pgfusepath{fill}%
\end{pgfscope}%
\begin{pgfscope}%
\pgfpathrectangle{\pgfqpoint{1.254980in}{0.150000in}}{\pgfqpoint{5.490039in}{5.490039in}}%
\pgfusepath{clip}%
\pgfsetbuttcap%
\pgfsetroundjoin%
\definecolor{currentfill}{rgb}{0.279574,0.170599,0.479997}%
\pgfsetfillcolor{currentfill}%
\pgfsetfillopacity{0.700000}%
\pgfsetlinewidth{0.000000pt}%
\definecolor{currentstroke}{rgb}{0.000000,0.000000,0.000000}%
\pgfsetstrokecolor{currentstroke}%
\pgfsetdash{}{0pt}%
\pgfpathmoveto{\pgfqpoint{2.716107in}{2.100208in}}%
\pgfpathlineto{\pgfqpoint{2.729413in}{2.085853in}}%
\pgfpathlineto{\pgfqpoint{2.742714in}{2.071734in}}%
\pgfpathlineto{\pgfqpoint{2.756009in}{2.057849in}}%
\pgfpathlineto{\pgfqpoint{2.769298in}{2.044197in}}%
\pgfpathlineto{\pgfqpoint{2.777602in}{2.047316in}}%
\pgfpathlineto{\pgfqpoint{2.785894in}{2.050620in}}%
\pgfpathlineto{\pgfqpoint{2.794173in}{2.054105in}}%
\pgfpathlineto{\pgfqpoint{2.802440in}{2.057766in}}%
\pgfpathlineto{\pgfqpoint{2.789184in}{2.070995in}}%
\pgfpathlineto{\pgfqpoint{2.775923in}{2.084455in}}%
\pgfpathlineto{\pgfqpoint{2.762657in}{2.098149in}}%
\pgfpathlineto{\pgfqpoint{2.749385in}{2.112079in}}%
\pgfpathlineto{\pgfqpoint{2.741085in}{2.108831in}}%
\pgfpathlineto{\pgfqpoint{2.732772in}{2.105767in}}%
\pgfpathlineto{\pgfqpoint{2.724446in}{2.102892in}}%
\pgfpathlineto{\pgfqpoint{2.716107in}{2.100208in}}%
\pgfpathclose%
\pgfusepath{fill}%
\end{pgfscope}%
\begin{pgfscope}%
\pgfpathrectangle{\pgfqpoint{1.254980in}{0.150000in}}{\pgfqpoint{5.490039in}{5.490039in}}%
\pgfusepath{clip}%
\pgfsetbuttcap%
\pgfsetroundjoin%
\definecolor{currentfill}{rgb}{0.282910,0.105393,0.426902}%
\pgfsetfillcolor{currentfill}%
\pgfsetfillopacity{0.700000}%
\pgfsetlinewidth{0.000000pt}%
\definecolor{currentstroke}{rgb}{0.000000,0.000000,0.000000}%
\pgfsetstrokecolor{currentstroke}%
\pgfsetdash{}{0pt}%
\pgfpathmoveto{\pgfqpoint{3.816039in}{1.925075in}}%
\pgfpathlineto{\pgfqpoint{3.829283in}{1.925793in}}%
\pgfpathlineto{\pgfqpoint{3.842535in}{1.926682in}}%
\pgfpathlineto{\pgfqpoint{3.855795in}{1.927740in}}%
\pgfpathlineto{\pgfqpoint{3.869062in}{1.928968in}}%
\pgfpathlineto{\pgfqpoint{3.876827in}{1.939560in}}%
\pgfpathlineto{\pgfqpoint{3.884587in}{1.950133in}}%
\pgfpathlineto{\pgfqpoint{3.892342in}{1.960687in}}%
\pgfpathlineto{\pgfqpoint{3.900092in}{1.971218in}}%
\pgfpathlineto{\pgfqpoint{3.886832in}{1.969802in}}%
\pgfpathlineto{\pgfqpoint{3.873580in}{1.968555in}}%
\pgfpathlineto{\pgfqpoint{3.860336in}{1.967478in}}%
\pgfpathlineto{\pgfqpoint{3.847099in}{1.966571in}}%
\pgfpathlineto{\pgfqpoint{3.839342in}{1.956217in}}%
\pgfpathlineto{\pgfqpoint{3.831579in}{1.945849in}}%
\pgfpathlineto{\pgfqpoint{3.823812in}{1.935468in}}%
\pgfpathlineto{\pgfqpoint{3.816039in}{1.925075in}}%
\pgfpathclose%
\pgfusepath{fill}%
\end{pgfscope}%
\begin{pgfscope}%
\pgfpathrectangle{\pgfqpoint{1.254980in}{0.150000in}}{\pgfqpoint{5.490039in}{5.490039in}}%
\pgfusepath{clip}%
\pgfsetbuttcap%
\pgfsetroundjoin%
\definecolor{currentfill}{rgb}{0.280894,0.078907,0.402329}%
\pgfsetfillcolor{currentfill}%
\pgfsetfillopacity{0.700000}%
\pgfsetlinewidth{0.000000pt}%
\definecolor{currentstroke}{rgb}{0.000000,0.000000,0.000000}%
\pgfsetstrokecolor{currentstroke}%
\pgfsetdash{}{0pt}%
\pgfpathmoveto{\pgfqpoint{3.731962in}{1.883121in}}%
\pgfpathlineto{\pgfqpoint{3.745185in}{1.882939in}}%
\pgfpathlineto{\pgfqpoint{3.758416in}{1.882928in}}%
\pgfpathlineto{\pgfqpoint{3.771654in}{1.883089in}}%
\pgfpathlineto{\pgfqpoint{3.784899in}{1.883420in}}%
\pgfpathlineto{\pgfqpoint{3.792692in}{1.893843in}}%
\pgfpathlineto{\pgfqpoint{3.800479in}{1.904260in}}%
\pgfpathlineto{\pgfqpoint{3.808262in}{1.914672in}}%
\pgfpathlineto{\pgfqpoint{3.816039in}{1.925075in}}%
\pgfpathlineto{\pgfqpoint{3.802803in}{1.924526in}}%
\pgfpathlineto{\pgfqpoint{3.789573in}{1.924149in}}%
\pgfpathlineto{\pgfqpoint{3.776352in}{1.923944in}}%
\pgfpathlineto{\pgfqpoint{3.763137in}{1.923910in}}%
\pgfpathlineto{\pgfqpoint{3.755351in}{1.913713in}}%
\pgfpathlineto{\pgfqpoint{3.747559in}{1.903515in}}%
\pgfpathlineto{\pgfqpoint{3.739763in}{1.893317in}}%
\pgfpathlineto{\pgfqpoint{3.731962in}{1.883121in}}%
\pgfpathclose%
\pgfusepath{fill}%
\end{pgfscope}%
\begin{pgfscope}%
\pgfpathrectangle{\pgfqpoint{1.254980in}{0.150000in}}{\pgfqpoint{5.490039in}{5.490039in}}%
\pgfusepath{clip}%
\pgfsetbuttcap%
\pgfsetroundjoin%
\definecolor{currentfill}{rgb}{0.132444,0.552216,0.553018}%
\pgfsetfillcolor{currentfill}%
\pgfsetfillopacity{0.700000}%
\pgfsetlinewidth{0.000000pt}%
\definecolor{currentstroke}{rgb}{0.000000,0.000000,0.000000}%
\pgfsetstrokecolor{currentstroke}%
\pgfsetdash{}{0pt}%
\pgfpathmoveto{\pgfqpoint{5.253147in}{2.961288in}}%
\pgfpathlineto{\pgfqpoint{5.267052in}{2.971070in}}%
\pgfpathlineto{\pgfqpoint{5.280975in}{2.981009in}}%
\pgfpathlineto{\pgfqpoint{5.294914in}{2.991105in}}%
\pgfpathlineto{\pgfqpoint{5.308871in}{3.001359in}}%
\pgfpathlineto{\pgfqpoint{5.316038in}{3.005725in}}%
\pgfpathlineto{\pgfqpoint{5.323199in}{3.010050in}}%
\pgfpathlineto{\pgfqpoint{5.330352in}{3.014337in}}%
\pgfpathlineto{\pgfqpoint{5.337498in}{3.018592in}}%
\pgfpathlineto{\pgfqpoint{5.323559in}{3.008704in}}%
\pgfpathlineto{\pgfqpoint{5.309638in}{2.998973in}}%
\pgfpathlineto{\pgfqpoint{5.295734in}{2.989398in}}%
\pgfpathlineto{\pgfqpoint{5.281847in}{2.979979in}}%
\pgfpathlineto{\pgfqpoint{5.274682in}{2.975350in}}%
\pgfpathlineto{\pgfqpoint{5.267510in}{2.970694in}}%
\pgfpathlineto{\pgfqpoint{5.260332in}{2.966008in}}%
\pgfpathlineto{\pgfqpoint{5.253147in}{2.961288in}}%
\pgfpathclose%
\pgfusepath{fill}%
\end{pgfscope}%
\begin{pgfscope}%
\pgfpathrectangle{\pgfqpoint{1.254980in}{0.150000in}}{\pgfqpoint{5.490039in}{5.490039in}}%
\pgfusepath{clip}%
\pgfsetbuttcap%
\pgfsetroundjoin%
\definecolor{currentfill}{rgb}{0.283072,0.130895,0.449241}%
\pgfsetfillcolor{currentfill}%
\pgfsetfillopacity{0.700000}%
\pgfsetlinewidth{0.000000pt}%
\definecolor{currentstroke}{rgb}{0.000000,0.000000,0.000000}%
\pgfsetstrokecolor{currentstroke}%
\pgfsetdash{}{0pt}%
\pgfpathmoveto{\pgfqpoint{3.900092in}{1.971218in}}%
\pgfpathlineto{\pgfqpoint{3.913361in}{1.972803in}}%
\pgfpathlineto{\pgfqpoint{3.926638in}{1.974557in}}%
\pgfpathlineto{\pgfqpoint{3.939923in}{1.976479in}}%
\pgfpathlineto{\pgfqpoint{3.953218in}{1.978568in}}%
\pgfpathlineto{\pgfqpoint{3.960956in}{1.989248in}}%
\pgfpathlineto{\pgfqpoint{3.968689in}{1.999897in}}%
\pgfpathlineto{\pgfqpoint{3.976418in}{2.010512in}}%
\pgfpathlineto{\pgfqpoint{3.984142in}{2.021094in}}%
\pgfpathlineto{\pgfqpoint{3.970854in}{2.018844in}}%
\pgfpathlineto{\pgfqpoint{3.957575in}{2.016761in}}%
\pgfpathlineto{\pgfqpoint{3.944305in}{2.014847in}}%
\pgfpathlineto{\pgfqpoint{3.931043in}{2.013101in}}%
\pgfpathlineto{\pgfqpoint{3.923313in}{2.002669in}}%
\pgfpathlineto{\pgfqpoint{3.915577in}{1.992211in}}%
\pgfpathlineto{\pgfqpoint{3.907837in}{1.981727in}}%
\pgfpathlineto{\pgfqpoint{3.900092in}{1.971218in}}%
\pgfpathclose%
\pgfusepath{fill}%
\end{pgfscope}%
\begin{pgfscope}%
\pgfpathrectangle{\pgfqpoint{1.254980in}{0.150000in}}{\pgfqpoint{5.490039in}{5.490039in}}%
\pgfusepath{clip}%
\pgfsetbuttcap%
\pgfsetroundjoin%
\definecolor{currentfill}{rgb}{0.271305,0.019942,0.347269}%
\pgfsetfillcolor{currentfill}%
\pgfsetfillopacity{0.700000}%
\pgfsetlinewidth{0.000000pt}%
\definecolor{currentstroke}{rgb}{0.000000,0.000000,0.000000}%
\pgfsetstrokecolor{currentstroke}%
\pgfsetdash{}{0pt}%
\pgfpathmoveto{\pgfqpoint{3.204694in}{1.802507in}}%
\pgfpathlineto{\pgfqpoint{3.217876in}{1.795754in}}%
\pgfpathlineto{\pgfqpoint{3.231059in}{1.789193in}}%
\pgfpathlineto{\pgfqpoint{3.244243in}{1.782823in}}%
\pgfpathlineto{\pgfqpoint{3.257429in}{1.776643in}}%
\pgfpathlineto{\pgfqpoint{3.265440in}{1.784092in}}%
\pgfpathlineto{\pgfqpoint{3.273443in}{1.791637in}}%
\pgfpathlineto{\pgfqpoint{3.281438in}{1.799277in}}%
\pgfpathlineto{\pgfqpoint{3.289425in}{1.807006in}}%
\pgfpathlineto{\pgfqpoint{3.276259in}{1.812830in}}%
\pgfpathlineto{\pgfqpoint{3.263094in}{1.818844in}}%
\pgfpathlineto{\pgfqpoint{3.249932in}{1.825049in}}%
\pgfpathlineto{\pgfqpoint{3.236770in}{1.831445in}}%
\pgfpathlineto{\pgfqpoint{3.228763in}{1.824061in}}%
\pgfpathlineto{\pgfqpoint{3.220748in}{1.816775in}}%
\pgfpathlineto{\pgfqpoint{3.212725in}{1.809589in}}%
\pgfpathlineto{\pgfqpoint{3.204694in}{1.802507in}}%
\pgfpathclose%
\pgfusepath{fill}%
\end{pgfscope}%
\begin{pgfscope}%
\pgfpathrectangle{\pgfqpoint{1.254980in}{0.150000in}}{\pgfqpoint{5.490039in}{5.490039in}}%
\pgfusepath{clip}%
\pgfsetbuttcap%
\pgfsetroundjoin%
\definecolor{currentfill}{rgb}{0.243113,0.292092,0.538516}%
\pgfsetfillcolor{currentfill}%
\pgfsetfillopacity{0.700000}%
\pgfsetlinewidth{0.000000pt}%
\definecolor{currentstroke}{rgb}{0.000000,0.000000,0.000000}%
\pgfsetstrokecolor{currentstroke}%
\pgfsetdash{}{0pt}%
\pgfpathmoveto{\pgfqpoint{2.502240in}{2.363538in}}%
\pgfpathlineto{\pgfqpoint{2.515670in}{2.345140in}}%
\pgfpathlineto{\pgfqpoint{2.529091in}{2.327011in}}%
\pgfpathlineto{\pgfqpoint{2.542502in}{2.309149in}}%
\pgfpathlineto{\pgfqpoint{2.555904in}{2.291551in}}%
\pgfpathlineto{\pgfqpoint{2.564357in}{2.292911in}}%
\pgfpathlineto{\pgfqpoint{2.572795in}{2.294487in}}%
\pgfpathlineto{\pgfqpoint{2.581218in}{2.296273in}}%
\pgfpathlineto{\pgfqpoint{2.589627in}{2.298266in}}%
\pgfpathlineto{\pgfqpoint{2.576264in}{2.315429in}}%
\pgfpathlineto{\pgfqpoint{2.562893in}{2.332856in}}%
\pgfpathlineto{\pgfqpoint{2.549513in}{2.350548in}}%
\pgfpathlineto{\pgfqpoint{2.536124in}{2.368509in}}%
\pgfpathlineto{\pgfqpoint{2.527676in}{2.366941in}}%
\pgfpathlineto{\pgfqpoint{2.519212in}{2.365586in}}%
\pgfpathlineto{\pgfqpoint{2.510734in}{2.364451in}}%
\pgfpathlineto{\pgfqpoint{2.502240in}{2.363538in}}%
\pgfpathclose%
\pgfusepath{fill}%
\end{pgfscope}%
\begin{pgfscope}%
\pgfpathrectangle{\pgfqpoint{1.254980in}{0.150000in}}{\pgfqpoint{5.490039in}{5.490039in}}%
\pgfusepath{clip}%
\pgfsetbuttcap%
\pgfsetroundjoin%
\definecolor{currentfill}{rgb}{0.278791,0.062145,0.386592}%
\pgfsetfillcolor{currentfill}%
\pgfsetfillopacity{0.700000}%
\pgfsetlinewidth{0.000000pt}%
\definecolor{currentstroke}{rgb}{0.000000,0.000000,0.000000}%
\pgfsetstrokecolor{currentstroke}%
\pgfsetdash{}{0pt}%
\pgfpathmoveto{\pgfqpoint{3.647834in}{1.845839in}}%
\pgfpathlineto{\pgfqpoint{3.661042in}{1.844720in}}%
\pgfpathlineto{\pgfqpoint{3.674256in}{1.843774in}}%
\pgfpathlineto{\pgfqpoint{3.687477in}{1.843002in}}%
\pgfpathlineto{\pgfqpoint{3.700704in}{1.842403in}}%
\pgfpathlineto{\pgfqpoint{3.708526in}{1.852569in}}%
\pgfpathlineto{\pgfqpoint{3.716343in}{1.862745in}}%
\pgfpathlineto{\pgfqpoint{3.724155in}{1.872930in}}%
\pgfpathlineto{\pgfqpoint{3.731962in}{1.883121in}}%
\pgfpathlineto{\pgfqpoint{3.718745in}{1.883477in}}%
\pgfpathlineto{\pgfqpoint{3.705534in}{1.884005in}}%
\pgfpathlineto{\pgfqpoint{3.692330in}{1.884706in}}%
\pgfpathlineto{\pgfqpoint{3.679132in}{1.885581in}}%
\pgfpathlineto{\pgfqpoint{3.671316in}{1.875624in}}%
\pgfpathlineto{\pgfqpoint{3.663494in}{1.865679in}}%
\pgfpathlineto{\pgfqpoint{3.655667in}{1.855750in}}%
\pgfpathlineto{\pgfqpoint{3.647834in}{1.845839in}}%
\pgfpathclose%
\pgfusepath{fill}%
\end{pgfscope}%
\begin{pgfscope}%
\pgfpathrectangle{\pgfqpoint{1.254980in}{0.150000in}}{\pgfqpoint{5.490039in}{5.490039in}}%
\pgfusepath{clip}%
\pgfsetbuttcap%
\pgfsetroundjoin%
\definecolor{currentfill}{rgb}{0.282290,0.145912,0.461510}%
\pgfsetfillcolor{currentfill}%
\pgfsetfillopacity{0.700000}%
\pgfsetlinewidth{0.000000pt}%
\definecolor{currentstroke}{rgb}{0.000000,0.000000,0.000000}%
\pgfsetstrokecolor{currentstroke}%
\pgfsetdash{}{0pt}%
\pgfpathmoveto{\pgfqpoint{2.769298in}{2.044197in}}%
\pgfpathlineto{\pgfqpoint{2.782583in}{2.030775in}}%
\pgfpathlineto{\pgfqpoint{2.795863in}{2.017581in}}%
\pgfpathlineto{\pgfqpoint{2.809138in}{2.004615in}}%
\pgfpathlineto{\pgfqpoint{2.822409in}{1.991875in}}%
\pgfpathlineto{\pgfqpoint{2.830680in}{1.995427in}}%
\pgfpathlineto{\pgfqpoint{2.838939in}{1.999157in}}%
\pgfpathlineto{\pgfqpoint{2.847186in}{2.003061in}}%
\pgfpathlineto{\pgfqpoint{2.855421in}{2.007133in}}%
\pgfpathlineto{\pgfqpoint{2.842182in}{2.019452in}}%
\pgfpathlineto{\pgfqpoint{2.828939in}{2.031996in}}%
\pgfpathlineto{\pgfqpoint{2.815692in}{2.044767in}}%
\pgfpathlineto{\pgfqpoint{2.802440in}{2.057766in}}%
\pgfpathlineto{\pgfqpoint{2.794173in}{2.054105in}}%
\pgfpathlineto{\pgfqpoint{2.785894in}{2.050620in}}%
\pgfpathlineto{\pgfqpoint{2.777602in}{2.047316in}}%
\pgfpathlineto{\pgfqpoint{2.769298in}{2.044197in}}%
\pgfpathclose%
\pgfusepath{fill}%
\end{pgfscope}%
\begin{pgfscope}%
\pgfpathrectangle{\pgfqpoint{1.254980in}{0.150000in}}{\pgfqpoint{5.490039in}{5.490039in}}%
\pgfusepath{clip}%
\pgfsetbuttcap%
\pgfsetroundjoin%
\definecolor{currentfill}{rgb}{0.271305,0.019942,0.347269}%
\pgfsetfillcolor{currentfill}%
\pgfsetfillopacity{0.700000}%
\pgfsetlinewidth{0.000000pt}%
\definecolor{currentstroke}{rgb}{0.000000,0.000000,0.000000}%
\pgfsetstrokecolor{currentstroke}%
\pgfsetdash{}{0pt}%
\pgfpathmoveto{\pgfqpoint{3.342114in}{1.785593in}}%
\pgfpathlineto{\pgfqpoint{3.355292in}{1.780706in}}%
\pgfpathlineto{\pgfqpoint{3.368474in}{1.776003in}}%
\pgfpathlineto{\pgfqpoint{3.381658in}{1.771485in}}%
\pgfpathlineto{\pgfqpoint{3.394846in}{1.767150in}}%
\pgfpathlineto{\pgfqpoint{3.402791in}{1.775643in}}%
\pgfpathlineto{\pgfqpoint{3.410729in}{1.784206in}}%
\pgfpathlineto{\pgfqpoint{3.418661in}{1.792836in}}%
\pgfpathlineto{\pgfqpoint{3.426586in}{1.801530in}}%
\pgfpathlineto{\pgfqpoint{3.413415in}{1.805538in}}%
\pgfpathlineto{\pgfqpoint{3.400247in}{1.809729in}}%
\pgfpathlineto{\pgfqpoint{3.387082in}{1.814104in}}%
\pgfpathlineto{\pgfqpoint{3.373921in}{1.818664in}}%
\pgfpathlineto{\pgfqpoint{3.365979in}{1.810286in}}%
\pgfpathlineto{\pgfqpoint{3.358031in}{1.801980in}}%
\pgfpathlineto{\pgfqpoint{3.350076in}{1.793748in}}%
\pgfpathlineto{\pgfqpoint{3.342114in}{1.785593in}}%
\pgfpathclose%
\pgfusepath{fill}%
\end{pgfscope}%
\begin{pgfscope}%
\pgfpathrectangle{\pgfqpoint{1.254980in}{0.150000in}}{\pgfqpoint{5.490039in}{5.490039in}}%
\pgfusepath{clip}%
\pgfsetbuttcap%
\pgfsetroundjoin%
\definecolor{currentfill}{rgb}{0.180629,0.429975,0.557282}%
\pgfsetfillcolor{currentfill}%
\pgfsetfillopacity{0.700000}%
\pgfsetlinewidth{0.000000pt}%
\definecolor{currentstroke}{rgb}{0.000000,0.000000,0.000000}%
\pgfsetstrokecolor{currentstroke}%
\pgfsetdash{}{0pt}%
\pgfpathmoveto{\pgfqpoint{4.802208in}{2.634618in}}%
\pgfpathlineto{\pgfqpoint{4.815872in}{2.642791in}}%
\pgfpathlineto{\pgfqpoint{4.829552in}{2.651124in}}%
\pgfpathlineto{\pgfqpoint{4.843247in}{2.659616in}}%
\pgfpathlineto{\pgfqpoint{4.856956in}{2.668268in}}%
\pgfpathlineto{\pgfqpoint{4.864362in}{2.675556in}}%
\pgfpathlineto{\pgfqpoint{4.871762in}{2.682762in}}%
\pgfpathlineto{\pgfqpoint{4.879154in}{2.689890in}}%
\pgfpathlineto{\pgfqpoint{4.886540in}{2.696940in}}%
\pgfpathlineto{\pgfqpoint{4.872840in}{2.688474in}}%
\pgfpathlineto{\pgfqpoint{4.859155in}{2.680167in}}%
\pgfpathlineto{\pgfqpoint{4.845486in}{2.672019in}}%
\pgfpathlineto{\pgfqpoint{4.831831in}{2.664031in}}%
\pgfpathlineto{\pgfqpoint{4.824435in}{2.656785in}}%
\pgfpathlineto{\pgfqpoint{4.817032in}{2.649469in}}%
\pgfpathlineto{\pgfqpoint{4.809623in}{2.642081in}}%
\pgfpathlineto{\pgfqpoint{4.802208in}{2.634618in}}%
\pgfpathclose%
\pgfusepath{fill}%
\end{pgfscope}%
\begin{pgfscope}%
\pgfpathrectangle{\pgfqpoint{1.254980in}{0.150000in}}{\pgfqpoint{5.490039in}{5.490039in}}%
\pgfusepath{clip}%
\pgfsetbuttcap%
\pgfsetroundjoin%
\definecolor{currentfill}{rgb}{0.281412,0.155834,0.469201}%
\pgfsetfillcolor{currentfill}%
\pgfsetfillopacity{0.700000}%
\pgfsetlinewidth{0.000000pt}%
\definecolor{currentstroke}{rgb}{0.000000,0.000000,0.000000}%
\pgfsetstrokecolor{currentstroke}%
\pgfsetdash{}{0pt}%
\pgfpathmoveto{\pgfqpoint{3.984142in}{2.021094in}}%
\pgfpathlineto{\pgfqpoint{3.997439in}{2.023512in}}%
\pgfpathlineto{\pgfqpoint{4.010745in}{2.026097in}}%
\pgfpathlineto{\pgfqpoint{4.024060in}{2.028849in}}%
\pgfpathlineto{\pgfqpoint{4.037385in}{2.031767in}}%
\pgfpathlineto{\pgfqpoint{4.045098in}{2.042457in}}%
\pgfpathlineto{\pgfqpoint{4.052806in}{2.053104in}}%
\pgfpathlineto{\pgfqpoint{4.060509in}{2.063707in}}%
\pgfpathlineto{\pgfqpoint{4.068207in}{2.074264in}}%
\pgfpathlineto{\pgfqpoint{4.054888in}{2.071213in}}%
\pgfpathlineto{\pgfqpoint{4.041579in}{2.068328in}}%
\pgfpathlineto{\pgfqpoint{4.028279in}{2.065610in}}%
\pgfpathlineto{\pgfqpoint{4.014988in}{2.063060in}}%
\pgfpathlineto{\pgfqpoint{4.007284in}{2.052624in}}%
\pgfpathlineto{\pgfqpoint{3.999575in}{2.042151in}}%
\pgfpathlineto{\pgfqpoint{3.991861in}{2.031641in}}%
\pgfpathlineto{\pgfqpoint{3.984142in}{2.021094in}}%
\pgfpathclose%
\pgfusepath{fill}%
\end{pgfscope}%
\begin{pgfscope}%
\pgfpathrectangle{\pgfqpoint{1.254980in}{0.150000in}}{\pgfqpoint{5.490039in}{5.490039in}}%
\pgfusepath{clip}%
\pgfsetbuttcap%
\pgfsetroundjoin%
\definecolor{currentfill}{rgb}{0.233603,0.313828,0.543914}%
\pgfsetfillcolor{currentfill}%
\pgfsetfillopacity{0.700000}%
\pgfsetlinewidth{0.000000pt}%
\definecolor{currentstroke}{rgb}{0.000000,0.000000,0.000000}%
\pgfsetstrokecolor{currentstroke}%
\pgfsetdash{}{0pt}%
\pgfpathmoveto{\pgfqpoint{4.435225in}{2.350258in}}%
\pgfpathlineto{\pgfqpoint{4.448707in}{2.356339in}}%
\pgfpathlineto{\pgfqpoint{4.462201in}{2.362582in}}%
\pgfpathlineto{\pgfqpoint{4.475708in}{2.368988in}}%
\pgfpathlineto{\pgfqpoint{4.489228in}{2.375555in}}%
\pgfpathlineto{\pgfqpoint{4.496789in}{2.384949in}}%
\pgfpathlineto{\pgfqpoint{4.504345in}{2.394261in}}%
\pgfpathlineto{\pgfqpoint{4.511895in}{2.403491in}}%
\pgfpathlineto{\pgfqpoint{4.519439in}{2.412640in}}%
\pgfpathlineto{\pgfqpoint{4.505925in}{2.406111in}}%
\pgfpathlineto{\pgfqpoint{4.492424in}{2.399744in}}%
\pgfpathlineto{\pgfqpoint{4.478936in}{2.393538in}}%
\pgfpathlineto{\pgfqpoint{4.465460in}{2.387495in}}%
\pgfpathlineto{\pgfqpoint{4.457910in}{2.378297in}}%
\pgfpathlineto{\pgfqpoint{4.450354in}{2.369025in}}%
\pgfpathlineto{\pgfqpoint{4.442792in}{2.359679in}}%
\pgfpathlineto{\pgfqpoint{4.435225in}{2.350258in}}%
\pgfpathclose%
\pgfusepath{fill}%
\end{pgfscope}%
\begin{pgfscope}%
\pgfpathrectangle{\pgfqpoint{1.254980in}{0.150000in}}{\pgfqpoint{5.490039in}{5.490039in}}%
\pgfusepath{clip}%
\pgfsetbuttcap%
\pgfsetroundjoin%
\definecolor{currentfill}{rgb}{0.125394,0.574318,0.549086}%
\pgfsetfillcolor{currentfill}%
\pgfsetfillopacity{0.700000}%
\pgfsetlinewidth{0.000000pt}%
\definecolor{currentstroke}{rgb}{0.000000,0.000000,0.000000}%
\pgfsetstrokecolor{currentstroke}%
\pgfsetdash{}{0pt}%
\pgfpathmoveto{\pgfqpoint{5.337498in}{3.018592in}}%
\pgfpathlineto{\pgfqpoint{5.351453in}{3.028637in}}%
\pgfpathlineto{\pgfqpoint{5.365427in}{3.038839in}}%
\pgfpathlineto{\pgfqpoint{5.379417in}{3.049197in}}%
\pgfpathlineto{\pgfqpoint{5.393426in}{3.059712in}}%
\pgfpathlineto{\pgfqpoint{5.400545in}{3.063554in}}%
\pgfpathlineto{\pgfqpoint{5.407658in}{3.067364in}}%
\pgfpathlineto{\pgfqpoint{5.414763in}{3.071147in}}%
\pgfpathlineto{\pgfqpoint{5.421862in}{3.074908in}}%
\pgfpathlineto{\pgfqpoint{5.407874in}{3.064789in}}%
\pgfpathlineto{\pgfqpoint{5.393903in}{3.054826in}}%
\pgfpathlineto{\pgfqpoint{5.379950in}{3.045019in}}%
\pgfpathlineto{\pgfqpoint{5.366015in}{3.035368in}}%
\pgfpathlineto{\pgfqpoint{5.358895in}{3.031202in}}%
\pgfpathlineto{\pgfqpoint{5.351769in}{3.027020in}}%
\pgfpathlineto{\pgfqpoint{5.344637in}{3.022818in}}%
\pgfpathlineto{\pgfqpoint{5.337498in}{3.018592in}}%
\pgfpathclose%
\pgfusepath{fill}%
\end{pgfscope}%
\begin{pgfscope}%
\pgfpathrectangle{\pgfqpoint{1.254980in}{0.150000in}}{\pgfqpoint{5.490039in}{5.490039in}}%
\pgfusepath{clip}%
\pgfsetbuttcap%
\pgfsetroundjoin%
\definecolor{currentfill}{rgb}{0.227802,0.326594,0.546532}%
\pgfsetfillcolor{currentfill}%
\pgfsetfillopacity{0.700000}%
\pgfsetlinewidth{0.000000pt}%
\definecolor{currentstroke}{rgb}{0.000000,0.000000,0.000000}%
\pgfsetstrokecolor{currentstroke}%
\pgfsetdash{}{0pt}%
\pgfpathmoveto{\pgfqpoint{2.448416in}{2.439869in}}%
\pgfpathlineto{\pgfqpoint{2.461888in}{2.420370in}}%
\pgfpathlineto{\pgfqpoint{2.475349in}{2.401150in}}%
\pgfpathlineto{\pgfqpoint{2.488800in}{2.382207in}}%
\pgfpathlineto{\pgfqpoint{2.502240in}{2.363538in}}%
\pgfpathlineto{\pgfqpoint{2.510734in}{2.364451in}}%
\pgfpathlineto{\pgfqpoint{2.519212in}{2.365586in}}%
\pgfpathlineto{\pgfqpoint{2.527676in}{2.366941in}}%
\pgfpathlineto{\pgfqpoint{2.536124in}{2.368509in}}%
\pgfpathlineto{\pgfqpoint{2.522725in}{2.386741in}}%
\pgfpathlineto{\pgfqpoint{2.509316in}{2.405246in}}%
\pgfpathlineto{\pgfqpoint{2.495897in}{2.424026in}}%
\pgfpathlineto{\pgfqpoint{2.482468in}{2.443085in}}%
\pgfpathlineto{\pgfqpoint{2.473978in}{2.441944in}}%
\pgfpathlineto{\pgfqpoint{2.465474in}{2.441024in}}%
\pgfpathlineto{\pgfqpoint{2.456953in}{2.440332in}}%
\pgfpathlineto{\pgfqpoint{2.448416in}{2.439869in}}%
\pgfpathclose%
\pgfusepath{fill}%
\end{pgfscope}%
\begin{pgfscope}%
\pgfpathrectangle{\pgfqpoint{1.254980in}{0.150000in}}{\pgfqpoint{5.490039in}{5.490039in}}%
\pgfusepath{clip}%
\pgfsetbuttcap%
\pgfsetroundjoin%
\definecolor{currentfill}{rgb}{0.276022,0.044167,0.370164}%
\pgfsetfillcolor{currentfill}%
\pgfsetfillopacity{0.700000}%
\pgfsetlinewidth{0.000000pt}%
\definecolor{currentstroke}{rgb}{0.000000,0.000000,0.000000}%
\pgfsetstrokecolor{currentstroke}%
\pgfsetdash{}{0pt}%
\pgfpathmoveto{\pgfqpoint{3.563628in}{1.813729in}}%
\pgfpathlineto{\pgfqpoint{3.576825in}{1.811636in}}%
\pgfpathlineto{\pgfqpoint{3.590027in}{1.809720in}}%
\pgfpathlineto{\pgfqpoint{3.603235in}{1.807979in}}%
\pgfpathlineto{\pgfqpoint{3.616448in}{1.806414in}}%
\pgfpathlineto{\pgfqpoint{3.624303in}{1.816232in}}%
\pgfpathlineto{\pgfqpoint{3.632152in}{1.826077in}}%
\pgfpathlineto{\pgfqpoint{3.639996in}{1.835947in}}%
\pgfpathlineto{\pgfqpoint{3.647834in}{1.845839in}}%
\pgfpathlineto{\pgfqpoint{3.634632in}{1.847133in}}%
\pgfpathlineto{\pgfqpoint{3.621436in}{1.848602in}}%
\pgfpathlineto{\pgfqpoint{3.608245in}{1.850246in}}%
\pgfpathlineto{\pgfqpoint{3.595060in}{1.852067in}}%
\pgfpathlineto{\pgfqpoint{3.587210in}{1.842437in}}%
\pgfpathlineto{\pgfqpoint{3.579355in}{1.832835in}}%
\pgfpathlineto{\pgfqpoint{3.571494in}{1.823265in}}%
\pgfpathlineto{\pgfqpoint{3.563628in}{1.813729in}}%
\pgfpathclose%
\pgfusepath{fill}%
\end{pgfscope}%
\begin{pgfscope}%
\pgfpathrectangle{\pgfqpoint{1.254980in}{0.150000in}}{\pgfqpoint{5.490039in}{5.490039in}}%
\pgfusepath{clip}%
\pgfsetbuttcap%
\pgfsetroundjoin%
\definecolor{currentfill}{rgb}{0.276022,0.044167,0.370164}%
\pgfsetfillcolor{currentfill}%
\pgfsetfillopacity{0.700000}%
\pgfsetlinewidth{0.000000pt}%
\definecolor{currentstroke}{rgb}{0.000000,0.000000,0.000000}%
\pgfsetstrokecolor{currentstroke}%
\pgfsetdash{}{0pt}%
\pgfpathmoveto{\pgfqpoint{3.066871in}{1.839435in}}%
\pgfpathlineto{\pgfqpoint{3.080073in}{1.830722in}}%
\pgfpathlineto{\pgfqpoint{3.093275in}{1.822210in}}%
\pgfpathlineto{\pgfqpoint{3.106476in}{1.813898in}}%
\pgfpathlineto{\pgfqpoint{3.119678in}{1.805784in}}%
\pgfpathlineto{\pgfqpoint{3.127766in}{1.812018in}}%
\pgfpathlineto{\pgfqpoint{3.135845in}{1.818378in}}%
\pgfpathlineto{\pgfqpoint{3.143915in}{1.824860in}}%
\pgfpathlineto{\pgfqpoint{3.151976in}{1.831460in}}%
\pgfpathlineto{\pgfqpoint{3.138798in}{1.839188in}}%
\pgfpathlineto{\pgfqpoint{3.125621in}{1.847114in}}%
\pgfpathlineto{\pgfqpoint{3.112443in}{1.855240in}}%
\pgfpathlineto{\pgfqpoint{3.099265in}{1.863566in}}%
\pgfpathlineto{\pgfqpoint{3.091181in}{1.857342in}}%
\pgfpathlineto{\pgfqpoint{3.083087in}{1.851243in}}%
\pgfpathlineto{\pgfqpoint{3.074983in}{1.845273in}}%
\pgfpathlineto{\pgfqpoint{3.066871in}{1.839435in}}%
\pgfpathclose%
\pgfusepath{fill}%
\end{pgfscope}%
\begin{pgfscope}%
\pgfpathrectangle{\pgfqpoint{1.254980in}{0.150000in}}{\pgfqpoint{5.490039in}{5.490039in}}%
\pgfusepath{clip}%
\pgfsetbuttcap%
\pgfsetroundjoin%
\definecolor{currentfill}{rgb}{0.283229,0.120777,0.440584}%
\pgfsetfillcolor{currentfill}%
\pgfsetfillopacity{0.700000}%
\pgfsetlinewidth{0.000000pt}%
\definecolor{currentstroke}{rgb}{0.000000,0.000000,0.000000}%
\pgfsetstrokecolor{currentstroke}%
\pgfsetdash{}{0pt}%
\pgfpathmoveto{\pgfqpoint{2.822409in}{1.991875in}}%
\pgfpathlineto{\pgfqpoint{2.835676in}{1.979358in}}%
\pgfpathlineto{\pgfqpoint{2.848939in}{1.967064in}}%
\pgfpathlineto{\pgfqpoint{2.862199in}{1.954991in}}%
\pgfpathlineto{\pgfqpoint{2.875454in}{1.943137in}}%
\pgfpathlineto{\pgfqpoint{2.883693in}{1.947121in}}%
\pgfpathlineto{\pgfqpoint{2.891921in}{1.951275in}}%
\pgfpathlineto{\pgfqpoint{2.900137in}{1.955595in}}%
\pgfpathlineto{\pgfqpoint{2.908342in}{1.960077in}}%
\pgfpathlineto{\pgfqpoint{2.895117in}{1.971512in}}%
\pgfpathlineto{\pgfqpoint{2.881888in}{1.983165in}}%
\pgfpathlineto{\pgfqpoint{2.868657in}{1.995038in}}%
\pgfpathlineto{\pgfqpoint{2.855421in}{2.007133in}}%
\pgfpathlineto{\pgfqpoint{2.847186in}{2.003061in}}%
\pgfpathlineto{\pgfqpoint{2.838939in}{1.999157in}}%
\pgfpathlineto{\pgfqpoint{2.830680in}{1.995427in}}%
\pgfpathlineto{\pgfqpoint{2.822409in}{1.991875in}}%
\pgfpathclose%
\pgfusepath{fill}%
\end{pgfscope}%
\begin{pgfscope}%
\pgfpathrectangle{\pgfqpoint{1.254980in}{0.150000in}}{\pgfqpoint{5.490039in}{5.490039in}}%
\pgfusepath{clip}%
\pgfsetbuttcap%
\pgfsetroundjoin%
\definecolor{currentfill}{rgb}{0.277134,0.185228,0.489898}%
\pgfsetfillcolor{currentfill}%
\pgfsetfillopacity{0.700000}%
\pgfsetlinewidth{0.000000pt}%
\definecolor{currentstroke}{rgb}{0.000000,0.000000,0.000000}%
\pgfsetstrokecolor{currentstroke}%
\pgfsetdash{}{0pt}%
\pgfpathmoveto{\pgfqpoint{4.068207in}{2.074264in}}%
\pgfpathlineto{\pgfqpoint{4.081536in}{2.077481in}}%
\pgfpathlineto{\pgfqpoint{4.094875in}{2.080865in}}%
\pgfpathlineto{\pgfqpoint{4.108224in}{2.084413in}}%
\pgfpathlineto{\pgfqpoint{4.121583in}{2.088127in}}%
\pgfpathlineto{\pgfqpoint{4.129270in}{2.098754in}}%
\pgfpathlineto{\pgfqpoint{4.136953in}{2.109328in}}%
\pgfpathlineto{\pgfqpoint{4.144631in}{2.119846in}}%
\pgfpathlineto{\pgfqpoint{4.152304in}{2.130310in}}%
\pgfpathlineto{\pgfqpoint{4.138951in}{2.126491in}}%
\pgfpathlineto{\pgfqpoint{4.125607in}{2.122837in}}%
\pgfpathlineto{\pgfqpoint{4.112274in}{2.119349in}}%
\pgfpathlineto{\pgfqpoint{4.098951in}{2.116027in}}%
\pgfpathlineto{\pgfqpoint{4.091272in}{2.105658in}}%
\pgfpathlineto{\pgfqpoint{4.083589in}{2.095241in}}%
\pgfpathlineto{\pgfqpoint{4.075901in}{2.084776in}}%
\pgfpathlineto{\pgfqpoint{4.068207in}{2.074264in}}%
\pgfpathclose%
\pgfusepath{fill}%
\end{pgfscope}%
\begin{pgfscope}%
\pgfpathrectangle{\pgfqpoint{1.254980in}{0.150000in}}{\pgfqpoint{5.490039in}{5.490039in}}%
\pgfusepath{clip}%
\pgfsetbuttcap%
\pgfsetroundjoin%
\definecolor{currentfill}{rgb}{0.120565,0.596422,0.543611}%
\pgfsetfillcolor{currentfill}%
\pgfsetfillopacity{0.700000}%
\pgfsetlinewidth{0.000000pt}%
\definecolor{currentstroke}{rgb}{0.000000,0.000000,0.000000}%
\pgfsetstrokecolor{currentstroke}%
\pgfsetdash{}{0pt}%
\pgfpathmoveto{\pgfqpoint{5.421862in}{3.074908in}}%
\pgfpathlineto{\pgfqpoint{5.435868in}{3.085184in}}%
\pgfpathlineto{\pgfqpoint{5.449891in}{3.095615in}}%
\pgfpathlineto{\pgfqpoint{5.463933in}{3.106204in}}%
\pgfpathlineto{\pgfqpoint{5.477993in}{3.116949in}}%
\pgfpathlineto{\pgfqpoint{5.485063in}{3.120277in}}%
\pgfpathlineto{\pgfqpoint{5.492126in}{3.123584in}}%
\pgfpathlineto{\pgfqpoint{5.499183in}{3.126877in}}%
\pgfpathlineto{\pgfqpoint{5.506233in}{3.130158in}}%
\pgfpathlineto{\pgfqpoint{5.492196in}{3.119840in}}%
\pgfpathlineto{\pgfqpoint{5.478176in}{3.109677in}}%
\pgfpathlineto{\pgfqpoint{5.464175in}{3.099670in}}%
\pgfpathlineto{\pgfqpoint{5.450191in}{3.089819in}}%
\pgfpathlineto{\pgfqpoint{5.443118in}{3.086102in}}%
\pgfpathlineto{\pgfqpoint{5.436039in}{3.082381in}}%
\pgfpathlineto{\pgfqpoint{5.428954in}{3.078651in}}%
\pgfpathlineto{\pgfqpoint{5.421862in}{3.074908in}}%
\pgfpathclose%
\pgfusepath{fill}%
\end{pgfscope}%
\begin{pgfscope}%
\pgfpathrectangle{\pgfqpoint{1.254980in}{0.150000in}}{\pgfqpoint{5.490039in}{5.490039in}}%
\pgfusepath{clip}%
\pgfsetbuttcap%
\pgfsetroundjoin%
\definecolor{currentfill}{rgb}{0.169646,0.456262,0.558030}%
\pgfsetfillcolor{currentfill}%
\pgfsetfillopacity{0.700000}%
\pgfsetlinewidth{0.000000pt}%
\definecolor{currentstroke}{rgb}{0.000000,0.000000,0.000000}%
\pgfsetstrokecolor{currentstroke}%
\pgfsetdash{}{0pt}%
\pgfpathmoveto{\pgfqpoint{4.886540in}{2.696940in}}%
\pgfpathlineto{\pgfqpoint{4.900255in}{2.705566in}}%
\pgfpathlineto{\pgfqpoint{4.913986in}{2.714351in}}%
\pgfpathlineto{\pgfqpoint{4.927732in}{2.723295in}}%
\pgfpathlineto{\pgfqpoint{4.941493in}{2.732398in}}%
\pgfpathlineto{\pgfqpoint{4.948862in}{2.739170in}}%
\pgfpathlineto{\pgfqpoint{4.956223in}{2.745863in}}%
\pgfpathlineto{\pgfqpoint{4.963577in}{2.752480in}}%
\pgfpathlineto{\pgfqpoint{4.970925in}{2.759023in}}%
\pgfpathlineto{\pgfqpoint{4.957174in}{2.750136in}}%
\pgfpathlineto{\pgfqpoint{4.943439in}{2.741407in}}%
\pgfpathlineto{\pgfqpoint{4.929720in}{2.732837in}}%
\pgfpathlineto{\pgfqpoint{4.916016in}{2.724426in}}%
\pgfpathlineto{\pgfqpoint{4.908657in}{2.717657in}}%
\pgfpathlineto{\pgfqpoint{4.901291in}{2.710822in}}%
\pgfpathlineto{\pgfqpoint{4.893919in}{2.703917in}}%
\pgfpathlineto{\pgfqpoint{4.886540in}{2.696940in}}%
\pgfpathclose%
\pgfusepath{fill}%
\end{pgfscope}%
\begin{pgfscope}%
\pgfpathrectangle{\pgfqpoint{1.254980in}{0.150000in}}{\pgfqpoint{5.490039in}{5.490039in}}%
\pgfusepath{clip}%
\pgfsetbuttcap%
\pgfsetroundjoin%
\definecolor{currentfill}{rgb}{0.220057,0.343307,0.549413}%
\pgfsetfillcolor{currentfill}%
\pgfsetfillopacity{0.700000}%
\pgfsetlinewidth{0.000000pt}%
\definecolor{currentstroke}{rgb}{0.000000,0.000000,0.000000}%
\pgfsetstrokecolor{currentstroke}%
\pgfsetdash{}{0pt}%
\pgfpathmoveto{\pgfqpoint{4.519439in}{2.412640in}}%
\pgfpathlineto{\pgfqpoint{4.532966in}{2.419331in}}%
\pgfpathlineto{\pgfqpoint{4.546506in}{2.426183in}}%
\pgfpathlineto{\pgfqpoint{4.560060in}{2.433196in}}%
\pgfpathlineto{\pgfqpoint{4.573627in}{2.440371in}}%
\pgfpathlineto{\pgfqpoint{4.581159in}{2.449384in}}%
\pgfpathlineto{\pgfqpoint{4.588685in}{2.458311in}}%
\pgfpathlineto{\pgfqpoint{4.596205in}{2.467153in}}%
\pgfpathlineto{\pgfqpoint{4.603719in}{2.475911in}}%
\pgfpathlineto{\pgfqpoint{4.590158in}{2.468803in}}%
\pgfpathlineto{\pgfqpoint{4.576611in}{2.461857in}}%
\pgfpathlineto{\pgfqpoint{4.563077in}{2.455072in}}%
\pgfpathlineto{\pgfqpoint{4.549557in}{2.448449in}}%
\pgfpathlineto{\pgfqpoint{4.542036in}{2.439613in}}%
\pgfpathlineto{\pgfqpoint{4.534510in}{2.430700in}}%
\pgfpathlineto{\pgfqpoint{4.526977in}{2.421709in}}%
\pgfpathlineto{\pgfqpoint{4.519439in}{2.412640in}}%
\pgfpathclose%
\pgfusepath{fill}%
\end{pgfscope}%
\begin{pgfscope}%
\pgfpathrectangle{\pgfqpoint{1.254980in}{0.150000in}}{\pgfqpoint{5.490039in}{5.490039in}}%
\pgfusepath{clip}%
\pgfsetbuttcap%
\pgfsetroundjoin%
\definecolor{currentfill}{rgb}{0.270595,0.214069,0.507052}%
\pgfsetfillcolor{currentfill}%
\pgfsetfillopacity{0.700000}%
\pgfsetlinewidth{0.000000pt}%
\definecolor{currentstroke}{rgb}{0.000000,0.000000,0.000000}%
\pgfsetstrokecolor{currentstroke}%
\pgfsetdash{}{0pt}%
\pgfpathmoveto{\pgfqpoint{4.152304in}{2.130310in}}%
\pgfpathlineto{\pgfqpoint{4.165668in}{2.134294in}}%
\pgfpathlineto{\pgfqpoint{4.179043in}{2.138442in}}%
\pgfpathlineto{\pgfqpoint{4.192428in}{2.142756in}}%
\pgfpathlineto{\pgfqpoint{4.205825in}{2.147233in}}%
\pgfpathlineto{\pgfqpoint{4.213487in}{2.157728in}}%
\pgfpathlineto{\pgfqpoint{4.221145in}{2.168160in}}%
\pgfpathlineto{\pgfqpoint{4.228798in}{2.178528in}}%
\pgfpathlineto{\pgfqpoint{4.236446in}{2.188832in}}%
\pgfpathlineto{\pgfqpoint{4.223054in}{2.184278in}}%
\pgfpathlineto{\pgfqpoint{4.209674in}{2.179889in}}%
\pgfpathlineto{\pgfqpoint{4.196305in}{2.175663in}}%
\pgfpathlineto{\pgfqpoint{4.182946in}{2.171603in}}%
\pgfpathlineto{\pgfqpoint{4.175293in}{2.161365in}}%
\pgfpathlineto{\pgfqpoint{4.167635in}{2.151069in}}%
\pgfpathlineto{\pgfqpoint{4.159972in}{2.140718in}}%
\pgfpathlineto{\pgfqpoint{4.152304in}{2.130310in}}%
\pgfpathclose%
\pgfusepath{fill}%
\end{pgfscope}%
\begin{pgfscope}%
\pgfpathrectangle{\pgfqpoint{1.254980in}{0.150000in}}{\pgfqpoint{5.490039in}{5.490039in}}%
\pgfusepath{clip}%
\pgfsetbuttcap%
\pgfsetroundjoin%
\definecolor{currentfill}{rgb}{0.212395,0.359683,0.551710}%
\pgfsetfillcolor{currentfill}%
\pgfsetfillopacity{0.700000}%
\pgfsetlinewidth{0.000000pt}%
\definecolor{currentstroke}{rgb}{0.000000,0.000000,0.000000}%
\pgfsetstrokecolor{currentstroke}%
\pgfsetdash{}{0pt}%
\pgfpathmoveto{\pgfqpoint{2.394414in}{2.520709in}}%
\pgfpathlineto{\pgfqpoint{2.407933in}{2.500067in}}%
\pgfpathlineto{\pgfqpoint{2.421439in}{2.479715in}}%
\pgfpathlineto{\pgfqpoint{2.434933in}{2.459650in}}%
\pgfpathlineto{\pgfqpoint{2.448416in}{2.439869in}}%
\pgfpathlineto{\pgfqpoint{2.456953in}{2.440332in}}%
\pgfpathlineto{\pgfqpoint{2.465474in}{2.441024in}}%
\pgfpathlineto{\pgfqpoint{2.473978in}{2.441944in}}%
\pgfpathlineto{\pgfqpoint{2.482468in}{2.443085in}}%
\pgfpathlineto{\pgfqpoint{2.469028in}{2.462424in}}%
\pgfpathlineto{\pgfqpoint{2.455576in}{2.482047in}}%
\pgfpathlineto{\pgfqpoint{2.442114in}{2.501956in}}%
\pgfpathlineto{\pgfqpoint{2.428640in}{2.522153in}}%
\pgfpathlineto{\pgfqpoint{2.420108in}{2.521443in}}%
\pgfpathlineto{\pgfqpoint{2.411560in}{2.520962in}}%
\pgfpathlineto{\pgfqpoint{2.402996in}{2.520716in}}%
\pgfpathlineto{\pgfqpoint{2.394414in}{2.520709in}}%
\pgfpathclose%
\pgfusepath{fill}%
\end{pgfscope}%
\begin{pgfscope}%
\pgfpathrectangle{\pgfqpoint{1.254980in}{0.150000in}}{\pgfqpoint{5.490039in}{5.490039in}}%
\pgfusepath{clip}%
\pgfsetbuttcap%
\pgfsetroundjoin%
\definecolor{currentfill}{rgb}{0.272594,0.025563,0.353093}%
\pgfsetfillcolor{currentfill}%
\pgfsetfillopacity{0.700000}%
\pgfsetlinewidth{0.000000pt}%
\definecolor{currentstroke}{rgb}{0.000000,0.000000,0.000000}%
\pgfsetstrokecolor{currentstroke}%
\pgfsetdash{}{0pt}%
\pgfpathmoveto{\pgfqpoint{3.479310in}{1.787316in}}%
\pgfpathlineto{\pgfqpoint{3.492502in}{1.784213in}}%
\pgfpathlineto{\pgfqpoint{3.505697in}{1.781288in}}%
\pgfpathlineto{\pgfqpoint{3.518898in}{1.778542in}}%
\pgfpathlineto{\pgfqpoint{3.532103in}{1.775973in}}%
\pgfpathlineto{\pgfqpoint{3.539993in}{1.785348in}}%
\pgfpathlineto{\pgfqpoint{3.547877in}{1.794768in}}%
\pgfpathlineto{\pgfqpoint{3.555755in}{1.804229in}}%
\pgfpathlineto{\pgfqpoint{3.563628in}{1.813729in}}%
\pgfpathlineto{\pgfqpoint{3.550436in}{1.815998in}}%
\pgfpathlineto{\pgfqpoint{3.537249in}{1.818445in}}%
\pgfpathlineto{\pgfqpoint{3.524067in}{1.821070in}}%
\pgfpathlineto{\pgfqpoint{3.510889in}{1.823873in}}%
\pgfpathlineto{\pgfqpoint{3.503004in}{1.814662in}}%
\pgfpathlineto{\pgfqpoint{3.495112in}{1.805497in}}%
\pgfpathlineto{\pgfqpoint{3.487214in}{1.796381in}}%
\pgfpathlineto{\pgfqpoint{3.479310in}{1.787316in}}%
\pgfpathclose%
\pgfusepath{fill}%
\end{pgfscope}%
\begin{pgfscope}%
\pgfpathrectangle{\pgfqpoint{1.254980in}{0.150000in}}{\pgfqpoint{5.490039in}{5.490039in}}%
\pgfusepath{clip}%
\pgfsetbuttcap%
\pgfsetroundjoin%
\definecolor{currentfill}{rgb}{0.119699,0.618490,0.536347}%
\pgfsetfillcolor{currentfill}%
\pgfsetfillopacity{0.700000}%
\pgfsetlinewidth{0.000000pt}%
\definecolor{currentstroke}{rgb}{0.000000,0.000000,0.000000}%
\pgfsetstrokecolor{currentstroke}%
\pgfsetdash{}{0pt}%
\pgfpathmoveto{\pgfqpoint{5.506233in}{3.130158in}}%
\pgfpathlineto{\pgfqpoint{5.520288in}{3.140632in}}%
\pgfpathlineto{\pgfqpoint{5.534362in}{3.151263in}}%
\pgfpathlineto{\pgfqpoint{5.548454in}{3.162049in}}%
\pgfpathlineto{\pgfqpoint{5.562564in}{3.172992in}}%
\pgfpathlineto{\pgfqpoint{5.569584in}{3.175822in}}%
\pgfpathlineto{\pgfqpoint{5.576597in}{3.178644in}}%
\pgfpathlineto{\pgfqpoint{5.583603in}{3.181464in}}%
\pgfpathlineto{\pgfqpoint{5.590604in}{3.184286in}}%
\pgfpathlineto{\pgfqpoint{5.576518in}{3.173800in}}%
\pgfpathlineto{\pgfqpoint{5.562450in}{3.163470in}}%
\pgfpathlineto{\pgfqpoint{5.548401in}{3.153295in}}%
\pgfpathlineto{\pgfqpoint{5.534370in}{3.143275in}}%
\pgfpathlineto{\pgfqpoint{5.527345in}{3.139987in}}%
\pgfpathlineto{\pgfqpoint{5.520314in}{3.136708in}}%
\pgfpathlineto{\pgfqpoint{5.513276in}{3.133434in}}%
\pgfpathlineto{\pgfqpoint{5.506233in}{3.130158in}}%
\pgfpathclose%
\pgfusepath{fill}%
\end{pgfscope}%
\begin{pgfscope}%
\pgfpathrectangle{\pgfqpoint{1.254980in}{0.150000in}}{\pgfqpoint{5.490039in}{5.490039in}}%
\pgfusepath{clip}%
\pgfsetbuttcap%
\pgfsetroundjoin%
\definecolor{currentfill}{rgb}{0.282656,0.100196,0.422160}%
\pgfsetfillcolor{currentfill}%
\pgfsetfillopacity{0.700000}%
\pgfsetlinewidth{0.000000pt}%
\definecolor{currentstroke}{rgb}{0.000000,0.000000,0.000000}%
\pgfsetstrokecolor{currentstroke}%
\pgfsetdash{}{0pt}%
\pgfpathmoveto{\pgfqpoint{2.875454in}{1.943137in}}%
\pgfpathlineto{\pgfqpoint{2.888707in}{1.931501in}}%
\pgfpathlineto{\pgfqpoint{2.901956in}{1.920081in}}%
\pgfpathlineto{\pgfqpoint{2.915203in}{1.908876in}}%
\pgfpathlineto{\pgfqpoint{2.928446in}{1.897885in}}%
\pgfpathlineto{\pgfqpoint{2.936655in}{1.902299in}}%
\pgfpathlineto{\pgfqpoint{2.944852in}{1.906876in}}%
\pgfpathlineto{\pgfqpoint{2.953039in}{1.911611in}}%
\pgfpathlineto{\pgfqpoint{2.961215in}{1.916501in}}%
\pgfpathlineto{\pgfqpoint{2.948000in}{1.927074in}}%
\pgfpathlineto{\pgfqpoint{2.934784in}{1.937860in}}%
\pgfpathlineto{\pgfqpoint{2.921564in}{1.948861in}}%
\pgfpathlineto{\pgfqpoint{2.908342in}{1.960077in}}%
\pgfpathlineto{\pgfqpoint{2.900137in}{1.955595in}}%
\pgfpathlineto{\pgfqpoint{2.891921in}{1.951275in}}%
\pgfpathlineto{\pgfqpoint{2.883693in}{1.947121in}}%
\pgfpathlineto{\pgfqpoint{2.875454in}{1.943137in}}%
\pgfpathclose%
\pgfusepath{fill}%
\end{pgfscope}%
\begin{pgfscope}%
\pgfpathrectangle{\pgfqpoint{1.254980in}{0.150000in}}{\pgfqpoint{5.490039in}{5.490039in}}%
\pgfusepath{clip}%
\pgfsetbuttcap%
\pgfsetroundjoin%
\definecolor{currentfill}{rgb}{0.123444,0.636809,0.528763}%
\pgfsetfillcolor{currentfill}%
\pgfsetfillopacity{0.700000}%
\pgfsetlinewidth{0.000000pt}%
\definecolor{currentstroke}{rgb}{0.000000,0.000000,0.000000}%
\pgfsetstrokecolor{currentstroke}%
\pgfsetdash{}{0pt}%
\pgfpathmoveto{\pgfqpoint{5.590604in}{3.184286in}}%
\pgfpathlineto{\pgfqpoint{5.604708in}{3.194927in}}%
\pgfpathlineto{\pgfqpoint{5.618831in}{3.205724in}}%
\pgfpathlineto{\pgfqpoint{5.632972in}{3.216676in}}%
\pgfpathlineto{\pgfqpoint{5.647133in}{3.227785in}}%
\pgfpathlineto{\pgfqpoint{5.654101in}{3.230138in}}%
\pgfpathlineto{\pgfqpoint{5.661062in}{3.232497in}}%
\pgfpathlineto{\pgfqpoint{5.668018in}{3.234867in}}%
\pgfpathlineto{\pgfqpoint{5.674967in}{3.237254in}}%
\pgfpathlineto{\pgfqpoint{5.660834in}{3.226633in}}%
\pgfpathlineto{\pgfqpoint{5.646719in}{3.216167in}}%
\pgfpathlineto{\pgfqpoint{5.632622in}{3.205856in}}%
\pgfpathlineto{\pgfqpoint{5.618544in}{3.195699in}}%
\pgfpathlineto{\pgfqpoint{5.611568in}{3.192816in}}%
\pgfpathlineto{\pgfqpoint{5.604586in}{3.189957in}}%
\pgfpathlineto{\pgfqpoint{5.597598in}{3.187115in}}%
\pgfpathlineto{\pgfqpoint{5.590604in}{3.184286in}}%
\pgfpathclose%
\pgfusepath{fill}%
\end{pgfscope}%
\begin{pgfscope}%
\pgfpathrectangle{\pgfqpoint{1.254980in}{0.150000in}}{\pgfqpoint{5.490039in}{5.490039in}}%
\pgfusepath{clip}%
\pgfsetbuttcap%
\pgfsetroundjoin%
\definecolor{currentfill}{rgb}{0.269944,0.014625,0.341379}%
\pgfsetfillcolor{currentfill}%
\pgfsetfillopacity{0.700000}%
\pgfsetlinewidth{0.000000pt}%
\definecolor{currentstroke}{rgb}{0.000000,0.000000,0.000000}%
\pgfsetstrokecolor{currentstroke}%
\pgfsetdash{}{0pt}%
\pgfpathmoveto{\pgfqpoint{3.257429in}{1.776643in}}%
\pgfpathlineto{\pgfqpoint{3.270617in}{1.770652in}}%
\pgfpathlineto{\pgfqpoint{3.283807in}{1.764850in}}%
\pgfpathlineto{\pgfqpoint{3.296998in}{1.759235in}}%
\pgfpathlineto{\pgfqpoint{3.310193in}{1.753806in}}%
\pgfpathlineto{\pgfqpoint{3.318184in}{1.761621in}}%
\pgfpathlineto{\pgfqpoint{3.326168in}{1.769526in}}%
\pgfpathlineto{\pgfqpoint{3.334144in}{1.777518in}}%
\pgfpathlineto{\pgfqpoint{3.342114in}{1.785593in}}%
\pgfpathlineto{\pgfqpoint{3.328938in}{1.790666in}}%
\pgfpathlineto{\pgfqpoint{3.315765in}{1.795925in}}%
\pgfpathlineto{\pgfqpoint{3.302594in}{1.801372in}}%
\pgfpathlineto{\pgfqpoint{3.289425in}{1.807006in}}%
\pgfpathlineto{\pgfqpoint{3.281438in}{1.799277in}}%
\pgfpathlineto{\pgfqpoint{3.273443in}{1.791637in}}%
\pgfpathlineto{\pgfqpoint{3.265440in}{1.784092in}}%
\pgfpathlineto{\pgfqpoint{3.257429in}{1.776643in}}%
\pgfpathclose%
\pgfusepath{fill}%
\end{pgfscope}%
\begin{pgfscope}%
\pgfpathrectangle{\pgfqpoint{1.254980in}{0.150000in}}{\pgfqpoint{5.490039in}{5.490039in}}%
\pgfusepath{clip}%
\pgfsetbuttcap%
\pgfsetroundjoin%
\definecolor{currentfill}{rgb}{0.159194,0.482237,0.558073}%
\pgfsetfillcolor{currentfill}%
\pgfsetfillopacity{0.700000}%
\pgfsetlinewidth{0.000000pt}%
\definecolor{currentstroke}{rgb}{0.000000,0.000000,0.000000}%
\pgfsetstrokecolor{currentstroke}%
\pgfsetdash{}{0pt}%
\pgfpathmoveto{\pgfqpoint{4.970925in}{2.759023in}}%
\pgfpathlineto{\pgfqpoint{4.984691in}{2.768070in}}%
\pgfpathlineto{\pgfqpoint{4.998473in}{2.777275in}}%
\pgfpathlineto{\pgfqpoint{5.012271in}{2.786639in}}%
\pgfpathlineto{\pgfqpoint{5.026085in}{2.796162in}}%
\pgfpathlineto{\pgfqpoint{5.033414in}{2.802399in}}%
\pgfpathlineto{\pgfqpoint{5.040735in}{2.808562in}}%
\pgfpathlineto{\pgfqpoint{5.048049in}{2.814652in}}%
\pgfpathlineto{\pgfqpoint{5.055356in}{2.820672in}}%
\pgfpathlineto{\pgfqpoint{5.041555in}{2.811395in}}%
\pgfpathlineto{\pgfqpoint{5.027769in}{2.802277in}}%
\pgfpathlineto{\pgfqpoint{5.013999in}{2.793317in}}%
\pgfpathlineto{\pgfqpoint{5.000245in}{2.784515in}}%
\pgfpathlineto{\pgfqpoint{4.992925in}{2.778239in}}%
\pgfpathlineto{\pgfqpoint{4.985599in}{2.771900in}}%
\pgfpathlineto{\pgfqpoint{4.978265in}{2.765496in}}%
\pgfpathlineto{\pgfqpoint{4.970925in}{2.759023in}}%
\pgfpathclose%
\pgfusepath{fill}%
\end{pgfscope}%
\begin{pgfscope}%
\pgfpathrectangle{\pgfqpoint{1.254980in}{0.150000in}}{\pgfqpoint{5.490039in}{5.490039in}}%
\pgfusepath{clip}%
\pgfsetbuttcap%
\pgfsetroundjoin%
\definecolor{currentfill}{rgb}{0.260571,0.246922,0.522828}%
\pgfsetfillcolor{currentfill}%
\pgfsetfillopacity{0.700000}%
\pgfsetlinewidth{0.000000pt}%
\definecolor{currentstroke}{rgb}{0.000000,0.000000,0.000000}%
\pgfsetstrokecolor{currentstroke}%
\pgfsetdash{}{0pt}%
\pgfpathmoveto{\pgfqpoint{4.236446in}{2.188832in}}%
\pgfpathlineto{\pgfqpoint{4.249848in}{2.193551in}}%
\pgfpathlineto{\pgfqpoint{4.263261in}{2.198432in}}%
\pgfpathlineto{\pgfqpoint{4.276687in}{2.203478in}}%
\pgfpathlineto{\pgfqpoint{4.290123in}{2.208687in}}%
\pgfpathlineto{\pgfqpoint{4.297761in}{2.218986in}}%
\pgfpathlineto{\pgfqpoint{4.305393in}{2.229213in}}%
\pgfpathlineto{\pgfqpoint{4.313020in}{2.239368in}}%
\pgfpathlineto{\pgfqpoint{4.320642in}{2.249452in}}%
\pgfpathlineto{\pgfqpoint{4.307210in}{2.244195in}}%
\pgfpathlineto{\pgfqpoint{4.293790in}{2.239101in}}%
\pgfpathlineto{\pgfqpoint{4.280381in}{2.234171in}}%
\pgfpathlineto{\pgfqpoint{4.266984in}{2.229405in}}%
\pgfpathlineto{\pgfqpoint{4.259357in}{2.219359in}}%
\pgfpathlineto{\pgfqpoint{4.251725in}{2.209248in}}%
\pgfpathlineto{\pgfqpoint{4.244088in}{2.199072in}}%
\pgfpathlineto{\pgfqpoint{4.236446in}{2.188832in}}%
\pgfpathclose%
\pgfusepath{fill}%
\end{pgfscope}%
\begin{pgfscope}%
\pgfpathrectangle{\pgfqpoint{1.254980in}{0.150000in}}{\pgfqpoint{5.490039in}{5.490039in}}%
\pgfusepath{clip}%
\pgfsetbuttcap%
\pgfsetroundjoin%
\definecolor{currentfill}{rgb}{0.206756,0.371758,0.553117}%
\pgfsetfillcolor{currentfill}%
\pgfsetfillopacity{0.700000}%
\pgfsetlinewidth{0.000000pt}%
\definecolor{currentstroke}{rgb}{0.000000,0.000000,0.000000}%
\pgfsetstrokecolor{currentstroke}%
\pgfsetdash{}{0pt}%
\pgfpathmoveto{\pgfqpoint{4.603719in}{2.475911in}}%
\pgfpathlineto{\pgfqpoint{4.617293in}{2.483179in}}%
\pgfpathlineto{\pgfqpoint{4.630882in}{2.490609in}}%
\pgfpathlineto{\pgfqpoint{4.644484in}{2.498199in}}%
\pgfpathlineto{\pgfqpoint{4.658100in}{2.505950in}}%
\pgfpathlineto{\pgfqpoint{4.665601in}{2.514539in}}%
\pgfpathlineto{\pgfqpoint{4.673096in}{2.523040in}}%
\pgfpathlineto{\pgfqpoint{4.680584in}{2.531454in}}%
\pgfpathlineto{\pgfqpoint{4.688066in}{2.539782in}}%
\pgfpathlineto{\pgfqpoint{4.674457in}{2.532129in}}%
\pgfpathlineto{\pgfqpoint{4.660862in}{2.524635in}}%
\pgfpathlineto{\pgfqpoint{4.647281in}{2.517303in}}%
\pgfpathlineto{\pgfqpoint{4.633714in}{2.510131in}}%
\pgfpathlineto{\pgfqpoint{4.626224in}{2.501695in}}%
\pgfpathlineto{\pgfqpoint{4.618729in}{2.493181in}}%
\pgfpathlineto{\pgfqpoint{4.611227in}{2.484586in}}%
\pgfpathlineto{\pgfqpoint{4.603719in}{2.475911in}}%
\pgfpathclose%
\pgfusepath{fill}%
\end{pgfscope}%
\begin{pgfscope}%
\pgfpathrectangle{\pgfqpoint{1.254980in}{0.150000in}}{\pgfqpoint{5.490039in}{5.490039in}}%
\pgfusepath{clip}%
\pgfsetbuttcap%
\pgfsetroundjoin%
\definecolor{currentfill}{rgb}{0.273809,0.031497,0.358853}%
\pgfsetfillcolor{currentfill}%
\pgfsetfillopacity{0.700000}%
\pgfsetlinewidth{0.000000pt}%
\definecolor{currentstroke}{rgb}{0.000000,0.000000,0.000000}%
\pgfsetstrokecolor{currentstroke}%
\pgfsetdash{}{0pt}%
\pgfpathmoveto{\pgfqpoint{3.119678in}{1.805784in}}%
\pgfpathlineto{\pgfqpoint{3.132879in}{1.797867in}}%
\pgfpathlineto{\pgfqpoint{3.146081in}{1.790147in}}%
\pgfpathlineto{\pgfqpoint{3.159283in}{1.782622in}}%
\pgfpathlineto{\pgfqpoint{3.172486in}{1.775290in}}%
\pgfpathlineto{\pgfqpoint{3.180551in}{1.781921in}}%
\pgfpathlineto{\pgfqpoint{3.188607in}{1.788670in}}%
\pgfpathlineto{\pgfqpoint{3.196655in}{1.795533in}}%
\pgfpathlineto{\pgfqpoint{3.204694in}{1.802507in}}%
\pgfpathlineto{\pgfqpoint{3.191513in}{1.809453in}}%
\pgfpathlineto{\pgfqpoint{3.178334in}{1.816593in}}%
\pgfpathlineto{\pgfqpoint{3.165155in}{1.823929in}}%
\pgfpathlineto{\pgfqpoint{3.151976in}{1.831460in}}%
\pgfpathlineto{\pgfqpoint{3.143915in}{1.824860in}}%
\pgfpathlineto{\pgfqpoint{3.135845in}{1.818378in}}%
\pgfpathlineto{\pgfqpoint{3.127766in}{1.812018in}}%
\pgfpathlineto{\pgfqpoint{3.119678in}{1.805784in}}%
\pgfpathclose%
\pgfusepath{fill}%
\end{pgfscope}%
\begin{pgfscope}%
\pgfpathrectangle{\pgfqpoint{1.254980in}{0.150000in}}{\pgfqpoint{5.490039in}{5.490039in}}%
\pgfusepath{clip}%
\pgfsetbuttcap%
\pgfsetroundjoin%
\definecolor{currentfill}{rgb}{0.134692,0.658636,0.517649}%
\pgfsetfillcolor{currentfill}%
\pgfsetfillopacity{0.700000}%
\pgfsetlinewidth{0.000000pt}%
\definecolor{currentstroke}{rgb}{0.000000,0.000000,0.000000}%
\pgfsetstrokecolor{currentstroke}%
\pgfsetdash{}{0pt}%
\pgfpathmoveto{\pgfqpoint{5.674967in}{3.237254in}}%
\pgfpathlineto{\pgfqpoint{5.689120in}{3.248030in}}%
\pgfpathlineto{\pgfqpoint{5.703291in}{3.258961in}}%
\pgfpathlineto{\pgfqpoint{5.717481in}{3.270048in}}%
\pgfpathlineto{\pgfqpoint{5.731691in}{3.281290in}}%
\pgfpathlineto{\pgfqpoint{5.738606in}{3.283192in}}%
\pgfpathlineto{\pgfqpoint{5.745516in}{3.285115in}}%
\pgfpathlineto{\pgfqpoint{5.752420in}{3.287065in}}%
\pgfpathlineto{\pgfqpoint{5.759318in}{3.289046in}}%
\pgfpathlineto{\pgfqpoint{5.745138in}{3.278322in}}%
\pgfpathlineto{\pgfqpoint{5.730976in}{3.267752in}}%
\pgfpathlineto{\pgfqpoint{5.716834in}{3.257337in}}%
\pgfpathlineto{\pgfqpoint{5.702710in}{3.247076in}}%
\pgfpathlineto{\pgfqpoint{5.695782in}{3.244568in}}%
\pgfpathlineto{\pgfqpoint{5.688849in}{3.242099in}}%
\pgfpathlineto{\pgfqpoint{5.681911in}{3.239662in}}%
\pgfpathlineto{\pgfqpoint{5.674967in}{3.237254in}}%
\pgfpathclose%
\pgfusepath{fill}%
\end{pgfscope}%
\begin{pgfscope}%
\pgfpathrectangle{\pgfqpoint{1.254980in}{0.150000in}}{\pgfqpoint{5.490039in}{5.490039in}}%
\pgfusepath{clip}%
\pgfsetbuttcap%
\pgfsetroundjoin%
\definecolor{currentfill}{rgb}{0.195860,0.395433,0.555276}%
\pgfsetfillcolor{currentfill}%
\pgfsetfillopacity{0.700000}%
\pgfsetlinewidth{0.000000pt}%
\definecolor{currentstroke}{rgb}{0.000000,0.000000,0.000000}%
\pgfsetstrokecolor{currentstroke}%
\pgfsetdash{}{0pt}%
\pgfpathmoveto{\pgfqpoint{2.340214in}{2.606230in}}%
\pgfpathlineto{\pgfqpoint{2.353784in}{2.584400in}}%
\pgfpathlineto{\pgfqpoint{2.367340in}{2.562873in}}%
\pgfpathlineto{\pgfqpoint{2.380884in}{2.541643in}}%
\pgfpathlineto{\pgfqpoint{2.394414in}{2.520709in}}%
\pgfpathlineto{\pgfqpoint{2.402996in}{2.520716in}}%
\pgfpathlineto{\pgfqpoint{2.411560in}{2.520962in}}%
\pgfpathlineto{\pgfqpoint{2.420108in}{2.521443in}}%
\pgfpathlineto{\pgfqpoint{2.428640in}{2.522153in}}%
\pgfpathlineto{\pgfqpoint{2.415154in}{2.542642in}}%
\pgfpathlineto{\pgfqpoint{2.401656in}{2.563426in}}%
\pgfpathlineto{\pgfqpoint{2.388145in}{2.584506in}}%
\pgfpathlineto{\pgfqpoint{2.374622in}{2.605887in}}%
\pgfpathlineto{\pgfqpoint{2.366045in}{2.605611in}}%
\pgfpathlineto{\pgfqpoint{2.357452in}{2.605574in}}%
\pgfpathlineto{\pgfqpoint{2.348842in}{2.605778in}}%
\pgfpathlineto{\pgfqpoint{2.340214in}{2.606230in}}%
\pgfpathclose%
\pgfusepath{fill}%
\end{pgfscope}%
\begin{pgfscope}%
\pgfpathrectangle{\pgfqpoint{1.254980in}{0.150000in}}{\pgfqpoint{5.490039in}{5.490039in}}%
\pgfusepath{clip}%
\pgfsetbuttcap%
\pgfsetroundjoin%
\definecolor{currentfill}{rgb}{0.281446,0.084320,0.407414}%
\pgfsetfillcolor{currentfill}%
\pgfsetfillopacity{0.700000}%
\pgfsetlinewidth{0.000000pt}%
\definecolor{currentstroke}{rgb}{0.000000,0.000000,0.000000}%
\pgfsetstrokecolor{currentstroke}%
\pgfsetdash{}{0pt}%
\pgfpathmoveto{\pgfqpoint{2.928446in}{1.897885in}}%
\pgfpathlineto{\pgfqpoint{2.941688in}{1.887107in}}%
\pgfpathlineto{\pgfqpoint{2.954927in}{1.876539in}}%
\pgfpathlineto{\pgfqpoint{2.968163in}{1.866180in}}%
\pgfpathlineto{\pgfqpoint{2.981398in}{1.856030in}}%
\pgfpathlineto{\pgfqpoint{2.989578in}{1.860872in}}%
\pgfpathlineto{\pgfqpoint{2.997747in}{1.865870in}}%
\pgfpathlineto{\pgfqpoint{3.005906in}{1.871019in}}%
\pgfpathlineto{\pgfqpoint{3.014054in}{1.876315in}}%
\pgfpathlineto{\pgfqpoint{3.000847in}{1.886048in}}%
\pgfpathlineto{\pgfqpoint{2.987638in}{1.895990in}}%
\pgfpathlineto{\pgfqpoint{2.974428in}{1.906140in}}%
\pgfpathlineto{\pgfqpoint{2.961215in}{1.916501in}}%
\pgfpathlineto{\pgfqpoint{2.953039in}{1.911611in}}%
\pgfpathlineto{\pgfqpoint{2.944852in}{1.906876in}}%
\pgfpathlineto{\pgfqpoint{2.936655in}{1.902299in}}%
\pgfpathlineto{\pgfqpoint{2.928446in}{1.897885in}}%
\pgfpathclose%
\pgfusepath{fill}%
\end{pgfscope}%
\begin{pgfscope}%
\pgfpathrectangle{\pgfqpoint{1.254980in}{0.150000in}}{\pgfqpoint{5.490039in}{5.490039in}}%
\pgfusepath{clip}%
\pgfsetbuttcap%
\pgfsetroundjoin%
\definecolor{currentfill}{rgb}{0.226397,0.728888,0.462789}%
\pgfsetfillcolor{currentfill}%
\pgfsetfillopacity{0.700000}%
\pgfsetlinewidth{0.000000pt}%
\definecolor{currentstroke}{rgb}{0.000000,0.000000,0.000000}%
\pgfsetstrokecolor{currentstroke}%
\pgfsetdash{}{0pt}%
\pgfpathmoveto{\pgfqpoint{6.012241in}{3.437519in}}%
\pgfpathlineto{\pgfqpoint{6.026574in}{3.448516in}}%
\pgfpathlineto{\pgfqpoint{6.040927in}{3.459667in}}%
\pgfpathlineto{\pgfqpoint{6.055300in}{3.470972in}}%
\pgfpathlineto{\pgfqpoint{6.062014in}{3.471591in}}%
\pgfpathlineto{\pgfqpoint{6.068725in}{3.472304in}}%
\pgfpathlineto{\pgfqpoint{6.075432in}{3.473118in}}%
\pgfpathlineto{\pgfqpoint{6.082138in}{3.474041in}}%
\pgfpathlineto{\pgfqpoint{6.067803in}{3.463374in}}%
\pgfpathlineto{\pgfqpoint{6.053488in}{3.452858in}}%
\pgfpathlineto{\pgfqpoint{6.039193in}{3.442496in}}%
\pgfpathlineto{\pgfqpoint{6.032459in}{3.441089in}}%
\pgfpathlineto{\pgfqpoint{6.025722in}{3.439795in}}%
\pgfpathlineto{\pgfqpoint{6.018983in}{3.438607in}}%
\pgfpathlineto{\pgfqpoint{6.012241in}{3.437519in}}%
\pgfpathclose%
\pgfusepath{fill}%
\end{pgfscope}%
\begin{pgfscope}%
\pgfpathrectangle{\pgfqpoint{1.254980in}{0.150000in}}{\pgfqpoint{5.490039in}{5.490039in}}%
\pgfusepath{clip}%
\pgfsetbuttcap%
\pgfsetroundjoin%
\definecolor{currentfill}{rgb}{0.271305,0.019942,0.347269}%
\pgfsetfillcolor{currentfill}%
\pgfsetfillopacity{0.700000}%
\pgfsetlinewidth{0.000000pt}%
\definecolor{currentstroke}{rgb}{0.000000,0.000000,0.000000}%
\pgfsetstrokecolor{currentstroke}%
\pgfsetdash{}{0pt}%
\pgfpathmoveto{\pgfqpoint{3.394846in}{1.767150in}}%
\pgfpathlineto{\pgfqpoint{3.408037in}{1.762997in}}%
\pgfpathlineto{\pgfqpoint{3.421231in}{1.759026in}}%
\pgfpathlineto{\pgfqpoint{3.434429in}{1.755236in}}%
\pgfpathlineto{\pgfqpoint{3.447631in}{1.751626in}}%
\pgfpathlineto{\pgfqpoint{3.455561in}{1.760457in}}%
\pgfpathlineto{\pgfqpoint{3.463484in}{1.769351in}}%
\pgfpathlineto{\pgfqpoint{3.471400in}{1.778305in}}%
\pgfpathlineto{\pgfqpoint{3.479310in}{1.787316in}}%
\pgfpathlineto{\pgfqpoint{3.466123in}{1.790598in}}%
\pgfpathlineto{\pgfqpoint{3.452940in}{1.794061in}}%
\pgfpathlineto{\pgfqpoint{3.439762in}{1.797705in}}%
\pgfpathlineto{\pgfqpoint{3.426586in}{1.801530in}}%
\pgfpathlineto{\pgfqpoint{3.418661in}{1.792836in}}%
\pgfpathlineto{\pgfqpoint{3.410729in}{1.784206in}}%
\pgfpathlineto{\pgfqpoint{3.402791in}{1.775643in}}%
\pgfpathlineto{\pgfqpoint{3.394846in}{1.767150in}}%
\pgfpathclose%
\pgfusepath{fill}%
\end{pgfscope}%
\begin{pgfscope}%
\pgfpathrectangle{\pgfqpoint{1.254980in}{0.150000in}}{\pgfqpoint{5.490039in}{5.490039in}}%
\pgfusepath{clip}%
\pgfsetbuttcap%
\pgfsetroundjoin%
\definecolor{currentfill}{rgb}{0.150148,0.676631,0.506589}%
\pgfsetfillcolor{currentfill}%
\pgfsetfillopacity{0.700000}%
\pgfsetlinewidth{0.000000pt}%
\definecolor{currentstroke}{rgb}{0.000000,0.000000,0.000000}%
\pgfsetstrokecolor{currentstroke}%
\pgfsetdash{}{0pt}%
\pgfpathmoveto{\pgfqpoint{5.759318in}{3.289046in}}%
\pgfpathlineto{\pgfqpoint{5.773518in}{3.299925in}}%
\pgfpathlineto{\pgfqpoint{5.787736in}{3.310959in}}%
\pgfpathlineto{\pgfqpoint{5.801974in}{3.322148in}}%
\pgfpathlineto{\pgfqpoint{5.816232in}{3.333493in}}%
\pgfpathlineto{\pgfqpoint{5.823094in}{3.334975in}}%
\pgfpathlineto{\pgfqpoint{5.829951in}{3.336494in}}%
\pgfpathlineto{\pgfqpoint{5.836803in}{3.338057in}}%
\pgfpathlineto{\pgfqpoint{5.843650in}{3.339668in}}%
\pgfpathlineto{\pgfqpoint{5.829424in}{3.328872in}}%
\pgfpathlineto{\pgfqpoint{5.815218in}{3.318230in}}%
\pgfpathlineto{\pgfqpoint{5.801030in}{3.307742in}}%
\pgfpathlineto{\pgfqpoint{5.786861in}{3.297408in}}%
\pgfpathlineto{\pgfqpoint{5.779983in}{3.295241in}}%
\pgfpathlineto{\pgfqpoint{5.773099in}{3.293128in}}%
\pgfpathlineto{\pgfqpoint{5.766211in}{3.291065in}}%
\pgfpathlineto{\pgfqpoint{5.759318in}{3.289046in}}%
\pgfpathclose%
\pgfusepath{fill}%
\end{pgfscope}%
\begin{pgfscope}%
\pgfpathrectangle{\pgfqpoint{1.254980in}{0.150000in}}{\pgfqpoint{5.490039in}{5.490039in}}%
\pgfusepath{clip}%
\pgfsetbuttcap%
\pgfsetroundjoin%
\definecolor{currentfill}{rgb}{0.149039,0.508051,0.557250}%
\pgfsetfillcolor{currentfill}%
\pgfsetfillopacity{0.700000}%
\pgfsetlinewidth{0.000000pt}%
\definecolor{currentstroke}{rgb}{0.000000,0.000000,0.000000}%
\pgfsetstrokecolor{currentstroke}%
\pgfsetdash{}{0pt}%
\pgfpathmoveto{\pgfqpoint{5.055356in}{2.820672in}}%
\pgfpathlineto{\pgfqpoint{5.069174in}{2.830108in}}%
\pgfpathlineto{\pgfqpoint{5.083009in}{2.839702in}}%
\pgfpathlineto{\pgfqpoint{5.096859in}{2.849454in}}%
\pgfpathlineto{\pgfqpoint{5.110727in}{2.859365in}}%
\pgfpathlineto{\pgfqpoint{5.118013in}{2.865054in}}%
\pgfpathlineto{\pgfqpoint{5.125292in}{2.870673in}}%
\pgfpathlineto{\pgfqpoint{5.132564in}{2.876225in}}%
\pgfpathlineto{\pgfqpoint{5.139829in}{2.881712in}}%
\pgfpathlineto{\pgfqpoint{5.125976in}{2.872078in}}%
\pgfpathlineto{\pgfqpoint{5.112139in}{2.862602in}}%
\pgfpathlineto{\pgfqpoint{5.098318in}{2.853283in}}%
\pgfpathlineto{\pgfqpoint{5.084514in}{2.844123in}}%
\pgfpathlineto{\pgfqpoint{5.077235in}{2.838349in}}%
\pgfpathlineto{\pgfqpoint{5.069949in}{2.832518in}}%
\pgfpathlineto{\pgfqpoint{5.062656in}{2.826627in}}%
\pgfpathlineto{\pgfqpoint{5.055356in}{2.820672in}}%
\pgfpathclose%
\pgfusepath{fill}%
\end{pgfscope}%
\begin{pgfscope}%
\pgfpathrectangle{\pgfqpoint{1.254980in}{0.150000in}}{\pgfqpoint{5.490039in}{5.490039in}}%
\pgfusepath{clip}%
\pgfsetbuttcap%
\pgfsetroundjoin%
\definecolor{currentfill}{rgb}{0.175707,0.697900,0.491033}%
\pgfsetfillcolor{currentfill}%
\pgfsetfillopacity{0.700000}%
\pgfsetlinewidth{0.000000pt}%
\definecolor{currentstroke}{rgb}{0.000000,0.000000,0.000000}%
\pgfsetstrokecolor{currentstroke}%
\pgfsetdash{}{0pt}%
\pgfpathmoveto{\pgfqpoint{5.843650in}{3.339668in}}%
\pgfpathlineto{\pgfqpoint{5.857896in}{3.350618in}}%
\pgfpathlineto{\pgfqpoint{5.872160in}{3.361723in}}%
\pgfpathlineto{\pgfqpoint{5.886445in}{3.372982in}}%
\pgfpathlineto{\pgfqpoint{5.900749in}{3.384397in}}%
\pgfpathlineto{\pgfqpoint{5.907558in}{3.385495in}}%
\pgfpathlineto{\pgfqpoint{5.914363in}{3.386649in}}%
\pgfpathlineto{\pgfqpoint{5.921163in}{3.387862in}}%
\pgfpathlineto{\pgfqpoint{5.927959in}{3.389143in}}%
\pgfpathlineto{\pgfqpoint{5.913689in}{3.378307in}}%
\pgfpathlineto{\pgfqpoint{5.899438in}{3.367625in}}%
\pgfpathlineto{\pgfqpoint{5.885206in}{3.357097in}}%
\pgfpathlineto{\pgfqpoint{5.870994in}{3.346722in}}%
\pgfpathlineto{\pgfqpoint{5.864164in}{3.344855in}}%
\pgfpathlineto{\pgfqpoint{5.857331in}{3.343061in}}%
\pgfpathlineto{\pgfqpoint{5.850493in}{3.341334in}}%
\pgfpathlineto{\pgfqpoint{5.843650in}{3.339668in}}%
\pgfpathclose%
\pgfusepath{fill}%
\end{pgfscope}%
\begin{pgfscope}%
\pgfpathrectangle{\pgfqpoint{1.254980in}{0.150000in}}{\pgfqpoint{5.490039in}{5.490039in}}%
\pgfusepath{clip}%
\pgfsetbuttcap%
\pgfsetroundjoin%
\definecolor{currentfill}{rgb}{0.248629,0.278775,0.534556}%
\pgfsetfillcolor{currentfill}%
\pgfsetfillopacity{0.700000}%
\pgfsetlinewidth{0.000000pt}%
\definecolor{currentstroke}{rgb}{0.000000,0.000000,0.000000}%
\pgfsetstrokecolor{currentstroke}%
\pgfsetdash{}{0pt}%
\pgfpathmoveto{\pgfqpoint{4.320642in}{2.249452in}}%
\pgfpathlineto{\pgfqpoint{4.334085in}{2.254872in}}%
\pgfpathlineto{\pgfqpoint{4.347540in}{2.260455in}}%
\pgfpathlineto{\pgfqpoint{4.361008in}{2.266201in}}%
\pgfpathlineto{\pgfqpoint{4.374487in}{2.272110in}}%
\pgfpathlineto{\pgfqpoint{4.382099in}{2.282152in}}%
\pgfpathlineto{\pgfqpoint{4.389705in}{2.292116in}}%
\pgfpathlineto{\pgfqpoint{4.397305in}{2.302001in}}%
\pgfpathlineto{\pgfqpoint{4.404901in}{2.311807in}}%
\pgfpathlineto{\pgfqpoint{4.391426in}{2.305879in}}%
\pgfpathlineto{\pgfqpoint{4.377964in}{2.300113in}}%
\pgfpathlineto{\pgfqpoint{4.364513in}{2.294511in}}%
\pgfpathlineto{\pgfqpoint{4.351075in}{2.289071in}}%
\pgfpathlineto{\pgfqpoint{4.343475in}{2.279273in}}%
\pgfpathlineto{\pgfqpoint{4.335869in}{2.269404in}}%
\pgfpathlineto{\pgfqpoint{4.328258in}{2.259464in}}%
\pgfpathlineto{\pgfqpoint{4.320642in}{2.249452in}}%
\pgfpathclose%
\pgfusepath{fill}%
\end{pgfscope}%
\begin{pgfscope}%
\pgfpathrectangle{\pgfqpoint{1.254980in}{0.150000in}}{\pgfqpoint{5.490039in}{5.490039in}}%
\pgfusepath{clip}%
\pgfsetbuttcap%
\pgfsetroundjoin%
\definecolor{currentfill}{rgb}{0.202219,0.715272,0.476084}%
\pgfsetfillcolor{currentfill}%
\pgfsetfillopacity{0.700000}%
\pgfsetlinewidth{0.000000pt}%
\definecolor{currentstroke}{rgb}{0.000000,0.000000,0.000000}%
\pgfsetstrokecolor{currentstroke}%
\pgfsetdash{}{0pt}%
\pgfpathmoveto{\pgfqpoint{5.927959in}{3.389143in}}%
\pgfpathlineto{\pgfqpoint{5.942249in}{3.400133in}}%
\pgfpathlineto{\pgfqpoint{5.956559in}{3.411277in}}%
\pgfpathlineto{\pgfqpoint{5.970888in}{3.422575in}}%
\pgfpathlineto{\pgfqpoint{5.985238in}{3.434027in}}%
\pgfpathlineto{\pgfqpoint{5.991994in}{3.434784in}}%
\pgfpathlineto{\pgfqpoint{5.998747in}{3.435614in}}%
\pgfpathlineto{\pgfqpoint{6.005495in}{3.436523in}}%
\pgfpathlineto{\pgfqpoint{6.012241in}{3.437519in}}%
\pgfpathlineto{\pgfqpoint{5.997927in}{3.426675in}}%
\pgfpathlineto{\pgfqpoint{5.983634in}{3.415984in}}%
\pgfpathlineto{\pgfqpoint{5.969360in}{3.405447in}}%
\pgfpathlineto{\pgfqpoint{5.955106in}{3.395062in}}%
\pgfpathlineto{\pgfqpoint{5.948324in}{3.393450in}}%
\pgfpathlineto{\pgfqpoint{5.941539in}{3.391931in}}%
\pgfpathlineto{\pgfqpoint{5.934751in}{3.390497in}}%
\pgfpathlineto{\pgfqpoint{5.927959in}{3.389143in}}%
\pgfpathclose%
\pgfusepath{fill}%
\end{pgfscope}%
\begin{pgfscope}%
\pgfpathrectangle{\pgfqpoint{1.254980in}{0.150000in}}{\pgfqpoint{5.490039in}{5.490039in}}%
\pgfusepath{clip}%
\pgfsetbuttcap%
\pgfsetroundjoin%
\definecolor{currentfill}{rgb}{0.282327,0.094955,0.417331}%
\pgfsetfillcolor{currentfill}%
\pgfsetfillopacity{0.700000}%
\pgfsetlinewidth{0.000000pt}%
\definecolor{currentstroke}{rgb}{0.000000,0.000000,0.000000}%
\pgfsetstrokecolor{currentstroke}%
\pgfsetdash{}{0pt}%
\pgfpathmoveto{\pgfqpoint{3.784899in}{1.883420in}}%
\pgfpathlineto{\pgfqpoint{3.798151in}{1.883922in}}%
\pgfpathlineto{\pgfqpoint{3.811411in}{1.884595in}}%
\pgfpathlineto{\pgfqpoint{3.824679in}{1.885436in}}%
\pgfpathlineto{\pgfqpoint{3.837954in}{1.886447in}}%
\pgfpathlineto{\pgfqpoint{3.845739in}{1.897097in}}%
\pgfpathlineto{\pgfqpoint{3.853518in}{1.907734in}}%
\pgfpathlineto{\pgfqpoint{3.861293in}{1.918359in}}%
\pgfpathlineto{\pgfqpoint{3.869062in}{1.928968in}}%
\pgfpathlineto{\pgfqpoint{3.855795in}{1.927740in}}%
\pgfpathlineto{\pgfqpoint{3.842535in}{1.926682in}}%
\pgfpathlineto{\pgfqpoint{3.829283in}{1.925793in}}%
\pgfpathlineto{\pgfqpoint{3.816039in}{1.925075in}}%
\pgfpathlineto{\pgfqpoint{3.808262in}{1.914672in}}%
\pgfpathlineto{\pgfqpoint{3.800479in}{1.904260in}}%
\pgfpathlineto{\pgfqpoint{3.792692in}{1.893843in}}%
\pgfpathlineto{\pgfqpoint{3.784899in}{1.883420in}}%
\pgfpathclose%
\pgfusepath{fill}%
\end{pgfscope}%
\begin{pgfscope}%
\pgfpathrectangle{\pgfqpoint{1.254980in}{0.150000in}}{\pgfqpoint{5.490039in}{5.490039in}}%
\pgfusepath{clip}%
\pgfsetbuttcap%
\pgfsetroundjoin%
\definecolor{currentfill}{rgb}{0.192357,0.403199,0.555836}%
\pgfsetfillcolor{currentfill}%
\pgfsetfillopacity{0.700000}%
\pgfsetlinewidth{0.000000pt}%
\definecolor{currentstroke}{rgb}{0.000000,0.000000,0.000000}%
\pgfsetstrokecolor{currentstroke}%
\pgfsetdash{}{0pt}%
\pgfpathmoveto{\pgfqpoint{4.688066in}{2.539782in}}%
\pgfpathlineto{\pgfqpoint{4.701689in}{2.547597in}}%
\pgfpathlineto{\pgfqpoint{4.715327in}{2.555572in}}%
\pgfpathlineto{\pgfqpoint{4.728979in}{2.563707in}}%
\pgfpathlineto{\pgfqpoint{4.742646in}{2.572003in}}%
\pgfpathlineto{\pgfqpoint{4.750114in}{2.580131in}}%
\pgfpathlineto{\pgfqpoint{4.757576in}{2.588170in}}%
\pgfpathlineto{\pgfqpoint{4.765031in}{2.596121in}}%
\pgfpathlineto{\pgfqpoint{4.772479in}{2.603985in}}%
\pgfpathlineto{\pgfqpoint{4.758821in}{2.595816in}}%
\pgfpathlineto{\pgfqpoint{4.745176in}{2.587808in}}%
\pgfpathlineto{\pgfqpoint{4.731547in}{2.579959in}}%
\pgfpathlineto{\pgfqpoint{4.717931in}{2.572271in}}%
\pgfpathlineto{\pgfqpoint{4.710474in}{2.564269in}}%
\pgfpathlineto{\pgfqpoint{4.703011in}{2.556188in}}%
\pgfpathlineto{\pgfqpoint{4.695542in}{2.548027in}}%
\pgfpathlineto{\pgfqpoint{4.688066in}{2.539782in}}%
\pgfpathclose%
\pgfusepath{fill}%
\end{pgfscope}%
\begin{pgfscope}%
\pgfpathrectangle{\pgfqpoint{1.254980in}{0.150000in}}{\pgfqpoint{5.490039in}{5.490039in}}%
\pgfusepath{clip}%
\pgfsetbuttcap%
\pgfsetroundjoin%
\definecolor{currentfill}{rgb}{0.280267,0.073417,0.397163}%
\pgfsetfillcolor{currentfill}%
\pgfsetfillopacity{0.700000}%
\pgfsetlinewidth{0.000000pt}%
\definecolor{currentstroke}{rgb}{0.000000,0.000000,0.000000}%
\pgfsetstrokecolor{currentstroke}%
\pgfsetdash{}{0pt}%
\pgfpathmoveto{\pgfqpoint{3.700704in}{1.842403in}}%
\pgfpathlineto{\pgfqpoint{3.713937in}{1.841976in}}%
\pgfpathlineto{\pgfqpoint{3.727177in}{1.841721in}}%
\pgfpathlineto{\pgfqpoint{3.740424in}{1.841638in}}%
\pgfpathlineto{\pgfqpoint{3.753678in}{1.841725in}}%
\pgfpathlineto{\pgfqpoint{3.761491in}{1.852146in}}%
\pgfpathlineto{\pgfqpoint{3.769299in}{1.862570in}}%
\pgfpathlineto{\pgfqpoint{3.777101in}{1.872995in}}%
\pgfpathlineto{\pgfqpoint{3.784899in}{1.883420in}}%
\pgfpathlineto{\pgfqpoint{3.771654in}{1.883089in}}%
\pgfpathlineto{\pgfqpoint{3.758416in}{1.882928in}}%
\pgfpathlineto{\pgfqpoint{3.745185in}{1.882939in}}%
\pgfpathlineto{\pgfqpoint{3.731962in}{1.883121in}}%
\pgfpathlineto{\pgfqpoint{3.724155in}{1.872930in}}%
\pgfpathlineto{\pgfqpoint{3.716343in}{1.862745in}}%
\pgfpathlineto{\pgfqpoint{3.708526in}{1.852569in}}%
\pgfpathlineto{\pgfqpoint{3.700704in}{1.842403in}}%
\pgfpathclose%
\pgfusepath{fill}%
\end{pgfscope}%
\begin{pgfscope}%
\pgfpathrectangle{\pgfqpoint{1.254980in}{0.150000in}}{\pgfqpoint{5.490039in}{5.490039in}}%
\pgfusepath{clip}%
\pgfsetbuttcap%
\pgfsetroundjoin%
\definecolor{currentfill}{rgb}{0.283229,0.120777,0.440584}%
\pgfsetfillcolor{currentfill}%
\pgfsetfillopacity{0.700000}%
\pgfsetlinewidth{0.000000pt}%
\definecolor{currentstroke}{rgb}{0.000000,0.000000,0.000000}%
\pgfsetstrokecolor{currentstroke}%
\pgfsetdash{}{0pt}%
\pgfpathmoveto{\pgfqpoint{3.869062in}{1.928968in}}%
\pgfpathlineto{\pgfqpoint{3.882338in}{1.930364in}}%
\pgfpathlineto{\pgfqpoint{3.895622in}{1.931929in}}%
\pgfpathlineto{\pgfqpoint{3.908915in}{1.933662in}}%
\pgfpathlineto{\pgfqpoint{3.922216in}{1.935562in}}%
\pgfpathlineto{\pgfqpoint{3.929973in}{1.946354in}}%
\pgfpathlineto{\pgfqpoint{3.937726in}{1.957120in}}%
\pgfpathlineto{\pgfqpoint{3.945474in}{1.967858in}}%
\pgfpathlineto{\pgfqpoint{3.953218in}{1.978568in}}%
\pgfpathlineto{\pgfqpoint{3.939923in}{1.976479in}}%
\pgfpathlineto{\pgfqpoint{3.926638in}{1.974557in}}%
\pgfpathlineto{\pgfqpoint{3.913361in}{1.972803in}}%
\pgfpathlineto{\pgfqpoint{3.900092in}{1.971218in}}%
\pgfpathlineto{\pgfqpoint{3.892342in}{1.960687in}}%
\pgfpathlineto{\pgfqpoint{3.884587in}{1.950133in}}%
\pgfpathlineto{\pgfqpoint{3.876827in}{1.939560in}}%
\pgfpathlineto{\pgfqpoint{3.869062in}{1.928968in}}%
\pgfpathclose%
\pgfusepath{fill}%
\end{pgfscope}%
\begin{pgfscope}%
\pgfpathrectangle{\pgfqpoint{1.254980in}{0.150000in}}{\pgfqpoint{5.490039in}{5.490039in}}%
\pgfusepath{clip}%
\pgfsetbuttcap%
\pgfsetroundjoin%
\definecolor{currentfill}{rgb}{0.277018,0.050344,0.375715}%
\pgfsetfillcolor{currentfill}%
\pgfsetfillopacity{0.700000}%
\pgfsetlinewidth{0.000000pt}%
\definecolor{currentstroke}{rgb}{0.000000,0.000000,0.000000}%
\pgfsetstrokecolor{currentstroke}%
\pgfsetdash{}{0pt}%
\pgfpathmoveto{\pgfqpoint{3.616448in}{1.806414in}}%
\pgfpathlineto{\pgfqpoint{3.629668in}{1.805023in}}%
\pgfpathlineto{\pgfqpoint{3.642893in}{1.803806in}}%
\pgfpathlineto{\pgfqpoint{3.656124in}{1.802762in}}%
\pgfpathlineto{\pgfqpoint{3.669361in}{1.801891in}}%
\pgfpathlineto{\pgfqpoint{3.677205in}{1.811992in}}%
\pgfpathlineto{\pgfqpoint{3.685043in}{1.822112in}}%
\pgfpathlineto{\pgfqpoint{3.692876in}{1.832250in}}%
\pgfpathlineto{\pgfqpoint{3.700704in}{1.842403in}}%
\pgfpathlineto{\pgfqpoint{3.687477in}{1.843002in}}%
\pgfpathlineto{\pgfqpoint{3.674256in}{1.843774in}}%
\pgfpathlineto{\pgfqpoint{3.661042in}{1.844720in}}%
\pgfpathlineto{\pgfqpoint{3.647834in}{1.845839in}}%
\pgfpathlineto{\pgfqpoint{3.639996in}{1.835947in}}%
\pgfpathlineto{\pgfqpoint{3.632152in}{1.826077in}}%
\pgfpathlineto{\pgfqpoint{3.624303in}{1.816232in}}%
\pgfpathlineto{\pgfqpoint{3.616448in}{1.806414in}}%
\pgfpathclose%
\pgfusepath{fill}%
\end{pgfscope}%
\begin{pgfscope}%
\pgfpathrectangle{\pgfqpoint{1.254980in}{0.150000in}}{\pgfqpoint{5.490039in}{5.490039in}}%
\pgfusepath{clip}%
\pgfsetbuttcap%
\pgfsetroundjoin%
\definecolor{currentfill}{rgb}{0.282290,0.145912,0.461510}%
\pgfsetfillcolor{currentfill}%
\pgfsetfillopacity{0.700000}%
\pgfsetlinewidth{0.000000pt}%
\definecolor{currentstroke}{rgb}{0.000000,0.000000,0.000000}%
\pgfsetstrokecolor{currentstroke}%
\pgfsetdash{}{0pt}%
\pgfpathmoveto{\pgfqpoint{3.953218in}{1.978568in}}%
\pgfpathlineto{\pgfqpoint{3.966521in}{1.980825in}}%
\pgfpathlineto{\pgfqpoint{3.979833in}{1.983249in}}%
\pgfpathlineto{\pgfqpoint{3.993154in}{1.985840in}}%
\pgfpathlineto{\pgfqpoint{4.006485in}{1.988597in}}%
\pgfpathlineto{\pgfqpoint{4.014217in}{1.999449in}}%
\pgfpathlineto{\pgfqpoint{4.021945in}{2.010262in}}%
\pgfpathlineto{\pgfqpoint{4.029667in}{2.021035in}}%
\pgfpathlineto{\pgfqpoint{4.037385in}{2.031767in}}%
\pgfpathlineto{\pgfqpoint{4.024060in}{2.028849in}}%
\pgfpathlineto{\pgfqpoint{4.010745in}{2.026097in}}%
\pgfpathlineto{\pgfqpoint{3.997439in}{2.023512in}}%
\pgfpathlineto{\pgfqpoint{3.984142in}{2.021094in}}%
\pgfpathlineto{\pgfqpoint{3.976418in}{2.010512in}}%
\pgfpathlineto{\pgfqpoint{3.968689in}{1.999897in}}%
\pgfpathlineto{\pgfqpoint{3.960956in}{1.989248in}}%
\pgfpathlineto{\pgfqpoint{3.953218in}{1.978568in}}%
\pgfpathclose%
\pgfusepath{fill}%
\end{pgfscope}%
\begin{pgfscope}%
\pgfpathrectangle{\pgfqpoint{1.254980in}{0.150000in}}{\pgfqpoint{5.490039in}{5.490039in}}%
\pgfusepath{clip}%
\pgfsetbuttcap%
\pgfsetroundjoin%
\definecolor{currentfill}{rgb}{0.140536,0.530132,0.555659}%
\pgfsetfillcolor{currentfill}%
\pgfsetfillopacity{0.700000}%
\pgfsetlinewidth{0.000000pt}%
\definecolor{currentstroke}{rgb}{0.000000,0.000000,0.000000}%
\pgfsetstrokecolor{currentstroke}%
\pgfsetdash{}{0pt}%
\pgfpathmoveto{\pgfqpoint{5.139829in}{2.881712in}}%
\pgfpathlineto{\pgfqpoint{5.153699in}{2.891505in}}%
\pgfpathlineto{\pgfqpoint{5.167586in}{2.901455in}}%
\pgfpathlineto{\pgfqpoint{5.181490in}{2.911564in}}%
\pgfpathlineto{\pgfqpoint{5.195411in}{2.921831in}}%
\pgfpathlineto{\pgfqpoint{5.202653in}{2.926963in}}%
\pgfpathlineto{\pgfqpoint{5.209888in}{2.932030in}}%
\pgfpathlineto{\pgfqpoint{5.217116in}{2.937037in}}%
\pgfpathlineto{\pgfqpoint{5.224336in}{2.941987in}}%
\pgfpathlineto{\pgfqpoint{5.210431in}{2.932027in}}%
\pgfpathlineto{\pgfqpoint{5.196543in}{2.922224in}}%
\pgfpathlineto{\pgfqpoint{5.182672in}{2.912580in}}%
\pgfpathlineto{\pgfqpoint{5.168817in}{2.903093in}}%
\pgfpathlineto{\pgfqpoint{5.161581in}{2.897826in}}%
\pgfpathlineto{\pgfqpoint{5.154337in}{2.892510in}}%
\pgfpathlineto{\pgfqpoint{5.147087in}{2.887140in}}%
\pgfpathlineto{\pgfqpoint{5.139829in}{2.881712in}}%
\pgfpathclose%
\pgfusepath{fill}%
\end{pgfscope}%
\begin{pgfscope}%
\pgfpathrectangle{\pgfqpoint{1.254980in}{0.150000in}}{\pgfqpoint{5.490039in}{5.490039in}}%
\pgfusepath{clip}%
\pgfsetbuttcap%
\pgfsetroundjoin%
\definecolor{currentfill}{rgb}{0.278791,0.062145,0.386592}%
\pgfsetfillcolor{currentfill}%
\pgfsetfillopacity{0.700000}%
\pgfsetlinewidth{0.000000pt}%
\definecolor{currentstroke}{rgb}{0.000000,0.000000,0.000000}%
\pgfsetstrokecolor{currentstroke}%
\pgfsetdash{}{0pt}%
\pgfpathmoveto{\pgfqpoint{2.981398in}{1.856030in}}%
\pgfpathlineto{\pgfqpoint{2.994632in}{1.846087in}}%
\pgfpathlineto{\pgfqpoint{3.007863in}{1.836349in}}%
\pgfpathlineto{\pgfqpoint{3.021094in}{1.826816in}}%
\pgfpathlineto{\pgfqpoint{3.034323in}{1.817486in}}%
\pgfpathlineto{\pgfqpoint{3.042475in}{1.822755in}}%
\pgfpathlineto{\pgfqpoint{3.050617in}{1.828172in}}%
\pgfpathlineto{\pgfqpoint{3.058749in}{1.833734in}}%
\pgfpathlineto{\pgfqpoint{3.066871in}{1.839435in}}%
\pgfpathlineto{\pgfqpoint{3.053668in}{1.848349in}}%
\pgfpathlineto{\pgfqpoint{3.040464in}{1.857467in}}%
\pgfpathlineto{\pgfqpoint{3.027260in}{1.866788in}}%
\pgfpathlineto{\pgfqpoint{3.014054in}{1.876315in}}%
\pgfpathlineto{\pgfqpoint{3.005906in}{1.871019in}}%
\pgfpathlineto{\pgfqpoint{2.997747in}{1.865870in}}%
\pgfpathlineto{\pgfqpoint{2.989578in}{1.860872in}}%
\pgfpathlineto{\pgfqpoint{2.981398in}{1.856030in}}%
\pgfpathclose%
\pgfusepath{fill}%
\end{pgfscope}%
\begin{pgfscope}%
\pgfpathrectangle{\pgfqpoint{1.254980in}{0.150000in}}{\pgfqpoint{5.490039in}{5.490039in}}%
\pgfusepath{clip}%
\pgfsetbuttcap%
\pgfsetroundjoin%
\definecolor{currentfill}{rgb}{0.179019,0.433756,0.557430}%
\pgfsetfillcolor{currentfill}%
\pgfsetfillopacity{0.700000}%
\pgfsetlinewidth{0.000000pt}%
\definecolor{currentstroke}{rgb}{0.000000,0.000000,0.000000}%
\pgfsetstrokecolor{currentstroke}%
\pgfsetdash{}{0pt}%
\pgfpathmoveto{\pgfqpoint{2.285796in}{2.696619in}}%
\pgfpathlineto{\pgfqpoint{2.299422in}{2.673555in}}%
\pgfpathlineto{\pgfqpoint{2.313034in}{2.650804in}}%
\pgfpathlineto{\pgfqpoint{2.326631in}{2.628363in}}%
\pgfpathlineto{\pgfqpoint{2.340214in}{2.606230in}}%
\pgfpathlineto{\pgfqpoint{2.348842in}{2.605778in}}%
\pgfpathlineto{\pgfqpoint{2.357452in}{2.605574in}}%
\pgfpathlineto{\pgfqpoint{2.366045in}{2.605611in}}%
\pgfpathlineto{\pgfqpoint{2.374622in}{2.605887in}}%
\pgfpathlineto{\pgfqpoint{2.361085in}{2.627571in}}%
\pgfpathlineto{\pgfqpoint{2.347535in}{2.649561in}}%
\pgfpathlineto{\pgfqpoint{2.333971in}{2.671860in}}%
\pgfpathlineto{\pgfqpoint{2.320392in}{2.694472in}}%
\pgfpathlineto{\pgfqpoint{2.311770in}{2.694636in}}%
\pgfpathlineto{\pgfqpoint{2.303130in}{2.695045in}}%
\pgfpathlineto{\pgfqpoint{2.294472in}{2.695705in}}%
\pgfpathlineto{\pgfqpoint{2.285796in}{2.696619in}}%
\pgfpathclose%
\pgfusepath{fill}%
\end{pgfscope}%
\begin{pgfscope}%
\pgfpathrectangle{\pgfqpoint{1.254980in}{0.150000in}}{\pgfqpoint{5.490039in}{5.490039in}}%
\pgfusepath{clip}%
\pgfsetbuttcap%
\pgfsetroundjoin%
\definecolor{currentfill}{rgb}{0.235526,0.309527,0.542944}%
\pgfsetfillcolor{currentfill}%
\pgfsetfillopacity{0.700000}%
\pgfsetlinewidth{0.000000pt}%
\definecolor{currentstroke}{rgb}{0.000000,0.000000,0.000000}%
\pgfsetstrokecolor{currentstroke}%
\pgfsetdash{}{0pt}%
\pgfpathmoveto{\pgfqpoint{4.404901in}{2.311807in}}%
\pgfpathlineto{\pgfqpoint{4.418388in}{2.317898in}}%
\pgfpathlineto{\pgfqpoint{4.431887in}{2.324150in}}%
\pgfpathlineto{\pgfqpoint{4.445399in}{2.330565in}}%
\pgfpathlineto{\pgfqpoint{4.458924in}{2.337142in}}%
\pgfpathlineto{\pgfqpoint{4.466509in}{2.346872in}}%
\pgfpathlineto{\pgfqpoint{4.474088in}{2.356517in}}%
\pgfpathlineto{\pgfqpoint{4.481661in}{2.366078in}}%
\pgfpathlineto{\pgfqpoint{4.489228in}{2.375555in}}%
\pgfpathlineto{\pgfqpoint{4.475708in}{2.368988in}}%
\pgfpathlineto{\pgfqpoint{4.462201in}{2.362582in}}%
\pgfpathlineto{\pgfqpoint{4.448707in}{2.356339in}}%
\pgfpathlineto{\pgfqpoint{4.435225in}{2.350258in}}%
\pgfpathlineto{\pgfqpoint{4.427652in}{2.340760in}}%
\pgfpathlineto{\pgfqpoint{4.420074in}{2.331187in}}%
\pgfpathlineto{\pgfqpoint{4.412490in}{2.321536in}}%
\pgfpathlineto{\pgfqpoint{4.404901in}{2.311807in}}%
\pgfpathclose%
\pgfusepath{fill}%
\end{pgfscope}%
\begin{pgfscope}%
\pgfpathrectangle{\pgfqpoint{1.254980in}{0.150000in}}{\pgfqpoint{5.490039in}{5.490039in}}%
\pgfusepath{clip}%
\pgfsetbuttcap%
\pgfsetroundjoin%
\definecolor{currentfill}{rgb}{0.271305,0.019942,0.347269}%
\pgfsetfillcolor{currentfill}%
\pgfsetfillopacity{0.700000}%
\pgfsetlinewidth{0.000000pt}%
\definecolor{currentstroke}{rgb}{0.000000,0.000000,0.000000}%
\pgfsetstrokecolor{currentstroke}%
\pgfsetdash{}{0pt}%
\pgfpathmoveto{\pgfqpoint{3.172486in}{1.775290in}}%
\pgfpathlineto{\pgfqpoint{3.185689in}{1.768153in}}%
\pgfpathlineto{\pgfqpoint{3.198894in}{1.761207in}}%
\pgfpathlineto{\pgfqpoint{3.212100in}{1.754452in}}%
\pgfpathlineto{\pgfqpoint{3.225306in}{1.747888in}}%
\pgfpathlineto{\pgfqpoint{3.233349in}{1.754914in}}%
\pgfpathlineto{\pgfqpoint{3.241384in}{1.762050in}}%
\pgfpathlineto{\pgfqpoint{3.249411in}{1.769295in}}%
\pgfpathlineto{\pgfqpoint{3.257429in}{1.776643in}}%
\pgfpathlineto{\pgfqpoint{3.244243in}{1.782823in}}%
\pgfpathlineto{\pgfqpoint{3.231059in}{1.789193in}}%
\pgfpathlineto{\pgfqpoint{3.217876in}{1.795754in}}%
\pgfpathlineto{\pgfqpoint{3.204694in}{1.802507in}}%
\pgfpathlineto{\pgfqpoint{3.196655in}{1.795533in}}%
\pgfpathlineto{\pgfqpoint{3.188607in}{1.788670in}}%
\pgfpathlineto{\pgfqpoint{3.180551in}{1.781921in}}%
\pgfpathlineto{\pgfqpoint{3.172486in}{1.775290in}}%
\pgfpathclose%
\pgfusepath{fill}%
\end{pgfscope}%
\begin{pgfscope}%
\pgfpathrectangle{\pgfqpoint{1.254980in}{0.150000in}}{\pgfqpoint{5.490039in}{5.490039in}}%
\pgfusepath{clip}%
\pgfsetbuttcap%
\pgfsetroundjoin%
\definecolor{currentfill}{rgb}{0.269944,0.014625,0.341379}%
\pgfsetfillcolor{currentfill}%
\pgfsetfillopacity{0.700000}%
\pgfsetlinewidth{0.000000pt}%
\definecolor{currentstroke}{rgb}{0.000000,0.000000,0.000000}%
\pgfsetstrokecolor{currentstroke}%
\pgfsetdash{}{0pt}%
\pgfpathmoveto{\pgfqpoint{3.310193in}{1.753806in}}%
\pgfpathlineto{\pgfqpoint{3.323389in}{1.748563in}}%
\pgfpathlineto{\pgfqpoint{3.336588in}{1.743506in}}%
\pgfpathlineto{\pgfqpoint{3.349790in}{1.738632in}}%
\pgfpathlineto{\pgfqpoint{3.362995in}{1.733941in}}%
\pgfpathlineto{\pgfqpoint{3.370968in}{1.742122in}}%
\pgfpathlineto{\pgfqpoint{3.378934in}{1.750386in}}%
\pgfpathlineto{\pgfqpoint{3.386893in}{1.758730in}}%
\pgfpathlineto{\pgfqpoint{3.394846in}{1.767150in}}%
\pgfpathlineto{\pgfqpoint{3.381658in}{1.771485in}}%
\pgfpathlineto{\pgfqpoint{3.368474in}{1.776003in}}%
\pgfpathlineto{\pgfqpoint{3.355292in}{1.780706in}}%
\pgfpathlineto{\pgfqpoint{3.342114in}{1.785593in}}%
\pgfpathlineto{\pgfqpoint{3.334144in}{1.777518in}}%
\pgfpathlineto{\pgfqpoint{3.326168in}{1.769526in}}%
\pgfpathlineto{\pgfqpoint{3.318184in}{1.761621in}}%
\pgfpathlineto{\pgfqpoint{3.310193in}{1.753806in}}%
\pgfpathclose%
\pgfusepath{fill}%
\end{pgfscope}%
\begin{pgfscope}%
\pgfpathrectangle{\pgfqpoint{1.254980in}{0.150000in}}{\pgfqpoint{5.490039in}{5.490039in}}%
\pgfusepath{clip}%
\pgfsetbuttcap%
\pgfsetroundjoin%
\definecolor{currentfill}{rgb}{0.278826,0.175490,0.483397}%
\pgfsetfillcolor{currentfill}%
\pgfsetfillopacity{0.700000}%
\pgfsetlinewidth{0.000000pt}%
\definecolor{currentstroke}{rgb}{0.000000,0.000000,0.000000}%
\pgfsetstrokecolor{currentstroke}%
\pgfsetdash{}{0pt}%
\pgfpathmoveto{\pgfqpoint{4.037385in}{2.031767in}}%
\pgfpathlineto{\pgfqpoint{4.050720in}{2.034851in}}%
\pgfpathlineto{\pgfqpoint{4.064064in}{2.038101in}}%
\pgfpathlineto{\pgfqpoint{4.077418in}{2.041517in}}%
\pgfpathlineto{\pgfqpoint{4.090782in}{2.045098in}}%
\pgfpathlineto{\pgfqpoint{4.098490in}{2.055931in}}%
\pgfpathlineto{\pgfqpoint{4.106192in}{2.066715in}}%
\pgfpathlineto{\pgfqpoint{4.113890in}{2.077447in}}%
\pgfpathlineto{\pgfqpoint{4.121583in}{2.088127in}}%
\pgfpathlineto{\pgfqpoint{4.108224in}{2.084413in}}%
\pgfpathlineto{\pgfqpoint{4.094875in}{2.080865in}}%
\pgfpathlineto{\pgfqpoint{4.081536in}{2.077481in}}%
\pgfpathlineto{\pgfqpoint{4.068207in}{2.074264in}}%
\pgfpathlineto{\pgfqpoint{4.060509in}{2.063707in}}%
\pgfpathlineto{\pgfqpoint{4.052806in}{2.053104in}}%
\pgfpathlineto{\pgfqpoint{4.045098in}{2.042457in}}%
\pgfpathlineto{\pgfqpoint{4.037385in}{2.031767in}}%
\pgfpathclose%
\pgfusepath{fill}%
\end{pgfscope}%
\begin{pgfscope}%
\pgfpathrectangle{\pgfqpoint{1.254980in}{0.150000in}}{\pgfqpoint{5.490039in}{5.490039in}}%
\pgfusepath{clip}%
\pgfsetbuttcap%
\pgfsetroundjoin%
\definecolor{currentfill}{rgb}{0.273809,0.031497,0.358853}%
\pgfsetfillcolor{currentfill}%
\pgfsetfillopacity{0.700000}%
\pgfsetlinewidth{0.000000pt}%
\definecolor{currentstroke}{rgb}{0.000000,0.000000,0.000000}%
\pgfsetstrokecolor{currentstroke}%
\pgfsetdash{}{0pt}%
\pgfpathmoveto{\pgfqpoint{3.532103in}{1.775973in}}%
\pgfpathlineto{\pgfqpoint{3.545313in}{1.773581in}}%
\pgfpathlineto{\pgfqpoint{3.558528in}{1.771366in}}%
\pgfpathlineto{\pgfqpoint{3.571748in}{1.769326in}}%
\pgfpathlineto{\pgfqpoint{3.584973in}{1.767461in}}%
\pgfpathlineto{\pgfqpoint{3.592851in}{1.777146in}}%
\pgfpathlineto{\pgfqpoint{3.600722in}{1.786868in}}%
\pgfpathlineto{\pgfqpoint{3.608588in}{1.796625in}}%
\pgfpathlineto{\pgfqpoint{3.616448in}{1.806414in}}%
\pgfpathlineto{\pgfqpoint{3.603235in}{1.807979in}}%
\pgfpathlineto{\pgfqpoint{3.590027in}{1.809720in}}%
\pgfpathlineto{\pgfqpoint{3.576825in}{1.811636in}}%
\pgfpathlineto{\pgfqpoint{3.563628in}{1.813729in}}%
\pgfpathlineto{\pgfqpoint{3.555755in}{1.804229in}}%
\pgfpathlineto{\pgfqpoint{3.547877in}{1.794768in}}%
\pgfpathlineto{\pgfqpoint{3.539993in}{1.785348in}}%
\pgfpathlineto{\pgfqpoint{3.532103in}{1.775973in}}%
\pgfpathclose%
\pgfusepath{fill}%
\end{pgfscope}%
\begin{pgfscope}%
\pgfpathrectangle{\pgfqpoint{1.254980in}{0.150000in}}{\pgfqpoint{5.490039in}{5.490039in}}%
\pgfusepath{clip}%
\pgfsetbuttcap%
\pgfsetroundjoin%
\definecolor{currentfill}{rgb}{0.180629,0.429975,0.557282}%
\pgfsetfillcolor{currentfill}%
\pgfsetfillopacity{0.700000}%
\pgfsetlinewidth{0.000000pt}%
\definecolor{currentstroke}{rgb}{0.000000,0.000000,0.000000}%
\pgfsetstrokecolor{currentstroke}%
\pgfsetdash{}{0pt}%
\pgfpathmoveto{\pgfqpoint{4.772479in}{2.603985in}}%
\pgfpathlineto{\pgfqpoint{4.786153in}{2.612314in}}%
\pgfpathlineto{\pgfqpoint{4.799841in}{2.620803in}}%
\pgfpathlineto{\pgfqpoint{4.813545in}{2.629452in}}%
\pgfpathlineto{\pgfqpoint{4.827264in}{2.638260in}}%
\pgfpathlineto{\pgfqpoint{4.834697in}{2.645895in}}%
\pgfpathlineto{\pgfqpoint{4.842124in}{2.653439in}}%
\pgfpathlineto{\pgfqpoint{4.849543in}{2.660896in}}%
\pgfpathlineto{\pgfqpoint{4.856956in}{2.668268in}}%
\pgfpathlineto{\pgfqpoint{4.843247in}{2.659616in}}%
\pgfpathlineto{\pgfqpoint{4.829552in}{2.651124in}}%
\pgfpathlineto{\pgfqpoint{4.815872in}{2.642791in}}%
\pgfpathlineto{\pgfqpoint{4.802208in}{2.634618in}}%
\pgfpathlineto{\pgfqpoint{4.794785in}{2.627080in}}%
\pgfpathlineto{\pgfqpoint{4.787357in}{2.619463in}}%
\pgfpathlineto{\pgfqpoint{4.779921in}{2.611765in}}%
\pgfpathlineto{\pgfqpoint{4.772479in}{2.603985in}}%
\pgfpathclose%
\pgfusepath{fill}%
\end{pgfscope}%
\begin{pgfscope}%
\pgfpathrectangle{\pgfqpoint{1.254980in}{0.150000in}}{\pgfqpoint{5.490039in}{5.490039in}}%
\pgfusepath{clip}%
\pgfsetbuttcap%
\pgfsetroundjoin%
\definecolor{currentfill}{rgb}{0.131172,0.555899,0.552459}%
\pgfsetfillcolor{currentfill}%
\pgfsetfillopacity{0.700000}%
\pgfsetlinewidth{0.000000pt}%
\definecolor{currentstroke}{rgb}{0.000000,0.000000,0.000000}%
\pgfsetstrokecolor{currentstroke}%
\pgfsetdash{}{0pt}%
\pgfpathmoveto{\pgfqpoint{5.224336in}{2.941987in}}%
\pgfpathlineto{\pgfqpoint{5.238259in}{2.952104in}}%
\pgfpathlineto{\pgfqpoint{5.252198in}{2.962380in}}%
\pgfpathlineto{\pgfqpoint{5.266155in}{2.972814in}}%
\pgfpathlineto{\pgfqpoint{5.280130in}{2.983405in}}%
\pgfpathlineto{\pgfqpoint{5.287326in}{2.987975in}}%
\pgfpathlineto{\pgfqpoint{5.294515in}{2.992488in}}%
\pgfpathlineto{\pgfqpoint{5.301697in}{2.996948in}}%
\pgfpathlineto{\pgfqpoint{5.308871in}{3.001359in}}%
\pgfpathlineto{\pgfqpoint{5.294914in}{2.991105in}}%
\pgfpathlineto{\pgfqpoint{5.280975in}{2.981009in}}%
\pgfpathlineto{\pgfqpoint{5.267052in}{2.971070in}}%
\pgfpathlineto{\pgfqpoint{5.253147in}{2.961288in}}%
\pgfpathlineto{\pgfqpoint{5.245955in}{2.956530in}}%
\pgfpathlineto{\pgfqpoint{5.238756in}{2.951729in}}%
\pgfpathlineto{\pgfqpoint{5.231550in}{2.946883in}}%
\pgfpathlineto{\pgfqpoint{5.224336in}{2.941987in}}%
\pgfpathclose%
\pgfusepath{fill}%
\end{pgfscope}%
\begin{pgfscope}%
\pgfpathrectangle{\pgfqpoint{1.254980in}{0.150000in}}{\pgfqpoint{5.490039in}{5.490039in}}%
\pgfusepath{clip}%
\pgfsetbuttcap%
\pgfsetroundjoin%
\definecolor{currentfill}{rgb}{0.271828,0.209303,0.504434}%
\pgfsetfillcolor{currentfill}%
\pgfsetfillopacity{0.700000}%
\pgfsetlinewidth{0.000000pt}%
\definecolor{currentstroke}{rgb}{0.000000,0.000000,0.000000}%
\pgfsetstrokecolor{currentstroke}%
\pgfsetdash{}{0pt}%
\pgfpathmoveto{\pgfqpoint{2.629179in}{2.153125in}}%
\pgfpathlineto{\pgfqpoint{2.642549in}{2.137350in}}%
\pgfpathlineto{\pgfqpoint{2.655911in}{2.121819in}}%
\pgfpathlineto{\pgfqpoint{2.669267in}{2.106531in}}%
\pgfpathlineto{\pgfqpoint{2.682616in}{2.091482in}}%
\pgfpathlineto{\pgfqpoint{2.691010in}{2.093354in}}%
\pgfpathlineto{\pgfqpoint{2.699389in}{2.095435in}}%
\pgfpathlineto{\pgfqpoint{2.707755in}{2.097721in}}%
\pgfpathlineto{\pgfqpoint{2.716107in}{2.100208in}}%
\pgfpathlineto{\pgfqpoint{2.702795in}{2.114801in}}%
\pgfpathlineto{\pgfqpoint{2.689477in}{2.129632in}}%
\pgfpathlineto{\pgfqpoint{2.676152in}{2.144706in}}%
\pgfpathlineto{\pgfqpoint{2.662821in}{2.160022in}}%
\pgfpathlineto{\pgfqpoint{2.654432in}{2.157981in}}%
\pgfpathlineto{\pgfqpoint{2.646029in}{2.156148in}}%
\pgfpathlineto{\pgfqpoint{2.637611in}{2.154528in}}%
\pgfpathlineto{\pgfqpoint{2.629179in}{2.153125in}}%
\pgfpathclose%
\pgfusepath{fill}%
\end{pgfscope}%
\begin{pgfscope}%
\pgfpathrectangle{\pgfqpoint{1.254980in}{0.150000in}}{\pgfqpoint{5.490039in}{5.490039in}}%
\pgfusepath{clip}%
\pgfsetbuttcap%
\pgfsetroundjoin%
\definecolor{currentfill}{rgb}{0.262138,0.242286,0.520837}%
\pgfsetfillcolor{currentfill}%
\pgfsetfillopacity{0.700000}%
\pgfsetlinewidth{0.000000pt}%
\definecolor{currentstroke}{rgb}{0.000000,0.000000,0.000000}%
\pgfsetstrokecolor{currentstroke}%
\pgfsetdash{}{0pt}%
\pgfpathmoveto{\pgfqpoint{2.575626in}{2.218703in}}%
\pgfpathlineto{\pgfqpoint{2.589026in}{2.201932in}}%
\pgfpathlineto{\pgfqpoint{2.602418in}{2.185414in}}%
\pgfpathlineto{\pgfqpoint{2.615802in}{2.169145in}}%
\pgfpathlineto{\pgfqpoint{2.629179in}{2.153125in}}%
\pgfpathlineto{\pgfqpoint{2.637611in}{2.154528in}}%
\pgfpathlineto{\pgfqpoint{2.646029in}{2.156148in}}%
\pgfpathlineto{\pgfqpoint{2.654432in}{2.157981in}}%
\pgfpathlineto{\pgfqpoint{2.662821in}{2.160022in}}%
\pgfpathlineto{\pgfqpoint{2.649483in}{2.175584in}}%
\pgfpathlineto{\pgfqpoint{2.636138in}{2.191392in}}%
\pgfpathlineto{\pgfqpoint{2.622786in}{2.207450in}}%
\pgfpathlineto{\pgfqpoint{2.609425in}{2.223759in}}%
\pgfpathlineto{\pgfqpoint{2.600998in}{2.222167in}}%
\pgfpathlineto{\pgfqpoint{2.592555in}{2.220790in}}%
\pgfpathlineto{\pgfqpoint{2.584098in}{2.219634in}}%
\pgfpathlineto{\pgfqpoint{2.575626in}{2.218703in}}%
\pgfpathclose%
\pgfusepath{fill}%
\end{pgfscope}%
\begin{pgfscope}%
\pgfpathrectangle{\pgfqpoint{1.254980in}{0.150000in}}{\pgfqpoint{5.490039in}{5.490039in}}%
\pgfusepath{clip}%
\pgfsetbuttcap%
\pgfsetroundjoin%
\definecolor{currentfill}{rgb}{0.278012,0.180367,0.486697}%
\pgfsetfillcolor{currentfill}%
\pgfsetfillopacity{0.700000}%
\pgfsetlinewidth{0.000000pt}%
\definecolor{currentstroke}{rgb}{0.000000,0.000000,0.000000}%
\pgfsetstrokecolor{currentstroke}%
\pgfsetdash{}{0pt}%
\pgfpathmoveto{\pgfqpoint{2.682616in}{2.091482in}}%
\pgfpathlineto{\pgfqpoint{2.695959in}{2.076672in}}%
\pgfpathlineto{\pgfqpoint{2.709296in}{2.062098in}}%
\pgfpathlineto{\pgfqpoint{2.722627in}{2.047759in}}%
\pgfpathlineto{\pgfqpoint{2.735952in}{2.033652in}}%
\pgfpathlineto{\pgfqpoint{2.744309in}{2.035990in}}%
\pgfpathlineto{\pgfqpoint{2.752652in}{2.038529in}}%
\pgfpathlineto{\pgfqpoint{2.760982in}{2.041266in}}%
\pgfpathlineto{\pgfqpoint{2.769298in}{2.044197in}}%
\pgfpathlineto{\pgfqpoint{2.756009in}{2.057849in}}%
\pgfpathlineto{\pgfqpoint{2.742714in}{2.071734in}}%
\pgfpathlineto{\pgfqpoint{2.729413in}{2.085853in}}%
\pgfpathlineto{\pgfqpoint{2.716107in}{2.100208in}}%
\pgfpathlineto{\pgfqpoint{2.707755in}{2.097721in}}%
\pgfpathlineto{\pgfqpoint{2.699389in}{2.095435in}}%
\pgfpathlineto{\pgfqpoint{2.691010in}{2.093354in}}%
\pgfpathlineto{\pgfqpoint{2.682616in}{2.091482in}}%
\pgfpathclose%
\pgfusepath{fill}%
\end{pgfscope}%
\begin{pgfscope}%
\pgfpathrectangle{\pgfqpoint{1.254980in}{0.150000in}}{\pgfqpoint{5.490039in}{5.490039in}}%
\pgfusepath{clip}%
\pgfsetbuttcap%
\pgfsetroundjoin%
\definecolor{currentfill}{rgb}{0.271828,0.209303,0.504434}%
\pgfsetfillcolor{currentfill}%
\pgfsetfillopacity{0.700000}%
\pgfsetlinewidth{0.000000pt}%
\definecolor{currentstroke}{rgb}{0.000000,0.000000,0.000000}%
\pgfsetstrokecolor{currentstroke}%
\pgfsetdash{}{0pt}%
\pgfpathmoveto{\pgfqpoint{4.121583in}{2.088127in}}%
\pgfpathlineto{\pgfqpoint{4.134952in}{2.092006in}}%
\pgfpathlineto{\pgfqpoint{4.148331in}{2.096050in}}%
\pgfpathlineto{\pgfqpoint{4.161722in}{2.100258in}}%
\pgfpathlineto{\pgfqpoint{4.175123in}{2.104631in}}%
\pgfpathlineto{\pgfqpoint{4.182806in}{2.115373in}}%
\pgfpathlineto{\pgfqpoint{4.190484in}{2.126055in}}%
\pgfpathlineto{\pgfqpoint{4.198157in}{2.136675in}}%
\pgfpathlineto{\pgfqpoint{4.205825in}{2.147233in}}%
\pgfpathlineto{\pgfqpoint{4.192428in}{2.142756in}}%
\pgfpathlineto{\pgfqpoint{4.179043in}{2.138442in}}%
\pgfpathlineto{\pgfqpoint{4.165668in}{2.134294in}}%
\pgfpathlineto{\pgfqpoint{4.152304in}{2.130310in}}%
\pgfpathlineto{\pgfqpoint{4.144631in}{2.119846in}}%
\pgfpathlineto{\pgfqpoint{4.136953in}{2.109328in}}%
\pgfpathlineto{\pgfqpoint{4.129270in}{2.098754in}}%
\pgfpathlineto{\pgfqpoint{4.121583in}{2.088127in}}%
\pgfpathclose%
\pgfusepath{fill}%
\end{pgfscope}%
\begin{pgfscope}%
\pgfpathrectangle{\pgfqpoint{1.254980in}{0.150000in}}{\pgfqpoint{5.490039in}{5.490039in}}%
\pgfusepath{clip}%
\pgfsetbuttcap%
\pgfsetroundjoin%
\definecolor{currentfill}{rgb}{0.250425,0.274290,0.533103}%
\pgfsetfillcolor{currentfill}%
\pgfsetfillopacity{0.700000}%
\pgfsetlinewidth{0.000000pt}%
\definecolor{currentstroke}{rgb}{0.000000,0.000000,0.000000}%
\pgfsetstrokecolor{currentstroke}%
\pgfsetdash{}{0pt}%
\pgfpathmoveto{\pgfqpoint{2.521940in}{2.288347in}}%
\pgfpathlineto{\pgfqpoint{2.535375in}{2.270547in}}%
\pgfpathlineto{\pgfqpoint{2.548800in}{2.253008in}}%
\pgfpathlineto{\pgfqpoint{2.562217in}{2.235727in}}%
\pgfpathlineto{\pgfqpoint{2.575626in}{2.218703in}}%
\pgfpathlineto{\pgfqpoint{2.584098in}{2.219634in}}%
\pgfpathlineto{\pgfqpoint{2.592555in}{2.220790in}}%
\pgfpathlineto{\pgfqpoint{2.600998in}{2.222167in}}%
\pgfpathlineto{\pgfqpoint{2.609425in}{2.223759in}}%
\pgfpathlineto{\pgfqpoint{2.596058in}{2.240322in}}%
\pgfpathlineto{\pgfqpoint{2.582681in}{2.257140in}}%
\pgfpathlineto{\pgfqpoint{2.569297in}{2.274216in}}%
\pgfpathlineto{\pgfqpoint{2.555904in}{2.291551in}}%
\pgfpathlineto{\pgfqpoint{2.547436in}{2.290410in}}%
\pgfpathlineto{\pgfqpoint{2.538953in}{2.289493in}}%
\pgfpathlineto{\pgfqpoint{2.530454in}{2.288804in}}%
\pgfpathlineto{\pgfqpoint{2.521940in}{2.288347in}}%
\pgfpathclose%
\pgfusepath{fill}%
\end{pgfscope}%
\begin{pgfscope}%
\pgfpathrectangle{\pgfqpoint{1.254980in}{0.150000in}}{\pgfqpoint{5.490039in}{5.490039in}}%
\pgfusepath{clip}%
\pgfsetbuttcap%
\pgfsetroundjoin%
\definecolor{currentfill}{rgb}{0.221989,0.339161,0.548752}%
\pgfsetfillcolor{currentfill}%
\pgfsetfillopacity{0.700000}%
\pgfsetlinewidth{0.000000pt}%
\definecolor{currentstroke}{rgb}{0.000000,0.000000,0.000000}%
\pgfsetstrokecolor{currentstroke}%
\pgfsetdash{}{0pt}%
\pgfpathmoveto{\pgfqpoint{4.489228in}{2.375555in}}%
\pgfpathlineto{\pgfqpoint{4.502761in}{2.382284in}}%
\pgfpathlineto{\pgfqpoint{4.516307in}{2.389175in}}%
\pgfpathlineto{\pgfqpoint{4.529866in}{2.396227in}}%
\pgfpathlineto{\pgfqpoint{4.543439in}{2.403441in}}%
\pgfpathlineto{\pgfqpoint{4.550995in}{2.412807in}}%
\pgfpathlineto{\pgfqpoint{4.558545in}{2.422084in}}%
\pgfpathlineto{\pgfqpoint{4.566089in}{2.431272in}}%
\pgfpathlineto{\pgfqpoint{4.573627in}{2.440371in}}%
\pgfpathlineto{\pgfqpoint{4.560060in}{2.433196in}}%
\pgfpathlineto{\pgfqpoint{4.546506in}{2.426183in}}%
\pgfpathlineto{\pgfqpoint{4.532966in}{2.419331in}}%
\pgfpathlineto{\pgfqpoint{4.519439in}{2.412640in}}%
\pgfpathlineto{\pgfqpoint{4.511895in}{2.403491in}}%
\pgfpathlineto{\pgfqpoint{4.504345in}{2.394261in}}%
\pgfpathlineto{\pgfqpoint{4.496789in}{2.384949in}}%
\pgfpathlineto{\pgfqpoint{4.489228in}{2.375555in}}%
\pgfpathclose%
\pgfusepath{fill}%
\end{pgfscope}%
\begin{pgfscope}%
\pgfpathrectangle{\pgfqpoint{1.254980in}{0.150000in}}{\pgfqpoint{5.490039in}{5.490039in}}%
\pgfusepath{clip}%
\pgfsetbuttcap%
\pgfsetroundjoin%
\definecolor{currentfill}{rgb}{0.281412,0.155834,0.469201}%
\pgfsetfillcolor{currentfill}%
\pgfsetfillopacity{0.700000}%
\pgfsetlinewidth{0.000000pt}%
\definecolor{currentstroke}{rgb}{0.000000,0.000000,0.000000}%
\pgfsetstrokecolor{currentstroke}%
\pgfsetdash{}{0pt}%
\pgfpathmoveto{\pgfqpoint{2.735952in}{2.033652in}}%
\pgfpathlineto{\pgfqpoint{2.749272in}{2.019777in}}%
\pgfpathlineto{\pgfqpoint{2.762587in}{2.006131in}}%
\pgfpathlineto{\pgfqpoint{2.775897in}{1.992713in}}%
\pgfpathlineto{\pgfqpoint{2.789202in}{1.979521in}}%
\pgfpathlineto{\pgfqpoint{2.797523in}{1.982322in}}%
\pgfpathlineto{\pgfqpoint{2.805831in}{1.985318in}}%
\pgfpathlineto{\pgfqpoint{2.814126in}{1.988503in}}%
\pgfpathlineto{\pgfqpoint{2.822409in}{1.991875in}}%
\pgfpathlineto{\pgfqpoint{2.809138in}{2.004615in}}%
\pgfpathlineto{\pgfqpoint{2.795863in}{2.017581in}}%
\pgfpathlineto{\pgfqpoint{2.782583in}{2.030775in}}%
\pgfpathlineto{\pgfqpoint{2.769298in}{2.044197in}}%
\pgfpathlineto{\pgfqpoint{2.760982in}{2.041266in}}%
\pgfpathlineto{\pgfqpoint{2.752652in}{2.038529in}}%
\pgfpathlineto{\pgfqpoint{2.744309in}{2.035990in}}%
\pgfpathlineto{\pgfqpoint{2.735952in}{2.033652in}}%
\pgfpathclose%
\pgfusepath{fill}%
\end{pgfscope}%
\begin{pgfscope}%
\pgfpathrectangle{\pgfqpoint{1.254980in}{0.150000in}}{\pgfqpoint{5.490039in}{5.490039in}}%
\pgfusepath{clip}%
\pgfsetbuttcap%
\pgfsetroundjoin%
\definecolor{currentfill}{rgb}{0.277018,0.050344,0.375715}%
\pgfsetfillcolor{currentfill}%
\pgfsetfillopacity{0.700000}%
\pgfsetlinewidth{0.000000pt}%
\definecolor{currentstroke}{rgb}{0.000000,0.000000,0.000000}%
\pgfsetstrokecolor{currentstroke}%
\pgfsetdash{}{0pt}%
\pgfpathmoveto{\pgfqpoint{3.034323in}{1.817486in}}%
\pgfpathlineto{\pgfqpoint{3.047551in}{1.808359in}}%
\pgfpathlineto{\pgfqpoint{3.060779in}{1.799432in}}%
\pgfpathlineto{\pgfqpoint{3.074006in}{1.790705in}}%
\pgfpathlineto{\pgfqpoint{3.087232in}{1.782176in}}%
\pgfpathlineto{\pgfqpoint{3.095358in}{1.787871in}}%
\pgfpathlineto{\pgfqpoint{3.103474in}{1.793706in}}%
\pgfpathlineto{\pgfqpoint{3.111580in}{1.799679in}}%
\pgfpathlineto{\pgfqpoint{3.119678in}{1.805784in}}%
\pgfpathlineto{\pgfqpoint{3.106476in}{1.813898in}}%
\pgfpathlineto{\pgfqpoint{3.093275in}{1.822210in}}%
\pgfpathlineto{\pgfqpoint{3.080073in}{1.830722in}}%
\pgfpathlineto{\pgfqpoint{3.066871in}{1.839435in}}%
\pgfpathlineto{\pgfqpoint{3.058749in}{1.833734in}}%
\pgfpathlineto{\pgfqpoint{3.050617in}{1.828172in}}%
\pgfpathlineto{\pgfqpoint{3.042475in}{1.822755in}}%
\pgfpathlineto{\pgfqpoint{3.034323in}{1.817486in}}%
\pgfpathclose%
\pgfusepath{fill}%
\end{pgfscope}%
\begin{pgfscope}%
\pgfpathrectangle{\pgfqpoint{1.254980in}{0.150000in}}{\pgfqpoint{5.490039in}{5.490039in}}%
\pgfusepath{clip}%
\pgfsetbuttcap%
\pgfsetroundjoin%
\definecolor{currentfill}{rgb}{0.124395,0.578002,0.548287}%
\pgfsetfillcolor{currentfill}%
\pgfsetfillopacity{0.700000}%
\pgfsetlinewidth{0.000000pt}%
\definecolor{currentstroke}{rgb}{0.000000,0.000000,0.000000}%
\pgfsetstrokecolor{currentstroke}%
\pgfsetdash{}{0pt}%
\pgfpathmoveto{\pgfqpoint{5.308871in}{3.001359in}}%
\pgfpathlineto{\pgfqpoint{5.322846in}{3.011770in}}%
\pgfpathlineto{\pgfqpoint{5.336838in}{3.022339in}}%
\pgfpathlineto{\pgfqpoint{5.350848in}{3.033065in}}%
\pgfpathlineto{\pgfqpoint{5.364876in}{3.043949in}}%
\pgfpathlineto{\pgfqpoint{5.372024in}{3.047958in}}%
\pgfpathlineto{\pgfqpoint{5.379166in}{3.051918in}}%
\pgfpathlineto{\pgfqpoint{5.386299in}{3.055835in}}%
\pgfpathlineto{\pgfqpoint{5.393426in}{3.059712in}}%
\pgfpathlineto{\pgfqpoint{5.379417in}{3.049197in}}%
\pgfpathlineto{\pgfqpoint{5.365427in}{3.038839in}}%
\pgfpathlineto{\pgfqpoint{5.351453in}{3.028637in}}%
\pgfpathlineto{\pgfqpoint{5.337498in}{3.018592in}}%
\pgfpathlineto{\pgfqpoint{5.330352in}{3.014337in}}%
\pgfpathlineto{\pgfqpoint{5.323199in}{3.010050in}}%
\pgfpathlineto{\pgfqpoint{5.316038in}{3.005725in}}%
\pgfpathlineto{\pgfqpoint{5.308871in}{3.001359in}}%
\pgfpathclose%
\pgfusepath{fill}%
\end{pgfscope}%
\begin{pgfscope}%
\pgfpathrectangle{\pgfqpoint{1.254980in}{0.150000in}}{\pgfqpoint{5.490039in}{5.490039in}}%
\pgfusepath{clip}%
\pgfsetbuttcap%
\pgfsetroundjoin%
\definecolor{currentfill}{rgb}{0.271305,0.019942,0.347269}%
\pgfsetfillcolor{currentfill}%
\pgfsetfillopacity{0.700000}%
\pgfsetlinewidth{0.000000pt}%
\definecolor{currentstroke}{rgb}{0.000000,0.000000,0.000000}%
\pgfsetstrokecolor{currentstroke}%
\pgfsetdash{}{0pt}%
\pgfpathmoveto{\pgfqpoint{3.447631in}{1.751626in}}%
\pgfpathlineto{\pgfqpoint{3.460837in}{1.748196in}}%
\pgfpathlineto{\pgfqpoint{3.474048in}{1.744944in}}%
\pgfpathlineto{\pgfqpoint{3.487262in}{1.741870in}}%
\pgfpathlineto{\pgfqpoint{3.500481in}{1.738974in}}%
\pgfpathlineto{\pgfqpoint{3.508396in}{1.748143in}}%
\pgfpathlineto{\pgfqpoint{3.516304in}{1.757368in}}%
\pgfpathlineto{\pgfqpoint{3.524207in}{1.766646in}}%
\pgfpathlineto{\pgfqpoint{3.532103in}{1.775973in}}%
\pgfpathlineto{\pgfqpoint{3.518898in}{1.778542in}}%
\pgfpathlineto{\pgfqpoint{3.505697in}{1.781288in}}%
\pgfpathlineto{\pgfqpoint{3.492502in}{1.784213in}}%
\pgfpathlineto{\pgfqpoint{3.479310in}{1.787316in}}%
\pgfpathlineto{\pgfqpoint{3.471400in}{1.778305in}}%
\pgfpathlineto{\pgfqpoint{3.463484in}{1.769351in}}%
\pgfpathlineto{\pgfqpoint{3.455561in}{1.760457in}}%
\pgfpathlineto{\pgfqpoint{3.447631in}{1.751626in}}%
\pgfpathclose%
\pgfusepath{fill}%
\end{pgfscope}%
\begin{pgfscope}%
\pgfpathrectangle{\pgfqpoint{1.254980in}{0.150000in}}{\pgfqpoint{5.490039in}{5.490039in}}%
\pgfusepath{clip}%
\pgfsetbuttcap%
\pgfsetroundjoin%
\definecolor{currentfill}{rgb}{0.169646,0.456262,0.558030}%
\pgfsetfillcolor{currentfill}%
\pgfsetfillopacity{0.700000}%
\pgfsetlinewidth{0.000000pt}%
\definecolor{currentstroke}{rgb}{0.000000,0.000000,0.000000}%
\pgfsetstrokecolor{currentstroke}%
\pgfsetdash{}{0pt}%
\pgfpathmoveto{\pgfqpoint{4.856956in}{2.668268in}}%
\pgfpathlineto{\pgfqpoint{4.870681in}{2.677079in}}%
\pgfpathlineto{\pgfqpoint{4.884422in}{2.686050in}}%
\pgfpathlineto{\pgfqpoint{4.898178in}{2.695181in}}%
\pgfpathlineto{\pgfqpoint{4.911950in}{2.704472in}}%
\pgfpathlineto{\pgfqpoint{4.919346in}{2.711584in}}%
\pgfpathlineto{\pgfqpoint{4.926736in}{2.718607in}}%
\pgfpathlineto{\pgfqpoint{4.934118in}{2.725544in}}%
\pgfpathlineto{\pgfqpoint{4.941493in}{2.732398in}}%
\pgfpathlineto{\pgfqpoint{4.927732in}{2.723295in}}%
\pgfpathlineto{\pgfqpoint{4.913986in}{2.714351in}}%
\pgfpathlineto{\pgfqpoint{4.900255in}{2.705566in}}%
\pgfpathlineto{\pgfqpoint{4.886540in}{2.696940in}}%
\pgfpathlineto{\pgfqpoint{4.879154in}{2.689890in}}%
\pgfpathlineto{\pgfqpoint{4.871762in}{2.682762in}}%
\pgfpathlineto{\pgfqpoint{4.864362in}{2.675556in}}%
\pgfpathlineto{\pgfqpoint{4.856956in}{2.668268in}}%
\pgfpathclose%
\pgfusepath{fill}%
\end{pgfscope}%
\begin{pgfscope}%
\pgfpathrectangle{\pgfqpoint{1.254980in}{0.150000in}}{\pgfqpoint{5.490039in}{5.490039in}}%
\pgfusepath{clip}%
\pgfsetbuttcap%
\pgfsetroundjoin%
\definecolor{currentfill}{rgb}{0.237441,0.305202,0.541921}%
\pgfsetfillcolor{currentfill}%
\pgfsetfillopacity{0.700000}%
\pgfsetlinewidth{0.000000pt}%
\definecolor{currentstroke}{rgb}{0.000000,0.000000,0.000000}%
\pgfsetstrokecolor{currentstroke}%
\pgfsetdash{}{0pt}%
\pgfpathmoveto{\pgfqpoint{2.468105in}{2.362201in}}%
\pgfpathlineto{\pgfqpoint{2.481579in}{2.343335in}}%
\pgfpathlineto{\pgfqpoint{2.495042in}{2.324739in}}%
\pgfpathlineto{\pgfqpoint{2.508496in}{2.306411in}}%
\pgfpathlineto{\pgfqpoint{2.521940in}{2.288347in}}%
\pgfpathlineto{\pgfqpoint{2.530454in}{2.288804in}}%
\pgfpathlineto{\pgfqpoint{2.538953in}{2.289493in}}%
\pgfpathlineto{\pgfqpoint{2.547436in}{2.290410in}}%
\pgfpathlineto{\pgfqpoint{2.555904in}{2.291551in}}%
\pgfpathlineto{\pgfqpoint{2.542502in}{2.309149in}}%
\pgfpathlineto{\pgfqpoint{2.529091in}{2.327011in}}%
\pgfpathlineto{\pgfqpoint{2.515670in}{2.345140in}}%
\pgfpathlineto{\pgfqpoint{2.502240in}{2.363538in}}%
\pgfpathlineto{\pgfqpoint{2.493731in}{2.362852in}}%
\pgfpathlineto{\pgfqpoint{2.485205in}{2.362397in}}%
\pgfpathlineto{\pgfqpoint{2.476663in}{2.362179in}}%
\pgfpathlineto{\pgfqpoint{2.468105in}{2.362201in}}%
\pgfpathclose%
\pgfusepath{fill}%
\end{pgfscope}%
\begin{pgfscope}%
\pgfpathrectangle{\pgfqpoint{1.254980in}{0.150000in}}{\pgfqpoint{5.490039in}{5.490039in}}%
\pgfusepath{clip}%
\pgfsetbuttcap%
\pgfsetroundjoin%
\definecolor{currentfill}{rgb}{0.283072,0.130895,0.449241}%
\pgfsetfillcolor{currentfill}%
\pgfsetfillopacity{0.700000}%
\pgfsetlinewidth{0.000000pt}%
\definecolor{currentstroke}{rgb}{0.000000,0.000000,0.000000}%
\pgfsetstrokecolor{currentstroke}%
\pgfsetdash{}{0pt}%
\pgfpathmoveto{\pgfqpoint{2.789202in}{1.979521in}}%
\pgfpathlineto{\pgfqpoint{2.802503in}{1.966554in}}%
\pgfpathlineto{\pgfqpoint{2.815799in}{1.953809in}}%
\pgfpathlineto{\pgfqpoint{2.829091in}{1.941286in}}%
\pgfpathlineto{\pgfqpoint{2.842380in}{1.928983in}}%
\pgfpathlineto{\pgfqpoint{2.850667in}{1.932245in}}%
\pgfpathlineto{\pgfqpoint{2.858941in}{1.935694in}}%
\pgfpathlineto{\pgfqpoint{2.867204in}{1.939326in}}%
\pgfpathlineto{\pgfqpoint{2.875454in}{1.943137in}}%
\pgfpathlineto{\pgfqpoint{2.862199in}{1.954991in}}%
\pgfpathlineto{\pgfqpoint{2.848939in}{1.967064in}}%
\pgfpathlineto{\pgfqpoint{2.835676in}{1.979358in}}%
\pgfpathlineto{\pgfqpoint{2.822409in}{1.991875in}}%
\pgfpathlineto{\pgfqpoint{2.814126in}{1.988503in}}%
\pgfpathlineto{\pgfqpoint{2.805831in}{1.985318in}}%
\pgfpathlineto{\pgfqpoint{2.797523in}{1.982322in}}%
\pgfpathlineto{\pgfqpoint{2.789202in}{1.979521in}}%
\pgfpathclose%
\pgfusepath{fill}%
\end{pgfscope}%
\begin{pgfscope}%
\pgfpathrectangle{\pgfqpoint{1.254980in}{0.150000in}}{\pgfqpoint{5.490039in}{5.490039in}}%
\pgfusepath{clip}%
\pgfsetbuttcap%
\pgfsetroundjoin%
\definecolor{currentfill}{rgb}{0.263663,0.237631,0.518762}%
\pgfsetfillcolor{currentfill}%
\pgfsetfillopacity{0.700000}%
\pgfsetlinewidth{0.000000pt}%
\definecolor{currentstroke}{rgb}{0.000000,0.000000,0.000000}%
\pgfsetstrokecolor{currentstroke}%
\pgfsetdash{}{0pt}%
\pgfpathmoveto{\pgfqpoint{4.205825in}{2.147233in}}%
\pgfpathlineto{\pgfqpoint{4.219232in}{2.151875in}}%
\pgfpathlineto{\pgfqpoint{4.232650in}{2.156680in}}%
\pgfpathlineto{\pgfqpoint{4.246080in}{2.161649in}}%
\pgfpathlineto{\pgfqpoint{4.259521in}{2.166781in}}%
\pgfpathlineto{\pgfqpoint{4.267179in}{2.177364in}}%
\pgfpathlineto{\pgfqpoint{4.274832in}{2.187876in}}%
\pgfpathlineto{\pgfqpoint{4.282480in}{2.198317in}}%
\pgfpathlineto{\pgfqpoint{4.290123in}{2.208687in}}%
\pgfpathlineto{\pgfqpoint{4.276687in}{2.203478in}}%
\pgfpathlineto{\pgfqpoint{4.263261in}{2.198432in}}%
\pgfpathlineto{\pgfqpoint{4.249848in}{2.193551in}}%
\pgfpathlineto{\pgfqpoint{4.236446in}{2.188832in}}%
\pgfpathlineto{\pgfqpoint{4.228798in}{2.178528in}}%
\pgfpathlineto{\pgfqpoint{4.221145in}{2.168160in}}%
\pgfpathlineto{\pgfqpoint{4.213487in}{2.157728in}}%
\pgfpathlineto{\pgfqpoint{4.205825in}{2.147233in}}%
\pgfpathclose%
\pgfusepath{fill}%
\end{pgfscope}%
\begin{pgfscope}%
\pgfpathrectangle{\pgfqpoint{1.254980in}{0.150000in}}{\pgfqpoint{5.490039in}{5.490039in}}%
\pgfusepath{clip}%
\pgfsetbuttcap%
\pgfsetroundjoin%
\definecolor{currentfill}{rgb}{0.120092,0.600104,0.542530}%
\pgfsetfillcolor{currentfill}%
\pgfsetfillopacity{0.700000}%
\pgfsetlinewidth{0.000000pt}%
\definecolor{currentstroke}{rgb}{0.000000,0.000000,0.000000}%
\pgfsetstrokecolor{currentstroke}%
\pgfsetdash{}{0pt}%
\pgfpathmoveto{\pgfqpoint{5.393426in}{3.059712in}}%
\pgfpathlineto{\pgfqpoint{5.407453in}{3.070385in}}%
\pgfpathlineto{\pgfqpoint{5.421497in}{3.081214in}}%
\pgfpathlineto{\pgfqpoint{5.435560in}{3.092201in}}%
\pgfpathlineto{\pgfqpoint{5.449641in}{3.103346in}}%
\pgfpathlineto{\pgfqpoint{5.456740in}{3.106799in}}%
\pgfpathlineto{\pgfqpoint{5.463831in}{3.110215in}}%
\pgfpathlineto{\pgfqpoint{5.470916in}{3.113596in}}%
\pgfpathlineto{\pgfqpoint{5.477993in}{3.116949in}}%
\pgfpathlineto{\pgfqpoint{5.463933in}{3.106204in}}%
\pgfpathlineto{\pgfqpoint{5.449891in}{3.095615in}}%
\pgfpathlineto{\pgfqpoint{5.435868in}{3.085184in}}%
\pgfpathlineto{\pgfqpoint{5.421862in}{3.074908in}}%
\pgfpathlineto{\pgfqpoint{5.414763in}{3.071147in}}%
\pgfpathlineto{\pgfqpoint{5.407658in}{3.067364in}}%
\pgfpathlineto{\pgfqpoint{5.400545in}{3.063554in}}%
\pgfpathlineto{\pgfqpoint{5.393426in}{3.059712in}}%
\pgfpathclose%
\pgfusepath{fill}%
\end{pgfscope}%
\begin{pgfscope}%
\pgfpathrectangle{\pgfqpoint{1.254980in}{0.150000in}}{\pgfqpoint{5.490039in}{5.490039in}}%
\pgfusepath{clip}%
\pgfsetbuttcap%
\pgfsetroundjoin%
\definecolor{currentfill}{rgb}{0.269944,0.014625,0.341379}%
\pgfsetfillcolor{currentfill}%
\pgfsetfillopacity{0.700000}%
\pgfsetlinewidth{0.000000pt}%
\definecolor{currentstroke}{rgb}{0.000000,0.000000,0.000000}%
\pgfsetstrokecolor{currentstroke}%
\pgfsetdash{}{0pt}%
\pgfpathmoveto{\pgfqpoint{3.225306in}{1.747888in}}%
\pgfpathlineto{\pgfqpoint{3.238515in}{1.741513in}}%
\pgfpathlineto{\pgfqpoint{3.251725in}{1.735327in}}%
\pgfpathlineto{\pgfqpoint{3.264937in}{1.729328in}}%
\pgfpathlineto{\pgfqpoint{3.278151in}{1.723515in}}%
\pgfpathlineto{\pgfqpoint{3.286173in}{1.730936in}}%
\pgfpathlineto{\pgfqpoint{3.294187in}{1.738460in}}%
\pgfpathlineto{\pgfqpoint{3.302194in}{1.746085in}}%
\pgfpathlineto{\pgfqpoint{3.310193in}{1.753806in}}%
\pgfpathlineto{\pgfqpoint{3.296998in}{1.759235in}}%
\pgfpathlineto{\pgfqpoint{3.283807in}{1.764850in}}%
\pgfpathlineto{\pgfqpoint{3.270617in}{1.770652in}}%
\pgfpathlineto{\pgfqpoint{3.257429in}{1.776643in}}%
\pgfpathlineto{\pgfqpoint{3.249411in}{1.769295in}}%
\pgfpathlineto{\pgfqpoint{3.241384in}{1.762050in}}%
\pgfpathlineto{\pgfqpoint{3.233349in}{1.754914in}}%
\pgfpathlineto{\pgfqpoint{3.225306in}{1.747888in}}%
\pgfpathclose%
\pgfusepath{fill}%
\end{pgfscope}%
\begin{pgfscope}%
\pgfpathrectangle{\pgfqpoint{1.254980in}{0.150000in}}{\pgfqpoint{5.490039in}{5.490039in}}%
\pgfusepath{clip}%
\pgfsetbuttcap%
\pgfsetroundjoin%
\definecolor{currentfill}{rgb}{0.208623,0.367752,0.552675}%
\pgfsetfillcolor{currentfill}%
\pgfsetfillopacity{0.700000}%
\pgfsetlinewidth{0.000000pt}%
\definecolor{currentstroke}{rgb}{0.000000,0.000000,0.000000}%
\pgfsetstrokecolor{currentstroke}%
\pgfsetdash{}{0pt}%
\pgfpathmoveto{\pgfqpoint{4.573627in}{2.440371in}}%
\pgfpathlineto{\pgfqpoint{4.587208in}{2.447708in}}%
\pgfpathlineto{\pgfqpoint{4.600802in}{2.455205in}}%
\pgfpathlineto{\pgfqpoint{4.614411in}{2.462863in}}%
\pgfpathlineto{\pgfqpoint{4.628033in}{2.470683in}}%
\pgfpathlineto{\pgfqpoint{4.635559in}{2.479638in}}%
\pgfpathlineto{\pgfqpoint{4.643079in}{2.488501in}}%
\pgfpathlineto{\pgfqpoint{4.650592in}{2.497271in}}%
\pgfpathlineto{\pgfqpoint{4.658100in}{2.505950in}}%
\pgfpathlineto{\pgfqpoint{4.644484in}{2.498199in}}%
\pgfpathlineto{\pgfqpoint{4.630882in}{2.490609in}}%
\pgfpathlineto{\pgfqpoint{4.617293in}{2.483179in}}%
\pgfpathlineto{\pgfqpoint{4.603719in}{2.475911in}}%
\pgfpathlineto{\pgfqpoint{4.596205in}{2.467153in}}%
\pgfpathlineto{\pgfqpoint{4.588685in}{2.458311in}}%
\pgfpathlineto{\pgfqpoint{4.581159in}{2.449384in}}%
\pgfpathlineto{\pgfqpoint{4.573627in}{2.440371in}}%
\pgfpathclose%
\pgfusepath{fill}%
\end{pgfscope}%
\begin{pgfscope}%
\pgfpathrectangle{\pgfqpoint{1.254980in}{0.150000in}}{\pgfqpoint{5.490039in}{5.490039in}}%
\pgfusepath{clip}%
\pgfsetbuttcap%
\pgfsetroundjoin%
\definecolor{currentfill}{rgb}{0.221989,0.339161,0.548752}%
\pgfsetfillcolor{currentfill}%
\pgfsetfillopacity{0.700000}%
\pgfsetlinewidth{0.000000pt}%
\definecolor{currentstroke}{rgb}{0.000000,0.000000,0.000000}%
\pgfsetstrokecolor{currentstroke}%
\pgfsetdash{}{0pt}%
\pgfpathmoveto{\pgfqpoint{2.414103in}{2.440414in}}%
\pgfpathlineto{\pgfqpoint{2.427620in}{2.420444in}}%
\pgfpathlineto{\pgfqpoint{2.441126in}{2.400753in}}%
\pgfpathlineto{\pgfqpoint{2.454621in}{2.381339in}}%
\pgfpathlineto{\pgfqpoint{2.468105in}{2.362201in}}%
\pgfpathlineto{\pgfqpoint{2.476663in}{2.362179in}}%
\pgfpathlineto{\pgfqpoint{2.485205in}{2.362397in}}%
\pgfpathlineto{\pgfqpoint{2.493731in}{2.362852in}}%
\pgfpathlineto{\pgfqpoint{2.502240in}{2.363538in}}%
\pgfpathlineto{\pgfqpoint{2.488800in}{2.382207in}}%
\pgfpathlineto{\pgfqpoint{2.475349in}{2.401150in}}%
\pgfpathlineto{\pgfqpoint{2.461888in}{2.420370in}}%
\pgfpathlineto{\pgfqpoint{2.448416in}{2.439869in}}%
\pgfpathlineto{\pgfqpoint{2.439863in}{2.439642in}}%
\pgfpathlineto{\pgfqpoint{2.431294in}{2.439654in}}%
\pgfpathlineto{\pgfqpoint{2.422707in}{2.439910in}}%
\pgfpathlineto{\pgfqpoint{2.414103in}{2.440414in}}%
\pgfpathclose%
\pgfusepath{fill}%
\end{pgfscope}%
\begin{pgfscope}%
\pgfpathrectangle{\pgfqpoint{1.254980in}{0.150000in}}{\pgfqpoint{5.490039in}{5.490039in}}%
\pgfusepath{clip}%
\pgfsetbuttcap%
\pgfsetroundjoin%
\definecolor{currentfill}{rgb}{0.159194,0.482237,0.558073}%
\pgfsetfillcolor{currentfill}%
\pgfsetfillopacity{0.700000}%
\pgfsetlinewidth{0.000000pt}%
\definecolor{currentstroke}{rgb}{0.000000,0.000000,0.000000}%
\pgfsetstrokecolor{currentstroke}%
\pgfsetdash{}{0pt}%
\pgfpathmoveto{\pgfqpoint{4.941493in}{2.732398in}}%
\pgfpathlineto{\pgfqpoint{4.955271in}{2.741661in}}%
\pgfpathlineto{\pgfqpoint{4.969064in}{2.751082in}}%
\pgfpathlineto{\pgfqpoint{4.982873in}{2.760664in}}%
\pgfpathlineto{\pgfqpoint{4.996699in}{2.770404in}}%
\pgfpathlineto{\pgfqpoint{5.004057in}{2.776970in}}%
\pgfpathlineto{\pgfqpoint{5.011407in}{2.783450in}}%
\pgfpathlineto{\pgfqpoint{5.018750in}{2.789846in}}%
\pgfpathlineto{\pgfqpoint{5.026085in}{2.796162in}}%
\pgfpathlineto{\pgfqpoint{5.012271in}{2.786639in}}%
\pgfpathlineto{\pgfqpoint{4.998473in}{2.777275in}}%
\pgfpathlineto{\pgfqpoint{4.984691in}{2.768070in}}%
\pgfpathlineto{\pgfqpoint{4.970925in}{2.759023in}}%
\pgfpathlineto{\pgfqpoint{4.963577in}{2.752480in}}%
\pgfpathlineto{\pgfqpoint{4.956223in}{2.745863in}}%
\pgfpathlineto{\pgfqpoint{4.948862in}{2.739170in}}%
\pgfpathlineto{\pgfqpoint{4.941493in}{2.732398in}}%
\pgfpathclose%
\pgfusepath{fill}%
\end{pgfscope}%
\begin{pgfscope}%
\pgfpathrectangle{\pgfqpoint{1.254980in}{0.150000in}}{\pgfqpoint{5.490039in}{5.490039in}}%
\pgfusepath{clip}%
\pgfsetbuttcap%
\pgfsetroundjoin%
\definecolor{currentfill}{rgb}{0.283091,0.110553,0.431554}%
\pgfsetfillcolor{currentfill}%
\pgfsetfillopacity{0.700000}%
\pgfsetlinewidth{0.000000pt}%
\definecolor{currentstroke}{rgb}{0.000000,0.000000,0.000000}%
\pgfsetstrokecolor{currentstroke}%
\pgfsetdash{}{0pt}%
\pgfpathmoveto{\pgfqpoint{2.842380in}{1.928983in}}%
\pgfpathlineto{\pgfqpoint{2.855664in}{1.916898in}}%
\pgfpathlineto{\pgfqpoint{2.868946in}{1.905030in}}%
\pgfpathlineto{\pgfqpoint{2.882224in}{1.893377in}}%
\pgfpathlineto{\pgfqpoint{2.895499in}{1.881939in}}%
\pgfpathlineto{\pgfqpoint{2.903753in}{1.885660in}}%
\pgfpathlineto{\pgfqpoint{2.911996in}{1.889562in}}%
\pgfpathlineto{\pgfqpoint{2.920227in}{1.893638in}}%
\pgfpathlineto{\pgfqpoint{2.928446in}{1.897885in}}%
\pgfpathlineto{\pgfqpoint{2.915203in}{1.908876in}}%
\pgfpathlineto{\pgfqpoint{2.901956in}{1.920081in}}%
\pgfpathlineto{\pgfqpoint{2.888707in}{1.931501in}}%
\pgfpathlineto{\pgfqpoint{2.875454in}{1.943137in}}%
\pgfpathlineto{\pgfqpoint{2.867204in}{1.939326in}}%
\pgfpathlineto{\pgfqpoint{2.858941in}{1.935694in}}%
\pgfpathlineto{\pgfqpoint{2.850667in}{1.932245in}}%
\pgfpathlineto{\pgfqpoint{2.842380in}{1.928983in}}%
\pgfpathclose%
\pgfusepath{fill}%
\end{pgfscope}%
\begin{pgfscope}%
\pgfpathrectangle{\pgfqpoint{1.254980in}{0.150000in}}{\pgfqpoint{5.490039in}{5.490039in}}%
\pgfusepath{clip}%
\pgfsetbuttcap%
\pgfsetroundjoin%
\definecolor{currentfill}{rgb}{0.120638,0.625828,0.533488}%
\pgfsetfillcolor{currentfill}%
\pgfsetfillopacity{0.700000}%
\pgfsetlinewidth{0.000000pt}%
\definecolor{currentstroke}{rgb}{0.000000,0.000000,0.000000}%
\pgfsetstrokecolor{currentstroke}%
\pgfsetdash{}{0pt}%
\pgfpathmoveto{\pgfqpoint{5.477993in}{3.116949in}}%
\pgfpathlineto{\pgfqpoint{5.492071in}{3.127851in}}%
\pgfpathlineto{\pgfqpoint{5.506168in}{3.138910in}}%
\pgfpathlineto{\pgfqpoint{5.520283in}{3.150125in}}%
\pgfpathlineto{\pgfqpoint{5.534417in}{3.161498in}}%
\pgfpathlineto{\pgfqpoint{5.541464in}{3.164407in}}%
\pgfpathlineto{\pgfqpoint{5.548505in}{3.167290in}}%
\pgfpathlineto{\pgfqpoint{5.555538in}{3.170150in}}%
\pgfpathlineto{\pgfqpoint{5.562564in}{3.172992in}}%
\pgfpathlineto{\pgfqpoint{5.548454in}{3.162049in}}%
\pgfpathlineto{\pgfqpoint{5.534362in}{3.151263in}}%
\pgfpathlineto{\pgfqpoint{5.520288in}{3.140632in}}%
\pgfpathlineto{\pgfqpoint{5.506233in}{3.130158in}}%
\pgfpathlineto{\pgfqpoint{5.499183in}{3.126877in}}%
\pgfpathlineto{\pgfqpoint{5.492126in}{3.123584in}}%
\pgfpathlineto{\pgfqpoint{5.485063in}{3.120277in}}%
\pgfpathlineto{\pgfqpoint{5.477993in}{3.116949in}}%
\pgfpathclose%
\pgfusepath{fill}%
\end{pgfscope}%
\begin{pgfscope}%
\pgfpathrectangle{\pgfqpoint{1.254980in}{0.150000in}}{\pgfqpoint{5.490039in}{5.490039in}}%
\pgfusepath{clip}%
\pgfsetbuttcap%
\pgfsetroundjoin%
\definecolor{currentfill}{rgb}{0.269944,0.014625,0.341379}%
\pgfsetfillcolor{currentfill}%
\pgfsetfillopacity{0.700000}%
\pgfsetlinewidth{0.000000pt}%
\definecolor{currentstroke}{rgb}{0.000000,0.000000,0.000000}%
\pgfsetstrokecolor{currentstroke}%
\pgfsetdash{}{0pt}%
\pgfpathmoveto{\pgfqpoint{3.362995in}{1.733941in}}%
\pgfpathlineto{\pgfqpoint{3.376203in}{1.729433in}}%
\pgfpathlineto{\pgfqpoint{3.389414in}{1.725107in}}%
\pgfpathlineto{\pgfqpoint{3.402629in}{1.720961in}}%
\pgfpathlineto{\pgfqpoint{3.415847in}{1.716996in}}%
\pgfpathlineto{\pgfqpoint{3.423803in}{1.725543in}}%
\pgfpathlineto{\pgfqpoint{3.431753in}{1.734166in}}%
\pgfpathlineto{\pgfqpoint{3.439695in}{1.742861in}}%
\pgfpathlineto{\pgfqpoint{3.447631in}{1.751626in}}%
\pgfpathlineto{\pgfqpoint{3.434429in}{1.755236in}}%
\pgfpathlineto{\pgfqpoint{3.421231in}{1.759026in}}%
\pgfpathlineto{\pgfqpoint{3.408037in}{1.762997in}}%
\pgfpathlineto{\pgfqpoint{3.394846in}{1.767150in}}%
\pgfpathlineto{\pgfqpoint{3.386893in}{1.758730in}}%
\pgfpathlineto{\pgfqpoint{3.378934in}{1.750386in}}%
\pgfpathlineto{\pgfqpoint{3.370968in}{1.742122in}}%
\pgfpathlineto{\pgfqpoint{3.362995in}{1.733941in}}%
\pgfpathclose%
\pgfusepath{fill}%
\end{pgfscope}%
\begin{pgfscope}%
\pgfpathrectangle{\pgfqpoint{1.254980in}{0.150000in}}{\pgfqpoint{5.490039in}{5.490039in}}%
\pgfusepath{clip}%
\pgfsetbuttcap%
\pgfsetroundjoin%
\definecolor{currentfill}{rgb}{0.252194,0.269783,0.531579}%
\pgfsetfillcolor{currentfill}%
\pgfsetfillopacity{0.700000}%
\pgfsetlinewidth{0.000000pt}%
\definecolor{currentstroke}{rgb}{0.000000,0.000000,0.000000}%
\pgfsetstrokecolor{currentstroke}%
\pgfsetdash{}{0pt}%
\pgfpathmoveto{\pgfqpoint{4.290123in}{2.208687in}}%
\pgfpathlineto{\pgfqpoint{4.303571in}{2.214059in}}%
\pgfpathlineto{\pgfqpoint{4.317031in}{2.219594in}}%
\pgfpathlineto{\pgfqpoint{4.330503in}{2.225293in}}%
\pgfpathlineto{\pgfqpoint{4.343987in}{2.231153in}}%
\pgfpathlineto{\pgfqpoint{4.351620in}{2.241511in}}%
\pgfpathlineto{\pgfqpoint{4.359248in}{2.251789in}}%
\pgfpathlineto{\pgfqpoint{4.366870in}{2.261989in}}%
\pgfpathlineto{\pgfqpoint{4.374487in}{2.272110in}}%
\pgfpathlineto{\pgfqpoint{4.361008in}{2.266201in}}%
\pgfpathlineto{\pgfqpoint{4.347540in}{2.260455in}}%
\pgfpathlineto{\pgfqpoint{4.334085in}{2.254872in}}%
\pgfpathlineto{\pgfqpoint{4.320642in}{2.249452in}}%
\pgfpathlineto{\pgfqpoint{4.313020in}{2.239368in}}%
\pgfpathlineto{\pgfqpoint{4.305393in}{2.229213in}}%
\pgfpathlineto{\pgfqpoint{4.297761in}{2.218986in}}%
\pgfpathlineto{\pgfqpoint{4.290123in}{2.208687in}}%
\pgfpathclose%
\pgfusepath{fill}%
\end{pgfscope}%
\begin{pgfscope}%
\pgfpathrectangle{\pgfqpoint{1.254980in}{0.150000in}}{\pgfqpoint{5.490039in}{5.490039in}}%
\pgfusepath{clip}%
\pgfsetbuttcap%
\pgfsetroundjoin%
\definecolor{currentfill}{rgb}{0.281446,0.084320,0.407414}%
\pgfsetfillcolor{currentfill}%
\pgfsetfillopacity{0.700000}%
\pgfsetlinewidth{0.000000pt}%
\definecolor{currentstroke}{rgb}{0.000000,0.000000,0.000000}%
\pgfsetstrokecolor{currentstroke}%
\pgfsetdash{}{0pt}%
\pgfpathmoveto{\pgfqpoint{3.753678in}{1.841725in}}%
\pgfpathlineto{\pgfqpoint{3.766939in}{1.841983in}}%
\pgfpathlineto{\pgfqpoint{3.780207in}{1.842411in}}%
\pgfpathlineto{\pgfqpoint{3.793483in}{1.843008in}}%
\pgfpathlineto{\pgfqpoint{3.806767in}{1.843774in}}%
\pgfpathlineto{\pgfqpoint{3.814571in}{1.854450in}}%
\pgfpathlineto{\pgfqpoint{3.822370in}{1.865122in}}%
\pgfpathlineto{\pgfqpoint{3.830165in}{1.875788in}}%
\pgfpathlineto{\pgfqpoint{3.837954in}{1.886447in}}%
\pgfpathlineto{\pgfqpoint{3.824679in}{1.885436in}}%
\pgfpathlineto{\pgfqpoint{3.811411in}{1.884595in}}%
\pgfpathlineto{\pgfqpoint{3.798151in}{1.883922in}}%
\pgfpathlineto{\pgfqpoint{3.784899in}{1.883420in}}%
\pgfpathlineto{\pgfqpoint{3.777101in}{1.872995in}}%
\pgfpathlineto{\pgfqpoint{3.769299in}{1.862570in}}%
\pgfpathlineto{\pgfqpoint{3.761491in}{1.852146in}}%
\pgfpathlineto{\pgfqpoint{3.753678in}{1.841725in}}%
\pgfpathclose%
\pgfusepath{fill}%
\end{pgfscope}%
\begin{pgfscope}%
\pgfpathrectangle{\pgfqpoint{1.254980in}{0.150000in}}{\pgfqpoint{5.490039in}{5.490039in}}%
\pgfusepath{clip}%
\pgfsetbuttcap%
\pgfsetroundjoin%
\definecolor{currentfill}{rgb}{0.273809,0.031497,0.358853}%
\pgfsetfillcolor{currentfill}%
\pgfsetfillopacity{0.700000}%
\pgfsetlinewidth{0.000000pt}%
\definecolor{currentstroke}{rgb}{0.000000,0.000000,0.000000}%
\pgfsetstrokecolor{currentstroke}%
\pgfsetdash{}{0pt}%
\pgfpathmoveto{\pgfqpoint{3.087232in}{1.782176in}}%
\pgfpathlineto{\pgfqpoint{3.100458in}{1.773846in}}%
\pgfpathlineto{\pgfqpoint{3.113684in}{1.765711in}}%
\pgfpathlineto{\pgfqpoint{3.126911in}{1.757772in}}%
\pgfpathlineto{\pgfqpoint{3.140137in}{1.750028in}}%
\pgfpathlineto{\pgfqpoint{3.148238in}{1.756147in}}%
\pgfpathlineto{\pgfqpoint{3.156330in}{1.762400in}}%
\pgfpathlineto{\pgfqpoint{3.164412in}{1.768782in}}%
\pgfpathlineto{\pgfqpoint{3.172486in}{1.775290in}}%
\pgfpathlineto{\pgfqpoint{3.159283in}{1.782622in}}%
\pgfpathlineto{\pgfqpoint{3.146081in}{1.790147in}}%
\pgfpathlineto{\pgfqpoint{3.132879in}{1.797867in}}%
\pgfpathlineto{\pgfqpoint{3.119678in}{1.805784in}}%
\pgfpathlineto{\pgfqpoint{3.111580in}{1.799679in}}%
\pgfpathlineto{\pgfqpoint{3.103474in}{1.793706in}}%
\pgfpathlineto{\pgfqpoint{3.095358in}{1.787871in}}%
\pgfpathlineto{\pgfqpoint{3.087232in}{1.782176in}}%
\pgfpathclose%
\pgfusepath{fill}%
\end{pgfscope}%
\begin{pgfscope}%
\pgfpathrectangle{\pgfqpoint{1.254980in}{0.150000in}}{\pgfqpoint{5.490039in}{5.490039in}}%
\pgfusepath{clip}%
\pgfsetbuttcap%
\pgfsetroundjoin%
\definecolor{currentfill}{rgb}{0.283091,0.110553,0.431554}%
\pgfsetfillcolor{currentfill}%
\pgfsetfillopacity{0.700000}%
\pgfsetlinewidth{0.000000pt}%
\definecolor{currentstroke}{rgb}{0.000000,0.000000,0.000000}%
\pgfsetstrokecolor{currentstroke}%
\pgfsetdash{}{0pt}%
\pgfpathmoveto{\pgfqpoint{3.837954in}{1.886447in}}%
\pgfpathlineto{\pgfqpoint{3.851237in}{1.887627in}}%
\pgfpathlineto{\pgfqpoint{3.864529in}{1.888975in}}%
\pgfpathlineto{\pgfqpoint{3.877828in}{1.890491in}}%
\pgfpathlineto{\pgfqpoint{3.891136in}{1.892175in}}%
\pgfpathlineto{\pgfqpoint{3.898913in}{1.903051in}}%
\pgfpathlineto{\pgfqpoint{3.906686in}{1.913909in}}%
\pgfpathlineto{\pgfqpoint{3.914453in}{1.924747in}}%
\pgfpathlineto{\pgfqpoint{3.922216in}{1.935562in}}%
\pgfpathlineto{\pgfqpoint{3.908915in}{1.933662in}}%
\pgfpathlineto{\pgfqpoint{3.895622in}{1.931929in}}%
\pgfpathlineto{\pgfqpoint{3.882338in}{1.930364in}}%
\pgfpathlineto{\pgfqpoint{3.869062in}{1.928968in}}%
\pgfpathlineto{\pgfqpoint{3.861293in}{1.918359in}}%
\pgfpathlineto{\pgfqpoint{3.853518in}{1.907734in}}%
\pgfpathlineto{\pgfqpoint{3.845739in}{1.897097in}}%
\pgfpathlineto{\pgfqpoint{3.837954in}{1.886447in}}%
\pgfpathclose%
\pgfusepath{fill}%
\end{pgfscope}%
\begin{pgfscope}%
\pgfpathrectangle{\pgfqpoint{1.254980in}{0.150000in}}{\pgfqpoint{5.490039in}{5.490039in}}%
\pgfusepath{clip}%
\pgfsetbuttcap%
\pgfsetroundjoin%
\definecolor{currentfill}{rgb}{0.278791,0.062145,0.386592}%
\pgfsetfillcolor{currentfill}%
\pgfsetfillopacity{0.700000}%
\pgfsetlinewidth{0.000000pt}%
\definecolor{currentstroke}{rgb}{0.000000,0.000000,0.000000}%
\pgfsetstrokecolor{currentstroke}%
\pgfsetdash{}{0pt}%
\pgfpathmoveto{\pgfqpoint{3.669361in}{1.801891in}}%
\pgfpathlineto{\pgfqpoint{3.682604in}{1.801192in}}%
\pgfpathlineto{\pgfqpoint{3.695854in}{1.800665in}}%
\pgfpathlineto{\pgfqpoint{3.709111in}{1.800310in}}%
\pgfpathlineto{\pgfqpoint{3.722374in}{1.800125in}}%
\pgfpathlineto{\pgfqpoint{3.730208in}{1.810508in}}%
\pgfpathlineto{\pgfqpoint{3.738037in}{1.820904in}}%
\pgfpathlineto{\pgfqpoint{3.745860in}{1.831311in}}%
\pgfpathlineto{\pgfqpoint{3.753678in}{1.841725in}}%
\pgfpathlineto{\pgfqpoint{3.740424in}{1.841638in}}%
\pgfpathlineto{\pgfqpoint{3.727177in}{1.841721in}}%
\pgfpathlineto{\pgfqpoint{3.713937in}{1.841976in}}%
\pgfpathlineto{\pgfqpoint{3.700704in}{1.842403in}}%
\pgfpathlineto{\pgfqpoint{3.692876in}{1.832250in}}%
\pgfpathlineto{\pgfqpoint{3.685043in}{1.822112in}}%
\pgfpathlineto{\pgfqpoint{3.677205in}{1.811992in}}%
\pgfpathlineto{\pgfqpoint{3.669361in}{1.801891in}}%
\pgfpathclose%
\pgfusepath{fill}%
\end{pgfscope}%
\begin{pgfscope}%
\pgfpathrectangle{\pgfqpoint{1.254980in}{0.150000in}}{\pgfqpoint{5.490039in}{5.490039in}}%
\pgfusepath{clip}%
\pgfsetbuttcap%
\pgfsetroundjoin%
\definecolor{currentfill}{rgb}{0.128087,0.647749,0.523491}%
\pgfsetfillcolor{currentfill}%
\pgfsetfillopacity{0.700000}%
\pgfsetlinewidth{0.000000pt}%
\definecolor{currentstroke}{rgb}{0.000000,0.000000,0.000000}%
\pgfsetstrokecolor{currentstroke}%
\pgfsetdash{}{0pt}%
\pgfpathmoveto{\pgfqpoint{5.562564in}{3.172992in}}%
\pgfpathlineto{\pgfqpoint{5.576694in}{3.184091in}}%
\pgfpathlineto{\pgfqpoint{5.590841in}{3.195347in}}%
\pgfpathlineto{\pgfqpoint{5.605008in}{3.206760in}}%
\pgfpathlineto{\pgfqpoint{5.619194in}{3.218329in}}%
\pgfpathlineto{\pgfqpoint{5.626189in}{3.220710in}}%
\pgfpathlineto{\pgfqpoint{5.633177in}{3.223076in}}%
\pgfpathlineto{\pgfqpoint{5.640158in}{3.225433in}}%
\pgfpathlineto{\pgfqpoint{5.647133in}{3.227785in}}%
\pgfpathlineto{\pgfqpoint{5.632972in}{3.216676in}}%
\pgfpathlineto{\pgfqpoint{5.618831in}{3.205724in}}%
\pgfpathlineto{\pgfqpoint{5.604708in}{3.194927in}}%
\pgfpathlineto{\pgfqpoint{5.590604in}{3.184286in}}%
\pgfpathlineto{\pgfqpoint{5.583603in}{3.181464in}}%
\pgfpathlineto{\pgfqpoint{5.576597in}{3.178644in}}%
\pgfpathlineto{\pgfqpoint{5.569584in}{3.175822in}}%
\pgfpathlineto{\pgfqpoint{5.562564in}{3.172992in}}%
\pgfpathclose%
\pgfusepath{fill}%
\end{pgfscope}%
\begin{pgfscope}%
\pgfpathrectangle{\pgfqpoint{1.254980in}{0.150000in}}{\pgfqpoint{5.490039in}{5.490039in}}%
\pgfusepath{clip}%
\pgfsetbuttcap%
\pgfsetroundjoin%
\definecolor{currentfill}{rgb}{0.204903,0.375746,0.553533}%
\pgfsetfillcolor{currentfill}%
\pgfsetfillopacity{0.700000}%
\pgfsetlinewidth{0.000000pt}%
\definecolor{currentstroke}{rgb}{0.000000,0.000000,0.000000}%
\pgfsetstrokecolor{currentstroke}%
\pgfsetdash{}{0pt}%
\pgfpathmoveto{\pgfqpoint{2.359916in}{2.523152in}}%
\pgfpathlineto{\pgfqpoint{2.373481in}{2.502034in}}%
\pgfpathlineto{\pgfqpoint{2.387034in}{2.481207in}}%
\pgfpathlineto{\pgfqpoint{2.400574in}{2.460668in}}%
\pgfpathlineto{\pgfqpoint{2.414103in}{2.440414in}}%
\pgfpathlineto{\pgfqpoint{2.422707in}{2.439910in}}%
\pgfpathlineto{\pgfqpoint{2.431294in}{2.439654in}}%
\pgfpathlineto{\pgfqpoint{2.439863in}{2.439642in}}%
\pgfpathlineto{\pgfqpoint{2.448416in}{2.439869in}}%
\pgfpathlineto{\pgfqpoint{2.434933in}{2.459650in}}%
\pgfpathlineto{\pgfqpoint{2.421439in}{2.479715in}}%
\pgfpathlineto{\pgfqpoint{2.407933in}{2.500067in}}%
\pgfpathlineto{\pgfqpoint{2.394414in}{2.520709in}}%
\pgfpathlineto{\pgfqpoint{2.385816in}{2.520944in}}%
\pgfpathlineto{\pgfqpoint{2.377200in}{2.521426in}}%
\pgfpathlineto{\pgfqpoint{2.368567in}{2.522161in}}%
\pgfpathlineto{\pgfqpoint{2.359916in}{2.523152in}}%
\pgfpathclose%
\pgfusepath{fill}%
\end{pgfscope}%
\begin{pgfscope}%
\pgfpathrectangle{\pgfqpoint{1.254980in}{0.150000in}}{\pgfqpoint{5.490039in}{5.490039in}}%
\pgfusepath{clip}%
\pgfsetbuttcap%
\pgfsetroundjoin%
\definecolor{currentfill}{rgb}{0.282884,0.135920,0.453427}%
\pgfsetfillcolor{currentfill}%
\pgfsetfillopacity{0.700000}%
\pgfsetlinewidth{0.000000pt}%
\definecolor{currentstroke}{rgb}{0.000000,0.000000,0.000000}%
\pgfsetstrokecolor{currentstroke}%
\pgfsetdash{}{0pt}%
\pgfpathmoveto{\pgfqpoint{3.922216in}{1.935562in}}%
\pgfpathlineto{\pgfqpoint{3.935525in}{1.937630in}}%
\pgfpathlineto{\pgfqpoint{3.948844in}{1.939865in}}%
\pgfpathlineto{\pgfqpoint{3.962171in}{1.942267in}}%
\pgfpathlineto{\pgfqpoint{3.975508in}{1.944834in}}%
\pgfpathlineto{\pgfqpoint{3.983260in}{1.955826in}}%
\pgfpathlineto{\pgfqpoint{3.991006in}{1.966784in}}%
\pgfpathlineto{\pgfqpoint{3.998748in}{1.977708in}}%
\pgfpathlineto{\pgfqpoint{4.006485in}{1.988597in}}%
\pgfpathlineto{\pgfqpoint{3.993154in}{1.985840in}}%
\pgfpathlineto{\pgfqpoint{3.979833in}{1.983249in}}%
\pgfpathlineto{\pgfqpoint{3.966521in}{1.980825in}}%
\pgfpathlineto{\pgfqpoint{3.953218in}{1.978568in}}%
\pgfpathlineto{\pgfqpoint{3.945474in}{1.967858in}}%
\pgfpathlineto{\pgfqpoint{3.937726in}{1.957120in}}%
\pgfpathlineto{\pgfqpoint{3.929973in}{1.946354in}}%
\pgfpathlineto{\pgfqpoint{3.922216in}{1.935562in}}%
\pgfpathclose%
\pgfusepath{fill}%
\end{pgfscope}%
\begin{pgfscope}%
\pgfpathrectangle{\pgfqpoint{1.254980in}{0.150000in}}{\pgfqpoint{5.490039in}{5.490039in}}%
\pgfusepath{clip}%
\pgfsetbuttcap%
\pgfsetroundjoin%
\definecolor{currentfill}{rgb}{0.194100,0.399323,0.555565}%
\pgfsetfillcolor{currentfill}%
\pgfsetfillopacity{0.700000}%
\pgfsetlinewidth{0.000000pt}%
\definecolor{currentstroke}{rgb}{0.000000,0.000000,0.000000}%
\pgfsetstrokecolor{currentstroke}%
\pgfsetdash{}{0pt}%
\pgfpathmoveto{\pgfqpoint{4.658100in}{2.505950in}}%
\pgfpathlineto{\pgfqpoint{4.671730in}{2.513862in}}%
\pgfpathlineto{\pgfqpoint{4.685375in}{2.521934in}}%
\pgfpathlineto{\pgfqpoint{4.699034in}{2.530168in}}%
\pgfpathlineto{\pgfqpoint{4.712708in}{2.538562in}}%
\pgfpathlineto{\pgfqpoint{4.720202in}{2.547064in}}%
\pgfpathlineto{\pgfqpoint{4.727690in}{2.555471in}}%
\pgfpathlineto{\pgfqpoint{4.735171in}{2.563783in}}%
\pgfpathlineto{\pgfqpoint{4.742646in}{2.572003in}}%
\pgfpathlineto{\pgfqpoint{4.728979in}{2.563707in}}%
\pgfpathlineto{\pgfqpoint{4.715327in}{2.555572in}}%
\pgfpathlineto{\pgfqpoint{4.701689in}{2.547597in}}%
\pgfpathlineto{\pgfqpoint{4.688066in}{2.539782in}}%
\pgfpathlineto{\pgfqpoint{4.680584in}{2.531454in}}%
\pgfpathlineto{\pgfqpoint{4.673096in}{2.523040in}}%
\pgfpathlineto{\pgfqpoint{4.665601in}{2.514539in}}%
\pgfpathlineto{\pgfqpoint{4.658100in}{2.505950in}}%
\pgfpathclose%
\pgfusepath{fill}%
\end{pgfscope}%
\begin{pgfscope}%
\pgfpathrectangle{\pgfqpoint{1.254980in}{0.150000in}}{\pgfqpoint{5.490039in}{5.490039in}}%
\pgfusepath{clip}%
\pgfsetbuttcap%
\pgfsetroundjoin%
\definecolor{currentfill}{rgb}{0.274952,0.037752,0.364543}%
\pgfsetfillcolor{currentfill}%
\pgfsetfillopacity{0.700000}%
\pgfsetlinewidth{0.000000pt}%
\definecolor{currentstroke}{rgb}{0.000000,0.000000,0.000000}%
\pgfsetstrokecolor{currentstroke}%
\pgfsetdash{}{0pt}%
\pgfpathmoveto{\pgfqpoint{3.584973in}{1.767461in}}%
\pgfpathlineto{\pgfqpoint{3.598204in}{1.765770in}}%
\pgfpathlineto{\pgfqpoint{3.611441in}{1.764253in}}%
\pgfpathlineto{\pgfqpoint{3.624683in}{1.762910in}}%
\pgfpathlineto{\pgfqpoint{3.637931in}{1.761739in}}%
\pgfpathlineto{\pgfqpoint{3.645797in}{1.771734in}}%
\pgfpathlineto{\pgfqpoint{3.653657in}{1.781760in}}%
\pgfpathlineto{\pgfqpoint{3.661512in}{1.791813in}}%
\pgfpathlineto{\pgfqpoint{3.669361in}{1.801891in}}%
\pgfpathlineto{\pgfqpoint{3.656124in}{1.802762in}}%
\pgfpathlineto{\pgfqpoint{3.642893in}{1.803806in}}%
\pgfpathlineto{\pgfqpoint{3.629668in}{1.805023in}}%
\pgfpathlineto{\pgfqpoint{3.616448in}{1.806414in}}%
\pgfpathlineto{\pgfqpoint{3.608588in}{1.796625in}}%
\pgfpathlineto{\pgfqpoint{3.600722in}{1.786868in}}%
\pgfpathlineto{\pgfqpoint{3.592851in}{1.777146in}}%
\pgfpathlineto{\pgfqpoint{3.584973in}{1.767461in}}%
\pgfpathclose%
\pgfusepath{fill}%
\end{pgfscope}%
\begin{pgfscope}%
\pgfpathrectangle{\pgfqpoint{1.254980in}{0.150000in}}{\pgfqpoint{5.490039in}{5.490039in}}%
\pgfusepath{clip}%
\pgfsetbuttcap%
\pgfsetroundjoin%
\definecolor{currentfill}{rgb}{0.149039,0.508051,0.557250}%
\pgfsetfillcolor{currentfill}%
\pgfsetfillopacity{0.700000}%
\pgfsetlinewidth{0.000000pt}%
\definecolor{currentstroke}{rgb}{0.000000,0.000000,0.000000}%
\pgfsetstrokecolor{currentstroke}%
\pgfsetdash{}{0pt}%
\pgfpathmoveto{\pgfqpoint{5.026085in}{2.796162in}}%
\pgfpathlineto{\pgfqpoint{5.039916in}{2.805844in}}%
\pgfpathlineto{\pgfqpoint{5.053763in}{2.815685in}}%
\pgfpathlineto{\pgfqpoint{5.067626in}{2.825685in}}%
\pgfpathlineto{\pgfqpoint{5.081507in}{2.835844in}}%
\pgfpathlineto{\pgfqpoint{5.088823in}{2.841845in}}%
\pgfpathlineto{\pgfqpoint{5.096132in}{2.847763in}}%
\pgfpathlineto{\pgfqpoint{5.103433in}{2.853602in}}%
\pgfpathlineto{\pgfqpoint{5.110727in}{2.859365in}}%
\pgfpathlineto{\pgfqpoint{5.096859in}{2.849454in}}%
\pgfpathlineto{\pgfqpoint{5.083009in}{2.839702in}}%
\pgfpathlineto{\pgfqpoint{5.069174in}{2.830108in}}%
\pgfpathlineto{\pgfqpoint{5.055356in}{2.820672in}}%
\pgfpathlineto{\pgfqpoint{5.048049in}{2.814652in}}%
\pgfpathlineto{\pgfqpoint{5.040735in}{2.808562in}}%
\pgfpathlineto{\pgfqpoint{5.033414in}{2.802399in}}%
\pgfpathlineto{\pgfqpoint{5.026085in}{2.796162in}}%
\pgfpathclose%
\pgfusepath{fill}%
\end{pgfscope}%
\begin{pgfscope}%
\pgfpathrectangle{\pgfqpoint{1.254980in}{0.150000in}}{\pgfqpoint{5.490039in}{5.490039in}}%
\pgfusepath{clip}%
\pgfsetbuttcap%
\pgfsetroundjoin%
\definecolor{currentfill}{rgb}{0.281924,0.089666,0.412415}%
\pgfsetfillcolor{currentfill}%
\pgfsetfillopacity{0.700000}%
\pgfsetlinewidth{0.000000pt}%
\definecolor{currentstroke}{rgb}{0.000000,0.000000,0.000000}%
\pgfsetstrokecolor{currentstroke}%
\pgfsetdash{}{0pt}%
\pgfpathmoveto{\pgfqpoint{2.895499in}{1.881939in}}%
\pgfpathlineto{\pgfqpoint{2.908771in}{1.870713in}}%
\pgfpathlineto{\pgfqpoint{2.922040in}{1.859698in}}%
\pgfpathlineto{\pgfqpoint{2.935307in}{1.848894in}}%
\pgfpathlineto{\pgfqpoint{2.948572in}{1.838298in}}%
\pgfpathlineto{\pgfqpoint{2.956795in}{1.842477in}}%
\pgfpathlineto{\pgfqpoint{2.965007in}{1.846828in}}%
\pgfpathlineto{\pgfqpoint{2.973208in}{1.851347in}}%
\pgfpathlineto{\pgfqpoint{2.981398in}{1.856030in}}%
\pgfpathlineto{\pgfqpoint{2.968163in}{1.866180in}}%
\pgfpathlineto{\pgfqpoint{2.954927in}{1.876539in}}%
\pgfpathlineto{\pgfqpoint{2.941688in}{1.887107in}}%
\pgfpathlineto{\pgfqpoint{2.928446in}{1.897885in}}%
\pgfpathlineto{\pgfqpoint{2.920227in}{1.893638in}}%
\pgfpathlineto{\pgfqpoint{2.911996in}{1.889562in}}%
\pgfpathlineto{\pgfqpoint{2.903753in}{1.885660in}}%
\pgfpathlineto{\pgfqpoint{2.895499in}{1.881939in}}%
\pgfpathclose%
\pgfusepath{fill}%
\end{pgfscope}%
\begin{pgfscope}%
\pgfpathrectangle{\pgfqpoint{1.254980in}{0.150000in}}{\pgfqpoint{5.490039in}{5.490039in}}%
\pgfusepath{clip}%
\pgfsetbuttcap%
\pgfsetroundjoin%
\definecolor{currentfill}{rgb}{0.140210,0.665859,0.513427}%
\pgfsetfillcolor{currentfill}%
\pgfsetfillopacity{0.700000}%
\pgfsetlinewidth{0.000000pt}%
\definecolor{currentstroke}{rgb}{0.000000,0.000000,0.000000}%
\pgfsetstrokecolor{currentstroke}%
\pgfsetdash{}{0pt}%
\pgfpathmoveto{\pgfqpoint{5.647133in}{3.227785in}}%
\pgfpathlineto{\pgfqpoint{5.661312in}{3.239050in}}%
\pgfpathlineto{\pgfqpoint{5.675511in}{3.250471in}}%
\pgfpathlineto{\pgfqpoint{5.689729in}{3.262048in}}%
\pgfpathlineto{\pgfqpoint{5.703966in}{3.273782in}}%
\pgfpathlineto{\pgfqpoint{5.710907in}{3.275655in}}%
\pgfpathlineto{\pgfqpoint{5.717841in}{3.277527in}}%
\pgfpathlineto{\pgfqpoint{5.724769in}{3.279404in}}%
\pgfpathlineto{\pgfqpoint{5.731691in}{3.281290in}}%
\pgfpathlineto{\pgfqpoint{5.717481in}{3.270048in}}%
\pgfpathlineto{\pgfqpoint{5.703291in}{3.258961in}}%
\pgfpathlineto{\pgfqpoint{5.689120in}{3.248030in}}%
\pgfpathlineto{\pgfqpoint{5.674967in}{3.237254in}}%
\pgfpathlineto{\pgfqpoint{5.668018in}{3.234867in}}%
\pgfpathlineto{\pgfqpoint{5.661062in}{3.232497in}}%
\pgfpathlineto{\pgfqpoint{5.654101in}{3.230138in}}%
\pgfpathlineto{\pgfqpoint{5.647133in}{3.227785in}}%
\pgfpathclose%
\pgfusepath{fill}%
\end{pgfscope}%
\begin{pgfscope}%
\pgfpathrectangle{\pgfqpoint{1.254980in}{0.150000in}}{\pgfqpoint{5.490039in}{5.490039in}}%
\pgfusepath{clip}%
\pgfsetbuttcap%
\pgfsetroundjoin%
\definecolor{currentfill}{rgb}{0.280255,0.165693,0.476498}%
\pgfsetfillcolor{currentfill}%
\pgfsetfillopacity{0.700000}%
\pgfsetlinewidth{0.000000pt}%
\definecolor{currentstroke}{rgb}{0.000000,0.000000,0.000000}%
\pgfsetstrokecolor{currentstroke}%
\pgfsetdash{}{0pt}%
\pgfpathmoveto{\pgfqpoint{4.006485in}{1.988597in}}%
\pgfpathlineto{\pgfqpoint{4.019825in}{1.991520in}}%
\pgfpathlineto{\pgfqpoint{4.033175in}{1.994609in}}%
\pgfpathlineto{\pgfqpoint{4.046534in}{1.997863in}}%
\pgfpathlineto{\pgfqpoint{4.059904in}{2.001283in}}%
\pgfpathlineto{\pgfqpoint{4.067631in}{2.012306in}}%
\pgfpathlineto{\pgfqpoint{4.075353in}{2.023284in}}%
\pgfpathlineto{\pgfqpoint{4.083070in}{2.034215in}}%
\pgfpathlineto{\pgfqpoint{4.090782in}{2.045098in}}%
\pgfpathlineto{\pgfqpoint{4.077418in}{2.041517in}}%
\pgfpathlineto{\pgfqpoint{4.064064in}{2.038101in}}%
\pgfpathlineto{\pgfqpoint{4.050720in}{2.034851in}}%
\pgfpathlineto{\pgfqpoint{4.037385in}{2.031767in}}%
\pgfpathlineto{\pgfqpoint{4.029667in}{2.021035in}}%
\pgfpathlineto{\pgfqpoint{4.021945in}{2.010262in}}%
\pgfpathlineto{\pgfqpoint{4.014217in}{1.999449in}}%
\pgfpathlineto{\pgfqpoint{4.006485in}{1.988597in}}%
\pgfpathclose%
\pgfusepath{fill}%
\end{pgfscope}%
\begin{pgfscope}%
\pgfpathrectangle{\pgfqpoint{1.254980in}{0.150000in}}{\pgfqpoint{5.490039in}{5.490039in}}%
\pgfusepath{clip}%
\pgfsetbuttcap%
\pgfsetroundjoin%
\definecolor{currentfill}{rgb}{0.239346,0.300855,0.540844}%
\pgfsetfillcolor{currentfill}%
\pgfsetfillopacity{0.700000}%
\pgfsetlinewidth{0.000000pt}%
\definecolor{currentstroke}{rgb}{0.000000,0.000000,0.000000}%
\pgfsetstrokecolor{currentstroke}%
\pgfsetdash{}{0pt}%
\pgfpathmoveto{\pgfqpoint{4.374487in}{2.272110in}}%
\pgfpathlineto{\pgfqpoint{4.387979in}{2.278181in}}%
\pgfpathlineto{\pgfqpoint{4.401484in}{2.284415in}}%
\pgfpathlineto{\pgfqpoint{4.415001in}{2.290811in}}%
\pgfpathlineto{\pgfqpoint{4.428530in}{2.297369in}}%
\pgfpathlineto{\pgfqpoint{4.436137in}{2.307441in}}%
\pgfpathlineto{\pgfqpoint{4.443739in}{2.317427in}}%
\pgfpathlineto{\pgfqpoint{4.451334in}{2.327327in}}%
\pgfpathlineto{\pgfqpoint{4.458924in}{2.337142in}}%
\pgfpathlineto{\pgfqpoint{4.445399in}{2.330565in}}%
\pgfpathlineto{\pgfqpoint{4.431887in}{2.324150in}}%
\pgfpathlineto{\pgfqpoint{4.418388in}{2.317898in}}%
\pgfpathlineto{\pgfqpoint{4.404901in}{2.311807in}}%
\pgfpathlineto{\pgfqpoint{4.397305in}{2.302001in}}%
\pgfpathlineto{\pgfqpoint{4.389705in}{2.292116in}}%
\pgfpathlineto{\pgfqpoint{4.382099in}{2.282152in}}%
\pgfpathlineto{\pgfqpoint{4.374487in}{2.272110in}}%
\pgfpathclose%
\pgfusepath{fill}%
\end{pgfscope}%
\begin{pgfscope}%
\pgfpathrectangle{\pgfqpoint{1.254980in}{0.150000in}}{\pgfqpoint{5.490039in}{5.490039in}}%
\pgfusepath{clip}%
\pgfsetbuttcap%
\pgfsetroundjoin%
\definecolor{currentfill}{rgb}{0.162016,0.687316,0.499129}%
\pgfsetfillcolor{currentfill}%
\pgfsetfillopacity{0.700000}%
\pgfsetlinewidth{0.000000pt}%
\definecolor{currentstroke}{rgb}{0.000000,0.000000,0.000000}%
\pgfsetstrokecolor{currentstroke}%
\pgfsetdash{}{0pt}%
\pgfpathmoveto{\pgfqpoint{5.731691in}{3.281290in}}%
\pgfpathlineto{\pgfqpoint{5.745919in}{3.292689in}}%
\pgfpathlineto{\pgfqpoint{5.760168in}{3.304243in}}%
\pgfpathlineto{\pgfqpoint{5.774436in}{3.315953in}}%
\pgfpathlineto{\pgfqpoint{5.788723in}{3.327820in}}%
\pgfpathlineto{\pgfqpoint{5.795609in}{3.329211in}}%
\pgfpathlineto{\pgfqpoint{5.802489in}{3.330616in}}%
\pgfpathlineto{\pgfqpoint{5.809363in}{3.332042in}}%
\pgfpathlineto{\pgfqpoint{5.816232in}{3.333493in}}%
\pgfpathlineto{\pgfqpoint{5.801974in}{3.322148in}}%
\pgfpathlineto{\pgfqpoint{5.787736in}{3.310959in}}%
\pgfpathlineto{\pgfqpoint{5.773518in}{3.299925in}}%
\pgfpathlineto{\pgfqpoint{5.759318in}{3.289046in}}%
\pgfpathlineto{\pgfqpoint{5.752420in}{3.287065in}}%
\pgfpathlineto{\pgfqpoint{5.745516in}{3.285115in}}%
\pgfpathlineto{\pgfqpoint{5.738606in}{3.283192in}}%
\pgfpathlineto{\pgfqpoint{5.731691in}{3.281290in}}%
\pgfpathclose%
\pgfusepath{fill}%
\end{pgfscope}%
\begin{pgfscope}%
\pgfpathrectangle{\pgfqpoint{1.254980in}{0.150000in}}{\pgfqpoint{5.490039in}{5.490039in}}%
\pgfusepath{clip}%
\pgfsetbuttcap%
\pgfsetroundjoin%
\definecolor{currentfill}{rgb}{0.246070,0.738910,0.452024}%
\pgfsetfillcolor{currentfill}%
\pgfsetfillopacity{0.700000}%
\pgfsetlinewidth{0.000000pt}%
\definecolor{currentstroke}{rgb}{0.000000,0.000000,0.000000}%
\pgfsetstrokecolor{currentstroke}%
\pgfsetdash{}{0pt}%
\pgfpathmoveto{\pgfqpoint{5.985238in}{3.434027in}}%
\pgfpathlineto{\pgfqpoint{5.999608in}{3.445635in}}%
\pgfpathlineto{\pgfqpoint{6.013998in}{3.457397in}}%
\pgfpathlineto{\pgfqpoint{6.028408in}{3.469314in}}%
\pgfpathlineto{\pgfqpoint{6.035137in}{3.469619in}}%
\pgfpathlineto{\pgfqpoint{6.041862in}{3.469993in}}%
\pgfpathlineto{\pgfqpoint{6.048583in}{3.470442in}}%
\pgfpathlineto{\pgfqpoint{6.055300in}{3.470972in}}%
\pgfpathlineto{\pgfqpoint{6.040927in}{3.459667in}}%
\pgfpathlineto{\pgfqpoint{6.026574in}{3.448516in}}%
\pgfpathlineto{\pgfqpoint{6.012241in}{3.437519in}}%
\pgfpathlineto{\pgfqpoint{6.005495in}{3.436523in}}%
\pgfpathlineto{\pgfqpoint{5.998747in}{3.435614in}}%
\pgfpathlineto{\pgfqpoint{5.991994in}{3.434784in}}%
\pgfpathlineto{\pgfqpoint{5.985238in}{3.434027in}}%
\pgfpathclose%
\pgfusepath{fill}%
\end{pgfscope}%
\begin{pgfscope}%
\pgfpathrectangle{\pgfqpoint{1.254980in}{0.150000in}}{\pgfqpoint{5.490039in}{5.490039in}}%
\pgfusepath{clip}%
\pgfsetbuttcap%
\pgfsetroundjoin%
\definecolor{currentfill}{rgb}{0.272594,0.025563,0.353093}%
\pgfsetfillcolor{currentfill}%
\pgfsetfillopacity{0.700000}%
\pgfsetlinewidth{0.000000pt}%
\definecolor{currentstroke}{rgb}{0.000000,0.000000,0.000000}%
\pgfsetstrokecolor{currentstroke}%
\pgfsetdash{}{0pt}%
\pgfpathmoveto{\pgfqpoint{3.500481in}{1.738974in}}%
\pgfpathlineto{\pgfqpoint{3.513705in}{1.736255in}}%
\pgfpathlineto{\pgfqpoint{3.526933in}{1.733712in}}%
\pgfpathlineto{\pgfqpoint{3.540166in}{1.731345in}}%
\pgfpathlineto{\pgfqpoint{3.553405in}{1.729152in}}%
\pgfpathlineto{\pgfqpoint{3.561306in}{1.738659in}}%
\pgfpathlineto{\pgfqpoint{3.569201in}{1.748214in}}%
\pgfpathlineto{\pgfqpoint{3.577090in}{1.757816in}}%
\pgfpathlineto{\pgfqpoint{3.584973in}{1.767461in}}%
\pgfpathlineto{\pgfqpoint{3.571748in}{1.769326in}}%
\pgfpathlineto{\pgfqpoint{3.558528in}{1.771366in}}%
\pgfpathlineto{\pgfqpoint{3.545313in}{1.773581in}}%
\pgfpathlineto{\pgfqpoint{3.532103in}{1.775973in}}%
\pgfpathlineto{\pgfqpoint{3.524207in}{1.766646in}}%
\pgfpathlineto{\pgfqpoint{3.516304in}{1.757368in}}%
\pgfpathlineto{\pgfqpoint{3.508396in}{1.748143in}}%
\pgfpathlineto{\pgfqpoint{3.500481in}{1.738974in}}%
\pgfpathclose%
\pgfusepath{fill}%
\end{pgfscope}%
\begin{pgfscope}%
\pgfpathrectangle{\pgfqpoint{1.254980in}{0.150000in}}{\pgfqpoint{5.490039in}{5.490039in}}%
\pgfusepath{clip}%
\pgfsetbuttcap%
\pgfsetroundjoin%
\definecolor{currentfill}{rgb}{0.139147,0.533812,0.555298}%
\pgfsetfillcolor{currentfill}%
\pgfsetfillopacity{0.700000}%
\pgfsetlinewidth{0.000000pt}%
\definecolor{currentstroke}{rgb}{0.000000,0.000000,0.000000}%
\pgfsetstrokecolor{currentstroke}%
\pgfsetdash{}{0pt}%
\pgfpathmoveto{\pgfqpoint{5.110727in}{2.859365in}}%
\pgfpathlineto{\pgfqpoint{5.124611in}{2.869435in}}%
\pgfpathlineto{\pgfqpoint{5.138512in}{2.879663in}}%
\pgfpathlineto{\pgfqpoint{5.152430in}{2.890050in}}%
\pgfpathlineto{\pgfqpoint{5.166365in}{2.900596in}}%
\pgfpathlineto{\pgfqpoint{5.173638in}{2.906018in}}%
\pgfpathlineto{\pgfqpoint{5.180903in}{2.911362in}}%
\pgfpathlineto{\pgfqpoint{5.188161in}{2.916632in}}%
\pgfpathlineto{\pgfqpoint{5.195411in}{2.921831in}}%
\pgfpathlineto{\pgfqpoint{5.181490in}{2.911564in}}%
\pgfpathlineto{\pgfqpoint{5.167586in}{2.901455in}}%
\pgfpathlineto{\pgfqpoint{5.153699in}{2.891505in}}%
\pgfpathlineto{\pgfqpoint{5.139829in}{2.881712in}}%
\pgfpathlineto{\pgfqpoint{5.132564in}{2.876225in}}%
\pgfpathlineto{\pgfqpoint{5.125292in}{2.870673in}}%
\pgfpathlineto{\pgfqpoint{5.118013in}{2.865054in}}%
\pgfpathlineto{\pgfqpoint{5.110727in}{2.859365in}}%
\pgfpathclose%
\pgfusepath{fill}%
\end{pgfscope}%
\begin{pgfscope}%
\pgfpathrectangle{\pgfqpoint{1.254980in}{0.150000in}}{\pgfqpoint{5.490039in}{5.490039in}}%
\pgfusepath{clip}%
\pgfsetbuttcap%
\pgfsetroundjoin%
\definecolor{currentfill}{rgb}{0.268510,0.009605,0.335427}%
\pgfsetfillcolor{currentfill}%
\pgfsetfillopacity{0.700000}%
\pgfsetlinewidth{0.000000pt}%
\definecolor{currentstroke}{rgb}{0.000000,0.000000,0.000000}%
\pgfsetstrokecolor{currentstroke}%
\pgfsetdash{}{0pt}%
\pgfpathmoveto{\pgfqpoint{3.278151in}{1.723515in}}%
\pgfpathlineto{\pgfqpoint{3.291367in}{1.717889in}}%
\pgfpathlineto{\pgfqpoint{3.304585in}{1.712447in}}%
\pgfpathlineto{\pgfqpoint{3.317806in}{1.707190in}}%
\pgfpathlineto{\pgfqpoint{3.331029in}{1.702116in}}%
\pgfpathlineto{\pgfqpoint{3.339032in}{1.709930in}}%
\pgfpathlineto{\pgfqpoint{3.347027in}{1.717842in}}%
\pgfpathlineto{\pgfqpoint{3.355015in}{1.725846in}}%
\pgfpathlineto{\pgfqpoint{3.362995in}{1.733941in}}%
\pgfpathlineto{\pgfqpoint{3.349790in}{1.738632in}}%
\pgfpathlineto{\pgfqpoint{3.336588in}{1.743506in}}%
\pgfpathlineto{\pgfqpoint{3.323389in}{1.748563in}}%
\pgfpathlineto{\pgfqpoint{3.310193in}{1.753806in}}%
\pgfpathlineto{\pgfqpoint{3.302194in}{1.746085in}}%
\pgfpathlineto{\pgfqpoint{3.294187in}{1.738460in}}%
\pgfpathlineto{\pgfqpoint{3.286173in}{1.730936in}}%
\pgfpathlineto{\pgfqpoint{3.278151in}{1.723515in}}%
\pgfpathclose%
\pgfusepath{fill}%
\end{pgfscope}%
\begin{pgfscope}%
\pgfpathrectangle{\pgfqpoint{1.254980in}{0.150000in}}{\pgfqpoint{5.490039in}{5.490039in}}%
\pgfusepath{clip}%
\pgfsetbuttcap%
\pgfsetroundjoin%
\definecolor{currentfill}{rgb}{0.185783,0.704891,0.485273}%
\pgfsetfillcolor{currentfill}%
\pgfsetfillopacity{0.700000}%
\pgfsetlinewidth{0.000000pt}%
\definecolor{currentstroke}{rgb}{0.000000,0.000000,0.000000}%
\pgfsetstrokecolor{currentstroke}%
\pgfsetdash{}{0pt}%
\pgfpathmoveto{\pgfqpoint{5.816232in}{3.333493in}}%
\pgfpathlineto{\pgfqpoint{5.830509in}{3.344993in}}%
\pgfpathlineto{\pgfqpoint{5.844805in}{3.356648in}}%
\pgfpathlineto{\pgfqpoint{5.859122in}{3.368459in}}%
\pgfpathlineto{\pgfqpoint{5.873459in}{3.380426in}}%
\pgfpathlineto{\pgfqpoint{5.880290in}{3.381367in}}%
\pgfpathlineto{\pgfqpoint{5.887115in}{3.382338in}}%
\pgfpathlineto{\pgfqpoint{5.893935in}{3.383346in}}%
\pgfpathlineto{\pgfqpoint{5.900749in}{3.384397in}}%
\pgfpathlineto{\pgfqpoint{5.886445in}{3.372982in}}%
\pgfpathlineto{\pgfqpoint{5.872160in}{3.361723in}}%
\pgfpathlineto{\pgfqpoint{5.857896in}{3.350618in}}%
\pgfpathlineto{\pgfqpoint{5.843650in}{3.339668in}}%
\pgfpathlineto{\pgfqpoint{5.836803in}{3.338057in}}%
\pgfpathlineto{\pgfqpoint{5.829951in}{3.336494in}}%
\pgfpathlineto{\pgfqpoint{5.823094in}{3.334975in}}%
\pgfpathlineto{\pgfqpoint{5.816232in}{3.333493in}}%
\pgfpathclose%
\pgfusepath{fill}%
\end{pgfscope}%
\begin{pgfscope}%
\pgfpathrectangle{\pgfqpoint{1.254980in}{0.150000in}}{\pgfqpoint{5.490039in}{5.490039in}}%
\pgfusepath{clip}%
\pgfsetbuttcap%
\pgfsetroundjoin%
\definecolor{currentfill}{rgb}{0.188923,0.410910,0.556326}%
\pgfsetfillcolor{currentfill}%
\pgfsetfillopacity{0.700000}%
\pgfsetlinewidth{0.000000pt}%
\definecolor{currentstroke}{rgb}{0.000000,0.000000,0.000000}%
\pgfsetstrokecolor{currentstroke}%
\pgfsetdash{}{0pt}%
\pgfpathmoveto{\pgfqpoint{2.305523in}{2.610589in}}%
\pgfpathlineto{\pgfqpoint{2.319142in}{2.588279in}}%
\pgfpathlineto{\pgfqpoint{2.332746in}{2.566271in}}%
\pgfpathlineto{\pgfqpoint{2.346338in}{2.544563in}}%
\pgfpathlineto{\pgfqpoint{2.359916in}{2.523152in}}%
\pgfpathlineto{\pgfqpoint{2.368567in}{2.522161in}}%
\pgfpathlineto{\pgfqpoint{2.377200in}{2.521426in}}%
\pgfpathlineto{\pgfqpoint{2.385816in}{2.520944in}}%
\pgfpathlineto{\pgfqpoint{2.394414in}{2.520709in}}%
\pgfpathlineto{\pgfqpoint{2.380884in}{2.541643in}}%
\pgfpathlineto{\pgfqpoint{2.367340in}{2.562873in}}%
\pgfpathlineto{\pgfqpoint{2.353784in}{2.584400in}}%
\pgfpathlineto{\pgfqpoint{2.340214in}{2.606230in}}%
\pgfpathlineto{\pgfqpoint{2.331569in}{2.606932in}}%
\pgfpathlineto{\pgfqpoint{2.322905in}{2.607889in}}%
\pgfpathlineto{\pgfqpoint{2.314224in}{2.609107in}}%
\pgfpathlineto{\pgfqpoint{2.305523in}{2.610589in}}%
\pgfpathclose%
\pgfusepath{fill}%
\end{pgfscope}%
\begin{pgfscope}%
\pgfpathrectangle{\pgfqpoint{1.254980in}{0.150000in}}{\pgfqpoint{5.490039in}{5.490039in}}%
\pgfusepath{clip}%
\pgfsetbuttcap%
\pgfsetroundjoin%
\definecolor{currentfill}{rgb}{0.274128,0.199721,0.498911}%
\pgfsetfillcolor{currentfill}%
\pgfsetfillopacity{0.700000}%
\pgfsetlinewidth{0.000000pt}%
\definecolor{currentstroke}{rgb}{0.000000,0.000000,0.000000}%
\pgfsetstrokecolor{currentstroke}%
\pgfsetdash{}{0pt}%
\pgfpathmoveto{\pgfqpoint{4.090782in}{2.045098in}}%
\pgfpathlineto{\pgfqpoint{4.104156in}{2.048844in}}%
\pgfpathlineto{\pgfqpoint{4.117541in}{2.052754in}}%
\pgfpathlineto{\pgfqpoint{4.130936in}{2.056829in}}%
\pgfpathlineto{\pgfqpoint{4.144342in}{2.061068in}}%
\pgfpathlineto{\pgfqpoint{4.152044in}{2.072046in}}%
\pgfpathlineto{\pgfqpoint{4.159742in}{2.082966in}}%
\pgfpathlineto{\pgfqpoint{4.167435in}{2.093828in}}%
\pgfpathlineto{\pgfqpoint{4.175123in}{2.104631in}}%
\pgfpathlineto{\pgfqpoint{4.161722in}{2.100258in}}%
\pgfpathlineto{\pgfqpoint{4.148331in}{2.096050in}}%
\pgfpathlineto{\pgfqpoint{4.134952in}{2.092006in}}%
\pgfpathlineto{\pgfqpoint{4.121583in}{2.088127in}}%
\pgfpathlineto{\pgfqpoint{4.113890in}{2.077447in}}%
\pgfpathlineto{\pgfqpoint{4.106192in}{2.066715in}}%
\pgfpathlineto{\pgfqpoint{4.098490in}{2.055931in}}%
\pgfpathlineto{\pgfqpoint{4.090782in}{2.045098in}}%
\pgfpathclose%
\pgfusepath{fill}%
\end{pgfscope}%
\begin{pgfscope}%
\pgfpathrectangle{\pgfqpoint{1.254980in}{0.150000in}}{\pgfqpoint{5.490039in}{5.490039in}}%
\pgfusepath{clip}%
\pgfsetbuttcap%
\pgfsetroundjoin%
\definecolor{currentfill}{rgb}{0.182256,0.426184,0.557120}%
\pgfsetfillcolor{currentfill}%
\pgfsetfillopacity{0.700000}%
\pgfsetlinewidth{0.000000pt}%
\definecolor{currentstroke}{rgb}{0.000000,0.000000,0.000000}%
\pgfsetstrokecolor{currentstroke}%
\pgfsetdash{}{0pt}%
\pgfpathmoveto{\pgfqpoint{4.742646in}{2.572003in}}%
\pgfpathlineto{\pgfqpoint{4.756327in}{2.580459in}}%
\pgfpathlineto{\pgfqpoint{4.770024in}{2.589075in}}%
\pgfpathlineto{\pgfqpoint{4.783735in}{2.597852in}}%
\pgfpathlineto{\pgfqpoint{4.797462in}{2.606789in}}%
\pgfpathlineto{\pgfqpoint{4.804923in}{2.614801in}}%
\pgfpathlineto{\pgfqpoint{4.812376in}{2.622715in}}%
\pgfpathlineto{\pgfqpoint{4.819823in}{2.630534in}}%
\pgfpathlineto{\pgfqpoint{4.827264in}{2.638260in}}%
\pgfpathlineto{\pgfqpoint{4.813545in}{2.629452in}}%
\pgfpathlineto{\pgfqpoint{4.799841in}{2.620803in}}%
\pgfpathlineto{\pgfqpoint{4.786153in}{2.612314in}}%
\pgfpathlineto{\pgfqpoint{4.772479in}{2.603985in}}%
\pgfpathlineto{\pgfqpoint{4.765031in}{2.596121in}}%
\pgfpathlineto{\pgfqpoint{4.757576in}{2.588170in}}%
\pgfpathlineto{\pgfqpoint{4.750114in}{2.580131in}}%
\pgfpathlineto{\pgfqpoint{4.742646in}{2.572003in}}%
\pgfpathclose%
\pgfusepath{fill}%
\end{pgfscope}%
\begin{pgfscope}%
\pgfpathrectangle{\pgfqpoint{1.254980in}{0.150000in}}{\pgfqpoint{5.490039in}{5.490039in}}%
\pgfusepath{clip}%
\pgfsetbuttcap%
\pgfsetroundjoin%
\definecolor{currentfill}{rgb}{0.272594,0.025563,0.353093}%
\pgfsetfillcolor{currentfill}%
\pgfsetfillopacity{0.700000}%
\pgfsetlinewidth{0.000000pt}%
\definecolor{currentstroke}{rgb}{0.000000,0.000000,0.000000}%
\pgfsetstrokecolor{currentstroke}%
\pgfsetdash{}{0pt}%
\pgfpathmoveto{\pgfqpoint{3.140137in}{1.750028in}}%
\pgfpathlineto{\pgfqpoint{3.153364in}{1.742477in}}%
\pgfpathlineto{\pgfqpoint{3.166592in}{1.735118in}}%
\pgfpathlineto{\pgfqpoint{3.179820in}{1.727950in}}%
\pgfpathlineto{\pgfqpoint{3.193050in}{1.720973in}}%
\pgfpathlineto{\pgfqpoint{3.201127in}{1.727516in}}%
\pgfpathlineto{\pgfqpoint{3.209195in}{1.734185in}}%
\pgfpathlineto{\pgfqpoint{3.217255in}{1.740977in}}%
\pgfpathlineto{\pgfqpoint{3.225306in}{1.747888in}}%
\pgfpathlineto{\pgfqpoint{3.212100in}{1.754452in}}%
\pgfpathlineto{\pgfqpoint{3.198894in}{1.761207in}}%
\pgfpathlineto{\pgfqpoint{3.185689in}{1.768153in}}%
\pgfpathlineto{\pgfqpoint{3.172486in}{1.775290in}}%
\pgfpathlineto{\pgfqpoint{3.164412in}{1.768782in}}%
\pgfpathlineto{\pgfqpoint{3.156330in}{1.762400in}}%
\pgfpathlineto{\pgfqpoint{3.148238in}{1.756147in}}%
\pgfpathlineto{\pgfqpoint{3.140137in}{1.750028in}}%
\pgfpathclose%
\pgfusepath{fill}%
\end{pgfscope}%
\begin{pgfscope}%
\pgfpathrectangle{\pgfqpoint{1.254980in}{0.150000in}}{\pgfqpoint{5.490039in}{5.490039in}}%
\pgfusepath{clip}%
\pgfsetbuttcap%
\pgfsetroundjoin%
\definecolor{currentfill}{rgb}{0.280267,0.073417,0.397163}%
\pgfsetfillcolor{currentfill}%
\pgfsetfillopacity{0.700000}%
\pgfsetlinewidth{0.000000pt}%
\definecolor{currentstroke}{rgb}{0.000000,0.000000,0.000000}%
\pgfsetstrokecolor{currentstroke}%
\pgfsetdash{}{0pt}%
\pgfpathmoveto{\pgfqpoint{2.948572in}{1.838298in}}%
\pgfpathlineto{\pgfqpoint{2.961834in}{1.827909in}}%
\pgfpathlineto{\pgfqpoint{2.975095in}{1.817726in}}%
\pgfpathlineto{\pgfqpoint{2.988354in}{1.807748in}}%
\pgfpathlineto{\pgfqpoint{3.001611in}{1.797974in}}%
\pgfpathlineto{\pgfqpoint{3.009805in}{1.802609in}}%
\pgfpathlineto{\pgfqpoint{3.017988in}{1.807409in}}%
\pgfpathlineto{\pgfqpoint{3.026161in}{1.812370in}}%
\pgfpathlineto{\pgfqpoint{3.034323in}{1.817486in}}%
\pgfpathlineto{\pgfqpoint{3.021094in}{1.826816in}}%
\pgfpathlineto{\pgfqpoint{3.007863in}{1.836349in}}%
\pgfpathlineto{\pgfqpoint{2.994632in}{1.846087in}}%
\pgfpathlineto{\pgfqpoint{2.981398in}{1.856030in}}%
\pgfpathlineto{\pgfqpoint{2.973208in}{1.851347in}}%
\pgfpathlineto{\pgfqpoint{2.965007in}{1.846828in}}%
\pgfpathlineto{\pgfqpoint{2.956795in}{1.842477in}}%
\pgfpathlineto{\pgfqpoint{2.948572in}{1.838298in}}%
\pgfpathclose%
\pgfusepath{fill}%
\end{pgfscope}%
\begin{pgfscope}%
\pgfpathrectangle{\pgfqpoint{1.254980in}{0.150000in}}{\pgfqpoint{5.490039in}{5.490039in}}%
\pgfusepath{clip}%
\pgfsetbuttcap%
\pgfsetroundjoin%
\definecolor{currentfill}{rgb}{0.220124,0.725509,0.466226}%
\pgfsetfillcolor{currentfill}%
\pgfsetfillopacity{0.700000}%
\pgfsetlinewidth{0.000000pt}%
\definecolor{currentstroke}{rgb}{0.000000,0.000000,0.000000}%
\pgfsetstrokecolor{currentstroke}%
\pgfsetdash{}{0pt}%
\pgfpathmoveto{\pgfqpoint{5.900749in}{3.384397in}}%
\pgfpathlineto{\pgfqpoint{5.915073in}{3.395966in}}%
\pgfpathlineto{\pgfqpoint{5.929417in}{3.407691in}}%
\pgfpathlineto{\pgfqpoint{5.943782in}{3.419571in}}%
\pgfpathlineto{\pgfqpoint{5.958166in}{3.431606in}}%
\pgfpathlineto{\pgfqpoint{5.964942in}{3.432133in}}%
\pgfpathlineto{\pgfqpoint{5.971712in}{3.432708in}}%
\pgfpathlineto{\pgfqpoint{5.978477in}{3.433338in}}%
\pgfpathlineto{\pgfqpoint{5.985238in}{3.434027in}}%
\pgfpathlineto{\pgfqpoint{5.970888in}{3.422575in}}%
\pgfpathlineto{\pgfqpoint{5.956559in}{3.411277in}}%
\pgfpathlineto{\pgfqpoint{5.942249in}{3.400133in}}%
\pgfpathlineto{\pgfqpoint{5.927959in}{3.389143in}}%
\pgfpathlineto{\pgfqpoint{5.921163in}{3.387862in}}%
\pgfpathlineto{\pgfqpoint{5.914363in}{3.386649in}}%
\pgfpathlineto{\pgfqpoint{5.907558in}{3.385495in}}%
\pgfpathlineto{\pgfqpoint{5.900749in}{3.384397in}}%
\pgfpathclose%
\pgfusepath{fill}%
\end{pgfscope}%
\begin{pgfscope}%
\pgfpathrectangle{\pgfqpoint{1.254980in}{0.150000in}}{\pgfqpoint{5.490039in}{5.490039in}}%
\pgfusepath{clip}%
\pgfsetbuttcap%
\pgfsetroundjoin%
\definecolor{currentfill}{rgb}{0.223925,0.334994,0.548053}%
\pgfsetfillcolor{currentfill}%
\pgfsetfillopacity{0.700000}%
\pgfsetlinewidth{0.000000pt}%
\definecolor{currentstroke}{rgb}{0.000000,0.000000,0.000000}%
\pgfsetstrokecolor{currentstroke}%
\pgfsetdash{}{0pt}%
\pgfpathmoveto{\pgfqpoint{4.458924in}{2.337142in}}%
\pgfpathlineto{\pgfqpoint{4.472462in}{2.343881in}}%
\pgfpathlineto{\pgfqpoint{4.486013in}{2.350782in}}%
\pgfpathlineto{\pgfqpoint{4.499578in}{2.357844in}}%
\pgfpathlineto{\pgfqpoint{4.513155in}{2.365069in}}%
\pgfpathlineto{\pgfqpoint{4.520735in}{2.374799in}}%
\pgfpathlineto{\pgfqpoint{4.528309in}{2.384438in}}%
\pgfpathlineto{\pgfqpoint{4.535877in}{2.393985in}}%
\pgfpathlineto{\pgfqpoint{4.543439in}{2.403441in}}%
\pgfpathlineto{\pgfqpoint{4.529866in}{2.396227in}}%
\pgfpathlineto{\pgfqpoint{4.516307in}{2.389175in}}%
\pgfpathlineto{\pgfqpoint{4.502761in}{2.382284in}}%
\pgfpathlineto{\pgfqpoint{4.489228in}{2.375555in}}%
\pgfpathlineto{\pgfqpoint{4.481661in}{2.366078in}}%
\pgfpathlineto{\pgfqpoint{4.474088in}{2.356517in}}%
\pgfpathlineto{\pgfqpoint{4.466509in}{2.346872in}}%
\pgfpathlineto{\pgfqpoint{4.458924in}{2.337142in}}%
\pgfpathclose%
\pgfusepath{fill}%
\end{pgfscope}%
\begin{pgfscope}%
\pgfpathrectangle{\pgfqpoint{1.254980in}{0.150000in}}{\pgfqpoint{5.490039in}{5.490039in}}%
\pgfusepath{clip}%
\pgfsetbuttcap%
\pgfsetroundjoin%
\definecolor{currentfill}{rgb}{0.269944,0.014625,0.341379}%
\pgfsetfillcolor{currentfill}%
\pgfsetfillopacity{0.700000}%
\pgfsetlinewidth{0.000000pt}%
\definecolor{currentstroke}{rgb}{0.000000,0.000000,0.000000}%
\pgfsetstrokecolor{currentstroke}%
\pgfsetdash{}{0pt}%
\pgfpathmoveto{\pgfqpoint{3.415847in}{1.716996in}}%
\pgfpathlineto{\pgfqpoint{3.429069in}{1.713211in}}%
\pgfpathlineto{\pgfqpoint{3.442295in}{1.709604in}}%
\pgfpathlineto{\pgfqpoint{3.455525in}{1.706175in}}%
\pgfpathlineto{\pgfqpoint{3.468759in}{1.702924in}}%
\pgfpathlineto{\pgfqpoint{3.476699in}{1.711836in}}%
\pgfpathlineto{\pgfqpoint{3.484633in}{1.720818in}}%
\pgfpathlineto{\pgfqpoint{3.492560in}{1.729865in}}%
\pgfpathlineto{\pgfqpoint{3.500481in}{1.738974in}}%
\pgfpathlineto{\pgfqpoint{3.487262in}{1.741870in}}%
\pgfpathlineto{\pgfqpoint{3.474048in}{1.744944in}}%
\pgfpathlineto{\pgfqpoint{3.460837in}{1.748196in}}%
\pgfpathlineto{\pgfqpoint{3.447631in}{1.751626in}}%
\pgfpathlineto{\pgfqpoint{3.439695in}{1.742861in}}%
\pgfpathlineto{\pgfqpoint{3.431753in}{1.734166in}}%
\pgfpathlineto{\pgfqpoint{3.423803in}{1.725543in}}%
\pgfpathlineto{\pgfqpoint{3.415847in}{1.716996in}}%
\pgfpathclose%
\pgfusepath{fill}%
\end{pgfscope}%
\begin{pgfscope}%
\pgfpathrectangle{\pgfqpoint{1.254980in}{0.150000in}}{\pgfqpoint{5.490039in}{5.490039in}}%
\pgfusepath{clip}%
\pgfsetbuttcap%
\pgfsetroundjoin%
\definecolor{currentfill}{rgb}{0.129933,0.559582,0.551864}%
\pgfsetfillcolor{currentfill}%
\pgfsetfillopacity{0.700000}%
\pgfsetlinewidth{0.000000pt}%
\definecolor{currentstroke}{rgb}{0.000000,0.000000,0.000000}%
\pgfsetstrokecolor{currentstroke}%
\pgfsetdash{}{0pt}%
\pgfpathmoveto{\pgfqpoint{5.195411in}{2.921831in}}%
\pgfpathlineto{\pgfqpoint{5.209349in}{2.932257in}}%
\pgfpathlineto{\pgfqpoint{5.223304in}{2.942841in}}%
\pgfpathlineto{\pgfqpoint{5.237277in}{2.953583in}}%
\pgfpathlineto{\pgfqpoint{5.251268in}{2.964485in}}%
\pgfpathlineto{\pgfqpoint{5.258495in}{2.969318in}}%
\pgfpathlineto{\pgfqpoint{5.265714in}{2.974080in}}%
\pgfpathlineto{\pgfqpoint{5.272926in}{2.978775in}}%
\pgfpathlineto{\pgfqpoint{5.280130in}{2.983405in}}%
\pgfpathlineto{\pgfqpoint{5.266155in}{2.972814in}}%
\pgfpathlineto{\pgfqpoint{5.252198in}{2.962380in}}%
\pgfpathlineto{\pgfqpoint{5.238259in}{2.952104in}}%
\pgfpathlineto{\pgfqpoint{5.224336in}{2.941987in}}%
\pgfpathlineto{\pgfqpoint{5.217116in}{2.937037in}}%
\pgfpathlineto{\pgfqpoint{5.209888in}{2.932030in}}%
\pgfpathlineto{\pgfqpoint{5.202653in}{2.926963in}}%
\pgfpathlineto{\pgfqpoint{5.195411in}{2.921831in}}%
\pgfpathclose%
\pgfusepath{fill}%
\end{pgfscope}%
\begin{pgfscope}%
\pgfpathrectangle{\pgfqpoint{1.254980in}{0.150000in}}{\pgfqpoint{5.490039in}{5.490039in}}%
\pgfusepath{clip}%
\pgfsetbuttcap%
\pgfsetroundjoin%
\definecolor{currentfill}{rgb}{0.266580,0.228262,0.514349}%
\pgfsetfillcolor{currentfill}%
\pgfsetfillopacity{0.700000}%
\pgfsetlinewidth{0.000000pt}%
\definecolor{currentstroke}{rgb}{0.000000,0.000000,0.000000}%
\pgfsetstrokecolor{currentstroke}%
\pgfsetdash{}{0pt}%
\pgfpathmoveto{\pgfqpoint{4.175123in}{2.104631in}}%
\pgfpathlineto{\pgfqpoint{4.188535in}{2.109167in}}%
\pgfpathlineto{\pgfqpoint{4.201958in}{2.113868in}}%
\pgfpathlineto{\pgfqpoint{4.215392in}{2.118732in}}%
\pgfpathlineto{\pgfqpoint{4.228837in}{2.123759in}}%
\pgfpathlineto{\pgfqpoint{4.236516in}{2.134617in}}%
\pgfpathlineto{\pgfqpoint{4.244189in}{2.145407in}}%
\pgfpathlineto{\pgfqpoint{4.251858in}{2.156129in}}%
\pgfpathlineto{\pgfqpoint{4.259521in}{2.166781in}}%
\pgfpathlineto{\pgfqpoint{4.246080in}{2.161649in}}%
\pgfpathlineto{\pgfqpoint{4.232650in}{2.156680in}}%
\pgfpathlineto{\pgfqpoint{4.219232in}{2.151875in}}%
\pgfpathlineto{\pgfqpoint{4.205825in}{2.147233in}}%
\pgfpathlineto{\pgfqpoint{4.198157in}{2.136675in}}%
\pgfpathlineto{\pgfqpoint{4.190484in}{2.126055in}}%
\pgfpathlineto{\pgfqpoint{4.182806in}{2.115373in}}%
\pgfpathlineto{\pgfqpoint{4.175123in}{2.104631in}}%
\pgfpathclose%
\pgfusepath{fill}%
\end{pgfscope}%
\begin{pgfscope}%
\pgfpathrectangle{\pgfqpoint{1.254980in}{0.150000in}}{\pgfqpoint{5.490039in}{5.490039in}}%
\pgfusepath{clip}%
\pgfsetbuttcap%
\pgfsetroundjoin%
\definecolor{currentfill}{rgb}{0.169646,0.456262,0.558030}%
\pgfsetfillcolor{currentfill}%
\pgfsetfillopacity{0.700000}%
\pgfsetlinewidth{0.000000pt}%
\definecolor{currentstroke}{rgb}{0.000000,0.000000,0.000000}%
\pgfsetstrokecolor{currentstroke}%
\pgfsetdash{}{0pt}%
\pgfpathmoveto{\pgfqpoint{4.827264in}{2.638260in}}%
\pgfpathlineto{\pgfqpoint{4.840998in}{2.647229in}}%
\pgfpathlineto{\pgfqpoint{4.854747in}{2.656358in}}%
\pgfpathlineto{\pgfqpoint{4.868512in}{2.665647in}}%
\pgfpathlineto{\pgfqpoint{4.882293in}{2.675096in}}%
\pgfpathlineto{\pgfqpoint{4.889718in}{2.682583in}}%
\pgfpathlineto{\pgfqpoint{4.897135in}{2.689974in}}%
\pgfpathlineto{\pgfqpoint{4.904546in}{2.697269in}}%
\pgfpathlineto{\pgfqpoint{4.911950in}{2.704472in}}%
\pgfpathlineto{\pgfqpoint{4.898178in}{2.695181in}}%
\pgfpathlineto{\pgfqpoint{4.884422in}{2.686050in}}%
\pgfpathlineto{\pgfqpoint{4.870681in}{2.677079in}}%
\pgfpathlineto{\pgfqpoint{4.856956in}{2.668268in}}%
\pgfpathlineto{\pgfqpoint{4.849543in}{2.660896in}}%
\pgfpathlineto{\pgfqpoint{4.842124in}{2.653439in}}%
\pgfpathlineto{\pgfqpoint{4.834697in}{2.645895in}}%
\pgfpathlineto{\pgfqpoint{4.827264in}{2.638260in}}%
\pgfpathclose%
\pgfusepath{fill}%
\end{pgfscope}%
\begin{pgfscope}%
\pgfpathrectangle{\pgfqpoint{1.254980in}{0.150000in}}{\pgfqpoint{5.490039in}{5.490039in}}%
\pgfusepath{clip}%
\pgfsetbuttcap%
\pgfsetroundjoin%
\definecolor{currentfill}{rgb}{0.172719,0.448791,0.557885}%
\pgfsetfillcolor{currentfill}%
\pgfsetfillopacity{0.700000}%
\pgfsetlinewidth{0.000000pt}%
\definecolor{currentstroke}{rgb}{0.000000,0.000000,0.000000}%
\pgfsetstrokecolor{currentstroke}%
\pgfsetdash{}{0pt}%
\pgfpathmoveto{\pgfqpoint{2.250905in}{2.702913in}}%
\pgfpathlineto{\pgfqpoint{2.264582in}{2.679363in}}%
\pgfpathlineto{\pgfqpoint{2.278244in}{2.656128in}}%
\pgfpathlineto{\pgfqpoint{2.291891in}{2.633204in}}%
\pgfpathlineto{\pgfqpoint{2.305523in}{2.610589in}}%
\pgfpathlineto{\pgfqpoint{2.314224in}{2.609107in}}%
\pgfpathlineto{\pgfqpoint{2.322905in}{2.607889in}}%
\pgfpathlineto{\pgfqpoint{2.331569in}{2.606932in}}%
\pgfpathlineto{\pgfqpoint{2.340214in}{2.606230in}}%
\pgfpathlineto{\pgfqpoint{2.326631in}{2.628363in}}%
\pgfpathlineto{\pgfqpoint{2.313034in}{2.650804in}}%
\pgfpathlineto{\pgfqpoint{2.299422in}{2.673555in}}%
\pgfpathlineto{\pgfqpoint{2.285796in}{2.696619in}}%
\pgfpathlineto{\pgfqpoint{2.277102in}{2.697793in}}%
\pgfpathlineto{\pgfqpoint{2.268389in}{2.699230in}}%
\pgfpathlineto{\pgfqpoint{2.259656in}{2.700935in}}%
\pgfpathlineto{\pgfqpoint{2.250905in}{2.702913in}}%
\pgfpathclose%
\pgfusepath{fill}%
\end{pgfscope}%
\begin{pgfscope}%
\pgfpathrectangle{\pgfqpoint{1.254980in}{0.150000in}}{\pgfqpoint{5.490039in}{5.490039in}}%
\pgfusepath{clip}%
\pgfsetbuttcap%
\pgfsetroundjoin%
\definecolor{currentfill}{rgb}{0.277941,0.056324,0.381191}%
\pgfsetfillcolor{currentfill}%
\pgfsetfillopacity{0.700000}%
\pgfsetlinewidth{0.000000pt}%
\definecolor{currentstroke}{rgb}{0.000000,0.000000,0.000000}%
\pgfsetstrokecolor{currentstroke}%
\pgfsetdash{}{0pt}%
\pgfpathmoveto{\pgfqpoint{3.001611in}{1.797974in}}%
\pgfpathlineto{\pgfqpoint{3.014868in}{1.788402in}}%
\pgfpathlineto{\pgfqpoint{3.028123in}{1.779032in}}%
\pgfpathlineto{\pgfqpoint{3.041377in}{1.769861in}}%
\pgfpathlineto{\pgfqpoint{3.054630in}{1.760890in}}%
\pgfpathlineto{\pgfqpoint{3.062796in}{1.765979in}}%
\pgfpathlineto{\pgfqpoint{3.070951in}{1.771226in}}%
\pgfpathlineto{\pgfqpoint{3.079097in}{1.776627in}}%
\pgfpathlineto{\pgfqpoint{3.087232in}{1.782176in}}%
\pgfpathlineto{\pgfqpoint{3.074006in}{1.790705in}}%
\pgfpathlineto{\pgfqpoint{3.060779in}{1.799432in}}%
\pgfpathlineto{\pgfqpoint{3.047551in}{1.808359in}}%
\pgfpathlineto{\pgfqpoint{3.034323in}{1.817486in}}%
\pgfpathlineto{\pgfqpoint{3.026161in}{1.812370in}}%
\pgfpathlineto{\pgfqpoint{3.017988in}{1.807409in}}%
\pgfpathlineto{\pgfqpoint{3.009805in}{1.802609in}}%
\pgfpathlineto{\pgfqpoint{3.001611in}{1.797974in}}%
\pgfpathclose%
\pgfusepath{fill}%
\end{pgfscope}%
\begin{pgfscope}%
\pgfpathrectangle{\pgfqpoint{1.254980in}{0.150000in}}{\pgfqpoint{5.490039in}{5.490039in}}%
\pgfusepath{clip}%
\pgfsetbuttcap%
\pgfsetroundjoin%
\definecolor{currentfill}{rgb}{0.210503,0.363727,0.552206}%
\pgfsetfillcolor{currentfill}%
\pgfsetfillopacity{0.700000}%
\pgfsetlinewidth{0.000000pt}%
\definecolor{currentstroke}{rgb}{0.000000,0.000000,0.000000}%
\pgfsetstrokecolor{currentstroke}%
\pgfsetdash{}{0pt}%
\pgfpathmoveto{\pgfqpoint{4.543439in}{2.403441in}}%
\pgfpathlineto{\pgfqpoint{4.557025in}{2.410817in}}%
\pgfpathlineto{\pgfqpoint{4.570625in}{2.418353in}}%
\pgfpathlineto{\pgfqpoint{4.584239in}{2.426051in}}%
\pgfpathlineto{\pgfqpoint{4.597867in}{2.433910in}}%
\pgfpathlineto{\pgfqpoint{4.605417in}{2.443248in}}%
\pgfpathlineto{\pgfqpoint{4.612962in}{2.452489in}}%
\pgfpathlineto{\pgfqpoint{4.620501in}{2.461633in}}%
\pgfpathlineto{\pgfqpoint{4.628033in}{2.470683in}}%
\pgfpathlineto{\pgfqpoint{4.614411in}{2.462863in}}%
\pgfpathlineto{\pgfqpoint{4.600802in}{2.455205in}}%
\pgfpathlineto{\pgfqpoint{4.587208in}{2.447708in}}%
\pgfpathlineto{\pgfqpoint{4.573627in}{2.440371in}}%
\pgfpathlineto{\pgfqpoint{4.566089in}{2.431272in}}%
\pgfpathlineto{\pgfqpoint{4.558545in}{2.422084in}}%
\pgfpathlineto{\pgfqpoint{4.550995in}{2.412807in}}%
\pgfpathlineto{\pgfqpoint{4.543439in}{2.403441in}}%
\pgfpathclose%
\pgfusepath{fill}%
\end{pgfscope}%
\begin{pgfscope}%
\pgfpathrectangle{\pgfqpoint{1.254980in}{0.150000in}}{\pgfqpoint{5.490039in}{5.490039in}}%
\pgfusepath{clip}%
\pgfsetbuttcap%
\pgfsetroundjoin%
\definecolor{currentfill}{rgb}{0.122606,0.585371,0.546557}%
\pgfsetfillcolor{currentfill}%
\pgfsetfillopacity{0.700000}%
\pgfsetlinewidth{0.000000pt}%
\definecolor{currentstroke}{rgb}{0.000000,0.000000,0.000000}%
\pgfsetstrokecolor{currentstroke}%
\pgfsetdash{}{0pt}%
\pgfpathmoveto{\pgfqpoint{5.280130in}{2.983405in}}%
\pgfpathlineto{\pgfqpoint{5.294122in}{2.994155in}}%
\pgfpathlineto{\pgfqpoint{5.308131in}{3.005062in}}%
\pgfpathlineto{\pgfqpoint{5.322159in}{3.016128in}}%
\pgfpathlineto{\pgfqpoint{5.336205in}{3.027353in}}%
\pgfpathlineto{\pgfqpoint{5.343384in}{3.031594in}}%
\pgfpathlineto{\pgfqpoint{5.350556in}{3.035771in}}%
\pgfpathlineto{\pgfqpoint{5.357720in}{3.039888in}}%
\pgfpathlineto{\pgfqpoint{5.364876in}{3.043949in}}%
\pgfpathlineto{\pgfqpoint{5.350848in}{3.033065in}}%
\pgfpathlineto{\pgfqpoint{5.336838in}{3.022339in}}%
\pgfpathlineto{\pgfqpoint{5.322846in}{3.011770in}}%
\pgfpathlineto{\pgfqpoint{5.308871in}{3.001359in}}%
\pgfpathlineto{\pgfqpoint{5.301697in}{2.996948in}}%
\pgfpathlineto{\pgfqpoint{5.294515in}{2.992488in}}%
\pgfpathlineto{\pgfqpoint{5.287326in}{2.987975in}}%
\pgfpathlineto{\pgfqpoint{5.280130in}{2.983405in}}%
\pgfpathclose%
\pgfusepath{fill}%
\end{pgfscope}%
\begin{pgfscope}%
\pgfpathrectangle{\pgfqpoint{1.254980in}{0.150000in}}{\pgfqpoint{5.490039in}{5.490039in}}%
\pgfusepath{clip}%
\pgfsetbuttcap%
\pgfsetroundjoin%
\definecolor{currentfill}{rgb}{0.267968,0.223549,0.512008}%
\pgfsetfillcolor{currentfill}%
\pgfsetfillopacity{0.700000}%
\pgfsetlinewidth{0.000000pt}%
\definecolor{currentstroke}{rgb}{0.000000,0.000000,0.000000}%
\pgfsetstrokecolor{currentstroke}%
\pgfsetdash{}{0pt}%
\pgfpathmoveto{\pgfqpoint{2.595299in}{2.149777in}}%
\pgfpathlineto{\pgfqpoint{2.608710in}{2.133513in}}%
\pgfpathlineto{\pgfqpoint{2.622113in}{2.117494in}}%
\pgfpathlineto{\pgfqpoint{2.635508in}{2.101718in}}%
\pgfpathlineto{\pgfqpoint{2.648897in}{2.086182in}}%
\pgfpathlineto{\pgfqpoint{2.657349in}{2.087170in}}%
\pgfpathlineto{\pgfqpoint{2.665786in}{2.088385in}}%
\pgfpathlineto{\pgfqpoint{2.674208in}{2.089824in}}%
\pgfpathlineto{\pgfqpoint{2.682616in}{2.091482in}}%
\pgfpathlineto{\pgfqpoint{2.669267in}{2.106531in}}%
\pgfpathlineto{\pgfqpoint{2.655911in}{2.121819in}}%
\pgfpathlineto{\pgfqpoint{2.642549in}{2.137350in}}%
\pgfpathlineto{\pgfqpoint{2.629179in}{2.153125in}}%
\pgfpathlineto{\pgfqpoint{2.620732in}{2.151943in}}%
\pgfpathlineto{\pgfqpoint{2.612270in}{2.150988in}}%
\pgfpathlineto{\pgfqpoint{2.603792in}{2.150265in}}%
\pgfpathlineto{\pgfqpoint{2.595299in}{2.149777in}}%
\pgfpathclose%
\pgfusepath{fill}%
\end{pgfscope}%
\begin{pgfscope}%
\pgfpathrectangle{\pgfqpoint{1.254980in}{0.150000in}}{\pgfqpoint{5.490039in}{5.490039in}}%
\pgfusepath{clip}%
\pgfsetbuttcap%
\pgfsetroundjoin%
\definecolor{currentfill}{rgb}{0.275191,0.194905,0.496005}%
\pgfsetfillcolor{currentfill}%
\pgfsetfillopacity{0.700000}%
\pgfsetlinewidth{0.000000pt}%
\definecolor{currentstroke}{rgb}{0.000000,0.000000,0.000000}%
\pgfsetstrokecolor{currentstroke}%
\pgfsetdash{}{0pt}%
\pgfpathmoveto{\pgfqpoint{2.648897in}{2.086182in}}%
\pgfpathlineto{\pgfqpoint{2.662279in}{2.070886in}}%
\pgfpathlineto{\pgfqpoint{2.675655in}{2.055827in}}%
\pgfpathlineto{\pgfqpoint{2.689024in}{2.041003in}}%
\pgfpathlineto{\pgfqpoint{2.702388in}{2.026412in}}%
\pgfpathlineto{\pgfqpoint{2.710800in}{2.027896in}}%
\pgfpathlineto{\pgfqpoint{2.719198in}{2.029601in}}%
\pgfpathlineto{\pgfqpoint{2.727582in}{2.031521in}}%
\pgfpathlineto{\pgfqpoint{2.735952in}{2.033652in}}%
\pgfpathlineto{\pgfqpoint{2.722627in}{2.047759in}}%
\pgfpathlineto{\pgfqpoint{2.709296in}{2.062098in}}%
\pgfpathlineto{\pgfqpoint{2.695959in}{2.076672in}}%
\pgfpathlineto{\pgfqpoint{2.682616in}{2.091482in}}%
\pgfpathlineto{\pgfqpoint{2.674208in}{2.089824in}}%
\pgfpathlineto{\pgfqpoint{2.665786in}{2.088385in}}%
\pgfpathlineto{\pgfqpoint{2.657349in}{2.087170in}}%
\pgfpathlineto{\pgfqpoint{2.648897in}{2.086182in}}%
\pgfpathclose%
\pgfusepath{fill}%
\end{pgfscope}%
\begin{pgfscope}%
\pgfpathrectangle{\pgfqpoint{1.254980in}{0.150000in}}{\pgfqpoint{5.490039in}{5.490039in}}%
\pgfusepath{clip}%
\pgfsetbuttcap%
\pgfsetroundjoin%
\definecolor{currentfill}{rgb}{0.255645,0.260703,0.528312}%
\pgfsetfillcolor{currentfill}%
\pgfsetfillopacity{0.700000}%
\pgfsetlinewidth{0.000000pt}%
\definecolor{currentstroke}{rgb}{0.000000,0.000000,0.000000}%
\pgfsetstrokecolor{currentstroke}%
\pgfsetdash{}{0pt}%
\pgfpathmoveto{\pgfqpoint{4.259521in}{2.166781in}}%
\pgfpathlineto{\pgfqpoint{4.272974in}{2.172077in}}%
\pgfpathlineto{\pgfqpoint{4.286438in}{2.177536in}}%
\pgfpathlineto{\pgfqpoint{4.299914in}{2.183157in}}%
\pgfpathlineto{\pgfqpoint{4.313402in}{2.188941in}}%
\pgfpathlineto{\pgfqpoint{4.321057in}{2.199611in}}%
\pgfpathlineto{\pgfqpoint{4.328705in}{2.210203in}}%
\pgfpathlineto{\pgfqpoint{4.336349in}{2.220717in}}%
\pgfpathlineto{\pgfqpoint{4.343987in}{2.231153in}}%
\pgfpathlineto{\pgfqpoint{4.330503in}{2.225293in}}%
\pgfpathlineto{\pgfqpoint{4.317031in}{2.219594in}}%
\pgfpathlineto{\pgfqpoint{4.303571in}{2.214059in}}%
\pgfpathlineto{\pgfqpoint{4.290123in}{2.208687in}}%
\pgfpathlineto{\pgfqpoint{4.282480in}{2.198317in}}%
\pgfpathlineto{\pgfqpoint{4.274832in}{2.187876in}}%
\pgfpathlineto{\pgfqpoint{4.267179in}{2.177364in}}%
\pgfpathlineto{\pgfqpoint{4.259521in}{2.166781in}}%
\pgfpathclose%
\pgfusepath{fill}%
\end{pgfscope}%
\begin{pgfscope}%
\pgfpathrectangle{\pgfqpoint{1.254980in}{0.150000in}}{\pgfqpoint{5.490039in}{5.490039in}}%
\pgfusepath{clip}%
\pgfsetbuttcap%
\pgfsetroundjoin%
\definecolor{currentfill}{rgb}{0.257322,0.256130,0.526563}%
\pgfsetfillcolor{currentfill}%
\pgfsetfillopacity{0.700000}%
\pgfsetlinewidth{0.000000pt}%
\definecolor{currentstroke}{rgb}{0.000000,0.000000,0.000000}%
\pgfsetstrokecolor{currentstroke}%
\pgfsetdash{}{0pt}%
\pgfpathmoveto{\pgfqpoint{2.541578in}{2.217319in}}%
\pgfpathlineto{\pgfqpoint{2.555021in}{2.200056in}}%
\pgfpathlineto{\pgfqpoint{2.568455in}{2.183046in}}%
\pgfpathlineto{\pgfqpoint{2.581881in}{2.166287in}}%
\pgfpathlineto{\pgfqpoint{2.595299in}{2.149777in}}%
\pgfpathlineto{\pgfqpoint{2.603792in}{2.150265in}}%
\pgfpathlineto{\pgfqpoint{2.612270in}{2.150988in}}%
\pgfpathlineto{\pgfqpoint{2.620732in}{2.151943in}}%
\pgfpathlineto{\pgfqpoint{2.629179in}{2.153125in}}%
\pgfpathlineto{\pgfqpoint{2.615802in}{2.169145in}}%
\pgfpathlineto{\pgfqpoint{2.602418in}{2.185414in}}%
\pgfpathlineto{\pgfqpoint{2.589026in}{2.201932in}}%
\pgfpathlineto{\pgfqpoint{2.575626in}{2.218703in}}%
\pgfpathlineto{\pgfqpoint{2.567138in}{2.218001in}}%
\pgfpathlineto{\pgfqpoint{2.558634in}{2.217533in}}%
\pgfpathlineto{\pgfqpoint{2.550114in}{2.217304in}}%
\pgfpathlineto{\pgfqpoint{2.541578in}{2.217319in}}%
\pgfpathclose%
\pgfusepath{fill}%
\end{pgfscope}%
\begin{pgfscope}%
\pgfpathrectangle{\pgfqpoint{1.254980in}{0.150000in}}{\pgfqpoint{5.490039in}{5.490039in}}%
\pgfusepath{clip}%
\pgfsetbuttcap%
\pgfsetroundjoin%
\definecolor{currentfill}{rgb}{0.280267,0.073417,0.397163}%
\pgfsetfillcolor{currentfill}%
\pgfsetfillopacity{0.700000}%
\pgfsetlinewidth{0.000000pt}%
\definecolor{currentstroke}{rgb}{0.000000,0.000000,0.000000}%
\pgfsetstrokecolor{currentstroke}%
\pgfsetdash{}{0pt}%
\pgfpathmoveto{\pgfqpoint{3.722374in}{1.800125in}}%
\pgfpathlineto{\pgfqpoint{3.735645in}{1.800110in}}%
\pgfpathlineto{\pgfqpoint{3.748922in}{1.800266in}}%
\pgfpathlineto{\pgfqpoint{3.762207in}{1.800591in}}%
\pgfpathlineto{\pgfqpoint{3.775499in}{1.801084in}}%
\pgfpathlineto{\pgfqpoint{3.783323in}{1.811750in}}%
\pgfpathlineto{\pgfqpoint{3.791143in}{1.822422in}}%
\pgfpathlineto{\pgfqpoint{3.798957in}{1.833098in}}%
\pgfpathlineto{\pgfqpoint{3.806767in}{1.843774in}}%
\pgfpathlineto{\pgfqpoint{3.793483in}{1.843008in}}%
\pgfpathlineto{\pgfqpoint{3.780207in}{1.842411in}}%
\pgfpathlineto{\pgfqpoint{3.766939in}{1.841983in}}%
\pgfpathlineto{\pgfqpoint{3.753678in}{1.841725in}}%
\pgfpathlineto{\pgfqpoint{3.745860in}{1.831311in}}%
\pgfpathlineto{\pgfqpoint{3.738037in}{1.820904in}}%
\pgfpathlineto{\pgfqpoint{3.730208in}{1.810508in}}%
\pgfpathlineto{\pgfqpoint{3.722374in}{1.800125in}}%
\pgfpathclose%
\pgfusepath{fill}%
\end{pgfscope}%
\begin{pgfscope}%
\pgfpathrectangle{\pgfqpoint{1.254980in}{0.150000in}}{\pgfqpoint{5.490039in}{5.490039in}}%
\pgfusepath{clip}%
\pgfsetbuttcap%
\pgfsetroundjoin%
\definecolor{currentfill}{rgb}{0.282656,0.100196,0.422160}%
\pgfsetfillcolor{currentfill}%
\pgfsetfillopacity{0.700000}%
\pgfsetlinewidth{0.000000pt}%
\definecolor{currentstroke}{rgb}{0.000000,0.000000,0.000000}%
\pgfsetstrokecolor{currentstroke}%
\pgfsetdash{}{0pt}%
\pgfpathmoveto{\pgfqpoint{3.806767in}{1.843774in}}%
\pgfpathlineto{\pgfqpoint{3.820058in}{1.844709in}}%
\pgfpathlineto{\pgfqpoint{3.833357in}{1.845813in}}%
\pgfpathlineto{\pgfqpoint{3.846664in}{1.847084in}}%
\pgfpathlineto{\pgfqpoint{3.859979in}{1.848522in}}%
\pgfpathlineto{\pgfqpoint{3.867775in}{1.859453in}}%
\pgfpathlineto{\pgfqpoint{3.875567in}{1.870374in}}%
\pgfpathlineto{\pgfqpoint{3.883354in}{1.881281in}}%
\pgfpathlineto{\pgfqpoint{3.891136in}{1.892175in}}%
\pgfpathlineto{\pgfqpoint{3.877828in}{1.890491in}}%
\pgfpathlineto{\pgfqpoint{3.864529in}{1.888975in}}%
\pgfpathlineto{\pgfqpoint{3.851237in}{1.887627in}}%
\pgfpathlineto{\pgfqpoint{3.837954in}{1.886447in}}%
\pgfpathlineto{\pgfqpoint{3.830165in}{1.875788in}}%
\pgfpathlineto{\pgfqpoint{3.822370in}{1.865122in}}%
\pgfpathlineto{\pgfqpoint{3.814571in}{1.854450in}}%
\pgfpathlineto{\pgfqpoint{3.806767in}{1.843774in}}%
\pgfpathclose%
\pgfusepath{fill}%
\end{pgfscope}%
\begin{pgfscope}%
\pgfpathrectangle{\pgfqpoint{1.254980in}{0.150000in}}{\pgfqpoint{5.490039in}{5.490039in}}%
\pgfusepath{clip}%
\pgfsetbuttcap%
\pgfsetroundjoin%
\definecolor{currentfill}{rgb}{0.269944,0.014625,0.341379}%
\pgfsetfillcolor{currentfill}%
\pgfsetfillopacity{0.700000}%
\pgfsetlinewidth{0.000000pt}%
\definecolor{currentstroke}{rgb}{0.000000,0.000000,0.000000}%
\pgfsetstrokecolor{currentstroke}%
\pgfsetdash{}{0pt}%
\pgfpathmoveto{\pgfqpoint{3.193050in}{1.720973in}}%
\pgfpathlineto{\pgfqpoint{3.206280in}{1.714186in}}%
\pgfpathlineto{\pgfqpoint{3.219512in}{1.707587in}}%
\pgfpathlineto{\pgfqpoint{3.232746in}{1.701176in}}%
\pgfpathlineto{\pgfqpoint{3.245981in}{1.694951in}}%
\pgfpathlineto{\pgfqpoint{3.254036in}{1.701917in}}%
\pgfpathlineto{\pgfqpoint{3.262082in}{1.709002in}}%
\pgfpathlineto{\pgfqpoint{3.270121in}{1.716203in}}%
\pgfpathlineto{\pgfqpoint{3.278151in}{1.723515in}}%
\pgfpathlineto{\pgfqpoint{3.264937in}{1.729328in}}%
\pgfpathlineto{\pgfqpoint{3.251725in}{1.735327in}}%
\pgfpathlineto{\pgfqpoint{3.238515in}{1.741513in}}%
\pgfpathlineto{\pgfqpoint{3.225306in}{1.747888in}}%
\pgfpathlineto{\pgfqpoint{3.217255in}{1.740977in}}%
\pgfpathlineto{\pgfqpoint{3.209195in}{1.734185in}}%
\pgfpathlineto{\pgfqpoint{3.201127in}{1.727516in}}%
\pgfpathlineto{\pgfqpoint{3.193050in}{1.720973in}}%
\pgfpathclose%
\pgfusepath{fill}%
\end{pgfscope}%
\begin{pgfscope}%
\pgfpathrectangle{\pgfqpoint{1.254980in}{0.150000in}}{\pgfqpoint{5.490039in}{5.490039in}}%
\pgfusepath{clip}%
\pgfsetbuttcap%
\pgfsetroundjoin%
\definecolor{currentfill}{rgb}{0.279574,0.170599,0.479997}%
\pgfsetfillcolor{currentfill}%
\pgfsetfillopacity{0.700000}%
\pgfsetlinewidth{0.000000pt}%
\definecolor{currentstroke}{rgb}{0.000000,0.000000,0.000000}%
\pgfsetstrokecolor{currentstroke}%
\pgfsetdash{}{0pt}%
\pgfpathmoveto{\pgfqpoint{2.702388in}{2.026412in}}%
\pgfpathlineto{\pgfqpoint{2.715745in}{2.012054in}}%
\pgfpathlineto{\pgfqpoint{2.729097in}{1.997925in}}%
\pgfpathlineto{\pgfqpoint{2.742444in}{1.984025in}}%
\pgfpathlineto{\pgfqpoint{2.755786in}{1.970352in}}%
\pgfpathlineto{\pgfqpoint{2.764160in}{1.972329in}}%
\pgfpathlineto{\pgfqpoint{2.772521in}{1.974520in}}%
\pgfpathlineto{\pgfqpoint{2.780868in}{1.976919in}}%
\pgfpathlineto{\pgfqpoint{2.789202in}{1.979521in}}%
\pgfpathlineto{\pgfqpoint{2.775897in}{1.992713in}}%
\pgfpathlineto{\pgfqpoint{2.762587in}{2.006131in}}%
\pgfpathlineto{\pgfqpoint{2.749272in}{2.019777in}}%
\pgfpathlineto{\pgfqpoint{2.735952in}{2.033652in}}%
\pgfpathlineto{\pgfqpoint{2.727582in}{2.031521in}}%
\pgfpathlineto{\pgfqpoint{2.719198in}{2.029601in}}%
\pgfpathlineto{\pgfqpoint{2.710800in}{2.027896in}}%
\pgfpathlineto{\pgfqpoint{2.702388in}{2.026412in}}%
\pgfpathclose%
\pgfusepath{fill}%
\end{pgfscope}%
\begin{pgfscope}%
\pgfpathrectangle{\pgfqpoint{1.254980in}{0.150000in}}{\pgfqpoint{5.490039in}{5.490039in}}%
\pgfusepath{clip}%
\pgfsetbuttcap%
\pgfsetroundjoin%
\definecolor{currentfill}{rgb}{0.277018,0.050344,0.375715}%
\pgfsetfillcolor{currentfill}%
\pgfsetfillopacity{0.700000}%
\pgfsetlinewidth{0.000000pt}%
\definecolor{currentstroke}{rgb}{0.000000,0.000000,0.000000}%
\pgfsetstrokecolor{currentstroke}%
\pgfsetdash{}{0pt}%
\pgfpathmoveto{\pgfqpoint{3.637931in}{1.761739in}}%
\pgfpathlineto{\pgfqpoint{3.651186in}{1.760741in}}%
\pgfpathlineto{\pgfqpoint{3.664446in}{1.759914in}}%
\pgfpathlineto{\pgfqpoint{3.677713in}{1.759258in}}%
\pgfpathlineto{\pgfqpoint{3.690987in}{1.758773in}}%
\pgfpathlineto{\pgfqpoint{3.698842in}{1.769079in}}%
\pgfpathlineto{\pgfqpoint{3.706691in}{1.779408in}}%
\pgfpathlineto{\pgfqpoint{3.714535in}{1.789757in}}%
\pgfpathlineto{\pgfqpoint{3.722374in}{1.800125in}}%
\pgfpathlineto{\pgfqpoint{3.709111in}{1.800310in}}%
\pgfpathlineto{\pgfqpoint{3.695854in}{1.800665in}}%
\pgfpathlineto{\pgfqpoint{3.682604in}{1.801192in}}%
\pgfpathlineto{\pgfqpoint{3.669361in}{1.801891in}}%
\pgfpathlineto{\pgfqpoint{3.661512in}{1.791813in}}%
\pgfpathlineto{\pgfqpoint{3.653657in}{1.781760in}}%
\pgfpathlineto{\pgfqpoint{3.645797in}{1.771734in}}%
\pgfpathlineto{\pgfqpoint{3.637931in}{1.761739in}}%
\pgfpathclose%
\pgfusepath{fill}%
\end{pgfscope}%
\begin{pgfscope}%
\pgfpathrectangle{\pgfqpoint{1.254980in}{0.150000in}}{\pgfqpoint{5.490039in}{5.490039in}}%
\pgfusepath{clip}%
\pgfsetbuttcap%
\pgfsetroundjoin%
\definecolor{currentfill}{rgb}{0.283187,0.125848,0.444960}%
\pgfsetfillcolor{currentfill}%
\pgfsetfillopacity{0.700000}%
\pgfsetlinewidth{0.000000pt}%
\definecolor{currentstroke}{rgb}{0.000000,0.000000,0.000000}%
\pgfsetstrokecolor{currentstroke}%
\pgfsetdash{}{0pt}%
\pgfpathmoveto{\pgfqpoint{3.891136in}{1.892175in}}%
\pgfpathlineto{\pgfqpoint{3.904453in}{1.894026in}}%
\pgfpathlineto{\pgfqpoint{3.917778in}{1.896043in}}%
\pgfpathlineto{\pgfqpoint{3.931111in}{1.898227in}}%
\pgfpathlineto{\pgfqpoint{3.944454in}{1.900578in}}%
\pgfpathlineto{\pgfqpoint{3.952225in}{1.911682in}}%
\pgfpathlineto{\pgfqpoint{3.959991in}{1.922761in}}%
\pgfpathlineto{\pgfqpoint{3.967752in}{1.933812in}}%
\pgfpathlineto{\pgfqpoint{3.975508in}{1.944834in}}%
\pgfpathlineto{\pgfqpoint{3.962171in}{1.942267in}}%
\pgfpathlineto{\pgfqpoint{3.948844in}{1.939865in}}%
\pgfpathlineto{\pgfqpoint{3.935525in}{1.937630in}}%
\pgfpathlineto{\pgfqpoint{3.922216in}{1.935562in}}%
\pgfpathlineto{\pgfqpoint{3.914453in}{1.924747in}}%
\pgfpathlineto{\pgfqpoint{3.906686in}{1.913909in}}%
\pgfpathlineto{\pgfqpoint{3.898913in}{1.903051in}}%
\pgfpathlineto{\pgfqpoint{3.891136in}{1.892175in}}%
\pgfpathclose%
\pgfusepath{fill}%
\end{pgfscope}%
\begin{pgfscope}%
\pgfpathrectangle{\pgfqpoint{1.254980in}{0.150000in}}{\pgfqpoint{5.490039in}{5.490039in}}%
\pgfusepath{clip}%
\pgfsetbuttcap%
\pgfsetroundjoin%
\definecolor{currentfill}{rgb}{0.268510,0.009605,0.335427}%
\pgfsetfillcolor{currentfill}%
\pgfsetfillopacity{0.700000}%
\pgfsetlinewidth{0.000000pt}%
\definecolor{currentstroke}{rgb}{0.000000,0.000000,0.000000}%
\pgfsetstrokecolor{currentstroke}%
\pgfsetdash{}{0pt}%
\pgfpathmoveto{\pgfqpoint{3.331029in}{1.702116in}}%
\pgfpathlineto{\pgfqpoint{3.344255in}{1.697224in}}%
\pgfpathlineto{\pgfqpoint{3.357485in}{1.692515in}}%
\pgfpathlineto{\pgfqpoint{3.370717in}{1.687986in}}%
\pgfpathlineto{\pgfqpoint{3.383952in}{1.683637in}}%
\pgfpathlineto{\pgfqpoint{3.391937in}{1.691846in}}%
\pgfpathlineto{\pgfqpoint{3.399914in}{1.700144in}}%
\pgfpathlineto{\pgfqpoint{3.407884in}{1.708529in}}%
\pgfpathlineto{\pgfqpoint{3.415847in}{1.716996in}}%
\pgfpathlineto{\pgfqpoint{3.402629in}{1.720961in}}%
\pgfpathlineto{\pgfqpoint{3.389414in}{1.725107in}}%
\pgfpathlineto{\pgfqpoint{3.376203in}{1.729433in}}%
\pgfpathlineto{\pgfqpoint{3.362995in}{1.733941in}}%
\pgfpathlineto{\pgfqpoint{3.355015in}{1.725846in}}%
\pgfpathlineto{\pgfqpoint{3.347027in}{1.717842in}}%
\pgfpathlineto{\pgfqpoint{3.339032in}{1.709930in}}%
\pgfpathlineto{\pgfqpoint{3.331029in}{1.702116in}}%
\pgfpathclose%
\pgfusepath{fill}%
\end{pgfscope}%
\begin{pgfscope}%
\pgfpathrectangle{\pgfqpoint{1.254980in}{0.150000in}}{\pgfqpoint{5.490039in}{5.490039in}}%
\pgfusepath{clip}%
\pgfsetbuttcap%
\pgfsetroundjoin%
\definecolor{currentfill}{rgb}{0.244972,0.287675,0.537260}%
\pgfsetfillcolor{currentfill}%
\pgfsetfillopacity{0.700000}%
\pgfsetlinewidth{0.000000pt}%
\definecolor{currentstroke}{rgb}{0.000000,0.000000,0.000000}%
\pgfsetstrokecolor{currentstroke}%
\pgfsetdash{}{0pt}%
\pgfpathmoveto{\pgfqpoint{2.487717in}{2.288942in}}%
\pgfpathlineto{\pgfqpoint{2.501196in}{2.270646in}}%
\pgfpathlineto{\pgfqpoint{2.514666in}{2.252612in}}%
\pgfpathlineto{\pgfqpoint{2.528126in}{2.234837in}}%
\pgfpathlineto{\pgfqpoint{2.541578in}{2.217319in}}%
\pgfpathlineto{\pgfqpoint{2.550114in}{2.217304in}}%
\pgfpathlineto{\pgfqpoint{2.558634in}{2.217533in}}%
\pgfpathlineto{\pgfqpoint{2.567138in}{2.218001in}}%
\pgfpathlineto{\pgfqpoint{2.575626in}{2.218703in}}%
\pgfpathlineto{\pgfqpoint{2.562217in}{2.235727in}}%
\pgfpathlineto{\pgfqpoint{2.548800in}{2.253008in}}%
\pgfpathlineto{\pgfqpoint{2.535375in}{2.270547in}}%
\pgfpathlineto{\pgfqpoint{2.521940in}{2.288347in}}%
\pgfpathlineto{\pgfqpoint{2.513409in}{2.288128in}}%
\pgfpathlineto{\pgfqpoint{2.504862in}{2.288151in}}%
\pgfpathlineto{\pgfqpoint{2.496298in}{2.288421in}}%
\pgfpathlineto{\pgfqpoint{2.487717in}{2.288942in}}%
\pgfpathclose%
\pgfusepath{fill}%
\end{pgfscope}%
\begin{pgfscope}%
\pgfpathrectangle{\pgfqpoint{1.254980in}{0.150000in}}{\pgfqpoint{5.490039in}{5.490039in}}%
\pgfusepath{clip}%
\pgfsetbuttcap%
\pgfsetroundjoin%
\definecolor{currentfill}{rgb}{0.159194,0.482237,0.558073}%
\pgfsetfillcolor{currentfill}%
\pgfsetfillopacity{0.700000}%
\pgfsetlinewidth{0.000000pt}%
\definecolor{currentstroke}{rgb}{0.000000,0.000000,0.000000}%
\pgfsetstrokecolor{currentstroke}%
\pgfsetdash{}{0pt}%
\pgfpathmoveto{\pgfqpoint{4.911950in}{2.704472in}}%
\pgfpathlineto{\pgfqpoint{4.925737in}{2.713922in}}%
\pgfpathlineto{\pgfqpoint{4.939541in}{2.723532in}}%
\pgfpathlineto{\pgfqpoint{4.953360in}{2.733302in}}%
\pgfpathlineto{\pgfqpoint{4.967196in}{2.743231in}}%
\pgfpathlineto{\pgfqpoint{4.974583in}{2.750166in}}%
\pgfpathlineto{\pgfqpoint{4.981962in}{2.757004in}}%
\pgfpathlineto{\pgfqpoint{4.989334in}{2.763750in}}%
\pgfpathlineto{\pgfqpoint{4.996699in}{2.770404in}}%
\pgfpathlineto{\pgfqpoint{4.982873in}{2.760664in}}%
\pgfpathlineto{\pgfqpoint{4.969064in}{2.751082in}}%
\pgfpathlineto{\pgfqpoint{4.955271in}{2.741661in}}%
\pgfpathlineto{\pgfqpoint{4.941493in}{2.732398in}}%
\pgfpathlineto{\pgfqpoint{4.934118in}{2.725544in}}%
\pgfpathlineto{\pgfqpoint{4.926736in}{2.718607in}}%
\pgfpathlineto{\pgfqpoint{4.919346in}{2.711584in}}%
\pgfpathlineto{\pgfqpoint{4.911950in}{2.704472in}}%
\pgfpathclose%
\pgfusepath{fill}%
\end{pgfscope}%
\begin{pgfscope}%
\pgfpathrectangle{\pgfqpoint{1.254980in}{0.150000in}}{\pgfqpoint{5.490039in}{5.490039in}}%
\pgfusepath{clip}%
\pgfsetbuttcap%
\pgfsetroundjoin%
\definecolor{currentfill}{rgb}{0.119512,0.607464,0.540218}%
\pgfsetfillcolor{currentfill}%
\pgfsetfillopacity{0.700000}%
\pgfsetlinewidth{0.000000pt}%
\definecolor{currentstroke}{rgb}{0.000000,0.000000,0.000000}%
\pgfsetstrokecolor{currentstroke}%
\pgfsetdash{}{0pt}%
\pgfpathmoveto{\pgfqpoint{5.364876in}{3.043949in}}%
\pgfpathlineto{\pgfqpoint{5.378922in}{3.054991in}}%
\pgfpathlineto{\pgfqpoint{5.392986in}{3.066190in}}%
\pgfpathlineto{\pgfqpoint{5.407068in}{3.077548in}}%
\pgfpathlineto{\pgfqpoint{5.421170in}{3.089064in}}%
\pgfpathlineto{\pgfqpoint{5.428299in}{3.092713in}}%
\pgfpathlineto{\pgfqpoint{5.435421in}{3.096307in}}%
\pgfpathlineto{\pgfqpoint{5.442535in}{3.099850in}}%
\pgfpathlineto{\pgfqpoint{5.449641in}{3.103346in}}%
\pgfpathlineto{\pgfqpoint{5.435560in}{3.092201in}}%
\pgfpathlineto{\pgfqpoint{5.421497in}{3.081214in}}%
\pgfpathlineto{\pgfqpoint{5.407453in}{3.070385in}}%
\pgfpathlineto{\pgfqpoint{5.393426in}{3.059712in}}%
\pgfpathlineto{\pgfqpoint{5.386299in}{3.055835in}}%
\pgfpathlineto{\pgfqpoint{5.379166in}{3.051918in}}%
\pgfpathlineto{\pgfqpoint{5.372024in}{3.047958in}}%
\pgfpathlineto{\pgfqpoint{5.364876in}{3.043949in}}%
\pgfpathclose%
\pgfusepath{fill}%
\end{pgfscope}%
\begin{pgfscope}%
\pgfpathrectangle{\pgfqpoint{1.254980in}{0.150000in}}{\pgfqpoint{5.490039in}{5.490039in}}%
\pgfusepath{clip}%
\pgfsetbuttcap%
\pgfsetroundjoin%
\definecolor{currentfill}{rgb}{0.282290,0.145912,0.461510}%
\pgfsetfillcolor{currentfill}%
\pgfsetfillopacity{0.700000}%
\pgfsetlinewidth{0.000000pt}%
\definecolor{currentstroke}{rgb}{0.000000,0.000000,0.000000}%
\pgfsetstrokecolor{currentstroke}%
\pgfsetdash{}{0pt}%
\pgfpathmoveto{\pgfqpoint{2.755786in}{1.970352in}}%
\pgfpathlineto{\pgfqpoint{2.769122in}{1.956903in}}%
\pgfpathlineto{\pgfqpoint{2.782454in}{1.943679in}}%
\pgfpathlineto{\pgfqpoint{2.795782in}{1.930676in}}%
\pgfpathlineto{\pgfqpoint{2.809105in}{1.917893in}}%
\pgfpathlineto{\pgfqpoint{2.817443in}{1.920363in}}%
\pgfpathlineto{\pgfqpoint{2.825768in}{1.923037in}}%
\pgfpathlineto{\pgfqpoint{2.834080in}{1.925912in}}%
\pgfpathlineto{\pgfqpoint{2.842380in}{1.928983in}}%
\pgfpathlineto{\pgfqpoint{2.829091in}{1.941286in}}%
\pgfpathlineto{\pgfqpoint{2.815799in}{1.953809in}}%
\pgfpathlineto{\pgfqpoint{2.802503in}{1.966554in}}%
\pgfpathlineto{\pgfqpoint{2.789202in}{1.979521in}}%
\pgfpathlineto{\pgfqpoint{2.780868in}{1.976919in}}%
\pgfpathlineto{\pgfqpoint{2.772521in}{1.974520in}}%
\pgfpathlineto{\pgfqpoint{2.764160in}{1.972329in}}%
\pgfpathlineto{\pgfqpoint{2.755786in}{1.970352in}}%
\pgfpathclose%
\pgfusepath{fill}%
\end{pgfscope}%
\begin{pgfscope}%
\pgfpathrectangle{\pgfqpoint{1.254980in}{0.150000in}}{\pgfqpoint{5.490039in}{5.490039in}}%
\pgfusepath{clip}%
\pgfsetbuttcap%
\pgfsetroundjoin%
\definecolor{currentfill}{rgb}{0.273809,0.031497,0.358853}%
\pgfsetfillcolor{currentfill}%
\pgfsetfillopacity{0.700000}%
\pgfsetlinewidth{0.000000pt}%
\definecolor{currentstroke}{rgb}{0.000000,0.000000,0.000000}%
\pgfsetstrokecolor{currentstroke}%
\pgfsetdash{}{0pt}%
\pgfpathmoveto{\pgfqpoint{3.553405in}{1.729152in}}%
\pgfpathlineto{\pgfqpoint{3.566648in}{1.727134in}}%
\pgfpathlineto{\pgfqpoint{3.579897in}{1.725290in}}%
\pgfpathlineto{\pgfqpoint{3.593152in}{1.723619in}}%
\pgfpathlineto{\pgfqpoint{3.606412in}{1.722121in}}%
\pgfpathlineto{\pgfqpoint{3.614300in}{1.731965in}}%
\pgfpathlineto{\pgfqpoint{3.622183in}{1.741852in}}%
\pgfpathlineto{\pgfqpoint{3.630060in}{1.751777in}}%
\pgfpathlineto{\pgfqpoint{3.637931in}{1.761739in}}%
\pgfpathlineto{\pgfqpoint{3.624683in}{1.762910in}}%
\pgfpathlineto{\pgfqpoint{3.611441in}{1.764253in}}%
\pgfpathlineto{\pgfqpoint{3.598204in}{1.765770in}}%
\pgfpathlineto{\pgfqpoint{3.584973in}{1.767461in}}%
\pgfpathlineto{\pgfqpoint{3.577090in}{1.757816in}}%
\pgfpathlineto{\pgfqpoint{3.569201in}{1.748214in}}%
\pgfpathlineto{\pgfqpoint{3.561306in}{1.738659in}}%
\pgfpathlineto{\pgfqpoint{3.553405in}{1.729152in}}%
\pgfpathclose%
\pgfusepath{fill}%
\end{pgfscope}%
\begin{pgfscope}%
\pgfpathrectangle{\pgfqpoint{1.254980in}{0.150000in}}{\pgfqpoint{5.490039in}{5.490039in}}%
\pgfusepath{clip}%
\pgfsetbuttcap%
\pgfsetroundjoin%
\definecolor{currentfill}{rgb}{0.281412,0.155834,0.469201}%
\pgfsetfillcolor{currentfill}%
\pgfsetfillopacity{0.700000}%
\pgfsetlinewidth{0.000000pt}%
\definecolor{currentstroke}{rgb}{0.000000,0.000000,0.000000}%
\pgfsetstrokecolor{currentstroke}%
\pgfsetdash{}{0pt}%
\pgfpathmoveto{\pgfqpoint{3.975508in}{1.944834in}}%
\pgfpathlineto{\pgfqpoint{3.988854in}{1.947568in}}%
\pgfpathlineto{\pgfqpoint{4.002209in}{1.950468in}}%
\pgfpathlineto{\pgfqpoint{4.015574in}{1.953532in}}%
\pgfpathlineto{\pgfqpoint{4.028949in}{1.956762in}}%
\pgfpathlineto{\pgfqpoint{4.036695in}{1.967953in}}%
\pgfpathlineto{\pgfqpoint{4.044436in}{1.979105in}}%
\pgfpathlineto{\pgfqpoint{4.052172in}{1.990215in}}%
\pgfpathlineto{\pgfqpoint{4.059904in}{2.001283in}}%
\pgfpathlineto{\pgfqpoint{4.046534in}{1.997863in}}%
\pgfpathlineto{\pgfqpoint{4.033175in}{1.994609in}}%
\pgfpathlineto{\pgfqpoint{4.019825in}{1.991520in}}%
\pgfpathlineto{\pgfqpoint{4.006485in}{1.988597in}}%
\pgfpathlineto{\pgfqpoint{3.998748in}{1.977708in}}%
\pgfpathlineto{\pgfqpoint{3.991006in}{1.966784in}}%
\pgfpathlineto{\pgfqpoint{3.983260in}{1.955826in}}%
\pgfpathlineto{\pgfqpoint{3.975508in}{1.944834in}}%
\pgfpathclose%
\pgfusepath{fill}%
\end{pgfscope}%
\begin{pgfscope}%
\pgfpathrectangle{\pgfqpoint{1.254980in}{0.150000in}}{\pgfqpoint{5.490039in}{5.490039in}}%
\pgfusepath{clip}%
\pgfsetbuttcap%
\pgfsetroundjoin%
\definecolor{currentfill}{rgb}{0.195860,0.395433,0.555276}%
\pgfsetfillcolor{currentfill}%
\pgfsetfillopacity{0.700000}%
\pgfsetlinewidth{0.000000pt}%
\definecolor{currentstroke}{rgb}{0.000000,0.000000,0.000000}%
\pgfsetstrokecolor{currentstroke}%
\pgfsetdash{}{0pt}%
\pgfpathmoveto{\pgfqpoint{4.628033in}{2.470683in}}%
\pgfpathlineto{\pgfqpoint{4.641669in}{2.478664in}}%
\pgfpathlineto{\pgfqpoint{4.655320in}{2.486805in}}%
\pgfpathlineto{\pgfqpoint{4.668986in}{2.495107in}}%
\pgfpathlineto{\pgfqpoint{4.682666in}{2.503570in}}%
\pgfpathlineto{\pgfqpoint{4.690186in}{2.512468in}}%
\pgfpathlineto{\pgfqpoint{4.697700in}{2.521265in}}%
\pgfpathlineto{\pgfqpoint{4.705207in}{2.529962in}}%
\pgfpathlineto{\pgfqpoint{4.712708in}{2.538562in}}%
\pgfpathlineto{\pgfqpoint{4.699034in}{2.530168in}}%
\pgfpathlineto{\pgfqpoint{4.685375in}{2.521934in}}%
\pgfpathlineto{\pgfqpoint{4.671730in}{2.513862in}}%
\pgfpathlineto{\pgfqpoint{4.658100in}{2.505950in}}%
\pgfpathlineto{\pgfqpoint{4.650592in}{2.497271in}}%
\pgfpathlineto{\pgfqpoint{4.643079in}{2.488501in}}%
\pgfpathlineto{\pgfqpoint{4.635559in}{2.479638in}}%
\pgfpathlineto{\pgfqpoint{4.628033in}{2.470683in}}%
\pgfpathclose%
\pgfusepath{fill}%
\end{pgfscope}%
\begin{pgfscope}%
\pgfpathrectangle{\pgfqpoint{1.254980in}{0.150000in}}{\pgfqpoint{5.490039in}{5.490039in}}%
\pgfusepath{clip}%
\pgfsetbuttcap%
\pgfsetroundjoin%
\definecolor{currentfill}{rgb}{0.229739,0.322361,0.545706}%
\pgfsetfillcolor{currentfill}%
\pgfsetfillopacity{0.700000}%
\pgfsetlinewidth{0.000000pt}%
\definecolor{currentstroke}{rgb}{0.000000,0.000000,0.000000}%
\pgfsetstrokecolor{currentstroke}%
\pgfsetdash{}{0pt}%
\pgfpathmoveto{\pgfqpoint{2.433700in}{2.364789in}}%
\pgfpathlineto{\pgfqpoint{2.447220in}{2.345423in}}%
\pgfpathlineto{\pgfqpoint{2.460729in}{2.326328in}}%
\pgfpathlineto{\pgfqpoint{2.474228in}{2.307502in}}%
\pgfpathlineto{\pgfqpoint{2.487717in}{2.288942in}}%
\pgfpathlineto{\pgfqpoint{2.496298in}{2.288421in}}%
\pgfpathlineto{\pgfqpoint{2.504862in}{2.288151in}}%
\pgfpathlineto{\pgfqpoint{2.513409in}{2.288128in}}%
\pgfpathlineto{\pgfqpoint{2.521940in}{2.288347in}}%
\pgfpathlineto{\pgfqpoint{2.508496in}{2.306411in}}%
\pgfpathlineto{\pgfqpoint{2.495042in}{2.324739in}}%
\pgfpathlineto{\pgfqpoint{2.481579in}{2.343335in}}%
\pgfpathlineto{\pgfqpoint{2.468105in}{2.362201in}}%
\pgfpathlineto{\pgfqpoint{2.459530in}{2.362468in}}%
\pgfpathlineto{\pgfqpoint{2.450937in}{2.362985in}}%
\pgfpathlineto{\pgfqpoint{2.442328in}{2.363757in}}%
\pgfpathlineto{\pgfqpoint{2.433700in}{2.364789in}}%
\pgfpathclose%
\pgfusepath{fill}%
\end{pgfscope}%
\begin{pgfscope}%
\pgfpathrectangle{\pgfqpoint{1.254980in}{0.150000in}}{\pgfqpoint{5.490039in}{5.490039in}}%
\pgfusepath{clip}%
\pgfsetbuttcap%
\pgfsetroundjoin%
\definecolor{currentfill}{rgb}{0.241237,0.296485,0.539709}%
\pgfsetfillcolor{currentfill}%
\pgfsetfillopacity{0.700000}%
\pgfsetlinewidth{0.000000pt}%
\definecolor{currentstroke}{rgb}{0.000000,0.000000,0.000000}%
\pgfsetstrokecolor{currentstroke}%
\pgfsetdash{}{0pt}%
\pgfpathmoveto{\pgfqpoint{4.343987in}{2.231153in}}%
\pgfpathlineto{\pgfqpoint{4.357484in}{2.237177in}}%
\pgfpathlineto{\pgfqpoint{4.370992in}{2.243363in}}%
\pgfpathlineto{\pgfqpoint{4.384514in}{2.249711in}}%
\pgfpathlineto{\pgfqpoint{4.398047in}{2.256221in}}%
\pgfpathlineto{\pgfqpoint{4.405676in}{2.266637in}}%
\pgfpathlineto{\pgfqpoint{4.413300in}{2.276967in}}%
\pgfpathlineto{\pgfqpoint{4.420918in}{2.287211in}}%
\pgfpathlineto{\pgfqpoint{4.428530in}{2.297369in}}%
\pgfpathlineto{\pgfqpoint{4.415001in}{2.290811in}}%
\pgfpathlineto{\pgfqpoint{4.401484in}{2.284415in}}%
\pgfpathlineto{\pgfqpoint{4.387979in}{2.278181in}}%
\pgfpathlineto{\pgfqpoint{4.374487in}{2.272110in}}%
\pgfpathlineto{\pgfqpoint{4.366870in}{2.261989in}}%
\pgfpathlineto{\pgfqpoint{4.359248in}{2.251789in}}%
\pgfpathlineto{\pgfqpoint{4.351620in}{2.241511in}}%
\pgfpathlineto{\pgfqpoint{4.343987in}{2.231153in}}%
\pgfpathclose%
\pgfusepath{fill}%
\end{pgfscope}%
\begin{pgfscope}%
\pgfpathrectangle{\pgfqpoint{1.254980in}{0.150000in}}{\pgfqpoint{5.490039in}{5.490039in}}%
\pgfusepath{clip}%
\pgfsetbuttcap%
\pgfsetroundjoin%
\definecolor{currentfill}{rgb}{0.274952,0.037752,0.364543}%
\pgfsetfillcolor{currentfill}%
\pgfsetfillopacity{0.700000}%
\pgfsetlinewidth{0.000000pt}%
\definecolor{currentstroke}{rgb}{0.000000,0.000000,0.000000}%
\pgfsetstrokecolor{currentstroke}%
\pgfsetdash{}{0pt}%
\pgfpathmoveto{\pgfqpoint{3.054630in}{1.760890in}}%
\pgfpathlineto{\pgfqpoint{3.067883in}{1.752116in}}%
\pgfpathlineto{\pgfqpoint{3.081135in}{1.743539in}}%
\pgfpathlineto{\pgfqpoint{3.094387in}{1.735158in}}%
\pgfpathlineto{\pgfqpoint{3.107639in}{1.726971in}}%
\pgfpathlineto{\pgfqpoint{3.115778in}{1.732514in}}%
\pgfpathlineto{\pgfqpoint{3.123908in}{1.738207in}}%
\pgfpathlineto{\pgfqpoint{3.132027in}{1.744046in}}%
\pgfpathlineto{\pgfqpoint{3.140137in}{1.750028in}}%
\pgfpathlineto{\pgfqpoint{3.126911in}{1.757772in}}%
\pgfpathlineto{\pgfqpoint{3.113684in}{1.765711in}}%
\pgfpathlineto{\pgfqpoint{3.100458in}{1.773846in}}%
\pgfpathlineto{\pgfqpoint{3.087232in}{1.782176in}}%
\pgfpathlineto{\pgfqpoint{3.079097in}{1.776627in}}%
\pgfpathlineto{\pgfqpoint{3.070951in}{1.771226in}}%
\pgfpathlineto{\pgfqpoint{3.062796in}{1.765979in}}%
\pgfpathlineto{\pgfqpoint{3.054630in}{1.760890in}}%
\pgfpathclose%
\pgfusepath{fill}%
\end{pgfscope}%
\begin{pgfscope}%
\pgfpathrectangle{\pgfqpoint{1.254980in}{0.150000in}}{\pgfqpoint{5.490039in}{5.490039in}}%
\pgfusepath{clip}%
\pgfsetbuttcap%
\pgfsetroundjoin%
\definecolor{currentfill}{rgb}{0.121380,0.629492,0.531973}%
\pgfsetfillcolor{currentfill}%
\pgfsetfillopacity{0.700000}%
\pgfsetlinewidth{0.000000pt}%
\definecolor{currentstroke}{rgb}{0.000000,0.000000,0.000000}%
\pgfsetstrokecolor{currentstroke}%
\pgfsetdash{}{0pt}%
\pgfpathmoveto{\pgfqpoint{5.449641in}{3.103346in}}%
\pgfpathlineto{\pgfqpoint{5.463741in}{3.114648in}}%
\pgfpathlineto{\pgfqpoint{5.477859in}{3.126107in}}%
\pgfpathlineto{\pgfqpoint{5.491996in}{3.137725in}}%
\pgfpathlineto{\pgfqpoint{5.506152in}{3.149501in}}%
\pgfpathlineto{\pgfqpoint{5.513229in}{3.152563in}}%
\pgfpathlineto{\pgfqpoint{5.520299in}{3.155581in}}%
\pgfpathlineto{\pgfqpoint{5.527362in}{3.158558in}}%
\pgfpathlineto{\pgfqpoint{5.534417in}{3.161498in}}%
\pgfpathlineto{\pgfqpoint{5.520283in}{3.150125in}}%
\pgfpathlineto{\pgfqpoint{5.506168in}{3.138910in}}%
\pgfpathlineto{\pgfqpoint{5.492071in}{3.127851in}}%
\pgfpathlineto{\pgfqpoint{5.477993in}{3.116949in}}%
\pgfpathlineto{\pgfqpoint{5.470916in}{3.113596in}}%
\pgfpathlineto{\pgfqpoint{5.463831in}{3.110215in}}%
\pgfpathlineto{\pgfqpoint{5.456740in}{3.106799in}}%
\pgfpathlineto{\pgfqpoint{5.449641in}{3.103346in}}%
\pgfpathclose%
\pgfusepath{fill}%
\end{pgfscope}%
\begin{pgfscope}%
\pgfpathrectangle{\pgfqpoint{1.254980in}{0.150000in}}{\pgfqpoint{5.490039in}{5.490039in}}%
\pgfusepath{clip}%
\pgfsetbuttcap%
\pgfsetroundjoin%
\definecolor{currentfill}{rgb}{0.283229,0.120777,0.440584}%
\pgfsetfillcolor{currentfill}%
\pgfsetfillopacity{0.700000}%
\pgfsetlinewidth{0.000000pt}%
\definecolor{currentstroke}{rgb}{0.000000,0.000000,0.000000}%
\pgfsetstrokecolor{currentstroke}%
\pgfsetdash{}{0pt}%
\pgfpathmoveto{\pgfqpoint{2.809105in}{1.917893in}}%
\pgfpathlineto{\pgfqpoint{2.822424in}{1.905330in}}%
\pgfpathlineto{\pgfqpoint{2.835739in}{1.892984in}}%
\pgfpathlineto{\pgfqpoint{2.849051in}{1.880854in}}%
\pgfpathlineto{\pgfqpoint{2.862359in}{1.868938in}}%
\pgfpathlineto{\pgfqpoint{2.870663in}{1.871896in}}%
\pgfpathlineto{\pgfqpoint{2.878954in}{1.875052in}}%
\pgfpathlineto{\pgfqpoint{2.887232in}{1.878401in}}%
\pgfpathlineto{\pgfqpoint{2.895499in}{1.881939in}}%
\pgfpathlineto{\pgfqpoint{2.882224in}{1.893377in}}%
\pgfpathlineto{\pgfqpoint{2.868946in}{1.905030in}}%
\pgfpathlineto{\pgfqpoint{2.855664in}{1.916898in}}%
\pgfpathlineto{\pgfqpoint{2.842380in}{1.928983in}}%
\pgfpathlineto{\pgfqpoint{2.834080in}{1.925912in}}%
\pgfpathlineto{\pgfqpoint{2.825768in}{1.923037in}}%
\pgfpathlineto{\pgfqpoint{2.817443in}{1.920363in}}%
\pgfpathlineto{\pgfqpoint{2.809105in}{1.917893in}}%
\pgfpathclose%
\pgfusepath{fill}%
\end{pgfscope}%
\begin{pgfscope}%
\pgfpathrectangle{\pgfqpoint{1.254980in}{0.150000in}}{\pgfqpoint{5.490039in}{5.490039in}}%
\pgfusepath{clip}%
\pgfsetbuttcap%
\pgfsetroundjoin%
\definecolor{currentfill}{rgb}{0.149039,0.508051,0.557250}%
\pgfsetfillcolor{currentfill}%
\pgfsetfillopacity{0.700000}%
\pgfsetlinewidth{0.000000pt}%
\definecolor{currentstroke}{rgb}{0.000000,0.000000,0.000000}%
\pgfsetstrokecolor{currentstroke}%
\pgfsetdash{}{0pt}%
\pgfpathmoveto{\pgfqpoint{4.996699in}{2.770404in}}%
\pgfpathlineto{\pgfqpoint{5.010541in}{2.780304in}}%
\pgfpathlineto{\pgfqpoint{5.024400in}{2.790364in}}%
\pgfpathlineto{\pgfqpoint{5.038275in}{2.800583in}}%
\pgfpathlineto{\pgfqpoint{5.052167in}{2.810962in}}%
\pgfpathlineto{\pgfqpoint{5.059513in}{2.817319in}}%
\pgfpathlineto{\pgfqpoint{5.066852in}{2.823584in}}%
\pgfpathlineto{\pgfqpoint{5.074183in}{2.829758in}}%
\pgfpathlineto{\pgfqpoint{5.081507in}{2.835844in}}%
\pgfpathlineto{\pgfqpoint{5.067626in}{2.825685in}}%
\pgfpathlineto{\pgfqpoint{5.053763in}{2.815685in}}%
\pgfpathlineto{\pgfqpoint{5.039916in}{2.805844in}}%
\pgfpathlineto{\pgfqpoint{5.026085in}{2.796162in}}%
\pgfpathlineto{\pgfqpoint{5.018750in}{2.789846in}}%
\pgfpathlineto{\pgfqpoint{5.011407in}{2.783450in}}%
\pgfpathlineto{\pgfqpoint{5.004057in}{2.776970in}}%
\pgfpathlineto{\pgfqpoint{4.996699in}{2.770404in}}%
\pgfpathclose%
\pgfusepath{fill}%
\end{pgfscope}%
\begin{pgfscope}%
\pgfpathrectangle{\pgfqpoint{1.254980in}{0.150000in}}{\pgfqpoint{5.490039in}{5.490039in}}%
\pgfusepath{clip}%
\pgfsetbuttcap%
\pgfsetroundjoin%
\definecolor{currentfill}{rgb}{0.277134,0.185228,0.489898}%
\pgfsetfillcolor{currentfill}%
\pgfsetfillopacity{0.700000}%
\pgfsetlinewidth{0.000000pt}%
\definecolor{currentstroke}{rgb}{0.000000,0.000000,0.000000}%
\pgfsetstrokecolor{currentstroke}%
\pgfsetdash{}{0pt}%
\pgfpathmoveto{\pgfqpoint{4.059904in}{2.001283in}}%
\pgfpathlineto{\pgfqpoint{4.073283in}{2.004867in}}%
\pgfpathlineto{\pgfqpoint{4.086673in}{2.008616in}}%
\pgfpathlineto{\pgfqpoint{4.100072in}{2.012529in}}%
\pgfpathlineto{\pgfqpoint{4.113483in}{2.016607in}}%
\pgfpathlineto{\pgfqpoint{4.121205in}{2.027802in}}%
\pgfpathlineto{\pgfqpoint{4.128922in}{2.038945in}}%
\pgfpathlineto{\pgfqpoint{4.136634in}{2.050034in}}%
\pgfpathlineto{\pgfqpoint{4.144342in}{2.061068in}}%
\pgfpathlineto{\pgfqpoint{4.130936in}{2.056829in}}%
\pgfpathlineto{\pgfqpoint{4.117541in}{2.052754in}}%
\pgfpathlineto{\pgfqpoint{4.104156in}{2.048844in}}%
\pgfpathlineto{\pgfqpoint{4.090782in}{2.045098in}}%
\pgfpathlineto{\pgfqpoint{4.083070in}{2.034215in}}%
\pgfpathlineto{\pgfqpoint{4.075353in}{2.023284in}}%
\pgfpathlineto{\pgfqpoint{4.067631in}{2.012306in}}%
\pgfpathlineto{\pgfqpoint{4.059904in}{2.001283in}}%
\pgfpathclose%
\pgfusepath{fill}%
\end{pgfscope}%
\begin{pgfscope}%
\pgfpathrectangle{\pgfqpoint{1.254980in}{0.150000in}}{\pgfqpoint{5.490039in}{5.490039in}}%
\pgfusepath{clip}%
\pgfsetbuttcap%
\pgfsetroundjoin%
\definecolor{currentfill}{rgb}{0.271305,0.019942,0.347269}%
\pgfsetfillcolor{currentfill}%
\pgfsetfillopacity{0.700000}%
\pgfsetlinewidth{0.000000pt}%
\definecolor{currentstroke}{rgb}{0.000000,0.000000,0.000000}%
\pgfsetstrokecolor{currentstroke}%
\pgfsetdash{}{0pt}%
\pgfpathmoveto{\pgfqpoint{3.468759in}{1.702924in}}%
\pgfpathlineto{\pgfqpoint{3.481997in}{1.699849in}}%
\pgfpathlineto{\pgfqpoint{3.495240in}{1.696951in}}%
\pgfpathlineto{\pgfqpoint{3.508487in}{1.694229in}}%
\pgfpathlineto{\pgfqpoint{3.521739in}{1.691681in}}%
\pgfpathlineto{\pgfqpoint{3.529665in}{1.700959in}}%
\pgfpathlineto{\pgfqpoint{3.537584in}{1.710299in}}%
\pgfpathlineto{\pgfqpoint{3.545498in}{1.719698in}}%
\pgfpathlineto{\pgfqpoint{3.553405in}{1.729152in}}%
\pgfpathlineto{\pgfqpoint{3.540166in}{1.731345in}}%
\pgfpathlineto{\pgfqpoint{3.526933in}{1.733712in}}%
\pgfpathlineto{\pgfqpoint{3.513705in}{1.736255in}}%
\pgfpathlineto{\pgfqpoint{3.500481in}{1.738974in}}%
\pgfpathlineto{\pgfqpoint{3.492560in}{1.729865in}}%
\pgfpathlineto{\pgfqpoint{3.484633in}{1.720818in}}%
\pgfpathlineto{\pgfqpoint{3.476699in}{1.711836in}}%
\pgfpathlineto{\pgfqpoint{3.468759in}{1.702924in}}%
\pgfpathclose%
\pgfusepath{fill}%
\end{pgfscope}%
\begin{pgfscope}%
\pgfpathrectangle{\pgfqpoint{1.254980in}{0.150000in}}{\pgfqpoint{5.490039in}{5.490039in}}%
\pgfusepath{clip}%
\pgfsetbuttcap%
\pgfsetroundjoin%
\definecolor{currentfill}{rgb}{0.214298,0.355619,0.551184}%
\pgfsetfillcolor{currentfill}%
\pgfsetfillopacity{0.700000}%
\pgfsetlinewidth{0.000000pt}%
\definecolor{currentstroke}{rgb}{0.000000,0.000000,0.000000}%
\pgfsetstrokecolor{currentstroke}%
\pgfsetdash{}{0pt}%
\pgfpathmoveto{\pgfqpoint{2.379509in}{2.445012in}}%
\pgfpathlineto{\pgfqpoint{2.393074in}{2.424537in}}%
\pgfpathlineto{\pgfqpoint{2.406628in}{2.404343in}}%
\pgfpathlineto{\pgfqpoint{2.420170in}{2.384428in}}%
\pgfpathlineto{\pgfqpoint{2.433700in}{2.364789in}}%
\pgfpathlineto{\pgfqpoint{2.442328in}{2.363757in}}%
\pgfpathlineto{\pgfqpoint{2.450937in}{2.362985in}}%
\pgfpathlineto{\pgfqpoint{2.459530in}{2.362468in}}%
\pgfpathlineto{\pgfqpoint{2.468105in}{2.362201in}}%
\pgfpathlineto{\pgfqpoint{2.454621in}{2.381339in}}%
\pgfpathlineto{\pgfqpoint{2.441126in}{2.400753in}}%
\pgfpathlineto{\pgfqpoint{2.427620in}{2.420444in}}%
\pgfpathlineto{\pgfqpoint{2.414103in}{2.440414in}}%
\pgfpathlineto{\pgfqpoint{2.405482in}{2.441172in}}%
\pgfpathlineto{\pgfqpoint{2.396842in}{2.442188in}}%
\pgfpathlineto{\pgfqpoint{2.388185in}{2.443466in}}%
\pgfpathlineto{\pgfqpoint{2.379509in}{2.445012in}}%
\pgfpathclose%
\pgfusepath{fill}%
\end{pgfscope}%
\begin{pgfscope}%
\pgfpathrectangle{\pgfqpoint{1.254980in}{0.150000in}}{\pgfqpoint{5.490039in}{5.490039in}}%
\pgfusepath{clip}%
\pgfsetbuttcap%
\pgfsetroundjoin%
\definecolor{currentfill}{rgb}{0.132268,0.655014,0.519661}%
\pgfsetfillcolor{currentfill}%
\pgfsetfillopacity{0.700000}%
\pgfsetlinewidth{0.000000pt}%
\definecolor{currentstroke}{rgb}{0.000000,0.000000,0.000000}%
\pgfsetstrokecolor{currentstroke}%
\pgfsetdash{}{0pt}%
\pgfpathmoveto{\pgfqpoint{5.534417in}{3.161498in}}%
\pgfpathlineto{\pgfqpoint{5.548569in}{3.173029in}}%
\pgfpathlineto{\pgfqpoint{5.562741in}{3.184716in}}%
\pgfpathlineto{\pgfqpoint{5.576932in}{3.196562in}}%
\pgfpathlineto{\pgfqpoint{5.591142in}{3.208565in}}%
\pgfpathlineto{\pgfqpoint{5.598166in}{3.211052in}}%
\pgfpathlineto{\pgfqpoint{5.605183in}{3.213505in}}%
\pgfpathlineto{\pgfqpoint{5.612192in}{3.215929in}}%
\pgfpathlineto{\pgfqpoint{5.619194in}{3.218329in}}%
\pgfpathlineto{\pgfqpoint{5.605008in}{3.206760in}}%
\pgfpathlineto{\pgfqpoint{5.590841in}{3.195347in}}%
\pgfpathlineto{\pgfqpoint{5.576694in}{3.184091in}}%
\pgfpathlineto{\pgfqpoint{5.562564in}{3.172992in}}%
\pgfpathlineto{\pgfqpoint{5.555538in}{3.170150in}}%
\pgfpathlineto{\pgfqpoint{5.548505in}{3.167290in}}%
\pgfpathlineto{\pgfqpoint{5.541464in}{3.164407in}}%
\pgfpathlineto{\pgfqpoint{5.534417in}{3.161498in}}%
\pgfpathclose%
\pgfusepath{fill}%
\end{pgfscope}%
\begin{pgfscope}%
\pgfpathrectangle{\pgfqpoint{1.254980in}{0.150000in}}{\pgfqpoint{5.490039in}{5.490039in}}%
\pgfusepath{clip}%
\pgfsetbuttcap%
\pgfsetroundjoin%
\definecolor{currentfill}{rgb}{0.282656,0.100196,0.422160}%
\pgfsetfillcolor{currentfill}%
\pgfsetfillopacity{0.700000}%
\pgfsetlinewidth{0.000000pt}%
\definecolor{currentstroke}{rgb}{0.000000,0.000000,0.000000}%
\pgfsetstrokecolor{currentstroke}%
\pgfsetdash{}{0pt}%
\pgfpathmoveto{\pgfqpoint{2.862359in}{1.868938in}}%
\pgfpathlineto{\pgfqpoint{2.875664in}{1.857236in}}%
\pgfpathlineto{\pgfqpoint{2.888966in}{1.845745in}}%
\pgfpathlineto{\pgfqpoint{2.902265in}{1.834464in}}%
\pgfpathlineto{\pgfqpoint{2.915561in}{1.823393in}}%
\pgfpathlineto{\pgfqpoint{2.923832in}{1.826838in}}%
\pgfpathlineto{\pgfqpoint{2.932090in}{1.830474in}}%
\pgfpathlineto{\pgfqpoint{2.940337in}{1.834295in}}%
\pgfpathlineto{\pgfqpoint{2.948572in}{1.838298in}}%
\pgfpathlineto{\pgfqpoint{2.935307in}{1.848894in}}%
\pgfpathlineto{\pgfqpoint{2.922040in}{1.859698in}}%
\pgfpathlineto{\pgfqpoint{2.908771in}{1.870713in}}%
\pgfpathlineto{\pgfqpoint{2.895499in}{1.881939in}}%
\pgfpathlineto{\pgfqpoint{2.887232in}{1.878401in}}%
\pgfpathlineto{\pgfqpoint{2.878954in}{1.875052in}}%
\pgfpathlineto{\pgfqpoint{2.870663in}{1.871896in}}%
\pgfpathlineto{\pgfqpoint{2.862359in}{1.868938in}}%
\pgfpathclose%
\pgfusepath{fill}%
\end{pgfscope}%
\begin{pgfscope}%
\pgfpathrectangle{\pgfqpoint{1.254980in}{0.150000in}}{\pgfqpoint{5.490039in}{5.490039in}}%
\pgfusepath{clip}%
\pgfsetbuttcap%
\pgfsetroundjoin%
\definecolor{currentfill}{rgb}{0.183898,0.422383,0.556944}%
\pgfsetfillcolor{currentfill}%
\pgfsetfillopacity{0.700000}%
\pgfsetlinewidth{0.000000pt}%
\definecolor{currentstroke}{rgb}{0.000000,0.000000,0.000000}%
\pgfsetstrokecolor{currentstroke}%
\pgfsetdash{}{0pt}%
\pgfpathmoveto{\pgfqpoint{4.712708in}{2.538562in}}%
\pgfpathlineto{\pgfqpoint{4.726396in}{2.547116in}}%
\pgfpathlineto{\pgfqpoint{4.740100in}{2.555831in}}%
\pgfpathlineto{\pgfqpoint{4.753818in}{2.564707in}}%
\pgfpathlineto{\pgfqpoint{4.767552in}{2.573743in}}%
\pgfpathlineto{\pgfqpoint{4.775039in}{2.582158in}}%
\pgfpathlineto{\pgfqpoint{4.782520in}{2.590469in}}%
\pgfpathlineto{\pgfqpoint{4.789995in}{2.598679in}}%
\pgfpathlineto{\pgfqpoint{4.797462in}{2.606789in}}%
\pgfpathlineto{\pgfqpoint{4.783735in}{2.597852in}}%
\pgfpathlineto{\pgfqpoint{4.770024in}{2.589075in}}%
\pgfpathlineto{\pgfqpoint{4.756327in}{2.580459in}}%
\pgfpathlineto{\pgfqpoint{4.742646in}{2.572003in}}%
\pgfpathlineto{\pgfqpoint{4.735171in}{2.563783in}}%
\pgfpathlineto{\pgfqpoint{4.727690in}{2.555471in}}%
\pgfpathlineto{\pgfqpoint{4.720202in}{2.547064in}}%
\pgfpathlineto{\pgfqpoint{4.712708in}{2.538562in}}%
\pgfpathclose%
\pgfusepath{fill}%
\end{pgfscope}%
\begin{pgfscope}%
\pgfpathrectangle{\pgfqpoint{1.254980in}{0.150000in}}{\pgfqpoint{5.490039in}{5.490039in}}%
\pgfusepath{clip}%
\pgfsetbuttcap%
\pgfsetroundjoin%
\definecolor{currentfill}{rgb}{0.227802,0.326594,0.546532}%
\pgfsetfillcolor{currentfill}%
\pgfsetfillopacity{0.700000}%
\pgfsetlinewidth{0.000000pt}%
\definecolor{currentstroke}{rgb}{0.000000,0.000000,0.000000}%
\pgfsetstrokecolor{currentstroke}%
\pgfsetdash{}{0pt}%
\pgfpathmoveto{\pgfqpoint{4.428530in}{2.297369in}}%
\pgfpathlineto{\pgfqpoint{4.442073in}{2.304089in}}%
\pgfpathlineto{\pgfqpoint{4.455628in}{2.310971in}}%
\pgfpathlineto{\pgfqpoint{4.469197in}{2.318014in}}%
\pgfpathlineto{\pgfqpoint{4.482779in}{2.325220in}}%
\pgfpathlineto{\pgfqpoint{4.490382in}{2.335321in}}%
\pgfpathlineto{\pgfqpoint{4.497979in}{2.345330in}}%
\pgfpathlineto{\pgfqpoint{4.505570in}{2.355246in}}%
\pgfpathlineto{\pgfqpoint{4.513155in}{2.365069in}}%
\pgfpathlineto{\pgfqpoint{4.499578in}{2.357844in}}%
\pgfpathlineto{\pgfqpoint{4.486013in}{2.350782in}}%
\pgfpathlineto{\pgfqpoint{4.472462in}{2.343881in}}%
\pgfpathlineto{\pgfqpoint{4.458924in}{2.337142in}}%
\pgfpathlineto{\pgfqpoint{4.451334in}{2.327327in}}%
\pgfpathlineto{\pgfqpoint{4.443739in}{2.317427in}}%
\pgfpathlineto{\pgfqpoint{4.436137in}{2.307441in}}%
\pgfpathlineto{\pgfqpoint{4.428530in}{2.297369in}}%
\pgfpathclose%
\pgfusepath{fill}%
\end{pgfscope}%
\begin{pgfscope}%
\pgfpathrectangle{\pgfqpoint{1.254980in}{0.150000in}}{\pgfqpoint{5.490039in}{5.490039in}}%
\pgfusepath{clip}%
\pgfsetbuttcap%
\pgfsetroundjoin%
\definecolor{currentfill}{rgb}{0.268510,0.009605,0.335427}%
\pgfsetfillcolor{currentfill}%
\pgfsetfillopacity{0.700000}%
\pgfsetlinewidth{0.000000pt}%
\definecolor{currentstroke}{rgb}{0.000000,0.000000,0.000000}%
\pgfsetstrokecolor{currentstroke}%
\pgfsetdash{}{0pt}%
\pgfpathmoveto{\pgfqpoint{3.245981in}{1.694951in}}%
\pgfpathlineto{\pgfqpoint{3.259218in}{1.688913in}}%
\pgfpathlineto{\pgfqpoint{3.272457in}{1.683059in}}%
\pgfpathlineto{\pgfqpoint{3.285698in}{1.677390in}}%
\pgfpathlineto{\pgfqpoint{3.298941in}{1.671904in}}%
\pgfpathlineto{\pgfqpoint{3.306975in}{1.679293in}}%
\pgfpathlineto{\pgfqpoint{3.315001in}{1.686793in}}%
\pgfpathlineto{\pgfqpoint{3.323019in}{1.694402in}}%
\pgfpathlineto{\pgfqpoint{3.331029in}{1.702116in}}%
\pgfpathlineto{\pgfqpoint{3.317806in}{1.707190in}}%
\pgfpathlineto{\pgfqpoint{3.304585in}{1.712447in}}%
\pgfpathlineto{\pgfqpoint{3.291367in}{1.717889in}}%
\pgfpathlineto{\pgfqpoint{3.278151in}{1.723515in}}%
\pgfpathlineto{\pgfqpoint{3.270121in}{1.716203in}}%
\pgfpathlineto{\pgfqpoint{3.262082in}{1.709002in}}%
\pgfpathlineto{\pgfqpoint{3.254036in}{1.701917in}}%
\pgfpathlineto{\pgfqpoint{3.245981in}{1.694951in}}%
\pgfpathclose%
\pgfusepath{fill}%
\end{pgfscope}%
\begin{pgfscope}%
\pgfpathrectangle{\pgfqpoint{1.254980in}{0.150000in}}{\pgfqpoint{5.490039in}{5.490039in}}%
\pgfusepath{clip}%
\pgfsetbuttcap%
\pgfsetroundjoin%
\definecolor{currentfill}{rgb}{0.269308,0.218818,0.509577}%
\pgfsetfillcolor{currentfill}%
\pgfsetfillopacity{0.700000}%
\pgfsetlinewidth{0.000000pt}%
\definecolor{currentstroke}{rgb}{0.000000,0.000000,0.000000}%
\pgfsetstrokecolor{currentstroke}%
\pgfsetdash{}{0pt}%
\pgfpathmoveto{\pgfqpoint{4.144342in}{2.061068in}}%
\pgfpathlineto{\pgfqpoint{4.157758in}{2.065471in}}%
\pgfpathlineto{\pgfqpoint{4.171185in}{2.070038in}}%
\pgfpathlineto{\pgfqpoint{4.184624in}{2.074768in}}%
\pgfpathlineto{\pgfqpoint{4.198073in}{2.079662in}}%
\pgfpathlineto{\pgfqpoint{4.205772in}{2.090784in}}%
\pgfpathlineto{\pgfqpoint{4.213465in}{2.101841in}}%
\pgfpathlineto{\pgfqpoint{4.221154in}{2.112833in}}%
\pgfpathlineto{\pgfqpoint{4.228837in}{2.123759in}}%
\pgfpathlineto{\pgfqpoint{4.215392in}{2.118732in}}%
\pgfpathlineto{\pgfqpoint{4.201958in}{2.113868in}}%
\pgfpathlineto{\pgfqpoint{4.188535in}{2.109167in}}%
\pgfpathlineto{\pgfqpoint{4.175123in}{2.104631in}}%
\pgfpathlineto{\pgfqpoint{4.167435in}{2.093828in}}%
\pgfpathlineto{\pgfqpoint{4.159742in}{2.082966in}}%
\pgfpathlineto{\pgfqpoint{4.152044in}{2.072046in}}%
\pgfpathlineto{\pgfqpoint{4.144342in}{2.061068in}}%
\pgfpathclose%
\pgfusepath{fill}%
\end{pgfscope}%
\begin{pgfscope}%
\pgfpathrectangle{\pgfqpoint{1.254980in}{0.150000in}}{\pgfqpoint{5.490039in}{5.490039in}}%
\pgfusepath{clip}%
\pgfsetbuttcap%
\pgfsetroundjoin%
\definecolor{currentfill}{rgb}{0.150148,0.676631,0.506589}%
\pgfsetfillcolor{currentfill}%
\pgfsetfillopacity{0.700000}%
\pgfsetlinewidth{0.000000pt}%
\definecolor{currentstroke}{rgb}{0.000000,0.000000,0.000000}%
\pgfsetstrokecolor{currentstroke}%
\pgfsetdash{}{0pt}%
\pgfpathmoveto{\pgfqpoint{5.619194in}{3.218329in}}%
\pgfpathlineto{\pgfqpoint{5.633400in}{3.230056in}}%
\pgfpathlineto{\pgfqpoint{5.647624in}{3.241940in}}%
\pgfpathlineto{\pgfqpoint{5.661868in}{3.253981in}}%
\pgfpathlineto{\pgfqpoint{5.676132in}{3.266180in}}%
\pgfpathlineto{\pgfqpoint{5.683101in}{3.268107in}}%
\pgfpathlineto{\pgfqpoint{5.690063in}{3.270013in}}%
\pgfpathlineto{\pgfqpoint{5.697018in}{3.271903in}}%
\pgfpathlineto{\pgfqpoint{5.703966in}{3.273782in}}%
\pgfpathlineto{\pgfqpoint{5.689729in}{3.262048in}}%
\pgfpathlineto{\pgfqpoint{5.675511in}{3.250471in}}%
\pgfpathlineto{\pgfqpoint{5.661312in}{3.239050in}}%
\pgfpathlineto{\pgfqpoint{5.647133in}{3.227785in}}%
\pgfpathlineto{\pgfqpoint{5.640158in}{3.225433in}}%
\pgfpathlineto{\pgfqpoint{5.633177in}{3.223076in}}%
\pgfpathlineto{\pgfqpoint{5.626189in}{3.220710in}}%
\pgfpathlineto{\pgfqpoint{5.619194in}{3.218329in}}%
\pgfpathclose%
\pgfusepath{fill}%
\end{pgfscope}%
\begin{pgfscope}%
\pgfpathrectangle{\pgfqpoint{1.254980in}{0.150000in}}{\pgfqpoint{5.490039in}{5.490039in}}%
\pgfusepath{clip}%
\pgfsetbuttcap%
\pgfsetroundjoin%
\definecolor{currentfill}{rgb}{0.137770,0.537492,0.554906}%
\pgfsetfillcolor{currentfill}%
\pgfsetfillopacity{0.700000}%
\pgfsetlinewidth{0.000000pt}%
\definecolor{currentstroke}{rgb}{0.000000,0.000000,0.000000}%
\pgfsetstrokecolor{currentstroke}%
\pgfsetdash{}{0pt}%
\pgfpathmoveto{\pgfqpoint{5.081507in}{2.835844in}}%
\pgfpathlineto{\pgfqpoint{5.095404in}{2.846162in}}%
\pgfpathlineto{\pgfqpoint{5.109318in}{2.856640in}}%
\pgfpathlineto{\pgfqpoint{5.123249in}{2.867277in}}%
\pgfpathlineto{\pgfqpoint{5.137197in}{2.878073in}}%
\pgfpathlineto{\pgfqpoint{5.144501in}{2.883835in}}%
\pgfpathlineto{\pgfqpoint{5.151797in}{2.889508in}}%
\pgfpathlineto{\pgfqpoint{5.159085in}{2.895094in}}%
\pgfpathlineto{\pgfqpoint{5.166365in}{2.900596in}}%
\pgfpathlineto{\pgfqpoint{5.152430in}{2.890050in}}%
\pgfpathlineto{\pgfqpoint{5.138512in}{2.879663in}}%
\pgfpathlineto{\pgfqpoint{5.124611in}{2.869435in}}%
\pgfpathlineto{\pgfqpoint{5.110727in}{2.859365in}}%
\pgfpathlineto{\pgfqpoint{5.103433in}{2.853602in}}%
\pgfpathlineto{\pgfqpoint{5.096132in}{2.847763in}}%
\pgfpathlineto{\pgfqpoint{5.088823in}{2.841845in}}%
\pgfpathlineto{\pgfqpoint{5.081507in}{2.835844in}}%
\pgfpathclose%
\pgfusepath{fill}%
\end{pgfscope}%
\begin{pgfscope}%
\pgfpathrectangle{\pgfqpoint{1.254980in}{0.150000in}}{\pgfqpoint{5.490039in}{5.490039in}}%
\pgfusepath{clip}%
\pgfsetbuttcap%
\pgfsetroundjoin%
\definecolor{currentfill}{rgb}{0.272594,0.025563,0.353093}%
\pgfsetfillcolor{currentfill}%
\pgfsetfillopacity{0.700000}%
\pgfsetlinewidth{0.000000pt}%
\definecolor{currentstroke}{rgb}{0.000000,0.000000,0.000000}%
\pgfsetstrokecolor{currentstroke}%
\pgfsetdash{}{0pt}%
\pgfpathmoveto{\pgfqpoint{3.107639in}{1.726971in}}%
\pgfpathlineto{\pgfqpoint{3.120891in}{1.718978in}}%
\pgfpathlineto{\pgfqpoint{3.134144in}{1.711177in}}%
\pgfpathlineto{\pgfqpoint{3.147397in}{1.703568in}}%
\pgfpathlineto{\pgfqpoint{3.160650in}{1.696150in}}%
\pgfpathlineto{\pgfqpoint{3.168764in}{1.702145in}}%
\pgfpathlineto{\pgfqpoint{3.176869in}{1.708284in}}%
\pgfpathlineto{\pgfqpoint{3.184964in}{1.714561in}}%
\pgfpathlineto{\pgfqpoint{3.193050in}{1.720973in}}%
\pgfpathlineto{\pgfqpoint{3.179820in}{1.727950in}}%
\pgfpathlineto{\pgfqpoint{3.166592in}{1.735118in}}%
\pgfpathlineto{\pgfqpoint{3.153364in}{1.742477in}}%
\pgfpathlineto{\pgfqpoint{3.140137in}{1.750028in}}%
\pgfpathlineto{\pgfqpoint{3.132027in}{1.744046in}}%
\pgfpathlineto{\pgfqpoint{3.123908in}{1.738207in}}%
\pgfpathlineto{\pgfqpoint{3.115778in}{1.732514in}}%
\pgfpathlineto{\pgfqpoint{3.107639in}{1.726971in}}%
\pgfpathclose%
\pgfusepath{fill}%
\end{pgfscope}%
\begin{pgfscope}%
\pgfpathrectangle{\pgfqpoint{1.254980in}{0.150000in}}{\pgfqpoint{5.490039in}{5.490039in}}%
\pgfusepath{clip}%
\pgfsetbuttcap%
\pgfsetroundjoin%
\definecolor{currentfill}{rgb}{0.197636,0.391528,0.554969}%
\pgfsetfillcolor{currentfill}%
\pgfsetfillopacity{0.700000}%
\pgfsetlinewidth{0.000000pt}%
\definecolor{currentstroke}{rgb}{0.000000,0.000000,0.000000}%
\pgfsetstrokecolor{currentstroke}%
\pgfsetdash{}{0pt}%
\pgfpathmoveto{\pgfqpoint{2.325125in}{2.529777in}}%
\pgfpathlineto{\pgfqpoint{2.338741in}{2.508151in}}%
\pgfpathlineto{\pgfqpoint{2.352343in}{2.486816in}}%
\pgfpathlineto{\pgfqpoint{2.365932in}{2.465771in}}%
\pgfpathlineto{\pgfqpoint{2.379509in}{2.445012in}}%
\pgfpathlineto{\pgfqpoint{2.388185in}{2.443466in}}%
\pgfpathlineto{\pgfqpoint{2.396842in}{2.442188in}}%
\pgfpathlineto{\pgfqpoint{2.405482in}{2.441172in}}%
\pgfpathlineto{\pgfqpoint{2.414103in}{2.440414in}}%
\pgfpathlineto{\pgfqpoint{2.400574in}{2.460668in}}%
\pgfpathlineto{\pgfqpoint{2.387034in}{2.481207in}}%
\pgfpathlineto{\pgfqpoint{2.373481in}{2.502034in}}%
\pgfpathlineto{\pgfqpoint{2.359916in}{2.523152in}}%
\pgfpathlineto{\pgfqpoint{2.351246in}{2.524404in}}%
\pgfpathlineto{\pgfqpoint{2.342558in}{2.525923in}}%
\pgfpathlineto{\pgfqpoint{2.333851in}{2.527712in}}%
\pgfpathlineto{\pgfqpoint{2.325125in}{2.529777in}}%
\pgfpathclose%
\pgfusepath{fill}%
\end{pgfscope}%
\begin{pgfscope}%
\pgfpathrectangle{\pgfqpoint{1.254980in}{0.150000in}}{\pgfqpoint{5.490039in}{5.490039in}}%
\pgfusepath{clip}%
\pgfsetbuttcap%
\pgfsetroundjoin%
\definecolor{currentfill}{rgb}{0.268510,0.009605,0.335427}%
\pgfsetfillcolor{currentfill}%
\pgfsetfillopacity{0.700000}%
\pgfsetlinewidth{0.000000pt}%
\definecolor{currentstroke}{rgb}{0.000000,0.000000,0.000000}%
\pgfsetstrokecolor{currentstroke}%
\pgfsetdash{}{0pt}%
\pgfpathmoveto{\pgfqpoint{3.383952in}{1.683637in}}%
\pgfpathlineto{\pgfqpoint{3.397191in}{1.679469in}}%
\pgfpathlineto{\pgfqpoint{3.410434in}{1.675479in}}%
\pgfpathlineto{\pgfqpoint{3.423680in}{1.671667in}}%
\pgfpathlineto{\pgfqpoint{3.436930in}{1.668032in}}%
\pgfpathlineto{\pgfqpoint{3.444897in}{1.676634in}}%
\pgfpathlineto{\pgfqpoint{3.452858in}{1.685319in}}%
\pgfpathlineto{\pgfqpoint{3.460812in}{1.694084in}}%
\pgfpathlineto{\pgfqpoint{3.468759in}{1.702924in}}%
\pgfpathlineto{\pgfqpoint{3.455525in}{1.706175in}}%
\pgfpathlineto{\pgfqpoint{3.442295in}{1.709604in}}%
\pgfpathlineto{\pgfqpoint{3.429069in}{1.713211in}}%
\pgfpathlineto{\pgfqpoint{3.415847in}{1.716996in}}%
\pgfpathlineto{\pgfqpoint{3.407884in}{1.708529in}}%
\pgfpathlineto{\pgfqpoint{3.399914in}{1.700144in}}%
\pgfpathlineto{\pgfqpoint{3.391937in}{1.691846in}}%
\pgfpathlineto{\pgfqpoint{3.383952in}{1.683637in}}%
\pgfpathclose%
\pgfusepath{fill}%
\end{pgfscope}%
\begin{pgfscope}%
\pgfpathrectangle{\pgfqpoint{1.254980in}{0.150000in}}{\pgfqpoint{5.490039in}{5.490039in}}%
\pgfusepath{clip}%
\pgfsetbuttcap%
\pgfsetroundjoin%
\definecolor{currentfill}{rgb}{0.175707,0.697900,0.491033}%
\pgfsetfillcolor{currentfill}%
\pgfsetfillopacity{0.700000}%
\pgfsetlinewidth{0.000000pt}%
\definecolor{currentstroke}{rgb}{0.000000,0.000000,0.000000}%
\pgfsetstrokecolor{currentstroke}%
\pgfsetdash{}{0pt}%
\pgfpathmoveto{\pgfqpoint{5.703966in}{3.273782in}}%
\pgfpathlineto{\pgfqpoint{5.718223in}{3.285673in}}%
\pgfpathlineto{\pgfqpoint{5.732499in}{3.297721in}}%
\pgfpathlineto{\pgfqpoint{5.746796in}{3.309925in}}%
\pgfpathlineto{\pgfqpoint{5.761112in}{3.322288in}}%
\pgfpathlineto{\pgfqpoint{5.768025in}{3.323676in}}%
\pgfpathlineto{\pgfqpoint{5.774931in}{3.325058in}}%
\pgfpathlineto{\pgfqpoint{5.781831in}{3.326437in}}%
\pgfpathlineto{\pgfqpoint{5.788723in}{3.327820in}}%
\pgfpathlineto{\pgfqpoint{5.774436in}{3.315953in}}%
\pgfpathlineto{\pgfqpoint{5.760168in}{3.304243in}}%
\pgfpathlineto{\pgfqpoint{5.745919in}{3.292689in}}%
\pgfpathlineto{\pgfqpoint{5.731691in}{3.281290in}}%
\pgfpathlineto{\pgfqpoint{5.724769in}{3.279404in}}%
\pgfpathlineto{\pgfqpoint{5.717841in}{3.277527in}}%
\pgfpathlineto{\pgfqpoint{5.710907in}{3.275655in}}%
\pgfpathlineto{\pgfqpoint{5.703966in}{3.273782in}}%
\pgfpathclose%
\pgfusepath{fill}%
\end{pgfscope}%
\begin{pgfscope}%
\pgfpathrectangle{\pgfqpoint{1.254980in}{0.150000in}}{\pgfqpoint{5.490039in}{5.490039in}}%
\pgfusepath{clip}%
\pgfsetbuttcap%
\pgfsetroundjoin%
\definecolor{currentfill}{rgb}{0.280894,0.078907,0.402329}%
\pgfsetfillcolor{currentfill}%
\pgfsetfillopacity{0.700000}%
\pgfsetlinewidth{0.000000pt}%
\definecolor{currentstroke}{rgb}{0.000000,0.000000,0.000000}%
\pgfsetstrokecolor{currentstroke}%
\pgfsetdash{}{0pt}%
\pgfpathmoveto{\pgfqpoint{2.915561in}{1.823393in}}%
\pgfpathlineto{\pgfqpoint{2.928855in}{1.812530in}}%
\pgfpathlineto{\pgfqpoint{2.942147in}{1.801872in}}%
\pgfpathlineto{\pgfqpoint{2.955437in}{1.791421in}}%
\pgfpathlineto{\pgfqpoint{2.968725in}{1.781173in}}%
\pgfpathlineto{\pgfqpoint{2.976963in}{1.785103in}}%
\pgfpathlineto{\pgfqpoint{2.985191in}{1.789217in}}%
\pgfpathlineto{\pgfqpoint{2.993407in}{1.793509in}}%
\pgfpathlineto{\pgfqpoint{3.001611in}{1.797974in}}%
\pgfpathlineto{\pgfqpoint{2.988354in}{1.807748in}}%
\pgfpathlineto{\pgfqpoint{2.975095in}{1.817726in}}%
\pgfpathlineto{\pgfqpoint{2.961834in}{1.827909in}}%
\pgfpathlineto{\pgfqpoint{2.948572in}{1.838298in}}%
\pgfpathlineto{\pgfqpoint{2.940337in}{1.834295in}}%
\pgfpathlineto{\pgfqpoint{2.932090in}{1.830474in}}%
\pgfpathlineto{\pgfqpoint{2.923832in}{1.826838in}}%
\pgfpathlineto{\pgfqpoint{2.915561in}{1.823393in}}%
\pgfpathclose%
\pgfusepath{fill}%
\end{pgfscope}%
\begin{pgfscope}%
\pgfpathrectangle{\pgfqpoint{1.254980in}{0.150000in}}{\pgfqpoint{5.490039in}{5.490039in}}%
\pgfusepath{clip}%
\pgfsetbuttcap%
\pgfsetroundjoin%
\definecolor{currentfill}{rgb}{0.266941,0.748751,0.440573}%
\pgfsetfillcolor{currentfill}%
\pgfsetfillopacity{0.700000}%
\pgfsetlinewidth{0.000000pt}%
\definecolor{currentstroke}{rgb}{0.000000,0.000000,0.000000}%
\pgfsetstrokecolor{currentstroke}%
\pgfsetdash{}{0pt}%
\pgfpathmoveto{\pgfqpoint{5.958166in}{3.431606in}}%
\pgfpathlineto{\pgfqpoint{5.972572in}{3.443798in}}%
\pgfpathlineto{\pgfqpoint{5.986998in}{3.456145in}}%
\pgfpathlineto{\pgfqpoint{6.001444in}{3.468649in}}%
\pgfpathlineto{\pgfqpoint{6.008193in}{3.468744in}}%
\pgfpathlineto{\pgfqpoint{6.014936in}{3.468883in}}%
\pgfpathlineto{\pgfqpoint{6.021675in}{3.469070in}}%
\pgfpathlineto{\pgfqpoint{6.028408in}{3.469314in}}%
\pgfpathlineto{\pgfqpoint{6.013998in}{3.457397in}}%
\pgfpathlineto{\pgfqpoint{5.999608in}{3.445635in}}%
\pgfpathlineto{\pgfqpoint{5.985238in}{3.434027in}}%
\pgfpathlineto{\pgfqpoint{5.978477in}{3.433338in}}%
\pgfpathlineto{\pgfqpoint{5.971712in}{3.432708in}}%
\pgfpathlineto{\pgfqpoint{5.964942in}{3.432133in}}%
\pgfpathlineto{\pgfqpoint{5.958166in}{3.431606in}}%
\pgfpathclose%
\pgfusepath{fill}%
\end{pgfscope}%
\begin{pgfscope}%
\pgfpathrectangle{\pgfqpoint{1.254980in}{0.150000in}}{\pgfqpoint{5.490039in}{5.490039in}}%
\pgfusepath{clip}%
\pgfsetbuttcap%
\pgfsetroundjoin%
\definecolor{currentfill}{rgb}{0.171176,0.452530,0.557965}%
\pgfsetfillcolor{currentfill}%
\pgfsetfillopacity{0.700000}%
\pgfsetlinewidth{0.000000pt}%
\definecolor{currentstroke}{rgb}{0.000000,0.000000,0.000000}%
\pgfsetstrokecolor{currentstroke}%
\pgfsetdash{}{0pt}%
\pgfpathmoveto{\pgfqpoint{4.797462in}{2.606789in}}%
\pgfpathlineto{\pgfqpoint{4.811204in}{2.615886in}}%
\pgfpathlineto{\pgfqpoint{4.824961in}{2.625144in}}%
\pgfpathlineto{\pgfqpoint{4.838734in}{2.634562in}}%
\pgfpathlineto{\pgfqpoint{4.852523in}{2.644141in}}%
\pgfpathlineto{\pgfqpoint{4.859976in}{2.652034in}}%
\pgfpathlineto{\pgfqpoint{4.867422in}{2.659823in}}%
\pgfpathlineto{\pgfqpoint{4.874861in}{2.667510in}}%
\pgfpathlineto{\pgfqpoint{4.882293in}{2.675096in}}%
\pgfpathlineto{\pgfqpoint{4.868512in}{2.665647in}}%
\pgfpathlineto{\pgfqpoint{4.854747in}{2.656358in}}%
\pgfpathlineto{\pgfqpoint{4.840998in}{2.647229in}}%
\pgfpathlineto{\pgfqpoint{4.827264in}{2.638260in}}%
\pgfpathlineto{\pgfqpoint{4.819823in}{2.630534in}}%
\pgfpathlineto{\pgfqpoint{4.812376in}{2.622715in}}%
\pgfpathlineto{\pgfqpoint{4.804923in}{2.614801in}}%
\pgfpathlineto{\pgfqpoint{4.797462in}{2.606789in}}%
\pgfpathclose%
\pgfusepath{fill}%
\end{pgfscope}%
\begin{pgfscope}%
\pgfpathrectangle{\pgfqpoint{1.254980in}{0.150000in}}{\pgfqpoint{5.490039in}{5.490039in}}%
\pgfusepath{clip}%
\pgfsetbuttcap%
\pgfsetroundjoin%
\definecolor{currentfill}{rgb}{0.258965,0.251537,0.524736}%
\pgfsetfillcolor{currentfill}%
\pgfsetfillopacity{0.700000}%
\pgfsetlinewidth{0.000000pt}%
\definecolor{currentstroke}{rgb}{0.000000,0.000000,0.000000}%
\pgfsetstrokecolor{currentstroke}%
\pgfsetdash{}{0pt}%
\pgfpathmoveto{\pgfqpoint{4.228837in}{2.123759in}}%
\pgfpathlineto{\pgfqpoint{4.242294in}{2.128949in}}%
\pgfpathlineto{\pgfqpoint{4.255762in}{2.134303in}}%
\pgfpathlineto{\pgfqpoint{4.269242in}{2.139819in}}%
\pgfpathlineto{\pgfqpoint{4.282734in}{2.145498in}}%
\pgfpathlineto{\pgfqpoint{4.290409in}{2.156472in}}%
\pgfpathlineto{\pgfqpoint{4.298079in}{2.167372in}}%
\pgfpathlineto{\pgfqpoint{4.305743in}{2.178195in}}%
\pgfpathlineto{\pgfqpoint{4.313402in}{2.188941in}}%
\pgfpathlineto{\pgfqpoint{4.299914in}{2.183157in}}%
\pgfpathlineto{\pgfqpoint{4.286438in}{2.177536in}}%
\pgfpathlineto{\pgfqpoint{4.272974in}{2.172077in}}%
\pgfpathlineto{\pgfqpoint{4.259521in}{2.166781in}}%
\pgfpathlineto{\pgfqpoint{4.251858in}{2.156129in}}%
\pgfpathlineto{\pgfqpoint{4.244189in}{2.145407in}}%
\pgfpathlineto{\pgfqpoint{4.236516in}{2.134617in}}%
\pgfpathlineto{\pgfqpoint{4.228837in}{2.123759in}}%
\pgfpathclose%
\pgfusepath{fill}%
\end{pgfscope}%
\begin{pgfscope}%
\pgfpathrectangle{\pgfqpoint{1.254980in}{0.150000in}}{\pgfqpoint{5.490039in}{5.490039in}}%
\pgfusepath{clip}%
\pgfsetbuttcap%
\pgfsetroundjoin%
\definecolor{currentfill}{rgb}{0.202219,0.715272,0.476084}%
\pgfsetfillcolor{currentfill}%
\pgfsetfillopacity{0.700000}%
\pgfsetlinewidth{0.000000pt}%
\definecolor{currentstroke}{rgb}{0.000000,0.000000,0.000000}%
\pgfsetstrokecolor{currentstroke}%
\pgfsetdash{}{0pt}%
\pgfpathmoveto{\pgfqpoint{5.788723in}{3.327820in}}%
\pgfpathlineto{\pgfqpoint{5.803031in}{3.339843in}}%
\pgfpathlineto{\pgfqpoint{5.817358in}{3.352022in}}%
\pgfpathlineto{\pgfqpoint{5.831706in}{3.364359in}}%
\pgfpathlineto{\pgfqpoint{5.846074in}{3.376852in}}%
\pgfpathlineto{\pgfqpoint{5.852930in}{3.377729in}}%
\pgfpathlineto{\pgfqpoint{5.859779in}{3.378613in}}%
\pgfpathlineto{\pgfqpoint{5.866622in}{3.379510in}}%
\pgfpathlineto{\pgfqpoint{5.873459in}{3.380426in}}%
\pgfpathlineto{\pgfqpoint{5.859122in}{3.368459in}}%
\pgfpathlineto{\pgfqpoint{5.844805in}{3.356648in}}%
\pgfpathlineto{\pgfqpoint{5.830509in}{3.344993in}}%
\pgfpathlineto{\pgfqpoint{5.816232in}{3.333493in}}%
\pgfpathlineto{\pgfqpoint{5.809363in}{3.332042in}}%
\pgfpathlineto{\pgfqpoint{5.802489in}{3.330616in}}%
\pgfpathlineto{\pgfqpoint{5.795609in}{3.329211in}}%
\pgfpathlineto{\pgfqpoint{5.788723in}{3.327820in}}%
\pgfpathclose%
\pgfusepath{fill}%
\end{pgfscope}%
\begin{pgfscope}%
\pgfpathrectangle{\pgfqpoint{1.254980in}{0.150000in}}{\pgfqpoint{5.490039in}{5.490039in}}%
\pgfusepath{clip}%
\pgfsetbuttcap%
\pgfsetroundjoin%
\definecolor{currentfill}{rgb}{0.212395,0.359683,0.551710}%
\pgfsetfillcolor{currentfill}%
\pgfsetfillopacity{0.700000}%
\pgfsetlinewidth{0.000000pt}%
\definecolor{currentstroke}{rgb}{0.000000,0.000000,0.000000}%
\pgfsetstrokecolor{currentstroke}%
\pgfsetdash{}{0pt}%
\pgfpathmoveto{\pgfqpoint{4.513155in}{2.365069in}}%
\pgfpathlineto{\pgfqpoint{4.526747in}{2.372454in}}%
\pgfpathlineto{\pgfqpoint{4.540352in}{2.380001in}}%
\pgfpathlineto{\pgfqpoint{4.553970in}{2.387710in}}%
\pgfpathlineto{\pgfqpoint{4.567603in}{2.395579in}}%
\pgfpathlineto{\pgfqpoint{4.575178in}{2.405310in}}%
\pgfpathlineto{\pgfqpoint{4.582747in}{2.414942in}}%
\pgfpathlineto{\pgfqpoint{4.590310in}{2.424475in}}%
\pgfpathlineto{\pgfqpoint{4.597867in}{2.433910in}}%
\pgfpathlineto{\pgfqpoint{4.584239in}{2.426051in}}%
\pgfpathlineto{\pgfqpoint{4.570625in}{2.418353in}}%
\pgfpathlineto{\pgfqpoint{4.557025in}{2.410817in}}%
\pgfpathlineto{\pgfqpoint{4.543439in}{2.403441in}}%
\pgfpathlineto{\pgfqpoint{4.535877in}{2.393985in}}%
\pgfpathlineto{\pgfqpoint{4.528309in}{2.384438in}}%
\pgfpathlineto{\pgfqpoint{4.520735in}{2.374799in}}%
\pgfpathlineto{\pgfqpoint{4.513155in}{2.365069in}}%
\pgfpathclose%
\pgfusepath{fill}%
\end{pgfscope}%
\begin{pgfscope}%
\pgfpathrectangle{\pgfqpoint{1.254980in}{0.150000in}}{\pgfqpoint{5.490039in}{5.490039in}}%
\pgfusepath{clip}%
\pgfsetbuttcap%
\pgfsetroundjoin%
\definecolor{currentfill}{rgb}{0.281924,0.089666,0.412415}%
\pgfsetfillcolor{currentfill}%
\pgfsetfillopacity{0.700000}%
\pgfsetlinewidth{0.000000pt}%
\definecolor{currentstroke}{rgb}{0.000000,0.000000,0.000000}%
\pgfsetstrokecolor{currentstroke}%
\pgfsetdash{}{0pt}%
\pgfpathmoveto{\pgfqpoint{3.775499in}{1.801084in}}%
\pgfpathlineto{\pgfqpoint{3.788798in}{1.801747in}}%
\pgfpathlineto{\pgfqpoint{3.802105in}{1.802577in}}%
\pgfpathlineto{\pgfqpoint{3.815420in}{1.803575in}}%
\pgfpathlineto{\pgfqpoint{3.828743in}{1.804741in}}%
\pgfpathlineto{\pgfqpoint{3.836559in}{1.815690in}}%
\pgfpathlineto{\pgfqpoint{3.844371in}{1.826639in}}%
\pgfpathlineto{\pgfqpoint{3.852177in}{1.837583in}}%
\pgfpathlineto{\pgfqpoint{3.859979in}{1.848522in}}%
\pgfpathlineto{\pgfqpoint{3.846664in}{1.847084in}}%
\pgfpathlineto{\pgfqpoint{3.833357in}{1.845813in}}%
\pgfpathlineto{\pgfqpoint{3.820058in}{1.844709in}}%
\pgfpathlineto{\pgfqpoint{3.806767in}{1.843774in}}%
\pgfpathlineto{\pgfqpoint{3.798957in}{1.833098in}}%
\pgfpathlineto{\pgfqpoint{3.791143in}{1.822422in}}%
\pgfpathlineto{\pgfqpoint{3.783323in}{1.811750in}}%
\pgfpathlineto{\pgfqpoint{3.775499in}{1.801084in}}%
\pgfpathclose%
\pgfusepath{fill}%
\end{pgfscope}%
\begin{pgfscope}%
\pgfpathrectangle{\pgfqpoint{1.254980in}{0.150000in}}{\pgfqpoint{5.490039in}{5.490039in}}%
\pgfusepath{clip}%
\pgfsetbuttcap%
\pgfsetroundjoin%
\definecolor{currentfill}{rgb}{0.278791,0.062145,0.386592}%
\pgfsetfillcolor{currentfill}%
\pgfsetfillopacity{0.700000}%
\pgfsetlinewidth{0.000000pt}%
\definecolor{currentstroke}{rgb}{0.000000,0.000000,0.000000}%
\pgfsetstrokecolor{currentstroke}%
\pgfsetdash{}{0pt}%
\pgfpathmoveto{\pgfqpoint{3.690987in}{1.758773in}}%
\pgfpathlineto{\pgfqpoint{3.704267in}{1.758459in}}%
\pgfpathlineto{\pgfqpoint{3.717554in}{1.758314in}}%
\pgfpathlineto{\pgfqpoint{3.730848in}{1.758338in}}%
\pgfpathlineto{\pgfqpoint{3.744149in}{1.758532in}}%
\pgfpathlineto{\pgfqpoint{3.751994in}{1.769148in}}%
\pgfpathlineto{\pgfqpoint{3.759834in}{1.779781in}}%
\pgfpathlineto{\pgfqpoint{3.767669in}{1.790427in}}%
\pgfpathlineto{\pgfqpoint{3.775499in}{1.801084in}}%
\pgfpathlineto{\pgfqpoint{3.762207in}{1.800591in}}%
\pgfpathlineto{\pgfqpoint{3.748922in}{1.800266in}}%
\pgfpathlineto{\pgfqpoint{3.735645in}{1.800110in}}%
\pgfpathlineto{\pgfqpoint{3.722374in}{1.800125in}}%
\pgfpathlineto{\pgfqpoint{3.714535in}{1.789757in}}%
\pgfpathlineto{\pgfqpoint{3.706691in}{1.779408in}}%
\pgfpathlineto{\pgfqpoint{3.698842in}{1.769079in}}%
\pgfpathlineto{\pgfqpoint{3.690987in}{1.758773in}}%
\pgfpathclose%
\pgfusepath{fill}%
\end{pgfscope}%
\begin{pgfscope}%
\pgfpathrectangle{\pgfqpoint{1.254980in}{0.150000in}}{\pgfqpoint{5.490039in}{5.490039in}}%
\pgfusepath{clip}%
\pgfsetbuttcap%
\pgfsetroundjoin%
\definecolor{currentfill}{rgb}{0.128729,0.563265,0.551229}%
\pgfsetfillcolor{currentfill}%
\pgfsetfillopacity{0.700000}%
\pgfsetlinewidth{0.000000pt}%
\definecolor{currentstroke}{rgb}{0.000000,0.000000,0.000000}%
\pgfsetstrokecolor{currentstroke}%
\pgfsetdash{}{0pt}%
\pgfpathmoveto{\pgfqpoint{5.166365in}{2.900596in}}%
\pgfpathlineto{\pgfqpoint{5.180318in}{2.911301in}}%
\pgfpathlineto{\pgfqpoint{5.194288in}{2.922165in}}%
\pgfpathlineto{\pgfqpoint{5.208276in}{2.933188in}}%
\pgfpathlineto{\pgfqpoint{5.222281in}{2.944371in}}%
\pgfpathlineto{\pgfqpoint{5.229539in}{2.949523in}}%
\pgfpathlineto{\pgfqpoint{5.236790in}{2.954590in}}%
\pgfpathlineto{\pgfqpoint{5.244033in}{2.959577in}}%
\pgfpathlineto{\pgfqpoint{5.251268in}{2.964485in}}%
\pgfpathlineto{\pgfqpoint{5.237277in}{2.953583in}}%
\pgfpathlineto{\pgfqpoint{5.223304in}{2.942841in}}%
\pgfpathlineto{\pgfqpoint{5.209349in}{2.932257in}}%
\pgfpathlineto{\pgfqpoint{5.195411in}{2.921831in}}%
\pgfpathlineto{\pgfqpoint{5.188161in}{2.916632in}}%
\pgfpathlineto{\pgfqpoint{5.180903in}{2.911362in}}%
\pgfpathlineto{\pgfqpoint{5.173638in}{2.906018in}}%
\pgfpathlineto{\pgfqpoint{5.166365in}{2.900596in}}%
\pgfpathclose%
\pgfusepath{fill}%
\end{pgfscope}%
\begin{pgfscope}%
\pgfpathrectangle{\pgfqpoint{1.254980in}{0.150000in}}{\pgfqpoint{5.490039in}{5.490039in}}%
\pgfusepath{clip}%
\pgfsetbuttcap%
\pgfsetroundjoin%
\definecolor{currentfill}{rgb}{0.283197,0.115680,0.436115}%
\pgfsetfillcolor{currentfill}%
\pgfsetfillopacity{0.700000}%
\pgfsetlinewidth{0.000000pt}%
\definecolor{currentstroke}{rgb}{0.000000,0.000000,0.000000}%
\pgfsetstrokecolor{currentstroke}%
\pgfsetdash{}{0pt}%
\pgfpathmoveto{\pgfqpoint{3.859979in}{1.848522in}}%
\pgfpathlineto{\pgfqpoint{3.873302in}{1.850128in}}%
\pgfpathlineto{\pgfqpoint{3.886634in}{1.851900in}}%
\pgfpathlineto{\pgfqpoint{3.899974in}{1.853839in}}%
\pgfpathlineto{\pgfqpoint{3.913323in}{1.855944in}}%
\pgfpathlineto{\pgfqpoint{3.921113in}{1.867131in}}%
\pgfpathlineto{\pgfqpoint{3.928898in}{1.878300in}}%
\pgfpathlineto{\pgfqpoint{3.936679in}{1.889450in}}%
\pgfpathlineto{\pgfqpoint{3.944454in}{1.900578in}}%
\pgfpathlineto{\pgfqpoint{3.931111in}{1.898227in}}%
\pgfpathlineto{\pgfqpoint{3.917778in}{1.896043in}}%
\pgfpathlineto{\pgfqpoint{3.904453in}{1.894026in}}%
\pgfpathlineto{\pgfqpoint{3.891136in}{1.892175in}}%
\pgfpathlineto{\pgfqpoint{3.883354in}{1.881281in}}%
\pgfpathlineto{\pgfqpoint{3.875567in}{1.870374in}}%
\pgfpathlineto{\pgfqpoint{3.867775in}{1.859453in}}%
\pgfpathlineto{\pgfqpoint{3.859979in}{1.848522in}}%
\pgfpathclose%
\pgfusepath{fill}%
\end{pgfscope}%
\begin{pgfscope}%
\pgfpathrectangle{\pgfqpoint{1.254980in}{0.150000in}}{\pgfqpoint{5.490039in}{5.490039in}}%
\pgfusepath{clip}%
\pgfsetbuttcap%
\pgfsetroundjoin%
\definecolor{currentfill}{rgb}{0.239374,0.735588,0.455688}%
\pgfsetfillcolor{currentfill}%
\pgfsetfillopacity{0.700000}%
\pgfsetlinewidth{0.000000pt}%
\definecolor{currentstroke}{rgb}{0.000000,0.000000,0.000000}%
\pgfsetstrokecolor{currentstroke}%
\pgfsetdash{}{0pt}%
\pgfpathmoveto{\pgfqpoint{5.873459in}{3.380426in}}%
\pgfpathlineto{\pgfqpoint{5.887816in}{3.392549in}}%
\pgfpathlineto{\pgfqpoint{5.902193in}{3.404829in}}%
\pgfpathlineto{\pgfqpoint{5.916591in}{3.417265in}}%
\pgfpathlineto{\pgfqpoint{5.931010in}{3.429858in}}%
\pgfpathlineto{\pgfqpoint{5.937808in}{3.430253in}}%
\pgfpathlineto{\pgfqpoint{5.944600in}{3.430672in}}%
\pgfpathlineto{\pgfqpoint{5.951386in}{3.431121in}}%
\pgfpathlineto{\pgfqpoint{5.958166in}{3.431606in}}%
\pgfpathlineto{\pgfqpoint{5.943782in}{3.419571in}}%
\pgfpathlineto{\pgfqpoint{5.929417in}{3.407691in}}%
\pgfpathlineto{\pgfqpoint{5.915073in}{3.395966in}}%
\pgfpathlineto{\pgfqpoint{5.900749in}{3.384397in}}%
\pgfpathlineto{\pgfqpoint{5.893935in}{3.383346in}}%
\pgfpathlineto{\pgfqpoint{5.887115in}{3.382338in}}%
\pgfpathlineto{\pgfqpoint{5.880290in}{3.381367in}}%
\pgfpathlineto{\pgfqpoint{5.873459in}{3.380426in}}%
\pgfpathclose%
\pgfusepath{fill}%
\end{pgfscope}%
\begin{pgfscope}%
\pgfpathrectangle{\pgfqpoint{1.254980in}{0.150000in}}{\pgfqpoint{5.490039in}{5.490039in}}%
\pgfusepath{clip}%
\pgfsetbuttcap%
\pgfsetroundjoin%
\definecolor{currentfill}{rgb}{0.276022,0.044167,0.370164}%
\pgfsetfillcolor{currentfill}%
\pgfsetfillopacity{0.700000}%
\pgfsetlinewidth{0.000000pt}%
\definecolor{currentstroke}{rgb}{0.000000,0.000000,0.000000}%
\pgfsetstrokecolor{currentstroke}%
\pgfsetdash{}{0pt}%
\pgfpathmoveto{\pgfqpoint{3.606412in}{1.722121in}}%
\pgfpathlineto{\pgfqpoint{3.619678in}{1.720795in}}%
\pgfpathlineto{\pgfqpoint{3.632950in}{1.719640in}}%
\pgfpathlineto{\pgfqpoint{3.646228in}{1.718657in}}%
\pgfpathlineto{\pgfqpoint{3.659512in}{1.717844in}}%
\pgfpathlineto{\pgfqpoint{3.667389in}{1.728026in}}%
\pgfpathlineto{\pgfqpoint{3.675260in}{1.738244in}}%
\pgfpathlineto{\pgfqpoint{3.683126in}{1.748494in}}%
\pgfpathlineto{\pgfqpoint{3.690987in}{1.758773in}}%
\pgfpathlineto{\pgfqpoint{3.677713in}{1.759258in}}%
\pgfpathlineto{\pgfqpoint{3.664446in}{1.759914in}}%
\pgfpathlineto{\pgfqpoint{3.651186in}{1.760741in}}%
\pgfpathlineto{\pgfqpoint{3.637931in}{1.761739in}}%
\pgfpathlineto{\pgfqpoint{3.630060in}{1.751777in}}%
\pgfpathlineto{\pgfqpoint{3.622183in}{1.741852in}}%
\pgfpathlineto{\pgfqpoint{3.614300in}{1.731965in}}%
\pgfpathlineto{\pgfqpoint{3.606412in}{1.722121in}}%
\pgfpathclose%
\pgfusepath{fill}%
\end{pgfscope}%
\begin{pgfscope}%
\pgfpathrectangle{\pgfqpoint{1.254980in}{0.150000in}}{\pgfqpoint{5.490039in}{5.490039in}}%
\pgfusepath{clip}%
\pgfsetbuttcap%
\pgfsetroundjoin%
\definecolor{currentfill}{rgb}{0.180629,0.429975,0.557282}%
\pgfsetfillcolor{currentfill}%
\pgfsetfillopacity{0.700000}%
\pgfsetlinewidth{0.000000pt}%
\definecolor{currentstroke}{rgb}{0.000000,0.000000,0.000000}%
\pgfsetstrokecolor{currentstroke}%
\pgfsetdash{}{0pt}%
\pgfpathmoveto{\pgfqpoint{2.270528in}{2.619261in}}%
\pgfpathlineto{\pgfqpoint{2.284199in}{2.596437in}}%
\pgfpathlineto{\pgfqpoint{2.297855in}{2.573918in}}%
\pgfpathlineto{\pgfqpoint{2.311497in}{2.551699in}}%
\pgfpathlineto{\pgfqpoint{2.325125in}{2.529777in}}%
\pgfpathlineto{\pgfqpoint{2.333851in}{2.527712in}}%
\pgfpathlineto{\pgfqpoint{2.342558in}{2.525923in}}%
\pgfpathlineto{\pgfqpoint{2.351246in}{2.524404in}}%
\pgfpathlineto{\pgfqpoint{2.359916in}{2.523152in}}%
\pgfpathlineto{\pgfqpoint{2.346338in}{2.544563in}}%
\pgfpathlineto{\pgfqpoint{2.332746in}{2.566271in}}%
\pgfpathlineto{\pgfqpoint{2.319142in}{2.588279in}}%
\pgfpathlineto{\pgfqpoint{2.305523in}{2.610589in}}%
\pgfpathlineto{\pgfqpoint{2.296804in}{2.612340in}}%
\pgfpathlineto{\pgfqpoint{2.288065in}{2.614366in}}%
\pgfpathlineto{\pgfqpoint{2.279307in}{2.616671in}}%
\pgfpathlineto{\pgfqpoint{2.270528in}{2.619261in}}%
\pgfpathclose%
\pgfusepath{fill}%
\end{pgfscope}%
\begin{pgfscope}%
\pgfpathrectangle{\pgfqpoint{1.254980in}{0.150000in}}{\pgfqpoint{5.490039in}{5.490039in}}%
\pgfusepath{clip}%
\pgfsetbuttcap%
\pgfsetroundjoin%
\definecolor{currentfill}{rgb}{0.282290,0.145912,0.461510}%
\pgfsetfillcolor{currentfill}%
\pgfsetfillopacity{0.700000}%
\pgfsetlinewidth{0.000000pt}%
\definecolor{currentstroke}{rgb}{0.000000,0.000000,0.000000}%
\pgfsetstrokecolor{currentstroke}%
\pgfsetdash{}{0pt}%
\pgfpathmoveto{\pgfqpoint{3.944454in}{1.900578in}}%
\pgfpathlineto{\pgfqpoint{3.957806in}{1.903094in}}%
\pgfpathlineto{\pgfqpoint{3.971167in}{1.905776in}}%
\pgfpathlineto{\pgfqpoint{3.984537in}{1.908623in}}%
\pgfpathlineto{\pgfqpoint{3.997917in}{1.911634in}}%
\pgfpathlineto{\pgfqpoint{4.005682in}{1.922967in}}%
\pgfpathlineto{\pgfqpoint{4.013442in}{1.934267in}}%
\pgfpathlineto{\pgfqpoint{4.021198in}{1.945533in}}%
\pgfpathlineto{\pgfqpoint{4.028949in}{1.956762in}}%
\pgfpathlineto{\pgfqpoint{4.015574in}{1.953532in}}%
\pgfpathlineto{\pgfqpoint{4.002209in}{1.950468in}}%
\pgfpathlineto{\pgfqpoint{3.988854in}{1.947568in}}%
\pgfpathlineto{\pgfqpoint{3.975508in}{1.944834in}}%
\pgfpathlineto{\pgfqpoint{3.967752in}{1.933812in}}%
\pgfpathlineto{\pgfqpoint{3.959991in}{1.922761in}}%
\pgfpathlineto{\pgfqpoint{3.952225in}{1.911682in}}%
\pgfpathlineto{\pgfqpoint{3.944454in}{1.900578in}}%
\pgfpathclose%
\pgfusepath{fill}%
\end{pgfscope}%
\begin{pgfscope}%
\pgfpathrectangle{\pgfqpoint{1.254980in}{0.150000in}}{\pgfqpoint{5.490039in}{5.490039in}}%
\pgfusepath{clip}%
\pgfsetbuttcap%
\pgfsetroundjoin%
\definecolor{currentfill}{rgb}{0.278791,0.062145,0.386592}%
\pgfsetfillcolor{currentfill}%
\pgfsetfillopacity{0.700000}%
\pgfsetlinewidth{0.000000pt}%
\definecolor{currentstroke}{rgb}{0.000000,0.000000,0.000000}%
\pgfsetstrokecolor{currentstroke}%
\pgfsetdash{}{0pt}%
\pgfpathmoveto{\pgfqpoint{2.968725in}{1.781173in}}%
\pgfpathlineto{\pgfqpoint{2.982011in}{1.771128in}}%
\pgfpathlineto{\pgfqpoint{2.995295in}{1.761284in}}%
\pgfpathlineto{\pgfqpoint{3.008578in}{1.751641in}}%
\pgfpathlineto{\pgfqpoint{3.021861in}{1.742197in}}%
\pgfpathlineto{\pgfqpoint{3.030069in}{1.746611in}}%
\pgfpathlineto{\pgfqpoint{3.038267in}{1.751201in}}%
\pgfpathlineto{\pgfqpoint{3.046454in}{1.755962in}}%
\pgfpathlineto{\pgfqpoint{3.054630in}{1.760890in}}%
\pgfpathlineto{\pgfqpoint{3.041377in}{1.769861in}}%
\pgfpathlineto{\pgfqpoint{3.028123in}{1.779032in}}%
\pgfpathlineto{\pgfqpoint{3.014868in}{1.788402in}}%
\pgfpathlineto{\pgfqpoint{3.001611in}{1.797974in}}%
\pgfpathlineto{\pgfqpoint{2.993407in}{1.793509in}}%
\pgfpathlineto{\pgfqpoint{2.985191in}{1.789217in}}%
\pgfpathlineto{\pgfqpoint{2.976963in}{1.785103in}}%
\pgfpathlineto{\pgfqpoint{2.968725in}{1.781173in}}%
\pgfpathclose%
\pgfusepath{fill}%
\end{pgfscope}%
\begin{pgfscope}%
\pgfpathrectangle{\pgfqpoint{1.254980in}{0.150000in}}{\pgfqpoint{5.490039in}{5.490039in}}%
\pgfusepath{clip}%
\pgfsetbuttcap%
\pgfsetroundjoin%
\definecolor{currentfill}{rgb}{0.268510,0.009605,0.335427}%
\pgfsetfillcolor{currentfill}%
\pgfsetfillopacity{0.700000}%
\pgfsetlinewidth{0.000000pt}%
\definecolor{currentstroke}{rgb}{0.000000,0.000000,0.000000}%
\pgfsetstrokecolor{currentstroke}%
\pgfsetdash{}{0pt}%
\pgfpathmoveto{\pgfqpoint{3.298941in}{1.671904in}}%
\pgfpathlineto{\pgfqpoint{3.312187in}{1.666601in}}%
\pgfpathlineto{\pgfqpoint{3.325435in}{1.661480in}}%
\pgfpathlineto{\pgfqpoint{3.338686in}{1.656540in}}%
\pgfpathlineto{\pgfqpoint{3.351940in}{1.651780in}}%
\pgfpathlineto{\pgfqpoint{3.359955in}{1.659590in}}%
\pgfpathlineto{\pgfqpoint{3.367961in}{1.667506in}}%
\pgfpathlineto{\pgfqpoint{3.375961in}{1.675523in}}%
\pgfpathlineto{\pgfqpoint{3.383952in}{1.683637in}}%
\pgfpathlineto{\pgfqpoint{3.370717in}{1.687986in}}%
\pgfpathlineto{\pgfqpoint{3.357485in}{1.692515in}}%
\pgfpathlineto{\pgfqpoint{3.344255in}{1.697224in}}%
\pgfpathlineto{\pgfqpoint{3.331029in}{1.702116in}}%
\pgfpathlineto{\pgfqpoint{3.323019in}{1.694402in}}%
\pgfpathlineto{\pgfqpoint{3.315001in}{1.686793in}}%
\pgfpathlineto{\pgfqpoint{3.306975in}{1.679293in}}%
\pgfpathlineto{\pgfqpoint{3.298941in}{1.671904in}}%
\pgfpathclose%
\pgfusepath{fill}%
\end{pgfscope}%
\begin{pgfscope}%
\pgfpathrectangle{\pgfqpoint{1.254980in}{0.150000in}}{\pgfqpoint{5.490039in}{5.490039in}}%
\pgfusepath{clip}%
\pgfsetbuttcap%
\pgfsetroundjoin%
\definecolor{currentfill}{rgb}{0.272594,0.025563,0.353093}%
\pgfsetfillcolor{currentfill}%
\pgfsetfillopacity{0.700000}%
\pgfsetlinewidth{0.000000pt}%
\definecolor{currentstroke}{rgb}{0.000000,0.000000,0.000000}%
\pgfsetstrokecolor{currentstroke}%
\pgfsetdash{}{0pt}%
\pgfpathmoveto{\pgfqpoint{3.521739in}{1.691681in}}%
\pgfpathlineto{\pgfqpoint{3.534997in}{1.689307in}}%
\pgfpathlineto{\pgfqpoint{3.548259in}{1.687108in}}%
\pgfpathlineto{\pgfqpoint{3.561526in}{1.685081in}}%
\pgfpathlineto{\pgfqpoint{3.574799in}{1.683227in}}%
\pgfpathlineto{\pgfqpoint{3.582711in}{1.692872in}}%
\pgfpathlineto{\pgfqpoint{3.590617in}{1.702571in}}%
\pgfpathlineto{\pgfqpoint{3.598518in}{1.712322in}}%
\pgfpathlineto{\pgfqpoint{3.606412in}{1.722121in}}%
\pgfpathlineto{\pgfqpoint{3.593152in}{1.723619in}}%
\pgfpathlineto{\pgfqpoint{3.579897in}{1.725290in}}%
\pgfpathlineto{\pgfqpoint{3.566648in}{1.727134in}}%
\pgfpathlineto{\pgfqpoint{3.553405in}{1.729152in}}%
\pgfpathlineto{\pgfqpoint{3.545498in}{1.719698in}}%
\pgfpathlineto{\pgfqpoint{3.537584in}{1.710299in}}%
\pgfpathlineto{\pgfqpoint{3.529665in}{1.700959in}}%
\pgfpathlineto{\pgfqpoint{3.521739in}{1.691681in}}%
\pgfpathclose%
\pgfusepath{fill}%
\end{pgfscope}%
\begin{pgfscope}%
\pgfpathrectangle{\pgfqpoint{1.254980in}{0.150000in}}{\pgfqpoint{5.490039in}{5.490039in}}%
\pgfusepath{clip}%
\pgfsetbuttcap%
\pgfsetroundjoin%
\definecolor{currentfill}{rgb}{0.121831,0.589055,0.545623}%
\pgfsetfillcolor{currentfill}%
\pgfsetfillopacity{0.700000}%
\pgfsetlinewidth{0.000000pt}%
\definecolor{currentstroke}{rgb}{0.000000,0.000000,0.000000}%
\pgfsetstrokecolor{currentstroke}%
\pgfsetdash{}{0pt}%
\pgfpathmoveto{\pgfqpoint{5.251268in}{2.964485in}}%
\pgfpathlineto{\pgfqpoint{5.265276in}{2.975545in}}%
\pgfpathlineto{\pgfqpoint{5.279302in}{2.986763in}}%
\pgfpathlineto{\pgfqpoint{5.293346in}{2.998141in}}%
\pgfpathlineto{\pgfqpoint{5.307409in}{3.009678in}}%
\pgfpathlineto{\pgfqpoint{5.314620in}{3.014211in}}%
\pgfpathlineto{\pgfqpoint{5.321823in}{3.018665in}}%
\pgfpathlineto{\pgfqpoint{5.329018in}{3.023045in}}%
\pgfpathlineto{\pgfqpoint{5.336205in}{3.027353in}}%
\pgfpathlineto{\pgfqpoint{5.322159in}{3.016128in}}%
\pgfpathlineto{\pgfqpoint{5.308131in}{3.005062in}}%
\pgfpathlineto{\pgfqpoint{5.294122in}{2.994155in}}%
\pgfpathlineto{\pgfqpoint{5.280130in}{2.983405in}}%
\pgfpathlineto{\pgfqpoint{5.272926in}{2.978775in}}%
\pgfpathlineto{\pgfqpoint{5.265714in}{2.974080in}}%
\pgfpathlineto{\pgfqpoint{5.258495in}{2.969318in}}%
\pgfpathlineto{\pgfqpoint{5.251268in}{2.964485in}}%
\pgfpathclose%
\pgfusepath{fill}%
\end{pgfscope}%
\begin{pgfscope}%
\pgfpathrectangle{\pgfqpoint{1.254980in}{0.150000in}}{\pgfqpoint{5.490039in}{5.490039in}}%
\pgfusepath{clip}%
\pgfsetbuttcap%
\pgfsetroundjoin%
\definecolor{currentfill}{rgb}{0.271305,0.019942,0.347269}%
\pgfsetfillcolor{currentfill}%
\pgfsetfillopacity{0.700000}%
\pgfsetlinewidth{0.000000pt}%
\definecolor{currentstroke}{rgb}{0.000000,0.000000,0.000000}%
\pgfsetstrokecolor{currentstroke}%
\pgfsetdash{}{0pt}%
\pgfpathmoveto{\pgfqpoint{3.160650in}{1.696150in}}%
\pgfpathlineto{\pgfqpoint{3.173905in}{1.688922in}}%
\pgfpathlineto{\pgfqpoint{3.187160in}{1.681882in}}%
\pgfpathlineto{\pgfqpoint{3.200417in}{1.675030in}}%
\pgfpathlineto{\pgfqpoint{3.213674in}{1.668365in}}%
\pgfpathlineto{\pgfqpoint{3.221764in}{1.674812in}}%
\pgfpathlineto{\pgfqpoint{3.229845in}{1.681395in}}%
\pgfpathlineto{\pgfqpoint{3.237917in}{1.688109in}}%
\pgfpathlineto{\pgfqpoint{3.245981in}{1.694951in}}%
\pgfpathlineto{\pgfqpoint{3.232746in}{1.701176in}}%
\pgfpathlineto{\pgfqpoint{3.219512in}{1.707587in}}%
\pgfpathlineto{\pgfqpoint{3.206280in}{1.714186in}}%
\pgfpathlineto{\pgfqpoint{3.193050in}{1.720973in}}%
\pgfpathlineto{\pgfqpoint{3.184964in}{1.714561in}}%
\pgfpathlineto{\pgfqpoint{3.176869in}{1.708284in}}%
\pgfpathlineto{\pgfqpoint{3.168764in}{1.702145in}}%
\pgfpathlineto{\pgfqpoint{3.160650in}{1.696150in}}%
\pgfpathclose%
\pgfusepath{fill}%
\end{pgfscope}%
\begin{pgfscope}%
\pgfpathrectangle{\pgfqpoint{1.254980in}{0.150000in}}{\pgfqpoint{5.490039in}{5.490039in}}%
\pgfusepath{clip}%
\pgfsetbuttcap%
\pgfsetroundjoin%
\definecolor{currentfill}{rgb}{0.244972,0.287675,0.537260}%
\pgfsetfillcolor{currentfill}%
\pgfsetfillopacity{0.700000}%
\pgfsetlinewidth{0.000000pt}%
\definecolor{currentstroke}{rgb}{0.000000,0.000000,0.000000}%
\pgfsetstrokecolor{currentstroke}%
\pgfsetdash{}{0pt}%
\pgfpathmoveto{\pgfqpoint{4.313402in}{2.188941in}}%
\pgfpathlineto{\pgfqpoint{4.326903in}{2.194888in}}%
\pgfpathlineto{\pgfqpoint{4.340415in}{2.200997in}}%
\pgfpathlineto{\pgfqpoint{4.353940in}{2.207269in}}%
\pgfpathlineto{\pgfqpoint{4.367478in}{2.213702in}}%
\pgfpathlineto{\pgfqpoint{4.375128in}{2.224460in}}%
\pgfpathlineto{\pgfqpoint{4.382773in}{2.235132in}}%
\pgfpathlineto{\pgfqpoint{4.390413in}{2.245719in}}%
\pgfpathlineto{\pgfqpoint{4.398047in}{2.256221in}}%
\pgfpathlineto{\pgfqpoint{4.384514in}{2.249711in}}%
\pgfpathlineto{\pgfqpoint{4.370992in}{2.243363in}}%
\pgfpathlineto{\pgfqpoint{4.357484in}{2.237177in}}%
\pgfpathlineto{\pgfqpoint{4.343987in}{2.231153in}}%
\pgfpathlineto{\pgfqpoint{4.336349in}{2.220717in}}%
\pgfpathlineto{\pgfqpoint{4.328705in}{2.210203in}}%
\pgfpathlineto{\pgfqpoint{4.321057in}{2.199611in}}%
\pgfpathlineto{\pgfqpoint{4.313402in}{2.188941in}}%
\pgfpathclose%
\pgfusepath{fill}%
\end{pgfscope}%
\begin{pgfscope}%
\pgfpathrectangle{\pgfqpoint{1.254980in}{0.150000in}}{\pgfqpoint{5.490039in}{5.490039in}}%
\pgfusepath{clip}%
\pgfsetbuttcap%
\pgfsetroundjoin%
\definecolor{currentfill}{rgb}{0.159194,0.482237,0.558073}%
\pgfsetfillcolor{currentfill}%
\pgfsetfillopacity{0.700000}%
\pgfsetlinewidth{0.000000pt}%
\definecolor{currentstroke}{rgb}{0.000000,0.000000,0.000000}%
\pgfsetstrokecolor{currentstroke}%
\pgfsetdash{}{0pt}%
\pgfpathmoveto{\pgfqpoint{4.882293in}{2.675096in}}%
\pgfpathlineto{\pgfqpoint{4.896089in}{2.684705in}}%
\pgfpathlineto{\pgfqpoint{4.909902in}{2.694474in}}%
\pgfpathlineto{\pgfqpoint{4.923731in}{2.704403in}}%
\pgfpathlineto{\pgfqpoint{4.937576in}{2.714493in}}%
\pgfpathlineto{\pgfqpoint{4.944992in}{2.721832in}}%
\pgfpathlineto{\pgfqpoint{4.952401in}{2.729066in}}%
\pgfpathlineto{\pgfqpoint{4.959802in}{2.736199in}}%
\pgfpathlineto{\pgfqpoint{4.967196in}{2.743231in}}%
\pgfpathlineto{\pgfqpoint{4.953360in}{2.733302in}}%
\pgfpathlineto{\pgfqpoint{4.939541in}{2.723532in}}%
\pgfpathlineto{\pgfqpoint{4.925737in}{2.713922in}}%
\pgfpathlineto{\pgfqpoint{4.911950in}{2.704472in}}%
\pgfpathlineto{\pgfqpoint{4.904546in}{2.697269in}}%
\pgfpathlineto{\pgfqpoint{4.897135in}{2.689974in}}%
\pgfpathlineto{\pgfqpoint{4.889718in}{2.682583in}}%
\pgfpathlineto{\pgfqpoint{4.882293in}{2.675096in}}%
\pgfpathclose%
\pgfusepath{fill}%
\end{pgfscope}%
\begin{pgfscope}%
\pgfpathrectangle{\pgfqpoint{1.254980in}{0.150000in}}{\pgfqpoint{5.490039in}{5.490039in}}%
\pgfusepath{clip}%
\pgfsetbuttcap%
\pgfsetroundjoin%
\definecolor{currentfill}{rgb}{0.197636,0.391528,0.554969}%
\pgfsetfillcolor{currentfill}%
\pgfsetfillopacity{0.700000}%
\pgfsetlinewidth{0.000000pt}%
\definecolor{currentstroke}{rgb}{0.000000,0.000000,0.000000}%
\pgfsetstrokecolor{currentstroke}%
\pgfsetdash{}{0pt}%
\pgfpathmoveto{\pgfqpoint{4.597867in}{2.433910in}}%
\pgfpathlineto{\pgfqpoint{4.611509in}{2.441931in}}%
\pgfpathlineto{\pgfqpoint{4.625165in}{2.450112in}}%
\pgfpathlineto{\pgfqpoint{4.638836in}{2.458454in}}%
\pgfpathlineto{\pgfqpoint{4.652521in}{2.466958in}}%
\pgfpathlineto{\pgfqpoint{4.660067in}{2.476266in}}%
\pgfpathlineto{\pgfqpoint{4.667606in}{2.485471in}}%
\pgfpathlineto{\pgfqpoint{4.675139in}{2.494572in}}%
\pgfpathlineto{\pgfqpoint{4.682666in}{2.503570in}}%
\pgfpathlineto{\pgfqpoint{4.668986in}{2.495107in}}%
\pgfpathlineto{\pgfqpoint{4.655320in}{2.486805in}}%
\pgfpathlineto{\pgfqpoint{4.641669in}{2.478664in}}%
\pgfpathlineto{\pgfqpoint{4.628033in}{2.470683in}}%
\pgfpathlineto{\pgfqpoint{4.620501in}{2.461633in}}%
\pgfpathlineto{\pgfqpoint{4.612962in}{2.452489in}}%
\pgfpathlineto{\pgfqpoint{4.605417in}{2.443248in}}%
\pgfpathlineto{\pgfqpoint{4.597867in}{2.433910in}}%
\pgfpathclose%
\pgfusepath{fill}%
\end{pgfscope}%
\begin{pgfscope}%
\pgfpathrectangle{\pgfqpoint{1.254980in}{0.150000in}}{\pgfqpoint{5.490039in}{5.490039in}}%
\pgfusepath{clip}%
\pgfsetbuttcap%
\pgfsetroundjoin%
\definecolor{currentfill}{rgb}{0.278826,0.175490,0.483397}%
\pgfsetfillcolor{currentfill}%
\pgfsetfillopacity{0.700000}%
\pgfsetlinewidth{0.000000pt}%
\definecolor{currentstroke}{rgb}{0.000000,0.000000,0.000000}%
\pgfsetstrokecolor{currentstroke}%
\pgfsetdash{}{0pt}%
\pgfpathmoveto{\pgfqpoint{4.028949in}{1.956762in}}%
\pgfpathlineto{\pgfqpoint{4.042333in}{1.960157in}}%
\pgfpathlineto{\pgfqpoint{4.055727in}{1.963716in}}%
\pgfpathlineto{\pgfqpoint{4.069132in}{1.967439in}}%
\pgfpathlineto{\pgfqpoint{4.082547in}{1.971326in}}%
\pgfpathlineto{\pgfqpoint{4.090288in}{1.982718in}}%
\pgfpathlineto{\pgfqpoint{4.098024in}{1.994063in}}%
\pgfpathlineto{\pgfqpoint{4.105756in}{2.005360in}}%
\pgfpathlineto{\pgfqpoint{4.113483in}{2.016607in}}%
\pgfpathlineto{\pgfqpoint{4.100072in}{2.012529in}}%
\pgfpathlineto{\pgfqpoint{4.086673in}{2.008616in}}%
\pgfpathlineto{\pgfqpoint{4.073283in}{2.004867in}}%
\pgfpathlineto{\pgfqpoint{4.059904in}{2.001283in}}%
\pgfpathlineto{\pgfqpoint{4.052172in}{1.990215in}}%
\pgfpathlineto{\pgfqpoint{4.044436in}{1.979105in}}%
\pgfpathlineto{\pgfqpoint{4.036695in}{1.967953in}}%
\pgfpathlineto{\pgfqpoint{4.028949in}{1.956762in}}%
\pgfpathclose%
\pgfusepath{fill}%
\end{pgfscope}%
\begin{pgfscope}%
\pgfpathrectangle{\pgfqpoint{1.254980in}{0.150000in}}{\pgfqpoint{5.490039in}{5.490039in}}%
\pgfusepath{clip}%
\pgfsetbuttcap%
\pgfsetroundjoin%
\definecolor{currentfill}{rgb}{0.119483,0.614817,0.537692}%
\pgfsetfillcolor{currentfill}%
\pgfsetfillopacity{0.700000}%
\pgfsetlinewidth{0.000000pt}%
\definecolor{currentstroke}{rgb}{0.000000,0.000000,0.000000}%
\pgfsetstrokecolor{currentstroke}%
\pgfsetdash{}{0pt}%
\pgfpathmoveto{\pgfqpoint{5.336205in}{3.027353in}}%
\pgfpathlineto{\pgfqpoint{5.350269in}{3.038736in}}%
\pgfpathlineto{\pgfqpoint{5.364352in}{3.050278in}}%
\pgfpathlineto{\pgfqpoint{5.378453in}{3.061979in}}%
\pgfpathlineto{\pgfqpoint{5.392572in}{3.073838in}}%
\pgfpathlineto{\pgfqpoint{5.399734in}{3.077747in}}%
\pgfpathlineto{\pgfqpoint{5.406887in}{3.081585in}}%
\pgfpathlineto{\pgfqpoint{5.414032in}{3.085356in}}%
\pgfpathlineto{\pgfqpoint{5.421170in}{3.089064in}}%
\pgfpathlineto{\pgfqpoint{5.407068in}{3.077548in}}%
\pgfpathlineto{\pgfqpoint{5.392986in}{3.066190in}}%
\pgfpathlineto{\pgfqpoint{5.378922in}{3.054991in}}%
\pgfpathlineto{\pgfqpoint{5.364876in}{3.043949in}}%
\pgfpathlineto{\pgfqpoint{5.357720in}{3.039888in}}%
\pgfpathlineto{\pgfqpoint{5.350556in}{3.035771in}}%
\pgfpathlineto{\pgfqpoint{5.343384in}{3.031594in}}%
\pgfpathlineto{\pgfqpoint{5.336205in}{3.027353in}}%
\pgfpathclose%
\pgfusepath{fill}%
\end{pgfscope}%
\begin{pgfscope}%
\pgfpathrectangle{\pgfqpoint{1.254980in}{0.150000in}}{\pgfqpoint{5.490039in}{5.490039in}}%
\pgfusepath{clip}%
\pgfsetbuttcap%
\pgfsetroundjoin%
\definecolor{currentfill}{rgb}{0.271828,0.209303,0.504434}%
\pgfsetfillcolor{currentfill}%
\pgfsetfillopacity{0.700000}%
\pgfsetlinewidth{0.000000pt}%
\definecolor{currentstroke}{rgb}{0.000000,0.000000,0.000000}%
\pgfsetstrokecolor{currentstroke}%
\pgfsetdash{}{0pt}%
\pgfpathmoveto{\pgfqpoint{2.614934in}{2.084613in}}%
\pgfpathlineto{\pgfqpoint{2.628358in}{2.068800in}}%
\pgfpathlineto{\pgfqpoint{2.641775in}{2.053225in}}%
\pgfpathlineto{\pgfqpoint{2.655185in}{2.037885in}}%
\pgfpathlineto{\pgfqpoint{2.668589in}{2.022780in}}%
\pgfpathlineto{\pgfqpoint{2.677062in}{2.023333in}}%
\pgfpathlineto{\pgfqpoint{2.685519in}{2.024126in}}%
\pgfpathlineto{\pgfqpoint{2.693961in}{2.025154in}}%
\pgfpathlineto{\pgfqpoint{2.702388in}{2.026412in}}%
\pgfpathlineto{\pgfqpoint{2.689024in}{2.041003in}}%
\pgfpathlineto{\pgfqpoint{2.675655in}{2.055827in}}%
\pgfpathlineto{\pgfqpoint{2.662279in}{2.070886in}}%
\pgfpathlineto{\pgfqpoint{2.648897in}{2.086182in}}%
\pgfpathlineto{\pgfqpoint{2.640430in}{2.085428in}}%
\pgfpathlineto{\pgfqpoint{2.631947in}{2.084912in}}%
\pgfpathlineto{\pgfqpoint{2.623449in}{2.084638in}}%
\pgfpathlineto{\pgfqpoint{2.614934in}{2.084613in}}%
\pgfpathclose%
\pgfusepath{fill}%
\end{pgfscope}%
\begin{pgfscope}%
\pgfpathrectangle{\pgfqpoint{1.254980in}{0.150000in}}{\pgfqpoint{5.490039in}{5.490039in}}%
\pgfusepath{clip}%
\pgfsetbuttcap%
\pgfsetroundjoin%
\definecolor{currentfill}{rgb}{0.269944,0.014625,0.341379}%
\pgfsetfillcolor{currentfill}%
\pgfsetfillopacity{0.700000}%
\pgfsetlinewidth{0.000000pt}%
\definecolor{currentstroke}{rgb}{0.000000,0.000000,0.000000}%
\pgfsetstrokecolor{currentstroke}%
\pgfsetdash{}{0pt}%
\pgfpathmoveto{\pgfqpoint{3.436930in}{1.668032in}}%
\pgfpathlineto{\pgfqpoint{3.450184in}{1.664575in}}%
\pgfpathlineto{\pgfqpoint{3.463443in}{1.661293in}}%
\pgfpathlineto{\pgfqpoint{3.476705in}{1.658187in}}%
\pgfpathlineto{\pgfqpoint{3.489972in}{1.655256in}}%
\pgfpathlineto{\pgfqpoint{3.497924in}{1.664252in}}%
\pgfpathlineto{\pgfqpoint{3.505869in}{1.673324in}}%
\pgfpathlineto{\pgfqpoint{3.513807in}{1.682468in}}%
\pgfpathlineto{\pgfqpoint{3.521739in}{1.691681in}}%
\pgfpathlineto{\pgfqpoint{3.508487in}{1.694229in}}%
\pgfpathlineto{\pgfqpoint{3.495240in}{1.696951in}}%
\pgfpathlineto{\pgfqpoint{3.481997in}{1.699849in}}%
\pgfpathlineto{\pgfqpoint{3.468759in}{1.702924in}}%
\pgfpathlineto{\pgfqpoint{3.460812in}{1.694084in}}%
\pgfpathlineto{\pgfqpoint{3.452858in}{1.685319in}}%
\pgfpathlineto{\pgfqpoint{3.444897in}{1.676634in}}%
\pgfpathlineto{\pgfqpoint{3.436930in}{1.668032in}}%
\pgfpathclose%
\pgfusepath{fill}%
\end{pgfscope}%
\begin{pgfscope}%
\pgfpathrectangle{\pgfqpoint{1.254980in}{0.150000in}}{\pgfqpoint{5.490039in}{5.490039in}}%
\pgfusepath{clip}%
\pgfsetbuttcap%
\pgfsetroundjoin%
\definecolor{currentfill}{rgb}{0.262138,0.242286,0.520837}%
\pgfsetfillcolor{currentfill}%
\pgfsetfillopacity{0.700000}%
\pgfsetlinewidth{0.000000pt}%
\definecolor{currentstroke}{rgb}{0.000000,0.000000,0.000000}%
\pgfsetstrokecolor{currentstroke}%
\pgfsetdash{}{0pt}%
\pgfpathmoveto{\pgfqpoint{2.561164in}{2.150284in}}%
\pgfpathlineto{\pgfqpoint{2.574618in}{2.133500in}}%
\pgfpathlineto{\pgfqpoint{2.588064in}{2.116962in}}%
\pgfpathlineto{\pgfqpoint{2.601503in}{2.100667in}}%
\pgfpathlineto{\pgfqpoint{2.614934in}{2.084613in}}%
\pgfpathlineto{\pgfqpoint{2.623449in}{2.084638in}}%
\pgfpathlineto{\pgfqpoint{2.631947in}{2.084912in}}%
\pgfpathlineto{\pgfqpoint{2.640430in}{2.085428in}}%
\pgfpathlineto{\pgfqpoint{2.648897in}{2.086182in}}%
\pgfpathlineto{\pgfqpoint{2.635508in}{2.101718in}}%
\pgfpathlineto{\pgfqpoint{2.622113in}{2.117494in}}%
\pgfpathlineto{\pgfqpoint{2.608710in}{2.133513in}}%
\pgfpathlineto{\pgfqpoint{2.595299in}{2.149777in}}%
\pgfpathlineto{\pgfqpoint{2.586790in}{2.149530in}}%
\pgfpathlineto{\pgfqpoint{2.578265in}{2.149528in}}%
\pgfpathlineto{\pgfqpoint{2.569723in}{2.149778in}}%
\pgfpathlineto{\pgfqpoint{2.561164in}{2.150284in}}%
\pgfpathclose%
\pgfusepath{fill}%
\end{pgfscope}%
\begin{pgfscope}%
\pgfpathrectangle{\pgfqpoint{1.254980in}{0.150000in}}{\pgfqpoint{5.490039in}{5.490039in}}%
\pgfusepath{clip}%
\pgfsetbuttcap%
\pgfsetroundjoin%
\definecolor{currentfill}{rgb}{0.277134,0.185228,0.489898}%
\pgfsetfillcolor{currentfill}%
\pgfsetfillopacity{0.700000}%
\pgfsetlinewidth{0.000000pt}%
\definecolor{currentstroke}{rgb}{0.000000,0.000000,0.000000}%
\pgfsetstrokecolor{currentstroke}%
\pgfsetdash{}{0pt}%
\pgfpathmoveto{\pgfqpoint{2.668589in}{2.022780in}}%
\pgfpathlineto{\pgfqpoint{2.681987in}{2.007907in}}%
\pgfpathlineto{\pgfqpoint{2.695378in}{1.993266in}}%
\pgfpathlineto{\pgfqpoint{2.708764in}{1.978853in}}%
\pgfpathlineto{\pgfqpoint{2.722145in}{1.964667in}}%
\pgfpathlineto{\pgfqpoint{2.730577in}{1.965745in}}%
\pgfpathlineto{\pgfqpoint{2.738994in}{1.967054in}}%
\pgfpathlineto{\pgfqpoint{2.747397in}{1.968592in}}%
\pgfpathlineto{\pgfqpoint{2.755786in}{1.970352in}}%
\pgfpathlineto{\pgfqpoint{2.742444in}{1.984025in}}%
\pgfpathlineto{\pgfqpoint{2.729097in}{1.997925in}}%
\pgfpathlineto{\pgfqpoint{2.715745in}{2.012054in}}%
\pgfpathlineto{\pgfqpoint{2.702388in}{2.026412in}}%
\pgfpathlineto{\pgfqpoint{2.693961in}{2.025154in}}%
\pgfpathlineto{\pgfqpoint{2.685519in}{2.024126in}}%
\pgfpathlineto{\pgfqpoint{2.677062in}{2.023333in}}%
\pgfpathlineto{\pgfqpoint{2.668589in}{2.022780in}}%
\pgfpathclose%
\pgfusepath{fill}%
\end{pgfscope}%
\begin{pgfscope}%
\pgfpathrectangle{\pgfqpoint{1.254980in}{0.150000in}}{\pgfqpoint{5.490039in}{5.490039in}}%
\pgfusepath{clip}%
\pgfsetbuttcap%
\pgfsetroundjoin%
\definecolor{currentfill}{rgb}{0.165117,0.467423,0.558141}%
\pgfsetfillcolor{currentfill}%
\pgfsetfillopacity{0.700000}%
\pgfsetlinewidth{0.000000pt}%
\definecolor{currentstroke}{rgb}{0.000000,0.000000,0.000000}%
\pgfsetstrokecolor{currentstroke}%
\pgfsetdash{}{0pt}%
\pgfpathmoveto{\pgfqpoint{2.215698in}{2.713653in}}%
\pgfpathlineto{\pgfqpoint{2.229428in}{2.689584in}}%
\pgfpathlineto{\pgfqpoint{2.243144in}{2.665831in}}%
\pgfpathlineto{\pgfqpoint{2.256843in}{2.642391in}}%
\pgfpathlineto{\pgfqpoint{2.270528in}{2.619261in}}%
\pgfpathlineto{\pgfqpoint{2.279307in}{2.616671in}}%
\pgfpathlineto{\pgfqpoint{2.288065in}{2.614366in}}%
\pgfpathlineto{\pgfqpoint{2.296804in}{2.612340in}}%
\pgfpathlineto{\pgfqpoint{2.305523in}{2.610589in}}%
\pgfpathlineto{\pgfqpoint{2.291891in}{2.633204in}}%
\pgfpathlineto{\pgfqpoint{2.278244in}{2.656128in}}%
\pgfpathlineto{\pgfqpoint{2.264582in}{2.679363in}}%
\pgfpathlineto{\pgfqpoint{2.250905in}{2.702913in}}%
\pgfpathlineto{\pgfqpoint{2.242133in}{2.705169in}}%
\pgfpathlineto{\pgfqpoint{2.233342in}{2.707708in}}%
\pgfpathlineto{\pgfqpoint{2.224530in}{2.710534in}}%
\pgfpathlineto{\pgfqpoint{2.215698in}{2.713653in}}%
\pgfpathclose%
\pgfusepath{fill}%
\end{pgfscope}%
\begin{pgfscope}%
\pgfpathrectangle{\pgfqpoint{1.254980in}{0.150000in}}{\pgfqpoint{5.490039in}{5.490039in}}%
\pgfusepath{clip}%
\pgfsetbuttcap%
\pgfsetroundjoin%
\definecolor{currentfill}{rgb}{0.276022,0.044167,0.370164}%
\pgfsetfillcolor{currentfill}%
\pgfsetfillopacity{0.700000}%
\pgfsetlinewidth{0.000000pt}%
\definecolor{currentstroke}{rgb}{0.000000,0.000000,0.000000}%
\pgfsetstrokecolor{currentstroke}%
\pgfsetdash{}{0pt}%
\pgfpathmoveto{\pgfqpoint{3.021861in}{1.742197in}}%
\pgfpathlineto{\pgfqpoint{3.035142in}{1.732951in}}%
\pgfpathlineto{\pgfqpoint{3.048422in}{1.723903in}}%
\pgfpathlineto{\pgfqpoint{3.061702in}{1.715050in}}%
\pgfpathlineto{\pgfqpoint{3.074981in}{1.706392in}}%
\pgfpathlineto{\pgfqpoint{3.083161in}{1.711289in}}%
\pgfpathlineto{\pgfqpoint{3.091331in}{1.716354in}}%
\pgfpathlineto{\pgfqpoint{3.099490in}{1.721583in}}%
\pgfpathlineto{\pgfqpoint{3.107639in}{1.726971in}}%
\pgfpathlineto{\pgfqpoint{3.094387in}{1.735158in}}%
\pgfpathlineto{\pgfqpoint{3.081135in}{1.743539in}}%
\pgfpathlineto{\pgfqpoint{3.067883in}{1.752116in}}%
\pgfpathlineto{\pgfqpoint{3.054630in}{1.760890in}}%
\pgfpathlineto{\pgfqpoint{3.046454in}{1.755962in}}%
\pgfpathlineto{\pgfqpoint{3.038267in}{1.751201in}}%
\pgfpathlineto{\pgfqpoint{3.030069in}{1.746611in}}%
\pgfpathlineto{\pgfqpoint{3.021861in}{1.742197in}}%
\pgfpathclose%
\pgfusepath{fill}%
\end{pgfscope}%
\begin{pgfscope}%
\pgfpathrectangle{\pgfqpoint{1.254980in}{0.150000in}}{\pgfqpoint{5.490039in}{5.490039in}}%
\pgfusepath{clip}%
\pgfsetbuttcap%
\pgfsetroundjoin%
\definecolor{currentfill}{rgb}{0.250425,0.274290,0.533103}%
\pgfsetfillcolor{currentfill}%
\pgfsetfillopacity{0.700000}%
\pgfsetlinewidth{0.000000pt}%
\definecolor{currentstroke}{rgb}{0.000000,0.000000,0.000000}%
\pgfsetstrokecolor{currentstroke}%
\pgfsetdash{}{0pt}%
\pgfpathmoveto{\pgfqpoint{2.507264in}{2.219917in}}%
\pgfpathlineto{\pgfqpoint{2.520752in}{2.202130in}}%
\pgfpathlineto{\pgfqpoint{2.534231in}{2.184597in}}%
\pgfpathlineto{\pgfqpoint{2.547702in}{2.167316in}}%
\pgfpathlineto{\pgfqpoint{2.561164in}{2.150284in}}%
\pgfpathlineto{\pgfqpoint{2.569723in}{2.149778in}}%
\pgfpathlineto{\pgfqpoint{2.578265in}{2.149528in}}%
\pgfpathlineto{\pgfqpoint{2.586790in}{2.149530in}}%
\pgfpathlineto{\pgfqpoint{2.595299in}{2.149777in}}%
\pgfpathlineto{\pgfqpoint{2.581881in}{2.166287in}}%
\pgfpathlineto{\pgfqpoint{2.568455in}{2.183046in}}%
\pgfpathlineto{\pgfqpoint{2.555021in}{2.200056in}}%
\pgfpathlineto{\pgfqpoint{2.541578in}{2.217319in}}%
\pgfpathlineto{\pgfqpoint{2.533025in}{2.217583in}}%
\pgfpathlineto{\pgfqpoint{2.524455in}{2.218100in}}%
\pgfpathlineto{\pgfqpoint{2.515869in}{2.218876in}}%
\pgfpathlineto{\pgfqpoint{2.507264in}{2.219917in}}%
\pgfpathclose%
\pgfusepath{fill}%
\end{pgfscope}%
\begin{pgfscope}%
\pgfpathrectangle{\pgfqpoint{1.254980in}{0.150000in}}{\pgfqpoint{5.490039in}{5.490039in}}%
\pgfusepath{clip}%
\pgfsetbuttcap%
\pgfsetroundjoin%
\definecolor{currentfill}{rgb}{0.231674,0.318106,0.544834}%
\pgfsetfillcolor{currentfill}%
\pgfsetfillopacity{0.700000}%
\pgfsetlinewidth{0.000000pt}%
\definecolor{currentstroke}{rgb}{0.000000,0.000000,0.000000}%
\pgfsetstrokecolor{currentstroke}%
\pgfsetdash{}{0pt}%
\pgfpathmoveto{\pgfqpoint{4.398047in}{2.256221in}}%
\pgfpathlineto{\pgfqpoint{4.411594in}{2.262893in}}%
\pgfpathlineto{\pgfqpoint{4.425154in}{2.269727in}}%
\pgfpathlineto{\pgfqpoint{4.438726in}{2.276723in}}%
\pgfpathlineto{\pgfqpoint{4.452312in}{2.283881in}}%
\pgfpathlineto{\pgfqpoint{4.459937in}{2.294355in}}%
\pgfpathlineto{\pgfqpoint{4.467557in}{2.304737in}}%
\pgfpathlineto{\pgfqpoint{4.475171in}{2.315025in}}%
\pgfpathlineto{\pgfqpoint{4.482779in}{2.325220in}}%
\pgfpathlineto{\pgfqpoint{4.469197in}{2.318014in}}%
\pgfpathlineto{\pgfqpoint{4.455628in}{2.310971in}}%
\pgfpathlineto{\pgfqpoint{4.442073in}{2.304089in}}%
\pgfpathlineto{\pgfqpoint{4.428530in}{2.297369in}}%
\pgfpathlineto{\pgfqpoint{4.420918in}{2.287211in}}%
\pgfpathlineto{\pgfqpoint{4.413300in}{2.276967in}}%
\pgfpathlineto{\pgfqpoint{4.405676in}{2.266637in}}%
\pgfpathlineto{\pgfqpoint{4.398047in}{2.256221in}}%
\pgfpathclose%
\pgfusepath{fill}%
\end{pgfscope}%
\begin{pgfscope}%
\pgfpathrectangle{\pgfqpoint{1.254980in}{0.150000in}}{\pgfqpoint{5.490039in}{5.490039in}}%
\pgfusepath{clip}%
\pgfsetbuttcap%
\pgfsetroundjoin%
\definecolor{currentfill}{rgb}{0.149039,0.508051,0.557250}%
\pgfsetfillcolor{currentfill}%
\pgfsetfillopacity{0.700000}%
\pgfsetlinewidth{0.000000pt}%
\definecolor{currentstroke}{rgb}{0.000000,0.000000,0.000000}%
\pgfsetstrokecolor{currentstroke}%
\pgfsetdash{}{0pt}%
\pgfpathmoveto{\pgfqpoint{4.967196in}{2.743231in}}%
\pgfpathlineto{\pgfqpoint{4.981049in}{2.753321in}}%
\pgfpathlineto{\pgfqpoint{4.994917in}{2.763570in}}%
\pgfpathlineto{\pgfqpoint{5.008803in}{2.773979in}}%
\pgfpathlineto{\pgfqpoint{5.022705in}{2.784548in}}%
\pgfpathlineto{\pgfqpoint{5.030082in}{2.791304in}}%
\pgfpathlineto{\pgfqpoint{5.037451in}{2.797956in}}%
\pgfpathlineto{\pgfqpoint{5.044813in}{2.804508in}}%
\pgfpathlineto{\pgfqpoint{5.052167in}{2.810962in}}%
\pgfpathlineto{\pgfqpoint{5.038275in}{2.800583in}}%
\pgfpathlineto{\pgfqpoint{5.024400in}{2.790364in}}%
\pgfpathlineto{\pgfqpoint{5.010541in}{2.780304in}}%
\pgfpathlineto{\pgfqpoint{4.996699in}{2.770404in}}%
\pgfpathlineto{\pgfqpoint{4.989334in}{2.763750in}}%
\pgfpathlineto{\pgfqpoint{4.981962in}{2.757004in}}%
\pgfpathlineto{\pgfqpoint{4.974583in}{2.750166in}}%
\pgfpathlineto{\pgfqpoint{4.967196in}{2.743231in}}%
\pgfpathclose%
\pgfusepath{fill}%
\end{pgfscope}%
\begin{pgfscope}%
\pgfpathrectangle{\pgfqpoint{1.254980in}{0.150000in}}{\pgfqpoint{5.490039in}{5.490039in}}%
\pgfusepath{clip}%
\pgfsetbuttcap%
\pgfsetroundjoin%
\definecolor{currentfill}{rgb}{0.271828,0.209303,0.504434}%
\pgfsetfillcolor{currentfill}%
\pgfsetfillopacity{0.700000}%
\pgfsetlinewidth{0.000000pt}%
\definecolor{currentstroke}{rgb}{0.000000,0.000000,0.000000}%
\pgfsetstrokecolor{currentstroke}%
\pgfsetdash{}{0pt}%
\pgfpathmoveto{\pgfqpoint{4.113483in}{2.016607in}}%
\pgfpathlineto{\pgfqpoint{4.126903in}{2.020848in}}%
\pgfpathlineto{\pgfqpoint{4.140335in}{2.025253in}}%
\pgfpathlineto{\pgfqpoint{4.153777in}{2.029821in}}%
\pgfpathlineto{\pgfqpoint{4.167230in}{2.034553in}}%
\pgfpathlineto{\pgfqpoint{4.174948in}{2.045921in}}%
\pgfpathlineto{\pgfqpoint{4.182661in}{2.057229in}}%
\pgfpathlineto{\pgfqpoint{4.190370in}{2.068477in}}%
\pgfpathlineto{\pgfqpoint{4.198073in}{2.079662in}}%
\pgfpathlineto{\pgfqpoint{4.184624in}{2.074768in}}%
\pgfpathlineto{\pgfqpoint{4.171185in}{2.070038in}}%
\pgfpathlineto{\pgfqpoint{4.157758in}{2.065471in}}%
\pgfpathlineto{\pgfqpoint{4.144342in}{2.061068in}}%
\pgfpathlineto{\pgfqpoint{4.136634in}{2.050034in}}%
\pgfpathlineto{\pgfqpoint{4.128922in}{2.038945in}}%
\pgfpathlineto{\pgfqpoint{4.121205in}{2.027802in}}%
\pgfpathlineto{\pgfqpoint{4.113483in}{2.016607in}}%
\pgfpathclose%
\pgfusepath{fill}%
\end{pgfscope}%
\begin{pgfscope}%
\pgfpathrectangle{\pgfqpoint{1.254980in}{0.150000in}}{\pgfqpoint{5.490039in}{5.490039in}}%
\pgfusepath{clip}%
\pgfsetbuttcap%
\pgfsetroundjoin%
\definecolor{currentfill}{rgb}{0.281412,0.155834,0.469201}%
\pgfsetfillcolor{currentfill}%
\pgfsetfillopacity{0.700000}%
\pgfsetlinewidth{0.000000pt}%
\definecolor{currentstroke}{rgb}{0.000000,0.000000,0.000000}%
\pgfsetstrokecolor{currentstroke}%
\pgfsetdash{}{0pt}%
\pgfpathmoveto{\pgfqpoint{2.722145in}{1.964667in}}%
\pgfpathlineto{\pgfqpoint{2.735520in}{1.950708in}}%
\pgfpathlineto{\pgfqpoint{2.748890in}{1.936973in}}%
\pgfpathlineto{\pgfqpoint{2.762255in}{1.923460in}}%
\pgfpathlineto{\pgfqpoint{2.775615in}{1.910168in}}%
\pgfpathlineto{\pgfqpoint{2.784008in}{1.911767in}}%
\pgfpathlineto{\pgfqpoint{2.792388in}{1.913591in}}%
\pgfpathlineto{\pgfqpoint{2.800753in}{1.915635in}}%
\pgfpathlineto{\pgfqpoint{2.809105in}{1.917893in}}%
\pgfpathlineto{\pgfqpoint{2.795782in}{1.930676in}}%
\pgfpathlineto{\pgfqpoint{2.782454in}{1.943679in}}%
\pgfpathlineto{\pgfqpoint{2.769122in}{1.956903in}}%
\pgfpathlineto{\pgfqpoint{2.755786in}{1.970352in}}%
\pgfpathlineto{\pgfqpoint{2.747397in}{1.968592in}}%
\pgfpathlineto{\pgfqpoint{2.738994in}{1.967054in}}%
\pgfpathlineto{\pgfqpoint{2.730577in}{1.965745in}}%
\pgfpathlineto{\pgfqpoint{2.722145in}{1.964667in}}%
\pgfpathclose%
\pgfusepath{fill}%
\end{pgfscope}%
\begin{pgfscope}%
\pgfpathrectangle{\pgfqpoint{1.254980in}{0.150000in}}{\pgfqpoint{5.490039in}{5.490039in}}%
\pgfusepath{clip}%
\pgfsetbuttcap%
\pgfsetroundjoin%
\definecolor{currentfill}{rgb}{0.183898,0.422383,0.556944}%
\pgfsetfillcolor{currentfill}%
\pgfsetfillopacity{0.700000}%
\pgfsetlinewidth{0.000000pt}%
\definecolor{currentstroke}{rgb}{0.000000,0.000000,0.000000}%
\pgfsetstrokecolor{currentstroke}%
\pgfsetdash{}{0pt}%
\pgfpathmoveto{\pgfqpoint{4.682666in}{2.503570in}}%
\pgfpathlineto{\pgfqpoint{4.696360in}{2.512194in}}%
\pgfpathlineto{\pgfqpoint{4.710070in}{2.520979in}}%
\pgfpathlineto{\pgfqpoint{4.723794in}{2.529925in}}%
\pgfpathlineto{\pgfqpoint{4.737534in}{2.539031in}}%
\pgfpathlineto{\pgfqpoint{4.745048in}{2.547870in}}%
\pgfpathlineto{\pgfqpoint{4.752556in}{2.556601in}}%
\pgfpathlineto{\pgfqpoint{4.760057in}{2.565225in}}%
\pgfpathlineto{\pgfqpoint{4.767552in}{2.573743in}}%
\pgfpathlineto{\pgfqpoint{4.753818in}{2.564707in}}%
\pgfpathlineto{\pgfqpoint{4.740100in}{2.555831in}}%
\pgfpathlineto{\pgfqpoint{4.726396in}{2.547116in}}%
\pgfpathlineto{\pgfqpoint{4.712708in}{2.538562in}}%
\pgfpathlineto{\pgfqpoint{4.705207in}{2.529962in}}%
\pgfpathlineto{\pgfqpoint{4.697700in}{2.521265in}}%
\pgfpathlineto{\pgfqpoint{4.690186in}{2.512468in}}%
\pgfpathlineto{\pgfqpoint{4.682666in}{2.503570in}}%
\pgfpathclose%
\pgfusepath{fill}%
\end{pgfscope}%
\begin{pgfscope}%
\pgfpathrectangle{\pgfqpoint{1.254980in}{0.150000in}}{\pgfqpoint{5.490039in}{5.490039in}}%
\pgfusepath{clip}%
\pgfsetbuttcap%
\pgfsetroundjoin%
\definecolor{currentfill}{rgb}{0.237441,0.305202,0.541921}%
\pgfsetfillcolor{currentfill}%
\pgfsetfillopacity{0.700000}%
\pgfsetlinewidth{0.000000pt}%
\definecolor{currentstroke}{rgb}{0.000000,0.000000,0.000000}%
\pgfsetstrokecolor{currentstroke}%
\pgfsetdash{}{0pt}%
\pgfpathmoveto{\pgfqpoint{2.453218in}{2.293645in}}%
\pgfpathlineto{\pgfqpoint{2.466744in}{2.274821in}}%
\pgfpathlineto{\pgfqpoint{2.480260in}{2.256260in}}%
\pgfpathlineto{\pgfqpoint{2.493767in}{2.237959in}}%
\pgfpathlineto{\pgfqpoint{2.507264in}{2.219917in}}%
\pgfpathlineto{\pgfqpoint{2.515869in}{2.218876in}}%
\pgfpathlineto{\pgfqpoint{2.524455in}{2.218100in}}%
\pgfpathlineto{\pgfqpoint{2.533025in}{2.217583in}}%
\pgfpathlineto{\pgfqpoint{2.541578in}{2.217319in}}%
\pgfpathlineto{\pgfqpoint{2.528126in}{2.234837in}}%
\pgfpathlineto{\pgfqpoint{2.514666in}{2.252612in}}%
\pgfpathlineto{\pgfqpoint{2.501196in}{2.270646in}}%
\pgfpathlineto{\pgfqpoint{2.487717in}{2.288942in}}%
\pgfpathlineto{\pgfqpoint{2.479119in}{2.289720in}}%
\pgfpathlineto{\pgfqpoint{2.470503in}{2.290760in}}%
\pgfpathlineto{\pgfqpoint{2.461870in}{2.292066in}}%
\pgfpathlineto{\pgfqpoint{2.453218in}{2.293645in}}%
\pgfpathclose%
\pgfusepath{fill}%
\end{pgfscope}%
\begin{pgfscope}%
\pgfpathrectangle{\pgfqpoint{1.254980in}{0.150000in}}{\pgfqpoint{5.490039in}{5.490039in}}%
\pgfusepath{clip}%
\pgfsetbuttcap%
\pgfsetroundjoin%
\definecolor{currentfill}{rgb}{0.123444,0.636809,0.528763}%
\pgfsetfillcolor{currentfill}%
\pgfsetfillopacity{0.700000}%
\pgfsetlinewidth{0.000000pt}%
\definecolor{currentstroke}{rgb}{0.000000,0.000000,0.000000}%
\pgfsetstrokecolor{currentstroke}%
\pgfsetdash{}{0pt}%
\pgfpathmoveto{\pgfqpoint{5.421170in}{3.089064in}}%
\pgfpathlineto{\pgfqpoint{5.435289in}{3.100739in}}%
\pgfpathlineto{\pgfqpoint{5.449428in}{3.112572in}}%
\pgfpathlineto{\pgfqpoint{5.463585in}{3.124563in}}%
\pgfpathlineto{\pgfqpoint{5.477762in}{3.136714in}}%
\pgfpathlineto{\pgfqpoint{5.484871in}{3.139999in}}%
\pgfpathlineto{\pgfqpoint{5.491973in}{3.143223in}}%
\pgfpathlineto{\pgfqpoint{5.499066in}{3.146389in}}%
\pgfpathlineto{\pgfqpoint{5.506152in}{3.149501in}}%
\pgfpathlineto{\pgfqpoint{5.491996in}{3.137725in}}%
\pgfpathlineto{\pgfqpoint{5.477859in}{3.126107in}}%
\pgfpathlineto{\pgfqpoint{5.463741in}{3.114648in}}%
\pgfpathlineto{\pgfqpoint{5.449641in}{3.103346in}}%
\pgfpathlineto{\pgfqpoint{5.442535in}{3.099850in}}%
\pgfpathlineto{\pgfqpoint{5.435421in}{3.096307in}}%
\pgfpathlineto{\pgfqpoint{5.428299in}{3.092713in}}%
\pgfpathlineto{\pgfqpoint{5.421170in}{3.089064in}}%
\pgfpathclose%
\pgfusepath{fill}%
\end{pgfscope}%
\begin{pgfscope}%
\pgfpathrectangle{\pgfqpoint{1.254980in}{0.150000in}}{\pgfqpoint{5.490039in}{5.490039in}}%
\pgfusepath{clip}%
\pgfsetbuttcap%
\pgfsetroundjoin%
\definecolor{currentfill}{rgb}{0.268510,0.009605,0.335427}%
\pgfsetfillcolor{currentfill}%
\pgfsetfillopacity{0.700000}%
\pgfsetlinewidth{0.000000pt}%
\definecolor{currentstroke}{rgb}{0.000000,0.000000,0.000000}%
\pgfsetstrokecolor{currentstroke}%
\pgfsetdash{}{0pt}%
\pgfpathmoveto{\pgfqpoint{3.213674in}{1.668365in}}%
\pgfpathlineto{\pgfqpoint{3.226934in}{1.661886in}}%
\pgfpathlineto{\pgfqpoint{3.240195in}{1.655592in}}%
\pgfpathlineto{\pgfqpoint{3.253458in}{1.649483in}}%
\pgfpathlineto{\pgfqpoint{3.266722in}{1.643557in}}%
\pgfpathlineto{\pgfqpoint{3.274790in}{1.650455in}}%
\pgfpathlineto{\pgfqpoint{3.282848in}{1.657481in}}%
\pgfpathlineto{\pgfqpoint{3.290899in}{1.664633in}}%
\pgfpathlineto{\pgfqpoint{3.298941in}{1.671904in}}%
\pgfpathlineto{\pgfqpoint{3.285698in}{1.677390in}}%
\pgfpathlineto{\pgfqpoint{3.272457in}{1.683059in}}%
\pgfpathlineto{\pgfqpoint{3.259218in}{1.688913in}}%
\pgfpathlineto{\pgfqpoint{3.245981in}{1.694951in}}%
\pgfpathlineto{\pgfqpoint{3.237917in}{1.688109in}}%
\pgfpathlineto{\pgfqpoint{3.229845in}{1.681395in}}%
\pgfpathlineto{\pgfqpoint{3.221764in}{1.674812in}}%
\pgfpathlineto{\pgfqpoint{3.213674in}{1.668365in}}%
\pgfpathclose%
\pgfusepath{fill}%
\end{pgfscope}%
\begin{pgfscope}%
\pgfpathrectangle{\pgfqpoint{1.254980in}{0.150000in}}{\pgfqpoint{5.490039in}{5.490039in}}%
\pgfusepath{clip}%
\pgfsetbuttcap%
\pgfsetroundjoin%
\definecolor{currentfill}{rgb}{0.283072,0.130895,0.449241}%
\pgfsetfillcolor{currentfill}%
\pgfsetfillopacity{0.700000}%
\pgfsetlinewidth{0.000000pt}%
\definecolor{currentstroke}{rgb}{0.000000,0.000000,0.000000}%
\pgfsetstrokecolor{currentstroke}%
\pgfsetdash{}{0pt}%
\pgfpathmoveto{\pgfqpoint{2.775615in}{1.910168in}}%
\pgfpathlineto{\pgfqpoint{2.788971in}{1.897096in}}%
\pgfpathlineto{\pgfqpoint{2.802323in}{1.884241in}}%
\pgfpathlineto{\pgfqpoint{2.815670in}{1.871603in}}%
\pgfpathlineto{\pgfqpoint{2.829014in}{1.859181in}}%
\pgfpathlineto{\pgfqpoint{2.837370in}{1.861299in}}%
\pgfpathlineto{\pgfqpoint{2.845713in}{1.863635in}}%
\pgfpathlineto{\pgfqpoint{2.854043in}{1.866183in}}%
\pgfpathlineto{\pgfqpoint{2.862359in}{1.868938in}}%
\pgfpathlineto{\pgfqpoint{2.849051in}{1.880854in}}%
\pgfpathlineto{\pgfqpoint{2.835739in}{1.892984in}}%
\pgfpathlineto{\pgfqpoint{2.822424in}{1.905330in}}%
\pgfpathlineto{\pgfqpoint{2.809105in}{1.917893in}}%
\pgfpathlineto{\pgfqpoint{2.800753in}{1.915635in}}%
\pgfpathlineto{\pgfqpoint{2.792388in}{1.913591in}}%
\pgfpathlineto{\pgfqpoint{2.784008in}{1.911767in}}%
\pgfpathlineto{\pgfqpoint{2.775615in}{1.910168in}}%
\pgfpathclose%
\pgfusepath{fill}%
\end{pgfscope}%
\begin{pgfscope}%
\pgfpathrectangle{\pgfqpoint{1.254980in}{0.150000in}}{\pgfqpoint{5.490039in}{5.490039in}}%
\pgfusepath{clip}%
\pgfsetbuttcap%
\pgfsetroundjoin%
\definecolor{currentfill}{rgb}{0.262138,0.242286,0.520837}%
\pgfsetfillcolor{currentfill}%
\pgfsetfillopacity{0.700000}%
\pgfsetlinewidth{0.000000pt}%
\definecolor{currentstroke}{rgb}{0.000000,0.000000,0.000000}%
\pgfsetstrokecolor{currentstroke}%
\pgfsetdash{}{0pt}%
\pgfpathmoveto{\pgfqpoint{4.198073in}{2.079662in}}%
\pgfpathlineto{\pgfqpoint{4.211534in}{2.084719in}}%
\pgfpathlineto{\pgfqpoint{4.225006in}{2.089939in}}%
\pgfpathlineto{\pgfqpoint{4.238490in}{2.095321in}}%
\pgfpathlineto{\pgfqpoint{4.251985in}{2.100866in}}%
\pgfpathlineto{\pgfqpoint{4.259680in}{2.112133in}}%
\pgfpathlineto{\pgfqpoint{4.267370in}{2.123328in}}%
\pgfpathlineto{\pgfqpoint{4.275055in}{2.134450in}}%
\pgfpathlineto{\pgfqpoint{4.282734in}{2.145498in}}%
\pgfpathlineto{\pgfqpoint{4.269242in}{2.139819in}}%
\pgfpathlineto{\pgfqpoint{4.255762in}{2.134303in}}%
\pgfpathlineto{\pgfqpoint{4.242294in}{2.128949in}}%
\pgfpathlineto{\pgfqpoint{4.228837in}{2.123759in}}%
\pgfpathlineto{\pgfqpoint{4.221154in}{2.112833in}}%
\pgfpathlineto{\pgfqpoint{4.213465in}{2.101841in}}%
\pgfpathlineto{\pgfqpoint{4.205772in}{2.090784in}}%
\pgfpathlineto{\pgfqpoint{4.198073in}{2.079662in}}%
\pgfpathclose%
\pgfusepath{fill}%
\end{pgfscope}%
\begin{pgfscope}%
\pgfpathrectangle{\pgfqpoint{1.254980in}{0.150000in}}{\pgfqpoint{5.490039in}{5.490039in}}%
\pgfusepath{clip}%
\pgfsetbuttcap%
\pgfsetroundjoin%
\definecolor{currentfill}{rgb}{0.268510,0.009605,0.335427}%
\pgfsetfillcolor{currentfill}%
\pgfsetfillopacity{0.700000}%
\pgfsetlinewidth{0.000000pt}%
\definecolor{currentstroke}{rgb}{0.000000,0.000000,0.000000}%
\pgfsetstrokecolor{currentstroke}%
\pgfsetdash{}{0pt}%
\pgfpathmoveto{\pgfqpoint{3.351940in}{1.651780in}}%
\pgfpathlineto{\pgfqpoint{3.365198in}{1.647200in}}%
\pgfpathlineto{\pgfqpoint{3.378458in}{1.642798in}}%
\pgfpathlineto{\pgfqpoint{3.391722in}{1.638575in}}%
\pgfpathlineto{\pgfqpoint{3.404990in}{1.634529in}}%
\pgfpathlineto{\pgfqpoint{3.412986in}{1.642762in}}%
\pgfpathlineto{\pgfqpoint{3.420974in}{1.651092in}}%
\pgfpathlineto{\pgfqpoint{3.428956in}{1.659517in}}%
\pgfpathlineto{\pgfqpoint{3.436930in}{1.668032in}}%
\pgfpathlineto{\pgfqpoint{3.423680in}{1.671667in}}%
\pgfpathlineto{\pgfqpoint{3.410434in}{1.675479in}}%
\pgfpathlineto{\pgfqpoint{3.397191in}{1.679469in}}%
\pgfpathlineto{\pgfqpoint{3.383952in}{1.683637in}}%
\pgfpathlineto{\pgfqpoint{3.375961in}{1.675523in}}%
\pgfpathlineto{\pgfqpoint{3.367961in}{1.667506in}}%
\pgfpathlineto{\pgfqpoint{3.359955in}{1.659590in}}%
\pgfpathlineto{\pgfqpoint{3.351940in}{1.651780in}}%
\pgfpathclose%
\pgfusepath{fill}%
\end{pgfscope}%
\begin{pgfscope}%
\pgfpathrectangle{\pgfqpoint{1.254980in}{0.150000in}}{\pgfqpoint{5.490039in}{5.490039in}}%
\pgfusepath{clip}%
\pgfsetbuttcap%
\pgfsetroundjoin%
\definecolor{currentfill}{rgb}{0.221989,0.339161,0.548752}%
\pgfsetfillcolor{currentfill}%
\pgfsetfillopacity{0.700000}%
\pgfsetlinewidth{0.000000pt}%
\definecolor{currentstroke}{rgb}{0.000000,0.000000,0.000000}%
\pgfsetstrokecolor{currentstroke}%
\pgfsetdash{}{0pt}%
\pgfpathmoveto{\pgfqpoint{2.399008in}{2.371614in}}%
\pgfpathlineto{\pgfqpoint{2.412576in}{2.351716in}}%
\pgfpathlineto{\pgfqpoint{2.426134in}{2.332090in}}%
\pgfpathlineto{\pgfqpoint{2.439681in}{2.312734in}}%
\pgfpathlineto{\pgfqpoint{2.453218in}{2.293645in}}%
\pgfpathlineto{\pgfqpoint{2.461870in}{2.292066in}}%
\pgfpathlineto{\pgfqpoint{2.470503in}{2.290760in}}%
\pgfpathlineto{\pgfqpoint{2.479119in}{2.289720in}}%
\pgfpathlineto{\pgfqpoint{2.487717in}{2.288942in}}%
\pgfpathlineto{\pgfqpoint{2.474228in}{2.307502in}}%
\pgfpathlineto{\pgfqpoint{2.460729in}{2.326328in}}%
\pgfpathlineto{\pgfqpoint{2.447220in}{2.345423in}}%
\pgfpathlineto{\pgfqpoint{2.433700in}{2.364789in}}%
\pgfpathlineto{\pgfqpoint{2.425055in}{2.366085in}}%
\pgfpathlineto{\pgfqpoint{2.416391in}{2.367651in}}%
\pgfpathlineto{\pgfqpoint{2.407709in}{2.369492in}}%
\pgfpathlineto{\pgfqpoint{2.399008in}{2.371614in}}%
\pgfpathclose%
\pgfusepath{fill}%
\end{pgfscope}%
\begin{pgfscope}%
\pgfpathrectangle{\pgfqpoint{1.254980in}{0.150000in}}{\pgfqpoint{5.490039in}{5.490039in}}%
\pgfusepath{clip}%
\pgfsetbuttcap%
\pgfsetroundjoin%
\definecolor{currentfill}{rgb}{0.137770,0.537492,0.554906}%
\pgfsetfillcolor{currentfill}%
\pgfsetfillopacity{0.700000}%
\pgfsetlinewidth{0.000000pt}%
\definecolor{currentstroke}{rgb}{0.000000,0.000000,0.000000}%
\pgfsetstrokecolor{currentstroke}%
\pgfsetdash{}{0pt}%
\pgfpathmoveto{\pgfqpoint{5.052167in}{2.810962in}}%
\pgfpathlineto{\pgfqpoint{5.066075in}{2.821500in}}%
\pgfpathlineto{\pgfqpoint{5.080001in}{2.832198in}}%
\pgfpathlineto{\pgfqpoint{5.093944in}{2.843056in}}%
\pgfpathlineto{\pgfqpoint{5.107905in}{2.854074in}}%
\pgfpathlineto{\pgfqpoint{5.115240in}{2.860222in}}%
\pgfpathlineto{\pgfqpoint{5.122567in}{2.866269in}}%
\pgfpathlineto{\pgfqpoint{5.129886in}{2.872219in}}%
\pgfpathlineto{\pgfqpoint{5.137197in}{2.878073in}}%
\pgfpathlineto{\pgfqpoint{5.123249in}{2.867277in}}%
\pgfpathlineto{\pgfqpoint{5.109318in}{2.856640in}}%
\pgfpathlineto{\pgfqpoint{5.095404in}{2.846162in}}%
\pgfpathlineto{\pgfqpoint{5.081507in}{2.835844in}}%
\pgfpathlineto{\pgfqpoint{5.074183in}{2.829758in}}%
\pgfpathlineto{\pgfqpoint{5.066852in}{2.823584in}}%
\pgfpathlineto{\pgfqpoint{5.059513in}{2.817319in}}%
\pgfpathlineto{\pgfqpoint{5.052167in}{2.810962in}}%
\pgfpathclose%
\pgfusepath{fill}%
\end{pgfscope}%
\begin{pgfscope}%
\pgfpathrectangle{\pgfqpoint{1.254980in}{0.150000in}}{\pgfqpoint{5.490039in}{5.490039in}}%
\pgfusepath{clip}%
\pgfsetbuttcap%
\pgfsetroundjoin%
\definecolor{currentfill}{rgb}{0.280894,0.078907,0.402329}%
\pgfsetfillcolor{currentfill}%
\pgfsetfillopacity{0.700000}%
\pgfsetlinewidth{0.000000pt}%
\definecolor{currentstroke}{rgb}{0.000000,0.000000,0.000000}%
\pgfsetstrokecolor{currentstroke}%
\pgfsetdash{}{0pt}%
\pgfpathmoveto{\pgfqpoint{3.744149in}{1.758532in}}%
\pgfpathlineto{\pgfqpoint{3.757457in}{1.758893in}}%
\pgfpathlineto{\pgfqpoint{3.770773in}{1.759423in}}%
\pgfpathlineto{\pgfqpoint{3.784096in}{1.760120in}}%
\pgfpathlineto{\pgfqpoint{3.797427in}{1.760985in}}%
\pgfpathlineto{\pgfqpoint{3.805263in}{1.771912in}}%
\pgfpathlineto{\pgfqpoint{3.813095in}{1.782849in}}%
\pgfpathlineto{\pgfqpoint{3.820921in}{1.793793in}}%
\pgfpathlineto{\pgfqpoint{3.828743in}{1.804741in}}%
\pgfpathlineto{\pgfqpoint{3.815420in}{1.803575in}}%
\pgfpathlineto{\pgfqpoint{3.802105in}{1.802577in}}%
\pgfpathlineto{\pgfqpoint{3.788798in}{1.801747in}}%
\pgfpathlineto{\pgfqpoint{3.775499in}{1.801084in}}%
\pgfpathlineto{\pgfqpoint{3.767669in}{1.790427in}}%
\pgfpathlineto{\pgfqpoint{3.759834in}{1.779781in}}%
\pgfpathlineto{\pgfqpoint{3.751994in}{1.769148in}}%
\pgfpathlineto{\pgfqpoint{3.744149in}{1.758532in}}%
\pgfpathclose%
\pgfusepath{fill}%
\end{pgfscope}%
\begin{pgfscope}%
\pgfpathrectangle{\pgfqpoint{1.254980in}{0.150000in}}{\pgfqpoint{5.490039in}{5.490039in}}%
\pgfusepath{clip}%
\pgfsetbuttcap%
\pgfsetroundjoin%
\definecolor{currentfill}{rgb}{0.137339,0.662252,0.515571}%
\pgfsetfillcolor{currentfill}%
\pgfsetfillopacity{0.700000}%
\pgfsetlinewidth{0.000000pt}%
\definecolor{currentstroke}{rgb}{0.000000,0.000000,0.000000}%
\pgfsetstrokecolor{currentstroke}%
\pgfsetdash{}{0pt}%
\pgfpathmoveto{\pgfqpoint{5.506152in}{3.149501in}}%
\pgfpathlineto{\pgfqpoint{5.520327in}{3.161434in}}%
\pgfpathlineto{\pgfqpoint{5.534521in}{3.173527in}}%
\pgfpathlineto{\pgfqpoint{5.548734in}{3.185777in}}%
\pgfpathlineto{\pgfqpoint{5.562967in}{3.198187in}}%
\pgfpathlineto{\pgfqpoint{5.570023in}{3.200855in}}%
\pgfpathlineto{\pgfqpoint{5.577070in}{3.203471in}}%
\pgfpathlineto{\pgfqpoint{5.584110in}{3.206039in}}%
\pgfpathlineto{\pgfqpoint{5.591142in}{3.208565in}}%
\pgfpathlineto{\pgfqpoint{5.576932in}{3.196562in}}%
\pgfpathlineto{\pgfqpoint{5.562741in}{3.184716in}}%
\pgfpathlineto{\pgfqpoint{5.548569in}{3.173029in}}%
\pgfpathlineto{\pgfqpoint{5.534417in}{3.161498in}}%
\pgfpathlineto{\pgfqpoint{5.527362in}{3.158558in}}%
\pgfpathlineto{\pgfqpoint{5.520299in}{3.155581in}}%
\pgfpathlineto{\pgfqpoint{5.513229in}{3.152563in}}%
\pgfpathlineto{\pgfqpoint{5.506152in}{3.149501in}}%
\pgfpathclose%
\pgfusepath{fill}%
\end{pgfscope}%
\begin{pgfscope}%
\pgfpathrectangle{\pgfqpoint{1.254980in}{0.150000in}}{\pgfqpoint{5.490039in}{5.490039in}}%
\pgfusepath{clip}%
\pgfsetbuttcap%
\pgfsetroundjoin%
\definecolor{currentfill}{rgb}{0.277941,0.056324,0.381191}%
\pgfsetfillcolor{currentfill}%
\pgfsetfillopacity{0.700000}%
\pgfsetlinewidth{0.000000pt}%
\definecolor{currentstroke}{rgb}{0.000000,0.000000,0.000000}%
\pgfsetstrokecolor{currentstroke}%
\pgfsetdash{}{0pt}%
\pgfpathmoveto{\pgfqpoint{3.659512in}{1.717844in}}%
\pgfpathlineto{\pgfqpoint{3.672803in}{1.717201in}}%
\pgfpathlineto{\pgfqpoint{3.686100in}{1.716728in}}%
\pgfpathlineto{\pgfqpoint{3.699404in}{1.716424in}}%
\pgfpathlineto{\pgfqpoint{3.712715in}{1.716289in}}%
\pgfpathlineto{\pgfqpoint{3.720581in}{1.726810in}}%
\pgfpathlineto{\pgfqpoint{3.728442in}{1.737360in}}%
\pgfpathlineto{\pgfqpoint{3.736298in}{1.747935in}}%
\pgfpathlineto{\pgfqpoint{3.744149in}{1.758532in}}%
\pgfpathlineto{\pgfqpoint{3.730848in}{1.758338in}}%
\pgfpathlineto{\pgfqpoint{3.717554in}{1.758314in}}%
\pgfpathlineto{\pgfqpoint{3.704267in}{1.758459in}}%
\pgfpathlineto{\pgfqpoint{3.690987in}{1.758773in}}%
\pgfpathlineto{\pgfqpoint{3.683126in}{1.748494in}}%
\pgfpathlineto{\pgfqpoint{3.675260in}{1.738244in}}%
\pgfpathlineto{\pgfqpoint{3.667389in}{1.728026in}}%
\pgfpathlineto{\pgfqpoint{3.659512in}{1.717844in}}%
\pgfpathclose%
\pgfusepath{fill}%
\end{pgfscope}%
\begin{pgfscope}%
\pgfpathrectangle{\pgfqpoint{1.254980in}{0.150000in}}{\pgfqpoint{5.490039in}{5.490039in}}%
\pgfusepath{clip}%
\pgfsetbuttcap%
\pgfsetroundjoin%
\definecolor{currentfill}{rgb}{0.216210,0.351535,0.550627}%
\pgfsetfillcolor{currentfill}%
\pgfsetfillopacity{0.700000}%
\pgfsetlinewidth{0.000000pt}%
\definecolor{currentstroke}{rgb}{0.000000,0.000000,0.000000}%
\pgfsetstrokecolor{currentstroke}%
\pgfsetdash{}{0pt}%
\pgfpathmoveto{\pgfqpoint{4.482779in}{2.325220in}}%
\pgfpathlineto{\pgfqpoint{4.496375in}{2.332587in}}%
\pgfpathlineto{\pgfqpoint{4.509984in}{2.340115in}}%
\pgfpathlineto{\pgfqpoint{4.523607in}{2.347805in}}%
\pgfpathlineto{\pgfqpoint{4.537243in}{2.355656in}}%
\pgfpathlineto{\pgfqpoint{4.544842in}{2.365787in}}%
\pgfpathlineto{\pgfqpoint{4.552435in}{2.375818in}}%
\pgfpathlineto{\pgfqpoint{4.560022in}{2.385749in}}%
\pgfpathlineto{\pgfqpoint{4.567603in}{2.395579in}}%
\pgfpathlineto{\pgfqpoint{4.553970in}{2.387710in}}%
\pgfpathlineto{\pgfqpoint{4.540352in}{2.380001in}}%
\pgfpathlineto{\pgfqpoint{4.526747in}{2.372454in}}%
\pgfpathlineto{\pgfqpoint{4.513155in}{2.365069in}}%
\pgfpathlineto{\pgfqpoint{4.505570in}{2.355246in}}%
\pgfpathlineto{\pgfqpoint{4.497979in}{2.345330in}}%
\pgfpathlineto{\pgfqpoint{4.490382in}{2.335321in}}%
\pgfpathlineto{\pgfqpoint{4.482779in}{2.325220in}}%
\pgfpathclose%
\pgfusepath{fill}%
\end{pgfscope}%
\begin{pgfscope}%
\pgfpathrectangle{\pgfqpoint{1.254980in}{0.150000in}}{\pgfqpoint{5.490039in}{5.490039in}}%
\pgfusepath{clip}%
\pgfsetbuttcap%
\pgfsetroundjoin%
\definecolor{currentfill}{rgb}{0.282910,0.105393,0.426902}%
\pgfsetfillcolor{currentfill}%
\pgfsetfillopacity{0.700000}%
\pgfsetlinewidth{0.000000pt}%
\definecolor{currentstroke}{rgb}{0.000000,0.000000,0.000000}%
\pgfsetstrokecolor{currentstroke}%
\pgfsetdash{}{0pt}%
\pgfpathmoveto{\pgfqpoint{3.828743in}{1.804741in}}%
\pgfpathlineto{\pgfqpoint{3.842073in}{1.806073in}}%
\pgfpathlineto{\pgfqpoint{3.855412in}{1.807573in}}%
\pgfpathlineto{\pgfqpoint{3.868760in}{1.809238in}}%
\pgfpathlineto{\pgfqpoint{3.882116in}{1.811069in}}%
\pgfpathlineto{\pgfqpoint{3.889925in}{1.822302in}}%
\pgfpathlineto{\pgfqpoint{3.897729in}{1.833528in}}%
\pgfpathlineto{\pgfqpoint{3.905529in}{1.844742in}}%
\pgfpathlineto{\pgfqpoint{3.913323in}{1.855944in}}%
\pgfpathlineto{\pgfqpoint{3.899974in}{1.853839in}}%
\pgfpathlineto{\pgfqpoint{3.886634in}{1.851900in}}%
\pgfpathlineto{\pgfqpoint{3.873302in}{1.850128in}}%
\pgfpathlineto{\pgfqpoint{3.859979in}{1.848522in}}%
\pgfpathlineto{\pgfqpoint{3.852177in}{1.837583in}}%
\pgfpathlineto{\pgfqpoint{3.844371in}{1.826639in}}%
\pgfpathlineto{\pgfqpoint{3.836559in}{1.815690in}}%
\pgfpathlineto{\pgfqpoint{3.828743in}{1.804741in}}%
\pgfpathclose%
\pgfusepath{fill}%
\end{pgfscope}%
\begin{pgfscope}%
\pgfpathrectangle{\pgfqpoint{1.254980in}{0.150000in}}{\pgfqpoint{5.490039in}{5.490039in}}%
\pgfusepath{clip}%
\pgfsetbuttcap%
\pgfsetroundjoin%
\definecolor{currentfill}{rgb}{0.283091,0.110553,0.431554}%
\pgfsetfillcolor{currentfill}%
\pgfsetfillopacity{0.700000}%
\pgfsetlinewidth{0.000000pt}%
\definecolor{currentstroke}{rgb}{0.000000,0.000000,0.000000}%
\pgfsetstrokecolor{currentstroke}%
\pgfsetdash{}{0pt}%
\pgfpathmoveto{\pgfqpoint{2.829014in}{1.859181in}}%
\pgfpathlineto{\pgfqpoint{2.842354in}{1.846972in}}%
\pgfpathlineto{\pgfqpoint{2.855691in}{1.834975in}}%
\pgfpathlineto{\pgfqpoint{2.869024in}{1.823189in}}%
\pgfpathlineto{\pgfqpoint{2.882354in}{1.811613in}}%
\pgfpathlineto{\pgfqpoint{2.890675in}{1.814248in}}%
\pgfpathlineto{\pgfqpoint{2.898983in}{1.817093in}}%
\pgfpathlineto{\pgfqpoint{2.907279in}{1.820143in}}%
\pgfpathlineto{\pgfqpoint{2.915561in}{1.823393in}}%
\pgfpathlineto{\pgfqpoint{2.902265in}{1.834464in}}%
\pgfpathlineto{\pgfqpoint{2.888966in}{1.845745in}}%
\pgfpathlineto{\pgfqpoint{2.875664in}{1.857236in}}%
\pgfpathlineto{\pgfqpoint{2.862359in}{1.868938in}}%
\pgfpathlineto{\pgfqpoint{2.854043in}{1.866183in}}%
\pgfpathlineto{\pgfqpoint{2.845713in}{1.863635in}}%
\pgfpathlineto{\pgfqpoint{2.837370in}{1.861299in}}%
\pgfpathlineto{\pgfqpoint{2.829014in}{1.859181in}}%
\pgfpathclose%
\pgfusepath{fill}%
\end{pgfscope}%
\begin{pgfscope}%
\pgfpathrectangle{\pgfqpoint{1.254980in}{0.150000in}}{\pgfqpoint{5.490039in}{5.490039in}}%
\pgfusepath{clip}%
\pgfsetbuttcap%
\pgfsetroundjoin%
\definecolor{currentfill}{rgb}{0.273809,0.031497,0.358853}%
\pgfsetfillcolor{currentfill}%
\pgfsetfillopacity{0.700000}%
\pgfsetlinewidth{0.000000pt}%
\definecolor{currentstroke}{rgb}{0.000000,0.000000,0.000000}%
\pgfsetstrokecolor{currentstroke}%
\pgfsetdash{}{0pt}%
\pgfpathmoveto{\pgfqpoint{3.074981in}{1.706392in}}%
\pgfpathlineto{\pgfqpoint{3.088260in}{1.697928in}}%
\pgfpathlineto{\pgfqpoint{3.101539in}{1.689657in}}%
\pgfpathlineto{\pgfqpoint{3.114819in}{1.681578in}}%
\pgfpathlineto{\pgfqpoint{3.128098in}{1.673690in}}%
\pgfpathlineto{\pgfqpoint{3.136251in}{1.679068in}}%
\pgfpathlineto{\pgfqpoint{3.144394in}{1.684607in}}%
\pgfpathlineto{\pgfqpoint{3.152527in}{1.690303in}}%
\pgfpathlineto{\pgfqpoint{3.160650in}{1.696150in}}%
\pgfpathlineto{\pgfqpoint{3.147397in}{1.703568in}}%
\pgfpathlineto{\pgfqpoint{3.134144in}{1.711177in}}%
\pgfpathlineto{\pgfqpoint{3.120891in}{1.718978in}}%
\pgfpathlineto{\pgfqpoint{3.107639in}{1.726971in}}%
\pgfpathlineto{\pgfqpoint{3.099490in}{1.721583in}}%
\pgfpathlineto{\pgfqpoint{3.091331in}{1.716354in}}%
\pgfpathlineto{\pgfqpoint{3.083161in}{1.711289in}}%
\pgfpathlineto{\pgfqpoint{3.074981in}{1.706392in}}%
\pgfpathclose%
\pgfusepath{fill}%
\end{pgfscope}%
\begin{pgfscope}%
\pgfpathrectangle{\pgfqpoint{1.254980in}{0.150000in}}{\pgfqpoint{5.490039in}{5.490039in}}%
\pgfusepath{clip}%
\pgfsetbuttcap%
\pgfsetroundjoin%
\definecolor{currentfill}{rgb}{0.171176,0.452530,0.557965}%
\pgfsetfillcolor{currentfill}%
\pgfsetfillopacity{0.700000}%
\pgfsetlinewidth{0.000000pt}%
\definecolor{currentstroke}{rgb}{0.000000,0.000000,0.000000}%
\pgfsetstrokecolor{currentstroke}%
\pgfsetdash{}{0pt}%
\pgfpathmoveto{\pgfqpoint{4.767552in}{2.573743in}}%
\pgfpathlineto{\pgfqpoint{4.781301in}{2.582940in}}%
\pgfpathlineto{\pgfqpoint{4.795065in}{2.592298in}}%
\pgfpathlineto{\pgfqpoint{4.808845in}{2.601816in}}%
\pgfpathlineto{\pgfqpoint{4.822641in}{2.611494in}}%
\pgfpathlineto{\pgfqpoint{4.830122in}{2.619820in}}%
\pgfpathlineto{\pgfqpoint{4.837596in}{2.628035in}}%
\pgfpathlineto{\pgfqpoint{4.845063in}{2.636141in}}%
\pgfpathlineto{\pgfqpoint{4.852523in}{2.644141in}}%
\pgfpathlineto{\pgfqpoint{4.838734in}{2.634562in}}%
\pgfpathlineto{\pgfqpoint{4.824961in}{2.625144in}}%
\pgfpathlineto{\pgfqpoint{4.811204in}{2.615886in}}%
\pgfpathlineto{\pgfqpoint{4.797462in}{2.606789in}}%
\pgfpathlineto{\pgfqpoint{4.789995in}{2.598679in}}%
\pgfpathlineto{\pgfqpoint{4.782520in}{2.590469in}}%
\pgfpathlineto{\pgfqpoint{4.775039in}{2.582158in}}%
\pgfpathlineto{\pgfqpoint{4.767552in}{2.573743in}}%
\pgfpathclose%
\pgfusepath{fill}%
\end{pgfscope}%
\begin{pgfscope}%
\pgfpathrectangle{\pgfqpoint{1.254980in}{0.150000in}}{\pgfqpoint{5.490039in}{5.490039in}}%
\pgfusepath{clip}%
\pgfsetbuttcap%
\pgfsetroundjoin%
\definecolor{currentfill}{rgb}{0.273809,0.031497,0.358853}%
\pgfsetfillcolor{currentfill}%
\pgfsetfillopacity{0.700000}%
\pgfsetlinewidth{0.000000pt}%
\definecolor{currentstroke}{rgb}{0.000000,0.000000,0.000000}%
\pgfsetstrokecolor{currentstroke}%
\pgfsetdash{}{0pt}%
\pgfpathmoveto{\pgfqpoint{3.574799in}{1.683227in}}%
\pgfpathlineto{\pgfqpoint{3.588078in}{1.681545in}}%
\pgfpathlineto{\pgfqpoint{3.601362in}{1.680035in}}%
\pgfpathlineto{\pgfqpoint{3.614652in}{1.678696in}}%
\pgfpathlineto{\pgfqpoint{3.627948in}{1.677527in}}%
\pgfpathlineto{\pgfqpoint{3.635848in}{1.687538in}}%
\pgfpathlineto{\pgfqpoint{3.643741in}{1.697596in}}%
\pgfpathlineto{\pgfqpoint{3.651630in}{1.707699in}}%
\pgfpathlineto{\pgfqpoint{3.659512in}{1.717844in}}%
\pgfpathlineto{\pgfqpoint{3.646228in}{1.718657in}}%
\pgfpathlineto{\pgfqpoint{3.632950in}{1.719640in}}%
\pgfpathlineto{\pgfqpoint{3.619678in}{1.720795in}}%
\pgfpathlineto{\pgfqpoint{3.606412in}{1.722121in}}%
\pgfpathlineto{\pgfqpoint{3.598518in}{1.712322in}}%
\pgfpathlineto{\pgfqpoint{3.590617in}{1.702571in}}%
\pgfpathlineto{\pgfqpoint{3.582711in}{1.692872in}}%
\pgfpathlineto{\pgfqpoint{3.574799in}{1.683227in}}%
\pgfpathclose%
\pgfusepath{fill}%
\end{pgfscope}%
\begin{pgfscope}%
\pgfpathrectangle{\pgfqpoint{1.254980in}{0.150000in}}{\pgfqpoint{5.490039in}{5.490039in}}%
\pgfusepath{clip}%
\pgfsetbuttcap%
\pgfsetroundjoin%
\definecolor{currentfill}{rgb}{0.282884,0.135920,0.453427}%
\pgfsetfillcolor{currentfill}%
\pgfsetfillopacity{0.700000}%
\pgfsetlinewidth{0.000000pt}%
\definecolor{currentstroke}{rgb}{0.000000,0.000000,0.000000}%
\pgfsetstrokecolor{currentstroke}%
\pgfsetdash{}{0pt}%
\pgfpathmoveto{\pgfqpoint{3.913323in}{1.855944in}}%
\pgfpathlineto{\pgfqpoint{3.926681in}{1.858215in}}%
\pgfpathlineto{\pgfqpoint{3.940048in}{1.860650in}}%
\pgfpathlineto{\pgfqpoint{3.953425in}{1.863252in}}%
\pgfpathlineto{\pgfqpoint{3.966810in}{1.866017in}}%
\pgfpathlineto{\pgfqpoint{3.974594in}{1.877460in}}%
\pgfpathlineto{\pgfqpoint{3.982373in}{1.888879in}}%
\pgfpathlineto{\pgfqpoint{3.990148in}{1.900271in}}%
\pgfpathlineto{\pgfqpoint{3.997917in}{1.911634in}}%
\pgfpathlineto{\pgfqpoint{3.984537in}{1.908623in}}%
\pgfpathlineto{\pgfqpoint{3.971167in}{1.905776in}}%
\pgfpathlineto{\pgfqpoint{3.957806in}{1.903094in}}%
\pgfpathlineto{\pgfqpoint{3.944454in}{1.900578in}}%
\pgfpathlineto{\pgfqpoint{3.936679in}{1.889450in}}%
\pgfpathlineto{\pgfqpoint{3.928898in}{1.878300in}}%
\pgfpathlineto{\pgfqpoint{3.921113in}{1.867131in}}%
\pgfpathlineto{\pgfqpoint{3.913323in}{1.855944in}}%
\pgfpathclose%
\pgfusepath{fill}%
\end{pgfscope}%
\begin{pgfscope}%
\pgfpathrectangle{\pgfqpoint{1.254980in}{0.150000in}}{\pgfqpoint{5.490039in}{5.490039in}}%
\pgfusepath{clip}%
\pgfsetbuttcap%
\pgfsetroundjoin%
\definecolor{currentfill}{rgb}{0.157851,0.683765,0.501686}%
\pgfsetfillcolor{currentfill}%
\pgfsetfillopacity{0.700000}%
\pgfsetlinewidth{0.000000pt}%
\definecolor{currentstroke}{rgb}{0.000000,0.000000,0.000000}%
\pgfsetstrokecolor{currentstroke}%
\pgfsetdash{}{0pt}%
\pgfpathmoveto{\pgfqpoint{5.591142in}{3.208565in}}%
\pgfpathlineto{\pgfqpoint{5.605372in}{3.220726in}}%
\pgfpathlineto{\pgfqpoint{5.619621in}{3.233045in}}%
\pgfpathlineto{\pgfqpoint{5.633890in}{3.245523in}}%
\pgfpathlineto{\pgfqpoint{5.648179in}{3.258159in}}%
\pgfpathlineto{\pgfqpoint{5.655179in}{3.260221in}}%
\pgfpathlineto{\pgfqpoint{5.662171in}{3.262241in}}%
\pgfpathlineto{\pgfqpoint{5.669155in}{3.264226in}}%
\pgfpathlineto{\pgfqpoint{5.676132in}{3.266180in}}%
\pgfpathlineto{\pgfqpoint{5.661868in}{3.253981in}}%
\pgfpathlineto{\pgfqpoint{5.647624in}{3.241940in}}%
\pgfpathlineto{\pgfqpoint{5.633400in}{3.230056in}}%
\pgfpathlineto{\pgfqpoint{5.619194in}{3.218329in}}%
\pgfpathlineto{\pgfqpoint{5.612192in}{3.215929in}}%
\pgfpathlineto{\pgfqpoint{5.605183in}{3.213505in}}%
\pgfpathlineto{\pgfqpoint{5.598166in}{3.211052in}}%
\pgfpathlineto{\pgfqpoint{5.591142in}{3.208565in}}%
\pgfpathclose%
\pgfusepath{fill}%
\end{pgfscope}%
\begin{pgfscope}%
\pgfpathrectangle{\pgfqpoint{1.254980in}{0.150000in}}{\pgfqpoint{5.490039in}{5.490039in}}%
\pgfusepath{clip}%
\pgfsetbuttcap%
\pgfsetroundjoin%
\definecolor{currentfill}{rgb}{0.204903,0.375746,0.553533}%
\pgfsetfillcolor{currentfill}%
\pgfsetfillopacity{0.700000}%
\pgfsetlinewidth{0.000000pt}%
\definecolor{currentstroke}{rgb}{0.000000,0.000000,0.000000}%
\pgfsetstrokecolor{currentstroke}%
\pgfsetdash{}{0pt}%
\pgfpathmoveto{\pgfqpoint{2.344616in}{2.453977in}}%
\pgfpathlineto{\pgfqpoint{2.358232in}{2.432965in}}%
\pgfpathlineto{\pgfqpoint{2.371836in}{2.412236in}}%
\pgfpathlineto{\pgfqpoint{2.385427in}{2.391786in}}%
\pgfpathlineto{\pgfqpoint{2.399008in}{2.371614in}}%
\pgfpathlineto{\pgfqpoint{2.407709in}{2.369492in}}%
\pgfpathlineto{\pgfqpoint{2.416391in}{2.367651in}}%
\pgfpathlineto{\pgfqpoint{2.425055in}{2.366085in}}%
\pgfpathlineto{\pgfqpoint{2.433700in}{2.364789in}}%
\pgfpathlineto{\pgfqpoint{2.420170in}{2.384428in}}%
\pgfpathlineto{\pgfqpoint{2.406628in}{2.404343in}}%
\pgfpathlineto{\pgfqpoint{2.393074in}{2.424537in}}%
\pgfpathlineto{\pgfqpoint{2.379509in}{2.445012in}}%
\pgfpathlineto{\pgfqpoint{2.370815in}{2.446831in}}%
\pgfpathlineto{\pgfqpoint{2.362101in}{2.448928in}}%
\pgfpathlineto{\pgfqpoint{2.353368in}{2.451308in}}%
\pgfpathlineto{\pgfqpoint{2.344616in}{2.453977in}}%
\pgfpathclose%
\pgfusepath{fill}%
\end{pgfscope}%
\begin{pgfscope}%
\pgfpathrectangle{\pgfqpoint{1.254980in}{0.150000in}}{\pgfqpoint{5.490039in}{5.490039in}}%
\pgfusepath{clip}%
\pgfsetbuttcap%
\pgfsetroundjoin%
\definecolor{currentfill}{rgb}{0.248629,0.278775,0.534556}%
\pgfsetfillcolor{currentfill}%
\pgfsetfillopacity{0.700000}%
\pgfsetlinewidth{0.000000pt}%
\definecolor{currentstroke}{rgb}{0.000000,0.000000,0.000000}%
\pgfsetstrokecolor{currentstroke}%
\pgfsetdash{}{0pt}%
\pgfpathmoveto{\pgfqpoint{4.282734in}{2.145498in}}%
\pgfpathlineto{\pgfqpoint{4.296238in}{2.151340in}}%
\pgfpathlineto{\pgfqpoint{4.309754in}{2.157344in}}%
\pgfpathlineto{\pgfqpoint{4.323283in}{2.163510in}}%
\pgfpathlineto{\pgfqpoint{4.336823in}{2.169838in}}%
\pgfpathlineto{\pgfqpoint{4.344495in}{2.180928in}}%
\pgfpathlineto{\pgfqpoint{4.352161in}{2.191936in}}%
\pgfpathlineto{\pgfqpoint{4.359822in}{2.202861in}}%
\pgfpathlineto{\pgfqpoint{4.367478in}{2.213702in}}%
\pgfpathlineto{\pgfqpoint{4.353940in}{2.207269in}}%
\pgfpathlineto{\pgfqpoint{4.340415in}{2.200997in}}%
\pgfpathlineto{\pgfqpoint{4.326903in}{2.194888in}}%
\pgfpathlineto{\pgfqpoint{4.313402in}{2.188941in}}%
\pgfpathlineto{\pgfqpoint{4.305743in}{2.178195in}}%
\pgfpathlineto{\pgfqpoint{4.298079in}{2.167372in}}%
\pgfpathlineto{\pgfqpoint{4.290409in}{2.156472in}}%
\pgfpathlineto{\pgfqpoint{4.282734in}{2.145498in}}%
\pgfpathclose%
\pgfusepath{fill}%
\end{pgfscope}%
\begin{pgfscope}%
\pgfpathrectangle{\pgfqpoint{1.254980in}{0.150000in}}{\pgfqpoint{5.490039in}{5.490039in}}%
\pgfusepath{clip}%
\pgfsetbuttcap%
\pgfsetroundjoin%
\definecolor{currentfill}{rgb}{0.128729,0.563265,0.551229}%
\pgfsetfillcolor{currentfill}%
\pgfsetfillopacity{0.700000}%
\pgfsetlinewidth{0.000000pt}%
\definecolor{currentstroke}{rgb}{0.000000,0.000000,0.000000}%
\pgfsetstrokecolor{currentstroke}%
\pgfsetdash{}{0pt}%
\pgfpathmoveto{\pgfqpoint{5.137197in}{2.878073in}}%
\pgfpathlineto{\pgfqpoint{5.151163in}{2.889029in}}%
\pgfpathlineto{\pgfqpoint{5.165147in}{2.900145in}}%
\pgfpathlineto{\pgfqpoint{5.179148in}{2.911420in}}%
\pgfpathlineto{\pgfqpoint{5.193167in}{2.922855in}}%
\pgfpathlineto{\pgfqpoint{5.200458in}{2.928376in}}%
\pgfpathlineto{\pgfqpoint{5.207740in}{2.933801in}}%
\pgfpathlineto{\pgfqpoint{5.215015in}{2.939131in}}%
\pgfpathlineto{\pgfqpoint{5.222281in}{2.944371in}}%
\pgfpathlineto{\pgfqpoint{5.208276in}{2.933188in}}%
\pgfpathlineto{\pgfqpoint{5.194288in}{2.922165in}}%
\pgfpathlineto{\pgfqpoint{5.180318in}{2.911301in}}%
\pgfpathlineto{\pgfqpoint{5.166365in}{2.900596in}}%
\pgfpathlineto{\pgfqpoint{5.159085in}{2.895094in}}%
\pgfpathlineto{\pgfqpoint{5.151797in}{2.889508in}}%
\pgfpathlineto{\pgfqpoint{5.144501in}{2.883835in}}%
\pgfpathlineto{\pgfqpoint{5.137197in}{2.878073in}}%
\pgfpathclose%
\pgfusepath{fill}%
\end{pgfscope}%
\begin{pgfscope}%
\pgfpathrectangle{\pgfqpoint{1.254980in}{0.150000in}}{\pgfqpoint{5.490039in}{5.490039in}}%
\pgfusepath{clip}%
\pgfsetbuttcap%
\pgfsetroundjoin%
\definecolor{currentfill}{rgb}{0.281924,0.089666,0.412415}%
\pgfsetfillcolor{currentfill}%
\pgfsetfillopacity{0.700000}%
\pgfsetlinewidth{0.000000pt}%
\definecolor{currentstroke}{rgb}{0.000000,0.000000,0.000000}%
\pgfsetstrokecolor{currentstroke}%
\pgfsetdash{}{0pt}%
\pgfpathmoveto{\pgfqpoint{2.882354in}{1.811613in}}%
\pgfpathlineto{\pgfqpoint{2.895682in}{1.800244in}}%
\pgfpathlineto{\pgfqpoint{2.909007in}{1.789083in}}%
\pgfpathlineto{\pgfqpoint{2.922329in}{1.778128in}}%
\pgfpathlineto{\pgfqpoint{2.935649in}{1.767377in}}%
\pgfpathlineto{\pgfqpoint{2.943937in}{1.770527in}}%
\pgfpathlineto{\pgfqpoint{2.952211in}{1.773880in}}%
\pgfpathlineto{\pgfqpoint{2.960474in}{1.777430in}}%
\pgfpathlineto{\pgfqpoint{2.968725in}{1.781173in}}%
\pgfpathlineto{\pgfqpoint{2.955437in}{1.791421in}}%
\pgfpathlineto{\pgfqpoint{2.942147in}{1.801872in}}%
\pgfpathlineto{\pgfqpoint{2.928855in}{1.812530in}}%
\pgfpathlineto{\pgfqpoint{2.915561in}{1.823393in}}%
\pgfpathlineto{\pgfqpoint{2.907279in}{1.820143in}}%
\pgfpathlineto{\pgfqpoint{2.898983in}{1.817093in}}%
\pgfpathlineto{\pgfqpoint{2.890675in}{1.814248in}}%
\pgfpathlineto{\pgfqpoint{2.882354in}{1.811613in}}%
\pgfpathclose%
\pgfusepath{fill}%
\end{pgfscope}%
\begin{pgfscope}%
\pgfpathrectangle{\pgfqpoint{1.254980in}{0.150000in}}{\pgfqpoint{5.490039in}{5.490039in}}%
\pgfusepath{clip}%
\pgfsetbuttcap%
\pgfsetroundjoin%
\definecolor{currentfill}{rgb}{0.269944,0.014625,0.341379}%
\pgfsetfillcolor{currentfill}%
\pgfsetfillopacity{0.700000}%
\pgfsetlinewidth{0.000000pt}%
\definecolor{currentstroke}{rgb}{0.000000,0.000000,0.000000}%
\pgfsetstrokecolor{currentstroke}%
\pgfsetdash{}{0pt}%
\pgfpathmoveto{\pgfqpoint{3.489972in}{1.655256in}}%
\pgfpathlineto{\pgfqpoint{3.503244in}{1.652499in}}%
\pgfpathlineto{\pgfqpoint{3.516521in}{1.649916in}}%
\pgfpathlineto{\pgfqpoint{3.529803in}{1.647506in}}%
\pgfpathlineto{\pgfqpoint{3.543089in}{1.645269in}}%
\pgfpathlineto{\pgfqpoint{3.551026in}{1.654659in}}%
\pgfpathlineto{\pgfqpoint{3.558957in}{1.664117in}}%
\pgfpathlineto{\pgfqpoint{3.566881in}{1.673641in}}%
\pgfpathlineto{\pgfqpoint{3.574799in}{1.683227in}}%
\pgfpathlineto{\pgfqpoint{3.561526in}{1.685081in}}%
\pgfpathlineto{\pgfqpoint{3.548259in}{1.687108in}}%
\pgfpathlineto{\pgfqpoint{3.534997in}{1.689307in}}%
\pgfpathlineto{\pgfqpoint{3.521739in}{1.691681in}}%
\pgfpathlineto{\pgfqpoint{3.513807in}{1.682468in}}%
\pgfpathlineto{\pgfqpoint{3.505869in}{1.673324in}}%
\pgfpathlineto{\pgfqpoint{3.497924in}{1.664252in}}%
\pgfpathlineto{\pgfqpoint{3.489972in}{1.655256in}}%
\pgfpathclose%
\pgfusepath{fill}%
\end{pgfscope}%
\begin{pgfscope}%
\pgfpathrectangle{\pgfqpoint{1.254980in}{0.150000in}}{\pgfqpoint{5.490039in}{5.490039in}}%
\pgfusepath{clip}%
\pgfsetbuttcap%
\pgfsetroundjoin%
\definecolor{currentfill}{rgb}{0.280255,0.165693,0.476498}%
\pgfsetfillcolor{currentfill}%
\pgfsetfillopacity{0.700000}%
\pgfsetlinewidth{0.000000pt}%
\definecolor{currentstroke}{rgb}{0.000000,0.000000,0.000000}%
\pgfsetstrokecolor{currentstroke}%
\pgfsetdash{}{0pt}%
\pgfpathmoveto{\pgfqpoint{3.997917in}{1.911634in}}%
\pgfpathlineto{\pgfqpoint{4.011307in}{1.914811in}}%
\pgfpathlineto{\pgfqpoint{4.024706in}{1.918152in}}%
\pgfpathlineto{\pgfqpoint{4.038115in}{1.921657in}}%
\pgfpathlineto{\pgfqpoint{4.051535in}{1.925326in}}%
\pgfpathlineto{\pgfqpoint{4.059295in}{1.936887in}}%
\pgfpathlineto{\pgfqpoint{4.067050in}{1.948409in}}%
\pgfpathlineto{\pgfqpoint{4.074801in}{1.959889in}}%
\pgfpathlineto{\pgfqpoint{4.082547in}{1.971326in}}%
\pgfpathlineto{\pgfqpoint{4.069132in}{1.967439in}}%
\pgfpathlineto{\pgfqpoint{4.055727in}{1.963716in}}%
\pgfpathlineto{\pgfqpoint{4.042333in}{1.960157in}}%
\pgfpathlineto{\pgfqpoint{4.028949in}{1.956762in}}%
\pgfpathlineto{\pgfqpoint{4.021198in}{1.945533in}}%
\pgfpathlineto{\pgfqpoint{4.013442in}{1.934267in}}%
\pgfpathlineto{\pgfqpoint{4.005682in}{1.922967in}}%
\pgfpathlineto{\pgfqpoint{3.997917in}{1.911634in}}%
\pgfpathclose%
\pgfusepath{fill}%
\end{pgfscope}%
\begin{pgfscope}%
\pgfpathrectangle{\pgfqpoint{1.254980in}{0.150000in}}{\pgfqpoint{5.490039in}{5.490039in}}%
\pgfusepath{clip}%
\pgfsetbuttcap%
\pgfsetroundjoin%
\definecolor{currentfill}{rgb}{0.185783,0.704891,0.485273}%
\pgfsetfillcolor{currentfill}%
\pgfsetfillopacity{0.700000}%
\pgfsetlinewidth{0.000000pt}%
\definecolor{currentstroke}{rgb}{0.000000,0.000000,0.000000}%
\pgfsetstrokecolor{currentstroke}%
\pgfsetdash{}{0pt}%
\pgfpathmoveto{\pgfqpoint{5.676132in}{3.266180in}}%
\pgfpathlineto{\pgfqpoint{5.690416in}{3.278536in}}%
\pgfpathlineto{\pgfqpoint{5.704719in}{3.291051in}}%
\pgfpathlineto{\pgfqpoint{5.719043in}{3.303723in}}%
\pgfpathlineto{\pgfqpoint{5.733387in}{3.316554in}}%
\pgfpathlineto{\pgfqpoint{5.740329in}{3.318024in}}%
\pgfpathlineto{\pgfqpoint{5.747264in}{3.319467in}}%
\pgfpathlineto{\pgfqpoint{5.754192in}{3.320886in}}%
\pgfpathlineto{\pgfqpoint{5.761112in}{3.322288in}}%
\pgfpathlineto{\pgfqpoint{5.746796in}{3.309925in}}%
\pgfpathlineto{\pgfqpoint{5.732499in}{3.297721in}}%
\pgfpathlineto{\pgfqpoint{5.718223in}{3.285673in}}%
\pgfpathlineto{\pgfqpoint{5.703966in}{3.273782in}}%
\pgfpathlineto{\pgfqpoint{5.697018in}{3.271903in}}%
\pgfpathlineto{\pgfqpoint{5.690063in}{3.270013in}}%
\pgfpathlineto{\pgfqpoint{5.683101in}{3.268107in}}%
\pgfpathlineto{\pgfqpoint{5.676132in}{3.266180in}}%
\pgfpathclose%
\pgfusepath{fill}%
\end{pgfscope}%
\begin{pgfscope}%
\pgfpathrectangle{\pgfqpoint{1.254980in}{0.150000in}}{\pgfqpoint{5.490039in}{5.490039in}}%
\pgfusepath{clip}%
\pgfsetbuttcap%
\pgfsetroundjoin%
\definecolor{currentfill}{rgb}{0.201239,0.383670,0.554294}%
\pgfsetfillcolor{currentfill}%
\pgfsetfillopacity{0.700000}%
\pgfsetlinewidth{0.000000pt}%
\definecolor{currentstroke}{rgb}{0.000000,0.000000,0.000000}%
\pgfsetstrokecolor{currentstroke}%
\pgfsetdash{}{0pt}%
\pgfpathmoveto{\pgfqpoint{4.567603in}{2.395579in}}%
\pgfpathlineto{\pgfqpoint{4.581250in}{2.403610in}}%
\pgfpathlineto{\pgfqpoint{4.594910in}{2.411802in}}%
\pgfpathlineto{\pgfqpoint{4.608586in}{2.420156in}}%
\pgfpathlineto{\pgfqpoint{4.622275in}{2.428670in}}%
\pgfpathlineto{\pgfqpoint{4.629846in}{2.438401in}}%
\pgfpathlineto{\pgfqpoint{4.637411in}{2.448026in}}%
\pgfpathlineto{\pgfqpoint{4.644969in}{2.457544in}}%
\pgfpathlineto{\pgfqpoint{4.652521in}{2.466958in}}%
\pgfpathlineto{\pgfqpoint{4.638836in}{2.458454in}}%
\pgfpathlineto{\pgfqpoint{4.625165in}{2.450112in}}%
\pgfpathlineto{\pgfqpoint{4.611509in}{2.441931in}}%
\pgfpathlineto{\pgfqpoint{4.597867in}{2.433910in}}%
\pgfpathlineto{\pgfqpoint{4.590310in}{2.424475in}}%
\pgfpathlineto{\pgfqpoint{4.582747in}{2.414942in}}%
\pgfpathlineto{\pgfqpoint{4.575178in}{2.405310in}}%
\pgfpathlineto{\pgfqpoint{4.567603in}{2.395579in}}%
\pgfpathclose%
\pgfusepath{fill}%
\end{pgfscope}%
\begin{pgfscope}%
\pgfpathrectangle{\pgfqpoint{1.254980in}{0.150000in}}{\pgfqpoint{5.490039in}{5.490039in}}%
\pgfusepath{clip}%
\pgfsetbuttcap%
\pgfsetroundjoin%
\definecolor{currentfill}{rgb}{0.268510,0.009605,0.335427}%
\pgfsetfillcolor{currentfill}%
\pgfsetfillopacity{0.700000}%
\pgfsetlinewidth{0.000000pt}%
\definecolor{currentstroke}{rgb}{0.000000,0.000000,0.000000}%
\pgfsetstrokecolor{currentstroke}%
\pgfsetdash{}{0pt}%
\pgfpathmoveto{\pgfqpoint{3.266722in}{1.643557in}}%
\pgfpathlineto{\pgfqpoint{3.279989in}{1.637814in}}%
\pgfpathlineto{\pgfqpoint{3.293258in}{1.632253in}}%
\pgfpathlineto{\pgfqpoint{3.306530in}{1.626873in}}%
\pgfpathlineto{\pgfqpoint{3.319804in}{1.621673in}}%
\pgfpathlineto{\pgfqpoint{3.327850in}{1.629022in}}%
\pgfpathlineto{\pgfqpoint{3.335888in}{1.636491in}}%
\pgfpathlineto{\pgfqpoint{3.343918in}{1.644079in}}%
\pgfpathlineto{\pgfqpoint{3.351940in}{1.651780in}}%
\pgfpathlineto{\pgfqpoint{3.338686in}{1.656540in}}%
\pgfpathlineto{\pgfqpoint{3.325435in}{1.661480in}}%
\pgfpathlineto{\pgfqpoint{3.312187in}{1.666601in}}%
\pgfpathlineto{\pgfqpoint{3.298941in}{1.671904in}}%
\pgfpathlineto{\pgfqpoint{3.290899in}{1.664633in}}%
\pgfpathlineto{\pgfqpoint{3.282848in}{1.657481in}}%
\pgfpathlineto{\pgfqpoint{3.274790in}{1.650455in}}%
\pgfpathlineto{\pgfqpoint{3.266722in}{1.643557in}}%
\pgfpathclose%
\pgfusepath{fill}%
\end{pgfscope}%
\begin{pgfscope}%
\pgfpathrectangle{\pgfqpoint{1.254980in}{0.150000in}}{\pgfqpoint{5.490039in}{5.490039in}}%
\pgfusepath{clip}%
\pgfsetbuttcap%
\pgfsetroundjoin%
\definecolor{currentfill}{rgb}{0.160665,0.478540,0.558115}%
\pgfsetfillcolor{currentfill}%
\pgfsetfillopacity{0.700000}%
\pgfsetlinewidth{0.000000pt}%
\definecolor{currentstroke}{rgb}{0.000000,0.000000,0.000000}%
\pgfsetstrokecolor{currentstroke}%
\pgfsetdash{}{0pt}%
\pgfpathmoveto{\pgfqpoint{4.852523in}{2.644141in}}%
\pgfpathlineto{\pgfqpoint{4.866328in}{2.653879in}}%
\pgfpathlineto{\pgfqpoint{4.880148in}{2.663779in}}%
\pgfpathlineto{\pgfqpoint{4.893985in}{2.673838in}}%
\pgfpathlineto{\pgfqpoint{4.907838in}{2.684059in}}%
\pgfpathlineto{\pgfqpoint{4.915284in}{2.691833in}}%
\pgfpathlineto{\pgfqpoint{4.922722in}{2.699495in}}%
\pgfpathlineto{\pgfqpoint{4.930153in}{2.707048in}}%
\pgfpathlineto{\pgfqpoint{4.937576in}{2.714493in}}%
\pgfpathlineto{\pgfqpoint{4.923731in}{2.704403in}}%
\pgfpathlineto{\pgfqpoint{4.909902in}{2.694474in}}%
\pgfpathlineto{\pgfqpoint{4.896089in}{2.684705in}}%
\pgfpathlineto{\pgfqpoint{4.882293in}{2.675096in}}%
\pgfpathlineto{\pgfqpoint{4.874861in}{2.667510in}}%
\pgfpathlineto{\pgfqpoint{4.867422in}{2.659823in}}%
\pgfpathlineto{\pgfqpoint{4.859976in}{2.652034in}}%
\pgfpathlineto{\pgfqpoint{4.852523in}{2.644141in}}%
\pgfpathclose%
\pgfusepath{fill}%
\end{pgfscope}%
\begin{pgfscope}%
\pgfpathrectangle{\pgfqpoint{1.254980in}{0.150000in}}{\pgfqpoint{5.490039in}{5.490039in}}%
\pgfusepath{clip}%
\pgfsetbuttcap%
\pgfsetroundjoin%
\definecolor{currentfill}{rgb}{0.220124,0.725509,0.466226}%
\pgfsetfillcolor{currentfill}%
\pgfsetfillopacity{0.700000}%
\pgfsetlinewidth{0.000000pt}%
\definecolor{currentstroke}{rgb}{0.000000,0.000000,0.000000}%
\pgfsetstrokecolor{currentstroke}%
\pgfsetdash{}{0pt}%
\pgfpathmoveto{\pgfqpoint{5.761112in}{3.322288in}}%
\pgfpathlineto{\pgfqpoint{5.775449in}{3.334807in}}%
\pgfpathlineto{\pgfqpoint{5.789806in}{3.347485in}}%
\pgfpathlineto{\pgfqpoint{5.804183in}{3.360320in}}%
\pgfpathlineto{\pgfqpoint{5.818581in}{3.373314in}}%
\pgfpathlineto{\pgfqpoint{5.825465in}{3.374214in}}%
\pgfpathlineto{\pgfqpoint{5.832342in}{3.375101in}}%
\pgfpathlineto{\pgfqpoint{5.839212in}{3.375978in}}%
\pgfpathlineto{\pgfqpoint{5.846074in}{3.376852in}}%
\pgfpathlineto{\pgfqpoint{5.831706in}{3.364359in}}%
\pgfpathlineto{\pgfqpoint{5.817358in}{3.352022in}}%
\pgfpathlineto{\pgfqpoint{5.803031in}{3.339843in}}%
\pgfpathlineto{\pgfqpoint{5.788723in}{3.327820in}}%
\pgfpathlineto{\pgfqpoint{5.781831in}{3.326437in}}%
\pgfpathlineto{\pgfqpoint{5.774931in}{3.325058in}}%
\pgfpathlineto{\pgfqpoint{5.768025in}{3.323676in}}%
\pgfpathlineto{\pgfqpoint{5.761112in}{3.322288in}}%
\pgfpathclose%
\pgfusepath{fill}%
\end{pgfscope}%
\begin{pgfscope}%
\pgfpathrectangle{\pgfqpoint{1.254980in}{0.150000in}}{\pgfqpoint{5.490039in}{5.490039in}}%
\pgfusepath{clip}%
\pgfsetbuttcap%
\pgfsetroundjoin%
\definecolor{currentfill}{rgb}{0.288921,0.758394,0.428426}%
\pgfsetfillcolor{currentfill}%
\pgfsetfillopacity{0.700000}%
\pgfsetlinewidth{0.000000pt}%
\definecolor{currentstroke}{rgb}{0.000000,0.000000,0.000000}%
\pgfsetstrokecolor{currentstroke}%
\pgfsetdash{}{0pt}%
\pgfpathmoveto{\pgfqpoint{5.931010in}{3.429858in}}%
\pgfpathlineto{\pgfqpoint{5.945449in}{3.442608in}}%
\pgfpathlineto{\pgfqpoint{5.959910in}{3.455515in}}%
\pgfpathlineto{\pgfqpoint{5.974391in}{3.468580in}}%
\pgfpathlineto{\pgfqpoint{5.981164in}{3.468563in}}%
\pgfpathlineto{\pgfqpoint{5.987930in}{3.468564in}}%
\pgfpathlineto{\pgfqpoint{5.994690in}{3.468591in}}%
\pgfpathlineto{\pgfqpoint{6.001444in}{3.468649in}}%
\pgfpathlineto{\pgfqpoint{5.986998in}{3.456145in}}%
\pgfpathlineto{\pgfqpoint{5.972572in}{3.443798in}}%
\pgfpathlineto{\pgfqpoint{5.958166in}{3.431606in}}%
\pgfpathlineto{\pgfqpoint{5.951386in}{3.431121in}}%
\pgfpathlineto{\pgfqpoint{5.944600in}{3.430672in}}%
\pgfpathlineto{\pgfqpoint{5.937808in}{3.430253in}}%
\pgfpathlineto{\pgfqpoint{5.931010in}{3.429858in}}%
\pgfpathclose%
\pgfusepath{fill}%
\end{pgfscope}%
\begin{pgfscope}%
\pgfpathrectangle{\pgfqpoint{1.254980in}{0.150000in}}{\pgfqpoint{5.490039in}{5.490039in}}%
\pgfusepath{clip}%
\pgfsetbuttcap%
\pgfsetroundjoin%
\definecolor{currentfill}{rgb}{0.188923,0.410910,0.556326}%
\pgfsetfillcolor{currentfill}%
\pgfsetfillopacity{0.700000}%
\pgfsetlinewidth{0.000000pt}%
\definecolor{currentstroke}{rgb}{0.000000,0.000000,0.000000}%
\pgfsetstrokecolor{currentstroke}%
\pgfsetdash{}{0pt}%
\pgfpathmoveto{\pgfqpoint{2.290023in}{2.540902in}}%
\pgfpathlineto{\pgfqpoint{2.303691in}{2.518733in}}%
\pgfpathlineto{\pgfqpoint{2.317346in}{2.496858in}}%
\pgfpathlineto{\pgfqpoint{2.330987in}{2.475274in}}%
\pgfpathlineto{\pgfqpoint{2.344616in}{2.453977in}}%
\pgfpathlineto{\pgfqpoint{2.353368in}{2.451308in}}%
\pgfpathlineto{\pgfqpoint{2.362101in}{2.448928in}}%
\pgfpathlineto{\pgfqpoint{2.370815in}{2.446831in}}%
\pgfpathlineto{\pgfqpoint{2.379509in}{2.445012in}}%
\pgfpathlineto{\pgfqpoint{2.365932in}{2.465771in}}%
\pgfpathlineto{\pgfqpoint{2.352343in}{2.486816in}}%
\pgfpathlineto{\pgfqpoint{2.338741in}{2.508151in}}%
\pgfpathlineto{\pgfqpoint{2.325125in}{2.529777in}}%
\pgfpathlineto{\pgfqpoint{2.316380in}{2.532124in}}%
\pgfpathlineto{\pgfqpoint{2.307614in}{2.534756in}}%
\pgfpathlineto{\pgfqpoint{2.298829in}{2.537680in}}%
\pgfpathlineto{\pgfqpoint{2.290023in}{2.540902in}}%
\pgfpathclose%
\pgfusepath{fill}%
\end{pgfscope}%
\begin{pgfscope}%
\pgfpathrectangle{\pgfqpoint{1.254980in}{0.150000in}}{\pgfqpoint{5.490039in}{5.490039in}}%
\pgfusepath{clip}%
\pgfsetbuttcap%
\pgfsetroundjoin%
\definecolor{currentfill}{rgb}{0.271305,0.019942,0.347269}%
\pgfsetfillcolor{currentfill}%
\pgfsetfillopacity{0.700000}%
\pgfsetlinewidth{0.000000pt}%
\definecolor{currentstroke}{rgb}{0.000000,0.000000,0.000000}%
\pgfsetstrokecolor{currentstroke}%
\pgfsetdash{}{0pt}%
\pgfpathmoveto{\pgfqpoint{3.128098in}{1.673690in}}%
\pgfpathlineto{\pgfqpoint{3.141378in}{1.665992in}}%
\pgfpathlineto{\pgfqpoint{3.154659in}{1.658482in}}%
\pgfpathlineto{\pgfqpoint{3.167940in}{1.651161in}}%
\pgfpathlineto{\pgfqpoint{3.181222in}{1.644027in}}%
\pgfpathlineto{\pgfqpoint{3.189350in}{1.649885in}}%
\pgfpathlineto{\pgfqpoint{3.197467in}{1.655897in}}%
\pgfpathlineto{\pgfqpoint{3.205575in}{1.662059in}}%
\pgfpathlineto{\pgfqpoint{3.213674in}{1.668365in}}%
\pgfpathlineto{\pgfqpoint{3.200417in}{1.675030in}}%
\pgfpathlineto{\pgfqpoint{3.187160in}{1.681882in}}%
\pgfpathlineto{\pgfqpoint{3.173905in}{1.688922in}}%
\pgfpathlineto{\pgfqpoint{3.160650in}{1.696150in}}%
\pgfpathlineto{\pgfqpoint{3.152527in}{1.690303in}}%
\pgfpathlineto{\pgfqpoint{3.144394in}{1.684607in}}%
\pgfpathlineto{\pgfqpoint{3.136251in}{1.679068in}}%
\pgfpathlineto{\pgfqpoint{3.128098in}{1.673690in}}%
\pgfpathclose%
\pgfusepath{fill}%
\end{pgfscope}%
\begin{pgfscope}%
\pgfpathrectangle{\pgfqpoint{1.254980in}{0.150000in}}{\pgfqpoint{5.490039in}{5.490039in}}%
\pgfusepath{clip}%
\pgfsetbuttcap%
\pgfsetroundjoin%
\definecolor{currentfill}{rgb}{0.274128,0.199721,0.498911}%
\pgfsetfillcolor{currentfill}%
\pgfsetfillopacity{0.700000}%
\pgfsetlinewidth{0.000000pt}%
\definecolor{currentstroke}{rgb}{0.000000,0.000000,0.000000}%
\pgfsetstrokecolor{currentstroke}%
\pgfsetdash{}{0pt}%
\pgfpathmoveto{\pgfqpoint{4.082547in}{1.971326in}}%
\pgfpathlineto{\pgfqpoint{4.095972in}{1.975377in}}%
\pgfpathlineto{\pgfqpoint{4.109407in}{1.979592in}}%
\pgfpathlineto{\pgfqpoint{4.122853in}{1.983970in}}%
\pgfpathlineto{\pgfqpoint{4.136311in}{1.988511in}}%
\pgfpathlineto{\pgfqpoint{4.144048in}{2.000104in}}%
\pgfpathlineto{\pgfqpoint{4.151780in}{2.011643in}}%
\pgfpathlineto{\pgfqpoint{4.159508in}{2.023126in}}%
\pgfpathlineto{\pgfqpoint{4.167230in}{2.034553in}}%
\pgfpathlineto{\pgfqpoint{4.153777in}{2.029821in}}%
\pgfpathlineto{\pgfqpoint{4.140335in}{2.025253in}}%
\pgfpathlineto{\pgfqpoint{4.126903in}{2.020848in}}%
\pgfpathlineto{\pgfqpoint{4.113483in}{2.016607in}}%
\pgfpathlineto{\pgfqpoint{4.105756in}{2.005360in}}%
\pgfpathlineto{\pgfqpoint{4.098024in}{1.994063in}}%
\pgfpathlineto{\pgfqpoint{4.090288in}{1.982718in}}%
\pgfpathlineto{\pgfqpoint{4.082547in}{1.971326in}}%
\pgfpathclose%
\pgfusepath{fill}%
\end{pgfscope}%
\begin{pgfscope}%
\pgfpathrectangle{\pgfqpoint{1.254980in}{0.150000in}}{\pgfqpoint{5.490039in}{5.490039in}}%
\pgfusepath{clip}%
\pgfsetbuttcap%
\pgfsetroundjoin%
\definecolor{currentfill}{rgb}{0.121148,0.592739,0.544641}%
\pgfsetfillcolor{currentfill}%
\pgfsetfillopacity{0.700000}%
\pgfsetlinewidth{0.000000pt}%
\definecolor{currentstroke}{rgb}{0.000000,0.000000,0.000000}%
\pgfsetstrokecolor{currentstroke}%
\pgfsetdash{}{0pt}%
\pgfpathmoveto{\pgfqpoint{5.222281in}{2.944371in}}%
\pgfpathlineto{\pgfqpoint{5.236304in}{2.955713in}}%
\pgfpathlineto{\pgfqpoint{5.250346in}{2.967214in}}%
\pgfpathlineto{\pgfqpoint{5.264405in}{2.978875in}}%
\pgfpathlineto{\pgfqpoint{5.278483in}{2.990696in}}%
\pgfpathlineto{\pgfqpoint{5.285727in}{2.995576in}}%
\pgfpathlineto{\pgfqpoint{5.292963in}{3.000364in}}%
\pgfpathlineto{\pgfqpoint{5.300190in}{3.005064in}}%
\pgfpathlineto{\pgfqpoint{5.307409in}{3.009678in}}%
\pgfpathlineto{\pgfqpoint{5.293346in}{2.998141in}}%
\pgfpathlineto{\pgfqpoint{5.279302in}{2.986763in}}%
\pgfpathlineto{\pgfqpoint{5.265276in}{2.975545in}}%
\pgfpathlineto{\pgfqpoint{5.251268in}{2.964485in}}%
\pgfpathlineto{\pgfqpoint{5.244033in}{2.959577in}}%
\pgfpathlineto{\pgfqpoint{5.236790in}{2.954590in}}%
\pgfpathlineto{\pgfqpoint{5.229539in}{2.949523in}}%
\pgfpathlineto{\pgfqpoint{5.222281in}{2.944371in}}%
\pgfpathclose%
\pgfusepath{fill}%
\end{pgfscope}%
\begin{pgfscope}%
\pgfpathrectangle{\pgfqpoint{1.254980in}{0.150000in}}{\pgfqpoint{5.490039in}{5.490039in}}%
\pgfusepath{clip}%
\pgfsetbuttcap%
\pgfsetroundjoin%
\definecolor{currentfill}{rgb}{0.259857,0.745492,0.444467}%
\pgfsetfillcolor{currentfill}%
\pgfsetfillopacity{0.700000}%
\pgfsetlinewidth{0.000000pt}%
\definecolor{currentstroke}{rgb}{0.000000,0.000000,0.000000}%
\pgfsetstrokecolor{currentstroke}%
\pgfsetdash{}{0pt}%
\pgfpathmoveto{\pgfqpoint{5.846074in}{3.376852in}}%
\pgfpathlineto{\pgfqpoint{5.860463in}{3.389503in}}%
\pgfpathlineto{\pgfqpoint{5.874872in}{3.402312in}}%
\pgfpathlineto{\pgfqpoint{5.889302in}{3.415278in}}%
\pgfpathlineto{\pgfqpoint{5.903753in}{3.428402in}}%
\pgfpathlineto{\pgfqpoint{5.910578in}{3.428759in}}%
\pgfpathlineto{\pgfqpoint{5.917395in}{3.429117in}}%
\pgfpathlineto{\pgfqpoint{5.924206in}{3.429481in}}%
\pgfpathlineto{\pgfqpoint{5.931010in}{3.429858in}}%
\pgfpathlineto{\pgfqpoint{5.916591in}{3.417265in}}%
\pgfpathlineto{\pgfqpoint{5.902193in}{3.404829in}}%
\pgfpathlineto{\pgfqpoint{5.887816in}{3.392549in}}%
\pgfpathlineto{\pgfqpoint{5.873459in}{3.380426in}}%
\pgfpathlineto{\pgfqpoint{5.866622in}{3.379510in}}%
\pgfpathlineto{\pgfqpoint{5.859779in}{3.378613in}}%
\pgfpathlineto{\pgfqpoint{5.852930in}{3.377729in}}%
\pgfpathlineto{\pgfqpoint{5.846074in}{3.376852in}}%
\pgfpathclose%
\pgfusepath{fill}%
\end{pgfscope}%
\begin{pgfscope}%
\pgfpathrectangle{\pgfqpoint{1.254980in}{0.150000in}}{\pgfqpoint{5.490039in}{5.490039in}}%
\pgfusepath{clip}%
\pgfsetbuttcap%
\pgfsetroundjoin%
\definecolor{currentfill}{rgb}{0.233603,0.313828,0.543914}%
\pgfsetfillcolor{currentfill}%
\pgfsetfillopacity{0.700000}%
\pgfsetlinewidth{0.000000pt}%
\definecolor{currentstroke}{rgb}{0.000000,0.000000,0.000000}%
\pgfsetstrokecolor{currentstroke}%
\pgfsetdash{}{0pt}%
\pgfpathmoveto{\pgfqpoint{4.367478in}{2.213702in}}%
\pgfpathlineto{\pgfqpoint{4.381028in}{2.220298in}}%
\pgfpathlineto{\pgfqpoint{4.394591in}{2.227056in}}%
\pgfpathlineto{\pgfqpoint{4.408167in}{2.233975in}}%
\pgfpathlineto{\pgfqpoint{4.421756in}{2.241056in}}%
\pgfpathlineto{\pgfqpoint{4.429403in}{2.251900in}}%
\pgfpathlineto{\pgfqpoint{4.437045in}{2.262653in}}%
\pgfpathlineto{\pgfqpoint{4.444681in}{2.273313in}}%
\pgfpathlineto{\pgfqpoint{4.452312in}{2.283881in}}%
\pgfpathlineto{\pgfqpoint{4.438726in}{2.276723in}}%
\pgfpathlineto{\pgfqpoint{4.425154in}{2.269727in}}%
\pgfpathlineto{\pgfqpoint{4.411594in}{2.262893in}}%
\pgfpathlineto{\pgfqpoint{4.398047in}{2.256221in}}%
\pgfpathlineto{\pgfqpoint{4.390413in}{2.245719in}}%
\pgfpathlineto{\pgfqpoint{4.382773in}{2.235132in}}%
\pgfpathlineto{\pgfqpoint{4.375128in}{2.224460in}}%
\pgfpathlineto{\pgfqpoint{4.367478in}{2.213702in}}%
\pgfpathclose%
\pgfusepath{fill}%
\end{pgfscope}%
\begin{pgfscope}%
\pgfpathrectangle{\pgfqpoint{1.254980in}{0.150000in}}{\pgfqpoint{5.490039in}{5.490039in}}%
\pgfusepath{clip}%
\pgfsetbuttcap%
\pgfsetroundjoin%
\definecolor{currentfill}{rgb}{0.280267,0.073417,0.397163}%
\pgfsetfillcolor{currentfill}%
\pgfsetfillopacity{0.700000}%
\pgfsetlinewidth{0.000000pt}%
\definecolor{currentstroke}{rgb}{0.000000,0.000000,0.000000}%
\pgfsetstrokecolor{currentstroke}%
\pgfsetdash{}{0pt}%
\pgfpathmoveto{\pgfqpoint{2.935649in}{1.767377in}}%
\pgfpathlineto{\pgfqpoint{2.948968in}{1.756829in}}%
\pgfpathlineto{\pgfqpoint{2.962284in}{1.746483in}}%
\pgfpathlineto{\pgfqpoint{2.975598in}{1.736337in}}%
\pgfpathlineto{\pgfqpoint{2.988911in}{1.726392in}}%
\pgfpathlineto{\pgfqpoint{2.997166in}{1.730056in}}%
\pgfpathlineto{\pgfqpoint{3.005409in}{1.733914in}}%
\pgfpathlineto{\pgfqpoint{3.013641in}{1.737963in}}%
\pgfpathlineto{\pgfqpoint{3.021861in}{1.742197in}}%
\pgfpathlineto{\pgfqpoint{3.008578in}{1.751641in}}%
\pgfpathlineto{\pgfqpoint{2.995295in}{1.761284in}}%
\pgfpathlineto{\pgfqpoint{2.982011in}{1.771128in}}%
\pgfpathlineto{\pgfqpoint{2.968725in}{1.781173in}}%
\pgfpathlineto{\pgfqpoint{2.960474in}{1.777430in}}%
\pgfpathlineto{\pgfqpoint{2.952211in}{1.773880in}}%
\pgfpathlineto{\pgfqpoint{2.943937in}{1.770527in}}%
\pgfpathlineto{\pgfqpoint{2.935649in}{1.767377in}}%
\pgfpathclose%
\pgfusepath{fill}%
\end{pgfscope}%
\begin{pgfscope}%
\pgfpathrectangle{\pgfqpoint{1.254980in}{0.150000in}}{\pgfqpoint{5.490039in}{5.490039in}}%
\pgfusepath{clip}%
\pgfsetbuttcap%
\pgfsetroundjoin%
\definecolor{currentfill}{rgb}{0.268510,0.009605,0.335427}%
\pgfsetfillcolor{currentfill}%
\pgfsetfillopacity{0.700000}%
\pgfsetlinewidth{0.000000pt}%
\definecolor{currentstroke}{rgb}{0.000000,0.000000,0.000000}%
\pgfsetstrokecolor{currentstroke}%
\pgfsetdash{}{0pt}%
\pgfpathmoveto{\pgfqpoint{3.404990in}{1.634529in}}%
\pgfpathlineto{\pgfqpoint{3.418261in}{1.630661in}}%
\pgfpathlineto{\pgfqpoint{3.431536in}{1.626968in}}%
\pgfpathlineto{\pgfqpoint{3.444815in}{1.623451in}}%
\pgfpathlineto{\pgfqpoint{3.458099in}{1.620108in}}%
\pgfpathlineto{\pgfqpoint{3.466077in}{1.628762in}}%
\pgfpathlineto{\pgfqpoint{3.474049in}{1.637507in}}%
\pgfpathlineto{\pgfqpoint{3.482014in}{1.646340in}}%
\pgfpathlineto{\pgfqpoint{3.489972in}{1.655256in}}%
\pgfpathlineto{\pgfqpoint{3.476705in}{1.658187in}}%
\pgfpathlineto{\pgfqpoint{3.463443in}{1.661293in}}%
\pgfpathlineto{\pgfqpoint{3.450184in}{1.664575in}}%
\pgfpathlineto{\pgfqpoint{3.436930in}{1.668032in}}%
\pgfpathlineto{\pgfqpoint{3.428956in}{1.659517in}}%
\pgfpathlineto{\pgfqpoint{3.420974in}{1.651092in}}%
\pgfpathlineto{\pgfqpoint{3.412986in}{1.642762in}}%
\pgfpathlineto{\pgfqpoint{3.404990in}{1.634529in}}%
\pgfpathclose%
\pgfusepath{fill}%
\end{pgfscope}%
\begin{pgfscope}%
\pgfpathrectangle{\pgfqpoint{1.254980in}{0.150000in}}{\pgfqpoint{5.490039in}{5.490039in}}%
\pgfusepath{clip}%
\pgfsetbuttcap%
\pgfsetroundjoin%
\definecolor{currentfill}{rgb}{0.187231,0.414746,0.556547}%
\pgfsetfillcolor{currentfill}%
\pgfsetfillopacity{0.700000}%
\pgfsetlinewidth{0.000000pt}%
\definecolor{currentstroke}{rgb}{0.000000,0.000000,0.000000}%
\pgfsetstrokecolor{currentstroke}%
\pgfsetdash{}{0pt}%
\pgfpathmoveto{\pgfqpoint{4.652521in}{2.466958in}}%
\pgfpathlineto{\pgfqpoint{4.666221in}{2.475622in}}%
\pgfpathlineto{\pgfqpoint{4.679936in}{2.484448in}}%
\pgfpathlineto{\pgfqpoint{4.693665in}{2.493434in}}%
\pgfpathlineto{\pgfqpoint{4.707410in}{2.502581in}}%
\pgfpathlineto{\pgfqpoint{4.714951in}{2.511860in}}%
\pgfpathlineto{\pgfqpoint{4.722485in}{2.521027in}}%
\pgfpathlineto{\pgfqpoint{4.730013in}{2.530084in}}%
\pgfpathlineto{\pgfqpoint{4.737534in}{2.539031in}}%
\pgfpathlineto{\pgfqpoint{4.723794in}{2.529925in}}%
\pgfpathlineto{\pgfqpoint{4.710070in}{2.520979in}}%
\pgfpathlineto{\pgfqpoint{4.696360in}{2.512194in}}%
\pgfpathlineto{\pgfqpoint{4.682666in}{2.503570in}}%
\pgfpathlineto{\pgfqpoint{4.675139in}{2.494572in}}%
\pgfpathlineto{\pgfqpoint{4.667606in}{2.485471in}}%
\pgfpathlineto{\pgfqpoint{4.660067in}{2.476266in}}%
\pgfpathlineto{\pgfqpoint{4.652521in}{2.466958in}}%
\pgfpathclose%
\pgfusepath{fill}%
\end{pgfscope}%
\begin{pgfscope}%
\pgfpathrectangle{\pgfqpoint{1.254980in}{0.150000in}}{\pgfqpoint{5.490039in}{5.490039in}}%
\pgfusepath{clip}%
\pgfsetbuttcap%
\pgfsetroundjoin%
\definecolor{currentfill}{rgb}{0.149039,0.508051,0.557250}%
\pgfsetfillcolor{currentfill}%
\pgfsetfillopacity{0.700000}%
\pgfsetlinewidth{0.000000pt}%
\definecolor{currentstroke}{rgb}{0.000000,0.000000,0.000000}%
\pgfsetstrokecolor{currentstroke}%
\pgfsetdash{}{0pt}%
\pgfpathmoveto{\pgfqpoint{4.937576in}{2.714493in}}%
\pgfpathlineto{\pgfqpoint{4.951438in}{2.724743in}}%
\pgfpathlineto{\pgfqpoint{4.965316in}{2.735153in}}%
\pgfpathlineto{\pgfqpoint{4.979210in}{2.745723in}}%
\pgfpathlineto{\pgfqpoint{4.993122in}{2.756454in}}%
\pgfpathlineto{\pgfqpoint{5.000529in}{2.763643in}}%
\pgfpathlineto{\pgfqpoint{5.007929in}{2.770720in}}%
\pgfpathlineto{\pgfqpoint{5.015321in}{2.777688in}}%
\pgfpathlineto{\pgfqpoint{5.022705in}{2.784548in}}%
\pgfpathlineto{\pgfqpoint{5.008803in}{2.773979in}}%
\pgfpathlineto{\pgfqpoint{4.994917in}{2.763570in}}%
\pgfpathlineto{\pgfqpoint{4.981049in}{2.753321in}}%
\pgfpathlineto{\pgfqpoint{4.967196in}{2.743231in}}%
\pgfpathlineto{\pgfqpoint{4.959802in}{2.736199in}}%
\pgfpathlineto{\pgfqpoint{4.952401in}{2.729066in}}%
\pgfpathlineto{\pgfqpoint{4.944992in}{2.721832in}}%
\pgfpathlineto{\pgfqpoint{4.937576in}{2.714493in}}%
\pgfpathclose%
\pgfusepath{fill}%
\end{pgfscope}%
\begin{pgfscope}%
\pgfpathrectangle{\pgfqpoint{1.254980in}{0.150000in}}{\pgfqpoint{5.490039in}{5.490039in}}%
\pgfusepath{clip}%
\pgfsetbuttcap%
\pgfsetroundjoin%
\definecolor{currentfill}{rgb}{0.265145,0.232956,0.516599}%
\pgfsetfillcolor{currentfill}%
\pgfsetfillopacity{0.700000}%
\pgfsetlinewidth{0.000000pt}%
\definecolor{currentstroke}{rgb}{0.000000,0.000000,0.000000}%
\pgfsetstrokecolor{currentstroke}%
\pgfsetdash{}{0pt}%
\pgfpathmoveto{\pgfqpoint{4.167230in}{2.034553in}}%
\pgfpathlineto{\pgfqpoint{4.180695in}{2.039447in}}%
\pgfpathlineto{\pgfqpoint{4.194171in}{2.044505in}}%
\pgfpathlineto{\pgfqpoint{4.207658in}{2.049725in}}%
\pgfpathlineto{\pgfqpoint{4.221156in}{2.055107in}}%
\pgfpathlineto{\pgfqpoint{4.228871in}{2.066649in}}%
\pgfpathlineto{\pgfqpoint{4.236580in}{2.078123in}}%
\pgfpathlineto{\pgfqpoint{4.244285in}{2.089530in}}%
\pgfpathlineto{\pgfqpoint{4.251985in}{2.100866in}}%
\pgfpathlineto{\pgfqpoint{4.238490in}{2.095321in}}%
\pgfpathlineto{\pgfqpoint{4.225006in}{2.089939in}}%
\pgfpathlineto{\pgfqpoint{4.211534in}{2.084719in}}%
\pgfpathlineto{\pgfqpoint{4.198073in}{2.079662in}}%
\pgfpathlineto{\pgfqpoint{4.190370in}{2.068477in}}%
\pgfpathlineto{\pgfqpoint{4.182661in}{2.057229in}}%
\pgfpathlineto{\pgfqpoint{4.174948in}{2.045921in}}%
\pgfpathlineto{\pgfqpoint{4.167230in}{2.034553in}}%
\pgfpathclose%
\pgfusepath{fill}%
\end{pgfscope}%
\begin{pgfscope}%
\pgfpathrectangle{\pgfqpoint{1.254980in}{0.150000in}}{\pgfqpoint{5.490039in}{5.490039in}}%
\pgfusepath{clip}%
\pgfsetbuttcap%
\pgfsetroundjoin%
\definecolor{currentfill}{rgb}{0.119699,0.618490,0.536347}%
\pgfsetfillcolor{currentfill}%
\pgfsetfillopacity{0.700000}%
\pgfsetlinewidth{0.000000pt}%
\definecolor{currentstroke}{rgb}{0.000000,0.000000,0.000000}%
\pgfsetstrokecolor{currentstroke}%
\pgfsetdash{}{0pt}%
\pgfpathmoveto{\pgfqpoint{5.307409in}{3.009678in}}%
\pgfpathlineto{\pgfqpoint{5.321490in}{3.021375in}}%
\pgfpathlineto{\pgfqpoint{5.335589in}{3.033230in}}%
\pgfpathlineto{\pgfqpoint{5.349707in}{3.045245in}}%
\pgfpathlineto{\pgfqpoint{5.363844in}{3.057420in}}%
\pgfpathlineto{\pgfqpoint{5.371039in}{3.061649in}}%
\pgfpathlineto{\pgfqpoint{5.378225in}{3.065793in}}%
\pgfpathlineto{\pgfqpoint{5.385403in}{3.069855in}}%
\pgfpathlineto{\pgfqpoint{5.392572in}{3.073838in}}%
\pgfpathlineto{\pgfqpoint{5.378453in}{3.061979in}}%
\pgfpathlineto{\pgfqpoint{5.364352in}{3.050278in}}%
\pgfpathlineto{\pgfqpoint{5.350269in}{3.038736in}}%
\pgfpathlineto{\pgfqpoint{5.336205in}{3.027353in}}%
\pgfpathlineto{\pgfqpoint{5.329018in}{3.023045in}}%
\pgfpathlineto{\pgfqpoint{5.321823in}{3.018665in}}%
\pgfpathlineto{\pgfqpoint{5.314620in}{3.014211in}}%
\pgfpathlineto{\pgfqpoint{5.307409in}{3.009678in}}%
\pgfpathclose%
\pgfusepath{fill}%
\end{pgfscope}%
\begin{pgfscope}%
\pgfpathrectangle{\pgfqpoint{1.254980in}{0.150000in}}{\pgfqpoint{5.490039in}{5.490039in}}%
\pgfusepath{clip}%
\pgfsetbuttcap%
\pgfsetroundjoin%
\definecolor{currentfill}{rgb}{0.172719,0.448791,0.557885}%
\pgfsetfillcolor{currentfill}%
\pgfsetfillopacity{0.700000}%
\pgfsetlinewidth{0.000000pt}%
\definecolor{currentstroke}{rgb}{0.000000,0.000000,0.000000}%
\pgfsetstrokecolor{currentstroke}%
\pgfsetdash{}{0pt}%
\pgfpathmoveto{\pgfqpoint{2.235209in}{2.632566in}}%
\pgfpathlineto{\pgfqpoint{2.248934in}{2.609195in}}%
\pgfpathlineto{\pgfqpoint{2.262645in}{2.586129in}}%
\pgfpathlineto{\pgfqpoint{2.276341in}{2.563366in}}%
\pgfpathlineto{\pgfqpoint{2.290023in}{2.540902in}}%
\pgfpathlineto{\pgfqpoint{2.298829in}{2.537680in}}%
\pgfpathlineto{\pgfqpoint{2.307614in}{2.534756in}}%
\pgfpathlineto{\pgfqpoint{2.316380in}{2.532124in}}%
\pgfpathlineto{\pgfqpoint{2.325125in}{2.529777in}}%
\pgfpathlineto{\pgfqpoint{2.311497in}{2.551699in}}%
\pgfpathlineto{\pgfqpoint{2.297855in}{2.573918in}}%
\pgfpathlineto{\pgfqpoint{2.284199in}{2.596437in}}%
\pgfpathlineto{\pgfqpoint{2.270528in}{2.619261in}}%
\pgfpathlineto{\pgfqpoint{2.261730in}{2.622140in}}%
\pgfpathlineto{\pgfqpoint{2.252911in}{2.625313in}}%
\pgfpathlineto{\pgfqpoint{2.244070in}{2.628787in}}%
\pgfpathlineto{\pgfqpoint{2.235209in}{2.632566in}}%
\pgfpathclose%
\pgfusepath{fill}%
\end{pgfscope}%
\begin{pgfscope}%
\pgfpathrectangle{\pgfqpoint{1.254980in}{0.150000in}}{\pgfqpoint{5.490039in}{5.490039in}}%
\pgfusepath{clip}%
\pgfsetbuttcap%
\pgfsetroundjoin%
\definecolor{currentfill}{rgb}{0.279566,0.067836,0.391917}%
\pgfsetfillcolor{currentfill}%
\pgfsetfillopacity{0.700000}%
\pgfsetlinewidth{0.000000pt}%
\definecolor{currentstroke}{rgb}{0.000000,0.000000,0.000000}%
\pgfsetstrokecolor{currentstroke}%
\pgfsetdash{}{0pt}%
\pgfpathmoveto{\pgfqpoint{3.712715in}{1.716289in}}%
\pgfpathlineto{\pgfqpoint{3.726033in}{1.716322in}}%
\pgfpathlineto{\pgfqpoint{3.739358in}{1.716523in}}%
\pgfpathlineto{\pgfqpoint{3.752690in}{1.716892in}}%
\pgfpathlineto{\pgfqpoint{3.766030in}{1.717427in}}%
\pgfpathlineto{\pgfqpoint{3.773887in}{1.728288in}}%
\pgfpathlineto{\pgfqpoint{3.781738in}{1.739170in}}%
\pgfpathlineto{\pgfqpoint{3.789585in}{1.750070in}}%
\pgfpathlineto{\pgfqpoint{3.797427in}{1.760985in}}%
\pgfpathlineto{\pgfqpoint{3.784096in}{1.760120in}}%
\pgfpathlineto{\pgfqpoint{3.770773in}{1.759423in}}%
\pgfpathlineto{\pgfqpoint{3.757457in}{1.758893in}}%
\pgfpathlineto{\pgfqpoint{3.744149in}{1.758532in}}%
\pgfpathlineto{\pgfqpoint{3.736298in}{1.747935in}}%
\pgfpathlineto{\pgfqpoint{3.728442in}{1.737360in}}%
\pgfpathlineto{\pgfqpoint{3.720581in}{1.726810in}}%
\pgfpathlineto{\pgfqpoint{3.712715in}{1.716289in}}%
\pgfpathclose%
\pgfusepath{fill}%
\end{pgfscope}%
\begin{pgfscope}%
\pgfpathrectangle{\pgfqpoint{1.254980in}{0.150000in}}{\pgfqpoint{5.490039in}{5.490039in}}%
\pgfusepath{clip}%
\pgfsetbuttcap%
\pgfsetroundjoin%
\definecolor{currentfill}{rgb}{0.277941,0.056324,0.381191}%
\pgfsetfillcolor{currentfill}%
\pgfsetfillopacity{0.700000}%
\pgfsetlinewidth{0.000000pt}%
\definecolor{currentstroke}{rgb}{0.000000,0.000000,0.000000}%
\pgfsetstrokecolor{currentstroke}%
\pgfsetdash{}{0pt}%
\pgfpathmoveto{\pgfqpoint{2.988911in}{1.726392in}}%
\pgfpathlineto{\pgfqpoint{3.002223in}{1.716645in}}%
\pgfpathlineto{\pgfqpoint{3.015533in}{1.707096in}}%
\pgfpathlineto{\pgfqpoint{3.028843in}{1.697742in}}%
\pgfpathlineto{\pgfqpoint{3.042152in}{1.688584in}}%
\pgfpathlineto{\pgfqpoint{3.050376in}{1.692760in}}%
\pgfpathlineto{\pgfqpoint{3.058589in}{1.697123in}}%
\pgfpathlineto{\pgfqpoint{3.066790in}{1.701669in}}%
\pgfpathlineto{\pgfqpoint{3.074981in}{1.706392in}}%
\pgfpathlineto{\pgfqpoint{3.061702in}{1.715050in}}%
\pgfpathlineto{\pgfqpoint{3.048422in}{1.723903in}}%
\pgfpathlineto{\pgfqpoint{3.035142in}{1.732951in}}%
\pgfpathlineto{\pgfqpoint{3.021861in}{1.742197in}}%
\pgfpathlineto{\pgfqpoint{3.013641in}{1.737963in}}%
\pgfpathlineto{\pgfqpoint{3.005409in}{1.733914in}}%
\pgfpathlineto{\pgfqpoint{2.997166in}{1.730056in}}%
\pgfpathlineto{\pgfqpoint{2.988911in}{1.726392in}}%
\pgfpathclose%
\pgfusepath{fill}%
\end{pgfscope}%
\begin{pgfscope}%
\pgfpathrectangle{\pgfqpoint{1.254980in}{0.150000in}}{\pgfqpoint{5.490039in}{5.490039in}}%
\pgfusepath{clip}%
\pgfsetbuttcap%
\pgfsetroundjoin%
\definecolor{currentfill}{rgb}{0.282327,0.094955,0.417331}%
\pgfsetfillcolor{currentfill}%
\pgfsetfillopacity{0.700000}%
\pgfsetlinewidth{0.000000pt}%
\definecolor{currentstroke}{rgb}{0.000000,0.000000,0.000000}%
\pgfsetstrokecolor{currentstroke}%
\pgfsetdash{}{0pt}%
\pgfpathmoveto{\pgfqpoint{3.797427in}{1.760985in}}%
\pgfpathlineto{\pgfqpoint{3.810765in}{1.762016in}}%
\pgfpathlineto{\pgfqpoint{3.824112in}{1.763214in}}%
\pgfpathlineto{\pgfqpoint{3.837467in}{1.764578in}}%
\pgfpathlineto{\pgfqpoint{3.850830in}{1.766108in}}%
\pgfpathlineto{\pgfqpoint{3.858659in}{1.777347in}}%
\pgfpathlineto{\pgfqpoint{3.866483in}{1.788589in}}%
\pgfpathlineto{\pgfqpoint{3.874302in}{1.799830in}}%
\pgfpathlineto{\pgfqpoint{3.882116in}{1.811069in}}%
\pgfpathlineto{\pgfqpoint{3.868760in}{1.809238in}}%
\pgfpathlineto{\pgfqpoint{3.855412in}{1.807573in}}%
\pgfpathlineto{\pgfqpoint{3.842073in}{1.806073in}}%
\pgfpathlineto{\pgfqpoint{3.828743in}{1.804741in}}%
\pgfpathlineto{\pgfqpoint{3.820921in}{1.793793in}}%
\pgfpathlineto{\pgfqpoint{3.813095in}{1.782849in}}%
\pgfpathlineto{\pgfqpoint{3.805263in}{1.771912in}}%
\pgfpathlineto{\pgfqpoint{3.797427in}{1.760985in}}%
\pgfpathclose%
\pgfusepath{fill}%
\end{pgfscope}%
\begin{pgfscope}%
\pgfpathrectangle{\pgfqpoint{1.254980in}{0.150000in}}{\pgfqpoint{5.490039in}{5.490039in}}%
\pgfusepath{clip}%
\pgfsetbuttcap%
\pgfsetroundjoin%
\definecolor{currentfill}{rgb}{0.218130,0.347432,0.550038}%
\pgfsetfillcolor{currentfill}%
\pgfsetfillopacity{0.700000}%
\pgfsetlinewidth{0.000000pt}%
\definecolor{currentstroke}{rgb}{0.000000,0.000000,0.000000}%
\pgfsetstrokecolor{currentstroke}%
\pgfsetdash{}{0pt}%
\pgfpathmoveto{\pgfqpoint{4.452312in}{2.283881in}}%
\pgfpathlineto{\pgfqpoint{4.465911in}{2.291200in}}%
\pgfpathlineto{\pgfqpoint{4.479524in}{2.298681in}}%
\pgfpathlineto{\pgfqpoint{4.493150in}{2.306323in}}%
\pgfpathlineto{\pgfqpoint{4.506790in}{2.314127in}}%
\pgfpathlineto{\pgfqpoint{4.514412in}{2.324660in}}%
\pgfpathlineto{\pgfqpoint{4.522028in}{2.335092in}}%
\pgfpathlineto{\pgfqpoint{4.529639in}{2.345425in}}%
\pgfpathlineto{\pgfqpoint{4.537243in}{2.355656in}}%
\pgfpathlineto{\pgfqpoint{4.523607in}{2.347805in}}%
\pgfpathlineto{\pgfqpoint{4.509984in}{2.340115in}}%
\pgfpathlineto{\pgfqpoint{4.496375in}{2.332587in}}%
\pgfpathlineto{\pgfqpoint{4.482779in}{2.325220in}}%
\pgfpathlineto{\pgfqpoint{4.475171in}{2.315025in}}%
\pgfpathlineto{\pgfqpoint{4.467557in}{2.304737in}}%
\pgfpathlineto{\pgfqpoint{4.459937in}{2.294355in}}%
\pgfpathlineto{\pgfqpoint{4.452312in}{2.283881in}}%
\pgfpathclose%
\pgfusepath{fill}%
\end{pgfscope}%
\begin{pgfscope}%
\pgfpathrectangle{\pgfqpoint{1.254980in}{0.150000in}}{\pgfqpoint{5.490039in}{5.490039in}}%
\pgfusepath{clip}%
\pgfsetbuttcap%
\pgfsetroundjoin%
\definecolor{currentfill}{rgb}{0.276022,0.044167,0.370164}%
\pgfsetfillcolor{currentfill}%
\pgfsetfillopacity{0.700000}%
\pgfsetlinewidth{0.000000pt}%
\definecolor{currentstroke}{rgb}{0.000000,0.000000,0.000000}%
\pgfsetstrokecolor{currentstroke}%
\pgfsetdash{}{0pt}%
\pgfpathmoveto{\pgfqpoint{3.627948in}{1.677527in}}%
\pgfpathlineto{\pgfqpoint{3.641250in}{1.676529in}}%
\pgfpathlineto{\pgfqpoint{3.654558in}{1.675699in}}%
\pgfpathlineto{\pgfqpoint{3.667873in}{1.675039in}}%
\pgfpathlineto{\pgfqpoint{3.681195in}{1.674548in}}%
\pgfpathlineto{\pgfqpoint{3.689083in}{1.684925in}}%
\pgfpathlineto{\pgfqpoint{3.696966in}{1.695343in}}%
\pgfpathlineto{\pgfqpoint{3.704843in}{1.705799in}}%
\pgfpathlineto{\pgfqpoint{3.712715in}{1.716289in}}%
\pgfpathlineto{\pgfqpoint{3.699404in}{1.716424in}}%
\pgfpathlineto{\pgfqpoint{3.686100in}{1.716728in}}%
\pgfpathlineto{\pgfqpoint{3.672803in}{1.717201in}}%
\pgfpathlineto{\pgfqpoint{3.659512in}{1.717844in}}%
\pgfpathlineto{\pgfqpoint{3.651630in}{1.707699in}}%
\pgfpathlineto{\pgfqpoint{3.643741in}{1.697596in}}%
\pgfpathlineto{\pgfqpoint{3.635848in}{1.687538in}}%
\pgfpathlineto{\pgfqpoint{3.627948in}{1.677527in}}%
\pgfpathclose%
\pgfusepath{fill}%
\end{pgfscope}%
\begin{pgfscope}%
\pgfpathrectangle{\pgfqpoint{1.254980in}{0.150000in}}{\pgfqpoint{5.490039in}{5.490039in}}%
\pgfusepath{clip}%
\pgfsetbuttcap%
\pgfsetroundjoin%
\definecolor{currentfill}{rgb}{0.269944,0.014625,0.341379}%
\pgfsetfillcolor{currentfill}%
\pgfsetfillopacity{0.700000}%
\pgfsetlinewidth{0.000000pt}%
\definecolor{currentstroke}{rgb}{0.000000,0.000000,0.000000}%
\pgfsetstrokecolor{currentstroke}%
\pgfsetdash{}{0pt}%
\pgfpathmoveto{\pgfqpoint{3.181222in}{1.644027in}}%
\pgfpathlineto{\pgfqpoint{3.194506in}{1.637079in}}%
\pgfpathlineto{\pgfqpoint{3.207790in}{1.630316in}}%
\pgfpathlineto{\pgfqpoint{3.221077in}{1.623738in}}%
\pgfpathlineto{\pgfqpoint{3.234364in}{1.617343in}}%
\pgfpathlineto{\pgfqpoint{3.242467in}{1.623681in}}%
\pgfpathlineto{\pgfqpoint{3.250561in}{1.630166in}}%
\pgfpathlineto{\pgfqpoint{3.258646in}{1.636793in}}%
\pgfpathlineto{\pgfqpoint{3.266722in}{1.643557in}}%
\pgfpathlineto{\pgfqpoint{3.253458in}{1.649483in}}%
\pgfpathlineto{\pgfqpoint{3.240195in}{1.655592in}}%
\pgfpathlineto{\pgfqpoint{3.226934in}{1.661886in}}%
\pgfpathlineto{\pgfqpoint{3.213674in}{1.668365in}}%
\pgfpathlineto{\pgfqpoint{3.205575in}{1.662059in}}%
\pgfpathlineto{\pgfqpoint{3.197467in}{1.655897in}}%
\pgfpathlineto{\pgfqpoint{3.189350in}{1.649885in}}%
\pgfpathlineto{\pgfqpoint{3.181222in}{1.644027in}}%
\pgfpathclose%
\pgfusepath{fill}%
\end{pgfscope}%
\begin{pgfscope}%
\pgfpathrectangle{\pgfqpoint{1.254980in}{0.150000in}}{\pgfqpoint{5.490039in}{5.490039in}}%
\pgfusepath{clip}%
\pgfsetbuttcap%
\pgfsetroundjoin%
\definecolor{currentfill}{rgb}{0.267004,0.004874,0.329415}%
\pgfsetfillcolor{currentfill}%
\pgfsetfillopacity{0.700000}%
\pgfsetlinewidth{0.000000pt}%
\definecolor{currentstroke}{rgb}{0.000000,0.000000,0.000000}%
\pgfsetstrokecolor{currentstroke}%
\pgfsetdash{}{0pt}%
\pgfpathmoveto{\pgfqpoint{3.319804in}{1.621673in}}%
\pgfpathlineto{\pgfqpoint{3.333081in}{1.616653in}}%
\pgfpathlineto{\pgfqpoint{3.346361in}{1.611813in}}%
\pgfpathlineto{\pgfqpoint{3.359644in}{1.607150in}}%
\pgfpathlineto{\pgfqpoint{3.372931in}{1.602665in}}%
\pgfpathlineto{\pgfqpoint{3.380957in}{1.610463in}}%
\pgfpathlineto{\pgfqpoint{3.388975in}{1.618376in}}%
\pgfpathlineto{\pgfqpoint{3.396986in}{1.626400in}}%
\pgfpathlineto{\pgfqpoint{3.404990in}{1.634529in}}%
\pgfpathlineto{\pgfqpoint{3.391722in}{1.638575in}}%
\pgfpathlineto{\pgfqpoint{3.378458in}{1.642798in}}%
\pgfpathlineto{\pgfqpoint{3.365198in}{1.647200in}}%
\pgfpathlineto{\pgfqpoint{3.351940in}{1.651780in}}%
\pgfpathlineto{\pgfqpoint{3.343918in}{1.644079in}}%
\pgfpathlineto{\pgfqpoint{3.335888in}{1.636491in}}%
\pgfpathlineto{\pgfqpoint{3.327850in}{1.629022in}}%
\pgfpathlineto{\pgfqpoint{3.319804in}{1.621673in}}%
\pgfpathclose%
\pgfusepath{fill}%
\end{pgfscope}%
\begin{pgfscope}%
\pgfpathrectangle{\pgfqpoint{1.254980in}{0.150000in}}{\pgfqpoint{5.490039in}{5.490039in}}%
\pgfusepath{clip}%
\pgfsetbuttcap%
\pgfsetroundjoin%
\definecolor{currentfill}{rgb}{0.283229,0.120777,0.440584}%
\pgfsetfillcolor{currentfill}%
\pgfsetfillopacity{0.700000}%
\pgfsetlinewidth{0.000000pt}%
\definecolor{currentstroke}{rgb}{0.000000,0.000000,0.000000}%
\pgfsetstrokecolor{currentstroke}%
\pgfsetdash{}{0pt}%
\pgfpathmoveto{\pgfqpoint{3.882116in}{1.811069in}}%
\pgfpathlineto{\pgfqpoint{3.895480in}{1.813066in}}%
\pgfpathlineto{\pgfqpoint{3.908853in}{1.815228in}}%
\pgfpathlineto{\pgfqpoint{3.922236in}{1.817555in}}%
\pgfpathlineto{\pgfqpoint{3.935627in}{1.820047in}}%
\pgfpathlineto{\pgfqpoint{3.943430in}{1.831564in}}%
\pgfpathlineto{\pgfqpoint{3.951228in}{1.843067in}}%
\pgfpathlineto{\pgfqpoint{3.959021in}{1.854552in}}%
\pgfpathlineto{\pgfqpoint{3.966810in}{1.866017in}}%
\pgfpathlineto{\pgfqpoint{3.953425in}{1.863252in}}%
\pgfpathlineto{\pgfqpoint{3.940048in}{1.860650in}}%
\pgfpathlineto{\pgfqpoint{3.926681in}{1.858215in}}%
\pgfpathlineto{\pgfqpoint{3.913323in}{1.855944in}}%
\pgfpathlineto{\pgfqpoint{3.905529in}{1.844742in}}%
\pgfpathlineto{\pgfqpoint{3.897729in}{1.833528in}}%
\pgfpathlineto{\pgfqpoint{3.889925in}{1.822302in}}%
\pgfpathlineto{\pgfqpoint{3.882116in}{1.811069in}}%
\pgfpathclose%
\pgfusepath{fill}%
\end{pgfscope}%
\begin{pgfscope}%
\pgfpathrectangle{\pgfqpoint{1.254980in}{0.150000in}}{\pgfqpoint{5.490039in}{5.490039in}}%
\pgfusepath{clip}%
\pgfsetbuttcap%
\pgfsetroundjoin%
\definecolor{currentfill}{rgb}{0.126326,0.644107,0.525311}%
\pgfsetfillcolor{currentfill}%
\pgfsetfillopacity{0.700000}%
\pgfsetlinewidth{0.000000pt}%
\definecolor{currentstroke}{rgb}{0.000000,0.000000,0.000000}%
\pgfsetstrokecolor{currentstroke}%
\pgfsetdash{}{0pt}%
\pgfpathmoveto{\pgfqpoint{5.392572in}{3.073838in}}%
\pgfpathlineto{\pgfqpoint{5.406711in}{3.085857in}}%
\pgfpathlineto{\pgfqpoint{5.420868in}{3.098035in}}%
\pgfpathlineto{\pgfqpoint{5.435045in}{3.110373in}}%
\pgfpathlineto{\pgfqpoint{5.449241in}{3.122870in}}%
\pgfpathlineto{\pgfqpoint{5.456383in}{3.126444in}}%
\pgfpathlineto{\pgfqpoint{5.463518in}{3.129940in}}%
\pgfpathlineto{\pgfqpoint{5.470644in}{3.133362in}}%
\pgfpathlineto{\pgfqpoint{5.477762in}{3.136714in}}%
\pgfpathlineto{\pgfqpoint{5.463585in}{3.124563in}}%
\pgfpathlineto{\pgfqpoint{5.449428in}{3.112572in}}%
\pgfpathlineto{\pgfqpoint{5.435289in}{3.100739in}}%
\pgfpathlineto{\pgfqpoint{5.421170in}{3.089064in}}%
\pgfpathlineto{\pgfqpoint{5.414032in}{3.085356in}}%
\pgfpathlineto{\pgfqpoint{5.406887in}{3.081585in}}%
\pgfpathlineto{\pgfqpoint{5.399734in}{3.077747in}}%
\pgfpathlineto{\pgfqpoint{5.392572in}{3.073838in}}%
\pgfpathclose%
\pgfusepath{fill}%
\end{pgfscope}%
\begin{pgfscope}%
\pgfpathrectangle{\pgfqpoint{1.254980in}{0.150000in}}{\pgfqpoint{5.490039in}{5.490039in}}%
\pgfusepath{clip}%
\pgfsetbuttcap%
\pgfsetroundjoin%
\definecolor{currentfill}{rgb}{0.266580,0.228262,0.514349}%
\pgfsetfillcolor{currentfill}%
\pgfsetfillopacity{0.700000}%
\pgfsetlinewidth{0.000000pt}%
\definecolor{currentstroke}{rgb}{0.000000,0.000000,0.000000}%
\pgfsetstrokecolor{currentstroke}%
\pgfsetdash{}{0pt}%
\pgfpathmoveto{\pgfqpoint{2.580708in}{2.087102in}}%
\pgfpathlineto{\pgfqpoint{2.594176in}{2.070740in}}%
\pgfpathlineto{\pgfqpoint{2.607637in}{2.054617in}}%
\pgfpathlineto{\pgfqpoint{2.621091in}{2.038731in}}%
\pgfpathlineto{\pgfqpoint{2.634538in}{2.023079in}}%
\pgfpathlineto{\pgfqpoint{2.643076in}{2.022617in}}%
\pgfpathlineto{\pgfqpoint{2.651596in}{2.022417in}}%
\pgfpathlineto{\pgfqpoint{2.660101in}{2.022473in}}%
\pgfpathlineto{\pgfqpoint{2.668589in}{2.022780in}}%
\pgfpathlineto{\pgfqpoint{2.655185in}{2.037885in}}%
\pgfpathlineto{\pgfqpoint{2.641775in}{2.053225in}}%
\pgfpathlineto{\pgfqpoint{2.628358in}{2.068800in}}%
\pgfpathlineto{\pgfqpoint{2.614934in}{2.084613in}}%
\pgfpathlineto{\pgfqpoint{2.606403in}{2.084842in}}%
\pgfpathlineto{\pgfqpoint{2.597855in}{2.085329in}}%
\pgfpathlineto{\pgfqpoint{2.589290in}{2.086081in}}%
\pgfpathlineto{\pgfqpoint{2.580708in}{2.087102in}}%
\pgfpathclose%
\pgfusepath{fill}%
\end{pgfscope}%
\begin{pgfscope}%
\pgfpathrectangle{\pgfqpoint{1.254980in}{0.150000in}}{\pgfqpoint{5.490039in}{5.490039in}}%
\pgfusepath{clip}%
\pgfsetbuttcap%
\pgfsetroundjoin%
\definecolor{currentfill}{rgb}{0.272594,0.025563,0.353093}%
\pgfsetfillcolor{currentfill}%
\pgfsetfillopacity{0.700000}%
\pgfsetlinewidth{0.000000pt}%
\definecolor{currentstroke}{rgb}{0.000000,0.000000,0.000000}%
\pgfsetstrokecolor{currentstroke}%
\pgfsetdash{}{0pt}%
\pgfpathmoveto{\pgfqpoint{3.543089in}{1.645269in}}%
\pgfpathlineto{\pgfqpoint{3.556381in}{1.643204in}}%
\pgfpathlineto{\pgfqpoint{3.569679in}{1.641310in}}%
\pgfpathlineto{\pgfqpoint{3.582982in}{1.639587in}}%
\pgfpathlineto{\pgfqpoint{3.596290in}{1.638034in}}%
\pgfpathlineto{\pgfqpoint{3.604214in}{1.647818in}}%
\pgfpathlineto{\pgfqpoint{3.612131in}{1.657664in}}%
\pgfpathlineto{\pgfqpoint{3.620042in}{1.667568in}}%
\pgfpathlineto{\pgfqpoint{3.627948in}{1.677527in}}%
\pgfpathlineto{\pgfqpoint{3.614652in}{1.678696in}}%
\pgfpathlineto{\pgfqpoint{3.601362in}{1.680035in}}%
\pgfpathlineto{\pgfqpoint{3.588078in}{1.681545in}}%
\pgfpathlineto{\pgfqpoint{3.574799in}{1.683227in}}%
\pgfpathlineto{\pgfqpoint{3.566881in}{1.673641in}}%
\pgfpathlineto{\pgfqpoint{3.558957in}{1.664117in}}%
\pgfpathlineto{\pgfqpoint{3.551026in}{1.654659in}}%
\pgfpathlineto{\pgfqpoint{3.543089in}{1.645269in}}%
\pgfpathclose%
\pgfusepath{fill}%
\end{pgfscope}%
\begin{pgfscope}%
\pgfpathrectangle{\pgfqpoint{1.254980in}{0.150000in}}{\pgfqpoint{5.490039in}{5.490039in}}%
\pgfusepath{clip}%
\pgfsetbuttcap%
\pgfsetroundjoin%
\definecolor{currentfill}{rgb}{0.274128,0.199721,0.498911}%
\pgfsetfillcolor{currentfill}%
\pgfsetfillopacity{0.700000}%
\pgfsetlinewidth{0.000000pt}%
\definecolor{currentstroke}{rgb}{0.000000,0.000000,0.000000}%
\pgfsetstrokecolor{currentstroke}%
\pgfsetdash{}{0pt}%
\pgfpathmoveto{\pgfqpoint{2.634538in}{2.023079in}}%
\pgfpathlineto{\pgfqpoint{2.647979in}{2.007662in}}%
\pgfpathlineto{\pgfqpoint{2.661413in}{1.992475in}}%
\pgfpathlineto{\pgfqpoint{2.674840in}{1.977519in}}%
\pgfpathlineto{\pgfqpoint{2.688262in}{1.962791in}}%
\pgfpathlineto{\pgfqpoint{2.696756in}{1.962885in}}%
\pgfpathlineto{\pgfqpoint{2.705235in}{1.963232in}}%
\pgfpathlineto{\pgfqpoint{2.713697in}{1.963828in}}%
\pgfpathlineto{\pgfqpoint{2.722145in}{1.964667in}}%
\pgfpathlineto{\pgfqpoint{2.708764in}{1.978853in}}%
\pgfpathlineto{\pgfqpoint{2.695378in}{1.993266in}}%
\pgfpathlineto{\pgfqpoint{2.681987in}{2.007907in}}%
\pgfpathlineto{\pgfqpoint{2.668589in}{2.022780in}}%
\pgfpathlineto{\pgfqpoint{2.660101in}{2.022473in}}%
\pgfpathlineto{\pgfqpoint{2.651596in}{2.022417in}}%
\pgfpathlineto{\pgfqpoint{2.643076in}{2.022617in}}%
\pgfpathlineto{\pgfqpoint{2.634538in}{2.023079in}}%
\pgfpathclose%
\pgfusepath{fill}%
\end{pgfscope}%
\begin{pgfscope}%
\pgfpathrectangle{\pgfqpoint{1.254980in}{0.150000in}}{\pgfqpoint{5.490039in}{5.490039in}}%
\pgfusepath{clip}%
\pgfsetbuttcap%
\pgfsetroundjoin%
\definecolor{currentfill}{rgb}{0.172719,0.448791,0.557885}%
\pgfsetfillcolor{currentfill}%
\pgfsetfillopacity{0.700000}%
\pgfsetlinewidth{0.000000pt}%
\definecolor{currentstroke}{rgb}{0.000000,0.000000,0.000000}%
\pgfsetstrokecolor{currentstroke}%
\pgfsetdash{}{0pt}%
\pgfpathmoveto{\pgfqpoint{4.737534in}{2.539031in}}%
\pgfpathlineto{\pgfqpoint{4.751289in}{2.548299in}}%
\pgfpathlineto{\pgfqpoint{4.765059in}{2.557727in}}%
\pgfpathlineto{\pgfqpoint{4.778845in}{2.567316in}}%
\pgfpathlineto{\pgfqpoint{4.792647in}{2.577065in}}%
\pgfpathlineto{\pgfqpoint{4.800156in}{2.585844in}}%
\pgfpathlineto{\pgfqpoint{4.807658in}{2.594508in}}%
\pgfpathlineto{\pgfqpoint{4.815153in}{2.603058in}}%
\pgfpathlineto{\pgfqpoint{4.822641in}{2.611494in}}%
\pgfpathlineto{\pgfqpoint{4.808845in}{2.601816in}}%
\pgfpathlineto{\pgfqpoint{4.795065in}{2.592298in}}%
\pgfpathlineto{\pgfqpoint{4.781301in}{2.582940in}}%
\pgfpathlineto{\pgfqpoint{4.767552in}{2.573743in}}%
\pgfpathlineto{\pgfqpoint{4.760057in}{2.565225in}}%
\pgfpathlineto{\pgfqpoint{4.752556in}{2.556601in}}%
\pgfpathlineto{\pgfqpoint{4.745048in}{2.547870in}}%
\pgfpathlineto{\pgfqpoint{4.737534in}{2.539031in}}%
\pgfpathclose%
\pgfusepath{fill}%
\end{pgfscope}%
\begin{pgfscope}%
\pgfpathrectangle{\pgfqpoint{1.254980in}{0.150000in}}{\pgfqpoint{5.490039in}{5.490039in}}%
\pgfusepath{clip}%
\pgfsetbuttcap%
\pgfsetroundjoin%
\definecolor{currentfill}{rgb}{0.137770,0.537492,0.554906}%
\pgfsetfillcolor{currentfill}%
\pgfsetfillopacity{0.700000}%
\pgfsetlinewidth{0.000000pt}%
\definecolor{currentstroke}{rgb}{0.000000,0.000000,0.000000}%
\pgfsetstrokecolor{currentstroke}%
\pgfsetdash{}{0pt}%
\pgfpathmoveto{\pgfqpoint{5.022705in}{2.784548in}}%
\pgfpathlineto{\pgfqpoint{5.036625in}{2.795278in}}%
\pgfpathlineto{\pgfqpoint{5.050561in}{2.806167in}}%
\pgfpathlineto{\pgfqpoint{5.064515in}{2.817217in}}%
\pgfpathlineto{\pgfqpoint{5.078486in}{2.828428in}}%
\pgfpathlineto{\pgfqpoint{5.085853in}{2.835002in}}%
\pgfpathlineto{\pgfqpoint{5.093211in}{2.841467in}}%
\pgfpathlineto{\pgfqpoint{5.100562in}{2.847823in}}%
\pgfpathlineto{\pgfqpoint{5.107905in}{2.854074in}}%
\pgfpathlineto{\pgfqpoint{5.093944in}{2.843056in}}%
\pgfpathlineto{\pgfqpoint{5.080001in}{2.832198in}}%
\pgfpathlineto{\pgfqpoint{5.066075in}{2.821500in}}%
\pgfpathlineto{\pgfqpoint{5.052167in}{2.810962in}}%
\pgfpathlineto{\pgfqpoint{5.044813in}{2.804508in}}%
\pgfpathlineto{\pgfqpoint{5.037451in}{2.797956in}}%
\pgfpathlineto{\pgfqpoint{5.030082in}{2.791304in}}%
\pgfpathlineto{\pgfqpoint{5.022705in}{2.784548in}}%
\pgfpathclose%
\pgfusepath{fill}%
\end{pgfscope}%
\begin{pgfscope}%
\pgfpathrectangle{\pgfqpoint{1.254980in}{0.150000in}}{\pgfqpoint{5.490039in}{5.490039in}}%
\pgfusepath{clip}%
\pgfsetbuttcap%
\pgfsetroundjoin%
\definecolor{currentfill}{rgb}{0.255645,0.260703,0.528312}%
\pgfsetfillcolor{currentfill}%
\pgfsetfillopacity{0.700000}%
\pgfsetlinewidth{0.000000pt}%
\definecolor{currentstroke}{rgb}{0.000000,0.000000,0.000000}%
\pgfsetstrokecolor{currentstroke}%
\pgfsetdash{}{0pt}%
\pgfpathmoveto{\pgfqpoint{2.526755in}{2.154976in}}%
\pgfpathlineto{\pgfqpoint{2.540256in}{2.137639in}}%
\pgfpathlineto{\pgfqpoint{2.553748in}{2.120550in}}%
\pgfpathlineto{\pgfqpoint{2.567232in}{2.103704in}}%
\pgfpathlineto{\pgfqpoint{2.580708in}{2.087102in}}%
\pgfpathlineto{\pgfqpoint{2.589290in}{2.086081in}}%
\pgfpathlineto{\pgfqpoint{2.597855in}{2.085329in}}%
\pgfpathlineto{\pgfqpoint{2.606403in}{2.084842in}}%
\pgfpathlineto{\pgfqpoint{2.614934in}{2.084613in}}%
\pgfpathlineto{\pgfqpoint{2.601503in}{2.100667in}}%
\pgfpathlineto{\pgfqpoint{2.588064in}{2.116962in}}%
\pgfpathlineto{\pgfqpoint{2.574618in}{2.133500in}}%
\pgfpathlineto{\pgfqpoint{2.561164in}{2.150284in}}%
\pgfpathlineto{\pgfqpoint{2.552588in}{2.151051in}}%
\pgfpathlineto{\pgfqpoint{2.543995in}{2.152085in}}%
\pgfpathlineto{\pgfqpoint{2.535384in}{2.153392in}}%
\pgfpathlineto{\pgfqpoint{2.526755in}{2.154976in}}%
\pgfpathclose%
\pgfusepath{fill}%
\end{pgfscope}%
\begin{pgfscope}%
\pgfpathrectangle{\pgfqpoint{1.254980in}{0.150000in}}{\pgfqpoint{5.490039in}{5.490039in}}%
\pgfusepath{clip}%
\pgfsetbuttcap%
\pgfsetroundjoin%
\definecolor{currentfill}{rgb}{0.253935,0.265254,0.529983}%
\pgfsetfillcolor{currentfill}%
\pgfsetfillopacity{0.700000}%
\pgfsetlinewidth{0.000000pt}%
\definecolor{currentstroke}{rgb}{0.000000,0.000000,0.000000}%
\pgfsetstrokecolor{currentstroke}%
\pgfsetdash{}{0pt}%
\pgfpathmoveto{\pgfqpoint{4.251985in}{2.100866in}}%
\pgfpathlineto{\pgfqpoint{4.265492in}{2.106574in}}%
\pgfpathlineto{\pgfqpoint{4.279011in}{2.112444in}}%
\pgfpathlineto{\pgfqpoint{4.292543in}{2.118476in}}%
\pgfpathlineto{\pgfqpoint{4.306086in}{2.124670in}}%
\pgfpathlineto{\pgfqpoint{4.313778in}{2.136081in}}%
\pgfpathlineto{\pgfqpoint{4.321465in}{2.147413in}}%
\pgfpathlineto{\pgfqpoint{4.329147in}{2.158666in}}%
\pgfpathlineto{\pgfqpoint{4.336823in}{2.169838in}}%
\pgfpathlineto{\pgfqpoint{4.323283in}{2.163510in}}%
\pgfpathlineto{\pgfqpoint{4.309754in}{2.157344in}}%
\pgfpathlineto{\pgfqpoint{4.296238in}{2.151340in}}%
\pgfpathlineto{\pgfqpoint{4.282734in}{2.145498in}}%
\pgfpathlineto{\pgfqpoint{4.275055in}{2.134450in}}%
\pgfpathlineto{\pgfqpoint{4.267370in}{2.123328in}}%
\pgfpathlineto{\pgfqpoint{4.259680in}{2.112133in}}%
\pgfpathlineto{\pgfqpoint{4.251985in}{2.100866in}}%
\pgfpathclose%
\pgfusepath{fill}%
\end{pgfscope}%
\begin{pgfscope}%
\pgfpathrectangle{\pgfqpoint{1.254980in}{0.150000in}}{\pgfqpoint{5.490039in}{5.490039in}}%
\pgfusepath{clip}%
\pgfsetbuttcap%
\pgfsetroundjoin%
\definecolor{currentfill}{rgb}{0.279574,0.170599,0.479997}%
\pgfsetfillcolor{currentfill}%
\pgfsetfillopacity{0.700000}%
\pgfsetlinewidth{0.000000pt}%
\definecolor{currentstroke}{rgb}{0.000000,0.000000,0.000000}%
\pgfsetstrokecolor{currentstroke}%
\pgfsetdash{}{0pt}%
\pgfpathmoveto{\pgfqpoint{2.688262in}{1.962791in}}%
\pgfpathlineto{\pgfqpoint{2.701678in}{1.948289in}}%
\pgfpathlineto{\pgfqpoint{2.715089in}{1.934012in}}%
\pgfpathlineto{\pgfqpoint{2.728494in}{1.919959in}}%
\pgfpathlineto{\pgfqpoint{2.741894in}{1.906127in}}%
\pgfpathlineto{\pgfqpoint{2.750347in}{1.906773in}}%
\pgfpathlineto{\pgfqpoint{2.758784in}{1.907666in}}%
\pgfpathlineto{\pgfqpoint{2.767207in}{1.908799in}}%
\pgfpathlineto{\pgfqpoint{2.775615in}{1.910168in}}%
\pgfpathlineto{\pgfqpoint{2.762255in}{1.923460in}}%
\pgfpathlineto{\pgfqpoint{2.748890in}{1.936973in}}%
\pgfpathlineto{\pgfqpoint{2.735520in}{1.950708in}}%
\pgfpathlineto{\pgfqpoint{2.722145in}{1.964667in}}%
\pgfpathlineto{\pgfqpoint{2.713697in}{1.963828in}}%
\pgfpathlineto{\pgfqpoint{2.705235in}{1.963232in}}%
\pgfpathlineto{\pgfqpoint{2.696756in}{1.962885in}}%
\pgfpathlineto{\pgfqpoint{2.688262in}{1.962791in}}%
\pgfpathclose%
\pgfusepath{fill}%
\end{pgfscope}%
\begin{pgfscope}%
\pgfpathrectangle{\pgfqpoint{1.254980in}{0.150000in}}{\pgfqpoint{5.490039in}{5.490039in}}%
\pgfusepath{clip}%
\pgfsetbuttcap%
\pgfsetroundjoin%
\definecolor{currentfill}{rgb}{0.281412,0.155834,0.469201}%
\pgfsetfillcolor{currentfill}%
\pgfsetfillopacity{0.700000}%
\pgfsetlinewidth{0.000000pt}%
\definecolor{currentstroke}{rgb}{0.000000,0.000000,0.000000}%
\pgfsetstrokecolor{currentstroke}%
\pgfsetdash{}{0pt}%
\pgfpathmoveto{\pgfqpoint{3.966810in}{1.866017in}}%
\pgfpathlineto{\pgfqpoint{3.980205in}{1.868948in}}%
\pgfpathlineto{\pgfqpoint{3.993609in}{1.872042in}}%
\pgfpathlineto{\pgfqpoint{4.007023in}{1.875301in}}%
\pgfpathlineto{\pgfqpoint{4.020447in}{1.878723in}}%
\pgfpathlineto{\pgfqpoint{4.028226in}{1.890423in}}%
\pgfpathlineto{\pgfqpoint{4.036000in}{1.902092in}}%
\pgfpathlineto{\pgfqpoint{4.043770in}{1.913727in}}%
\pgfpathlineto{\pgfqpoint{4.051535in}{1.925326in}}%
\pgfpathlineto{\pgfqpoint{4.038115in}{1.921657in}}%
\pgfpathlineto{\pgfqpoint{4.024706in}{1.918152in}}%
\pgfpathlineto{\pgfqpoint{4.011307in}{1.914811in}}%
\pgfpathlineto{\pgfqpoint{3.997917in}{1.911634in}}%
\pgfpathlineto{\pgfqpoint{3.990148in}{1.900271in}}%
\pgfpathlineto{\pgfqpoint{3.982373in}{1.888879in}}%
\pgfpathlineto{\pgfqpoint{3.974594in}{1.877460in}}%
\pgfpathlineto{\pgfqpoint{3.966810in}{1.866017in}}%
\pgfpathclose%
\pgfusepath{fill}%
\end{pgfscope}%
\begin{pgfscope}%
\pgfpathrectangle{\pgfqpoint{1.254980in}{0.150000in}}{\pgfqpoint{5.490039in}{5.490039in}}%
\pgfusepath{clip}%
\pgfsetbuttcap%
\pgfsetroundjoin%
\definecolor{currentfill}{rgb}{0.243113,0.292092,0.538516}%
\pgfsetfillcolor{currentfill}%
\pgfsetfillopacity{0.700000}%
\pgfsetlinewidth{0.000000pt}%
\definecolor{currentstroke}{rgb}{0.000000,0.000000,0.000000}%
\pgfsetstrokecolor{currentstroke}%
\pgfsetdash{}{0pt}%
\pgfpathmoveto{\pgfqpoint{2.472665in}{2.226826in}}%
\pgfpathlineto{\pgfqpoint{2.486202in}{2.208484in}}%
\pgfpathlineto{\pgfqpoint{2.499729in}{2.190396in}}%
\pgfpathlineto{\pgfqpoint{2.513246in}{2.172560in}}%
\pgfpathlineto{\pgfqpoint{2.526755in}{2.154976in}}%
\pgfpathlineto{\pgfqpoint{2.535384in}{2.153392in}}%
\pgfpathlineto{\pgfqpoint{2.543995in}{2.152085in}}%
\pgfpathlineto{\pgfqpoint{2.552588in}{2.151051in}}%
\pgfpathlineto{\pgfqpoint{2.561164in}{2.150284in}}%
\pgfpathlineto{\pgfqpoint{2.547702in}{2.167316in}}%
\pgfpathlineto{\pgfqpoint{2.534231in}{2.184597in}}%
\pgfpathlineto{\pgfqpoint{2.520752in}{2.202130in}}%
\pgfpathlineto{\pgfqpoint{2.507264in}{2.219917in}}%
\pgfpathlineto{\pgfqpoint{2.498642in}{2.221226in}}%
\pgfpathlineto{\pgfqpoint{2.490002in}{2.222811in}}%
\pgfpathlineto{\pgfqpoint{2.481343in}{2.224676in}}%
\pgfpathlineto{\pgfqpoint{2.472665in}{2.226826in}}%
\pgfpathclose%
\pgfusepath{fill}%
\end{pgfscope}%
\begin{pgfscope}%
\pgfpathrectangle{\pgfqpoint{1.254980in}{0.150000in}}{\pgfqpoint{5.490039in}{5.490039in}}%
\pgfusepath{clip}%
\pgfsetbuttcap%
\pgfsetroundjoin%
\definecolor{currentfill}{rgb}{0.282290,0.145912,0.461510}%
\pgfsetfillcolor{currentfill}%
\pgfsetfillopacity{0.700000}%
\pgfsetlinewidth{0.000000pt}%
\definecolor{currentstroke}{rgb}{0.000000,0.000000,0.000000}%
\pgfsetstrokecolor{currentstroke}%
\pgfsetdash{}{0pt}%
\pgfpathmoveto{\pgfqpoint{2.741894in}{1.906127in}}%
\pgfpathlineto{\pgfqpoint{2.755289in}{1.892515in}}%
\pgfpathlineto{\pgfqpoint{2.768680in}{1.879122in}}%
\pgfpathlineto{\pgfqpoint{2.782066in}{1.865946in}}%
\pgfpathlineto{\pgfqpoint{2.795447in}{1.852986in}}%
\pgfpathlineto{\pgfqpoint{2.803861in}{1.854182in}}%
\pgfpathlineto{\pgfqpoint{2.812259in}{1.855617in}}%
\pgfpathlineto{\pgfqpoint{2.820644in}{1.857285in}}%
\pgfpathlineto{\pgfqpoint{2.829014in}{1.859181in}}%
\pgfpathlineto{\pgfqpoint{2.815670in}{1.871603in}}%
\pgfpathlineto{\pgfqpoint{2.802323in}{1.884241in}}%
\pgfpathlineto{\pgfqpoint{2.788971in}{1.897096in}}%
\pgfpathlineto{\pgfqpoint{2.775615in}{1.910168in}}%
\pgfpathlineto{\pgfqpoint{2.767207in}{1.908799in}}%
\pgfpathlineto{\pgfqpoint{2.758784in}{1.907666in}}%
\pgfpathlineto{\pgfqpoint{2.750347in}{1.906773in}}%
\pgfpathlineto{\pgfqpoint{2.741894in}{1.906127in}}%
\pgfpathclose%
\pgfusepath{fill}%
\end{pgfscope}%
\begin{pgfscope}%
\pgfpathrectangle{\pgfqpoint{1.254980in}{0.150000in}}{\pgfqpoint{5.490039in}{5.490039in}}%
\pgfusepath{clip}%
\pgfsetbuttcap%
\pgfsetroundjoin%
\definecolor{currentfill}{rgb}{0.274952,0.037752,0.364543}%
\pgfsetfillcolor{currentfill}%
\pgfsetfillopacity{0.700000}%
\pgfsetlinewidth{0.000000pt}%
\definecolor{currentstroke}{rgb}{0.000000,0.000000,0.000000}%
\pgfsetstrokecolor{currentstroke}%
\pgfsetdash{}{0pt}%
\pgfpathmoveto{\pgfqpoint{3.042152in}{1.688584in}}%
\pgfpathlineto{\pgfqpoint{3.055460in}{1.679620in}}%
\pgfpathlineto{\pgfqpoint{3.068767in}{1.670850in}}%
\pgfpathlineto{\pgfqpoint{3.082075in}{1.662271in}}%
\pgfpathlineto{\pgfqpoint{3.095382in}{1.653884in}}%
\pgfpathlineto{\pgfqpoint{3.103577in}{1.658570in}}%
\pgfpathlineto{\pgfqpoint{3.111761in}{1.663436in}}%
\pgfpathlineto{\pgfqpoint{3.119935in}{1.668478in}}%
\pgfpathlineto{\pgfqpoint{3.128098in}{1.673690in}}%
\pgfpathlineto{\pgfqpoint{3.114819in}{1.681578in}}%
\pgfpathlineto{\pgfqpoint{3.101539in}{1.689657in}}%
\pgfpathlineto{\pgfqpoint{3.088260in}{1.697928in}}%
\pgfpathlineto{\pgfqpoint{3.074981in}{1.706392in}}%
\pgfpathlineto{\pgfqpoint{3.066790in}{1.701669in}}%
\pgfpathlineto{\pgfqpoint{3.058589in}{1.697123in}}%
\pgfpathlineto{\pgfqpoint{3.050376in}{1.692760in}}%
\pgfpathlineto{\pgfqpoint{3.042152in}{1.688584in}}%
\pgfpathclose%
\pgfusepath{fill}%
\end{pgfscope}%
\begin{pgfscope}%
\pgfpathrectangle{\pgfqpoint{1.254980in}{0.150000in}}{\pgfqpoint{5.490039in}{5.490039in}}%
\pgfusepath{clip}%
\pgfsetbuttcap%
\pgfsetroundjoin%
\definecolor{currentfill}{rgb}{0.157729,0.485932,0.558013}%
\pgfsetfillcolor{currentfill}%
\pgfsetfillopacity{0.700000}%
\pgfsetlinewidth{0.000000pt}%
\definecolor{currentstroke}{rgb}{0.000000,0.000000,0.000000}%
\pgfsetstrokecolor{currentstroke}%
\pgfsetdash{}{0pt}%
\pgfpathmoveto{\pgfqpoint{2.180153in}{2.729161in}}%
\pgfpathlineto{\pgfqpoint{2.193941in}{2.704539in}}%
\pgfpathlineto{\pgfqpoint{2.207712in}{2.680234in}}%
\pgfpathlineto{\pgfqpoint{2.221468in}{2.656244in}}%
\pgfpathlineto{\pgfqpoint{2.235209in}{2.632566in}}%
\pgfpathlineto{\pgfqpoint{2.244070in}{2.628787in}}%
\pgfpathlineto{\pgfqpoint{2.252911in}{2.625313in}}%
\pgfpathlineto{\pgfqpoint{2.261730in}{2.622140in}}%
\pgfpathlineto{\pgfqpoint{2.270528in}{2.619261in}}%
\pgfpathlineto{\pgfqpoint{2.256843in}{2.642391in}}%
\pgfpathlineto{\pgfqpoint{2.243144in}{2.665831in}}%
\pgfpathlineto{\pgfqpoint{2.229428in}{2.689584in}}%
\pgfpathlineto{\pgfqpoint{2.215698in}{2.713653in}}%
\pgfpathlineto{\pgfqpoint{2.206844in}{2.717070in}}%
\pgfpathlineto{\pgfqpoint{2.197969in}{2.720790in}}%
\pgfpathlineto{\pgfqpoint{2.189072in}{2.724819in}}%
\pgfpathlineto{\pgfqpoint{2.180153in}{2.729161in}}%
\pgfpathclose%
\pgfusepath{fill}%
\end{pgfscope}%
\begin{pgfscope}%
\pgfpathrectangle{\pgfqpoint{1.254980in}{0.150000in}}{\pgfqpoint{5.490039in}{5.490039in}}%
\pgfusepath{clip}%
\pgfsetbuttcap%
\pgfsetroundjoin%
\definecolor{currentfill}{rgb}{0.143303,0.669459,0.511215}%
\pgfsetfillcolor{currentfill}%
\pgfsetfillopacity{0.700000}%
\pgfsetlinewidth{0.000000pt}%
\definecolor{currentstroke}{rgb}{0.000000,0.000000,0.000000}%
\pgfsetstrokecolor{currentstroke}%
\pgfsetdash{}{0pt}%
\pgfpathmoveto{\pgfqpoint{5.477762in}{3.136714in}}%
\pgfpathlineto{\pgfqpoint{5.491957in}{3.149023in}}%
\pgfpathlineto{\pgfqpoint{5.506173in}{3.161492in}}%
\pgfpathlineto{\pgfqpoint{5.520407in}{3.174120in}}%
\pgfpathlineto{\pgfqpoint{5.534662in}{3.186907in}}%
\pgfpathlineto{\pgfqpoint{5.541751in}{3.189827in}}%
\pgfpathlineto{\pgfqpoint{5.548831in}{3.192677in}}%
\pgfpathlineto{\pgfqpoint{5.555903in}{3.195462in}}%
\pgfpathlineto{\pgfqpoint{5.562967in}{3.198187in}}%
\pgfpathlineto{\pgfqpoint{5.548734in}{3.185777in}}%
\pgfpathlineto{\pgfqpoint{5.534521in}{3.173527in}}%
\pgfpathlineto{\pgfqpoint{5.520327in}{3.161434in}}%
\pgfpathlineto{\pgfqpoint{5.506152in}{3.149501in}}%
\pgfpathlineto{\pgfqpoint{5.499066in}{3.146389in}}%
\pgfpathlineto{\pgfqpoint{5.491973in}{3.143223in}}%
\pgfpathlineto{\pgfqpoint{5.484871in}{3.139999in}}%
\pgfpathlineto{\pgfqpoint{5.477762in}{3.136714in}}%
\pgfpathclose%
\pgfusepath{fill}%
\end{pgfscope}%
\begin{pgfscope}%
\pgfpathrectangle{\pgfqpoint{1.254980in}{0.150000in}}{\pgfqpoint{5.490039in}{5.490039in}}%
\pgfusepath{clip}%
\pgfsetbuttcap%
\pgfsetroundjoin%
\definecolor{currentfill}{rgb}{0.268510,0.009605,0.335427}%
\pgfsetfillcolor{currentfill}%
\pgfsetfillopacity{0.700000}%
\pgfsetlinewidth{0.000000pt}%
\definecolor{currentstroke}{rgb}{0.000000,0.000000,0.000000}%
\pgfsetstrokecolor{currentstroke}%
\pgfsetdash{}{0pt}%
\pgfpathmoveto{\pgfqpoint{3.458099in}{1.620108in}}%
\pgfpathlineto{\pgfqpoint{3.471386in}{1.616940in}}%
\pgfpathlineto{\pgfqpoint{3.484678in}{1.613946in}}%
\pgfpathlineto{\pgfqpoint{3.497975in}{1.611124in}}%
\pgfpathlineto{\pgfqpoint{3.511277in}{1.608475in}}%
\pgfpathlineto{\pgfqpoint{3.519240in}{1.617551in}}%
\pgfpathlineto{\pgfqpoint{3.527196in}{1.626712in}}%
\pgfpathlineto{\pgfqpoint{3.535146in}{1.635952in}}%
\pgfpathlineto{\pgfqpoint{3.543089in}{1.645269in}}%
\pgfpathlineto{\pgfqpoint{3.529803in}{1.647506in}}%
\pgfpathlineto{\pgfqpoint{3.516521in}{1.649916in}}%
\pgfpathlineto{\pgfqpoint{3.503244in}{1.652499in}}%
\pgfpathlineto{\pgfqpoint{3.489972in}{1.655256in}}%
\pgfpathlineto{\pgfqpoint{3.482014in}{1.646340in}}%
\pgfpathlineto{\pgfqpoint{3.474049in}{1.637507in}}%
\pgfpathlineto{\pgfqpoint{3.466077in}{1.628762in}}%
\pgfpathlineto{\pgfqpoint{3.458099in}{1.620108in}}%
\pgfpathclose%
\pgfusepath{fill}%
\end{pgfscope}%
\begin{pgfscope}%
\pgfpathrectangle{\pgfqpoint{1.254980in}{0.150000in}}{\pgfqpoint{5.490039in}{5.490039in}}%
\pgfusepath{clip}%
\pgfsetbuttcap%
\pgfsetroundjoin%
\definecolor{currentfill}{rgb}{0.203063,0.379716,0.553925}%
\pgfsetfillcolor{currentfill}%
\pgfsetfillopacity{0.700000}%
\pgfsetlinewidth{0.000000pt}%
\definecolor{currentstroke}{rgb}{0.000000,0.000000,0.000000}%
\pgfsetstrokecolor{currentstroke}%
\pgfsetdash{}{0pt}%
\pgfpathmoveto{\pgfqpoint{4.537243in}{2.355656in}}%
\pgfpathlineto{\pgfqpoint{4.550894in}{2.363669in}}%
\pgfpathlineto{\pgfqpoint{4.564559in}{2.371843in}}%
\pgfpathlineto{\pgfqpoint{4.578238in}{2.380178in}}%
\pgfpathlineto{\pgfqpoint{4.591931in}{2.388674in}}%
\pgfpathlineto{\pgfqpoint{4.599526in}{2.398834in}}%
\pgfpathlineto{\pgfqpoint{4.607115in}{2.408887in}}%
\pgfpathlineto{\pgfqpoint{4.614699in}{2.418832in}}%
\pgfpathlineto{\pgfqpoint{4.622275in}{2.428670in}}%
\pgfpathlineto{\pgfqpoint{4.608586in}{2.420156in}}%
\pgfpathlineto{\pgfqpoint{4.594910in}{2.411802in}}%
\pgfpathlineto{\pgfqpoint{4.581250in}{2.403610in}}%
\pgfpathlineto{\pgfqpoint{4.567603in}{2.395579in}}%
\pgfpathlineto{\pgfqpoint{4.560022in}{2.385749in}}%
\pgfpathlineto{\pgfqpoint{4.552435in}{2.375818in}}%
\pgfpathlineto{\pgfqpoint{4.544842in}{2.365787in}}%
\pgfpathlineto{\pgfqpoint{4.537243in}{2.355656in}}%
\pgfpathclose%
\pgfusepath{fill}%
\end{pgfscope}%
\begin{pgfscope}%
\pgfpathrectangle{\pgfqpoint{1.254980in}{0.150000in}}{\pgfqpoint{5.490039in}{5.490039in}}%
\pgfusepath{clip}%
\pgfsetbuttcap%
\pgfsetroundjoin%
\definecolor{currentfill}{rgb}{0.227802,0.326594,0.546532}%
\pgfsetfillcolor{currentfill}%
\pgfsetfillopacity{0.700000}%
\pgfsetlinewidth{0.000000pt}%
\definecolor{currentstroke}{rgb}{0.000000,0.000000,0.000000}%
\pgfsetstrokecolor{currentstroke}%
\pgfsetdash{}{0pt}%
\pgfpathmoveto{\pgfqpoint{2.418421in}{2.302790in}}%
\pgfpathlineto{\pgfqpoint{2.431998in}{2.283406in}}%
\pgfpathlineto{\pgfqpoint{2.445564in}{2.264285in}}%
\pgfpathlineto{\pgfqpoint{2.459119in}{2.245426in}}%
\pgfpathlineto{\pgfqpoint{2.472665in}{2.226826in}}%
\pgfpathlineto{\pgfqpoint{2.481343in}{2.224676in}}%
\pgfpathlineto{\pgfqpoint{2.490002in}{2.222811in}}%
\pgfpathlineto{\pgfqpoint{2.498642in}{2.221226in}}%
\pgfpathlineto{\pgfqpoint{2.507264in}{2.219917in}}%
\pgfpathlineto{\pgfqpoint{2.493767in}{2.237959in}}%
\pgfpathlineto{\pgfqpoint{2.480260in}{2.256260in}}%
\pgfpathlineto{\pgfqpoint{2.466744in}{2.274821in}}%
\pgfpathlineto{\pgfqpoint{2.453218in}{2.293645in}}%
\pgfpathlineto{\pgfqpoint{2.444547in}{2.295501in}}%
\pgfpathlineto{\pgfqpoint{2.435858in}{2.297641in}}%
\pgfpathlineto{\pgfqpoint{2.427149in}{2.300068in}}%
\pgfpathlineto{\pgfqpoint{2.418421in}{2.302790in}}%
\pgfpathclose%
\pgfusepath{fill}%
\end{pgfscope}%
\begin{pgfscope}%
\pgfpathrectangle{\pgfqpoint{1.254980in}{0.150000in}}{\pgfqpoint{5.490039in}{5.490039in}}%
\pgfusepath{clip}%
\pgfsetbuttcap%
\pgfsetroundjoin%
\definecolor{currentfill}{rgb}{0.277134,0.185228,0.489898}%
\pgfsetfillcolor{currentfill}%
\pgfsetfillopacity{0.700000}%
\pgfsetlinewidth{0.000000pt}%
\definecolor{currentstroke}{rgb}{0.000000,0.000000,0.000000}%
\pgfsetstrokecolor{currentstroke}%
\pgfsetdash{}{0pt}%
\pgfpathmoveto{\pgfqpoint{4.051535in}{1.925326in}}%
\pgfpathlineto{\pgfqpoint{4.064964in}{1.929158in}}%
\pgfpathlineto{\pgfqpoint{4.078404in}{1.933154in}}%
\pgfpathlineto{\pgfqpoint{4.091854in}{1.937313in}}%
\pgfpathlineto{\pgfqpoint{4.105315in}{1.941635in}}%
\pgfpathlineto{\pgfqpoint{4.113071in}{1.953426in}}%
\pgfpathlineto{\pgfqpoint{4.120822in}{1.965170in}}%
\pgfpathlineto{\pgfqpoint{4.128569in}{1.976866in}}%
\pgfpathlineto{\pgfqpoint{4.136311in}{1.988511in}}%
\pgfpathlineto{\pgfqpoint{4.122853in}{1.983970in}}%
\pgfpathlineto{\pgfqpoint{4.109407in}{1.979592in}}%
\pgfpathlineto{\pgfqpoint{4.095972in}{1.975377in}}%
\pgfpathlineto{\pgfqpoint{4.082547in}{1.971326in}}%
\pgfpathlineto{\pgfqpoint{4.074801in}{1.959889in}}%
\pgfpathlineto{\pgfqpoint{4.067050in}{1.948409in}}%
\pgfpathlineto{\pgfqpoint{4.059295in}{1.936887in}}%
\pgfpathlineto{\pgfqpoint{4.051535in}{1.925326in}}%
\pgfpathclose%
\pgfusepath{fill}%
\end{pgfscope}%
\begin{pgfscope}%
\pgfpathrectangle{\pgfqpoint{1.254980in}{0.150000in}}{\pgfqpoint{5.490039in}{5.490039in}}%
\pgfusepath{clip}%
\pgfsetbuttcap%
\pgfsetroundjoin%
\definecolor{currentfill}{rgb}{0.283187,0.125848,0.444960}%
\pgfsetfillcolor{currentfill}%
\pgfsetfillopacity{0.700000}%
\pgfsetlinewidth{0.000000pt}%
\definecolor{currentstroke}{rgb}{0.000000,0.000000,0.000000}%
\pgfsetstrokecolor{currentstroke}%
\pgfsetdash{}{0pt}%
\pgfpathmoveto{\pgfqpoint{2.795447in}{1.852986in}}%
\pgfpathlineto{\pgfqpoint{2.808825in}{1.840240in}}%
\pgfpathlineto{\pgfqpoint{2.822199in}{1.827707in}}%
\pgfpathlineto{\pgfqpoint{2.835569in}{1.815385in}}%
\pgfpathlineto{\pgfqpoint{2.848936in}{1.803274in}}%
\pgfpathlineto{\pgfqpoint{2.857311in}{1.805017in}}%
\pgfpathlineto{\pgfqpoint{2.865673in}{1.806992in}}%
\pgfpathlineto{\pgfqpoint{2.874020in}{1.809192in}}%
\pgfpathlineto{\pgfqpoint{2.882354in}{1.811613in}}%
\pgfpathlineto{\pgfqpoint{2.869024in}{1.823189in}}%
\pgfpathlineto{\pgfqpoint{2.855691in}{1.834975in}}%
\pgfpathlineto{\pgfqpoint{2.842354in}{1.846972in}}%
\pgfpathlineto{\pgfqpoint{2.829014in}{1.859181in}}%
\pgfpathlineto{\pgfqpoint{2.820644in}{1.857285in}}%
\pgfpathlineto{\pgfqpoint{2.812259in}{1.855617in}}%
\pgfpathlineto{\pgfqpoint{2.803861in}{1.854182in}}%
\pgfpathlineto{\pgfqpoint{2.795447in}{1.852986in}}%
\pgfpathclose%
\pgfusepath{fill}%
\end{pgfscope}%
\begin{pgfscope}%
\pgfpathrectangle{\pgfqpoint{1.254980in}{0.150000in}}{\pgfqpoint{5.490039in}{5.490039in}}%
\pgfusepath{clip}%
\pgfsetbuttcap%
\pgfsetroundjoin%
\definecolor{currentfill}{rgb}{0.127568,0.566949,0.550556}%
\pgfsetfillcolor{currentfill}%
\pgfsetfillopacity{0.700000}%
\pgfsetlinewidth{0.000000pt}%
\definecolor{currentstroke}{rgb}{0.000000,0.000000,0.000000}%
\pgfsetstrokecolor{currentstroke}%
\pgfsetdash{}{0pt}%
\pgfpathmoveto{\pgfqpoint{5.107905in}{2.854074in}}%
\pgfpathlineto{\pgfqpoint{5.121883in}{2.865252in}}%
\pgfpathlineto{\pgfqpoint{5.135878in}{2.876590in}}%
\pgfpathlineto{\pgfqpoint{5.149892in}{2.888088in}}%
\pgfpathlineto{\pgfqpoint{5.163923in}{2.899747in}}%
\pgfpathlineto{\pgfqpoint{5.171246in}{2.905683in}}%
\pgfpathlineto{\pgfqpoint{5.178561in}{2.911511in}}%
\pgfpathlineto{\pgfqpoint{5.185868in}{2.917234in}}%
\pgfpathlineto{\pgfqpoint{5.193167in}{2.922855in}}%
\pgfpathlineto{\pgfqpoint{5.179148in}{2.911420in}}%
\pgfpathlineto{\pgfqpoint{5.165147in}{2.900145in}}%
\pgfpathlineto{\pgfqpoint{5.151163in}{2.889029in}}%
\pgfpathlineto{\pgfqpoint{5.137197in}{2.878073in}}%
\pgfpathlineto{\pgfqpoint{5.129886in}{2.872219in}}%
\pgfpathlineto{\pgfqpoint{5.122567in}{2.866269in}}%
\pgfpathlineto{\pgfqpoint{5.115240in}{2.860222in}}%
\pgfpathlineto{\pgfqpoint{5.107905in}{2.854074in}}%
\pgfpathclose%
\pgfusepath{fill}%
\end{pgfscope}%
\begin{pgfscope}%
\pgfpathrectangle{\pgfqpoint{1.254980in}{0.150000in}}{\pgfqpoint{5.490039in}{5.490039in}}%
\pgfusepath{clip}%
\pgfsetbuttcap%
\pgfsetroundjoin%
\definecolor{currentfill}{rgb}{0.239346,0.300855,0.540844}%
\pgfsetfillcolor{currentfill}%
\pgfsetfillopacity{0.700000}%
\pgfsetlinewidth{0.000000pt}%
\definecolor{currentstroke}{rgb}{0.000000,0.000000,0.000000}%
\pgfsetstrokecolor{currentstroke}%
\pgfsetdash{}{0pt}%
\pgfpathmoveto{\pgfqpoint{4.336823in}{2.169838in}}%
\pgfpathlineto{\pgfqpoint{4.350377in}{2.176328in}}%
\pgfpathlineto{\pgfqpoint{4.363943in}{2.182980in}}%
\pgfpathlineto{\pgfqpoint{4.377521in}{2.189794in}}%
\pgfpathlineto{\pgfqpoint{4.391113in}{2.196769in}}%
\pgfpathlineto{\pgfqpoint{4.398782in}{2.207976in}}%
\pgfpathlineto{\pgfqpoint{4.406445in}{2.219093in}}%
\pgfpathlineto{\pgfqpoint{4.414103in}{2.230120in}}%
\pgfpathlineto{\pgfqpoint{4.421756in}{2.241056in}}%
\pgfpathlineto{\pgfqpoint{4.408167in}{2.233975in}}%
\pgfpathlineto{\pgfqpoint{4.394591in}{2.227056in}}%
\pgfpathlineto{\pgfqpoint{4.381028in}{2.220298in}}%
\pgfpathlineto{\pgfqpoint{4.367478in}{2.213702in}}%
\pgfpathlineto{\pgfqpoint{4.359822in}{2.202861in}}%
\pgfpathlineto{\pgfqpoint{4.352161in}{2.191936in}}%
\pgfpathlineto{\pgfqpoint{4.344495in}{2.180928in}}%
\pgfpathlineto{\pgfqpoint{4.336823in}{2.169838in}}%
\pgfpathclose%
\pgfusepath{fill}%
\end{pgfscope}%
\begin{pgfscope}%
\pgfpathrectangle{\pgfqpoint{1.254980in}{0.150000in}}{\pgfqpoint{5.490039in}{5.490039in}}%
\pgfusepath{clip}%
\pgfsetbuttcap%
\pgfsetroundjoin%
\definecolor{currentfill}{rgb}{0.160665,0.478540,0.558115}%
\pgfsetfillcolor{currentfill}%
\pgfsetfillopacity{0.700000}%
\pgfsetlinewidth{0.000000pt}%
\definecolor{currentstroke}{rgb}{0.000000,0.000000,0.000000}%
\pgfsetstrokecolor{currentstroke}%
\pgfsetdash{}{0pt}%
\pgfpathmoveto{\pgfqpoint{4.822641in}{2.611494in}}%
\pgfpathlineto{\pgfqpoint{4.836452in}{2.621334in}}%
\pgfpathlineto{\pgfqpoint{4.850280in}{2.631334in}}%
\pgfpathlineto{\pgfqpoint{4.864124in}{2.641495in}}%
\pgfpathlineto{\pgfqpoint{4.877984in}{2.651817in}}%
\pgfpathlineto{\pgfqpoint{4.885459in}{2.660052in}}%
\pgfpathlineto{\pgfqpoint{4.892926in}{2.668170in}}%
\pgfpathlineto{\pgfqpoint{4.900386in}{2.676172in}}%
\pgfpathlineto{\pgfqpoint{4.907838in}{2.684059in}}%
\pgfpathlineto{\pgfqpoint{4.893985in}{2.673838in}}%
\pgfpathlineto{\pgfqpoint{4.880148in}{2.663779in}}%
\pgfpathlineto{\pgfqpoint{4.866328in}{2.653879in}}%
\pgfpathlineto{\pgfqpoint{4.852523in}{2.644141in}}%
\pgfpathlineto{\pgfqpoint{4.845063in}{2.636141in}}%
\pgfpathlineto{\pgfqpoint{4.837596in}{2.628035in}}%
\pgfpathlineto{\pgfqpoint{4.830122in}{2.619820in}}%
\pgfpathlineto{\pgfqpoint{4.822641in}{2.611494in}}%
\pgfpathclose%
\pgfusepath{fill}%
\end{pgfscope}%
\begin{pgfscope}%
\pgfpathrectangle{\pgfqpoint{1.254980in}{0.150000in}}{\pgfqpoint{5.490039in}{5.490039in}}%
\pgfusepath{clip}%
\pgfsetbuttcap%
\pgfsetroundjoin%
\definecolor{currentfill}{rgb}{0.166383,0.690856,0.496502}%
\pgfsetfillcolor{currentfill}%
\pgfsetfillopacity{0.700000}%
\pgfsetlinewidth{0.000000pt}%
\definecolor{currentstroke}{rgb}{0.000000,0.000000,0.000000}%
\pgfsetstrokecolor{currentstroke}%
\pgfsetdash{}{0pt}%
\pgfpathmoveto{\pgfqpoint{5.562967in}{3.198187in}}%
\pgfpathlineto{\pgfqpoint{5.577220in}{3.210755in}}%
\pgfpathlineto{\pgfqpoint{5.591492in}{3.223482in}}%
\pgfpathlineto{\pgfqpoint{5.605785in}{3.236369in}}%
\pgfpathlineto{\pgfqpoint{5.620097in}{3.249415in}}%
\pgfpathlineto{\pgfqpoint{5.627130in}{3.251685in}}%
\pgfpathlineto{\pgfqpoint{5.634154in}{3.253896in}}%
\pgfpathlineto{\pgfqpoint{5.641171in}{3.256053in}}%
\pgfpathlineto{\pgfqpoint{5.648179in}{3.258159in}}%
\pgfpathlineto{\pgfqpoint{5.633890in}{3.245523in}}%
\pgfpathlineto{\pgfqpoint{5.619621in}{3.233045in}}%
\pgfpathlineto{\pgfqpoint{5.605372in}{3.220726in}}%
\pgfpathlineto{\pgfqpoint{5.591142in}{3.208565in}}%
\pgfpathlineto{\pgfqpoint{5.584110in}{3.206039in}}%
\pgfpathlineto{\pgfqpoint{5.577070in}{3.203471in}}%
\pgfpathlineto{\pgfqpoint{5.570023in}{3.200855in}}%
\pgfpathlineto{\pgfqpoint{5.562967in}{3.198187in}}%
\pgfpathclose%
\pgfusepath{fill}%
\end{pgfscope}%
\begin{pgfscope}%
\pgfpathrectangle{\pgfqpoint{1.254980in}{0.150000in}}{\pgfqpoint{5.490039in}{5.490039in}}%
\pgfusepath{clip}%
\pgfsetbuttcap%
\pgfsetroundjoin%
\definecolor{currentfill}{rgb}{0.268510,0.009605,0.335427}%
\pgfsetfillcolor{currentfill}%
\pgfsetfillopacity{0.700000}%
\pgfsetlinewidth{0.000000pt}%
\definecolor{currentstroke}{rgb}{0.000000,0.000000,0.000000}%
\pgfsetstrokecolor{currentstroke}%
\pgfsetdash{}{0pt}%
\pgfpathmoveto{\pgfqpoint{3.234364in}{1.617343in}}%
\pgfpathlineto{\pgfqpoint{3.247654in}{1.611132in}}%
\pgfpathlineto{\pgfqpoint{3.260946in}{1.605102in}}%
\pgfpathlineto{\pgfqpoint{3.274239in}{1.599254in}}%
\pgfpathlineto{\pgfqpoint{3.287535in}{1.593587in}}%
\pgfpathlineto{\pgfqpoint{3.295615in}{1.600404in}}%
\pgfpathlineto{\pgfqpoint{3.303687in}{1.607360in}}%
\pgfpathlineto{\pgfqpoint{3.311750in}{1.614451in}}%
\pgfpathlineto{\pgfqpoint{3.319804in}{1.621673in}}%
\pgfpathlineto{\pgfqpoint{3.306530in}{1.626873in}}%
\pgfpathlineto{\pgfqpoint{3.293258in}{1.632253in}}%
\pgfpathlineto{\pgfqpoint{3.279989in}{1.637814in}}%
\pgfpathlineto{\pgfqpoint{3.266722in}{1.643557in}}%
\pgfpathlineto{\pgfqpoint{3.258646in}{1.636793in}}%
\pgfpathlineto{\pgfqpoint{3.250561in}{1.630166in}}%
\pgfpathlineto{\pgfqpoint{3.242467in}{1.623681in}}%
\pgfpathlineto{\pgfqpoint{3.234364in}{1.617343in}}%
\pgfpathclose%
\pgfusepath{fill}%
\end{pgfscope}%
\begin{pgfscope}%
\pgfpathrectangle{\pgfqpoint{1.254980in}{0.150000in}}{\pgfqpoint{5.490039in}{5.490039in}}%
\pgfusepath{clip}%
\pgfsetbuttcap%
\pgfsetroundjoin%
\definecolor{currentfill}{rgb}{0.212395,0.359683,0.551710}%
\pgfsetfillcolor{currentfill}%
\pgfsetfillopacity{0.700000}%
\pgfsetlinewidth{0.000000pt}%
\definecolor{currentstroke}{rgb}{0.000000,0.000000,0.000000}%
\pgfsetstrokecolor{currentstroke}%
\pgfsetdash{}{0pt}%
\pgfpathmoveto{\pgfqpoint{2.364006in}{2.383012in}}%
\pgfpathlineto{\pgfqpoint{2.377627in}{2.362549in}}%
\pgfpathlineto{\pgfqpoint{2.391236in}{2.342359in}}%
\pgfpathlineto{\pgfqpoint{2.404834in}{2.322440in}}%
\pgfpathlineto{\pgfqpoint{2.418421in}{2.302790in}}%
\pgfpathlineto{\pgfqpoint{2.427149in}{2.300068in}}%
\pgfpathlineto{\pgfqpoint{2.435858in}{2.297641in}}%
\pgfpathlineto{\pgfqpoint{2.444547in}{2.295501in}}%
\pgfpathlineto{\pgfqpoint{2.453218in}{2.293645in}}%
\pgfpathlineto{\pgfqpoint{2.439681in}{2.312734in}}%
\pgfpathlineto{\pgfqpoint{2.426134in}{2.332090in}}%
\pgfpathlineto{\pgfqpoint{2.412576in}{2.351716in}}%
\pgfpathlineto{\pgfqpoint{2.399008in}{2.371614in}}%
\pgfpathlineto{\pgfqpoint{2.390287in}{2.374021in}}%
\pgfpathlineto{\pgfqpoint{2.381547in}{2.376719in}}%
\pgfpathlineto{\pgfqpoint{2.372786in}{2.379714in}}%
\pgfpathlineto{\pgfqpoint{2.364006in}{2.383012in}}%
\pgfpathclose%
\pgfusepath{fill}%
\end{pgfscope}%
\begin{pgfscope}%
\pgfpathrectangle{\pgfqpoint{1.254980in}{0.150000in}}{\pgfqpoint{5.490039in}{5.490039in}}%
\pgfusepath{clip}%
\pgfsetbuttcap%
\pgfsetroundjoin%
\definecolor{currentfill}{rgb}{0.282910,0.105393,0.426902}%
\pgfsetfillcolor{currentfill}%
\pgfsetfillopacity{0.700000}%
\pgfsetlinewidth{0.000000pt}%
\definecolor{currentstroke}{rgb}{0.000000,0.000000,0.000000}%
\pgfsetstrokecolor{currentstroke}%
\pgfsetdash{}{0pt}%
\pgfpathmoveto{\pgfqpoint{2.848936in}{1.803274in}}%
\pgfpathlineto{\pgfqpoint{2.862299in}{1.791371in}}%
\pgfpathlineto{\pgfqpoint{2.875660in}{1.779675in}}%
\pgfpathlineto{\pgfqpoint{2.889017in}{1.768186in}}%
\pgfpathlineto{\pgfqpoint{2.902372in}{1.756902in}}%
\pgfpathlineto{\pgfqpoint{2.910711in}{1.759191in}}%
\pgfpathlineto{\pgfqpoint{2.919037in}{1.761703in}}%
\pgfpathlineto{\pgfqpoint{2.927350in}{1.764434in}}%
\pgfpathlineto{\pgfqpoint{2.935649in}{1.767377in}}%
\pgfpathlineto{\pgfqpoint{2.922329in}{1.778128in}}%
\pgfpathlineto{\pgfqpoint{2.909007in}{1.789083in}}%
\pgfpathlineto{\pgfqpoint{2.895682in}{1.800244in}}%
\pgfpathlineto{\pgfqpoint{2.882354in}{1.811613in}}%
\pgfpathlineto{\pgfqpoint{2.874020in}{1.809192in}}%
\pgfpathlineto{\pgfqpoint{2.865673in}{1.806992in}}%
\pgfpathlineto{\pgfqpoint{2.857311in}{1.805017in}}%
\pgfpathlineto{\pgfqpoint{2.848936in}{1.803274in}}%
\pgfpathclose%
\pgfusepath{fill}%
\end{pgfscope}%
\begin{pgfscope}%
\pgfpathrectangle{\pgfqpoint{1.254980in}{0.150000in}}{\pgfqpoint{5.490039in}{5.490039in}}%
\pgfusepath{clip}%
\pgfsetbuttcap%
\pgfsetroundjoin%
\definecolor{currentfill}{rgb}{0.267004,0.004874,0.329415}%
\pgfsetfillcolor{currentfill}%
\pgfsetfillopacity{0.700000}%
\pgfsetlinewidth{0.000000pt}%
\definecolor{currentstroke}{rgb}{0.000000,0.000000,0.000000}%
\pgfsetstrokecolor{currentstroke}%
\pgfsetdash{}{0pt}%
\pgfpathmoveto{\pgfqpoint{3.372931in}{1.602665in}}%
\pgfpathlineto{\pgfqpoint{3.386220in}{1.598356in}}%
\pgfpathlineto{\pgfqpoint{3.399514in}{1.594224in}}%
\pgfpathlineto{\pgfqpoint{3.412811in}{1.590267in}}%
\pgfpathlineto{\pgfqpoint{3.426111in}{1.586485in}}%
\pgfpathlineto{\pgfqpoint{3.434119in}{1.594734in}}%
\pgfpathlineto{\pgfqpoint{3.442120in}{1.603090in}}%
\pgfpathlineto{\pgfqpoint{3.450113in}{1.611549in}}%
\pgfpathlineto{\pgfqpoint{3.458099in}{1.620108in}}%
\pgfpathlineto{\pgfqpoint{3.444815in}{1.623451in}}%
\pgfpathlineto{\pgfqpoint{3.431536in}{1.626968in}}%
\pgfpathlineto{\pgfqpoint{3.418261in}{1.630661in}}%
\pgfpathlineto{\pgfqpoint{3.404990in}{1.634529in}}%
\pgfpathlineto{\pgfqpoint{3.396986in}{1.626400in}}%
\pgfpathlineto{\pgfqpoint{3.388975in}{1.618376in}}%
\pgfpathlineto{\pgfqpoint{3.380957in}{1.610463in}}%
\pgfpathlineto{\pgfqpoint{3.372931in}{1.602665in}}%
\pgfpathclose%
\pgfusepath{fill}%
\end{pgfscope}%
\begin{pgfscope}%
\pgfpathrectangle{\pgfqpoint{1.254980in}{0.150000in}}{\pgfqpoint{5.490039in}{5.490039in}}%
\pgfusepath{clip}%
\pgfsetbuttcap%
\pgfsetroundjoin%
\definecolor{currentfill}{rgb}{0.269308,0.218818,0.509577}%
\pgfsetfillcolor{currentfill}%
\pgfsetfillopacity{0.700000}%
\pgfsetlinewidth{0.000000pt}%
\definecolor{currentstroke}{rgb}{0.000000,0.000000,0.000000}%
\pgfsetstrokecolor{currentstroke}%
\pgfsetdash{}{0pt}%
\pgfpathmoveto{\pgfqpoint{4.136311in}{1.988511in}}%
\pgfpathlineto{\pgfqpoint{4.149779in}{1.993215in}}%
\pgfpathlineto{\pgfqpoint{4.163258in}{1.998081in}}%
\pgfpathlineto{\pgfqpoint{4.176748in}{2.003110in}}%
\pgfpathlineto{\pgfqpoint{4.190250in}{2.008302in}}%
\pgfpathlineto{\pgfqpoint{4.197984in}{2.020096in}}%
\pgfpathlineto{\pgfqpoint{4.205713in}{2.031830in}}%
\pgfpathlineto{\pgfqpoint{4.213437in}{2.043501in}}%
\pgfpathlineto{\pgfqpoint{4.221156in}{2.055107in}}%
\pgfpathlineto{\pgfqpoint{4.207658in}{2.049725in}}%
\pgfpathlineto{\pgfqpoint{4.194171in}{2.044505in}}%
\pgfpathlineto{\pgfqpoint{4.180695in}{2.039447in}}%
\pgfpathlineto{\pgfqpoint{4.167230in}{2.034553in}}%
\pgfpathlineto{\pgfqpoint{4.159508in}{2.023126in}}%
\pgfpathlineto{\pgfqpoint{4.151780in}{2.011643in}}%
\pgfpathlineto{\pgfqpoint{4.144048in}{2.000104in}}%
\pgfpathlineto{\pgfqpoint{4.136311in}{1.988511in}}%
\pgfpathclose%
\pgfusepath{fill}%
\end{pgfscope}%
\begin{pgfscope}%
\pgfpathrectangle{\pgfqpoint{1.254980in}{0.150000in}}{\pgfqpoint{5.490039in}{5.490039in}}%
\pgfusepath{clip}%
\pgfsetbuttcap%
\pgfsetroundjoin%
\definecolor{currentfill}{rgb}{0.202219,0.715272,0.476084}%
\pgfsetfillcolor{currentfill}%
\pgfsetfillopacity{0.700000}%
\pgfsetlinewidth{0.000000pt}%
\definecolor{currentstroke}{rgb}{0.000000,0.000000,0.000000}%
\pgfsetstrokecolor{currentstroke}%
\pgfsetdash{}{0pt}%
\pgfpathmoveto{\pgfqpoint{5.648179in}{3.258159in}}%
\pgfpathlineto{\pgfqpoint{5.662488in}{3.270954in}}%
\pgfpathlineto{\pgfqpoint{5.676817in}{3.283908in}}%
\pgfpathlineto{\pgfqpoint{5.691166in}{3.297021in}}%
\pgfpathlineto{\pgfqpoint{5.705536in}{3.310293in}}%
\pgfpathlineto{\pgfqpoint{5.712511in}{3.311925in}}%
\pgfpathlineto{\pgfqpoint{5.719478in}{3.313509in}}%
\pgfpathlineto{\pgfqpoint{5.726436in}{3.315051in}}%
\pgfpathlineto{\pgfqpoint{5.733387in}{3.316554in}}%
\pgfpathlineto{\pgfqpoint{5.719043in}{3.303723in}}%
\pgfpathlineto{\pgfqpoint{5.704719in}{3.291051in}}%
\pgfpathlineto{\pgfqpoint{5.690416in}{3.278536in}}%
\pgfpathlineto{\pgfqpoint{5.676132in}{3.266180in}}%
\pgfpathlineto{\pgfqpoint{5.669155in}{3.264226in}}%
\pgfpathlineto{\pgfqpoint{5.662171in}{3.262241in}}%
\pgfpathlineto{\pgfqpoint{5.655179in}{3.260221in}}%
\pgfpathlineto{\pgfqpoint{5.648179in}{3.258159in}}%
\pgfpathclose%
\pgfusepath{fill}%
\end{pgfscope}%
\begin{pgfscope}%
\pgfpathrectangle{\pgfqpoint{1.254980in}{0.150000in}}{\pgfqpoint{5.490039in}{5.490039in}}%
\pgfusepath{clip}%
\pgfsetbuttcap%
\pgfsetroundjoin%
\definecolor{currentfill}{rgb}{0.188923,0.410910,0.556326}%
\pgfsetfillcolor{currentfill}%
\pgfsetfillopacity{0.700000}%
\pgfsetlinewidth{0.000000pt}%
\definecolor{currentstroke}{rgb}{0.000000,0.000000,0.000000}%
\pgfsetstrokecolor{currentstroke}%
\pgfsetdash{}{0pt}%
\pgfpathmoveto{\pgfqpoint{4.622275in}{2.428670in}}%
\pgfpathlineto{\pgfqpoint{4.635980in}{2.437346in}}%
\pgfpathlineto{\pgfqpoint{4.649699in}{2.446183in}}%
\pgfpathlineto{\pgfqpoint{4.663433in}{2.455180in}}%
\pgfpathlineto{\pgfqpoint{4.677182in}{2.464339in}}%
\pgfpathlineto{\pgfqpoint{4.684749in}{2.474070in}}%
\pgfpathlineto{\pgfqpoint{4.692309in}{2.483687in}}%
\pgfpathlineto{\pgfqpoint{4.699863in}{2.493190in}}%
\pgfpathlineto{\pgfqpoint{4.707410in}{2.502581in}}%
\pgfpathlineto{\pgfqpoint{4.693665in}{2.493434in}}%
\pgfpathlineto{\pgfqpoint{4.679936in}{2.484448in}}%
\pgfpathlineto{\pgfqpoint{4.666221in}{2.475622in}}%
\pgfpathlineto{\pgfqpoint{4.652521in}{2.466958in}}%
\pgfpathlineto{\pgfqpoint{4.644969in}{2.457544in}}%
\pgfpathlineto{\pgfqpoint{4.637411in}{2.448026in}}%
\pgfpathlineto{\pgfqpoint{4.629846in}{2.438401in}}%
\pgfpathlineto{\pgfqpoint{4.622275in}{2.428670in}}%
\pgfpathclose%
\pgfusepath{fill}%
\end{pgfscope}%
\begin{pgfscope}%
\pgfpathrectangle{\pgfqpoint{1.254980in}{0.150000in}}{\pgfqpoint{5.490039in}{5.490039in}}%
\pgfusepath{clip}%
\pgfsetbuttcap%
\pgfsetroundjoin%
\definecolor{currentfill}{rgb}{0.272594,0.025563,0.353093}%
\pgfsetfillcolor{currentfill}%
\pgfsetfillopacity{0.700000}%
\pgfsetlinewidth{0.000000pt}%
\definecolor{currentstroke}{rgb}{0.000000,0.000000,0.000000}%
\pgfsetstrokecolor{currentstroke}%
\pgfsetdash{}{0pt}%
\pgfpathmoveto{\pgfqpoint{3.095382in}{1.653884in}}%
\pgfpathlineto{\pgfqpoint{3.108689in}{1.645687in}}%
\pgfpathlineto{\pgfqpoint{3.121997in}{1.637679in}}%
\pgfpathlineto{\pgfqpoint{3.135305in}{1.629859in}}%
\pgfpathlineto{\pgfqpoint{3.148614in}{1.622227in}}%
\pgfpathlineto{\pgfqpoint{3.156781in}{1.627422in}}%
\pgfpathlineto{\pgfqpoint{3.164938in}{1.632790in}}%
\pgfpathlineto{\pgfqpoint{3.173085in}{1.638327in}}%
\pgfpathlineto{\pgfqpoint{3.181222in}{1.644027in}}%
\pgfpathlineto{\pgfqpoint{3.167940in}{1.651161in}}%
\pgfpathlineto{\pgfqpoint{3.154659in}{1.658482in}}%
\pgfpathlineto{\pgfqpoint{3.141378in}{1.665992in}}%
\pgfpathlineto{\pgfqpoint{3.128098in}{1.673690in}}%
\pgfpathlineto{\pgfqpoint{3.119935in}{1.668478in}}%
\pgfpathlineto{\pgfqpoint{3.111761in}{1.663436in}}%
\pgfpathlineto{\pgfqpoint{3.103577in}{1.658570in}}%
\pgfpathlineto{\pgfqpoint{3.095382in}{1.653884in}}%
\pgfpathclose%
\pgfusepath{fill}%
\end{pgfscope}%
\begin{pgfscope}%
\pgfpathrectangle{\pgfqpoint{1.254980in}{0.150000in}}{\pgfqpoint{5.490039in}{5.490039in}}%
\pgfusepath{clip}%
\pgfsetbuttcap%
\pgfsetroundjoin%
\definecolor{currentfill}{rgb}{0.120565,0.596422,0.543611}%
\pgfsetfillcolor{currentfill}%
\pgfsetfillopacity{0.700000}%
\pgfsetlinewidth{0.000000pt}%
\definecolor{currentstroke}{rgb}{0.000000,0.000000,0.000000}%
\pgfsetstrokecolor{currentstroke}%
\pgfsetdash{}{0pt}%
\pgfpathmoveto{\pgfqpoint{5.193167in}{2.922855in}}%
\pgfpathlineto{\pgfqpoint{5.207204in}{2.934450in}}%
\pgfpathlineto{\pgfqpoint{5.221259in}{2.946205in}}%
\pgfpathlineto{\pgfqpoint{5.235332in}{2.958121in}}%
\pgfpathlineto{\pgfqpoint{5.249424in}{2.970197in}}%
\pgfpathlineto{\pgfqpoint{5.256702in}{2.975475in}}%
\pgfpathlineto{\pgfqpoint{5.263970in}{2.980649in}}%
\pgfpathlineto{\pgfqpoint{5.271231in}{2.985721in}}%
\pgfpathlineto{\pgfqpoint{5.278483in}{2.990696in}}%
\pgfpathlineto{\pgfqpoint{5.264405in}{2.978875in}}%
\pgfpathlineto{\pgfqpoint{5.250346in}{2.967214in}}%
\pgfpathlineto{\pgfqpoint{5.236304in}{2.955713in}}%
\pgfpathlineto{\pgfqpoint{5.222281in}{2.944371in}}%
\pgfpathlineto{\pgfqpoint{5.215015in}{2.939131in}}%
\pgfpathlineto{\pgfqpoint{5.207740in}{2.933801in}}%
\pgfpathlineto{\pgfqpoint{5.200458in}{2.928376in}}%
\pgfpathlineto{\pgfqpoint{5.193167in}{2.922855in}}%
\pgfpathclose%
\pgfusepath{fill}%
\end{pgfscope}%
\begin{pgfscope}%
\pgfpathrectangle{\pgfqpoint{1.254980in}{0.150000in}}{\pgfqpoint{5.490039in}{5.490039in}}%
\pgfusepath{clip}%
\pgfsetbuttcap%
\pgfsetroundjoin%
\definecolor{currentfill}{rgb}{0.277941,0.056324,0.381191}%
\pgfsetfillcolor{currentfill}%
\pgfsetfillopacity{0.700000}%
\pgfsetlinewidth{0.000000pt}%
\definecolor{currentstroke}{rgb}{0.000000,0.000000,0.000000}%
\pgfsetstrokecolor{currentstroke}%
\pgfsetdash{}{0pt}%
\pgfpathmoveto{\pgfqpoint{3.681195in}{1.674548in}}%
\pgfpathlineto{\pgfqpoint{3.694523in}{1.674224in}}%
\pgfpathlineto{\pgfqpoint{3.707858in}{1.674069in}}%
\pgfpathlineto{\pgfqpoint{3.721200in}{1.674081in}}%
\pgfpathlineto{\pgfqpoint{3.734549in}{1.674259in}}%
\pgfpathlineto{\pgfqpoint{3.742427in}{1.685004in}}%
\pgfpathlineto{\pgfqpoint{3.750300in}{1.695782in}}%
\pgfpathlineto{\pgfqpoint{3.758167in}{1.706591in}}%
\pgfpathlineto{\pgfqpoint{3.766030in}{1.717427in}}%
\pgfpathlineto{\pgfqpoint{3.752690in}{1.716892in}}%
\pgfpathlineto{\pgfqpoint{3.739358in}{1.716523in}}%
\pgfpathlineto{\pgfqpoint{3.726033in}{1.716322in}}%
\pgfpathlineto{\pgfqpoint{3.712715in}{1.716289in}}%
\pgfpathlineto{\pgfqpoint{3.704843in}{1.705799in}}%
\pgfpathlineto{\pgfqpoint{3.696966in}{1.695343in}}%
\pgfpathlineto{\pgfqpoint{3.689083in}{1.684925in}}%
\pgfpathlineto{\pgfqpoint{3.681195in}{1.674548in}}%
\pgfpathclose%
\pgfusepath{fill}%
\end{pgfscope}%
\begin{pgfscope}%
\pgfpathrectangle{\pgfqpoint{1.254980in}{0.150000in}}{\pgfqpoint{5.490039in}{5.490039in}}%
\pgfusepath{clip}%
\pgfsetbuttcap%
\pgfsetroundjoin%
\definecolor{currentfill}{rgb}{0.281446,0.084320,0.407414}%
\pgfsetfillcolor{currentfill}%
\pgfsetfillopacity{0.700000}%
\pgfsetlinewidth{0.000000pt}%
\definecolor{currentstroke}{rgb}{0.000000,0.000000,0.000000}%
\pgfsetstrokecolor{currentstroke}%
\pgfsetdash{}{0pt}%
\pgfpathmoveto{\pgfqpoint{3.766030in}{1.717427in}}%
\pgfpathlineto{\pgfqpoint{3.779377in}{1.718129in}}%
\pgfpathlineto{\pgfqpoint{3.792732in}{1.718998in}}%
\pgfpathlineto{\pgfqpoint{3.806095in}{1.720032in}}%
\pgfpathlineto{\pgfqpoint{3.819465in}{1.721232in}}%
\pgfpathlineto{\pgfqpoint{3.827314in}{1.732433in}}%
\pgfpathlineto{\pgfqpoint{3.835158in}{1.743648in}}%
\pgfpathlineto{\pgfqpoint{3.842996in}{1.754874in}}%
\pgfpathlineto{\pgfqpoint{3.850830in}{1.766108in}}%
\pgfpathlineto{\pgfqpoint{3.837467in}{1.764578in}}%
\pgfpathlineto{\pgfqpoint{3.824112in}{1.763214in}}%
\pgfpathlineto{\pgfqpoint{3.810765in}{1.762016in}}%
\pgfpathlineto{\pgfqpoint{3.797427in}{1.760985in}}%
\pgfpathlineto{\pgfqpoint{3.789585in}{1.750070in}}%
\pgfpathlineto{\pgfqpoint{3.781738in}{1.739170in}}%
\pgfpathlineto{\pgfqpoint{3.773887in}{1.728288in}}%
\pgfpathlineto{\pgfqpoint{3.766030in}{1.717427in}}%
\pgfpathclose%
\pgfusepath{fill}%
\end{pgfscope}%
\begin{pgfscope}%
\pgfpathrectangle{\pgfqpoint{1.254980in}{0.150000in}}{\pgfqpoint{5.490039in}{5.490039in}}%
\pgfusepath{clip}%
\pgfsetbuttcap%
\pgfsetroundjoin%
\definecolor{currentfill}{rgb}{0.195860,0.395433,0.555276}%
\pgfsetfillcolor{currentfill}%
\pgfsetfillopacity{0.700000}%
\pgfsetlinewidth{0.000000pt}%
\definecolor{currentstroke}{rgb}{0.000000,0.000000,0.000000}%
\pgfsetstrokecolor{currentstroke}%
\pgfsetdash{}{0pt}%
\pgfpathmoveto{\pgfqpoint{2.309400in}{2.467648in}}%
\pgfpathlineto{\pgfqpoint{2.323070in}{2.446066in}}%
\pgfpathlineto{\pgfqpoint{2.336728in}{2.424768in}}%
\pgfpathlineto{\pgfqpoint{2.350373in}{2.403751in}}%
\pgfpathlineto{\pgfqpoint{2.364006in}{2.383012in}}%
\pgfpathlineto{\pgfqpoint{2.372786in}{2.379714in}}%
\pgfpathlineto{\pgfqpoint{2.381547in}{2.376719in}}%
\pgfpathlineto{\pgfqpoint{2.390287in}{2.374021in}}%
\pgfpathlineto{\pgfqpoint{2.399008in}{2.371614in}}%
\pgfpathlineto{\pgfqpoint{2.385427in}{2.391786in}}%
\pgfpathlineto{\pgfqpoint{2.371836in}{2.412236in}}%
\pgfpathlineto{\pgfqpoint{2.358232in}{2.432965in}}%
\pgfpathlineto{\pgfqpoint{2.344616in}{2.453977in}}%
\pgfpathlineto{\pgfqpoint{2.335843in}{2.456940in}}%
\pgfpathlineto{\pgfqpoint{2.327050in}{2.460202in}}%
\pgfpathlineto{\pgfqpoint{2.318235in}{2.463770in}}%
\pgfpathlineto{\pgfqpoint{2.309400in}{2.467648in}}%
\pgfpathclose%
\pgfusepath{fill}%
\end{pgfscope}%
\begin{pgfscope}%
\pgfpathrectangle{\pgfqpoint{1.254980in}{0.150000in}}{\pgfqpoint{5.490039in}{5.490039in}}%
\pgfusepath{clip}%
\pgfsetbuttcap%
\pgfsetroundjoin%
\definecolor{currentfill}{rgb}{0.221989,0.339161,0.548752}%
\pgfsetfillcolor{currentfill}%
\pgfsetfillopacity{0.700000}%
\pgfsetlinewidth{0.000000pt}%
\definecolor{currentstroke}{rgb}{0.000000,0.000000,0.000000}%
\pgfsetstrokecolor{currentstroke}%
\pgfsetdash{}{0pt}%
\pgfpathmoveto{\pgfqpoint{4.421756in}{2.241056in}}%
\pgfpathlineto{\pgfqpoint{4.435358in}{2.248299in}}%
\pgfpathlineto{\pgfqpoint{4.448974in}{2.255703in}}%
\pgfpathlineto{\pgfqpoint{4.462603in}{2.263268in}}%
\pgfpathlineto{\pgfqpoint{4.476246in}{2.270995in}}%
\pgfpathlineto{\pgfqpoint{4.483890in}{2.281927in}}%
\pgfpathlineto{\pgfqpoint{4.491529in}{2.292760in}}%
\pgfpathlineto{\pgfqpoint{4.499162in}{2.303493in}}%
\pgfpathlineto{\pgfqpoint{4.506790in}{2.314127in}}%
\pgfpathlineto{\pgfqpoint{4.493150in}{2.306323in}}%
\pgfpathlineto{\pgfqpoint{4.479524in}{2.298681in}}%
\pgfpathlineto{\pgfqpoint{4.465911in}{2.291200in}}%
\pgfpathlineto{\pgfqpoint{4.452312in}{2.283881in}}%
\pgfpathlineto{\pgfqpoint{4.444681in}{2.273313in}}%
\pgfpathlineto{\pgfqpoint{4.437045in}{2.262653in}}%
\pgfpathlineto{\pgfqpoint{4.429403in}{2.251900in}}%
\pgfpathlineto{\pgfqpoint{4.421756in}{2.241056in}}%
\pgfpathclose%
\pgfusepath{fill}%
\end{pgfscope}%
\begin{pgfscope}%
\pgfpathrectangle{\pgfqpoint{1.254980in}{0.150000in}}{\pgfqpoint{5.490039in}{5.490039in}}%
\pgfusepath{clip}%
\pgfsetbuttcap%
\pgfsetroundjoin%
\definecolor{currentfill}{rgb}{0.149039,0.508051,0.557250}%
\pgfsetfillcolor{currentfill}%
\pgfsetfillopacity{0.700000}%
\pgfsetlinewidth{0.000000pt}%
\definecolor{currentstroke}{rgb}{0.000000,0.000000,0.000000}%
\pgfsetstrokecolor{currentstroke}%
\pgfsetdash{}{0pt}%
\pgfpathmoveto{\pgfqpoint{4.907838in}{2.684059in}}%
\pgfpathlineto{\pgfqpoint{4.921708in}{2.694439in}}%
\pgfpathlineto{\pgfqpoint{4.935594in}{2.704981in}}%
\pgfpathlineto{\pgfqpoint{4.949497in}{2.715683in}}%
\pgfpathlineto{\pgfqpoint{4.963417in}{2.726546in}}%
\pgfpathlineto{\pgfqpoint{4.970855in}{2.734199in}}%
\pgfpathlineto{\pgfqpoint{4.978285in}{2.741734in}}%
\pgfpathlineto{\pgfqpoint{4.985707in}{2.749152in}}%
\pgfpathlineto{\pgfqpoint{4.993122in}{2.756454in}}%
\pgfpathlineto{\pgfqpoint{4.979210in}{2.745723in}}%
\pgfpathlineto{\pgfqpoint{4.965316in}{2.735153in}}%
\pgfpathlineto{\pgfqpoint{4.951438in}{2.724743in}}%
\pgfpathlineto{\pgfqpoint{4.937576in}{2.714493in}}%
\pgfpathlineto{\pgfqpoint{4.930153in}{2.707048in}}%
\pgfpathlineto{\pgfqpoint{4.922722in}{2.699495in}}%
\pgfpathlineto{\pgfqpoint{4.915284in}{2.691833in}}%
\pgfpathlineto{\pgfqpoint{4.907838in}{2.684059in}}%
\pgfpathclose%
\pgfusepath{fill}%
\end{pgfscope}%
\begin{pgfscope}%
\pgfpathrectangle{\pgfqpoint{1.254980in}{0.150000in}}{\pgfqpoint{5.490039in}{5.490039in}}%
\pgfusepath{clip}%
\pgfsetbuttcap%
\pgfsetroundjoin%
\definecolor{currentfill}{rgb}{0.239374,0.735588,0.455688}%
\pgfsetfillcolor{currentfill}%
\pgfsetfillopacity{0.700000}%
\pgfsetlinewidth{0.000000pt}%
\definecolor{currentstroke}{rgb}{0.000000,0.000000,0.000000}%
\pgfsetstrokecolor{currentstroke}%
\pgfsetdash{}{0pt}%
\pgfpathmoveto{\pgfqpoint{5.733387in}{3.316554in}}%
\pgfpathlineto{\pgfqpoint{5.747751in}{3.329544in}}%
\pgfpathlineto{\pgfqpoint{5.762136in}{3.342692in}}%
\pgfpathlineto{\pgfqpoint{5.776541in}{3.355999in}}%
\pgfpathlineto{\pgfqpoint{5.790968in}{3.369465in}}%
\pgfpathlineto{\pgfqpoint{5.797883in}{3.370475in}}%
\pgfpathlineto{\pgfqpoint{5.804790in}{3.371449in}}%
\pgfpathlineto{\pgfqpoint{5.811690in}{3.372394in}}%
\pgfpathlineto{\pgfqpoint{5.818581in}{3.373314in}}%
\pgfpathlineto{\pgfqpoint{5.804183in}{3.360320in}}%
\pgfpathlineto{\pgfqpoint{5.789806in}{3.347485in}}%
\pgfpathlineto{\pgfqpoint{5.775449in}{3.334807in}}%
\pgfpathlineto{\pgfqpoint{5.761112in}{3.322288in}}%
\pgfpathlineto{\pgfqpoint{5.754192in}{3.320886in}}%
\pgfpathlineto{\pgfqpoint{5.747264in}{3.319467in}}%
\pgfpathlineto{\pgfqpoint{5.740329in}{3.318024in}}%
\pgfpathlineto{\pgfqpoint{5.733387in}{3.316554in}}%
\pgfpathclose%
\pgfusepath{fill}%
\end{pgfscope}%
\begin{pgfscope}%
\pgfpathrectangle{\pgfqpoint{1.254980in}{0.150000in}}{\pgfqpoint{5.490039in}{5.490039in}}%
\pgfusepath{clip}%
\pgfsetbuttcap%
\pgfsetroundjoin%
\definecolor{currentfill}{rgb}{0.273809,0.031497,0.358853}%
\pgfsetfillcolor{currentfill}%
\pgfsetfillopacity{0.700000}%
\pgfsetlinewidth{0.000000pt}%
\definecolor{currentstroke}{rgb}{0.000000,0.000000,0.000000}%
\pgfsetstrokecolor{currentstroke}%
\pgfsetdash{}{0pt}%
\pgfpathmoveto{\pgfqpoint{3.596290in}{1.638034in}}%
\pgfpathlineto{\pgfqpoint{3.609605in}{1.636652in}}%
\pgfpathlineto{\pgfqpoint{3.622925in}{1.635438in}}%
\pgfpathlineto{\pgfqpoint{3.636252in}{1.634394in}}%
\pgfpathlineto{\pgfqpoint{3.649585in}{1.633518in}}%
\pgfpathlineto{\pgfqpoint{3.657496in}{1.643697in}}%
\pgfpathlineto{\pgfqpoint{3.665401in}{1.653930in}}%
\pgfpathlineto{\pgfqpoint{3.673301in}{1.664215in}}%
\pgfpathlineto{\pgfqpoint{3.681195in}{1.674548in}}%
\pgfpathlineto{\pgfqpoint{3.667873in}{1.675039in}}%
\pgfpathlineto{\pgfqpoint{3.654558in}{1.675699in}}%
\pgfpathlineto{\pgfqpoint{3.641250in}{1.676529in}}%
\pgfpathlineto{\pgfqpoint{3.627948in}{1.677527in}}%
\pgfpathlineto{\pgfqpoint{3.620042in}{1.667568in}}%
\pgfpathlineto{\pgfqpoint{3.612131in}{1.657664in}}%
\pgfpathlineto{\pgfqpoint{3.604214in}{1.647818in}}%
\pgfpathlineto{\pgfqpoint{3.596290in}{1.638034in}}%
\pgfpathclose%
\pgfusepath{fill}%
\end{pgfscope}%
\begin{pgfscope}%
\pgfpathrectangle{\pgfqpoint{1.254980in}{0.150000in}}{\pgfqpoint{5.490039in}{5.490039in}}%
\pgfusepath{clip}%
\pgfsetbuttcap%
\pgfsetroundjoin%
\definecolor{currentfill}{rgb}{0.283091,0.110553,0.431554}%
\pgfsetfillcolor{currentfill}%
\pgfsetfillopacity{0.700000}%
\pgfsetlinewidth{0.000000pt}%
\definecolor{currentstroke}{rgb}{0.000000,0.000000,0.000000}%
\pgfsetstrokecolor{currentstroke}%
\pgfsetdash{}{0pt}%
\pgfpathmoveto{\pgfqpoint{3.850830in}{1.766108in}}%
\pgfpathlineto{\pgfqpoint{3.864201in}{1.767803in}}%
\pgfpathlineto{\pgfqpoint{3.877581in}{1.769663in}}%
\pgfpathlineto{\pgfqpoint{3.890970in}{1.771688in}}%
\pgfpathlineto{\pgfqpoint{3.904368in}{1.773877in}}%
\pgfpathlineto{\pgfqpoint{3.912190in}{1.785429in}}%
\pgfpathlineto{\pgfqpoint{3.920007in}{1.796976in}}%
\pgfpathlineto{\pgfqpoint{3.927819in}{1.808516in}}%
\pgfpathlineto{\pgfqpoint{3.935627in}{1.820047in}}%
\pgfpathlineto{\pgfqpoint{3.922236in}{1.817555in}}%
\pgfpathlineto{\pgfqpoint{3.908853in}{1.815228in}}%
\pgfpathlineto{\pgfqpoint{3.895480in}{1.813066in}}%
\pgfpathlineto{\pgfqpoint{3.882116in}{1.811069in}}%
\pgfpathlineto{\pgfqpoint{3.874302in}{1.799830in}}%
\pgfpathlineto{\pgfqpoint{3.866483in}{1.788589in}}%
\pgfpathlineto{\pgfqpoint{3.858659in}{1.777347in}}%
\pgfpathlineto{\pgfqpoint{3.850830in}{1.766108in}}%
\pgfpathclose%
\pgfusepath{fill}%
\end{pgfscope}%
\begin{pgfscope}%
\pgfpathrectangle{\pgfqpoint{1.254980in}{0.150000in}}{\pgfqpoint{5.490039in}{5.490039in}}%
\pgfusepath{clip}%
\pgfsetbuttcap%
\pgfsetroundjoin%
\definecolor{currentfill}{rgb}{0.281446,0.084320,0.407414}%
\pgfsetfillcolor{currentfill}%
\pgfsetfillopacity{0.700000}%
\pgfsetlinewidth{0.000000pt}%
\definecolor{currentstroke}{rgb}{0.000000,0.000000,0.000000}%
\pgfsetstrokecolor{currentstroke}%
\pgfsetdash{}{0pt}%
\pgfpathmoveto{\pgfqpoint{2.902372in}{1.756902in}}%
\pgfpathlineto{\pgfqpoint{2.915724in}{1.745821in}}%
\pgfpathlineto{\pgfqpoint{2.929074in}{1.734943in}}%
\pgfpathlineto{\pgfqpoint{2.942422in}{1.724266in}}%
\pgfpathlineto{\pgfqpoint{2.955769in}{1.713789in}}%
\pgfpathlineto{\pgfqpoint{2.964073in}{1.716622in}}%
\pgfpathlineto{\pgfqpoint{2.972365in}{1.719670in}}%
\pgfpathlineto{\pgfqpoint{2.980644in}{1.722928in}}%
\pgfpathlineto{\pgfqpoint{2.988911in}{1.726392in}}%
\pgfpathlineto{\pgfqpoint{2.975598in}{1.736337in}}%
\pgfpathlineto{\pgfqpoint{2.962284in}{1.746483in}}%
\pgfpathlineto{\pgfqpoint{2.948968in}{1.756829in}}%
\pgfpathlineto{\pgfqpoint{2.935649in}{1.767377in}}%
\pgfpathlineto{\pgfqpoint{2.927350in}{1.764434in}}%
\pgfpathlineto{\pgfqpoint{2.919037in}{1.761703in}}%
\pgfpathlineto{\pgfqpoint{2.910711in}{1.759191in}}%
\pgfpathlineto{\pgfqpoint{2.902372in}{1.756902in}}%
\pgfpathclose%
\pgfusepath{fill}%
\end{pgfscope}%
\begin{pgfscope}%
\pgfpathrectangle{\pgfqpoint{1.254980in}{0.150000in}}{\pgfqpoint{5.490039in}{5.490039in}}%
\pgfusepath{clip}%
\pgfsetbuttcap%
\pgfsetroundjoin%
\definecolor{currentfill}{rgb}{0.319809,0.770914,0.411152}%
\pgfsetfillcolor{currentfill}%
\pgfsetfillopacity{0.700000}%
\pgfsetlinewidth{0.000000pt}%
\definecolor{currentstroke}{rgb}{0.000000,0.000000,0.000000}%
\pgfsetstrokecolor{currentstroke}%
\pgfsetdash{}{0pt}%
\pgfpathmoveto{\pgfqpoint{5.903753in}{3.428402in}}%
\pgfpathlineto{\pgfqpoint{5.918225in}{3.441684in}}%
\pgfpathlineto{\pgfqpoint{5.932719in}{3.455124in}}%
\pgfpathlineto{\pgfqpoint{5.947233in}{3.468724in}}%
\pgfpathlineto{\pgfqpoint{5.954033in}{3.468688in}}%
\pgfpathlineto{\pgfqpoint{5.960826in}{3.468648in}}%
\pgfpathlineto{\pgfqpoint{5.967612in}{3.468610in}}%
\pgfpathlineto{\pgfqpoint{5.974391in}{3.468580in}}%
\pgfpathlineto{\pgfqpoint{5.959910in}{3.455515in}}%
\pgfpathlineto{\pgfqpoint{5.945449in}{3.442608in}}%
\pgfpathlineto{\pgfqpoint{5.931010in}{3.429858in}}%
\pgfpathlineto{\pgfqpoint{5.924206in}{3.429481in}}%
\pgfpathlineto{\pgfqpoint{5.917395in}{3.429117in}}%
\pgfpathlineto{\pgfqpoint{5.910578in}{3.428759in}}%
\pgfpathlineto{\pgfqpoint{5.903753in}{3.428402in}}%
\pgfpathclose%
\pgfusepath{fill}%
\end{pgfscope}%
\begin{pgfscope}%
\pgfpathrectangle{\pgfqpoint{1.254980in}{0.150000in}}{\pgfqpoint{5.490039in}{5.490039in}}%
\pgfusepath{clip}%
\pgfsetbuttcap%
\pgfsetroundjoin%
\definecolor{currentfill}{rgb}{0.257322,0.256130,0.526563}%
\pgfsetfillcolor{currentfill}%
\pgfsetfillopacity{0.700000}%
\pgfsetlinewidth{0.000000pt}%
\definecolor{currentstroke}{rgb}{0.000000,0.000000,0.000000}%
\pgfsetstrokecolor{currentstroke}%
\pgfsetdash{}{0pt}%
\pgfpathmoveto{\pgfqpoint{4.221156in}{2.055107in}}%
\pgfpathlineto{\pgfqpoint{4.234667in}{2.060652in}}%
\pgfpathlineto{\pgfqpoint{4.248189in}{2.066359in}}%
\pgfpathlineto{\pgfqpoint{4.261723in}{2.072229in}}%
\pgfpathlineto{\pgfqpoint{4.275269in}{2.078259in}}%
\pgfpathlineto{\pgfqpoint{4.282981in}{2.089974in}}%
\pgfpathlineto{\pgfqpoint{4.290688in}{2.101615in}}%
\pgfpathlineto{\pgfqpoint{4.298390in}{2.113181in}}%
\pgfpathlineto{\pgfqpoint{4.306086in}{2.124670in}}%
\pgfpathlineto{\pgfqpoint{4.292543in}{2.118476in}}%
\pgfpathlineto{\pgfqpoint{4.279011in}{2.112444in}}%
\pgfpathlineto{\pgfqpoint{4.265492in}{2.106574in}}%
\pgfpathlineto{\pgfqpoint{4.251985in}{2.100866in}}%
\pgfpathlineto{\pgfqpoint{4.244285in}{2.089530in}}%
\pgfpathlineto{\pgfqpoint{4.236580in}{2.078123in}}%
\pgfpathlineto{\pgfqpoint{4.228871in}{2.066649in}}%
\pgfpathlineto{\pgfqpoint{4.221156in}{2.055107in}}%
\pgfpathclose%
\pgfusepath{fill}%
\end{pgfscope}%
\begin{pgfscope}%
\pgfpathrectangle{\pgfqpoint{1.254980in}{0.150000in}}{\pgfqpoint{5.490039in}{5.490039in}}%
\pgfusepath{clip}%
\pgfsetbuttcap%
\pgfsetroundjoin%
\definecolor{currentfill}{rgb}{0.281477,0.755203,0.432552}%
\pgfsetfillcolor{currentfill}%
\pgfsetfillopacity{0.700000}%
\pgfsetlinewidth{0.000000pt}%
\definecolor{currentstroke}{rgb}{0.000000,0.000000,0.000000}%
\pgfsetstrokecolor{currentstroke}%
\pgfsetdash{}{0pt}%
\pgfpathmoveto{\pgfqpoint{5.818581in}{3.373314in}}%
\pgfpathlineto{\pgfqpoint{5.833000in}{3.386466in}}%
\pgfpathlineto{\pgfqpoint{5.847440in}{3.399776in}}%
\pgfpathlineto{\pgfqpoint{5.861900in}{3.413245in}}%
\pgfpathlineto{\pgfqpoint{5.876382in}{3.426874in}}%
\pgfpathlineto{\pgfqpoint{5.883236in}{3.427282in}}%
\pgfpathlineto{\pgfqpoint{5.890083in}{3.427669in}}%
\pgfpathlineto{\pgfqpoint{5.896922in}{3.428040in}}%
\pgfpathlineto{\pgfqpoint{5.903753in}{3.428402in}}%
\pgfpathlineto{\pgfqpoint{5.889302in}{3.415278in}}%
\pgfpathlineto{\pgfqpoint{5.874872in}{3.402312in}}%
\pgfpathlineto{\pgfqpoint{5.860463in}{3.389503in}}%
\pgfpathlineto{\pgfqpoint{5.846074in}{3.376852in}}%
\pgfpathlineto{\pgfqpoint{5.839212in}{3.375978in}}%
\pgfpathlineto{\pgfqpoint{5.832342in}{3.375101in}}%
\pgfpathlineto{\pgfqpoint{5.825465in}{3.374214in}}%
\pgfpathlineto{\pgfqpoint{5.818581in}{3.373314in}}%
\pgfpathclose%
\pgfusepath{fill}%
\end{pgfscope}%
\begin{pgfscope}%
\pgfpathrectangle{\pgfqpoint{1.254980in}{0.150000in}}{\pgfqpoint{5.490039in}{5.490039in}}%
\pgfusepath{clip}%
\pgfsetbuttcap%
\pgfsetroundjoin%
\definecolor{currentfill}{rgb}{0.282623,0.140926,0.457517}%
\pgfsetfillcolor{currentfill}%
\pgfsetfillopacity{0.700000}%
\pgfsetlinewidth{0.000000pt}%
\definecolor{currentstroke}{rgb}{0.000000,0.000000,0.000000}%
\pgfsetstrokecolor{currentstroke}%
\pgfsetdash{}{0pt}%
\pgfpathmoveto{\pgfqpoint{3.935627in}{1.820047in}}%
\pgfpathlineto{\pgfqpoint{3.949027in}{1.822702in}}%
\pgfpathlineto{\pgfqpoint{3.962437in}{1.825522in}}%
\pgfpathlineto{\pgfqpoint{3.975856in}{1.828506in}}%
\pgfpathlineto{\pgfqpoint{3.989285in}{1.831653in}}%
\pgfpathlineto{\pgfqpoint{3.997083in}{1.843456in}}%
\pgfpathlineto{\pgfqpoint{4.004876in}{1.855237in}}%
\pgfpathlineto{\pgfqpoint{4.012664in}{1.866993in}}%
\pgfpathlineto{\pgfqpoint{4.020447in}{1.878723in}}%
\pgfpathlineto{\pgfqpoint{4.007023in}{1.875301in}}%
\pgfpathlineto{\pgfqpoint{3.993609in}{1.872042in}}%
\pgfpathlineto{\pgfqpoint{3.980205in}{1.868948in}}%
\pgfpathlineto{\pgfqpoint{3.966810in}{1.866017in}}%
\pgfpathlineto{\pgfqpoint{3.959021in}{1.854552in}}%
\pgfpathlineto{\pgfqpoint{3.951228in}{1.843067in}}%
\pgfpathlineto{\pgfqpoint{3.943430in}{1.831564in}}%
\pgfpathlineto{\pgfqpoint{3.935627in}{1.820047in}}%
\pgfpathclose%
\pgfusepath{fill}%
\end{pgfscope}%
\begin{pgfscope}%
\pgfpathrectangle{\pgfqpoint{1.254980in}{0.150000in}}{\pgfqpoint{5.490039in}{5.490039in}}%
\pgfusepath{clip}%
\pgfsetbuttcap%
\pgfsetroundjoin%
\definecolor{currentfill}{rgb}{0.120081,0.622161,0.534946}%
\pgfsetfillcolor{currentfill}%
\pgfsetfillopacity{0.700000}%
\pgfsetlinewidth{0.000000pt}%
\definecolor{currentstroke}{rgb}{0.000000,0.000000,0.000000}%
\pgfsetstrokecolor{currentstroke}%
\pgfsetdash{}{0pt}%
\pgfpathmoveto{\pgfqpoint{5.278483in}{2.990696in}}%
\pgfpathlineto{\pgfqpoint{5.292579in}{3.002677in}}%
\pgfpathlineto{\pgfqpoint{5.306694in}{3.014818in}}%
\pgfpathlineto{\pgfqpoint{5.320828in}{3.027119in}}%
\pgfpathlineto{\pgfqpoint{5.334981in}{3.039580in}}%
\pgfpathlineto{\pgfqpoint{5.342210in}{3.044185in}}%
\pgfpathlineto{\pgfqpoint{5.349430in}{3.048692in}}%
\pgfpathlineto{\pgfqpoint{5.356641in}{3.053102in}}%
\pgfpathlineto{\pgfqpoint{5.363844in}{3.057420in}}%
\pgfpathlineto{\pgfqpoint{5.349707in}{3.045245in}}%
\pgfpathlineto{\pgfqpoint{5.335589in}{3.033230in}}%
\pgfpathlineto{\pgfqpoint{5.321490in}{3.021375in}}%
\pgfpathlineto{\pgfqpoint{5.307409in}{3.009678in}}%
\pgfpathlineto{\pgfqpoint{5.300190in}{3.005064in}}%
\pgfpathlineto{\pgfqpoint{5.292963in}{3.000364in}}%
\pgfpathlineto{\pgfqpoint{5.285727in}{2.995576in}}%
\pgfpathlineto{\pgfqpoint{5.278483in}{2.990696in}}%
\pgfpathclose%
\pgfusepath{fill}%
\end{pgfscope}%
\begin{pgfscope}%
\pgfpathrectangle{\pgfqpoint{1.254980in}{0.150000in}}{\pgfqpoint{5.490039in}{5.490039in}}%
\pgfusepath{clip}%
\pgfsetbuttcap%
\pgfsetroundjoin%
\definecolor{currentfill}{rgb}{0.269944,0.014625,0.341379}%
\pgfsetfillcolor{currentfill}%
\pgfsetfillopacity{0.700000}%
\pgfsetlinewidth{0.000000pt}%
\definecolor{currentstroke}{rgb}{0.000000,0.000000,0.000000}%
\pgfsetstrokecolor{currentstroke}%
\pgfsetdash{}{0pt}%
\pgfpathmoveto{\pgfqpoint{3.511277in}{1.608475in}}%
\pgfpathlineto{\pgfqpoint{3.524584in}{1.605998in}}%
\pgfpathlineto{\pgfqpoint{3.537895in}{1.603693in}}%
\pgfpathlineto{\pgfqpoint{3.551212in}{1.601558in}}%
\pgfpathlineto{\pgfqpoint{3.564534in}{1.599594in}}%
\pgfpathlineto{\pgfqpoint{3.572483in}{1.609092in}}%
\pgfpathlineto{\pgfqpoint{3.580425in}{1.618667in}}%
\pgfpathlineto{\pgfqpoint{3.588361in}{1.628316in}}%
\pgfpathlineto{\pgfqpoint{3.596290in}{1.638034in}}%
\pgfpathlineto{\pgfqpoint{3.582982in}{1.639587in}}%
\pgfpathlineto{\pgfqpoint{3.569679in}{1.641310in}}%
\pgfpathlineto{\pgfqpoint{3.556381in}{1.643204in}}%
\pgfpathlineto{\pgfqpoint{3.543089in}{1.645269in}}%
\pgfpathlineto{\pgfqpoint{3.535146in}{1.635952in}}%
\pgfpathlineto{\pgfqpoint{3.527196in}{1.626712in}}%
\pgfpathlineto{\pgfqpoint{3.519240in}{1.617551in}}%
\pgfpathlineto{\pgfqpoint{3.511277in}{1.608475in}}%
\pgfpathclose%
\pgfusepath{fill}%
\end{pgfscope}%
\begin{pgfscope}%
\pgfpathrectangle{\pgfqpoint{1.254980in}{0.150000in}}{\pgfqpoint{5.490039in}{5.490039in}}%
\pgfusepath{clip}%
\pgfsetbuttcap%
\pgfsetroundjoin%
\definecolor{currentfill}{rgb}{0.174274,0.445044,0.557792}%
\pgfsetfillcolor{currentfill}%
\pgfsetfillopacity{0.700000}%
\pgfsetlinewidth{0.000000pt}%
\definecolor{currentstroke}{rgb}{0.000000,0.000000,0.000000}%
\pgfsetstrokecolor{currentstroke}%
\pgfsetdash{}{0pt}%
\pgfpathmoveto{\pgfqpoint{4.707410in}{2.502581in}}%
\pgfpathlineto{\pgfqpoint{4.721170in}{2.511890in}}%
\pgfpathlineto{\pgfqpoint{4.734946in}{2.521359in}}%
\pgfpathlineto{\pgfqpoint{4.748737in}{2.530989in}}%
\pgfpathlineto{\pgfqpoint{4.762543in}{2.540781in}}%
\pgfpathlineto{\pgfqpoint{4.770080in}{2.550029in}}%
\pgfpathlineto{\pgfqpoint{4.777609in}{2.559159in}}%
\pgfpathlineto{\pgfqpoint{4.785131in}{2.568171in}}%
\pgfpathlineto{\pgfqpoint{4.792647in}{2.577065in}}%
\pgfpathlineto{\pgfqpoint{4.778845in}{2.567316in}}%
\pgfpathlineto{\pgfqpoint{4.765059in}{2.557727in}}%
\pgfpathlineto{\pgfqpoint{4.751289in}{2.548299in}}%
\pgfpathlineto{\pgfqpoint{4.737534in}{2.539031in}}%
\pgfpathlineto{\pgfqpoint{4.730013in}{2.530084in}}%
\pgfpathlineto{\pgfqpoint{4.722485in}{2.521027in}}%
\pgfpathlineto{\pgfqpoint{4.714951in}{2.511860in}}%
\pgfpathlineto{\pgfqpoint{4.707410in}{2.502581in}}%
\pgfpathclose%
\pgfusepath{fill}%
\end{pgfscope}%
\begin{pgfscope}%
\pgfpathrectangle{\pgfqpoint{1.254980in}{0.150000in}}{\pgfqpoint{5.490039in}{5.490039in}}%
\pgfusepath{clip}%
\pgfsetbuttcap%
\pgfsetroundjoin%
\definecolor{currentfill}{rgb}{0.267004,0.004874,0.329415}%
\pgfsetfillcolor{currentfill}%
\pgfsetfillopacity{0.700000}%
\pgfsetlinewidth{0.000000pt}%
\definecolor{currentstroke}{rgb}{0.000000,0.000000,0.000000}%
\pgfsetstrokecolor{currentstroke}%
\pgfsetdash{}{0pt}%
\pgfpathmoveto{\pgfqpoint{3.287535in}{1.593587in}}%
\pgfpathlineto{\pgfqpoint{3.300834in}{1.588099in}}%
\pgfpathlineto{\pgfqpoint{3.314135in}{1.582790in}}%
\pgfpathlineto{\pgfqpoint{3.327438in}{1.577659in}}%
\pgfpathlineto{\pgfqpoint{3.340745in}{1.572706in}}%
\pgfpathlineto{\pgfqpoint{3.348804in}{1.580002in}}%
\pgfpathlineto{\pgfqpoint{3.356854in}{1.587430in}}%
\pgfpathlineto{\pgfqpoint{3.364896in}{1.594985in}}%
\pgfpathlineto{\pgfqpoint{3.372931in}{1.602665in}}%
\pgfpathlineto{\pgfqpoint{3.359644in}{1.607150in}}%
\pgfpathlineto{\pgfqpoint{3.346361in}{1.611813in}}%
\pgfpathlineto{\pgfqpoint{3.333081in}{1.616653in}}%
\pgfpathlineto{\pgfqpoint{3.319804in}{1.621673in}}%
\pgfpathlineto{\pgfqpoint{3.311750in}{1.614451in}}%
\pgfpathlineto{\pgfqpoint{3.303687in}{1.607360in}}%
\pgfpathlineto{\pgfqpoint{3.295615in}{1.600404in}}%
\pgfpathlineto{\pgfqpoint{3.287535in}{1.593587in}}%
\pgfpathclose%
\pgfusepath{fill}%
\end{pgfscope}%
\begin{pgfscope}%
\pgfpathrectangle{\pgfqpoint{1.254980in}{0.150000in}}{\pgfqpoint{5.490039in}{5.490039in}}%
\pgfusepath{clip}%
\pgfsetbuttcap%
\pgfsetroundjoin%
\definecolor{currentfill}{rgb}{0.180629,0.429975,0.557282}%
\pgfsetfillcolor{currentfill}%
\pgfsetfillopacity{0.700000}%
\pgfsetlinewidth{0.000000pt}%
\definecolor{currentstroke}{rgb}{0.000000,0.000000,0.000000}%
\pgfsetstrokecolor{currentstroke}%
\pgfsetdash{}{0pt}%
\pgfpathmoveto{\pgfqpoint{2.254585in}{2.556866in}}%
\pgfpathlineto{\pgfqpoint{2.268310in}{2.534123in}}%
\pgfpathlineto{\pgfqpoint{2.282020in}{2.511674in}}%
\pgfpathlineto{\pgfqpoint{2.295717in}{2.489516in}}%
\pgfpathlineto{\pgfqpoint{2.309400in}{2.467648in}}%
\pgfpathlineto{\pgfqpoint{2.318235in}{2.463770in}}%
\pgfpathlineto{\pgfqpoint{2.327050in}{2.460202in}}%
\pgfpathlineto{\pgfqpoint{2.335843in}{2.456940in}}%
\pgfpathlineto{\pgfqpoint{2.344616in}{2.453977in}}%
\pgfpathlineto{\pgfqpoint{2.330987in}{2.475274in}}%
\pgfpathlineto{\pgfqpoint{2.317346in}{2.496858in}}%
\pgfpathlineto{\pgfqpoint{2.303691in}{2.518733in}}%
\pgfpathlineto{\pgfqpoint{2.290023in}{2.540902in}}%
\pgfpathlineto{\pgfqpoint{2.281196in}{2.544425in}}%
\pgfpathlineto{\pgfqpoint{2.272347in}{2.548257in}}%
\pgfpathlineto{\pgfqpoint{2.263477in}{2.552402in}}%
\pgfpathlineto{\pgfqpoint{2.254585in}{2.556866in}}%
\pgfpathclose%
\pgfusepath{fill}%
\end{pgfscope}%
\begin{pgfscope}%
\pgfpathrectangle{\pgfqpoint{1.254980in}{0.150000in}}{\pgfqpoint{5.490039in}{5.490039in}}%
\pgfusepath{clip}%
\pgfsetbuttcap%
\pgfsetroundjoin%
\definecolor{currentfill}{rgb}{0.269944,0.014625,0.341379}%
\pgfsetfillcolor{currentfill}%
\pgfsetfillopacity{0.700000}%
\pgfsetlinewidth{0.000000pt}%
\definecolor{currentstroke}{rgb}{0.000000,0.000000,0.000000}%
\pgfsetstrokecolor{currentstroke}%
\pgfsetdash{}{0pt}%
\pgfpathmoveto{\pgfqpoint{3.148614in}{1.622227in}}%
\pgfpathlineto{\pgfqpoint{3.161923in}{1.614781in}}%
\pgfpathlineto{\pgfqpoint{3.175234in}{1.607521in}}%
\pgfpathlineto{\pgfqpoint{3.188545in}{1.600445in}}%
\pgfpathlineto{\pgfqpoint{3.201858in}{1.593553in}}%
\pgfpathlineto{\pgfqpoint{3.209999in}{1.599257in}}%
\pgfpathlineto{\pgfqpoint{3.218130in}{1.605127in}}%
\pgfpathlineto{\pgfqpoint{3.226252in}{1.611157in}}%
\pgfpathlineto{\pgfqpoint{3.234364in}{1.617343in}}%
\pgfpathlineto{\pgfqpoint{3.221077in}{1.623738in}}%
\pgfpathlineto{\pgfqpoint{3.207790in}{1.630316in}}%
\pgfpathlineto{\pgfqpoint{3.194506in}{1.637079in}}%
\pgfpathlineto{\pgfqpoint{3.181222in}{1.644027in}}%
\pgfpathlineto{\pgfqpoint{3.173085in}{1.638327in}}%
\pgfpathlineto{\pgfqpoint{3.164938in}{1.632790in}}%
\pgfpathlineto{\pgfqpoint{3.156781in}{1.627422in}}%
\pgfpathlineto{\pgfqpoint{3.148614in}{1.622227in}}%
\pgfpathclose%
\pgfusepath{fill}%
\end{pgfscope}%
\begin{pgfscope}%
\pgfpathrectangle{\pgfqpoint{1.254980in}{0.150000in}}{\pgfqpoint{5.490039in}{5.490039in}}%
\pgfusepath{clip}%
\pgfsetbuttcap%
\pgfsetroundjoin%
\definecolor{currentfill}{rgb}{0.279566,0.067836,0.391917}%
\pgfsetfillcolor{currentfill}%
\pgfsetfillopacity{0.700000}%
\pgfsetlinewidth{0.000000pt}%
\definecolor{currentstroke}{rgb}{0.000000,0.000000,0.000000}%
\pgfsetstrokecolor{currentstroke}%
\pgfsetdash{}{0pt}%
\pgfpathmoveto{\pgfqpoint{2.955769in}{1.713789in}}%
\pgfpathlineto{\pgfqpoint{2.969113in}{1.703511in}}%
\pgfpathlineto{\pgfqpoint{2.982456in}{1.693431in}}%
\pgfpathlineto{\pgfqpoint{2.995797in}{1.683548in}}%
\pgfpathlineto{\pgfqpoint{3.009137in}{1.673860in}}%
\pgfpathlineto{\pgfqpoint{3.017409in}{1.677234in}}%
\pgfpathlineto{\pgfqpoint{3.025668in}{1.680816in}}%
\pgfpathlineto{\pgfqpoint{3.033916in}{1.684601in}}%
\pgfpathlineto{\pgfqpoint{3.042152in}{1.688584in}}%
\pgfpathlineto{\pgfqpoint{3.028843in}{1.697742in}}%
\pgfpathlineto{\pgfqpoint{3.015533in}{1.707096in}}%
\pgfpathlineto{\pgfqpoint{3.002223in}{1.716645in}}%
\pgfpathlineto{\pgfqpoint{2.988911in}{1.726392in}}%
\pgfpathlineto{\pgfqpoint{2.980644in}{1.722928in}}%
\pgfpathlineto{\pgfqpoint{2.972365in}{1.719670in}}%
\pgfpathlineto{\pgfqpoint{2.964073in}{1.716622in}}%
\pgfpathlineto{\pgfqpoint{2.955769in}{1.713789in}}%
\pgfpathclose%
\pgfusepath{fill}%
\end{pgfscope}%
\begin{pgfscope}%
\pgfpathrectangle{\pgfqpoint{1.254980in}{0.150000in}}{\pgfqpoint{5.490039in}{5.490039in}}%
\pgfusepath{clip}%
\pgfsetbuttcap%
\pgfsetroundjoin%
\definecolor{currentfill}{rgb}{0.137770,0.537492,0.554906}%
\pgfsetfillcolor{currentfill}%
\pgfsetfillopacity{0.700000}%
\pgfsetlinewidth{0.000000pt}%
\definecolor{currentstroke}{rgb}{0.000000,0.000000,0.000000}%
\pgfsetstrokecolor{currentstroke}%
\pgfsetdash{}{0pt}%
\pgfpathmoveto{\pgfqpoint{4.993122in}{2.756454in}}%
\pgfpathlineto{\pgfqpoint{5.007051in}{2.767345in}}%
\pgfpathlineto{\pgfqpoint{5.020997in}{2.778397in}}%
\pgfpathlineto{\pgfqpoint{5.034960in}{2.789610in}}%
\pgfpathlineto{\pgfqpoint{5.048940in}{2.800983in}}%
\pgfpathlineto{\pgfqpoint{5.056339in}{2.808021in}}%
\pgfpathlineto{\pgfqpoint{5.063729in}{2.814939in}}%
\pgfpathlineto{\pgfqpoint{5.071112in}{2.821741in}}%
\pgfpathlineto{\pgfqpoint{5.078486in}{2.828428in}}%
\pgfpathlineto{\pgfqpoint{5.064515in}{2.817217in}}%
\pgfpathlineto{\pgfqpoint{5.050561in}{2.806167in}}%
\pgfpathlineto{\pgfqpoint{5.036625in}{2.795278in}}%
\pgfpathlineto{\pgfqpoint{5.022705in}{2.784548in}}%
\pgfpathlineto{\pgfqpoint{5.015321in}{2.777688in}}%
\pgfpathlineto{\pgfqpoint{5.007929in}{2.770720in}}%
\pgfpathlineto{\pgfqpoint{5.000529in}{2.763643in}}%
\pgfpathlineto{\pgfqpoint{4.993122in}{2.756454in}}%
\pgfpathclose%
\pgfusepath{fill}%
\end{pgfscope}%
\begin{pgfscope}%
\pgfpathrectangle{\pgfqpoint{1.254980in}{0.150000in}}{\pgfqpoint{5.490039in}{5.490039in}}%
\pgfusepath{clip}%
\pgfsetbuttcap%
\pgfsetroundjoin%
\definecolor{currentfill}{rgb}{0.206756,0.371758,0.553117}%
\pgfsetfillcolor{currentfill}%
\pgfsetfillopacity{0.700000}%
\pgfsetlinewidth{0.000000pt}%
\definecolor{currentstroke}{rgb}{0.000000,0.000000,0.000000}%
\pgfsetstrokecolor{currentstroke}%
\pgfsetdash{}{0pt}%
\pgfpathmoveto{\pgfqpoint{4.506790in}{2.314127in}}%
\pgfpathlineto{\pgfqpoint{4.520444in}{2.322092in}}%
\pgfpathlineto{\pgfqpoint{4.534112in}{2.330218in}}%
\pgfpathlineto{\pgfqpoint{4.547794in}{2.338506in}}%
\pgfpathlineto{\pgfqpoint{4.561491in}{2.346955in}}%
\pgfpathlineto{\pgfqpoint{4.569110in}{2.357546in}}%
\pgfpathlineto{\pgfqpoint{4.576723in}{2.368030in}}%
\pgfpathlineto{\pgfqpoint{4.584330in}{2.378406in}}%
\pgfpathlineto{\pgfqpoint{4.591931in}{2.388674in}}%
\pgfpathlineto{\pgfqpoint{4.578238in}{2.380178in}}%
\pgfpathlineto{\pgfqpoint{4.564559in}{2.371843in}}%
\pgfpathlineto{\pgfqpoint{4.550894in}{2.363669in}}%
\pgfpathlineto{\pgfqpoint{4.537243in}{2.355656in}}%
\pgfpathlineto{\pgfqpoint{4.529639in}{2.345425in}}%
\pgfpathlineto{\pgfqpoint{4.522028in}{2.335092in}}%
\pgfpathlineto{\pgfqpoint{4.514412in}{2.324660in}}%
\pgfpathlineto{\pgfqpoint{4.506790in}{2.314127in}}%
\pgfpathclose%
\pgfusepath{fill}%
\end{pgfscope}%
\begin{pgfscope}%
\pgfpathrectangle{\pgfqpoint{1.254980in}{0.150000in}}{\pgfqpoint{5.490039in}{5.490039in}}%
\pgfusepath{clip}%
\pgfsetbuttcap%
\pgfsetroundjoin%
\definecolor{currentfill}{rgb}{0.278826,0.175490,0.483397}%
\pgfsetfillcolor{currentfill}%
\pgfsetfillopacity{0.700000}%
\pgfsetlinewidth{0.000000pt}%
\definecolor{currentstroke}{rgb}{0.000000,0.000000,0.000000}%
\pgfsetstrokecolor{currentstroke}%
\pgfsetdash{}{0pt}%
\pgfpathmoveto{\pgfqpoint{4.020447in}{1.878723in}}%
\pgfpathlineto{\pgfqpoint{4.033881in}{1.882308in}}%
\pgfpathlineto{\pgfqpoint{4.047325in}{1.886057in}}%
\pgfpathlineto{\pgfqpoint{4.060780in}{1.889969in}}%
\pgfpathlineto{\pgfqpoint{4.074244in}{1.894043in}}%
\pgfpathlineto{\pgfqpoint{4.082019in}{1.906001in}}%
\pgfpathlineto{\pgfqpoint{4.089789in}{1.917921in}}%
\pgfpathlineto{\pgfqpoint{4.097554in}{1.929799in}}%
\pgfpathlineto{\pgfqpoint{4.105315in}{1.941635in}}%
\pgfpathlineto{\pgfqpoint{4.091854in}{1.937313in}}%
\pgfpathlineto{\pgfqpoint{4.078404in}{1.933154in}}%
\pgfpathlineto{\pgfqpoint{4.064964in}{1.929158in}}%
\pgfpathlineto{\pgfqpoint{4.051535in}{1.925326in}}%
\pgfpathlineto{\pgfqpoint{4.043770in}{1.913727in}}%
\pgfpathlineto{\pgfqpoint{4.036000in}{1.902092in}}%
\pgfpathlineto{\pgfqpoint{4.028226in}{1.890423in}}%
\pgfpathlineto{\pgfqpoint{4.020447in}{1.878723in}}%
\pgfpathclose%
\pgfusepath{fill}%
\end{pgfscope}%
\begin{pgfscope}%
\pgfpathrectangle{\pgfqpoint{1.254980in}{0.150000in}}{\pgfqpoint{5.490039in}{5.490039in}}%
\pgfusepath{clip}%
\pgfsetbuttcap%
\pgfsetroundjoin%
\definecolor{currentfill}{rgb}{0.268510,0.009605,0.335427}%
\pgfsetfillcolor{currentfill}%
\pgfsetfillopacity{0.700000}%
\pgfsetlinewidth{0.000000pt}%
\definecolor{currentstroke}{rgb}{0.000000,0.000000,0.000000}%
\pgfsetstrokecolor{currentstroke}%
\pgfsetdash{}{0pt}%
\pgfpathmoveto{\pgfqpoint{3.426111in}{1.586485in}}%
\pgfpathlineto{\pgfqpoint{3.439416in}{1.582878in}}%
\pgfpathlineto{\pgfqpoint{3.452725in}{1.579444in}}%
\pgfpathlineto{\pgfqpoint{3.466039in}{1.576183in}}%
\pgfpathlineto{\pgfqpoint{3.479356in}{1.573095in}}%
\pgfpathlineto{\pgfqpoint{3.487347in}{1.581793in}}%
\pgfpathlineto{\pgfqpoint{3.495331in}{1.590592in}}%
\pgfpathlineto{\pgfqpoint{3.503307in}{1.599488in}}%
\pgfpathlineto{\pgfqpoint{3.511277in}{1.608475in}}%
\pgfpathlineto{\pgfqpoint{3.497975in}{1.611124in}}%
\pgfpathlineto{\pgfqpoint{3.484678in}{1.613946in}}%
\pgfpathlineto{\pgfqpoint{3.471386in}{1.616940in}}%
\pgfpathlineto{\pgfqpoint{3.458099in}{1.620108in}}%
\pgfpathlineto{\pgfqpoint{3.450113in}{1.611549in}}%
\pgfpathlineto{\pgfqpoint{3.442120in}{1.603090in}}%
\pgfpathlineto{\pgfqpoint{3.434119in}{1.594734in}}%
\pgfpathlineto{\pgfqpoint{3.426111in}{1.586485in}}%
\pgfpathclose%
\pgfusepath{fill}%
\end{pgfscope}%
\begin{pgfscope}%
\pgfpathrectangle{\pgfqpoint{1.254980in}{0.150000in}}{\pgfqpoint{5.490039in}{5.490039in}}%
\pgfusepath{clip}%
\pgfsetbuttcap%
\pgfsetroundjoin%
\definecolor{currentfill}{rgb}{0.128087,0.647749,0.523491}%
\pgfsetfillcolor{currentfill}%
\pgfsetfillopacity{0.700000}%
\pgfsetlinewidth{0.000000pt}%
\definecolor{currentstroke}{rgb}{0.000000,0.000000,0.000000}%
\pgfsetstrokecolor{currentstroke}%
\pgfsetdash{}{0pt}%
\pgfpathmoveto{\pgfqpoint{5.363844in}{3.057420in}}%
\pgfpathlineto{\pgfqpoint{5.378000in}{3.069755in}}%
\pgfpathlineto{\pgfqpoint{5.392175in}{3.082250in}}%
\pgfpathlineto{\pgfqpoint{5.406369in}{3.094905in}}%
\pgfpathlineto{\pgfqpoint{5.420583in}{3.107720in}}%
\pgfpathlineto{\pgfqpoint{5.427760in}{3.111643in}}%
\pgfpathlineto{\pgfqpoint{5.434929in}{3.115473in}}%
\pgfpathlineto{\pgfqpoint{5.442089in}{3.119214in}}%
\pgfpathlineto{\pgfqpoint{5.449241in}{3.122870in}}%
\pgfpathlineto{\pgfqpoint{5.435045in}{3.110373in}}%
\pgfpathlineto{\pgfqpoint{5.420868in}{3.098035in}}%
\pgfpathlineto{\pgfqpoint{5.406711in}{3.085857in}}%
\pgfpathlineto{\pgfqpoint{5.392572in}{3.073838in}}%
\pgfpathlineto{\pgfqpoint{5.385403in}{3.069855in}}%
\pgfpathlineto{\pgfqpoint{5.378225in}{3.065793in}}%
\pgfpathlineto{\pgfqpoint{5.371039in}{3.061649in}}%
\pgfpathlineto{\pgfqpoint{5.363844in}{3.057420in}}%
\pgfpathclose%
\pgfusepath{fill}%
\end{pgfscope}%
\begin{pgfscope}%
\pgfpathrectangle{\pgfqpoint{1.254980in}{0.150000in}}{\pgfqpoint{5.490039in}{5.490039in}}%
\pgfusepath{clip}%
\pgfsetbuttcap%
\pgfsetroundjoin%
\definecolor{currentfill}{rgb}{0.243113,0.292092,0.538516}%
\pgfsetfillcolor{currentfill}%
\pgfsetfillopacity{0.700000}%
\pgfsetlinewidth{0.000000pt}%
\definecolor{currentstroke}{rgb}{0.000000,0.000000,0.000000}%
\pgfsetstrokecolor{currentstroke}%
\pgfsetdash{}{0pt}%
\pgfpathmoveto{\pgfqpoint{4.306086in}{2.124670in}}%
\pgfpathlineto{\pgfqpoint{4.319643in}{2.131026in}}%
\pgfpathlineto{\pgfqpoint{4.333211in}{2.137543in}}%
\pgfpathlineto{\pgfqpoint{4.346792in}{2.144222in}}%
\pgfpathlineto{\pgfqpoint{4.360386in}{2.151063in}}%
\pgfpathlineto{\pgfqpoint{4.368076in}{2.162620in}}%
\pgfpathlineto{\pgfqpoint{4.375760in}{2.174090in}}%
\pgfpathlineto{\pgfqpoint{4.383439in}{2.185474in}}%
\pgfpathlineto{\pgfqpoint{4.391113in}{2.196769in}}%
\pgfpathlineto{\pgfqpoint{4.377521in}{2.189794in}}%
\pgfpathlineto{\pgfqpoint{4.363943in}{2.182980in}}%
\pgfpathlineto{\pgfqpoint{4.350377in}{2.176328in}}%
\pgfpathlineto{\pgfqpoint{4.336823in}{2.169838in}}%
\pgfpathlineto{\pgfqpoint{4.329147in}{2.158666in}}%
\pgfpathlineto{\pgfqpoint{4.321465in}{2.147413in}}%
\pgfpathlineto{\pgfqpoint{4.313778in}{2.136081in}}%
\pgfpathlineto{\pgfqpoint{4.306086in}{2.124670in}}%
\pgfpathclose%
\pgfusepath{fill}%
\end{pgfscope}%
\begin{pgfscope}%
\pgfpathrectangle{\pgfqpoint{1.254980in}{0.150000in}}{\pgfqpoint{5.490039in}{5.490039in}}%
\pgfusepath{clip}%
\pgfsetbuttcap%
\pgfsetroundjoin%
\definecolor{currentfill}{rgb}{0.162142,0.474838,0.558140}%
\pgfsetfillcolor{currentfill}%
\pgfsetfillopacity{0.700000}%
\pgfsetlinewidth{0.000000pt}%
\definecolor{currentstroke}{rgb}{0.000000,0.000000,0.000000}%
\pgfsetstrokecolor{currentstroke}%
\pgfsetdash{}{0pt}%
\pgfpathmoveto{\pgfqpoint{4.792647in}{2.577065in}}%
\pgfpathlineto{\pgfqpoint{4.806465in}{2.586976in}}%
\pgfpathlineto{\pgfqpoint{4.820298in}{2.597048in}}%
\pgfpathlineto{\pgfqpoint{4.834148in}{2.607280in}}%
\pgfpathlineto{\pgfqpoint{4.848014in}{2.617674in}}%
\pgfpathlineto{\pgfqpoint{4.855517in}{2.626392in}}%
\pgfpathlineto{\pgfqpoint{4.863013in}{2.634988in}}%
\pgfpathlineto{\pgfqpoint{4.870502in}{2.643462in}}%
\pgfpathlineto{\pgfqpoint{4.877984in}{2.651817in}}%
\pgfpathlineto{\pgfqpoint{4.864124in}{2.641495in}}%
\pgfpathlineto{\pgfqpoint{4.850280in}{2.631334in}}%
\pgfpathlineto{\pgfqpoint{4.836452in}{2.621334in}}%
\pgfpathlineto{\pgfqpoint{4.822641in}{2.611494in}}%
\pgfpathlineto{\pgfqpoint{4.815153in}{2.603058in}}%
\pgfpathlineto{\pgfqpoint{4.807658in}{2.594508in}}%
\pgfpathlineto{\pgfqpoint{4.800156in}{2.585844in}}%
\pgfpathlineto{\pgfqpoint{4.792647in}{2.577065in}}%
\pgfpathclose%
\pgfusepath{fill}%
\end{pgfscope}%
\begin{pgfscope}%
\pgfpathrectangle{\pgfqpoint{1.254980in}{0.150000in}}{\pgfqpoint{5.490039in}{5.490039in}}%
\pgfusepath{clip}%
\pgfsetbuttcap%
\pgfsetroundjoin%
\definecolor{currentfill}{rgb}{0.271828,0.209303,0.504434}%
\pgfsetfillcolor{currentfill}%
\pgfsetfillopacity{0.700000}%
\pgfsetlinewidth{0.000000pt}%
\definecolor{currentstroke}{rgb}{0.000000,0.000000,0.000000}%
\pgfsetstrokecolor{currentstroke}%
\pgfsetdash{}{0pt}%
\pgfpathmoveto{\pgfqpoint{4.105315in}{1.941635in}}%
\pgfpathlineto{\pgfqpoint{4.118787in}{1.946120in}}%
\pgfpathlineto{\pgfqpoint{4.132269in}{1.950767in}}%
\pgfpathlineto{\pgfqpoint{4.145763in}{1.955577in}}%
\pgfpathlineto{\pgfqpoint{4.159267in}{1.960548in}}%
\pgfpathlineto{\pgfqpoint{4.167020in}{1.972569in}}%
\pgfpathlineto{\pgfqpoint{4.174768in}{1.984536in}}%
\pgfpathlineto{\pgfqpoint{4.182511in}{1.996448in}}%
\pgfpathlineto{\pgfqpoint{4.190250in}{2.008302in}}%
\pgfpathlineto{\pgfqpoint{4.176748in}{2.003110in}}%
\pgfpathlineto{\pgfqpoint{4.163258in}{1.998081in}}%
\pgfpathlineto{\pgfqpoint{4.149779in}{1.993215in}}%
\pgfpathlineto{\pgfqpoint{4.136311in}{1.988511in}}%
\pgfpathlineto{\pgfqpoint{4.128569in}{1.976866in}}%
\pgfpathlineto{\pgfqpoint{4.120822in}{1.965170in}}%
\pgfpathlineto{\pgfqpoint{4.113071in}{1.953426in}}%
\pgfpathlineto{\pgfqpoint{4.105315in}{1.941635in}}%
\pgfpathclose%
\pgfusepath{fill}%
\end{pgfscope}%
\begin{pgfscope}%
\pgfpathrectangle{\pgfqpoint{1.254980in}{0.150000in}}{\pgfqpoint{5.490039in}{5.490039in}}%
\pgfusepath{clip}%
\pgfsetbuttcap%
\pgfsetroundjoin%
\definecolor{currentfill}{rgb}{0.165117,0.467423,0.558141}%
\pgfsetfillcolor{currentfill}%
\pgfsetfillopacity{0.700000}%
\pgfsetlinewidth{0.000000pt}%
\definecolor{currentstroke}{rgb}{0.000000,0.000000,0.000000}%
\pgfsetstrokecolor{currentstroke}%
\pgfsetdash{}{0pt}%
\pgfpathmoveto{\pgfqpoint{2.199541in}{2.650847in}}%
\pgfpathlineto{\pgfqpoint{2.213325in}{2.626895in}}%
\pgfpathlineto{\pgfqpoint{2.227093in}{2.603250in}}%
\pgfpathlineto{\pgfqpoint{2.240847in}{2.579908in}}%
\pgfpathlineto{\pgfqpoint{2.254585in}{2.556866in}}%
\pgfpathlineto{\pgfqpoint{2.263477in}{2.552402in}}%
\pgfpathlineto{\pgfqpoint{2.272347in}{2.548257in}}%
\pgfpathlineto{\pgfqpoint{2.281196in}{2.544425in}}%
\pgfpathlineto{\pgfqpoint{2.290023in}{2.540902in}}%
\pgfpathlineto{\pgfqpoint{2.276341in}{2.563366in}}%
\pgfpathlineto{\pgfqpoint{2.262645in}{2.586129in}}%
\pgfpathlineto{\pgfqpoint{2.248934in}{2.609195in}}%
\pgfpathlineto{\pgfqpoint{2.235209in}{2.632566in}}%
\pgfpathlineto{\pgfqpoint{2.226326in}{2.636655in}}%
\pgfpathlineto{\pgfqpoint{2.217420in}{2.641062in}}%
\pgfpathlineto{\pgfqpoint{2.208492in}{2.645791in}}%
\pgfpathlineto{\pgfqpoint{2.199541in}{2.650847in}}%
\pgfpathclose%
\pgfusepath{fill}%
\end{pgfscope}%
\begin{pgfscope}%
\pgfpathrectangle{\pgfqpoint{1.254980in}{0.150000in}}{\pgfqpoint{5.490039in}{5.490039in}}%
\pgfusepath{clip}%
\pgfsetbuttcap%
\pgfsetroundjoin%
\definecolor{currentfill}{rgb}{0.277018,0.050344,0.375715}%
\pgfsetfillcolor{currentfill}%
\pgfsetfillopacity{0.700000}%
\pgfsetlinewidth{0.000000pt}%
\definecolor{currentstroke}{rgb}{0.000000,0.000000,0.000000}%
\pgfsetstrokecolor{currentstroke}%
\pgfsetdash{}{0pt}%
\pgfpathmoveto{\pgfqpoint{3.009137in}{1.673860in}}%
\pgfpathlineto{\pgfqpoint{3.022476in}{1.664367in}}%
\pgfpathlineto{\pgfqpoint{3.035815in}{1.655067in}}%
\pgfpathlineto{\pgfqpoint{3.049152in}{1.645960in}}%
\pgfpathlineto{\pgfqpoint{3.062490in}{1.637044in}}%
\pgfpathlineto{\pgfqpoint{3.070730in}{1.640958in}}%
\pgfpathlineto{\pgfqpoint{3.078959in}{1.645073in}}%
\pgfpathlineto{\pgfqpoint{3.087176in}{1.649383in}}%
\pgfpathlineto{\pgfqpoint{3.095382in}{1.653884in}}%
\pgfpathlineto{\pgfqpoint{3.082075in}{1.662271in}}%
\pgfpathlineto{\pgfqpoint{3.068767in}{1.670850in}}%
\pgfpathlineto{\pgfqpoint{3.055460in}{1.679620in}}%
\pgfpathlineto{\pgfqpoint{3.042152in}{1.688584in}}%
\pgfpathlineto{\pgfqpoint{3.033916in}{1.684601in}}%
\pgfpathlineto{\pgfqpoint{3.025668in}{1.680816in}}%
\pgfpathlineto{\pgfqpoint{3.017409in}{1.677234in}}%
\pgfpathlineto{\pgfqpoint{3.009137in}{1.673860in}}%
\pgfpathclose%
\pgfusepath{fill}%
\end{pgfscope}%
\begin{pgfscope}%
\pgfpathrectangle{\pgfqpoint{1.254980in}{0.150000in}}{\pgfqpoint{5.490039in}{5.490039in}}%
\pgfusepath{clip}%
\pgfsetbuttcap%
\pgfsetroundjoin%
\definecolor{currentfill}{rgb}{0.127568,0.566949,0.550556}%
\pgfsetfillcolor{currentfill}%
\pgfsetfillopacity{0.700000}%
\pgfsetlinewidth{0.000000pt}%
\definecolor{currentstroke}{rgb}{0.000000,0.000000,0.000000}%
\pgfsetstrokecolor{currentstroke}%
\pgfsetdash{}{0pt}%
\pgfpathmoveto{\pgfqpoint{5.078486in}{2.828428in}}%
\pgfpathlineto{\pgfqpoint{5.092475in}{2.839799in}}%
\pgfpathlineto{\pgfqpoint{5.106481in}{2.851330in}}%
\pgfpathlineto{\pgfqpoint{5.120505in}{2.863022in}}%
\pgfpathlineto{\pgfqpoint{5.134548in}{2.874875in}}%
\pgfpathlineto{\pgfqpoint{5.141904in}{2.881267in}}%
\pgfpathlineto{\pgfqpoint{5.149252in}{2.887542in}}%
\pgfpathlineto{\pgfqpoint{5.156591in}{2.893701in}}%
\pgfpathlineto{\pgfqpoint{5.163923in}{2.899747in}}%
\pgfpathlineto{\pgfqpoint{5.149892in}{2.888088in}}%
\pgfpathlineto{\pgfqpoint{5.135878in}{2.876590in}}%
\pgfpathlineto{\pgfqpoint{5.121883in}{2.865252in}}%
\pgfpathlineto{\pgfqpoint{5.107905in}{2.854074in}}%
\pgfpathlineto{\pgfqpoint{5.100562in}{2.847823in}}%
\pgfpathlineto{\pgfqpoint{5.093211in}{2.841467in}}%
\pgfpathlineto{\pgfqpoint{5.085853in}{2.835002in}}%
\pgfpathlineto{\pgfqpoint{5.078486in}{2.828428in}}%
\pgfpathclose%
\pgfusepath{fill}%
\end{pgfscope}%
\begin{pgfscope}%
\pgfpathrectangle{\pgfqpoint{1.254980in}{0.150000in}}{\pgfqpoint{5.490039in}{5.490039in}}%
\pgfusepath{clip}%
\pgfsetbuttcap%
\pgfsetroundjoin%
\definecolor{currentfill}{rgb}{0.150148,0.676631,0.506589}%
\pgfsetfillcolor{currentfill}%
\pgfsetfillopacity{0.700000}%
\pgfsetlinewidth{0.000000pt}%
\definecolor{currentstroke}{rgb}{0.000000,0.000000,0.000000}%
\pgfsetstrokecolor{currentstroke}%
\pgfsetdash{}{0pt}%
\pgfpathmoveto{\pgfqpoint{5.449241in}{3.122870in}}%
\pgfpathlineto{\pgfqpoint{5.463456in}{3.135527in}}%
\pgfpathlineto{\pgfqpoint{5.477690in}{3.148344in}}%
\pgfpathlineto{\pgfqpoint{5.491945in}{3.161321in}}%
\pgfpathlineto{\pgfqpoint{5.506219in}{3.174458in}}%
\pgfpathlineto{\pgfqpoint{5.513343in}{3.177694in}}%
\pgfpathlineto{\pgfqpoint{5.520458in}{3.180845in}}%
\pgfpathlineto{\pgfqpoint{5.527564in}{3.183915in}}%
\pgfpathlineto{\pgfqpoint{5.534662in}{3.186907in}}%
\pgfpathlineto{\pgfqpoint{5.520407in}{3.174120in}}%
\pgfpathlineto{\pgfqpoint{5.506173in}{3.161492in}}%
\pgfpathlineto{\pgfqpoint{5.491957in}{3.149023in}}%
\pgfpathlineto{\pgfqpoint{5.477762in}{3.136714in}}%
\pgfpathlineto{\pgfqpoint{5.470644in}{3.133362in}}%
\pgfpathlineto{\pgfqpoint{5.463518in}{3.129940in}}%
\pgfpathlineto{\pgfqpoint{5.456383in}{3.126444in}}%
\pgfpathlineto{\pgfqpoint{5.449241in}{3.122870in}}%
\pgfpathclose%
\pgfusepath{fill}%
\end{pgfscope}%
\begin{pgfscope}%
\pgfpathrectangle{\pgfqpoint{1.254980in}{0.150000in}}{\pgfqpoint{5.490039in}{5.490039in}}%
\pgfusepath{clip}%
\pgfsetbuttcap%
\pgfsetroundjoin%
\definecolor{currentfill}{rgb}{0.269308,0.218818,0.509577}%
\pgfsetfillcolor{currentfill}%
\pgfsetfillopacity{0.700000}%
\pgfsetlinewidth{0.000000pt}%
\definecolor{currentstroke}{rgb}{0.000000,0.000000,0.000000}%
\pgfsetstrokecolor{currentstroke}%
\pgfsetdash{}{0pt}%
\pgfpathmoveto{\pgfqpoint{2.600217in}{2.027658in}}%
\pgfpathlineto{\pgfqpoint{2.613703in}{2.011664in}}%
\pgfpathlineto{\pgfqpoint{2.627182in}{1.995901in}}%
\pgfpathlineto{\pgfqpoint{2.640654in}{1.980370in}}%
\pgfpathlineto{\pgfqpoint{2.654120in}{1.965067in}}%
\pgfpathlineto{\pgfqpoint{2.662681in}{1.964089in}}%
\pgfpathlineto{\pgfqpoint{2.671225in}{1.963387in}}%
\pgfpathlineto{\pgfqpoint{2.679752in}{1.962957in}}%
\pgfpathlineto{\pgfqpoint{2.688262in}{1.962791in}}%
\pgfpathlineto{\pgfqpoint{2.674840in}{1.977519in}}%
\pgfpathlineto{\pgfqpoint{2.661413in}{1.992475in}}%
\pgfpathlineto{\pgfqpoint{2.647979in}{2.007662in}}%
\pgfpathlineto{\pgfqpoint{2.634538in}{2.023079in}}%
\pgfpathlineto{\pgfqpoint{2.625984in}{2.023809in}}%
\pgfpathlineto{\pgfqpoint{2.617413in}{2.024811in}}%
\pgfpathlineto{\pgfqpoint{2.608824in}{2.026092in}}%
\pgfpathlineto{\pgfqpoint{2.600217in}{2.027658in}}%
\pgfpathclose%
\pgfusepath{fill}%
\end{pgfscope}%
\begin{pgfscope}%
\pgfpathrectangle{\pgfqpoint{1.254980in}{0.150000in}}{\pgfqpoint{5.490039in}{5.490039in}}%
\pgfusepath{clip}%
\pgfsetbuttcap%
\pgfsetroundjoin%
\definecolor{currentfill}{rgb}{0.258965,0.251537,0.524736}%
\pgfsetfillcolor{currentfill}%
\pgfsetfillopacity{0.700000}%
\pgfsetlinewidth{0.000000pt}%
\definecolor{currentstroke}{rgb}{0.000000,0.000000,0.000000}%
\pgfsetstrokecolor{currentstroke}%
\pgfsetdash{}{0pt}%
\pgfpathmoveto{\pgfqpoint{2.546199in}{2.093997in}}%
\pgfpathlineto{\pgfqpoint{2.559715in}{2.077055in}}%
\pgfpathlineto{\pgfqpoint{2.573223in}{2.060352in}}%
\pgfpathlineto{\pgfqpoint{2.586724in}{2.043887in}}%
\pgfpathlineto{\pgfqpoint{2.600217in}{2.027658in}}%
\pgfpathlineto{\pgfqpoint{2.608824in}{2.026092in}}%
\pgfpathlineto{\pgfqpoint{2.617413in}{2.024811in}}%
\pgfpathlineto{\pgfqpoint{2.625984in}{2.023809in}}%
\pgfpathlineto{\pgfqpoint{2.634538in}{2.023079in}}%
\pgfpathlineto{\pgfqpoint{2.621091in}{2.038731in}}%
\pgfpathlineto{\pgfqpoint{2.607637in}{2.054617in}}%
\pgfpathlineto{\pgfqpoint{2.594176in}{2.070740in}}%
\pgfpathlineto{\pgfqpoint{2.580708in}{2.087102in}}%
\pgfpathlineto{\pgfqpoint{2.572108in}{2.088398in}}%
\pgfpathlineto{\pgfqpoint{2.563490in}{2.089976in}}%
\pgfpathlineto{\pgfqpoint{2.554854in}{2.091840in}}%
\pgfpathlineto{\pgfqpoint{2.546199in}{2.093997in}}%
\pgfpathclose%
\pgfusepath{fill}%
\end{pgfscope}%
\begin{pgfscope}%
\pgfpathrectangle{\pgfqpoint{1.254980in}{0.150000in}}{\pgfqpoint{5.490039in}{5.490039in}}%
\pgfusepath{clip}%
\pgfsetbuttcap%
\pgfsetroundjoin%
\definecolor{currentfill}{rgb}{0.276194,0.190074,0.493001}%
\pgfsetfillcolor{currentfill}%
\pgfsetfillopacity{0.700000}%
\pgfsetlinewidth{0.000000pt}%
\definecolor{currentstroke}{rgb}{0.000000,0.000000,0.000000}%
\pgfsetstrokecolor{currentstroke}%
\pgfsetdash{}{0pt}%
\pgfpathmoveto{\pgfqpoint{2.654120in}{1.965067in}}%
\pgfpathlineto{\pgfqpoint{2.667580in}{1.949992in}}%
\pgfpathlineto{\pgfqpoint{2.681034in}{1.935142in}}%
\pgfpathlineto{\pgfqpoint{2.694482in}{1.920517in}}%
\pgfpathlineto{\pgfqpoint{2.707924in}{1.906114in}}%
\pgfpathlineto{\pgfqpoint{2.716441in}{1.905720in}}%
\pgfpathlineto{\pgfqpoint{2.724941in}{1.905594in}}%
\pgfpathlineto{\pgfqpoint{2.733426in}{1.905732in}}%
\pgfpathlineto{\pgfqpoint{2.741894in}{1.906127in}}%
\pgfpathlineto{\pgfqpoint{2.728494in}{1.919959in}}%
\pgfpathlineto{\pgfqpoint{2.715089in}{1.934012in}}%
\pgfpathlineto{\pgfqpoint{2.701678in}{1.948289in}}%
\pgfpathlineto{\pgfqpoint{2.688262in}{1.962791in}}%
\pgfpathlineto{\pgfqpoint{2.679752in}{1.962957in}}%
\pgfpathlineto{\pgfqpoint{2.671225in}{1.963387in}}%
\pgfpathlineto{\pgfqpoint{2.662681in}{1.964089in}}%
\pgfpathlineto{\pgfqpoint{2.654120in}{1.965067in}}%
\pgfpathclose%
\pgfusepath{fill}%
\end{pgfscope}%
\begin{pgfscope}%
\pgfpathrectangle{\pgfqpoint{1.254980in}{0.150000in}}{\pgfqpoint{5.490039in}{5.490039in}}%
\pgfusepath{clip}%
\pgfsetbuttcap%
\pgfsetroundjoin%
\definecolor{currentfill}{rgb}{0.190631,0.407061,0.556089}%
\pgfsetfillcolor{currentfill}%
\pgfsetfillopacity{0.700000}%
\pgfsetlinewidth{0.000000pt}%
\definecolor{currentstroke}{rgb}{0.000000,0.000000,0.000000}%
\pgfsetstrokecolor{currentstroke}%
\pgfsetdash{}{0pt}%
\pgfpathmoveto{\pgfqpoint{4.591931in}{2.388674in}}%
\pgfpathlineto{\pgfqpoint{4.605639in}{2.397332in}}%
\pgfpathlineto{\pgfqpoint{4.619362in}{2.406150in}}%
\pgfpathlineto{\pgfqpoint{4.633100in}{2.415130in}}%
\pgfpathlineto{\pgfqpoint{4.646852in}{2.424271in}}%
\pgfpathlineto{\pgfqpoint{4.654444in}{2.434461in}}%
\pgfpathlineto{\pgfqpoint{4.662030in}{2.444535in}}%
\pgfpathlineto{\pgfqpoint{4.669609in}{2.454494in}}%
\pgfpathlineto{\pgfqpoint{4.677182in}{2.464339in}}%
\pgfpathlineto{\pgfqpoint{4.663433in}{2.455180in}}%
\pgfpathlineto{\pgfqpoint{4.649699in}{2.446183in}}%
\pgfpathlineto{\pgfqpoint{4.635980in}{2.437346in}}%
\pgfpathlineto{\pgfqpoint{4.622275in}{2.428670in}}%
\pgfpathlineto{\pgfqpoint{4.614699in}{2.418832in}}%
\pgfpathlineto{\pgfqpoint{4.607115in}{2.408887in}}%
\pgfpathlineto{\pgfqpoint{4.599526in}{2.398834in}}%
\pgfpathlineto{\pgfqpoint{4.591931in}{2.388674in}}%
\pgfpathclose%
\pgfusepath{fill}%
\end{pgfscope}%
\begin{pgfscope}%
\pgfpathrectangle{\pgfqpoint{1.254980in}{0.150000in}}{\pgfqpoint{5.490039in}{5.490039in}}%
\pgfusepath{clip}%
\pgfsetbuttcap%
\pgfsetroundjoin%
\definecolor{currentfill}{rgb}{0.268510,0.009605,0.335427}%
\pgfsetfillcolor{currentfill}%
\pgfsetfillopacity{0.700000}%
\pgfsetlinewidth{0.000000pt}%
\definecolor{currentstroke}{rgb}{0.000000,0.000000,0.000000}%
\pgfsetstrokecolor{currentstroke}%
\pgfsetdash{}{0pt}%
\pgfpathmoveto{\pgfqpoint{3.201858in}{1.593553in}}%
\pgfpathlineto{\pgfqpoint{3.215172in}{1.586845in}}%
\pgfpathlineto{\pgfqpoint{3.228488in}{1.580318in}}%
\pgfpathlineto{\pgfqpoint{3.241805in}{1.573973in}}%
\pgfpathlineto{\pgfqpoint{3.255124in}{1.567809in}}%
\pgfpathlineto{\pgfqpoint{3.263241in}{1.574020in}}%
\pgfpathlineto{\pgfqpoint{3.271348in}{1.580390in}}%
\pgfpathlineto{\pgfqpoint{3.279446in}{1.586914in}}%
\pgfpathlineto{\pgfqpoint{3.287535in}{1.593587in}}%
\pgfpathlineto{\pgfqpoint{3.274239in}{1.599254in}}%
\pgfpathlineto{\pgfqpoint{3.260946in}{1.605102in}}%
\pgfpathlineto{\pgfqpoint{3.247654in}{1.611132in}}%
\pgfpathlineto{\pgfqpoint{3.234364in}{1.617343in}}%
\pgfpathlineto{\pgfqpoint{3.226252in}{1.611157in}}%
\pgfpathlineto{\pgfqpoint{3.218130in}{1.605127in}}%
\pgfpathlineto{\pgfqpoint{3.209999in}{1.599257in}}%
\pgfpathlineto{\pgfqpoint{3.201858in}{1.593553in}}%
\pgfpathclose%
\pgfusepath{fill}%
\end{pgfscope}%
\begin{pgfscope}%
\pgfpathrectangle{\pgfqpoint{1.254980in}{0.150000in}}{\pgfqpoint{5.490039in}{5.490039in}}%
\pgfusepath{clip}%
\pgfsetbuttcap%
\pgfsetroundjoin%
\definecolor{currentfill}{rgb}{0.280267,0.073417,0.397163}%
\pgfsetfillcolor{currentfill}%
\pgfsetfillopacity{0.700000}%
\pgfsetlinewidth{0.000000pt}%
\definecolor{currentstroke}{rgb}{0.000000,0.000000,0.000000}%
\pgfsetstrokecolor{currentstroke}%
\pgfsetdash{}{0pt}%
\pgfpathmoveto{\pgfqpoint{3.734549in}{1.674259in}}%
\pgfpathlineto{\pgfqpoint{3.747905in}{1.674604in}}%
\pgfpathlineto{\pgfqpoint{3.761269in}{1.675116in}}%
\pgfpathlineto{\pgfqpoint{3.774641in}{1.675793in}}%
\pgfpathlineto{\pgfqpoint{3.788020in}{1.676635in}}%
\pgfpathlineto{\pgfqpoint{3.795889in}{1.687747in}}%
\pgfpathlineto{\pgfqpoint{3.803753in}{1.698886in}}%
\pgfpathlineto{\pgfqpoint{3.811612in}{1.710049in}}%
\pgfpathlineto{\pgfqpoint{3.819465in}{1.721232in}}%
\pgfpathlineto{\pgfqpoint{3.806095in}{1.720032in}}%
\pgfpathlineto{\pgfqpoint{3.792732in}{1.718998in}}%
\pgfpathlineto{\pgfqpoint{3.779377in}{1.718129in}}%
\pgfpathlineto{\pgfqpoint{3.766030in}{1.717427in}}%
\pgfpathlineto{\pgfqpoint{3.758167in}{1.706591in}}%
\pgfpathlineto{\pgfqpoint{3.750300in}{1.695782in}}%
\pgfpathlineto{\pgfqpoint{3.742427in}{1.685004in}}%
\pgfpathlineto{\pgfqpoint{3.734549in}{1.674259in}}%
\pgfpathclose%
\pgfusepath{fill}%
\end{pgfscope}%
\begin{pgfscope}%
\pgfpathrectangle{\pgfqpoint{1.254980in}{0.150000in}}{\pgfqpoint{5.490039in}{5.490039in}}%
\pgfusepath{clip}%
\pgfsetbuttcap%
\pgfsetroundjoin%
\definecolor{currentfill}{rgb}{0.248629,0.278775,0.534556}%
\pgfsetfillcolor{currentfill}%
\pgfsetfillopacity{0.700000}%
\pgfsetlinewidth{0.000000pt}%
\definecolor{currentstroke}{rgb}{0.000000,0.000000,0.000000}%
\pgfsetstrokecolor{currentstroke}%
\pgfsetdash{}{0pt}%
\pgfpathmoveto{\pgfqpoint{2.492052in}{2.164203in}}%
\pgfpathlineto{\pgfqpoint{2.505602in}{2.146282in}}%
\pgfpathlineto{\pgfqpoint{2.519143in}{2.128609in}}%
\pgfpathlineto{\pgfqpoint{2.532675in}{2.111182in}}%
\pgfpathlineto{\pgfqpoint{2.546199in}{2.093997in}}%
\pgfpathlineto{\pgfqpoint{2.554854in}{2.091840in}}%
\pgfpathlineto{\pgfqpoint{2.563490in}{2.089976in}}%
\pgfpathlineto{\pgfqpoint{2.572108in}{2.088398in}}%
\pgfpathlineto{\pgfqpoint{2.580708in}{2.087102in}}%
\pgfpathlineto{\pgfqpoint{2.567232in}{2.103704in}}%
\pgfpathlineto{\pgfqpoint{2.553748in}{2.120550in}}%
\pgfpathlineto{\pgfqpoint{2.540256in}{2.137639in}}%
\pgfpathlineto{\pgfqpoint{2.526755in}{2.154976in}}%
\pgfpathlineto{\pgfqpoint{2.518108in}{2.156843in}}%
\pgfpathlineto{\pgfqpoint{2.509442in}{2.158999in}}%
\pgfpathlineto{\pgfqpoint{2.500757in}{2.161451in}}%
\pgfpathlineto{\pgfqpoint{2.492052in}{2.164203in}}%
\pgfpathclose%
\pgfusepath{fill}%
\end{pgfscope}%
\begin{pgfscope}%
\pgfpathrectangle{\pgfqpoint{1.254980in}{0.150000in}}{\pgfqpoint{5.490039in}{5.490039in}}%
\pgfusepath{clip}%
\pgfsetbuttcap%
\pgfsetroundjoin%
\definecolor{currentfill}{rgb}{0.267004,0.004874,0.329415}%
\pgfsetfillcolor{currentfill}%
\pgfsetfillopacity{0.700000}%
\pgfsetlinewidth{0.000000pt}%
\definecolor{currentstroke}{rgb}{0.000000,0.000000,0.000000}%
\pgfsetstrokecolor{currentstroke}%
\pgfsetdash{}{0pt}%
\pgfpathmoveto{\pgfqpoint{3.340745in}{1.572706in}}%
\pgfpathlineto{\pgfqpoint{3.354055in}{1.567930in}}%
\pgfpathlineto{\pgfqpoint{3.367368in}{1.563330in}}%
\pgfpathlineto{\pgfqpoint{3.380684in}{1.558906in}}%
\pgfpathlineto{\pgfqpoint{3.394004in}{1.554656in}}%
\pgfpathlineto{\pgfqpoint{3.402042in}{1.562430in}}%
\pgfpathlineto{\pgfqpoint{3.410073in}{1.570329in}}%
\pgfpathlineto{\pgfqpoint{3.418096in}{1.578349in}}%
\pgfpathlineto{\pgfqpoint{3.426111in}{1.586485in}}%
\pgfpathlineto{\pgfqpoint{3.412811in}{1.590267in}}%
\pgfpathlineto{\pgfqpoint{3.399514in}{1.594224in}}%
\pgfpathlineto{\pgfqpoint{3.386220in}{1.598356in}}%
\pgfpathlineto{\pgfqpoint{3.372931in}{1.602665in}}%
\pgfpathlineto{\pgfqpoint{3.364896in}{1.594985in}}%
\pgfpathlineto{\pgfqpoint{3.356854in}{1.587430in}}%
\pgfpathlineto{\pgfqpoint{3.348804in}{1.580002in}}%
\pgfpathlineto{\pgfqpoint{3.340745in}{1.572706in}}%
\pgfpathclose%
\pgfusepath{fill}%
\end{pgfscope}%
\begin{pgfscope}%
\pgfpathrectangle{\pgfqpoint{1.254980in}{0.150000in}}{\pgfqpoint{5.490039in}{5.490039in}}%
\pgfusepath{clip}%
\pgfsetbuttcap%
\pgfsetroundjoin%
\definecolor{currentfill}{rgb}{0.276022,0.044167,0.370164}%
\pgfsetfillcolor{currentfill}%
\pgfsetfillopacity{0.700000}%
\pgfsetlinewidth{0.000000pt}%
\definecolor{currentstroke}{rgb}{0.000000,0.000000,0.000000}%
\pgfsetstrokecolor{currentstroke}%
\pgfsetdash{}{0pt}%
\pgfpathmoveto{\pgfqpoint{3.649585in}{1.633518in}}%
\pgfpathlineto{\pgfqpoint{3.662924in}{1.632811in}}%
\pgfpathlineto{\pgfqpoint{3.676270in}{1.632271in}}%
\pgfpathlineto{\pgfqpoint{3.689622in}{1.631898in}}%
\pgfpathlineto{\pgfqpoint{3.702982in}{1.631691in}}%
\pgfpathlineto{\pgfqpoint{3.710882in}{1.642265in}}%
\pgfpathlineto{\pgfqpoint{3.718776in}{1.652886in}}%
\pgfpathlineto{\pgfqpoint{3.726665in}{1.663552in}}%
\pgfpathlineto{\pgfqpoint{3.734549in}{1.674259in}}%
\pgfpathlineto{\pgfqpoint{3.721200in}{1.674081in}}%
\pgfpathlineto{\pgfqpoint{3.707858in}{1.674069in}}%
\pgfpathlineto{\pgfqpoint{3.694523in}{1.674224in}}%
\pgfpathlineto{\pgfqpoint{3.681195in}{1.674548in}}%
\pgfpathlineto{\pgfqpoint{3.673301in}{1.664215in}}%
\pgfpathlineto{\pgfqpoint{3.665401in}{1.653930in}}%
\pgfpathlineto{\pgfqpoint{3.657496in}{1.643697in}}%
\pgfpathlineto{\pgfqpoint{3.649585in}{1.633518in}}%
\pgfpathclose%
\pgfusepath{fill}%
\end{pgfscope}%
\begin{pgfscope}%
\pgfpathrectangle{\pgfqpoint{1.254980in}{0.150000in}}{\pgfqpoint{5.490039in}{5.490039in}}%
\pgfusepath{clip}%
\pgfsetbuttcap%
\pgfsetroundjoin%
\definecolor{currentfill}{rgb}{0.227802,0.326594,0.546532}%
\pgfsetfillcolor{currentfill}%
\pgfsetfillopacity{0.700000}%
\pgfsetlinewidth{0.000000pt}%
\definecolor{currentstroke}{rgb}{0.000000,0.000000,0.000000}%
\pgfsetstrokecolor{currentstroke}%
\pgfsetdash{}{0pt}%
\pgfpathmoveto{\pgfqpoint{4.391113in}{2.196769in}}%
\pgfpathlineto{\pgfqpoint{4.404718in}{2.203906in}}%
\pgfpathlineto{\pgfqpoint{4.418336in}{2.211205in}}%
\pgfpathlineto{\pgfqpoint{4.431968in}{2.218665in}}%
\pgfpathlineto{\pgfqpoint{4.445613in}{2.226286in}}%
\pgfpathlineto{\pgfqpoint{4.453279in}{2.237609in}}%
\pgfpathlineto{\pgfqpoint{4.460940in}{2.248836in}}%
\pgfpathlineto{\pgfqpoint{4.468596in}{2.259965in}}%
\pgfpathlineto{\pgfqpoint{4.476246in}{2.270995in}}%
\pgfpathlineto{\pgfqpoint{4.462603in}{2.263268in}}%
\pgfpathlineto{\pgfqpoint{4.448974in}{2.255703in}}%
\pgfpathlineto{\pgfqpoint{4.435358in}{2.248299in}}%
\pgfpathlineto{\pgfqpoint{4.421756in}{2.241056in}}%
\pgfpathlineto{\pgfqpoint{4.414103in}{2.230120in}}%
\pgfpathlineto{\pgfqpoint{4.406445in}{2.219093in}}%
\pgfpathlineto{\pgfqpoint{4.398782in}{2.207976in}}%
\pgfpathlineto{\pgfqpoint{4.391113in}{2.196769in}}%
\pgfpathclose%
\pgfusepath{fill}%
\end{pgfscope}%
\begin{pgfscope}%
\pgfpathrectangle{\pgfqpoint{1.254980in}{0.150000in}}{\pgfqpoint{5.490039in}{5.490039in}}%
\pgfusepath{clip}%
\pgfsetbuttcap%
\pgfsetroundjoin%
\definecolor{currentfill}{rgb}{0.280255,0.165693,0.476498}%
\pgfsetfillcolor{currentfill}%
\pgfsetfillopacity{0.700000}%
\pgfsetlinewidth{0.000000pt}%
\definecolor{currentstroke}{rgb}{0.000000,0.000000,0.000000}%
\pgfsetstrokecolor{currentstroke}%
\pgfsetdash{}{0pt}%
\pgfpathmoveto{\pgfqpoint{2.707924in}{1.906114in}}%
\pgfpathlineto{\pgfqpoint{2.721361in}{1.891931in}}%
\pgfpathlineto{\pgfqpoint{2.734793in}{1.877969in}}%
\pgfpathlineto{\pgfqpoint{2.748220in}{1.864224in}}%
\pgfpathlineto{\pgfqpoint{2.761643in}{1.850695in}}%
\pgfpathlineto{\pgfqpoint{2.770117in}{1.850882in}}%
\pgfpathlineto{\pgfqpoint{2.778576in}{1.851330in}}%
\pgfpathlineto{\pgfqpoint{2.787019in}{1.852033in}}%
\pgfpathlineto{\pgfqpoint{2.795447in}{1.852986in}}%
\pgfpathlineto{\pgfqpoint{2.782066in}{1.865946in}}%
\pgfpathlineto{\pgfqpoint{2.768680in}{1.879122in}}%
\pgfpathlineto{\pgfqpoint{2.755289in}{1.892515in}}%
\pgfpathlineto{\pgfqpoint{2.741894in}{1.906127in}}%
\pgfpathlineto{\pgfqpoint{2.733426in}{1.905732in}}%
\pgfpathlineto{\pgfqpoint{2.724941in}{1.905594in}}%
\pgfpathlineto{\pgfqpoint{2.716441in}{1.905720in}}%
\pgfpathlineto{\pgfqpoint{2.707924in}{1.906114in}}%
\pgfpathclose%
\pgfusepath{fill}%
\end{pgfscope}%
\begin{pgfscope}%
\pgfpathrectangle{\pgfqpoint{1.254980in}{0.150000in}}{\pgfqpoint{5.490039in}{5.490039in}}%
\pgfusepath{clip}%
\pgfsetbuttcap%
\pgfsetroundjoin%
\definecolor{currentfill}{rgb}{0.282656,0.100196,0.422160}%
\pgfsetfillcolor{currentfill}%
\pgfsetfillopacity{0.700000}%
\pgfsetlinewidth{0.000000pt}%
\definecolor{currentstroke}{rgb}{0.000000,0.000000,0.000000}%
\pgfsetstrokecolor{currentstroke}%
\pgfsetdash{}{0pt}%
\pgfpathmoveto{\pgfqpoint{3.819465in}{1.721232in}}%
\pgfpathlineto{\pgfqpoint{3.832844in}{1.722598in}}%
\pgfpathlineto{\pgfqpoint{3.846231in}{1.724128in}}%
\pgfpathlineto{\pgfqpoint{3.859627in}{1.725822in}}%
\pgfpathlineto{\pgfqpoint{3.873031in}{1.727681in}}%
\pgfpathlineto{\pgfqpoint{3.880873in}{1.739222in}}%
\pgfpathlineto{\pgfqpoint{3.888709in}{1.750771in}}%
\pgfpathlineto{\pgfqpoint{3.896541in}{1.762323in}}%
\pgfpathlineto{\pgfqpoint{3.904368in}{1.773877in}}%
\pgfpathlineto{\pgfqpoint{3.890970in}{1.771688in}}%
\pgfpathlineto{\pgfqpoint{3.877581in}{1.769663in}}%
\pgfpathlineto{\pgfqpoint{3.864201in}{1.767803in}}%
\pgfpathlineto{\pgfqpoint{3.850830in}{1.766108in}}%
\pgfpathlineto{\pgfqpoint{3.842996in}{1.754874in}}%
\pgfpathlineto{\pgfqpoint{3.835158in}{1.743648in}}%
\pgfpathlineto{\pgfqpoint{3.827314in}{1.732433in}}%
\pgfpathlineto{\pgfqpoint{3.819465in}{1.721232in}}%
\pgfpathclose%
\pgfusepath{fill}%
\end{pgfscope}%
\begin{pgfscope}%
\pgfpathrectangle{\pgfqpoint{1.254980in}{0.150000in}}{\pgfqpoint{5.490039in}{5.490039in}}%
\pgfusepath{clip}%
\pgfsetbuttcap%
\pgfsetroundjoin%
\definecolor{currentfill}{rgb}{0.233603,0.313828,0.543914}%
\pgfsetfillcolor{currentfill}%
\pgfsetfillopacity{0.700000}%
\pgfsetlinewidth{0.000000pt}%
\definecolor{currentstroke}{rgb}{0.000000,0.000000,0.000000}%
\pgfsetstrokecolor{currentstroke}%
\pgfsetdash{}{0pt}%
\pgfpathmoveto{\pgfqpoint{2.437760in}{2.238403in}}%
\pgfpathlineto{\pgfqpoint{2.451348in}{2.219471in}}%
\pgfpathlineto{\pgfqpoint{2.464925in}{2.200796in}}%
\pgfpathlineto{\pgfqpoint{2.478493in}{2.182374in}}%
\pgfpathlineto{\pgfqpoint{2.492052in}{2.164203in}}%
\pgfpathlineto{\pgfqpoint{2.500757in}{2.161451in}}%
\pgfpathlineto{\pgfqpoint{2.509442in}{2.158999in}}%
\pgfpathlineto{\pgfqpoint{2.518108in}{2.156843in}}%
\pgfpathlineto{\pgfqpoint{2.526755in}{2.154976in}}%
\pgfpathlineto{\pgfqpoint{2.513246in}{2.172560in}}%
\pgfpathlineto{\pgfqpoint{2.499729in}{2.190396in}}%
\pgfpathlineto{\pgfqpoint{2.486202in}{2.208484in}}%
\pgfpathlineto{\pgfqpoint{2.472665in}{2.226826in}}%
\pgfpathlineto{\pgfqpoint{2.463969in}{2.229269in}}%
\pgfpathlineto{\pgfqpoint{2.455252in}{2.232008in}}%
\pgfpathlineto{\pgfqpoint{2.446516in}{2.235051in}}%
\pgfpathlineto{\pgfqpoint{2.437760in}{2.238403in}}%
\pgfpathclose%
\pgfusepath{fill}%
\end{pgfscope}%
\begin{pgfscope}%
\pgfpathrectangle{\pgfqpoint{1.254980in}{0.150000in}}{\pgfqpoint{5.490039in}{5.490039in}}%
\pgfusepath{clip}%
\pgfsetbuttcap%
\pgfsetroundjoin%
\definecolor{currentfill}{rgb}{0.272594,0.025563,0.353093}%
\pgfsetfillcolor{currentfill}%
\pgfsetfillopacity{0.700000}%
\pgfsetlinewidth{0.000000pt}%
\definecolor{currentstroke}{rgb}{0.000000,0.000000,0.000000}%
\pgfsetstrokecolor{currentstroke}%
\pgfsetdash{}{0pt}%
\pgfpathmoveto{\pgfqpoint{3.564534in}{1.599594in}}%
\pgfpathlineto{\pgfqpoint{3.577862in}{1.597799in}}%
\pgfpathlineto{\pgfqpoint{3.591196in}{1.596174in}}%
\pgfpathlineto{\pgfqpoint{3.604535in}{1.594718in}}%
\pgfpathlineto{\pgfqpoint{3.617881in}{1.593430in}}%
\pgfpathlineto{\pgfqpoint{3.625816in}{1.603351in}}%
\pgfpathlineto{\pgfqpoint{3.633745in}{1.613341in}}%
\pgfpathlineto{\pgfqpoint{3.641668in}{1.623399in}}%
\pgfpathlineto{\pgfqpoint{3.649585in}{1.633518in}}%
\pgfpathlineto{\pgfqpoint{3.636252in}{1.634394in}}%
\pgfpathlineto{\pgfqpoint{3.622925in}{1.635438in}}%
\pgfpathlineto{\pgfqpoint{3.609605in}{1.636652in}}%
\pgfpathlineto{\pgfqpoint{3.596290in}{1.638034in}}%
\pgfpathlineto{\pgfqpoint{3.588361in}{1.628316in}}%
\pgfpathlineto{\pgfqpoint{3.580425in}{1.618667in}}%
\pgfpathlineto{\pgfqpoint{3.572483in}{1.609092in}}%
\pgfpathlineto{\pgfqpoint{3.564534in}{1.599594in}}%
\pgfpathclose%
\pgfusepath{fill}%
\end{pgfscope}%
\begin{pgfscope}%
\pgfpathrectangle{\pgfqpoint{1.254980in}{0.150000in}}{\pgfqpoint{5.490039in}{5.490039in}}%
\pgfusepath{clip}%
\pgfsetbuttcap%
\pgfsetroundjoin%
\definecolor{currentfill}{rgb}{0.180653,0.701402,0.488189}%
\pgfsetfillcolor{currentfill}%
\pgfsetfillopacity{0.700000}%
\pgfsetlinewidth{0.000000pt}%
\definecolor{currentstroke}{rgb}{0.000000,0.000000,0.000000}%
\pgfsetstrokecolor{currentstroke}%
\pgfsetdash{}{0pt}%
\pgfpathmoveto{\pgfqpoint{5.534662in}{3.186907in}}%
\pgfpathlineto{\pgfqpoint{5.548936in}{3.199855in}}%
\pgfpathlineto{\pgfqpoint{5.563230in}{3.212962in}}%
\pgfpathlineto{\pgfqpoint{5.577544in}{3.226229in}}%
\pgfpathlineto{\pgfqpoint{5.591879in}{3.239657in}}%
\pgfpathlineto{\pgfqpoint{5.598946in}{3.242206in}}%
\pgfpathlineto{\pgfqpoint{5.606005in}{3.244680in}}%
\pgfpathlineto{\pgfqpoint{5.613055in}{3.247081in}}%
\pgfpathlineto{\pgfqpoint{5.620097in}{3.249415in}}%
\pgfpathlineto{\pgfqpoint{5.605785in}{3.236369in}}%
\pgfpathlineto{\pgfqpoint{5.591492in}{3.223482in}}%
\pgfpathlineto{\pgfqpoint{5.577220in}{3.210755in}}%
\pgfpathlineto{\pgfqpoint{5.562967in}{3.198187in}}%
\pgfpathlineto{\pgfqpoint{5.555903in}{3.195462in}}%
\pgfpathlineto{\pgfqpoint{5.548831in}{3.192677in}}%
\pgfpathlineto{\pgfqpoint{5.541751in}{3.189827in}}%
\pgfpathlineto{\pgfqpoint{5.534662in}{3.186907in}}%
\pgfpathclose%
\pgfusepath{fill}%
\end{pgfscope}%
\begin{pgfscope}%
\pgfpathrectangle{\pgfqpoint{1.254980in}{0.150000in}}{\pgfqpoint{5.490039in}{5.490039in}}%
\pgfusepath{clip}%
\pgfsetbuttcap%
\pgfsetroundjoin%
\definecolor{currentfill}{rgb}{0.260571,0.246922,0.522828}%
\pgfsetfillcolor{currentfill}%
\pgfsetfillopacity{0.700000}%
\pgfsetlinewidth{0.000000pt}%
\definecolor{currentstroke}{rgb}{0.000000,0.000000,0.000000}%
\pgfsetstrokecolor{currentstroke}%
\pgfsetdash{}{0pt}%
\pgfpathmoveto{\pgfqpoint{4.190250in}{2.008302in}}%
\pgfpathlineto{\pgfqpoint{4.203763in}{2.013655in}}%
\pgfpathlineto{\pgfqpoint{4.217288in}{2.019171in}}%
\pgfpathlineto{\pgfqpoint{4.230825in}{2.024848in}}%
\pgfpathlineto{\pgfqpoint{4.244373in}{2.030688in}}%
\pgfpathlineto{\pgfqpoint{4.252105in}{2.042684in}}%
\pgfpathlineto{\pgfqpoint{4.259831in}{2.054613in}}%
\pgfpathlineto{\pgfqpoint{4.267552in}{2.066472in}}%
\pgfpathlineto{\pgfqpoint{4.275269in}{2.078259in}}%
\pgfpathlineto{\pgfqpoint{4.261723in}{2.072229in}}%
\pgfpathlineto{\pgfqpoint{4.248189in}{2.066359in}}%
\pgfpathlineto{\pgfqpoint{4.234667in}{2.060652in}}%
\pgfpathlineto{\pgfqpoint{4.221156in}{2.055107in}}%
\pgfpathlineto{\pgfqpoint{4.213437in}{2.043501in}}%
\pgfpathlineto{\pgfqpoint{4.205713in}{2.031830in}}%
\pgfpathlineto{\pgfqpoint{4.197984in}{2.020096in}}%
\pgfpathlineto{\pgfqpoint{4.190250in}{2.008302in}}%
\pgfpathclose%
\pgfusepath{fill}%
\end{pgfscope}%
\begin{pgfscope}%
\pgfpathrectangle{\pgfqpoint{1.254980in}{0.150000in}}{\pgfqpoint{5.490039in}{5.490039in}}%
\pgfusepath{clip}%
\pgfsetbuttcap%
\pgfsetroundjoin%
\definecolor{currentfill}{rgb}{0.282623,0.140926,0.457517}%
\pgfsetfillcolor{currentfill}%
\pgfsetfillopacity{0.700000}%
\pgfsetlinewidth{0.000000pt}%
\definecolor{currentstroke}{rgb}{0.000000,0.000000,0.000000}%
\pgfsetstrokecolor{currentstroke}%
\pgfsetdash{}{0pt}%
\pgfpathmoveto{\pgfqpoint{2.761643in}{1.850695in}}%
\pgfpathlineto{\pgfqpoint{2.775061in}{1.837381in}}%
\pgfpathlineto{\pgfqpoint{2.788474in}{1.824281in}}%
\pgfpathlineto{\pgfqpoint{2.801884in}{1.811393in}}%
\pgfpathlineto{\pgfqpoint{2.815289in}{1.798716in}}%
\pgfpathlineto{\pgfqpoint{2.823723in}{1.799481in}}%
\pgfpathlineto{\pgfqpoint{2.832142in}{1.800500in}}%
\pgfpathlineto{\pgfqpoint{2.840546in}{1.801766in}}%
\pgfpathlineto{\pgfqpoint{2.848936in}{1.803274in}}%
\pgfpathlineto{\pgfqpoint{2.835569in}{1.815385in}}%
\pgfpathlineto{\pgfqpoint{2.822199in}{1.827707in}}%
\pgfpathlineto{\pgfqpoint{2.808825in}{1.840240in}}%
\pgfpathlineto{\pgfqpoint{2.795447in}{1.852986in}}%
\pgfpathlineto{\pgfqpoint{2.787019in}{1.852033in}}%
\pgfpathlineto{\pgfqpoint{2.778576in}{1.851330in}}%
\pgfpathlineto{\pgfqpoint{2.770117in}{1.850882in}}%
\pgfpathlineto{\pgfqpoint{2.761643in}{1.850695in}}%
\pgfpathclose%
\pgfusepath{fill}%
\end{pgfscope}%
\begin{pgfscope}%
\pgfpathrectangle{\pgfqpoint{1.254980in}{0.150000in}}{\pgfqpoint{5.490039in}{5.490039in}}%
\pgfusepath{clip}%
\pgfsetbuttcap%
\pgfsetroundjoin%
\definecolor{currentfill}{rgb}{0.283072,0.130895,0.449241}%
\pgfsetfillcolor{currentfill}%
\pgfsetfillopacity{0.700000}%
\pgfsetlinewidth{0.000000pt}%
\definecolor{currentstroke}{rgb}{0.000000,0.000000,0.000000}%
\pgfsetstrokecolor{currentstroke}%
\pgfsetdash{}{0pt}%
\pgfpathmoveto{\pgfqpoint{3.904368in}{1.773877in}}%
\pgfpathlineto{\pgfqpoint{3.917774in}{1.776230in}}%
\pgfpathlineto{\pgfqpoint{3.931190in}{1.778747in}}%
\pgfpathlineto{\pgfqpoint{3.944614in}{1.781428in}}%
\pgfpathlineto{\pgfqpoint{3.958049in}{1.784272in}}%
\pgfpathlineto{\pgfqpoint{3.965865in}{1.796138in}}%
\pgfpathlineto{\pgfqpoint{3.973676in}{1.807991in}}%
\pgfpathlineto{\pgfqpoint{3.981483in}{1.819831in}}%
\pgfpathlineto{\pgfqpoint{3.989285in}{1.831653in}}%
\pgfpathlineto{\pgfqpoint{3.975856in}{1.828506in}}%
\pgfpathlineto{\pgfqpoint{3.962437in}{1.825522in}}%
\pgfpathlineto{\pgfqpoint{3.949027in}{1.822702in}}%
\pgfpathlineto{\pgfqpoint{3.935627in}{1.820047in}}%
\pgfpathlineto{\pgfqpoint{3.927819in}{1.808516in}}%
\pgfpathlineto{\pgfqpoint{3.920007in}{1.796976in}}%
\pgfpathlineto{\pgfqpoint{3.912190in}{1.785429in}}%
\pgfpathlineto{\pgfqpoint{3.904368in}{1.773877in}}%
\pgfpathclose%
\pgfusepath{fill}%
\end{pgfscope}%
\begin{pgfscope}%
\pgfpathrectangle{\pgfqpoint{1.254980in}{0.150000in}}{\pgfqpoint{5.490039in}{5.490039in}}%
\pgfusepath{clip}%
\pgfsetbuttcap%
\pgfsetroundjoin%
\definecolor{currentfill}{rgb}{0.149039,0.508051,0.557250}%
\pgfsetfillcolor{currentfill}%
\pgfsetfillopacity{0.700000}%
\pgfsetlinewidth{0.000000pt}%
\definecolor{currentstroke}{rgb}{0.000000,0.000000,0.000000}%
\pgfsetstrokecolor{currentstroke}%
\pgfsetdash{}{0pt}%
\pgfpathmoveto{\pgfqpoint{4.877984in}{2.651817in}}%
\pgfpathlineto{\pgfqpoint{4.891861in}{2.662299in}}%
\pgfpathlineto{\pgfqpoint{4.905754in}{2.672943in}}%
\pgfpathlineto{\pgfqpoint{4.919664in}{2.683747in}}%
\pgfpathlineto{\pgfqpoint{4.933590in}{2.694713in}}%
\pgfpathlineto{\pgfqpoint{4.941058in}{2.702857in}}%
\pgfpathlineto{\pgfqpoint{4.948519in}{2.710876in}}%
\pgfpathlineto{\pgfqpoint{4.955972in}{2.718772in}}%
\pgfpathlineto{\pgfqpoint{4.963417in}{2.726546in}}%
\pgfpathlineto{\pgfqpoint{4.949497in}{2.715683in}}%
\pgfpathlineto{\pgfqpoint{4.935594in}{2.704981in}}%
\pgfpathlineto{\pgfqpoint{4.921708in}{2.694439in}}%
\pgfpathlineto{\pgfqpoint{4.907838in}{2.684059in}}%
\pgfpathlineto{\pgfqpoint{4.900386in}{2.676172in}}%
\pgfpathlineto{\pgfqpoint{4.892926in}{2.668170in}}%
\pgfpathlineto{\pgfqpoint{4.885459in}{2.660052in}}%
\pgfpathlineto{\pgfqpoint{4.877984in}{2.651817in}}%
\pgfpathclose%
\pgfusepath{fill}%
\end{pgfscope}%
\begin{pgfscope}%
\pgfpathrectangle{\pgfqpoint{1.254980in}{0.150000in}}{\pgfqpoint{5.490039in}{5.490039in}}%
\pgfusepath{clip}%
\pgfsetbuttcap%
\pgfsetroundjoin%
\definecolor{currentfill}{rgb}{0.120565,0.596422,0.543611}%
\pgfsetfillcolor{currentfill}%
\pgfsetfillopacity{0.700000}%
\pgfsetlinewidth{0.000000pt}%
\definecolor{currentstroke}{rgb}{0.000000,0.000000,0.000000}%
\pgfsetstrokecolor{currentstroke}%
\pgfsetdash{}{0pt}%
\pgfpathmoveto{\pgfqpoint{5.163923in}{2.899747in}}%
\pgfpathlineto{\pgfqpoint{5.177972in}{2.911566in}}%
\pgfpathlineto{\pgfqpoint{5.192040in}{2.923546in}}%
\pgfpathlineto{\pgfqpoint{5.206126in}{2.935687in}}%
\pgfpathlineto{\pgfqpoint{5.220230in}{2.947988in}}%
\pgfpathlineto{\pgfqpoint{5.227541in}{2.953710in}}%
\pgfpathlineto{\pgfqpoint{5.234844in}{2.959317in}}%
\pgfpathlineto{\pgfqpoint{5.242138in}{2.964812in}}%
\pgfpathlineto{\pgfqpoint{5.249424in}{2.970197in}}%
\pgfpathlineto{\pgfqpoint{5.235332in}{2.958121in}}%
\pgfpathlineto{\pgfqpoint{5.221259in}{2.946205in}}%
\pgfpathlineto{\pgfqpoint{5.207204in}{2.934450in}}%
\pgfpathlineto{\pgfqpoint{5.193167in}{2.922855in}}%
\pgfpathlineto{\pgfqpoint{5.185868in}{2.917234in}}%
\pgfpathlineto{\pgfqpoint{5.178561in}{2.911511in}}%
\pgfpathlineto{\pgfqpoint{5.171246in}{2.905683in}}%
\pgfpathlineto{\pgfqpoint{5.163923in}{2.899747in}}%
\pgfpathclose%
\pgfusepath{fill}%
\end{pgfscope}%
\begin{pgfscope}%
\pgfpathrectangle{\pgfqpoint{1.254980in}{0.150000in}}{\pgfqpoint{5.490039in}{5.490039in}}%
\pgfusepath{clip}%
\pgfsetbuttcap%
\pgfsetroundjoin%
\definecolor{currentfill}{rgb}{0.273809,0.031497,0.358853}%
\pgfsetfillcolor{currentfill}%
\pgfsetfillopacity{0.700000}%
\pgfsetlinewidth{0.000000pt}%
\definecolor{currentstroke}{rgb}{0.000000,0.000000,0.000000}%
\pgfsetstrokecolor{currentstroke}%
\pgfsetdash{}{0pt}%
\pgfpathmoveto{\pgfqpoint{3.062490in}{1.637044in}}%
\pgfpathlineto{\pgfqpoint{3.075827in}{1.628319in}}%
\pgfpathlineto{\pgfqpoint{3.089163in}{1.619783in}}%
\pgfpathlineto{\pgfqpoint{3.102500in}{1.611436in}}%
\pgfpathlineto{\pgfqpoint{3.115837in}{1.603276in}}%
\pgfpathlineto{\pgfqpoint{3.124048in}{1.607729in}}%
\pgfpathlineto{\pgfqpoint{3.132247in}{1.612375in}}%
\pgfpathlineto{\pgfqpoint{3.140436in}{1.617209in}}%
\pgfpathlineto{\pgfqpoint{3.148614in}{1.622227in}}%
\pgfpathlineto{\pgfqpoint{3.135305in}{1.629859in}}%
\pgfpathlineto{\pgfqpoint{3.121997in}{1.637679in}}%
\pgfpathlineto{\pgfqpoint{3.108689in}{1.645687in}}%
\pgfpathlineto{\pgfqpoint{3.095382in}{1.653884in}}%
\pgfpathlineto{\pgfqpoint{3.087176in}{1.649383in}}%
\pgfpathlineto{\pgfqpoint{3.078959in}{1.645073in}}%
\pgfpathlineto{\pgfqpoint{3.070730in}{1.640958in}}%
\pgfpathlineto{\pgfqpoint{3.062490in}{1.637044in}}%
\pgfpathclose%
\pgfusepath{fill}%
\end{pgfscope}%
\begin{pgfscope}%
\pgfpathrectangle{\pgfqpoint{1.254980in}{0.150000in}}{\pgfqpoint{5.490039in}{5.490039in}}%
\pgfusepath{clip}%
\pgfsetbuttcap%
\pgfsetroundjoin%
\definecolor{currentfill}{rgb}{0.218130,0.347432,0.550038}%
\pgfsetfillcolor{currentfill}%
\pgfsetfillopacity{0.700000}%
\pgfsetlinewidth{0.000000pt}%
\definecolor{currentstroke}{rgb}{0.000000,0.000000,0.000000}%
\pgfsetstrokecolor{currentstroke}%
\pgfsetdash{}{0pt}%
\pgfpathmoveto{\pgfqpoint{2.383306in}{2.316734in}}%
\pgfpathlineto{\pgfqpoint{2.396935in}{2.296756in}}%
\pgfpathlineto{\pgfqpoint{2.410554in}{2.277043in}}%
\pgfpathlineto{\pgfqpoint{2.424162in}{2.257593in}}%
\pgfpathlineto{\pgfqpoint{2.437760in}{2.238403in}}%
\pgfpathlineto{\pgfqpoint{2.446516in}{2.235051in}}%
\pgfpathlineto{\pgfqpoint{2.455252in}{2.232008in}}%
\pgfpathlineto{\pgfqpoint{2.463969in}{2.229269in}}%
\pgfpathlineto{\pgfqpoint{2.472665in}{2.226826in}}%
\pgfpathlineto{\pgfqpoint{2.459119in}{2.245426in}}%
\pgfpathlineto{\pgfqpoint{2.445564in}{2.264285in}}%
\pgfpathlineto{\pgfqpoint{2.431998in}{2.283406in}}%
\pgfpathlineto{\pgfqpoint{2.418421in}{2.302790in}}%
\pgfpathlineto{\pgfqpoint{2.409673in}{2.305812in}}%
\pgfpathlineto{\pgfqpoint{2.400905in}{2.309139in}}%
\pgfpathlineto{\pgfqpoint{2.392116in}{2.312778in}}%
\pgfpathlineto{\pgfqpoint{2.383306in}{2.316734in}}%
\pgfpathclose%
\pgfusepath{fill}%
\end{pgfscope}%
\begin{pgfscope}%
\pgfpathrectangle{\pgfqpoint{1.254980in}{0.150000in}}{\pgfqpoint{5.490039in}{5.490039in}}%
\pgfusepath{clip}%
\pgfsetbuttcap%
\pgfsetroundjoin%
\definecolor{currentfill}{rgb}{0.149039,0.508051,0.557250}%
\pgfsetfillcolor{currentfill}%
\pgfsetfillopacity{0.700000}%
\pgfsetlinewidth{0.000000pt}%
\definecolor{currentstroke}{rgb}{0.000000,0.000000,0.000000}%
\pgfsetstrokecolor{currentstroke}%
\pgfsetdash{}{0pt}%
\pgfpathmoveto{\pgfqpoint{2.144246in}{2.749785in}}%
\pgfpathlineto{\pgfqpoint{2.158095in}{2.724574in}}%
\pgfpathlineto{\pgfqpoint{2.171927in}{2.699684in}}%
\pgfpathlineto{\pgfqpoint{2.185742in}{2.675109in}}%
\pgfpathlineto{\pgfqpoint{2.199541in}{2.650847in}}%
\pgfpathlineto{\pgfqpoint{2.208492in}{2.645791in}}%
\pgfpathlineto{\pgfqpoint{2.217420in}{2.641062in}}%
\pgfpathlineto{\pgfqpoint{2.226326in}{2.636655in}}%
\pgfpathlineto{\pgfqpoint{2.235209in}{2.632566in}}%
\pgfpathlineto{\pgfqpoint{2.221468in}{2.656244in}}%
\pgfpathlineto{\pgfqpoint{2.207712in}{2.680234in}}%
\pgfpathlineto{\pgfqpoint{2.193941in}{2.704539in}}%
\pgfpathlineto{\pgfqpoint{2.180153in}{2.729161in}}%
\pgfpathlineto{\pgfqpoint{2.171211in}{2.733823in}}%
\pgfpathlineto{\pgfqpoint{2.162246in}{2.738811in}}%
\pgfpathlineto{\pgfqpoint{2.153258in}{2.744129in}}%
\pgfpathlineto{\pgfqpoint{2.144246in}{2.749785in}}%
\pgfpathclose%
\pgfusepath{fill}%
\end{pgfscope}%
\begin{pgfscope}%
\pgfpathrectangle{\pgfqpoint{1.254980in}{0.150000in}}{\pgfqpoint{5.490039in}{5.490039in}}%
\pgfusepath{clip}%
\pgfsetbuttcap%
\pgfsetroundjoin%
\definecolor{currentfill}{rgb}{0.268510,0.009605,0.335427}%
\pgfsetfillcolor{currentfill}%
\pgfsetfillopacity{0.700000}%
\pgfsetlinewidth{0.000000pt}%
\definecolor{currentstroke}{rgb}{0.000000,0.000000,0.000000}%
\pgfsetstrokecolor{currentstroke}%
\pgfsetdash{}{0pt}%
\pgfpathmoveto{\pgfqpoint{3.479356in}{1.573095in}}%
\pgfpathlineto{\pgfqpoint{3.492679in}{1.570178in}}%
\pgfpathlineto{\pgfqpoint{3.506006in}{1.567433in}}%
\pgfpathlineto{\pgfqpoint{3.519338in}{1.564859in}}%
\pgfpathlineto{\pgfqpoint{3.532676in}{1.562454in}}%
\pgfpathlineto{\pgfqpoint{3.540650in}{1.571603in}}%
\pgfpathlineto{\pgfqpoint{3.548618in}{1.580845in}}%
\pgfpathlineto{\pgfqpoint{3.556580in}{1.590177in}}%
\pgfpathlineto{\pgfqpoint{3.564534in}{1.599594in}}%
\pgfpathlineto{\pgfqpoint{3.551212in}{1.601558in}}%
\pgfpathlineto{\pgfqpoint{3.537895in}{1.603693in}}%
\pgfpathlineto{\pgfqpoint{3.524584in}{1.605998in}}%
\pgfpathlineto{\pgfqpoint{3.511277in}{1.608475in}}%
\pgfpathlineto{\pgfqpoint{3.503307in}{1.599488in}}%
\pgfpathlineto{\pgfqpoint{3.495331in}{1.590592in}}%
\pgfpathlineto{\pgfqpoint{3.487347in}{1.581793in}}%
\pgfpathlineto{\pgfqpoint{3.479356in}{1.573095in}}%
\pgfpathclose%
\pgfusepath{fill}%
\end{pgfscope}%
\begin{pgfscope}%
\pgfpathrectangle{\pgfqpoint{1.254980in}{0.150000in}}{\pgfqpoint{5.490039in}{5.490039in}}%
\pgfusepath{clip}%
\pgfsetbuttcap%
\pgfsetroundjoin%
\definecolor{currentfill}{rgb}{0.283197,0.115680,0.436115}%
\pgfsetfillcolor{currentfill}%
\pgfsetfillopacity{0.700000}%
\pgfsetlinewidth{0.000000pt}%
\definecolor{currentstroke}{rgb}{0.000000,0.000000,0.000000}%
\pgfsetstrokecolor{currentstroke}%
\pgfsetdash{}{0pt}%
\pgfpathmoveto{\pgfqpoint{2.815289in}{1.798716in}}%
\pgfpathlineto{\pgfqpoint{2.828691in}{1.786248in}}%
\pgfpathlineto{\pgfqpoint{2.842090in}{1.773988in}}%
\pgfpathlineto{\pgfqpoint{2.855485in}{1.761934in}}%
\pgfpathlineto{\pgfqpoint{2.868877in}{1.750087in}}%
\pgfpathlineto{\pgfqpoint{2.877272in}{1.751428in}}%
\pgfpathlineto{\pgfqpoint{2.885653in}{1.753015in}}%
\pgfpathlineto{\pgfqpoint{2.894019in}{1.754841in}}%
\pgfpathlineto{\pgfqpoint{2.902372in}{1.756902in}}%
\pgfpathlineto{\pgfqpoint{2.889017in}{1.768186in}}%
\pgfpathlineto{\pgfqpoint{2.875660in}{1.779675in}}%
\pgfpathlineto{\pgfqpoint{2.862299in}{1.791371in}}%
\pgfpathlineto{\pgfqpoint{2.848936in}{1.803274in}}%
\pgfpathlineto{\pgfqpoint{2.840546in}{1.801766in}}%
\pgfpathlineto{\pgfqpoint{2.832142in}{1.800500in}}%
\pgfpathlineto{\pgfqpoint{2.823723in}{1.799481in}}%
\pgfpathlineto{\pgfqpoint{2.815289in}{1.798716in}}%
\pgfpathclose%
\pgfusepath{fill}%
\end{pgfscope}%
\begin{pgfscope}%
\pgfpathrectangle{\pgfqpoint{1.254980in}{0.150000in}}{\pgfqpoint{5.490039in}{5.490039in}}%
\pgfusepath{clip}%
\pgfsetbuttcap%
\pgfsetroundjoin%
\definecolor{currentfill}{rgb}{0.214000,0.722114,0.469588}%
\pgfsetfillcolor{currentfill}%
\pgfsetfillopacity{0.700000}%
\pgfsetlinewidth{0.000000pt}%
\definecolor{currentstroke}{rgb}{0.000000,0.000000,0.000000}%
\pgfsetstrokecolor{currentstroke}%
\pgfsetdash{}{0pt}%
\pgfpathmoveto{\pgfqpoint{5.620097in}{3.249415in}}%
\pgfpathlineto{\pgfqpoint{5.634430in}{3.262620in}}%
\pgfpathlineto{\pgfqpoint{5.648783in}{3.275986in}}%
\pgfpathlineto{\pgfqpoint{5.663156in}{3.289511in}}%
\pgfpathlineto{\pgfqpoint{5.677550in}{3.303197in}}%
\pgfpathlineto{\pgfqpoint{5.684560in}{3.305065in}}%
\pgfpathlineto{\pgfqpoint{5.691561in}{3.306868in}}%
\pgfpathlineto{\pgfqpoint{5.698552in}{3.308609in}}%
\pgfpathlineto{\pgfqpoint{5.705536in}{3.310293in}}%
\pgfpathlineto{\pgfqpoint{5.691166in}{3.297021in}}%
\pgfpathlineto{\pgfqpoint{5.676817in}{3.283908in}}%
\pgfpathlineto{\pgfqpoint{5.662488in}{3.270954in}}%
\pgfpathlineto{\pgfqpoint{5.648179in}{3.258159in}}%
\pgfpathlineto{\pgfqpoint{5.641171in}{3.256053in}}%
\pgfpathlineto{\pgfqpoint{5.634154in}{3.253896in}}%
\pgfpathlineto{\pgfqpoint{5.627130in}{3.251685in}}%
\pgfpathlineto{\pgfqpoint{5.620097in}{3.249415in}}%
\pgfpathclose%
\pgfusepath{fill}%
\end{pgfscope}%
\begin{pgfscope}%
\pgfpathrectangle{\pgfqpoint{1.254980in}{0.150000in}}{\pgfqpoint{5.490039in}{5.490039in}}%
\pgfusepath{clip}%
\pgfsetbuttcap%
\pgfsetroundjoin%
\definecolor{currentfill}{rgb}{0.177423,0.437527,0.557565}%
\pgfsetfillcolor{currentfill}%
\pgfsetfillopacity{0.700000}%
\pgfsetlinewidth{0.000000pt}%
\definecolor{currentstroke}{rgb}{0.000000,0.000000,0.000000}%
\pgfsetstrokecolor{currentstroke}%
\pgfsetdash{}{0pt}%
\pgfpathmoveto{\pgfqpoint{4.677182in}{2.464339in}}%
\pgfpathlineto{\pgfqpoint{4.690946in}{2.473659in}}%
\pgfpathlineto{\pgfqpoint{4.704726in}{2.483140in}}%
\pgfpathlineto{\pgfqpoint{4.718521in}{2.492783in}}%
\pgfpathlineto{\pgfqpoint{4.732332in}{2.502586in}}%
\pgfpathlineto{\pgfqpoint{4.739895in}{2.512316in}}%
\pgfpathlineto{\pgfqpoint{4.747451in}{2.521925in}}%
\pgfpathlineto{\pgfqpoint{4.755001in}{2.531413in}}%
\pgfpathlineto{\pgfqpoint{4.762543in}{2.540781in}}%
\pgfpathlineto{\pgfqpoint{4.748737in}{2.530989in}}%
\pgfpathlineto{\pgfqpoint{4.734946in}{2.521359in}}%
\pgfpathlineto{\pgfqpoint{4.721170in}{2.511890in}}%
\pgfpathlineto{\pgfqpoint{4.707410in}{2.502581in}}%
\pgfpathlineto{\pgfqpoint{4.699863in}{2.493190in}}%
\pgfpathlineto{\pgfqpoint{4.692309in}{2.483687in}}%
\pgfpathlineto{\pgfqpoint{4.684749in}{2.474070in}}%
\pgfpathlineto{\pgfqpoint{4.677182in}{2.464339in}}%
\pgfpathclose%
\pgfusepath{fill}%
\end{pgfscope}%
\begin{pgfscope}%
\pgfpathrectangle{\pgfqpoint{1.254980in}{0.150000in}}{\pgfqpoint{5.490039in}{5.490039in}}%
\pgfusepath{clip}%
\pgfsetbuttcap%
\pgfsetroundjoin%
\definecolor{currentfill}{rgb}{0.280868,0.160771,0.472899}%
\pgfsetfillcolor{currentfill}%
\pgfsetfillopacity{0.700000}%
\pgfsetlinewidth{0.000000pt}%
\definecolor{currentstroke}{rgb}{0.000000,0.000000,0.000000}%
\pgfsetstrokecolor{currentstroke}%
\pgfsetdash{}{0pt}%
\pgfpathmoveto{\pgfqpoint{3.989285in}{1.831653in}}%
\pgfpathlineto{\pgfqpoint{4.002724in}{1.834964in}}%
\pgfpathlineto{\pgfqpoint{4.016173in}{1.838437in}}%
\pgfpathlineto{\pgfqpoint{4.029631in}{1.842073in}}%
\pgfpathlineto{\pgfqpoint{4.043100in}{1.845872in}}%
\pgfpathlineto{\pgfqpoint{4.050893in}{1.857961in}}%
\pgfpathlineto{\pgfqpoint{4.058681in}{1.870021in}}%
\pgfpathlineto{\pgfqpoint{4.066465in}{1.882049in}}%
\pgfpathlineto{\pgfqpoint{4.074244in}{1.894043in}}%
\pgfpathlineto{\pgfqpoint{4.060780in}{1.889969in}}%
\pgfpathlineto{\pgfqpoint{4.047325in}{1.886057in}}%
\pgfpathlineto{\pgfqpoint{4.033881in}{1.882308in}}%
\pgfpathlineto{\pgfqpoint{4.020447in}{1.878723in}}%
\pgfpathlineto{\pgfqpoint{4.012664in}{1.866993in}}%
\pgfpathlineto{\pgfqpoint{4.004876in}{1.855237in}}%
\pgfpathlineto{\pgfqpoint{3.997083in}{1.843456in}}%
\pgfpathlineto{\pgfqpoint{3.989285in}{1.831653in}}%
\pgfpathclose%
\pgfusepath{fill}%
\end{pgfscope}%
\begin{pgfscope}%
\pgfpathrectangle{\pgfqpoint{1.254980in}{0.150000in}}{\pgfqpoint{5.490039in}{5.490039in}}%
\pgfusepath{clip}%
\pgfsetbuttcap%
\pgfsetroundjoin%
\definecolor{currentfill}{rgb}{0.210503,0.363727,0.552206}%
\pgfsetfillcolor{currentfill}%
\pgfsetfillopacity{0.700000}%
\pgfsetlinewidth{0.000000pt}%
\definecolor{currentstroke}{rgb}{0.000000,0.000000,0.000000}%
\pgfsetstrokecolor{currentstroke}%
\pgfsetdash{}{0pt}%
\pgfpathmoveto{\pgfqpoint{4.476246in}{2.270995in}}%
\pgfpathlineto{\pgfqpoint{4.489902in}{2.278884in}}%
\pgfpathlineto{\pgfqpoint{4.503573in}{2.286933in}}%
\pgfpathlineto{\pgfqpoint{4.517258in}{2.295144in}}%
\pgfpathlineto{\pgfqpoint{4.530956in}{2.303516in}}%
\pgfpathlineto{\pgfqpoint{4.538599in}{2.314536in}}%
\pgfpathlineto{\pgfqpoint{4.546235in}{2.325449in}}%
\pgfpathlineto{\pgfqpoint{4.553866in}{2.336256in}}%
\pgfpathlineto{\pgfqpoint{4.561491in}{2.346955in}}%
\pgfpathlineto{\pgfqpoint{4.547794in}{2.338506in}}%
\pgfpathlineto{\pgfqpoint{4.534112in}{2.330218in}}%
\pgfpathlineto{\pgfqpoint{4.520444in}{2.322092in}}%
\pgfpathlineto{\pgfqpoint{4.506790in}{2.314127in}}%
\pgfpathlineto{\pgfqpoint{4.499162in}{2.303493in}}%
\pgfpathlineto{\pgfqpoint{4.491529in}{2.292760in}}%
\pgfpathlineto{\pgfqpoint{4.483890in}{2.281927in}}%
\pgfpathlineto{\pgfqpoint{4.476246in}{2.270995in}}%
\pgfpathclose%
\pgfusepath{fill}%
\end{pgfscope}%
\begin{pgfscope}%
\pgfpathrectangle{\pgfqpoint{1.254980in}{0.150000in}}{\pgfqpoint{5.490039in}{5.490039in}}%
\pgfusepath{clip}%
\pgfsetbuttcap%
\pgfsetroundjoin%
\definecolor{currentfill}{rgb}{0.246811,0.283237,0.535941}%
\pgfsetfillcolor{currentfill}%
\pgfsetfillopacity{0.700000}%
\pgfsetlinewidth{0.000000pt}%
\definecolor{currentstroke}{rgb}{0.000000,0.000000,0.000000}%
\pgfsetstrokecolor{currentstroke}%
\pgfsetdash{}{0pt}%
\pgfpathmoveto{\pgfqpoint{4.275269in}{2.078259in}}%
\pgfpathlineto{\pgfqpoint{4.288828in}{2.084452in}}%
\pgfpathlineto{\pgfqpoint{4.302398in}{2.090807in}}%
\pgfpathlineto{\pgfqpoint{4.315982in}{2.097322in}}%
\pgfpathlineto{\pgfqpoint{4.329578in}{2.104000in}}%
\pgfpathlineto{\pgfqpoint{4.337288in}{2.115889in}}%
\pgfpathlineto{\pgfqpoint{4.344992in}{2.127697in}}%
\pgfpathlineto{\pgfqpoint{4.352692in}{2.139422in}}%
\pgfpathlineto{\pgfqpoint{4.360386in}{2.151063in}}%
\pgfpathlineto{\pgfqpoint{4.346792in}{2.144222in}}%
\pgfpathlineto{\pgfqpoint{4.333211in}{2.137543in}}%
\pgfpathlineto{\pgfqpoint{4.319643in}{2.131026in}}%
\pgfpathlineto{\pgfqpoint{4.306086in}{2.124670in}}%
\pgfpathlineto{\pgfqpoint{4.298390in}{2.113181in}}%
\pgfpathlineto{\pgfqpoint{4.290688in}{2.101615in}}%
\pgfpathlineto{\pgfqpoint{4.282981in}{2.089974in}}%
\pgfpathlineto{\pgfqpoint{4.275269in}{2.078259in}}%
\pgfpathclose%
\pgfusepath{fill}%
\end{pgfscope}%
\begin{pgfscope}%
\pgfpathrectangle{\pgfqpoint{1.254980in}{0.150000in}}{\pgfqpoint{5.490039in}{5.490039in}}%
\pgfusepath{clip}%
\pgfsetbuttcap%
\pgfsetroundjoin%
\definecolor{currentfill}{rgb}{0.267004,0.004874,0.329415}%
\pgfsetfillcolor{currentfill}%
\pgfsetfillopacity{0.700000}%
\pgfsetlinewidth{0.000000pt}%
\definecolor{currentstroke}{rgb}{0.000000,0.000000,0.000000}%
\pgfsetstrokecolor{currentstroke}%
\pgfsetdash{}{0pt}%
\pgfpathmoveto{\pgfqpoint{3.255124in}{1.567809in}}%
\pgfpathlineto{\pgfqpoint{3.268446in}{1.561824in}}%
\pgfpathlineto{\pgfqpoint{3.281769in}{1.556019in}}%
\pgfpathlineto{\pgfqpoint{3.295096in}{1.550392in}}%
\pgfpathlineto{\pgfqpoint{3.308424in}{1.544942in}}%
\pgfpathlineto{\pgfqpoint{3.316518in}{1.551661in}}%
\pgfpathlineto{\pgfqpoint{3.324602in}{1.558531in}}%
\pgfpathlineto{\pgfqpoint{3.332678in}{1.565548in}}%
\pgfpathlineto{\pgfqpoint{3.340745in}{1.572706in}}%
\pgfpathlineto{\pgfqpoint{3.327438in}{1.577659in}}%
\pgfpathlineto{\pgfqpoint{3.314135in}{1.582790in}}%
\pgfpathlineto{\pgfqpoint{3.300834in}{1.588099in}}%
\pgfpathlineto{\pgfqpoint{3.287535in}{1.593587in}}%
\pgfpathlineto{\pgfqpoint{3.279446in}{1.586914in}}%
\pgfpathlineto{\pgfqpoint{3.271348in}{1.580390in}}%
\pgfpathlineto{\pgfqpoint{3.263241in}{1.574020in}}%
\pgfpathlineto{\pgfqpoint{3.255124in}{1.567809in}}%
\pgfpathclose%
\pgfusepath{fill}%
\end{pgfscope}%
\begin{pgfscope}%
\pgfpathrectangle{\pgfqpoint{1.254980in}{0.150000in}}{\pgfqpoint{5.490039in}{5.490039in}}%
\pgfusepath{clip}%
\pgfsetbuttcap%
\pgfsetroundjoin%
\definecolor{currentfill}{rgb}{0.203063,0.379716,0.553925}%
\pgfsetfillcolor{currentfill}%
\pgfsetfillopacity{0.700000}%
\pgfsetlinewidth{0.000000pt}%
\definecolor{currentstroke}{rgb}{0.000000,0.000000,0.000000}%
\pgfsetstrokecolor{currentstroke}%
\pgfsetdash{}{0pt}%
\pgfpathmoveto{\pgfqpoint{2.328672in}{2.399343in}}%
\pgfpathlineto{\pgfqpoint{2.342348in}{2.378281in}}%
\pgfpathlineto{\pgfqpoint{2.356012in}{2.357494in}}%
\pgfpathlineto{\pgfqpoint{2.369665in}{2.336979in}}%
\pgfpathlineto{\pgfqpoint{2.383306in}{2.316734in}}%
\pgfpathlineto{\pgfqpoint{2.392116in}{2.312778in}}%
\pgfpathlineto{\pgfqpoint{2.400905in}{2.309139in}}%
\pgfpathlineto{\pgfqpoint{2.409673in}{2.305812in}}%
\pgfpathlineto{\pgfqpoint{2.418421in}{2.302790in}}%
\pgfpathlineto{\pgfqpoint{2.404834in}{2.322440in}}%
\pgfpathlineto{\pgfqpoint{2.391236in}{2.342359in}}%
\pgfpathlineto{\pgfqpoint{2.377627in}{2.362549in}}%
\pgfpathlineto{\pgfqpoint{2.364006in}{2.383012in}}%
\pgfpathlineto{\pgfqpoint{2.355204in}{2.386618in}}%
\pgfpathlineto{\pgfqpoint{2.346382in}{2.390537in}}%
\pgfpathlineto{\pgfqpoint{2.337538in}{2.394777in}}%
\pgfpathlineto{\pgfqpoint{2.328672in}{2.399343in}}%
\pgfpathclose%
\pgfusepath{fill}%
\end{pgfscope}%
\begin{pgfscope}%
\pgfpathrectangle{\pgfqpoint{1.254980in}{0.150000in}}{\pgfqpoint{5.490039in}{5.490039in}}%
\pgfusepath{clip}%
\pgfsetbuttcap%
\pgfsetroundjoin%
\definecolor{currentfill}{rgb}{0.259857,0.745492,0.444467}%
\pgfsetfillcolor{currentfill}%
\pgfsetfillopacity{0.700000}%
\pgfsetlinewidth{0.000000pt}%
\definecolor{currentstroke}{rgb}{0.000000,0.000000,0.000000}%
\pgfsetstrokecolor{currentstroke}%
\pgfsetdash{}{0pt}%
\pgfpathmoveto{\pgfqpoint{5.705536in}{3.310293in}}%
\pgfpathlineto{\pgfqpoint{5.719926in}{3.323725in}}%
\pgfpathlineto{\pgfqpoint{5.734338in}{3.337317in}}%
\pgfpathlineto{\pgfqpoint{5.748770in}{3.351068in}}%
\pgfpathlineto{\pgfqpoint{5.763223in}{3.364980in}}%
\pgfpathlineto{\pgfqpoint{5.770172in}{3.366178in}}%
\pgfpathlineto{\pgfqpoint{5.777112in}{3.367322in}}%
\pgfpathlineto{\pgfqpoint{5.784044in}{3.368416in}}%
\pgfpathlineto{\pgfqpoint{5.790968in}{3.369465in}}%
\pgfpathlineto{\pgfqpoint{5.776541in}{3.355999in}}%
\pgfpathlineto{\pgfqpoint{5.762136in}{3.342692in}}%
\pgfpathlineto{\pgfqpoint{5.747751in}{3.329544in}}%
\pgfpathlineto{\pgfqpoint{5.733387in}{3.316554in}}%
\pgfpathlineto{\pgfqpoint{5.726436in}{3.315051in}}%
\pgfpathlineto{\pgfqpoint{5.719478in}{3.313509in}}%
\pgfpathlineto{\pgfqpoint{5.712511in}{3.311925in}}%
\pgfpathlineto{\pgfqpoint{5.705536in}{3.310293in}}%
\pgfpathclose%
\pgfusepath{fill}%
\end{pgfscope}%
\begin{pgfscope}%
\pgfpathrectangle{\pgfqpoint{1.254980in}{0.150000in}}{\pgfqpoint{5.490039in}{5.490039in}}%
\pgfusepath{clip}%
\pgfsetbuttcap%
\pgfsetroundjoin%
\definecolor{currentfill}{rgb}{0.120638,0.625828,0.533488}%
\pgfsetfillcolor{currentfill}%
\pgfsetfillopacity{0.700000}%
\pgfsetlinewidth{0.000000pt}%
\definecolor{currentstroke}{rgb}{0.000000,0.000000,0.000000}%
\pgfsetstrokecolor{currentstroke}%
\pgfsetdash{}{0pt}%
\pgfpathmoveto{\pgfqpoint{5.249424in}{2.970197in}}%
\pgfpathlineto{\pgfqpoint{5.263535in}{2.982433in}}%
\pgfpathlineto{\pgfqpoint{5.277664in}{2.994830in}}%
\pgfpathlineto{\pgfqpoint{5.291812in}{3.007388in}}%
\pgfpathlineto{\pgfqpoint{5.305979in}{3.020106in}}%
\pgfpathlineto{\pgfqpoint{5.313242in}{3.025139in}}%
\pgfpathlineto{\pgfqpoint{5.320497in}{3.030060in}}%
\pgfpathlineto{\pgfqpoint{5.327743in}{3.034873in}}%
\pgfpathlineto{\pgfqpoint{5.334981in}{3.039580in}}%
\pgfpathlineto{\pgfqpoint{5.320828in}{3.027119in}}%
\pgfpathlineto{\pgfqpoint{5.306694in}{3.014818in}}%
\pgfpathlineto{\pgfqpoint{5.292579in}{3.002677in}}%
\pgfpathlineto{\pgfqpoint{5.278483in}{2.990696in}}%
\pgfpathlineto{\pgfqpoint{5.271231in}{2.985721in}}%
\pgfpathlineto{\pgfqpoint{5.263970in}{2.980649in}}%
\pgfpathlineto{\pgfqpoint{5.256702in}{2.975475in}}%
\pgfpathlineto{\pgfqpoint{5.249424in}{2.970197in}}%
\pgfpathclose%
\pgfusepath{fill}%
\end{pgfscope}%
\begin{pgfscope}%
\pgfpathrectangle{\pgfqpoint{1.254980in}{0.150000in}}{\pgfqpoint{5.490039in}{5.490039in}}%
\pgfusepath{clip}%
\pgfsetbuttcap%
\pgfsetroundjoin%
\definecolor{currentfill}{rgb}{0.282327,0.094955,0.417331}%
\pgfsetfillcolor{currentfill}%
\pgfsetfillopacity{0.700000}%
\pgfsetlinewidth{0.000000pt}%
\definecolor{currentstroke}{rgb}{0.000000,0.000000,0.000000}%
\pgfsetstrokecolor{currentstroke}%
\pgfsetdash{}{0pt}%
\pgfpathmoveto{\pgfqpoint{2.868877in}{1.750087in}}%
\pgfpathlineto{\pgfqpoint{2.882266in}{1.738443in}}%
\pgfpathlineto{\pgfqpoint{2.895652in}{1.727002in}}%
\pgfpathlineto{\pgfqpoint{2.909036in}{1.715763in}}%
\pgfpathlineto{\pgfqpoint{2.922418in}{1.704725in}}%
\pgfpathlineto{\pgfqpoint{2.930776in}{1.706640in}}%
\pgfpathlineto{\pgfqpoint{2.939120in}{1.708793in}}%
\pgfpathlineto{\pgfqpoint{2.947451in}{1.711178in}}%
\pgfpathlineto{\pgfqpoint{2.955769in}{1.713789in}}%
\pgfpathlineto{\pgfqpoint{2.942422in}{1.724266in}}%
\pgfpathlineto{\pgfqpoint{2.929074in}{1.734943in}}%
\pgfpathlineto{\pgfqpoint{2.915724in}{1.745821in}}%
\pgfpathlineto{\pgfqpoint{2.902372in}{1.756902in}}%
\pgfpathlineto{\pgfqpoint{2.894019in}{1.754841in}}%
\pgfpathlineto{\pgfqpoint{2.885653in}{1.753015in}}%
\pgfpathlineto{\pgfqpoint{2.877272in}{1.751428in}}%
\pgfpathlineto{\pgfqpoint{2.868877in}{1.750087in}}%
\pgfpathclose%
\pgfusepath{fill}%
\end{pgfscope}%
\begin{pgfscope}%
\pgfpathrectangle{\pgfqpoint{1.254980in}{0.150000in}}{\pgfqpoint{5.490039in}{5.490039in}}%
\pgfusepath{clip}%
\pgfsetbuttcap%
\pgfsetroundjoin%
\definecolor{currentfill}{rgb}{0.137770,0.537492,0.554906}%
\pgfsetfillcolor{currentfill}%
\pgfsetfillopacity{0.700000}%
\pgfsetlinewidth{0.000000pt}%
\definecolor{currentstroke}{rgb}{0.000000,0.000000,0.000000}%
\pgfsetstrokecolor{currentstroke}%
\pgfsetdash{}{0pt}%
\pgfpathmoveto{\pgfqpoint{4.963417in}{2.726546in}}%
\pgfpathlineto{\pgfqpoint{4.977354in}{2.737570in}}%
\pgfpathlineto{\pgfqpoint{4.991308in}{2.748755in}}%
\pgfpathlineto{\pgfqpoint{5.005279in}{2.760101in}}%
\pgfpathlineto{\pgfqpoint{5.019268in}{2.771608in}}%
\pgfpathlineto{\pgfqpoint{5.026698in}{2.779139in}}%
\pgfpathlineto{\pgfqpoint{5.034120in}{2.786544in}}%
\pgfpathlineto{\pgfqpoint{5.041534in}{2.793825in}}%
\pgfpathlineto{\pgfqpoint{5.048940in}{2.800983in}}%
\pgfpathlineto{\pgfqpoint{5.034960in}{2.789610in}}%
\pgfpathlineto{\pgfqpoint{5.020997in}{2.778397in}}%
\pgfpathlineto{\pgfqpoint{5.007051in}{2.767345in}}%
\pgfpathlineto{\pgfqpoint{4.993122in}{2.756454in}}%
\pgfpathlineto{\pgfqpoint{4.985707in}{2.749152in}}%
\pgfpathlineto{\pgfqpoint{4.978285in}{2.741734in}}%
\pgfpathlineto{\pgfqpoint{4.970855in}{2.734199in}}%
\pgfpathlineto{\pgfqpoint{4.963417in}{2.726546in}}%
\pgfpathclose%
\pgfusepath{fill}%
\end{pgfscope}%
\begin{pgfscope}%
\pgfpathrectangle{\pgfqpoint{1.254980in}{0.150000in}}{\pgfqpoint{5.490039in}{5.490039in}}%
\pgfusepath{clip}%
\pgfsetbuttcap%
\pgfsetroundjoin%
\definecolor{currentfill}{rgb}{0.275191,0.194905,0.496005}%
\pgfsetfillcolor{currentfill}%
\pgfsetfillopacity{0.700000}%
\pgfsetlinewidth{0.000000pt}%
\definecolor{currentstroke}{rgb}{0.000000,0.000000,0.000000}%
\pgfsetstrokecolor{currentstroke}%
\pgfsetdash{}{0pt}%
\pgfpathmoveto{\pgfqpoint{4.074244in}{1.894043in}}%
\pgfpathlineto{\pgfqpoint{4.087720in}{1.898280in}}%
\pgfpathlineto{\pgfqpoint{4.101206in}{1.902680in}}%
\pgfpathlineto{\pgfqpoint{4.114703in}{1.907241in}}%
\pgfpathlineto{\pgfqpoint{4.128210in}{1.911965in}}%
\pgfpathlineto{\pgfqpoint{4.135982in}{1.924181in}}%
\pgfpathlineto{\pgfqpoint{4.143748in}{1.936352in}}%
\pgfpathlineto{\pgfqpoint{4.151510in}{1.948475in}}%
\pgfpathlineto{\pgfqpoint{4.159267in}{1.960548in}}%
\pgfpathlineto{\pgfqpoint{4.145763in}{1.955577in}}%
\pgfpathlineto{\pgfqpoint{4.132269in}{1.950767in}}%
\pgfpathlineto{\pgfqpoint{4.118787in}{1.946120in}}%
\pgfpathlineto{\pgfqpoint{4.105315in}{1.941635in}}%
\pgfpathlineto{\pgfqpoint{4.097554in}{1.929799in}}%
\pgfpathlineto{\pgfqpoint{4.089789in}{1.917921in}}%
\pgfpathlineto{\pgfqpoint{4.082019in}{1.906001in}}%
\pgfpathlineto{\pgfqpoint{4.074244in}{1.894043in}}%
\pgfpathclose%
\pgfusepath{fill}%
\end{pgfscope}%
\begin{pgfscope}%
\pgfpathrectangle{\pgfqpoint{1.254980in}{0.150000in}}{\pgfqpoint{5.490039in}{5.490039in}}%
\pgfusepath{clip}%
\pgfsetbuttcap%
\pgfsetroundjoin%
\definecolor{currentfill}{rgb}{0.267004,0.004874,0.329415}%
\pgfsetfillcolor{currentfill}%
\pgfsetfillopacity{0.700000}%
\pgfsetlinewidth{0.000000pt}%
\definecolor{currentstroke}{rgb}{0.000000,0.000000,0.000000}%
\pgfsetstrokecolor{currentstroke}%
\pgfsetdash{}{0pt}%
\pgfpathmoveto{\pgfqpoint{3.394004in}{1.554656in}}%
\pgfpathlineto{\pgfqpoint{3.407327in}{1.550581in}}%
\pgfpathlineto{\pgfqpoint{3.420654in}{1.546679in}}%
\pgfpathlineto{\pgfqpoint{3.433986in}{1.542951in}}%
\pgfpathlineto{\pgfqpoint{3.447321in}{1.539395in}}%
\pgfpathlineto{\pgfqpoint{3.455341in}{1.547647in}}%
\pgfpathlineto{\pgfqpoint{3.463354in}{1.556017in}}%
\pgfpathlineto{\pgfqpoint{3.471359in}{1.564501in}}%
\pgfpathlineto{\pgfqpoint{3.479356in}{1.573095in}}%
\pgfpathlineto{\pgfqpoint{3.466039in}{1.576183in}}%
\pgfpathlineto{\pgfqpoint{3.452725in}{1.579444in}}%
\pgfpathlineto{\pgfqpoint{3.439416in}{1.582878in}}%
\pgfpathlineto{\pgfqpoint{3.426111in}{1.586485in}}%
\pgfpathlineto{\pgfqpoint{3.418096in}{1.578349in}}%
\pgfpathlineto{\pgfqpoint{3.410073in}{1.570329in}}%
\pgfpathlineto{\pgfqpoint{3.402042in}{1.562430in}}%
\pgfpathlineto{\pgfqpoint{3.394004in}{1.554656in}}%
\pgfpathclose%
\pgfusepath{fill}%
\end{pgfscope}%
\begin{pgfscope}%
\pgfpathrectangle{\pgfqpoint{1.254980in}{0.150000in}}{\pgfqpoint{5.490039in}{5.490039in}}%
\pgfusepath{clip}%
\pgfsetbuttcap%
\pgfsetroundjoin%
\definecolor{currentfill}{rgb}{0.344074,0.780029,0.397381}%
\pgfsetfillcolor{currentfill}%
\pgfsetfillopacity{0.700000}%
\pgfsetlinewidth{0.000000pt}%
\definecolor{currentstroke}{rgb}{0.000000,0.000000,0.000000}%
\pgfsetstrokecolor{currentstroke}%
\pgfsetdash{}{0pt}%
\pgfpathmoveto{\pgfqpoint{5.876382in}{3.426874in}}%
\pgfpathlineto{\pgfqpoint{5.890886in}{3.440661in}}%
\pgfpathlineto{\pgfqpoint{5.905410in}{3.454609in}}%
\pgfpathlineto{\pgfqpoint{5.919957in}{3.468715in}}%
\pgfpathlineto{\pgfqpoint{5.926788in}{3.468751in}}%
\pgfpathlineto{\pgfqpoint{5.933611in}{3.468761in}}%
\pgfpathlineto{\pgfqpoint{5.940426in}{3.468750in}}%
\pgfpathlineto{\pgfqpoint{5.947233in}{3.468724in}}%
\pgfpathlineto{\pgfqpoint{5.932719in}{3.455124in}}%
\pgfpathlineto{\pgfqpoint{5.918225in}{3.441684in}}%
\pgfpathlineto{\pgfqpoint{5.903753in}{3.428402in}}%
\pgfpathlineto{\pgfqpoint{5.896922in}{3.428040in}}%
\pgfpathlineto{\pgfqpoint{5.890083in}{3.427669in}}%
\pgfpathlineto{\pgfqpoint{5.883236in}{3.427282in}}%
\pgfpathlineto{\pgfqpoint{5.876382in}{3.426874in}}%
\pgfpathclose%
\pgfusepath{fill}%
\end{pgfscope}%
\begin{pgfscope}%
\pgfpathrectangle{\pgfqpoint{1.254980in}{0.150000in}}{\pgfqpoint{5.490039in}{5.490039in}}%
\pgfusepath{clip}%
\pgfsetbuttcap%
\pgfsetroundjoin%
\definecolor{currentfill}{rgb}{0.271305,0.019942,0.347269}%
\pgfsetfillcolor{currentfill}%
\pgfsetfillopacity{0.700000}%
\pgfsetlinewidth{0.000000pt}%
\definecolor{currentstroke}{rgb}{0.000000,0.000000,0.000000}%
\pgfsetstrokecolor{currentstroke}%
\pgfsetdash{}{0pt}%
\pgfpathmoveto{\pgfqpoint{3.115837in}{1.603276in}}%
\pgfpathlineto{\pgfqpoint{3.129175in}{1.595304in}}%
\pgfpathlineto{\pgfqpoint{3.142513in}{1.587517in}}%
\pgfpathlineto{\pgfqpoint{3.155851in}{1.579914in}}%
\pgfpathlineto{\pgfqpoint{3.169191in}{1.572496in}}%
\pgfpathlineto{\pgfqpoint{3.177373in}{1.577487in}}%
\pgfpathlineto{\pgfqpoint{3.185545in}{1.582663in}}%
\pgfpathlineto{\pgfqpoint{3.193706in}{1.588020in}}%
\pgfpathlineto{\pgfqpoint{3.201858in}{1.593553in}}%
\pgfpathlineto{\pgfqpoint{3.188545in}{1.600445in}}%
\pgfpathlineto{\pgfqpoint{3.175234in}{1.607521in}}%
\pgfpathlineto{\pgfqpoint{3.161923in}{1.614781in}}%
\pgfpathlineto{\pgfqpoint{3.148614in}{1.622227in}}%
\pgfpathlineto{\pgfqpoint{3.140436in}{1.617209in}}%
\pgfpathlineto{\pgfqpoint{3.132247in}{1.612375in}}%
\pgfpathlineto{\pgfqpoint{3.124048in}{1.607729in}}%
\pgfpathlineto{\pgfqpoint{3.115837in}{1.603276in}}%
\pgfpathclose%
\pgfusepath{fill}%
\end{pgfscope}%
\begin{pgfscope}%
\pgfpathrectangle{\pgfqpoint{1.254980in}{0.150000in}}{\pgfqpoint{5.490039in}{5.490039in}}%
\pgfusepath{clip}%
\pgfsetbuttcap%
\pgfsetroundjoin%
\definecolor{currentfill}{rgb}{0.304148,0.764704,0.419943}%
\pgfsetfillcolor{currentfill}%
\pgfsetfillopacity{0.700000}%
\pgfsetlinewidth{0.000000pt}%
\definecolor{currentstroke}{rgb}{0.000000,0.000000,0.000000}%
\pgfsetstrokecolor{currentstroke}%
\pgfsetdash{}{0pt}%
\pgfpathmoveto{\pgfqpoint{5.790968in}{3.369465in}}%
\pgfpathlineto{\pgfqpoint{5.805415in}{3.383091in}}%
\pgfpathlineto{\pgfqpoint{5.819884in}{3.396877in}}%
\pgfpathlineto{\pgfqpoint{5.834374in}{3.410822in}}%
\pgfpathlineto{\pgfqpoint{5.848885in}{3.424928in}}%
\pgfpathlineto{\pgfqpoint{5.855772in}{3.425472in}}%
\pgfpathlineto{\pgfqpoint{5.862650in}{3.425974in}}%
\pgfpathlineto{\pgfqpoint{5.869520in}{3.426440in}}%
\pgfpathlineto{\pgfqpoint{5.876382in}{3.426874in}}%
\pgfpathlineto{\pgfqpoint{5.861900in}{3.413245in}}%
\pgfpathlineto{\pgfqpoint{5.847440in}{3.399776in}}%
\pgfpathlineto{\pgfqpoint{5.833000in}{3.386466in}}%
\pgfpathlineto{\pgfqpoint{5.818581in}{3.373314in}}%
\pgfpathlineto{\pgfqpoint{5.811690in}{3.372394in}}%
\pgfpathlineto{\pgfqpoint{5.804790in}{3.371449in}}%
\pgfpathlineto{\pgfqpoint{5.797883in}{3.370475in}}%
\pgfpathlineto{\pgfqpoint{5.790968in}{3.369465in}}%
\pgfpathclose%
\pgfusepath{fill}%
\end{pgfscope}%
\begin{pgfscope}%
\pgfpathrectangle{\pgfqpoint{1.254980in}{0.150000in}}{\pgfqpoint{5.490039in}{5.490039in}}%
\pgfusepath{clip}%
\pgfsetbuttcap%
\pgfsetroundjoin%
\definecolor{currentfill}{rgb}{0.163625,0.471133,0.558148}%
\pgfsetfillcolor{currentfill}%
\pgfsetfillopacity{0.700000}%
\pgfsetlinewidth{0.000000pt}%
\definecolor{currentstroke}{rgb}{0.000000,0.000000,0.000000}%
\pgfsetstrokecolor{currentstroke}%
\pgfsetdash{}{0pt}%
\pgfpathmoveto{\pgfqpoint{4.762543in}{2.540781in}}%
\pgfpathlineto{\pgfqpoint{4.776366in}{2.550733in}}%
\pgfpathlineto{\pgfqpoint{4.790205in}{2.560847in}}%
\pgfpathlineto{\pgfqpoint{4.804059in}{2.571121in}}%
\pgfpathlineto{\pgfqpoint{4.817930in}{2.581557in}}%
\pgfpathlineto{\pgfqpoint{4.825462in}{2.590775in}}%
\pgfpathlineto{\pgfqpoint{4.832986in}{2.599866in}}%
\pgfpathlineto{\pgfqpoint{4.840504in}{2.608832in}}%
\pgfpathlineto{\pgfqpoint{4.848014in}{2.617674in}}%
\pgfpathlineto{\pgfqpoint{4.834148in}{2.607280in}}%
\pgfpathlineto{\pgfqpoint{4.820298in}{2.597048in}}%
\pgfpathlineto{\pgfqpoint{4.806465in}{2.586976in}}%
\pgfpathlineto{\pgfqpoint{4.792647in}{2.577065in}}%
\pgfpathlineto{\pgfqpoint{4.785131in}{2.568171in}}%
\pgfpathlineto{\pgfqpoint{4.777609in}{2.559159in}}%
\pgfpathlineto{\pgfqpoint{4.770080in}{2.550029in}}%
\pgfpathlineto{\pgfqpoint{4.762543in}{2.540781in}}%
\pgfpathclose%
\pgfusepath{fill}%
\end{pgfscope}%
\begin{pgfscope}%
\pgfpathrectangle{\pgfqpoint{1.254980in}{0.150000in}}{\pgfqpoint{5.490039in}{5.490039in}}%
\pgfusepath{clip}%
\pgfsetbuttcap%
\pgfsetroundjoin%
\definecolor{currentfill}{rgb}{0.185556,0.418570,0.556753}%
\pgfsetfillcolor{currentfill}%
\pgfsetfillopacity{0.700000}%
\pgfsetlinewidth{0.000000pt}%
\definecolor{currentstroke}{rgb}{0.000000,0.000000,0.000000}%
\pgfsetstrokecolor{currentstroke}%
\pgfsetdash{}{0pt}%
\pgfpathmoveto{\pgfqpoint{2.273839in}{2.486388in}}%
\pgfpathlineto{\pgfqpoint{2.287567in}{2.464202in}}%
\pgfpathlineto{\pgfqpoint{2.301281in}{2.442301in}}%
\pgfpathlineto{\pgfqpoint{2.314983in}{2.420682in}}%
\pgfpathlineto{\pgfqpoint{2.328672in}{2.399343in}}%
\pgfpathlineto{\pgfqpoint{2.337538in}{2.394777in}}%
\pgfpathlineto{\pgfqpoint{2.346382in}{2.390537in}}%
\pgfpathlineto{\pgfqpoint{2.355204in}{2.386618in}}%
\pgfpathlineto{\pgfqpoint{2.364006in}{2.383012in}}%
\pgfpathlineto{\pgfqpoint{2.350373in}{2.403751in}}%
\pgfpathlineto{\pgfqpoint{2.336728in}{2.424768in}}%
\pgfpathlineto{\pgfqpoint{2.323070in}{2.446066in}}%
\pgfpathlineto{\pgfqpoint{2.309400in}{2.467648in}}%
\pgfpathlineto{\pgfqpoint{2.300543in}{2.471843in}}%
\pgfpathlineto{\pgfqpoint{2.291664in}{2.476361in}}%
\pgfpathlineto{\pgfqpoint{2.282763in}{2.481207in}}%
\pgfpathlineto{\pgfqpoint{2.273839in}{2.486388in}}%
\pgfpathclose%
\pgfusepath{fill}%
\end{pgfscope}%
\begin{pgfscope}%
\pgfpathrectangle{\pgfqpoint{1.254980in}{0.150000in}}{\pgfqpoint{5.490039in}{5.490039in}}%
\pgfusepath{clip}%
\pgfsetbuttcap%
\pgfsetroundjoin%
\definecolor{currentfill}{rgb}{0.280894,0.078907,0.402329}%
\pgfsetfillcolor{currentfill}%
\pgfsetfillopacity{0.700000}%
\pgfsetlinewidth{0.000000pt}%
\definecolor{currentstroke}{rgb}{0.000000,0.000000,0.000000}%
\pgfsetstrokecolor{currentstroke}%
\pgfsetdash{}{0pt}%
\pgfpathmoveto{\pgfqpoint{2.922418in}{1.704725in}}%
\pgfpathlineto{\pgfqpoint{2.935797in}{1.693886in}}%
\pgfpathlineto{\pgfqpoint{2.949174in}{1.683246in}}%
\pgfpathlineto{\pgfqpoint{2.962550in}{1.672802in}}%
\pgfpathlineto{\pgfqpoint{2.975924in}{1.662555in}}%
\pgfpathlineto{\pgfqpoint{2.984247in}{1.665042in}}%
\pgfpathlineto{\pgfqpoint{2.992556in}{1.667758in}}%
\pgfpathlineto{\pgfqpoint{3.000853in}{1.670700in}}%
\pgfpathlineto{\pgfqpoint{3.009137in}{1.673860in}}%
\pgfpathlineto{\pgfqpoint{2.995797in}{1.683548in}}%
\pgfpathlineto{\pgfqpoint{2.982456in}{1.693431in}}%
\pgfpathlineto{\pgfqpoint{2.969113in}{1.703511in}}%
\pgfpathlineto{\pgfqpoint{2.955769in}{1.713789in}}%
\pgfpathlineto{\pgfqpoint{2.947451in}{1.711178in}}%
\pgfpathlineto{\pgfqpoint{2.939120in}{1.708793in}}%
\pgfpathlineto{\pgfqpoint{2.930776in}{1.706640in}}%
\pgfpathlineto{\pgfqpoint{2.922418in}{1.704725in}}%
\pgfpathclose%
\pgfusepath{fill}%
\end{pgfscope}%
\begin{pgfscope}%
\pgfpathrectangle{\pgfqpoint{1.254980in}{0.150000in}}{\pgfqpoint{5.490039in}{5.490039in}}%
\pgfusepath{clip}%
\pgfsetbuttcap%
\pgfsetroundjoin%
\definecolor{currentfill}{rgb}{0.194100,0.399323,0.555565}%
\pgfsetfillcolor{currentfill}%
\pgfsetfillopacity{0.700000}%
\pgfsetlinewidth{0.000000pt}%
\definecolor{currentstroke}{rgb}{0.000000,0.000000,0.000000}%
\pgfsetstrokecolor{currentstroke}%
\pgfsetdash{}{0pt}%
\pgfpathmoveto{\pgfqpoint{4.561491in}{2.346955in}}%
\pgfpathlineto{\pgfqpoint{4.575202in}{2.355565in}}%
\pgfpathlineto{\pgfqpoint{4.588927in}{2.364336in}}%
\pgfpathlineto{\pgfqpoint{4.602668in}{2.373269in}}%
\pgfpathlineto{\pgfqpoint{4.616423in}{2.382362in}}%
\pgfpathlineto{\pgfqpoint{4.624039in}{2.393012in}}%
\pgfpathlineto{\pgfqpoint{4.631650in}{2.403547in}}%
\pgfpathlineto{\pgfqpoint{4.639254in}{2.413967in}}%
\pgfpathlineto{\pgfqpoint{4.646852in}{2.424271in}}%
\pgfpathlineto{\pgfqpoint{4.633100in}{2.415130in}}%
\pgfpathlineto{\pgfqpoint{4.619362in}{2.406150in}}%
\pgfpathlineto{\pgfqpoint{4.605639in}{2.397332in}}%
\pgfpathlineto{\pgfqpoint{4.591931in}{2.388674in}}%
\pgfpathlineto{\pgfqpoint{4.584330in}{2.378406in}}%
\pgfpathlineto{\pgfqpoint{4.576723in}{2.368030in}}%
\pgfpathlineto{\pgfqpoint{4.569110in}{2.357546in}}%
\pgfpathlineto{\pgfqpoint{4.561491in}{2.346955in}}%
\pgfpathclose%
\pgfusepath{fill}%
\end{pgfscope}%
\begin{pgfscope}%
\pgfpathrectangle{\pgfqpoint{1.254980in}{0.150000in}}{\pgfqpoint{5.490039in}{5.490039in}}%
\pgfusepath{clip}%
\pgfsetbuttcap%
\pgfsetroundjoin%
\definecolor{currentfill}{rgb}{0.132268,0.655014,0.519661}%
\pgfsetfillcolor{currentfill}%
\pgfsetfillopacity{0.700000}%
\pgfsetlinewidth{0.000000pt}%
\definecolor{currentstroke}{rgb}{0.000000,0.000000,0.000000}%
\pgfsetstrokecolor{currentstroke}%
\pgfsetdash{}{0pt}%
\pgfpathmoveto{\pgfqpoint{5.334981in}{3.039580in}}%
\pgfpathlineto{\pgfqpoint{5.349153in}{3.052202in}}%
\pgfpathlineto{\pgfqpoint{5.363343in}{3.064984in}}%
\pgfpathlineto{\pgfqpoint{5.377554in}{3.077928in}}%
\pgfpathlineto{\pgfqpoint{5.391783in}{3.091032in}}%
\pgfpathlineto{\pgfqpoint{5.398997in}{3.095360in}}%
\pgfpathlineto{\pgfqpoint{5.406201in}{3.099582in}}%
\pgfpathlineto{\pgfqpoint{5.413396in}{3.103701in}}%
\pgfpathlineto{\pgfqpoint{5.420583in}{3.107720in}}%
\pgfpathlineto{\pgfqpoint{5.406369in}{3.094905in}}%
\pgfpathlineto{\pgfqpoint{5.392175in}{3.082250in}}%
\pgfpathlineto{\pgfqpoint{5.378000in}{3.069755in}}%
\pgfpathlineto{\pgfqpoint{5.363844in}{3.057420in}}%
\pgfpathlineto{\pgfqpoint{5.356641in}{3.053102in}}%
\pgfpathlineto{\pgfqpoint{5.349430in}{3.048692in}}%
\pgfpathlineto{\pgfqpoint{5.342210in}{3.044185in}}%
\pgfpathlineto{\pgfqpoint{5.334981in}{3.039580in}}%
\pgfpathclose%
\pgfusepath{fill}%
\end{pgfscope}%
\begin{pgfscope}%
\pgfpathrectangle{\pgfqpoint{1.254980in}{0.150000in}}{\pgfqpoint{5.490039in}{5.490039in}}%
\pgfusepath{clip}%
\pgfsetbuttcap%
\pgfsetroundjoin%
\definecolor{currentfill}{rgb}{0.278791,0.062145,0.386592}%
\pgfsetfillcolor{currentfill}%
\pgfsetfillopacity{0.700000}%
\pgfsetlinewidth{0.000000pt}%
\definecolor{currentstroke}{rgb}{0.000000,0.000000,0.000000}%
\pgfsetstrokecolor{currentstroke}%
\pgfsetdash{}{0pt}%
\pgfpathmoveto{\pgfqpoint{3.702982in}{1.631691in}}%
\pgfpathlineto{\pgfqpoint{3.716348in}{1.631651in}}%
\pgfpathlineto{\pgfqpoint{3.729722in}{1.631777in}}%
\pgfpathlineto{\pgfqpoint{3.743103in}{1.632069in}}%
\pgfpathlineto{\pgfqpoint{3.756491in}{1.632526in}}%
\pgfpathlineto{\pgfqpoint{3.764381in}{1.643495in}}%
\pgfpathlineto{\pgfqpoint{3.772266in}{1.654506in}}%
\pgfpathlineto{\pgfqpoint{3.780146in}{1.665553in}}%
\pgfpathlineto{\pgfqpoint{3.788020in}{1.676635in}}%
\pgfpathlineto{\pgfqpoint{3.774641in}{1.675793in}}%
\pgfpathlineto{\pgfqpoint{3.761269in}{1.675116in}}%
\pgfpathlineto{\pgfqpoint{3.747905in}{1.674604in}}%
\pgfpathlineto{\pgfqpoint{3.734549in}{1.674259in}}%
\pgfpathlineto{\pgfqpoint{3.726665in}{1.663552in}}%
\pgfpathlineto{\pgfqpoint{3.718776in}{1.652886in}}%
\pgfpathlineto{\pgfqpoint{3.710882in}{1.642265in}}%
\pgfpathlineto{\pgfqpoint{3.702982in}{1.631691in}}%
\pgfpathclose%
\pgfusepath{fill}%
\end{pgfscope}%
\begin{pgfscope}%
\pgfpathrectangle{\pgfqpoint{1.254980in}{0.150000in}}{\pgfqpoint{5.490039in}{5.490039in}}%
\pgfusepath{clip}%
\pgfsetbuttcap%
\pgfsetroundjoin%
\definecolor{currentfill}{rgb}{0.231674,0.318106,0.544834}%
\pgfsetfillcolor{currentfill}%
\pgfsetfillopacity{0.700000}%
\pgfsetlinewidth{0.000000pt}%
\definecolor{currentstroke}{rgb}{0.000000,0.000000,0.000000}%
\pgfsetstrokecolor{currentstroke}%
\pgfsetdash{}{0pt}%
\pgfpathmoveto{\pgfqpoint{4.360386in}{2.151063in}}%
\pgfpathlineto{\pgfqpoint{4.373993in}{2.158066in}}%
\pgfpathlineto{\pgfqpoint{4.387614in}{2.165229in}}%
\pgfpathlineto{\pgfqpoint{4.401247in}{2.172554in}}%
\pgfpathlineto{\pgfqpoint{4.414894in}{2.180040in}}%
\pgfpathlineto{\pgfqpoint{4.422582in}{2.191743in}}%
\pgfpathlineto{\pgfqpoint{4.430264in}{2.203352in}}%
\pgfpathlineto{\pgfqpoint{4.437941in}{2.214866in}}%
\pgfpathlineto{\pgfqpoint{4.445613in}{2.226286in}}%
\pgfpathlineto{\pgfqpoint{4.431968in}{2.218665in}}%
\pgfpathlineto{\pgfqpoint{4.418336in}{2.211205in}}%
\pgfpathlineto{\pgfqpoint{4.404718in}{2.203906in}}%
\pgfpathlineto{\pgfqpoint{4.391113in}{2.196769in}}%
\pgfpathlineto{\pgfqpoint{4.383439in}{2.185474in}}%
\pgfpathlineto{\pgfqpoint{4.375760in}{2.174090in}}%
\pgfpathlineto{\pgfqpoint{4.368076in}{2.162620in}}%
\pgfpathlineto{\pgfqpoint{4.360386in}{2.151063in}}%
\pgfpathclose%
\pgfusepath{fill}%
\end{pgfscope}%
\begin{pgfscope}%
\pgfpathrectangle{\pgfqpoint{1.254980in}{0.150000in}}{\pgfqpoint{5.490039in}{5.490039in}}%
\pgfusepath{clip}%
\pgfsetbuttcap%
\pgfsetroundjoin%
\definecolor{currentfill}{rgb}{0.281446,0.084320,0.407414}%
\pgfsetfillcolor{currentfill}%
\pgfsetfillopacity{0.700000}%
\pgfsetlinewidth{0.000000pt}%
\definecolor{currentstroke}{rgb}{0.000000,0.000000,0.000000}%
\pgfsetstrokecolor{currentstroke}%
\pgfsetdash{}{0pt}%
\pgfpathmoveto{\pgfqpoint{3.788020in}{1.676635in}}%
\pgfpathlineto{\pgfqpoint{3.801407in}{1.677642in}}%
\pgfpathlineto{\pgfqpoint{3.814802in}{1.678814in}}%
\pgfpathlineto{\pgfqpoint{3.828205in}{1.680151in}}%
\pgfpathlineto{\pgfqpoint{3.841617in}{1.681651in}}%
\pgfpathlineto{\pgfqpoint{3.849478in}{1.693132in}}%
\pgfpathlineto{\pgfqpoint{3.857334in}{1.704633in}}%
\pgfpathlineto{\pgfqpoint{3.865185in}{1.716150in}}%
\pgfpathlineto{\pgfqpoint{3.873031in}{1.727681in}}%
\pgfpathlineto{\pgfqpoint{3.859627in}{1.725822in}}%
\pgfpathlineto{\pgfqpoint{3.846231in}{1.724128in}}%
\pgfpathlineto{\pgfqpoint{3.832844in}{1.722598in}}%
\pgfpathlineto{\pgfqpoint{3.819465in}{1.721232in}}%
\pgfpathlineto{\pgfqpoint{3.811612in}{1.710049in}}%
\pgfpathlineto{\pgfqpoint{3.803753in}{1.698886in}}%
\pgfpathlineto{\pgfqpoint{3.795889in}{1.687747in}}%
\pgfpathlineto{\pgfqpoint{3.788020in}{1.676635in}}%
\pgfpathclose%
\pgfusepath{fill}%
\end{pgfscope}%
\begin{pgfscope}%
\pgfpathrectangle{\pgfqpoint{1.254980in}{0.150000in}}{\pgfqpoint{5.490039in}{5.490039in}}%
\pgfusepath{clip}%
\pgfsetbuttcap%
\pgfsetroundjoin%
\definecolor{currentfill}{rgb}{0.274952,0.037752,0.364543}%
\pgfsetfillcolor{currentfill}%
\pgfsetfillopacity{0.700000}%
\pgfsetlinewidth{0.000000pt}%
\definecolor{currentstroke}{rgb}{0.000000,0.000000,0.000000}%
\pgfsetstrokecolor{currentstroke}%
\pgfsetdash{}{0pt}%
\pgfpathmoveto{\pgfqpoint{3.617881in}{1.593430in}}%
\pgfpathlineto{\pgfqpoint{3.631232in}{1.592310in}}%
\pgfpathlineto{\pgfqpoint{3.644590in}{1.591357in}}%
\pgfpathlineto{\pgfqpoint{3.657954in}{1.590571in}}%
\pgfpathlineto{\pgfqpoint{3.671325in}{1.589952in}}%
\pgfpathlineto{\pgfqpoint{3.679248in}{1.600296in}}%
\pgfpathlineto{\pgfqpoint{3.687165in}{1.610703in}}%
\pgfpathlineto{\pgfqpoint{3.695076in}{1.621170in}}%
\pgfpathlineto{\pgfqpoint{3.702982in}{1.631691in}}%
\pgfpathlineto{\pgfqpoint{3.689622in}{1.631898in}}%
\pgfpathlineto{\pgfqpoint{3.676270in}{1.632271in}}%
\pgfpathlineto{\pgfqpoint{3.662924in}{1.632811in}}%
\pgfpathlineto{\pgfqpoint{3.649585in}{1.633518in}}%
\pgfpathlineto{\pgfqpoint{3.641668in}{1.623399in}}%
\pgfpathlineto{\pgfqpoint{3.633745in}{1.613341in}}%
\pgfpathlineto{\pgfqpoint{3.625816in}{1.603351in}}%
\pgfpathlineto{\pgfqpoint{3.617881in}{1.593430in}}%
\pgfpathclose%
\pgfusepath{fill}%
\end{pgfscope}%
\begin{pgfscope}%
\pgfpathrectangle{\pgfqpoint{1.254980in}{0.150000in}}{\pgfqpoint{5.490039in}{5.490039in}}%
\pgfusepath{clip}%
\pgfsetbuttcap%
\pgfsetroundjoin%
\definecolor{currentfill}{rgb}{0.265145,0.232956,0.516599}%
\pgfsetfillcolor{currentfill}%
\pgfsetfillopacity{0.700000}%
\pgfsetlinewidth{0.000000pt}%
\definecolor{currentstroke}{rgb}{0.000000,0.000000,0.000000}%
\pgfsetstrokecolor{currentstroke}%
\pgfsetdash{}{0pt}%
\pgfpathmoveto{\pgfqpoint{4.159267in}{1.960548in}}%
\pgfpathlineto{\pgfqpoint{4.172784in}{1.965682in}}%
\pgfpathlineto{\pgfqpoint{4.186311in}{1.970977in}}%
\pgfpathlineto{\pgfqpoint{4.199850in}{1.976435in}}%
\pgfpathlineto{\pgfqpoint{4.213401in}{1.982053in}}%
\pgfpathlineto{\pgfqpoint{4.221151in}{1.994305in}}%
\pgfpathlineto{\pgfqpoint{4.228897in}{2.006496in}}%
\pgfpathlineto{\pgfqpoint{4.236637in}{2.018624in}}%
\pgfpathlineto{\pgfqpoint{4.244373in}{2.030688in}}%
\pgfpathlineto{\pgfqpoint{4.230825in}{2.024848in}}%
\pgfpathlineto{\pgfqpoint{4.217288in}{2.019171in}}%
\pgfpathlineto{\pgfqpoint{4.203763in}{2.013655in}}%
\pgfpathlineto{\pgfqpoint{4.190250in}{2.008302in}}%
\pgfpathlineto{\pgfqpoint{4.182511in}{1.996448in}}%
\pgfpathlineto{\pgfqpoint{4.174768in}{1.984536in}}%
\pgfpathlineto{\pgfqpoint{4.167020in}{1.972569in}}%
\pgfpathlineto{\pgfqpoint{4.159267in}{1.960548in}}%
\pgfpathclose%
\pgfusepath{fill}%
\end{pgfscope}%
\begin{pgfscope}%
\pgfpathrectangle{\pgfqpoint{1.254980in}{0.150000in}}{\pgfqpoint{5.490039in}{5.490039in}}%
\pgfusepath{clip}%
\pgfsetbuttcap%
\pgfsetroundjoin%
\definecolor{currentfill}{rgb}{0.126453,0.570633,0.549841}%
\pgfsetfillcolor{currentfill}%
\pgfsetfillopacity{0.700000}%
\pgfsetlinewidth{0.000000pt}%
\definecolor{currentstroke}{rgb}{0.000000,0.000000,0.000000}%
\pgfsetstrokecolor{currentstroke}%
\pgfsetdash{}{0pt}%
\pgfpathmoveto{\pgfqpoint{5.048940in}{2.800983in}}%
\pgfpathlineto{\pgfqpoint{5.062939in}{2.812518in}}%
\pgfpathlineto{\pgfqpoint{5.076955in}{2.824213in}}%
\pgfpathlineto{\pgfqpoint{5.090988in}{2.836070in}}%
\pgfpathlineto{\pgfqpoint{5.105040in}{2.848088in}}%
\pgfpathlineto{\pgfqpoint{5.112429in}{2.854972in}}%
\pgfpathlineto{\pgfqpoint{5.119810in}{2.861730in}}%
\pgfpathlineto{\pgfqpoint{5.127183in}{2.868364in}}%
\pgfpathlineto{\pgfqpoint{5.134548in}{2.874875in}}%
\pgfpathlineto{\pgfqpoint{5.120505in}{2.863022in}}%
\pgfpathlineto{\pgfqpoint{5.106481in}{2.851330in}}%
\pgfpathlineto{\pgfqpoint{5.092475in}{2.839799in}}%
\pgfpathlineto{\pgfqpoint{5.078486in}{2.828428in}}%
\pgfpathlineto{\pgfqpoint{5.071112in}{2.821741in}}%
\pgfpathlineto{\pgfqpoint{5.063729in}{2.814939in}}%
\pgfpathlineto{\pgfqpoint{5.056339in}{2.808021in}}%
\pgfpathlineto{\pgfqpoint{5.048940in}{2.800983in}}%
\pgfpathclose%
\pgfusepath{fill}%
\end{pgfscope}%
\begin{pgfscope}%
\pgfpathrectangle{\pgfqpoint{1.254980in}{0.150000in}}{\pgfqpoint{5.490039in}{5.490039in}}%
\pgfusepath{clip}%
\pgfsetbuttcap%
\pgfsetroundjoin%
\definecolor{currentfill}{rgb}{0.283197,0.115680,0.436115}%
\pgfsetfillcolor{currentfill}%
\pgfsetfillopacity{0.700000}%
\pgfsetlinewidth{0.000000pt}%
\definecolor{currentstroke}{rgb}{0.000000,0.000000,0.000000}%
\pgfsetstrokecolor{currentstroke}%
\pgfsetdash{}{0pt}%
\pgfpathmoveto{\pgfqpoint{3.873031in}{1.727681in}}%
\pgfpathlineto{\pgfqpoint{3.886444in}{1.729704in}}%
\pgfpathlineto{\pgfqpoint{3.899866in}{1.731890in}}%
\pgfpathlineto{\pgfqpoint{3.913297in}{1.734240in}}%
\pgfpathlineto{\pgfqpoint{3.926737in}{1.736753in}}%
\pgfpathlineto{\pgfqpoint{3.934572in}{1.748635in}}%
\pgfpathlineto{\pgfqpoint{3.942402in}{1.760518in}}%
\pgfpathlineto{\pgfqpoint{3.950228in}{1.772398in}}%
\pgfpathlineto{\pgfqpoint{3.958049in}{1.784272in}}%
\pgfpathlineto{\pgfqpoint{3.944614in}{1.781428in}}%
\pgfpathlineto{\pgfqpoint{3.931190in}{1.778747in}}%
\pgfpathlineto{\pgfqpoint{3.917774in}{1.776230in}}%
\pgfpathlineto{\pgfqpoint{3.904368in}{1.773877in}}%
\pgfpathlineto{\pgfqpoint{3.896541in}{1.762323in}}%
\pgfpathlineto{\pgfqpoint{3.888709in}{1.750771in}}%
\pgfpathlineto{\pgfqpoint{3.880873in}{1.739222in}}%
\pgfpathlineto{\pgfqpoint{3.873031in}{1.727681in}}%
\pgfpathclose%
\pgfusepath{fill}%
\end{pgfscope}%
\begin{pgfscope}%
\pgfpathrectangle{\pgfqpoint{1.254980in}{0.150000in}}{\pgfqpoint{5.490039in}{5.490039in}}%
\pgfusepath{clip}%
\pgfsetbuttcap%
\pgfsetroundjoin%
\definecolor{currentfill}{rgb}{0.269944,0.014625,0.341379}%
\pgfsetfillcolor{currentfill}%
\pgfsetfillopacity{0.700000}%
\pgfsetlinewidth{0.000000pt}%
\definecolor{currentstroke}{rgb}{0.000000,0.000000,0.000000}%
\pgfsetstrokecolor{currentstroke}%
\pgfsetdash{}{0pt}%
\pgfpathmoveto{\pgfqpoint{3.532676in}{1.562454in}}%
\pgfpathlineto{\pgfqpoint{3.546018in}{1.560220in}}%
\pgfpathlineto{\pgfqpoint{3.559366in}{1.558155in}}%
\pgfpathlineto{\pgfqpoint{3.572719in}{1.556258in}}%
\pgfpathlineto{\pgfqpoint{3.586078in}{1.554530in}}%
\pgfpathlineto{\pgfqpoint{3.594038in}{1.564129in}}%
\pgfpathlineto{\pgfqpoint{3.601992in}{1.573815in}}%
\pgfpathlineto{\pgfqpoint{3.609939in}{1.583583in}}%
\pgfpathlineto{\pgfqpoint{3.617881in}{1.593430in}}%
\pgfpathlineto{\pgfqpoint{3.604535in}{1.594718in}}%
\pgfpathlineto{\pgfqpoint{3.591196in}{1.596174in}}%
\pgfpathlineto{\pgfqpoint{3.577862in}{1.597799in}}%
\pgfpathlineto{\pgfqpoint{3.564534in}{1.599594in}}%
\pgfpathlineto{\pgfqpoint{3.556580in}{1.590177in}}%
\pgfpathlineto{\pgfqpoint{3.548618in}{1.580845in}}%
\pgfpathlineto{\pgfqpoint{3.540650in}{1.571603in}}%
\pgfpathlineto{\pgfqpoint{3.532676in}{1.562454in}}%
\pgfpathclose%
\pgfusepath{fill}%
\end{pgfscope}%
\begin{pgfscope}%
\pgfpathrectangle{\pgfqpoint{1.254980in}{0.150000in}}{\pgfqpoint{5.490039in}{5.490039in}}%
\pgfusepath{clip}%
\pgfsetbuttcap%
\pgfsetroundjoin%
\definecolor{currentfill}{rgb}{0.267004,0.004874,0.329415}%
\pgfsetfillcolor{currentfill}%
\pgfsetfillopacity{0.700000}%
\pgfsetlinewidth{0.000000pt}%
\definecolor{currentstroke}{rgb}{0.000000,0.000000,0.000000}%
\pgfsetstrokecolor{currentstroke}%
\pgfsetdash{}{0pt}%
\pgfpathmoveto{\pgfqpoint{3.308424in}{1.544942in}}%
\pgfpathlineto{\pgfqpoint{3.321755in}{1.539670in}}%
\pgfpathlineto{\pgfqpoint{3.335090in}{1.534574in}}%
\pgfpathlineto{\pgfqpoint{3.348427in}{1.529653in}}%
\pgfpathlineto{\pgfqpoint{3.361767in}{1.524908in}}%
\pgfpathlineto{\pgfqpoint{3.369839in}{1.532133in}}%
\pgfpathlineto{\pgfqpoint{3.377902in}{1.539503in}}%
\pgfpathlineto{\pgfqpoint{3.385957in}{1.547012in}}%
\pgfpathlineto{\pgfqpoint{3.394004in}{1.554656in}}%
\pgfpathlineto{\pgfqpoint{3.380684in}{1.558906in}}%
\pgfpathlineto{\pgfqpoint{3.367368in}{1.563330in}}%
\pgfpathlineto{\pgfqpoint{3.354055in}{1.567930in}}%
\pgfpathlineto{\pgfqpoint{3.340745in}{1.572706in}}%
\pgfpathlineto{\pgfqpoint{3.332678in}{1.565548in}}%
\pgfpathlineto{\pgfqpoint{3.324602in}{1.558531in}}%
\pgfpathlineto{\pgfqpoint{3.316518in}{1.551661in}}%
\pgfpathlineto{\pgfqpoint{3.308424in}{1.544942in}}%
\pgfpathclose%
\pgfusepath{fill}%
\end{pgfscope}%
\begin{pgfscope}%
\pgfpathrectangle{\pgfqpoint{1.254980in}{0.150000in}}{\pgfqpoint{5.490039in}{5.490039in}}%
\pgfusepath{clip}%
\pgfsetbuttcap%
\pgfsetroundjoin%
\definecolor{currentfill}{rgb}{0.269944,0.014625,0.341379}%
\pgfsetfillcolor{currentfill}%
\pgfsetfillopacity{0.700000}%
\pgfsetlinewidth{0.000000pt}%
\definecolor{currentstroke}{rgb}{0.000000,0.000000,0.000000}%
\pgfsetstrokecolor{currentstroke}%
\pgfsetdash{}{0pt}%
\pgfpathmoveto{\pgfqpoint{3.169191in}{1.572496in}}%
\pgfpathlineto{\pgfqpoint{3.182531in}{1.565262in}}%
\pgfpathlineto{\pgfqpoint{3.195873in}{1.558209in}}%
\pgfpathlineto{\pgfqpoint{3.209216in}{1.551339in}}%
\pgfpathlineto{\pgfqpoint{3.222561in}{1.544649in}}%
\pgfpathlineto{\pgfqpoint{3.230717in}{1.550175in}}%
\pgfpathlineto{\pgfqpoint{3.238862in}{1.555881in}}%
\pgfpathlineto{\pgfqpoint{3.246998in}{1.561761in}}%
\pgfpathlineto{\pgfqpoint{3.255124in}{1.567809in}}%
\pgfpathlineto{\pgfqpoint{3.241805in}{1.573973in}}%
\pgfpathlineto{\pgfqpoint{3.228488in}{1.580318in}}%
\pgfpathlineto{\pgfqpoint{3.215172in}{1.586845in}}%
\pgfpathlineto{\pgfqpoint{3.201858in}{1.593553in}}%
\pgfpathlineto{\pgfqpoint{3.193706in}{1.588020in}}%
\pgfpathlineto{\pgfqpoint{3.185545in}{1.582663in}}%
\pgfpathlineto{\pgfqpoint{3.177373in}{1.577487in}}%
\pgfpathlineto{\pgfqpoint{3.169191in}{1.572496in}}%
\pgfpathclose%
\pgfusepath{fill}%
\end{pgfscope}%
\begin{pgfscope}%
\pgfpathrectangle{\pgfqpoint{1.254980in}{0.150000in}}{\pgfqpoint{5.490039in}{5.490039in}}%
\pgfusepath{clip}%
\pgfsetbuttcap%
\pgfsetroundjoin%
\definecolor{currentfill}{rgb}{0.278791,0.062145,0.386592}%
\pgfsetfillcolor{currentfill}%
\pgfsetfillopacity{0.700000}%
\pgfsetlinewidth{0.000000pt}%
\definecolor{currentstroke}{rgb}{0.000000,0.000000,0.000000}%
\pgfsetstrokecolor{currentstroke}%
\pgfsetdash{}{0pt}%
\pgfpathmoveto{\pgfqpoint{2.975924in}{1.662555in}}%
\pgfpathlineto{\pgfqpoint{2.989297in}{1.652502in}}%
\pgfpathlineto{\pgfqpoint{3.002668in}{1.642644in}}%
\pgfpathlineto{\pgfqpoint{3.016038in}{1.632978in}}%
\pgfpathlineto{\pgfqpoint{3.029408in}{1.623504in}}%
\pgfpathlineto{\pgfqpoint{3.037697in}{1.626561in}}%
\pgfpathlineto{\pgfqpoint{3.045973in}{1.629840in}}%
\pgfpathlineto{\pgfqpoint{3.054238in}{1.633336in}}%
\pgfpathlineto{\pgfqpoint{3.062490in}{1.637044in}}%
\pgfpathlineto{\pgfqpoint{3.049152in}{1.645960in}}%
\pgfpathlineto{\pgfqpoint{3.035815in}{1.655067in}}%
\pgfpathlineto{\pgfqpoint{3.022476in}{1.664367in}}%
\pgfpathlineto{\pgfqpoint{3.009137in}{1.673860in}}%
\pgfpathlineto{\pgfqpoint{3.000853in}{1.670700in}}%
\pgfpathlineto{\pgfqpoint{2.992556in}{1.667758in}}%
\pgfpathlineto{\pgfqpoint{2.984247in}{1.665042in}}%
\pgfpathlineto{\pgfqpoint{2.975924in}{1.662555in}}%
\pgfpathclose%
\pgfusepath{fill}%
\end{pgfscope}%
\begin{pgfscope}%
\pgfpathrectangle{\pgfqpoint{1.254980in}{0.150000in}}{\pgfqpoint{5.490039in}{5.490039in}}%
\pgfusepath{clip}%
\pgfsetbuttcap%
\pgfsetroundjoin%
\definecolor{currentfill}{rgb}{0.153894,0.680203,0.504172}%
\pgfsetfillcolor{currentfill}%
\pgfsetfillopacity{0.700000}%
\pgfsetlinewidth{0.000000pt}%
\definecolor{currentstroke}{rgb}{0.000000,0.000000,0.000000}%
\pgfsetstrokecolor{currentstroke}%
\pgfsetdash{}{0pt}%
\pgfpathmoveto{\pgfqpoint{5.420583in}{3.107720in}}%
\pgfpathlineto{\pgfqpoint{5.434816in}{3.120696in}}%
\pgfpathlineto{\pgfqpoint{5.449068in}{3.133832in}}%
\pgfpathlineto{\pgfqpoint{5.463341in}{3.147130in}}%
\pgfpathlineto{\pgfqpoint{5.477633in}{3.160588in}}%
\pgfpathlineto{\pgfqpoint{5.484793in}{3.164202in}}%
\pgfpathlineto{\pgfqpoint{5.491944in}{3.167716in}}%
\pgfpathlineto{\pgfqpoint{5.499086in}{3.171133in}}%
\pgfpathlineto{\pgfqpoint{5.506219in}{3.174458in}}%
\pgfpathlineto{\pgfqpoint{5.491945in}{3.161321in}}%
\pgfpathlineto{\pgfqpoint{5.477690in}{3.148344in}}%
\pgfpathlineto{\pgfqpoint{5.463456in}{3.135527in}}%
\pgfpathlineto{\pgfqpoint{5.449241in}{3.122870in}}%
\pgfpathlineto{\pgfqpoint{5.442089in}{3.119214in}}%
\pgfpathlineto{\pgfqpoint{5.434929in}{3.115473in}}%
\pgfpathlineto{\pgfqpoint{5.427760in}{3.111643in}}%
\pgfpathlineto{\pgfqpoint{5.420583in}{3.107720in}}%
\pgfpathclose%
\pgfusepath{fill}%
\end{pgfscope}%
\begin{pgfscope}%
\pgfpathrectangle{\pgfqpoint{1.254980in}{0.150000in}}{\pgfqpoint{5.490039in}{5.490039in}}%
\pgfusepath{clip}%
\pgfsetbuttcap%
\pgfsetroundjoin%
\definecolor{currentfill}{rgb}{0.171176,0.452530,0.557965}%
\pgfsetfillcolor{currentfill}%
\pgfsetfillopacity{0.700000}%
\pgfsetlinewidth{0.000000pt}%
\definecolor{currentstroke}{rgb}{0.000000,0.000000,0.000000}%
\pgfsetstrokecolor{currentstroke}%
\pgfsetdash{}{0pt}%
\pgfpathmoveto{\pgfqpoint{2.218789in}{2.578038in}}%
\pgfpathlineto{\pgfqpoint{2.232573in}{2.554684in}}%
\pgfpathlineto{\pgfqpoint{2.246343in}{2.531626in}}%
\pgfpathlineto{\pgfqpoint{2.260098in}{2.508862in}}%
\pgfpathlineto{\pgfqpoint{2.273839in}{2.486388in}}%
\pgfpathlineto{\pgfqpoint{2.282763in}{2.481207in}}%
\pgfpathlineto{\pgfqpoint{2.291664in}{2.476361in}}%
\pgfpathlineto{\pgfqpoint{2.300543in}{2.471843in}}%
\pgfpathlineto{\pgfqpoint{2.309400in}{2.467648in}}%
\pgfpathlineto{\pgfqpoint{2.295717in}{2.489516in}}%
\pgfpathlineto{\pgfqpoint{2.282020in}{2.511674in}}%
\pgfpathlineto{\pgfqpoint{2.268310in}{2.534123in}}%
\pgfpathlineto{\pgfqpoint{2.254585in}{2.556866in}}%
\pgfpathlineto{\pgfqpoint{2.245671in}{2.561656in}}%
\pgfpathlineto{\pgfqpoint{2.236734in}{2.566778in}}%
\pgfpathlineto{\pgfqpoint{2.227773in}{2.572236in}}%
\pgfpathlineto{\pgfqpoint{2.218789in}{2.578038in}}%
\pgfpathclose%
\pgfusepath{fill}%
\end{pgfscope}%
\begin{pgfscope}%
\pgfpathrectangle{\pgfqpoint{1.254980in}{0.150000in}}{\pgfqpoint{5.490039in}{5.490039in}}%
\pgfusepath{clip}%
\pgfsetbuttcap%
\pgfsetroundjoin%
\definecolor{currentfill}{rgb}{0.150476,0.504369,0.557430}%
\pgfsetfillcolor{currentfill}%
\pgfsetfillopacity{0.700000}%
\pgfsetlinewidth{0.000000pt}%
\definecolor{currentstroke}{rgb}{0.000000,0.000000,0.000000}%
\pgfsetstrokecolor{currentstroke}%
\pgfsetdash{}{0pt}%
\pgfpathmoveto{\pgfqpoint{4.848014in}{2.617674in}}%
\pgfpathlineto{\pgfqpoint{4.861896in}{2.628229in}}%
\pgfpathlineto{\pgfqpoint{4.875796in}{2.638945in}}%
\pgfpathlineto{\pgfqpoint{4.889711in}{2.649822in}}%
\pgfpathlineto{\pgfqpoint{4.903644in}{2.660860in}}%
\pgfpathlineto{\pgfqpoint{4.911142in}{2.669517in}}%
\pgfpathlineto{\pgfqpoint{4.918632in}{2.678044in}}%
\pgfpathlineto{\pgfqpoint{4.926115in}{2.686442in}}%
\pgfpathlineto{\pgfqpoint{4.933590in}{2.694713in}}%
\pgfpathlineto{\pgfqpoint{4.919664in}{2.683747in}}%
\pgfpathlineto{\pgfqpoint{4.905754in}{2.672943in}}%
\pgfpathlineto{\pgfqpoint{4.891861in}{2.662299in}}%
\pgfpathlineto{\pgfqpoint{4.877984in}{2.651817in}}%
\pgfpathlineto{\pgfqpoint{4.870502in}{2.643462in}}%
\pgfpathlineto{\pgfqpoint{4.863013in}{2.634988in}}%
\pgfpathlineto{\pgfqpoint{4.855517in}{2.626392in}}%
\pgfpathlineto{\pgfqpoint{4.848014in}{2.617674in}}%
\pgfpathclose%
\pgfusepath{fill}%
\end{pgfscope}%
\begin{pgfscope}%
\pgfpathrectangle{\pgfqpoint{1.254980in}{0.150000in}}{\pgfqpoint{5.490039in}{5.490039in}}%
\pgfusepath{clip}%
\pgfsetbuttcap%
\pgfsetroundjoin%
\definecolor{currentfill}{rgb}{0.281887,0.150881,0.465405}%
\pgfsetfillcolor{currentfill}%
\pgfsetfillopacity{0.700000}%
\pgfsetlinewidth{0.000000pt}%
\definecolor{currentstroke}{rgb}{0.000000,0.000000,0.000000}%
\pgfsetstrokecolor{currentstroke}%
\pgfsetdash{}{0pt}%
\pgfpathmoveto{\pgfqpoint{3.958049in}{1.784272in}}%
\pgfpathlineto{\pgfqpoint{3.971492in}{1.787279in}}%
\pgfpathlineto{\pgfqpoint{3.984946in}{1.790449in}}%
\pgfpathlineto{\pgfqpoint{3.998409in}{1.793782in}}%
\pgfpathlineto{\pgfqpoint{4.011882in}{1.797276in}}%
\pgfpathlineto{\pgfqpoint{4.019693in}{1.809456in}}%
\pgfpathlineto{\pgfqpoint{4.027500in}{1.821617in}}%
\pgfpathlineto{\pgfqpoint{4.035302in}{1.833757in}}%
\pgfpathlineto{\pgfqpoint{4.043100in}{1.845872in}}%
\pgfpathlineto{\pgfqpoint{4.029631in}{1.842073in}}%
\pgfpathlineto{\pgfqpoint{4.016173in}{1.838437in}}%
\pgfpathlineto{\pgfqpoint{4.002724in}{1.834964in}}%
\pgfpathlineto{\pgfqpoint{3.989285in}{1.831653in}}%
\pgfpathlineto{\pgfqpoint{3.981483in}{1.819831in}}%
\pgfpathlineto{\pgfqpoint{3.973676in}{1.807991in}}%
\pgfpathlineto{\pgfqpoint{3.965865in}{1.796138in}}%
\pgfpathlineto{\pgfqpoint{3.958049in}{1.784272in}}%
\pgfpathclose%
\pgfusepath{fill}%
\end{pgfscope}%
\begin{pgfscope}%
\pgfpathrectangle{\pgfqpoint{1.254980in}{0.150000in}}{\pgfqpoint{5.490039in}{5.490039in}}%
\pgfusepath{clip}%
\pgfsetbuttcap%
\pgfsetroundjoin%
\definecolor{currentfill}{rgb}{0.179019,0.433756,0.557430}%
\pgfsetfillcolor{currentfill}%
\pgfsetfillopacity{0.700000}%
\pgfsetlinewidth{0.000000pt}%
\definecolor{currentstroke}{rgb}{0.000000,0.000000,0.000000}%
\pgfsetstrokecolor{currentstroke}%
\pgfsetdash{}{0pt}%
\pgfpathmoveto{\pgfqpoint{4.646852in}{2.424271in}}%
\pgfpathlineto{\pgfqpoint{4.660620in}{2.433573in}}%
\pgfpathlineto{\pgfqpoint{4.674403in}{2.443037in}}%
\pgfpathlineto{\pgfqpoint{4.688201in}{2.452662in}}%
\pgfpathlineto{\pgfqpoint{4.702015in}{2.462447in}}%
\pgfpathlineto{\pgfqpoint{4.709604in}{2.472665in}}%
\pgfpathlineto{\pgfqpoint{4.717187in}{2.482761in}}%
\pgfpathlineto{\pgfqpoint{4.724763in}{2.492735in}}%
\pgfpathlineto{\pgfqpoint{4.732332in}{2.502586in}}%
\pgfpathlineto{\pgfqpoint{4.718521in}{2.492783in}}%
\pgfpathlineto{\pgfqpoint{4.704726in}{2.483140in}}%
\pgfpathlineto{\pgfqpoint{4.690946in}{2.473659in}}%
\pgfpathlineto{\pgfqpoint{4.677182in}{2.464339in}}%
\pgfpathlineto{\pgfqpoint{4.669609in}{2.454494in}}%
\pgfpathlineto{\pgfqpoint{4.662030in}{2.444535in}}%
\pgfpathlineto{\pgfqpoint{4.654444in}{2.434461in}}%
\pgfpathlineto{\pgfqpoint{4.646852in}{2.424271in}}%
\pgfpathclose%
\pgfusepath{fill}%
\end{pgfscope}%
\begin{pgfscope}%
\pgfpathrectangle{\pgfqpoint{1.254980in}{0.150000in}}{\pgfqpoint{5.490039in}{5.490039in}}%
\pgfusepath{clip}%
\pgfsetbuttcap%
\pgfsetroundjoin%
\definecolor{currentfill}{rgb}{0.214298,0.355619,0.551184}%
\pgfsetfillcolor{currentfill}%
\pgfsetfillopacity{0.700000}%
\pgfsetlinewidth{0.000000pt}%
\definecolor{currentstroke}{rgb}{0.000000,0.000000,0.000000}%
\pgfsetstrokecolor{currentstroke}%
\pgfsetdash{}{0pt}%
\pgfpathmoveto{\pgfqpoint{4.445613in}{2.226286in}}%
\pgfpathlineto{\pgfqpoint{4.459272in}{2.234068in}}%
\pgfpathlineto{\pgfqpoint{4.472944in}{2.242012in}}%
\pgfpathlineto{\pgfqpoint{4.486631in}{2.250117in}}%
\pgfpathlineto{\pgfqpoint{4.500331in}{2.258383in}}%
\pgfpathlineto{\pgfqpoint{4.507996in}{2.269823in}}%
\pgfpathlineto{\pgfqpoint{4.515655in}{2.281159in}}%
\pgfpathlineto{\pgfqpoint{4.523308in}{2.292391in}}%
\pgfpathlineto{\pgfqpoint{4.530956in}{2.303516in}}%
\pgfpathlineto{\pgfqpoint{4.517258in}{2.295144in}}%
\pgfpathlineto{\pgfqpoint{4.503573in}{2.286933in}}%
\pgfpathlineto{\pgfqpoint{4.489902in}{2.278884in}}%
\pgfpathlineto{\pgfqpoint{4.476246in}{2.270995in}}%
\pgfpathlineto{\pgfqpoint{4.468596in}{2.259965in}}%
\pgfpathlineto{\pgfqpoint{4.460940in}{2.248836in}}%
\pgfpathlineto{\pgfqpoint{4.453279in}{2.237609in}}%
\pgfpathlineto{\pgfqpoint{4.445613in}{2.226286in}}%
\pgfpathclose%
\pgfusepath{fill}%
\end{pgfscope}%
\begin{pgfscope}%
\pgfpathrectangle{\pgfqpoint{1.254980in}{0.150000in}}{\pgfqpoint{5.490039in}{5.490039in}}%
\pgfusepath{clip}%
\pgfsetbuttcap%
\pgfsetroundjoin%
\definecolor{currentfill}{rgb}{0.252194,0.269783,0.531579}%
\pgfsetfillcolor{currentfill}%
\pgfsetfillopacity{0.700000}%
\pgfsetlinewidth{0.000000pt}%
\definecolor{currentstroke}{rgb}{0.000000,0.000000,0.000000}%
\pgfsetstrokecolor{currentstroke}%
\pgfsetdash{}{0pt}%
\pgfpathmoveto{\pgfqpoint{4.244373in}{2.030688in}}%
\pgfpathlineto{\pgfqpoint{4.257934in}{2.036688in}}%
\pgfpathlineto{\pgfqpoint{4.271507in}{2.042851in}}%
\pgfpathlineto{\pgfqpoint{4.285092in}{2.049174in}}%
\pgfpathlineto{\pgfqpoint{4.298690in}{2.055659in}}%
\pgfpathlineto{\pgfqpoint{4.306419in}{2.067859in}}%
\pgfpathlineto{\pgfqpoint{4.314144in}{2.079983in}}%
\pgfpathlineto{\pgfqpoint{4.321863in}{2.092031in}}%
\pgfpathlineto{\pgfqpoint{4.329578in}{2.104000in}}%
\pgfpathlineto{\pgfqpoint{4.315982in}{2.097322in}}%
\pgfpathlineto{\pgfqpoint{4.302398in}{2.090807in}}%
\pgfpathlineto{\pgfqpoint{4.288828in}{2.084452in}}%
\pgfpathlineto{\pgfqpoint{4.275269in}{2.078259in}}%
\pgfpathlineto{\pgfqpoint{4.267552in}{2.066472in}}%
\pgfpathlineto{\pgfqpoint{4.259831in}{2.054613in}}%
\pgfpathlineto{\pgfqpoint{4.252105in}{2.042684in}}%
\pgfpathlineto{\pgfqpoint{4.244373in}{2.030688in}}%
\pgfpathclose%
\pgfusepath{fill}%
\end{pgfscope}%
\begin{pgfscope}%
\pgfpathrectangle{\pgfqpoint{1.254980in}{0.150000in}}{\pgfqpoint{5.490039in}{5.490039in}}%
\pgfusepath{clip}%
\pgfsetbuttcap%
\pgfsetroundjoin%
\definecolor{currentfill}{rgb}{0.267004,0.004874,0.329415}%
\pgfsetfillcolor{currentfill}%
\pgfsetfillopacity{0.700000}%
\pgfsetlinewidth{0.000000pt}%
\definecolor{currentstroke}{rgb}{0.000000,0.000000,0.000000}%
\pgfsetstrokecolor{currentstroke}%
\pgfsetdash{}{0pt}%
\pgfpathmoveto{\pgfqpoint{3.447321in}{1.539395in}}%
\pgfpathlineto{\pgfqpoint{3.460661in}{1.536010in}}%
\pgfpathlineto{\pgfqpoint{3.474005in}{1.532797in}}%
\pgfpathlineto{\pgfqpoint{3.487354in}{1.529755in}}%
\pgfpathlineto{\pgfqpoint{3.500707in}{1.526883in}}%
\pgfpathlineto{\pgfqpoint{3.508710in}{1.535613in}}%
\pgfpathlineto{\pgfqpoint{3.516705in}{1.544455in}}%
\pgfpathlineto{\pgfqpoint{3.524694in}{1.553404in}}%
\pgfpathlineto{\pgfqpoint{3.532676in}{1.562454in}}%
\pgfpathlineto{\pgfqpoint{3.519338in}{1.564859in}}%
\pgfpathlineto{\pgfqpoint{3.506006in}{1.567433in}}%
\pgfpathlineto{\pgfqpoint{3.492679in}{1.570178in}}%
\pgfpathlineto{\pgfqpoint{3.479356in}{1.573095in}}%
\pgfpathlineto{\pgfqpoint{3.471359in}{1.564501in}}%
\pgfpathlineto{\pgfqpoint{3.463354in}{1.556017in}}%
\pgfpathlineto{\pgfqpoint{3.455341in}{1.547647in}}%
\pgfpathlineto{\pgfqpoint{3.447321in}{1.539395in}}%
\pgfpathclose%
\pgfusepath{fill}%
\end{pgfscope}%
\begin{pgfscope}%
\pgfpathrectangle{\pgfqpoint{1.254980in}{0.150000in}}{\pgfqpoint{5.490039in}{5.490039in}}%
\pgfusepath{clip}%
\pgfsetbuttcap%
\pgfsetroundjoin%
\definecolor{currentfill}{rgb}{0.120092,0.600104,0.542530}%
\pgfsetfillcolor{currentfill}%
\pgfsetfillopacity{0.700000}%
\pgfsetlinewidth{0.000000pt}%
\definecolor{currentstroke}{rgb}{0.000000,0.000000,0.000000}%
\pgfsetstrokecolor{currentstroke}%
\pgfsetdash{}{0pt}%
\pgfpathmoveto{\pgfqpoint{5.134548in}{2.874875in}}%
\pgfpathlineto{\pgfqpoint{5.148608in}{2.886890in}}%
\pgfpathlineto{\pgfqpoint{5.162686in}{2.899065in}}%
\pgfpathlineto{\pgfqpoint{5.176783in}{2.911401in}}%
\pgfpathlineto{\pgfqpoint{5.190899in}{2.923899in}}%
\pgfpathlineto{\pgfqpoint{5.198245in}{2.930106in}}%
\pgfpathlineto{\pgfqpoint{5.205582in}{2.936189in}}%
\pgfpathlineto{\pgfqpoint{5.212910in}{2.942149in}}%
\pgfpathlineto{\pgfqpoint{5.220230in}{2.947988in}}%
\pgfpathlineto{\pgfqpoint{5.206126in}{2.935687in}}%
\pgfpathlineto{\pgfqpoint{5.192040in}{2.923546in}}%
\pgfpathlineto{\pgfqpoint{5.177972in}{2.911566in}}%
\pgfpathlineto{\pgfqpoint{5.163923in}{2.899747in}}%
\pgfpathlineto{\pgfqpoint{5.156591in}{2.893701in}}%
\pgfpathlineto{\pgfqpoint{5.149252in}{2.887542in}}%
\pgfpathlineto{\pgfqpoint{5.141904in}{2.881267in}}%
\pgfpathlineto{\pgfqpoint{5.134548in}{2.874875in}}%
\pgfpathclose%
\pgfusepath{fill}%
\end{pgfscope}%
\begin{pgfscope}%
\pgfpathrectangle{\pgfqpoint{1.254980in}{0.150000in}}{\pgfqpoint{5.490039in}{5.490039in}}%
\pgfusepath{clip}%
\pgfsetbuttcap%
\pgfsetroundjoin%
\definecolor{currentfill}{rgb}{0.191090,0.708366,0.482284}%
\pgfsetfillcolor{currentfill}%
\pgfsetfillopacity{0.700000}%
\pgfsetlinewidth{0.000000pt}%
\definecolor{currentstroke}{rgb}{0.000000,0.000000,0.000000}%
\pgfsetstrokecolor{currentstroke}%
\pgfsetdash{}{0pt}%
\pgfpathmoveto{\pgfqpoint{5.506219in}{3.174458in}}%
\pgfpathlineto{\pgfqpoint{5.520513in}{3.187756in}}%
\pgfpathlineto{\pgfqpoint{5.534827in}{3.201215in}}%
\pgfpathlineto{\pgfqpoint{5.549162in}{3.214835in}}%
\pgfpathlineto{\pgfqpoint{5.563517in}{3.228615in}}%
\pgfpathlineto{\pgfqpoint{5.570621in}{3.231510in}}%
\pgfpathlineto{\pgfqpoint{5.577716in}{3.234313in}}%
\pgfpathlineto{\pgfqpoint{5.584802in}{3.237027in}}%
\pgfpathlineto{\pgfqpoint{5.591879in}{3.239657in}}%
\pgfpathlineto{\pgfqpoint{5.577544in}{3.226229in}}%
\pgfpathlineto{\pgfqpoint{5.563230in}{3.212962in}}%
\pgfpathlineto{\pgfqpoint{5.548936in}{3.199855in}}%
\pgfpathlineto{\pgfqpoint{5.534662in}{3.186907in}}%
\pgfpathlineto{\pgfqpoint{5.527564in}{3.183915in}}%
\pgfpathlineto{\pgfqpoint{5.520458in}{3.180845in}}%
\pgfpathlineto{\pgfqpoint{5.513343in}{3.177694in}}%
\pgfpathlineto{\pgfqpoint{5.506219in}{3.174458in}}%
\pgfpathclose%
\pgfusepath{fill}%
\end{pgfscope}%
\begin{pgfscope}%
\pgfpathrectangle{\pgfqpoint{1.254980in}{0.150000in}}{\pgfqpoint{5.490039in}{5.490039in}}%
\pgfusepath{clip}%
\pgfsetbuttcap%
\pgfsetroundjoin%
\definecolor{currentfill}{rgb}{0.277134,0.185228,0.489898}%
\pgfsetfillcolor{currentfill}%
\pgfsetfillopacity{0.700000}%
\pgfsetlinewidth{0.000000pt}%
\definecolor{currentstroke}{rgb}{0.000000,0.000000,0.000000}%
\pgfsetstrokecolor{currentstroke}%
\pgfsetdash{}{0pt}%
\pgfpathmoveto{\pgfqpoint{4.043100in}{1.845872in}}%
\pgfpathlineto{\pgfqpoint{4.056579in}{1.849833in}}%
\pgfpathlineto{\pgfqpoint{4.070069in}{1.853956in}}%
\pgfpathlineto{\pgfqpoint{4.083569in}{1.858241in}}%
\pgfpathlineto{\pgfqpoint{4.097080in}{1.862688in}}%
\pgfpathlineto{\pgfqpoint{4.104869in}{1.875064in}}%
\pgfpathlineto{\pgfqpoint{4.112654in}{1.887404in}}%
\pgfpathlineto{\pgfqpoint{4.120435in}{1.899705in}}%
\pgfpathlineto{\pgfqpoint{4.128210in}{1.911965in}}%
\pgfpathlineto{\pgfqpoint{4.114703in}{1.907241in}}%
\pgfpathlineto{\pgfqpoint{4.101206in}{1.902680in}}%
\pgfpathlineto{\pgfqpoint{4.087720in}{1.898280in}}%
\pgfpathlineto{\pgfqpoint{4.074244in}{1.894043in}}%
\pgfpathlineto{\pgfqpoint{4.066465in}{1.882049in}}%
\pgfpathlineto{\pgfqpoint{4.058681in}{1.870021in}}%
\pgfpathlineto{\pgfqpoint{4.050893in}{1.857961in}}%
\pgfpathlineto{\pgfqpoint{4.043100in}{1.845872in}}%
\pgfpathclose%
\pgfusepath{fill}%
\end{pgfscope}%
\begin{pgfscope}%
\pgfpathrectangle{\pgfqpoint{1.254980in}{0.150000in}}{\pgfqpoint{5.490039in}{5.490039in}}%
\pgfusepath{clip}%
\pgfsetbuttcap%
\pgfsetroundjoin%
\definecolor{currentfill}{rgb}{0.262138,0.242286,0.520837}%
\pgfsetfillcolor{currentfill}%
\pgfsetfillopacity{0.700000}%
\pgfsetlinewidth{0.000000pt}%
\definecolor{currentstroke}{rgb}{0.000000,0.000000,0.000000}%
\pgfsetstrokecolor{currentstroke}%
\pgfsetdash{}{0pt}%
\pgfpathmoveto{\pgfqpoint{2.565604in}{2.036888in}}%
\pgfpathlineto{\pgfqpoint{2.579138in}{2.020284in}}%
\pgfpathlineto{\pgfqpoint{2.592665in}{2.003913in}}%
\pgfpathlineto{\pgfqpoint{2.606185in}{1.987774in}}%
\pgfpathlineto{\pgfqpoint{2.619699in}{1.971865in}}%
\pgfpathlineto{\pgfqpoint{2.628331in}{1.969721in}}%
\pgfpathlineto{\pgfqpoint{2.636946in}{1.967877in}}%
\pgfpathlineto{\pgfqpoint{2.645542in}{1.966328in}}%
\pgfpathlineto{\pgfqpoint{2.654120in}{1.965067in}}%
\pgfpathlineto{\pgfqpoint{2.640654in}{1.980370in}}%
\pgfpathlineto{\pgfqpoint{2.627182in}{1.995901in}}%
\pgfpathlineto{\pgfqpoint{2.613703in}{2.011664in}}%
\pgfpathlineto{\pgfqpoint{2.600217in}{2.027658in}}%
\pgfpathlineto{\pgfqpoint{2.591592in}{2.029515in}}%
\pgfpathlineto{\pgfqpoint{2.582948in}{2.031667in}}%
\pgfpathlineto{\pgfqpoint{2.574285in}{2.034123in}}%
\pgfpathlineto{\pgfqpoint{2.565604in}{2.036888in}}%
\pgfpathclose%
\pgfusepath{fill}%
\end{pgfscope}%
\begin{pgfscope}%
\pgfpathrectangle{\pgfqpoint{1.254980in}{0.150000in}}{\pgfqpoint{5.490039in}{5.490039in}}%
\pgfusepath{clip}%
\pgfsetbuttcap%
\pgfsetroundjoin%
\definecolor{currentfill}{rgb}{0.270595,0.214069,0.507052}%
\pgfsetfillcolor{currentfill}%
\pgfsetfillopacity{0.700000}%
\pgfsetlinewidth{0.000000pt}%
\definecolor{currentstroke}{rgb}{0.000000,0.000000,0.000000}%
\pgfsetstrokecolor{currentstroke}%
\pgfsetdash{}{0pt}%
\pgfpathmoveto{\pgfqpoint{2.619699in}{1.971865in}}%
\pgfpathlineto{\pgfqpoint{2.633205in}{1.956184in}}%
\pgfpathlineto{\pgfqpoint{2.646705in}{1.940730in}}%
\pgfpathlineto{\pgfqpoint{2.660199in}{1.925501in}}%
\pgfpathlineto{\pgfqpoint{2.673686in}{1.910495in}}%
\pgfpathlineto{\pgfqpoint{2.682272in}{1.908967in}}%
\pgfpathlineto{\pgfqpoint{2.690840in}{1.907731in}}%
\pgfpathlineto{\pgfqpoint{2.699391in}{1.906782in}}%
\pgfpathlineto{\pgfqpoint{2.707924in}{1.906114in}}%
\pgfpathlineto{\pgfqpoint{2.694482in}{1.920517in}}%
\pgfpathlineto{\pgfqpoint{2.681034in}{1.935142in}}%
\pgfpathlineto{\pgfqpoint{2.667580in}{1.949992in}}%
\pgfpathlineto{\pgfqpoint{2.654120in}{1.965067in}}%
\pgfpathlineto{\pgfqpoint{2.645542in}{1.966328in}}%
\pgfpathlineto{\pgfqpoint{2.636946in}{1.967877in}}%
\pgfpathlineto{\pgfqpoint{2.628331in}{1.969721in}}%
\pgfpathlineto{\pgfqpoint{2.619699in}{1.971865in}}%
\pgfpathclose%
\pgfusepath{fill}%
\end{pgfscope}%
\begin{pgfscope}%
\pgfpathrectangle{\pgfqpoint{1.254980in}{0.150000in}}{\pgfqpoint{5.490039in}{5.490039in}}%
\pgfusepath{clip}%
\pgfsetbuttcap%
\pgfsetroundjoin%
\definecolor{currentfill}{rgb}{0.276022,0.044167,0.370164}%
\pgfsetfillcolor{currentfill}%
\pgfsetfillopacity{0.700000}%
\pgfsetlinewidth{0.000000pt}%
\definecolor{currentstroke}{rgb}{0.000000,0.000000,0.000000}%
\pgfsetstrokecolor{currentstroke}%
\pgfsetdash{}{0pt}%
\pgfpathmoveto{\pgfqpoint{3.029408in}{1.623504in}}%
\pgfpathlineto{\pgfqpoint{3.042776in}{1.614221in}}%
\pgfpathlineto{\pgfqpoint{3.056144in}{1.605128in}}%
\pgfpathlineto{\pgfqpoint{3.069512in}{1.596224in}}%
\pgfpathlineto{\pgfqpoint{3.082880in}{1.587508in}}%
\pgfpathlineto{\pgfqpoint{3.091137in}{1.591133in}}%
\pgfpathlineto{\pgfqpoint{3.099382in}{1.594973in}}%
\pgfpathlineto{\pgfqpoint{3.107615in}{1.599023in}}%
\pgfpathlineto{\pgfqpoint{3.115837in}{1.603276in}}%
\pgfpathlineto{\pgfqpoint{3.102500in}{1.611436in}}%
\pgfpathlineto{\pgfqpoint{3.089163in}{1.619783in}}%
\pgfpathlineto{\pgfqpoint{3.075827in}{1.628319in}}%
\pgfpathlineto{\pgfqpoint{3.062490in}{1.637044in}}%
\pgfpathlineto{\pgfqpoint{3.054238in}{1.633336in}}%
\pgfpathlineto{\pgfqpoint{3.045973in}{1.629840in}}%
\pgfpathlineto{\pgfqpoint{3.037697in}{1.626561in}}%
\pgfpathlineto{\pgfqpoint{3.029408in}{1.623504in}}%
\pgfpathclose%
\pgfusepath{fill}%
\end{pgfscope}%
\begin{pgfscope}%
\pgfpathrectangle{\pgfqpoint{1.254980in}{0.150000in}}{\pgfqpoint{5.490039in}{5.490039in}}%
\pgfusepath{clip}%
\pgfsetbuttcap%
\pgfsetroundjoin%
\definecolor{currentfill}{rgb}{0.250425,0.274290,0.533103}%
\pgfsetfillcolor{currentfill}%
\pgfsetfillopacity{0.700000}%
\pgfsetlinewidth{0.000000pt}%
\definecolor{currentstroke}{rgb}{0.000000,0.000000,0.000000}%
\pgfsetstrokecolor{currentstroke}%
\pgfsetdash{}{0pt}%
\pgfpathmoveto{\pgfqpoint{2.511387in}{2.105674in}}%
\pgfpathlineto{\pgfqpoint{2.524953in}{2.088118in}}%
\pgfpathlineto{\pgfqpoint{2.538511in}{2.070803in}}%
\pgfpathlineto{\pgfqpoint{2.552061in}{2.053727in}}%
\pgfpathlineto{\pgfqpoint{2.565604in}{2.036888in}}%
\pgfpathlineto{\pgfqpoint{2.574285in}{2.034123in}}%
\pgfpathlineto{\pgfqpoint{2.582948in}{2.031667in}}%
\pgfpathlineto{\pgfqpoint{2.591592in}{2.029515in}}%
\pgfpathlineto{\pgfqpoint{2.600217in}{2.027658in}}%
\pgfpathlineto{\pgfqpoint{2.586724in}{2.043887in}}%
\pgfpathlineto{\pgfqpoint{2.573223in}{2.060352in}}%
\pgfpathlineto{\pgfqpoint{2.559715in}{2.077055in}}%
\pgfpathlineto{\pgfqpoint{2.546199in}{2.093997in}}%
\pgfpathlineto{\pgfqpoint{2.537525in}{2.096453in}}%
\pgfpathlineto{\pgfqpoint{2.528832in}{2.099214in}}%
\pgfpathlineto{\pgfqpoint{2.520120in}{2.102285in}}%
\pgfpathlineto{\pgfqpoint{2.511387in}{2.105674in}}%
\pgfpathclose%
\pgfusepath{fill}%
\end{pgfscope}%
\begin{pgfscope}%
\pgfpathrectangle{\pgfqpoint{1.254980in}{0.150000in}}{\pgfqpoint{5.490039in}{5.490039in}}%
\pgfusepath{clip}%
\pgfsetbuttcap%
\pgfsetroundjoin%
\definecolor{currentfill}{rgb}{0.277134,0.185228,0.489898}%
\pgfsetfillcolor{currentfill}%
\pgfsetfillopacity{0.700000}%
\pgfsetlinewidth{0.000000pt}%
\definecolor{currentstroke}{rgb}{0.000000,0.000000,0.000000}%
\pgfsetstrokecolor{currentstroke}%
\pgfsetdash{}{0pt}%
\pgfpathmoveto{\pgfqpoint{2.673686in}{1.910495in}}%
\pgfpathlineto{\pgfqpoint{2.687168in}{1.895710in}}%
\pgfpathlineto{\pgfqpoint{2.700645in}{1.881146in}}%
\pgfpathlineto{\pgfqpoint{2.714116in}{1.866801in}}%
\pgfpathlineto{\pgfqpoint{2.727581in}{1.852672in}}%
\pgfpathlineto{\pgfqpoint{2.736122in}{1.851757in}}%
\pgfpathlineto{\pgfqpoint{2.744645in}{1.851126in}}%
\pgfpathlineto{\pgfqpoint{2.753152in}{1.850774in}}%
\pgfpathlineto{\pgfqpoint{2.761643in}{1.850695in}}%
\pgfpathlineto{\pgfqpoint{2.748220in}{1.864224in}}%
\pgfpathlineto{\pgfqpoint{2.734793in}{1.877969in}}%
\pgfpathlineto{\pgfqpoint{2.721361in}{1.891931in}}%
\pgfpathlineto{\pgfqpoint{2.707924in}{1.906114in}}%
\pgfpathlineto{\pgfqpoint{2.699391in}{1.906782in}}%
\pgfpathlineto{\pgfqpoint{2.690840in}{1.907731in}}%
\pgfpathlineto{\pgfqpoint{2.682272in}{1.908967in}}%
\pgfpathlineto{\pgfqpoint{2.673686in}{1.910495in}}%
\pgfpathclose%
\pgfusepath{fill}%
\end{pgfscope}%
\begin{pgfscope}%
\pgfpathrectangle{\pgfqpoint{1.254980in}{0.150000in}}{\pgfqpoint{5.490039in}{5.490039in}}%
\pgfusepath{clip}%
\pgfsetbuttcap%
\pgfsetroundjoin%
\definecolor{currentfill}{rgb}{0.154815,0.493313,0.557840}%
\pgfsetfillcolor{currentfill}%
\pgfsetfillopacity{0.700000}%
\pgfsetlinewidth{0.000000pt}%
\definecolor{currentstroke}{rgb}{0.000000,0.000000,0.000000}%
\pgfsetstrokecolor{currentstroke}%
\pgfsetdash{}{0pt}%
\pgfpathmoveto{\pgfqpoint{2.163501in}{2.674476in}}%
\pgfpathlineto{\pgfqpoint{2.177346in}{2.649907in}}%
\pgfpathlineto{\pgfqpoint{2.191176in}{2.625647in}}%
\pgfpathlineto{\pgfqpoint{2.204990in}{2.601691in}}%
\pgfpathlineto{\pgfqpoint{2.218789in}{2.578038in}}%
\pgfpathlineto{\pgfqpoint{2.227773in}{2.572236in}}%
\pgfpathlineto{\pgfqpoint{2.236734in}{2.566778in}}%
\pgfpathlineto{\pgfqpoint{2.245671in}{2.561656in}}%
\pgfpathlineto{\pgfqpoint{2.254585in}{2.556866in}}%
\pgfpathlineto{\pgfqpoint{2.240847in}{2.579908in}}%
\pgfpathlineto{\pgfqpoint{2.227093in}{2.603250in}}%
\pgfpathlineto{\pgfqpoint{2.213325in}{2.626895in}}%
\pgfpathlineto{\pgfqpoint{2.199541in}{2.650847in}}%
\pgfpathlineto{\pgfqpoint{2.190567in}{2.656238in}}%
\pgfpathlineto{\pgfqpoint{2.181569in}{2.661969in}}%
\pgfpathlineto{\pgfqpoint{2.172547in}{2.668046in}}%
\pgfpathlineto{\pgfqpoint{2.163501in}{2.674476in}}%
\pgfpathclose%
\pgfusepath{fill}%
\end{pgfscope}%
\begin{pgfscope}%
\pgfpathrectangle{\pgfqpoint{1.254980in}{0.150000in}}{\pgfqpoint{5.490039in}{5.490039in}}%
\pgfusepath{clip}%
\pgfsetbuttcap%
\pgfsetroundjoin%
\definecolor{currentfill}{rgb}{0.237441,0.305202,0.541921}%
\pgfsetfillcolor{currentfill}%
\pgfsetfillopacity{0.700000}%
\pgfsetlinewidth{0.000000pt}%
\definecolor{currentstroke}{rgb}{0.000000,0.000000,0.000000}%
\pgfsetstrokecolor{currentstroke}%
\pgfsetdash{}{0pt}%
\pgfpathmoveto{\pgfqpoint{2.457033in}{2.178344in}}%
\pgfpathlineto{\pgfqpoint{2.470635in}{2.159805in}}%
\pgfpathlineto{\pgfqpoint{2.484228in}{2.141515in}}%
\pgfpathlineto{\pgfqpoint{2.497812in}{2.123472in}}%
\pgfpathlineto{\pgfqpoint{2.511387in}{2.105674in}}%
\pgfpathlineto{\pgfqpoint{2.520120in}{2.102285in}}%
\pgfpathlineto{\pgfqpoint{2.528832in}{2.099214in}}%
\pgfpathlineto{\pgfqpoint{2.537525in}{2.096453in}}%
\pgfpathlineto{\pgfqpoint{2.546199in}{2.093997in}}%
\pgfpathlineto{\pgfqpoint{2.532675in}{2.111182in}}%
\pgfpathlineto{\pgfqpoint{2.519143in}{2.128609in}}%
\pgfpathlineto{\pgfqpoint{2.505602in}{2.146282in}}%
\pgfpathlineto{\pgfqpoint{2.492052in}{2.164203in}}%
\pgfpathlineto{\pgfqpoint{2.483328in}{2.167263in}}%
\pgfpathlineto{\pgfqpoint{2.474584in}{2.170635in}}%
\pgfpathlineto{\pgfqpoint{2.465819in}{2.174326in}}%
\pgfpathlineto{\pgfqpoint{2.457033in}{2.178344in}}%
\pgfpathclose%
\pgfusepath{fill}%
\end{pgfscope}%
\begin{pgfscope}%
\pgfpathrectangle{\pgfqpoint{1.254980in}{0.150000in}}{\pgfqpoint{5.490039in}{5.490039in}}%
\pgfusepath{clip}%
\pgfsetbuttcap%
\pgfsetroundjoin%
\definecolor{currentfill}{rgb}{0.280868,0.160771,0.472899}%
\pgfsetfillcolor{currentfill}%
\pgfsetfillopacity{0.700000}%
\pgfsetlinewidth{0.000000pt}%
\definecolor{currentstroke}{rgb}{0.000000,0.000000,0.000000}%
\pgfsetstrokecolor{currentstroke}%
\pgfsetdash{}{0pt}%
\pgfpathmoveto{\pgfqpoint{2.727581in}{1.852672in}}%
\pgfpathlineto{\pgfqpoint{2.741042in}{1.838759in}}%
\pgfpathlineto{\pgfqpoint{2.754498in}{1.825061in}}%
\pgfpathlineto{\pgfqpoint{2.767950in}{1.811575in}}%
\pgfpathlineto{\pgfqpoint{2.781397in}{1.798301in}}%
\pgfpathlineto{\pgfqpoint{2.789894in}{1.797995in}}%
\pgfpathlineto{\pgfqpoint{2.798375in}{1.797967in}}%
\pgfpathlineto{\pgfqpoint{2.806840in}{1.798209in}}%
\pgfpathlineto{\pgfqpoint{2.815289in}{1.798716in}}%
\pgfpathlineto{\pgfqpoint{2.801884in}{1.811393in}}%
\pgfpathlineto{\pgfqpoint{2.788474in}{1.824281in}}%
\pgfpathlineto{\pgfqpoint{2.775061in}{1.837381in}}%
\pgfpathlineto{\pgfqpoint{2.761643in}{1.850695in}}%
\pgfpathlineto{\pgfqpoint{2.753152in}{1.850774in}}%
\pgfpathlineto{\pgfqpoint{2.744645in}{1.851126in}}%
\pgfpathlineto{\pgfqpoint{2.736122in}{1.851757in}}%
\pgfpathlineto{\pgfqpoint{2.727581in}{1.852672in}}%
\pgfpathclose%
\pgfusepath{fill}%
\end{pgfscope}%
\begin{pgfscope}%
\pgfpathrectangle{\pgfqpoint{1.254980in}{0.150000in}}{\pgfqpoint{5.490039in}{5.490039in}}%
\pgfusepath{clip}%
\pgfsetbuttcap%
\pgfsetroundjoin%
\definecolor{currentfill}{rgb}{0.137770,0.537492,0.554906}%
\pgfsetfillcolor{currentfill}%
\pgfsetfillopacity{0.700000}%
\pgfsetlinewidth{0.000000pt}%
\definecolor{currentstroke}{rgb}{0.000000,0.000000,0.000000}%
\pgfsetstrokecolor{currentstroke}%
\pgfsetdash{}{0pt}%
\pgfpathmoveto{\pgfqpoint{4.933590in}{2.694713in}}%
\pgfpathlineto{\pgfqpoint{4.947534in}{2.705839in}}%
\pgfpathlineto{\pgfqpoint{4.961495in}{2.717127in}}%
\pgfpathlineto{\pgfqpoint{4.975474in}{2.728577in}}%
\pgfpathlineto{\pgfqpoint{4.989469in}{2.740188in}}%
\pgfpathlineto{\pgfqpoint{4.996931in}{2.748240in}}%
\pgfpathlineto{\pgfqpoint{5.004384in}{2.756160in}}%
\pgfpathlineto{\pgfqpoint{5.011830in}{2.763948in}}%
\pgfpathlineto{\pgfqpoint{5.019268in}{2.771608in}}%
\pgfpathlineto{\pgfqpoint{5.005279in}{2.760101in}}%
\pgfpathlineto{\pgfqpoint{4.991308in}{2.748755in}}%
\pgfpathlineto{\pgfqpoint{4.977354in}{2.737570in}}%
\pgfpathlineto{\pgfqpoint{4.963417in}{2.726546in}}%
\pgfpathlineto{\pgfqpoint{4.955972in}{2.718772in}}%
\pgfpathlineto{\pgfqpoint{4.948519in}{2.710876in}}%
\pgfpathlineto{\pgfqpoint{4.941058in}{2.702857in}}%
\pgfpathlineto{\pgfqpoint{4.933590in}{2.694713in}}%
\pgfpathclose%
\pgfusepath{fill}%
\end{pgfscope}%
\begin{pgfscope}%
\pgfpathrectangle{\pgfqpoint{1.254980in}{0.150000in}}{\pgfqpoint{5.490039in}{5.490039in}}%
\pgfusepath{clip}%
\pgfsetbuttcap%
\pgfsetroundjoin%
\definecolor{currentfill}{rgb}{0.268510,0.009605,0.335427}%
\pgfsetfillcolor{currentfill}%
\pgfsetfillopacity{0.700000}%
\pgfsetlinewidth{0.000000pt}%
\definecolor{currentstroke}{rgb}{0.000000,0.000000,0.000000}%
\pgfsetstrokecolor{currentstroke}%
\pgfsetdash{}{0pt}%
\pgfpathmoveto{\pgfqpoint{3.222561in}{1.544649in}}%
\pgfpathlineto{\pgfqpoint{3.235907in}{1.538139in}}%
\pgfpathlineto{\pgfqpoint{3.249256in}{1.531808in}}%
\pgfpathlineto{\pgfqpoint{3.262606in}{1.525656in}}%
\pgfpathlineto{\pgfqpoint{3.275958in}{1.519681in}}%
\pgfpathlineto{\pgfqpoint{3.284089in}{1.525744in}}%
\pgfpathlineto{\pgfqpoint{3.292210in}{1.531979in}}%
\pgfpathlineto{\pgfqpoint{3.300322in}{1.538380in}}%
\pgfpathlineto{\pgfqpoint{3.308424in}{1.544942in}}%
\pgfpathlineto{\pgfqpoint{3.295096in}{1.550392in}}%
\pgfpathlineto{\pgfqpoint{3.281769in}{1.556019in}}%
\pgfpathlineto{\pgfqpoint{3.268446in}{1.561824in}}%
\pgfpathlineto{\pgfqpoint{3.255124in}{1.567809in}}%
\pgfpathlineto{\pgfqpoint{3.246998in}{1.561761in}}%
\pgfpathlineto{\pgfqpoint{3.238862in}{1.555881in}}%
\pgfpathlineto{\pgfqpoint{3.230717in}{1.550175in}}%
\pgfpathlineto{\pgfqpoint{3.222561in}{1.544649in}}%
\pgfpathclose%
\pgfusepath{fill}%
\end{pgfscope}%
\begin{pgfscope}%
\pgfpathrectangle{\pgfqpoint{1.254980in}{0.150000in}}{\pgfqpoint{5.490039in}{5.490039in}}%
\pgfusepath{clip}%
\pgfsetbuttcap%
\pgfsetroundjoin%
\definecolor{currentfill}{rgb}{0.232815,0.732247,0.459277}%
\pgfsetfillcolor{currentfill}%
\pgfsetfillopacity{0.700000}%
\pgfsetlinewidth{0.000000pt}%
\definecolor{currentstroke}{rgb}{0.000000,0.000000,0.000000}%
\pgfsetstrokecolor{currentstroke}%
\pgfsetdash{}{0pt}%
\pgfpathmoveto{\pgfqpoint{5.591879in}{3.239657in}}%
\pgfpathlineto{\pgfqpoint{5.606234in}{3.253245in}}%
\pgfpathlineto{\pgfqpoint{5.620609in}{3.266994in}}%
\pgfpathlineto{\pgfqpoint{5.635005in}{3.280904in}}%
\pgfpathlineto{\pgfqpoint{5.649422in}{3.294975in}}%
\pgfpathlineto{\pgfqpoint{5.656468in}{3.297151in}}%
\pgfpathlineto{\pgfqpoint{5.663505in}{3.299244in}}%
\pgfpathlineto{\pgfqpoint{5.670532in}{3.301258in}}%
\pgfpathlineto{\pgfqpoint{5.677550in}{3.303197in}}%
\pgfpathlineto{\pgfqpoint{5.663156in}{3.289511in}}%
\pgfpathlineto{\pgfqpoint{5.648783in}{3.275986in}}%
\pgfpathlineto{\pgfqpoint{5.634430in}{3.262620in}}%
\pgfpathlineto{\pgfqpoint{5.620097in}{3.249415in}}%
\pgfpathlineto{\pgfqpoint{5.613055in}{3.247081in}}%
\pgfpathlineto{\pgfqpoint{5.606005in}{3.244680in}}%
\pgfpathlineto{\pgfqpoint{5.598946in}{3.242206in}}%
\pgfpathlineto{\pgfqpoint{5.591879in}{3.239657in}}%
\pgfpathclose%
\pgfusepath{fill}%
\end{pgfscope}%
\begin{pgfscope}%
\pgfpathrectangle{\pgfqpoint{1.254980in}{0.150000in}}{\pgfqpoint{5.490039in}{5.490039in}}%
\pgfusepath{clip}%
\pgfsetbuttcap%
\pgfsetroundjoin%
\definecolor{currentfill}{rgb}{0.267004,0.004874,0.329415}%
\pgfsetfillcolor{currentfill}%
\pgfsetfillopacity{0.700000}%
\pgfsetlinewidth{0.000000pt}%
\definecolor{currentstroke}{rgb}{0.000000,0.000000,0.000000}%
\pgfsetstrokecolor{currentstroke}%
\pgfsetdash{}{0pt}%
\pgfpathmoveto{\pgfqpoint{3.361767in}{1.524908in}}%
\pgfpathlineto{\pgfqpoint{3.375111in}{1.520336in}}%
\pgfpathlineto{\pgfqpoint{3.388458in}{1.515939in}}%
\pgfpathlineto{\pgfqpoint{3.401809in}{1.511714in}}%
\pgfpathlineto{\pgfqpoint{3.415163in}{1.507662in}}%
\pgfpathlineto{\pgfqpoint{3.423215in}{1.515394in}}%
\pgfpathlineto{\pgfqpoint{3.431258in}{1.523264in}}%
\pgfpathlineto{\pgfqpoint{3.439294in}{1.531265in}}%
\pgfpathlineto{\pgfqpoint{3.447321in}{1.539395in}}%
\pgfpathlineto{\pgfqpoint{3.433986in}{1.542951in}}%
\pgfpathlineto{\pgfqpoint{3.420654in}{1.546679in}}%
\pgfpathlineto{\pgfqpoint{3.407327in}{1.550581in}}%
\pgfpathlineto{\pgfqpoint{3.394004in}{1.554656in}}%
\pgfpathlineto{\pgfqpoint{3.385957in}{1.547012in}}%
\pgfpathlineto{\pgfqpoint{3.377902in}{1.539503in}}%
\pgfpathlineto{\pgfqpoint{3.369839in}{1.532133in}}%
\pgfpathlineto{\pgfqpoint{3.361767in}{1.524908in}}%
\pgfpathclose%
\pgfusepath{fill}%
\end{pgfscope}%
\begin{pgfscope}%
\pgfpathrectangle{\pgfqpoint{1.254980in}{0.150000in}}{\pgfqpoint{5.490039in}{5.490039in}}%
\pgfusepath{clip}%
\pgfsetbuttcap%
\pgfsetroundjoin%
\definecolor{currentfill}{rgb}{0.235526,0.309527,0.542944}%
\pgfsetfillcolor{currentfill}%
\pgfsetfillopacity{0.700000}%
\pgfsetlinewidth{0.000000pt}%
\definecolor{currentstroke}{rgb}{0.000000,0.000000,0.000000}%
\pgfsetstrokecolor{currentstroke}%
\pgfsetdash{}{0pt}%
\pgfpathmoveto{\pgfqpoint{4.329578in}{2.104000in}}%
\pgfpathlineto{\pgfqpoint{4.343187in}{2.110838in}}%
\pgfpathlineto{\pgfqpoint{4.356809in}{2.117838in}}%
\pgfpathlineto{\pgfqpoint{4.370444in}{2.124999in}}%
\pgfpathlineto{\pgfqpoint{4.384092in}{2.132321in}}%
\pgfpathlineto{\pgfqpoint{4.391800in}{2.144385in}}%
\pgfpathlineto{\pgfqpoint{4.399503in}{2.156361in}}%
\pgfpathlineto{\pgfqpoint{4.407201in}{2.168246in}}%
\pgfpathlineto{\pgfqpoint{4.414894in}{2.180040in}}%
\pgfpathlineto{\pgfqpoint{4.401247in}{2.172554in}}%
\pgfpathlineto{\pgfqpoint{4.387614in}{2.165229in}}%
\pgfpathlineto{\pgfqpoint{4.373993in}{2.158066in}}%
\pgfpathlineto{\pgfqpoint{4.360386in}{2.151063in}}%
\pgfpathlineto{\pgfqpoint{4.352692in}{2.139422in}}%
\pgfpathlineto{\pgfqpoint{4.344992in}{2.127697in}}%
\pgfpathlineto{\pgfqpoint{4.337288in}{2.115889in}}%
\pgfpathlineto{\pgfqpoint{4.329578in}{2.104000in}}%
\pgfpathclose%
\pgfusepath{fill}%
\end{pgfscope}%
\begin{pgfscope}%
\pgfpathrectangle{\pgfqpoint{1.254980in}{0.150000in}}{\pgfqpoint{5.490039in}{5.490039in}}%
\pgfusepath{clip}%
\pgfsetbuttcap%
\pgfsetroundjoin%
\definecolor{currentfill}{rgb}{0.121380,0.629492,0.531973}%
\pgfsetfillcolor{currentfill}%
\pgfsetfillopacity{0.700000}%
\pgfsetlinewidth{0.000000pt}%
\definecolor{currentstroke}{rgb}{0.000000,0.000000,0.000000}%
\pgfsetstrokecolor{currentstroke}%
\pgfsetdash{}{0pt}%
\pgfpathmoveto{\pgfqpoint{5.220230in}{2.947988in}}%
\pgfpathlineto{\pgfqpoint{5.234353in}{2.960451in}}%
\pgfpathlineto{\pgfqpoint{5.248495in}{2.973075in}}%
\pgfpathlineto{\pgfqpoint{5.262656in}{2.985860in}}%
\pgfpathlineto{\pgfqpoint{5.276836in}{2.998807in}}%
\pgfpathlineto{\pgfqpoint{5.284135in}{3.004313in}}%
\pgfpathlineto{\pgfqpoint{5.291425in}{3.009696in}}%
\pgfpathlineto{\pgfqpoint{5.298706in}{3.014960in}}%
\pgfpathlineto{\pgfqpoint{5.305979in}{3.020106in}}%
\pgfpathlineto{\pgfqpoint{5.291812in}{3.007388in}}%
\pgfpathlineto{\pgfqpoint{5.277664in}{2.994830in}}%
\pgfpathlineto{\pgfqpoint{5.263535in}{2.982433in}}%
\pgfpathlineto{\pgfqpoint{5.249424in}{2.970197in}}%
\pgfpathlineto{\pgfqpoint{5.242138in}{2.964812in}}%
\pgfpathlineto{\pgfqpoint{5.234844in}{2.959317in}}%
\pgfpathlineto{\pgfqpoint{5.227541in}{2.953710in}}%
\pgfpathlineto{\pgfqpoint{5.220230in}{2.947988in}}%
\pgfpathclose%
\pgfusepath{fill}%
\end{pgfscope}%
\begin{pgfscope}%
\pgfpathrectangle{\pgfqpoint{1.254980in}{0.150000in}}{\pgfqpoint{5.490039in}{5.490039in}}%
\pgfusepath{clip}%
\pgfsetbuttcap%
\pgfsetroundjoin%
\definecolor{currentfill}{rgb}{0.197636,0.391528,0.554969}%
\pgfsetfillcolor{currentfill}%
\pgfsetfillopacity{0.700000}%
\pgfsetlinewidth{0.000000pt}%
\definecolor{currentstroke}{rgb}{0.000000,0.000000,0.000000}%
\pgfsetstrokecolor{currentstroke}%
\pgfsetdash{}{0pt}%
\pgfpathmoveto{\pgfqpoint{4.530956in}{2.303516in}}%
\pgfpathlineto{\pgfqpoint{4.544670in}{2.312050in}}%
\pgfpathlineto{\pgfqpoint{4.558397in}{2.320744in}}%
\pgfpathlineto{\pgfqpoint{4.572140in}{2.329600in}}%
\pgfpathlineto{\pgfqpoint{4.585897in}{2.338617in}}%
\pgfpathlineto{\pgfqpoint{4.593537in}{2.349724in}}%
\pgfpathlineto{\pgfqpoint{4.601172in}{2.360718in}}%
\pgfpathlineto{\pgfqpoint{4.608800in}{2.371598in}}%
\pgfpathlineto{\pgfqpoint{4.616423in}{2.382362in}}%
\pgfpathlineto{\pgfqpoint{4.602668in}{2.373269in}}%
\pgfpathlineto{\pgfqpoint{4.588927in}{2.364336in}}%
\pgfpathlineto{\pgfqpoint{4.575202in}{2.355565in}}%
\pgfpathlineto{\pgfqpoint{4.561491in}{2.346955in}}%
\pgfpathlineto{\pgfqpoint{4.553866in}{2.336256in}}%
\pgfpathlineto{\pgfqpoint{4.546235in}{2.325449in}}%
\pgfpathlineto{\pgfqpoint{4.538599in}{2.314536in}}%
\pgfpathlineto{\pgfqpoint{4.530956in}{2.303516in}}%
\pgfpathclose%
\pgfusepath{fill}%
\end{pgfscope}%
\begin{pgfscope}%
\pgfpathrectangle{\pgfqpoint{1.254980in}{0.150000in}}{\pgfqpoint{5.490039in}{5.490039in}}%
\pgfusepath{clip}%
\pgfsetbuttcap%
\pgfsetroundjoin%
\definecolor{currentfill}{rgb}{0.165117,0.467423,0.558141}%
\pgfsetfillcolor{currentfill}%
\pgfsetfillopacity{0.700000}%
\pgfsetlinewidth{0.000000pt}%
\definecolor{currentstroke}{rgb}{0.000000,0.000000,0.000000}%
\pgfsetstrokecolor{currentstroke}%
\pgfsetdash{}{0pt}%
\pgfpathmoveto{\pgfqpoint{4.732332in}{2.502586in}}%
\pgfpathlineto{\pgfqpoint{4.746159in}{2.512551in}}%
\pgfpathlineto{\pgfqpoint{4.760001in}{2.522677in}}%
\pgfpathlineto{\pgfqpoint{4.773860in}{2.532964in}}%
\pgfpathlineto{\pgfqpoint{4.787734in}{2.543412in}}%
\pgfpathlineto{\pgfqpoint{4.795294in}{2.553141in}}%
\pgfpathlineto{\pgfqpoint{4.802846in}{2.562741in}}%
\pgfpathlineto{\pgfqpoint{4.810392in}{2.572213in}}%
\pgfpathlineto{\pgfqpoint{4.817930in}{2.581557in}}%
\pgfpathlineto{\pgfqpoint{4.804059in}{2.571121in}}%
\pgfpathlineto{\pgfqpoint{4.790205in}{2.560847in}}%
\pgfpathlineto{\pgfqpoint{4.776366in}{2.550733in}}%
\pgfpathlineto{\pgfqpoint{4.762543in}{2.540781in}}%
\pgfpathlineto{\pgfqpoint{4.755001in}{2.531413in}}%
\pgfpathlineto{\pgfqpoint{4.747451in}{2.521925in}}%
\pgfpathlineto{\pgfqpoint{4.739895in}{2.512316in}}%
\pgfpathlineto{\pgfqpoint{4.732332in}{2.502586in}}%
\pgfpathclose%
\pgfusepath{fill}%
\end{pgfscope}%
\begin{pgfscope}%
\pgfpathrectangle{\pgfqpoint{1.254980in}{0.150000in}}{\pgfqpoint{5.490039in}{5.490039in}}%
\pgfusepath{clip}%
\pgfsetbuttcap%
\pgfsetroundjoin%
\definecolor{currentfill}{rgb}{0.223925,0.334994,0.548053}%
\pgfsetfillcolor{currentfill}%
\pgfsetfillopacity{0.700000}%
\pgfsetlinewidth{0.000000pt}%
\definecolor{currentstroke}{rgb}{0.000000,0.000000,0.000000}%
\pgfsetstrokecolor{currentstroke}%
\pgfsetdash{}{0pt}%
\pgfpathmoveto{\pgfqpoint{2.402525in}{2.255026in}}%
\pgfpathlineto{\pgfqpoint{2.416167in}{2.235472in}}%
\pgfpathlineto{\pgfqpoint{2.429799in}{2.216175in}}%
\pgfpathlineto{\pgfqpoint{2.443421in}{2.197133in}}%
\pgfpathlineto{\pgfqpoint{2.457033in}{2.178344in}}%
\pgfpathlineto{\pgfqpoint{2.465819in}{2.174326in}}%
\pgfpathlineto{\pgfqpoint{2.474584in}{2.170635in}}%
\pgfpathlineto{\pgfqpoint{2.483328in}{2.167263in}}%
\pgfpathlineto{\pgfqpoint{2.492052in}{2.164203in}}%
\pgfpathlineto{\pgfqpoint{2.478493in}{2.182374in}}%
\pgfpathlineto{\pgfqpoint{2.464925in}{2.200796in}}%
\pgfpathlineto{\pgfqpoint{2.451348in}{2.219471in}}%
\pgfpathlineto{\pgfqpoint{2.437760in}{2.238403in}}%
\pgfpathlineto{\pgfqpoint{2.428983in}{2.242070in}}%
\pgfpathlineto{\pgfqpoint{2.420185in}{2.246059in}}%
\pgfpathlineto{\pgfqpoint{2.411366in}{2.250376in}}%
\pgfpathlineto{\pgfqpoint{2.402525in}{2.255026in}}%
\pgfpathclose%
\pgfusepath{fill}%
\end{pgfscope}%
\begin{pgfscope}%
\pgfpathrectangle{\pgfqpoint{1.254980in}{0.150000in}}{\pgfqpoint{5.490039in}{5.490039in}}%
\pgfusepath{clip}%
\pgfsetbuttcap%
\pgfsetroundjoin%
\definecolor{currentfill}{rgb}{0.269308,0.218818,0.509577}%
\pgfsetfillcolor{currentfill}%
\pgfsetfillopacity{0.700000}%
\pgfsetlinewidth{0.000000pt}%
\definecolor{currentstroke}{rgb}{0.000000,0.000000,0.000000}%
\pgfsetstrokecolor{currentstroke}%
\pgfsetdash{}{0pt}%
\pgfpathmoveto{\pgfqpoint{4.128210in}{1.911965in}}%
\pgfpathlineto{\pgfqpoint{4.141729in}{1.916850in}}%
\pgfpathlineto{\pgfqpoint{4.155260in}{1.921897in}}%
\pgfpathlineto{\pgfqpoint{4.168801in}{1.927105in}}%
\pgfpathlineto{\pgfqpoint{4.182354in}{1.932475in}}%
\pgfpathlineto{\pgfqpoint{4.190123in}{1.944951in}}%
\pgfpathlineto{\pgfqpoint{4.197887in}{1.957374in}}%
\pgfpathlineto{\pgfqpoint{4.205646in}{1.969742in}}%
\pgfpathlineto{\pgfqpoint{4.213401in}{1.982053in}}%
\pgfpathlineto{\pgfqpoint{4.199850in}{1.976435in}}%
\pgfpathlineto{\pgfqpoint{4.186311in}{1.970977in}}%
\pgfpathlineto{\pgfqpoint{4.172784in}{1.965682in}}%
\pgfpathlineto{\pgfqpoint{4.159267in}{1.960548in}}%
\pgfpathlineto{\pgfqpoint{4.151510in}{1.948475in}}%
\pgfpathlineto{\pgfqpoint{4.143748in}{1.936352in}}%
\pgfpathlineto{\pgfqpoint{4.135982in}{1.924181in}}%
\pgfpathlineto{\pgfqpoint{4.128210in}{1.911965in}}%
\pgfpathclose%
\pgfusepath{fill}%
\end{pgfscope}%
\begin{pgfscope}%
\pgfpathrectangle{\pgfqpoint{1.254980in}{0.150000in}}{\pgfqpoint{5.490039in}{5.490039in}}%
\pgfusepath{clip}%
\pgfsetbuttcap%
\pgfsetroundjoin%
\definecolor{currentfill}{rgb}{0.277018,0.050344,0.375715}%
\pgfsetfillcolor{currentfill}%
\pgfsetfillopacity{0.700000}%
\pgfsetlinewidth{0.000000pt}%
\definecolor{currentstroke}{rgb}{0.000000,0.000000,0.000000}%
\pgfsetstrokecolor{currentstroke}%
\pgfsetdash{}{0pt}%
\pgfpathmoveto{\pgfqpoint{3.671325in}{1.589952in}}%
\pgfpathlineto{\pgfqpoint{3.684702in}{1.589500in}}%
\pgfpathlineto{\pgfqpoint{3.698087in}{1.589212in}}%
\pgfpathlineto{\pgfqpoint{3.711478in}{1.589091in}}%
\pgfpathlineto{\pgfqpoint{3.724876in}{1.589134in}}%
\pgfpathlineto{\pgfqpoint{3.732788in}{1.599902in}}%
\pgfpathlineto{\pgfqpoint{3.740695in}{1.610725in}}%
\pgfpathlineto{\pgfqpoint{3.748596in}{1.621601in}}%
\pgfpathlineto{\pgfqpoint{3.756491in}{1.632526in}}%
\pgfpathlineto{\pgfqpoint{3.743103in}{1.632069in}}%
\pgfpathlineto{\pgfqpoint{3.729722in}{1.631777in}}%
\pgfpathlineto{\pgfqpoint{3.716348in}{1.631651in}}%
\pgfpathlineto{\pgfqpoint{3.702982in}{1.631691in}}%
\pgfpathlineto{\pgfqpoint{3.695076in}{1.621170in}}%
\pgfpathlineto{\pgfqpoint{3.687165in}{1.610703in}}%
\pgfpathlineto{\pgfqpoint{3.679248in}{1.600296in}}%
\pgfpathlineto{\pgfqpoint{3.671325in}{1.589952in}}%
\pgfpathclose%
\pgfusepath{fill}%
\end{pgfscope}%
\begin{pgfscope}%
\pgfpathrectangle{\pgfqpoint{1.254980in}{0.150000in}}{\pgfqpoint{5.490039in}{5.490039in}}%
\pgfusepath{clip}%
\pgfsetbuttcap%
\pgfsetroundjoin%
\definecolor{currentfill}{rgb}{0.282884,0.135920,0.453427}%
\pgfsetfillcolor{currentfill}%
\pgfsetfillopacity{0.700000}%
\pgfsetlinewidth{0.000000pt}%
\definecolor{currentstroke}{rgb}{0.000000,0.000000,0.000000}%
\pgfsetstrokecolor{currentstroke}%
\pgfsetdash{}{0pt}%
\pgfpathmoveto{\pgfqpoint{2.781397in}{1.798301in}}%
\pgfpathlineto{\pgfqpoint{2.794840in}{1.785237in}}%
\pgfpathlineto{\pgfqpoint{2.808279in}{1.772381in}}%
\pgfpathlineto{\pgfqpoint{2.821714in}{1.759733in}}%
\pgfpathlineto{\pgfqpoint{2.835146in}{1.747291in}}%
\pgfpathlineto{\pgfqpoint{2.843602in}{1.747592in}}%
\pgfpathlineto{\pgfqpoint{2.852042in}{1.748163in}}%
\pgfpathlineto{\pgfqpoint{2.860467in}{1.748996in}}%
\pgfpathlineto{\pgfqpoint{2.868877in}{1.750087in}}%
\pgfpathlineto{\pgfqpoint{2.855485in}{1.761934in}}%
\pgfpathlineto{\pgfqpoint{2.842090in}{1.773988in}}%
\pgfpathlineto{\pgfqpoint{2.828691in}{1.786248in}}%
\pgfpathlineto{\pgfqpoint{2.815289in}{1.798716in}}%
\pgfpathlineto{\pgfqpoint{2.806840in}{1.798209in}}%
\pgfpathlineto{\pgfqpoint{2.798375in}{1.797967in}}%
\pgfpathlineto{\pgfqpoint{2.789894in}{1.797995in}}%
\pgfpathlineto{\pgfqpoint{2.781397in}{1.798301in}}%
\pgfpathclose%
\pgfusepath{fill}%
\end{pgfscope}%
\begin{pgfscope}%
\pgfpathrectangle{\pgfqpoint{1.254980in}{0.150000in}}{\pgfqpoint{5.490039in}{5.490039in}}%
\pgfusepath{clip}%
\pgfsetbuttcap%
\pgfsetroundjoin%
\definecolor{currentfill}{rgb}{0.280267,0.073417,0.397163}%
\pgfsetfillcolor{currentfill}%
\pgfsetfillopacity{0.700000}%
\pgfsetlinewidth{0.000000pt}%
\definecolor{currentstroke}{rgb}{0.000000,0.000000,0.000000}%
\pgfsetstrokecolor{currentstroke}%
\pgfsetdash{}{0pt}%
\pgfpathmoveto{\pgfqpoint{3.756491in}{1.632526in}}%
\pgfpathlineto{\pgfqpoint{3.769887in}{1.633148in}}%
\pgfpathlineto{\pgfqpoint{3.783291in}{1.633934in}}%
\pgfpathlineto{\pgfqpoint{3.796702in}{1.634884in}}%
\pgfpathlineto{\pgfqpoint{3.810122in}{1.635998in}}%
\pgfpathlineto{\pgfqpoint{3.818003in}{1.647364in}}%
\pgfpathlineto{\pgfqpoint{3.825879in}{1.658764in}}%
\pgfpathlineto{\pgfqpoint{3.833751in}{1.670194in}}%
\pgfpathlineto{\pgfqpoint{3.841617in}{1.681651in}}%
\pgfpathlineto{\pgfqpoint{3.828205in}{1.680151in}}%
\pgfpathlineto{\pgfqpoint{3.814802in}{1.678814in}}%
\pgfpathlineto{\pgfqpoint{3.801407in}{1.677642in}}%
\pgfpathlineto{\pgfqpoint{3.788020in}{1.676635in}}%
\pgfpathlineto{\pgfqpoint{3.780146in}{1.665553in}}%
\pgfpathlineto{\pgfqpoint{3.772266in}{1.654506in}}%
\pgfpathlineto{\pgfqpoint{3.764381in}{1.643495in}}%
\pgfpathlineto{\pgfqpoint{3.756491in}{1.632526in}}%
\pgfpathclose%
\pgfusepath{fill}%
\end{pgfscope}%
\begin{pgfscope}%
\pgfpathrectangle{\pgfqpoint{1.254980in}{0.150000in}}{\pgfqpoint{5.490039in}{5.490039in}}%
\pgfusepath{clip}%
\pgfsetbuttcap%
\pgfsetroundjoin%
\definecolor{currentfill}{rgb}{0.272594,0.025563,0.353093}%
\pgfsetfillcolor{currentfill}%
\pgfsetfillopacity{0.700000}%
\pgfsetlinewidth{0.000000pt}%
\definecolor{currentstroke}{rgb}{0.000000,0.000000,0.000000}%
\pgfsetstrokecolor{currentstroke}%
\pgfsetdash{}{0pt}%
\pgfpathmoveto{\pgfqpoint{3.586078in}{1.554530in}}%
\pgfpathlineto{\pgfqpoint{3.599443in}{1.552970in}}%
\pgfpathlineto{\pgfqpoint{3.612814in}{1.551576in}}%
\pgfpathlineto{\pgfqpoint{3.626191in}{1.550350in}}%
\pgfpathlineto{\pgfqpoint{3.639574in}{1.549290in}}%
\pgfpathlineto{\pgfqpoint{3.647520in}{1.559341in}}%
\pgfpathlineto{\pgfqpoint{3.655461in}{1.569471in}}%
\pgfpathlineto{\pgfqpoint{3.663396in}{1.579676in}}%
\pgfpathlineto{\pgfqpoint{3.671325in}{1.589952in}}%
\pgfpathlineto{\pgfqpoint{3.657954in}{1.590571in}}%
\pgfpathlineto{\pgfqpoint{3.644590in}{1.591357in}}%
\pgfpathlineto{\pgfqpoint{3.631232in}{1.592310in}}%
\pgfpathlineto{\pgfqpoint{3.617881in}{1.593430in}}%
\pgfpathlineto{\pgfqpoint{3.609939in}{1.583583in}}%
\pgfpathlineto{\pgfqpoint{3.601992in}{1.573815in}}%
\pgfpathlineto{\pgfqpoint{3.594038in}{1.564129in}}%
\pgfpathlineto{\pgfqpoint{3.586078in}{1.554530in}}%
\pgfpathclose%
\pgfusepath{fill}%
\end{pgfscope}%
\begin{pgfscope}%
\pgfpathrectangle{\pgfqpoint{1.254980in}{0.150000in}}{\pgfqpoint{5.490039in}{5.490039in}}%
\pgfusepath{clip}%
\pgfsetbuttcap%
\pgfsetroundjoin%
\definecolor{currentfill}{rgb}{0.282910,0.105393,0.426902}%
\pgfsetfillcolor{currentfill}%
\pgfsetfillopacity{0.700000}%
\pgfsetlinewidth{0.000000pt}%
\definecolor{currentstroke}{rgb}{0.000000,0.000000,0.000000}%
\pgfsetstrokecolor{currentstroke}%
\pgfsetdash{}{0pt}%
\pgfpathmoveto{\pgfqpoint{3.841617in}{1.681651in}}%
\pgfpathlineto{\pgfqpoint{3.855037in}{1.683315in}}%
\pgfpathlineto{\pgfqpoint{3.868465in}{1.685143in}}%
\pgfpathlineto{\pgfqpoint{3.881902in}{1.687134in}}%
\pgfpathlineto{\pgfqpoint{3.895349in}{1.689287in}}%
\pgfpathlineto{\pgfqpoint{3.903203in}{1.701137in}}%
\pgfpathlineto{\pgfqpoint{3.911052in}{1.713000in}}%
\pgfpathlineto{\pgfqpoint{3.918897in}{1.724873in}}%
\pgfpathlineto{\pgfqpoint{3.926737in}{1.736753in}}%
\pgfpathlineto{\pgfqpoint{3.913297in}{1.734240in}}%
\pgfpathlineto{\pgfqpoint{3.899866in}{1.731890in}}%
\pgfpathlineto{\pgfqpoint{3.886444in}{1.729704in}}%
\pgfpathlineto{\pgfqpoint{3.873031in}{1.727681in}}%
\pgfpathlineto{\pgfqpoint{3.865185in}{1.716150in}}%
\pgfpathlineto{\pgfqpoint{3.857334in}{1.704633in}}%
\pgfpathlineto{\pgfqpoint{3.849478in}{1.693132in}}%
\pgfpathlineto{\pgfqpoint{3.841617in}{1.681651in}}%
\pgfpathclose%
\pgfusepath{fill}%
\end{pgfscope}%
\begin{pgfscope}%
\pgfpathrectangle{\pgfqpoint{1.254980in}{0.150000in}}{\pgfqpoint{5.490039in}{5.490039in}}%
\pgfusepath{clip}%
\pgfsetbuttcap%
\pgfsetroundjoin%
\definecolor{currentfill}{rgb}{0.281477,0.755203,0.432552}%
\pgfsetfillcolor{currentfill}%
\pgfsetfillopacity{0.700000}%
\pgfsetlinewidth{0.000000pt}%
\definecolor{currentstroke}{rgb}{0.000000,0.000000,0.000000}%
\pgfsetstrokecolor{currentstroke}%
\pgfsetdash{}{0pt}%
\pgfpathmoveto{\pgfqpoint{5.677550in}{3.303197in}}%
\pgfpathlineto{\pgfqpoint{5.691966in}{3.317043in}}%
\pgfpathlineto{\pgfqpoint{5.706402in}{3.331050in}}%
\pgfpathlineto{\pgfqpoint{5.720859in}{3.345218in}}%
\pgfpathlineto{\pgfqpoint{5.735337in}{3.359548in}}%
\pgfpathlineto{\pgfqpoint{5.742322in}{3.361011in}}%
\pgfpathlineto{\pgfqpoint{5.749298in}{3.362401in}}%
\pgfpathlineto{\pgfqpoint{5.756265in}{3.363722in}}%
\pgfpathlineto{\pgfqpoint{5.763223in}{3.364980in}}%
\pgfpathlineto{\pgfqpoint{5.748770in}{3.351068in}}%
\pgfpathlineto{\pgfqpoint{5.734338in}{3.337317in}}%
\pgfpathlineto{\pgfqpoint{5.719926in}{3.323725in}}%
\pgfpathlineto{\pgfqpoint{5.705536in}{3.310293in}}%
\pgfpathlineto{\pgfqpoint{5.698552in}{3.308609in}}%
\pgfpathlineto{\pgfqpoint{5.691561in}{3.306868in}}%
\pgfpathlineto{\pgfqpoint{5.684560in}{3.305065in}}%
\pgfpathlineto{\pgfqpoint{5.677550in}{3.303197in}}%
\pgfpathclose%
\pgfusepath{fill}%
\end{pgfscope}%
\begin{pgfscope}%
\pgfpathrectangle{\pgfqpoint{1.254980in}{0.150000in}}{\pgfqpoint{5.490039in}{5.490039in}}%
\pgfusepath{clip}%
\pgfsetbuttcap%
\pgfsetroundjoin%
\definecolor{currentfill}{rgb}{0.273809,0.031497,0.358853}%
\pgfsetfillcolor{currentfill}%
\pgfsetfillopacity{0.700000}%
\pgfsetlinewidth{0.000000pt}%
\definecolor{currentstroke}{rgb}{0.000000,0.000000,0.000000}%
\pgfsetstrokecolor{currentstroke}%
\pgfsetdash{}{0pt}%
\pgfpathmoveto{\pgfqpoint{3.082880in}{1.587508in}}%
\pgfpathlineto{\pgfqpoint{3.096247in}{1.578979in}}%
\pgfpathlineto{\pgfqpoint{3.109615in}{1.570635in}}%
\pgfpathlineto{\pgfqpoint{3.122983in}{1.562477in}}%
\pgfpathlineto{\pgfqpoint{3.136351in}{1.554504in}}%
\pgfpathlineto{\pgfqpoint{3.144578in}{1.558696in}}%
\pgfpathlineto{\pgfqpoint{3.152793in}{1.563096in}}%
\pgfpathlineto{\pgfqpoint{3.160998in}{1.567698in}}%
\pgfpathlineto{\pgfqpoint{3.169191in}{1.572496in}}%
\pgfpathlineto{\pgfqpoint{3.155851in}{1.579914in}}%
\pgfpathlineto{\pgfqpoint{3.142513in}{1.587517in}}%
\pgfpathlineto{\pgfqpoint{3.129175in}{1.595304in}}%
\pgfpathlineto{\pgfqpoint{3.115837in}{1.603276in}}%
\pgfpathlineto{\pgfqpoint{3.107615in}{1.599023in}}%
\pgfpathlineto{\pgfqpoint{3.099382in}{1.594973in}}%
\pgfpathlineto{\pgfqpoint{3.091137in}{1.591133in}}%
\pgfpathlineto{\pgfqpoint{3.082880in}{1.587508in}}%
\pgfpathclose%
\pgfusepath{fill}%
\end{pgfscope}%
\begin{pgfscope}%
\pgfpathrectangle{\pgfqpoint{1.254980in}{0.150000in}}{\pgfqpoint{5.490039in}{5.490039in}}%
\pgfusepath{clip}%
\pgfsetbuttcap%
\pgfsetroundjoin%
\definecolor{currentfill}{rgb}{0.206756,0.371758,0.553117}%
\pgfsetfillcolor{currentfill}%
\pgfsetfillopacity{0.700000}%
\pgfsetlinewidth{0.000000pt}%
\definecolor{currentstroke}{rgb}{0.000000,0.000000,0.000000}%
\pgfsetstrokecolor{currentstroke}%
\pgfsetdash{}{0pt}%
\pgfpathmoveto{\pgfqpoint{2.347847in}{2.335860in}}%
\pgfpathlineto{\pgfqpoint{2.361534in}{2.315254in}}%
\pgfpathlineto{\pgfqpoint{2.375209in}{2.294915in}}%
\pgfpathlineto{\pgfqpoint{2.388872in}{2.274840in}}%
\pgfpathlineto{\pgfqpoint{2.402525in}{2.255026in}}%
\pgfpathlineto{\pgfqpoint{2.411366in}{2.250376in}}%
\pgfpathlineto{\pgfqpoint{2.420185in}{2.246059in}}%
\pgfpathlineto{\pgfqpoint{2.428983in}{2.242070in}}%
\pgfpathlineto{\pgfqpoint{2.437760in}{2.238403in}}%
\pgfpathlineto{\pgfqpoint{2.424162in}{2.257593in}}%
\pgfpathlineto{\pgfqpoint{2.410554in}{2.277043in}}%
\pgfpathlineto{\pgfqpoint{2.396935in}{2.296756in}}%
\pgfpathlineto{\pgfqpoint{2.383306in}{2.316734in}}%
\pgfpathlineto{\pgfqpoint{2.374474in}{2.321014in}}%
\pgfpathlineto{\pgfqpoint{2.365621in}{2.325624in}}%
\pgfpathlineto{\pgfqpoint{2.356745in}{2.330571in}}%
\pgfpathlineto{\pgfqpoint{2.347847in}{2.335860in}}%
\pgfpathclose%
\pgfusepath{fill}%
\end{pgfscope}%
\begin{pgfscope}%
\pgfpathrectangle{\pgfqpoint{1.254980in}{0.150000in}}{\pgfqpoint{5.490039in}{5.490039in}}%
\pgfusepath{clip}%
\pgfsetbuttcap%
\pgfsetroundjoin%
\definecolor{currentfill}{rgb}{0.283197,0.115680,0.436115}%
\pgfsetfillcolor{currentfill}%
\pgfsetfillopacity{0.700000}%
\pgfsetlinewidth{0.000000pt}%
\definecolor{currentstroke}{rgb}{0.000000,0.000000,0.000000}%
\pgfsetstrokecolor{currentstroke}%
\pgfsetdash{}{0pt}%
\pgfpathmoveto{\pgfqpoint{2.835146in}{1.747291in}}%
\pgfpathlineto{\pgfqpoint{2.848575in}{1.735053in}}%
\pgfpathlineto{\pgfqpoint{2.862000in}{1.723020in}}%
\pgfpathlineto{\pgfqpoint{2.875422in}{1.711188in}}%
\pgfpathlineto{\pgfqpoint{2.888842in}{1.699558in}}%
\pgfpathlineto{\pgfqpoint{2.897258in}{1.700464in}}%
\pgfpathlineto{\pgfqpoint{2.905659in}{1.701631in}}%
\pgfpathlineto{\pgfqpoint{2.914045in}{1.703054in}}%
\pgfpathlineto{\pgfqpoint{2.922418in}{1.704725in}}%
\pgfpathlineto{\pgfqpoint{2.909036in}{1.715763in}}%
\pgfpathlineto{\pgfqpoint{2.895652in}{1.727002in}}%
\pgfpathlineto{\pgfqpoint{2.882266in}{1.738443in}}%
\pgfpathlineto{\pgfqpoint{2.868877in}{1.750087in}}%
\pgfpathlineto{\pgfqpoint{2.860467in}{1.748996in}}%
\pgfpathlineto{\pgfqpoint{2.852042in}{1.748163in}}%
\pgfpathlineto{\pgfqpoint{2.843602in}{1.747592in}}%
\pgfpathlineto{\pgfqpoint{2.835146in}{1.747291in}}%
\pgfpathclose%
\pgfusepath{fill}%
\end{pgfscope}%
\begin{pgfscope}%
\pgfpathrectangle{\pgfqpoint{1.254980in}{0.150000in}}{\pgfqpoint{5.490039in}{5.490039in}}%
\pgfusepath{clip}%
\pgfsetbuttcap%
\pgfsetroundjoin%
\definecolor{currentfill}{rgb}{0.282884,0.135920,0.453427}%
\pgfsetfillcolor{currentfill}%
\pgfsetfillopacity{0.700000}%
\pgfsetlinewidth{0.000000pt}%
\definecolor{currentstroke}{rgb}{0.000000,0.000000,0.000000}%
\pgfsetstrokecolor{currentstroke}%
\pgfsetdash{}{0pt}%
\pgfpathmoveto{\pgfqpoint{3.926737in}{1.736753in}}%
\pgfpathlineto{\pgfqpoint{3.940186in}{1.739428in}}%
\pgfpathlineto{\pgfqpoint{3.953644in}{1.742266in}}%
\pgfpathlineto{\pgfqpoint{3.967112in}{1.745267in}}%
\pgfpathlineto{\pgfqpoint{3.980590in}{1.748429in}}%
\pgfpathlineto{\pgfqpoint{3.988420in}{1.760654in}}%
\pgfpathlineto{\pgfqpoint{3.996245in}{1.772873in}}%
\pgfpathlineto{\pgfqpoint{4.004066in}{1.785081in}}%
\pgfpathlineto{\pgfqpoint{4.011882in}{1.797276in}}%
\pgfpathlineto{\pgfqpoint{3.998409in}{1.793782in}}%
\pgfpathlineto{\pgfqpoint{3.984946in}{1.790449in}}%
\pgfpathlineto{\pgfqpoint{3.971492in}{1.787279in}}%
\pgfpathlineto{\pgfqpoint{3.958049in}{1.784272in}}%
\pgfpathlineto{\pgfqpoint{3.950228in}{1.772398in}}%
\pgfpathlineto{\pgfqpoint{3.942402in}{1.760518in}}%
\pgfpathlineto{\pgfqpoint{3.934572in}{1.748635in}}%
\pgfpathlineto{\pgfqpoint{3.926737in}{1.736753in}}%
\pgfpathclose%
\pgfusepath{fill}%
\end{pgfscope}%
\begin{pgfscope}%
\pgfpathrectangle{\pgfqpoint{1.254980in}{0.150000in}}{\pgfqpoint{5.490039in}{5.490039in}}%
\pgfusepath{clip}%
\pgfsetbuttcap%
\pgfsetroundjoin%
\definecolor{currentfill}{rgb}{0.127568,0.566949,0.550556}%
\pgfsetfillcolor{currentfill}%
\pgfsetfillopacity{0.700000}%
\pgfsetlinewidth{0.000000pt}%
\definecolor{currentstroke}{rgb}{0.000000,0.000000,0.000000}%
\pgfsetstrokecolor{currentstroke}%
\pgfsetdash{}{0pt}%
\pgfpathmoveto{\pgfqpoint{5.019268in}{2.771608in}}%
\pgfpathlineto{\pgfqpoint{5.033274in}{2.783276in}}%
\pgfpathlineto{\pgfqpoint{5.047299in}{2.795106in}}%
\pgfpathlineto{\pgfqpoint{5.061341in}{2.807097in}}%
\pgfpathlineto{\pgfqpoint{5.075401in}{2.819250in}}%
\pgfpathlineto{\pgfqpoint{5.082823in}{2.826658in}}%
\pgfpathlineto{\pgfqpoint{5.090237in}{2.833933in}}%
\pgfpathlineto{\pgfqpoint{5.097643in}{2.841075in}}%
\pgfpathlineto{\pgfqpoint{5.105040in}{2.848088in}}%
\pgfpathlineto{\pgfqpoint{5.090988in}{2.836070in}}%
\pgfpathlineto{\pgfqpoint{5.076955in}{2.824213in}}%
\pgfpathlineto{\pgfqpoint{5.062939in}{2.812518in}}%
\pgfpathlineto{\pgfqpoint{5.048940in}{2.800983in}}%
\pgfpathlineto{\pgfqpoint{5.041534in}{2.793825in}}%
\pgfpathlineto{\pgfqpoint{5.034120in}{2.786544in}}%
\pgfpathlineto{\pgfqpoint{5.026698in}{2.779139in}}%
\pgfpathlineto{\pgfqpoint{5.019268in}{2.771608in}}%
\pgfpathclose%
\pgfusepath{fill}%
\end{pgfscope}%
\begin{pgfscope}%
\pgfpathrectangle{\pgfqpoint{1.254980in}{0.150000in}}{\pgfqpoint{5.490039in}{5.490039in}}%
\pgfusepath{clip}%
\pgfsetbuttcap%
\pgfsetroundjoin%
\definecolor{currentfill}{rgb}{0.139147,0.533812,0.555298}%
\pgfsetfillcolor{currentfill}%
\pgfsetfillopacity{0.700000}%
\pgfsetlinewidth{0.000000pt}%
\definecolor{currentstroke}{rgb}{0.000000,0.000000,0.000000}%
\pgfsetstrokecolor{currentstroke}%
\pgfsetdash{}{0pt}%
\pgfpathmoveto{\pgfqpoint{2.107952in}{2.775896in}}%
\pgfpathlineto{\pgfqpoint{2.121865in}{2.750063in}}%
\pgfpathlineto{\pgfqpoint{2.135760in}{2.724550in}}%
\pgfpathlineto{\pgfqpoint{2.149639in}{2.699356in}}%
\pgfpathlineto{\pgfqpoint{2.163501in}{2.674476in}}%
\pgfpathlineto{\pgfqpoint{2.172547in}{2.668046in}}%
\pgfpathlineto{\pgfqpoint{2.181569in}{2.661969in}}%
\pgfpathlineto{\pgfqpoint{2.190567in}{2.656238in}}%
\pgfpathlineto{\pgfqpoint{2.199541in}{2.650847in}}%
\pgfpathlineto{\pgfqpoint{2.185742in}{2.675109in}}%
\pgfpathlineto{\pgfqpoint{2.171927in}{2.699684in}}%
\pgfpathlineto{\pgfqpoint{2.158095in}{2.724574in}}%
\pgfpathlineto{\pgfqpoint{2.144246in}{2.749785in}}%
\pgfpathlineto{\pgfqpoint{2.135210in}{2.755783in}}%
\pgfpathlineto{\pgfqpoint{2.126149in}{2.762130in}}%
\pgfpathlineto{\pgfqpoint{2.117063in}{2.768832in}}%
\pgfpathlineto{\pgfqpoint{2.107952in}{2.775896in}}%
\pgfpathclose%
\pgfusepath{fill}%
\end{pgfscope}%
\begin{pgfscope}%
\pgfpathrectangle{\pgfqpoint{1.254980in}{0.150000in}}{\pgfqpoint{5.490039in}{5.490039in}}%
\pgfusepath{clip}%
\pgfsetbuttcap%
\pgfsetroundjoin%
\definecolor{currentfill}{rgb}{0.268510,0.009605,0.335427}%
\pgfsetfillcolor{currentfill}%
\pgfsetfillopacity{0.700000}%
\pgfsetlinewidth{0.000000pt}%
\definecolor{currentstroke}{rgb}{0.000000,0.000000,0.000000}%
\pgfsetstrokecolor{currentstroke}%
\pgfsetdash{}{0pt}%
\pgfpathmoveto{\pgfqpoint{3.500707in}{1.526883in}}%
\pgfpathlineto{\pgfqpoint{3.514066in}{1.524180in}}%
\pgfpathlineto{\pgfqpoint{3.527429in}{1.521647in}}%
\pgfpathlineto{\pgfqpoint{3.540798in}{1.519283in}}%
\pgfpathlineto{\pgfqpoint{3.554171in}{1.517086in}}%
\pgfpathlineto{\pgfqpoint{3.562158in}{1.526295in}}%
\pgfpathlineto{\pgfqpoint{3.570138in}{1.535609in}}%
\pgfpathlineto{\pgfqpoint{3.578111in}{1.545022in}}%
\pgfpathlineto{\pgfqpoint{3.586078in}{1.554530in}}%
\pgfpathlineto{\pgfqpoint{3.572719in}{1.556258in}}%
\pgfpathlineto{\pgfqpoint{3.559366in}{1.558155in}}%
\pgfpathlineto{\pgfqpoint{3.546018in}{1.560220in}}%
\pgfpathlineto{\pgfqpoint{3.532676in}{1.562454in}}%
\pgfpathlineto{\pgfqpoint{3.524694in}{1.553404in}}%
\pgfpathlineto{\pgfqpoint{3.516705in}{1.544455in}}%
\pgfpathlineto{\pgfqpoint{3.508710in}{1.535613in}}%
\pgfpathlineto{\pgfqpoint{3.500707in}{1.526883in}}%
\pgfpathclose%
\pgfusepath{fill}%
\end{pgfscope}%
\begin{pgfscope}%
\pgfpathrectangle{\pgfqpoint{1.254980in}{0.150000in}}{\pgfqpoint{5.490039in}{5.490039in}}%
\pgfusepath{clip}%
\pgfsetbuttcap%
\pgfsetroundjoin%
\definecolor{currentfill}{rgb}{0.134692,0.658636,0.517649}%
\pgfsetfillcolor{currentfill}%
\pgfsetfillopacity{0.700000}%
\pgfsetlinewidth{0.000000pt}%
\definecolor{currentstroke}{rgb}{0.000000,0.000000,0.000000}%
\pgfsetstrokecolor{currentstroke}%
\pgfsetdash{}{0pt}%
\pgfpathmoveto{\pgfqpoint{5.305979in}{3.020106in}}%
\pgfpathlineto{\pgfqpoint{5.320165in}{3.032986in}}%
\pgfpathlineto{\pgfqpoint{5.334370in}{3.046028in}}%
\pgfpathlineto{\pgfqpoint{5.348595in}{3.059230in}}%
\pgfpathlineto{\pgfqpoint{5.362840in}{3.072595in}}%
\pgfpathlineto{\pgfqpoint{5.370089in}{3.077379in}}%
\pgfpathlineto{\pgfqpoint{5.377330in}{3.082045in}}%
\pgfpathlineto{\pgfqpoint{5.384561in}{3.086595in}}%
\pgfpathlineto{\pgfqpoint{5.391783in}{3.091032in}}%
\pgfpathlineto{\pgfqpoint{5.377554in}{3.077928in}}%
\pgfpathlineto{\pgfqpoint{5.363343in}{3.064984in}}%
\pgfpathlineto{\pgfqpoint{5.349153in}{3.052202in}}%
\pgfpathlineto{\pgfqpoint{5.334981in}{3.039580in}}%
\pgfpathlineto{\pgfqpoint{5.327743in}{3.034873in}}%
\pgfpathlineto{\pgfqpoint{5.320497in}{3.030060in}}%
\pgfpathlineto{\pgfqpoint{5.313242in}{3.025139in}}%
\pgfpathlineto{\pgfqpoint{5.305979in}{3.020106in}}%
\pgfpathclose%
\pgfusepath{fill}%
\end{pgfscope}%
\begin{pgfscope}%
\pgfpathrectangle{\pgfqpoint{1.254980in}{0.150000in}}{\pgfqpoint{5.490039in}{5.490039in}}%
\pgfusepath{clip}%
\pgfsetbuttcap%
\pgfsetroundjoin%
\definecolor{currentfill}{rgb}{0.369214,0.788888,0.382914}%
\pgfsetfillcolor{currentfill}%
\pgfsetfillopacity{0.700000}%
\pgfsetlinewidth{0.000000pt}%
\definecolor{currentstroke}{rgb}{0.000000,0.000000,0.000000}%
\pgfsetstrokecolor{currentstroke}%
\pgfsetdash{}{0pt}%
\pgfpathmoveto{\pgfqpoint{5.848885in}{3.424928in}}%
\pgfpathlineto{\pgfqpoint{5.863418in}{3.439194in}}%
\pgfpathlineto{\pgfqpoint{5.877973in}{3.453620in}}%
\pgfpathlineto{\pgfqpoint{5.892549in}{3.468208in}}%
\pgfpathlineto{\pgfqpoint{5.899414in}{3.468400in}}%
\pgfpathlineto{\pgfqpoint{5.906270in}{3.468545in}}%
\pgfpathlineto{\pgfqpoint{5.913118in}{3.468649in}}%
\pgfpathlineto{\pgfqpoint{5.919957in}{3.468715in}}%
\pgfpathlineto{\pgfqpoint{5.905410in}{3.454609in}}%
\pgfpathlineto{\pgfqpoint{5.890886in}{3.440661in}}%
\pgfpathlineto{\pgfqpoint{5.876382in}{3.426874in}}%
\pgfpathlineto{\pgfqpoint{5.869520in}{3.426440in}}%
\pgfpathlineto{\pgfqpoint{5.862650in}{3.425974in}}%
\pgfpathlineto{\pgfqpoint{5.855772in}{3.425472in}}%
\pgfpathlineto{\pgfqpoint{5.848885in}{3.424928in}}%
\pgfpathclose%
\pgfusepath{fill}%
\end{pgfscope}%
\begin{pgfscope}%
\pgfpathrectangle{\pgfqpoint{1.254980in}{0.150000in}}{\pgfqpoint{5.490039in}{5.490039in}}%
\pgfusepath{clip}%
\pgfsetbuttcap%
\pgfsetroundjoin%
\definecolor{currentfill}{rgb}{0.327796,0.773980,0.406640}%
\pgfsetfillcolor{currentfill}%
\pgfsetfillopacity{0.700000}%
\pgfsetlinewidth{0.000000pt}%
\definecolor{currentstroke}{rgb}{0.000000,0.000000,0.000000}%
\pgfsetstrokecolor{currentstroke}%
\pgfsetdash{}{0pt}%
\pgfpathmoveto{\pgfqpoint{5.763223in}{3.364980in}}%
\pgfpathlineto{\pgfqpoint{5.777698in}{3.379052in}}%
\pgfpathlineto{\pgfqpoint{5.792193in}{3.393285in}}%
\pgfpathlineto{\pgfqpoint{5.806711in}{3.407679in}}%
\pgfpathlineto{\pgfqpoint{5.821250in}{3.422234in}}%
\pgfpathlineto{\pgfqpoint{5.828172in}{3.422995in}}%
\pgfpathlineto{\pgfqpoint{5.835085in}{3.423694in}}%
\pgfpathlineto{\pgfqpoint{5.841989in}{3.424337in}}%
\pgfpathlineto{\pgfqpoint{5.848885in}{3.424928in}}%
\pgfpathlineto{\pgfqpoint{5.834374in}{3.410822in}}%
\pgfpathlineto{\pgfqpoint{5.819884in}{3.396877in}}%
\pgfpathlineto{\pgfqpoint{5.805415in}{3.383091in}}%
\pgfpathlineto{\pgfqpoint{5.790968in}{3.369465in}}%
\pgfpathlineto{\pgfqpoint{5.784044in}{3.368416in}}%
\pgfpathlineto{\pgfqpoint{5.777112in}{3.367322in}}%
\pgfpathlineto{\pgfqpoint{5.770172in}{3.366178in}}%
\pgfpathlineto{\pgfqpoint{5.763223in}{3.364980in}}%
\pgfpathclose%
\pgfusepath{fill}%
\end{pgfscope}%
\begin{pgfscope}%
\pgfpathrectangle{\pgfqpoint{1.254980in}{0.150000in}}{\pgfqpoint{5.490039in}{5.490039in}}%
\pgfusepath{clip}%
\pgfsetbuttcap%
\pgfsetroundjoin%
\definecolor{currentfill}{rgb}{0.257322,0.256130,0.526563}%
\pgfsetfillcolor{currentfill}%
\pgfsetfillopacity{0.700000}%
\pgfsetlinewidth{0.000000pt}%
\definecolor{currentstroke}{rgb}{0.000000,0.000000,0.000000}%
\pgfsetstrokecolor{currentstroke}%
\pgfsetdash{}{0pt}%
\pgfpathmoveto{\pgfqpoint{4.213401in}{1.982053in}}%
\pgfpathlineto{\pgfqpoint{4.226964in}{1.987834in}}%
\pgfpathlineto{\pgfqpoint{4.240539in}{1.993775in}}%
\pgfpathlineto{\pgfqpoint{4.254126in}{1.999878in}}%
\pgfpathlineto{\pgfqpoint{4.267725in}{2.006141in}}%
\pgfpathlineto{\pgfqpoint{4.275473in}{2.018625in}}%
\pgfpathlineto{\pgfqpoint{4.283217in}{2.031040in}}%
\pgfpathlineto{\pgfqpoint{4.290956in}{2.043386in}}%
\pgfpathlineto{\pgfqpoint{4.298690in}{2.055659in}}%
\pgfpathlineto{\pgfqpoint{4.285092in}{2.049174in}}%
\pgfpathlineto{\pgfqpoint{4.271507in}{2.042851in}}%
\pgfpathlineto{\pgfqpoint{4.257934in}{2.036688in}}%
\pgfpathlineto{\pgfqpoint{4.244373in}{2.030688in}}%
\pgfpathlineto{\pgfqpoint{4.236637in}{2.018624in}}%
\pgfpathlineto{\pgfqpoint{4.228897in}{2.006496in}}%
\pgfpathlineto{\pgfqpoint{4.221151in}{1.994305in}}%
\pgfpathlineto{\pgfqpoint{4.213401in}{1.982053in}}%
\pgfpathclose%
\pgfusepath{fill}%
\end{pgfscope}%
\begin{pgfscope}%
\pgfpathrectangle{\pgfqpoint{1.254980in}{0.150000in}}{\pgfqpoint{5.490039in}{5.490039in}}%
\pgfusepath{clip}%
\pgfsetbuttcap%
\pgfsetroundjoin%
\definecolor{currentfill}{rgb}{0.218130,0.347432,0.550038}%
\pgfsetfillcolor{currentfill}%
\pgfsetfillopacity{0.700000}%
\pgfsetlinewidth{0.000000pt}%
\definecolor{currentstroke}{rgb}{0.000000,0.000000,0.000000}%
\pgfsetstrokecolor{currentstroke}%
\pgfsetdash{}{0pt}%
\pgfpathmoveto{\pgfqpoint{4.414894in}{2.180040in}}%
\pgfpathlineto{\pgfqpoint{4.428554in}{2.187688in}}%
\pgfpathlineto{\pgfqpoint{4.442228in}{2.195496in}}%
\pgfpathlineto{\pgfqpoint{4.455916in}{2.203466in}}%
\pgfpathlineto{\pgfqpoint{4.469618in}{2.211596in}}%
\pgfpathlineto{\pgfqpoint{4.477304in}{2.223445in}}%
\pgfpathlineto{\pgfqpoint{4.484985in}{2.235193in}}%
\pgfpathlineto{\pgfqpoint{4.492661in}{2.246839in}}%
\pgfpathlineto{\pgfqpoint{4.500331in}{2.258383in}}%
\pgfpathlineto{\pgfqpoint{4.486631in}{2.250117in}}%
\pgfpathlineto{\pgfqpoint{4.472944in}{2.242012in}}%
\pgfpathlineto{\pgfqpoint{4.459272in}{2.234068in}}%
\pgfpathlineto{\pgfqpoint{4.445613in}{2.226286in}}%
\pgfpathlineto{\pgfqpoint{4.437941in}{2.214866in}}%
\pgfpathlineto{\pgfqpoint{4.430264in}{2.203352in}}%
\pgfpathlineto{\pgfqpoint{4.422582in}{2.191743in}}%
\pgfpathlineto{\pgfqpoint{4.414894in}{2.180040in}}%
\pgfpathclose%
\pgfusepath{fill}%
\end{pgfscope}%
\begin{pgfscope}%
\pgfpathrectangle{\pgfqpoint{1.254980in}{0.150000in}}{\pgfqpoint{5.490039in}{5.490039in}}%
\pgfusepath{clip}%
\pgfsetbuttcap%
\pgfsetroundjoin%
\definecolor{currentfill}{rgb}{0.151918,0.500685,0.557587}%
\pgfsetfillcolor{currentfill}%
\pgfsetfillopacity{0.700000}%
\pgfsetlinewidth{0.000000pt}%
\definecolor{currentstroke}{rgb}{0.000000,0.000000,0.000000}%
\pgfsetstrokecolor{currentstroke}%
\pgfsetdash{}{0pt}%
\pgfpathmoveto{\pgfqpoint{4.817930in}{2.581557in}}%
\pgfpathlineto{\pgfqpoint{4.831817in}{2.592154in}}%
\pgfpathlineto{\pgfqpoint{4.845721in}{2.602913in}}%
\pgfpathlineto{\pgfqpoint{4.859642in}{2.613833in}}%
\pgfpathlineto{\pgfqpoint{4.873579in}{2.624915in}}%
\pgfpathlineto{\pgfqpoint{4.881106in}{2.634101in}}%
\pgfpathlineto{\pgfqpoint{4.888626in}{2.643153in}}%
\pgfpathlineto{\pgfqpoint{4.896139in}{2.652073in}}%
\pgfpathlineto{\pgfqpoint{4.903644in}{2.660860in}}%
\pgfpathlineto{\pgfqpoint{4.889711in}{2.649822in}}%
\pgfpathlineto{\pgfqpoint{4.875796in}{2.638945in}}%
\pgfpathlineto{\pgfqpoint{4.861896in}{2.628229in}}%
\pgfpathlineto{\pgfqpoint{4.848014in}{2.617674in}}%
\pgfpathlineto{\pgfqpoint{4.840504in}{2.608832in}}%
\pgfpathlineto{\pgfqpoint{4.832986in}{2.599866in}}%
\pgfpathlineto{\pgfqpoint{4.825462in}{2.590775in}}%
\pgfpathlineto{\pgfqpoint{4.817930in}{2.581557in}}%
\pgfpathclose%
\pgfusepath{fill}%
\end{pgfscope}%
\begin{pgfscope}%
\pgfpathrectangle{\pgfqpoint{1.254980in}{0.150000in}}{\pgfqpoint{5.490039in}{5.490039in}}%
\pgfusepath{clip}%
\pgfsetbuttcap%
\pgfsetroundjoin%
\definecolor{currentfill}{rgb}{0.182256,0.426184,0.557120}%
\pgfsetfillcolor{currentfill}%
\pgfsetfillopacity{0.700000}%
\pgfsetlinewidth{0.000000pt}%
\definecolor{currentstroke}{rgb}{0.000000,0.000000,0.000000}%
\pgfsetstrokecolor{currentstroke}%
\pgfsetdash{}{0pt}%
\pgfpathmoveto{\pgfqpoint{4.616423in}{2.382362in}}%
\pgfpathlineto{\pgfqpoint{4.630193in}{2.391617in}}%
\pgfpathlineto{\pgfqpoint{4.643979in}{2.401033in}}%
\pgfpathlineto{\pgfqpoint{4.657779in}{2.410611in}}%
\pgfpathlineto{\pgfqpoint{4.671596in}{2.420349in}}%
\pgfpathlineto{\pgfqpoint{4.679210in}{2.431058in}}%
\pgfpathlineto{\pgfqpoint{4.686818in}{2.441643in}}%
\pgfpathlineto{\pgfqpoint{4.694420in}{2.452107in}}%
\pgfpathlineto{\pgfqpoint{4.702015in}{2.462447in}}%
\pgfpathlineto{\pgfqpoint{4.688201in}{2.452662in}}%
\pgfpathlineto{\pgfqpoint{4.674403in}{2.443037in}}%
\pgfpathlineto{\pgfqpoint{4.660620in}{2.433573in}}%
\pgfpathlineto{\pgfqpoint{4.646852in}{2.424271in}}%
\pgfpathlineto{\pgfqpoint{4.639254in}{2.413967in}}%
\pgfpathlineto{\pgfqpoint{4.631650in}{2.403547in}}%
\pgfpathlineto{\pgfqpoint{4.624039in}{2.393012in}}%
\pgfpathlineto{\pgfqpoint{4.616423in}{2.382362in}}%
\pgfpathclose%
\pgfusepath{fill}%
\end{pgfscope}%
\begin{pgfscope}%
\pgfpathrectangle{\pgfqpoint{1.254980in}{0.150000in}}{\pgfqpoint{5.490039in}{5.490039in}}%
\pgfusepath{clip}%
\pgfsetbuttcap%
\pgfsetroundjoin%
\definecolor{currentfill}{rgb}{0.267004,0.004874,0.329415}%
\pgfsetfillcolor{currentfill}%
\pgfsetfillopacity{0.700000}%
\pgfsetlinewidth{0.000000pt}%
\definecolor{currentstroke}{rgb}{0.000000,0.000000,0.000000}%
\pgfsetstrokecolor{currentstroke}%
\pgfsetdash{}{0pt}%
\pgfpathmoveto{\pgfqpoint{3.275958in}{1.519681in}}%
\pgfpathlineto{\pgfqpoint{3.289313in}{1.513884in}}%
\pgfpathlineto{\pgfqpoint{3.302670in}{1.508263in}}%
\pgfpathlineto{\pgfqpoint{3.316030in}{1.502818in}}%
\pgfpathlineto{\pgfqpoint{3.329393in}{1.497548in}}%
\pgfpathlineto{\pgfqpoint{3.337500in}{1.504146in}}%
\pgfpathlineto{\pgfqpoint{3.345598in}{1.510909in}}%
\pgfpathlineto{\pgfqpoint{3.353687in}{1.517831in}}%
\pgfpathlineto{\pgfqpoint{3.361767in}{1.524908in}}%
\pgfpathlineto{\pgfqpoint{3.348427in}{1.529653in}}%
\pgfpathlineto{\pgfqpoint{3.335090in}{1.534574in}}%
\pgfpathlineto{\pgfqpoint{3.321755in}{1.539670in}}%
\pgfpathlineto{\pgfqpoint{3.308424in}{1.544942in}}%
\pgfpathlineto{\pgfqpoint{3.300322in}{1.538380in}}%
\pgfpathlineto{\pgfqpoint{3.292210in}{1.531979in}}%
\pgfpathlineto{\pgfqpoint{3.284089in}{1.525744in}}%
\pgfpathlineto{\pgfqpoint{3.275958in}{1.519681in}}%
\pgfpathclose%
\pgfusepath{fill}%
\end{pgfscope}%
\begin{pgfscope}%
\pgfpathrectangle{\pgfqpoint{1.254980in}{0.150000in}}{\pgfqpoint{5.490039in}{5.490039in}}%
\pgfusepath{clip}%
\pgfsetbuttcap%
\pgfsetroundjoin%
\definecolor{currentfill}{rgb}{0.282327,0.094955,0.417331}%
\pgfsetfillcolor{currentfill}%
\pgfsetfillopacity{0.700000}%
\pgfsetlinewidth{0.000000pt}%
\definecolor{currentstroke}{rgb}{0.000000,0.000000,0.000000}%
\pgfsetstrokecolor{currentstroke}%
\pgfsetdash{}{0pt}%
\pgfpathmoveto{\pgfqpoint{2.888842in}{1.699558in}}%
\pgfpathlineto{\pgfqpoint{2.902259in}{1.688128in}}%
\pgfpathlineto{\pgfqpoint{2.915673in}{1.676896in}}%
\pgfpathlineto{\pgfqpoint{2.929086in}{1.665862in}}%
\pgfpathlineto{\pgfqpoint{2.942496in}{1.655025in}}%
\pgfpathlineto{\pgfqpoint{2.950874in}{1.656533in}}%
\pgfpathlineto{\pgfqpoint{2.959238in}{1.658295in}}%
\pgfpathlineto{\pgfqpoint{2.967588in}{1.660304in}}%
\pgfpathlineto{\pgfqpoint{2.975924in}{1.662555in}}%
\pgfpathlineto{\pgfqpoint{2.962550in}{1.672802in}}%
\pgfpathlineto{\pgfqpoint{2.949174in}{1.683246in}}%
\pgfpathlineto{\pgfqpoint{2.935797in}{1.693886in}}%
\pgfpathlineto{\pgfqpoint{2.922418in}{1.704725in}}%
\pgfpathlineto{\pgfqpoint{2.914045in}{1.703054in}}%
\pgfpathlineto{\pgfqpoint{2.905659in}{1.701631in}}%
\pgfpathlineto{\pgfqpoint{2.897258in}{1.700464in}}%
\pgfpathlineto{\pgfqpoint{2.888842in}{1.699558in}}%
\pgfpathclose%
\pgfusepath{fill}%
\end{pgfscope}%
\begin{pgfscope}%
\pgfpathrectangle{\pgfqpoint{1.254980in}{0.150000in}}{\pgfqpoint{5.490039in}{5.490039in}}%
\pgfusepath{clip}%
\pgfsetbuttcap%
\pgfsetroundjoin%
\definecolor{currentfill}{rgb}{0.190631,0.407061,0.556089}%
\pgfsetfillcolor{currentfill}%
\pgfsetfillopacity{0.700000}%
\pgfsetlinewidth{0.000000pt}%
\definecolor{currentstroke}{rgb}{0.000000,0.000000,0.000000}%
\pgfsetstrokecolor{currentstroke}%
\pgfsetdash{}{0pt}%
\pgfpathmoveto{\pgfqpoint{2.292981in}{2.420993in}}%
\pgfpathlineto{\pgfqpoint{2.306716in}{2.399298in}}%
\pgfpathlineto{\pgfqpoint{2.320439in}{2.377879in}}%
\pgfpathlineto{\pgfqpoint{2.334149in}{2.356734in}}%
\pgfpathlineto{\pgfqpoint{2.347847in}{2.335860in}}%
\pgfpathlineto{\pgfqpoint{2.356745in}{2.330571in}}%
\pgfpathlineto{\pgfqpoint{2.365621in}{2.325624in}}%
\pgfpathlineto{\pgfqpoint{2.374474in}{2.321014in}}%
\pgfpathlineto{\pgfqpoint{2.383306in}{2.316734in}}%
\pgfpathlineto{\pgfqpoint{2.369665in}{2.336979in}}%
\pgfpathlineto{\pgfqpoint{2.356012in}{2.357494in}}%
\pgfpathlineto{\pgfqpoint{2.342348in}{2.378281in}}%
\pgfpathlineto{\pgfqpoint{2.328672in}{2.399343in}}%
\pgfpathlineto{\pgfqpoint{2.319783in}{2.404241in}}%
\pgfpathlineto{\pgfqpoint{2.310872in}{2.409478in}}%
\pgfpathlineto{\pgfqpoint{2.301938in}{2.415060in}}%
\pgfpathlineto{\pgfqpoint{2.292981in}{2.420993in}}%
\pgfpathclose%
\pgfusepath{fill}%
\end{pgfscope}%
\begin{pgfscope}%
\pgfpathrectangle{\pgfqpoint{1.254980in}{0.150000in}}{\pgfqpoint{5.490039in}{5.490039in}}%
\pgfusepath{clip}%
\pgfsetbuttcap%
\pgfsetroundjoin%
\definecolor{currentfill}{rgb}{0.279574,0.170599,0.479997}%
\pgfsetfillcolor{currentfill}%
\pgfsetfillopacity{0.700000}%
\pgfsetlinewidth{0.000000pt}%
\definecolor{currentstroke}{rgb}{0.000000,0.000000,0.000000}%
\pgfsetstrokecolor{currentstroke}%
\pgfsetdash{}{0pt}%
\pgfpathmoveto{\pgfqpoint{4.011882in}{1.797276in}}%
\pgfpathlineto{\pgfqpoint{4.025365in}{1.800933in}}%
\pgfpathlineto{\pgfqpoint{4.038859in}{1.804752in}}%
\pgfpathlineto{\pgfqpoint{4.052362in}{1.808732in}}%
\pgfpathlineto{\pgfqpoint{4.065877in}{1.812874in}}%
\pgfpathlineto{\pgfqpoint{4.073684in}{1.825369in}}%
\pgfpathlineto{\pgfqpoint{4.081487in}{1.837838in}}%
\pgfpathlineto{\pgfqpoint{4.089286in}{1.850279in}}%
\pgfpathlineto{\pgfqpoint{4.097080in}{1.862688in}}%
\pgfpathlineto{\pgfqpoint{4.083569in}{1.858241in}}%
\pgfpathlineto{\pgfqpoint{4.070069in}{1.853956in}}%
\pgfpathlineto{\pgfqpoint{4.056579in}{1.849833in}}%
\pgfpathlineto{\pgfqpoint{4.043100in}{1.845872in}}%
\pgfpathlineto{\pgfqpoint{4.035302in}{1.833757in}}%
\pgfpathlineto{\pgfqpoint{4.027500in}{1.821617in}}%
\pgfpathlineto{\pgfqpoint{4.019693in}{1.809456in}}%
\pgfpathlineto{\pgfqpoint{4.011882in}{1.797276in}}%
\pgfpathclose%
\pgfusepath{fill}%
\end{pgfscope}%
\begin{pgfscope}%
\pgfpathrectangle{\pgfqpoint{1.254980in}{0.150000in}}{\pgfqpoint{5.490039in}{5.490039in}}%
\pgfusepath{clip}%
\pgfsetbuttcap%
\pgfsetroundjoin%
\definecolor{currentfill}{rgb}{0.271305,0.019942,0.347269}%
\pgfsetfillcolor{currentfill}%
\pgfsetfillopacity{0.700000}%
\pgfsetlinewidth{0.000000pt}%
\definecolor{currentstroke}{rgb}{0.000000,0.000000,0.000000}%
\pgfsetstrokecolor{currentstroke}%
\pgfsetdash{}{0pt}%
\pgfpathmoveto{\pgfqpoint{3.136351in}{1.554504in}}%
\pgfpathlineto{\pgfqpoint{3.149720in}{1.546714in}}%
\pgfpathlineto{\pgfqpoint{3.163090in}{1.539106in}}%
\pgfpathlineto{\pgfqpoint{3.176461in}{1.531681in}}%
\pgfpathlineto{\pgfqpoint{3.189833in}{1.524436in}}%
\pgfpathlineto{\pgfqpoint{3.198031in}{1.529194in}}%
\pgfpathlineto{\pgfqpoint{3.206218in}{1.534152in}}%
\pgfpathlineto{\pgfqpoint{3.214395in}{1.539306in}}%
\pgfpathlineto{\pgfqpoint{3.222561in}{1.544649in}}%
\pgfpathlineto{\pgfqpoint{3.209216in}{1.551339in}}%
\pgfpathlineto{\pgfqpoint{3.195873in}{1.558209in}}%
\pgfpathlineto{\pgfqpoint{3.182531in}{1.565262in}}%
\pgfpathlineto{\pgfqpoint{3.169191in}{1.572496in}}%
\pgfpathlineto{\pgfqpoint{3.160998in}{1.567698in}}%
\pgfpathlineto{\pgfqpoint{3.152793in}{1.563096in}}%
\pgfpathlineto{\pgfqpoint{3.144578in}{1.558696in}}%
\pgfpathlineto{\pgfqpoint{3.136351in}{1.554504in}}%
\pgfpathclose%
\pgfusepath{fill}%
\end{pgfscope}%
\begin{pgfscope}%
\pgfpathrectangle{\pgfqpoint{1.254980in}{0.150000in}}{\pgfqpoint{5.490039in}{5.490039in}}%
\pgfusepath{clip}%
\pgfsetbuttcap%
\pgfsetroundjoin%
\definecolor{currentfill}{rgb}{0.267004,0.004874,0.329415}%
\pgfsetfillcolor{currentfill}%
\pgfsetfillopacity{0.700000}%
\pgfsetlinewidth{0.000000pt}%
\definecolor{currentstroke}{rgb}{0.000000,0.000000,0.000000}%
\pgfsetstrokecolor{currentstroke}%
\pgfsetdash{}{0pt}%
\pgfpathmoveto{\pgfqpoint{3.415163in}{1.507662in}}%
\pgfpathlineto{\pgfqpoint{3.428522in}{1.503782in}}%
\pgfpathlineto{\pgfqpoint{3.441884in}{1.500073in}}%
\pgfpathlineto{\pgfqpoint{3.455251in}{1.496534in}}%
\pgfpathlineto{\pgfqpoint{3.468622in}{1.493166in}}%
\pgfpathlineto{\pgfqpoint{3.476655in}{1.501405in}}%
\pgfpathlineto{\pgfqpoint{3.484680in}{1.509774in}}%
\pgfpathlineto{\pgfqpoint{3.492697in}{1.518268in}}%
\pgfpathlineto{\pgfqpoint{3.500707in}{1.526883in}}%
\pgfpathlineto{\pgfqpoint{3.487354in}{1.529755in}}%
\pgfpathlineto{\pgfqpoint{3.474005in}{1.532797in}}%
\pgfpathlineto{\pgfqpoint{3.460661in}{1.536010in}}%
\pgfpathlineto{\pgfqpoint{3.447321in}{1.539395in}}%
\pgfpathlineto{\pgfqpoint{3.439294in}{1.531265in}}%
\pgfpathlineto{\pgfqpoint{3.431258in}{1.523264in}}%
\pgfpathlineto{\pgfqpoint{3.423215in}{1.515394in}}%
\pgfpathlineto{\pgfqpoint{3.415163in}{1.507662in}}%
\pgfpathclose%
\pgfusepath{fill}%
\end{pgfscope}%
\begin{pgfscope}%
\pgfpathrectangle{\pgfqpoint{1.254980in}{0.150000in}}{\pgfqpoint{5.490039in}{5.490039in}}%
\pgfusepath{clip}%
\pgfsetbuttcap%
\pgfsetroundjoin%
\definecolor{currentfill}{rgb}{0.162016,0.687316,0.499129}%
\pgfsetfillcolor{currentfill}%
\pgfsetfillopacity{0.700000}%
\pgfsetlinewidth{0.000000pt}%
\definecolor{currentstroke}{rgb}{0.000000,0.000000,0.000000}%
\pgfsetstrokecolor{currentstroke}%
\pgfsetdash{}{0pt}%
\pgfpathmoveto{\pgfqpoint{5.391783in}{3.091032in}}%
\pgfpathlineto{\pgfqpoint{5.406033in}{3.104298in}}%
\pgfpathlineto{\pgfqpoint{5.420302in}{3.117725in}}%
\pgfpathlineto{\pgfqpoint{5.434591in}{3.131314in}}%
\pgfpathlineto{\pgfqpoint{5.448901in}{3.145065in}}%
\pgfpathlineto{\pgfqpoint{5.456098in}{3.149113in}}%
\pgfpathlineto{\pgfqpoint{5.463286in}{3.153047in}}%
\pgfpathlineto{\pgfqpoint{5.470464in}{3.156871in}}%
\pgfpathlineto{\pgfqpoint{5.477633in}{3.160588in}}%
\pgfpathlineto{\pgfqpoint{5.463341in}{3.147130in}}%
\pgfpathlineto{\pgfqpoint{5.449068in}{3.133832in}}%
\pgfpathlineto{\pgfqpoint{5.434816in}{3.120696in}}%
\pgfpathlineto{\pgfqpoint{5.420583in}{3.107720in}}%
\pgfpathlineto{\pgfqpoint{5.413396in}{3.103701in}}%
\pgfpathlineto{\pgfqpoint{5.406201in}{3.099582in}}%
\pgfpathlineto{\pgfqpoint{5.398997in}{3.095360in}}%
\pgfpathlineto{\pgfqpoint{5.391783in}{3.091032in}}%
\pgfpathclose%
\pgfusepath{fill}%
\end{pgfscope}%
\begin{pgfscope}%
\pgfpathrectangle{\pgfqpoint{1.254980in}{0.150000in}}{\pgfqpoint{5.490039in}{5.490039in}}%
\pgfusepath{clip}%
\pgfsetbuttcap%
\pgfsetroundjoin%
\definecolor{currentfill}{rgb}{0.120092,0.600104,0.542530}%
\pgfsetfillcolor{currentfill}%
\pgfsetfillopacity{0.700000}%
\pgfsetlinewidth{0.000000pt}%
\definecolor{currentstroke}{rgb}{0.000000,0.000000,0.000000}%
\pgfsetstrokecolor{currentstroke}%
\pgfsetdash{}{0pt}%
\pgfpathmoveto{\pgfqpoint{5.105040in}{2.848088in}}%
\pgfpathlineto{\pgfqpoint{5.119110in}{2.860267in}}%
\pgfpathlineto{\pgfqpoint{5.133198in}{2.872608in}}%
\pgfpathlineto{\pgfqpoint{5.147305in}{2.885110in}}%
\pgfpathlineto{\pgfqpoint{5.161430in}{2.897775in}}%
\pgfpathlineto{\pgfqpoint{5.168810in}{2.904504in}}%
\pgfpathlineto{\pgfqpoint{5.176182in}{2.911100in}}%
\pgfpathlineto{\pgfqpoint{5.183545in}{2.917564in}}%
\pgfpathlineto{\pgfqpoint{5.190899in}{2.923899in}}%
\pgfpathlineto{\pgfqpoint{5.176783in}{2.911401in}}%
\pgfpathlineto{\pgfqpoint{5.162686in}{2.899065in}}%
\pgfpathlineto{\pgfqpoint{5.148608in}{2.886890in}}%
\pgfpathlineto{\pgfqpoint{5.134548in}{2.874875in}}%
\pgfpathlineto{\pgfqpoint{5.127183in}{2.868364in}}%
\pgfpathlineto{\pgfqpoint{5.119810in}{2.861730in}}%
\pgfpathlineto{\pgfqpoint{5.112429in}{2.854972in}}%
\pgfpathlineto{\pgfqpoint{5.105040in}{2.848088in}}%
\pgfpathclose%
\pgfusepath{fill}%
\end{pgfscope}%
\begin{pgfscope}%
\pgfpathrectangle{\pgfqpoint{1.254980in}{0.150000in}}{\pgfqpoint{5.490039in}{5.490039in}}%
\pgfusepath{clip}%
\pgfsetbuttcap%
\pgfsetroundjoin%
\definecolor{currentfill}{rgb}{0.280267,0.073417,0.397163}%
\pgfsetfillcolor{currentfill}%
\pgfsetfillopacity{0.700000}%
\pgfsetlinewidth{0.000000pt}%
\definecolor{currentstroke}{rgb}{0.000000,0.000000,0.000000}%
\pgfsetstrokecolor{currentstroke}%
\pgfsetdash{}{0pt}%
\pgfpathmoveto{\pgfqpoint{2.942496in}{1.655025in}}%
\pgfpathlineto{\pgfqpoint{2.955905in}{1.644383in}}%
\pgfpathlineto{\pgfqpoint{2.969311in}{1.633935in}}%
\pgfpathlineto{\pgfqpoint{2.982717in}{1.623681in}}%
\pgfpathlineto{\pgfqpoint{2.996121in}{1.613619in}}%
\pgfpathlineto{\pgfqpoint{3.004462in}{1.615728in}}%
\pgfpathlineto{\pgfqpoint{3.012791in}{1.618082in}}%
\pgfpathlineto{\pgfqpoint{3.021106in}{1.620676in}}%
\pgfpathlineto{\pgfqpoint{3.029408in}{1.623504in}}%
\pgfpathlineto{\pgfqpoint{3.016038in}{1.632978in}}%
\pgfpathlineto{\pgfqpoint{3.002668in}{1.642644in}}%
\pgfpathlineto{\pgfqpoint{2.989297in}{1.652502in}}%
\pgfpathlineto{\pgfqpoint{2.975924in}{1.662555in}}%
\pgfpathlineto{\pgfqpoint{2.967588in}{1.660304in}}%
\pgfpathlineto{\pgfqpoint{2.959238in}{1.658295in}}%
\pgfpathlineto{\pgfqpoint{2.950874in}{1.656533in}}%
\pgfpathlineto{\pgfqpoint{2.942496in}{1.655025in}}%
\pgfpathclose%
\pgfusepath{fill}%
\end{pgfscope}%
\begin{pgfscope}%
\pgfpathrectangle{\pgfqpoint{1.254980in}{0.150000in}}{\pgfqpoint{5.490039in}{5.490039in}}%
\pgfusepath{clip}%
\pgfsetbuttcap%
\pgfsetroundjoin%
\definecolor{currentfill}{rgb}{0.241237,0.296485,0.539709}%
\pgfsetfillcolor{currentfill}%
\pgfsetfillopacity{0.700000}%
\pgfsetlinewidth{0.000000pt}%
\definecolor{currentstroke}{rgb}{0.000000,0.000000,0.000000}%
\pgfsetstrokecolor{currentstroke}%
\pgfsetdash{}{0pt}%
\pgfpathmoveto{\pgfqpoint{4.298690in}{2.055659in}}%
\pgfpathlineto{\pgfqpoint{4.312300in}{2.062305in}}%
\pgfpathlineto{\pgfqpoint{4.325924in}{2.069112in}}%
\pgfpathlineto{\pgfqpoint{4.339560in}{2.076080in}}%
\pgfpathlineto{\pgfqpoint{4.353209in}{2.083209in}}%
\pgfpathlineto{\pgfqpoint{4.360937in}{2.095613in}}%
\pgfpathlineto{\pgfqpoint{4.368660in}{2.107934in}}%
\pgfpathlineto{\pgfqpoint{4.376378in}{2.120170in}}%
\pgfpathlineto{\pgfqpoint{4.384092in}{2.132321in}}%
\pgfpathlineto{\pgfqpoint{4.370444in}{2.124999in}}%
\pgfpathlineto{\pgfqpoint{4.356809in}{2.117838in}}%
\pgfpathlineto{\pgfqpoint{4.343187in}{2.110838in}}%
\pgfpathlineto{\pgfqpoint{4.329578in}{2.104000in}}%
\pgfpathlineto{\pgfqpoint{4.321863in}{2.092031in}}%
\pgfpathlineto{\pgfqpoint{4.314144in}{2.079983in}}%
\pgfpathlineto{\pgfqpoint{4.306419in}{2.067859in}}%
\pgfpathlineto{\pgfqpoint{4.298690in}{2.055659in}}%
\pgfpathclose%
\pgfusepath{fill}%
\end{pgfscope}%
\begin{pgfscope}%
\pgfpathrectangle{\pgfqpoint{1.254980in}{0.150000in}}{\pgfqpoint{5.490039in}{5.490039in}}%
\pgfusepath{clip}%
\pgfsetbuttcap%
\pgfsetroundjoin%
\definecolor{currentfill}{rgb}{0.175841,0.441290,0.557685}%
\pgfsetfillcolor{currentfill}%
\pgfsetfillopacity{0.700000}%
\pgfsetlinewidth{0.000000pt}%
\definecolor{currentstroke}{rgb}{0.000000,0.000000,0.000000}%
\pgfsetstrokecolor{currentstroke}%
\pgfsetdash{}{0pt}%
\pgfpathmoveto{\pgfqpoint{2.237908in}{2.510586in}}%
\pgfpathlineto{\pgfqpoint{2.251696in}{2.487760in}}%
\pgfpathlineto{\pgfqpoint{2.265471in}{2.465222in}}%
\pgfpathlineto{\pgfqpoint{2.279233in}{2.442967in}}%
\pgfpathlineto{\pgfqpoint{2.292981in}{2.420993in}}%
\pgfpathlineto{\pgfqpoint{2.301938in}{2.415060in}}%
\pgfpathlineto{\pgfqpoint{2.310872in}{2.409478in}}%
\pgfpathlineto{\pgfqpoint{2.319783in}{2.404241in}}%
\pgfpathlineto{\pgfqpoint{2.328672in}{2.399343in}}%
\pgfpathlineto{\pgfqpoint{2.314983in}{2.420682in}}%
\pgfpathlineto{\pgfqpoint{2.301281in}{2.442301in}}%
\pgfpathlineto{\pgfqpoint{2.287567in}{2.464202in}}%
\pgfpathlineto{\pgfqpoint{2.273839in}{2.486388in}}%
\pgfpathlineto{\pgfqpoint{2.264892in}{2.491910in}}%
\pgfpathlineto{\pgfqpoint{2.255921in}{2.497779in}}%
\pgfpathlineto{\pgfqpoint{2.246927in}{2.504002in}}%
\pgfpathlineto{\pgfqpoint{2.237908in}{2.510586in}}%
\pgfpathclose%
\pgfusepath{fill}%
\end{pgfscope}%
\begin{pgfscope}%
\pgfpathrectangle{\pgfqpoint{1.254980in}{0.150000in}}{\pgfqpoint{5.490039in}{5.490039in}}%
\pgfusepath{clip}%
\pgfsetbuttcap%
\pgfsetroundjoin%
\definecolor{currentfill}{rgb}{0.273006,0.204520,0.501721}%
\pgfsetfillcolor{currentfill}%
\pgfsetfillopacity{0.700000}%
\pgfsetlinewidth{0.000000pt}%
\definecolor{currentstroke}{rgb}{0.000000,0.000000,0.000000}%
\pgfsetstrokecolor{currentstroke}%
\pgfsetdash{}{0pt}%
\pgfpathmoveto{\pgfqpoint{4.097080in}{1.862688in}}%
\pgfpathlineto{\pgfqpoint{4.110602in}{1.867297in}}%
\pgfpathlineto{\pgfqpoint{4.124135in}{1.872066in}}%
\pgfpathlineto{\pgfqpoint{4.137679in}{1.876997in}}%
\pgfpathlineto{\pgfqpoint{4.151234in}{1.882089in}}%
\pgfpathlineto{\pgfqpoint{4.159021in}{1.894753in}}%
\pgfpathlineto{\pgfqpoint{4.166803in}{1.907374in}}%
\pgfpathlineto{\pgfqpoint{4.174581in}{1.919949in}}%
\pgfpathlineto{\pgfqpoint{4.182354in}{1.932475in}}%
\pgfpathlineto{\pgfqpoint{4.168801in}{1.927105in}}%
\pgfpathlineto{\pgfqpoint{4.155260in}{1.921897in}}%
\pgfpathlineto{\pgfqpoint{4.141729in}{1.916850in}}%
\pgfpathlineto{\pgfqpoint{4.128210in}{1.911965in}}%
\pgfpathlineto{\pgfqpoint{4.120435in}{1.899705in}}%
\pgfpathlineto{\pgfqpoint{4.112654in}{1.887404in}}%
\pgfpathlineto{\pgfqpoint{4.104869in}{1.875064in}}%
\pgfpathlineto{\pgfqpoint{4.097080in}{1.862688in}}%
\pgfpathclose%
\pgfusepath{fill}%
\end{pgfscope}%
\begin{pgfscope}%
\pgfpathrectangle{\pgfqpoint{1.254980in}{0.150000in}}{\pgfqpoint{5.490039in}{5.490039in}}%
\pgfusepath{clip}%
\pgfsetbuttcap%
\pgfsetroundjoin%
\definecolor{currentfill}{rgb}{0.201239,0.383670,0.554294}%
\pgfsetfillcolor{currentfill}%
\pgfsetfillopacity{0.700000}%
\pgfsetlinewidth{0.000000pt}%
\definecolor{currentstroke}{rgb}{0.000000,0.000000,0.000000}%
\pgfsetstrokecolor{currentstroke}%
\pgfsetdash{}{0pt}%
\pgfpathmoveto{\pgfqpoint{4.500331in}{2.258383in}}%
\pgfpathlineto{\pgfqpoint{4.514046in}{2.266810in}}%
\pgfpathlineto{\pgfqpoint{4.527775in}{2.275398in}}%
\pgfpathlineto{\pgfqpoint{4.541519in}{2.284148in}}%
\pgfpathlineto{\pgfqpoint{4.555277in}{2.293058in}}%
\pgfpathlineto{\pgfqpoint{4.562941in}{2.304616in}}%
\pgfpathlineto{\pgfqpoint{4.570599in}{2.316062in}}%
\pgfpathlineto{\pgfqpoint{4.578251in}{2.327396in}}%
\pgfpathlineto{\pgfqpoint{4.585897in}{2.338617in}}%
\pgfpathlineto{\pgfqpoint{4.572140in}{2.329600in}}%
\pgfpathlineto{\pgfqpoint{4.558397in}{2.320744in}}%
\pgfpathlineto{\pgfqpoint{4.544670in}{2.312050in}}%
\pgfpathlineto{\pgfqpoint{4.530956in}{2.303516in}}%
\pgfpathlineto{\pgfqpoint{4.523308in}{2.292391in}}%
\pgfpathlineto{\pgfqpoint{4.515655in}{2.281159in}}%
\pgfpathlineto{\pgfqpoint{4.507996in}{2.269823in}}%
\pgfpathlineto{\pgfqpoint{4.500331in}{2.258383in}}%
\pgfpathclose%
\pgfusepath{fill}%
\end{pgfscope}%
\begin{pgfscope}%
\pgfpathrectangle{\pgfqpoint{1.254980in}{0.150000in}}{\pgfqpoint{5.490039in}{5.490039in}}%
\pgfusepath{clip}%
\pgfsetbuttcap%
\pgfsetroundjoin%
\definecolor{currentfill}{rgb}{0.139147,0.533812,0.555298}%
\pgfsetfillcolor{currentfill}%
\pgfsetfillopacity{0.700000}%
\pgfsetlinewidth{0.000000pt}%
\definecolor{currentstroke}{rgb}{0.000000,0.000000,0.000000}%
\pgfsetstrokecolor{currentstroke}%
\pgfsetdash{}{0pt}%
\pgfpathmoveto{\pgfqpoint{4.903644in}{2.660860in}}%
\pgfpathlineto{\pgfqpoint{4.917594in}{2.672060in}}%
\pgfpathlineto{\pgfqpoint{4.931560in}{2.683422in}}%
\pgfpathlineto{\pgfqpoint{4.945544in}{2.694945in}}%
\pgfpathlineto{\pgfqpoint{4.959546in}{2.706629in}}%
\pgfpathlineto{\pgfqpoint{4.967038in}{2.715224in}}%
\pgfpathlineto{\pgfqpoint{4.974523in}{2.723681in}}%
\pgfpathlineto{\pgfqpoint{4.982000in}{2.732002in}}%
\pgfpathlineto{\pgfqpoint{4.989469in}{2.740188in}}%
\pgfpathlineto{\pgfqpoint{4.975474in}{2.728577in}}%
\pgfpathlineto{\pgfqpoint{4.961495in}{2.717127in}}%
\pgfpathlineto{\pgfqpoint{4.947534in}{2.705839in}}%
\pgfpathlineto{\pgfqpoint{4.933590in}{2.694713in}}%
\pgfpathlineto{\pgfqpoint{4.926115in}{2.686442in}}%
\pgfpathlineto{\pgfqpoint{4.918632in}{2.678044in}}%
\pgfpathlineto{\pgfqpoint{4.911142in}{2.669517in}}%
\pgfpathlineto{\pgfqpoint{4.903644in}{2.660860in}}%
\pgfpathclose%
\pgfusepath{fill}%
\end{pgfscope}%
\begin{pgfscope}%
\pgfpathrectangle{\pgfqpoint{1.254980in}{0.150000in}}{\pgfqpoint{5.490039in}{5.490039in}}%
\pgfusepath{clip}%
\pgfsetbuttcap%
\pgfsetroundjoin%
\definecolor{currentfill}{rgb}{0.278791,0.062145,0.386592}%
\pgfsetfillcolor{currentfill}%
\pgfsetfillopacity{0.700000}%
\pgfsetlinewidth{0.000000pt}%
\definecolor{currentstroke}{rgb}{0.000000,0.000000,0.000000}%
\pgfsetstrokecolor{currentstroke}%
\pgfsetdash{}{0pt}%
\pgfpathmoveto{\pgfqpoint{3.724876in}{1.589134in}}%
\pgfpathlineto{\pgfqpoint{3.738282in}{1.589342in}}%
\pgfpathlineto{\pgfqpoint{3.751695in}{1.589714in}}%
\pgfpathlineto{\pgfqpoint{3.765116in}{1.590250in}}%
\pgfpathlineto{\pgfqpoint{3.778544in}{1.590950in}}%
\pgfpathlineto{\pgfqpoint{3.786446in}{1.602142in}}%
\pgfpathlineto{\pgfqpoint{3.794344in}{1.613383in}}%
\pgfpathlineto{\pgfqpoint{3.802235in}{1.624670in}}%
\pgfpathlineto{\pgfqpoint{3.810122in}{1.635998in}}%
\pgfpathlineto{\pgfqpoint{3.796702in}{1.634884in}}%
\pgfpathlineto{\pgfqpoint{3.783291in}{1.633934in}}%
\pgfpathlineto{\pgfqpoint{3.769887in}{1.633148in}}%
\pgfpathlineto{\pgfqpoint{3.756491in}{1.632526in}}%
\pgfpathlineto{\pgfqpoint{3.748596in}{1.621601in}}%
\pgfpathlineto{\pgfqpoint{3.740695in}{1.610725in}}%
\pgfpathlineto{\pgfqpoint{3.732788in}{1.599902in}}%
\pgfpathlineto{\pgfqpoint{3.724876in}{1.589134in}}%
\pgfpathclose%
\pgfusepath{fill}%
\end{pgfscope}%
\begin{pgfscope}%
\pgfpathrectangle{\pgfqpoint{1.254980in}{0.150000in}}{\pgfqpoint{5.490039in}{5.490039in}}%
\pgfusepath{clip}%
\pgfsetbuttcap%
\pgfsetroundjoin%
\definecolor{currentfill}{rgb}{0.168126,0.459988,0.558082}%
\pgfsetfillcolor{currentfill}%
\pgfsetfillopacity{0.700000}%
\pgfsetlinewidth{0.000000pt}%
\definecolor{currentstroke}{rgb}{0.000000,0.000000,0.000000}%
\pgfsetstrokecolor{currentstroke}%
\pgfsetdash{}{0pt}%
\pgfpathmoveto{\pgfqpoint{4.702015in}{2.462447in}}%
\pgfpathlineto{\pgfqpoint{4.715845in}{2.472394in}}%
\pgfpathlineto{\pgfqpoint{4.729690in}{2.482503in}}%
\pgfpathlineto{\pgfqpoint{4.743552in}{2.492773in}}%
\pgfpathlineto{\pgfqpoint{4.757429in}{2.503204in}}%
\pgfpathlineto{\pgfqpoint{4.765016in}{2.513451in}}%
\pgfpathlineto{\pgfqpoint{4.772595in}{2.523567in}}%
\pgfpathlineto{\pgfqpoint{4.780168in}{2.533555in}}%
\pgfpathlineto{\pgfqpoint{4.787734in}{2.543412in}}%
\pgfpathlineto{\pgfqpoint{4.773860in}{2.532964in}}%
\pgfpathlineto{\pgfqpoint{4.760001in}{2.522677in}}%
\pgfpathlineto{\pgfqpoint{4.746159in}{2.512551in}}%
\pgfpathlineto{\pgfqpoint{4.732332in}{2.502586in}}%
\pgfpathlineto{\pgfqpoint{4.724763in}{2.492735in}}%
\pgfpathlineto{\pgfqpoint{4.717187in}{2.482761in}}%
\pgfpathlineto{\pgfqpoint{4.709604in}{2.472665in}}%
\pgfpathlineto{\pgfqpoint{4.702015in}{2.462447in}}%
\pgfpathclose%
\pgfusepath{fill}%
\end{pgfscope}%
\begin{pgfscope}%
\pgfpathrectangle{\pgfqpoint{1.254980in}{0.150000in}}{\pgfqpoint{5.490039in}{5.490039in}}%
\pgfusepath{clip}%
\pgfsetbuttcap%
\pgfsetroundjoin%
\definecolor{currentfill}{rgb}{0.274952,0.037752,0.364543}%
\pgfsetfillcolor{currentfill}%
\pgfsetfillopacity{0.700000}%
\pgfsetlinewidth{0.000000pt}%
\definecolor{currentstroke}{rgb}{0.000000,0.000000,0.000000}%
\pgfsetstrokecolor{currentstroke}%
\pgfsetdash{}{0pt}%
\pgfpathmoveto{\pgfqpoint{3.639574in}{1.549290in}}%
\pgfpathlineto{\pgfqpoint{3.652963in}{1.548397in}}%
\pgfpathlineto{\pgfqpoint{3.666359in}{1.547668in}}%
\pgfpathlineto{\pgfqpoint{3.679762in}{1.547105in}}%
\pgfpathlineto{\pgfqpoint{3.693171in}{1.546707in}}%
\pgfpathlineto{\pgfqpoint{3.701106in}{1.557209in}}%
\pgfpathlineto{\pgfqpoint{3.709035in}{1.567784in}}%
\pgfpathlineto{\pgfqpoint{3.716959in}{1.578427in}}%
\pgfpathlineto{\pgfqpoint{3.724876in}{1.589134in}}%
\pgfpathlineto{\pgfqpoint{3.711478in}{1.589091in}}%
\pgfpathlineto{\pgfqpoint{3.698087in}{1.589212in}}%
\pgfpathlineto{\pgfqpoint{3.684702in}{1.589500in}}%
\pgfpathlineto{\pgfqpoint{3.671325in}{1.589952in}}%
\pgfpathlineto{\pgfqpoint{3.663396in}{1.579676in}}%
\pgfpathlineto{\pgfqpoint{3.655461in}{1.569471in}}%
\pgfpathlineto{\pgfqpoint{3.647520in}{1.559341in}}%
\pgfpathlineto{\pgfqpoint{3.639574in}{1.549290in}}%
\pgfpathclose%
\pgfusepath{fill}%
\end{pgfscope}%
\begin{pgfscope}%
\pgfpathrectangle{\pgfqpoint{1.254980in}{0.150000in}}{\pgfqpoint{5.490039in}{5.490039in}}%
\pgfusepath{clip}%
\pgfsetbuttcap%
\pgfsetroundjoin%
\definecolor{currentfill}{rgb}{0.202219,0.715272,0.476084}%
\pgfsetfillcolor{currentfill}%
\pgfsetfillopacity{0.700000}%
\pgfsetlinewidth{0.000000pt}%
\definecolor{currentstroke}{rgb}{0.000000,0.000000,0.000000}%
\pgfsetstrokecolor{currentstroke}%
\pgfsetdash{}{0pt}%
\pgfpathmoveto{\pgfqpoint{5.477633in}{3.160588in}}%
\pgfpathlineto{\pgfqpoint{5.491946in}{3.174208in}}%
\pgfpathlineto{\pgfqpoint{5.506279in}{3.187990in}}%
\pgfpathlineto{\pgfqpoint{5.520632in}{3.201933in}}%
\pgfpathlineto{\pgfqpoint{5.535006in}{3.216038in}}%
\pgfpathlineto{\pgfqpoint{5.542148in}{3.219339in}}%
\pgfpathlineto{\pgfqpoint{5.549281in}{3.222534in}}%
\pgfpathlineto{\pgfqpoint{5.556403in}{3.225625in}}%
\pgfpathlineto{\pgfqpoint{5.563517in}{3.228615in}}%
\pgfpathlineto{\pgfqpoint{5.549162in}{3.214835in}}%
\pgfpathlineto{\pgfqpoint{5.534827in}{3.201215in}}%
\pgfpathlineto{\pgfqpoint{5.520513in}{3.187756in}}%
\pgfpathlineto{\pgfqpoint{5.506219in}{3.174458in}}%
\pgfpathlineto{\pgfqpoint{5.499086in}{3.171133in}}%
\pgfpathlineto{\pgfqpoint{5.491944in}{3.167716in}}%
\pgfpathlineto{\pgfqpoint{5.484793in}{3.164202in}}%
\pgfpathlineto{\pgfqpoint{5.477633in}{3.160588in}}%
\pgfpathclose%
\pgfusepath{fill}%
\end{pgfscope}%
\begin{pgfscope}%
\pgfpathrectangle{\pgfqpoint{1.254980in}{0.150000in}}{\pgfqpoint{5.490039in}{5.490039in}}%
\pgfusepath{clip}%
\pgfsetbuttcap%
\pgfsetroundjoin%
\definecolor{currentfill}{rgb}{0.281924,0.089666,0.412415}%
\pgfsetfillcolor{currentfill}%
\pgfsetfillopacity{0.700000}%
\pgfsetlinewidth{0.000000pt}%
\definecolor{currentstroke}{rgb}{0.000000,0.000000,0.000000}%
\pgfsetstrokecolor{currentstroke}%
\pgfsetdash{}{0pt}%
\pgfpathmoveto{\pgfqpoint{3.810122in}{1.635998in}}%
\pgfpathlineto{\pgfqpoint{3.823550in}{1.637275in}}%
\pgfpathlineto{\pgfqpoint{3.836985in}{1.638716in}}%
\pgfpathlineto{\pgfqpoint{3.850430in}{1.640319in}}%
\pgfpathlineto{\pgfqpoint{3.863883in}{1.642086in}}%
\pgfpathlineto{\pgfqpoint{3.871757in}{1.653849in}}%
\pgfpathlineto{\pgfqpoint{3.879626in}{1.665639in}}%
\pgfpathlineto{\pgfqpoint{3.887489in}{1.677453in}}%
\pgfpathlineto{\pgfqpoint{3.895349in}{1.689287in}}%
\pgfpathlineto{\pgfqpoint{3.881902in}{1.687134in}}%
\pgfpathlineto{\pgfqpoint{3.868465in}{1.685143in}}%
\pgfpathlineto{\pgfqpoint{3.855037in}{1.683315in}}%
\pgfpathlineto{\pgfqpoint{3.841617in}{1.681651in}}%
\pgfpathlineto{\pgfqpoint{3.833751in}{1.670194in}}%
\pgfpathlineto{\pgfqpoint{3.825879in}{1.658764in}}%
\pgfpathlineto{\pgfqpoint{3.818003in}{1.647364in}}%
\pgfpathlineto{\pgfqpoint{3.810122in}{1.635998in}}%
\pgfpathclose%
\pgfusepath{fill}%
\end{pgfscope}%
\begin{pgfscope}%
\pgfpathrectangle{\pgfqpoint{1.254980in}{0.150000in}}{\pgfqpoint{5.490039in}{5.490039in}}%
\pgfusepath{clip}%
\pgfsetbuttcap%
\pgfsetroundjoin%
\definecolor{currentfill}{rgb}{0.267004,0.004874,0.329415}%
\pgfsetfillcolor{currentfill}%
\pgfsetfillopacity{0.700000}%
\pgfsetlinewidth{0.000000pt}%
\definecolor{currentstroke}{rgb}{0.000000,0.000000,0.000000}%
\pgfsetstrokecolor{currentstroke}%
\pgfsetdash{}{0pt}%
\pgfpathmoveto{\pgfqpoint{3.329393in}{1.497548in}}%
\pgfpathlineto{\pgfqpoint{3.342758in}{1.492452in}}%
\pgfpathlineto{\pgfqpoint{3.356127in}{1.487530in}}%
\pgfpathlineto{\pgfqpoint{3.369499in}{1.482781in}}%
\pgfpathlineto{\pgfqpoint{3.382874in}{1.478204in}}%
\pgfpathlineto{\pgfqpoint{3.390959in}{1.485338in}}%
\pgfpathlineto{\pgfqpoint{3.399036in}{1.492629in}}%
\pgfpathlineto{\pgfqpoint{3.407104in}{1.500072in}}%
\pgfpathlineto{\pgfqpoint{3.415163in}{1.507662in}}%
\pgfpathlineto{\pgfqpoint{3.401809in}{1.511714in}}%
\pgfpathlineto{\pgfqpoint{3.388458in}{1.515939in}}%
\pgfpathlineto{\pgfqpoint{3.375111in}{1.520336in}}%
\pgfpathlineto{\pgfqpoint{3.361767in}{1.524908in}}%
\pgfpathlineto{\pgfqpoint{3.353687in}{1.517831in}}%
\pgfpathlineto{\pgfqpoint{3.345598in}{1.510909in}}%
\pgfpathlineto{\pgfqpoint{3.337500in}{1.504146in}}%
\pgfpathlineto{\pgfqpoint{3.329393in}{1.497548in}}%
\pgfpathclose%
\pgfusepath{fill}%
\end{pgfscope}%
\begin{pgfscope}%
\pgfpathrectangle{\pgfqpoint{1.254980in}{0.150000in}}{\pgfqpoint{5.490039in}{5.490039in}}%
\pgfusepath{clip}%
\pgfsetbuttcap%
\pgfsetroundjoin%
\definecolor{currentfill}{rgb}{0.271305,0.019942,0.347269}%
\pgfsetfillcolor{currentfill}%
\pgfsetfillopacity{0.700000}%
\pgfsetlinewidth{0.000000pt}%
\definecolor{currentstroke}{rgb}{0.000000,0.000000,0.000000}%
\pgfsetstrokecolor{currentstroke}%
\pgfsetdash{}{0pt}%
\pgfpathmoveto{\pgfqpoint{3.554171in}{1.517086in}}%
\pgfpathlineto{\pgfqpoint{3.567551in}{1.515057in}}%
\pgfpathlineto{\pgfqpoint{3.580936in}{1.513196in}}%
\pgfpathlineto{\pgfqpoint{3.594326in}{1.511501in}}%
\pgfpathlineto{\pgfqpoint{3.607723in}{1.509972in}}%
\pgfpathlineto{\pgfqpoint{3.615695in}{1.519660in}}%
\pgfpathlineto{\pgfqpoint{3.623661in}{1.529446in}}%
\pgfpathlineto{\pgfqpoint{3.631621in}{1.539324in}}%
\pgfpathlineto{\pgfqpoint{3.639574in}{1.549290in}}%
\pgfpathlineto{\pgfqpoint{3.626191in}{1.550350in}}%
\pgfpathlineto{\pgfqpoint{3.612814in}{1.551576in}}%
\pgfpathlineto{\pgfqpoint{3.599443in}{1.552970in}}%
\pgfpathlineto{\pgfqpoint{3.586078in}{1.554530in}}%
\pgfpathlineto{\pgfqpoint{3.578111in}{1.545022in}}%
\pgfpathlineto{\pgfqpoint{3.570138in}{1.535609in}}%
\pgfpathlineto{\pgfqpoint{3.562158in}{1.526295in}}%
\pgfpathlineto{\pgfqpoint{3.554171in}{1.517086in}}%
\pgfpathclose%
\pgfusepath{fill}%
\end{pgfscope}%
\begin{pgfscope}%
\pgfpathrectangle{\pgfqpoint{1.254980in}{0.150000in}}{\pgfqpoint{5.490039in}{5.490039in}}%
\pgfusepath{clip}%
\pgfsetbuttcap%
\pgfsetroundjoin%
\definecolor{currentfill}{rgb}{0.277941,0.056324,0.381191}%
\pgfsetfillcolor{currentfill}%
\pgfsetfillopacity{0.700000}%
\pgfsetlinewidth{0.000000pt}%
\definecolor{currentstroke}{rgb}{0.000000,0.000000,0.000000}%
\pgfsetstrokecolor{currentstroke}%
\pgfsetdash{}{0pt}%
\pgfpathmoveto{\pgfqpoint{2.996121in}{1.613619in}}%
\pgfpathlineto{\pgfqpoint{3.009523in}{1.603749in}}%
\pgfpathlineto{\pgfqpoint{3.022925in}{1.594068in}}%
\pgfpathlineto{\pgfqpoint{3.036326in}{1.584577in}}%
\pgfpathlineto{\pgfqpoint{3.049727in}{1.575274in}}%
\pgfpathlineto{\pgfqpoint{3.058034in}{1.577981in}}%
\pgfpathlineto{\pgfqpoint{3.066329in}{1.580926in}}%
\pgfpathlineto{\pgfqpoint{3.074610in}{1.584104in}}%
\pgfpathlineto{\pgfqpoint{3.082880in}{1.587508in}}%
\pgfpathlineto{\pgfqpoint{3.069512in}{1.596224in}}%
\pgfpathlineto{\pgfqpoint{3.056144in}{1.605128in}}%
\pgfpathlineto{\pgfqpoint{3.042776in}{1.614221in}}%
\pgfpathlineto{\pgfqpoint{3.029408in}{1.623504in}}%
\pgfpathlineto{\pgfqpoint{3.021106in}{1.620676in}}%
\pgfpathlineto{\pgfqpoint{3.012791in}{1.618082in}}%
\pgfpathlineto{\pgfqpoint{3.004462in}{1.615728in}}%
\pgfpathlineto{\pgfqpoint{2.996121in}{1.613619in}}%
\pgfpathclose%
\pgfusepath{fill}%
\end{pgfscope}%
\begin{pgfscope}%
\pgfpathrectangle{\pgfqpoint{1.254980in}{0.150000in}}{\pgfqpoint{5.490039in}{5.490039in}}%
\pgfusepath{clip}%
\pgfsetbuttcap%
\pgfsetroundjoin%
\definecolor{currentfill}{rgb}{0.269944,0.014625,0.341379}%
\pgfsetfillcolor{currentfill}%
\pgfsetfillopacity{0.700000}%
\pgfsetlinewidth{0.000000pt}%
\definecolor{currentstroke}{rgb}{0.000000,0.000000,0.000000}%
\pgfsetstrokecolor{currentstroke}%
\pgfsetdash{}{0pt}%
\pgfpathmoveto{\pgfqpoint{3.189833in}{1.524436in}}%
\pgfpathlineto{\pgfqpoint{3.203207in}{1.517372in}}%
\pgfpathlineto{\pgfqpoint{3.216581in}{1.510487in}}%
\pgfpathlineto{\pgfqpoint{3.229958in}{1.503780in}}%
\pgfpathlineto{\pgfqpoint{3.243336in}{1.497252in}}%
\pgfpathlineto{\pgfqpoint{3.251507in}{1.502575in}}%
\pgfpathlineto{\pgfqpoint{3.259667in}{1.508092in}}%
\pgfpathlineto{\pgfqpoint{3.267818in}{1.513795in}}%
\pgfpathlineto{\pgfqpoint{3.275958in}{1.519681in}}%
\pgfpathlineto{\pgfqpoint{3.262606in}{1.525656in}}%
\pgfpathlineto{\pgfqpoint{3.249256in}{1.531808in}}%
\pgfpathlineto{\pgfqpoint{3.235907in}{1.538139in}}%
\pgfpathlineto{\pgfqpoint{3.222561in}{1.544649in}}%
\pgfpathlineto{\pgfqpoint{3.214395in}{1.539306in}}%
\pgfpathlineto{\pgfqpoint{3.206218in}{1.534152in}}%
\pgfpathlineto{\pgfqpoint{3.198031in}{1.529194in}}%
\pgfpathlineto{\pgfqpoint{3.189833in}{1.524436in}}%
\pgfpathclose%
\pgfusepath{fill}%
\end{pgfscope}%
\begin{pgfscope}%
\pgfpathrectangle{\pgfqpoint{1.254980in}{0.150000in}}{\pgfqpoint{5.490039in}{5.490039in}}%
\pgfusepath{clip}%
\pgfsetbuttcap%
\pgfsetroundjoin%
\definecolor{currentfill}{rgb}{0.283229,0.120777,0.440584}%
\pgfsetfillcolor{currentfill}%
\pgfsetfillopacity{0.700000}%
\pgfsetlinewidth{0.000000pt}%
\definecolor{currentstroke}{rgb}{0.000000,0.000000,0.000000}%
\pgfsetstrokecolor{currentstroke}%
\pgfsetdash{}{0pt}%
\pgfpathmoveto{\pgfqpoint{3.895349in}{1.689287in}}%
\pgfpathlineto{\pgfqpoint{3.908804in}{1.691603in}}%
\pgfpathlineto{\pgfqpoint{3.922268in}{1.694081in}}%
\pgfpathlineto{\pgfqpoint{3.935741in}{1.696722in}}%
\pgfpathlineto{\pgfqpoint{3.949224in}{1.699524in}}%
\pgfpathlineto{\pgfqpoint{3.957073in}{1.711744in}}%
\pgfpathlineto{\pgfqpoint{3.964917in}{1.723971in}}%
\pgfpathlineto{\pgfqpoint{3.972756in}{1.736200in}}%
\pgfpathlineto{\pgfqpoint{3.980590in}{1.748429in}}%
\pgfpathlineto{\pgfqpoint{3.967112in}{1.745267in}}%
\pgfpathlineto{\pgfqpoint{3.953644in}{1.742266in}}%
\pgfpathlineto{\pgfqpoint{3.940186in}{1.739428in}}%
\pgfpathlineto{\pgfqpoint{3.926737in}{1.736753in}}%
\pgfpathlineto{\pgfqpoint{3.918897in}{1.724873in}}%
\pgfpathlineto{\pgfqpoint{3.911052in}{1.713000in}}%
\pgfpathlineto{\pgfqpoint{3.903203in}{1.701137in}}%
\pgfpathlineto{\pgfqpoint{3.895349in}{1.689287in}}%
\pgfpathclose%
\pgfusepath{fill}%
\end{pgfscope}%
\begin{pgfscope}%
\pgfpathrectangle{\pgfqpoint{1.254980in}{0.150000in}}{\pgfqpoint{5.490039in}{5.490039in}}%
\pgfusepath{clip}%
\pgfsetbuttcap%
\pgfsetroundjoin%
\definecolor{currentfill}{rgb}{0.122312,0.633153,0.530398}%
\pgfsetfillcolor{currentfill}%
\pgfsetfillopacity{0.700000}%
\pgfsetlinewidth{0.000000pt}%
\definecolor{currentstroke}{rgb}{0.000000,0.000000,0.000000}%
\pgfsetstrokecolor{currentstroke}%
\pgfsetdash{}{0pt}%
\pgfpathmoveto{\pgfqpoint{5.190899in}{2.923899in}}%
\pgfpathlineto{\pgfqpoint{5.205033in}{2.936558in}}%
\pgfpathlineto{\pgfqpoint{5.219186in}{2.949380in}}%
\pgfpathlineto{\pgfqpoint{5.233359in}{2.962363in}}%
\pgfpathlineto{\pgfqpoint{5.247550in}{2.975508in}}%
\pgfpathlineto{\pgfqpoint{5.254885in}{2.981529in}}%
\pgfpathlineto{\pgfqpoint{5.262211in}{2.987417in}}%
\pgfpathlineto{\pgfqpoint{5.269528in}{2.993176in}}%
\pgfpathlineto{\pgfqpoint{5.276836in}{2.998807in}}%
\pgfpathlineto{\pgfqpoint{5.262656in}{2.985860in}}%
\pgfpathlineto{\pgfqpoint{5.248495in}{2.973075in}}%
\pgfpathlineto{\pgfqpoint{5.234353in}{2.960451in}}%
\pgfpathlineto{\pgfqpoint{5.220230in}{2.947988in}}%
\pgfpathlineto{\pgfqpoint{5.212910in}{2.942149in}}%
\pgfpathlineto{\pgfqpoint{5.205582in}{2.936189in}}%
\pgfpathlineto{\pgfqpoint{5.198245in}{2.930106in}}%
\pgfpathlineto{\pgfqpoint{5.190899in}{2.923899in}}%
\pgfpathclose%
\pgfusepath{fill}%
\end{pgfscope}%
\begin{pgfscope}%
\pgfpathrectangle{\pgfqpoint{1.254980in}{0.150000in}}{\pgfqpoint{5.490039in}{5.490039in}}%
\pgfusepath{clip}%
\pgfsetbuttcap%
\pgfsetroundjoin%
\definecolor{currentfill}{rgb}{0.260571,0.246922,0.522828}%
\pgfsetfillcolor{currentfill}%
\pgfsetfillopacity{0.700000}%
\pgfsetlinewidth{0.000000pt}%
\definecolor{currentstroke}{rgb}{0.000000,0.000000,0.000000}%
\pgfsetstrokecolor{currentstroke}%
\pgfsetdash{}{0pt}%
\pgfpathmoveto{\pgfqpoint{4.182354in}{1.932475in}}%
\pgfpathlineto{\pgfqpoint{4.195919in}{1.938006in}}%
\pgfpathlineto{\pgfqpoint{4.209496in}{1.943698in}}%
\pgfpathlineto{\pgfqpoint{4.223084in}{1.949550in}}%
\pgfpathlineto{\pgfqpoint{4.236685in}{1.955564in}}%
\pgfpathlineto{\pgfqpoint{4.244452in}{1.968300in}}%
\pgfpathlineto{\pgfqpoint{4.252214in}{1.980977in}}%
\pgfpathlineto{\pgfqpoint{4.259972in}{1.993591in}}%
\pgfpathlineto{\pgfqpoint{4.267725in}{2.006141in}}%
\pgfpathlineto{\pgfqpoint{4.254126in}{1.999878in}}%
\pgfpathlineto{\pgfqpoint{4.240539in}{1.993775in}}%
\pgfpathlineto{\pgfqpoint{4.226964in}{1.987834in}}%
\pgfpathlineto{\pgfqpoint{4.213401in}{1.982053in}}%
\pgfpathlineto{\pgfqpoint{4.205646in}{1.969742in}}%
\pgfpathlineto{\pgfqpoint{4.197887in}{1.957374in}}%
\pgfpathlineto{\pgfqpoint{4.190123in}{1.944951in}}%
\pgfpathlineto{\pgfqpoint{4.182354in}{1.932475in}}%
\pgfpathclose%
\pgfusepath{fill}%
\end{pgfscope}%
\begin{pgfscope}%
\pgfpathrectangle{\pgfqpoint{1.254980in}{0.150000in}}{\pgfqpoint{5.490039in}{5.490039in}}%
\pgfusepath{clip}%
\pgfsetbuttcap%
\pgfsetroundjoin%
\definecolor{currentfill}{rgb}{0.160665,0.478540,0.558115}%
\pgfsetfillcolor{currentfill}%
\pgfsetfillopacity{0.700000}%
\pgfsetlinewidth{0.000000pt}%
\definecolor{currentstroke}{rgb}{0.000000,0.000000,0.000000}%
\pgfsetstrokecolor{currentstroke}%
\pgfsetdash{}{0pt}%
\pgfpathmoveto{\pgfqpoint{2.182607in}{2.604809in}}%
\pgfpathlineto{\pgfqpoint{2.196455in}{2.580809in}}%
\pgfpathlineto{\pgfqpoint{2.210287in}{2.557107in}}%
\pgfpathlineto{\pgfqpoint{2.224105in}{2.533700in}}%
\pgfpathlineto{\pgfqpoint{2.237908in}{2.510586in}}%
\pgfpathlineto{\pgfqpoint{2.246927in}{2.504002in}}%
\pgfpathlineto{\pgfqpoint{2.255921in}{2.497779in}}%
\pgfpathlineto{\pgfqpoint{2.264892in}{2.491910in}}%
\pgfpathlineto{\pgfqpoint{2.273839in}{2.486388in}}%
\pgfpathlineto{\pgfqpoint{2.260098in}{2.508862in}}%
\pgfpathlineto{\pgfqpoint{2.246343in}{2.531626in}}%
\pgfpathlineto{\pgfqpoint{2.232573in}{2.554684in}}%
\pgfpathlineto{\pgfqpoint{2.218789in}{2.578038in}}%
\pgfpathlineto{\pgfqpoint{2.209781in}{2.584190in}}%
\pgfpathlineto{\pgfqpoint{2.200748in}{2.590698in}}%
\pgfpathlineto{\pgfqpoint{2.191691in}{2.597569in}}%
\pgfpathlineto{\pgfqpoint{2.182607in}{2.604809in}}%
\pgfpathclose%
\pgfusepath{fill}%
\end{pgfscope}%
\begin{pgfscope}%
\pgfpathrectangle{\pgfqpoint{1.254980in}{0.150000in}}{\pgfqpoint{5.490039in}{5.490039in}}%
\pgfusepath{clip}%
\pgfsetbuttcap%
\pgfsetroundjoin%
\definecolor{currentfill}{rgb}{0.223925,0.334994,0.548053}%
\pgfsetfillcolor{currentfill}%
\pgfsetfillopacity{0.700000}%
\pgfsetlinewidth{0.000000pt}%
\definecolor{currentstroke}{rgb}{0.000000,0.000000,0.000000}%
\pgfsetstrokecolor{currentstroke}%
\pgfsetdash{}{0pt}%
\pgfpathmoveto{\pgfqpoint{4.384092in}{2.132321in}}%
\pgfpathlineto{\pgfqpoint{4.397753in}{2.139804in}}%
\pgfpathlineto{\pgfqpoint{4.411428in}{2.147449in}}%
\pgfpathlineto{\pgfqpoint{4.425117in}{2.155254in}}%
\pgfpathlineto{\pgfqpoint{4.438819in}{2.163220in}}%
\pgfpathlineto{\pgfqpoint{4.446527in}{2.175459in}}%
\pgfpathlineto{\pgfqpoint{4.454229in}{2.187602in}}%
\pgfpathlineto{\pgfqpoint{4.461926in}{2.199649in}}%
\pgfpathlineto{\pgfqpoint{4.469618in}{2.211596in}}%
\pgfpathlineto{\pgfqpoint{4.455916in}{2.203466in}}%
\pgfpathlineto{\pgfqpoint{4.442228in}{2.195496in}}%
\pgfpathlineto{\pgfqpoint{4.428554in}{2.187688in}}%
\pgfpathlineto{\pgfqpoint{4.414894in}{2.180040in}}%
\pgfpathlineto{\pgfqpoint{4.407201in}{2.168246in}}%
\pgfpathlineto{\pgfqpoint{4.399503in}{2.156361in}}%
\pgfpathlineto{\pgfqpoint{4.391800in}{2.144385in}}%
\pgfpathlineto{\pgfqpoint{4.384092in}{2.132321in}}%
\pgfpathclose%
\pgfusepath{fill}%
\end{pgfscope}%
\begin{pgfscope}%
\pgfpathrectangle{\pgfqpoint{1.254980in}{0.150000in}}{\pgfqpoint{5.490039in}{5.490039in}}%
\pgfusepath{clip}%
\pgfsetbuttcap%
\pgfsetroundjoin%
\definecolor{currentfill}{rgb}{0.246070,0.738910,0.452024}%
\pgfsetfillcolor{currentfill}%
\pgfsetfillopacity{0.700000}%
\pgfsetlinewidth{0.000000pt}%
\definecolor{currentstroke}{rgb}{0.000000,0.000000,0.000000}%
\pgfsetstrokecolor{currentstroke}%
\pgfsetdash{}{0pt}%
\pgfpathmoveto{\pgfqpoint{5.563517in}{3.228615in}}%
\pgfpathlineto{\pgfqpoint{5.577893in}{3.242558in}}%
\pgfpathlineto{\pgfqpoint{5.592289in}{3.256661in}}%
\pgfpathlineto{\pgfqpoint{5.606706in}{3.270927in}}%
\pgfpathlineto{\pgfqpoint{5.621144in}{3.285355in}}%
\pgfpathlineto{\pgfqpoint{5.628228in}{3.287905in}}%
\pgfpathlineto{\pgfqpoint{5.635303in}{3.290356in}}%
\pgfpathlineto{\pgfqpoint{5.642367in}{3.292711in}}%
\pgfpathlineto{\pgfqpoint{5.649422in}{3.294975in}}%
\pgfpathlineto{\pgfqpoint{5.635005in}{3.280904in}}%
\pgfpathlineto{\pgfqpoint{5.620609in}{3.266994in}}%
\pgfpathlineto{\pgfqpoint{5.606234in}{3.253245in}}%
\pgfpathlineto{\pgfqpoint{5.591879in}{3.239657in}}%
\pgfpathlineto{\pgfqpoint{5.584802in}{3.237027in}}%
\pgfpathlineto{\pgfqpoint{5.577716in}{3.234313in}}%
\pgfpathlineto{\pgfqpoint{5.570621in}{3.231510in}}%
\pgfpathlineto{\pgfqpoint{5.563517in}{3.228615in}}%
\pgfpathclose%
\pgfusepath{fill}%
\end{pgfscope}%
\begin{pgfscope}%
\pgfpathrectangle{\pgfqpoint{1.254980in}{0.150000in}}{\pgfqpoint{5.490039in}{5.490039in}}%
\pgfusepath{clip}%
\pgfsetbuttcap%
\pgfsetroundjoin%
\definecolor{currentfill}{rgb}{0.267004,0.004874,0.329415}%
\pgfsetfillcolor{currentfill}%
\pgfsetfillopacity{0.700000}%
\pgfsetlinewidth{0.000000pt}%
\definecolor{currentstroke}{rgb}{0.000000,0.000000,0.000000}%
\pgfsetstrokecolor{currentstroke}%
\pgfsetdash{}{0pt}%
\pgfpathmoveto{\pgfqpoint{3.468622in}{1.493166in}}%
\pgfpathlineto{\pgfqpoint{3.481998in}{1.489967in}}%
\pgfpathlineto{\pgfqpoint{3.495378in}{1.486938in}}%
\pgfpathlineto{\pgfqpoint{3.508764in}{1.484077in}}%
\pgfpathlineto{\pgfqpoint{3.522154in}{1.481384in}}%
\pgfpathlineto{\pgfqpoint{3.530169in}{1.490130in}}%
\pgfpathlineto{\pgfqpoint{3.538177in}{1.498999in}}%
\pgfpathlineto{\pgfqpoint{3.546178in}{1.507986in}}%
\pgfpathlineto{\pgfqpoint{3.554171in}{1.517086in}}%
\pgfpathlineto{\pgfqpoint{3.540798in}{1.519283in}}%
\pgfpathlineto{\pgfqpoint{3.527429in}{1.521647in}}%
\pgfpathlineto{\pgfqpoint{3.514066in}{1.524180in}}%
\pgfpathlineto{\pgfqpoint{3.500707in}{1.526883in}}%
\pgfpathlineto{\pgfqpoint{3.492697in}{1.518268in}}%
\pgfpathlineto{\pgfqpoint{3.484680in}{1.509774in}}%
\pgfpathlineto{\pgfqpoint{3.476655in}{1.501405in}}%
\pgfpathlineto{\pgfqpoint{3.468622in}{1.493166in}}%
\pgfpathclose%
\pgfusepath{fill}%
\end{pgfscope}%
\begin{pgfscope}%
\pgfpathrectangle{\pgfqpoint{1.254980in}{0.150000in}}{\pgfqpoint{5.490039in}{5.490039in}}%
\pgfusepath{clip}%
\pgfsetbuttcap%
\pgfsetroundjoin%
\definecolor{currentfill}{rgb}{0.263663,0.237631,0.518762}%
\pgfsetfillcolor{currentfill}%
\pgfsetfillopacity{0.700000}%
\pgfsetlinewidth{0.000000pt}%
\definecolor{currentstroke}{rgb}{0.000000,0.000000,0.000000}%
\pgfsetstrokecolor{currentstroke}%
\pgfsetdash{}{0pt}%
\pgfpathmoveto{\pgfqpoint{2.584975in}{1.983578in}}%
\pgfpathlineto{\pgfqpoint{2.598532in}{1.967259in}}%
\pgfpathlineto{\pgfqpoint{2.612081in}{1.951167in}}%
\pgfpathlineto{\pgfqpoint{2.625623in}{1.935302in}}%
\pgfpathlineto{\pgfqpoint{2.639160in}{1.919660in}}%
\pgfpathlineto{\pgfqpoint{2.647820in}{1.916898in}}%
\pgfpathlineto{\pgfqpoint{2.656460in}{1.914454in}}%
\pgfpathlineto{\pgfqpoint{2.665083in}{1.912322in}}%
\pgfpathlineto{\pgfqpoint{2.673686in}{1.910495in}}%
\pgfpathlineto{\pgfqpoint{2.660199in}{1.925501in}}%
\pgfpathlineto{\pgfqpoint{2.646705in}{1.940730in}}%
\pgfpathlineto{\pgfqpoint{2.633205in}{1.956184in}}%
\pgfpathlineto{\pgfqpoint{2.619699in}{1.971865in}}%
\pgfpathlineto{\pgfqpoint{2.611047in}{1.974317in}}%
\pgfpathlineto{\pgfqpoint{2.602376in}{1.977082in}}%
\pgfpathlineto{\pgfqpoint{2.593686in}{1.980167in}}%
\pgfpathlineto{\pgfqpoint{2.584975in}{1.983578in}}%
\pgfpathclose%
\pgfusepath{fill}%
\end{pgfscope}%
\begin{pgfscope}%
\pgfpathrectangle{\pgfqpoint{1.254980in}{0.150000in}}{\pgfqpoint{5.490039in}{5.490039in}}%
\pgfusepath{clip}%
\pgfsetbuttcap%
\pgfsetroundjoin%
\definecolor{currentfill}{rgb}{0.281412,0.155834,0.469201}%
\pgfsetfillcolor{currentfill}%
\pgfsetfillopacity{0.700000}%
\pgfsetlinewidth{0.000000pt}%
\definecolor{currentstroke}{rgb}{0.000000,0.000000,0.000000}%
\pgfsetstrokecolor{currentstroke}%
\pgfsetdash{}{0pt}%
\pgfpathmoveto{\pgfqpoint{3.980590in}{1.748429in}}%
\pgfpathlineto{\pgfqpoint{3.994078in}{1.751754in}}%
\pgfpathlineto{\pgfqpoint{4.007575in}{1.755239in}}%
\pgfpathlineto{\pgfqpoint{4.021083in}{1.758887in}}%
\pgfpathlineto{\pgfqpoint{4.034601in}{1.762695in}}%
\pgfpathlineto{\pgfqpoint{4.042427in}{1.775264in}}%
\pgfpathlineto{\pgfqpoint{4.050248in}{1.787818in}}%
\pgfpathlineto{\pgfqpoint{4.058065in}{1.800356in}}%
\pgfpathlineto{\pgfqpoint{4.065877in}{1.812874in}}%
\pgfpathlineto{\pgfqpoint{4.052362in}{1.808732in}}%
\pgfpathlineto{\pgfqpoint{4.038859in}{1.804752in}}%
\pgfpathlineto{\pgfqpoint{4.025365in}{1.800933in}}%
\pgfpathlineto{\pgfqpoint{4.011882in}{1.797276in}}%
\pgfpathlineto{\pgfqpoint{4.004066in}{1.785081in}}%
\pgfpathlineto{\pgfqpoint{3.996245in}{1.772873in}}%
\pgfpathlineto{\pgfqpoint{3.988420in}{1.760654in}}%
\pgfpathlineto{\pgfqpoint{3.980590in}{1.748429in}}%
\pgfpathclose%
\pgfusepath{fill}%
\end{pgfscope}%
\begin{pgfscope}%
\pgfpathrectangle{\pgfqpoint{1.254980in}{0.150000in}}{\pgfqpoint{5.490039in}{5.490039in}}%
\pgfusepath{clip}%
\pgfsetbuttcap%
\pgfsetroundjoin%
\definecolor{currentfill}{rgb}{0.271828,0.209303,0.504434}%
\pgfsetfillcolor{currentfill}%
\pgfsetfillopacity{0.700000}%
\pgfsetlinewidth{0.000000pt}%
\definecolor{currentstroke}{rgb}{0.000000,0.000000,0.000000}%
\pgfsetstrokecolor{currentstroke}%
\pgfsetdash{}{0pt}%
\pgfpathmoveto{\pgfqpoint{2.639160in}{1.919660in}}%
\pgfpathlineto{\pgfqpoint{2.652689in}{1.904242in}}%
\pgfpathlineto{\pgfqpoint{2.666213in}{1.889044in}}%
\pgfpathlineto{\pgfqpoint{2.679731in}{1.874066in}}%
\pgfpathlineto{\pgfqpoint{2.693243in}{1.859305in}}%
\pgfpathlineto{\pgfqpoint{2.701855in}{1.857189in}}%
\pgfpathlineto{\pgfqpoint{2.710448in}{1.855382in}}%
\pgfpathlineto{\pgfqpoint{2.719024in}{1.853878in}}%
\pgfpathlineto{\pgfqpoint{2.727581in}{1.852672in}}%
\pgfpathlineto{\pgfqpoint{2.714116in}{1.866801in}}%
\pgfpathlineto{\pgfqpoint{2.700645in}{1.881146in}}%
\pgfpathlineto{\pgfqpoint{2.687168in}{1.895710in}}%
\pgfpathlineto{\pgfqpoint{2.673686in}{1.910495in}}%
\pgfpathlineto{\pgfqpoint{2.665083in}{1.912322in}}%
\pgfpathlineto{\pgfqpoint{2.656460in}{1.914454in}}%
\pgfpathlineto{\pgfqpoint{2.647820in}{1.916898in}}%
\pgfpathlineto{\pgfqpoint{2.639160in}{1.919660in}}%
\pgfpathclose%
\pgfusepath{fill}%
\end{pgfscope}%
\begin{pgfscope}%
\pgfpathrectangle{\pgfqpoint{1.254980in}{0.150000in}}{\pgfqpoint{5.490039in}{5.490039in}}%
\pgfusepath{clip}%
\pgfsetbuttcap%
\pgfsetroundjoin%
\definecolor{currentfill}{rgb}{0.185556,0.418570,0.556753}%
\pgfsetfillcolor{currentfill}%
\pgfsetfillopacity{0.700000}%
\pgfsetlinewidth{0.000000pt}%
\definecolor{currentstroke}{rgb}{0.000000,0.000000,0.000000}%
\pgfsetstrokecolor{currentstroke}%
\pgfsetdash{}{0pt}%
\pgfpathmoveto{\pgfqpoint{4.585897in}{2.338617in}}%
\pgfpathlineto{\pgfqpoint{4.599669in}{2.347795in}}%
\pgfpathlineto{\pgfqpoint{4.613456in}{2.357134in}}%
\pgfpathlineto{\pgfqpoint{4.627258in}{2.366634in}}%
\pgfpathlineto{\pgfqpoint{4.641076in}{2.376296in}}%
\pgfpathlineto{\pgfqpoint{4.648715in}{2.387492in}}%
\pgfpathlineto{\pgfqpoint{4.656348in}{2.398566in}}%
\pgfpathlineto{\pgfqpoint{4.663975in}{2.409519in}}%
\pgfpathlineto{\pgfqpoint{4.671596in}{2.420349in}}%
\pgfpathlineto{\pgfqpoint{4.657779in}{2.410611in}}%
\pgfpathlineto{\pgfqpoint{4.643979in}{2.401033in}}%
\pgfpathlineto{\pgfqpoint{4.630193in}{2.391617in}}%
\pgfpathlineto{\pgfqpoint{4.616423in}{2.382362in}}%
\pgfpathlineto{\pgfqpoint{4.608800in}{2.371598in}}%
\pgfpathlineto{\pgfqpoint{4.601172in}{2.360718in}}%
\pgfpathlineto{\pgfqpoint{4.593537in}{2.349724in}}%
\pgfpathlineto{\pgfqpoint{4.585897in}{2.338617in}}%
\pgfpathclose%
\pgfusepath{fill}%
\end{pgfscope}%
\begin{pgfscope}%
\pgfpathrectangle{\pgfqpoint{1.254980in}{0.150000in}}{\pgfqpoint{5.490039in}{5.490039in}}%
\pgfusepath{clip}%
\pgfsetbuttcap%
\pgfsetroundjoin%
\definecolor{currentfill}{rgb}{0.253935,0.265254,0.529983}%
\pgfsetfillcolor{currentfill}%
\pgfsetfillopacity{0.700000}%
\pgfsetlinewidth{0.000000pt}%
\definecolor{currentstroke}{rgb}{0.000000,0.000000,0.000000}%
\pgfsetstrokecolor{currentstroke}%
\pgfsetdash{}{0pt}%
\pgfpathmoveto{\pgfqpoint{2.530676in}{2.051163in}}%
\pgfpathlineto{\pgfqpoint{2.544263in}{2.033917in}}%
\pgfpathlineto{\pgfqpoint{2.557841in}{2.016905in}}%
\pgfpathlineto{\pgfqpoint{2.571412in}{2.000126in}}%
\pgfpathlineto{\pgfqpoint{2.584975in}{1.983578in}}%
\pgfpathlineto{\pgfqpoint{2.593686in}{1.980167in}}%
\pgfpathlineto{\pgfqpoint{2.602376in}{1.977082in}}%
\pgfpathlineto{\pgfqpoint{2.611047in}{1.974317in}}%
\pgfpathlineto{\pgfqpoint{2.619699in}{1.971865in}}%
\pgfpathlineto{\pgfqpoint{2.606185in}{1.987774in}}%
\pgfpathlineto{\pgfqpoint{2.592665in}{2.003913in}}%
\pgfpathlineto{\pgfqpoint{2.579138in}{2.020284in}}%
\pgfpathlineto{\pgfqpoint{2.565604in}{2.036888in}}%
\pgfpathlineto{\pgfqpoint{2.556902in}{2.039967in}}%
\pgfpathlineto{\pgfqpoint{2.548181in}{2.043369in}}%
\pgfpathlineto{\pgfqpoint{2.539439in}{2.047099in}}%
\pgfpathlineto{\pgfqpoint{2.530676in}{2.051163in}}%
\pgfpathclose%
\pgfusepath{fill}%
\end{pgfscope}%
\begin{pgfscope}%
\pgfpathrectangle{\pgfqpoint{1.254980in}{0.150000in}}{\pgfqpoint{5.490039in}{5.490039in}}%
\pgfusepath{clip}%
\pgfsetbuttcap%
\pgfsetroundjoin%
\definecolor{currentfill}{rgb}{0.127568,0.566949,0.550556}%
\pgfsetfillcolor{currentfill}%
\pgfsetfillopacity{0.700000}%
\pgfsetlinewidth{0.000000pt}%
\definecolor{currentstroke}{rgb}{0.000000,0.000000,0.000000}%
\pgfsetstrokecolor{currentstroke}%
\pgfsetdash{}{0pt}%
\pgfpathmoveto{\pgfqpoint{4.989469in}{2.740188in}}%
\pgfpathlineto{\pgfqpoint{5.003483in}{2.751960in}}%
\pgfpathlineto{\pgfqpoint{5.017514in}{2.763894in}}%
\pgfpathlineto{\pgfqpoint{5.031563in}{2.775990in}}%
\pgfpathlineto{\pgfqpoint{5.045630in}{2.788248in}}%
\pgfpathlineto{\pgfqpoint{5.053085in}{2.796207in}}%
\pgfpathlineto{\pgfqpoint{5.060532in}{2.804026in}}%
\pgfpathlineto{\pgfqpoint{5.067970in}{2.811706in}}%
\pgfpathlineto{\pgfqpoint{5.075401in}{2.819250in}}%
\pgfpathlineto{\pgfqpoint{5.061341in}{2.807097in}}%
\pgfpathlineto{\pgfqpoint{5.047299in}{2.795106in}}%
\pgfpathlineto{\pgfqpoint{5.033274in}{2.783276in}}%
\pgfpathlineto{\pgfqpoint{5.019268in}{2.771608in}}%
\pgfpathlineto{\pgfqpoint{5.011830in}{2.763948in}}%
\pgfpathlineto{\pgfqpoint{5.004384in}{2.756160in}}%
\pgfpathlineto{\pgfqpoint{4.996931in}{2.748240in}}%
\pgfpathlineto{\pgfqpoint{4.989469in}{2.740188in}}%
\pgfpathclose%
\pgfusepath{fill}%
\end{pgfscope}%
\begin{pgfscope}%
\pgfpathrectangle{\pgfqpoint{1.254980in}{0.150000in}}{\pgfqpoint{5.490039in}{5.490039in}}%
\pgfusepath{clip}%
\pgfsetbuttcap%
\pgfsetroundjoin%
\definecolor{currentfill}{rgb}{0.278012,0.180367,0.486697}%
\pgfsetfillcolor{currentfill}%
\pgfsetfillopacity{0.700000}%
\pgfsetlinewidth{0.000000pt}%
\definecolor{currentstroke}{rgb}{0.000000,0.000000,0.000000}%
\pgfsetstrokecolor{currentstroke}%
\pgfsetdash{}{0pt}%
\pgfpathmoveto{\pgfqpoint{2.693243in}{1.859305in}}%
\pgfpathlineto{\pgfqpoint{2.706750in}{1.844762in}}%
\pgfpathlineto{\pgfqpoint{2.720251in}{1.830433in}}%
\pgfpathlineto{\pgfqpoint{2.733748in}{1.816318in}}%
\pgfpathlineto{\pgfqpoint{2.747239in}{1.802415in}}%
\pgfpathlineto{\pgfqpoint{2.755805in}{1.800940in}}%
\pgfpathlineto{\pgfqpoint{2.764353in}{1.799767in}}%
\pgfpathlineto{\pgfqpoint{2.772883in}{1.798889in}}%
\pgfpathlineto{\pgfqpoint{2.781397in}{1.798301in}}%
\pgfpathlineto{\pgfqpoint{2.767950in}{1.811575in}}%
\pgfpathlineto{\pgfqpoint{2.754498in}{1.825061in}}%
\pgfpathlineto{\pgfqpoint{2.741042in}{1.838759in}}%
\pgfpathlineto{\pgfqpoint{2.727581in}{1.852672in}}%
\pgfpathlineto{\pgfqpoint{2.719024in}{1.853878in}}%
\pgfpathlineto{\pgfqpoint{2.710448in}{1.855382in}}%
\pgfpathlineto{\pgfqpoint{2.701855in}{1.857189in}}%
\pgfpathlineto{\pgfqpoint{2.693243in}{1.859305in}}%
\pgfpathclose%
\pgfusepath{fill}%
\end{pgfscope}%
\begin{pgfscope}%
\pgfpathrectangle{\pgfqpoint{1.254980in}{0.150000in}}{\pgfqpoint{5.490039in}{5.490039in}}%
\pgfusepath{clip}%
\pgfsetbuttcap%
\pgfsetroundjoin%
\definecolor{currentfill}{rgb}{0.153364,0.497000,0.557724}%
\pgfsetfillcolor{currentfill}%
\pgfsetfillopacity{0.700000}%
\pgfsetlinewidth{0.000000pt}%
\definecolor{currentstroke}{rgb}{0.000000,0.000000,0.000000}%
\pgfsetstrokecolor{currentstroke}%
\pgfsetdash{}{0pt}%
\pgfpathmoveto{\pgfqpoint{4.787734in}{2.543412in}}%
\pgfpathlineto{\pgfqpoint{4.801625in}{2.554022in}}%
\pgfpathlineto{\pgfqpoint{4.815533in}{2.564794in}}%
\pgfpathlineto{\pgfqpoint{4.829457in}{2.575727in}}%
\pgfpathlineto{\pgfqpoint{4.843398in}{2.586821in}}%
\pgfpathlineto{\pgfqpoint{4.850954in}{2.596548in}}%
\pgfpathlineto{\pgfqpoint{4.858503in}{2.606139in}}%
\pgfpathlineto{\pgfqpoint{4.866045in}{2.615594in}}%
\pgfpathlineto{\pgfqpoint{4.873579in}{2.624915in}}%
\pgfpathlineto{\pgfqpoint{4.859642in}{2.613833in}}%
\pgfpathlineto{\pgfqpoint{4.845721in}{2.602913in}}%
\pgfpathlineto{\pgfqpoint{4.831817in}{2.592154in}}%
\pgfpathlineto{\pgfqpoint{4.817930in}{2.581557in}}%
\pgfpathlineto{\pgfqpoint{4.810392in}{2.572213in}}%
\pgfpathlineto{\pgfqpoint{4.802846in}{2.562741in}}%
\pgfpathlineto{\pgfqpoint{4.795294in}{2.553141in}}%
\pgfpathlineto{\pgfqpoint{4.787734in}{2.543412in}}%
\pgfpathclose%
\pgfusepath{fill}%
\end{pgfscope}%
\begin{pgfscope}%
\pgfpathrectangle{\pgfqpoint{1.254980in}{0.150000in}}{\pgfqpoint{5.490039in}{5.490039in}}%
\pgfusepath{clip}%
\pgfsetbuttcap%
\pgfsetroundjoin%
\definecolor{currentfill}{rgb}{0.241237,0.296485,0.539709}%
\pgfsetfillcolor{currentfill}%
\pgfsetfillopacity{0.700000}%
\pgfsetlinewidth{0.000000pt}%
\definecolor{currentstroke}{rgb}{0.000000,0.000000,0.000000}%
\pgfsetstrokecolor{currentstroke}%
\pgfsetdash{}{0pt}%
\pgfpathmoveto{\pgfqpoint{2.476247in}{2.122530in}}%
\pgfpathlineto{\pgfqpoint{2.489867in}{2.104327in}}%
\pgfpathlineto{\pgfqpoint{2.503479in}{2.086367in}}%
\pgfpathlineto{\pgfqpoint{2.517082in}{2.068646in}}%
\pgfpathlineto{\pgfqpoint{2.530676in}{2.051163in}}%
\pgfpathlineto{\pgfqpoint{2.539439in}{2.047099in}}%
\pgfpathlineto{\pgfqpoint{2.548181in}{2.043369in}}%
\pgfpathlineto{\pgfqpoint{2.556902in}{2.039967in}}%
\pgfpathlineto{\pgfqpoint{2.565604in}{2.036888in}}%
\pgfpathlineto{\pgfqpoint{2.552061in}{2.053727in}}%
\pgfpathlineto{\pgfqpoint{2.538511in}{2.070803in}}%
\pgfpathlineto{\pgfqpoint{2.524953in}{2.088118in}}%
\pgfpathlineto{\pgfqpoint{2.511387in}{2.105674in}}%
\pgfpathlineto{\pgfqpoint{2.502634in}{2.109386in}}%
\pgfpathlineto{\pgfqpoint{2.493859in}{2.113428in}}%
\pgfpathlineto{\pgfqpoint{2.485064in}{2.117807in}}%
\pgfpathlineto{\pgfqpoint{2.476247in}{2.122530in}}%
\pgfpathclose%
\pgfusepath{fill}%
\end{pgfscope}%
\begin{pgfscope}%
\pgfpathrectangle{\pgfqpoint{1.254980in}{0.150000in}}{\pgfqpoint{5.490039in}{5.490039in}}%
\pgfusepath{clip}%
\pgfsetbuttcap%
\pgfsetroundjoin%
\definecolor{currentfill}{rgb}{0.276022,0.044167,0.370164}%
\pgfsetfillcolor{currentfill}%
\pgfsetfillopacity{0.700000}%
\pgfsetlinewidth{0.000000pt}%
\definecolor{currentstroke}{rgb}{0.000000,0.000000,0.000000}%
\pgfsetstrokecolor{currentstroke}%
\pgfsetdash{}{0pt}%
\pgfpathmoveto{\pgfqpoint{3.049727in}{1.575274in}}%
\pgfpathlineto{\pgfqpoint{3.063127in}{1.566159in}}%
\pgfpathlineto{\pgfqpoint{3.076526in}{1.557230in}}%
\pgfpathlineto{\pgfqpoint{3.089926in}{1.548486in}}%
\pgfpathlineto{\pgfqpoint{3.103326in}{1.539928in}}%
\pgfpathlineto{\pgfqpoint{3.111600in}{1.543231in}}%
\pgfpathlineto{\pgfqpoint{3.119863in}{1.546766in}}%
\pgfpathlineto{\pgfqpoint{3.128113in}{1.550525in}}%
\pgfpathlineto{\pgfqpoint{3.136351in}{1.554504in}}%
\pgfpathlineto{\pgfqpoint{3.122983in}{1.562477in}}%
\pgfpathlineto{\pgfqpoint{3.109615in}{1.570635in}}%
\pgfpathlineto{\pgfqpoint{3.096247in}{1.578979in}}%
\pgfpathlineto{\pgfqpoint{3.082880in}{1.587508in}}%
\pgfpathlineto{\pgfqpoint{3.074610in}{1.584104in}}%
\pgfpathlineto{\pgfqpoint{3.066329in}{1.580926in}}%
\pgfpathlineto{\pgfqpoint{3.058034in}{1.577981in}}%
\pgfpathlineto{\pgfqpoint{3.049727in}{1.575274in}}%
\pgfpathclose%
\pgfusepath{fill}%
\end{pgfscope}%
\begin{pgfscope}%
\pgfpathrectangle{\pgfqpoint{1.254980in}{0.150000in}}{\pgfqpoint{5.490039in}{5.490039in}}%
\pgfusepath{clip}%
\pgfsetbuttcap%
\pgfsetroundjoin%
\definecolor{currentfill}{rgb}{0.137339,0.662252,0.515571}%
\pgfsetfillcolor{currentfill}%
\pgfsetfillopacity{0.700000}%
\pgfsetlinewidth{0.000000pt}%
\definecolor{currentstroke}{rgb}{0.000000,0.000000,0.000000}%
\pgfsetstrokecolor{currentstroke}%
\pgfsetdash{}{0pt}%
\pgfpathmoveto{\pgfqpoint{5.276836in}{2.998807in}}%
\pgfpathlineto{\pgfqpoint{5.291035in}{3.011915in}}%
\pgfpathlineto{\pgfqpoint{5.305253in}{3.025186in}}%
\pgfpathlineto{\pgfqpoint{5.319491in}{3.038618in}}%
\pgfpathlineto{\pgfqpoint{5.333749in}{3.052213in}}%
\pgfpathlineto{\pgfqpoint{5.341036in}{3.057501in}}%
\pgfpathlineto{\pgfqpoint{5.348313in}{3.062658in}}%
\pgfpathlineto{\pgfqpoint{5.355581in}{3.067689in}}%
\pgfpathlineto{\pgfqpoint{5.362840in}{3.072595in}}%
\pgfpathlineto{\pgfqpoint{5.348595in}{3.059230in}}%
\pgfpathlineto{\pgfqpoint{5.334370in}{3.046028in}}%
\pgfpathlineto{\pgfqpoint{5.320165in}{3.032986in}}%
\pgfpathlineto{\pgfqpoint{5.305979in}{3.020106in}}%
\pgfpathlineto{\pgfqpoint{5.298706in}{3.014960in}}%
\pgfpathlineto{\pgfqpoint{5.291425in}{3.009696in}}%
\pgfpathlineto{\pgfqpoint{5.284135in}{3.004313in}}%
\pgfpathlineto{\pgfqpoint{5.276836in}{2.998807in}}%
\pgfpathclose%
\pgfusepath{fill}%
\end{pgfscope}%
\begin{pgfscope}%
\pgfpathrectangle{\pgfqpoint{1.254980in}{0.150000in}}{\pgfqpoint{5.490039in}{5.490039in}}%
\pgfusepath{clip}%
\pgfsetbuttcap%
\pgfsetroundjoin%
\definecolor{currentfill}{rgb}{0.296479,0.761561,0.424223}%
\pgfsetfillcolor{currentfill}%
\pgfsetfillopacity{0.700000}%
\pgfsetlinewidth{0.000000pt}%
\definecolor{currentstroke}{rgb}{0.000000,0.000000,0.000000}%
\pgfsetstrokecolor{currentstroke}%
\pgfsetdash{}{0pt}%
\pgfpathmoveto{\pgfqpoint{5.649422in}{3.294975in}}%
\pgfpathlineto{\pgfqpoint{5.663860in}{3.309207in}}%
\pgfpathlineto{\pgfqpoint{5.678320in}{3.323602in}}%
\pgfpathlineto{\pgfqpoint{5.692800in}{3.338158in}}%
\pgfpathlineto{\pgfqpoint{5.707302in}{3.352876in}}%
\pgfpathlineto{\pgfqpoint{5.714326in}{3.354675in}}%
\pgfpathlineto{\pgfqpoint{5.721339in}{3.356384in}}%
\pgfpathlineto{\pgfqpoint{5.728343in}{3.358007in}}%
\pgfpathlineto{\pgfqpoint{5.735337in}{3.359548in}}%
\pgfpathlineto{\pgfqpoint{5.720859in}{3.345218in}}%
\pgfpathlineto{\pgfqpoint{5.706402in}{3.331050in}}%
\pgfpathlineto{\pgfqpoint{5.691966in}{3.317043in}}%
\pgfpathlineto{\pgfqpoint{5.677550in}{3.303197in}}%
\pgfpathlineto{\pgfqpoint{5.670532in}{3.301258in}}%
\pgfpathlineto{\pgfqpoint{5.663505in}{3.299244in}}%
\pgfpathlineto{\pgfqpoint{5.656468in}{3.297151in}}%
\pgfpathlineto{\pgfqpoint{5.649422in}{3.294975in}}%
\pgfpathclose%
\pgfusepath{fill}%
\end{pgfscope}%
\begin{pgfscope}%
\pgfpathrectangle{\pgfqpoint{1.254980in}{0.150000in}}{\pgfqpoint{5.490039in}{5.490039in}}%
\pgfusepath{clip}%
\pgfsetbuttcap%
\pgfsetroundjoin%
\definecolor{currentfill}{rgb}{0.281412,0.155834,0.469201}%
\pgfsetfillcolor{currentfill}%
\pgfsetfillopacity{0.700000}%
\pgfsetlinewidth{0.000000pt}%
\definecolor{currentstroke}{rgb}{0.000000,0.000000,0.000000}%
\pgfsetstrokecolor{currentstroke}%
\pgfsetdash{}{0pt}%
\pgfpathmoveto{\pgfqpoint{2.747239in}{1.802415in}}%
\pgfpathlineto{\pgfqpoint{2.760726in}{1.788723in}}%
\pgfpathlineto{\pgfqpoint{2.774209in}{1.775240in}}%
\pgfpathlineto{\pgfqpoint{2.787687in}{1.761966in}}%
\pgfpathlineto{\pgfqpoint{2.801162in}{1.748898in}}%
\pgfpathlineto{\pgfqpoint{2.809683in}{1.748061in}}%
\pgfpathlineto{\pgfqpoint{2.818187in}{1.747519in}}%
\pgfpathlineto{\pgfqpoint{2.826675in}{1.747264in}}%
\pgfpathlineto{\pgfqpoint{2.835146in}{1.747291in}}%
\pgfpathlineto{\pgfqpoint{2.821714in}{1.759733in}}%
\pgfpathlineto{\pgfqpoint{2.808279in}{1.772381in}}%
\pgfpathlineto{\pgfqpoint{2.794840in}{1.785237in}}%
\pgfpathlineto{\pgfqpoint{2.781397in}{1.798301in}}%
\pgfpathlineto{\pgfqpoint{2.772883in}{1.798889in}}%
\pgfpathlineto{\pgfqpoint{2.764353in}{1.799767in}}%
\pgfpathlineto{\pgfqpoint{2.755805in}{1.800940in}}%
\pgfpathlineto{\pgfqpoint{2.747239in}{1.802415in}}%
\pgfpathclose%
\pgfusepath{fill}%
\end{pgfscope}%
\begin{pgfscope}%
\pgfpathrectangle{\pgfqpoint{1.254980in}{0.150000in}}{\pgfqpoint{5.490039in}{5.490039in}}%
\pgfusepath{clip}%
\pgfsetbuttcap%
\pgfsetroundjoin%
\definecolor{currentfill}{rgb}{0.225863,0.330805,0.547314}%
\pgfsetfillcolor{currentfill}%
\pgfsetfillopacity{0.700000}%
\pgfsetlinewidth{0.000000pt}%
\definecolor{currentstroke}{rgb}{0.000000,0.000000,0.000000}%
\pgfsetstrokecolor{currentstroke}%
\pgfsetdash{}{0pt}%
\pgfpathmoveto{\pgfqpoint{2.421672in}{2.197799in}}%
\pgfpathlineto{\pgfqpoint{2.435331in}{2.178609in}}%
\pgfpathlineto{\pgfqpoint{2.448979in}{2.159668in}}%
\pgfpathlineto{\pgfqpoint{2.462618in}{2.140976in}}%
\pgfpathlineto{\pgfqpoint{2.476247in}{2.122530in}}%
\pgfpathlineto{\pgfqpoint{2.485064in}{2.117807in}}%
\pgfpathlineto{\pgfqpoint{2.493859in}{2.113428in}}%
\pgfpathlineto{\pgfqpoint{2.502634in}{2.109386in}}%
\pgfpathlineto{\pgfqpoint{2.511387in}{2.105674in}}%
\pgfpathlineto{\pgfqpoint{2.497812in}{2.123472in}}%
\pgfpathlineto{\pgfqpoint{2.484228in}{2.141515in}}%
\pgfpathlineto{\pgfqpoint{2.470635in}{2.159805in}}%
\pgfpathlineto{\pgfqpoint{2.457033in}{2.178344in}}%
\pgfpathlineto{\pgfqpoint{2.448226in}{2.182693in}}%
\pgfpathlineto{\pgfqpoint{2.439397in}{2.187381in}}%
\pgfpathlineto{\pgfqpoint{2.430546in}{2.192414in}}%
\pgfpathlineto{\pgfqpoint{2.421672in}{2.197799in}}%
\pgfpathclose%
\pgfusepath{fill}%
\end{pgfscope}%
\begin{pgfscope}%
\pgfpathrectangle{\pgfqpoint{1.254980in}{0.150000in}}{\pgfqpoint{5.490039in}{5.490039in}}%
\pgfusepath{clip}%
\pgfsetbuttcap%
\pgfsetroundjoin%
\definecolor{currentfill}{rgb}{0.246811,0.283237,0.535941}%
\pgfsetfillcolor{currentfill}%
\pgfsetfillopacity{0.700000}%
\pgfsetlinewidth{0.000000pt}%
\definecolor{currentstroke}{rgb}{0.000000,0.000000,0.000000}%
\pgfsetstrokecolor{currentstroke}%
\pgfsetdash{}{0pt}%
\pgfpathmoveto{\pgfqpoint{4.267725in}{2.006141in}}%
\pgfpathlineto{\pgfqpoint{4.281337in}{2.012566in}}%
\pgfpathlineto{\pgfqpoint{4.294961in}{2.019151in}}%
\pgfpathlineto{\pgfqpoint{4.308598in}{2.025897in}}%
\pgfpathlineto{\pgfqpoint{4.322248in}{2.032804in}}%
\pgfpathlineto{\pgfqpoint{4.329995in}{2.045520in}}%
\pgfpathlineto{\pgfqpoint{4.337738in}{2.058161in}}%
\pgfpathlineto{\pgfqpoint{4.345476in}{2.070725in}}%
\pgfpathlineto{\pgfqpoint{4.353209in}{2.083209in}}%
\pgfpathlineto{\pgfqpoint{4.339560in}{2.076080in}}%
\pgfpathlineto{\pgfqpoint{4.325924in}{2.069112in}}%
\pgfpathlineto{\pgfqpoint{4.312300in}{2.062305in}}%
\pgfpathlineto{\pgfqpoint{4.298690in}{2.055659in}}%
\pgfpathlineto{\pgfqpoint{4.290956in}{2.043386in}}%
\pgfpathlineto{\pgfqpoint{4.283217in}{2.031040in}}%
\pgfpathlineto{\pgfqpoint{4.275473in}{2.018625in}}%
\pgfpathlineto{\pgfqpoint{4.267725in}{2.006141in}}%
\pgfpathclose%
\pgfusepath{fill}%
\end{pgfscope}%
\begin{pgfscope}%
\pgfpathrectangle{\pgfqpoint{1.254980in}{0.150000in}}{\pgfqpoint{5.490039in}{5.490039in}}%
\pgfusepath{clip}%
\pgfsetbuttcap%
\pgfsetroundjoin%
\definecolor{currentfill}{rgb}{0.267004,0.004874,0.329415}%
\pgfsetfillcolor{currentfill}%
\pgfsetfillopacity{0.700000}%
\pgfsetlinewidth{0.000000pt}%
\definecolor{currentstroke}{rgb}{0.000000,0.000000,0.000000}%
\pgfsetstrokecolor{currentstroke}%
\pgfsetdash{}{0pt}%
\pgfpathmoveto{\pgfqpoint{3.243336in}{1.497252in}}%
\pgfpathlineto{\pgfqpoint{3.256716in}{1.490901in}}%
\pgfpathlineto{\pgfqpoint{3.270098in}{1.484727in}}%
\pgfpathlineto{\pgfqpoint{3.283483in}{1.478728in}}%
\pgfpathlineto{\pgfqpoint{3.296869in}{1.472904in}}%
\pgfpathlineto{\pgfqpoint{3.305015in}{1.478792in}}%
\pgfpathlineto{\pgfqpoint{3.313150in}{1.484865in}}%
\pgfpathlineto{\pgfqpoint{3.321276in}{1.491119in}}%
\pgfpathlineto{\pgfqpoint{3.329393in}{1.497548in}}%
\pgfpathlineto{\pgfqpoint{3.316030in}{1.502818in}}%
\pgfpathlineto{\pgfqpoint{3.302670in}{1.508263in}}%
\pgfpathlineto{\pgfqpoint{3.289313in}{1.513884in}}%
\pgfpathlineto{\pgfqpoint{3.275958in}{1.519681in}}%
\pgfpathlineto{\pgfqpoint{3.267818in}{1.513795in}}%
\pgfpathlineto{\pgfqpoint{3.259667in}{1.508092in}}%
\pgfpathlineto{\pgfqpoint{3.251507in}{1.502575in}}%
\pgfpathlineto{\pgfqpoint{3.243336in}{1.497252in}}%
\pgfpathclose%
\pgfusepath{fill}%
\end{pgfscope}%
\begin{pgfscope}%
\pgfpathrectangle{\pgfqpoint{1.254980in}{0.150000in}}{\pgfqpoint{5.490039in}{5.490039in}}%
\pgfusepath{clip}%
\pgfsetbuttcap%
\pgfsetroundjoin%
\definecolor{currentfill}{rgb}{0.275191,0.194905,0.496005}%
\pgfsetfillcolor{currentfill}%
\pgfsetfillopacity{0.700000}%
\pgfsetlinewidth{0.000000pt}%
\definecolor{currentstroke}{rgb}{0.000000,0.000000,0.000000}%
\pgfsetstrokecolor{currentstroke}%
\pgfsetdash{}{0pt}%
\pgfpathmoveto{\pgfqpoint{4.065877in}{1.812874in}}%
\pgfpathlineto{\pgfqpoint{4.079402in}{1.817177in}}%
\pgfpathlineto{\pgfqpoint{4.092937in}{1.821641in}}%
\pgfpathlineto{\pgfqpoint{4.106484in}{1.826266in}}%
\pgfpathlineto{\pgfqpoint{4.120042in}{1.831052in}}%
\pgfpathlineto{\pgfqpoint{4.127847in}{1.843863in}}%
\pgfpathlineto{\pgfqpoint{4.135647in}{1.856642in}}%
\pgfpathlineto{\pgfqpoint{4.143443in}{1.869385in}}%
\pgfpathlineto{\pgfqpoint{4.151234in}{1.882089in}}%
\pgfpathlineto{\pgfqpoint{4.137679in}{1.876997in}}%
\pgfpathlineto{\pgfqpoint{4.124135in}{1.872066in}}%
\pgfpathlineto{\pgfqpoint{4.110602in}{1.867297in}}%
\pgfpathlineto{\pgfqpoint{4.097080in}{1.862688in}}%
\pgfpathlineto{\pgfqpoint{4.089286in}{1.850279in}}%
\pgfpathlineto{\pgfqpoint{4.081487in}{1.837838in}}%
\pgfpathlineto{\pgfqpoint{4.073684in}{1.825369in}}%
\pgfpathlineto{\pgfqpoint{4.065877in}{1.812874in}}%
\pgfpathclose%
\pgfusepath{fill}%
\end{pgfscope}%
\begin{pgfscope}%
\pgfpathrectangle{\pgfqpoint{1.254980in}{0.150000in}}{\pgfqpoint{5.490039in}{5.490039in}}%
\pgfusepath{clip}%
\pgfsetbuttcap%
\pgfsetroundjoin%
\definecolor{currentfill}{rgb}{0.144759,0.519093,0.556572}%
\pgfsetfillcolor{currentfill}%
\pgfsetfillopacity{0.700000}%
\pgfsetlinewidth{0.000000pt}%
\definecolor{currentstroke}{rgb}{0.000000,0.000000,0.000000}%
\pgfsetstrokecolor{currentstroke}%
\pgfsetdash{}{0pt}%
\pgfpathmoveto{\pgfqpoint{2.127060in}{2.703846in}}%
\pgfpathlineto{\pgfqpoint{2.140971in}{2.678625in}}%
\pgfpathlineto{\pgfqpoint{2.154866in}{2.653714in}}%
\pgfpathlineto{\pgfqpoint{2.168745in}{2.629109in}}%
\pgfpathlineto{\pgfqpoint{2.182607in}{2.604809in}}%
\pgfpathlineto{\pgfqpoint{2.191691in}{2.597569in}}%
\pgfpathlineto{\pgfqpoint{2.200748in}{2.590698in}}%
\pgfpathlineto{\pgfqpoint{2.209781in}{2.584190in}}%
\pgfpathlineto{\pgfqpoint{2.218789in}{2.578038in}}%
\pgfpathlineto{\pgfqpoint{2.204990in}{2.601691in}}%
\pgfpathlineto{\pgfqpoint{2.191176in}{2.625647in}}%
\pgfpathlineto{\pgfqpoint{2.177346in}{2.649907in}}%
\pgfpathlineto{\pgfqpoint{2.163501in}{2.674476in}}%
\pgfpathlineto{\pgfqpoint{2.154429in}{2.681264in}}%
\pgfpathlineto{\pgfqpoint{2.145332in}{2.688417in}}%
\pgfpathlineto{\pgfqpoint{2.136209in}{2.695943in}}%
\pgfpathlineto{\pgfqpoint{2.127060in}{2.703846in}}%
\pgfpathclose%
\pgfusepath{fill}%
\end{pgfscope}%
\begin{pgfscope}%
\pgfpathrectangle{\pgfqpoint{1.254980in}{0.150000in}}{\pgfqpoint{5.490039in}{5.490039in}}%
\pgfusepath{clip}%
\pgfsetbuttcap%
\pgfsetroundjoin%
\definecolor{currentfill}{rgb}{0.267004,0.004874,0.329415}%
\pgfsetfillcolor{currentfill}%
\pgfsetfillopacity{0.700000}%
\pgfsetlinewidth{0.000000pt}%
\definecolor{currentstroke}{rgb}{0.000000,0.000000,0.000000}%
\pgfsetstrokecolor{currentstroke}%
\pgfsetdash{}{0pt}%
\pgfpathmoveto{\pgfqpoint{3.382874in}{1.478204in}}%
\pgfpathlineto{\pgfqpoint{3.396253in}{1.473799in}}%
\pgfpathlineto{\pgfqpoint{3.409636in}{1.469566in}}%
\pgfpathlineto{\pgfqpoint{3.423022in}{1.465503in}}%
\pgfpathlineto{\pgfqpoint{3.436413in}{1.461610in}}%
\pgfpathlineto{\pgfqpoint{3.444477in}{1.469279in}}%
\pgfpathlineto{\pgfqpoint{3.452534in}{1.477098in}}%
\pgfpathlineto{\pgfqpoint{3.460582in}{1.485062in}}%
\pgfpathlineto{\pgfqpoint{3.468622in}{1.493166in}}%
\pgfpathlineto{\pgfqpoint{3.455251in}{1.496534in}}%
\pgfpathlineto{\pgfqpoint{3.441884in}{1.500073in}}%
\pgfpathlineto{\pgfqpoint{3.428522in}{1.503782in}}%
\pgfpathlineto{\pgfqpoint{3.415163in}{1.507662in}}%
\pgfpathlineto{\pgfqpoint{3.407104in}{1.500072in}}%
\pgfpathlineto{\pgfqpoint{3.399036in}{1.492629in}}%
\pgfpathlineto{\pgfqpoint{3.390959in}{1.485338in}}%
\pgfpathlineto{\pgfqpoint{3.382874in}{1.478204in}}%
\pgfpathclose%
\pgfusepath{fill}%
\end{pgfscope}%
\begin{pgfscope}%
\pgfpathrectangle{\pgfqpoint{1.254980in}{0.150000in}}{\pgfqpoint{5.490039in}{5.490039in}}%
\pgfusepath{clip}%
\pgfsetbuttcap%
\pgfsetroundjoin%
\definecolor{currentfill}{rgb}{0.283072,0.130895,0.449241}%
\pgfsetfillcolor{currentfill}%
\pgfsetfillopacity{0.700000}%
\pgfsetlinewidth{0.000000pt}%
\definecolor{currentstroke}{rgb}{0.000000,0.000000,0.000000}%
\pgfsetstrokecolor{currentstroke}%
\pgfsetdash{}{0pt}%
\pgfpathmoveto{\pgfqpoint{2.801162in}{1.748898in}}%
\pgfpathlineto{\pgfqpoint{2.814633in}{1.736035in}}%
\pgfpathlineto{\pgfqpoint{2.828100in}{1.723377in}}%
\pgfpathlineto{\pgfqpoint{2.841563in}{1.710922in}}%
\pgfpathlineto{\pgfqpoint{2.855023in}{1.698669in}}%
\pgfpathlineto{\pgfqpoint{2.863502in}{1.698468in}}%
\pgfpathlineto{\pgfqpoint{2.871964in}{1.698554in}}%
\pgfpathlineto{\pgfqpoint{2.880411in}{1.698919in}}%
\pgfpathlineto{\pgfqpoint{2.888842in}{1.699558in}}%
\pgfpathlineto{\pgfqpoint{2.875422in}{1.711188in}}%
\pgfpathlineto{\pgfqpoint{2.862000in}{1.723020in}}%
\pgfpathlineto{\pgfqpoint{2.848575in}{1.735053in}}%
\pgfpathlineto{\pgfqpoint{2.835146in}{1.747291in}}%
\pgfpathlineto{\pgfqpoint{2.826675in}{1.747264in}}%
\pgfpathlineto{\pgfqpoint{2.818187in}{1.747519in}}%
\pgfpathlineto{\pgfqpoint{2.809683in}{1.748061in}}%
\pgfpathlineto{\pgfqpoint{2.801162in}{1.748898in}}%
\pgfpathclose%
\pgfusepath{fill}%
\end{pgfscope}%
\begin{pgfscope}%
\pgfpathrectangle{\pgfqpoint{1.254980in}{0.150000in}}{\pgfqpoint{5.490039in}{5.490039in}}%
\pgfusepath{clip}%
\pgfsetbuttcap%
\pgfsetroundjoin%
\definecolor{currentfill}{rgb}{0.206756,0.371758,0.553117}%
\pgfsetfillcolor{currentfill}%
\pgfsetfillopacity{0.700000}%
\pgfsetlinewidth{0.000000pt}%
\definecolor{currentstroke}{rgb}{0.000000,0.000000,0.000000}%
\pgfsetstrokecolor{currentstroke}%
\pgfsetdash{}{0pt}%
\pgfpathmoveto{\pgfqpoint{4.469618in}{2.211596in}}%
\pgfpathlineto{\pgfqpoint{4.483334in}{2.219888in}}%
\pgfpathlineto{\pgfqpoint{4.497064in}{2.228341in}}%
\pgfpathlineto{\pgfqpoint{4.510808in}{2.236954in}}%
\pgfpathlineto{\pgfqpoint{4.524567in}{2.245729in}}%
\pgfpathlineto{\pgfqpoint{4.532253in}{2.257724in}}%
\pgfpathlineto{\pgfqpoint{4.539933in}{2.269611in}}%
\pgfpathlineto{\pgfqpoint{4.547608in}{2.281390in}}%
\pgfpathlineto{\pgfqpoint{4.555277in}{2.293058in}}%
\pgfpathlineto{\pgfqpoint{4.541519in}{2.284148in}}%
\pgfpathlineto{\pgfqpoint{4.527775in}{2.275398in}}%
\pgfpathlineto{\pgfqpoint{4.514046in}{2.266810in}}%
\pgfpathlineto{\pgfqpoint{4.500331in}{2.258383in}}%
\pgfpathlineto{\pgfqpoint{4.492661in}{2.246839in}}%
\pgfpathlineto{\pgfqpoint{4.484985in}{2.235193in}}%
\pgfpathlineto{\pgfqpoint{4.477304in}{2.223445in}}%
\pgfpathlineto{\pgfqpoint{4.469618in}{2.211596in}}%
\pgfpathclose%
\pgfusepath{fill}%
\end{pgfscope}%
\begin{pgfscope}%
\pgfpathrectangle{\pgfqpoint{1.254980in}{0.150000in}}{\pgfqpoint{5.490039in}{5.490039in}}%
\pgfusepath{clip}%
\pgfsetbuttcap%
\pgfsetroundjoin%
\definecolor{currentfill}{rgb}{0.210503,0.363727,0.552206}%
\pgfsetfillcolor{currentfill}%
\pgfsetfillopacity{0.700000}%
\pgfsetlinewidth{0.000000pt}%
\definecolor{currentstroke}{rgb}{0.000000,0.000000,0.000000}%
\pgfsetstrokecolor{currentstroke}%
\pgfsetdash{}{0pt}%
\pgfpathmoveto{\pgfqpoint{2.366936in}{2.277101in}}%
\pgfpathlineto{\pgfqpoint{2.380636in}{2.256890in}}%
\pgfpathlineto{\pgfqpoint{2.394325in}{2.236937in}}%
\pgfpathlineto{\pgfqpoint{2.408004in}{2.217241in}}%
\pgfpathlineto{\pgfqpoint{2.421672in}{2.197799in}}%
\pgfpathlineto{\pgfqpoint{2.430546in}{2.192414in}}%
\pgfpathlineto{\pgfqpoint{2.439397in}{2.187381in}}%
\pgfpathlineto{\pgfqpoint{2.448226in}{2.182693in}}%
\pgfpathlineto{\pgfqpoint{2.457033in}{2.178344in}}%
\pgfpathlineto{\pgfqpoint{2.443421in}{2.197133in}}%
\pgfpathlineto{\pgfqpoint{2.429799in}{2.216175in}}%
\pgfpathlineto{\pgfqpoint{2.416167in}{2.235472in}}%
\pgfpathlineto{\pgfqpoint{2.402525in}{2.255026in}}%
\pgfpathlineto{\pgfqpoint{2.393662in}{2.260017in}}%
\pgfpathlineto{\pgfqpoint{2.384776in}{2.265356in}}%
\pgfpathlineto{\pgfqpoint{2.375868in}{2.271048in}}%
\pgfpathlineto{\pgfqpoint{2.366936in}{2.277101in}}%
\pgfpathclose%
\pgfusepath{fill}%
\end{pgfscope}%
\begin{pgfscope}%
\pgfpathrectangle{\pgfqpoint{1.254980in}{0.150000in}}{\pgfqpoint{5.490039in}{5.490039in}}%
\pgfusepath{clip}%
\pgfsetbuttcap%
\pgfsetroundjoin%
\definecolor{currentfill}{rgb}{0.352360,0.783011,0.392636}%
\pgfsetfillcolor{currentfill}%
\pgfsetfillopacity{0.700000}%
\pgfsetlinewidth{0.000000pt}%
\definecolor{currentstroke}{rgb}{0.000000,0.000000,0.000000}%
\pgfsetstrokecolor{currentstroke}%
\pgfsetdash{}{0pt}%
\pgfpathmoveto{\pgfqpoint{5.735337in}{3.359548in}}%
\pgfpathlineto{\pgfqpoint{5.749837in}{3.374038in}}%
\pgfpathlineto{\pgfqpoint{5.764359in}{3.388691in}}%
\pgfpathlineto{\pgfqpoint{5.778902in}{3.403505in}}%
\pgfpathlineto{\pgfqpoint{5.793467in}{3.418482in}}%
\pgfpathlineto{\pgfqpoint{5.800427in}{3.419536in}}%
\pgfpathlineto{\pgfqpoint{5.807378in}{3.420509in}}%
\pgfpathlineto{\pgfqpoint{5.814318in}{3.421407in}}%
\pgfpathlineto{\pgfqpoint{5.821250in}{3.422234in}}%
\pgfpathlineto{\pgfqpoint{5.806711in}{3.407679in}}%
\pgfpathlineto{\pgfqpoint{5.792193in}{3.393285in}}%
\pgfpathlineto{\pgfqpoint{5.777698in}{3.379052in}}%
\pgfpathlineto{\pgfqpoint{5.763223in}{3.364980in}}%
\pgfpathlineto{\pgfqpoint{5.756265in}{3.363722in}}%
\pgfpathlineto{\pgfqpoint{5.749298in}{3.362401in}}%
\pgfpathlineto{\pgfqpoint{5.742322in}{3.361011in}}%
\pgfpathlineto{\pgfqpoint{5.735337in}{3.359548in}}%
\pgfpathclose%
\pgfusepath{fill}%
\end{pgfscope}%
\begin{pgfscope}%
\pgfpathrectangle{\pgfqpoint{1.254980in}{0.150000in}}{\pgfqpoint{5.490039in}{5.490039in}}%
\pgfusepath{clip}%
\pgfsetbuttcap%
\pgfsetroundjoin%
\definecolor{currentfill}{rgb}{0.395174,0.797475,0.367757}%
\pgfsetfillcolor{currentfill}%
\pgfsetfillopacity{0.700000}%
\pgfsetlinewidth{0.000000pt}%
\definecolor{currentstroke}{rgb}{0.000000,0.000000,0.000000}%
\pgfsetstrokecolor{currentstroke}%
\pgfsetdash{}{0pt}%
\pgfpathmoveto{\pgfqpoint{5.821250in}{3.422234in}}%
\pgfpathlineto{\pgfqpoint{5.835811in}{3.436951in}}%
\pgfpathlineto{\pgfqpoint{5.850394in}{3.451829in}}%
\pgfpathlineto{\pgfqpoint{5.864999in}{3.466869in}}%
\pgfpathlineto{\pgfqpoint{5.871900in}{3.467299in}}%
\pgfpathlineto{\pgfqpoint{5.878793in}{3.467662in}}%
\pgfpathlineto{\pgfqpoint{5.885675in}{3.467964in}}%
\pgfpathlineto{\pgfqpoint{5.892549in}{3.468208in}}%
\pgfpathlineto{\pgfqpoint{5.877973in}{3.453620in}}%
\pgfpathlineto{\pgfqpoint{5.863418in}{3.439194in}}%
\pgfpathlineto{\pgfqpoint{5.848885in}{3.424928in}}%
\pgfpathlineto{\pgfqpoint{5.841989in}{3.424337in}}%
\pgfpathlineto{\pgfqpoint{5.835085in}{3.423694in}}%
\pgfpathlineto{\pgfqpoint{5.828172in}{3.422995in}}%
\pgfpathlineto{\pgfqpoint{5.821250in}{3.422234in}}%
\pgfpathclose%
\pgfusepath{fill}%
\end{pgfscope}%
\begin{pgfscope}%
\pgfpathrectangle{\pgfqpoint{1.254980in}{0.150000in}}{\pgfqpoint{5.490039in}{5.490039in}}%
\pgfusepath{clip}%
\pgfsetbuttcap%
\pgfsetroundjoin%
\definecolor{currentfill}{rgb}{0.277018,0.050344,0.375715}%
\pgfsetfillcolor{currentfill}%
\pgfsetfillopacity{0.700000}%
\pgfsetlinewidth{0.000000pt}%
\definecolor{currentstroke}{rgb}{0.000000,0.000000,0.000000}%
\pgfsetstrokecolor{currentstroke}%
\pgfsetdash{}{0pt}%
\pgfpathmoveto{\pgfqpoint{3.693171in}{1.546707in}}%
\pgfpathlineto{\pgfqpoint{3.706588in}{1.546473in}}%
\pgfpathlineto{\pgfqpoint{3.720011in}{1.546404in}}%
\pgfpathlineto{\pgfqpoint{3.733442in}{1.546498in}}%
\pgfpathlineto{\pgfqpoint{3.746880in}{1.546755in}}%
\pgfpathlineto{\pgfqpoint{3.754805in}{1.557709in}}%
\pgfpathlineto{\pgfqpoint{3.762723in}{1.568729in}}%
\pgfpathlineto{\pgfqpoint{3.770636in}{1.579811in}}%
\pgfpathlineto{\pgfqpoint{3.778544in}{1.590950in}}%
\pgfpathlineto{\pgfqpoint{3.765116in}{1.590250in}}%
\pgfpathlineto{\pgfqpoint{3.751695in}{1.589714in}}%
\pgfpathlineto{\pgfqpoint{3.738282in}{1.589342in}}%
\pgfpathlineto{\pgfqpoint{3.724876in}{1.589134in}}%
\pgfpathlineto{\pgfqpoint{3.716959in}{1.578427in}}%
\pgfpathlineto{\pgfqpoint{3.709035in}{1.567784in}}%
\pgfpathlineto{\pgfqpoint{3.701106in}{1.557209in}}%
\pgfpathlineto{\pgfqpoint{3.693171in}{1.546707in}}%
\pgfpathclose%
\pgfusepath{fill}%
\end{pgfscope}%
\begin{pgfscope}%
\pgfpathrectangle{\pgfqpoint{1.254980in}{0.150000in}}{\pgfqpoint{5.490039in}{5.490039in}}%
\pgfusepath{clip}%
\pgfsetbuttcap%
\pgfsetroundjoin%
\definecolor{currentfill}{rgb}{0.120092,0.600104,0.542530}%
\pgfsetfillcolor{currentfill}%
\pgfsetfillopacity{0.700000}%
\pgfsetlinewidth{0.000000pt}%
\definecolor{currentstroke}{rgb}{0.000000,0.000000,0.000000}%
\pgfsetstrokecolor{currentstroke}%
\pgfsetdash{}{0pt}%
\pgfpathmoveto{\pgfqpoint{5.075401in}{2.819250in}}%
\pgfpathlineto{\pgfqpoint{5.089479in}{2.831565in}}%
\pgfpathlineto{\pgfqpoint{5.103576in}{2.844041in}}%
\pgfpathlineto{\pgfqpoint{5.117691in}{2.856680in}}%
\pgfpathlineto{\pgfqpoint{5.131824in}{2.869481in}}%
\pgfpathlineto{\pgfqpoint{5.139239in}{2.876764in}}%
\pgfpathlineto{\pgfqpoint{5.146645in}{2.883907in}}%
\pgfpathlineto{\pgfqpoint{5.154042in}{2.890909in}}%
\pgfpathlineto{\pgfqpoint{5.161430in}{2.897775in}}%
\pgfpathlineto{\pgfqpoint{5.147305in}{2.885110in}}%
\pgfpathlineto{\pgfqpoint{5.133198in}{2.872608in}}%
\pgfpathlineto{\pgfqpoint{5.119110in}{2.860267in}}%
\pgfpathlineto{\pgfqpoint{5.105040in}{2.848088in}}%
\pgfpathlineto{\pgfqpoint{5.097643in}{2.841075in}}%
\pgfpathlineto{\pgfqpoint{5.090237in}{2.833933in}}%
\pgfpathlineto{\pgfqpoint{5.082823in}{2.826658in}}%
\pgfpathlineto{\pgfqpoint{5.075401in}{2.819250in}}%
\pgfpathclose%
\pgfusepath{fill}%
\end{pgfscope}%
\begin{pgfscope}%
\pgfpathrectangle{\pgfqpoint{1.254980in}{0.150000in}}{\pgfqpoint{5.490039in}{5.490039in}}%
\pgfusepath{clip}%
\pgfsetbuttcap%
\pgfsetroundjoin%
\definecolor{currentfill}{rgb}{0.280894,0.078907,0.402329}%
\pgfsetfillcolor{currentfill}%
\pgfsetfillopacity{0.700000}%
\pgfsetlinewidth{0.000000pt}%
\definecolor{currentstroke}{rgb}{0.000000,0.000000,0.000000}%
\pgfsetstrokecolor{currentstroke}%
\pgfsetdash{}{0pt}%
\pgfpathmoveto{\pgfqpoint{3.778544in}{1.590950in}}%
\pgfpathlineto{\pgfqpoint{3.791980in}{1.591813in}}%
\pgfpathlineto{\pgfqpoint{3.805425in}{1.592838in}}%
\pgfpathlineto{\pgfqpoint{3.818877in}{1.594027in}}%
\pgfpathlineto{\pgfqpoint{3.832338in}{1.595378in}}%
\pgfpathlineto{\pgfqpoint{3.840232in}{1.606995in}}%
\pgfpathlineto{\pgfqpoint{3.848120in}{1.618655in}}%
\pgfpathlineto{\pgfqpoint{3.856004in}{1.630353in}}%
\pgfpathlineto{\pgfqpoint{3.863883in}{1.642086in}}%
\pgfpathlineto{\pgfqpoint{3.850430in}{1.640319in}}%
\pgfpathlineto{\pgfqpoint{3.836985in}{1.638716in}}%
\pgfpathlineto{\pgfqpoint{3.823550in}{1.637275in}}%
\pgfpathlineto{\pgfqpoint{3.810122in}{1.635998in}}%
\pgfpathlineto{\pgfqpoint{3.802235in}{1.624670in}}%
\pgfpathlineto{\pgfqpoint{3.794344in}{1.613383in}}%
\pgfpathlineto{\pgfqpoint{3.786446in}{1.602142in}}%
\pgfpathlineto{\pgfqpoint{3.778544in}{1.590950in}}%
\pgfpathclose%
\pgfusepath{fill}%
\end{pgfscope}%
\begin{pgfscope}%
\pgfpathrectangle{\pgfqpoint{1.254980in}{0.150000in}}{\pgfqpoint{5.490039in}{5.490039in}}%
\pgfusepath{clip}%
\pgfsetbuttcap%
\pgfsetroundjoin%
\definecolor{currentfill}{rgb}{0.171176,0.452530,0.557965}%
\pgfsetfillcolor{currentfill}%
\pgfsetfillopacity{0.700000}%
\pgfsetlinewidth{0.000000pt}%
\definecolor{currentstroke}{rgb}{0.000000,0.000000,0.000000}%
\pgfsetstrokecolor{currentstroke}%
\pgfsetdash{}{0pt}%
\pgfpathmoveto{\pgfqpoint{4.671596in}{2.420349in}}%
\pgfpathlineto{\pgfqpoint{4.685427in}{2.430249in}}%
\pgfpathlineto{\pgfqpoint{4.699275in}{2.440310in}}%
\pgfpathlineto{\pgfqpoint{4.713138in}{2.450533in}}%
\pgfpathlineto{\pgfqpoint{4.727018in}{2.460917in}}%
\pgfpathlineto{\pgfqpoint{4.734631in}{2.471684in}}%
\pgfpathlineto{\pgfqpoint{4.742237in}{2.482321in}}%
\pgfpathlineto{\pgfqpoint{4.749836in}{2.492827in}}%
\pgfpathlineto{\pgfqpoint{4.757429in}{2.503204in}}%
\pgfpathlineto{\pgfqpoint{4.743552in}{2.492773in}}%
\pgfpathlineto{\pgfqpoint{4.729690in}{2.482503in}}%
\pgfpathlineto{\pgfqpoint{4.715845in}{2.472394in}}%
\pgfpathlineto{\pgfqpoint{4.702015in}{2.462447in}}%
\pgfpathlineto{\pgfqpoint{4.694420in}{2.452107in}}%
\pgfpathlineto{\pgfqpoint{4.686818in}{2.441643in}}%
\pgfpathlineto{\pgfqpoint{4.679210in}{2.431058in}}%
\pgfpathlineto{\pgfqpoint{4.671596in}{2.420349in}}%
\pgfpathclose%
\pgfusepath{fill}%
\end{pgfscope}%
\begin{pgfscope}%
\pgfpathrectangle{\pgfqpoint{1.254980in}{0.150000in}}{\pgfqpoint{5.490039in}{5.490039in}}%
\pgfusepath{clip}%
\pgfsetbuttcap%
\pgfsetroundjoin%
\definecolor{currentfill}{rgb}{0.273809,0.031497,0.358853}%
\pgfsetfillcolor{currentfill}%
\pgfsetfillopacity{0.700000}%
\pgfsetlinewidth{0.000000pt}%
\definecolor{currentstroke}{rgb}{0.000000,0.000000,0.000000}%
\pgfsetstrokecolor{currentstroke}%
\pgfsetdash{}{0pt}%
\pgfpathmoveto{\pgfqpoint{3.103326in}{1.539928in}}%
\pgfpathlineto{\pgfqpoint{3.116726in}{1.531553in}}%
\pgfpathlineto{\pgfqpoint{3.130126in}{1.523361in}}%
\pgfpathlineto{\pgfqpoint{3.143527in}{1.515351in}}%
\pgfpathlineto{\pgfqpoint{3.156928in}{1.507522in}}%
\pgfpathlineto{\pgfqpoint{3.165172in}{1.511421in}}%
\pgfpathlineto{\pgfqpoint{3.173404in}{1.515544in}}%
\pgfpathlineto{\pgfqpoint{3.181624in}{1.519884in}}%
\pgfpathlineto{\pgfqpoint{3.189833in}{1.524436in}}%
\pgfpathlineto{\pgfqpoint{3.176461in}{1.531681in}}%
\pgfpathlineto{\pgfqpoint{3.163090in}{1.539106in}}%
\pgfpathlineto{\pgfqpoint{3.149720in}{1.546714in}}%
\pgfpathlineto{\pgfqpoint{3.136351in}{1.554504in}}%
\pgfpathlineto{\pgfqpoint{3.128113in}{1.550525in}}%
\pgfpathlineto{\pgfqpoint{3.119863in}{1.546766in}}%
\pgfpathlineto{\pgfqpoint{3.111600in}{1.543231in}}%
\pgfpathlineto{\pgfqpoint{3.103326in}{1.539928in}}%
\pgfpathclose%
\pgfusepath{fill}%
\end{pgfscope}%
\begin{pgfscope}%
\pgfpathrectangle{\pgfqpoint{1.254980in}{0.150000in}}{\pgfqpoint{5.490039in}{5.490039in}}%
\pgfusepath{clip}%
\pgfsetbuttcap%
\pgfsetroundjoin%
\definecolor{currentfill}{rgb}{0.272594,0.025563,0.353093}%
\pgfsetfillcolor{currentfill}%
\pgfsetfillopacity{0.700000}%
\pgfsetlinewidth{0.000000pt}%
\definecolor{currentstroke}{rgb}{0.000000,0.000000,0.000000}%
\pgfsetstrokecolor{currentstroke}%
\pgfsetdash{}{0pt}%
\pgfpathmoveto{\pgfqpoint{3.607723in}{1.509972in}}%
\pgfpathlineto{\pgfqpoint{3.621126in}{1.508609in}}%
\pgfpathlineto{\pgfqpoint{3.634535in}{1.507412in}}%
\pgfpathlineto{\pgfqpoint{3.647950in}{1.506380in}}%
\pgfpathlineto{\pgfqpoint{3.661372in}{1.505512in}}%
\pgfpathlineto{\pgfqpoint{3.669331in}{1.515680in}}%
\pgfpathlineto{\pgfqpoint{3.677284in}{1.525938in}}%
\pgfpathlineto{\pgfqpoint{3.685230in}{1.536282in}}%
\pgfpathlineto{\pgfqpoint{3.693171in}{1.546707in}}%
\pgfpathlineto{\pgfqpoint{3.679762in}{1.547105in}}%
\pgfpathlineto{\pgfqpoint{3.666359in}{1.547668in}}%
\pgfpathlineto{\pgfqpoint{3.652963in}{1.548397in}}%
\pgfpathlineto{\pgfqpoint{3.639574in}{1.549290in}}%
\pgfpathlineto{\pgfqpoint{3.631621in}{1.539324in}}%
\pgfpathlineto{\pgfqpoint{3.623661in}{1.529446in}}%
\pgfpathlineto{\pgfqpoint{3.615695in}{1.519660in}}%
\pgfpathlineto{\pgfqpoint{3.607723in}{1.509972in}}%
\pgfpathclose%
\pgfusepath{fill}%
\end{pgfscope}%
\begin{pgfscope}%
\pgfpathrectangle{\pgfqpoint{1.254980in}{0.150000in}}{\pgfqpoint{5.490039in}{5.490039in}}%
\pgfusepath{clip}%
\pgfsetbuttcap%
\pgfsetroundjoin%
\definecolor{currentfill}{rgb}{0.166383,0.690856,0.496502}%
\pgfsetfillcolor{currentfill}%
\pgfsetfillopacity{0.700000}%
\pgfsetlinewidth{0.000000pt}%
\definecolor{currentstroke}{rgb}{0.000000,0.000000,0.000000}%
\pgfsetstrokecolor{currentstroke}%
\pgfsetdash{}{0pt}%
\pgfpathmoveto{\pgfqpoint{5.362840in}{3.072595in}}%
\pgfpathlineto{\pgfqpoint{5.377104in}{3.086121in}}%
\pgfpathlineto{\pgfqpoint{5.391388in}{3.099809in}}%
\pgfpathlineto{\pgfqpoint{5.405693in}{3.113660in}}%
\pgfpathlineto{\pgfqpoint{5.420017in}{3.127673in}}%
\pgfpathlineto{\pgfqpoint{5.427252in}{3.132207in}}%
\pgfpathlineto{\pgfqpoint{5.434478in}{3.136615in}}%
\pgfpathlineto{\pgfqpoint{5.441694in}{3.140900in}}%
\pgfpathlineto{\pgfqpoint{5.448901in}{3.145065in}}%
\pgfpathlineto{\pgfqpoint{5.434591in}{3.131314in}}%
\pgfpathlineto{\pgfqpoint{5.420302in}{3.117725in}}%
\pgfpathlineto{\pgfqpoint{5.406033in}{3.104298in}}%
\pgfpathlineto{\pgfqpoint{5.391783in}{3.091032in}}%
\pgfpathlineto{\pgfqpoint{5.384561in}{3.086595in}}%
\pgfpathlineto{\pgfqpoint{5.377330in}{3.082045in}}%
\pgfpathlineto{\pgfqpoint{5.370089in}{3.077379in}}%
\pgfpathlineto{\pgfqpoint{5.362840in}{3.072595in}}%
\pgfpathclose%
\pgfusepath{fill}%
\end{pgfscope}%
\begin{pgfscope}%
\pgfpathrectangle{\pgfqpoint{1.254980in}{0.150000in}}{\pgfqpoint{5.490039in}{5.490039in}}%
\pgfusepath{clip}%
\pgfsetbuttcap%
\pgfsetroundjoin%
\definecolor{currentfill}{rgb}{0.283091,0.110553,0.431554}%
\pgfsetfillcolor{currentfill}%
\pgfsetfillopacity{0.700000}%
\pgfsetlinewidth{0.000000pt}%
\definecolor{currentstroke}{rgb}{0.000000,0.000000,0.000000}%
\pgfsetstrokecolor{currentstroke}%
\pgfsetdash{}{0pt}%
\pgfpathmoveto{\pgfqpoint{2.855023in}{1.698669in}}%
\pgfpathlineto{\pgfqpoint{2.868481in}{1.686616in}}%
\pgfpathlineto{\pgfqpoint{2.881935in}{1.674763in}}%
\pgfpathlineto{\pgfqpoint{2.895387in}{1.663108in}}%
\pgfpathlineto{\pgfqpoint{2.908836in}{1.651649in}}%
\pgfpathlineto{\pgfqpoint{2.917274in}{1.652082in}}%
\pgfpathlineto{\pgfqpoint{2.925696in}{1.652793in}}%
\pgfpathlineto{\pgfqpoint{2.934104in}{1.653776in}}%
\pgfpathlineto{\pgfqpoint{2.942496in}{1.655025in}}%
\pgfpathlineto{\pgfqpoint{2.929086in}{1.665862in}}%
\pgfpathlineto{\pgfqpoint{2.915673in}{1.676896in}}%
\pgfpathlineto{\pgfqpoint{2.902259in}{1.688128in}}%
\pgfpathlineto{\pgfqpoint{2.888842in}{1.699558in}}%
\pgfpathlineto{\pgfqpoint{2.880411in}{1.698919in}}%
\pgfpathlineto{\pgfqpoint{2.871964in}{1.698554in}}%
\pgfpathlineto{\pgfqpoint{2.863502in}{1.698468in}}%
\pgfpathlineto{\pgfqpoint{2.855023in}{1.698669in}}%
\pgfpathclose%
\pgfusepath{fill}%
\end{pgfscope}%
\begin{pgfscope}%
\pgfpathrectangle{\pgfqpoint{1.254980in}{0.150000in}}{\pgfqpoint{5.490039in}{5.490039in}}%
\pgfusepath{clip}%
\pgfsetbuttcap%
\pgfsetroundjoin%
\definecolor{currentfill}{rgb}{0.140536,0.530132,0.555659}%
\pgfsetfillcolor{currentfill}%
\pgfsetfillopacity{0.700000}%
\pgfsetlinewidth{0.000000pt}%
\definecolor{currentstroke}{rgb}{0.000000,0.000000,0.000000}%
\pgfsetstrokecolor{currentstroke}%
\pgfsetdash{}{0pt}%
\pgfpathmoveto{\pgfqpoint{4.873579in}{2.624915in}}%
\pgfpathlineto{\pgfqpoint{4.887533in}{2.636158in}}%
\pgfpathlineto{\pgfqpoint{4.901505in}{2.647563in}}%
\pgfpathlineto{\pgfqpoint{4.915493in}{2.659129in}}%
\pgfpathlineto{\pgfqpoint{4.929500in}{2.670858in}}%
\pgfpathlineto{\pgfqpoint{4.937023in}{2.680012in}}%
\pgfpathlineto{\pgfqpoint{4.944538in}{2.689024in}}%
\pgfpathlineto{\pgfqpoint{4.952046in}{2.697896in}}%
\pgfpathlineto{\pgfqpoint{4.959546in}{2.706629in}}%
\pgfpathlineto{\pgfqpoint{4.945544in}{2.694945in}}%
\pgfpathlineto{\pgfqpoint{4.931560in}{2.683422in}}%
\pgfpathlineto{\pgfqpoint{4.917594in}{2.672060in}}%
\pgfpathlineto{\pgfqpoint{4.903644in}{2.660860in}}%
\pgfpathlineto{\pgfqpoint{4.896139in}{2.652073in}}%
\pgfpathlineto{\pgfqpoint{4.888626in}{2.643153in}}%
\pgfpathlineto{\pgfqpoint{4.881106in}{2.634101in}}%
\pgfpathlineto{\pgfqpoint{4.873579in}{2.624915in}}%
\pgfpathclose%
\pgfusepath{fill}%
\end{pgfscope}%
\begin{pgfscope}%
\pgfpathrectangle{\pgfqpoint{1.254980in}{0.150000in}}{\pgfqpoint{5.490039in}{5.490039in}}%
\pgfusepath{clip}%
\pgfsetbuttcap%
\pgfsetroundjoin%
\definecolor{currentfill}{rgb}{0.283091,0.110553,0.431554}%
\pgfsetfillcolor{currentfill}%
\pgfsetfillopacity{0.700000}%
\pgfsetlinewidth{0.000000pt}%
\definecolor{currentstroke}{rgb}{0.000000,0.000000,0.000000}%
\pgfsetstrokecolor{currentstroke}%
\pgfsetdash{}{0pt}%
\pgfpathmoveto{\pgfqpoint{3.863883in}{1.642086in}}%
\pgfpathlineto{\pgfqpoint{3.877345in}{1.644014in}}%
\pgfpathlineto{\pgfqpoint{3.890815in}{1.646104in}}%
\pgfpathlineto{\pgfqpoint{3.904295in}{1.648356in}}%
\pgfpathlineto{\pgfqpoint{3.917783in}{1.650770in}}%
\pgfpathlineto{\pgfqpoint{3.925651in}{1.662932in}}%
\pgfpathlineto{\pgfqpoint{3.933513in}{1.675114in}}%
\pgfpathlineto{\pgfqpoint{3.941371in}{1.687312in}}%
\pgfpathlineto{\pgfqpoint{3.949224in}{1.699524in}}%
\pgfpathlineto{\pgfqpoint{3.935741in}{1.696722in}}%
\pgfpathlineto{\pgfqpoint{3.922268in}{1.694081in}}%
\pgfpathlineto{\pgfqpoint{3.908804in}{1.691603in}}%
\pgfpathlineto{\pgfqpoint{3.895349in}{1.689287in}}%
\pgfpathlineto{\pgfqpoint{3.887489in}{1.677453in}}%
\pgfpathlineto{\pgfqpoint{3.879626in}{1.665639in}}%
\pgfpathlineto{\pgfqpoint{3.871757in}{1.653849in}}%
\pgfpathlineto{\pgfqpoint{3.863883in}{1.642086in}}%
\pgfpathclose%
\pgfusepath{fill}%
\end{pgfscope}%
\begin{pgfscope}%
\pgfpathrectangle{\pgfqpoint{1.254980in}{0.150000in}}{\pgfqpoint{5.490039in}{5.490039in}}%
\pgfusepath{clip}%
\pgfsetbuttcap%
\pgfsetroundjoin%
\definecolor{currentfill}{rgb}{0.195860,0.395433,0.555276}%
\pgfsetfillcolor{currentfill}%
\pgfsetfillopacity{0.700000}%
\pgfsetlinewidth{0.000000pt}%
\definecolor{currentstroke}{rgb}{0.000000,0.000000,0.000000}%
\pgfsetstrokecolor{currentstroke}%
\pgfsetdash{}{0pt}%
\pgfpathmoveto{\pgfqpoint{2.312020in}{2.360575in}}%
\pgfpathlineto{\pgfqpoint{2.325766in}{2.339307in}}%
\pgfpathlineto{\pgfqpoint{2.339501in}{2.318307in}}%
\pgfpathlineto{\pgfqpoint{2.353224in}{2.297572in}}%
\pgfpathlineto{\pgfqpoint{2.366936in}{2.277101in}}%
\pgfpathlineto{\pgfqpoint{2.375868in}{2.271048in}}%
\pgfpathlineto{\pgfqpoint{2.384776in}{2.265356in}}%
\pgfpathlineto{\pgfqpoint{2.393662in}{2.260017in}}%
\pgfpathlineto{\pgfqpoint{2.402525in}{2.255026in}}%
\pgfpathlineto{\pgfqpoint{2.388872in}{2.274840in}}%
\pgfpathlineto{\pgfqpoint{2.375209in}{2.294915in}}%
\pgfpathlineto{\pgfqpoint{2.361534in}{2.315254in}}%
\pgfpathlineto{\pgfqpoint{2.347847in}{2.335860in}}%
\pgfpathlineto{\pgfqpoint{2.338926in}{2.341498in}}%
\pgfpathlineto{\pgfqpoint{2.329981in}{2.347492in}}%
\pgfpathlineto{\pgfqpoint{2.321013in}{2.353849in}}%
\pgfpathlineto{\pgfqpoint{2.312020in}{2.360575in}}%
\pgfpathclose%
\pgfusepath{fill}%
\end{pgfscope}%
\begin{pgfscope}%
\pgfpathrectangle{\pgfqpoint{1.254980in}{0.150000in}}{\pgfqpoint{5.490039in}{5.490039in}}%
\pgfusepath{clip}%
\pgfsetbuttcap%
\pgfsetroundjoin%
\definecolor{currentfill}{rgb}{0.265145,0.232956,0.516599}%
\pgfsetfillcolor{currentfill}%
\pgfsetfillopacity{0.700000}%
\pgfsetlinewidth{0.000000pt}%
\definecolor{currentstroke}{rgb}{0.000000,0.000000,0.000000}%
\pgfsetstrokecolor{currentstroke}%
\pgfsetdash{}{0pt}%
\pgfpathmoveto{\pgfqpoint{4.151234in}{1.882089in}}%
\pgfpathlineto{\pgfqpoint{4.164801in}{1.887342in}}%
\pgfpathlineto{\pgfqpoint{4.178379in}{1.892756in}}%
\pgfpathlineto{\pgfqpoint{4.191969in}{1.898330in}}%
\pgfpathlineto{\pgfqpoint{4.205571in}{1.904065in}}%
\pgfpathlineto{\pgfqpoint{4.213357in}{1.917018in}}%
\pgfpathlineto{\pgfqpoint{4.221137in}{1.929920in}}%
\pgfpathlineto{\pgfqpoint{4.228913in}{1.942770in}}%
\pgfpathlineto{\pgfqpoint{4.236685in}{1.955564in}}%
\pgfpathlineto{\pgfqpoint{4.223084in}{1.949550in}}%
\pgfpathlineto{\pgfqpoint{4.209496in}{1.943698in}}%
\pgfpathlineto{\pgfqpoint{4.195919in}{1.938006in}}%
\pgfpathlineto{\pgfqpoint{4.182354in}{1.932475in}}%
\pgfpathlineto{\pgfqpoint{4.174581in}{1.919949in}}%
\pgfpathlineto{\pgfqpoint{4.166803in}{1.907374in}}%
\pgfpathlineto{\pgfqpoint{4.159021in}{1.894753in}}%
\pgfpathlineto{\pgfqpoint{4.151234in}{1.882089in}}%
\pgfpathclose%
\pgfusepath{fill}%
\end{pgfscope}%
\begin{pgfscope}%
\pgfpathrectangle{\pgfqpoint{1.254980in}{0.150000in}}{\pgfqpoint{5.490039in}{5.490039in}}%
\pgfusepath{clip}%
\pgfsetbuttcap%
\pgfsetroundjoin%
\definecolor{currentfill}{rgb}{0.229739,0.322361,0.545706}%
\pgfsetfillcolor{currentfill}%
\pgfsetfillopacity{0.700000}%
\pgfsetlinewidth{0.000000pt}%
\definecolor{currentstroke}{rgb}{0.000000,0.000000,0.000000}%
\pgfsetstrokecolor{currentstroke}%
\pgfsetdash{}{0pt}%
\pgfpathmoveto{\pgfqpoint{4.353209in}{2.083209in}}%
\pgfpathlineto{\pgfqpoint{4.366871in}{2.090499in}}%
\pgfpathlineto{\pgfqpoint{4.380547in}{2.097950in}}%
\pgfpathlineto{\pgfqpoint{4.394236in}{2.105561in}}%
\pgfpathlineto{\pgfqpoint{4.407939in}{2.113333in}}%
\pgfpathlineto{\pgfqpoint{4.415666in}{2.125941in}}%
\pgfpathlineto{\pgfqpoint{4.423389in}{2.138459in}}%
\pgfpathlineto{\pgfqpoint{4.431107in}{2.150886in}}%
\pgfpathlineto{\pgfqpoint{4.438819in}{2.163220in}}%
\pgfpathlineto{\pgfqpoint{4.425117in}{2.155254in}}%
\pgfpathlineto{\pgfqpoint{4.411428in}{2.147449in}}%
\pgfpathlineto{\pgfqpoint{4.397753in}{2.139804in}}%
\pgfpathlineto{\pgfqpoint{4.384092in}{2.132321in}}%
\pgfpathlineto{\pgfqpoint{4.376378in}{2.120170in}}%
\pgfpathlineto{\pgfqpoint{4.368660in}{2.107934in}}%
\pgfpathlineto{\pgfqpoint{4.360937in}{2.095613in}}%
\pgfpathlineto{\pgfqpoint{4.353209in}{2.083209in}}%
\pgfpathclose%
\pgfusepath{fill}%
\end{pgfscope}%
\begin{pgfscope}%
\pgfpathrectangle{\pgfqpoint{1.254980in}{0.150000in}}{\pgfqpoint{5.490039in}{5.490039in}}%
\pgfusepath{clip}%
\pgfsetbuttcap%
\pgfsetroundjoin%
\definecolor{currentfill}{rgb}{0.269944,0.014625,0.341379}%
\pgfsetfillcolor{currentfill}%
\pgfsetfillopacity{0.700000}%
\pgfsetlinewidth{0.000000pt}%
\definecolor{currentstroke}{rgb}{0.000000,0.000000,0.000000}%
\pgfsetstrokecolor{currentstroke}%
\pgfsetdash{}{0pt}%
\pgfpathmoveto{\pgfqpoint{3.522154in}{1.481384in}}%
\pgfpathlineto{\pgfqpoint{3.535549in}{1.478859in}}%
\pgfpathlineto{\pgfqpoint{3.548950in}{1.476500in}}%
\pgfpathlineto{\pgfqpoint{3.562356in}{1.474309in}}%
\pgfpathlineto{\pgfqpoint{3.575767in}{1.472283in}}%
\pgfpathlineto{\pgfqpoint{3.583766in}{1.481536in}}%
\pgfpathlineto{\pgfqpoint{3.591759in}{1.490905in}}%
\pgfpathlineto{\pgfqpoint{3.599744in}{1.500385in}}%
\pgfpathlineto{\pgfqpoint{3.607723in}{1.509972in}}%
\pgfpathlineto{\pgfqpoint{3.594326in}{1.511501in}}%
\pgfpathlineto{\pgfqpoint{3.580936in}{1.513196in}}%
\pgfpathlineto{\pgfqpoint{3.567551in}{1.515057in}}%
\pgfpathlineto{\pgfqpoint{3.554171in}{1.517086in}}%
\pgfpathlineto{\pgfqpoint{3.546178in}{1.507986in}}%
\pgfpathlineto{\pgfqpoint{3.538177in}{1.498999in}}%
\pgfpathlineto{\pgfqpoint{3.530169in}{1.490130in}}%
\pgfpathlineto{\pgfqpoint{3.522154in}{1.481384in}}%
\pgfpathclose%
\pgfusepath{fill}%
\end{pgfscope}%
\begin{pgfscope}%
\pgfpathrectangle{\pgfqpoint{1.254980in}{0.150000in}}{\pgfqpoint{5.490039in}{5.490039in}}%
\pgfusepath{clip}%
\pgfsetbuttcap%
\pgfsetroundjoin%
\definecolor{currentfill}{rgb}{0.282623,0.140926,0.457517}%
\pgfsetfillcolor{currentfill}%
\pgfsetfillopacity{0.700000}%
\pgfsetlinewidth{0.000000pt}%
\definecolor{currentstroke}{rgb}{0.000000,0.000000,0.000000}%
\pgfsetstrokecolor{currentstroke}%
\pgfsetdash{}{0pt}%
\pgfpathmoveto{\pgfqpoint{3.949224in}{1.699524in}}%
\pgfpathlineto{\pgfqpoint{3.962717in}{1.702487in}}%
\pgfpathlineto{\pgfqpoint{3.976219in}{1.705612in}}%
\pgfpathlineto{\pgfqpoint{3.989731in}{1.708898in}}%
\pgfpathlineto{\pgfqpoint{4.003253in}{1.712345in}}%
\pgfpathlineto{\pgfqpoint{4.011097in}{1.724938in}}%
\pgfpathlineto{\pgfqpoint{4.018936in}{1.737529in}}%
\pgfpathlineto{\pgfqpoint{4.026771in}{1.750116in}}%
\pgfpathlineto{\pgfqpoint{4.034601in}{1.762695in}}%
\pgfpathlineto{\pgfqpoint{4.021083in}{1.758887in}}%
\pgfpathlineto{\pgfqpoint{4.007575in}{1.755239in}}%
\pgfpathlineto{\pgfqpoint{3.994078in}{1.751754in}}%
\pgfpathlineto{\pgfqpoint{3.980590in}{1.748429in}}%
\pgfpathlineto{\pgfqpoint{3.972756in}{1.736200in}}%
\pgfpathlineto{\pgfqpoint{3.964917in}{1.723971in}}%
\pgfpathlineto{\pgfqpoint{3.957073in}{1.711744in}}%
\pgfpathlineto{\pgfqpoint{3.949224in}{1.699524in}}%
\pgfpathclose%
\pgfusepath{fill}%
\end{pgfscope}%
\begin{pgfscope}%
\pgfpathrectangle{\pgfqpoint{1.254980in}{0.150000in}}{\pgfqpoint{5.490039in}{5.490039in}}%
\pgfusepath{clip}%
\pgfsetbuttcap%
\pgfsetroundjoin%
\definecolor{currentfill}{rgb}{0.129933,0.559582,0.551864}%
\pgfsetfillcolor{currentfill}%
\pgfsetfillopacity{0.700000}%
\pgfsetlinewidth{0.000000pt}%
\definecolor{currentstroke}{rgb}{0.000000,0.000000,0.000000}%
\pgfsetstrokecolor{currentstroke}%
\pgfsetdash{}{0pt}%
\pgfpathmoveto{\pgfqpoint{2.071242in}{2.807896in}}%
\pgfpathlineto{\pgfqpoint{2.085223in}{2.781403in}}%
\pgfpathlineto{\pgfqpoint{2.099186in}{2.755232in}}%
\pgfpathlineto{\pgfqpoint{2.113131in}{2.729381in}}%
\pgfpathlineto{\pgfqpoint{2.127060in}{2.703846in}}%
\pgfpathlineto{\pgfqpoint{2.136209in}{2.695943in}}%
\pgfpathlineto{\pgfqpoint{2.145332in}{2.688417in}}%
\pgfpathlineto{\pgfqpoint{2.154429in}{2.681264in}}%
\pgfpathlineto{\pgfqpoint{2.163501in}{2.674476in}}%
\pgfpathlineto{\pgfqpoint{2.149639in}{2.699356in}}%
\pgfpathlineto{\pgfqpoint{2.135760in}{2.724550in}}%
\pgfpathlineto{\pgfqpoint{2.121865in}{2.750063in}}%
\pgfpathlineto{\pgfqpoint{2.107952in}{2.775896in}}%
\pgfpathlineto{\pgfqpoint{2.098815in}{2.783328in}}%
\pgfpathlineto{\pgfqpoint{2.089651in}{2.791134in}}%
\pgfpathlineto{\pgfqpoint{2.080460in}{2.799321in}}%
\pgfpathlineto{\pgfqpoint{2.071242in}{2.807896in}}%
\pgfpathclose%
\pgfusepath{fill}%
\end{pgfscope}%
\begin{pgfscope}%
\pgfpathrectangle{\pgfqpoint{1.254980in}{0.150000in}}{\pgfqpoint{5.490039in}{5.490039in}}%
\pgfusepath{clip}%
\pgfsetbuttcap%
\pgfsetroundjoin%
\definecolor{currentfill}{rgb}{0.281924,0.089666,0.412415}%
\pgfsetfillcolor{currentfill}%
\pgfsetfillopacity{0.700000}%
\pgfsetlinewidth{0.000000pt}%
\definecolor{currentstroke}{rgb}{0.000000,0.000000,0.000000}%
\pgfsetstrokecolor{currentstroke}%
\pgfsetdash{}{0pt}%
\pgfpathmoveto{\pgfqpoint{2.908836in}{1.651649in}}%
\pgfpathlineto{\pgfqpoint{2.922283in}{1.640387in}}%
\pgfpathlineto{\pgfqpoint{2.935728in}{1.629320in}}%
\pgfpathlineto{\pgfqpoint{2.949171in}{1.618447in}}%
\pgfpathlineto{\pgfqpoint{2.962612in}{1.607766in}}%
\pgfpathlineto{\pgfqpoint{2.971011in}{1.608830in}}%
\pgfpathlineto{\pgfqpoint{2.979395in}{1.610164in}}%
\pgfpathlineto{\pgfqpoint{2.987765in}{1.611763in}}%
\pgfpathlineto{\pgfqpoint{2.996121in}{1.613619in}}%
\pgfpathlineto{\pgfqpoint{2.982717in}{1.623681in}}%
\pgfpathlineto{\pgfqpoint{2.969311in}{1.633935in}}%
\pgfpathlineto{\pgfqpoint{2.955905in}{1.644383in}}%
\pgfpathlineto{\pgfqpoint{2.942496in}{1.655025in}}%
\pgfpathlineto{\pgfqpoint{2.934104in}{1.653776in}}%
\pgfpathlineto{\pgfqpoint{2.925696in}{1.652793in}}%
\pgfpathlineto{\pgfqpoint{2.917274in}{1.652082in}}%
\pgfpathlineto{\pgfqpoint{2.908836in}{1.651649in}}%
\pgfpathclose%
\pgfusepath{fill}%
\end{pgfscope}%
\begin{pgfscope}%
\pgfpathrectangle{\pgfqpoint{1.254980in}{0.150000in}}{\pgfqpoint{5.490039in}{5.490039in}}%
\pgfusepath{clip}%
\pgfsetbuttcap%
\pgfsetroundjoin%
\definecolor{currentfill}{rgb}{0.267004,0.004874,0.329415}%
\pgfsetfillcolor{currentfill}%
\pgfsetfillopacity{0.700000}%
\pgfsetlinewidth{0.000000pt}%
\definecolor{currentstroke}{rgb}{0.000000,0.000000,0.000000}%
\pgfsetstrokecolor{currentstroke}%
\pgfsetdash{}{0pt}%
\pgfpathmoveto{\pgfqpoint{3.296869in}{1.472904in}}%
\pgfpathlineto{\pgfqpoint{3.310259in}{1.467255in}}%
\pgfpathlineto{\pgfqpoint{3.323651in}{1.461779in}}%
\pgfpathlineto{\pgfqpoint{3.337046in}{1.456477in}}%
\pgfpathlineto{\pgfqpoint{3.350444in}{1.451347in}}%
\pgfpathlineto{\pgfqpoint{3.358565in}{1.457799in}}%
\pgfpathlineto{\pgfqpoint{3.366677in}{1.464429in}}%
\pgfpathlineto{\pgfqpoint{3.374780in}{1.471233in}}%
\pgfpathlineto{\pgfqpoint{3.382874in}{1.478204in}}%
\pgfpathlineto{\pgfqpoint{3.369499in}{1.482781in}}%
\pgfpathlineto{\pgfqpoint{3.356127in}{1.487530in}}%
\pgfpathlineto{\pgfqpoint{3.342758in}{1.492452in}}%
\pgfpathlineto{\pgfqpoint{3.329393in}{1.497548in}}%
\pgfpathlineto{\pgfqpoint{3.321276in}{1.491119in}}%
\pgfpathlineto{\pgfqpoint{3.313150in}{1.484865in}}%
\pgfpathlineto{\pgfqpoint{3.305015in}{1.478792in}}%
\pgfpathlineto{\pgfqpoint{3.296869in}{1.472904in}}%
\pgfpathclose%
\pgfusepath{fill}%
\end{pgfscope}%
\begin{pgfscope}%
\pgfpathrectangle{\pgfqpoint{1.254980in}{0.150000in}}{\pgfqpoint{5.490039in}{5.490039in}}%
\pgfusepath{clip}%
\pgfsetbuttcap%
\pgfsetroundjoin%
\definecolor{currentfill}{rgb}{0.188923,0.410910,0.556326}%
\pgfsetfillcolor{currentfill}%
\pgfsetfillopacity{0.700000}%
\pgfsetlinewidth{0.000000pt}%
\definecolor{currentstroke}{rgb}{0.000000,0.000000,0.000000}%
\pgfsetstrokecolor{currentstroke}%
\pgfsetdash{}{0pt}%
\pgfpathmoveto{\pgfqpoint{4.555277in}{2.293058in}}%
\pgfpathlineto{\pgfqpoint{4.569050in}{2.302130in}}%
\pgfpathlineto{\pgfqpoint{4.582839in}{2.311362in}}%
\pgfpathlineto{\pgfqpoint{4.596642in}{2.320756in}}%
\pgfpathlineto{\pgfqpoint{4.610460in}{2.330311in}}%
\pgfpathlineto{\pgfqpoint{4.618123in}{2.341986in}}%
\pgfpathlineto{\pgfqpoint{4.625780in}{2.353543in}}%
\pgfpathlineto{\pgfqpoint{4.633431in}{2.364979in}}%
\pgfpathlineto{\pgfqpoint{4.641076in}{2.376296in}}%
\pgfpathlineto{\pgfqpoint{4.627258in}{2.366634in}}%
\pgfpathlineto{\pgfqpoint{4.613456in}{2.357134in}}%
\pgfpathlineto{\pgfqpoint{4.599669in}{2.347795in}}%
\pgfpathlineto{\pgfqpoint{4.585897in}{2.338617in}}%
\pgfpathlineto{\pgfqpoint{4.578251in}{2.327396in}}%
\pgfpathlineto{\pgfqpoint{4.570599in}{2.316062in}}%
\pgfpathlineto{\pgfqpoint{4.562941in}{2.304616in}}%
\pgfpathlineto{\pgfqpoint{4.555277in}{2.293058in}}%
\pgfpathclose%
\pgfusepath{fill}%
\end{pgfscope}%
\begin{pgfscope}%
\pgfpathrectangle{\pgfqpoint{1.254980in}{0.150000in}}{\pgfqpoint{5.490039in}{5.490039in}}%
\pgfusepath{clip}%
\pgfsetbuttcap%
\pgfsetroundjoin%
\definecolor{currentfill}{rgb}{0.208030,0.718701,0.472873}%
\pgfsetfillcolor{currentfill}%
\pgfsetfillopacity{0.700000}%
\pgfsetlinewidth{0.000000pt}%
\definecolor{currentstroke}{rgb}{0.000000,0.000000,0.000000}%
\pgfsetstrokecolor{currentstroke}%
\pgfsetdash{}{0pt}%
\pgfpathmoveto{\pgfqpoint{5.448901in}{3.145065in}}%
\pgfpathlineto{\pgfqpoint{5.463230in}{3.158977in}}%
\pgfpathlineto{\pgfqpoint{5.477580in}{3.173052in}}%
\pgfpathlineto{\pgfqpoint{5.491951in}{3.187289in}}%
\pgfpathlineto{\pgfqpoint{5.506342in}{3.201690in}}%
\pgfpathlineto{\pgfqpoint{5.513523in}{3.205455in}}%
\pgfpathlineto{\pgfqpoint{5.520694in}{3.209099in}}%
\pgfpathlineto{\pgfqpoint{5.527855in}{3.212625in}}%
\pgfpathlineto{\pgfqpoint{5.535006in}{3.216038in}}%
\pgfpathlineto{\pgfqpoint{5.520632in}{3.201933in}}%
\pgfpathlineto{\pgfqpoint{5.506279in}{3.187990in}}%
\pgfpathlineto{\pgfqpoint{5.491946in}{3.174208in}}%
\pgfpathlineto{\pgfqpoint{5.477633in}{3.160588in}}%
\pgfpathlineto{\pgfqpoint{5.470464in}{3.156871in}}%
\pgfpathlineto{\pgfqpoint{5.463286in}{3.153047in}}%
\pgfpathlineto{\pgfqpoint{5.456098in}{3.149113in}}%
\pgfpathlineto{\pgfqpoint{5.448901in}{3.145065in}}%
\pgfpathclose%
\pgfusepath{fill}%
\end{pgfscope}%
\begin{pgfscope}%
\pgfpathrectangle{\pgfqpoint{1.254980in}{0.150000in}}{\pgfqpoint{5.490039in}{5.490039in}}%
\pgfusepath{clip}%
\pgfsetbuttcap%
\pgfsetroundjoin%
\definecolor{currentfill}{rgb}{0.122312,0.633153,0.530398}%
\pgfsetfillcolor{currentfill}%
\pgfsetfillopacity{0.700000}%
\pgfsetlinewidth{0.000000pt}%
\definecolor{currentstroke}{rgb}{0.000000,0.000000,0.000000}%
\pgfsetstrokecolor{currentstroke}%
\pgfsetdash{}{0pt}%
\pgfpathmoveto{\pgfqpoint{5.161430in}{2.897775in}}%
\pgfpathlineto{\pgfqpoint{5.175575in}{2.910601in}}%
\pgfpathlineto{\pgfqpoint{5.189738in}{2.923590in}}%
\pgfpathlineto{\pgfqpoint{5.203920in}{2.936740in}}%
\pgfpathlineto{\pgfqpoint{5.218121in}{2.950054in}}%
\pgfpathlineto{\pgfqpoint{5.225492in}{2.956627in}}%
\pgfpathlineto{\pgfqpoint{5.232853in}{2.963059in}}%
\pgfpathlineto{\pgfqpoint{5.240206in}{2.969352in}}%
\pgfpathlineto{\pgfqpoint{5.247550in}{2.975508in}}%
\pgfpathlineto{\pgfqpoint{5.233359in}{2.962363in}}%
\pgfpathlineto{\pgfqpoint{5.219186in}{2.949380in}}%
\pgfpathlineto{\pgfqpoint{5.205033in}{2.936558in}}%
\pgfpathlineto{\pgfqpoint{5.190899in}{2.923899in}}%
\pgfpathlineto{\pgfqpoint{5.183545in}{2.917564in}}%
\pgfpathlineto{\pgfqpoint{5.176182in}{2.911100in}}%
\pgfpathlineto{\pgfqpoint{5.168810in}{2.904504in}}%
\pgfpathlineto{\pgfqpoint{5.161430in}{2.897775in}}%
\pgfpathclose%
\pgfusepath{fill}%
\end{pgfscope}%
\begin{pgfscope}%
\pgfpathrectangle{\pgfqpoint{1.254980in}{0.150000in}}{\pgfqpoint{5.490039in}{5.490039in}}%
\pgfusepath{clip}%
\pgfsetbuttcap%
\pgfsetroundjoin%
\definecolor{currentfill}{rgb}{0.179019,0.433756,0.557430}%
\pgfsetfillcolor{currentfill}%
\pgfsetfillopacity{0.700000}%
\pgfsetlinewidth{0.000000pt}%
\definecolor{currentstroke}{rgb}{0.000000,0.000000,0.000000}%
\pgfsetstrokecolor{currentstroke}%
\pgfsetdash{}{0pt}%
\pgfpathmoveto{\pgfqpoint{2.256907in}{2.448374in}}%
\pgfpathlineto{\pgfqpoint{2.270704in}{2.426010in}}%
\pgfpathlineto{\pgfqpoint{2.284489in}{2.403924in}}%
\pgfpathlineto{\pgfqpoint{2.298261in}{2.382114in}}%
\pgfpathlineto{\pgfqpoint{2.312020in}{2.360575in}}%
\pgfpathlineto{\pgfqpoint{2.321013in}{2.353849in}}%
\pgfpathlineto{\pgfqpoint{2.329981in}{2.347492in}}%
\pgfpathlineto{\pgfqpoint{2.338926in}{2.341498in}}%
\pgfpathlineto{\pgfqpoint{2.347847in}{2.335860in}}%
\pgfpathlineto{\pgfqpoint{2.334149in}{2.356734in}}%
\pgfpathlineto{\pgfqpoint{2.320439in}{2.377879in}}%
\pgfpathlineto{\pgfqpoint{2.306716in}{2.399298in}}%
\pgfpathlineto{\pgfqpoint{2.292981in}{2.420993in}}%
\pgfpathlineto{\pgfqpoint{2.283999in}{2.427284in}}%
\pgfpathlineto{\pgfqpoint{2.274993in}{2.433940in}}%
\pgfpathlineto{\pgfqpoint{2.265963in}{2.440968in}}%
\pgfpathlineto{\pgfqpoint{2.256907in}{2.448374in}}%
\pgfpathclose%
\pgfusepath{fill}%
\end{pgfscope}%
\begin{pgfscope}%
\pgfpathrectangle{\pgfqpoint{1.254980in}{0.150000in}}{\pgfqpoint{5.490039in}{5.490039in}}%
\pgfusepath{clip}%
\pgfsetbuttcap%
\pgfsetroundjoin%
\definecolor{currentfill}{rgb}{0.271305,0.019942,0.347269}%
\pgfsetfillcolor{currentfill}%
\pgfsetfillopacity{0.700000}%
\pgfsetlinewidth{0.000000pt}%
\definecolor{currentstroke}{rgb}{0.000000,0.000000,0.000000}%
\pgfsetstrokecolor{currentstroke}%
\pgfsetdash{}{0pt}%
\pgfpathmoveto{\pgfqpoint{3.156928in}{1.507522in}}%
\pgfpathlineto{\pgfqpoint{3.170331in}{1.499874in}}%
\pgfpathlineto{\pgfqpoint{3.183734in}{1.492405in}}%
\pgfpathlineto{\pgfqpoint{3.197139in}{1.485116in}}%
\pgfpathlineto{\pgfqpoint{3.210545in}{1.478004in}}%
\pgfpathlineto{\pgfqpoint{3.218759in}{1.482498in}}%
\pgfpathlineto{\pgfqpoint{3.226962in}{1.487208in}}%
\pgfpathlineto{\pgfqpoint{3.235154in}{1.492128in}}%
\pgfpathlineto{\pgfqpoint{3.243336in}{1.497252in}}%
\pgfpathlineto{\pgfqpoint{3.229958in}{1.503780in}}%
\pgfpathlineto{\pgfqpoint{3.216581in}{1.510487in}}%
\pgfpathlineto{\pgfqpoint{3.203207in}{1.517372in}}%
\pgfpathlineto{\pgfqpoint{3.189833in}{1.524436in}}%
\pgfpathlineto{\pgfqpoint{3.181624in}{1.519884in}}%
\pgfpathlineto{\pgfqpoint{3.173404in}{1.515544in}}%
\pgfpathlineto{\pgfqpoint{3.165172in}{1.511421in}}%
\pgfpathlineto{\pgfqpoint{3.156928in}{1.507522in}}%
\pgfpathclose%
\pgfusepath{fill}%
\end{pgfscope}%
\begin{pgfscope}%
\pgfpathrectangle{\pgfqpoint{1.254980in}{0.150000in}}{\pgfqpoint{5.490039in}{5.490039in}}%
\pgfusepath{clip}%
\pgfsetbuttcap%
\pgfsetroundjoin%
\definecolor{currentfill}{rgb}{0.156270,0.489624,0.557936}%
\pgfsetfillcolor{currentfill}%
\pgfsetfillopacity{0.700000}%
\pgfsetlinewidth{0.000000pt}%
\definecolor{currentstroke}{rgb}{0.000000,0.000000,0.000000}%
\pgfsetstrokecolor{currentstroke}%
\pgfsetdash{}{0pt}%
\pgfpathmoveto{\pgfqpoint{4.757429in}{2.503204in}}%
\pgfpathlineto{\pgfqpoint{4.771323in}{2.513797in}}%
\pgfpathlineto{\pgfqpoint{4.785234in}{2.524551in}}%
\pgfpathlineto{\pgfqpoint{4.799160in}{2.535467in}}%
\pgfpathlineto{\pgfqpoint{4.813104in}{2.546545in}}%
\pgfpathlineto{\pgfqpoint{4.820688in}{2.556820in}}%
\pgfpathlineto{\pgfqpoint{4.828265in}{2.566957in}}%
\pgfpathlineto{\pgfqpoint{4.835835in}{2.576958in}}%
\pgfpathlineto{\pgfqpoint{4.843398in}{2.586821in}}%
\pgfpathlineto{\pgfqpoint{4.829457in}{2.575727in}}%
\pgfpathlineto{\pgfqpoint{4.815533in}{2.564794in}}%
\pgfpathlineto{\pgfqpoint{4.801625in}{2.554022in}}%
\pgfpathlineto{\pgfqpoint{4.787734in}{2.543412in}}%
\pgfpathlineto{\pgfqpoint{4.780168in}{2.533555in}}%
\pgfpathlineto{\pgfqpoint{4.772595in}{2.523567in}}%
\pgfpathlineto{\pgfqpoint{4.765016in}{2.513451in}}%
\pgfpathlineto{\pgfqpoint{4.757429in}{2.503204in}}%
\pgfpathclose%
\pgfusepath{fill}%
\end{pgfscope}%
\begin{pgfscope}%
\pgfpathrectangle{\pgfqpoint{1.254980in}{0.150000in}}{\pgfqpoint{5.490039in}{5.490039in}}%
\pgfusepath{clip}%
\pgfsetbuttcap%
\pgfsetroundjoin%
\definecolor{currentfill}{rgb}{0.267004,0.004874,0.329415}%
\pgfsetfillcolor{currentfill}%
\pgfsetfillopacity{0.700000}%
\pgfsetlinewidth{0.000000pt}%
\definecolor{currentstroke}{rgb}{0.000000,0.000000,0.000000}%
\pgfsetstrokecolor{currentstroke}%
\pgfsetdash{}{0pt}%
\pgfpathmoveto{\pgfqpoint{3.436413in}{1.461610in}}%
\pgfpathlineto{\pgfqpoint{3.449807in}{1.457887in}}%
\pgfpathlineto{\pgfqpoint{3.463206in}{1.454333in}}%
\pgfpathlineto{\pgfqpoint{3.476610in}{1.450948in}}%
\pgfpathlineto{\pgfqpoint{3.490018in}{1.447730in}}%
\pgfpathlineto{\pgfqpoint{3.498063in}{1.455934in}}%
\pgfpathlineto{\pgfqpoint{3.506101in}{1.464281in}}%
\pgfpathlineto{\pgfqpoint{3.514131in}{1.472766in}}%
\pgfpathlineto{\pgfqpoint{3.522154in}{1.481384in}}%
\pgfpathlineto{\pgfqpoint{3.508764in}{1.484077in}}%
\pgfpathlineto{\pgfqpoint{3.495378in}{1.486938in}}%
\pgfpathlineto{\pgfqpoint{3.481998in}{1.489967in}}%
\pgfpathlineto{\pgfqpoint{3.468622in}{1.493166in}}%
\pgfpathlineto{\pgfqpoint{3.460582in}{1.485062in}}%
\pgfpathlineto{\pgfqpoint{3.452534in}{1.477098in}}%
\pgfpathlineto{\pgfqpoint{3.444477in}{1.469279in}}%
\pgfpathlineto{\pgfqpoint{3.436413in}{1.461610in}}%
\pgfpathclose%
\pgfusepath{fill}%
\end{pgfscope}%
\begin{pgfscope}%
\pgfpathrectangle{\pgfqpoint{1.254980in}{0.150000in}}{\pgfqpoint{5.490039in}{5.490039in}}%
\pgfusepath{clip}%
\pgfsetbuttcap%
\pgfsetroundjoin%
\definecolor{currentfill}{rgb}{0.128729,0.563265,0.551229}%
\pgfsetfillcolor{currentfill}%
\pgfsetfillopacity{0.700000}%
\pgfsetlinewidth{0.000000pt}%
\definecolor{currentstroke}{rgb}{0.000000,0.000000,0.000000}%
\pgfsetstrokecolor{currentstroke}%
\pgfsetdash{}{0pt}%
\pgfpathmoveto{\pgfqpoint{4.959546in}{2.706629in}}%
\pgfpathlineto{\pgfqpoint{4.973565in}{2.718476in}}%
\pgfpathlineto{\pgfqpoint{4.987602in}{2.730485in}}%
\pgfpathlineto{\pgfqpoint{5.001657in}{2.742655in}}%
\pgfpathlineto{\pgfqpoint{5.015730in}{2.754988in}}%
\pgfpathlineto{\pgfqpoint{5.023217in}{2.763519in}}%
\pgfpathlineto{\pgfqpoint{5.030696in}{2.771906in}}%
\pgfpathlineto{\pgfqpoint{5.038167in}{2.780148in}}%
\pgfpathlineto{\pgfqpoint{5.045630in}{2.788248in}}%
\pgfpathlineto{\pgfqpoint{5.031563in}{2.775990in}}%
\pgfpathlineto{\pgfqpoint{5.017514in}{2.763894in}}%
\pgfpathlineto{\pgfqpoint{5.003483in}{2.751960in}}%
\pgfpathlineto{\pgfqpoint{4.989469in}{2.740188in}}%
\pgfpathlineto{\pgfqpoint{4.982000in}{2.732002in}}%
\pgfpathlineto{\pgfqpoint{4.974523in}{2.723681in}}%
\pgfpathlineto{\pgfqpoint{4.967038in}{2.715224in}}%
\pgfpathlineto{\pgfqpoint{4.959546in}{2.706629in}}%
\pgfpathclose%
\pgfusepath{fill}%
\end{pgfscope}%
\begin{pgfscope}%
\pgfpathrectangle{\pgfqpoint{1.254980in}{0.150000in}}{\pgfqpoint{5.490039in}{5.490039in}}%
\pgfusepath{clip}%
\pgfsetbuttcap%
\pgfsetroundjoin%
\definecolor{currentfill}{rgb}{0.252194,0.269783,0.531579}%
\pgfsetfillcolor{currentfill}%
\pgfsetfillopacity{0.700000}%
\pgfsetlinewidth{0.000000pt}%
\definecolor{currentstroke}{rgb}{0.000000,0.000000,0.000000}%
\pgfsetstrokecolor{currentstroke}%
\pgfsetdash{}{0pt}%
\pgfpathmoveto{\pgfqpoint{4.236685in}{1.955564in}}%
\pgfpathlineto{\pgfqpoint{4.250298in}{1.961738in}}%
\pgfpathlineto{\pgfqpoint{4.263923in}{1.968073in}}%
\pgfpathlineto{\pgfqpoint{4.277561in}{1.974568in}}%
\pgfpathlineto{\pgfqpoint{4.291211in}{1.981224in}}%
\pgfpathlineto{\pgfqpoint{4.298977in}{1.994222in}}%
\pgfpathlineto{\pgfqpoint{4.306739in}{2.007153in}}%
\pgfpathlineto{\pgfqpoint{4.314496in}{2.020014in}}%
\pgfpathlineto{\pgfqpoint{4.322248in}{2.032804in}}%
\pgfpathlineto{\pgfqpoint{4.308598in}{2.025897in}}%
\pgfpathlineto{\pgfqpoint{4.294961in}{2.019151in}}%
\pgfpathlineto{\pgfqpoint{4.281337in}{2.012566in}}%
\pgfpathlineto{\pgfqpoint{4.267725in}{2.006141in}}%
\pgfpathlineto{\pgfqpoint{4.259972in}{1.993591in}}%
\pgfpathlineto{\pgfqpoint{4.252214in}{1.980977in}}%
\pgfpathlineto{\pgfqpoint{4.244452in}{1.968300in}}%
\pgfpathlineto{\pgfqpoint{4.236685in}{1.955564in}}%
\pgfpathclose%
\pgfusepath{fill}%
\end{pgfscope}%
\begin{pgfscope}%
\pgfpathrectangle{\pgfqpoint{1.254980in}{0.150000in}}{\pgfqpoint{5.490039in}{5.490039in}}%
\pgfusepath{clip}%
\pgfsetbuttcap%
\pgfsetroundjoin%
\definecolor{currentfill}{rgb}{0.278012,0.180367,0.486697}%
\pgfsetfillcolor{currentfill}%
\pgfsetfillopacity{0.700000}%
\pgfsetlinewidth{0.000000pt}%
\definecolor{currentstroke}{rgb}{0.000000,0.000000,0.000000}%
\pgfsetstrokecolor{currentstroke}%
\pgfsetdash{}{0pt}%
\pgfpathmoveto{\pgfqpoint{4.034601in}{1.762695in}}%
\pgfpathlineto{\pgfqpoint{4.048130in}{1.766665in}}%
\pgfpathlineto{\pgfqpoint{4.061668in}{1.770795in}}%
\pgfpathlineto{\pgfqpoint{4.075218in}{1.775086in}}%
\pgfpathlineto{\pgfqpoint{4.088778in}{1.779537in}}%
\pgfpathlineto{\pgfqpoint{4.096601in}{1.792450in}}%
\pgfpathlineto{\pgfqpoint{4.104419in}{1.805343in}}%
\pgfpathlineto{\pgfqpoint{4.112233in}{1.818211in}}%
\pgfpathlineto{\pgfqpoint{4.120042in}{1.831052in}}%
\pgfpathlineto{\pgfqpoint{4.106484in}{1.826266in}}%
\pgfpathlineto{\pgfqpoint{4.092937in}{1.821641in}}%
\pgfpathlineto{\pgfqpoint{4.079402in}{1.817177in}}%
\pgfpathlineto{\pgfqpoint{4.065877in}{1.812874in}}%
\pgfpathlineto{\pgfqpoint{4.058065in}{1.800356in}}%
\pgfpathlineto{\pgfqpoint{4.050248in}{1.787818in}}%
\pgfpathlineto{\pgfqpoint{4.042427in}{1.775264in}}%
\pgfpathlineto{\pgfqpoint{4.034601in}{1.762695in}}%
\pgfpathclose%
\pgfusepath{fill}%
\end{pgfscope}%
\begin{pgfscope}%
\pgfpathrectangle{\pgfqpoint{1.254980in}{0.150000in}}{\pgfqpoint{5.490039in}{5.490039in}}%
\pgfusepath{clip}%
\pgfsetbuttcap%
\pgfsetroundjoin%
\definecolor{currentfill}{rgb}{0.280267,0.073417,0.397163}%
\pgfsetfillcolor{currentfill}%
\pgfsetfillopacity{0.700000}%
\pgfsetlinewidth{0.000000pt}%
\definecolor{currentstroke}{rgb}{0.000000,0.000000,0.000000}%
\pgfsetstrokecolor{currentstroke}%
\pgfsetdash{}{0pt}%
\pgfpathmoveto{\pgfqpoint{2.962612in}{1.607766in}}%
\pgfpathlineto{\pgfqpoint{2.976052in}{1.597277in}}%
\pgfpathlineto{\pgfqpoint{2.989490in}{1.586979in}}%
\pgfpathlineto{\pgfqpoint{3.002927in}{1.576871in}}%
\pgfpathlineto{\pgfqpoint{3.016362in}{1.566951in}}%
\pgfpathlineto{\pgfqpoint{3.024724in}{1.568644in}}%
\pgfpathlineto{\pgfqpoint{3.033072in}{1.570600in}}%
\pgfpathlineto{\pgfqpoint{3.041406in}{1.572812in}}%
\pgfpathlineto{\pgfqpoint{3.049727in}{1.575274in}}%
\pgfpathlineto{\pgfqpoint{3.036326in}{1.584577in}}%
\pgfpathlineto{\pgfqpoint{3.022925in}{1.594068in}}%
\pgfpathlineto{\pgfqpoint{3.009523in}{1.603749in}}%
\pgfpathlineto{\pgfqpoint{2.996121in}{1.613619in}}%
\pgfpathlineto{\pgfqpoint{2.987765in}{1.611763in}}%
\pgfpathlineto{\pgfqpoint{2.979395in}{1.610164in}}%
\pgfpathlineto{\pgfqpoint{2.971011in}{1.608830in}}%
\pgfpathlineto{\pgfqpoint{2.962612in}{1.607766in}}%
\pgfpathclose%
\pgfusepath{fill}%
\end{pgfscope}%
\begin{pgfscope}%
\pgfpathrectangle{\pgfqpoint{1.254980in}{0.150000in}}{\pgfqpoint{5.490039in}{5.490039in}}%
\pgfusepath{clip}%
\pgfsetbuttcap%
\pgfsetroundjoin%
\definecolor{currentfill}{rgb}{0.210503,0.363727,0.552206}%
\pgfsetfillcolor{currentfill}%
\pgfsetfillopacity{0.700000}%
\pgfsetlinewidth{0.000000pt}%
\definecolor{currentstroke}{rgb}{0.000000,0.000000,0.000000}%
\pgfsetstrokecolor{currentstroke}%
\pgfsetdash{}{0pt}%
\pgfpathmoveto{\pgfqpoint{4.438819in}{2.163220in}}%
\pgfpathlineto{\pgfqpoint{4.452536in}{2.171346in}}%
\pgfpathlineto{\pgfqpoint{4.466266in}{2.179634in}}%
\pgfpathlineto{\pgfqpoint{4.480011in}{2.188082in}}%
\pgfpathlineto{\pgfqpoint{4.493770in}{2.196692in}}%
\pgfpathlineto{\pgfqpoint{4.501477in}{2.209107in}}%
\pgfpathlineto{\pgfqpoint{4.509179in}{2.221419in}}%
\pgfpathlineto{\pgfqpoint{4.516876in}{2.233627in}}%
\pgfpathlineto{\pgfqpoint{4.524567in}{2.245729in}}%
\pgfpathlineto{\pgfqpoint{4.510808in}{2.236954in}}%
\pgfpathlineto{\pgfqpoint{4.497064in}{2.228341in}}%
\pgfpathlineto{\pgfqpoint{4.483334in}{2.219888in}}%
\pgfpathlineto{\pgfqpoint{4.469618in}{2.211596in}}%
\pgfpathlineto{\pgfqpoint{4.461926in}{2.199649in}}%
\pgfpathlineto{\pgfqpoint{4.454229in}{2.187602in}}%
\pgfpathlineto{\pgfqpoint{4.446527in}{2.175459in}}%
\pgfpathlineto{\pgfqpoint{4.438819in}{2.163220in}}%
\pgfpathclose%
\pgfusepath{fill}%
\end{pgfscope}%
\begin{pgfscope}%
\pgfpathrectangle{\pgfqpoint{1.254980in}{0.150000in}}{\pgfqpoint{5.490039in}{5.490039in}}%
\pgfusepath{clip}%
\pgfsetbuttcap%
\pgfsetroundjoin%
\definecolor{currentfill}{rgb}{0.259857,0.745492,0.444467}%
\pgfsetfillcolor{currentfill}%
\pgfsetfillopacity{0.700000}%
\pgfsetlinewidth{0.000000pt}%
\definecolor{currentstroke}{rgb}{0.000000,0.000000,0.000000}%
\pgfsetstrokecolor{currentstroke}%
\pgfsetdash{}{0pt}%
\pgfpathmoveto{\pgfqpoint{5.535006in}{3.216038in}}%
\pgfpathlineto{\pgfqpoint{5.549401in}{3.230305in}}%
\pgfpathlineto{\pgfqpoint{5.563817in}{3.244735in}}%
\pgfpathlineto{\pgfqpoint{5.578253in}{3.259327in}}%
\pgfpathlineto{\pgfqpoint{5.592711in}{3.274082in}}%
\pgfpathlineto{\pgfqpoint{5.599834in}{3.277069in}}%
\pgfpathlineto{\pgfqpoint{5.606947in}{3.279940in}}%
\pgfpathlineto{\pgfqpoint{5.614051in}{3.282701in}}%
\pgfpathlineto{\pgfqpoint{5.621144in}{3.285355in}}%
\pgfpathlineto{\pgfqpoint{5.606706in}{3.270927in}}%
\pgfpathlineto{\pgfqpoint{5.592289in}{3.256661in}}%
\pgfpathlineto{\pgfqpoint{5.577893in}{3.242558in}}%
\pgfpathlineto{\pgfqpoint{5.563517in}{3.228615in}}%
\pgfpathlineto{\pgfqpoint{5.556403in}{3.225625in}}%
\pgfpathlineto{\pgfqpoint{5.549281in}{3.222534in}}%
\pgfpathlineto{\pgfqpoint{5.542148in}{3.219339in}}%
\pgfpathlineto{\pgfqpoint{5.535006in}{3.216038in}}%
\pgfpathclose%
\pgfusepath{fill}%
\end{pgfscope}%
\begin{pgfscope}%
\pgfpathrectangle{\pgfqpoint{1.254980in}{0.150000in}}{\pgfqpoint{5.490039in}{5.490039in}}%
\pgfusepath{clip}%
\pgfsetbuttcap%
\pgfsetroundjoin%
\definecolor{currentfill}{rgb}{0.165117,0.467423,0.558141}%
\pgfsetfillcolor{currentfill}%
\pgfsetfillopacity{0.700000}%
\pgfsetlinewidth{0.000000pt}%
\definecolor{currentstroke}{rgb}{0.000000,0.000000,0.000000}%
\pgfsetstrokecolor{currentstroke}%
\pgfsetdash{}{0pt}%
\pgfpathmoveto{\pgfqpoint{2.201577in}{2.540657in}}%
\pgfpathlineto{\pgfqpoint{2.215431in}{2.517156in}}%
\pgfpathlineto{\pgfqpoint{2.229270in}{2.493944in}}%
\pgfpathlineto{\pgfqpoint{2.243095in}{2.471017in}}%
\pgfpathlineto{\pgfqpoint{2.256907in}{2.448374in}}%
\pgfpathlineto{\pgfqpoint{2.265963in}{2.440968in}}%
\pgfpathlineto{\pgfqpoint{2.274993in}{2.433940in}}%
\pgfpathlineto{\pgfqpoint{2.283999in}{2.427284in}}%
\pgfpathlineto{\pgfqpoint{2.292981in}{2.420993in}}%
\pgfpathlineto{\pgfqpoint{2.279233in}{2.442967in}}%
\pgfpathlineto{\pgfqpoint{2.265471in}{2.465222in}}%
\pgfpathlineto{\pgfqpoint{2.251696in}{2.487760in}}%
\pgfpathlineto{\pgfqpoint{2.237908in}{2.510586in}}%
\pgfpathlineto{\pgfqpoint{2.228864in}{2.517536in}}%
\pgfpathlineto{\pgfqpoint{2.219794in}{2.524860in}}%
\pgfpathlineto{\pgfqpoint{2.210699in}{2.532564in}}%
\pgfpathlineto{\pgfqpoint{2.201577in}{2.540657in}}%
\pgfpathclose%
\pgfusepath{fill}%
\end{pgfscope}%
\begin{pgfscope}%
\pgfpathrectangle{\pgfqpoint{1.254980in}{0.150000in}}{\pgfqpoint{5.490039in}{5.490039in}}%
\pgfusepath{clip}%
\pgfsetbuttcap%
\pgfsetroundjoin%
\definecolor{currentfill}{rgb}{0.140210,0.665859,0.513427}%
\pgfsetfillcolor{currentfill}%
\pgfsetfillopacity{0.700000}%
\pgfsetlinewidth{0.000000pt}%
\definecolor{currentstroke}{rgb}{0.000000,0.000000,0.000000}%
\pgfsetstrokecolor{currentstroke}%
\pgfsetdash{}{0pt}%
\pgfpathmoveto{\pgfqpoint{5.247550in}{2.975508in}}%
\pgfpathlineto{\pgfqpoint{5.261761in}{2.988815in}}%
\pgfpathlineto{\pgfqpoint{5.275991in}{3.002285in}}%
\pgfpathlineto{\pgfqpoint{5.290240in}{3.015917in}}%
\pgfpathlineto{\pgfqpoint{5.304510in}{3.029712in}}%
\pgfpathlineto{\pgfqpoint{5.311834in}{3.035545in}}%
\pgfpathlineto{\pgfqpoint{5.319148in}{3.041238in}}%
\pgfpathlineto{\pgfqpoint{5.326453in}{3.046793in}}%
\pgfpathlineto{\pgfqpoint{5.333749in}{3.052213in}}%
\pgfpathlineto{\pgfqpoint{5.319491in}{3.038618in}}%
\pgfpathlineto{\pgfqpoint{5.305253in}{3.025186in}}%
\pgfpathlineto{\pgfqpoint{5.291035in}{3.011915in}}%
\pgfpathlineto{\pgfqpoint{5.276836in}{2.998807in}}%
\pgfpathlineto{\pgfqpoint{5.269528in}{2.993176in}}%
\pgfpathlineto{\pgfqpoint{5.262211in}{2.987417in}}%
\pgfpathlineto{\pgfqpoint{5.254885in}{2.981529in}}%
\pgfpathlineto{\pgfqpoint{5.247550in}{2.975508in}}%
\pgfpathclose%
\pgfusepath{fill}%
\end{pgfscope}%
\begin{pgfscope}%
\pgfpathrectangle{\pgfqpoint{1.254980in}{0.150000in}}{\pgfqpoint{5.490039in}{5.490039in}}%
\pgfusepath{clip}%
\pgfsetbuttcap%
\pgfsetroundjoin%
\definecolor{currentfill}{rgb}{0.279566,0.067836,0.391917}%
\pgfsetfillcolor{currentfill}%
\pgfsetfillopacity{0.700000}%
\pgfsetlinewidth{0.000000pt}%
\definecolor{currentstroke}{rgb}{0.000000,0.000000,0.000000}%
\pgfsetstrokecolor{currentstroke}%
\pgfsetdash{}{0pt}%
\pgfpathmoveto{\pgfqpoint{3.746880in}{1.546755in}}%
\pgfpathlineto{\pgfqpoint{3.760326in}{1.547175in}}%
\pgfpathlineto{\pgfqpoint{3.773779in}{1.547758in}}%
\pgfpathlineto{\pgfqpoint{3.787241in}{1.548504in}}%
\pgfpathlineto{\pgfqpoint{3.800710in}{1.549411in}}%
\pgfpathlineto{\pgfqpoint{3.808625in}{1.560819in}}%
\pgfpathlineto{\pgfqpoint{3.816534in}{1.572285in}}%
\pgfpathlineto{\pgfqpoint{3.824438in}{1.583806in}}%
\pgfpathlineto{\pgfqpoint{3.832338in}{1.595378in}}%
\pgfpathlineto{\pgfqpoint{3.818877in}{1.594027in}}%
\pgfpathlineto{\pgfqpoint{3.805425in}{1.592838in}}%
\pgfpathlineto{\pgfqpoint{3.791980in}{1.591813in}}%
\pgfpathlineto{\pgfqpoint{3.778544in}{1.590950in}}%
\pgfpathlineto{\pgfqpoint{3.770636in}{1.579811in}}%
\pgfpathlineto{\pgfqpoint{3.762723in}{1.568729in}}%
\pgfpathlineto{\pgfqpoint{3.754805in}{1.557709in}}%
\pgfpathlineto{\pgfqpoint{3.746880in}{1.546755in}}%
\pgfpathclose%
\pgfusepath{fill}%
\end{pgfscope}%
\begin{pgfscope}%
\pgfpathrectangle{\pgfqpoint{1.254980in}{0.150000in}}{\pgfqpoint{5.490039in}{5.490039in}}%
\pgfusepath{clip}%
\pgfsetbuttcap%
\pgfsetroundjoin%
\definecolor{currentfill}{rgb}{0.274952,0.037752,0.364543}%
\pgfsetfillcolor{currentfill}%
\pgfsetfillopacity{0.700000}%
\pgfsetlinewidth{0.000000pt}%
\definecolor{currentstroke}{rgb}{0.000000,0.000000,0.000000}%
\pgfsetstrokecolor{currentstroke}%
\pgfsetdash{}{0pt}%
\pgfpathmoveto{\pgfqpoint{3.661372in}{1.505512in}}%
\pgfpathlineto{\pgfqpoint{3.674800in}{1.504809in}}%
\pgfpathlineto{\pgfqpoint{3.688235in}{1.504269in}}%
\pgfpathlineto{\pgfqpoint{3.701677in}{1.503893in}}%
\pgfpathlineto{\pgfqpoint{3.715126in}{1.503680in}}%
\pgfpathlineto{\pgfqpoint{3.723073in}{1.514329in}}%
\pgfpathlineto{\pgfqpoint{3.731015in}{1.525060in}}%
\pgfpathlineto{\pgfqpoint{3.738950in}{1.535870in}}%
\pgfpathlineto{\pgfqpoint{3.746880in}{1.546755in}}%
\pgfpathlineto{\pgfqpoint{3.733442in}{1.546498in}}%
\pgfpathlineto{\pgfqpoint{3.720011in}{1.546404in}}%
\pgfpathlineto{\pgfqpoint{3.706588in}{1.546473in}}%
\pgfpathlineto{\pgfqpoint{3.693171in}{1.546707in}}%
\pgfpathlineto{\pgfqpoint{3.685230in}{1.536282in}}%
\pgfpathlineto{\pgfqpoint{3.677284in}{1.525938in}}%
\pgfpathlineto{\pgfqpoint{3.669331in}{1.515680in}}%
\pgfpathlineto{\pgfqpoint{3.661372in}{1.505512in}}%
\pgfpathclose%
\pgfusepath{fill}%
\end{pgfscope}%
\begin{pgfscope}%
\pgfpathrectangle{\pgfqpoint{1.254980in}{0.150000in}}{\pgfqpoint{5.490039in}{5.490039in}}%
\pgfusepath{clip}%
\pgfsetbuttcap%
\pgfsetroundjoin%
\definecolor{currentfill}{rgb}{0.174274,0.445044,0.557792}%
\pgfsetfillcolor{currentfill}%
\pgfsetfillopacity{0.700000}%
\pgfsetlinewidth{0.000000pt}%
\definecolor{currentstroke}{rgb}{0.000000,0.000000,0.000000}%
\pgfsetstrokecolor{currentstroke}%
\pgfsetdash{}{0pt}%
\pgfpathmoveto{\pgfqpoint{4.641076in}{2.376296in}}%
\pgfpathlineto{\pgfqpoint{4.654909in}{2.386119in}}%
\pgfpathlineto{\pgfqpoint{4.668758in}{2.396103in}}%
\pgfpathlineto{\pgfqpoint{4.682623in}{2.406249in}}%
\pgfpathlineto{\pgfqpoint{4.696504in}{2.416556in}}%
\pgfpathlineto{\pgfqpoint{4.704142in}{2.427840in}}%
\pgfpathlineto{\pgfqpoint{4.711773in}{2.438995in}}%
\pgfpathlineto{\pgfqpoint{4.719399in}{2.450021in}}%
\pgfpathlineto{\pgfqpoint{4.727018in}{2.460917in}}%
\pgfpathlineto{\pgfqpoint{4.713138in}{2.450533in}}%
\pgfpathlineto{\pgfqpoint{4.699275in}{2.440310in}}%
\pgfpathlineto{\pgfqpoint{4.685427in}{2.430249in}}%
\pgfpathlineto{\pgfqpoint{4.671596in}{2.420349in}}%
\pgfpathlineto{\pgfqpoint{4.663975in}{2.409519in}}%
\pgfpathlineto{\pgfqpoint{4.656348in}{2.398566in}}%
\pgfpathlineto{\pgfqpoint{4.648715in}{2.387492in}}%
\pgfpathlineto{\pgfqpoint{4.641076in}{2.376296in}}%
\pgfpathclose%
\pgfusepath{fill}%
\end{pgfscope}%
\begin{pgfscope}%
\pgfpathrectangle{\pgfqpoint{1.254980in}{0.150000in}}{\pgfqpoint{5.490039in}{5.490039in}}%
\pgfusepath{clip}%
\pgfsetbuttcap%
\pgfsetroundjoin%
\definecolor{currentfill}{rgb}{0.269308,0.218818,0.509577}%
\pgfsetfillcolor{currentfill}%
\pgfsetfillopacity{0.700000}%
\pgfsetlinewidth{0.000000pt}%
\definecolor{currentstroke}{rgb}{0.000000,0.000000,0.000000}%
\pgfsetstrokecolor{currentstroke}%
\pgfsetdash{}{0pt}%
\pgfpathmoveto{\pgfqpoint{4.120042in}{1.831052in}}%
\pgfpathlineto{\pgfqpoint{4.133611in}{1.835998in}}%
\pgfpathlineto{\pgfqpoint{4.147191in}{1.841105in}}%
\pgfpathlineto{\pgfqpoint{4.160783in}{1.846373in}}%
\pgfpathlineto{\pgfqpoint{4.174386in}{1.851800in}}%
\pgfpathlineto{\pgfqpoint{4.182189in}{1.864929in}}%
\pgfpathlineto{\pgfqpoint{4.189988in}{1.878018in}}%
\pgfpathlineto{\pgfqpoint{4.197782in}{1.891064in}}%
\pgfpathlineto{\pgfqpoint{4.205571in}{1.904065in}}%
\pgfpathlineto{\pgfqpoint{4.191969in}{1.898330in}}%
\pgfpathlineto{\pgfqpoint{4.178379in}{1.892756in}}%
\pgfpathlineto{\pgfqpoint{4.164801in}{1.887342in}}%
\pgfpathlineto{\pgfqpoint{4.151234in}{1.882089in}}%
\pgfpathlineto{\pgfqpoint{4.143443in}{1.869385in}}%
\pgfpathlineto{\pgfqpoint{4.135647in}{1.856642in}}%
\pgfpathlineto{\pgfqpoint{4.127847in}{1.843863in}}%
\pgfpathlineto{\pgfqpoint{4.120042in}{1.831052in}}%
\pgfpathclose%
\pgfusepath{fill}%
\end{pgfscope}%
\begin{pgfscope}%
\pgfpathrectangle{\pgfqpoint{1.254980in}{0.150000in}}{\pgfqpoint{5.490039in}{5.490039in}}%
\pgfusepath{clip}%
\pgfsetbuttcap%
\pgfsetroundjoin%
\definecolor{currentfill}{rgb}{0.267004,0.004874,0.329415}%
\pgfsetfillcolor{currentfill}%
\pgfsetfillopacity{0.700000}%
\pgfsetlinewidth{0.000000pt}%
\definecolor{currentstroke}{rgb}{0.000000,0.000000,0.000000}%
\pgfsetstrokecolor{currentstroke}%
\pgfsetdash{}{0pt}%
\pgfpathmoveto{\pgfqpoint{3.350444in}{1.451347in}}%
\pgfpathlineto{\pgfqpoint{3.363845in}{1.446390in}}%
\pgfpathlineto{\pgfqpoint{3.377250in}{1.441603in}}%
\pgfpathlineto{\pgfqpoint{3.390658in}{1.436987in}}%
\pgfpathlineto{\pgfqpoint{3.404069in}{1.432542in}}%
\pgfpathlineto{\pgfqpoint{3.412168in}{1.439557in}}%
\pgfpathlineto{\pgfqpoint{3.420258in}{1.446744in}}%
\pgfpathlineto{\pgfqpoint{3.428340in}{1.454097in}}%
\pgfpathlineto{\pgfqpoint{3.436413in}{1.461610in}}%
\pgfpathlineto{\pgfqpoint{3.423022in}{1.465503in}}%
\pgfpathlineto{\pgfqpoint{3.409636in}{1.469566in}}%
\pgfpathlineto{\pgfqpoint{3.396253in}{1.473799in}}%
\pgfpathlineto{\pgfqpoint{3.382874in}{1.478204in}}%
\pgfpathlineto{\pgfqpoint{3.374780in}{1.471233in}}%
\pgfpathlineto{\pgfqpoint{3.366677in}{1.464429in}}%
\pgfpathlineto{\pgfqpoint{3.358565in}{1.457799in}}%
\pgfpathlineto{\pgfqpoint{3.350444in}{1.451347in}}%
\pgfpathclose%
\pgfusepath{fill}%
\end{pgfscope}%
\begin{pgfscope}%
\pgfpathrectangle{\pgfqpoint{1.254980in}{0.150000in}}{\pgfqpoint{5.490039in}{5.490039in}}%
\pgfusepath{clip}%
\pgfsetbuttcap%
\pgfsetroundjoin%
\definecolor{currentfill}{rgb}{0.282327,0.094955,0.417331}%
\pgfsetfillcolor{currentfill}%
\pgfsetfillopacity{0.700000}%
\pgfsetlinewidth{0.000000pt}%
\definecolor{currentstroke}{rgb}{0.000000,0.000000,0.000000}%
\pgfsetstrokecolor{currentstroke}%
\pgfsetdash{}{0pt}%
\pgfpathmoveto{\pgfqpoint{3.832338in}{1.595378in}}%
\pgfpathlineto{\pgfqpoint{3.845807in}{1.596890in}}%
\pgfpathlineto{\pgfqpoint{3.859284in}{1.598565in}}%
\pgfpathlineto{\pgfqpoint{3.872771in}{1.600401in}}%
\pgfpathlineto{\pgfqpoint{3.886266in}{1.602398in}}%
\pgfpathlineto{\pgfqpoint{3.894152in}{1.614442in}}%
\pgfpathlineto{\pgfqpoint{3.902034in}{1.626521in}}%
\pgfpathlineto{\pgfqpoint{3.909911in}{1.638632in}}%
\pgfpathlineto{\pgfqpoint{3.917783in}{1.650770in}}%
\pgfpathlineto{\pgfqpoint{3.904295in}{1.648356in}}%
\pgfpathlineto{\pgfqpoint{3.890815in}{1.646104in}}%
\pgfpathlineto{\pgfqpoint{3.877345in}{1.644014in}}%
\pgfpathlineto{\pgfqpoint{3.863883in}{1.642086in}}%
\pgfpathlineto{\pgfqpoint{3.856004in}{1.630353in}}%
\pgfpathlineto{\pgfqpoint{3.848120in}{1.618655in}}%
\pgfpathlineto{\pgfqpoint{3.840232in}{1.606995in}}%
\pgfpathlineto{\pgfqpoint{3.832338in}{1.595378in}}%
\pgfpathclose%
\pgfusepath{fill}%
\end{pgfscope}%
\begin{pgfscope}%
\pgfpathrectangle{\pgfqpoint{1.254980in}{0.150000in}}{\pgfqpoint{5.490039in}{5.490039in}}%
\pgfusepath{clip}%
\pgfsetbuttcap%
\pgfsetroundjoin%
\definecolor{currentfill}{rgb}{0.277941,0.056324,0.381191}%
\pgfsetfillcolor{currentfill}%
\pgfsetfillopacity{0.700000}%
\pgfsetlinewidth{0.000000pt}%
\definecolor{currentstroke}{rgb}{0.000000,0.000000,0.000000}%
\pgfsetstrokecolor{currentstroke}%
\pgfsetdash{}{0pt}%
\pgfpathmoveto{\pgfqpoint{3.016362in}{1.566951in}}%
\pgfpathlineto{\pgfqpoint{3.029797in}{1.557220in}}%
\pgfpathlineto{\pgfqpoint{3.043232in}{1.547675in}}%
\pgfpathlineto{\pgfqpoint{3.056665in}{1.538316in}}%
\pgfpathlineto{\pgfqpoint{3.070099in}{1.529142in}}%
\pgfpathlineto{\pgfqpoint{3.078425in}{1.531461in}}%
\pgfpathlineto{\pgfqpoint{3.086738in}{1.534037in}}%
\pgfpathlineto{\pgfqpoint{3.095038in}{1.536861in}}%
\pgfpathlineto{\pgfqpoint{3.103326in}{1.539928in}}%
\pgfpathlineto{\pgfqpoint{3.089926in}{1.548486in}}%
\pgfpathlineto{\pgfqpoint{3.076526in}{1.557230in}}%
\pgfpathlineto{\pgfqpoint{3.063127in}{1.566159in}}%
\pgfpathlineto{\pgfqpoint{3.049727in}{1.575274in}}%
\pgfpathlineto{\pgfqpoint{3.041406in}{1.572812in}}%
\pgfpathlineto{\pgfqpoint{3.033072in}{1.570600in}}%
\pgfpathlineto{\pgfqpoint{3.024724in}{1.568644in}}%
\pgfpathlineto{\pgfqpoint{3.016362in}{1.566951in}}%
\pgfpathclose%
\pgfusepath{fill}%
\end{pgfscope}%
\begin{pgfscope}%
\pgfpathrectangle{\pgfqpoint{1.254980in}{0.150000in}}{\pgfqpoint{5.490039in}{5.490039in}}%
\pgfusepath{clip}%
\pgfsetbuttcap%
\pgfsetroundjoin%
\definecolor{currentfill}{rgb}{0.268510,0.009605,0.335427}%
\pgfsetfillcolor{currentfill}%
\pgfsetfillopacity{0.700000}%
\pgfsetlinewidth{0.000000pt}%
\definecolor{currentstroke}{rgb}{0.000000,0.000000,0.000000}%
\pgfsetstrokecolor{currentstroke}%
\pgfsetdash{}{0pt}%
\pgfpathmoveto{\pgfqpoint{3.210545in}{1.478004in}}%
\pgfpathlineto{\pgfqpoint{3.223953in}{1.471070in}}%
\pgfpathlineto{\pgfqpoint{3.237362in}{1.464313in}}%
\pgfpathlineto{\pgfqpoint{3.250773in}{1.457731in}}%
\pgfpathlineto{\pgfqpoint{3.264186in}{1.451325in}}%
\pgfpathlineto{\pgfqpoint{3.272373in}{1.456413in}}%
\pgfpathlineto{\pgfqpoint{3.280549in}{1.461709in}}%
\pgfpathlineto{\pgfqpoint{3.288714in}{1.467208in}}%
\pgfpathlineto{\pgfqpoint{3.296869in}{1.472904in}}%
\pgfpathlineto{\pgfqpoint{3.283483in}{1.478728in}}%
\pgfpathlineto{\pgfqpoint{3.270098in}{1.484727in}}%
\pgfpathlineto{\pgfqpoint{3.256716in}{1.490901in}}%
\pgfpathlineto{\pgfqpoint{3.243336in}{1.497252in}}%
\pgfpathlineto{\pgfqpoint{3.235154in}{1.492128in}}%
\pgfpathlineto{\pgfqpoint{3.226962in}{1.487208in}}%
\pgfpathlineto{\pgfqpoint{3.218759in}{1.482498in}}%
\pgfpathlineto{\pgfqpoint{3.210545in}{1.478004in}}%
\pgfpathclose%
\pgfusepath{fill}%
\end{pgfscope}%
\begin{pgfscope}%
\pgfpathrectangle{\pgfqpoint{1.254980in}{0.150000in}}{\pgfqpoint{5.490039in}{5.490039in}}%
\pgfusepath{clip}%
\pgfsetbuttcap%
\pgfsetroundjoin%
\definecolor{currentfill}{rgb}{0.271305,0.019942,0.347269}%
\pgfsetfillcolor{currentfill}%
\pgfsetfillopacity{0.700000}%
\pgfsetlinewidth{0.000000pt}%
\definecolor{currentstroke}{rgb}{0.000000,0.000000,0.000000}%
\pgfsetstrokecolor{currentstroke}%
\pgfsetdash{}{0pt}%
\pgfpathmoveto{\pgfqpoint{3.575767in}{1.472283in}}%
\pgfpathlineto{\pgfqpoint{3.589185in}{1.470424in}}%
\pgfpathlineto{\pgfqpoint{3.602608in}{1.468729in}}%
\pgfpathlineto{\pgfqpoint{3.616037in}{1.467200in}}%
\pgfpathlineto{\pgfqpoint{3.629472in}{1.465835in}}%
\pgfpathlineto{\pgfqpoint{3.637457in}{1.475595in}}%
\pgfpathlineto{\pgfqpoint{3.645435in}{1.485465in}}%
\pgfpathlineto{\pgfqpoint{3.653406in}{1.495439in}}%
\pgfpathlineto{\pgfqpoint{3.661372in}{1.505512in}}%
\pgfpathlineto{\pgfqpoint{3.647950in}{1.506380in}}%
\pgfpathlineto{\pgfqpoint{3.634535in}{1.507412in}}%
\pgfpathlineto{\pgfqpoint{3.621126in}{1.508609in}}%
\pgfpathlineto{\pgfqpoint{3.607723in}{1.509972in}}%
\pgfpathlineto{\pgfqpoint{3.599744in}{1.500385in}}%
\pgfpathlineto{\pgfqpoint{3.591759in}{1.490905in}}%
\pgfpathlineto{\pgfqpoint{3.583766in}{1.481536in}}%
\pgfpathlineto{\pgfqpoint{3.575767in}{1.472283in}}%
\pgfpathclose%
\pgfusepath{fill}%
\end{pgfscope}%
\begin{pgfscope}%
\pgfpathrectangle{\pgfqpoint{1.254980in}{0.150000in}}{\pgfqpoint{5.490039in}{5.490039in}}%
\pgfusepath{clip}%
\pgfsetbuttcap%
\pgfsetroundjoin%
\definecolor{currentfill}{rgb}{0.235526,0.309527,0.542944}%
\pgfsetfillcolor{currentfill}%
\pgfsetfillopacity{0.700000}%
\pgfsetlinewidth{0.000000pt}%
\definecolor{currentstroke}{rgb}{0.000000,0.000000,0.000000}%
\pgfsetstrokecolor{currentstroke}%
\pgfsetdash{}{0pt}%
\pgfpathmoveto{\pgfqpoint{4.322248in}{2.032804in}}%
\pgfpathlineto{\pgfqpoint{4.335911in}{2.039871in}}%
\pgfpathlineto{\pgfqpoint{4.349587in}{2.047099in}}%
\pgfpathlineto{\pgfqpoint{4.363276in}{2.054488in}}%
\pgfpathlineto{\pgfqpoint{4.376979in}{2.062037in}}%
\pgfpathlineto{\pgfqpoint{4.384726in}{2.074987in}}%
\pgfpathlineto{\pgfqpoint{4.392468in}{2.087854in}}%
\pgfpathlineto{\pgfqpoint{4.400206in}{2.100637in}}%
\pgfpathlineto{\pgfqpoint{4.407939in}{2.113333in}}%
\pgfpathlineto{\pgfqpoint{4.394236in}{2.105561in}}%
\pgfpathlineto{\pgfqpoint{4.380547in}{2.097950in}}%
\pgfpathlineto{\pgfqpoint{4.366871in}{2.090499in}}%
\pgfpathlineto{\pgfqpoint{4.353209in}{2.083209in}}%
\pgfpathlineto{\pgfqpoint{4.345476in}{2.070725in}}%
\pgfpathlineto{\pgfqpoint{4.337738in}{2.058161in}}%
\pgfpathlineto{\pgfqpoint{4.329995in}{2.045520in}}%
\pgfpathlineto{\pgfqpoint{4.322248in}{2.032804in}}%
\pgfpathclose%
\pgfusepath{fill}%
\end{pgfscope}%
\begin{pgfscope}%
\pgfpathrectangle{\pgfqpoint{1.254980in}{0.150000in}}{\pgfqpoint{5.490039in}{5.490039in}}%
\pgfusepath{clip}%
\pgfsetbuttcap%
\pgfsetroundjoin%
\definecolor{currentfill}{rgb}{0.319809,0.770914,0.411152}%
\pgfsetfillcolor{currentfill}%
\pgfsetfillopacity{0.700000}%
\pgfsetlinewidth{0.000000pt}%
\definecolor{currentstroke}{rgb}{0.000000,0.000000,0.000000}%
\pgfsetstrokecolor{currentstroke}%
\pgfsetdash{}{0pt}%
\pgfpathmoveto{\pgfqpoint{5.621144in}{3.285355in}}%
\pgfpathlineto{\pgfqpoint{5.635604in}{3.299945in}}%
\pgfpathlineto{\pgfqpoint{5.650085in}{3.314697in}}%
\pgfpathlineto{\pgfqpoint{5.664587in}{3.329613in}}%
\pgfpathlineto{\pgfqpoint{5.679111in}{3.344691in}}%
\pgfpathlineto{\pgfqpoint{5.686174in}{3.346894in}}%
\pgfpathlineto{\pgfqpoint{5.693227in}{3.348989in}}%
\pgfpathlineto{\pgfqpoint{5.700270in}{3.350982in}}%
\pgfpathlineto{\pgfqpoint{5.707302in}{3.352876in}}%
\pgfpathlineto{\pgfqpoint{5.692800in}{3.338158in}}%
\pgfpathlineto{\pgfqpoint{5.678320in}{3.323602in}}%
\pgfpathlineto{\pgfqpoint{5.663860in}{3.309207in}}%
\pgfpathlineto{\pgfqpoint{5.649422in}{3.294975in}}%
\pgfpathlineto{\pgfqpoint{5.642367in}{3.292711in}}%
\pgfpathlineto{\pgfqpoint{5.635303in}{3.290356in}}%
\pgfpathlineto{\pgfqpoint{5.628228in}{3.287905in}}%
\pgfpathlineto{\pgfqpoint{5.621144in}{3.285355in}}%
\pgfpathclose%
\pgfusepath{fill}%
\end{pgfscope}%
\begin{pgfscope}%
\pgfpathrectangle{\pgfqpoint{1.254980in}{0.150000in}}{\pgfqpoint{5.490039in}{5.490039in}}%
\pgfusepath{clip}%
\pgfsetbuttcap%
\pgfsetroundjoin%
\definecolor{currentfill}{rgb}{0.141935,0.526453,0.555991}%
\pgfsetfillcolor{currentfill}%
\pgfsetfillopacity{0.700000}%
\pgfsetlinewidth{0.000000pt}%
\definecolor{currentstroke}{rgb}{0.000000,0.000000,0.000000}%
\pgfsetstrokecolor{currentstroke}%
\pgfsetdash{}{0pt}%
\pgfpathmoveto{\pgfqpoint{4.843398in}{2.586821in}}%
\pgfpathlineto{\pgfqpoint{4.857356in}{2.598078in}}%
\pgfpathlineto{\pgfqpoint{4.871331in}{2.609496in}}%
\pgfpathlineto{\pgfqpoint{4.885323in}{2.621076in}}%
\pgfpathlineto{\pgfqpoint{4.899333in}{2.632818in}}%
\pgfpathlineto{\pgfqpoint{4.906886in}{2.642543in}}%
\pgfpathlineto{\pgfqpoint{4.914431in}{2.652124in}}%
\pgfpathlineto{\pgfqpoint{4.921969in}{2.661562in}}%
\pgfpathlineto{\pgfqpoint{4.929500in}{2.670858in}}%
\pgfpathlineto{\pgfqpoint{4.915493in}{2.659129in}}%
\pgfpathlineto{\pgfqpoint{4.901505in}{2.647563in}}%
\pgfpathlineto{\pgfqpoint{4.887533in}{2.636158in}}%
\pgfpathlineto{\pgfqpoint{4.873579in}{2.624915in}}%
\pgfpathlineto{\pgfqpoint{4.866045in}{2.615594in}}%
\pgfpathlineto{\pgfqpoint{4.858503in}{2.606139in}}%
\pgfpathlineto{\pgfqpoint{4.850954in}{2.596548in}}%
\pgfpathlineto{\pgfqpoint{4.843398in}{2.586821in}}%
\pgfpathclose%
\pgfusepath{fill}%
\end{pgfscope}%
\begin{pgfscope}%
\pgfpathrectangle{\pgfqpoint{1.254980in}{0.150000in}}{\pgfqpoint{5.490039in}{5.490039in}}%
\pgfusepath{clip}%
\pgfsetbuttcap%
\pgfsetroundjoin%
\definecolor{currentfill}{rgb}{0.120092,0.600104,0.542530}%
\pgfsetfillcolor{currentfill}%
\pgfsetfillopacity{0.700000}%
\pgfsetlinewidth{0.000000pt}%
\definecolor{currentstroke}{rgb}{0.000000,0.000000,0.000000}%
\pgfsetstrokecolor{currentstroke}%
\pgfsetdash{}{0pt}%
\pgfpathmoveto{\pgfqpoint{5.045630in}{2.788248in}}%
\pgfpathlineto{\pgfqpoint{5.059715in}{2.800668in}}%
\pgfpathlineto{\pgfqpoint{5.073819in}{2.813251in}}%
\pgfpathlineto{\pgfqpoint{5.087941in}{2.825996in}}%
\pgfpathlineto{\pgfqpoint{5.102082in}{2.838903in}}%
\pgfpathlineto{\pgfqpoint{5.109530in}{2.846767in}}%
\pgfpathlineto{\pgfqpoint{5.116970in}{2.854484in}}%
\pgfpathlineto{\pgfqpoint{5.124402in}{2.862055in}}%
\pgfpathlineto{\pgfqpoint{5.131824in}{2.869481in}}%
\pgfpathlineto{\pgfqpoint{5.117691in}{2.856680in}}%
\pgfpathlineto{\pgfqpoint{5.103576in}{2.844041in}}%
\pgfpathlineto{\pgfqpoint{5.089479in}{2.831565in}}%
\pgfpathlineto{\pgfqpoint{5.075401in}{2.819250in}}%
\pgfpathlineto{\pgfqpoint{5.067970in}{2.811706in}}%
\pgfpathlineto{\pgfqpoint{5.060532in}{2.804026in}}%
\pgfpathlineto{\pgfqpoint{5.053085in}{2.796207in}}%
\pgfpathlineto{\pgfqpoint{5.045630in}{2.788248in}}%
\pgfpathclose%
\pgfusepath{fill}%
\end{pgfscope}%
\begin{pgfscope}%
\pgfpathrectangle{\pgfqpoint{1.254980in}{0.150000in}}{\pgfqpoint{5.490039in}{5.490039in}}%
\pgfusepath{clip}%
\pgfsetbuttcap%
\pgfsetroundjoin%
\definecolor{currentfill}{rgb}{0.283187,0.125848,0.444960}%
\pgfsetfillcolor{currentfill}%
\pgfsetfillopacity{0.700000}%
\pgfsetlinewidth{0.000000pt}%
\definecolor{currentstroke}{rgb}{0.000000,0.000000,0.000000}%
\pgfsetstrokecolor{currentstroke}%
\pgfsetdash{}{0pt}%
\pgfpathmoveto{\pgfqpoint{3.917783in}{1.650770in}}%
\pgfpathlineto{\pgfqpoint{3.931281in}{1.653345in}}%
\pgfpathlineto{\pgfqpoint{3.944789in}{1.656081in}}%
\pgfpathlineto{\pgfqpoint{3.958306in}{1.658978in}}%
\pgfpathlineto{\pgfqpoint{3.971832in}{1.662035in}}%
\pgfpathlineto{\pgfqpoint{3.979694in}{1.674596in}}%
\pgfpathlineto{\pgfqpoint{3.987552in}{1.687171in}}%
\pgfpathlineto{\pgfqpoint{3.995405in}{1.699755in}}%
\pgfpathlineto{\pgfqpoint{4.003253in}{1.712345in}}%
\pgfpathlineto{\pgfqpoint{3.989731in}{1.708898in}}%
\pgfpathlineto{\pgfqpoint{3.976219in}{1.705612in}}%
\pgfpathlineto{\pgfqpoint{3.962717in}{1.702487in}}%
\pgfpathlineto{\pgfqpoint{3.949224in}{1.699524in}}%
\pgfpathlineto{\pgfqpoint{3.941371in}{1.687312in}}%
\pgfpathlineto{\pgfqpoint{3.933513in}{1.675114in}}%
\pgfpathlineto{\pgfqpoint{3.925651in}{1.662932in}}%
\pgfpathlineto{\pgfqpoint{3.917783in}{1.650770in}}%
\pgfpathclose%
\pgfusepath{fill}%
\end{pgfscope}%
\begin{pgfscope}%
\pgfpathrectangle{\pgfqpoint{1.254980in}{0.150000in}}{\pgfqpoint{5.490039in}{5.490039in}}%
\pgfusepath{clip}%
\pgfsetbuttcap%
\pgfsetroundjoin%
\definecolor{currentfill}{rgb}{0.265145,0.232956,0.516599}%
\pgfsetfillcolor{currentfill}%
\pgfsetfillopacity{0.700000}%
\pgfsetlinewidth{0.000000pt}%
\definecolor{currentstroke}{rgb}{0.000000,0.000000,0.000000}%
\pgfsetstrokecolor{currentstroke}%
\pgfsetdash{}{0pt}%
\pgfpathmoveto{\pgfqpoint{2.604321in}{1.934026in}}%
\pgfpathlineto{\pgfqpoint{2.617901in}{1.917940in}}%
\pgfpathlineto{\pgfqpoint{2.631476in}{1.902076in}}%
\pgfpathlineto{\pgfqpoint{2.645044in}{1.886432in}}%
\pgfpathlineto{\pgfqpoint{2.658605in}{1.871007in}}%
\pgfpathlineto{\pgfqpoint{2.667294in}{1.867583in}}%
\pgfpathlineto{\pgfqpoint{2.675963in}{1.864496in}}%
\pgfpathlineto{\pgfqpoint{2.684612in}{1.861739in}}%
\pgfpathlineto{\pgfqpoint{2.693243in}{1.859305in}}%
\pgfpathlineto{\pgfqpoint{2.679731in}{1.874066in}}%
\pgfpathlineto{\pgfqpoint{2.666213in}{1.889044in}}%
\pgfpathlineto{\pgfqpoint{2.652689in}{1.904242in}}%
\pgfpathlineto{\pgfqpoint{2.639160in}{1.919660in}}%
\pgfpathlineto{\pgfqpoint{2.630480in}{1.922748in}}%
\pgfpathlineto{\pgfqpoint{2.621781in}{1.926166in}}%
\pgfpathlineto{\pgfqpoint{2.613061in}{1.929924in}}%
\pgfpathlineto{\pgfqpoint{2.604321in}{1.934026in}}%
\pgfpathclose%
\pgfusepath{fill}%
\end{pgfscope}%
\begin{pgfscope}%
\pgfpathrectangle{\pgfqpoint{1.254980in}{0.150000in}}{\pgfqpoint{5.490039in}{5.490039in}}%
\pgfusepath{clip}%
\pgfsetbuttcap%
\pgfsetroundjoin%
\definecolor{currentfill}{rgb}{0.255645,0.260703,0.528312}%
\pgfsetfillcolor{currentfill}%
\pgfsetfillopacity{0.700000}%
\pgfsetlinewidth{0.000000pt}%
\definecolor{currentstroke}{rgb}{0.000000,0.000000,0.000000}%
\pgfsetstrokecolor{currentstroke}%
\pgfsetdash{}{0pt}%
\pgfpathmoveto{\pgfqpoint{2.549927in}{2.000624in}}%
\pgfpathlineto{\pgfqpoint{2.563536in}{1.983633in}}%
\pgfpathlineto{\pgfqpoint{2.577138in}{1.966871in}}%
\pgfpathlineto{\pgfqpoint{2.590733in}{1.950336in}}%
\pgfpathlineto{\pgfqpoint{2.604321in}{1.934026in}}%
\pgfpathlineto{\pgfqpoint{2.613061in}{1.929924in}}%
\pgfpathlineto{\pgfqpoint{2.621781in}{1.926166in}}%
\pgfpathlineto{\pgfqpoint{2.630480in}{1.922748in}}%
\pgfpathlineto{\pgfqpoint{2.639160in}{1.919660in}}%
\pgfpathlineto{\pgfqpoint{2.625623in}{1.935302in}}%
\pgfpathlineto{\pgfqpoint{2.612081in}{1.951167in}}%
\pgfpathlineto{\pgfqpoint{2.598532in}{1.967259in}}%
\pgfpathlineto{\pgfqpoint{2.584975in}{1.983578in}}%
\pgfpathlineto{\pgfqpoint{2.576245in}{1.987323in}}%
\pgfpathlineto{\pgfqpoint{2.567493in}{1.991407in}}%
\pgfpathlineto{\pgfqpoint{2.558721in}{1.995839in}}%
\pgfpathlineto{\pgfqpoint{2.549927in}{2.000624in}}%
\pgfpathclose%
\pgfusepath{fill}%
\end{pgfscope}%
\begin{pgfscope}%
\pgfpathrectangle{\pgfqpoint{1.254980in}{0.150000in}}{\pgfqpoint{5.490039in}{5.490039in}}%
\pgfusepath{clip}%
\pgfsetbuttcap%
\pgfsetroundjoin%
\definecolor{currentfill}{rgb}{0.273006,0.204520,0.501721}%
\pgfsetfillcolor{currentfill}%
\pgfsetfillopacity{0.700000}%
\pgfsetlinewidth{0.000000pt}%
\definecolor{currentstroke}{rgb}{0.000000,0.000000,0.000000}%
\pgfsetstrokecolor{currentstroke}%
\pgfsetdash{}{0pt}%
\pgfpathmoveto{\pgfqpoint{2.658605in}{1.871007in}}%
\pgfpathlineto{\pgfqpoint{2.672161in}{1.855800in}}%
\pgfpathlineto{\pgfqpoint{2.685711in}{1.840809in}}%
\pgfpathlineto{\pgfqpoint{2.699256in}{1.826032in}}%
\pgfpathlineto{\pgfqpoint{2.712795in}{1.811468in}}%
\pgfpathlineto{\pgfqpoint{2.721434in}{1.808718in}}%
\pgfpathlineto{\pgfqpoint{2.730054in}{1.806297in}}%
\pgfpathlineto{\pgfqpoint{2.738656in}{1.804199in}}%
\pgfpathlineto{\pgfqpoint{2.747239in}{1.802415in}}%
\pgfpathlineto{\pgfqpoint{2.733748in}{1.816318in}}%
\pgfpathlineto{\pgfqpoint{2.720251in}{1.830433in}}%
\pgfpathlineto{\pgfqpoint{2.706750in}{1.844762in}}%
\pgfpathlineto{\pgfqpoint{2.693243in}{1.859305in}}%
\pgfpathlineto{\pgfqpoint{2.684612in}{1.861739in}}%
\pgfpathlineto{\pgfqpoint{2.675963in}{1.864496in}}%
\pgfpathlineto{\pgfqpoint{2.667294in}{1.867583in}}%
\pgfpathlineto{\pgfqpoint{2.658605in}{1.871007in}}%
\pgfpathclose%
\pgfusepath{fill}%
\end{pgfscope}%
\begin{pgfscope}%
\pgfpathrectangle{\pgfqpoint{1.254980in}{0.150000in}}{\pgfqpoint{5.490039in}{5.490039in}}%
\pgfusepath{clip}%
\pgfsetbuttcap%
\pgfsetroundjoin%
\definecolor{currentfill}{rgb}{0.243113,0.292092,0.538516}%
\pgfsetfillcolor{currentfill}%
\pgfsetfillopacity{0.700000}%
\pgfsetlinewidth{0.000000pt}%
\definecolor{currentstroke}{rgb}{0.000000,0.000000,0.000000}%
\pgfsetstrokecolor{currentstroke}%
\pgfsetdash{}{0pt}%
\pgfpathmoveto{\pgfqpoint{2.495410in}{2.070907in}}%
\pgfpathlineto{\pgfqpoint{2.509052in}{2.052984in}}%
\pgfpathlineto{\pgfqpoint{2.522685in}{2.035298in}}%
\pgfpathlineto{\pgfqpoint{2.536310in}{2.017845in}}%
\pgfpathlineto{\pgfqpoint{2.549927in}{2.000624in}}%
\pgfpathlineto{\pgfqpoint{2.558721in}{1.995839in}}%
\pgfpathlineto{\pgfqpoint{2.567493in}{1.991407in}}%
\pgfpathlineto{\pgfqpoint{2.576245in}{1.987323in}}%
\pgfpathlineto{\pgfqpoint{2.584975in}{1.983578in}}%
\pgfpathlineto{\pgfqpoint{2.571412in}{2.000126in}}%
\pgfpathlineto{\pgfqpoint{2.557841in}{2.016905in}}%
\pgfpathlineto{\pgfqpoint{2.544263in}{2.033917in}}%
\pgfpathlineto{\pgfqpoint{2.530676in}{2.051163in}}%
\pgfpathlineto{\pgfqpoint{2.521892in}{2.055570in}}%
\pgfpathlineto{\pgfqpoint{2.513087in}{2.060324in}}%
\pgfpathlineto{\pgfqpoint{2.504260in}{2.065434in}}%
\pgfpathlineto{\pgfqpoint{2.495410in}{2.070907in}}%
\pgfpathclose%
\pgfusepath{fill}%
\end{pgfscope}%
\begin{pgfscope}%
\pgfpathrectangle{\pgfqpoint{1.254980in}{0.150000in}}{\pgfqpoint{5.490039in}{5.490039in}}%
\pgfusepath{clip}%
\pgfsetbuttcap%
\pgfsetroundjoin%
\definecolor{currentfill}{rgb}{0.149039,0.508051,0.557250}%
\pgfsetfillcolor{currentfill}%
\pgfsetfillopacity{0.700000}%
\pgfsetlinewidth{0.000000pt}%
\definecolor{currentstroke}{rgb}{0.000000,0.000000,0.000000}%
\pgfsetstrokecolor{currentstroke}%
\pgfsetdash{}{0pt}%
\pgfpathmoveto{\pgfqpoint{2.146012in}{2.637598in}}%
\pgfpathlineto{\pgfqpoint{2.159926in}{2.612916in}}%
\pgfpathlineto{\pgfqpoint{2.173825in}{2.588533in}}%
\pgfpathlineto{\pgfqpoint{2.187709in}{2.564448in}}%
\pgfpathlineto{\pgfqpoint{2.201577in}{2.540657in}}%
\pgfpathlineto{\pgfqpoint{2.210699in}{2.532564in}}%
\pgfpathlineto{\pgfqpoint{2.219794in}{2.524860in}}%
\pgfpathlineto{\pgfqpoint{2.228864in}{2.517536in}}%
\pgfpathlineto{\pgfqpoint{2.237908in}{2.510586in}}%
\pgfpathlineto{\pgfqpoint{2.224105in}{2.533700in}}%
\pgfpathlineto{\pgfqpoint{2.210287in}{2.557107in}}%
\pgfpathlineto{\pgfqpoint{2.196455in}{2.580809in}}%
\pgfpathlineto{\pgfqpoint{2.182607in}{2.604809in}}%
\pgfpathlineto{\pgfqpoint{2.173499in}{2.612425in}}%
\pgfpathlineto{\pgfqpoint{2.164363in}{2.620424in}}%
\pgfpathlineto{\pgfqpoint{2.155201in}{2.628812in}}%
\pgfpathlineto{\pgfqpoint{2.146012in}{2.637598in}}%
\pgfpathclose%
\pgfusepath{fill}%
\end{pgfscope}%
\begin{pgfscope}%
\pgfpathrectangle{\pgfqpoint{1.254980in}{0.150000in}}{\pgfqpoint{5.490039in}{5.490039in}}%
\pgfusepath{clip}%
\pgfsetbuttcap%
\pgfsetroundjoin%
\definecolor{currentfill}{rgb}{0.194100,0.399323,0.555565}%
\pgfsetfillcolor{currentfill}%
\pgfsetfillopacity{0.700000}%
\pgfsetlinewidth{0.000000pt}%
\definecolor{currentstroke}{rgb}{0.000000,0.000000,0.000000}%
\pgfsetstrokecolor{currentstroke}%
\pgfsetdash{}{0pt}%
\pgfpathmoveto{\pgfqpoint{4.524567in}{2.245729in}}%
\pgfpathlineto{\pgfqpoint{4.538341in}{2.254664in}}%
\pgfpathlineto{\pgfqpoint{4.552129in}{2.263761in}}%
\pgfpathlineto{\pgfqpoint{4.565933in}{2.273018in}}%
\pgfpathlineto{\pgfqpoint{4.579751in}{2.282437in}}%
\pgfpathlineto{\pgfqpoint{4.587437in}{2.294580in}}%
\pgfpathlineto{\pgfqpoint{4.595117in}{2.306607in}}%
\pgfpathlineto{\pgfqpoint{4.602792in}{2.318517in}}%
\pgfpathlineto{\pgfqpoint{4.610460in}{2.330311in}}%
\pgfpathlineto{\pgfqpoint{4.596642in}{2.320756in}}%
\pgfpathlineto{\pgfqpoint{4.582839in}{2.311362in}}%
\pgfpathlineto{\pgfqpoint{4.569050in}{2.302130in}}%
\pgfpathlineto{\pgfqpoint{4.555277in}{2.293058in}}%
\pgfpathlineto{\pgfqpoint{4.547608in}{2.281390in}}%
\pgfpathlineto{\pgfqpoint{4.539933in}{2.269611in}}%
\pgfpathlineto{\pgfqpoint{4.532253in}{2.257724in}}%
\pgfpathlineto{\pgfqpoint{4.524567in}{2.245729in}}%
\pgfpathclose%
\pgfusepath{fill}%
\end{pgfscope}%
\begin{pgfscope}%
\pgfpathrectangle{\pgfqpoint{1.254980in}{0.150000in}}{\pgfqpoint{5.490039in}{5.490039in}}%
\pgfusepath{clip}%
\pgfsetbuttcap%
\pgfsetroundjoin%
\definecolor{currentfill}{rgb}{0.268510,0.009605,0.335427}%
\pgfsetfillcolor{currentfill}%
\pgfsetfillopacity{0.700000}%
\pgfsetlinewidth{0.000000pt}%
\definecolor{currentstroke}{rgb}{0.000000,0.000000,0.000000}%
\pgfsetstrokecolor{currentstroke}%
\pgfsetdash{}{0pt}%
\pgfpathmoveto{\pgfqpoint{3.490018in}{1.447730in}}%
\pgfpathlineto{\pgfqpoint{3.503431in}{1.444680in}}%
\pgfpathlineto{\pgfqpoint{3.516848in}{1.441797in}}%
\pgfpathlineto{\pgfqpoint{3.530271in}{1.439080in}}%
\pgfpathlineto{\pgfqpoint{3.543699in}{1.436530in}}%
\pgfpathlineto{\pgfqpoint{3.551727in}{1.445269in}}%
\pgfpathlineto{\pgfqpoint{3.559748in}{1.454145in}}%
\pgfpathlineto{\pgfqpoint{3.567761in}{1.463151in}}%
\pgfpathlineto{\pgfqpoint{3.575767in}{1.472283in}}%
\pgfpathlineto{\pgfqpoint{3.562356in}{1.474309in}}%
\pgfpathlineto{\pgfqpoint{3.548950in}{1.476500in}}%
\pgfpathlineto{\pgfqpoint{3.535549in}{1.478859in}}%
\pgfpathlineto{\pgfqpoint{3.522154in}{1.481384in}}%
\pgfpathlineto{\pgfqpoint{3.514131in}{1.472766in}}%
\pgfpathlineto{\pgfqpoint{3.506101in}{1.464281in}}%
\pgfpathlineto{\pgfqpoint{3.498063in}{1.455934in}}%
\pgfpathlineto{\pgfqpoint{3.490018in}{1.447730in}}%
\pgfpathclose%
\pgfusepath{fill}%
\end{pgfscope}%
\begin{pgfscope}%
\pgfpathrectangle{\pgfqpoint{1.254980in}{0.150000in}}{\pgfqpoint{5.490039in}{5.490039in}}%
\pgfusepath{clip}%
\pgfsetbuttcap%
\pgfsetroundjoin%
\definecolor{currentfill}{rgb}{0.170948,0.694384,0.493803}%
\pgfsetfillcolor{currentfill}%
\pgfsetfillopacity{0.700000}%
\pgfsetlinewidth{0.000000pt}%
\definecolor{currentstroke}{rgb}{0.000000,0.000000,0.000000}%
\pgfsetstrokecolor{currentstroke}%
\pgfsetdash{}{0pt}%
\pgfpathmoveto{\pgfqpoint{5.333749in}{3.052213in}}%
\pgfpathlineto{\pgfqpoint{5.348027in}{3.065970in}}%
\pgfpathlineto{\pgfqpoint{5.362324in}{3.079890in}}%
\pgfpathlineto{\pgfqpoint{5.376642in}{3.093973in}}%
\pgfpathlineto{\pgfqpoint{5.390980in}{3.108219in}}%
\pgfpathlineto{\pgfqpoint{5.398254in}{3.113286in}}%
\pgfpathlineto{\pgfqpoint{5.405518in}{3.118215in}}%
\pgfpathlineto{\pgfqpoint{5.412772in}{3.123010in}}%
\pgfpathlineto{\pgfqpoint{5.420017in}{3.127673in}}%
\pgfpathlineto{\pgfqpoint{5.405693in}{3.113660in}}%
\pgfpathlineto{\pgfqpoint{5.391388in}{3.099809in}}%
\pgfpathlineto{\pgfqpoint{5.377104in}{3.086121in}}%
\pgfpathlineto{\pgfqpoint{5.362840in}{3.072595in}}%
\pgfpathlineto{\pgfqpoint{5.355581in}{3.067689in}}%
\pgfpathlineto{\pgfqpoint{5.348313in}{3.062658in}}%
\pgfpathlineto{\pgfqpoint{5.341036in}{3.057501in}}%
\pgfpathlineto{\pgfqpoint{5.333749in}{3.052213in}}%
\pgfpathclose%
\pgfusepath{fill}%
\end{pgfscope}%
\begin{pgfscope}%
\pgfpathrectangle{\pgfqpoint{1.254980in}{0.150000in}}{\pgfqpoint{5.490039in}{5.490039in}}%
\pgfusepath{clip}%
\pgfsetbuttcap%
\pgfsetroundjoin%
\definecolor{currentfill}{rgb}{0.278012,0.180367,0.486697}%
\pgfsetfillcolor{currentfill}%
\pgfsetfillopacity{0.700000}%
\pgfsetlinewidth{0.000000pt}%
\definecolor{currentstroke}{rgb}{0.000000,0.000000,0.000000}%
\pgfsetstrokecolor{currentstroke}%
\pgfsetdash{}{0pt}%
\pgfpathmoveto{\pgfqpoint{2.712795in}{1.811468in}}%
\pgfpathlineto{\pgfqpoint{2.726329in}{1.797116in}}%
\pgfpathlineto{\pgfqpoint{2.739859in}{1.782974in}}%
\pgfpathlineto{\pgfqpoint{2.753383in}{1.769041in}}%
\pgfpathlineto{\pgfqpoint{2.766903in}{1.755315in}}%
\pgfpathlineto{\pgfqpoint{2.775495in}{1.753237in}}%
\pgfpathlineto{\pgfqpoint{2.784068in}{1.751479in}}%
\pgfpathlineto{\pgfqpoint{2.792624in}{1.750035in}}%
\pgfpathlineto{\pgfqpoint{2.801162in}{1.748898in}}%
\pgfpathlineto{\pgfqpoint{2.787687in}{1.761966in}}%
\pgfpathlineto{\pgfqpoint{2.774209in}{1.775240in}}%
\pgfpathlineto{\pgfqpoint{2.760726in}{1.788723in}}%
\pgfpathlineto{\pgfqpoint{2.747239in}{1.802415in}}%
\pgfpathlineto{\pgfqpoint{2.738656in}{1.804199in}}%
\pgfpathlineto{\pgfqpoint{2.730054in}{1.806297in}}%
\pgfpathlineto{\pgfqpoint{2.721434in}{1.808718in}}%
\pgfpathlineto{\pgfqpoint{2.712795in}{1.811468in}}%
\pgfpathclose%
\pgfusepath{fill}%
\end{pgfscope}%
\begin{pgfscope}%
\pgfpathrectangle{\pgfqpoint{1.254980in}{0.150000in}}{\pgfqpoint{5.490039in}{5.490039in}}%
\pgfusepath{clip}%
\pgfsetbuttcap%
\pgfsetroundjoin%
\definecolor{currentfill}{rgb}{0.377779,0.791781,0.377939}%
\pgfsetfillcolor{currentfill}%
\pgfsetfillopacity{0.700000}%
\pgfsetlinewidth{0.000000pt}%
\definecolor{currentstroke}{rgb}{0.000000,0.000000,0.000000}%
\pgfsetstrokecolor{currentstroke}%
\pgfsetdash{}{0pt}%
\pgfpathmoveto{\pgfqpoint{5.707302in}{3.352876in}}%
\pgfpathlineto{\pgfqpoint{5.721826in}{3.367757in}}%
\pgfpathlineto{\pgfqpoint{5.736372in}{3.382800in}}%
\pgfpathlineto{\pgfqpoint{5.750939in}{3.398007in}}%
\pgfpathlineto{\pgfqpoint{5.765529in}{3.413376in}}%
\pgfpathlineto{\pgfqpoint{5.772529in}{3.414795in}}%
\pgfpathlineto{\pgfqpoint{5.779518in}{3.416116in}}%
\pgfpathlineto{\pgfqpoint{5.786498in}{3.417343in}}%
\pgfpathlineto{\pgfqpoint{5.793467in}{3.418482in}}%
\pgfpathlineto{\pgfqpoint{5.778902in}{3.403505in}}%
\pgfpathlineto{\pgfqpoint{5.764359in}{3.388691in}}%
\pgfpathlineto{\pgfqpoint{5.749837in}{3.374038in}}%
\pgfpathlineto{\pgfqpoint{5.735337in}{3.359548in}}%
\pgfpathlineto{\pgfqpoint{5.728343in}{3.358007in}}%
\pgfpathlineto{\pgfqpoint{5.721339in}{3.356384in}}%
\pgfpathlineto{\pgfqpoint{5.714326in}{3.354675in}}%
\pgfpathlineto{\pgfqpoint{5.707302in}{3.352876in}}%
\pgfpathclose%
\pgfusepath{fill}%
\end{pgfscope}%
\begin{pgfscope}%
\pgfpathrectangle{\pgfqpoint{1.254980in}{0.150000in}}{\pgfqpoint{5.490039in}{5.490039in}}%
\pgfusepath{clip}%
\pgfsetbuttcap%
\pgfsetroundjoin%
\definecolor{currentfill}{rgb}{0.229739,0.322361,0.545706}%
\pgfsetfillcolor{currentfill}%
\pgfsetfillopacity{0.700000}%
\pgfsetlinewidth{0.000000pt}%
\definecolor{currentstroke}{rgb}{0.000000,0.000000,0.000000}%
\pgfsetstrokecolor{currentstroke}%
\pgfsetdash{}{0pt}%
\pgfpathmoveto{\pgfqpoint{2.440754in}{2.144990in}}%
\pgfpathlineto{\pgfqpoint{2.454432in}{2.126106in}}%
\pgfpathlineto{\pgfqpoint{2.468101in}{2.107466in}}%
\pgfpathlineto{\pgfqpoint{2.481760in}{2.089067in}}%
\pgfpathlineto{\pgfqpoint{2.495410in}{2.070907in}}%
\pgfpathlineto{\pgfqpoint{2.504260in}{2.065434in}}%
\pgfpathlineto{\pgfqpoint{2.513087in}{2.060324in}}%
\pgfpathlineto{\pgfqpoint{2.521892in}{2.055570in}}%
\pgfpathlineto{\pgfqpoint{2.530676in}{2.051163in}}%
\pgfpathlineto{\pgfqpoint{2.517082in}{2.068646in}}%
\pgfpathlineto{\pgfqpoint{2.503479in}{2.086367in}}%
\pgfpathlineto{\pgfqpoint{2.489867in}{2.104327in}}%
\pgfpathlineto{\pgfqpoint{2.476247in}{2.122530in}}%
\pgfpathlineto{\pgfqpoint{2.467408in}{2.127602in}}%
\pgfpathlineto{\pgfqpoint{2.458547in}{2.133032in}}%
\pgfpathlineto{\pgfqpoint{2.449662in}{2.138825in}}%
\pgfpathlineto{\pgfqpoint{2.440754in}{2.144990in}}%
\pgfpathclose%
\pgfusepath{fill}%
\end{pgfscope}%
\begin{pgfscope}%
\pgfpathrectangle{\pgfqpoint{1.254980in}{0.150000in}}{\pgfqpoint{5.490039in}{5.490039in}}%
\pgfusepath{clip}%
\pgfsetbuttcap%
\pgfsetroundjoin%
\definecolor{currentfill}{rgb}{0.276022,0.044167,0.370164}%
\pgfsetfillcolor{currentfill}%
\pgfsetfillopacity{0.700000}%
\pgfsetlinewidth{0.000000pt}%
\definecolor{currentstroke}{rgb}{0.000000,0.000000,0.000000}%
\pgfsetstrokecolor{currentstroke}%
\pgfsetdash{}{0pt}%
\pgfpathmoveto{\pgfqpoint{3.070099in}{1.529142in}}%
\pgfpathlineto{\pgfqpoint{3.083532in}{1.520152in}}%
\pgfpathlineto{\pgfqpoint{3.096965in}{1.511345in}}%
\pgfpathlineto{\pgfqpoint{3.110398in}{1.502721in}}%
\pgfpathlineto{\pgfqpoint{3.123831in}{1.494279in}}%
\pgfpathlineto{\pgfqpoint{3.132125in}{1.497224in}}%
\pgfpathlineto{\pgfqpoint{3.140405in}{1.500417in}}%
\pgfpathlineto{\pgfqpoint{3.148673in}{1.503852in}}%
\pgfpathlineto{\pgfqpoint{3.156928in}{1.507522in}}%
\pgfpathlineto{\pgfqpoint{3.143527in}{1.515351in}}%
\pgfpathlineto{\pgfqpoint{3.130126in}{1.523361in}}%
\pgfpathlineto{\pgfqpoint{3.116726in}{1.531553in}}%
\pgfpathlineto{\pgfqpoint{3.103326in}{1.539928in}}%
\pgfpathlineto{\pgfqpoint{3.095038in}{1.536861in}}%
\pgfpathlineto{\pgfqpoint{3.086738in}{1.534037in}}%
\pgfpathlineto{\pgfqpoint{3.078425in}{1.531461in}}%
\pgfpathlineto{\pgfqpoint{3.070099in}{1.529142in}}%
\pgfpathclose%
\pgfusepath{fill}%
\end{pgfscope}%
\begin{pgfscope}%
\pgfpathrectangle{\pgfqpoint{1.254980in}{0.150000in}}{\pgfqpoint{5.490039in}{5.490039in}}%
\pgfusepath{clip}%
\pgfsetbuttcap%
\pgfsetroundjoin%
\definecolor{currentfill}{rgb}{0.257322,0.256130,0.526563}%
\pgfsetfillcolor{currentfill}%
\pgfsetfillopacity{0.700000}%
\pgfsetlinewidth{0.000000pt}%
\definecolor{currentstroke}{rgb}{0.000000,0.000000,0.000000}%
\pgfsetstrokecolor{currentstroke}%
\pgfsetdash{}{0pt}%
\pgfpathmoveto{\pgfqpoint{4.205571in}{1.904065in}}%
\pgfpathlineto{\pgfqpoint{4.219185in}{1.909960in}}%
\pgfpathlineto{\pgfqpoint{4.232812in}{1.916016in}}%
\pgfpathlineto{\pgfqpoint{4.246450in}{1.922232in}}%
\pgfpathlineto{\pgfqpoint{4.260101in}{1.928608in}}%
\pgfpathlineto{\pgfqpoint{4.267885in}{1.941851in}}%
\pgfpathlineto{\pgfqpoint{4.275665in}{1.955036in}}%
\pgfpathlineto{\pgfqpoint{4.283440in}{1.968161in}}%
\pgfpathlineto{\pgfqpoint{4.291211in}{1.981224in}}%
\pgfpathlineto{\pgfqpoint{4.277561in}{1.974568in}}%
\pgfpathlineto{\pgfqpoint{4.263923in}{1.968073in}}%
\pgfpathlineto{\pgfqpoint{4.250298in}{1.961738in}}%
\pgfpathlineto{\pgfqpoint{4.236685in}{1.955564in}}%
\pgfpathlineto{\pgfqpoint{4.228913in}{1.942770in}}%
\pgfpathlineto{\pgfqpoint{4.221137in}{1.929920in}}%
\pgfpathlineto{\pgfqpoint{4.213357in}{1.917018in}}%
\pgfpathlineto{\pgfqpoint{4.205571in}{1.904065in}}%
\pgfpathclose%
\pgfusepath{fill}%
\end{pgfscope}%
\begin{pgfscope}%
\pgfpathrectangle{\pgfqpoint{1.254980in}{0.150000in}}{\pgfqpoint{5.490039in}{5.490039in}}%
\pgfusepath{clip}%
\pgfsetbuttcap%
\pgfsetroundjoin%
\definecolor{currentfill}{rgb}{0.430983,0.808473,0.346476}%
\pgfsetfillcolor{currentfill}%
\pgfsetfillopacity{0.700000}%
\pgfsetlinewidth{0.000000pt}%
\definecolor{currentstroke}{rgb}{0.000000,0.000000,0.000000}%
\pgfsetstrokecolor{currentstroke}%
\pgfsetdash{}{0pt}%
\pgfpathmoveto{\pgfqpoint{5.793467in}{3.418482in}}%
\pgfpathlineto{\pgfqpoint{5.808055in}{3.433621in}}%
\pgfpathlineto{\pgfqpoint{5.822664in}{3.448923in}}%
\pgfpathlineto{\pgfqpoint{5.837296in}{3.464388in}}%
\pgfpathlineto{\pgfqpoint{5.844237in}{3.465132in}}%
\pgfpathlineto{\pgfqpoint{5.851167in}{3.465790in}}%
\pgfpathlineto{\pgfqpoint{5.858088in}{3.466368in}}%
\pgfpathlineto{\pgfqpoint{5.864999in}{3.466869in}}%
\pgfpathlineto{\pgfqpoint{5.850394in}{3.451829in}}%
\pgfpathlineto{\pgfqpoint{5.835811in}{3.436951in}}%
\pgfpathlineto{\pgfqpoint{5.821250in}{3.422234in}}%
\pgfpathlineto{\pgfqpoint{5.814318in}{3.421407in}}%
\pgfpathlineto{\pgfqpoint{5.807378in}{3.420509in}}%
\pgfpathlineto{\pgfqpoint{5.800427in}{3.419536in}}%
\pgfpathlineto{\pgfqpoint{5.793467in}{3.418482in}}%
\pgfpathclose%
\pgfusepath{fill}%
\end{pgfscope}%
\begin{pgfscope}%
\pgfpathrectangle{\pgfqpoint{1.254980in}{0.150000in}}{\pgfqpoint{5.490039in}{5.490039in}}%
\pgfusepath{clip}%
\pgfsetbuttcap%
\pgfsetroundjoin%
\definecolor{currentfill}{rgb}{0.280255,0.165693,0.476498}%
\pgfsetfillcolor{currentfill}%
\pgfsetfillopacity{0.700000}%
\pgfsetlinewidth{0.000000pt}%
\definecolor{currentstroke}{rgb}{0.000000,0.000000,0.000000}%
\pgfsetstrokecolor{currentstroke}%
\pgfsetdash{}{0pt}%
\pgfpathmoveto{\pgfqpoint{4.003253in}{1.712345in}}%
\pgfpathlineto{\pgfqpoint{4.016785in}{1.715953in}}%
\pgfpathlineto{\pgfqpoint{4.030328in}{1.719721in}}%
\pgfpathlineto{\pgfqpoint{4.043881in}{1.723649in}}%
\pgfpathlineto{\pgfqpoint{4.057444in}{1.727738in}}%
\pgfpathlineto{\pgfqpoint{4.065284in}{1.740703in}}%
\pgfpathlineto{\pgfqpoint{4.073120in}{1.753660in}}%
\pgfpathlineto{\pgfqpoint{4.080951in}{1.766606in}}%
\pgfpathlineto{\pgfqpoint{4.088778in}{1.779537in}}%
\pgfpathlineto{\pgfqpoint{4.075218in}{1.775086in}}%
\pgfpathlineto{\pgfqpoint{4.061668in}{1.770795in}}%
\pgfpathlineto{\pgfqpoint{4.048130in}{1.766665in}}%
\pgfpathlineto{\pgfqpoint{4.034601in}{1.762695in}}%
\pgfpathlineto{\pgfqpoint{4.026771in}{1.750116in}}%
\pgfpathlineto{\pgfqpoint{4.018936in}{1.737529in}}%
\pgfpathlineto{\pgfqpoint{4.011097in}{1.724938in}}%
\pgfpathlineto{\pgfqpoint{4.003253in}{1.712345in}}%
\pgfpathclose%
\pgfusepath{fill}%
\end{pgfscope}%
\begin{pgfscope}%
\pgfpathrectangle{\pgfqpoint{1.254980in}{0.150000in}}{\pgfqpoint{5.490039in}{5.490039in}}%
\pgfusepath{clip}%
\pgfsetbuttcap%
\pgfsetroundjoin%
\definecolor{currentfill}{rgb}{0.281412,0.155834,0.469201}%
\pgfsetfillcolor{currentfill}%
\pgfsetfillopacity{0.700000}%
\pgfsetlinewidth{0.000000pt}%
\definecolor{currentstroke}{rgb}{0.000000,0.000000,0.000000}%
\pgfsetstrokecolor{currentstroke}%
\pgfsetdash{}{0pt}%
\pgfpathmoveto{\pgfqpoint{2.766903in}{1.755315in}}%
\pgfpathlineto{\pgfqpoint{2.780419in}{1.741796in}}%
\pgfpathlineto{\pgfqpoint{2.793931in}{1.728481in}}%
\pgfpathlineto{\pgfqpoint{2.807439in}{1.715371in}}%
\pgfpathlineto{\pgfqpoint{2.820943in}{1.702463in}}%
\pgfpathlineto{\pgfqpoint{2.829489in}{1.701052in}}%
\pgfpathlineto{\pgfqpoint{2.838017in}{1.699954in}}%
\pgfpathlineto{\pgfqpoint{2.846529in}{1.699162in}}%
\pgfpathlineto{\pgfqpoint{2.855023in}{1.698669in}}%
\pgfpathlineto{\pgfqpoint{2.841563in}{1.710922in}}%
\pgfpathlineto{\pgfqpoint{2.828100in}{1.723377in}}%
\pgfpathlineto{\pgfqpoint{2.814633in}{1.736035in}}%
\pgfpathlineto{\pgfqpoint{2.801162in}{1.748898in}}%
\pgfpathlineto{\pgfqpoint{2.792624in}{1.750035in}}%
\pgfpathlineto{\pgfqpoint{2.784068in}{1.751479in}}%
\pgfpathlineto{\pgfqpoint{2.775495in}{1.753237in}}%
\pgfpathlineto{\pgfqpoint{2.766903in}{1.755315in}}%
\pgfpathclose%
\pgfusepath{fill}%
\end{pgfscope}%
\begin{pgfscope}%
\pgfpathrectangle{\pgfqpoint{1.254980in}{0.150000in}}{\pgfqpoint{5.490039in}{5.490039in}}%
\pgfusepath{clip}%
\pgfsetbuttcap%
\pgfsetroundjoin%
\definecolor{currentfill}{rgb}{0.159194,0.482237,0.558073}%
\pgfsetfillcolor{currentfill}%
\pgfsetfillopacity{0.700000}%
\pgfsetlinewidth{0.000000pt}%
\definecolor{currentstroke}{rgb}{0.000000,0.000000,0.000000}%
\pgfsetstrokecolor{currentstroke}%
\pgfsetdash{}{0pt}%
\pgfpathmoveto{\pgfqpoint{4.727018in}{2.460917in}}%
\pgfpathlineto{\pgfqpoint{4.740914in}{2.471463in}}%
\pgfpathlineto{\pgfqpoint{4.754826in}{2.482170in}}%
\pgfpathlineto{\pgfqpoint{4.768754in}{2.493039in}}%
\pgfpathlineto{\pgfqpoint{4.782700in}{2.504069in}}%
\pgfpathlineto{\pgfqpoint{4.790311in}{2.514894in}}%
\pgfpathlineto{\pgfqpoint{4.797915in}{2.525582in}}%
\pgfpathlineto{\pgfqpoint{4.805513in}{2.536132in}}%
\pgfpathlineto{\pgfqpoint{4.813104in}{2.546545in}}%
\pgfpathlineto{\pgfqpoint{4.799160in}{2.535467in}}%
\pgfpathlineto{\pgfqpoint{4.785234in}{2.524551in}}%
\pgfpathlineto{\pgfqpoint{4.771323in}{2.513797in}}%
\pgfpathlineto{\pgfqpoint{4.757429in}{2.503204in}}%
\pgfpathlineto{\pgfqpoint{4.749836in}{2.492827in}}%
\pgfpathlineto{\pgfqpoint{4.742237in}{2.482321in}}%
\pgfpathlineto{\pgfqpoint{4.734631in}{2.471684in}}%
\pgfpathlineto{\pgfqpoint{4.727018in}{2.460917in}}%
\pgfpathclose%
\pgfusepath{fill}%
\end{pgfscope}%
\begin{pgfscope}%
\pgfpathrectangle{\pgfqpoint{1.254980in}{0.150000in}}{\pgfqpoint{5.490039in}{5.490039in}}%
\pgfusepath{clip}%
\pgfsetbuttcap%
\pgfsetroundjoin%
\definecolor{currentfill}{rgb}{0.214298,0.355619,0.551184}%
\pgfsetfillcolor{currentfill}%
\pgfsetfillopacity{0.700000}%
\pgfsetlinewidth{0.000000pt}%
\definecolor{currentstroke}{rgb}{0.000000,0.000000,0.000000}%
\pgfsetstrokecolor{currentstroke}%
\pgfsetdash{}{0pt}%
\pgfpathmoveto{\pgfqpoint{2.385945in}{2.222996in}}%
\pgfpathlineto{\pgfqpoint{2.399663in}{2.203120in}}%
\pgfpathlineto{\pgfqpoint{2.413370in}{2.183495in}}%
\pgfpathlineto{\pgfqpoint{2.427067in}{2.164119in}}%
\pgfpathlineto{\pgfqpoint{2.440754in}{2.144990in}}%
\pgfpathlineto{\pgfqpoint{2.449662in}{2.138825in}}%
\pgfpathlineto{\pgfqpoint{2.458547in}{2.133032in}}%
\pgfpathlineto{\pgfqpoint{2.467408in}{2.127602in}}%
\pgfpathlineto{\pgfqpoint{2.476247in}{2.122530in}}%
\pgfpathlineto{\pgfqpoint{2.462618in}{2.140976in}}%
\pgfpathlineto{\pgfqpoint{2.448979in}{2.159668in}}%
\pgfpathlineto{\pgfqpoint{2.435331in}{2.178609in}}%
\pgfpathlineto{\pgfqpoint{2.421672in}{2.197799in}}%
\pgfpathlineto{\pgfqpoint{2.412776in}{2.203542in}}%
\pgfpathlineto{\pgfqpoint{2.403856in}{2.209652in}}%
\pgfpathlineto{\pgfqpoint{2.394913in}{2.216134in}}%
\pgfpathlineto{\pgfqpoint{2.385945in}{2.222996in}}%
\pgfpathclose%
\pgfusepath{fill}%
\end{pgfscope}%
\begin{pgfscope}%
\pgfpathrectangle{\pgfqpoint{1.254980in}{0.150000in}}{\pgfqpoint{5.490039in}{5.490039in}}%
\pgfusepath{clip}%
\pgfsetbuttcap%
\pgfsetroundjoin%
\definecolor{currentfill}{rgb}{0.216210,0.351535,0.550627}%
\pgfsetfillcolor{currentfill}%
\pgfsetfillopacity{0.700000}%
\pgfsetlinewidth{0.000000pt}%
\definecolor{currentstroke}{rgb}{0.000000,0.000000,0.000000}%
\pgfsetstrokecolor{currentstroke}%
\pgfsetdash{}{0pt}%
\pgfpathmoveto{\pgfqpoint{4.407939in}{2.113333in}}%
\pgfpathlineto{\pgfqpoint{4.421655in}{2.121266in}}%
\pgfpathlineto{\pgfqpoint{4.435386in}{2.129359in}}%
\pgfpathlineto{\pgfqpoint{4.449130in}{2.137613in}}%
\pgfpathlineto{\pgfqpoint{4.462889in}{2.146027in}}%
\pgfpathlineto{\pgfqpoint{4.470617in}{2.158841in}}%
\pgfpathlineto{\pgfqpoint{4.478340in}{2.171557in}}%
\pgfpathlineto{\pgfqpoint{4.486057in}{2.184174in}}%
\pgfpathlineto{\pgfqpoint{4.493770in}{2.196692in}}%
\pgfpathlineto{\pgfqpoint{4.480011in}{2.188082in}}%
\pgfpathlineto{\pgfqpoint{4.466266in}{2.179634in}}%
\pgfpathlineto{\pgfqpoint{4.452536in}{2.171346in}}%
\pgfpathlineto{\pgfqpoint{4.438819in}{2.163220in}}%
\pgfpathlineto{\pgfqpoint{4.431107in}{2.150886in}}%
\pgfpathlineto{\pgfqpoint{4.423389in}{2.138459in}}%
\pgfpathlineto{\pgfqpoint{4.415666in}{2.125941in}}%
\pgfpathlineto{\pgfqpoint{4.407939in}{2.113333in}}%
\pgfpathclose%
\pgfusepath{fill}%
\end{pgfscope}%
\begin{pgfscope}%
\pgfpathrectangle{\pgfqpoint{1.254980in}{0.150000in}}{\pgfqpoint{5.490039in}{5.490039in}}%
\pgfusepath{clip}%
\pgfsetbuttcap%
\pgfsetroundjoin%
\definecolor{currentfill}{rgb}{0.268510,0.009605,0.335427}%
\pgfsetfillcolor{currentfill}%
\pgfsetfillopacity{0.700000}%
\pgfsetlinewidth{0.000000pt}%
\definecolor{currentstroke}{rgb}{0.000000,0.000000,0.000000}%
\pgfsetstrokecolor{currentstroke}%
\pgfsetdash{}{0pt}%
\pgfpathmoveto{\pgfqpoint{3.264186in}{1.451325in}}%
\pgfpathlineto{\pgfqpoint{3.277601in}{1.445094in}}%
\pgfpathlineto{\pgfqpoint{3.291019in}{1.439036in}}%
\pgfpathlineto{\pgfqpoint{3.304439in}{1.433152in}}%
\pgfpathlineto{\pgfqpoint{3.317861in}{1.427440in}}%
\pgfpathlineto{\pgfqpoint{3.326022in}{1.433120in}}%
\pgfpathlineto{\pgfqpoint{3.334173in}{1.439003in}}%
\pgfpathlineto{\pgfqpoint{3.342313in}{1.445080in}}%
\pgfpathlineto{\pgfqpoint{3.350444in}{1.451347in}}%
\pgfpathlineto{\pgfqpoint{3.337046in}{1.456477in}}%
\pgfpathlineto{\pgfqpoint{3.323651in}{1.461779in}}%
\pgfpathlineto{\pgfqpoint{3.310259in}{1.467255in}}%
\pgfpathlineto{\pgfqpoint{3.296869in}{1.472904in}}%
\pgfpathlineto{\pgfqpoint{3.288714in}{1.467208in}}%
\pgfpathlineto{\pgfqpoint{3.280549in}{1.461709in}}%
\pgfpathlineto{\pgfqpoint{3.272373in}{1.456413in}}%
\pgfpathlineto{\pgfqpoint{3.264186in}{1.451325in}}%
\pgfpathclose%
\pgfusepath{fill}%
\end{pgfscope}%
\begin{pgfscope}%
\pgfpathrectangle{\pgfqpoint{1.254980in}{0.150000in}}{\pgfqpoint{5.490039in}{5.490039in}}%
\pgfusepath{clip}%
\pgfsetbuttcap%
\pgfsetroundjoin%
\definecolor{currentfill}{rgb}{0.122312,0.633153,0.530398}%
\pgfsetfillcolor{currentfill}%
\pgfsetfillopacity{0.700000}%
\pgfsetlinewidth{0.000000pt}%
\definecolor{currentstroke}{rgb}{0.000000,0.000000,0.000000}%
\pgfsetstrokecolor{currentstroke}%
\pgfsetdash{}{0pt}%
\pgfpathmoveto{\pgfqpoint{5.131824in}{2.869481in}}%
\pgfpathlineto{\pgfqpoint{5.145977in}{2.882444in}}%
\pgfpathlineto{\pgfqpoint{5.160148in}{2.895570in}}%
\pgfpathlineto{\pgfqpoint{5.174339in}{2.908859in}}%
\pgfpathlineto{\pgfqpoint{5.188549in}{2.922310in}}%
\pgfpathlineto{\pgfqpoint{5.195955in}{2.929468in}}%
\pgfpathlineto{\pgfqpoint{5.203353in}{2.936476in}}%
\pgfpathlineto{\pgfqpoint{5.210741in}{2.943338in}}%
\pgfpathlineto{\pgfqpoint{5.218121in}{2.950054in}}%
\pgfpathlineto{\pgfqpoint{5.203920in}{2.936740in}}%
\pgfpathlineto{\pgfqpoint{5.189738in}{2.923590in}}%
\pgfpathlineto{\pgfqpoint{5.175575in}{2.910601in}}%
\pgfpathlineto{\pgfqpoint{5.161430in}{2.897775in}}%
\pgfpathlineto{\pgfqpoint{5.154042in}{2.890909in}}%
\pgfpathlineto{\pgfqpoint{5.146645in}{2.883907in}}%
\pgfpathlineto{\pgfqpoint{5.139239in}{2.876764in}}%
\pgfpathlineto{\pgfqpoint{5.131824in}{2.869481in}}%
\pgfpathclose%
\pgfusepath{fill}%
\end{pgfscope}%
\begin{pgfscope}%
\pgfpathrectangle{\pgfqpoint{1.254980in}{0.150000in}}{\pgfqpoint{5.490039in}{5.490039in}}%
\pgfusepath{clip}%
\pgfsetbuttcap%
\pgfsetroundjoin%
\definecolor{currentfill}{rgb}{0.129933,0.559582,0.551864}%
\pgfsetfillcolor{currentfill}%
\pgfsetfillopacity{0.700000}%
\pgfsetlinewidth{0.000000pt}%
\definecolor{currentstroke}{rgb}{0.000000,0.000000,0.000000}%
\pgfsetstrokecolor{currentstroke}%
\pgfsetdash{}{0pt}%
\pgfpathmoveto{\pgfqpoint{4.929500in}{2.670858in}}%
\pgfpathlineto{\pgfqpoint{4.943523in}{2.682749in}}%
\pgfpathlineto{\pgfqpoint{4.957565in}{2.694801in}}%
\pgfpathlineto{\pgfqpoint{4.971624in}{2.707016in}}%
\pgfpathlineto{\pgfqpoint{4.985701in}{2.719394in}}%
\pgfpathlineto{\pgfqpoint{4.993220in}{2.728515in}}%
\pgfpathlineto{\pgfqpoint{5.000731in}{2.737487in}}%
\pgfpathlineto{\pgfqpoint{5.008234in}{2.746311in}}%
\pgfpathlineto{\pgfqpoint{5.015730in}{2.754988in}}%
\pgfpathlineto{\pgfqpoint{5.001657in}{2.742655in}}%
\pgfpathlineto{\pgfqpoint{4.987602in}{2.730485in}}%
\pgfpathlineto{\pgfqpoint{4.973565in}{2.718476in}}%
\pgfpathlineto{\pgfqpoint{4.959546in}{2.706629in}}%
\pgfpathlineto{\pgfqpoint{4.952046in}{2.697896in}}%
\pgfpathlineto{\pgfqpoint{4.944538in}{2.689024in}}%
\pgfpathlineto{\pgfqpoint{4.937023in}{2.680012in}}%
\pgfpathlineto{\pgfqpoint{4.929500in}{2.670858in}}%
\pgfpathclose%
\pgfusepath{fill}%
\end{pgfscope}%
\begin{pgfscope}%
\pgfpathrectangle{\pgfqpoint{1.254980in}{0.150000in}}{\pgfqpoint{5.490039in}{5.490039in}}%
\pgfusepath{clip}%
\pgfsetbuttcap%
\pgfsetroundjoin%
\definecolor{currentfill}{rgb}{0.283072,0.130895,0.449241}%
\pgfsetfillcolor{currentfill}%
\pgfsetfillopacity{0.700000}%
\pgfsetlinewidth{0.000000pt}%
\definecolor{currentstroke}{rgb}{0.000000,0.000000,0.000000}%
\pgfsetstrokecolor{currentstroke}%
\pgfsetdash{}{0pt}%
\pgfpathmoveto{\pgfqpoint{2.820943in}{1.702463in}}%
\pgfpathlineto{\pgfqpoint{2.834443in}{1.689756in}}%
\pgfpathlineto{\pgfqpoint{2.847940in}{1.677249in}}%
\pgfpathlineto{\pgfqpoint{2.861434in}{1.664941in}}%
\pgfpathlineto{\pgfqpoint{2.874925in}{1.652831in}}%
\pgfpathlineto{\pgfqpoint{2.883428in}{1.652086in}}%
\pgfpathlineto{\pgfqpoint{2.891913in}{1.651645in}}%
\pgfpathlineto{\pgfqpoint{2.900383in}{1.651501in}}%
\pgfpathlineto{\pgfqpoint{2.908836in}{1.651649in}}%
\pgfpathlineto{\pgfqpoint{2.895387in}{1.663108in}}%
\pgfpathlineto{\pgfqpoint{2.881935in}{1.674763in}}%
\pgfpathlineto{\pgfqpoint{2.868481in}{1.686616in}}%
\pgfpathlineto{\pgfqpoint{2.855023in}{1.698669in}}%
\pgfpathlineto{\pgfqpoint{2.846529in}{1.699162in}}%
\pgfpathlineto{\pgfqpoint{2.838017in}{1.699954in}}%
\pgfpathlineto{\pgfqpoint{2.829489in}{1.701052in}}%
\pgfpathlineto{\pgfqpoint{2.820943in}{1.702463in}}%
\pgfpathclose%
\pgfusepath{fill}%
\end{pgfscope}%
\begin{pgfscope}%
\pgfpathrectangle{\pgfqpoint{1.254980in}{0.150000in}}{\pgfqpoint{5.490039in}{5.490039in}}%
\pgfusepath{clip}%
\pgfsetbuttcap%
\pgfsetroundjoin%
\definecolor{currentfill}{rgb}{0.267004,0.004874,0.329415}%
\pgfsetfillcolor{currentfill}%
\pgfsetfillopacity{0.700000}%
\pgfsetlinewidth{0.000000pt}%
\definecolor{currentstroke}{rgb}{0.000000,0.000000,0.000000}%
\pgfsetstrokecolor{currentstroke}%
\pgfsetdash{}{0pt}%
\pgfpathmoveto{\pgfqpoint{3.404069in}{1.432542in}}%
\pgfpathlineto{\pgfqpoint{3.417485in}{1.428266in}}%
\pgfpathlineto{\pgfqpoint{3.430904in}{1.424158in}}%
\pgfpathlineto{\pgfqpoint{3.444327in}{1.420220in}}%
\pgfpathlineto{\pgfqpoint{3.457755in}{1.416449in}}%
\pgfpathlineto{\pgfqpoint{3.465833in}{1.424028in}}%
\pgfpathlineto{\pgfqpoint{3.473903in}{1.431772in}}%
\pgfpathlineto{\pgfqpoint{3.481964in}{1.439674in}}%
\pgfpathlineto{\pgfqpoint{3.490018in}{1.447730in}}%
\pgfpathlineto{\pgfqpoint{3.476610in}{1.450948in}}%
\pgfpathlineto{\pgfqpoint{3.463206in}{1.454333in}}%
\pgfpathlineto{\pgfqpoint{3.449807in}{1.457887in}}%
\pgfpathlineto{\pgfqpoint{3.436413in}{1.461610in}}%
\pgfpathlineto{\pgfqpoint{3.428340in}{1.454097in}}%
\pgfpathlineto{\pgfqpoint{3.420258in}{1.446744in}}%
\pgfpathlineto{\pgfqpoint{3.412168in}{1.439557in}}%
\pgfpathlineto{\pgfqpoint{3.404069in}{1.432542in}}%
\pgfpathclose%
\pgfusepath{fill}%
\end{pgfscope}%
\begin{pgfscope}%
\pgfpathrectangle{\pgfqpoint{1.254980in}{0.150000in}}{\pgfqpoint{5.490039in}{5.490039in}}%
\pgfusepath{clip}%
\pgfsetbuttcap%
\pgfsetroundjoin%
\definecolor{currentfill}{rgb}{0.220124,0.725509,0.466226}%
\pgfsetfillcolor{currentfill}%
\pgfsetfillopacity{0.700000}%
\pgfsetlinewidth{0.000000pt}%
\definecolor{currentstroke}{rgb}{0.000000,0.000000,0.000000}%
\pgfsetstrokecolor{currentstroke}%
\pgfsetdash{}{0pt}%
\pgfpathmoveto{\pgfqpoint{5.420017in}{3.127673in}}%
\pgfpathlineto{\pgfqpoint{5.434362in}{3.141849in}}%
\pgfpathlineto{\pgfqpoint{5.448727in}{3.156188in}}%
\pgfpathlineto{\pgfqpoint{5.463113in}{3.170690in}}%
\pgfpathlineto{\pgfqpoint{5.477520in}{3.185355in}}%
\pgfpathlineto{\pgfqpoint{5.484740in}{3.189636in}}%
\pgfpathlineto{\pgfqpoint{5.491951in}{3.193783in}}%
\pgfpathlineto{\pgfqpoint{5.499151in}{3.197800in}}%
\pgfpathlineto{\pgfqpoint{5.506342in}{3.201690in}}%
\pgfpathlineto{\pgfqpoint{5.491951in}{3.187289in}}%
\pgfpathlineto{\pgfqpoint{5.477580in}{3.173052in}}%
\pgfpathlineto{\pgfqpoint{5.463230in}{3.158977in}}%
\pgfpathlineto{\pgfqpoint{5.448901in}{3.145065in}}%
\pgfpathlineto{\pgfqpoint{5.441694in}{3.140900in}}%
\pgfpathlineto{\pgfqpoint{5.434478in}{3.136615in}}%
\pgfpathlineto{\pgfqpoint{5.427252in}{3.132207in}}%
\pgfpathlineto{\pgfqpoint{5.420017in}{3.127673in}}%
\pgfpathclose%
\pgfusepath{fill}%
\end{pgfscope}%
\begin{pgfscope}%
\pgfpathrectangle{\pgfqpoint{1.254980in}{0.150000in}}{\pgfqpoint{5.490039in}{5.490039in}}%
\pgfusepath{clip}%
\pgfsetbuttcap%
\pgfsetroundjoin%
\definecolor{currentfill}{rgb}{0.133743,0.548535,0.553541}%
\pgfsetfillcolor{currentfill}%
\pgfsetfillopacity{0.700000}%
\pgfsetlinewidth{0.000000pt}%
\definecolor{currentstroke}{rgb}{0.000000,0.000000,0.000000}%
\pgfsetstrokecolor{currentstroke}%
\pgfsetdash{}{0pt}%
\pgfpathmoveto{\pgfqpoint{2.090189in}{2.739383in}}%
\pgfpathlineto{\pgfqpoint{2.104170in}{2.713472in}}%
\pgfpathlineto{\pgfqpoint{2.118134in}{2.687872in}}%
\pgfpathlineto{\pgfqpoint{2.132081in}{2.662582in}}%
\pgfpathlineto{\pgfqpoint{2.146012in}{2.637598in}}%
\pgfpathlineto{\pgfqpoint{2.155201in}{2.628812in}}%
\pgfpathlineto{\pgfqpoint{2.164363in}{2.620424in}}%
\pgfpathlineto{\pgfqpoint{2.173499in}{2.612425in}}%
\pgfpathlineto{\pgfqpoint{2.182607in}{2.604809in}}%
\pgfpathlineto{\pgfqpoint{2.168745in}{2.629109in}}%
\pgfpathlineto{\pgfqpoint{2.154866in}{2.653714in}}%
\pgfpathlineto{\pgfqpoint{2.140971in}{2.678625in}}%
\pgfpathlineto{\pgfqpoint{2.127060in}{2.703846in}}%
\pgfpathlineto{\pgfqpoint{2.117883in}{2.712135in}}%
\pgfpathlineto{\pgfqpoint{2.108680in}{2.720816in}}%
\pgfpathlineto{\pgfqpoint{2.099449in}{2.729896in}}%
\pgfpathlineto{\pgfqpoint{2.090189in}{2.739383in}}%
\pgfpathclose%
\pgfusepath{fill}%
\end{pgfscope}%
\begin{pgfscope}%
\pgfpathrectangle{\pgfqpoint{1.254980in}{0.150000in}}{\pgfqpoint{5.490039in}{5.490039in}}%
\pgfusepath{clip}%
\pgfsetbuttcap%
\pgfsetroundjoin%
\definecolor{currentfill}{rgb}{0.197636,0.391528,0.554969}%
\pgfsetfillcolor{currentfill}%
\pgfsetfillopacity{0.700000}%
\pgfsetlinewidth{0.000000pt}%
\definecolor{currentstroke}{rgb}{0.000000,0.000000,0.000000}%
\pgfsetstrokecolor{currentstroke}%
\pgfsetdash{}{0pt}%
\pgfpathmoveto{\pgfqpoint{2.330964in}{2.305058in}}%
\pgfpathlineto{\pgfqpoint{2.344726in}{2.284155in}}%
\pgfpathlineto{\pgfqpoint{2.358477in}{2.263511in}}%
\pgfpathlineto{\pgfqpoint{2.372216in}{2.243126in}}%
\pgfpathlineto{\pgfqpoint{2.385945in}{2.222996in}}%
\pgfpathlineto{\pgfqpoint{2.394913in}{2.216134in}}%
\pgfpathlineto{\pgfqpoint{2.403856in}{2.209652in}}%
\pgfpathlineto{\pgfqpoint{2.412776in}{2.203542in}}%
\pgfpathlineto{\pgfqpoint{2.421672in}{2.197799in}}%
\pgfpathlineto{\pgfqpoint{2.408004in}{2.217241in}}%
\pgfpathlineto{\pgfqpoint{2.394325in}{2.236937in}}%
\pgfpathlineto{\pgfqpoint{2.380636in}{2.256890in}}%
\pgfpathlineto{\pgfqpoint{2.366936in}{2.277101in}}%
\pgfpathlineto{\pgfqpoint{2.357980in}{2.283521in}}%
\pgfpathlineto{\pgfqpoint{2.348999in}{2.290316in}}%
\pgfpathlineto{\pgfqpoint{2.339994in}{2.297492in}}%
\pgfpathlineto{\pgfqpoint{2.330964in}{2.305058in}}%
\pgfpathclose%
\pgfusepath{fill}%
\end{pgfscope}%
\begin{pgfscope}%
\pgfpathrectangle{\pgfqpoint{1.254980in}{0.150000in}}{\pgfqpoint{5.490039in}{5.490039in}}%
\pgfusepath{clip}%
\pgfsetbuttcap%
\pgfsetroundjoin%
\definecolor{currentfill}{rgb}{0.274128,0.199721,0.498911}%
\pgfsetfillcolor{currentfill}%
\pgfsetfillopacity{0.700000}%
\pgfsetlinewidth{0.000000pt}%
\definecolor{currentstroke}{rgb}{0.000000,0.000000,0.000000}%
\pgfsetstrokecolor{currentstroke}%
\pgfsetdash{}{0pt}%
\pgfpathmoveto{\pgfqpoint{4.088778in}{1.779537in}}%
\pgfpathlineto{\pgfqpoint{4.102350in}{1.784149in}}%
\pgfpathlineto{\pgfqpoint{4.115932in}{1.788921in}}%
\pgfpathlineto{\pgfqpoint{4.129526in}{1.793853in}}%
\pgfpathlineto{\pgfqpoint{4.143131in}{1.798945in}}%
\pgfpathlineto{\pgfqpoint{4.150951in}{1.812204in}}%
\pgfpathlineto{\pgfqpoint{4.158767in}{1.825435in}}%
\pgfpathlineto{\pgfqpoint{4.166579in}{1.838635in}}%
\pgfpathlineto{\pgfqpoint{4.174386in}{1.851800in}}%
\pgfpathlineto{\pgfqpoint{4.160783in}{1.846373in}}%
\pgfpathlineto{\pgfqpoint{4.147191in}{1.841105in}}%
\pgfpathlineto{\pgfqpoint{4.133611in}{1.835998in}}%
\pgfpathlineto{\pgfqpoint{4.120042in}{1.831052in}}%
\pgfpathlineto{\pgfqpoint{4.112233in}{1.818211in}}%
\pgfpathlineto{\pgfqpoint{4.104419in}{1.805343in}}%
\pgfpathlineto{\pgfqpoint{4.096601in}{1.792450in}}%
\pgfpathlineto{\pgfqpoint{4.088778in}{1.779537in}}%
\pgfpathclose%
\pgfusepath{fill}%
\end{pgfscope}%
\begin{pgfscope}%
\pgfpathrectangle{\pgfqpoint{1.254980in}{0.150000in}}{\pgfqpoint{5.490039in}{5.490039in}}%
\pgfusepath{clip}%
\pgfsetbuttcap%
\pgfsetroundjoin%
\definecolor{currentfill}{rgb}{0.277941,0.056324,0.381191}%
\pgfsetfillcolor{currentfill}%
\pgfsetfillopacity{0.700000}%
\pgfsetlinewidth{0.000000pt}%
\definecolor{currentstroke}{rgb}{0.000000,0.000000,0.000000}%
\pgfsetstrokecolor{currentstroke}%
\pgfsetdash{}{0pt}%
\pgfpathmoveto{\pgfqpoint{3.715126in}{1.503680in}}%
\pgfpathlineto{\pgfqpoint{3.728582in}{1.503630in}}%
\pgfpathlineto{\pgfqpoint{3.742046in}{1.503742in}}%
\pgfpathlineto{\pgfqpoint{3.755517in}{1.504017in}}%
\pgfpathlineto{\pgfqpoint{3.768996in}{1.504453in}}%
\pgfpathlineto{\pgfqpoint{3.776933in}{1.515583in}}%
\pgfpathlineto{\pgfqpoint{3.784864in}{1.526789in}}%
\pgfpathlineto{\pgfqpoint{3.792790in}{1.538066in}}%
\pgfpathlineto{\pgfqpoint{3.800710in}{1.549411in}}%
\pgfpathlineto{\pgfqpoint{3.787241in}{1.548504in}}%
\pgfpathlineto{\pgfqpoint{3.773779in}{1.547758in}}%
\pgfpathlineto{\pgfqpoint{3.760326in}{1.547175in}}%
\pgfpathlineto{\pgfqpoint{3.746880in}{1.546755in}}%
\pgfpathlineto{\pgfqpoint{3.738950in}{1.535870in}}%
\pgfpathlineto{\pgfqpoint{3.731015in}{1.525060in}}%
\pgfpathlineto{\pgfqpoint{3.723073in}{1.514329in}}%
\pgfpathlineto{\pgfqpoint{3.715126in}{1.503680in}}%
\pgfpathclose%
\pgfusepath{fill}%
\end{pgfscope}%
\begin{pgfscope}%
\pgfpathrectangle{\pgfqpoint{1.254980in}{0.150000in}}{\pgfqpoint{5.490039in}{5.490039in}}%
\pgfusepath{clip}%
\pgfsetbuttcap%
\pgfsetroundjoin%
\definecolor{currentfill}{rgb}{0.177423,0.437527,0.557565}%
\pgfsetfillcolor{currentfill}%
\pgfsetfillopacity{0.700000}%
\pgfsetlinewidth{0.000000pt}%
\definecolor{currentstroke}{rgb}{0.000000,0.000000,0.000000}%
\pgfsetstrokecolor{currentstroke}%
\pgfsetdash{}{0pt}%
\pgfpathmoveto{\pgfqpoint{4.610460in}{2.330311in}}%
\pgfpathlineto{\pgfqpoint{4.624294in}{2.340027in}}%
\pgfpathlineto{\pgfqpoint{4.638144in}{2.349904in}}%
\pgfpathlineto{\pgfqpoint{4.652009in}{2.359943in}}%
\pgfpathlineto{\pgfqpoint{4.665889in}{2.370143in}}%
\pgfpathlineto{\pgfqpoint{4.673552in}{2.381936in}}%
\pgfpathlineto{\pgfqpoint{4.681209in}{2.393603in}}%
\pgfpathlineto{\pgfqpoint{4.688859in}{2.405143in}}%
\pgfpathlineto{\pgfqpoint{4.696504in}{2.416556in}}%
\pgfpathlineto{\pgfqpoint{4.682623in}{2.406249in}}%
\pgfpathlineto{\pgfqpoint{4.668758in}{2.396103in}}%
\pgfpathlineto{\pgfqpoint{4.654909in}{2.386119in}}%
\pgfpathlineto{\pgfqpoint{4.641076in}{2.376296in}}%
\pgfpathlineto{\pgfqpoint{4.633431in}{2.364979in}}%
\pgfpathlineto{\pgfqpoint{4.625780in}{2.353543in}}%
\pgfpathlineto{\pgfqpoint{4.618123in}{2.341986in}}%
\pgfpathlineto{\pgfqpoint{4.610460in}{2.330311in}}%
\pgfpathclose%
\pgfusepath{fill}%
\end{pgfscope}%
\begin{pgfscope}%
\pgfpathrectangle{\pgfqpoint{1.254980in}{0.150000in}}{\pgfqpoint{5.490039in}{5.490039in}}%
\pgfusepath{clip}%
\pgfsetbuttcap%
\pgfsetroundjoin%
\definecolor{currentfill}{rgb}{0.273809,0.031497,0.358853}%
\pgfsetfillcolor{currentfill}%
\pgfsetfillopacity{0.700000}%
\pgfsetlinewidth{0.000000pt}%
\definecolor{currentstroke}{rgb}{0.000000,0.000000,0.000000}%
\pgfsetstrokecolor{currentstroke}%
\pgfsetdash{}{0pt}%
\pgfpathmoveto{\pgfqpoint{3.123831in}{1.494279in}}%
\pgfpathlineto{\pgfqpoint{3.137265in}{1.486017in}}%
\pgfpathlineto{\pgfqpoint{3.150700in}{1.477935in}}%
\pgfpathlineto{\pgfqpoint{3.164135in}{1.470032in}}%
\pgfpathlineto{\pgfqpoint{3.177572in}{1.462308in}}%
\pgfpathlineto{\pgfqpoint{3.185833in}{1.465878in}}%
\pgfpathlineto{\pgfqpoint{3.194082in}{1.469688in}}%
\pgfpathlineto{\pgfqpoint{3.202319in}{1.473732in}}%
\pgfpathlineto{\pgfqpoint{3.210545in}{1.478004in}}%
\pgfpathlineto{\pgfqpoint{3.197139in}{1.485116in}}%
\pgfpathlineto{\pgfqpoint{3.183734in}{1.492405in}}%
\pgfpathlineto{\pgfqpoint{3.170331in}{1.499874in}}%
\pgfpathlineto{\pgfqpoint{3.156928in}{1.507522in}}%
\pgfpathlineto{\pgfqpoint{3.148673in}{1.503852in}}%
\pgfpathlineto{\pgfqpoint{3.140405in}{1.500417in}}%
\pgfpathlineto{\pgfqpoint{3.132125in}{1.497224in}}%
\pgfpathlineto{\pgfqpoint{3.123831in}{1.494279in}}%
\pgfpathclose%
\pgfusepath{fill}%
\end{pgfscope}%
\begin{pgfscope}%
\pgfpathrectangle{\pgfqpoint{1.254980in}{0.150000in}}{\pgfqpoint{5.490039in}{5.490039in}}%
\pgfusepath{clip}%
\pgfsetbuttcap%
\pgfsetroundjoin%
\definecolor{currentfill}{rgb}{0.283091,0.110553,0.431554}%
\pgfsetfillcolor{currentfill}%
\pgfsetfillopacity{0.700000}%
\pgfsetlinewidth{0.000000pt}%
\definecolor{currentstroke}{rgb}{0.000000,0.000000,0.000000}%
\pgfsetstrokecolor{currentstroke}%
\pgfsetdash{}{0pt}%
\pgfpathmoveto{\pgfqpoint{2.874925in}{1.652831in}}%
\pgfpathlineto{\pgfqpoint{2.888414in}{1.640918in}}%
\pgfpathlineto{\pgfqpoint{2.901900in}{1.629200in}}%
\pgfpathlineto{\pgfqpoint{2.915383in}{1.617676in}}%
\pgfpathlineto{\pgfqpoint{2.928864in}{1.606346in}}%
\pgfpathlineto{\pgfqpoint{2.937324in}{1.606263in}}%
\pgfpathlineto{\pgfqpoint{2.945769in}{1.606476in}}%
\pgfpathlineto{\pgfqpoint{2.954198in}{1.606980in}}%
\pgfpathlineto{\pgfqpoint{2.962612in}{1.607766in}}%
\pgfpathlineto{\pgfqpoint{2.949171in}{1.618447in}}%
\pgfpathlineto{\pgfqpoint{2.935728in}{1.629320in}}%
\pgfpathlineto{\pgfqpoint{2.922283in}{1.640387in}}%
\pgfpathlineto{\pgfqpoint{2.908836in}{1.651649in}}%
\pgfpathlineto{\pgfqpoint{2.900383in}{1.651501in}}%
\pgfpathlineto{\pgfqpoint{2.891913in}{1.651645in}}%
\pgfpathlineto{\pgfqpoint{2.883428in}{1.652086in}}%
\pgfpathlineto{\pgfqpoint{2.874925in}{1.652831in}}%
\pgfpathclose%
\pgfusepath{fill}%
\end{pgfscope}%
\begin{pgfscope}%
\pgfpathrectangle{\pgfqpoint{1.254980in}{0.150000in}}{\pgfqpoint{5.490039in}{5.490039in}}%
\pgfusepath{clip}%
\pgfsetbuttcap%
\pgfsetroundjoin%
\definecolor{currentfill}{rgb}{0.241237,0.296485,0.539709}%
\pgfsetfillcolor{currentfill}%
\pgfsetfillopacity{0.700000}%
\pgfsetlinewidth{0.000000pt}%
\definecolor{currentstroke}{rgb}{0.000000,0.000000,0.000000}%
\pgfsetstrokecolor{currentstroke}%
\pgfsetdash{}{0pt}%
\pgfpathmoveto{\pgfqpoint{4.291211in}{1.981224in}}%
\pgfpathlineto{\pgfqpoint{4.304874in}{1.988040in}}%
\pgfpathlineto{\pgfqpoint{4.318550in}{1.995017in}}%
\pgfpathlineto{\pgfqpoint{4.332240in}{2.002153in}}%
\pgfpathlineto{\pgfqpoint{4.345942in}{2.009450in}}%
\pgfpathlineto{\pgfqpoint{4.353708in}{2.022711in}}%
\pgfpathlineto{\pgfqpoint{4.361470in}{2.035897in}}%
\pgfpathlineto{\pgfqpoint{4.369227in}{2.049006in}}%
\pgfpathlineto{\pgfqpoint{4.376979in}{2.062037in}}%
\pgfpathlineto{\pgfqpoint{4.363276in}{2.054488in}}%
\pgfpathlineto{\pgfqpoint{4.349587in}{2.047099in}}%
\pgfpathlineto{\pgfqpoint{4.335911in}{2.039871in}}%
\pgfpathlineto{\pgfqpoint{4.322248in}{2.032804in}}%
\pgfpathlineto{\pgfqpoint{4.314496in}{2.020014in}}%
\pgfpathlineto{\pgfqpoint{4.306739in}{2.007153in}}%
\pgfpathlineto{\pgfqpoint{4.298977in}{1.994222in}}%
\pgfpathlineto{\pgfqpoint{4.291211in}{1.981224in}}%
\pgfpathclose%
\pgfusepath{fill}%
\end{pgfscope}%
\begin{pgfscope}%
\pgfpathrectangle{\pgfqpoint{1.254980in}{0.150000in}}{\pgfqpoint{5.490039in}{5.490039in}}%
\pgfusepath{clip}%
\pgfsetbuttcap%
\pgfsetroundjoin%
\definecolor{currentfill}{rgb}{0.281446,0.084320,0.407414}%
\pgfsetfillcolor{currentfill}%
\pgfsetfillopacity{0.700000}%
\pgfsetlinewidth{0.000000pt}%
\definecolor{currentstroke}{rgb}{0.000000,0.000000,0.000000}%
\pgfsetstrokecolor{currentstroke}%
\pgfsetdash{}{0pt}%
\pgfpathmoveto{\pgfqpoint{3.800710in}{1.549411in}}%
\pgfpathlineto{\pgfqpoint{3.814187in}{1.550480in}}%
\pgfpathlineto{\pgfqpoint{3.827673in}{1.551711in}}%
\pgfpathlineto{\pgfqpoint{3.841166in}{1.553103in}}%
\pgfpathlineto{\pgfqpoint{3.854669in}{1.554656in}}%
\pgfpathlineto{\pgfqpoint{3.862576in}{1.566518in}}%
\pgfpathlineto{\pgfqpoint{3.870477in}{1.578432in}}%
\pgfpathlineto{\pgfqpoint{3.878374in}{1.590393in}}%
\pgfpathlineto{\pgfqpoint{3.886266in}{1.602398in}}%
\pgfpathlineto{\pgfqpoint{3.872771in}{1.600401in}}%
\pgfpathlineto{\pgfqpoint{3.859284in}{1.598565in}}%
\pgfpathlineto{\pgfqpoint{3.845807in}{1.596890in}}%
\pgfpathlineto{\pgfqpoint{3.832338in}{1.595378in}}%
\pgfpathlineto{\pgfqpoint{3.824438in}{1.583806in}}%
\pgfpathlineto{\pgfqpoint{3.816534in}{1.572285in}}%
\pgfpathlineto{\pgfqpoint{3.808625in}{1.560819in}}%
\pgfpathlineto{\pgfqpoint{3.800710in}{1.549411in}}%
\pgfpathclose%
\pgfusepath{fill}%
\end{pgfscope}%
\begin{pgfscope}%
\pgfpathrectangle{\pgfqpoint{1.254980in}{0.150000in}}{\pgfqpoint{5.490039in}{5.490039in}}%
\pgfusepath{clip}%
\pgfsetbuttcap%
\pgfsetroundjoin%
\definecolor{currentfill}{rgb}{0.273809,0.031497,0.358853}%
\pgfsetfillcolor{currentfill}%
\pgfsetfillopacity{0.700000}%
\pgfsetlinewidth{0.000000pt}%
\definecolor{currentstroke}{rgb}{0.000000,0.000000,0.000000}%
\pgfsetstrokecolor{currentstroke}%
\pgfsetdash{}{0pt}%
\pgfpathmoveto{\pgfqpoint{3.629472in}{1.465835in}}%
\pgfpathlineto{\pgfqpoint{3.642913in}{1.464634in}}%
\pgfpathlineto{\pgfqpoint{3.656361in}{1.463596in}}%
\pgfpathlineto{\pgfqpoint{3.669816in}{1.462722in}}%
\pgfpathlineto{\pgfqpoint{3.683277in}{1.462011in}}%
\pgfpathlineto{\pgfqpoint{3.691249in}{1.472280in}}%
\pgfpathlineto{\pgfqpoint{3.699214in}{1.482651in}}%
\pgfpathlineto{\pgfqpoint{3.707173in}{1.493119in}}%
\pgfpathlineto{\pgfqpoint{3.715126in}{1.503680in}}%
\pgfpathlineto{\pgfqpoint{3.701677in}{1.503893in}}%
\pgfpathlineto{\pgfqpoint{3.688235in}{1.504269in}}%
\pgfpathlineto{\pgfqpoint{3.674800in}{1.504809in}}%
\pgfpathlineto{\pgfqpoint{3.661372in}{1.505512in}}%
\pgfpathlineto{\pgfqpoint{3.653406in}{1.495439in}}%
\pgfpathlineto{\pgfqpoint{3.645435in}{1.485465in}}%
\pgfpathlineto{\pgfqpoint{3.637457in}{1.475595in}}%
\pgfpathlineto{\pgfqpoint{3.629472in}{1.465835in}}%
\pgfpathclose%
\pgfusepath{fill}%
\end{pgfscope}%
\begin{pgfscope}%
\pgfpathrectangle{\pgfqpoint{1.254980in}{0.150000in}}{\pgfqpoint{5.490039in}{5.490039in}}%
\pgfusepath{clip}%
\pgfsetbuttcap%
\pgfsetroundjoin%
\definecolor{currentfill}{rgb}{0.283197,0.115680,0.436115}%
\pgfsetfillcolor{currentfill}%
\pgfsetfillopacity{0.700000}%
\pgfsetlinewidth{0.000000pt}%
\definecolor{currentstroke}{rgb}{0.000000,0.000000,0.000000}%
\pgfsetstrokecolor{currentstroke}%
\pgfsetdash{}{0pt}%
\pgfpathmoveto{\pgfqpoint{3.886266in}{1.602398in}}%
\pgfpathlineto{\pgfqpoint{3.899770in}{1.604556in}}%
\pgfpathlineto{\pgfqpoint{3.913283in}{1.606875in}}%
\pgfpathlineto{\pgfqpoint{3.926805in}{1.609354in}}%
\pgfpathlineto{\pgfqpoint{3.940337in}{1.611994in}}%
\pgfpathlineto{\pgfqpoint{3.948218in}{1.624466in}}%
\pgfpathlineto{\pgfqpoint{3.956094in}{1.636966in}}%
\pgfpathlineto{\pgfqpoint{3.963966in}{1.649490in}}%
\pgfpathlineto{\pgfqpoint{3.971832in}{1.662035in}}%
\pgfpathlineto{\pgfqpoint{3.958306in}{1.658978in}}%
\pgfpathlineto{\pgfqpoint{3.944789in}{1.656081in}}%
\pgfpathlineto{\pgfqpoint{3.931281in}{1.653345in}}%
\pgfpathlineto{\pgfqpoint{3.917783in}{1.650770in}}%
\pgfpathlineto{\pgfqpoint{3.909911in}{1.638632in}}%
\pgfpathlineto{\pgfqpoint{3.902034in}{1.626521in}}%
\pgfpathlineto{\pgfqpoint{3.894152in}{1.614442in}}%
\pgfpathlineto{\pgfqpoint{3.886266in}{1.602398in}}%
\pgfpathclose%
\pgfusepath{fill}%
\end{pgfscope}%
\begin{pgfscope}%
\pgfpathrectangle{\pgfqpoint{1.254980in}{0.150000in}}{\pgfqpoint{5.490039in}{5.490039in}}%
\pgfusepath{clip}%
\pgfsetbuttcap%
\pgfsetroundjoin%
\definecolor{currentfill}{rgb}{0.140210,0.665859,0.513427}%
\pgfsetfillcolor{currentfill}%
\pgfsetfillopacity{0.700000}%
\pgfsetlinewidth{0.000000pt}%
\definecolor{currentstroke}{rgb}{0.000000,0.000000,0.000000}%
\pgfsetstrokecolor{currentstroke}%
\pgfsetdash{}{0pt}%
\pgfpathmoveto{\pgfqpoint{5.218121in}{2.950054in}}%
\pgfpathlineto{\pgfqpoint{5.232342in}{2.963530in}}%
\pgfpathlineto{\pgfqpoint{5.246582in}{2.977169in}}%
\pgfpathlineto{\pgfqpoint{5.260842in}{2.990971in}}%
\pgfpathlineto{\pgfqpoint{5.275121in}{3.004937in}}%
\pgfpathlineto{\pgfqpoint{5.282482in}{3.011352in}}%
\pgfpathlineto{\pgfqpoint{5.289834in}{3.017618in}}%
\pgfpathlineto{\pgfqpoint{5.297177in}{3.023737in}}%
\pgfpathlineto{\pgfqpoint{5.304510in}{3.029712in}}%
\pgfpathlineto{\pgfqpoint{5.290240in}{3.015917in}}%
\pgfpathlineto{\pgfqpoint{5.275991in}{3.002285in}}%
\pgfpathlineto{\pgfqpoint{5.261761in}{2.988815in}}%
\pgfpathlineto{\pgfqpoint{5.247550in}{2.975508in}}%
\pgfpathlineto{\pgfqpoint{5.240206in}{2.969352in}}%
\pgfpathlineto{\pgfqpoint{5.232853in}{2.963059in}}%
\pgfpathlineto{\pgfqpoint{5.225492in}{2.956627in}}%
\pgfpathlineto{\pgfqpoint{5.218121in}{2.950054in}}%
\pgfpathclose%
\pgfusepath{fill}%
\end{pgfscope}%
\begin{pgfscope}%
\pgfpathrectangle{\pgfqpoint{1.254980in}{0.150000in}}{\pgfqpoint{5.490039in}{5.490039in}}%
\pgfusepath{clip}%
\pgfsetbuttcap%
\pgfsetroundjoin%
\definecolor{currentfill}{rgb}{0.269944,0.014625,0.341379}%
\pgfsetfillcolor{currentfill}%
\pgfsetfillopacity{0.700000}%
\pgfsetlinewidth{0.000000pt}%
\definecolor{currentstroke}{rgb}{0.000000,0.000000,0.000000}%
\pgfsetstrokecolor{currentstroke}%
\pgfsetdash{}{0pt}%
\pgfpathmoveto{\pgfqpoint{3.543699in}{1.436530in}}%
\pgfpathlineto{\pgfqpoint{3.557132in}{1.434145in}}%
\pgfpathlineto{\pgfqpoint{3.570571in}{1.431925in}}%
\pgfpathlineto{\pgfqpoint{3.584015in}{1.429870in}}%
\pgfpathlineto{\pgfqpoint{3.597466in}{1.427980in}}%
\pgfpathlineto{\pgfqpoint{3.605478in}{1.437255in}}%
\pgfpathlineto{\pgfqpoint{3.613483in}{1.446659in}}%
\pgfpathlineto{\pgfqpoint{3.621481in}{1.456188in}}%
\pgfpathlineto{\pgfqpoint{3.629472in}{1.465835in}}%
\pgfpathlineto{\pgfqpoint{3.616037in}{1.467200in}}%
\pgfpathlineto{\pgfqpoint{3.602608in}{1.468729in}}%
\pgfpathlineto{\pgfqpoint{3.589185in}{1.470424in}}%
\pgfpathlineto{\pgfqpoint{3.575767in}{1.472283in}}%
\pgfpathlineto{\pgfqpoint{3.567761in}{1.463151in}}%
\pgfpathlineto{\pgfqpoint{3.559748in}{1.454145in}}%
\pgfpathlineto{\pgfqpoint{3.551727in}{1.445269in}}%
\pgfpathlineto{\pgfqpoint{3.543699in}{1.436530in}}%
\pgfpathclose%
\pgfusepath{fill}%
\end{pgfscope}%
\begin{pgfscope}%
\pgfpathrectangle{\pgfqpoint{1.254980in}{0.150000in}}{\pgfqpoint{5.490039in}{5.490039in}}%
\pgfusepath{clip}%
\pgfsetbuttcap%
\pgfsetroundjoin%
\definecolor{currentfill}{rgb}{0.144759,0.519093,0.556572}%
\pgfsetfillcolor{currentfill}%
\pgfsetfillopacity{0.700000}%
\pgfsetlinewidth{0.000000pt}%
\definecolor{currentstroke}{rgb}{0.000000,0.000000,0.000000}%
\pgfsetstrokecolor{currentstroke}%
\pgfsetdash{}{0pt}%
\pgfpathmoveto{\pgfqpoint{4.813104in}{2.546545in}}%
\pgfpathlineto{\pgfqpoint{4.827064in}{2.557784in}}%
\pgfpathlineto{\pgfqpoint{4.841042in}{2.569185in}}%
\pgfpathlineto{\pgfqpoint{4.855036in}{2.580749in}}%
\pgfpathlineto{\pgfqpoint{4.869048in}{2.592474in}}%
\pgfpathlineto{\pgfqpoint{4.876630in}{2.602778in}}%
\pgfpathlineto{\pgfqpoint{4.884205in}{2.612936in}}%
\pgfpathlineto{\pgfqpoint{4.891773in}{2.622949in}}%
\pgfpathlineto{\pgfqpoint{4.899333in}{2.632818in}}%
\pgfpathlineto{\pgfqpoint{4.885323in}{2.621076in}}%
\pgfpathlineto{\pgfqpoint{4.871331in}{2.609496in}}%
\pgfpathlineto{\pgfqpoint{4.857356in}{2.598078in}}%
\pgfpathlineto{\pgfqpoint{4.843398in}{2.586821in}}%
\pgfpathlineto{\pgfqpoint{4.835835in}{2.576958in}}%
\pgfpathlineto{\pgfqpoint{4.828265in}{2.566957in}}%
\pgfpathlineto{\pgfqpoint{4.820688in}{2.556820in}}%
\pgfpathlineto{\pgfqpoint{4.813104in}{2.546545in}}%
\pgfpathclose%
\pgfusepath{fill}%
\end{pgfscope}%
\begin{pgfscope}%
\pgfpathrectangle{\pgfqpoint{1.254980in}{0.150000in}}{\pgfqpoint{5.490039in}{5.490039in}}%
\pgfusepath{clip}%
\pgfsetbuttcap%
\pgfsetroundjoin%
\definecolor{currentfill}{rgb}{0.182256,0.426184,0.557120}%
\pgfsetfillcolor{currentfill}%
\pgfsetfillopacity{0.700000}%
\pgfsetlinewidth{0.000000pt}%
\definecolor{currentstroke}{rgb}{0.000000,0.000000,0.000000}%
\pgfsetstrokecolor{currentstroke}%
\pgfsetdash{}{0pt}%
\pgfpathmoveto{\pgfqpoint{2.275795in}{2.391316in}}%
\pgfpathlineto{\pgfqpoint{2.289606in}{2.369350in}}%
\pgfpathlineto{\pgfqpoint{2.303404in}{2.347653in}}%
\pgfpathlineto{\pgfqpoint{2.317190in}{2.326223in}}%
\pgfpathlineto{\pgfqpoint{2.330964in}{2.305058in}}%
\pgfpathlineto{\pgfqpoint{2.339994in}{2.297492in}}%
\pgfpathlineto{\pgfqpoint{2.348999in}{2.290316in}}%
\pgfpathlineto{\pgfqpoint{2.357980in}{2.283521in}}%
\pgfpathlineto{\pgfqpoint{2.366936in}{2.277101in}}%
\pgfpathlineto{\pgfqpoint{2.353224in}{2.297572in}}%
\pgfpathlineto{\pgfqpoint{2.339501in}{2.318307in}}%
\pgfpathlineto{\pgfqpoint{2.325766in}{2.339307in}}%
\pgfpathlineto{\pgfqpoint{2.312020in}{2.360575in}}%
\pgfpathlineto{\pgfqpoint{2.303002in}{2.367678in}}%
\pgfpathlineto{\pgfqpoint{2.293959in}{2.375164in}}%
\pgfpathlineto{\pgfqpoint{2.284890in}{2.383041in}}%
\pgfpathlineto{\pgfqpoint{2.275795in}{2.391316in}}%
\pgfpathclose%
\pgfusepath{fill}%
\end{pgfscope}%
\begin{pgfscope}%
\pgfpathrectangle{\pgfqpoint{1.254980in}{0.150000in}}{\pgfqpoint{5.490039in}{5.490039in}}%
\pgfusepath{clip}%
\pgfsetbuttcap%
\pgfsetroundjoin%
\definecolor{currentfill}{rgb}{0.274149,0.751988,0.436601}%
\pgfsetfillcolor{currentfill}%
\pgfsetfillopacity{0.700000}%
\pgfsetlinewidth{0.000000pt}%
\definecolor{currentstroke}{rgb}{0.000000,0.000000,0.000000}%
\pgfsetstrokecolor{currentstroke}%
\pgfsetdash{}{0pt}%
\pgfpathmoveto{\pgfqpoint{5.506342in}{3.201690in}}%
\pgfpathlineto{\pgfqpoint{5.520754in}{3.216253in}}%
\pgfpathlineto{\pgfqpoint{5.535187in}{3.230979in}}%
\pgfpathlineto{\pgfqpoint{5.549641in}{3.245868in}}%
\pgfpathlineto{\pgfqpoint{5.564117in}{3.260922in}}%
\pgfpathlineto{\pgfqpoint{5.571281in}{3.264401in}}%
\pgfpathlineto{\pgfqpoint{5.578434in}{3.267752in}}%
\pgfpathlineto{\pgfqpoint{5.585578in}{3.270978in}}%
\pgfpathlineto{\pgfqpoint{5.592711in}{3.274082in}}%
\pgfpathlineto{\pgfqpoint{5.578253in}{3.259327in}}%
\pgfpathlineto{\pgfqpoint{5.563817in}{3.244735in}}%
\pgfpathlineto{\pgfqpoint{5.549401in}{3.230305in}}%
\pgfpathlineto{\pgfqpoint{5.535006in}{3.216038in}}%
\pgfpathlineto{\pgfqpoint{5.527855in}{3.212625in}}%
\pgfpathlineto{\pgfqpoint{5.520694in}{3.209099in}}%
\pgfpathlineto{\pgfqpoint{5.513523in}{3.205455in}}%
\pgfpathlineto{\pgfqpoint{5.506342in}{3.201690in}}%
\pgfpathclose%
\pgfusepath{fill}%
\end{pgfscope}%
\begin{pgfscope}%
\pgfpathrectangle{\pgfqpoint{1.254980in}{0.150000in}}{\pgfqpoint{5.490039in}{5.490039in}}%
\pgfusepath{clip}%
\pgfsetbuttcap%
\pgfsetroundjoin%
\definecolor{currentfill}{rgb}{0.199430,0.387607,0.554642}%
\pgfsetfillcolor{currentfill}%
\pgfsetfillopacity{0.700000}%
\pgfsetlinewidth{0.000000pt}%
\definecolor{currentstroke}{rgb}{0.000000,0.000000,0.000000}%
\pgfsetstrokecolor{currentstroke}%
\pgfsetdash{}{0pt}%
\pgfpathmoveto{\pgfqpoint{4.493770in}{2.196692in}}%
\pgfpathlineto{\pgfqpoint{4.507544in}{2.205462in}}%
\pgfpathlineto{\pgfqpoint{4.521332in}{2.214392in}}%
\pgfpathlineto{\pgfqpoint{4.535135in}{2.223484in}}%
\pgfpathlineto{\pgfqpoint{4.548953in}{2.232737in}}%
\pgfpathlineto{\pgfqpoint{4.556661in}{2.245329in}}%
\pgfpathlineto{\pgfqpoint{4.564363in}{2.257810in}}%
\pgfpathlineto{\pgfqpoint{4.572060in}{2.270180in}}%
\pgfpathlineto{\pgfqpoint{4.579751in}{2.282437in}}%
\pgfpathlineto{\pgfqpoint{4.565933in}{2.273018in}}%
\pgfpathlineto{\pgfqpoint{4.552129in}{2.263761in}}%
\pgfpathlineto{\pgfqpoint{4.538341in}{2.254664in}}%
\pgfpathlineto{\pgfqpoint{4.524567in}{2.245729in}}%
\pgfpathlineto{\pgfqpoint{4.516876in}{2.233627in}}%
\pgfpathlineto{\pgfqpoint{4.509179in}{2.221419in}}%
\pgfpathlineto{\pgfqpoint{4.501477in}{2.209107in}}%
\pgfpathlineto{\pgfqpoint{4.493770in}{2.196692in}}%
\pgfpathclose%
\pgfusepath{fill}%
\end{pgfscope}%
\begin{pgfscope}%
\pgfpathrectangle{\pgfqpoint{1.254980in}{0.150000in}}{\pgfqpoint{5.490039in}{5.490039in}}%
\pgfusepath{clip}%
\pgfsetbuttcap%
\pgfsetroundjoin%
\definecolor{currentfill}{rgb}{0.281924,0.089666,0.412415}%
\pgfsetfillcolor{currentfill}%
\pgfsetfillopacity{0.700000}%
\pgfsetlinewidth{0.000000pt}%
\definecolor{currentstroke}{rgb}{0.000000,0.000000,0.000000}%
\pgfsetstrokecolor{currentstroke}%
\pgfsetdash{}{0pt}%
\pgfpathmoveto{\pgfqpoint{2.928864in}{1.606346in}}%
\pgfpathlineto{\pgfqpoint{2.942343in}{1.595208in}}%
\pgfpathlineto{\pgfqpoint{2.955820in}{1.584261in}}%
\pgfpathlineto{\pgfqpoint{2.969296in}{1.573505in}}%
\pgfpathlineto{\pgfqpoint{2.982770in}{1.562938in}}%
\pgfpathlineto{\pgfqpoint{2.991190in}{1.563514in}}%
\pgfpathlineto{\pgfqpoint{2.999596in}{1.564380in}}%
\pgfpathlineto{\pgfqpoint{3.007986in}{1.565528in}}%
\pgfpathlineto{\pgfqpoint{3.016362in}{1.566951in}}%
\pgfpathlineto{\pgfqpoint{3.002927in}{1.576871in}}%
\pgfpathlineto{\pgfqpoint{2.989490in}{1.586979in}}%
\pgfpathlineto{\pgfqpoint{2.976052in}{1.597277in}}%
\pgfpathlineto{\pgfqpoint{2.962612in}{1.607766in}}%
\pgfpathlineto{\pgfqpoint{2.954198in}{1.606980in}}%
\pgfpathlineto{\pgfqpoint{2.945769in}{1.606476in}}%
\pgfpathlineto{\pgfqpoint{2.937324in}{1.606263in}}%
\pgfpathlineto{\pgfqpoint{2.928864in}{1.606346in}}%
\pgfpathclose%
\pgfusepath{fill}%
\end{pgfscope}%
\begin{pgfscope}%
\pgfpathrectangle{\pgfqpoint{1.254980in}{0.150000in}}{\pgfqpoint{5.490039in}{5.490039in}}%
\pgfusepath{clip}%
\pgfsetbuttcap%
\pgfsetroundjoin%
\definecolor{currentfill}{rgb}{0.120565,0.596422,0.543611}%
\pgfsetfillcolor{currentfill}%
\pgfsetfillopacity{0.700000}%
\pgfsetlinewidth{0.000000pt}%
\definecolor{currentstroke}{rgb}{0.000000,0.000000,0.000000}%
\pgfsetstrokecolor{currentstroke}%
\pgfsetdash{}{0pt}%
\pgfpathmoveto{\pgfqpoint{5.015730in}{2.754988in}}%
\pgfpathlineto{\pgfqpoint{5.029821in}{2.767483in}}%
\pgfpathlineto{\pgfqpoint{5.043930in}{2.780141in}}%
\pgfpathlineto{\pgfqpoint{5.058058in}{2.792962in}}%
\pgfpathlineto{\pgfqpoint{5.072204in}{2.805945in}}%
\pgfpathlineto{\pgfqpoint{5.079686in}{2.814412in}}%
\pgfpathlineto{\pgfqpoint{5.087160in}{2.822727in}}%
\pgfpathlineto{\pgfqpoint{5.094625in}{2.830890in}}%
\pgfpathlineto{\pgfqpoint{5.102082in}{2.838903in}}%
\pgfpathlineto{\pgfqpoint{5.087941in}{2.825996in}}%
\pgfpathlineto{\pgfqpoint{5.073819in}{2.813251in}}%
\pgfpathlineto{\pgfqpoint{5.059715in}{2.800668in}}%
\pgfpathlineto{\pgfqpoint{5.045630in}{2.788248in}}%
\pgfpathlineto{\pgfqpoint{5.038167in}{2.780148in}}%
\pgfpathlineto{\pgfqpoint{5.030696in}{2.771906in}}%
\pgfpathlineto{\pgfqpoint{5.023217in}{2.763519in}}%
\pgfpathlineto{\pgfqpoint{5.015730in}{2.754988in}}%
\pgfpathclose%
\pgfusepath{fill}%
\end{pgfscope}%
\begin{pgfscope}%
\pgfpathrectangle{\pgfqpoint{1.254980in}{0.150000in}}{\pgfqpoint{5.490039in}{5.490039in}}%
\pgfusepath{clip}%
\pgfsetbuttcap%
\pgfsetroundjoin%
\definecolor{currentfill}{rgb}{0.267004,0.004874,0.329415}%
\pgfsetfillcolor{currentfill}%
\pgfsetfillopacity{0.700000}%
\pgfsetlinewidth{0.000000pt}%
\definecolor{currentstroke}{rgb}{0.000000,0.000000,0.000000}%
\pgfsetstrokecolor{currentstroke}%
\pgfsetdash{}{0pt}%
\pgfpathmoveto{\pgfqpoint{3.317861in}{1.427440in}}%
\pgfpathlineto{\pgfqpoint{3.331287in}{1.421900in}}%
\pgfpathlineto{\pgfqpoint{3.344715in}{1.416532in}}%
\pgfpathlineto{\pgfqpoint{3.358147in}{1.411334in}}%
\pgfpathlineto{\pgfqpoint{3.371581in}{1.406307in}}%
\pgfpathlineto{\pgfqpoint{3.379717in}{1.412580in}}%
\pgfpathlineto{\pgfqpoint{3.387844in}{1.419047in}}%
\pgfpathlineto{\pgfqpoint{3.395961in}{1.425703in}}%
\pgfpathlineto{\pgfqpoint{3.404069in}{1.432542in}}%
\pgfpathlineto{\pgfqpoint{3.390658in}{1.436987in}}%
\pgfpathlineto{\pgfqpoint{3.377250in}{1.441603in}}%
\pgfpathlineto{\pgfqpoint{3.363845in}{1.446390in}}%
\pgfpathlineto{\pgfqpoint{3.350444in}{1.451347in}}%
\pgfpathlineto{\pgfqpoint{3.342313in}{1.445080in}}%
\pgfpathlineto{\pgfqpoint{3.334173in}{1.439003in}}%
\pgfpathlineto{\pgfqpoint{3.326022in}{1.433120in}}%
\pgfpathlineto{\pgfqpoint{3.317861in}{1.427440in}}%
\pgfpathclose%
\pgfusepath{fill}%
\end{pgfscope}%
\begin{pgfscope}%
\pgfpathrectangle{\pgfqpoint{1.254980in}{0.150000in}}{\pgfqpoint{5.490039in}{5.490039in}}%
\pgfusepath{clip}%
\pgfsetbuttcap%
\pgfsetroundjoin%
\definecolor{currentfill}{rgb}{0.262138,0.242286,0.520837}%
\pgfsetfillcolor{currentfill}%
\pgfsetfillopacity{0.700000}%
\pgfsetlinewidth{0.000000pt}%
\definecolor{currentstroke}{rgb}{0.000000,0.000000,0.000000}%
\pgfsetstrokecolor{currentstroke}%
\pgfsetdash{}{0pt}%
\pgfpathmoveto{\pgfqpoint{4.174386in}{1.851800in}}%
\pgfpathlineto{\pgfqpoint{4.188002in}{1.857388in}}%
\pgfpathlineto{\pgfqpoint{4.201629in}{1.863136in}}%
\pgfpathlineto{\pgfqpoint{4.215268in}{1.869043in}}%
\pgfpathlineto{\pgfqpoint{4.228919in}{1.875111in}}%
\pgfpathlineto{\pgfqpoint{4.236721in}{1.888558in}}%
\pgfpathlineto{\pgfqpoint{4.244519in}{1.901959in}}%
\pgfpathlineto{\pgfqpoint{4.252312in}{1.915309in}}%
\pgfpathlineto{\pgfqpoint{4.260101in}{1.928608in}}%
\pgfpathlineto{\pgfqpoint{4.246450in}{1.922232in}}%
\pgfpathlineto{\pgfqpoint{4.232812in}{1.916016in}}%
\pgfpathlineto{\pgfqpoint{4.219185in}{1.909960in}}%
\pgfpathlineto{\pgfqpoint{4.205571in}{1.904065in}}%
\pgfpathlineto{\pgfqpoint{4.197782in}{1.891064in}}%
\pgfpathlineto{\pgfqpoint{4.189988in}{1.878018in}}%
\pgfpathlineto{\pgfqpoint{4.182189in}{1.864929in}}%
\pgfpathlineto{\pgfqpoint{4.174386in}{1.851800in}}%
\pgfpathclose%
\pgfusepath{fill}%
\end{pgfscope}%
\begin{pgfscope}%
\pgfpathrectangle{\pgfqpoint{1.254980in}{0.150000in}}{\pgfqpoint{5.490039in}{5.490039in}}%
\pgfusepath{clip}%
\pgfsetbuttcap%
\pgfsetroundjoin%
\definecolor{currentfill}{rgb}{0.281887,0.150881,0.465405}%
\pgfsetfillcolor{currentfill}%
\pgfsetfillopacity{0.700000}%
\pgfsetlinewidth{0.000000pt}%
\definecolor{currentstroke}{rgb}{0.000000,0.000000,0.000000}%
\pgfsetstrokecolor{currentstroke}%
\pgfsetdash{}{0pt}%
\pgfpathmoveto{\pgfqpoint{3.971832in}{1.662035in}}%
\pgfpathlineto{\pgfqpoint{3.985369in}{1.665253in}}%
\pgfpathlineto{\pgfqpoint{3.998915in}{1.668630in}}%
\pgfpathlineto{\pgfqpoint{4.012472in}{1.672168in}}%
\pgfpathlineto{\pgfqpoint{4.026039in}{1.675866in}}%
\pgfpathlineto{\pgfqpoint{4.033897in}{1.688829in}}%
\pgfpathlineto{\pgfqpoint{4.041750in}{1.701797in}}%
\pgfpathlineto{\pgfqpoint{4.049599in}{1.714768in}}%
\pgfpathlineto{\pgfqpoint{4.057444in}{1.727738in}}%
\pgfpathlineto{\pgfqpoint{4.043881in}{1.723649in}}%
\pgfpathlineto{\pgfqpoint{4.030328in}{1.719721in}}%
\pgfpathlineto{\pgfqpoint{4.016785in}{1.715953in}}%
\pgfpathlineto{\pgfqpoint{4.003253in}{1.712345in}}%
\pgfpathlineto{\pgfqpoint{3.995405in}{1.699755in}}%
\pgfpathlineto{\pgfqpoint{3.987552in}{1.687171in}}%
\pgfpathlineto{\pgfqpoint{3.979694in}{1.674596in}}%
\pgfpathlineto{\pgfqpoint{3.971832in}{1.662035in}}%
\pgfpathclose%
\pgfusepath{fill}%
\end{pgfscope}%
\begin{pgfscope}%
\pgfpathrectangle{\pgfqpoint{1.254980in}{0.150000in}}{\pgfqpoint{5.490039in}{5.490039in}}%
\pgfusepath{clip}%
\pgfsetbuttcap%
\pgfsetroundjoin%
\definecolor{currentfill}{rgb}{0.271305,0.019942,0.347269}%
\pgfsetfillcolor{currentfill}%
\pgfsetfillopacity{0.700000}%
\pgfsetlinewidth{0.000000pt}%
\definecolor{currentstroke}{rgb}{0.000000,0.000000,0.000000}%
\pgfsetstrokecolor{currentstroke}%
\pgfsetdash{}{0pt}%
\pgfpathmoveto{\pgfqpoint{3.177572in}{1.462308in}}%
\pgfpathlineto{\pgfqpoint{3.191009in}{1.454762in}}%
\pgfpathlineto{\pgfqpoint{3.204447in}{1.447392in}}%
\pgfpathlineto{\pgfqpoint{3.217888in}{1.440198in}}%
\pgfpathlineto{\pgfqpoint{3.231329in}{1.433180in}}%
\pgfpathlineto{\pgfqpoint{3.239560in}{1.437374in}}%
\pgfpathlineto{\pgfqpoint{3.247780in}{1.441800in}}%
\pgfpathlineto{\pgfqpoint{3.255989in}{1.446452in}}%
\pgfpathlineto{\pgfqpoint{3.264186in}{1.451325in}}%
\pgfpathlineto{\pgfqpoint{3.250773in}{1.457731in}}%
\pgfpathlineto{\pgfqpoint{3.237362in}{1.464313in}}%
\pgfpathlineto{\pgfqpoint{3.223953in}{1.471070in}}%
\pgfpathlineto{\pgfqpoint{3.210545in}{1.478004in}}%
\pgfpathlineto{\pgfqpoint{3.202319in}{1.473732in}}%
\pgfpathlineto{\pgfqpoint{3.194082in}{1.469688in}}%
\pgfpathlineto{\pgfqpoint{3.185833in}{1.465878in}}%
\pgfpathlineto{\pgfqpoint{3.177572in}{1.462308in}}%
\pgfpathclose%
\pgfusepath{fill}%
\end{pgfscope}%
\begin{pgfscope}%
\pgfpathrectangle{\pgfqpoint{1.254980in}{0.150000in}}{\pgfqpoint{5.490039in}{5.490039in}}%
\pgfusepath{clip}%
\pgfsetbuttcap%
\pgfsetroundjoin%
\definecolor{currentfill}{rgb}{0.121831,0.589055,0.545623}%
\pgfsetfillcolor{currentfill}%
\pgfsetfillopacity{0.700000}%
\pgfsetlinewidth{0.000000pt}%
\definecolor{currentstroke}{rgb}{0.000000,0.000000,0.000000}%
\pgfsetstrokecolor{currentstroke}%
\pgfsetdash{}{0pt}%
\pgfpathmoveto{\pgfqpoint{2.034086in}{2.846212in}}%
\pgfpathlineto{\pgfqpoint{2.048139in}{2.819020in}}%
\pgfpathlineto{\pgfqpoint{2.062174in}{2.792154in}}%
\pgfpathlineto{\pgfqpoint{2.076190in}{2.765609in}}%
\pgfpathlineto{\pgfqpoint{2.090189in}{2.739383in}}%
\pgfpathlineto{\pgfqpoint{2.099449in}{2.729896in}}%
\pgfpathlineto{\pgfqpoint{2.108680in}{2.720816in}}%
\pgfpathlineto{\pgfqpoint{2.117883in}{2.712135in}}%
\pgfpathlineto{\pgfqpoint{2.127060in}{2.703846in}}%
\pgfpathlineto{\pgfqpoint{2.113131in}{2.729381in}}%
\pgfpathlineto{\pgfqpoint{2.099186in}{2.755232in}}%
\pgfpathlineto{\pgfqpoint{2.085223in}{2.781403in}}%
\pgfpathlineto{\pgfqpoint{2.071242in}{2.807896in}}%
\pgfpathlineto{\pgfqpoint{2.061996in}{2.816866in}}%
\pgfpathlineto{\pgfqpoint{2.052722in}{2.826237in}}%
\pgfpathlineto{\pgfqpoint{2.043419in}{2.836016in}}%
\pgfpathlineto{\pgfqpoint{2.034086in}{2.846212in}}%
\pgfpathclose%
\pgfusepath{fill}%
\end{pgfscope}%
\begin{pgfscope}%
\pgfpathrectangle{\pgfqpoint{1.254980in}{0.150000in}}{\pgfqpoint{5.490039in}{5.490039in}}%
\pgfusepath{clip}%
\pgfsetbuttcap%
\pgfsetroundjoin%
\definecolor{currentfill}{rgb}{0.267004,0.004874,0.329415}%
\pgfsetfillcolor{currentfill}%
\pgfsetfillopacity{0.700000}%
\pgfsetlinewidth{0.000000pt}%
\definecolor{currentstroke}{rgb}{0.000000,0.000000,0.000000}%
\pgfsetstrokecolor{currentstroke}%
\pgfsetdash{}{0pt}%
\pgfpathmoveto{\pgfqpoint{3.457755in}{1.416449in}}%
\pgfpathlineto{\pgfqpoint{3.471187in}{1.412846in}}%
\pgfpathlineto{\pgfqpoint{3.484623in}{1.409410in}}%
\pgfpathlineto{\pgfqpoint{3.498064in}{1.406140in}}%
\pgfpathlineto{\pgfqpoint{3.511510in}{1.403036in}}%
\pgfpathlineto{\pgfqpoint{3.519569in}{1.411179in}}%
\pgfpathlineto{\pgfqpoint{3.527620in}{1.419480in}}%
\pgfpathlineto{\pgfqpoint{3.535664in}{1.427931in}}%
\pgfpathlineto{\pgfqpoint{3.543699in}{1.436530in}}%
\pgfpathlineto{\pgfqpoint{3.530271in}{1.439080in}}%
\pgfpathlineto{\pgfqpoint{3.516848in}{1.441797in}}%
\pgfpathlineto{\pgfqpoint{3.503431in}{1.444680in}}%
\pgfpathlineto{\pgfqpoint{3.490018in}{1.447730in}}%
\pgfpathlineto{\pgfqpoint{3.481964in}{1.439674in}}%
\pgfpathlineto{\pgfqpoint{3.473903in}{1.431772in}}%
\pgfpathlineto{\pgfqpoint{3.465833in}{1.424028in}}%
\pgfpathlineto{\pgfqpoint{3.457755in}{1.416449in}}%
\pgfpathclose%
\pgfusepath{fill}%
\end{pgfscope}%
\begin{pgfscope}%
\pgfpathrectangle{\pgfqpoint{1.254980in}{0.150000in}}{\pgfqpoint{5.490039in}{5.490039in}}%
\pgfusepath{clip}%
\pgfsetbuttcap%
\pgfsetroundjoin%
\definecolor{currentfill}{rgb}{0.221989,0.339161,0.548752}%
\pgfsetfillcolor{currentfill}%
\pgfsetfillopacity{0.700000}%
\pgfsetlinewidth{0.000000pt}%
\definecolor{currentstroke}{rgb}{0.000000,0.000000,0.000000}%
\pgfsetstrokecolor{currentstroke}%
\pgfsetdash{}{0pt}%
\pgfpathmoveto{\pgfqpoint{4.376979in}{2.062037in}}%
\pgfpathlineto{\pgfqpoint{4.390695in}{2.069746in}}%
\pgfpathlineto{\pgfqpoint{4.404425in}{2.077616in}}%
\pgfpathlineto{\pgfqpoint{4.418169in}{2.085646in}}%
\pgfpathlineto{\pgfqpoint{4.431927in}{2.093836in}}%
\pgfpathlineto{\pgfqpoint{4.439675in}{2.107021in}}%
\pgfpathlineto{\pgfqpoint{4.447418in}{2.120115in}}%
\pgfpathlineto{\pgfqpoint{4.455156in}{2.133118in}}%
\pgfpathlineto{\pgfqpoint{4.462889in}{2.146027in}}%
\pgfpathlineto{\pgfqpoint{4.449130in}{2.137613in}}%
\pgfpathlineto{\pgfqpoint{4.435386in}{2.129359in}}%
\pgfpathlineto{\pgfqpoint{4.421655in}{2.121266in}}%
\pgfpathlineto{\pgfqpoint{4.407939in}{2.113333in}}%
\pgfpathlineto{\pgfqpoint{4.400206in}{2.100637in}}%
\pgfpathlineto{\pgfqpoint{4.392468in}{2.087854in}}%
\pgfpathlineto{\pgfqpoint{4.384726in}{2.074987in}}%
\pgfpathlineto{\pgfqpoint{4.376979in}{2.062037in}}%
\pgfpathclose%
\pgfusepath{fill}%
\end{pgfscope}%
\begin{pgfscope}%
\pgfpathrectangle{\pgfqpoint{1.254980in}{0.150000in}}{\pgfqpoint{5.490039in}{5.490039in}}%
\pgfusepath{clip}%
\pgfsetbuttcap%
\pgfsetroundjoin%
\definecolor{currentfill}{rgb}{0.162142,0.474838,0.558140}%
\pgfsetfillcolor{currentfill}%
\pgfsetfillopacity{0.700000}%
\pgfsetlinewidth{0.000000pt}%
\definecolor{currentstroke}{rgb}{0.000000,0.000000,0.000000}%
\pgfsetstrokecolor{currentstroke}%
\pgfsetdash{}{0pt}%
\pgfpathmoveto{\pgfqpoint{4.696504in}{2.416556in}}%
\pgfpathlineto{\pgfqpoint{4.710400in}{2.427024in}}%
\pgfpathlineto{\pgfqpoint{4.724313in}{2.437654in}}%
\pgfpathlineto{\pgfqpoint{4.738242in}{2.448446in}}%
\pgfpathlineto{\pgfqpoint{4.752188in}{2.459399in}}%
\pgfpathlineto{\pgfqpoint{4.759826in}{2.470771in}}%
\pgfpathlineto{\pgfqpoint{4.767457in}{2.482007in}}%
\pgfpathlineto{\pgfqpoint{4.775082in}{2.493107in}}%
\pgfpathlineto{\pgfqpoint{4.782700in}{2.504069in}}%
\pgfpathlineto{\pgfqpoint{4.768754in}{2.493039in}}%
\pgfpathlineto{\pgfqpoint{4.754826in}{2.482170in}}%
\pgfpathlineto{\pgfqpoint{4.740914in}{2.471463in}}%
\pgfpathlineto{\pgfqpoint{4.727018in}{2.460917in}}%
\pgfpathlineto{\pgfqpoint{4.719399in}{2.450021in}}%
\pgfpathlineto{\pgfqpoint{4.711773in}{2.438995in}}%
\pgfpathlineto{\pgfqpoint{4.704142in}{2.427840in}}%
\pgfpathlineto{\pgfqpoint{4.696504in}{2.416556in}}%
\pgfpathclose%
\pgfusepath{fill}%
\end{pgfscope}%
\begin{pgfscope}%
\pgfpathrectangle{\pgfqpoint{1.254980in}{0.150000in}}{\pgfqpoint{5.490039in}{5.490039in}}%
\pgfusepath{clip}%
\pgfsetbuttcap%
\pgfsetroundjoin%
\definecolor{currentfill}{rgb}{0.168126,0.459988,0.558082}%
\pgfsetfillcolor{currentfill}%
\pgfsetfillopacity{0.700000}%
\pgfsetlinewidth{0.000000pt}%
\definecolor{currentstroke}{rgb}{0.000000,0.000000,0.000000}%
\pgfsetstrokecolor{currentstroke}%
\pgfsetdash{}{0pt}%
\pgfpathmoveto{\pgfqpoint{2.220420in}{2.481924in}}%
\pgfpathlineto{\pgfqpoint{2.234284in}{2.458856in}}%
\pgfpathlineto{\pgfqpoint{2.248134in}{2.436067in}}%
\pgfpathlineto{\pgfqpoint{2.261971in}{2.413554in}}%
\pgfpathlineto{\pgfqpoint{2.275795in}{2.391316in}}%
\pgfpathlineto{\pgfqpoint{2.284890in}{2.383041in}}%
\pgfpathlineto{\pgfqpoint{2.293959in}{2.375164in}}%
\pgfpathlineto{\pgfqpoint{2.303002in}{2.367678in}}%
\pgfpathlineto{\pgfqpoint{2.312020in}{2.360575in}}%
\pgfpathlineto{\pgfqpoint{2.298261in}{2.382114in}}%
\pgfpathlineto{\pgfqpoint{2.284489in}{2.403924in}}%
\pgfpathlineto{\pgfqpoint{2.270704in}{2.426010in}}%
\pgfpathlineto{\pgfqpoint{2.256907in}{2.448374in}}%
\pgfpathlineto{\pgfqpoint{2.247825in}{2.456165in}}%
\pgfpathlineto{\pgfqpoint{2.238717in}{2.464349in}}%
\pgfpathlineto{\pgfqpoint{2.229582in}{2.472933in}}%
\pgfpathlineto{\pgfqpoint{2.220420in}{2.481924in}}%
\pgfpathclose%
\pgfusepath{fill}%
\end{pgfscope}%
\begin{pgfscope}%
\pgfpathrectangle{\pgfqpoint{1.254980in}{0.150000in}}{\pgfqpoint{5.490039in}{5.490039in}}%
\pgfusepath{clip}%
\pgfsetbuttcap%
\pgfsetroundjoin%
\definecolor{currentfill}{rgb}{0.335885,0.777018,0.402049}%
\pgfsetfillcolor{currentfill}%
\pgfsetfillopacity{0.700000}%
\pgfsetlinewidth{0.000000pt}%
\definecolor{currentstroke}{rgb}{0.000000,0.000000,0.000000}%
\pgfsetstrokecolor{currentstroke}%
\pgfsetdash{}{0pt}%
\pgfpathmoveto{\pgfqpoint{5.592711in}{3.274082in}}%
\pgfpathlineto{\pgfqpoint{5.607190in}{3.289001in}}%
\pgfpathlineto{\pgfqpoint{5.621691in}{3.304083in}}%
\pgfpathlineto{\pgfqpoint{5.636213in}{3.319328in}}%
\pgfpathlineto{\pgfqpoint{5.650757in}{3.334737in}}%
\pgfpathlineto{\pgfqpoint{5.657861in}{3.337405in}}%
\pgfpathlineto{\pgfqpoint{5.664955in}{3.339951in}}%
\pgfpathlineto{\pgfqpoint{5.672038in}{3.342378in}}%
\pgfpathlineto{\pgfqpoint{5.679111in}{3.344691in}}%
\pgfpathlineto{\pgfqpoint{5.664587in}{3.329613in}}%
\pgfpathlineto{\pgfqpoint{5.650085in}{3.314697in}}%
\pgfpathlineto{\pgfqpoint{5.635604in}{3.299945in}}%
\pgfpathlineto{\pgfqpoint{5.621144in}{3.285355in}}%
\pgfpathlineto{\pgfqpoint{5.614051in}{3.282701in}}%
\pgfpathlineto{\pgfqpoint{5.606947in}{3.279940in}}%
\pgfpathlineto{\pgfqpoint{5.599834in}{3.277069in}}%
\pgfpathlineto{\pgfqpoint{5.592711in}{3.274082in}}%
\pgfpathclose%
\pgfusepath{fill}%
\end{pgfscope}%
\begin{pgfscope}%
\pgfpathrectangle{\pgfqpoint{1.254980in}{0.150000in}}{\pgfqpoint{5.490039in}{5.490039in}}%
\pgfusepath{clip}%
\pgfsetbuttcap%
\pgfsetroundjoin%
\definecolor{currentfill}{rgb}{0.280267,0.073417,0.397163}%
\pgfsetfillcolor{currentfill}%
\pgfsetfillopacity{0.700000}%
\pgfsetlinewidth{0.000000pt}%
\definecolor{currentstroke}{rgb}{0.000000,0.000000,0.000000}%
\pgfsetstrokecolor{currentstroke}%
\pgfsetdash{}{0pt}%
\pgfpathmoveto{\pgfqpoint{2.982770in}{1.562938in}}%
\pgfpathlineto{\pgfqpoint{2.996242in}{1.552559in}}%
\pgfpathlineto{\pgfqpoint{3.009714in}{1.542367in}}%
\pgfpathlineto{\pgfqpoint{3.023184in}{1.532362in}}%
\pgfpathlineto{\pgfqpoint{3.036654in}{1.522542in}}%
\pgfpathlineto{\pgfqpoint{3.045036in}{1.523777in}}%
\pgfpathlineto{\pgfqpoint{3.053404in}{1.525293in}}%
\pgfpathlineto{\pgfqpoint{3.061759in}{1.527083in}}%
\pgfpathlineto{\pgfqpoint{3.070099in}{1.529142in}}%
\pgfpathlineto{\pgfqpoint{3.056665in}{1.538316in}}%
\pgfpathlineto{\pgfqpoint{3.043232in}{1.547675in}}%
\pgfpathlineto{\pgfqpoint{3.029797in}{1.557220in}}%
\pgfpathlineto{\pgfqpoint{3.016362in}{1.566951in}}%
\pgfpathlineto{\pgfqpoint{3.007986in}{1.565528in}}%
\pgfpathlineto{\pgfqpoint{2.999596in}{1.564380in}}%
\pgfpathlineto{\pgfqpoint{2.991190in}{1.563514in}}%
\pgfpathlineto{\pgfqpoint{2.982770in}{1.562938in}}%
\pgfpathclose%
\pgfusepath{fill}%
\end{pgfscope}%
\begin{pgfscope}%
\pgfpathrectangle{\pgfqpoint{1.254980in}{0.150000in}}{\pgfqpoint{5.490039in}{5.490039in}}%
\pgfusepath{clip}%
\pgfsetbuttcap%
\pgfsetroundjoin%
\definecolor{currentfill}{rgb}{0.175707,0.697900,0.491033}%
\pgfsetfillcolor{currentfill}%
\pgfsetfillopacity{0.700000}%
\pgfsetlinewidth{0.000000pt}%
\definecolor{currentstroke}{rgb}{0.000000,0.000000,0.000000}%
\pgfsetstrokecolor{currentstroke}%
\pgfsetdash{}{0pt}%
\pgfpathmoveto{\pgfqpoint{5.304510in}{3.029712in}}%
\pgfpathlineto{\pgfqpoint{5.318799in}{3.043670in}}%
\pgfpathlineto{\pgfqpoint{5.333109in}{3.057792in}}%
\pgfpathlineto{\pgfqpoint{5.347438in}{3.072077in}}%
\pgfpathlineto{\pgfqpoint{5.361788in}{3.086525in}}%
\pgfpathlineto{\pgfqpoint{5.369101in}{3.092168in}}%
\pgfpathlineto{\pgfqpoint{5.376404in}{3.097662in}}%
\pgfpathlineto{\pgfqpoint{5.383697in}{3.103012in}}%
\pgfpathlineto{\pgfqpoint{5.390980in}{3.108219in}}%
\pgfpathlineto{\pgfqpoint{5.376642in}{3.093973in}}%
\pgfpathlineto{\pgfqpoint{5.362324in}{3.079890in}}%
\pgfpathlineto{\pgfqpoint{5.348027in}{3.065970in}}%
\pgfpathlineto{\pgfqpoint{5.333749in}{3.052213in}}%
\pgfpathlineto{\pgfqpoint{5.326453in}{3.046793in}}%
\pgfpathlineto{\pgfqpoint{5.319148in}{3.041238in}}%
\pgfpathlineto{\pgfqpoint{5.311834in}{3.035545in}}%
\pgfpathlineto{\pgfqpoint{5.304510in}{3.029712in}}%
\pgfpathclose%
\pgfusepath{fill}%
\end{pgfscope}%
\begin{pgfscope}%
\pgfpathrectangle{\pgfqpoint{1.254980in}{0.150000in}}{\pgfqpoint{5.490039in}{5.490039in}}%
\pgfusepath{clip}%
\pgfsetbuttcap%
\pgfsetroundjoin%
\definecolor{currentfill}{rgb}{0.277134,0.185228,0.489898}%
\pgfsetfillcolor{currentfill}%
\pgfsetfillopacity{0.700000}%
\pgfsetlinewidth{0.000000pt}%
\definecolor{currentstroke}{rgb}{0.000000,0.000000,0.000000}%
\pgfsetstrokecolor{currentstroke}%
\pgfsetdash{}{0pt}%
\pgfpathmoveto{\pgfqpoint{4.057444in}{1.727738in}}%
\pgfpathlineto{\pgfqpoint{4.071018in}{1.731987in}}%
\pgfpathlineto{\pgfqpoint{4.084603in}{1.736395in}}%
\pgfpathlineto{\pgfqpoint{4.098199in}{1.740963in}}%
\pgfpathlineto{\pgfqpoint{4.111806in}{1.745691in}}%
\pgfpathlineto{\pgfqpoint{4.119644in}{1.759030in}}%
\pgfpathlineto{\pgfqpoint{4.127477in}{1.772355in}}%
\pgfpathlineto{\pgfqpoint{4.135306in}{1.785660in}}%
\pgfpathlineto{\pgfqpoint{4.143131in}{1.798945in}}%
\pgfpathlineto{\pgfqpoint{4.129526in}{1.793853in}}%
\pgfpathlineto{\pgfqpoint{4.115932in}{1.788921in}}%
\pgfpathlineto{\pgfqpoint{4.102350in}{1.784149in}}%
\pgfpathlineto{\pgfqpoint{4.088778in}{1.779537in}}%
\pgfpathlineto{\pgfqpoint{4.080951in}{1.766606in}}%
\pgfpathlineto{\pgfqpoint{4.073120in}{1.753660in}}%
\pgfpathlineto{\pgfqpoint{4.065284in}{1.740703in}}%
\pgfpathlineto{\pgfqpoint{4.057444in}{1.727738in}}%
\pgfpathclose%
\pgfusepath{fill}%
\end{pgfscope}%
\begin{pgfscope}%
\pgfpathrectangle{\pgfqpoint{1.254980in}{0.150000in}}{\pgfqpoint{5.490039in}{5.490039in}}%
\pgfusepath{clip}%
\pgfsetbuttcap%
\pgfsetroundjoin%
\definecolor{currentfill}{rgb}{0.131172,0.555899,0.552459}%
\pgfsetfillcolor{currentfill}%
\pgfsetfillopacity{0.700000}%
\pgfsetlinewidth{0.000000pt}%
\definecolor{currentstroke}{rgb}{0.000000,0.000000,0.000000}%
\pgfsetstrokecolor{currentstroke}%
\pgfsetdash{}{0pt}%
\pgfpathmoveto{\pgfqpoint{4.899333in}{2.632818in}}%
\pgfpathlineto{\pgfqpoint{4.913360in}{2.644722in}}%
\pgfpathlineto{\pgfqpoint{4.927405in}{2.656789in}}%
\pgfpathlineto{\pgfqpoint{4.941467in}{2.669018in}}%
\pgfpathlineto{\pgfqpoint{4.955548in}{2.681410in}}%
\pgfpathlineto{\pgfqpoint{4.963098in}{2.691132in}}%
\pgfpathlineto{\pgfqpoint{4.970640in}{2.700703in}}%
\pgfpathlineto{\pgfqpoint{4.978174in}{2.710124in}}%
\pgfpathlineto{\pgfqpoint{4.985701in}{2.719394in}}%
\pgfpathlineto{\pgfqpoint{4.971624in}{2.707016in}}%
\pgfpathlineto{\pgfqpoint{4.957565in}{2.694801in}}%
\pgfpathlineto{\pgfqpoint{4.943523in}{2.682749in}}%
\pgfpathlineto{\pgfqpoint{4.929500in}{2.670858in}}%
\pgfpathlineto{\pgfqpoint{4.921969in}{2.661562in}}%
\pgfpathlineto{\pgfqpoint{4.914431in}{2.652124in}}%
\pgfpathlineto{\pgfqpoint{4.906886in}{2.642543in}}%
\pgfpathlineto{\pgfqpoint{4.899333in}{2.632818in}}%
\pgfpathclose%
\pgfusepath{fill}%
\end{pgfscope}%
\begin{pgfscope}%
\pgfpathrectangle{\pgfqpoint{1.254980in}{0.150000in}}{\pgfqpoint{5.490039in}{5.490039in}}%
\pgfusepath{clip}%
\pgfsetbuttcap%
\pgfsetroundjoin%
\definecolor{currentfill}{rgb}{0.246811,0.283237,0.535941}%
\pgfsetfillcolor{currentfill}%
\pgfsetfillopacity{0.700000}%
\pgfsetlinewidth{0.000000pt}%
\definecolor{currentstroke}{rgb}{0.000000,0.000000,0.000000}%
\pgfsetstrokecolor{currentstroke}%
\pgfsetdash{}{0pt}%
\pgfpathmoveto{\pgfqpoint{4.260101in}{1.928608in}}%
\pgfpathlineto{\pgfqpoint{4.273764in}{1.935144in}}%
\pgfpathlineto{\pgfqpoint{4.287441in}{1.941839in}}%
\pgfpathlineto{\pgfqpoint{4.301130in}{1.948695in}}%
\pgfpathlineto{\pgfqpoint{4.314832in}{1.955711in}}%
\pgfpathlineto{\pgfqpoint{4.322616in}{1.969246in}}%
\pgfpathlineto{\pgfqpoint{4.330396in}{1.982715in}}%
\pgfpathlineto{\pgfqpoint{4.338171in}{1.996118in}}%
\pgfpathlineto{\pgfqpoint{4.345942in}{2.009450in}}%
\pgfpathlineto{\pgfqpoint{4.332240in}{2.002153in}}%
\pgfpathlineto{\pgfqpoint{4.318550in}{1.995017in}}%
\pgfpathlineto{\pgfqpoint{4.304874in}{1.988040in}}%
\pgfpathlineto{\pgfqpoint{4.291211in}{1.981224in}}%
\pgfpathlineto{\pgfqpoint{4.283440in}{1.968161in}}%
\pgfpathlineto{\pgfqpoint{4.275665in}{1.955036in}}%
\pgfpathlineto{\pgfqpoint{4.267885in}{1.941851in}}%
\pgfpathlineto{\pgfqpoint{4.260101in}{1.928608in}}%
\pgfpathclose%
\pgfusepath{fill}%
\end{pgfscope}%
\begin{pgfscope}%
\pgfpathrectangle{\pgfqpoint{1.254980in}{0.150000in}}{\pgfqpoint{5.490039in}{5.490039in}}%
\pgfusepath{clip}%
\pgfsetbuttcap%
\pgfsetroundjoin%
\definecolor{currentfill}{rgb}{0.122312,0.633153,0.530398}%
\pgfsetfillcolor{currentfill}%
\pgfsetfillopacity{0.700000}%
\pgfsetlinewidth{0.000000pt}%
\definecolor{currentstroke}{rgb}{0.000000,0.000000,0.000000}%
\pgfsetstrokecolor{currentstroke}%
\pgfsetdash{}{0pt}%
\pgfpathmoveto{\pgfqpoint{5.102082in}{2.838903in}}%
\pgfpathlineto{\pgfqpoint{5.116241in}{2.851973in}}%
\pgfpathlineto{\pgfqpoint{5.130420in}{2.865206in}}%
\pgfpathlineto{\pgfqpoint{5.144618in}{2.878602in}}%
\pgfpathlineto{\pgfqpoint{5.158834in}{2.892161in}}%
\pgfpathlineto{\pgfqpoint{5.166276in}{2.899930in}}%
\pgfpathlineto{\pgfqpoint{5.173709in}{2.907543in}}%
\pgfpathlineto{\pgfqpoint{5.181134in}{2.915003in}}%
\pgfpathlineto{\pgfqpoint{5.188549in}{2.922310in}}%
\pgfpathlineto{\pgfqpoint{5.174339in}{2.908859in}}%
\pgfpathlineto{\pgfqpoint{5.160148in}{2.895570in}}%
\pgfpathlineto{\pgfqpoint{5.145977in}{2.882444in}}%
\pgfpathlineto{\pgfqpoint{5.131824in}{2.869481in}}%
\pgfpathlineto{\pgfqpoint{5.124402in}{2.862055in}}%
\pgfpathlineto{\pgfqpoint{5.116970in}{2.854484in}}%
\pgfpathlineto{\pgfqpoint{5.109530in}{2.846767in}}%
\pgfpathlineto{\pgfqpoint{5.102082in}{2.838903in}}%
\pgfpathclose%
\pgfusepath{fill}%
\end{pgfscope}%
\begin{pgfscope}%
\pgfpathrectangle{\pgfqpoint{1.254980in}{0.150000in}}{\pgfqpoint{5.490039in}{5.490039in}}%
\pgfusepath{clip}%
\pgfsetbuttcap%
\pgfsetroundjoin%
\definecolor{currentfill}{rgb}{0.180629,0.429975,0.557282}%
\pgfsetfillcolor{currentfill}%
\pgfsetfillopacity{0.700000}%
\pgfsetlinewidth{0.000000pt}%
\definecolor{currentstroke}{rgb}{0.000000,0.000000,0.000000}%
\pgfsetstrokecolor{currentstroke}%
\pgfsetdash{}{0pt}%
\pgfpathmoveto{\pgfqpoint{4.579751in}{2.282437in}}%
\pgfpathlineto{\pgfqpoint{4.593585in}{2.292017in}}%
\pgfpathlineto{\pgfqpoint{4.607434in}{2.301757in}}%
\pgfpathlineto{\pgfqpoint{4.621299in}{2.311659in}}%
\pgfpathlineto{\pgfqpoint{4.635179in}{2.321722in}}%
\pgfpathlineto{\pgfqpoint{4.642866in}{2.334013in}}%
\pgfpathlineto{\pgfqpoint{4.650546in}{2.346180in}}%
\pgfpathlineto{\pgfqpoint{4.658221in}{2.358224in}}%
\pgfpathlineto{\pgfqpoint{4.665889in}{2.370143in}}%
\pgfpathlineto{\pgfqpoint{4.652009in}{2.359943in}}%
\pgfpathlineto{\pgfqpoint{4.638144in}{2.349904in}}%
\pgfpathlineto{\pgfqpoint{4.624294in}{2.340027in}}%
\pgfpathlineto{\pgfqpoint{4.610460in}{2.330311in}}%
\pgfpathlineto{\pgfqpoint{4.602792in}{2.318517in}}%
\pgfpathlineto{\pgfqpoint{4.595117in}{2.306607in}}%
\pgfpathlineto{\pgfqpoint{4.587437in}{2.294580in}}%
\pgfpathlineto{\pgfqpoint{4.579751in}{2.282437in}}%
\pgfpathclose%
\pgfusepath{fill}%
\end{pgfscope}%
\begin{pgfscope}%
\pgfpathrectangle{\pgfqpoint{1.254980in}{0.150000in}}{\pgfqpoint{5.490039in}{5.490039in}}%
\pgfusepath{clip}%
\pgfsetbuttcap%
\pgfsetroundjoin%
\definecolor{currentfill}{rgb}{0.276022,0.044167,0.370164}%
\pgfsetfillcolor{currentfill}%
\pgfsetfillopacity{0.700000}%
\pgfsetlinewidth{0.000000pt}%
\definecolor{currentstroke}{rgb}{0.000000,0.000000,0.000000}%
\pgfsetstrokecolor{currentstroke}%
\pgfsetdash{}{0pt}%
\pgfpathmoveto{\pgfqpoint{3.683277in}{1.462011in}}%
\pgfpathlineto{\pgfqpoint{3.696745in}{1.461462in}}%
\pgfpathlineto{\pgfqpoint{3.710220in}{1.461076in}}%
\pgfpathlineto{\pgfqpoint{3.723702in}{1.460852in}}%
\pgfpathlineto{\pgfqpoint{3.737192in}{1.460789in}}%
\pgfpathlineto{\pgfqpoint{3.745151in}{1.471567in}}%
\pgfpathlineto{\pgfqpoint{3.753105in}{1.482440in}}%
\pgfpathlineto{\pgfqpoint{3.761054in}{1.493404in}}%
\pgfpathlineto{\pgfqpoint{3.768996in}{1.504453in}}%
\pgfpathlineto{\pgfqpoint{3.755517in}{1.504017in}}%
\pgfpathlineto{\pgfqpoint{3.742046in}{1.503742in}}%
\pgfpathlineto{\pgfqpoint{3.728582in}{1.503630in}}%
\pgfpathlineto{\pgfqpoint{3.715126in}{1.503680in}}%
\pgfpathlineto{\pgfqpoint{3.707173in}{1.493119in}}%
\pgfpathlineto{\pgfqpoint{3.699214in}{1.482651in}}%
\pgfpathlineto{\pgfqpoint{3.691249in}{1.472280in}}%
\pgfpathlineto{\pgfqpoint{3.683277in}{1.462011in}}%
\pgfpathclose%
\pgfusepath{fill}%
\end{pgfscope}%
\begin{pgfscope}%
\pgfpathrectangle{\pgfqpoint{1.254980in}{0.150000in}}{\pgfqpoint{5.490039in}{5.490039in}}%
\pgfusepath{clip}%
\pgfsetbuttcap%
\pgfsetroundjoin%
\definecolor{currentfill}{rgb}{0.395174,0.797475,0.367757}%
\pgfsetfillcolor{currentfill}%
\pgfsetfillopacity{0.700000}%
\pgfsetlinewidth{0.000000pt}%
\definecolor{currentstroke}{rgb}{0.000000,0.000000,0.000000}%
\pgfsetstrokecolor{currentstroke}%
\pgfsetdash{}{0pt}%
\pgfpathmoveto{\pgfqpoint{5.679111in}{3.344691in}}%
\pgfpathlineto{\pgfqpoint{5.693657in}{3.359933in}}%
\pgfpathlineto{\pgfqpoint{5.708225in}{3.375339in}}%
\pgfpathlineto{\pgfqpoint{5.722815in}{3.390908in}}%
\pgfpathlineto{\pgfqpoint{5.737427in}{3.406641in}}%
\pgfpathlineto{\pgfqpoint{5.744468in}{3.408492in}}%
\pgfpathlineto{\pgfqpoint{5.751499in}{3.410229in}}%
\pgfpathlineto{\pgfqpoint{5.758519in}{3.411856in}}%
\pgfpathlineto{\pgfqpoint{5.765529in}{3.413376in}}%
\pgfpathlineto{\pgfqpoint{5.750939in}{3.398007in}}%
\pgfpathlineto{\pgfqpoint{5.736372in}{3.382800in}}%
\pgfpathlineto{\pgfqpoint{5.721826in}{3.367757in}}%
\pgfpathlineto{\pgfqpoint{5.707302in}{3.352876in}}%
\pgfpathlineto{\pgfqpoint{5.700270in}{3.350982in}}%
\pgfpathlineto{\pgfqpoint{5.693227in}{3.348989in}}%
\pgfpathlineto{\pgfqpoint{5.686174in}{3.346894in}}%
\pgfpathlineto{\pgfqpoint{5.679111in}{3.344691in}}%
\pgfpathclose%
\pgfusepath{fill}%
\end{pgfscope}%
\begin{pgfscope}%
\pgfpathrectangle{\pgfqpoint{1.254980in}{0.150000in}}{\pgfqpoint{5.490039in}{5.490039in}}%
\pgfusepath{clip}%
\pgfsetbuttcap%
\pgfsetroundjoin%
\definecolor{currentfill}{rgb}{0.279566,0.067836,0.391917}%
\pgfsetfillcolor{currentfill}%
\pgfsetfillopacity{0.700000}%
\pgfsetlinewidth{0.000000pt}%
\definecolor{currentstroke}{rgb}{0.000000,0.000000,0.000000}%
\pgfsetstrokecolor{currentstroke}%
\pgfsetdash{}{0pt}%
\pgfpathmoveto{\pgfqpoint{3.768996in}{1.504453in}}%
\pgfpathlineto{\pgfqpoint{3.782482in}{1.505051in}}%
\pgfpathlineto{\pgfqpoint{3.795977in}{1.505810in}}%
\pgfpathlineto{\pgfqpoint{3.809479in}{1.506729in}}%
\pgfpathlineto{\pgfqpoint{3.822990in}{1.507810in}}%
\pgfpathlineto{\pgfqpoint{3.830917in}{1.519422in}}%
\pgfpathlineto{\pgfqpoint{3.838840in}{1.531103in}}%
\pgfpathlineto{\pgfqpoint{3.846757in}{1.542849in}}%
\pgfpathlineto{\pgfqpoint{3.854669in}{1.554656in}}%
\pgfpathlineto{\pgfqpoint{3.841166in}{1.553103in}}%
\pgfpathlineto{\pgfqpoint{3.827673in}{1.551711in}}%
\pgfpathlineto{\pgfqpoint{3.814187in}{1.550480in}}%
\pgfpathlineto{\pgfqpoint{3.800710in}{1.549411in}}%
\pgfpathlineto{\pgfqpoint{3.792790in}{1.538066in}}%
\pgfpathlineto{\pgfqpoint{3.784864in}{1.526789in}}%
\pgfpathlineto{\pgfqpoint{3.776933in}{1.515583in}}%
\pgfpathlineto{\pgfqpoint{3.768996in}{1.504453in}}%
\pgfpathclose%
\pgfusepath{fill}%
\end{pgfscope}%
\begin{pgfscope}%
\pgfpathrectangle{\pgfqpoint{1.254980in}{0.150000in}}{\pgfqpoint{5.490039in}{5.490039in}}%
\pgfusepath{clip}%
\pgfsetbuttcap%
\pgfsetroundjoin%
\definecolor{currentfill}{rgb}{0.267004,0.004874,0.329415}%
\pgfsetfillcolor{currentfill}%
\pgfsetfillopacity{0.700000}%
\pgfsetlinewidth{0.000000pt}%
\definecolor{currentstroke}{rgb}{0.000000,0.000000,0.000000}%
\pgfsetstrokecolor{currentstroke}%
\pgfsetdash{}{0pt}%
\pgfpathmoveto{\pgfqpoint{3.371581in}{1.406307in}}%
\pgfpathlineto{\pgfqpoint{3.385019in}{1.401449in}}%
\pgfpathlineto{\pgfqpoint{3.398461in}{1.396760in}}%
\pgfpathlineto{\pgfqpoint{3.411906in}{1.392240in}}%
\pgfpathlineto{\pgfqpoint{3.425355in}{1.387888in}}%
\pgfpathlineto{\pgfqpoint{3.433468in}{1.394754in}}%
\pgfpathlineto{\pgfqpoint{3.441573in}{1.401806in}}%
\pgfpathlineto{\pgfqpoint{3.449668in}{1.409040in}}%
\pgfpathlineto{\pgfqpoint{3.457755in}{1.416449in}}%
\pgfpathlineto{\pgfqpoint{3.444327in}{1.420220in}}%
\pgfpathlineto{\pgfqpoint{3.430904in}{1.424158in}}%
\pgfpathlineto{\pgfqpoint{3.417485in}{1.428266in}}%
\pgfpathlineto{\pgfqpoint{3.404069in}{1.432542in}}%
\pgfpathlineto{\pgfqpoint{3.395961in}{1.425703in}}%
\pgfpathlineto{\pgfqpoint{3.387844in}{1.419047in}}%
\pgfpathlineto{\pgfqpoint{3.379717in}{1.412580in}}%
\pgfpathlineto{\pgfqpoint{3.371581in}{1.406307in}}%
\pgfpathclose%
\pgfusepath{fill}%
\end{pgfscope}%
\begin{pgfscope}%
\pgfpathrectangle{\pgfqpoint{1.254980in}{0.150000in}}{\pgfqpoint{5.490039in}{5.490039in}}%
\pgfusepath{clip}%
\pgfsetbuttcap%
\pgfsetroundjoin%
\definecolor{currentfill}{rgb}{0.277941,0.056324,0.381191}%
\pgfsetfillcolor{currentfill}%
\pgfsetfillopacity{0.700000}%
\pgfsetlinewidth{0.000000pt}%
\definecolor{currentstroke}{rgb}{0.000000,0.000000,0.000000}%
\pgfsetstrokecolor{currentstroke}%
\pgfsetdash{}{0pt}%
\pgfpathmoveto{\pgfqpoint{3.036654in}{1.522542in}}%
\pgfpathlineto{\pgfqpoint{3.050123in}{1.512907in}}%
\pgfpathlineto{\pgfqpoint{3.063591in}{1.503456in}}%
\pgfpathlineto{\pgfqpoint{3.077059in}{1.494187in}}%
\pgfpathlineto{\pgfqpoint{3.090527in}{1.485101in}}%
\pgfpathlineto{\pgfqpoint{3.098874in}{1.486992in}}%
\pgfpathlineto{\pgfqpoint{3.107206in}{1.489156in}}%
\pgfpathlineto{\pgfqpoint{3.115526in}{1.491587in}}%
\pgfpathlineto{\pgfqpoint{3.123831in}{1.494279in}}%
\pgfpathlineto{\pgfqpoint{3.110398in}{1.502721in}}%
\pgfpathlineto{\pgfqpoint{3.096965in}{1.511345in}}%
\pgfpathlineto{\pgfqpoint{3.083532in}{1.520152in}}%
\pgfpathlineto{\pgfqpoint{3.070099in}{1.529142in}}%
\pgfpathlineto{\pgfqpoint{3.061759in}{1.527083in}}%
\pgfpathlineto{\pgfqpoint{3.053404in}{1.525293in}}%
\pgfpathlineto{\pgfqpoint{3.045036in}{1.523777in}}%
\pgfpathlineto{\pgfqpoint{3.036654in}{1.522542in}}%
\pgfpathclose%
\pgfusepath{fill}%
\end{pgfscope}%
\begin{pgfscope}%
\pgfpathrectangle{\pgfqpoint{1.254980in}{0.150000in}}{\pgfqpoint{5.490039in}{5.490039in}}%
\pgfusepath{clip}%
\pgfsetbuttcap%
\pgfsetroundjoin%
\definecolor{currentfill}{rgb}{0.153364,0.497000,0.557724}%
\pgfsetfillcolor{currentfill}%
\pgfsetfillopacity{0.700000}%
\pgfsetlinewidth{0.000000pt}%
\definecolor{currentstroke}{rgb}{0.000000,0.000000,0.000000}%
\pgfsetstrokecolor{currentstroke}%
\pgfsetdash{}{0pt}%
\pgfpathmoveto{\pgfqpoint{2.164819in}{2.577044in}}%
\pgfpathlineto{\pgfqpoint{2.178741in}{2.552832in}}%
\pgfpathlineto{\pgfqpoint{2.192649in}{2.528909in}}%
\pgfpathlineto{\pgfqpoint{2.206541in}{2.505274in}}%
\pgfpathlineto{\pgfqpoint{2.220420in}{2.481924in}}%
\pgfpathlineto{\pgfqpoint{2.229582in}{2.472933in}}%
\pgfpathlineto{\pgfqpoint{2.238717in}{2.464349in}}%
\pgfpathlineto{\pgfqpoint{2.247825in}{2.456165in}}%
\pgfpathlineto{\pgfqpoint{2.256907in}{2.448374in}}%
\pgfpathlineto{\pgfqpoint{2.243095in}{2.471017in}}%
\pgfpathlineto{\pgfqpoint{2.229270in}{2.493944in}}%
\pgfpathlineto{\pgfqpoint{2.215431in}{2.517156in}}%
\pgfpathlineto{\pgfqpoint{2.201577in}{2.540657in}}%
\pgfpathlineto{\pgfqpoint{2.192429in}{2.549143in}}%
\pgfpathlineto{\pgfqpoint{2.183253in}{2.558032in}}%
\pgfpathlineto{\pgfqpoint{2.174050in}{2.567330in}}%
\pgfpathlineto{\pgfqpoint{2.164819in}{2.577044in}}%
\pgfpathclose%
\pgfusepath{fill}%
\end{pgfscope}%
\begin{pgfscope}%
\pgfpathrectangle{\pgfqpoint{1.254980in}{0.150000in}}{\pgfqpoint{5.490039in}{5.490039in}}%
\pgfusepath{clip}%
\pgfsetbuttcap%
\pgfsetroundjoin%
\definecolor{currentfill}{rgb}{0.269944,0.014625,0.341379}%
\pgfsetfillcolor{currentfill}%
\pgfsetfillopacity{0.700000}%
\pgfsetlinewidth{0.000000pt}%
\definecolor{currentstroke}{rgb}{0.000000,0.000000,0.000000}%
\pgfsetstrokecolor{currentstroke}%
\pgfsetdash{}{0pt}%
\pgfpathmoveto{\pgfqpoint{3.231329in}{1.433180in}}%
\pgfpathlineto{\pgfqpoint{3.244772in}{1.426337in}}%
\pgfpathlineto{\pgfqpoint{3.258218in}{1.419668in}}%
\pgfpathlineto{\pgfqpoint{3.271665in}{1.413173in}}%
\pgfpathlineto{\pgfqpoint{3.285114in}{1.406850in}}%
\pgfpathlineto{\pgfqpoint{3.293317in}{1.411665in}}%
\pgfpathlineto{\pgfqpoint{3.301509in}{1.416706in}}%
\pgfpathlineto{\pgfqpoint{3.309691in}{1.421966in}}%
\pgfpathlineto{\pgfqpoint{3.317861in}{1.427440in}}%
\pgfpathlineto{\pgfqpoint{3.304439in}{1.433152in}}%
\pgfpathlineto{\pgfqpoint{3.291019in}{1.439036in}}%
\pgfpathlineto{\pgfqpoint{3.277601in}{1.445094in}}%
\pgfpathlineto{\pgfqpoint{3.264186in}{1.451325in}}%
\pgfpathlineto{\pgfqpoint{3.255989in}{1.446452in}}%
\pgfpathlineto{\pgfqpoint{3.247780in}{1.441800in}}%
\pgfpathlineto{\pgfqpoint{3.239560in}{1.437374in}}%
\pgfpathlineto{\pgfqpoint{3.231329in}{1.433180in}}%
\pgfpathclose%
\pgfusepath{fill}%
\end{pgfscope}%
\begin{pgfscope}%
\pgfpathrectangle{\pgfqpoint{1.254980in}{0.150000in}}{\pgfqpoint{5.490039in}{5.490039in}}%
\pgfusepath{clip}%
\pgfsetbuttcap%
\pgfsetroundjoin%
\definecolor{currentfill}{rgb}{0.458674,0.816363,0.329727}%
\pgfsetfillcolor{currentfill}%
\pgfsetfillopacity{0.700000}%
\pgfsetlinewidth{0.000000pt}%
\definecolor{currentstroke}{rgb}{0.000000,0.000000,0.000000}%
\pgfsetstrokecolor{currentstroke}%
\pgfsetdash{}{0pt}%
\pgfpathmoveto{\pgfqpoint{5.765529in}{3.413376in}}%
\pgfpathlineto{\pgfqpoint{5.780141in}{3.428909in}}%
\pgfpathlineto{\pgfqpoint{5.794775in}{3.444606in}}%
\pgfpathlineto{\pgfqpoint{5.809432in}{3.460467in}}%
\pgfpathlineto{\pgfqpoint{5.816414in}{3.461598in}}%
\pgfpathlineto{\pgfqpoint{5.823385in}{3.462625in}}%
\pgfpathlineto{\pgfqpoint{5.830345in}{3.463554in}}%
\pgfpathlineto{\pgfqpoint{5.837296in}{3.464388in}}%
\pgfpathlineto{\pgfqpoint{5.822664in}{3.448923in}}%
\pgfpathlineto{\pgfqpoint{5.808055in}{3.433621in}}%
\pgfpathlineto{\pgfqpoint{5.793467in}{3.418482in}}%
\pgfpathlineto{\pgfqpoint{5.786498in}{3.417343in}}%
\pgfpathlineto{\pgfqpoint{5.779518in}{3.416116in}}%
\pgfpathlineto{\pgfqpoint{5.772529in}{3.414795in}}%
\pgfpathlineto{\pgfqpoint{5.765529in}{3.413376in}}%
\pgfpathclose%
\pgfusepath{fill}%
\end{pgfscope}%
\begin{pgfscope}%
\pgfpathrectangle{\pgfqpoint{1.254980in}{0.150000in}}{\pgfqpoint{5.490039in}{5.490039in}}%
\pgfusepath{clip}%
\pgfsetbuttcap%
\pgfsetroundjoin%
\definecolor{currentfill}{rgb}{0.271305,0.019942,0.347269}%
\pgfsetfillcolor{currentfill}%
\pgfsetfillopacity{0.700000}%
\pgfsetlinewidth{0.000000pt}%
\definecolor{currentstroke}{rgb}{0.000000,0.000000,0.000000}%
\pgfsetstrokecolor{currentstroke}%
\pgfsetdash{}{0pt}%
\pgfpathmoveto{\pgfqpoint{3.597466in}{1.427980in}}%
\pgfpathlineto{\pgfqpoint{3.610922in}{1.426253in}}%
\pgfpathlineto{\pgfqpoint{3.624384in}{1.424690in}}%
\pgfpathlineto{\pgfqpoint{3.637852in}{1.423290in}}%
\pgfpathlineto{\pgfqpoint{3.651327in}{1.422052in}}%
\pgfpathlineto{\pgfqpoint{3.659324in}{1.431864in}}%
\pgfpathlineto{\pgfqpoint{3.667315in}{1.441798in}}%
\pgfpathlineto{\pgfqpoint{3.675299in}{1.451849in}}%
\pgfpathlineto{\pgfqpoint{3.683277in}{1.462011in}}%
\pgfpathlineto{\pgfqpoint{3.669816in}{1.462722in}}%
\pgfpathlineto{\pgfqpoint{3.656361in}{1.463596in}}%
\pgfpathlineto{\pgfqpoint{3.642913in}{1.464634in}}%
\pgfpathlineto{\pgfqpoint{3.629472in}{1.465835in}}%
\pgfpathlineto{\pgfqpoint{3.621481in}{1.456188in}}%
\pgfpathlineto{\pgfqpoint{3.613483in}{1.446659in}}%
\pgfpathlineto{\pgfqpoint{3.605478in}{1.437255in}}%
\pgfpathlineto{\pgfqpoint{3.597466in}{1.427980in}}%
\pgfpathclose%
\pgfusepath{fill}%
\end{pgfscope}%
\begin{pgfscope}%
\pgfpathrectangle{\pgfqpoint{1.254980in}{0.150000in}}{\pgfqpoint{5.490039in}{5.490039in}}%
\pgfusepath{clip}%
\pgfsetbuttcap%
\pgfsetroundjoin%
\definecolor{currentfill}{rgb}{0.282656,0.100196,0.422160}%
\pgfsetfillcolor{currentfill}%
\pgfsetfillopacity{0.700000}%
\pgfsetlinewidth{0.000000pt}%
\definecolor{currentstroke}{rgb}{0.000000,0.000000,0.000000}%
\pgfsetstrokecolor{currentstroke}%
\pgfsetdash{}{0pt}%
\pgfpathmoveto{\pgfqpoint{3.854669in}{1.554656in}}%
\pgfpathlineto{\pgfqpoint{3.868180in}{1.556369in}}%
\pgfpathlineto{\pgfqpoint{3.881700in}{1.558243in}}%
\pgfpathlineto{\pgfqpoint{3.895228in}{1.560277in}}%
\pgfpathlineto{\pgfqpoint{3.908766in}{1.562471in}}%
\pgfpathlineto{\pgfqpoint{3.916666in}{1.574789in}}%
\pgfpathlineto{\pgfqpoint{3.924561in}{1.587152in}}%
\pgfpathlineto{\pgfqpoint{3.932452in}{1.599555in}}%
\pgfpathlineto{\pgfqpoint{3.940337in}{1.611994in}}%
\pgfpathlineto{\pgfqpoint{3.926805in}{1.609354in}}%
\pgfpathlineto{\pgfqpoint{3.913283in}{1.606875in}}%
\pgfpathlineto{\pgfqpoint{3.899770in}{1.604556in}}%
\pgfpathlineto{\pgfqpoint{3.886266in}{1.602398in}}%
\pgfpathlineto{\pgfqpoint{3.878374in}{1.590393in}}%
\pgfpathlineto{\pgfqpoint{3.870477in}{1.578432in}}%
\pgfpathlineto{\pgfqpoint{3.862576in}{1.566518in}}%
\pgfpathlineto{\pgfqpoint{3.854669in}{1.554656in}}%
\pgfpathclose%
\pgfusepath{fill}%
\end{pgfscope}%
\begin{pgfscope}%
\pgfpathrectangle{\pgfqpoint{1.254980in}{0.150000in}}{\pgfqpoint{5.490039in}{5.490039in}}%
\pgfusepath{clip}%
\pgfsetbuttcap%
\pgfsetroundjoin%
\definecolor{currentfill}{rgb}{0.203063,0.379716,0.553925}%
\pgfsetfillcolor{currentfill}%
\pgfsetfillopacity{0.700000}%
\pgfsetlinewidth{0.000000pt}%
\definecolor{currentstroke}{rgb}{0.000000,0.000000,0.000000}%
\pgfsetstrokecolor{currentstroke}%
\pgfsetdash{}{0pt}%
\pgfpathmoveto{\pgfqpoint{4.462889in}{2.146027in}}%
\pgfpathlineto{\pgfqpoint{4.476662in}{2.154602in}}%
\pgfpathlineto{\pgfqpoint{4.490449in}{2.163338in}}%
\pgfpathlineto{\pgfqpoint{4.504251in}{2.172234in}}%
\pgfpathlineto{\pgfqpoint{4.518068in}{2.181291in}}%
\pgfpathlineto{\pgfqpoint{4.525797in}{2.194311in}}%
\pgfpathlineto{\pgfqpoint{4.533521in}{2.207226in}}%
\pgfpathlineto{\pgfqpoint{4.541240in}{2.220035in}}%
\pgfpathlineto{\pgfqpoint{4.548953in}{2.232737in}}%
\pgfpathlineto{\pgfqpoint{4.535135in}{2.223484in}}%
\pgfpathlineto{\pgfqpoint{4.521332in}{2.214392in}}%
\pgfpathlineto{\pgfqpoint{4.507544in}{2.205462in}}%
\pgfpathlineto{\pgfqpoint{4.493770in}{2.196692in}}%
\pgfpathlineto{\pgfqpoint{4.486057in}{2.184174in}}%
\pgfpathlineto{\pgfqpoint{4.478340in}{2.171557in}}%
\pgfpathlineto{\pgfqpoint{4.470617in}{2.158841in}}%
\pgfpathlineto{\pgfqpoint{4.462889in}{2.146027in}}%
\pgfpathclose%
\pgfusepath{fill}%
\end{pgfscope}%
\begin{pgfscope}%
\pgfpathrectangle{\pgfqpoint{1.254980in}{0.150000in}}{\pgfqpoint{5.490039in}{5.490039in}}%
\pgfusepath{clip}%
\pgfsetbuttcap%
\pgfsetroundjoin%
\definecolor{currentfill}{rgb}{0.226397,0.728888,0.462789}%
\pgfsetfillcolor{currentfill}%
\pgfsetfillopacity{0.700000}%
\pgfsetlinewidth{0.000000pt}%
\definecolor{currentstroke}{rgb}{0.000000,0.000000,0.000000}%
\pgfsetstrokecolor{currentstroke}%
\pgfsetdash{}{0pt}%
\pgfpathmoveto{\pgfqpoint{5.390980in}{3.108219in}}%
\pgfpathlineto{\pgfqpoint{5.405339in}{3.122628in}}%
\pgfpathlineto{\pgfqpoint{5.419718in}{3.137201in}}%
\pgfpathlineto{\pgfqpoint{5.434118in}{3.151938in}}%
\pgfpathlineto{\pgfqpoint{5.448539in}{3.166838in}}%
\pgfpathlineto{\pgfqpoint{5.455799in}{3.171682in}}%
\pgfpathlineto{\pgfqpoint{5.463049in}{3.176381in}}%
\pgfpathlineto{\pgfqpoint{5.470290in}{3.180938in}}%
\pgfpathlineto{\pgfqpoint{5.477520in}{3.185355in}}%
\pgfpathlineto{\pgfqpoint{5.463113in}{3.170690in}}%
\pgfpathlineto{\pgfqpoint{5.448727in}{3.156188in}}%
\pgfpathlineto{\pgfqpoint{5.434362in}{3.141849in}}%
\pgfpathlineto{\pgfqpoint{5.420017in}{3.127673in}}%
\pgfpathlineto{\pgfqpoint{5.412772in}{3.123010in}}%
\pgfpathlineto{\pgfqpoint{5.405518in}{3.118215in}}%
\pgfpathlineto{\pgfqpoint{5.398254in}{3.113286in}}%
\pgfpathlineto{\pgfqpoint{5.390980in}{3.108219in}}%
\pgfpathclose%
\pgfusepath{fill}%
\end{pgfscope}%
\begin{pgfscope}%
\pgfpathrectangle{\pgfqpoint{1.254980in}{0.150000in}}{\pgfqpoint{5.490039in}{5.490039in}}%
\pgfusepath{clip}%
\pgfsetbuttcap%
\pgfsetroundjoin%
\definecolor{currentfill}{rgb}{0.255645,0.260703,0.528312}%
\pgfsetfillcolor{currentfill}%
\pgfsetfillopacity{0.700000}%
\pgfsetlinewidth{0.000000pt}%
\definecolor{currentstroke}{rgb}{0.000000,0.000000,0.000000}%
\pgfsetstrokecolor{currentstroke}%
\pgfsetdash{}{0pt}%
\pgfpathmoveto{\pgfqpoint{2.569145in}{1.954034in}}%
\pgfpathlineto{\pgfqpoint{2.582780in}{1.937246in}}%
\pgfpathlineto{\pgfqpoint{2.596409in}{1.920682in}}%
\pgfpathlineto{\pgfqpoint{2.610030in}{1.904339in}}%
\pgfpathlineto{\pgfqpoint{2.623645in}{1.888216in}}%
\pgfpathlineto{\pgfqpoint{2.632417in}{1.883373in}}%
\pgfpathlineto{\pgfqpoint{2.641167in}{1.878895in}}%
\pgfpathlineto{\pgfqpoint{2.649897in}{1.874776in}}%
\pgfpathlineto{\pgfqpoint{2.658605in}{1.871007in}}%
\pgfpathlineto{\pgfqpoint{2.645044in}{1.886432in}}%
\pgfpathlineto{\pgfqpoint{2.631476in}{1.902076in}}%
\pgfpathlineto{\pgfqpoint{2.617901in}{1.917940in}}%
\pgfpathlineto{\pgfqpoint{2.604321in}{1.934026in}}%
\pgfpathlineto{\pgfqpoint{2.595559in}{1.938481in}}%
\pgfpathlineto{\pgfqpoint{2.586776in}{1.943296in}}%
\pgfpathlineto{\pgfqpoint{2.577972in}{1.948478in}}%
\pgfpathlineto{\pgfqpoint{2.569145in}{1.954034in}}%
\pgfpathclose%
\pgfusepath{fill}%
\end{pgfscope}%
\begin{pgfscope}%
\pgfpathrectangle{\pgfqpoint{1.254980in}{0.150000in}}{\pgfqpoint{5.490039in}{5.490039in}}%
\pgfusepath{clip}%
\pgfsetbuttcap%
\pgfsetroundjoin%
\definecolor{currentfill}{rgb}{0.265145,0.232956,0.516599}%
\pgfsetfillcolor{currentfill}%
\pgfsetfillopacity{0.700000}%
\pgfsetlinewidth{0.000000pt}%
\definecolor{currentstroke}{rgb}{0.000000,0.000000,0.000000}%
\pgfsetstrokecolor{currentstroke}%
\pgfsetdash{}{0pt}%
\pgfpathmoveto{\pgfqpoint{2.623645in}{1.888216in}}%
\pgfpathlineto{\pgfqpoint{2.637253in}{1.872312in}}%
\pgfpathlineto{\pgfqpoint{2.650855in}{1.856624in}}%
\pgfpathlineto{\pgfqpoint{2.664451in}{1.841153in}}%
\pgfpathlineto{\pgfqpoint{2.678041in}{1.825895in}}%
\pgfpathlineto{\pgfqpoint{2.686760in}{1.821759in}}%
\pgfpathlineto{\pgfqpoint{2.695458in}{1.817981in}}%
\pgfpathlineto{\pgfqpoint{2.704137in}{1.814553in}}%
\pgfpathlineto{\pgfqpoint{2.712795in}{1.811468in}}%
\pgfpathlineto{\pgfqpoint{2.699256in}{1.826032in}}%
\pgfpathlineto{\pgfqpoint{2.685711in}{1.840809in}}%
\pgfpathlineto{\pgfqpoint{2.672161in}{1.855800in}}%
\pgfpathlineto{\pgfqpoint{2.658605in}{1.871007in}}%
\pgfpathlineto{\pgfqpoint{2.649897in}{1.874776in}}%
\pgfpathlineto{\pgfqpoint{2.641167in}{1.878895in}}%
\pgfpathlineto{\pgfqpoint{2.632417in}{1.883373in}}%
\pgfpathlineto{\pgfqpoint{2.623645in}{1.888216in}}%
\pgfpathclose%
\pgfusepath{fill}%
\end{pgfscope}%
\begin{pgfscope}%
\pgfpathrectangle{\pgfqpoint{1.254980in}{0.150000in}}{\pgfqpoint{5.490039in}{5.490039in}}%
\pgfusepath{clip}%
\pgfsetbuttcap%
\pgfsetroundjoin%
\definecolor{currentfill}{rgb}{0.266580,0.228262,0.514349}%
\pgfsetfillcolor{currentfill}%
\pgfsetfillopacity{0.700000}%
\pgfsetlinewidth{0.000000pt}%
\definecolor{currentstroke}{rgb}{0.000000,0.000000,0.000000}%
\pgfsetstrokecolor{currentstroke}%
\pgfsetdash{}{0pt}%
\pgfpathmoveto{\pgfqpoint{4.143131in}{1.798945in}}%
\pgfpathlineto{\pgfqpoint{4.156748in}{1.804196in}}%
\pgfpathlineto{\pgfqpoint{4.170376in}{1.809607in}}%
\pgfpathlineto{\pgfqpoint{4.184016in}{1.815178in}}%
\pgfpathlineto{\pgfqpoint{4.197668in}{1.820908in}}%
\pgfpathlineto{\pgfqpoint{4.205487in}{1.834515in}}%
\pgfpathlineto{\pgfqpoint{4.213302in}{1.848086in}}%
\pgfpathlineto{\pgfqpoint{4.221113in}{1.861619in}}%
\pgfpathlineto{\pgfqpoint{4.228919in}{1.875111in}}%
\pgfpathlineto{\pgfqpoint{4.215268in}{1.869043in}}%
\pgfpathlineto{\pgfqpoint{4.201629in}{1.863136in}}%
\pgfpathlineto{\pgfqpoint{4.188002in}{1.857388in}}%
\pgfpathlineto{\pgfqpoint{4.174386in}{1.851800in}}%
\pgfpathlineto{\pgfqpoint{4.166579in}{1.838635in}}%
\pgfpathlineto{\pgfqpoint{4.158767in}{1.825435in}}%
\pgfpathlineto{\pgfqpoint{4.150951in}{1.812204in}}%
\pgfpathlineto{\pgfqpoint{4.143131in}{1.798945in}}%
\pgfpathclose%
\pgfusepath{fill}%
\end{pgfscope}%
\begin{pgfscope}%
\pgfpathrectangle{\pgfqpoint{1.254980in}{0.150000in}}{\pgfqpoint{5.490039in}{5.490039in}}%
\pgfusepath{clip}%
\pgfsetbuttcap%
\pgfsetroundjoin%
\definecolor{currentfill}{rgb}{0.147607,0.511733,0.557049}%
\pgfsetfillcolor{currentfill}%
\pgfsetfillopacity{0.700000}%
\pgfsetlinewidth{0.000000pt}%
\definecolor{currentstroke}{rgb}{0.000000,0.000000,0.000000}%
\pgfsetstrokecolor{currentstroke}%
\pgfsetdash{}{0pt}%
\pgfpathmoveto{\pgfqpoint{4.782700in}{2.504069in}}%
\pgfpathlineto{\pgfqpoint{4.796661in}{2.515262in}}%
\pgfpathlineto{\pgfqpoint{4.810640in}{2.526616in}}%
\pgfpathlineto{\pgfqpoint{4.824636in}{2.538132in}}%
\pgfpathlineto{\pgfqpoint{4.838649in}{2.549811in}}%
\pgfpathlineto{\pgfqpoint{4.846260in}{2.560694in}}%
\pgfpathlineto{\pgfqpoint{4.853863in}{2.571433in}}%
\pgfpathlineto{\pgfqpoint{4.861459in}{2.582026in}}%
\pgfpathlineto{\pgfqpoint{4.869048in}{2.592474in}}%
\pgfpathlineto{\pgfqpoint{4.855036in}{2.580749in}}%
\pgfpathlineto{\pgfqpoint{4.841042in}{2.569185in}}%
\pgfpathlineto{\pgfqpoint{4.827064in}{2.557784in}}%
\pgfpathlineto{\pgfqpoint{4.813104in}{2.546545in}}%
\pgfpathlineto{\pgfqpoint{4.805513in}{2.536132in}}%
\pgfpathlineto{\pgfqpoint{4.797915in}{2.525582in}}%
\pgfpathlineto{\pgfqpoint{4.790311in}{2.514894in}}%
\pgfpathlineto{\pgfqpoint{4.782700in}{2.504069in}}%
\pgfpathclose%
\pgfusepath{fill}%
\end{pgfscope}%
\begin{pgfscope}%
\pgfpathrectangle{\pgfqpoint{1.254980in}{0.150000in}}{\pgfqpoint{5.490039in}{5.490039in}}%
\pgfusepath{clip}%
\pgfsetbuttcap%
\pgfsetroundjoin%
\definecolor{currentfill}{rgb}{0.243113,0.292092,0.538516}%
\pgfsetfillcolor{currentfill}%
\pgfsetfillopacity{0.700000}%
\pgfsetlinewidth{0.000000pt}%
\definecolor{currentstroke}{rgb}{0.000000,0.000000,0.000000}%
\pgfsetstrokecolor{currentstroke}%
\pgfsetdash{}{0pt}%
\pgfpathmoveto{\pgfqpoint{2.514528in}{2.023448in}}%
\pgfpathlineto{\pgfqpoint{2.528194in}{2.005751in}}%
\pgfpathlineto{\pgfqpoint{2.541852in}{1.988284in}}%
\pgfpathlineto{\pgfqpoint{2.555502in}{1.971046in}}%
\pgfpathlineto{\pgfqpoint{2.569145in}{1.954034in}}%
\pgfpathlineto{\pgfqpoint{2.577972in}{1.948478in}}%
\pgfpathlineto{\pgfqpoint{2.586776in}{1.943296in}}%
\pgfpathlineto{\pgfqpoint{2.595559in}{1.938481in}}%
\pgfpathlineto{\pgfqpoint{2.604321in}{1.934026in}}%
\pgfpathlineto{\pgfqpoint{2.590733in}{1.950336in}}%
\pgfpathlineto{\pgfqpoint{2.577138in}{1.966871in}}%
\pgfpathlineto{\pgfqpoint{2.563536in}{1.983633in}}%
\pgfpathlineto{\pgfqpoint{2.549927in}{2.000624in}}%
\pgfpathlineto{\pgfqpoint{2.541111in}{2.005770in}}%
\pgfpathlineto{\pgfqpoint{2.532273in}{2.011285in}}%
\pgfpathlineto{\pgfqpoint{2.523412in}{2.017175in}}%
\pgfpathlineto{\pgfqpoint{2.514528in}{2.023448in}}%
\pgfpathclose%
\pgfusepath{fill}%
\end{pgfscope}%
\begin{pgfscope}%
\pgfpathrectangle{\pgfqpoint{1.254980in}{0.150000in}}{\pgfqpoint{5.490039in}{5.490039in}}%
\pgfusepath{clip}%
\pgfsetbuttcap%
\pgfsetroundjoin%
\definecolor{currentfill}{rgb}{0.273006,0.204520,0.501721}%
\pgfsetfillcolor{currentfill}%
\pgfsetfillopacity{0.700000}%
\pgfsetlinewidth{0.000000pt}%
\definecolor{currentstroke}{rgb}{0.000000,0.000000,0.000000}%
\pgfsetstrokecolor{currentstroke}%
\pgfsetdash{}{0pt}%
\pgfpathmoveto{\pgfqpoint{2.678041in}{1.825895in}}%
\pgfpathlineto{\pgfqpoint{2.691626in}{1.810849in}}%
\pgfpathlineto{\pgfqpoint{2.705205in}{1.796015in}}%
\pgfpathlineto{\pgfqpoint{2.718779in}{1.781391in}}%
\pgfpathlineto{\pgfqpoint{2.732348in}{1.766975in}}%
\pgfpathlineto{\pgfqpoint{2.741016in}{1.763544in}}%
\pgfpathlineto{\pgfqpoint{2.749664in}{1.760462in}}%
\pgfpathlineto{\pgfqpoint{2.758293in}{1.757721in}}%
\pgfpathlineto{\pgfqpoint{2.766903in}{1.755315in}}%
\pgfpathlineto{\pgfqpoint{2.753383in}{1.769041in}}%
\pgfpathlineto{\pgfqpoint{2.739859in}{1.782974in}}%
\pgfpathlineto{\pgfqpoint{2.726329in}{1.797116in}}%
\pgfpathlineto{\pgfqpoint{2.712795in}{1.811468in}}%
\pgfpathlineto{\pgfqpoint{2.704137in}{1.814553in}}%
\pgfpathlineto{\pgfqpoint{2.695458in}{1.817981in}}%
\pgfpathlineto{\pgfqpoint{2.686760in}{1.821759in}}%
\pgfpathlineto{\pgfqpoint{2.678041in}{1.825895in}}%
\pgfpathclose%
\pgfusepath{fill}%
\end{pgfscope}%
\begin{pgfscope}%
\pgfpathrectangle{\pgfqpoint{1.254980in}{0.150000in}}{\pgfqpoint{5.490039in}{5.490039in}}%
\pgfusepath{clip}%
\pgfsetbuttcap%
\pgfsetroundjoin%
\definecolor{currentfill}{rgb}{0.282884,0.135920,0.453427}%
\pgfsetfillcolor{currentfill}%
\pgfsetfillopacity{0.700000}%
\pgfsetlinewidth{0.000000pt}%
\definecolor{currentstroke}{rgb}{0.000000,0.000000,0.000000}%
\pgfsetstrokecolor{currentstroke}%
\pgfsetdash{}{0pt}%
\pgfpathmoveto{\pgfqpoint{3.940337in}{1.611994in}}%
\pgfpathlineto{\pgfqpoint{3.953879in}{1.614794in}}%
\pgfpathlineto{\pgfqpoint{3.967430in}{1.617753in}}%
\pgfpathlineto{\pgfqpoint{3.980991in}{1.620872in}}%
\pgfpathlineto{\pgfqpoint{3.994562in}{1.624151in}}%
\pgfpathlineto{\pgfqpoint{4.002438in}{1.637052in}}%
\pgfpathlineto{\pgfqpoint{4.010309in}{1.649974in}}%
\pgfpathlineto{\pgfqpoint{4.018176in}{1.662913in}}%
\pgfpathlineto{\pgfqpoint{4.026039in}{1.675866in}}%
\pgfpathlineto{\pgfqpoint{4.012472in}{1.672168in}}%
\pgfpathlineto{\pgfqpoint{3.998915in}{1.668630in}}%
\pgfpathlineto{\pgfqpoint{3.985369in}{1.665253in}}%
\pgfpathlineto{\pgfqpoint{3.971832in}{1.662035in}}%
\pgfpathlineto{\pgfqpoint{3.963966in}{1.649490in}}%
\pgfpathlineto{\pgfqpoint{3.956094in}{1.636966in}}%
\pgfpathlineto{\pgfqpoint{3.948218in}{1.624466in}}%
\pgfpathlineto{\pgfqpoint{3.940337in}{1.611994in}}%
\pgfpathclose%
\pgfusepath{fill}%
\end{pgfscope}%
\begin{pgfscope}%
\pgfpathrectangle{\pgfqpoint{1.254980in}{0.150000in}}{\pgfqpoint{5.490039in}{5.490039in}}%
\pgfusepath{clip}%
\pgfsetbuttcap%
\pgfsetroundjoin%
\definecolor{currentfill}{rgb}{0.268510,0.009605,0.335427}%
\pgfsetfillcolor{currentfill}%
\pgfsetfillopacity{0.700000}%
\pgfsetlinewidth{0.000000pt}%
\definecolor{currentstroke}{rgb}{0.000000,0.000000,0.000000}%
\pgfsetstrokecolor{currentstroke}%
\pgfsetdash{}{0pt}%
\pgfpathmoveto{\pgfqpoint{3.511510in}{1.403036in}}%
\pgfpathlineto{\pgfqpoint{3.524961in}{1.400098in}}%
\pgfpathlineto{\pgfqpoint{3.538417in}{1.397325in}}%
\pgfpathlineto{\pgfqpoint{3.551879in}{1.394716in}}%
\pgfpathlineto{\pgfqpoint{3.565345in}{1.392272in}}%
\pgfpathlineto{\pgfqpoint{3.573386in}{1.400979in}}%
\pgfpathlineto{\pgfqpoint{3.581420in}{1.409836in}}%
\pgfpathlineto{\pgfqpoint{3.589447in}{1.418838in}}%
\pgfpathlineto{\pgfqpoint{3.597466in}{1.427980in}}%
\pgfpathlineto{\pgfqpoint{3.584015in}{1.429870in}}%
\pgfpathlineto{\pgfqpoint{3.570571in}{1.431925in}}%
\pgfpathlineto{\pgfqpoint{3.557132in}{1.434145in}}%
\pgfpathlineto{\pgfqpoint{3.543699in}{1.436530in}}%
\pgfpathlineto{\pgfqpoint{3.535664in}{1.427931in}}%
\pgfpathlineto{\pgfqpoint{3.527620in}{1.419480in}}%
\pgfpathlineto{\pgfqpoint{3.519569in}{1.411179in}}%
\pgfpathlineto{\pgfqpoint{3.511510in}{1.403036in}}%
\pgfpathclose%
\pgfusepath{fill}%
\end{pgfscope}%
\begin{pgfscope}%
\pgfpathrectangle{\pgfqpoint{1.254980in}{0.150000in}}{\pgfqpoint{5.490039in}{5.490039in}}%
\pgfusepath{clip}%
\pgfsetbuttcap%
\pgfsetroundjoin%
\definecolor{currentfill}{rgb}{0.229739,0.322361,0.545706}%
\pgfsetfillcolor{currentfill}%
\pgfsetfillopacity{0.700000}%
\pgfsetlinewidth{0.000000pt}%
\definecolor{currentstroke}{rgb}{0.000000,0.000000,0.000000}%
\pgfsetstrokecolor{currentstroke}%
\pgfsetdash{}{0pt}%
\pgfpathmoveto{\pgfqpoint{2.459778in}{2.096566in}}%
\pgfpathlineto{\pgfqpoint{2.473479in}{2.077933in}}%
\pgfpathlineto{\pgfqpoint{2.487170in}{2.059537in}}%
\pgfpathlineto{\pgfqpoint{2.500853in}{2.041376in}}%
\pgfpathlineto{\pgfqpoint{2.514528in}{2.023448in}}%
\pgfpathlineto{\pgfqpoint{2.523412in}{2.017175in}}%
\pgfpathlineto{\pgfqpoint{2.532273in}{2.011285in}}%
\pgfpathlineto{\pgfqpoint{2.541111in}{2.005770in}}%
\pgfpathlineto{\pgfqpoint{2.549927in}{2.000624in}}%
\pgfpathlineto{\pgfqpoint{2.536310in}{2.017845in}}%
\pgfpathlineto{\pgfqpoint{2.522685in}{2.035298in}}%
\pgfpathlineto{\pgfqpoint{2.509052in}{2.052984in}}%
\pgfpathlineto{\pgfqpoint{2.495410in}{2.070907in}}%
\pgfpathlineto{\pgfqpoint{2.486537in}{2.076749in}}%
\pgfpathlineto{\pgfqpoint{2.477642in}{2.082968in}}%
\pgfpathlineto{\pgfqpoint{2.468722in}{2.089571in}}%
\pgfpathlineto{\pgfqpoint{2.459778in}{2.096566in}}%
\pgfpathclose%
\pgfusepath{fill}%
\end{pgfscope}%
\begin{pgfscope}%
\pgfpathrectangle{\pgfqpoint{1.254980in}{0.150000in}}{\pgfqpoint{5.490039in}{5.490039in}}%
\pgfusepath{clip}%
\pgfsetbuttcap%
\pgfsetroundjoin%
\definecolor{currentfill}{rgb}{0.278012,0.180367,0.486697}%
\pgfsetfillcolor{currentfill}%
\pgfsetfillopacity{0.700000}%
\pgfsetlinewidth{0.000000pt}%
\definecolor{currentstroke}{rgb}{0.000000,0.000000,0.000000}%
\pgfsetstrokecolor{currentstroke}%
\pgfsetdash{}{0pt}%
\pgfpathmoveto{\pgfqpoint{2.732348in}{1.766975in}}%
\pgfpathlineto{\pgfqpoint{2.745912in}{1.752766in}}%
\pgfpathlineto{\pgfqpoint{2.759472in}{1.738763in}}%
\pgfpathlineto{\pgfqpoint{2.773027in}{1.724965in}}%
\pgfpathlineto{\pgfqpoint{2.786578in}{1.711370in}}%
\pgfpathlineto{\pgfqpoint{2.795197in}{1.708639in}}%
\pgfpathlineto{\pgfqpoint{2.803797in}{1.706249in}}%
\pgfpathlineto{\pgfqpoint{2.812379in}{1.704193in}}%
\pgfpathlineto{\pgfqpoint{2.820943in}{1.702463in}}%
\pgfpathlineto{\pgfqpoint{2.807439in}{1.715371in}}%
\pgfpathlineto{\pgfqpoint{2.793931in}{1.728481in}}%
\pgfpathlineto{\pgfqpoint{2.780419in}{1.741796in}}%
\pgfpathlineto{\pgfqpoint{2.766903in}{1.755315in}}%
\pgfpathlineto{\pgfqpoint{2.758293in}{1.757721in}}%
\pgfpathlineto{\pgfqpoint{2.749664in}{1.760462in}}%
\pgfpathlineto{\pgfqpoint{2.741016in}{1.763544in}}%
\pgfpathlineto{\pgfqpoint{2.732348in}{1.766975in}}%
\pgfpathclose%
\pgfusepath{fill}%
\end{pgfscope}%
\begin{pgfscope}%
\pgfpathrectangle{\pgfqpoint{1.254980in}{0.150000in}}{\pgfqpoint{5.490039in}{5.490039in}}%
\pgfusepath{clip}%
\pgfsetbuttcap%
\pgfsetroundjoin%
\definecolor{currentfill}{rgb}{0.229739,0.322361,0.545706}%
\pgfsetfillcolor{currentfill}%
\pgfsetfillopacity{0.700000}%
\pgfsetlinewidth{0.000000pt}%
\definecolor{currentstroke}{rgb}{0.000000,0.000000,0.000000}%
\pgfsetstrokecolor{currentstroke}%
\pgfsetdash{}{0pt}%
\pgfpathmoveto{\pgfqpoint{4.345942in}{2.009450in}}%
\pgfpathlineto{\pgfqpoint{4.359658in}{2.016907in}}%
\pgfpathlineto{\pgfqpoint{4.373387in}{2.024524in}}%
\pgfpathlineto{\pgfqpoint{4.387130in}{2.032301in}}%
\pgfpathlineto{\pgfqpoint{4.400887in}{2.040238in}}%
\pgfpathlineto{\pgfqpoint{4.408654in}{2.053762in}}%
\pgfpathlineto{\pgfqpoint{4.416416in}{2.067205in}}%
\pgfpathlineto{\pgfqpoint{4.424174in}{2.080564in}}%
\pgfpathlineto{\pgfqpoint{4.431927in}{2.093836in}}%
\pgfpathlineto{\pgfqpoint{4.418169in}{2.085646in}}%
\pgfpathlineto{\pgfqpoint{4.404425in}{2.077616in}}%
\pgfpathlineto{\pgfqpoint{4.390695in}{2.069746in}}%
\pgfpathlineto{\pgfqpoint{4.376979in}{2.062037in}}%
\pgfpathlineto{\pgfqpoint{4.369227in}{2.049006in}}%
\pgfpathlineto{\pgfqpoint{4.361470in}{2.035897in}}%
\pgfpathlineto{\pgfqpoint{4.353708in}{2.022711in}}%
\pgfpathlineto{\pgfqpoint{4.345942in}{2.009450in}}%
\pgfpathclose%
\pgfusepath{fill}%
\end{pgfscope}%
\begin{pgfscope}%
\pgfpathrectangle{\pgfqpoint{1.254980in}{0.150000in}}{\pgfqpoint{5.490039in}{5.490039in}}%
\pgfusepath{clip}%
\pgfsetbuttcap%
\pgfsetroundjoin%
\definecolor{currentfill}{rgb}{0.121148,0.592739,0.544641}%
\pgfsetfillcolor{currentfill}%
\pgfsetfillopacity{0.700000}%
\pgfsetlinewidth{0.000000pt}%
\definecolor{currentstroke}{rgb}{0.000000,0.000000,0.000000}%
\pgfsetstrokecolor{currentstroke}%
\pgfsetdash{}{0pt}%
\pgfpathmoveto{\pgfqpoint{4.985701in}{2.719394in}}%
\pgfpathlineto{\pgfqpoint{4.999797in}{2.731934in}}%
\pgfpathlineto{\pgfqpoint{5.013910in}{2.744637in}}%
\pgfpathlineto{\pgfqpoint{5.028043in}{2.757502in}}%
\pgfpathlineto{\pgfqpoint{5.042193in}{2.770531in}}%
\pgfpathlineto{\pgfqpoint{5.049708in}{2.779618in}}%
\pgfpathlineto{\pgfqpoint{5.057215in}{2.788549in}}%
\pgfpathlineto{\pgfqpoint{5.064714in}{2.797324in}}%
\pgfpathlineto{\pgfqpoint{5.072204in}{2.805945in}}%
\pgfpathlineto{\pgfqpoint{5.058058in}{2.792962in}}%
\pgfpathlineto{\pgfqpoint{5.043930in}{2.780141in}}%
\pgfpathlineto{\pgfqpoint{5.029821in}{2.767483in}}%
\pgfpathlineto{\pgfqpoint{5.015730in}{2.754988in}}%
\pgfpathlineto{\pgfqpoint{5.008234in}{2.746311in}}%
\pgfpathlineto{\pgfqpoint{5.000731in}{2.737487in}}%
\pgfpathlineto{\pgfqpoint{4.993220in}{2.728515in}}%
\pgfpathlineto{\pgfqpoint{4.985701in}{2.719394in}}%
\pgfpathclose%
\pgfusepath{fill}%
\end{pgfscope}%
\begin{pgfscope}%
\pgfpathrectangle{\pgfqpoint{1.254980in}{0.150000in}}{\pgfqpoint{5.490039in}{5.490039in}}%
\pgfusepath{clip}%
\pgfsetbuttcap%
\pgfsetroundjoin%
\definecolor{currentfill}{rgb}{0.140210,0.665859,0.513427}%
\pgfsetfillcolor{currentfill}%
\pgfsetfillopacity{0.700000}%
\pgfsetlinewidth{0.000000pt}%
\definecolor{currentstroke}{rgb}{0.000000,0.000000,0.000000}%
\pgfsetstrokecolor{currentstroke}%
\pgfsetdash{}{0pt}%
\pgfpathmoveto{\pgfqpoint{5.188549in}{2.922310in}}%
\pgfpathlineto{\pgfqpoint{5.202778in}{2.935925in}}%
\pgfpathlineto{\pgfqpoint{5.217027in}{2.949703in}}%
\pgfpathlineto{\pgfqpoint{5.231295in}{2.963645in}}%
\pgfpathlineto{\pgfqpoint{5.245583in}{2.977750in}}%
\pgfpathlineto{\pgfqpoint{5.252982in}{2.984779in}}%
\pgfpathlineto{\pgfqpoint{5.260371in}{2.991652in}}%
\pgfpathlineto{\pgfqpoint{5.267751in}{2.998371in}}%
\pgfpathlineto{\pgfqpoint{5.275121in}{3.004937in}}%
\pgfpathlineto{\pgfqpoint{5.260842in}{2.990971in}}%
\pgfpathlineto{\pgfqpoint{5.246582in}{2.977169in}}%
\pgfpathlineto{\pgfqpoint{5.232342in}{2.963530in}}%
\pgfpathlineto{\pgfqpoint{5.218121in}{2.950054in}}%
\pgfpathlineto{\pgfqpoint{5.210741in}{2.943338in}}%
\pgfpathlineto{\pgfqpoint{5.203353in}{2.936476in}}%
\pgfpathlineto{\pgfqpoint{5.195955in}{2.929468in}}%
\pgfpathlineto{\pgfqpoint{5.188549in}{2.922310in}}%
\pgfpathclose%
\pgfusepath{fill}%
\end{pgfscope}%
\begin{pgfscope}%
\pgfpathrectangle{\pgfqpoint{1.254980in}{0.150000in}}{\pgfqpoint{5.490039in}{5.490039in}}%
\pgfusepath{clip}%
\pgfsetbuttcap%
\pgfsetroundjoin%
\definecolor{currentfill}{rgb}{0.276022,0.044167,0.370164}%
\pgfsetfillcolor{currentfill}%
\pgfsetfillopacity{0.700000}%
\pgfsetlinewidth{0.000000pt}%
\definecolor{currentstroke}{rgb}{0.000000,0.000000,0.000000}%
\pgfsetstrokecolor{currentstroke}%
\pgfsetdash{}{0pt}%
\pgfpathmoveto{\pgfqpoint{3.090527in}{1.485101in}}%
\pgfpathlineto{\pgfqpoint{3.103995in}{1.476195in}}%
\pgfpathlineto{\pgfqpoint{3.117463in}{1.467470in}}%
\pgfpathlineto{\pgfqpoint{3.130931in}{1.458924in}}%
\pgfpathlineto{\pgfqpoint{3.144400in}{1.450557in}}%
\pgfpathlineto{\pgfqpoint{3.152713in}{1.453103in}}%
\pgfpathlineto{\pgfqpoint{3.161012in}{1.455914in}}%
\pgfpathlineto{\pgfqpoint{3.169298in}{1.458985in}}%
\pgfpathlineto{\pgfqpoint{3.177572in}{1.462308in}}%
\pgfpathlineto{\pgfqpoint{3.164135in}{1.470032in}}%
\pgfpathlineto{\pgfqpoint{3.150700in}{1.477935in}}%
\pgfpathlineto{\pgfqpoint{3.137265in}{1.486017in}}%
\pgfpathlineto{\pgfqpoint{3.123831in}{1.494279in}}%
\pgfpathlineto{\pgfqpoint{3.115526in}{1.491587in}}%
\pgfpathlineto{\pgfqpoint{3.107206in}{1.489156in}}%
\pgfpathlineto{\pgfqpoint{3.098874in}{1.486992in}}%
\pgfpathlineto{\pgfqpoint{3.090527in}{1.485101in}}%
\pgfpathclose%
\pgfusepath{fill}%
\end{pgfscope}%
\begin{pgfscope}%
\pgfpathrectangle{\pgfqpoint{1.254980in}{0.150000in}}{\pgfqpoint{5.490039in}{5.490039in}}%
\pgfusepath{clip}%
\pgfsetbuttcap%
\pgfsetroundjoin%
\definecolor{currentfill}{rgb}{0.281412,0.155834,0.469201}%
\pgfsetfillcolor{currentfill}%
\pgfsetfillopacity{0.700000}%
\pgfsetlinewidth{0.000000pt}%
\definecolor{currentstroke}{rgb}{0.000000,0.000000,0.000000}%
\pgfsetstrokecolor{currentstroke}%
\pgfsetdash{}{0pt}%
\pgfpathmoveto{\pgfqpoint{2.786578in}{1.711370in}}%
\pgfpathlineto{\pgfqpoint{2.800125in}{1.697976in}}%
\pgfpathlineto{\pgfqpoint{2.813668in}{1.684784in}}%
\pgfpathlineto{\pgfqpoint{2.827207in}{1.671792in}}%
\pgfpathlineto{\pgfqpoint{2.840743in}{1.658997in}}%
\pgfpathlineto{\pgfqpoint{2.849315in}{1.656964in}}%
\pgfpathlineto{\pgfqpoint{2.857870in}{1.655264in}}%
\pgfpathlineto{\pgfqpoint{2.866406in}{1.653888in}}%
\pgfpathlineto{\pgfqpoint{2.874925in}{1.652831in}}%
\pgfpathlineto{\pgfqpoint{2.861434in}{1.664941in}}%
\pgfpathlineto{\pgfqpoint{2.847940in}{1.677249in}}%
\pgfpathlineto{\pgfqpoint{2.834443in}{1.689756in}}%
\pgfpathlineto{\pgfqpoint{2.820943in}{1.702463in}}%
\pgfpathlineto{\pgfqpoint{2.812379in}{1.704193in}}%
\pgfpathlineto{\pgfqpoint{2.803797in}{1.706249in}}%
\pgfpathlineto{\pgfqpoint{2.795197in}{1.708639in}}%
\pgfpathlineto{\pgfqpoint{2.786578in}{1.711370in}}%
\pgfpathclose%
\pgfusepath{fill}%
\end{pgfscope}%
\begin{pgfscope}%
\pgfpathrectangle{\pgfqpoint{1.254980in}{0.150000in}}{\pgfqpoint{5.490039in}{5.490039in}}%
\pgfusepath{clip}%
\pgfsetbuttcap%
\pgfsetroundjoin%
\definecolor{currentfill}{rgb}{0.214298,0.355619,0.551184}%
\pgfsetfillcolor{currentfill}%
\pgfsetfillopacity{0.700000}%
\pgfsetlinewidth{0.000000pt}%
\definecolor{currentstroke}{rgb}{0.000000,0.000000,0.000000}%
\pgfsetstrokecolor{currentstroke}%
\pgfsetdash{}{0pt}%
\pgfpathmoveto{\pgfqpoint{2.404881in}{2.173506in}}%
\pgfpathlineto{\pgfqpoint{2.418620in}{2.153907in}}%
\pgfpathlineto{\pgfqpoint{2.432349in}{2.134551in}}%
\pgfpathlineto{\pgfqpoint{2.446068in}{2.115439in}}%
\pgfpathlineto{\pgfqpoint{2.459778in}{2.096566in}}%
\pgfpathlineto{\pgfqpoint{2.468722in}{2.089571in}}%
\pgfpathlineto{\pgfqpoint{2.477642in}{2.082968in}}%
\pgfpathlineto{\pgfqpoint{2.486537in}{2.076749in}}%
\pgfpathlineto{\pgfqpoint{2.495410in}{2.070907in}}%
\pgfpathlineto{\pgfqpoint{2.481760in}{2.089067in}}%
\pgfpathlineto{\pgfqpoint{2.468101in}{2.107466in}}%
\pgfpathlineto{\pgfqpoint{2.454432in}{2.126106in}}%
\pgfpathlineto{\pgfqpoint{2.440754in}{2.144990in}}%
\pgfpathlineto{\pgfqpoint{2.431823in}{2.151533in}}%
\pgfpathlineto{\pgfqpoint{2.422867in}{2.158462in}}%
\pgfpathlineto{\pgfqpoint{2.413887in}{2.165784in}}%
\pgfpathlineto{\pgfqpoint{2.404881in}{2.173506in}}%
\pgfpathclose%
\pgfusepath{fill}%
\end{pgfscope}%
\begin{pgfscope}%
\pgfpathrectangle{\pgfqpoint{1.254980in}{0.150000in}}{\pgfqpoint{5.490039in}{5.490039in}}%
\pgfusepath{clip}%
\pgfsetbuttcap%
\pgfsetroundjoin%
\definecolor{currentfill}{rgb}{0.137770,0.537492,0.554906}%
\pgfsetfillcolor{currentfill}%
\pgfsetfillopacity{0.700000}%
\pgfsetlinewidth{0.000000pt}%
\definecolor{currentstroke}{rgb}{0.000000,0.000000,0.000000}%
\pgfsetstrokecolor{currentstroke}%
\pgfsetdash{}{0pt}%
\pgfpathmoveto{\pgfqpoint{2.108971in}{2.676852in}}%
\pgfpathlineto{\pgfqpoint{2.122957in}{2.651451in}}%
\pgfpathlineto{\pgfqpoint{2.136927in}{2.626351in}}%
\pgfpathlineto{\pgfqpoint{2.150881in}{2.601550in}}%
\pgfpathlineto{\pgfqpoint{2.164819in}{2.577044in}}%
\pgfpathlineto{\pgfqpoint{2.174050in}{2.567330in}}%
\pgfpathlineto{\pgfqpoint{2.183253in}{2.558032in}}%
\pgfpathlineto{\pgfqpoint{2.192429in}{2.549143in}}%
\pgfpathlineto{\pgfqpoint{2.201577in}{2.540657in}}%
\pgfpathlineto{\pgfqpoint{2.187709in}{2.564448in}}%
\pgfpathlineto{\pgfqpoint{2.173825in}{2.588533in}}%
\pgfpathlineto{\pgfqpoint{2.159926in}{2.612916in}}%
\pgfpathlineto{\pgfqpoint{2.146012in}{2.637598in}}%
\pgfpathlineto{\pgfqpoint{2.136795in}{2.646787in}}%
\pgfpathlineto{\pgfqpoint{2.127549in}{2.656388in}}%
\pgfpathlineto{\pgfqpoint{2.118275in}{2.666407in}}%
\pgfpathlineto{\pgfqpoint{2.108971in}{2.676852in}}%
\pgfpathclose%
\pgfusepath{fill}%
\end{pgfscope}%
\begin{pgfscope}%
\pgfpathrectangle{\pgfqpoint{1.254980in}{0.150000in}}{\pgfqpoint{5.490039in}{5.490039in}}%
\pgfusepath{clip}%
\pgfsetbuttcap%
\pgfsetroundjoin%
\definecolor{currentfill}{rgb}{0.165117,0.467423,0.558141}%
\pgfsetfillcolor{currentfill}%
\pgfsetfillopacity{0.700000}%
\pgfsetlinewidth{0.000000pt}%
\definecolor{currentstroke}{rgb}{0.000000,0.000000,0.000000}%
\pgfsetstrokecolor{currentstroke}%
\pgfsetdash{}{0pt}%
\pgfpathmoveto{\pgfqpoint{4.665889in}{2.370143in}}%
\pgfpathlineto{\pgfqpoint{4.679786in}{2.380504in}}%
\pgfpathlineto{\pgfqpoint{4.693699in}{2.391027in}}%
\pgfpathlineto{\pgfqpoint{4.707628in}{2.401711in}}%
\pgfpathlineto{\pgfqpoint{4.721574in}{2.412557in}}%
\pgfpathlineto{\pgfqpoint{4.729237in}{2.424469in}}%
\pgfpathlineto{\pgfqpoint{4.736894in}{2.436247in}}%
\pgfpathlineto{\pgfqpoint{4.744544in}{2.447891in}}%
\pgfpathlineto{\pgfqpoint{4.752188in}{2.459399in}}%
\pgfpathlineto{\pgfqpoint{4.738242in}{2.448446in}}%
\pgfpathlineto{\pgfqpoint{4.724313in}{2.437654in}}%
\pgfpathlineto{\pgfqpoint{4.710400in}{2.427024in}}%
\pgfpathlineto{\pgfqpoint{4.696504in}{2.416556in}}%
\pgfpathlineto{\pgfqpoint{4.688859in}{2.405143in}}%
\pgfpathlineto{\pgfqpoint{4.681209in}{2.393603in}}%
\pgfpathlineto{\pgfqpoint{4.673552in}{2.381936in}}%
\pgfpathlineto{\pgfqpoint{4.665889in}{2.370143in}}%
\pgfpathclose%
\pgfusepath{fill}%
\end{pgfscope}%
\begin{pgfscope}%
\pgfpathrectangle{\pgfqpoint{1.254980in}{0.150000in}}{\pgfqpoint{5.490039in}{5.490039in}}%
\pgfusepath{clip}%
\pgfsetbuttcap%
\pgfsetroundjoin%
\definecolor{currentfill}{rgb}{0.281477,0.755203,0.432552}%
\pgfsetfillcolor{currentfill}%
\pgfsetfillopacity{0.700000}%
\pgfsetlinewidth{0.000000pt}%
\definecolor{currentstroke}{rgb}{0.000000,0.000000,0.000000}%
\pgfsetstrokecolor{currentstroke}%
\pgfsetdash{}{0pt}%
\pgfpathmoveto{\pgfqpoint{5.477520in}{3.185355in}}%
\pgfpathlineto{\pgfqpoint{5.491948in}{3.200184in}}%
\pgfpathlineto{\pgfqpoint{5.506397in}{3.215177in}}%
\pgfpathlineto{\pgfqpoint{5.520867in}{3.230334in}}%
\pgfpathlineto{\pgfqpoint{5.535359in}{3.245656in}}%
\pgfpathlineto{\pgfqpoint{5.542564in}{3.249681in}}%
\pgfpathlineto{\pgfqpoint{5.549759in}{3.253565in}}%
\pgfpathlineto{\pgfqpoint{5.556943in}{3.257311in}}%
\pgfpathlineto{\pgfqpoint{5.564117in}{3.260922in}}%
\pgfpathlineto{\pgfqpoint{5.549641in}{3.245868in}}%
\pgfpathlineto{\pgfqpoint{5.535187in}{3.230979in}}%
\pgfpathlineto{\pgfqpoint{5.520754in}{3.216253in}}%
\pgfpathlineto{\pgfqpoint{5.506342in}{3.201690in}}%
\pgfpathlineto{\pgfqpoint{5.499151in}{3.197800in}}%
\pgfpathlineto{\pgfqpoint{5.491951in}{3.193783in}}%
\pgfpathlineto{\pgfqpoint{5.484740in}{3.189636in}}%
\pgfpathlineto{\pgfqpoint{5.477520in}{3.185355in}}%
\pgfpathclose%
\pgfusepath{fill}%
\end{pgfscope}%
\begin{pgfscope}%
\pgfpathrectangle{\pgfqpoint{1.254980in}{0.150000in}}{\pgfqpoint{5.490039in}{5.490039in}}%
\pgfusepath{clip}%
\pgfsetbuttcap%
\pgfsetroundjoin%
\definecolor{currentfill}{rgb}{0.279574,0.170599,0.479997}%
\pgfsetfillcolor{currentfill}%
\pgfsetfillopacity{0.700000}%
\pgfsetlinewidth{0.000000pt}%
\definecolor{currentstroke}{rgb}{0.000000,0.000000,0.000000}%
\pgfsetstrokecolor{currentstroke}%
\pgfsetdash{}{0pt}%
\pgfpathmoveto{\pgfqpoint{4.026039in}{1.675866in}}%
\pgfpathlineto{\pgfqpoint{4.039616in}{1.679723in}}%
\pgfpathlineto{\pgfqpoint{4.053204in}{1.683740in}}%
\pgfpathlineto{\pgfqpoint{4.066802in}{1.687916in}}%
\pgfpathlineto{\pgfqpoint{4.080411in}{1.692251in}}%
\pgfpathlineto{\pgfqpoint{4.088266in}{1.705616in}}%
\pgfpathlineto{\pgfqpoint{4.096117in}{1.718980in}}%
\pgfpathlineto{\pgfqpoint{4.103964in}{1.732340in}}%
\pgfpathlineto{\pgfqpoint{4.111806in}{1.745691in}}%
\pgfpathlineto{\pgfqpoint{4.098199in}{1.740963in}}%
\pgfpathlineto{\pgfqpoint{4.084603in}{1.736395in}}%
\pgfpathlineto{\pgfqpoint{4.071018in}{1.731987in}}%
\pgfpathlineto{\pgfqpoint{4.057444in}{1.727738in}}%
\pgfpathlineto{\pgfqpoint{4.049599in}{1.714768in}}%
\pgfpathlineto{\pgfqpoint{4.041750in}{1.701797in}}%
\pgfpathlineto{\pgfqpoint{4.033897in}{1.688829in}}%
\pgfpathlineto{\pgfqpoint{4.026039in}{1.675866in}}%
\pgfpathclose%
\pgfusepath{fill}%
\end{pgfscope}%
\begin{pgfscope}%
\pgfpathrectangle{\pgfqpoint{1.254980in}{0.150000in}}{\pgfqpoint{5.490039in}{5.490039in}}%
\pgfusepath{clip}%
\pgfsetbuttcap%
\pgfsetroundjoin%
\definecolor{currentfill}{rgb}{0.268510,0.009605,0.335427}%
\pgfsetfillcolor{currentfill}%
\pgfsetfillopacity{0.700000}%
\pgfsetlinewidth{0.000000pt}%
\definecolor{currentstroke}{rgb}{0.000000,0.000000,0.000000}%
\pgfsetstrokecolor{currentstroke}%
\pgfsetdash{}{0pt}%
\pgfpathmoveto{\pgfqpoint{3.285114in}{1.406850in}}%
\pgfpathlineto{\pgfqpoint{3.298566in}{1.400699in}}%
\pgfpathlineto{\pgfqpoint{3.312020in}{1.394720in}}%
\pgfpathlineto{\pgfqpoint{3.325477in}{1.388911in}}%
\pgfpathlineto{\pgfqpoint{3.338937in}{1.383273in}}%
\pgfpathlineto{\pgfqpoint{3.347113in}{1.388710in}}%
\pgfpathlineto{\pgfqpoint{3.355279in}{1.394366in}}%
\pgfpathlineto{\pgfqpoint{3.363435in}{1.400233in}}%
\pgfpathlineto{\pgfqpoint{3.371581in}{1.406307in}}%
\pgfpathlineto{\pgfqpoint{3.358147in}{1.411334in}}%
\pgfpathlineto{\pgfqpoint{3.344715in}{1.416532in}}%
\pgfpathlineto{\pgfqpoint{3.331287in}{1.421900in}}%
\pgfpathlineto{\pgfqpoint{3.317861in}{1.427440in}}%
\pgfpathlineto{\pgfqpoint{3.309691in}{1.421966in}}%
\pgfpathlineto{\pgfqpoint{3.301509in}{1.416706in}}%
\pgfpathlineto{\pgfqpoint{3.293317in}{1.411665in}}%
\pgfpathlineto{\pgfqpoint{3.285114in}{1.406850in}}%
\pgfpathclose%
\pgfusepath{fill}%
\end{pgfscope}%
\begin{pgfscope}%
\pgfpathrectangle{\pgfqpoint{1.254980in}{0.150000in}}{\pgfqpoint{5.490039in}{5.490039in}}%
\pgfusepath{clip}%
\pgfsetbuttcap%
\pgfsetroundjoin%
\definecolor{currentfill}{rgb}{0.252194,0.269783,0.531579}%
\pgfsetfillcolor{currentfill}%
\pgfsetfillopacity{0.700000}%
\pgfsetlinewidth{0.000000pt}%
\definecolor{currentstroke}{rgb}{0.000000,0.000000,0.000000}%
\pgfsetstrokecolor{currentstroke}%
\pgfsetdash{}{0pt}%
\pgfpathmoveto{\pgfqpoint{4.228919in}{1.875111in}}%
\pgfpathlineto{\pgfqpoint{4.242583in}{1.881338in}}%
\pgfpathlineto{\pgfqpoint{4.256259in}{1.887724in}}%
\pgfpathlineto{\pgfqpoint{4.269948in}{1.894271in}}%
\pgfpathlineto{\pgfqpoint{4.283650in}{1.900976in}}%
\pgfpathlineto{\pgfqpoint{4.291452in}{1.914744in}}%
\pgfpathlineto{\pgfqpoint{4.299249in}{1.928458in}}%
\pgfpathlineto{\pgfqpoint{4.307043in}{1.942114in}}%
\pgfpathlineto{\pgfqpoint{4.314832in}{1.955711in}}%
\pgfpathlineto{\pgfqpoint{4.301130in}{1.948695in}}%
\pgfpathlineto{\pgfqpoint{4.287441in}{1.941839in}}%
\pgfpathlineto{\pgfqpoint{4.273764in}{1.935144in}}%
\pgfpathlineto{\pgfqpoint{4.260101in}{1.928608in}}%
\pgfpathlineto{\pgfqpoint{4.252312in}{1.915309in}}%
\pgfpathlineto{\pgfqpoint{4.244519in}{1.901959in}}%
\pgfpathlineto{\pgfqpoint{4.236721in}{1.888558in}}%
\pgfpathlineto{\pgfqpoint{4.228919in}{1.875111in}}%
\pgfpathclose%
\pgfusepath{fill}%
\end{pgfscope}%
\begin{pgfscope}%
\pgfpathrectangle{\pgfqpoint{1.254980in}{0.150000in}}{\pgfqpoint{5.490039in}{5.490039in}}%
\pgfusepath{clip}%
\pgfsetbuttcap%
\pgfsetroundjoin%
\definecolor{currentfill}{rgb}{0.283072,0.130895,0.449241}%
\pgfsetfillcolor{currentfill}%
\pgfsetfillopacity{0.700000}%
\pgfsetlinewidth{0.000000pt}%
\definecolor{currentstroke}{rgb}{0.000000,0.000000,0.000000}%
\pgfsetstrokecolor{currentstroke}%
\pgfsetdash{}{0pt}%
\pgfpathmoveto{\pgfqpoint{2.840743in}{1.658997in}}%
\pgfpathlineto{\pgfqpoint{2.854276in}{1.646401in}}%
\pgfpathlineto{\pgfqpoint{2.867805in}{1.634000in}}%
\pgfpathlineto{\pgfqpoint{2.881332in}{1.621794in}}%
\pgfpathlineto{\pgfqpoint{2.894856in}{1.609783in}}%
\pgfpathlineto{\pgfqpoint{2.903384in}{1.608444in}}%
\pgfpathlineto{\pgfqpoint{2.911894in}{1.607430in}}%
\pgfpathlineto{\pgfqpoint{2.920387in}{1.606733in}}%
\pgfpathlineto{\pgfqpoint{2.928864in}{1.606346in}}%
\pgfpathlineto{\pgfqpoint{2.915383in}{1.617676in}}%
\pgfpathlineto{\pgfqpoint{2.901900in}{1.629200in}}%
\pgfpathlineto{\pgfqpoint{2.888414in}{1.640918in}}%
\pgfpathlineto{\pgfqpoint{2.874925in}{1.652831in}}%
\pgfpathlineto{\pgfqpoint{2.866406in}{1.653888in}}%
\pgfpathlineto{\pgfqpoint{2.857870in}{1.655264in}}%
\pgfpathlineto{\pgfqpoint{2.849315in}{1.656964in}}%
\pgfpathlineto{\pgfqpoint{2.840743in}{1.658997in}}%
\pgfpathclose%
\pgfusepath{fill}%
\end{pgfscope}%
\begin{pgfscope}%
\pgfpathrectangle{\pgfqpoint{1.254980in}{0.150000in}}{\pgfqpoint{5.490039in}{5.490039in}}%
\pgfusepath{clip}%
\pgfsetbuttcap%
\pgfsetroundjoin%
\definecolor{currentfill}{rgb}{0.199430,0.387607,0.554642}%
\pgfsetfillcolor{currentfill}%
\pgfsetfillopacity{0.700000}%
\pgfsetlinewidth{0.000000pt}%
\definecolor{currentstroke}{rgb}{0.000000,0.000000,0.000000}%
\pgfsetstrokecolor{currentstroke}%
\pgfsetdash{}{0pt}%
\pgfpathmoveto{\pgfqpoint{2.349821in}{2.254392in}}%
\pgfpathlineto{\pgfqpoint{2.363603in}{2.233794in}}%
\pgfpathlineto{\pgfqpoint{2.377373in}{2.213448in}}%
\pgfpathlineto{\pgfqpoint{2.391132in}{2.193353in}}%
\pgfpathlineto{\pgfqpoint{2.404881in}{2.173506in}}%
\pgfpathlineto{\pgfqpoint{2.413887in}{2.165784in}}%
\pgfpathlineto{\pgfqpoint{2.422867in}{2.158462in}}%
\pgfpathlineto{\pgfqpoint{2.431823in}{2.151533in}}%
\pgfpathlineto{\pgfqpoint{2.440754in}{2.144990in}}%
\pgfpathlineto{\pgfqpoint{2.427067in}{2.164119in}}%
\pgfpathlineto{\pgfqpoint{2.413370in}{2.183495in}}%
\pgfpathlineto{\pgfqpoint{2.399663in}{2.203120in}}%
\pgfpathlineto{\pgfqpoint{2.385945in}{2.222996in}}%
\pgfpathlineto{\pgfqpoint{2.376952in}{2.230246in}}%
\pgfpathlineto{\pgfqpoint{2.367934in}{2.237890in}}%
\pgfpathlineto{\pgfqpoint{2.358891in}{2.245936in}}%
\pgfpathlineto{\pgfqpoint{2.349821in}{2.254392in}}%
\pgfpathclose%
\pgfusepath{fill}%
\end{pgfscope}%
\begin{pgfscope}%
\pgfpathrectangle{\pgfqpoint{1.254980in}{0.150000in}}{\pgfqpoint{5.490039in}{5.490039in}}%
\pgfusepath{clip}%
\pgfsetbuttcap%
\pgfsetroundjoin%
\definecolor{currentfill}{rgb}{0.267004,0.004874,0.329415}%
\pgfsetfillcolor{currentfill}%
\pgfsetfillopacity{0.700000}%
\pgfsetlinewidth{0.000000pt}%
\definecolor{currentstroke}{rgb}{0.000000,0.000000,0.000000}%
\pgfsetstrokecolor{currentstroke}%
\pgfsetdash{}{0pt}%
\pgfpathmoveto{\pgfqpoint{3.425355in}{1.387888in}}%
\pgfpathlineto{\pgfqpoint{3.438808in}{1.383703in}}%
\pgfpathlineto{\pgfqpoint{3.452265in}{1.379685in}}%
\pgfpathlineto{\pgfqpoint{3.465726in}{1.375834in}}%
\pgfpathlineto{\pgfqpoint{3.479192in}{1.372148in}}%
\pgfpathlineto{\pgfqpoint{3.487284in}{1.379606in}}%
\pgfpathlineto{\pgfqpoint{3.495368in}{1.387244in}}%
\pgfpathlineto{\pgfqpoint{3.503443in}{1.395056in}}%
\pgfpathlineto{\pgfqpoint{3.511510in}{1.403036in}}%
\pgfpathlineto{\pgfqpoint{3.498064in}{1.406140in}}%
\pgfpathlineto{\pgfqpoint{3.484623in}{1.409410in}}%
\pgfpathlineto{\pgfqpoint{3.471187in}{1.412846in}}%
\pgfpathlineto{\pgfqpoint{3.457755in}{1.416449in}}%
\pgfpathlineto{\pgfqpoint{3.449668in}{1.409040in}}%
\pgfpathlineto{\pgfqpoint{3.441573in}{1.401806in}}%
\pgfpathlineto{\pgfqpoint{3.433468in}{1.394754in}}%
\pgfpathlineto{\pgfqpoint{3.425355in}{1.387888in}}%
\pgfpathclose%
\pgfusepath{fill}%
\end{pgfscope}%
\begin{pgfscope}%
\pgfpathrectangle{\pgfqpoint{1.254980in}{0.150000in}}{\pgfqpoint{5.490039in}{5.490039in}}%
\pgfusepath{clip}%
\pgfsetbuttcap%
\pgfsetroundjoin%
\definecolor{currentfill}{rgb}{0.185556,0.418570,0.556753}%
\pgfsetfillcolor{currentfill}%
\pgfsetfillopacity{0.700000}%
\pgfsetlinewidth{0.000000pt}%
\definecolor{currentstroke}{rgb}{0.000000,0.000000,0.000000}%
\pgfsetstrokecolor{currentstroke}%
\pgfsetdash{}{0pt}%
\pgfpathmoveto{\pgfqpoint{4.548953in}{2.232737in}}%
\pgfpathlineto{\pgfqpoint{4.562786in}{2.242150in}}%
\pgfpathlineto{\pgfqpoint{4.576634in}{2.251724in}}%
\pgfpathlineto{\pgfqpoint{4.590498in}{2.261459in}}%
\pgfpathlineto{\pgfqpoint{4.604377in}{2.271356in}}%
\pgfpathlineto{\pgfqpoint{4.612086in}{2.284125in}}%
\pgfpathlineto{\pgfqpoint{4.619789in}{2.296777in}}%
\pgfpathlineto{\pgfqpoint{4.627487in}{2.309310in}}%
\pgfpathlineto{\pgfqpoint{4.635179in}{2.321722in}}%
\pgfpathlineto{\pgfqpoint{4.621299in}{2.311659in}}%
\pgfpathlineto{\pgfqpoint{4.607434in}{2.301757in}}%
\pgfpathlineto{\pgfqpoint{4.593585in}{2.292017in}}%
\pgfpathlineto{\pgfqpoint{4.579751in}{2.282437in}}%
\pgfpathlineto{\pgfqpoint{4.572060in}{2.270180in}}%
\pgfpathlineto{\pgfqpoint{4.564363in}{2.257810in}}%
\pgfpathlineto{\pgfqpoint{4.556661in}{2.245329in}}%
\pgfpathlineto{\pgfqpoint{4.548953in}{2.232737in}}%
\pgfpathclose%
\pgfusepath{fill}%
\end{pgfscope}%
\begin{pgfscope}%
\pgfpathrectangle{\pgfqpoint{1.254980in}{0.150000in}}{\pgfqpoint{5.490039in}{5.490039in}}%
\pgfusepath{clip}%
\pgfsetbuttcap%
\pgfsetroundjoin%
\definecolor{currentfill}{rgb}{0.132444,0.552216,0.553018}%
\pgfsetfillcolor{currentfill}%
\pgfsetfillopacity{0.700000}%
\pgfsetlinewidth{0.000000pt}%
\definecolor{currentstroke}{rgb}{0.000000,0.000000,0.000000}%
\pgfsetstrokecolor{currentstroke}%
\pgfsetdash{}{0pt}%
\pgfpathmoveto{\pgfqpoint{4.869048in}{2.592474in}}%
\pgfpathlineto{\pgfqpoint{4.883078in}{2.604362in}}%
\pgfpathlineto{\pgfqpoint{4.897124in}{2.616412in}}%
\pgfpathlineto{\pgfqpoint{4.911189in}{2.628625in}}%
\pgfpathlineto{\pgfqpoint{4.925272in}{2.641000in}}%
\pgfpathlineto{\pgfqpoint{4.932852in}{2.651331in}}%
\pgfpathlineto{\pgfqpoint{4.940425in}{2.661510in}}%
\pgfpathlineto{\pgfqpoint{4.947990in}{2.671536in}}%
\pgfpathlineto{\pgfqpoint{4.955548in}{2.681410in}}%
\pgfpathlineto{\pgfqpoint{4.941467in}{2.669018in}}%
\pgfpathlineto{\pgfqpoint{4.927405in}{2.656789in}}%
\pgfpathlineto{\pgfqpoint{4.913360in}{2.644722in}}%
\pgfpathlineto{\pgfqpoint{4.899333in}{2.632818in}}%
\pgfpathlineto{\pgfqpoint{4.891773in}{2.622949in}}%
\pgfpathlineto{\pgfqpoint{4.884205in}{2.612936in}}%
\pgfpathlineto{\pgfqpoint{4.876630in}{2.602778in}}%
\pgfpathlineto{\pgfqpoint{4.869048in}{2.592474in}}%
\pgfpathclose%
\pgfusepath{fill}%
\end{pgfscope}%
\begin{pgfscope}%
\pgfpathrectangle{\pgfqpoint{1.254980in}{0.150000in}}{\pgfqpoint{5.490039in}{5.490039in}}%
\pgfusepath{clip}%
\pgfsetbuttcap%
\pgfsetroundjoin%
\definecolor{currentfill}{rgb}{0.273809,0.031497,0.358853}%
\pgfsetfillcolor{currentfill}%
\pgfsetfillopacity{0.700000}%
\pgfsetlinewidth{0.000000pt}%
\definecolor{currentstroke}{rgb}{0.000000,0.000000,0.000000}%
\pgfsetstrokecolor{currentstroke}%
\pgfsetdash{}{0pt}%
\pgfpathmoveto{\pgfqpoint{3.144400in}{1.450557in}}%
\pgfpathlineto{\pgfqpoint{3.157870in}{1.442368in}}%
\pgfpathlineto{\pgfqpoint{3.171340in}{1.434357in}}%
\pgfpathlineto{\pgfqpoint{3.184812in}{1.426521in}}%
\pgfpathlineto{\pgfqpoint{3.198284in}{1.418862in}}%
\pgfpathlineto{\pgfqpoint{3.206564in}{1.422061in}}%
\pgfpathlineto{\pgfqpoint{3.214831in}{1.425518in}}%
\pgfpathlineto{\pgfqpoint{3.223086in}{1.429226in}}%
\pgfpathlineto{\pgfqpoint{3.231329in}{1.433180in}}%
\pgfpathlineto{\pgfqpoint{3.217888in}{1.440198in}}%
\pgfpathlineto{\pgfqpoint{3.204447in}{1.447392in}}%
\pgfpathlineto{\pgfqpoint{3.191009in}{1.454762in}}%
\pgfpathlineto{\pgfqpoint{3.177572in}{1.462308in}}%
\pgfpathlineto{\pgfqpoint{3.169298in}{1.458985in}}%
\pgfpathlineto{\pgfqpoint{3.161012in}{1.455914in}}%
\pgfpathlineto{\pgfqpoint{3.152713in}{1.453103in}}%
\pgfpathlineto{\pgfqpoint{3.144400in}{1.450557in}}%
\pgfpathclose%
\pgfusepath{fill}%
\end{pgfscope}%
\begin{pgfscope}%
\pgfpathrectangle{\pgfqpoint{1.254980in}{0.150000in}}{\pgfqpoint{5.490039in}{5.490039in}}%
\pgfusepath{clip}%
\pgfsetbuttcap%
\pgfsetroundjoin%
\definecolor{currentfill}{rgb}{0.277941,0.056324,0.381191}%
\pgfsetfillcolor{currentfill}%
\pgfsetfillopacity{0.700000}%
\pgfsetlinewidth{0.000000pt}%
\definecolor{currentstroke}{rgb}{0.000000,0.000000,0.000000}%
\pgfsetstrokecolor{currentstroke}%
\pgfsetdash{}{0pt}%
\pgfpathmoveto{\pgfqpoint{3.737192in}{1.460789in}}%
\pgfpathlineto{\pgfqpoint{3.750688in}{1.460887in}}%
\pgfpathlineto{\pgfqpoint{3.764193in}{1.461146in}}%
\pgfpathlineto{\pgfqpoint{3.777705in}{1.461566in}}%
\pgfpathlineto{\pgfqpoint{3.791225in}{1.462146in}}%
\pgfpathlineto{\pgfqpoint{3.799174in}{1.473435in}}%
\pgfpathlineto{\pgfqpoint{3.807118in}{1.484812in}}%
\pgfpathlineto{\pgfqpoint{3.815057in}{1.496272in}}%
\pgfpathlineto{\pgfqpoint{3.822990in}{1.507810in}}%
\pgfpathlineto{\pgfqpoint{3.809479in}{1.506729in}}%
\pgfpathlineto{\pgfqpoint{3.795977in}{1.505810in}}%
\pgfpathlineto{\pgfqpoint{3.782482in}{1.505051in}}%
\pgfpathlineto{\pgfqpoint{3.768996in}{1.504453in}}%
\pgfpathlineto{\pgfqpoint{3.761054in}{1.493404in}}%
\pgfpathlineto{\pgfqpoint{3.753105in}{1.482440in}}%
\pgfpathlineto{\pgfqpoint{3.745151in}{1.471567in}}%
\pgfpathlineto{\pgfqpoint{3.737192in}{1.460789in}}%
\pgfpathclose%
\pgfusepath{fill}%
\end{pgfscope}%
\begin{pgfscope}%
\pgfpathrectangle{\pgfqpoint{1.254980in}{0.150000in}}{\pgfqpoint{5.490039in}{5.490039in}}%
\pgfusepath{clip}%
\pgfsetbuttcap%
\pgfsetroundjoin%
\definecolor{currentfill}{rgb}{0.283091,0.110553,0.431554}%
\pgfsetfillcolor{currentfill}%
\pgfsetfillopacity{0.700000}%
\pgfsetlinewidth{0.000000pt}%
\definecolor{currentstroke}{rgb}{0.000000,0.000000,0.000000}%
\pgfsetstrokecolor{currentstroke}%
\pgfsetdash{}{0pt}%
\pgfpathmoveto{\pgfqpoint{2.894856in}{1.609783in}}%
\pgfpathlineto{\pgfqpoint{2.908378in}{1.597964in}}%
\pgfpathlineto{\pgfqpoint{2.921897in}{1.586337in}}%
\pgfpathlineto{\pgfqpoint{2.935414in}{1.574901in}}%
\pgfpathlineto{\pgfqpoint{2.948929in}{1.563655in}}%
\pgfpathlineto{\pgfqpoint{2.957413in}{1.563008in}}%
\pgfpathlineto{\pgfqpoint{2.965881in}{1.562678in}}%
\pgfpathlineto{\pgfqpoint{2.974333in}{1.562657in}}%
\pgfpathlineto{\pgfqpoint{2.982770in}{1.562938in}}%
\pgfpathlineto{\pgfqpoint{2.969296in}{1.573505in}}%
\pgfpathlineto{\pgfqpoint{2.955820in}{1.584261in}}%
\pgfpathlineto{\pgfqpoint{2.942343in}{1.595208in}}%
\pgfpathlineto{\pgfqpoint{2.928864in}{1.606346in}}%
\pgfpathlineto{\pgfqpoint{2.920387in}{1.606733in}}%
\pgfpathlineto{\pgfqpoint{2.911894in}{1.607430in}}%
\pgfpathlineto{\pgfqpoint{2.903384in}{1.608444in}}%
\pgfpathlineto{\pgfqpoint{2.894856in}{1.609783in}}%
\pgfpathclose%
\pgfusepath{fill}%
\end{pgfscope}%
\begin{pgfscope}%
\pgfpathrectangle{\pgfqpoint{1.254980in}{0.150000in}}{\pgfqpoint{5.490039in}{5.490039in}}%
\pgfusepath{clip}%
\pgfsetbuttcap%
\pgfsetroundjoin%
\definecolor{currentfill}{rgb}{0.180653,0.701402,0.488189}%
\pgfsetfillcolor{currentfill}%
\pgfsetfillopacity{0.700000}%
\pgfsetlinewidth{0.000000pt}%
\definecolor{currentstroke}{rgb}{0.000000,0.000000,0.000000}%
\pgfsetstrokecolor{currentstroke}%
\pgfsetdash{}{0pt}%
\pgfpathmoveto{\pgfqpoint{5.275121in}{3.004937in}}%
\pgfpathlineto{\pgfqpoint{5.289421in}{3.019066in}}%
\pgfpathlineto{\pgfqpoint{5.303741in}{3.033358in}}%
\pgfpathlineto{\pgfqpoint{5.318080in}{3.047815in}}%
\pgfpathlineto{\pgfqpoint{5.332441in}{3.062435in}}%
\pgfpathlineto{\pgfqpoint{5.339792in}{3.068690in}}%
\pgfpathlineto{\pgfqpoint{5.347134in}{3.074789in}}%
\pgfpathlineto{\pgfqpoint{5.354466in}{3.080733in}}%
\pgfpathlineto{\pgfqpoint{5.361788in}{3.086525in}}%
\pgfpathlineto{\pgfqpoint{5.347438in}{3.072077in}}%
\pgfpathlineto{\pgfqpoint{5.333109in}{3.057792in}}%
\pgfpathlineto{\pgfqpoint{5.318799in}{3.043670in}}%
\pgfpathlineto{\pgfqpoint{5.304510in}{3.029712in}}%
\pgfpathlineto{\pgfqpoint{5.297177in}{3.023737in}}%
\pgfpathlineto{\pgfqpoint{5.289834in}{3.017618in}}%
\pgfpathlineto{\pgfqpoint{5.282482in}{3.011352in}}%
\pgfpathlineto{\pgfqpoint{5.275121in}{3.004937in}}%
\pgfpathclose%
\pgfusepath{fill}%
\end{pgfscope}%
\begin{pgfscope}%
\pgfpathrectangle{\pgfqpoint{1.254980in}{0.150000in}}{\pgfqpoint{5.490039in}{5.490039in}}%
\pgfusepath{clip}%
\pgfsetbuttcap%
\pgfsetroundjoin%
\definecolor{currentfill}{rgb}{0.273809,0.031497,0.358853}%
\pgfsetfillcolor{currentfill}%
\pgfsetfillopacity{0.700000}%
\pgfsetlinewidth{0.000000pt}%
\definecolor{currentstroke}{rgb}{0.000000,0.000000,0.000000}%
\pgfsetstrokecolor{currentstroke}%
\pgfsetdash{}{0pt}%
\pgfpathmoveto{\pgfqpoint{3.651327in}{1.422052in}}%
\pgfpathlineto{\pgfqpoint{3.664808in}{1.420977in}}%
\pgfpathlineto{\pgfqpoint{3.678296in}{1.420064in}}%
\pgfpathlineto{\pgfqpoint{3.691790in}{1.419313in}}%
\pgfpathlineto{\pgfqpoint{3.705292in}{1.418723in}}%
\pgfpathlineto{\pgfqpoint{3.713276in}{1.429072in}}%
\pgfpathlineto{\pgfqpoint{3.721254in}{1.439536in}}%
\pgfpathlineto{\pgfqpoint{3.729226in}{1.450110in}}%
\pgfpathlineto{\pgfqpoint{3.737192in}{1.460789in}}%
\pgfpathlineto{\pgfqpoint{3.723702in}{1.460852in}}%
\pgfpathlineto{\pgfqpoint{3.710220in}{1.461076in}}%
\pgfpathlineto{\pgfqpoint{3.696745in}{1.461462in}}%
\pgfpathlineto{\pgfqpoint{3.683277in}{1.462011in}}%
\pgfpathlineto{\pgfqpoint{3.675299in}{1.451849in}}%
\pgfpathlineto{\pgfqpoint{3.667315in}{1.441798in}}%
\pgfpathlineto{\pgfqpoint{3.659324in}{1.431864in}}%
\pgfpathlineto{\pgfqpoint{3.651327in}{1.422052in}}%
\pgfpathclose%
\pgfusepath{fill}%
\end{pgfscope}%
\begin{pgfscope}%
\pgfpathrectangle{\pgfqpoint{1.254980in}{0.150000in}}{\pgfqpoint{5.490039in}{5.490039in}}%
\pgfusepath{clip}%
\pgfsetbuttcap%
\pgfsetroundjoin%
\definecolor{currentfill}{rgb}{0.183898,0.422383,0.556944}%
\pgfsetfillcolor{currentfill}%
\pgfsetfillopacity{0.700000}%
\pgfsetlinewidth{0.000000pt}%
\definecolor{currentstroke}{rgb}{0.000000,0.000000,0.000000}%
\pgfsetstrokecolor{currentstroke}%
\pgfsetdash{}{0pt}%
\pgfpathmoveto{\pgfqpoint{2.294581in}{2.339358in}}%
\pgfpathlineto{\pgfqpoint{2.308409in}{2.317726in}}%
\pgfpathlineto{\pgfqpoint{2.322225in}{2.296356in}}%
\pgfpathlineto{\pgfqpoint{2.336029in}{2.275246in}}%
\pgfpathlineto{\pgfqpoint{2.349821in}{2.254392in}}%
\pgfpathlineto{\pgfqpoint{2.358891in}{2.245936in}}%
\pgfpathlineto{\pgfqpoint{2.367934in}{2.237890in}}%
\pgfpathlineto{\pgfqpoint{2.376952in}{2.230246in}}%
\pgfpathlineto{\pgfqpoint{2.385945in}{2.222996in}}%
\pgfpathlineto{\pgfqpoint{2.372216in}{2.243126in}}%
\pgfpathlineto{\pgfqpoint{2.358477in}{2.263511in}}%
\pgfpathlineto{\pgfqpoint{2.344726in}{2.284155in}}%
\pgfpathlineto{\pgfqpoint{2.330964in}{2.305058in}}%
\pgfpathlineto{\pgfqpoint{2.321908in}{2.313019in}}%
\pgfpathlineto{\pgfqpoint{2.312826in}{2.321385in}}%
\pgfpathlineto{\pgfqpoint{2.303717in}{2.330162in}}%
\pgfpathlineto{\pgfqpoint{2.294581in}{2.339358in}}%
\pgfpathclose%
\pgfusepath{fill}%
\end{pgfscope}%
\begin{pgfscope}%
\pgfpathrectangle{\pgfqpoint{1.254980in}{0.150000in}}{\pgfqpoint{5.490039in}{5.490039in}}%
\pgfusepath{clip}%
\pgfsetbuttcap%
\pgfsetroundjoin%
\definecolor{currentfill}{rgb}{0.281446,0.084320,0.407414}%
\pgfsetfillcolor{currentfill}%
\pgfsetfillopacity{0.700000}%
\pgfsetlinewidth{0.000000pt}%
\definecolor{currentstroke}{rgb}{0.000000,0.000000,0.000000}%
\pgfsetstrokecolor{currentstroke}%
\pgfsetdash{}{0pt}%
\pgfpathmoveto{\pgfqpoint{3.822990in}{1.507810in}}%
\pgfpathlineto{\pgfqpoint{3.836509in}{1.509051in}}%
\pgfpathlineto{\pgfqpoint{3.850036in}{1.510452in}}%
\pgfpathlineto{\pgfqpoint{3.863572in}{1.512013in}}%
\pgfpathlineto{\pgfqpoint{3.877117in}{1.513733in}}%
\pgfpathlineto{\pgfqpoint{3.885037in}{1.525829in}}%
\pgfpathlineto{\pgfqpoint{3.892952in}{1.537986in}}%
\pgfpathlineto{\pgfqpoint{3.900861in}{1.550202in}}%
\pgfpathlineto{\pgfqpoint{3.908766in}{1.562471in}}%
\pgfpathlineto{\pgfqpoint{3.895228in}{1.560277in}}%
\pgfpathlineto{\pgfqpoint{3.881700in}{1.558243in}}%
\pgfpathlineto{\pgfqpoint{3.868180in}{1.556369in}}%
\pgfpathlineto{\pgfqpoint{3.854669in}{1.554656in}}%
\pgfpathlineto{\pgfqpoint{3.846757in}{1.542849in}}%
\pgfpathlineto{\pgfqpoint{3.838840in}{1.531103in}}%
\pgfpathlineto{\pgfqpoint{3.830917in}{1.519422in}}%
\pgfpathlineto{\pgfqpoint{3.822990in}{1.507810in}}%
\pgfpathclose%
\pgfusepath{fill}%
\end{pgfscope}%
\begin{pgfscope}%
\pgfpathrectangle{\pgfqpoint{1.254980in}{0.150000in}}{\pgfqpoint{5.490039in}{5.490039in}}%
\pgfusepath{clip}%
\pgfsetbuttcap%
\pgfsetroundjoin%
\definecolor{currentfill}{rgb}{0.210503,0.363727,0.552206}%
\pgfsetfillcolor{currentfill}%
\pgfsetfillopacity{0.700000}%
\pgfsetlinewidth{0.000000pt}%
\definecolor{currentstroke}{rgb}{0.000000,0.000000,0.000000}%
\pgfsetstrokecolor{currentstroke}%
\pgfsetdash{}{0pt}%
\pgfpathmoveto{\pgfqpoint{4.431927in}{2.093836in}}%
\pgfpathlineto{\pgfqpoint{4.445699in}{2.102187in}}%
\pgfpathlineto{\pgfqpoint{4.459485in}{2.110698in}}%
\pgfpathlineto{\pgfqpoint{4.473286in}{2.119369in}}%
\pgfpathlineto{\pgfqpoint{4.487101in}{2.128201in}}%
\pgfpathlineto{\pgfqpoint{4.494850in}{2.141621in}}%
\pgfpathlineto{\pgfqpoint{4.502595in}{2.154944in}}%
\pgfpathlineto{\pgfqpoint{4.510334in}{2.168168in}}%
\pgfpathlineto{\pgfqpoint{4.518068in}{2.181291in}}%
\pgfpathlineto{\pgfqpoint{4.504251in}{2.172234in}}%
\pgfpathlineto{\pgfqpoint{4.490449in}{2.163338in}}%
\pgfpathlineto{\pgfqpoint{4.476662in}{2.154602in}}%
\pgfpathlineto{\pgfqpoint{4.462889in}{2.146027in}}%
\pgfpathlineto{\pgfqpoint{4.455156in}{2.133118in}}%
\pgfpathlineto{\pgfqpoint{4.447418in}{2.120115in}}%
\pgfpathlineto{\pgfqpoint{4.439675in}{2.107021in}}%
\pgfpathlineto{\pgfqpoint{4.431927in}{2.093836in}}%
\pgfpathclose%
\pgfusepath{fill}%
\end{pgfscope}%
\begin{pgfscope}%
\pgfpathrectangle{\pgfqpoint{1.254980in}{0.150000in}}{\pgfqpoint{5.490039in}{5.490039in}}%
\pgfusepath{clip}%
\pgfsetbuttcap%
\pgfsetroundjoin%
\definecolor{currentfill}{rgb}{0.121380,0.629492,0.531973}%
\pgfsetfillcolor{currentfill}%
\pgfsetfillopacity{0.700000}%
\pgfsetlinewidth{0.000000pt}%
\definecolor{currentstroke}{rgb}{0.000000,0.000000,0.000000}%
\pgfsetstrokecolor{currentstroke}%
\pgfsetdash{}{0pt}%
\pgfpathmoveto{\pgfqpoint{5.072204in}{2.805945in}}%
\pgfpathlineto{\pgfqpoint{5.086369in}{2.819091in}}%
\pgfpathlineto{\pgfqpoint{5.100554in}{2.832401in}}%
\pgfpathlineto{\pgfqpoint{5.114757in}{2.845873in}}%
\pgfpathlineto{\pgfqpoint{5.128979in}{2.859510in}}%
\pgfpathlineto{\pgfqpoint{5.136456in}{2.867912in}}%
\pgfpathlineto{\pgfqpoint{5.143924in}{2.876153in}}%
\pgfpathlineto{\pgfqpoint{5.151384in}{2.884236in}}%
\pgfpathlineto{\pgfqpoint{5.158834in}{2.892161in}}%
\pgfpathlineto{\pgfqpoint{5.144618in}{2.878602in}}%
\pgfpathlineto{\pgfqpoint{5.130420in}{2.865206in}}%
\pgfpathlineto{\pgfqpoint{5.116241in}{2.851973in}}%
\pgfpathlineto{\pgfqpoint{5.102082in}{2.838903in}}%
\pgfpathlineto{\pgfqpoint{5.094625in}{2.830890in}}%
\pgfpathlineto{\pgfqpoint{5.087160in}{2.822727in}}%
\pgfpathlineto{\pgfqpoint{5.079686in}{2.814412in}}%
\pgfpathlineto{\pgfqpoint{5.072204in}{2.805945in}}%
\pgfpathclose%
\pgfusepath{fill}%
\end{pgfscope}%
\begin{pgfscope}%
\pgfpathrectangle{\pgfqpoint{1.254980in}{0.150000in}}{\pgfqpoint{5.490039in}{5.490039in}}%
\pgfusepath{clip}%
\pgfsetbuttcap%
\pgfsetroundjoin%
\definecolor{currentfill}{rgb}{0.271828,0.209303,0.504434}%
\pgfsetfillcolor{currentfill}%
\pgfsetfillopacity{0.700000}%
\pgfsetlinewidth{0.000000pt}%
\definecolor{currentstroke}{rgb}{0.000000,0.000000,0.000000}%
\pgfsetstrokecolor{currentstroke}%
\pgfsetdash{}{0pt}%
\pgfpathmoveto{\pgfqpoint{4.111806in}{1.745691in}}%
\pgfpathlineto{\pgfqpoint{4.125424in}{1.750578in}}%
\pgfpathlineto{\pgfqpoint{4.139054in}{1.755624in}}%
\pgfpathlineto{\pgfqpoint{4.152695in}{1.760830in}}%
\pgfpathlineto{\pgfqpoint{4.166347in}{1.766194in}}%
\pgfpathlineto{\pgfqpoint{4.174184in}{1.779910in}}%
\pgfpathlineto{\pgfqpoint{4.182016in}{1.793602in}}%
\pgfpathlineto{\pgfqpoint{4.189844in}{1.807270in}}%
\pgfpathlineto{\pgfqpoint{4.197668in}{1.820908in}}%
\pgfpathlineto{\pgfqpoint{4.184016in}{1.815178in}}%
\pgfpathlineto{\pgfqpoint{4.170376in}{1.809607in}}%
\pgfpathlineto{\pgfqpoint{4.156748in}{1.804196in}}%
\pgfpathlineto{\pgfqpoint{4.143131in}{1.798945in}}%
\pgfpathlineto{\pgfqpoint{4.135306in}{1.785660in}}%
\pgfpathlineto{\pgfqpoint{4.127477in}{1.772355in}}%
\pgfpathlineto{\pgfqpoint{4.119644in}{1.759030in}}%
\pgfpathlineto{\pgfqpoint{4.111806in}{1.745691in}}%
\pgfpathclose%
\pgfusepath{fill}%
\end{pgfscope}%
\begin{pgfscope}%
\pgfpathrectangle{\pgfqpoint{1.254980in}{0.150000in}}{\pgfqpoint{5.490039in}{5.490039in}}%
\pgfusepath{clip}%
\pgfsetbuttcap%
\pgfsetroundjoin%
\definecolor{currentfill}{rgb}{0.352360,0.783011,0.392636}%
\pgfsetfillcolor{currentfill}%
\pgfsetfillopacity{0.700000}%
\pgfsetlinewidth{0.000000pt}%
\definecolor{currentstroke}{rgb}{0.000000,0.000000,0.000000}%
\pgfsetstrokecolor{currentstroke}%
\pgfsetdash{}{0pt}%
\pgfpathmoveto{\pgfqpoint{5.564117in}{3.260922in}}%
\pgfpathlineto{\pgfqpoint{5.578614in}{3.276139in}}%
\pgfpathlineto{\pgfqpoint{5.593133in}{3.291520in}}%
\pgfpathlineto{\pgfqpoint{5.607673in}{3.307066in}}%
\pgfpathlineto{\pgfqpoint{5.622236in}{3.322777in}}%
\pgfpathlineto{\pgfqpoint{5.629382in}{3.325967in}}%
\pgfpathlineto{\pgfqpoint{5.636518in}{3.329022in}}%
\pgfpathlineto{\pgfqpoint{5.643643in}{3.331944in}}%
\pgfpathlineto{\pgfqpoint{5.650757in}{3.334737in}}%
\pgfpathlineto{\pgfqpoint{5.636213in}{3.319328in}}%
\pgfpathlineto{\pgfqpoint{5.621691in}{3.304083in}}%
\pgfpathlineto{\pgfqpoint{5.607190in}{3.289001in}}%
\pgfpathlineto{\pgfqpoint{5.592711in}{3.274082in}}%
\pgfpathlineto{\pgfqpoint{5.585578in}{3.270978in}}%
\pgfpathlineto{\pgfqpoint{5.578434in}{3.267752in}}%
\pgfpathlineto{\pgfqpoint{5.571281in}{3.264401in}}%
\pgfpathlineto{\pgfqpoint{5.564117in}{3.260922in}}%
\pgfpathclose%
\pgfusepath{fill}%
\end{pgfscope}%
\begin{pgfscope}%
\pgfpathrectangle{\pgfqpoint{1.254980in}{0.150000in}}{\pgfqpoint{5.490039in}{5.490039in}}%
\pgfusepath{clip}%
\pgfsetbuttcap%
\pgfsetroundjoin%
\definecolor{currentfill}{rgb}{0.124395,0.578002,0.548287}%
\pgfsetfillcolor{currentfill}%
\pgfsetfillopacity{0.700000}%
\pgfsetlinewidth{0.000000pt}%
\definecolor{currentstroke}{rgb}{0.000000,0.000000,0.000000}%
\pgfsetstrokecolor{currentstroke}%
\pgfsetdash{}{0pt}%
\pgfpathmoveto{\pgfqpoint{2.052856in}{2.781537in}}%
\pgfpathlineto{\pgfqpoint{2.066911in}{2.754898in}}%
\pgfpathlineto{\pgfqpoint{2.080948in}{2.728573in}}%
\pgfpathlineto{\pgfqpoint{2.094968in}{2.702559in}}%
\pgfpathlineto{\pgfqpoint{2.108971in}{2.676852in}}%
\pgfpathlineto{\pgfqpoint{2.118275in}{2.666407in}}%
\pgfpathlineto{\pgfqpoint{2.127549in}{2.656388in}}%
\pgfpathlineto{\pgfqpoint{2.136795in}{2.646787in}}%
\pgfpathlineto{\pgfqpoint{2.146012in}{2.637598in}}%
\pgfpathlineto{\pgfqpoint{2.132081in}{2.662582in}}%
\pgfpathlineto{\pgfqpoint{2.118134in}{2.687872in}}%
\pgfpathlineto{\pgfqpoint{2.104170in}{2.713472in}}%
\pgfpathlineto{\pgfqpoint{2.090189in}{2.739383in}}%
\pgfpathlineto{\pgfqpoint{2.080900in}{2.749282in}}%
\pgfpathlineto{\pgfqpoint{2.071582in}{2.759603in}}%
\pgfpathlineto{\pgfqpoint{2.062234in}{2.770352in}}%
\pgfpathlineto{\pgfqpoint{2.052856in}{2.781537in}}%
\pgfpathclose%
\pgfusepath{fill}%
\end{pgfscope}%
\begin{pgfscope}%
\pgfpathrectangle{\pgfqpoint{1.254980in}{0.150000in}}{\pgfqpoint{5.490039in}{5.490039in}}%
\pgfusepath{clip}%
\pgfsetbuttcap%
\pgfsetroundjoin%
\definecolor{currentfill}{rgb}{0.269944,0.014625,0.341379}%
\pgfsetfillcolor{currentfill}%
\pgfsetfillopacity{0.700000}%
\pgfsetlinewidth{0.000000pt}%
\definecolor{currentstroke}{rgb}{0.000000,0.000000,0.000000}%
\pgfsetstrokecolor{currentstroke}%
\pgfsetdash{}{0pt}%
\pgfpathmoveto{\pgfqpoint{3.565345in}{1.392272in}}%
\pgfpathlineto{\pgfqpoint{3.578817in}{1.389991in}}%
\pgfpathlineto{\pgfqpoint{3.592295in}{1.387874in}}%
\pgfpathlineto{\pgfqpoint{3.605779in}{1.385920in}}%
\pgfpathlineto{\pgfqpoint{3.619269in}{1.384128in}}%
\pgfpathlineto{\pgfqpoint{3.627294in}{1.393400in}}%
\pgfpathlineto{\pgfqpoint{3.635312in}{1.402815in}}%
\pgfpathlineto{\pgfqpoint{3.643323in}{1.412367in}}%
\pgfpathlineto{\pgfqpoint{3.651327in}{1.422052in}}%
\pgfpathlineto{\pgfqpoint{3.637852in}{1.423290in}}%
\pgfpathlineto{\pgfqpoint{3.624384in}{1.424690in}}%
\pgfpathlineto{\pgfqpoint{3.610922in}{1.426253in}}%
\pgfpathlineto{\pgfqpoint{3.597466in}{1.427980in}}%
\pgfpathlineto{\pgfqpoint{3.589447in}{1.418838in}}%
\pgfpathlineto{\pgfqpoint{3.581420in}{1.409836in}}%
\pgfpathlineto{\pgfqpoint{3.573386in}{1.400979in}}%
\pgfpathlineto{\pgfqpoint{3.565345in}{1.392272in}}%
\pgfpathclose%
\pgfusepath{fill}%
\end{pgfscope}%
\begin{pgfscope}%
\pgfpathrectangle{\pgfqpoint{1.254980in}{0.150000in}}{\pgfqpoint{5.490039in}{5.490039in}}%
\pgfusepath{clip}%
\pgfsetbuttcap%
\pgfsetroundjoin%
\definecolor{currentfill}{rgb}{0.150476,0.504369,0.557430}%
\pgfsetfillcolor{currentfill}%
\pgfsetfillopacity{0.700000}%
\pgfsetlinewidth{0.000000pt}%
\definecolor{currentstroke}{rgb}{0.000000,0.000000,0.000000}%
\pgfsetstrokecolor{currentstroke}%
\pgfsetdash{}{0pt}%
\pgfpathmoveto{\pgfqpoint{4.752188in}{2.459399in}}%
\pgfpathlineto{\pgfqpoint{4.766151in}{2.470514in}}%
\pgfpathlineto{\pgfqpoint{4.780130in}{2.481791in}}%
\pgfpathlineto{\pgfqpoint{4.794126in}{2.493230in}}%
\pgfpathlineto{\pgfqpoint{4.808140in}{2.504831in}}%
\pgfpathlineto{\pgfqpoint{4.815777in}{2.516292in}}%
\pgfpathlineto{\pgfqpoint{4.823408in}{2.527610in}}%
\pgfpathlineto{\pgfqpoint{4.831032in}{2.538782in}}%
\pgfpathlineto{\pgfqpoint{4.838649in}{2.549811in}}%
\pgfpathlineto{\pgfqpoint{4.824636in}{2.538132in}}%
\pgfpathlineto{\pgfqpoint{4.810640in}{2.526616in}}%
\pgfpathlineto{\pgfqpoint{4.796661in}{2.515262in}}%
\pgfpathlineto{\pgfqpoint{4.782700in}{2.504069in}}%
\pgfpathlineto{\pgfqpoint{4.775082in}{2.493107in}}%
\pgfpathlineto{\pgfqpoint{4.767457in}{2.482007in}}%
\pgfpathlineto{\pgfqpoint{4.759826in}{2.470771in}}%
\pgfpathlineto{\pgfqpoint{4.752188in}{2.459399in}}%
\pgfpathclose%
\pgfusepath{fill}%
\end{pgfscope}%
\begin{pgfscope}%
\pgfpathrectangle{\pgfqpoint{1.254980in}{0.150000in}}{\pgfqpoint{5.490039in}{5.490039in}}%
\pgfusepath{clip}%
\pgfsetbuttcap%
\pgfsetroundjoin%
\definecolor{currentfill}{rgb}{0.283229,0.120777,0.440584}%
\pgfsetfillcolor{currentfill}%
\pgfsetfillopacity{0.700000}%
\pgfsetlinewidth{0.000000pt}%
\definecolor{currentstroke}{rgb}{0.000000,0.000000,0.000000}%
\pgfsetstrokecolor{currentstroke}%
\pgfsetdash{}{0pt}%
\pgfpathmoveto{\pgfqpoint{3.908766in}{1.562471in}}%
\pgfpathlineto{\pgfqpoint{3.922313in}{1.564825in}}%
\pgfpathlineto{\pgfqpoint{3.935870in}{1.567338in}}%
\pgfpathlineto{\pgfqpoint{3.949436in}{1.570010in}}%
\pgfpathlineto{\pgfqpoint{3.963011in}{1.572842in}}%
\pgfpathlineto{\pgfqpoint{3.970906in}{1.585617in}}%
\pgfpathlineto{\pgfqpoint{3.978796in}{1.598430in}}%
\pgfpathlineto{\pgfqpoint{3.986681in}{1.611276in}}%
\pgfpathlineto{\pgfqpoint{3.994562in}{1.624151in}}%
\pgfpathlineto{\pgfqpoint{3.980991in}{1.620872in}}%
\pgfpathlineto{\pgfqpoint{3.967430in}{1.617753in}}%
\pgfpathlineto{\pgfqpoint{3.953879in}{1.614794in}}%
\pgfpathlineto{\pgfqpoint{3.940337in}{1.611994in}}%
\pgfpathlineto{\pgfqpoint{3.932452in}{1.599555in}}%
\pgfpathlineto{\pgfqpoint{3.924561in}{1.587152in}}%
\pgfpathlineto{\pgfqpoint{3.916666in}{1.574789in}}%
\pgfpathlineto{\pgfqpoint{3.908766in}{1.562471in}}%
\pgfpathclose%
\pgfusepath{fill}%
\end{pgfscope}%
\begin{pgfscope}%
\pgfpathrectangle{\pgfqpoint{1.254980in}{0.150000in}}{\pgfqpoint{5.490039in}{5.490039in}}%
\pgfusepath{clip}%
\pgfsetbuttcap%
\pgfsetroundjoin%
\definecolor{currentfill}{rgb}{0.282327,0.094955,0.417331}%
\pgfsetfillcolor{currentfill}%
\pgfsetfillopacity{0.700000}%
\pgfsetlinewidth{0.000000pt}%
\definecolor{currentstroke}{rgb}{0.000000,0.000000,0.000000}%
\pgfsetstrokecolor{currentstroke}%
\pgfsetdash{}{0pt}%
\pgfpathmoveto{\pgfqpoint{2.948929in}{1.563655in}}%
\pgfpathlineto{\pgfqpoint{2.962442in}{1.552598in}}%
\pgfpathlineto{\pgfqpoint{2.975953in}{1.541729in}}%
\pgfpathlineto{\pgfqpoint{2.989463in}{1.531047in}}%
\pgfpathlineto{\pgfqpoint{3.002972in}{1.520550in}}%
\pgfpathlineto{\pgfqpoint{3.011416in}{1.520592in}}%
\pgfpathlineto{\pgfqpoint{3.019844in}{1.520943in}}%
\pgfpathlineto{\pgfqpoint{3.028256in}{1.521596in}}%
\pgfpathlineto{\pgfqpoint{3.036654in}{1.522542in}}%
\pgfpathlineto{\pgfqpoint{3.023184in}{1.532362in}}%
\pgfpathlineto{\pgfqpoint{3.009714in}{1.542367in}}%
\pgfpathlineto{\pgfqpoint{2.996242in}{1.552559in}}%
\pgfpathlineto{\pgfqpoint{2.982770in}{1.562938in}}%
\pgfpathlineto{\pgfqpoint{2.974333in}{1.562657in}}%
\pgfpathlineto{\pgfqpoint{2.965881in}{1.562678in}}%
\pgfpathlineto{\pgfqpoint{2.957413in}{1.563008in}}%
\pgfpathlineto{\pgfqpoint{2.948929in}{1.563655in}}%
\pgfpathclose%
\pgfusepath{fill}%
\end{pgfscope}%
\begin{pgfscope}%
\pgfpathrectangle{\pgfqpoint{1.254980in}{0.150000in}}{\pgfqpoint{5.490039in}{5.490039in}}%
\pgfusepath{clip}%
\pgfsetbuttcap%
\pgfsetroundjoin%
\definecolor{currentfill}{rgb}{0.235526,0.309527,0.542944}%
\pgfsetfillcolor{currentfill}%
\pgfsetfillopacity{0.700000}%
\pgfsetlinewidth{0.000000pt}%
\definecolor{currentstroke}{rgb}{0.000000,0.000000,0.000000}%
\pgfsetstrokecolor{currentstroke}%
\pgfsetdash{}{0pt}%
\pgfpathmoveto{\pgfqpoint{4.314832in}{1.955711in}}%
\pgfpathlineto{\pgfqpoint{4.328547in}{1.962886in}}%
\pgfpathlineto{\pgfqpoint{4.342275in}{1.970222in}}%
\pgfpathlineto{\pgfqpoint{4.356017in}{1.977717in}}%
\pgfpathlineto{\pgfqpoint{4.369773in}{1.985371in}}%
\pgfpathlineto{\pgfqpoint{4.377558in}{1.999199in}}%
\pgfpathlineto{\pgfqpoint{4.385339in}{2.012954in}}%
\pgfpathlineto{\pgfqpoint{4.393115in}{2.026635in}}%
\pgfpathlineto{\pgfqpoint{4.400887in}{2.040238in}}%
\pgfpathlineto{\pgfqpoint{4.387130in}{2.032301in}}%
\pgfpathlineto{\pgfqpoint{4.373387in}{2.024524in}}%
\pgfpathlineto{\pgfqpoint{4.359658in}{2.016907in}}%
\pgfpathlineto{\pgfqpoint{4.345942in}{2.009450in}}%
\pgfpathlineto{\pgfqpoint{4.338171in}{1.996118in}}%
\pgfpathlineto{\pgfqpoint{4.330396in}{1.982715in}}%
\pgfpathlineto{\pgfqpoint{4.322616in}{1.969246in}}%
\pgfpathlineto{\pgfqpoint{4.314832in}{1.955711in}}%
\pgfpathclose%
\pgfusepath{fill}%
\end{pgfscope}%
\begin{pgfscope}%
\pgfpathrectangle{\pgfqpoint{1.254980in}{0.150000in}}{\pgfqpoint{5.490039in}{5.490039in}}%
\pgfusepath{clip}%
\pgfsetbuttcap%
\pgfsetroundjoin%
\definecolor{currentfill}{rgb}{0.267004,0.004874,0.329415}%
\pgfsetfillcolor{currentfill}%
\pgfsetfillopacity{0.700000}%
\pgfsetlinewidth{0.000000pt}%
\definecolor{currentstroke}{rgb}{0.000000,0.000000,0.000000}%
\pgfsetstrokecolor{currentstroke}%
\pgfsetdash{}{0pt}%
\pgfpathmoveto{\pgfqpoint{3.338937in}{1.383273in}}%
\pgfpathlineto{\pgfqpoint{3.352400in}{1.377805in}}%
\pgfpathlineto{\pgfqpoint{3.365866in}{1.372505in}}%
\pgfpathlineto{\pgfqpoint{3.379335in}{1.367375in}}%
\pgfpathlineto{\pgfqpoint{3.392807in}{1.362412in}}%
\pgfpathlineto{\pgfqpoint{3.400959in}{1.368471in}}%
\pgfpathlineto{\pgfqpoint{3.409100in}{1.374740in}}%
\pgfpathlineto{\pgfqpoint{3.417232in}{1.381215in}}%
\pgfpathlineto{\pgfqpoint{3.425355in}{1.387888in}}%
\pgfpathlineto{\pgfqpoint{3.411906in}{1.392240in}}%
\pgfpathlineto{\pgfqpoint{3.398461in}{1.396760in}}%
\pgfpathlineto{\pgfqpoint{3.385019in}{1.401449in}}%
\pgfpathlineto{\pgfqpoint{3.371581in}{1.406307in}}%
\pgfpathlineto{\pgfqpoint{3.363435in}{1.400233in}}%
\pgfpathlineto{\pgfqpoint{3.355279in}{1.394366in}}%
\pgfpathlineto{\pgfqpoint{3.347113in}{1.388710in}}%
\pgfpathlineto{\pgfqpoint{3.338937in}{1.383273in}}%
\pgfpathclose%
\pgfusepath{fill}%
\end{pgfscope}%
\begin{pgfscope}%
\pgfpathrectangle{\pgfqpoint{1.254980in}{0.150000in}}{\pgfqpoint{5.490039in}{5.490039in}}%
\pgfusepath{clip}%
\pgfsetbuttcap%
\pgfsetroundjoin%
\definecolor{currentfill}{rgb}{0.169646,0.456262,0.558030}%
\pgfsetfillcolor{currentfill}%
\pgfsetfillopacity{0.700000}%
\pgfsetlinewidth{0.000000pt}%
\definecolor{currentstroke}{rgb}{0.000000,0.000000,0.000000}%
\pgfsetstrokecolor{currentstroke}%
\pgfsetdash{}{0pt}%
\pgfpathmoveto{\pgfqpoint{2.239143in}{2.428546in}}%
\pgfpathlineto{\pgfqpoint{2.253022in}{2.405845in}}%
\pgfpathlineto{\pgfqpoint{2.266888in}{2.383415in}}%
\pgfpathlineto{\pgfqpoint{2.280741in}{2.361253in}}%
\pgfpathlineto{\pgfqpoint{2.294581in}{2.339358in}}%
\pgfpathlineto{\pgfqpoint{2.303717in}{2.330162in}}%
\pgfpathlineto{\pgfqpoint{2.312826in}{2.321385in}}%
\pgfpathlineto{\pgfqpoint{2.321908in}{2.313019in}}%
\pgfpathlineto{\pgfqpoint{2.330964in}{2.305058in}}%
\pgfpathlineto{\pgfqpoint{2.317190in}{2.326223in}}%
\pgfpathlineto{\pgfqpoint{2.303404in}{2.347653in}}%
\pgfpathlineto{\pgfqpoint{2.289606in}{2.369350in}}%
\pgfpathlineto{\pgfqpoint{2.275795in}{2.391316in}}%
\pgfpathlineto{\pgfqpoint{2.266674in}{2.399997in}}%
\pgfpathlineto{\pgfqpoint{2.257525in}{2.409090in}}%
\pgfpathlineto{\pgfqpoint{2.248348in}{2.418604in}}%
\pgfpathlineto{\pgfqpoint{2.239143in}{2.428546in}}%
\pgfpathclose%
\pgfusepath{fill}%
\end{pgfscope}%
\begin{pgfscope}%
\pgfpathrectangle{\pgfqpoint{1.254980in}{0.150000in}}{\pgfqpoint{5.490039in}{5.490039in}}%
\pgfusepath{clip}%
\pgfsetbuttcap%
\pgfsetroundjoin%
\definecolor{currentfill}{rgb}{0.271305,0.019942,0.347269}%
\pgfsetfillcolor{currentfill}%
\pgfsetfillopacity{0.700000}%
\pgfsetlinewidth{0.000000pt}%
\definecolor{currentstroke}{rgb}{0.000000,0.000000,0.000000}%
\pgfsetstrokecolor{currentstroke}%
\pgfsetdash{}{0pt}%
\pgfpathmoveto{\pgfqpoint{3.198284in}{1.418862in}}%
\pgfpathlineto{\pgfqpoint{3.211758in}{1.411377in}}%
\pgfpathlineto{\pgfqpoint{3.225233in}{1.404067in}}%
\pgfpathlineto{\pgfqpoint{3.238710in}{1.396930in}}%
\pgfpathlineto{\pgfqpoint{3.252189in}{1.389967in}}%
\pgfpathlineto{\pgfqpoint{3.260438in}{1.393818in}}%
\pgfpathlineto{\pgfqpoint{3.268675in}{1.397920in}}%
\pgfpathlineto{\pgfqpoint{3.276900in}{1.402266in}}%
\pgfpathlineto{\pgfqpoint{3.285114in}{1.406850in}}%
\pgfpathlineto{\pgfqpoint{3.271665in}{1.413173in}}%
\pgfpathlineto{\pgfqpoint{3.258218in}{1.419668in}}%
\pgfpathlineto{\pgfqpoint{3.244772in}{1.426337in}}%
\pgfpathlineto{\pgfqpoint{3.231329in}{1.433180in}}%
\pgfpathlineto{\pgfqpoint{3.223086in}{1.429226in}}%
\pgfpathlineto{\pgfqpoint{3.214831in}{1.425518in}}%
\pgfpathlineto{\pgfqpoint{3.206564in}{1.422061in}}%
\pgfpathlineto{\pgfqpoint{3.198284in}{1.418862in}}%
\pgfpathclose%
\pgfusepath{fill}%
\end{pgfscope}%
\begin{pgfscope}%
\pgfpathrectangle{\pgfqpoint{1.254980in}{0.150000in}}{\pgfqpoint{5.490039in}{5.490039in}}%
\pgfusepath{clip}%
\pgfsetbuttcap%
\pgfsetroundjoin%
\definecolor{currentfill}{rgb}{0.421908,0.805774,0.351910}%
\pgfsetfillcolor{currentfill}%
\pgfsetfillopacity{0.700000}%
\pgfsetlinewidth{0.000000pt}%
\definecolor{currentstroke}{rgb}{0.000000,0.000000,0.000000}%
\pgfsetstrokecolor{currentstroke}%
\pgfsetdash{}{0pt}%
\pgfpathmoveto{\pgfqpoint{5.650757in}{3.334737in}}%
\pgfpathlineto{\pgfqpoint{5.665323in}{3.350311in}}%
\pgfpathlineto{\pgfqpoint{5.679912in}{3.366049in}}%
\pgfpathlineto{\pgfqpoint{5.694522in}{3.381952in}}%
\pgfpathlineto{\pgfqpoint{5.709155in}{3.398020in}}%
\pgfpathlineto{\pgfqpoint{5.716239in}{3.400366in}}%
\pgfpathlineto{\pgfqpoint{5.723313in}{3.402582in}}%
\pgfpathlineto{\pgfqpoint{5.730375in}{3.404673in}}%
\pgfpathlineto{\pgfqpoint{5.737427in}{3.406641in}}%
\pgfpathlineto{\pgfqpoint{5.722815in}{3.390908in}}%
\pgfpathlineto{\pgfqpoint{5.708225in}{3.375339in}}%
\pgfpathlineto{\pgfqpoint{5.693657in}{3.359933in}}%
\pgfpathlineto{\pgfqpoint{5.679111in}{3.344691in}}%
\pgfpathlineto{\pgfqpoint{5.672038in}{3.342378in}}%
\pgfpathlineto{\pgfqpoint{5.664955in}{3.339951in}}%
\pgfpathlineto{\pgfqpoint{5.657861in}{3.337405in}}%
\pgfpathlineto{\pgfqpoint{5.650757in}{3.334737in}}%
\pgfpathclose%
\pgfusepath{fill}%
\end{pgfscope}%
\begin{pgfscope}%
\pgfpathrectangle{\pgfqpoint{1.254980in}{0.150000in}}{\pgfqpoint{5.490039in}{5.490039in}}%
\pgfusepath{clip}%
\pgfsetbuttcap%
\pgfsetroundjoin%
\definecolor{currentfill}{rgb}{0.169646,0.456262,0.558030}%
\pgfsetfillcolor{currentfill}%
\pgfsetfillopacity{0.700000}%
\pgfsetlinewidth{0.000000pt}%
\definecolor{currentstroke}{rgb}{0.000000,0.000000,0.000000}%
\pgfsetstrokecolor{currentstroke}%
\pgfsetdash{}{0pt}%
\pgfpathmoveto{\pgfqpoint{4.635179in}{2.321722in}}%
\pgfpathlineto{\pgfqpoint{4.649076in}{2.331946in}}%
\pgfpathlineto{\pgfqpoint{4.662988in}{2.342332in}}%
\pgfpathlineto{\pgfqpoint{4.676916in}{2.352879in}}%
\pgfpathlineto{\pgfqpoint{4.690860in}{2.363587in}}%
\pgfpathlineto{\pgfqpoint{4.698548in}{2.376026in}}%
\pgfpathlineto{\pgfqpoint{4.706229in}{2.388335in}}%
\pgfpathlineto{\pgfqpoint{4.713905in}{2.400512in}}%
\pgfpathlineto{\pgfqpoint{4.721574in}{2.412557in}}%
\pgfpathlineto{\pgfqpoint{4.707628in}{2.401711in}}%
\pgfpathlineto{\pgfqpoint{4.693699in}{2.391027in}}%
\pgfpathlineto{\pgfqpoint{4.679786in}{2.380504in}}%
\pgfpathlineto{\pgfqpoint{4.665889in}{2.370143in}}%
\pgfpathlineto{\pgfqpoint{4.658221in}{2.358224in}}%
\pgfpathlineto{\pgfqpoint{4.650546in}{2.346180in}}%
\pgfpathlineto{\pgfqpoint{4.642866in}{2.334013in}}%
\pgfpathlineto{\pgfqpoint{4.635179in}{2.321722in}}%
\pgfpathclose%
\pgfusepath{fill}%
\end{pgfscope}%
\begin{pgfscope}%
\pgfpathrectangle{\pgfqpoint{1.254980in}{0.150000in}}{\pgfqpoint{5.490039in}{5.490039in}}%
\pgfusepath{clip}%
\pgfsetbuttcap%
\pgfsetroundjoin%
\definecolor{currentfill}{rgb}{0.267004,0.004874,0.329415}%
\pgfsetfillcolor{currentfill}%
\pgfsetfillopacity{0.700000}%
\pgfsetlinewidth{0.000000pt}%
\definecolor{currentstroke}{rgb}{0.000000,0.000000,0.000000}%
\pgfsetstrokecolor{currentstroke}%
\pgfsetdash{}{0pt}%
\pgfpathmoveto{\pgfqpoint{3.479192in}{1.372148in}}%
\pgfpathlineto{\pgfqpoint{3.492663in}{1.368628in}}%
\pgfpathlineto{\pgfqpoint{3.506138in}{1.365273in}}%
\pgfpathlineto{\pgfqpoint{3.519617in}{1.362083in}}%
\pgfpathlineto{\pgfqpoint{3.533102in}{1.359056in}}%
\pgfpathlineto{\pgfqpoint{3.541175in}{1.367107in}}%
\pgfpathlineto{\pgfqpoint{3.549240in}{1.375330in}}%
\pgfpathlineto{\pgfqpoint{3.557296in}{1.383721in}}%
\pgfpathlineto{\pgfqpoint{3.565345in}{1.392272in}}%
\pgfpathlineto{\pgfqpoint{3.551879in}{1.394716in}}%
\pgfpathlineto{\pgfqpoint{3.538417in}{1.397325in}}%
\pgfpathlineto{\pgfqpoint{3.524961in}{1.400098in}}%
\pgfpathlineto{\pgfqpoint{3.511510in}{1.403036in}}%
\pgfpathlineto{\pgfqpoint{3.503443in}{1.395056in}}%
\pgfpathlineto{\pgfqpoint{3.495368in}{1.387244in}}%
\pgfpathlineto{\pgfqpoint{3.487284in}{1.379606in}}%
\pgfpathlineto{\pgfqpoint{3.479192in}{1.372148in}}%
\pgfpathclose%
\pgfusepath{fill}%
\end{pgfscope}%
\begin{pgfscope}%
\pgfpathrectangle{\pgfqpoint{1.254980in}{0.150000in}}{\pgfqpoint{5.490039in}{5.490039in}}%
\pgfusepath{clip}%
\pgfsetbuttcap%
\pgfsetroundjoin%
\definecolor{currentfill}{rgb}{0.281412,0.155834,0.469201}%
\pgfsetfillcolor{currentfill}%
\pgfsetfillopacity{0.700000}%
\pgfsetlinewidth{0.000000pt}%
\definecolor{currentstroke}{rgb}{0.000000,0.000000,0.000000}%
\pgfsetstrokecolor{currentstroke}%
\pgfsetdash{}{0pt}%
\pgfpathmoveto{\pgfqpoint{3.994562in}{1.624151in}}%
\pgfpathlineto{\pgfqpoint{4.008142in}{1.627589in}}%
\pgfpathlineto{\pgfqpoint{4.021734in}{1.631186in}}%
\pgfpathlineto{\pgfqpoint{4.035335in}{1.634942in}}%
\pgfpathlineto{\pgfqpoint{4.048947in}{1.638857in}}%
\pgfpathlineto{\pgfqpoint{4.056820in}{1.652188in}}%
\pgfpathlineto{\pgfqpoint{4.064688in}{1.665533in}}%
\pgfpathlineto{\pgfqpoint{4.072552in}{1.678889in}}%
\pgfpathlineto{\pgfqpoint{4.080411in}{1.692251in}}%
\pgfpathlineto{\pgfqpoint{4.066802in}{1.687916in}}%
\pgfpathlineto{\pgfqpoint{4.053204in}{1.683740in}}%
\pgfpathlineto{\pgfqpoint{4.039616in}{1.679723in}}%
\pgfpathlineto{\pgfqpoint{4.026039in}{1.675866in}}%
\pgfpathlineto{\pgfqpoint{4.018176in}{1.662913in}}%
\pgfpathlineto{\pgfqpoint{4.010309in}{1.649974in}}%
\pgfpathlineto{\pgfqpoint{4.002438in}{1.637052in}}%
\pgfpathlineto{\pgfqpoint{3.994562in}{1.624151in}}%
\pgfpathclose%
\pgfusepath{fill}%
\end{pgfscope}%
\begin{pgfscope}%
\pgfpathrectangle{\pgfqpoint{1.254980in}{0.150000in}}{\pgfqpoint{5.490039in}{5.490039in}}%
\pgfusepath{clip}%
\pgfsetbuttcap%
\pgfsetroundjoin%
\definecolor{currentfill}{rgb}{0.232815,0.732247,0.459277}%
\pgfsetfillcolor{currentfill}%
\pgfsetfillopacity{0.700000}%
\pgfsetlinewidth{0.000000pt}%
\definecolor{currentstroke}{rgb}{0.000000,0.000000,0.000000}%
\pgfsetstrokecolor{currentstroke}%
\pgfsetdash{}{0pt}%
\pgfpathmoveto{\pgfqpoint{5.361788in}{3.086525in}}%
\pgfpathlineto{\pgfqpoint{5.376159in}{3.101138in}}%
\pgfpathlineto{\pgfqpoint{5.390550in}{3.115914in}}%
\pgfpathlineto{\pgfqpoint{5.404962in}{3.130855in}}%
\pgfpathlineto{\pgfqpoint{5.419395in}{3.145961in}}%
\pgfpathlineto{\pgfqpoint{5.426696in}{3.151410in}}%
\pgfpathlineto{\pgfqpoint{5.433987in}{3.156705in}}%
\pgfpathlineto{\pgfqpoint{5.441268in}{3.161847in}}%
\pgfpathlineto{\pgfqpoint{5.448539in}{3.166838in}}%
\pgfpathlineto{\pgfqpoint{5.434118in}{3.151938in}}%
\pgfpathlineto{\pgfqpoint{5.419718in}{3.137201in}}%
\pgfpathlineto{\pgfqpoint{5.405339in}{3.122628in}}%
\pgfpathlineto{\pgfqpoint{5.390980in}{3.108219in}}%
\pgfpathlineto{\pgfqpoint{5.383697in}{3.103012in}}%
\pgfpathlineto{\pgfqpoint{5.376404in}{3.097662in}}%
\pgfpathlineto{\pgfqpoint{5.369101in}{3.092168in}}%
\pgfpathlineto{\pgfqpoint{5.361788in}{3.086525in}}%
\pgfpathclose%
\pgfusepath{fill}%
\end{pgfscope}%
\begin{pgfscope}%
\pgfpathrectangle{\pgfqpoint{1.254980in}{0.150000in}}{\pgfqpoint{5.490039in}{5.490039in}}%
\pgfusepath{clip}%
\pgfsetbuttcap%
\pgfsetroundjoin%
\definecolor{currentfill}{rgb}{0.121831,0.589055,0.545623}%
\pgfsetfillcolor{currentfill}%
\pgfsetfillopacity{0.700000}%
\pgfsetlinewidth{0.000000pt}%
\definecolor{currentstroke}{rgb}{0.000000,0.000000,0.000000}%
\pgfsetstrokecolor{currentstroke}%
\pgfsetdash{}{0pt}%
\pgfpathmoveto{\pgfqpoint{4.955548in}{2.681410in}}%
\pgfpathlineto{\pgfqpoint{4.969646in}{2.693964in}}%
\pgfpathlineto{\pgfqpoint{4.983763in}{2.706681in}}%
\pgfpathlineto{\pgfqpoint{4.997898in}{2.719562in}}%
\pgfpathlineto{\pgfqpoint{5.012052in}{2.732605in}}%
\pgfpathlineto{\pgfqpoint{5.019599in}{2.742324in}}%
\pgfpathlineto{\pgfqpoint{5.027139in}{2.751885in}}%
\pgfpathlineto{\pgfqpoint{5.034670in}{2.761287in}}%
\pgfpathlineto{\pgfqpoint{5.042193in}{2.770531in}}%
\pgfpathlineto{\pgfqpoint{5.028043in}{2.757502in}}%
\pgfpathlineto{\pgfqpoint{5.013910in}{2.744637in}}%
\pgfpathlineto{\pgfqpoint{4.999797in}{2.731934in}}%
\pgfpathlineto{\pgfqpoint{4.985701in}{2.719394in}}%
\pgfpathlineto{\pgfqpoint{4.978174in}{2.710124in}}%
\pgfpathlineto{\pgfqpoint{4.970640in}{2.700703in}}%
\pgfpathlineto{\pgfqpoint{4.963098in}{2.691132in}}%
\pgfpathlineto{\pgfqpoint{4.955548in}{2.681410in}}%
\pgfpathclose%
\pgfusepath{fill}%
\end{pgfscope}%
\begin{pgfscope}%
\pgfpathrectangle{\pgfqpoint{1.254980in}{0.150000in}}{\pgfqpoint{5.490039in}{5.490039in}}%
\pgfusepath{clip}%
\pgfsetbuttcap%
\pgfsetroundjoin%
\definecolor{currentfill}{rgb}{0.258965,0.251537,0.524736}%
\pgfsetfillcolor{currentfill}%
\pgfsetfillopacity{0.700000}%
\pgfsetlinewidth{0.000000pt}%
\definecolor{currentstroke}{rgb}{0.000000,0.000000,0.000000}%
\pgfsetstrokecolor{currentstroke}%
\pgfsetdash{}{0pt}%
\pgfpathmoveto{\pgfqpoint{4.197668in}{1.820908in}}%
\pgfpathlineto{\pgfqpoint{4.211332in}{1.826798in}}%
\pgfpathlineto{\pgfqpoint{4.225008in}{1.832847in}}%
\pgfpathlineto{\pgfqpoint{4.238697in}{1.839055in}}%
\pgfpathlineto{\pgfqpoint{4.252398in}{1.845422in}}%
\pgfpathlineto{\pgfqpoint{4.260217in}{1.859377in}}%
\pgfpathlineto{\pgfqpoint{4.268032in}{1.873290in}}%
\pgfpathlineto{\pgfqpoint{4.275843in}{1.887157in}}%
\pgfpathlineto{\pgfqpoint{4.283650in}{1.900976in}}%
\pgfpathlineto{\pgfqpoint{4.269948in}{1.894271in}}%
\pgfpathlineto{\pgfqpoint{4.256259in}{1.887724in}}%
\pgfpathlineto{\pgfqpoint{4.242583in}{1.881338in}}%
\pgfpathlineto{\pgfqpoint{4.228919in}{1.875111in}}%
\pgfpathlineto{\pgfqpoint{4.221113in}{1.861619in}}%
\pgfpathlineto{\pgfqpoint{4.213302in}{1.848086in}}%
\pgfpathlineto{\pgfqpoint{4.205487in}{1.834515in}}%
\pgfpathlineto{\pgfqpoint{4.197668in}{1.820908in}}%
\pgfpathclose%
\pgfusepath{fill}%
\end{pgfscope}%
\begin{pgfscope}%
\pgfpathrectangle{\pgfqpoint{1.254980in}{0.150000in}}{\pgfqpoint{5.490039in}{5.490039in}}%
\pgfusepath{clip}%
\pgfsetbuttcap%
\pgfsetroundjoin%
\definecolor{currentfill}{rgb}{0.477504,0.821444,0.318195}%
\pgfsetfillcolor{currentfill}%
\pgfsetfillopacity{0.700000}%
\pgfsetlinewidth{0.000000pt}%
\definecolor{currentstroke}{rgb}{0.000000,0.000000,0.000000}%
\pgfsetstrokecolor{currentstroke}%
\pgfsetdash{}{0pt}%
\pgfpathmoveto{\pgfqpoint{5.737427in}{3.406641in}}%
\pgfpathlineto{\pgfqpoint{5.752062in}{3.422539in}}%
\pgfpathlineto{\pgfqpoint{5.766719in}{3.438602in}}%
\pgfpathlineto{\pgfqpoint{5.781399in}{3.454830in}}%
\pgfpathlineto{\pgfqpoint{5.788423in}{3.456414in}}%
\pgfpathlineto{\pgfqpoint{5.795437in}{3.457880in}}%
\pgfpathlineto{\pgfqpoint{5.802440in}{3.459229in}}%
\pgfpathlineto{\pgfqpoint{5.809432in}{3.460467in}}%
\pgfpathlineto{\pgfqpoint{5.794775in}{3.444606in}}%
\pgfpathlineto{\pgfqpoint{5.780141in}{3.428909in}}%
\pgfpathlineto{\pgfqpoint{5.765529in}{3.413376in}}%
\pgfpathlineto{\pgfqpoint{5.758519in}{3.411856in}}%
\pgfpathlineto{\pgfqpoint{5.751499in}{3.410229in}}%
\pgfpathlineto{\pgfqpoint{5.744468in}{3.408492in}}%
\pgfpathlineto{\pgfqpoint{5.737427in}{3.406641in}}%
\pgfpathclose%
\pgfusepath{fill}%
\end{pgfscope}%
\begin{pgfscope}%
\pgfpathrectangle{\pgfqpoint{1.254980in}{0.150000in}}{\pgfqpoint{5.490039in}{5.490039in}}%
\pgfusepath{clip}%
\pgfsetbuttcap%
\pgfsetroundjoin%
\definecolor{currentfill}{rgb}{0.280894,0.078907,0.402329}%
\pgfsetfillcolor{currentfill}%
\pgfsetfillopacity{0.700000}%
\pgfsetlinewidth{0.000000pt}%
\definecolor{currentstroke}{rgb}{0.000000,0.000000,0.000000}%
\pgfsetstrokecolor{currentstroke}%
\pgfsetdash{}{0pt}%
\pgfpathmoveto{\pgfqpoint{3.002972in}{1.520550in}}%
\pgfpathlineto{\pgfqpoint{3.016480in}{1.510239in}}%
\pgfpathlineto{\pgfqpoint{3.029986in}{1.500112in}}%
\pgfpathlineto{\pgfqpoint{3.043492in}{1.490168in}}%
\pgfpathlineto{\pgfqpoint{3.056997in}{1.480406in}}%
\pgfpathlineto{\pgfqpoint{3.065402in}{1.481136in}}%
\pgfpathlineto{\pgfqpoint{3.073791in}{1.482166in}}%
\pgfpathlineto{\pgfqpoint{3.082166in}{1.483490in}}%
\pgfpathlineto{\pgfqpoint{3.090527in}{1.485101in}}%
\pgfpathlineto{\pgfqpoint{3.077059in}{1.494187in}}%
\pgfpathlineto{\pgfqpoint{3.063591in}{1.503456in}}%
\pgfpathlineto{\pgfqpoint{3.050123in}{1.512907in}}%
\pgfpathlineto{\pgfqpoint{3.036654in}{1.522542in}}%
\pgfpathlineto{\pgfqpoint{3.028256in}{1.521596in}}%
\pgfpathlineto{\pgfqpoint{3.019844in}{1.520943in}}%
\pgfpathlineto{\pgfqpoint{3.011416in}{1.520592in}}%
\pgfpathlineto{\pgfqpoint{3.002972in}{1.520550in}}%
\pgfpathclose%
\pgfusepath{fill}%
\end{pgfscope}%
\begin{pgfscope}%
\pgfpathrectangle{\pgfqpoint{1.254980in}{0.150000in}}{\pgfqpoint{5.490039in}{5.490039in}}%
\pgfusepath{clip}%
\pgfsetbuttcap%
\pgfsetroundjoin%
\definecolor{currentfill}{rgb}{0.140210,0.665859,0.513427}%
\pgfsetfillcolor{currentfill}%
\pgfsetfillopacity{0.700000}%
\pgfsetlinewidth{0.000000pt}%
\definecolor{currentstroke}{rgb}{0.000000,0.000000,0.000000}%
\pgfsetstrokecolor{currentstroke}%
\pgfsetdash{}{0pt}%
\pgfpathmoveto{\pgfqpoint{5.158834in}{2.892161in}}%
\pgfpathlineto{\pgfqpoint{5.173071in}{2.905884in}}%
\pgfpathlineto{\pgfqpoint{5.187326in}{2.919770in}}%
\pgfpathlineto{\pgfqpoint{5.201602in}{2.933820in}}%
\pgfpathlineto{\pgfqpoint{5.215897in}{2.948035in}}%
\pgfpathlineto{\pgfqpoint{5.223333in}{2.955706in}}%
\pgfpathlineto{\pgfqpoint{5.230759in}{2.963215in}}%
\pgfpathlineto{\pgfqpoint{5.238176in}{2.970562in}}%
\pgfpathlineto{\pgfqpoint{5.245583in}{2.977750in}}%
\pgfpathlineto{\pgfqpoint{5.231295in}{2.963645in}}%
\pgfpathlineto{\pgfqpoint{5.217027in}{2.949703in}}%
\pgfpathlineto{\pgfqpoint{5.202778in}{2.935925in}}%
\pgfpathlineto{\pgfqpoint{5.188549in}{2.922310in}}%
\pgfpathlineto{\pgfqpoint{5.181134in}{2.915003in}}%
\pgfpathlineto{\pgfqpoint{5.173709in}{2.907543in}}%
\pgfpathlineto{\pgfqpoint{5.166276in}{2.899930in}}%
\pgfpathlineto{\pgfqpoint{5.158834in}{2.892161in}}%
\pgfpathclose%
\pgfusepath{fill}%
\end{pgfscope}%
\begin{pgfscope}%
\pgfpathrectangle{\pgfqpoint{1.254980in}{0.150000in}}{\pgfqpoint{5.490039in}{5.490039in}}%
\pgfusepath{clip}%
\pgfsetbuttcap%
\pgfsetroundjoin%
\definecolor{currentfill}{rgb}{0.190631,0.407061,0.556089}%
\pgfsetfillcolor{currentfill}%
\pgfsetfillopacity{0.700000}%
\pgfsetlinewidth{0.000000pt}%
\definecolor{currentstroke}{rgb}{0.000000,0.000000,0.000000}%
\pgfsetstrokecolor{currentstroke}%
\pgfsetdash{}{0pt}%
\pgfpathmoveto{\pgfqpoint{4.518068in}{2.181291in}}%
\pgfpathlineto{\pgfqpoint{4.531900in}{2.190508in}}%
\pgfpathlineto{\pgfqpoint{4.545747in}{2.199887in}}%
\pgfpathlineto{\pgfqpoint{4.559609in}{2.209425in}}%
\pgfpathlineto{\pgfqpoint{4.573486in}{2.219125in}}%
\pgfpathlineto{\pgfqpoint{4.581217in}{2.232352in}}%
\pgfpathlineto{\pgfqpoint{4.588942in}{2.245467in}}%
\pgfpathlineto{\pgfqpoint{4.596662in}{2.258469in}}%
\pgfpathlineto{\pgfqpoint{4.604377in}{2.271356in}}%
\pgfpathlineto{\pgfqpoint{4.590498in}{2.261459in}}%
\pgfpathlineto{\pgfqpoint{4.576634in}{2.251724in}}%
\pgfpathlineto{\pgfqpoint{4.562786in}{2.242150in}}%
\pgfpathlineto{\pgfqpoint{4.548953in}{2.232737in}}%
\pgfpathlineto{\pgfqpoint{4.541240in}{2.220035in}}%
\pgfpathlineto{\pgfqpoint{4.533521in}{2.207226in}}%
\pgfpathlineto{\pgfqpoint{4.525797in}{2.194311in}}%
\pgfpathlineto{\pgfqpoint{4.518068in}{2.181291in}}%
\pgfpathclose%
\pgfusepath{fill}%
\end{pgfscope}%
\begin{pgfscope}%
\pgfpathrectangle{\pgfqpoint{1.254980in}{0.150000in}}{\pgfqpoint{5.490039in}{5.490039in}}%
\pgfusepath{clip}%
\pgfsetbuttcap%
\pgfsetroundjoin%
\definecolor{currentfill}{rgb}{0.120081,0.622161,0.534946}%
\pgfsetfillcolor{currentfill}%
\pgfsetfillopacity{0.700000}%
\pgfsetlinewidth{0.000000pt}%
\definecolor{currentstroke}{rgb}{0.000000,0.000000,0.000000}%
\pgfsetstrokecolor{currentstroke}%
\pgfsetdash{}{0pt}%
\pgfpathmoveto{\pgfqpoint{1.996451in}{2.891299in}}%
\pgfpathlineto{\pgfqpoint{2.010580in}{2.863371in}}%
\pgfpathlineto{\pgfqpoint{2.024691in}{2.835770in}}%
\pgfpathlineto{\pgfqpoint{2.038783in}{2.808493in}}%
\pgfpathlineto{\pgfqpoint{2.052856in}{2.781537in}}%
\pgfpathlineto{\pgfqpoint{2.062234in}{2.770352in}}%
\pgfpathlineto{\pgfqpoint{2.071582in}{2.759603in}}%
\pgfpathlineto{\pgfqpoint{2.080900in}{2.749282in}}%
\pgfpathlineto{\pgfqpoint{2.090189in}{2.739383in}}%
\pgfpathlineto{\pgfqpoint{2.076190in}{2.765609in}}%
\pgfpathlineto{\pgfqpoint{2.062174in}{2.792154in}}%
\pgfpathlineto{\pgfqpoint{2.048139in}{2.819020in}}%
\pgfpathlineto{\pgfqpoint{2.034086in}{2.846212in}}%
\pgfpathlineto{\pgfqpoint{2.024723in}{2.856830in}}%
\pgfpathlineto{\pgfqpoint{2.015330in}{2.867879in}}%
\pgfpathlineto{\pgfqpoint{2.005906in}{2.879366in}}%
\pgfpathlineto{\pgfqpoint{1.996451in}{2.891299in}}%
\pgfpathclose%
\pgfusepath{fill}%
\end{pgfscope}%
\begin{pgfscope}%
\pgfpathrectangle{\pgfqpoint{1.254980in}{0.150000in}}{\pgfqpoint{5.490039in}{5.490039in}}%
\pgfusepath{clip}%
\pgfsetbuttcap%
\pgfsetroundjoin%
\definecolor{currentfill}{rgb}{0.154815,0.493313,0.557840}%
\pgfsetfillcolor{currentfill}%
\pgfsetfillopacity{0.700000}%
\pgfsetlinewidth{0.000000pt}%
\definecolor{currentstroke}{rgb}{0.000000,0.000000,0.000000}%
\pgfsetstrokecolor{currentstroke}%
\pgfsetdash{}{0pt}%
\pgfpathmoveto{\pgfqpoint{2.183489in}{2.522112in}}%
\pgfpathlineto{\pgfqpoint{2.197424in}{2.498301in}}%
\pgfpathlineto{\pgfqpoint{2.211344in}{2.474772in}}%
\pgfpathlineto{\pgfqpoint{2.225251in}{2.451521in}}%
\pgfpathlineto{\pgfqpoint{2.239143in}{2.428546in}}%
\pgfpathlineto{\pgfqpoint{2.248348in}{2.418604in}}%
\pgfpathlineto{\pgfqpoint{2.257525in}{2.409090in}}%
\pgfpathlineto{\pgfqpoint{2.266674in}{2.399997in}}%
\pgfpathlineto{\pgfqpoint{2.275795in}{2.391316in}}%
\pgfpathlineto{\pgfqpoint{2.261971in}{2.413554in}}%
\pgfpathlineto{\pgfqpoint{2.248134in}{2.436067in}}%
\pgfpathlineto{\pgfqpoint{2.234284in}{2.458856in}}%
\pgfpathlineto{\pgfqpoint{2.220420in}{2.481924in}}%
\pgfpathlineto{\pgfqpoint{2.211230in}{2.491330in}}%
\pgfpathlineto{\pgfqpoint{2.202012in}{2.501158in}}%
\pgfpathlineto{\pgfqpoint{2.192765in}{2.511416in}}%
\pgfpathlineto{\pgfqpoint{2.183489in}{2.522112in}}%
\pgfpathclose%
\pgfusepath{fill}%
\end{pgfscope}%
\begin{pgfscope}%
\pgfpathrectangle{\pgfqpoint{1.254980in}{0.150000in}}{\pgfqpoint{5.490039in}{5.490039in}}%
\pgfusepath{clip}%
\pgfsetbuttcap%
\pgfsetroundjoin%
\definecolor{currentfill}{rgb}{0.216210,0.351535,0.550627}%
\pgfsetfillcolor{currentfill}%
\pgfsetfillopacity{0.700000}%
\pgfsetlinewidth{0.000000pt}%
\definecolor{currentstroke}{rgb}{0.000000,0.000000,0.000000}%
\pgfsetstrokecolor{currentstroke}%
\pgfsetdash{}{0pt}%
\pgfpathmoveto{\pgfqpoint{4.400887in}{2.040238in}}%
\pgfpathlineto{\pgfqpoint{4.414658in}{2.048335in}}%
\pgfpathlineto{\pgfqpoint{4.428443in}{2.056592in}}%
\pgfpathlineto{\pgfqpoint{4.442242in}{2.065009in}}%
\pgfpathlineto{\pgfqpoint{4.456055in}{2.073587in}}%
\pgfpathlineto{\pgfqpoint{4.463824in}{2.087376in}}%
\pgfpathlineto{\pgfqpoint{4.471588in}{2.101076in}}%
\pgfpathlineto{\pgfqpoint{4.479347in}{2.114685in}}%
\pgfpathlineto{\pgfqpoint{4.487101in}{2.128201in}}%
\pgfpathlineto{\pgfqpoint{4.473286in}{2.119369in}}%
\pgfpathlineto{\pgfqpoint{4.459485in}{2.110698in}}%
\pgfpathlineto{\pgfqpoint{4.445699in}{2.102187in}}%
\pgfpathlineto{\pgfqpoint{4.431927in}{2.093836in}}%
\pgfpathlineto{\pgfqpoint{4.424174in}{2.080564in}}%
\pgfpathlineto{\pgfqpoint{4.416416in}{2.067205in}}%
\pgfpathlineto{\pgfqpoint{4.408654in}{2.053762in}}%
\pgfpathlineto{\pgfqpoint{4.400887in}{2.040238in}}%
\pgfpathclose%
\pgfusepath{fill}%
\end{pgfscope}%
\begin{pgfscope}%
\pgfpathrectangle{\pgfqpoint{1.254980in}{0.150000in}}{\pgfqpoint{5.490039in}{5.490039in}}%
\pgfusepath{clip}%
\pgfsetbuttcap%
\pgfsetroundjoin%
\definecolor{currentfill}{rgb}{0.135066,0.544853,0.554029}%
\pgfsetfillcolor{currentfill}%
\pgfsetfillopacity{0.700000}%
\pgfsetlinewidth{0.000000pt}%
\definecolor{currentstroke}{rgb}{0.000000,0.000000,0.000000}%
\pgfsetstrokecolor{currentstroke}%
\pgfsetdash{}{0pt}%
\pgfpathmoveto{\pgfqpoint{4.838649in}{2.549811in}}%
\pgfpathlineto{\pgfqpoint{4.852680in}{2.561652in}}%
\pgfpathlineto{\pgfqpoint{4.866728in}{2.573655in}}%
\pgfpathlineto{\pgfqpoint{4.880793in}{2.585820in}}%
\pgfpathlineto{\pgfqpoint{4.894877in}{2.598149in}}%
\pgfpathlineto{\pgfqpoint{4.902486in}{2.609091in}}%
\pgfpathlineto{\pgfqpoint{4.910089in}{2.619880in}}%
\pgfpathlineto{\pgfqpoint{4.917684in}{2.630516in}}%
\pgfpathlineto{\pgfqpoint{4.925272in}{2.641000in}}%
\pgfpathlineto{\pgfqpoint{4.911189in}{2.628625in}}%
\pgfpathlineto{\pgfqpoint{4.897124in}{2.616412in}}%
\pgfpathlineto{\pgfqpoint{4.883078in}{2.604362in}}%
\pgfpathlineto{\pgfqpoint{4.869048in}{2.592474in}}%
\pgfpathlineto{\pgfqpoint{4.861459in}{2.582026in}}%
\pgfpathlineto{\pgfqpoint{4.853863in}{2.571433in}}%
\pgfpathlineto{\pgfqpoint{4.846260in}{2.560694in}}%
\pgfpathlineto{\pgfqpoint{4.838649in}{2.549811in}}%
\pgfpathclose%
\pgfusepath{fill}%
\end{pgfscope}%
\begin{pgfscope}%
\pgfpathrectangle{\pgfqpoint{1.254980in}{0.150000in}}{\pgfqpoint{5.490039in}{5.490039in}}%
\pgfusepath{clip}%
\pgfsetbuttcap%
\pgfsetroundjoin%
\definecolor{currentfill}{rgb}{0.275191,0.194905,0.496005}%
\pgfsetfillcolor{currentfill}%
\pgfsetfillopacity{0.700000}%
\pgfsetlinewidth{0.000000pt}%
\definecolor{currentstroke}{rgb}{0.000000,0.000000,0.000000}%
\pgfsetstrokecolor{currentstroke}%
\pgfsetdash{}{0pt}%
\pgfpathmoveto{\pgfqpoint{4.080411in}{1.692251in}}%
\pgfpathlineto{\pgfqpoint{4.094032in}{1.696746in}}%
\pgfpathlineto{\pgfqpoint{4.107663in}{1.701399in}}%
\pgfpathlineto{\pgfqpoint{4.121305in}{1.706211in}}%
\pgfpathlineto{\pgfqpoint{4.134959in}{1.711182in}}%
\pgfpathlineto{\pgfqpoint{4.142813in}{1.724950in}}%
\pgfpathlineto{\pgfqpoint{4.150662in}{1.738711in}}%
\pgfpathlineto{\pgfqpoint{4.158507in}{1.752461in}}%
\pgfpathlineto{\pgfqpoint{4.166347in}{1.766194in}}%
\pgfpathlineto{\pgfqpoint{4.152695in}{1.760830in}}%
\pgfpathlineto{\pgfqpoint{4.139054in}{1.755624in}}%
\pgfpathlineto{\pgfqpoint{4.125424in}{1.750578in}}%
\pgfpathlineto{\pgfqpoint{4.111806in}{1.745691in}}%
\pgfpathlineto{\pgfqpoint{4.103964in}{1.732340in}}%
\pgfpathlineto{\pgfqpoint{4.096117in}{1.718980in}}%
\pgfpathlineto{\pgfqpoint{4.088266in}{1.705616in}}%
\pgfpathlineto{\pgfqpoint{4.080411in}{1.692251in}}%
\pgfpathclose%
\pgfusepath{fill}%
\end{pgfscope}%
\begin{pgfscope}%
\pgfpathrectangle{\pgfqpoint{1.254980in}{0.150000in}}{\pgfqpoint{5.490039in}{5.490039in}}%
\pgfusepath{clip}%
\pgfsetbuttcap%
\pgfsetroundjoin%
\definecolor{currentfill}{rgb}{0.276022,0.044167,0.370164}%
\pgfsetfillcolor{currentfill}%
\pgfsetfillopacity{0.700000}%
\pgfsetlinewidth{0.000000pt}%
\definecolor{currentstroke}{rgb}{0.000000,0.000000,0.000000}%
\pgfsetstrokecolor{currentstroke}%
\pgfsetdash{}{0pt}%
\pgfpathmoveto{\pgfqpoint{3.705292in}{1.418723in}}%
\pgfpathlineto{\pgfqpoint{3.718800in}{1.418294in}}%
\pgfpathlineto{\pgfqpoint{3.732316in}{1.418026in}}%
\pgfpathlineto{\pgfqpoint{3.745839in}{1.417918in}}%
\pgfpathlineto{\pgfqpoint{3.759369in}{1.417970in}}%
\pgfpathlineto{\pgfqpoint{3.767342in}{1.428857in}}%
\pgfpathlineto{\pgfqpoint{3.775309in}{1.439852in}}%
\pgfpathlineto{\pgfqpoint{3.783270in}{1.450951in}}%
\pgfpathlineto{\pgfqpoint{3.791225in}{1.462146in}}%
\pgfpathlineto{\pgfqpoint{3.777705in}{1.461566in}}%
\pgfpathlineto{\pgfqpoint{3.764193in}{1.461146in}}%
\pgfpathlineto{\pgfqpoint{3.750688in}{1.460887in}}%
\pgfpathlineto{\pgfqpoint{3.737192in}{1.460789in}}%
\pgfpathlineto{\pgfqpoint{3.729226in}{1.450110in}}%
\pgfpathlineto{\pgfqpoint{3.721254in}{1.439536in}}%
\pgfpathlineto{\pgfqpoint{3.713276in}{1.429072in}}%
\pgfpathlineto{\pgfqpoint{3.705292in}{1.418723in}}%
\pgfpathclose%
\pgfusepath{fill}%
\end{pgfscope}%
\begin{pgfscope}%
\pgfpathrectangle{\pgfqpoint{1.254980in}{0.150000in}}{\pgfqpoint{5.490039in}{5.490039in}}%
\pgfusepath{clip}%
\pgfsetbuttcap%
\pgfsetroundjoin%
\definecolor{currentfill}{rgb}{0.280267,0.073417,0.397163}%
\pgfsetfillcolor{currentfill}%
\pgfsetfillopacity{0.700000}%
\pgfsetlinewidth{0.000000pt}%
\definecolor{currentstroke}{rgb}{0.000000,0.000000,0.000000}%
\pgfsetstrokecolor{currentstroke}%
\pgfsetdash{}{0pt}%
\pgfpathmoveto{\pgfqpoint{3.791225in}{1.462146in}}%
\pgfpathlineto{\pgfqpoint{3.804753in}{1.462887in}}%
\pgfpathlineto{\pgfqpoint{3.818289in}{1.463787in}}%
\pgfpathlineto{\pgfqpoint{3.831833in}{1.464847in}}%
\pgfpathlineto{\pgfqpoint{3.845386in}{1.466066in}}%
\pgfpathlineto{\pgfqpoint{3.853326in}{1.477866in}}%
\pgfpathlineto{\pgfqpoint{3.861262in}{1.489747in}}%
\pgfpathlineto{\pgfqpoint{3.869192in}{1.501704in}}%
\pgfpathlineto{\pgfqpoint{3.877117in}{1.513733in}}%
\pgfpathlineto{\pgfqpoint{3.863572in}{1.512013in}}%
\pgfpathlineto{\pgfqpoint{3.850036in}{1.510452in}}%
\pgfpathlineto{\pgfqpoint{3.836509in}{1.509051in}}%
\pgfpathlineto{\pgfqpoint{3.822990in}{1.507810in}}%
\pgfpathlineto{\pgfqpoint{3.815057in}{1.496272in}}%
\pgfpathlineto{\pgfqpoint{3.807118in}{1.484812in}}%
\pgfpathlineto{\pgfqpoint{3.799174in}{1.473435in}}%
\pgfpathlineto{\pgfqpoint{3.791225in}{1.462146in}}%
\pgfpathclose%
\pgfusepath{fill}%
\end{pgfscope}%
\begin{pgfscope}%
\pgfpathrectangle{\pgfqpoint{1.254980in}{0.150000in}}{\pgfqpoint{5.490039in}{5.490039in}}%
\pgfusepath{clip}%
\pgfsetbuttcap%
\pgfsetroundjoin%
\definecolor{currentfill}{rgb}{0.267004,0.004874,0.329415}%
\pgfsetfillcolor{currentfill}%
\pgfsetfillopacity{0.700000}%
\pgfsetlinewidth{0.000000pt}%
\definecolor{currentstroke}{rgb}{0.000000,0.000000,0.000000}%
\pgfsetstrokecolor{currentstroke}%
\pgfsetdash{}{0pt}%
\pgfpathmoveto{\pgfqpoint{3.392807in}{1.362412in}}%
\pgfpathlineto{\pgfqpoint{3.406283in}{1.357617in}}%
\pgfpathlineto{\pgfqpoint{3.419763in}{1.352988in}}%
\pgfpathlineto{\pgfqpoint{3.433246in}{1.348527in}}%
\pgfpathlineto{\pgfqpoint{3.446734in}{1.344231in}}%
\pgfpathlineto{\pgfqpoint{3.454862in}{1.350911in}}%
\pgfpathlineto{\pgfqpoint{3.462981in}{1.357794in}}%
\pgfpathlineto{\pgfqpoint{3.471091in}{1.364876in}}%
\pgfpathlineto{\pgfqpoint{3.479192in}{1.372148in}}%
\pgfpathlineto{\pgfqpoint{3.465726in}{1.375834in}}%
\pgfpathlineto{\pgfqpoint{3.452265in}{1.379685in}}%
\pgfpathlineto{\pgfqpoint{3.438808in}{1.383703in}}%
\pgfpathlineto{\pgfqpoint{3.425355in}{1.387888in}}%
\pgfpathlineto{\pgfqpoint{3.417232in}{1.381215in}}%
\pgfpathlineto{\pgfqpoint{3.409100in}{1.374740in}}%
\pgfpathlineto{\pgfqpoint{3.400959in}{1.368471in}}%
\pgfpathlineto{\pgfqpoint{3.392807in}{1.362412in}}%
\pgfpathclose%
\pgfusepath{fill}%
\end{pgfscope}%
\begin{pgfscope}%
\pgfpathrectangle{\pgfqpoint{1.254980in}{0.150000in}}{\pgfqpoint{5.490039in}{5.490039in}}%
\pgfusepath{clip}%
\pgfsetbuttcap%
\pgfsetroundjoin%
\definecolor{currentfill}{rgb}{0.278791,0.062145,0.386592}%
\pgfsetfillcolor{currentfill}%
\pgfsetfillopacity{0.700000}%
\pgfsetlinewidth{0.000000pt}%
\definecolor{currentstroke}{rgb}{0.000000,0.000000,0.000000}%
\pgfsetstrokecolor{currentstroke}%
\pgfsetdash{}{0pt}%
\pgfpathmoveto{\pgfqpoint{3.056997in}{1.480406in}}%
\pgfpathlineto{\pgfqpoint{3.070502in}{1.470826in}}%
\pgfpathlineto{\pgfqpoint{3.084006in}{1.461427in}}%
\pgfpathlineto{\pgfqpoint{3.097511in}{1.452208in}}%
\pgfpathlineto{\pgfqpoint{3.111015in}{1.443168in}}%
\pgfpathlineto{\pgfqpoint{3.119382in}{1.444583in}}%
\pgfpathlineto{\pgfqpoint{3.127736in}{1.446291in}}%
\pgfpathlineto{\pgfqpoint{3.136075in}{1.448284in}}%
\pgfpathlineto{\pgfqpoint{3.144400in}{1.450557in}}%
\pgfpathlineto{\pgfqpoint{3.130931in}{1.458924in}}%
\pgfpathlineto{\pgfqpoint{3.117463in}{1.467470in}}%
\pgfpathlineto{\pgfqpoint{3.103995in}{1.476195in}}%
\pgfpathlineto{\pgfqpoint{3.090527in}{1.485101in}}%
\pgfpathlineto{\pgfqpoint{3.082166in}{1.483490in}}%
\pgfpathlineto{\pgfqpoint{3.073791in}{1.482166in}}%
\pgfpathlineto{\pgfqpoint{3.065402in}{1.481136in}}%
\pgfpathlineto{\pgfqpoint{3.056997in}{1.480406in}}%
\pgfpathclose%
\pgfusepath{fill}%
\end{pgfscope}%
\begin{pgfscope}%
\pgfpathrectangle{\pgfqpoint{1.254980in}{0.150000in}}{\pgfqpoint{5.490039in}{5.490039in}}%
\pgfusepath{clip}%
\pgfsetbuttcap%
\pgfsetroundjoin%
\definecolor{currentfill}{rgb}{0.296479,0.761561,0.424223}%
\pgfsetfillcolor{currentfill}%
\pgfsetfillopacity{0.700000}%
\pgfsetlinewidth{0.000000pt}%
\definecolor{currentstroke}{rgb}{0.000000,0.000000,0.000000}%
\pgfsetstrokecolor{currentstroke}%
\pgfsetdash{}{0pt}%
\pgfpathmoveto{\pgfqpoint{5.448539in}{3.166838in}}%
\pgfpathlineto{\pgfqpoint{5.462980in}{3.181903in}}%
\pgfpathlineto{\pgfqpoint{5.477444in}{3.197133in}}%
\pgfpathlineto{\pgfqpoint{5.491928in}{3.212527in}}%
\pgfpathlineto{\pgfqpoint{5.506434in}{3.228086in}}%
\pgfpathlineto{\pgfqpoint{5.513681in}{3.232705in}}%
\pgfpathlineto{\pgfqpoint{5.520918in}{3.237170in}}%
\pgfpathlineto{\pgfqpoint{5.528143in}{3.241487in}}%
\pgfpathlineto{\pgfqpoint{5.535359in}{3.245656in}}%
\pgfpathlineto{\pgfqpoint{5.520867in}{3.230334in}}%
\pgfpathlineto{\pgfqpoint{5.506397in}{3.215177in}}%
\pgfpathlineto{\pgfqpoint{5.491948in}{3.200184in}}%
\pgfpathlineto{\pgfqpoint{5.477520in}{3.185355in}}%
\pgfpathlineto{\pgfqpoint{5.470290in}{3.180938in}}%
\pgfpathlineto{\pgfqpoint{5.463049in}{3.176381in}}%
\pgfpathlineto{\pgfqpoint{5.455799in}{3.171682in}}%
\pgfpathlineto{\pgfqpoint{5.448539in}{3.166838in}}%
\pgfpathclose%
\pgfusepath{fill}%
\end{pgfscope}%
\begin{pgfscope}%
\pgfpathrectangle{\pgfqpoint{1.254980in}{0.150000in}}{\pgfqpoint{5.490039in}{5.490039in}}%
\pgfusepath{clip}%
\pgfsetbuttcap%
\pgfsetroundjoin%
\definecolor{currentfill}{rgb}{0.269944,0.014625,0.341379}%
\pgfsetfillcolor{currentfill}%
\pgfsetfillopacity{0.700000}%
\pgfsetlinewidth{0.000000pt}%
\definecolor{currentstroke}{rgb}{0.000000,0.000000,0.000000}%
\pgfsetstrokecolor{currentstroke}%
\pgfsetdash{}{0pt}%
\pgfpathmoveto{\pgfqpoint{3.252189in}{1.389967in}}%
\pgfpathlineto{\pgfqpoint{3.265669in}{1.383176in}}%
\pgfpathlineto{\pgfqpoint{3.279152in}{1.376556in}}%
\pgfpathlineto{\pgfqpoint{3.292637in}{1.370107in}}%
\pgfpathlineto{\pgfqpoint{3.306124in}{1.363829in}}%
\pgfpathlineto{\pgfqpoint{3.314344in}{1.368332in}}%
\pgfpathlineto{\pgfqpoint{3.322552in}{1.373077in}}%
\pgfpathlineto{\pgfqpoint{3.330750in}{1.378060in}}%
\pgfpathlineto{\pgfqpoint{3.338937in}{1.383273in}}%
\pgfpathlineto{\pgfqpoint{3.325477in}{1.388911in}}%
\pgfpathlineto{\pgfqpoint{3.312020in}{1.394720in}}%
\pgfpathlineto{\pgfqpoint{3.298566in}{1.400699in}}%
\pgfpathlineto{\pgfqpoint{3.285114in}{1.406850in}}%
\pgfpathlineto{\pgfqpoint{3.276900in}{1.402266in}}%
\pgfpathlineto{\pgfqpoint{3.268675in}{1.397920in}}%
\pgfpathlineto{\pgfqpoint{3.260438in}{1.393818in}}%
\pgfpathlineto{\pgfqpoint{3.252189in}{1.389967in}}%
\pgfpathclose%
\pgfusepath{fill}%
\end{pgfscope}%
\begin{pgfscope}%
\pgfpathrectangle{\pgfqpoint{1.254980in}{0.150000in}}{\pgfqpoint{5.490039in}{5.490039in}}%
\pgfusepath{clip}%
\pgfsetbuttcap%
\pgfsetroundjoin%
\definecolor{currentfill}{rgb}{0.272594,0.025563,0.353093}%
\pgfsetfillcolor{currentfill}%
\pgfsetfillopacity{0.700000}%
\pgfsetlinewidth{0.000000pt}%
\definecolor{currentstroke}{rgb}{0.000000,0.000000,0.000000}%
\pgfsetstrokecolor{currentstroke}%
\pgfsetdash{}{0pt}%
\pgfpathmoveto{\pgfqpoint{3.619269in}{1.384128in}}%
\pgfpathlineto{\pgfqpoint{3.632764in}{1.382499in}}%
\pgfpathlineto{\pgfqpoint{3.646266in}{1.381031in}}%
\pgfpathlineto{\pgfqpoint{3.659775in}{1.379725in}}%
\pgfpathlineto{\pgfqpoint{3.673290in}{1.378580in}}%
\pgfpathlineto{\pgfqpoint{3.681300in}{1.388417in}}%
\pgfpathlineto{\pgfqpoint{3.689304in}{1.398390in}}%
\pgfpathlineto{\pgfqpoint{3.697301in}{1.408494in}}%
\pgfpathlineto{\pgfqpoint{3.705292in}{1.418723in}}%
\pgfpathlineto{\pgfqpoint{3.691790in}{1.419313in}}%
\pgfpathlineto{\pgfqpoint{3.678296in}{1.420064in}}%
\pgfpathlineto{\pgfqpoint{3.664808in}{1.420977in}}%
\pgfpathlineto{\pgfqpoint{3.651327in}{1.422052in}}%
\pgfpathlineto{\pgfqpoint{3.643323in}{1.412367in}}%
\pgfpathlineto{\pgfqpoint{3.635312in}{1.402815in}}%
\pgfpathlineto{\pgfqpoint{3.627294in}{1.393400in}}%
\pgfpathlineto{\pgfqpoint{3.619269in}{1.384128in}}%
\pgfpathclose%
\pgfusepath{fill}%
\end{pgfscope}%
\begin{pgfscope}%
\pgfpathrectangle{\pgfqpoint{1.254980in}{0.150000in}}{\pgfqpoint{5.490039in}{5.490039in}}%
\pgfusepath{clip}%
\pgfsetbuttcap%
\pgfsetroundjoin%
\definecolor{currentfill}{rgb}{0.241237,0.296485,0.539709}%
\pgfsetfillcolor{currentfill}%
\pgfsetfillopacity{0.700000}%
\pgfsetlinewidth{0.000000pt}%
\definecolor{currentstroke}{rgb}{0.000000,0.000000,0.000000}%
\pgfsetstrokecolor{currentstroke}%
\pgfsetdash{}{0pt}%
\pgfpathmoveto{\pgfqpoint{4.283650in}{1.900976in}}%
\pgfpathlineto{\pgfqpoint{4.297364in}{1.907842in}}%
\pgfpathlineto{\pgfqpoint{4.311091in}{1.914866in}}%
\pgfpathlineto{\pgfqpoint{4.324832in}{1.922050in}}%
\pgfpathlineto{\pgfqpoint{4.338586in}{1.929393in}}%
\pgfpathlineto{\pgfqpoint{4.346389in}{1.943483in}}%
\pgfpathlineto{\pgfqpoint{4.354188in}{1.957511in}}%
\pgfpathlineto{\pgfqpoint{4.361983in}{1.971474in}}%
\pgfpathlineto{\pgfqpoint{4.369773in}{1.985371in}}%
\pgfpathlineto{\pgfqpoint{4.356017in}{1.977717in}}%
\pgfpathlineto{\pgfqpoint{4.342275in}{1.970222in}}%
\pgfpathlineto{\pgfqpoint{4.328547in}{1.962886in}}%
\pgfpathlineto{\pgfqpoint{4.314832in}{1.955711in}}%
\pgfpathlineto{\pgfqpoint{4.307043in}{1.942114in}}%
\pgfpathlineto{\pgfqpoint{4.299249in}{1.928458in}}%
\pgfpathlineto{\pgfqpoint{4.291452in}{1.914744in}}%
\pgfpathlineto{\pgfqpoint{4.283650in}{1.900976in}}%
\pgfpathclose%
\pgfusepath{fill}%
\end{pgfscope}%
\begin{pgfscope}%
\pgfpathrectangle{\pgfqpoint{1.254980in}{0.150000in}}{\pgfqpoint{5.490039in}{5.490039in}}%
\pgfusepath{clip}%
\pgfsetbuttcap%
\pgfsetroundjoin%
\definecolor{currentfill}{rgb}{0.282910,0.105393,0.426902}%
\pgfsetfillcolor{currentfill}%
\pgfsetfillopacity{0.700000}%
\pgfsetlinewidth{0.000000pt}%
\definecolor{currentstroke}{rgb}{0.000000,0.000000,0.000000}%
\pgfsetstrokecolor{currentstroke}%
\pgfsetdash{}{0pt}%
\pgfpathmoveto{\pgfqpoint{3.877117in}{1.513733in}}%
\pgfpathlineto{\pgfqpoint{3.890671in}{1.515613in}}%
\pgfpathlineto{\pgfqpoint{3.904233in}{1.517652in}}%
\pgfpathlineto{\pgfqpoint{3.917805in}{1.519850in}}%
\pgfpathlineto{\pgfqpoint{3.931386in}{1.522206in}}%
\pgfpathlineto{\pgfqpoint{3.939300in}{1.534787in}}%
\pgfpathlineto{\pgfqpoint{3.947208in}{1.547423in}}%
\pgfpathlineto{\pgfqpoint{3.955112in}{1.560109in}}%
\pgfpathlineto{\pgfqpoint{3.963011in}{1.572842in}}%
\pgfpathlineto{\pgfqpoint{3.949436in}{1.570010in}}%
\pgfpathlineto{\pgfqpoint{3.935870in}{1.567338in}}%
\pgfpathlineto{\pgfqpoint{3.922313in}{1.564825in}}%
\pgfpathlineto{\pgfqpoint{3.908766in}{1.562471in}}%
\pgfpathlineto{\pgfqpoint{3.900861in}{1.550202in}}%
\pgfpathlineto{\pgfqpoint{3.892952in}{1.537986in}}%
\pgfpathlineto{\pgfqpoint{3.885037in}{1.525829in}}%
\pgfpathlineto{\pgfqpoint{3.877117in}{1.513733in}}%
\pgfpathclose%
\pgfusepath{fill}%
\end{pgfscope}%
\begin{pgfscope}%
\pgfpathrectangle{\pgfqpoint{1.254980in}{0.150000in}}{\pgfqpoint{5.490039in}{5.490039in}}%
\pgfusepath{clip}%
\pgfsetbuttcap%
\pgfsetroundjoin%
\definecolor{currentfill}{rgb}{0.255645,0.260703,0.528312}%
\pgfsetfillcolor{currentfill}%
\pgfsetfillopacity{0.700000}%
\pgfsetlinewidth{0.000000pt}%
\definecolor{currentstroke}{rgb}{0.000000,0.000000,0.000000}%
\pgfsetstrokecolor{currentstroke}%
\pgfsetdash{}{0pt}%
\pgfpathmoveto{\pgfqpoint{2.588335in}{1.911397in}}%
\pgfpathlineto{\pgfqpoint{2.601999in}{1.894761in}}%
\pgfpathlineto{\pgfqpoint{2.615657in}{1.878344in}}%
\pgfpathlineto{\pgfqpoint{2.629308in}{1.862143in}}%
\pgfpathlineto{\pgfqpoint{2.642952in}{1.846157in}}%
\pgfpathlineto{\pgfqpoint{2.651757in}{1.840518in}}%
\pgfpathlineto{\pgfqpoint{2.660540in}{1.835266in}}%
\pgfpathlineto{\pgfqpoint{2.669301in}{1.830394in}}%
\pgfpathlineto{\pgfqpoint{2.678041in}{1.825895in}}%
\pgfpathlineto{\pgfqpoint{2.664451in}{1.841153in}}%
\pgfpathlineto{\pgfqpoint{2.650855in}{1.856624in}}%
\pgfpathlineto{\pgfqpoint{2.637253in}{1.872312in}}%
\pgfpathlineto{\pgfqpoint{2.623645in}{1.888216in}}%
\pgfpathlineto{\pgfqpoint{2.614851in}{1.893433in}}%
\pgfpathlineto{\pgfqpoint{2.606035in}{1.899030in}}%
\pgfpathlineto{\pgfqpoint{2.597196in}{1.905015in}}%
\pgfpathlineto{\pgfqpoint{2.588335in}{1.911397in}}%
\pgfpathclose%
\pgfusepath{fill}%
\end{pgfscope}%
\begin{pgfscope}%
\pgfpathrectangle{\pgfqpoint{1.254980in}{0.150000in}}{\pgfqpoint{5.490039in}{5.490039in}}%
\pgfusepath{clip}%
\pgfsetbuttcap%
\pgfsetroundjoin%
\definecolor{currentfill}{rgb}{0.265145,0.232956,0.516599}%
\pgfsetfillcolor{currentfill}%
\pgfsetfillopacity{0.700000}%
\pgfsetlinewidth{0.000000pt}%
\definecolor{currentstroke}{rgb}{0.000000,0.000000,0.000000}%
\pgfsetstrokecolor{currentstroke}%
\pgfsetdash{}{0pt}%
\pgfpathmoveto{\pgfqpoint{2.642952in}{1.846157in}}%
\pgfpathlineto{\pgfqpoint{2.656591in}{1.830385in}}%
\pgfpathlineto{\pgfqpoint{2.670223in}{1.814825in}}%
\pgfpathlineto{\pgfqpoint{2.683850in}{1.799475in}}%
\pgfpathlineto{\pgfqpoint{2.697471in}{1.784336in}}%
\pgfpathlineto{\pgfqpoint{2.706222in}{1.779435in}}%
\pgfpathlineto{\pgfqpoint{2.714951in}{1.774913in}}%
\pgfpathlineto{\pgfqpoint{2.723660in}{1.770762in}}%
\pgfpathlineto{\pgfqpoint{2.732348in}{1.766975in}}%
\pgfpathlineto{\pgfqpoint{2.718779in}{1.781391in}}%
\pgfpathlineto{\pgfqpoint{2.705205in}{1.796015in}}%
\pgfpathlineto{\pgfqpoint{2.691626in}{1.810849in}}%
\pgfpathlineto{\pgfqpoint{2.678041in}{1.825895in}}%
\pgfpathlineto{\pgfqpoint{2.669301in}{1.830394in}}%
\pgfpathlineto{\pgfqpoint{2.660540in}{1.835266in}}%
\pgfpathlineto{\pgfqpoint{2.651757in}{1.840518in}}%
\pgfpathlineto{\pgfqpoint{2.642952in}{1.846157in}}%
\pgfpathclose%
\pgfusepath{fill}%
\end{pgfscope}%
\begin{pgfscope}%
\pgfpathrectangle{\pgfqpoint{1.254980in}{0.150000in}}{\pgfqpoint{5.490039in}{5.490039in}}%
\pgfusepath{clip}%
\pgfsetbuttcap%
\pgfsetroundjoin%
\definecolor{currentfill}{rgb}{0.153364,0.497000,0.557724}%
\pgfsetfillcolor{currentfill}%
\pgfsetfillopacity{0.700000}%
\pgfsetlinewidth{0.000000pt}%
\definecolor{currentstroke}{rgb}{0.000000,0.000000,0.000000}%
\pgfsetstrokecolor{currentstroke}%
\pgfsetdash{}{0pt}%
\pgfpathmoveto{\pgfqpoint{4.721574in}{2.412557in}}%
\pgfpathlineto{\pgfqpoint{4.735536in}{2.423565in}}%
\pgfpathlineto{\pgfqpoint{4.749515in}{2.434734in}}%
\pgfpathlineto{\pgfqpoint{4.763510in}{2.446066in}}%
\pgfpathlineto{\pgfqpoint{4.777523in}{2.457559in}}%
\pgfpathlineto{\pgfqpoint{4.785187in}{2.469590in}}%
\pgfpathlineto{\pgfqpoint{4.792844in}{2.481479in}}%
\pgfpathlineto{\pgfqpoint{4.800495in}{2.493227in}}%
\pgfpathlineto{\pgfqpoint{4.808140in}{2.504831in}}%
\pgfpathlineto{\pgfqpoint{4.794126in}{2.493230in}}%
\pgfpathlineto{\pgfqpoint{4.780130in}{2.481791in}}%
\pgfpathlineto{\pgfqpoint{4.766151in}{2.470514in}}%
\pgfpathlineto{\pgfqpoint{4.752188in}{2.459399in}}%
\pgfpathlineto{\pgfqpoint{4.744544in}{2.447891in}}%
\pgfpathlineto{\pgfqpoint{4.736894in}{2.436247in}}%
\pgfpathlineto{\pgfqpoint{4.729237in}{2.424469in}}%
\pgfpathlineto{\pgfqpoint{4.721574in}{2.412557in}}%
\pgfpathclose%
\pgfusepath{fill}%
\end{pgfscope}%
\begin{pgfscope}%
\pgfpathrectangle{\pgfqpoint{1.254980in}{0.150000in}}{\pgfqpoint{5.490039in}{5.490039in}}%
\pgfusepath{clip}%
\pgfsetbuttcap%
\pgfsetroundjoin%
\definecolor{currentfill}{rgb}{0.120638,0.625828,0.533488}%
\pgfsetfillcolor{currentfill}%
\pgfsetfillopacity{0.700000}%
\pgfsetlinewidth{0.000000pt}%
\definecolor{currentstroke}{rgb}{0.000000,0.000000,0.000000}%
\pgfsetstrokecolor{currentstroke}%
\pgfsetdash{}{0pt}%
\pgfpathmoveto{\pgfqpoint{5.042193in}{2.770531in}}%
\pgfpathlineto{\pgfqpoint{5.056363in}{2.783723in}}%
\pgfpathlineto{\pgfqpoint{5.070551in}{2.797078in}}%
\pgfpathlineto{\pgfqpoint{5.084759in}{2.810597in}}%
\pgfpathlineto{\pgfqpoint{5.098985in}{2.824280in}}%
\pgfpathlineto{\pgfqpoint{5.106497in}{2.833332in}}%
\pgfpathlineto{\pgfqpoint{5.114000in}{2.842221in}}%
\pgfpathlineto{\pgfqpoint{5.121494in}{2.850946in}}%
\pgfpathlineto{\pgfqpoint{5.128979in}{2.859510in}}%
\pgfpathlineto{\pgfqpoint{5.114757in}{2.845873in}}%
\pgfpathlineto{\pgfqpoint{5.100554in}{2.832401in}}%
\pgfpathlineto{\pgfqpoint{5.086369in}{2.819091in}}%
\pgfpathlineto{\pgfqpoint{5.072204in}{2.805945in}}%
\pgfpathlineto{\pgfqpoint{5.064714in}{2.797324in}}%
\pgfpathlineto{\pgfqpoint{5.057215in}{2.788549in}}%
\pgfpathlineto{\pgfqpoint{5.049708in}{2.779618in}}%
\pgfpathlineto{\pgfqpoint{5.042193in}{2.770531in}}%
\pgfpathclose%
\pgfusepath{fill}%
\end{pgfscope}%
\begin{pgfscope}%
\pgfpathrectangle{\pgfqpoint{1.254980in}{0.150000in}}{\pgfqpoint{5.490039in}{5.490039in}}%
\pgfusepath{clip}%
\pgfsetbuttcap%
\pgfsetroundjoin%
\definecolor{currentfill}{rgb}{0.243113,0.292092,0.538516}%
\pgfsetfillcolor{currentfill}%
\pgfsetfillopacity{0.700000}%
\pgfsetlinewidth{0.000000pt}%
\definecolor{currentstroke}{rgb}{0.000000,0.000000,0.000000}%
\pgfsetstrokecolor{currentstroke}%
\pgfsetdash{}{0pt}%
\pgfpathmoveto{\pgfqpoint{2.533605in}{1.980151in}}%
\pgfpathlineto{\pgfqpoint{2.547299in}{1.962628in}}%
\pgfpathlineto{\pgfqpoint{2.560985in}{1.945329in}}%
\pgfpathlineto{\pgfqpoint{2.574663in}{1.928252in}}%
\pgfpathlineto{\pgfqpoint{2.588335in}{1.911397in}}%
\pgfpathlineto{\pgfqpoint{2.597196in}{1.905015in}}%
\pgfpathlineto{\pgfqpoint{2.606035in}{1.899030in}}%
\pgfpathlineto{\pgfqpoint{2.614851in}{1.893433in}}%
\pgfpathlineto{\pgfqpoint{2.623645in}{1.888216in}}%
\pgfpathlineto{\pgfqpoint{2.610030in}{1.904339in}}%
\pgfpathlineto{\pgfqpoint{2.596409in}{1.920682in}}%
\pgfpathlineto{\pgfqpoint{2.582780in}{1.937246in}}%
\pgfpathlineto{\pgfqpoint{2.569145in}{1.954034in}}%
\pgfpathlineto{\pgfqpoint{2.560295in}{1.959971in}}%
\pgfpathlineto{\pgfqpoint{2.551422in}{1.966298in}}%
\pgfpathlineto{\pgfqpoint{2.542526in}{1.973022in}}%
\pgfpathlineto{\pgfqpoint{2.533605in}{1.980151in}}%
\pgfpathclose%
\pgfusepath{fill}%
\end{pgfscope}%
\begin{pgfscope}%
\pgfpathrectangle{\pgfqpoint{1.254980in}{0.150000in}}{\pgfqpoint{5.490039in}{5.490039in}}%
\pgfusepath{clip}%
\pgfsetbuttcap%
\pgfsetroundjoin%
\definecolor{currentfill}{rgb}{0.273006,0.204520,0.501721}%
\pgfsetfillcolor{currentfill}%
\pgfsetfillopacity{0.700000}%
\pgfsetlinewidth{0.000000pt}%
\definecolor{currentstroke}{rgb}{0.000000,0.000000,0.000000}%
\pgfsetstrokecolor{currentstroke}%
\pgfsetdash{}{0pt}%
\pgfpathmoveto{\pgfqpoint{2.697471in}{1.784336in}}%
\pgfpathlineto{\pgfqpoint{2.711087in}{1.769404in}}%
\pgfpathlineto{\pgfqpoint{2.724698in}{1.754679in}}%
\pgfpathlineto{\pgfqpoint{2.738304in}{1.740159in}}%
\pgfpathlineto{\pgfqpoint{2.751905in}{1.725844in}}%
\pgfpathlineto{\pgfqpoint{2.760604in}{1.721678in}}%
\pgfpathlineto{\pgfqpoint{2.769281in}{1.717881in}}%
\pgfpathlineto{\pgfqpoint{2.777939in}{1.714448in}}%
\pgfpathlineto{\pgfqpoint{2.786578in}{1.711370in}}%
\pgfpathlineto{\pgfqpoint{2.773027in}{1.724965in}}%
\pgfpathlineto{\pgfqpoint{2.759472in}{1.738763in}}%
\pgfpathlineto{\pgfqpoint{2.745912in}{1.752766in}}%
\pgfpathlineto{\pgfqpoint{2.732348in}{1.766975in}}%
\pgfpathlineto{\pgfqpoint{2.723660in}{1.770762in}}%
\pgfpathlineto{\pgfqpoint{2.714951in}{1.774913in}}%
\pgfpathlineto{\pgfqpoint{2.706222in}{1.779435in}}%
\pgfpathlineto{\pgfqpoint{2.697471in}{1.784336in}}%
\pgfpathclose%
\pgfusepath{fill}%
\end{pgfscope}%
\begin{pgfscope}%
\pgfpathrectangle{\pgfqpoint{1.254980in}{0.150000in}}{\pgfqpoint{5.490039in}{5.490039in}}%
\pgfusepath{clip}%
\pgfsetbuttcap%
\pgfsetroundjoin%
\definecolor{currentfill}{rgb}{0.180653,0.701402,0.488189}%
\pgfsetfillcolor{currentfill}%
\pgfsetfillopacity{0.700000}%
\pgfsetlinewidth{0.000000pt}%
\definecolor{currentstroke}{rgb}{0.000000,0.000000,0.000000}%
\pgfsetstrokecolor{currentstroke}%
\pgfsetdash{}{0pt}%
\pgfpathmoveto{\pgfqpoint{5.245583in}{2.977750in}}%
\pgfpathlineto{\pgfqpoint{5.259892in}{2.992019in}}%
\pgfpathlineto{\pgfqpoint{5.274220in}{3.006452in}}%
\pgfpathlineto{\pgfqpoint{5.288569in}{3.021050in}}%
\pgfpathlineto{\pgfqpoint{5.302938in}{3.035812in}}%
\pgfpathlineto{\pgfqpoint{5.310328in}{3.042712in}}%
\pgfpathlineto{\pgfqpoint{5.317709in}{3.049448in}}%
\pgfpathlineto{\pgfqpoint{5.325080in}{3.056022in}}%
\pgfpathlineto{\pgfqpoint{5.332441in}{3.062435in}}%
\pgfpathlineto{\pgfqpoint{5.318080in}{3.047815in}}%
\pgfpathlineto{\pgfqpoint{5.303741in}{3.033358in}}%
\pgfpathlineto{\pgfqpoint{5.289421in}{3.019066in}}%
\pgfpathlineto{\pgfqpoint{5.275121in}{3.004937in}}%
\pgfpathlineto{\pgfqpoint{5.267751in}{2.998371in}}%
\pgfpathlineto{\pgfqpoint{5.260371in}{2.991652in}}%
\pgfpathlineto{\pgfqpoint{5.252982in}{2.984779in}}%
\pgfpathlineto{\pgfqpoint{5.245583in}{2.977750in}}%
\pgfpathclose%
\pgfusepath{fill}%
\end{pgfscope}%
\begin{pgfscope}%
\pgfpathrectangle{\pgfqpoint{1.254980in}{0.150000in}}{\pgfqpoint{5.490039in}{5.490039in}}%
\pgfusepath{clip}%
\pgfsetbuttcap%
\pgfsetroundjoin%
\definecolor{currentfill}{rgb}{0.140536,0.530132,0.555659}%
\pgfsetfillcolor{currentfill}%
\pgfsetfillopacity{0.700000}%
\pgfsetlinewidth{0.000000pt}%
\definecolor{currentstroke}{rgb}{0.000000,0.000000,0.000000}%
\pgfsetstrokecolor{currentstroke}%
\pgfsetdash{}{0pt}%
\pgfpathmoveto{\pgfqpoint{2.127599in}{2.620221in}}%
\pgfpathlineto{\pgfqpoint{2.141594in}{2.595258in}}%
\pgfpathlineto{\pgfqpoint{2.155575in}{2.570588in}}%
\pgfpathlineto{\pgfqpoint{2.169539in}{2.546207in}}%
\pgfpathlineto{\pgfqpoint{2.183489in}{2.522112in}}%
\pgfpathlineto{\pgfqpoint{2.192765in}{2.511416in}}%
\pgfpathlineto{\pgfqpoint{2.202012in}{2.501158in}}%
\pgfpathlineto{\pgfqpoint{2.211230in}{2.491330in}}%
\pgfpathlineto{\pgfqpoint{2.220420in}{2.481924in}}%
\pgfpathlineto{\pgfqpoint{2.206541in}{2.505274in}}%
\pgfpathlineto{\pgfqpoint{2.192649in}{2.528909in}}%
\pgfpathlineto{\pgfqpoint{2.178741in}{2.552832in}}%
\pgfpathlineto{\pgfqpoint{2.164819in}{2.577044in}}%
\pgfpathlineto{\pgfqpoint{2.155558in}{2.587183in}}%
\pgfpathlineto{\pgfqpoint{2.146268in}{2.597753in}}%
\pgfpathlineto{\pgfqpoint{2.136949in}{2.608763in}}%
\pgfpathlineto{\pgfqpoint{2.127599in}{2.620221in}}%
\pgfpathclose%
\pgfusepath{fill}%
\end{pgfscope}%
\begin{pgfscope}%
\pgfpathrectangle{\pgfqpoint{1.254980in}{0.150000in}}{\pgfqpoint{5.490039in}{5.490039in}}%
\pgfusepath{clip}%
\pgfsetbuttcap%
\pgfsetroundjoin%
\definecolor{currentfill}{rgb}{0.229739,0.322361,0.545706}%
\pgfsetfillcolor{currentfill}%
\pgfsetfillopacity{0.700000}%
\pgfsetlinewidth{0.000000pt}%
\definecolor{currentstroke}{rgb}{0.000000,0.000000,0.000000}%
\pgfsetstrokecolor{currentstroke}%
\pgfsetdash{}{0pt}%
\pgfpathmoveto{\pgfqpoint{2.478749in}{2.052522in}}%
\pgfpathlineto{\pgfqpoint{2.492476in}{2.034084in}}%
\pgfpathlineto{\pgfqpoint{2.506194in}{2.015878in}}%
\pgfpathlineto{\pgfqpoint{2.519903in}{1.997901in}}%
\pgfpathlineto{\pgfqpoint{2.533605in}{1.980151in}}%
\pgfpathlineto{\pgfqpoint{2.542526in}{1.973022in}}%
\pgfpathlineto{\pgfqpoint{2.551422in}{1.966298in}}%
\pgfpathlineto{\pgfqpoint{2.560295in}{1.959971in}}%
\pgfpathlineto{\pgfqpoint{2.569145in}{1.954034in}}%
\pgfpathlineto{\pgfqpoint{2.555502in}{1.971046in}}%
\pgfpathlineto{\pgfqpoint{2.541852in}{1.988284in}}%
\pgfpathlineto{\pgfqpoint{2.528194in}{2.005751in}}%
\pgfpathlineto{\pgfqpoint{2.514528in}{2.023448in}}%
\pgfpathlineto{\pgfqpoint{2.505620in}{2.030111in}}%
\pgfpathlineto{\pgfqpoint{2.496688in}{2.037173in}}%
\pgfpathlineto{\pgfqpoint{2.487731in}{2.044640in}}%
\pgfpathlineto{\pgfqpoint{2.478749in}{2.052522in}}%
\pgfpathclose%
\pgfusepath{fill}%
\end{pgfscope}%
\begin{pgfscope}%
\pgfpathrectangle{\pgfqpoint{1.254980in}{0.150000in}}{\pgfqpoint{5.490039in}{5.490039in}}%
\pgfusepath{clip}%
\pgfsetbuttcap%
\pgfsetroundjoin%
\definecolor{currentfill}{rgb}{0.268510,0.009605,0.335427}%
\pgfsetfillcolor{currentfill}%
\pgfsetfillopacity{0.700000}%
\pgfsetlinewidth{0.000000pt}%
\definecolor{currentstroke}{rgb}{0.000000,0.000000,0.000000}%
\pgfsetstrokecolor{currentstroke}%
\pgfsetdash{}{0pt}%
\pgfpathmoveto{\pgfqpoint{3.533102in}{1.359056in}}%
\pgfpathlineto{\pgfqpoint{3.546592in}{1.356193in}}%
\pgfpathlineto{\pgfqpoint{3.560088in}{1.353494in}}%
\pgfpathlineto{\pgfqpoint{3.573588in}{1.350957in}}%
\pgfpathlineto{\pgfqpoint{3.587095in}{1.348583in}}%
\pgfpathlineto{\pgfqpoint{3.595149in}{1.357227in}}%
\pgfpathlineto{\pgfqpoint{3.603197in}{1.366036in}}%
\pgfpathlineto{\pgfqpoint{3.611236in}{1.375005in}}%
\pgfpathlineto{\pgfqpoint{3.619269in}{1.384128in}}%
\pgfpathlineto{\pgfqpoint{3.605779in}{1.385920in}}%
\pgfpathlineto{\pgfqpoint{3.592295in}{1.387874in}}%
\pgfpathlineto{\pgfqpoint{3.578817in}{1.389991in}}%
\pgfpathlineto{\pgfqpoint{3.565345in}{1.392272in}}%
\pgfpathlineto{\pgfqpoint{3.557296in}{1.383721in}}%
\pgfpathlineto{\pgfqpoint{3.549240in}{1.375330in}}%
\pgfpathlineto{\pgfqpoint{3.541175in}{1.367107in}}%
\pgfpathlineto{\pgfqpoint{3.533102in}{1.359056in}}%
\pgfpathclose%
\pgfusepath{fill}%
\end{pgfscope}%
\begin{pgfscope}%
\pgfpathrectangle{\pgfqpoint{1.254980in}{0.150000in}}{\pgfqpoint{5.490039in}{5.490039in}}%
\pgfusepath{clip}%
\pgfsetbuttcap%
\pgfsetroundjoin%
\definecolor{currentfill}{rgb}{0.278012,0.180367,0.486697}%
\pgfsetfillcolor{currentfill}%
\pgfsetfillopacity{0.700000}%
\pgfsetlinewidth{0.000000pt}%
\definecolor{currentstroke}{rgb}{0.000000,0.000000,0.000000}%
\pgfsetstrokecolor{currentstroke}%
\pgfsetdash{}{0pt}%
\pgfpathmoveto{\pgfqpoint{2.751905in}{1.725844in}}%
\pgfpathlineto{\pgfqpoint{2.765502in}{1.711732in}}%
\pgfpathlineto{\pgfqpoint{2.779094in}{1.697821in}}%
\pgfpathlineto{\pgfqpoint{2.792682in}{1.684111in}}%
\pgfpathlineto{\pgfqpoint{2.806266in}{1.670600in}}%
\pgfpathlineto{\pgfqpoint{2.814915in}{1.667164in}}%
\pgfpathlineto{\pgfqpoint{2.823543in}{1.664090in}}%
\pgfpathlineto{\pgfqpoint{2.832152in}{1.661370in}}%
\pgfpathlineto{\pgfqpoint{2.840743in}{1.658997in}}%
\pgfpathlineto{\pgfqpoint{2.827207in}{1.671792in}}%
\pgfpathlineto{\pgfqpoint{2.813668in}{1.684784in}}%
\pgfpathlineto{\pgfqpoint{2.800125in}{1.697976in}}%
\pgfpathlineto{\pgfqpoint{2.786578in}{1.711370in}}%
\pgfpathlineto{\pgfqpoint{2.777939in}{1.714448in}}%
\pgfpathlineto{\pgfqpoint{2.769281in}{1.717881in}}%
\pgfpathlineto{\pgfqpoint{2.760604in}{1.721678in}}%
\pgfpathlineto{\pgfqpoint{2.751905in}{1.725844in}}%
\pgfpathclose%
\pgfusepath{fill}%
\end{pgfscope}%
\begin{pgfscope}%
\pgfpathrectangle{\pgfqpoint{1.254980in}{0.150000in}}{\pgfqpoint{5.490039in}{5.490039in}}%
\pgfusepath{clip}%
\pgfsetbuttcap%
\pgfsetroundjoin%
\definecolor{currentfill}{rgb}{0.263663,0.237631,0.518762}%
\pgfsetfillcolor{currentfill}%
\pgfsetfillopacity{0.700000}%
\pgfsetlinewidth{0.000000pt}%
\definecolor{currentstroke}{rgb}{0.000000,0.000000,0.000000}%
\pgfsetstrokecolor{currentstroke}%
\pgfsetdash{}{0pt}%
\pgfpathmoveto{\pgfqpoint{4.166347in}{1.766194in}}%
\pgfpathlineto{\pgfqpoint{4.180012in}{1.771718in}}%
\pgfpathlineto{\pgfqpoint{4.193689in}{1.777400in}}%
\pgfpathlineto{\pgfqpoint{4.207377in}{1.783242in}}%
\pgfpathlineto{\pgfqpoint{4.221078in}{1.789242in}}%
\pgfpathlineto{\pgfqpoint{4.228914in}{1.803334in}}%
\pgfpathlineto{\pgfqpoint{4.236746in}{1.817397in}}%
\pgfpathlineto{\pgfqpoint{4.244574in}{1.831427in}}%
\pgfpathlineto{\pgfqpoint{4.252398in}{1.845422in}}%
\pgfpathlineto{\pgfqpoint{4.238697in}{1.839055in}}%
\pgfpathlineto{\pgfqpoint{4.225008in}{1.832847in}}%
\pgfpathlineto{\pgfqpoint{4.211332in}{1.826798in}}%
\pgfpathlineto{\pgfqpoint{4.197668in}{1.820908in}}%
\pgfpathlineto{\pgfqpoint{4.189844in}{1.807270in}}%
\pgfpathlineto{\pgfqpoint{4.182016in}{1.793602in}}%
\pgfpathlineto{\pgfqpoint{4.174184in}{1.779910in}}%
\pgfpathlineto{\pgfqpoint{4.166347in}{1.766194in}}%
\pgfpathclose%
\pgfusepath{fill}%
\end{pgfscope}%
\begin{pgfscope}%
\pgfpathrectangle{\pgfqpoint{1.254980in}{0.150000in}}{\pgfqpoint{5.490039in}{5.490039in}}%
\pgfusepath{clip}%
\pgfsetbuttcap%
\pgfsetroundjoin%
\definecolor{currentfill}{rgb}{0.174274,0.445044,0.557792}%
\pgfsetfillcolor{currentfill}%
\pgfsetfillopacity{0.700000}%
\pgfsetlinewidth{0.000000pt}%
\definecolor{currentstroke}{rgb}{0.000000,0.000000,0.000000}%
\pgfsetstrokecolor{currentstroke}%
\pgfsetdash{}{0pt}%
\pgfpathmoveto{\pgfqpoint{4.604377in}{2.271356in}}%
\pgfpathlineto{\pgfqpoint{4.618272in}{2.281413in}}%
\pgfpathlineto{\pgfqpoint{4.632182in}{2.291631in}}%
\pgfpathlineto{\pgfqpoint{4.646109in}{2.302011in}}%
\pgfpathlineto{\pgfqpoint{4.660052in}{2.312552in}}%
\pgfpathlineto{\pgfqpoint{4.667762in}{2.325500in}}%
\pgfpathlineto{\pgfqpoint{4.675468in}{2.338323in}}%
\pgfpathlineto{\pgfqpoint{4.683167in}{2.351019in}}%
\pgfpathlineto{\pgfqpoint{4.690860in}{2.363587in}}%
\pgfpathlineto{\pgfqpoint{4.676916in}{2.352879in}}%
\pgfpathlineto{\pgfqpoint{4.662988in}{2.342332in}}%
\pgfpathlineto{\pgfqpoint{4.649076in}{2.331946in}}%
\pgfpathlineto{\pgfqpoint{4.635179in}{2.321722in}}%
\pgfpathlineto{\pgfqpoint{4.627487in}{2.309310in}}%
\pgfpathlineto{\pgfqpoint{4.619789in}{2.296777in}}%
\pgfpathlineto{\pgfqpoint{4.612086in}{2.284125in}}%
\pgfpathlineto{\pgfqpoint{4.604377in}{2.271356in}}%
\pgfpathclose%
\pgfusepath{fill}%
\end{pgfscope}%
\begin{pgfscope}%
\pgfpathrectangle{\pgfqpoint{1.254980in}{0.150000in}}{\pgfqpoint{5.490039in}{5.490039in}}%
\pgfusepath{clip}%
\pgfsetbuttcap%
\pgfsetroundjoin%
\definecolor{currentfill}{rgb}{0.282623,0.140926,0.457517}%
\pgfsetfillcolor{currentfill}%
\pgfsetfillopacity{0.700000}%
\pgfsetlinewidth{0.000000pt}%
\definecolor{currentstroke}{rgb}{0.000000,0.000000,0.000000}%
\pgfsetstrokecolor{currentstroke}%
\pgfsetdash{}{0pt}%
\pgfpathmoveto{\pgfqpoint{3.963011in}{1.572842in}}%
\pgfpathlineto{\pgfqpoint{3.976597in}{1.575833in}}%
\pgfpathlineto{\pgfqpoint{3.990192in}{1.578982in}}%
\pgfpathlineto{\pgfqpoint{4.003797in}{1.582290in}}%
\pgfpathlineto{\pgfqpoint{4.017413in}{1.585756in}}%
\pgfpathlineto{\pgfqpoint{4.025303in}{1.598989in}}%
\pgfpathlineto{\pgfqpoint{4.033189in}{1.612253in}}%
\pgfpathlineto{\pgfqpoint{4.041070in}{1.625544in}}%
\pgfpathlineto{\pgfqpoint{4.048947in}{1.638857in}}%
\pgfpathlineto{\pgfqpoint{4.035335in}{1.634942in}}%
\pgfpathlineto{\pgfqpoint{4.021734in}{1.631186in}}%
\pgfpathlineto{\pgfqpoint{4.008142in}{1.627589in}}%
\pgfpathlineto{\pgfqpoint{3.994562in}{1.624151in}}%
\pgfpathlineto{\pgfqpoint{3.986681in}{1.611276in}}%
\pgfpathlineto{\pgfqpoint{3.978796in}{1.598430in}}%
\pgfpathlineto{\pgfqpoint{3.970906in}{1.585617in}}%
\pgfpathlineto{\pgfqpoint{3.963011in}{1.572842in}}%
\pgfpathclose%
\pgfusepath{fill}%
\end{pgfscope}%
\begin{pgfscope}%
\pgfpathrectangle{\pgfqpoint{1.254980in}{0.150000in}}{\pgfqpoint{5.490039in}{5.490039in}}%
\pgfusepath{clip}%
\pgfsetbuttcap%
\pgfsetroundjoin%
\definecolor{currentfill}{rgb}{0.216210,0.351535,0.550627}%
\pgfsetfillcolor{currentfill}%
\pgfsetfillopacity{0.700000}%
\pgfsetlinewidth{0.000000pt}%
\definecolor{currentstroke}{rgb}{0.000000,0.000000,0.000000}%
\pgfsetstrokecolor{currentstroke}%
\pgfsetdash{}{0pt}%
\pgfpathmoveto{\pgfqpoint{2.423752in}{2.128618in}}%
\pgfpathlineto{\pgfqpoint{2.437515in}{2.109239in}}%
\pgfpathlineto{\pgfqpoint{2.451269in}{2.090097in}}%
\pgfpathlineto{\pgfqpoint{2.465014in}{2.071192in}}%
\pgfpathlineto{\pgfqpoint{2.478749in}{2.052522in}}%
\pgfpathlineto{\pgfqpoint{2.487731in}{2.044640in}}%
\pgfpathlineto{\pgfqpoint{2.496688in}{2.037173in}}%
\pgfpathlineto{\pgfqpoint{2.505620in}{2.030111in}}%
\pgfpathlineto{\pgfqpoint{2.514528in}{2.023448in}}%
\pgfpathlineto{\pgfqpoint{2.500853in}{2.041376in}}%
\pgfpathlineto{\pgfqpoint{2.487170in}{2.059537in}}%
\pgfpathlineto{\pgfqpoint{2.473479in}{2.077933in}}%
\pgfpathlineto{\pgfqpoint{2.459778in}{2.096566in}}%
\pgfpathlineto{\pgfqpoint{2.450810in}{2.103961in}}%
\pgfpathlineto{\pgfqpoint{2.441816in}{2.111763in}}%
\pgfpathlineto{\pgfqpoint{2.432797in}{2.119979in}}%
\pgfpathlineto{\pgfqpoint{2.423752in}{2.128618in}}%
\pgfpathclose%
\pgfusepath{fill}%
\end{pgfscope}%
\begin{pgfscope}%
\pgfpathrectangle{\pgfqpoint{1.254980in}{0.150000in}}{\pgfqpoint{5.490039in}{5.490039in}}%
\pgfusepath{clip}%
\pgfsetbuttcap%
\pgfsetroundjoin%
\definecolor{currentfill}{rgb}{0.277018,0.050344,0.375715}%
\pgfsetfillcolor{currentfill}%
\pgfsetfillopacity{0.700000}%
\pgfsetlinewidth{0.000000pt}%
\definecolor{currentstroke}{rgb}{0.000000,0.000000,0.000000}%
\pgfsetstrokecolor{currentstroke}%
\pgfsetdash{}{0pt}%
\pgfpathmoveto{\pgfqpoint{3.111015in}{1.443168in}}%
\pgfpathlineto{\pgfqpoint{3.124519in}{1.434306in}}%
\pgfpathlineto{\pgfqpoint{3.138024in}{1.425622in}}%
\pgfpathlineto{\pgfqpoint{3.151529in}{1.417115in}}%
\pgfpathlineto{\pgfqpoint{3.165035in}{1.408783in}}%
\pgfpathlineto{\pgfqpoint{3.173368in}{1.410882in}}%
\pgfpathlineto{\pgfqpoint{3.181686in}{1.413266in}}%
\pgfpathlineto{\pgfqpoint{3.189992in}{1.415928in}}%
\pgfpathlineto{\pgfqpoint{3.198284in}{1.418862in}}%
\pgfpathlineto{\pgfqpoint{3.184812in}{1.426521in}}%
\pgfpathlineto{\pgfqpoint{3.171340in}{1.434357in}}%
\pgfpathlineto{\pgfqpoint{3.157870in}{1.442368in}}%
\pgfpathlineto{\pgfqpoint{3.144400in}{1.450557in}}%
\pgfpathlineto{\pgfqpoint{3.136075in}{1.448284in}}%
\pgfpathlineto{\pgfqpoint{3.127736in}{1.446291in}}%
\pgfpathlineto{\pgfqpoint{3.119382in}{1.444583in}}%
\pgfpathlineto{\pgfqpoint{3.111015in}{1.443168in}}%
\pgfpathclose%
\pgfusepath{fill}%
\end{pgfscope}%
\begin{pgfscope}%
\pgfpathrectangle{\pgfqpoint{1.254980in}{0.150000in}}{\pgfqpoint{5.490039in}{5.490039in}}%
\pgfusepath{clip}%
\pgfsetbuttcap%
\pgfsetroundjoin%
\definecolor{currentfill}{rgb}{0.360741,0.785964,0.387814}%
\pgfsetfillcolor{currentfill}%
\pgfsetfillopacity{0.700000}%
\pgfsetlinewidth{0.000000pt}%
\definecolor{currentstroke}{rgb}{0.000000,0.000000,0.000000}%
\pgfsetstrokecolor{currentstroke}%
\pgfsetdash{}{0pt}%
\pgfpathmoveto{\pgfqpoint{5.535359in}{3.245656in}}%
\pgfpathlineto{\pgfqpoint{5.549872in}{3.261142in}}%
\pgfpathlineto{\pgfqpoint{5.564407in}{3.276793in}}%
\pgfpathlineto{\pgfqpoint{5.578964in}{3.292609in}}%
\pgfpathlineto{\pgfqpoint{5.593543in}{3.308591in}}%
\pgfpathlineto{\pgfqpoint{5.600733in}{3.312357in}}%
\pgfpathlineto{\pgfqpoint{5.607911in}{3.315975in}}%
\pgfpathlineto{\pgfqpoint{5.615079in}{3.319447in}}%
\pgfpathlineto{\pgfqpoint{5.622236in}{3.322777in}}%
\pgfpathlineto{\pgfqpoint{5.607673in}{3.307066in}}%
\pgfpathlineto{\pgfqpoint{5.593133in}{3.291520in}}%
\pgfpathlineto{\pgfqpoint{5.578614in}{3.276139in}}%
\pgfpathlineto{\pgfqpoint{5.564117in}{3.260922in}}%
\pgfpathlineto{\pgfqpoint{5.556943in}{3.257311in}}%
\pgfpathlineto{\pgfqpoint{5.549759in}{3.253565in}}%
\pgfpathlineto{\pgfqpoint{5.542564in}{3.249681in}}%
\pgfpathlineto{\pgfqpoint{5.535359in}{3.245656in}}%
\pgfpathclose%
\pgfusepath{fill}%
\end{pgfscope}%
\begin{pgfscope}%
\pgfpathrectangle{\pgfqpoint{1.254980in}{0.150000in}}{\pgfqpoint{5.490039in}{5.490039in}}%
\pgfusepath{clip}%
\pgfsetbuttcap%
\pgfsetroundjoin%
\definecolor{currentfill}{rgb}{0.281412,0.155834,0.469201}%
\pgfsetfillcolor{currentfill}%
\pgfsetfillopacity{0.700000}%
\pgfsetlinewidth{0.000000pt}%
\definecolor{currentstroke}{rgb}{0.000000,0.000000,0.000000}%
\pgfsetstrokecolor{currentstroke}%
\pgfsetdash{}{0pt}%
\pgfpathmoveto{\pgfqpoint{2.806266in}{1.670600in}}%
\pgfpathlineto{\pgfqpoint{2.819847in}{1.657287in}}%
\pgfpathlineto{\pgfqpoint{2.833424in}{1.644171in}}%
\pgfpathlineto{\pgfqpoint{2.846997in}{1.631251in}}%
\pgfpathlineto{\pgfqpoint{2.860567in}{1.618526in}}%
\pgfpathlineto{\pgfqpoint{2.869167in}{1.615817in}}%
\pgfpathlineto{\pgfqpoint{2.877748in}{1.613462in}}%
\pgfpathlineto{\pgfqpoint{2.886311in}{1.611453in}}%
\pgfpathlineto{\pgfqpoint{2.894856in}{1.609783in}}%
\pgfpathlineto{\pgfqpoint{2.881332in}{1.621794in}}%
\pgfpathlineto{\pgfqpoint{2.867805in}{1.634000in}}%
\pgfpathlineto{\pgfqpoint{2.854276in}{1.646401in}}%
\pgfpathlineto{\pgfqpoint{2.840743in}{1.658997in}}%
\pgfpathlineto{\pgfqpoint{2.832152in}{1.661370in}}%
\pgfpathlineto{\pgfqpoint{2.823543in}{1.664090in}}%
\pgfpathlineto{\pgfqpoint{2.814915in}{1.667164in}}%
\pgfpathlineto{\pgfqpoint{2.806266in}{1.670600in}}%
\pgfpathclose%
\pgfusepath{fill}%
\end{pgfscope}%
\begin{pgfscope}%
\pgfpathrectangle{\pgfqpoint{1.254980in}{0.150000in}}{\pgfqpoint{5.490039in}{5.490039in}}%
\pgfusepath{clip}%
\pgfsetbuttcap%
\pgfsetroundjoin%
\definecolor{currentfill}{rgb}{0.197636,0.391528,0.554969}%
\pgfsetfillcolor{currentfill}%
\pgfsetfillopacity{0.700000}%
\pgfsetlinewidth{0.000000pt}%
\definecolor{currentstroke}{rgb}{0.000000,0.000000,0.000000}%
\pgfsetstrokecolor{currentstroke}%
\pgfsetdash{}{0pt}%
\pgfpathmoveto{\pgfqpoint{4.487101in}{2.128201in}}%
\pgfpathlineto{\pgfqpoint{4.500931in}{2.137193in}}%
\pgfpathlineto{\pgfqpoint{4.514776in}{2.146345in}}%
\pgfpathlineto{\pgfqpoint{4.528636in}{2.155658in}}%
\pgfpathlineto{\pgfqpoint{4.542511in}{2.165132in}}%
\pgfpathlineto{\pgfqpoint{4.550262in}{2.178789in}}%
\pgfpathlineto{\pgfqpoint{4.558009in}{2.192342in}}%
\pgfpathlineto{\pgfqpoint{4.565750in}{2.205788in}}%
\pgfpathlineto{\pgfqpoint{4.573486in}{2.219125in}}%
\pgfpathlineto{\pgfqpoint{4.559609in}{2.209425in}}%
\pgfpathlineto{\pgfqpoint{4.545747in}{2.199887in}}%
\pgfpathlineto{\pgfqpoint{4.531900in}{2.190508in}}%
\pgfpathlineto{\pgfqpoint{4.518068in}{2.181291in}}%
\pgfpathlineto{\pgfqpoint{4.510334in}{2.168168in}}%
\pgfpathlineto{\pgfqpoint{4.502595in}{2.154944in}}%
\pgfpathlineto{\pgfqpoint{4.494850in}{2.141621in}}%
\pgfpathlineto{\pgfqpoint{4.487101in}{2.128201in}}%
\pgfpathclose%
\pgfusepath{fill}%
\end{pgfscope}%
\begin{pgfscope}%
\pgfpathrectangle{\pgfqpoint{1.254980in}{0.150000in}}{\pgfqpoint{5.490039in}{5.490039in}}%
\pgfusepath{clip}%
\pgfsetbuttcap%
\pgfsetroundjoin%
\definecolor{currentfill}{rgb}{0.123463,0.581687,0.547445}%
\pgfsetfillcolor{currentfill}%
\pgfsetfillopacity{0.700000}%
\pgfsetlinewidth{0.000000pt}%
\definecolor{currentstroke}{rgb}{0.000000,0.000000,0.000000}%
\pgfsetstrokecolor{currentstroke}%
\pgfsetdash{}{0pt}%
\pgfpathmoveto{\pgfqpoint{4.925272in}{2.641000in}}%
\pgfpathlineto{\pgfqpoint{4.939372in}{2.653538in}}%
\pgfpathlineto{\pgfqpoint{4.953491in}{2.666239in}}%
\pgfpathlineto{\pgfqpoint{4.967628in}{2.679103in}}%
\pgfpathlineto{\pgfqpoint{4.981784in}{2.692131in}}%
\pgfpathlineto{\pgfqpoint{4.989362in}{2.702489in}}%
\pgfpathlineto{\pgfqpoint{4.996934in}{2.712688in}}%
\pgfpathlineto{\pgfqpoint{5.004497in}{2.722726in}}%
\pgfpathlineto{\pgfqpoint{5.012052in}{2.732605in}}%
\pgfpathlineto{\pgfqpoint{4.997898in}{2.719562in}}%
\pgfpathlineto{\pgfqpoint{4.983763in}{2.706681in}}%
\pgfpathlineto{\pgfqpoint{4.969646in}{2.693964in}}%
\pgfpathlineto{\pgfqpoint{4.955548in}{2.681410in}}%
\pgfpathlineto{\pgfqpoint{4.947990in}{2.671536in}}%
\pgfpathlineto{\pgfqpoint{4.940425in}{2.661510in}}%
\pgfpathlineto{\pgfqpoint{4.932852in}{2.651331in}}%
\pgfpathlineto{\pgfqpoint{4.925272in}{2.641000in}}%
\pgfpathclose%
\pgfusepath{fill}%
\end{pgfscope}%
\begin{pgfscope}%
\pgfpathrectangle{\pgfqpoint{1.254980in}{0.150000in}}{\pgfqpoint{5.490039in}{5.490039in}}%
\pgfusepath{clip}%
\pgfsetbuttcap%
\pgfsetroundjoin%
\definecolor{currentfill}{rgb}{0.268510,0.009605,0.335427}%
\pgfsetfillcolor{currentfill}%
\pgfsetfillopacity{0.700000}%
\pgfsetlinewidth{0.000000pt}%
\definecolor{currentstroke}{rgb}{0.000000,0.000000,0.000000}%
\pgfsetstrokecolor{currentstroke}%
\pgfsetdash{}{0pt}%
\pgfpathmoveto{\pgfqpoint{3.306124in}{1.363829in}}%
\pgfpathlineto{\pgfqpoint{3.319613in}{1.357721in}}%
\pgfpathlineto{\pgfqpoint{3.333106in}{1.351782in}}%
\pgfpathlineto{\pgfqpoint{3.346601in}{1.346011in}}%
\pgfpathlineto{\pgfqpoint{3.360099in}{1.340409in}}%
\pgfpathlineto{\pgfqpoint{3.368292in}{1.345562in}}%
\pgfpathlineto{\pgfqpoint{3.376474in}{1.350951in}}%
\pgfpathlineto{\pgfqpoint{3.384646in}{1.356570in}}%
\pgfpathlineto{\pgfqpoint{3.392807in}{1.362412in}}%
\pgfpathlineto{\pgfqpoint{3.379335in}{1.367375in}}%
\pgfpathlineto{\pgfqpoint{3.365866in}{1.372505in}}%
\pgfpathlineto{\pgfqpoint{3.352400in}{1.377805in}}%
\pgfpathlineto{\pgfqpoint{3.338937in}{1.383273in}}%
\pgfpathlineto{\pgfqpoint{3.330750in}{1.378060in}}%
\pgfpathlineto{\pgfqpoint{3.322552in}{1.373077in}}%
\pgfpathlineto{\pgfqpoint{3.314344in}{1.368332in}}%
\pgfpathlineto{\pgfqpoint{3.306124in}{1.363829in}}%
\pgfpathclose%
\pgfusepath{fill}%
\end{pgfscope}%
\begin{pgfscope}%
\pgfpathrectangle{\pgfqpoint{1.254980in}{0.150000in}}{\pgfqpoint{5.490039in}{5.490039in}}%
\pgfusepath{clip}%
\pgfsetbuttcap%
\pgfsetroundjoin%
\definecolor{currentfill}{rgb}{0.201239,0.383670,0.554294}%
\pgfsetfillcolor{currentfill}%
\pgfsetfillopacity{0.700000}%
\pgfsetlinewidth{0.000000pt}%
\definecolor{currentstroke}{rgb}{0.000000,0.000000,0.000000}%
\pgfsetstrokecolor{currentstroke}%
\pgfsetdash{}{0pt}%
\pgfpathmoveto{\pgfqpoint{2.368598in}{2.208559in}}%
\pgfpathlineto{\pgfqpoint{2.382402in}{2.188207in}}%
\pgfpathlineto{\pgfqpoint{2.396196in}{2.168101in}}%
\pgfpathlineto{\pgfqpoint{2.409979in}{2.148238in}}%
\pgfpathlineto{\pgfqpoint{2.423752in}{2.128618in}}%
\pgfpathlineto{\pgfqpoint{2.432797in}{2.119979in}}%
\pgfpathlineto{\pgfqpoint{2.441816in}{2.111763in}}%
\pgfpathlineto{\pgfqpoint{2.450810in}{2.103961in}}%
\pgfpathlineto{\pgfqpoint{2.459778in}{2.096566in}}%
\pgfpathlineto{\pgfqpoint{2.446068in}{2.115439in}}%
\pgfpathlineto{\pgfqpoint{2.432349in}{2.134551in}}%
\pgfpathlineto{\pgfqpoint{2.418620in}{2.153907in}}%
\pgfpathlineto{\pgfqpoint{2.404881in}{2.173506in}}%
\pgfpathlineto{\pgfqpoint{2.395850in}{2.181637in}}%
\pgfpathlineto{\pgfqpoint{2.386793in}{2.190185in}}%
\pgfpathlineto{\pgfqpoint{2.377709in}{2.199156in}}%
\pgfpathlineto{\pgfqpoint{2.368598in}{2.208559in}}%
\pgfpathclose%
\pgfusepath{fill}%
\end{pgfscope}%
\begin{pgfscope}%
\pgfpathrectangle{\pgfqpoint{1.254980in}{0.150000in}}{\pgfqpoint{5.490039in}{5.490039in}}%
\pgfusepath{clip}%
\pgfsetbuttcap%
\pgfsetroundjoin%
\definecolor{currentfill}{rgb}{0.221989,0.339161,0.548752}%
\pgfsetfillcolor{currentfill}%
\pgfsetfillopacity{0.700000}%
\pgfsetlinewidth{0.000000pt}%
\definecolor{currentstroke}{rgb}{0.000000,0.000000,0.000000}%
\pgfsetstrokecolor{currentstroke}%
\pgfsetdash{}{0pt}%
\pgfpathmoveto{\pgfqpoint{4.369773in}{1.985371in}}%
\pgfpathlineto{\pgfqpoint{4.383542in}{1.993186in}}%
\pgfpathlineto{\pgfqpoint{4.397325in}{2.001160in}}%
\pgfpathlineto{\pgfqpoint{4.411122in}{2.009293in}}%
\pgfpathlineto{\pgfqpoint{4.424933in}{2.017587in}}%
\pgfpathlineto{\pgfqpoint{4.432720in}{2.031708in}}%
\pgfpathlineto{\pgfqpoint{4.440503in}{2.045750in}}%
\pgfpathlineto{\pgfqpoint{4.448281in}{2.059711in}}%
\pgfpathlineto{\pgfqpoint{4.456055in}{2.073587in}}%
\pgfpathlineto{\pgfqpoint{4.442242in}{2.065009in}}%
\pgfpathlineto{\pgfqpoint{4.428443in}{2.056592in}}%
\pgfpathlineto{\pgfqpoint{4.414658in}{2.048335in}}%
\pgfpathlineto{\pgfqpoint{4.400887in}{2.040238in}}%
\pgfpathlineto{\pgfqpoint{4.393115in}{2.026635in}}%
\pgfpathlineto{\pgfqpoint{4.385339in}{2.012954in}}%
\pgfpathlineto{\pgfqpoint{4.377558in}{1.999199in}}%
\pgfpathlineto{\pgfqpoint{4.369773in}{1.985371in}}%
\pgfpathclose%
\pgfusepath{fill}%
\end{pgfscope}%
\begin{pgfscope}%
\pgfpathrectangle{\pgfqpoint{1.254980in}{0.150000in}}{\pgfqpoint{5.490039in}{5.490039in}}%
\pgfusepath{clip}%
\pgfsetbuttcap%
\pgfsetroundjoin%
\definecolor{currentfill}{rgb}{0.282884,0.135920,0.453427}%
\pgfsetfillcolor{currentfill}%
\pgfsetfillopacity{0.700000}%
\pgfsetlinewidth{0.000000pt}%
\definecolor{currentstroke}{rgb}{0.000000,0.000000,0.000000}%
\pgfsetstrokecolor{currentstroke}%
\pgfsetdash{}{0pt}%
\pgfpathmoveto{\pgfqpoint{2.860567in}{1.618526in}}%
\pgfpathlineto{\pgfqpoint{2.874134in}{1.605995in}}%
\pgfpathlineto{\pgfqpoint{2.887699in}{1.593656in}}%
\pgfpathlineto{\pgfqpoint{2.901260in}{1.581508in}}%
\pgfpathlineto{\pgfqpoint{2.914819in}{1.569551in}}%
\pgfpathlineto{\pgfqpoint{2.923373in}{1.567567in}}%
\pgfpathlineto{\pgfqpoint{2.931909in}{1.565927in}}%
\pgfpathlineto{\pgfqpoint{2.940427in}{1.564626in}}%
\pgfpathlineto{\pgfqpoint{2.948929in}{1.563655in}}%
\pgfpathlineto{\pgfqpoint{2.935414in}{1.574901in}}%
\pgfpathlineto{\pgfqpoint{2.921897in}{1.586337in}}%
\pgfpathlineto{\pgfqpoint{2.908378in}{1.597964in}}%
\pgfpathlineto{\pgfqpoint{2.894856in}{1.609783in}}%
\pgfpathlineto{\pgfqpoint{2.886311in}{1.611453in}}%
\pgfpathlineto{\pgfqpoint{2.877748in}{1.613462in}}%
\pgfpathlineto{\pgfqpoint{2.869167in}{1.615817in}}%
\pgfpathlineto{\pgfqpoint{2.860567in}{1.618526in}}%
\pgfpathclose%
\pgfusepath{fill}%
\end{pgfscope}%
\begin{pgfscope}%
\pgfpathrectangle{\pgfqpoint{1.254980in}{0.150000in}}{\pgfqpoint{5.490039in}{5.490039in}}%
\pgfusepath{clip}%
\pgfsetbuttcap%
\pgfsetroundjoin%
\definecolor{currentfill}{rgb}{0.267004,0.004874,0.329415}%
\pgfsetfillcolor{currentfill}%
\pgfsetfillopacity{0.700000}%
\pgfsetlinewidth{0.000000pt}%
\definecolor{currentstroke}{rgb}{0.000000,0.000000,0.000000}%
\pgfsetstrokecolor{currentstroke}%
\pgfsetdash{}{0pt}%
\pgfpathmoveto{\pgfqpoint{3.446734in}{1.344231in}}%
\pgfpathlineto{\pgfqpoint{3.460226in}{1.340100in}}%
\pgfpathlineto{\pgfqpoint{3.473722in}{1.336135in}}%
\pgfpathlineto{\pgfqpoint{3.487222in}{1.332334in}}%
\pgfpathlineto{\pgfqpoint{3.500727in}{1.328697in}}%
\pgfpathlineto{\pgfqpoint{3.508834in}{1.335998in}}%
\pgfpathlineto{\pgfqpoint{3.516932in}{1.343496in}}%
\pgfpathlineto{\pgfqpoint{3.525021in}{1.351184in}}%
\pgfpathlineto{\pgfqpoint{3.533102in}{1.359056in}}%
\pgfpathlineto{\pgfqpoint{3.519617in}{1.362083in}}%
\pgfpathlineto{\pgfqpoint{3.506138in}{1.365273in}}%
\pgfpathlineto{\pgfqpoint{3.492663in}{1.368628in}}%
\pgfpathlineto{\pgfqpoint{3.479192in}{1.372148in}}%
\pgfpathlineto{\pgfqpoint{3.471091in}{1.364876in}}%
\pgfpathlineto{\pgfqpoint{3.462981in}{1.357794in}}%
\pgfpathlineto{\pgfqpoint{3.454862in}{1.350911in}}%
\pgfpathlineto{\pgfqpoint{3.446734in}{1.344231in}}%
\pgfpathclose%
\pgfusepath{fill}%
\end{pgfscope}%
\begin{pgfscope}%
\pgfpathrectangle{\pgfqpoint{1.254980in}{0.150000in}}{\pgfqpoint{5.490039in}{5.490039in}}%
\pgfusepath{clip}%
\pgfsetbuttcap%
\pgfsetroundjoin%
\definecolor{currentfill}{rgb}{0.278826,0.175490,0.483397}%
\pgfsetfillcolor{currentfill}%
\pgfsetfillopacity{0.700000}%
\pgfsetlinewidth{0.000000pt}%
\definecolor{currentstroke}{rgb}{0.000000,0.000000,0.000000}%
\pgfsetstrokecolor{currentstroke}%
\pgfsetdash{}{0pt}%
\pgfpathmoveto{\pgfqpoint{4.048947in}{1.638857in}}%
\pgfpathlineto{\pgfqpoint{4.062570in}{1.642930in}}%
\pgfpathlineto{\pgfqpoint{4.076204in}{1.647162in}}%
\pgfpathlineto{\pgfqpoint{4.089848in}{1.651552in}}%
\pgfpathlineto{\pgfqpoint{4.103503in}{1.656101in}}%
\pgfpathlineto{\pgfqpoint{4.111374in}{1.669864in}}%
\pgfpathlineto{\pgfqpoint{4.119240in}{1.683634in}}%
\pgfpathlineto{\pgfqpoint{4.127102in}{1.697408in}}%
\pgfpathlineto{\pgfqpoint{4.134959in}{1.711182in}}%
\pgfpathlineto{\pgfqpoint{4.121305in}{1.706211in}}%
\pgfpathlineto{\pgfqpoint{4.107663in}{1.701399in}}%
\pgfpathlineto{\pgfqpoint{4.094032in}{1.696746in}}%
\pgfpathlineto{\pgfqpoint{4.080411in}{1.692251in}}%
\pgfpathlineto{\pgfqpoint{4.072552in}{1.678889in}}%
\pgfpathlineto{\pgfqpoint{4.064688in}{1.665533in}}%
\pgfpathlineto{\pgfqpoint{4.056820in}{1.652188in}}%
\pgfpathlineto{\pgfqpoint{4.048947in}{1.638857in}}%
\pgfpathclose%
\pgfusepath{fill}%
\end{pgfscope}%
\begin{pgfscope}%
\pgfpathrectangle{\pgfqpoint{1.254980in}{0.150000in}}{\pgfqpoint{5.490039in}{5.490039in}}%
\pgfusepath{clip}%
\pgfsetbuttcap%
\pgfsetroundjoin%
\definecolor{currentfill}{rgb}{0.126453,0.570633,0.549841}%
\pgfsetfillcolor{currentfill}%
\pgfsetfillopacity{0.700000}%
\pgfsetlinewidth{0.000000pt}%
\definecolor{currentstroke}{rgb}{0.000000,0.000000,0.000000}%
\pgfsetstrokecolor{currentstroke}%
\pgfsetdash{}{0pt}%
\pgfpathmoveto{\pgfqpoint{2.071452in}{2.723050in}}%
\pgfpathlineto{\pgfqpoint{2.085514in}{2.696890in}}%
\pgfpathlineto{\pgfqpoint{2.099559in}{2.671034in}}%
\pgfpathlineto{\pgfqpoint{2.113587in}{2.645479in}}%
\pgfpathlineto{\pgfqpoint{2.127599in}{2.620221in}}%
\pgfpathlineto{\pgfqpoint{2.136949in}{2.608763in}}%
\pgfpathlineto{\pgfqpoint{2.146268in}{2.597753in}}%
\pgfpathlineto{\pgfqpoint{2.155558in}{2.587183in}}%
\pgfpathlineto{\pgfqpoint{2.164819in}{2.577044in}}%
\pgfpathlineto{\pgfqpoint{2.150881in}{2.601550in}}%
\pgfpathlineto{\pgfqpoint{2.136927in}{2.626351in}}%
\pgfpathlineto{\pgfqpoint{2.122957in}{2.651451in}}%
\pgfpathlineto{\pgfqpoint{2.108971in}{2.676852in}}%
\pgfpathlineto{\pgfqpoint{2.099638in}{2.687732in}}%
\pgfpathlineto{\pgfqpoint{2.090274in}{2.699052in}}%
\pgfpathlineto{\pgfqpoint{2.080879in}{2.710823in}}%
\pgfpathlineto{\pgfqpoint{2.071452in}{2.723050in}}%
\pgfpathclose%
\pgfusepath{fill}%
\end{pgfscope}%
\begin{pgfscope}%
\pgfpathrectangle{\pgfqpoint{1.254980in}{0.150000in}}{\pgfqpoint{5.490039in}{5.490039in}}%
\pgfusepath{clip}%
\pgfsetbuttcap%
\pgfsetroundjoin%
\definecolor{currentfill}{rgb}{0.232815,0.732247,0.459277}%
\pgfsetfillcolor{currentfill}%
\pgfsetfillopacity{0.700000}%
\pgfsetlinewidth{0.000000pt}%
\definecolor{currentstroke}{rgb}{0.000000,0.000000,0.000000}%
\pgfsetstrokecolor{currentstroke}%
\pgfsetdash{}{0pt}%
\pgfpathmoveto{\pgfqpoint{5.332441in}{3.062435in}}%
\pgfpathlineto{\pgfqpoint{5.346822in}{3.077221in}}%
\pgfpathlineto{\pgfqpoint{5.361224in}{3.092170in}}%
\pgfpathlineto{\pgfqpoint{5.375646in}{3.107285in}}%
\pgfpathlineto{\pgfqpoint{5.390090in}{3.122565in}}%
\pgfpathlineto{\pgfqpoint{5.397432in}{3.128658in}}%
\pgfpathlineto{\pgfqpoint{5.404763in}{3.134587in}}%
\pgfpathlineto{\pgfqpoint{5.412084in}{3.140354in}}%
\pgfpathlineto{\pgfqpoint{5.419395in}{3.145961in}}%
\pgfpathlineto{\pgfqpoint{5.404962in}{3.130855in}}%
\pgfpathlineto{\pgfqpoint{5.390550in}{3.115914in}}%
\pgfpathlineto{\pgfqpoint{5.376159in}{3.101138in}}%
\pgfpathlineto{\pgfqpoint{5.361788in}{3.086525in}}%
\pgfpathlineto{\pgfqpoint{5.354466in}{3.080733in}}%
\pgfpathlineto{\pgfqpoint{5.347134in}{3.074789in}}%
\pgfpathlineto{\pgfqpoint{5.339792in}{3.068690in}}%
\pgfpathlineto{\pgfqpoint{5.332441in}{3.062435in}}%
\pgfpathclose%
\pgfusepath{fill}%
\end{pgfscope}%
\begin{pgfscope}%
\pgfpathrectangle{\pgfqpoint{1.254980in}{0.150000in}}{\pgfqpoint{5.490039in}{5.490039in}}%
\pgfusepath{clip}%
\pgfsetbuttcap%
\pgfsetroundjoin%
\definecolor{currentfill}{rgb}{0.137339,0.662252,0.515571}%
\pgfsetfillcolor{currentfill}%
\pgfsetfillopacity{0.700000}%
\pgfsetlinewidth{0.000000pt}%
\definecolor{currentstroke}{rgb}{0.000000,0.000000,0.000000}%
\pgfsetstrokecolor{currentstroke}%
\pgfsetdash{}{0pt}%
\pgfpathmoveto{\pgfqpoint{5.128979in}{2.859510in}}%
\pgfpathlineto{\pgfqpoint{5.143221in}{2.873310in}}%
\pgfpathlineto{\pgfqpoint{5.157483in}{2.887274in}}%
\pgfpathlineto{\pgfqpoint{5.171764in}{2.901402in}}%
\pgfpathlineto{\pgfqpoint{5.186065in}{2.915694in}}%
\pgfpathlineto{\pgfqpoint{5.193536in}{2.924030in}}%
\pgfpathlineto{\pgfqpoint{5.200999in}{2.932198in}}%
\pgfpathlineto{\pgfqpoint{5.208453in}{2.940199in}}%
\pgfpathlineto{\pgfqpoint{5.215897in}{2.948035in}}%
\pgfpathlineto{\pgfqpoint{5.201602in}{2.933820in}}%
\pgfpathlineto{\pgfqpoint{5.187326in}{2.919770in}}%
\pgfpathlineto{\pgfqpoint{5.173071in}{2.905884in}}%
\pgfpathlineto{\pgfqpoint{5.158834in}{2.892161in}}%
\pgfpathlineto{\pgfqpoint{5.151384in}{2.884236in}}%
\pgfpathlineto{\pgfqpoint{5.143924in}{2.876153in}}%
\pgfpathlineto{\pgfqpoint{5.136456in}{2.867912in}}%
\pgfpathlineto{\pgfqpoint{5.128979in}{2.859510in}}%
\pgfpathclose%
\pgfusepath{fill}%
\end{pgfscope}%
\begin{pgfscope}%
\pgfpathrectangle{\pgfqpoint{1.254980in}{0.150000in}}{\pgfqpoint{5.490039in}{5.490039in}}%
\pgfusepath{clip}%
\pgfsetbuttcap%
\pgfsetroundjoin%
\definecolor{currentfill}{rgb}{0.137770,0.537492,0.554906}%
\pgfsetfillcolor{currentfill}%
\pgfsetfillopacity{0.700000}%
\pgfsetlinewidth{0.000000pt}%
\definecolor{currentstroke}{rgb}{0.000000,0.000000,0.000000}%
\pgfsetstrokecolor{currentstroke}%
\pgfsetdash{}{0pt}%
\pgfpathmoveto{\pgfqpoint{4.808140in}{2.504831in}}%
\pgfpathlineto{\pgfqpoint{4.822170in}{2.516595in}}%
\pgfpathlineto{\pgfqpoint{4.836218in}{2.528520in}}%
\pgfpathlineto{\pgfqpoint{4.850284in}{2.540608in}}%
\pgfpathlineto{\pgfqpoint{4.864367in}{2.552859in}}%
\pgfpathlineto{\pgfqpoint{4.872005in}{2.564409in}}%
\pgfpathlineto{\pgfqpoint{4.879636in}{2.575808in}}%
\pgfpathlineto{\pgfqpoint{4.887260in}{2.587054in}}%
\pgfpathlineto{\pgfqpoint{4.894877in}{2.598149in}}%
\pgfpathlineto{\pgfqpoint{4.880793in}{2.585820in}}%
\pgfpathlineto{\pgfqpoint{4.866728in}{2.573655in}}%
\pgfpathlineto{\pgfqpoint{4.852680in}{2.561652in}}%
\pgfpathlineto{\pgfqpoint{4.838649in}{2.549811in}}%
\pgfpathlineto{\pgfqpoint{4.831032in}{2.538782in}}%
\pgfpathlineto{\pgfqpoint{4.823408in}{2.527610in}}%
\pgfpathlineto{\pgfqpoint{4.815777in}{2.516292in}}%
\pgfpathlineto{\pgfqpoint{4.808140in}{2.504831in}}%
\pgfpathclose%
\pgfusepath{fill}%
\end{pgfscope}%
\begin{pgfscope}%
\pgfpathrectangle{\pgfqpoint{1.254980in}{0.150000in}}{\pgfqpoint{5.490039in}{5.490039in}}%
\pgfusepath{clip}%
\pgfsetbuttcap%
\pgfsetroundjoin%
\definecolor{currentfill}{rgb}{0.185556,0.418570,0.556753}%
\pgfsetfillcolor{currentfill}%
\pgfsetfillopacity{0.700000}%
\pgfsetlinewidth{0.000000pt}%
\definecolor{currentstroke}{rgb}{0.000000,0.000000,0.000000}%
\pgfsetstrokecolor{currentstroke}%
\pgfsetdash{}{0pt}%
\pgfpathmoveto{\pgfqpoint{2.313271in}{2.292471in}}%
\pgfpathlineto{\pgfqpoint{2.327120in}{2.271114in}}%
\pgfpathlineto{\pgfqpoint{2.340957in}{2.250011in}}%
\pgfpathlineto{\pgfqpoint{2.354783in}{2.229160in}}%
\pgfpathlineto{\pgfqpoint{2.368598in}{2.208559in}}%
\pgfpathlineto{\pgfqpoint{2.377709in}{2.199156in}}%
\pgfpathlineto{\pgfqpoint{2.386793in}{2.190185in}}%
\pgfpathlineto{\pgfqpoint{2.395850in}{2.181637in}}%
\pgfpathlineto{\pgfqpoint{2.404881in}{2.173506in}}%
\pgfpathlineto{\pgfqpoint{2.391132in}{2.193353in}}%
\pgfpathlineto{\pgfqpoint{2.377373in}{2.213448in}}%
\pgfpathlineto{\pgfqpoint{2.363603in}{2.233794in}}%
\pgfpathlineto{\pgfqpoint{2.349821in}{2.254392in}}%
\pgfpathlineto{\pgfqpoint{2.340725in}{2.263266in}}%
\pgfpathlineto{\pgfqpoint{2.331602in}{2.272565in}}%
\pgfpathlineto{\pgfqpoint{2.322451in}{2.282297in}}%
\pgfpathlineto{\pgfqpoint{2.313271in}{2.292471in}}%
\pgfpathclose%
\pgfusepath{fill}%
\end{pgfscope}%
\begin{pgfscope}%
\pgfpathrectangle{\pgfqpoint{1.254980in}{0.150000in}}{\pgfqpoint{5.490039in}{5.490039in}}%
\pgfusepath{clip}%
\pgfsetbuttcap%
\pgfsetroundjoin%
\definecolor{currentfill}{rgb}{0.248629,0.278775,0.534556}%
\pgfsetfillcolor{currentfill}%
\pgfsetfillopacity{0.700000}%
\pgfsetlinewidth{0.000000pt}%
\definecolor{currentstroke}{rgb}{0.000000,0.000000,0.000000}%
\pgfsetstrokecolor{currentstroke}%
\pgfsetdash{}{0pt}%
\pgfpathmoveto{\pgfqpoint{4.252398in}{1.845422in}}%
\pgfpathlineto{\pgfqpoint{4.266112in}{1.851948in}}%
\pgfpathlineto{\pgfqpoint{4.279838in}{1.858633in}}%
\pgfpathlineto{\pgfqpoint{4.293578in}{1.865477in}}%
\pgfpathlineto{\pgfqpoint{4.307330in}{1.872480in}}%
\pgfpathlineto{\pgfqpoint{4.315150in}{1.886786in}}%
\pgfpathlineto{\pgfqpoint{4.322967in}{1.901042in}}%
\pgfpathlineto{\pgfqpoint{4.330779in}{1.915245in}}%
\pgfpathlineto{\pgfqpoint{4.338586in}{1.929393in}}%
\pgfpathlineto{\pgfqpoint{4.324832in}{1.922050in}}%
\pgfpathlineto{\pgfqpoint{4.311091in}{1.914866in}}%
\pgfpathlineto{\pgfqpoint{4.297364in}{1.907842in}}%
\pgfpathlineto{\pgfqpoint{4.283650in}{1.900976in}}%
\pgfpathlineto{\pgfqpoint{4.275843in}{1.887157in}}%
\pgfpathlineto{\pgfqpoint{4.268032in}{1.873290in}}%
\pgfpathlineto{\pgfqpoint{4.260217in}{1.859377in}}%
\pgfpathlineto{\pgfqpoint{4.252398in}{1.845422in}}%
\pgfpathclose%
\pgfusepath{fill}%
\end{pgfscope}%
\begin{pgfscope}%
\pgfpathrectangle{\pgfqpoint{1.254980in}{0.150000in}}{\pgfqpoint{5.490039in}{5.490039in}}%
\pgfusepath{clip}%
\pgfsetbuttcap%
\pgfsetroundjoin%
\definecolor{currentfill}{rgb}{0.274952,0.037752,0.364543}%
\pgfsetfillcolor{currentfill}%
\pgfsetfillopacity{0.700000}%
\pgfsetlinewidth{0.000000pt}%
\definecolor{currentstroke}{rgb}{0.000000,0.000000,0.000000}%
\pgfsetstrokecolor{currentstroke}%
\pgfsetdash{}{0pt}%
\pgfpathmoveto{\pgfqpoint{3.165035in}{1.408783in}}%
\pgfpathlineto{\pgfqpoint{3.178542in}{1.400627in}}%
\pgfpathlineto{\pgfqpoint{3.192050in}{1.392646in}}%
\pgfpathlineto{\pgfqpoint{3.205559in}{1.384838in}}%
\pgfpathlineto{\pgfqpoint{3.219069in}{1.377204in}}%
\pgfpathlineto{\pgfqpoint{3.227368in}{1.379985in}}%
\pgfpathlineto{\pgfqpoint{3.235654in}{1.383044in}}%
\pgfpathlineto{\pgfqpoint{3.243928in}{1.386373in}}%
\pgfpathlineto{\pgfqpoint{3.252189in}{1.389967in}}%
\pgfpathlineto{\pgfqpoint{3.238710in}{1.396930in}}%
\pgfpathlineto{\pgfqpoint{3.225233in}{1.404067in}}%
\pgfpathlineto{\pgfqpoint{3.211758in}{1.411377in}}%
\pgfpathlineto{\pgfqpoint{3.198284in}{1.418862in}}%
\pgfpathlineto{\pgfqpoint{3.189992in}{1.415928in}}%
\pgfpathlineto{\pgfqpoint{3.181686in}{1.413266in}}%
\pgfpathlineto{\pgfqpoint{3.173368in}{1.410882in}}%
\pgfpathlineto{\pgfqpoint{3.165035in}{1.408783in}}%
\pgfpathclose%
\pgfusepath{fill}%
\end{pgfscope}%
\begin{pgfscope}%
\pgfpathrectangle{\pgfqpoint{1.254980in}{0.150000in}}{\pgfqpoint{5.490039in}{5.490039in}}%
\pgfusepath{clip}%
\pgfsetbuttcap%
\pgfsetroundjoin%
\definecolor{currentfill}{rgb}{0.440137,0.811138,0.340967}%
\pgfsetfillcolor{currentfill}%
\pgfsetfillopacity{0.700000}%
\pgfsetlinewidth{0.000000pt}%
\definecolor{currentstroke}{rgb}{0.000000,0.000000,0.000000}%
\pgfsetstrokecolor{currentstroke}%
\pgfsetdash{}{0pt}%
\pgfpathmoveto{\pgfqpoint{5.622236in}{3.322777in}}%
\pgfpathlineto{\pgfqpoint{5.636821in}{3.338652in}}%
\pgfpathlineto{\pgfqpoint{5.651428in}{3.354693in}}%
\pgfpathlineto{\pgfqpoint{5.666057in}{3.370900in}}%
\pgfpathlineto{\pgfqpoint{5.680709in}{3.387272in}}%
\pgfpathlineto{\pgfqpoint{5.687837in}{3.390171in}}%
\pgfpathlineto{\pgfqpoint{5.694954in}{3.392926in}}%
\pgfpathlineto{\pgfqpoint{5.702060in}{3.395541in}}%
\pgfpathlineto{\pgfqpoint{5.709155in}{3.398020in}}%
\pgfpathlineto{\pgfqpoint{5.694522in}{3.381952in}}%
\pgfpathlineto{\pgfqpoint{5.679912in}{3.366049in}}%
\pgfpathlineto{\pgfqpoint{5.665323in}{3.350311in}}%
\pgfpathlineto{\pgfqpoint{5.650757in}{3.334737in}}%
\pgfpathlineto{\pgfqpoint{5.643643in}{3.331944in}}%
\pgfpathlineto{\pgfqpoint{5.636518in}{3.329022in}}%
\pgfpathlineto{\pgfqpoint{5.629382in}{3.325967in}}%
\pgfpathlineto{\pgfqpoint{5.622236in}{3.322777in}}%
\pgfpathclose%
\pgfusepath{fill}%
\end{pgfscope}%
\begin{pgfscope}%
\pgfpathrectangle{\pgfqpoint{1.254980in}{0.150000in}}{\pgfqpoint{5.490039in}{5.490039in}}%
\pgfusepath{clip}%
\pgfsetbuttcap%
\pgfsetroundjoin%
\definecolor{currentfill}{rgb}{0.278791,0.062145,0.386592}%
\pgfsetfillcolor{currentfill}%
\pgfsetfillopacity{0.700000}%
\pgfsetlinewidth{0.000000pt}%
\definecolor{currentstroke}{rgb}{0.000000,0.000000,0.000000}%
\pgfsetstrokecolor{currentstroke}%
\pgfsetdash{}{0pt}%
\pgfpathmoveto{\pgfqpoint{3.759369in}{1.417970in}}%
\pgfpathlineto{\pgfqpoint{3.772907in}{1.418182in}}%
\pgfpathlineto{\pgfqpoint{3.786453in}{1.418553in}}%
\pgfpathlineto{\pgfqpoint{3.800007in}{1.419084in}}%
\pgfpathlineto{\pgfqpoint{3.813569in}{1.419773in}}%
\pgfpathlineto{\pgfqpoint{3.821531in}{1.431200in}}%
\pgfpathlineto{\pgfqpoint{3.829488in}{1.442728in}}%
\pgfpathlineto{\pgfqpoint{3.837440in}{1.454351in}}%
\pgfpathlineto{\pgfqpoint{3.845386in}{1.466066in}}%
\pgfpathlineto{\pgfqpoint{3.831833in}{1.464847in}}%
\pgfpathlineto{\pgfqpoint{3.818289in}{1.463787in}}%
\pgfpathlineto{\pgfqpoint{3.804753in}{1.462887in}}%
\pgfpathlineto{\pgfqpoint{3.791225in}{1.462146in}}%
\pgfpathlineto{\pgfqpoint{3.783270in}{1.450951in}}%
\pgfpathlineto{\pgfqpoint{3.775309in}{1.439852in}}%
\pgfpathlineto{\pgfqpoint{3.767342in}{1.428857in}}%
\pgfpathlineto{\pgfqpoint{3.759369in}{1.417970in}}%
\pgfpathclose%
\pgfusepath{fill}%
\end{pgfscope}%
\begin{pgfscope}%
\pgfpathrectangle{\pgfqpoint{1.254980in}{0.150000in}}{\pgfqpoint{5.490039in}{5.490039in}}%
\pgfusepath{clip}%
\pgfsetbuttcap%
\pgfsetroundjoin%
\definecolor{currentfill}{rgb}{0.283197,0.115680,0.436115}%
\pgfsetfillcolor{currentfill}%
\pgfsetfillopacity{0.700000}%
\pgfsetlinewidth{0.000000pt}%
\definecolor{currentstroke}{rgb}{0.000000,0.000000,0.000000}%
\pgfsetstrokecolor{currentstroke}%
\pgfsetdash{}{0pt}%
\pgfpathmoveto{\pgfqpoint{2.914819in}{1.569551in}}%
\pgfpathlineto{\pgfqpoint{2.928376in}{1.557784in}}%
\pgfpathlineto{\pgfqpoint{2.941931in}{1.546205in}}%
\pgfpathlineto{\pgfqpoint{2.955483in}{1.534814in}}%
\pgfpathlineto{\pgfqpoint{2.969034in}{1.523609in}}%
\pgfpathlineto{\pgfqpoint{2.977544in}{1.522345in}}%
\pgfpathlineto{\pgfqpoint{2.986036in}{1.521419in}}%
\pgfpathlineto{\pgfqpoint{2.994512in}{1.520823in}}%
\pgfpathlineto{\pgfqpoint{3.002972in}{1.520550in}}%
\pgfpathlineto{\pgfqpoint{2.989463in}{1.531047in}}%
\pgfpathlineto{\pgfqpoint{2.975953in}{1.541729in}}%
\pgfpathlineto{\pgfqpoint{2.962442in}{1.552598in}}%
\pgfpathlineto{\pgfqpoint{2.948929in}{1.563655in}}%
\pgfpathlineto{\pgfqpoint{2.940427in}{1.564626in}}%
\pgfpathlineto{\pgfqpoint{2.931909in}{1.565927in}}%
\pgfpathlineto{\pgfqpoint{2.923373in}{1.567567in}}%
\pgfpathlineto{\pgfqpoint{2.914819in}{1.569551in}}%
\pgfpathclose%
\pgfusepath{fill}%
\end{pgfscope}%
\begin{pgfscope}%
\pgfpathrectangle{\pgfqpoint{1.254980in}{0.150000in}}{\pgfqpoint{5.490039in}{5.490039in}}%
\pgfusepath{clip}%
\pgfsetbuttcap%
\pgfsetroundjoin%
\definecolor{currentfill}{rgb}{0.274952,0.037752,0.364543}%
\pgfsetfillcolor{currentfill}%
\pgfsetfillopacity{0.700000}%
\pgfsetlinewidth{0.000000pt}%
\definecolor{currentstroke}{rgb}{0.000000,0.000000,0.000000}%
\pgfsetstrokecolor{currentstroke}%
\pgfsetdash{}{0pt}%
\pgfpathmoveto{\pgfqpoint{3.673290in}{1.378580in}}%
\pgfpathlineto{\pgfqpoint{3.686811in}{1.377595in}}%
\pgfpathlineto{\pgfqpoint{3.700339in}{1.376771in}}%
\pgfpathlineto{\pgfqpoint{3.713875in}{1.376108in}}%
\pgfpathlineto{\pgfqpoint{3.727417in}{1.375604in}}%
\pgfpathlineto{\pgfqpoint{3.735415in}{1.386007in}}%
\pgfpathlineto{\pgfqpoint{3.743406in}{1.396539in}}%
\pgfpathlineto{\pgfqpoint{3.751390in}{1.407195in}}%
\pgfpathlineto{\pgfqpoint{3.759369in}{1.417970in}}%
\pgfpathlineto{\pgfqpoint{3.745839in}{1.417918in}}%
\pgfpathlineto{\pgfqpoint{3.732316in}{1.418026in}}%
\pgfpathlineto{\pgfqpoint{3.718800in}{1.418294in}}%
\pgfpathlineto{\pgfqpoint{3.705292in}{1.418723in}}%
\pgfpathlineto{\pgfqpoint{3.697301in}{1.408494in}}%
\pgfpathlineto{\pgfqpoint{3.689304in}{1.398390in}}%
\pgfpathlineto{\pgfqpoint{3.681300in}{1.388417in}}%
\pgfpathlineto{\pgfqpoint{3.673290in}{1.378580in}}%
\pgfpathclose%
\pgfusepath{fill}%
\end{pgfscope}%
\begin{pgfscope}%
\pgfpathrectangle{\pgfqpoint{1.254980in}{0.150000in}}{\pgfqpoint{5.490039in}{5.490039in}}%
\pgfusepath{clip}%
\pgfsetbuttcap%
\pgfsetroundjoin%
\definecolor{currentfill}{rgb}{0.281924,0.089666,0.412415}%
\pgfsetfillcolor{currentfill}%
\pgfsetfillopacity{0.700000}%
\pgfsetlinewidth{0.000000pt}%
\definecolor{currentstroke}{rgb}{0.000000,0.000000,0.000000}%
\pgfsetstrokecolor{currentstroke}%
\pgfsetdash{}{0pt}%
\pgfpathmoveto{\pgfqpoint{3.845386in}{1.466066in}}%
\pgfpathlineto{\pgfqpoint{3.858947in}{1.467444in}}%
\pgfpathlineto{\pgfqpoint{3.872517in}{1.468980in}}%
\pgfpathlineto{\pgfqpoint{3.886096in}{1.470676in}}%
\pgfpathlineto{\pgfqpoint{3.899683in}{1.472530in}}%
\pgfpathlineto{\pgfqpoint{3.907616in}{1.484842in}}%
\pgfpathlineto{\pgfqpoint{3.915545in}{1.497229in}}%
\pgfpathlineto{\pgfqpoint{3.923468in}{1.509686in}}%
\pgfpathlineto{\pgfqpoint{3.931386in}{1.522206in}}%
\pgfpathlineto{\pgfqpoint{3.917805in}{1.519850in}}%
\pgfpathlineto{\pgfqpoint{3.904233in}{1.517652in}}%
\pgfpathlineto{\pgfqpoint{3.890671in}{1.515613in}}%
\pgfpathlineto{\pgfqpoint{3.877117in}{1.513733in}}%
\pgfpathlineto{\pgfqpoint{3.869192in}{1.501704in}}%
\pgfpathlineto{\pgfqpoint{3.861262in}{1.489747in}}%
\pgfpathlineto{\pgfqpoint{3.853326in}{1.477866in}}%
\pgfpathlineto{\pgfqpoint{3.845386in}{1.466066in}}%
\pgfpathclose%
\pgfusepath{fill}%
\end{pgfscope}%
\begin{pgfscope}%
\pgfpathrectangle{\pgfqpoint{1.254980in}{0.150000in}}{\pgfqpoint{5.490039in}{5.490039in}}%
\pgfusepath{clip}%
\pgfsetbuttcap%
\pgfsetroundjoin%
\definecolor{currentfill}{rgb}{0.157729,0.485932,0.558013}%
\pgfsetfillcolor{currentfill}%
\pgfsetfillopacity{0.700000}%
\pgfsetlinewidth{0.000000pt}%
\definecolor{currentstroke}{rgb}{0.000000,0.000000,0.000000}%
\pgfsetstrokecolor{currentstroke}%
\pgfsetdash{}{0pt}%
\pgfpathmoveto{\pgfqpoint{4.690860in}{2.363587in}}%
\pgfpathlineto{\pgfqpoint{4.704821in}{2.374457in}}%
\pgfpathlineto{\pgfqpoint{4.718799in}{2.385489in}}%
\pgfpathlineto{\pgfqpoint{4.732793in}{2.396682in}}%
\pgfpathlineto{\pgfqpoint{4.746804in}{2.408037in}}%
\pgfpathlineto{\pgfqpoint{4.754493in}{2.420626in}}%
\pgfpathlineto{\pgfqpoint{4.762176in}{2.433076in}}%
\pgfpathlineto{\pgfqpoint{4.769853in}{2.445387in}}%
\pgfpathlineto{\pgfqpoint{4.777523in}{2.457559in}}%
\pgfpathlineto{\pgfqpoint{4.763510in}{2.446066in}}%
\pgfpathlineto{\pgfqpoint{4.749515in}{2.434734in}}%
\pgfpathlineto{\pgfqpoint{4.735536in}{2.423565in}}%
\pgfpathlineto{\pgfqpoint{4.721574in}{2.412557in}}%
\pgfpathlineto{\pgfqpoint{4.713905in}{2.400512in}}%
\pgfpathlineto{\pgfqpoint{4.706229in}{2.388335in}}%
\pgfpathlineto{\pgfqpoint{4.698548in}{2.376026in}}%
\pgfpathlineto{\pgfqpoint{4.690860in}{2.363587in}}%
\pgfpathclose%
\pgfusepath{fill}%
\end{pgfscope}%
\begin{pgfscope}%
\pgfpathrectangle{\pgfqpoint{1.254980in}{0.150000in}}{\pgfqpoint{5.490039in}{5.490039in}}%
\pgfusepath{clip}%
\pgfsetbuttcap%
\pgfsetroundjoin%
\definecolor{currentfill}{rgb}{0.506271,0.828786,0.300362}%
\pgfsetfillcolor{currentfill}%
\pgfsetfillopacity{0.700000}%
\pgfsetlinewidth{0.000000pt}%
\definecolor{currentstroke}{rgb}{0.000000,0.000000,0.000000}%
\pgfsetstrokecolor{currentstroke}%
\pgfsetdash{}{0pt}%
\pgfpathmoveto{\pgfqpoint{5.709155in}{3.398020in}}%
\pgfpathlineto{\pgfqpoint{5.723811in}{3.414253in}}%
\pgfpathlineto{\pgfqpoint{5.738489in}{3.430652in}}%
\pgfpathlineto{\pgfqpoint{5.753191in}{3.447217in}}%
\pgfpathlineto{\pgfqpoint{5.760259in}{3.449319in}}%
\pgfpathlineto{\pgfqpoint{5.767317in}{3.451286in}}%
\pgfpathlineto{\pgfqpoint{5.774363in}{3.453121in}}%
\pgfpathlineto{\pgfqpoint{5.781399in}{3.454830in}}%
\pgfpathlineto{\pgfqpoint{5.766719in}{3.438602in}}%
\pgfpathlineto{\pgfqpoint{5.752062in}{3.422539in}}%
\pgfpathlineto{\pgfqpoint{5.737427in}{3.406641in}}%
\pgfpathlineto{\pgfqpoint{5.730375in}{3.404673in}}%
\pgfpathlineto{\pgfqpoint{5.723313in}{3.402582in}}%
\pgfpathlineto{\pgfqpoint{5.716239in}{3.400366in}}%
\pgfpathlineto{\pgfqpoint{5.709155in}{3.398020in}}%
\pgfpathclose%
\pgfusepath{fill}%
\end{pgfscope}%
\begin{pgfscope}%
\pgfpathrectangle{\pgfqpoint{1.254980in}{0.150000in}}{\pgfqpoint{5.490039in}{5.490039in}}%
\pgfusepath{clip}%
\pgfsetbuttcap%
\pgfsetroundjoin%
\definecolor{currentfill}{rgb}{0.271305,0.019942,0.347269}%
\pgfsetfillcolor{currentfill}%
\pgfsetfillopacity{0.700000}%
\pgfsetlinewidth{0.000000pt}%
\definecolor{currentstroke}{rgb}{0.000000,0.000000,0.000000}%
\pgfsetstrokecolor{currentstroke}%
\pgfsetdash{}{0pt}%
\pgfpathmoveto{\pgfqpoint{3.587095in}{1.348583in}}%
\pgfpathlineto{\pgfqpoint{3.600607in}{1.346371in}}%
\pgfpathlineto{\pgfqpoint{3.614124in}{1.344320in}}%
\pgfpathlineto{\pgfqpoint{3.627648in}{1.342431in}}%
\pgfpathlineto{\pgfqpoint{3.641178in}{1.340703in}}%
\pgfpathlineto{\pgfqpoint{3.649217in}{1.349940in}}%
\pgfpathlineto{\pgfqpoint{3.657248in}{1.359336in}}%
\pgfpathlineto{\pgfqpoint{3.665272in}{1.368884in}}%
\pgfpathlineto{\pgfqpoint{3.673290in}{1.378580in}}%
\pgfpathlineto{\pgfqpoint{3.659775in}{1.379725in}}%
\pgfpathlineto{\pgfqpoint{3.646266in}{1.381031in}}%
\pgfpathlineto{\pgfqpoint{3.632764in}{1.382499in}}%
\pgfpathlineto{\pgfqpoint{3.619269in}{1.384128in}}%
\pgfpathlineto{\pgfqpoint{3.611236in}{1.375005in}}%
\pgfpathlineto{\pgfqpoint{3.603197in}{1.366036in}}%
\pgfpathlineto{\pgfqpoint{3.595149in}{1.357227in}}%
\pgfpathlineto{\pgfqpoint{3.587095in}{1.348583in}}%
\pgfpathclose%
\pgfusepath{fill}%
\end{pgfscope}%
\begin{pgfscope}%
\pgfpathrectangle{\pgfqpoint{1.254980in}{0.150000in}}{\pgfqpoint{5.490039in}{5.490039in}}%
\pgfusepath{clip}%
\pgfsetbuttcap%
\pgfsetroundjoin%
\definecolor{currentfill}{rgb}{0.269308,0.218818,0.509577}%
\pgfsetfillcolor{currentfill}%
\pgfsetfillopacity{0.700000}%
\pgfsetlinewidth{0.000000pt}%
\definecolor{currentstroke}{rgb}{0.000000,0.000000,0.000000}%
\pgfsetstrokecolor{currentstroke}%
\pgfsetdash{}{0pt}%
\pgfpathmoveto{\pgfqpoint{4.134959in}{1.711182in}}%
\pgfpathlineto{\pgfqpoint{4.148625in}{1.716311in}}%
\pgfpathlineto{\pgfqpoint{4.162302in}{1.721598in}}%
\pgfpathlineto{\pgfqpoint{4.175991in}{1.727044in}}%
\pgfpathlineto{\pgfqpoint{4.189692in}{1.732649in}}%
\pgfpathlineto{\pgfqpoint{4.197544in}{1.746823in}}%
\pgfpathlineto{\pgfqpoint{4.205393in}{1.760983in}}%
\pgfpathlineto{\pgfqpoint{4.213238in}{1.775123in}}%
\pgfpathlineto{\pgfqpoint{4.221078in}{1.789242in}}%
\pgfpathlineto{\pgfqpoint{4.207377in}{1.783242in}}%
\pgfpathlineto{\pgfqpoint{4.193689in}{1.777400in}}%
\pgfpathlineto{\pgfqpoint{4.180012in}{1.771718in}}%
\pgfpathlineto{\pgfqpoint{4.166347in}{1.766194in}}%
\pgfpathlineto{\pgfqpoint{4.158507in}{1.752461in}}%
\pgfpathlineto{\pgfqpoint{4.150662in}{1.738711in}}%
\pgfpathlineto{\pgfqpoint{4.142813in}{1.724950in}}%
\pgfpathlineto{\pgfqpoint{4.134959in}{1.711182in}}%
\pgfpathclose%
\pgfusepath{fill}%
\end{pgfscope}%
\begin{pgfscope}%
\pgfpathrectangle{\pgfqpoint{1.254980in}{0.150000in}}{\pgfqpoint{5.490039in}{5.490039in}}%
\pgfusepath{clip}%
\pgfsetbuttcap%
\pgfsetroundjoin%
\definecolor{currentfill}{rgb}{0.171176,0.452530,0.557965}%
\pgfsetfillcolor{currentfill}%
\pgfsetfillopacity{0.700000}%
\pgfsetlinewidth{0.000000pt}%
\definecolor{currentstroke}{rgb}{0.000000,0.000000,0.000000}%
\pgfsetstrokecolor{currentstroke}%
\pgfsetdash{}{0pt}%
\pgfpathmoveto{\pgfqpoint{2.257755in}{2.380489in}}%
\pgfpathlineto{\pgfqpoint{2.271653in}{2.358092in}}%
\pgfpathlineto{\pgfqpoint{2.285538in}{2.335958in}}%
\pgfpathlineto{\pgfqpoint{2.299411in}{2.314085in}}%
\pgfpathlineto{\pgfqpoint{2.313271in}{2.292471in}}%
\pgfpathlineto{\pgfqpoint{2.322451in}{2.282297in}}%
\pgfpathlineto{\pgfqpoint{2.331602in}{2.272565in}}%
\pgfpathlineto{\pgfqpoint{2.340725in}{2.263266in}}%
\pgfpathlineto{\pgfqpoint{2.349821in}{2.254392in}}%
\pgfpathlineto{\pgfqpoint{2.336029in}{2.275246in}}%
\pgfpathlineto{\pgfqpoint{2.322225in}{2.296356in}}%
\pgfpathlineto{\pgfqpoint{2.308409in}{2.317726in}}%
\pgfpathlineto{\pgfqpoint{2.294581in}{2.339358in}}%
\pgfpathlineto{\pgfqpoint{2.285418in}{2.348980in}}%
\pgfpathlineto{\pgfqpoint{2.276226in}{2.359038in}}%
\pgfpathlineto{\pgfqpoint{2.267005in}{2.369538in}}%
\pgfpathlineto{\pgfqpoint{2.257755in}{2.380489in}}%
\pgfpathclose%
\pgfusepath{fill}%
\end{pgfscope}%
\begin{pgfscope}%
\pgfpathrectangle{\pgfqpoint{1.254980in}{0.150000in}}{\pgfqpoint{5.490039in}{5.490039in}}%
\pgfusepath{clip}%
\pgfsetbuttcap%
\pgfsetroundjoin%
\definecolor{currentfill}{rgb}{0.179019,0.433756,0.557430}%
\pgfsetfillcolor{currentfill}%
\pgfsetfillopacity{0.700000}%
\pgfsetlinewidth{0.000000pt}%
\definecolor{currentstroke}{rgb}{0.000000,0.000000,0.000000}%
\pgfsetstrokecolor{currentstroke}%
\pgfsetdash{}{0pt}%
\pgfpathmoveto{\pgfqpoint{4.573486in}{2.219125in}}%
\pgfpathlineto{\pgfqpoint{4.587379in}{2.228985in}}%
\pgfpathlineto{\pgfqpoint{4.601287in}{2.239007in}}%
\pgfpathlineto{\pgfqpoint{4.615212in}{2.249189in}}%
\pgfpathlineto{\pgfqpoint{4.629152in}{2.259532in}}%
\pgfpathlineto{\pgfqpoint{4.636885in}{2.272968in}}%
\pgfpathlineto{\pgfqpoint{4.644613in}{2.286284in}}%
\pgfpathlineto{\pgfqpoint{4.652335in}{2.299479in}}%
\pgfpathlineto{\pgfqpoint{4.660052in}{2.312552in}}%
\pgfpathlineto{\pgfqpoint{4.646109in}{2.302011in}}%
\pgfpathlineto{\pgfqpoint{4.632182in}{2.291631in}}%
\pgfpathlineto{\pgfqpoint{4.618272in}{2.281413in}}%
\pgfpathlineto{\pgfqpoint{4.604377in}{2.271356in}}%
\pgfpathlineto{\pgfqpoint{4.596662in}{2.258469in}}%
\pgfpathlineto{\pgfqpoint{4.588942in}{2.245467in}}%
\pgfpathlineto{\pgfqpoint{4.581217in}{2.232352in}}%
\pgfpathlineto{\pgfqpoint{4.573486in}{2.219125in}}%
\pgfpathclose%
\pgfusepath{fill}%
\end{pgfscope}%
\begin{pgfscope}%
\pgfpathrectangle{\pgfqpoint{1.254980in}{0.150000in}}{\pgfqpoint{5.490039in}{5.490039in}}%
\pgfusepath{clip}%
\pgfsetbuttcap%
\pgfsetroundjoin%
\definecolor{currentfill}{rgb}{0.283187,0.125848,0.444960}%
\pgfsetfillcolor{currentfill}%
\pgfsetfillopacity{0.700000}%
\pgfsetlinewidth{0.000000pt}%
\definecolor{currentstroke}{rgb}{0.000000,0.000000,0.000000}%
\pgfsetstrokecolor{currentstroke}%
\pgfsetdash{}{0pt}%
\pgfpathmoveto{\pgfqpoint{3.931386in}{1.522206in}}%
\pgfpathlineto{\pgfqpoint{3.944977in}{1.524721in}}%
\pgfpathlineto{\pgfqpoint{3.958577in}{1.527395in}}%
\pgfpathlineto{\pgfqpoint{3.972187in}{1.530227in}}%
\pgfpathlineto{\pgfqpoint{3.985807in}{1.533216in}}%
\pgfpathlineto{\pgfqpoint{3.993715in}{1.546283in}}%
\pgfpathlineto{\pgfqpoint{4.001619in}{1.559398in}}%
\pgfpathlineto{\pgfqpoint{4.009518in}{1.572557in}}%
\pgfpathlineto{\pgfqpoint{4.017413in}{1.585756in}}%
\pgfpathlineto{\pgfqpoint{4.003797in}{1.582290in}}%
\pgfpathlineto{\pgfqpoint{3.990192in}{1.578982in}}%
\pgfpathlineto{\pgfqpoint{3.976597in}{1.575833in}}%
\pgfpathlineto{\pgfqpoint{3.963011in}{1.572842in}}%
\pgfpathlineto{\pgfqpoint{3.955112in}{1.560109in}}%
\pgfpathlineto{\pgfqpoint{3.947208in}{1.547423in}}%
\pgfpathlineto{\pgfqpoint{3.939300in}{1.534787in}}%
\pgfpathlineto{\pgfqpoint{3.931386in}{1.522206in}}%
\pgfpathclose%
\pgfusepath{fill}%
\end{pgfscope}%
\begin{pgfscope}%
\pgfpathrectangle{\pgfqpoint{1.254980in}{0.150000in}}{\pgfqpoint{5.490039in}{5.490039in}}%
\pgfusepath{clip}%
\pgfsetbuttcap%
\pgfsetroundjoin%
\definecolor{currentfill}{rgb}{0.120081,0.622161,0.534946}%
\pgfsetfillcolor{currentfill}%
\pgfsetfillopacity{0.700000}%
\pgfsetlinewidth{0.000000pt}%
\definecolor{currentstroke}{rgb}{0.000000,0.000000,0.000000}%
\pgfsetstrokecolor{currentstroke}%
\pgfsetdash{}{0pt}%
\pgfpathmoveto{\pgfqpoint{5.012052in}{2.732605in}}%
\pgfpathlineto{\pgfqpoint{5.026225in}{2.745812in}}%
\pgfpathlineto{\pgfqpoint{5.040416in}{2.759182in}}%
\pgfpathlineto{\pgfqpoint{5.054626in}{2.772716in}}%
\pgfpathlineto{\pgfqpoint{5.068856in}{2.786414in}}%
\pgfpathlineto{\pgfqpoint{5.076401in}{2.796130in}}%
\pgfpathlineto{\pgfqpoint{5.083937in}{2.805679in}}%
\pgfpathlineto{\pgfqpoint{5.091466in}{2.815062in}}%
\pgfpathlineto{\pgfqpoint{5.098985in}{2.824280in}}%
\pgfpathlineto{\pgfqpoint{5.084759in}{2.810597in}}%
\pgfpathlineto{\pgfqpoint{5.070551in}{2.797078in}}%
\pgfpathlineto{\pgfqpoint{5.056363in}{2.783723in}}%
\pgfpathlineto{\pgfqpoint{5.042193in}{2.770531in}}%
\pgfpathlineto{\pgfqpoint{5.034670in}{2.761287in}}%
\pgfpathlineto{\pgfqpoint{5.027139in}{2.751885in}}%
\pgfpathlineto{\pgfqpoint{5.019599in}{2.742324in}}%
\pgfpathlineto{\pgfqpoint{5.012052in}{2.732605in}}%
\pgfpathclose%
\pgfusepath{fill}%
\end{pgfscope}%
\begin{pgfscope}%
\pgfpathrectangle{\pgfqpoint{1.254980in}{0.150000in}}{\pgfqpoint{5.490039in}{5.490039in}}%
\pgfusepath{clip}%
\pgfsetbuttcap%
\pgfsetroundjoin%
\definecolor{currentfill}{rgb}{0.282656,0.100196,0.422160}%
\pgfsetfillcolor{currentfill}%
\pgfsetfillopacity{0.700000}%
\pgfsetlinewidth{0.000000pt}%
\definecolor{currentstroke}{rgb}{0.000000,0.000000,0.000000}%
\pgfsetstrokecolor{currentstroke}%
\pgfsetdash{}{0pt}%
\pgfpathmoveto{\pgfqpoint{2.969034in}{1.523609in}}%
\pgfpathlineto{\pgfqpoint{2.982583in}{1.512590in}}%
\pgfpathlineto{\pgfqpoint{2.996131in}{1.501756in}}%
\pgfpathlineto{\pgfqpoint{3.009677in}{1.491105in}}%
\pgfpathlineto{\pgfqpoint{3.023223in}{1.480637in}}%
\pgfpathlineto{\pgfqpoint{3.031690in}{1.480093in}}%
\pgfpathlineto{\pgfqpoint{3.040142in}{1.479878in}}%
\pgfpathlineto{\pgfqpoint{3.048577in}{1.479985in}}%
\pgfpathlineto{\pgfqpoint{3.056997in}{1.480406in}}%
\pgfpathlineto{\pgfqpoint{3.043492in}{1.490168in}}%
\pgfpathlineto{\pgfqpoint{3.029986in}{1.500112in}}%
\pgfpathlineto{\pgfqpoint{3.016480in}{1.510239in}}%
\pgfpathlineto{\pgfqpoint{3.002972in}{1.520550in}}%
\pgfpathlineto{\pgfqpoint{2.994512in}{1.520823in}}%
\pgfpathlineto{\pgfqpoint{2.986036in}{1.521419in}}%
\pgfpathlineto{\pgfqpoint{2.977544in}{1.522345in}}%
\pgfpathlineto{\pgfqpoint{2.969034in}{1.523609in}}%
\pgfpathclose%
\pgfusepath{fill}%
\end{pgfscope}%
\begin{pgfscope}%
\pgfpathrectangle{\pgfqpoint{1.254980in}{0.150000in}}{\pgfqpoint{5.490039in}{5.490039in}}%
\pgfusepath{clip}%
\pgfsetbuttcap%
\pgfsetroundjoin%
\definecolor{currentfill}{rgb}{0.268510,0.009605,0.335427}%
\pgfsetfillcolor{currentfill}%
\pgfsetfillopacity{0.700000}%
\pgfsetlinewidth{0.000000pt}%
\definecolor{currentstroke}{rgb}{0.000000,0.000000,0.000000}%
\pgfsetstrokecolor{currentstroke}%
\pgfsetdash{}{0pt}%
\pgfpathmoveto{\pgfqpoint{3.360099in}{1.340409in}}%
\pgfpathlineto{\pgfqpoint{3.373600in}{1.334974in}}%
\pgfpathlineto{\pgfqpoint{3.387105in}{1.329707in}}%
\pgfpathlineto{\pgfqpoint{3.400613in}{1.324605in}}%
\pgfpathlineto{\pgfqpoint{3.414124in}{1.319670in}}%
\pgfpathlineto{\pgfqpoint{3.422291in}{1.325474in}}%
\pgfpathlineto{\pgfqpoint{3.430449in}{1.331506in}}%
\pgfpathlineto{\pgfqpoint{3.438596in}{1.337760in}}%
\pgfpathlineto{\pgfqpoint{3.446734in}{1.344231in}}%
\pgfpathlineto{\pgfqpoint{3.433246in}{1.348527in}}%
\pgfpathlineto{\pgfqpoint{3.419763in}{1.352988in}}%
\pgfpathlineto{\pgfqpoint{3.406283in}{1.357617in}}%
\pgfpathlineto{\pgfqpoint{3.392807in}{1.362412in}}%
\pgfpathlineto{\pgfqpoint{3.384646in}{1.356570in}}%
\pgfpathlineto{\pgfqpoint{3.376474in}{1.350951in}}%
\pgfpathlineto{\pgfqpoint{3.368292in}{1.345562in}}%
\pgfpathlineto{\pgfqpoint{3.360099in}{1.340409in}}%
\pgfpathclose%
\pgfusepath{fill}%
\end{pgfscope}%
\begin{pgfscope}%
\pgfpathrectangle{\pgfqpoint{1.254980in}{0.150000in}}{\pgfqpoint{5.490039in}{5.490039in}}%
\pgfusepath{clip}%
\pgfsetbuttcap%
\pgfsetroundjoin%
\definecolor{currentfill}{rgb}{0.304148,0.764704,0.419943}%
\pgfsetfillcolor{currentfill}%
\pgfsetfillopacity{0.700000}%
\pgfsetlinewidth{0.000000pt}%
\definecolor{currentstroke}{rgb}{0.000000,0.000000,0.000000}%
\pgfsetstrokecolor{currentstroke}%
\pgfsetdash{}{0pt}%
\pgfpathmoveto{\pgfqpoint{5.419395in}{3.145961in}}%
\pgfpathlineto{\pgfqpoint{5.433850in}{3.161231in}}%
\pgfpathlineto{\pgfqpoint{5.448325in}{3.176667in}}%
\pgfpathlineto{\pgfqpoint{5.462822in}{3.192267in}}%
\pgfpathlineto{\pgfqpoint{5.477341in}{3.208034in}}%
\pgfpathlineto{\pgfqpoint{5.484630in}{3.213289in}}%
\pgfpathlineto{\pgfqpoint{5.491909in}{3.218381in}}%
\pgfpathlineto{\pgfqpoint{5.499177in}{3.223312in}}%
\pgfpathlineto{\pgfqpoint{5.506434in}{3.228086in}}%
\pgfpathlineto{\pgfqpoint{5.491928in}{3.212527in}}%
\pgfpathlineto{\pgfqpoint{5.477444in}{3.197133in}}%
\pgfpathlineto{\pgfqpoint{5.462980in}{3.181903in}}%
\pgfpathlineto{\pgfqpoint{5.448539in}{3.166838in}}%
\pgfpathlineto{\pgfqpoint{5.441268in}{3.161847in}}%
\pgfpathlineto{\pgfqpoint{5.433987in}{3.156705in}}%
\pgfpathlineto{\pgfqpoint{5.426696in}{3.151410in}}%
\pgfpathlineto{\pgfqpoint{5.419395in}{3.145961in}}%
\pgfpathclose%
\pgfusepath{fill}%
\end{pgfscope}%
\begin{pgfscope}%
\pgfpathrectangle{\pgfqpoint{1.254980in}{0.150000in}}{\pgfqpoint{5.490039in}{5.490039in}}%
\pgfusepath{clip}%
\pgfsetbuttcap%
\pgfsetroundjoin%
\definecolor{currentfill}{rgb}{0.203063,0.379716,0.553925}%
\pgfsetfillcolor{currentfill}%
\pgfsetfillopacity{0.700000}%
\pgfsetlinewidth{0.000000pt}%
\definecolor{currentstroke}{rgb}{0.000000,0.000000,0.000000}%
\pgfsetstrokecolor{currentstroke}%
\pgfsetdash{}{0pt}%
\pgfpathmoveto{\pgfqpoint{4.456055in}{2.073587in}}%
\pgfpathlineto{\pgfqpoint{4.469883in}{2.082324in}}%
\pgfpathlineto{\pgfqpoint{4.483725in}{2.091221in}}%
\pgfpathlineto{\pgfqpoint{4.497582in}{2.100278in}}%
\pgfpathlineto{\pgfqpoint{4.511455in}{2.109496in}}%
\pgfpathlineto{\pgfqpoint{4.519226in}{2.123551in}}%
\pgfpathlineto{\pgfqpoint{4.526993in}{2.137511in}}%
\pgfpathlineto{\pgfqpoint{4.534754in}{2.151372in}}%
\pgfpathlineto{\pgfqpoint{4.542511in}{2.165132in}}%
\pgfpathlineto{\pgfqpoint{4.528636in}{2.155658in}}%
\pgfpathlineto{\pgfqpoint{4.514776in}{2.146345in}}%
\pgfpathlineto{\pgfqpoint{4.500931in}{2.137193in}}%
\pgfpathlineto{\pgfqpoint{4.487101in}{2.128201in}}%
\pgfpathlineto{\pgfqpoint{4.479347in}{2.114685in}}%
\pgfpathlineto{\pgfqpoint{4.471588in}{2.101076in}}%
\pgfpathlineto{\pgfqpoint{4.463824in}{2.087376in}}%
\pgfpathlineto{\pgfqpoint{4.456055in}{2.073587in}}%
\pgfpathclose%
\pgfusepath{fill}%
\end{pgfscope}%
\begin{pgfscope}%
\pgfpathrectangle{\pgfqpoint{1.254980in}{0.150000in}}{\pgfqpoint{5.490039in}{5.490039in}}%
\pgfusepath{clip}%
\pgfsetbuttcap%
\pgfsetroundjoin%
\definecolor{currentfill}{rgb}{0.119483,0.614817,0.537692}%
\pgfsetfillcolor{currentfill}%
\pgfsetfillopacity{0.700000}%
\pgfsetlinewidth{0.000000pt}%
\definecolor{currentstroke}{rgb}{0.000000,0.000000,0.000000}%
\pgfsetstrokecolor{currentstroke}%
\pgfsetdash{}{0pt}%
\pgfpathmoveto{\pgfqpoint{2.015027in}{2.830791in}}%
\pgfpathlineto{\pgfqpoint{2.029160in}{2.803384in}}%
\pgfpathlineto{\pgfqpoint{2.043276in}{2.776294in}}%
\pgfpathlineto{\pgfqpoint{2.057373in}{2.749517in}}%
\pgfpathlineto{\pgfqpoint{2.071452in}{2.723050in}}%
\pgfpathlineto{\pgfqpoint{2.080879in}{2.710823in}}%
\pgfpathlineto{\pgfqpoint{2.090274in}{2.699052in}}%
\pgfpathlineto{\pgfqpoint{2.099638in}{2.687732in}}%
\pgfpathlineto{\pgfqpoint{2.108971in}{2.676852in}}%
\pgfpathlineto{\pgfqpoint{2.094968in}{2.702559in}}%
\pgfpathlineto{\pgfqpoint{2.080948in}{2.728573in}}%
\pgfpathlineto{\pgfqpoint{2.066911in}{2.754898in}}%
\pgfpathlineto{\pgfqpoint{2.052856in}{2.781537in}}%
\pgfpathlineto{\pgfqpoint{2.043447in}{2.793165in}}%
\pgfpathlineto{\pgfqpoint{2.034006in}{2.805245in}}%
\pgfpathlineto{\pgfqpoint{2.024533in}{2.817784in}}%
\pgfpathlineto{\pgfqpoint{2.015027in}{2.830791in}}%
\pgfpathclose%
\pgfusepath{fill}%
\end{pgfscope}%
\begin{pgfscope}%
\pgfpathrectangle{\pgfqpoint{1.254980in}{0.150000in}}{\pgfqpoint{5.490039in}{5.490039in}}%
\pgfusepath{clip}%
\pgfsetbuttcap%
\pgfsetroundjoin%
\definecolor{currentfill}{rgb}{0.180653,0.701402,0.488189}%
\pgfsetfillcolor{currentfill}%
\pgfsetfillopacity{0.700000}%
\pgfsetlinewidth{0.000000pt}%
\definecolor{currentstroke}{rgb}{0.000000,0.000000,0.000000}%
\pgfsetstrokecolor{currentstroke}%
\pgfsetdash{}{0pt}%
\pgfpathmoveto{\pgfqpoint{5.215897in}{2.948035in}}%
\pgfpathlineto{\pgfqpoint{5.230213in}{2.962413in}}%
\pgfpathlineto{\pgfqpoint{5.244548in}{2.976956in}}%
\pgfpathlineto{\pgfqpoint{5.258904in}{2.991664in}}%
\pgfpathlineto{\pgfqpoint{5.273280in}{3.006537in}}%
\pgfpathlineto{\pgfqpoint{5.280709in}{3.014110in}}%
\pgfpathlineto{\pgfqpoint{5.288128in}{3.021512in}}%
\pgfpathlineto{\pgfqpoint{5.295538in}{3.028746in}}%
\pgfpathlineto{\pgfqpoint{5.302938in}{3.035812in}}%
\pgfpathlineto{\pgfqpoint{5.288569in}{3.021050in}}%
\pgfpathlineto{\pgfqpoint{5.274220in}{3.006452in}}%
\pgfpathlineto{\pgfqpoint{5.259892in}{2.992019in}}%
\pgfpathlineto{\pgfqpoint{5.245583in}{2.977750in}}%
\pgfpathlineto{\pgfqpoint{5.238176in}{2.970562in}}%
\pgfpathlineto{\pgfqpoint{5.230759in}{2.963215in}}%
\pgfpathlineto{\pgfqpoint{5.223333in}{2.955706in}}%
\pgfpathlineto{\pgfqpoint{5.215897in}{2.948035in}}%
\pgfpathclose%
\pgfusepath{fill}%
\end{pgfscope}%
\begin{pgfscope}%
\pgfpathrectangle{\pgfqpoint{1.254980in}{0.150000in}}{\pgfqpoint{5.490039in}{5.490039in}}%
\pgfusepath{clip}%
\pgfsetbuttcap%
\pgfsetroundjoin%
\definecolor{currentfill}{rgb}{0.272594,0.025563,0.353093}%
\pgfsetfillcolor{currentfill}%
\pgfsetfillopacity{0.700000}%
\pgfsetlinewidth{0.000000pt}%
\definecolor{currentstroke}{rgb}{0.000000,0.000000,0.000000}%
\pgfsetstrokecolor{currentstroke}%
\pgfsetdash{}{0pt}%
\pgfpathmoveto{\pgfqpoint{3.219069in}{1.377204in}}%
\pgfpathlineto{\pgfqpoint{3.232581in}{1.369742in}}%
\pgfpathlineto{\pgfqpoint{3.246094in}{1.362453in}}%
\pgfpathlineto{\pgfqpoint{3.259609in}{1.355334in}}%
\pgfpathlineto{\pgfqpoint{3.273126in}{1.348386in}}%
\pgfpathlineto{\pgfqpoint{3.281394in}{1.351849in}}%
\pgfpathlineto{\pgfqpoint{3.289649in}{1.355581in}}%
\pgfpathlineto{\pgfqpoint{3.297892in}{1.359577in}}%
\pgfpathlineto{\pgfqpoint{3.306124in}{1.363829in}}%
\pgfpathlineto{\pgfqpoint{3.292637in}{1.370107in}}%
\pgfpathlineto{\pgfqpoint{3.279152in}{1.376556in}}%
\pgfpathlineto{\pgfqpoint{3.265669in}{1.383176in}}%
\pgfpathlineto{\pgfqpoint{3.252189in}{1.389967in}}%
\pgfpathlineto{\pgfqpoint{3.243928in}{1.386373in}}%
\pgfpathlineto{\pgfqpoint{3.235654in}{1.383044in}}%
\pgfpathlineto{\pgfqpoint{3.227368in}{1.379985in}}%
\pgfpathlineto{\pgfqpoint{3.219069in}{1.377204in}}%
\pgfpathclose%
\pgfusepath{fill}%
\end{pgfscope}%
\begin{pgfscope}%
\pgfpathrectangle{\pgfqpoint{1.254980in}{0.150000in}}{\pgfqpoint{5.490039in}{5.490039in}}%
\pgfusepath{clip}%
\pgfsetbuttcap%
\pgfsetroundjoin%
\definecolor{currentfill}{rgb}{0.268510,0.009605,0.335427}%
\pgfsetfillcolor{currentfill}%
\pgfsetfillopacity{0.700000}%
\pgfsetlinewidth{0.000000pt}%
\definecolor{currentstroke}{rgb}{0.000000,0.000000,0.000000}%
\pgfsetstrokecolor{currentstroke}%
\pgfsetdash{}{0pt}%
\pgfpathmoveto{\pgfqpoint{3.500727in}{1.328697in}}%
\pgfpathlineto{\pgfqpoint{3.514237in}{1.325223in}}%
\pgfpathlineto{\pgfqpoint{3.527751in}{1.321913in}}%
\pgfpathlineto{\pgfqpoint{3.541271in}{1.318765in}}%
\pgfpathlineto{\pgfqpoint{3.554796in}{1.315780in}}%
\pgfpathlineto{\pgfqpoint{3.562883in}{1.323703in}}%
\pgfpathlineto{\pgfqpoint{3.570961in}{1.331815in}}%
\pgfpathlineto{\pgfqpoint{3.579032in}{1.340110in}}%
\pgfpathlineto{\pgfqpoint{3.587095in}{1.348583in}}%
\pgfpathlineto{\pgfqpoint{3.573588in}{1.350957in}}%
\pgfpathlineto{\pgfqpoint{3.560088in}{1.353494in}}%
\pgfpathlineto{\pgfqpoint{3.546592in}{1.356193in}}%
\pgfpathlineto{\pgfqpoint{3.533102in}{1.359056in}}%
\pgfpathlineto{\pgfqpoint{3.525021in}{1.351184in}}%
\pgfpathlineto{\pgfqpoint{3.516932in}{1.343496in}}%
\pgfpathlineto{\pgfqpoint{3.508834in}{1.335998in}}%
\pgfpathlineto{\pgfqpoint{3.500727in}{1.328697in}}%
\pgfpathclose%
\pgfusepath{fill}%
\end{pgfscope}%
\begin{pgfscope}%
\pgfpathrectangle{\pgfqpoint{1.254980in}{0.150000in}}{\pgfqpoint{5.490039in}{5.490039in}}%
\pgfusepath{clip}%
\pgfsetbuttcap%
\pgfsetroundjoin%
\definecolor{currentfill}{rgb}{0.229739,0.322361,0.545706}%
\pgfsetfillcolor{currentfill}%
\pgfsetfillopacity{0.700000}%
\pgfsetlinewidth{0.000000pt}%
\definecolor{currentstroke}{rgb}{0.000000,0.000000,0.000000}%
\pgfsetstrokecolor{currentstroke}%
\pgfsetdash{}{0pt}%
\pgfpathmoveto{\pgfqpoint{4.338586in}{1.929393in}}%
\pgfpathlineto{\pgfqpoint{4.352354in}{1.936896in}}%
\pgfpathlineto{\pgfqpoint{4.366135in}{1.944558in}}%
\pgfpathlineto{\pgfqpoint{4.379929in}{1.952379in}}%
\pgfpathlineto{\pgfqpoint{4.393738in}{1.960359in}}%
\pgfpathlineto{\pgfqpoint{4.401543in}{1.974772in}}%
\pgfpathlineto{\pgfqpoint{4.409344in}{1.989116in}}%
\pgfpathlineto{\pgfqpoint{4.417141in}{2.003388in}}%
\pgfpathlineto{\pgfqpoint{4.424933in}{2.017587in}}%
\pgfpathlineto{\pgfqpoint{4.411122in}{2.009293in}}%
\pgfpathlineto{\pgfqpoint{4.397325in}{2.001160in}}%
\pgfpathlineto{\pgfqpoint{4.383542in}{1.993186in}}%
\pgfpathlineto{\pgfqpoint{4.369773in}{1.985371in}}%
\pgfpathlineto{\pgfqpoint{4.361983in}{1.971474in}}%
\pgfpathlineto{\pgfqpoint{4.354188in}{1.957511in}}%
\pgfpathlineto{\pgfqpoint{4.346389in}{1.943483in}}%
\pgfpathlineto{\pgfqpoint{4.338586in}{1.929393in}}%
\pgfpathclose%
\pgfusepath{fill}%
\end{pgfscope}%
\begin{pgfscope}%
\pgfpathrectangle{\pgfqpoint{1.254980in}{0.150000in}}{\pgfqpoint{5.490039in}{5.490039in}}%
\pgfusepath{clip}%
\pgfsetbuttcap%
\pgfsetroundjoin%
\definecolor{currentfill}{rgb}{0.280868,0.160771,0.472899}%
\pgfsetfillcolor{currentfill}%
\pgfsetfillopacity{0.700000}%
\pgfsetlinewidth{0.000000pt}%
\definecolor{currentstroke}{rgb}{0.000000,0.000000,0.000000}%
\pgfsetstrokecolor{currentstroke}%
\pgfsetdash{}{0pt}%
\pgfpathmoveto{\pgfqpoint{4.017413in}{1.585756in}}%
\pgfpathlineto{\pgfqpoint{4.031039in}{1.589380in}}%
\pgfpathlineto{\pgfqpoint{4.044675in}{1.593163in}}%
\pgfpathlineto{\pgfqpoint{4.058322in}{1.597103in}}%
\pgfpathlineto{\pgfqpoint{4.071980in}{1.601201in}}%
\pgfpathlineto{\pgfqpoint{4.079867in}{1.614895in}}%
\pgfpathlineto{\pgfqpoint{4.087750in}{1.628612in}}%
\pgfpathlineto{\pgfqpoint{4.095629in}{1.642349in}}%
\pgfpathlineto{\pgfqpoint{4.103503in}{1.656101in}}%
\pgfpathlineto{\pgfqpoint{4.089848in}{1.651552in}}%
\pgfpathlineto{\pgfqpoint{4.076204in}{1.647162in}}%
\pgfpathlineto{\pgfqpoint{4.062570in}{1.642930in}}%
\pgfpathlineto{\pgfqpoint{4.048947in}{1.638857in}}%
\pgfpathlineto{\pgfqpoint{4.041070in}{1.625544in}}%
\pgfpathlineto{\pgfqpoint{4.033189in}{1.612253in}}%
\pgfpathlineto{\pgfqpoint{4.025303in}{1.598989in}}%
\pgfpathlineto{\pgfqpoint{4.017413in}{1.585756in}}%
\pgfpathclose%
\pgfusepath{fill}%
\end{pgfscope}%
\begin{pgfscope}%
\pgfpathrectangle{\pgfqpoint{1.254980in}{0.150000in}}{\pgfqpoint{5.490039in}{5.490039in}}%
\pgfusepath{clip}%
\pgfsetbuttcap%
\pgfsetroundjoin%
\definecolor{currentfill}{rgb}{0.124395,0.578002,0.548287}%
\pgfsetfillcolor{currentfill}%
\pgfsetfillopacity{0.700000}%
\pgfsetlinewidth{0.000000pt}%
\definecolor{currentstroke}{rgb}{0.000000,0.000000,0.000000}%
\pgfsetstrokecolor{currentstroke}%
\pgfsetdash{}{0pt}%
\pgfpathmoveto{\pgfqpoint{4.894877in}{2.598149in}}%
\pgfpathlineto{\pgfqpoint{4.908978in}{2.610640in}}%
\pgfpathlineto{\pgfqpoint{4.923098in}{2.623294in}}%
\pgfpathlineto{\pgfqpoint{4.937235in}{2.636111in}}%
\pgfpathlineto{\pgfqpoint{4.951392in}{2.649092in}}%
\pgfpathlineto{\pgfqpoint{4.959001in}{2.660092in}}%
\pgfpathlineto{\pgfqpoint{4.966603in}{2.670932in}}%
\pgfpathlineto{\pgfqpoint{4.974197in}{2.681612in}}%
\pgfpathlineto{\pgfqpoint{4.981784in}{2.692131in}}%
\pgfpathlineto{\pgfqpoint{4.967628in}{2.679103in}}%
\pgfpathlineto{\pgfqpoint{4.953491in}{2.666239in}}%
\pgfpathlineto{\pgfqpoint{4.939372in}{2.653538in}}%
\pgfpathlineto{\pgfqpoint{4.925272in}{2.641000in}}%
\pgfpathlineto{\pgfqpoint{4.917684in}{2.630516in}}%
\pgfpathlineto{\pgfqpoint{4.910089in}{2.619880in}}%
\pgfpathlineto{\pgfqpoint{4.902486in}{2.609091in}}%
\pgfpathlineto{\pgfqpoint{4.894877in}{2.598149in}}%
\pgfpathclose%
\pgfusepath{fill}%
\end{pgfscope}%
\begin{pgfscope}%
\pgfpathrectangle{\pgfqpoint{1.254980in}{0.150000in}}{\pgfqpoint{5.490039in}{5.490039in}}%
\pgfusepath{clip}%
\pgfsetbuttcap%
\pgfsetroundjoin%
\definecolor{currentfill}{rgb}{0.156270,0.489624,0.557936}%
\pgfsetfillcolor{currentfill}%
\pgfsetfillopacity{0.700000}%
\pgfsetlinewidth{0.000000pt}%
\definecolor{currentstroke}{rgb}{0.000000,0.000000,0.000000}%
\pgfsetstrokecolor{currentstroke}%
\pgfsetdash{}{0pt}%
\pgfpathmoveto{\pgfqpoint{2.202031in}{2.472759in}}%
\pgfpathlineto{\pgfqpoint{2.215982in}{2.449284in}}%
\pgfpathlineto{\pgfqpoint{2.229920in}{2.426083in}}%
\pgfpathlineto{\pgfqpoint{2.243844in}{2.403152in}}%
\pgfpathlineto{\pgfqpoint{2.257755in}{2.380489in}}%
\pgfpathlineto{\pgfqpoint{2.267005in}{2.369538in}}%
\pgfpathlineto{\pgfqpoint{2.276226in}{2.359038in}}%
\pgfpathlineto{\pgfqpoint{2.285418in}{2.348980in}}%
\pgfpathlineto{\pgfqpoint{2.294581in}{2.339358in}}%
\pgfpathlineto{\pgfqpoint{2.280741in}{2.361253in}}%
\pgfpathlineto{\pgfqpoint{2.266888in}{2.383415in}}%
\pgfpathlineto{\pgfqpoint{2.253022in}{2.405845in}}%
\pgfpathlineto{\pgfqpoint{2.239143in}{2.428546in}}%
\pgfpathlineto{\pgfqpoint{2.229910in}{2.438925in}}%
\pgfpathlineto{\pgfqpoint{2.220647in}{2.449748in}}%
\pgfpathlineto{\pgfqpoint{2.211354in}{2.461023in}}%
\pgfpathlineto{\pgfqpoint{2.202031in}{2.472759in}}%
\pgfpathclose%
\pgfusepath{fill}%
\end{pgfscope}%
\begin{pgfscope}%
\pgfpathrectangle{\pgfqpoint{1.254980in}{0.150000in}}{\pgfqpoint{5.490039in}{5.490039in}}%
\pgfusepath{clip}%
\pgfsetbuttcap%
\pgfsetroundjoin%
\definecolor{currentfill}{rgb}{0.280894,0.078907,0.402329}%
\pgfsetfillcolor{currentfill}%
\pgfsetfillopacity{0.700000}%
\pgfsetlinewidth{0.000000pt}%
\definecolor{currentstroke}{rgb}{0.000000,0.000000,0.000000}%
\pgfsetstrokecolor{currentstroke}%
\pgfsetdash{}{0pt}%
\pgfpathmoveto{\pgfqpoint{3.023223in}{1.480637in}}%
\pgfpathlineto{\pgfqpoint{3.036767in}{1.470352in}}%
\pgfpathlineto{\pgfqpoint{3.050311in}{1.460248in}}%
\pgfpathlineto{\pgfqpoint{3.063854in}{1.450324in}}%
\pgfpathlineto{\pgfqpoint{3.077396in}{1.440579in}}%
\pgfpathlineto{\pgfqpoint{3.085824in}{1.440752in}}%
\pgfpathlineto{\pgfqpoint{3.094236in}{1.441245in}}%
\pgfpathlineto{\pgfqpoint{3.102633in}{1.442053in}}%
\pgfpathlineto{\pgfqpoint{3.111015in}{1.443168in}}%
\pgfpathlineto{\pgfqpoint{3.097511in}{1.452208in}}%
\pgfpathlineto{\pgfqpoint{3.084006in}{1.461427in}}%
\pgfpathlineto{\pgfqpoint{3.070502in}{1.470826in}}%
\pgfpathlineto{\pgfqpoint{3.056997in}{1.480406in}}%
\pgfpathlineto{\pgfqpoint{3.048577in}{1.479985in}}%
\pgfpathlineto{\pgfqpoint{3.040142in}{1.479878in}}%
\pgfpathlineto{\pgfqpoint{3.031690in}{1.480093in}}%
\pgfpathlineto{\pgfqpoint{3.023223in}{1.480637in}}%
\pgfpathclose%
\pgfusepath{fill}%
\end{pgfscope}%
\begin{pgfscope}%
\pgfpathrectangle{\pgfqpoint{1.254980in}{0.150000in}}{\pgfqpoint{5.490039in}{5.490039in}}%
\pgfusepath{clip}%
\pgfsetbuttcap%
\pgfsetroundjoin%
\definecolor{currentfill}{rgb}{0.255645,0.260703,0.528312}%
\pgfsetfillcolor{currentfill}%
\pgfsetfillopacity{0.700000}%
\pgfsetlinewidth{0.000000pt}%
\definecolor{currentstroke}{rgb}{0.000000,0.000000,0.000000}%
\pgfsetstrokecolor{currentstroke}%
\pgfsetdash{}{0pt}%
\pgfpathmoveto{\pgfqpoint{4.221078in}{1.789242in}}%
\pgfpathlineto{\pgfqpoint{4.234791in}{1.795400in}}%
\pgfpathlineto{\pgfqpoint{4.248517in}{1.801717in}}%
\pgfpathlineto{\pgfqpoint{4.262255in}{1.808193in}}%
\pgfpathlineto{\pgfqpoint{4.276006in}{1.814827in}}%
\pgfpathlineto{\pgfqpoint{4.283844in}{1.829299in}}%
\pgfpathlineto{\pgfqpoint{4.291677in}{1.843733in}}%
\pgfpathlineto{\pgfqpoint{4.299505in}{1.858128in}}%
\pgfpathlineto{\pgfqpoint{4.307330in}{1.872480in}}%
\pgfpathlineto{\pgfqpoint{4.293578in}{1.865477in}}%
\pgfpathlineto{\pgfqpoint{4.279838in}{1.858633in}}%
\pgfpathlineto{\pgfqpoint{4.266112in}{1.851948in}}%
\pgfpathlineto{\pgfqpoint{4.252398in}{1.845422in}}%
\pgfpathlineto{\pgfqpoint{4.244574in}{1.831427in}}%
\pgfpathlineto{\pgfqpoint{4.236746in}{1.817397in}}%
\pgfpathlineto{\pgfqpoint{4.228914in}{1.803334in}}%
\pgfpathlineto{\pgfqpoint{4.221078in}{1.789242in}}%
\pgfpathclose%
\pgfusepath{fill}%
\end{pgfscope}%
\begin{pgfscope}%
\pgfpathrectangle{\pgfqpoint{1.254980in}{0.150000in}}{\pgfqpoint{5.490039in}{5.490039in}}%
\pgfusepath{clip}%
\pgfsetbuttcap%
\pgfsetroundjoin%
\definecolor{currentfill}{rgb}{0.141935,0.526453,0.555991}%
\pgfsetfillcolor{currentfill}%
\pgfsetfillopacity{0.700000}%
\pgfsetlinewidth{0.000000pt}%
\definecolor{currentstroke}{rgb}{0.000000,0.000000,0.000000}%
\pgfsetstrokecolor{currentstroke}%
\pgfsetdash{}{0pt}%
\pgfpathmoveto{\pgfqpoint{4.777523in}{2.457559in}}%
\pgfpathlineto{\pgfqpoint{4.791553in}{2.469214in}}%
\pgfpathlineto{\pgfqpoint{4.805600in}{2.481032in}}%
\pgfpathlineto{\pgfqpoint{4.819664in}{2.493012in}}%
\pgfpathlineto{\pgfqpoint{4.833746in}{2.505155in}}%
\pgfpathlineto{\pgfqpoint{4.841412in}{2.517305in}}%
\pgfpathlineto{\pgfqpoint{4.849070in}{2.529306in}}%
\pgfpathlineto{\pgfqpoint{4.856722in}{2.541158in}}%
\pgfpathlineto{\pgfqpoint{4.864367in}{2.552859in}}%
\pgfpathlineto{\pgfqpoint{4.850284in}{2.540608in}}%
\pgfpathlineto{\pgfqpoint{4.836218in}{2.528520in}}%
\pgfpathlineto{\pgfqpoint{4.822170in}{2.516595in}}%
\pgfpathlineto{\pgfqpoint{4.808140in}{2.504831in}}%
\pgfpathlineto{\pgfqpoint{4.800495in}{2.493227in}}%
\pgfpathlineto{\pgfqpoint{4.792844in}{2.481479in}}%
\pgfpathlineto{\pgfqpoint{4.785187in}{2.469590in}}%
\pgfpathlineto{\pgfqpoint{4.777523in}{2.457559in}}%
\pgfpathclose%
\pgfusepath{fill}%
\end{pgfscope}%
\begin{pgfscope}%
\pgfpathrectangle{\pgfqpoint{1.254980in}{0.150000in}}{\pgfqpoint{5.490039in}{5.490039in}}%
\pgfusepath{clip}%
\pgfsetbuttcap%
\pgfsetroundjoin%
\definecolor{currentfill}{rgb}{0.377779,0.791781,0.377939}%
\pgfsetfillcolor{currentfill}%
\pgfsetfillopacity{0.700000}%
\pgfsetlinewidth{0.000000pt}%
\definecolor{currentstroke}{rgb}{0.000000,0.000000,0.000000}%
\pgfsetstrokecolor{currentstroke}%
\pgfsetdash{}{0pt}%
\pgfpathmoveto{\pgfqpoint{5.506434in}{3.228086in}}%
\pgfpathlineto{\pgfqpoint{5.520962in}{3.243811in}}%
\pgfpathlineto{\pgfqpoint{5.535512in}{3.259701in}}%
\pgfpathlineto{\pgfqpoint{5.550083in}{3.275757in}}%
\pgfpathlineto{\pgfqpoint{5.564677in}{3.291979in}}%
\pgfpathlineto{\pgfqpoint{5.571910in}{3.296370in}}%
\pgfpathlineto{\pgfqpoint{5.579132in}{3.300600in}}%
\pgfpathlineto{\pgfqpoint{5.586343in}{3.304672in}}%
\pgfpathlineto{\pgfqpoint{5.593543in}{3.308591in}}%
\pgfpathlineto{\pgfqpoint{5.578964in}{3.292609in}}%
\pgfpathlineto{\pgfqpoint{5.564407in}{3.276793in}}%
\pgfpathlineto{\pgfqpoint{5.549872in}{3.261142in}}%
\pgfpathlineto{\pgfqpoint{5.535359in}{3.245656in}}%
\pgfpathlineto{\pgfqpoint{5.528143in}{3.241487in}}%
\pgfpathlineto{\pgfqpoint{5.520918in}{3.237170in}}%
\pgfpathlineto{\pgfqpoint{5.513681in}{3.232705in}}%
\pgfpathlineto{\pgfqpoint{5.506434in}{3.228086in}}%
\pgfpathclose%
\pgfusepath{fill}%
\end{pgfscope}%
\begin{pgfscope}%
\pgfpathrectangle{\pgfqpoint{1.254980in}{0.150000in}}{\pgfqpoint{5.490039in}{5.490039in}}%
\pgfusepath{clip}%
\pgfsetbuttcap%
\pgfsetroundjoin%
\definecolor{currentfill}{rgb}{0.137339,0.662252,0.515571}%
\pgfsetfillcolor{currentfill}%
\pgfsetfillopacity{0.700000}%
\pgfsetlinewidth{0.000000pt}%
\definecolor{currentstroke}{rgb}{0.000000,0.000000,0.000000}%
\pgfsetstrokecolor{currentstroke}%
\pgfsetdash{}{0pt}%
\pgfpathmoveto{\pgfqpoint{5.098985in}{2.824280in}}%
\pgfpathlineto{\pgfqpoint{5.113232in}{2.838126in}}%
\pgfpathlineto{\pgfqpoint{5.127497in}{2.852137in}}%
\pgfpathlineto{\pgfqpoint{5.141782in}{2.866312in}}%
\pgfpathlineto{\pgfqpoint{5.156087in}{2.880651in}}%
\pgfpathlineto{\pgfqpoint{5.163595in}{2.889669in}}%
\pgfpathlineto{\pgfqpoint{5.171094in}{2.898515in}}%
\pgfpathlineto{\pgfqpoint{5.178584in}{2.907189in}}%
\pgfpathlineto{\pgfqpoint{5.186065in}{2.915694in}}%
\pgfpathlineto{\pgfqpoint{5.171764in}{2.901402in}}%
\pgfpathlineto{\pgfqpoint{5.157483in}{2.887274in}}%
\pgfpathlineto{\pgfqpoint{5.143221in}{2.873310in}}%
\pgfpathlineto{\pgfqpoint{5.128979in}{2.859510in}}%
\pgfpathlineto{\pgfqpoint{5.121494in}{2.850946in}}%
\pgfpathlineto{\pgfqpoint{5.114000in}{2.842221in}}%
\pgfpathlineto{\pgfqpoint{5.106497in}{2.833332in}}%
\pgfpathlineto{\pgfqpoint{5.098985in}{2.824280in}}%
\pgfpathclose%
\pgfusepath{fill}%
\end{pgfscope}%
\begin{pgfscope}%
\pgfpathrectangle{\pgfqpoint{1.254980in}{0.150000in}}{\pgfqpoint{5.490039in}{5.490039in}}%
\pgfusepath{clip}%
\pgfsetbuttcap%
\pgfsetroundjoin%
\definecolor{currentfill}{rgb}{0.277018,0.050344,0.375715}%
\pgfsetfillcolor{currentfill}%
\pgfsetfillopacity{0.700000}%
\pgfsetlinewidth{0.000000pt}%
\definecolor{currentstroke}{rgb}{0.000000,0.000000,0.000000}%
\pgfsetstrokecolor{currentstroke}%
\pgfsetdash{}{0pt}%
\pgfpathmoveto{\pgfqpoint{3.727417in}{1.375604in}}%
\pgfpathlineto{\pgfqpoint{3.740967in}{1.375259in}}%
\pgfpathlineto{\pgfqpoint{3.754524in}{1.375074in}}%
\pgfpathlineto{\pgfqpoint{3.768088in}{1.375048in}}%
\pgfpathlineto{\pgfqpoint{3.781661in}{1.375180in}}%
\pgfpathlineto{\pgfqpoint{3.789646in}{1.386151in}}%
\pgfpathlineto{\pgfqpoint{3.797626in}{1.397243in}}%
\pgfpathlineto{\pgfqpoint{3.805600in}{1.408453in}}%
\pgfpathlineto{\pgfqpoint{3.813569in}{1.419773in}}%
\pgfpathlineto{\pgfqpoint{3.800007in}{1.419084in}}%
\pgfpathlineto{\pgfqpoint{3.786453in}{1.418553in}}%
\pgfpathlineto{\pgfqpoint{3.772907in}{1.418182in}}%
\pgfpathlineto{\pgfqpoint{3.759369in}{1.417970in}}%
\pgfpathlineto{\pgfqpoint{3.751390in}{1.407195in}}%
\pgfpathlineto{\pgfqpoint{3.743406in}{1.396539in}}%
\pgfpathlineto{\pgfqpoint{3.735415in}{1.386007in}}%
\pgfpathlineto{\pgfqpoint{3.727417in}{1.375604in}}%
\pgfpathclose%
\pgfusepath{fill}%
\end{pgfscope}%
\begin{pgfscope}%
\pgfpathrectangle{\pgfqpoint{1.254980in}{0.150000in}}{\pgfqpoint{5.490039in}{5.490039in}}%
\pgfusepath{clip}%
\pgfsetbuttcap%
\pgfsetroundjoin%
\definecolor{currentfill}{rgb}{0.162142,0.474838,0.558140}%
\pgfsetfillcolor{currentfill}%
\pgfsetfillopacity{0.700000}%
\pgfsetlinewidth{0.000000pt}%
\definecolor{currentstroke}{rgb}{0.000000,0.000000,0.000000}%
\pgfsetstrokecolor{currentstroke}%
\pgfsetdash{}{0pt}%
\pgfpathmoveto{\pgfqpoint{4.660052in}{2.312552in}}%
\pgfpathlineto{\pgfqpoint{4.674011in}{2.323254in}}%
\pgfpathlineto{\pgfqpoint{4.687986in}{2.334117in}}%
\pgfpathlineto{\pgfqpoint{4.701978in}{2.345142in}}%
\pgfpathlineto{\pgfqpoint{4.715986in}{2.356329in}}%
\pgfpathlineto{\pgfqpoint{4.723699in}{2.369457in}}%
\pgfpathlineto{\pgfqpoint{4.731407in}{2.382452in}}%
\pgfpathlineto{\pgfqpoint{4.739108in}{2.395312in}}%
\pgfpathlineto{\pgfqpoint{4.746804in}{2.408037in}}%
\pgfpathlineto{\pgfqpoint{4.732793in}{2.396682in}}%
\pgfpathlineto{\pgfqpoint{4.718799in}{2.385489in}}%
\pgfpathlineto{\pgfqpoint{4.704821in}{2.374457in}}%
\pgfpathlineto{\pgfqpoint{4.690860in}{2.363587in}}%
\pgfpathlineto{\pgfqpoint{4.683167in}{2.351019in}}%
\pgfpathlineto{\pgfqpoint{4.675468in}{2.338323in}}%
\pgfpathlineto{\pgfqpoint{4.667762in}{2.325500in}}%
\pgfpathlineto{\pgfqpoint{4.660052in}{2.312552in}}%
\pgfpathclose%
\pgfusepath{fill}%
\end{pgfscope}%
\begin{pgfscope}%
\pgfpathrectangle{\pgfqpoint{1.254980in}{0.150000in}}{\pgfqpoint{5.490039in}{5.490039in}}%
\pgfusepath{clip}%
\pgfsetbuttcap%
\pgfsetroundjoin%
\definecolor{currentfill}{rgb}{0.268510,0.009605,0.335427}%
\pgfsetfillcolor{currentfill}%
\pgfsetfillopacity{0.700000}%
\pgfsetlinewidth{0.000000pt}%
\definecolor{currentstroke}{rgb}{0.000000,0.000000,0.000000}%
\pgfsetstrokecolor{currentstroke}%
\pgfsetdash{}{0pt}%
\pgfpathmoveto{\pgfqpoint{3.414124in}{1.319670in}}%
\pgfpathlineto{\pgfqpoint{3.427639in}{1.314900in}}%
\pgfpathlineto{\pgfqpoint{3.441158in}{1.310295in}}%
\pgfpathlineto{\pgfqpoint{3.454681in}{1.305855in}}%
\pgfpathlineto{\pgfqpoint{3.468208in}{1.301579in}}%
\pgfpathlineto{\pgfqpoint{3.476352in}{1.308032in}}%
\pgfpathlineto{\pgfqpoint{3.484486in}{1.314708in}}%
\pgfpathlineto{\pgfqpoint{3.492611in}{1.321598in}}%
\pgfpathlineto{\pgfqpoint{3.500727in}{1.328697in}}%
\pgfpathlineto{\pgfqpoint{3.487222in}{1.332334in}}%
\pgfpathlineto{\pgfqpoint{3.473722in}{1.336135in}}%
\pgfpathlineto{\pgfqpoint{3.460226in}{1.340100in}}%
\pgfpathlineto{\pgfqpoint{3.446734in}{1.344231in}}%
\pgfpathlineto{\pgfqpoint{3.438596in}{1.337760in}}%
\pgfpathlineto{\pgfqpoint{3.430449in}{1.331506in}}%
\pgfpathlineto{\pgfqpoint{3.422291in}{1.325474in}}%
\pgfpathlineto{\pgfqpoint{3.414124in}{1.319670in}}%
\pgfpathclose%
\pgfusepath{fill}%
\end{pgfscope}%
\begin{pgfscope}%
\pgfpathrectangle{\pgfqpoint{1.254980in}{0.150000in}}{\pgfqpoint{5.490039in}{5.490039in}}%
\pgfusepath{clip}%
\pgfsetbuttcap%
\pgfsetroundjoin%
\definecolor{currentfill}{rgb}{0.280894,0.078907,0.402329}%
\pgfsetfillcolor{currentfill}%
\pgfsetfillopacity{0.700000}%
\pgfsetlinewidth{0.000000pt}%
\definecolor{currentstroke}{rgb}{0.000000,0.000000,0.000000}%
\pgfsetstrokecolor{currentstroke}%
\pgfsetdash{}{0pt}%
\pgfpathmoveto{\pgfqpoint{3.813569in}{1.419773in}}%
\pgfpathlineto{\pgfqpoint{3.827139in}{1.420622in}}%
\pgfpathlineto{\pgfqpoint{3.840717in}{1.421629in}}%
\pgfpathlineto{\pgfqpoint{3.854304in}{1.422794in}}%
\pgfpathlineto{\pgfqpoint{3.867899in}{1.424117in}}%
\pgfpathlineto{\pgfqpoint{3.875853in}{1.436084in}}%
\pgfpathlineto{\pgfqpoint{3.883801in}{1.448145in}}%
\pgfpathlineto{\pgfqpoint{3.891745in}{1.460296in}}%
\pgfpathlineto{\pgfqpoint{3.899683in}{1.472530in}}%
\pgfpathlineto{\pgfqpoint{3.886096in}{1.470676in}}%
\pgfpathlineto{\pgfqpoint{3.872517in}{1.468980in}}%
\pgfpathlineto{\pgfqpoint{3.858947in}{1.467444in}}%
\pgfpathlineto{\pgfqpoint{3.845386in}{1.466066in}}%
\pgfpathlineto{\pgfqpoint{3.837440in}{1.454351in}}%
\pgfpathlineto{\pgfqpoint{3.829488in}{1.442728in}}%
\pgfpathlineto{\pgfqpoint{3.821531in}{1.431200in}}%
\pgfpathlineto{\pgfqpoint{3.813569in}{1.419773in}}%
\pgfpathclose%
\pgfusepath{fill}%
\end{pgfscope}%
\begin{pgfscope}%
\pgfpathrectangle{\pgfqpoint{1.254980in}{0.150000in}}{\pgfqpoint{5.490039in}{5.490039in}}%
\pgfusepath{clip}%
\pgfsetbuttcap%
\pgfsetroundjoin%
\definecolor{currentfill}{rgb}{0.274128,0.199721,0.498911}%
\pgfsetfillcolor{currentfill}%
\pgfsetfillopacity{0.700000}%
\pgfsetlinewidth{0.000000pt}%
\definecolor{currentstroke}{rgb}{0.000000,0.000000,0.000000}%
\pgfsetstrokecolor{currentstroke}%
\pgfsetdash{}{0pt}%
\pgfpathmoveto{\pgfqpoint{4.103503in}{1.656101in}}%
\pgfpathlineto{\pgfqpoint{4.117170in}{1.660807in}}%
\pgfpathlineto{\pgfqpoint{4.130848in}{1.665672in}}%
\pgfpathlineto{\pgfqpoint{4.144538in}{1.670695in}}%
\pgfpathlineto{\pgfqpoint{4.158239in}{1.675875in}}%
\pgfpathlineto{\pgfqpoint{4.166108in}{1.690072in}}%
\pgfpathlineto{\pgfqpoint{4.173974in}{1.704269in}}%
\pgfpathlineto{\pgfqpoint{4.181835in}{1.718463in}}%
\pgfpathlineto{\pgfqpoint{4.189692in}{1.732649in}}%
\pgfpathlineto{\pgfqpoint{4.175991in}{1.727044in}}%
\pgfpathlineto{\pgfqpoint{4.162302in}{1.721598in}}%
\pgfpathlineto{\pgfqpoint{4.148625in}{1.716311in}}%
\pgfpathlineto{\pgfqpoint{4.134959in}{1.711182in}}%
\pgfpathlineto{\pgfqpoint{4.127102in}{1.697408in}}%
\pgfpathlineto{\pgfqpoint{4.119240in}{1.683634in}}%
\pgfpathlineto{\pgfqpoint{4.111374in}{1.669864in}}%
\pgfpathlineto{\pgfqpoint{4.103503in}{1.656101in}}%
\pgfpathclose%
\pgfusepath{fill}%
\end{pgfscope}%
\begin{pgfscope}%
\pgfpathrectangle{\pgfqpoint{1.254980in}{0.150000in}}{\pgfqpoint{5.490039in}{5.490039in}}%
\pgfusepath{clip}%
\pgfsetbuttcap%
\pgfsetroundjoin%
\definecolor{currentfill}{rgb}{0.271305,0.019942,0.347269}%
\pgfsetfillcolor{currentfill}%
\pgfsetfillopacity{0.700000}%
\pgfsetlinewidth{0.000000pt}%
\definecolor{currentstroke}{rgb}{0.000000,0.000000,0.000000}%
\pgfsetstrokecolor{currentstroke}%
\pgfsetdash{}{0pt}%
\pgfpathmoveto{\pgfqpoint{3.273126in}{1.348386in}}%
\pgfpathlineto{\pgfqpoint{3.286645in}{1.341608in}}%
\pgfpathlineto{\pgfqpoint{3.300167in}{1.335000in}}%
\pgfpathlineto{\pgfqpoint{3.313690in}{1.328560in}}%
\pgfpathlineto{\pgfqpoint{3.327217in}{1.322289in}}%
\pgfpathlineto{\pgfqpoint{3.335454in}{1.326432in}}%
\pgfpathlineto{\pgfqpoint{3.343680in}{1.330837in}}%
\pgfpathlineto{\pgfqpoint{3.351895in}{1.335499in}}%
\pgfpathlineto{\pgfqpoint{3.360099in}{1.340409in}}%
\pgfpathlineto{\pgfqpoint{3.346601in}{1.346011in}}%
\pgfpathlineto{\pgfqpoint{3.333106in}{1.351782in}}%
\pgfpathlineto{\pgfqpoint{3.319613in}{1.357721in}}%
\pgfpathlineto{\pgfqpoint{3.306124in}{1.363829in}}%
\pgfpathlineto{\pgfqpoint{3.297892in}{1.359577in}}%
\pgfpathlineto{\pgfqpoint{3.289649in}{1.355581in}}%
\pgfpathlineto{\pgfqpoint{3.281394in}{1.351849in}}%
\pgfpathlineto{\pgfqpoint{3.273126in}{1.348386in}}%
\pgfpathclose%
\pgfusepath{fill}%
\end{pgfscope}%
\begin{pgfscope}%
\pgfpathrectangle{\pgfqpoint{1.254980in}{0.150000in}}{\pgfqpoint{5.490039in}{5.490039in}}%
\pgfusepath{clip}%
\pgfsetbuttcap%
\pgfsetroundjoin%
\definecolor{currentfill}{rgb}{0.279566,0.067836,0.391917}%
\pgfsetfillcolor{currentfill}%
\pgfsetfillopacity{0.700000}%
\pgfsetlinewidth{0.000000pt}%
\definecolor{currentstroke}{rgb}{0.000000,0.000000,0.000000}%
\pgfsetstrokecolor{currentstroke}%
\pgfsetdash{}{0pt}%
\pgfpathmoveto{\pgfqpoint{3.077396in}{1.440579in}}%
\pgfpathlineto{\pgfqpoint{3.090939in}{1.431014in}}%
\pgfpathlineto{\pgfqpoint{3.104481in}{1.421626in}}%
\pgfpathlineto{\pgfqpoint{3.118023in}{1.412416in}}%
\pgfpathlineto{\pgfqpoint{3.131565in}{1.403382in}}%
\pgfpathlineto{\pgfqpoint{3.139954in}{1.404269in}}%
\pgfpathlineto{\pgfqpoint{3.148329in}{1.405470in}}%
\pgfpathlineto{\pgfqpoint{3.156689in}{1.406977in}}%
\pgfpathlineto{\pgfqpoint{3.165035in}{1.408783in}}%
\pgfpathlineto{\pgfqpoint{3.151529in}{1.417115in}}%
\pgfpathlineto{\pgfqpoint{3.138024in}{1.425622in}}%
\pgfpathlineto{\pgfqpoint{3.124519in}{1.434306in}}%
\pgfpathlineto{\pgfqpoint{3.111015in}{1.443168in}}%
\pgfpathlineto{\pgfqpoint{3.102633in}{1.442053in}}%
\pgfpathlineto{\pgfqpoint{3.094236in}{1.441245in}}%
\pgfpathlineto{\pgfqpoint{3.085824in}{1.440752in}}%
\pgfpathlineto{\pgfqpoint{3.077396in}{1.440579in}}%
\pgfpathclose%
\pgfusepath{fill}%
\end{pgfscope}%
\begin{pgfscope}%
\pgfpathrectangle{\pgfqpoint{1.254980in}{0.150000in}}{\pgfqpoint{5.490039in}{5.490039in}}%
\pgfusepath{clip}%
\pgfsetbuttcap%
\pgfsetroundjoin%
\definecolor{currentfill}{rgb}{0.272594,0.025563,0.353093}%
\pgfsetfillcolor{currentfill}%
\pgfsetfillopacity{0.700000}%
\pgfsetlinewidth{0.000000pt}%
\definecolor{currentstroke}{rgb}{0.000000,0.000000,0.000000}%
\pgfsetstrokecolor{currentstroke}%
\pgfsetdash{}{0pt}%
\pgfpathmoveto{\pgfqpoint{3.641178in}{1.340703in}}%
\pgfpathlineto{\pgfqpoint{3.654714in}{1.339135in}}%
\pgfpathlineto{\pgfqpoint{3.668257in}{1.337727in}}%
\pgfpathlineto{\pgfqpoint{3.681806in}{1.336480in}}%
\pgfpathlineto{\pgfqpoint{3.695362in}{1.335392in}}%
\pgfpathlineto{\pgfqpoint{3.703386in}{1.345223in}}%
\pgfpathlineto{\pgfqpoint{3.711403in}{1.355206in}}%
\pgfpathlineto{\pgfqpoint{3.719413in}{1.365335in}}%
\pgfpathlineto{\pgfqpoint{3.727417in}{1.375604in}}%
\pgfpathlineto{\pgfqpoint{3.713875in}{1.376108in}}%
\pgfpathlineto{\pgfqpoint{3.700339in}{1.376771in}}%
\pgfpathlineto{\pgfqpoint{3.686811in}{1.377595in}}%
\pgfpathlineto{\pgfqpoint{3.673290in}{1.378580in}}%
\pgfpathlineto{\pgfqpoint{3.665272in}{1.368884in}}%
\pgfpathlineto{\pgfqpoint{3.657248in}{1.359336in}}%
\pgfpathlineto{\pgfqpoint{3.649217in}{1.349940in}}%
\pgfpathlineto{\pgfqpoint{3.641178in}{1.340703in}}%
\pgfpathclose%
\pgfusepath{fill}%
\end{pgfscope}%
\begin{pgfscope}%
\pgfpathrectangle{\pgfqpoint{1.254980in}{0.150000in}}{\pgfqpoint{5.490039in}{5.490039in}}%
\pgfusepath{clip}%
\pgfsetbuttcap%
\pgfsetroundjoin%
\definecolor{currentfill}{rgb}{0.239374,0.735588,0.455688}%
\pgfsetfillcolor{currentfill}%
\pgfsetfillopacity{0.700000}%
\pgfsetlinewidth{0.000000pt}%
\definecolor{currentstroke}{rgb}{0.000000,0.000000,0.000000}%
\pgfsetstrokecolor{currentstroke}%
\pgfsetdash{}{0pt}%
\pgfpathmoveto{\pgfqpoint{5.302938in}{3.035812in}}%
\pgfpathlineto{\pgfqpoint{5.317328in}{3.050739in}}%
\pgfpathlineto{\pgfqpoint{5.331738in}{3.065831in}}%
\pgfpathlineto{\pgfqpoint{5.346170in}{3.081088in}}%
\pgfpathlineto{\pgfqpoint{5.360622in}{3.096511in}}%
\pgfpathlineto{\pgfqpoint{5.368004in}{3.103281in}}%
\pgfpathlineto{\pgfqpoint{5.375376in}{3.109878in}}%
\pgfpathlineto{\pgfqpoint{5.382738in}{3.116305in}}%
\pgfpathlineto{\pgfqpoint{5.390090in}{3.122565in}}%
\pgfpathlineto{\pgfqpoint{5.375646in}{3.107285in}}%
\pgfpathlineto{\pgfqpoint{5.361224in}{3.092170in}}%
\pgfpathlineto{\pgfqpoint{5.346822in}{3.077221in}}%
\pgfpathlineto{\pgfqpoint{5.332441in}{3.062435in}}%
\pgfpathlineto{\pgfqpoint{5.325080in}{3.056022in}}%
\pgfpathlineto{\pgfqpoint{5.317709in}{3.049448in}}%
\pgfpathlineto{\pgfqpoint{5.310328in}{3.042712in}}%
\pgfpathlineto{\pgfqpoint{5.302938in}{3.035812in}}%
\pgfpathclose%
\pgfusepath{fill}%
\end{pgfscope}%
\begin{pgfscope}%
\pgfpathrectangle{\pgfqpoint{1.254980in}{0.150000in}}{\pgfqpoint{5.490039in}{5.490039in}}%
\pgfusepath{clip}%
\pgfsetbuttcap%
\pgfsetroundjoin%
\definecolor{currentfill}{rgb}{0.141935,0.526453,0.555991}%
\pgfsetfillcolor{currentfill}%
\pgfsetfillopacity{0.700000}%
\pgfsetlinewidth{0.000000pt}%
\definecolor{currentstroke}{rgb}{0.000000,0.000000,0.000000}%
\pgfsetstrokecolor{currentstroke}%
\pgfsetdash{}{0pt}%
\pgfpathmoveto{\pgfqpoint{2.146080in}{2.569437in}}%
\pgfpathlineto{\pgfqpoint{2.160090in}{2.544845in}}%
\pgfpathlineto{\pgfqpoint{2.174085in}{2.520537in}}%
\pgfpathlineto{\pgfqpoint{2.188065in}{2.496509in}}%
\pgfpathlineto{\pgfqpoint{2.202031in}{2.472759in}}%
\pgfpathlineto{\pgfqpoint{2.211354in}{2.461023in}}%
\pgfpathlineto{\pgfqpoint{2.220647in}{2.449748in}}%
\pgfpathlineto{\pgfqpoint{2.229910in}{2.438925in}}%
\pgfpathlineto{\pgfqpoint{2.239143in}{2.428546in}}%
\pgfpathlineto{\pgfqpoint{2.225251in}{2.451521in}}%
\pgfpathlineto{\pgfqpoint{2.211344in}{2.474772in}}%
\pgfpathlineto{\pgfqpoint{2.197424in}{2.498301in}}%
\pgfpathlineto{\pgfqpoint{2.183489in}{2.522112in}}%
\pgfpathlineto{\pgfqpoint{2.174183in}{2.533254in}}%
\pgfpathlineto{\pgfqpoint{2.164846in}{2.544850in}}%
\pgfpathlineto{\pgfqpoint{2.155479in}{2.556908in}}%
\pgfpathlineto{\pgfqpoint{2.146080in}{2.569437in}}%
\pgfpathclose%
\pgfusepath{fill}%
\end{pgfscope}%
\begin{pgfscope}%
\pgfpathrectangle{\pgfqpoint{1.254980in}{0.150000in}}{\pgfqpoint{5.490039in}{5.490039in}}%
\pgfusepath{clip}%
\pgfsetbuttcap%
\pgfsetroundjoin%
\definecolor{currentfill}{rgb}{0.183898,0.422383,0.556944}%
\pgfsetfillcolor{currentfill}%
\pgfsetfillopacity{0.700000}%
\pgfsetlinewidth{0.000000pt}%
\definecolor{currentstroke}{rgb}{0.000000,0.000000,0.000000}%
\pgfsetstrokecolor{currentstroke}%
\pgfsetdash{}{0pt}%
\pgfpathmoveto{\pgfqpoint{4.542511in}{2.165132in}}%
\pgfpathlineto{\pgfqpoint{4.556401in}{2.174766in}}%
\pgfpathlineto{\pgfqpoint{4.570307in}{2.184560in}}%
\pgfpathlineto{\pgfqpoint{4.584228in}{2.194515in}}%
\pgfpathlineto{\pgfqpoint{4.598165in}{2.204631in}}%
\pgfpathlineto{\pgfqpoint{4.605920in}{2.218526in}}%
\pgfpathlineto{\pgfqpoint{4.613669in}{2.232310in}}%
\pgfpathlineto{\pgfqpoint{4.621413in}{2.245979in}}%
\pgfpathlineto{\pgfqpoint{4.629152in}{2.259532in}}%
\pgfpathlineto{\pgfqpoint{4.615212in}{2.249189in}}%
\pgfpathlineto{\pgfqpoint{4.601287in}{2.239007in}}%
\pgfpathlineto{\pgfqpoint{4.587379in}{2.228985in}}%
\pgfpathlineto{\pgfqpoint{4.573486in}{2.219125in}}%
\pgfpathlineto{\pgfqpoint{4.565750in}{2.205788in}}%
\pgfpathlineto{\pgfqpoint{4.558009in}{2.192342in}}%
\pgfpathlineto{\pgfqpoint{4.550262in}{2.178789in}}%
\pgfpathlineto{\pgfqpoint{4.542511in}{2.165132in}}%
\pgfpathclose%
\pgfusepath{fill}%
\end{pgfscope}%
\begin{pgfscope}%
\pgfpathrectangle{\pgfqpoint{1.254980in}{0.150000in}}{\pgfqpoint{5.490039in}{5.490039in}}%
\pgfusepath{clip}%
\pgfsetbuttcap%
\pgfsetroundjoin%
\definecolor{currentfill}{rgb}{0.134692,0.658636,0.517649}%
\pgfsetfillcolor{currentfill}%
\pgfsetfillopacity{0.700000}%
\pgfsetlinewidth{0.000000pt}%
\definecolor{currentstroke}{rgb}{0.000000,0.000000,0.000000}%
\pgfsetstrokecolor{currentstroke}%
\pgfsetdash{}{0pt}%
\pgfpathmoveto{\pgfqpoint{1.958300in}{2.943647in}}%
\pgfpathlineto{\pgfqpoint{1.972512in}{2.914942in}}%
\pgfpathlineto{\pgfqpoint{1.986703in}{2.886566in}}%
\pgfpathlineto{\pgfqpoint{2.000874in}{2.858517in}}%
\pgfpathlineto{\pgfqpoint{2.015027in}{2.830791in}}%
\pgfpathlineto{\pgfqpoint{2.024533in}{2.817784in}}%
\pgfpathlineto{\pgfqpoint{2.034006in}{2.805245in}}%
\pgfpathlineto{\pgfqpoint{2.043447in}{2.793165in}}%
\pgfpathlineto{\pgfqpoint{2.052856in}{2.781537in}}%
\pgfpathlineto{\pgfqpoint{2.038783in}{2.808493in}}%
\pgfpathlineto{\pgfqpoint{2.024691in}{2.835770in}}%
\pgfpathlineto{\pgfqpoint{2.010580in}{2.863371in}}%
\pgfpathlineto{\pgfqpoint{1.996451in}{2.891299in}}%
\pgfpathlineto{\pgfqpoint{1.986963in}{2.903686in}}%
\pgfpathlineto{\pgfqpoint{1.977442in}{2.916534in}}%
\pgfpathlineto{\pgfqpoint{1.967888in}{2.929852in}}%
\pgfpathlineto{\pgfqpoint{1.958300in}{2.943647in}}%
\pgfpathclose%
\pgfusepath{fill}%
\end{pgfscope}%
\begin{pgfscope}%
\pgfpathrectangle{\pgfqpoint{1.254980in}{0.150000in}}{\pgfqpoint{5.490039in}{5.490039in}}%
\pgfusepath{clip}%
\pgfsetbuttcap%
\pgfsetroundjoin%
\definecolor{currentfill}{rgb}{0.283091,0.110553,0.431554}%
\pgfsetfillcolor{currentfill}%
\pgfsetfillopacity{0.700000}%
\pgfsetlinewidth{0.000000pt}%
\definecolor{currentstroke}{rgb}{0.000000,0.000000,0.000000}%
\pgfsetstrokecolor{currentstroke}%
\pgfsetdash{}{0pt}%
\pgfpathmoveto{\pgfqpoint{3.899683in}{1.472530in}}%
\pgfpathlineto{\pgfqpoint{3.913280in}{1.474541in}}%
\pgfpathlineto{\pgfqpoint{3.926886in}{1.476711in}}%
\pgfpathlineto{\pgfqpoint{3.940501in}{1.479039in}}%
\pgfpathlineto{\pgfqpoint{3.954126in}{1.481524in}}%
\pgfpathlineto{\pgfqpoint{3.962053in}{1.494351in}}%
\pgfpathlineto{\pgfqpoint{3.969976in}{1.507246in}}%
\pgfpathlineto{\pgfqpoint{3.977894in}{1.520202in}}%
\pgfpathlineto{\pgfqpoint{3.985807in}{1.533216in}}%
\pgfpathlineto{\pgfqpoint{3.972187in}{1.530227in}}%
\pgfpathlineto{\pgfqpoint{3.958577in}{1.527395in}}%
\pgfpathlineto{\pgfqpoint{3.944977in}{1.524721in}}%
\pgfpathlineto{\pgfqpoint{3.931386in}{1.522206in}}%
\pgfpathlineto{\pgfqpoint{3.923468in}{1.509686in}}%
\pgfpathlineto{\pgfqpoint{3.915545in}{1.497229in}}%
\pgfpathlineto{\pgfqpoint{3.907616in}{1.484842in}}%
\pgfpathlineto{\pgfqpoint{3.899683in}{1.472530in}}%
\pgfpathclose%
\pgfusepath{fill}%
\end{pgfscope}%
\begin{pgfscope}%
\pgfpathrectangle{\pgfqpoint{1.254980in}{0.150000in}}{\pgfqpoint{5.490039in}{5.490039in}}%
\pgfusepath{clip}%
\pgfsetbuttcap%
\pgfsetroundjoin%
\definecolor{currentfill}{rgb}{0.210503,0.363727,0.552206}%
\pgfsetfillcolor{currentfill}%
\pgfsetfillopacity{0.700000}%
\pgfsetlinewidth{0.000000pt}%
\definecolor{currentstroke}{rgb}{0.000000,0.000000,0.000000}%
\pgfsetstrokecolor{currentstroke}%
\pgfsetdash{}{0pt}%
\pgfpathmoveto{\pgfqpoint{4.424933in}{2.017587in}}%
\pgfpathlineto{\pgfqpoint{4.438758in}{2.026040in}}%
\pgfpathlineto{\pgfqpoint{4.452598in}{2.034652in}}%
\pgfpathlineto{\pgfqpoint{4.466452in}{2.043425in}}%
\pgfpathlineto{\pgfqpoint{4.480321in}{2.052357in}}%
\pgfpathlineto{\pgfqpoint{4.488112in}{2.066774in}}%
\pgfpathlineto{\pgfqpoint{4.495897in}{2.081105in}}%
\pgfpathlineto{\pgfqpoint{4.503678in}{2.095346in}}%
\pgfpathlineto{\pgfqpoint{4.511455in}{2.109496in}}%
\pgfpathlineto{\pgfqpoint{4.497582in}{2.100278in}}%
\pgfpathlineto{\pgfqpoint{4.483725in}{2.091221in}}%
\pgfpathlineto{\pgfqpoint{4.469883in}{2.082324in}}%
\pgfpathlineto{\pgfqpoint{4.456055in}{2.073587in}}%
\pgfpathlineto{\pgfqpoint{4.448281in}{2.059711in}}%
\pgfpathlineto{\pgfqpoint{4.440503in}{2.045750in}}%
\pgfpathlineto{\pgfqpoint{4.432720in}{2.031708in}}%
\pgfpathlineto{\pgfqpoint{4.424933in}{2.017587in}}%
\pgfpathclose%
\pgfusepath{fill}%
\end{pgfscope}%
\begin{pgfscope}%
\pgfpathrectangle{\pgfqpoint{1.254980in}{0.150000in}}{\pgfqpoint{5.490039in}{5.490039in}}%
\pgfusepath{clip}%
\pgfsetbuttcap%
\pgfsetroundjoin%
\definecolor{currentfill}{rgb}{0.269944,0.014625,0.341379}%
\pgfsetfillcolor{currentfill}%
\pgfsetfillopacity{0.700000}%
\pgfsetlinewidth{0.000000pt}%
\definecolor{currentstroke}{rgb}{0.000000,0.000000,0.000000}%
\pgfsetstrokecolor{currentstroke}%
\pgfsetdash{}{0pt}%
\pgfpathmoveto{\pgfqpoint{3.554796in}{1.315780in}}%
\pgfpathlineto{\pgfqpoint{3.568326in}{1.312957in}}%
\pgfpathlineto{\pgfqpoint{3.581861in}{1.310295in}}%
\pgfpathlineto{\pgfqpoint{3.595402in}{1.307795in}}%
\pgfpathlineto{\pgfqpoint{3.608949in}{1.305455in}}%
\pgfpathlineto{\pgfqpoint{3.617018in}{1.314000in}}%
\pgfpathlineto{\pgfqpoint{3.625079in}{1.322726in}}%
\pgfpathlineto{\pgfqpoint{3.633132in}{1.331629in}}%
\pgfpathlineto{\pgfqpoint{3.641178in}{1.340703in}}%
\pgfpathlineto{\pgfqpoint{3.627648in}{1.342431in}}%
\pgfpathlineto{\pgfqpoint{3.614124in}{1.344320in}}%
\pgfpathlineto{\pgfqpoint{3.600607in}{1.346371in}}%
\pgfpathlineto{\pgfqpoint{3.587095in}{1.348583in}}%
\pgfpathlineto{\pgfqpoint{3.579032in}{1.340110in}}%
\pgfpathlineto{\pgfqpoint{3.570961in}{1.331815in}}%
\pgfpathlineto{\pgfqpoint{3.562883in}{1.323703in}}%
\pgfpathlineto{\pgfqpoint{3.554796in}{1.315780in}}%
\pgfpathclose%
\pgfusepath{fill}%
\end{pgfscope}%
\begin{pgfscope}%
\pgfpathrectangle{\pgfqpoint{1.254980in}{0.150000in}}{\pgfqpoint{5.490039in}{5.490039in}}%
\pgfusepath{clip}%
\pgfsetbuttcap%
\pgfsetroundjoin%
\definecolor{currentfill}{rgb}{0.119483,0.614817,0.537692}%
\pgfsetfillcolor{currentfill}%
\pgfsetfillopacity{0.700000}%
\pgfsetlinewidth{0.000000pt}%
\definecolor{currentstroke}{rgb}{0.000000,0.000000,0.000000}%
\pgfsetstrokecolor{currentstroke}%
\pgfsetdash{}{0pt}%
\pgfpathmoveto{\pgfqpoint{4.981784in}{2.692131in}}%
\pgfpathlineto{\pgfqpoint{4.995958in}{2.705321in}}%
\pgfpathlineto{\pgfqpoint{5.010151in}{2.718676in}}%
\pgfpathlineto{\pgfqpoint{5.024363in}{2.732194in}}%
\pgfpathlineto{\pgfqpoint{5.038594in}{2.745876in}}%
\pgfpathlineto{\pgfqpoint{5.046172in}{2.756262in}}%
\pgfpathlineto{\pgfqpoint{5.053741in}{2.766481in}}%
\pgfpathlineto{\pgfqpoint{5.061303in}{2.776531in}}%
\pgfpathlineto{\pgfqpoint{5.068856in}{2.786414in}}%
\pgfpathlineto{\pgfqpoint{5.054626in}{2.772716in}}%
\pgfpathlineto{\pgfqpoint{5.040416in}{2.759182in}}%
\pgfpathlineto{\pgfqpoint{5.026225in}{2.745812in}}%
\pgfpathlineto{\pgfqpoint{5.012052in}{2.732605in}}%
\pgfpathlineto{\pgfqpoint{5.004497in}{2.722726in}}%
\pgfpathlineto{\pgfqpoint{4.996934in}{2.712688in}}%
\pgfpathlineto{\pgfqpoint{4.989362in}{2.702489in}}%
\pgfpathlineto{\pgfqpoint{4.981784in}{2.692131in}}%
\pgfpathclose%
\pgfusepath{fill}%
\end{pgfscope}%
\begin{pgfscope}%
\pgfpathrectangle{\pgfqpoint{1.254980in}{0.150000in}}{\pgfqpoint{5.490039in}{5.490039in}}%
\pgfusepath{clip}%
\pgfsetbuttcap%
\pgfsetroundjoin%
\definecolor{currentfill}{rgb}{0.237441,0.305202,0.541921}%
\pgfsetfillcolor{currentfill}%
\pgfsetfillopacity{0.700000}%
\pgfsetlinewidth{0.000000pt}%
\definecolor{currentstroke}{rgb}{0.000000,0.000000,0.000000}%
\pgfsetstrokecolor{currentstroke}%
\pgfsetdash{}{0pt}%
\pgfpathmoveto{\pgfqpoint{4.307330in}{1.872480in}}%
\pgfpathlineto{\pgfqpoint{4.321096in}{1.879642in}}%
\pgfpathlineto{\pgfqpoint{4.334875in}{1.886963in}}%
\pgfpathlineto{\pgfqpoint{4.348667in}{1.894443in}}%
\pgfpathlineto{\pgfqpoint{4.362474in}{1.902081in}}%
\pgfpathlineto{\pgfqpoint{4.370296in}{1.916739in}}%
\pgfpathlineto{\pgfqpoint{4.378114in}{1.931340in}}%
\pgfpathlineto{\pgfqpoint{4.385928in}{1.945881in}}%
\pgfpathlineto{\pgfqpoint{4.393738in}{1.960359in}}%
\pgfpathlineto{\pgfqpoint{4.379929in}{1.952379in}}%
\pgfpathlineto{\pgfqpoint{4.366135in}{1.944558in}}%
\pgfpathlineto{\pgfqpoint{4.352354in}{1.936896in}}%
\pgfpathlineto{\pgfqpoint{4.338586in}{1.929393in}}%
\pgfpathlineto{\pgfqpoint{4.330779in}{1.915245in}}%
\pgfpathlineto{\pgfqpoint{4.322967in}{1.901042in}}%
\pgfpathlineto{\pgfqpoint{4.315150in}{1.886786in}}%
\pgfpathlineto{\pgfqpoint{4.307330in}{1.872480in}}%
\pgfpathclose%
\pgfusepath{fill}%
\end{pgfscope}%
\begin{pgfscope}%
\pgfpathrectangle{\pgfqpoint{1.254980in}{0.150000in}}{\pgfqpoint{5.490039in}{5.490039in}}%
\pgfusepath{clip}%
\pgfsetbuttcap%
\pgfsetroundjoin%
\definecolor{currentfill}{rgb}{0.458674,0.816363,0.329727}%
\pgfsetfillcolor{currentfill}%
\pgfsetfillopacity{0.700000}%
\pgfsetlinewidth{0.000000pt}%
\definecolor{currentstroke}{rgb}{0.000000,0.000000,0.000000}%
\pgfsetstrokecolor{currentstroke}%
\pgfsetdash{}{0pt}%
\pgfpathmoveto{\pgfqpoint{5.593543in}{3.308591in}}%
\pgfpathlineto{\pgfqpoint{5.608145in}{3.324738in}}%
\pgfpathlineto{\pgfqpoint{5.622769in}{3.341051in}}%
\pgfpathlineto{\pgfqpoint{5.637415in}{3.357531in}}%
\pgfpathlineto{\pgfqpoint{5.652084in}{3.374178in}}%
\pgfpathlineto{\pgfqpoint{5.659257in}{3.377683in}}%
\pgfpathlineto{\pgfqpoint{5.666419in}{3.381031in}}%
\pgfpathlineto{\pgfqpoint{5.673569in}{3.384227in}}%
\pgfpathlineto{\pgfqpoint{5.680709in}{3.387272in}}%
\pgfpathlineto{\pgfqpoint{5.666057in}{3.370900in}}%
\pgfpathlineto{\pgfqpoint{5.651428in}{3.354693in}}%
\pgfpathlineto{\pgfqpoint{5.636821in}{3.338652in}}%
\pgfpathlineto{\pgfqpoint{5.622236in}{3.322777in}}%
\pgfpathlineto{\pgfqpoint{5.615079in}{3.319447in}}%
\pgfpathlineto{\pgfqpoint{5.607911in}{3.315975in}}%
\pgfpathlineto{\pgfqpoint{5.600733in}{3.312357in}}%
\pgfpathlineto{\pgfqpoint{5.593543in}{3.308591in}}%
\pgfpathclose%
\pgfusepath{fill}%
\end{pgfscope}%
\begin{pgfscope}%
\pgfpathrectangle{\pgfqpoint{1.254980in}{0.150000in}}{\pgfqpoint{5.490039in}{5.490039in}}%
\pgfusepath{clip}%
\pgfsetbuttcap%
\pgfsetroundjoin%
\definecolor{currentfill}{rgb}{0.282290,0.145912,0.461510}%
\pgfsetfillcolor{currentfill}%
\pgfsetfillopacity{0.700000}%
\pgfsetlinewidth{0.000000pt}%
\definecolor{currentstroke}{rgb}{0.000000,0.000000,0.000000}%
\pgfsetstrokecolor{currentstroke}%
\pgfsetdash{}{0pt}%
\pgfpathmoveto{\pgfqpoint{3.985807in}{1.533216in}}%
\pgfpathlineto{\pgfqpoint{3.999436in}{1.536364in}}%
\pgfpathlineto{\pgfqpoint{4.013076in}{1.539669in}}%
\pgfpathlineto{\pgfqpoint{4.026726in}{1.543132in}}%
\pgfpathlineto{\pgfqpoint{4.040387in}{1.546752in}}%
\pgfpathlineto{\pgfqpoint{4.048292in}{1.560307in}}%
\pgfpathlineto{\pgfqpoint{4.056192in}{1.573903in}}%
\pgfpathlineto{\pgfqpoint{4.064088in}{1.587536in}}%
\pgfpathlineto{\pgfqpoint{4.071980in}{1.601201in}}%
\pgfpathlineto{\pgfqpoint{4.058322in}{1.597103in}}%
\pgfpathlineto{\pgfqpoint{4.044675in}{1.593163in}}%
\pgfpathlineto{\pgfqpoint{4.031039in}{1.589380in}}%
\pgfpathlineto{\pgfqpoint{4.017413in}{1.585756in}}%
\pgfpathlineto{\pgfqpoint{4.009518in}{1.572557in}}%
\pgfpathlineto{\pgfqpoint{4.001619in}{1.559398in}}%
\pgfpathlineto{\pgfqpoint{3.993715in}{1.546283in}}%
\pgfpathlineto{\pgfqpoint{3.985807in}{1.533216in}}%
\pgfpathclose%
\pgfusepath{fill}%
\end{pgfscope}%
\begin{pgfscope}%
\pgfpathrectangle{\pgfqpoint{1.254980in}{0.150000in}}{\pgfqpoint{5.490039in}{5.490039in}}%
\pgfusepath{clip}%
\pgfsetbuttcap%
\pgfsetroundjoin%
\definecolor{currentfill}{rgb}{0.277018,0.050344,0.375715}%
\pgfsetfillcolor{currentfill}%
\pgfsetfillopacity{0.700000}%
\pgfsetlinewidth{0.000000pt}%
\definecolor{currentstroke}{rgb}{0.000000,0.000000,0.000000}%
\pgfsetstrokecolor{currentstroke}%
\pgfsetdash{}{0pt}%
\pgfpathmoveto{\pgfqpoint{3.131565in}{1.403382in}}%
\pgfpathlineto{\pgfqpoint{3.145108in}{1.394524in}}%
\pgfpathlineto{\pgfqpoint{3.158651in}{1.385841in}}%
\pgfpathlineto{\pgfqpoint{3.172195in}{1.377332in}}%
\pgfpathlineto{\pgfqpoint{3.185739in}{1.368997in}}%
\pgfpathlineto{\pgfqpoint{3.194093in}{1.370597in}}%
\pgfpathlineto{\pgfqpoint{3.202432in}{1.372503in}}%
\pgfpathlineto{\pgfqpoint{3.210757in}{1.374708in}}%
\pgfpathlineto{\pgfqpoint{3.219069in}{1.377204in}}%
\pgfpathlineto{\pgfqpoint{3.205559in}{1.384838in}}%
\pgfpathlineto{\pgfqpoint{3.192050in}{1.392646in}}%
\pgfpathlineto{\pgfqpoint{3.178542in}{1.400627in}}%
\pgfpathlineto{\pgfqpoint{3.165035in}{1.408783in}}%
\pgfpathlineto{\pgfqpoint{3.156689in}{1.406977in}}%
\pgfpathlineto{\pgfqpoint{3.148329in}{1.405470in}}%
\pgfpathlineto{\pgfqpoint{3.139954in}{1.404269in}}%
\pgfpathlineto{\pgfqpoint{3.131565in}{1.403382in}}%
\pgfpathclose%
\pgfusepath{fill}%
\end{pgfscope}%
\begin{pgfscope}%
\pgfpathrectangle{\pgfqpoint{1.254980in}{0.150000in}}{\pgfqpoint{5.490039in}{5.490039in}}%
\pgfusepath{clip}%
\pgfsetbuttcap%
\pgfsetroundjoin%
\definecolor{currentfill}{rgb}{0.262138,0.242286,0.520837}%
\pgfsetfillcolor{currentfill}%
\pgfsetfillopacity{0.700000}%
\pgfsetlinewidth{0.000000pt}%
\definecolor{currentstroke}{rgb}{0.000000,0.000000,0.000000}%
\pgfsetstrokecolor{currentstroke}%
\pgfsetdash{}{0pt}%
\pgfpathmoveto{\pgfqpoint{4.189692in}{1.732649in}}%
\pgfpathlineto{\pgfqpoint{4.203404in}{1.738411in}}%
\pgfpathlineto{\pgfqpoint{4.217130in}{1.744332in}}%
\pgfpathlineto{\pgfqpoint{4.230867in}{1.750411in}}%
\pgfpathlineto{\pgfqpoint{4.244617in}{1.756648in}}%
\pgfpathlineto{\pgfqpoint{4.252470in}{1.771229in}}%
\pgfpathlineto{\pgfqpoint{4.260320in}{1.785789in}}%
\pgfpathlineto{\pgfqpoint{4.268165in}{1.800323in}}%
\pgfpathlineto{\pgfqpoint{4.276006in}{1.814827in}}%
\pgfpathlineto{\pgfqpoint{4.262255in}{1.808193in}}%
\pgfpathlineto{\pgfqpoint{4.248517in}{1.801717in}}%
\pgfpathlineto{\pgfqpoint{4.234791in}{1.795400in}}%
\pgfpathlineto{\pgfqpoint{4.221078in}{1.789242in}}%
\pgfpathlineto{\pgfqpoint{4.213238in}{1.775123in}}%
\pgfpathlineto{\pgfqpoint{4.205393in}{1.760983in}}%
\pgfpathlineto{\pgfqpoint{4.197544in}{1.746823in}}%
\pgfpathlineto{\pgfqpoint{4.189692in}{1.732649in}}%
\pgfpathclose%
\pgfusepath{fill}%
\end{pgfscope}%
\begin{pgfscope}%
\pgfpathrectangle{\pgfqpoint{1.254980in}{0.150000in}}{\pgfqpoint{5.490039in}{5.490039in}}%
\pgfusepath{clip}%
\pgfsetbuttcap%
\pgfsetroundjoin%
\definecolor{currentfill}{rgb}{0.127568,0.566949,0.550556}%
\pgfsetfillcolor{currentfill}%
\pgfsetfillopacity{0.700000}%
\pgfsetlinewidth{0.000000pt}%
\definecolor{currentstroke}{rgb}{0.000000,0.000000,0.000000}%
\pgfsetstrokecolor{currentstroke}%
\pgfsetdash{}{0pt}%
\pgfpathmoveto{\pgfqpoint{4.864367in}{2.552859in}}%
\pgfpathlineto{\pgfqpoint{4.878468in}{2.565273in}}%
\pgfpathlineto{\pgfqpoint{4.892587in}{2.577849in}}%
\pgfpathlineto{\pgfqpoint{4.906724in}{2.590588in}}%
\pgfpathlineto{\pgfqpoint{4.920880in}{2.603491in}}%
\pgfpathlineto{\pgfqpoint{4.928519in}{2.615130in}}%
\pgfpathlineto{\pgfqpoint{4.936150in}{2.626610in}}%
\pgfpathlineto{\pgfqpoint{4.943775in}{2.637931in}}%
\pgfpathlineto{\pgfqpoint{4.951392in}{2.649092in}}%
\pgfpathlineto{\pgfqpoint{4.937235in}{2.636111in}}%
\pgfpathlineto{\pgfqpoint{4.923098in}{2.623294in}}%
\pgfpathlineto{\pgfqpoint{4.908978in}{2.610640in}}%
\pgfpathlineto{\pgfqpoint{4.894877in}{2.598149in}}%
\pgfpathlineto{\pgfqpoint{4.887260in}{2.587054in}}%
\pgfpathlineto{\pgfqpoint{4.879636in}{2.575808in}}%
\pgfpathlineto{\pgfqpoint{4.872005in}{2.564409in}}%
\pgfpathlineto{\pgfqpoint{4.864367in}{2.552859in}}%
\pgfpathclose%
\pgfusepath{fill}%
\end{pgfscope}%
\begin{pgfscope}%
\pgfpathrectangle{\pgfqpoint{1.254980in}{0.150000in}}{\pgfqpoint{5.490039in}{5.490039in}}%
\pgfusepath{clip}%
\pgfsetbuttcap%
\pgfsetroundjoin%
\definecolor{currentfill}{rgb}{0.127568,0.566949,0.550556}%
\pgfsetfillcolor{currentfill}%
\pgfsetfillopacity{0.700000}%
\pgfsetlinewidth{0.000000pt}%
\definecolor{currentstroke}{rgb}{0.000000,0.000000,0.000000}%
\pgfsetstrokecolor{currentstroke}%
\pgfsetdash{}{0pt}%
\pgfpathmoveto{\pgfqpoint{2.089883in}{2.670691in}}%
\pgfpathlineto{\pgfqpoint{2.103956in}{2.644939in}}%
\pgfpathlineto{\pgfqpoint{2.118013in}{2.619481in}}%
\pgfpathlineto{\pgfqpoint{2.132055in}{2.594315in}}%
\pgfpathlineto{\pgfqpoint{2.146080in}{2.569437in}}%
\pgfpathlineto{\pgfqpoint{2.155479in}{2.556908in}}%
\pgfpathlineto{\pgfqpoint{2.164846in}{2.544850in}}%
\pgfpathlineto{\pgfqpoint{2.174183in}{2.533254in}}%
\pgfpathlineto{\pgfqpoint{2.183489in}{2.522112in}}%
\pgfpathlineto{\pgfqpoint{2.169539in}{2.546207in}}%
\pgfpathlineto{\pgfqpoint{2.155575in}{2.570588in}}%
\pgfpathlineto{\pgfqpoint{2.141594in}{2.595258in}}%
\pgfpathlineto{\pgfqpoint{2.127599in}{2.620221in}}%
\pgfpathlineto{\pgfqpoint{2.118218in}{2.632134in}}%
\pgfpathlineto{\pgfqpoint{2.108805in}{2.644512in}}%
\pgfpathlineto{\pgfqpoint{2.099360in}{2.657361in}}%
\pgfpathlineto{\pgfqpoint{2.089883in}{2.670691in}}%
\pgfpathclose%
\pgfusepath{fill}%
\end{pgfscope}%
\begin{pgfscope}%
\pgfpathrectangle{\pgfqpoint{1.254980in}{0.150000in}}{\pgfqpoint{5.490039in}{5.490039in}}%
\pgfusepath{clip}%
\pgfsetbuttcap%
\pgfsetroundjoin%
\definecolor{currentfill}{rgb}{0.525776,0.833491,0.288127}%
\pgfsetfillcolor{currentfill}%
\pgfsetfillopacity{0.700000}%
\pgfsetlinewidth{0.000000pt}%
\definecolor{currentstroke}{rgb}{0.000000,0.000000,0.000000}%
\pgfsetstrokecolor{currentstroke}%
\pgfsetdash{}{0pt}%
\pgfpathmoveto{\pgfqpoint{5.680709in}{3.387272in}}%
\pgfpathlineto{\pgfqpoint{5.695384in}{3.403811in}}%
\pgfpathlineto{\pgfqpoint{5.710081in}{3.420516in}}%
\pgfpathlineto{\pgfqpoint{5.724802in}{3.437387in}}%
\pgfpathlineto{\pgfqpoint{5.731916in}{3.440065in}}%
\pgfpathlineto{\pgfqpoint{5.739019in}{3.442593in}}%
\pgfpathlineto{\pgfqpoint{5.746110in}{3.444976in}}%
\pgfpathlineto{\pgfqpoint{5.753191in}{3.447217in}}%
\pgfpathlineto{\pgfqpoint{5.738489in}{3.430652in}}%
\pgfpathlineto{\pgfqpoint{5.723811in}{3.414253in}}%
\pgfpathlineto{\pgfqpoint{5.709155in}{3.398020in}}%
\pgfpathlineto{\pgfqpoint{5.702060in}{3.395541in}}%
\pgfpathlineto{\pgfqpoint{5.694954in}{3.392926in}}%
\pgfpathlineto{\pgfqpoint{5.687837in}{3.390171in}}%
\pgfpathlineto{\pgfqpoint{5.680709in}{3.387272in}}%
\pgfpathclose%
\pgfusepath{fill}%
\end{pgfscope}%
\begin{pgfscope}%
\pgfpathrectangle{\pgfqpoint{1.254980in}{0.150000in}}{\pgfqpoint{5.490039in}{5.490039in}}%
\pgfusepath{clip}%
\pgfsetbuttcap%
\pgfsetroundjoin%
\definecolor{currentfill}{rgb}{0.175707,0.697900,0.491033}%
\pgfsetfillcolor{currentfill}%
\pgfsetfillopacity{0.700000}%
\pgfsetlinewidth{0.000000pt}%
\definecolor{currentstroke}{rgb}{0.000000,0.000000,0.000000}%
\pgfsetstrokecolor{currentstroke}%
\pgfsetdash{}{0pt}%
\pgfpathmoveto{\pgfqpoint{5.186065in}{2.915694in}}%
\pgfpathlineto{\pgfqpoint{5.200386in}{2.930151in}}%
\pgfpathlineto{\pgfqpoint{5.214727in}{2.944773in}}%
\pgfpathlineto{\pgfqpoint{5.229088in}{2.959560in}}%
\pgfpathlineto{\pgfqpoint{5.243470in}{2.974512in}}%
\pgfpathlineto{\pgfqpoint{5.250937in}{2.982780in}}%
\pgfpathlineto{\pgfqpoint{5.258394in}{2.990873in}}%
\pgfpathlineto{\pgfqpoint{5.265842in}{2.998792in}}%
\pgfpathlineto{\pgfqpoint{5.273280in}{3.006537in}}%
\pgfpathlineto{\pgfqpoint{5.258904in}{2.991664in}}%
\pgfpathlineto{\pgfqpoint{5.244548in}{2.976956in}}%
\pgfpathlineto{\pgfqpoint{5.230213in}{2.962413in}}%
\pgfpathlineto{\pgfqpoint{5.215897in}{2.948035in}}%
\pgfpathlineto{\pgfqpoint{5.208453in}{2.940199in}}%
\pgfpathlineto{\pgfqpoint{5.200999in}{2.932198in}}%
\pgfpathlineto{\pgfqpoint{5.193536in}{2.924030in}}%
\pgfpathlineto{\pgfqpoint{5.186065in}{2.915694in}}%
\pgfpathclose%
\pgfusepath{fill}%
\end{pgfscope}%
\begin{pgfscope}%
\pgfpathrectangle{\pgfqpoint{1.254980in}{0.150000in}}{\pgfqpoint{5.490039in}{5.490039in}}%
\pgfusepath{clip}%
\pgfsetbuttcap%
\pgfsetroundjoin%
\definecolor{currentfill}{rgb}{0.269944,0.014625,0.341379}%
\pgfsetfillcolor{currentfill}%
\pgfsetfillopacity{0.700000}%
\pgfsetlinewidth{0.000000pt}%
\definecolor{currentstroke}{rgb}{0.000000,0.000000,0.000000}%
\pgfsetstrokecolor{currentstroke}%
\pgfsetdash{}{0pt}%
\pgfpathmoveto{\pgfqpoint{3.327217in}{1.322289in}}%
\pgfpathlineto{\pgfqpoint{3.340745in}{1.316185in}}%
\pgfpathlineto{\pgfqpoint{3.354277in}{1.310249in}}%
\pgfpathlineto{\pgfqpoint{3.367812in}{1.304479in}}%
\pgfpathlineto{\pgfqpoint{3.381349in}{1.298875in}}%
\pgfpathlineto{\pgfqpoint{3.389559in}{1.303697in}}%
\pgfpathlineto{\pgfqpoint{3.397758in}{1.308775in}}%
\pgfpathlineto{\pgfqpoint{3.405946in}{1.314102in}}%
\pgfpathlineto{\pgfqpoint{3.414124in}{1.319670in}}%
\pgfpathlineto{\pgfqpoint{3.400613in}{1.324605in}}%
\pgfpathlineto{\pgfqpoint{3.387105in}{1.329707in}}%
\pgfpathlineto{\pgfqpoint{3.373600in}{1.334974in}}%
\pgfpathlineto{\pgfqpoint{3.360099in}{1.340409in}}%
\pgfpathlineto{\pgfqpoint{3.351895in}{1.335499in}}%
\pgfpathlineto{\pgfqpoint{3.343680in}{1.330837in}}%
\pgfpathlineto{\pgfqpoint{3.335454in}{1.326432in}}%
\pgfpathlineto{\pgfqpoint{3.327217in}{1.322289in}}%
\pgfpathclose%
\pgfusepath{fill}%
\end{pgfscope}%
\begin{pgfscope}%
\pgfpathrectangle{\pgfqpoint{1.254980in}{0.150000in}}{\pgfqpoint{5.490039in}{5.490039in}}%
\pgfusepath{clip}%
\pgfsetbuttcap%
\pgfsetroundjoin%
\definecolor{currentfill}{rgb}{0.304148,0.764704,0.419943}%
\pgfsetfillcolor{currentfill}%
\pgfsetfillopacity{0.700000}%
\pgfsetlinewidth{0.000000pt}%
\definecolor{currentstroke}{rgb}{0.000000,0.000000,0.000000}%
\pgfsetstrokecolor{currentstroke}%
\pgfsetdash{}{0pt}%
\pgfpathmoveto{\pgfqpoint{5.390090in}{3.122565in}}%
\pgfpathlineto{\pgfqpoint{5.404555in}{3.138010in}}%
\pgfpathlineto{\pgfqpoint{5.419041in}{3.153620in}}%
\pgfpathlineto{\pgfqpoint{5.433549in}{3.169397in}}%
\pgfpathlineto{\pgfqpoint{5.448078in}{3.185340in}}%
\pgfpathlineto{\pgfqpoint{5.455410in}{3.191269in}}%
\pgfpathlineto{\pgfqpoint{5.462731in}{3.197026in}}%
\pgfpathlineto{\pgfqpoint{5.470041in}{3.202614in}}%
\pgfpathlineto{\pgfqpoint{5.477341in}{3.208034in}}%
\pgfpathlineto{\pgfqpoint{5.462822in}{3.192267in}}%
\pgfpathlineto{\pgfqpoint{5.448325in}{3.176667in}}%
\pgfpathlineto{\pgfqpoint{5.433850in}{3.161231in}}%
\pgfpathlineto{\pgfqpoint{5.419395in}{3.145961in}}%
\pgfpathlineto{\pgfqpoint{5.412084in}{3.140354in}}%
\pgfpathlineto{\pgfqpoint{5.404763in}{3.134587in}}%
\pgfpathlineto{\pgfqpoint{5.397432in}{3.128658in}}%
\pgfpathlineto{\pgfqpoint{5.390090in}{3.122565in}}%
\pgfpathclose%
\pgfusepath{fill}%
\end{pgfscope}%
\begin{pgfscope}%
\pgfpathrectangle{\pgfqpoint{1.254980in}{0.150000in}}{\pgfqpoint{5.490039in}{5.490039in}}%
\pgfusepath{clip}%
\pgfsetbuttcap%
\pgfsetroundjoin%
\definecolor{currentfill}{rgb}{0.268510,0.009605,0.335427}%
\pgfsetfillcolor{currentfill}%
\pgfsetfillopacity{0.700000}%
\pgfsetlinewidth{0.000000pt}%
\definecolor{currentstroke}{rgb}{0.000000,0.000000,0.000000}%
\pgfsetstrokecolor{currentstroke}%
\pgfsetdash{}{0pt}%
\pgfpathmoveto{\pgfqpoint{3.468208in}{1.301579in}}%
\pgfpathlineto{\pgfqpoint{3.481740in}{1.297466in}}%
\pgfpathlineto{\pgfqpoint{3.495276in}{1.293516in}}%
\pgfpathlineto{\pgfqpoint{3.508816in}{1.289729in}}%
\pgfpathlineto{\pgfqpoint{3.522362in}{1.286105in}}%
\pgfpathlineto{\pgfqpoint{3.530483in}{1.293209in}}%
\pgfpathlineto{\pgfqpoint{3.538596in}{1.300527in}}%
\pgfpathlineto{\pgfqpoint{3.546700in}{1.308053in}}%
\pgfpathlineto{\pgfqpoint{3.554796in}{1.315780in}}%
\pgfpathlineto{\pgfqpoint{3.541271in}{1.318765in}}%
\pgfpathlineto{\pgfqpoint{3.527751in}{1.321913in}}%
\pgfpathlineto{\pgfqpoint{3.514237in}{1.325223in}}%
\pgfpathlineto{\pgfqpoint{3.500727in}{1.328697in}}%
\pgfpathlineto{\pgfqpoint{3.492611in}{1.321598in}}%
\pgfpathlineto{\pgfqpoint{3.484486in}{1.314708in}}%
\pgfpathlineto{\pgfqpoint{3.476352in}{1.308032in}}%
\pgfpathlineto{\pgfqpoint{3.468208in}{1.301579in}}%
\pgfpathclose%
\pgfusepath{fill}%
\end{pgfscope}%
\begin{pgfscope}%
\pgfpathrectangle{\pgfqpoint{1.254980in}{0.150000in}}{\pgfqpoint{5.490039in}{5.490039in}}%
\pgfusepath{clip}%
\pgfsetbuttcap%
\pgfsetroundjoin%
\definecolor{currentfill}{rgb}{0.146180,0.515413,0.556823}%
\pgfsetfillcolor{currentfill}%
\pgfsetfillopacity{0.700000}%
\pgfsetlinewidth{0.000000pt}%
\definecolor{currentstroke}{rgb}{0.000000,0.000000,0.000000}%
\pgfsetstrokecolor{currentstroke}%
\pgfsetdash{}{0pt}%
\pgfpathmoveto{\pgfqpoint{4.746804in}{2.408037in}}%
\pgfpathlineto{\pgfqpoint{4.760832in}{2.419554in}}%
\pgfpathlineto{\pgfqpoint{4.774877in}{2.431234in}}%
\pgfpathlineto{\pgfqpoint{4.788939in}{2.443075in}}%
\pgfpathlineto{\pgfqpoint{4.803019in}{2.455079in}}%
\pgfpathlineto{\pgfqpoint{4.810710in}{2.467817in}}%
\pgfpathlineto{\pgfqpoint{4.818396in}{2.480410in}}%
\pgfpathlineto{\pgfqpoint{4.826074in}{2.492856in}}%
\pgfpathlineto{\pgfqpoint{4.833746in}{2.505155in}}%
\pgfpathlineto{\pgfqpoint{4.819664in}{2.493012in}}%
\pgfpathlineto{\pgfqpoint{4.805600in}{2.481032in}}%
\pgfpathlineto{\pgfqpoint{4.791553in}{2.469214in}}%
\pgfpathlineto{\pgfqpoint{4.777523in}{2.457559in}}%
\pgfpathlineto{\pgfqpoint{4.769853in}{2.445387in}}%
\pgfpathlineto{\pgfqpoint{4.762176in}{2.433076in}}%
\pgfpathlineto{\pgfqpoint{4.754493in}{2.420626in}}%
\pgfpathlineto{\pgfqpoint{4.746804in}{2.408037in}}%
\pgfpathclose%
\pgfusepath{fill}%
\end{pgfscope}%
\begin{pgfscope}%
\pgfpathrectangle{\pgfqpoint{1.254980in}{0.150000in}}{\pgfqpoint{5.490039in}{5.490039in}}%
\pgfusepath{clip}%
\pgfsetbuttcap%
\pgfsetroundjoin%
\definecolor{currentfill}{rgb}{0.277134,0.185228,0.489898}%
\pgfsetfillcolor{currentfill}%
\pgfsetfillopacity{0.700000}%
\pgfsetlinewidth{0.000000pt}%
\definecolor{currentstroke}{rgb}{0.000000,0.000000,0.000000}%
\pgfsetstrokecolor{currentstroke}%
\pgfsetdash{}{0pt}%
\pgfpathmoveto{\pgfqpoint{4.071980in}{1.601201in}}%
\pgfpathlineto{\pgfqpoint{4.085648in}{1.605457in}}%
\pgfpathlineto{\pgfqpoint{4.099328in}{1.609871in}}%
\pgfpathlineto{\pgfqpoint{4.113019in}{1.614442in}}%
\pgfpathlineto{\pgfqpoint{4.126721in}{1.619170in}}%
\pgfpathlineto{\pgfqpoint{4.134607in}{1.633326in}}%
\pgfpathlineto{\pgfqpoint{4.142488in}{1.647498in}}%
\pgfpathlineto{\pgfqpoint{4.150366in}{1.661682in}}%
\pgfpathlineto{\pgfqpoint{4.158239in}{1.675875in}}%
\pgfpathlineto{\pgfqpoint{4.144538in}{1.670695in}}%
\pgfpathlineto{\pgfqpoint{4.130848in}{1.665672in}}%
\pgfpathlineto{\pgfqpoint{4.117170in}{1.660807in}}%
\pgfpathlineto{\pgfqpoint{4.103503in}{1.656101in}}%
\pgfpathlineto{\pgfqpoint{4.095629in}{1.642349in}}%
\pgfpathlineto{\pgfqpoint{4.087750in}{1.628612in}}%
\pgfpathlineto{\pgfqpoint{4.079867in}{1.614895in}}%
\pgfpathlineto{\pgfqpoint{4.071980in}{1.601201in}}%
\pgfpathclose%
\pgfusepath{fill}%
\end{pgfscope}%
\begin{pgfscope}%
\pgfpathrectangle{\pgfqpoint{1.254980in}{0.150000in}}{\pgfqpoint{5.490039in}{5.490039in}}%
\pgfusepath{clip}%
\pgfsetbuttcap%
\pgfsetroundjoin%
\definecolor{currentfill}{rgb}{0.166617,0.463708,0.558119}%
\pgfsetfillcolor{currentfill}%
\pgfsetfillopacity{0.700000}%
\pgfsetlinewidth{0.000000pt}%
\definecolor{currentstroke}{rgb}{0.000000,0.000000,0.000000}%
\pgfsetstrokecolor{currentstroke}%
\pgfsetdash{}{0pt}%
\pgfpathmoveto{\pgfqpoint{4.629152in}{2.259532in}}%
\pgfpathlineto{\pgfqpoint{4.643108in}{2.270036in}}%
\pgfpathlineto{\pgfqpoint{4.657080in}{2.280702in}}%
\pgfpathlineto{\pgfqpoint{4.671069in}{2.291529in}}%
\pgfpathlineto{\pgfqpoint{4.685074in}{2.302517in}}%
\pgfpathlineto{\pgfqpoint{4.692811in}{2.316162in}}%
\pgfpathlineto{\pgfqpoint{4.700542in}{2.329680in}}%
\pgfpathlineto{\pgfqpoint{4.708267in}{2.343070in}}%
\pgfpathlineto{\pgfqpoint{4.715986in}{2.356329in}}%
\pgfpathlineto{\pgfqpoint{4.701978in}{2.345142in}}%
\pgfpathlineto{\pgfqpoint{4.687986in}{2.334117in}}%
\pgfpathlineto{\pgfqpoint{4.674011in}{2.323254in}}%
\pgfpathlineto{\pgfqpoint{4.660052in}{2.312552in}}%
\pgfpathlineto{\pgfqpoint{4.652335in}{2.299479in}}%
\pgfpathlineto{\pgfqpoint{4.644613in}{2.286284in}}%
\pgfpathlineto{\pgfqpoint{4.636885in}{2.272968in}}%
\pgfpathlineto{\pgfqpoint{4.629152in}{2.259532in}}%
\pgfpathclose%
\pgfusepath{fill}%
\end{pgfscope}%
\begin{pgfscope}%
\pgfpathrectangle{\pgfqpoint{1.254980in}{0.150000in}}{\pgfqpoint{5.490039in}{5.490039in}}%
\pgfusepath{clip}%
\pgfsetbuttcap%
\pgfsetroundjoin%
\definecolor{currentfill}{rgb}{0.190631,0.407061,0.556089}%
\pgfsetfillcolor{currentfill}%
\pgfsetfillopacity{0.700000}%
\pgfsetlinewidth{0.000000pt}%
\definecolor{currentstroke}{rgb}{0.000000,0.000000,0.000000}%
\pgfsetstrokecolor{currentstroke}%
\pgfsetdash{}{0pt}%
\pgfpathmoveto{\pgfqpoint{4.511455in}{2.109496in}}%
\pgfpathlineto{\pgfqpoint{4.525342in}{2.118874in}}%
\pgfpathlineto{\pgfqpoint{4.539245in}{2.128411in}}%
\pgfpathlineto{\pgfqpoint{4.553162in}{2.138110in}}%
\pgfpathlineto{\pgfqpoint{4.567096in}{2.147968in}}%
\pgfpathlineto{\pgfqpoint{4.574871in}{2.162291in}}%
\pgfpathlineto{\pgfqpoint{4.582641in}{2.176511in}}%
\pgfpathlineto{\pgfqpoint{4.590405in}{2.190625in}}%
\pgfpathlineto{\pgfqpoint{4.598165in}{2.204631in}}%
\pgfpathlineto{\pgfqpoint{4.584228in}{2.194515in}}%
\pgfpathlineto{\pgfqpoint{4.570307in}{2.184560in}}%
\pgfpathlineto{\pgfqpoint{4.556401in}{2.174766in}}%
\pgfpathlineto{\pgfqpoint{4.542511in}{2.165132in}}%
\pgfpathlineto{\pgfqpoint{4.534754in}{2.151372in}}%
\pgfpathlineto{\pgfqpoint{4.526993in}{2.137511in}}%
\pgfpathlineto{\pgfqpoint{4.519226in}{2.123551in}}%
\pgfpathlineto{\pgfqpoint{4.511455in}{2.109496in}}%
\pgfpathclose%
\pgfusepath{fill}%
\end{pgfscope}%
\begin{pgfscope}%
\pgfpathrectangle{\pgfqpoint{1.254980in}{0.150000in}}{\pgfqpoint{5.490039in}{5.490039in}}%
\pgfusepath{clip}%
\pgfsetbuttcap%
\pgfsetroundjoin%
\definecolor{currentfill}{rgb}{0.278791,0.062145,0.386592}%
\pgfsetfillcolor{currentfill}%
\pgfsetfillopacity{0.700000}%
\pgfsetlinewidth{0.000000pt}%
\definecolor{currentstroke}{rgb}{0.000000,0.000000,0.000000}%
\pgfsetstrokecolor{currentstroke}%
\pgfsetdash{}{0pt}%
\pgfpathmoveto{\pgfqpoint{3.781661in}{1.375180in}}%
\pgfpathlineto{\pgfqpoint{3.795241in}{1.375471in}}%
\pgfpathlineto{\pgfqpoint{3.808828in}{1.375920in}}%
\pgfpathlineto{\pgfqpoint{3.822424in}{1.376527in}}%
\pgfpathlineto{\pgfqpoint{3.836029in}{1.377292in}}%
\pgfpathlineto{\pgfqpoint{3.844004in}{1.388831in}}%
\pgfpathlineto{\pgfqpoint{3.851975in}{1.400485in}}%
\pgfpathlineto{\pgfqpoint{3.859940in}{1.412249in}}%
\pgfpathlineto{\pgfqpoint{3.867899in}{1.424117in}}%
\pgfpathlineto{\pgfqpoint{3.854304in}{1.422794in}}%
\pgfpathlineto{\pgfqpoint{3.840717in}{1.421629in}}%
\pgfpathlineto{\pgfqpoint{3.827139in}{1.420622in}}%
\pgfpathlineto{\pgfqpoint{3.813569in}{1.419773in}}%
\pgfpathlineto{\pgfqpoint{3.805600in}{1.408453in}}%
\pgfpathlineto{\pgfqpoint{3.797626in}{1.397243in}}%
\pgfpathlineto{\pgfqpoint{3.789646in}{1.386151in}}%
\pgfpathlineto{\pgfqpoint{3.781661in}{1.375180in}}%
\pgfpathclose%
\pgfusepath{fill}%
\end{pgfscope}%
\begin{pgfscope}%
\pgfpathrectangle{\pgfqpoint{1.254980in}{0.150000in}}{\pgfqpoint{5.490039in}{5.490039in}}%
\pgfusepath{clip}%
\pgfsetbuttcap%
\pgfsetroundjoin%
\definecolor{currentfill}{rgb}{0.274952,0.037752,0.364543}%
\pgfsetfillcolor{currentfill}%
\pgfsetfillopacity{0.700000}%
\pgfsetlinewidth{0.000000pt}%
\definecolor{currentstroke}{rgb}{0.000000,0.000000,0.000000}%
\pgfsetstrokecolor{currentstroke}%
\pgfsetdash{}{0pt}%
\pgfpathmoveto{\pgfqpoint{3.185739in}{1.368997in}}%
\pgfpathlineto{\pgfqpoint{3.199285in}{1.360834in}}%
\pgfpathlineto{\pgfqpoint{3.212832in}{1.352844in}}%
\pgfpathlineto{\pgfqpoint{3.226380in}{1.345025in}}%
\pgfpathlineto{\pgfqpoint{3.239929in}{1.337377in}}%
\pgfpathlineto{\pgfqpoint{3.248248in}{1.339689in}}%
\pgfpathlineto{\pgfqpoint{3.256554in}{1.342299in}}%
\pgfpathlineto{\pgfqpoint{3.264847in}{1.345201in}}%
\pgfpathlineto{\pgfqpoint{3.273126in}{1.348386in}}%
\pgfpathlineto{\pgfqpoint{3.259609in}{1.355334in}}%
\pgfpathlineto{\pgfqpoint{3.246094in}{1.362453in}}%
\pgfpathlineto{\pgfqpoint{3.232581in}{1.369742in}}%
\pgfpathlineto{\pgfqpoint{3.219069in}{1.377204in}}%
\pgfpathlineto{\pgfqpoint{3.210757in}{1.374708in}}%
\pgfpathlineto{\pgfqpoint{3.202432in}{1.372503in}}%
\pgfpathlineto{\pgfqpoint{3.194093in}{1.370597in}}%
\pgfpathlineto{\pgfqpoint{3.185739in}{1.368997in}}%
\pgfpathclose%
\pgfusepath{fill}%
\end{pgfscope}%
\begin{pgfscope}%
\pgfpathrectangle{\pgfqpoint{1.254980in}{0.150000in}}{\pgfqpoint{5.490039in}{5.490039in}}%
\pgfusepath{clip}%
\pgfsetbuttcap%
\pgfsetroundjoin%
\definecolor{currentfill}{rgb}{0.216210,0.351535,0.550627}%
\pgfsetfillcolor{currentfill}%
\pgfsetfillopacity{0.700000}%
\pgfsetlinewidth{0.000000pt}%
\definecolor{currentstroke}{rgb}{0.000000,0.000000,0.000000}%
\pgfsetstrokecolor{currentstroke}%
\pgfsetdash{}{0pt}%
\pgfpathmoveto{\pgfqpoint{4.393738in}{1.960359in}}%
\pgfpathlineto{\pgfqpoint{4.407561in}{1.968499in}}%
\pgfpathlineto{\pgfqpoint{4.421398in}{1.976798in}}%
\pgfpathlineto{\pgfqpoint{4.435249in}{1.985256in}}%
\pgfpathlineto{\pgfqpoint{4.449115in}{1.993874in}}%
\pgfpathlineto{\pgfqpoint{4.456923in}{2.008611in}}%
\pgfpathlineto{\pgfqpoint{4.464727in}{2.023273in}}%
\pgfpathlineto{\pgfqpoint{4.472527in}{2.037856in}}%
\pgfpathlineto{\pgfqpoint{4.480321in}{2.052357in}}%
\pgfpathlineto{\pgfqpoint{4.466452in}{2.043425in}}%
\pgfpathlineto{\pgfqpoint{4.452598in}{2.034652in}}%
\pgfpathlineto{\pgfqpoint{4.438758in}{2.026040in}}%
\pgfpathlineto{\pgfqpoint{4.424933in}{2.017587in}}%
\pgfpathlineto{\pgfqpoint{4.417141in}{2.003388in}}%
\pgfpathlineto{\pgfqpoint{4.409344in}{1.989116in}}%
\pgfpathlineto{\pgfqpoint{4.401543in}{1.974772in}}%
\pgfpathlineto{\pgfqpoint{4.393738in}{1.960359in}}%
\pgfpathclose%
\pgfusepath{fill}%
\end{pgfscope}%
\begin{pgfscope}%
\pgfpathrectangle{\pgfqpoint{1.254980in}{0.150000in}}{\pgfqpoint{5.490039in}{5.490039in}}%
\pgfusepath{clip}%
\pgfsetbuttcap%
\pgfsetroundjoin%
\definecolor{currentfill}{rgb}{0.274952,0.037752,0.364543}%
\pgfsetfillcolor{currentfill}%
\pgfsetfillopacity{0.700000}%
\pgfsetlinewidth{0.000000pt}%
\definecolor{currentstroke}{rgb}{0.000000,0.000000,0.000000}%
\pgfsetstrokecolor{currentstroke}%
\pgfsetdash{}{0pt}%
\pgfpathmoveto{\pgfqpoint{3.695362in}{1.335392in}}%
\pgfpathlineto{\pgfqpoint{3.708925in}{1.334463in}}%
\pgfpathlineto{\pgfqpoint{3.722494in}{1.333693in}}%
\pgfpathlineto{\pgfqpoint{3.736071in}{1.333082in}}%
\pgfpathlineto{\pgfqpoint{3.749655in}{1.332629in}}%
\pgfpathlineto{\pgfqpoint{3.757666in}{1.343056in}}%
\pgfpathlineto{\pgfqpoint{3.765670in}{1.353627in}}%
\pgfpathlineto{\pgfqpoint{3.773669in}{1.364337in}}%
\pgfpathlineto{\pgfqpoint{3.781661in}{1.375180in}}%
\pgfpathlineto{\pgfqpoint{3.768088in}{1.375048in}}%
\pgfpathlineto{\pgfqpoint{3.754524in}{1.375074in}}%
\pgfpathlineto{\pgfqpoint{3.740967in}{1.375259in}}%
\pgfpathlineto{\pgfqpoint{3.727417in}{1.375604in}}%
\pgfpathlineto{\pgfqpoint{3.719413in}{1.365335in}}%
\pgfpathlineto{\pgfqpoint{3.711403in}{1.355206in}}%
\pgfpathlineto{\pgfqpoint{3.703386in}{1.345223in}}%
\pgfpathlineto{\pgfqpoint{3.695362in}{1.335392in}}%
\pgfpathclose%
\pgfusepath{fill}%
\end{pgfscope}%
\begin{pgfscope}%
\pgfpathrectangle{\pgfqpoint{1.254980in}{0.150000in}}{\pgfqpoint{5.490039in}{5.490039in}}%
\pgfusepath{clip}%
\pgfsetbuttcap%
\pgfsetroundjoin%
\definecolor{currentfill}{rgb}{0.132268,0.655014,0.519661}%
\pgfsetfillcolor{currentfill}%
\pgfsetfillopacity{0.700000}%
\pgfsetlinewidth{0.000000pt}%
\definecolor{currentstroke}{rgb}{0.000000,0.000000,0.000000}%
\pgfsetstrokecolor{currentstroke}%
\pgfsetdash{}{0pt}%
\pgfpathmoveto{\pgfqpoint{5.068856in}{2.786414in}}%
\pgfpathlineto{\pgfqpoint{5.083105in}{2.800276in}}%
\pgfpathlineto{\pgfqpoint{5.097373in}{2.814302in}}%
\pgfpathlineto{\pgfqpoint{5.111661in}{2.828493in}}%
\pgfpathlineto{\pgfqpoint{5.125969in}{2.842848in}}%
\pgfpathlineto{\pgfqpoint{5.133512in}{2.852560in}}%
\pgfpathlineto{\pgfqpoint{5.141046in}{2.862098in}}%
\pgfpathlineto{\pgfqpoint{5.148571in}{2.871461in}}%
\pgfpathlineto{\pgfqpoint{5.156087in}{2.880651in}}%
\pgfpathlineto{\pgfqpoint{5.141782in}{2.866312in}}%
\pgfpathlineto{\pgfqpoint{5.127497in}{2.852137in}}%
\pgfpathlineto{\pgfqpoint{5.113232in}{2.838126in}}%
\pgfpathlineto{\pgfqpoint{5.098985in}{2.824280in}}%
\pgfpathlineto{\pgfqpoint{5.091466in}{2.815062in}}%
\pgfpathlineto{\pgfqpoint{5.083937in}{2.805679in}}%
\pgfpathlineto{\pgfqpoint{5.076401in}{2.796130in}}%
\pgfpathlineto{\pgfqpoint{5.068856in}{2.786414in}}%
\pgfpathclose%
\pgfusepath{fill}%
\end{pgfscope}%
\begin{pgfscope}%
\pgfpathrectangle{\pgfqpoint{1.254980in}{0.150000in}}{\pgfqpoint{5.490039in}{5.490039in}}%
\pgfusepath{clip}%
\pgfsetbuttcap%
\pgfsetroundjoin%
\definecolor{currentfill}{rgb}{0.282327,0.094955,0.417331}%
\pgfsetfillcolor{currentfill}%
\pgfsetfillopacity{0.700000}%
\pgfsetlinewidth{0.000000pt}%
\definecolor{currentstroke}{rgb}{0.000000,0.000000,0.000000}%
\pgfsetstrokecolor{currentstroke}%
\pgfsetdash{}{0pt}%
\pgfpathmoveto{\pgfqpoint{3.867899in}{1.424117in}}%
\pgfpathlineto{\pgfqpoint{3.881503in}{1.425598in}}%
\pgfpathlineto{\pgfqpoint{3.895116in}{1.427236in}}%
\pgfpathlineto{\pgfqpoint{3.908738in}{1.429032in}}%
\pgfpathlineto{\pgfqpoint{3.922369in}{1.430985in}}%
\pgfpathlineto{\pgfqpoint{3.930315in}{1.443494in}}%
\pgfpathlineto{\pgfqpoint{3.938257in}{1.456091in}}%
\pgfpathlineto{\pgfqpoint{3.946194in}{1.468769in}}%
\pgfpathlineto{\pgfqpoint{3.954126in}{1.481524in}}%
\pgfpathlineto{\pgfqpoint{3.940501in}{1.479039in}}%
\pgfpathlineto{\pgfqpoint{3.926886in}{1.476711in}}%
\pgfpathlineto{\pgfqpoint{3.913280in}{1.474541in}}%
\pgfpathlineto{\pgfqpoint{3.899683in}{1.472530in}}%
\pgfpathlineto{\pgfqpoint{3.891745in}{1.460296in}}%
\pgfpathlineto{\pgfqpoint{3.883801in}{1.448145in}}%
\pgfpathlineto{\pgfqpoint{3.875853in}{1.436084in}}%
\pgfpathlineto{\pgfqpoint{3.867899in}{1.424117in}}%
\pgfpathclose%
\pgfusepath{fill}%
\end{pgfscope}%
\begin{pgfscope}%
\pgfpathrectangle{\pgfqpoint{1.254980in}{0.150000in}}{\pgfqpoint{5.490039in}{5.490039in}}%
\pgfusepath{clip}%
\pgfsetbuttcap%
\pgfsetroundjoin%
\definecolor{currentfill}{rgb}{0.244972,0.287675,0.537260}%
\pgfsetfillcolor{currentfill}%
\pgfsetfillopacity{0.700000}%
\pgfsetlinewidth{0.000000pt}%
\definecolor{currentstroke}{rgb}{0.000000,0.000000,0.000000}%
\pgfsetstrokecolor{currentstroke}%
\pgfsetdash{}{0pt}%
\pgfpathmoveto{\pgfqpoint{4.276006in}{1.814827in}}%
\pgfpathlineto{\pgfqpoint{4.289771in}{1.821620in}}%
\pgfpathlineto{\pgfqpoint{4.303548in}{1.828571in}}%
\pgfpathlineto{\pgfqpoint{4.317338in}{1.835680in}}%
\pgfpathlineto{\pgfqpoint{4.331142in}{1.842948in}}%
\pgfpathlineto{\pgfqpoint{4.338981in}{1.857800in}}%
\pgfpathlineto{\pgfqpoint{4.346816in}{1.872609in}}%
\pgfpathlineto{\pgfqpoint{4.354647in}{1.887370in}}%
\pgfpathlineto{\pgfqpoint{4.362474in}{1.902081in}}%
\pgfpathlineto{\pgfqpoint{4.348667in}{1.894443in}}%
\pgfpathlineto{\pgfqpoint{4.334875in}{1.886963in}}%
\pgfpathlineto{\pgfqpoint{4.321096in}{1.879642in}}%
\pgfpathlineto{\pgfqpoint{4.307330in}{1.872480in}}%
\pgfpathlineto{\pgfqpoint{4.299505in}{1.858128in}}%
\pgfpathlineto{\pgfqpoint{4.291677in}{1.843733in}}%
\pgfpathlineto{\pgfqpoint{4.283844in}{1.829299in}}%
\pgfpathlineto{\pgfqpoint{4.276006in}{1.814827in}}%
\pgfpathclose%
\pgfusepath{fill}%
\end{pgfscope}%
\begin{pgfscope}%
\pgfpathrectangle{\pgfqpoint{1.254980in}{0.150000in}}{\pgfqpoint{5.490039in}{5.490039in}}%
\pgfusepath{clip}%
\pgfsetbuttcap%
\pgfsetroundjoin%
\definecolor{currentfill}{rgb}{0.271305,0.019942,0.347269}%
\pgfsetfillcolor{currentfill}%
\pgfsetfillopacity{0.700000}%
\pgfsetlinewidth{0.000000pt}%
\definecolor{currentstroke}{rgb}{0.000000,0.000000,0.000000}%
\pgfsetstrokecolor{currentstroke}%
\pgfsetdash{}{0pt}%
\pgfpathmoveto{\pgfqpoint{3.608949in}{1.305455in}}%
\pgfpathlineto{\pgfqpoint{3.622502in}{1.303275in}}%
\pgfpathlineto{\pgfqpoint{3.636060in}{1.301256in}}%
\pgfpathlineto{\pgfqpoint{3.649625in}{1.299396in}}%
\pgfpathlineto{\pgfqpoint{3.663196in}{1.297696in}}%
\pgfpathlineto{\pgfqpoint{3.671248in}{1.306863in}}%
\pgfpathlineto{\pgfqpoint{3.679293in}{1.316206in}}%
\pgfpathlineto{\pgfqpoint{3.687331in}{1.325717in}}%
\pgfpathlineto{\pgfqpoint{3.695362in}{1.335392in}}%
\pgfpathlineto{\pgfqpoint{3.681806in}{1.336480in}}%
\pgfpathlineto{\pgfqpoint{3.668257in}{1.337727in}}%
\pgfpathlineto{\pgfqpoint{3.654714in}{1.339135in}}%
\pgfpathlineto{\pgfqpoint{3.641178in}{1.340703in}}%
\pgfpathlineto{\pgfqpoint{3.633132in}{1.331629in}}%
\pgfpathlineto{\pgfqpoint{3.625079in}{1.322726in}}%
\pgfpathlineto{\pgfqpoint{3.617018in}{1.314000in}}%
\pgfpathlineto{\pgfqpoint{3.608949in}{1.305455in}}%
\pgfpathclose%
\pgfusepath{fill}%
\end{pgfscope}%
\begin{pgfscope}%
\pgfpathrectangle{\pgfqpoint{1.254980in}{0.150000in}}{\pgfqpoint{5.490039in}{5.490039in}}%
\pgfusepath{clip}%
\pgfsetbuttcap%
\pgfsetroundjoin%
\definecolor{currentfill}{rgb}{0.119512,0.607464,0.540218}%
\pgfsetfillcolor{currentfill}%
\pgfsetfillopacity{0.700000}%
\pgfsetlinewidth{0.000000pt}%
\definecolor{currentstroke}{rgb}{0.000000,0.000000,0.000000}%
\pgfsetstrokecolor{currentstroke}%
\pgfsetdash{}{0pt}%
\pgfpathmoveto{\pgfqpoint{2.033418in}{2.776701in}}%
\pgfpathlineto{\pgfqpoint{2.047560in}{2.749742in}}%
\pgfpathlineto{\pgfqpoint{2.061685in}{2.723090in}}%
\pgfpathlineto{\pgfqpoint{2.075792in}{2.696740in}}%
\pgfpathlineto{\pgfqpoint{2.089883in}{2.670691in}}%
\pgfpathlineto{\pgfqpoint{2.099360in}{2.657361in}}%
\pgfpathlineto{\pgfqpoint{2.108805in}{2.644512in}}%
\pgfpathlineto{\pgfqpoint{2.118218in}{2.632134in}}%
\pgfpathlineto{\pgfqpoint{2.127599in}{2.620221in}}%
\pgfpathlineto{\pgfqpoint{2.113587in}{2.645479in}}%
\pgfpathlineto{\pgfqpoint{2.099559in}{2.671034in}}%
\pgfpathlineto{\pgfqpoint{2.085514in}{2.696890in}}%
\pgfpathlineto{\pgfqpoint{2.071452in}{2.723050in}}%
\pgfpathlineto{\pgfqpoint{2.061993in}{2.735744in}}%
\pgfpathlineto{\pgfqpoint{2.052502in}{2.748911in}}%
\pgfpathlineto{\pgfqpoint{2.042977in}{2.762560in}}%
\pgfpathlineto{\pgfqpoint{2.033418in}{2.776701in}}%
\pgfpathclose%
\pgfusepath{fill}%
\end{pgfscope}%
\begin{pgfscope}%
\pgfpathrectangle{\pgfqpoint{1.254980in}{0.150000in}}{\pgfqpoint{5.490039in}{5.490039in}}%
\pgfusepath{clip}%
\pgfsetbuttcap%
\pgfsetroundjoin%
\definecolor{currentfill}{rgb}{0.386433,0.794644,0.372886}%
\pgfsetfillcolor{currentfill}%
\pgfsetfillopacity{0.700000}%
\pgfsetlinewidth{0.000000pt}%
\definecolor{currentstroke}{rgb}{0.000000,0.000000,0.000000}%
\pgfsetstrokecolor{currentstroke}%
\pgfsetdash{}{0pt}%
\pgfpathmoveto{\pgfqpoint{5.477341in}{3.208034in}}%
\pgfpathlineto{\pgfqpoint{5.491881in}{3.223966in}}%
\pgfpathlineto{\pgfqpoint{5.506444in}{3.240065in}}%
\pgfpathlineto{\pgfqpoint{5.521028in}{3.256330in}}%
\pgfpathlineto{\pgfqpoint{5.535635in}{3.272762in}}%
\pgfpathlineto{\pgfqpoint{5.542912in}{3.277820in}}%
\pgfpathlineto{\pgfqpoint{5.550178in}{3.282707in}}%
\pgfpathlineto{\pgfqpoint{5.557433in}{3.287426in}}%
\pgfpathlineto{\pgfqpoint{5.564677in}{3.291979in}}%
\pgfpathlineto{\pgfqpoint{5.550083in}{3.275757in}}%
\pgfpathlineto{\pgfqpoint{5.535512in}{3.259701in}}%
\pgfpathlineto{\pgfqpoint{5.520962in}{3.243811in}}%
\pgfpathlineto{\pgfqpoint{5.506434in}{3.228086in}}%
\pgfpathlineto{\pgfqpoint{5.499177in}{3.223312in}}%
\pgfpathlineto{\pgfqpoint{5.491909in}{3.218381in}}%
\pgfpathlineto{\pgfqpoint{5.484630in}{3.213289in}}%
\pgfpathlineto{\pgfqpoint{5.477341in}{3.208034in}}%
\pgfpathclose%
\pgfusepath{fill}%
\end{pgfscope}%
\begin{pgfscope}%
\pgfpathrectangle{\pgfqpoint{1.254980in}{0.150000in}}{\pgfqpoint{5.490039in}{5.490039in}}%
\pgfusepath{clip}%
\pgfsetbuttcap%
\pgfsetroundjoin%
\definecolor{currentfill}{rgb}{0.239374,0.735588,0.455688}%
\pgfsetfillcolor{currentfill}%
\pgfsetfillopacity{0.700000}%
\pgfsetlinewidth{0.000000pt}%
\definecolor{currentstroke}{rgb}{0.000000,0.000000,0.000000}%
\pgfsetstrokecolor{currentstroke}%
\pgfsetdash{}{0pt}%
\pgfpathmoveto{\pgfqpoint{5.273280in}{3.006537in}}%
\pgfpathlineto{\pgfqpoint{5.287677in}{3.021575in}}%
\pgfpathlineto{\pgfqpoint{5.302095in}{3.036778in}}%
\pgfpathlineto{\pgfqpoint{5.316534in}{3.052147in}}%
\pgfpathlineto{\pgfqpoint{5.330994in}{3.067682in}}%
\pgfpathlineto{\pgfqpoint{5.338416in}{3.075155in}}%
\pgfpathlineto{\pgfqpoint{5.345828in}{3.082450in}}%
\pgfpathlineto{\pgfqpoint{5.353230in}{3.089568in}}%
\pgfpathlineto{\pgfqpoint{5.360622in}{3.096511in}}%
\pgfpathlineto{\pgfqpoint{5.346170in}{3.081088in}}%
\pgfpathlineto{\pgfqpoint{5.331738in}{3.065831in}}%
\pgfpathlineto{\pgfqpoint{5.317328in}{3.050739in}}%
\pgfpathlineto{\pgfqpoint{5.302938in}{3.035812in}}%
\pgfpathlineto{\pgfqpoint{5.295538in}{3.028746in}}%
\pgfpathlineto{\pgfqpoint{5.288128in}{3.021512in}}%
\pgfpathlineto{\pgfqpoint{5.280709in}{3.014110in}}%
\pgfpathlineto{\pgfqpoint{5.273280in}{3.006537in}}%
\pgfpathclose%
\pgfusepath{fill}%
\end{pgfscope}%
\begin{pgfscope}%
\pgfpathrectangle{\pgfqpoint{1.254980in}{0.150000in}}{\pgfqpoint{5.490039in}{5.490039in}}%
\pgfusepath{clip}%
\pgfsetbuttcap%
\pgfsetroundjoin%
\definecolor{currentfill}{rgb}{0.283072,0.130895,0.449241}%
\pgfsetfillcolor{currentfill}%
\pgfsetfillopacity{0.700000}%
\pgfsetlinewidth{0.000000pt}%
\definecolor{currentstroke}{rgb}{0.000000,0.000000,0.000000}%
\pgfsetstrokecolor{currentstroke}%
\pgfsetdash{}{0pt}%
\pgfpathmoveto{\pgfqpoint{3.954126in}{1.481524in}}%
\pgfpathlineto{\pgfqpoint{3.967761in}{1.484167in}}%
\pgfpathlineto{\pgfqpoint{3.981405in}{1.486967in}}%
\pgfpathlineto{\pgfqpoint{3.995059in}{1.489924in}}%
\pgfpathlineto{\pgfqpoint{4.008724in}{1.493038in}}%
\pgfpathlineto{\pgfqpoint{4.016646in}{1.506381in}}%
\pgfpathlineto{\pgfqpoint{4.024564in}{1.519784in}}%
\pgfpathlineto{\pgfqpoint{4.032478in}{1.533243in}}%
\pgfpathlineto{\pgfqpoint{4.040387in}{1.546752in}}%
\pgfpathlineto{\pgfqpoint{4.026726in}{1.543132in}}%
\pgfpathlineto{\pgfqpoint{4.013076in}{1.539669in}}%
\pgfpathlineto{\pgfqpoint{3.999436in}{1.536364in}}%
\pgfpathlineto{\pgfqpoint{3.985807in}{1.533216in}}%
\pgfpathlineto{\pgfqpoint{3.977894in}{1.520202in}}%
\pgfpathlineto{\pgfqpoint{3.969976in}{1.507246in}}%
\pgfpathlineto{\pgfqpoint{3.962053in}{1.494351in}}%
\pgfpathlineto{\pgfqpoint{3.954126in}{1.481524in}}%
\pgfpathclose%
\pgfusepath{fill}%
\end{pgfscope}%
\begin{pgfscope}%
\pgfpathrectangle{\pgfqpoint{1.254980in}{0.150000in}}{\pgfqpoint{5.490039in}{5.490039in}}%
\pgfusepath{clip}%
\pgfsetbuttcap%
\pgfsetroundjoin%
\definecolor{currentfill}{rgb}{0.268510,0.009605,0.335427}%
\pgfsetfillcolor{currentfill}%
\pgfsetfillopacity{0.700000}%
\pgfsetlinewidth{0.000000pt}%
\definecolor{currentstroke}{rgb}{0.000000,0.000000,0.000000}%
\pgfsetstrokecolor{currentstroke}%
\pgfsetdash{}{0pt}%
\pgfpathmoveto{\pgfqpoint{3.381349in}{1.298875in}}%
\pgfpathlineto{\pgfqpoint{3.394890in}{1.293436in}}%
\pgfpathlineto{\pgfqpoint{3.408435in}{1.288163in}}%
\pgfpathlineto{\pgfqpoint{3.421983in}{1.283054in}}%
\pgfpathlineto{\pgfqpoint{3.435534in}{1.278110in}}%
\pgfpathlineto{\pgfqpoint{3.443718in}{1.283612in}}%
\pgfpathlineto{\pgfqpoint{3.451892in}{1.289362in}}%
\pgfpathlineto{\pgfqpoint{3.460055in}{1.295353in}}%
\pgfpathlineto{\pgfqpoint{3.468208in}{1.301579in}}%
\pgfpathlineto{\pgfqpoint{3.454681in}{1.305855in}}%
\pgfpathlineto{\pgfqpoint{3.441158in}{1.310295in}}%
\pgfpathlineto{\pgfqpoint{3.427639in}{1.314900in}}%
\pgfpathlineto{\pgfqpoint{3.414124in}{1.319670in}}%
\pgfpathlineto{\pgfqpoint{3.405946in}{1.314102in}}%
\pgfpathlineto{\pgfqpoint{3.397758in}{1.308775in}}%
\pgfpathlineto{\pgfqpoint{3.389559in}{1.303697in}}%
\pgfpathlineto{\pgfqpoint{3.381349in}{1.298875in}}%
\pgfpathclose%
\pgfusepath{fill}%
\end{pgfscope}%
\begin{pgfscope}%
\pgfpathrectangle{\pgfqpoint{1.254980in}{0.150000in}}{\pgfqpoint{5.490039in}{5.490039in}}%
\pgfusepath{clip}%
\pgfsetbuttcap%
\pgfsetroundjoin%
\definecolor{currentfill}{rgb}{0.266580,0.228262,0.514349}%
\pgfsetfillcolor{currentfill}%
\pgfsetfillopacity{0.700000}%
\pgfsetlinewidth{0.000000pt}%
\definecolor{currentstroke}{rgb}{0.000000,0.000000,0.000000}%
\pgfsetstrokecolor{currentstroke}%
\pgfsetdash{}{0pt}%
\pgfpathmoveto{\pgfqpoint{4.158239in}{1.675875in}}%
\pgfpathlineto{\pgfqpoint{4.171952in}{1.681213in}}%
\pgfpathlineto{\pgfqpoint{4.185677in}{1.686709in}}%
\pgfpathlineto{\pgfqpoint{4.199414in}{1.692363in}}%
\pgfpathlineto{\pgfqpoint{4.213163in}{1.698174in}}%
\pgfpathlineto{\pgfqpoint{4.221033in}{1.712806in}}%
\pgfpathlineto{\pgfqpoint{4.228898in}{1.727432in}}%
\pgfpathlineto{\pgfqpoint{4.236760in}{1.742047in}}%
\pgfpathlineto{\pgfqpoint{4.244617in}{1.756648in}}%
\pgfpathlineto{\pgfqpoint{4.230867in}{1.750411in}}%
\pgfpathlineto{\pgfqpoint{4.217130in}{1.744332in}}%
\pgfpathlineto{\pgfqpoint{4.203404in}{1.738411in}}%
\pgfpathlineto{\pgfqpoint{4.189692in}{1.732649in}}%
\pgfpathlineto{\pgfqpoint{4.181835in}{1.718463in}}%
\pgfpathlineto{\pgfqpoint{4.173974in}{1.704269in}}%
\pgfpathlineto{\pgfqpoint{4.166108in}{1.690072in}}%
\pgfpathlineto{\pgfqpoint{4.158239in}{1.675875in}}%
\pgfpathclose%
\pgfusepath{fill}%
\end{pgfscope}%
\begin{pgfscope}%
\pgfpathrectangle{\pgfqpoint{1.254980in}{0.150000in}}{\pgfqpoint{5.490039in}{5.490039in}}%
\pgfusepath{clip}%
\pgfsetbuttcap%
\pgfsetroundjoin%
\definecolor{currentfill}{rgb}{0.119423,0.611141,0.538982}%
\pgfsetfillcolor{currentfill}%
\pgfsetfillopacity{0.700000}%
\pgfsetlinewidth{0.000000pt}%
\definecolor{currentstroke}{rgb}{0.000000,0.000000,0.000000}%
\pgfsetstrokecolor{currentstroke}%
\pgfsetdash{}{0pt}%
\pgfpathmoveto{\pgfqpoint{4.951392in}{2.649092in}}%
\pgfpathlineto{\pgfqpoint{4.965566in}{2.662235in}}%
\pgfpathlineto{\pgfqpoint{4.979760in}{2.675543in}}%
\pgfpathlineto{\pgfqpoint{4.993972in}{2.689014in}}%
\pgfpathlineto{\pgfqpoint{5.008203in}{2.702650in}}%
\pgfpathlineto{\pgfqpoint{5.015813in}{2.713708in}}%
\pgfpathlineto{\pgfqpoint{5.023414in}{2.724599in}}%
\pgfpathlineto{\pgfqpoint{5.031008in}{2.735322in}}%
\pgfpathlineto{\pgfqpoint{5.038594in}{2.745876in}}%
\pgfpathlineto{\pgfqpoint{5.024363in}{2.732194in}}%
\pgfpathlineto{\pgfqpoint{5.010151in}{2.718676in}}%
\pgfpathlineto{\pgfqpoint{4.995958in}{2.705321in}}%
\pgfpathlineto{\pgfqpoint{4.981784in}{2.692131in}}%
\pgfpathlineto{\pgfqpoint{4.974197in}{2.681612in}}%
\pgfpathlineto{\pgfqpoint{4.966603in}{2.670932in}}%
\pgfpathlineto{\pgfqpoint{4.959001in}{2.660092in}}%
\pgfpathlineto{\pgfqpoint{4.951392in}{2.649092in}}%
\pgfpathclose%
\pgfusepath{fill}%
\end{pgfscope}%
\begin{pgfscope}%
\pgfpathrectangle{\pgfqpoint{1.254980in}{0.150000in}}{\pgfqpoint{5.490039in}{5.490039in}}%
\pgfusepath{clip}%
\pgfsetbuttcap%
\pgfsetroundjoin%
\definecolor{currentfill}{rgb}{0.268510,0.009605,0.335427}%
\pgfsetfillcolor{currentfill}%
\pgfsetfillopacity{0.700000}%
\pgfsetlinewidth{0.000000pt}%
\definecolor{currentstroke}{rgb}{0.000000,0.000000,0.000000}%
\pgfsetstrokecolor{currentstroke}%
\pgfsetdash{}{0pt}%
\pgfpathmoveto{\pgfqpoint{3.522362in}{1.286105in}}%
\pgfpathlineto{\pgfqpoint{3.535912in}{1.282642in}}%
\pgfpathlineto{\pgfqpoint{3.549467in}{1.279340in}}%
\pgfpathlineto{\pgfqpoint{3.563027in}{1.276200in}}%
\pgfpathlineto{\pgfqpoint{3.576592in}{1.273220in}}%
\pgfpathlineto{\pgfqpoint{3.584694in}{1.280975in}}%
\pgfpathlineto{\pgfqpoint{3.592787in}{1.288936in}}%
\pgfpathlineto{\pgfqpoint{3.600872in}{1.297098in}}%
\pgfpathlineto{\pgfqpoint{3.608949in}{1.305455in}}%
\pgfpathlineto{\pgfqpoint{3.595402in}{1.307795in}}%
\pgfpathlineto{\pgfqpoint{3.581861in}{1.310295in}}%
\pgfpathlineto{\pgfqpoint{3.568326in}{1.312957in}}%
\pgfpathlineto{\pgfqpoint{3.554796in}{1.315780in}}%
\pgfpathlineto{\pgfqpoint{3.546700in}{1.308053in}}%
\pgfpathlineto{\pgfqpoint{3.538596in}{1.300527in}}%
\pgfpathlineto{\pgfqpoint{3.530483in}{1.293209in}}%
\pgfpathlineto{\pgfqpoint{3.522362in}{1.286105in}}%
\pgfpathclose%
\pgfusepath{fill}%
\end{pgfscope}%
\begin{pgfscope}%
\pgfpathrectangle{\pgfqpoint{1.254980in}{0.150000in}}{\pgfqpoint{5.490039in}{5.490039in}}%
\pgfusepath{clip}%
\pgfsetbuttcap%
\pgfsetroundjoin%
\definecolor{currentfill}{rgb}{0.273809,0.031497,0.358853}%
\pgfsetfillcolor{currentfill}%
\pgfsetfillopacity{0.700000}%
\pgfsetlinewidth{0.000000pt}%
\definecolor{currentstroke}{rgb}{0.000000,0.000000,0.000000}%
\pgfsetstrokecolor{currentstroke}%
\pgfsetdash{}{0pt}%
\pgfpathmoveto{\pgfqpoint{3.239929in}{1.337377in}}%
\pgfpathlineto{\pgfqpoint{3.253480in}{1.329900in}}%
\pgfpathlineto{\pgfqpoint{3.267033in}{1.322592in}}%
\pgfpathlineto{\pgfqpoint{3.280588in}{1.315453in}}%
\pgfpathlineto{\pgfqpoint{3.294145in}{1.308483in}}%
\pgfpathlineto{\pgfqpoint{3.302431in}{1.311505in}}%
\pgfpathlineto{\pgfqpoint{3.310705in}{1.314819in}}%
\pgfpathlineto{\pgfqpoint{3.318967in}{1.318416in}}%
\pgfpathlineto{\pgfqpoint{3.327217in}{1.322289in}}%
\pgfpathlineto{\pgfqpoint{3.313690in}{1.328560in}}%
\pgfpathlineto{\pgfqpoint{3.300167in}{1.335000in}}%
\pgfpathlineto{\pgfqpoint{3.286645in}{1.341608in}}%
\pgfpathlineto{\pgfqpoint{3.273126in}{1.348386in}}%
\pgfpathlineto{\pgfqpoint{3.264847in}{1.345201in}}%
\pgfpathlineto{\pgfqpoint{3.256554in}{1.342299in}}%
\pgfpathlineto{\pgfqpoint{3.248248in}{1.339689in}}%
\pgfpathlineto{\pgfqpoint{3.239929in}{1.337377in}}%
\pgfpathclose%
\pgfusepath{fill}%
\end{pgfscope}%
\begin{pgfscope}%
\pgfpathrectangle{\pgfqpoint{1.254980in}{0.150000in}}{\pgfqpoint{5.490039in}{5.490039in}}%
\pgfusepath{clip}%
\pgfsetbuttcap%
\pgfsetroundjoin%
\definecolor{currentfill}{rgb}{0.129933,0.559582,0.551864}%
\pgfsetfillcolor{currentfill}%
\pgfsetfillopacity{0.700000}%
\pgfsetlinewidth{0.000000pt}%
\definecolor{currentstroke}{rgb}{0.000000,0.000000,0.000000}%
\pgfsetstrokecolor{currentstroke}%
\pgfsetdash{}{0pt}%
\pgfpathmoveto{\pgfqpoint{4.833746in}{2.505155in}}%
\pgfpathlineto{\pgfqpoint{4.847846in}{2.517460in}}%
\pgfpathlineto{\pgfqpoint{4.861963in}{2.529928in}}%
\pgfpathlineto{\pgfqpoint{4.876099in}{2.542559in}}%
\pgfpathlineto{\pgfqpoint{4.890253in}{2.555353in}}%
\pgfpathlineto{\pgfqpoint{4.897920in}{2.567623in}}%
\pgfpathlineto{\pgfqpoint{4.905580in}{2.579737in}}%
\pgfpathlineto{\pgfqpoint{4.913234in}{2.591693in}}%
\pgfpathlineto{\pgfqpoint{4.920880in}{2.603491in}}%
\pgfpathlineto{\pgfqpoint{4.906724in}{2.590588in}}%
\pgfpathlineto{\pgfqpoint{4.892587in}{2.577849in}}%
\pgfpathlineto{\pgfqpoint{4.878468in}{2.565273in}}%
\pgfpathlineto{\pgfqpoint{4.864367in}{2.552859in}}%
\pgfpathlineto{\pgfqpoint{4.856722in}{2.541158in}}%
\pgfpathlineto{\pgfqpoint{4.849070in}{2.529306in}}%
\pgfpathlineto{\pgfqpoint{4.841412in}{2.517305in}}%
\pgfpathlineto{\pgfqpoint{4.833746in}{2.505155in}}%
\pgfpathclose%
\pgfusepath{fill}%
\end{pgfscope}%
\begin{pgfscope}%
\pgfpathrectangle{\pgfqpoint{1.254980in}{0.150000in}}{\pgfqpoint{5.490039in}{5.490039in}}%
\pgfusepath{clip}%
\pgfsetbuttcap%
\pgfsetroundjoin%
\definecolor{currentfill}{rgb}{0.150476,0.504369,0.557430}%
\pgfsetfillcolor{currentfill}%
\pgfsetfillopacity{0.700000}%
\pgfsetlinewidth{0.000000pt}%
\definecolor{currentstroke}{rgb}{0.000000,0.000000,0.000000}%
\pgfsetstrokecolor{currentstroke}%
\pgfsetdash{}{0pt}%
\pgfpathmoveto{\pgfqpoint{4.715986in}{2.356329in}}%
\pgfpathlineto{\pgfqpoint{4.730011in}{2.367678in}}%
\pgfpathlineto{\pgfqpoint{4.744054in}{2.379188in}}%
\pgfpathlineto{\pgfqpoint{4.758113in}{2.390860in}}%
\pgfpathlineto{\pgfqpoint{4.772190in}{2.402695in}}%
\pgfpathlineto{\pgfqpoint{4.779906in}{2.416003in}}%
\pgfpathlineto{\pgfqpoint{4.787617in}{2.429171in}}%
\pgfpathlineto{\pgfqpoint{4.795321in}{2.442196in}}%
\pgfpathlineto{\pgfqpoint{4.803019in}{2.455079in}}%
\pgfpathlineto{\pgfqpoint{4.788939in}{2.443075in}}%
\pgfpathlineto{\pgfqpoint{4.774877in}{2.431234in}}%
\pgfpathlineto{\pgfqpoint{4.760832in}{2.419554in}}%
\pgfpathlineto{\pgfqpoint{4.746804in}{2.408037in}}%
\pgfpathlineto{\pgfqpoint{4.739108in}{2.395312in}}%
\pgfpathlineto{\pgfqpoint{4.731407in}{2.382452in}}%
\pgfpathlineto{\pgfqpoint{4.723699in}{2.369457in}}%
\pgfpathlineto{\pgfqpoint{4.715986in}{2.356329in}}%
\pgfpathclose%
\pgfusepath{fill}%
\end{pgfscope}%
\begin{pgfscope}%
\pgfpathrectangle{\pgfqpoint{1.254980in}{0.150000in}}{\pgfqpoint{5.490039in}{5.490039in}}%
\pgfusepath{clip}%
\pgfsetbuttcap%
\pgfsetroundjoin%
\definecolor{currentfill}{rgb}{0.280255,0.165693,0.476498}%
\pgfsetfillcolor{currentfill}%
\pgfsetfillopacity{0.700000}%
\pgfsetlinewidth{0.000000pt}%
\definecolor{currentstroke}{rgb}{0.000000,0.000000,0.000000}%
\pgfsetstrokecolor{currentstroke}%
\pgfsetdash{}{0pt}%
\pgfpathmoveto{\pgfqpoint{4.040387in}{1.546752in}}%
\pgfpathlineto{\pgfqpoint{4.054058in}{1.550529in}}%
\pgfpathlineto{\pgfqpoint{4.067740in}{1.554463in}}%
\pgfpathlineto{\pgfqpoint{4.081433in}{1.558555in}}%
\pgfpathlineto{\pgfqpoint{4.095136in}{1.562803in}}%
\pgfpathlineto{\pgfqpoint{4.103039in}{1.576848in}}%
\pgfpathlineto{\pgfqpoint{4.110937in}{1.590927in}}%
\pgfpathlineto{\pgfqpoint{4.118831in}{1.605036in}}%
\pgfpathlineto{\pgfqpoint{4.126721in}{1.619170in}}%
\pgfpathlineto{\pgfqpoint{4.113019in}{1.614442in}}%
\pgfpathlineto{\pgfqpoint{4.099328in}{1.609871in}}%
\pgfpathlineto{\pgfqpoint{4.085648in}{1.605457in}}%
\pgfpathlineto{\pgfqpoint{4.071980in}{1.601201in}}%
\pgfpathlineto{\pgfqpoint{4.064088in}{1.587536in}}%
\pgfpathlineto{\pgfqpoint{4.056192in}{1.573903in}}%
\pgfpathlineto{\pgfqpoint{4.048292in}{1.560307in}}%
\pgfpathlineto{\pgfqpoint{4.040387in}{1.546752in}}%
\pgfpathclose%
\pgfusepath{fill}%
\end{pgfscope}%
\begin{pgfscope}%
\pgfpathrectangle{\pgfqpoint{1.254980in}{0.150000in}}{\pgfqpoint{5.490039in}{5.490039in}}%
\pgfusepath{clip}%
\pgfsetbuttcap%
\pgfsetroundjoin%
\definecolor{currentfill}{rgb}{0.171176,0.452530,0.557965}%
\pgfsetfillcolor{currentfill}%
\pgfsetfillopacity{0.700000}%
\pgfsetlinewidth{0.000000pt}%
\definecolor{currentstroke}{rgb}{0.000000,0.000000,0.000000}%
\pgfsetstrokecolor{currentstroke}%
\pgfsetdash{}{0pt}%
\pgfpathmoveto{\pgfqpoint{4.598165in}{2.204631in}}%
\pgfpathlineto{\pgfqpoint{4.612118in}{2.214907in}}%
\pgfpathlineto{\pgfqpoint{4.626087in}{2.225345in}}%
\pgfpathlineto{\pgfqpoint{4.640072in}{2.235943in}}%
\pgfpathlineto{\pgfqpoint{4.654073in}{2.246702in}}%
\pgfpathlineto{\pgfqpoint{4.661832in}{2.260837in}}%
\pgfpathlineto{\pgfqpoint{4.669585in}{2.274853in}}%
\pgfpathlineto{\pgfqpoint{4.677332in}{2.288746in}}%
\pgfpathlineto{\pgfqpoint{4.685074in}{2.302517in}}%
\pgfpathlineto{\pgfqpoint{4.671069in}{2.291529in}}%
\pgfpathlineto{\pgfqpoint{4.657080in}{2.280702in}}%
\pgfpathlineto{\pgfqpoint{4.643108in}{2.270036in}}%
\pgfpathlineto{\pgfqpoint{4.629152in}{2.259532in}}%
\pgfpathlineto{\pgfqpoint{4.621413in}{2.245979in}}%
\pgfpathlineto{\pgfqpoint{4.613669in}{2.232310in}}%
\pgfpathlineto{\pgfqpoint{4.605920in}{2.218526in}}%
\pgfpathlineto{\pgfqpoint{4.598165in}{2.204631in}}%
\pgfpathclose%
\pgfusepath{fill}%
\end{pgfscope}%
\begin{pgfscope}%
\pgfpathrectangle{\pgfqpoint{1.254980in}{0.150000in}}{\pgfqpoint{5.490039in}{5.490039in}}%
\pgfusepath{clip}%
\pgfsetbuttcap%
\pgfsetroundjoin%
\definecolor{currentfill}{rgb}{0.197636,0.391528,0.554969}%
\pgfsetfillcolor{currentfill}%
\pgfsetfillopacity{0.700000}%
\pgfsetlinewidth{0.000000pt}%
\definecolor{currentstroke}{rgb}{0.000000,0.000000,0.000000}%
\pgfsetstrokecolor{currentstroke}%
\pgfsetdash{}{0pt}%
\pgfpathmoveto{\pgfqpoint{4.480321in}{2.052357in}}%
\pgfpathlineto{\pgfqpoint{4.494205in}{2.061449in}}%
\pgfpathlineto{\pgfqpoint{4.508104in}{2.070701in}}%
\pgfpathlineto{\pgfqpoint{4.522018in}{2.080113in}}%
\pgfpathlineto{\pgfqpoint{4.535948in}{2.089684in}}%
\pgfpathlineto{\pgfqpoint{4.543742in}{2.104399in}}%
\pgfpathlineto{\pgfqpoint{4.551531in}{2.119019in}}%
\pgfpathlineto{\pgfqpoint{4.559316in}{2.133543in}}%
\pgfpathlineto{\pgfqpoint{4.567096in}{2.147968in}}%
\pgfpathlineto{\pgfqpoint{4.553162in}{2.138110in}}%
\pgfpathlineto{\pgfqpoint{4.539245in}{2.128411in}}%
\pgfpathlineto{\pgfqpoint{4.525342in}{2.118874in}}%
\pgfpathlineto{\pgfqpoint{4.511455in}{2.109496in}}%
\pgfpathlineto{\pgfqpoint{4.503678in}{2.095346in}}%
\pgfpathlineto{\pgfqpoint{4.495897in}{2.081105in}}%
\pgfpathlineto{\pgfqpoint{4.488112in}{2.066774in}}%
\pgfpathlineto{\pgfqpoint{4.480321in}{2.052357in}}%
\pgfpathclose%
\pgfusepath{fill}%
\end{pgfscope}%
\begin{pgfscope}%
\pgfpathrectangle{\pgfqpoint{1.254980in}{0.150000in}}{\pgfqpoint{5.490039in}{5.490039in}}%
\pgfusepath{clip}%
\pgfsetbuttcap%
\pgfsetroundjoin%
\definecolor{currentfill}{rgb}{0.170948,0.694384,0.493803}%
\pgfsetfillcolor{currentfill}%
\pgfsetfillopacity{0.700000}%
\pgfsetlinewidth{0.000000pt}%
\definecolor{currentstroke}{rgb}{0.000000,0.000000,0.000000}%
\pgfsetstrokecolor{currentstroke}%
\pgfsetdash{}{0pt}%
\pgfpathmoveto{\pgfqpoint{5.156087in}{2.880651in}}%
\pgfpathlineto{\pgfqpoint{5.170412in}{2.895156in}}%
\pgfpathlineto{\pgfqpoint{5.184758in}{2.909825in}}%
\pgfpathlineto{\pgfqpoint{5.199123in}{2.924659in}}%
\pgfpathlineto{\pgfqpoint{5.213509in}{2.939659in}}%
\pgfpathlineto{\pgfqpoint{5.221013in}{2.948641in}}%
\pgfpathlineto{\pgfqpoint{5.228508in}{2.957443in}}%
\pgfpathlineto{\pgfqpoint{5.235994in}{2.966066in}}%
\pgfpathlineto{\pgfqpoint{5.243470in}{2.974512in}}%
\pgfpathlineto{\pgfqpoint{5.229088in}{2.959560in}}%
\pgfpathlineto{\pgfqpoint{5.214727in}{2.944773in}}%
\pgfpathlineto{\pgfqpoint{5.200386in}{2.930151in}}%
\pgfpathlineto{\pgfqpoint{5.186065in}{2.915694in}}%
\pgfpathlineto{\pgfqpoint{5.178584in}{2.907189in}}%
\pgfpathlineto{\pgfqpoint{5.171094in}{2.898515in}}%
\pgfpathlineto{\pgfqpoint{5.163595in}{2.889669in}}%
\pgfpathlineto{\pgfqpoint{5.156087in}{2.880651in}}%
\pgfpathclose%
\pgfusepath{fill}%
\end{pgfscope}%
\begin{pgfscope}%
\pgfpathrectangle{\pgfqpoint{1.254980in}{0.150000in}}{\pgfqpoint{5.490039in}{5.490039in}}%
\pgfusepath{clip}%
\pgfsetbuttcap%
\pgfsetroundjoin%
\definecolor{currentfill}{rgb}{0.223925,0.334994,0.548053}%
\pgfsetfillcolor{currentfill}%
\pgfsetfillopacity{0.700000}%
\pgfsetlinewidth{0.000000pt}%
\definecolor{currentstroke}{rgb}{0.000000,0.000000,0.000000}%
\pgfsetstrokecolor{currentstroke}%
\pgfsetdash{}{0pt}%
\pgfpathmoveto{\pgfqpoint{4.362474in}{1.902081in}}%
\pgfpathlineto{\pgfqpoint{4.376294in}{1.909879in}}%
\pgfpathlineto{\pgfqpoint{4.390127in}{1.917835in}}%
\pgfpathlineto{\pgfqpoint{4.403975in}{1.925950in}}%
\pgfpathlineto{\pgfqpoint{4.417838in}{1.934224in}}%
\pgfpathlineto{\pgfqpoint{4.425663in}{1.949235in}}%
\pgfpathlineto{\pgfqpoint{4.433485in}{1.964183in}}%
\pgfpathlineto{\pgfqpoint{4.441302in}{1.979064in}}%
\pgfpathlineto{\pgfqpoint{4.449115in}{1.993874in}}%
\pgfpathlineto{\pgfqpoint{4.435249in}{1.985256in}}%
\pgfpathlineto{\pgfqpoint{4.421398in}{1.976798in}}%
\pgfpathlineto{\pgfqpoint{4.407561in}{1.968499in}}%
\pgfpathlineto{\pgfqpoint{4.393738in}{1.960359in}}%
\pgfpathlineto{\pgfqpoint{4.385928in}{1.945881in}}%
\pgfpathlineto{\pgfqpoint{4.378114in}{1.931340in}}%
\pgfpathlineto{\pgfqpoint{4.370296in}{1.916739in}}%
\pgfpathlineto{\pgfqpoint{4.362474in}{1.902081in}}%
\pgfpathclose%
\pgfusepath{fill}%
\end{pgfscope}%
\begin{pgfscope}%
\pgfpathrectangle{\pgfqpoint{1.254980in}{0.150000in}}{\pgfqpoint{5.490039in}{5.490039in}}%
\pgfusepath{clip}%
\pgfsetbuttcap%
\pgfsetroundjoin%
\definecolor{currentfill}{rgb}{0.468053,0.818921,0.323998}%
\pgfsetfillcolor{currentfill}%
\pgfsetfillopacity{0.700000}%
\pgfsetlinewidth{0.000000pt}%
\definecolor{currentstroke}{rgb}{0.000000,0.000000,0.000000}%
\pgfsetstrokecolor{currentstroke}%
\pgfsetdash{}{0pt}%
\pgfpathmoveto{\pgfqpoint{5.564677in}{3.291979in}}%
\pgfpathlineto{\pgfqpoint{5.579293in}{3.308368in}}%
\pgfpathlineto{\pgfqpoint{5.593932in}{3.324923in}}%
\pgfpathlineto{\pgfqpoint{5.608593in}{3.341646in}}%
\pgfpathlineto{\pgfqpoint{5.623277in}{3.358535in}}%
\pgfpathlineto{\pgfqpoint{5.630496in}{3.362695in}}%
\pgfpathlineto{\pgfqpoint{5.637703in}{3.366687in}}%
\pgfpathlineto{\pgfqpoint{5.644899in}{3.370513in}}%
\pgfpathlineto{\pgfqpoint{5.652084in}{3.374178in}}%
\pgfpathlineto{\pgfqpoint{5.637415in}{3.357531in}}%
\pgfpathlineto{\pgfqpoint{5.622769in}{3.341051in}}%
\pgfpathlineto{\pgfqpoint{5.608145in}{3.324738in}}%
\pgfpathlineto{\pgfqpoint{5.593543in}{3.308591in}}%
\pgfpathlineto{\pgfqpoint{5.586343in}{3.304672in}}%
\pgfpathlineto{\pgfqpoint{5.579132in}{3.300600in}}%
\pgfpathlineto{\pgfqpoint{5.571910in}{3.296370in}}%
\pgfpathlineto{\pgfqpoint{5.564677in}{3.291979in}}%
\pgfpathclose%
\pgfusepath{fill}%
\end{pgfscope}%
\begin{pgfscope}%
\pgfpathrectangle{\pgfqpoint{1.254980in}{0.150000in}}{\pgfqpoint{5.490039in}{5.490039in}}%
\pgfusepath{clip}%
\pgfsetbuttcap%
\pgfsetroundjoin%
\definecolor{currentfill}{rgb}{0.130067,0.651384,0.521608}%
\pgfsetfillcolor{currentfill}%
\pgfsetfillopacity{0.700000}%
\pgfsetlinewidth{0.000000pt}%
\definecolor{currentstroke}{rgb}{0.000000,0.000000,0.000000}%
\pgfsetstrokecolor{currentstroke}%
\pgfsetdash{}{0pt}%
\pgfpathmoveto{\pgfqpoint{1.976664in}{2.887660in}}%
\pgfpathlineto{\pgfqpoint{1.990881in}{2.859445in}}%
\pgfpathlineto{\pgfqpoint{2.005079in}{2.831549in}}%
\pgfpathlineto{\pgfqpoint{2.019258in}{2.803969in}}%
\pgfpathlineto{\pgfqpoint{2.033418in}{2.776701in}}%
\pgfpathlineto{\pgfqpoint{2.042977in}{2.762560in}}%
\pgfpathlineto{\pgfqpoint{2.052502in}{2.748911in}}%
\pgfpathlineto{\pgfqpoint{2.061993in}{2.735744in}}%
\pgfpathlineto{\pgfqpoint{2.071452in}{2.723050in}}%
\pgfpathlineto{\pgfqpoint{2.057373in}{2.749517in}}%
\pgfpathlineto{\pgfqpoint{2.043276in}{2.776294in}}%
\pgfpathlineto{\pgfqpoint{2.029160in}{2.803384in}}%
\pgfpathlineto{\pgfqpoint{2.015027in}{2.830791in}}%
\pgfpathlineto{\pgfqpoint{2.005488in}{2.844273in}}%
\pgfpathlineto{\pgfqpoint{1.995915in}{2.858240in}}%
\pgfpathlineto{\pgfqpoint{1.986307in}{2.872699in}}%
\pgfpathlineto{\pgfqpoint{1.976664in}{2.887660in}}%
\pgfpathclose%
\pgfusepath{fill}%
\end{pgfscope}%
\begin{pgfscope}%
\pgfpathrectangle{\pgfqpoint{1.254980in}{0.150000in}}{\pgfqpoint{5.490039in}{5.490039in}}%
\pgfusepath{clip}%
\pgfsetbuttcap%
\pgfsetroundjoin%
\definecolor{currentfill}{rgb}{0.252194,0.269783,0.531579}%
\pgfsetfillcolor{currentfill}%
\pgfsetfillopacity{0.700000}%
\pgfsetlinewidth{0.000000pt}%
\definecolor{currentstroke}{rgb}{0.000000,0.000000,0.000000}%
\pgfsetstrokecolor{currentstroke}%
\pgfsetdash{}{0pt}%
\pgfpathmoveto{\pgfqpoint{4.244617in}{1.756648in}}%
\pgfpathlineto{\pgfqpoint{4.258380in}{1.763042in}}%
\pgfpathlineto{\pgfqpoint{4.272155in}{1.769595in}}%
\pgfpathlineto{\pgfqpoint{4.285943in}{1.776306in}}%
\pgfpathlineto{\pgfqpoint{4.299745in}{1.783174in}}%
\pgfpathlineto{\pgfqpoint{4.307600in}{1.798165in}}%
\pgfpathlineto{\pgfqpoint{4.315451in}{1.813127in}}%
\pgfpathlineto{\pgfqpoint{4.323299in}{1.828056in}}%
\pgfpathlineto{\pgfqpoint{4.331142in}{1.842948in}}%
\pgfpathlineto{\pgfqpoint{4.317338in}{1.835680in}}%
\pgfpathlineto{\pgfqpoint{4.303548in}{1.828571in}}%
\pgfpathlineto{\pgfqpoint{4.289771in}{1.821620in}}%
\pgfpathlineto{\pgfqpoint{4.276006in}{1.814827in}}%
\pgfpathlineto{\pgfqpoint{4.268165in}{1.800323in}}%
\pgfpathlineto{\pgfqpoint{4.260320in}{1.785789in}}%
\pgfpathlineto{\pgfqpoint{4.252470in}{1.771229in}}%
\pgfpathlineto{\pgfqpoint{4.244617in}{1.756648in}}%
\pgfpathclose%
\pgfusepath{fill}%
\end{pgfscope}%
\begin{pgfscope}%
\pgfpathrectangle{\pgfqpoint{1.254980in}{0.150000in}}{\pgfqpoint{5.490039in}{5.490039in}}%
\pgfusepath{clip}%
\pgfsetbuttcap%
\pgfsetroundjoin%
\definecolor{currentfill}{rgb}{0.311925,0.767822,0.415586}%
\pgfsetfillcolor{currentfill}%
\pgfsetfillopacity{0.700000}%
\pgfsetlinewidth{0.000000pt}%
\definecolor{currentstroke}{rgb}{0.000000,0.000000,0.000000}%
\pgfsetstrokecolor{currentstroke}%
\pgfsetdash{}{0pt}%
\pgfpathmoveto{\pgfqpoint{5.360622in}{3.096511in}}%
\pgfpathlineto{\pgfqpoint{5.375096in}{3.112100in}}%
\pgfpathlineto{\pgfqpoint{5.389591in}{3.127855in}}%
\pgfpathlineto{\pgfqpoint{5.404108in}{3.143776in}}%
\pgfpathlineto{\pgfqpoint{5.418647in}{3.159864in}}%
\pgfpathlineto{\pgfqpoint{5.426020in}{3.166501in}}%
\pgfpathlineto{\pgfqpoint{5.433384in}{3.172958in}}%
\pgfpathlineto{\pgfqpoint{5.440736in}{3.179237in}}%
\pgfpathlineto{\pgfqpoint{5.448078in}{3.185340in}}%
\pgfpathlineto{\pgfqpoint{5.433549in}{3.169397in}}%
\pgfpathlineto{\pgfqpoint{5.419041in}{3.153620in}}%
\pgfpathlineto{\pgfqpoint{5.404555in}{3.138010in}}%
\pgfpathlineto{\pgfqpoint{5.390090in}{3.122565in}}%
\pgfpathlineto{\pgfqpoint{5.382738in}{3.116305in}}%
\pgfpathlineto{\pgfqpoint{5.375376in}{3.109878in}}%
\pgfpathlineto{\pgfqpoint{5.368004in}{3.103281in}}%
\pgfpathlineto{\pgfqpoint{5.360622in}{3.096511in}}%
\pgfpathclose%
\pgfusepath{fill}%
\end{pgfscope}%
\begin{pgfscope}%
\pgfpathrectangle{\pgfqpoint{1.254980in}{0.150000in}}{\pgfqpoint{5.490039in}{5.490039in}}%
\pgfusepath{clip}%
\pgfsetbuttcap%
\pgfsetroundjoin%
\definecolor{currentfill}{rgb}{0.277018,0.050344,0.375715}%
\pgfsetfillcolor{currentfill}%
\pgfsetfillopacity{0.700000}%
\pgfsetlinewidth{0.000000pt}%
\definecolor{currentstroke}{rgb}{0.000000,0.000000,0.000000}%
\pgfsetstrokecolor{currentstroke}%
\pgfsetdash{}{0pt}%
\pgfpathmoveto{\pgfqpoint{3.749655in}{1.332629in}}%
\pgfpathlineto{\pgfqpoint{3.763247in}{1.332335in}}%
\pgfpathlineto{\pgfqpoint{3.776846in}{1.332198in}}%
\pgfpathlineto{\pgfqpoint{3.790452in}{1.332219in}}%
\pgfpathlineto{\pgfqpoint{3.804067in}{1.332397in}}%
\pgfpathlineto{\pgfqpoint{3.812066in}{1.343420in}}%
\pgfpathlineto{\pgfqpoint{3.820060in}{1.354581in}}%
\pgfpathlineto{\pgfqpoint{3.828047in}{1.365873in}}%
\pgfpathlineto{\pgfqpoint{3.836029in}{1.377292in}}%
\pgfpathlineto{\pgfqpoint{3.822424in}{1.376527in}}%
\pgfpathlineto{\pgfqpoint{3.808828in}{1.375920in}}%
\pgfpathlineto{\pgfqpoint{3.795241in}{1.375471in}}%
\pgfpathlineto{\pgfqpoint{3.781661in}{1.375180in}}%
\pgfpathlineto{\pgfqpoint{3.773669in}{1.364337in}}%
\pgfpathlineto{\pgfqpoint{3.765670in}{1.353627in}}%
\pgfpathlineto{\pgfqpoint{3.757666in}{1.343056in}}%
\pgfpathlineto{\pgfqpoint{3.749655in}{1.332629in}}%
\pgfpathclose%
\pgfusepath{fill}%
\end{pgfscope}%
\begin{pgfscope}%
\pgfpathrectangle{\pgfqpoint{1.254980in}{0.150000in}}{\pgfqpoint{5.490039in}{5.490039in}}%
\pgfusepath{clip}%
\pgfsetbuttcap%
\pgfsetroundjoin%
\definecolor{currentfill}{rgb}{0.268510,0.009605,0.335427}%
\pgfsetfillcolor{currentfill}%
\pgfsetfillopacity{0.700000}%
\pgfsetlinewidth{0.000000pt}%
\definecolor{currentstroke}{rgb}{0.000000,0.000000,0.000000}%
\pgfsetstrokecolor{currentstroke}%
\pgfsetdash{}{0pt}%
\pgfpathmoveto{\pgfqpoint{3.435534in}{1.278110in}}%
\pgfpathlineto{\pgfqpoint{3.449090in}{1.273328in}}%
\pgfpathlineto{\pgfqpoint{3.462649in}{1.268710in}}%
\pgfpathlineto{\pgfqpoint{3.476213in}{1.264255in}}%
\pgfpathlineto{\pgfqpoint{3.489780in}{1.259962in}}%
\pgfpathlineto{\pgfqpoint{3.497940in}{1.266143in}}%
\pgfpathlineto{\pgfqpoint{3.506090in}{1.272566in}}%
\pgfpathlineto{\pgfqpoint{3.514231in}{1.279221in}}%
\pgfpathlineto{\pgfqpoint{3.522362in}{1.286105in}}%
\pgfpathlineto{\pgfqpoint{3.508816in}{1.289729in}}%
\pgfpathlineto{\pgfqpoint{3.495276in}{1.293516in}}%
\pgfpathlineto{\pgfqpoint{3.481740in}{1.297466in}}%
\pgfpathlineto{\pgfqpoint{3.468208in}{1.301579in}}%
\pgfpathlineto{\pgfqpoint{3.460055in}{1.295353in}}%
\pgfpathlineto{\pgfqpoint{3.451892in}{1.289362in}}%
\pgfpathlineto{\pgfqpoint{3.443718in}{1.283612in}}%
\pgfpathlineto{\pgfqpoint{3.435534in}{1.278110in}}%
\pgfpathclose%
\pgfusepath{fill}%
\end{pgfscope}%
\begin{pgfscope}%
\pgfpathrectangle{\pgfqpoint{1.254980in}{0.150000in}}{\pgfqpoint{5.490039in}{5.490039in}}%
\pgfusepath{clip}%
\pgfsetbuttcap%
\pgfsetroundjoin%
\definecolor{currentfill}{rgb}{0.280894,0.078907,0.402329}%
\pgfsetfillcolor{currentfill}%
\pgfsetfillopacity{0.700000}%
\pgfsetlinewidth{0.000000pt}%
\definecolor{currentstroke}{rgb}{0.000000,0.000000,0.000000}%
\pgfsetstrokecolor{currentstroke}%
\pgfsetdash{}{0pt}%
\pgfpathmoveto{\pgfqpoint{3.836029in}{1.377292in}}%
\pgfpathlineto{\pgfqpoint{3.849641in}{1.378214in}}%
\pgfpathlineto{\pgfqpoint{3.863262in}{1.379293in}}%
\pgfpathlineto{\pgfqpoint{3.876892in}{1.380529in}}%
\pgfpathlineto{\pgfqpoint{3.890530in}{1.381922in}}%
\pgfpathlineto{\pgfqpoint{3.898498in}{1.394031in}}%
\pgfpathlineto{\pgfqpoint{3.906460in}{1.406248in}}%
\pgfpathlineto{\pgfqpoint{3.914417in}{1.418568in}}%
\pgfpathlineto{\pgfqpoint{3.922369in}{1.430985in}}%
\pgfpathlineto{\pgfqpoint{3.908738in}{1.429032in}}%
\pgfpathlineto{\pgfqpoint{3.895116in}{1.427236in}}%
\pgfpathlineto{\pgfqpoint{3.881503in}{1.425598in}}%
\pgfpathlineto{\pgfqpoint{3.867899in}{1.424117in}}%
\pgfpathlineto{\pgfqpoint{3.859940in}{1.412249in}}%
\pgfpathlineto{\pgfqpoint{3.851975in}{1.400485in}}%
\pgfpathlineto{\pgfqpoint{3.844004in}{1.388831in}}%
\pgfpathlineto{\pgfqpoint{3.836029in}{1.377292in}}%
\pgfpathclose%
\pgfusepath{fill}%
\end{pgfscope}%
\begin{pgfscope}%
\pgfpathrectangle{\pgfqpoint{1.254980in}{0.150000in}}{\pgfqpoint{5.490039in}{5.490039in}}%
\pgfusepath{clip}%
\pgfsetbuttcap%
\pgfsetroundjoin%
\definecolor{currentfill}{rgb}{0.545524,0.838039,0.275626}%
\pgfsetfillcolor{currentfill}%
\pgfsetfillopacity{0.700000}%
\pgfsetlinewidth{0.000000pt}%
\definecolor{currentstroke}{rgb}{0.000000,0.000000,0.000000}%
\pgfsetstrokecolor{currentstroke}%
\pgfsetdash{}{0pt}%
\pgfpathmoveto{\pgfqpoint{5.652084in}{3.374178in}}%
\pgfpathlineto{\pgfqpoint{5.666776in}{3.390991in}}%
\pgfpathlineto{\pgfqpoint{5.681491in}{3.407972in}}%
\pgfpathlineto{\pgfqpoint{5.696229in}{3.425120in}}%
\pgfpathlineto{\pgfqpoint{5.703390in}{3.428427in}}%
\pgfpathlineto{\pgfqpoint{5.710539in}{3.431572in}}%
\pgfpathlineto{\pgfqpoint{5.717676in}{3.434557in}}%
\pgfpathlineto{\pgfqpoint{5.724802in}{3.437387in}}%
\pgfpathlineto{\pgfqpoint{5.710081in}{3.420516in}}%
\pgfpathlineto{\pgfqpoint{5.695384in}{3.403811in}}%
\pgfpathlineto{\pgfqpoint{5.680709in}{3.387272in}}%
\pgfpathlineto{\pgfqpoint{5.673569in}{3.384227in}}%
\pgfpathlineto{\pgfqpoint{5.666419in}{3.381031in}}%
\pgfpathlineto{\pgfqpoint{5.659257in}{3.377683in}}%
\pgfpathlineto{\pgfqpoint{5.652084in}{3.374178in}}%
\pgfpathclose%
\pgfusepath{fill}%
\end{pgfscope}%
\begin{pgfscope}%
\pgfpathrectangle{\pgfqpoint{1.254980in}{0.150000in}}{\pgfqpoint{5.490039in}{5.490039in}}%
\pgfusepath{clip}%
\pgfsetbuttcap%
\pgfsetroundjoin%
\definecolor{currentfill}{rgb}{0.273809,0.031497,0.358853}%
\pgfsetfillcolor{currentfill}%
\pgfsetfillopacity{0.700000}%
\pgfsetlinewidth{0.000000pt}%
\definecolor{currentstroke}{rgb}{0.000000,0.000000,0.000000}%
\pgfsetstrokecolor{currentstroke}%
\pgfsetdash{}{0pt}%
\pgfpathmoveto{\pgfqpoint{3.663196in}{1.297696in}}%
\pgfpathlineto{\pgfqpoint{3.676773in}{1.296155in}}%
\pgfpathlineto{\pgfqpoint{3.690358in}{1.294772in}}%
\pgfpathlineto{\pgfqpoint{3.703948in}{1.293548in}}%
\pgfpathlineto{\pgfqpoint{3.717546in}{1.292482in}}%
\pgfpathlineto{\pgfqpoint{3.725583in}{1.302273in}}%
\pgfpathlineto{\pgfqpoint{3.733614in}{1.312232in}}%
\pgfpathlineto{\pgfqpoint{3.741638in}{1.322352in}}%
\pgfpathlineto{\pgfqpoint{3.749655in}{1.332629in}}%
\pgfpathlineto{\pgfqpoint{3.736071in}{1.333082in}}%
\pgfpathlineto{\pgfqpoint{3.722494in}{1.333693in}}%
\pgfpathlineto{\pgfqpoint{3.708925in}{1.334463in}}%
\pgfpathlineto{\pgfqpoint{3.695362in}{1.335392in}}%
\pgfpathlineto{\pgfqpoint{3.687331in}{1.325717in}}%
\pgfpathlineto{\pgfqpoint{3.679293in}{1.316206in}}%
\pgfpathlineto{\pgfqpoint{3.671248in}{1.306863in}}%
\pgfpathlineto{\pgfqpoint{3.663196in}{1.297696in}}%
\pgfpathclose%
\pgfusepath{fill}%
\end{pgfscope}%
\begin{pgfscope}%
\pgfpathrectangle{\pgfqpoint{1.254980in}{0.150000in}}{\pgfqpoint{5.490039in}{5.490039in}}%
\pgfusepath{clip}%
\pgfsetbuttcap%
\pgfsetroundjoin%
\definecolor{currentfill}{rgb}{0.272594,0.025563,0.353093}%
\pgfsetfillcolor{currentfill}%
\pgfsetfillopacity{0.700000}%
\pgfsetlinewidth{0.000000pt}%
\definecolor{currentstroke}{rgb}{0.000000,0.000000,0.000000}%
\pgfsetstrokecolor{currentstroke}%
\pgfsetdash{}{0pt}%
\pgfpathmoveto{\pgfqpoint{3.294145in}{1.308483in}}%
\pgfpathlineto{\pgfqpoint{3.307704in}{1.301680in}}%
\pgfpathlineto{\pgfqpoint{3.321265in}{1.295045in}}%
\pgfpathlineto{\pgfqpoint{3.334829in}{1.288577in}}%
\pgfpathlineto{\pgfqpoint{3.348395in}{1.282275in}}%
\pgfpathlineto{\pgfqpoint{3.356651in}{1.286007in}}%
\pgfpathlineto{\pgfqpoint{3.364896in}{1.290022in}}%
\pgfpathlineto{\pgfqpoint{3.373128in}{1.294314in}}%
\pgfpathlineto{\pgfqpoint{3.381349in}{1.298875in}}%
\pgfpathlineto{\pgfqpoint{3.367812in}{1.304479in}}%
\pgfpathlineto{\pgfqpoint{3.354277in}{1.310249in}}%
\pgfpathlineto{\pgfqpoint{3.340745in}{1.316185in}}%
\pgfpathlineto{\pgfqpoint{3.327217in}{1.322289in}}%
\pgfpathlineto{\pgfqpoint{3.318967in}{1.318416in}}%
\pgfpathlineto{\pgfqpoint{3.310705in}{1.314819in}}%
\pgfpathlineto{\pgfqpoint{3.302431in}{1.311505in}}%
\pgfpathlineto{\pgfqpoint{3.294145in}{1.308483in}}%
\pgfpathclose%
\pgfusepath{fill}%
\end{pgfscope}%
\begin{pgfscope}%
\pgfpathrectangle{\pgfqpoint{1.254980in}{0.150000in}}{\pgfqpoint{5.490039in}{5.490039in}}%
\pgfusepath{clip}%
\pgfsetbuttcap%
\pgfsetroundjoin%
\definecolor{currentfill}{rgb}{0.130067,0.651384,0.521608}%
\pgfsetfillcolor{currentfill}%
\pgfsetfillopacity{0.700000}%
\pgfsetlinewidth{0.000000pt}%
\definecolor{currentstroke}{rgb}{0.000000,0.000000,0.000000}%
\pgfsetstrokecolor{currentstroke}%
\pgfsetdash{}{0pt}%
\pgfpathmoveto{\pgfqpoint{5.038594in}{2.745876in}}%
\pgfpathlineto{\pgfqpoint{5.052844in}{2.759722in}}%
\pgfpathlineto{\pgfqpoint{5.067114in}{2.773733in}}%
\pgfpathlineto{\pgfqpoint{5.081403in}{2.787908in}}%
\pgfpathlineto{\pgfqpoint{5.095712in}{2.802248in}}%
\pgfpathlineto{\pgfqpoint{5.103289in}{2.812662in}}%
\pgfpathlineto{\pgfqpoint{5.110857in}{2.822899in}}%
\pgfpathlineto{\pgfqpoint{5.118417in}{2.832962in}}%
\pgfpathlineto{\pgfqpoint{5.125969in}{2.842848in}}%
\pgfpathlineto{\pgfqpoint{5.111661in}{2.828493in}}%
\pgfpathlineto{\pgfqpoint{5.097373in}{2.814302in}}%
\pgfpathlineto{\pgfqpoint{5.083105in}{2.800276in}}%
\pgfpathlineto{\pgfqpoint{5.068856in}{2.786414in}}%
\pgfpathlineto{\pgfqpoint{5.061303in}{2.776531in}}%
\pgfpathlineto{\pgfqpoint{5.053741in}{2.766481in}}%
\pgfpathlineto{\pgfqpoint{5.046172in}{2.756262in}}%
\pgfpathlineto{\pgfqpoint{5.038594in}{2.745876in}}%
\pgfpathclose%
\pgfusepath{fill}%
\end{pgfscope}%
\begin{pgfscope}%
\pgfpathrectangle{\pgfqpoint{1.254980in}{0.150000in}}{\pgfqpoint{5.490039in}{5.490039in}}%
\pgfusepath{clip}%
\pgfsetbuttcap%
\pgfsetroundjoin%
\definecolor{currentfill}{rgb}{0.271828,0.209303,0.504434}%
\pgfsetfillcolor{currentfill}%
\pgfsetfillopacity{0.700000}%
\pgfsetlinewidth{0.000000pt}%
\definecolor{currentstroke}{rgb}{0.000000,0.000000,0.000000}%
\pgfsetstrokecolor{currentstroke}%
\pgfsetdash{}{0pt}%
\pgfpathmoveto{\pgfqpoint{4.126721in}{1.619170in}}%
\pgfpathlineto{\pgfqpoint{4.140434in}{1.624056in}}%
\pgfpathlineto{\pgfqpoint{4.154159in}{1.629099in}}%
\pgfpathlineto{\pgfqpoint{4.167896in}{1.634299in}}%
\pgfpathlineto{\pgfqpoint{4.181645in}{1.639656in}}%
\pgfpathlineto{\pgfqpoint{4.189531in}{1.654275in}}%
\pgfpathlineto{\pgfqpoint{4.197412in}{1.668904in}}%
\pgfpathlineto{\pgfqpoint{4.205290in}{1.683538in}}%
\pgfpathlineto{\pgfqpoint{4.213163in}{1.698174in}}%
\pgfpathlineto{\pgfqpoint{4.199414in}{1.692363in}}%
\pgfpathlineto{\pgfqpoint{4.185677in}{1.686709in}}%
\pgfpathlineto{\pgfqpoint{4.171952in}{1.681213in}}%
\pgfpathlineto{\pgfqpoint{4.158239in}{1.675875in}}%
\pgfpathlineto{\pgfqpoint{4.150366in}{1.661682in}}%
\pgfpathlineto{\pgfqpoint{4.142488in}{1.647498in}}%
\pgfpathlineto{\pgfqpoint{4.134607in}{1.633326in}}%
\pgfpathlineto{\pgfqpoint{4.126721in}{1.619170in}}%
\pgfpathclose%
\pgfusepath{fill}%
\end{pgfscope}%
\begin{pgfscope}%
\pgfpathrectangle{\pgfqpoint{1.254980in}{0.150000in}}{\pgfqpoint{5.490039in}{5.490039in}}%
\pgfusepath{clip}%
\pgfsetbuttcap%
\pgfsetroundjoin%
\definecolor{currentfill}{rgb}{0.283197,0.115680,0.436115}%
\pgfsetfillcolor{currentfill}%
\pgfsetfillopacity{0.700000}%
\pgfsetlinewidth{0.000000pt}%
\definecolor{currentstroke}{rgb}{0.000000,0.000000,0.000000}%
\pgfsetstrokecolor{currentstroke}%
\pgfsetdash{}{0pt}%
\pgfpathmoveto{\pgfqpoint{3.922369in}{1.430985in}}%
\pgfpathlineto{\pgfqpoint{3.936009in}{1.433095in}}%
\pgfpathlineto{\pgfqpoint{3.949659in}{1.435362in}}%
\pgfpathlineto{\pgfqpoint{3.963318in}{1.437786in}}%
\pgfpathlineto{\pgfqpoint{3.976987in}{1.440366in}}%
\pgfpathlineto{\pgfqpoint{3.984928in}{1.453419in}}%
\pgfpathlineto{\pgfqpoint{3.992865in}{1.466552in}}%
\pgfpathlineto{\pgfqpoint{4.000796in}{1.479760in}}%
\pgfpathlineto{\pgfqpoint{4.008724in}{1.493038in}}%
\pgfpathlineto{\pgfqpoint{3.995059in}{1.489924in}}%
\pgfpathlineto{\pgfqpoint{3.981405in}{1.486967in}}%
\pgfpathlineto{\pgfqpoint{3.967761in}{1.484167in}}%
\pgfpathlineto{\pgfqpoint{3.954126in}{1.481524in}}%
\pgfpathlineto{\pgfqpoint{3.946194in}{1.468769in}}%
\pgfpathlineto{\pgfqpoint{3.938257in}{1.456091in}}%
\pgfpathlineto{\pgfqpoint{3.930315in}{1.443494in}}%
\pgfpathlineto{\pgfqpoint{3.922369in}{1.430985in}}%
\pgfpathclose%
\pgfusepath{fill}%
\end{pgfscope}%
\begin{pgfscope}%
\pgfpathrectangle{\pgfqpoint{1.254980in}{0.150000in}}{\pgfqpoint{5.490039in}{5.490039in}}%
\pgfusepath{clip}%
\pgfsetbuttcap%
\pgfsetroundjoin%
\definecolor{currentfill}{rgb}{0.120092,0.600104,0.542530}%
\pgfsetfillcolor{currentfill}%
\pgfsetfillopacity{0.700000}%
\pgfsetlinewidth{0.000000pt}%
\definecolor{currentstroke}{rgb}{0.000000,0.000000,0.000000}%
\pgfsetstrokecolor{currentstroke}%
\pgfsetdash{}{0pt}%
\pgfpathmoveto{\pgfqpoint{4.920880in}{2.603491in}}%
\pgfpathlineto{\pgfqpoint{4.935054in}{2.616557in}}%
\pgfpathlineto{\pgfqpoint{4.949246in}{2.629787in}}%
\pgfpathlineto{\pgfqpoint{4.963458in}{2.643180in}}%
\pgfpathlineto{\pgfqpoint{4.977688in}{2.656737in}}%
\pgfpathlineto{\pgfqpoint{4.985328in}{2.668466in}}%
\pgfpathlineto{\pgfqpoint{4.992961in}{2.680028in}}%
\pgfpathlineto{\pgfqpoint{5.000586in}{2.691423in}}%
\pgfpathlineto{\pgfqpoint{5.008203in}{2.702650in}}%
\pgfpathlineto{\pgfqpoint{4.993972in}{2.689014in}}%
\pgfpathlineto{\pgfqpoint{4.979760in}{2.675543in}}%
\pgfpathlineto{\pgfqpoint{4.965566in}{2.662235in}}%
\pgfpathlineto{\pgfqpoint{4.951392in}{2.649092in}}%
\pgfpathlineto{\pgfqpoint{4.943775in}{2.637931in}}%
\pgfpathlineto{\pgfqpoint{4.936150in}{2.626610in}}%
\pgfpathlineto{\pgfqpoint{4.928519in}{2.615130in}}%
\pgfpathlineto{\pgfqpoint{4.920880in}{2.603491in}}%
\pgfpathclose%
\pgfusepath{fill}%
\end{pgfscope}%
\begin{pgfscope}%
\pgfpathrectangle{\pgfqpoint{1.254980in}{0.150000in}}{\pgfqpoint{5.490039in}{5.490039in}}%
\pgfusepath{clip}%
\pgfsetbuttcap%
\pgfsetroundjoin%
\definecolor{currentfill}{rgb}{0.269944,0.014625,0.341379}%
\pgfsetfillcolor{currentfill}%
\pgfsetfillopacity{0.700000}%
\pgfsetlinewidth{0.000000pt}%
\definecolor{currentstroke}{rgb}{0.000000,0.000000,0.000000}%
\pgfsetstrokecolor{currentstroke}%
\pgfsetdash{}{0pt}%
\pgfpathmoveto{\pgfqpoint{3.576592in}{1.273220in}}%
\pgfpathlineto{\pgfqpoint{3.590163in}{1.270401in}}%
\pgfpathlineto{\pgfqpoint{3.603740in}{1.267741in}}%
\pgfpathlineto{\pgfqpoint{3.617322in}{1.265241in}}%
\pgfpathlineto{\pgfqpoint{3.630910in}{1.262900in}}%
\pgfpathlineto{\pgfqpoint{3.638993in}{1.271306in}}%
\pgfpathlineto{\pgfqpoint{3.647069in}{1.279911in}}%
\pgfpathlineto{\pgfqpoint{3.655136in}{1.288710in}}%
\pgfpathlineto{\pgfqpoint{3.663196in}{1.297696in}}%
\pgfpathlineto{\pgfqpoint{3.649625in}{1.299396in}}%
\pgfpathlineto{\pgfqpoint{3.636060in}{1.301256in}}%
\pgfpathlineto{\pgfqpoint{3.622502in}{1.303275in}}%
\pgfpathlineto{\pgfqpoint{3.608949in}{1.305455in}}%
\pgfpathlineto{\pgfqpoint{3.600872in}{1.297098in}}%
\pgfpathlineto{\pgfqpoint{3.592787in}{1.288936in}}%
\pgfpathlineto{\pgfqpoint{3.584694in}{1.280975in}}%
\pgfpathlineto{\pgfqpoint{3.576592in}{1.273220in}}%
\pgfpathclose%
\pgfusepath{fill}%
\end{pgfscope}%
\begin{pgfscope}%
\pgfpathrectangle{\pgfqpoint{1.254980in}{0.150000in}}{\pgfqpoint{5.490039in}{5.490039in}}%
\pgfusepath{clip}%
\pgfsetbuttcap%
\pgfsetroundjoin%
\definecolor{currentfill}{rgb}{0.133743,0.548535,0.553541}%
\pgfsetfillcolor{currentfill}%
\pgfsetfillopacity{0.700000}%
\pgfsetlinewidth{0.000000pt}%
\definecolor{currentstroke}{rgb}{0.000000,0.000000,0.000000}%
\pgfsetstrokecolor{currentstroke}%
\pgfsetdash{}{0pt}%
\pgfpathmoveto{\pgfqpoint{4.803019in}{2.455079in}}%
\pgfpathlineto{\pgfqpoint{4.817116in}{2.467245in}}%
\pgfpathlineto{\pgfqpoint{4.831231in}{2.479574in}}%
\pgfpathlineto{\pgfqpoint{4.845364in}{2.492065in}}%
\pgfpathlineto{\pgfqpoint{4.859515in}{2.504720in}}%
\pgfpathlineto{\pgfqpoint{4.867210in}{2.517609in}}%
\pgfpathlineto{\pgfqpoint{4.874898in}{2.530345in}}%
\pgfpathlineto{\pgfqpoint{4.882579in}{2.542926in}}%
\pgfpathlineto{\pgfqpoint{4.890253in}{2.555353in}}%
\pgfpathlineto{\pgfqpoint{4.876099in}{2.542559in}}%
\pgfpathlineto{\pgfqpoint{4.861963in}{2.529928in}}%
\pgfpathlineto{\pgfqpoint{4.847846in}{2.517460in}}%
\pgfpathlineto{\pgfqpoint{4.833746in}{2.505155in}}%
\pgfpathlineto{\pgfqpoint{4.826074in}{2.492856in}}%
\pgfpathlineto{\pgfqpoint{4.818396in}{2.480410in}}%
\pgfpathlineto{\pgfqpoint{4.810710in}{2.467817in}}%
\pgfpathlineto{\pgfqpoint{4.803019in}{2.455079in}}%
\pgfpathclose%
\pgfusepath{fill}%
\end{pgfscope}%
\begin{pgfscope}%
\pgfpathrectangle{\pgfqpoint{1.254980in}{0.150000in}}{\pgfqpoint{5.490039in}{5.490039in}}%
\pgfusepath{clip}%
\pgfsetbuttcap%
\pgfsetroundjoin%
\definecolor{currentfill}{rgb}{0.232815,0.732247,0.459277}%
\pgfsetfillcolor{currentfill}%
\pgfsetfillopacity{0.700000}%
\pgfsetlinewidth{0.000000pt}%
\definecolor{currentstroke}{rgb}{0.000000,0.000000,0.000000}%
\pgfsetstrokecolor{currentstroke}%
\pgfsetdash{}{0pt}%
\pgfpathmoveto{\pgfqpoint{5.243470in}{2.974512in}}%
\pgfpathlineto{\pgfqpoint{5.257872in}{2.989629in}}%
\pgfpathlineto{\pgfqpoint{5.272296in}{3.004913in}}%
\pgfpathlineto{\pgfqpoint{5.286740in}{3.020362in}}%
\pgfpathlineto{\pgfqpoint{5.301205in}{3.035977in}}%
\pgfpathlineto{\pgfqpoint{5.308667in}{3.044178in}}%
\pgfpathlineto{\pgfqpoint{5.316119in}{3.052194in}}%
\pgfpathlineto{\pgfqpoint{5.323561in}{3.060029in}}%
\pgfpathlineto{\pgfqpoint{5.330994in}{3.067682in}}%
\pgfpathlineto{\pgfqpoint{5.316534in}{3.052147in}}%
\pgfpathlineto{\pgfqpoint{5.302095in}{3.036778in}}%
\pgfpathlineto{\pgfqpoint{5.287677in}{3.021575in}}%
\pgfpathlineto{\pgfqpoint{5.273280in}{3.006537in}}%
\pgfpathlineto{\pgfqpoint{5.265842in}{2.998792in}}%
\pgfpathlineto{\pgfqpoint{5.258394in}{2.990873in}}%
\pgfpathlineto{\pgfqpoint{5.250937in}{2.982780in}}%
\pgfpathlineto{\pgfqpoint{5.243470in}{2.974512in}}%
\pgfpathclose%
\pgfusepath{fill}%
\end{pgfscope}%
\begin{pgfscope}%
\pgfpathrectangle{\pgfqpoint{1.254980in}{0.150000in}}{\pgfqpoint{5.490039in}{5.490039in}}%
\pgfusepath{clip}%
\pgfsetbuttcap%
\pgfsetroundjoin%
\definecolor{currentfill}{rgb}{0.154815,0.493313,0.557840}%
\pgfsetfillcolor{currentfill}%
\pgfsetfillopacity{0.700000}%
\pgfsetlinewidth{0.000000pt}%
\definecolor{currentstroke}{rgb}{0.000000,0.000000,0.000000}%
\pgfsetstrokecolor{currentstroke}%
\pgfsetdash{}{0pt}%
\pgfpathmoveto{\pgfqpoint{4.685074in}{2.302517in}}%
\pgfpathlineto{\pgfqpoint{4.699096in}{2.313666in}}%
\pgfpathlineto{\pgfqpoint{4.713135in}{2.324977in}}%
\pgfpathlineto{\pgfqpoint{4.727191in}{2.336450in}}%
\pgfpathlineto{\pgfqpoint{4.741264in}{2.348085in}}%
\pgfpathlineto{\pgfqpoint{4.749004in}{2.361941in}}%
\pgfpathlineto{\pgfqpoint{4.756739in}{2.375662in}}%
\pgfpathlineto{\pgfqpoint{4.764467in}{2.389247in}}%
\pgfpathlineto{\pgfqpoint{4.772190in}{2.402695in}}%
\pgfpathlineto{\pgfqpoint{4.758113in}{2.390860in}}%
\pgfpathlineto{\pgfqpoint{4.744054in}{2.379188in}}%
\pgfpathlineto{\pgfqpoint{4.730011in}{2.367678in}}%
\pgfpathlineto{\pgfqpoint{4.715986in}{2.356329in}}%
\pgfpathlineto{\pgfqpoint{4.708267in}{2.343070in}}%
\pgfpathlineto{\pgfqpoint{4.700542in}{2.329680in}}%
\pgfpathlineto{\pgfqpoint{4.692811in}{2.316162in}}%
\pgfpathlineto{\pgfqpoint{4.685074in}{2.302517in}}%
\pgfpathclose%
\pgfusepath{fill}%
\end{pgfscope}%
\begin{pgfscope}%
\pgfpathrectangle{\pgfqpoint{1.254980in}{0.150000in}}{\pgfqpoint{5.490039in}{5.490039in}}%
\pgfusepath{clip}%
\pgfsetbuttcap%
\pgfsetroundjoin%
\definecolor{currentfill}{rgb}{0.177423,0.437527,0.557565}%
\pgfsetfillcolor{currentfill}%
\pgfsetfillopacity{0.700000}%
\pgfsetlinewidth{0.000000pt}%
\definecolor{currentstroke}{rgb}{0.000000,0.000000,0.000000}%
\pgfsetstrokecolor{currentstroke}%
\pgfsetdash{}{0pt}%
\pgfpathmoveto{\pgfqpoint{4.567096in}{2.147968in}}%
\pgfpathlineto{\pgfqpoint{4.581045in}{2.157987in}}%
\pgfpathlineto{\pgfqpoint{4.595010in}{2.168166in}}%
\pgfpathlineto{\pgfqpoint{4.608991in}{2.178506in}}%
\pgfpathlineto{\pgfqpoint{4.622987in}{2.189007in}}%
\pgfpathlineto{\pgfqpoint{4.630767in}{2.203600in}}%
\pgfpathlineto{\pgfqpoint{4.638541in}{2.218081in}}%
\pgfpathlineto{\pgfqpoint{4.646310in}{2.232450in}}%
\pgfpathlineto{\pgfqpoint{4.654073in}{2.246702in}}%
\pgfpathlineto{\pgfqpoint{4.640072in}{2.235943in}}%
\pgfpathlineto{\pgfqpoint{4.626087in}{2.225345in}}%
\pgfpathlineto{\pgfqpoint{4.612118in}{2.214907in}}%
\pgfpathlineto{\pgfqpoint{4.598165in}{2.204631in}}%
\pgfpathlineto{\pgfqpoint{4.590405in}{2.190625in}}%
\pgfpathlineto{\pgfqpoint{4.582641in}{2.176511in}}%
\pgfpathlineto{\pgfqpoint{4.574871in}{2.162291in}}%
\pgfpathlineto{\pgfqpoint{4.567096in}{2.147968in}}%
\pgfpathclose%
\pgfusepath{fill}%
\end{pgfscope}%
\begin{pgfscope}%
\pgfpathrectangle{\pgfqpoint{1.254980in}{0.150000in}}{\pgfqpoint{5.490039in}{5.490039in}}%
\pgfusepath{clip}%
\pgfsetbuttcap%
\pgfsetroundjoin%
\definecolor{currentfill}{rgb}{0.281887,0.150881,0.465405}%
\pgfsetfillcolor{currentfill}%
\pgfsetfillopacity{0.700000}%
\pgfsetlinewidth{0.000000pt}%
\definecolor{currentstroke}{rgb}{0.000000,0.000000,0.000000}%
\pgfsetstrokecolor{currentstroke}%
\pgfsetdash{}{0pt}%
\pgfpathmoveto{\pgfqpoint{4.008724in}{1.493038in}}%
\pgfpathlineto{\pgfqpoint{4.022398in}{1.496309in}}%
\pgfpathlineto{\pgfqpoint{4.036083in}{1.499736in}}%
\pgfpathlineto{\pgfqpoint{4.049778in}{1.503320in}}%
\pgfpathlineto{\pgfqpoint{4.063484in}{1.507061in}}%
\pgfpathlineto{\pgfqpoint{4.071404in}{1.520921in}}%
\pgfpathlineto{\pgfqpoint{4.079319in}{1.534835in}}%
\pgfpathlineto{\pgfqpoint{4.087230in}{1.548797in}}%
\pgfpathlineto{\pgfqpoint{4.095136in}{1.562803in}}%
\pgfpathlineto{\pgfqpoint{4.081433in}{1.558555in}}%
\pgfpathlineto{\pgfqpoint{4.067740in}{1.554463in}}%
\pgfpathlineto{\pgfqpoint{4.054058in}{1.550529in}}%
\pgfpathlineto{\pgfqpoint{4.040387in}{1.546752in}}%
\pgfpathlineto{\pgfqpoint{4.032478in}{1.533243in}}%
\pgfpathlineto{\pgfqpoint{4.024564in}{1.519784in}}%
\pgfpathlineto{\pgfqpoint{4.016646in}{1.506381in}}%
\pgfpathlineto{\pgfqpoint{4.008724in}{1.493038in}}%
\pgfpathclose%
\pgfusepath{fill}%
\end{pgfscope}%
\begin{pgfscope}%
\pgfpathrectangle{\pgfqpoint{1.254980in}{0.150000in}}{\pgfqpoint{5.490039in}{5.490039in}}%
\pgfusepath{clip}%
\pgfsetbuttcap%
\pgfsetroundjoin%
\definecolor{currentfill}{rgb}{0.203063,0.379716,0.553925}%
\pgfsetfillcolor{currentfill}%
\pgfsetfillopacity{0.700000}%
\pgfsetlinewidth{0.000000pt}%
\definecolor{currentstroke}{rgb}{0.000000,0.000000,0.000000}%
\pgfsetstrokecolor{currentstroke}%
\pgfsetdash{}{0pt}%
\pgfpathmoveto{\pgfqpoint{4.449115in}{1.993874in}}%
\pgfpathlineto{\pgfqpoint{4.462995in}{2.002651in}}%
\pgfpathlineto{\pgfqpoint{4.476890in}{2.011587in}}%
\pgfpathlineto{\pgfqpoint{4.490800in}{2.020683in}}%
\pgfpathlineto{\pgfqpoint{4.504725in}{2.029938in}}%
\pgfpathlineto{\pgfqpoint{4.512537in}{2.045003in}}%
\pgfpathlineto{\pgfqpoint{4.520345in}{2.059983in}}%
\pgfpathlineto{\pgfqpoint{4.528149in}{2.074878in}}%
\pgfpathlineto{\pgfqpoint{4.535948in}{2.089684in}}%
\pgfpathlineto{\pgfqpoint{4.522018in}{2.080113in}}%
\pgfpathlineto{\pgfqpoint{4.508104in}{2.070701in}}%
\pgfpathlineto{\pgfqpoint{4.494205in}{2.061449in}}%
\pgfpathlineto{\pgfqpoint{4.480321in}{2.052357in}}%
\pgfpathlineto{\pgfqpoint{4.472527in}{2.037856in}}%
\pgfpathlineto{\pgfqpoint{4.464727in}{2.023273in}}%
\pgfpathlineto{\pgfqpoint{4.456923in}{2.008611in}}%
\pgfpathlineto{\pgfqpoint{4.449115in}{1.993874in}}%
\pgfpathclose%
\pgfusepath{fill}%
\end{pgfscope}%
\begin{pgfscope}%
\pgfpathrectangle{\pgfqpoint{1.254980in}{0.150000in}}{\pgfqpoint{5.490039in}{5.490039in}}%
\pgfusepath{clip}%
\pgfsetbuttcap%
\pgfsetroundjoin%
\definecolor{currentfill}{rgb}{0.233603,0.313828,0.543914}%
\pgfsetfillcolor{currentfill}%
\pgfsetfillopacity{0.700000}%
\pgfsetlinewidth{0.000000pt}%
\definecolor{currentstroke}{rgb}{0.000000,0.000000,0.000000}%
\pgfsetstrokecolor{currentstroke}%
\pgfsetdash{}{0pt}%
\pgfpathmoveto{\pgfqpoint{4.331142in}{1.842948in}}%
\pgfpathlineto{\pgfqpoint{4.344959in}{1.850374in}}%
\pgfpathlineto{\pgfqpoint{4.358790in}{1.857959in}}%
\pgfpathlineto{\pgfqpoint{4.372634in}{1.865702in}}%
\pgfpathlineto{\pgfqpoint{4.386493in}{1.873603in}}%
\pgfpathlineto{\pgfqpoint{4.394335in}{1.888838in}}%
\pgfpathlineto{\pgfqpoint{4.402174in}{1.904022in}}%
\pgfpathlineto{\pgfqpoint{4.410008in}{1.919152in}}%
\pgfpathlineto{\pgfqpoint{4.417838in}{1.934224in}}%
\pgfpathlineto{\pgfqpoint{4.403975in}{1.925950in}}%
\pgfpathlineto{\pgfqpoint{4.390127in}{1.917835in}}%
\pgfpathlineto{\pgfqpoint{4.376294in}{1.909879in}}%
\pgfpathlineto{\pgfqpoint{4.362474in}{1.902081in}}%
\pgfpathlineto{\pgfqpoint{4.354647in}{1.887370in}}%
\pgfpathlineto{\pgfqpoint{4.346816in}{1.872609in}}%
\pgfpathlineto{\pgfqpoint{4.338981in}{1.857800in}}%
\pgfpathlineto{\pgfqpoint{4.331142in}{1.842948in}}%
\pgfpathclose%
\pgfusepath{fill}%
\end{pgfscope}%
\begin{pgfscope}%
\pgfpathrectangle{\pgfqpoint{1.254980in}{0.150000in}}{\pgfqpoint{5.490039in}{5.490039in}}%
\pgfusepath{clip}%
\pgfsetbuttcap%
\pgfsetroundjoin%
\definecolor{currentfill}{rgb}{0.395174,0.797475,0.367757}%
\pgfsetfillcolor{currentfill}%
\pgfsetfillopacity{0.700000}%
\pgfsetlinewidth{0.000000pt}%
\definecolor{currentstroke}{rgb}{0.000000,0.000000,0.000000}%
\pgfsetstrokecolor{currentstroke}%
\pgfsetdash{}{0pt}%
\pgfpathmoveto{\pgfqpoint{5.448078in}{3.185340in}}%
\pgfpathlineto{\pgfqpoint{5.462630in}{3.201449in}}%
\pgfpathlineto{\pgfqpoint{5.477203in}{3.217725in}}%
\pgfpathlineto{\pgfqpoint{5.491798in}{3.234168in}}%
\pgfpathlineto{\pgfqpoint{5.506416in}{3.250778in}}%
\pgfpathlineto{\pgfqpoint{5.513737in}{3.256542in}}%
\pgfpathlineto{\pgfqpoint{5.521048in}{3.262125in}}%
\pgfpathlineto{\pgfqpoint{5.528347in}{3.267531in}}%
\pgfpathlineto{\pgfqpoint{5.535635in}{3.272762in}}%
\pgfpathlineto{\pgfqpoint{5.521028in}{3.256330in}}%
\pgfpathlineto{\pgfqpoint{5.506444in}{3.240065in}}%
\pgfpathlineto{\pgfqpoint{5.491881in}{3.223966in}}%
\pgfpathlineto{\pgfqpoint{5.477341in}{3.208034in}}%
\pgfpathlineto{\pgfqpoint{5.470041in}{3.202614in}}%
\pgfpathlineto{\pgfqpoint{5.462731in}{3.197026in}}%
\pgfpathlineto{\pgfqpoint{5.455410in}{3.191269in}}%
\pgfpathlineto{\pgfqpoint{5.448078in}{3.185340in}}%
\pgfpathclose%
\pgfusepath{fill}%
\end{pgfscope}%
\begin{pgfscope}%
\pgfpathrectangle{\pgfqpoint{1.254980in}{0.150000in}}{\pgfqpoint{5.490039in}{5.490039in}}%
\pgfusepath{clip}%
\pgfsetbuttcap%
\pgfsetroundjoin%
\definecolor{currentfill}{rgb}{0.175707,0.697900,0.491033}%
\pgfsetfillcolor{currentfill}%
\pgfsetfillopacity{0.700000}%
\pgfsetlinewidth{0.000000pt}%
\definecolor{currentstroke}{rgb}{0.000000,0.000000,0.000000}%
\pgfsetstrokecolor{currentstroke}%
\pgfsetdash{}{0pt}%
\pgfpathmoveto{\pgfqpoint{1.919597in}{3.003775in}}%
\pgfpathlineto{\pgfqpoint{1.933894in}{2.974251in}}%
\pgfpathlineto{\pgfqpoint{1.948171in}{2.945059in}}%
\pgfpathlineto{\pgfqpoint{1.962427in}{2.916197in}}%
\pgfpathlineto{\pgfqpoint{1.976664in}{2.887660in}}%
\pgfpathlineto{\pgfqpoint{1.986307in}{2.872699in}}%
\pgfpathlineto{\pgfqpoint{1.995915in}{2.858240in}}%
\pgfpathlineto{\pgfqpoint{2.005488in}{2.844273in}}%
\pgfpathlineto{\pgfqpoint{2.015027in}{2.830791in}}%
\pgfpathlineto{\pgfqpoint{2.000874in}{2.858517in}}%
\pgfpathlineto{\pgfqpoint{1.986703in}{2.886566in}}%
\pgfpathlineto{\pgfqpoint{1.972512in}{2.914942in}}%
\pgfpathlineto{\pgfqpoint{1.958300in}{2.943647in}}%
\pgfpathlineto{\pgfqpoint{1.948678in}{2.957929in}}%
\pgfpathlineto{\pgfqpoint{1.939020in}{2.972705in}}%
\pgfpathlineto{\pgfqpoint{1.929327in}{2.987984in}}%
\pgfpathlineto{\pgfqpoint{1.919597in}{3.003775in}}%
\pgfpathclose%
\pgfusepath{fill}%
\end{pgfscope}%
\begin{pgfscope}%
\pgfpathrectangle{\pgfqpoint{1.254980in}{0.150000in}}{\pgfqpoint{5.490039in}{5.490039in}}%
\pgfusepath{clip}%
\pgfsetbuttcap%
\pgfsetroundjoin%
\definecolor{currentfill}{rgb}{0.258965,0.251537,0.524736}%
\pgfsetfillcolor{currentfill}%
\pgfsetfillopacity{0.700000}%
\pgfsetlinewidth{0.000000pt}%
\definecolor{currentstroke}{rgb}{0.000000,0.000000,0.000000}%
\pgfsetstrokecolor{currentstroke}%
\pgfsetdash{}{0pt}%
\pgfpathmoveto{\pgfqpoint{4.213163in}{1.698174in}}%
\pgfpathlineto{\pgfqpoint{4.226924in}{1.704142in}}%
\pgfpathlineto{\pgfqpoint{4.240698in}{1.710268in}}%
\pgfpathlineto{\pgfqpoint{4.254485in}{1.716552in}}%
\pgfpathlineto{\pgfqpoint{4.268284in}{1.722992in}}%
\pgfpathlineto{\pgfqpoint{4.276155in}{1.738063in}}%
\pgfpathlineto{\pgfqpoint{4.284022in}{1.753119in}}%
\pgfpathlineto{\pgfqpoint{4.291886in}{1.768158in}}%
\pgfpathlineto{\pgfqpoint{4.299745in}{1.783174in}}%
\pgfpathlineto{\pgfqpoint{4.285943in}{1.776306in}}%
\pgfpathlineto{\pgfqpoint{4.272155in}{1.769595in}}%
\pgfpathlineto{\pgfqpoint{4.258380in}{1.763042in}}%
\pgfpathlineto{\pgfqpoint{4.244617in}{1.756648in}}%
\pgfpathlineto{\pgfqpoint{4.236760in}{1.742047in}}%
\pgfpathlineto{\pgfqpoint{4.228898in}{1.727432in}}%
\pgfpathlineto{\pgfqpoint{4.221033in}{1.712806in}}%
\pgfpathlineto{\pgfqpoint{4.213163in}{1.698174in}}%
\pgfpathclose%
\pgfusepath{fill}%
\end{pgfscope}%
\begin{pgfscope}%
\pgfpathrectangle{\pgfqpoint{1.254980in}{0.150000in}}{\pgfqpoint{5.490039in}{5.490039in}}%
\pgfusepath{clip}%
\pgfsetbuttcap%
\pgfsetroundjoin%
\definecolor{currentfill}{rgb}{0.271305,0.019942,0.347269}%
\pgfsetfillcolor{currentfill}%
\pgfsetfillopacity{0.700000}%
\pgfsetlinewidth{0.000000pt}%
\definecolor{currentstroke}{rgb}{0.000000,0.000000,0.000000}%
\pgfsetstrokecolor{currentstroke}%
\pgfsetdash{}{0pt}%
\pgfpathmoveto{\pgfqpoint{3.348395in}{1.282275in}}%
\pgfpathlineto{\pgfqpoint{3.361964in}{1.276138in}}%
\pgfpathlineto{\pgfqpoint{3.375537in}{1.270167in}}%
\pgfpathlineto{\pgfqpoint{3.389112in}{1.264360in}}%
\pgfpathlineto{\pgfqpoint{3.402690in}{1.258717in}}%
\pgfpathlineto{\pgfqpoint{3.410918in}{1.263159in}}%
\pgfpathlineto{\pgfqpoint{3.419134in}{1.267876in}}%
\pgfpathlineto{\pgfqpoint{3.427340in}{1.272862in}}%
\pgfpathlineto{\pgfqpoint{3.435534in}{1.278110in}}%
\pgfpathlineto{\pgfqpoint{3.421983in}{1.283054in}}%
\pgfpathlineto{\pgfqpoint{3.408435in}{1.288163in}}%
\pgfpathlineto{\pgfqpoint{3.394890in}{1.293436in}}%
\pgfpathlineto{\pgfqpoint{3.381349in}{1.298875in}}%
\pgfpathlineto{\pgfqpoint{3.373128in}{1.294314in}}%
\pgfpathlineto{\pgfqpoint{3.364896in}{1.290022in}}%
\pgfpathlineto{\pgfqpoint{3.356651in}{1.286007in}}%
\pgfpathlineto{\pgfqpoint{3.348395in}{1.282275in}}%
\pgfpathclose%
\pgfusepath{fill}%
\end{pgfscope}%
\begin{pgfscope}%
\pgfpathrectangle{\pgfqpoint{1.254980in}{0.150000in}}{\pgfqpoint{5.490039in}{5.490039in}}%
\pgfusepath{clip}%
\pgfsetbuttcap%
\pgfsetroundjoin%
\definecolor{currentfill}{rgb}{0.269944,0.014625,0.341379}%
\pgfsetfillcolor{currentfill}%
\pgfsetfillopacity{0.700000}%
\pgfsetlinewidth{0.000000pt}%
\definecolor{currentstroke}{rgb}{0.000000,0.000000,0.000000}%
\pgfsetstrokecolor{currentstroke}%
\pgfsetdash{}{0pt}%
\pgfpathmoveto{\pgfqpoint{3.489780in}{1.259962in}}%
\pgfpathlineto{\pgfqpoint{3.503353in}{1.255831in}}%
\pgfpathlineto{\pgfqpoint{3.516930in}{1.251861in}}%
\pgfpathlineto{\pgfqpoint{3.530511in}{1.248052in}}%
\pgfpathlineto{\pgfqpoint{3.544097in}{1.244403in}}%
\pgfpathlineto{\pgfqpoint{3.552235in}{1.251264in}}%
\pgfpathlineto{\pgfqpoint{3.560363in}{1.258358in}}%
\pgfpathlineto{\pgfqpoint{3.568482in}{1.265679in}}%
\pgfpathlineto{\pgfqpoint{3.576592in}{1.273220in}}%
\pgfpathlineto{\pgfqpoint{3.563027in}{1.276200in}}%
\pgfpathlineto{\pgfqpoint{3.549467in}{1.279340in}}%
\pgfpathlineto{\pgfqpoint{3.535912in}{1.282642in}}%
\pgfpathlineto{\pgfqpoint{3.522362in}{1.286105in}}%
\pgfpathlineto{\pgfqpoint{3.514231in}{1.279221in}}%
\pgfpathlineto{\pgfqpoint{3.506090in}{1.272566in}}%
\pgfpathlineto{\pgfqpoint{3.497940in}{1.266143in}}%
\pgfpathlineto{\pgfqpoint{3.489780in}{1.259962in}}%
\pgfpathclose%
\pgfusepath{fill}%
\end{pgfscope}%
\begin{pgfscope}%
\pgfpathrectangle{\pgfqpoint{1.254980in}{0.150000in}}{\pgfqpoint{5.490039in}{5.490039in}}%
\pgfusepath{clip}%
\pgfsetbuttcap%
\pgfsetroundjoin%
\definecolor{currentfill}{rgb}{0.166383,0.690856,0.496502}%
\pgfsetfillcolor{currentfill}%
\pgfsetfillopacity{0.700000}%
\pgfsetlinewidth{0.000000pt}%
\definecolor{currentstroke}{rgb}{0.000000,0.000000,0.000000}%
\pgfsetstrokecolor{currentstroke}%
\pgfsetdash{}{0pt}%
\pgfpathmoveto{\pgfqpoint{5.125969in}{2.842848in}}%
\pgfpathlineto{\pgfqpoint{5.140296in}{2.857369in}}%
\pgfpathlineto{\pgfqpoint{5.154644in}{2.872054in}}%
\pgfpathlineto{\pgfqpoint{5.169012in}{2.886905in}}%
\pgfpathlineto{\pgfqpoint{5.183400in}{2.901922in}}%
\pgfpathlineto{\pgfqpoint{5.190941in}{2.911629in}}%
\pgfpathlineto{\pgfqpoint{5.198473in}{2.921154in}}%
\pgfpathlineto{\pgfqpoint{5.205996in}{2.930497in}}%
\pgfpathlineto{\pgfqpoint{5.213509in}{2.939659in}}%
\pgfpathlineto{\pgfqpoint{5.199123in}{2.924659in}}%
\pgfpathlineto{\pgfqpoint{5.184758in}{2.909825in}}%
\pgfpathlineto{\pgfqpoint{5.170412in}{2.895156in}}%
\pgfpathlineto{\pgfqpoint{5.156087in}{2.880651in}}%
\pgfpathlineto{\pgfqpoint{5.148571in}{2.871461in}}%
\pgfpathlineto{\pgfqpoint{5.141046in}{2.862098in}}%
\pgfpathlineto{\pgfqpoint{5.133512in}{2.852560in}}%
\pgfpathlineto{\pgfqpoint{5.125969in}{2.842848in}}%
\pgfpathclose%
\pgfusepath{fill}%
\end{pgfscope}%
\begin{pgfscope}%
\pgfpathrectangle{\pgfqpoint{1.254980in}{0.150000in}}{\pgfqpoint{5.490039in}{5.490039in}}%
\pgfusepath{clip}%
\pgfsetbuttcap%
\pgfsetroundjoin%
\definecolor{currentfill}{rgb}{0.276194,0.190074,0.493001}%
\pgfsetfillcolor{currentfill}%
\pgfsetfillopacity{0.700000}%
\pgfsetlinewidth{0.000000pt}%
\definecolor{currentstroke}{rgb}{0.000000,0.000000,0.000000}%
\pgfsetstrokecolor{currentstroke}%
\pgfsetdash{}{0pt}%
\pgfpathmoveto{\pgfqpoint{4.095136in}{1.562803in}}%
\pgfpathlineto{\pgfqpoint{4.108851in}{1.567208in}}%
\pgfpathlineto{\pgfqpoint{4.122577in}{1.571770in}}%
\pgfpathlineto{\pgfqpoint{4.136314in}{1.576489in}}%
\pgfpathlineto{\pgfqpoint{4.150063in}{1.581364in}}%
\pgfpathlineto{\pgfqpoint{4.157965in}{1.595900in}}%
\pgfpathlineto{\pgfqpoint{4.165862in}{1.610464in}}%
\pgfpathlineto{\pgfqpoint{4.173756in}{1.625051in}}%
\pgfpathlineto{\pgfqpoint{4.181645in}{1.639656in}}%
\pgfpathlineto{\pgfqpoint{4.167896in}{1.634299in}}%
\pgfpathlineto{\pgfqpoint{4.154159in}{1.629099in}}%
\pgfpathlineto{\pgfqpoint{4.140434in}{1.624056in}}%
\pgfpathlineto{\pgfqpoint{4.126721in}{1.619170in}}%
\pgfpathlineto{\pgfqpoint{4.118831in}{1.605036in}}%
\pgfpathlineto{\pgfqpoint{4.110937in}{1.590927in}}%
\pgfpathlineto{\pgfqpoint{4.103039in}{1.576848in}}%
\pgfpathlineto{\pgfqpoint{4.095136in}{1.562803in}}%
\pgfpathclose%
\pgfusepath{fill}%
\end{pgfscope}%
\begin{pgfscope}%
\pgfpathrectangle{\pgfqpoint{1.254980in}{0.150000in}}{\pgfqpoint{5.490039in}{5.490039in}}%
\pgfusepath{clip}%
\pgfsetbuttcap%
\pgfsetroundjoin%
\definecolor{currentfill}{rgb}{0.279566,0.067836,0.391917}%
\pgfsetfillcolor{currentfill}%
\pgfsetfillopacity{0.700000}%
\pgfsetlinewidth{0.000000pt}%
\definecolor{currentstroke}{rgb}{0.000000,0.000000,0.000000}%
\pgfsetstrokecolor{currentstroke}%
\pgfsetdash{}{0pt}%
\pgfpathmoveto{\pgfqpoint{3.804067in}{1.332397in}}%
\pgfpathlineto{\pgfqpoint{3.817689in}{1.332732in}}%
\pgfpathlineto{\pgfqpoint{3.831320in}{1.333224in}}%
\pgfpathlineto{\pgfqpoint{3.844959in}{1.333873in}}%
\pgfpathlineto{\pgfqpoint{3.858606in}{1.334678in}}%
\pgfpathlineto{\pgfqpoint{3.866595in}{1.346299in}}%
\pgfpathlineto{\pgfqpoint{3.874579in}{1.358051in}}%
\pgfpathlineto{\pgfqpoint{3.882557in}{1.369927in}}%
\pgfpathlineto{\pgfqpoint{3.890530in}{1.381922in}}%
\pgfpathlineto{\pgfqpoint{3.876892in}{1.380529in}}%
\pgfpathlineto{\pgfqpoint{3.863262in}{1.379293in}}%
\pgfpathlineto{\pgfqpoint{3.849641in}{1.378214in}}%
\pgfpathlineto{\pgfqpoint{3.836029in}{1.377292in}}%
\pgfpathlineto{\pgfqpoint{3.828047in}{1.365873in}}%
\pgfpathlineto{\pgfqpoint{3.820060in}{1.354581in}}%
\pgfpathlineto{\pgfqpoint{3.812066in}{1.343420in}}%
\pgfpathlineto{\pgfqpoint{3.804067in}{1.332397in}}%
\pgfpathclose%
\pgfusepath{fill}%
\end{pgfscope}%
\begin{pgfscope}%
\pgfpathrectangle{\pgfqpoint{1.254980in}{0.150000in}}{\pgfqpoint{5.490039in}{5.490039in}}%
\pgfusepath{clip}%
\pgfsetbuttcap%
\pgfsetroundjoin%
\definecolor{currentfill}{rgb}{0.276022,0.044167,0.370164}%
\pgfsetfillcolor{currentfill}%
\pgfsetfillopacity{0.700000}%
\pgfsetlinewidth{0.000000pt}%
\definecolor{currentstroke}{rgb}{0.000000,0.000000,0.000000}%
\pgfsetstrokecolor{currentstroke}%
\pgfsetdash{}{0pt}%
\pgfpathmoveto{\pgfqpoint{3.717546in}{1.292482in}}%
\pgfpathlineto{\pgfqpoint{3.731150in}{1.291574in}}%
\pgfpathlineto{\pgfqpoint{3.744762in}{1.290824in}}%
\pgfpathlineto{\pgfqpoint{3.758381in}{1.290231in}}%
\pgfpathlineto{\pgfqpoint{3.772007in}{1.289795in}}%
\pgfpathlineto{\pgfqpoint{3.780032in}{1.300210in}}%
\pgfpathlineto{\pgfqpoint{3.788050in}{1.310786in}}%
\pgfpathlineto{\pgfqpoint{3.796061in}{1.321517in}}%
\pgfpathlineto{\pgfqpoint{3.804067in}{1.332397in}}%
\pgfpathlineto{\pgfqpoint{3.790452in}{1.332219in}}%
\pgfpathlineto{\pgfqpoint{3.776846in}{1.332198in}}%
\pgfpathlineto{\pgfqpoint{3.763247in}{1.332335in}}%
\pgfpathlineto{\pgfqpoint{3.749655in}{1.332629in}}%
\pgfpathlineto{\pgfqpoint{3.741638in}{1.322352in}}%
\pgfpathlineto{\pgfqpoint{3.733614in}{1.312232in}}%
\pgfpathlineto{\pgfqpoint{3.725583in}{1.302273in}}%
\pgfpathlineto{\pgfqpoint{3.717546in}{1.292482in}}%
\pgfpathclose%
\pgfusepath{fill}%
\end{pgfscope}%
\begin{pgfscope}%
\pgfpathrectangle{\pgfqpoint{1.254980in}{0.150000in}}{\pgfqpoint{5.490039in}{5.490039in}}%
\pgfusepath{clip}%
\pgfsetbuttcap%
\pgfsetroundjoin%
\definecolor{currentfill}{rgb}{0.282656,0.100196,0.422160}%
\pgfsetfillcolor{currentfill}%
\pgfsetfillopacity{0.700000}%
\pgfsetlinewidth{0.000000pt}%
\definecolor{currentstroke}{rgb}{0.000000,0.000000,0.000000}%
\pgfsetstrokecolor{currentstroke}%
\pgfsetdash{}{0pt}%
\pgfpathmoveto{\pgfqpoint{3.890530in}{1.381922in}}%
\pgfpathlineto{\pgfqpoint{3.904177in}{1.383472in}}%
\pgfpathlineto{\pgfqpoint{3.917834in}{1.385178in}}%
\pgfpathlineto{\pgfqpoint{3.931499in}{1.387040in}}%
\pgfpathlineto{\pgfqpoint{3.945174in}{1.389059in}}%
\pgfpathlineto{\pgfqpoint{3.953135in}{1.401739in}}%
\pgfpathlineto{\pgfqpoint{3.961090in}{1.414521in}}%
\pgfpathlineto{\pgfqpoint{3.969041in}{1.427398in}}%
\pgfpathlineto{\pgfqpoint{3.976987in}{1.440366in}}%
\pgfpathlineto{\pgfqpoint{3.963318in}{1.437786in}}%
\pgfpathlineto{\pgfqpoint{3.949659in}{1.435362in}}%
\pgfpathlineto{\pgfqpoint{3.936009in}{1.433095in}}%
\pgfpathlineto{\pgfqpoint{3.922369in}{1.430985in}}%
\pgfpathlineto{\pgfqpoint{3.914417in}{1.418568in}}%
\pgfpathlineto{\pgfqpoint{3.906460in}{1.406248in}}%
\pgfpathlineto{\pgfqpoint{3.898498in}{1.394031in}}%
\pgfpathlineto{\pgfqpoint{3.890530in}{1.381922in}}%
\pgfpathclose%
\pgfusepath{fill}%
\end{pgfscope}%
\begin{pgfscope}%
\pgfpathrectangle{\pgfqpoint{1.254980in}{0.150000in}}{\pgfqpoint{5.490039in}{5.490039in}}%
\pgfusepath{clip}%
\pgfsetbuttcap%
\pgfsetroundjoin%
\definecolor{currentfill}{rgb}{0.126326,0.644107,0.525311}%
\pgfsetfillcolor{currentfill}%
\pgfsetfillopacity{0.700000}%
\pgfsetlinewidth{0.000000pt}%
\definecolor{currentstroke}{rgb}{0.000000,0.000000,0.000000}%
\pgfsetstrokecolor{currentstroke}%
\pgfsetdash{}{0pt}%
\pgfpathmoveto{\pgfqpoint{5.008203in}{2.702650in}}%
\pgfpathlineto{\pgfqpoint{5.022454in}{2.716449in}}%
\pgfpathlineto{\pgfqpoint{5.036723in}{2.730413in}}%
\pgfpathlineto{\pgfqpoint{5.051012in}{2.744541in}}%
\pgfpathlineto{\pgfqpoint{5.065321in}{2.758834in}}%
\pgfpathlineto{\pgfqpoint{5.072931in}{2.769952in}}%
\pgfpathlineto{\pgfqpoint{5.080533in}{2.780893in}}%
\pgfpathlineto{\pgfqpoint{5.088127in}{2.791659in}}%
\pgfpathlineto{\pgfqpoint{5.095712in}{2.802248in}}%
\pgfpathlineto{\pgfqpoint{5.081403in}{2.787908in}}%
\pgfpathlineto{\pgfqpoint{5.067114in}{2.773733in}}%
\pgfpathlineto{\pgfqpoint{5.052844in}{2.759722in}}%
\pgfpathlineto{\pgfqpoint{5.038594in}{2.745876in}}%
\pgfpathlineto{\pgfqpoint{5.031008in}{2.735322in}}%
\pgfpathlineto{\pgfqpoint{5.023414in}{2.724599in}}%
\pgfpathlineto{\pgfqpoint{5.015813in}{2.713708in}}%
\pgfpathlineto{\pgfqpoint{5.008203in}{2.702650in}}%
\pgfpathclose%
\pgfusepath{fill}%
\end{pgfscope}%
\begin{pgfscope}%
\pgfpathrectangle{\pgfqpoint{1.254980in}{0.150000in}}{\pgfqpoint{5.490039in}{5.490039in}}%
\pgfusepath{clip}%
\pgfsetbuttcap%
\pgfsetroundjoin%
\definecolor{currentfill}{rgb}{0.487026,0.823929,0.312321}%
\pgfsetfillcolor{currentfill}%
\pgfsetfillopacity{0.700000}%
\pgfsetlinewidth{0.000000pt}%
\definecolor{currentstroke}{rgb}{0.000000,0.000000,0.000000}%
\pgfsetstrokecolor{currentstroke}%
\pgfsetdash{}{0pt}%
\pgfpathmoveto{\pgfqpoint{5.535635in}{3.272762in}}%
\pgfpathlineto{\pgfqpoint{5.550264in}{3.289361in}}%
\pgfpathlineto{\pgfqpoint{5.564916in}{3.306127in}}%
\pgfpathlineto{\pgfqpoint{5.579590in}{3.323061in}}%
\pgfpathlineto{\pgfqpoint{5.594288in}{3.340163in}}%
\pgfpathlineto{\pgfqpoint{5.601552in}{3.345021in}}%
\pgfpathlineto{\pgfqpoint{5.608805in}{3.349701in}}%
\pgfpathlineto{\pgfqpoint{5.616047in}{3.354205in}}%
\pgfpathlineto{\pgfqpoint{5.623277in}{3.358535in}}%
\pgfpathlineto{\pgfqpoint{5.608593in}{3.341646in}}%
\pgfpathlineto{\pgfqpoint{5.593932in}{3.324923in}}%
\pgfpathlineto{\pgfqpoint{5.579293in}{3.308368in}}%
\pgfpathlineto{\pgfqpoint{5.564677in}{3.291979in}}%
\pgfpathlineto{\pgfqpoint{5.557433in}{3.287426in}}%
\pgfpathlineto{\pgfqpoint{5.550178in}{3.282707in}}%
\pgfpathlineto{\pgfqpoint{5.542912in}{3.277820in}}%
\pgfpathlineto{\pgfqpoint{5.535635in}{3.272762in}}%
\pgfpathclose%
\pgfusepath{fill}%
\end{pgfscope}%
\begin{pgfscope}%
\pgfpathrectangle{\pgfqpoint{1.254980in}{0.150000in}}{\pgfqpoint{5.490039in}{5.490039in}}%
\pgfusepath{clip}%
\pgfsetbuttcap%
\pgfsetroundjoin%
\definecolor{currentfill}{rgb}{0.311925,0.767822,0.415586}%
\pgfsetfillcolor{currentfill}%
\pgfsetfillopacity{0.700000}%
\pgfsetlinewidth{0.000000pt}%
\definecolor{currentstroke}{rgb}{0.000000,0.000000,0.000000}%
\pgfsetstrokecolor{currentstroke}%
\pgfsetdash{}{0pt}%
\pgfpathmoveto{\pgfqpoint{5.330994in}{3.067682in}}%
\pgfpathlineto{\pgfqpoint{5.345475in}{3.083383in}}%
\pgfpathlineto{\pgfqpoint{5.359977in}{3.099251in}}%
\pgfpathlineto{\pgfqpoint{5.374501in}{3.115285in}}%
\pgfpathlineto{\pgfqpoint{5.389047in}{3.131486in}}%
\pgfpathlineto{\pgfqpoint{5.396462in}{3.138858in}}%
\pgfpathlineto{\pgfqpoint{5.403868in}{3.146044in}}%
\pgfpathlineto{\pgfqpoint{5.411262in}{3.153046in}}%
\pgfpathlineto{\pgfqpoint{5.418647in}{3.159864in}}%
\pgfpathlineto{\pgfqpoint{5.404108in}{3.143776in}}%
\pgfpathlineto{\pgfqpoint{5.389591in}{3.127855in}}%
\pgfpathlineto{\pgfqpoint{5.375096in}{3.112100in}}%
\pgfpathlineto{\pgfqpoint{5.360622in}{3.096511in}}%
\pgfpathlineto{\pgfqpoint{5.353230in}{3.089568in}}%
\pgfpathlineto{\pgfqpoint{5.345828in}{3.082450in}}%
\pgfpathlineto{\pgfqpoint{5.338416in}{3.075155in}}%
\pgfpathlineto{\pgfqpoint{5.330994in}{3.067682in}}%
\pgfpathclose%
\pgfusepath{fill}%
\end{pgfscope}%
\begin{pgfscope}%
\pgfpathrectangle{\pgfqpoint{1.254980in}{0.150000in}}{\pgfqpoint{5.490039in}{5.490039in}}%
\pgfusepath{clip}%
\pgfsetbuttcap%
\pgfsetroundjoin%
\definecolor{currentfill}{rgb}{0.272594,0.025563,0.353093}%
\pgfsetfillcolor{currentfill}%
\pgfsetfillopacity{0.700000}%
\pgfsetlinewidth{0.000000pt}%
\definecolor{currentstroke}{rgb}{0.000000,0.000000,0.000000}%
\pgfsetstrokecolor{currentstroke}%
\pgfsetdash{}{0pt}%
\pgfpathmoveto{\pgfqpoint{3.630910in}{1.262900in}}%
\pgfpathlineto{\pgfqpoint{3.644504in}{1.260718in}}%
\pgfpathlineto{\pgfqpoint{3.658104in}{1.258695in}}%
\pgfpathlineto{\pgfqpoint{3.671711in}{1.256830in}}%
\pgfpathlineto{\pgfqpoint{3.685324in}{1.255123in}}%
\pgfpathlineto{\pgfqpoint{3.693390in}{1.264180in}}%
\pgfpathlineto{\pgfqpoint{3.701450in}{1.273429in}}%
\pgfpathlineto{\pgfqpoint{3.709501in}{1.282866in}}%
\pgfpathlineto{\pgfqpoint{3.717546in}{1.292482in}}%
\pgfpathlineto{\pgfqpoint{3.703948in}{1.293548in}}%
\pgfpathlineto{\pgfqpoint{3.690358in}{1.294772in}}%
\pgfpathlineto{\pgfqpoint{3.676773in}{1.296155in}}%
\pgfpathlineto{\pgfqpoint{3.663196in}{1.297696in}}%
\pgfpathlineto{\pgfqpoint{3.655136in}{1.288710in}}%
\pgfpathlineto{\pgfqpoint{3.647069in}{1.279911in}}%
\pgfpathlineto{\pgfqpoint{3.638993in}{1.271306in}}%
\pgfpathlineto{\pgfqpoint{3.630910in}{1.262900in}}%
\pgfpathclose%
\pgfusepath{fill}%
\end{pgfscope}%
\begin{pgfscope}%
\pgfpathrectangle{\pgfqpoint{1.254980in}{0.150000in}}{\pgfqpoint{5.490039in}{5.490039in}}%
\pgfusepath{clip}%
\pgfsetbuttcap%
\pgfsetroundjoin%
\definecolor{currentfill}{rgb}{0.212395,0.359683,0.551710}%
\pgfsetfillcolor{currentfill}%
\pgfsetfillopacity{0.700000}%
\pgfsetlinewidth{0.000000pt}%
\definecolor{currentstroke}{rgb}{0.000000,0.000000,0.000000}%
\pgfsetstrokecolor{currentstroke}%
\pgfsetdash{}{0pt}%
\pgfpathmoveto{\pgfqpoint{4.417838in}{1.934224in}}%
\pgfpathlineto{\pgfqpoint{4.431714in}{1.942657in}}%
\pgfpathlineto{\pgfqpoint{4.445605in}{1.951249in}}%
\pgfpathlineto{\pgfqpoint{4.459511in}{1.959999in}}%
\pgfpathlineto{\pgfqpoint{4.473431in}{1.968909in}}%
\pgfpathlineto{\pgfqpoint{4.481261in}{1.984276in}}%
\pgfpathlineto{\pgfqpoint{4.489087in}{1.999572in}}%
\pgfpathlineto{\pgfqpoint{4.496908in}{2.014794in}}%
\pgfpathlineto{\pgfqpoint{4.504725in}{2.029938in}}%
\pgfpathlineto{\pgfqpoint{4.490800in}{2.020683in}}%
\pgfpathlineto{\pgfqpoint{4.476890in}{2.011587in}}%
\pgfpathlineto{\pgfqpoint{4.462995in}{2.002651in}}%
\pgfpathlineto{\pgfqpoint{4.449115in}{1.993874in}}%
\pgfpathlineto{\pgfqpoint{4.441302in}{1.979064in}}%
\pgfpathlineto{\pgfqpoint{4.433485in}{1.964183in}}%
\pgfpathlineto{\pgfqpoint{4.425663in}{1.949235in}}%
\pgfpathlineto{\pgfqpoint{4.417838in}{1.934224in}}%
\pgfpathclose%
\pgfusepath{fill}%
\end{pgfscope}%
\begin{pgfscope}%
\pgfpathrectangle{\pgfqpoint{1.254980in}{0.150000in}}{\pgfqpoint{5.490039in}{5.490039in}}%
\pgfusepath{clip}%
\pgfsetbuttcap%
\pgfsetroundjoin%
\definecolor{currentfill}{rgb}{0.121148,0.592739,0.544641}%
\pgfsetfillcolor{currentfill}%
\pgfsetfillopacity{0.700000}%
\pgfsetlinewidth{0.000000pt}%
\definecolor{currentstroke}{rgb}{0.000000,0.000000,0.000000}%
\pgfsetstrokecolor{currentstroke}%
\pgfsetdash{}{0pt}%
\pgfpathmoveto{\pgfqpoint{4.890253in}{2.555353in}}%
\pgfpathlineto{\pgfqpoint{4.904425in}{2.568310in}}%
\pgfpathlineto{\pgfqpoint{4.918616in}{2.581431in}}%
\pgfpathlineto{\pgfqpoint{4.932825in}{2.594715in}}%
\pgfpathlineto{\pgfqpoint{4.947053in}{2.608163in}}%
\pgfpathlineto{\pgfqpoint{4.954723in}{2.620554in}}%
\pgfpathlineto{\pgfqpoint{4.962385in}{2.632781in}}%
\pgfpathlineto{\pgfqpoint{4.970040in}{2.644842in}}%
\pgfpathlineto{\pgfqpoint{4.977688in}{2.656737in}}%
\pgfpathlineto{\pgfqpoint{4.963458in}{2.643180in}}%
\pgfpathlineto{\pgfqpoint{4.949246in}{2.629787in}}%
\pgfpathlineto{\pgfqpoint{4.935054in}{2.616557in}}%
\pgfpathlineto{\pgfqpoint{4.920880in}{2.603491in}}%
\pgfpathlineto{\pgfqpoint{4.913234in}{2.591693in}}%
\pgfpathlineto{\pgfqpoint{4.905580in}{2.579737in}}%
\pgfpathlineto{\pgfqpoint{4.897920in}{2.567623in}}%
\pgfpathlineto{\pgfqpoint{4.890253in}{2.555353in}}%
\pgfpathclose%
\pgfusepath{fill}%
\end{pgfscope}%
\begin{pgfscope}%
\pgfpathrectangle{\pgfqpoint{1.254980in}{0.150000in}}{\pgfqpoint{5.490039in}{5.490039in}}%
\pgfusepath{clip}%
\pgfsetbuttcap%
\pgfsetroundjoin%
\definecolor{currentfill}{rgb}{0.183898,0.422383,0.556944}%
\pgfsetfillcolor{currentfill}%
\pgfsetfillopacity{0.700000}%
\pgfsetlinewidth{0.000000pt}%
\definecolor{currentstroke}{rgb}{0.000000,0.000000,0.000000}%
\pgfsetstrokecolor{currentstroke}%
\pgfsetdash{}{0pt}%
\pgfpathmoveto{\pgfqpoint{4.535948in}{2.089684in}}%
\pgfpathlineto{\pgfqpoint{4.549893in}{2.099416in}}%
\pgfpathlineto{\pgfqpoint{4.563853in}{2.109308in}}%
\pgfpathlineto{\pgfqpoint{4.577829in}{2.119359in}}%
\pgfpathlineto{\pgfqpoint{4.591821in}{2.129572in}}%
\pgfpathlineto{\pgfqpoint{4.599620in}{2.144585in}}%
\pgfpathlineto{\pgfqpoint{4.607414in}{2.159497in}}%
\pgfpathlineto{\pgfqpoint{4.615203in}{2.174305in}}%
\pgfpathlineto{\pgfqpoint{4.622987in}{2.189007in}}%
\pgfpathlineto{\pgfqpoint{4.608991in}{2.178506in}}%
\pgfpathlineto{\pgfqpoint{4.595010in}{2.168166in}}%
\pgfpathlineto{\pgfqpoint{4.581045in}{2.157987in}}%
\pgfpathlineto{\pgfqpoint{4.567096in}{2.147968in}}%
\pgfpathlineto{\pgfqpoint{4.559316in}{2.133543in}}%
\pgfpathlineto{\pgfqpoint{4.551531in}{2.119019in}}%
\pgfpathlineto{\pgfqpoint{4.543742in}{2.104399in}}%
\pgfpathlineto{\pgfqpoint{4.535948in}{2.089684in}}%
\pgfpathclose%
\pgfusepath{fill}%
\end{pgfscope}%
\begin{pgfscope}%
\pgfpathrectangle{\pgfqpoint{1.254980in}{0.150000in}}{\pgfqpoint{5.490039in}{5.490039in}}%
\pgfusepath{clip}%
\pgfsetbuttcap%
\pgfsetroundjoin%
\definecolor{currentfill}{rgb}{0.241237,0.296485,0.539709}%
\pgfsetfillcolor{currentfill}%
\pgfsetfillopacity{0.700000}%
\pgfsetlinewidth{0.000000pt}%
\definecolor{currentstroke}{rgb}{0.000000,0.000000,0.000000}%
\pgfsetstrokecolor{currentstroke}%
\pgfsetdash{}{0pt}%
\pgfpathmoveto{\pgfqpoint{4.299745in}{1.783174in}}%
\pgfpathlineto{\pgfqpoint{4.313559in}{1.790201in}}%
\pgfpathlineto{\pgfqpoint{4.327387in}{1.797385in}}%
\pgfpathlineto{\pgfqpoint{4.341229in}{1.804727in}}%
\pgfpathlineto{\pgfqpoint{4.355084in}{1.812227in}}%
\pgfpathlineto{\pgfqpoint{4.362942in}{1.827630in}}%
\pgfpathlineto{\pgfqpoint{4.370796in}{1.842995in}}%
\pgfpathlineto{\pgfqpoint{4.378647in}{1.858321in}}%
\pgfpathlineto{\pgfqpoint{4.386493in}{1.873603in}}%
\pgfpathlineto{\pgfqpoint{4.372634in}{1.865702in}}%
\pgfpathlineto{\pgfqpoint{4.358790in}{1.857959in}}%
\pgfpathlineto{\pgfqpoint{4.344959in}{1.850374in}}%
\pgfpathlineto{\pgfqpoint{4.331142in}{1.842948in}}%
\pgfpathlineto{\pgfqpoint{4.323299in}{1.828056in}}%
\pgfpathlineto{\pgfqpoint{4.315451in}{1.813127in}}%
\pgfpathlineto{\pgfqpoint{4.307600in}{1.798165in}}%
\pgfpathlineto{\pgfqpoint{4.299745in}{1.783174in}}%
\pgfpathclose%
\pgfusepath{fill}%
\end{pgfscope}%
\begin{pgfscope}%
\pgfpathrectangle{\pgfqpoint{1.254980in}{0.150000in}}{\pgfqpoint{5.490039in}{5.490039in}}%
\pgfusepath{clip}%
\pgfsetbuttcap%
\pgfsetroundjoin%
\definecolor{currentfill}{rgb}{0.159194,0.482237,0.558073}%
\pgfsetfillcolor{currentfill}%
\pgfsetfillopacity{0.700000}%
\pgfsetlinewidth{0.000000pt}%
\definecolor{currentstroke}{rgb}{0.000000,0.000000,0.000000}%
\pgfsetstrokecolor{currentstroke}%
\pgfsetdash{}{0pt}%
\pgfpathmoveto{\pgfqpoint{4.654073in}{2.246702in}}%
\pgfpathlineto{\pgfqpoint{4.668091in}{2.257623in}}%
\pgfpathlineto{\pgfqpoint{4.682126in}{2.268704in}}%
\pgfpathlineto{\pgfqpoint{4.696177in}{2.279947in}}%
\pgfpathlineto{\pgfqpoint{4.710245in}{2.291352in}}%
\pgfpathlineto{\pgfqpoint{4.718008in}{2.305728in}}%
\pgfpathlineto{\pgfqpoint{4.725765in}{2.319977in}}%
\pgfpathlineto{\pgfqpoint{4.733517in}{2.334096in}}%
\pgfpathlineto{\pgfqpoint{4.741264in}{2.348085in}}%
\pgfpathlineto{\pgfqpoint{4.727191in}{2.336450in}}%
\pgfpathlineto{\pgfqpoint{4.713135in}{2.324977in}}%
\pgfpathlineto{\pgfqpoint{4.699096in}{2.313666in}}%
\pgfpathlineto{\pgfqpoint{4.685074in}{2.302517in}}%
\pgfpathlineto{\pgfqpoint{4.677332in}{2.288746in}}%
\pgfpathlineto{\pgfqpoint{4.669585in}{2.274853in}}%
\pgfpathlineto{\pgfqpoint{4.661832in}{2.260837in}}%
\pgfpathlineto{\pgfqpoint{4.654073in}{2.246702in}}%
\pgfpathclose%
\pgfusepath{fill}%
\end{pgfscope}%
\begin{pgfscope}%
\pgfpathrectangle{\pgfqpoint{1.254980in}{0.150000in}}{\pgfqpoint{5.490039in}{5.490039in}}%
\pgfusepath{clip}%
\pgfsetbuttcap%
\pgfsetroundjoin%
\definecolor{currentfill}{rgb}{0.137770,0.537492,0.554906}%
\pgfsetfillcolor{currentfill}%
\pgfsetfillopacity{0.700000}%
\pgfsetlinewidth{0.000000pt}%
\definecolor{currentstroke}{rgb}{0.000000,0.000000,0.000000}%
\pgfsetstrokecolor{currentstroke}%
\pgfsetdash{}{0pt}%
\pgfpathmoveto{\pgfqpoint{4.772190in}{2.402695in}}%
\pgfpathlineto{\pgfqpoint{4.786284in}{2.414691in}}%
\pgfpathlineto{\pgfqpoint{4.800396in}{2.426850in}}%
\pgfpathlineto{\pgfqpoint{4.814525in}{2.439172in}}%
\pgfpathlineto{\pgfqpoint{4.828672in}{2.451656in}}%
\pgfpathlineto{\pgfqpoint{4.836393in}{2.465146in}}%
\pgfpathlineto{\pgfqpoint{4.844107in}{2.478487in}}%
\pgfpathlineto{\pgfqpoint{4.851814in}{2.491679in}}%
\pgfpathlineto{\pgfqpoint{4.859515in}{2.504720in}}%
\pgfpathlineto{\pgfqpoint{4.845364in}{2.492065in}}%
\pgfpathlineto{\pgfqpoint{4.831231in}{2.479574in}}%
\pgfpathlineto{\pgfqpoint{4.817116in}{2.467245in}}%
\pgfpathlineto{\pgfqpoint{4.803019in}{2.455079in}}%
\pgfpathlineto{\pgfqpoint{4.795321in}{2.442196in}}%
\pgfpathlineto{\pgfqpoint{4.787617in}{2.429171in}}%
\pgfpathlineto{\pgfqpoint{4.779906in}{2.416003in}}%
\pgfpathlineto{\pgfqpoint{4.772190in}{2.402695in}}%
\pgfpathclose%
\pgfusepath{fill}%
\end{pgfscope}%
\begin{pgfscope}%
\pgfpathrectangle{\pgfqpoint{1.254980in}{0.150000in}}{\pgfqpoint{5.490039in}{5.490039in}}%
\pgfusepath{clip}%
\pgfsetbuttcap%
\pgfsetroundjoin%
\definecolor{currentfill}{rgb}{0.283072,0.130895,0.449241}%
\pgfsetfillcolor{currentfill}%
\pgfsetfillopacity{0.700000}%
\pgfsetlinewidth{0.000000pt}%
\definecolor{currentstroke}{rgb}{0.000000,0.000000,0.000000}%
\pgfsetstrokecolor{currentstroke}%
\pgfsetdash{}{0pt}%
\pgfpathmoveto{\pgfqpoint{3.976987in}{1.440366in}}%
\pgfpathlineto{\pgfqpoint{3.990666in}{1.443102in}}%
\pgfpathlineto{\pgfqpoint{4.004355in}{1.445995in}}%
\pgfpathlineto{\pgfqpoint{4.018054in}{1.449044in}}%
\pgfpathlineto{\pgfqpoint{4.031763in}{1.452248in}}%
\pgfpathlineto{\pgfqpoint{4.039700in}{1.465847in}}%
\pgfpathlineto{\pgfqpoint{4.047632in}{1.479518in}}%
\pgfpathlineto{\pgfqpoint{4.055560in}{1.493258in}}%
\pgfpathlineto{\pgfqpoint{4.063484in}{1.507061in}}%
\pgfpathlineto{\pgfqpoint{4.049778in}{1.503320in}}%
\pgfpathlineto{\pgfqpoint{4.036083in}{1.499736in}}%
\pgfpathlineto{\pgfqpoint{4.022398in}{1.496309in}}%
\pgfpathlineto{\pgfqpoint{4.008724in}{1.493038in}}%
\pgfpathlineto{\pgfqpoint{4.000796in}{1.479760in}}%
\pgfpathlineto{\pgfqpoint{3.992865in}{1.466552in}}%
\pgfpathlineto{\pgfqpoint{3.984928in}{1.453419in}}%
\pgfpathlineto{\pgfqpoint{3.976987in}{1.440366in}}%
\pgfpathclose%
\pgfusepath{fill}%
\end{pgfscope}%
\begin{pgfscope}%
\pgfpathrectangle{\pgfqpoint{1.254980in}{0.150000in}}{\pgfqpoint{5.490039in}{5.490039in}}%
\pgfusepath{clip}%
\pgfsetbuttcap%
\pgfsetroundjoin%
\definecolor{currentfill}{rgb}{0.269944,0.014625,0.341379}%
\pgfsetfillcolor{currentfill}%
\pgfsetfillopacity{0.700000}%
\pgfsetlinewidth{0.000000pt}%
\definecolor{currentstroke}{rgb}{0.000000,0.000000,0.000000}%
\pgfsetstrokecolor{currentstroke}%
\pgfsetdash{}{0pt}%
\pgfpathmoveto{\pgfqpoint{3.402690in}{1.258717in}}%
\pgfpathlineto{\pgfqpoint{3.416272in}{1.253238in}}%
\pgfpathlineto{\pgfqpoint{3.429858in}{1.247923in}}%
\pgfpathlineto{\pgfqpoint{3.443447in}{1.242770in}}%
\pgfpathlineto{\pgfqpoint{3.457040in}{1.237779in}}%
\pgfpathlineto{\pgfqpoint{3.465241in}{1.242929in}}%
\pgfpathlineto{\pgfqpoint{3.473431in}{1.248348in}}%
\pgfpathlineto{\pgfqpoint{3.481611in}{1.254028in}}%
\pgfpathlineto{\pgfqpoint{3.489780in}{1.259962in}}%
\pgfpathlineto{\pgfqpoint{3.476213in}{1.264255in}}%
\pgfpathlineto{\pgfqpoint{3.462649in}{1.268710in}}%
\pgfpathlineto{\pgfqpoint{3.449090in}{1.273328in}}%
\pgfpathlineto{\pgfqpoint{3.435534in}{1.278110in}}%
\pgfpathlineto{\pgfqpoint{3.427340in}{1.272862in}}%
\pgfpathlineto{\pgfqpoint{3.419134in}{1.267876in}}%
\pgfpathlineto{\pgfqpoint{3.410918in}{1.263159in}}%
\pgfpathlineto{\pgfqpoint{3.402690in}{1.258717in}}%
\pgfpathclose%
\pgfusepath{fill}%
\end{pgfscope}%
\begin{pgfscope}%
\pgfpathrectangle{\pgfqpoint{1.254980in}{0.150000in}}{\pgfqpoint{5.490039in}{5.490039in}}%
\pgfusepath{clip}%
\pgfsetbuttcap%
\pgfsetroundjoin%
\definecolor{currentfill}{rgb}{0.265145,0.232956,0.516599}%
\pgfsetfillcolor{currentfill}%
\pgfsetfillopacity{0.700000}%
\pgfsetlinewidth{0.000000pt}%
\definecolor{currentstroke}{rgb}{0.000000,0.000000,0.000000}%
\pgfsetstrokecolor{currentstroke}%
\pgfsetdash{}{0pt}%
\pgfpathmoveto{\pgfqpoint{4.181645in}{1.639656in}}%
\pgfpathlineto{\pgfqpoint{4.195406in}{1.645170in}}%
\pgfpathlineto{\pgfqpoint{4.209179in}{1.650841in}}%
\pgfpathlineto{\pgfqpoint{4.222964in}{1.656669in}}%
\pgfpathlineto{\pgfqpoint{4.236762in}{1.662654in}}%
\pgfpathlineto{\pgfqpoint{4.244648in}{1.677739in}}%
\pgfpathlineto{\pgfqpoint{4.252531in}{1.692826in}}%
\pgfpathlineto{\pgfqpoint{4.260409in}{1.707912in}}%
\pgfpathlineto{\pgfqpoint{4.268284in}{1.722992in}}%
\pgfpathlineto{\pgfqpoint{4.254485in}{1.716552in}}%
\pgfpathlineto{\pgfqpoint{4.240698in}{1.710268in}}%
\pgfpathlineto{\pgfqpoint{4.226924in}{1.704142in}}%
\pgfpathlineto{\pgfqpoint{4.213163in}{1.698174in}}%
\pgfpathlineto{\pgfqpoint{4.205290in}{1.683538in}}%
\pgfpathlineto{\pgfqpoint{4.197412in}{1.668904in}}%
\pgfpathlineto{\pgfqpoint{4.189531in}{1.654275in}}%
\pgfpathlineto{\pgfqpoint{4.181645in}{1.639656in}}%
\pgfpathclose%
\pgfusepath{fill}%
\end{pgfscope}%
\begin{pgfscope}%
\pgfpathrectangle{\pgfqpoint{1.254980in}{0.150000in}}{\pgfqpoint{5.490039in}{5.490039in}}%
\pgfusepath{clip}%
\pgfsetbuttcap%
\pgfsetroundjoin%
\definecolor{currentfill}{rgb}{0.565498,0.842430,0.262877}%
\pgfsetfillcolor{currentfill}%
\pgfsetfillopacity{0.700000}%
\pgfsetlinewidth{0.000000pt}%
\definecolor{currentstroke}{rgb}{0.000000,0.000000,0.000000}%
\pgfsetstrokecolor{currentstroke}%
\pgfsetdash{}{0pt}%
\pgfpathmoveto{\pgfqpoint{5.623277in}{3.358535in}}%
\pgfpathlineto{\pgfqpoint{5.637985in}{3.375593in}}%
\pgfpathlineto{\pgfqpoint{5.652715in}{3.392818in}}%
\pgfpathlineto{\pgfqpoint{5.667468in}{3.410212in}}%
\pgfpathlineto{\pgfqpoint{5.674676in}{3.414197in}}%
\pgfpathlineto{\pgfqpoint{5.681872in}{3.418008in}}%
\pgfpathlineto{\pgfqpoint{5.689056in}{3.421648in}}%
\pgfpathlineto{\pgfqpoint{5.696229in}{3.425120in}}%
\pgfpathlineto{\pgfqpoint{5.681491in}{3.407972in}}%
\pgfpathlineto{\pgfqpoint{5.666776in}{3.390991in}}%
\pgfpathlineto{\pgfqpoint{5.652084in}{3.374178in}}%
\pgfpathlineto{\pgfqpoint{5.644899in}{3.370513in}}%
\pgfpathlineto{\pgfqpoint{5.637703in}{3.366687in}}%
\pgfpathlineto{\pgfqpoint{5.630496in}{3.362695in}}%
\pgfpathlineto{\pgfqpoint{5.623277in}{3.358535in}}%
\pgfpathclose%
\pgfusepath{fill}%
\end{pgfscope}%
\begin{pgfscope}%
\pgfpathrectangle{\pgfqpoint{1.254980in}{0.150000in}}{\pgfqpoint{5.490039in}{5.490039in}}%
\pgfusepath{clip}%
\pgfsetbuttcap%
\pgfsetroundjoin%
\definecolor{currentfill}{rgb}{0.269944,0.014625,0.341379}%
\pgfsetfillcolor{currentfill}%
\pgfsetfillopacity{0.700000}%
\pgfsetlinewidth{0.000000pt}%
\definecolor{currentstroke}{rgb}{0.000000,0.000000,0.000000}%
\pgfsetstrokecolor{currentstroke}%
\pgfsetdash{}{0pt}%
\pgfpathmoveto{\pgfqpoint{3.544097in}{1.244403in}}%
\pgfpathlineto{\pgfqpoint{3.557689in}{1.240915in}}%
\pgfpathlineto{\pgfqpoint{3.571285in}{1.237587in}}%
\pgfpathlineto{\pgfqpoint{3.584887in}{1.234418in}}%
\pgfpathlineto{\pgfqpoint{3.598494in}{1.231408in}}%
\pgfpathlineto{\pgfqpoint{3.606611in}{1.238948in}}%
\pgfpathlineto{\pgfqpoint{3.614719in}{1.246715in}}%
\pgfpathlineto{\pgfqpoint{3.622819in}{1.254701in}}%
\pgfpathlineto{\pgfqpoint{3.630910in}{1.262900in}}%
\pgfpathlineto{\pgfqpoint{3.617322in}{1.265241in}}%
\pgfpathlineto{\pgfqpoint{3.603740in}{1.267741in}}%
\pgfpathlineto{\pgfqpoint{3.590163in}{1.270401in}}%
\pgfpathlineto{\pgfqpoint{3.576592in}{1.273220in}}%
\pgfpathlineto{\pgfqpoint{3.568482in}{1.265679in}}%
\pgfpathlineto{\pgfqpoint{3.560363in}{1.258358in}}%
\pgfpathlineto{\pgfqpoint{3.552235in}{1.251264in}}%
\pgfpathlineto{\pgfqpoint{3.544097in}{1.244403in}}%
\pgfpathclose%
\pgfusepath{fill}%
\end{pgfscope}%
\begin{pgfscope}%
\pgfpathrectangle{\pgfqpoint{1.254980in}{0.150000in}}{\pgfqpoint{5.490039in}{5.490039in}}%
\pgfusepath{clip}%
\pgfsetbuttcap%
\pgfsetroundjoin%
\definecolor{currentfill}{rgb}{0.226397,0.728888,0.462789}%
\pgfsetfillcolor{currentfill}%
\pgfsetfillopacity{0.700000}%
\pgfsetlinewidth{0.000000pt}%
\definecolor{currentstroke}{rgb}{0.000000,0.000000,0.000000}%
\pgfsetstrokecolor{currentstroke}%
\pgfsetdash{}{0pt}%
\pgfpathmoveto{\pgfqpoint{5.213509in}{2.939659in}}%
\pgfpathlineto{\pgfqpoint{5.227915in}{2.954825in}}%
\pgfpathlineto{\pgfqpoint{5.242343in}{2.970157in}}%
\pgfpathlineto{\pgfqpoint{5.256791in}{2.985655in}}%
\pgfpathlineto{\pgfqpoint{5.271260in}{3.001319in}}%
\pgfpathlineto{\pgfqpoint{5.278761in}{3.010264in}}%
\pgfpathlineto{\pgfqpoint{5.286252in}{3.019022in}}%
\pgfpathlineto{\pgfqpoint{5.293734in}{3.027592in}}%
\pgfpathlineto{\pgfqpoint{5.301205in}{3.035977in}}%
\pgfpathlineto{\pgfqpoint{5.286740in}{3.020362in}}%
\pgfpathlineto{\pgfqpoint{5.272296in}{3.004913in}}%
\pgfpathlineto{\pgfqpoint{5.257872in}{2.989629in}}%
\pgfpathlineto{\pgfqpoint{5.243470in}{2.974512in}}%
\pgfpathlineto{\pgfqpoint{5.235994in}{2.966066in}}%
\pgfpathlineto{\pgfqpoint{5.228508in}{2.957443in}}%
\pgfpathlineto{\pgfqpoint{5.221013in}{2.948641in}}%
\pgfpathlineto{\pgfqpoint{5.213509in}{2.939659in}}%
\pgfpathclose%
\pgfusepath{fill}%
\end{pgfscope}%
\begin{pgfscope}%
\pgfpathrectangle{\pgfqpoint{1.254980in}{0.150000in}}{\pgfqpoint{5.490039in}{5.490039in}}%
\pgfusepath{clip}%
\pgfsetbuttcap%
\pgfsetroundjoin%
\definecolor{currentfill}{rgb}{0.279574,0.170599,0.479997}%
\pgfsetfillcolor{currentfill}%
\pgfsetfillopacity{0.700000}%
\pgfsetlinewidth{0.000000pt}%
\definecolor{currentstroke}{rgb}{0.000000,0.000000,0.000000}%
\pgfsetstrokecolor{currentstroke}%
\pgfsetdash{}{0pt}%
\pgfpathmoveto{\pgfqpoint{4.063484in}{1.507061in}}%
\pgfpathlineto{\pgfqpoint{4.077201in}{1.510957in}}%
\pgfpathlineto{\pgfqpoint{4.090928in}{1.515010in}}%
\pgfpathlineto{\pgfqpoint{4.104667in}{1.519219in}}%
\pgfpathlineto{\pgfqpoint{4.118417in}{1.523585in}}%
\pgfpathlineto{\pgfqpoint{4.126335in}{1.537965in}}%
\pgfpathlineto{\pgfqpoint{4.134248in}{1.552391in}}%
\pgfpathlineto{\pgfqpoint{4.142158in}{1.566859in}}%
\pgfpathlineto{\pgfqpoint{4.150063in}{1.581364in}}%
\pgfpathlineto{\pgfqpoint{4.136314in}{1.576489in}}%
\pgfpathlineto{\pgfqpoint{4.122577in}{1.571770in}}%
\pgfpathlineto{\pgfqpoint{4.108851in}{1.567208in}}%
\pgfpathlineto{\pgfqpoint{4.095136in}{1.562803in}}%
\pgfpathlineto{\pgfqpoint{4.087230in}{1.548797in}}%
\pgfpathlineto{\pgfqpoint{4.079319in}{1.534835in}}%
\pgfpathlineto{\pgfqpoint{4.071404in}{1.520921in}}%
\pgfpathlineto{\pgfqpoint{4.063484in}{1.507061in}}%
\pgfpathclose%
\pgfusepath{fill}%
\end{pgfscope}%
\begin{pgfscope}%
\pgfpathrectangle{\pgfqpoint{1.254980in}{0.150000in}}{\pgfqpoint{5.490039in}{5.490039in}}%
\pgfusepath{clip}%
\pgfsetbuttcap%
\pgfsetroundjoin%
\definecolor{currentfill}{rgb}{0.395174,0.797475,0.367757}%
\pgfsetfillcolor{currentfill}%
\pgfsetfillopacity{0.700000}%
\pgfsetlinewidth{0.000000pt}%
\definecolor{currentstroke}{rgb}{0.000000,0.000000,0.000000}%
\pgfsetstrokecolor{currentstroke}%
\pgfsetdash{}{0pt}%
\pgfpathmoveto{\pgfqpoint{5.418647in}{3.159864in}}%
\pgfpathlineto{\pgfqpoint{5.433207in}{3.176119in}}%
\pgfpathlineto{\pgfqpoint{5.447789in}{3.192541in}}%
\pgfpathlineto{\pgfqpoint{5.462394in}{3.209130in}}%
\pgfpathlineto{\pgfqpoint{5.477021in}{3.225887in}}%
\pgfpathlineto{\pgfqpoint{5.484386in}{3.232390in}}%
\pgfpathlineto{\pgfqpoint{5.491741in}{3.238704in}}%
\pgfpathlineto{\pgfqpoint{5.499084in}{3.244833in}}%
\pgfpathlineto{\pgfqpoint{5.506416in}{3.250778in}}%
\pgfpathlineto{\pgfqpoint{5.491798in}{3.234168in}}%
\pgfpathlineto{\pgfqpoint{5.477203in}{3.217725in}}%
\pgfpathlineto{\pgfqpoint{5.462630in}{3.201449in}}%
\pgfpathlineto{\pgfqpoint{5.448078in}{3.185340in}}%
\pgfpathlineto{\pgfqpoint{5.440736in}{3.179237in}}%
\pgfpathlineto{\pgfqpoint{5.433384in}{3.172958in}}%
\pgfpathlineto{\pgfqpoint{5.426020in}{3.166501in}}%
\pgfpathlineto{\pgfqpoint{5.418647in}{3.159864in}}%
\pgfpathclose%
\pgfusepath{fill}%
\end{pgfscope}%
\begin{pgfscope}%
\pgfpathrectangle{\pgfqpoint{1.254980in}{0.150000in}}{\pgfqpoint{5.490039in}{5.490039in}}%
\pgfusepath{clip}%
\pgfsetbuttcap%
\pgfsetroundjoin%
\definecolor{currentfill}{rgb}{0.162016,0.687316,0.499129}%
\pgfsetfillcolor{currentfill}%
\pgfsetfillopacity{0.700000}%
\pgfsetlinewidth{0.000000pt}%
\definecolor{currentstroke}{rgb}{0.000000,0.000000,0.000000}%
\pgfsetstrokecolor{currentstroke}%
\pgfsetdash{}{0pt}%
\pgfpathmoveto{\pgfqpoint{5.095712in}{2.802248in}}%
\pgfpathlineto{\pgfqpoint{5.110041in}{2.816753in}}%
\pgfpathlineto{\pgfqpoint{5.124390in}{2.831424in}}%
\pgfpathlineto{\pgfqpoint{5.138759in}{2.846259in}}%
\pgfpathlineto{\pgfqpoint{5.153148in}{2.861261in}}%
\pgfpathlineto{\pgfqpoint{5.160724in}{2.871701in}}%
\pgfpathlineto{\pgfqpoint{5.168292in}{2.881958in}}%
\pgfpathlineto{\pgfqpoint{5.175851in}{2.892031in}}%
\pgfpathlineto{\pgfqpoint{5.183400in}{2.901922in}}%
\pgfpathlineto{\pgfqpoint{5.169012in}{2.886905in}}%
\pgfpathlineto{\pgfqpoint{5.154644in}{2.872054in}}%
\pgfpathlineto{\pgfqpoint{5.140296in}{2.857369in}}%
\pgfpathlineto{\pgfqpoint{5.125969in}{2.842848in}}%
\pgfpathlineto{\pgfqpoint{5.118417in}{2.832962in}}%
\pgfpathlineto{\pgfqpoint{5.110857in}{2.822899in}}%
\pgfpathlineto{\pgfqpoint{5.103289in}{2.812662in}}%
\pgfpathlineto{\pgfqpoint{5.095712in}{2.802248in}}%
\pgfpathclose%
\pgfusepath{fill}%
\end{pgfscope}%
\begin{pgfscope}%
\pgfpathrectangle{\pgfqpoint{1.254980in}{0.150000in}}{\pgfqpoint{5.490039in}{5.490039in}}%
\pgfusepath{clip}%
\pgfsetbuttcap%
\pgfsetroundjoin%
\definecolor{currentfill}{rgb}{0.277941,0.056324,0.381191}%
\pgfsetfillcolor{currentfill}%
\pgfsetfillopacity{0.700000}%
\pgfsetlinewidth{0.000000pt}%
\definecolor{currentstroke}{rgb}{0.000000,0.000000,0.000000}%
\pgfsetstrokecolor{currentstroke}%
\pgfsetdash{}{0pt}%
\pgfpathmoveto{\pgfqpoint{3.772007in}{1.289795in}}%
\pgfpathlineto{\pgfqpoint{3.785641in}{1.289515in}}%
\pgfpathlineto{\pgfqpoint{3.799283in}{1.289393in}}%
\pgfpathlineto{\pgfqpoint{3.812932in}{1.289426in}}%
\pgfpathlineto{\pgfqpoint{3.826590in}{1.289616in}}%
\pgfpathlineto{\pgfqpoint{3.834603in}{1.300657in}}%
\pgfpathlineto{\pgfqpoint{3.842610in}{1.311851in}}%
\pgfpathlineto{\pgfqpoint{3.850611in}{1.323194in}}%
\pgfpathlineto{\pgfqpoint{3.858606in}{1.334678in}}%
\pgfpathlineto{\pgfqpoint{3.844959in}{1.333873in}}%
\pgfpathlineto{\pgfqpoint{3.831320in}{1.333224in}}%
\pgfpathlineto{\pgfqpoint{3.817689in}{1.332732in}}%
\pgfpathlineto{\pgfqpoint{3.804067in}{1.332397in}}%
\pgfpathlineto{\pgfqpoint{3.796061in}{1.321517in}}%
\pgfpathlineto{\pgfqpoint{3.788050in}{1.310786in}}%
\pgfpathlineto{\pgfqpoint{3.780032in}{1.300210in}}%
\pgfpathlineto{\pgfqpoint{3.772007in}{1.289795in}}%
\pgfpathclose%
\pgfusepath{fill}%
\end{pgfscope}%
\begin{pgfscope}%
\pgfpathrectangle{\pgfqpoint{1.254980in}{0.150000in}}{\pgfqpoint{5.490039in}{5.490039in}}%
\pgfusepath{clip}%
\pgfsetbuttcap%
\pgfsetroundjoin%
\definecolor{currentfill}{rgb}{0.220057,0.343307,0.549413}%
\pgfsetfillcolor{currentfill}%
\pgfsetfillopacity{0.700000}%
\pgfsetlinewidth{0.000000pt}%
\definecolor{currentstroke}{rgb}{0.000000,0.000000,0.000000}%
\pgfsetstrokecolor{currentstroke}%
\pgfsetdash{}{0pt}%
\pgfpathmoveto{\pgfqpoint{4.386493in}{1.873603in}}%
\pgfpathlineto{\pgfqpoint{4.400366in}{1.881663in}}%
\pgfpathlineto{\pgfqpoint{4.414253in}{1.889881in}}%
\pgfpathlineto{\pgfqpoint{4.428154in}{1.898258in}}%
\pgfpathlineto{\pgfqpoint{4.442069in}{1.906793in}}%
\pgfpathlineto{\pgfqpoint{4.449916in}{1.922412in}}%
\pgfpathlineto{\pgfqpoint{4.457758in}{1.937974in}}%
\pgfpathlineto{\pgfqpoint{4.465597in}{1.953474in}}%
\pgfpathlineto{\pgfqpoint{4.473431in}{1.968909in}}%
\pgfpathlineto{\pgfqpoint{4.459511in}{1.959999in}}%
\pgfpathlineto{\pgfqpoint{4.445605in}{1.951249in}}%
\pgfpathlineto{\pgfqpoint{4.431714in}{1.942657in}}%
\pgfpathlineto{\pgfqpoint{4.417838in}{1.934224in}}%
\pgfpathlineto{\pgfqpoint{4.410008in}{1.919152in}}%
\pgfpathlineto{\pgfqpoint{4.402174in}{1.904022in}}%
\pgfpathlineto{\pgfqpoint{4.394335in}{1.888838in}}%
\pgfpathlineto{\pgfqpoint{4.386493in}{1.873603in}}%
\pgfpathclose%
\pgfusepath{fill}%
\end{pgfscope}%
\begin{pgfscope}%
\pgfpathrectangle{\pgfqpoint{1.254980in}{0.150000in}}{\pgfqpoint{5.490039in}{5.490039in}}%
\pgfusepath{clip}%
\pgfsetbuttcap%
\pgfsetroundjoin%
\definecolor{currentfill}{rgb}{0.281446,0.084320,0.407414}%
\pgfsetfillcolor{currentfill}%
\pgfsetfillopacity{0.700000}%
\pgfsetlinewidth{0.000000pt}%
\definecolor{currentstroke}{rgb}{0.000000,0.000000,0.000000}%
\pgfsetstrokecolor{currentstroke}%
\pgfsetdash{}{0pt}%
\pgfpathmoveto{\pgfqpoint{3.858606in}{1.334678in}}%
\pgfpathlineto{\pgfqpoint{3.872261in}{1.335640in}}%
\pgfpathlineto{\pgfqpoint{3.885926in}{1.336758in}}%
\pgfpathlineto{\pgfqpoint{3.899599in}{1.338031in}}%
\pgfpathlineto{\pgfqpoint{3.913281in}{1.339460in}}%
\pgfpathlineto{\pgfqpoint{3.921262in}{1.351680in}}%
\pgfpathlineto{\pgfqpoint{3.929238in}{1.364024in}}%
\pgfpathlineto{\pgfqpoint{3.937209in}{1.376485in}}%
\pgfpathlineto{\pgfqpoint{3.945174in}{1.389059in}}%
\pgfpathlineto{\pgfqpoint{3.931499in}{1.387040in}}%
\pgfpathlineto{\pgfqpoint{3.917834in}{1.385178in}}%
\pgfpathlineto{\pgfqpoint{3.904177in}{1.383472in}}%
\pgfpathlineto{\pgfqpoint{3.890530in}{1.381922in}}%
\pgfpathlineto{\pgfqpoint{3.882557in}{1.369927in}}%
\pgfpathlineto{\pgfqpoint{3.874579in}{1.358051in}}%
\pgfpathlineto{\pgfqpoint{3.866595in}{1.346299in}}%
\pgfpathlineto{\pgfqpoint{3.858606in}{1.334678in}}%
\pgfpathclose%
\pgfusepath{fill}%
\end{pgfscope}%
\begin{pgfscope}%
\pgfpathrectangle{\pgfqpoint{1.254980in}{0.150000in}}{\pgfqpoint{5.490039in}{5.490039in}}%
\pgfusepath{clip}%
\pgfsetbuttcap%
\pgfsetroundjoin%
\definecolor{currentfill}{rgb}{0.190631,0.407061,0.556089}%
\pgfsetfillcolor{currentfill}%
\pgfsetfillopacity{0.700000}%
\pgfsetlinewidth{0.000000pt}%
\definecolor{currentstroke}{rgb}{0.000000,0.000000,0.000000}%
\pgfsetstrokecolor{currentstroke}%
\pgfsetdash{}{0pt}%
\pgfpathmoveto{\pgfqpoint{4.504725in}{2.029938in}}%
\pgfpathlineto{\pgfqpoint{4.518665in}{2.039353in}}%
\pgfpathlineto{\pgfqpoint{4.532621in}{2.048928in}}%
\pgfpathlineto{\pgfqpoint{4.546592in}{2.058662in}}%
\pgfpathlineto{\pgfqpoint{4.560578in}{2.068556in}}%
\pgfpathlineto{\pgfqpoint{4.568396in}{2.083949in}}%
\pgfpathlineto{\pgfqpoint{4.576209in}{2.099251in}}%
\pgfpathlineto{\pgfqpoint{4.584017in}{2.114459in}}%
\pgfpathlineto{\pgfqpoint{4.591821in}{2.129572in}}%
\pgfpathlineto{\pgfqpoint{4.577829in}{2.119359in}}%
\pgfpathlineto{\pgfqpoint{4.563853in}{2.109308in}}%
\pgfpathlineto{\pgfqpoint{4.549893in}{2.099416in}}%
\pgfpathlineto{\pgfqpoint{4.535948in}{2.089684in}}%
\pgfpathlineto{\pgfqpoint{4.528149in}{2.074878in}}%
\pgfpathlineto{\pgfqpoint{4.520345in}{2.059983in}}%
\pgfpathlineto{\pgfqpoint{4.512537in}{2.045003in}}%
\pgfpathlineto{\pgfqpoint{4.504725in}{2.029938in}}%
\pgfpathclose%
\pgfusepath{fill}%
\end{pgfscope}%
\begin{pgfscope}%
\pgfpathrectangle{\pgfqpoint{1.254980in}{0.150000in}}{\pgfqpoint{5.490039in}{5.490039in}}%
\pgfusepath{clip}%
\pgfsetbuttcap%
\pgfsetroundjoin%
\definecolor{currentfill}{rgb}{0.248629,0.278775,0.534556}%
\pgfsetfillcolor{currentfill}%
\pgfsetfillopacity{0.700000}%
\pgfsetlinewidth{0.000000pt}%
\definecolor{currentstroke}{rgb}{0.000000,0.000000,0.000000}%
\pgfsetstrokecolor{currentstroke}%
\pgfsetdash{}{0pt}%
\pgfpathmoveto{\pgfqpoint{4.268284in}{1.722992in}}%
\pgfpathlineto{\pgfqpoint{4.282096in}{1.729590in}}%
\pgfpathlineto{\pgfqpoint{4.295922in}{1.736346in}}%
\pgfpathlineto{\pgfqpoint{4.309760in}{1.743259in}}%
\pgfpathlineto{\pgfqpoint{4.323612in}{1.750329in}}%
\pgfpathlineto{\pgfqpoint{4.331486in}{1.765839in}}%
\pgfpathlineto{\pgfqpoint{4.339356in}{1.781328in}}%
\pgfpathlineto{\pgfqpoint{4.347222in}{1.796792in}}%
\pgfpathlineto{\pgfqpoint{4.355084in}{1.812227in}}%
\pgfpathlineto{\pgfqpoint{4.341229in}{1.804727in}}%
\pgfpathlineto{\pgfqpoint{4.327387in}{1.797385in}}%
\pgfpathlineto{\pgfqpoint{4.313559in}{1.790201in}}%
\pgfpathlineto{\pgfqpoint{4.299745in}{1.783174in}}%
\pgfpathlineto{\pgfqpoint{4.291886in}{1.768158in}}%
\pgfpathlineto{\pgfqpoint{4.284022in}{1.753119in}}%
\pgfpathlineto{\pgfqpoint{4.276155in}{1.738063in}}%
\pgfpathlineto{\pgfqpoint{4.268284in}{1.722992in}}%
\pgfpathclose%
\pgfusepath{fill}%
\end{pgfscope}%
\begin{pgfscope}%
\pgfpathrectangle{\pgfqpoint{1.254980in}{0.150000in}}{\pgfqpoint{5.490039in}{5.490039in}}%
\pgfusepath{clip}%
\pgfsetbuttcap%
\pgfsetroundjoin%
\definecolor{currentfill}{rgb}{0.269944,0.014625,0.341379}%
\pgfsetfillcolor{currentfill}%
\pgfsetfillopacity{0.700000}%
\pgfsetlinewidth{0.000000pt}%
\definecolor{currentstroke}{rgb}{0.000000,0.000000,0.000000}%
\pgfsetstrokecolor{currentstroke}%
\pgfsetdash{}{0pt}%
\pgfpathmoveto{\pgfqpoint{3.457040in}{1.237779in}}%
\pgfpathlineto{\pgfqpoint{3.470636in}{1.232950in}}%
\pgfpathlineto{\pgfqpoint{3.484237in}{1.228283in}}%
\pgfpathlineto{\pgfqpoint{3.497842in}{1.223776in}}%
\pgfpathlineto{\pgfqpoint{3.511452in}{1.219430in}}%
\pgfpathlineto{\pgfqpoint{3.519628in}{1.225289in}}%
\pgfpathlineto{\pgfqpoint{3.527794in}{1.231409in}}%
\pgfpathlineto{\pgfqpoint{3.535951in}{1.237783in}}%
\pgfpathlineto{\pgfqpoint{3.544097in}{1.244403in}}%
\pgfpathlineto{\pgfqpoint{3.530511in}{1.248052in}}%
\pgfpathlineto{\pgfqpoint{3.516930in}{1.251861in}}%
\pgfpathlineto{\pgfqpoint{3.503353in}{1.255831in}}%
\pgfpathlineto{\pgfqpoint{3.489780in}{1.259962in}}%
\pgfpathlineto{\pgfqpoint{3.481611in}{1.254028in}}%
\pgfpathlineto{\pgfqpoint{3.473431in}{1.248348in}}%
\pgfpathlineto{\pgfqpoint{3.465241in}{1.242929in}}%
\pgfpathlineto{\pgfqpoint{3.457040in}{1.237779in}}%
\pgfpathclose%
\pgfusepath{fill}%
\end{pgfscope}%
\begin{pgfscope}%
\pgfpathrectangle{\pgfqpoint{1.254980in}{0.150000in}}{\pgfqpoint{5.490039in}{5.490039in}}%
\pgfusepath{clip}%
\pgfsetbuttcap%
\pgfsetroundjoin%
\definecolor{currentfill}{rgb}{0.273809,0.031497,0.358853}%
\pgfsetfillcolor{currentfill}%
\pgfsetfillopacity{0.700000}%
\pgfsetlinewidth{0.000000pt}%
\definecolor{currentstroke}{rgb}{0.000000,0.000000,0.000000}%
\pgfsetstrokecolor{currentstroke}%
\pgfsetdash{}{0pt}%
\pgfpathmoveto{\pgfqpoint{3.685324in}{1.255123in}}%
\pgfpathlineto{\pgfqpoint{3.698943in}{1.253573in}}%
\pgfpathlineto{\pgfqpoint{3.712570in}{1.252181in}}%
\pgfpathlineto{\pgfqpoint{3.726203in}{1.250946in}}%
\pgfpathlineto{\pgfqpoint{3.739842in}{1.249867in}}%
\pgfpathlineto{\pgfqpoint{3.747894in}{1.259577in}}%
\pgfpathlineto{\pgfqpoint{3.755939in}{1.269472in}}%
\pgfpathlineto{\pgfqpoint{3.763976in}{1.279547in}}%
\pgfpathlineto{\pgfqpoint{3.772007in}{1.289795in}}%
\pgfpathlineto{\pgfqpoint{3.758381in}{1.290231in}}%
\pgfpathlineto{\pgfqpoint{3.744762in}{1.290824in}}%
\pgfpathlineto{\pgfqpoint{3.731150in}{1.291574in}}%
\pgfpathlineto{\pgfqpoint{3.717546in}{1.292482in}}%
\pgfpathlineto{\pgfqpoint{3.709501in}{1.282866in}}%
\pgfpathlineto{\pgfqpoint{3.701450in}{1.273429in}}%
\pgfpathlineto{\pgfqpoint{3.693390in}{1.264180in}}%
\pgfpathlineto{\pgfqpoint{3.685324in}{1.255123in}}%
\pgfpathclose%
\pgfusepath{fill}%
\end{pgfscope}%
\begin{pgfscope}%
\pgfpathrectangle{\pgfqpoint{1.254980in}{0.150000in}}{\pgfqpoint{5.490039in}{5.490039in}}%
\pgfusepath{clip}%
\pgfsetbuttcap%
\pgfsetroundjoin%
\definecolor{currentfill}{rgb}{0.123444,0.636809,0.528763}%
\pgfsetfillcolor{currentfill}%
\pgfsetfillopacity{0.700000}%
\pgfsetlinewidth{0.000000pt}%
\definecolor{currentstroke}{rgb}{0.000000,0.000000,0.000000}%
\pgfsetstrokecolor{currentstroke}%
\pgfsetdash{}{0pt}%
\pgfpathmoveto{\pgfqpoint{4.977688in}{2.656737in}}%
\pgfpathlineto{\pgfqpoint{4.991937in}{2.670459in}}%
\pgfpathlineto{\pgfqpoint{5.006206in}{2.684345in}}%
\pgfpathlineto{\pgfqpoint{5.020494in}{2.698395in}}%
\pgfpathlineto{\pgfqpoint{5.034801in}{2.712610in}}%
\pgfpathlineto{\pgfqpoint{5.042443in}{2.724429in}}%
\pgfpathlineto{\pgfqpoint{5.050077in}{2.736072in}}%
\pgfpathlineto{\pgfqpoint{5.057703in}{2.747541in}}%
\pgfpathlineto{\pgfqpoint{5.065321in}{2.758834in}}%
\pgfpathlineto{\pgfqpoint{5.051012in}{2.744541in}}%
\pgfpathlineto{\pgfqpoint{5.036723in}{2.730413in}}%
\pgfpathlineto{\pgfqpoint{5.022454in}{2.716449in}}%
\pgfpathlineto{\pgfqpoint{5.008203in}{2.702650in}}%
\pgfpathlineto{\pgfqpoint{5.000586in}{2.691423in}}%
\pgfpathlineto{\pgfqpoint{4.992961in}{2.680028in}}%
\pgfpathlineto{\pgfqpoint{4.985328in}{2.668466in}}%
\pgfpathlineto{\pgfqpoint{4.977688in}{2.656737in}}%
\pgfpathclose%
\pgfusepath{fill}%
\end{pgfscope}%
\begin{pgfscope}%
\pgfpathrectangle{\pgfqpoint{1.254980in}{0.150000in}}{\pgfqpoint{5.490039in}{5.490039in}}%
\pgfusepath{clip}%
\pgfsetbuttcap%
\pgfsetroundjoin%
\definecolor{currentfill}{rgb}{0.165117,0.467423,0.558141}%
\pgfsetfillcolor{currentfill}%
\pgfsetfillopacity{0.700000}%
\pgfsetlinewidth{0.000000pt}%
\definecolor{currentstroke}{rgb}{0.000000,0.000000,0.000000}%
\pgfsetstrokecolor{currentstroke}%
\pgfsetdash{}{0pt}%
\pgfpathmoveto{\pgfqpoint{4.622987in}{2.189007in}}%
\pgfpathlineto{\pgfqpoint{4.637001in}{2.199668in}}%
\pgfpathlineto{\pgfqpoint{4.651030in}{2.210491in}}%
\pgfpathlineto{\pgfqpoint{4.665076in}{2.221474in}}%
\pgfpathlineto{\pgfqpoint{4.679139in}{2.232618in}}%
\pgfpathlineto{\pgfqpoint{4.686923in}{2.247482in}}%
\pgfpathlineto{\pgfqpoint{4.694703in}{2.262227in}}%
\pgfpathlineto{\pgfqpoint{4.702476in}{2.276851in}}%
\pgfpathlineto{\pgfqpoint{4.710245in}{2.291352in}}%
\pgfpathlineto{\pgfqpoint{4.696177in}{2.279947in}}%
\pgfpathlineto{\pgfqpoint{4.682126in}{2.268704in}}%
\pgfpathlineto{\pgfqpoint{4.668091in}{2.257623in}}%
\pgfpathlineto{\pgfqpoint{4.654073in}{2.246702in}}%
\pgfpathlineto{\pgfqpoint{4.646310in}{2.232450in}}%
\pgfpathlineto{\pgfqpoint{4.638541in}{2.218081in}}%
\pgfpathlineto{\pgfqpoint{4.630767in}{2.203600in}}%
\pgfpathlineto{\pgfqpoint{4.622987in}{2.189007in}}%
\pgfpathclose%
\pgfusepath{fill}%
\end{pgfscope}%
\begin{pgfscope}%
\pgfpathrectangle{\pgfqpoint{1.254980in}{0.150000in}}{\pgfqpoint{5.490039in}{5.490039in}}%
\pgfusepath{clip}%
\pgfsetbuttcap%
\pgfsetroundjoin%
\definecolor{currentfill}{rgb}{0.141935,0.526453,0.555991}%
\pgfsetfillcolor{currentfill}%
\pgfsetfillopacity{0.700000}%
\pgfsetlinewidth{0.000000pt}%
\definecolor{currentstroke}{rgb}{0.000000,0.000000,0.000000}%
\pgfsetstrokecolor{currentstroke}%
\pgfsetdash{}{0pt}%
\pgfpathmoveto{\pgfqpoint{4.741264in}{2.348085in}}%
\pgfpathlineto{\pgfqpoint{4.755354in}{2.359881in}}%
\pgfpathlineto{\pgfqpoint{4.769461in}{2.371840in}}%
\pgfpathlineto{\pgfqpoint{4.783586in}{2.383961in}}%
\pgfpathlineto{\pgfqpoint{4.797729in}{2.396244in}}%
\pgfpathlineto{\pgfqpoint{4.805474in}{2.410312in}}%
\pgfpathlineto{\pgfqpoint{4.813213in}{2.424238in}}%
\pgfpathlineto{\pgfqpoint{4.820946in}{2.438020in}}%
\pgfpathlineto{\pgfqpoint{4.828672in}{2.451656in}}%
\pgfpathlineto{\pgfqpoint{4.814525in}{2.439172in}}%
\pgfpathlineto{\pgfqpoint{4.800396in}{2.426850in}}%
\pgfpathlineto{\pgfqpoint{4.786284in}{2.414691in}}%
\pgfpathlineto{\pgfqpoint{4.772190in}{2.402695in}}%
\pgfpathlineto{\pgfqpoint{4.764467in}{2.389247in}}%
\pgfpathlineto{\pgfqpoint{4.756739in}{2.375662in}}%
\pgfpathlineto{\pgfqpoint{4.749004in}{2.361941in}}%
\pgfpathlineto{\pgfqpoint{4.741264in}{2.348085in}}%
\pgfpathclose%
\pgfusepath{fill}%
\end{pgfscope}%
\begin{pgfscope}%
\pgfpathrectangle{\pgfqpoint{1.254980in}{0.150000in}}{\pgfqpoint{5.490039in}{5.490039in}}%
\pgfusepath{clip}%
\pgfsetbuttcap%
\pgfsetroundjoin%
\definecolor{currentfill}{rgb}{0.123463,0.581687,0.547445}%
\pgfsetfillcolor{currentfill}%
\pgfsetfillopacity{0.700000}%
\pgfsetlinewidth{0.000000pt}%
\definecolor{currentstroke}{rgb}{0.000000,0.000000,0.000000}%
\pgfsetstrokecolor{currentstroke}%
\pgfsetdash{}{0pt}%
\pgfpathmoveto{\pgfqpoint{4.859515in}{2.504720in}}%
\pgfpathlineto{\pgfqpoint{4.873685in}{2.517538in}}%
\pgfpathlineto{\pgfqpoint{4.887872in}{2.530518in}}%
\pgfpathlineto{\pgfqpoint{4.902078in}{2.543663in}}%
\pgfpathlineto{\pgfqpoint{4.916303in}{2.556971in}}%
\pgfpathlineto{\pgfqpoint{4.924001in}{2.570011in}}%
\pgfpathlineto{\pgfqpoint{4.931692in}{2.582891in}}%
\pgfpathlineto{\pgfqpoint{4.939376in}{2.595608in}}%
\pgfpathlineto{\pgfqpoint{4.947053in}{2.608163in}}%
\pgfpathlineto{\pgfqpoint{4.932825in}{2.594715in}}%
\pgfpathlineto{\pgfqpoint{4.918616in}{2.581431in}}%
\pgfpathlineto{\pgfqpoint{4.904425in}{2.568310in}}%
\pgfpathlineto{\pgfqpoint{4.890253in}{2.555353in}}%
\pgfpathlineto{\pgfqpoint{4.882579in}{2.542926in}}%
\pgfpathlineto{\pgfqpoint{4.874898in}{2.530345in}}%
\pgfpathlineto{\pgfqpoint{4.867210in}{2.517609in}}%
\pgfpathlineto{\pgfqpoint{4.859515in}{2.504720in}}%
\pgfpathclose%
\pgfusepath{fill}%
\end{pgfscope}%
\begin{pgfscope}%
\pgfpathrectangle{\pgfqpoint{1.254980in}{0.150000in}}{\pgfqpoint{5.490039in}{5.490039in}}%
\pgfusepath{clip}%
\pgfsetbuttcap%
\pgfsetroundjoin%
\definecolor{currentfill}{rgb}{0.270595,0.214069,0.507052}%
\pgfsetfillcolor{currentfill}%
\pgfsetfillopacity{0.700000}%
\pgfsetlinewidth{0.000000pt}%
\definecolor{currentstroke}{rgb}{0.000000,0.000000,0.000000}%
\pgfsetstrokecolor{currentstroke}%
\pgfsetdash{}{0pt}%
\pgfpathmoveto{\pgfqpoint{4.150063in}{1.581364in}}%
\pgfpathlineto{\pgfqpoint{4.163824in}{1.586395in}}%
\pgfpathlineto{\pgfqpoint{4.177596in}{1.591583in}}%
\pgfpathlineto{\pgfqpoint{4.191381in}{1.596928in}}%
\pgfpathlineto{\pgfqpoint{4.205177in}{1.602428in}}%
\pgfpathlineto{\pgfqpoint{4.213079in}{1.617458in}}%
\pgfpathlineto{\pgfqpoint{4.220977in}{1.632509in}}%
\pgfpathlineto{\pgfqpoint{4.228871in}{1.647576in}}%
\pgfpathlineto{\pgfqpoint{4.236762in}{1.662654in}}%
\pgfpathlineto{\pgfqpoint{4.222964in}{1.656669in}}%
\pgfpathlineto{\pgfqpoint{4.209179in}{1.650841in}}%
\pgfpathlineto{\pgfqpoint{4.195406in}{1.645170in}}%
\pgfpathlineto{\pgfqpoint{4.181645in}{1.639656in}}%
\pgfpathlineto{\pgfqpoint{4.173756in}{1.625051in}}%
\pgfpathlineto{\pgfqpoint{4.165862in}{1.610464in}}%
\pgfpathlineto{\pgfqpoint{4.157965in}{1.595900in}}%
\pgfpathlineto{\pgfqpoint{4.150063in}{1.581364in}}%
\pgfpathclose%
\pgfusepath{fill}%
\end{pgfscope}%
\begin{pgfscope}%
\pgfpathrectangle{\pgfqpoint{1.254980in}{0.150000in}}{\pgfqpoint{5.490039in}{5.490039in}}%
\pgfusepath{clip}%
\pgfsetbuttcap%
\pgfsetroundjoin%
\definecolor{currentfill}{rgb}{0.283197,0.115680,0.436115}%
\pgfsetfillcolor{currentfill}%
\pgfsetfillopacity{0.700000}%
\pgfsetlinewidth{0.000000pt}%
\definecolor{currentstroke}{rgb}{0.000000,0.000000,0.000000}%
\pgfsetstrokecolor{currentstroke}%
\pgfsetdash{}{0pt}%
\pgfpathmoveto{\pgfqpoint{3.945174in}{1.389059in}}%
\pgfpathlineto{\pgfqpoint{3.958858in}{1.391233in}}%
\pgfpathlineto{\pgfqpoint{3.972552in}{1.393563in}}%
\pgfpathlineto{\pgfqpoint{3.986256in}{1.396049in}}%
\pgfpathlineto{\pgfqpoint{3.999969in}{1.398690in}}%
\pgfpathlineto{\pgfqpoint{4.007925in}{1.411944in}}%
\pgfpathlineto{\pgfqpoint{4.015875in}{1.425292in}}%
\pgfpathlineto{\pgfqpoint{4.023821in}{1.438728in}}%
\pgfpathlineto{\pgfqpoint{4.031763in}{1.452248in}}%
\pgfpathlineto{\pgfqpoint{4.018054in}{1.449044in}}%
\pgfpathlineto{\pgfqpoint{4.004355in}{1.445995in}}%
\pgfpathlineto{\pgfqpoint{3.990666in}{1.443102in}}%
\pgfpathlineto{\pgfqpoint{3.976987in}{1.440366in}}%
\pgfpathlineto{\pgfqpoint{3.969041in}{1.427398in}}%
\pgfpathlineto{\pgfqpoint{3.961090in}{1.414521in}}%
\pgfpathlineto{\pgfqpoint{3.953135in}{1.401739in}}%
\pgfpathlineto{\pgfqpoint{3.945174in}{1.389059in}}%
\pgfpathclose%
\pgfusepath{fill}%
\end{pgfscope}%
\begin{pgfscope}%
\pgfpathrectangle{\pgfqpoint{1.254980in}{0.150000in}}{\pgfqpoint{5.490039in}{5.490039in}}%
\pgfusepath{clip}%
\pgfsetbuttcap%
\pgfsetroundjoin%
\definecolor{currentfill}{rgb}{0.271305,0.019942,0.347269}%
\pgfsetfillcolor{currentfill}%
\pgfsetfillopacity{0.700000}%
\pgfsetlinewidth{0.000000pt}%
\definecolor{currentstroke}{rgb}{0.000000,0.000000,0.000000}%
\pgfsetstrokecolor{currentstroke}%
\pgfsetdash{}{0pt}%
\pgfpathmoveto{\pgfqpoint{3.598494in}{1.231408in}}%
\pgfpathlineto{\pgfqpoint{3.612107in}{1.228557in}}%
\pgfpathlineto{\pgfqpoint{3.625725in}{1.225864in}}%
\pgfpathlineto{\pgfqpoint{3.639350in}{1.223330in}}%
\pgfpathlineto{\pgfqpoint{3.652980in}{1.220953in}}%
\pgfpathlineto{\pgfqpoint{3.661078in}{1.229173in}}%
\pgfpathlineto{\pgfqpoint{3.669168in}{1.237613in}}%
\pgfpathlineto{\pgfqpoint{3.677250in}{1.246265in}}%
\pgfpathlineto{\pgfqpoint{3.685324in}{1.255123in}}%
\pgfpathlineto{\pgfqpoint{3.671711in}{1.256830in}}%
\pgfpathlineto{\pgfqpoint{3.658104in}{1.258695in}}%
\pgfpathlineto{\pgfqpoint{3.644504in}{1.260718in}}%
\pgfpathlineto{\pgfqpoint{3.630910in}{1.262900in}}%
\pgfpathlineto{\pgfqpoint{3.622819in}{1.254701in}}%
\pgfpathlineto{\pgfqpoint{3.614719in}{1.246715in}}%
\pgfpathlineto{\pgfqpoint{3.606611in}{1.238948in}}%
\pgfpathlineto{\pgfqpoint{3.598494in}{1.231408in}}%
\pgfpathclose%
\pgfusepath{fill}%
\end{pgfscope}%
\begin{pgfscope}%
\pgfpathrectangle{\pgfqpoint{1.254980in}{0.150000in}}{\pgfqpoint{5.490039in}{5.490039in}}%
\pgfusepath{clip}%
\pgfsetbuttcap%
\pgfsetroundjoin%
\definecolor{currentfill}{rgb}{0.304148,0.764704,0.419943}%
\pgfsetfillcolor{currentfill}%
\pgfsetfillopacity{0.700000}%
\pgfsetlinewidth{0.000000pt}%
\definecolor{currentstroke}{rgb}{0.000000,0.000000,0.000000}%
\pgfsetstrokecolor{currentstroke}%
\pgfsetdash{}{0pt}%
\pgfpathmoveto{\pgfqpoint{5.301205in}{3.035977in}}%
\pgfpathlineto{\pgfqpoint{5.315692in}{3.051759in}}%
\pgfpathlineto{\pgfqpoint{5.330200in}{3.067708in}}%
\pgfpathlineto{\pgfqpoint{5.344730in}{3.083824in}}%
\pgfpathlineto{\pgfqpoint{5.359281in}{3.100107in}}%
\pgfpathlineto{\pgfqpoint{5.366738in}{3.108238in}}%
\pgfpathlineto{\pgfqpoint{5.374184in}{3.116177in}}%
\pgfpathlineto{\pgfqpoint{5.381621in}{3.123926in}}%
\pgfpathlineto{\pgfqpoint{5.389047in}{3.131486in}}%
\pgfpathlineto{\pgfqpoint{5.374501in}{3.115285in}}%
\pgfpathlineto{\pgfqpoint{5.359977in}{3.099251in}}%
\pgfpathlineto{\pgfqpoint{5.345475in}{3.083383in}}%
\pgfpathlineto{\pgfqpoint{5.330994in}{3.067682in}}%
\pgfpathlineto{\pgfqpoint{5.323561in}{3.060029in}}%
\pgfpathlineto{\pgfqpoint{5.316119in}{3.052194in}}%
\pgfpathlineto{\pgfqpoint{5.308667in}{3.044178in}}%
\pgfpathlineto{\pgfqpoint{5.301205in}{3.035977in}}%
\pgfpathclose%
\pgfusepath{fill}%
\end{pgfscope}%
\begin{pgfscope}%
\pgfpathrectangle{\pgfqpoint{1.254980in}{0.150000in}}{\pgfqpoint{5.490039in}{5.490039in}}%
\pgfusepath{clip}%
\pgfsetbuttcap%
\pgfsetroundjoin%
\definecolor{currentfill}{rgb}{0.487026,0.823929,0.312321}%
\pgfsetfillcolor{currentfill}%
\pgfsetfillopacity{0.700000}%
\pgfsetlinewidth{0.000000pt}%
\definecolor{currentstroke}{rgb}{0.000000,0.000000,0.000000}%
\pgfsetstrokecolor{currentstroke}%
\pgfsetdash{}{0pt}%
\pgfpathmoveto{\pgfqpoint{5.506416in}{3.250778in}}%
\pgfpathlineto{\pgfqpoint{5.521056in}{3.267556in}}%
\pgfpathlineto{\pgfqpoint{5.535719in}{3.284502in}}%
\pgfpathlineto{\pgfqpoint{5.550405in}{3.301616in}}%
\pgfpathlineto{\pgfqpoint{5.565114in}{3.318899in}}%
\pgfpathlineto{\pgfqpoint{5.572424in}{3.324494in}}%
\pgfpathlineto{\pgfqpoint{5.579724in}{3.329902in}}%
\pgfpathlineto{\pgfqpoint{5.587011in}{3.335124in}}%
\pgfpathlineto{\pgfqpoint{5.594288in}{3.340163in}}%
\pgfpathlineto{\pgfqpoint{5.579590in}{3.323061in}}%
\pgfpathlineto{\pgfqpoint{5.564916in}{3.306127in}}%
\pgfpathlineto{\pgfqpoint{5.550264in}{3.289361in}}%
\pgfpathlineto{\pgfqpoint{5.535635in}{3.272762in}}%
\pgfpathlineto{\pgfqpoint{5.528347in}{3.267531in}}%
\pgfpathlineto{\pgfqpoint{5.521048in}{3.262125in}}%
\pgfpathlineto{\pgfqpoint{5.513737in}{3.256542in}}%
\pgfpathlineto{\pgfqpoint{5.506416in}{3.250778in}}%
\pgfpathclose%
\pgfusepath{fill}%
\end{pgfscope}%
\begin{pgfscope}%
\pgfpathrectangle{\pgfqpoint{1.254980in}{0.150000in}}{\pgfqpoint{5.490039in}{5.490039in}}%
\pgfusepath{clip}%
\pgfsetbuttcap%
\pgfsetroundjoin%
\definecolor{currentfill}{rgb}{0.281412,0.155834,0.469201}%
\pgfsetfillcolor{currentfill}%
\pgfsetfillopacity{0.700000}%
\pgfsetlinewidth{0.000000pt}%
\definecolor{currentstroke}{rgb}{0.000000,0.000000,0.000000}%
\pgfsetstrokecolor{currentstroke}%
\pgfsetdash{}{0pt}%
\pgfpathmoveto{\pgfqpoint{4.031763in}{1.452248in}}%
\pgfpathlineto{\pgfqpoint{4.045482in}{1.455609in}}%
\pgfpathlineto{\pgfqpoint{4.059212in}{1.459125in}}%
\pgfpathlineto{\pgfqpoint{4.072953in}{1.462797in}}%
\pgfpathlineto{\pgfqpoint{4.086705in}{1.466624in}}%
\pgfpathlineto{\pgfqpoint{4.094639in}{1.480770in}}%
\pgfpathlineto{\pgfqpoint{4.102569in}{1.494982in}}%
\pgfpathlineto{\pgfqpoint{4.110495in}{1.509255in}}%
\pgfpathlineto{\pgfqpoint{4.118417in}{1.523585in}}%
\pgfpathlineto{\pgfqpoint{4.104667in}{1.519219in}}%
\pgfpathlineto{\pgfqpoint{4.090928in}{1.515010in}}%
\pgfpathlineto{\pgfqpoint{4.077201in}{1.510957in}}%
\pgfpathlineto{\pgfqpoint{4.063484in}{1.507061in}}%
\pgfpathlineto{\pgfqpoint{4.055560in}{1.493258in}}%
\pgfpathlineto{\pgfqpoint{4.047632in}{1.479518in}}%
\pgfpathlineto{\pgfqpoint{4.039700in}{1.465847in}}%
\pgfpathlineto{\pgfqpoint{4.031763in}{1.452248in}}%
\pgfpathclose%
\pgfusepath{fill}%
\end{pgfscope}%
\begin{pgfscope}%
\pgfpathrectangle{\pgfqpoint{1.254980in}{0.150000in}}{\pgfqpoint{5.490039in}{5.490039in}}%
\pgfusepath{clip}%
\pgfsetbuttcap%
\pgfsetroundjoin%
\definecolor{currentfill}{rgb}{0.220124,0.725509,0.466226}%
\pgfsetfillcolor{currentfill}%
\pgfsetfillopacity{0.700000}%
\pgfsetlinewidth{0.000000pt}%
\definecolor{currentstroke}{rgb}{0.000000,0.000000,0.000000}%
\pgfsetstrokecolor{currentstroke}%
\pgfsetdash{}{0pt}%
\pgfpathmoveto{\pgfqpoint{5.183400in}{2.901922in}}%
\pgfpathlineto{\pgfqpoint{5.197809in}{2.917104in}}%
\pgfpathlineto{\pgfqpoint{5.212239in}{2.932453in}}%
\pgfpathlineto{\pgfqpoint{5.226690in}{2.947967in}}%
\pgfpathlineto{\pgfqpoint{5.241161in}{2.963649in}}%
\pgfpathlineto{\pgfqpoint{5.248700in}{2.973351in}}%
\pgfpathlineto{\pgfqpoint{5.256230in}{2.982863in}}%
\pgfpathlineto{\pgfqpoint{5.263750in}{2.992186in}}%
\pgfpathlineto{\pgfqpoint{5.271260in}{3.001319in}}%
\pgfpathlineto{\pgfqpoint{5.256791in}{2.985655in}}%
\pgfpathlineto{\pgfqpoint{5.242343in}{2.970157in}}%
\pgfpathlineto{\pgfqpoint{5.227915in}{2.954825in}}%
\pgfpathlineto{\pgfqpoint{5.213509in}{2.939659in}}%
\pgfpathlineto{\pgfqpoint{5.205996in}{2.930497in}}%
\pgfpathlineto{\pgfqpoint{5.198473in}{2.921154in}}%
\pgfpathlineto{\pgfqpoint{5.190941in}{2.911629in}}%
\pgfpathlineto{\pgfqpoint{5.183400in}{2.901922in}}%
\pgfpathclose%
\pgfusepath{fill}%
\end{pgfscope}%
\begin{pgfscope}%
\pgfpathrectangle{\pgfqpoint{1.254980in}{0.150000in}}{\pgfqpoint{5.490039in}{5.490039in}}%
\pgfusepath{clip}%
\pgfsetbuttcap%
\pgfsetroundjoin%
\definecolor{currentfill}{rgb}{0.227802,0.326594,0.546532}%
\pgfsetfillcolor{currentfill}%
\pgfsetfillopacity{0.700000}%
\pgfsetlinewidth{0.000000pt}%
\definecolor{currentstroke}{rgb}{0.000000,0.000000,0.000000}%
\pgfsetstrokecolor{currentstroke}%
\pgfsetdash{}{0pt}%
\pgfpathmoveto{\pgfqpoint{4.355084in}{1.812227in}}%
\pgfpathlineto{\pgfqpoint{4.368953in}{1.819885in}}%
\pgfpathlineto{\pgfqpoint{4.382835in}{1.827701in}}%
\pgfpathlineto{\pgfqpoint{4.396732in}{1.835674in}}%
\pgfpathlineto{\pgfqpoint{4.410643in}{1.843806in}}%
\pgfpathlineto{\pgfqpoint{4.418506in}{1.859622in}}%
\pgfpathlineto{\pgfqpoint{4.426364in}{1.875394in}}%
\pgfpathlineto{\pgfqpoint{4.434219in}{1.891119in}}%
\pgfpathlineto{\pgfqpoint{4.442069in}{1.906793in}}%
\pgfpathlineto{\pgfqpoint{4.428154in}{1.898258in}}%
\pgfpathlineto{\pgfqpoint{4.414253in}{1.889881in}}%
\pgfpathlineto{\pgfqpoint{4.400366in}{1.881663in}}%
\pgfpathlineto{\pgfqpoint{4.386493in}{1.873603in}}%
\pgfpathlineto{\pgfqpoint{4.378647in}{1.858321in}}%
\pgfpathlineto{\pgfqpoint{4.370796in}{1.842995in}}%
\pgfpathlineto{\pgfqpoint{4.362942in}{1.827630in}}%
\pgfpathlineto{\pgfqpoint{4.355084in}{1.812227in}}%
\pgfpathclose%
\pgfusepath{fill}%
\end{pgfscope}%
\begin{pgfscope}%
\pgfpathrectangle{\pgfqpoint{1.254980in}{0.150000in}}{\pgfqpoint{5.490039in}{5.490039in}}%
\pgfusepath{clip}%
\pgfsetbuttcap%
\pgfsetroundjoin%
\definecolor{currentfill}{rgb}{0.197636,0.391528,0.554969}%
\pgfsetfillcolor{currentfill}%
\pgfsetfillopacity{0.700000}%
\pgfsetlinewidth{0.000000pt}%
\definecolor{currentstroke}{rgb}{0.000000,0.000000,0.000000}%
\pgfsetstrokecolor{currentstroke}%
\pgfsetdash{}{0pt}%
\pgfpathmoveto{\pgfqpoint{4.473431in}{1.968909in}}%
\pgfpathlineto{\pgfqpoint{4.487366in}{1.977978in}}%
\pgfpathlineto{\pgfqpoint{4.501317in}{1.987206in}}%
\pgfpathlineto{\pgfqpoint{4.515282in}{1.996593in}}%
\pgfpathlineto{\pgfqpoint{4.529263in}{2.006139in}}%
\pgfpathlineto{\pgfqpoint{4.537098in}{2.021864in}}%
\pgfpathlineto{\pgfqpoint{4.544929in}{2.037511in}}%
\pgfpathlineto{\pgfqpoint{4.552756in}{2.053076in}}%
\pgfpathlineto{\pgfqpoint{4.560578in}{2.068556in}}%
\pgfpathlineto{\pgfqpoint{4.546592in}{2.058662in}}%
\pgfpathlineto{\pgfqpoint{4.532621in}{2.048928in}}%
\pgfpathlineto{\pgfqpoint{4.518665in}{2.039353in}}%
\pgfpathlineto{\pgfqpoint{4.504725in}{2.029938in}}%
\pgfpathlineto{\pgfqpoint{4.496908in}{2.014794in}}%
\pgfpathlineto{\pgfqpoint{4.489087in}{1.999572in}}%
\pgfpathlineto{\pgfqpoint{4.481261in}{1.984276in}}%
\pgfpathlineto{\pgfqpoint{4.473431in}{1.968909in}}%
\pgfpathclose%
\pgfusepath{fill}%
\end{pgfscope}%
\begin{pgfscope}%
\pgfpathrectangle{\pgfqpoint{1.254980in}{0.150000in}}{\pgfqpoint{5.490039in}{5.490039in}}%
\pgfusepath{clip}%
\pgfsetbuttcap%
\pgfsetroundjoin%
\definecolor{currentfill}{rgb}{0.255645,0.260703,0.528312}%
\pgfsetfillcolor{currentfill}%
\pgfsetfillopacity{0.700000}%
\pgfsetlinewidth{0.000000pt}%
\definecolor{currentstroke}{rgb}{0.000000,0.000000,0.000000}%
\pgfsetstrokecolor{currentstroke}%
\pgfsetdash{}{0pt}%
\pgfpathmoveto{\pgfqpoint{4.236762in}{1.662654in}}%
\pgfpathlineto{\pgfqpoint{4.250572in}{1.668795in}}%
\pgfpathlineto{\pgfqpoint{4.264395in}{1.675094in}}%
\pgfpathlineto{\pgfqpoint{4.278230in}{1.681549in}}%
\pgfpathlineto{\pgfqpoint{4.292079in}{1.688160in}}%
\pgfpathlineto{\pgfqpoint{4.299968in}{1.703713in}}%
\pgfpathlineto{\pgfqpoint{4.307853in}{1.719262in}}%
\pgfpathlineto{\pgfqpoint{4.315734in}{1.734802in}}%
\pgfpathlineto{\pgfqpoint{4.323612in}{1.750329in}}%
\pgfpathlineto{\pgfqpoint{4.309760in}{1.743259in}}%
\pgfpathlineto{\pgfqpoint{4.295922in}{1.736346in}}%
\pgfpathlineto{\pgfqpoint{4.282096in}{1.729590in}}%
\pgfpathlineto{\pgfqpoint{4.268284in}{1.722992in}}%
\pgfpathlineto{\pgfqpoint{4.260409in}{1.707912in}}%
\pgfpathlineto{\pgfqpoint{4.252531in}{1.692826in}}%
\pgfpathlineto{\pgfqpoint{4.244648in}{1.677739in}}%
\pgfpathlineto{\pgfqpoint{4.236762in}{1.662654in}}%
\pgfpathclose%
\pgfusepath{fill}%
\end{pgfscope}%
\begin{pgfscope}%
\pgfpathrectangle{\pgfqpoint{1.254980in}{0.150000in}}{\pgfqpoint{5.490039in}{5.490039in}}%
\pgfusepath{clip}%
\pgfsetbuttcap%
\pgfsetroundjoin%
\definecolor{currentfill}{rgb}{0.269944,0.014625,0.341379}%
\pgfsetfillcolor{currentfill}%
\pgfsetfillopacity{0.700000}%
\pgfsetlinewidth{0.000000pt}%
\definecolor{currentstroke}{rgb}{0.000000,0.000000,0.000000}%
\pgfsetstrokecolor{currentstroke}%
\pgfsetdash{}{0pt}%
\pgfpathmoveto{\pgfqpoint{3.511452in}{1.219430in}}%
\pgfpathlineto{\pgfqpoint{3.525066in}{1.215244in}}%
\pgfpathlineto{\pgfqpoint{3.538685in}{1.211218in}}%
\pgfpathlineto{\pgfqpoint{3.552308in}{1.207351in}}%
\pgfpathlineto{\pgfqpoint{3.565937in}{1.203644in}}%
\pgfpathlineto{\pgfqpoint{3.574090in}{1.210211in}}%
\pgfpathlineto{\pgfqpoint{3.582234in}{1.217032in}}%
\pgfpathlineto{\pgfqpoint{3.590369in}{1.224100in}}%
\pgfpathlineto{\pgfqpoint{3.598494in}{1.231408in}}%
\pgfpathlineto{\pgfqpoint{3.584887in}{1.234418in}}%
\pgfpathlineto{\pgfqpoint{3.571285in}{1.237587in}}%
\pgfpathlineto{\pgfqpoint{3.557689in}{1.240915in}}%
\pgfpathlineto{\pgfqpoint{3.544097in}{1.244403in}}%
\pgfpathlineto{\pgfqpoint{3.535951in}{1.237783in}}%
\pgfpathlineto{\pgfqpoint{3.527794in}{1.231409in}}%
\pgfpathlineto{\pgfqpoint{3.519628in}{1.225289in}}%
\pgfpathlineto{\pgfqpoint{3.511452in}{1.219430in}}%
\pgfpathclose%
\pgfusepath{fill}%
\end{pgfscope}%
\begin{pgfscope}%
\pgfpathrectangle{\pgfqpoint{1.254980in}{0.150000in}}{\pgfqpoint{5.490039in}{5.490039in}}%
\pgfusepath{clip}%
\pgfsetbuttcap%
\pgfsetroundjoin%
\definecolor{currentfill}{rgb}{0.575563,0.844566,0.256415}%
\pgfsetfillcolor{currentfill}%
\pgfsetfillopacity{0.700000}%
\pgfsetlinewidth{0.000000pt}%
\definecolor{currentstroke}{rgb}{0.000000,0.000000,0.000000}%
\pgfsetstrokecolor{currentstroke}%
\pgfsetdash{}{0pt}%
\pgfpathmoveto{\pgfqpoint{5.594288in}{3.340163in}}%
\pgfpathlineto{\pgfqpoint{5.609008in}{3.357434in}}%
\pgfpathlineto{\pgfqpoint{5.623752in}{3.374872in}}%
\pgfpathlineto{\pgfqpoint{5.638519in}{3.392480in}}%
\pgfpathlineto{\pgfqpoint{5.645774in}{3.397187in}}%
\pgfpathlineto{\pgfqpoint{5.653017in}{3.401710in}}%
\pgfpathlineto{\pgfqpoint{5.660249in}{3.406051in}}%
\pgfpathlineto{\pgfqpoint{5.667468in}{3.410212in}}%
\pgfpathlineto{\pgfqpoint{5.652715in}{3.392818in}}%
\pgfpathlineto{\pgfqpoint{5.637985in}{3.375593in}}%
\pgfpathlineto{\pgfqpoint{5.623277in}{3.358535in}}%
\pgfpathlineto{\pgfqpoint{5.616047in}{3.354205in}}%
\pgfpathlineto{\pgfqpoint{5.608805in}{3.349701in}}%
\pgfpathlineto{\pgfqpoint{5.601552in}{3.345021in}}%
\pgfpathlineto{\pgfqpoint{5.594288in}{3.340163in}}%
\pgfpathclose%
\pgfusepath{fill}%
\end{pgfscope}%
\begin{pgfscope}%
\pgfpathrectangle{\pgfqpoint{1.254980in}{0.150000in}}{\pgfqpoint{5.490039in}{5.490039in}}%
\pgfusepath{clip}%
\pgfsetbuttcap%
\pgfsetroundjoin%
\definecolor{currentfill}{rgb}{0.171176,0.452530,0.557965}%
\pgfsetfillcolor{currentfill}%
\pgfsetfillopacity{0.700000}%
\pgfsetlinewidth{0.000000pt}%
\definecolor{currentstroke}{rgb}{0.000000,0.000000,0.000000}%
\pgfsetstrokecolor{currentstroke}%
\pgfsetdash{}{0pt}%
\pgfpathmoveto{\pgfqpoint{4.591821in}{2.129572in}}%
\pgfpathlineto{\pgfqpoint{4.605829in}{2.139944in}}%
\pgfpathlineto{\pgfqpoint{4.619853in}{2.150477in}}%
\pgfpathlineto{\pgfqpoint{4.633893in}{2.161170in}}%
\pgfpathlineto{\pgfqpoint{4.647950in}{2.172024in}}%
\pgfpathlineto{\pgfqpoint{4.655755in}{2.187339in}}%
\pgfpathlineto{\pgfqpoint{4.663555in}{2.202544in}}%
\pgfpathlineto{\pgfqpoint{4.671349in}{2.217638in}}%
\pgfpathlineto{\pgfqpoint{4.679139in}{2.232618in}}%
\pgfpathlineto{\pgfqpoint{4.665076in}{2.221474in}}%
\pgfpathlineto{\pgfqpoint{4.651030in}{2.210491in}}%
\pgfpathlineto{\pgfqpoint{4.637001in}{2.199668in}}%
\pgfpathlineto{\pgfqpoint{4.622987in}{2.189007in}}%
\pgfpathlineto{\pgfqpoint{4.615203in}{2.174305in}}%
\pgfpathlineto{\pgfqpoint{4.607414in}{2.159497in}}%
\pgfpathlineto{\pgfqpoint{4.599620in}{2.144585in}}%
\pgfpathlineto{\pgfqpoint{4.591821in}{2.129572in}}%
\pgfpathclose%
\pgfusepath{fill}%
\end{pgfscope}%
\begin{pgfscope}%
\pgfpathrectangle{\pgfqpoint{1.254980in}{0.150000in}}{\pgfqpoint{5.490039in}{5.490039in}}%
\pgfusepath{clip}%
\pgfsetbuttcap%
\pgfsetroundjoin%
\definecolor{currentfill}{rgb}{0.153894,0.680203,0.504172}%
\pgfsetfillcolor{currentfill}%
\pgfsetfillopacity{0.700000}%
\pgfsetlinewidth{0.000000pt}%
\definecolor{currentstroke}{rgb}{0.000000,0.000000,0.000000}%
\pgfsetstrokecolor{currentstroke}%
\pgfsetdash{}{0pt}%
\pgfpathmoveto{\pgfqpoint{5.065321in}{2.758834in}}%
\pgfpathlineto{\pgfqpoint{5.079650in}{2.773293in}}%
\pgfpathlineto{\pgfqpoint{5.093998in}{2.787916in}}%
\pgfpathlineto{\pgfqpoint{5.108367in}{2.802705in}}%
\pgfpathlineto{\pgfqpoint{5.122756in}{2.817660in}}%
\pgfpathlineto{\pgfqpoint{5.130367in}{2.828836in}}%
\pgfpathlineto{\pgfqpoint{5.137969in}{2.839828in}}%
\pgfpathlineto{\pgfqpoint{5.145563in}{2.850636in}}%
\pgfpathlineto{\pgfqpoint{5.153148in}{2.861261in}}%
\pgfpathlineto{\pgfqpoint{5.138759in}{2.846259in}}%
\pgfpathlineto{\pgfqpoint{5.124390in}{2.831424in}}%
\pgfpathlineto{\pgfqpoint{5.110041in}{2.816753in}}%
\pgfpathlineto{\pgfqpoint{5.095712in}{2.802248in}}%
\pgfpathlineto{\pgfqpoint{5.088127in}{2.791659in}}%
\pgfpathlineto{\pgfqpoint{5.080533in}{2.780893in}}%
\pgfpathlineto{\pgfqpoint{5.072931in}{2.769952in}}%
\pgfpathlineto{\pgfqpoint{5.065321in}{2.758834in}}%
\pgfpathclose%
\pgfusepath{fill}%
\end{pgfscope}%
\begin{pgfscope}%
\pgfpathrectangle{\pgfqpoint{1.254980in}{0.150000in}}{\pgfqpoint{5.490039in}{5.490039in}}%
\pgfusepath{clip}%
\pgfsetbuttcap%
\pgfsetroundjoin%
\definecolor{currentfill}{rgb}{0.147607,0.511733,0.557049}%
\pgfsetfillcolor{currentfill}%
\pgfsetfillopacity{0.700000}%
\pgfsetlinewidth{0.000000pt}%
\definecolor{currentstroke}{rgb}{0.000000,0.000000,0.000000}%
\pgfsetstrokecolor{currentstroke}%
\pgfsetdash{}{0pt}%
\pgfpathmoveto{\pgfqpoint{4.710245in}{2.291352in}}%
\pgfpathlineto{\pgfqpoint{4.724330in}{2.302918in}}%
\pgfpathlineto{\pgfqpoint{4.738432in}{2.314646in}}%
\pgfpathlineto{\pgfqpoint{4.752552in}{2.326536in}}%
\pgfpathlineto{\pgfqpoint{4.766689in}{2.338588in}}%
\pgfpathlineto{\pgfqpoint{4.774458in}{2.353206in}}%
\pgfpathlineto{\pgfqpoint{4.782221in}{2.367690in}}%
\pgfpathlineto{\pgfqpoint{4.789977in}{2.382036in}}%
\pgfpathlineto{\pgfqpoint{4.797729in}{2.396244in}}%
\pgfpathlineto{\pgfqpoint{4.783586in}{2.383961in}}%
\pgfpathlineto{\pgfqpoint{4.769461in}{2.371840in}}%
\pgfpathlineto{\pgfqpoint{4.755354in}{2.359881in}}%
\pgfpathlineto{\pgfqpoint{4.741264in}{2.348085in}}%
\pgfpathlineto{\pgfqpoint{4.733517in}{2.334096in}}%
\pgfpathlineto{\pgfqpoint{4.725765in}{2.319977in}}%
\pgfpathlineto{\pgfqpoint{4.718008in}{2.305728in}}%
\pgfpathlineto{\pgfqpoint{4.710245in}{2.291352in}}%
\pgfpathclose%
\pgfusepath{fill}%
\end{pgfscope}%
\begin{pgfscope}%
\pgfpathrectangle{\pgfqpoint{1.254980in}{0.150000in}}{\pgfqpoint{5.490039in}{5.490039in}}%
\pgfusepath{clip}%
\pgfsetbuttcap%
\pgfsetroundjoin%
\definecolor{currentfill}{rgb}{0.275191,0.194905,0.496005}%
\pgfsetfillcolor{currentfill}%
\pgfsetfillopacity{0.700000}%
\pgfsetlinewidth{0.000000pt}%
\definecolor{currentstroke}{rgb}{0.000000,0.000000,0.000000}%
\pgfsetstrokecolor{currentstroke}%
\pgfsetdash{}{0pt}%
\pgfpathmoveto{\pgfqpoint{4.118417in}{1.523585in}}%
\pgfpathlineto{\pgfqpoint{4.132178in}{1.528106in}}%
\pgfpathlineto{\pgfqpoint{4.145951in}{1.532783in}}%
\pgfpathlineto{\pgfqpoint{4.159735in}{1.537615in}}%
\pgfpathlineto{\pgfqpoint{4.173531in}{1.542604in}}%
\pgfpathlineto{\pgfqpoint{4.181448in}{1.557506in}}%
\pgfpathlineto{\pgfqpoint{4.189362in}{1.572447in}}%
\pgfpathlineto{\pgfqpoint{4.197271in}{1.587423in}}%
\pgfpathlineto{\pgfqpoint{4.205177in}{1.602428in}}%
\pgfpathlineto{\pgfqpoint{4.191381in}{1.596928in}}%
\pgfpathlineto{\pgfqpoint{4.177596in}{1.591583in}}%
\pgfpathlineto{\pgfqpoint{4.163824in}{1.586395in}}%
\pgfpathlineto{\pgfqpoint{4.150063in}{1.581364in}}%
\pgfpathlineto{\pgfqpoint{4.142158in}{1.566859in}}%
\pgfpathlineto{\pgfqpoint{4.134248in}{1.552391in}}%
\pgfpathlineto{\pgfqpoint{4.126335in}{1.537965in}}%
\pgfpathlineto{\pgfqpoint{4.118417in}{1.523585in}}%
\pgfpathclose%
\pgfusepath{fill}%
\end{pgfscope}%
\begin{pgfscope}%
\pgfpathrectangle{\pgfqpoint{1.254980in}{0.150000in}}{\pgfqpoint{5.490039in}{5.490039in}}%
\pgfusepath{clip}%
\pgfsetbuttcap%
\pgfsetroundjoin%
\definecolor{currentfill}{rgb}{0.280267,0.073417,0.397163}%
\pgfsetfillcolor{currentfill}%
\pgfsetfillopacity{0.700000}%
\pgfsetlinewidth{0.000000pt}%
\definecolor{currentstroke}{rgb}{0.000000,0.000000,0.000000}%
\pgfsetstrokecolor{currentstroke}%
\pgfsetdash{}{0pt}%
\pgfpathmoveto{\pgfqpoint{3.826590in}{1.289616in}}%
\pgfpathlineto{\pgfqpoint{3.840255in}{1.289962in}}%
\pgfpathlineto{\pgfqpoint{3.853929in}{1.290463in}}%
\pgfpathlineto{\pgfqpoint{3.867611in}{1.291119in}}%
\pgfpathlineto{\pgfqpoint{3.881301in}{1.291931in}}%
\pgfpathlineto{\pgfqpoint{3.889304in}{1.303599in}}%
\pgfpathlineto{\pgfqpoint{3.897302in}{1.315414in}}%
\pgfpathlineto{\pgfqpoint{3.905294in}{1.327369in}}%
\pgfpathlineto{\pgfqpoint{3.913281in}{1.339460in}}%
\pgfpathlineto{\pgfqpoint{3.899599in}{1.338031in}}%
\pgfpathlineto{\pgfqpoint{3.885926in}{1.336758in}}%
\pgfpathlineto{\pgfqpoint{3.872261in}{1.335640in}}%
\pgfpathlineto{\pgfqpoint{3.858606in}{1.334678in}}%
\pgfpathlineto{\pgfqpoint{3.850611in}{1.323194in}}%
\pgfpathlineto{\pgfqpoint{3.842610in}{1.311851in}}%
\pgfpathlineto{\pgfqpoint{3.834603in}{1.300657in}}%
\pgfpathlineto{\pgfqpoint{3.826590in}{1.289616in}}%
\pgfpathclose%
\pgfusepath{fill}%
\end{pgfscope}%
\begin{pgfscope}%
\pgfpathrectangle{\pgfqpoint{1.254980in}{0.150000in}}{\pgfqpoint{5.490039in}{5.490039in}}%
\pgfusepath{clip}%
\pgfsetbuttcap%
\pgfsetroundjoin%
\definecolor{currentfill}{rgb}{0.276022,0.044167,0.370164}%
\pgfsetfillcolor{currentfill}%
\pgfsetfillopacity{0.700000}%
\pgfsetlinewidth{0.000000pt}%
\definecolor{currentstroke}{rgb}{0.000000,0.000000,0.000000}%
\pgfsetstrokecolor{currentstroke}%
\pgfsetdash{}{0pt}%
\pgfpathmoveto{\pgfqpoint{3.739842in}{1.249867in}}%
\pgfpathlineto{\pgfqpoint{3.753490in}{1.248945in}}%
\pgfpathlineto{\pgfqpoint{3.767144in}{1.248180in}}%
\pgfpathlineto{\pgfqpoint{3.780806in}{1.247570in}}%
\pgfpathlineto{\pgfqpoint{3.794475in}{1.247116in}}%
\pgfpathlineto{\pgfqpoint{3.802513in}{1.257479in}}%
\pgfpathlineto{\pgfqpoint{3.810545in}{1.268021in}}%
\pgfpathlineto{\pgfqpoint{3.818570in}{1.278735in}}%
\pgfpathlineto{\pgfqpoint{3.826590in}{1.289616in}}%
\pgfpathlineto{\pgfqpoint{3.812932in}{1.289426in}}%
\pgfpathlineto{\pgfqpoint{3.799283in}{1.289393in}}%
\pgfpathlineto{\pgfqpoint{3.785641in}{1.289515in}}%
\pgfpathlineto{\pgfqpoint{3.772007in}{1.289795in}}%
\pgfpathlineto{\pgfqpoint{3.763976in}{1.279547in}}%
\pgfpathlineto{\pgfqpoint{3.755939in}{1.269472in}}%
\pgfpathlineto{\pgfqpoint{3.747894in}{1.259577in}}%
\pgfpathlineto{\pgfqpoint{3.739842in}{1.249867in}}%
\pgfpathclose%
\pgfusepath{fill}%
\end{pgfscope}%
\begin{pgfscope}%
\pgfpathrectangle{\pgfqpoint{1.254980in}{0.150000in}}{\pgfqpoint{5.490039in}{5.490039in}}%
\pgfusepath{clip}%
\pgfsetbuttcap%
\pgfsetroundjoin%
\definecolor{currentfill}{rgb}{0.120638,0.625828,0.533488}%
\pgfsetfillcolor{currentfill}%
\pgfsetfillopacity{0.700000}%
\pgfsetlinewidth{0.000000pt}%
\definecolor{currentstroke}{rgb}{0.000000,0.000000,0.000000}%
\pgfsetstrokecolor{currentstroke}%
\pgfsetdash{}{0pt}%
\pgfpathmoveto{\pgfqpoint{4.947053in}{2.608163in}}%
\pgfpathlineto{\pgfqpoint{4.961300in}{2.621775in}}%
\pgfpathlineto{\pgfqpoint{4.975566in}{2.635552in}}%
\pgfpathlineto{\pgfqpoint{4.989851in}{2.649493in}}%
\pgfpathlineto{\pgfqpoint{5.004156in}{2.663598in}}%
\pgfpathlineto{\pgfqpoint{5.011828in}{2.676110in}}%
\pgfpathlineto{\pgfqpoint{5.019494in}{2.688450in}}%
\pgfpathlineto{\pgfqpoint{5.027151in}{2.700617in}}%
\pgfpathlineto{\pgfqpoint{5.034801in}{2.712610in}}%
\pgfpathlineto{\pgfqpoint{5.020494in}{2.698395in}}%
\pgfpathlineto{\pgfqpoint{5.006206in}{2.684345in}}%
\pgfpathlineto{\pgfqpoint{4.991937in}{2.670459in}}%
\pgfpathlineto{\pgfqpoint{4.977688in}{2.656737in}}%
\pgfpathlineto{\pgfqpoint{4.970040in}{2.644842in}}%
\pgfpathlineto{\pgfqpoint{4.962385in}{2.632781in}}%
\pgfpathlineto{\pgfqpoint{4.954723in}{2.620554in}}%
\pgfpathlineto{\pgfqpoint{4.947053in}{2.608163in}}%
\pgfpathclose%
\pgfusepath{fill}%
\end{pgfscope}%
\begin{pgfscope}%
\pgfpathrectangle{\pgfqpoint{1.254980in}{0.150000in}}{\pgfqpoint{5.490039in}{5.490039in}}%
\pgfusepath{clip}%
\pgfsetbuttcap%
\pgfsetroundjoin%
\definecolor{currentfill}{rgb}{0.126453,0.570633,0.549841}%
\pgfsetfillcolor{currentfill}%
\pgfsetfillopacity{0.700000}%
\pgfsetlinewidth{0.000000pt}%
\definecolor{currentstroke}{rgb}{0.000000,0.000000,0.000000}%
\pgfsetstrokecolor{currentstroke}%
\pgfsetdash{}{0pt}%
\pgfpathmoveto{\pgfqpoint{4.828672in}{2.451656in}}%
\pgfpathlineto{\pgfqpoint{4.842838in}{2.464303in}}%
\pgfpathlineto{\pgfqpoint{4.857021in}{2.477113in}}%
\pgfpathlineto{\pgfqpoint{4.871223in}{2.490087in}}%
\pgfpathlineto{\pgfqpoint{4.885443in}{2.503223in}}%
\pgfpathlineto{\pgfqpoint{4.893168in}{2.516896in}}%
\pgfpathlineto{\pgfqpoint{4.900886in}{2.530412in}}%
\pgfpathlineto{\pgfqpoint{4.908598in}{2.543771in}}%
\pgfpathlineto{\pgfqpoint{4.916303in}{2.556971in}}%
\pgfpathlineto{\pgfqpoint{4.902078in}{2.543663in}}%
\pgfpathlineto{\pgfqpoint{4.887872in}{2.530518in}}%
\pgfpathlineto{\pgfqpoint{4.873685in}{2.517538in}}%
\pgfpathlineto{\pgfqpoint{4.859515in}{2.504720in}}%
\pgfpathlineto{\pgfqpoint{4.851814in}{2.491679in}}%
\pgfpathlineto{\pgfqpoint{4.844107in}{2.478487in}}%
\pgfpathlineto{\pgfqpoint{4.836393in}{2.465146in}}%
\pgfpathlineto{\pgfqpoint{4.828672in}{2.451656in}}%
\pgfpathclose%
\pgfusepath{fill}%
\end{pgfscope}%
\begin{pgfscope}%
\pgfpathrectangle{\pgfqpoint{1.254980in}{0.150000in}}{\pgfqpoint{5.490039in}{5.490039in}}%
\pgfusepath{clip}%
\pgfsetbuttcap%
\pgfsetroundjoin%
\definecolor{currentfill}{rgb}{0.395174,0.797475,0.367757}%
\pgfsetfillcolor{currentfill}%
\pgfsetfillopacity{0.700000}%
\pgfsetlinewidth{0.000000pt}%
\definecolor{currentstroke}{rgb}{0.000000,0.000000,0.000000}%
\pgfsetstrokecolor{currentstroke}%
\pgfsetdash{}{0pt}%
\pgfpathmoveto{\pgfqpoint{5.389047in}{3.131486in}}%
\pgfpathlineto{\pgfqpoint{5.403615in}{3.147855in}}%
\pgfpathlineto{\pgfqpoint{5.418204in}{3.164391in}}%
\pgfpathlineto{\pgfqpoint{5.432816in}{3.181095in}}%
\pgfpathlineto{\pgfqpoint{5.447451in}{3.197968in}}%
\pgfpathlineto{\pgfqpoint{5.454860in}{3.205237in}}%
\pgfpathlineto{\pgfqpoint{5.462258in}{3.212313in}}%
\pgfpathlineto{\pgfqpoint{5.469645in}{3.219196in}}%
\pgfpathlineto{\pgfqpoint{5.477021in}{3.225887in}}%
\pgfpathlineto{\pgfqpoint{5.462394in}{3.209130in}}%
\pgfpathlineto{\pgfqpoint{5.447789in}{3.192541in}}%
\pgfpathlineto{\pgfqpoint{5.433207in}{3.176119in}}%
\pgfpathlineto{\pgfqpoint{5.418647in}{3.159864in}}%
\pgfpathlineto{\pgfqpoint{5.411262in}{3.153046in}}%
\pgfpathlineto{\pgfqpoint{5.403868in}{3.146044in}}%
\pgfpathlineto{\pgfqpoint{5.396462in}{3.138858in}}%
\pgfpathlineto{\pgfqpoint{5.389047in}{3.131486in}}%
\pgfpathclose%
\pgfusepath{fill}%
\end{pgfscope}%
\begin{pgfscope}%
\pgfpathrectangle{\pgfqpoint{1.254980in}{0.150000in}}{\pgfqpoint{5.490039in}{5.490039in}}%
\pgfusepath{clip}%
\pgfsetbuttcap%
\pgfsetroundjoin%
\definecolor{currentfill}{rgb}{0.282656,0.100196,0.422160}%
\pgfsetfillcolor{currentfill}%
\pgfsetfillopacity{0.700000}%
\pgfsetlinewidth{0.000000pt}%
\definecolor{currentstroke}{rgb}{0.000000,0.000000,0.000000}%
\pgfsetstrokecolor{currentstroke}%
\pgfsetdash{}{0pt}%
\pgfpathmoveto{\pgfqpoint{3.913281in}{1.339460in}}%
\pgfpathlineto{\pgfqpoint{3.926972in}{1.341045in}}%
\pgfpathlineto{\pgfqpoint{3.940672in}{1.342785in}}%
\pgfpathlineto{\pgfqpoint{3.954381in}{1.344680in}}%
\pgfpathlineto{\pgfqpoint{3.968100in}{1.346730in}}%
\pgfpathlineto{\pgfqpoint{3.976075in}{1.359551in}}%
\pgfpathlineto{\pgfqpoint{3.984045in}{1.372488in}}%
\pgfpathlineto{\pgfqpoint{3.992009in}{1.385537in}}%
\pgfpathlineto{\pgfqpoint{3.999969in}{1.398690in}}%
\pgfpathlineto{\pgfqpoint{3.986256in}{1.396049in}}%
\pgfpathlineto{\pgfqpoint{3.972552in}{1.393563in}}%
\pgfpathlineto{\pgfqpoint{3.958858in}{1.391233in}}%
\pgfpathlineto{\pgfqpoint{3.945174in}{1.389059in}}%
\pgfpathlineto{\pgfqpoint{3.937209in}{1.376485in}}%
\pgfpathlineto{\pgfqpoint{3.929238in}{1.364024in}}%
\pgfpathlineto{\pgfqpoint{3.921262in}{1.351680in}}%
\pgfpathlineto{\pgfqpoint{3.913281in}{1.339460in}}%
\pgfpathclose%
\pgfusepath{fill}%
\end{pgfscope}%
\begin{pgfscope}%
\pgfpathrectangle{\pgfqpoint{1.254980in}{0.150000in}}{\pgfqpoint{5.490039in}{5.490039in}}%
\pgfusepath{clip}%
\pgfsetbuttcap%
\pgfsetroundjoin%
\definecolor{currentfill}{rgb}{0.272594,0.025563,0.353093}%
\pgfsetfillcolor{currentfill}%
\pgfsetfillopacity{0.700000}%
\pgfsetlinewidth{0.000000pt}%
\definecolor{currentstroke}{rgb}{0.000000,0.000000,0.000000}%
\pgfsetstrokecolor{currentstroke}%
\pgfsetdash{}{0pt}%
\pgfpathmoveto{\pgfqpoint{3.652980in}{1.220953in}}%
\pgfpathlineto{\pgfqpoint{3.666616in}{1.218733in}}%
\pgfpathlineto{\pgfqpoint{3.680259in}{1.216671in}}%
\pgfpathlineto{\pgfqpoint{3.693908in}{1.214765in}}%
\pgfpathlineto{\pgfqpoint{3.707563in}{1.213016in}}%
\pgfpathlineto{\pgfqpoint{3.715644in}{1.221918in}}%
\pgfpathlineto{\pgfqpoint{3.723718in}{1.231031in}}%
\pgfpathlineto{\pgfqpoint{3.731784in}{1.240350in}}%
\pgfpathlineto{\pgfqpoint{3.739842in}{1.249867in}}%
\pgfpathlineto{\pgfqpoint{3.726203in}{1.250946in}}%
\pgfpathlineto{\pgfqpoint{3.712570in}{1.252181in}}%
\pgfpathlineto{\pgfqpoint{3.698943in}{1.253573in}}%
\pgfpathlineto{\pgfqpoint{3.685324in}{1.255123in}}%
\pgfpathlineto{\pgfqpoint{3.677250in}{1.246265in}}%
\pgfpathlineto{\pgfqpoint{3.669168in}{1.237613in}}%
\pgfpathlineto{\pgfqpoint{3.661078in}{1.229173in}}%
\pgfpathlineto{\pgfqpoint{3.652980in}{1.220953in}}%
\pgfpathclose%
\pgfusepath{fill}%
\end{pgfscope}%
\begin{pgfscope}%
\pgfpathrectangle{\pgfqpoint{1.254980in}{0.150000in}}{\pgfqpoint{5.490039in}{5.490039in}}%
\pgfusepath{clip}%
\pgfsetbuttcap%
\pgfsetroundjoin%
\definecolor{currentfill}{rgb}{0.282884,0.135920,0.453427}%
\pgfsetfillcolor{currentfill}%
\pgfsetfillopacity{0.700000}%
\pgfsetlinewidth{0.000000pt}%
\definecolor{currentstroke}{rgb}{0.000000,0.000000,0.000000}%
\pgfsetstrokecolor{currentstroke}%
\pgfsetdash{}{0pt}%
\pgfpathmoveto{\pgfqpoint{3.999969in}{1.398690in}}%
\pgfpathlineto{\pgfqpoint{4.013693in}{1.401487in}}%
\pgfpathlineto{\pgfqpoint{4.027427in}{1.404439in}}%
\pgfpathlineto{\pgfqpoint{4.041170in}{1.407546in}}%
\pgfpathlineto{\pgfqpoint{4.054925in}{1.410808in}}%
\pgfpathlineto{\pgfqpoint{4.062876in}{1.424636in}}%
\pgfpathlineto{\pgfqpoint{4.070824in}{1.438552in}}%
\pgfpathlineto{\pgfqpoint{4.078766in}{1.452550in}}%
\pgfpathlineto{\pgfqpoint{4.086705in}{1.466624in}}%
\pgfpathlineto{\pgfqpoint{4.072953in}{1.462797in}}%
\pgfpathlineto{\pgfqpoint{4.059212in}{1.459125in}}%
\pgfpathlineto{\pgfqpoint{4.045482in}{1.455609in}}%
\pgfpathlineto{\pgfqpoint{4.031763in}{1.452248in}}%
\pgfpathlineto{\pgfqpoint{4.023821in}{1.438728in}}%
\pgfpathlineto{\pgfqpoint{4.015875in}{1.425292in}}%
\pgfpathlineto{\pgfqpoint{4.007925in}{1.411944in}}%
\pgfpathlineto{\pgfqpoint{3.999969in}{1.398690in}}%
\pgfpathclose%
\pgfusepath{fill}%
\end{pgfscope}%
\begin{pgfscope}%
\pgfpathrectangle{\pgfqpoint{1.254980in}{0.150000in}}{\pgfqpoint{5.490039in}{5.490039in}}%
\pgfusepath{clip}%
\pgfsetbuttcap%
\pgfsetroundjoin%
\definecolor{currentfill}{rgb}{0.237441,0.305202,0.541921}%
\pgfsetfillcolor{currentfill}%
\pgfsetfillopacity{0.700000}%
\pgfsetlinewidth{0.000000pt}%
\definecolor{currentstroke}{rgb}{0.000000,0.000000,0.000000}%
\pgfsetstrokecolor{currentstroke}%
\pgfsetdash{}{0pt}%
\pgfpathmoveto{\pgfqpoint{4.323612in}{1.750329in}}%
\pgfpathlineto{\pgfqpoint{4.337477in}{1.757556in}}%
\pgfpathlineto{\pgfqpoint{4.351356in}{1.764941in}}%
\pgfpathlineto{\pgfqpoint{4.365249in}{1.772483in}}%
\pgfpathlineto{\pgfqpoint{4.379155in}{1.780183in}}%
\pgfpathlineto{\pgfqpoint{4.387033in}{1.796135in}}%
\pgfpathlineto{\pgfqpoint{4.394907in}{1.812058in}}%
\pgfpathlineto{\pgfqpoint{4.402777in}{1.827950in}}%
\pgfpathlineto{\pgfqpoint{4.410643in}{1.843806in}}%
\pgfpathlineto{\pgfqpoint{4.396732in}{1.835674in}}%
\pgfpathlineto{\pgfqpoint{4.382835in}{1.827701in}}%
\pgfpathlineto{\pgfqpoint{4.368953in}{1.819885in}}%
\pgfpathlineto{\pgfqpoint{4.355084in}{1.812227in}}%
\pgfpathlineto{\pgfqpoint{4.347222in}{1.796792in}}%
\pgfpathlineto{\pgfqpoint{4.339356in}{1.781328in}}%
\pgfpathlineto{\pgfqpoint{4.331486in}{1.765839in}}%
\pgfpathlineto{\pgfqpoint{4.323612in}{1.750329in}}%
\pgfpathclose%
\pgfusepath{fill}%
\end{pgfscope}%
\begin{pgfscope}%
\pgfpathrectangle{\pgfqpoint{1.254980in}{0.150000in}}{\pgfqpoint{5.490039in}{5.490039in}}%
\pgfusepath{clip}%
\pgfsetbuttcap%
\pgfsetroundjoin%
\definecolor{currentfill}{rgb}{0.304148,0.764704,0.419943}%
\pgfsetfillcolor{currentfill}%
\pgfsetfillopacity{0.700000}%
\pgfsetlinewidth{0.000000pt}%
\definecolor{currentstroke}{rgb}{0.000000,0.000000,0.000000}%
\pgfsetstrokecolor{currentstroke}%
\pgfsetdash{}{0pt}%
\pgfpathmoveto{\pgfqpoint{5.271260in}{3.001319in}}%
\pgfpathlineto{\pgfqpoint{5.285751in}{3.017150in}}%
\pgfpathlineto{\pgfqpoint{5.300262in}{3.033148in}}%
\pgfpathlineto{\pgfqpoint{5.314796in}{3.049314in}}%
\pgfpathlineto{\pgfqpoint{5.329351in}{3.065647in}}%
\pgfpathlineto{\pgfqpoint{5.336849in}{3.074554in}}%
\pgfpathlineto{\pgfqpoint{5.344336in}{3.083266in}}%
\pgfpathlineto{\pgfqpoint{5.351814in}{3.091783in}}%
\pgfpathlineto{\pgfqpoint{5.359281in}{3.100107in}}%
\pgfpathlineto{\pgfqpoint{5.344730in}{3.083824in}}%
\pgfpathlineto{\pgfqpoint{5.330200in}{3.067708in}}%
\pgfpathlineto{\pgfqpoint{5.315692in}{3.051759in}}%
\pgfpathlineto{\pgfqpoint{5.301205in}{3.035977in}}%
\pgfpathlineto{\pgfqpoint{5.293734in}{3.027592in}}%
\pgfpathlineto{\pgfqpoint{5.286252in}{3.019022in}}%
\pgfpathlineto{\pgfqpoint{5.278761in}{3.010264in}}%
\pgfpathlineto{\pgfqpoint{5.271260in}{3.001319in}}%
\pgfpathclose%
\pgfusepath{fill}%
\end{pgfscope}%
\begin{pgfscope}%
\pgfpathrectangle{\pgfqpoint{1.254980in}{0.150000in}}{\pgfqpoint{5.490039in}{5.490039in}}%
\pgfusepath{clip}%
\pgfsetbuttcap%
\pgfsetroundjoin%
\definecolor{currentfill}{rgb}{0.206756,0.371758,0.553117}%
\pgfsetfillcolor{currentfill}%
\pgfsetfillopacity{0.700000}%
\pgfsetlinewidth{0.000000pt}%
\definecolor{currentstroke}{rgb}{0.000000,0.000000,0.000000}%
\pgfsetstrokecolor{currentstroke}%
\pgfsetdash{}{0pt}%
\pgfpathmoveto{\pgfqpoint{4.442069in}{1.906793in}}%
\pgfpathlineto{\pgfqpoint{4.456000in}{1.915487in}}%
\pgfpathlineto{\pgfqpoint{4.469945in}{1.924339in}}%
\pgfpathlineto{\pgfqpoint{4.483905in}{1.933350in}}%
\pgfpathlineto{\pgfqpoint{4.497880in}{1.942519in}}%
\pgfpathlineto{\pgfqpoint{4.505732in}{1.958526in}}%
\pgfpathlineto{\pgfqpoint{4.513580in}{1.974467in}}%
\pgfpathlineto{\pgfqpoint{4.521424in}{1.990339in}}%
\pgfpathlineto{\pgfqpoint{4.529263in}{2.006139in}}%
\pgfpathlineto{\pgfqpoint{4.515282in}{1.996593in}}%
\pgfpathlineto{\pgfqpoint{4.501317in}{1.987206in}}%
\pgfpathlineto{\pgfqpoint{4.487366in}{1.977978in}}%
\pgfpathlineto{\pgfqpoint{4.473431in}{1.968909in}}%
\pgfpathlineto{\pgfqpoint{4.465597in}{1.953474in}}%
\pgfpathlineto{\pgfqpoint{4.457758in}{1.937974in}}%
\pgfpathlineto{\pgfqpoint{4.449916in}{1.922412in}}%
\pgfpathlineto{\pgfqpoint{4.442069in}{1.906793in}}%
\pgfpathclose%
\pgfusepath{fill}%
\end{pgfscope}%
\begin{pgfscope}%
\pgfpathrectangle{\pgfqpoint{1.254980in}{0.150000in}}{\pgfqpoint{5.490039in}{5.490039in}}%
\pgfusepath{clip}%
\pgfsetbuttcap%
\pgfsetroundjoin%
\definecolor{currentfill}{rgb}{0.262138,0.242286,0.520837}%
\pgfsetfillcolor{currentfill}%
\pgfsetfillopacity{0.700000}%
\pgfsetlinewidth{0.000000pt}%
\definecolor{currentstroke}{rgb}{0.000000,0.000000,0.000000}%
\pgfsetstrokecolor{currentstroke}%
\pgfsetdash{}{0pt}%
\pgfpathmoveto{\pgfqpoint{4.205177in}{1.602428in}}%
\pgfpathlineto{\pgfqpoint{4.218986in}{1.608085in}}%
\pgfpathlineto{\pgfqpoint{4.232807in}{1.613898in}}%
\pgfpathlineto{\pgfqpoint{4.246640in}{1.619867in}}%
\pgfpathlineto{\pgfqpoint{4.260487in}{1.625993in}}%
\pgfpathlineto{\pgfqpoint{4.268390in}{1.641519in}}%
\pgfpathlineto{\pgfqpoint{4.276290in}{1.657059in}}%
\pgfpathlineto{\pgfqpoint{4.284187in}{1.672608in}}%
\pgfpathlineto{\pgfqpoint{4.292079in}{1.688160in}}%
\pgfpathlineto{\pgfqpoint{4.278230in}{1.681549in}}%
\pgfpathlineto{\pgfqpoint{4.264395in}{1.675094in}}%
\pgfpathlineto{\pgfqpoint{4.250572in}{1.668795in}}%
\pgfpathlineto{\pgfqpoint{4.236762in}{1.662654in}}%
\pgfpathlineto{\pgfqpoint{4.228871in}{1.647576in}}%
\pgfpathlineto{\pgfqpoint{4.220977in}{1.632509in}}%
\pgfpathlineto{\pgfqpoint{4.213079in}{1.617458in}}%
\pgfpathlineto{\pgfqpoint{4.205177in}{1.602428in}}%
\pgfpathclose%
\pgfusepath{fill}%
\end{pgfscope}%
\begin{pgfscope}%
\pgfpathrectangle{\pgfqpoint{1.254980in}{0.150000in}}{\pgfqpoint{5.490039in}{5.490039in}}%
\pgfusepath{clip}%
\pgfsetbuttcap%
\pgfsetroundjoin%
\definecolor{currentfill}{rgb}{0.271305,0.019942,0.347269}%
\pgfsetfillcolor{currentfill}%
\pgfsetfillopacity{0.700000}%
\pgfsetlinewidth{0.000000pt}%
\definecolor{currentstroke}{rgb}{0.000000,0.000000,0.000000}%
\pgfsetstrokecolor{currentstroke}%
\pgfsetdash{}{0pt}%
\pgfpathmoveto{\pgfqpoint{3.565937in}{1.203644in}}%
\pgfpathlineto{\pgfqpoint{3.579570in}{1.200095in}}%
\pgfpathlineto{\pgfqpoint{3.593209in}{1.196704in}}%
\pgfpathlineto{\pgfqpoint{3.606853in}{1.193471in}}%
\pgfpathlineto{\pgfqpoint{3.620503in}{1.190396in}}%
\pgfpathlineto{\pgfqpoint{3.628635in}{1.197672in}}%
\pgfpathlineto{\pgfqpoint{3.636759in}{1.205195in}}%
\pgfpathlineto{\pgfqpoint{3.644874in}{1.212958in}}%
\pgfpathlineto{\pgfqpoint{3.652980in}{1.220953in}}%
\pgfpathlineto{\pgfqpoint{3.639350in}{1.223330in}}%
\pgfpathlineto{\pgfqpoint{3.625725in}{1.225864in}}%
\pgfpathlineto{\pgfqpoint{3.612107in}{1.228557in}}%
\pgfpathlineto{\pgfqpoint{3.598494in}{1.231408in}}%
\pgfpathlineto{\pgfqpoint{3.590369in}{1.224100in}}%
\pgfpathlineto{\pgfqpoint{3.582234in}{1.217032in}}%
\pgfpathlineto{\pgfqpoint{3.574090in}{1.210211in}}%
\pgfpathlineto{\pgfqpoint{3.565937in}{1.203644in}}%
\pgfpathclose%
\pgfusepath{fill}%
\end{pgfscope}%
\begin{pgfscope}%
\pgfpathrectangle{\pgfqpoint{1.254980in}{0.150000in}}{\pgfqpoint{5.490039in}{5.490039in}}%
\pgfusepath{clip}%
\pgfsetbuttcap%
\pgfsetroundjoin%
\definecolor{currentfill}{rgb}{0.179019,0.433756,0.557430}%
\pgfsetfillcolor{currentfill}%
\pgfsetfillopacity{0.700000}%
\pgfsetlinewidth{0.000000pt}%
\definecolor{currentstroke}{rgb}{0.000000,0.000000,0.000000}%
\pgfsetstrokecolor{currentstroke}%
\pgfsetdash{}{0pt}%
\pgfpathmoveto{\pgfqpoint{4.560578in}{2.068556in}}%
\pgfpathlineto{\pgfqpoint{4.574580in}{2.078610in}}%
\pgfpathlineto{\pgfqpoint{4.588599in}{2.088824in}}%
\pgfpathlineto{\pgfqpoint{4.602633in}{2.099198in}}%
\pgfpathlineto{\pgfqpoint{4.616684in}{2.109732in}}%
\pgfpathlineto{\pgfqpoint{4.624507in}{2.125455in}}%
\pgfpathlineto{\pgfqpoint{4.632326in}{2.141080in}}%
\pgfpathlineto{\pgfqpoint{4.640141in}{2.156604in}}%
\pgfpathlineto{\pgfqpoint{4.647950in}{2.172024in}}%
\pgfpathlineto{\pgfqpoint{4.633893in}{2.161170in}}%
\pgfpathlineto{\pgfqpoint{4.619853in}{2.150477in}}%
\pgfpathlineto{\pgfqpoint{4.605829in}{2.139944in}}%
\pgfpathlineto{\pgfqpoint{4.591821in}{2.129572in}}%
\pgfpathlineto{\pgfqpoint{4.584017in}{2.114459in}}%
\pgfpathlineto{\pgfqpoint{4.576209in}{2.099251in}}%
\pgfpathlineto{\pgfqpoint{4.568396in}{2.083949in}}%
\pgfpathlineto{\pgfqpoint{4.560578in}{2.068556in}}%
\pgfpathclose%
\pgfusepath{fill}%
\end{pgfscope}%
\begin{pgfscope}%
\pgfpathrectangle{\pgfqpoint{1.254980in}{0.150000in}}{\pgfqpoint{5.490039in}{5.490039in}}%
\pgfusepath{clip}%
\pgfsetbuttcap%
\pgfsetroundjoin%
\definecolor{currentfill}{rgb}{0.153364,0.497000,0.557724}%
\pgfsetfillcolor{currentfill}%
\pgfsetfillopacity{0.700000}%
\pgfsetlinewidth{0.000000pt}%
\definecolor{currentstroke}{rgb}{0.000000,0.000000,0.000000}%
\pgfsetstrokecolor{currentstroke}%
\pgfsetdash{}{0pt}%
\pgfpathmoveto{\pgfqpoint{4.679139in}{2.232618in}}%
\pgfpathlineto{\pgfqpoint{4.693219in}{2.243924in}}%
\pgfpathlineto{\pgfqpoint{4.707315in}{2.255391in}}%
\pgfpathlineto{\pgfqpoint{4.721429in}{2.267019in}}%
\pgfpathlineto{\pgfqpoint{4.735560in}{2.278809in}}%
\pgfpathlineto{\pgfqpoint{4.743350in}{2.293945in}}%
\pgfpathlineto{\pgfqpoint{4.751136in}{2.308955in}}%
\pgfpathlineto{\pgfqpoint{4.758915in}{2.323837in}}%
\pgfpathlineto{\pgfqpoint{4.766689in}{2.338588in}}%
\pgfpathlineto{\pgfqpoint{4.752552in}{2.326536in}}%
\pgfpathlineto{\pgfqpoint{4.738432in}{2.314646in}}%
\pgfpathlineto{\pgfqpoint{4.724330in}{2.302918in}}%
\pgfpathlineto{\pgfqpoint{4.710245in}{2.291352in}}%
\pgfpathlineto{\pgfqpoint{4.702476in}{2.276851in}}%
\pgfpathlineto{\pgfqpoint{4.694703in}{2.262227in}}%
\pgfpathlineto{\pgfqpoint{4.686923in}{2.247482in}}%
\pgfpathlineto{\pgfqpoint{4.679139in}{2.232618in}}%
\pgfpathclose%
\pgfusepath{fill}%
\end{pgfscope}%
\begin{pgfscope}%
\pgfpathrectangle{\pgfqpoint{1.254980in}{0.150000in}}{\pgfqpoint{5.490039in}{5.490039in}}%
\pgfusepath{clip}%
\pgfsetbuttcap%
\pgfsetroundjoin%
\definecolor{currentfill}{rgb}{0.214000,0.722114,0.469588}%
\pgfsetfillcolor{currentfill}%
\pgfsetfillopacity{0.700000}%
\pgfsetlinewidth{0.000000pt}%
\definecolor{currentstroke}{rgb}{0.000000,0.000000,0.000000}%
\pgfsetstrokecolor{currentstroke}%
\pgfsetdash{}{0pt}%
\pgfpathmoveto{\pgfqpoint{5.153148in}{2.861261in}}%
\pgfpathlineto{\pgfqpoint{5.167558in}{2.876428in}}%
\pgfpathlineto{\pgfqpoint{5.181988in}{2.891762in}}%
\pgfpathlineto{\pgfqpoint{5.196440in}{2.907262in}}%
\pgfpathlineto{\pgfqpoint{5.210912in}{2.922929in}}%
\pgfpathlineto{\pgfqpoint{5.218488in}{2.933396in}}%
\pgfpathlineto{\pgfqpoint{5.226055in}{2.943672in}}%
\pgfpathlineto{\pgfqpoint{5.233613in}{2.953756in}}%
\pgfpathlineto{\pgfqpoint{5.241161in}{2.963649in}}%
\pgfpathlineto{\pgfqpoint{5.226690in}{2.947967in}}%
\pgfpathlineto{\pgfqpoint{5.212239in}{2.932453in}}%
\pgfpathlineto{\pgfqpoint{5.197809in}{2.917104in}}%
\pgfpathlineto{\pgfqpoint{5.183400in}{2.901922in}}%
\pgfpathlineto{\pgfqpoint{5.175851in}{2.892031in}}%
\pgfpathlineto{\pgfqpoint{5.168292in}{2.881958in}}%
\pgfpathlineto{\pgfqpoint{5.160724in}{2.871701in}}%
\pgfpathlineto{\pgfqpoint{5.153148in}{2.861261in}}%
\pgfpathclose%
\pgfusepath{fill}%
\end{pgfscope}%
\begin{pgfscope}%
\pgfpathrectangle{\pgfqpoint{1.254980in}{0.150000in}}{\pgfqpoint{5.490039in}{5.490039in}}%
\pgfusepath{clip}%
\pgfsetbuttcap%
\pgfsetroundjoin%
\definecolor{currentfill}{rgb}{0.496615,0.826376,0.306377}%
\pgfsetfillcolor{currentfill}%
\pgfsetfillopacity{0.700000}%
\pgfsetlinewidth{0.000000pt}%
\definecolor{currentstroke}{rgb}{0.000000,0.000000,0.000000}%
\pgfsetstrokecolor{currentstroke}%
\pgfsetdash{}{0pt}%
\pgfpathmoveto{\pgfqpoint{5.477021in}{3.225887in}}%
\pgfpathlineto{\pgfqpoint{5.491671in}{3.242813in}}%
\pgfpathlineto{\pgfqpoint{5.506343in}{3.259906in}}%
\pgfpathlineto{\pgfqpoint{5.521038in}{3.277169in}}%
\pgfpathlineto{\pgfqpoint{5.535756in}{3.294600in}}%
\pgfpathlineto{\pgfqpoint{5.543113in}{3.300966in}}%
\pgfpathlineto{\pgfqpoint{5.550458in}{3.307137in}}%
\pgfpathlineto{\pgfqpoint{5.557791in}{3.313114in}}%
\pgfpathlineto{\pgfqpoint{5.565114in}{3.318899in}}%
\pgfpathlineto{\pgfqpoint{5.550405in}{3.301616in}}%
\pgfpathlineto{\pgfqpoint{5.535719in}{3.284502in}}%
\pgfpathlineto{\pgfqpoint{5.521056in}{3.267556in}}%
\pgfpathlineto{\pgfqpoint{5.506416in}{3.250778in}}%
\pgfpathlineto{\pgfqpoint{5.499084in}{3.244833in}}%
\pgfpathlineto{\pgfqpoint{5.491741in}{3.238704in}}%
\pgfpathlineto{\pgfqpoint{5.484386in}{3.232390in}}%
\pgfpathlineto{\pgfqpoint{5.477021in}{3.225887in}}%
\pgfpathclose%
\pgfusepath{fill}%
\end{pgfscope}%
\begin{pgfscope}%
\pgfpathrectangle{\pgfqpoint{1.254980in}{0.150000in}}{\pgfqpoint{5.490039in}{5.490039in}}%
\pgfusepath{clip}%
\pgfsetbuttcap%
\pgfsetroundjoin%
\definecolor{currentfill}{rgb}{0.278826,0.175490,0.483397}%
\pgfsetfillcolor{currentfill}%
\pgfsetfillopacity{0.700000}%
\pgfsetlinewidth{0.000000pt}%
\definecolor{currentstroke}{rgb}{0.000000,0.000000,0.000000}%
\pgfsetstrokecolor{currentstroke}%
\pgfsetdash{}{0pt}%
\pgfpathmoveto{\pgfqpoint{4.086705in}{1.466624in}}%
\pgfpathlineto{\pgfqpoint{4.100467in}{1.470607in}}%
\pgfpathlineto{\pgfqpoint{4.114241in}{1.474745in}}%
\pgfpathlineto{\pgfqpoint{4.128026in}{1.479039in}}%
\pgfpathlineto{\pgfqpoint{4.141822in}{1.483487in}}%
\pgfpathlineto{\pgfqpoint{4.149755in}{1.498182in}}%
\pgfpathlineto{\pgfqpoint{4.157684in}{1.512937in}}%
\pgfpathlineto{\pgfqpoint{4.165610in}{1.527746in}}%
\pgfpathlineto{\pgfqpoint{4.173531in}{1.542604in}}%
\pgfpathlineto{\pgfqpoint{4.159735in}{1.537615in}}%
\pgfpathlineto{\pgfqpoint{4.145951in}{1.532783in}}%
\pgfpathlineto{\pgfqpoint{4.132178in}{1.528106in}}%
\pgfpathlineto{\pgfqpoint{4.118417in}{1.523585in}}%
\pgfpathlineto{\pgfqpoint{4.110495in}{1.509255in}}%
\pgfpathlineto{\pgfqpoint{4.102569in}{1.494982in}}%
\pgfpathlineto{\pgfqpoint{4.094639in}{1.480770in}}%
\pgfpathlineto{\pgfqpoint{4.086705in}{1.466624in}}%
\pgfpathclose%
\pgfusepath{fill}%
\end{pgfscope}%
\begin{pgfscope}%
\pgfpathrectangle{\pgfqpoint{1.254980in}{0.150000in}}{\pgfqpoint{5.490039in}{5.490039in}}%
\pgfusepath{clip}%
\pgfsetbuttcap%
\pgfsetroundjoin%
\definecolor{currentfill}{rgb}{0.131172,0.555899,0.552459}%
\pgfsetfillcolor{currentfill}%
\pgfsetfillopacity{0.700000}%
\pgfsetlinewidth{0.000000pt}%
\definecolor{currentstroke}{rgb}{0.000000,0.000000,0.000000}%
\pgfsetstrokecolor{currentstroke}%
\pgfsetdash{}{0pt}%
\pgfpathmoveto{\pgfqpoint{4.797729in}{2.396244in}}%
\pgfpathlineto{\pgfqpoint{4.811889in}{2.408690in}}%
\pgfpathlineto{\pgfqpoint{4.826067in}{2.421299in}}%
\pgfpathlineto{\pgfqpoint{4.840264in}{2.434070in}}%
\pgfpathlineto{\pgfqpoint{4.854479in}{2.447005in}}%
\pgfpathlineto{\pgfqpoint{4.862230in}{2.461286in}}%
\pgfpathlineto{\pgfqpoint{4.869974in}{2.475417in}}%
\pgfpathlineto{\pgfqpoint{4.877712in}{2.489397in}}%
\pgfpathlineto{\pgfqpoint{4.885443in}{2.503223in}}%
\pgfpathlineto{\pgfqpoint{4.871223in}{2.490087in}}%
\pgfpathlineto{\pgfqpoint{4.857021in}{2.477113in}}%
\pgfpathlineto{\pgfqpoint{4.842838in}{2.464303in}}%
\pgfpathlineto{\pgfqpoint{4.828672in}{2.451656in}}%
\pgfpathlineto{\pgfqpoint{4.820946in}{2.438020in}}%
\pgfpathlineto{\pgfqpoint{4.813213in}{2.424238in}}%
\pgfpathlineto{\pgfqpoint{4.805474in}{2.410312in}}%
\pgfpathlineto{\pgfqpoint{4.797729in}{2.396244in}}%
\pgfpathclose%
\pgfusepath{fill}%
\end{pgfscope}%
\begin{pgfscope}%
\pgfpathrectangle{\pgfqpoint{1.254980in}{0.150000in}}{\pgfqpoint{5.490039in}{5.490039in}}%
\pgfusepath{clip}%
\pgfsetbuttcap%
\pgfsetroundjoin%
\definecolor{currentfill}{rgb}{0.143303,0.669459,0.511215}%
\pgfsetfillcolor{currentfill}%
\pgfsetfillopacity{0.700000}%
\pgfsetlinewidth{0.000000pt}%
\definecolor{currentstroke}{rgb}{0.000000,0.000000,0.000000}%
\pgfsetstrokecolor{currentstroke}%
\pgfsetdash{}{0pt}%
\pgfpathmoveto{\pgfqpoint{5.034801in}{2.712610in}}%
\pgfpathlineto{\pgfqpoint{5.049128in}{2.726990in}}%
\pgfpathlineto{\pgfqpoint{5.063475in}{2.741535in}}%
\pgfpathlineto{\pgfqpoint{5.077842in}{2.756245in}}%
\pgfpathlineto{\pgfqpoint{5.092229in}{2.771122in}}%
\pgfpathlineto{\pgfqpoint{5.099873in}{2.783031in}}%
\pgfpathlineto{\pgfqpoint{5.107509in}{2.794757in}}%
\pgfpathlineto{\pgfqpoint{5.115137in}{2.806300in}}%
\pgfpathlineto{\pgfqpoint{5.122756in}{2.817660in}}%
\pgfpathlineto{\pgfqpoint{5.108367in}{2.802705in}}%
\pgfpathlineto{\pgfqpoint{5.093998in}{2.787916in}}%
\pgfpathlineto{\pgfqpoint{5.079650in}{2.773293in}}%
\pgfpathlineto{\pgfqpoint{5.065321in}{2.758834in}}%
\pgfpathlineto{\pgfqpoint{5.057703in}{2.747541in}}%
\pgfpathlineto{\pgfqpoint{5.050077in}{2.736072in}}%
\pgfpathlineto{\pgfqpoint{5.042443in}{2.724429in}}%
\pgfpathlineto{\pgfqpoint{5.034801in}{2.712610in}}%
\pgfpathclose%
\pgfusepath{fill}%
\end{pgfscope}%
\begin{pgfscope}%
\pgfpathrectangle{\pgfqpoint{1.254980in}{0.150000in}}{\pgfqpoint{5.490039in}{5.490039in}}%
\pgfusepath{clip}%
\pgfsetbuttcap%
\pgfsetroundjoin%
\definecolor{currentfill}{rgb}{0.119483,0.614817,0.537692}%
\pgfsetfillcolor{currentfill}%
\pgfsetfillopacity{0.700000}%
\pgfsetlinewidth{0.000000pt}%
\definecolor{currentstroke}{rgb}{0.000000,0.000000,0.000000}%
\pgfsetstrokecolor{currentstroke}%
\pgfsetdash{}{0pt}%
\pgfpathmoveto{\pgfqpoint{4.916303in}{2.556971in}}%
\pgfpathlineto{\pgfqpoint{4.930546in}{2.570443in}}%
\pgfpathlineto{\pgfqpoint{4.944809in}{2.584078in}}%
\pgfpathlineto{\pgfqpoint{4.959090in}{2.597878in}}%
\pgfpathlineto{\pgfqpoint{4.973391in}{2.611843in}}%
\pgfpathlineto{\pgfqpoint{4.981093in}{2.625036in}}%
\pgfpathlineto{\pgfqpoint{4.988788in}{2.638060in}}%
\pgfpathlineto{\pgfqpoint{4.996476in}{2.650914in}}%
\pgfpathlineto{\pgfqpoint{5.004156in}{2.663598in}}%
\pgfpathlineto{\pgfqpoint{4.989851in}{2.649493in}}%
\pgfpathlineto{\pgfqpoint{4.975566in}{2.635552in}}%
\pgfpathlineto{\pgfqpoint{4.961300in}{2.621775in}}%
\pgfpathlineto{\pgfqpoint{4.947053in}{2.608163in}}%
\pgfpathlineto{\pgfqpoint{4.939376in}{2.595608in}}%
\pgfpathlineto{\pgfqpoint{4.931692in}{2.582891in}}%
\pgfpathlineto{\pgfqpoint{4.924001in}{2.570011in}}%
\pgfpathlineto{\pgfqpoint{4.916303in}{2.556971in}}%
\pgfpathclose%
\pgfusepath{fill}%
\end{pgfscope}%
\begin{pgfscope}%
\pgfpathrectangle{\pgfqpoint{1.254980in}{0.150000in}}{\pgfqpoint{5.490039in}{5.490039in}}%
\pgfusepath{clip}%
\pgfsetbuttcap%
\pgfsetroundjoin%
\definecolor{currentfill}{rgb}{0.278791,0.062145,0.386592}%
\pgfsetfillcolor{currentfill}%
\pgfsetfillopacity{0.700000}%
\pgfsetlinewidth{0.000000pt}%
\definecolor{currentstroke}{rgb}{0.000000,0.000000,0.000000}%
\pgfsetstrokecolor{currentstroke}%
\pgfsetdash{}{0pt}%
\pgfpathmoveto{\pgfqpoint{3.794475in}{1.247116in}}%
\pgfpathlineto{\pgfqpoint{3.808152in}{1.246818in}}%
\pgfpathlineto{\pgfqpoint{3.821836in}{1.246675in}}%
\pgfpathlineto{\pgfqpoint{3.835529in}{1.246687in}}%
\pgfpathlineto{\pgfqpoint{3.849230in}{1.246854in}}%
\pgfpathlineto{\pgfqpoint{3.857257in}{1.257872in}}%
\pgfpathlineto{\pgfqpoint{3.865277in}{1.269062in}}%
\pgfpathlineto{\pgfqpoint{3.873292in}{1.280417in}}%
\pgfpathlineto{\pgfqpoint{3.881301in}{1.291931in}}%
\pgfpathlineto{\pgfqpoint{3.867611in}{1.291119in}}%
\pgfpathlineto{\pgfqpoint{3.853929in}{1.290463in}}%
\pgfpathlineto{\pgfqpoint{3.840255in}{1.289962in}}%
\pgfpathlineto{\pgfqpoint{3.826590in}{1.289616in}}%
\pgfpathlineto{\pgfqpoint{3.818570in}{1.278735in}}%
\pgfpathlineto{\pgfqpoint{3.810545in}{1.268021in}}%
\pgfpathlineto{\pgfqpoint{3.802513in}{1.257479in}}%
\pgfpathlineto{\pgfqpoint{3.794475in}{1.247116in}}%
\pgfpathclose%
\pgfusepath{fill}%
\end{pgfscope}%
\begin{pgfscope}%
\pgfpathrectangle{\pgfqpoint{1.254980in}{0.150000in}}{\pgfqpoint{5.490039in}{5.490039in}}%
\pgfusepath{clip}%
\pgfsetbuttcap%
\pgfsetroundjoin%
\definecolor{currentfill}{rgb}{0.281924,0.089666,0.412415}%
\pgfsetfillcolor{currentfill}%
\pgfsetfillopacity{0.700000}%
\pgfsetlinewidth{0.000000pt}%
\definecolor{currentstroke}{rgb}{0.000000,0.000000,0.000000}%
\pgfsetstrokecolor{currentstroke}%
\pgfsetdash{}{0pt}%
\pgfpathmoveto{\pgfqpoint{3.881301in}{1.291931in}}%
\pgfpathlineto{\pgfqpoint{3.895000in}{1.292898in}}%
\pgfpathlineto{\pgfqpoint{3.908708in}{1.294020in}}%
\pgfpathlineto{\pgfqpoint{3.922425in}{1.295297in}}%
\pgfpathlineto{\pgfqpoint{3.936151in}{1.296728in}}%
\pgfpathlineto{\pgfqpoint{3.944146in}{1.309025in}}%
\pgfpathlineto{\pgfqpoint{3.952136in}{1.321461in}}%
\pgfpathlineto{\pgfqpoint{3.960121in}{1.334031in}}%
\pgfpathlineto{\pgfqpoint{3.968100in}{1.346730in}}%
\pgfpathlineto{\pgfqpoint{3.954381in}{1.344680in}}%
\pgfpathlineto{\pgfqpoint{3.940672in}{1.342785in}}%
\pgfpathlineto{\pgfqpoint{3.926972in}{1.341045in}}%
\pgfpathlineto{\pgfqpoint{3.913281in}{1.339460in}}%
\pgfpathlineto{\pgfqpoint{3.905294in}{1.327369in}}%
\pgfpathlineto{\pgfqpoint{3.897302in}{1.315414in}}%
\pgfpathlineto{\pgfqpoint{3.889304in}{1.303599in}}%
\pgfpathlineto{\pgfqpoint{3.881301in}{1.291931in}}%
\pgfpathclose%
\pgfusepath{fill}%
\end{pgfscope}%
\begin{pgfscope}%
\pgfpathrectangle{\pgfqpoint{1.254980in}{0.150000in}}{\pgfqpoint{5.490039in}{5.490039in}}%
\pgfusepath{clip}%
\pgfsetbuttcap%
\pgfsetroundjoin%
\definecolor{currentfill}{rgb}{0.274952,0.037752,0.364543}%
\pgfsetfillcolor{currentfill}%
\pgfsetfillopacity{0.700000}%
\pgfsetlinewidth{0.000000pt}%
\definecolor{currentstroke}{rgb}{0.000000,0.000000,0.000000}%
\pgfsetstrokecolor{currentstroke}%
\pgfsetdash{}{0pt}%
\pgfpathmoveto{\pgfqpoint{3.707563in}{1.213016in}}%
\pgfpathlineto{\pgfqpoint{3.721225in}{1.211424in}}%
\pgfpathlineto{\pgfqpoint{3.734894in}{1.209987in}}%
\pgfpathlineto{\pgfqpoint{3.748570in}{1.208706in}}%
\pgfpathlineto{\pgfqpoint{3.762253in}{1.207581in}}%
\pgfpathlineto{\pgfqpoint{3.770319in}{1.217164in}}%
\pgfpathlineto{\pgfqpoint{3.778378in}{1.226952in}}%
\pgfpathlineto{\pgfqpoint{3.786430in}{1.236938in}}%
\pgfpathlineto{\pgfqpoint{3.794475in}{1.247116in}}%
\pgfpathlineto{\pgfqpoint{3.780806in}{1.247570in}}%
\pgfpathlineto{\pgfqpoint{3.767144in}{1.248180in}}%
\pgfpathlineto{\pgfqpoint{3.753490in}{1.248945in}}%
\pgfpathlineto{\pgfqpoint{3.739842in}{1.249867in}}%
\pgfpathlineto{\pgfqpoint{3.731784in}{1.240350in}}%
\pgfpathlineto{\pgfqpoint{3.723718in}{1.231031in}}%
\pgfpathlineto{\pgfqpoint{3.715644in}{1.221918in}}%
\pgfpathlineto{\pgfqpoint{3.707563in}{1.213016in}}%
\pgfpathclose%
\pgfusepath{fill}%
\end{pgfscope}%
\begin{pgfscope}%
\pgfpathrectangle{\pgfqpoint{1.254980in}{0.150000in}}{\pgfqpoint{5.490039in}{5.490039in}}%
\pgfusepath{clip}%
\pgfsetbuttcap%
\pgfsetroundjoin%
\definecolor{currentfill}{rgb}{0.244972,0.287675,0.537260}%
\pgfsetfillcolor{currentfill}%
\pgfsetfillopacity{0.700000}%
\pgfsetlinewidth{0.000000pt}%
\definecolor{currentstroke}{rgb}{0.000000,0.000000,0.000000}%
\pgfsetstrokecolor{currentstroke}%
\pgfsetdash{}{0pt}%
\pgfpathmoveto{\pgfqpoint{4.292079in}{1.688160in}}%
\pgfpathlineto{\pgfqpoint{4.305941in}{1.694929in}}%
\pgfpathlineto{\pgfqpoint{4.319816in}{1.701854in}}%
\pgfpathlineto{\pgfqpoint{4.333705in}{1.708937in}}%
\pgfpathlineto{\pgfqpoint{4.347607in}{1.716175in}}%
\pgfpathlineto{\pgfqpoint{4.355499in}{1.732199in}}%
\pgfpathlineto{\pgfqpoint{4.363388in}{1.748210in}}%
\pgfpathlineto{\pgfqpoint{4.371274in}{1.764206in}}%
\pgfpathlineto{\pgfqpoint{4.379155in}{1.780183in}}%
\pgfpathlineto{\pgfqpoint{4.365249in}{1.772483in}}%
\pgfpathlineto{\pgfqpoint{4.351356in}{1.764941in}}%
\pgfpathlineto{\pgfqpoint{4.337477in}{1.757556in}}%
\pgfpathlineto{\pgfqpoint{4.323612in}{1.750329in}}%
\pgfpathlineto{\pgfqpoint{4.315734in}{1.734802in}}%
\pgfpathlineto{\pgfqpoint{4.307853in}{1.719262in}}%
\pgfpathlineto{\pgfqpoint{4.299968in}{1.703713in}}%
\pgfpathlineto{\pgfqpoint{4.292079in}{1.688160in}}%
\pgfpathclose%
\pgfusepath{fill}%
\end{pgfscope}%
\begin{pgfscope}%
\pgfpathrectangle{\pgfqpoint{1.254980in}{0.150000in}}{\pgfqpoint{5.490039in}{5.490039in}}%
\pgfusepath{clip}%
\pgfsetbuttcap%
\pgfsetroundjoin%
\definecolor{currentfill}{rgb}{0.585678,0.846661,0.249897}%
\pgfsetfillcolor{currentfill}%
\pgfsetfillopacity{0.700000}%
\pgfsetlinewidth{0.000000pt}%
\definecolor{currentstroke}{rgb}{0.000000,0.000000,0.000000}%
\pgfsetstrokecolor{currentstroke}%
\pgfsetdash{}{0pt}%
\pgfpathmoveto{\pgfqpoint{5.565114in}{3.318899in}}%
\pgfpathlineto{\pgfqpoint{5.579846in}{3.336351in}}%
\pgfpathlineto{\pgfqpoint{5.594601in}{3.353971in}}%
\pgfpathlineto{\pgfqpoint{5.609379in}{3.371762in}}%
\pgfpathlineto{\pgfqpoint{5.616682in}{3.377229in}}%
\pgfpathlineto{\pgfqpoint{5.623973in}{3.382504in}}%
\pgfpathlineto{\pgfqpoint{5.631252in}{3.387587in}}%
\pgfpathlineto{\pgfqpoint{5.638519in}{3.392480in}}%
\pgfpathlineto{\pgfqpoint{5.623752in}{3.374872in}}%
\pgfpathlineto{\pgfqpoint{5.609008in}{3.357434in}}%
\pgfpathlineto{\pgfqpoint{5.594288in}{3.340163in}}%
\pgfpathlineto{\pgfqpoint{5.587011in}{3.335124in}}%
\pgfpathlineto{\pgfqpoint{5.579724in}{3.329902in}}%
\pgfpathlineto{\pgfqpoint{5.572424in}{3.324494in}}%
\pgfpathlineto{\pgfqpoint{5.565114in}{3.318899in}}%
\pgfpathclose%
\pgfusepath{fill}%
\end{pgfscope}%
\begin{pgfscope}%
\pgfpathrectangle{\pgfqpoint{1.254980in}{0.150000in}}{\pgfqpoint{5.490039in}{5.490039in}}%
\pgfusepath{clip}%
\pgfsetbuttcap%
\pgfsetroundjoin%
\definecolor{currentfill}{rgb}{0.214298,0.355619,0.551184}%
\pgfsetfillcolor{currentfill}%
\pgfsetfillopacity{0.700000}%
\pgfsetlinewidth{0.000000pt}%
\definecolor{currentstroke}{rgb}{0.000000,0.000000,0.000000}%
\pgfsetstrokecolor{currentstroke}%
\pgfsetdash{}{0pt}%
\pgfpathmoveto{\pgfqpoint{4.410643in}{1.843806in}}%
\pgfpathlineto{\pgfqpoint{4.424569in}{1.852096in}}%
\pgfpathlineto{\pgfqpoint{4.438508in}{1.860543in}}%
\pgfpathlineto{\pgfqpoint{4.452463in}{1.869149in}}%
\pgfpathlineto{\pgfqpoint{4.466432in}{1.877913in}}%
\pgfpathlineto{\pgfqpoint{4.474300in}{1.894144in}}%
\pgfpathlineto{\pgfqpoint{4.482164in}{1.910325in}}%
\pgfpathlineto{\pgfqpoint{4.490024in}{1.926451in}}%
\pgfpathlineto{\pgfqpoint{4.497880in}{1.942519in}}%
\pgfpathlineto{\pgfqpoint{4.483905in}{1.933350in}}%
\pgfpathlineto{\pgfqpoint{4.469945in}{1.924339in}}%
\pgfpathlineto{\pgfqpoint{4.456000in}{1.915487in}}%
\pgfpathlineto{\pgfqpoint{4.442069in}{1.906793in}}%
\pgfpathlineto{\pgfqpoint{4.434219in}{1.891119in}}%
\pgfpathlineto{\pgfqpoint{4.426364in}{1.875394in}}%
\pgfpathlineto{\pgfqpoint{4.418506in}{1.859622in}}%
\pgfpathlineto{\pgfqpoint{4.410643in}{1.843806in}}%
\pgfpathclose%
\pgfusepath{fill}%
\end{pgfscope}%
\begin{pgfscope}%
\pgfpathrectangle{\pgfqpoint{1.254980in}{0.150000in}}{\pgfqpoint{5.490039in}{5.490039in}}%
\pgfusepath{clip}%
\pgfsetbuttcap%
\pgfsetroundjoin%
\definecolor{currentfill}{rgb}{0.283229,0.120777,0.440584}%
\pgfsetfillcolor{currentfill}%
\pgfsetfillopacity{0.700000}%
\pgfsetlinewidth{0.000000pt}%
\definecolor{currentstroke}{rgb}{0.000000,0.000000,0.000000}%
\pgfsetstrokecolor{currentstroke}%
\pgfsetdash{}{0pt}%
\pgfpathmoveto{\pgfqpoint{3.968100in}{1.346730in}}%
\pgfpathlineto{\pgfqpoint{3.981829in}{1.348935in}}%
\pgfpathlineto{\pgfqpoint{3.995567in}{1.351294in}}%
\pgfpathlineto{\pgfqpoint{4.009316in}{1.353809in}}%
\pgfpathlineto{\pgfqpoint{4.023074in}{1.356478in}}%
\pgfpathlineto{\pgfqpoint{4.031044in}{1.369901in}}%
\pgfpathlineto{\pgfqpoint{4.039008in}{1.383435in}}%
\pgfpathlineto{\pgfqpoint{4.046969in}{1.397072in}}%
\pgfpathlineto{\pgfqpoint{4.054925in}{1.410808in}}%
\pgfpathlineto{\pgfqpoint{4.041170in}{1.407546in}}%
\pgfpathlineto{\pgfqpoint{4.027427in}{1.404439in}}%
\pgfpathlineto{\pgfqpoint{4.013693in}{1.401487in}}%
\pgfpathlineto{\pgfqpoint{3.999969in}{1.398690in}}%
\pgfpathlineto{\pgfqpoint{3.992009in}{1.385537in}}%
\pgfpathlineto{\pgfqpoint{3.984045in}{1.372488in}}%
\pgfpathlineto{\pgfqpoint{3.976075in}{1.359551in}}%
\pgfpathlineto{\pgfqpoint{3.968100in}{1.346730in}}%
\pgfpathclose%
\pgfusepath{fill}%
\end{pgfscope}%
\begin{pgfscope}%
\pgfpathrectangle{\pgfqpoint{1.254980in}{0.150000in}}{\pgfqpoint{5.490039in}{5.490039in}}%
\pgfusepath{clip}%
\pgfsetbuttcap%
\pgfsetroundjoin%
\definecolor{currentfill}{rgb}{0.395174,0.797475,0.367757}%
\pgfsetfillcolor{currentfill}%
\pgfsetfillopacity{0.700000}%
\pgfsetlinewidth{0.000000pt}%
\definecolor{currentstroke}{rgb}{0.000000,0.000000,0.000000}%
\pgfsetstrokecolor{currentstroke}%
\pgfsetdash{}{0pt}%
\pgfpathmoveto{\pgfqpoint{5.359281in}{3.100107in}}%
\pgfpathlineto{\pgfqpoint{5.373854in}{3.116558in}}%
\pgfpathlineto{\pgfqpoint{5.388449in}{3.133176in}}%
\pgfpathlineto{\pgfqpoint{5.403067in}{3.149963in}}%
\pgfpathlineto{\pgfqpoint{5.417707in}{3.166919in}}%
\pgfpathlineto{\pgfqpoint{5.425159in}{3.174979in}}%
\pgfpathlineto{\pgfqpoint{5.432600in}{3.182840in}}%
\pgfpathlineto{\pgfqpoint{5.440031in}{3.190502in}}%
\pgfpathlineto{\pgfqpoint{5.447451in}{3.197968in}}%
\pgfpathlineto{\pgfqpoint{5.432816in}{3.181095in}}%
\pgfpathlineto{\pgfqpoint{5.418204in}{3.164391in}}%
\pgfpathlineto{\pgfqpoint{5.403615in}{3.147855in}}%
\pgfpathlineto{\pgfqpoint{5.389047in}{3.131486in}}%
\pgfpathlineto{\pgfqpoint{5.381621in}{3.123926in}}%
\pgfpathlineto{\pgfqpoint{5.374184in}{3.116177in}}%
\pgfpathlineto{\pgfqpoint{5.366738in}{3.108238in}}%
\pgfpathlineto{\pgfqpoint{5.359281in}{3.100107in}}%
\pgfpathclose%
\pgfusepath{fill}%
\end{pgfscope}%
\begin{pgfscope}%
\pgfpathrectangle{\pgfqpoint{1.254980in}{0.150000in}}{\pgfqpoint{5.490039in}{5.490039in}}%
\pgfusepath{clip}%
\pgfsetbuttcap%
\pgfsetroundjoin%
\definecolor{currentfill}{rgb}{0.269308,0.218818,0.509577}%
\pgfsetfillcolor{currentfill}%
\pgfsetfillopacity{0.700000}%
\pgfsetlinewidth{0.000000pt}%
\definecolor{currentstroke}{rgb}{0.000000,0.000000,0.000000}%
\pgfsetstrokecolor{currentstroke}%
\pgfsetdash{}{0pt}%
\pgfpathmoveto{\pgfqpoint{4.173531in}{1.542604in}}%
\pgfpathlineto{\pgfqpoint{4.187339in}{1.547748in}}%
\pgfpathlineto{\pgfqpoint{4.201158in}{1.553048in}}%
\pgfpathlineto{\pgfqpoint{4.214990in}{1.558503in}}%
\pgfpathlineto{\pgfqpoint{4.228835in}{1.564114in}}%
\pgfpathlineto{\pgfqpoint{4.236753in}{1.579540in}}%
\pgfpathlineto{\pgfqpoint{4.244668in}{1.594998in}}%
\pgfpathlineto{\pgfqpoint{4.252579in}{1.610484in}}%
\pgfpathlineto{\pgfqpoint{4.260487in}{1.625993in}}%
\pgfpathlineto{\pgfqpoint{4.246640in}{1.619867in}}%
\pgfpathlineto{\pgfqpoint{4.232807in}{1.613898in}}%
\pgfpathlineto{\pgfqpoint{4.218986in}{1.608085in}}%
\pgfpathlineto{\pgfqpoint{4.205177in}{1.602428in}}%
\pgfpathlineto{\pgfqpoint{4.197271in}{1.587423in}}%
\pgfpathlineto{\pgfqpoint{4.189362in}{1.572447in}}%
\pgfpathlineto{\pgfqpoint{4.181448in}{1.557506in}}%
\pgfpathlineto{\pgfqpoint{4.173531in}{1.542604in}}%
\pgfpathclose%
\pgfusepath{fill}%
\end{pgfscope}%
\begin{pgfscope}%
\pgfpathrectangle{\pgfqpoint{1.254980in}{0.150000in}}{\pgfqpoint{5.490039in}{5.490039in}}%
\pgfusepath{clip}%
\pgfsetbuttcap%
\pgfsetroundjoin%
\definecolor{currentfill}{rgb}{0.185556,0.418570,0.556753}%
\pgfsetfillcolor{currentfill}%
\pgfsetfillopacity{0.700000}%
\pgfsetlinewidth{0.000000pt}%
\definecolor{currentstroke}{rgb}{0.000000,0.000000,0.000000}%
\pgfsetstrokecolor{currentstroke}%
\pgfsetdash{}{0pt}%
\pgfpathmoveto{\pgfqpoint{4.529263in}{2.006139in}}%
\pgfpathlineto{\pgfqpoint{4.543260in}{2.015845in}}%
\pgfpathlineto{\pgfqpoint{4.557272in}{2.025710in}}%
\pgfpathlineto{\pgfqpoint{4.571300in}{2.035735in}}%
\pgfpathlineto{\pgfqpoint{4.585343in}{2.045919in}}%
\pgfpathlineto{\pgfqpoint{4.593185in}{2.062004in}}%
\pgfpathlineto{\pgfqpoint{4.601022in}{2.078003in}}%
\pgfpathlineto{\pgfqpoint{4.608855in}{2.093914in}}%
\pgfpathlineto{\pgfqpoint{4.616684in}{2.109732in}}%
\pgfpathlineto{\pgfqpoint{4.602633in}{2.099198in}}%
\pgfpathlineto{\pgfqpoint{4.588599in}{2.088824in}}%
\pgfpathlineto{\pgfqpoint{4.574580in}{2.078610in}}%
\pgfpathlineto{\pgfqpoint{4.560578in}{2.068556in}}%
\pgfpathlineto{\pgfqpoint{4.552756in}{2.053076in}}%
\pgfpathlineto{\pgfqpoint{4.544929in}{2.037511in}}%
\pgfpathlineto{\pgfqpoint{4.537098in}{2.021864in}}%
\pgfpathlineto{\pgfqpoint{4.529263in}{2.006139in}}%
\pgfpathclose%
\pgfusepath{fill}%
\end{pgfscope}%
\begin{pgfscope}%
\pgfpathrectangle{\pgfqpoint{1.254980in}{0.150000in}}{\pgfqpoint{5.490039in}{5.490039in}}%
\pgfusepath{clip}%
\pgfsetbuttcap%
\pgfsetroundjoin%
\definecolor{currentfill}{rgb}{0.272594,0.025563,0.353093}%
\pgfsetfillcolor{currentfill}%
\pgfsetfillopacity{0.700000}%
\pgfsetlinewidth{0.000000pt}%
\definecolor{currentstroke}{rgb}{0.000000,0.000000,0.000000}%
\pgfsetstrokecolor{currentstroke}%
\pgfsetdash{}{0pt}%
\pgfpathmoveto{\pgfqpoint{3.620503in}{1.190396in}}%
\pgfpathlineto{\pgfqpoint{3.634158in}{1.187478in}}%
\pgfpathlineto{\pgfqpoint{3.647819in}{1.184717in}}%
\pgfpathlineto{\pgfqpoint{3.661487in}{1.182113in}}%
\pgfpathlineto{\pgfqpoint{3.675160in}{1.179665in}}%
\pgfpathlineto{\pgfqpoint{3.683273in}{1.187651in}}%
\pgfpathlineto{\pgfqpoint{3.691378in}{1.195876in}}%
\pgfpathlineto{\pgfqpoint{3.699474in}{1.204334in}}%
\pgfpathlineto{\pgfqpoint{3.707563in}{1.213016in}}%
\pgfpathlineto{\pgfqpoint{3.693908in}{1.214765in}}%
\pgfpathlineto{\pgfqpoint{3.680259in}{1.216671in}}%
\pgfpathlineto{\pgfqpoint{3.666616in}{1.218733in}}%
\pgfpathlineto{\pgfqpoint{3.652980in}{1.220953in}}%
\pgfpathlineto{\pgfqpoint{3.644874in}{1.212958in}}%
\pgfpathlineto{\pgfqpoint{3.636759in}{1.205195in}}%
\pgfpathlineto{\pgfqpoint{3.628635in}{1.197672in}}%
\pgfpathlineto{\pgfqpoint{3.620503in}{1.190396in}}%
\pgfpathclose%
\pgfusepath{fill}%
\end{pgfscope}%
\begin{pgfscope}%
\pgfpathrectangle{\pgfqpoint{1.254980in}{0.150000in}}{\pgfqpoint{5.490039in}{5.490039in}}%
\pgfusepath{clip}%
\pgfsetbuttcap%
\pgfsetroundjoin%
\definecolor{currentfill}{rgb}{0.159194,0.482237,0.558073}%
\pgfsetfillcolor{currentfill}%
\pgfsetfillopacity{0.700000}%
\pgfsetlinewidth{0.000000pt}%
\definecolor{currentstroke}{rgb}{0.000000,0.000000,0.000000}%
\pgfsetstrokecolor{currentstroke}%
\pgfsetdash{}{0pt}%
\pgfpathmoveto{\pgfqpoint{4.647950in}{2.172024in}}%
\pgfpathlineto{\pgfqpoint{4.662024in}{2.183039in}}%
\pgfpathlineto{\pgfqpoint{4.676114in}{2.194215in}}%
\pgfpathlineto{\pgfqpoint{4.690221in}{2.205552in}}%
\pgfpathlineto{\pgfqpoint{4.704345in}{2.217050in}}%
\pgfpathlineto{\pgfqpoint{4.712157in}{2.232667in}}%
\pgfpathlineto{\pgfqpoint{4.719963in}{2.248167in}}%
\pgfpathlineto{\pgfqpoint{4.727764in}{2.263549in}}%
\pgfpathlineto{\pgfqpoint{4.735560in}{2.278809in}}%
\pgfpathlineto{\pgfqpoint{4.721429in}{2.267019in}}%
\pgfpathlineto{\pgfqpoint{4.707315in}{2.255391in}}%
\pgfpathlineto{\pgfqpoint{4.693219in}{2.243924in}}%
\pgfpathlineto{\pgfqpoint{4.679139in}{2.232618in}}%
\pgfpathlineto{\pgfqpoint{4.671349in}{2.217638in}}%
\pgfpathlineto{\pgfqpoint{4.663555in}{2.202544in}}%
\pgfpathlineto{\pgfqpoint{4.655755in}{2.187339in}}%
\pgfpathlineto{\pgfqpoint{4.647950in}{2.172024in}}%
\pgfpathclose%
\pgfusepath{fill}%
\end{pgfscope}%
\begin{pgfscope}%
\pgfpathrectangle{\pgfqpoint{1.254980in}{0.150000in}}{\pgfqpoint{5.490039in}{5.490039in}}%
\pgfusepath{clip}%
\pgfsetbuttcap%
\pgfsetroundjoin%
\definecolor{currentfill}{rgb}{0.280868,0.160771,0.472899}%
\pgfsetfillcolor{currentfill}%
\pgfsetfillopacity{0.700000}%
\pgfsetlinewidth{0.000000pt}%
\definecolor{currentstroke}{rgb}{0.000000,0.000000,0.000000}%
\pgfsetstrokecolor{currentstroke}%
\pgfsetdash{}{0pt}%
\pgfpathmoveto{\pgfqpoint{4.054925in}{1.410808in}}%
\pgfpathlineto{\pgfqpoint{4.068690in}{1.414224in}}%
\pgfpathlineto{\pgfqpoint{4.082465in}{1.417796in}}%
\pgfpathlineto{\pgfqpoint{4.096252in}{1.421522in}}%
\pgfpathlineto{\pgfqpoint{4.110049in}{1.425403in}}%
\pgfpathlineto{\pgfqpoint{4.117999in}{1.439809in}}%
\pgfpathlineto{\pgfqpoint{4.125944in}{1.454295in}}%
\pgfpathlineto{\pgfqpoint{4.133885in}{1.468856in}}%
\pgfpathlineto{\pgfqpoint{4.141822in}{1.483487in}}%
\pgfpathlineto{\pgfqpoint{4.128026in}{1.479039in}}%
\pgfpathlineto{\pgfqpoint{4.114241in}{1.474745in}}%
\pgfpathlineto{\pgfqpoint{4.100467in}{1.470607in}}%
\pgfpathlineto{\pgfqpoint{4.086705in}{1.466624in}}%
\pgfpathlineto{\pgfqpoint{4.078766in}{1.452550in}}%
\pgfpathlineto{\pgfqpoint{4.070824in}{1.438552in}}%
\pgfpathlineto{\pgfqpoint{4.062876in}{1.424636in}}%
\pgfpathlineto{\pgfqpoint{4.054925in}{1.410808in}}%
\pgfpathclose%
\pgfusepath{fill}%
\end{pgfscope}%
\begin{pgfscope}%
\pgfpathrectangle{\pgfqpoint{1.254980in}{0.150000in}}{\pgfqpoint{5.490039in}{5.490039in}}%
\pgfusepath{clip}%
\pgfsetbuttcap%
\pgfsetroundjoin%
\definecolor{currentfill}{rgb}{0.288921,0.758394,0.428426}%
\pgfsetfillcolor{currentfill}%
\pgfsetfillopacity{0.700000}%
\pgfsetlinewidth{0.000000pt}%
\definecolor{currentstroke}{rgb}{0.000000,0.000000,0.000000}%
\pgfsetstrokecolor{currentstroke}%
\pgfsetdash{}{0pt}%
\pgfpathmoveto{\pgfqpoint{5.241161in}{2.963649in}}%
\pgfpathlineto{\pgfqpoint{5.255654in}{2.979497in}}%
\pgfpathlineto{\pgfqpoint{5.270168in}{2.995513in}}%
\pgfpathlineto{\pgfqpoint{5.284703in}{3.011696in}}%
\pgfpathlineto{\pgfqpoint{5.299261in}{3.028047in}}%
\pgfpathlineto{\pgfqpoint{5.306798in}{3.037743in}}%
\pgfpathlineto{\pgfqpoint{5.314326in}{3.047242in}}%
\pgfpathlineto{\pgfqpoint{5.321843in}{3.056543in}}%
\pgfpathlineto{\pgfqpoint{5.329351in}{3.065647in}}%
\pgfpathlineto{\pgfqpoint{5.314796in}{3.049314in}}%
\pgfpathlineto{\pgfqpoint{5.300262in}{3.033148in}}%
\pgfpathlineto{\pgfqpoint{5.285751in}{3.017150in}}%
\pgfpathlineto{\pgfqpoint{5.271260in}{3.001319in}}%
\pgfpathlineto{\pgfqpoint{5.263750in}{2.992186in}}%
\pgfpathlineto{\pgfqpoint{5.256230in}{2.982863in}}%
\pgfpathlineto{\pgfqpoint{5.248700in}{2.973351in}}%
\pgfpathlineto{\pgfqpoint{5.241161in}{2.963649in}}%
\pgfpathclose%
\pgfusepath{fill}%
\end{pgfscope}%
\begin{pgfscope}%
\pgfpathrectangle{\pgfqpoint{1.254980in}{0.150000in}}{\pgfqpoint{5.490039in}{5.490039in}}%
\pgfusepath{clip}%
\pgfsetbuttcap%
\pgfsetroundjoin%
\definecolor{currentfill}{rgb}{0.135066,0.544853,0.554029}%
\pgfsetfillcolor{currentfill}%
\pgfsetfillopacity{0.700000}%
\pgfsetlinewidth{0.000000pt}%
\definecolor{currentstroke}{rgb}{0.000000,0.000000,0.000000}%
\pgfsetstrokecolor{currentstroke}%
\pgfsetdash{}{0pt}%
\pgfpathmoveto{\pgfqpoint{4.766689in}{2.338588in}}%
\pgfpathlineto{\pgfqpoint{4.780844in}{2.350802in}}%
\pgfpathlineto{\pgfqpoint{4.795017in}{2.363178in}}%
\pgfpathlineto{\pgfqpoint{4.809207in}{2.375717in}}%
\pgfpathlineto{\pgfqpoint{4.823416in}{2.388419in}}%
\pgfpathlineto{\pgfqpoint{4.831191in}{2.403281in}}%
\pgfpathlineto{\pgfqpoint{4.838960in}{2.418001in}}%
\pgfpathlineto{\pgfqpoint{4.846722in}{2.432576in}}%
\pgfpathlineto{\pgfqpoint{4.854479in}{2.447005in}}%
\pgfpathlineto{\pgfqpoint{4.840264in}{2.434070in}}%
\pgfpathlineto{\pgfqpoint{4.826067in}{2.421299in}}%
\pgfpathlineto{\pgfqpoint{4.811889in}{2.408690in}}%
\pgfpathlineto{\pgfqpoint{4.797729in}{2.396244in}}%
\pgfpathlineto{\pgfqpoint{4.789977in}{2.382036in}}%
\pgfpathlineto{\pgfqpoint{4.782221in}{2.367690in}}%
\pgfpathlineto{\pgfqpoint{4.774458in}{2.353206in}}%
\pgfpathlineto{\pgfqpoint{4.766689in}{2.338588in}}%
\pgfpathclose%
\pgfusepath{fill}%
\end{pgfscope}%
\begin{pgfscope}%
\pgfpathrectangle{\pgfqpoint{1.254980in}{0.150000in}}{\pgfqpoint{5.490039in}{5.490039in}}%
\pgfusepath{clip}%
\pgfsetbuttcap%
\pgfsetroundjoin%
\definecolor{currentfill}{rgb}{0.202219,0.715272,0.476084}%
\pgfsetfillcolor{currentfill}%
\pgfsetfillopacity{0.700000}%
\pgfsetlinewidth{0.000000pt}%
\definecolor{currentstroke}{rgb}{0.000000,0.000000,0.000000}%
\pgfsetstrokecolor{currentstroke}%
\pgfsetdash{}{0pt}%
\pgfpathmoveto{\pgfqpoint{5.122756in}{2.817660in}}%
\pgfpathlineto{\pgfqpoint{5.137165in}{2.832780in}}%
\pgfpathlineto{\pgfqpoint{5.151595in}{2.848067in}}%
\pgfpathlineto{\pgfqpoint{5.166046in}{2.863521in}}%
\pgfpathlineto{\pgfqpoint{5.180517in}{2.879141in}}%
\pgfpathlineto{\pgfqpoint{5.188129in}{2.890375in}}%
\pgfpathlineto{\pgfqpoint{5.195733in}{2.901418in}}%
\pgfpathlineto{\pgfqpoint{5.203327in}{2.912269in}}%
\pgfpathlineto{\pgfqpoint{5.210912in}{2.922929in}}%
\pgfpathlineto{\pgfqpoint{5.196440in}{2.907262in}}%
\pgfpathlineto{\pgfqpoint{5.181988in}{2.891762in}}%
\pgfpathlineto{\pgfqpoint{5.167558in}{2.876428in}}%
\pgfpathlineto{\pgfqpoint{5.153148in}{2.861261in}}%
\pgfpathlineto{\pgfqpoint{5.145563in}{2.850636in}}%
\pgfpathlineto{\pgfqpoint{5.137969in}{2.839828in}}%
\pgfpathlineto{\pgfqpoint{5.130367in}{2.828836in}}%
\pgfpathlineto{\pgfqpoint{5.122756in}{2.817660in}}%
\pgfpathclose%
\pgfusepath{fill}%
\end{pgfscope}%
\begin{pgfscope}%
\pgfpathrectangle{\pgfqpoint{1.254980in}{0.150000in}}{\pgfqpoint{5.490039in}{5.490039in}}%
\pgfusepath{clip}%
\pgfsetbuttcap%
\pgfsetroundjoin%
\definecolor{currentfill}{rgb}{0.119738,0.603785,0.541400}%
\pgfsetfillcolor{currentfill}%
\pgfsetfillopacity{0.700000}%
\pgfsetlinewidth{0.000000pt}%
\definecolor{currentstroke}{rgb}{0.000000,0.000000,0.000000}%
\pgfsetstrokecolor{currentstroke}%
\pgfsetdash{}{0pt}%
\pgfpathmoveto{\pgfqpoint{4.885443in}{2.503223in}}%
\pgfpathlineto{\pgfqpoint{4.899682in}{2.516524in}}%
\pgfpathlineto{\pgfqpoint{4.913940in}{2.529988in}}%
\pgfpathlineto{\pgfqpoint{4.928216in}{2.543616in}}%
\pgfpathlineto{\pgfqpoint{4.942512in}{2.557408in}}%
\pgfpathlineto{\pgfqpoint{4.950242in}{2.571263in}}%
\pgfpathlineto{\pgfqpoint{4.957966in}{2.584955in}}%
\pgfpathlineto{\pgfqpoint{4.965682in}{2.598482in}}%
\pgfpathlineto{\pgfqpoint{4.973391in}{2.611843in}}%
\pgfpathlineto{\pgfqpoint{4.959090in}{2.597878in}}%
\pgfpathlineto{\pgfqpoint{4.944809in}{2.584078in}}%
\pgfpathlineto{\pgfqpoint{4.930546in}{2.570443in}}%
\pgfpathlineto{\pgfqpoint{4.916303in}{2.556971in}}%
\pgfpathlineto{\pgfqpoint{4.908598in}{2.543771in}}%
\pgfpathlineto{\pgfqpoint{4.900886in}{2.530412in}}%
\pgfpathlineto{\pgfqpoint{4.893168in}{2.516896in}}%
\pgfpathlineto{\pgfqpoint{4.885443in}{2.503223in}}%
\pgfpathclose%
\pgfusepath{fill}%
\end{pgfscope}%
\begin{pgfscope}%
\pgfpathrectangle{\pgfqpoint{1.254980in}{0.150000in}}{\pgfqpoint{5.490039in}{5.490039in}}%
\pgfusepath{clip}%
\pgfsetbuttcap%
\pgfsetroundjoin%
\definecolor{currentfill}{rgb}{0.137339,0.662252,0.515571}%
\pgfsetfillcolor{currentfill}%
\pgfsetfillopacity{0.700000}%
\pgfsetlinewidth{0.000000pt}%
\definecolor{currentstroke}{rgb}{0.000000,0.000000,0.000000}%
\pgfsetstrokecolor{currentstroke}%
\pgfsetdash{}{0pt}%
\pgfpathmoveto{\pgfqpoint{5.004156in}{2.663598in}}%
\pgfpathlineto{\pgfqpoint{5.018480in}{2.677868in}}%
\pgfpathlineto{\pgfqpoint{5.032824in}{2.692303in}}%
\pgfpathlineto{\pgfqpoint{5.047188in}{2.706904in}}%
\pgfpathlineto{\pgfqpoint{5.061572in}{2.721670in}}%
\pgfpathlineto{\pgfqpoint{5.069248in}{2.734304in}}%
\pgfpathlineto{\pgfqpoint{5.076916in}{2.746758in}}%
\pgfpathlineto{\pgfqpoint{5.084576in}{2.759031in}}%
\pgfpathlineto{\pgfqpoint{5.092229in}{2.771122in}}%
\pgfpathlineto{\pgfqpoint{5.077842in}{2.756245in}}%
\pgfpathlineto{\pgfqpoint{5.063475in}{2.741535in}}%
\pgfpathlineto{\pgfqpoint{5.049128in}{2.726990in}}%
\pgfpathlineto{\pgfqpoint{5.034801in}{2.712610in}}%
\pgfpathlineto{\pgfqpoint{5.027151in}{2.700617in}}%
\pgfpathlineto{\pgfqpoint{5.019494in}{2.688450in}}%
\pgfpathlineto{\pgfqpoint{5.011828in}{2.676110in}}%
\pgfpathlineto{\pgfqpoint{5.004156in}{2.663598in}}%
\pgfpathclose%
\pgfusepath{fill}%
\end{pgfscope}%
\begin{pgfscope}%
\pgfpathrectangle{\pgfqpoint{1.254980in}{0.150000in}}{\pgfqpoint{5.490039in}{5.490039in}}%
\pgfusepath{clip}%
\pgfsetbuttcap%
\pgfsetroundjoin%
\definecolor{currentfill}{rgb}{0.253935,0.265254,0.529983}%
\pgfsetfillcolor{currentfill}%
\pgfsetfillopacity{0.700000}%
\pgfsetlinewidth{0.000000pt}%
\definecolor{currentstroke}{rgb}{0.000000,0.000000,0.000000}%
\pgfsetstrokecolor{currentstroke}%
\pgfsetdash{}{0pt}%
\pgfpathmoveto{\pgfqpoint{4.260487in}{1.625993in}}%
\pgfpathlineto{\pgfqpoint{4.274346in}{1.632274in}}%
\pgfpathlineto{\pgfqpoint{4.288218in}{1.638712in}}%
\pgfpathlineto{\pgfqpoint{4.302103in}{1.645306in}}%
\pgfpathlineto{\pgfqpoint{4.316001in}{1.652056in}}%
\pgfpathlineto{\pgfqpoint{4.323908in}{1.668081in}}%
\pgfpathlineto{\pgfqpoint{4.331811in}{1.684112in}}%
\pgfpathlineto{\pgfqpoint{4.339711in}{1.700145in}}%
\pgfpathlineto{\pgfqpoint{4.347607in}{1.716175in}}%
\pgfpathlineto{\pgfqpoint{4.333705in}{1.708937in}}%
\pgfpathlineto{\pgfqpoint{4.319816in}{1.701854in}}%
\pgfpathlineto{\pgfqpoint{4.305941in}{1.694929in}}%
\pgfpathlineto{\pgfqpoint{4.292079in}{1.688160in}}%
\pgfpathlineto{\pgfqpoint{4.284187in}{1.672608in}}%
\pgfpathlineto{\pgfqpoint{4.276290in}{1.657059in}}%
\pgfpathlineto{\pgfqpoint{4.268390in}{1.641519in}}%
\pgfpathlineto{\pgfqpoint{4.260487in}{1.625993in}}%
\pgfpathclose%
\pgfusepath{fill}%
\end{pgfscope}%
\begin{pgfscope}%
\pgfpathrectangle{\pgfqpoint{1.254980in}{0.150000in}}{\pgfqpoint{5.490039in}{5.490039in}}%
\pgfusepath{clip}%
\pgfsetbuttcap%
\pgfsetroundjoin%
\definecolor{currentfill}{rgb}{0.223925,0.334994,0.548053}%
\pgfsetfillcolor{currentfill}%
\pgfsetfillopacity{0.700000}%
\pgfsetlinewidth{0.000000pt}%
\definecolor{currentstroke}{rgb}{0.000000,0.000000,0.000000}%
\pgfsetstrokecolor{currentstroke}%
\pgfsetdash{}{0pt}%
\pgfpathmoveto{\pgfqpoint{4.379155in}{1.780183in}}%
\pgfpathlineto{\pgfqpoint{4.393075in}{1.788039in}}%
\pgfpathlineto{\pgfqpoint{4.407010in}{1.796054in}}%
\pgfpathlineto{\pgfqpoint{4.420959in}{1.804225in}}%
\pgfpathlineto{\pgfqpoint{4.434923in}{1.812555in}}%
\pgfpathlineto{\pgfqpoint{4.442806in}{1.828951in}}%
\pgfpathlineto{\pgfqpoint{4.450685in}{1.845312in}}%
\pgfpathlineto{\pgfqpoint{4.458560in}{1.861634in}}%
\pgfpathlineto{\pgfqpoint{4.466432in}{1.877913in}}%
\pgfpathlineto{\pgfqpoint{4.452463in}{1.869149in}}%
\pgfpathlineto{\pgfqpoint{4.438508in}{1.860543in}}%
\pgfpathlineto{\pgfqpoint{4.424569in}{1.852096in}}%
\pgfpathlineto{\pgfqpoint{4.410643in}{1.843806in}}%
\pgfpathlineto{\pgfqpoint{4.402777in}{1.827950in}}%
\pgfpathlineto{\pgfqpoint{4.394907in}{1.812058in}}%
\pgfpathlineto{\pgfqpoint{4.387033in}{1.796135in}}%
\pgfpathlineto{\pgfqpoint{4.379155in}{1.780183in}}%
\pgfpathclose%
\pgfusepath{fill}%
\end{pgfscope}%
\begin{pgfscope}%
\pgfpathrectangle{\pgfqpoint{1.254980in}{0.150000in}}{\pgfqpoint{5.490039in}{5.490039in}}%
\pgfusepath{clip}%
\pgfsetbuttcap%
\pgfsetroundjoin%
\definecolor{currentfill}{rgb}{0.280267,0.073417,0.397163}%
\pgfsetfillcolor{currentfill}%
\pgfsetfillopacity{0.700000}%
\pgfsetlinewidth{0.000000pt}%
\definecolor{currentstroke}{rgb}{0.000000,0.000000,0.000000}%
\pgfsetstrokecolor{currentstroke}%
\pgfsetdash{}{0pt}%
\pgfpathmoveto{\pgfqpoint{3.849230in}{1.246854in}}%
\pgfpathlineto{\pgfqpoint{3.862939in}{1.247176in}}%
\pgfpathlineto{\pgfqpoint{3.876656in}{1.247652in}}%
\pgfpathlineto{\pgfqpoint{3.890382in}{1.248283in}}%
\pgfpathlineto{\pgfqpoint{3.904116in}{1.249068in}}%
\pgfpathlineto{\pgfqpoint{3.912133in}{1.260742in}}%
\pgfpathlineto{\pgfqpoint{3.920145in}{1.272581in}}%
\pgfpathlineto{\pgfqpoint{3.928151in}{1.284578in}}%
\pgfpathlineto{\pgfqpoint{3.936151in}{1.296728in}}%
\pgfpathlineto{\pgfqpoint{3.922425in}{1.295297in}}%
\pgfpathlineto{\pgfqpoint{3.908708in}{1.294020in}}%
\pgfpathlineto{\pgfqpoint{3.895000in}{1.292898in}}%
\pgfpathlineto{\pgfqpoint{3.881301in}{1.291931in}}%
\pgfpathlineto{\pgfqpoint{3.873292in}{1.280417in}}%
\pgfpathlineto{\pgfqpoint{3.865277in}{1.269062in}}%
\pgfpathlineto{\pgfqpoint{3.857257in}{1.257872in}}%
\pgfpathlineto{\pgfqpoint{3.849230in}{1.246854in}}%
\pgfpathclose%
\pgfusepath{fill}%
\end{pgfscope}%
\begin{pgfscope}%
\pgfpathrectangle{\pgfqpoint{1.254980in}{0.150000in}}{\pgfqpoint{5.490039in}{5.490039in}}%
\pgfusepath{clip}%
\pgfsetbuttcap%
\pgfsetroundjoin%
\definecolor{currentfill}{rgb}{0.496615,0.826376,0.306377}%
\pgfsetfillcolor{currentfill}%
\pgfsetfillopacity{0.700000}%
\pgfsetlinewidth{0.000000pt}%
\definecolor{currentstroke}{rgb}{0.000000,0.000000,0.000000}%
\pgfsetstrokecolor{currentstroke}%
\pgfsetdash{}{0pt}%
\pgfpathmoveto{\pgfqpoint{5.447451in}{3.197968in}}%
\pgfpathlineto{\pgfqpoint{5.462107in}{3.215008in}}%
\pgfpathlineto{\pgfqpoint{5.476787in}{3.232218in}}%
\pgfpathlineto{\pgfqpoint{5.491490in}{3.249597in}}%
\pgfpathlineto{\pgfqpoint{5.506215in}{3.267145in}}%
\pgfpathlineto{\pgfqpoint{5.513617in}{3.274311in}}%
\pgfpathlineto{\pgfqpoint{5.521008in}{3.281274in}}%
\pgfpathlineto{\pgfqpoint{5.528388in}{3.288037in}}%
\pgfpathlineto{\pgfqpoint{5.535756in}{3.294600in}}%
\pgfpathlineto{\pgfqpoint{5.521038in}{3.277169in}}%
\pgfpathlineto{\pgfqpoint{5.506343in}{3.259906in}}%
\pgfpathlineto{\pgfqpoint{5.491671in}{3.242813in}}%
\pgfpathlineto{\pgfqpoint{5.477021in}{3.225887in}}%
\pgfpathlineto{\pgfqpoint{5.469645in}{3.219196in}}%
\pgfpathlineto{\pgfqpoint{5.462258in}{3.212313in}}%
\pgfpathlineto{\pgfqpoint{5.454860in}{3.205237in}}%
\pgfpathlineto{\pgfqpoint{5.447451in}{3.197968in}}%
\pgfpathclose%
\pgfusepath{fill}%
\end{pgfscope}%
\begin{pgfscope}%
\pgfpathrectangle{\pgfqpoint{1.254980in}{0.150000in}}{\pgfqpoint{5.490039in}{5.490039in}}%
\pgfusepath{clip}%
\pgfsetbuttcap%
\pgfsetroundjoin%
\definecolor{currentfill}{rgb}{0.277018,0.050344,0.375715}%
\pgfsetfillcolor{currentfill}%
\pgfsetfillopacity{0.700000}%
\pgfsetlinewidth{0.000000pt}%
\definecolor{currentstroke}{rgb}{0.000000,0.000000,0.000000}%
\pgfsetstrokecolor{currentstroke}%
\pgfsetdash{}{0pt}%
\pgfpathmoveto{\pgfqpoint{3.762253in}{1.207581in}}%
\pgfpathlineto{\pgfqpoint{3.775943in}{1.206611in}}%
\pgfpathlineto{\pgfqpoint{3.789641in}{1.205795in}}%
\pgfpathlineto{\pgfqpoint{3.803346in}{1.205135in}}%
\pgfpathlineto{\pgfqpoint{3.817059in}{1.204629in}}%
\pgfpathlineto{\pgfqpoint{3.825111in}{1.214895in}}%
\pgfpathlineto{\pgfqpoint{3.833157in}{1.225359in}}%
\pgfpathlineto{\pgfqpoint{3.841197in}{1.236014in}}%
\pgfpathlineto{\pgfqpoint{3.849230in}{1.246854in}}%
\pgfpathlineto{\pgfqpoint{3.835529in}{1.246687in}}%
\pgfpathlineto{\pgfqpoint{3.821836in}{1.246675in}}%
\pgfpathlineto{\pgfqpoint{3.808152in}{1.246818in}}%
\pgfpathlineto{\pgfqpoint{3.794475in}{1.247116in}}%
\pgfpathlineto{\pgfqpoint{3.786430in}{1.236938in}}%
\pgfpathlineto{\pgfqpoint{3.778378in}{1.226952in}}%
\pgfpathlineto{\pgfqpoint{3.770319in}{1.217164in}}%
\pgfpathlineto{\pgfqpoint{3.762253in}{1.207581in}}%
\pgfpathclose%
\pgfusepath{fill}%
\end{pgfscope}%
\begin{pgfscope}%
\pgfpathrectangle{\pgfqpoint{1.254980in}{0.150000in}}{\pgfqpoint{5.490039in}{5.490039in}}%
\pgfusepath{clip}%
\pgfsetbuttcap%
\pgfsetroundjoin%
\definecolor{currentfill}{rgb}{0.274128,0.199721,0.498911}%
\pgfsetfillcolor{currentfill}%
\pgfsetfillopacity{0.700000}%
\pgfsetlinewidth{0.000000pt}%
\definecolor{currentstroke}{rgb}{0.000000,0.000000,0.000000}%
\pgfsetstrokecolor{currentstroke}%
\pgfsetdash{}{0pt}%
\pgfpathmoveto{\pgfqpoint{4.141822in}{1.483487in}}%
\pgfpathlineto{\pgfqpoint{4.155630in}{1.488091in}}%
\pgfpathlineto{\pgfqpoint{4.169449in}{1.492850in}}%
\pgfpathlineto{\pgfqpoint{4.183280in}{1.497763in}}%
\pgfpathlineto{\pgfqpoint{4.197123in}{1.502832in}}%
\pgfpathlineto{\pgfqpoint{4.205057in}{1.518079in}}%
\pgfpathlineto{\pgfqpoint{4.212987in}{1.533379in}}%
\pgfpathlineto{\pgfqpoint{4.220912in}{1.548725in}}%
\pgfpathlineto{\pgfqpoint{4.228835in}{1.564114in}}%
\pgfpathlineto{\pgfqpoint{4.214990in}{1.558503in}}%
\pgfpathlineto{\pgfqpoint{4.201158in}{1.553048in}}%
\pgfpathlineto{\pgfqpoint{4.187339in}{1.547748in}}%
\pgfpathlineto{\pgfqpoint{4.173531in}{1.542604in}}%
\pgfpathlineto{\pgfqpoint{4.165610in}{1.527746in}}%
\pgfpathlineto{\pgfqpoint{4.157684in}{1.512937in}}%
\pgfpathlineto{\pgfqpoint{4.149755in}{1.498182in}}%
\pgfpathlineto{\pgfqpoint{4.141822in}{1.483487in}}%
\pgfpathclose%
\pgfusepath{fill}%
\end{pgfscope}%
\begin{pgfscope}%
\pgfpathrectangle{\pgfqpoint{1.254980in}{0.150000in}}{\pgfqpoint{5.490039in}{5.490039in}}%
\pgfusepath{clip}%
\pgfsetbuttcap%
\pgfsetroundjoin%
\definecolor{currentfill}{rgb}{0.194100,0.399323,0.555565}%
\pgfsetfillcolor{currentfill}%
\pgfsetfillopacity{0.700000}%
\pgfsetlinewidth{0.000000pt}%
\definecolor{currentstroke}{rgb}{0.000000,0.000000,0.000000}%
\pgfsetstrokecolor{currentstroke}%
\pgfsetdash{}{0pt}%
\pgfpathmoveto{\pgfqpoint{4.497880in}{1.942519in}}%
\pgfpathlineto{\pgfqpoint{4.511870in}{1.951848in}}%
\pgfpathlineto{\pgfqpoint{4.525876in}{1.961335in}}%
\pgfpathlineto{\pgfqpoint{4.539897in}{1.970981in}}%
\pgfpathlineto{\pgfqpoint{4.553934in}{1.980787in}}%
\pgfpathlineto{\pgfqpoint{4.561793in}{1.997182in}}%
\pgfpathlineto{\pgfqpoint{4.569647in}{2.013505in}}%
\pgfpathlineto{\pgfqpoint{4.577497in}{2.029752in}}%
\pgfpathlineto{\pgfqpoint{4.585343in}{2.045919in}}%
\pgfpathlineto{\pgfqpoint{4.571300in}{2.035735in}}%
\pgfpathlineto{\pgfqpoint{4.557272in}{2.025710in}}%
\pgfpathlineto{\pgfqpoint{4.543260in}{2.015845in}}%
\pgfpathlineto{\pgfqpoint{4.529263in}{2.006139in}}%
\pgfpathlineto{\pgfqpoint{4.521424in}{1.990339in}}%
\pgfpathlineto{\pgfqpoint{4.513580in}{1.974467in}}%
\pgfpathlineto{\pgfqpoint{4.505732in}{1.958526in}}%
\pgfpathlineto{\pgfqpoint{4.497880in}{1.942519in}}%
\pgfpathclose%
\pgfusepath{fill}%
\end{pgfscope}%
\begin{pgfscope}%
\pgfpathrectangle{\pgfqpoint{1.254980in}{0.150000in}}{\pgfqpoint{5.490039in}{5.490039in}}%
\pgfusepath{clip}%
\pgfsetbuttcap%
\pgfsetroundjoin%
\definecolor{currentfill}{rgb}{0.282910,0.105393,0.426902}%
\pgfsetfillcolor{currentfill}%
\pgfsetfillopacity{0.700000}%
\pgfsetlinewidth{0.000000pt}%
\definecolor{currentstroke}{rgb}{0.000000,0.000000,0.000000}%
\pgfsetstrokecolor{currentstroke}%
\pgfsetdash{}{0pt}%
\pgfpathmoveto{\pgfqpoint{3.936151in}{1.296728in}}%
\pgfpathlineto{\pgfqpoint{3.949886in}{1.298314in}}%
\pgfpathlineto{\pgfqpoint{3.963631in}{1.300054in}}%
\pgfpathlineto{\pgfqpoint{3.977385in}{1.301948in}}%
\pgfpathlineto{\pgfqpoint{3.991148in}{1.303996in}}%
\pgfpathlineto{\pgfqpoint{3.999137in}{1.316923in}}%
\pgfpathlineto{\pgfqpoint{4.007121in}{1.329983in}}%
\pgfpathlineto{\pgfqpoint{4.015100in}{1.343169in}}%
\pgfpathlineto{\pgfqpoint{4.023074in}{1.356478in}}%
\pgfpathlineto{\pgfqpoint{4.009316in}{1.353809in}}%
\pgfpathlineto{\pgfqpoint{3.995567in}{1.351294in}}%
\pgfpathlineto{\pgfqpoint{3.981829in}{1.348935in}}%
\pgfpathlineto{\pgfqpoint{3.968100in}{1.346730in}}%
\pgfpathlineto{\pgfqpoint{3.960121in}{1.334031in}}%
\pgfpathlineto{\pgfqpoint{3.952136in}{1.321461in}}%
\pgfpathlineto{\pgfqpoint{3.944146in}{1.309025in}}%
\pgfpathlineto{\pgfqpoint{3.936151in}{1.296728in}}%
\pgfpathclose%
\pgfusepath{fill}%
\end{pgfscope}%
\begin{pgfscope}%
\pgfpathrectangle{\pgfqpoint{1.254980in}{0.150000in}}{\pgfqpoint{5.490039in}{5.490039in}}%
\pgfusepath{clip}%
\pgfsetbuttcap%
\pgfsetroundjoin%
\definecolor{currentfill}{rgb}{0.166617,0.463708,0.558119}%
\pgfsetfillcolor{currentfill}%
\pgfsetfillopacity{0.700000}%
\pgfsetlinewidth{0.000000pt}%
\definecolor{currentstroke}{rgb}{0.000000,0.000000,0.000000}%
\pgfsetstrokecolor{currentstroke}%
\pgfsetdash{}{0pt}%
\pgfpathmoveto{\pgfqpoint{4.616684in}{2.109732in}}%
\pgfpathlineto{\pgfqpoint{4.630750in}{2.120426in}}%
\pgfpathlineto{\pgfqpoint{4.644834in}{2.131281in}}%
\pgfpathlineto{\pgfqpoint{4.658934in}{2.142296in}}%
\pgfpathlineto{\pgfqpoint{4.673051in}{2.153472in}}%
\pgfpathlineto{\pgfqpoint{4.680882in}{2.169527in}}%
\pgfpathlineto{\pgfqpoint{4.688708in}{2.185477in}}%
\pgfpathlineto{\pgfqpoint{4.696529in}{2.201319in}}%
\pgfpathlineto{\pgfqpoint{4.704345in}{2.217050in}}%
\pgfpathlineto{\pgfqpoint{4.690221in}{2.205552in}}%
\pgfpathlineto{\pgfqpoint{4.676114in}{2.194215in}}%
\pgfpathlineto{\pgfqpoint{4.662024in}{2.183039in}}%
\pgfpathlineto{\pgfqpoint{4.647950in}{2.172024in}}%
\pgfpathlineto{\pgfqpoint{4.640141in}{2.156604in}}%
\pgfpathlineto{\pgfqpoint{4.632326in}{2.141080in}}%
\pgfpathlineto{\pgfqpoint{4.624507in}{2.125455in}}%
\pgfpathlineto{\pgfqpoint{4.616684in}{2.109732in}}%
\pgfpathclose%
\pgfusepath{fill}%
\end{pgfscope}%
\begin{pgfscope}%
\pgfpathrectangle{\pgfqpoint{1.254980in}{0.150000in}}{\pgfqpoint{5.490039in}{5.490039in}}%
\pgfusepath{clip}%
\pgfsetbuttcap%
\pgfsetroundjoin%
\definecolor{currentfill}{rgb}{0.273809,0.031497,0.358853}%
\pgfsetfillcolor{currentfill}%
\pgfsetfillopacity{0.700000}%
\pgfsetlinewidth{0.000000pt}%
\definecolor{currentstroke}{rgb}{0.000000,0.000000,0.000000}%
\pgfsetstrokecolor{currentstroke}%
\pgfsetdash{}{0pt}%
\pgfpathmoveto{\pgfqpoint{3.675160in}{1.179665in}}%
\pgfpathlineto{\pgfqpoint{3.688839in}{1.177373in}}%
\pgfpathlineto{\pgfqpoint{3.702525in}{1.175237in}}%
\pgfpathlineto{\pgfqpoint{3.716217in}{1.173257in}}%
\pgfpathlineto{\pgfqpoint{3.729915in}{1.171432in}}%
\pgfpathlineto{\pgfqpoint{3.738011in}{1.180128in}}%
\pgfpathlineto{\pgfqpoint{3.746100in}{1.189056in}}%
\pgfpathlineto{\pgfqpoint{3.754180in}{1.198209in}}%
\pgfpathlineto{\pgfqpoint{3.762253in}{1.207581in}}%
\pgfpathlineto{\pgfqpoint{3.748570in}{1.208706in}}%
\pgfpathlineto{\pgfqpoint{3.734894in}{1.209987in}}%
\pgfpathlineto{\pgfqpoint{3.721225in}{1.211424in}}%
\pgfpathlineto{\pgfqpoint{3.707563in}{1.213016in}}%
\pgfpathlineto{\pgfqpoint{3.699474in}{1.204334in}}%
\pgfpathlineto{\pgfqpoint{3.691378in}{1.195876in}}%
\pgfpathlineto{\pgfqpoint{3.683273in}{1.187651in}}%
\pgfpathlineto{\pgfqpoint{3.675160in}{1.179665in}}%
\pgfpathclose%
\pgfusepath{fill}%
\end{pgfscope}%
\begin{pgfscope}%
\pgfpathrectangle{\pgfqpoint{1.254980in}{0.150000in}}{\pgfqpoint{5.490039in}{5.490039in}}%
\pgfusepath{clip}%
\pgfsetbuttcap%
\pgfsetroundjoin%
\definecolor{currentfill}{rgb}{0.282623,0.140926,0.457517}%
\pgfsetfillcolor{currentfill}%
\pgfsetfillopacity{0.700000}%
\pgfsetlinewidth{0.000000pt}%
\definecolor{currentstroke}{rgb}{0.000000,0.000000,0.000000}%
\pgfsetstrokecolor{currentstroke}%
\pgfsetdash{}{0pt}%
\pgfpathmoveto{\pgfqpoint{4.023074in}{1.356478in}}%
\pgfpathlineto{\pgfqpoint{4.036842in}{1.359301in}}%
\pgfpathlineto{\pgfqpoint{4.050621in}{1.362278in}}%
\pgfpathlineto{\pgfqpoint{4.064410in}{1.365410in}}%
\pgfpathlineto{\pgfqpoint{4.078210in}{1.368696in}}%
\pgfpathlineto{\pgfqpoint{4.086176in}{1.382724in}}%
\pgfpathlineto{\pgfqpoint{4.094138in}{1.396855in}}%
\pgfpathlineto{\pgfqpoint{4.102096in}{1.411084in}}%
\pgfpathlineto{\pgfqpoint{4.110049in}{1.425403in}}%
\pgfpathlineto{\pgfqpoint{4.096252in}{1.421522in}}%
\pgfpathlineto{\pgfqpoint{4.082465in}{1.417796in}}%
\pgfpathlineto{\pgfqpoint{4.068690in}{1.414224in}}%
\pgfpathlineto{\pgfqpoint{4.054925in}{1.410808in}}%
\pgfpathlineto{\pgfqpoint{4.046969in}{1.397072in}}%
\pgfpathlineto{\pgfqpoint{4.039008in}{1.383435in}}%
\pgfpathlineto{\pgfqpoint{4.031044in}{1.369901in}}%
\pgfpathlineto{\pgfqpoint{4.023074in}{1.356478in}}%
\pgfpathclose%
\pgfusepath{fill}%
\end{pgfscope}%
\begin{pgfscope}%
\pgfpathrectangle{\pgfqpoint{1.254980in}{0.150000in}}{\pgfqpoint{5.490039in}{5.490039in}}%
\pgfusepath{clip}%
\pgfsetbuttcap%
\pgfsetroundjoin%
\definecolor{currentfill}{rgb}{0.386433,0.794644,0.372886}%
\pgfsetfillcolor{currentfill}%
\pgfsetfillopacity{0.700000}%
\pgfsetlinewidth{0.000000pt}%
\definecolor{currentstroke}{rgb}{0.000000,0.000000,0.000000}%
\pgfsetstrokecolor{currentstroke}%
\pgfsetdash{}{0pt}%
\pgfpathmoveto{\pgfqpoint{5.329351in}{3.065647in}}%
\pgfpathlineto{\pgfqpoint{5.343928in}{3.082148in}}%
\pgfpathlineto{\pgfqpoint{5.358527in}{3.098817in}}%
\pgfpathlineto{\pgfqpoint{5.373148in}{3.115654in}}%
\pgfpathlineto{\pgfqpoint{5.387792in}{3.132660in}}%
\pgfpathlineto{\pgfqpoint{5.395286in}{3.141529in}}%
\pgfpathlineto{\pgfqpoint{5.402770in}{3.150195in}}%
\pgfpathlineto{\pgfqpoint{5.410244in}{3.158658in}}%
\pgfpathlineto{\pgfqpoint{5.417707in}{3.166919in}}%
\pgfpathlineto{\pgfqpoint{5.403067in}{3.149963in}}%
\pgfpathlineto{\pgfqpoint{5.388449in}{3.133176in}}%
\pgfpathlineto{\pgfqpoint{5.373854in}{3.116558in}}%
\pgfpathlineto{\pgfqpoint{5.359281in}{3.100107in}}%
\pgfpathlineto{\pgfqpoint{5.351814in}{3.091783in}}%
\pgfpathlineto{\pgfqpoint{5.344336in}{3.083266in}}%
\pgfpathlineto{\pgfqpoint{5.336849in}{3.074554in}}%
\pgfpathlineto{\pgfqpoint{5.329351in}{3.065647in}}%
\pgfpathclose%
\pgfusepath{fill}%
\end{pgfscope}%
\begin{pgfscope}%
\pgfpathrectangle{\pgfqpoint{1.254980in}{0.150000in}}{\pgfqpoint{5.490039in}{5.490039in}}%
\pgfusepath{clip}%
\pgfsetbuttcap%
\pgfsetroundjoin%
\definecolor{currentfill}{rgb}{0.140536,0.530132,0.555659}%
\pgfsetfillcolor{currentfill}%
\pgfsetfillopacity{0.700000}%
\pgfsetlinewidth{0.000000pt}%
\definecolor{currentstroke}{rgb}{0.000000,0.000000,0.000000}%
\pgfsetstrokecolor{currentstroke}%
\pgfsetdash{}{0pt}%
\pgfpathmoveto{\pgfqpoint{4.735560in}{2.278809in}}%
\pgfpathlineto{\pgfqpoint{4.749708in}{2.290761in}}%
\pgfpathlineto{\pgfqpoint{4.763874in}{2.302874in}}%
\pgfpathlineto{\pgfqpoint{4.778058in}{2.315150in}}%
\pgfpathlineto{\pgfqpoint{4.792260in}{2.327588in}}%
\pgfpathlineto{\pgfqpoint{4.800057in}{2.342999in}}%
\pgfpathlineto{\pgfqpoint{4.807849in}{2.358276in}}%
\pgfpathlineto{\pgfqpoint{4.815635in}{2.373416in}}%
\pgfpathlineto{\pgfqpoint{4.823416in}{2.388419in}}%
\pgfpathlineto{\pgfqpoint{4.809207in}{2.375717in}}%
\pgfpathlineto{\pgfqpoint{4.795017in}{2.363178in}}%
\pgfpathlineto{\pgfqpoint{4.780844in}{2.350802in}}%
\pgfpathlineto{\pgfqpoint{4.766689in}{2.338588in}}%
\pgfpathlineto{\pgfqpoint{4.758915in}{2.323837in}}%
\pgfpathlineto{\pgfqpoint{4.751136in}{2.308955in}}%
\pgfpathlineto{\pgfqpoint{4.743350in}{2.293945in}}%
\pgfpathlineto{\pgfqpoint{4.735560in}{2.278809in}}%
\pgfpathclose%
\pgfusepath{fill}%
\end{pgfscope}%
\begin{pgfscope}%
\pgfpathrectangle{\pgfqpoint{1.254980in}{0.150000in}}{\pgfqpoint{5.490039in}{5.490039in}}%
\pgfusepath{clip}%
\pgfsetbuttcap%
\pgfsetroundjoin%
\definecolor{currentfill}{rgb}{0.595839,0.848717,0.243329}%
\pgfsetfillcolor{currentfill}%
\pgfsetfillopacity{0.700000}%
\pgfsetlinewidth{0.000000pt}%
\definecolor{currentstroke}{rgb}{0.000000,0.000000,0.000000}%
\pgfsetstrokecolor{currentstroke}%
\pgfsetdash{}{0pt}%
\pgfpathmoveto{\pgfqpoint{5.535756in}{3.294600in}}%
\pgfpathlineto{\pgfqpoint{5.550497in}{3.312201in}}%
\pgfpathlineto{\pgfqpoint{5.565262in}{3.329972in}}%
\pgfpathlineto{\pgfqpoint{5.580050in}{3.347912in}}%
\pgfpathlineto{\pgfqpoint{5.587400in}{3.354175in}}%
\pgfpathlineto{\pgfqpoint{5.594738in}{3.360236in}}%
\pgfpathlineto{\pgfqpoint{5.602065in}{3.366098in}}%
\pgfpathlineto{\pgfqpoint{5.609379in}{3.371762in}}%
\pgfpathlineto{\pgfqpoint{5.594601in}{3.353971in}}%
\pgfpathlineto{\pgfqpoint{5.579846in}{3.336351in}}%
\pgfpathlineto{\pgfqpoint{5.565114in}{3.318899in}}%
\pgfpathlineto{\pgfqpoint{5.557791in}{3.313114in}}%
\pgfpathlineto{\pgfqpoint{5.550458in}{3.307137in}}%
\pgfpathlineto{\pgfqpoint{5.543113in}{3.300966in}}%
\pgfpathlineto{\pgfqpoint{5.535756in}{3.294600in}}%
\pgfpathclose%
\pgfusepath{fill}%
\end{pgfscope}%
\begin{pgfscope}%
\pgfpathrectangle{\pgfqpoint{1.254980in}{0.150000in}}{\pgfqpoint{5.490039in}{5.490039in}}%
\pgfusepath{clip}%
\pgfsetbuttcap%
\pgfsetroundjoin%
\definecolor{currentfill}{rgb}{0.233603,0.313828,0.543914}%
\pgfsetfillcolor{currentfill}%
\pgfsetfillopacity{0.700000}%
\pgfsetlinewidth{0.000000pt}%
\definecolor{currentstroke}{rgb}{0.000000,0.000000,0.000000}%
\pgfsetstrokecolor{currentstroke}%
\pgfsetdash{}{0pt}%
\pgfpathmoveto{\pgfqpoint{4.347607in}{1.716175in}}%
\pgfpathlineto{\pgfqpoint{4.361523in}{1.723571in}}%
\pgfpathlineto{\pgfqpoint{4.375453in}{1.731124in}}%
\pgfpathlineto{\pgfqpoint{4.389396in}{1.738833in}}%
\pgfpathlineto{\pgfqpoint{4.403354in}{1.746699in}}%
\pgfpathlineto{\pgfqpoint{4.411252in}{1.763195in}}%
\pgfpathlineto{\pgfqpoint{4.419146in}{1.779673in}}%
\pgfpathlineto{\pgfqpoint{4.427036in}{1.796127in}}%
\pgfpathlineto{\pgfqpoint{4.434923in}{1.812555in}}%
\pgfpathlineto{\pgfqpoint{4.420959in}{1.804225in}}%
\pgfpathlineto{\pgfqpoint{4.407010in}{1.796054in}}%
\pgfpathlineto{\pgfqpoint{4.393075in}{1.788039in}}%
\pgfpathlineto{\pgfqpoint{4.379155in}{1.780183in}}%
\pgfpathlineto{\pgfqpoint{4.371274in}{1.764206in}}%
\pgfpathlineto{\pgfqpoint{4.363388in}{1.748210in}}%
\pgfpathlineto{\pgfqpoint{4.355499in}{1.732199in}}%
\pgfpathlineto{\pgfqpoint{4.347607in}{1.716175in}}%
\pgfpathclose%
\pgfusepath{fill}%
\end{pgfscope}%
\begin{pgfscope}%
\pgfpathrectangle{\pgfqpoint{1.254980in}{0.150000in}}{\pgfqpoint{5.490039in}{5.490039in}}%
\pgfusepath{clip}%
\pgfsetbuttcap%
\pgfsetroundjoin%
\definecolor{currentfill}{rgb}{0.121831,0.589055,0.545623}%
\pgfsetfillcolor{currentfill}%
\pgfsetfillopacity{0.700000}%
\pgfsetlinewidth{0.000000pt}%
\definecolor{currentstroke}{rgb}{0.000000,0.000000,0.000000}%
\pgfsetstrokecolor{currentstroke}%
\pgfsetdash{}{0pt}%
\pgfpathmoveto{\pgfqpoint{4.854479in}{2.447005in}}%
\pgfpathlineto{\pgfqpoint{4.868712in}{2.460103in}}%
\pgfpathlineto{\pgfqpoint{4.882964in}{2.473364in}}%
\pgfpathlineto{\pgfqpoint{4.897235in}{2.486789in}}%
\pgfpathlineto{\pgfqpoint{4.911525in}{2.500377in}}%
\pgfpathlineto{\pgfqpoint{4.919281in}{2.514873in}}%
\pgfpathlineto{\pgfqpoint{4.927032in}{2.529211in}}%
\pgfpathlineto{\pgfqpoint{4.934775in}{2.543390in}}%
\pgfpathlineto{\pgfqpoint{4.942512in}{2.557408in}}%
\pgfpathlineto{\pgfqpoint{4.928216in}{2.543616in}}%
\pgfpathlineto{\pgfqpoint{4.913940in}{2.529988in}}%
\pgfpathlineto{\pgfqpoint{4.899682in}{2.516524in}}%
\pgfpathlineto{\pgfqpoint{4.885443in}{2.503223in}}%
\pgfpathlineto{\pgfqpoint{4.877712in}{2.489397in}}%
\pgfpathlineto{\pgfqpoint{4.869974in}{2.475417in}}%
\pgfpathlineto{\pgfqpoint{4.862230in}{2.461286in}}%
\pgfpathlineto{\pgfqpoint{4.854479in}{2.447005in}}%
\pgfpathclose%
\pgfusepath{fill}%
\end{pgfscope}%
\begin{pgfscope}%
\pgfpathrectangle{\pgfqpoint{1.254980in}{0.150000in}}{\pgfqpoint{5.490039in}{5.490039in}}%
\pgfusepath{clip}%
\pgfsetbuttcap%
\pgfsetroundjoin%
\definecolor{currentfill}{rgb}{0.260571,0.246922,0.522828}%
\pgfsetfillcolor{currentfill}%
\pgfsetfillopacity{0.700000}%
\pgfsetlinewidth{0.000000pt}%
\definecolor{currentstroke}{rgb}{0.000000,0.000000,0.000000}%
\pgfsetstrokecolor{currentstroke}%
\pgfsetdash{}{0pt}%
\pgfpathmoveto{\pgfqpoint{4.228835in}{1.564114in}}%
\pgfpathlineto{\pgfqpoint{4.242691in}{1.569881in}}%
\pgfpathlineto{\pgfqpoint{4.256561in}{1.575803in}}%
\pgfpathlineto{\pgfqpoint{4.270443in}{1.581880in}}%
\pgfpathlineto{\pgfqpoint{4.284338in}{1.588114in}}%
\pgfpathlineto{\pgfqpoint{4.292259in}{1.604066in}}%
\pgfpathlineto{\pgfqpoint{4.300176in}{1.620044in}}%
\pgfpathlineto{\pgfqpoint{4.308090in}{1.636042in}}%
\pgfpathlineto{\pgfqpoint{4.316001in}{1.652056in}}%
\pgfpathlineto{\pgfqpoint{4.302103in}{1.645306in}}%
\pgfpathlineto{\pgfqpoint{4.288218in}{1.638712in}}%
\pgfpathlineto{\pgfqpoint{4.274346in}{1.632274in}}%
\pgfpathlineto{\pgfqpoint{4.260487in}{1.625993in}}%
\pgfpathlineto{\pgfqpoint{4.252579in}{1.610484in}}%
\pgfpathlineto{\pgfqpoint{4.244668in}{1.594998in}}%
\pgfpathlineto{\pgfqpoint{4.236753in}{1.579540in}}%
\pgfpathlineto{\pgfqpoint{4.228835in}{1.564114in}}%
\pgfpathclose%
\pgfusepath{fill}%
\end{pgfscope}%
\begin{pgfscope}%
\pgfpathrectangle{\pgfqpoint{1.254980in}{0.150000in}}{\pgfqpoint{5.490039in}{5.490039in}}%
\pgfusepath{clip}%
\pgfsetbuttcap%
\pgfsetroundjoin%
\definecolor{currentfill}{rgb}{0.281477,0.755203,0.432552}%
\pgfsetfillcolor{currentfill}%
\pgfsetfillopacity{0.700000}%
\pgfsetlinewidth{0.000000pt}%
\definecolor{currentstroke}{rgb}{0.000000,0.000000,0.000000}%
\pgfsetstrokecolor{currentstroke}%
\pgfsetdash{}{0pt}%
\pgfpathmoveto{\pgfqpoint{5.210912in}{2.922929in}}%
\pgfpathlineto{\pgfqpoint{5.225405in}{2.938762in}}%
\pgfpathlineto{\pgfqpoint{5.239920in}{2.954763in}}%
\pgfpathlineto{\pgfqpoint{5.254456in}{2.970932in}}%
\pgfpathlineto{\pgfqpoint{5.269014in}{2.987268in}}%
\pgfpathlineto{\pgfqpoint{5.276590in}{2.997762in}}%
\pgfpathlineto{\pgfqpoint{5.284157in}{3.008056in}}%
\pgfpathlineto{\pgfqpoint{5.291714in}{3.018151in}}%
\pgfpathlineto{\pgfqpoint{5.299261in}{3.028047in}}%
\pgfpathlineto{\pgfqpoint{5.284703in}{3.011696in}}%
\pgfpathlineto{\pgfqpoint{5.270168in}{2.995513in}}%
\pgfpathlineto{\pgfqpoint{5.255654in}{2.979497in}}%
\pgfpathlineto{\pgfqpoint{5.241161in}{2.963649in}}%
\pgfpathlineto{\pgfqpoint{5.233613in}{2.953756in}}%
\pgfpathlineto{\pgfqpoint{5.226055in}{2.943672in}}%
\pgfpathlineto{\pgfqpoint{5.218488in}{2.933396in}}%
\pgfpathlineto{\pgfqpoint{5.210912in}{2.922929in}}%
\pgfpathclose%
\pgfusepath{fill}%
\end{pgfscope}%
\begin{pgfscope}%
\pgfpathrectangle{\pgfqpoint{1.254980in}{0.150000in}}{\pgfqpoint{5.490039in}{5.490039in}}%
\pgfusepath{clip}%
\pgfsetbuttcap%
\pgfsetroundjoin%
\definecolor{currentfill}{rgb}{0.130067,0.651384,0.521608}%
\pgfsetfillcolor{currentfill}%
\pgfsetfillopacity{0.700000}%
\pgfsetlinewidth{0.000000pt}%
\definecolor{currentstroke}{rgb}{0.000000,0.000000,0.000000}%
\pgfsetstrokecolor{currentstroke}%
\pgfsetdash{}{0pt}%
\pgfpathmoveto{\pgfqpoint{4.973391in}{2.611843in}}%
\pgfpathlineto{\pgfqpoint{4.987711in}{2.625972in}}%
\pgfpathlineto{\pgfqpoint{5.002051in}{2.640266in}}%
\pgfpathlineto{\pgfqpoint{5.016410in}{2.654725in}}%
\pgfpathlineto{\pgfqpoint{5.030790in}{2.669349in}}%
\pgfpathlineto{\pgfqpoint{5.038497in}{2.682695in}}%
\pgfpathlineto{\pgfqpoint{5.046196in}{2.695865in}}%
\pgfpathlineto{\pgfqpoint{5.053888in}{2.708857in}}%
\pgfpathlineto{\pgfqpoint{5.061572in}{2.721670in}}%
\pgfpathlineto{\pgfqpoint{5.047188in}{2.706904in}}%
\pgfpathlineto{\pgfqpoint{5.032824in}{2.692303in}}%
\pgfpathlineto{\pgfqpoint{5.018480in}{2.677868in}}%
\pgfpathlineto{\pgfqpoint{5.004156in}{2.663598in}}%
\pgfpathlineto{\pgfqpoint{4.996476in}{2.650914in}}%
\pgfpathlineto{\pgfqpoint{4.988788in}{2.638060in}}%
\pgfpathlineto{\pgfqpoint{4.981093in}{2.625036in}}%
\pgfpathlineto{\pgfqpoint{4.973391in}{2.611843in}}%
\pgfpathclose%
\pgfusepath{fill}%
\end{pgfscope}%
\begin{pgfscope}%
\pgfpathrectangle{\pgfqpoint{1.254980in}{0.150000in}}{\pgfqpoint{5.490039in}{5.490039in}}%
\pgfusepath{clip}%
\pgfsetbuttcap%
\pgfsetroundjoin%
\definecolor{currentfill}{rgb}{0.201239,0.383670,0.554294}%
\pgfsetfillcolor{currentfill}%
\pgfsetfillopacity{0.700000}%
\pgfsetlinewidth{0.000000pt}%
\definecolor{currentstroke}{rgb}{0.000000,0.000000,0.000000}%
\pgfsetstrokecolor{currentstroke}%
\pgfsetdash{}{0pt}%
\pgfpathmoveto{\pgfqpoint{4.466432in}{1.877913in}}%
\pgfpathlineto{\pgfqpoint{4.480416in}{1.886835in}}%
\pgfpathlineto{\pgfqpoint{4.494416in}{1.895915in}}%
\pgfpathlineto{\pgfqpoint{4.508430in}{1.905154in}}%
\pgfpathlineto{\pgfqpoint{4.522460in}{1.914551in}}%
\pgfpathlineto{\pgfqpoint{4.530334in}{1.931201in}}%
\pgfpathlineto{\pgfqpoint{4.538205in}{1.947793in}}%
\pgfpathlineto{\pgfqpoint{4.546072in}{1.964322in}}%
\pgfpathlineto{\pgfqpoint{4.553934in}{1.980787in}}%
\pgfpathlineto{\pgfqpoint{4.539897in}{1.970981in}}%
\pgfpathlineto{\pgfqpoint{4.525876in}{1.961335in}}%
\pgfpathlineto{\pgfqpoint{4.511870in}{1.951848in}}%
\pgfpathlineto{\pgfqpoint{4.497880in}{1.942519in}}%
\pgfpathlineto{\pgfqpoint{4.490024in}{1.926451in}}%
\pgfpathlineto{\pgfqpoint{4.482164in}{1.910325in}}%
\pgfpathlineto{\pgfqpoint{4.474300in}{1.894144in}}%
\pgfpathlineto{\pgfqpoint{4.466432in}{1.877913in}}%
\pgfpathclose%
\pgfusepath{fill}%
\end{pgfscope}%
\begin{pgfscope}%
\pgfpathrectangle{\pgfqpoint{1.254980in}{0.150000in}}{\pgfqpoint{5.490039in}{5.490039in}}%
\pgfusepath{clip}%
\pgfsetbuttcap%
\pgfsetroundjoin%
\definecolor{currentfill}{rgb}{0.185783,0.704891,0.485273}%
\pgfsetfillcolor{currentfill}%
\pgfsetfillopacity{0.700000}%
\pgfsetlinewidth{0.000000pt}%
\definecolor{currentstroke}{rgb}{0.000000,0.000000,0.000000}%
\pgfsetstrokecolor{currentstroke}%
\pgfsetdash{}{0pt}%
\pgfpathmoveto{\pgfqpoint{5.092229in}{2.771122in}}%
\pgfpathlineto{\pgfqpoint{5.106636in}{2.786164in}}%
\pgfpathlineto{\pgfqpoint{5.121064in}{2.801372in}}%
\pgfpathlineto{\pgfqpoint{5.135512in}{2.816747in}}%
\pgfpathlineto{\pgfqpoint{5.149981in}{2.832288in}}%
\pgfpathlineto{\pgfqpoint{5.157628in}{2.844287in}}%
\pgfpathlineto{\pgfqpoint{5.165267in}{2.856096in}}%
\pgfpathlineto{\pgfqpoint{5.172896in}{2.867714in}}%
\pgfpathlineto{\pgfqpoint{5.180517in}{2.879141in}}%
\pgfpathlineto{\pgfqpoint{5.166046in}{2.863521in}}%
\pgfpathlineto{\pgfqpoint{5.151595in}{2.848067in}}%
\pgfpathlineto{\pgfqpoint{5.137165in}{2.832780in}}%
\pgfpathlineto{\pgfqpoint{5.122756in}{2.817660in}}%
\pgfpathlineto{\pgfqpoint{5.115137in}{2.806300in}}%
\pgfpathlineto{\pgfqpoint{5.107509in}{2.794757in}}%
\pgfpathlineto{\pgfqpoint{5.099873in}{2.783031in}}%
\pgfpathlineto{\pgfqpoint{5.092229in}{2.771122in}}%
\pgfpathclose%
\pgfusepath{fill}%
\end{pgfscope}%
\begin{pgfscope}%
\pgfpathrectangle{\pgfqpoint{1.254980in}{0.150000in}}{\pgfqpoint{5.490039in}{5.490039in}}%
\pgfusepath{clip}%
\pgfsetbuttcap%
\pgfsetroundjoin%
\definecolor{currentfill}{rgb}{0.278012,0.180367,0.486697}%
\pgfsetfillcolor{currentfill}%
\pgfsetfillopacity{0.700000}%
\pgfsetlinewidth{0.000000pt}%
\definecolor{currentstroke}{rgb}{0.000000,0.000000,0.000000}%
\pgfsetstrokecolor{currentstroke}%
\pgfsetdash{}{0pt}%
\pgfpathmoveto{\pgfqpoint{4.110049in}{1.425403in}}%
\pgfpathlineto{\pgfqpoint{4.123858in}{1.429439in}}%
\pgfpathlineto{\pgfqpoint{4.137678in}{1.433629in}}%
\pgfpathlineto{\pgfqpoint{4.151509in}{1.437973in}}%
\pgfpathlineto{\pgfqpoint{4.165352in}{1.442472in}}%
\pgfpathlineto{\pgfqpoint{4.173300in}{1.457457in}}%
\pgfpathlineto{\pgfqpoint{4.181245in}{1.472516in}}%
\pgfpathlineto{\pgfqpoint{4.189186in}{1.487643in}}%
\pgfpathlineto{\pgfqpoint{4.197123in}{1.502832in}}%
\pgfpathlineto{\pgfqpoint{4.183280in}{1.497763in}}%
\pgfpathlineto{\pgfqpoint{4.169449in}{1.492850in}}%
\pgfpathlineto{\pgfqpoint{4.155630in}{1.488091in}}%
\pgfpathlineto{\pgfqpoint{4.141822in}{1.483487in}}%
\pgfpathlineto{\pgfqpoint{4.133885in}{1.468856in}}%
\pgfpathlineto{\pgfqpoint{4.125944in}{1.454295in}}%
\pgfpathlineto{\pgfqpoint{4.117999in}{1.439809in}}%
\pgfpathlineto{\pgfqpoint{4.110049in}{1.425403in}}%
\pgfpathclose%
\pgfusepath{fill}%
\end{pgfscope}%
\begin{pgfscope}%
\pgfpathrectangle{\pgfqpoint{1.254980in}{0.150000in}}{\pgfqpoint{5.490039in}{5.490039in}}%
\pgfusepath{clip}%
\pgfsetbuttcap%
\pgfsetroundjoin%
\definecolor{currentfill}{rgb}{0.172719,0.448791,0.557885}%
\pgfsetfillcolor{currentfill}%
\pgfsetfillopacity{0.700000}%
\pgfsetlinewidth{0.000000pt}%
\definecolor{currentstroke}{rgb}{0.000000,0.000000,0.000000}%
\pgfsetstrokecolor{currentstroke}%
\pgfsetdash{}{0pt}%
\pgfpathmoveto{\pgfqpoint{4.585343in}{2.045919in}}%
\pgfpathlineto{\pgfqpoint{4.599403in}{2.056264in}}%
\pgfpathlineto{\pgfqpoint{4.613480in}{2.066767in}}%
\pgfpathlineto{\pgfqpoint{4.627572in}{2.077431in}}%
\pgfpathlineto{\pgfqpoint{4.641681in}{2.088255in}}%
\pgfpathlineto{\pgfqpoint{4.649530in}{2.104702in}}%
\pgfpathlineto{\pgfqpoint{4.657375in}{2.121056in}}%
\pgfpathlineto{\pgfqpoint{4.665215in}{2.137314in}}%
\pgfpathlineto{\pgfqpoint{4.673051in}{2.153472in}}%
\pgfpathlineto{\pgfqpoint{4.658934in}{2.142296in}}%
\pgfpathlineto{\pgfqpoint{4.644834in}{2.131281in}}%
\pgfpathlineto{\pgfqpoint{4.630750in}{2.120426in}}%
\pgfpathlineto{\pgfqpoint{4.616684in}{2.109732in}}%
\pgfpathlineto{\pgfqpoint{4.608855in}{2.093914in}}%
\pgfpathlineto{\pgfqpoint{4.601022in}{2.078003in}}%
\pgfpathlineto{\pgfqpoint{4.593185in}{2.062004in}}%
\pgfpathlineto{\pgfqpoint{4.585343in}{2.045919in}}%
\pgfpathclose%
\pgfusepath{fill}%
\end{pgfscope}%
\begin{pgfscope}%
\pgfpathrectangle{\pgfqpoint{1.254980in}{0.150000in}}{\pgfqpoint{5.490039in}{5.490039in}}%
\pgfusepath{clip}%
\pgfsetbuttcap%
\pgfsetroundjoin%
\definecolor{currentfill}{rgb}{0.278791,0.062145,0.386592}%
\pgfsetfillcolor{currentfill}%
\pgfsetfillopacity{0.700000}%
\pgfsetlinewidth{0.000000pt}%
\definecolor{currentstroke}{rgb}{0.000000,0.000000,0.000000}%
\pgfsetstrokecolor{currentstroke}%
\pgfsetdash{}{0pt}%
\pgfpathmoveto{\pgfqpoint{3.817059in}{1.204629in}}%
\pgfpathlineto{\pgfqpoint{3.830779in}{1.204278in}}%
\pgfpathlineto{\pgfqpoint{3.844507in}{1.204081in}}%
\pgfpathlineto{\pgfqpoint{3.858244in}{1.204037in}}%
\pgfpathlineto{\pgfqpoint{3.871988in}{1.204148in}}%
\pgfpathlineto{\pgfqpoint{3.880029in}{1.215098in}}%
\pgfpathlineto{\pgfqpoint{3.888064in}{1.226239in}}%
\pgfpathlineto{\pgfqpoint{3.896093in}{1.237565in}}%
\pgfpathlineto{\pgfqpoint{3.904116in}{1.249068in}}%
\pgfpathlineto{\pgfqpoint{3.890382in}{1.248283in}}%
\pgfpathlineto{\pgfqpoint{3.876656in}{1.247652in}}%
\pgfpathlineto{\pgfqpoint{3.862939in}{1.247176in}}%
\pgfpathlineto{\pgfqpoint{3.849230in}{1.246854in}}%
\pgfpathlineto{\pgfqpoint{3.841197in}{1.236014in}}%
\pgfpathlineto{\pgfqpoint{3.833157in}{1.225359in}}%
\pgfpathlineto{\pgfqpoint{3.825111in}{1.214895in}}%
\pgfpathlineto{\pgfqpoint{3.817059in}{1.204629in}}%
\pgfpathclose%
\pgfusepath{fill}%
\end{pgfscope}%
\begin{pgfscope}%
\pgfpathrectangle{\pgfqpoint{1.254980in}{0.150000in}}{\pgfqpoint{5.490039in}{5.490039in}}%
\pgfusepath{clip}%
\pgfsetbuttcap%
\pgfsetroundjoin%
\definecolor{currentfill}{rgb}{0.281924,0.089666,0.412415}%
\pgfsetfillcolor{currentfill}%
\pgfsetfillopacity{0.700000}%
\pgfsetlinewidth{0.000000pt}%
\definecolor{currentstroke}{rgb}{0.000000,0.000000,0.000000}%
\pgfsetstrokecolor{currentstroke}%
\pgfsetdash{}{0pt}%
\pgfpathmoveto{\pgfqpoint{3.904116in}{1.249068in}}%
\pgfpathlineto{\pgfqpoint{3.917859in}{1.250006in}}%
\pgfpathlineto{\pgfqpoint{3.931611in}{1.251099in}}%
\pgfpathlineto{\pgfqpoint{3.945372in}{1.252345in}}%
\pgfpathlineto{\pgfqpoint{3.959143in}{1.253745in}}%
\pgfpathlineto{\pgfqpoint{3.967152in}{1.266077in}}%
\pgfpathlineto{\pgfqpoint{3.975156in}{1.278567in}}%
\pgfpathlineto{\pgfqpoint{3.983154in}{1.291209in}}%
\pgfpathlineto{\pgfqpoint{3.991148in}{1.303996in}}%
\pgfpathlineto{\pgfqpoint{3.977385in}{1.301948in}}%
\pgfpathlineto{\pgfqpoint{3.963631in}{1.300054in}}%
\pgfpathlineto{\pgfqpoint{3.949886in}{1.298314in}}%
\pgfpathlineto{\pgfqpoint{3.936151in}{1.296728in}}%
\pgfpathlineto{\pgfqpoint{3.928151in}{1.284578in}}%
\pgfpathlineto{\pgfqpoint{3.920145in}{1.272581in}}%
\pgfpathlineto{\pgfqpoint{3.912133in}{1.260742in}}%
\pgfpathlineto{\pgfqpoint{3.904116in}{1.249068in}}%
\pgfpathclose%
\pgfusepath{fill}%
\end{pgfscope}%
\begin{pgfscope}%
\pgfpathrectangle{\pgfqpoint{1.254980in}{0.150000in}}{\pgfqpoint{5.490039in}{5.490039in}}%
\pgfusepath{clip}%
\pgfsetbuttcap%
\pgfsetroundjoin%
\definecolor{currentfill}{rgb}{0.276022,0.044167,0.370164}%
\pgfsetfillcolor{currentfill}%
\pgfsetfillopacity{0.700000}%
\pgfsetlinewidth{0.000000pt}%
\definecolor{currentstroke}{rgb}{0.000000,0.000000,0.000000}%
\pgfsetstrokecolor{currentstroke}%
\pgfsetdash{}{0pt}%
\pgfpathmoveto{\pgfqpoint{3.729915in}{1.171432in}}%
\pgfpathlineto{\pgfqpoint{3.743621in}{1.169761in}}%
\pgfpathlineto{\pgfqpoint{3.757333in}{1.168246in}}%
\pgfpathlineto{\pgfqpoint{3.771053in}{1.166885in}}%
\pgfpathlineto{\pgfqpoint{3.784779in}{1.165678in}}%
\pgfpathlineto{\pgfqpoint{3.792860in}{1.175085in}}%
\pgfpathlineto{\pgfqpoint{3.800933in}{1.184717in}}%
\pgfpathlineto{\pgfqpoint{3.808999in}{1.194568in}}%
\pgfpathlineto{\pgfqpoint{3.817059in}{1.204629in}}%
\pgfpathlineto{\pgfqpoint{3.803346in}{1.205135in}}%
\pgfpathlineto{\pgfqpoint{3.789641in}{1.205795in}}%
\pgfpathlineto{\pgfqpoint{3.775943in}{1.206611in}}%
\pgfpathlineto{\pgfqpoint{3.762253in}{1.207581in}}%
\pgfpathlineto{\pgfqpoint{3.754180in}{1.198209in}}%
\pgfpathlineto{\pgfqpoint{3.746100in}{1.189056in}}%
\pgfpathlineto{\pgfqpoint{3.738011in}{1.180128in}}%
\pgfpathlineto{\pgfqpoint{3.729915in}{1.171432in}}%
\pgfpathclose%
\pgfusepath{fill}%
\end{pgfscope}%
\begin{pgfscope}%
\pgfpathrectangle{\pgfqpoint{1.254980in}{0.150000in}}{\pgfqpoint{5.490039in}{5.490039in}}%
\pgfusepath{clip}%
\pgfsetbuttcap%
\pgfsetroundjoin%
\definecolor{currentfill}{rgb}{0.147607,0.511733,0.557049}%
\pgfsetfillcolor{currentfill}%
\pgfsetfillopacity{0.700000}%
\pgfsetlinewidth{0.000000pt}%
\definecolor{currentstroke}{rgb}{0.000000,0.000000,0.000000}%
\pgfsetstrokecolor{currentstroke}%
\pgfsetdash{}{0pt}%
\pgfpathmoveto{\pgfqpoint{4.704345in}{2.217050in}}%
\pgfpathlineto{\pgfqpoint{4.718487in}{2.228709in}}%
\pgfpathlineto{\pgfqpoint{4.732645in}{2.240529in}}%
\pgfpathlineto{\pgfqpoint{4.746822in}{2.252512in}}%
\pgfpathlineto{\pgfqpoint{4.761015in}{2.264656in}}%
\pgfpathlineto{\pgfqpoint{4.768834in}{2.280577in}}%
\pgfpathlineto{\pgfqpoint{4.776648in}{2.296375in}}%
\pgfpathlineto{\pgfqpoint{4.784457in}{2.312046in}}%
\pgfpathlineto{\pgfqpoint{4.792260in}{2.327588in}}%
\pgfpathlineto{\pgfqpoint{4.778058in}{2.315150in}}%
\pgfpathlineto{\pgfqpoint{4.763874in}{2.302874in}}%
\pgfpathlineto{\pgfqpoint{4.749708in}{2.290761in}}%
\pgfpathlineto{\pgfqpoint{4.735560in}{2.278809in}}%
\pgfpathlineto{\pgfqpoint{4.727764in}{2.263549in}}%
\pgfpathlineto{\pgfqpoint{4.719963in}{2.248167in}}%
\pgfpathlineto{\pgfqpoint{4.712157in}{2.232667in}}%
\pgfpathlineto{\pgfqpoint{4.704345in}{2.217050in}}%
\pgfpathclose%
\pgfusepath{fill}%
\end{pgfscope}%
\begin{pgfscope}%
\pgfpathrectangle{\pgfqpoint{1.254980in}{0.150000in}}{\pgfqpoint{5.490039in}{5.490039in}}%
\pgfusepath{clip}%
\pgfsetbuttcap%
\pgfsetroundjoin%
\definecolor{currentfill}{rgb}{0.496615,0.826376,0.306377}%
\pgfsetfillcolor{currentfill}%
\pgfsetfillopacity{0.700000}%
\pgfsetlinewidth{0.000000pt}%
\definecolor{currentstroke}{rgb}{0.000000,0.000000,0.000000}%
\pgfsetstrokecolor{currentstroke}%
\pgfsetdash{}{0pt}%
\pgfpathmoveto{\pgfqpoint{5.417707in}{3.166919in}}%
\pgfpathlineto{\pgfqpoint{5.432369in}{3.184043in}}%
\pgfpathlineto{\pgfqpoint{5.447054in}{3.201336in}}%
\pgfpathlineto{\pgfqpoint{5.461762in}{3.218799in}}%
\pgfpathlineto{\pgfqpoint{5.476493in}{3.236432in}}%
\pgfpathlineto{\pgfqpoint{5.483941in}{3.244420in}}%
\pgfpathlineto{\pgfqpoint{5.491377in}{3.252201in}}%
\pgfpathlineto{\pgfqpoint{5.498802in}{3.259776in}}%
\pgfpathlineto{\pgfqpoint{5.506215in}{3.267145in}}%
\pgfpathlineto{\pgfqpoint{5.491490in}{3.249597in}}%
\pgfpathlineto{\pgfqpoint{5.476787in}{3.232218in}}%
\pgfpathlineto{\pgfqpoint{5.462107in}{3.215008in}}%
\pgfpathlineto{\pgfqpoint{5.447451in}{3.197968in}}%
\pgfpathlineto{\pgfqpoint{5.440031in}{3.190502in}}%
\pgfpathlineto{\pgfqpoint{5.432600in}{3.182840in}}%
\pgfpathlineto{\pgfqpoint{5.425159in}{3.174979in}}%
\pgfpathlineto{\pgfqpoint{5.417707in}{3.166919in}}%
\pgfpathclose%
\pgfusepath{fill}%
\end{pgfscope}%
\begin{pgfscope}%
\pgfpathrectangle{\pgfqpoint{1.254980in}{0.150000in}}{\pgfqpoint{5.490039in}{5.490039in}}%
\pgfusepath{clip}%
\pgfsetbuttcap%
\pgfsetroundjoin%
\definecolor{currentfill}{rgb}{0.283187,0.125848,0.444960}%
\pgfsetfillcolor{currentfill}%
\pgfsetfillopacity{0.700000}%
\pgfsetlinewidth{0.000000pt}%
\definecolor{currentstroke}{rgb}{0.000000,0.000000,0.000000}%
\pgfsetstrokecolor{currentstroke}%
\pgfsetdash{}{0pt}%
\pgfpathmoveto{\pgfqpoint{3.991148in}{1.303996in}}%
\pgfpathlineto{\pgfqpoint{4.004921in}{1.306198in}}%
\pgfpathlineto{\pgfqpoint{4.018705in}{1.308554in}}%
\pgfpathlineto{\pgfqpoint{4.032498in}{1.311064in}}%
\pgfpathlineto{\pgfqpoint{4.046301in}{1.313727in}}%
\pgfpathlineto{\pgfqpoint{4.054285in}{1.327285in}}%
\pgfpathlineto{\pgfqpoint{4.062265in}{1.340971in}}%
\pgfpathlineto{\pgfqpoint{4.070239in}{1.354776in}}%
\pgfpathlineto{\pgfqpoint{4.078210in}{1.368696in}}%
\pgfpathlineto{\pgfqpoint{4.064410in}{1.365410in}}%
\pgfpathlineto{\pgfqpoint{4.050621in}{1.362278in}}%
\pgfpathlineto{\pgfqpoint{4.036842in}{1.359301in}}%
\pgfpathlineto{\pgfqpoint{4.023074in}{1.356478in}}%
\pgfpathlineto{\pgfqpoint{4.015100in}{1.343169in}}%
\pgfpathlineto{\pgfqpoint{4.007121in}{1.329983in}}%
\pgfpathlineto{\pgfqpoint{3.999137in}{1.316923in}}%
\pgfpathlineto{\pgfqpoint{3.991148in}{1.303996in}}%
\pgfpathclose%
\pgfusepath{fill}%
\end{pgfscope}%
\begin{pgfscope}%
\pgfpathrectangle{\pgfqpoint{1.254980in}{0.150000in}}{\pgfqpoint{5.490039in}{5.490039in}}%
\pgfusepath{clip}%
\pgfsetbuttcap%
\pgfsetroundjoin%
\definecolor{currentfill}{rgb}{0.243113,0.292092,0.538516}%
\pgfsetfillcolor{currentfill}%
\pgfsetfillopacity{0.700000}%
\pgfsetlinewidth{0.000000pt}%
\definecolor{currentstroke}{rgb}{0.000000,0.000000,0.000000}%
\pgfsetstrokecolor{currentstroke}%
\pgfsetdash{}{0pt}%
\pgfpathmoveto{\pgfqpoint{4.316001in}{1.652056in}}%
\pgfpathlineto{\pgfqpoint{4.329912in}{1.658962in}}%
\pgfpathlineto{\pgfqpoint{4.343838in}{1.666024in}}%
\pgfpathlineto{\pgfqpoint{4.357776in}{1.673243in}}%
\pgfpathlineto{\pgfqpoint{4.371729in}{1.680618in}}%
\pgfpathlineto{\pgfqpoint{4.379640in}{1.697144in}}%
\pgfpathlineto{\pgfqpoint{4.387548in}{1.713669in}}%
\pgfpathlineto{\pgfqpoint{4.395453in}{1.730189in}}%
\pgfpathlineto{\pgfqpoint{4.403354in}{1.746699in}}%
\pgfpathlineto{\pgfqpoint{4.389396in}{1.738833in}}%
\pgfpathlineto{\pgfqpoint{4.375453in}{1.731124in}}%
\pgfpathlineto{\pgfqpoint{4.361523in}{1.723571in}}%
\pgfpathlineto{\pgfqpoint{4.347607in}{1.716175in}}%
\pgfpathlineto{\pgfqpoint{4.339711in}{1.700145in}}%
\pgfpathlineto{\pgfqpoint{4.331811in}{1.684112in}}%
\pgfpathlineto{\pgfqpoint{4.323908in}{1.668081in}}%
\pgfpathlineto{\pgfqpoint{4.316001in}{1.652056in}}%
\pgfpathclose%
\pgfusepath{fill}%
\end{pgfscope}%
\begin{pgfscope}%
\pgfpathrectangle{\pgfqpoint{1.254980in}{0.150000in}}{\pgfqpoint{5.490039in}{5.490039in}}%
\pgfusepath{clip}%
\pgfsetbuttcap%
\pgfsetroundjoin%
\definecolor{currentfill}{rgb}{0.267968,0.223549,0.512008}%
\pgfsetfillcolor{currentfill}%
\pgfsetfillopacity{0.700000}%
\pgfsetlinewidth{0.000000pt}%
\definecolor{currentstroke}{rgb}{0.000000,0.000000,0.000000}%
\pgfsetstrokecolor{currentstroke}%
\pgfsetdash{}{0pt}%
\pgfpathmoveto{\pgfqpoint{4.197123in}{1.502832in}}%
\pgfpathlineto{\pgfqpoint{4.210978in}{1.508056in}}%
\pgfpathlineto{\pgfqpoint{4.224846in}{1.513434in}}%
\pgfpathlineto{\pgfqpoint{4.238725in}{1.518968in}}%
\pgfpathlineto{\pgfqpoint{4.252618in}{1.524656in}}%
\pgfpathlineto{\pgfqpoint{4.260553in}{1.540458in}}%
\pgfpathlineto{\pgfqpoint{4.268485in}{1.556304in}}%
\pgfpathlineto{\pgfqpoint{4.276413in}{1.572191in}}%
\pgfpathlineto{\pgfqpoint{4.284338in}{1.588114in}}%
\pgfpathlineto{\pgfqpoint{4.270443in}{1.581880in}}%
\pgfpathlineto{\pgfqpoint{4.256561in}{1.575803in}}%
\pgfpathlineto{\pgfqpoint{4.242691in}{1.569881in}}%
\pgfpathlineto{\pgfqpoint{4.228835in}{1.564114in}}%
\pgfpathlineto{\pgfqpoint{4.220912in}{1.548725in}}%
\pgfpathlineto{\pgfqpoint{4.212987in}{1.533379in}}%
\pgfpathlineto{\pgfqpoint{4.205057in}{1.518079in}}%
\pgfpathlineto{\pgfqpoint{4.197123in}{1.502832in}}%
\pgfpathclose%
\pgfusepath{fill}%
\end{pgfscope}%
\begin{pgfscope}%
\pgfpathrectangle{\pgfqpoint{1.254980in}{0.150000in}}{\pgfqpoint{5.490039in}{5.490039in}}%
\pgfusepath{clip}%
\pgfsetbuttcap%
\pgfsetroundjoin%
\definecolor{currentfill}{rgb}{0.124395,0.578002,0.548287}%
\pgfsetfillcolor{currentfill}%
\pgfsetfillopacity{0.700000}%
\pgfsetlinewidth{0.000000pt}%
\definecolor{currentstroke}{rgb}{0.000000,0.000000,0.000000}%
\pgfsetstrokecolor{currentstroke}%
\pgfsetdash{}{0pt}%
\pgfpathmoveto{\pgfqpoint{4.823416in}{2.388419in}}%
\pgfpathlineto{\pgfqpoint{4.837643in}{2.401283in}}%
\pgfpathlineto{\pgfqpoint{4.851888in}{2.414311in}}%
\pgfpathlineto{\pgfqpoint{4.866152in}{2.427502in}}%
\pgfpathlineto{\pgfqpoint{4.880435in}{2.440856in}}%
\pgfpathlineto{\pgfqpoint{4.888217in}{2.455963in}}%
\pgfpathlineto{\pgfqpoint{4.895992in}{2.470921in}}%
\pgfpathlineto{\pgfqpoint{4.903762in}{2.485726in}}%
\pgfpathlineto{\pgfqpoint{4.911525in}{2.500377in}}%
\pgfpathlineto{\pgfqpoint{4.897235in}{2.486789in}}%
\pgfpathlineto{\pgfqpoint{4.882964in}{2.473364in}}%
\pgfpathlineto{\pgfqpoint{4.868712in}{2.460103in}}%
\pgfpathlineto{\pgfqpoint{4.854479in}{2.447005in}}%
\pgfpathlineto{\pgfqpoint{4.846722in}{2.432576in}}%
\pgfpathlineto{\pgfqpoint{4.838960in}{2.418001in}}%
\pgfpathlineto{\pgfqpoint{4.831191in}{2.403281in}}%
\pgfpathlineto{\pgfqpoint{4.823416in}{2.388419in}}%
\pgfpathclose%
\pgfusepath{fill}%
\end{pgfscope}%
\begin{pgfscope}%
\pgfpathrectangle{\pgfqpoint{1.254980in}{0.150000in}}{\pgfqpoint{5.490039in}{5.490039in}}%
\pgfusepath{clip}%
\pgfsetbuttcap%
\pgfsetroundjoin%
\definecolor{currentfill}{rgb}{0.210503,0.363727,0.552206}%
\pgfsetfillcolor{currentfill}%
\pgfsetfillopacity{0.700000}%
\pgfsetlinewidth{0.000000pt}%
\definecolor{currentstroke}{rgb}{0.000000,0.000000,0.000000}%
\pgfsetstrokecolor{currentstroke}%
\pgfsetdash{}{0pt}%
\pgfpathmoveto{\pgfqpoint{4.434923in}{1.812555in}}%
\pgfpathlineto{\pgfqpoint{4.448901in}{1.821042in}}%
\pgfpathlineto{\pgfqpoint{4.462893in}{1.829686in}}%
\pgfpathlineto{\pgfqpoint{4.476901in}{1.838489in}}%
\pgfpathlineto{\pgfqpoint{4.490924in}{1.847449in}}%
\pgfpathlineto{\pgfqpoint{4.498813in}{1.864292in}}%
\pgfpathlineto{\pgfqpoint{4.506699in}{1.881093in}}%
\pgfpathlineto{\pgfqpoint{4.514581in}{1.897847in}}%
\pgfpathlineto{\pgfqpoint{4.522460in}{1.914551in}}%
\pgfpathlineto{\pgfqpoint{4.508430in}{1.905154in}}%
\pgfpathlineto{\pgfqpoint{4.494416in}{1.895915in}}%
\pgfpathlineto{\pgfqpoint{4.480416in}{1.886835in}}%
\pgfpathlineto{\pgfqpoint{4.466432in}{1.877913in}}%
\pgfpathlineto{\pgfqpoint{4.458560in}{1.861634in}}%
\pgfpathlineto{\pgfqpoint{4.450685in}{1.845312in}}%
\pgfpathlineto{\pgfqpoint{4.442806in}{1.828951in}}%
\pgfpathlineto{\pgfqpoint{4.434923in}{1.812555in}}%
\pgfpathclose%
\pgfusepath{fill}%
\end{pgfscope}%
\begin{pgfscope}%
\pgfpathrectangle{\pgfqpoint{1.254980in}{0.150000in}}{\pgfqpoint{5.490039in}{5.490039in}}%
\pgfusepath{clip}%
\pgfsetbuttcap%
\pgfsetroundjoin%
\definecolor{currentfill}{rgb}{0.377779,0.791781,0.377939}%
\pgfsetfillcolor{currentfill}%
\pgfsetfillopacity{0.700000}%
\pgfsetlinewidth{0.000000pt}%
\definecolor{currentstroke}{rgb}{0.000000,0.000000,0.000000}%
\pgfsetstrokecolor{currentstroke}%
\pgfsetdash{}{0pt}%
\pgfpathmoveto{\pgfqpoint{5.299261in}{3.028047in}}%
\pgfpathlineto{\pgfqpoint{5.313840in}{3.044565in}}%
\pgfpathlineto{\pgfqpoint{5.328441in}{3.061252in}}%
\pgfpathlineto{\pgfqpoint{5.343064in}{3.078108in}}%
\pgfpathlineto{\pgfqpoint{5.357710in}{3.095133in}}%
\pgfpathlineto{\pgfqpoint{5.365246in}{3.104824in}}%
\pgfpathlineto{\pgfqpoint{5.372772in}{3.114308in}}%
\pgfpathlineto{\pgfqpoint{5.380287in}{3.123587in}}%
\pgfpathlineto{\pgfqpoint{5.387792in}{3.132660in}}%
\pgfpathlineto{\pgfqpoint{5.373148in}{3.115654in}}%
\pgfpathlineto{\pgfqpoint{5.358527in}{3.098817in}}%
\pgfpathlineto{\pgfqpoint{5.343928in}{3.082148in}}%
\pgfpathlineto{\pgfqpoint{5.329351in}{3.065647in}}%
\pgfpathlineto{\pgfqpoint{5.321843in}{3.056543in}}%
\pgfpathlineto{\pgfqpoint{5.314326in}{3.047242in}}%
\pgfpathlineto{\pgfqpoint{5.306798in}{3.037743in}}%
\pgfpathlineto{\pgfqpoint{5.299261in}{3.028047in}}%
\pgfpathclose%
\pgfusepath{fill}%
\end{pgfscope}%
\begin{pgfscope}%
\pgfpathrectangle{\pgfqpoint{1.254980in}{0.150000in}}{\pgfqpoint{5.490039in}{5.490039in}}%
\pgfusepath{clip}%
\pgfsetbuttcap%
\pgfsetroundjoin%
\definecolor{currentfill}{rgb}{0.124780,0.640461,0.527068}%
\pgfsetfillcolor{currentfill}%
\pgfsetfillopacity{0.700000}%
\pgfsetlinewidth{0.000000pt}%
\definecolor{currentstroke}{rgb}{0.000000,0.000000,0.000000}%
\pgfsetstrokecolor{currentstroke}%
\pgfsetdash{}{0pt}%
\pgfpathmoveto{\pgfqpoint{4.942512in}{2.557408in}}%
\pgfpathlineto{\pgfqpoint{4.956827in}{2.571364in}}%
\pgfpathlineto{\pgfqpoint{4.971161in}{2.585485in}}%
\pgfpathlineto{\pgfqpoint{4.985515in}{2.599771in}}%
\pgfpathlineto{\pgfqpoint{4.999889in}{2.614222in}}%
\pgfpathlineto{\pgfqpoint{5.007625in}{2.628263in}}%
\pgfpathlineto{\pgfqpoint{5.015354in}{2.642131in}}%
\pgfpathlineto{\pgfqpoint{5.023075in}{2.655827in}}%
\pgfpathlineto{\pgfqpoint{5.030790in}{2.669349in}}%
\pgfpathlineto{\pgfqpoint{5.016410in}{2.654725in}}%
\pgfpathlineto{\pgfqpoint{5.002051in}{2.640266in}}%
\pgfpathlineto{\pgfqpoint{4.987711in}{2.625972in}}%
\pgfpathlineto{\pgfqpoint{4.973391in}{2.611843in}}%
\pgfpathlineto{\pgfqpoint{4.965682in}{2.598482in}}%
\pgfpathlineto{\pgfqpoint{4.957966in}{2.584955in}}%
\pgfpathlineto{\pgfqpoint{4.950242in}{2.571263in}}%
\pgfpathlineto{\pgfqpoint{4.942512in}{2.557408in}}%
\pgfpathclose%
\pgfusepath{fill}%
\end{pgfscope}%
\begin{pgfscope}%
\pgfpathrectangle{\pgfqpoint{1.254980in}{0.150000in}}{\pgfqpoint{5.490039in}{5.490039in}}%
\pgfusepath{clip}%
\pgfsetbuttcap%
\pgfsetroundjoin%
\definecolor{currentfill}{rgb}{0.280868,0.160771,0.472899}%
\pgfsetfillcolor{currentfill}%
\pgfsetfillopacity{0.700000}%
\pgfsetlinewidth{0.000000pt}%
\definecolor{currentstroke}{rgb}{0.000000,0.000000,0.000000}%
\pgfsetstrokecolor{currentstroke}%
\pgfsetdash{}{0pt}%
\pgfpathmoveto{\pgfqpoint{4.078210in}{1.368696in}}%
\pgfpathlineto{\pgfqpoint{4.092020in}{1.372135in}}%
\pgfpathlineto{\pgfqpoint{4.105842in}{1.375729in}}%
\pgfpathlineto{\pgfqpoint{4.119674in}{1.379477in}}%
\pgfpathlineto{\pgfqpoint{4.133518in}{1.383378in}}%
\pgfpathlineto{\pgfqpoint{4.141482in}{1.398013in}}%
\pgfpathlineto{\pgfqpoint{4.149442in}{1.412745in}}%
\pgfpathlineto{\pgfqpoint{4.157399in}{1.427566in}}%
\pgfpathlineto{\pgfqpoint{4.165352in}{1.442472in}}%
\pgfpathlineto{\pgfqpoint{4.151509in}{1.437973in}}%
\pgfpathlineto{\pgfqpoint{4.137678in}{1.433629in}}%
\pgfpathlineto{\pgfqpoint{4.123858in}{1.429439in}}%
\pgfpathlineto{\pgfqpoint{4.110049in}{1.425403in}}%
\pgfpathlineto{\pgfqpoint{4.102096in}{1.411084in}}%
\pgfpathlineto{\pgfqpoint{4.094138in}{1.396855in}}%
\pgfpathlineto{\pgfqpoint{4.086176in}{1.382724in}}%
\pgfpathlineto{\pgfqpoint{4.078210in}{1.368696in}}%
\pgfpathclose%
\pgfusepath{fill}%
\end{pgfscope}%
\begin{pgfscope}%
\pgfpathrectangle{\pgfqpoint{1.254980in}{0.150000in}}{\pgfqpoint{5.490039in}{5.490039in}}%
\pgfusepath{clip}%
\pgfsetbuttcap%
\pgfsetroundjoin%
\definecolor{currentfill}{rgb}{0.180629,0.429975,0.557282}%
\pgfsetfillcolor{currentfill}%
\pgfsetfillopacity{0.700000}%
\pgfsetlinewidth{0.000000pt}%
\definecolor{currentstroke}{rgb}{0.000000,0.000000,0.000000}%
\pgfsetstrokecolor{currentstroke}%
\pgfsetdash{}{0pt}%
\pgfpathmoveto{\pgfqpoint{4.553934in}{1.980787in}}%
\pgfpathlineto{\pgfqpoint{4.567987in}{1.990751in}}%
\pgfpathlineto{\pgfqpoint{4.582056in}{2.000875in}}%
\pgfpathlineto{\pgfqpoint{4.596140in}{2.011158in}}%
\pgfpathlineto{\pgfqpoint{4.610241in}{2.021601in}}%
\pgfpathlineto{\pgfqpoint{4.618108in}{2.038388in}}%
\pgfpathlineto{\pgfqpoint{4.625970in}{2.055095in}}%
\pgfpathlineto{\pgfqpoint{4.633828in}{2.071718in}}%
\pgfpathlineto{\pgfqpoint{4.641681in}{2.088255in}}%
\pgfpathlineto{\pgfqpoint{4.627572in}{2.077431in}}%
\pgfpathlineto{\pgfqpoint{4.613480in}{2.066767in}}%
\pgfpathlineto{\pgfqpoint{4.599403in}{2.056264in}}%
\pgfpathlineto{\pgfqpoint{4.585343in}{2.045919in}}%
\pgfpathlineto{\pgfqpoint{4.577497in}{2.029752in}}%
\pgfpathlineto{\pgfqpoint{4.569647in}{2.013505in}}%
\pgfpathlineto{\pgfqpoint{4.561793in}{1.997182in}}%
\pgfpathlineto{\pgfqpoint{4.553934in}{1.980787in}}%
\pgfpathclose%
\pgfusepath{fill}%
\end{pgfscope}%
\begin{pgfscope}%
\pgfpathrectangle{\pgfqpoint{1.254980in}{0.150000in}}{\pgfqpoint{5.490039in}{5.490039in}}%
\pgfusepath{clip}%
\pgfsetbuttcap%
\pgfsetroundjoin%
\definecolor{currentfill}{rgb}{0.266941,0.748751,0.440573}%
\pgfsetfillcolor{currentfill}%
\pgfsetfillopacity{0.700000}%
\pgfsetlinewidth{0.000000pt}%
\definecolor{currentstroke}{rgb}{0.000000,0.000000,0.000000}%
\pgfsetstrokecolor{currentstroke}%
\pgfsetdash{}{0pt}%
\pgfpathmoveto{\pgfqpoint{5.180517in}{2.879141in}}%
\pgfpathlineto{\pgfqpoint{5.195010in}{2.894928in}}%
\pgfpathlineto{\pgfqpoint{5.209523in}{2.910882in}}%
\pgfpathlineto{\pgfqpoint{5.224058in}{2.927004in}}%
\pgfpathlineto{\pgfqpoint{5.238615in}{2.943293in}}%
\pgfpathlineto{\pgfqpoint{5.246229in}{2.954587in}}%
\pgfpathlineto{\pgfqpoint{5.253833in}{2.965680in}}%
\pgfpathlineto{\pgfqpoint{5.261428in}{2.976574in}}%
\pgfpathlineto{\pgfqpoint{5.269014in}{2.987268in}}%
\pgfpathlineto{\pgfqpoint{5.254456in}{2.970932in}}%
\pgfpathlineto{\pgfqpoint{5.239920in}{2.954763in}}%
\pgfpathlineto{\pgfqpoint{5.225405in}{2.938762in}}%
\pgfpathlineto{\pgfqpoint{5.210912in}{2.922929in}}%
\pgfpathlineto{\pgfqpoint{5.203327in}{2.912269in}}%
\pgfpathlineto{\pgfqpoint{5.195733in}{2.901418in}}%
\pgfpathlineto{\pgfqpoint{5.188129in}{2.890375in}}%
\pgfpathlineto{\pgfqpoint{5.180517in}{2.879141in}}%
\pgfpathclose%
\pgfusepath{fill}%
\end{pgfscope}%
\begin{pgfscope}%
\pgfpathrectangle{\pgfqpoint{1.254980in}{0.150000in}}{\pgfqpoint{5.490039in}{5.490039in}}%
\pgfusepath{clip}%
\pgfsetbuttcap%
\pgfsetroundjoin%
\definecolor{currentfill}{rgb}{0.175707,0.697900,0.491033}%
\pgfsetfillcolor{currentfill}%
\pgfsetfillopacity{0.700000}%
\pgfsetlinewidth{0.000000pt}%
\definecolor{currentstroke}{rgb}{0.000000,0.000000,0.000000}%
\pgfsetstrokecolor{currentstroke}%
\pgfsetdash{}{0pt}%
\pgfpathmoveto{\pgfqpoint{5.061572in}{2.721670in}}%
\pgfpathlineto{\pgfqpoint{5.075976in}{2.736602in}}%
\pgfpathlineto{\pgfqpoint{5.090400in}{2.751700in}}%
\pgfpathlineto{\pgfqpoint{5.104845in}{2.766964in}}%
\pgfpathlineto{\pgfqpoint{5.119310in}{2.782394in}}%
\pgfpathlineto{\pgfqpoint{5.126990in}{2.795151in}}%
\pgfpathlineto{\pgfqpoint{5.134662in}{2.807719in}}%
\pgfpathlineto{\pgfqpoint{5.142326in}{2.820098in}}%
\pgfpathlineto{\pgfqpoint{5.149981in}{2.832288in}}%
\pgfpathlineto{\pgfqpoint{5.135512in}{2.816747in}}%
\pgfpathlineto{\pgfqpoint{5.121064in}{2.801372in}}%
\pgfpathlineto{\pgfqpoint{5.106636in}{2.786164in}}%
\pgfpathlineto{\pgfqpoint{5.092229in}{2.771122in}}%
\pgfpathlineto{\pgfqpoint{5.084576in}{2.759031in}}%
\pgfpathlineto{\pgfqpoint{5.076916in}{2.746758in}}%
\pgfpathlineto{\pgfqpoint{5.069248in}{2.734304in}}%
\pgfpathlineto{\pgfqpoint{5.061572in}{2.721670in}}%
\pgfpathclose%
\pgfusepath{fill}%
\end{pgfscope}%
\begin{pgfscope}%
\pgfpathrectangle{\pgfqpoint{1.254980in}{0.150000in}}{\pgfqpoint{5.490039in}{5.490039in}}%
\pgfusepath{clip}%
\pgfsetbuttcap%
\pgfsetroundjoin%
\definecolor{currentfill}{rgb}{0.595839,0.848717,0.243329}%
\pgfsetfillcolor{currentfill}%
\pgfsetfillopacity{0.700000}%
\pgfsetlinewidth{0.000000pt}%
\definecolor{currentstroke}{rgb}{0.000000,0.000000,0.000000}%
\pgfsetstrokecolor{currentstroke}%
\pgfsetdash{}{0pt}%
\pgfpathmoveto{\pgfqpoint{5.506215in}{3.267145in}}%
\pgfpathlineto{\pgfqpoint{5.520964in}{3.284863in}}%
\pgfpathlineto{\pgfqpoint{5.535736in}{3.302751in}}%
\pgfpathlineto{\pgfqpoint{5.550532in}{3.320810in}}%
\pgfpathlineto{\pgfqpoint{5.557929in}{3.327897in}}%
\pgfpathlineto{\pgfqpoint{5.565314in}{3.334775in}}%
\pgfpathlineto{\pgfqpoint{5.572688in}{3.341446in}}%
\pgfpathlineto{\pgfqpoint{5.580050in}{3.347912in}}%
\pgfpathlineto{\pgfqpoint{5.565262in}{3.329972in}}%
\pgfpathlineto{\pgfqpoint{5.550497in}{3.312201in}}%
\pgfpathlineto{\pgfqpoint{5.535756in}{3.294600in}}%
\pgfpathlineto{\pgfqpoint{5.528388in}{3.288037in}}%
\pgfpathlineto{\pgfqpoint{5.521008in}{3.281274in}}%
\pgfpathlineto{\pgfqpoint{5.513617in}{3.274311in}}%
\pgfpathlineto{\pgfqpoint{5.506215in}{3.267145in}}%
\pgfpathclose%
\pgfusepath{fill}%
\end{pgfscope}%
\begin{pgfscope}%
\pgfpathrectangle{\pgfqpoint{1.254980in}{0.150000in}}{\pgfqpoint{5.490039in}{5.490039in}}%
\pgfusepath{clip}%
\pgfsetbuttcap%
\pgfsetroundjoin%
\definecolor{currentfill}{rgb}{0.153364,0.497000,0.557724}%
\pgfsetfillcolor{currentfill}%
\pgfsetfillopacity{0.700000}%
\pgfsetlinewidth{0.000000pt}%
\definecolor{currentstroke}{rgb}{0.000000,0.000000,0.000000}%
\pgfsetstrokecolor{currentstroke}%
\pgfsetdash{}{0pt}%
\pgfpathmoveto{\pgfqpoint{4.673051in}{2.153472in}}%
\pgfpathlineto{\pgfqpoint{4.687184in}{2.164808in}}%
\pgfpathlineto{\pgfqpoint{4.701335in}{2.176306in}}%
\pgfpathlineto{\pgfqpoint{4.715504in}{2.187964in}}%
\pgfpathlineto{\pgfqpoint{4.729689in}{2.199784in}}%
\pgfpathlineto{\pgfqpoint{4.737528in}{2.216174in}}%
\pgfpathlineto{\pgfqpoint{4.745362in}{2.232452in}}%
\pgfpathlineto{\pgfqpoint{4.753191in}{2.248613in}}%
\pgfpathlineto{\pgfqpoint{4.761015in}{2.264656in}}%
\pgfpathlineto{\pgfqpoint{4.746822in}{2.252512in}}%
\pgfpathlineto{\pgfqpoint{4.732645in}{2.240529in}}%
\pgfpathlineto{\pgfqpoint{4.718487in}{2.228709in}}%
\pgfpathlineto{\pgfqpoint{4.704345in}{2.217050in}}%
\pgfpathlineto{\pgfqpoint{4.696529in}{2.201319in}}%
\pgfpathlineto{\pgfqpoint{4.688708in}{2.185477in}}%
\pgfpathlineto{\pgfqpoint{4.680882in}{2.169527in}}%
\pgfpathlineto{\pgfqpoint{4.673051in}{2.153472in}}%
\pgfpathclose%
\pgfusepath{fill}%
\end{pgfscope}%
\begin{pgfscope}%
\pgfpathrectangle{\pgfqpoint{1.254980in}{0.150000in}}{\pgfqpoint{5.490039in}{5.490039in}}%
\pgfusepath{clip}%
\pgfsetbuttcap%
\pgfsetroundjoin%
\definecolor{currentfill}{rgb}{0.250425,0.274290,0.533103}%
\pgfsetfillcolor{currentfill}%
\pgfsetfillopacity{0.700000}%
\pgfsetlinewidth{0.000000pt}%
\definecolor{currentstroke}{rgb}{0.000000,0.000000,0.000000}%
\pgfsetstrokecolor{currentstroke}%
\pgfsetdash{}{0pt}%
\pgfpathmoveto{\pgfqpoint{4.284338in}{1.588114in}}%
\pgfpathlineto{\pgfqpoint{4.298245in}{1.594502in}}%
\pgfpathlineto{\pgfqpoint{4.312166in}{1.601046in}}%
\pgfpathlineto{\pgfqpoint{4.326101in}{1.607746in}}%
\pgfpathlineto{\pgfqpoint{4.340048in}{1.614602in}}%
\pgfpathlineto{\pgfqpoint{4.347974in}{1.631083in}}%
\pgfpathlineto{\pgfqpoint{4.355895in}{1.647583in}}%
\pgfpathlineto{\pgfqpoint{4.363814in}{1.664096in}}%
\pgfpathlineto{\pgfqpoint{4.371729in}{1.680618in}}%
\pgfpathlineto{\pgfqpoint{4.357776in}{1.673243in}}%
\pgfpathlineto{\pgfqpoint{4.343838in}{1.666024in}}%
\pgfpathlineto{\pgfqpoint{4.329912in}{1.658962in}}%
\pgfpathlineto{\pgfqpoint{4.316001in}{1.652056in}}%
\pgfpathlineto{\pgfqpoint{4.308090in}{1.636042in}}%
\pgfpathlineto{\pgfqpoint{4.300176in}{1.620044in}}%
\pgfpathlineto{\pgfqpoint{4.292259in}{1.604066in}}%
\pgfpathlineto{\pgfqpoint{4.284338in}{1.588114in}}%
\pgfpathclose%
\pgfusepath{fill}%
\end{pgfscope}%
\begin{pgfscope}%
\pgfpathrectangle{\pgfqpoint{1.254980in}{0.150000in}}{\pgfqpoint{5.490039in}{5.490039in}}%
\pgfusepath{clip}%
\pgfsetbuttcap%
\pgfsetroundjoin%
\definecolor{currentfill}{rgb}{0.280894,0.078907,0.402329}%
\pgfsetfillcolor{currentfill}%
\pgfsetfillopacity{0.700000}%
\pgfsetlinewidth{0.000000pt}%
\definecolor{currentstroke}{rgb}{0.000000,0.000000,0.000000}%
\pgfsetstrokecolor{currentstroke}%
\pgfsetdash{}{0pt}%
\pgfpathmoveto{\pgfqpoint{3.871988in}{1.204148in}}%
\pgfpathlineto{\pgfqpoint{3.885741in}{1.204412in}}%
\pgfpathlineto{\pgfqpoint{3.899503in}{1.204829in}}%
\pgfpathlineto{\pgfqpoint{3.913273in}{1.205400in}}%
\pgfpathlineto{\pgfqpoint{3.927051in}{1.206124in}}%
\pgfpathlineto{\pgfqpoint{3.935082in}{1.217760in}}%
\pgfpathlineto{\pgfqpoint{3.943108in}{1.229580in}}%
\pgfpathlineto{\pgfqpoint{3.951128in}{1.241577in}}%
\pgfpathlineto{\pgfqpoint{3.959143in}{1.253745in}}%
\pgfpathlineto{\pgfqpoint{3.945372in}{1.252345in}}%
\pgfpathlineto{\pgfqpoint{3.931611in}{1.251099in}}%
\pgfpathlineto{\pgfqpoint{3.917859in}{1.250006in}}%
\pgfpathlineto{\pgfqpoint{3.904116in}{1.249068in}}%
\pgfpathlineto{\pgfqpoint{3.896093in}{1.237565in}}%
\pgfpathlineto{\pgfqpoint{3.888064in}{1.226239in}}%
\pgfpathlineto{\pgfqpoint{3.880029in}{1.215098in}}%
\pgfpathlineto{\pgfqpoint{3.871988in}{1.204148in}}%
\pgfpathclose%
\pgfusepath{fill}%
\end{pgfscope}%
\begin{pgfscope}%
\pgfpathrectangle{\pgfqpoint{1.254980in}{0.150000in}}{\pgfqpoint{5.490039in}{5.490039in}}%
\pgfusepath{clip}%
\pgfsetbuttcap%
\pgfsetroundjoin%
\definecolor{currentfill}{rgb}{0.277941,0.056324,0.381191}%
\pgfsetfillcolor{currentfill}%
\pgfsetfillopacity{0.700000}%
\pgfsetlinewidth{0.000000pt}%
\definecolor{currentstroke}{rgb}{0.000000,0.000000,0.000000}%
\pgfsetstrokecolor{currentstroke}%
\pgfsetdash{}{0pt}%
\pgfpathmoveto{\pgfqpoint{3.784779in}{1.165678in}}%
\pgfpathlineto{\pgfqpoint{3.798513in}{1.164626in}}%
\pgfpathlineto{\pgfqpoint{3.812255in}{1.163727in}}%
\pgfpathlineto{\pgfqpoint{3.826004in}{1.162982in}}%
\pgfpathlineto{\pgfqpoint{3.839760in}{1.162390in}}%
\pgfpathlineto{\pgfqpoint{3.847827in}{1.172509in}}%
\pgfpathlineto{\pgfqpoint{3.855887in}{1.182847in}}%
\pgfpathlineto{\pgfqpoint{3.863941in}{1.193395in}}%
\pgfpathlineto{\pgfqpoint{3.871988in}{1.204148in}}%
\pgfpathlineto{\pgfqpoint{3.858244in}{1.204037in}}%
\pgfpathlineto{\pgfqpoint{3.844507in}{1.204081in}}%
\pgfpathlineto{\pgfqpoint{3.830779in}{1.204278in}}%
\pgfpathlineto{\pgfqpoint{3.817059in}{1.204629in}}%
\pgfpathlineto{\pgfqpoint{3.808999in}{1.194568in}}%
\pgfpathlineto{\pgfqpoint{3.800933in}{1.184717in}}%
\pgfpathlineto{\pgfqpoint{3.792860in}{1.175085in}}%
\pgfpathlineto{\pgfqpoint{3.784779in}{1.165678in}}%
\pgfpathclose%
\pgfusepath{fill}%
\end{pgfscope}%
\begin{pgfscope}%
\pgfpathrectangle{\pgfqpoint{1.254980in}{0.150000in}}{\pgfqpoint{5.490039in}{5.490039in}}%
\pgfusepath{clip}%
\pgfsetbuttcap%
\pgfsetroundjoin%
\definecolor{currentfill}{rgb}{0.220057,0.343307,0.549413}%
\pgfsetfillcolor{currentfill}%
\pgfsetfillopacity{0.700000}%
\pgfsetlinewidth{0.000000pt}%
\definecolor{currentstroke}{rgb}{0.000000,0.000000,0.000000}%
\pgfsetstrokecolor{currentstroke}%
\pgfsetdash{}{0pt}%
\pgfpathmoveto{\pgfqpoint{4.403354in}{1.746699in}}%
\pgfpathlineto{\pgfqpoint{4.417326in}{1.754722in}}%
\pgfpathlineto{\pgfqpoint{4.431313in}{1.762902in}}%
\pgfpathlineto{\pgfqpoint{4.445314in}{1.771240in}}%
\pgfpathlineto{\pgfqpoint{4.459330in}{1.779734in}}%
\pgfpathlineto{\pgfqpoint{4.467234in}{1.796705in}}%
\pgfpathlineto{\pgfqpoint{4.475134in}{1.813651in}}%
\pgfpathlineto{\pgfqpoint{4.483031in}{1.830567in}}%
\pgfpathlineto{\pgfqpoint{4.490924in}{1.847449in}}%
\pgfpathlineto{\pgfqpoint{4.476901in}{1.838489in}}%
\pgfpathlineto{\pgfqpoint{4.462893in}{1.829686in}}%
\pgfpathlineto{\pgfqpoint{4.448901in}{1.821042in}}%
\pgfpathlineto{\pgfqpoint{4.434923in}{1.812555in}}%
\pgfpathlineto{\pgfqpoint{4.427036in}{1.796127in}}%
\pgfpathlineto{\pgfqpoint{4.419146in}{1.779673in}}%
\pgfpathlineto{\pgfqpoint{4.411252in}{1.763195in}}%
\pgfpathlineto{\pgfqpoint{4.403354in}{1.746699in}}%
\pgfpathclose%
\pgfusepath{fill}%
\end{pgfscope}%
\begin{pgfscope}%
\pgfpathrectangle{\pgfqpoint{1.254980in}{0.150000in}}{\pgfqpoint{5.490039in}{5.490039in}}%
\pgfusepath{clip}%
\pgfsetbuttcap%
\pgfsetroundjoin%
\definecolor{currentfill}{rgb}{0.273006,0.204520,0.501721}%
\pgfsetfillcolor{currentfill}%
\pgfsetfillopacity{0.700000}%
\pgfsetlinewidth{0.000000pt}%
\definecolor{currentstroke}{rgb}{0.000000,0.000000,0.000000}%
\pgfsetstrokecolor{currentstroke}%
\pgfsetdash{}{0pt}%
\pgfpathmoveto{\pgfqpoint{4.165352in}{1.442472in}}%
\pgfpathlineto{\pgfqpoint{4.179206in}{1.447125in}}%
\pgfpathlineto{\pgfqpoint{4.193072in}{1.451933in}}%
\pgfpathlineto{\pgfqpoint{4.206950in}{1.456894in}}%
\pgfpathlineto{\pgfqpoint{4.220841in}{1.462010in}}%
\pgfpathlineto{\pgfqpoint{4.228790in}{1.477577in}}%
\pgfpathlineto{\pgfqpoint{4.236736in}{1.493211in}}%
\pgfpathlineto{\pgfqpoint{4.244679in}{1.508906in}}%
\pgfpathlineto{\pgfqpoint{4.252618in}{1.524656in}}%
\pgfpathlineto{\pgfqpoint{4.238725in}{1.518968in}}%
\pgfpathlineto{\pgfqpoint{4.224846in}{1.513434in}}%
\pgfpathlineto{\pgfqpoint{4.210978in}{1.508056in}}%
\pgfpathlineto{\pgfqpoint{4.197123in}{1.502832in}}%
\pgfpathlineto{\pgfqpoint{4.189186in}{1.487643in}}%
\pgfpathlineto{\pgfqpoint{4.181245in}{1.472516in}}%
\pgfpathlineto{\pgfqpoint{4.173300in}{1.457457in}}%
\pgfpathlineto{\pgfqpoint{4.165352in}{1.442472in}}%
\pgfpathclose%
\pgfusepath{fill}%
\end{pgfscope}%
\begin{pgfscope}%
\pgfpathrectangle{\pgfqpoint{1.254980in}{0.150000in}}{\pgfqpoint{5.490039in}{5.490039in}}%
\pgfusepath{clip}%
\pgfsetbuttcap%
\pgfsetroundjoin%
\definecolor{currentfill}{rgb}{0.283091,0.110553,0.431554}%
\pgfsetfillcolor{currentfill}%
\pgfsetfillopacity{0.700000}%
\pgfsetlinewidth{0.000000pt}%
\definecolor{currentstroke}{rgb}{0.000000,0.000000,0.000000}%
\pgfsetstrokecolor{currentstroke}%
\pgfsetdash{}{0pt}%
\pgfpathmoveto{\pgfqpoint{3.959143in}{1.253745in}}%
\pgfpathlineto{\pgfqpoint{3.972922in}{1.255298in}}%
\pgfpathlineto{\pgfqpoint{3.986711in}{1.257005in}}%
\pgfpathlineto{\pgfqpoint{4.000510in}{1.258865in}}%
\pgfpathlineto{\pgfqpoint{4.014318in}{1.260877in}}%
\pgfpathlineto{\pgfqpoint{4.022321in}{1.273869in}}%
\pgfpathlineto{\pgfqpoint{4.030319in}{1.287012in}}%
\pgfpathlineto{\pgfqpoint{4.038312in}{1.300300in}}%
\pgfpathlineto{\pgfqpoint{4.046301in}{1.313727in}}%
\pgfpathlineto{\pgfqpoint{4.032498in}{1.311064in}}%
\pgfpathlineto{\pgfqpoint{4.018705in}{1.308554in}}%
\pgfpathlineto{\pgfqpoint{4.004921in}{1.306198in}}%
\pgfpathlineto{\pgfqpoint{3.991148in}{1.303996in}}%
\pgfpathlineto{\pgfqpoint{3.983154in}{1.291209in}}%
\pgfpathlineto{\pgfqpoint{3.975156in}{1.278567in}}%
\pgfpathlineto{\pgfqpoint{3.967152in}{1.266077in}}%
\pgfpathlineto{\pgfqpoint{3.959143in}{1.253745in}}%
\pgfpathclose%
\pgfusepath{fill}%
\end{pgfscope}%
\begin{pgfscope}%
\pgfpathrectangle{\pgfqpoint{1.254980in}{0.150000in}}{\pgfqpoint{5.490039in}{5.490039in}}%
\pgfusepath{clip}%
\pgfsetbuttcap%
\pgfsetroundjoin%
\definecolor{currentfill}{rgb}{0.129933,0.559582,0.551864}%
\pgfsetfillcolor{currentfill}%
\pgfsetfillopacity{0.700000}%
\pgfsetlinewidth{0.000000pt}%
\definecolor{currentstroke}{rgb}{0.000000,0.000000,0.000000}%
\pgfsetstrokecolor{currentstroke}%
\pgfsetdash{}{0pt}%
\pgfpathmoveto{\pgfqpoint{4.792260in}{2.327588in}}%
\pgfpathlineto{\pgfqpoint{4.806479in}{2.340189in}}%
\pgfpathlineto{\pgfqpoint{4.820717in}{2.352952in}}%
\pgfpathlineto{\pgfqpoint{4.834973in}{2.365878in}}%
\pgfpathlineto{\pgfqpoint{4.849248in}{2.378966in}}%
\pgfpathlineto{\pgfqpoint{4.857053in}{2.394653in}}%
\pgfpathlineto{\pgfqpoint{4.864853in}{2.410198in}}%
\pgfpathlineto{\pgfqpoint{4.872647in}{2.425600in}}%
\pgfpathlineto{\pgfqpoint{4.880435in}{2.440856in}}%
\pgfpathlineto{\pgfqpoint{4.866152in}{2.427502in}}%
\pgfpathlineto{\pgfqpoint{4.851888in}{2.414311in}}%
\pgfpathlineto{\pgfqpoint{4.837643in}{2.401283in}}%
\pgfpathlineto{\pgfqpoint{4.823416in}{2.388419in}}%
\pgfpathlineto{\pgfqpoint{4.815635in}{2.373416in}}%
\pgfpathlineto{\pgfqpoint{4.807849in}{2.358276in}}%
\pgfpathlineto{\pgfqpoint{4.800057in}{2.342999in}}%
\pgfpathlineto{\pgfqpoint{4.792260in}{2.327588in}}%
\pgfpathclose%
\pgfusepath{fill}%
\end{pgfscope}%
\begin{pgfscope}%
\pgfpathrectangle{\pgfqpoint{1.254980in}{0.150000in}}{\pgfqpoint{5.490039in}{5.490039in}}%
\pgfusepath{clip}%
\pgfsetbuttcap%
\pgfsetroundjoin%
\definecolor{currentfill}{rgb}{0.188923,0.410910,0.556326}%
\pgfsetfillcolor{currentfill}%
\pgfsetfillopacity{0.700000}%
\pgfsetlinewidth{0.000000pt}%
\definecolor{currentstroke}{rgb}{0.000000,0.000000,0.000000}%
\pgfsetstrokecolor{currentstroke}%
\pgfsetdash{}{0pt}%
\pgfpathmoveto{\pgfqpoint{4.522460in}{1.914551in}}%
\pgfpathlineto{\pgfqpoint{4.536505in}{1.924107in}}%
\pgfpathlineto{\pgfqpoint{4.550566in}{1.933821in}}%
\pgfpathlineto{\pgfqpoint{4.564643in}{1.943694in}}%
\pgfpathlineto{\pgfqpoint{4.578736in}{1.953726in}}%
\pgfpathlineto{\pgfqpoint{4.586618in}{1.970796in}}%
\pgfpathlineto{\pgfqpoint{4.594496in}{1.987801in}}%
\pgfpathlineto{\pgfqpoint{4.602371in}{2.004737in}}%
\pgfpathlineto{\pgfqpoint{4.610241in}{2.021601in}}%
\pgfpathlineto{\pgfqpoint{4.596140in}{2.011158in}}%
\pgfpathlineto{\pgfqpoint{4.582056in}{2.000875in}}%
\pgfpathlineto{\pgfqpoint{4.567987in}{1.990751in}}%
\pgfpathlineto{\pgfqpoint{4.553934in}{1.980787in}}%
\pgfpathlineto{\pgfqpoint{4.546072in}{1.964322in}}%
\pgfpathlineto{\pgfqpoint{4.538205in}{1.947793in}}%
\pgfpathlineto{\pgfqpoint{4.530334in}{1.931201in}}%
\pgfpathlineto{\pgfqpoint{4.522460in}{1.914551in}}%
\pgfpathclose%
\pgfusepath{fill}%
\end{pgfscope}%
\begin{pgfscope}%
\pgfpathrectangle{\pgfqpoint{1.254980in}{0.150000in}}{\pgfqpoint{5.490039in}{5.490039in}}%
\pgfusepath{clip}%
\pgfsetbuttcap%
\pgfsetroundjoin%
\definecolor{currentfill}{rgb}{0.487026,0.823929,0.312321}%
\pgfsetfillcolor{currentfill}%
\pgfsetfillopacity{0.700000}%
\pgfsetlinewidth{0.000000pt}%
\definecolor{currentstroke}{rgb}{0.000000,0.000000,0.000000}%
\pgfsetstrokecolor{currentstroke}%
\pgfsetdash{}{0pt}%
\pgfpathmoveto{\pgfqpoint{5.387792in}{3.132660in}}%
\pgfpathlineto{\pgfqpoint{5.402458in}{3.149836in}}%
\pgfpathlineto{\pgfqpoint{5.417147in}{3.167180in}}%
\pgfpathlineto{\pgfqpoint{5.431859in}{3.184695in}}%
\pgfpathlineto{\pgfqpoint{5.446593in}{3.202379in}}%
\pgfpathlineto{\pgfqpoint{5.454085in}{3.211209in}}%
\pgfpathlineto{\pgfqpoint{5.461565in}{3.219827in}}%
\pgfpathlineto{\pgfqpoint{5.469035in}{3.228234in}}%
\pgfpathlineto{\pgfqpoint{5.476493in}{3.236432in}}%
\pgfpathlineto{\pgfqpoint{5.461762in}{3.218799in}}%
\pgfpathlineto{\pgfqpoint{5.447054in}{3.201336in}}%
\pgfpathlineto{\pgfqpoint{5.432369in}{3.184043in}}%
\pgfpathlineto{\pgfqpoint{5.417707in}{3.166919in}}%
\pgfpathlineto{\pgfqpoint{5.410244in}{3.158658in}}%
\pgfpathlineto{\pgfqpoint{5.402770in}{3.150195in}}%
\pgfpathlineto{\pgfqpoint{5.395286in}{3.141529in}}%
\pgfpathlineto{\pgfqpoint{5.387792in}{3.132660in}}%
\pgfpathclose%
\pgfusepath{fill}%
\end{pgfscope}%
\begin{pgfscope}%
\pgfpathrectangle{\pgfqpoint{1.254980in}{0.150000in}}{\pgfqpoint{5.490039in}{5.490039in}}%
\pgfusepath{clip}%
\pgfsetbuttcap%
\pgfsetroundjoin%
\definecolor{currentfill}{rgb}{0.120638,0.625828,0.533488}%
\pgfsetfillcolor{currentfill}%
\pgfsetfillopacity{0.700000}%
\pgfsetlinewidth{0.000000pt}%
\definecolor{currentstroke}{rgb}{0.000000,0.000000,0.000000}%
\pgfsetstrokecolor{currentstroke}%
\pgfsetdash{}{0pt}%
\pgfpathmoveto{\pgfqpoint{4.911525in}{2.500377in}}%
\pgfpathlineto{\pgfqpoint{4.925833in}{2.514130in}}%
\pgfpathlineto{\pgfqpoint{4.940161in}{2.528047in}}%
\pgfpathlineto{\pgfqpoint{4.954509in}{2.542129in}}%
\pgfpathlineto{\pgfqpoint{4.968876in}{2.556375in}}%
\pgfpathlineto{\pgfqpoint{4.976639in}{2.571086in}}%
\pgfpathlineto{\pgfqpoint{4.984396in}{2.585632in}}%
\pgfpathlineto{\pgfqpoint{4.992146in}{2.600012in}}%
\pgfpathlineto{\pgfqpoint{4.999889in}{2.614222in}}%
\pgfpathlineto{\pgfqpoint{4.985515in}{2.599771in}}%
\pgfpathlineto{\pgfqpoint{4.971161in}{2.585485in}}%
\pgfpathlineto{\pgfqpoint{4.956827in}{2.571364in}}%
\pgfpathlineto{\pgfqpoint{4.942512in}{2.557408in}}%
\pgfpathlineto{\pgfqpoint{4.934775in}{2.543390in}}%
\pgfpathlineto{\pgfqpoint{4.927032in}{2.529211in}}%
\pgfpathlineto{\pgfqpoint{4.919281in}{2.514873in}}%
\pgfpathlineto{\pgfqpoint{4.911525in}{2.500377in}}%
\pgfpathclose%
\pgfusepath{fill}%
\end{pgfscope}%
\begin{pgfscope}%
\pgfpathrectangle{\pgfqpoint{1.254980in}{0.150000in}}{\pgfqpoint{5.490039in}{5.490039in}}%
\pgfusepath{clip}%
\pgfsetbuttcap%
\pgfsetroundjoin%
\definecolor{currentfill}{rgb}{0.282290,0.145912,0.461510}%
\pgfsetfillcolor{currentfill}%
\pgfsetfillopacity{0.700000}%
\pgfsetlinewidth{0.000000pt}%
\definecolor{currentstroke}{rgb}{0.000000,0.000000,0.000000}%
\pgfsetstrokecolor{currentstroke}%
\pgfsetdash{}{0pt}%
\pgfpathmoveto{\pgfqpoint{4.046301in}{1.313727in}}%
\pgfpathlineto{\pgfqpoint{4.060115in}{1.316543in}}%
\pgfpathlineto{\pgfqpoint{4.073939in}{1.319513in}}%
\pgfpathlineto{\pgfqpoint{4.087773in}{1.322636in}}%
\pgfpathlineto{\pgfqpoint{4.101619in}{1.325912in}}%
\pgfpathlineto{\pgfqpoint{4.109600in}{1.340106in}}%
\pgfpathlineto{\pgfqpoint{4.117576in}{1.354418in}}%
\pgfpathlineto{\pgfqpoint{4.125549in}{1.368844in}}%
\pgfpathlineto{\pgfqpoint{4.133518in}{1.383378in}}%
\pgfpathlineto{\pgfqpoint{4.119674in}{1.379477in}}%
\pgfpathlineto{\pgfqpoint{4.105842in}{1.375729in}}%
\pgfpathlineto{\pgfqpoint{4.092020in}{1.372135in}}%
\pgfpathlineto{\pgfqpoint{4.078210in}{1.368696in}}%
\pgfpathlineto{\pgfqpoint{4.070239in}{1.354776in}}%
\pgfpathlineto{\pgfqpoint{4.062265in}{1.340971in}}%
\pgfpathlineto{\pgfqpoint{4.054285in}{1.327285in}}%
\pgfpathlineto{\pgfqpoint{4.046301in}{1.313727in}}%
\pgfpathclose%
\pgfusepath{fill}%
\end{pgfscope}%
\begin{pgfscope}%
\pgfpathrectangle{\pgfqpoint{1.254980in}{0.150000in}}{\pgfqpoint{5.490039in}{5.490039in}}%
\pgfusepath{clip}%
\pgfsetbuttcap%
\pgfsetroundjoin%
\definecolor{currentfill}{rgb}{0.162016,0.687316,0.499129}%
\pgfsetfillcolor{currentfill}%
\pgfsetfillopacity{0.700000}%
\pgfsetlinewidth{0.000000pt}%
\definecolor{currentstroke}{rgb}{0.000000,0.000000,0.000000}%
\pgfsetstrokecolor{currentstroke}%
\pgfsetdash{}{0pt}%
\pgfpathmoveto{\pgfqpoint{5.030790in}{2.669349in}}%
\pgfpathlineto{\pgfqpoint{5.045189in}{2.684139in}}%
\pgfpathlineto{\pgfqpoint{5.059609in}{2.699094in}}%
\pgfpathlineto{\pgfqpoint{5.074049in}{2.714216in}}%
\pgfpathlineto{\pgfqpoint{5.088509in}{2.729504in}}%
\pgfpathlineto{\pgfqpoint{5.096221in}{2.743004in}}%
\pgfpathlineto{\pgfqpoint{5.103925in}{2.756320in}}%
\pgfpathlineto{\pgfqpoint{5.111622in}{2.769450in}}%
\pgfpathlineto{\pgfqpoint{5.119310in}{2.782394in}}%
\pgfpathlineto{\pgfqpoint{5.104845in}{2.766964in}}%
\pgfpathlineto{\pgfqpoint{5.090400in}{2.751700in}}%
\pgfpathlineto{\pgfqpoint{5.075976in}{2.736602in}}%
\pgfpathlineto{\pgfqpoint{5.061572in}{2.721670in}}%
\pgfpathlineto{\pgfqpoint{5.053888in}{2.708857in}}%
\pgfpathlineto{\pgfqpoint{5.046196in}{2.695865in}}%
\pgfpathlineto{\pgfqpoint{5.038497in}{2.682695in}}%
\pgfpathlineto{\pgfqpoint{5.030790in}{2.669349in}}%
\pgfpathclose%
\pgfusepath{fill}%
\end{pgfscope}%
\begin{pgfscope}%
\pgfpathrectangle{\pgfqpoint{1.254980in}{0.150000in}}{\pgfqpoint{5.490039in}{5.490039in}}%
\pgfusepath{clip}%
\pgfsetbuttcap%
\pgfsetroundjoin%
\definecolor{currentfill}{rgb}{0.160665,0.478540,0.558115}%
\pgfsetfillcolor{currentfill}%
\pgfsetfillopacity{0.700000}%
\pgfsetlinewidth{0.000000pt}%
\definecolor{currentstroke}{rgb}{0.000000,0.000000,0.000000}%
\pgfsetstrokecolor{currentstroke}%
\pgfsetdash{}{0pt}%
\pgfpathmoveto{\pgfqpoint{4.641681in}{2.088255in}}%
\pgfpathlineto{\pgfqpoint{4.655807in}{2.099240in}}%
\pgfpathlineto{\pgfqpoint{4.669949in}{2.110384in}}%
\pgfpathlineto{\pgfqpoint{4.684109in}{2.121689in}}%
\pgfpathlineto{\pgfqpoint{4.698286in}{2.133155in}}%
\pgfpathlineto{\pgfqpoint{4.706143in}{2.149966in}}%
\pgfpathlineto{\pgfqpoint{4.713997in}{2.166677in}}%
\pgfpathlineto{\pgfqpoint{4.721845in}{2.183284in}}%
\pgfpathlineto{\pgfqpoint{4.729689in}{2.199784in}}%
\pgfpathlineto{\pgfqpoint{4.715504in}{2.187964in}}%
\pgfpathlineto{\pgfqpoint{4.701335in}{2.176306in}}%
\pgfpathlineto{\pgfqpoint{4.687184in}{2.164808in}}%
\pgfpathlineto{\pgfqpoint{4.673051in}{2.153472in}}%
\pgfpathlineto{\pgfqpoint{4.665215in}{2.137314in}}%
\pgfpathlineto{\pgfqpoint{4.657375in}{2.121056in}}%
\pgfpathlineto{\pgfqpoint{4.649530in}{2.104702in}}%
\pgfpathlineto{\pgfqpoint{4.641681in}{2.088255in}}%
\pgfpathclose%
\pgfusepath{fill}%
\end{pgfscope}%
\begin{pgfscope}%
\pgfpathrectangle{\pgfqpoint{1.254980in}{0.150000in}}{\pgfqpoint{5.490039in}{5.490039in}}%
\pgfusepath{clip}%
\pgfsetbuttcap%
\pgfsetroundjoin%
\definecolor{currentfill}{rgb}{0.369214,0.788888,0.382914}%
\pgfsetfillcolor{currentfill}%
\pgfsetfillopacity{0.700000}%
\pgfsetlinewidth{0.000000pt}%
\definecolor{currentstroke}{rgb}{0.000000,0.000000,0.000000}%
\pgfsetstrokecolor{currentstroke}%
\pgfsetdash{}{0pt}%
\pgfpathmoveto{\pgfqpoint{5.269014in}{2.987268in}}%
\pgfpathlineto{\pgfqpoint{5.283593in}{3.003772in}}%
\pgfpathlineto{\pgfqpoint{5.298195in}{3.020445in}}%
\pgfpathlineto{\pgfqpoint{5.312818in}{3.037287in}}%
\pgfpathlineto{\pgfqpoint{5.327464in}{3.054297in}}%
\pgfpathlineto{\pgfqpoint{5.335041in}{3.064818in}}%
\pgfpathlineto{\pgfqpoint{5.342607in}{3.075130in}}%
\pgfpathlineto{\pgfqpoint{5.350164in}{3.085235in}}%
\pgfpathlineto{\pgfqpoint{5.357710in}{3.095133in}}%
\pgfpathlineto{\pgfqpoint{5.343064in}{3.078108in}}%
\pgfpathlineto{\pgfqpoint{5.328441in}{3.061252in}}%
\pgfpathlineto{\pgfqpoint{5.313840in}{3.044565in}}%
\pgfpathlineto{\pgfqpoint{5.299261in}{3.028047in}}%
\pgfpathlineto{\pgfqpoint{5.291714in}{3.018151in}}%
\pgfpathlineto{\pgfqpoint{5.284157in}{3.008056in}}%
\pgfpathlineto{\pgfqpoint{5.276590in}{2.997762in}}%
\pgfpathlineto{\pgfqpoint{5.269014in}{2.987268in}}%
\pgfpathclose%
\pgfusepath{fill}%
\end{pgfscope}%
\begin{pgfscope}%
\pgfpathrectangle{\pgfqpoint{1.254980in}{0.150000in}}{\pgfqpoint{5.490039in}{5.490039in}}%
\pgfusepath{clip}%
\pgfsetbuttcap%
\pgfsetroundjoin%
\definecolor{currentfill}{rgb}{0.252899,0.742211,0.448284}%
\pgfsetfillcolor{currentfill}%
\pgfsetfillopacity{0.700000}%
\pgfsetlinewidth{0.000000pt}%
\definecolor{currentstroke}{rgb}{0.000000,0.000000,0.000000}%
\pgfsetstrokecolor{currentstroke}%
\pgfsetdash{}{0pt}%
\pgfpathmoveto{\pgfqpoint{5.149981in}{2.832288in}}%
\pgfpathlineto{\pgfqpoint{5.164472in}{2.847996in}}%
\pgfpathlineto{\pgfqpoint{5.178983in}{2.863872in}}%
\pgfpathlineto{\pgfqpoint{5.193515in}{2.879914in}}%
\pgfpathlineto{\pgfqpoint{5.208069in}{2.896125in}}%
\pgfpathlineto{\pgfqpoint{5.215719in}{2.908216in}}%
\pgfpathlineto{\pgfqpoint{5.223360in}{2.920107in}}%
\pgfpathlineto{\pgfqpoint{5.230992in}{2.931800in}}%
\pgfpathlineto{\pgfqpoint{5.238615in}{2.943293in}}%
\pgfpathlineto{\pgfqpoint{5.224058in}{2.927004in}}%
\pgfpathlineto{\pgfqpoint{5.209523in}{2.910882in}}%
\pgfpathlineto{\pgfqpoint{5.195010in}{2.894928in}}%
\pgfpathlineto{\pgfqpoint{5.180517in}{2.879141in}}%
\pgfpathlineto{\pgfqpoint{5.172896in}{2.867714in}}%
\pgfpathlineto{\pgfqpoint{5.165267in}{2.856096in}}%
\pgfpathlineto{\pgfqpoint{5.157628in}{2.844287in}}%
\pgfpathlineto{\pgfqpoint{5.149981in}{2.832288in}}%
\pgfpathclose%
\pgfusepath{fill}%
\end{pgfscope}%
\begin{pgfscope}%
\pgfpathrectangle{\pgfqpoint{1.254980in}{0.150000in}}{\pgfqpoint{5.490039in}{5.490039in}}%
\pgfusepath{clip}%
\pgfsetbuttcap%
\pgfsetroundjoin%
\definecolor{currentfill}{rgb}{0.258965,0.251537,0.524736}%
\pgfsetfillcolor{currentfill}%
\pgfsetfillopacity{0.700000}%
\pgfsetlinewidth{0.000000pt}%
\definecolor{currentstroke}{rgb}{0.000000,0.000000,0.000000}%
\pgfsetstrokecolor{currentstroke}%
\pgfsetdash{}{0pt}%
\pgfpathmoveto{\pgfqpoint{4.252618in}{1.524656in}}%
\pgfpathlineto{\pgfqpoint{4.266522in}{1.530500in}}%
\pgfpathlineto{\pgfqpoint{4.280440in}{1.536498in}}%
\pgfpathlineto{\pgfqpoint{4.294370in}{1.542651in}}%
\pgfpathlineto{\pgfqpoint{4.308314in}{1.548959in}}%
\pgfpathlineto{\pgfqpoint{4.316253in}{1.565317in}}%
\pgfpathlineto{\pgfqpoint{4.324188in}{1.581713in}}%
\pgfpathlineto{\pgfqpoint{4.332120in}{1.598143in}}%
\pgfpathlineto{\pgfqpoint{4.340048in}{1.614602in}}%
\pgfpathlineto{\pgfqpoint{4.326101in}{1.607746in}}%
\pgfpathlineto{\pgfqpoint{4.312166in}{1.601046in}}%
\pgfpathlineto{\pgfqpoint{4.298245in}{1.594502in}}%
\pgfpathlineto{\pgfqpoint{4.284338in}{1.588114in}}%
\pgfpathlineto{\pgfqpoint{4.276413in}{1.572191in}}%
\pgfpathlineto{\pgfqpoint{4.268485in}{1.556304in}}%
\pgfpathlineto{\pgfqpoint{4.260553in}{1.540458in}}%
\pgfpathlineto{\pgfqpoint{4.252618in}{1.524656in}}%
\pgfpathclose%
\pgfusepath{fill}%
\end{pgfscope}%
\begin{pgfscope}%
\pgfpathrectangle{\pgfqpoint{1.254980in}{0.150000in}}{\pgfqpoint{5.490039in}{5.490039in}}%
\pgfusepath{clip}%
\pgfsetbuttcap%
\pgfsetroundjoin%
\definecolor{currentfill}{rgb}{0.229739,0.322361,0.545706}%
\pgfsetfillcolor{currentfill}%
\pgfsetfillopacity{0.700000}%
\pgfsetlinewidth{0.000000pt}%
\definecolor{currentstroke}{rgb}{0.000000,0.000000,0.000000}%
\pgfsetstrokecolor{currentstroke}%
\pgfsetdash{}{0pt}%
\pgfpathmoveto{\pgfqpoint{4.371729in}{1.680618in}}%
\pgfpathlineto{\pgfqpoint{4.385695in}{1.688149in}}%
\pgfpathlineto{\pgfqpoint{4.399676in}{1.695836in}}%
\pgfpathlineto{\pgfqpoint{4.413671in}{1.703680in}}%
\pgfpathlineto{\pgfqpoint{4.427680in}{1.711680in}}%
\pgfpathlineto{\pgfqpoint{4.435597in}{1.728710in}}%
\pgfpathlineto{\pgfqpoint{4.443512in}{1.745732in}}%
\pgfpathlineto{\pgfqpoint{4.451422in}{1.762741in}}%
\pgfpathlineto{\pgfqpoint{4.459330in}{1.779734in}}%
\pgfpathlineto{\pgfqpoint{4.445314in}{1.771240in}}%
\pgfpathlineto{\pgfqpoint{4.431313in}{1.762902in}}%
\pgfpathlineto{\pgfqpoint{4.417326in}{1.754722in}}%
\pgfpathlineto{\pgfqpoint{4.403354in}{1.746699in}}%
\pgfpathlineto{\pgfqpoint{4.395453in}{1.730189in}}%
\pgfpathlineto{\pgfqpoint{4.387548in}{1.713669in}}%
\pgfpathlineto{\pgfqpoint{4.379640in}{1.697144in}}%
\pgfpathlineto{\pgfqpoint{4.371729in}{1.680618in}}%
\pgfpathclose%
\pgfusepath{fill}%
\end{pgfscope}%
\begin{pgfscope}%
\pgfpathrectangle{\pgfqpoint{1.254980in}{0.150000in}}{\pgfqpoint{5.490039in}{5.490039in}}%
\pgfusepath{clip}%
\pgfsetbuttcap%
\pgfsetroundjoin%
\definecolor{currentfill}{rgb}{0.277134,0.185228,0.489898}%
\pgfsetfillcolor{currentfill}%
\pgfsetfillopacity{0.700000}%
\pgfsetlinewidth{0.000000pt}%
\definecolor{currentstroke}{rgb}{0.000000,0.000000,0.000000}%
\pgfsetstrokecolor{currentstroke}%
\pgfsetdash{}{0pt}%
\pgfpathmoveto{\pgfqpoint{4.133518in}{1.383378in}}%
\pgfpathlineto{\pgfqpoint{4.147372in}{1.387433in}}%
\pgfpathlineto{\pgfqpoint{4.161238in}{1.391642in}}%
\pgfpathlineto{\pgfqpoint{4.175116in}{1.396004in}}%
\pgfpathlineto{\pgfqpoint{4.189005in}{1.400520in}}%
\pgfpathlineto{\pgfqpoint{4.196970in}{1.415765in}}%
\pgfpathlineto{\pgfqpoint{4.204930in}{1.431099in}}%
\pgfpathlineto{\pgfqpoint{4.212887in}{1.446516in}}%
\pgfpathlineto{\pgfqpoint{4.220841in}{1.462010in}}%
\pgfpathlineto{\pgfqpoint{4.206950in}{1.456894in}}%
\pgfpathlineto{\pgfqpoint{4.193072in}{1.451933in}}%
\pgfpathlineto{\pgfqpoint{4.179206in}{1.447125in}}%
\pgfpathlineto{\pgfqpoint{4.165352in}{1.442472in}}%
\pgfpathlineto{\pgfqpoint{4.157399in}{1.427566in}}%
\pgfpathlineto{\pgfqpoint{4.149442in}{1.412745in}}%
\pgfpathlineto{\pgfqpoint{4.141482in}{1.398013in}}%
\pgfpathlineto{\pgfqpoint{4.133518in}{1.383378in}}%
\pgfpathclose%
\pgfusepath{fill}%
\end{pgfscope}%
\begin{pgfscope}%
\pgfpathrectangle{\pgfqpoint{1.254980in}{0.150000in}}{\pgfqpoint{5.490039in}{5.490039in}}%
\pgfusepath{clip}%
\pgfsetbuttcap%
\pgfsetroundjoin%
\definecolor{currentfill}{rgb}{0.135066,0.544853,0.554029}%
\pgfsetfillcolor{currentfill}%
\pgfsetfillopacity{0.700000}%
\pgfsetlinewidth{0.000000pt}%
\definecolor{currentstroke}{rgb}{0.000000,0.000000,0.000000}%
\pgfsetstrokecolor{currentstroke}%
\pgfsetdash{}{0pt}%
\pgfpathmoveto{\pgfqpoint{4.761015in}{2.264656in}}%
\pgfpathlineto{\pgfqpoint{4.775227in}{2.276962in}}%
\pgfpathlineto{\pgfqpoint{4.789457in}{2.289430in}}%
\pgfpathlineto{\pgfqpoint{4.803705in}{2.302060in}}%
\pgfpathlineto{\pgfqpoint{4.817971in}{2.314853in}}%
\pgfpathlineto{\pgfqpoint{4.825798in}{2.331081in}}%
\pgfpathlineto{\pgfqpoint{4.833620in}{2.347178in}}%
\pgfpathlineto{\pgfqpoint{4.841437in}{2.363141in}}%
\pgfpathlineto{\pgfqpoint{4.849248in}{2.378966in}}%
\pgfpathlineto{\pgfqpoint{4.834973in}{2.365878in}}%
\pgfpathlineto{\pgfqpoint{4.820717in}{2.352952in}}%
\pgfpathlineto{\pgfqpoint{4.806479in}{2.340189in}}%
\pgfpathlineto{\pgfqpoint{4.792260in}{2.327588in}}%
\pgfpathlineto{\pgfqpoint{4.784457in}{2.312046in}}%
\pgfpathlineto{\pgfqpoint{4.776648in}{2.296375in}}%
\pgfpathlineto{\pgfqpoint{4.768834in}{2.280577in}}%
\pgfpathlineto{\pgfqpoint{4.761015in}{2.264656in}}%
\pgfpathclose%
\pgfusepath{fill}%
\end{pgfscope}%
\begin{pgfscope}%
\pgfpathrectangle{\pgfqpoint{1.254980in}{0.150000in}}{\pgfqpoint{5.490039in}{5.490039in}}%
\pgfusepath{clip}%
\pgfsetbuttcap%
\pgfsetroundjoin%
\definecolor{currentfill}{rgb}{0.197636,0.391528,0.554969}%
\pgfsetfillcolor{currentfill}%
\pgfsetfillopacity{0.700000}%
\pgfsetlinewidth{0.000000pt}%
\definecolor{currentstroke}{rgb}{0.000000,0.000000,0.000000}%
\pgfsetstrokecolor{currentstroke}%
\pgfsetdash{}{0pt}%
\pgfpathmoveto{\pgfqpoint{4.490924in}{1.847449in}}%
\pgfpathlineto{\pgfqpoint{4.504962in}{1.856567in}}%
\pgfpathlineto{\pgfqpoint{4.519015in}{1.865842in}}%
\pgfpathlineto{\pgfqpoint{4.533084in}{1.875276in}}%
\pgfpathlineto{\pgfqpoint{4.547168in}{1.884869in}}%
\pgfpathlineto{\pgfqpoint{4.555066in}{1.902161in}}%
\pgfpathlineto{\pgfqpoint{4.562959in}{1.919405in}}%
\pgfpathlineto{\pgfqpoint{4.570849in}{1.936594in}}%
\pgfpathlineto{\pgfqpoint{4.578736in}{1.953726in}}%
\pgfpathlineto{\pgfqpoint{4.564643in}{1.943694in}}%
\pgfpathlineto{\pgfqpoint{4.550566in}{1.933821in}}%
\pgfpathlineto{\pgfqpoint{4.536505in}{1.924107in}}%
\pgfpathlineto{\pgfqpoint{4.522460in}{1.914551in}}%
\pgfpathlineto{\pgfqpoint{4.514581in}{1.897847in}}%
\pgfpathlineto{\pgfqpoint{4.506699in}{1.881093in}}%
\pgfpathlineto{\pgfqpoint{4.498813in}{1.864292in}}%
\pgfpathlineto{\pgfqpoint{4.490924in}{1.847449in}}%
\pgfpathclose%
\pgfusepath{fill}%
\end{pgfscope}%
\begin{pgfscope}%
\pgfpathrectangle{\pgfqpoint{1.254980in}{0.150000in}}{\pgfqpoint{5.490039in}{5.490039in}}%
\pgfusepath{clip}%
\pgfsetbuttcap%
\pgfsetroundjoin%
\definecolor{currentfill}{rgb}{0.595839,0.848717,0.243329}%
\pgfsetfillcolor{currentfill}%
\pgfsetfillopacity{0.700000}%
\pgfsetlinewidth{0.000000pt}%
\definecolor{currentstroke}{rgb}{0.000000,0.000000,0.000000}%
\pgfsetstrokecolor{currentstroke}%
\pgfsetdash{}{0pt}%
\pgfpathmoveto{\pgfqpoint{5.476493in}{3.236432in}}%
\pgfpathlineto{\pgfqpoint{5.491248in}{3.254235in}}%
\pgfpathlineto{\pgfqpoint{5.506025in}{3.272208in}}%
\pgfpathlineto{\pgfqpoint{5.520827in}{3.290352in}}%
\pgfpathlineto{\pgfqpoint{5.528270in}{3.298286in}}%
\pgfpathlineto{\pgfqpoint{5.535702in}{3.306006in}}%
\pgfpathlineto{\pgfqpoint{5.543123in}{3.313514in}}%
\pgfpathlineto{\pgfqpoint{5.550532in}{3.320810in}}%
\pgfpathlineto{\pgfqpoint{5.535736in}{3.302751in}}%
\pgfpathlineto{\pgfqpoint{5.520964in}{3.284863in}}%
\pgfpathlineto{\pgfqpoint{5.506215in}{3.267145in}}%
\pgfpathlineto{\pgfqpoint{5.498802in}{3.259776in}}%
\pgfpathlineto{\pgfqpoint{5.491377in}{3.252201in}}%
\pgfpathlineto{\pgfqpoint{5.483941in}{3.244420in}}%
\pgfpathlineto{\pgfqpoint{5.476493in}{3.236432in}}%
\pgfpathclose%
\pgfusepath{fill}%
\end{pgfscope}%
\begin{pgfscope}%
\pgfpathrectangle{\pgfqpoint{1.254980in}{0.150000in}}{\pgfqpoint{5.490039in}{5.490039in}}%
\pgfusepath{clip}%
\pgfsetbuttcap%
\pgfsetroundjoin%
\definecolor{currentfill}{rgb}{0.279566,0.067836,0.391917}%
\pgfsetfillcolor{currentfill}%
\pgfsetfillopacity{0.700000}%
\pgfsetlinewidth{0.000000pt}%
\definecolor{currentstroke}{rgb}{0.000000,0.000000,0.000000}%
\pgfsetstrokecolor{currentstroke}%
\pgfsetdash{}{0pt}%
\pgfpathmoveto{\pgfqpoint{3.839760in}{1.162390in}}%
\pgfpathlineto{\pgfqpoint{3.853525in}{1.161951in}}%
\pgfpathlineto{\pgfqpoint{3.867297in}{1.161666in}}%
\pgfpathlineto{\pgfqpoint{3.881078in}{1.161533in}}%
\pgfpathlineto{\pgfqpoint{3.894867in}{1.161554in}}%
\pgfpathlineto{\pgfqpoint{3.902922in}{1.172387in}}%
\pgfpathlineto{\pgfqpoint{3.910971in}{1.183430in}}%
\pgfpathlineto{\pgfqpoint{3.919014in}{1.194679in}}%
\pgfpathlineto{\pgfqpoint{3.927051in}{1.206124in}}%
\pgfpathlineto{\pgfqpoint{3.913273in}{1.205400in}}%
\pgfpathlineto{\pgfqpoint{3.899503in}{1.204829in}}%
\pgfpathlineto{\pgfqpoint{3.885741in}{1.204412in}}%
\pgfpathlineto{\pgfqpoint{3.871988in}{1.204148in}}%
\pgfpathlineto{\pgfqpoint{3.863941in}{1.193395in}}%
\pgfpathlineto{\pgfqpoint{3.855887in}{1.182847in}}%
\pgfpathlineto{\pgfqpoint{3.847827in}{1.172509in}}%
\pgfpathlineto{\pgfqpoint{3.839760in}{1.162390in}}%
\pgfpathclose%
\pgfusepath{fill}%
\end{pgfscope}%
\begin{pgfscope}%
\pgfpathrectangle{\pgfqpoint{1.254980in}{0.150000in}}{\pgfqpoint{5.490039in}{5.490039in}}%
\pgfusepath{clip}%
\pgfsetbuttcap%
\pgfsetroundjoin%
\definecolor{currentfill}{rgb}{0.282327,0.094955,0.417331}%
\pgfsetfillcolor{currentfill}%
\pgfsetfillopacity{0.700000}%
\pgfsetlinewidth{0.000000pt}%
\definecolor{currentstroke}{rgb}{0.000000,0.000000,0.000000}%
\pgfsetstrokecolor{currentstroke}%
\pgfsetdash{}{0pt}%
\pgfpathmoveto{\pgfqpoint{3.927051in}{1.206124in}}%
\pgfpathlineto{\pgfqpoint{3.940839in}{1.207001in}}%
\pgfpathlineto{\pgfqpoint{3.954635in}{1.208031in}}%
\pgfpathlineto{\pgfqpoint{3.968441in}{1.209213in}}%
\pgfpathlineto{\pgfqpoint{3.982256in}{1.210548in}}%
\pgfpathlineto{\pgfqpoint{3.990279in}{1.222871in}}%
\pgfpathlineto{\pgfqpoint{3.998297in}{1.235372in}}%
\pgfpathlineto{\pgfqpoint{4.006310in}{1.248043in}}%
\pgfpathlineto{\pgfqpoint{4.014318in}{1.260877in}}%
\pgfpathlineto{\pgfqpoint{4.000510in}{1.258865in}}%
\pgfpathlineto{\pgfqpoint{3.986711in}{1.257005in}}%
\pgfpathlineto{\pgfqpoint{3.972922in}{1.255298in}}%
\pgfpathlineto{\pgfqpoint{3.959143in}{1.253745in}}%
\pgfpathlineto{\pgfqpoint{3.951128in}{1.241577in}}%
\pgfpathlineto{\pgfqpoint{3.943108in}{1.229580in}}%
\pgfpathlineto{\pgfqpoint{3.935082in}{1.217760in}}%
\pgfpathlineto{\pgfqpoint{3.927051in}{1.206124in}}%
\pgfpathclose%
\pgfusepath{fill}%
\end{pgfscope}%
\begin{pgfscope}%
\pgfpathrectangle{\pgfqpoint{1.254980in}{0.150000in}}{\pgfqpoint{5.490039in}{5.490039in}}%
\pgfusepath{clip}%
\pgfsetbuttcap%
\pgfsetroundjoin%
\definecolor{currentfill}{rgb}{0.119423,0.611141,0.538982}%
\pgfsetfillcolor{currentfill}%
\pgfsetfillopacity{0.700000}%
\pgfsetlinewidth{0.000000pt}%
\definecolor{currentstroke}{rgb}{0.000000,0.000000,0.000000}%
\pgfsetstrokecolor{currentstroke}%
\pgfsetdash{}{0pt}%
\pgfpathmoveto{\pgfqpoint{4.880435in}{2.440856in}}%
\pgfpathlineto{\pgfqpoint{4.894736in}{2.454374in}}%
\pgfpathlineto{\pgfqpoint{4.909057in}{2.468055in}}%
\pgfpathlineto{\pgfqpoint{4.923396in}{2.481901in}}%
\pgfpathlineto{\pgfqpoint{4.937755in}{2.495911in}}%
\pgfpathlineto{\pgfqpoint{4.945545in}{2.511266in}}%
\pgfpathlineto{\pgfqpoint{4.953328in}{2.526463in}}%
\pgfpathlineto{\pgfqpoint{4.961105in}{2.541500in}}%
\pgfpathlineto{\pgfqpoint{4.968876in}{2.556375in}}%
\pgfpathlineto{\pgfqpoint{4.954509in}{2.542129in}}%
\pgfpathlineto{\pgfqpoint{4.940161in}{2.528047in}}%
\pgfpathlineto{\pgfqpoint{4.925833in}{2.514130in}}%
\pgfpathlineto{\pgfqpoint{4.911525in}{2.500377in}}%
\pgfpathlineto{\pgfqpoint{4.903762in}{2.485726in}}%
\pgfpathlineto{\pgfqpoint{4.895992in}{2.470921in}}%
\pgfpathlineto{\pgfqpoint{4.888217in}{2.455963in}}%
\pgfpathlineto{\pgfqpoint{4.880435in}{2.440856in}}%
\pgfpathclose%
\pgfusepath{fill}%
\end{pgfscope}%
\begin{pgfscope}%
\pgfpathrectangle{\pgfqpoint{1.254980in}{0.150000in}}{\pgfqpoint{5.490039in}{5.490039in}}%
\pgfusepath{clip}%
\pgfsetbuttcap%
\pgfsetroundjoin%
\definecolor{currentfill}{rgb}{0.168126,0.459988,0.558082}%
\pgfsetfillcolor{currentfill}%
\pgfsetfillopacity{0.700000}%
\pgfsetlinewidth{0.000000pt}%
\definecolor{currentstroke}{rgb}{0.000000,0.000000,0.000000}%
\pgfsetstrokecolor{currentstroke}%
\pgfsetdash{}{0pt}%
\pgfpathmoveto{\pgfqpoint{4.610241in}{2.021601in}}%
\pgfpathlineto{\pgfqpoint{4.624359in}{2.032203in}}%
\pgfpathlineto{\pgfqpoint{4.638493in}{2.042965in}}%
\pgfpathlineto{\pgfqpoint{4.652643in}{2.053886in}}%
\pgfpathlineto{\pgfqpoint{4.666811in}{2.064968in}}%
\pgfpathlineto{\pgfqpoint{4.674686in}{2.082149in}}%
\pgfpathlineto{\pgfqpoint{4.682557in}{2.099243in}}%
\pgfpathlineto{\pgfqpoint{4.690423in}{2.116246in}}%
\pgfpathlineto{\pgfqpoint{4.698286in}{2.133155in}}%
\pgfpathlineto{\pgfqpoint{4.684109in}{2.121689in}}%
\pgfpathlineto{\pgfqpoint{4.669949in}{2.110384in}}%
\pgfpathlineto{\pgfqpoint{4.655807in}{2.099240in}}%
\pgfpathlineto{\pgfqpoint{4.641681in}{2.088255in}}%
\pgfpathlineto{\pgfqpoint{4.633828in}{2.071718in}}%
\pgfpathlineto{\pgfqpoint{4.625970in}{2.055095in}}%
\pgfpathlineto{\pgfqpoint{4.618108in}{2.038388in}}%
\pgfpathlineto{\pgfqpoint{4.610241in}{2.021601in}}%
\pgfpathclose%
\pgfusepath{fill}%
\end{pgfscope}%
\begin{pgfscope}%
\pgfpathrectangle{\pgfqpoint{1.254980in}{0.150000in}}{\pgfqpoint{5.490039in}{5.490039in}}%
\pgfusepath{clip}%
\pgfsetbuttcap%
\pgfsetroundjoin%
\definecolor{currentfill}{rgb}{0.283187,0.125848,0.444960}%
\pgfsetfillcolor{currentfill}%
\pgfsetfillopacity{0.700000}%
\pgfsetlinewidth{0.000000pt}%
\definecolor{currentstroke}{rgb}{0.000000,0.000000,0.000000}%
\pgfsetstrokecolor{currentstroke}%
\pgfsetdash{}{0pt}%
\pgfpathmoveto{\pgfqpoint{4.014318in}{1.260877in}}%
\pgfpathlineto{\pgfqpoint{4.028136in}{1.263043in}}%
\pgfpathlineto{\pgfqpoint{4.041964in}{1.265361in}}%
\pgfpathlineto{\pgfqpoint{4.055803in}{1.267833in}}%
\pgfpathlineto{\pgfqpoint{4.069651in}{1.270456in}}%
\pgfpathlineto{\pgfqpoint{4.077650in}{1.284110in}}%
\pgfpathlineto{\pgfqpoint{4.085644in}{1.297909in}}%
\pgfpathlineto{\pgfqpoint{4.093633in}{1.311845in}}%
\pgfpathlineto{\pgfqpoint{4.101619in}{1.325912in}}%
\pgfpathlineto{\pgfqpoint{4.087773in}{1.322636in}}%
\pgfpathlineto{\pgfqpoint{4.073939in}{1.319513in}}%
\pgfpathlineto{\pgfqpoint{4.060115in}{1.316543in}}%
\pgfpathlineto{\pgfqpoint{4.046301in}{1.313727in}}%
\pgfpathlineto{\pgfqpoint{4.038312in}{1.300300in}}%
\pgfpathlineto{\pgfqpoint{4.030319in}{1.287012in}}%
\pgfpathlineto{\pgfqpoint{4.022321in}{1.273869in}}%
\pgfpathlineto{\pgfqpoint{4.014318in}{1.260877in}}%
\pgfpathclose%
\pgfusepath{fill}%
\end{pgfscope}%
\begin{pgfscope}%
\pgfpathrectangle{\pgfqpoint{1.254980in}{0.150000in}}{\pgfqpoint{5.490039in}{5.490039in}}%
\pgfusepath{clip}%
\pgfsetbuttcap%
\pgfsetroundjoin%
\definecolor{currentfill}{rgb}{0.266580,0.228262,0.514349}%
\pgfsetfillcolor{currentfill}%
\pgfsetfillopacity{0.700000}%
\pgfsetlinewidth{0.000000pt}%
\definecolor{currentstroke}{rgb}{0.000000,0.000000,0.000000}%
\pgfsetstrokecolor{currentstroke}%
\pgfsetdash{}{0pt}%
\pgfpathmoveto{\pgfqpoint{4.220841in}{1.462010in}}%
\pgfpathlineto{\pgfqpoint{4.234743in}{1.467281in}}%
\pgfpathlineto{\pgfqpoint{4.248658in}{1.472705in}}%
\pgfpathlineto{\pgfqpoint{4.262585in}{1.478284in}}%
\pgfpathlineto{\pgfqpoint{4.276525in}{1.484016in}}%
\pgfpathlineto{\pgfqpoint{4.284478in}{1.500168in}}%
\pgfpathlineto{\pgfqpoint{4.292426in}{1.516379in}}%
\pgfpathlineto{\pgfqpoint{4.300372in}{1.532644in}}%
\pgfpathlineto{\pgfqpoint{4.308314in}{1.548959in}}%
\pgfpathlineto{\pgfqpoint{4.294370in}{1.542651in}}%
\pgfpathlineto{\pgfqpoint{4.280440in}{1.536498in}}%
\pgfpathlineto{\pgfqpoint{4.266522in}{1.530500in}}%
\pgfpathlineto{\pgfqpoint{4.252618in}{1.524656in}}%
\pgfpathlineto{\pgfqpoint{4.244679in}{1.508906in}}%
\pgfpathlineto{\pgfqpoint{4.236736in}{1.493211in}}%
\pgfpathlineto{\pgfqpoint{4.228790in}{1.477577in}}%
\pgfpathlineto{\pgfqpoint{4.220841in}{1.462010in}}%
\pgfpathclose%
\pgfusepath{fill}%
\end{pgfscope}%
\begin{pgfscope}%
\pgfpathrectangle{\pgfqpoint{1.254980in}{0.150000in}}{\pgfqpoint{5.490039in}{5.490039in}}%
\pgfusepath{clip}%
\pgfsetbuttcap%
\pgfsetroundjoin%
\definecolor{currentfill}{rgb}{0.239346,0.300855,0.540844}%
\pgfsetfillcolor{currentfill}%
\pgfsetfillopacity{0.700000}%
\pgfsetlinewidth{0.000000pt}%
\definecolor{currentstroke}{rgb}{0.000000,0.000000,0.000000}%
\pgfsetstrokecolor{currentstroke}%
\pgfsetdash{}{0pt}%
\pgfpathmoveto{\pgfqpoint{4.340048in}{1.614602in}}%
\pgfpathlineto{\pgfqpoint{4.354010in}{1.621612in}}%
\pgfpathlineto{\pgfqpoint{4.367985in}{1.628779in}}%
\pgfpathlineto{\pgfqpoint{4.381974in}{1.636101in}}%
\pgfpathlineto{\pgfqpoint{4.395976in}{1.643579in}}%
\pgfpathlineto{\pgfqpoint{4.403907in}{1.660592in}}%
\pgfpathlineto{\pgfqpoint{4.411835in}{1.677617in}}%
\pgfpathlineto{\pgfqpoint{4.419759in}{1.694648in}}%
\pgfpathlineto{\pgfqpoint{4.427680in}{1.711680in}}%
\pgfpathlineto{\pgfqpoint{4.413671in}{1.703680in}}%
\pgfpathlineto{\pgfqpoint{4.399676in}{1.695836in}}%
\pgfpathlineto{\pgfqpoint{4.385695in}{1.688149in}}%
\pgfpathlineto{\pgfqpoint{4.371729in}{1.680618in}}%
\pgfpathlineto{\pgfqpoint{4.363814in}{1.664096in}}%
\pgfpathlineto{\pgfqpoint{4.355895in}{1.647583in}}%
\pgfpathlineto{\pgfqpoint{4.347974in}{1.631083in}}%
\pgfpathlineto{\pgfqpoint{4.340048in}{1.614602in}}%
\pgfpathclose%
\pgfusepath{fill}%
\end{pgfscope}%
\begin{pgfscope}%
\pgfpathrectangle{\pgfqpoint{1.254980in}{0.150000in}}{\pgfqpoint{5.490039in}{5.490039in}}%
\pgfusepath{clip}%
\pgfsetbuttcap%
\pgfsetroundjoin%
\definecolor{currentfill}{rgb}{0.146616,0.673050,0.508936}%
\pgfsetfillcolor{currentfill}%
\pgfsetfillopacity{0.700000}%
\pgfsetlinewidth{0.000000pt}%
\definecolor{currentstroke}{rgb}{0.000000,0.000000,0.000000}%
\pgfsetstrokecolor{currentstroke}%
\pgfsetdash{}{0pt}%
\pgfpathmoveto{\pgfqpoint{4.999889in}{2.614222in}}%
\pgfpathlineto{\pgfqpoint{5.014283in}{2.628838in}}%
\pgfpathlineto{\pgfqpoint{5.028696in}{2.643620in}}%
\pgfpathlineto{\pgfqpoint{5.043130in}{2.658568in}}%
\pgfpathlineto{\pgfqpoint{5.057584in}{2.673681in}}%
\pgfpathlineto{\pgfqpoint{5.065327in}{2.687907in}}%
\pgfpathlineto{\pgfqpoint{5.073062in}{2.701954in}}%
\pgfpathlineto{\pgfqpoint{5.080789in}{2.715820in}}%
\pgfpathlineto{\pgfqpoint{5.088509in}{2.729504in}}%
\pgfpathlineto{\pgfqpoint{5.074049in}{2.714216in}}%
\pgfpathlineto{\pgfqpoint{5.059609in}{2.699094in}}%
\pgfpathlineto{\pgfqpoint{5.045189in}{2.684139in}}%
\pgfpathlineto{\pgfqpoint{5.030790in}{2.669349in}}%
\pgfpathlineto{\pgfqpoint{5.023075in}{2.655827in}}%
\pgfpathlineto{\pgfqpoint{5.015354in}{2.642131in}}%
\pgfpathlineto{\pgfqpoint{5.007625in}{2.628263in}}%
\pgfpathlineto{\pgfqpoint{4.999889in}{2.614222in}}%
\pgfpathclose%
\pgfusepath{fill}%
\end{pgfscope}%
\begin{pgfscope}%
\pgfpathrectangle{\pgfqpoint{1.254980in}{0.150000in}}{\pgfqpoint{5.490039in}{5.490039in}}%
\pgfusepath{clip}%
\pgfsetbuttcap%
\pgfsetroundjoin%
\definecolor{currentfill}{rgb}{0.477504,0.821444,0.318195}%
\pgfsetfillcolor{currentfill}%
\pgfsetfillopacity{0.700000}%
\pgfsetlinewidth{0.000000pt}%
\definecolor{currentstroke}{rgb}{0.000000,0.000000,0.000000}%
\pgfsetstrokecolor{currentstroke}%
\pgfsetdash{}{0pt}%
\pgfpathmoveto{\pgfqpoint{5.357710in}{3.095133in}}%
\pgfpathlineto{\pgfqpoint{5.372378in}{3.112327in}}%
\pgfpathlineto{\pgfqpoint{5.387069in}{3.129690in}}%
\pgfpathlineto{\pgfqpoint{5.401782in}{3.147224in}}%
\pgfpathlineto{\pgfqpoint{5.416519in}{3.164927in}}%
\pgfpathlineto{\pgfqpoint{5.424054in}{3.174611in}}%
\pgfpathlineto{\pgfqpoint{5.431578in}{3.184081in}}%
\pgfpathlineto{\pgfqpoint{5.439091in}{3.193337in}}%
\pgfpathlineto{\pgfqpoint{5.446593in}{3.202379in}}%
\pgfpathlineto{\pgfqpoint{5.431859in}{3.184695in}}%
\pgfpathlineto{\pgfqpoint{5.417147in}{3.167180in}}%
\pgfpathlineto{\pgfqpoint{5.402458in}{3.149836in}}%
\pgfpathlineto{\pgfqpoint{5.387792in}{3.132660in}}%
\pgfpathlineto{\pgfqpoint{5.380287in}{3.123587in}}%
\pgfpathlineto{\pgfqpoint{5.372772in}{3.114308in}}%
\pgfpathlineto{\pgfqpoint{5.365246in}{3.104824in}}%
\pgfpathlineto{\pgfqpoint{5.357710in}{3.095133in}}%
\pgfpathclose%
\pgfusepath{fill}%
\end{pgfscope}%
\begin{pgfscope}%
\pgfpathrectangle{\pgfqpoint{1.254980in}{0.150000in}}{\pgfqpoint{5.490039in}{5.490039in}}%
\pgfusepath{clip}%
\pgfsetbuttcap%
\pgfsetroundjoin%
\definecolor{currentfill}{rgb}{0.232815,0.732247,0.459277}%
\pgfsetfillcolor{currentfill}%
\pgfsetfillopacity{0.700000}%
\pgfsetlinewidth{0.000000pt}%
\definecolor{currentstroke}{rgb}{0.000000,0.000000,0.000000}%
\pgfsetstrokecolor{currentstroke}%
\pgfsetdash{}{0pt}%
\pgfpathmoveto{\pgfqpoint{5.119310in}{2.782394in}}%
\pgfpathlineto{\pgfqpoint{5.133796in}{2.797992in}}%
\pgfpathlineto{\pgfqpoint{5.148304in}{2.813756in}}%
\pgfpathlineto{\pgfqpoint{5.162832in}{2.829688in}}%
\pgfpathlineto{\pgfqpoint{5.177382in}{2.845787in}}%
\pgfpathlineto{\pgfqpoint{5.185067in}{2.858667in}}%
\pgfpathlineto{\pgfqpoint{5.192743in}{2.871350in}}%
\pgfpathlineto{\pgfqpoint{5.200410in}{2.883836in}}%
\pgfpathlineto{\pgfqpoint{5.208069in}{2.896125in}}%
\pgfpathlineto{\pgfqpoint{5.193515in}{2.879914in}}%
\pgfpathlineto{\pgfqpoint{5.178983in}{2.863872in}}%
\pgfpathlineto{\pgfqpoint{5.164472in}{2.847996in}}%
\pgfpathlineto{\pgfqpoint{5.149981in}{2.832288in}}%
\pgfpathlineto{\pgfqpoint{5.142326in}{2.820098in}}%
\pgfpathlineto{\pgfqpoint{5.134662in}{2.807719in}}%
\pgfpathlineto{\pgfqpoint{5.126990in}{2.795151in}}%
\pgfpathlineto{\pgfqpoint{5.119310in}{2.782394in}}%
\pgfpathclose%
\pgfusepath{fill}%
\end{pgfscope}%
\begin{pgfscope}%
\pgfpathrectangle{\pgfqpoint{1.254980in}{0.150000in}}{\pgfqpoint{5.490039in}{5.490039in}}%
\pgfusepath{clip}%
\pgfsetbuttcap%
\pgfsetroundjoin%
\definecolor{currentfill}{rgb}{0.141935,0.526453,0.555991}%
\pgfsetfillcolor{currentfill}%
\pgfsetfillopacity{0.700000}%
\pgfsetlinewidth{0.000000pt}%
\definecolor{currentstroke}{rgb}{0.000000,0.000000,0.000000}%
\pgfsetstrokecolor{currentstroke}%
\pgfsetdash{}{0pt}%
\pgfpathmoveto{\pgfqpoint{4.729689in}{2.199784in}}%
\pgfpathlineto{\pgfqpoint{4.743892in}{2.211765in}}%
\pgfpathlineto{\pgfqpoint{4.758113in}{2.223908in}}%
\pgfpathlineto{\pgfqpoint{4.772352in}{2.236213in}}%
\pgfpathlineto{\pgfqpoint{4.786608in}{2.248679in}}%
\pgfpathlineto{\pgfqpoint{4.794456in}{2.265406in}}%
\pgfpathlineto{\pgfqpoint{4.802300in}{2.282013in}}%
\pgfpathlineto{\pgfqpoint{4.810138in}{2.298496in}}%
\pgfpathlineto{\pgfqpoint{4.817971in}{2.314853in}}%
\pgfpathlineto{\pgfqpoint{4.803705in}{2.302060in}}%
\pgfpathlineto{\pgfqpoint{4.789457in}{2.289430in}}%
\pgfpathlineto{\pgfqpoint{4.775227in}{2.276962in}}%
\pgfpathlineto{\pgfqpoint{4.761015in}{2.264656in}}%
\pgfpathlineto{\pgfqpoint{4.753191in}{2.248613in}}%
\pgfpathlineto{\pgfqpoint{4.745362in}{2.232452in}}%
\pgfpathlineto{\pgfqpoint{4.737528in}{2.216174in}}%
\pgfpathlineto{\pgfqpoint{4.729689in}{2.199784in}}%
\pgfpathclose%
\pgfusepath{fill}%
\end{pgfscope}%
\begin{pgfscope}%
\pgfpathrectangle{\pgfqpoint{1.254980in}{0.150000in}}{\pgfqpoint{5.490039in}{5.490039in}}%
\pgfusepath{clip}%
\pgfsetbuttcap%
\pgfsetroundjoin%
\definecolor{currentfill}{rgb}{0.206756,0.371758,0.553117}%
\pgfsetfillcolor{currentfill}%
\pgfsetfillopacity{0.700000}%
\pgfsetlinewidth{0.000000pt}%
\definecolor{currentstroke}{rgb}{0.000000,0.000000,0.000000}%
\pgfsetstrokecolor{currentstroke}%
\pgfsetdash{}{0pt}%
\pgfpathmoveto{\pgfqpoint{4.459330in}{1.779734in}}%
\pgfpathlineto{\pgfqpoint{4.473360in}{1.788386in}}%
\pgfpathlineto{\pgfqpoint{4.487406in}{1.797194in}}%
\pgfpathlineto{\pgfqpoint{4.501467in}{1.806161in}}%
\pgfpathlineto{\pgfqpoint{4.515543in}{1.815284in}}%
\pgfpathlineto{\pgfqpoint{4.523454in}{1.832734in}}%
\pgfpathlineto{\pgfqpoint{4.531363in}{1.850151in}}%
\pgfpathlineto{\pgfqpoint{4.539267in}{1.867530in}}%
\pgfpathlineto{\pgfqpoint{4.547168in}{1.884869in}}%
\pgfpathlineto{\pgfqpoint{4.533084in}{1.875276in}}%
\pgfpathlineto{\pgfqpoint{4.519015in}{1.865842in}}%
\pgfpathlineto{\pgfqpoint{4.504962in}{1.856567in}}%
\pgfpathlineto{\pgfqpoint{4.490924in}{1.847449in}}%
\pgfpathlineto{\pgfqpoint{4.483031in}{1.830567in}}%
\pgfpathlineto{\pgfqpoint{4.475134in}{1.813651in}}%
\pgfpathlineto{\pgfqpoint{4.467234in}{1.796705in}}%
\pgfpathlineto{\pgfqpoint{4.459330in}{1.779734in}}%
\pgfpathclose%
\pgfusepath{fill}%
\end{pgfscope}%
\begin{pgfscope}%
\pgfpathrectangle{\pgfqpoint{1.254980in}{0.150000in}}{\pgfqpoint{5.490039in}{5.490039in}}%
\pgfusepath{clip}%
\pgfsetbuttcap%
\pgfsetroundjoin%
\definecolor{currentfill}{rgb}{0.352360,0.783011,0.392636}%
\pgfsetfillcolor{currentfill}%
\pgfsetfillopacity{0.700000}%
\pgfsetlinewidth{0.000000pt}%
\definecolor{currentstroke}{rgb}{0.000000,0.000000,0.000000}%
\pgfsetstrokecolor{currentstroke}%
\pgfsetdash{}{0pt}%
\pgfpathmoveto{\pgfqpoint{5.238615in}{2.943293in}}%
\pgfpathlineto{\pgfqpoint{5.253193in}{2.959751in}}%
\pgfpathlineto{\pgfqpoint{5.267793in}{2.976377in}}%
\pgfpathlineto{\pgfqpoint{5.282416in}{2.993172in}}%
\pgfpathlineto{\pgfqpoint{5.297060in}{3.010136in}}%
\pgfpathlineto{\pgfqpoint{5.304676in}{3.021488in}}%
\pgfpathlineto{\pgfqpoint{5.312282in}{3.032633in}}%
\pgfpathlineto{\pgfqpoint{5.319878in}{3.043569in}}%
\pgfpathlineto{\pgfqpoint{5.327464in}{3.054297in}}%
\pgfpathlineto{\pgfqpoint{5.312818in}{3.037287in}}%
\pgfpathlineto{\pgfqpoint{5.298195in}{3.020445in}}%
\pgfpathlineto{\pgfqpoint{5.283593in}{3.003772in}}%
\pgfpathlineto{\pgfqpoint{5.269014in}{2.987268in}}%
\pgfpathlineto{\pgfqpoint{5.261428in}{2.976574in}}%
\pgfpathlineto{\pgfqpoint{5.253833in}{2.965680in}}%
\pgfpathlineto{\pgfqpoint{5.246229in}{2.954587in}}%
\pgfpathlineto{\pgfqpoint{5.238615in}{2.943293in}}%
\pgfpathclose%
\pgfusepath{fill}%
\end{pgfscope}%
\begin{pgfscope}%
\pgfpathrectangle{\pgfqpoint{1.254980in}{0.150000in}}{\pgfqpoint{5.490039in}{5.490039in}}%
\pgfusepath{clip}%
\pgfsetbuttcap%
\pgfsetroundjoin%
\definecolor{currentfill}{rgb}{0.280255,0.165693,0.476498}%
\pgfsetfillcolor{currentfill}%
\pgfsetfillopacity{0.700000}%
\pgfsetlinewidth{0.000000pt}%
\definecolor{currentstroke}{rgb}{0.000000,0.000000,0.000000}%
\pgfsetstrokecolor{currentstroke}%
\pgfsetdash{}{0pt}%
\pgfpathmoveto{\pgfqpoint{4.101619in}{1.325912in}}%
\pgfpathlineto{\pgfqpoint{4.115475in}{1.329342in}}%
\pgfpathlineto{\pgfqpoint{4.129342in}{1.332924in}}%
\pgfpathlineto{\pgfqpoint{4.143220in}{1.336660in}}%
\pgfpathlineto{\pgfqpoint{4.157110in}{1.340548in}}%
\pgfpathlineto{\pgfqpoint{4.165090in}{1.355378in}}%
\pgfpathlineto{\pgfqpoint{4.173065in}{1.370321in}}%
\pgfpathlineto{\pgfqpoint{4.181037in}{1.385370in}}%
\pgfpathlineto{\pgfqpoint{4.189005in}{1.400520in}}%
\pgfpathlineto{\pgfqpoint{4.175116in}{1.396004in}}%
\pgfpathlineto{\pgfqpoint{4.161238in}{1.391642in}}%
\pgfpathlineto{\pgfqpoint{4.147372in}{1.387433in}}%
\pgfpathlineto{\pgfqpoint{4.133518in}{1.383378in}}%
\pgfpathlineto{\pgfqpoint{4.125549in}{1.368844in}}%
\pgfpathlineto{\pgfqpoint{4.117576in}{1.354418in}}%
\pgfpathlineto{\pgfqpoint{4.109600in}{1.340106in}}%
\pgfpathlineto{\pgfqpoint{4.101619in}{1.325912in}}%
\pgfpathclose%
\pgfusepath{fill}%
\end{pgfscope}%
\begin{pgfscope}%
\pgfpathrectangle{\pgfqpoint{1.254980in}{0.150000in}}{\pgfqpoint{5.490039in}{5.490039in}}%
\pgfusepath{clip}%
\pgfsetbuttcap%
\pgfsetroundjoin%
\definecolor{currentfill}{rgb}{0.175841,0.441290,0.557685}%
\pgfsetfillcolor{currentfill}%
\pgfsetfillopacity{0.700000}%
\pgfsetlinewidth{0.000000pt}%
\definecolor{currentstroke}{rgb}{0.000000,0.000000,0.000000}%
\pgfsetstrokecolor{currentstroke}%
\pgfsetdash{}{0pt}%
\pgfpathmoveto{\pgfqpoint{4.578736in}{1.953726in}}%
\pgfpathlineto{\pgfqpoint{4.592844in}{1.963917in}}%
\pgfpathlineto{\pgfqpoint{4.606969in}{1.974267in}}%
\pgfpathlineto{\pgfqpoint{4.621111in}{1.984776in}}%
\pgfpathlineto{\pgfqpoint{4.635269in}{1.995444in}}%
\pgfpathlineto{\pgfqpoint{4.643160in}{2.012938in}}%
\pgfpathlineto{\pgfqpoint{4.651048in}{2.030359in}}%
\pgfpathlineto{\pgfqpoint{4.658931in}{2.047704in}}%
\pgfpathlineto{\pgfqpoint{4.666811in}{2.064968in}}%
\pgfpathlineto{\pgfqpoint{4.652643in}{2.053886in}}%
\pgfpathlineto{\pgfqpoint{4.638493in}{2.042965in}}%
\pgfpathlineto{\pgfqpoint{4.624359in}{2.032203in}}%
\pgfpathlineto{\pgfqpoint{4.610241in}{2.021601in}}%
\pgfpathlineto{\pgfqpoint{4.602371in}{2.004737in}}%
\pgfpathlineto{\pgfqpoint{4.594496in}{1.987801in}}%
\pgfpathlineto{\pgfqpoint{4.586618in}{1.970796in}}%
\pgfpathlineto{\pgfqpoint{4.578736in}{1.953726in}}%
\pgfpathclose%
\pgfusepath{fill}%
\end{pgfscope}%
\begin{pgfscope}%
\pgfpathrectangle{\pgfqpoint{1.254980in}{0.150000in}}{\pgfqpoint{5.490039in}{5.490039in}}%
\pgfusepath{clip}%
\pgfsetbuttcap%
\pgfsetroundjoin%
\definecolor{currentfill}{rgb}{0.120565,0.596422,0.543611}%
\pgfsetfillcolor{currentfill}%
\pgfsetfillopacity{0.700000}%
\pgfsetlinewidth{0.000000pt}%
\definecolor{currentstroke}{rgb}{0.000000,0.000000,0.000000}%
\pgfsetstrokecolor{currentstroke}%
\pgfsetdash{}{0pt}%
\pgfpathmoveto{\pgfqpoint{4.849248in}{2.378966in}}%
\pgfpathlineto{\pgfqpoint{4.863541in}{2.392218in}}%
\pgfpathlineto{\pgfqpoint{4.877854in}{2.405634in}}%
\pgfpathlineto{\pgfqpoint{4.892185in}{2.419213in}}%
\pgfpathlineto{\pgfqpoint{4.906535in}{2.432956in}}%
\pgfpathlineto{\pgfqpoint{4.914349in}{2.448921in}}%
\pgfpathlineto{\pgfqpoint{4.922157in}{2.464736in}}%
\pgfpathlineto{\pgfqpoint{4.929959in}{2.480400in}}%
\pgfpathlineto{\pgfqpoint{4.937755in}{2.495911in}}%
\pgfpathlineto{\pgfqpoint{4.923396in}{2.481901in}}%
\pgfpathlineto{\pgfqpoint{4.909057in}{2.468055in}}%
\pgfpathlineto{\pgfqpoint{4.894736in}{2.454374in}}%
\pgfpathlineto{\pgfqpoint{4.880435in}{2.440856in}}%
\pgfpathlineto{\pgfqpoint{4.872647in}{2.425600in}}%
\pgfpathlineto{\pgfqpoint{4.864853in}{2.410198in}}%
\pgfpathlineto{\pgfqpoint{4.857053in}{2.394653in}}%
\pgfpathlineto{\pgfqpoint{4.849248in}{2.378966in}}%
\pgfpathclose%
\pgfusepath{fill}%
\end{pgfscope}%
\begin{pgfscope}%
\pgfpathrectangle{\pgfqpoint{1.254980in}{0.150000in}}{\pgfqpoint{5.490039in}{5.490039in}}%
\pgfusepath{clip}%
\pgfsetbuttcap%
\pgfsetroundjoin%
\definecolor{currentfill}{rgb}{0.281446,0.084320,0.407414}%
\pgfsetfillcolor{currentfill}%
\pgfsetfillopacity{0.700000}%
\pgfsetlinewidth{0.000000pt}%
\definecolor{currentstroke}{rgb}{0.000000,0.000000,0.000000}%
\pgfsetstrokecolor{currentstroke}%
\pgfsetdash{}{0pt}%
\pgfpathmoveto{\pgfqpoint{3.894867in}{1.161554in}}%
\pgfpathlineto{\pgfqpoint{3.908664in}{1.161726in}}%
\pgfpathlineto{\pgfqpoint{3.922470in}{1.162051in}}%
\pgfpathlineto{\pgfqpoint{3.936284in}{1.162529in}}%
\pgfpathlineto{\pgfqpoint{3.950108in}{1.163158in}}%
\pgfpathlineto{\pgfqpoint{3.958153in}{1.174706in}}%
\pgfpathlineto{\pgfqpoint{3.966193in}{1.186459in}}%
\pgfpathlineto{\pgfqpoint{3.974227in}{1.198408in}}%
\pgfpathlineto{\pgfqpoint{3.982256in}{1.210548in}}%
\pgfpathlineto{\pgfqpoint{3.968441in}{1.209213in}}%
\pgfpathlineto{\pgfqpoint{3.954635in}{1.208031in}}%
\pgfpathlineto{\pgfqpoint{3.940839in}{1.207001in}}%
\pgfpathlineto{\pgfqpoint{3.927051in}{1.206124in}}%
\pgfpathlineto{\pgfqpoint{3.919014in}{1.194679in}}%
\pgfpathlineto{\pgfqpoint{3.910971in}{1.183430in}}%
\pgfpathlineto{\pgfqpoint{3.902922in}{1.172387in}}%
\pgfpathlineto{\pgfqpoint{3.894867in}{1.161554in}}%
\pgfpathclose%
\pgfusepath{fill}%
\end{pgfscope}%
\begin{pgfscope}%
\pgfpathrectangle{\pgfqpoint{1.254980in}{0.150000in}}{\pgfqpoint{5.490039in}{5.490039in}}%
\pgfusepath{clip}%
\pgfsetbuttcap%
\pgfsetroundjoin%
\definecolor{currentfill}{rgb}{0.248629,0.278775,0.534556}%
\pgfsetfillcolor{currentfill}%
\pgfsetfillopacity{0.700000}%
\pgfsetlinewidth{0.000000pt}%
\definecolor{currentstroke}{rgb}{0.000000,0.000000,0.000000}%
\pgfsetstrokecolor{currentstroke}%
\pgfsetdash{}{0pt}%
\pgfpathmoveto{\pgfqpoint{4.308314in}{1.548959in}}%
\pgfpathlineto{\pgfqpoint{4.322271in}{1.555422in}}%
\pgfpathlineto{\pgfqpoint{4.336241in}{1.562039in}}%
\pgfpathlineto{\pgfqpoint{4.350224in}{1.568812in}}%
\pgfpathlineto{\pgfqpoint{4.364221in}{1.575740in}}%
\pgfpathlineto{\pgfqpoint{4.372165in}{1.592657in}}%
\pgfpathlineto{\pgfqpoint{4.380105in}{1.609607in}}%
\pgfpathlineto{\pgfqpoint{4.388042in}{1.626582in}}%
\pgfpathlineto{\pgfqpoint{4.395976in}{1.643579in}}%
\pgfpathlineto{\pgfqpoint{4.381974in}{1.636101in}}%
\pgfpathlineto{\pgfqpoint{4.367985in}{1.628779in}}%
\pgfpathlineto{\pgfqpoint{4.354010in}{1.621612in}}%
\pgfpathlineto{\pgfqpoint{4.340048in}{1.614602in}}%
\pgfpathlineto{\pgfqpoint{4.332120in}{1.598143in}}%
\pgfpathlineto{\pgfqpoint{4.324188in}{1.581713in}}%
\pgfpathlineto{\pgfqpoint{4.316253in}{1.565317in}}%
\pgfpathlineto{\pgfqpoint{4.308314in}{1.548959in}}%
\pgfpathclose%
\pgfusepath{fill}%
\end{pgfscope}%
\begin{pgfscope}%
\pgfpathrectangle{\pgfqpoint{1.254980in}{0.150000in}}{\pgfqpoint{5.490039in}{5.490039in}}%
\pgfusepath{clip}%
\pgfsetbuttcap%
\pgfsetroundjoin%
\definecolor{currentfill}{rgb}{0.585678,0.846661,0.249897}%
\pgfsetfillcolor{currentfill}%
\pgfsetfillopacity{0.700000}%
\pgfsetlinewidth{0.000000pt}%
\definecolor{currentstroke}{rgb}{0.000000,0.000000,0.000000}%
\pgfsetstrokecolor{currentstroke}%
\pgfsetdash{}{0pt}%
\pgfpathmoveto{\pgfqpoint{5.446593in}{3.202379in}}%
\pgfpathlineto{\pgfqpoint{5.461351in}{3.220234in}}%
\pgfpathlineto{\pgfqpoint{5.476133in}{3.238260in}}%
\pgfpathlineto{\pgfqpoint{5.490937in}{3.256456in}}%
\pgfpathlineto{\pgfqpoint{5.498427in}{3.265256in}}%
\pgfpathlineto{\pgfqpoint{5.505905in}{3.273838in}}%
\pgfpathlineto{\pgfqpoint{5.513371in}{3.282203in}}%
\pgfpathlineto{\pgfqpoint{5.520827in}{3.290352in}}%
\pgfpathlineto{\pgfqpoint{5.506025in}{3.272208in}}%
\pgfpathlineto{\pgfqpoint{5.491248in}{3.254235in}}%
\pgfpathlineto{\pgfqpoint{5.476493in}{3.236432in}}%
\pgfpathlineto{\pgfqpoint{5.469035in}{3.228234in}}%
\pgfpathlineto{\pgfqpoint{5.461565in}{3.219827in}}%
\pgfpathlineto{\pgfqpoint{5.454085in}{3.211209in}}%
\pgfpathlineto{\pgfqpoint{5.446593in}{3.202379in}}%
\pgfpathclose%
\pgfusepath{fill}%
\end{pgfscope}%
\begin{pgfscope}%
\pgfpathrectangle{\pgfqpoint{1.254980in}{0.150000in}}{\pgfqpoint{5.490039in}{5.490039in}}%
\pgfusepath{clip}%
\pgfsetbuttcap%
\pgfsetroundjoin%
\definecolor{currentfill}{rgb}{0.271828,0.209303,0.504434}%
\pgfsetfillcolor{currentfill}%
\pgfsetfillopacity{0.700000}%
\pgfsetlinewidth{0.000000pt}%
\definecolor{currentstroke}{rgb}{0.000000,0.000000,0.000000}%
\pgfsetstrokecolor{currentstroke}%
\pgfsetdash{}{0pt}%
\pgfpathmoveto{\pgfqpoint{4.189005in}{1.400520in}}%
\pgfpathlineto{\pgfqpoint{4.202907in}{1.405190in}}%
\pgfpathlineto{\pgfqpoint{4.216820in}{1.410013in}}%
\pgfpathlineto{\pgfqpoint{4.230745in}{1.414989in}}%
\pgfpathlineto{\pgfqpoint{4.244682in}{1.420119in}}%
\pgfpathlineto{\pgfqpoint{4.252648in}{1.435976in}}%
\pgfpathlineto{\pgfqpoint{4.260611in}{1.451915in}}%
\pgfpathlineto{\pgfqpoint{4.268570in}{1.467930in}}%
\pgfpathlineto{\pgfqpoint{4.276525in}{1.484016in}}%
\pgfpathlineto{\pgfqpoint{4.262585in}{1.478284in}}%
\pgfpathlineto{\pgfqpoint{4.248658in}{1.472705in}}%
\pgfpathlineto{\pgfqpoint{4.234743in}{1.467281in}}%
\pgfpathlineto{\pgfqpoint{4.220841in}{1.462010in}}%
\pgfpathlineto{\pgfqpoint{4.212887in}{1.446516in}}%
\pgfpathlineto{\pgfqpoint{4.204930in}{1.431099in}}%
\pgfpathlineto{\pgfqpoint{4.196970in}{1.415765in}}%
\pgfpathlineto{\pgfqpoint{4.189005in}{1.400520in}}%
\pgfpathclose%
\pgfusepath{fill}%
\end{pgfscope}%
\begin{pgfscope}%
\pgfpathrectangle{\pgfqpoint{1.254980in}{0.150000in}}{\pgfqpoint{5.490039in}{5.490039in}}%
\pgfusepath{clip}%
\pgfsetbuttcap%
\pgfsetroundjoin%
\definecolor{currentfill}{rgb}{0.283091,0.110553,0.431554}%
\pgfsetfillcolor{currentfill}%
\pgfsetfillopacity{0.700000}%
\pgfsetlinewidth{0.000000pt}%
\definecolor{currentstroke}{rgb}{0.000000,0.000000,0.000000}%
\pgfsetstrokecolor{currentstroke}%
\pgfsetdash{}{0pt}%
\pgfpathmoveto{\pgfqpoint{3.982256in}{1.210548in}}%
\pgfpathlineto{\pgfqpoint{3.996080in}{1.212035in}}%
\pgfpathlineto{\pgfqpoint{4.009914in}{1.213675in}}%
\pgfpathlineto{\pgfqpoint{4.023758in}{1.215467in}}%
\pgfpathlineto{\pgfqpoint{4.037611in}{1.217411in}}%
\pgfpathlineto{\pgfqpoint{4.045628in}{1.230424in}}%
\pgfpathlineto{\pgfqpoint{4.053641in}{1.243607in}}%
\pgfpathlineto{\pgfqpoint{4.061648in}{1.256953in}}%
\pgfpathlineto{\pgfqpoint{4.069651in}{1.270456in}}%
\pgfpathlineto{\pgfqpoint{4.055803in}{1.267833in}}%
\pgfpathlineto{\pgfqpoint{4.041964in}{1.265361in}}%
\pgfpathlineto{\pgfqpoint{4.028136in}{1.263043in}}%
\pgfpathlineto{\pgfqpoint{4.014318in}{1.260877in}}%
\pgfpathlineto{\pgfqpoint{4.006310in}{1.248043in}}%
\pgfpathlineto{\pgfqpoint{3.998297in}{1.235372in}}%
\pgfpathlineto{\pgfqpoint{3.990279in}{1.222871in}}%
\pgfpathlineto{\pgfqpoint{3.982256in}{1.210548in}}%
\pgfpathclose%
\pgfusepath{fill}%
\end{pgfscope}%
\begin{pgfscope}%
\pgfpathrectangle{\pgfqpoint{1.254980in}{0.150000in}}{\pgfqpoint{5.490039in}{5.490039in}}%
\pgfusepath{clip}%
\pgfsetbuttcap%
\pgfsetroundjoin%
\definecolor{currentfill}{rgb}{0.137339,0.662252,0.515571}%
\pgfsetfillcolor{currentfill}%
\pgfsetfillopacity{0.700000}%
\pgfsetlinewidth{0.000000pt}%
\definecolor{currentstroke}{rgb}{0.000000,0.000000,0.000000}%
\pgfsetstrokecolor{currentstroke}%
\pgfsetdash{}{0pt}%
\pgfpathmoveto{\pgfqpoint{4.968876in}{2.556375in}}%
\pgfpathlineto{\pgfqpoint{4.983262in}{2.570786in}}%
\pgfpathlineto{\pgfqpoint{4.997669in}{2.585362in}}%
\pgfpathlineto{\pgfqpoint{5.012095in}{2.600104in}}%
\pgfpathlineto{\pgfqpoint{5.026541in}{2.615011in}}%
\pgfpathlineto{\pgfqpoint{5.034313in}{2.629940in}}%
\pgfpathlineto{\pgfqpoint{5.042077in}{2.644696in}}%
\pgfpathlineto{\pgfqpoint{5.049834in}{2.659277in}}%
\pgfpathlineto{\pgfqpoint{5.057584in}{2.673681in}}%
\pgfpathlineto{\pgfqpoint{5.043130in}{2.658568in}}%
\pgfpathlineto{\pgfqpoint{5.028696in}{2.643620in}}%
\pgfpathlineto{\pgfqpoint{5.014283in}{2.628838in}}%
\pgfpathlineto{\pgfqpoint{4.999889in}{2.614222in}}%
\pgfpathlineto{\pgfqpoint{4.992146in}{2.600012in}}%
\pgfpathlineto{\pgfqpoint{4.984396in}{2.585632in}}%
\pgfpathlineto{\pgfqpoint{4.976639in}{2.571086in}}%
\pgfpathlineto{\pgfqpoint{4.968876in}{2.556375in}}%
\pgfpathclose%
\pgfusepath{fill}%
\end{pgfscope}%
\begin{pgfscope}%
\pgfpathrectangle{\pgfqpoint{1.254980in}{0.150000in}}{\pgfqpoint{5.490039in}{5.490039in}}%
\pgfusepath{clip}%
\pgfsetbuttcap%
\pgfsetroundjoin%
\definecolor{currentfill}{rgb}{0.216210,0.351535,0.550627}%
\pgfsetfillcolor{currentfill}%
\pgfsetfillopacity{0.700000}%
\pgfsetlinewidth{0.000000pt}%
\definecolor{currentstroke}{rgb}{0.000000,0.000000,0.000000}%
\pgfsetstrokecolor{currentstroke}%
\pgfsetdash{}{0pt}%
\pgfpathmoveto{\pgfqpoint{4.427680in}{1.711680in}}%
\pgfpathlineto{\pgfqpoint{4.441703in}{1.719837in}}%
\pgfpathlineto{\pgfqpoint{4.455742in}{1.728150in}}%
\pgfpathlineto{\pgfqpoint{4.469795in}{1.736620in}}%
\pgfpathlineto{\pgfqpoint{4.483863in}{1.745247in}}%
\pgfpathlineto{\pgfqpoint{4.491788in}{1.762783in}}%
\pgfpathlineto{\pgfqpoint{4.499709in}{1.780304in}}%
\pgfpathlineto{\pgfqpoint{4.507628in}{1.797806in}}%
\pgfpathlineto{\pgfqpoint{4.515543in}{1.815284in}}%
\pgfpathlineto{\pgfqpoint{4.501467in}{1.806161in}}%
\pgfpathlineto{\pgfqpoint{4.487406in}{1.797194in}}%
\pgfpathlineto{\pgfqpoint{4.473360in}{1.788386in}}%
\pgfpathlineto{\pgfqpoint{4.459330in}{1.779734in}}%
\pgfpathlineto{\pgfqpoint{4.451422in}{1.762741in}}%
\pgfpathlineto{\pgfqpoint{4.443512in}{1.745732in}}%
\pgfpathlineto{\pgfqpoint{4.435597in}{1.728710in}}%
\pgfpathlineto{\pgfqpoint{4.427680in}{1.711680in}}%
\pgfpathclose%
\pgfusepath{fill}%
\end{pgfscope}%
\begin{pgfscope}%
\pgfpathrectangle{\pgfqpoint{1.254980in}{0.150000in}}{\pgfqpoint{5.490039in}{5.490039in}}%
\pgfusepath{clip}%
\pgfsetbuttcap%
\pgfsetroundjoin%
\definecolor{currentfill}{rgb}{0.149039,0.508051,0.557250}%
\pgfsetfillcolor{currentfill}%
\pgfsetfillopacity{0.700000}%
\pgfsetlinewidth{0.000000pt}%
\definecolor{currentstroke}{rgb}{0.000000,0.000000,0.000000}%
\pgfsetstrokecolor{currentstroke}%
\pgfsetdash{}{0pt}%
\pgfpathmoveto{\pgfqpoint{4.698286in}{2.133155in}}%
\pgfpathlineto{\pgfqpoint{4.712480in}{2.144781in}}%
\pgfpathlineto{\pgfqpoint{4.726691in}{2.156569in}}%
\pgfpathlineto{\pgfqpoint{4.740920in}{2.168517in}}%
\pgfpathlineto{\pgfqpoint{4.755167in}{2.180627in}}%
\pgfpathlineto{\pgfqpoint{4.763034in}{2.197805in}}%
\pgfpathlineto{\pgfqpoint{4.770897in}{2.214876in}}%
\pgfpathlineto{\pgfqpoint{4.778755in}{2.231835in}}%
\pgfpathlineto{\pgfqpoint{4.786608in}{2.248679in}}%
\pgfpathlineto{\pgfqpoint{4.772352in}{2.236213in}}%
\pgfpathlineto{\pgfqpoint{4.758113in}{2.223908in}}%
\pgfpathlineto{\pgfqpoint{4.743892in}{2.211765in}}%
\pgfpathlineto{\pgfqpoint{4.729689in}{2.199784in}}%
\pgfpathlineto{\pgfqpoint{4.721845in}{2.183284in}}%
\pgfpathlineto{\pgfqpoint{4.713997in}{2.166677in}}%
\pgfpathlineto{\pgfqpoint{4.706143in}{2.149966in}}%
\pgfpathlineto{\pgfqpoint{4.698286in}{2.133155in}}%
\pgfpathclose%
\pgfusepath{fill}%
\end{pgfscope}%
\begin{pgfscope}%
\pgfpathrectangle{\pgfqpoint{1.254980in}{0.150000in}}{\pgfqpoint{5.490039in}{5.490039in}}%
\pgfusepath{clip}%
\pgfsetbuttcap%
\pgfsetroundjoin%
\definecolor{currentfill}{rgb}{0.214000,0.722114,0.469588}%
\pgfsetfillcolor{currentfill}%
\pgfsetfillopacity{0.700000}%
\pgfsetlinewidth{0.000000pt}%
\definecolor{currentstroke}{rgb}{0.000000,0.000000,0.000000}%
\pgfsetstrokecolor{currentstroke}%
\pgfsetdash{}{0pt}%
\pgfpathmoveto{\pgfqpoint{5.088509in}{2.729504in}}%
\pgfpathlineto{\pgfqpoint{5.102990in}{2.744958in}}%
\pgfpathlineto{\pgfqpoint{5.117492in}{2.760580in}}%
\pgfpathlineto{\pgfqpoint{5.132015in}{2.776368in}}%
\pgfpathlineto{\pgfqpoint{5.146559in}{2.792324in}}%
\pgfpathlineto{\pgfqpoint{5.154277in}{2.805980in}}%
\pgfpathlineto{\pgfqpoint{5.161987in}{2.819443in}}%
\pgfpathlineto{\pgfqpoint{5.169689in}{2.832712in}}%
\pgfpathlineto{\pgfqpoint{5.177382in}{2.845787in}}%
\pgfpathlineto{\pgfqpoint{5.162832in}{2.829688in}}%
\pgfpathlineto{\pgfqpoint{5.148304in}{2.813756in}}%
\pgfpathlineto{\pgfqpoint{5.133796in}{2.797992in}}%
\pgfpathlineto{\pgfqpoint{5.119310in}{2.782394in}}%
\pgfpathlineto{\pgfqpoint{5.111622in}{2.769450in}}%
\pgfpathlineto{\pgfqpoint{5.103925in}{2.756320in}}%
\pgfpathlineto{\pgfqpoint{5.096221in}{2.743004in}}%
\pgfpathlineto{\pgfqpoint{5.088509in}{2.729504in}}%
\pgfpathclose%
\pgfusepath{fill}%
\end{pgfscope}%
\begin{pgfscope}%
\pgfpathrectangle{\pgfqpoint{1.254980in}{0.150000in}}{\pgfqpoint{5.490039in}{5.490039in}}%
\pgfusepath{clip}%
\pgfsetbuttcap%
\pgfsetroundjoin%
\definecolor{currentfill}{rgb}{0.282290,0.145912,0.461510}%
\pgfsetfillcolor{currentfill}%
\pgfsetfillopacity{0.700000}%
\pgfsetlinewidth{0.000000pt}%
\definecolor{currentstroke}{rgb}{0.000000,0.000000,0.000000}%
\pgfsetstrokecolor{currentstroke}%
\pgfsetdash{}{0pt}%
\pgfpathmoveto{\pgfqpoint{4.069651in}{1.270456in}}%
\pgfpathlineto{\pgfqpoint{4.083510in}{1.273233in}}%
\pgfpathlineto{\pgfqpoint{4.097380in}{1.276162in}}%
\pgfpathlineto{\pgfqpoint{4.111260in}{1.279243in}}%
\pgfpathlineto{\pgfqpoint{4.125151in}{1.282477in}}%
\pgfpathlineto{\pgfqpoint{4.133147in}{1.296795in}}%
\pgfpathlineto{\pgfqpoint{4.141139in}{1.311251in}}%
\pgfpathlineto{\pgfqpoint{4.149126in}{1.325837in}}%
\pgfpathlineto{\pgfqpoint{4.157110in}{1.340548in}}%
\pgfpathlineto{\pgfqpoint{4.143220in}{1.336660in}}%
\pgfpathlineto{\pgfqpoint{4.129342in}{1.332924in}}%
\pgfpathlineto{\pgfqpoint{4.115475in}{1.329342in}}%
\pgfpathlineto{\pgfqpoint{4.101619in}{1.325912in}}%
\pgfpathlineto{\pgfqpoint{4.093633in}{1.311845in}}%
\pgfpathlineto{\pgfqpoint{4.085644in}{1.297909in}}%
\pgfpathlineto{\pgfqpoint{4.077650in}{1.284110in}}%
\pgfpathlineto{\pgfqpoint{4.069651in}{1.270456in}}%
\pgfpathclose%
\pgfusepath{fill}%
\end{pgfscope}%
\begin{pgfscope}%
\pgfpathrectangle{\pgfqpoint{1.254980in}{0.150000in}}{\pgfqpoint{5.490039in}{5.490039in}}%
\pgfusepath{clip}%
\pgfsetbuttcap%
\pgfsetroundjoin%
\definecolor{currentfill}{rgb}{0.185556,0.418570,0.556753}%
\pgfsetfillcolor{currentfill}%
\pgfsetfillopacity{0.700000}%
\pgfsetlinewidth{0.000000pt}%
\definecolor{currentstroke}{rgb}{0.000000,0.000000,0.000000}%
\pgfsetstrokecolor{currentstroke}%
\pgfsetdash{}{0pt}%
\pgfpathmoveto{\pgfqpoint{4.547168in}{1.884869in}}%
\pgfpathlineto{\pgfqpoint{4.561268in}{1.894619in}}%
\pgfpathlineto{\pgfqpoint{4.575384in}{1.904528in}}%
\pgfpathlineto{\pgfqpoint{4.589517in}{1.914595in}}%
\pgfpathlineto{\pgfqpoint{4.603665in}{1.924821in}}%
\pgfpathlineto{\pgfqpoint{4.611572in}{1.942566in}}%
\pgfpathlineto{\pgfqpoint{4.619474in}{1.960254in}}%
\pgfpathlineto{\pgfqpoint{4.627374in}{1.977882in}}%
\pgfpathlineto{\pgfqpoint{4.635269in}{1.995444in}}%
\pgfpathlineto{\pgfqpoint{4.621111in}{1.984776in}}%
\pgfpathlineto{\pgfqpoint{4.606969in}{1.974267in}}%
\pgfpathlineto{\pgfqpoint{4.592844in}{1.963917in}}%
\pgfpathlineto{\pgfqpoint{4.578736in}{1.953726in}}%
\pgfpathlineto{\pgfqpoint{4.570849in}{1.936594in}}%
\pgfpathlineto{\pgfqpoint{4.562959in}{1.919405in}}%
\pgfpathlineto{\pgfqpoint{4.555066in}{1.902161in}}%
\pgfpathlineto{\pgfqpoint{4.547168in}{1.884869in}}%
\pgfpathclose%
\pgfusepath{fill}%
\end{pgfscope}%
\begin{pgfscope}%
\pgfpathrectangle{\pgfqpoint{1.254980in}{0.150000in}}{\pgfqpoint{5.490039in}{5.490039in}}%
\pgfusepath{clip}%
\pgfsetbuttcap%
\pgfsetroundjoin%
\definecolor{currentfill}{rgb}{0.468053,0.818921,0.323998}%
\pgfsetfillcolor{currentfill}%
\pgfsetfillopacity{0.700000}%
\pgfsetlinewidth{0.000000pt}%
\definecolor{currentstroke}{rgb}{0.000000,0.000000,0.000000}%
\pgfsetstrokecolor{currentstroke}%
\pgfsetdash{}{0pt}%
\pgfpathmoveto{\pgfqpoint{5.327464in}{3.054297in}}%
\pgfpathlineto{\pgfqpoint{5.342132in}{3.071477in}}%
\pgfpathlineto{\pgfqpoint{5.356823in}{3.088827in}}%
\pgfpathlineto{\pgfqpoint{5.371537in}{3.106346in}}%
\pgfpathlineto{\pgfqpoint{5.386273in}{3.124036in}}%
\pgfpathlineto{\pgfqpoint{5.393850in}{3.134583in}}%
\pgfpathlineto{\pgfqpoint{5.401417in}{3.144913in}}%
\pgfpathlineto{\pgfqpoint{5.408973in}{3.155028in}}%
\pgfpathlineto{\pgfqpoint{5.416519in}{3.164927in}}%
\pgfpathlineto{\pgfqpoint{5.401782in}{3.147224in}}%
\pgfpathlineto{\pgfqpoint{5.387069in}{3.129690in}}%
\pgfpathlineto{\pgfqpoint{5.372378in}{3.112327in}}%
\pgfpathlineto{\pgfqpoint{5.357710in}{3.095133in}}%
\pgfpathlineto{\pgfqpoint{5.350164in}{3.085235in}}%
\pgfpathlineto{\pgfqpoint{5.342607in}{3.075130in}}%
\pgfpathlineto{\pgfqpoint{5.335041in}{3.064818in}}%
\pgfpathlineto{\pgfqpoint{5.327464in}{3.054297in}}%
\pgfpathclose%
\pgfusepath{fill}%
\end{pgfscope}%
\begin{pgfscope}%
\pgfpathrectangle{\pgfqpoint{1.254980in}{0.150000in}}{\pgfqpoint{5.490039in}{5.490039in}}%
\pgfusepath{clip}%
\pgfsetbuttcap%
\pgfsetroundjoin%
\definecolor{currentfill}{rgb}{0.327796,0.773980,0.406640}%
\pgfsetfillcolor{currentfill}%
\pgfsetfillopacity{0.700000}%
\pgfsetlinewidth{0.000000pt}%
\definecolor{currentstroke}{rgb}{0.000000,0.000000,0.000000}%
\pgfsetstrokecolor{currentstroke}%
\pgfsetdash{}{0pt}%
\pgfpathmoveto{\pgfqpoint{5.208069in}{2.896125in}}%
\pgfpathlineto{\pgfqpoint{5.222645in}{2.912504in}}%
\pgfpathlineto{\pgfqpoint{5.237242in}{2.929051in}}%
\pgfpathlineto{\pgfqpoint{5.251861in}{2.945766in}}%
\pgfpathlineto{\pgfqpoint{5.266502in}{2.962651in}}%
\pgfpathlineto{\pgfqpoint{5.274156in}{2.974833in}}%
\pgfpathlineto{\pgfqpoint{5.281800in}{2.986808in}}%
\pgfpathlineto{\pgfqpoint{5.289435in}{2.998576in}}%
\pgfpathlineto{\pgfqpoint{5.297060in}{3.010136in}}%
\pgfpathlineto{\pgfqpoint{5.282416in}{2.993172in}}%
\pgfpathlineto{\pgfqpoint{5.267793in}{2.976377in}}%
\pgfpathlineto{\pgfqpoint{5.253193in}{2.959751in}}%
\pgfpathlineto{\pgfqpoint{5.238615in}{2.943293in}}%
\pgfpathlineto{\pgfqpoint{5.230992in}{2.931800in}}%
\pgfpathlineto{\pgfqpoint{5.223360in}{2.920107in}}%
\pgfpathlineto{\pgfqpoint{5.215719in}{2.908216in}}%
\pgfpathlineto{\pgfqpoint{5.208069in}{2.896125in}}%
\pgfpathclose%
\pgfusepath{fill}%
\end{pgfscope}%
\begin{pgfscope}%
\pgfpathrectangle{\pgfqpoint{1.254980in}{0.150000in}}{\pgfqpoint{5.490039in}{5.490039in}}%
\pgfusepath{clip}%
\pgfsetbuttcap%
\pgfsetroundjoin%
\definecolor{currentfill}{rgb}{0.124395,0.578002,0.548287}%
\pgfsetfillcolor{currentfill}%
\pgfsetfillopacity{0.700000}%
\pgfsetlinewidth{0.000000pt}%
\definecolor{currentstroke}{rgb}{0.000000,0.000000,0.000000}%
\pgfsetstrokecolor{currentstroke}%
\pgfsetdash{}{0pt}%
\pgfpathmoveto{\pgfqpoint{4.817971in}{2.314853in}}%
\pgfpathlineto{\pgfqpoint{4.832255in}{2.327809in}}%
\pgfpathlineto{\pgfqpoint{4.846558in}{2.340927in}}%
\pgfpathlineto{\pgfqpoint{4.860880in}{2.354209in}}%
\pgfpathlineto{\pgfqpoint{4.875220in}{2.367653in}}%
\pgfpathlineto{\pgfqpoint{4.883057in}{2.384190in}}%
\pgfpathlineto{\pgfqpoint{4.890889in}{2.400588in}}%
\pgfpathlineto{\pgfqpoint{4.898715in}{2.416844in}}%
\pgfpathlineto{\pgfqpoint{4.906535in}{2.432956in}}%
\pgfpathlineto{\pgfqpoint{4.892185in}{2.419213in}}%
\pgfpathlineto{\pgfqpoint{4.877854in}{2.405634in}}%
\pgfpathlineto{\pgfqpoint{4.863541in}{2.392218in}}%
\pgfpathlineto{\pgfqpoint{4.849248in}{2.378966in}}%
\pgfpathlineto{\pgfqpoint{4.841437in}{2.363141in}}%
\pgfpathlineto{\pgfqpoint{4.833620in}{2.347178in}}%
\pgfpathlineto{\pgfqpoint{4.825798in}{2.331081in}}%
\pgfpathlineto{\pgfqpoint{4.817971in}{2.314853in}}%
\pgfpathclose%
\pgfusepath{fill}%
\end{pgfscope}%
\begin{pgfscope}%
\pgfpathrectangle{\pgfqpoint{1.254980in}{0.150000in}}{\pgfqpoint{5.490039in}{5.490039in}}%
\pgfusepath{clip}%
\pgfsetbuttcap%
\pgfsetroundjoin%
\definecolor{currentfill}{rgb}{0.257322,0.256130,0.526563}%
\pgfsetfillcolor{currentfill}%
\pgfsetfillopacity{0.700000}%
\pgfsetlinewidth{0.000000pt}%
\definecolor{currentstroke}{rgb}{0.000000,0.000000,0.000000}%
\pgfsetstrokecolor{currentstroke}%
\pgfsetdash{}{0pt}%
\pgfpathmoveto{\pgfqpoint{4.276525in}{1.484016in}}%
\pgfpathlineto{\pgfqpoint{4.290478in}{1.489903in}}%
\pgfpathlineto{\pgfqpoint{4.304444in}{1.495945in}}%
\pgfpathlineto{\pgfqpoint{4.318423in}{1.502140in}}%
\pgfpathlineto{\pgfqpoint{4.332415in}{1.508490in}}%
\pgfpathlineto{\pgfqpoint{4.340371in}{1.525228in}}%
\pgfpathlineto{\pgfqpoint{4.348324in}{1.542020in}}%
\pgfpathlineto{\pgfqpoint{4.356274in}{1.558859in}}%
\pgfpathlineto{\pgfqpoint{4.364221in}{1.575740in}}%
\pgfpathlineto{\pgfqpoint{4.350224in}{1.568812in}}%
\pgfpathlineto{\pgfqpoint{4.336241in}{1.562039in}}%
\pgfpathlineto{\pgfqpoint{4.322271in}{1.555422in}}%
\pgfpathlineto{\pgfqpoint{4.308314in}{1.548959in}}%
\pgfpathlineto{\pgfqpoint{4.300372in}{1.532644in}}%
\pgfpathlineto{\pgfqpoint{4.292426in}{1.516379in}}%
\pgfpathlineto{\pgfqpoint{4.284478in}{1.500168in}}%
\pgfpathlineto{\pgfqpoint{4.276525in}{1.484016in}}%
\pgfpathclose%
\pgfusepath{fill}%
\end{pgfscope}%
\begin{pgfscope}%
\pgfpathrectangle{\pgfqpoint{1.254980in}{0.150000in}}{\pgfqpoint{5.490039in}{5.490039in}}%
\pgfusepath{clip}%
\pgfsetbuttcap%
\pgfsetroundjoin%
\definecolor{currentfill}{rgb}{0.227802,0.326594,0.546532}%
\pgfsetfillcolor{currentfill}%
\pgfsetfillopacity{0.700000}%
\pgfsetlinewidth{0.000000pt}%
\definecolor{currentstroke}{rgb}{0.000000,0.000000,0.000000}%
\pgfsetstrokecolor{currentstroke}%
\pgfsetdash{}{0pt}%
\pgfpathmoveto{\pgfqpoint{4.395976in}{1.643579in}}%
\pgfpathlineto{\pgfqpoint{4.409993in}{1.651213in}}%
\pgfpathlineto{\pgfqpoint{4.424025in}{1.659002in}}%
\pgfpathlineto{\pgfqpoint{4.438070in}{1.666948in}}%
\pgfpathlineto{\pgfqpoint{4.452131in}{1.675049in}}%
\pgfpathlineto{\pgfqpoint{4.460068in}{1.692597in}}%
\pgfpathlineto{\pgfqpoint{4.468003in}{1.710149in}}%
\pgfpathlineto{\pgfqpoint{4.475935in}{1.727701in}}%
\pgfpathlineto{\pgfqpoint{4.483863in}{1.745247in}}%
\pgfpathlineto{\pgfqpoint{4.469795in}{1.736620in}}%
\pgfpathlineto{\pgfqpoint{4.455742in}{1.728150in}}%
\pgfpathlineto{\pgfqpoint{4.441703in}{1.719837in}}%
\pgfpathlineto{\pgfqpoint{4.427680in}{1.711680in}}%
\pgfpathlineto{\pgfqpoint{4.419759in}{1.694648in}}%
\pgfpathlineto{\pgfqpoint{4.411835in}{1.677617in}}%
\pgfpathlineto{\pgfqpoint{4.403907in}{1.660592in}}%
\pgfpathlineto{\pgfqpoint{4.395976in}{1.643579in}}%
\pgfpathclose%
\pgfusepath{fill}%
\end{pgfscope}%
\begin{pgfscope}%
\pgfpathrectangle{\pgfqpoint{1.254980in}{0.150000in}}{\pgfqpoint{5.490039in}{5.490039in}}%
\pgfusepath{clip}%
\pgfsetbuttcap%
\pgfsetroundjoin%
\definecolor{currentfill}{rgb}{0.276194,0.190074,0.493001}%
\pgfsetfillcolor{currentfill}%
\pgfsetfillopacity{0.700000}%
\pgfsetlinewidth{0.000000pt}%
\definecolor{currentstroke}{rgb}{0.000000,0.000000,0.000000}%
\pgfsetstrokecolor{currentstroke}%
\pgfsetdash{}{0pt}%
\pgfpathmoveto{\pgfqpoint{4.157110in}{1.340548in}}%
\pgfpathlineto{\pgfqpoint{4.171011in}{1.344590in}}%
\pgfpathlineto{\pgfqpoint{4.184923in}{1.348784in}}%
\pgfpathlineto{\pgfqpoint{4.198847in}{1.353131in}}%
\pgfpathlineto{\pgfqpoint{4.212783in}{1.357631in}}%
\pgfpathlineto{\pgfqpoint{4.220764in}{1.373101in}}%
\pgfpathlineto{\pgfqpoint{4.228740in}{1.388676in}}%
\pgfpathlineto{\pgfqpoint{4.236713in}{1.404351in}}%
\pgfpathlineto{\pgfqpoint{4.244682in}{1.420119in}}%
\pgfpathlineto{\pgfqpoint{4.230745in}{1.414989in}}%
\pgfpathlineto{\pgfqpoint{4.216820in}{1.410013in}}%
\pgfpathlineto{\pgfqpoint{4.202907in}{1.405190in}}%
\pgfpathlineto{\pgfqpoint{4.189005in}{1.400520in}}%
\pgfpathlineto{\pgfqpoint{4.181037in}{1.385370in}}%
\pgfpathlineto{\pgfqpoint{4.173065in}{1.370321in}}%
\pgfpathlineto{\pgfqpoint{4.165090in}{1.355378in}}%
\pgfpathlineto{\pgfqpoint{4.157110in}{1.340548in}}%
\pgfpathclose%
\pgfusepath{fill}%
\end{pgfscope}%
\begin{pgfscope}%
\pgfpathrectangle{\pgfqpoint{1.254980in}{0.150000in}}{\pgfqpoint{5.490039in}{5.490039in}}%
\pgfusepath{clip}%
\pgfsetbuttcap%
\pgfsetroundjoin%
\definecolor{currentfill}{rgb}{0.156270,0.489624,0.557936}%
\pgfsetfillcolor{currentfill}%
\pgfsetfillopacity{0.700000}%
\pgfsetlinewidth{0.000000pt}%
\definecolor{currentstroke}{rgb}{0.000000,0.000000,0.000000}%
\pgfsetstrokecolor{currentstroke}%
\pgfsetdash{}{0pt}%
\pgfpathmoveto{\pgfqpoint{4.666811in}{2.064968in}}%
\pgfpathlineto{\pgfqpoint{4.680995in}{2.076210in}}%
\pgfpathlineto{\pgfqpoint{4.695197in}{2.087612in}}%
\pgfpathlineto{\pgfqpoint{4.709416in}{2.099175in}}%
\pgfpathlineto{\pgfqpoint{4.723652in}{2.110898in}}%
\pgfpathlineto{\pgfqpoint{4.731537in}{2.128476in}}%
\pgfpathlineto{\pgfqpoint{4.739418in}{2.145959in}}%
\pgfpathlineto{\pgfqpoint{4.747295in}{2.163344in}}%
\pgfpathlineto{\pgfqpoint{4.755167in}{2.180627in}}%
\pgfpathlineto{\pgfqpoint{4.740920in}{2.168517in}}%
\pgfpathlineto{\pgfqpoint{4.726691in}{2.156569in}}%
\pgfpathlineto{\pgfqpoint{4.712480in}{2.144781in}}%
\pgfpathlineto{\pgfqpoint{4.698286in}{2.133155in}}%
\pgfpathlineto{\pgfqpoint{4.690423in}{2.116246in}}%
\pgfpathlineto{\pgfqpoint{4.682557in}{2.099243in}}%
\pgfpathlineto{\pgfqpoint{4.674686in}{2.082149in}}%
\pgfpathlineto{\pgfqpoint{4.666811in}{2.064968in}}%
\pgfpathclose%
\pgfusepath{fill}%
\end{pgfscope}%
\begin{pgfscope}%
\pgfpathrectangle{\pgfqpoint{1.254980in}{0.150000in}}{\pgfqpoint{5.490039in}{5.490039in}}%
\pgfusepath{clip}%
\pgfsetbuttcap%
\pgfsetroundjoin%
\definecolor{currentfill}{rgb}{0.128087,0.647749,0.523491}%
\pgfsetfillcolor{currentfill}%
\pgfsetfillopacity{0.700000}%
\pgfsetlinewidth{0.000000pt}%
\definecolor{currentstroke}{rgb}{0.000000,0.000000,0.000000}%
\pgfsetstrokecolor{currentstroke}%
\pgfsetdash{}{0pt}%
\pgfpathmoveto{\pgfqpoint{4.937755in}{2.495911in}}%
\pgfpathlineto{\pgfqpoint{4.952134in}{2.510086in}}%
\pgfpathlineto{\pgfqpoint{4.966532in}{2.524425in}}%
\pgfpathlineto{\pgfqpoint{4.980950in}{2.538929in}}%
\pgfpathlineto{\pgfqpoint{4.995388in}{2.553599in}}%
\pgfpathlineto{\pgfqpoint{5.003186in}{2.569202in}}%
\pgfpathlineto{\pgfqpoint{5.010978in}{2.584640in}}%
\pgfpathlineto{\pgfqpoint{5.018763in}{2.599910in}}%
\pgfpathlineto{\pgfqpoint{5.026541in}{2.615011in}}%
\pgfpathlineto{\pgfqpoint{5.012095in}{2.600104in}}%
\pgfpathlineto{\pgfqpoint{4.997669in}{2.585362in}}%
\pgfpathlineto{\pgfqpoint{4.983262in}{2.570786in}}%
\pgfpathlineto{\pgfqpoint{4.968876in}{2.556375in}}%
\pgfpathlineto{\pgfqpoint{4.961105in}{2.541500in}}%
\pgfpathlineto{\pgfqpoint{4.953328in}{2.526463in}}%
\pgfpathlineto{\pgfqpoint{4.945545in}{2.511266in}}%
\pgfpathlineto{\pgfqpoint{4.937755in}{2.495911in}}%
\pgfpathclose%
\pgfusepath{fill}%
\end{pgfscope}%
\begin{pgfscope}%
\pgfpathrectangle{\pgfqpoint{1.254980in}{0.150000in}}{\pgfqpoint{5.490039in}{5.490039in}}%
\pgfusepath{clip}%
\pgfsetbuttcap%
\pgfsetroundjoin%
\definecolor{currentfill}{rgb}{0.282656,0.100196,0.422160}%
\pgfsetfillcolor{currentfill}%
\pgfsetfillopacity{0.700000}%
\pgfsetlinewidth{0.000000pt}%
\definecolor{currentstroke}{rgb}{0.000000,0.000000,0.000000}%
\pgfsetstrokecolor{currentstroke}%
\pgfsetdash{}{0pt}%
\pgfpathmoveto{\pgfqpoint{3.950108in}{1.163158in}}%
\pgfpathlineto{\pgfqpoint{3.963940in}{1.163940in}}%
\pgfpathlineto{\pgfqpoint{3.977781in}{1.164873in}}%
\pgfpathlineto{\pgfqpoint{3.991632in}{1.165958in}}%
\pgfpathlineto{\pgfqpoint{4.005492in}{1.167194in}}%
\pgfpathlineto{\pgfqpoint{4.013529in}{1.179459in}}%
\pgfpathlineto{\pgfqpoint{4.021562in}{1.191922in}}%
\pgfpathlineto{\pgfqpoint{4.029589in}{1.204574in}}%
\pgfpathlineto{\pgfqpoint{4.037611in}{1.217411in}}%
\pgfpathlineto{\pgfqpoint{4.023758in}{1.215467in}}%
\pgfpathlineto{\pgfqpoint{4.009914in}{1.213675in}}%
\pgfpathlineto{\pgfqpoint{3.996080in}{1.212035in}}%
\pgfpathlineto{\pgfqpoint{3.982256in}{1.210548in}}%
\pgfpathlineto{\pgfqpoint{3.974227in}{1.198408in}}%
\pgfpathlineto{\pgfqpoint{3.966193in}{1.186459in}}%
\pgfpathlineto{\pgfqpoint{3.958153in}{1.174706in}}%
\pgfpathlineto{\pgfqpoint{3.950108in}{1.163158in}}%
\pgfpathclose%
\pgfusepath{fill}%
\end{pgfscope}%
\begin{pgfscope}%
\pgfpathrectangle{\pgfqpoint{1.254980in}{0.150000in}}{\pgfqpoint{5.490039in}{5.490039in}}%
\pgfusepath{clip}%
\pgfsetbuttcap%
\pgfsetroundjoin%
\definecolor{currentfill}{rgb}{0.194100,0.399323,0.555565}%
\pgfsetfillcolor{currentfill}%
\pgfsetfillopacity{0.700000}%
\pgfsetlinewidth{0.000000pt}%
\definecolor{currentstroke}{rgb}{0.000000,0.000000,0.000000}%
\pgfsetstrokecolor{currentstroke}%
\pgfsetdash{}{0pt}%
\pgfpathmoveto{\pgfqpoint{4.515543in}{1.815284in}}%
\pgfpathlineto{\pgfqpoint{4.529634in}{1.824565in}}%
\pgfpathlineto{\pgfqpoint{4.543742in}{1.834004in}}%
\pgfpathlineto{\pgfqpoint{4.557864in}{1.843601in}}%
\pgfpathlineto{\pgfqpoint{4.572003in}{1.853355in}}%
\pgfpathlineto{\pgfqpoint{4.579924in}{1.871286in}}%
\pgfpathlineto{\pgfqpoint{4.587841in}{1.889177in}}%
\pgfpathlineto{\pgfqpoint{4.595755in}{1.907023in}}%
\pgfpathlineto{\pgfqpoint{4.603665in}{1.924821in}}%
\pgfpathlineto{\pgfqpoint{4.589517in}{1.914595in}}%
\pgfpathlineto{\pgfqpoint{4.575384in}{1.904528in}}%
\pgfpathlineto{\pgfqpoint{4.561268in}{1.894619in}}%
\pgfpathlineto{\pgfqpoint{4.547168in}{1.884869in}}%
\pgfpathlineto{\pgfqpoint{4.539267in}{1.867530in}}%
\pgfpathlineto{\pgfqpoint{4.531363in}{1.850151in}}%
\pgfpathlineto{\pgfqpoint{4.523454in}{1.832734in}}%
\pgfpathlineto{\pgfqpoint{4.515543in}{1.815284in}}%
\pgfpathclose%
\pgfusepath{fill}%
\end{pgfscope}%
\begin{pgfscope}%
\pgfpathrectangle{\pgfqpoint{1.254980in}{0.150000in}}{\pgfqpoint{5.490039in}{5.490039in}}%
\pgfusepath{clip}%
\pgfsetbuttcap%
\pgfsetroundjoin%
\definecolor{currentfill}{rgb}{0.575563,0.844566,0.256415}%
\pgfsetfillcolor{currentfill}%
\pgfsetfillopacity{0.700000}%
\pgfsetlinewidth{0.000000pt}%
\definecolor{currentstroke}{rgb}{0.000000,0.000000,0.000000}%
\pgfsetstrokecolor{currentstroke}%
\pgfsetdash{}{0pt}%
\pgfpathmoveto{\pgfqpoint{5.416519in}{3.164927in}}%
\pgfpathlineto{\pgfqpoint{5.431278in}{3.182801in}}%
\pgfpathlineto{\pgfqpoint{5.446061in}{3.200846in}}%
\pgfpathlineto{\pgfqpoint{5.460868in}{3.219063in}}%
\pgfpathlineto{\pgfqpoint{5.468402in}{3.228742in}}%
\pgfpathlineto{\pgfqpoint{5.475925in}{3.238200in}}%
\pgfpathlineto{\pgfqpoint{5.483437in}{3.247438in}}%
\pgfpathlineto{\pgfqpoint{5.490937in}{3.256456in}}%
\pgfpathlineto{\pgfqpoint{5.476133in}{3.238260in}}%
\pgfpathlineto{\pgfqpoint{5.461351in}{3.220234in}}%
\pgfpathlineto{\pgfqpoint{5.446593in}{3.202379in}}%
\pgfpathlineto{\pgfqpoint{5.439091in}{3.193337in}}%
\pgfpathlineto{\pgfqpoint{5.431578in}{3.184081in}}%
\pgfpathlineto{\pgfqpoint{5.424054in}{3.174611in}}%
\pgfpathlineto{\pgfqpoint{5.416519in}{3.164927in}}%
\pgfpathclose%
\pgfusepath{fill}%
\end{pgfscope}%
\begin{pgfscope}%
\pgfpathrectangle{\pgfqpoint{1.254980in}{0.150000in}}{\pgfqpoint{5.490039in}{5.490039in}}%
\pgfusepath{clip}%
\pgfsetbuttcap%
\pgfsetroundjoin%
\definecolor{currentfill}{rgb}{0.191090,0.708366,0.482284}%
\pgfsetfillcolor{currentfill}%
\pgfsetfillopacity{0.700000}%
\pgfsetlinewidth{0.000000pt}%
\definecolor{currentstroke}{rgb}{0.000000,0.000000,0.000000}%
\pgfsetstrokecolor{currentstroke}%
\pgfsetdash{}{0pt}%
\pgfpathmoveto{\pgfqpoint{5.057584in}{2.673681in}}%
\pgfpathlineto{\pgfqpoint{5.072059in}{2.688961in}}%
\pgfpathlineto{\pgfqpoint{5.086554in}{2.704407in}}%
\pgfpathlineto{\pgfqpoint{5.101070in}{2.720020in}}%
\pgfpathlineto{\pgfqpoint{5.115607in}{2.735800in}}%
\pgfpathlineto{\pgfqpoint{5.123357in}{2.750214in}}%
\pgfpathlineto{\pgfqpoint{5.131099in}{2.764440in}}%
\pgfpathlineto{\pgfqpoint{5.138833in}{2.778477in}}%
\pgfpathlineto{\pgfqpoint{5.146559in}{2.792324in}}%
\pgfpathlineto{\pgfqpoint{5.132015in}{2.776368in}}%
\pgfpathlineto{\pgfqpoint{5.117492in}{2.760580in}}%
\pgfpathlineto{\pgfqpoint{5.102990in}{2.744958in}}%
\pgfpathlineto{\pgfqpoint{5.088509in}{2.729504in}}%
\pgfpathlineto{\pgfqpoint{5.080789in}{2.715820in}}%
\pgfpathlineto{\pgfqpoint{5.073062in}{2.701954in}}%
\pgfpathlineto{\pgfqpoint{5.065327in}{2.687907in}}%
\pgfpathlineto{\pgfqpoint{5.057584in}{2.673681in}}%
\pgfpathclose%
\pgfusepath{fill}%
\end{pgfscope}%
\begin{pgfscope}%
\pgfpathrectangle{\pgfqpoint{1.254980in}{0.150000in}}{\pgfqpoint{5.490039in}{5.490039in}}%
\pgfusepath{clip}%
\pgfsetbuttcap%
\pgfsetroundjoin%
\definecolor{currentfill}{rgb}{0.283072,0.130895,0.449241}%
\pgfsetfillcolor{currentfill}%
\pgfsetfillopacity{0.700000}%
\pgfsetlinewidth{0.000000pt}%
\definecolor{currentstroke}{rgb}{0.000000,0.000000,0.000000}%
\pgfsetstrokecolor{currentstroke}%
\pgfsetdash{}{0pt}%
\pgfpathmoveto{\pgfqpoint{4.037611in}{1.217411in}}%
\pgfpathlineto{\pgfqpoint{4.051474in}{1.219507in}}%
\pgfpathlineto{\pgfqpoint{4.065348in}{1.221754in}}%
\pgfpathlineto{\pgfqpoint{4.079231in}{1.224154in}}%
\pgfpathlineto{\pgfqpoint{4.093125in}{1.226706in}}%
\pgfpathlineto{\pgfqpoint{4.101138in}{1.240410in}}%
\pgfpathlineto{\pgfqpoint{4.109147in}{1.254278in}}%
\pgfpathlineto{\pgfqpoint{4.117151in}{1.268302in}}%
\pgfpathlineto{\pgfqpoint{4.125151in}{1.282477in}}%
\pgfpathlineto{\pgfqpoint{4.111260in}{1.279243in}}%
\pgfpathlineto{\pgfqpoint{4.097380in}{1.276162in}}%
\pgfpathlineto{\pgfqpoint{4.083510in}{1.273233in}}%
\pgfpathlineto{\pgfqpoint{4.069651in}{1.270456in}}%
\pgfpathlineto{\pgfqpoint{4.061648in}{1.256953in}}%
\pgfpathlineto{\pgfqpoint{4.053641in}{1.243607in}}%
\pgfpathlineto{\pgfqpoint{4.045628in}{1.230424in}}%
\pgfpathlineto{\pgfqpoint{4.037611in}{1.217411in}}%
\pgfpathclose%
\pgfusepath{fill}%
\end{pgfscope}%
\begin{pgfscope}%
\pgfpathrectangle{\pgfqpoint{1.254980in}{0.150000in}}{\pgfqpoint{5.490039in}{5.490039in}}%
\pgfusepath{clip}%
\pgfsetbuttcap%
\pgfsetroundjoin%
\definecolor{currentfill}{rgb}{0.129933,0.559582,0.551864}%
\pgfsetfillcolor{currentfill}%
\pgfsetfillopacity{0.700000}%
\pgfsetlinewidth{0.000000pt}%
\definecolor{currentstroke}{rgb}{0.000000,0.000000,0.000000}%
\pgfsetstrokecolor{currentstroke}%
\pgfsetdash{}{0pt}%
\pgfpathmoveto{\pgfqpoint{4.786608in}{2.248679in}}%
\pgfpathlineto{\pgfqpoint{4.800883in}{2.261308in}}%
\pgfpathlineto{\pgfqpoint{4.815176in}{2.274099in}}%
\pgfpathlineto{\pgfqpoint{4.829488in}{2.287052in}}%
\pgfpathlineto{\pgfqpoint{4.843818in}{2.300168in}}%
\pgfpathlineto{\pgfqpoint{4.851676in}{2.317234in}}%
\pgfpathlineto{\pgfqpoint{4.859530in}{2.334172in}}%
\pgfpathlineto{\pgfqpoint{4.867378in}{2.350980in}}%
\pgfpathlineto{\pgfqpoint{4.875220in}{2.367653in}}%
\pgfpathlineto{\pgfqpoint{4.860880in}{2.354209in}}%
\pgfpathlineto{\pgfqpoint{4.846558in}{2.340927in}}%
\pgfpathlineto{\pgfqpoint{4.832255in}{2.327809in}}%
\pgfpathlineto{\pgfqpoint{4.817971in}{2.314853in}}%
\pgfpathlineto{\pgfqpoint{4.810138in}{2.298496in}}%
\pgfpathlineto{\pgfqpoint{4.802300in}{2.282013in}}%
\pgfpathlineto{\pgfqpoint{4.794456in}{2.265406in}}%
\pgfpathlineto{\pgfqpoint{4.786608in}{2.248679in}}%
\pgfpathclose%
\pgfusepath{fill}%
\end{pgfscope}%
\begin{pgfscope}%
\pgfpathrectangle{\pgfqpoint{1.254980in}{0.150000in}}{\pgfqpoint{5.490039in}{5.490039in}}%
\pgfusepath{clip}%
\pgfsetbuttcap%
\pgfsetroundjoin%
\definecolor{currentfill}{rgb}{0.265145,0.232956,0.516599}%
\pgfsetfillcolor{currentfill}%
\pgfsetfillopacity{0.700000}%
\pgfsetlinewidth{0.000000pt}%
\definecolor{currentstroke}{rgb}{0.000000,0.000000,0.000000}%
\pgfsetstrokecolor{currentstroke}%
\pgfsetdash{}{0pt}%
\pgfpathmoveto{\pgfqpoint{4.244682in}{1.420119in}}%
\pgfpathlineto{\pgfqpoint{4.258632in}{1.425403in}}%
\pgfpathlineto{\pgfqpoint{4.272595in}{1.430840in}}%
\pgfpathlineto{\pgfqpoint{4.286570in}{1.436431in}}%
\pgfpathlineto{\pgfqpoint{4.300557in}{1.442175in}}%
\pgfpathlineto{\pgfqpoint{4.308527in}{1.458647in}}%
\pgfpathlineto{\pgfqpoint{4.316493in}{1.475193in}}%
\pgfpathlineto{\pgfqpoint{4.324455in}{1.491809in}}%
\pgfpathlineto{\pgfqpoint{4.332415in}{1.508490in}}%
\pgfpathlineto{\pgfqpoint{4.318423in}{1.502140in}}%
\pgfpathlineto{\pgfqpoint{4.304444in}{1.495945in}}%
\pgfpathlineto{\pgfqpoint{4.290478in}{1.489903in}}%
\pgfpathlineto{\pgfqpoint{4.276525in}{1.484016in}}%
\pgfpathlineto{\pgfqpoint{4.268570in}{1.467930in}}%
\pgfpathlineto{\pgfqpoint{4.260611in}{1.451915in}}%
\pgfpathlineto{\pgfqpoint{4.252648in}{1.435976in}}%
\pgfpathlineto{\pgfqpoint{4.244682in}{1.420119in}}%
\pgfpathclose%
\pgfusepath{fill}%
\end{pgfscope}%
\begin{pgfscope}%
\pgfpathrectangle{\pgfqpoint{1.254980in}{0.150000in}}{\pgfqpoint{5.490039in}{5.490039in}}%
\pgfusepath{clip}%
\pgfsetbuttcap%
\pgfsetroundjoin%
\definecolor{currentfill}{rgb}{0.311925,0.767822,0.415586}%
\pgfsetfillcolor{currentfill}%
\pgfsetfillopacity{0.700000}%
\pgfsetlinewidth{0.000000pt}%
\definecolor{currentstroke}{rgb}{0.000000,0.000000,0.000000}%
\pgfsetstrokecolor{currentstroke}%
\pgfsetdash{}{0pt}%
\pgfpathmoveto{\pgfqpoint{5.177382in}{2.845787in}}%
\pgfpathlineto{\pgfqpoint{5.191953in}{2.862054in}}%
\pgfpathlineto{\pgfqpoint{5.206546in}{2.878490in}}%
\pgfpathlineto{\pgfqpoint{5.221160in}{2.895094in}}%
\pgfpathlineto{\pgfqpoint{5.235797in}{2.911867in}}%
\pgfpathlineto{\pgfqpoint{5.243487in}{2.924870in}}%
\pgfpathlineto{\pgfqpoint{5.251168in}{2.937669in}}%
\pgfpathlineto{\pgfqpoint{5.258840in}{2.950263in}}%
\pgfpathlineto{\pgfqpoint{5.266502in}{2.962651in}}%
\pgfpathlineto{\pgfqpoint{5.251861in}{2.945766in}}%
\pgfpathlineto{\pgfqpoint{5.237242in}{2.929051in}}%
\pgfpathlineto{\pgfqpoint{5.222645in}{2.912504in}}%
\pgfpathlineto{\pgfqpoint{5.208069in}{2.896125in}}%
\pgfpathlineto{\pgfqpoint{5.200410in}{2.883836in}}%
\pgfpathlineto{\pgfqpoint{5.192743in}{2.871350in}}%
\pgfpathlineto{\pgfqpoint{5.185067in}{2.858667in}}%
\pgfpathlineto{\pgfqpoint{5.177382in}{2.845787in}}%
\pgfpathclose%
\pgfusepath{fill}%
\end{pgfscope}%
\begin{pgfscope}%
\pgfpathrectangle{\pgfqpoint{1.254980in}{0.150000in}}{\pgfqpoint{5.490039in}{5.490039in}}%
\pgfusepath{clip}%
\pgfsetbuttcap%
\pgfsetroundjoin%
\definecolor{currentfill}{rgb}{0.237441,0.305202,0.541921}%
\pgfsetfillcolor{currentfill}%
\pgfsetfillopacity{0.700000}%
\pgfsetlinewidth{0.000000pt}%
\definecolor{currentstroke}{rgb}{0.000000,0.000000,0.000000}%
\pgfsetstrokecolor{currentstroke}%
\pgfsetdash{}{0pt}%
\pgfpathmoveto{\pgfqpoint{4.364221in}{1.575740in}}%
\pgfpathlineto{\pgfqpoint{4.378232in}{1.582822in}}%
\pgfpathlineto{\pgfqpoint{4.392257in}{1.590060in}}%
\pgfpathlineto{\pgfqpoint{4.406296in}{1.597453in}}%
\pgfpathlineto{\pgfqpoint{4.420349in}{1.605001in}}%
\pgfpathlineto{\pgfqpoint{4.428299in}{1.622482in}}%
\pgfpathlineto{\pgfqpoint{4.436246in}{1.639986in}}%
\pgfpathlineto{\pgfqpoint{4.444190in}{1.657511in}}%
\pgfpathlineto{\pgfqpoint{4.452131in}{1.675049in}}%
\pgfpathlineto{\pgfqpoint{4.438070in}{1.666948in}}%
\pgfpathlineto{\pgfqpoint{4.424025in}{1.659002in}}%
\pgfpathlineto{\pgfqpoint{4.409993in}{1.651213in}}%
\pgfpathlineto{\pgfqpoint{4.395976in}{1.643579in}}%
\pgfpathlineto{\pgfqpoint{4.388042in}{1.626582in}}%
\pgfpathlineto{\pgfqpoint{4.380105in}{1.609607in}}%
\pgfpathlineto{\pgfqpoint{4.372165in}{1.592657in}}%
\pgfpathlineto{\pgfqpoint{4.364221in}{1.575740in}}%
\pgfpathclose%
\pgfusepath{fill}%
\end{pgfscope}%
\begin{pgfscope}%
\pgfpathrectangle{\pgfqpoint{1.254980in}{0.150000in}}{\pgfqpoint{5.490039in}{5.490039in}}%
\pgfusepath{clip}%
\pgfsetbuttcap%
\pgfsetroundjoin%
\definecolor{currentfill}{rgb}{0.163625,0.471133,0.558148}%
\pgfsetfillcolor{currentfill}%
\pgfsetfillopacity{0.700000}%
\pgfsetlinewidth{0.000000pt}%
\definecolor{currentstroke}{rgb}{0.000000,0.000000,0.000000}%
\pgfsetstrokecolor{currentstroke}%
\pgfsetdash{}{0pt}%
\pgfpathmoveto{\pgfqpoint{4.635269in}{1.995444in}}%
\pgfpathlineto{\pgfqpoint{4.649444in}{2.006272in}}%
\pgfpathlineto{\pgfqpoint{4.663635in}{2.017260in}}%
\pgfpathlineto{\pgfqpoint{4.677843in}{2.028407in}}%
\pgfpathlineto{\pgfqpoint{4.692069in}{2.039714in}}%
\pgfpathlineto{\pgfqpoint{4.699971in}{2.057634in}}%
\pgfpathlineto{\pgfqpoint{4.707869in}{2.075474in}}%
\pgfpathlineto{\pgfqpoint{4.715762in}{2.093230in}}%
\pgfpathlineto{\pgfqpoint{4.723652in}{2.110898in}}%
\pgfpathlineto{\pgfqpoint{4.709416in}{2.099175in}}%
\pgfpathlineto{\pgfqpoint{4.695197in}{2.087612in}}%
\pgfpathlineto{\pgfqpoint{4.680995in}{2.076210in}}%
\pgfpathlineto{\pgfqpoint{4.666811in}{2.064968in}}%
\pgfpathlineto{\pgfqpoint{4.658931in}{2.047704in}}%
\pgfpathlineto{\pgfqpoint{4.651048in}{2.030359in}}%
\pgfpathlineto{\pgfqpoint{4.643160in}{2.012938in}}%
\pgfpathlineto{\pgfqpoint{4.635269in}{1.995444in}}%
\pgfpathclose%
\pgfusepath{fill}%
\end{pgfscope}%
\begin{pgfscope}%
\pgfpathrectangle{\pgfqpoint{1.254980in}{0.150000in}}{\pgfqpoint{5.490039in}{5.490039in}}%
\pgfusepath{clip}%
\pgfsetbuttcap%
\pgfsetroundjoin%
\definecolor{currentfill}{rgb}{0.449368,0.813768,0.335384}%
\pgfsetfillcolor{currentfill}%
\pgfsetfillopacity{0.700000}%
\pgfsetlinewidth{0.000000pt}%
\definecolor{currentstroke}{rgb}{0.000000,0.000000,0.000000}%
\pgfsetstrokecolor{currentstroke}%
\pgfsetdash{}{0pt}%
\pgfpathmoveto{\pgfqpoint{5.297060in}{3.010136in}}%
\pgfpathlineto{\pgfqpoint{5.311727in}{3.027269in}}%
\pgfpathlineto{\pgfqpoint{5.326416in}{3.044572in}}%
\pgfpathlineto{\pgfqpoint{5.341128in}{3.062045in}}%
\pgfpathlineto{\pgfqpoint{5.355862in}{3.079688in}}%
\pgfpathlineto{\pgfqpoint{5.363480in}{3.091100in}}%
\pgfpathlineto{\pgfqpoint{5.371088in}{3.102295in}}%
\pgfpathlineto{\pgfqpoint{5.378686in}{3.113274in}}%
\pgfpathlineto{\pgfqpoint{5.386273in}{3.124036in}}%
\pgfpathlineto{\pgfqpoint{5.371537in}{3.106346in}}%
\pgfpathlineto{\pgfqpoint{5.356823in}{3.088827in}}%
\pgfpathlineto{\pgfqpoint{5.342132in}{3.071477in}}%
\pgfpathlineto{\pgfqpoint{5.327464in}{3.054297in}}%
\pgfpathlineto{\pgfqpoint{5.319878in}{3.043569in}}%
\pgfpathlineto{\pgfqpoint{5.312282in}{3.032633in}}%
\pgfpathlineto{\pgfqpoint{5.304676in}{3.021488in}}%
\pgfpathlineto{\pgfqpoint{5.297060in}{3.010136in}}%
\pgfpathclose%
\pgfusepath{fill}%
\end{pgfscope}%
\begin{pgfscope}%
\pgfpathrectangle{\pgfqpoint{1.254980in}{0.150000in}}{\pgfqpoint{5.490039in}{5.490039in}}%
\pgfusepath{clip}%
\pgfsetbuttcap%
\pgfsetroundjoin%
\definecolor{currentfill}{rgb}{0.279574,0.170599,0.479997}%
\pgfsetfillcolor{currentfill}%
\pgfsetfillopacity{0.700000}%
\pgfsetlinewidth{0.000000pt}%
\definecolor{currentstroke}{rgb}{0.000000,0.000000,0.000000}%
\pgfsetstrokecolor{currentstroke}%
\pgfsetdash{}{0pt}%
\pgfpathmoveto{\pgfqpoint{4.125151in}{1.282477in}}%
\pgfpathlineto{\pgfqpoint{4.139053in}{1.285863in}}%
\pgfpathlineto{\pgfqpoint{4.152966in}{1.289401in}}%
\pgfpathlineto{\pgfqpoint{4.166891in}{1.293092in}}%
\pgfpathlineto{\pgfqpoint{4.180826in}{1.296934in}}%
\pgfpathlineto{\pgfqpoint{4.188821in}{1.311919in}}%
\pgfpathlineto{\pgfqpoint{4.196812in}{1.327034in}}%
\pgfpathlineto{\pgfqpoint{4.204800in}{1.342274in}}%
\pgfpathlineto{\pgfqpoint{4.212783in}{1.357631in}}%
\pgfpathlineto{\pgfqpoint{4.198847in}{1.353131in}}%
\pgfpathlineto{\pgfqpoint{4.184923in}{1.348784in}}%
\pgfpathlineto{\pgfqpoint{4.171011in}{1.344590in}}%
\pgfpathlineto{\pgfqpoint{4.157110in}{1.340548in}}%
\pgfpathlineto{\pgfqpoint{4.149126in}{1.325837in}}%
\pgfpathlineto{\pgfqpoint{4.141139in}{1.311251in}}%
\pgfpathlineto{\pgfqpoint{4.133147in}{1.296795in}}%
\pgfpathlineto{\pgfqpoint{4.125151in}{1.282477in}}%
\pgfpathclose%
\pgfusepath{fill}%
\end{pgfscope}%
\begin{pgfscope}%
\pgfpathrectangle{\pgfqpoint{1.254980in}{0.150000in}}{\pgfqpoint{5.490039in}{5.490039in}}%
\pgfusepath{clip}%
\pgfsetbuttcap%
\pgfsetroundjoin%
\definecolor{currentfill}{rgb}{0.121380,0.629492,0.531973}%
\pgfsetfillcolor{currentfill}%
\pgfsetfillopacity{0.700000}%
\pgfsetlinewidth{0.000000pt}%
\definecolor{currentstroke}{rgb}{0.000000,0.000000,0.000000}%
\pgfsetstrokecolor{currentstroke}%
\pgfsetdash{}{0pt}%
\pgfpathmoveto{\pgfqpoint{4.906535in}{2.432956in}}%
\pgfpathlineto{\pgfqpoint{4.920904in}{2.446863in}}%
\pgfpathlineto{\pgfqpoint{4.935293in}{2.460934in}}%
\pgfpathlineto{\pgfqpoint{4.949702in}{2.475169in}}%
\pgfpathlineto{\pgfqpoint{4.964130in}{2.489570in}}%
\pgfpathlineto{\pgfqpoint{4.971954in}{2.505815in}}%
\pgfpathlineto{\pgfqpoint{4.979772in}{2.521903in}}%
\pgfpathlineto{\pgfqpoint{4.987583in}{2.537831in}}%
\pgfpathlineto{\pgfqpoint{4.995388in}{2.553599in}}%
\pgfpathlineto{\pgfqpoint{4.980950in}{2.538929in}}%
\pgfpathlineto{\pgfqpoint{4.966532in}{2.524425in}}%
\pgfpathlineto{\pgfqpoint{4.952134in}{2.510086in}}%
\pgfpathlineto{\pgfqpoint{4.937755in}{2.495911in}}%
\pgfpathlineto{\pgfqpoint{4.929959in}{2.480400in}}%
\pgfpathlineto{\pgfqpoint{4.922157in}{2.464736in}}%
\pgfpathlineto{\pgfqpoint{4.914349in}{2.448921in}}%
\pgfpathlineto{\pgfqpoint{4.906535in}{2.432956in}}%
\pgfpathclose%
\pgfusepath{fill}%
\end{pgfscope}%
\begin{pgfscope}%
\pgfpathrectangle{\pgfqpoint{1.254980in}{0.150000in}}{\pgfqpoint{5.490039in}{5.490039in}}%
\pgfusepath{clip}%
\pgfsetbuttcap%
\pgfsetroundjoin%
\definecolor{currentfill}{rgb}{0.203063,0.379716,0.553925}%
\pgfsetfillcolor{currentfill}%
\pgfsetfillopacity{0.700000}%
\pgfsetlinewidth{0.000000pt}%
\definecolor{currentstroke}{rgb}{0.000000,0.000000,0.000000}%
\pgfsetstrokecolor{currentstroke}%
\pgfsetdash{}{0pt}%
\pgfpathmoveto{\pgfqpoint{4.483863in}{1.745247in}}%
\pgfpathlineto{\pgfqpoint{4.497946in}{1.754030in}}%
\pgfpathlineto{\pgfqpoint{4.512044in}{1.762971in}}%
\pgfpathlineto{\pgfqpoint{4.526158in}{1.772068in}}%
\pgfpathlineto{\pgfqpoint{4.540287in}{1.781323in}}%
\pgfpathlineto{\pgfqpoint{4.548221in}{1.799368in}}%
\pgfpathlineto{\pgfqpoint{4.556152in}{1.817392in}}%
\pgfpathlineto{\pgfqpoint{4.564079in}{1.835389in}}%
\pgfpathlineto{\pgfqpoint{4.572003in}{1.853355in}}%
\pgfpathlineto{\pgfqpoint{4.557864in}{1.843601in}}%
\pgfpathlineto{\pgfqpoint{4.543742in}{1.834004in}}%
\pgfpathlineto{\pgfqpoint{4.529634in}{1.824565in}}%
\pgfpathlineto{\pgfqpoint{4.515543in}{1.815284in}}%
\pgfpathlineto{\pgfqpoint{4.507628in}{1.797806in}}%
\pgfpathlineto{\pgfqpoint{4.499709in}{1.780304in}}%
\pgfpathlineto{\pgfqpoint{4.491788in}{1.762783in}}%
\pgfpathlineto{\pgfqpoint{4.483863in}{1.745247in}}%
\pgfpathclose%
\pgfusepath{fill}%
\end{pgfscope}%
\begin{pgfscope}%
\pgfpathrectangle{\pgfqpoint{1.254980in}{0.150000in}}{\pgfqpoint{5.490039in}{5.490039in}}%
\pgfusepath{clip}%
\pgfsetbuttcap%
\pgfsetroundjoin%
\definecolor{currentfill}{rgb}{0.136408,0.541173,0.554483}%
\pgfsetfillcolor{currentfill}%
\pgfsetfillopacity{0.700000}%
\pgfsetlinewidth{0.000000pt}%
\definecolor{currentstroke}{rgb}{0.000000,0.000000,0.000000}%
\pgfsetstrokecolor{currentstroke}%
\pgfsetdash{}{0pt}%
\pgfpathmoveto{\pgfqpoint{4.755167in}{2.180627in}}%
\pgfpathlineto{\pgfqpoint{4.769431in}{2.192898in}}%
\pgfpathlineto{\pgfqpoint{4.783714in}{2.205331in}}%
\pgfpathlineto{\pgfqpoint{4.798015in}{2.217926in}}%
\pgfpathlineto{\pgfqpoint{4.812334in}{2.230683in}}%
\pgfpathlineto{\pgfqpoint{4.820212in}{2.248231in}}%
\pgfpathlineto{\pgfqpoint{4.828086in}{2.265663in}}%
\pgfpathlineto{\pgfqpoint{4.835954in}{2.282976in}}%
\pgfpathlineto{\pgfqpoint{4.843818in}{2.300168in}}%
\pgfpathlineto{\pgfqpoint{4.829488in}{2.287052in}}%
\pgfpathlineto{\pgfqpoint{4.815176in}{2.274099in}}%
\pgfpathlineto{\pgfqpoint{4.800883in}{2.261308in}}%
\pgfpathlineto{\pgfqpoint{4.786608in}{2.248679in}}%
\pgfpathlineto{\pgfqpoint{4.778755in}{2.231835in}}%
\pgfpathlineto{\pgfqpoint{4.770897in}{2.214876in}}%
\pgfpathlineto{\pgfqpoint{4.763034in}{2.197805in}}%
\pgfpathlineto{\pgfqpoint{4.755167in}{2.180627in}}%
\pgfpathclose%
\pgfusepath{fill}%
\end{pgfscope}%
\begin{pgfscope}%
\pgfpathrectangle{\pgfqpoint{1.254980in}{0.150000in}}{\pgfqpoint{5.490039in}{5.490039in}}%
\pgfusepath{clip}%
\pgfsetbuttcap%
\pgfsetroundjoin%
\definecolor{currentfill}{rgb}{0.175707,0.697900,0.491033}%
\pgfsetfillcolor{currentfill}%
\pgfsetfillopacity{0.700000}%
\pgfsetlinewidth{0.000000pt}%
\definecolor{currentstroke}{rgb}{0.000000,0.000000,0.000000}%
\pgfsetstrokecolor{currentstroke}%
\pgfsetdash{}{0pt}%
\pgfpathmoveto{\pgfqpoint{5.026541in}{2.615011in}}%
\pgfpathlineto{\pgfqpoint{5.041008in}{2.630084in}}%
\pgfpathlineto{\pgfqpoint{5.055496in}{2.645323in}}%
\pgfpathlineto{\pgfqpoint{5.070004in}{2.660729in}}%
\pgfpathlineto{\pgfqpoint{5.084533in}{2.676302in}}%
\pgfpathlineto{\pgfqpoint{5.092312in}{2.691450in}}%
\pgfpathlineto{\pgfqpoint{5.100085in}{2.706417in}}%
\pgfpathlineto{\pgfqpoint{5.107850in}{2.721201in}}%
\pgfpathlineto{\pgfqpoint{5.115607in}{2.735800in}}%
\pgfpathlineto{\pgfqpoint{5.101070in}{2.720020in}}%
\pgfpathlineto{\pgfqpoint{5.086554in}{2.704407in}}%
\pgfpathlineto{\pgfqpoint{5.072059in}{2.688961in}}%
\pgfpathlineto{\pgfqpoint{5.057584in}{2.673681in}}%
\pgfpathlineto{\pgfqpoint{5.049834in}{2.659277in}}%
\pgfpathlineto{\pgfqpoint{5.042077in}{2.644696in}}%
\pgfpathlineto{\pgfqpoint{5.034313in}{2.629940in}}%
\pgfpathlineto{\pgfqpoint{5.026541in}{2.615011in}}%
\pgfpathclose%
\pgfusepath{fill}%
\end{pgfscope}%
\begin{pgfscope}%
\pgfpathrectangle{\pgfqpoint{1.254980in}{0.150000in}}{\pgfqpoint{5.490039in}{5.490039in}}%
\pgfusepath{clip}%
\pgfsetbuttcap%
\pgfsetroundjoin%
\definecolor{currentfill}{rgb}{0.172719,0.448791,0.557885}%
\pgfsetfillcolor{currentfill}%
\pgfsetfillopacity{0.700000}%
\pgfsetlinewidth{0.000000pt}%
\definecolor{currentstroke}{rgb}{0.000000,0.000000,0.000000}%
\pgfsetstrokecolor{currentstroke}%
\pgfsetdash{}{0pt}%
\pgfpathmoveto{\pgfqpoint{4.603665in}{1.924821in}}%
\pgfpathlineto{\pgfqpoint{4.617830in}{1.935206in}}%
\pgfpathlineto{\pgfqpoint{4.632011in}{1.945749in}}%
\pgfpathlineto{\pgfqpoint{4.646209in}{1.956452in}}%
\pgfpathlineto{\pgfqpoint{4.660423in}{1.967313in}}%
\pgfpathlineto{\pgfqpoint{4.668340in}{1.985514in}}%
\pgfpathlineto{\pgfqpoint{4.676254in}{2.003650in}}%
\pgfpathlineto{\pgfqpoint{4.684163in}{2.021718in}}%
\pgfpathlineto{\pgfqpoint{4.692069in}{2.039714in}}%
\pgfpathlineto{\pgfqpoint{4.677843in}{2.028407in}}%
\pgfpathlineto{\pgfqpoint{4.663635in}{2.017260in}}%
\pgfpathlineto{\pgfqpoint{4.649444in}{2.006272in}}%
\pgfpathlineto{\pgfqpoint{4.635269in}{1.995444in}}%
\pgfpathlineto{\pgfqpoint{4.627374in}{1.977882in}}%
\pgfpathlineto{\pgfqpoint{4.619474in}{1.960254in}}%
\pgfpathlineto{\pgfqpoint{4.611572in}{1.942566in}}%
\pgfpathlineto{\pgfqpoint{4.603665in}{1.924821in}}%
\pgfpathclose%
\pgfusepath{fill}%
\end{pgfscope}%
\begin{pgfscope}%
\pgfpathrectangle{\pgfqpoint{1.254980in}{0.150000in}}{\pgfqpoint{5.490039in}{5.490039in}}%
\pgfusepath{clip}%
\pgfsetbuttcap%
\pgfsetroundjoin%
\definecolor{currentfill}{rgb}{0.246811,0.283237,0.535941}%
\pgfsetfillcolor{currentfill}%
\pgfsetfillopacity{0.700000}%
\pgfsetlinewidth{0.000000pt}%
\definecolor{currentstroke}{rgb}{0.000000,0.000000,0.000000}%
\pgfsetstrokecolor{currentstroke}%
\pgfsetdash{}{0pt}%
\pgfpathmoveto{\pgfqpoint{4.332415in}{1.508490in}}%
\pgfpathlineto{\pgfqpoint{4.346420in}{1.514994in}}%
\pgfpathlineto{\pgfqpoint{4.360439in}{1.521652in}}%
\pgfpathlineto{\pgfqpoint{4.374472in}{1.528465in}}%
\pgfpathlineto{\pgfqpoint{4.388518in}{1.535432in}}%
\pgfpathlineto{\pgfqpoint{4.396480in}{1.552761in}}%
\pgfpathlineto{\pgfqpoint{4.404439in}{1.570136in}}%
\pgfpathlineto{\pgfqpoint{4.412395in}{1.587551in}}%
\pgfpathlineto{\pgfqpoint{4.420349in}{1.605001in}}%
\pgfpathlineto{\pgfqpoint{4.406296in}{1.597453in}}%
\pgfpathlineto{\pgfqpoint{4.392257in}{1.590060in}}%
\pgfpathlineto{\pgfqpoint{4.378232in}{1.582822in}}%
\pgfpathlineto{\pgfqpoint{4.364221in}{1.575740in}}%
\pgfpathlineto{\pgfqpoint{4.356274in}{1.558859in}}%
\pgfpathlineto{\pgfqpoint{4.348324in}{1.542020in}}%
\pgfpathlineto{\pgfqpoint{4.340371in}{1.525228in}}%
\pgfpathlineto{\pgfqpoint{4.332415in}{1.508490in}}%
\pgfpathclose%
\pgfusepath{fill}%
\end{pgfscope}%
\begin{pgfscope}%
\pgfpathrectangle{\pgfqpoint{1.254980in}{0.150000in}}{\pgfqpoint{5.490039in}{5.490039in}}%
\pgfusepath{clip}%
\pgfsetbuttcap%
\pgfsetroundjoin%
\definecolor{currentfill}{rgb}{0.270595,0.214069,0.507052}%
\pgfsetfillcolor{currentfill}%
\pgfsetfillopacity{0.700000}%
\pgfsetlinewidth{0.000000pt}%
\definecolor{currentstroke}{rgb}{0.000000,0.000000,0.000000}%
\pgfsetstrokecolor{currentstroke}%
\pgfsetdash{}{0pt}%
\pgfpathmoveto{\pgfqpoint{4.212783in}{1.357631in}}%
\pgfpathlineto{\pgfqpoint{4.226731in}{1.362284in}}%
\pgfpathlineto{\pgfqpoint{4.240691in}{1.367090in}}%
\pgfpathlineto{\pgfqpoint{4.254664in}{1.372049in}}%
\pgfpathlineto{\pgfqpoint{4.268648in}{1.377160in}}%
\pgfpathlineto{\pgfqpoint{4.276631in}{1.393271in}}%
\pgfpathlineto{\pgfqpoint{4.284609in}{1.409482in}}%
\pgfpathlineto{\pgfqpoint{4.292585in}{1.425785in}}%
\pgfpathlineto{\pgfqpoint{4.300557in}{1.442175in}}%
\pgfpathlineto{\pgfqpoint{4.286570in}{1.436431in}}%
\pgfpathlineto{\pgfqpoint{4.272595in}{1.430840in}}%
\pgfpathlineto{\pgfqpoint{4.258632in}{1.425403in}}%
\pgfpathlineto{\pgfqpoint{4.244682in}{1.420119in}}%
\pgfpathlineto{\pgfqpoint{4.236713in}{1.404351in}}%
\pgfpathlineto{\pgfqpoint{4.228740in}{1.388676in}}%
\pgfpathlineto{\pgfqpoint{4.220764in}{1.373101in}}%
\pgfpathlineto{\pgfqpoint{4.212783in}{1.357631in}}%
\pgfpathclose%
\pgfusepath{fill}%
\end{pgfscope}%
\begin{pgfscope}%
\pgfpathrectangle{\pgfqpoint{1.254980in}{0.150000in}}{\pgfqpoint{5.490039in}{5.490039in}}%
\pgfusepath{clip}%
\pgfsetbuttcap%
\pgfsetroundjoin%
\definecolor{currentfill}{rgb}{0.283197,0.115680,0.436115}%
\pgfsetfillcolor{currentfill}%
\pgfsetfillopacity{0.700000}%
\pgfsetlinewidth{0.000000pt}%
\definecolor{currentstroke}{rgb}{0.000000,0.000000,0.000000}%
\pgfsetstrokecolor{currentstroke}%
\pgfsetdash{}{0pt}%
\pgfpathmoveto{\pgfqpoint{4.005492in}{1.167194in}}%
\pgfpathlineto{\pgfqpoint{4.019361in}{1.168582in}}%
\pgfpathlineto{\pgfqpoint{4.033240in}{1.170122in}}%
\pgfpathlineto{\pgfqpoint{4.047129in}{1.171813in}}%
\pgfpathlineto{\pgfqpoint{4.061027in}{1.173655in}}%
\pgfpathlineto{\pgfqpoint{4.069059in}{1.186639in}}%
\pgfpathlineto{\pgfqpoint{4.077086in}{1.199813in}}%
\pgfpathlineto{\pgfqpoint{4.085108in}{1.213171in}}%
\pgfpathlineto{\pgfqpoint{4.093125in}{1.226706in}}%
\pgfpathlineto{\pgfqpoint{4.079231in}{1.224154in}}%
\pgfpathlineto{\pgfqpoint{4.065348in}{1.221754in}}%
\pgfpathlineto{\pgfqpoint{4.051474in}{1.219507in}}%
\pgfpathlineto{\pgfqpoint{4.037611in}{1.217411in}}%
\pgfpathlineto{\pgfqpoint{4.029589in}{1.204574in}}%
\pgfpathlineto{\pgfqpoint{4.021562in}{1.191922in}}%
\pgfpathlineto{\pgfqpoint{4.013529in}{1.179459in}}%
\pgfpathlineto{\pgfqpoint{4.005492in}{1.167194in}}%
\pgfpathclose%
\pgfusepath{fill}%
\end{pgfscope}%
\begin{pgfscope}%
\pgfpathrectangle{\pgfqpoint{1.254980in}{0.150000in}}{\pgfqpoint{5.490039in}{5.490039in}}%
\pgfusepath{clip}%
\pgfsetbuttcap%
\pgfsetroundjoin%
\definecolor{currentfill}{rgb}{0.555484,0.840254,0.269281}%
\pgfsetfillcolor{currentfill}%
\pgfsetfillopacity{0.700000}%
\pgfsetlinewidth{0.000000pt}%
\definecolor{currentstroke}{rgb}{0.000000,0.000000,0.000000}%
\pgfsetstrokecolor{currentstroke}%
\pgfsetdash{}{0pt}%
\pgfpathmoveto{\pgfqpoint{5.386273in}{3.124036in}}%
\pgfpathlineto{\pgfqpoint{5.401033in}{3.141897in}}%
\pgfpathlineto{\pgfqpoint{5.415816in}{3.159928in}}%
\pgfpathlineto{\pgfqpoint{5.430622in}{3.178131in}}%
\pgfpathlineto{\pgfqpoint{5.438200in}{3.188697in}}%
\pgfpathlineto{\pgfqpoint{5.445767in}{3.199041in}}%
\pgfpathlineto{\pgfqpoint{5.453323in}{3.209163in}}%
\pgfpathlineto{\pgfqpoint{5.460868in}{3.219063in}}%
\pgfpathlineto{\pgfqpoint{5.446061in}{3.200846in}}%
\pgfpathlineto{\pgfqpoint{5.431278in}{3.182801in}}%
\pgfpathlineto{\pgfqpoint{5.416519in}{3.164927in}}%
\pgfpathlineto{\pgfqpoint{5.408973in}{3.155028in}}%
\pgfpathlineto{\pgfqpoint{5.401417in}{3.144913in}}%
\pgfpathlineto{\pgfqpoint{5.393850in}{3.134583in}}%
\pgfpathlineto{\pgfqpoint{5.386273in}{3.124036in}}%
\pgfpathclose%
\pgfusepath{fill}%
\end{pgfscope}%
\begin{pgfscope}%
\pgfpathrectangle{\pgfqpoint{1.254980in}{0.150000in}}{\pgfqpoint{5.490039in}{5.490039in}}%
\pgfusepath{clip}%
\pgfsetbuttcap%
\pgfsetroundjoin%
\definecolor{currentfill}{rgb}{0.214298,0.355619,0.551184}%
\pgfsetfillcolor{currentfill}%
\pgfsetfillopacity{0.700000}%
\pgfsetlinewidth{0.000000pt}%
\definecolor{currentstroke}{rgb}{0.000000,0.000000,0.000000}%
\pgfsetstrokecolor{currentstroke}%
\pgfsetdash{}{0pt}%
\pgfpathmoveto{\pgfqpoint{4.452131in}{1.675049in}}%
\pgfpathlineto{\pgfqpoint{4.466206in}{1.683307in}}%
\pgfpathlineto{\pgfqpoint{4.480295in}{1.691720in}}%
\pgfpathlineto{\pgfqpoint{4.494400in}{1.700290in}}%
\pgfpathlineto{\pgfqpoint{4.508520in}{1.709016in}}%
\pgfpathlineto{\pgfqpoint{4.516467in}{1.727102in}}%
\pgfpathlineto{\pgfqpoint{4.524410in}{1.745185in}}%
\pgfpathlineto{\pgfqpoint{4.532350in}{1.763260in}}%
\pgfpathlineto{\pgfqpoint{4.540287in}{1.781323in}}%
\pgfpathlineto{\pgfqpoint{4.526158in}{1.772068in}}%
\pgfpathlineto{\pgfqpoint{4.512044in}{1.762971in}}%
\pgfpathlineto{\pgfqpoint{4.497946in}{1.754030in}}%
\pgfpathlineto{\pgfqpoint{4.483863in}{1.745247in}}%
\pgfpathlineto{\pgfqpoint{4.475935in}{1.727701in}}%
\pgfpathlineto{\pgfqpoint{4.468003in}{1.710149in}}%
\pgfpathlineto{\pgfqpoint{4.460068in}{1.692597in}}%
\pgfpathlineto{\pgfqpoint{4.452131in}{1.675049in}}%
\pgfpathclose%
\pgfusepath{fill}%
\end{pgfscope}%
\begin{pgfscope}%
\pgfpathrectangle{\pgfqpoint{1.254980in}{0.150000in}}{\pgfqpoint{5.490039in}{5.490039in}}%
\pgfusepath{clip}%
\pgfsetbuttcap%
\pgfsetroundjoin%
\definecolor{currentfill}{rgb}{0.281477,0.755203,0.432552}%
\pgfsetfillcolor{currentfill}%
\pgfsetfillopacity{0.700000}%
\pgfsetlinewidth{0.000000pt}%
\definecolor{currentstroke}{rgb}{0.000000,0.000000,0.000000}%
\pgfsetstrokecolor{currentstroke}%
\pgfsetdash{}{0pt}%
\pgfpathmoveto{\pgfqpoint{5.146559in}{2.792324in}}%
\pgfpathlineto{\pgfqpoint{5.161124in}{2.808448in}}%
\pgfpathlineto{\pgfqpoint{5.175711in}{2.824739in}}%
\pgfpathlineto{\pgfqpoint{5.190320in}{2.841199in}}%
\pgfpathlineto{\pgfqpoint{5.204950in}{2.857827in}}%
\pgfpathlineto{\pgfqpoint{5.212675in}{2.871639in}}%
\pgfpathlineto{\pgfqpoint{5.220391in}{2.885250in}}%
\pgfpathlineto{\pgfqpoint{5.228098in}{2.898660in}}%
\pgfpathlineto{\pgfqpoint{5.235797in}{2.911867in}}%
\pgfpathlineto{\pgfqpoint{5.221160in}{2.895094in}}%
\pgfpathlineto{\pgfqpoint{5.206546in}{2.878490in}}%
\pgfpathlineto{\pgfqpoint{5.191953in}{2.862054in}}%
\pgfpathlineto{\pgfqpoint{5.177382in}{2.845787in}}%
\pgfpathlineto{\pgfqpoint{5.169689in}{2.832712in}}%
\pgfpathlineto{\pgfqpoint{5.161987in}{2.819443in}}%
\pgfpathlineto{\pgfqpoint{5.154277in}{2.805980in}}%
\pgfpathlineto{\pgfqpoint{5.146559in}{2.792324in}}%
\pgfpathclose%
\pgfusepath{fill}%
\end{pgfscope}%
\begin{pgfscope}%
\pgfpathrectangle{\pgfqpoint{1.254980in}{0.150000in}}{\pgfqpoint{5.490039in}{5.490039in}}%
\pgfusepath{clip}%
\pgfsetbuttcap%
\pgfsetroundjoin%
\definecolor{currentfill}{rgb}{0.119423,0.611141,0.538982}%
\pgfsetfillcolor{currentfill}%
\pgfsetfillopacity{0.700000}%
\pgfsetlinewidth{0.000000pt}%
\definecolor{currentstroke}{rgb}{0.000000,0.000000,0.000000}%
\pgfsetstrokecolor{currentstroke}%
\pgfsetdash{}{0pt}%
\pgfpathmoveto{\pgfqpoint{4.875220in}{2.367653in}}%
\pgfpathlineto{\pgfqpoint{4.889580in}{2.381262in}}%
\pgfpathlineto{\pgfqpoint{4.903959in}{2.395033in}}%
\pgfpathlineto{\pgfqpoint{4.918357in}{2.408969in}}%
\pgfpathlineto{\pgfqpoint{4.932774in}{2.423070in}}%
\pgfpathlineto{\pgfqpoint{4.940622in}{2.439918in}}%
\pgfpathlineto{\pgfqpoint{4.948464in}{2.456619in}}%
\pgfpathlineto{\pgfqpoint{4.956300in}{2.473170in}}%
\pgfpathlineto{\pgfqpoint{4.964130in}{2.489570in}}%
\pgfpathlineto{\pgfqpoint{4.949702in}{2.475169in}}%
\pgfpathlineto{\pgfqpoint{4.935293in}{2.460934in}}%
\pgfpathlineto{\pgfqpoint{4.920904in}{2.446863in}}%
\pgfpathlineto{\pgfqpoint{4.906535in}{2.432956in}}%
\pgfpathlineto{\pgfqpoint{4.898715in}{2.416844in}}%
\pgfpathlineto{\pgfqpoint{4.890889in}{2.400588in}}%
\pgfpathlineto{\pgfqpoint{4.883057in}{2.384190in}}%
\pgfpathlineto{\pgfqpoint{4.875220in}{2.367653in}}%
\pgfpathclose%
\pgfusepath{fill}%
\end{pgfscope}%
\begin{pgfscope}%
\pgfpathrectangle{\pgfqpoint{1.254980in}{0.150000in}}{\pgfqpoint{5.490039in}{5.490039in}}%
\pgfusepath{clip}%
\pgfsetbuttcap%
\pgfsetroundjoin%
\definecolor{currentfill}{rgb}{0.281887,0.150881,0.465405}%
\pgfsetfillcolor{currentfill}%
\pgfsetfillopacity{0.700000}%
\pgfsetlinewidth{0.000000pt}%
\definecolor{currentstroke}{rgb}{0.000000,0.000000,0.000000}%
\pgfsetstrokecolor{currentstroke}%
\pgfsetdash{}{0pt}%
\pgfpathmoveto{\pgfqpoint{4.093125in}{1.226706in}}%
\pgfpathlineto{\pgfqpoint{4.107030in}{1.229409in}}%
\pgfpathlineto{\pgfqpoint{4.120945in}{1.232264in}}%
\pgfpathlineto{\pgfqpoint{4.134871in}{1.235270in}}%
\pgfpathlineto{\pgfqpoint{4.148808in}{1.238428in}}%
\pgfpathlineto{\pgfqpoint{4.156819in}{1.252827in}}%
\pgfpathlineto{\pgfqpoint{4.164825in}{1.267382in}}%
\pgfpathlineto{\pgfqpoint{4.172828in}{1.282086in}}%
\pgfpathlineto{\pgfqpoint{4.180826in}{1.296934in}}%
\pgfpathlineto{\pgfqpoint{4.166891in}{1.293092in}}%
\pgfpathlineto{\pgfqpoint{4.152966in}{1.289401in}}%
\pgfpathlineto{\pgfqpoint{4.139053in}{1.285863in}}%
\pgfpathlineto{\pgfqpoint{4.125151in}{1.282477in}}%
\pgfpathlineto{\pgfqpoint{4.117151in}{1.268302in}}%
\pgfpathlineto{\pgfqpoint{4.109147in}{1.254278in}}%
\pgfpathlineto{\pgfqpoint{4.101138in}{1.240410in}}%
\pgfpathlineto{\pgfqpoint{4.093125in}{1.226706in}}%
\pgfpathclose%
\pgfusepath{fill}%
\end{pgfscope}%
\begin{pgfscope}%
\pgfpathrectangle{\pgfqpoint{1.254980in}{0.150000in}}{\pgfqpoint{5.490039in}{5.490039in}}%
\pgfusepath{clip}%
\pgfsetbuttcap%
\pgfsetroundjoin%
\definecolor{currentfill}{rgb}{0.421908,0.805774,0.351910}%
\pgfsetfillcolor{currentfill}%
\pgfsetfillopacity{0.700000}%
\pgfsetlinewidth{0.000000pt}%
\definecolor{currentstroke}{rgb}{0.000000,0.000000,0.000000}%
\pgfsetstrokecolor{currentstroke}%
\pgfsetdash{}{0pt}%
\pgfpathmoveto{\pgfqpoint{5.266502in}{2.962651in}}%
\pgfpathlineto{\pgfqpoint{5.281166in}{2.979705in}}%
\pgfpathlineto{\pgfqpoint{5.295852in}{2.996928in}}%
\pgfpathlineto{\pgfqpoint{5.310560in}{3.014322in}}%
\pgfpathlineto{\pgfqpoint{5.325291in}{3.031885in}}%
\pgfpathlineto{\pgfqpoint{5.332949in}{3.044159in}}%
\pgfpathlineto{\pgfqpoint{5.340596in}{3.056218in}}%
\pgfpathlineto{\pgfqpoint{5.348234in}{3.068061in}}%
\pgfpathlineto{\pgfqpoint{5.355862in}{3.079688in}}%
\pgfpathlineto{\pgfqpoint{5.341128in}{3.062045in}}%
\pgfpathlineto{\pgfqpoint{5.326416in}{3.044572in}}%
\pgfpathlineto{\pgfqpoint{5.311727in}{3.027269in}}%
\pgfpathlineto{\pgfqpoint{5.297060in}{3.010136in}}%
\pgfpathlineto{\pgfqpoint{5.289435in}{2.998576in}}%
\pgfpathlineto{\pgfqpoint{5.281800in}{2.986808in}}%
\pgfpathlineto{\pgfqpoint{5.274156in}{2.974833in}}%
\pgfpathlineto{\pgfqpoint{5.266502in}{2.962651in}}%
\pgfpathclose%
\pgfusepath{fill}%
\end{pgfscope}%
\begin{pgfscope}%
\pgfpathrectangle{\pgfqpoint{1.254980in}{0.150000in}}{\pgfqpoint{5.490039in}{5.490039in}}%
\pgfusepath{clip}%
\pgfsetbuttcap%
\pgfsetroundjoin%
\definecolor{currentfill}{rgb}{0.143343,0.522773,0.556295}%
\pgfsetfillcolor{currentfill}%
\pgfsetfillopacity{0.700000}%
\pgfsetlinewidth{0.000000pt}%
\definecolor{currentstroke}{rgb}{0.000000,0.000000,0.000000}%
\pgfsetstrokecolor{currentstroke}%
\pgfsetdash{}{0pt}%
\pgfpathmoveto{\pgfqpoint{4.723652in}{2.110898in}}%
\pgfpathlineto{\pgfqpoint{4.737906in}{2.122783in}}%
\pgfpathlineto{\pgfqpoint{4.752177in}{2.134828in}}%
\pgfpathlineto{\pgfqpoint{4.766467in}{2.147034in}}%
\pgfpathlineto{\pgfqpoint{4.780775in}{2.159402in}}%
\pgfpathlineto{\pgfqpoint{4.788671in}{2.177379in}}%
\pgfpathlineto{\pgfqpoint{4.796563in}{2.195253in}}%
\pgfpathlineto{\pgfqpoint{4.804451in}{2.213023in}}%
\pgfpathlineto{\pgfqpoint{4.812334in}{2.230683in}}%
\pgfpathlineto{\pgfqpoint{4.798015in}{2.217926in}}%
\pgfpathlineto{\pgfqpoint{4.783714in}{2.205331in}}%
\pgfpathlineto{\pgfqpoint{4.769431in}{2.192898in}}%
\pgfpathlineto{\pgfqpoint{4.755167in}{2.180627in}}%
\pgfpathlineto{\pgfqpoint{4.747295in}{2.163344in}}%
\pgfpathlineto{\pgfqpoint{4.739418in}{2.145959in}}%
\pgfpathlineto{\pgfqpoint{4.731537in}{2.128476in}}%
\pgfpathlineto{\pgfqpoint{4.723652in}{2.110898in}}%
\pgfpathclose%
\pgfusepath{fill}%
\end{pgfscope}%
\begin{pgfscope}%
\pgfpathrectangle{\pgfqpoint{1.254980in}{0.150000in}}{\pgfqpoint{5.490039in}{5.490039in}}%
\pgfusepath{clip}%
\pgfsetbuttcap%
\pgfsetroundjoin%
\definecolor{currentfill}{rgb}{0.180629,0.429975,0.557282}%
\pgfsetfillcolor{currentfill}%
\pgfsetfillopacity{0.700000}%
\pgfsetlinewidth{0.000000pt}%
\definecolor{currentstroke}{rgb}{0.000000,0.000000,0.000000}%
\pgfsetstrokecolor{currentstroke}%
\pgfsetdash{}{0pt}%
\pgfpathmoveto{\pgfqpoint{4.572003in}{1.853355in}}%
\pgfpathlineto{\pgfqpoint{4.586158in}{1.863268in}}%
\pgfpathlineto{\pgfqpoint{4.600329in}{1.873338in}}%
\pgfpathlineto{\pgfqpoint{4.614516in}{1.883567in}}%
\pgfpathlineto{\pgfqpoint{4.628720in}{1.893955in}}%
\pgfpathlineto{\pgfqpoint{4.636651in}{1.912369in}}%
\pgfpathlineto{\pgfqpoint{4.644578in}{1.930737in}}%
\pgfpathlineto{\pgfqpoint{4.652503in}{1.949053in}}%
\pgfpathlineto{\pgfqpoint{4.660423in}{1.967313in}}%
\pgfpathlineto{\pgfqpoint{4.646209in}{1.956452in}}%
\pgfpathlineto{\pgfqpoint{4.632011in}{1.945749in}}%
\pgfpathlineto{\pgfqpoint{4.617830in}{1.935206in}}%
\pgfpathlineto{\pgfqpoint{4.603665in}{1.924821in}}%
\pgfpathlineto{\pgfqpoint{4.595755in}{1.907023in}}%
\pgfpathlineto{\pgfqpoint{4.587841in}{1.889177in}}%
\pgfpathlineto{\pgfqpoint{4.579924in}{1.871286in}}%
\pgfpathlineto{\pgfqpoint{4.572003in}{1.853355in}}%
\pgfpathclose%
\pgfusepath{fill}%
\end{pgfscope}%
\begin{pgfscope}%
\pgfpathrectangle{\pgfqpoint{1.254980in}{0.150000in}}{\pgfqpoint{5.490039in}{5.490039in}}%
\pgfusepath{clip}%
\pgfsetbuttcap%
\pgfsetroundjoin%
\definecolor{currentfill}{rgb}{0.255645,0.260703,0.528312}%
\pgfsetfillcolor{currentfill}%
\pgfsetfillopacity{0.700000}%
\pgfsetlinewidth{0.000000pt}%
\definecolor{currentstroke}{rgb}{0.000000,0.000000,0.000000}%
\pgfsetstrokecolor{currentstroke}%
\pgfsetdash{}{0pt}%
\pgfpathmoveto{\pgfqpoint{4.300557in}{1.442175in}}%
\pgfpathlineto{\pgfqpoint{4.314558in}{1.448073in}}%
\pgfpathlineto{\pgfqpoint{4.328572in}{1.454124in}}%
\pgfpathlineto{\pgfqpoint{4.342599in}{1.460329in}}%
\pgfpathlineto{\pgfqpoint{4.356640in}{1.466688in}}%
\pgfpathlineto{\pgfqpoint{4.364614in}{1.483777in}}%
\pgfpathlineto{\pgfqpoint{4.372585in}{1.500934in}}%
\pgfpathlineto{\pgfqpoint{4.380553in}{1.518155in}}%
\pgfpathlineto{\pgfqpoint{4.388518in}{1.535432in}}%
\pgfpathlineto{\pgfqpoint{4.374472in}{1.528465in}}%
\pgfpathlineto{\pgfqpoint{4.360439in}{1.521652in}}%
\pgfpathlineto{\pgfqpoint{4.346420in}{1.514994in}}%
\pgfpathlineto{\pgfqpoint{4.332415in}{1.508490in}}%
\pgfpathlineto{\pgfqpoint{4.324455in}{1.491809in}}%
\pgfpathlineto{\pgfqpoint{4.316493in}{1.475193in}}%
\pgfpathlineto{\pgfqpoint{4.308527in}{1.458647in}}%
\pgfpathlineto{\pgfqpoint{4.300557in}{1.442175in}}%
\pgfpathclose%
\pgfusepath{fill}%
\end{pgfscope}%
\begin{pgfscope}%
\pgfpathrectangle{\pgfqpoint{1.254980in}{0.150000in}}{\pgfqpoint{5.490039in}{5.490039in}}%
\pgfusepath{clip}%
\pgfsetbuttcap%
\pgfsetroundjoin%
\definecolor{currentfill}{rgb}{0.153894,0.680203,0.504172}%
\pgfsetfillcolor{currentfill}%
\pgfsetfillopacity{0.700000}%
\pgfsetlinewidth{0.000000pt}%
\definecolor{currentstroke}{rgb}{0.000000,0.000000,0.000000}%
\pgfsetstrokecolor{currentstroke}%
\pgfsetdash{}{0pt}%
\pgfpathmoveto{\pgfqpoint{4.995388in}{2.553599in}}%
\pgfpathlineto{\pgfqpoint{5.009846in}{2.568434in}}%
\pgfpathlineto{\pgfqpoint{5.024324in}{2.583434in}}%
\pgfpathlineto{\pgfqpoint{5.038823in}{2.598601in}}%
\pgfpathlineto{\pgfqpoint{5.053342in}{2.613934in}}%
\pgfpathlineto{\pgfqpoint{5.061150in}{2.629788in}}%
\pgfpathlineto{\pgfqpoint{5.068951in}{2.645469in}}%
\pgfpathlineto{\pgfqpoint{5.076746in}{2.660974in}}%
\pgfpathlineto{\pgfqpoint{5.084533in}{2.676302in}}%
\pgfpathlineto{\pgfqpoint{5.070004in}{2.660729in}}%
\pgfpathlineto{\pgfqpoint{5.055496in}{2.645323in}}%
\pgfpathlineto{\pgfqpoint{5.041008in}{2.630084in}}%
\pgfpathlineto{\pgfqpoint{5.026541in}{2.615011in}}%
\pgfpathlineto{\pgfqpoint{5.018763in}{2.599910in}}%
\pgfpathlineto{\pgfqpoint{5.010978in}{2.584640in}}%
\pgfpathlineto{\pgfqpoint{5.003186in}{2.569202in}}%
\pgfpathlineto{\pgfqpoint{4.995388in}{2.553599in}}%
\pgfpathclose%
\pgfusepath{fill}%
\end{pgfscope}%
\begin{pgfscope}%
\pgfpathrectangle{\pgfqpoint{1.254980in}{0.150000in}}{\pgfqpoint{5.490039in}{5.490039in}}%
\pgfusepath{clip}%
\pgfsetbuttcap%
\pgfsetroundjoin%
\definecolor{currentfill}{rgb}{0.276194,0.190074,0.493001}%
\pgfsetfillcolor{currentfill}%
\pgfsetfillopacity{0.700000}%
\pgfsetlinewidth{0.000000pt}%
\definecolor{currentstroke}{rgb}{0.000000,0.000000,0.000000}%
\pgfsetstrokecolor{currentstroke}%
\pgfsetdash{}{0pt}%
\pgfpathmoveto{\pgfqpoint{4.180826in}{1.296934in}}%
\pgfpathlineto{\pgfqpoint{4.194774in}{1.300929in}}%
\pgfpathlineto{\pgfqpoint{4.208733in}{1.305076in}}%
\pgfpathlineto{\pgfqpoint{4.222703in}{1.309375in}}%
\pgfpathlineto{\pgfqpoint{4.236686in}{1.313826in}}%
\pgfpathlineto{\pgfqpoint{4.244682in}{1.329480in}}%
\pgfpathlineto{\pgfqpoint{4.252674in}{1.345258in}}%
\pgfpathlineto{\pgfqpoint{4.260663in}{1.361153in}}%
\pgfpathlineto{\pgfqpoint{4.268648in}{1.377160in}}%
\pgfpathlineto{\pgfqpoint{4.254664in}{1.372049in}}%
\pgfpathlineto{\pgfqpoint{4.240691in}{1.367090in}}%
\pgfpathlineto{\pgfqpoint{4.226731in}{1.362284in}}%
\pgfpathlineto{\pgfqpoint{4.212783in}{1.357631in}}%
\pgfpathlineto{\pgfqpoint{4.204800in}{1.342274in}}%
\pgfpathlineto{\pgfqpoint{4.196812in}{1.327034in}}%
\pgfpathlineto{\pgfqpoint{4.188821in}{1.311919in}}%
\pgfpathlineto{\pgfqpoint{4.180826in}{1.296934in}}%
\pgfpathclose%
\pgfusepath{fill}%
\end{pgfscope}%
\begin{pgfscope}%
\pgfpathrectangle{\pgfqpoint{1.254980in}{0.150000in}}{\pgfqpoint{5.490039in}{5.490039in}}%
\pgfusepath{clip}%
\pgfsetbuttcap%
\pgfsetroundjoin%
\definecolor{currentfill}{rgb}{0.223925,0.334994,0.548053}%
\pgfsetfillcolor{currentfill}%
\pgfsetfillopacity{0.700000}%
\pgfsetlinewidth{0.000000pt}%
\definecolor{currentstroke}{rgb}{0.000000,0.000000,0.000000}%
\pgfsetstrokecolor{currentstroke}%
\pgfsetdash{}{0pt}%
\pgfpathmoveto{\pgfqpoint{4.420349in}{1.605001in}}%
\pgfpathlineto{\pgfqpoint{4.434416in}{1.612705in}}%
\pgfpathlineto{\pgfqpoint{4.448498in}{1.620564in}}%
\pgfpathlineto{\pgfqpoint{4.462594in}{1.628578in}}%
\pgfpathlineto{\pgfqpoint{4.476705in}{1.636748in}}%
\pgfpathlineto{\pgfqpoint{4.484663in}{1.654793in}}%
\pgfpathlineto{\pgfqpoint{4.492619in}{1.672857in}}%
\pgfpathlineto{\pgfqpoint{4.500571in}{1.690933in}}%
\pgfpathlineto{\pgfqpoint{4.508520in}{1.709016in}}%
\pgfpathlineto{\pgfqpoint{4.494400in}{1.700290in}}%
\pgfpathlineto{\pgfqpoint{4.480295in}{1.691720in}}%
\pgfpathlineto{\pgfqpoint{4.466206in}{1.683307in}}%
\pgfpathlineto{\pgfqpoint{4.452131in}{1.675049in}}%
\pgfpathlineto{\pgfqpoint{4.444190in}{1.657511in}}%
\pgfpathlineto{\pgfqpoint{4.436246in}{1.639986in}}%
\pgfpathlineto{\pgfqpoint{4.428299in}{1.622482in}}%
\pgfpathlineto{\pgfqpoint{4.420349in}{1.605001in}}%
\pgfpathclose%
\pgfusepath{fill}%
\end{pgfscope}%
\begin{pgfscope}%
\pgfpathrectangle{\pgfqpoint{1.254980in}{0.150000in}}{\pgfqpoint{5.490039in}{5.490039in}}%
\pgfusepath{clip}%
\pgfsetbuttcap%
\pgfsetroundjoin%
\definecolor{currentfill}{rgb}{0.121148,0.592739,0.544641}%
\pgfsetfillcolor{currentfill}%
\pgfsetfillopacity{0.700000}%
\pgfsetlinewidth{0.000000pt}%
\definecolor{currentstroke}{rgb}{0.000000,0.000000,0.000000}%
\pgfsetstrokecolor{currentstroke}%
\pgfsetdash{}{0pt}%
\pgfpathmoveto{\pgfqpoint{4.843818in}{2.300168in}}%
\pgfpathlineto{\pgfqpoint{4.858167in}{2.313447in}}%
\pgfpathlineto{\pgfqpoint{4.872535in}{2.326889in}}%
\pgfpathlineto{\pgfqpoint{4.886922in}{2.340494in}}%
\pgfpathlineto{\pgfqpoint{4.901328in}{2.354263in}}%
\pgfpathlineto{\pgfqpoint{4.909198in}{2.371671in}}%
\pgfpathlineto{\pgfqpoint{4.917062in}{2.388943in}}%
\pgfpathlineto{\pgfqpoint{4.924921in}{2.406077in}}%
\pgfpathlineto{\pgfqpoint{4.932774in}{2.423070in}}%
\pgfpathlineto{\pgfqpoint{4.918357in}{2.408969in}}%
\pgfpathlineto{\pgfqpoint{4.903959in}{2.395033in}}%
\pgfpathlineto{\pgfqpoint{4.889580in}{2.381262in}}%
\pgfpathlineto{\pgfqpoint{4.875220in}{2.367653in}}%
\pgfpathlineto{\pgfqpoint{4.867378in}{2.350980in}}%
\pgfpathlineto{\pgfqpoint{4.859530in}{2.334172in}}%
\pgfpathlineto{\pgfqpoint{4.851676in}{2.317234in}}%
\pgfpathlineto{\pgfqpoint{4.843818in}{2.300168in}}%
\pgfpathclose%
\pgfusepath{fill}%
\end{pgfscope}%
\begin{pgfscope}%
\pgfpathrectangle{\pgfqpoint{1.254980in}{0.150000in}}{\pgfqpoint{5.490039in}{5.490039in}}%
\pgfusepath{clip}%
\pgfsetbuttcap%
\pgfsetroundjoin%
\definecolor{currentfill}{rgb}{0.151918,0.500685,0.557587}%
\pgfsetfillcolor{currentfill}%
\pgfsetfillopacity{0.700000}%
\pgfsetlinewidth{0.000000pt}%
\definecolor{currentstroke}{rgb}{0.000000,0.000000,0.000000}%
\pgfsetstrokecolor{currentstroke}%
\pgfsetdash{}{0pt}%
\pgfpathmoveto{\pgfqpoint{4.692069in}{2.039714in}}%
\pgfpathlineto{\pgfqpoint{4.706312in}{2.051181in}}%
\pgfpathlineto{\pgfqpoint{4.720572in}{2.062809in}}%
\pgfpathlineto{\pgfqpoint{4.734850in}{2.074597in}}%
\pgfpathlineto{\pgfqpoint{4.749146in}{2.086546in}}%
\pgfpathlineto{\pgfqpoint{4.757059in}{2.104894in}}%
\pgfpathlineto{\pgfqpoint{4.764969in}{2.123156in}}%
\pgfpathlineto{\pgfqpoint{4.772874in}{2.141326in}}%
\pgfpathlineto{\pgfqpoint{4.780775in}{2.159402in}}%
\pgfpathlineto{\pgfqpoint{4.766467in}{2.147034in}}%
\pgfpathlineto{\pgfqpoint{4.752177in}{2.134828in}}%
\pgfpathlineto{\pgfqpoint{4.737906in}{2.122783in}}%
\pgfpathlineto{\pgfqpoint{4.723652in}{2.110898in}}%
\pgfpathlineto{\pgfqpoint{4.715762in}{2.093230in}}%
\pgfpathlineto{\pgfqpoint{4.707869in}{2.075474in}}%
\pgfpathlineto{\pgfqpoint{4.699971in}{2.057634in}}%
\pgfpathlineto{\pgfqpoint{4.692069in}{2.039714in}}%
\pgfpathclose%
\pgfusepath{fill}%
\end{pgfscope}%
\begin{pgfscope}%
\pgfpathrectangle{\pgfqpoint{1.254980in}{0.150000in}}{\pgfqpoint{5.490039in}{5.490039in}}%
\pgfusepath{clip}%
\pgfsetbuttcap%
\pgfsetroundjoin%
\definecolor{currentfill}{rgb}{0.259857,0.745492,0.444467}%
\pgfsetfillcolor{currentfill}%
\pgfsetfillopacity{0.700000}%
\pgfsetlinewidth{0.000000pt}%
\definecolor{currentstroke}{rgb}{0.000000,0.000000,0.000000}%
\pgfsetstrokecolor{currentstroke}%
\pgfsetdash{}{0pt}%
\pgfpathmoveto{\pgfqpoint{5.115607in}{2.735800in}}%
\pgfpathlineto{\pgfqpoint{5.130165in}{2.751748in}}%
\pgfpathlineto{\pgfqpoint{5.144745in}{2.767863in}}%
\pgfpathlineto{\pgfqpoint{5.159346in}{2.784146in}}%
\pgfpathlineto{\pgfqpoint{5.173969in}{2.800598in}}%
\pgfpathlineto{\pgfqpoint{5.181726in}{2.815200in}}%
\pgfpathlineto{\pgfqpoint{5.189476in}{2.829606in}}%
\pgfpathlineto{\pgfqpoint{5.197217in}{2.843816in}}%
\pgfpathlineto{\pgfqpoint{5.204950in}{2.857827in}}%
\pgfpathlineto{\pgfqpoint{5.190320in}{2.841199in}}%
\pgfpathlineto{\pgfqpoint{5.175711in}{2.824739in}}%
\pgfpathlineto{\pgfqpoint{5.161124in}{2.808448in}}%
\pgfpathlineto{\pgfqpoint{5.146559in}{2.792324in}}%
\pgfpathlineto{\pgfqpoint{5.138833in}{2.778477in}}%
\pgfpathlineto{\pgfqpoint{5.131099in}{2.764440in}}%
\pgfpathlineto{\pgfqpoint{5.123357in}{2.750214in}}%
\pgfpathlineto{\pgfqpoint{5.115607in}{2.735800in}}%
\pgfpathclose%
\pgfusepath{fill}%
\end{pgfscope}%
\begin{pgfscope}%
\pgfpathrectangle{\pgfqpoint{1.254980in}{0.150000in}}{\pgfqpoint{5.490039in}{5.490039in}}%
\pgfusepath{clip}%
\pgfsetbuttcap%
\pgfsetroundjoin%
\definecolor{currentfill}{rgb}{0.190631,0.407061,0.556089}%
\pgfsetfillcolor{currentfill}%
\pgfsetfillopacity{0.700000}%
\pgfsetlinewidth{0.000000pt}%
\definecolor{currentstroke}{rgb}{0.000000,0.000000,0.000000}%
\pgfsetstrokecolor{currentstroke}%
\pgfsetdash{}{0pt}%
\pgfpathmoveto{\pgfqpoint{4.540287in}{1.781323in}}%
\pgfpathlineto{\pgfqpoint{4.554432in}{1.790734in}}%
\pgfpathlineto{\pgfqpoint{4.568593in}{1.800303in}}%
\pgfpathlineto{\pgfqpoint{4.582769in}{1.810030in}}%
\pgfpathlineto{\pgfqpoint{4.596962in}{1.819914in}}%
\pgfpathlineto{\pgfqpoint{4.604906in}{1.838472in}}%
\pgfpathlineto{\pgfqpoint{4.612847in}{1.857001in}}%
\pgfpathlineto{\pgfqpoint{4.620785in}{1.875497in}}%
\pgfpathlineto{\pgfqpoint{4.628720in}{1.893955in}}%
\pgfpathlineto{\pgfqpoint{4.614516in}{1.883567in}}%
\pgfpathlineto{\pgfqpoint{4.600329in}{1.873338in}}%
\pgfpathlineto{\pgfqpoint{4.586158in}{1.863268in}}%
\pgfpathlineto{\pgfqpoint{4.572003in}{1.853355in}}%
\pgfpathlineto{\pgfqpoint{4.564079in}{1.835389in}}%
\pgfpathlineto{\pgfqpoint{4.556152in}{1.817392in}}%
\pgfpathlineto{\pgfqpoint{4.548221in}{1.799368in}}%
\pgfpathlineto{\pgfqpoint{4.540287in}{1.781323in}}%
\pgfpathclose%
\pgfusepath{fill}%
\end{pgfscope}%
\begin{pgfscope}%
\pgfpathrectangle{\pgfqpoint{1.254980in}{0.150000in}}{\pgfqpoint{5.490039in}{5.490039in}}%
\pgfusepath{clip}%
\pgfsetbuttcap%
\pgfsetroundjoin%
\definecolor{currentfill}{rgb}{0.535621,0.835785,0.281908}%
\pgfsetfillcolor{currentfill}%
\pgfsetfillopacity{0.700000}%
\pgfsetlinewidth{0.000000pt}%
\definecolor{currentstroke}{rgb}{0.000000,0.000000,0.000000}%
\pgfsetstrokecolor{currentstroke}%
\pgfsetdash{}{0pt}%
\pgfpathmoveto{\pgfqpoint{5.355862in}{3.079688in}}%
\pgfpathlineto{\pgfqpoint{5.370620in}{3.097502in}}%
\pgfpathlineto{\pgfqpoint{5.385401in}{3.115487in}}%
\pgfpathlineto{\pgfqpoint{5.400205in}{3.133643in}}%
\pgfpathlineto{\pgfqpoint{5.407825in}{3.145099in}}%
\pgfpathlineto{\pgfqpoint{5.415435in}{3.156332in}}%
\pgfpathlineto{\pgfqpoint{5.423034in}{3.167343in}}%
\pgfpathlineto{\pgfqpoint{5.430622in}{3.178131in}}%
\pgfpathlineto{\pgfqpoint{5.415816in}{3.159928in}}%
\pgfpathlineto{\pgfqpoint{5.401033in}{3.141897in}}%
\pgfpathlineto{\pgfqpoint{5.386273in}{3.124036in}}%
\pgfpathlineto{\pgfqpoint{5.378686in}{3.113274in}}%
\pgfpathlineto{\pgfqpoint{5.371088in}{3.102295in}}%
\pgfpathlineto{\pgfqpoint{5.363480in}{3.091100in}}%
\pgfpathlineto{\pgfqpoint{5.355862in}{3.079688in}}%
\pgfpathclose%
\pgfusepath{fill}%
\end{pgfscope}%
\begin{pgfscope}%
\pgfpathrectangle{\pgfqpoint{1.254980in}{0.150000in}}{\pgfqpoint{5.490039in}{5.490039in}}%
\pgfusepath{clip}%
\pgfsetbuttcap%
\pgfsetroundjoin%
\definecolor{currentfill}{rgb}{0.282884,0.135920,0.453427}%
\pgfsetfillcolor{currentfill}%
\pgfsetfillopacity{0.700000}%
\pgfsetlinewidth{0.000000pt}%
\definecolor{currentstroke}{rgb}{0.000000,0.000000,0.000000}%
\pgfsetstrokecolor{currentstroke}%
\pgfsetdash{}{0pt}%
\pgfpathmoveto{\pgfqpoint{4.061027in}{1.173655in}}%
\pgfpathlineto{\pgfqpoint{4.074936in}{1.175648in}}%
\pgfpathlineto{\pgfqpoint{4.088855in}{1.177792in}}%
\pgfpathlineto{\pgfqpoint{4.102784in}{1.180087in}}%
\pgfpathlineto{\pgfqpoint{4.116724in}{1.182533in}}%
\pgfpathlineto{\pgfqpoint{4.124751in}{1.196239in}}%
\pgfpathlineto{\pgfqpoint{4.132774in}{1.210127in}}%
\pgfpathlineto{\pgfqpoint{4.140793in}{1.224193in}}%
\pgfpathlineto{\pgfqpoint{4.148808in}{1.238428in}}%
\pgfpathlineto{\pgfqpoint{4.134871in}{1.235270in}}%
\pgfpathlineto{\pgfqpoint{4.120945in}{1.232264in}}%
\pgfpathlineto{\pgfqpoint{4.107030in}{1.229409in}}%
\pgfpathlineto{\pgfqpoint{4.093125in}{1.226706in}}%
\pgfpathlineto{\pgfqpoint{4.085108in}{1.213171in}}%
\pgfpathlineto{\pgfqpoint{4.077086in}{1.199813in}}%
\pgfpathlineto{\pgfqpoint{4.069059in}{1.186639in}}%
\pgfpathlineto{\pgfqpoint{4.061027in}{1.173655in}}%
\pgfpathclose%
\pgfusepath{fill}%
\end{pgfscope}%
\begin{pgfscope}%
\pgfpathrectangle{\pgfqpoint{1.254980in}{0.150000in}}{\pgfqpoint{5.490039in}{5.490039in}}%
\pgfusepath{clip}%
\pgfsetbuttcap%
\pgfsetroundjoin%
\definecolor{currentfill}{rgb}{0.395174,0.797475,0.367757}%
\pgfsetfillcolor{currentfill}%
\pgfsetfillopacity{0.700000}%
\pgfsetlinewidth{0.000000pt}%
\definecolor{currentstroke}{rgb}{0.000000,0.000000,0.000000}%
\pgfsetstrokecolor{currentstroke}%
\pgfsetdash{}{0pt}%
\pgfpathmoveto{\pgfqpoint{5.235797in}{2.911867in}}%
\pgfpathlineto{\pgfqpoint{5.250456in}{2.928808in}}%
\pgfpathlineto{\pgfqpoint{5.265137in}{2.945920in}}%
\pgfpathlineto{\pgfqpoint{5.279840in}{2.963201in}}%
\pgfpathlineto{\pgfqpoint{5.294566in}{2.980652in}}%
\pgfpathlineto{\pgfqpoint{5.302261in}{2.993780in}}%
\pgfpathlineto{\pgfqpoint{5.309947in}{3.006695in}}%
\pgfpathlineto{\pgfqpoint{5.317624in}{3.019397in}}%
\pgfpathlineto{\pgfqpoint{5.325291in}{3.031885in}}%
\pgfpathlineto{\pgfqpoint{5.310560in}{3.014322in}}%
\pgfpathlineto{\pgfqpoint{5.295852in}{2.996928in}}%
\pgfpathlineto{\pgfqpoint{5.281166in}{2.979705in}}%
\pgfpathlineto{\pgfqpoint{5.266502in}{2.962651in}}%
\pgfpathlineto{\pgfqpoint{5.258840in}{2.950263in}}%
\pgfpathlineto{\pgfqpoint{5.251168in}{2.937669in}}%
\pgfpathlineto{\pgfqpoint{5.243487in}{2.924870in}}%
\pgfpathlineto{\pgfqpoint{5.235797in}{2.911867in}}%
\pgfpathclose%
\pgfusepath{fill}%
\end{pgfscope}%
\begin{pgfscope}%
\pgfpathrectangle{\pgfqpoint{1.254980in}{0.150000in}}{\pgfqpoint{5.490039in}{5.490039in}}%
\pgfusepath{clip}%
\pgfsetbuttcap%
\pgfsetroundjoin%
\definecolor{currentfill}{rgb}{0.263663,0.237631,0.518762}%
\pgfsetfillcolor{currentfill}%
\pgfsetfillopacity{0.700000}%
\pgfsetlinewidth{0.000000pt}%
\definecolor{currentstroke}{rgb}{0.000000,0.000000,0.000000}%
\pgfsetstrokecolor{currentstroke}%
\pgfsetdash{}{0pt}%
\pgfpathmoveto{\pgfqpoint{4.268648in}{1.377160in}}%
\pgfpathlineto{\pgfqpoint{4.282646in}{1.382424in}}%
\pgfpathlineto{\pgfqpoint{4.296655in}{1.387841in}}%
\pgfpathlineto{\pgfqpoint{4.310678in}{1.393411in}}%
\pgfpathlineto{\pgfqpoint{4.324714in}{1.399134in}}%
\pgfpathlineto{\pgfqpoint{4.332700in}{1.415890in}}%
\pgfpathlineto{\pgfqpoint{4.340683in}{1.432739in}}%
\pgfpathlineto{\pgfqpoint{4.348663in}{1.449673in}}%
\pgfpathlineto{\pgfqpoint{4.356640in}{1.466688in}}%
\pgfpathlineto{\pgfqpoint{4.342599in}{1.460329in}}%
\pgfpathlineto{\pgfqpoint{4.328572in}{1.454124in}}%
\pgfpathlineto{\pgfqpoint{4.314558in}{1.448073in}}%
\pgfpathlineto{\pgfqpoint{4.300557in}{1.442175in}}%
\pgfpathlineto{\pgfqpoint{4.292585in}{1.425785in}}%
\pgfpathlineto{\pgfqpoint{4.284609in}{1.409482in}}%
\pgfpathlineto{\pgfqpoint{4.276631in}{1.393271in}}%
\pgfpathlineto{\pgfqpoint{4.268648in}{1.377160in}}%
\pgfpathclose%
\pgfusepath{fill}%
\end{pgfscope}%
\begin{pgfscope}%
\pgfpathrectangle{\pgfqpoint{1.254980in}{0.150000in}}{\pgfqpoint{5.490039in}{5.490039in}}%
\pgfusepath{clip}%
\pgfsetbuttcap%
\pgfsetroundjoin%
\definecolor{currentfill}{rgb}{0.140210,0.665859,0.513427}%
\pgfsetfillcolor{currentfill}%
\pgfsetfillopacity{0.700000}%
\pgfsetlinewidth{0.000000pt}%
\definecolor{currentstroke}{rgb}{0.000000,0.000000,0.000000}%
\pgfsetstrokecolor{currentstroke}%
\pgfsetdash{}{0pt}%
\pgfpathmoveto{\pgfqpoint{4.964130in}{2.489570in}}%
\pgfpathlineto{\pgfqpoint{4.978578in}{2.504135in}}%
\pgfpathlineto{\pgfqpoint{4.993046in}{2.518866in}}%
\pgfpathlineto{\pgfqpoint{5.007535in}{2.533762in}}%
\pgfpathlineto{\pgfqpoint{5.022043in}{2.548824in}}%
\pgfpathlineto{\pgfqpoint{5.029878in}{2.565351in}}%
\pgfpathlineto{\pgfqpoint{5.037706in}{2.581713in}}%
\pgfpathlineto{\pgfqpoint{5.045528in}{2.597908in}}%
\pgfpathlineto{\pgfqpoint{5.053342in}{2.613934in}}%
\pgfpathlineto{\pgfqpoint{5.038823in}{2.598601in}}%
\pgfpathlineto{\pgfqpoint{5.024324in}{2.583434in}}%
\pgfpathlineto{\pgfqpoint{5.009846in}{2.568434in}}%
\pgfpathlineto{\pgfqpoint{4.995388in}{2.553599in}}%
\pgfpathlineto{\pgfqpoint{4.987583in}{2.537831in}}%
\pgfpathlineto{\pgfqpoint{4.979772in}{2.521903in}}%
\pgfpathlineto{\pgfqpoint{4.971954in}{2.505815in}}%
\pgfpathlineto{\pgfqpoint{4.964130in}{2.489570in}}%
\pgfpathclose%
\pgfusepath{fill}%
\end{pgfscope}%
\begin{pgfscope}%
\pgfpathrectangle{\pgfqpoint{1.254980in}{0.150000in}}{\pgfqpoint{5.490039in}{5.490039in}}%
\pgfusepath{clip}%
\pgfsetbuttcap%
\pgfsetroundjoin%
\definecolor{currentfill}{rgb}{0.235526,0.309527,0.542944}%
\pgfsetfillcolor{currentfill}%
\pgfsetfillopacity{0.700000}%
\pgfsetlinewidth{0.000000pt}%
\definecolor{currentstroke}{rgb}{0.000000,0.000000,0.000000}%
\pgfsetstrokecolor{currentstroke}%
\pgfsetdash{}{0pt}%
\pgfpathmoveto{\pgfqpoint{4.388518in}{1.535432in}}%
\pgfpathlineto{\pgfqpoint{4.402578in}{1.542554in}}%
\pgfpathlineto{\pgfqpoint{4.416653in}{1.549830in}}%
\pgfpathlineto{\pgfqpoint{4.430741in}{1.557261in}}%
\pgfpathlineto{\pgfqpoint{4.444844in}{1.564846in}}%
\pgfpathlineto{\pgfqpoint{4.452814in}{1.582768in}}%
\pgfpathlineto{\pgfqpoint{4.460780in}{1.600730in}}%
\pgfpathlineto{\pgfqpoint{4.468744in}{1.618725in}}%
\pgfpathlineto{\pgfqpoint{4.476705in}{1.636748in}}%
\pgfpathlineto{\pgfqpoint{4.462594in}{1.628578in}}%
\pgfpathlineto{\pgfqpoint{4.448498in}{1.620564in}}%
\pgfpathlineto{\pgfqpoint{4.434416in}{1.612705in}}%
\pgfpathlineto{\pgfqpoint{4.420349in}{1.605001in}}%
\pgfpathlineto{\pgfqpoint{4.412395in}{1.587551in}}%
\pgfpathlineto{\pgfqpoint{4.404439in}{1.570136in}}%
\pgfpathlineto{\pgfqpoint{4.396480in}{1.552761in}}%
\pgfpathlineto{\pgfqpoint{4.388518in}{1.535432in}}%
\pgfpathclose%
\pgfusepath{fill}%
\end{pgfscope}%
\begin{pgfscope}%
\pgfpathrectangle{\pgfqpoint{1.254980in}{0.150000in}}{\pgfqpoint{5.490039in}{5.490039in}}%
\pgfusepath{clip}%
\pgfsetbuttcap%
\pgfsetroundjoin%
\definecolor{currentfill}{rgb}{0.279574,0.170599,0.479997}%
\pgfsetfillcolor{currentfill}%
\pgfsetfillopacity{0.700000}%
\pgfsetlinewidth{0.000000pt}%
\definecolor{currentstroke}{rgb}{0.000000,0.000000,0.000000}%
\pgfsetstrokecolor{currentstroke}%
\pgfsetdash{}{0pt}%
\pgfpathmoveto{\pgfqpoint{4.148808in}{1.238428in}}%
\pgfpathlineto{\pgfqpoint{4.162756in}{1.241738in}}%
\pgfpathlineto{\pgfqpoint{4.176715in}{1.245199in}}%
\pgfpathlineto{\pgfqpoint{4.190686in}{1.248811in}}%
\pgfpathlineto{\pgfqpoint{4.204668in}{1.252575in}}%
\pgfpathlineto{\pgfqpoint{4.212678in}{1.267670in}}%
\pgfpathlineto{\pgfqpoint{4.220684in}{1.282915in}}%
\pgfpathlineto{\pgfqpoint{4.228687in}{1.298302in}}%
\pgfpathlineto{\pgfqpoint{4.236686in}{1.313826in}}%
\pgfpathlineto{\pgfqpoint{4.222703in}{1.309375in}}%
\pgfpathlineto{\pgfqpoint{4.208733in}{1.305076in}}%
\pgfpathlineto{\pgfqpoint{4.194774in}{1.300929in}}%
\pgfpathlineto{\pgfqpoint{4.180826in}{1.296934in}}%
\pgfpathlineto{\pgfqpoint{4.172828in}{1.282086in}}%
\pgfpathlineto{\pgfqpoint{4.164825in}{1.267382in}}%
\pgfpathlineto{\pgfqpoint{4.156819in}{1.252827in}}%
\pgfpathlineto{\pgfqpoint{4.148808in}{1.238428in}}%
\pgfpathclose%
\pgfusepath{fill}%
\end{pgfscope}%
\begin{pgfscope}%
\pgfpathrectangle{\pgfqpoint{1.254980in}{0.150000in}}{\pgfqpoint{5.490039in}{5.490039in}}%
\pgfusepath{clip}%
\pgfsetbuttcap%
\pgfsetroundjoin%
\definecolor{currentfill}{rgb}{0.125394,0.574318,0.549086}%
\pgfsetfillcolor{currentfill}%
\pgfsetfillopacity{0.700000}%
\pgfsetlinewidth{0.000000pt}%
\definecolor{currentstroke}{rgb}{0.000000,0.000000,0.000000}%
\pgfsetstrokecolor{currentstroke}%
\pgfsetdash{}{0pt}%
\pgfpathmoveto{\pgfqpoint{4.812334in}{2.230683in}}%
\pgfpathlineto{\pgfqpoint{4.826672in}{2.243602in}}%
\pgfpathlineto{\pgfqpoint{4.841028in}{2.256684in}}%
\pgfpathlineto{\pgfqpoint{4.855403in}{2.269928in}}%
\pgfpathlineto{\pgfqpoint{4.869797in}{2.283335in}}%
\pgfpathlineto{\pgfqpoint{4.877687in}{2.301255in}}%
\pgfpathlineto{\pgfqpoint{4.885572in}{2.319052in}}%
\pgfpathlineto{\pgfqpoint{4.893453in}{2.336722in}}%
\pgfpathlineto{\pgfqpoint{4.901328in}{2.354263in}}%
\pgfpathlineto{\pgfqpoint{4.886922in}{2.340494in}}%
\pgfpathlineto{\pgfqpoint{4.872535in}{2.326889in}}%
\pgfpathlineto{\pgfqpoint{4.858167in}{2.313447in}}%
\pgfpathlineto{\pgfqpoint{4.843818in}{2.300168in}}%
\pgfpathlineto{\pgfqpoint{4.835954in}{2.282976in}}%
\pgfpathlineto{\pgfqpoint{4.828086in}{2.265663in}}%
\pgfpathlineto{\pgfqpoint{4.820212in}{2.248231in}}%
\pgfpathlineto{\pgfqpoint{4.812334in}{2.230683in}}%
\pgfpathclose%
\pgfusepath{fill}%
\end{pgfscope}%
\begin{pgfscope}%
\pgfpathrectangle{\pgfqpoint{1.254980in}{0.150000in}}{\pgfqpoint{5.490039in}{5.490039in}}%
\pgfusepath{clip}%
\pgfsetbuttcap%
\pgfsetroundjoin%
\definecolor{currentfill}{rgb}{0.160665,0.478540,0.558115}%
\pgfsetfillcolor{currentfill}%
\pgfsetfillopacity{0.700000}%
\pgfsetlinewidth{0.000000pt}%
\definecolor{currentstroke}{rgb}{0.000000,0.000000,0.000000}%
\pgfsetstrokecolor{currentstroke}%
\pgfsetdash{}{0pt}%
\pgfpathmoveto{\pgfqpoint{4.660423in}{1.967313in}}%
\pgfpathlineto{\pgfqpoint{4.674655in}{1.978335in}}%
\pgfpathlineto{\pgfqpoint{4.688904in}{1.989515in}}%
\pgfpathlineto{\pgfqpoint{4.703170in}{2.000855in}}%
\pgfpathlineto{\pgfqpoint{4.717453in}{2.012355in}}%
\pgfpathlineto{\pgfqpoint{4.725382in}{2.031014in}}%
\pgfpathlineto{\pgfqpoint{4.733307in}{2.049601in}}%
\pgfpathlineto{\pgfqpoint{4.741228in}{2.068113in}}%
\pgfpathlineto{\pgfqpoint{4.749146in}{2.086546in}}%
\pgfpathlineto{\pgfqpoint{4.734850in}{2.074597in}}%
\pgfpathlineto{\pgfqpoint{4.720572in}{2.062809in}}%
\pgfpathlineto{\pgfqpoint{4.706312in}{2.051181in}}%
\pgfpathlineto{\pgfqpoint{4.692069in}{2.039714in}}%
\pgfpathlineto{\pgfqpoint{4.684163in}{2.021718in}}%
\pgfpathlineto{\pgfqpoint{4.676254in}{2.003650in}}%
\pgfpathlineto{\pgfqpoint{4.668340in}{1.985514in}}%
\pgfpathlineto{\pgfqpoint{4.660423in}{1.967313in}}%
\pgfpathclose%
\pgfusepath{fill}%
\end{pgfscope}%
\begin{pgfscope}%
\pgfpathrectangle{\pgfqpoint{1.254980in}{0.150000in}}{\pgfqpoint{5.490039in}{5.490039in}}%
\pgfusepath{clip}%
\pgfsetbuttcap%
\pgfsetroundjoin%
\definecolor{currentfill}{rgb}{0.201239,0.383670,0.554294}%
\pgfsetfillcolor{currentfill}%
\pgfsetfillopacity{0.700000}%
\pgfsetlinewidth{0.000000pt}%
\definecolor{currentstroke}{rgb}{0.000000,0.000000,0.000000}%
\pgfsetstrokecolor{currentstroke}%
\pgfsetdash{}{0pt}%
\pgfpathmoveto{\pgfqpoint{4.508520in}{1.709016in}}%
\pgfpathlineto{\pgfqpoint{4.522655in}{1.717899in}}%
\pgfpathlineto{\pgfqpoint{4.536806in}{1.726938in}}%
\pgfpathlineto{\pgfqpoint{4.550972in}{1.736133in}}%
\pgfpathlineto{\pgfqpoint{4.565155in}{1.745486in}}%
\pgfpathlineto{\pgfqpoint{4.573111in}{1.764112in}}%
\pgfpathlineto{\pgfqpoint{4.581064in}{1.782729in}}%
\pgfpathlineto{\pgfqpoint{4.589015in}{1.801331in}}%
\pgfpathlineto{\pgfqpoint{4.596962in}{1.819914in}}%
\pgfpathlineto{\pgfqpoint{4.582769in}{1.810030in}}%
\pgfpathlineto{\pgfqpoint{4.568593in}{1.800303in}}%
\pgfpathlineto{\pgfqpoint{4.554432in}{1.790734in}}%
\pgfpathlineto{\pgfqpoint{4.540287in}{1.781323in}}%
\pgfpathlineto{\pgfqpoint{4.532350in}{1.763260in}}%
\pgfpathlineto{\pgfqpoint{4.524410in}{1.745185in}}%
\pgfpathlineto{\pgfqpoint{4.516467in}{1.727102in}}%
\pgfpathlineto{\pgfqpoint{4.508520in}{1.709016in}}%
\pgfpathclose%
\pgfusepath{fill}%
\end{pgfscope}%
\begin{pgfscope}%
\pgfpathrectangle{\pgfqpoint{1.254980in}{0.150000in}}{\pgfqpoint{5.490039in}{5.490039in}}%
\pgfusepath{clip}%
\pgfsetbuttcap%
\pgfsetroundjoin%
\definecolor{currentfill}{rgb}{0.232815,0.732247,0.459277}%
\pgfsetfillcolor{currentfill}%
\pgfsetfillopacity{0.700000}%
\pgfsetlinewidth{0.000000pt}%
\definecolor{currentstroke}{rgb}{0.000000,0.000000,0.000000}%
\pgfsetstrokecolor{currentstroke}%
\pgfsetdash{}{0pt}%
\pgfpathmoveto{\pgfqpoint{5.084533in}{2.676302in}}%
\pgfpathlineto{\pgfqpoint{5.099082in}{2.692041in}}%
\pgfpathlineto{\pgfqpoint{5.113653in}{2.707948in}}%
\pgfpathlineto{\pgfqpoint{5.128245in}{2.724022in}}%
\pgfpathlineto{\pgfqpoint{5.142859in}{2.740265in}}%
\pgfpathlineto{\pgfqpoint{5.150648in}{2.755633in}}%
\pgfpathlineto{\pgfqpoint{5.158429in}{2.770813in}}%
\pgfpathlineto{\pgfqpoint{5.166203in}{2.785802in}}%
\pgfpathlineto{\pgfqpoint{5.173969in}{2.800598in}}%
\pgfpathlineto{\pgfqpoint{5.159346in}{2.784146in}}%
\pgfpathlineto{\pgfqpoint{5.144745in}{2.767863in}}%
\pgfpathlineto{\pgfqpoint{5.130165in}{2.751748in}}%
\pgfpathlineto{\pgfqpoint{5.115607in}{2.735800in}}%
\pgfpathlineto{\pgfqpoint{5.107850in}{2.721201in}}%
\pgfpathlineto{\pgfqpoint{5.100085in}{2.706417in}}%
\pgfpathlineto{\pgfqpoint{5.092312in}{2.691450in}}%
\pgfpathlineto{\pgfqpoint{5.084533in}{2.676302in}}%
\pgfpathclose%
\pgfusepath{fill}%
\end{pgfscope}%
\begin{pgfscope}%
\pgfpathrectangle{\pgfqpoint{1.254980in}{0.150000in}}{\pgfqpoint{5.490039in}{5.490039in}}%
\pgfusepath{clip}%
\pgfsetbuttcap%
\pgfsetroundjoin%
\definecolor{currentfill}{rgb}{0.270595,0.214069,0.507052}%
\pgfsetfillcolor{currentfill}%
\pgfsetfillopacity{0.700000}%
\pgfsetlinewidth{0.000000pt}%
\definecolor{currentstroke}{rgb}{0.000000,0.000000,0.000000}%
\pgfsetstrokecolor{currentstroke}%
\pgfsetdash{}{0pt}%
\pgfpathmoveto{\pgfqpoint{4.236686in}{1.313826in}}%
\pgfpathlineto{\pgfqpoint{4.250681in}{1.318430in}}%
\pgfpathlineto{\pgfqpoint{4.264688in}{1.323185in}}%
\pgfpathlineto{\pgfqpoint{4.278707in}{1.328092in}}%
\pgfpathlineto{\pgfqpoint{4.292739in}{1.333152in}}%
\pgfpathlineto{\pgfqpoint{4.300737in}{1.349478in}}%
\pgfpathlineto{\pgfqpoint{4.308732in}{1.365922in}}%
\pgfpathlineto{\pgfqpoint{4.316724in}{1.382475in}}%
\pgfpathlineto{\pgfqpoint{4.324714in}{1.399134in}}%
\pgfpathlineto{\pgfqpoint{4.310678in}{1.393411in}}%
\pgfpathlineto{\pgfqpoint{4.296655in}{1.387841in}}%
\pgfpathlineto{\pgfqpoint{4.282646in}{1.382424in}}%
\pgfpathlineto{\pgfqpoint{4.268648in}{1.377160in}}%
\pgfpathlineto{\pgfqpoint{4.260663in}{1.361153in}}%
\pgfpathlineto{\pgfqpoint{4.252674in}{1.345258in}}%
\pgfpathlineto{\pgfqpoint{4.244682in}{1.329480in}}%
\pgfpathlineto{\pgfqpoint{4.236686in}{1.313826in}}%
\pgfpathclose%
\pgfusepath{fill}%
\end{pgfscope}%
\begin{pgfscope}%
\pgfpathrectangle{\pgfqpoint{1.254980in}{0.150000in}}{\pgfqpoint{5.490039in}{5.490039in}}%
\pgfusepath{clip}%
\pgfsetbuttcap%
\pgfsetroundjoin%
\definecolor{currentfill}{rgb}{0.244972,0.287675,0.537260}%
\pgfsetfillcolor{currentfill}%
\pgfsetfillopacity{0.700000}%
\pgfsetlinewidth{0.000000pt}%
\definecolor{currentstroke}{rgb}{0.000000,0.000000,0.000000}%
\pgfsetstrokecolor{currentstroke}%
\pgfsetdash{}{0pt}%
\pgfpathmoveto{\pgfqpoint{4.356640in}{1.466688in}}%
\pgfpathlineto{\pgfqpoint{4.370694in}{1.473200in}}%
\pgfpathlineto{\pgfqpoint{4.384761in}{1.479866in}}%
\pgfpathlineto{\pgfqpoint{4.398843in}{1.486686in}}%
\pgfpathlineto{\pgfqpoint{4.412938in}{1.493660in}}%
\pgfpathlineto{\pgfqpoint{4.420919in}{1.511370in}}%
\pgfpathlineto{\pgfqpoint{4.428897in}{1.529141in}}%
\pgfpathlineto{\pgfqpoint{4.436872in}{1.546969in}}%
\pgfpathlineto{\pgfqpoint{4.444844in}{1.564846in}}%
\pgfpathlineto{\pgfqpoint{4.430741in}{1.557261in}}%
\pgfpathlineto{\pgfqpoint{4.416653in}{1.549830in}}%
\pgfpathlineto{\pgfqpoint{4.402578in}{1.542554in}}%
\pgfpathlineto{\pgfqpoint{4.388518in}{1.535432in}}%
\pgfpathlineto{\pgfqpoint{4.380553in}{1.518155in}}%
\pgfpathlineto{\pgfqpoint{4.372585in}{1.500934in}}%
\pgfpathlineto{\pgfqpoint{4.364614in}{1.483777in}}%
\pgfpathlineto{\pgfqpoint{4.356640in}{1.466688in}}%
\pgfpathclose%
\pgfusepath{fill}%
\end{pgfscope}%
\begin{pgfscope}%
\pgfpathrectangle{\pgfqpoint{1.254980in}{0.150000in}}{\pgfqpoint{5.490039in}{5.490039in}}%
\pgfusepath{clip}%
\pgfsetbuttcap%
\pgfsetroundjoin%
\definecolor{currentfill}{rgb}{0.128087,0.647749,0.523491}%
\pgfsetfillcolor{currentfill}%
\pgfsetfillopacity{0.700000}%
\pgfsetlinewidth{0.000000pt}%
\definecolor{currentstroke}{rgb}{0.000000,0.000000,0.000000}%
\pgfsetstrokecolor{currentstroke}%
\pgfsetdash{}{0pt}%
\pgfpathmoveto{\pgfqpoint{4.932774in}{2.423070in}}%
\pgfpathlineto{\pgfqpoint{4.947212in}{2.437334in}}%
\pgfpathlineto{\pgfqpoint{4.961669in}{2.451763in}}%
\pgfpathlineto{\pgfqpoint{4.976146in}{2.466357in}}%
\pgfpathlineto{\pgfqpoint{4.990643in}{2.481117in}}%
\pgfpathlineto{\pgfqpoint{4.998502in}{2.498278in}}%
\pgfpathlineto{\pgfqpoint{5.006355in}{2.515285in}}%
\pgfpathlineto{\pgfqpoint{5.014203in}{2.532134in}}%
\pgfpathlineto{\pgfqpoint{5.022043in}{2.548824in}}%
\pgfpathlineto{\pgfqpoint{5.007535in}{2.533762in}}%
\pgfpathlineto{\pgfqpoint{4.993046in}{2.518866in}}%
\pgfpathlineto{\pgfqpoint{4.978578in}{2.504135in}}%
\pgfpathlineto{\pgfqpoint{4.964130in}{2.489570in}}%
\pgfpathlineto{\pgfqpoint{4.956300in}{2.473170in}}%
\pgfpathlineto{\pgfqpoint{4.948464in}{2.456619in}}%
\pgfpathlineto{\pgfqpoint{4.940622in}{2.439918in}}%
\pgfpathlineto{\pgfqpoint{4.932774in}{2.423070in}}%
\pgfpathclose%
\pgfusepath{fill}%
\end{pgfscope}%
\begin{pgfscope}%
\pgfpathrectangle{\pgfqpoint{1.254980in}{0.150000in}}{\pgfqpoint{5.490039in}{5.490039in}}%
\pgfusepath{clip}%
\pgfsetbuttcap%
\pgfsetroundjoin%
\definecolor{currentfill}{rgb}{0.515992,0.831158,0.294279}%
\pgfsetfillcolor{currentfill}%
\pgfsetfillopacity{0.700000}%
\pgfsetlinewidth{0.000000pt}%
\definecolor{currentstroke}{rgb}{0.000000,0.000000,0.000000}%
\pgfsetstrokecolor{currentstroke}%
\pgfsetdash{}{0pt}%
\pgfpathmoveto{\pgfqpoint{5.325291in}{3.031885in}}%
\pgfpathlineto{\pgfqpoint{5.340045in}{3.049620in}}%
\pgfpathlineto{\pgfqpoint{5.354822in}{3.067525in}}%
\pgfpathlineto{\pgfqpoint{5.369623in}{3.085601in}}%
\pgfpathlineto{\pgfqpoint{5.377284in}{3.097944in}}%
\pgfpathlineto{\pgfqpoint{5.384934in}{3.110065in}}%
\pgfpathlineto{\pgfqpoint{5.392575in}{3.121965in}}%
\pgfpathlineto{\pgfqpoint{5.400205in}{3.133643in}}%
\pgfpathlineto{\pgfqpoint{5.385401in}{3.115487in}}%
\pgfpathlineto{\pgfqpoint{5.370620in}{3.097502in}}%
\pgfpathlineto{\pgfqpoint{5.355862in}{3.079688in}}%
\pgfpathlineto{\pgfqpoint{5.348234in}{3.068061in}}%
\pgfpathlineto{\pgfqpoint{5.340596in}{3.056218in}}%
\pgfpathlineto{\pgfqpoint{5.332949in}{3.044159in}}%
\pgfpathlineto{\pgfqpoint{5.325291in}{3.031885in}}%
\pgfpathclose%
\pgfusepath{fill}%
\end{pgfscope}%
\begin{pgfscope}%
\pgfpathrectangle{\pgfqpoint{1.254980in}{0.150000in}}{\pgfqpoint{5.490039in}{5.490039in}}%
\pgfusepath{clip}%
\pgfsetbuttcap%
\pgfsetroundjoin%
\definecolor{currentfill}{rgb}{0.369214,0.788888,0.382914}%
\pgfsetfillcolor{currentfill}%
\pgfsetfillopacity{0.700000}%
\pgfsetlinewidth{0.000000pt}%
\definecolor{currentstroke}{rgb}{0.000000,0.000000,0.000000}%
\pgfsetstrokecolor{currentstroke}%
\pgfsetdash{}{0pt}%
\pgfpathmoveto{\pgfqpoint{5.204950in}{2.857827in}}%
\pgfpathlineto{\pgfqpoint{5.219602in}{2.874624in}}%
\pgfpathlineto{\pgfqpoint{5.234277in}{2.891591in}}%
\pgfpathlineto{\pgfqpoint{5.248974in}{2.908726in}}%
\pgfpathlineto{\pgfqpoint{5.263693in}{2.926032in}}%
\pgfpathlineto{\pgfqpoint{5.271425in}{2.940001in}}%
\pgfpathlineto{\pgfqpoint{5.279147in}{2.953762in}}%
\pgfpathlineto{\pgfqpoint{5.286861in}{2.967312in}}%
\pgfpathlineto{\pgfqpoint{5.294566in}{2.980652in}}%
\pgfpathlineto{\pgfqpoint{5.279840in}{2.963201in}}%
\pgfpathlineto{\pgfqpoint{5.265137in}{2.945920in}}%
\pgfpathlineto{\pgfqpoint{5.250456in}{2.928808in}}%
\pgfpathlineto{\pgfqpoint{5.235797in}{2.911867in}}%
\pgfpathlineto{\pgfqpoint{5.228098in}{2.898660in}}%
\pgfpathlineto{\pgfqpoint{5.220391in}{2.885250in}}%
\pgfpathlineto{\pgfqpoint{5.212675in}{2.871639in}}%
\pgfpathlineto{\pgfqpoint{5.204950in}{2.857827in}}%
\pgfpathclose%
\pgfusepath{fill}%
\end{pgfscope}%
\begin{pgfscope}%
\pgfpathrectangle{\pgfqpoint{1.254980in}{0.150000in}}{\pgfqpoint{5.490039in}{5.490039in}}%
\pgfusepath{clip}%
\pgfsetbuttcap%
\pgfsetroundjoin%
\definecolor{currentfill}{rgb}{0.131172,0.555899,0.552459}%
\pgfsetfillcolor{currentfill}%
\pgfsetfillopacity{0.700000}%
\pgfsetlinewidth{0.000000pt}%
\definecolor{currentstroke}{rgb}{0.000000,0.000000,0.000000}%
\pgfsetstrokecolor{currentstroke}%
\pgfsetdash{}{0pt}%
\pgfpathmoveto{\pgfqpoint{4.780775in}{2.159402in}}%
\pgfpathlineto{\pgfqpoint{4.795100in}{2.171931in}}%
\pgfpathlineto{\pgfqpoint{4.809445in}{2.184622in}}%
\pgfpathlineto{\pgfqpoint{4.823807in}{2.197475in}}%
\pgfpathlineto{\pgfqpoint{4.838189in}{2.210490in}}%
\pgfpathlineto{\pgfqpoint{4.846098in}{2.228869in}}%
\pgfpathlineto{\pgfqpoint{4.854002in}{2.247139in}}%
\pgfpathlineto{\pgfqpoint{4.861902in}{2.265295in}}%
\pgfpathlineto{\pgfqpoint{4.869797in}{2.283335in}}%
\pgfpathlineto{\pgfqpoint{4.855403in}{2.269928in}}%
\pgfpathlineto{\pgfqpoint{4.841028in}{2.256684in}}%
\pgfpathlineto{\pgfqpoint{4.826672in}{2.243602in}}%
\pgfpathlineto{\pgfqpoint{4.812334in}{2.230683in}}%
\pgfpathlineto{\pgfqpoint{4.804451in}{2.213023in}}%
\pgfpathlineto{\pgfqpoint{4.796563in}{2.195253in}}%
\pgfpathlineto{\pgfqpoint{4.788671in}{2.177379in}}%
\pgfpathlineto{\pgfqpoint{4.780775in}{2.159402in}}%
\pgfpathclose%
\pgfusepath{fill}%
\end{pgfscope}%
\begin{pgfscope}%
\pgfpathrectangle{\pgfqpoint{1.254980in}{0.150000in}}{\pgfqpoint{5.490039in}{5.490039in}}%
\pgfusepath{clip}%
\pgfsetbuttcap%
\pgfsetroundjoin%
\definecolor{currentfill}{rgb}{0.168126,0.459988,0.558082}%
\pgfsetfillcolor{currentfill}%
\pgfsetfillopacity{0.700000}%
\pgfsetlinewidth{0.000000pt}%
\definecolor{currentstroke}{rgb}{0.000000,0.000000,0.000000}%
\pgfsetstrokecolor{currentstroke}%
\pgfsetdash{}{0pt}%
\pgfpathmoveto{\pgfqpoint{4.628720in}{1.893955in}}%
\pgfpathlineto{\pgfqpoint{4.642940in}{1.904500in}}%
\pgfpathlineto{\pgfqpoint{4.657177in}{1.915205in}}%
\pgfpathlineto{\pgfqpoint{4.671431in}{1.926068in}}%
\pgfpathlineto{\pgfqpoint{4.685702in}{1.937091in}}%
\pgfpathlineto{\pgfqpoint{4.693645in}{1.955993in}}%
\pgfpathlineto{\pgfqpoint{4.701585in}{1.974841in}}%
\pgfpathlineto{\pgfqpoint{4.709521in}{1.993630in}}%
\pgfpathlineto{\pgfqpoint{4.717453in}{2.012355in}}%
\pgfpathlineto{\pgfqpoint{4.703170in}{2.000855in}}%
\pgfpathlineto{\pgfqpoint{4.688904in}{1.989515in}}%
\pgfpathlineto{\pgfqpoint{4.674655in}{1.978335in}}%
\pgfpathlineto{\pgfqpoint{4.660423in}{1.967313in}}%
\pgfpathlineto{\pgfqpoint{4.652503in}{1.949053in}}%
\pgfpathlineto{\pgfqpoint{4.644578in}{1.930737in}}%
\pgfpathlineto{\pgfqpoint{4.636651in}{1.912369in}}%
\pgfpathlineto{\pgfqpoint{4.628720in}{1.893955in}}%
\pgfpathclose%
\pgfusepath{fill}%
\end{pgfscope}%
\begin{pgfscope}%
\pgfpathrectangle{\pgfqpoint{1.254980in}{0.150000in}}{\pgfqpoint{5.490039in}{5.490039in}}%
\pgfusepath{clip}%
\pgfsetbuttcap%
\pgfsetroundjoin%
\definecolor{currentfill}{rgb}{0.212395,0.359683,0.551710}%
\pgfsetfillcolor{currentfill}%
\pgfsetfillopacity{0.700000}%
\pgfsetlinewidth{0.000000pt}%
\definecolor{currentstroke}{rgb}{0.000000,0.000000,0.000000}%
\pgfsetstrokecolor{currentstroke}%
\pgfsetdash{}{0pt}%
\pgfpathmoveto{\pgfqpoint{4.476705in}{1.636748in}}%
\pgfpathlineto{\pgfqpoint{4.490831in}{1.645073in}}%
\pgfpathlineto{\pgfqpoint{4.504972in}{1.653554in}}%
\pgfpathlineto{\pgfqpoint{4.519129in}{1.662191in}}%
\pgfpathlineto{\pgfqpoint{4.533300in}{1.670983in}}%
\pgfpathlineto{\pgfqpoint{4.541268in}{1.689598in}}%
\pgfpathlineto{\pgfqpoint{4.549233in}{1.708224in}}%
\pgfpathlineto{\pgfqpoint{4.557195in}{1.726855in}}%
\pgfpathlineto{\pgfqpoint{4.565155in}{1.745486in}}%
\pgfpathlineto{\pgfqpoint{4.550972in}{1.736133in}}%
\pgfpathlineto{\pgfqpoint{4.536806in}{1.726938in}}%
\pgfpathlineto{\pgfqpoint{4.522655in}{1.717899in}}%
\pgfpathlineto{\pgfqpoint{4.508520in}{1.709016in}}%
\pgfpathlineto{\pgfqpoint{4.500571in}{1.690933in}}%
\pgfpathlineto{\pgfqpoint{4.492619in}{1.672857in}}%
\pgfpathlineto{\pgfqpoint{4.484663in}{1.654793in}}%
\pgfpathlineto{\pgfqpoint{4.476705in}{1.636748in}}%
\pgfpathclose%
\pgfusepath{fill}%
\end{pgfscope}%
\begin{pgfscope}%
\pgfpathrectangle{\pgfqpoint{1.254980in}{0.150000in}}{\pgfqpoint{5.490039in}{5.490039in}}%
\pgfusepath{clip}%
\pgfsetbuttcap%
\pgfsetroundjoin%
\definecolor{currentfill}{rgb}{0.281412,0.155834,0.469201}%
\pgfsetfillcolor{currentfill}%
\pgfsetfillopacity{0.700000}%
\pgfsetlinewidth{0.000000pt}%
\definecolor{currentstroke}{rgb}{0.000000,0.000000,0.000000}%
\pgfsetstrokecolor{currentstroke}%
\pgfsetdash{}{0pt}%
\pgfpathmoveto{\pgfqpoint{4.116724in}{1.182533in}}%
\pgfpathlineto{\pgfqpoint{4.130674in}{1.185130in}}%
\pgfpathlineto{\pgfqpoint{4.144635in}{1.187878in}}%
\pgfpathlineto{\pgfqpoint{4.158607in}{1.190777in}}%
\pgfpathlineto{\pgfqpoint{4.172590in}{1.193826in}}%
\pgfpathlineto{\pgfqpoint{4.180615in}{1.208255in}}%
\pgfpathlineto{\pgfqpoint{4.188636in}{1.222861in}}%
\pgfpathlineto{\pgfqpoint{4.196654in}{1.237637in}}%
\pgfpathlineto{\pgfqpoint{4.204668in}{1.252575in}}%
\pgfpathlineto{\pgfqpoint{4.190686in}{1.248811in}}%
\pgfpathlineto{\pgfqpoint{4.176715in}{1.245199in}}%
\pgfpathlineto{\pgfqpoint{4.162756in}{1.241738in}}%
\pgfpathlineto{\pgfqpoint{4.148808in}{1.238428in}}%
\pgfpathlineto{\pgfqpoint{4.140793in}{1.224193in}}%
\pgfpathlineto{\pgfqpoint{4.132774in}{1.210127in}}%
\pgfpathlineto{\pgfqpoint{4.124751in}{1.196239in}}%
\pgfpathlineto{\pgfqpoint{4.116724in}{1.182533in}}%
\pgfpathclose%
\pgfusepath{fill}%
\end{pgfscope}%
\begin{pgfscope}%
\pgfpathrectangle{\pgfqpoint{1.254980in}{0.150000in}}{\pgfqpoint{5.490039in}{5.490039in}}%
\pgfusepath{clip}%
\pgfsetbuttcap%
\pgfsetroundjoin%
\definecolor{currentfill}{rgb}{0.208030,0.718701,0.472873}%
\pgfsetfillcolor{currentfill}%
\pgfsetfillopacity{0.700000}%
\pgfsetlinewidth{0.000000pt}%
\definecolor{currentstroke}{rgb}{0.000000,0.000000,0.000000}%
\pgfsetstrokecolor{currentstroke}%
\pgfsetdash{}{0pt}%
\pgfpathmoveto{\pgfqpoint{5.053342in}{2.613934in}}%
\pgfpathlineto{\pgfqpoint{5.067882in}{2.629434in}}%
\pgfpathlineto{\pgfqpoint{5.082443in}{2.645100in}}%
\pgfpathlineto{\pgfqpoint{5.097026in}{2.660934in}}%
\pgfpathlineto{\pgfqpoint{5.111629in}{2.676934in}}%
\pgfpathlineto{\pgfqpoint{5.119447in}{2.693041in}}%
\pgfpathlineto{\pgfqpoint{5.127259in}{2.708967in}}%
\pgfpathlineto{\pgfqpoint{5.135063in}{2.724708in}}%
\pgfpathlineto{\pgfqpoint{5.142859in}{2.740265in}}%
\pgfpathlineto{\pgfqpoint{5.128245in}{2.724022in}}%
\pgfpathlineto{\pgfqpoint{5.113653in}{2.707948in}}%
\pgfpathlineto{\pgfqpoint{5.099082in}{2.692041in}}%
\pgfpathlineto{\pgfqpoint{5.084533in}{2.676302in}}%
\pgfpathlineto{\pgfqpoint{5.076746in}{2.660974in}}%
\pgfpathlineto{\pgfqpoint{5.068951in}{2.645469in}}%
\pgfpathlineto{\pgfqpoint{5.061150in}{2.629788in}}%
\pgfpathlineto{\pgfqpoint{5.053342in}{2.613934in}}%
\pgfpathclose%
\pgfusepath{fill}%
\end{pgfscope}%
\begin{pgfscope}%
\pgfpathrectangle{\pgfqpoint{1.254980in}{0.150000in}}{\pgfqpoint{5.490039in}{5.490039in}}%
\pgfusepath{clip}%
\pgfsetbuttcap%
\pgfsetroundjoin%
\definecolor{currentfill}{rgb}{0.253935,0.265254,0.529983}%
\pgfsetfillcolor{currentfill}%
\pgfsetfillopacity{0.700000}%
\pgfsetlinewidth{0.000000pt}%
\definecolor{currentstroke}{rgb}{0.000000,0.000000,0.000000}%
\pgfsetstrokecolor{currentstroke}%
\pgfsetdash{}{0pt}%
\pgfpathmoveto{\pgfqpoint{4.324714in}{1.399134in}}%
\pgfpathlineto{\pgfqpoint{4.338762in}{1.405009in}}%
\pgfpathlineto{\pgfqpoint{4.352824in}{1.411038in}}%
\pgfpathlineto{\pgfqpoint{4.366899in}{1.417220in}}%
\pgfpathlineto{\pgfqpoint{4.380988in}{1.423554in}}%
\pgfpathlineto{\pgfqpoint{4.388980in}{1.440959in}}%
\pgfpathlineto{\pgfqpoint{4.396969in}{1.458449in}}%
\pgfpathlineto{\pgfqpoint{4.404955in}{1.476018in}}%
\pgfpathlineto{\pgfqpoint{4.412938in}{1.493660in}}%
\pgfpathlineto{\pgfqpoint{4.398843in}{1.486686in}}%
\pgfpathlineto{\pgfqpoint{4.384761in}{1.479866in}}%
\pgfpathlineto{\pgfqpoint{4.370694in}{1.473200in}}%
\pgfpathlineto{\pgfqpoint{4.356640in}{1.466688in}}%
\pgfpathlineto{\pgfqpoint{4.348663in}{1.449673in}}%
\pgfpathlineto{\pgfqpoint{4.340683in}{1.432739in}}%
\pgfpathlineto{\pgfqpoint{4.332700in}{1.415890in}}%
\pgfpathlineto{\pgfqpoint{4.324714in}{1.399134in}}%
\pgfpathclose%
\pgfusepath{fill}%
\end{pgfscope}%
\begin{pgfscope}%
\pgfpathrectangle{\pgfqpoint{1.254980in}{0.150000in}}{\pgfqpoint{5.490039in}{5.490039in}}%
\pgfusepath{clip}%
\pgfsetbuttcap%
\pgfsetroundjoin%
\definecolor{currentfill}{rgb}{0.121380,0.629492,0.531973}%
\pgfsetfillcolor{currentfill}%
\pgfsetfillopacity{0.700000}%
\pgfsetlinewidth{0.000000pt}%
\definecolor{currentstroke}{rgb}{0.000000,0.000000,0.000000}%
\pgfsetstrokecolor{currentstroke}%
\pgfsetdash{}{0pt}%
\pgfpathmoveto{\pgfqpoint{4.901328in}{2.354263in}}%
\pgfpathlineto{\pgfqpoint{4.915753in}{2.368196in}}%
\pgfpathlineto{\pgfqpoint{4.930198in}{2.382293in}}%
\pgfpathlineto{\pgfqpoint{4.944663in}{2.396554in}}%
\pgfpathlineto{\pgfqpoint{4.959148in}{2.410979in}}%
\pgfpathlineto{\pgfqpoint{4.967030in}{2.428732in}}%
\pgfpathlineto{\pgfqpoint{4.974907in}{2.446341in}}%
\pgfpathlineto{\pgfqpoint{4.982778in}{2.463803in}}%
\pgfpathlineto{\pgfqpoint{4.990643in}{2.481117in}}%
\pgfpathlineto{\pgfqpoint{4.976146in}{2.466357in}}%
\pgfpathlineto{\pgfqpoint{4.961669in}{2.451763in}}%
\pgfpathlineto{\pgfqpoint{4.947212in}{2.437334in}}%
\pgfpathlineto{\pgfqpoint{4.932774in}{2.423070in}}%
\pgfpathlineto{\pgfqpoint{4.924921in}{2.406077in}}%
\pgfpathlineto{\pgfqpoint{4.917062in}{2.388943in}}%
\pgfpathlineto{\pgfqpoint{4.909198in}{2.371671in}}%
\pgfpathlineto{\pgfqpoint{4.901328in}{2.354263in}}%
\pgfpathclose%
\pgfusepath{fill}%
\end{pgfscope}%
\begin{pgfscope}%
\pgfpathrectangle{\pgfqpoint{1.254980in}{0.150000in}}{\pgfqpoint{5.490039in}{5.490039in}}%
\pgfusepath{clip}%
\pgfsetbuttcap%
\pgfsetroundjoin%
\definecolor{currentfill}{rgb}{0.139147,0.533812,0.555298}%
\pgfsetfillcolor{currentfill}%
\pgfsetfillopacity{0.700000}%
\pgfsetlinewidth{0.000000pt}%
\definecolor{currentstroke}{rgb}{0.000000,0.000000,0.000000}%
\pgfsetstrokecolor{currentstroke}%
\pgfsetdash{}{0pt}%
\pgfpathmoveto{\pgfqpoint{4.749146in}{2.086546in}}%
\pgfpathlineto{\pgfqpoint{4.763459in}{2.098655in}}%
\pgfpathlineto{\pgfqpoint{4.777791in}{2.110925in}}%
\pgfpathlineto{\pgfqpoint{4.792141in}{2.123357in}}%
\pgfpathlineto{\pgfqpoint{4.806509in}{2.135950in}}%
\pgfpathlineto{\pgfqpoint{4.814435in}{2.154731in}}%
\pgfpathlineto{\pgfqpoint{4.822357in}{2.173417in}}%
\pgfpathlineto{\pgfqpoint{4.830275in}{2.192005in}}%
\pgfpathlineto{\pgfqpoint{4.838189in}{2.210490in}}%
\pgfpathlineto{\pgfqpoint{4.823807in}{2.197475in}}%
\pgfpathlineto{\pgfqpoint{4.809445in}{2.184622in}}%
\pgfpathlineto{\pgfqpoint{4.795100in}{2.171931in}}%
\pgfpathlineto{\pgfqpoint{4.780775in}{2.159402in}}%
\pgfpathlineto{\pgfqpoint{4.772874in}{2.141326in}}%
\pgfpathlineto{\pgfqpoint{4.764969in}{2.123156in}}%
\pgfpathlineto{\pgfqpoint{4.757059in}{2.104894in}}%
\pgfpathlineto{\pgfqpoint{4.749146in}{2.086546in}}%
\pgfpathclose%
\pgfusepath{fill}%
\end{pgfscope}%
\begin{pgfscope}%
\pgfpathrectangle{\pgfqpoint{1.254980in}{0.150000in}}{\pgfqpoint{5.490039in}{5.490039in}}%
\pgfusepath{clip}%
\pgfsetbuttcap%
\pgfsetroundjoin%
\definecolor{currentfill}{rgb}{0.177423,0.437527,0.557565}%
\pgfsetfillcolor{currentfill}%
\pgfsetfillopacity{0.700000}%
\pgfsetlinewidth{0.000000pt}%
\definecolor{currentstroke}{rgb}{0.000000,0.000000,0.000000}%
\pgfsetstrokecolor{currentstroke}%
\pgfsetdash{}{0pt}%
\pgfpathmoveto{\pgfqpoint{4.596962in}{1.819914in}}%
\pgfpathlineto{\pgfqpoint{4.611171in}{1.829956in}}%
\pgfpathlineto{\pgfqpoint{4.625396in}{1.840155in}}%
\pgfpathlineto{\pgfqpoint{4.639638in}{1.850513in}}%
\pgfpathlineto{\pgfqpoint{4.653897in}{1.861029in}}%
\pgfpathlineto{\pgfqpoint{4.661853in}{1.880103in}}%
\pgfpathlineto{\pgfqpoint{4.669806in}{1.899141in}}%
\pgfpathlineto{\pgfqpoint{4.677756in}{1.918139in}}%
\pgfpathlineto{\pgfqpoint{4.685702in}{1.937091in}}%
\pgfpathlineto{\pgfqpoint{4.671431in}{1.926068in}}%
\pgfpathlineto{\pgfqpoint{4.657177in}{1.915205in}}%
\pgfpathlineto{\pgfqpoint{4.642940in}{1.904500in}}%
\pgfpathlineto{\pgfqpoint{4.628720in}{1.893955in}}%
\pgfpathlineto{\pgfqpoint{4.620785in}{1.875497in}}%
\pgfpathlineto{\pgfqpoint{4.612847in}{1.857001in}}%
\pgfpathlineto{\pgfqpoint{4.604906in}{1.838472in}}%
\pgfpathlineto{\pgfqpoint{4.596962in}{1.819914in}}%
\pgfpathclose%
\pgfusepath{fill}%
\end{pgfscope}%
\begin{pgfscope}%
\pgfpathrectangle{\pgfqpoint{1.254980in}{0.150000in}}{\pgfqpoint{5.490039in}{5.490039in}}%
\pgfusepath{clip}%
\pgfsetbuttcap%
\pgfsetroundjoin%
\definecolor{currentfill}{rgb}{0.275191,0.194905,0.496005}%
\pgfsetfillcolor{currentfill}%
\pgfsetfillopacity{0.700000}%
\pgfsetlinewidth{0.000000pt}%
\definecolor{currentstroke}{rgb}{0.000000,0.000000,0.000000}%
\pgfsetstrokecolor{currentstroke}%
\pgfsetdash{}{0pt}%
\pgfpathmoveto{\pgfqpoint{4.204668in}{1.252575in}}%
\pgfpathlineto{\pgfqpoint{4.218661in}{1.256490in}}%
\pgfpathlineto{\pgfqpoint{4.232667in}{1.260557in}}%
\pgfpathlineto{\pgfqpoint{4.246684in}{1.264775in}}%
\pgfpathlineto{\pgfqpoint{4.260713in}{1.269144in}}%
\pgfpathlineto{\pgfqpoint{4.268724in}{1.284939in}}%
\pgfpathlineto{\pgfqpoint{4.276732in}{1.300876in}}%
\pgfpathlineto{\pgfqpoint{4.284737in}{1.316949in}}%
\pgfpathlineto{\pgfqpoint{4.292739in}{1.333152in}}%
\pgfpathlineto{\pgfqpoint{4.278707in}{1.328092in}}%
\pgfpathlineto{\pgfqpoint{4.264688in}{1.323185in}}%
\pgfpathlineto{\pgfqpoint{4.250681in}{1.318430in}}%
\pgfpathlineto{\pgfqpoint{4.236686in}{1.313826in}}%
\pgfpathlineto{\pgfqpoint{4.228687in}{1.298302in}}%
\pgfpathlineto{\pgfqpoint{4.220684in}{1.282915in}}%
\pgfpathlineto{\pgfqpoint{4.212678in}{1.267670in}}%
\pgfpathlineto{\pgfqpoint{4.204668in}{1.252575in}}%
\pgfpathclose%
\pgfusepath{fill}%
\end{pgfscope}%
\begin{pgfscope}%
\pgfpathrectangle{\pgfqpoint{1.254980in}{0.150000in}}{\pgfqpoint{5.490039in}{5.490039in}}%
\pgfusepath{clip}%
\pgfsetbuttcap%
\pgfsetroundjoin%
\definecolor{currentfill}{rgb}{0.221989,0.339161,0.548752}%
\pgfsetfillcolor{currentfill}%
\pgfsetfillopacity{0.700000}%
\pgfsetlinewidth{0.000000pt}%
\definecolor{currentstroke}{rgb}{0.000000,0.000000,0.000000}%
\pgfsetstrokecolor{currentstroke}%
\pgfsetdash{}{0pt}%
\pgfpathmoveto{\pgfqpoint{4.444844in}{1.564846in}}%
\pgfpathlineto{\pgfqpoint{4.458961in}{1.572587in}}%
\pgfpathlineto{\pgfqpoint{4.473093in}{1.580482in}}%
\pgfpathlineto{\pgfqpoint{4.487240in}{1.588532in}}%
\pgfpathlineto{\pgfqpoint{4.501402in}{1.596737in}}%
\pgfpathlineto{\pgfqpoint{4.509381in}{1.615256in}}%
\pgfpathlineto{\pgfqpoint{4.517357in}{1.633807in}}%
\pgfpathlineto{\pgfqpoint{4.525330in}{1.652385in}}%
\pgfpathlineto{\pgfqpoint{4.533300in}{1.670983in}}%
\pgfpathlineto{\pgfqpoint{4.519129in}{1.662191in}}%
\pgfpathlineto{\pgfqpoint{4.504972in}{1.653554in}}%
\pgfpathlineto{\pgfqpoint{4.490831in}{1.645073in}}%
\pgfpathlineto{\pgfqpoint{4.476705in}{1.636748in}}%
\pgfpathlineto{\pgfqpoint{4.468744in}{1.618725in}}%
\pgfpathlineto{\pgfqpoint{4.460780in}{1.600730in}}%
\pgfpathlineto{\pgfqpoint{4.452814in}{1.582768in}}%
\pgfpathlineto{\pgfqpoint{4.444844in}{1.564846in}}%
\pgfpathclose%
\pgfusepath{fill}%
\end{pgfscope}%
\begin{pgfscope}%
\pgfpathrectangle{\pgfqpoint{1.254980in}{0.150000in}}{\pgfqpoint{5.490039in}{5.490039in}}%
\pgfusepath{clip}%
\pgfsetbuttcap%
\pgfsetroundjoin%
\definecolor{currentfill}{rgb}{0.344074,0.780029,0.397381}%
\pgfsetfillcolor{currentfill}%
\pgfsetfillopacity{0.700000}%
\pgfsetlinewidth{0.000000pt}%
\definecolor{currentstroke}{rgb}{0.000000,0.000000,0.000000}%
\pgfsetstrokecolor{currentstroke}%
\pgfsetdash{}{0pt}%
\pgfpathmoveto{\pgfqpoint{5.173969in}{2.800598in}}%
\pgfpathlineto{\pgfqpoint{5.188613in}{2.817218in}}%
\pgfpathlineto{\pgfqpoint{5.203279in}{2.834007in}}%
\pgfpathlineto{\pgfqpoint{5.217968in}{2.850965in}}%
\pgfpathlineto{\pgfqpoint{5.232679in}{2.868092in}}%
\pgfpathlineto{\pgfqpoint{5.240445in}{2.882884in}}%
\pgfpathlineto{\pgfqpoint{5.248203in}{2.897472in}}%
\pgfpathlineto{\pgfqpoint{5.255952in}{2.911855in}}%
\pgfpathlineto{\pgfqpoint{5.263693in}{2.926032in}}%
\pgfpathlineto{\pgfqpoint{5.248974in}{2.908726in}}%
\pgfpathlineto{\pgfqpoint{5.234277in}{2.891591in}}%
\pgfpathlineto{\pgfqpoint{5.219602in}{2.874624in}}%
\pgfpathlineto{\pgfqpoint{5.204950in}{2.857827in}}%
\pgfpathlineto{\pgfqpoint{5.197217in}{2.843816in}}%
\pgfpathlineto{\pgfqpoint{5.189476in}{2.829606in}}%
\pgfpathlineto{\pgfqpoint{5.181726in}{2.815200in}}%
\pgfpathlineto{\pgfqpoint{5.173969in}{2.800598in}}%
\pgfpathclose%
\pgfusepath{fill}%
\end{pgfscope}%
\begin{pgfscope}%
\pgfpathrectangle{\pgfqpoint{1.254980in}{0.150000in}}{\pgfqpoint{5.490039in}{5.490039in}}%
\pgfusepath{clip}%
\pgfsetbuttcap%
\pgfsetroundjoin%
\definecolor{currentfill}{rgb}{0.487026,0.823929,0.312321}%
\pgfsetfillcolor{currentfill}%
\pgfsetfillopacity{0.700000}%
\pgfsetlinewidth{0.000000pt}%
\definecolor{currentstroke}{rgb}{0.000000,0.000000,0.000000}%
\pgfsetstrokecolor{currentstroke}%
\pgfsetdash{}{0pt}%
\pgfpathmoveto{\pgfqpoint{5.294566in}{2.980652in}}%
\pgfpathlineto{\pgfqpoint{5.309315in}{2.998273in}}%
\pgfpathlineto{\pgfqpoint{5.324086in}{3.016066in}}%
\pgfpathlineto{\pgfqpoint{5.338881in}{3.034029in}}%
\pgfpathlineto{\pgfqpoint{5.346581in}{3.047251in}}%
\pgfpathlineto{\pgfqpoint{5.354272in}{3.060254in}}%
\pgfpathlineto{\pgfqpoint{5.361952in}{3.073038in}}%
\pgfpathlineto{\pgfqpoint{5.369623in}{3.085601in}}%
\pgfpathlineto{\pgfqpoint{5.354822in}{3.067525in}}%
\pgfpathlineto{\pgfqpoint{5.340045in}{3.049620in}}%
\pgfpathlineto{\pgfqpoint{5.325291in}{3.031885in}}%
\pgfpathlineto{\pgfqpoint{5.317624in}{3.019397in}}%
\pgfpathlineto{\pgfqpoint{5.309947in}{3.006695in}}%
\pgfpathlineto{\pgfqpoint{5.302261in}{2.993780in}}%
\pgfpathlineto{\pgfqpoint{5.294566in}{2.980652in}}%
\pgfpathclose%
\pgfusepath{fill}%
\end{pgfscope}%
\begin{pgfscope}%
\pgfpathrectangle{\pgfqpoint{1.254980in}{0.150000in}}{\pgfqpoint{5.490039in}{5.490039in}}%
\pgfusepath{clip}%
\pgfsetbuttcap%
\pgfsetroundjoin%
\definecolor{currentfill}{rgb}{0.180653,0.701402,0.488189}%
\pgfsetfillcolor{currentfill}%
\pgfsetfillopacity{0.700000}%
\pgfsetlinewidth{0.000000pt}%
\definecolor{currentstroke}{rgb}{0.000000,0.000000,0.000000}%
\pgfsetstrokecolor{currentstroke}%
\pgfsetdash{}{0pt}%
\pgfpathmoveto{\pgfqpoint{5.022043in}{2.548824in}}%
\pgfpathlineto{\pgfqpoint{5.036573in}{2.564052in}}%
\pgfpathlineto{\pgfqpoint{5.051123in}{2.579446in}}%
\pgfpathlineto{\pgfqpoint{5.065694in}{2.595007in}}%
\pgfpathlineto{\pgfqpoint{5.080285in}{2.610734in}}%
\pgfpathlineto{\pgfqpoint{5.088132in}{2.627546in}}%
\pgfpathlineto{\pgfqpoint{5.095971in}{2.644184in}}%
\pgfpathlineto{\pgfqpoint{5.103803in}{2.660648in}}%
\pgfpathlineto{\pgfqpoint{5.111629in}{2.676934in}}%
\pgfpathlineto{\pgfqpoint{5.097026in}{2.660934in}}%
\pgfpathlineto{\pgfqpoint{5.082443in}{2.645100in}}%
\pgfpathlineto{\pgfqpoint{5.067882in}{2.629434in}}%
\pgfpathlineto{\pgfqpoint{5.053342in}{2.613934in}}%
\pgfpathlineto{\pgfqpoint{5.045528in}{2.597908in}}%
\pgfpathlineto{\pgfqpoint{5.037706in}{2.581713in}}%
\pgfpathlineto{\pgfqpoint{5.029878in}{2.565351in}}%
\pgfpathlineto{\pgfqpoint{5.022043in}{2.548824in}}%
\pgfpathclose%
\pgfusepath{fill}%
\end{pgfscope}%
\begin{pgfscope}%
\pgfpathrectangle{\pgfqpoint{1.254980in}{0.150000in}}{\pgfqpoint{5.490039in}{5.490039in}}%
\pgfusepath{clip}%
\pgfsetbuttcap%
\pgfsetroundjoin%
\definecolor{currentfill}{rgb}{0.187231,0.414746,0.556547}%
\pgfsetfillcolor{currentfill}%
\pgfsetfillopacity{0.700000}%
\pgfsetlinewidth{0.000000pt}%
\definecolor{currentstroke}{rgb}{0.000000,0.000000,0.000000}%
\pgfsetstrokecolor{currentstroke}%
\pgfsetdash{}{0pt}%
\pgfpathmoveto{\pgfqpoint{4.565155in}{1.745486in}}%
\pgfpathlineto{\pgfqpoint{4.579352in}{1.754995in}}%
\pgfpathlineto{\pgfqpoint{4.593566in}{1.764662in}}%
\pgfpathlineto{\pgfqpoint{4.607796in}{1.774485in}}%
\pgfpathlineto{\pgfqpoint{4.622043in}{1.784466in}}%
\pgfpathlineto{\pgfqpoint{4.630011in}{1.803637in}}%
\pgfpathlineto{\pgfqpoint{4.637976in}{1.822791in}}%
\pgfpathlineto{\pgfqpoint{4.645938in}{1.841923in}}%
\pgfpathlineto{\pgfqpoint{4.653897in}{1.861029in}}%
\pgfpathlineto{\pgfqpoint{4.639638in}{1.850513in}}%
\pgfpathlineto{\pgfqpoint{4.625396in}{1.840155in}}%
\pgfpathlineto{\pgfqpoint{4.611171in}{1.829956in}}%
\pgfpathlineto{\pgfqpoint{4.596962in}{1.819914in}}%
\pgfpathlineto{\pgfqpoint{4.589015in}{1.801331in}}%
\pgfpathlineto{\pgfqpoint{4.581064in}{1.782729in}}%
\pgfpathlineto{\pgfqpoint{4.573111in}{1.764112in}}%
\pgfpathlineto{\pgfqpoint{4.565155in}{1.745486in}}%
\pgfpathclose%
\pgfusepath{fill}%
\end{pgfscope}%
\begin{pgfscope}%
\pgfpathrectangle{\pgfqpoint{1.254980in}{0.150000in}}{\pgfqpoint{5.490039in}{5.490039in}}%
\pgfusepath{clip}%
\pgfsetbuttcap%
\pgfsetroundjoin%
\definecolor{currentfill}{rgb}{0.147607,0.511733,0.557049}%
\pgfsetfillcolor{currentfill}%
\pgfsetfillopacity{0.700000}%
\pgfsetlinewidth{0.000000pt}%
\definecolor{currentstroke}{rgb}{0.000000,0.000000,0.000000}%
\pgfsetstrokecolor{currentstroke}%
\pgfsetdash{}{0pt}%
\pgfpathmoveto{\pgfqpoint{4.717453in}{2.012355in}}%
\pgfpathlineto{\pgfqpoint{4.731754in}{2.024016in}}%
\pgfpathlineto{\pgfqpoint{4.746073in}{2.035836in}}%
\pgfpathlineto{\pgfqpoint{4.760409in}{2.047817in}}%
\pgfpathlineto{\pgfqpoint{4.774764in}{2.059958in}}%
\pgfpathlineto{\pgfqpoint{4.782706in}{2.079078in}}%
\pgfpathlineto{\pgfqpoint{4.790644in}{2.098120in}}%
\pgfpathlineto{\pgfqpoint{4.798579in}{2.117078in}}%
\pgfpathlineto{\pgfqpoint{4.806509in}{2.135950in}}%
\pgfpathlineto{\pgfqpoint{4.792141in}{2.123357in}}%
\pgfpathlineto{\pgfqpoint{4.777791in}{2.110925in}}%
\pgfpathlineto{\pgfqpoint{4.763459in}{2.098655in}}%
\pgfpathlineto{\pgfqpoint{4.749146in}{2.086546in}}%
\pgfpathlineto{\pgfqpoint{4.741228in}{2.068113in}}%
\pgfpathlineto{\pgfqpoint{4.733307in}{2.049601in}}%
\pgfpathlineto{\pgfqpoint{4.725382in}{2.031014in}}%
\pgfpathlineto{\pgfqpoint{4.717453in}{2.012355in}}%
\pgfpathclose%
\pgfusepath{fill}%
\end{pgfscope}%
\begin{pgfscope}%
\pgfpathrectangle{\pgfqpoint{1.254980in}{0.150000in}}{\pgfqpoint{5.490039in}{5.490039in}}%
\pgfusepath{clip}%
\pgfsetbuttcap%
\pgfsetroundjoin%
\definecolor{currentfill}{rgb}{0.119512,0.607464,0.540218}%
\pgfsetfillcolor{currentfill}%
\pgfsetfillopacity{0.700000}%
\pgfsetlinewidth{0.000000pt}%
\definecolor{currentstroke}{rgb}{0.000000,0.000000,0.000000}%
\pgfsetstrokecolor{currentstroke}%
\pgfsetdash{}{0pt}%
\pgfpathmoveto{\pgfqpoint{4.869797in}{2.283335in}}%
\pgfpathlineto{\pgfqpoint{4.884210in}{2.296905in}}%
\pgfpathlineto{\pgfqpoint{4.898643in}{2.310639in}}%
\pgfpathlineto{\pgfqpoint{4.913094in}{2.324536in}}%
\pgfpathlineto{\pgfqpoint{4.927566in}{2.338597in}}%
\pgfpathlineto{\pgfqpoint{4.935469in}{2.356892in}}%
\pgfpathlineto{\pgfqpoint{4.943367in}{2.375057in}}%
\pgfpathlineto{\pgfqpoint{4.951260in}{2.393087in}}%
\pgfpathlineto{\pgfqpoint{4.959148in}{2.410979in}}%
\pgfpathlineto{\pgfqpoint{4.944663in}{2.396554in}}%
\pgfpathlineto{\pgfqpoint{4.930198in}{2.382293in}}%
\pgfpathlineto{\pgfqpoint{4.915753in}{2.368196in}}%
\pgfpathlineto{\pgfqpoint{4.901328in}{2.354263in}}%
\pgfpathlineto{\pgfqpoint{4.893453in}{2.336722in}}%
\pgfpathlineto{\pgfqpoint{4.885572in}{2.319052in}}%
\pgfpathlineto{\pgfqpoint{4.877687in}{2.301255in}}%
\pgfpathlineto{\pgfqpoint{4.869797in}{2.283335in}}%
\pgfpathclose%
\pgfusepath{fill}%
\end{pgfscope}%
\begin{pgfscope}%
\pgfpathrectangle{\pgfqpoint{1.254980in}{0.150000in}}{\pgfqpoint{5.490039in}{5.490039in}}%
\pgfusepath{clip}%
\pgfsetbuttcap%
\pgfsetroundjoin%
\definecolor{currentfill}{rgb}{0.262138,0.242286,0.520837}%
\pgfsetfillcolor{currentfill}%
\pgfsetfillopacity{0.700000}%
\pgfsetlinewidth{0.000000pt}%
\definecolor{currentstroke}{rgb}{0.000000,0.000000,0.000000}%
\pgfsetstrokecolor{currentstroke}%
\pgfsetdash{}{0pt}%
\pgfpathmoveto{\pgfqpoint{4.292739in}{1.333152in}}%
\pgfpathlineto{\pgfqpoint{4.306783in}{1.338364in}}%
\pgfpathlineto{\pgfqpoint{4.320840in}{1.343728in}}%
\pgfpathlineto{\pgfqpoint{4.334910in}{1.349244in}}%
\pgfpathlineto{\pgfqpoint{4.348993in}{1.354912in}}%
\pgfpathlineto{\pgfqpoint{4.356996in}{1.371914in}}%
\pgfpathlineto{\pgfqpoint{4.364996in}{1.389026in}}%
\pgfpathlineto{\pgfqpoint{4.372994in}{1.406241in}}%
\pgfpathlineto{\pgfqpoint{4.380988in}{1.423554in}}%
\pgfpathlineto{\pgfqpoint{4.366899in}{1.417220in}}%
\pgfpathlineto{\pgfqpoint{4.352824in}{1.411038in}}%
\pgfpathlineto{\pgfqpoint{4.338762in}{1.405009in}}%
\pgfpathlineto{\pgfqpoint{4.324714in}{1.399134in}}%
\pgfpathlineto{\pgfqpoint{4.316724in}{1.382475in}}%
\pgfpathlineto{\pgfqpoint{4.308732in}{1.365922in}}%
\pgfpathlineto{\pgfqpoint{4.300737in}{1.349478in}}%
\pgfpathlineto{\pgfqpoint{4.292739in}{1.333152in}}%
\pgfpathclose%
\pgfusepath{fill}%
\end{pgfscope}%
\begin{pgfscope}%
\pgfpathrectangle{\pgfqpoint{1.254980in}{0.150000in}}{\pgfqpoint{5.490039in}{5.490039in}}%
\pgfusepath{clip}%
\pgfsetbuttcap%
\pgfsetroundjoin%
\definecolor{currentfill}{rgb}{0.233603,0.313828,0.543914}%
\pgfsetfillcolor{currentfill}%
\pgfsetfillopacity{0.700000}%
\pgfsetlinewidth{0.000000pt}%
\definecolor{currentstroke}{rgb}{0.000000,0.000000,0.000000}%
\pgfsetstrokecolor{currentstroke}%
\pgfsetdash{}{0pt}%
\pgfpathmoveto{\pgfqpoint{4.412938in}{1.493660in}}%
\pgfpathlineto{\pgfqpoint{4.427048in}{1.500788in}}%
\pgfpathlineto{\pgfqpoint{4.441171in}{1.508070in}}%
\pgfpathlineto{\pgfqpoint{4.455309in}{1.515505in}}%
\pgfpathlineto{\pgfqpoint{4.469462in}{1.523095in}}%
\pgfpathlineto{\pgfqpoint{4.477451in}{1.541429in}}%
\pgfpathlineto{\pgfqpoint{4.485437in}{1.559818in}}%
\pgfpathlineto{\pgfqpoint{4.493421in}{1.578256in}}%
\pgfpathlineto{\pgfqpoint{4.501402in}{1.596737in}}%
\pgfpathlineto{\pgfqpoint{4.487240in}{1.588532in}}%
\pgfpathlineto{\pgfqpoint{4.473093in}{1.580482in}}%
\pgfpathlineto{\pgfqpoint{4.458961in}{1.572587in}}%
\pgfpathlineto{\pgfqpoint{4.444844in}{1.564846in}}%
\pgfpathlineto{\pgfqpoint{4.436872in}{1.546969in}}%
\pgfpathlineto{\pgfqpoint{4.428897in}{1.529141in}}%
\pgfpathlineto{\pgfqpoint{4.420919in}{1.511370in}}%
\pgfpathlineto{\pgfqpoint{4.412938in}{1.493660in}}%
\pgfpathclose%
\pgfusepath{fill}%
\end{pgfscope}%
\begin{pgfscope}%
\pgfpathrectangle{\pgfqpoint{1.254980in}{0.150000in}}{\pgfqpoint{5.490039in}{5.490039in}}%
\pgfusepath{clip}%
\pgfsetbuttcap%
\pgfsetroundjoin%
\definecolor{currentfill}{rgb}{0.278826,0.175490,0.483397}%
\pgfsetfillcolor{currentfill}%
\pgfsetfillopacity{0.700000}%
\pgfsetlinewidth{0.000000pt}%
\definecolor{currentstroke}{rgb}{0.000000,0.000000,0.000000}%
\pgfsetstrokecolor{currentstroke}%
\pgfsetdash{}{0pt}%
\pgfpathmoveto{\pgfqpoint{4.172590in}{1.193826in}}%
\pgfpathlineto{\pgfqpoint{4.186584in}{1.197026in}}%
\pgfpathlineto{\pgfqpoint{4.200589in}{1.200377in}}%
\pgfpathlineto{\pgfqpoint{4.214606in}{1.203878in}}%
\pgfpathlineto{\pgfqpoint{4.228634in}{1.207531in}}%
\pgfpathlineto{\pgfqpoint{4.236659in}{1.222686in}}%
\pgfpathlineto{\pgfqpoint{4.244680in}{1.238011in}}%
\pgfpathlineto{\pgfqpoint{4.252698in}{1.253500in}}%
\pgfpathlineto{\pgfqpoint{4.260713in}{1.269144in}}%
\pgfpathlineto{\pgfqpoint{4.246684in}{1.264775in}}%
\pgfpathlineto{\pgfqpoint{4.232667in}{1.260557in}}%
\pgfpathlineto{\pgfqpoint{4.218661in}{1.256490in}}%
\pgfpathlineto{\pgfqpoint{4.204668in}{1.252575in}}%
\pgfpathlineto{\pgfqpoint{4.196654in}{1.237637in}}%
\pgfpathlineto{\pgfqpoint{4.188636in}{1.222861in}}%
\pgfpathlineto{\pgfqpoint{4.180615in}{1.208255in}}%
\pgfpathlineto{\pgfqpoint{4.172590in}{1.193826in}}%
\pgfpathclose%
\pgfusepath{fill}%
\end{pgfscope}%
\begin{pgfscope}%
\pgfpathrectangle{\pgfqpoint{1.254980in}{0.150000in}}{\pgfqpoint{5.490039in}{5.490039in}}%
\pgfusepath{clip}%
\pgfsetbuttcap%
\pgfsetroundjoin%
\definecolor{currentfill}{rgb}{0.311925,0.767822,0.415586}%
\pgfsetfillcolor{currentfill}%
\pgfsetfillopacity{0.700000}%
\pgfsetlinewidth{0.000000pt}%
\definecolor{currentstroke}{rgb}{0.000000,0.000000,0.000000}%
\pgfsetstrokecolor{currentstroke}%
\pgfsetdash{}{0pt}%
\pgfpathmoveto{\pgfqpoint{5.142859in}{2.740265in}}%
\pgfpathlineto{\pgfqpoint{5.157494in}{2.756675in}}%
\pgfpathlineto{\pgfqpoint{5.172151in}{2.773254in}}%
\pgfpathlineto{\pgfqpoint{5.186830in}{2.790002in}}%
\pgfpathlineto{\pgfqpoint{5.201531in}{2.806918in}}%
\pgfpathlineto{\pgfqpoint{5.209330in}{2.822509in}}%
\pgfpathlineto{\pgfqpoint{5.217121in}{2.837903in}}%
\pgfpathlineto{\pgfqpoint{5.224904in}{2.853098in}}%
\pgfpathlineto{\pgfqpoint{5.232679in}{2.868092in}}%
\pgfpathlineto{\pgfqpoint{5.217968in}{2.850965in}}%
\pgfpathlineto{\pgfqpoint{5.203279in}{2.834007in}}%
\pgfpathlineto{\pgfqpoint{5.188613in}{2.817218in}}%
\pgfpathlineto{\pgfqpoint{5.173969in}{2.800598in}}%
\pgfpathlineto{\pgfqpoint{5.166203in}{2.785802in}}%
\pgfpathlineto{\pgfqpoint{5.158429in}{2.770813in}}%
\pgfpathlineto{\pgfqpoint{5.150648in}{2.755633in}}%
\pgfpathlineto{\pgfqpoint{5.142859in}{2.740265in}}%
\pgfpathclose%
\pgfusepath{fill}%
\end{pgfscope}%
\begin{pgfscope}%
\pgfpathrectangle{\pgfqpoint{1.254980in}{0.150000in}}{\pgfqpoint{5.490039in}{5.490039in}}%
\pgfusepath{clip}%
\pgfsetbuttcap%
\pgfsetroundjoin%
\definecolor{currentfill}{rgb}{0.197636,0.391528,0.554969}%
\pgfsetfillcolor{currentfill}%
\pgfsetfillopacity{0.700000}%
\pgfsetlinewidth{0.000000pt}%
\definecolor{currentstroke}{rgb}{0.000000,0.000000,0.000000}%
\pgfsetstrokecolor{currentstroke}%
\pgfsetdash{}{0pt}%
\pgfpathmoveto{\pgfqpoint{4.533300in}{1.670983in}}%
\pgfpathlineto{\pgfqpoint{4.547487in}{1.679932in}}%
\pgfpathlineto{\pgfqpoint{4.561690in}{1.689037in}}%
\pgfpathlineto{\pgfqpoint{4.575909in}{1.698298in}}%
\pgfpathlineto{\pgfqpoint{4.590143in}{1.707716in}}%
\pgfpathlineto{\pgfqpoint{4.598122in}{1.726903in}}%
\pgfpathlineto{\pgfqpoint{4.606099in}{1.746094in}}%
\pgfpathlineto{\pgfqpoint{4.614072in}{1.765283in}}%
\pgfpathlineto{\pgfqpoint{4.622043in}{1.784466in}}%
\pgfpathlineto{\pgfqpoint{4.607796in}{1.774485in}}%
\pgfpathlineto{\pgfqpoint{4.593566in}{1.764662in}}%
\pgfpathlineto{\pgfqpoint{4.579352in}{1.754995in}}%
\pgfpathlineto{\pgfqpoint{4.565155in}{1.745486in}}%
\pgfpathlineto{\pgfqpoint{4.557195in}{1.726855in}}%
\pgfpathlineto{\pgfqpoint{4.549233in}{1.708224in}}%
\pgfpathlineto{\pgfqpoint{4.541268in}{1.689598in}}%
\pgfpathlineto{\pgfqpoint{4.533300in}{1.670983in}}%
\pgfpathclose%
\pgfusepath{fill}%
\end{pgfscope}%
\begin{pgfscope}%
\pgfpathrectangle{\pgfqpoint{1.254980in}{0.150000in}}{\pgfqpoint{5.490039in}{5.490039in}}%
\pgfusepath{clip}%
\pgfsetbuttcap%
\pgfsetroundjoin%
\definecolor{currentfill}{rgb}{0.156270,0.489624,0.557936}%
\pgfsetfillcolor{currentfill}%
\pgfsetfillopacity{0.700000}%
\pgfsetlinewidth{0.000000pt}%
\definecolor{currentstroke}{rgb}{0.000000,0.000000,0.000000}%
\pgfsetstrokecolor{currentstroke}%
\pgfsetdash{}{0pt}%
\pgfpathmoveto{\pgfqpoint{4.685702in}{1.937091in}}%
\pgfpathlineto{\pgfqpoint{4.699990in}{1.948272in}}%
\pgfpathlineto{\pgfqpoint{4.714296in}{1.959613in}}%
\pgfpathlineto{\pgfqpoint{4.728619in}{1.971114in}}%
\pgfpathlineto{\pgfqpoint{4.742960in}{1.982774in}}%
\pgfpathlineto{\pgfqpoint{4.750916in}{2.002167in}}%
\pgfpathlineto{\pgfqpoint{4.758869in}{2.021498in}}%
\pgfpathlineto{\pgfqpoint{4.766818in}{2.040763in}}%
\pgfpathlineto{\pgfqpoint{4.774764in}{2.059958in}}%
\pgfpathlineto{\pgfqpoint{4.760409in}{2.047817in}}%
\pgfpathlineto{\pgfqpoint{4.746073in}{2.035836in}}%
\pgfpathlineto{\pgfqpoint{4.731754in}{2.024016in}}%
\pgfpathlineto{\pgfqpoint{4.717453in}{2.012355in}}%
\pgfpathlineto{\pgfqpoint{4.709521in}{1.993630in}}%
\pgfpathlineto{\pgfqpoint{4.701585in}{1.974841in}}%
\pgfpathlineto{\pgfqpoint{4.693645in}{1.955993in}}%
\pgfpathlineto{\pgfqpoint{4.685702in}{1.937091in}}%
\pgfpathclose%
\pgfusepath{fill}%
\end{pgfscope}%
\begin{pgfscope}%
\pgfpathrectangle{\pgfqpoint{1.254980in}{0.150000in}}{\pgfqpoint{5.490039in}{5.490039in}}%
\pgfusepath{clip}%
\pgfsetbuttcap%
\pgfsetroundjoin%
\definecolor{currentfill}{rgb}{0.458674,0.816363,0.329727}%
\pgfsetfillcolor{currentfill}%
\pgfsetfillopacity{0.700000}%
\pgfsetlinewidth{0.000000pt}%
\definecolor{currentstroke}{rgb}{0.000000,0.000000,0.000000}%
\pgfsetstrokecolor{currentstroke}%
\pgfsetdash{}{0pt}%
\pgfpathmoveto{\pgfqpoint{5.263693in}{2.926032in}}%
\pgfpathlineto{\pgfqpoint{5.278435in}{2.943508in}}%
\pgfpathlineto{\pgfqpoint{5.293199in}{2.961154in}}%
\pgfpathlineto{\pgfqpoint{5.307987in}{2.978972in}}%
\pgfpathlineto{\pgfqpoint{5.315724in}{2.993060in}}%
\pgfpathlineto{\pgfqpoint{5.323453in}{3.006932in}}%
\pgfpathlineto{\pgfqpoint{5.331172in}{3.020589in}}%
\pgfpathlineto{\pgfqpoint{5.338881in}{3.034029in}}%
\pgfpathlineto{\pgfqpoint{5.324086in}{3.016066in}}%
\pgfpathlineto{\pgfqpoint{5.309315in}{2.998273in}}%
\pgfpathlineto{\pgfqpoint{5.294566in}{2.980652in}}%
\pgfpathlineto{\pgfqpoint{5.286861in}{2.967312in}}%
\pgfpathlineto{\pgfqpoint{5.279147in}{2.953762in}}%
\pgfpathlineto{\pgfqpoint{5.271425in}{2.940001in}}%
\pgfpathlineto{\pgfqpoint{5.263693in}{2.926032in}}%
\pgfpathclose%
\pgfusepath{fill}%
\end{pgfscope}%
\begin{pgfscope}%
\pgfpathrectangle{\pgfqpoint{1.254980in}{0.150000in}}{\pgfqpoint{5.490039in}{5.490039in}}%
\pgfusepath{clip}%
\pgfsetbuttcap%
\pgfsetroundjoin%
\definecolor{currentfill}{rgb}{0.121831,0.589055,0.545623}%
\pgfsetfillcolor{currentfill}%
\pgfsetfillopacity{0.700000}%
\pgfsetlinewidth{0.000000pt}%
\definecolor{currentstroke}{rgb}{0.000000,0.000000,0.000000}%
\pgfsetstrokecolor{currentstroke}%
\pgfsetdash{}{0pt}%
\pgfpathmoveto{\pgfqpoint{4.838189in}{2.210490in}}%
\pgfpathlineto{\pgfqpoint{4.852589in}{2.223667in}}%
\pgfpathlineto{\pgfqpoint{4.867008in}{2.237007in}}%
\pgfpathlineto{\pgfqpoint{4.881446in}{2.250510in}}%
\pgfpathlineto{\pgfqpoint{4.895904in}{2.264176in}}%
\pgfpathlineto{\pgfqpoint{4.903826in}{2.282961in}}%
\pgfpathlineto{\pgfqpoint{4.911744in}{2.301628in}}%
\pgfpathlineto{\pgfqpoint{4.919658in}{2.320175in}}%
\pgfpathlineto{\pgfqpoint{4.927566in}{2.338597in}}%
\pgfpathlineto{\pgfqpoint{4.913094in}{2.324536in}}%
\pgfpathlineto{\pgfqpoint{4.898643in}{2.310639in}}%
\pgfpathlineto{\pgfqpoint{4.884210in}{2.296905in}}%
\pgfpathlineto{\pgfqpoint{4.869797in}{2.283335in}}%
\pgfpathlineto{\pgfqpoint{4.861902in}{2.265295in}}%
\pgfpathlineto{\pgfqpoint{4.854002in}{2.247139in}}%
\pgfpathlineto{\pgfqpoint{4.846098in}{2.228869in}}%
\pgfpathlineto{\pgfqpoint{4.838189in}{2.210490in}}%
\pgfpathclose%
\pgfusepath{fill}%
\end{pgfscope}%
\begin{pgfscope}%
\pgfpathrectangle{\pgfqpoint{1.254980in}{0.150000in}}{\pgfqpoint{5.490039in}{5.490039in}}%
\pgfusepath{clip}%
\pgfsetbuttcap%
\pgfsetroundjoin%
\definecolor{currentfill}{rgb}{0.157851,0.683765,0.501686}%
\pgfsetfillcolor{currentfill}%
\pgfsetfillopacity{0.700000}%
\pgfsetlinewidth{0.000000pt}%
\definecolor{currentstroke}{rgb}{0.000000,0.000000,0.000000}%
\pgfsetstrokecolor{currentstroke}%
\pgfsetdash{}{0pt}%
\pgfpathmoveto{\pgfqpoint{4.990643in}{2.481117in}}%
\pgfpathlineto{\pgfqpoint{5.005160in}{2.496041in}}%
\pgfpathlineto{\pgfqpoint{5.019698in}{2.511132in}}%
\pgfpathlineto{\pgfqpoint{5.034257in}{2.526388in}}%
\pgfpathlineto{\pgfqpoint{5.048836in}{2.541811in}}%
\pgfpathlineto{\pgfqpoint{5.056708in}{2.559288in}}%
\pgfpathlineto{\pgfqpoint{5.064574in}{2.576603in}}%
\pgfpathlineto{\pgfqpoint{5.072433in}{2.593752in}}%
\pgfpathlineto{\pgfqpoint{5.080285in}{2.610734in}}%
\pgfpathlineto{\pgfqpoint{5.065694in}{2.595007in}}%
\pgfpathlineto{\pgfqpoint{5.051123in}{2.579446in}}%
\pgfpathlineto{\pgfqpoint{5.036573in}{2.564052in}}%
\pgfpathlineto{\pgfqpoint{5.022043in}{2.548824in}}%
\pgfpathlineto{\pgfqpoint{5.014203in}{2.532134in}}%
\pgfpathlineto{\pgfqpoint{5.006355in}{2.515285in}}%
\pgfpathlineto{\pgfqpoint{4.998502in}{2.498278in}}%
\pgfpathlineto{\pgfqpoint{4.990643in}{2.481117in}}%
\pgfpathclose%
\pgfusepath{fill}%
\end{pgfscope}%
\begin{pgfscope}%
\pgfpathrectangle{\pgfqpoint{1.254980in}{0.150000in}}{\pgfqpoint{5.490039in}{5.490039in}}%
\pgfusepath{clip}%
\pgfsetbuttcap%
\pgfsetroundjoin%
\definecolor{currentfill}{rgb}{0.269308,0.218818,0.509577}%
\pgfsetfillcolor{currentfill}%
\pgfsetfillopacity{0.700000}%
\pgfsetlinewidth{0.000000pt}%
\definecolor{currentstroke}{rgb}{0.000000,0.000000,0.000000}%
\pgfsetstrokecolor{currentstroke}%
\pgfsetdash{}{0pt}%
\pgfpathmoveto{\pgfqpoint{4.260713in}{1.269144in}}%
\pgfpathlineto{\pgfqpoint{4.274754in}{1.273665in}}%
\pgfpathlineto{\pgfqpoint{4.288808in}{1.278338in}}%
\pgfpathlineto{\pgfqpoint{4.302874in}{1.283161in}}%
\pgfpathlineto{\pgfqpoint{4.316953in}{1.288136in}}%
\pgfpathlineto{\pgfqpoint{4.324968in}{1.304633in}}%
\pgfpathlineto{\pgfqpoint{4.332979in}{1.321265in}}%
\pgfpathlineto{\pgfqpoint{4.340988in}{1.338027in}}%
\pgfpathlineto{\pgfqpoint{4.348993in}{1.354912in}}%
\pgfpathlineto{\pgfqpoint{4.334910in}{1.349244in}}%
\pgfpathlineto{\pgfqpoint{4.320840in}{1.343728in}}%
\pgfpathlineto{\pgfqpoint{4.306783in}{1.338364in}}%
\pgfpathlineto{\pgfqpoint{4.292739in}{1.333152in}}%
\pgfpathlineto{\pgfqpoint{4.284737in}{1.316949in}}%
\pgfpathlineto{\pgfqpoint{4.276732in}{1.300876in}}%
\pgfpathlineto{\pgfqpoint{4.268724in}{1.284939in}}%
\pgfpathlineto{\pgfqpoint{4.260713in}{1.269144in}}%
\pgfpathclose%
\pgfusepath{fill}%
\end{pgfscope}%
\begin{pgfscope}%
\pgfpathrectangle{\pgfqpoint{1.254980in}{0.150000in}}{\pgfqpoint{5.490039in}{5.490039in}}%
\pgfusepath{clip}%
\pgfsetbuttcap%
\pgfsetroundjoin%
\definecolor{currentfill}{rgb}{0.244972,0.287675,0.537260}%
\pgfsetfillcolor{currentfill}%
\pgfsetfillopacity{0.700000}%
\pgfsetlinewidth{0.000000pt}%
\definecolor{currentstroke}{rgb}{0.000000,0.000000,0.000000}%
\pgfsetstrokecolor{currentstroke}%
\pgfsetdash{}{0pt}%
\pgfpathmoveto{\pgfqpoint{4.380988in}{1.423554in}}%
\pgfpathlineto{\pgfqpoint{4.395090in}{1.430042in}}%
\pgfpathlineto{\pgfqpoint{4.409207in}{1.436683in}}%
\pgfpathlineto{\pgfqpoint{4.423337in}{1.443477in}}%
\pgfpathlineto{\pgfqpoint{4.437481in}{1.450425in}}%
\pgfpathlineto{\pgfqpoint{4.445480in}{1.468481in}}%
\pgfpathlineto{\pgfqpoint{4.453476in}{1.486615in}}%
\pgfpathlineto{\pgfqpoint{4.461470in}{1.504822in}}%
\pgfpathlineto{\pgfqpoint{4.469462in}{1.523095in}}%
\pgfpathlineto{\pgfqpoint{4.455309in}{1.515505in}}%
\pgfpathlineto{\pgfqpoint{4.441171in}{1.508070in}}%
\pgfpathlineto{\pgfqpoint{4.427048in}{1.500788in}}%
\pgfpathlineto{\pgfqpoint{4.412938in}{1.493660in}}%
\pgfpathlineto{\pgfqpoint{4.404955in}{1.476018in}}%
\pgfpathlineto{\pgfqpoint{4.396969in}{1.458449in}}%
\pgfpathlineto{\pgfqpoint{4.388980in}{1.440959in}}%
\pgfpathlineto{\pgfqpoint{4.380988in}{1.423554in}}%
\pgfpathclose%
\pgfusepath{fill}%
\end{pgfscope}%
\begin{pgfscope}%
\pgfpathrectangle{\pgfqpoint{1.254980in}{0.150000in}}{\pgfqpoint{5.490039in}{5.490039in}}%
\pgfusepath{clip}%
\pgfsetbuttcap%
\pgfsetroundjoin%
\definecolor{currentfill}{rgb}{0.165117,0.467423,0.558141}%
\pgfsetfillcolor{currentfill}%
\pgfsetfillopacity{0.700000}%
\pgfsetlinewidth{0.000000pt}%
\definecolor{currentstroke}{rgb}{0.000000,0.000000,0.000000}%
\pgfsetstrokecolor{currentstroke}%
\pgfsetdash{}{0pt}%
\pgfpathmoveto{\pgfqpoint{4.653897in}{1.861029in}}%
\pgfpathlineto{\pgfqpoint{4.668173in}{1.871703in}}%
\pgfpathlineto{\pgfqpoint{4.682465in}{1.882536in}}%
\pgfpathlineto{\pgfqpoint{4.696775in}{1.893527in}}%
\pgfpathlineto{\pgfqpoint{4.711102in}{1.904677in}}%
\pgfpathlineto{\pgfqpoint{4.719071in}{1.924271in}}%
\pgfpathlineto{\pgfqpoint{4.727037in}{1.943822in}}%
\pgfpathlineto{\pgfqpoint{4.735000in}{1.963324in}}%
\pgfpathlineto{\pgfqpoint{4.742960in}{1.982774in}}%
\pgfpathlineto{\pgfqpoint{4.728619in}{1.971114in}}%
\pgfpathlineto{\pgfqpoint{4.714296in}{1.959613in}}%
\pgfpathlineto{\pgfqpoint{4.699990in}{1.948272in}}%
\pgfpathlineto{\pgfqpoint{4.685702in}{1.937091in}}%
\pgfpathlineto{\pgfqpoint{4.677756in}{1.918139in}}%
\pgfpathlineto{\pgfqpoint{4.669806in}{1.899141in}}%
\pgfpathlineto{\pgfqpoint{4.661853in}{1.880103in}}%
\pgfpathlineto{\pgfqpoint{4.653897in}{1.861029in}}%
\pgfpathclose%
\pgfusepath{fill}%
\end{pgfscope}%
\begin{pgfscope}%
\pgfpathrectangle{\pgfqpoint{1.254980in}{0.150000in}}{\pgfqpoint{5.490039in}{5.490039in}}%
\pgfusepath{clip}%
\pgfsetbuttcap%
\pgfsetroundjoin%
\definecolor{currentfill}{rgb}{0.208623,0.367752,0.552675}%
\pgfsetfillcolor{currentfill}%
\pgfsetfillopacity{0.700000}%
\pgfsetlinewidth{0.000000pt}%
\definecolor{currentstroke}{rgb}{0.000000,0.000000,0.000000}%
\pgfsetstrokecolor{currentstroke}%
\pgfsetdash{}{0pt}%
\pgfpathmoveto{\pgfqpoint{4.501402in}{1.596737in}}%
\pgfpathlineto{\pgfqpoint{4.515579in}{1.605097in}}%
\pgfpathlineto{\pgfqpoint{4.529771in}{1.613613in}}%
\pgfpathlineto{\pgfqpoint{4.543978in}{1.622284in}}%
\pgfpathlineto{\pgfqpoint{4.558201in}{1.631111in}}%
\pgfpathlineto{\pgfqpoint{4.566191in}{1.650229in}}%
\pgfpathlineto{\pgfqpoint{4.574177in}{1.669374in}}%
\pgfpathlineto{\pgfqpoint{4.582162in}{1.688538in}}%
\pgfpathlineto{\pgfqpoint{4.590143in}{1.707716in}}%
\pgfpathlineto{\pgfqpoint{4.575909in}{1.698298in}}%
\pgfpathlineto{\pgfqpoint{4.561690in}{1.689037in}}%
\pgfpathlineto{\pgfqpoint{4.547487in}{1.679932in}}%
\pgfpathlineto{\pgfqpoint{4.533300in}{1.670983in}}%
\pgfpathlineto{\pgfqpoint{4.525330in}{1.652385in}}%
\pgfpathlineto{\pgfqpoint{4.517357in}{1.633807in}}%
\pgfpathlineto{\pgfqpoint{4.509381in}{1.615256in}}%
\pgfpathlineto{\pgfqpoint{4.501402in}{1.596737in}}%
\pgfpathclose%
\pgfusepath{fill}%
\end{pgfscope}%
\begin{pgfscope}%
\pgfpathrectangle{\pgfqpoint{1.254980in}{0.150000in}}{\pgfqpoint{5.490039in}{5.490039in}}%
\pgfusepath{clip}%
\pgfsetbuttcap%
\pgfsetroundjoin%
\definecolor{currentfill}{rgb}{0.127568,0.566949,0.550556}%
\pgfsetfillcolor{currentfill}%
\pgfsetfillopacity{0.700000}%
\pgfsetlinewidth{0.000000pt}%
\definecolor{currentstroke}{rgb}{0.000000,0.000000,0.000000}%
\pgfsetstrokecolor{currentstroke}%
\pgfsetdash{}{0pt}%
\pgfpathmoveto{\pgfqpoint{4.806509in}{2.135950in}}%
\pgfpathlineto{\pgfqpoint{4.820895in}{2.148705in}}%
\pgfpathlineto{\pgfqpoint{4.835301in}{2.161621in}}%
\pgfpathlineto{\pgfqpoint{4.849725in}{2.174700in}}%
\pgfpathlineto{\pgfqpoint{4.864168in}{2.187940in}}%
\pgfpathlineto{\pgfqpoint{4.872108in}{2.207156in}}%
\pgfpathlineto{\pgfqpoint{4.880045in}{2.226270in}}%
\pgfpathlineto{\pgfqpoint{4.887976in}{2.245278in}}%
\pgfpathlineto{\pgfqpoint{4.895904in}{2.264176in}}%
\pgfpathlineto{\pgfqpoint{4.881446in}{2.250510in}}%
\pgfpathlineto{\pgfqpoint{4.867008in}{2.237007in}}%
\pgfpathlineto{\pgfqpoint{4.852589in}{2.223667in}}%
\pgfpathlineto{\pgfqpoint{4.838189in}{2.210490in}}%
\pgfpathlineto{\pgfqpoint{4.830275in}{2.192005in}}%
\pgfpathlineto{\pgfqpoint{4.822357in}{2.173417in}}%
\pgfpathlineto{\pgfqpoint{4.814435in}{2.154731in}}%
\pgfpathlineto{\pgfqpoint{4.806509in}{2.135950in}}%
\pgfpathclose%
\pgfusepath{fill}%
\end{pgfscope}%
\begin{pgfscope}%
\pgfpathrectangle{\pgfqpoint{1.254980in}{0.150000in}}{\pgfqpoint{5.490039in}{5.490039in}}%
\pgfusepath{clip}%
\pgfsetbuttcap%
\pgfsetroundjoin%
\definecolor{currentfill}{rgb}{0.274149,0.751988,0.436601}%
\pgfsetfillcolor{currentfill}%
\pgfsetfillopacity{0.700000}%
\pgfsetlinewidth{0.000000pt}%
\definecolor{currentstroke}{rgb}{0.000000,0.000000,0.000000}%
\pgfsetstrokecolor{currentstroke}%
\pgfsetdash{}{0pt}%
\pgfpathmoveto{\pgfqpoint{5.111629in}{2.676934in}}%
\pgfpathlineto{\pgfqpoint{5.126254in}{2.693103in}}%
\pgfpathlineto{\pgfqpoint{5.140900in}{2.709440in}}%
\pgfpathlineto{\pgfqpoint{5.155568in}{2.725945in}}%
\pgfpathlineto{\pgfqpoint{5.170258in}{2.742618in}}%
\pgfpathlineto{\pgfqpoint{5.178088in}{2.758980in}}%
\pgfpathlineto{\pgfqpoint{5.185910in}{2.775151in}}%
\pgfpathlineto{\pgfqpoint{5.193725in}{2.791132in}}%
\pgfpathlineto{\pgfqpoint{5.201531in}{2.806918in}}%
\pgfpathlineto{\pgfqpoint{5.186830in}{2.790002in}}%
\pgfpathlineto{\pgfqpoint{5.172151in}{2.773254in}}%
\pgfpathlineto{\pgfqpoint{5.157494in}{2.756675in}}%
\pgfpathlineto{\pgfqpoint{5.142859in}{2.740265in}}%
\pgfpathlineto{\pgfqpoint{5.135063in}{2.724708in}}%
\pgfpathlineto{\pgfqpoint{5.127259in}{2.708967in}}%
\pgfpathlineto{\pgfqpoint{5.119447in}{2.693041in}}%
\pgfpathlineto{\pgfqpoint{5.111629in}{2.676934in}}%
\pgfpathclose%
\pgfusepath{fill}%
\end{pgfscope}%
\begin{pgfscope}%
\pgfpathrectangle{\pgfqpoint{1.254980in}{0.150000in}}{\pgfqpoint{5.490039in}{5.490039in}}%
\pgfusepath{clip}%
\pgfsetbuttcap%
\pgfsetroundjoin%
\definecolor{currentfill}{rgb}{0.140210,0.665859,0.513427}%
\pgfsetfillcolor{currentfill}%
\pgfsetfillopacity{0.700000}%
\pgfsetlinewidth{0.000000pt}%
\definecolor{currentstroke}{rgb}{0.000000,0.000000,0.000000}%
\pgfsetstrokecolor{currentstroke}%
\pgfsetdash{}{0pt}%
\pgfpathmoveto{\pgfqpoint{4.959148in}{2.410979in}}%
\pgfpathlineto{\pgfqpoint{4.973653in}{2.425570in}}%
\pgfpathlineto{\pgfqpoint{4.988178in}{2.440325in}}%
\pgfpathlineto{\pgfqpoint{5.002723in}{2.455246in}}%
\pgfpathlineto{\pgfqpoint{5.017289in}{2.470332in}}%
\pgfpathlineto{\pgfqpoint{5.025185in}{2.488432in}}%
\pgfpathlineto{\pgfqpoint{5.033074in}{2.506380in}}%
\pgfpathlineto{\pgfqpoint{5.040958in}{2.524174in}}%
\pgfpathlineto{\pgfqpoint{5.048836in}{2.541811in}}%
\pgfpathlineto{\pgfqpoint{5.034257in}{2.526388in}}%
\pgfpathlineto{\pgfqpoint{5.019698in}{2.511132in}}%
\pgfpathlineto{\pgfqpoint{5.005160in}{2.496041in}}%
\pgfpathlineto{\pgfqpoint{4.990643in}{2.481117in}}%
\pgfpathlineto{\pgfqpoint{4.982778in}{2.463803in}}%
\pgfpathlineto{\pgfqpoint{4.974907in}{2.446341in}}%
\pgfpathlineto{\pgfqpoint{4.967030in}{2.428732in}}%
\pgfpathlineto{\pgfqpoint{4.959148in}{2.410979in}}%
\pgfpathclose%
\pgfusepath{fill}%
\end{pgfscope}%
\begin{pgfscope}%
\pgfpathrectangle{\pgfqpoint{1.254980in}{0.150000in}}{\pgfqpoint{5.490039in}{5.490039in}}%
\pgfusepath{clip}%
\pgfsetbuttcap%
\pgfsetroundjoin%
\definecolor{currentfill}{rgb}{0.253935,0.265254,0.529983}%
\pgfsetfillcolor{currentfill}%
\pgfsetfillopacity{0.700000}%
\pgfsetlinewidth{0.000000pt}%
\definecolor{currentstroke}{rgb}{0.000000,0.000000,0.000000}%
\pgfsetstrokecolor{currentstroke}%
\pgfsetdash{}{0pt}%
\pgfpathmoveto{\pgfqpoint{4.348993in}{1.354912in}}%
\pgfpathlineto{\pgfqpoint{4.363090in}{1.360733in}}%
\pgfpathlineto{\pgfqpoint{4.377199in}{1.366706in}}%
\pgfpathlineto{\pgfqpoint{4.391322in}{1.372832in}}%
\pgfpathlineto{\pgfqpoint{4.405459in}{1.379110in}}%
\pgfpathlineto{\pgfqpoint{4.413468in}{1.396790in}}%
\pgfpathlineto{\pgfqpoint{4.421475in}{1.414573in}}%
\pgfpathlineto{\pgfqpoint{4.429479in}{1.432453in}}%
\pgfpathlineto{\pgfqpoint{4.437481in}{1.450425in}}%
\pgfpathlineto{\pgfqpoint{4.423337in}{1.443477in}}%
\pgfpathlineto{\pgfqpoint{4.409207in}{1.436683in}}%
\pgfpathlineto{\pgfqpoint{4.395090in}{1.430042in}}%
\pgfpathlineto{\pgfqpoint{4.380988in}{1.423554in}}%
\pgfpathlineto{\pgfqpoint{4.372994in}{1.406241in}}%
\pgfpathlineto{\pgfqpoint{4.364996in}{1.389026in}}%
\pgfpathlineto{\pgfqpoint{4.356996in}{1.371914in}}%
\pgfpathlineto{\pgfqpoint{4.348993in}{1.354912in}}%
\pgfpathclose%
\pgfusepath{fill}%
\end{pgfscope}%
\begin{pgfscope}%
\pgfpathrectangle{\pgfqpoint{1.254980in}{0.150000in}}{\pgfqpoint{5.490039in}{5.490039in}}%
\pgfusepath{clip}%
\pgfsetbuttcap%
\pgfsetroundjoin%
\definecolor{currentfill}{rgb}{0.421908,0.805774,0.351910}%
\pgfsetfillcolor{currentfill}%
\pgfsetfillopacity{0.700000}%
\pgfsetlinewidth{0.000000pt}%
\definecolor{currentstroke}{rgb}{0.000000,0.000000,0.000000}%
\pgfsetstrokecolor{currentstroke}%
\pgfsetdash{}{0pt}%
\pgfpathmoveto{\pgfqpoint{5.232679in}{2.868092in}}%
\pgfpathlineto{\pgfqpoint{5.247412in}{2.885390in}}%
\pgfpathlineto{\pgfqpoint{5.262168in}{2.902857in}}%
\pgfpathlineto{\pgfqpoint{5.276947in}{2.920495in}}%
\pgfpathlineto{\pgfqpoint{5.284720in}{2.935430in}}%
\pgfpathlineto{\pgfqpoint{5.292485in}{2.950156in}}%
\pgfpathlineto{\pgfqpoint{5.300240in}{2.964670in}}%
\pgfpathlineto{\pgfqpoint{5.307987in}{2.978972in}}%
\pgfpathlineto{\pgfqpoint{5.293199in}{2.961154in}}%
\pgfpathlineto{\pgfqpoint{5.278435in}{2.943508in}}%
\pgfpathlineto{\pgfqpoint{5.263693in}{2.926032in}}%
\pgfpathlineto{\pgfqpoint{5.255952in}{2.911855in}}%
\pgfpathlineto{\pgfqpoint{5.248203in}{2.897472in}}%
\pgfpathlineto{\pgfqpoint{5.240445in}{2.882884in}}%
\pgfpathlineto{\pgfqpoint{5.232679in}{2.868092in}}%
\pgfpathclose%
\pgfusepath{fill}%
\end{pgfscope}%
\begin{pgfscope}%
\pgfpathrectangle{\pgfqpoint{1.254980in}{0.150000in}}{\pgfqpoint{5.490039in}{5.490039in}}%
\pgfusepath{clip}%
\pgfsetbuttcap%
\pgfsetroundjoin%
\definecolor{currentfill}{rgb}{0.275191,0.194905,0.496005}%
\pgfsetfillcolor{currentfill}%
\pgfsetfillopacity{0.700000}%
\pgfsetlinewidth{0.000000pt}%
\definecolor{currentstroke}{rgb}{0.000000,0.000000,0.000000}%
\pgfsetstrokecolor{currentstroke}%
\pgfsetdash{}{0pt}%
\pgfpathmoveto{\pgfqpoint{4.228634in}{1.207531in}}%
\pgfpathlineto{\pgfqpoint{4.242674in}{1.211333in}}%
\pgfpathlineto{\pgfqpoint{4.256725in}{1.215287in}}%
\pgfpathlineto{\pgfqpoint{4.270789in}{1.219391in}}%
\pgfpathlineto{\pgfqpoint{4.284865in}{1.223646in}}%
\pgfpathlineto{\pgfqpoint{4.292892in}{1.239531in}}%
\pgfpathlineto{\pgfqpoint{4.300915in}{1.255578in}}%
\pgfpathlineto{\pgfqpoint{4.308936in}{1.271783in}}%
\pgfpathlineto{\pgfqpoint{4.316953in}{1.288136in}}%
\pgfpathlineto{\pgfqpoint{4.302874in}{1.283161in}}%
\pgfpathlineto{\pgfqpoint{4.288808in}{1.278338in}}%
\pgfpathlineto{\pgfqpoint{4.274754in}{1.273665in}}%
\pgfpathlineto{\pgfqpoint{4.260713in}{1.269144in}}%
\pgfpathlineto{\pgfqpoint{4.252698in}{1.253500in}}%
\pgfpathlineto{\pgfqpoint{4.244680in}{1.238011in}}%
\pgfpathlineto{\pgfqpoint{4.236659in}{1.222686in}}%
\pgfpathlineto{\pgfqpoint{4.228634in}{1.207531in}}%
\pgfpathclose%
\pgfusepath{fill}%
\end{pgfscope}%
\begin{pgfscope}%
\pgfpathrectangle{\pgfqpoint{1.254980in}{0.150000in}}{\pgfqpoint{5.490039in}{5.490039in}}%
\pgfusepath{clip}%
\pgfsetbuttcap%
\pgfsetroundjoin%
\definecolor{currentfill}{rgb}{0.175841,0.441290,0.557685}%
\pgfsetfillcolor{currentfill}%
\pgfsetfillopacity{0.700000}%
\pgfsetlinewidth{0.000000pt}%
\definecolor{currentstroke}{rgb}{0.000000,0.000000,0.000000}%
\pgfsetstrokecolor{currentstroke}%
\pgfsetdash{}{0pt}%
\pgfpathmoveto{\pgfqpoint{4.622043in}{1.784466in}}%
\pgfpathlineto{\pgfqpoint{4.636306in}{1.794604in}}%
\pgfpathlineto{\pgfqpoint{4.650585in}{1.804900in}}%
\pgfpathlineto{\pgfqpoint{4.664882in}{1.815354in}}%
\pgfpathlineto{\pgfqpoint{4.679195in}{1.825965in}}%
\pgfpathlineto{\pgfqpoint{4.687176in}{1.845684in}}%
\pgfpathlineto{\pgfqpoint{4.695154in}{1.865379in}}%
\pgfpathlineto{\pgfqpoint{4.703129in}{1.885045in}}%
\pgfpathlineto{\pgfqpoint{4.711102in}{1.904677in}}%
\pgfpathlineto{\pgfqpoint{4.696775in}{1.893527in}}%
\pgfpathlineto{\pgfqpoint{4.682465in}{1.882536in}}%
\pgfpathlineto{\pgfqpoint{4.668173in}{1.871703in}}%
\pgfpathlineto{\pgfqpoint{4.653897in}{1.861029in}}%
\pgfpathlineto{\pgfqpoint{4.645938in}{1.841923in}}%
\pgfpathlineto{\pgfqpoint{4.637976in}{1.822791in}}%
\pgfpathlineto{\pgfqpoint{4.630011in}{1.803637in}}%
\pgfpathlineto{\pgfqpoint{4.622043in}{1.784466in}}%
\pgfpathclose%
\pgfusepath{fill}%
\end{pgfscope}%
\begin{pgfscope}%
\pgfpathrectangle{\pgfqpoint{1.254980in}{0.150000in}}{\pgfqpoint{5.490039in}{5.490039in}}%
\pgfusepath{clip}%
\pgfsetbuttcap%
\pgfsetroundjoin%
\definecolor{currentfill}{rgb}{0.220057,0.343307,0.549413}%
\pgfsetfillcolor{currentfill}%
\pgfsetfillopacity{0.700000}%
\pgfsetlinewidth{0.000000pt}%
\definecolor{currentstroke}{rgb}{0.000000,0.000000,0.000000}%
\pgfsetstrokecolor{currentstroke}%
\pgfsetdash{}{0pt}%
\pgfpathmoveto{\pgfqpoint{4.469462in}{1.523095in}}%
\pgfpathlineto{\pgfqpoint{4.483629in}{1.530840in}}%
\pgfpathlineto{\pgfqpoint{4.497811in}{1.538738in}}%
\pgfpathlineto{\pgfqpoint{4.512008in}{1.546792in}}%
\pgfpathlineto{\pgfqpoint{4.526220in}{1.554999in}}%
\pgfpathlineto{\pgfqpoint{4.534219in}{1.573961in}}%
\pgfpathlineto{\pgfqpoint{4.542216in}{1.592971in}}%
\pgfpathlineto{\pgfqpoint{4.550210in}{1.612022in}}%
\pgfpathlineto{\pgfqpoint{4.558201in}{1.631111in}}%
\pgfpathlineto{\pgfqpoint{4.543978in}{1.622284in}}%
\pgfpathlineto{\pgfqpoint{4.529771in}{1.613613in}}%
\pgfpathlineto{\pgfqpoint{4.515579in}{1.605097in}}%
\pgfpathlineto{\pgfqpoint{4.501402in}{1.596737in}}%
\pgfpathlineto{\pgfqpoint{4.493421in}{1.578256in}}%
\pgfpathlineto{\pgfqpoint{4.485437in}{1.559818in}}%
\pgfpathlineto{\pgfqpoint{4.477451in}{1.541429in}}%
\pgfpathlineto{\pgfqpoint{4.469462in}{1.523095in}}%
\pgfpathclose%
\pgfusepath{fill}%
\end{pgfscope}%
\begin{pgfscope}%
\pgfpathrectangle{\pgfqpoint{1.254980in}{0.150000in}}{\pgfqpoint{5.490039in}{5.490039in}}%
\pgfusepath{clip}%
\pgfsetbuttcap%
\pgfsetroundjoin%
\definecolor{currentfill}{rgb}{0.135066,0.544853,0.554029}%
\pgfsetfillcolor{currentfill}%
\pgfsetfillopacity{0.700000}%
\pgfsetlinewidth{0.000000pt}%
\definecolor{currentstroke}{rgb}{0.000000,0.000000,0.000000}%
\pgfsetstrokecolor{currentstroke}%
\pgfsetdash{}{0pt}%
\pgfpathmoveto{\pgfqpoint{4.774764in}{2.059958in}}%
\pgfpathlineto{\pgfqpoint{4.789137in}{2.072260in}}%
\pgfpathlineto{\pgfqpoint{4.803528in}{2.084723in}}%
\pgfpathlineto{\pgfqpoint{4.817937in}{2.097347in}}%
\pgfpathlineto{\pgfqpoint{4.832366in}{2.110133in}}%
\pgfpathlineto{\pgfqpoint{4.840322in}{2.129718in}}%
\pgfpathlineto{\pgfqpoint{4.848275in}{2.149217in}}%
\pgfpathlineto{\pgfqpoint{4.856223in}{2.168626in}}%
\pgfpathlineto{\pgfqpoint{4.864168in}{2.187940in}}%
\pgfpathlineto{\pgfqpoint{4.849725in}{2.174700in}}%
\pgfpathlineto{\pgfqpoint{4.835301in}{2.161621in}}%
\pgfpathlineto{\pgfqpoint{4.820895in}{2.148705in}}%
\pgfpathlineto{\pgfqpoint{4.806509in}{2.135950in}}%
\pgfpathlineto{\pgfqpoint{4.798579in}{2.117078in}}%
\pgfpathlineto{\pgfqpoint{4.790644in}{2.098120in}}%
\pgfpathlineto{\pgfqpoint{4.782706in}{2.079078in}}%
\pgfpathlineto{\pgfqpoint{4.774764in}{2.059958in}}%
\pgfpathclose%
\pgfusepath{fill}%
\end{pgfscope}%
\begin{pgfscope}%
\pgfpathrectangle{\pgfqpoint{1.254980in}{0.150000in}}{\pgfqpoint{5.490039in}{5.490039in}}%
\pgfusepath{clip}%
\pgfsetbuttcap%
\pgfsetroundjoin%
\definecolor{currentfill}{rgb}{0.126326,0.644107,0.525311}%
\pgfsetfillcolor{currentfill}%
\pgfsetfillopacity{0.700000}%
\pgfsetlinewidth{0.000000pt}%
\definecolor{currentstroke}{rgb}{0.000000,0.000000,0.000000}%
\pgfsetstrokecolor{currentstroke}%
\pgfsetdash{}{0pt}%
\pgfpathmoveto{\pgfqpoint{4.927566in}{2.338597in}}%
\pgfpathlineto{\pgfqpoint{4.942057in}{2.352823in}}%
\pgfpathlineto{\pgfqpoint{4.956568in}{2.367212in}}%
\pgfpathlineto{\pgfqpoint{4.971099in}{2.381766in}}%
\pgfpathlineto{\pgfqpoint{4.985651in}{2.396485in}}%
\pgfpathlineto{\pgfqpoint{4.993568in}{2.415158in}}%
\pgfpathlineto{\pgfqpoint{5.001481in}{2.433692in}}%
\pgfpathlineto{\pgfqpoint{5.009388in}{2.452085in}}%
\pgfpathlineto{\pgfqpoint{5.017289in}{2.470332in}}%
\pgfpathlineto{\pgfqpoint{5.002723in}{2.455246in}}%
\pgfpathlineto{\pgfqpoint{4.988178in}{2.440325in}}%
\pgfpathlineto{\pgfqpoint{4.973653in}{2.425570in}}%
\pgfpathlineto{\pgfqpoint{4.959148in}{2.410979in}}%
\pgfpathlineto{\pgfqpoint{4.951260in}{2.393087in}}%
\pgfpathlineto{\pgfqpoint{4.943367in}{2.375057in}}%
\pgfpathlineto{\pgfqpoint{4.935469in}{2.356892in}}%
\pgfpathlineto{\pgfqpoint{4.927566in}{2.338597in}}%
\pgfpathclose%
\pgfusepath{fill}%
\end{pgfscope}%
\begin{pgfscope}%
\pgfpathrectangle{\pgfqpoint{1.254980in}{0.150000in}}{\pgfqpoint{5.490039in}{5.490039in}}%
\pgfusepath{clip}%
\pgfsetbuttcap%
\pgfsetroundjoin%
\definecolor{currentfill}{rgb}{0.239374,0.735588,0.455688}%
\pgfsetfillcolor{currentfill}%
\pgfsetfillopacity{0.700000}%
\pgfsetlinewidth{0.000000pt}%
\definecolor{currentstroke}{rgb}{0.000000,0.000000,0.000000}%
\pgfsetstrokecolor{currentstroke}%
\pgfsetdash{}{0pt}%
\pgfpathmoveto{\pgfqpoint{5.080285in}{2.610734in}}%
\pgfpathlineto{\pgfqpoint{5.094899in}{2.626629in}}%
\pgfpathlineto{\pgfqpoint{5.109533in}{2.642691in}}%
\pgfpathlineto{\pgfqpoint{5.124189in}{2.658921in}}%
\pgfpathlineto{\pgfqpoint{5.138866in}{2.675319in}}%
\pgfpathlineto{\pgfqpoint{5.146725in}{2.692417in}}%
\pgfpathlineto{\pgfqpoint{5.154577in}{2.709335in}}%
\pgfpathlineto{\pgfqpoint{5.162421in}{2.726069in}}%
\pgfpathlineto{\pgfqpoint{5.170258in}{2.742618in}}%
\pgfpathlineto{\pgfqpoint{5.155568in}{2.725945in}}%
\pgfpathlineto{\pgfqpoint{5.140900in}{2.709440in}}%
\pgfpathlineto{\pgfqpoint{5.126254in}{2.693103in}}%
\pgfpathlineto{\pgfqpoint{5.111629in}{2.676934in}}%
\pgfpathlineto{\pgfqpoint{5.103803in}{2.660648in}}%
\pgfpathlineto{\pgfqpoint{5.095971in}{2.644184in}}%
\pgfpathlineto{\pgfqpoint{5.088132in}{2.627546in}}%
\pgfpathlineto{\pgfqpoint{5.080285in}{2.610734in}}%
\pgfpathclose%
\pgfusepath{fill}%
\end{pgfscope}%
\begin{pgfscope}%
\pgfpathrectangle{\pgfqpoint{1.254980in}{0.150000in}}{\pgfqpoint{5.490039in}{5.490039in}}%
\pgfusepath{clip}%
\pgfsetbuttcap%
\pgfsetroundjoin%
\definecolor{currentfill}{rgb}{0.262138,0.242286,0.520837}%
\pgfsetfillcolor{currentfill}%
\pgfsetfillopacity{0.700000}%
\pgfsetlinewidth{0.000000pt}%
\definecolor{currentstroke}{rgb}{0.000000,0.000000,0.000000}%
\pgfsetstrokecolor{currentstroke}%
\pgfsetdash{}{0pt}%
\pgfpathmoveto{\pgfqpoint{4.316953in}{1.288136in}}%
\pgfpathlineto{\pgfqpoint{4.331045in}{1.293263in}}%
\pgfpathlineto{\pgfqpoint{4.345149in}{1.298541in}}%
\pgfpathlineto{\pgfqpoint{4.359266in}{1.303971in}}%
\pgfpathlineto{\pgfqpoint{4.373397in}{1.309552in}}%
\pgfpathlineto{\pgfqpoint{4.381416in}{1.326754in}}%
\pgfpathlineto{\pgfqpoint{4.389433in}{1.344085in}}%
\pgfpathlineto{\pgfqpoint{4.397447in}{1.361539in}}%
\pgfpathlineto{\pgfqpoint{4.405459in}{1.379110in}}%
\pgfpathlineto{\pgfqpoint{4.391322in}{1.372832in}}%
\pgfpathlineto{\pgfqpoint{4.377199in}{1.366706in}}%
\pgfpathlineto{\pgfqpoint{4.363090in}{1.360733in}}%
\pgfpathlineto{\pgfqpoint{4.348993in}{1.354912in}}%
\pgfpathlineto{\pgfqpoint{4.340988in}{1.338027in}}%
\pgfpathlineto{\pgfqpoint{4.332979in}{1.321265in}}%
\pgfpathlineto{\pgfqpoint{4.324968in}{1.304633in}}%
\pgfpathlineto{\pgfqpoint{4.316953in}{1.288136in}}%
\pgfpathclose%
\pgfusepath{fill}%
\end{pgfscope}%
\begin{pgfscope}%
\pgfpathrectangle{\pgfqpoint{1.254980in}{0.150000in}}{\pgfqpoint{5.490039in}{5.490039in}}%
\pgfusepath{clip}%
\pgfsetbuttcap%
\pgfsetroundjoin%
\definecolor{currentfill}{rgb}{0.185556,0.418570,0.556753}%
\pgfsetfillcolor{currentfill}%
\pgfsetfillopacity{0.700000}%
\pgfsetlinewidth{0.000000pt}%
\definecolor{currentstroke}{rgb}{0.000000,0.000000,0.000000}%
\pgfsetstrokecolor{currentstroke}%
\pgfsetdash{}{0pt}%
\pgfpathmoveto{\pgfqpoint{4.590143in}{1.707716in}}%
\pgfpathlineto{\pgfqpoint{4.604394in}{1.717290in}}%
\pgfpathlineto{\pgfqpoint{4.618661in}{1.727021in}}%
\pgfpathlineto{\pgfqpoint{4.632944in}{1.736909in}}%
\pgfpathlineto{\pgfqpoint{4.647243in}{1.746953in}}%
\pgfpathlineto{\pgfqpoint{4.655235in}{1.766717in}}%
\pgfpathlineto{\pgfqpoint{4.663224in}{1.786477in}}%
\pgfpathlineto{\pgfqpoint{4.671211in}{1.806228in}}%
\pgfpathlineto{\pgfqpoint{4.679195in}{1.825965in}}%
\pgfpathlineto{\pgfqpoint{4.664882in}{1.815354in}}%
\pgfpathlineto{\pgfqpoint{4.650585in}{1.804900in}}%
\pgfpathlineto{\pgfqpoint{4.636306in}{1.794604in}}%
\pgfpathlineto{\pgfqpoint{4.622043in}{1.784466in}}%
\pgfpathlineto{\pgfqpoint{4.614072in}{1.765283in}}%
\pgfpathlineto{\pgfqpoint{4.606099in}{1.746094in}}%
\pgfpathlineto{\pgfqpoint{4.598122in}{1.726903in}}%
\pgfpathlineto{\pgfqpoint{4.590143in}{1.707716in}}%
\pgfpathclose%
\pgfusepath{fill}%
\end{pgfscope}%
\begin{pgfscope}%
\pgfpathrectangle{\pgfqpoint{1.254980in}{0.150000in}}{\pgfqpoint{5.490039in}{5.490039in}}%
\pgfusepath{clip}%
\pgfsetbuttcap%
\pgfsetroundjoin%
\definecolor{currentfill}{rgb}{0.386433,0.794644,0.372886}%
\pgfsetfillcolor{currentfill}%
\pgfsetfillopacity{0.700000}%
\pgfsetlinewidth{0.000000pt}%
\definecolor{currentstroke}{rgb}{0.000000,0.000000,0.000000}%
\pgfsetstrokecolor{currentstroke}%
\pgfsetdash{}{0pt}%
\pgfpathmoveto{\pgfqpoint{5.201531in}{2.806918in}}%
\pgfpathlineto{\pgfqpoint{5.216255in}{2.824005in}}%
\pgfpathlineto{\pgfqpoint{5.231001in}{2.841260in}}%
\pgfpathlineto{\pgfqpoint{5.245769in}{2.858687in}}%
\pgfpathlineto{\pgfqpoint{5.253576in}{2.874446in}}%
\pgfpathlineto{\pgfqpoint{5.261375in}{2.890001in}}%
\pgfpathlineto{\pgfqpoint{5.269165in}{2.905352in}}%
\pgfpathlineto{\pgfqpoint{5.276947in}{2.920495in}}%
\pgfpathlineto{\pgfqpoint{5.262168in}{2.902857in}}%
\pgfpathlineto{\pgfqpoint{5.247412in}{2.885390in}}%
\pgfpathlineto{\pgfqpoint{5.232679in}{2.868092in}}%
\pgfpathlineto{\pgfqpoint{5.224904in}{2.853098in}}%
\pgfpathlineto{\pgfqpoint{5.217121in}{2.837903in}}%
\pgfpathlineto{\pgfqpoint{5.209330in}{2.822509in}}%
\pgfpathlineto{\pgfqpoint{5.201531in}{2.806918in}}%
\pgfpathclose%
\pgfusepath{fill}%
\end{pgfscope}%
\begin{pgfscope}%
\pgfpathrectangle{\pgfqpoint{1.254980in}{0.150000in}}{\pgfqpoint{5.490039in}{5.490039in}}%
\pgfusepath{clip}%
\pgfsetbuttcap%
\pgfsetroundjoin%
\definecolor{currentfill}{rgb}{0.231674,0.318106,0.544834}%
\pgfsetfillcolor{currentfill}%
\pgfsetfillopacity{0.700000}%
\pgfsetlinewidth{0.000000pt}%
\definecolor{currentstroke}{rgb}{0.000000,0.000000,0.000000}%
\pgfsetstrokecolor{currentstroke}%
\pgfsetdash{}{0pt}%
\pgfpathmoveto{\pgfqpoint{4.437481in}{1.450425in}}%
\pgfpathlineto{\pgfqpoint{4.451639in}{1.457526in}}%
\pgfpathlineto{\pgfqpoint{4.465812in}{1.464780in}}%
\pgfpathlineto{\pgfqpoint{4.479999in}{1.472188in}}%
\pgfpathlineto{\pgfqpoint{4.494201in}{1.479750in}}%
\pgfpathlineto{\pgfqpoint{4.502209in}{1.498461in}}%
\pgfpathlineto{\pgfqpoint{4.510215in}{1.517243in}}%
\pgfpathlineto{\pgfqpoint{4.518219in}{1.536091in}}%
\pgfpathlineto{\pgfqpoint{4.526220in}{1.554999in}}%
\pgfpathlineto{\pgfqpoint{4.512008in}{1.546792in}}%
\pgfpathlineto{\pgfqpoint{4.497811in}{1.538738in}}%
\pgfpathlineto{\pgfqpoint{4.483629in}{1.530840in}}%
\pgfpathlineto{\pgfqpoint{4.469462in}{1.523095in}}%
\pgfpathlineto{\pgfqpoint{4.461470in}{1.504822in}}%
\pgfpathlineto{\pgfqpoint{4.453476in}{1.486615in}}%
\pgfpathlineto{\pgfqpoint{4.445480in}{1.468481in}}%
\pgfpathlineto{\pgfqpoint{4.437481in}{1.450425in}}%
\pgfpathclose%
\pgfusepath{fill}%
\end{pgfscope}%
\begin{pgfscope}%
\pgfpathrectangle{\pgfqpoint{1.254980in}{0.150000in}}{\pgfqpoint{5.490039in}{5.490039in}}%
\pgfusepath{clip}%
\pgfsetbuttcap%
\pgfsetroundjoin%
\definecolor{currentfill}{rgb}{0.144759,0.519093,0.556572}%
\pgfsetfillcolor{currentfill}%
\pgfsetfillopacity{0.700000}%
\pgfsetlinewidth{0.000000pt}%
\definecolor{currentstroke}{rgb}{0.000000,0.000000,0.000000}%
\pgfsetstrokecolor{currentstroke}%
\pgfsetdash{}{0pt}%
\pgfpathmoveto{\pgfqpoint{4.742960in}{1.982774in}}%
\pgfpathlineto{\pgfqpoint{4.757318in}{1.994594in}}%
\pgfpathlineto{\pgfqpoint{4.771695in}{2.006574in}}%
\pgfpathlineto{\pgfqpoint{4.786090in}{2.018715in}}%
\pgfpathlineto{\pgfqpoint{4.800503in}{2.031016in}}%
\pgfpathlineto{\pgfqpoint{4.808474in}{2.050903in}}%
\pgfpathlineto{\pgfqpoint{4.816441in}{2.070721in}}%
\pgfpathlineto{\pgfqpoint{4.824405in}{2.090466in}}%
\pgfpathlineto{\pgfqpoint{4.832366in}{2.110133in}}%
\pgfpathlineto{\pgfqpoint{4.817937in}{2.097347in}}%
\pgfpathlineto{\pgfqpoint{4.803528in}{2.084723in}}%
\pgfpathlineto{\pgfqpoint{4.789137in}{2.072260in}}%
\pgfpathlineto{\pgfqpoint{4.774764in}{2.059958in}}%
\pgfpathlineto{\pgfqpoint{4.766818in}{2.040763in}}%
\pgfpathlineto{\pgfqpoint{4.758869in}{2.021498in}}%
\pgfpathlineto{\pgfqpoint{4.750916in}{2.002167in}}%
\pgfpathlineto{\pgfqpoint{4.742960in}{1.982774in}}%
\pgfpathclose%
\pgfusepath{fill}%
\end{pgfscope}%
\begin{pgfscope}%
\pgfpathrectangle{\pgfqpoint{1.254980in}{0.150000in}}{\pgfqpoint{5.490039in}{5.490039in}}%
\pgfusepath{clip}%
\pgfsetbuttcap%
\pgfsetroundjoin%
\definecolor{currentfill}{rgb}{0.120081,0.622161,0.534946}%
\pgfsetfillcolor{currentfill}%
\pgfsetfillopacity{0.700000}%
\pgfsetlinewidth{0.000000pt}%
\definecolor{currentstroke}{rgb}{0.000000,0.000000,0.000000}%
\pgfsetstrokecolor{currentstroke}%
\pgfsetdash{}{0pt}%
\pgfpathmoveto{\pgfqpoint{4.895904in}{2.264176in}}%
\pgfpathlineto{\pgfqpoint{4.910381in}{2.278006in}}%
\pgfpathlineto{\pgfqpoint{4.924877in}{2.291999in}}%
\pgfpathlineto{\pgfqpoint{4.939394in}{2.306155in}}%
\pgfpathlineto{\pgfqpoint{4.953930in}{2.320476in}}%
\pgfpathlineto{\pgfqpoint{4.961867in}{2.339669in}}%
\pgfpathlineto{\pgfqpoint{4.969800in}{2.358737in}}%
\pgfpathlineto{\pgfqpoint{4.977728in}{2.377677in}}%
\pgfpathlineto{\pgfqpoint{4.985651in}{2.396485in}}%
\pgfpathlineto{\pgfqpoint{4.971099in}{2.381766in}}%
\pgfpathlineto{\pgfqpoint{4.956568in}{2.367212in}}%
\pgfpathlineto{\pgfqpoint{4.942057in}{2.352823in}}%
\pgfpathlineto{\pgfqpoint{4.927566in}{2.338597in}}%
\pgfpathlineto{\pgfqpoint{4.919658in}{2.320175in}}%
\pgfpathlineto{\pgfqpoint{4.911744in}{2.301628in}}%
\pgfpathlineto{\pgfqpoint{4.903826in}{2.282961in}}%
\pgfpathlineto{\pgfqpoint{4.895904in}{2.264176in}}%
\pgfpathclose%
\pgfusepath{fill}%
\end{pgfscope}%
\begin{pgfscope}%
\pgfpathrectangle{\pgfqpoint{1.254980in}{0.150000in}}{\pgfqpoint{5.490039in}{5.490039in}}%
\pgfusepath{clip}%
\pgfsetbuttcap%
\pgfsetroundjoin%
\definecolor{currentfill}{rgb}{0.208030,0.718701,0.472873}%
\pgfsetfillcolor{currentfill}%
\pgfsetfillopacity{0.700000}%
\pgfsetlinewidth{0.000000pt}%
\definecolor{currentstroke}{rgb}{0.000000,0.000000,0.000000}%
\pgfsetstrokecolor{currentstroke}%
\pgfsetdash{}{0pt}%
\pgfpathmoveto{\pgfqpoint{5.048836in}{2.541811in}}%
\pgfpathlineto{\pgfqpoint{5.063437in}{2.557400in}}%
\pgfpathlineto{\pgfqpoint{5.078058in}{2.573156in}}%
\pgfpathlineto{\pgfqpoint{5.092701in}{2.589080in}}%
\pgfpathlineto{\pgfqpoint{5.107365in}{2.605170in}}%
\pgfpathlineto{\pgfqpoint{5.115250in}{2.622966in}}%
\pgfpathlineto{\pgfqpoint{5.123129in}{2.640591in}}%
\pgfpathlineto{\pgfqpoint{5.131001in}{2.658043in}}%
\pgfpathlineto{\pgfqpoint{5.138866in}{2.675319in}}%
\pgfpathlineto{\pgfqpoint{5.124189in}{2.658921in}}%
\pgfpathlineto{\pgfqpoint{5.109533in}{2.642691in}}%
\pgfpathlineto{\pgfqpoint{5.094899in}{2.626629in}}%
\pgfpathlineto{\pgfqpoint{5.080285in}{2.610734in}}%
\pgfpathlineto{\pgfqpoint{5.072433in}{2.593752in}}%
\pgfpathlineto{\pgfqpoint{5.064574in}{2.576603in}}%
\pgfpathlineto{\pgfqpoint{5.056708in}{2.559288in}}%
\pgfpathlineto{\pgfqpoint{5.048836in}{2.541811in}}%
\pgfpathclose%
\pgfusepath{fill}%
\end{pgfscope}%
\begin{pgfscope}%
\pgfpathrectangle{\pgfqpoint{1.254980in}{0.150000in}}{\pgfqpoint{5.490039in}{5.490039in}}%
\pgfusepath{clip}%
\pgfsetbuttcap%
\pgfsetroundjoin%
\definecolor{currentfill}{rgb}{0.195860,0.395433,0.555276}%
\pgfsetfillcolor{currentfill}%
\pgfsetfillopacity{0.700000}%
\pgfsetlinewidth{0.000000pt}%
\definecolor{currentstroke}{rgb}{0.000000,0.000000,0.000000}%
\pgfsetstrokecolor{currentstroke}%
\pgfsetdash{}{0pt}%
\pgfpathmoveto{\pgfqpoint{4.558201in}{1.631111in}}%
\pgfpathlineto{\pgfqpoint{4.572440in}{1.640093in}}%
\pgfpathlineto{\pgfqpoint{4.586694in}{1.649230in}}%
\pgfpathlineto{\pgfqpoint{4.600965in}{1.658524in}}%
\pgfpathlineto{\pgfqpoint{4.615251in}{1.667974in}}%
\pgfpathlineto{\pgfqpoint{4.623253in}{1.687697in}}%
\pgfpathlineto{\pgfqpoint{4.631252in}{1.707438in}}%
\pgfpathlineto{\pgfqpoint{4.639249in}{1.727192in}}%
\pgfpathlineto{\pgfqpoint{4.647243in}{1.746953in}}%
\pgfpathlineto{\pgfqpoint{4.632944in}{1.736909in}}%
\pgfpathlineto{\pgfqpoint{4.618661in}{1.727021in}}%
\pgfpathlineto{\pgfqpoint{4.604394in}{1.717290in}}%
\pgfpathlineto{\pgfqpoint{4.590143in}{1.707716in}}%
\pgfpathlineto{\pgfqpoint{4.582162in}{1.688538in}}%
\pgfpathlineto{\pgfqpoint{4.574177in}{1.669374in}}%
\pgfpathlineto{\pgfqpoint{4.566191in}{1.650229in}}%
\pgfpathlineto{\pgfqpoint{4.558201in}{1.631111in}}%
\pgfpathclose%
\pgfusepath{fill}%
\end{pgfscope}%
\begin{pgfscope}%
\pgfpathrectangle{\pgfqpoint{1.254980in}{0.150000in}}{\pgfqpoint{5.490039in}{5.490039in}}%
\pgfusepath{clip}%
\pgfsetbuttcap%
\pgfsetroundjoin%
\definecolor{currentfill}{rgb}{0.153364,0.497000,0.557724}%
\pgfsetfillcolor{currentfill}%
\pgfsetfillopacity{0.700000}%
\pgfsetlinewidth{0.000000pt}%
\definecolor{currentstroke}{rgb}{0.000000,0.000000,0.000000}%
\pgfsetstrokecolor{currentstroke}%
\pgfsetdash{}{0pt}%
\pgfpathmoveto{\pgfqpoint{4.711102in}{1.904677in}}%
\pgfpathlineto{\pgfqpoint{4.725446in}{1.915987in}}%
\pgfpathlineto{\pgfqpoint{4.739808in}{1.927455in}}%
\pgfpathlineto{\pgfqpoint{4.754188in}{1.939083in}}%
\pgfpathlineto{\pgfqpoint{4.768586in}{1.950871in}}%
\pgfpathlineto{\pgfqpoint{4.776570in}{1.970987in}}%
\pgfpathlineto{\pgfqpoint{4.784551in}{1.991054in}}%
\pgfpathlineto{\pgfqpoint{4.792528in}{2.011065in}}%
\pgfpathlineto{\pgfqpoint{4.800503in}{2.031016in}}%
\pgfpathlineto{\pgfqpoint{4.786090in}{2.018715in}}%
\pgfpathlineto{\pgfqpoint{4.771695in}{2.006574in}}%
\pgfpathlineto{\pgfqpoint{4.757318in}{1.994594in}}%
\pgfpathlineto{\pgfqpoint{4.742960in}{1.982774in}}%
\pgfpathlineto{\pgfqpoint{4.735000in}{1.963324in}}%
\pgfpathlineto{\pgfqpoint{4.727037in}{1.943822in}}%
\pgfpathlineto{\pgfqpoint{4.719071in}{1.924271in}}%
\pgfpathlineto{\pgfqpoint{4.711102in}{1.904677in}}%
\pgfpathclose%
\pgfusepath{fill}%
\end{pgfscope}%
\begin{pgfscope}%
\pgfpathrectangle{\pgfqpoint{1.254980in}{0.150000in}}{\pgfqpoint{5.490039in}{5.490039in}}%
\pgfusepath{clip}%
\pgfsetbuttcap%
\pgfsetroundjoin%
\definecolor{currentfill}{rgb}{0.269308,0.218818,0.509577}%
\pgfsetfillcolor{currentfill}%
\pgfsetfillopacity{0.700000}%
\pgfsetlinewidth{0.000000pt}%
\definecolor{currentstroke}{rgb}{0.000000,0.000000,0.000000}%
\pgfsetstrokecolor{currentstroke}%
\pgfsetdash{}{0pt}%
\pgfpathmoveto{\pgfqpoint{4.284865in}{1.223646in}}%
\pgfpathlineto{\pgfqpoint{4.298953in}{1.228052in}}%
\pgfpathlineto{\pgfqpoint{4.313053in}{1.232608in}}%
\pgfpathlineto{\pgfqpoint{4.327166in}{1.237315in}}%
\pgfpathlineto{\pgfqpoint{4.341292in}{1.242173in}}%
\pgfpathlineto{\pgfqpoint{4.349322in}{1.258790in}}%
\pgfpathlineto{\pgfqpoint{4.357350in}{1.275563in}}%
\pgfpathlineto{\pgfqpoint{4.365375in}{1.292486in}}%
\pgfpathlineto{\pgfqpoint{4.373397in}{1.309552in}}%
\pgfpathlineto{\pgfqpoint{4.359266in}{1.303971in}}%
\pgfpathlineto{\pgfqpoint{4.345149in}{1.298541in}}%
\pgfpathlineto{\pgfqpoint{4.331045in}{1.293263in}}%
\pgfpathlineto{\pgfqpoint{4.316953in}{1.288136in}}%
\pgfpathlineto{\pgfqpoint{4.308936in}{1.271783in}}%
\pgfpathlineto{\pgfqpoint{4.300915in}{1.255578in}}%
\pgfpathlineto{\pgfqpoint{4.292892in}{1.239531in}}%
\pgfpathlineto{\pgfqpoint{4.284865in}{1.223646in}}%
\pgfpathclose%
\pgfusepath{fill}%
\end{pgfscope}%
\begin{pgfscope}%
\pgfpathrectangle{\pgfqpoint{1.254980in}{0.150000in}}{\pgfqpoint{5.490039in}{5.490039in}}%
\pgfusepath{clip}%
\pgfsetbuttcap%
\pgfsetroundjoin%
\definecolor{currentfill}{rgb}{0.243113,0.292092,0.538516}%
\pgfsetfillcolor{currentfill}%
\pgfsetfillopacity{0.700000}%
\pgfsetlinewidth{0.000000pt}%
\definecolor{currentstroke}{rgb}{0.000000,0.000000,0.000000}%
\pgfsetstrokecolor{currentstroke}%
\pgfsetdash{}{0pt}%
\pgfpathmoveto{\pgfqpoint{4.405459in}{1.379110in}}%
\pgfpathlineto{\pgfqpoint{4.419610in}{1.385540in}}%
\pgfpathlineto{\pgfqpoint{4.433774in}{1.392123in}}%
\pgfpathlineto{\pgfqpoint{4.447953in}{1.398859in}}%
\pgfpathlineto{\pgfqpoint{4.462145in}{1.405747in}}%
\pgfpathlineto{\pgfqpoint{4.470163in}{1.424109in}}%
\pgfpathlineto{\pgfqpoint{4.478178in}{1.442568in}}%
\pgfpathlineto{\pgfqpoint{4.486190in}{1.461117in}}%
\pgfpathlineto{\pgfqpoint{4.494201in}{1.479750in}}%
\pgfpathlineto{\pgfqpoint{4.479999in}{1.472188in}}%
\pgfpathlineto{\pgfqpoint{4.465812in}{1.464780in}}%
\pgfpathlineto{\pgfqpoint{4.451639in}{1.457526in}}%
\pgfpathlineto{\pgfqpoint{4.437481in}{1.450425in}}%
\pgfpathlineto{\pgfqpoint{4.429479in}{1.432453in}}%
\pgfpathlineto{\pgfqpoint{4.421475in}{1.414573in}}%
\pgfpathlineto{\pgfqpoint{4.413468in}{1.396790in}}%
\pgfpathlineto{\pgfqpoint{4.405459in}{1.379110in}}%
\pgfpathclose%
\pgfusepath{fill}%
\end{pgfscope}%
\begin{pgfscope}%
\pgfpathrectangle{\pgfqpoint{1.254980in}{0.150000in}}{\pgfqpoint{5.490039in}{5.490039in}}%
\pgfusepath{clip}%
\pgfsetbuttcap%
\pgfsetroundjoin%
\definecolor{currentfill}{rgb}{0.352360,0.783011,0.392636}%
\pgfsetfillcolor{currentfill}%
\pgfsetfillopacity{0.700000}%
\pgfsetlinewidth{0.000000pt}%
\definecolor{currentstroke}{rgb}{0.000000,0.000000,0.000000}%
\pgfsetstrokecolor{currentstroke}%
\pgfsetdash{}{0pt}%
\pgfpathmoveto{\pgfqpoint{5.170258in}{2.742618in}}%
\pgfpathlineto{\pgfqpoint{5.184970in}{2.759461in}}%
\pgfpathlineto{\pgfqpoint{5.199704in}{2.776472in}}%
\pgfpathlineto{\pgfqpoint{5.214461in}{2.793654in}}%
\pgfpathlineto{\pgfqpoint{5.222300in}{2.810208in}}%
\pgfpathlineto{\pgfqpoint{5.230131in}{2.826566in}}%
\pgfpathlineto{\pgfqpoint{5.237954in}{2.842726in}}%
\pgfpathlineto{\pgfqpoint{5.245769in}{2.858687in}}%
\pgfpathlineto{\pgfqpoint{5.231001in}{2.841260in}}%
\pgfpathlineto{\pgfqpoint{5.216255in}{2.824005in}}%
\pgfpathlineto{\pgfqpoint{5.201531in}{2.806918in}}%
\pgfpathlineto{\pgfqpoint{5.193725in}{2.791132in}}%
\pgfpathlineto{\pgfqpoint{5.185910in}{2.775151in}}%
\pgfpathlineto{\pgfqpoint{5.178088in}{2.758980in}}%
\pgfpathlineto{\pgfqpoint{5.170258in}{2.742618in}}%
\pgfpathclose%
\pgfusepath{fill}%
\end{pgfscope}%
\begin{pgfscope}%
\pgfpathrectangle{\pgfqpoint{1.254980in}{0.150000in}}{\pgfqpoint{5.490039in}{5.490039in}}%
\pgfusepath{clip}%
\pgfsetbuttcap%
\pgfsetroundjoin%
\definecolor{currentfill}{rgb}{0.120092,0.600104,0.542530}%
\pgfsetfillcolor{currentfill}%
\pgfsetfillopacity{0.700000}%
\pgfsetlinewidth{0.000000pt}%
\definecolor{currentstroke}{rgb}{0.000000,0.000000,0.000000}%
\pgfsetstrokecolor{currentstroke}%
\pgfsetdash{}{0pt}%
\pgfpathmoveto{\pgfqpoint{4.864168in}{2.187940in}}%
\pgfpathlineto{\pgfqpoint{4.878630in}{2.201344in}}%
\pgfpathlineto{\pgfqpoint{4.893112in}{2.214910in}}%
\pgfpathlineto{\pgfqpoint{4.907613in}{2.228639in}}%
\pgfpathlineto{\pgfqpoint{4.922133in}{2.242531in}}%
\pgfpathlineto{\pgfqpoint{4.930089in}{2.262186in}}%
\pgfpathlineto{\pgfqpoint{4.938040in}{2.281731in}}%
\pgfpathlineto{\pgfqpoint{4.945987in}{2.301162in}}%
\pgfpathlineto{\pgfqpoint{4.953930in}{2.320476in}}%
\pgfpathlineto{\pgfqpoint{4.939394in}{2.306155in}}%
\pgfpathlineto{\pgfqpoint{4.924877in}{2.291999in}}%
\pgfpathlineto{\pgfqpoint{4.910381in}{2.278006in}}%
\pgfpathlineto{\pgfqpoint{4.895904in}{2.264176in}}%
\pgfpathlineto{\pgfqpoint{4.887976in}{2.245278in}}%
\pgfpathlineto{\pgfqpoint{4.880045in}{2.226270in}}%
\pgfpathlineto{\pgfqpoint{4.872108in}{2.207156in}}%
\pgfpathlineto{\pgfqpoint{4.864168in}{2.187940in}}%
\pgfpathclose%
\pgfusepath{fill}%
\end{pgfscope}%
\begin{pgfscope}%
\pgfpathrectangle{\pgfqpoint{1.254980in}{0.150000in}}{\pgfqpoint{5.490039in}{5.490039in}}%
\pgfusepath{clip}%
\pgfsetbuttcap%
\pgfsetroundjoin%
\definecolor{currentfill}{rgb}{0.208623,0.367752,0.552675}%
\pgfsetfillcolor{currentfill}%
\pgfsetfillopacity{0.700000}%
\pgfsetlinewidth{0.000000pt}%
\definecolor{currentstroke}{rgb}{0.000000,0.000000,0.000000}%
\pgfsetstrokecolor{currentstroke}%
\pgfsetdash{}{0pt}%
\pgfpathmoveto{\pgfqpoint{4.526220in}{1.554999in}}%
\pgfpathlineto{\pgfqpoint{4.540447in}{1.563362in}}%
\pgfpathlineto{\pgfqpoint{4.554690in}{1.571879in}}%
\pgfpathlineto{\pgfqpoint{4.568948in}{1.580551in}}%
\pgfpathlineto{\pgfqpoint{4.583222in}{1.589378in}}%
\pgfpathlineto{\pgfqpoint{4.591233in}{1.608972in}}%
\pgfpathlineto{\pgfqpoint{4.599241in}{1.628606in}}%
\pgfpathlineto{\pgfqpoint{4.607247in}{1.648275in}}%
\pgfpathlineto{\pgfqpoint{4.615251in}{1.667974in}}%
\pgfpathlineto{\pgfqpoint{4.600965in}{1.658524in}}%
\pgfpathlineto{\pgfqpoint{4.586694in}{1.649230in}}%
\pgfpathlineto{\pgfqpoint{4.572440in}{1.640093in}}%
\pgfpathlineto{\pgfqpoint{4.558201in}{1.631111in}}%
\pgfpathlineto{\pgfqpoint{4.550210in}{1.612022in}}%
\pgfpathlineto{\pgfqpoint{4.542216in}{1.592971in}}%
\pgfpathlineto{\pgfqpoint{4.534219in}{1.573961in}}%
\pgfpathlineto{\pgfqpoint{4.526220in}{1.554999in}}%
\pgfpathclose%
\pgfusepath{fill}%
\end{pgfscope}%
\begin{pgfscope}%
\pgfpathrectangle{\pgfqpoint{1.254980in}{0.150000in}}{\pgfqpoint{5.490039in}{5.490039in}}%
\pgfusepath{clip}%
\pgfsetbuttcap%
\pgfsetroundjoin%
\definecolor{currentfill}{rgb}{0.180653,0.701402,0.488189}%
\pgfsetfillcolor{currentfill}%
\pgfsetfillopacity{0.700000}%
\pgfsetlinewidth{0.000000pt}%
\definecolor{currentstroke}{rgb}{0.000000,0.000000,0.000000}%
\pgfsetstrokecolor{currentstroke}%
\pgfsetdash{}{0pt}%
\pgfpathmoveto{\pgfqpoint{5.017289in}{2.470332in}}%
\pgfpathlineto{\pgfqpoint{5.031875in}{2.485585in}}%
\pgfpathlineto{\pgfqpoint{5.046483in}{2.501003in}}%
\pgfpathlineto{\pgfqpoint{5.061111in}{2.516588in}}%
\pgfpathlineto{\pgfqpoint{5.075760in}{2.532340in}}%
\pgfpathlineto{\pgfqpoint{5.083671in}{2.550789in}}%
\pgfpathlineto{\pgfqpoint{5.091575in}{2.569079in}}%
\pgfpathlineto{\pgfqpoint{5.099473in}{2.587207in}}%
\pgfpathlineto{\pgfqpoint{5.107365in}{2.605170in}}%
\pgfpathlineto{\pgfqpoint{5.092701in}{2.589080in}}%
\pgfpathlineto{\pgfqpoint{5.078058in}{2.573156in}}%
\pgfpathlineto{\pgfqpoint{5.063437in}{2.557400in}}%
\pgfpathlineto{\pgfqpoint{5.048836in}{2.541811in}}%
\pgfpathlineto{\pgfqpoint{5.040958in}{2.524174in}}%
\pgfpathlineto{\pgfqpoint{5.033074in}{2.506380in}}%
\pgfpathlineto{\pgfqpoint{5.025185in}{2.488432in}}%
\pgfpathlineto{\pgfqpoint{5.017289in}{2.470332in}}%
\pgfpathclose%
\pgfusepath{fill}%
\end{pgfscope}%
\begin{pgfscope}%
\pgfpathrectangle{\pgfqpoint{1.254980in}{0.150000in}}{\pgfqpoint{5.490039in}{5.490039in}}%
\pgfusepath{clip}%
\pgfsetbuttcap%
\pgfsetroundjoin%
\definecolor{currentfill}{rgb}{0.163625,0.471133,0.558148}%
\pgfsetfillcolor{currentfill}%
\pgfsetfillopacity{0.700000}%
\pgfsetlinewidth{0.000000pt}%
\definecolor{currentstroke}{rgb}{0.000000,0.000000,0.000000}%
\pgfsetstrokecolor{currentstroke}%
\pgfsetdash{}{0pt}%
\pgfpathmoveto{\pgfqpoint{4.679195in}{1.825965in}}%
\pgfpathlineto{\pgfqpoint{4.693525in}{1.836735in}}%
\pgfpathlineto{\pgfqpoint{4.707873in}{1.847663in}}%
\pgfpathlineto{\pgfqpoint{4.722237in}{1.858750in}}%
\pgfpathlineto{\pgfqpoint{4.736620in}{1.869995in}}%
\pgfpathlineto{\pgfqpoint{4.744616in}{1.890265in}}%
\pgfpathlineto{\pgfqpoint{4.752609in}{1.910504in}}%
\pgfpathlineto{\pgfqpoint{4.760599in}{1.930708in}}%
\pgfpathlineto{\pgfqpoint{4.768586in}{1.950871in}}%
\pgfpathlineto{\pgfqpoint{4.754188in}{1.939083in}}%
\pgfpathlineto{\pgfqpoint{4.739808in}{1.927455in}}%
\pgfpathlineto{\pgfqpoint{4.725446in}{1.915987in}}%
\pgfpathlineto{\pgfqpoint{4.711102in}{1.904677in}}%
\pgfpathlineto{\pgfqpoint{4.703129in}{1.885045in}}%
\pgfpathlineto{\pgfqpoint{4.695154in}{1.865379in}}%
\pgfpathlineto{\pgfqpoint{4.687176in}{1.845684in}}%
\pgfpathlineto{\pgfqpoint{4.679195in}{1.825965in}}%
\pgfpathclose%
\pgfusepath{fill}%
\end{pgfscope}%
\begin{pgfscope}%
\pgfpathrectangle{\pgfqpoint{1.254980in}{0.150000in}}{\pgfqpoint{5.490039in}{5.490039in}}%
\pgfusepath{clip}%
\pgfsetbuttcap%
\pgfsetroundjoin%
\definecolor{currentfill}{rgb}{0.252194,0.269783,0.531579}%
\pgfsetfillcolor{currentfill}%
\pgfsetfillopacity{0.700000}%
\pgfsetlinewidth{0.000000pt}%
\definecolor{currentstroke}{rgb}{0.000000,0.000000,0.000000}%
\pgfsetstrokecolor{currentstroke}%
\pgfsetdash{}{0pt}%
\pgfpathmoveto{\pgfqpoint{4.373397in}{1.309552in}}%
\pgfpathlineto{\pgfqpoint{4.387541in}{1.315285in}}%
\pgfpathlineto{\pgfqpoint{4.401698in}{1.321170in}}%
\pgfpathlineto{\pgfqpoint{4.415869in}{1.327206in}}%
\pgfpathlineto{\pgfqpoint{4.430053in}{1.333394in}}%
\pgfpathlineto{\pgfqpoint{4.438080in}{1.351305in}}%
\pgfpathlineto{\pgfqpoint{4.446104in}{1.369339in}}%
\pgfpathlineto{\pgfqpoint{4.454126in}{1.387488in}}%
\pgfpathlineto{\pgfqpoint{4.462145in}{1.405747in}}%
\pgfpathlineto{\pgfqpoint{4.447953in}{1.398859in}}%
\pgfpathlineto{\pgfqpoint{4.433774in}{1.392123in}}%
\pgfpathlineto{\pgfqpoint{4.419610in}{1.385540in}}%
\pgfpathlineto{\pgfqpoint{4.405459in}{1.379110in}}%
\pgfpathlineto{\pgfqpoint{4.397447in}{1.361539in}}%
\pgfpathlineto{\pgfqpoint{4.389433in}{1.344085in}}%
\pgfpathlineto{\pgfqpoint{4.381416in}{1.326754in}}%
\pgfpathlineto{\pgfqpoint{4.373397in}{1.309552in}}%
\pgfpathclose%
\pgfusepath{fill}%
\end{pgfscope}%
\begin{pgfscope}%
\pgfpathrectangle{\pgfqpoint{1.254980in}{0.150000in}}{\pgfqpoint{5.490039in}{5.490039in}}%
\pgfusepath{clip}%
\pgfsetbuttcap%
\pgfsetroundjoin%
\definecolor{currentfill}{rgb}{0.124395,0.578002,0.548287}%
\pgfsetfillcolor{currentfill}%
\pgfsetfillopacity{0.700000}%
\pgfsetlinewidth{0.000000pt}%
\definecolor{currentstroke}{rgb}{0.000000,0.000000,0.000000}%
\pgfsetstrokecolor{currentstroke}%
\pgfsetdash{}{0pt}%
\pgfpathmoveto{\pgfqpoint{4.832366in}{2.110133in}}%
\pgfpathlineto{\pgfqpoint{4.846813in}{2.123080in}}%
\pgfpathlineto{\pgfqpoint{4.861278in}{2.136190in}}%
\pgfpathlineto{\pgfqpoint{4.875764in}{2.149461in}}%
\pgfpathlineto{\pgfqpoint{4.890268in}{2.162895in}}%
\pgfpathlineto{\pgfqpoint{4.898240in}{2.182948in}}%
\pgfpathlineto{\pgfqpoint{4.906209in}{2.202908in}}%
\pgfpathlineto{\pgfqpoint{4.914173in}{2.222770in}}%
\pgfpathlineto{\pgfqpoint{4.922133in}{2.242531in}}%
\pgfpathlineto{\pgfqpoint{4.907613in}{2.228639in}}%
\pgfpathlineto{\pgfqpoint{4.893112in}{2.214910in}}%
\pgfpathlineto{\pgfqpoint{4.878630in}{2.201344in}}%
\pgfpathlineto{\pgfqpoint{4.864168in}{2.187940in}}%
\pgfpathlineto{\pgfqpoint{4.856223in}{2.168626in}}%
\pgfpathlineto{\pgfqpoint{4.848275in}{2.149217in}}%
\pgfpathlineto{\pgfqpoint{4.840322in}{2.129718in}}%
\pgfpathlineto{\pgfqpoint{4.832366in}{2.110133in}}%
\pgfpathclose%
\pgfusepath{fill}%
\end{pgfscope}%
\begin{pgfscope}%
\pgfpathrectangle{\pgfqpoint{1.254980in}{0.150000in}}{\pgfqpoint{5.490039in}{5.490039in}}%
\pgfusepath{clip}%
\pgfsetbuttcap%
\pgfsetroundjoin%
\definecolor{currentfill}{rgb}{0.311925,0.767822,0.415586}%
\pgfsetfillcolor{currentfill}%
\pgfsetfillopacity{0.700000}%
\pgfsetlinewidth{0.000000pt}%
\definecolor{currentstroke}{rgb}{0.000000,0.000000,0.000000}%
\pgfsetstrokecolor{currentstroke}%
\pgfsetdash{}{0pt}%
\pgfpathmoveto{\pgfqpoint{5.138866in}{2.675319in}}%
\pgfpathlineto{\pgfqpoint{5.153566in}{2.691886in}}%
\pgfpathlineto{\pgfqpoint{5.168287in}{2.708621in}}%
\pgfpathlineto{\pgfqpoint{5.183031in}{2.725525in}}%
\pgfpathlineto{\pgfqpoint{5.190900in}{2.742840in}}%
\pgfpathlineto{\pgfqpoint{5.198761in}{2.759968in}}%
\pgfpathlineto{\pgfqpoint{5.206615in}{2.776906in}}%
\pgfpathlineto{\pgfqpoint{5.214461in}{2.793654in}}%
\pgfpathlineto{\pgfqpoint{5.199704in}{2.776472in}}%
\pgfpathlineto{\pgfqpoint{5.184970in}{2.759461in}}%
\pgfpathlineto{\pgfqpoint{5.170258in}{2.742618in}}%
\pgfpathlineto{\pgfqpoint{5.162421in}{2.726069in}}%
\pgfpathlineto{\pgfqpoint{5.154577in}{2.709335in}}%
\pgfpathlineto{\pgfqpoint{5.146725in}{2.692417in}}%
\pgfpathlineto{\pgfqpoint{5.138866in}{2.675319in}}%
\pgfpathclose%
\pgfusepath{fill}%
\end{pgfscope}%
\begin{pgfscope}%
\pgfpathrectangle{\pgfqpoint{1.254980in}{0.150000in}}{\pgfqpoint{5.490039in}{5.490039in}}%
\pgfusepath{clip}%
\pgfsetbuttcap%
\pgfsetroundjoin%
\definecolor{currentfill}{rgb}{0.220057,0.343307,0.549413}%
\pgfsetfillcolor{currentfill}%
\pgfsetfillopacity{0.700000}%
\pgfsetlinewidth{0.000000pt}%
\definecolor{currentstroke}{rgb}{0.000000,0.000000,0.000000}%
\pgfsetstrokecolor{currentstroke}%
\pgfsetdash{}{0pt}%
\pgfpathmoveto{\pgfqpoint{4.494201in}{1.479750in}}%
\pgfpathlineto{\pgfqpoint{4.508418in}{1.487465in}}%
\pgfpathlineto{\pgfqpoint{4.522649in}{1.495334in}}%
\pgfpathlineto{\pgfqpoint{4.536896in}{1.503358in}}%
\pgfpathlineto{\pgfqpoint{4.551158in}{1.511535in}}%
\pgfpathlineto{\pgfqpoint{4.559177in}{1.530905in}}%
\pgfpathlineto{\pgfqpoint{4.567194in}{1.550339in}}%
\pgfpathlineto{\pgfqpoint{4.575209in}{1.569832in}}%
\pgfpathlineto{\pgfqpoint{4.583222in}{1.589378in}}%
\pgfpathlineto{\pgfqpoint{4.568948in}{1.580551in}}%
\pgfpathlineto{\pgfqpoint{4.554690in}{1.571879in}}%
\pgfpathlineto{\pgfqpoint{4.540447in}{1.563362in}}%
\pgfpathlineto{\pgfqpoint{4.526220in}{1.554999in}}%
\pgfpathlineto{\pgfqpoint{4.518219in}{1.536091in}}%
\pgfpathlineto{\pgfqpoint{4.510215in}{1.517243in}}%
\pgfpathlineto{\pgfqpoint{4.502209in}{1.498461in}}%
\pgfpathlineto{\pgfqpoint{4.494201in}{1.479750in}}%
\pgfpathclose%
\pgfusepath{fill}%
\end{pgfscope}%
\begin{pgfscope}%
\pgfpathrectangle{\pgfqpoint{1.254980in}{0.150000in}}{\pgfqpoint{5.490039in}{5.490039in}}%
\pgfusepath{clip}%
\pgfsetbuttcap%
\pgfsetroundjoin%
\definecolor{currentfill}{rgb}{0.172719,0.448791,0.557885}%
\pgfsetfillcolor{currentfill}%
\pgfsetfillopacity{0.700000}%
\pgfsetlinewidth{0.000000pt}%
\definecolor{currentstroke}{rgb}{0.000000,0.000000,0.000000}%
\pgfsetstrokecolor{currentstroke}%
\pgfsetdash{}{0pt}%
\pgfpathmoveto{\pgfqpoint{4.647243in}{1.746953in}}%
\pgfpathlineto{\pgfqpoint{4.661560in}{1.757155in}}%
\pgfpathlineto{\pgfqpoint{4.675893in}{1.767514in}}%
\pgfpathlineto{\pgfqpoint{4.690243in}{1.778031in}}%
\pgfpathlineto{\pgfqpoint{4.704610in}{1.788705in}}%
\pgfpathlineto{\pgfqpoint{4.712617in}{1.809048in}}%
\pgfpathlineto{\pgfqpoint{4.720620in}{1.829381in}}%
\pgfpathlineto{\pgfqpoint{4.728621in}{1.849698in}}%
\pgfpathlineto{\pgfqpoint{4.736620in}{1.869995in}}%
\pgfpathlineto{\pgfqpoint{4.722237in}{1.858750in}}%
\pgfpathlineto{\pgfqpoint{4.707873in}{1.847663in}}%
\pgfpathlineto{\pgfqpoint{4.693525in}{1.836735in}}%
\pgfpathlineto{\pgfqpoint{4.679195in}{1.825965in}}%
\pgfpathlineto{\pgfqpoint{4.671211in}{1.806228in}}%
\pgfpathlineto{\pgfqpoint{4.663224in}{1.786477in}}%
\pgfpathlineto{\pgfqpoint{4.655235in}{1.766717in}}%
\pgfpathlineto{\pgfqpoint{4.647243in}{1.746953in}}%
\pgfpathclose%
\pgfusepath{fill}%
\end{pgfscope}%
\begin{pgfscope}%
\pgfpathrectangle{\pgfqpoint{1.254980in}{0.150000in}}{\pgfqpoint{5.490039in}{5.490039in}}%
\pgfusepath{clip}%
\pgfsetbuttcap%
\pgfsetroundjoin%
\definecolor{currentfill}{rgb}{0.153894,0.680203,0.504172}%
\pgfsetfillcolor{currentfill}%
\pgfsetfillopacity{0.700000}%
\pgfsetlinewidth{0.000000pt}%
\definecolor{currentstroke}{rgb}{0.000000,0.000000,0.000000}%
\pgfsetstrokecolor{currentstroke}%
\pgfsetdash{}{0pt}%
\pgfpathmoveto{\pgfqpoint{4.985651in}{2.396485in}}%
\pgfpathlineto{\pgfqpoint{5.000223in}{2.411369in}}%
\pgfpathlineto{\pgfqpoint{5.014815in}{2.426419in}}%
\pgfpathlineto{\pgfqpoint{5.029428in}{2.441634in}}%
\pgfpathlineto{\pgfqpoint{5.044062in}{2.457015in}}%
\pgfpathlineto{\pgfqpoint{5.051995in}{2.476069in}}%
\pgfpathlineto{\pgfqpoint{5.059923in}{2.494977in}}%
\pgfpathlineto{\pgfqpoint{5.067844in}{2.513734in}}%
\pgfpathlineto{\pgfqpoint{5.075760in}{2.532340in}}%
\pgfpathlineto{\pgfqpoint{5.061111in}{2.516588in}}%
\pgfpathlineto{\pgfqpoint{5.046483in}{2.501003in}}%
\pgfpathlineto{\pgfqpoint{5.031875in}{2.485585in}}%
\pgfpathlineto{\pgfqpoint{5.017289in}{2.470332in}}%
\pgfpathlineto{\pgfqpoint{5.009388in}{2.452085in}}%
\pgfpathlineto{\pgfqpoint{5.001481in}{2.433692in}}%
\pgfpathlineto{\pgfqpoint{4.993568in}{2.415158in}}%
\pgfpathlineto{\pgfqpoint{4.985651in}{2.396485in}}%
\pgfpathclose%
\pgfusepath{fill}%
\end{pgfscope}%
\begin{pgfscope}%
\pgfpathrectangle{\pgfqpoint{1.254980in}{0.150000in}}{\pgfqpoint{5.490039in}{5.490039in}}%
\pgfusepath{clip}%
\pgfsetbuttcap%
\pgfsetroundjoin%
\definecolor{currentfill}{rgb}{0.132444,0.552216,0.553018}%
\pgfsetfillcolor{currentfill}%
\pgfsetfillopacity{0.700000}%
\pgfsetlinewidth{0.000000pt}%
\definecolor{currentstroke}{rgb}{0.000000,0.000000,0.000000}%
\pgfsetstrokecolor{currentstroke}%
\pgfsetdash{}{0pt}%
\pgfpathmoveto{\pgfqpoint{4.800503in}{2.031016in}}%
\pgfpathlineto{\pgfqpoint{4.814934in}{2.043478in}}%
\pgfpathlineto{\pgfqpoint{4.829384in}{2.056101in}}%
\pgfpathlineto{\pgfqpoint{4.843853in}{2.068885in}}%
\pgfpathlineto{\pgfqpoint{4.858341in}{2.081830in}}%
\pgfpathlineto{\pgfqpoint{4.866328in}{2.102215in}}%
\pgfpathlineto{\pgfqpoint{4.874312in}{2.122524in}}%
\pgfpathlineto{\pgfqpoint{4.882292in}{2.142752in}}%
\pgfpathlineto{\pgfqpoint{4.890268in}{2.162895in}}%
\pgfpathlineto{\pgfqpoint{4.875764in}{2.149461in}}%
\pgfpathlineto{\pgfqpoint{4.861278in}{2.136190in}}%
\pgfpathlineto{\pgfqpoint{4.846813in}{2.123080in}}%
\pgfpathlineto{\pgfqpoint{4.832366in}{2.110133in}}%
\pgfpathlineto{\pgfqpoint{4.824405in}{2.090466in}}%
\pgfpathlineto{\pgfqpoint{4.816441in}{2.070721in}}%
\pgfpathlineto{\pgfqpoint{4.808474in}{2.050903in}}%
\pgfpathlineto{\pgfqpoint{4.800503in}{2.031016in}}%
\pgfpathclose%
\pgfusepath{fill}%
\end{pgfscope}%
\begin{pgfscope}%
\pgfpathrectangle{\pgfqpoint{1.254980in}{0.150000in}}{\pgfqpoint{5.490039in}{5.490039in}}%
\pgfusepath{clip}%
\pgfsetbuttcap%
\pgfsetroundjoin%
\definecolor{currentfill}{rgb}{0.262138,0.242286,0.520837}%
\pgfsetfillcolor{currentfill}%
\pgfsetfillopacity{0.700000}%
\pgfsetlinewidth{0.000000pt}%
\definecolor{currentstroke}{rgb}{0.000000,0.000000,0.000000}%
\pgfsetstrokecolor{currentstroke}%
\pgfsetdash{}{0pt}%
\pgfpathmoveto{\pgfqpoint{4.341292in}{1.242173in}}%
\pgfpathlineto{\pgfqpoint{4.355430in}{1.247181in}}%
\pgfpathlineto{\pgfqpoint{4.369582in}{1.252341in}}%
\pgfpathlineto{\pgfqpoint{4.383746in}{1.257651in}}%
\pgfpathlineto{\pgfqpoint{4.397924in}{1.263113in}}%
\pgfpathlineto{\pgfqpoint{4.405960in}{1.280465in}}%
\pgfpathlineto{\pgfqpoint{4.413993in}{1.297968in}}%
\pgfpathlineto{\pgfqpoint{4.422024in}{1.315613in}}%
\pgfpathlineto{\pgfqpoint{4.430053in}{1.333394in}}%
\pgfpathlineto{\pgfqpoint{4.415869in}{1.327206in}}%
\pgfpathlineto{\pgfqpoint{4.401698in}{1.321170in}}%
\pgfpathlineto{\pgfqpoint{4.387541in}{1.315285in}}%
\pgfpathlineto{\pgfqpoint{4.373397in}{1.309552in}}%
\pgfpathlineto{\pgfqpoint{4.365375in}{1.292486in}}%
\pgfpathlineto{\pgfqpoint{4.357350in}{1.275563in}}%
\pgfpathlineto{\pgfqpoint{4.349322in}{1.258790in}}%
\pgfpathlineto{\pgfqpoint{4.341292in}{1.242173in}}%
\pgfpathclose%
\pgfusepath{fill}%
\end{pgfscope}%
\begin{pgfscope}%
\pgfpathrectangle{\pgfqpoint{1.254980in}{0.150000in}}{\pgfqpoint{5.490039in}{5.490039in}}%
\pgfusepath{clip}%
\pgfsetbuttcap%
\pgfsetroundjoin%
\definecolor{currentfill}{rgb}{0.183898,0.422383,0.556944}%
\pgfsetfillcolor{currentfill}%
\pgfsetfillopacity{0.700000}%
\pgfsetlinewidth{0.000000pt}%
\definecolor{currentstroke}{rgb}{0.000000,0.000000,0.000000}%
\pgfsetstrokecolor{currentstroke}%
\pgfsetdash{}{0pt}%
\pgfpathmoveto{\pgfqpoint{4.615251in}{1.667974in}}%
\pgfpathlineto{\pgfqpoint{4.629554in}{1.677580in}}%
\pgfpathlineto{\pgfqpoint{4.643873in}{1.687342in}}%
\pgfpathlineto{\pgfqpoint{4.658209in}{1.697261in}}%
\pgfpathlineto{\pgfqpoint{4.672561in}{1.707337in}}%
\pgfpathlineto{\pgfqpoint{4.680577in}{1.727667in}}%
\pgfpathlineto{\pgfqpoint{4.688590in}{1.748009in}}%
\pgfpathlineto{\pgfqpoint{4.696602in}{1.768357in}}%
\pgfpathlineto{\pgfqpoint{4.704610in}{1.788705in}}%
\pgfpathlineto{\pgfqpoint{4.690243in}{1.778031in}}%
\pgfpathlineto{\pgfqpoint{4.675893in}{1.767514in}}%
\pgfpathlineto{\pgfqpoint{4.661560in}{1.757155in}}%
\pgfpathlineto{\pgfqpoint{4.647243in}{1.746953in}}%
\pgfpathlineto{\pgfqpoint{4.639249in}{1.727192in}}%
\pgfpathlineto{\pgfqpoint{4.631252in}{1.707438in}}%
\pgfpathlineto{\pgfqpoint{4.623253in}{1.687697in}}%
\pgfpathlineto{\pgfqpoint{4.615251in}{1.667974in}}%
\pgfpathclose%
\pgfusepath{fill}%
\end{pgfscope}%
\begin{pgfscope}%
\pgfpathrectangle{\pgfqpoint{1.254980in}{0.150000in}}{\pgfqpoint{5.490039in}{5.490039in}}%
\pgfusepath{clip}%
\pgfsetbuttcap%
\pgfsetroundjoin%
\definecolor{currentfill}{rgb}{0.266941,0.748751,0.440573}%
\pgfsetfillcolor{currentfill}%
\pgfsetfillopacity{0.700000}%
\pgfsetlinewidth{0.000000pt}%
\definecolor{currentstroke}{rgb}{0.000000,0.000000,0.000000}%
\pgfsetstrokecolor{currentstroke}%
\pgfsetdash{}{0pt}%
\pgfpathmoveto{\pgfqpoint{5.107365in}{2.605170in}}%
\pgfpathlineto{\pgfqpoint{5.122050in}{2.621429in}}%
\pgfpathlineto{\pgfqpoint{5.136758in}{2.637855in}}%
\pgfpathlineto{\pgfqpoint{5.151487in}{2.654450in}}%
\pgfpathlineto{\pgfqpoint{5.159383in}{2.672486in}}%
\pgfpathlineto{\pgfqpoint{5.167273in}{2.690346in}}%
\pgfpathlineto{\pgfqpoint{5.175155in}{2.708026in}}%
\pgfpathlineto{\pgfqpoint{5.183031in}{2.725525in}}%
\pgfpathlineto{\pgfqpoint{5.168287in}{2.708621in}}%
\pgfpathlineto{\pgfqpoint{5.153566in}{2.691886in}}%
\pgfpathlineto{\pgfqpoint{5.138866in}{2.675319in}}%
\pgfpathlineto{\pgfqpoint{5.131001in}{2.658043in}}%
\pgfpathlineto{\pgfqpoint{5.123129in}{2.640591in}}%
\pgfpathlineto{\pgfqpoint{5.115250in}{2.622966in}}%
\pgfpathlineto{\pgfqpoint{5.107365in}{2.605170in}}%
\pgfpathclose%
\pgfusepath{fill}%
\end{pgfscope}%
\begin{pgfscope}%
\pgfpathrectangle{\pgfqpoint{1.254980in}{0.150000in}}{\pgfqpoint{5.490039in}{5.490039in}}%
\pgfusepath{clip}%
\pgfsetbuttcap%
\pgfsetroundjoin%
\definecolor{currentfill}{rgb}{0.231674,0.318106,0.544834}%
\pgfsetfillcolor{currentfill}%
\pgfsetfillopacity{0.700000}%
\pgfsetlinewidth{0.000000pt}%
\definecolor{currentstroke}{rgb}{0.000000,0.000000,0.000000}%
\pgfsetstrokecolor{currentstroke}%
\pgfsetdash{}{0pt}%
\pgfpathmoveto{\pgfqpoint{4.462145in}{1.405747in}}%
\pgfpathlineto{\pgfqpoint{4.476352in}{1.412788in}}%
\pgfpathlineto{\pgfqpoint{4.490574in}{1.419983in}}%
\pgfpathlineto{\pgfqpoint{4.504810in}{1.427330in}}%
\pgfpathlineto{\pgfqpoint{4.519061in}{1.434830in}}%
\pgfpathlineto{\pgfqpoint{4.527088in}{1.453878in}}%
\pgfpathlineto{\pgfqpoint{4.535113in}{1.473016in}}%
\pgfpathlineto{\pgfqpoint{4.543137in}{1.492237in}}%
\pgfpathlineto{\pgfqpoint{4.551158in}{1.511535in}}%
\pgfpathlineto{\pgfqpoint{4.536896in}{1.503358in}}%
\pgfpathlineto{\pgfqpoint{4.522649in}{1.495334in}}%
\pgfpathlineto{\pgfqpoint{4.508418in}{1.487465in}}%
\pgfpathlineto{\pgfqpoint{4.494201in}{1.479750in}}%
\pgfpathlineto{\pgfqpoint{4.486190in}{1.461117in}}%
\pgfpathlineto{\pgfqpoint{4.478178in}{1.442568in}}%
\pgfpathlineto{\pgfqpoint{4.470163in}{1.424109in}}%
\pgfpathlineto{\pgfqpoint{4.462145in}{1.405747in}}%
\pgfpathclose%
\pgfusepath{fill}%
\end{pgfscope}%
\begin{pgfscope}%
\pgfpathrectangle{\pgfqpoint{1.254980in}{0.150000in}}{\pgfqpoint{5.490039in}{5.490039in}}%
\pgfusepath{clip}%
\pgfsetbuttcap%
\pgfsetroundjoin%
\definecolor{currentfill}{rgb}{0.134692,0.658636,0.517649}%
\pgfsetfillcolor{currentfill}%
\pgfsetfillopacity{0.700000}%
\pgfsetlinewidth{0.000000pt}%
\definecolor{currentstroke}{rgb}{0.000000,0.000000,0.000000}%
\pgfsetstrokecolor{currentstroke}%
\pgfsetdash{}{0pt}%
\pgfpathmoveto{\pgfqpoint{4.953930in}{2.320476in}}%
\pgfpathlineto{\pgfqpoint{4.968486in}{2.334961in}}%
\pgfpathlineto{\pgfqpoint{4.983063in}{2.349611in}}%
\pgfpathlineto{\pgfqpoint{4.997660in}{2.364426in}}%
\pgfpathlineto{\pgfqpoint{5.012277in}{2.379406in}}%
\pgfpathlineto{\pgfqpoint{5.020231in}{2.399010in}}%
\pgfpathlineto{\pgfqpoint{5.028180in}{2.418482in}}%
\pgfpathlineto{\pgfqpoint{5.036124in}{2.437819in}}%
\pgfpathlineto{\pgfqpoint{5.044062in}{2.457015in}}%
\pgfpathlineto{\pgfqpoint{5.029428in}{2.441634in}}%
\pgfpathlineto{\pgfqpoint{5.014815in}{2.426419in}}%
\pgfpathlineto{\pgfqpoint{5.000223in}{2.411369in}}%
\pgfpathlineto{\pgfqpoint{4.985651in}{2.396485in}}%
\pgfpathlineto{\pgfqpoint{4.977728in}{2.377677in}}%
\pgfpathlineto{\pgfqpoint{4.969800in}{2.358737in}}%
\pgfpathlineto{\pgfqpoint{4.961867in}{2.339669in}}%
\pgfpathlineto{\pgfqpoint{4.953930in}{2.320476in}}%
\pgfpathclose%
\pgfusepath{fill}%
\end{pgfscope}%
\begin{pgfscope}%
\pgfpathrectangle{\pgfqpoint{1.254980in}{0.150000in}}{\pgfqpoint{5.490039in}{5.490039in}}%
\pgfusepath{clip}%
\pgfsetbuttcap%
\pgfsetroundjoin%
\definecolor{currentfill}{rgb}{0.141935,0.526453,0.555991}%
\pgfsetfillcolor{currentfill}%
\pgfsetfillopacity{0.700000}%
\pgfsetlinewidth{0.000000pt}%
\definecolor{currentstroke}{rgb}{0.000000,0.000000,0.000000}%
\pgfsetstrokecolor{currentstroke}%
\pgfsetdash{}{0pt}%
\pgfpathmoveto{\pgfqpoint{4.768586in}{1.950871in}}%
\pgfpathlineto{\pgfqpoint{4.783002in}{1.962818in}}%
\pgfpathlineto{\pgfqpoint{4.797436in}{1.974925in}}%
\pgfpathlineto{\pgfqpoint{4.811888in}{1.987193in}}%
\pgfpathlineto{\pgfqpoint{4.826359in}{1.999621in}}%
\pgfpathlineto{\pgfqpoint{4.834359in}{2.020264in}}%
\pgfpathlineto{\pgfqpoint{4.842356in}{2.040850in}}%
\pgfpathlineto{\pgfqpoint{4.850350in}{2.061374in}}%
\pgfpathlineto{\pgfqpoint{4.858341in}{2.081830in}}%
\pgfpathlineto{\pgfqpoint{4.843853in}{2.068885in}}%
\pgfpathlineto{\pgfqpoint{4.829384in}{2.056101in}}%
\pgfpathlineto{\pgfqpoint{4.814934in}{2.043478in}}%
\pgfpathlineto{\pgfqpoint{4.800503in}{2.031016in}}%
\pgfpathlineto{\pgfqpoint{4.792528in}{2.011065in}}%
\pgfpathlineto{\pgfqpoint{4.784551in}{1.991054in}}%
\pgfpathlineto{\pgfqpoint{4.776570in}{1.970987in}}%
\pgfpathlineto{\pgfqpoint{4.768586in}{1.950871in}}%
\pgfpathclose%
\pgfusepath{fill}%
\end{pgfscope}%
\begin{pgfscope}%
\pgfpathrectangle{\pgfqpoint{1.254980in}{0.150000in}}{\pgfqpoint{5.490039in}{5.490039in}}%
\pgfusepath{clip}%
\pgfsetbuttcap%
\pgfsetroundjoin%
\definecolor{currentfill}{rgb}{0.194100,0.399323,0.555565}%
\pgfsetfillcolor{currentfill}%
\pgfsetfillopacity{0.700000}%
\pgfsetlinewidth{0.000000pt}%
\definecolor{currentstroke}{rgb}{0.000000,0.000000,0.000000}%
\pgfsetstrokecolor{currentstroke}%
\pgfsetdash{}{0pt}%
\pgfpathmoveto{\pgfqpoint{4.583222in}{1.589378in}}%
\pgfpathlineto{\pgfqpoint{4.597512in}{1.598361in}}%
\pgfpathlineto{\pgfqpoint{4.611817in}{1.607499in}}%
\pgfpathlineto{\pgfqpoint{4.626139in}{1.616792in}}%
\pgfpathlineto{\pgfqpoint{4.640477in}{1.626241in}}%
\pgfpathlineto{\pgfqpoint{4.648501in}{1.646469in}}%
\pgfpathlineto{\pgfqpoint{4.656524in}{1.666732in}}%
\pgfpathlineto{\pgfqpoint{4.664544in}{1.687023in}}%
\pgfpathlineto{\pgfqpoint{4.672561in}{1.707337in}}%
\pgfpathlineto{\pgfqpoint{4.658209in}{1.697261in}}%
\pgfpathlineto{\pgfqpoint{4.643873in}{1.687342in}}%
\pgfpathlineto{\pgfqpoint{4.629554in}{1.677580in}}%
\pgfpathlineto{\pgfqpoint{4.615251in}{1.667974in}}%
\pgfpathlineto{\pgfqpoint{4.607247in}{1.648275in}}%
\pgfpathlineto{\pgfqpoint{4.599241in}{1.628606in}}%
\pgfpathlineto{\pgfqpoint{4.591233in}{1.608972in}}%
\pgfpathlineto{\pgfqpoint{4.583222in}{1.589378in}}%
\pgfpathclose%
\pgfusepath{fill}%
\end{pgfscope}%
\begin{pgfscope}%
\pgfpathrectangle{\pgfqpoint{1.254980in}{0.150000in}}{\pgfqpoint{5.490039in}{5.490039in}}%
\pgfusepath{clip}%
\pgfsetbuttcap%
\pgfsetroundjoin%
\definecolor{currentfill}{rgb}{0.123444,0.636809,0.528763}%
\pgfsetfillcolor{currentfill}%
\pgfsetfillopacity{0.700000}%
\pgfsetlinewidth{0.000000pt}%
\definecolor{currentstroke}{rgb}{0.000000,0.000000,0.000000}%
\pgfsetstrokecolor{currentstroke}%
\pgfsetdash{}{0pt}%
\pgfpathmoveto{\pgfqpoint{4.922133in}{2.242531in}}%
\pgfpathlineto{\pgfqpoint{4.936673in}{2.256587in}}%
\pgfpathlineto{\pgfqpoint{4.951234in}{2.270806in}}%
\pgfpathlineto{\pgfqpoint{4.965814in}{2.285190in}}%
\pgfpathlineto{\pgfqpoint{4.980414in}{2.299738in}}%
\pgfpathlineto{\pgfqpoint{4.988387in}{2.319834in}}%
\pgfpathlineto{\pgfqpoint{4.996355in}{2.339814in}}%
\pgfpathlineto{\pgfqpoint{5.004319in}{2.359672in}}%
\pgfpathlineto{\pgfqpoint{5.012277in}{2.379406in}}%
\pgfpathlineto{\pgfqpoint{4.997660in}{2.364426in}}%
\pgfpathlineto{\pgfqpoint{4.983063in}{2.349611in}}%
\pgfpathlineto{\pgfqpoint{4.968486in}{2.334961in}}%
\pgfpathlineto{\pgfqpoint{4.953930in}{2.320476in}}%
\pgfpathlineto{\pgfqpoint{4.945987in}{2.301162in}}%
\pgfpathlineto{\pgfqpoint{4.938040in}{2.281731in}}%
\pgfpathlineto{\pgfqpoint{4.930089in}{2.262186in}}%
\pgfpathlineto{\pgfqpoint{4.922133in}{2.242531in}}%
\pgfpathclose%
\pgfusepath{fill}%
\end{pgfscope}%
\begin{pgfscope}%
\pgfpathrectangle{\pgfqpoint{1.254980in}{0.150000in}}{\pgfqpoint{5.490039in}{5.490039in}}%
\pgfusepath{clip}%
\pgfsetbuttcap%
\pgfsetroundjoin%
\definecolor{currentfill}{rgb}{0.243113,0.292092,0.538516}%
\pgfsetfillcolor{currentfill}%
\pgfsetfillopacity{0.700000}%
\pgfsetlinewidth{0.000000pt}%
\definecolor{currentstroke}{rgb}{0.000000,0.000000,0.000000}%
\pgfsetstrokecolor{currentstroke}%
\pgfsetdash{}{0pt}%
\pgfpathmoveto{\pgfqpoint{4.430053in}{1.333394in}}%
\pgfpathlineto{\pgfqpoint{4.444252in}{1.339735in}}%
\pgfpathlineto{\pgfqpoint{4.458464in}{1.346227in}}%
\pgfpathlineto{\pgfqpoint{4.472690in}{1.352871in}}%
\pgfpathlineto{\pgfqpoint{4.486931in}{1.359668in}}%
\pgfpathlineto{\pgfqpoint{4.494967in}{1.378291in}}%
\pgfpathlineto{\pgfqpoint{4.503000in}{1.397030in}}%
\pgfpathlineto{\pgfqpoint{4.511031in}{1.415879in}}%
\pgfpathlineto{\pgfqpoint{4.519061in}{1.434830in}}%
\pgfpathlineto{\pgfqpoint{4.504810in}{1.427330in}}%
\pgfpathlineto{\pgfqpoint{4.490574in}{1.419983in}}%
\pgfpathlineto{\pgfqpoint{4.476352in}{1.412788in}}%
\pgfpathlineto{\pgfqpoint{4.462145in}{1.405747in}}%
\pgfpathlineto{\pgfqpoint{4.454126in}{1.387488in}}%
\pgfpathlineto{\pgfqpoint{4.446104in}{1.369339in}}%
\pgfpathlineto{\pgfqpoint{4.438080in}{1.351305in}}%
\pgfpathlineto{\pgfqpoint{4.430053in}{1.333394in}}%
\pgfpathclose%
\pgfusepath{fill}%
\end{pgfscope}%
\begin{pgfscope}%
\pgfpathrectangle{\pgfqpoint{1.254980in}{0.150000in}}{\pgfqpoint{5.490039in}{5.490039in}}%
\pgfusepath{clip}%
\pgfsetbuttcap%
\pgfsetroundjoin%
\definecolor{currentfill}{rgb}{0.232815,0.732247,0.459277}%
\pgfsetfillcolor{currentfill}%
\pgfsetfillopacity{0.700000}%
\pgfsetlinewidth{0.000000pt}%
\definecolor{currentstroke}{rgb}{0.000000,0.000000,0.000000}%
\pgfsetstrokecolor{currentstroke}%
\pgfsetdash{}{0pt}%
\pgfpathmoveto{\pgfqpoint{5.075760in}{2.532340in}}%
\pgfpathlineto{\pgfqpoint{5.090431in}{2.548258in}}%
\pgfpathlineto{\pgfqpoint{5.105123in}{2.564344in}}%
\pgfpathlineto{\pgfqpoint{5.119837in}{2.580598in}}%
\pgfpathlineto{\pgfqpoint{5.127759in}{2.599311in}}%
\pgfpathlineto{\pgfqpoint{5.135675in}{2.617860in}}%
\pgfpathlineto{\pgfqpoint{5.143584in}{2.636241in}}%
\pgfpathlineto{\pgfqpoint{5.151487in}{2.654450in}}%
\pgfpathlineto{\pgfqpoint{5.136758in}{2.637855in}}%
\pgfpathlineto{\pgfqpoint{5.122050in}{2.621429in}}%
\pgfpathlineto{\pgfqpoint{5.107365in}{2.605170in}}%
\pgfpathlineto{\pgfqpoint{5.099473in}{2.587207in}}%
\pgfpathlineto{\pgfqpoint{5.091575in}{2.569079in}}%
\pgfpathlineto{\pgfqpoint{5.083671in}{2.550789in}}%
\pgfpathlineto{\pgfqpoint{5.075760in}{2.532340in}}%
\pgfpathclose%
\pgfusepath{fill}%
\end{pgfscope}%
\begin{pgfscope}%
\pgfpathrectangle{\pgfqpoint{1.254980in}{0.150000in}}{\pgfqpoint{5.490039in}{5.490039in}}%
\pgfusepath{clip}%
\pgfsetbuttcap%
\pgfsetroundjoin%
\definecolor{currentfill}{rgb}{0.150476,0.504369,0.557430}%
\pgfsetfillcolor{currentfill}%
\pgfsetfillopacity{0.700000}%
\pgfsetlinewidth{0.000000pt}%
\definecolor{currentstroke}{rgb}{0.000000,0.000000,0.000000}%
\pgfsetstrokecolor{currentstroke}%
\pgfsetdash{}{0pt}%
\pgfpathmoveto{\pgfqpoint{4.736620in}{1.869995in}}%
\pgfpathlineto{\pgfqpoint{4.751020in}{1.881399in}}%
\pgfpathlineto{\pgfqpoint{4.765438in}{1.892961in}}%
\pgfpathlineto{\pgfqpoint{4.779874in}{1.904684in}}%
\pgfpathlineto{\pgfqpoint{4.794328in}{1.916565in}}%
\pgfpathlineto{\pgfqpoint{4.802340in}{1.937391in}}%
\pgfpathlineto{\pgfqpoint{4.810349in}{1.958179in}}%
\pgfpathlineto{\pgfqpoint{4.818355in}{1.978924in}}%
\pgfpathlineto{\pgfqpoint{4.826359in}{1.999621in}}%
\pgfpathlineto{\pgfqpoint{4.811888in}{1.987193in}}%
\pgfpathlineto{\pgfqpoint{4.797436in}{1.974925in}}%
\pgfpathlineto{\pgfqpoint{4.783002in}{1.962818in}}%
\pgfpathlineto{\pgfqpoint{4.768586in}{1.950871in}}%
\pgfpathlineto{\pgfqpoint{4.760599in}{1.930708in}}%
\pgfpathlineto{\pgfqpoint{4.752609in}{1.910504in}}%
\pgfpathlineto{\pgfqpoint{4.744616in}{1.890265in}}%
\pgfpathlineto{\pgfqpoint{4.736620in}{1.869995in}}%
\pgfpathclose%
\pgfusepath{fill}%
\end{pgfscope}%
\begin{pgfscope}%
\pgfpathrectangle{\pgfqpoint{1.254980in}{0.150000in}}{\pgfqpoint{5.490039in}{5.490039in}}%
\pgfusepath{clip}%
\pgfsetbuttcap%
\pgfsetroundjoin%
\definecolor{currentfill}{rgb}{0.206756,0.371758,0.553117}%
\pgfsetfillcolor{currentfill}%
\pgfsetfillopacity{0.700000}%
\pgfsetlinewidth{0.000000pt}%
\definecolor{currentstroke}{rgb}{0.000000,0.000000,0.000000}%
\pgfsetstrokecolor{currentstroke}%
\pgfsetdash{}{0pt}%
\pgfpathmoveto{\pgfqpoint{4.551158in}{1.511535in}}%
\pgfpathlineto{\pgfqpoint{4.565435in}{1.519867in}}%
\pgfpathlineto{\pgfqpoint{4.579728in}{1.528353in}}%
\pgfpathlineto{\pgfqpoint{4.594036in}{1.536993in}}%
\pgfpathlineto{\pgfqpoint{4.608361in}{1.545788in}}%
\pgfpathlineto{\pgfqpoint{4.616393in}{1.565820in}}%
\pgfpathlineto{\pgfqpoint{4.624423in}{1.585911in}}%
\pgfpathlineto{\pgfqpoint{4.632451in}{1.606053in}}%
\pgfpathlineto{\pgfqpoint{4.640477in}{1.626241in}}%
\pgfpathlineto{\pgfqpoint{4.626139in}{1.616792in}}%
\pgfpathlineto{\pgfqpoint{4.611817in}{1.607499in}}%
\pgfpathlineto{\pgfqpoint{4.597512in}{1.598361in}}%
\pgfpathlineto{\pgfqpoint{4.583222in}{1.589378in}}%
\pgfpathlineto{\pgfqpoint{4.575209in}{1.569832in}}%
\pgfpathlineto{\pgfqpoint{4.567194in}{1.550339in}}%
\pgfpathlineto{\pgfqpoint{4.559177in}{1.530905in}}%
\pgfpathlineto{\pgfqpoint{4.551158in}{1.511535in}}%
\pgfpathclose%
\pgfusepath{fill}%
\end{pgfscope}%
\begin{pgfscope}%
\pgfpathrectangle{\pgfqpoint{1.254980in}{0.150000in}}{\pgfqpoint{5.490039in}{5.490039in}}%
\pgfusepath{clip}%
\pgfsetbuttcap%
\pgfsetroundjoin%
\definecolor{currentfill}{rgb}{0.119423,0.611141,0.538982}%
\pgfsetfillcolor{currentfill}%
\pgfsetfillopacity{0.700000}%
\pgfsetlinewidth{0.000000pt}%
\definecolor{currentstroke}{rgb}{0.000000,0.000000,0.000000}%
\pgfsetstrokecolor{currentstroke}%
\pgfsetdash{}{0pt}%
\pgfpathmoveto{\pgfqpoint{4.890268in}{2.162895in}}%
\pgfpathlineto{\pgfqpoint{4.904792in}{2.176491in}}%
\pgfpathlineto{\pgfqpoint{4.919335in}{2.190250in}}%
\pgfpathlineto{\pgfqpoint{4.933898in}{2.204172in}}%
\pgfpathlineto{\pgfqpoint{4.948481in}{2.218258in}}%
\pgfpathlineto{\pgfqpoint{4.956470in}{2.238783in}}%
\pgfpathlineto{\pgfqpoint{4.964456in}{2.259208in}}%
\pgfpathlineto{\pgfqpoint{4.972437in}{2.279528in}}%
\pgfpathlineto{\pgfqpoint{4.980414in}{2.299738in}}%
\pgfpathlineto{\pgfqpoint{4.965814in}{2.285190in}}%
\pgfpathlineto{\pgfqpoint{4.951234in}{2.270806in}}%
\pgfpathlineto{\pgfqpoint{4.936673in}{2.256587in}}%
\pgfpathlineto{\pgfqpoint{4.922133in}{2.242531in}}%
\pgfpathlineto{\pgfqpoint{4.914173in}{2.222770in}}%
\pgfpathlineto{\pgfqpoint{4.906209in}{2.202908in}}%
\pgfpathlineto{\pgfqpoint{4.898240in}{2.182948in}}%
\pgfpathlineto{\pgfqpoint{4.890268in}{2.162895in}}%
\pgfpathclose%
\pgfusepath{fill}%
\end{pgfscope}%
\begin{pgfscope}%
\pgfpathrectangle{\pgfqpoint{1.254980in}{0.150000in}}{\pgfqpoint{5.490039in}{5.490039in}}%
\pgfusepath{clip}%
\pgfsetbuttcap%
\pgfsetroundjoin%
\definecolor{currentfill}{rgb}{0.160665,0.478540,0.558115}%
\pgfsetfillcolor{currentfill}%
\pgfsetfillopacity{0.700000}%
\pgfsetlinewidth{0.000000pt}%
\definecolor{currentstroke}{rgb}{0.000000,0.000000,0.000000}%
\pgfsetstrokecolor{currentstroke}%
\pgfsetdash{}{0pt}%
\pgfpathmoveto{\pgfqpoint{4.704610in}{1.788705in}}%
\pgfpathlineto{\pgfqpoint{4.718995in}{1.799537in}}%
\pgfpathlineto{\pgfqpoint{4.733397in}{1.810527in}}%
\pgfpathlineto{\pgfqpoint{4.747816in}{1.821676in}}%
\pgfpathlineto{\pgfqpoint{4.762254in}{1.832983in}}%
\pgfpathlineto{\pgfqpoint{4.770276in}{1.853910in}}%
\pgfpathlineto{\pgfqpoint{4.778296in}{1.874819in}}%
\pgfpathlineto{\pgfqpoint{4.786313in}{1.895706in}}%
\pgfpathlineto{\pgfqpoint{4.794328in}{1.916565in}}%
\pgfpathlineto{\pgfqpoint{4.779874in}{1.904684in}}%
\pgfpathlineto{\pgfqpoint{4.765438in}{1.892961in}}%
\pgfpathlineto{\pgfqpoint{4.751020in}{1.881399in}}%
\pgfpathlineto{\pgfqpoint{4.736620in}{1.869995in}}%
\pgfpathlineto{\pgfqpoint{4.728621in}{1.849698in}}%
\pgfpathlineto{\pgfqpoint{4.720620in}{1.829381in}}%
\pgfpathlineto{\pgfqpoint{4.712617in}{1.809048in}}%
\pgfpathlineto{\pgfqpoint{4.704610in}{1.788705in}}%
\pgfpathclose%
\pgfusepath{fill}%
\end{pgfscope}%
\begin{pgfscope}%
\pgfpathrectangle{\pgfqpoint{1.254980in}{0.150000in}}{\pgfqpoint{5.490039in}{5.490039in}}%
\pgfusepath{clip}%
\pgfsetbuttcap%
\pgfsetroundjoin%
\definecolor{currentfill}{rgb}{0.196571,0.711827,0.479221}%
\pgfsetfillcolor{currentfill}%
\pgfsetfillopacity{0.700000}%
\pgfsetlinewidth{0.000000pt}%
\definecolor{currentstroke}{rgb}{0.000000,0.000000,0.000000}%
\pgfsetstrokecolor{currentstroke}%
\pgfsetdash{}{0pt}%
\pgfpathmoveto{\pgfqpoint{5.044062in}{2.457015in}}%
\pgfpathlineto{\pgfqpoint{5.058717in}{2.472563in}}%
\pgfpathlineto{\pgfqpoint{5.073393in}{2.488277in}}%
\pgfpathlineto{\pgfqpoint{5.088091in}{2.504158in}}%
\pgfpathlineto{\pgfqpoint{5.096036in}{2.523500in}}%
\pgfpathlineto{\pgfqpoint{5.103976in}{2.542689in}}%
\pgfpathlineto{\pgfqpoint{5.111910in}{2.561723in}}%
\pgfpathlineto{\pgfqpoint{5.119837in}{2.580598in}}%
\pgfpathlineto{\pgfqpoint{5.105123in}{2.564344in}}%
\pgfpathlineto{\pgfqpoint{5.090431in}{2.548258in}}%
\pgfpathlineto{\pgfqpoint{5.075760in}{2.532340in}}%
\pgfpathlineto{\pgfqpoint{5.067844in}{2.513734in}}%
\pgfpathlineto{\pgfqpoint{5.059923in}{2.494977in}}%
\pgfpathlineto{\pgfqpoint{5.051995in}{2.476069in}}%
\pgfpathlineto{\pgfqpoint{5.044062in}{2.457015in}}%
\pgfpathclose%
\pgfusepath{fill}%
\end{pgfscope}%
\begin{pgfscope}%
\pgfpathrectangle{\pgfqpoint{1.254980in}{0.150000in}}{\pgfqpoint{5.490039in}{5.490039in}}%
\pgfusepath{clip}%
\pgfsetbuttcap%
\pgfsetroundjoin%
\definecolor{currentfill}{rgb}{0.252194,0.269783,0.531579}%
\pgfsetfillcolor{currentfill}%
\pgfsetfillopacity{0.700000}%
\pgfsetlinewidth{0.000000pt}%
\definecolor{currentstroke}{rgb}{0.000000,0.000000,0.000000}%
\pgfsetstrokecolor{currentstroke}%
\pgfsetdash{}{0pt}%
\pgfpathmoveto{\pgfqpoint{4.397924in}{1.263113in}}%
\pgfpathlineto{\pgfqpoint{4.412115in}{1.268725in}}%
\pgfpathlineto{\pgfqpoint{4.426319in}{1.274489in}}%
\pgfpathlineto{\pgfqpoint{4.440538in}{1.280403in}}%
\pgfpathlineto{\pgfqpoint{4.454770in}{1.286469in}}%
\pgfpathlineto{\pgfqpoint{4.462813in}{1.304561in}}%
\pgfpathlineto{\pgfqpoint{4.470855in}{1.322796in}}%
\pgfpathlineto{\pgfqpoint{4.478894in}{1.341167in}}%
\pgfpathlineto{\pgfqpoint{4.486931in}{1.359668in}}%
\pgfpathlineto{\pgfqpoint{4.472690in}{1.352871in}}%
\pgfpathlineto{\pgfqpoint{4.458464in}{1.346227in}}%
\pgfpathlineto{\pgfqpoint{4.444252in}{1.339735in}}%
\pgfpathlineto{\pgfqpoint{4.430053in}{1.333394in}}%
\pgfpathlineto{\pgfqpoint{4.422024in}{1.315613in}}%
\pgfpathlineto{\pgfqpoint{4.413993in}{1.297968in}}%
\pgfpathlineto{\pgfqpoint{4.405960in}{1.280465in}}%
\pgfpathlineto{\pgfqpoint{4.397924in}{1.263113in}}%
\pgfpathclose%
\pgfusepath{fill}%
\end{pgfscope}%
\begin{pgfscope}%
\pgfpathrectangle{\pgfqpoint{1.254980in}{0.150000in}}{\pgfqpoint{5.490039in}{5.490039in}}%
\pgfusepath{clip}%
\pgfsetbuttcap%
\pgfsetroundjoin%
\definecolor{currentfill}{rgb}{0.218130,0.347432,0.550038}%
\pgfsetfillcolor{currentfill}%
\pgfsetfillopacity{0.700000}%
\pgfsetlinewidth{0.000000pt}%
\definecolor{currentstroke}{rgb}{0.000000,0.000000,0.000000}%
\pgfsetstrokecolor{currentstroke}%
\pgfsetdash{}{0pt}%
\pgfpathmoveto{\pgfqpoint{4.519061in}{1.434830in}}%
\pgfpathlineto{\pgfqpoint{4.533326in}{1.442484in}}%
\pgfpathlineto{\pgfqpoint{4.547607in}{1.450291in}}%
\pgfpathlineto{\pgfqpoint{4.561903in}{1.458252in}}%
\pgfpathlineto{\pgfqpoint{4.576215in}{1.466366in}}%
\pgfpathlineto{\pgfqpoint{4.584254in}{1.486103in}}%
\pgfpathlineto{\pgfqpoint{4.592291in}{1.505923in}}%
\pgfpathlineto{\pgfqpoint{4.600327in}{1.525821in}}%
\pgfpathlineto{\pgfqpoint{4.608361in}{1.545788in}}%
\pgfpathlineto{\pgfqpoint{4.594036in}{1.536993in}}%
\pgfpathlineto{\pgfqpoint{4.579728in}{1.528353in}}%
\pgfpathlineto{\pgfqpoint{4.565435in}{1.519867in}}%
\pgfpathlineto{\pgfqpoint{4.551158in}{1.511535in}}%
\pgfpathlineto{\pgfqpoint{4.543137in}{1.492237in}}%
\pgfpathlineto{\pgfqpoint{4.535113in}{1.473016in}}%
\pgfpathlineto{\pgfqpoint{4.527088in}{1.453878in}}%
\pgfpathlineto{\pgfqpoint{4.519061in}{1.434830in}}%
\pgfpathclose%
\pgfusepath{fill}%
\end{pgfscope}%
\begin{pgfscope}%
\pgfpathrectangle{\pgfqpoint{1.254980in}{0.150000in}}{\pgfqpoint{5.490039in}{5.490039in}}%
\pgfusepath{clip}%
\pgfsetbuttcap%
\pgfsetroundjoin%
\definecolor{currentfill}{rgb}{0.122606,0.585371,0.546557}%
\pgfsetfillcolor{currentfill}%
\pgfsetfillopacity{0.700000}%
\pgfsetlinewidth{0.000000pt}%
\definecolor{currentstroke}{rgb}{0.000000,0.000000,0.000000}%
\pgfsetstrokecolor{currentstroke}%
\pgfsetdash{}{0pt}%
\pgfpathmoveto{\pgfqpoint{4.858341in}{2.081830in}}%
\pgfpathlineto{\pgfqpoint{4.872848in}{2.094938in}}%
\pgfpathlineto{\pgfqpoint{4.887374in}{2.108207in}}%
\pgfpathlineto{\pgfqpoint{4.901919in}{2.121638in}}%
\pgfpathlineto{\pgfqpoint{4.916484in}{2.135231in}}%
\pgfpathlineto{\pgfqpoint{4.924489in}{2.156118in}}%
\pgfpathlineto{\pgfqpoint{4.932490in}{2.176921in}}%
\pgfpathlineto{\pgfqpoint{4.940487in}{2.197635in}}%
\pgfpathlineto{\pgfqpoint{4.948481in}{2.218258in}}%
\pgfpathlineto{\pgfqpoint{4.933898in}{2.204172in}}%
\pgfpathlineto{\pgfqpoint{4.919335in}{2.190250in}}%
\pgfpathlineto{\pgfqpoint{4.904792in}{2.176491in}}%
\pgfpathlineto{\pgfqpoint{4.890268in}{2.162895in}}%
\pgfpathlineto{\pgfqpoint{4.882292in}{2.142752in}}%
\pgfpathlineto{\pgfqpoint{4.874312in}{2.122524in}}%
\pgfpathlineto{\pgfqpoint{4.866328in}{2.102215in}}%
\pgfpathlineto{\pgfqpoint{4.858341in}{2.081830in}}%
\pgfpathclose%
\pgfusepath{fill}%
\end{pgfscope}%
\begin{pgfscope}%
\pgfpathrectangle{\pgfqpoint{1.254980in}{0.150000in}}{\pgfqpoint{5.490039in}{5.490039in}}%
\pgfusepath{clip}%
\pgfsetbuttcap%
\pgfsetroundjoin%
\definecolor{currentfill}{rgb}{0.171176,0.452530,0.557965}%
\pgfsetfillcolor{currentfill}%
\pgfsetfillopacity{0.700000}%
\pgfsetlinewidth{0.000000pt}%
\definecolor{currentstroke}{rgb}{0.000000,0.000000,0.000000}%
\pgfsetstrokecolor{currentstroke}%
\pgfsetdash{}{0pt}%
\pgfpathmoveto{\pgfqpoint{4.672561in}{1.707337in}}%
\pgfpathlineto{\pgfqpoint{4.686931in}{1.717569in}}%
\pgfpathlineto{\pgfqpoint{4.701317in}{1.727958in}}%
\pgfpathlineto{\pgfqpoint{4.715720in}{1.738505in}}%
\pgfpathlineto{\pgfqpoint{4.730141in}{1.749209in}}%
\pgfpathlineto{\pgfqpoint{4.738173in}{1.770151in}}%
\pgfpathlineto{\pgfqpoint{4.746202in}{1.791098in}}%
\pgfpathlineto{\pgfqpoint{4.754229in}{1.812044in}}%
\pgfpathlineto{\pgfqpoint{4.762254in}{1.832983in}}%
\pgfpathlineto{\pgfqpoint{4.747816in}{1.821676in}}%
\pgfpathlineto{\pgfqpoint{4.733397in}{1.810527in}}%
\pgfpathlineto{\pgfqpoint{4.718995in}{1.799537in}}%
\pgfpathlineto{\pgfqpoint{4.704610in}{1.788705in}}%
\pgfpathlineto{\pgfqpoint{4.696602in}{1.768357in}}%
\pgfpathlineto{\pgfqpoint{4.688590in}{1.748009in}}%
\pgfpathlineto{\pgfqpoint{4.680577in}{1.727667in}}%
\pgfpathlineto{\pgfqpoint{4.672561in}{1.707337in}}%
\pgfpathclose%
\pgfusepath{fill}%
\end{pgfscope}%
\begin{pgfscope}%
\pgfpathrectangle{\pgfqpoint{1.254980in}{0.150000in}}{\pgfqpoint{5.490039in}{5.490039in}}%
\pgfusepath{clip}%
\pgfsetbuttcap%
\pgfsetroundjoin%
\definecolor{currentfill}{rgb}{0.166383,0.690856,0.496502}%
\pgfsetfillcolor{currentfill}%
\pgfsetfillopacity{0.700000}%
\pgfsetlinewidth{0.000000pt}%
\definecolor{currentstroke}{rgb}{0.000000,0.000000,0.000000}%
\pgfsetstrokecolor{currentstroke}%
\pgfsetdash{}{0pt}%
\pgfpathmoveto{\pgfqpoint{5.012277in}{2.379406in}}%
\pgfpathlineto{\pgfqpoint{5.026916in}{2.394551in}}%
\pgfpathlineto{\pgfqpoint{5.041575in}{2.409862in}}%
\pgfpathlineto{\pgfqpoint{5.056255in}{2.425339in}}%
\pgfpathlineto{\pgfqpoint{5.064222in}{2.445255in}}%
\pgfpathlineto{\pgfqpoint{5.072183in}{2.465032in}}%
\pgfpathlineto{\pgfqpoint{5.080140in}{2.484668in}}%
\pgfpathlineto{\pgfqpoint{5.088091in}{2.504158in}}%
\pgfpathlineto{\pgfqpoint{5.073393in}{2.488277in}}%
\pgfpathlineto{\pgfqpoint{5.058717in}{2.472563in}}%
\pgfpathlineto{\pgfqpoint{5.044062in}{2.457015in}}%
\pgfpathlineto{\pgfqpoint{5.036124in}{2.437819in}}%
\pgfpathlineto{\pgfqpoint{5.028180in}{2.418482in}}%
\pgfpathlineto{\pgfqpoint{5.020231in}{2.399010in}}%
\pgfpathlineto{\pgfqpoint{5.012277in}{2.379406in}}%
\pgfpathclose%
\pgfusepath{fill}%
\end{pgfscope}%
\begin{pgfscope}%
\pgfpathrectangle{\pgfqpoint{1.254980in}{0.150000in}}{\pgfqpoint{5.490039in}{5.490039in}}%
\pgfusepath{clip}%
\pgfsetbuttcap%
\pgfsetroundjoin%
\definecolor{currentfill}{rgb}{0.129933,0.559582,0.551864}%
\pgfsetfillcolor{currentfill}%
\pgfsetfillopacity{0.700000}%
\pgfsetlinewidth{0.000000pt}%
\definecolor{currentstroke}{rgb}{0.000000,0.000000,0.000000}%
\pgfsetstrokecolor{currentstroke}%
\pgfsetdash{}{0pt}%
\pgfpathmoveto{\pgfqpoint{4.826359in}{1.999621in}}%
\pgfpathlineto{\pgfqpoint{4.840848in}{2.012209in}}%
\pgfpathlineto{\pgfqpoint{4.855357in}{2.024959in}}%
\pgfpathlineto{\pgfqpoint{4.869884in}{2.037870in}}%
\pgfpathlineto{\pgfqpoint{4.884431in}{2.050942in}}%
\pgfpathlineto{\pgfqpoint{4.892449in}{2.072116in}}%
\pgfpathlineto{\pgfqpoint{4.900464in}{2.093226in}}%
\pgfpathlineto{\pgfqpoint{4.908476in}{2.114266in}}%
\pgfpathlineto{\pgfqpoint{4.916484in}{2.135231in}}%
\pgfpathlineto{\pgfqpoint{4.901919in}{2.121638in}}%
\pgfpathlineto{\pgfqpoint{4.887374in}{2.108207in}}%
\pgfpathlineto{\pgfqpoint{4.872848in}{2.094938in}}%
\pgfpathlineto{\pgfqpoint{4.858341in}{2.081830in}}%
\pgfpathlineto{\pgfqpoint{4.850350in}{2.061374in}}%
\pgfpathlineto{\pgfqpoint{4.842356in}{2.040850in}}%
\pgfpathlineto{\pgfqpoint{4.834359in}{2.020264in}}%
\pgfpathlineto{\pgfqpoint{4.826359in}{1.999621in}}%
\pgfpathclose%
\pgfusepath{fill}%
\end{pgfscope}%
\begin{pgfscope}%
\pgfpathrectangle{\pgfqpoint{1.254980in}{0.150000in}}{\pgfqpoint{5.490039in}{5.490039in}}%
\pgfusepath{clip}%
\pgfsetbuttcap%
\pgfsetroundjoin%
\definecolor{currentfill}{rgb}{0.229739,0.322361,0.545706}%
\pgfsetfillcolor{currentfill}%
\pgfsetfillopacity{0.700000}%
\pgfsetlinewidth{0.000000pt}%
\definecolor{currentstroke}{rgb}{0.000000,0.000000,0.000000}%
\pgfsetstrokecolor{currentstroke}%
\pgfsetdash{}{0pt}%
\pgfpathmoveto{\pgfqpoint{4.486931in}{1.359668in}}%
\pgfpathlineto{\pgfqpoint{4.501187in}{1.366617in}}%
\pgfpathlineto{\pgfqpoint{4.515456in}{1.373718in}}%
\pgfpathlineto{\pgfqpoint{4.529741in}{1.380972in}}%
\pgfpathlineto{\pgfqpoint{4.544040in}{1.388378in}}%
\pgfpathlineto{\pgfqpoint{4.552086in}{1.407718in}}%
\pgfpathlineto{\pgfqpoint{4.560131in}{1.427167in}}%
\pgfpathlineto{\pgfqpoint{4.568174in}{1.446718in}}%
\pgfpathlineto{\pgfqpoint{4.576215in}{1.466366in}}%
\pgfpathlineto{\pgfqpoint{4.561903in}{1.458252in}}%
\pgfpathlineto{\pgfqpoint{4.547607in}{1.450291in}}%
\pgfpathlineto{\pgfqpoint{4.533326in}{1.442484in}}%
\pgfpathlineto{\pgfqpoint{4.519061in}{1.434830in}}%
\pgfpathlineto{\pgfqpoint{4.511031in}{1.415879in}}%
\pgfpathlineto{\pgfqpoint{4.503000in}{1.397030in}}%
\pgfpathlineto{\pgfqpoint{4.494967in}{1.378291in}}%
\pgfpathlineto{\pgfqpoint{4.486931in}{1.359668in}}%
\pgfpathclose%
\pgfusepath{fill}%
\end{pgfscope}%
\begin{pgfscope}%
\pgfpathrectangle{\pgfqpoint{1.254980in}{0.150000in}}{\pgfqpoint{5.490039in}{5.490039in}}%
\pgfusepath{clip}%
\pgfsetbuttcap%
\pgfsetroundjoin%
\definecolor{currentfill}{rgb}{0.182256,0.426184,0.557120}%
\pgfsetfillcolor{currentfill}%
\pgfsetfillopacity{0.700000}%
\pgfsetlinewidth{0.000000pt}%
\definecolor{currentstroke}{rgb}{0.000000,0.000000,0.000000}%
\pgfsetstrokecolor{currentstroke}%
\pgfsetdash{}{0pt}%
\pgfpathmoveto{\pgfqpoint{4.640477in}{1.626241in}}%
\pgfpathlineto{\pgfqpoint{4.654832in}{1.635846in}}%
\pgfpathlineto{\pgfqpoint{4.669203in}{1.645607in}}%
\pgfpathlineto{\pgfqpoint{4.683590in}{1.655524in}}%
\pgfpathlineto{\pgfqpoint{4.697995in}{1.665597in}}%
\pgfpathlineto{\pgfqpoint{4.706034in}{1.686465in}}%
\pgfpathlineto{\pgfqpoint{4.714072in}{1.707360in}}%
\pgfpathlineto{\pgfqpoint{4.722108in}{1.728276in}}%
\pgfpathlineto{\pgfqpoint{4.730141in}{1.749209in}}%
\pgfpathlineto{\pgfqpoint{4.715720in}{1.738505in}}%
\pgfpathlineto{\pgfqpoint{4.701317in}{1.727958in}}%
\pgfpathlineto{\pgfqpoint{4.686931in}{1.717569in}}%
\pgfpathlineto{\pgfqpoint{4.672561in}{1.707337in}}%
\pgfpathlineto{\pgfqpoint{4.664544in}{1.687023in}}%
\pgfpathlineto{\pgfqpoint{4.656524in}{1.666732in}}%
\pgfpathlineto{\pgfqpoint{4.648501in}{1.646469in}}%
\pgfpathlineto{\pgfqpoint{4.640477in}{1.626241in}}%
\pgfpathclose%
\pgfusepath{fill}%
\end{pgfscope}%
\begin{pgfscope}%
\pgfpathrectangle{\pgfqpoint{1.254980in}{0.150000in}}{\pgfqpoint{5.490039in}{5.490039in}}%
\pgfusepath{clip}%
\pgfsetbuttcap%
\pgfsetroundjoin%
\definecolor{currentfill}{rgb}{0.140210,0.665859,0.513427}%
\pgfsetfillcolor{currentfill}%
\pgfsetfillopacity{0.700000}%
\pgfsetlinewidth{0.000000pt}%
\definecolor{currentstroke}{rgb}{0.000000,0.000000,0.000000}%
\pgfsetstrokecolor{currentstroke}%
\pgfsetdash{}{0pt}%
\pgfpathmoveto{\pgfqpoint{4.980414in}{2.299738in}}%
\pgfpathlineto{\pgfqpoint{4.995035in}{2.314450in}}%
\pgfpathlineto{\pgfqpoint{5.009677in}{2.329327in}}%
\pgfpathlineto{\pgfqpoint{5.024339in}{2.344370in}}%
\pgfpathlineto{\pgfqpoint{5.032325in}{2.364801in}}%
\pgfpathlineto{\pgfqpoint{5.040307in}{2.385108in}}%
\pgfpathlineto{\pgfqpoint{5.048283in}{2.405289in}}%
\pgfpathlineto{\pgfqpoint{5.056255in}{2.425339in}}%
\pgfpathlineto{\pgfqpoint{5.041575in}{2.409862in}}%
\pgfpathlineto{\pgfqpoint{5.026916in}{2.394551in}}%
\pgfpathlineto{\pgfqpoint{5.012277in}{2.379406in}}%
\pgfpathlineto{\pgfqpoint{5.004319in}{2.359672in}}%
\pgfpathlineto{\pgfqpoint{4.996355in}{2.339814in}}%
\pgfpathlineto{\pgfqpoint{4.988387in}{2.319834in}}%
\pgfpathlineto{\pgfqpoint{4.980414in}{2.299738in}}%
\pgfpathclose%
\pgfusepath{fill}%
\end{pgfscope}%
\begin{pgfscope}%
\pgfpathrectangle{\pgfqpoint{1.254980in}{0.150000in}}{\pgfqpoint{5.490039in}{5.490039in}}%
\pgfusepath{clip}%
\pgfsetbuttcap%
\pgfsetroundjoin%
\definecolor{currentfill}{rgb}{0.139147,0.533812,0.555298}%
\pgfsetfillcolor{currentfill}%
\pgfsetfillopacity{0.700000}%
\pgfsetlinewidth{0.000000pt}%
\definecolor{currentstroke}{rgb}{0.000000,0.000000,0.000000}%
\pgfsetstrokecolor{currentstroke}%
\pgfsetdash{}{0pt}%
\pgfpathmoveto{\pgfqpoint{4.794328in}{1.916565in}}%
\pgfpathlineto{\pgfqpoint{4.808800in}{1.928607in}}%
\pgfpathlineto{\pgfqpoint{4.823291in}{1.940808in}}%
\pgfpathlineto{\pgfqpoint{4.837800in}{1.953169in}}%
\pgfpathlineto{\pgfqpoint{4.852328in}{1.965691in}}%
\pgfpathlineto{\pgfqpoint{4.860358in}{1.987076in}}%
\pgfpathlineto{\pgfqpoint{4.868385in}{2.008417in}}%
\pgfpathlineto{\pgfqpoint{4.876410in}{2.029707in}}%
\pgfpathlineto{\pgfqpoint{4.884431in}{2.050942in}}%
\pgfpathlineto{\pgfqpoint{4.869884in}{2.037870in}}%
\pgfpathlineto{\pgfqpoint{4.855357in}{2.024959in}}%
\pgfpathlineto{\pgfqpoint{4.840848in}{2.012209in}}%
\pgfpathlineto{\pgfqpoint{4.826359in}{1.999621in}}%
\pgfpathlineto{\pgfqpoint{4.818355in}{1.978924in}}%
\pgfpathlineto{\pgfqpoint{4.810349in}{1.958179in}}%
\pgfpathlineto{\pgfqpoint{4.802340in}{1.937391in}}%
\pgfpathlineto{\pgfqpoint{4.794328in}{1.916565in}}%
\pgfpathclose%
\pgfusepath{fill}%
\end{pgfscope}%
\begin{pgfscope}%
\pgfpathrectangle{\pgfqpoint{1.254980in}{0.150000in}}{\pgfqpoint{5.490039in}{5.490039in}}%
\pgfusepath{clip}%
\pgfsetbuttcap%
\pgfsetroundjoin%
\definecolor{currentfill}{rgb}{0.194100,0.399323,0.555565}%
\pgfsetfillcolor{currentfill}%
\pgfsetfillopacity{0.700000}%
\pgfsetlinewidth{0.000000pt}%
\definecolor{currentstroke}{rgb}{0.000000,0.000000,0.000000}%
\pgfsetstrokecolor{currentstroke}%
\pgfsetdash{}{0pt}%
\pgfpathmoveto{\pgfqpoint{4.608361in}{1.545788in}}%
\pgfpathlineto{\pgfqpoint{4.622701in}{1.554738in}}%
\pgfpathlineto{\pgfqpoint{4.637057in}{1.563843in}}%
\pgfpathlineto{\pgfqpoint{4.651430in}{1.573103in}}%
\pgfpathlineto{\pgfqpoint{4.665819in}{1.582518in}}%
\pgfpathlineto{\pgfqpoint{4.673866in}{1.603217in}}%
\pgfpathlineto{\pgfqpoint{4.681911in}{1.623967in}}%
\pgfpathlineto{\pgfqpoint{4.689954in}{1.644762in}}%
\pgfpathlineto{\pgfqpoint{4.697995in}{1.665597in}}%
\pgfpathlineto{\pgfqpoint{4.683590in}{1.655524in}}%
\pgfpathlineto{\pgfqpoint{4.669203in}{1.645607in}}%
\pgfpathlineto{\pgfqpoint{4.654832in}{1.635846in}}%
\pgfpathlineto{\pgfqpoint{4.640477in}{1.626241in}}%
\pgfpathlineto{\pgfqpoint{4.632451in}{1.606053in}}%
\pgfpathlineto{\pgfqpoint{4.624423in}{1.585911in}}%
\pgfpathlineto{\pgfqpoint{4.616393in}{1.565820in}}%
\pgfpathlineto{\pgfqpoint{4.608361in}{1.545788in}}%
\pgfpathclose%
\pgfusepath{fill}%
\end{pgfscope}%
\begin{pgfscope}%
\pgfpathrectangle{\pgfqpoint{1.254980in}{0.150000in}}{\pgfqpoint{5.490039in}{5.490039in}}%
\pgfusepath{clip}%
\pgfsetbuttcap%
\pgfsetroundjoin%
\definecolor{currentfill}{rgb}{0.241237,0.296485,0.539709}%
\pgfsetfillcolor{currentfill}%
\pgfsetfillopacity{0.700000}%
\pgfsetlinewidth{0.000000pt}%
\definecolor{currentstroke}{rgb}{0.000000,0.000000,0.000000}%
\pgfsetstrokecolor{currentstroke}%
\pgfsetdash{}{0pt}%
\pgfpathmoveto{\pgfqpoint{4.454770in}{1.286469in}}%
\pgfpathlineto{\pgfqpoint{4.469016in}{1.292687in}}%
\pgfpathlineto{\pgfqpoint{4.483276in}{1.299056in}}%
\pgfpathlineto{\pgfqpoint{4.497550in}{1.305576in}}%
\pgfpathlineto{\pgfqpoint{4.511839in}{1.312248in}}%
\pgfpathlineto{\pgfqpoint{4.519892in}{1.331083in}}%
\pgfpathlineto{\pgfqpoint{4.527943in}{1.350054in}}%
\pgfpathlineto{\pgfqpoint{4.535992in}{1.369155in}}%
\pgfpathlineto{\pgfqpoint{4.544040in}{1.388378in}}%
\pgfpathlineto{\pgfqpoint{4.529741in}{1.380972in}}%
\pgfpathlineto{\pgfqpoint{4.515456in}{1.373718in}}%
\pgfpathlineto{\pgfqpoint{4.501187in}{1.366617in}}%
\pgfpathlineto{\pgfqpoint{4.486931in}{1.359668in}}%
\pgfpathlineto{\pgfqpoint{4.478894in}{1.341167in}}%
\pgfpathlineto{\pgfqpoint{4.470855in}{1.322796in}}%
\pgfpathlineto{\pgfqpoint{4.462813in}{1.304561in}}%
\pgfpathlineto{\pgfqpoint{4.454770in}{1.286469in}}%
\pgfpathclose%
\pgfusepath{fill}%
\end{pgfscope}%
\begin{pgfscope}%
\pgfpathrectangle{\pgfqpoint{1.254980in}{0.150000in}}{\pgfqpoint{5.490039in}{5.490039in}}%
\pgfusepath{clip}%
\pgfsetbuttcap%
\pgfsetroundjoin%
\definecolor{currentfill}{rgb}{0.124780,0.640461,0.527068}%
\pgfsetfillcolor{currentfill}%
\pgfsetfillopacity{0.700000}%
\pgfsetlinewidth{0.000000pt}%
\definecolor{currentstroke}{rgb}{0.000000,0.000000,0.000000}%
\pgfsetstrokecolor{currentstroke}%
\pgfsetdash{}{0pt}%
\pgfpathmoveto{\pgfqpoint{4.948481in}{2.218258in}}%
\pgfpathlineto{\pgfqpoint{4.963084in}{2.232507in}}%
\pgfpathlineto{\pgfqpoint{4.977707in}{2.246920in}}%
\pgfpathlineto{\pgfqpoint{4.992350in}{2.261498in}}%
\pgfpathlineto{\pgfqpoint{5.000354in}{2.282380in}}%
\pgfpathlineto{\pgfqpoint{5.008353in}{2.303155in}}%
\pgfpathlineto{\pgfqpoint{5.016348in}{2.323820in}}%
\pgfpathlineto{\pgfqpoint{5.024339in}{2.344370in}}%
\pgfpathlineto{\pgfqpoint{5.009677in}{2.329327in}}%
\pgfpathlineto{\pgfqpoint{4.995035in}{2.314450in}}%
\pgfpathlineto{\pgfqpoint{4.980414in}{2.299738in}}%
\pgfpathlineto{\pgfqpoint{4.972437in}{2.279528in}}%
\pgfpathlineto{\pgfqpoint{4.964456in}{2.259208in}}%
\pgfpathlineto{\pgfqpoint{4.956470in}{2.238783in}}%
\pgfpathlineto{\pgfqpoint{4.948481in}{2.218258in}}%
\pgfpathclose%
\pgfusepath{fill}%
\end{pgfscope}%
\begin{pgfscope}%
\pgfpathrectangle{\pgfqpoint{1.254980in}{0.150000in}}{\pgfqpoint{5.490039in}{5.490039in}}%
\pgfusepath{clip}%
\pgfsetbuttcap%
\pgfsetroundjoin%
\definecolor{currentfill}{rgb}{0.149039,0.508051,0.557250}%
\pgfsetfillcolor{currentfill}%
\pgfsetfillopacity{0.700000}%
\pgfsetlinewidth{0.000000pt}%
\definecolor{currentstroke}{rgb}{0.000000,0.000000,0.000000}%
\pgfsetstrokecolor{currentstroke}%
\pgfsetdash{}{0pt}%
\pgfpathmoveto{\pgfqpoint{4.762254in}{1.832983in}}%
\pgfpathlineto{\pgfqpoint{4.776709in}{1.844448in}}%
\pgfpathlineto{\pgfqpoint{4.791182in}{1.856072in}}%
\pgfpathlineto{\pgfqpoint{4.805673in}{1.867856in}}%
\pgfpathlineto{\pgfqpoint{4.820183in}{1.879798in}}%
\pgfpathlineto{\pgfqpoint{4.828223in}{1.901313in}}%
\pgfpathlineto{\pgfqpoint{4.836261in}{1.922804in}}%
\pgfpathlineto{\pgfqpoint{4.844296in}{1.944265in}}%
\pgfpathlineto{\pgfqpoint{4.852328in}{1.965691in}}%
\pgfpathlineto{\pgfqpoint{4.837800in}{1.953169in}}%
\pgfpathlineto{\pgfqpoint{4.823291in}{1.940808in}}%
\pgfpathlineto{\pgfqpoint{4.808800in}{1.928607in}}%
\pgfpathlineto{\pgfqpoint{4.794328in}{1.916565in}}%
\pgfpathlineto{\pgfqpoint{4.786313in}{1.895706in}}%
\pgfpathlineto{\pgfqpoint{4.778296in}{1.874819in}}%
\pgfpathlineto{\pgfqpoint{4.770276in}{1.853910in}}%
\pgfpathlineto{\pgfqpoint{4.762254in}{1.832983in}}%
\pgfpathclose%
\pgfusepath{fill}%
\end{pgfscope}%
\begin{pgfscope}%
\pgfpathrectangle{\pgfqpoint{1.254980in}{0.150000in}}{\pgfqpoint{5.490039in}{5.490039in}}%
\pgfusepath{clip}%
\pgfsetbuttcap%
\pgfsetroundjoin%
\definecolor{currentfill}{rgb}{0.204903,0.375746,0.553533}%
\pgfsetfillcolor{currentfill}%
\pgfsetfillopacity{0.700000}%
\pgfsetlinewidth{0.000000pt}%
\definecolor{currentstroke}{rgb}{0.000000,0.000000,0.000000}%
\pgfsetstrokecolor{currentstroke}%
\pgfsetdash{}{0pt}%
\pgfpathmoveto{\pgfqpoint{4.576215in}{1.466366in}}%
\pgfpathlineto{\pgfqpoint{4.590541in}{1.474634in}}%
\pgfpathlineto{\pgfqpoint{4.604884in}{1.483056in}}%
\pgfpathlineto{\pgfqpoint{4.619242in}{1.491631in}}%
\pgfpathlineto{\pgfqpoint{4.633617in}{1.500362in}}%
\pgfpathlineto{\pgfqpoint{4.641670in}{1.520792in}}%
\pgfpathlineto{\pgfqpoint{4.649721in}{1.541300in}}%
\pgfpathlineto{\pgfqpoint{4.657771in}{1.561877in}}%
\pgfpathlineto{\pgfqpoint{4.665819in}{1.582518in}}%
\pgfpathlineto{\pgfqpoint{4.651430in}{1.573103in}}%
\pgfpathlineto{\pgfqpoint{4.637057in}{1.563843in}}%
\pgfpathlineto{\pgfqpoint{4.622701in}{1.554738in}}%
\pgfpathlineto{\pgfqpoint{4.608361in}{1.545788in}}%
\pgfpathlineto{\pgfqpoint{4.600327in}{1.525821in}}%
\pgfpathlineto{\pgfqpoint{4.592291in}{1.505923in}}%
\pgfpathlineto{\pgfqpoint{4.584254in}{1.486103in}}%
\pgfpathlineto{\pgfqpoint{4.576215in}{1.466366in}}%
\pgfpathclose%
\pgfusepath{fill}%
\end{pgfscope}%
\begin{pgfscope}%
\pgfpathrectangle{\pgfqpoint{1.254980in}{0.150000in}}{\pgfqpoint{5.490039in}{5.490039in}}%
\pgfusepath{clip}%
\pgfsetbuttcap%
\pgfsetroundjoin%
\definecolor{currentfill}{rgb}{0.119699,0.618490,0.536347}%
\pgfsetfillcolor{currentfill}%
\pgfsetfillopacity{0.700000}%
\pgfsetlinewidth{0.000000pt}%
\definecolor{currentstroke}{rgb}{0.000000,0.000000,0.000000}%
\pgfsetstrokecolor{currentstroke}%
\pgfsetdash{}{0pt}%
\pgfpathmoveto{\pgfqpoint{4.916484in}{2.135231in}}%
\pgfpathlineto{\pgfqpoint{4.931068in}{2.148988in}}%
\pgfpathlineto{\pgfqpoint{4.945673in}{2.162907in}}%
\pgfpathlineto{\pgfqpoint{4.960297in}{2.176989in}}%
\pgfpathlineto{\pgfqpoint{4.968316in}{2.198254in}}%
\pgfpathlineto{\pgfqpoint{4.976332in}{2.219431in}}%
\pgfpathlineto{\pgfqpoint{4.984343in}{2.240513in}}%
\pgfpathlineto{\pgfqpoint{4.992350in}{2.261498in}}%
\pgfpathlineto{\pgfqpoint{4.977707in}{2.246920in}}%
\pgfpathlineto{\pgfqpoint{4.963084in}{2.232507in}}%
\pgfpathlineto{\pgfqpoint{4.948481in}{2.218258in}}%
\pgfpathlineto{\pgfqpoint{4.940487in}{2.197635in}}%
\pgfpathlineto{\pgfqpoint{4.932490in}{2.176921in}}%
\pgfpathlineto{\pgfqpoint{4.924489in}{2.156118in}}%
\pgfpathlineto{\pgfqpoint{4.916484in}{2.135231in}}%
\pgfpathclose%
\pgfusepath{fill}%
\end{pgfscope}%
\begin{pgfscope}%
\pgfpathrectangle{\pgfqpoint{1.254980in}{0.150000in}}{\pgfqpoint{5.490039in}{5.490039in}}%
\pgfusepath{clip}%
\pgfsetbuttcap%
\pgfsetroundjoin%
\definecolor{currentfill}{rgb}{0.159194,0.482237,0.558073}%
\pgfsetfillcolor{currentfill}%
\pgfsetfillopacity{0.700000}%
\pgfsetlinewidth{0.000000pt}%
\definecolor{currentstroke}{rgb}{0.000000,0.000000,0.000000}%
\pgfsetstrokecolor{currentstroke}%
\pgfsetdash{}{0pt}%
\pgfpathmoveto{\pgfqpoint{4.730141in}{1.749209in}}%
\pgfpathlineto{\pgfqpoint{4.744579in}{1.760070in}}%
\pgfpathlineto{\pgfqpoint{4.759035in}{1.771089in}}%
\pgfpathlineto{\pgfqpoint{4.773509in}{1.782267in}}%
\pgfpathlineto{\pgfqpoint{4.788000in}{1.793602in}}%
\pgfpathlineto{\pgfqpoint{4.796049in}{1.815160in}}%
\pgfpathlineto{\pgfqpoint{4.804096in}{1.836716in}}%
\pgfpathlineto{\pgfqpoint{4.812141in}{1.858264in}}%
\pgfpathlineto{\pgfqpoint{4.820183in}{1.879798in}}%
\pgfpathlineto{\pgfqpoint{4.805673in}{1.867856in}}%
\pgfpathlineto{\pgfqpoint{4.791182in}{1.856072in}}%
\pgfpathlineto{\pgfqpoint{4.776709in}{1.844448in}}%
\pgfpathlineto{\pgfqpoint{4.762254in}{1.832983in}}%
\pgfpathlineto{\pgfqpoint{4.754229in}{1.812044in}}%
\pgfpathlineto{\pgfqpoint{4.746202in}{1.791098in}}%
\pgfpathlineto{\pgfqpoint{4.738173in}{1.770151in}}%
\pgfpathlineto{\pgfqpoint{4.730141in}{1.749209in}}%
\pgfpathclose%
\pgfusepath{fill}%
\end{pgfscope}%
\begin{pgfscope}%
\pgfpathrectangle{\pgfqpoint{1.254980in}{0.150000in}}{\pgfqpoint{5.490039in}{5.490039in}}%
\pgfusepath{clip}%
\pgfsetbuttcap%
\pgfsetroundjoin%
\definecolor{currentfill}{rgb}{0.218130,0.347432,0.550038}%
\pgfsetfillcolor{currentfill}%
\pgfsetfillopacity{0.700000}%
\pgfsetlinewidth{0.000000pt}%
\definecolor{currentstroke}{rgb}{0.000000,0.000000,0.000000}%
\pgfsetstrokecolor{currentstroke}%
\pgfsetdash{}{0pt}%
\pgfpathmoveto{\pgfqpoint{4.544040in}{1.388378in}}%
\pgfpathlineto{\pgfqpoint{4.558355in}{1.395937in}}%
\pgfpathlineto{\pgfqpoint{4.572684in}{1.403649in}}%
\pgfpathlineto{\pgfqpoint{4.587029in}{1.411514in}}%
\pgfpathlineto{\pgfqpoint{4.601390in}{1.419532in}}%
\pgfpathlineto{\pgfqpoint{4.609448in}{1.439592in}}%
\pgfpathlineto{\pgfqpoint{4.617506in}{1.459755in}}%
\pgfpathlineto{\pgfqpoint{4.625562in}{1.480014in}}%
\pgfpathlineto{\pgfqpoint{4.633617in}{1.500362in}}%
\pgfpathlineto{\pgfqpoint{4.619242in}{1.491631in}}%
\pgfpathlineto{\pgfqpoint{4.604884in}{1.483056in}}%
\pgfpathlineto{\pgfqpoint{4.590541in}{1.474634in}}%
\pgfpathlineto{\pgfqpoint{4.576215in}{1.466366in}}%
\pgfpathlineto{\pgfqpoint{4.568174in}{1.446718in}}%
\pgfpathlineto{\pgfqpoint{4.560131in}{1.427167in}}%
\pgfpathlineto{\pgfqpoint{4.552086in}{1.407718in}}%
\pgfpathlineto{\pgfqpoint{4.544040in}{1.388378in}}%
\pgfpathclose%
\pgfusepath{fill}%
\end{pgfscope}%
\begin{pgfscope}%
\pgfpathrectangle{\pgfqpoint{1.254980in}{0.150000in}}{\pgfqpoint{5.490039in}{5.490039in}}%
\pgfusepath{clip}%
\pgfsetbuttcap%
\pgfsetroundjoin%
\definecolor{currentfill}{rgb}{0.121831,0.589055,0.545623}%
\pgfsetfillcolor{currentfill}%
\pgfsetfillopacity{0.700000}%
\pgfsetlinewidth{0.000000pt}%
\definecolor{currentstroke}{rgb}{0.000000,0.000000,0.000000}%
\pgfsetstrokecolor{currentstroke}%
\pgfsetdash{}{0pt}%
\pgfpathmoveto{\pgfqpoint{4.884431in}{2.050942in}}%
\pgfpathlineto{\pgfqpoint{4.898997in}{2.064175in}}%
\pgfpathlineto{\pgfqpoint{4.913582in}{2.077571in}}%
\pgfpathlineto{\pgfqpoint{4.928187in}{2.091129in}}%
\pgfpathlineto{\pgfqpoint{4.936220in}{2.112704in}}%
\pgfpathlineto{\pgfqpoint{4.944249in}{2.134209in}}%
\pgfpathlineto{\pgfqpoint{4.952275in}{2.155639in}}%
\pgfpathlineto{\pgfqpoint{4.960297in}{2.176989in}}%
\pgfpathlineto{\pgfqpoint{4.945673in}{2.162907in}}%
\pgfpathlineto{\pgfqpoint{4.931068in}{2.148988in}}%
\pgfpathlineto{\pgfqpoint{4.916484in}{2.135231in}}%
\pgfpathlineto{\pgfqpoint{4.908476in}{2.114266in}}%
\pgfpathlineto{\pgfqpoint{4.900464in}{2.093226in}}%
\pgfpathlineto{\pgfqpoint{4.892449in}{2.072116in}}%
\pgfpathlineto{\pgfqpoint{4.884431in}{2.050942in}}%
\pgfpathclose%
\pgfusepath{fill}%
\end{pgfscope}%
\begin{pgfscope}%
\pgfpathrectangle{\pgfqpoint{1.254980in}{0.150000in}}{\pgfqpoint{5.490039in}{5.490039in}}%
\pgfusepath{clip}%
\pgfsetbuttcap%
\pgfsetroundjoin%
\definecolor{currentfill}{rgb}{0.169646,0.456262,0.558030}%
\pgfsetfillcolor{currentfill}%
\pgfsetfillopacity{0.700000}%
\pgfsetlinewidth{0.000000pt}%
\definecolor{currentstroke}{rgb}{0.000000,0.000000,0.000000}%
\pgfsetstrokecolor{currentstroke}%
\pgfsetdash{}{0pt}%
\pgfpathmoveto{\pgfqpoint{4.697995in}{1.665597in}}%
\pgfpathlineto{\pgfqpoint{4.712417in}{1.675827in}}%
\pgfpathlineto{\pgfqpoint{4.726855in}{1.686213in}}%
\pgfpathlineto{\pgfqpoint{4.741311in}{1.696756in}}%
\pgfpathlineto{\pgfqpoint{4.755785in}{1.707457in}}%
\pgfpathlineto{\pgfqpoint{4.763842in}{1.728968in}}%
\pgfpathlineto{\pgfqpoint{4.771896in}{1.750500in}}%
\pgfpathlineto{\pgfqpoint{4.779949in}{1.772047in}}%
\pgfpathlineto{\pgfqpoint{4.788000in}{1.793602in}}%
\pgfpathlineto{\pgfqpoint{4.773509in}{1.782267in}}%
\pgfpathlineto{\pgfqpoint{4.759035in}{1.771089in}}%
\pgfpathlineto{\pgfqpoint{4.744579in}{1.760070in}}%
\pgfpathlineto{\pgfqpoint{4.730141in}{1.749209in}}%
\pgfpathlineto{\pgfqpoint{4.722108in}{1.728276in}}%
\pgfpathlineto{\pgfqpoint{4.714072in}{1.707360in}}%
\pgfpathlineto{\pgfqpoint{4.706034in}{1.686465in}}%
\pgfpathlineto{\pgfqpoint{4.697995in}{1.665597in}}%
\pgfpathclose%
\pgfusepath{fill}%
\end{pgfscope}%
\begin{pgfscope}%
\pgfpathrectangle{\pgfqpoint{1.254980in}{0.150000in}}{\pgfqpoint{5.490039in}{5.490039in}}%
\pgfusepath{clip}%
\pgfsetbuttcap%
\pgfsetroundjoin%
\definecolor{currentfill}{rgb}{0.128729,0.563265,0.551229}%
\pgfsetfillcolor{currentfill}%
\pgfsetfillopacity{0.700000}%
\pgfsetlinewidth{0.000000pt}%
\definecolor{currentstroke}{rgb}{0.000000,0.000000,0.000000}%
\pgfsetstrokecolor{currentstroke}%
\pgfsetdash{}{0pt}%
\pgfpathmoveto{\pgfqpoint{4.852328in}{1.965691in}}%
\pgfpathlineto{\pgfqpoint{4.866876in}{1.978373in}}%
\pgfpathlineto{\pgfqpoint{4.881442in}{1.991216in}}%
\pgfpathlineto{\pgfqpoint{4.896027in}{2.004220in}}%
\pgfpathlineto{\pgfqpoint{4.904071in}{2.026029in}}%
\pgfpathlineto{\pgfqpoint{4.912113in}{2.047786in}}%
\pgfpathlineto{\pgfqpoint{4.920151in}{2.069488in}}%
\pgfpathlineto{\pgfqpoint{4.928187in}{2.091129in}}%
\pgfpathlineto{\pgfqpoint{4.913582in}{2.077571in}}%
\pgfpathlineto{\pgfqpoint{4.898997in}{2.064175in}}%
\pgfpathlineto{\pgfqpoint{4.884431in}{2.050942in}}%
\pgfpathlineto{\pgfqpoint{4.876410in}{2.029707in}}%
\pgfpathlineto{\pgfqpoint{4.868385in}{2.008417in}}%
\pgfpathlineto{\pgfqpoint{4.860358in}{1.987076in}}%
\pgfpathlineto{\pgfqpoint{4.852328in}{1.965691in}}%
\pgfpathclose%
\pgfusepath{fill}%
\end{pgfscope}%
\begin{pgfscope}%
\pgfpathrectangle{\pgfqpoint{1.254980in}{0.150000in}}{\pgfqpoint{5.490039in}{5.490039in}}%
\pgfusepath{clip}%
\pgfsetbuttcap%
\pgfsetroundjoin%
\definecolor{currentfill}{rgb}{0.229739,0.322361,0.545706}%
\pgfsetfillcolor{currentfill}%
\pgfsetfillopacity{0.700000}%
\pgfsetlinewidth{0.000000pt}%
\definecolor{currentstroke}{rgb}{0.000000,0.000000,0.000000}%
\pgfsetstrokecolor{currentstroke}%
\pgfsetdash{}{0pt}%
\pgfpathmoveto{\pgfqpoint{4.511839in}{1.312248in}}%
\pgfpathlineto{\pgfqpoint{4.526142in}{1.319072in}}%
\pgfpathlineto{\pgfqpoint{4.540459in}{1.326047in}}%
\pgfpathlineto{\pgfqpoint{4.554792in}{1.333175in}}%
\pgfpathlineto{\pgfqpoint{4.569140in}{1.340454in}}%
\pgfpathlineto{\pgfqpoint{4.577204in}{1.360036in}}%
\pgfpathlineto{\pgfqpoint{4.585267in}{1.379747in}}%
\pgfpathlineto{\pgfqpoint{4.593329in}{1.399582in}}%
\pgfpathlineto{\pgfqpoint{4.601390in}{1.419532in}}%
\pgfpathlineto{\pgfqpoint{4.587029in}{1.411514in}}%
\pgfpathlineto{\pgfqpoint{4.572684in}{1.403649in}}%
\pgfpathlineto{\pgfqpoint{4.558355in}{1.395937in}}%
\pgfpathlineto{\pgfqpoint{4.544040in}{1.388378in}}%
\pgfpathlineto{\pgfqpoint{4.535992in}{1.369155in}}%
\pgfpathlineto{\pgfqpoint{4.527943in}{1.350054in}}%
\pgfpathlineto{\pgfqpoint{4.519892in}{1.331083in}}%
\pgfpathlineto{\pgfqpoint{4.511839in}{1.312248in}}%
\pgfpathclose%
\pgfusepath{fill}%
\end{pgfscope}%
\begin{pgfscope}%
\pgfpathrectangle{\pgfqpoint{1.254980in}{0.150000in}}{\pgfqpoint{5.490039in}{5.490039in}}%
\pgfusepath{clip}%
\pgfsetbuttcap%
\pgfsetroundjoin%
\definecolor{currentfill}{rgb}{0.180629,0.429975,0.557282}%
\pgfsetfillcolor{currentfill}%
\pgfsetfillopacity{0.700000}%
\pgfsetlinewidth{0.000000pt}%
\definecolor{currentstroke}{rgb}{0.000000,0.000000,0.000000}%
\pgfsetstrokecolor{currentstroke}%
\pgfsetdash{}{0pt}%
\pgfpathmoveto{\pgfqpoint{4.665819in}{1.582518in}}%
\pgfpathlineto{\pgfqpoint{4.680225in}{1.592089in}}%
\pgfpathlineto{\pgfqpoint{4.694647in}{1.601815in}}%
\pgfpathlineto{\pgfqpoint{4.709086in}{1.611697in}}%
\pgfpathlineto{\pgfqpoint{4.723542in}{1.621735in}}%
\pgfpathlineto{\pgfqpoint{4.731605in}{1.643105in}}%
\pgfpathlineto{\pgfqpoint{4.739667in}{1.664519in}}%
\pgfpathlineto{\pgfqpoint{4.747727in}{1.685972in}}%
\pgfpathlineto{\pgfqpoint{4.755785in}{1.707457in}}%
\pgfpathlineto{\pgfqpoint{4.741311in}{1.696756in}}%
\pgfpathlineto{\pgfqpoint{4.726855in}{1.686213in}}%
\pgfpathlineto{\pgfqpoint{4.712417in}{1.675827in}}%
\pgfpathlineto{\pgfqpoint{4.697995in}{1.665597in}}%
\pgfpathlineto{\pgfqpoint{4.689954in}{1.644762in}}%
\pgfpathlineto{\pgfqpoint{4.681911in}{1.623967in}}%
\pgfpathlineto{\pgfqpoint{4.673866in}{1.603217in}}%
\pgfpathlineto{\pgfqpoint{4.665819in}{1.582518in}}%
\pgfpathclose%
\pgfusepath{fill}%
\end{pgfscope}%
\begin{pgfscope}%
\pgfpathrectangle{\pgfqpoint{1.254980in}{0.150000in}}{\pgfqpoint{5.490039in}{5.490039in}}%
\pgfusepath{clip}%
\pgfsetbuttcap%
\pgfsetroundjoin%
\definecolor{currentfill}{rgb}{0.137770,0.537492,0.554906}%
\pgfsetfillcolor{currentfill}%
\pgfsetfillopacity{0.700000}%
\pgfsetlinewidth{0.000000pt}%
\definecolor{currentstroke}{rgb}{0.000000,0.000000,0.000000}%
\pgfsetstrokecolor{currentstroke}%
\pgfsetdash{}{0pt}%
\pgfpathmoveto{\pgfqpoint{4.820183in}{1.879798in}}%
\pgfpathlineto{\pgfqpoint{4.834711in}{1.891901in}}%
\pgfpathlineto{\pgfqpoint{4.849258in}{1.904162in}}%
\pgfpathlineto{\pgfqpoint{4.863824in}{1.916584in}}%
\pgfpathlineto{\pgfqpoint{4.871878in}{1.938543in}}%
\pgfpathlineto{\pgfqpoint{4.879930in}{1.960472in}}%
\pgfpathlineto{\pgfqpoint{4.887980in}{1.982367in}}%
\pgfpathlineto{\pgfqpoint{4.896027in}{2.004220in}}%
\pgfpathlineto{\pgfqpoint{4.881442in}{1.991216in}}%
\pgfpathlineto{\pgfqpoint{4.866876in}{1.978373in}}%
\pgfpathlineto{\pgfqpoint{4.852328in}{1.965691in}}%
\pgfpathlineto{\pgfqpoint{4.844296in}{1.944265in}}%
\pgfpathlineto{\pgfqpoint{4.836261in}{1.922804in}}%
\pgfpathlineto{\pgfqpoint{4.828223in}{1.901313in}}%
\pgfpathlineto{\pgfqpoint{4.820183in}{1.879798in}}%
\pgfpathclose%
\pgfusepath{fill}%
\end{pgfscope}%
\begin{pgfscope}%
\pgfpathrectangle{\pgfqpoint{1.254980in}{0.150000in}}{\pgfqpoint{5.490039in}{5.490039in}}%
\pgfusepath{clip}%
\pgfsetbuttcap%
\pgfsetroundjoin%
\definecolor{currentfill}{rgb}{0.192357,0.403199,0.555836}%
\pgfsetfillcolor{currentfill}%
\pgfsetfillopacity{0.700000}%
\pgfsetlinewidth{0.000000pt}%
\definecolor{currentstroke}{rgb}{0.000000,0.000000,0.000000}%
\pgfsetstrokecolor{currentstroke}%
\pgfsetdash{}{0pt}%
\pgfpathmoveto{\pgfqpoint{4.633617in}{1.500362in}}%
\pgfpathlineto{\pgfqpoint{4.648007in}{1.509246in}}%
\pgfpathlineto{\pgfqpoint{4.662413in}{1.518285in}}%
\pgfpathlineto{\pgfqpoint{4.676836in}{1.527478in}}%
\pgfpathlineto{\pgfqpoint{4.691276in}{1.536827in}}%
\pgfpathlineto{\pgfqpoint{4.699344in}{1.557956in}}%
\pgfpathlineto{\pgfqpoint{4.707412in}{1.579154in}}%
\pgfpathlineto{\pgfqpoint{4.715478in}{1.600416in}}%
\pgfpathlineto{\pgfqpoint{4.723542in}{1.621735in}}%
\pgfpathlineto{\pgfqpoint{4.709086in}{1.611697in}}%
\pgfpathlineto{\pgfqpoint{4.694647in}{1.601815in}}%
\pgfpathlineto{\pgfqpoint{4.680225in}{1.592089in}}%
\pgfpathlineto{\pgfqpoint{4.665819in}{1.582518in}}%
\pgfpathlineto{\pgfqpoint{4.657771in}{1.561877in}}%
\pgfpathlineto{\pgfqpoint{4.649721in}{1.541300in}}%
\pgfpathlineto{\pgfqpoint{4.641670in}{1.520792in}}%
\pgfpathlineto{\pgfqpoint{4.633617in}{1.500362in}}%
\pgfpathclose%
\pgfusepath{fill}%
\end{pgfscope}%
\begin{pgfscope}%
\pgfpathrectangle{\pgfqpoint{1.254980in}{0.150000in}}{\pgfqpoint{5.490039in}{5.490039in}}%
\pgfusepath{clip}%
\pgfsetbuttcap%
\pgfsetroundjoin%
\definecolor{currentfill}{rgb}{0.149039,0.508051,0.557250}%
\pgfsetfillcolor{currentfill}%
\pgfsetfillopacity{0.700000}%
\pgfsetlinewidth{0.000000pt}%
\definecolor{currentstroke}{rgb}{0.000000,0.000000,0.000000}%
\pgfsetstrokecolor{currentstroke}%
\pgfsetdash{}{0pt}%
\pgfpathmoveto{\pgfqpoint{4.788000in}{1.793602in}}%
\pgfpathlineto{\pgfqpoint{4.802510in}{1.805096in}}%
\pgfpathlineto{\pgfqpoint{4.817037in}{1.816748in}}%
\pgfpathlineto{\pgfqpoint{4.831584in}{1.828559in}}%
\pgfpathlineto{\pgfqpoint{4.839647in}{1.850583in}}%
\pgfpathlineto{\pgfqpoint{4.847708in}{1.872599in}}%
\pgfpathlineto{\pgfqpoint{4.855767in}{1.894601in}}%
\pgfpathlineto{\pgfqpoint{4.863824in}{1.916584in}}%
\pgfpathlineto{\pgfqpoint{4.849258in}{1.904162in}}%
\pgfpathlineto{\pgfqpoint{4.834711in}{1.891901in}}%
\pgfpathlineto{\pgfqpoint{4.820183in}{1.879798in}}%
\pgfpathlineto{\pgfqpoint{4.812141in}{1.858264in}}%
\pgfpathlineto{\pgfqpoint{4.804096in}{1.836716in}}%
\pgfpathlineto{\pgfqpoint{4.796049in}{1.815160in}}%
\pgfpathlineto{\pgfqpoint{4.788000in}{1.793602in}}%
\pgfpathclose%
\pgfusepath{fill}%
\end{pgfscope}%
\begin{pgfscope}%
\pgfpathrectangle{\pgfqpoint{1.254980in}{0.150000in}}{\pgfqpoint{5.490039in}{5.490039in}}%
\pgfusepath{clip}%
\pgfsetbuttcap%
\pgfsetroundjoin%
\definecolor{currentfill}{rgb}{0.204903,0.375746,0.553533}%
\pgfsetfillcolor{currentfill}%
\pgfsetfillopacity{0.700000}%
\pgfsetlinewidth{0.000000pt}%
\definecolor{currentstroke}{rgb}{0.000000,0.000000,0.000000}%
\pgfsetstrokecolor{currentstroke}%
\pgfsetdash{}{0pt}%
\pgfpathmoveto{\pgfqpoint{4.601390in}{1.419532in}}%
\pgfpathlineto{\pgfqpoint{4.615765in}{1.427704in}}%
\pgfpathlineto{\pgfqpoint{4.630157in}{1.436029in}}%
\pgfpathlineto{\pgfqpoint{4.644565in}{1.444507in}}%
\pgfpathlineto{\pgfqpoint{4.658988in}{1.453139in}}%
\pgfpathlineto{\pgfqpoint{4.667062in}{1.473924in}}%
\pgfpathlineto{\pgfqpoint{4.675134in}{1.494804in}}%
\pgfpathlineto{\pgfqpoint{4.683206in}{1.515774in}}%
\pgfpathlineto{\pgfqpoint{4.691276in}{1.536827in}}%
\pgfpathlineto{\pgfqpoint{4.676836in}{1.527478in}}%
\pgfpathlineto{\pgfqpoint{4.662413in}{1.518285in}}%
\pgfpathlineto{\pgfqpoint{4.648007in}{1.509246in}}%
\pgfpathlineto{\pgfqpoint{4.633617in}{1.500362in}}%
\pgfpathlineto{\pgfqpoint{4.625562in}{1.480014in}}%
\pgfpathlineto{\pgfqpoint{4.617506in}{1.459755in}}%
\pgfpathlineto{\pgfqpoint{4.609448in}{1.439592in}}%
\pgfpathlineto{\pgfqpoint{4.601390in}{1.419532in}}%
\pgfpathclose%
\pgfusepath{fill}%
\end{pgfscope}%
\begin{pgfscope}%
\pgfpathrectangle{\pgfqpoint{1.254980in}{0.150000in}}{\pgfqpoint{5.490039in}{5.490039in}}%
\pgfusepath{clip}%
\pgfsetbuttcap%
\pgfsetroundjoin%
\definecolor{currentfill}{rgb}{0.159194,0.482237,0.558073}%
\pgfsetfillcolor{currentfill}%
\pgfsetfillopacity{0.700000}%
\pgfsetlinewidth{0.000000pt}%
\definecolor{currentstroke}{rgb}{0.000000,0.000000,0.000000}%
\pgfsetstrokecolor{currentstroke}%
\pgfsetdash{}{0pt}%
\pgfpathmoveto{\pgfqpoint{4.755785in}{1.707457in}}%
\pgfpathlineto{\pgfqpoint{4.770276in}{1.718314in}}%
\pgfpathlineto{\pgfqpoint{4.784785in}{1.729330in}}%
\pgfpathlineto{\pgfqpoint{4.799312in}{1.740502in}}%
\pgfpathlineto{\pgfqpoint{4.807383in}{1.762500in}}%
\pgfpathlineto{\pgfqpoint{4.815451in}{1.784512in}}%
\pgfpathlineto{\pgfqpoint{4.823518in}{1.806534in}}%
\pgfpathlineto{\pgfqpoint{4.831584in}{1.828559in}}%
\pgfpathlineto{\pgfqpoint{4.817037in}{1.816748in}}%
\pgfpathlineto{\pgfqpoint{4.802510in}{1.805096in}}%
\pgfpathlineto{\pgfqpoint{4.788000in}{1.793602in}}%
\pgfpathlineto{\pgfqpoint{4.779949in}{1.772047in}}%
\pgfpathlineto{\pgfqpoint{4.771896in}{1.750500in}}%
\pgfpathlineto{\pgfqpoint{4.763842in}{1.728968in}}%
\pgfpathlineto{\pgfqpoint{4.755785in}{1.707457in}}%
\pgfpathclose%
\pgfusepath{fill}%
\end{pgfscope}%
\begin{pgfscope}%
\pgfpathrectangle{\pgfqpoint{1.254980in}{0.150000in}}{\pgfqpoint{5.490039in}{5.490039in}}%
\pgfusepath{clip}%
\pgfsetbuttcap%
\pgfsetroundjoin%
\definecolor{currentfill}{rgb}{0.218130,0.347432,0.550038}%
\pgfsetfillcolor{currentfill}%
\pgfsetfillopacity{0.700000}%
\pgfsetlinewidth{0.000000pt}%
\definecolor{currentstroke}{rgb}{0.000000,0.000000,0.000000}%
\pgfsetstrokecolor{currentstroke}%
\pgfsetdash{}{0pt}%
\pgfpathmoveto{\pgfqpoint{4.569140in}{1.340454in}}%
\pgfpathlineto{\pgfqpoint{4.583502in}{1.347886in}}%
\pgfpathlineto{\pgfqpoint{4.597880in}{1.355470in}}%
\pgfpathlineto{\pgfqpoint{4.612273in}{1.363207in}}%
\pgfpathlineto{\pgfqpoint{4.626682in}{1.371096in}}%
\pgfpathlineto{\pgfqpoint{4.634760in}{1.391429in}}%
\pgfpathlineto{\pgfqpoint{4.642837in}{1.411885in}}%
\pgfpathlineto{\pgfqpoint{4.650913in}{1.432457in}}%
\pgfpathlineto{\pgfqpoint{4.658988in}{1.453139in}}%
\pgfpathlineto{\pgfqpoint{4.644565in}{1.444507in}}%
\pgfpathlineto{\pgfqpoint{4.630157in}{1.436029in}}%
\pgfpathlineto{\pgfqpoint{4.615765in}{1.427704in}}%
\pgfpathlineto{\pgfqpoint{4.601390in}{1.419532in}}%
\pgfpathlineto{\pgfqpoint{4.593329in}{1.399582in}}%
\pgfpathlineto{\pgfqpoint{4.585267in}{1.379747in}}%
\pgfpathlineto{\pgfqpoint{4.577204in}{1.360036in}}%
\pgfpathlineto{\pgfqpoint{4.569140in}{1.340454in}}%
\pgfpathclose%
\pgfusepath{fill}%
\end{pgfscope}%
\begin{pgfscope}%
\pgfpathrectangle{\pgfqpoint{1.254980in}{0.150000in}}{\pgfqpoint{5.490039in}{5.490039in}}%
\pgfusepath{clip}%
\pgfsetbuttcap%
\pgfsetroundjoin%
\definecolor{currentfill}{rgb}{0.171176,0.452530,0.557965}%
\pgfsetfillcolor{currentfill}%
\pgfsetfillopacity{0.700000}%
\pgfsetlinewidth{0.000000pt}%
\definecolor{currentstroke}{rgb}{0.000000,0.000000,0.000000}%
\pgfsetstrokecolor{currentstroke}%
\pgfsetdash{}{0pt}%
\pgfpathmoveto{\pgfqpoint{4.723542in}{1.621735in}}%
\pgfpathlineto{\pgfqpoint{4.738016in}{1.631929in}}%
\pgfpathlineto{\pgfqpoint{4.752506in}{1.642280in}}%
\pgfpathlineto{\pgfqpoint{4.767014in}{1.652786in}}%
\pgfpathlineto{\pgfqpoint{4.775091in}{1.674662in}}%
\pgfpathlineto{\pgfqpoint{4.783166in}{1.696578in}}%
\pgfpathlineto{\pgfqpoint{4.791240in}{1.718526in}}%
\pgfpathlineto{\pgfqpoint{4.799312in}{1.740502in}}%
\pgfpathlineto{\pgfqpoint{4.784785in}{1.729330in}}%
\pgfpathlineto{\pgfqpoint{4.770276in}{1.718314in}}%
\pgfpathlineto{\pgfqpoint{4.755785in}{1.707457in}}%
\pgfpathlineto{\pgfqpoint{4.747727in}{1.685972in}}%
\pgfpathlineto{\pgfqpoint{4.739667in}{1.664519in}}%
\pgfpathlineto{\pgfqpoint{4.731605in}{1.643105in}}%
\pgfpathlineto{\pgfqpoint{4.723542in}{1.621735in}}%
\pgfpathclose%
\pgfusepath{fill}%
\end{pgfscope}%
\begin{pgfscope}%
\pgfpathrectangle{\pgfqpoint{1.254980in}{0.150000in}}{\pgfqpoint{5.490039in}{5.490039in}}%
\pgfusepath{clip}%
\pgfsetbuttcap%
\pgfsetroundjoin%
\definecolor{currentfill}{rgb}{0.182256,0.426184,0.557120}%
\pgfsetfillcolor{currentfill}%
\pgfsetfillopacity{0.700000}%
\pgfsetlinewidth{0.000000pt}%
\definecolor{currentstroke}{rgb}{0.000000,0.000000,0.000000}%
\pgfsetstrokecolor{currentstroke}%
\pgfsetdash{}{0pt}%
\pgfpathmoveto{\pgfqpoint{4.691276in}{1.536827in}}%
\pgfpathlineto{\pgfqpoint{4.705732in}{1.546330in}}%
\pgfpathlineto{\pgfqpoint{4.720205in}{1.555989in}}%
\pgfpathlineto{\pgfqpoint{4.734695in}{1.565803in}}%
\pgfpathlineto{\pgfqpoint{4.742777in}{1.587458in}}%
\pgfpathlineto{\pgfqpoint{4.750857in}{1.609178in}}%
\pgfpathlineto{\pgfqpoint{4.758936in}{1.630956in}}%
\pgfpathlineto{\pgfqpoint{4.767014in}{1.652786in}}%
\pgfpathlineto{\pgfqpoint{4.752506in}{1.642280in}}%
\pgfpathlineto{\pgfqpoint{4.738016in}{1.631929in}}%
\pgfpathlineto{\pgfqpoint{4.723542in}{1.621735in}}%
\pgfpathlineto{\pgfqpoint{4.715478in}{1.600416in}}%
\pgfpathlineto{\pgfqpoint{4.707412in}{1.579154in}}%
\pgfpathlineto{\pgfqpoint{4.699344in}{1.557956in}}%
\pgfpathlineto{\pgfqpoint{4.691276in}{1.536827in}}%
\pgfpathclose%
\pgfusepath{fill}%
\end{pgfscope}%
\begin{pgfscope}%
\pgfpathrectangle{\pgfqpoint{1.254980in}{0.150000in}}{\pgfqpoint{5.490039in}{5.490039in}}%
\pgfusepath{clip}%
\pgfsetbuttcap%
\pgfsetroundjoin%
\definecolor{currentfill}{rgb}{0.194100,0.399323,0.555565}%
\pgfsetfillcolor{currentfill}%
\pgfsetfillopacity{0.700000}%
\pgfsetlinewidth{0.000000pt}%
\definecolor{currentstroke}{rgb}{0.000000,0.000000,0.000000}%
\pgfsetstrokecolor{currentstroke}%
\pgfsetdash{}{0pt}%
\pgfpathmoveto{\pgfqpoint{4.658988in}{1.453139in}}%
\pgfpathlineto{\pgfqpoint{4.673428in}{1.461925in}}%
\pgfpathlineto{\pgfqpoint{4.687884in}{1.470865in}}%
\pgfpathlineto{\pgfqpoint{4.702357in}{1.479959in}}%
\pgfpathlineto{\pgfqpoint{4.710443in}{1.501290in}}%
\pgfpathlineto{\pgfqpoint{4.718528in}{1.522712in}}%
\pgfpathlineto{\pgfqpoint{4.726612in}{1.544219in}}%
\pgfpathlineto{\pgfqpoint{4.734695in}{1.565803in}}%
\pgfpathlineto{\pgfqpoint{4.720205in}{1.555989in}}%
\pgfpathlineto{\pgfqpoint{4.705732in}{1.546330in}}%
\pgfpathlineto{\pgfqpoint{4.691276in}{1.536827in}}%
\pgfpathlineto{\pgfqpoint{4.683206in}{1.515774in}}%
\pgfpathlineto{\pgfqpoint{4.675134in}{1.494804in}}%
\pgfpathlineto{\pgfqpoint{4.667062in}{1.473924in}}%
\pgfpathlineto{\pgfqpoint{4.658988in}{1.453139in}}%
\pgfpathclose%
\pgfusepath{fill}%
\end{pgfscope}%
\begin{pgfscope}%
\pgfpathrectangle{\pgfqpoint{1.254980in}{0.150000in}}{\pgfqpoint{5.490039in}{5.490039in}}%
\pgfusepath{clip}%
\pgfsetbuttcap%
\pgfsetroundjoin%
\definecolor{currentfill}{rgb}{0.206756,0.371758,0.553117}%
\pgfsetfillcolor{currentfill}%
\pgfsetfillopacity{0.700000}%
\pgfsetlinewidth{0.000000pt}%
\definecolor{currentstroke}{rgb}{0.000000,0.000000,0.000000}%
\pgfsetstrokecolor{currentstroke}%
\pgfsetdash{}{0pt}%
\pgfpathmoveto{\pgfqpoint{4.626682in}{1.371096in}}%
\pgfpathlineto{\pgfqpoint{4.641107in}{1.379138in}}%
\pgfpathlineto{\pgfqpoint{4.655547in}{1.387333in}}%
\pgfpathlineto{\pgfqpoint{4.670003in}{1.395681in}}%
\pgfpathlineto{\pgfqpoint{4.678093in}{1.416580in}}%
\pgfpathlineto{\pgfqpoint{4.686182in}{1.437597in}}%
\pgfpathlineto{\pgfqpoint{4.694270in}{1.458726in}}%
\pgfpathlineto{\pgfqpoint{4.702357in}{1.479959in}}%
\pgfpathlineto{\pgfqpoint{4.687884in}{1.470865in}}%
\pgfpathlineto{\pgfqpoint{4.673428in}{1.461925in}}%
\pgfpathlineto{\pgfqpoint{4.658988in}{1.453139in}}%
\pgfpathlineto{\pgfqpoint{4.650913in}{1.432457in}}%
\pgfpathlineto{\pgfqpoint{4.642837in}{1.411885in}}%
\pgfpathlineto{\pgfqpoint{4.634760in}{1.391429in}}%
\pgfpathlineto{\pgfqpoint{4.626682in}{1.371096in}}%
\pgfpathclose%
\pgfusepath{fill}%
\end{pgfscope}%
\begin{pgfscope}%
\pgfsetbuttcap%
\pgfsetmiterjoin%
\definecolor{currentfill}{rgb}{1.000000,1.000000,1.000000}%
\pgfsetfillcolor{currentfill}%
\pgfsetfillopacity{0.800000}%
\pgfsetlinewidth{1.003750pt}%
\definecolor{currentstroke}{rgb}{0.800000,0.800000,0.800000}%
\pgfsetstrokecolor{currentstroke}%
\pgfsetstrokeopacity{0.800000}%
\pgfsetdash{}{0pt}%
\pgfpathmoveto{\pgfqpoint{5.541867in}{5.121213in}}%
\pgfpathlineto{\pgfqpoint{6.647797in}{5.121213in}}%
\pgfpathquadraticcurveto{\pgfqpoint{6.675575in}{5.121213in}}{\pgfqpoint{6.675575in}{5.148991in}}%
\pgfpathlineto{\pgfqpoint{6.675575in}{5.542817in}}%
\pgfpathquadraticcurveto{\pgfqpoint{6.675575in}{5.570595in}}{\pgfqpoint{6.647797in}{5.570595in}}%
\pgfpathlineto{\pgfqpoint{5.541867in}{5.570595in}}%
\pgfpathquadraticcurveto{\pgfqpoint{5.514090in}{5.570595in}}{\pgfqpoint{5.514090in}{5.542817in}}%
\pgfpathlineto{\pgfqpoint{5.514090in}{5.148991in}}%
\pgfpathquadraticcurveto{\pgfqpoint{5.514090in}{5.121213in}}{\pgfqpoint{5.541867in}{5.121213in}}%
\pgfpathlineto{\pgfqpoint{5.541867in}{5.121213in}}%
\pgfpathclose%
\pgfusepath{stroke,fill}%
\end{pgfscope}%
\begin{pgfscope}%
\pgfsetrectcap%
\pgfsetroundjoin%
\pgfsetlinewidth{1.505625pt}%
\definecolor{currentstroke}{rgb}{1.000000,0.000000,0.000000}%
\pgfsetstrokecolor{currentstroke}%
\pgfsetdash{}{0pt}%
\pgfpathmoveto{\pgfqpoint{5.569645in}{5.458127in}}%
\pgfpathlineto{\pgfqpoint{5.708534in}{5.458127in}}%
\pgfpathlineto{\pgfqpoint{5.847423in}{5.458127in}}%
\pgfusepath{stroke}%
\end{pgfscope}%
\begin{pgfscope}%
\pgfsetbuttcap%
\pgfsetroundjoin%
\definecolor{currentfill}{rgb}{1.000000,0.000000,0.000000}%
\pgfsetfillcolor{currentfill}%
\pgfsetlinewidth{1.003750pt}%
\definecolor{currentstroke}{rgb}{1.000000,0.000000,0.000000}%
\pgfsetstrokecolor{currentstroke}%
\pgfsetdash{}{0pt}%
\pgfsys@defobject{currentmarker}{\pgfqpoint{-0.041667in}{-0.041667in}}{\pgfqpoint{0.041667in}{0.041667in}}{%
\pgfpathmoveto{\pgfqpoint{0.000000in}{-0.041667in}}%
\pgfpathcurveto{\pgfqpoint{0.011050in}{-0.041667in}}{\pgfqpoint{0.021649in}{-0.037276in}}{\pgfqpoint{0.029463in}{-0.029463in}}%
\pgfpathcurveto{\pgfqpoint{0.037276in}{-0.021649in}}{\pgfqpoint{0.041667in}{-0.011050in}}{\pgfqpoint{0.041667in}{0.000000in}}%
\pgfpathcurveto{\pgfqpoint{0.041667in}{0.011050in}}{\pgfqpoint{0.037276in}{0.021649in}}{\pgfqpoint{0.029463in}{0.029463in}}%
\pgfpathcurveto{\pgfqpoint{0.021649in}{0.037276in}}{\pgfqpoint{0.011050in}{0.041667in}}{\pgfqpoint{0.000000in}{0.041667in}}%
\pgfpathcurveto{\pgfqpoint{-0.011050in}{0.041667in}}{\pgfqpoint{-0.021649in}{0.037276in}}{\pgfqpoint{-0.029463in}{0.029463in}}%
\pgfpathcurveto{\pgfqpoint{-0.037276in}{0.021649in}}{\pgfqpoint{-0.041667in}{0.011050in}}{\pgfqpoint{-0.041667in}{0.000000in}}%
\pgfpathcurveto{\pgfqpoint{-0.041667in}{-0.011050in}}{\pgfqpoint{-0.037276in}{-0.021649in}}{\pgfqpoint{-0.029463in}{-0.029463in}}%
\pgfpathcurveto{\pgfqpoint{-0.021649in}{-0.037276in}}{\pgfqpoint{-0.011050in}{-0.041667in}}{\pgfqpoint{0.000000in}{-0.041667in}}%
\pgfpathlineto{\pgfqpoint{0.000000in}{-0.041667in}}%
\pgfpathclose%
\pgfusepath{stroke,fill}%
}%
\begin{pgfscope}%
\pgfsys@transformshift{5.708534in}{5.458127in}%
\pgfsys@useobject{currentmarker}{}%
\end{pgfscope}%
\end{pgfscope}%
\begin{pgfscope}%
\definecolor{textcolor}{rgb}{0.000000,0.000000,0.000000}%
\pgfsetstrokecolor{textcolor}%
\pgfsetfillcolor{textcolor}%
\pgftext[x=5.958534in,y=5.409516in,left,base]{\color{textcolor}\sffamily\fontsize{10.000000}{12.000000}\selectfont Iterations}%
\end{pgfscope}%
\begin{pgfscope}%
\pgfsetbuttcap%
\pgfsetroundjoin%
\definecolor{currentfill}{rgb}{0.000000,0.000000,1.000000}%
\pgfsetfillcolor{currentfill}%
\pgfsetlinewidth{1.003750pt}%
\definecolor{currentstroke}{rgb}{0.000000,0.000000,1.000000}%
\pgfsetstrokecolor{currentstroke}%
\pgfsetdash{}{0pt}%
\pgfsys@defobject{currentmarker}{\pgfqpoint{-0.069444in}{-0.069444in}}{\pgfqpoint{0.069444in}{0.069444in}}{%
\pgfpathmoveto{\pgfqpoint{0.000000in}{-0.069444in}}%
\pgfpathcurveto{\pgfqpoint{0.018417in}{-0.069444in}}{\pgfqpoint{0.036082in}{-0.062127in}}{\pgfqpoint{0.049105in}{-0.049105in}}%
\pgfpathcurveto{\pgfqpoint{0.062127in}{-0.036082in}}{\pgfqpoint{0.069444in}{-0.018417in}}{\pgfqpoint{0.069444in}{0.000000in}}%
\pgfpathcurveto{\pgfqpoint{0.069444in}{0.018417in}}{\pgfqpoint{0.062127in}{0.036082in}}{\pgfqpoint{0.049105in}{0.049105in}}%
\pgfpathcurveto{\pgfqpoint{0.036082in}{0.062127in}}{\pgfqpoint{0.018417in}{0.069444in}}{\pgfqpoint{0.000000in}{0.069444in}}%
\pgfpathcurveto{\pgfqpoint{-0.018417in}{0.069444in}}{\pgfqpoint{-0.036082in}{0.062127in}}{\pgfqpoint{-0.049105in}{0.049105in}}%
\pgfpathcurveto{\pgfqpoint{-0.062127in}{0.036082in}}{\pgfqpoint{-0.069444in}{0.018417in}}{\pgfqpoint{-0.069444in}{0.000000in}}%
\pgfpathcurveto{\pgfqpoint{-0.069444in}{-0.018417in}}{\pgfqpoint{-0.062127in}{-0.036082in}}{\pgfqpoint{-0.049105in}{-0.049105in}}%
\pgfpathcurveto{\pgfqpoint{-0.036082in}{-0.062127in}}{\pgfqpoint{-0.018417in}{-0.069444in}}{\pgfqpoint{0.000000in}{-0.069444in}}%
\pgfpathlineto{\pgfqpoint{0.000000in}{-0.069444in}}%
\pgfpathclose%
\pgfusepath{stroke,fill}%
}%
\begin{pgfscope}%
\pgfsys@transformshift{5.708534in}{5.242117in}%
\pgfsys@useobject{currentmarker}{}%
\end{pgfscope}%
\end{pgfscope}%
\begin{pgfscope}%
\definecolor{textcolor}{rgb}{0.000000,0.000000,0.000000}%
\pgfsetstrokecolor{textcolor}%
\pgfsetfillcolor{textcolor}%
\pgftext[x=5.958534in,y=5.205659in,left,base]{\color{textcolor}\sffamily\fontsize{10.000000}{12.000000}\selectfont Minimum}%
\end{pgfscope}%
\end{pgfpicture}%
\makeatother%
\endgroup%
}
    \caption{Your figure caption}
    \label{fig:cg_surface}
\end{figure}
\newpage


%%%%%%%%%%%%%%%%%%%%%%%%%%%%%%%%%%%%%%%%%%%%%%%%%%%%%%%%%%%%%%%%%%%%%
\newpage
\section{Záverečné zhrnutie a porovnanie metód}

Cieľom tohto projektu bola analýza vybraných numerických metód pre minimalizáciu funkcií jednej a viacerých premenných. Na základe vykonaných experimentov a teoretických poznatkov môžeme formulovať nasledujúce závery.

\subsection*{Funkcia jednej premennej}
Pre minimalizáciu funkcie $f(x)$ sme porovnávali \textit{Metódu polenia intervalu} (MPI) a \textit{Metódu zlatého rezu} (MZR).

\begin{itemize}
    \item \textbf{Efektivita vyčíslení:} Zatiaľ čo MPI zmenšuje interval lokalizácie minima v každom kroku na polovicu (faktor 0,5), vyžaduje pre každé zmenšenie dve nové funkčné hodnoty (resp. v optimalizovanej verzii jednu novú hodnotu na jednu iteráciu, ale s menším ziskom presnosti na krok). MZR zmenšuje interval s faktorom $1/\tau \approx 0,618$, ale po prvotnej inicializácii vyžaduje v každom kroku iba \textbf{jedno} nové vyčíslenie funkcie[cite: 476].
    \item \textbf{Záver:} Pre našu testovaciu funkciu sa MZR ukázala ako efektívnejšia metóda, pretože dosiahla porovnateľnú presnosť s menším celkovým počtom vyčíslení funkčných hodnôt než MPI. Potvrdilo sa, že MZR je vhodnejšia v prípadoch, kde je vyčíslenie funkčnej hodnoty výpočtovo náročné.
\end{itemize}

\subsection*{Funkcia viac premenných}
Pre funkciu dvoch premenných $f(x,y)$ sme analyzovali \textit{Newtonovu metódu} (NM) a \textit{Metódu združených gradientov} (MSG).

\begin{itemize}
    \item \textbf{Rýchlosť konvergencie:} Newtonova metóda preukázala svoju teoretickú kvadratickú konvergenciu. Ak sme zvolili počiatočný bod dostatočne blízko minima, metóda našla presné riešenie v extrémne malom počte krokov (cca 3--4). MSG konvergovala pomalšie, čo je dané tým, že ide o metódu, ktorá nevyužíva informácie druhého rádu priamo v každom kroku tak agresívne ako NM.
    \item \textbf{Výpočtová náročnosť:} Hoci NM potrebovala menej iterácií, v každom kroku vyžadovala výpočet a inverziu Hessovej matice $\nabla^2 f(x)$. Pre našu funkciu dvoch premenných to bolo zvládnuteľné, ale pre úlohy s vysokou dimenziou $n$ rastie náročnosť tejto operácie s $n^3$.
    \item \textbf{Záver:} MSG sa ukázala ako výhodný kompromis[cite: 1222]. Nevyžaduje výpočet inverznej matice, má malé pamäťové nároky a hoci potrebuje viac iterácií než NM, každá iterácia je výpočtovo lacnejšia. Pre našu nekvadratickú funkciu sme museli využiť reštartovanie metódy, aby sme zabezpečili konvergenciu.
\end{itemize}

Z experimentov vyplýva, že neexistuje univerzálna „najlepšia“ metóda. Voľba algoritmu musí vždy zohľadňovať vlastnosti minimalizovanej funkcie, dimenziu problému, dostupný výpočtový výkon a požiadavky na presnosť.

\newpage
%%%%%%%%%%%%%%%%%%%%%%%%%%%%%%%%%%%%%%%%%%%%%%%%%%%%%%%%%%%%%%%%%%%%%

%ZOZNAM LITERATÚRY ---
\newpage
\addcontentsline{toc}{section}{Zoznam použitej literatúry} % Pridá sekciu do obsahu, hoci nemá číslo

\begin{thebibliography}{9} % {9} znamená zarovnanie pre max jednociferný počet zdrojov

    % 1. Vaše skriptá (Zdroj z PDF)
    \bibitem{zemanek}
    ZEMÁNEK, Petr. \textit{M5170: Matematické programování -- Kapitola 3: Numerické metody řešení úloh matematického programování I}. [učebný text]. Brno: Masarykova univerzita, Přírodovědecká fakulta, 2024.

    % 2. Štandardná učebnica na MU (odporúčam tam nechať, vyzerá to dobre)
    \bibitem{horova}
    HOROVÁ, Ivana a Jiří ZELINKA. \textit{Numerické metody}. 2. rozšířené vydání. Brno: Masarykova univerzita, 2008. ISBN 978-80-210-4644-8.

    % 3. Softvér, ktorý ste použili (voliteľné, ale vhodné)
    \bibitem{matlab}
    MATHWORKS. \textit{MATLAB Documentation}. [online]. Dostupné z: \url{https://www.mathworks.com/help/matlab/}.

\end{thebibliography}

\end{document}
